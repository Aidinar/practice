\documentclass[10pt]{book}
\usepackage[utf8]{inputenc}

\usepackage{latexsym,amssymb,amsfonts,amsmath,indentfirst,shapepar,%fleqn,%
picinpar,shadow,floatflt,enumerate,multicol,colortbl,ipi}

\usepackage{rotating}
\input{epsf}

%\nofiles

%\includeonly{avtor,avtor-eng}
%\includeonly{avtor-eng}
%\includeonly{pred}  %+
%\includeonly{podgot-2str}  %+

%\includeonly{pechinkin} %1pdf
%\includeonly{gavr} %2pdf
%\includeonly{cristof}    %3pdf
%\includeonly{korolev} %4+pdf
%\includeonly{nefedova}  %5pdf
%\includeonly{chubich} %6pdf
%\includeonly{chirkunov} %7pdf
%\includeonly{demin} %8+pdf
%\includeonly{pavel} %9pdf
%\includeonly{basha} %10pdf
%\includeonly{kudr} %11pdf

%\includeonly{toc-rus, toc-en}
%\includeonly{toc-en}

%\includeonly{obchak}
%\includeonly{reshal}
%\includeonly{eng-index}
%\includeonly{cover3}

\usepackage{acad}
\usepackage{courier}
\usepackage{decor}
\usepackage{newton}
\usepackage{pragmatica}
\usepackage{zapfchan}
\usepackage{petrotex}
\usepackage{bm}                     % полужирные греческие буквы
\usepackage{upgreek}                % прямые греческие буквы
%\usepackage{verbatim}

\renewcommand{\bottomfraction}{0.99}
\renewcommand{\topfraction}{0.99}
\renewcommand{\textfraction}{0.01}

\setcounter{secnumdepth}{1} %здесь - 3 + chapter = 4

\arraycolsep=1.5pt

%\usepackage[pdftex]{graphicx}

%\usepackage{oz}

%NEW COMMANDS


\renewcommand*{\hm}[1]{#1\nobreak\discretionary{}%
            {\hbox{$\mathsurround=0pt #1$}}{}} %% Дублирует знаки операций
                               %при переносе в формуле (перед знаком, который 
                               %надо продублировать ставится команда \hm)


%\renewcommand{\r}{{\rm I\hspace{-0.7mm}\rm R}}
\renewcommand{\r}{\mathbb{R}}
\newcommand{\I}{{\rm I\hspace{-0.7mm}I}}
\newcommand{\Ik}{\mbox{{\small \tt {1}}\hspace{-1.5mm}{\tt 1}}}
%\newcommand{\Ikl}{{\small \tt{1}}\hspace*{-0.4mm}\mathtt{1}}

%\mathrm{I}\hspace*{-0.7mm}\mathrm{R}

\newcommand{\il}[2]{\int\limits_{#1}^{#2}}%интеграл с пределами #1 и #2

%\newcommand{\il}[0]{\int\limits_{#1}^{#2}}%интеграл с пределами #1 и #2

\newcommand{\h}{{\bf H}}
\newcommand{\p}{{\sf P}}  % вероятность
\newcommand{\e}{{\sf E}}  % мат. ожидание
\newcommand{\D}{{\sf D}}  % дисперсия
\newcommand{\eps}{\varepsilon}
\newcommand{\vp}{\mathrm{v.p.}}
\newcommand{\F}{{\mathcal F}}
%\def\iint{\int\limits_{-\infty}^{\infty}}
\newcommand{\abs}[1]{\left|#1\right|}

\DeclareMathOperator{\sign}{sign}

%\newcommand{\gr}{{\geqslant}}

\newcommand{\g}{\mbox{\textit{g}}}

%\renewcommand{\la}{\lambda}
\newcommand{\si}{\sigma}
%\renewcommand{\a}{\alpha}

%\newcommand{\pto}{\stackrel{P}{\longrightarrow}} % сходимость по веpоятности

%\newcommand{\eqd}{\stackrel{d}{=}} % равенство по pаспpеделению

%\newcommand{\kp}{\kappa}
%\def\Q{{\cal Q}} \def\H{{\cal H}}
%\newcommand{\bet}{\beta_{2+\delta}}


%\newtheorem{definition}{Определение}
%\renewcommand{\thedefinition}{\arabic{definition}.}
%END NEW COMMANDS

%\renewcommand{\baselinestretch}{1.2}

%\pagestyle{myheadings}

\setlength{\textwidth}{167mm}      % 122mm
\setlength{\textheight}{658pt}
%\setlength{\textheight}{635.6pt}
\setlength{\columnsep}{4.5mm}

\setcounter{secnumdepth}{4}

%\addtolength{\headheight}{2pt}
%\addtolength{\headsep}{-2mm}

%\addtolength{\topmargin}{-20mm}  % for printing


\hoffset=-30mm  % From Yap
%\hoffset=-20mm  % From Acrobat

%\voffset=0mm % From Yap
%\voffset=-15mm   % From Acrobat

\addtolength{\evensidemargin}{-9.5mm} % for printing
\addtolength{\oddsidemargin}{9.5mm}  % for printing

%\renewcommand{\thefootnote}{\fnsymbol{footnote}}
%\renewcommand{\thefootnote}{\arabic{footnote}}
\renewcommand{\figurename}{\protect\bf Рис.}
\renewcommand{\tablename}{\protect\bf Таблица}

\newcommand{\Caption}[1]{\caption{\protect\small %\baselineskip=2.5ex
#1}}

\renewcommand{\thefigure}{\arabic{figure}}
\renewcommand{\thetable}{\arabic{table}}
\renewcommand{\theequation}{\arabic{equation}}
\renewcommand{\thesection}{\arabic{section}}

\renewcommand{\contentsname}{СОДЕРЖАНИЕ}
\newcommand{\fr}[2]{\displaystyle\frac{\displaystyle #1\mathstrut}{\displaystyle #2\mathstrut}}

%\renewcommand{\thefootnote}{\fnsymbol{footnote}}
%\newcommand{\g}{\mbox{\textit{g}}}

%\newcommand{\Caption}[1]{\caption{\protect\small\baselineskip=2ex #1}}
\newcounter{razdel}
\setcounter{razdel}{0}


\newcommand{\titel}[4]{%
\

\vspace*{5pt}

\ifodd\therazdel {\raggedright\noindent\Large\textrm\textbf
 \lineskip .75em
  \baselineskip=3.2ex #1 \par}
\vskip 1em {\noindent\large\textrm\textbf #2 \par}
\addcontentsline{toc}{subsection}{{\textrm\textbf #3}\protect\newline #1}
\def\rightheadline{\underline{\noindent\hbox to \textwidth{\hfill\small\textrm{#4}
%\hfill \large\bf\thepage
}}}
\def\leftheadline{\underline{\noindent\parbox{\textwidth}{
%\raggedleft\large\bf\thepage \hfill
\small\textit{#3}\hfill}}}
\def\leftfootline{\small{\textbf{\thepage}
\hfill ИНФОРМАТИКА И ЕЁ ПРИМЕНЕНИЯ\ \ \ том~5\ \ \ выпуск 1\ \ \ 2011}
}%
 \def\rightfootline{\small{ИНФОРМАТИКА И ЕЁ ПРИМЕНЕНИЯ\ \ \ том~5\ \ \ выпуск~1\ \ \ 2011
\hfill \textbf{\thepage}}} 
\vskip 2em \setcounter{figure}{0}
\setcounter{table}{0} 
\setcounter{equation}{0} 
\setcounter{section}{0}
\setcounter{subsection}{0} 
\setcounter{subsubsection}{0}
\setcounter{footnote}{0} 
\setcounter{razdel}{0}
%\end{flushleft}
\else {
 \raggedright\noindent\Large\textrm\textbf
 \lineskip .75em
\baselineskip=3.2ex #1 \par} \vskip 1em
%\begin{flushleft}
{\noindent\large\textrm\textbf #2 \par}
\addcontentsline{toc}{subsection}{{\textrm\textbf #3}\protect\newline #1}
\def\rightheadline{\underline{\noindent\hbox to \textwidth{\hfill\small\textrm{#4}
%\hfill \large\bf\thepage
}}}
\def\leftheadline{\underline{\noindent\parbox{\textwidth}{%\raggedleft\large\bf\thepage \hfill
\small\textit{#3}\hfill}}}
\def\leftfootline{\small{\textbf{\thepage}
\hfill ИНФОРМАТИКА И ЕЁ ПРИМЕНЕНИЯ\ \ \ том~5\ \ \ выпуск~1\ \ \ 2011}
}%
 \def\rightfootline{\small{ИНФОРМАТИКА И ЕЁ ПРИМЕНЕНИЯ\ \ \ том~5\ \ \ выпуск~1\ \ \ 2011
\hfill \textbf{\thepage}}} \vskip 2em \setcounter{figure}{0}
\setcounter{table}{0} \setcounter{equation}{0} \setcounter{section}{0}
\setcounter{subsection}{0} \setcounter{subsubsection}{0}
\setcounter{footnote}{0}
%\end{flushleft}
\fi}

\newcommand{\titelr}[2]{%
\

\vspace*{5pt}

\ifodd\therazdel {\raggedright\noindent\large\textrm\textbf
 \lineskip .75em
  \baselineskip=3.2ex #1 \par}
\vskip 1em {\noindent\normalsize\textrm\textbf #2 \par}
\else {
 \raggedright\noindent\large\textrm\textbf
 \lineskip .75em
\baselineskip=3.2ex #1 \par} \vskip 1em
%\begin{flushleft}
{\noindent\normalsize\textrm\textbf #2 \par}
\fi}

\newcommand{\titele}[5]{%
\

%\vspace*{5pt}

\ifodd\therazdel {\raggedright\noindent%\large
\textrm\textbf
 \lineskip .75em
%  \baselineskip=3.2ex
#1 \par}
\vskip .5em {\noindent\large\textrm\textbf #2 \par}
\vskip .5em
 {\noindent\textrm #3 \par}
\addcontentsline{toc}{subsection}{{\textrm\textbf #1}\protect\newline #2}
\def\rightheadline{\underline{\noindent\hbox to \textwidth{\hfill\small\textrm{#4}
%\hfill \large\bf\thepage
}}}
\def\leftheadline{\underline{\noindent\parbox{\textwidth}{
%\raggedleft\large\bf\thepage \hfill
\small\textrm{#5}\hfill}}}
\def\leftfootline{\small{\textbf{\thepage}
\hfill ИНФОРМАТИКА И ЕЁ ПРИМЕНЕНИЯ\ \ \ том~5\ \ \ выпуск~1\ \ \ 2011}
}%
 \def\rightfootline{\small{ИНФОРМАТИКА И ЕЁ ПРИМЕНЕНИЯ\ \ \ том~5\ \ \ выпуск~1\ \ \ 2011
\hfill \textbf{\thepage}}} \vskip 1em \setcounter{figure}{0}
\setcounter{table}{0} \setcounter{equation}{0} \setcounter{section}{0}
\setcounter{subsection}{0} \setcounter{subsubsection}{0}
\setcounter{footnote}{0} \setcounter{razdel}{0}
%\end{flushleft}
\else {
 \raggedright\noindent%\large
 \textrm\textbf
 \lineskip .75em
%\baselineskip=3.2ex
#1 \par} \vskip .5em
%\begin{flushleft}
{\noindent\large\textrm\textbf #2 \par} \vskip .5em
 {\noindent\textrm #3 \par}
\addcontentsline{toc}{subsection}{{\textrm\textbf #1}\protect\newline #2}
\def\rightheadline{\underline{\noindent\hbox to \textwidth{\hfill\small\textrm{#4}
%\hfill \large\bf\thepage
}}}
\def\leftheadline{\underline{\noindent\parbox{\textwidth}{%\raggedleft\large\bf\thepage \hfill
\small\textrm{#5}\hfill}}}
\def\leftfootline{\small{\textbf{\thepage}
\hfill ИНФОРМАТИКА И ЕЁ ПРИМЕНЕНИЯ\ \ \ том~5\ \ \ выпуск~1\ \ \ 2011}
}%
 \def\rightfootline{\small{ИНФОРМАТИКА И ЕЁ ПРИМЕНЕНИЯ\ \ \ том~5\ \ \ выпуск~1\ \ \ 2011
\hfill \textbf{\thepage}}} \vskip 1em \setcounter{figure}{0}
\setcounter{table}{0} \setcounter{equation}{0} \setcounter{section}{0}
\setcounter{subsection}{0} \setcounter{subsubsection}{0}
\setcounter{footnote}{0}
%\end{flushleft}
\fi}

\def\Abst#1{
\begin{center}\small\nwt
\parbox{150mm}{%\baselineskip=2.5ex
\textbf{Аннотация:}\ \
%\hspace*{\parindent}
#1}
\end{center}}
\def\Abste#1{
\begin{center}\small\nwt
\parbox{150mm}{%\baselineskip=2.5ex
\textbf{Abstract:}\ \
%\hspace*{\parindent}
#1}
\end{center}}

\def\KW#1{
\begin{center}\small\nwt
\parbox{150mm}{%\baselineskip=2.5ex
\textbf{Ключевые слова:}\ \ #1}
\end{center}}

\def\KWE#1{
\begin{center}\small\nwt
\parbox{150mm}{%\baselineskip=2.5ex
\textbf{Keywords:}\ \ #1}
\end{center}}


\def\KWN#1{
%\begin{center}
%\small
%\parbox{150mm}\end{center}
}

\renewcommand{\thesubsection}{\thesection.\arabic{subsection}\hspace*{-5pt}}
\renewcommand{\thesubsubsection}{\thesubsection\hspace*{5pt}.\arabic{subsubsection}\hspace*{-3pt}}

\begin{document}
\Rus

\nwt
%\ptb

%\renewcommand{\contentsname}{\protect\Large\bf Содержание}

\setcounter{tocdepth}{2}

%\tableofcontents

\renewcommand{\bibname}{\protect\rmfamily Литература}
  \def\Au#1{{\it #1}}

%\newcommand{\No}{№}
  \newcommand{\tg}{\,\mathrm{tg}\,}
    \newcommand{\ctg}{\,\mathrm{ctg}\,}
  \newcommand{\arctg}{\,\mathrm{arctg}\,}
  
\def\forallb{\mathop{\forall}}
\def\existsb{\mathop{\exists}}

\setcounter{page}{1}

\newpage
\addtocounter{razdel}{1}
%\def\razd{РЕГУЛИРУЕМЫЙ ЭЛЕКТРОПРИВОД ДЛЯ ЭЛЕКТРОЭНЕРГЕТИКИ}
%\newpage
%\def\stat{zakh}
\def\tit{СРЕДСТВА ОБЕСПЕЧЕНИЯ ОТКАЗОУСТОЙЧИВОСТИ ПРИЛОЖЕНИЙ}
\def\titkol{Средства обеспечения отказоустойчивости приложений}

\def\aut{В.\,Н.~Захаров$^1$, В.\,А.~Козмидиади$^2$}
\titel{\razd}{\tit}{\aut}{\titkol}


\Abst{Рассмотрены проблемы построения отказоустойчивых серверов, возникающие в связи с недетерминированностью поведения приложений. Предложена формальная модель, описывающая поведение приложения, основными объектами которой являются ресурсы и события. Предложены алгоритмы протоколирования работы приложения на резервном узле кластера, а также восстановления и продолжения его работы при отказе основного узла. При этом для клиентов сбой остается незаметным, за исключением некоторого увеличения времени обслуживания.}

\KW{сервер приложений $\bullet$ прозрачная отказоустойчивость $\diamond$
 процесс $\diamond$ ресурс $\diamond$ событие $\diamond$ контрольная точка
$\bullet$ детерминированность}

\vskip 12pt plus 6pt minus 3pt

\begin{multicols}{2}

\section*{ВВЕДЕНИЕ}

Средства вычислительной техники стали использоваться в областях,
требующих безотказной работы систем в течение многих лет (24 часа
в сутки, 365 дней в году).

\label{st\stat}

\footnotetext{$^1$ФГУП Центральный институт авиационного моторостроения
им. П.И. Баранова, Москва, Россия}
\footnotetext{$^2$ФГУП Центральный институт авиационного моторостроения
им. П.И. Баранова, Москва, Россия}

К таким областям относятся, например, центры хранения и обработки данных  в сетях (системы резервирования билетов, биллинговые,  банковские и т.д.), массированные распределенные вычисления (GRID-вычисления) и другие.

\thispagestyle{headings}

Обычно в подобных системах применяются частные решения, ориентированные в основном на обеспечение надежного хранения данных (например, файловые серверы, использующие для хранения RAID-контроллеры) и корректного их состояния при отказах (серверы баз данных с транзакционным выполнением запросов). Однако большинство приложений не гарантируют, что не произойдет потери части данных при отказе системы. Обычно предполагается, что клиентские средства должны повторять запросы после восстановления серверов, для того, чтобы данные не были потеряны, или что можно сделать возврат по времени на некоторое время назад и повторить работу с этого места. Однако далеко не все клиентские средства и условия применения приложений допускают это.

Отказоустойчивые системы для критически важных приложений, корректно решающие проблемы восстановления после сбоев,   предлагаемые ведущими производителями, как правило, дороги. Кроме того, они включают специфические серверные и клиентские приложения, не совместимые со стандартными приложениями, не обеспечивающими отказоустойчивость. Примером такого подхода к решению проблемы отказоустойчивости  хранения данных являются системы NetApp FAS компании Network Appliance, работающие на базе специализированной операционной системы Data ONTAP [1].

Построение отказоустойчивых систем, использующих серверы со стандартными приложениями, в свете вышесказанного, является актуальной проблемой, вызывающей значительный интерес. Рассмотрение методов достижения прозрачной отказоустойчивости таких систем и является предметом статьи.
\begin{figure*} %fig1
\vspace*{1pt}
\begin{center}
\mbox{%
\epsfxsize=1.6in
\epsfxsize=100mm
\epsfbox{BbR-1.eps}
}
\end{center}
\vspace*{-9pt}
\Caption{Базовый вариант трубы с разными выходными устройствами
(цилиндрическое, расширяющееся и сужающееся сопло)
\label{f1bab}}
\vspace*{-3pt}
\end{figure*}


\section{ОСНОВНЫЕ ПОНЯТИЯ И ПОДХОДЫ}

Под сервером в данной работе понимается вычислительный центр
(отдельный компьютер или кластер) в сети, предоставляющий клиентам
(пользователям, клиентским компьютерам) определенные услуги, разделяя
между ними свои ресурсы. Подобные серверы названы серверами приложений.
Широко распространенным примером сервера такого типа является файловый сервер, обеспечивающий удаленный коллективный доступ к файловой системе. Часто используются вычислительные серверы, предоставляющие клиентам возможность выполнять на них свои программы (например, в центрах коллективного пользования).


Обычно приложение представляет собой программу или группу программ, работающих в операционной среде, создаваемой операционной системой (в другой терминологии - один или несколько взаимодействующих процессов или потоков (threads)), которые реализуют функциональность сервера. Для построения отказоустойчивых серверов приложений широко используется кластерная технология. Следуя [2], кластером, названа разновидность параллельной или распределенной системы, которая:
\begin{itemize}
\item состоит из нескольких компьютеров (узлов кластера), связанных как минимум необходимыми коммуникационными каналами;
\item используется как единый, унифицированный компьютерный ресурс.
\end{itemize}

Прозрачная отказоустойчивость (Transparent Fault Tolerance, TFT) сервера приложений - это такое его поведение при возникновении аппаратных или программных отказов либо отказов в сети, при котором:
\begin{itemize}
\item отказ не вызывает потери или искажения данных, находящихся в базе данных сервера;
\item сервер продолжает нормально функционировать, несмотря на имевшие место отказы.
\end{itemize}

Клиенты сервера "не замечают" произошедших отказов. Единственным\footnote{допустимым
отклонением сервера от нормального поведения с точки зрения клиента является
некоторое увеличение времени обслуживания} (на несколько секунд или десятков секунд).

Обычно приложения, работающие на серверах приложений, не ориентированы на прозрачную отказоустойчивость. Они "заботятся" лишь о собственной целостности (например, состояния файловой системы или базы данных). Восстановление работоспособности сервера приводит к разрыву соединений с клиентами и потере их запросов. Это замечают клиенты - запросы следует повторять, на что клиентские приложения далеко не всегда рассчитаны. В данной работе предполагается, что приложения (прикладные программные средства), выполняемые на сервере, являются стандартными, то есть не имеют специальных средств, обеспечивающих отказоустойчивость.
\begin{figure*}[b] %fig1
\vspace*{1pt}
\begin{center}
\mbox{%
\epsfxsize=1.6in
\epsfxsize=100mm
\epsfbox{BbR-1.eps}
}
\end{center}
\vspace*{-9pt}
\Caption{Базовый вариант трубы с разными выходными устройствами
(цилиндрическое, расширяющееся и сужающееся сопло)
\label{f1bab}}
\vspace*{-3pt}
\end{figure*}

Серьезные исследования в области обеспечения отказоустойчивости серверов были развернуты после создания вычислительных серверов, предназначенных для решения задач, требующих больших вычислительных ресурсов. Решение этих задач выполняется на суперкомпьютерах, обеспечивающих массово-параллельные вычисления и представляющих собой кластеры из сотен и тысяч узлов (процессоров). Однако даже на этих "монстрах" решение может требовать десятков или сотен часов, и одиночный сбой, если не предприняты специальные меры, может привести к необходимости начинать работу сначала. Обычно решение вычислительной задачи в таких случаях осуществляется в модели относительно редко взаимодействующих между собой процессов, выполняемых на разных узлах кластера. Эти взаимодействия нужны для координации работы процессов, в частности, для обмена данными и промежуточными результатами. Взаимодействия опираются на специальный протокол, называемый MPI (Message-Passing Interface) и представляющий собой стандарт "de facto" [3].

Для преодоления последствий сбоя достаточно давно была разработана и широко применяется технология, опирающаяся на механизм контрольных точек (checkpoints) [4-6]. По этой технологии система должна иметь стабильную память, которая не меняется при отказах. Соответствующие программные средства периодически сохраняют информацию о состоянии процессов приложения в стабильной памяти. Все процессы также имеют доступ к устройству стабильной памяти.  В случае отказа или сбоя, записанная в стабильную память информация используется для повторения вычисления с момента, когда была записана эта информация, то есть выполняется откат назад по времени. Данные, сохранение которых позволяет выполнить откат, называются контрольной точкой. В качестве устройства стабильной памяти может использоваться дисковый том, энергонезависимая оперативная память, память другого узла или узлов кластера. В последнем случае узел, которому требуется сохранить информацию, пересылает ее через быстрый канал связи на другой узел. Стабильная память после отказа одного из узлов должна быть доступной узлу, на котором делается повтор.

Однако решение, опирающееся только на контрольные точки, не является прозрачным, поскольку не скрывает от клиентов факт отказа системы и требует от них выполнения определенных действий. Так как при работе процессы обмениваются сообщениями, возможны два варианта решения проблемы. Первый - все процессы выполняют записи контрольных точек одновременно, что затруднительно. Второй вариант, при несоблюдении синхронности, - возврат в каждом процессе к такому скоординированному набору контрольных точек, при котором невозможна противоречивая ситуация. Такая ситуация возникает, когда один процесс вернулся к контрольной точке, после которой он должен получить сообщение от другого процесса, а этот другой процесс вернулся к точке, которая следует за выдачей этого сообщения. Однако при повторе ожидаемое первым процессом сообщение не поступит. В этом случае  возможен эффект домино, в результате процессы оказываются отброшены как угодно далеко назад.

В этом состоит первая проблема, которую необходимо преодолеть.

Если нужно, чтобы последствия отказа узла не были видны клиенту,  это означает:
\begin{itemize}
\item клиент не должен терять и потом восстанавливать соединения с сервером;
\item клиент не должен повторять свои запросы;
\item клиент не должен повторно получать сообщения, которые он уже получил.
\end{itemize}

Вторая проблема, которую надо решать, связана с недетерминированностью поведения сервера приложений. Приведем пример.  Пусть имеется система продажи билетов на самолеты. Два клиента одновременно обратились к системе с запросом билета на один и тот же рейс. Клиентам безразлично, какие места им зарезервирует система. Система выполняет запросы клиентов параллельно, поэтому в какой-то момент между процессами, обрабатывающими эти запросы, может возникнуть конкуренция за ресурс - в данном случае, скажем, рейс. Один из процессов захватывает ресурс первым, резервирует место и освобождает ресурс. Потом второй процесс проделывает то же самое.

Порядок, в котором в этом примере процессы захватили ресурс, зависит от многих факторов и, в конечном счете, случаен. Однако  это не мешает правильному функционированию системы, поскольку клиентам важно одно - получить билеты, причем на разные места. Однако отсутствие детерминизма в поведении приложения приводит к тому, что при повторном выполнении могут быть получены другие результаты: например, клиенту уже сообщено, что ему зарезервировано место №5, а при повторе может получиться, что зарезервировано место №6. Система должна устранить это несоответствие и сделать его невидимым для клиента.
\begin{figure*} %fig1
\vspace*{1pt}
\begin{center}
\mbox{%
\epsfxsize=1.6in
\epsfxsize=100mm
\epsfbox{BbR-1.eps}
}
\end{center}
\vspace*{-9pt}
\Caption{Базовый вариант трубы с разными выходными устройствами
(цилиндрическое, расширяющееся и сужающееся сопло)
\label{f1bab}}
\vspace*{-3pt}
\end{figure*}

Недетерминированность поведения системы это следствие, по крайней мере, двух обстоятельств. Во-первых, это присущая системам с разделением времени неопределенность в порядке выполнения процессов. Во-вторых, это конкуренция процессов за общие ресурсы. Перечислим некоторые причины недетерминированного поведения приложений:
\begin{itemize}
\item синхронизация процессов с помощью семафоров или атомарных операций над операндами в общей памяти процессов;
\item зависимость от порядка получения клиентских запросов;
\item время, затраченное процессом на обработку полученного запроса;
\item генераторы случайных чисел;
\item системное управление процессами и потоками;
\item локальные таймеры;
\item доступ к реальному времени.
\end{itemize}

По различным  причинам время, которое тратится на выполнение вычислительной задачи с одними и теми же исходными данными, не является константой, то есть повторное выполнение может дать другое время. Процессы используют общие ресурсы, обращение к которым требует организации очередности выполнения (сериализации) - первым пришел, первым захватил. И, наконец,  результат работы процесса может зависеть от состояния ресурса, а это состояние может изменить другой процесс, ранее захвативший ресурс. Все это создает значительные трудности при попытках воспроизведения поведения процессов с сохраненной контрольной точки.

Прозрачная отказоустойчивость серверов приложений обычно осуществляется переносом приложения на другой узел кластера, идентичный первому по конфигурации аппаратных средств и операционной среды. Это делается методом, называемым snapshot/restore. На основном узле (оригинале)  периодически фиксируется состояние приложения на этом узле кластера (так называемый снимок или snapshot). После отказа оригинала на резервном узле (копии) делается восстановление (restore), то есть восстанавливается последнее зафиксированное состояние приложения. Операционная среда при этом приводится в состояние, которое соответствует моменту изготовления снимка. После этого узел-копия продолжает работу с зафиксированного места. Сравнение метода  snapshot/restore с другими подходами приведено в [7].

Ниже рассматриваются информационные  технологии, позволяющие решить ряд принципиальных вопросов, связанных с реализацией прозрачной отказоустойчивости серверов приложений. Ими являются:
\begin{itemize}
\item виртуализация операционной среды, в которой работает серверное приложение;
\item отказоустойчивая реализация протокола TCP;
\item создание контрольных точек состояния приложения и файловой системы, которые делаются внешним по отношению к приложению образом;
\item восстановление серверного приложения на основании контрольной точки.
\end{itemize}
\begin{figure*} %fig1
\vspace*{1pt}
\begin{center}
\mbox{%
\epsfxsize=1.6in
\epsfxsize=100mm
\epsfbox{BbR-1.eps}
}
\end{center}
\vspace*{-9pt}
\Caption{Базовый вариант трубы с разными выходными устройствами
(цилиндрическое, расширяющееся и сужающееся сопло)
\label{f1bab}}
\vspace*{-3pt}
\end{figure*}

\section{МОДЕЛЬ ОПИСАНИЯ ПОВЕДЕНИЯ ПРИЛОЖЕНИЯ}

Предлагаемый подход опирается на построение модели вычислений, связанной с использованием понятия времени в многопроцессных приложениях. Впервые подобные проблемы были изучены в классической работе Л. Лампорта [8].

Многопроцессными приложения называются потому, что в них параллельно работают несколько процессов. Процесс ведет себя детерминированно, пока в предписанном кодом порядке выполняет процессорные инструкции. Конечно, его работа может быть прервана практически в любой момент и процессор передан другому процессу или ядру. Поэтому абсолютное время, которое затрачивает процесс на выполнение определенной работы, не  константа, а случайная  величина. То же  относится к относительному времени, то есть времени, которое процесс занимал процессор,  поскольку одни и те же обращения к операционной среде могут вызвать работы разной длительности, а значит потребовать разное время на свое выполнение.

Кэшированность инструкций и данных, а также длина хэш-списков влияют на действительное время пребывания в операционной среде. Утрачивает смысл понятие одновременность действий, поскольку  нельзя установить, выполнили ли два разных процесса какие-либо действия одновременно или одно из них предшествовало другому. Таким образом, с процессом можно связать только его локальное время, которое линейно упорядочивает события,  происходившие в этом процессе.  Глобальное время, линейно упорядочивающее действия во всех процессах, отсутствует. Расстояние (в этом качестве используется время) между действиями оказывается случайной величиной.

Эти соображения важны, поскольку процессы в интересующих нас приложениях взаимодействуют и используют общие ресурсы. Для взаимодействия они используют средства синхронизации, предоставляемые операционной средой - например, наборы семафоров SVR4 (System V Release 4), POSIX-семафоры, бинарные семафоры и другие примитивы взаимного исключения (POSIX- mutual exclusion locks) и т.д. Подобные средства операционной среды, которые позволяют процессам синхронизировать свою деятельность друг с другом или сериализовать обращения к совместно используемым объектам,  будут ниже  называться ресурсами.

С каждым ресурсом связано свое локальное время, линейно упорядочивающее события в жизни ресурса. Например, в случае двоичных семафоров это создание семафора, а также его захват и освобождение процессом. Заметим, что событие - это не намерение процесса (например, захватить бинарный семафор), а сам факт захвата семафора процессом (т.е. успешное выполнение намерения). От изъявления намерения до его осуществления может многое произойти. Например, семафор, который хочет захватить рассматриваемый процесс, принадлежал другому процессу, потом тот процесс его освободил, но семафор был сначала передан операционной средой третьему процессу, который также на него претендовал, и т.д. Поведение рассматриваемого процесса в это время нас не интересует - он ресурсом еще не овладел, а только его захват определяет его дальнейшее поведение. По причинам,  изложенным выше, расстояние между двумя событиями - случайная величина. Однако, события замечательны тем, что они одновременно присутствуют и в локальном времени процесса, и в локальном времени ресурса. Поэтому все, что произошло в истории процесса или/и ресурса до этого события, предшествует ему. Далее  будет считаться, что истории и ресурсов и процессов состоят только из событий, причем между двумя последовательными событиями в жизни процесса последний ведет себя детерминированно.

Это означает, что на  поведении процесса сказывается только его предыдущая история, то есть состояние ресурсов, с которыми он взаимодействовал. Это свойство процессов ниже будет называться локальной детерминированностью. Этим свойством не обладают ресурсы, поскольку - следующее событие в истории ресурса не определяется однозначно по его предыдущей истории. Утверждение, касающееся детерминированного поведения процессов, неявно опирается на предположение,  что учтены все ресурсы, которые могут привести к  недетерминированности процессов.

Таким образом, описанное нами очень неформально время в многопроцессном комплексе представляет собой отношение частичного порядка, введенное на множестве событий. Зная полное состояние комплекса в некоторый момент времени,  нельзя однозначно определить, какое событие в истории ресурса наступит следующим. Можно говорить только о вероятности наступления того или иного события. Недетерминированность поведения есть следствие двух обстоятельств. Во-первых, это неопределенность времени, которое тратит процесс на переход от одного события к другому. Во-вторых, конкуренция процессов за общие ресурсы.

Выполнение приложения, на множестве событий которого введена частичная упорядоченность, можно описать направленным ациклическим графом выполнения. Вершинами этого графа являются события, с каждым  из которых связаны две входящие в него дуги. Одна дуга начинается в событии, которое непосредственно предшествует данному событию в истории процесса, другая - в истории ресурса.

Построение средств обеспечения прозрачной отказоустойчивости приложений опирается на следующее утверждение: для восстановления работы приложения после отказа достаточно располагать:
\begin{itemize}
\item контрольной точкой, которая отражает на некоторый момент времени состояния процессов и других ресурсов, образующих приложение;
\item графом выполнения приложения, который описывает работу приложения, начинающуюся с контрольной точки и заканчивающуюся отказом. Данные, которые нужны для построения графа выполнения, далее называются протоколом.
\end{itemize}
\begin{figure*} %fig1
\vspace*{1pt}
\begin{center}
\mbox{%
\epsfxsize=1.6in
\epsfxsize=100mm
\epsfbox{BbR-1.eps}
}
\end{center}
\vspace*{-9pt}
\Caption{Базовый вариант трубы с разными выходными устройствами
(цилиндрическое, расширяющееся и сужающееся сопло)
\label{f1bab}}
\vspace*{-3pt}
\end{figure*}

Вся эта информация должна находиться в стабильной памяти, не разрушающейся при отказе.

Ниже неформально описан алгоритм восстановления работы приложения после отказа, который опирается на наличие контрольной точки и графа выполнения. Будем считать, что в распоряжении имеются средства, позволяющие остановить процесс в тот момент, когда он намерен совершить некоторую операцию над ресурсом. Заметим, что событие в графе выполнения соответствует не изъявлению намерения, а его удовлетворению, то есть завершению выполнения операции.

Предварительно сделаем следующее:
\begin{itemize}
\item используя контрольную точку, приведем приложение в состояние, соответствующее этой контрольной точке;
\item в графе выполнения пометим все вершины (события) как "не наступившие". У некоторых вершин графа отсутствуют им непосредственно предшествующие; соответствующие события наступили сразу же после создания контрольной точки. Для каждой такой вершины включим в граф дополнительную вершину, ей предшествующую в истории процесса, и отметим эту дополнительную вершину как "наступившую";
\item разрешим процессам приложения выполняться.
\end{itemize}

Пусть некоторый процесс проявляет намерение выполнить операцию над каким-либо ресурсом. Отыщем для этого процесса в его истории последнее наступившее событие. Следующее в его истории событие - это то, которое соответствует требуемой операции. Посмотрим, наступило ли событие в истории ресурса, которое ему предшествует. Если нет, переведем процесс в состояния ожидания, отметив в предшествующем событии, что данный процесс ожидает его наступления. Если да, разрешим процессу выполняться, то есть выполнить операцию над ресурсом.

Пусть некоторый процесс объявляет, что он выполнил операцию над каким-либо ресурсом (это соответствует моменту протоколирования при оригинальном выполнении). Отыщем для этого процесса в его истории последнее наступившее событие и перейдем к следующему событию в его истории. Это опять то событие, которое мы рассматриваем. Отметим его как "наступившее". Если наступления этого события ожидал какой-нибудь процесс, выведем этот процесс из состояния ожидания. Наконец, разрешим процессу, выполнившему операцию, продолжаться дальше.

Когда выясняется, что наступили все события графа выполнения, повторное выполнение считается законченным.

Естественным следствием из сказанного является следующее утверждение: для того, чтобы размер протокола не рос неограниченно, нужно периодически создавать контрольные точки, очищая при этом протокол.

\section{ФОРМАЛЬНОЕ ОПИСАНИЕ МОДЕЛИ ПОВЕДЕНИЯ МНОГОПРОЦЕССНОГО ПРИЛОЖЕНИЯ}
\begin{figure*} %fig1
\vspace*{1pt}
\begin{center}
\mbox{%
\epsfxsize=1.6in
\epsfxsize=100mm
\epsfbox{BbR-1.eps}
}
\end{center}
\vspace*{-9pt}
\Caption{Базовый вариант трубы с разными выходными устройствами
(цилиндрическое, расширяющееся и сужающееся сопло)
\label{f1bab}}
\vspace*{-3pt}
\end{figure*}

Опишем формально поведение приложения, неформальное описание которого было приведено выше. Рассматриваются два типа объектов:
\begin{itemize}
\item ресурсы (r), например, наборы семафоров (POSIX- или SVR4-семафоры), бинарные семафоры (POSIX-mutex's), таймер реального времени, сокеты (sockets), то есть двусторонние виртуальные соединения с внешним миром;
\item процессы (p), например, процессы или потоки (threads) пользователя.
\end{itemize}

\end{multicols}

\label{end\stat}

%\def\stat{batr}

\def\tit{НОВЫЙ МЕТОД ВЕРОЯТНОСТНО-СТАТИСТИЧЕСКОГО\newline
АНАЛИЗА ИНФОРМАЦИОННЫХ ПОТОКОВ
В~ТЕЛЕКОММУНИКАЦИОННЫХ СЕТЯХ$^*$}
\def\titkol{Новый метод вероятностно-статистического
анализа информационных потоков
в~телекоммуникационных сетях}
\def\autkol{Д.\,А.~Батракова, В.\,Ю.~Королев, С.\,Я.~Шоргин}
\def\aut{Д.\,А.~Батракова$^1$, В.\,Ю.~Королев$^2$, С.\,Я.~Шоргин$^3$}

\titel{\tit}{\aut}{\autkol}{\titkol}

{\renewcommand{\thefootnote}{\fnsymbol{footnote}}\footnotetext[1]{Работа 
выполнена при поддержке РФФИ, проекты №№\,04-01-00671, 05-07-90103.} 
\renewcommand{\thefootnote}{\arabic{footnote}}}
 \footnotetext[1]{ИПИ РАН, 
daria.batrakova@gmail.com} \footnotetext[2]{Факультет вычислительной математики 
и кибернетики МГУ им.~М.\,В.~Ломоносова, ИПИ РАН, vkorolev@comtv.ru} 
\footnotetext[3]{ИПИ РАН, sshorgin@ipiran.ru}



\Abst{В данной работе предлагается метод исследования стохастической структуры
хаотических информационных потоков в сложных телекоммуникационных
сетях. Предлагаемый метод основан на стохастической модели
телекоммуникационной сети, в рамках которой она представляется в виде
суперпозиции некоторых простых последовательно-параллельных структур.
Эта модель естественно порождает смеси гамма-распределений для времени
выполнения (обработки) запроса сетью. Параметры получаемой смеси
гамма-распределений характеризуют стохастическую структуру
информационных потоков в сети. Для решения задачи статистического
оценивания параметров смесей экспоненциальных и гамма-распределений
(задачи разделения смесей) используется ЕМ-алгоритм. Чтобы проследить
изменение стохастической структуры информационных потоков во времени,
ЕМ-алгоритм применяется в режиме скользящего окна. Описывается
программный инструментарий для применения полученного решения к
реальным статистическим данным. Приводится интерпретация результатов.}

\KW{телекоммуникационные сети; информационные потоки;
разделение смесей  распределений;
метод скользящего окна;  программа для разделения смесей}

\vskip 24pt plus 9pt minus 6pt

\thispagestyle{headings}

\begin{multicols}{2}


\label{st\stat}

\section{Введение}

Развитие телекоммуникационных сетей, их усложнение поставило перед
инженерами важную прикладную задачу исследования характеристик
информационных потоков, возникающих в этих сетях. Здесь под
информационным потоком мы будем понимать упорядоченное движение
любого вида информации по сети.

Если на заре эры телекоммуникаций, в эпоху первых телефонных линий и
телеграфа эта проблема не была столь насущной, то со временем, при
постепенном охвате мирового пространства сетями возникла необходимость в
построении и исследовании адекватных моделей сетей и процессов,
происходящих в них.

\thispagestyle{headings}


Современные сети~--- это \textit{конвергентные} сети, т.е.\ совокупность крайне
разнородных как по топологии, так и по физической архитектуре сетей, которые
предлагают конечному пользователю самые разнообразные сервисы. Это~--- огромное
виртуальное и физическое пространство, состоящее из миллионов процессоров,
операционных платформ, линий передачи данных и стыковочных узлов.
%
Существует множество классификаций телекоммуникационных сетей по различным
признакам:
\begin{itemize}
\item масштабу (локальные сети~--- LAN, масштаба города~---
MAN, широкого масштаба~--- WAN);
\item топологии, или логической организации (<<звезда>>,
<<кольцо>>, <<шина>>);
\item физической организации (оптические сети, радио);
\item предлагаемым услугам (сотовые сети, для доступа в
Интернет);
\item назначению (военные, гражданские) и~др.
\end{itemize}


Конвергентная сеть входит во все эти классы, причем, как правило,
обладает всеми этими признаками. Поэтому построение модели для ее анализа
является и очень важной, и очень сложной задачей.

Существуют достаточно многочисленные математические методы, ориентированные на
моделирование и анализ телекоммуникационных сетей. В~большинстве своем они
основываются на теории массового обслуживания, то есть разделе теории
вероятностей, посвященном описанию функционирования сложных систем обслуживания
(в том чис\-ле телекоммуникационных сетей и систем) с помощью стохастических
процессов особого вида и анализу таких процессов. Указанная теория является
весьма развитой и широко применяется на практике. Тем не менее, ее применимость
ограничена~--- во-первых, все возрастающей сложностью структур и дисциплин
обслуживания в рас\-смат\-ри\-ва\-емых реальных сетях. Эта сложность во многих
случаях принципиально не может найти адекватного отображения в моделях
массового обслуживания, даже несмотря на постоянно растущую сложность самих
этих моделей. В результате даже модели, допускающие точный математический
анализ, дают возможность расчета всего лишь приближенных значений характеристик
реальных сетей, ибо предположения, принимаемые при построении моделей, во
многих случаях не соответствуют практике. Во-вторых, для описания
телекоммуникационной сети в виде сети массового обслуживания исследователь
должен располагать детальным описанием структуры сети, что далеко не всегда
имеет мес\-то на практике. В-третьих, разработано крайне мало моделей массового
обслуживания, в которых используется в качестве входной информация о
наблюдаемых (статистических) показателях функционирования сети; в то же время,
такая информация очень часто доступна исследователю, и ее использование при
анализе сети весьма целесообразно.

В данной работе предлагается в определенной степени альтернативный подход к
моделированию сложных телекоммуникационных сетей. Строится и исследуется
вероятностная модель сложной телекоммуникационной сети как суперпозиции
достаточно простых структур. При этом практически никакая априорная информация
о структуре исследуемой сети не используется~--- наоборот, в результате
исследования модели исследователь получает приближенное представление об этой
структуре. Характеристики типовых простых структур, составляющих в совокупности
модель сети, и сети в целом при этом подходе описываются
гам\-ма-рас\-пре\-де\-ле\-ни\-я\-ми. Ставится задача оценки параметров модели
на основе статистических данных о функционировании сети, а также предлагается
математическое решение этой задачи. В статье описан также созданный на основе
разработанных математических методов программный инструментарий и приведены
результаты расчетов для реального трафика. {\looseness=-1

}

\section{Математическая модель и~постановка задачи}

\subsection{Логическая модель сети}
 %1.1

Рассмотрим абстрактную \textit{конвергентную телекоммуникационную
сеть}. Это может быть как крупномасштабная транспортная сеть (WAN), сеть
отдельного оператора масштаба города (MAN) с различными сервисами, так и
локальная сеть (LAN).

Любой из этих случаев можно описать как ($p,\,q$)-\textit{сеть}.

\medskip
\textbf{Определение 1.} В теории графов и сетей под ($p,\,q)$-сетью понимается
набор вида $S =$\linebreak $=(G,\,V^\prime ,\,V^{\prime\prime})$, где $G$~---
граф, а $V^\prime$ и $V^{\prime\prime}$~--- выборки из множества $V(G)$ (вершин
графа) длины~$p$ и $q$ соответственно. При этом выборка $V^\prime$
($V^{\prime\prime}$) считается \textit{входной} (\textit{выходной}) выборкой, а
ее $i$-я вершина называется $i$-\textit{м} \textit{входным} (\textit{выходным})
\textit{полюсом} или, иначе, $i$-\textit{м} \textit{входом} (\textit{выходом})
сети~$S$. Вершины, не участвующие во входной и выходной выборках сети,
считаются ее внутренними вершинами~\cite{1bat}.

Сеть $S$ (рис.~\ref{f1bat}) имеет $p$ точек входа~--- точек соединения
с внешней средой (это могут быть точки стыковки разнородных сетей, сетей
различных операторов, физические подключения к интерфейсам
маршрутизаторов и~т.п.). Под \textit{внешней средой} будем понимать другие
сети, которые передают данные в сеть~$S$. Отдельные <<единицы>> данных
(кадры, сообщения, датаграммы, пакеты) поступают на входы сети~$S$,
обрабатываются и подаются на каждый из $q$ выходов, которые могут быть
соединены как с конечными пользователями, так и с другими сетями.
\begin{figure*} %fig1
\vspace*{1pt}
\begin{center}
\mbox{%
\epsfxsize=139.7mm \epsfbox{bat-1.eps}
%\epsfxsize=139.698mm
%\epsfbox{bek-3.eps}
}
\end{center}
\vspace*{-9pt} \Caption{Абстрактная телекоммуникационная сеть \label{f1bat}}
\end{figure*}

Следует отметить, что структура сложных телекоммуникационных сетей обладает
свойством некоторого самоподобия, т.е.\ на каком бы уровне сетевой архитектуры
мы ни рассматривали поведение информационных потоков, мы можем выделить
некоторые базовые структуры, субпотоки, суперпозицией которых мы можем получить
модель конкретной сети, какой бы уровень <<детализации>> сегментов сети мы ни
взяли. Так, например, физические подключения к интерфейсам оптического
кросс-коннекта в этом смысле подобны <<виртуальным>> подключениям к портам TCP
на сервере приложений.

Итак, независимо от уровня сетевой архитектуры мы можем
рассматривать некоторую величину, характеризующую количество каких-либо
ресурсов сети~$S$, занимаемых в процессе передачи и обработки данных.  Это
могут быть ресурсы, относящиеся как к <<объему>> (памяти сетевого
устройства, количеству занятых линий, размеру пакета), так и ко <<времени>>
(времени обслуживания заявки, времени простоя). Эта величина случайна, т.к.\
мы не можем абсолютно точно сказать для сложной телекоммуникационной
сети, какое сообщение на какой из входов поступит и какого размера оно будет.
Таким образом, случайный характер данной величины определяется
случайностью поведения внешней среды.

Пусть $R$~--- это описанная выше случайная величина, $R>0$. Далее, не
ограничивая общности, будем подразумевать под ней время, необходимое для
какой-либо операции сети (обработки <<единицы>> данных), предполагая, что
время обработки прямо зависит от объема сообщения.

\subsection{Вероятностная модель сети} %1.2.

Даже не зная реальной топологии сети, мы можем описать
функционирование некоторых ее участков как процесс выполнения операций
(задач сети) в последовательном  порядке (например, если доступен только
один канал данных) или как процесс одновременного выполнения субопераций
(когда доступно более одного пути выполнения). Это значит, что мы можем
представить функционирование сложной телекоммуникационной сети как
\textit{суперпозицию} таких <<последовательных>> и <<параллельных>>
блоков.

Для построения вероятностной модели распределения~$R$ используется
комбинация асимптотического подхода, основанного на предельных теоремах
теории вероятностей, и принципа максимальной неопределенности (энтропии).

Рассмотрим следующую модель. Предположим, что мы можем разделить
сеть~$S$ на несколько сегментов $S_i$. Пусть $T$~--- случайная величина,
время выполнения операции в отдельно взятом блоке $S_i$ (сегменте сети).

Если операции выполняются \textit{параллельно}, то время, необходимое
для их выполнения~--- это максимальное время, затрачиваемое на какую-либо
субоперацию:
$$
T = \underset{i}{\max}\, T_i\,,
$$
где $T_i$~--- случайные величины для со\-от\-вет\-ст\-ву\-ющих субопераций.
Модель такого выполнения пред\-став\-ле\-на на рис.~\ref{f2bat}.

\begin{figure*} %fig2
\vspace*{1pt}
\begin{center}
\mbox{%
\epsfxsize=117.271mm
\epsfbox{bat-2.eps}
}
\end{center}
\vspace*{-9pt}
\Caption{Параллельное выполнение
\label{f2bat}}
\end{figure*}

Известно, что предельное распределение экстремальных значений для
выборок ~--- это экспоненциальное распределение с плотностью~\cite{2bat}
$$
f(x) =
\begin{cases}
\lambda e^{-\lambda x}\,, & x>0\,,\\
0\,, & x\leq 0\,,
\end{cases}
$$
где $\lambda >0$~--- параметр масштаба.

 Учитывая также энтропийный подход, естественно будет считать
распределение $T$ экспоненциальным, т.к.\ экспоненциальное распределение
обладает наибольшей энтропией среди всех распределений с $x>0$.

Если же операции сети выполняются \textit{последовательно}, то величина
$T$~--- это сумма времен $T_i$, необходимых для выполнения каждой
субоперации:
$$
T = \sum\limits_i T_i\,,
$$
где $T_i$~--- случайные величины для со\-от\-вет\-ст\-ву\-ющих субопераций.
%
Такая модель представлена на рис.~\ref{f3bat}.

\begin{figure*} %fig3
\vspace*{1pt}
\begin{center}
\mbox{%
\epsfxsize=139.592mm
\epsfbox{bat-3.eps}
}
\end{center}
\vspace*{-9pt}
\Caption{Последовательное  выполнение
\label{f3bat}}
\end{figure*}

Это значит, что распределение общей длительности $T$ выполнения
блока~--- это свертка распределений <<элементарных>> времен $T_i$
(экспоненциально распределенных).

Известно, что результатом свертки экспоненциальных распределений
является гамма-распределение, определяемое плотностью
$$
\g_{\lambda , \alpha} (x) =
\begin{cases}
\fr{\lambda_0^{\alpha_0}}{\Gamma (\alpha_0 )}\,x^{\alpha_0-1}
e^{\lambda_0 x}\,, & x>0\,,\\
0,\, & x\leq 0\,,
\end{cases}
$$
где $\alpha >0$~--- параметр формы,  $\lambda >0$  параметр масштаба, а
$\Gamma (z)$~--- гамма-функция Эйлера:
$$
\Gamma (z) = \int\limits_0^\infty x^{z-1} e^{-x}\,dx\,.
$$

\begin{figure*} %fig4
\vspace*{1pt}
\begin{center}
\mbox{%
\epsfxsize=120.831mm
\epsfbox{bat-4.eps}
}
\end{center}
\vspace*{-9pt}
\Caption{Модель пути  обработки сообщения сетью~$S$
\label{f4bat}}
\end{figure*}

Известно~\cite{2bat}, что класс гамма-распределений замкнут над операцией
свертки, поэтому ре\-зуль\-ти\-ру\-ющее распределение случайной величины~$R$
будет также гамма-распределением
$$
\g_{\lambda , \alpha} (x) =
\begin{cases}
\fr{\lambda^{\alpha}}{\Gamma (\alpha )}\,x^{\alpha -1} e^{-\lambda x}\,, &
x>0\,,\\
0,\, & x\leq 0\,.
\end{cases}
$$

В силу случайного характера ввода данных в сеть~$S$ из внешней среды маршрут
передачи данных становится случайным, что представлено на рис.~\ref{f4bat}. Это
означает, что параметры ре\-зуль\-ти\-ру\-юще\-го распределения~$R$ тоже
случайны. Отсюда имеем следующую модель \textit{смеси
гам\-ма-рас\-пре\-де\-ле\-ний}, определяемой плотностью

\begin{equation} %1
p(x) = \iint \g_{\lambda , \alpha}(x)\,dH (\lambda ,\,\alpha )\,,
\end{equation}
где $H(\lambda , \alpha )$~--- смешивающая функция, функция распределения
входных параметров.

Поясним понятие \textit{смеси распределений}.

\medskip
\textbf{Определение~2.} Пусть имеется двух\-па\-ра\-мет\-ри\-че\-ское
семейство $n$-мерных плотностей  распределения
\begin{equation}
F = \{ f_\omega (x;\, \theta (\omega ))\}\,,
\end{equation}
где одномерный (целочисленный или непрерывный) параметр $\omega$ в
качестве нижнего индекса функции $f$ определяет специфику общего вида
каж\-до\-го компонента~--- распределения смеси, а в качестве аргумента при
многомерном, вообще говоря, параметре $\theta$ определяет зависимость
значений хотя бы части компонентов этого параметра от того, в каком именно
составляющем распределении $f_\omega$ он присутствует. Кроме того, пусть
$P = \{P(\omega )\}$~--- \textit{семейство смешивающих функций}
распределения.

Функция плотности распределения
\begin{equation}
f(x) = \int f_\omega (x;\,\theta(\omega ))\,dP (\omega )
\end{equation}
называется $P$-\textit{смесью} (или просто \textit{смесью})
\textit{распределений} семейства~$F$,  интеграл в~(3) понимается в смысле
Лебега--Стильтьеса~\cite{3bat}.

\medskip
\textbf{Определение 3.} Семейство смесей~(3) называется
\textit{идентифицируемым} (\textit{различимым}), если из равенства
$$
\int f_\omega (x;\,\theta(\omega ))\,dP (\omega ) =\int f_\omega
(x,\,\theta(\omega )) dP^*(\omega )
$$
следует, что $P(\omega ) \equiv P^*(\omega )$ для всех $P \in P(\omega
)$~\cite{3bat}.

\subsection{Постановка задачи} %1.3.

Перед нами встает задача \textit{разделения} такой смеси. Вообще говоря,
задача разделения смесей распределений со смешивающими функциями
общего вида является \textit{некорректно поставленной}, т.к.\ она допускает
существование нескольких решений. Поэтому будем искать решение в классе
\textit{конечных идентифицируемых смесей распределений}, где смешивающая
функция дискретна.

Для этого сузим данное выше определение и будем рассматривать в дальнейшем лишь 
случай конечного числа $k$ возможных значений па\-ра\-мет\-ра~$\omega$, что 
соответствует конечному числу скачков смешивающих функций $P(\omega )$.  
Величины этих скачков как раз и будут играть роль \textit{удельных весов} 
(\textit{априорных вероятностей}) $p_j$ компонентов смеси. Более того, в нашем 
случае мы постулируем также однотипность компонентов распределений $f_j$, т.е.\ 
принадлежность всех $f_j$ к одному общему па\-ра\-мет\-ри\-че\-ско\-му 
семейству $\{ f(X;\,\theta )\}$, где $\theta$~--- многомерный, вообще говоря, 
параметр. Так что~(3) в этом случае может быть записано в виде
\begin{equation} %4
p(x) = \sum\limits^k_{j=1} p_j f_j (x;\,\theta_j )\,.
\end{equation}

Переформулируем понятие идентифицируемости (различимости) смесей
специально применительно к такому виду смесей.

\medskip
\textbf{Определение 4.} \textit{Конечная смесь}~(3) называется
\textit{идентифицируемой} (\textit{различимой}), если из равенства
$$
\sum\limits_{j=1}^k p_j f_j (x;\,\theta_j ) = \sum\limits_{l=1}^{k^*} p_l^* f_l
(x;\,\theta_l^* )
$$
следует, что $k=k^*$ и для любого $j$ ($1\leq j \leq k$) найдется такое $l$ 
($1\leq l \leq k^*$), что $p_j = p_l^*$ и $f_j (x;\,\theta_j ) = f_l 
(x;\,\theta_l^* )$~\cite{3bat}.

Решить эту задачу в выборочном варианте~--- значит по выборке
классифицируемых наблюдений
$X_1,\ldots , X_n, $ извлеченной из генеральной совокупности, яв\-ля\-ющей\-ся смесью~(3)
генеральных совокупностей типа~(2) (при заданном общем виде составляющих
смесь функций $f_j (x;\,\theta_j )$), построить статистические оценки для числа
компонентов смеси~$k$, их удельных весов $p_j$ и, главное, для каждого из
компонентов %f_j (x;\,\theta_j )$ анализируемой смеси. Далее будет считать, что
функции $f_j$ однозначно определяются своими параметрами $\theta_j$: $f_j
=f(x;\,\theta_j)$.

Однако не следует ставить знак тождества между задачей разделения смеси
и задачей статистического оценивания параметров в модели~(4) по выборке $
X_1,\ldots , X_n$, поскольку задача разделения сохраняет смысл и
применительно к генеральным совокупностям, т.е.\ в теоретическом
варианте~\cite{3bat}.

Итак, для статистического анализа на основе реальных данных мы
аппроксимируем нашу общую модель~(1) следующей:
$$
p(x) \approx \hat{p}(x) = \sum\limits_{j=1}^k p_j \g_{\lambda_j , \alpha_j}
(x)\,,
$$
где $p_j$~--- дискретные смешивающие параметры, $\g_{\lambda_j , \alpha_j}
(x)$~--- плотности гамма-распределений.

Такая аппроксимация не только позволяет решить поставленную статистическую
задачу, но и полу\-чить наглядную визуализацию результатов. Существуют
достаточно эффективные методики разделения смесей распределений, среди них~---
семейство так называемых \textit{ЕМ-алгоритмов}
(\textit{Expectation-Maximization Algorithms}).

Полученные результаты могут быть достаточно просто интерпретированы. Число
компонентов смеси символизирует число типичных параллельных или
последовательных структур. Значения параметров составляющих смесь
гам\-ма-рас\-пре\-де\-ле\-ний показывают <<степень параллелизма>>
соответствующей структуры: чем ближе параметр формы к~1, тем выше эта
<<степень>>. И наоборот, чем дальше значение параметра формы от~1, тем больше
последовательных операций выполняется в соответствующем блоке.

Веса компонентов характеризуют примерную долю использования
ресурсов для сообщений, соответствующих каждому распределению входных
данных.

Итак, предложенный подход позволяет получить представление о
стохастической структуре телекоммуникационной сети.

\section{ЕМ-алгоритм разделения смесей распределений}

\subsection{Описание алгоритма} %2.1.

Определяемый ниже итерационный алгоритм решения поставленной в
предыдущем разделе задачи относится к процедурам, базирующимся на
\textit{методе максимального правдоподобия}.

Этот алгоритм позволяет находить максимум логарифмической функции
правдоподобия по параметрам $p_1,\,p_2,\ldots ,\,p_k$, $\theta_1 ,\,\theta_2,\ldots ,\,
\theta_k$ при фиксированном $k$ по выборке $X_1, \ldots , X_n$, т.е.\ решение
оптимизационной задачи вида

\begin{equation} \sum\limits_{i=1}^n \ln \left ( \sum\limits_{j=1}^k p_j f_j
(X_i;\,\theta_j )\right ) \rightarrow \underset{p_j,\,\theta_j}{\max}\,.
\end{equation}

Конкретные алгоритмы, построенные по этой схеме, часто называют
\textit{алгоритмами типа ЕМ}, поскольку в каждом из них можно выделить два
этапа, находящихся по отношению друг к другу в последовательности
итерационного взаимодействия: \textit{оценивание} (\textit{Estimation}) и
\textit{максимизация} (\textit{Maximization})~\cite{4bat}.

Введем в рассмотрение так называемые апостериорные вероятности
$\g_{ij}$ принадлежности наблюдения $X_i$ к $j$-му классу:
\begin{equation} %6
\g_{ij} = \fr{p_j f(X_i;\,\theta_j )}{\sum\limits_{l=1}^k p_l f(X_i;\,\theta_l 
)} \ (i=1,\ldots , n;\ j=1,\ldots ,k)\,.\!\!\end{equation} 
Очевидно, что для 
всех $i=1,\ldots ,n$ и $j=1,\ldots ,k$
$$
\g_{ij} \geq 0,\quad \sum_{j=1}^k \g_{ij} =1\,.
$$


Далее обозначим $\Theta = (p_1,\ldots p_k,\,\theta_1,\ldots ,\theta_k )$ и
представим анализируемую логарифмическую функцию правдоподобия
$$
\ln L(\Theta ) = \sum\limits_{i=1}^n \ln \left (\sum\limits_{j=1}^k p_j f_j
(X_i;\,\theta_j )\right )
$$
в виде
\begin{multline}
\ln L (\Theta ) = \sum\limits_{j=1}^k\sum\limits_{i=1}^n \g_{ij} \ln p_j+{}\\
{}+\sum\limits_{j=1}^k\sum\limits_{i=1}^n \g_{ij} f(X_i;\,\theta_j)-
\sum\limits_{j=1}^k\sum\limits_{i=1}^n \g_{ij} \ln \g_{ij}\,.
\end{multline}

Справедливость этого тождества легко проверяется с учетом
$$
\sum\limits_{j=1}^k \g_{ij} =1\,.
$$

Далее идея построения итерационного алгоритма вычисления оценок
$\hat{\Theta} = (\hat{p}_1,\ldots , \hat{p}_k,\
\hat{\theta}_1,\ldots , \hat{\theta}_k)$
для параметров $\Theta = (p_1,\ldots , p_k,\ \theta_1,\ldots , \theta_k)$ состоит в
следующем:
\begin{enumerate}[1.]
\item Выбирается некоторое \textit{начальное приближение}~$\hat{\Theta}^0$.
\item \textbf{E-step:} вычисляются по формулам~(6) начальные приближения
$\g_{ij}^0$ для апостериорных вероятностей $\g_{ij}$~--- \textit{этап
оценивания}.
\item \textbf{M-step:} затем, возвращаясь к~(7), при вычисленных значениях
$\g^0_{ij}$ следует определить значения $\hat{\Theta}^1$ из условия
максимизации отдельно каждого из первых двух слагаемых правой
части~(7), поскольку первое слагаемое
$$
\sum_{j=1}^k \sum_{i=1}^n \g_{ij} \ln p_j
$$
зависит только от параметров $p_j$, а второе слагаемое
$$
\sum_{j=1}^k \sum_{i=1}^n \g_{ij} f(X_i;\,\theta_j )
$$
зависит только от параметров $\theta_j$~--- \textit{этап максимизации}.
\item Проверяется \textit{условие останова}:
$$
\parallel \Theta^{(t)} - \Theta^{t-1}\parallel <\varepsilon\,,
$$
где $t$~--- номер итерации, а
$\parallel\bullet\parallel$~--- евклидова норма, для некоторого $\varepsilon
>0$.
\end{enumerate}

Очевидно, решение оптимизационной задачи
$$
\sum\limits_{j=1}^k\sum\limits_{i=1}^n \g_{ij}^{(t)}\ln p_j \rightarrow
\underset{p_j}{\max}
$$
дается выражением (с учетом $\sum_{j=1}^k p_j =1$):
$$
p_{ij}^{(t+1)} =\fr{1}{n}\,\sum\limits_{i=1}^n \g_{ij}^{(t)}\,,
$$
где $t$~--- номер итерации, $t = 0$, 1, 2,\,\ldots

Решение оптимизационной задачи
$$
\sum\limits_{j=1}^k \sum\limits_{i=1}^n \g_{ij}^{(t)} f(X_i;\,\theta_j )
\rightarrow \underset{\theta_j}{\max}
$$
получить намного проще решения задачи~(5): выражение для $\theta_j$
записывается с учетом знания конкретного вида функций
$f(X,\,\theta)$~\cite{3bat}.

\subsection{О сходимости алгоритма} %2.2.

В работе М.\,И.~Шлезингера~\cite{5bat}, где эта схема (позднее названная
ЕМ-схемой) впервые предложена, установлены и основные свойства
реа\-ли\-зу\-ющих ее алгоритмов. В частности, было доказано, что при достаточно
широких предположениях \textit{предельные точки} всякой последовательности,
порожденной итерациями ЕМ-алгоритма, являются стационарными точками
оптимизируемой логарифмической функции правдоподобия $\ln L(\Theta )$ и что
найдется неподвижная точка алгоритма, к которой будет сходиться каждая из таких
последовательностей. Если дополнительно потребовать положительной
определенности информационной мат\-ри\-цы Фишера для $\ln L(\Theta )$ при
истинных зна\-че\-ни\-ях па\-ра\-мет\-ра $\Theta$, то можно показать, что
асимптотически по $n\rightarrow\infty$ (т.е.\ при больших выборках) существует
единственное сходящееся (по веро\-ят\-но\-сти) решение $\hat{\Theta}(n)$
уравнений метода максимального правдоподобия и, кроме того, существует в
пространстве параметров $\Theta$ норма, в которой последовательность
$\Theta^{(t)}(n)$, порожденная ЕМ-ал\-го\-рит\-мом, сходится к $\hat{\Theta}
(n)$, если только начальная аппроксимация $\hat{\Theta}^0$ не была слишком
далека от~$\hat{\Theta} (n)$. {%\looseness=1

}

Таким образом, результаты исследования свойств ЕМ-алгоритмов метода
максимального правдоподобия разделения смеси и их практическое
использование показали, что они являются достаточно работоспособными (при
известном чис\-ле компонентов смеси) даже при большом чис\-ле $k$ компонентов и
при высоких размерностях анализируемого признака~$X$~\cite{3bat}.

\subsection{Уравнения для смеси экспоненциальных распределений}
%2.3.

Применим описанный выше алгоритм к разделению смеси
экспоненциальных распределений:
$$
p(x) = \sum\limits_{j=1}^k p_j \lambda_j e^{-\lambda_j x}\,.
$$
Получаем следующие итерационные уравнения:
\begin{align*}
\g_{ij}^{(t+1)} & = \fr{p_j^{(t)}\lambda_j^{(t)}e^{-
\lambda_j^{(t)}X_i}}{\sum\limits_{l=1}^k p_l^{(t)}\lambda_l^{(t)}
e^{-\lambda_l^{(t)}X_i}}\,,\\
p_j^{(t+1)} & = \fr{1}{n}\,\sum\limits_{i=1}^n \g_{ij}^{(t)}\,.
\end{align*}

Чтобы найти  оценки $\lambda_j$, подсчитаем первую производную функции
$$\sum_{j=1}^k\sum_{i=1}^n \g_{ij}^{(t)} \ln (\lambda_j e^{-\lambda_j X_i}):$$
\vspace*{-8pt}
\begin{multline*}
\left ( \sum\limits_{j=1}^k \sum\limits_{i=1}^n
\g_{ij}^{(t)}\ln \left ( \lambda_j
e^{-\lambda_j X_i} \right ) \right )^\prime \lambda_j =\\[-3pt]
{}= \left (
\sum\limits_{j=1}^k\sum\limits_{i=1}^n \g_{ij}^{(t)}\ln (\lambda_j -\lambda_j X_i )
\right )^\prime \lambda_j =\\[-3pt]
{}= \sum\limits_{i=1}^n \g_{ij}^{(t)}\left (
\fr{1}{\lambda_j} - X_i \right )\,.
\end{multline*}

Разрешая уравнение
$$
\sum\limits_{i=1}^n \g_{ij}^{(t)}\left ( \fr{1}{\lambda_j} -X_i\right ) =0
$$
относительно $\lambda_j$, получаем следующее итерационное уравнение:
$$
\lambda_j^{(t+1)} = \fr{\sum\limits_{i=1}^n
\g_{ij}^{(t)}}{\sum\limits_{i=1}^n \g_{ij}^{(t)} X_i}\,.
$$

\subsection{Уравнения для смеси гамма-распределений } %2.4.

Применим теперь ЕМ-алгоритм к смеси гам\-ма-рас\-пре\-де\-ле\-ний вида
$$
p(x) = \sum\limits_{j=1}^k p_j \fr{\alpha_j^{\alpha_j} x^{\alpha_j -
1}}{\lambda_j^{\alpha_j} \Gamma (\alpha_j )}\,e^{-(\alpha_j / \lambda_j)x}\,.
$$

Такая параметризация удобна для нахождения
оценок~$\alpha_j$~\cite{6bat}.

Аналогичным способом выписываются итерационные уравнения:
\begin{align*}
\g_{ij}^{(t+1)} & =   \fr{p_j^{(t)}\fr{(\alpha_j^{\alpha_j} )^{(t)}
x^{\alpha_j - 1}}{(\lambda_j^{\alpha_j} )^{(t)}\Gamma (\alpha_j)}\,
e^{-(\alpha_j /\gamma_j)^{(t)}x}}{\sum\limits_{l=1}^k
p_l^{(t)}\fr{(\alpha_l^{\alpha_l})^{(t)} x^{\alpha_l -
1}}{(\lambda_l^{\alpha_l})^{(t)}\Gamma (\alpha_l )}\,
e^{-(\alpha_l /\lambda_l)^{(t)} x}}\,,\\
p_j^{(t+1)} & = \fr{1}{n}\,\sum\limits_{i=1}^n \g_{ij}^{(t)}\,.
\end{align*}

Далее найдем оценки $\lambda_j$ для данного случая, приравнивая
производную
\begin{equation} %8
\sum\limits_{j=1}^k \sum\limits_{i=1}^n \g_{ij}^{(t)} \ln \left (
\fr{\alpha_j^{\alpha_j} x^{\alpha_j -1}}{\lambda_j^{\alpha_j}\Gamma
(\alpha_j)}\,e^{-(\alpha_j /\lambda_j) x}\right )
\end{equation}
по $\lambda_j$ к нулю и разрешая относительно~$\lambda_j$ уравнение:
$$
\sum\limits_{i=1}^n \g_{ij}^{(t+1)}\left ( \fr{\alpha_j^{(t)}}{\lambda_j^{(t)}}
- \fr{\alpha_j^{(t)}X_i}{\left ( \lambda_j^{(t)}\right )^2}\right ) =0 \,.
$$
Получаем
$$
\lambda_j^{(t+1)} = \fr{\sum\limits_{i=1}^n \g_{ij}^{(t)}
X_i}{\sum\limits_{i=1}^n \g_{ij}^{(t)}}\,.
$$

Для того чтобы получить итерационные уравнения для $\alpha_j$, найдем
первую производную~(8):
\begin{multline*}
\left ( \sum\limits_{j=1}^k\sum\limits_{i=1}^n \g_{ij}^{(t)}\ln \left (
\fr{\alpha_j^{\alpha_j} x^{\alpha_j -1}}{\lambda_j^{\alpha_j}\Gamma (\alpha_j
)}\,e^{-(\alpha_j /\lambda_j ) x} \right ) \right )^\prime \alpha_j ={}\\[-3pt]
{}=\left ( \sum\limits_{j=1}^k\sum\limits_{i=1}^n \g_{ij}^{(t)}\ln \left (
\fr{\alpha_j^{\alpha_j}}{\lambda_j^{\alpha_j}}\right ) - \ln \Gamma (\alpha_j )+{} \right.\\[-3pt]
{}+\left.
(\alpha_j -1 )\ln X_i - \fr{\alpha_j}{\lambda_j}\,X_i \right )^\prime \alpha_j =\\[-3pt]
{}=\sum\limits_{i=1}^n \g_{ij}^{(t)} \left ( \ln \alpha_j +1-\ln \lambda_j -
\fr{\Gamma^\prime (\alpha_j )}{\Gamma (\alpha_j)}\right.+\\[-3pt]
{}+\left. \ln X_i - \fr{X_i}{\lambda_j}\right )\,;
\end{multline*}
\begin{multline*}
\sum\limits_{i=1}^n \g_{ij}^{(t)} \left(  \ln \alpha_j +1 -\ln \lambda_j -{}\right. \\[-3pt]
\left. {}-\fr{\Gamma^\prime (\alpha_j )}{\Gamma (\alpha_j )}+\ln X_i 
-\fr{X_i}{\lambda_j} \right) =0\,;
\end{multline*}
\begin{multline}
\fr{\Gamma^\prime (\alpha_j )}{\Gamma (\alpha_j )} ={}\\[-3pt]
{}= \fr{\sum\limits_{i=1}^n \g_{ij}^{(t)} \left ( \ln \alpha_j +1-\ln\lambda_j 
+\ln X_i -\fr{X_i}{\lambda_j} \right )}{\sum\limits_{i=1}^n \g_{ij}^{(t)}}\,.
\end{multline}
%
Здесь $\Gamma^\prime (\alpha_j ) / \Gamma (\alpha_j )$~--- это
\textit{логарифмическая производная гамма-функции}. Для нее существует так
называемое \textit{разложение Абрамовитца}--\textit{Стигана}~\cite{4bat}:
$$
\fr{\Gamma^\prime (\alpha ) }{ \Gamma (\alpha )} = \mathrm{log}\,\alpha -
\fr{1}{2\alpha }-\fr{1}{12\alpha^2 }+\ldots
$$

Подставим первые три члена разложения в~(9) и разрешим это уравнение
относительно~$\alpha_j$:
$$
\alpha_{ij}^{(t+1)} = \fr{\sum\limits_{i=1}^n
\g_{ij}^{(t+1)}}{2\sum\limits_{i=1}^n \g_{ij}^{(t +1)}\left ( \fr{X_i}{\lambda_j^{(t)}} -
\ln \fr{X_i}{\lambda_j^{(t)}} -1\right )}\,.
$$
В итоге получаем итерационные уравнения для ~$\alpha_j$.

\section{Описание программного обеспечения (программа~ЕМ)}

\subsection{Назначение программы} %3.1.

Разработанная авторами статьи программа ЕМ предназначена для решения задачи
разделения смесей экспоненциальных и гамма-распределений, поставленной в
разд.~2, с использованием ЕМ-ал\-го\-рит\-ма и формул, описанных в разд.~3.

\subsection{Инструменты разработки} %3.2.

Для создания программы была использована среда разработки Microsoft
Visual Studio .NET 2005 и объектно-ориентированный язык C\#. Для
визуализации результатов была использована свободно распространяемая
графическая библиотека ZedGraph~\cite{7bat}.


\subsection{Возможности  программы} %3.3.

\noindent
\begin{itemize}
\item Загрузка выборочных данных из текстового файла
\item Оценивание по выборке параметров смеси экспоненциальных
распределений
\item Оценивание по выборке параметров смеси гамма-распределений
\item Отслеживание изменений параметров смесей распределений во
времени в режиме <<скользящего окна>>
\item Построение гистограммы по выборке
\end{itemize}

\subsection{Входные и выходные данные. Функционирование
программы} %3.4.

В качестве \textit{входных данных} программа ЕМ получает:
\begin{itemize}
\item выборочные данные из текстового файла;
\item число компонентов смеси;
\item размер <<скользящего окна>>;
\item размер класса гистограммы.
\end{itemize}

На \textit{выходе} мы получаем:
\begin{itemize}
\item точечные оценки параметров смеси экспоненциальных
распределений;
\item точечные оценки параметров смеси гамма-распределений;
\item графическое изображение результирующей смеси распределения;
\item графическое изображение компонентов каж\-дой смеси;
\item графическое изображение того, как меняются параметры смесей
распределений с течением времени в режиме <<скользящего окна>>;
\item гистограмма, построенная по выборке;
\item значение статистического теста.
\end{itemize}

Выборочные данные загружаются из текстового файла в память программы и подаются
на вход двум независимо работающим реализациям ЕМ-алгоритма~--- для
идентификации смеси экспоненциальных распределений и для идентификации смеси
гамма-распределений. Результатом их работы являются наборы значений оцениваемых
параметров модели, предложенной в разд.~2. Кроме того, результирующие
распределения визуализируются в виде графиков. В программе можно запустить
режим <<скользящего окна>>, который для всех подвыборок заданного
размера с помощью ЕМ-алгоритма оценивает параметры смесей распределений этих
подвыборок. Все действия программы документируются в окне информации.

\section{Описание тестовых расчетов}

С использованием разработанной программы были проведены тестовые
расчеты на выборочных данных реального сетевого трафика.

На вход программы EM были поданы выборки трафика:
\begin{enumerate}[I]
\item Между лабораторией Lawrence Berkeley (Berkeley, California) и
внешним миром размера примерно 7000~\cite{8bat}~--- \textit{выборка~1}.
\item
Сети радиодоступа ЗАО <<Синтерра>> размера примерно 1000~\cite{9bat}~---
 \textit{выборка~2}.
\end{enumerate}

\subsection{Выборка 1 ``Berkeley''} %5.1.

При числе компонентов смеси~5 и случайном начальном приближении
были получены результаты, представленные в табл.~\ref{t1bat}.


Данные результаты иллюстрирует рис.~\ref{f5bat}.

Гистограмма  на рис.~\ref{f6bat} показывает статистическую значимость
полученных результатов.

Данная выборка обладает той особенностью, что она собиралась в течение
достаточно длительного времени и в ней агрегирован самый разнородный
трафик. Поэтому в ней присутствует не только большое количество
<<коротких>> сообщений (что обычно для выборок из телетрафика), но и
некоторый массив сообщений средней длины, а также определенный
<<выброс>> больших сообщений. Это свидетельствует о \textit{пиковости}
телетрафика на довольно больших промежутках времени.

Как мы видим, ЕМ-алгоритм удачно справился с задачей идентификации
смеси.

\subsection{Выборка~2 ``Synterra''} %5.2.

Результаты применения ЕМ-алгоритма к выборке ``Synterra''
представлены в табл.~\ref{t2bat}.
\begin{table*}\small
\begin{minipage}[t]{76mm}
\begin{center}
\Caption{Результаты применения ЕМ-алго\-рит\-ма к выборке~1 ``Berkeley'' 
\label{t1bat}} \vspace*{2ex}

\tabcolsep=8.7pt
\begin{tabular}{|c|c|c|}
\hline
№&Начальное приближение&Результат\\
\hline
\multicolumn{3}{|c|}{$P$}\\
\hline
0&0,2&0,1896\\
1&0,2&0,1858\\
2&0,2&0,1830\\
3&0,2&0,2259\\
4&0,2&0,2154\\
\hline
\multicolumn{3}{|c|}{$\alpha$}\\
\hline
0&2,7028&10,9783\hphantom{9}\\
1&3,6273&5,8621 \\
2&5,7598&2,7092\\
3&0,2315&1,0235\\
4&0,9110&0,4772\\
\hline
\multicolumn{3}{|c|}{$\lambda$}\\
\hline
0&85,2066&137,1714  \\
1&23,9592&136,7349\\
2&63,8425&132,6482\\
3&\hphantom{9}1,8026&116,7317\\
4&98,3882&102,5278\\
\hline
\end{tabular}
\end{center}
\end{minipage}\hfill
\begin{minipage}[t]{76mm}
%\end{table*}
%\begin{table*}\small
\begin{center}
\Caption{Результаты применения ЕМ-алго\-рит\-ма к выборке~2 ``Synterra'' 
\label{t2bat}} \vspace*{2ex}

\tabcolsep=8.7pt
\begin{tabular}{|c|c|c|}
\hline
№&Начальное приближение&Результат\\
\hline
\multicolumn{3}{|c|}{$P$}\\
\hline
0&0,2&$0{,}3815\hphantom{{}\cdot 10^{-9}}$\\
1&0,2&$0{,}3594\hphantom{{}\cdot 10^{-9}}$\\
2&0,2&$0{,}2589\hphantom{{}\cdot 10^{-9}}$\\
3&0,2&$0{,}4401\cdot 10^{-9}$\\
4&0,2&$0{,}0\hphantom{{}\cdot 10^{-9}999}$\\
\hline
\multicolumn{3}{|c|}{$\alpha$}\\
\hline
0&6,0804&1,5833\\
1&3,1838&0,8554\\
2&1,4886&0,4557\\
3&4,6407&0,2278\\
4&3,7843&0,1139\\
\hline
\multicolumn{3}{|c|}{$\lambda$}\\
\hline
0&17,3387&15,8682\\
1&47,8294&16,9150\\
2&54,1984&19,2866\\
3&\hphantom{1}8,6254&19,2866\\
4&\hphantom{1}5,7252&19,2866\\
\hline
\end{tabular}
\end{center}
\end{minipage}
\end{table*}


Данные результаты иллюстрируют рис.~\ref{f7bat}.


Эти результаты также отражают действительную картину, как показано на
рис.~\ref{f8bat}.


Этот трафик был снят с базовой станции <<Лукойл-Юго-Запад>> сети
широкополосного радиодоступа ЗАО <<Синтерра>>. Сеть радиодоступа
является реализацией так называемой <<последней мили>>, переносящей два
разных вида трафика: данные (Ethernet пакеты) и голос (IP-телефония, VoIP).
Поэтому здесь присутствуют в качестве основной массы короткие, но
интенсивные сообщения (пакеты SIP и голосовые фреймы), а также длинные
сообщения, содержащие данные.

Как мы видим, программная реализация ЕМ-ал\-го\-рит\-ма успешно справилась с
задачей разделения смесей распределений для этих двух выборок, что делает
данную программу удобным инструментом построения стохастической картины
конкретной сети. По полученным данным, используя метод интерпретации,
предложенный в разд.~2, можно получить представление о количестве
последовательных и параллельных структур вероятностной модели сети.

\subsection{Режим <<скользящего окна>>} %5.3.

Результаты для выборки
``Berkeley'' в режиме <<скользящего окна>>  представлены
на рис.~\ref{f9bat}.


Данные графики показывают изменение параметров распределений подвыборок выборки 
``Berkeley''. Видно, что параметры распределений подвыборок не остаются 
неизменными во времени, наоборот, они имеют внешне случайный характер. На 
рис.~\ref{f9bat},\,\textit{в} видна даже своеобразная пульсация первой 
компоненты.
%
На основании расчетов можно сделать вывод о том, что пиковость трафика
обусловливается как формой, так и интенсивностью сообщений.

\section{Заключение}

В данной работе исследована вероятностная модель  информационных потоков,
возникающих в сложных телекоммуникационных конвергентных сетях, построенная с
помощью асимптотического и энтропийного подходов. Эта модель предполагает, что
функционирование сложной телекоммуникационной сети можно представить в виде
суперпозиции довольно простых стохастических структур~--- последовательных и
параллельных, которые по\-рож\-да\-ют смеси гамма-распределений для случайной
величины времени обработки и передачи сообщений в сети. Предложена простая
интерпретация параметров данной модели.
\begin{figure*} %fig5
\vspace*{1pt}
\begin{center}
\mbox{%
\epsfxsize=130mm %145.109mm 
\epsfbox{bat-5.eps} }
\end{center}
\vspace*{-13pt} \Caption{Компоненты смеси начального приближения~(\textit{а}) и 
результата~(\textit{б}) для выборки~1 ``Berkeley'' \label{f5bat}}
%\end{figure*}
%\begin{figure*} %fig6
\vspace*{12pt}
\begin{center}
\mbox{%
\epsfxsize=130mm %148.256mm 
\epsfbox{bat-7.eps} }
\end{center}
\vspace*{-13pt} \Caption{График смеси распределений~(\textit{1}) и гистограмма 
для выборки~1 ``Berkeley''~(\textit{2}) \label{f6bat}}
\end{figure*}



\begin{figure*} %fig7
\vspace*{1pt}
\begin{center}
\mbox{%
\epsfxsize=130mm %144.283mm 
\epsfbox{bat-8.eps} }
\end{center}
\vspace*{-16pt} \Caption{Компоненты смеси начального приближения~(\textit{а}) и 
результата~(\textit{б}) для выборки~2 ``Synterra'' \label{f7bat}}
%\end{figure*}
%\begin{figure*} %fig8
\vspace*{12pt}
\begin{center}
\mbox{%
\epsfxsize=130mm %148.256mm 
\epsfbox{bat-10.eps} }
\end{center}
\vspace*{-11pt} \Caption{График смеси распределений~(\textit{1}) и гистограмма
для выборки~2 ``Synterra''~(\textit{2}) \label{f8bat}}
\end{figure*}

\begin{figure*} %fig9
\vspace*{1pt}
\begin{center}
\mbox{%
\epsfxsize=119.041mm
\epsfbox{bat-11.eps} }
\end{center}
\vspace*{-9pt} \Caption{Изменение  смешивающих параметров~(\textit{а}), 
параметров формы~(\textit{б}) и параметров масштаба~(\textit{в}) во времени для 
выборки~1 ``Berkeley'' \label{f9bat}}
\end{figure*}

Для решения вытекающей из модели задачи предложен итерационный алгоритм,
базирующийся на методе максимального правдоподобия~--- ЕМ-ал\-го\-ритм, для
которого получены формулы для конкретного вида смесей~--- экспоненциальных и
гамма-распределений.
%
Кроме того, разработан программный инструментарий для оценки параметров 
предложенной модели на выборках из реальных трафиковых данных. Проведены 
исследования, которые подтвердили предположения вероятностной модели. 


Получение информации о стохастической структуре
телекоммуникационных сетей и наличие программных инструментов для
выявления более или менее стабильных структур позволит понять причины
возникновения неожиданных больших нагрузок, предотвратить такие нагрузки,
а также поможет в будущем в проектировании надежных, оптимальных по
стоимости и уровню сервиса телекоммуникационных сетей нового поколения.

%\vspace*{-15pt} 
{\small\frenchspacing
{%\baselineskip=10.8pt
\addcontentsline{toc}{section}{Литература}
\begin{thebibliography}{9}
\bibitem{1bat}
Teletraffic Engeneering Handbook. International Telecommunication Union, 
Geneva, 2005 {\sf http://www.itu.int}. \vspace*{5pt} 
\bibitem{2bat}
\Au{Севастьянов~Б.\,А.} Курс теории вероятностей и математической статистики. 
М., 2004. \vspace*{5pt} 
\bibitem{3bat}
\Au{Айвазян~C.\,А., Бухштабер~В.\,М., Енюков~И.\,С, Мешалкин~Л.\,Д.} Прикладная 
статистика. Классификация и снижение размерности~// Финансы и статистика. М., 
1989. \vspace*{5pt} 
\bibitem{4bat}
\Au{Bilmes~J.\,A.} A gentle tutorial of the EM algorithm and its application to 
parameter estimation for Gaussian mixture and hidden Markov models. Berkeley, 
CA, USA: International Computer Science Institute,  1998. \vspace*{5pt} 
\bibitem{5bat}
\Au{Шлезингер~М.\,И.} О самопроизвольном различении образов~// Шлезингер~М.\,И. 
Читающие. автоматы. Киев: Наукова думка, 1965. С.~38--45. \vspace*{5pt} 
\bibitem{6bat}
\Au{Hsiao~I.-T., Rangarajan~A., Gindi~G.}. Joint-MAP 
reconstruction/segmentation for transmission tomography using mixture-models as 
priors. Yale University, 1998. \vspace*{5pt} 
\bibitem{7bat}
{\sf http://zedgraph.org}. \vspace*{4pt} 
\bibitem{8bat}
{\sf http://ita.ee.lbl.gov/html/contrib/LBL-PKT.html}. \vspace*{5pt} 
\bibitem{9bat}
{\sf http://www.synterra.ru}.
\end{thebibliography}

} } \label{end\stat}
\end{multicols}


%\addtocounter{razdel}{1}
%\def\razd{НЕРЕГУЛИРУЕМЫЙ ЭЛЕКТРОПРИВОД ДЛЯ ЭЛЕКТРОЭНЕРГЕТИКИ}

\setcounter{page}{2}

%   { %\Large  
   { %\baselineskip=16.6pt
   
   \vspace*{-48pt}
   \begin{center}\LARGE
   \textit{Предисловие}
   \end{center}
   
   %\vspace*{2.5mm}
   
   \vspace*{25mm}
   
   \thispagestyle{empty}
   
   { %\small 

    
Вниманию читателей журнала <<Информатика и её применения>> предлагается 
очередной тематический выпуск <<Вероятностно-статистические методы и 
задачи информатики и информационных технологий>>. Предыдущие тематические 
выпуски журнала по данному направлению вышли в 2008~г.\ (т.~2, вып.~2), 
в 2009~г.\ (т.~3, вып.~3) и в 2010~г.\ (т.~4, вып.~2). 

Статьи, собранные в данном журнале, посвящены разработке новых вероятностно-статистических 
методов, ориентированных на применение к решению конкретных задач информатики и информационных 
технологий, а также~--- в ряде случаев~--- и других прикладных задач. Проблематика, охватываемая 
публикуемыми работами, развивается в рамках научного сотрудничества между Институтом проблем 
информатики Российской академии наук (ИПИ РАН) и Факультетом вычислительной математики и 
кибернетики Московского государственного университета им.\ М.\,В.~Ломоносова в ходе работ 
над совместными научными проектами (в том числе в рамках функционирования 
Научно-образовательного центра <<Вероятностно-статистические методы анализа рисков>>). 
Многие из авторов статей, включенных в данный номер журнала, являются активными участниками 
традиционного международного семинара по проблемам устойчивости стохастических моделей, 
руководимого В.\,М.~Золотаревым и В.\,Ю.~Королевым; регулярные сессии этого семинара 
проводятся под эгидой МГУ и ИПИ РАН (в 2011~г.\ указанный семинар проводится в октябре 
в Калининградской области РФ). 

Наряду с представителями ИПИ РАН и МГУ в число авторов данного выпуска журнала входят 
ученые из Научно-исследовательского института системных исследований РАН, Института 
проблем технологии микроэлектроники и особочистых материалов РАН, Института 
прикладных математических исследований Карельского НЦ РАН, Московского 
авиационного института, Вологодского государственного педагогического университета, 
НИИММ им.\ Н.\,Г.~Чеботарева, Казанского государственного университета, Дебреценского 
университета (Венгрия).

Несколько статей выпуска посвящено разработке и применению стохастических методов и 
информационных технологий для решения различных прикладных задач. В~работе В.\,Г.~Ушакова 
и О.\,В.~Шестакова рассмотрена задача определения вероятностных характеристик случайных 
функций по распределениям интегральных преобразований, возникающих в задачах эмиссионной 
томографии. В~статье Д.\,О.~Яковенко и М.\,А.~Целищева рассмотрены некоторые вопросы 
математической теории риска и предложен новый подход к диверсификации инвестиционных 
портфелей. Работа И.\,А.~Кудрявцевой и А.\,В.~Пантелеева посвящена построению и 
исследованию математической модели, описывающей динамику сильноионизованной плазмы. 
В~статье П.\,П.~Кольцова изучается качество работы ряда алгоритмов сегментации изображений. 
Статья А.\,Н.~Чупрунова и И.~Фазекаша посвящена вероятностному анализу числа без\-оши\-бочных 
блоков при помехоустойчивом кодировании; получены усиленные законы больших чисел для указанных 
величин.

В данном выпуске традиционно присутствует тематика, весьма активно разрабатываемая в течение 
многих лет специалистами ИПИ РАН и МГУ,~--- методы моделирования и управления для 
информационно-телекоммуникационных и вычислительных систем, в частности методы 
теории массового обслуживания. В~статье А.\,И.~Зейфмана с соавторами рассматриваются 
модели обслуживания, описываемые марковскими цепями с непрерывным временем в случае 
наличия катастроф. В~работе М.\,М.~Лери и И.\,А.~Чеплюковой рассматриваются случайные 
графы Интернет-типа, т.\,е.\ графы, степени вершин которых имеют степенные распределения; 
такие задачи находят применение при исследовании глобальных сетей передачи данных. 
Работа Р.\,В.~Разумчика посвящена исследованию систем массового обслуживания специального 
вида~--- с отрицательными заявками и хранением вытесненных заявок.

Ряд статей посвящен развитию перспективных теоретических 
вероятностно-статистических методов, которые находят широкое применение в различных 
задачах информатики и информационных технологий. В~работе В.\,Е.~Бенинга, А.\,К.~Горшенина 
и В.\,Ю.~Королева рассмотрена задача статистической проверки гипотез о числе компонент 
смеси вероятностных распределений, приводится конструкция асимптотически наиболее мощного 
критерия. Результаты этой работы найдут применение в ряде прикладных задач, использующих 
математическую модель смеси вероятностных распределений (в информатике, моделировании 
финансовых рынков, физике турбулентной плазмы и~т.\,д.). В~статье В.\,Ю.~Королева, 
И.\,Г.~Шевцовой и С.\,Я.~Шоргина строится новая, улучшенная оценка точности нормальной 
аппроксимации для пуассоновских случайных сумм; как известно, указанные случайные суммы 
широко используются в качестве моделей многих реальных объектов, в том числе в информатике, 
физике и других прикладных областях. Работа В.\,Г.~Ушакова и Н.\,Г.~Ушакова посвящена 
исследованию ядерной оценки плотности распределения; эти результаты могут применяться, 
в част\-ности, при анализе трафика в телекоммуникационных системах. Серьезные приложения 
в статистике могут получить результаты работы О.\,В.~Шестакова, в которой доказаны оценки 
скорости сходимости распределения выборочного абсолютного медианного отклонения к нормальному 
закону. 

\smallskip

Редакционная коллегия журнала выражает надежду, что данный тематический  выпуск 
будет интересен специалистам в области теории вероятностей и математической статистики 
и их применения к решению задач информатики и информационных технологий.
     
     %\vfill 
     \vspace*{20mm}
     \noindent
     Заместитель главного редактора журнала <<Информатика и её 
применения>>,\\
     директор ИПИ РАН, академик  \hfill
     \textit{И.\,А.~Соколов}\\
     
     \noindent
     Редактор-составитель тематического выпуска,\\
     профессор кафедры математической статистики факультета\\
      вычислительной математики и кибернетики МГУ им.\ М.\,В.~Ломоносова,\\
     ведущий научный сотрудник ИПИ РАН,\\ 
доктор физико-математических наук \hfill
      \textit{В.\,Ю.~Королев}
     
     } }
     }



%   { %\Large  
   { %\baselineskip=16.6pt
   
   \vspace*{-48pt}
   \begin{center}\LARGE
   \textit{Предисловие}
   \end{center}
   
   %\vspace*{2.5mm}
   
   \vspace*{25mm}
   
   \thispagestyle{empty}
   
   { %\small 

    
Вниманию читателей журнала <<Информатика и её применения>> предлагается 
очередной тематический выпуск <<Вероятностно-статистические методы и 
задачи информатики и информационных технологий>>. Предыдущие тематические 
выпуски журнала по данному направлению вышли в 2008~г.\ (т.~2, вып.~2), 
в 2009~г.\ (т.~3, вып.~3) и в 2010~г.\ (т.~4, вып.~2). 

Статьи, собранные в данном журнале, посвящены разработке новых вероятностно-статистических 
методов, ориентированных на применение к решению конкретных задач информатики и информационных 
технологий, а также~--- в ряде случаев~--- и других прикладных задач. Проблематика, охватываемая 
публикуемыми работами, развивается в рамках научного сотрудничества между Институтом проблем 
информатики Российской академии наук (ИПИ РАН) и Факультетом вычислительной математики и 
кибернетики Московского государственного университета им.\ М.\,В.~Ломоносова в ходе работ 
над совместными научными проектами (в том числе в рамках функционирования 
Научно-образовательного центра <<Вероятностно-статистические методы анализа рисков>>). 
Многие из авторов статей, включенных в данный номер журнала, являются активными участниками 
традиционного международного семинара по проблемам устойчивости стохастических моделей, 
руководимого В.\,М.~Золотаревым и В.\,Ю.~Королевым; регулярные сессии этого семинара 
проводятся под эгидой МГУ и ИПИ РАН (в 2011~г.\ указанный семинар проводится в октябре 
в Калининградской области РФ). 

Наряду с представителями ИПИ РАН и МГУ в число авторов данного выпуска журнала входят 
ученые из Научно-исследовательского института системных исследований РАН, Института 
проблем технологии микроэлектроники и особочистых материалов РАН, Института 
прикладных математических исследований Карельского НЦ РАН, Московского 
авиационного института, Вологодского государственного педагогического университета, 
НИИММ им.\ Н.\,Г.~Чеботарева, Казанского государственного университета, Дебреценского 
университета (Венгрия).

Несколько статей выпуска посвящено разработке и применению стохастических методов и 
информационных технологий для решения различных прикладных задач. В~работе В.\,Г.~Ушакова 
и О.\,В.~Шестакова рассмотрена задача определения вероятностных характеристик случайных 
функций по распределениям интегральных преобразований, возникающих в задачах эмиссионной 
томографии. В~статье Д.\,О.~Яковенко и М.\,А.~Целищева рассмотрены некоторые вопросы 
математической теории риска и предложен новый подход к диверсификации инвестиционных 
портфелей. Работа И.\,А.~Кудрявцевой и А.\,В.~Пантелеева посвящена построению и 
исследованию математической модели, описывающей динамику сильноионизованной плазмы. 
В~статье П.\,П.~Кольцова изучается качество работы ряда алгоритмов сегментации изображений. 
Статья А.\,Н.~Чупрунова и И.~Фазекаша посвящена вероятностному анализу числа без\-оши\-бочных 
блоков при помехоустойчивом кодировании; получены усиленные законы больших чисел для указанных 
величин.

В данном выпуске традиционно присутствует тематика, весьма активно разрабатываемая в течение 
многих лет специалистами ИПИ РАН и МГУ,~--- методы моделирования и управления для 
информационно-телекоммуникационных и вычислительных систем, в частности методы 
теории массового обслуживания. В~статье А.\,И.~Зейфмана с соавторами рассматриваются 
модели обслуживания, описываемые марковскими цепями с непрерывным временем в случае 
наличия катастроф. В~работе М.\,М.~Лери и И.\,А.~Чеплюковой рассматриваются случайные 
графы Интернет-типа, т.\,е.\ графы, степени вершин которых имеют степенные распределения; 
такие задачи находят применение при исследовании глобальных сетей передачи данных. 
Работа Р.\,В.~Разумчика посвящена исследованию систем массового обслуживания специального 
вида~--- с отрицательными заявками и хранением вытесненных заявок.

Ряд статей посвящен развитию перспективных теоретических 
вероятностно-статистических методов, которые находят широкое применение в различных 
задачах информатики и информационных технологий. В~работе В.\,Е.~Бенинга, А.\,К.~Горшенина 
и В.\,Ю.~Королева рассмотрена задача статистической проверки гипотез о числе компонент 
смеси вероятностных распределений, приводится конструкция асимптотически наиболее мощного 
критерия. Результаты этой работы найдут применение в ряде прикладных задач, использующих 
математическую модель смеси вероятностных распределений (в информатике, моделировании 
финансовых рынков, физике турбулентной плазмы и~т.\,д.). В~статье В.\,Ю.~Королева, 
И.\,Г.~Шевцовой и С.\,Я.~Шоргина строится новая, улучшенная оценка точности нормальной 
аппроксимации для пуассоновских случайных сумм; как известно, указанные случайные суммы 
широко используются в качестве моделей многих реальных объектов, в том числе в информатике, 
физике и других прикладных областях. Работа В.\,Г.~Ушакова и Н.\,Г.~Ушакова посвящена 
исследованию ядерной оценки плотности распределения; эти результаты могут применяться, 
в част\-ности, при анализе трафика в телекоммуникационных системах. Серьезные приложения 
в статистике могут получить результаты работы О.\,В.~Шестакова, в которой доказаны оценки 
скорости сходимости распределения выборочного абсолютного медианного отклонения к нормальному 
закону. 

\smallskip

Редакционная коллегия журнала выражает надежду, что данный тематический  выпуск 
будет интересен специалистам в области теории вероятностей и математической статистики 
и их применения к решению задач информатики и информационных технологий.
     
     %\vfill 
     \vspace*{20mm}
     \noindent
     Заместитель главного редактора журнала <<Информатика и её 
применения>>,\\
     директор ИПИ РАН, академик  \hfill
     \textit{И.\,А.~Соколов}\\
     
     \noindent
     Редактор-составитель тематического выпуска,\\
     профессор кафедры математической статистики факультета\\
      вычислительной математики и кибернетики МГУ им.\ М.\,В.~Ломоносова,\\
     ведущий научный сотрудник ИПИ РАН,\\ 
доктор физико-математических наук \hfill
      \textit{В.\,Ю.~Королев}
     
     } }
     }

\def\stat{pechinkin}


\def\tit{СОВМЕСТНОЕ СТАЦИОНАРНОЕ РАСПРЕДЕЛЕНИЕ
ЧИСЛА ЗАЯВОК В~НАКОПИТЕЛЕ И~В~БУНКЕРЕ
ПЕРЕУПОРЯДОЧЕНИЯ В~МНОГОКАНАЛЬНОЙ СИСТЕМЕ
ОБСЛУЖИВАНИЯ С~ПЕРЕУПОРЯДОЧЕНИЕМ
ЗАЯВОК$^*$}


\def\titkol{Совместное стационарное распределение
числа заявок в~накопителе и~в~бункере
переупорядочения} % в~многоканальной системе обслуживания с~переупорядочением заявок}

\def\aut{\fbox{А.\,В.\~Печинкин}$^1$, Р.\,В.~Разумчик$^2$}

\def\autkol{А.\,В.\~Печинкин, Р.\,В.~Разумчик}

\titel{\tit}{\aut}{\autkol}{\titkol}

{\renewcommand{\thefootnote}{\fnsymbol{footnote}} \footnotetext[1]
{Работа выполнена при частичной поддержке РФФИ (проект 13-07-00223).}}


\renewcommand{\thefootnote}{\arabic{footnote}}
\footnotetext[1]{Институт проблем информатики Российской академии наук}
\footnotetext[2]{Институт проблем информатики Российской академии наук; Российский
университет дружбы народов, rrazumchik@ieee.org}

%\vspace*{3pt}

\Abst{Рассматривается функционирующая в~непрерывном времени
многоканальная система обслуживания с~накопителем
бесконечной емкости и переупорядочением заявок.
В~систему поступает пуассоновский поток заявок, время
обслуживания каждым прибором распределено по
экспоненциальному закону с~одним и~тем же параметром.
При поступлении в~систему всем заявкам  присваивается
порядковый номер. На выходе из системы сохраняется
порядок между заявками, установленный при входе в~нее.
Заявки, завершившие обслуживание и~нарушившие установленный порядок,
накапливаются на выходе системы
в~бункере переупорядочения (БП), который также имеет неограниченную емкость.
Найдено совместное стационарное распределение
числа заявок в~накопителе и~суммарного числа
заявок в~БП в~терминах
вычислительных алгоритмов и~производящих функций (ПФ).
Приведены примеры расчетов по полученным
соотношениям.}

\KW{многолинейная система массового обслуживания;
переупорядочение; стационарное распределение
числа заявок}

\DOI{10.14357/19922264140401}


%\vspace*{3pt}

\vskip 12pt plus 9pt minus 6pt

\thispagestyle{headings}

\begin{multicols}{2}

\label{st\stat}


\section{Введение}

Для функционирования ряда
ин\-фор\-ма\-ци\-он\-но-те\-ле\-ком\-му\-ни\-ка\-ци\-он\-ных сис\-тем
и для предоставления на их основе услуг
необходимо соблюдение\linebreak требования сохранения порядка в~потоке передаваемых сообщений.
Различные действия, необходимые для этого, можно объединить
в~одно понятие~--- переупорядочение.
Для изучения влияния\linebreak \mbox{переупорядочения} на качество
функционирования ин\-фор\-ма\-ци\-он\-но-те\-ле\-ком\-му\-ни\-ка\-ци\-он\-ных
сис\-тем к~настоящему времени предложено множество
моделей, которые в~своей основе используют методы
и~модели теории массового обслуживания.
Исследуемая сис\-те\-ма обычно представляется в~виде
системы или сети массового обслуживания с одним\linebreak или
несколькими входящими потоками сообщений.
Эффект переупорядочения часто моделируется с~помощью
дополнительной очереди (БП),
в~которую попадают сообщения, обработанные\linebreak в~системе,
и~ожидают там до тех пор, пока порядок следования сообщений
нельзя будет восстановить.
Некоторый обзор работ в~этом направлении можно найти
в~\cite{a1, a2},
а~некоторые последние результаты~--- в~[3--8].

Настоящая работа является развитием \cite{a8}, в~которой
рассматривается система массового обслуживания (СМО)
с~переупорядочением в~виде марковской многоканальной
системы обслуживания неограниченной емкости и~бункером
переупорядочения, также имеющим неограниченную
емкость.
В~\cite{a8} была получена система уравнений равновесия для
совместного стационарного распределения чис\-ла заявок в~системе
и~бункере переупорядочения и~приведены некоторые результаты
численных расчетов.
Однако несомненный интерес представляют
две задачи, не освещенные в~\cite{a8}, которые и~являются
предметом данной статьи, а~именно:
разработка рекуррентного алгоритма расчета вышеупомянутого
совместного стационарного распределения и~нахождение
этого распределения в~терминах ПФ.

Статья организована таким образом.
В~разд.~2 приводится подробное описание
системы.
В~разд.~3 дается рекуррентный алгоритм расчета
совместного стационарного распределения, а~в~разд.~4
показано, как совместное стационарное распределение
можно найти в~терминах ПФ.
Примеры расчетов, проведенных по формулам разд.~4,
представлены в~разд.~5.
В~заключении сформулированы основные результаты работы.

\section{Описание системы}

Рассмотрим функционирующую в~непрерывном времени
$N$-ли\-ней\-ную ($N\hm\ge 2$) СМО с накопителем
неограниченной емкости, входящим пуассоновским
потоком заявок интенсивности~$\lambda$ \mbox{и~экспоненциальным}
распределением времени
обслуживания заявки каждым прибором с~па\-ра\-мет\-ром~$\mu$.


При поступлении в~систему всем заявкам  присваивается
порядковый номер.
На выходе из СМО сохраняется порядок между заявками,
установленный при входе в~нее.
Заявки, завершившие обслуживание и~нарушившие
установленный порядок, накапливаются на выходе
системы в~БП и~покидают СМО только
после того, как закончится обслуживание всех заявок с~меньшими номерами.
Такая СМО носит название системы с переупорядочением
заявок.

Предполагается также выполнение необходимого и~достаточного условия
существования стационарного режима функционирования СМО
$$\tilde {\rho}\hm=\fr{\rho}{N}<1\,,
$$
 где $\rho\hm=\lambda/\mu$.

\vspace*{-9pt}

\section{Алгоритм нахождения совместного стационарного распределения}

Предположим, что на приборах находится $n$, $n\hm=\overline{1,N}$, заявок.
Тогда заявкой первого уровня будем называть ту из них,
которая в~систему поступила последней, второго уровня~--- предпоследней,
$\ldots,$ $n$-го уровня~--- первой. При этом если $n\hm=N$ (все приборы
заняты), то находящиеся в~БП заявки, поступившие между заявками
второго и~первого уровней, будем называть заявками первой очереди,
заявки, поступившие между заявками третьего и~второго уровней,~---
заявками второй очереди, $\ldots,$ заявки, поступившие между
заявками $N$-го и~$(N-1)$-го уровней,~--- заявками $(N-1)$-й
очереди. Если же $n<N$, то  заявками первой очереди будем называть
заявки из БП, поступившие после заявки первого уровня, заявками
второй очереди~--- заявки, поступившие между заявками второго и~первого уровней,
и~т.\,д.

При $n\ge N$ обозначим через
$p^{(m)}_{n;i}$, ${m\hm=\overline{1,N-1}}$, ${i\hm\ge 0}$,
стационарную вероятность того, что в~системе на
приборах и~в накопителе находится~$n$~заявок,
а~в~БП имеется в~сумме~$i$~заявок первой,
второй, $\ldots,$ $m$-й очереди.
Через
$p^{(m)}_{n;i}$, ${m\hm=\overline{1,n}}$, ${i\hm\ge 0}$,
обозначим аналогичную стационарную вероятность
при $n\hm=\overline{1,N-1}$.
Через~$p_n$, $n\hm\ge 0$, обозначим
стационарную вероятность того, что в~системе на
приборах и~в накопителе (без учета числа заявок в~БП) находится~$n$~заявок.
Очевидно, что стационарные вероятности~$p_n$
определяются теми же самыми формулами, что и~в~обычной
марковской СМО $M/M/N/\infty$
(см., например,~\cite{boch}):
\begin{align}
p_{0} &= \left( \sum\limits_{i=0}^{N-1} \fr{\rho^i}{i!} +
\fr{\rho^N}{(N-1)! (N-\rho)}
\right)^{-1} \,;\label{3-1}
\\
p_{i} &= \begin{cases}
\fr{\rho^i }{i!} p_{0}\,, &\ i=\overline{1,N}\,,
\\
%\label{3-3}
\fr{\rho^i}{N!\, N^{i-N}} p_{0}
= \tilde \rho^{i-N} p_{N}\,, &\ i\ge N+1\,.
\end{cases}
\label{3-2}
\end{align}

Наконец, через $p_{n;i}$, ${n\hm\ge 1}$, ${i\hm\ge 0}$, обозначим
стационарную вероятность того, что в~системе на
приборах и~в накопителе находится~$n$~заявок,
а~в~БП~--- $i$~заявок.

Используя принцип глобального баланса, можно выписать систему уравнений для
вероятностей~$p^{(m)}_{n;i}$.
Для вероятностей $p^{(1)}_{n;i}$, $n\hm\ge N$,
$i \hm\ge 0$, справедливы уравнения:
\begin{align}
\hspace*{-2.8mm}p^{(1)}_{n;0} (\lambda+N\mu) &= p^{(1)}_{n-1;0} \lambda +
p_{n+1} (N-1) \mu \,,\ n\ge N;
\!\!\label{eq-1-1}
\\
\hspace*{-2.8mm}p^{(1)}_{n;i} (\lambda+N\mu) &= p^{(1)}_{n-1;i} \lambda +
p^{(1)}_{n+1;i-1} \mu \,,\notag\\
&\hspace*{25mm} n\ge N\,,\enskip i \ge 1\,.
\label{eq-1-2}\!\!
\end{align}
%%%%%%%%%%%%%%%%%%%%%%%
%%%%%%%%%%%%%%%%%%%%%%%
Для вероятностей $p^{(1)}_{N-1;i}$,\ \ $i \ge 0$,
справедливы уравнения:
%%%%%%%%%%%%%%%%%%%
\begin{align}
\label{eq-1-3}
p^{(1)}_{N-1;0} [\lambda+(N-1)\mu] &=
p_{N-2} \lambda + p_{N} (N-1)\mu\,;
\\
\label{eq-1-4}
p^{(1)}_{N-1;i} [\lambda+(N-1)\mu] &= p^{(1)}_{N;i-1} \mu\,,\enskip i \ge 1\,.
\end{align}
Для вероятностей
$p^{(1)}_{n;i}$, $n\hm=\overline{1,N-2}$, $i \hm\ge 0$,
справедливы уравнения
\begin{align}
\label{eq-1-5}
\hspace*{-2mm}p^{(1)}_{n;0} (\lambda+n\mu) &= p_{n-1} \lambda +
p^{(1)}_{n+1;0} n\mu ,\  n=\overline{1,N-2};
\\
\label{eq-1-6}
\hspace*{-2mm}p^{(1)}_{n;i} (\lambda+n\mu) &= p^{(1)}_{n+1;i} n\mu
+ p^{(2)}_{n+1;i-1} \mu \,,\notag\\
&\hspace*{15mm}n=\overline{1,N-2},\ \ i \ge 1.
\end{align}


Для остальных вероятностей
$p^{(m)}_{n;i}$, $m\hm=\overline{2,N-1}$, справедливы формулы:
\begin{align}
p^{(m)}_{n;0} (\lambda+N\mu) &= p^{(m)}_{n-1;0} \lambda +
p^{(m-1)}_{n+1;0} (N-m) \mu\,,\notag\\
& \hspace*{30mm}n\ge N\,; \label{bat-1}
\\
p^{(m)}_{n;i} (\lambda+N\mu) &= p^{(m)}_{n-1;i} \lambda +
p^{(m-1)}_{n+1;i} (N-m) \mu +{}\notag\\
&\hspace*{-10mm}{}+p^{(m)}_{n+1;i-1} m \mu \,,\enskip
n\ge N\,,\ \ i\ge 1\,;
\label{bat-2}
\end{align}

\noindent
\begin{align}
p^{(m)}_{N-1;0} [\lambda+(N-1)\mu] &={}\notag\\
{}=p^{(m-1)}_{N-2;0} \lambda
&{}=+ p^{(m-1)}_{N;0} (N-m) \mu \,;
\label{bat-3}
\end{align}

\noindent
\begin{multline}
p^{(m)}_{N-1;i} [\lambda+(N-1)\mu] =p^{(m-1)}_{N-2;i} \lambda+{}\\
{}+
p^{(m-1)}_{N;i} (N-m) \mu +p^{(m)}_{N;i-1} m \mu\,,\enskip i\ge 1\,;
\label{bat-4}
\end{multline}

\vspace*{-12pt}



\noindent
\begin{multline}
\label{bat-5}
p^{(m)}_{n;0} (\lambda+n\mu) = p^{(m-1)}_{n-1;0} \lambda+
p^{(m)}_{n+1;0} (n-m+1) \mu \,,\\
 n=\overline{m,N-2}\,;
\end{multline}

\noindent
\begin{multline}
\label{bat-6}
p^{(m)}_{n;i} (\lambda+n\mu) = p^{(m-1)}_{n-1;i} \lambda
+
p^{(m)}_{n+1;i} (n-m+1) \mu +{}\\
{}+ p^{(m+1)}_{n+1;i-1} m \mu\,,\enskip
 n=\overline{m,N-2}\,,\ \ i\ge 1\,.
\end{multline}

Решение данной системы уравнений позволяет
найти совместное стационарное распределение
$p_{n;i}$ числа заявок на приборах и~в
накопителе и~суммарного числа заявок в~БП в~виде следующих ра\-венств:
\begin{alignat*}{2}
%\label{bat-7}
p_{n;i} &= p^{(N-1)}_{n;i}\,, &\quad  n&\ge N\,,\ \ i\ge 0\,,
\\
%\label{bat-8}
p_{n;i} &= p^{(n)}_{n;i} \,, &\quad n&=\overline{1,N-1}\,,\ \ i\ge 0\,.
\end{alignat*}

Анализ системы~\eqref{eq-1-1}--\eqref{bat-6}
показал, что вычисление стационарных
вероятностей $p^{(m)}_{n;i}$ можно проводить
рекуррентным образом по следующему алгоритму.

\bigskip

\noindent
А\,л\,г\,о\,р\,и\,т\,м~1\ (\textbf{Алгоритм решения системы уравнений равновесия}).

\noindent
\textit{Задать} $\lambda$, $\mu$ и $n$.

\noindent
\textit{Для $n\ge 0$ рассчитать $p_{n}$ по
формулам}~\eqref{3-1} и~\eqref{3-2}.

\noindent
\textit{Рассчитать $p^{(1)}_{N-1;0}$ по формуле}~\eqref{eq-1-3}.

\noindent
\textit{Для $n\ge N$ рассчитать $p^{(1)}_{n;0}$ по
формуле}~\eqref{eq-1-1}.

\noindent
\textit{Для $i\ge1$}


\textit{рассчитать $p^{(1)}_{N-1;i}$ по формуле}~\eqref{eq-1-4}.

\textit{для $n\ge N$ рассчитать $p^{(1)}_{n;i}$ по формуле}~\eqref{eq-1-2}.

\noindent
\textit{Для $n=\overline{N-2,1}$ рассчитать $p^{(1)}_{n;0}$
по формуле}~\eqref{eq-1-5}.

\noindent
\textit{Для $m=\overline{2,N-1}$}

\textit{рассчитать $p^{(m)}_{N-1;0}$ по формуле}~\eqref{bat-3}.


\textit{для $n\ge N$ рассчитать $p^{(m)}_{n;0}$
   по формуле}~\eqref{bat-1};

\textit{для} $i\hm\ge1$

    \hspace*{9pt}\textit{рассчитать $p^{(1)}_{N-m;i}$ по
    формуле}~\eqref{eq-1-6};


    \hspace*{9pt}\textit{если $m \ne 2$, для}  $j\hm=\overline{2,m-1}$ \textit{рассчитать}\linebreak\vspace*{-12pt}

 \hspace*{9pt}\textit{$p^{(j)}_{N-m+j-1;i}$ по формуле}~\eqref{bat-6};

\hspace*{9pt}\textit{рассчитать $p^{(m)}_{N-1;i}$ по формуле}~\eqref{bat-4};

\hspace*{9pt}\textit{для $n\ge N$ рассчитать $p^{(m)}_{n;i}$
    по формуле}~\eqref{bat-2};

\textit{если {$m \ne N-1$}, для $m\hm=\overline{N-2,m}$
   рассчитать}\linebreak

   \textit{$p^{(m)}_{n;0}$ по формуле}~\eqref{bat-5}.

\bigskip

В~связи с~тем, что вычисление моментов после расчета
вероятностей по представленному алгоритму
может давать погрешности, в~следующем разделе
находятся формулы для совместного стационарного
распределения в~терминах ПФ.


\section{Использование производящих функций}

Система уравнений~\eqref{eq-1-1}--\eqref{bat-6}
допускает также решение с~помощью ПФ.
Для нахождения этого решения положим
\begin{equation*}
\label{f-m}
f_m(u,z) = \lambda u^2 - (\lambda + N\mu) u + m \mu z\,,\
 m=\overline{1,N-1}\,.
\end{equation*}

Обозначим через $u_m\hm=u_m(z)$, $m\hm=\overline{1,N-1}$,
минимальное решение уравнения
$$
f_m(u,z) = 0\,,
$$
т.\,е.
\begin{equation*}
%\label{sqrt}
u_m = \fr{\lambda + N\mu - \sqrt{(\lambda + N\mu)^2 - 4 m \lambda \mu z}}
{2 \lambda }\,.
\end{equation*}


Введем ПФ
\begin{multline*}
P^{(m)}_{n}(z) = \sum\limits_{i=0}^{\infty}
z^{i} p^{(m)}_{n;i}\,, \\
0<z<1\,, \ \ n\ge1\,,\ \
m=\overline{1,\min(n,N-1)} \,;
\end{multline*}

\vspace*{-12pt}


\noindent
\begin{multline*}
P^{(m)}(u,z) = \sum\limits_{n=N}^{\infty} u^{n-N} P^{(m)}_{n}(z)\,, \\
0<u,z<1\,, \ \ m=\overline{1,N-1}\,,
\end{multline*}
и, кроме того, положим
$$
P(u) = \sum\limits_{n=N}^{\infty} u^{n-N} p_{n}
= \fr{1}{1 - \tilde{\rho} u}\, p_N \,.
$$

Тогда, умножая~\eqref{eq-1-1} и~\eqref{eq-1-2}
на~$z^i$ и~суммируя по всем~$i$ от нуля до
бесконечности, получаем:
\begin{multline*}
%\label{eq-z-1}
(\lambda+N\mu) P^{(1)}_{n}(z) =
\lambda P^{(1)}_{n-1}(z) +
(N-1) \mu p_{n+1}
+ {}\\
{}+\mu z P^{(1)}_{n+1}(z)\,,\enskip n\ge N\,.
\end{multline*}
Умножая последнее выражение на $u^{n-N}$ и~суммируя по всем значениям $n\hm\ge N$,
после приведения подобных слагаемых имеем:
\begin{multline}
\label{eq-z-2}
f_1(u,z) P^{(1)}(u,z) =
\mu z P^{(1)}_{N}(z) -{}\\
{}- \lambda u P^{(1)}_{N-1}(z) -
(N-1) \mu [P(u) - p_{N}] \,.
\end{multline}


Теперь умножим \eqref{bat-1} и~\eqref{bat-2}
на~$z^i$ и~просуммируем по всем значениям $i\hm\ge0$.
В~результате приходим к~выражению:
\begin{multline*}
%\label{bat-2*}
(\lambda+N\mu) P^{(m)}_{n}(z) = \lambda P^{(m)}_{n-1}(z)
+{}\\
{}+(N-m) \mu P^{(m-1)}_{n+1}(z) +
m \mu z P^{(m)}_{n+1}(z) \,,\enskip n\ge N\,.
\end{multline*}
Умножая последнее выражение на $u^{n-N}$, после
суммирования по всем $n\hm\ge N$ получаем:

\pagebreak

\noindent
\begin{multline}
\label{bat-2*}
f_m(u,z) P^{(m)}(u,z) = m \mu z P^{(m)}_{N}(z)
- \lambda u P^{(m)}_{N-1}(z) -{}\\
{}-
(N-m) \mu [P^{(m-1)}(u,z) - P^{(m-1)}_{N}(z)]\,,\\ m=\overline{2,N-1}\,.
\end{multline}

Из уравнений~\eqref{eq-1-3} и~\eqref{eq-1-4}
после умножения на~$z^i$ и~суммирования по
всем значениям $i \hm\ge 0$ находим:
\begin{multline}
\label{eq-z-3}
P^{(1)}_{N-1}(z)=\fr{\lambda p_{N-2} + (N-1)\mu p_{N}}
{\lambda+(N-1)\mu }+{}\\
{}+ \fr{\mu z}{\lambda+(N-1)\mu} \,P^{(1)}_N(z)\,.
\end{multline}

Действуя аналогичным образом
с~уравнениями~\eqref{bat-3} и~\eqref{bat-4}, как и~с~уравнениями~\eqref{eq-1-3}
и~\eqref{eq-1-4}, приходим к выражению:
\begin{multline}
\label{bat-4*}
P^{(m)}_{N-1}(z) = \fr{ \lambda P^{(m-1)}_{N-2}(z) + (N-m) \mu P^{(m-1)}_{N}(z)
}{\lambda+(N-1)\mu }+{}\\
{}+\fr{m \mu z}{\lambda+(N-1)\mu}\,P^{(m)}_{N}(z) \,,\enskip m=\overline{2,N-1}\,.
\end{multline}


Домножая уравнения~\eqref{eq-1-5} и~\eqref{eq-1-6}
на~$z^i$, после суммирования по всем
значениям $i \hm\ge 0$ имеем:
\begin{multline}
\label{eq-z-4}
P^{(1)}_{n}(z)= \fr{ \lambda p_{n-1} + n \mu P^{(1)}_{n+1}(z) }{
\lambda+n\mu }+ \fr{\mu z}{\lambda+n\mu}\,P^{(2)}_{n+1}(z) \,,\\
n=\overline{1,N-2}\,.
\end{multline}

Наконец, производя аналогичные преобразования
с~уравнениями~\eqref{bat-5} и~\eqref{bat-6}, получаем:
\begin{multline}
\label{bat-6*}
P^{(m)}_n(z)= \fr {\lambda P^{(m-1)}_{n-1}(z) +
(n-m+1) \mu P^{(m)}_{n+1}(z)} {\lambda+n\mu}
+{}
\\
{}+
\fr{m \mu z}{\lambda+n\mu} P^{(m+1)}_{n+1}(z)\,,\enskip
m=\overline{2,N-2}\,,\\
n=\overline{m,N-2}\,.
\end{multline}

Уравнения~\eqref{eq-z-2}--\eqref{bat-6*} позволяют
находить выражения для всех
ПФ $P^{(m)}_{n}(z)$, $m\hm=\overline{1,N-1}$,
$n\hm=\overline{1,N-1}$, а~так\-же совместное
стационарное распределение рекуррентным образом.
Подставляя выражение для $P^{(1)}_{N-1}(z)$ из
формулы~\eqref{eq-z-3} в~формулу~\eqref{eq-z-2}, получаем:
\begin{multline}
P^{(1)}(u,z) = \left(
\left[
\mu z - \fr{\lambda \mu z u}{\lambda+(N-1)\mu}
\right] P^{(1)}_N(z) -{}\right.\\
{}-
\left[
\lambda u \fr{\lambda p_{N-2} + (N-1)\mu p_{N}}{\lambda+(N-1)\mu}+{}\right.\\
\left.\left.{}+
 (N-1) \mu [P(u) - p_{N}]
\vphantom{\fr{\lambda p_{N-2} + (N-1)\mu p_{N}}{\lambda+(N-1)\mu}}\right]
\right)
\Bigg /
f_1(u,z)\,,
\label{m25}
\end{multline}
откуда из равенства нулю в~точке $u_1(z)$ числителя и~знаменателя
правой части формулы~\eqref{m25} следует:
\columnbreak


%%%%%%%%%%%%%%%%%%%%%%%%%%%
\noindent
\begin{multline*}
%\label{r1}
P^{(1)}_N(z)= \left(
\lambda u_1(z) [\lambda p_{N-2} + (N-1)\mu p_{N}]
+{}\right.\\
{}+
\left.(\lambda+(N-1)\mu)(N-1) \mu \left[P(u_1(z)) - p_{N}\right]\right)\!\!\Big/\!\!
\left(\mu z \left[\lambda+{}\right.\right.\\
\left.\left.{}+(N-1)\mu  - \lambda u_1(z)\right]\right)\,.
\end{multline*}
%%%%%%%%%%%%%%%%%%%%%%%%%%%%%%%%%%%%%%%%%%
%%%%%%%%%%%%%%%%%%%%%%%%%%%%%%%%%%%%%%%%
Теперь, возвращаясь к~формуле~\eqref{eq-z-3},
получаем выражение для $P^{(1)}_{N-1}(z)$:
\begin{multline*}
%\label{r2}
P^{(1)}_{N-1}(z)=
\left([\lambda p_{N-2} + (N-1)\mu p_{N}]+{}\right.\\
\left.{}
+ (N-1) \mu [P(u_1(z)) - p_{N}]\right)\Big /
\left(\lambda+(N-1)\mu  - {}\right.\\
\left.{}-\lambda u_1(z)\right)\,.
\end{multline*}

Далее из равенства~\eqref{eq-z-4} выражаем $P^{(1)}_{N-2}(z)$ через
$P^{(2)}_{N-1}(z)$. Из равенства~\eqref{bat-4*} выражаем
$P^{(2)}_{N-1}(z)$ через $P^{(2)}_{N}(z)$. Подставляя полученное
выражение для $P^{(2)}_{N-1}(z)$ в~формулу~\eqref{bat-2*}, из
равенства нулю в~точке~$u_2$ левой и~правой части получившегося
равенства находим $P^{(2)}(u,z)$. Затем из равенства~\eqref{eq-z-4}
выражаем $P^{(1)}_{N-3}(z)$ через $P^{(2)}_{N-2}(z)$ и~т.\,д.

Продолжая эту процедуру, можно найти
соотношения для вычисления всех
ПФ $P^{(m)}_{n}(z)$, $m\hm=\overline{1,N-1}$, $n\hm=\overline{1,N-1}$.

С каждым шагом выражение для очередной ПФ становится все сложнее,
и~в итоге при большом числе приборов выписать явный вид всех ПФ не
удается. Тем не менее нахождение значений ПФ в~каждой точке $z \hm\ne
0$ можно свести к последовательному решению систем линейных
уравнений. Для этого обозначим через $A_n(z)$, $n\hm =\overline{2,N-1}$,
мат\-ри\-цы размера $(n+1)\times (n+1)$, име\-ющие
следующую структуру:
\begin{gather*}
\setcounter{MaxMatrixCols}{3}
A_2(z)=
\begin{pmatrix}
 2 \mu z   & 0  & -2 \mu z         \\
 - \lambda u_2(z) & - \mu z   &  \lambda +(N-1) \mu       \\
0  & \lambda +(N-2)\mu &    - \lambda
\end{pmatrix}\,;
\end{gather*}

\vspace*{-12pt}

\noindent
{ %\scriptsize
\begin{multline*}
\setcounter{MaxMatrixCols}{7}
A_n(z)=\left(
\begin{matrix}
 n \mu z   & 0  & - n \mu z &      \!\cdots\!          \\
 - \lambda u_n(z) \! & 0  & \! \lambda +(N-1) \mu \! & \!\cdots\!  \\
  \vdots   & \vdots & \vdots &  \!\cdots\! \\
 0   & 0 & 0&  \!\cdots\!  \\
 0      & 0 & 0    &     \!\cdots\! \\
 0   & - \mu z  &0   &  \cdots\! \\
0 & \!\lambda +(N-n)\mu \!&0  &  \!\cdots\!
\end{matrix}\right.\\
\left.\begin{matrix}
    \cdots\!     & 0    & 0       \\
    \cdots\!  & 0 & 0 \\
    \cdots\! & \vdots     & \vdots  \\
    \cdots\!  & - 3 \mu z     & 0   \\
    \cdots\! & \! \lambda +(N-n+2) &-2\mu z\\
    \cdots\! & - \lambda  & \! \lambda+(N-n+1)\mu\\
    \cdots\!  & 0  & - \lambda
\end{matrix}\right)\,,
\\ n =\overline{3,N-1}\,.
\end{multline*}
}

\noindent
Определим вектор-стр$\acute{\mbox{о}}$\-ки $\vec{a}_n(z)$ и~$\vec{b}_n(z)$
длины $(n+1)$ следующим образом:
\begin{multline*}
\vec{a}_n(z) = \left (
P^{(n)}_{N}(z), P^{(n)}_{N-1}(z), \dots\right.\\
\left.\dots,  P^{(2)}_{N-n+1}(z), P^{(1)}_{N-n}(z)
\right )\,,\enskip
n =\overline{2,N-1}\,;
\end{multline*}

\vspace*{-12pt}
\noindent
\begin{multline*}
\vec{b}_2(z) = \left (
(N-2) \mu [P^{(1)}(u_2,z) - P^{(1)}_{N}(z)] ,
\lambda p_{N-3}+{}\right.\\
\left.{}+ (N-2) \mu P^{(1)}_{N-1}(z),
(N-2) \mu P^{(1)}_{N}(z) \right)\,;
\end{multline*}

\vspace*{-12pt}

\noindent
\begin{multline*}
\vec{b}_n(z) = \left (
(N-n) \mu
[P^{(n-1)}(u_n,z) - P^{(n-1)}_{N}(z)],\right.
\\
\lambda p_{N-1-n}+ (N-n) \mu P^{(1)}_{N-1-(n-2)}(z),\\
(N-n) \mu P^{(n-1)}_{N}(z), (N-n)\mu  P^{(n-1)}_{N-1}(z),
\dots ,
\\
\left.
(N-n)\mu  P^{(3)}_{N-n+3}(z), (N-n)\mu  P^{(2)}_{N-n+2}(z)
\right )\,,\\
n =\overline{3,N-1}\,.
\end{multline*}
Тогда алгоритм нахождения ПФ состоит в~последовательном начиная с~$n\hm=2$ решении
системы линейных уравнений
$$
\vec{a}_n(z) A_n(z) = \vec{b}_n(z) \,.
$$
Из структуры матрицы $A_n(z)$, $n \hm=\overline{3,N-1}$, видно, что
она неприводима и~обладает свойством диагонального преобладания
т.\,е.\ перестановкой строк и~столбцов можно добиться того,
что в~каждой строке модуль диагонального элемента будет либо строго
больше, либо не меньше суммы модулей всех остальных элементов в~строке.
Покажем это. Если определить матрицы перестановки~$P^L_n$ и~$P^R_n$
размера $(n+1)\times (n+1)$ при $n \hm=\overline{3,N-1}$
следующим образом:
\begin{gather*}
\setcounter{MaxMatrixCols}{5}
P^L_n=
\begin{pmatrix}
 0   & 0  & \cdots & 0& 1 \\
 1   & 0  & \cdots & 0& 0 \\
  \vdots   &  \vdots  & \cdots &  \vdots &  \vdots \\
  0   & 0  & \cdots & 0& 0 \\
   0   & 0  & \cdots & 1& 0
\end{pmatrix}\,;
\enskip
\setcounter{MaxMatrixCols}{5}
P^R_n=
\begin{pmatrix}
 0   & 1  & \cdots & 0& 0         \\
 1   & 0  & \cdots & 0& 0 \\
   \vdots   &  \vdots  & \cdots &  \vdots &  \vdots \\
  0   & 0  & \cdots & 1& 0 \\
   0   & 0  & \cdots & 0& 1
\end{pmatrix}\,,
\end{gather*}
то матрица $P^L_n A_n(z)P^R_n$, $n \hm=\overline{3,N-1}$,
примет вид:
\begin{multline*}
\setcounter{MaxMatrixCols}{7}
P^L_n A_n(z)P^R_n={}\\
{}=\left(
\begin{matrix}
 \lambda +(N-n)\mu &0 & 0  & \cdots\\
 0  &  n \mu z   & - n \mu z &  \cdots       \\
  0  & - \lambda u_n(z) & \lambda +(N-1) \mu  & \cdots  \\
  \vdots   & \vdots & \vdots & \cdots   \\
 0   & 0 & 0& \cdots \\
 0      & 0 & 0    & \cdots  \\
 - \mu z  & 0   &0   & \cdots
\end{matrix}\right.
\end{multline*}

\noindent
\begin{equation*}
\hspace*{15mm}\left.\begin{matrix}
\cdots  & 0  & - \lambda\\
\cdots        & 0    & 0       \\
\cdots    & 0    & 0      \\
\cdots   & \vdots     & \vdots       \\
\cdots  & - 3 \mu z     & 0       \\
\cdots   & \lambda +(N-n+2) \mu & - 2 \mu z       \\
\cdots      & - \lambda  & \lambda +(N-n+1)\mu
\end{matrix}\right).
\end{equation*}
Легко видеть, что в~каждой строке модуль диагонального
элемента либо строго больше, либо не меньше суммы
модулей всех остальных элементов в~строке.
Тогда, как вытекает из следствия~6.2.27 в~\cite{horn},
у~матрицы $A_n(z)$ существует обратная
и,~значит, система $\vec{a}_n(z) A_n(z) \hm= \vec{b}_n(z)$
при $z\hm\neq 0$ имеет единственное решение.

\vspace*{-4pt}

\section{Примеры расчетов}

На основе полученных в~разд.~4 результатов {были} проведены расчеты
среднего и~дисперсии чис\-ла заявок в~БП,
а~также коэффициента корреляции числа заявок в~накопителе и~числа
заявок в~БП для различного чис\-ла
приборов~$N$~и~значений загрузки системы $\rho/N$. \mbox{Напомним}, что аналогичные
показатели были рассчитаны в~\cite{a8} по определению, на основе
стационарных вероятностей, рассчитанных по приведенному выше
алгоритму. Далее можно видеть, что результаты, полученные с~по\-мощью
ПФ, как и~ожидалось, полностью совпадают с~результатами,
представленными в~\cite{a8}.

На рис.~1 отражено поведение значения среднего числа заявок
в~БП в~зависимости от загрузки системы $\rho/N$.
Отметим, что полученные в~предыдущих  разделах результаты позволяют
рассчитывать такие
 характеристики, как среднее число заявок только
в~первой очереди в~БП, в~сумме в~первой и~во второй очередях в~БП
(когда обе очереди существуют), в~сумме в~первой, второй,\ldots ,
$(N-1)$-й очере-\linebreak\vspace*{-12pt}

\vspace*{6pt}

\begin{center}  %fig1
\vspace*{2pt}
\mbox{%
 \epsfxsize=75.145mm
 \epsfbox{pec-1.eps}
 }
\end{center}

\noindent
{{\figurename~1}\ \ \small{Поведение
 среднего числа заявок в~БП в~зависимости от загрузки
системы  $\rho/N$: \textit{1}~--- $N\hm=4$; \textit{2}~--- 7;
\textit{3}~--- $N=9$}}

%\vspace*{9pt}


\addtocounter{figure}{1}


\begin{center}  %fig2
\vspace*{2pt}
 \mbox{%
 \epsfxsize=75.027mm
 \epsfbox{pec-2.eps}
 }
 \end{center}

\noindent
{{\figurename~2}\ \ \small{Поведение среднего числа заявок в~первой
очереди в~БП~(\textit{1}), в~сумме в~первой и~во второй очередях в~БП~(\textit{2}),
в~сумме в~первой, второй и~третьей очередях в~БП~(\textit{3})
в~зависимости от загрузки системы $\rho/N$. Число
приборов $N\hm=4$}}

\vspace*{18pt}


\begin{center}  %fig3
\vspace*{2pt}
 \mbox{%
 \epsfxsize=74.929mm
 \epsfbox{pec-3.eps}
 }
 \end{center}

\noindent
{{\figurename~3}\ \ \small{Поведение
 дисперсии числа заявок в~БП в~зависимости от загрузки
системы  $\rho/N$: \textit{1}~--- $N\hm=4$; \textit{2}~--- 7; \textit{3}~--- $N=9$}}

\vspace*{18pt}

\begin{center}  %fig4
\vspace*{2pt}
 \mbox{%
 \epsfxsize=75.192mm
 \epsfbox{pec-4.eps}
 }
 \end{center}

\noindent
{{\figurename~4}\ \ \small{Поведение
 коэффициента корреляции числа заявок в~накопителе и~числа
заявок в~БП в~зависимости от загрузки системы  $\rho/N$:
\textit{1}~--- $N\hm=4$; \textit{2}~--- 7; \textit{3}~--- $N=9$}}


%\vspace*{9pt}


\noindent
дях в~БП (когда каждая из очередей существует).
Поведение данных характеристик в~зависимости от загрузки системы
$\rho/N$ для случая $N\hm=4$ пред\-став\-ле\-но на рис.~2.

На рис.~3 и~4 изображено поведение дисперсии числа
заявок в~БП и~поведение
коэффициента корреляции числа заявок в~накопителе и~числа
заявок в~БП соответственно.

Во всех расчетах интенсивность обслуживания заявок~$\mu$ принималась
равной~1.

%\addtocounter{figure}{1}
%%%%%%%%%%%%%%%%%%%%%%%%%%%%%%%%%%%%%%%%%%%%%%%%%%%%%

Анализируя графики на рис.~1--4, стоит отметить два момента. Среднее
число заявок в~БП не уходит в~бесконечность с ростом загрузки
(и~даже при загрузке больше единицы), что следует из формулы Литтла.
Число заявок в~накопителе и~число заявок в~БП весьма слабо
коррелированы, и~с~рос\-том числа приборов коэффициент корреляции
уменьшается.

\section{Заключение}

В настоящей работе рассмотрена функционирующая в~непрерывном времени
многоканальная система обслуживания с~накопителем бесконечной емкости
и~переупорядочением заявок.
В~систему поступает пуассоновский поток заявок, время
обслуживания каждым прибором распределено по
экспоненциальному закону с~одним и~тем же параметром.
Для нахождения совместного стационарного распределения
числа заявок в~накопителе и~суммарного числа
заявок в~БП получен рекуррентный алгоритм.
Также показано, как можно находить совместное распределение
в~терминах ПФ, которые облегчают расчет его моментов.

{\small\frenchspacing
 {%\baselineskip=10.8pt
 \addcontentsline{toc}{section}{References}
 \begin{thebibliography}{99}
 \bibitem{a1} %1
\Au{Boxma O., Koole G., Liu~Z.}
Queueing-theoretic solution methods for
models of parallel and distributed systems~//
Performance Evaluation of Parallel and Distributed Systems Solution
Methods, 1994. CWI Tract~105 and~106. P.~1--24.

\bibitem{a2} %2
\Au{Dimitrov B.}
Queues with resequencing. A~survey and recent results~//
{2nd World Congress on Nonlinear Analysis,
Theory, Methods, Applications Proceedings}, 1997. Vol.~30. No.\,8. P.~5447--5456.

\bibitem{a3} %3
\Au{Huisman T., Boucherie R.\,J.}
The sojourn time distribution in an infinite server
resequencing queue with dependent interarrival and
service times~// J.~Appl. Probab., 2002.
Vol.~39. No.\,3. P.~590--603.

\bibitem{a5} %4
\Au{Xia Y., Tse D.\,N.\,C.}
On the large deviations of resequencing
queue size: 2-$M$/$M$/1 сase~// IEEE Trans. Inform. Theory, 2008.
Vol.~54. No.\,9. P.~4107--4118.

\bibitem{a4} %5
\Au{Leung K., Li V.\,O.\,K.}
A~resequencing model for high-speed packet-switching networks~//
J.~Comput. Commun., 2010.
Vol.~33. No.\,4. P.~443--453.

\bibitem{a7} %6
\Au{Матюшенко С.\,И.} Стационарные характеристики двухканальной
системы обслуживания с~переупорядочением заявок и~распределениями
фазового типа~// Информатика и~её применения, 2010. Т.~4. Вып.~4.
С.~67--71.

\bibitem{a6} %7
\Au{De Nicola C., Pechinkin A.\,V., Razumchik~R.\,V.}
Stationary characteristics of homogenous Geo/Geo/2
queue with resequencing in discrete time~//
27th European Conference on Modelling and
Simulation Proceedings.~---- Aalesund, 2013. P.~594--600.

\bibitem{a7+} %8
\Au{Pechinkin A.\,V., Caraccio~I., Razumchik~R.\,V.}
Joint stationary distribution of queues in
homogenous $M\vert M\vert$3 queue with resequencing~//
28th European Conference on
Modelling and Simulation Proceedings.~--- Brescia, 2014. P.~558--564.

\bibitem{a8}
\Au{Pechinkin A.\,V., Caraccio~I., Razumchik~R.\,V.}
On joint stationary distribution in exponential
multiserver reordering queue~// 12th  Conference (International) on
Numerical Analysis and Applied Mathematics Proceedings, 2014 (in press).

\bibitem{boch}
\Au{Bocharov P.\,P., D'Apice C., Pechinkin~A.\,V., Salerno~S.}
Queueing theory.~--- Urecht, Boston: VSP, 2004. 446~p.

\bibitem{horn}
\Au{Horn R.\,A., Johnson C.\,R.}
Matrix analysis.~--- 2nd ed.~--- Cambridge: Cambridge University Press, 2013.
662~p.
 \end{thebibliography}

 }
 }

\end{multicols}

\vspace*{-9pt}

\hfill{\small\textit{Поступила в редакцию 28.10.14}}

%\newpage

\vspace*{12pt}

\hrule

\vspace*{2pt}

\hrule

%\vspace*{12pt}

\def\tit{JOINT STATIONARY DISTRIBUTION OF~THE~NUMBER OF~CUSTOMERS IN~THE~SYSTEM
AND REORDERING BUFFER IN~THE~MULTISERVER REORDERING QUEUE}

\def\titkol{Joint stationary distribution of~the~number of~customers in~the~system
and reordering buffer in~the~multiserver reordering queue}



\def\aut{\fbox{A.\,V.~Pechinkin}$^1$ and R.\,V.~Razumchik$^{1,2}$}

\def\autkol{A.\,V.~Pechinkin and R.\,V.~Razumchik}

\titel{\tit}{\aut}{\autkol}{\titkol}

\vspace*{-9pt}

\noindent
$^1$Institute of Informatics Problems, Russian Academy of Sciences,
44-2 Vavilov Str., Moscow 119333, Russian\\
$\hphantom{^1}$Federation


\noindent
$^2$Peoples' Friendship University of Russia,
6~Miklukho-Maklaya Str., Moscow 117198, Russian Federation



\def\leftfootline{\small{\textbf{\thepage}
\hfill INFORMATIKA I EE PRIMENENIYA~--- INFORMATICS AND
APPLICATIONS\ \ \ 2014\ \ \ volume~8\ \ \ issue\ 4}
}%
 \def\rightfootline{\small{INFORMATIKA I EE PRIMENENIYA~---
INFORMATICS AND APPLICATIONS\ \ \ 2014\ \ \ volume~8\ \ \ issue\ 4
\hfill \textbf{\thepage}}}

\vspace*{3pt}



\Abste{The paper considers a continuous-time multiserver queueing
system with buffer on infinite capacity and reordering. The Poisson
flow of customers arrives at the system. Service times of customers at
each server are exponentially distributed with the same parameter.
Each customer obtains a~sequential number upon arrival. The order of
customers upon arrival should be preserved upon departure from the system.
Customers whose service finished but which violated the order are kept in
the reordering buffer of infinite capacity. A~joint stationary distribution
of the number of customers in the buffer, servers, and
reordering buffer is obtained in terms of a~computational algorithm and
a~generating function. A~numerical example is provided.}


\KWE{queueing system; reordering; infinite capacity; joint distribution}

\DOI{10.14357/19922264140401}

%\vspace*{3pt}

\Ack
\noindent
The research was partially financially supported by the Russian Foundation for
Basic Research (project 13-07-00223).


  \begin{multicols}{2}

\renewcommand{\bibname}{\protect\rmfamily References}
%\renewcommand{\bibname}{\large\protect\rm References}



{\small\frenchspacing
 {%\baselineskip=10.8pt
 \addcontentsline{toc}{section}{References}
 \begin{thebibliography}{99}


 \bibitem{a1-1}
\Aue{Boxma O., G. Koole, and Z.~Liu}. 1994.
Queueing-theoretic solution methods for
models of parallel and distributed systems.
\textit{Performance Evaluation of Parallel and
Distributed Systems Solution Methods}.  CWI Tract 105
and 106:1--24.

\bibitem{a2-1}
\Aue{Dimitrov, B.} 1997.
Queues with resequencing. A~survey and recent results.
\textit{2nd World Congress on Nonlinear
Analysis, Theory, Methods, Applications Proceedings}. 30(8):5447--5456.

\bibitem{a3-1}
\Aue{Huisman, T., and R.\,J.~Boucherie}. 2002.
The sojourn time distribution in an infinite server
resequencing queue with dependent interarrival and service times.
\textit{J.~Appl. Probab}. 39(3):590--603.

\bibitem{a5-1}
\Aue{Xia, Y., and D.\,N.\,C.~Tse}. 2008.
On the large deviations of resequencing
queue size: 2-$M$/$M$/1 case.
\textit{IEEE Trans. Inform. Theory} 54(9):4107--4118.

\bibitem{a4-1} %5
\Aue{Leung, K., and V.\,O.\,K.~Li}. 2010.
A~resequencing model for high-speed
packet-switching networks.
\textit{J.~ Comput. Commun.} 33(4):443--453.

\bibitem{a7-1} %6
\Aue{Matyushenko, S.\,I.} 2010.
 Statsionarnye kharakteristiki
dvukh\-ka\-nal'\-noy sistemy obsluzhivaniya s~pe\-re\-upo\-rya\-do\-chi\-va\-ni\-em zayavok
i~raspredeleniyami
fazovogo tipa [Stationary characteristics of the two-channel
queueing system with reordering customers and distributions of phase type].
\textit{Informatika i ee Primemeniya}~--- \textit{Inform. Appl.}
4(4):67--71.

\bibitem{a6-1} %7
\Aue{De Nicola, C., A.\,V.~Pechinkin, and R.\,V.~Razumchik}. 2013.
Stationary characteristics of homogenous Geo/Geo/2
queue with resequencing in discrete time.
\textit{27th European Conference
on Modelling and Simulation Proceedings}. Aalesund. 594--600.

\bibitem{a7+-1}
\Aue{Pechinkin, A.\,V., I.~Caraccio, and R.\,V.~Razumchik}. 2014.
joint stationary distribution of queues
in homogenous $M \vert M \vert3$ queue with resequencing.
\textit{28th European Conference
on Modelling and Simulation Proceedings}. Brescia. 558--564.

\bibitem{a8-1}
\Aue{Pechinkin, A.\,V., I.~Caraccio, and R.\,V.~Razumchik}. 2014 (in press).
On joint stationary distribution in exponential
multiserver reordering queue.
\textit{12th  Conference (International) on
Numerical Analysis and Applied Mathematics Proceedings}.

\bibitem{boch-1}
\Aue{Bocharov,  P.\,P., C.~D'Apice, A.\,V.~Pechinkin, and S.~Salerno}. 2004.
\textit{Queueing theory}. Urecht, Boston: VSP. 446~p.

\bibitem{horn-1}
\Aue{Horn, R.\,A., and C.\,R.~Johnson}. 2013.
\textit{Matrix analysis}. Cambridge: Cambridge University Press. 662~p.
\end{thebibliography}

 }
 }

\end{multicols}

\vspace*{-6pt}

\hfill{\small\textit{Received October 28, 2014}}

\vspace*{-18pt}

\Contr

\noindent
\textbf{Pechinkin Alexander V.} (1946--2014)~--- Doctor
of Science in physics and mathematics; principal
scientist, Institute of Informatics Problems of
the Russian Academy of Sciences, 44-2 Vavilov Str.,
Moscow 119333, Russian Federation


\vspace*{3pt}

\noindent
\textbf{Razumchik Rostislav V.} (b.\ 1984)~--- Candidate
of Science (PhD) in physics and mathematics,
senior scientist, Institute of Informatics
Problems of the Russian Academy of Sciences, 44-2 Vavilov Str.,
Moscow 119333, Russian Federation;
associate professor,
Peoples' Friendship University of Russia,
6~Miklukho-Maklaya Str., Moscow 117198, Russian Federation;
rrazumchik@ieee.org


\label{end\stat}

\renewcommand{\bibname}{\protect\rm Литература} %1
\newcommand{\R}{\mathbb R}
\newcommand{\dd}{3}
\renewcommand{\d}{1}
\newcommand{\betr}{\beta_{3}}
\newcommand{\Lo}{\fr{\betr}{\sqrt{n}}}
\newcommand{\Ll}{\fr{\betr+1}{\sqrt{n}}}
\newcommand{\pto}{\stackrel{P}{\longrightarrow}}

\def\stat{gavr}

\def\tit{УТОЧНЕНИЕ НЕРАВНОМЕРНОЙ ОЦЕНКИ СКОРОСТИ СХОДИМОСТИ
РАСПРЕДЕЛЕНИЙ ПУАССОНОВСКИХ СЛУЧАЙНЫХ СУММ К НОРМАЛЬНОМУ
ЗАКОНУ$^*$}

\def\titkol{Уточнение неравномерной оценки скорости сходимости
распределений пуассоновских случайных сумм % к нормальному закону
}

\def\autkol{С.\,В.~Гавриленко}
\def\aut{С.\,В.~Гавриленко$^1$}

\titel{\tit}{\aut}{\autkol}{\titkol}

{\renewcommand{\thefootnote}{\fnsymbol{footnote}}\footnotetext[1]
{Работа выполнена при поддержке Министерства
образования и науки (государственный контракт 16.740.11.0133 от
02.09.2010).}}

\renewcommand{\thefootnote}{\arabic{footnote}}
\footnotetext[1]{Московский государственный университет им.\ М.\,В.~Ломоносова, 
факультет вычислительной математики и кибернетики,
gavrilenko.cmc@gmail.com}


\Abst{Строятся неравномерные оценки скорости сходимости
в классической центральной предельной теореме с уточненной
структурой. С~помощью этих структурных уточнений показано, что
абсолютная константа в неравномерной оценке скорости сходимости в
центральной предельной теореме (ЦПТ) для пуассоновских случайных сумм
строго меньше, чем аналогичная константа в неравномерной оценке
скорости сходимости в классической ЦПТ,
и при условии существования третьих моментов слагаемых не
превосходит 22,7707. В~качестве следствия построены неравномерные
оценки скорости сходимости смешанных пуассоновских, в частности
отрицательных биномиальных случайных сумм.}

\KW{центральная предельная теорема; скорость
сходимости; неравномерная оценка; абсолютная константа;
пуассоновская случайная сумма; смешанное пуассоновское
распределение}

      \vskip 14pt plus 9pt minus 6pt

      \thispagestyle{headings}

      \begin{multicols}{2}
      
            \label{st\stat}

\section{Введение}

Пусть $X_1,X_2,\ldots$~--- последовательность независимых одинаково
распределенных случайных величин таких, что ${\sf E}X_1=0$, ${\sf
E}X_1^2=1$, ${\sf E}|X_1|^{\dd}=\betr<\infty$. Положим $F_n(x)={\sf
P}\big(X_1+\ldots+X_n<x \sqrt{n}\big)$. Пусть $\Phi(x)$~--- стандартная нормальная функция распределения, т.\,е.\
$$
\Phi(x) = \fr{1}{\sqrt{2\pi}}\int_{-\infty}^x e^{-t^2/2}\,dt\,.
$$

Известно, что при указанных условиях существуют абсолютные
положительные конечные константы $C_0$ и $C_1$ такие, что~[1, 2]
\begin{equation}
\sup_x|F_n (x)-\Phi(x)|\le C_0 \Lo
\label{e1gv}
\end{equation}
 и~[3, 4]:
\begin{multline}
\sup_x\left|F_n(x)-\Phi(x)\right|\le C_1\Ll={}\\
{}=
C_1\left(1+\fr{1}{\betr}\right)\Lo\label{e2gv}\,.
\end{multline}
 Для констант $C_0$ и $C_1$ известны следующие численные
оценки~[4, 5]:
\begin{equation*}
0{,}4097\approx\fr{\sqrt{10}+3}{6\sqrt{2\pi}}\le C_0\le
0{,}4784
%\label{e3gv}
\end{equation*}
и~[3, 4]:
\begin{equation*}
0{,}2659\approx\fr{2}{3\sqrt{2\pi}}\le C_1\le 0{,}3041\,.
%\label{e4gv}
\end{equation*}
При этом, поскольку всегда $\betr\ge1$, при больших значениях
$\betr$ оценка~(\ref{e2gv}) точнее, чем~(\ref{e1gv}), за счет меньших значений
абсолютных констант.

Оценка скорости сходимости $F_n(x)$ к $\Phi(x)$,
устанавливаемая неравенствами~(\ref{e1gv}) и~(\ref{e2gv}), \textit{равномерна} по~$x$.
Но поскольку и $F_n(x)$, и $\Phi(x)$~--- функции распределения, то
должно выполняться соотношение $|F_n(x)-\Phi(x)|\rightarrow 0$ при
$|x|\to0$. Это обстоятельство не учитывается в равномерных
оценках. Вместе с тем точность нормальной аппроксимации для
функции распределения сумм случайных величин именно при больших
значениях аргумента представляет особый интерес, например, при
вычислении рисков критически больших потерь. В~данной статье будут
рассмотрены неравномерные оценки скорости сходимости в центральной
предельной теореме.

По-видимому, исторически первая оценка такого рода была получена в
работе~\cite{Meshalkin}, где для $\delta=1$, то есть для случая
существования третьего момента слагаемых, было доказано
существование конечной положительной абсолютной постоянной~$A$
такой, что для любого $x\in\R$ справедливо неравенство
$$
(1+x^2)|F_n(x)-\Phi(x)|\le A\fr{\beta_3}{\sqrt{n}}\,.
$$

Этот результат был усилен в работе~\cite{Nagaev}, где было
показано, что существует такое положительное конечное число~$C$,
что
\begin{equation}
\sup_x\left(1+|x|^{\dd}\right)\left|F_n(x)-\Phi(x)\right|\le
C\Lo\,.\label{e5gv}
\end{equation}
При этом для рассматриваемых условий на моменты слагаемых порядок
оценки~(\ref{e5gv}) по $x$ неулучшаем без дополнительных предположений.

Что касается значения абсолютной константы~$C$ в~(\ref{e5gv}), то в работе~\cite{Mich81} 
было показано, что $C\le C_0+8(1+e)$, что с учетом
оценки $C_0\le 0{,}4784$, полученной в~\cite{KorolevBEs, KorSchev}, влечет оценку $C\le
30{,}2247$. Недавно эта оценка была уточнена в работе~\cite{Nefedova},
где было показано, что $C \le 25{,}7984$.

В данной работе с помощью модификации метода Л.~Падитца~\cite{Paditz89} будут построены альтернативные неравномерные
оценки скорости сходимости в центральной предельной теореме,
имеющие структуру, аналогичную неравенству~(\ref{e2gv}). Полученные оценки
затем будут использованы для уточнения абсолютной константы в
аналоге неравенства~(\ref{e5gv}) для пуассоновских случайных сумм.

Согласно результатам работы~\cite{Mich93}, в качестве абсолютной
константы в неравномерной оценке скорости сходимости в центральной
предельной тео\-ре\-ме для пуассоновских случайных сумм можно брать
абсолютную константу~$C$ из неравенства~(\ref{e5gv}). В~предлагаемой статье
с использованием упомянутых выше структурных уточнений неравенства~(\ref{e5gv})\linebreak 
будет показано, что на самом деле абсолютная константа в
неравномерной оценке ско\-рости схо\-ди\-мости в центральной предельной
теореме для пуассоновских случайных сумм строго меньше, чем\linebreak
аналогичная константа в неравенстве~(\ref{e5gv}), и не превосходит
22,7707. С~помощью этих результатов затем будут построены
неравномерные оценки скорости сходимости распределений смешанных
пуассоновских, в частности отрицательных биномиальных, случайных
сумм.

\section{Неравномерные оценки скорости сходимости в~центральной предельной теореме с~уточненной структурой}

Идея, лежащая в основе метода построения неравномерных оценок
точности нормальной аппроксимации для распределений сумм независимых
случайных величин, описанного в работе~\cite{Paditz89}, заключается
в подходящем разбиении вещественной прямой на зоны <<малых>>,
<<умеренных>> и <<больших>> значений~$x$. Традиционно используются
разбиения следующего вида:
\begin{enumerate}[$i$]
\item
<<малые>> значения $x$: $0\le x^2\le K^2$;
\item
<<умеренные>> значения $x$: $K^2\le x^2\le c_n(x;a,b)$;
\item
<<большие>> значения $x$: $c_n(x;a,b)\le x^2<\infty$,
\end{enumerate}
где $K>0$, $a>0$, $b>1$~--- вспомогательные свободные
параметры,
$$
c_n(x;a,b)=2b\left[\log|x|^{\dd}-\log\left(a\Lo\right)\right]
$$
(см., в частности,~\cite{Paditz89, Rychlik}).

\subsection{Случаи \boldmath{$i$} и \boldmath{$iii$}, то есть <<малые>> и~<<большие>> значения \boldmath{$x$}}

В случае~$i$, т.\,е.\ для $0\le |x|\le K$, в соответствии с
неравенством~(\ref{e2gv}) имеем
\begin{multline}
|x|^{\dd}\left|F_n(x)-\Phi(x)\right|\le
C_1K^{\dd}\Ll={}\\
{}=C_1K^{\dd}\Lo+C_1K^{\dd}\fr{1}{n^{\d/2}}\,. 
\label{e6gv}
\end{multline}


В случае же $iii$, т.\,е.\ для
$x^2\in[2b(\log|x|^{\dd}\hm-\log(a\beta_3/\sqrt{n})),\infty]$, в
соответствии с работами~\cite{Paditz81, Tysiak} справедлив следующий результат.
Обозначим
$$
P(a,b,K)=(2b)^{\dd}+a\exp\left\{\fr{(2b)^{\dd}}{a}-\fr{(b-1)K^2}{2b}\right\}\,.
$$

\smallskip

\noindent
\textbf{Лемма 2.1}. \textit{Предположим, что $x^2\ge c_n(x;a,b)\ge$\linebreak
$\ge K^2\ge (2\pi)^{-1}$. Тогда для любого $n\ge1$}
\begin{equation*}
|x|^{\dd}\left|F_n(x)-\Phi(x)\right|\le P(a,b,K)\Lo\,.
%\label{e7gv}
\end{equation*}

\smallskip

\noindent
Д\,о\,к\,а\,з\,а\,т\,е\,л\,ь\,с\,т\,в\,о\,\ см.\ в работах~\cite{Paditz81, Tysiak}.

\subsection{Случай \boldmath{$ii$}, то есть <<умеренные>> значения~\boldmath{$x$}}

Рассмотрение этого случая базируется на следующем фундаментальном
неравенстве (см.~\cite{Mich81} и~\cite{Tysiak}), в котором без
ограничения общности $x\ge0$:
\begin{multline}
\left|F_n(x)-\Phi(x)\right|\le n{\sf P}(|X_1|>y)
+
\left|
\vphantom{\fr{1}{2}}
f^n(h)-{}\right.\\
\left.{}-\exp\left\{\fr{1}{2}\,h^2n\right\}\right|
\exp\left\{-hx\sqrt{n}\right\}{\sf P}\left(S_n^*>x\sqrt{n}\right)+{}\\
{}+
2\exp\left\{\fr{1}{2}\,h^2n-hx\sqrt{n}\right\}\cdot\sup_{u\ge x}
\left|{\sf P}\left(S_n^*<u\sqrt{n}\right)-{}\right.\\
\left.{}-\Phi\left(u-h\sqrt{n}\right)\right|\,,
\label{e8gv}
\end{multline}
где $f(h)={\sf E}\exp\left\{h\overline{X}_1\right\}$,
$\overline{X}_1=X_1\I\{|X_1|<y\}$, $S_n^*=X_1^*+\ldots+X_n^*$~---
сумма независимых одинаково распределенных случайных величин с
общей функцией распределения
\begin{gather*}
{\sf P}(X_1^*<u)=\fr{1}{f(h)}\int\limits_{-\infty}^{u}e^{ht}\,d{\sf
P}\left(\overline{X}_1<t\right)\,,\\
y=\gamma x\sqrt{n}\,,\enskip h=\fr{(1-\gamma)x}{\sqrt{n}}\,,\enskip \gamma\in\left(0,\fr{1}{2}\right)\,.
\end{gather*}

Для начала сформулируем два утверждения, которые будут
неоднократно использоваться в дальнейшем. Во-первых, если $x^2\le
c_n(x;a,b)$, то (см.~$ii$)
\begin{equation}
\Lo\le\fr{|x|^{\dd}}{a}\,\exp\left\{-\fr{x^2}{2b}\right\}\,.\label{e9gv}
\end{equation}
Во-вторых, если $x^2\ge K^2$, то (см.~\cite{Paditz81})
\begin{equation}
x^r\exp\left\{-sx^2\right\}\le K^r\exp\left\{-sK^2\right\}\label{e10gv}
\end{equation}
при $x\ge\sqrt{r/(2s)}$ ($r>0$, $s>0$) или $r\le 0$.

Чтобы оценить выражение
$$
I_1=\left|f^n(h)-\exp\left\{\fr{1}{2}h^2n\right\}\right|
\exp\left\{-hx\sqrt{n}\right\}\,,
$$
воспользуемся результатом из~\cite{Tysiak}, согласно
которому
\begin{multline*}
I_1\le
\max\left\{
\vphantom{\left(1-\fr{\betr e^{hy}}{n^{\d/2}(\gamma
x)^{\dd}}\right)^{-1}}
\exp\left\{\fr{1}{2}\,h^2n-hx \sqrt{n}+
\fr{\betr e^{hy}}{n^{\d/2}(\gamma x)^{\dd}}\right\}\times\right.{}\\
\times \fr{\betr
e^{hy}}{n^{\d/2}(\gamma x)^{\dd}}\,, n\exp\left\{\fr{1}{2}h^2n-hx\sqrt{n}\right\}\times{}\\
\left.{}\times
\left(\fr{h^4}{4}+\fr{\betr
e^{hy}}{y^{\dd}}\right)\left(1-\fr{\betr e^{hy}}{n^{\d/2}(\gamma
x)^{\dd}}\right)^{-1}
\vphantom{}
\right\}.
\end{multline*}
Из~(\ref{e9gv}) вытекает, что
\begin{multline*}
1-\fr{\betr e^{hy}}{n^{\d/2}(\gamma x)^{\dd}}\ge{}\\
{}\ge 1-\fr{1}{a
\gamma^{\dd}}\exp\left\{\left(\gamma(1-\gamma)-\fr{1}{2b}\right)x^2\right\}\equiv
A_1(x)\,.
\end{multline*}
Здесь и далее символами $A(x)$, $A_1(x)$, $A_2(x)$,\ldots будут
обозначаться положительные функции аргумента $x$, а также
зависящие от параметров $a$, $b$, $\gamma$. Принимая во внимание
оценку
$$
n^{-1/2}\le\left(\Lo\right)^{1/3}
$$
и неравенство~(\ref{e9gv}), получим
\begin{multline*}
n\left(\fr{h^4}{4}+\fr{\betr e^{hy}}{y^{\dd}}\right)\le{}\\
\!{}\le
\fr{\betr}{n^{\d/2}|x|^{\dd}}\!\left[\fr{(1-\gamma)^4x^8}{4a^{1/3}}\,
\exp\!\left\{-\fr{x^2}{6b}\right\}+\fr{e^{\gamma(1-\gamma)x^2}}{\gamma^{\dd}}\right].
\end{multline*}
С учетом неравенства
$$
\exp\{1-A_1(x)\}\le\fr{1}{A_1(x)}\,,\enskip A_1(x)>0\,,
$$
также получаем
\begin{multline*}
\exp\left\{\fr{1}{2}\,h^2n-hx\sqrt{n}+ \fr{\betr
e^{hy}}{n^{1/2}(\gamma x)^{\dd}}\right\}\le{}\\
{}\le
\exp\left\{\fr{1}{2}\,h^2n-hx\sqrt{n}+1-A_1(x)\right\}\le{}\\
{}\le
\fr{1}{A_1(x)}\,\exp\left\{\fr{1}{2}h^2n-hx\sqrt{n}\right\}\,.
\end{multline*}
Наконец, если $\gamma(1-\gamma)-1/(2b)<0$, т.\,е.\
$b<$\linebreak $<[2\gamma(1-\gamma)]^{-1}$, то
\begin{equation}
I_1\le\fr{A_2(x)\betr}{n^{\d/2}A_1(K)|x|^{\dd}}\,,\label{e11gv}
\end{equation}
где $A_1(K)>0$ и
\begin{multline*}
A_2(x)=\fr{(1-\gamma)^4x^8}{4a^{1/3}}
\exp\left\{-\fr{x^2}{2}\left[\fr{1}{3b}+1-\gamma^2\right]\right\}+{}\\
{}+
\fr{1}{\gamma^{\dd}}\,\exp\left\{-\fr{x^2}{2}(1-\gamma)^2\right\}\,.
\end{multline*}
Для удобства дальнейших ссылок заметим, что с учетом~(\ref{e9gv})
$$
f(h)\ge 1-\fr{h\betr}{\gamma^{2}}\ge
1-\fr{x^2(1-\gamma)}{a\gamma^{2}}\,\exp\left\{-\fr{x^2}{2b}\right\}\equiv
A_3(x)
$$
(см.\ соотношение~(4.18) в \cite{Tysiak}).

Теперь оценим ${\sf E}\big(X_1^*\big)^2$. В~соответствии с~\cite{Tysiak} получаем, что если $K\ge\sqrt{2b}$ 
(см.~(\ref{e10gv}), то
\begin{equation}
{\sf E}\left(X_1^*\right)^2\le \fr{1}{A_3^{2}(K)}\left(h+\fr{\betr
e^{hy}}{y^{3}}\right)^2\!\le A_4(x)\Lo,\!\label{e12gv}
\end{equation}
где
\begin{multline*}
A_4(x)=\fr{1}{A_3^2(x)x^{\d}a^{1/3}}\,\exp\left\{-\fr{x^2}{6b}\right\}
\left[
\vphantom{\fr{1}{6}}
x^2(1-\gamma)+{}\right.\\
\left.{}+\fr{1}{\gamma^{2}a^{\d/3}}\exp\left\{\left(\gamma(1-\gamma)-
\fr{\d}{6b}\right)x^2\right\}
\right]^2\,.
\end{multline*}
Далее, согласно~\cite{Tysiak}

\noindent
\begin{multline*}
f^{-1}(h)\ge 2-f(h)\ge 1-\fr{h^2}{2}-\fr{\betr
e^{hy}}{y^{3}}\ge{}\\
{}\ge 1-A_5(x)\Lo\,,
\end{multline*}
где

\noindent
\begin{multline*}
A_5(x)=\fr{(1-\gamma)^2x^{3}}{2a^{1/3}}\,\exp\left\{-\fr{x^2}{6b}\right\}+{}\\
{}+
\fr{1}{\gamma^{3}a^{2/3}x}\,\exp\left\{\left(\gamma(1-\gamma)-
\fr{1}{3b}\right)x^2\right\}\,,
\end{multline*}
а также
$$
f(h){\sf E}\left(X_1^*\right)^2\ge 1-\betr\max\{\gamma^{-\d},\,h\}\,.
$$
Таким образом,
\begin{multline*}
B_n^2(h)\equiv {\sf D}S_n^*= n\left[{\sf
E}\left(X_1^*\right)^2-\left({\sf E}X_1^*\right)^2\right]\ge{}\\
{}\ge
n\left[f^{-1}(h)\left(1-\betr\max\{\gamma^{-\d},\,h\}\right)\right]-{}\\
{}- n\left({\sf
E}X_1^*\right)^2 \ge
n\left(1-\fr{A_5(x)\betr}{n^{\d/2}}\right)-{}\\
{}-\fr{n^{\d/2}\betr}{A_3(K)\gamma
x}\,\max\left\{1,\,\gamma(1-\gamma)x^2\right\}-{}\\
{}-A_4(x)n^{\d/2}\betr=n\left[1-A_6(x)\Lo\right]\,,
\end{multline*}
где
$$
A_6(x)=A_4(x)+A_5(x)+\fr{\max\{1,\,\gamma(1-\gamma)x^2\}}{A_3(K)\gamma
x}\,.
$$
С учетом~(\ref{e9gv}) имеем
\begin{multline}
B_n^2(h)\ge
n\left(1-\fr{x^{3}A_6(x)}{a}\exp\left\{-\fr{x^2}{2b}\right\}\right)\equiv{}\\
{}\equiv 
nA_7(x)\,;
\label{e13gv}
\end{multline}

\vspace*{-6pt}

\noindent
\begin{equation}
\fr{n-B_n^2(h)}{B_n^2(h)}\le\fr{A_6(x)}{A_7(x)}\Lo\,.\label{e14gv}
\end{equation}
Справедливы неравенства
\begin{multline*}
{\sf E}|X_1^*-{\sf E}X_1^*|^3\le{}\\
{}\le {\sf E}|X_1^*|^3+3{\sf
E}\left(X_1^*\right)^2|{\sf E}X_1^*|
+{\sf E}|X_1^*|\left({\sf E}X_1^*\right)^2\,;
\end{multline*}

\noindent
$$
{\sf
E}|X_1^*|^3\le\fr{e^{hy}\betr}{A_3(K)}\le\fr{1}{A_3(K)}\exp\{\gamma(1-\gamma)x^2\}\betr\,;
$$


\noindent
\begin{multline*}
3{\sf E}\left(X_1^*\right)^2\left|{\sf
E}X_1^*\right|\le{}\\
{}\le\fr{3\betr}{A_3^2(K)x}\left[\fr{(1-\gamma)x^3}{a^{1/3}}\,
\exp\left\{-\fr{x^2}{6b}\right\}+{}\right.\\
{}+
\fr{x+x^3\gamma(1-\gamma){\gamma^{2}a^{2/3}}}{\exp}\left\{\left(\gamma(1-\gamma)-
\fr{1}{3b}\right)x^2\right\}+{}\\
\left.{}+
\fr{x}{\gamma^{3}a}\exp\left\{\left(2\gamma(1-\gamma)-\fr{1}{2b}\right)x^2\right\}\right]
\equiv A_8(x)\betr\,;
\end{multline*}

\noindent
\begin{multline*}
{\sf E}|X_1^*|\left({\sf E}X_1^*\right)^2\le\sqrt{{\sf
E}\left(X_1^*\right)^2}\left({\sf E}X_1^*\right)^2\le{}\\
{}\le
\fr{\betr }{A_3^{5/2}(K)a^{2/3}}\,
\exp\left\{\left(\fr{5}{2}\gamma(1-\gamma)-\fr{3}{4b}\right)x^2\right\}\times{}\\
{}\times
\left[\fr{1}{\gamma
a^{1/3}}+\exp\left\{\left(\fr{1}{6b}-\gamma(1-\gamma)\right)x^2\right\}\right]^{1/2}\times{}\\
{}
\times\left[\fr{1}{\gamma^{2}a^{1/3}}+\right.\\
\left.{}+x^2(1-\gamma)\,\exp\left\{\left(\fr{1}{6b}-
\gamma(1-\gamma)\right)x^2\right\}\right]^2\equiv{}\\
{}\equiv A_9(x)\betr\,.
\end{multline*}
Таким образом,

\noindent
\begin{equation}
{\sf E}|X_1^*-{\sf E}X_1^*|^3\le A_{10}(x)\betr\,,\label{e15gv}
\end{equation}
где

\noindent
$$
A_{10}(x)=\fr{1}{A_3(K)}\,\exp\{\gamma(1-\gamma)x^2\}+A_8(x)+A_9(x)\,.
$$
Следовательно, в соответствии с неравенством~(\ref{e2gv}), учитывая~(\ref{e15gv}) и~(\ref{e13gv}), имеем

\noindent
\begin{multline}
\left|{\sf P}\left(\fr{S_n^*-{\sf E}S_n^*}{\sqrt{{\sf
D}S_n^*}}<u\right)-\Phi(u)\right|\le{}\\
{}\le 0{,}3041\fr{{\sf E}|X_1^*-{\sf
E}X_1^*|^3+1}{\sqrt{n}\big({\sf D}X_1^*\big)^{3/2}}\le{}\\
{}
\le 0{,}3041
\fr{A_{10}(x)\betr}{A_7^{3/2}(x)n^{\d/2}}+\fr{0{,}3041}{A_7^{3/2}(x)\sqrt{n}}\,.\label{e16gv}
\end{multline}
Далее в предположении, что

\noindent
\begin{multline*}
A(x)\equiv\fr{1}{2\gamma
a^{1/3}}\,\exp\left\{-\fr{x^2}{6b}\right\}+{}\\
{}+
\fr{1}{\gamma^4(1-\gamma)^2x^4a^{2/3}}\,\exp\left\{\!\left(\gamma(1-\gamma)-
\fr{1}{3b}\right)x^2\!\right\}\le\fr{1}{6}\hspace*{-2.1413pt}
\end{multline*}
(см.\ выражение для $g_{19}(x)$ в~\cite{Tysiak}), имеем
\begin{multline}
\left|{\sf
E}S_n^*-hB_n^*\right|\le\fr{\betr}{A_3(K)\gamma^{2}x^{2}}\left[
\vphantom{\fr{x^2}{3}}\exp\left\{\gamma(1-\gamma)x^2\right\}+{}\right.\\
\left.{}+
\fr{(1-\gamma)^2x^4}{a^{2/3}}\,\exp\left\{-\fr{x^2}{3b}\right\}\right]\equiv
A_{11}(x)\betr\,.\label{e17gv}
\end{multline}
Наряду с~(\ref{e14gv}) имеет место оценка
\begin{equation*}
B_n^2(h)-n\le
n\left[\fr{e^{hy}\betr}{f(h)y}+\fr{1}{f(h)}-1\right]\le{}\hspace*{8mm}
\end{equation*}
\pagebreak

\noindent
\begin{multline*}
{}\le 
\fr{n^{\d/2}\betr}{A_3(K)\gamma
x}\left[\exp\left\{\gamma(1-\gamma)x^2\right\}+{}\right.\\
\left.{}+\fr{(1-\gamma)x^2}{\gamma
a^{2/3}}\exp\left\{-\fr{x^2}{3b}\right\}\right]\equiv
A_{12}(x)n^{\d/2}\betr\,.
\end{multline*}
Таким образом,
\begin{multline}
\left|B_n^2(h)-n\right|\max\left\{\fr{1}{n},\,\fr{1}{B_n^2(h)}\right\}\le{}\\
{}\le \max\left\{A_{12}(x),\,\fr{A_6(x)}{A_7(x)}\right\}\Lo\,.\label{e18gv}
\end{multline}
Теперь можно приступить к оцениванию величин
(см.~(\ref{e8gv})):
\begin{align*}
I_2&\equiv {\sf P}\left(S_n^*>x\sqrt{n}\right) \  \mbox{и}\ \\
I_3&\equiv \sup_{u\ge x}\left|{\sf
P}\left(S_n^*<u\sqrt{n}\right)-\Phi\left(u-h\sqrt{n}\right)\right|
\end{align*}


Наряду с~(\ref{e13gv}) понадобится верхняя оценка для $B_n^2(h)$, которую
получим с учетом тождества $(1-z)^{-1}=z(1-z)^{-1}+1$:
\begin{multline*}
B_n^2(h)\le n{\sf E}\left(X_1^*\right)^2\le{}\\
{}\le
 n\left(1+\fr{e^{hy}\betr}{y}\right)\left(1-\fr{h\betr}{y^{2}}\right)^{-1}={}\\
{} =
n\left(1+\fr{e^{hy}\betr}{y}\right)\left[\fr{h\betr}{y^{2}}\left(1-\fr{h\betr}{y^{2}}\right)^{-1}+1\right]\le{}\\
{}\le
\fr{n}{A_3(K)}\left[\fr{h\betr}{y^{2}}+
\fr{he^{hy}\beta^2_{\dd}}{y^{3}}+A_3(K)\left(\!1+\fr{e^{hy}\betr}{y}\!\right)\right]
\equiv{}\hspace*{-0.7442pt}\\
{}\equiv n A_{13}(x)\,,
\end{multline*}
где
\begin{multline*}
A_{13}(x)=1+(\gamma x)^2\left(1-A_1(K)\right)+{}\\
{}
+\fr{\left(1-A_3(K)\right)x^2}{aA_3(K)}\left[a^{\d/3}\exp\left\{-\fr{x^2}{3b}\right\}+{}\right.\\
\left.{}+\fr{1}{\gamma}\,\exp\left\{\left(\gamma(1-\gamma)-\fr{1}{2b}\right)x^2\right\}\right]\,.
\end{multline*}
Найдем верхнюю оценку для ${\sf E}S_n^*-x\sqrt{n}$ (см.~(\ref{e17gv}):
\begin{multline*}
{\sf E}S_n^*-x\sqrt{n}={\sf
E}S_n^*-nh+nh-x\sqrt{n}\le{}\\
{}\le
\sqrt{n}\left(\fr{A_{11}(x)x^{\dd}}{a}\exp\left\{-\fr{x^2}{2b}\right\}-\gamma
x\right)\,,
\end{multline*}
т.\,е.\
\begin{multline}
\fr{x\sqrt{n}-{\sf E} S_n^*}{B_n(h)}\ge
\fr{1}{\sqrt{A_{13}(x)}}\left(\vphantom{\fr{x^3}{a}}\gamma x-{}\right.\\
\left.{}-\fr{x^{\dd}A_{11}(x)}{a}\exp\left\{-\fr{x^2}{2b}\right\}\right)\equiv
A_{14}(x)\,.\label{e19gv}
\end{multline}
Отсюда с учетом~(\ref{e1gv}), (\ref{e2gv}), (\ref{e9gv}) и (\ref{e16gv}) при
$A_{14}(x)\ge1$ получаем
\begin{multline}
I_2\le\left|{\sf
P}\left(S_n^*<x\sqrt{n}\right)-\Phi\left(\fr{x\sqrt{n}-{\sf
E}S_n^*}{\sqrt{{\sf
D}S_n^*}}\right)\right|+{}\\
{}+\Phi\left(-\fr{x\sqrt{n}-{\sf
E}S_n^*}{\sqrt{{\sf D}S_n^*}}\right)\le{}\\
{}\le 
0{,}3041\fr{A_{10}(x)x^{\dd}}{aA_7^{3/2}(x)}\exp\left\{-\fr{x^2}{2b}\right\}+{}\\
{}+
\fr{0{,}3041}{A_7^{3/2}(x)\sqrt{n}}+
\fr{1}{\sqrt{2\pi}A_{14}(x)}\exp\left\{-\fr{A_{14}^2(x)}{2}\right\}\equiv{}\\
{}
\equiv A_{15}(x)+\fr{0{,}3041}{A_7^{3/2}(x)\sqrt{n}}\,.\label{e20gv}
\end{multline}

Чтобы оценить $I_3$, воспользуемся соотношениями~(\ref{e13gv}),
(\ref{e16gv})--(\ref{e19gv}) и~(\ref{e2gv}) и получим
\begin{multline}
I_3=\sup_{u\ge x}\bigg|{\sf P}\left(\fr{S_n^*-{\sf
E}S_n^*}{B_n(h)}<\fr{u\sqrt{n}-{\sf
E}S_n^*}{B_n(h)}\right)-{}\\
{}-\Phi\left(\fr{u\sqrt{n}-{\sf
E}S_n^*}{B_n(h)}\right)+\Phi\left(\fr{u\sqrt{n}-{\sf
E}S_n^*}{B_n(h)}\right)-{}\\
{}-
\Phi\left(\fr{u\sqrt{n}-{\sf
E}S_n^*}{\sqrt{n}}\right)+\Phi\left(\fr{u\sqrt{n}-{\sf
E}S_n^*}{\sqrt{n}}\right)-{}\\
{}-\Phi\left(u-h\sqrt{n}\right)\bigg|\le{}\\
{}\le
\sup_{v\ge(x\sqrt{n}-{\sf E}S_n^*)/B_n(h)}\bigg[\bigg|{\sf
P}\bigg(\fr{S_n^*-{\sf
E}S_n^*}{B_n(h)}<v\bigg)-{}\\
{}-
\Phi(v)\bigg|+\bigg|\Phi(v)-\Phi\bigg(v\fr{B_n(h)}{\sqrt{n}}\bigg)
\bigg|\bigg]+{}\\
{}
+\sup_{u\ge x}\left|\Phi\left(u-{\sf
E}S_n^*/\sqrt{n}\right)-\Phi\left(u-h\sqrt{n}\right)\right|\le{}\\
{}\le
0{,}3041\fr{A_{10}(x)\betr}{A_7^{3/2}(x)n^{\d/2}}+\fr{0{,}3041}{A_7^{3/2}(x)\sqrt{n}}+{}\\
{}
+\fr{1}{\sqrt{8\pi}}\left|\fr{B_n^2(h)}{n}-1\right|\max\left\{1,\,\fr{n}{B_n^2(h)}\right\}\times{}\\
{}\times
\sup\left\{\vphantom{\fr{B_h(h)}{B_h(h)}}
|s|e^{-s^2/2}:\right.\\
\left.s\ge\fr{x\sqrt{n}-{\sf
E}S_n^*}{B_n(h)}\min\left\{1,\,\fr{B_n(h)}{\sqrt{n}}\right\}\right\}+{}\\
{}+
\fr{1}{\sqrt{2\pi}}\bigg|\fr{{\sf
E}S_n^*-nh}{\sqrt{n}}\bigg|
\sup\bigg\{e^{-s^2/2}:\\
s\ge\min\Big\{x-\fr{{\sf
E}S_n^*}{\sqrt{n}},\,x-h\sqrt{n}\Big\}\bigg\}\le{}\\
{}\le
A_{16}(x)\Lo+\fr{0{,}3041}{A_7^{3/2}(x)\sqrt{n}}\,,\label{e21gv}
\end{multline}
где
\begin{multline*}
A_{16}(x)=0{,}3041\fr{A_{10}(x)}{A_7^{3/2}(x)}+{}\\
{}+\fr{A_{14}(x)}{\sqrt{8\pi}}\max\left\{A_{12}(x),\,\fr{A_6(x)}{A_7(x)}\right\}
\exp\left\{-\fr{A_{14}^2(x)}{2}\right\}+{}\\
{}+
\fr{A_{11}(x)}{\sqrt{2\pi}}\exp\!\bigg\{\!-\fr{1}{2}\Big[\gamma
x-\fr{A_{11}(x)|x|^{\dd}}{a}\exp\left\{-\fr{x^2}{2b}\right\}\Big]^2\!\bigg\}.\hspace*{-1.467pt}
\end{multline*}

Соотношение~(\ref{e21gv}) справедливо в предположении
$A_{14}(x)\ge 1$, так что
$$
\gamma
x-\fr{A_{11}(x)|x|^{\dd}}{a}\,\exp\left\{-\fr{x^2}{2b}\right\}\ge 0\,.
$$
Первое слагаемое в правой части~(\ref{e8gv}) оценим с помощью неравенства
Маркова:
\begin{equation}
n{\sf P}(|X_1|>y)\le\fr{\betr}{\gamma^{\dd}
|x|^{\dd}n^{\d/2}}\,.\label{e22gv}
\end{equation}
В итоге из~(\ref{e8gv}) с учетом~(\ref{e11gv}), (\ref{e20gv})--(\ref{e22gv}) мы
получаем: для каждого~$x$ из рассматриваемого диапазона~ii
справедливо неравенство
\begin{multline}
\left|F_n(x)-\Phi(x)\right|\le\fr{\betr}{\gamma^{\dd}|x|^{\dd}n^{\d/2}}+I_1
I_2+{}\\
{}+2I_3\exp\left\{-(1-\gamma^2)\fr{x^2}{2}\right\}\le{}\\
{}\le 
Q_n(x;\,a,b,\gamma,K)\fr{\betr}{|x|^{\dd}n^{\d/2}}+{}\\
{}+
0{,}6082\fr{\exp\left\{-(1-\gamma^2)x^2/2\right\}}{A_7^{3/2}(x)\sqrt{n}}\,,\label{e23gv}
\end{multline}
где
\begin{multline*}
Q_n(x;\,a,b,\gamma,K)=\fr{1}{\gamma^{\dd}}+{}\\
{}+
\fr{A_2(x)}{A_1(K)}\left[1+A_{15}(x)+
\fr{0{,}3041}{A_7^{3/2}(x)\sqrt{n}}\right]+{}\\
{}+2A_{16}(x)|x|^{\dd}\exp\left\{-(1-\gamma^2)\fr{x^2}{2}\right\}
\end{multline*}
%$\big 
(отметим, что
$ %\begin{multline*}
\lim_{n\to\infty}Q_n(x;\,a,b,\gamma,K)={1}/{\gamma^{\dd}}+$\linebreak $+
{A_2(x)\left(1+A_{15}(x)\right)}/{A_1(K)}+2A_{16}(x)|x|^{\dd}\exp\{-(1-$\linebreak 
$-\gamma^2){x^2}/{2}\}$).
%\end{multline*}

Обозначим
$$
R(x;\,a,b,\gamma,K)=0{,}6082\fr{|x|^{\dd}\exp\left\{-(1-\gamma^2){x^2}/{2}\right\}}{A_7^{3/2}(x)}\,.
$$
Тогда справедливо следующее утверждение.

\columnbreak

\noindent
\textbf{Лемма 2.2} \textit{Предположим, что $K^2\le x^2\le
c_n(x;\,a,b)$,}

\noindent
\begin{multline*}
K^2\ge\fr{1}{2\pi}\,,\enskip  0<\gamma<\fr{1}{2}\,,\enskip  a>0\,,\\
1<b<\min\left\{\fr{1}{2\gamma(1-\gamma)},\,\fr{K^2}{2}\right\}
\end{multline*}
\textit{и}

\noindent
\begin{multline*}
A(x)\le\fr{1}{6}\,, \enskip A_1(K)>0\,,\enskip  A_3(K)>0\,,\\
 A_7(x)>0\,,\enskip 
A_{14}(x)\ge1\,.
\end{multline*}
\textit{Тогда для всех $n\ge1$}

\noindent
\begin{multline*}
|x|^{\dd}\left|F_n(x)-\Phi(x)\right|
\le
Q_n(x;\,a,b,\gamma,K)\Lo+{}\\
{}+R(x;\,a,b,\gamma,K)\frac{1}{\sqrt{n}}\,.
\end{multline*}

\smallskip

Приведем две мажоранты функции $Q_n(x;\,a,b,\gamma,K)$,
не зависящие от~$n$.

Во-первых, очевидно, что
\begin{multline}
Q_n(x;\,a,b,\gamma,K)\le Q_1(x;\,a,b,\gamma,K)={}\\
{}
=\fr{1}{\gamma^{\dd}}+\fr{A_2(x)}{A_1(K)}\left[1+A_{15}(x)+
\fr{0{,}3041}{A_7^{3/2}(x)}\right]+{}\\
{}+2A_{16}(x)|x|^{\dd}\exp\left\{-(1-\gamma^2)\fr{x^2}{2}\right\}\,.\label{e24gv}
\end{multline}


Во-вторых, с учетом того, что в сделанных предположениях о
моментах случайной величины~$X_1$ всегда $\betr\ge1$, из
соотношения~(\ref{e9gv}) вытекает неравенство
$$
\fr{1}{\sqrt{n}}\le\fr{|x|^{\dd}}{a}\exp\left\{-\fr{x^2}{2b}\right\}\,.
$$
Поэтому
\begin{multline}
Q_n(x;\,a,b,\gamma,K)\le Q'(x;\,a,b,\gamma,K)\equiv{}\\
{}\equiv
\fr{1}{\gamma^{\dd}}+\fr{A_2(x)}{A_1(K)}\bigg[1+{}\\
{}+A_{15}(x)+
\fr{0{,}3041}{A_7^{3/2}(x)}\bigg(\fr{|x|^{\dd}}{a}\exp\left\{-\fr{x^2}{2b}\right\}\bigg)\bigg]
+{}\\
{}+
2A_{16}(x)|x|^{\dd}\exp\left\{-(1-\gamma^2)\fr{x^2}{2}\right\}\,.\label{e25gv}
\end{multline}

Наконец, пытаясь ограничить $Q_n(x;\,a,b,\gamma,K)$, вместо
неравенства~(\ref{e2gv}) при оценивании величины~$I_2$ можно
воспользоваться неравенством~(\ref{e1gv}) с наилучшей известной на
сегодняшний день оценкой константы $C_0$: $C_0 \le 0{,}4784$~\cite{KorolevBEs}. 
Тогда из~(\ref{e1gv}), (\ref{e9gv}) и~(\ref{e16gv}) при $A_{14}(x)\ge1$
будет следовать оценка

\noindent
\begin{multline*}
I_2\le\bigg|{\sf
P}\big(S_n^*<x\sqrt{n}\big)-\Phi\bigg(\fr{x\sqrt{n}-{\sf
E}S_n^*}{\sqrt{{\sf
D}S_n^*}}\bigg)\bigg|+{}\\
{}+\Phi\bigg(\!-\fr{x\sqrt{n}-{\sf
E}S_n^*}{\sqrt{{\sf D}S_n^*}}\!\bigg)\le
0{,}4784\fr{A_{10}(x)x^{\dd}}{aA_7^{3/2}(x)}\exp\Big\{-\frac{x^2}{2b}\Big\}+{}\\
{}+
\fr{1}{\sqrt{2\pi}A_{14}(x)}\exp\left\{-\fr{A_{14}^2(x)}{2}\right\}\equiv
A'_{15}(x)\,.
\end{multline*}
При этом в~(\ref{e23gv}) и последующих выкладках и вычислениях вместо
$Q_n(x;\,a,b,\gamma,K)$ cоответственно следует использовать
величину

\noindent
\begin{multline}
Q''(x;\,a,b,\gamma,K)=\fr{1}{\gamma^{\dd}}+\fr{A_2(x)}{A_1(K)}\left[1+A'_{15}(x)\right]
+{}\\
{}+2A_{16}(x)|x|^{\dd}\exp\left\{-(1-\gamma^2)\fr{x^2}{2}\right\}\,.\label{e26gv}
\end{multline}

Положим

\noindent
\begin{multline*}
Q(x;\,a,b,\gamma,K)=\min\left\{Q_1(x;\,a,b,\gamma,K),\right.\\
\left.Q'(x;\,a,b,\gamma,K),\,Q''(x;\,a,b,\gamma,K)\right\}\,.
\end{multline*}

%\medskip

\noindent
\textbf{Следствие 2.1}. \textit{В условиях леммы~$2.2$ для всех $n\ge1$
справедливо неравенство}

\noindent
\begin{multline*}
|x|^{\dd}\left|F_n(x)-\Phi(x)\right|\le
Q(x;\,a,b,\gamma,K)\Lo+{}\\
{}+R(x;\,\d,a,b,\gamma,K)\fr{1}{\sqrt{n}}\,.
\end{multline*}

\subsection{Неравномерные оценки скорости сходимости в центральной предельной теореме}

Положим

\noindent
\begin{align*}
U_n&=\min_{a,b,\gamma,K}\max\Big\{C_1
K^{\dd}\,,\\
&\hspace*{-8mm}\max_{K\le|x|\le\sqrt{c_n(x;\,a,b)}}Q_n(x;\,a,b,\gamma,K),\,
P(a,b,K)\Big\}\,;\\
U&=\min_{a,b,\gamma,K}\max\Big\{C_1
K^{\dd}\,,\\
&\hspace*{-8mm}\max_{K\le|x|\le\sqrt{c_n(x;\,a,b)}}Q(x;\,a,b,\gamma,K),\,
P(a,b,K)\Big\}\,,
\end{align*}
где минимум берется по множеству значений вспомогательных
параметров, описанному в формулировке леммы~2. Значения параметров
$a$, $b$, $\gamma$, $K$, доставляющие вышеуказанные минимумы,
обозначим соответственно $a^{(n)}_0$, $b^{(n)}_0$,
$\gamma^{(n)}_0$, $K^{(n)}_0$ и $a_0$, $b_0$, $\gamma_0$, $K_0$.
Положим

\noindent
\begin{multline*}
R^{(n)}_0={}\hspace*{60mm}\\
=\!\!\!\max_{K^{(n)}_0\le|x|\le\sqrt{c_n(x;\,a^{(n)}_0,b^{(n)}_0)}}R(x;a^{(n)}_0,b^{(n)}_0,\gamma^{(n)}_0,K^{(n)}_0);\hspace*{-2.76pt}
\end{multline*}
\begin{gather*}
R_0=\max_{K_0\le|x|\le\sqrt{c_n(x;\,a_0,b_0)}}R(x;a_0,b_0,\gamma_0,K_0)\,;
\\
D_n=U_n+C_1,\enskip D=U+C_1\,;
\\
V_n=\max\left\{R^{(n)}_0,\,C_1(K^{(n)}_0)^{\dd}\right\}+C_1\,;
\\
V(n)=\max\left\{R_0,\,C_1 K_0^{\dd}\right\}+C_1.
\end{gather*}
Очевидно, что $V_n\le V_1$ и $V(n)\le V(1)\equiv V$.

Из соотношения~(\ref{e6gv}), лемм~2.1 и~2.2 вытекает следующее утверждение.

\medskip

\noindent
\textbf{Теорема 2.1.} \textit{Для всех $x\in\R$ и всех $n\ge1$
справедливы неравенства:}
\begin{align*}
\left(1+|x|^{\dd}\right)\left|F_n(x)-\Phi(x)\right|&\le
D_n\Lo+\fr{V_n}{n^{\d/2}}\,;
\\
\left(1+|x|^{\dd}\right)\left|F_n(x)-\Phi(x)\right|&\le
D\Lo+\fr{V(n)}{n^{\d/2}}\,.
\end{align*}

\medskip

\noindent
\textbf{Следствие 2.2} \textit{Для всех $x\in\R$ и всех $n\ge1$
справедливо неравенство}
$$
\left(1+|x|^{\dd}\right)\left|F_n(x)-\Phi(x)\right|\le
D\Lo+\fr{V}{n^{\d/2}}\,.
$$

\medskip

\noindent
\textbf{Замечание 2.1} На практике процедуру поиска оптимальных
значений $a$, $b$, $\gamma$, $K$ можно организовать следующим
образом. Поскольку выражение $C_1(1)K^{\dd}$ не зависит от $a$,
$b$, $\gamma$, сначала можно найти значение $K^*=K^*(a,b)$ из
условия
$$
C_1(K^*)^{\dd}=P(a,b,K^*)\,,
$$
затем найти пару $(a^*,b^*)$ из условия
$$
(a^*,b^*)=\mathrm{arg}\,\min_{a,b}P\left(a,b,K^*(a,b)\right)\,,
$$
а затем провести оптимизацию
\begin{align*}
&\hspace*{-5mm}\max_{
K^*(a^*,b^*)\le|x|\le\sqrt{c_n(x;\,a^*,b^*)}}Q_n(x;\,a^*,b^*,\gamma,\\
&\hspace*{35mm}K^*(a^*,b^*)) \longrightarrow\min_{\gamma}\,,
\\
&\hspace*{-5mm}\max_{
K^*(a^*,b^*)\le|x|\le\sqrt{c_n(x;\,a^*,b^*)}}Q(x;\,a^*,b^*,\gamma, \\
&\hspace*{35mm}K^*(a^*,b^*)) \longrightarrow\min_{\gamma}\,,
\\
&\hspace*{-5mm}\max_{
K^*(a^*,b^*)\le|x|\le\sqrt{c_n(x;\,a^*,b^*)}}\bar{Q}\left(x;\,a^*,b^*,\gamma,\right.\\
&\hspace*{35mm}\left.K^*(a^*,b^*)\right) \longrightarrow\min_{\gamma}\,.
\end{align*}
При этом вычисления показывают, что все три вышеперечисленных
максимума достигаются в точке $x = K^*(a^*,b^*).$

\smallskip

\noindent
\textbf{Замечание 2.2} Конкретные вычисления показали, что
значения функций $Q(x,a,b,\gamma,K)$ и $R(x,a,b,\gamma,K)$
оказываются строго меньшими, чем значения
$P(a,b,\gamma,K)$ и $C_1K^3,$ поэтому максимум в выражениях для
$U$ и $V(n)$ определяется лишь значениями функций
$P(a,b,\gamma,K)$ и $C_1K^3,$ а следовательно, $D$ и $V$
совпадают, причем оптимальное значение $D$ не превосходит~22,7707.
Таким образом, приведенная в следствии~2.2 оценка имеет структуру,
аналогичную оценке~(\ref{e2gv}), и справедливо следующее
утверждение.

\smallskip

\noindent
\textbf{Следствие 2.2\boldmath{$'$}} \textit{Для всех $x\in\R$ и всех $n\ge1$
справедливо неравенство}

\noindent
$$
(1+|x|^3)\left|F_n(x)-\Phi(x)\right|\leqslant
22{,}7707\fr{\beta_3+1}{n^{1/2}}\,.
$$

\smallskip

Следствие 2.2$'$ позволяет уточнить константу в неравномерной
оценке скорости сходимости в ЦПТ для пуассоновских случайных сумм,
чему посвящен следующий раздел.

\vspace*{-3pt}

\section{Неравномерная оценка скорости сходимости в~центральной предельной
теореме для~пуассоновских случайных сумм}

Пусть теперь $X_1, X_2, \ldots$~--- последовательность независимых
одинаково распределенных случайных величин таких, что
\begin{equation}
{\sf E}X_1\equiv\mu\,, \ \ {\sf D}X_1 \equiv \sigma^2>0\,,\ \ {\sf
E}|X_1|^{\dd} \equiv \betr<\infty\,. \label{e27gv}
\end{equation}
Пусть $N_\lambda$~--- случайная величина, имеющая распределение
Пуассона с параметром $\lambda>0.$ Предположим, что при каждом~$\lambda>0$ 
случайные величины $N_\lambda, X_1, X_2,\ldots$
независимы.

Рассмотрим пуассоновскую случайную сумму
$$
S_{\lambda} = X_1+\cdots+ X_{N_\lambda}\,.
$$
Для определенности полагаем, что  $S_\lambda = 0$ при $N_\lambda =
0.$ Несложно видеть, что в рас\-смат\-ри\-ва\-емых условиях на моменты
случайной величины~$X_1$ справедливы соотношения
$$
{\sf E}S_\lambda=\lambda\mu\,, \enskip {\sf
D}S_\lambda=\lambda(\mu^2+\sigma^2)\,.
$$
Функцию распределения стандартизованной пуассоновской случайной
суммы
$$
\widetilde{S}_\lambda
\equiv\fr{S_\lambda-\lambda\mu}{\sqrt{\lambda(\mu^2+\sigma^2)}}
$$
обозначим~$F_\lambda(x)$.

\columnbreak

Пуассоновские случайные суммы~$S_{\lambda}$ являются очень
популярными математическими моделями многих объектов и процессов в
самых разных областях, в том числе в страховании, где они
используются для описания суммы страховых требований, поступивших
в течение определенного периода времени, в теории управления
запасами, где они описывают суммарные заявки на продукт. При
анализе информационных систем также традиционным предположением
является пуассоновский характер потока заявок (клиентов,
требований, задач, сообщений), так что суммарные характеристики
заявок в информационных системах имеют вид пуассоновских случайных
сумм. Задаче изуче\-ния точ\-ности нормальной аппроксимации для
распределений пуассоновских случайных сумм~--- так называемых
обобщенных пуассоновских распределений~--- посвящена обширная
литература (см., например, библиографию в книгах~\cite{KorBenShorg} и~\cite{BenKor2002}).

В работе~\cite{Mich93} показано, что для любых $x\in\R$
и любых $n\ge1$ справедливо неравенство
\begin{multline}
\left(1+|x|^{3}\right)\left|F_{\lambda}(x)-\Phi(x)\right|\le{}\\
{}\le
C\cdot\fr{\betr}{\lambda^{\delta/2}(\mu^2+\sigma^2)^{1+\delta/2}}\,,\label{e28gv}
\end{multline}
где $C$~--- та же константа, что и в <<классической>> оценке~(\ref{e5gv}). 
В~данном разделе будет показано, что на самом деле неравенство~(\ref{e28gv})
справедливо с заменой~$C$ на~$D$ (см.\ следствие~2.2$'$). Для этого
понадобятся некоторые вспомогательные утверждения.

Обозначим
$$
\nu = \fr{\lambda}{n}\,.
$$

\medskip

\noindent
\textbf{Лемма 3.1.} \textit{Распределение пуассоновской случайной суммы
$S_\lambda$ совпадает с распределением суммы $n$ независимых
одинаково распределенных случайных величин, каким бы ни было
натуральное число $n\geq1:$}
$$
X_1+\cdots+X_{N_\lambda} \stackrel{d}{=} Y_{\nu,1}+\cdots+Y_{\nu,n}\,,
$$
\textit{где при каждом $n$ случайные величины $Y_{\nu,1},\ldots,
Y_{\nu,n}$ независимы и одинаково распределены. При этом если
случайная величина $X_1$ удовлетворяет условиям}~(\ref{e27gv}), \textit{то для
моментов случайной величины $Y_{\nu,1}$ имеют мес\-то соотношения:}
\begin{gather*}
{\sf E}Y_{\nu,1} = \mu\nu\,; \quad {\sf D}Y_{\nu,1} =
(\mu^2+\sigma^2)\nu\,;\\
{\sf E}|Y_{\nu,1} - \mu\nu|^{3}\leq \nu\beta_{3}(1+40\nu) \,, \quad
n\geq\lambda\,.
\end{gather*}

\medskip

Д\,о\,к\,а\,з\,а\,т\,е\,л\,ь\,с\,т\,в\,о\,\ см.\ в работе~\cite{Shev2007}.

\medskip

\noindent
\textbf{Следствие 3.1} \textit{Если выполнены условия $(27),$ то для
любого $n=1,2, \ldots$}
$$
\widetilde{S}_\lambda \stackrel{d}{=}
\fr{1}{\sqrt{n}}\sum_{k=1}^n Z_{\nu,k}\,,
$$
\textit{где при каждом $n$ случайные величины $Z_{\nu,1},\ldots,
Z_{\nu,n}$ независимы и одинаково распределены. Более того, эти
случайные величины имеют нулевое среднее и единичную дисперсию и
при всех $n\geq\lambda$}
\begin{equation}
{\sf E}|Z_{\nu,1}|^{3}\le
\fr{\beta_{3}(1+40\nu)}{(\mu^2+\sigma^2)^{3/2}}\left(\fr{n}{\lambda}\right)^{1/2}\,.
\label{e29gv}
\end{equation}

\medskip

Следующее утверждение представляет собой основной результат данной
статьи.

\medskip

\noindent
\textbf{Теорема 3.1} \textit{При условиях}~(\ref{e27gv}) \textit{для любого $\lambda>0$
справедливо неравенство}
$$
\sup_{x\in\R}(1+|x|^{3})\left|F_\lambda(x) - \Phi(x)\right|\leq
\fr{D\betr}{\lambda^{1/2}(\mu^2+\sigma^2)^{3/2}}\,,
$$
\textit{где константа~$D$ та же, что и в следствии~$2.2'$,
т.\,е.\ $D\le22.7707$.}

\medskip

\noindent
Д\,о\,к\,а\,з\,а\,т\,е\,л\,ь\,с\,т\,в\,о\,.\ Из леммы~3.1 и следствия~3.1
вытекает, что для любого целого $n\geq1$
$$
|F_\lambda(x) - \Phi(x)| = \left|{\sf
P}\left(\fr{1}{\sqrt{n}}\sum_{k=1}^n
Z_{\nu,k}<x\right)-\Phi(x)\right|\,.
$$
Следовательно, в силу следствия~2.2 для произвольного целого
$n\geq1$ при каждом фиксированном $x\in\R$ имеем
\begin{equation}
\left(1+|x|^{3}\right)\left|F_\lambda(x) - \Phi(x)\right| \leq D\fr{{\sf
E}|Z_{\nu,1}|^{3}}{n^{1/2}}+\fr{V}{n^{\d/2}}\,. \label{e30gv}
\end{equation}
Поскольку в~(\ref{e30gv}) $n$ произвольно, можно считать, что
$n\geq\lambda.$ Тогда, используя оценку~(\ref{e29gv}), в продолжение~(\ref{e30gv})
получаем неравенство
\begin{multline*}
\left(1+|x|^{3}\right)\left|F_\lambda(x) - \Phi(x)\right|\leq{}\\
{}\leq
\fr{D\betr}{\lambda^{1/2}(\mu^2+\sigma^2)^{3/2}}
\left(1+40\fr{\lambda}{n}\right)+\fr{V}{n^{\d/2}}\,.
\end{multline*}
Так как здесь $n\geq\lambda$ произвольно, устремляя
$n\rightarrow\infty,$ окончательно получаем
\begin{multline*}
\left(1+|x|^{3}\right)\left|F_\lambda(x) - \Phi(x)\right|\leq{}\\
{}\leq
\lim_{n\rightarrow\infty}\left[\fr{D\betr}{\lambda^{1/2}(\mu^2+\sigma^2)^{3/2}}
\left(1+40\fr{\lambda}{n}\right)+\fr{V}{n^{\d/2}}\right]={}\\
{}=
\fr{D\betr}{\lambda^{1/2}(\mu^2+\sigma^2)^{3/2}}\,,
\end{multline*}
что и требовалось доказать.

\section{Неравномерные оценки скорости сходимости в~предельных
теоремах для~смешанных пуассоновских случайных сумм}

Пусть $\Lambda_t$~--- положительная случайная величина, функция
распределения $G_t(x)={\sf P}(\Lambda_t<x)$ которой зависит от
некоторого параметра $t>0$. Под смешанным пуассоновским
распределением со структурным распределением~$G_t$ будем
подразумевать распределение случайной величины~$N(t)$, принимающей
целые неотрицательные значения с вероятностями

\noindent
$$
{\sf P}\left(N(t)=k\right)=\fr{1}{k!}\int\limits_{0}^{\infty}
e^{-\lambda}\lambda^kdG_t(\lambda),\enskip k=0,1,2,\ldots
$$
Известно несколько конкретных примеров смешанных пуассоновских
распределений, наиболее широко используемым среди которых, пожалуй,
является отрицательное биномиальное распределение (это распределение
было использовано в виде смешанного пуассоновского еще в работе~\cite{Greenwood1920} 
для\linebreak моделирования частоты несчастных
случаев на\linebreak производстве). Отрицательное биномиальное распределение
порождается структурным гам\-ма-рас\-пре\-де\-ле\-ни\-ем. Другими примерами
смешанных пуассоновских распределений являются распределение
Делапорте, порождаемое сдвинутым гам\-ма-струк\-тур\-ным распределением~\cite{Delaporte}, 
распределение Зихеля, порожденное
обратным нормальным структурным распределением~\cite{Holla}--\cite{Willmot}, обобщенное
распределение Варинга~\cite{Irwin, Seal}. Свойства смешанных пуассоновских
распределений подробно описаны в книгах~\cite{Grandell} и~\cite{BenKor2002}.

Пусть, как и ранее, $X_1,X_2,\ldots$~--- независимые одинаково
распределенные случайные величины. Предположим, что случайные
величины $N(t),X_1,X_2,\ldots$ независимы при каждом $t>0$. Положим

\noindent
$$
S(t)=X_1+ \ldots +X_{N(t)}
$$
(как и ранее, для определенности будем считать, что $S(t)=0$, если
$N(t)=0$). Случайную величину~$S(t)$ будем называть смешанной
пуассоновской случайной суммой, а ее распределение~--- обобщенным
смешанным пуассоновским.

Асимптотическое поведение смешанных пуассоновских случайных сумм~$S(t)$, когда $N(t)$ 
в определенном смысле неограниченно
возрастает, принципиально различно в зависимости от того, равно
нулю математическое ожидание~$\mu$ слагаемых или нет.

Сходимость по распределению и по вероятности будет обозначаться
символами $\Longrightarrow$ и~$\pto$ соответственно.

В этом разделе сосредоточимся на ситуации, когда ${\sf E}X_1=0$. 
В~таком случае предельными распределениями для стандартизованных
смешанных пуассоновских случайных сумм являются масштабные смеси
нормальных законов. Не ограничивая общности, будем считать, что
${\sf D}X_1=1$.

\smallskip

\noindent
\textbf{Теорема 4.1} \cite{Korolev96, BenKor2002}. \textit{Предположим, что
$\Lambda_t\pto\infty$ при $t\to\infty$. Тогда для положительной
неограниченно возрастающей функции~$d(t)$ существует функция
распределения $H(x)$ такая, что}
$$
{\sf P}\left(\fr{S(t)}{\sqrt{d(t)}}<x\right)\Longrightarrow H(x)\
\enskip (t\to\infty)\,,
$$
\textit{в том и только в том случае, когда существует функция
распределения~$G(x)$ такая, что при той же функции~$d(t)$}
\begin{equation}
G_t\left(xd(t)\right)\Longrightarrow G(x) \enskip (t\to\infty)\label{e31gv}
\end{equation}
\textit{и}
$$
H(x)=\int\limits_{0}^{\infty}\Phi\left(\frac{x}{\sqrt{y}}\right)dG(y)\,,\enskip
x\in\R\,.
$$

\smallskip

Легко видеть, что распределение смешанной пуассоновской случайной
суммы~$S(t)$ можно записать в виде
\begin{multline}
{\sf P}(S(t)<x)=\int_{0}^{\infty}{\sf
P}\left(\sum_{j=1}^{N_{\lambda}}X_j<x\right)dG_t(\lambda)\,,\\
x\in\R\,,
\label{e32gv}
\end{multline}
где $N_\lambda$~--- пуассоновская случайная величина с параметром
$\lambda>0$, такая что при каждом $\lambda>0$ случайные величины
$N_\lambda,X_1,X_2,\ldots$ независимы. (Следует заметить, что запись
приводимых соотношений в терминах случайных величин использована
лишь для удобства и наглядности. На самом деле речь идет о
соответствующих соотношениях для распределений, но такая форма
записи оказывается более громоздкой. Поэтому предположение о
существовании вероятностного пространства, на котором определены
упомянутые выше случайные величины с указанными свойствами, ни в
коей мере не ограничивает общности.) Пусть
\begin{equation}
{\sf E}X_1=0\,,\ \ {\sf E}X_1^2=1\,,\ \ \beta^3={\sf
E}|X_1|^3<\infty\label{e33gv}
\end{equation}
и $d(t)$, $t>0$,~--- некоторая положительная неограниченно
возрастающая функция. Равномерные оценки скорости сходимости в
теореме~4.1 приведены в работах~\cite{KorolevBEs, KorSchev, Gavrilenko} (см.\ также~\cite{KorBenShorg}). 
В~данном разделе будут приведены неравномерные
оценки скорости сходимости в теореме~4.1 и ее частных случаях.

Пусть $G(x)$~--- функция распределения такая, что $G(0)=0$. Если
выполнено условие~(\ref{e31gv}), то в соответствии с теоремой~4.1
обобщенные смешанные пуассоновские распределения случайной
величины~$S(t)$, нормированной квадратным корнем из функции~$d(t)$, 
сходятся к масштабным смесям нормальных законов, в которых
смешивающим распределением является~$G(x)$. Обозначим
\begin{gather*}
\Delta_t(x)=\left|{\sf P}\left(\fr{S(t)}{\sqrt{d(t)}}<x\right)-
\int\limits_{0}^{\infty}
\Phi\left(\fr{x}{\sqrt{\lambda}}\right)dG(\lambda)\right|\,,\\
\delta_t(x)=G_t\left(d(t)x\right)-G(x)\,.
\end{gather*}

\smallskip

\noindent
\textbf{Теорема 4.2.} \textit{Предположим, что выполнены условия}~(\ref{e28gv}).
\textit{Тогда при каждом $t>0$ при любом $x\in\R$ имеет место оценка}
\begin{multline}
\Delta_t(x)\le 22{,}7707\fr{\beta_3}{\sqrt{d(t)}}\,{\sf
E}\left\{\fr{\Lambda_t}{d(t)}\left[\left(\fr{\Lambda_t}{d(t)}\right)^{3/2}+{}\right.\right.\\
\left.\left.{}+|x|^3
\vphantom{\fr{\Lambda_t}{d(t)}}\right]^{-1}\right\}+
\int\limits_0^{\infty}|\delta_t(\lambda)|\,d_{\lambda}\Phi\left(\fr{x}{\sqrt{\lambda}}\right)\,.\label{e34gv}
\end{multline}

\smallskip

\noindent
Д\,о\,к\,а\,з\,а\,т\,е\,л\,ь\,с\,т\,в\,о\,.\ Идея доказательства аналогична идее
доказательства равномерной оценки в работе~\cite{Gavrilenko} и основана на использовании
представления~(\ref{e32gv}). Имеем
\begin{multline*}
\Delta_t(x)\le\int\limits_0^{\infty}\left|{\sf
P}\left(\fr{1}{\sqrt{\lambda d(t)}}\sum_{j=1}^{N_{\lambda
d(t)}}X_j<\fr{x}{\sqrt{\lambda}}\right)-{}\right.\\
\left.{}-\Phi\left(\fr{x}{\sqrt{\lambda}}\right)\right|\,dG_t\left(\lambda
d(t)\right)+
\left|\int\limits_0^{\infty}\Phi\left(\fr{x}{\sqrt{\lambda}}\right)\,d\delta_t(\lambda)\right|\equiv{}\\
{}\equiv
I_1(t;x)+I_2(t;x)\,.
\end{multline*}
Подынтегральное выражение в $I_1(t;x)$ оценим с помощью теоремы~3.1 и получим
\begin{multline}
I_1(t;x)\le{}\\
{}\le
D\beta_3\int\limits_0^{\infty}\fr{\lambda^{3/2}}{\sqrt{\lambda
d(t)}\left(\lambda^{3/2}+|x|^3\right)}\,dG_t\left(\lambda d(t)\right)={}\\
{}=
\fr{D\beta_3}{\sqrt{d(t)}}{\sf E}\left\{
\fr{\Lambda_t}{d(t)}\left[\left(\fr{\Lambda_t}{d(t)}\right)^{3/2}+|x|^3\right]^{-1}\right\}\,.\label{e35gv}
\end{multline}
Интегрируя по частям, получаем
\begin{equation}
I_2(t;x)\le\int\limits_0^{\infty}|\delta_t(\lambda)|\,d_{\lambda}\Phi\left(\fr{x}{\sqrt{\lambda}}\right)\,.\label{e36gv}
\end{equation}
Требуемое утверждение вытекает из~(\ref{e35gv}) и~(\ref{e36gv}). Тео\-ре\-ма доказана.

\smallskip

В качестве примера применения теоремы~4.2 рассмотрим случай, когда
при каждом $t>0$ случайная величина $\Lambda_t$ имеет
гам\-ма-рас\-пре\-де\-ле\-ние. Этот случай представляет особый интерес с
точки зрения его применения в финансовой мате-\linebreak матике для
асимптотического обоснования адекватности таких популярных моделей
эволюции\linebreak финансовых индексов, как дисперсионные гам\-ма-про\-цес\-сы Леви
(variance-gamma L$\acute{\mbox{e}}$vy processes, VG-processes)~\cite{Madan} или
двусторонние гам\-ма-про\-цес\-сы~\cite{Carr} (также см.~\cite{Korolev2010}).

Смешанное пуассоновское распределение со структурным
гам\-ма-рас\-пре\-де\-ле\-ни\-ем является не чем иным, как отрицательным
биномиальным распределением. Убедимся в этом. Плотность
гам\-ма-рас\-пре\-де\-ле\-ния с параметром формы $r>0$ и параметром масштаба
$\sigma>0$, как известно, имеет вид
$$
g_{r,\si}(x)=\fr{\si^r}{\Gamma(r)}e^{-\si x}x^{r-1}\,,\enskip x>0\,.
$$
Таким образом, смешанное пуассоновское распределение со
структурным гам\-ма-рас\-пре\-де\-ле\-ни\-ем имеет характеристическую функцию
\begin{multline*}
\psi(z)=\int\limits_{0}^{\infty}\exp\left\{y(e^{iz}-1)\right\}\fr{\si^r}{\Gamma(r)}
e^{-\si y}y^{r-1}\,dy={}\\
{}=
\fr{\si^r}{\Gamma(r)}\int\limits_{0}^{\infty}\exp\left\{-\si
y\left(1+ \fr{1-e^{iz}}{\si}\right)\right\}y^{r-1}\,dy={}\\
{}=
\left(1+\fr{1-e^{iz}}{\si}\right)^{-r}\,.
\end{multline*}
Вводя новую параметризацию
$$
\si=\fr{p}{1-p}\  \ \left(p=\fr{\si}{1+\si}\right)\,,\ \ 
p\in(0,1)\,,
$$
окончательно получаем
$$
\psi(z)=\left(\fr{p}{1-(1-p)e^{iz}}\right)^r\,,\enskip z\in\R\,,
$$
что совпадает с характеристической функцией отрицательного
биномиального распределения с параметрами $r>0$ и $p\in(0,1)$.

Функцию гамма-распределения с параметром масштаба $\si$ и
параметром формы $r$ обозначим $G_{r,\si}(x)$. Несложно убедиться,
что
\begin{equation}
G_{r,\si}(x)\equiv G_{r,1}(\si x)\,.\label{e37gv}
\end{equation}
Случайную величину с функцией распределения $G_{r,\si}(x)$
обозначим $U(r,\si)$. Хорошо известно, что
$$
{\sf E}U(r,\si)=\fr{r}{\si}\,.
$$
Зафиксируем параметр $r$ и в качестве структурной случайной
величины~$\Lambda_t$ возьмем $U(r,\si)$, предполагая, что
$t=\si^{-1}$:
$$
\Lambda_t=U(r,t^{-1})\,.
$$
В качестве функции~$d(t)$ возьмем $d(t)\equiv{\sf E}\Lambda_t=$\linebreak $={\sf
E}U(r,t^{-1})$. Очевидно, что ${\sf E}U(r,t^{-1})=rt$. Тогда с
учетом~(\ref{e37gv}) будем иметь
\begin{multline*}
G_t\left(xd(t)\right)={\sf P}\left(U(r,t^{-1})<xrt\right)={}\\
{}={\sf P}\left(U(r,1)<xr\right)={\sf P}\left(U(r,r)<x\right)=G_{r,r}(x)\,.
\end{multline*}
Заметим, что функция распределения в правой час\-ти последнего
соотношения не зависит от~$t$. Поэтому указанный выбор функции
$d(t)$ тривиальным образом гарантирует выполнение условия~(\ref{e31gv})
теоремы~4.1. Более того, в таком случае $\delta_t(x)=0$ для всех
$t>0$ и $x\in\R$.

При этом случайная величина~$N(t)$ имеет отрицательное
биномиальное распределение с па\-ра\-мет\-ра\-ми~$r$ и $p=(t+1)^{-1}$:
\begin{multline}
{\sf P}\left(N(t)=k\right)=C_{r+k-1}^{k}p^r(1-p)^k
={}\\
{}=C_{r+k-1}^{k}\fr{t^k}{(1+t)^{r+k}}\,,\enskip k=0,1,2,\ldots
\label{e38gv}
\end{multline}
Здесь для нецелых $r$ величина $C_{r+k-1}^{k}$ определяется как
$$
C_{r+k-1}^{k} = \fr{\Gamma(r+k)}{k!\Gamma(r)}\,.
$$

Итак, в рассматриваемом случае $G_t\big(xd(t)\big)\equiv$\linebreak $\equiv
G_{r,r}(x)$, $d(t)\equiv rt$ и второе слагаемое в правой час\-ти~(\ref{e34gv}) 
равно нулю. Вычислим первое слагаемое в правой части~(\ref{e34gv}) для
рассматриваемой ситуации в предположении, что
\begin{equation}
r>\fr{1}{2}\,.\label{e39gv}
\end{equation}
Имеем
\begin{multline}
\fr{1}{\sqrt{d(t)}}\,{\sf
E}\left\{\fr{\Lambda_t}{d(t)}\left[\left(\fr{\Lambda_t}{d(t)}\right)^{3/2}+|x|^3\right]^{-1}\right\}={}\\
{}=
\fr{1}{\sqrt{rt}}\int\limits_0^{\infty}\fr{y}{y^{3/2}+|x|^3}\,dG_{r,r}(x)={}\\
{}=
\fr{r^{r-1/2}}{\sqrt{t}\Gamma(r)}\int\limits_0^{\infty}\fr{y^re^{-ry}\,dy}{y^{3/2}+|x|^3}\,.
\label{e40gv}
\end{multline}

Приведем несколько оценок для правой час\-ти~(\ref{e40gv}). Во-пер\-вых,
несложно убедиться, что при $y>0$
$$
\sup_x\fr{1+|x|^3}{y^{3/2}+|x|^3}=\max\{1,\,y^{-3/2}\}\,.
$$
Поэтому
\begin{multline*}
J_r(x)\equiv\int\limits_0^{\infty}\fr{y^re^{-ry}dy}{y^{3/2}+|x|^3}\le{}\\
{}\le
\fr{1}{1+|x|^3}
\int\limits_0^{\infty}y^re^{-ry}\sup_x\left\{\fr{1+|x|^3}{y^{3/2}+|x|^3}\right\}\,dy\le{}\\
{}\le
\fr{1}{1+|x|^3}\left[\int\limits_0^1y^{r-3/2}e^{-ry}\,dy+\int\limits_1^{\infty}y^re^{-ry}\,dy\right]
={}\\
{}=
\fr{1}{1+|x|^3}\left[\fr{\gamma_r(r-{1}/{2})}{r^{r-1/2}}+\fr{\Gamma(r+1)-\gamma_r(r+1)}{r^{r+1}}\right]\,,
\end{multline*}
где $\gamma_x(\alpha)$~--- неполная гам\-ма-функ\-ция,
$$
\gamma_x(\alpha)=\int\limits_0^x e^{-z}z^{\alpha-1}\,dz\,,\ \ \
\alpha>0\,,\ x\ge0\,.
$$
Следовательно,
\begin{multline*}
\fr{r^{r-1/2}}{\sqrt{t}\Gamma(r)}J_r(x)\le\fr{1}{\sqrt{t}(1+|x|^3)}\left[
\fr{\gamma_r(r-{1}/{2})}{\Gamma(r)}+{}\right.\\
\left.{}+
\fr{1}{\sqrt{r}}-\fr{\gamma_r(r+1)}{\Gamma(r+1)\sqrt{r}}\right]\,.
\end{multline*}
Таким образом, справедливо

\smallskip

\noindent
\textbf{Следствие 4.1.} \textit{Пусть случайная величина $N(t)$ имеет
отрицательное биномиальное распределение}~(\ref{e38gv}), $t>0$.
\textit{Предположим, что выполнены условия}~(\ref{e33gv}) \textit{и}~(\ref{e39gv}). \textit{Тогда для
любых $t>0$ и $x\in\R$ справедлива оценка}

\noindent
\begin{multline*}
\left|{\sf P}(S(t)<x\sqrt{rt})-\int\limits_{0}^{\infty}
\Phi\left(\fr{x}{\sqrt{y}}\right)dG_{r,r}(y)\right|\le{}\\
{}\le \fr{22{,}7707\beta_3}{\sqrt{t}(1+|x|^3)}
K_1(r)\,,
\end{multline*}
\textit{где}
$$
K_1(r)=\fr{\gamma_r(r-{1}/{2})}{\Gamma(r)}+
\fr{1}{\sqrt{r}}-\fr{\gamma_r(r+1)}{\Gamma(r+1)\sqrt{r}}\,.
$$

\smallskip

С другой стороны, очевидно, что
$\gamma_x(\alpha)\le\Gamma(\alpha)$ при любых $x\ge0$ и
$\alpha>0$. Поэтому величину $K_1(r)$ можно оценить выражением,
содержащим только <<полные>> гам\-ма-функ\-ции:
\begin{equation}
K_1(r)\le \fr{\Gamma(r-{1}/{2})}{\Gamma(r)}+
\fr{1}{\sqrt{r}}\,.\label{e41gv}
\end{equation}

\smallskip

При $r=1$ случайная величина $\Lambda_t=U(1,t^{-1})$ имеет
показательное распределение с параметром $\sigma=1/t$.
Следовательно, как легко убедиться, смешанная пуассоновская
случайная величина~$N(t)$ с таким структурным распределением имеет
геометрическое распределение с параметром
$p=(t+1)^{-1}$. При этом предельное (при 
$t\rightarrow \infty$) распределение для стандартизованной
геометрической суммы~$S(t)$ является распределение Лапласа с
плотностью
$$
l(x)=\fr{1}{\sqrt{2}}\,e^{-\sqrt{2}\,|x|},\enskip x\in\R\,.
$$
Функцию распределения, соответствующую плотности $l(x)$, обозначим~$L(x)$:
$$
L(x)=\begin{cases}
\fr{1}{2}\,e^{\sqrt{2}x}\,, & \mbox{если}\ \ x<0\,,\\
1-\fr{1}{2}\,e^{-\sqrt{2}x}\,, & \mbox{если}\ \ x\ge0\,.
\end{cases}
$$
Из следствия~4.1 при этом получаем

\smallskip

\noindent
\textbf{Следствие 4.2.} \textit{Пусть случайная величина $N(t)$ имеет
геометрическое распределение с параметром $p=(1+t)^{-1}$, $t>0$.
Предположим, что выполнены условия}~(\ref{e33gv}). \textit{Тогда для любых $t>0$ и
$x\in\R$ справедлива оценка}
$$
\left|{\sf P}(S(t)<x\sqrt{t})-L(x)\right|\le
\fr{50{,}7652\beta_3}{\sqrt{t}\left(1+|x|^3\right)}\,.
$$

\smallskip

Заметим, что оценка, приведенная в следствии~4.2, точнее
равномерной оценки, приведенной в работах~\cite{KorolevBEs} и~\cite{KorSchev}, для
$|x|>4{,}5334$.

Заметим также, что использование оценки~(\ref{e41gv}) для $K_1(r)$
увеличивает абсолютную константу в следствии~4.2 до 63,1308~---
такова цена упрощения.

{\small\frenchspacing
{%\baselineskip=10.8pt
%\addcontentsline{toc}{section}{Литература}
\begin{thebibliography}{99}


\bibitem{Esseen42} 
\Au{Esseen C.\,G.} On the Liapunoff limit of error in
the theory of probability~// Ark. Mat. Astron. Fys., 1942. Vol.~A28.
No.\,9. P.~1--19.

\bibitem{Berry41} 
\Au{Berry A.\,C. } The accuracy of the Gaussian
approximation to the sum of the distributed random variables~// J.~Theor. Probab., 1994. Vol.~2. No.\,2. P.~211--224.

\bibitem{KorolevVOz} 
\Au{Королев В.\,Ю., Шевцова И.\,Г.} Уточнение верхней оценки
абсолютной постоянной в неравенстве Берри--Эссеена для смешанных
пуассоновских случайных сумм~// Докл. РАН,
2010. Т.~431. Вып.~1. С.~16--19.

\bibitem{KorolevBEs} 
\Au{Королев В.\,Ю., Шевцова И.\,Г.} Уточнение неравенства
Берри--Эссеена с приложениями к пуассоновским и смешанным
пуассоновским случайным суммам~// Обозрение прикладной и
промышленной математики, 2010. Т.~17. Вып.~1. С.~25--56.

\bibitem{Ess} \Au{Esseen C.\,G.} 
A moment inequality with an application to
the central limit theorem~// Skand. Aktuarrietidskr, 1956. Vol.~39.
P.~160--170.

\bibitem{Meshalkin} \Au{Мешалкин Л.\,Д., Рогозин Б.\,А.} Оценка расстояния между
функциями распределения по близости их характеристических функций
и ее применение к центральной предельной теореме~// Предельные
теоремы теории вероятностей.~--- Ташкент: АН УзССР, 1963. С.~40--55.

\bibitem{Nagaev} \Au{Нагаев С.\,В.} Некоторые предельные теоремы для больших
уклонений~// Теория вероятностей и ее применения, 1965. Т.~10.
Вып.~2. С.~231--254.

\bibitem{Mich81} \Au{Michel R.} On the constant in the nonuniform version of the
Berry--Esseen theorem~// Z.~Wahrsch. verw. Geb., 1981. Bd.~55. P.~109--117.

\bibitem{KorSchev} \Au{Korolev  V., Shevtsova~I.} An improvement of the
Berry--Esseen inequality with applications to Poisson and mixed
Poisson random sums~// Scandinavian Actuarial J., 2010.
{\sf http://www.informaworld.com/10.1080/03461238.\newline 2010.485370}.

\bibitem{Nefedova} \Au{Нефедова Ю.\,С., Шевцова И.\,Г.} О~точности нормальной
аппроксимации для распределений пуассоновских случайных сумм~//
Информатика и её применения, 2011. Т.~5. Вып.~1. С.~\pageref{nefedova1}--\pageref{end-nefedova}.

\bibitem{Paditz89} \Au{Paditz L.} On the analytical structure of the constant in
the nonuniform version of the Esseen inequality~// Statistics, 1989. Vol.~20. No.~3. P.~453--464.

\bibitem{Mich93} \Au{Michel R.} On Berry--Esseen results for the compound
Poisson distribution~// Insurance: Mathematics and Economics,
1993. Vol.~13. No.\,1. P.~35--37.

\bibitem{Rychlik} \Au{Rychlik Z.} Nonuniform central limit bounds and their
applications~// Теория вероятностей и ее применения, 1983. T.~28.
Вып.~3. С.~646--652.

\bibitem{Paditz81} \Au{Paditz L.} Einseitige Fehlerabsch$\ddot{\mbox{a}}$tzungen im zentralen
Grenzwertsatz~// Math. Operationsforsch. und Statist., ser.
Statist., 1981. Bd.~12. P.~587--604.

\bibitem{Tysiak} \Au{Tysiak W.} Gleichm$\ddot{\mbox{a}}${\ss}ige und
nicht-gleichm$\ddot{\mbox{a}}${\ss}ige Berry--Esseen-Absch$\ddot{\mbox{a}}$tzungen.
Dissertation.~--- Wuppertal, 1983.

\bibitem{KorBenShorg} \Au{Королев В.\,Ю., Бенинг В.\,Е., Шоргин С.\,Я.} 
Математические основы теории риска.~--- М.: Физматлит, 2007.

\bibitem{BenKor2002} \Au{Bening V., Korolev V.}  Generalized Poisson models
and their applications in insurance and finance.~--- Utrecht: VSP, 2002.

\bibitem{Shev2007} \Au{Шевцова И.\,Г.} О~точности нормальной аппроксимации для
распределений пуассоновских случайных сумм~// Обозрение
промышленной и прикладной математики, 2007. Т.~14. Вып.~1. С.~3--28.

\bibitem{Greenwood1920} \Au{Greenwood M., Yule G.\,U.} An inquiry into the nature of
frequency-distributions of multiple happenings, etc.~// J.~Roy.
Statist. Soc., 1920. Vol.~83. P.~255--279.

\bibitem{Delaporte} \Au{Delaporte P.} Un probl$\acute{\mbox{e}}$me de tarification de l'assurance
accidents d'automobile examin$\acute{\mbox{e}}$ par la statistique math$\acute{\mbox{e}}$matique~// 
Trans. 16th  Congress (International) of Actuaries.~--- Brussels,
1960. Vol.~2. P.~121--135.

\bibitem{Holla} \Au{Holla M. S.} On a Poisson-inverse Gaussian distribution~//
Metrika, 1967. Vol.~11. P.~115--121.

\bibitem{Sichel} \Au{Sichel H. S.} On a family of discrete distributions
particular suited to represent long tailed frequency data~// 
3rd Symposium on Mathematical Statistics Proceedings~/ Ed.\ N.\,F.~Laubscher.~---
Pretoria: CSIR, 1971. P.~51--97.

\bibitem{Willmot} \Au{Willmot G.\,E.} The Poisson-inverse Gaussian distribution
as an alternative to the negative binomial~// Scandinavian Actuar.~J., 1987. P.~113--127.

\bibitem{Irwin} \Au{Irwin J.\,O.} The generalized Waring distribution applied to
accident theory~// J.~Royal Statist. Soc., Ser. A, 1968. Vol.~130. P.~205--225.

\bibitem{Seal} \Au{Seal H.} Survival probabilities. The goal of risk
theory.~--- Chichester\,--\,New York\,--\,Brisbane\,--\,Toronto: Wiley, 1978.

\bibitem{Grandell} \Au{Grandell J.} Mixed Poisson processes.~--- London:
Chapman and Hall, 1997.

\bibitem{Korolev96} \Au{Korolev V.\,Yu.} A general theorem on the limit behavior of
superpositions of independent random processes with applications
to Cox processes~// J. Math. Sci., 1996. Vol.~81. No.\,5. P.~2951--2956.

\bibitem{Gavrilenko} \Au{Гавриленко С.\,В., Королев В.\,Ю.} Оценки скорости
сходимости смешанных пуассоновских случайных сумм~// 
Системы и средства информатики. Спец. вып. Математические модели в информационных технологиях.~--- М.: ИПИ РАН,
2006. С.~248--257.

\bibitem{Madan} \Au{Madan D.\,B.,  Seneta E.} The variance gamma ($V.G.$)
model for share market return~// J. Business, 1990. Vol.~63. P.~511--524.

\bibitem{Carr} \Au{Carr P.\,P., Madan D.\,B., Chang~E.\,C.} The variance gamma
process and option pricing~// Eur. Finance Rev., 1998. Vol.~2. P.~79--105.

 \label{end\stat}

\bibitem{Korolev2010} \Au{Королев В.\,Ю.} Ве\-ро\-ят\-ност\-но-ста\-ти\-сти\-че\-ские методы
декомпозиции волатильности хаотических процессов.~--- М.: МГУ, 2010.
 \end{thebibliography}
}
}


\end{multicols}   %2
\def\stat{ul}

\def\tit{О ТОЧНОСТИ ПРИБЛИЖЕНИЙ НОРМИРОВАННЫХ ХИ-КВАДРАТ РАСПРЕДЕЛЕНИЙ   
   АСИМПТОТИЧЕСКИМИ РАЗЛОЖЕНИЯМИ ЭДЖВОРТА--ЧЕБЫШЕВА$^*$}

\def\titkol{О точности приближений нормированных хи-квадрат распределений   
%асимптотическими разложениями Эджворта--Чебышева
}

\def\autkol{Г.~Кристоф, В.\,В.~Ульянов}
\def\aut{Г.~Кристоф$^1$, В.\,В.~Ульянов$^2$}

\titel{\tit}{\aut}{\autkol}{\titkol}

{\renewcommand{\thefootnote}{\fnsymbol{footnote}}\footnotetext[1]
{Исследования выполнены при частичной поддержке РФФИ, гранты 08-01-00567, 09-01-12180.}}

\renewcommand{\thefootnote}{\arabic{footnote}}
\footnotetext[1]{Магдебургский университет, факультет математики,  gerd.christoph@ovgu.de}
\footnotetext[2]{Московский государственный университет им.\ М.\,В.~Ломоносова, 
факультет вычислительной математики и кибернетики, vulyan@gmail.com}

\Abst{Рассмотрено распределение нормированной хи-квад\-рат случайной величины с  $n$~степенями свободы. 
С~помощью разложений Эджвор\-та--Че\-бы\-ше\-ва построены вычислимые оценки приближений этого распределения 
различного порядка: $O(n^{-1/2})$, $O(n^{-1})$ и $O(n^{-3/2})$. Результаты такого типа полезны в приложениях, 
в частности при анализе свойств статистик отношения правдоподобия.
}

\KW{асимптотические разложения; оценки погрешности; хи-квадрат распределение}

      \vskip 14pt plus 9pt minus 6pt

      \thispagestyle{headings}

      \begin{multicols}{2}
      
            \label{st\stat}

\section{Введение}

Пусть ${\cal X}_n^2$ есть случайная величина, имеющая хи-квад\-рат функцию распределения   $G_n(x)$ с 
$n$~степенями свободы  и функцию плотности
\begin{equation*}
%\label{g0}
p_{{\cal X}_n^2}\,(x) =
\fr{1}{2^{n/2}\,\Gamma(n/2)} \, x^{-1+n/2}  e^{-x/2}\,I_{(0,\,\infty)}(x)\,,
\end{equation*}
где $I_A(x)$ обозначает индикаторную функцию множества~$A$. Отметим, что
$ E({\cal X}_n^2) = n $,  $ Var({\cal X}_n^2) =2\,n$.

\smallskip

В прикладных исследованиях   хи-квад\-рат распределение возникает часто. 
Например, распределение нормированной хи-квад\-рат статистики
\begin{equation*}
T_1 = {\cal X}^2_n - n \log \fr{{\cal X}^2_n}{n} - n
%\label{chi1}
\end{equation*}
появляется в критериях отношения правдоподобия при проверке нулевой гипотезы  о равенстве дисперсии~$\sigma^2$ 
заданному значению, когда выборка объема $n+1$  берется из нормального распределения  $N(\mu, \sigma^2)$, 
см.\ также соответствующие многомерные обобщения в~[1]. Другие примеры использования хи-квад\-рат 
случайных величин в приложениях, в частности  в информатике, см. в~\cite{9-cr}; в гл.~5, 13, 15, 16 в~\cite{2-cr}; 
а также в~\cite{10-cr}, где описан новый подход по <<очистке>> изображений. При этом оказывается, 
что уже при  малом числе степеней свободы хи-квад\-рат распределение может быть успешно заменено 
нормальным распределением. В настоящей работе найдены оценки погрешности такой замены. 
В~[1] доказано, что
\begin{equation}
\label{g1}
\sup_x\,|P (T_1 \leq x) - G_1 (x)| \leq C(n)\,,
\end{equation}
где  $C(n)$ есть вычислимая величина, при этом  $C(n) = O(n^{-1})$ при
$n \rightarrow\infty$. В лемме~3 в~\cite{1-cr}
найдена оценка сверху для приближения функции распределения нормированной случайной величины ${\cal X}_n^2$ с 
помощью разложения Эджворта--Чебышева первого порядка. Отметим, что эта оценка вносит заметный вклад в величину
 $C(n)$  в правой части~(\ref{g1}).
 
 \smallskip
 
Цель настоящей работы~--- уточнить эту оценку, используя  разложения Эджворта--Чебышева второго порядка.
\smallskip

Пусть $X= ({\cal X}_1^2 - 1)/\sqrt{2}$ есть нормированная хи-квадрат случайная величина 
с одной степенью свободы и   $X, X_1, X_2,\ldots$~--- независимые одинаково распределенные 
случайные величины. Тогда нормирование  ${\cal X}_n^2$ приводит к случайной величине
\begin{multline*}
V_n =\,(2\,n)^{-1/2}\left({\cal X}_n^2-n\right)= {}\\
{}=n^{-1/2}\left(X_1 + X_2 + \ldots + X_n\right)\,. 
\end{multline*}
В силу центральной предельной теоремы функция распределения
$$ 
F_n(x) = P(V_n \leq x)=P\left({\cal X}_n^2-n \leq \sqrt{2\,n}\,x\right)
$$
стремится к нормальному закону   
$$ 
\Phi(x) =  (2\pi)^{-1/2} \int\limits_{- \infty}^x e^{-u^2/2}\,du 
$$
при $n \rightarrow~\infty$.

Кроме этого, из неравенства Берри--Эссеена находим

\noindent
\begin{equation}
\label{g1a}
\sup\nolimits_x |F_n(x) - \Phi(x)| \leq \fr{C\,E|X|^3 }{ \sqrt{n}} \leq \fr{1{,}4720}{\sqrt{n}}\,,
\end{equation}
где $E|X|^3 = 3{,}0729\ldots$ и $C = 0{,}478\ldots$ (см., например,~\cite{3-cr, 4-cr}).
%
Если использовать не только информацию о третьем абсолютном моменте распределения~$V_n$, 
но и другие свойства   нормированной величины  ${\cal X}_n^2$, то можно получить оценку, 
более точную, чем~(\ref{g1a}). В~\cite{5-cr} доказано, что
при всех $\lambda \in (0,\sqrt{3}-1)$  и целых $n\geq 1$
$$\sup\nolimits_x |F_n(x) - \Phi(x)| \leq D(\lambda, n)\,,
$$
где

\noindent
\begin{multline} 
D(\lambda, n) = \fr{1}{3\,\sqrt{\pi\,n}} +
\fr{2}{\pi n}\left(
\vphantom{\fr{(1+\lambda^2)^{1-n/4}}{\lambda^2}}
\fr{2(1-\lambda)}{(2-2\lambda - \lambda^2)^2} +{}\right.\\
\left.{}+
\fr{(1+\lambda^2)^{1-n/4}}{\lambda^2} +
\fr{1}{\lambda^2}e^{-\lambda^2 n/4}
\right)\,.
\label{g2}
\end{multline}
Поскольку   $E(X^3) = 2\sqrt{2}$ и $E(X^4)=15$, используя
разложения Эджворта--Чебышева первого и второго порядков
\begin{equation}
\label{g2a} 
\Phi_{1,n}(x) = \Phi (x) - \fr{\sqrt{2}\,(x^2-1)}{3 \sqrt{n}}
\,\fr{1}{\sqrt{2\pi}}\, e^{-x^2/2}\,;
\end{equation}

\vspace*{-12pt}

\noindent
\begin{multline}
\Phi_{2,n}(x) = \Phi (x) -\fr{e^{-x^2/2}}{\sqrt{2\,\pi}}
\left( \fr{\sqrt{2}\left(x^2-1\right)}{3\sqrt{n}}+{}\right.\\
\left.{}+
   \fr{x^5 -10x^3+15x}{9n}+\fr{x^3-3x}{2n}\right)\,,
   \label{g2b}
\end{multline}
получаем

\noindent
\begin{align*}
 F_n(x) = \Phi_{1,n}(x) + O(n ^{-1})\,;\\
 F_n(x) = \Phi_{2,n}(x) + O(n^{-3/2})
\end{align*}
 при $n \to \infty$.
В~\cite{6-cr}  доказано следующее неравенство (см.\ пример~3 в~\cite{6-cr} при  $a=b=1/2$):
\begin{multline}
\sup\nolimits_x\, |F_n(x) - \Phi_{1,n}(x)|
\leq D_1^*(n)=\fr{1{,}9}{n} \left( \fr{n}{n-1}\right)^2
+{}\\
{}+\fr{15{,}59}{n}\, 0{,}9906^n+ C(n) 0{,}9894^n
\label{g3} 
\end{multline}
с
$C(n) = 15{,}21/(n-4)$, для $n > 32$ и $C(n) = 0{,}5271$ для $4 \leq n \leq 32.$
%
Аналогичные оценки в   $L^1$-норме для разности плотности   $p_{V_n} (x)$ и
$d\Phi_{1,n}(x)/dx$ получены в~\cite{7-cr}.
%
Приведем исправленный вариант леммы~3 из~\cite{1-cr} в виде теоремы.

\medskip

\noindent
\textbf{Теорема~1.}
\textit{Для всех $\lambda \in (0,\sqrt{3}-1)$  и целых $n\geq 1$ справедливо неравенство
$$\sup\nolimits_x |F_n(x) - \Phi_{1,n}(x)| \leq D_1(\lambda, n)\,, $$
где}
\columnbreak

\noindent
\begin{multline}
D_1(\lambda, n) = \fr{2}{\pi n}\left(
\vphantom{\fr{(1+\lambda^2)^{1-n/4}}{\lambda^2}}
\fr{4}{9} +
\fr{2(1-\lambda)}{(2-2\lambda - \lambda^2)^2} +{}\right.\\
\!\left.{}+\fr{(1+\lambda^2)^{1-n/4}}{\lambda^2} +
\fr{3+\lambda}{3\lambda^2}\,e^{-\lambda^2(3-\lambda)n/(12+4\lambda)}
\right).\!\!
\label{g4}
\end{multline}

\smallskip

Отметим, что первое слагаемое в скобках в правой части~(\ref{g4}) есть 
$4/9$, а не $1/9$, как было напечатано в~\cite{1-cr}. Это  вытекает из оценки 
$ I_{11} \leq 16/(9n)$ на с.~1160. В~связи с этим ко всем значениям в таблице для~ $D(n)$ 
на с.~1155 в~\cite{1-cr} следует прибавить  $2/(3\pi n)$.

\section{Основные результаты и их обсуждение}

Сначала приведем две оценки точности приближений для распределения нормированной хи-квад\-рат 
случайной величины ${\cal X}_n^2$, полученных с помощью разложения Эджворта--Чебышева второго порядка.

\medskip

\noindent
\textbf{Теорема~2.} \textit{При всех $\lambda \in (0,\,0{,}75)$  и целых $n\geq 1$ имеем
$$
\sup\nolimits_x \left\vert F_n(x) - \Phi_{2,n}(x)\right\vert \leq D_2(\lambda, n)\,, 
$$
где с $a=a(\lambda)=1-\lambda^2/2-\lambda^4/(3-3\lambda)$ и $b=b(\lambda)=$\linebreak $=1-\lambda^2/2$,}

\noindent
\begin{multline*}
D_2(\lambda, n)  = \fr{1}{\sqrt{\pi}}\left(\fr{1349}{270n^{3/2}}+\fr{21}{n^{5/2}}\right)+{}\\
{}+
\fr{1}{2\pi}\left[\fr{32}{(3-3\lambda)a^3n^2} +\fr{12}{b^4n^2}+{}\right.\\
\left. {}+\fr{4}{\lambda^2n} (1+\lambda^2)^{1-n/4}+
\left( \ln\left(1+\fr{4}{\lambda^2n}\right)+ \fr{2\lambda}{3}+{}\right.\right.\\
\left. \left.{}+\fr{8}{3\lambda n(1+\sqrt{1+16/(\pi \lambda^2 n)})}
+\fr{\lambda^4 n}{18}+{}\right.\right.\\
\left. \left. {}+ \fr{17\lambda^2}{18} +\fr{34}{9\,n}\right)e^{-\lambda^2 n/4}\right]
\,.
%\label{g5111}
\end{multline*}



Значения для $D_2(\lambda,n)$ при различных значениях  $\lambda$ и~$n$, вычисленные в    MAPLE,
представлены в табл.~1.

Результат теоремы~2 можно улучшить, построив более точные, но и более громоздкие оценки. Определим функции
\begin{align*}
  U_8(A)&= \Bigg(A^7+7A^5+35A^3+ 105A + {}\\
&{}+\fr{210}{A+\sqrt{A^2+4}}\Bigg)e^{-A^2/2};\\
  U_7(B)&= \left(B^3+3B^2+6B+6\right)e^{-B};
\end{align*}


  \noindent
\begin{center} %tabl1
\parbox{56mm}{{\tablename~1}\ \ \small{Значения для $D_2(\lambda,n)$ 
при различных значениях  $\lambda$ и~$n$, вычисленные в    MAPLE}}
%\begin{center} %fig1

\vspace*{2ex}
\tabcolsep=14.6pt
{\small
\begin{tabular}{|c|c|c|}
\hline
$n$   & $\lambda$& $D_2(\lambda,n)$\\
\hline
\hphantom{9}50 &0,60 & 0,01862\\
100  & 0,54& 0,00423\\
200  & 0,44 & 0,00121\\
300&   0,37 & 0,00062\\
500&  0,30 & 0,00028\\
1000\hphantom{9} & 0,23 & 0,00009\\
\hline 
\end{tabular}}
\end{center}
%\vspace*{12pt}
%\begin{center}
%\end{center}
\vspace*{9pt}

\smallskip
\addtocounter{table}{1}

\noindent
  \begin{align*}
  U_6(A)&= \Bigg(\!A^5+5A^3+ 15A + \fr{30}{A+\sqrt{A^2+4}}\!\Bigg)e^{-A^2/2};\\
U_5(B)& = \left(B^2+2B+2\right)e^{-B}\,;\\
U_4(A)&= \left(A^3+ 3A + \fr{6}{A+\sqrt{A^2+4}}\right)e^{-A^2/2} \,.
\end{align*}

%\vspace*{6pt}




\noindent
\textbf{Теорема 3.} \textit{Пусть $a=a(\lambda) $ и $b=b(\lambda) $  те же, что в теореме~2. 
При всех $\lambda \in (0,0.75)$  и целых $n\geq 1$ имеем
$$
\sup\nolimits_x \left\vert F_n(x) - \Phi_{2,n}(x)\right\vert \leq D_2^*(\lambda, n)\,, 
$$
где
\begin{multline*}
D_2^*(\lambda, n) = \fr{2{,}8189}{n^{3/2}}+\fr{1{,}6977}{ (1-\lambda)a^3n^2} +
\fr{1{,}9099}{b^4n^2}+{}\\
{}+\fr{11{,}8480}{n^{5/2}} -U(\lambda,n)
+\fr{0{,}6366}{\lambda^2 n} (1+\lambda^2)^{1-n/4}
+{}\\
{}+\left(\fr{0{,}6366}{\lambda^2 n}+
\fr{0{,}2122}{\lambda \,n}+
\fr{0{,}6012 }{n}+ 0{,}1061\lambda + {}\right.\\
\left.{}+0{,}1503 \lambda^2 +0{,}0088 \lambda^4 n
\vphantom{\fr{0{,}}{0{,}}}\right)e^{-\lambda^2 n/4}
\end{multline*}
и функция $U(\lambda,n)$ определяется формулой}
\begin{multline*} %\label{g5a}
 U(\lambda,n)  = \fr{\sqrt{2}}{\pi n^{3/2}}\left[ 
 \vphantom{\left(\fr{1}{2}\right)^1}
 \left(\fr{1}{81}\,U_8\left(\lambda \sqrt{\fr{n}{2}}\right) + {}\right.\right.\\
\left. {}+
 \fr{1}{6}\, U_6\left(\lambda\,\sqrt{\fr{n}{2}}\right) + \fr{2}{5}\, U_4\left(\lambda \sqrt{\fr{n}{2}}\right) \right)+{}\\
{} + \left(\fr{4}{3(1-\lambda)a^3}\, U_5\left(a\lambda^2 \fr{n}{4}\right) + {}\right.\\
{}+\left.\left.\fr{1}{2b^4}\, 
U_7\left(b \lambda^2 \fr{n}{4}\right)\right)\!\left(\fr{2}{n}\right)^{1/2} 
+ \fr{1}{5 n}\, U_8\left(\lambda\,\sqrt{\fr{n}{2}}\right) \!\right].
\end{multline*}

Значения для $D_2^*(\lambda,n)$ при различных значениях  $\lambda$ и~$n$, вычисленные в    MAPLE,
представлены в табл.~2.

%\medskip


Теперь можно оценить   $\sup\nolimits_x \left\vert F_n(x) - \Phi_{1,n}(x)\right\vert$ и 
$\sup\nolimits_x \left\vert F_n(x) - \Phi(x)\right\vert$ с помощью теорем~2 и~3.   

Пусть $Q_2(x)/n$ есть член порядка   $1/n$ в разложении Эджворта--Чебышева (\ref{g2b}). Тогда

\columnbreak


\noindent
\begin{center}
\parbox{56mm}{{\tablename~2}\ \ \small{Значения для 
$D_2^*(\lambda,n)$ при различных значениях  $\lambda$ и~$n$, вычисленные в    MAPLE}}
%\begin{center} %fig1

\vspace*{2ex}
\tabcolsep=15.6pt
{\small
\begin{tabular}{|c|c|c|}
\hline
$n$    & $\lambda$ & $D_2^*(\lambda,n)$\\
\hline
\hphantom{9}20   & 0,69 & 0,06698\\
\hphantom{9}30   &0,62 & 0,03981\\
\hphantom{9}40    &0,61 & 0,02030\\
\hphantom{9}50  & 0,59& 0,01353\\
100   & 0,51 & 0,00389\\
150 & 0,45 & 0,00193\\
\hline 
\end{tabular}
}
\end{center}
%\vspace*{9pt}

\smallskip
\addtocounter{table}{1}


%\vspace*{-6pt}

\noindent
\begin{multline*} 
\sup\nolimits_x|Q_2(x)|=\sup\nolimits_x \left|\fr{e^{-x^2/2}}{\sqrt{2\pi}}
  \left( \fr{x^5 -10x^3+15x}{9}+{}\right.\right.\\
\left.\left.  {}+\fr{x^3-3x}{2}\right)\right|\leq 0{,}1269\,.
  \end{multline*}
Заметим, что $Q_2(x)$ принимает экстремальные значения при   $x=\pm 1{,}43$.

\smallskip

\noindent
\textbf{Следствие 1.} \textit{При всех $\lambda \in (0,3/4)$  и целых $n\geq 1$ имеем}

\vspace*{-3pt}

\noindent
\begin{multline}
\!\! \!\!\!\sup\nolimits_x \left\vert F_n(x) - \Phi_{1,n}(x)\right\vert
  \leq \sup_x \left\vert Q_2(x)\right\vert  \fr{1}{n} + 
  D_2(\lambda,n) \leq {}\\
  {}\leq \fr{0{,}12690}{n} +D_2(\lambda,n)=D_1^{**}(\lambda,n)\,.
  \label{g6a}
\end{multline}

%\vspace*{-3pt}

\noindent
\begin{center}
\parbox{80mm}{{\tablename~3}\ \ \small{Значения для $D^*_1(n)$, $D_1(\lambda,n)$ и $D_1^{**}(\lambda,n)$ 
при различных значениях  $\lambda$  и~$n$, вычисленные в    MAPLE}}
%\begin{center} %fig1

\vspace*{2ex}
%\tabcolsep=5pt
{\small
\begin{tabular}{|c|c||c|c||c|c|}
\hline
$n$ & $D^*_1(n)$ & $\lambda$ &   $D_1(\lambda,n)$ & $\lambda$ &   $D_1^{**}(\lambda,n)$\\
\hline
\hphantom{9}50   & 0,42808 & 0,50 & 0,03858 & 0,60 & 0,02115\\
100   & 0,13460 & 0,45 & 0,01275 & 0,54 & 0,00550\\
150  & 0,05911 & 0,42 & 0,00712 & 0,48 & \hphantom{9}0,002839\\
200  & 0,03059 & 0,40 & 0,00501 & 0,44 & 0,00185\\
300& 0,01153 & 0,31 & 0,00281 & 0,38 & 0,00105\\
500 & 0,00424 & 0,25 & 0,00153 & 0,31 & 0,00053\\
\hline 
\end{tabular}
}
\end{center}
%\vspace*{9pt}

\smallskip
\addtocounter{table}{1}


В табл.~3 для различных значений $n$ и $\lambda$ и в случае приближения~$F_n(x)$  
разложением первого порядка $\Phi_{1,n}(x)$ приведены  величины погрешностей   $D^*_1(n)$ 
из неравенства~(\ref{g3}), см.~\cite{6-cr}; величины   $D_1(\lambda,n)$ из~(\ref{g4}), см.~\cite{1-cr}, 
и $D_1^{**}(\lambda,n)$ из неравенства~(\ref{g6a}), доказанного в настоящей работе.



Пусть $Q_1(x)$ есть член порядка   $1/\sqrt{n}$ в разложениях Эджворта--Чебышева~(\ref{g2a})  
или~(\ref{g2b}). Тогда

\noindent
\begin{multline*} 
\sup\nolimits_x|Q_1(x)|=\sup\nolimits_x \left|\fr{e^{-x^2/2}}{\sqrt{2\pi}}\,
  \fr{\sqrt{2}\left(x^2-1\right)}{3 \sqrt{n}}\right|={}\\
  {}= \fr{1}{3\sqrt{\pi n}}\leq \fr{0{,}18806}{\sqrt{n}}\,.
  \end{multline*}
Заметим, что $Q_1(x)$ достигает экстремального значения при   $x=0$.

\medskip

\noindent
\textbf{Следствие 2.} \textit{При всех $\lambda \in (0,3/4)$  и целых $n\geq 1$ имеем}
\begin{multline}
\sup\nolimits_x \left\vert F_n(x) - \Phi(x)\right\vert \leq  \fr{0{,}18806}{\sqrt{n}}+
\fr{0{,}12690}{n} +{}\\
{}+D_2^*(\lambda,n)= D_0(\lambda,n)\,.
\label{g6b}
\end{multline}

\noindent
\begin{center}
\parbox{80mm}{{\tablename~4}\ \ \small{Значения для БЭ-оценки, 
$D(\lambda,n)$ и $D_0(\lambda,n)$ при различных значениях~ $\lambda$ и~$n$, 
вычисленные в    MAPLE}}
%\begin{center} %fig1

\vspace*{2ex}
\tabcolsep=4.5pt
{\small 
\begin{tabular}{|c|c||c|c||c|c|}
\hline
$n$    &БЭ-оценка & $\lambda$ &   $D(\lambda,n)$ & $\lambda$ & $D_0(\lambda,n)$\\
\hline
\hphantom{9}50   & 0,20818 & 0,483 & 0,05498 & 0,57 & 0,04266\\
100  & 0,14720 & 0,416 & 0,02763 & 0,49 & 0,02397\\
300 & 0,08499 & 0,295 & 0,1267\hphantom{9} & 0,37 & 0,01190\\
500 & 0,06583 & 0,244 & 0,00935 & 0,30 & 0,00894\\
1000\hphantom{9} & 0,04655 & 0,196 & 0,00638 & 0,23 & 0,00628\\
\hline 
\end{tabular}
}
\end{center}
\vspace*{9pt}

\smallskip
\addtocounter{table}{1}

В табл.~4 для различных значений~$n$ и~$\lambda$ и в случае приближения $F_n(x)$  
стандартной нормальной функцией распределения   $\Phi(x)$ приведены  величины 
погрешностей, даваемые неравенством Берри--Эссеена (БЭ-оцен\-ка)~(\ref{g1a}), 
а также значения $D(\lambda,n)$, определенного в~(\ref{g2}), см.~\cite{5-cr}, 
и значения $D_0(\lambda,n)$ из неравенства~(\ref{g6b}), доказанного в настоящей работе.


Значения параметра  $\lambda$, появляющегося в выражениях для   $D_2(\lambda, n)$, 
$D_1(\lambda, n)$ и $D_0(\lambda, n)$, следует выбирать в зависимости от размера выборки~$n$.

\section{Доказательство теорем~2 и~3}

Пусть $f_n(t)$ и $g^*_n(t)$ есть характеристическая функция распределения   $F_n(x)$ и преобразование Фурье--Стил\-тье\-са 
функции   $\Phi_{2,n}(x)$ соответственно. Тогда, обозначая   $m=n/2$, имеем
\begin{align*}
 f_n(t) &= e^{-it\sqrt{m}}\left(1-\fr{i\,t}{\sqrt{m}}\right)^{-m}\,;
\\
\quad g^*_n(t) &=
\left(1+\fr{(i\,t)^3}{3\sqrt{m}}+\fr{(it)^6}{18m}+\fr{(it)^4}{4m} \right) e^{-t^2/2}\,. 
\end{align*}
Поскольку функции $F_n(x)$ и $\Phi_{2,n}(x)$ имеют непрерывные производные, воспользуемся 
формулой обращения для характеристических функций
\begin{multline*}
\left| F_n(x) - \Phi_{2,n}(x) \right| ={}\\
{}=  \fr{ 1}{ 2\pi}\,\left|
\int\limits_{-\infty}^\infty e^{-\,itx}\fr{f_n(t)\,-\,g^*_n(t)}{-\,it}\,dt \right| \leq{}\\
{}\leq  \fr{ 1}{ 2\pi}\left( I_1+I_2+I_{31}+I_{32}\right)\,, 
\end{multline*}
где
\begin{align*}
 I_1 &= \int\limits_{|t|<\lambda\,\sqrt{m}} \fr{1}{|t|}
\left|\vphantom{\fr{(it)^3}{\sqrt{m}}}
f_n(t)-{}\right.\\
&\left.{}-e^{-t^2/2}\left(1+\fr{(it)^3}{3\sqrt{m}}+
\fr{(it)^6}{18m}+\fr{(it)^4}{4m} \right) \right| \, dt\,;\\
 I_2&=\int\limits_{|t|\geq \lambda \sqrt{m}} \fr{|f_n(t)|}{|t|} \, dt \,;\\
  I_{31} &= \int\limits_{|t|\geq\lambda \sqrt{m}} \fr{e^{-t^2/2}}{|t|}
  \left|\left(1+\fr{(it)^3}{3\sqrt{m}}\right)\right| \,dt\,;\\
 I_{32}&= \int\limits_{|t|\geq\lambda \sqrt{m}} \fr{e^{-t^2/2}}{|t|}
  \left|\left(\fr{(it)^6}{18m}+\fr{(it)^4}{4m} \right) \right|\, dt\,.
  \end{align*}

\medskip


Сначала оценим   $|f_n(t)\,-\,g^*_n(t)|$ для $|t| \le \lambda\,\sqrt{m}$, когда    $\lambda < 3/4$ и $m=n/2$.
Используя ту ветвь комплексной функции    $\tau(z)=\ln(1 - z)$, для которой
  $\tau(0) =0$, запишем
\begin{multline}
f_n(t)  =  \exp\left\{-it\sqrt{m}-m\ln\left(1-\fr{it}{\sqrt{m}}\right)\right\}={}\\
{}= \exp\left\{\!
-\fr{t^2}{2} +\fr{(it)^3}{3\sqrt{m}}+\fr{(it)^4}{4m}+ %{}\right.\\
%\left.{} +
\fr{(it)^5}{5m^{3/2}}+mR_m(t)\!\right\}={}\hspace{-1.603pt}\\
{} = \exp\left\{-\fr{t^2}{2} +\fr{(it)^3}{3\sqrt{m}}+\fr{(it)^4}{4m}+
\fr{(it)^5}{5m^{3/2}}\right\}+S_{1,m}(t)={}\\
{} =  e^{-t^2/2}\left(1+\fr{(it)^3}{3\sqrt{m}}+\fr{(it)^6}{18m}+\fr{(it)^4}{4m} \right)+{}\\
{} +S_{2,m}(t)+S_{1,m}(t)\,,
\label{g7}
\end{multline}
где
\begin{multline*}
 R_m(t) =  -\ln\left(1-\fr{ it}{\sqrt{m}}\right) -\fr{ it}{ \sqrt{m}}-{}\\
 {}-\fr{(it)^2}{ 2m} -
\fr{ (it)^3}{ 3m^{3/2}}-\fr{ (it)^4}{ 4m^{2}}-\fr{ (it)^5}{ 5m^{5/2}}\,;
\end{multline*}

\vspace*{-12pt}

\begin{multline*}
S_{1,m}(t) = \exp\left\{-\fr{t^2}{2} +\fr{(it)^3}{3\sqrt{m}}+
\fr{(it)^4}{4m}+{}\right.\\
\left.{}+\fr{(i\,t)^5}{5m^{3/2}}\right\}\left(\exp\{mR_m(t)\}-1\right)\,;
\end{multline*} 

\vspace*{-12pt}

\noindent
\begin{multline*}
 S_{2,m}(t)=e^{-t^2/2}\left(\!\exp\!\left\{\fr{(it)^3}{3\sqrt{m}}+\fr{(it)^4}{4m} +
 \fr{ (it)^5}{ 5m^{3/2}}\right\}-{}\right.\hspace*{-0.5347pt}\\
\left.{}-1-\fr{(it)^3}{3\sqrt{m}}-\fr{(it)^6}{18m}-\fr{(it)^4}{4m}\right)\,.
\end{multline*}

Используя 
$$
\left| -\ln(1-z) -z -\fr{z^2}{2} - \fr{z^3}{3} - \fr{z^4}{4} - \fr{z^5}{5} \right| \leq \fr{|z|^6}{6(1-|z|)}
$$ 
при 
$|z| < 1$, находим, что для  $|t| < \lambda \sqrt{m}$ справедливы неравенства:

\noindent
$$
m|R_m(t)|\leq \fr{  |t|^6}{  6m^2(1-|t|/\sqrt{m})}
\leq \fr{t^2\lambda^4}{6(1- \lambda)}\,,
$$
и, обозначая 

\noindent
$$
a=a(\lambda)=1-\fr{\lambda^2}{2} - \fr{\lambda^4}{3(1-\lambda)}\,,
$$ 
получаем:

\noindent
\begin{multline}
\left|S_{1,m}(t)\right| \leq  e^{-t^2/2+t^4/(4m)} \left|e^{m R_m(t)}-1\right| \leq {}\\
{} \leq  e^{-t^2/2+t^2\lambda^2/4} m|R_m(t)| e^{m |R_m(t)|} 
\leq{}\\
{}\leq  \fr{|t|^6 e^{-at^2/2}}{6m^2(1-\lambda)} \,.
\label{g8}
\end{multline}
%
 Для оценки $S_{2,m}(t)$ положим:
 $$
 u= \fr{(it)^3}{3\sqrt{m}}\,;\enskip
w= \fr{(it)^4}{4\,m}\,; \enskip z=\fr{(it)^5}{5m^{3/2}}\,.
$$  
Тогда имеем:

\noindent
\begin{gather*}
|e^u|= 1\,;\enskip \left|e^u-1-u-\fr{u^2}{2}\right|\leq \fr{u^3}{6}\,;\enskip\\[-1pt]
|e^w-1-w|\leq \fr{w^2e^{|w|}}{2}\,;\enskip |e^z|=1\,;\enskip |e^z - 1|\leq |z|\,;
\end{gather*}

\vspace*{-12pt}

\noindent
\begin{multline}
\big|S_{2,m}(t)\big|  ={}\\[-2pt]
{}=  e^{-t^2/2}\left(\left|e^u e^w e^z- 1 - u - \fr{u^2}{2} - w \right|\right)\leq{}\\[-1pt]
{} \leq e^{-t^2/2}\left(
\left|e^u (1+w) e^z- 1 - u - \fr{u^2}{2} - w \right| +{} \right.\\[-1pt]
\left.{} + \left|e^u(e^w - 1 -w)e^z\right| \vphantom{\fr{u^2}{2}}\right)\leq{}\\[-1pt]
{} \leq  e^{-t^2/2}\left(
\vphantom{\fr{|w|^2 e^{|w|}}{2}}\left|e^u(1+w) - 1 - u - \fr{u^2}{2} - w \right| + {}\right.\\[-1pt]
\left.{} + \left|e^u (1 + w)(e^z -1)\right|  +\fr{|w|^2e^{|w|}}{2}
\right)\leq{}\\[-1pt]
{} \leq  e^{-t^2/2}\left( 
\vphantom{\fr{|w|^2 e^{|w|}}{2}}
\left| e^u -1 - u - \fr{u^2}{2}\right| + |w|\cdot |(e^u - 1)| +{} \right.\\[-1pt]
\left.{} + (1+|w|)|z| + \fr{|w|^2 e^{|w|}}{2}\right)\leq{}\\[-1pt]
{} \leq  e^{-t^2/2}\left( \vphantom{\fr{|u|^3 e^{|w|}}{6}}
\fr{|u|^3}{6}+|w|\cdot |u|+ |z| +{}\right.\\[-1pt]
\left.{}+ |w||z| +\fr{|w|^2 e^{|w|}}{2}\right)\leq{}\\[-1pt]
{} \leq  e^{-t^2/2}\left(\fr{|t|^9}{162m^{3/2}}+\fr{|t|^7}{12m^{3/2}}+\fr{ |t|^5}{5m^{3/2}}+{}\right.\\[-1pt]
\left.{} + \fr{ |t|^9}{20m^{5/2}}+\fr{|t|^8}{32m^2}\, e^{\lambda^2 t^2 /4}\right)\,.
\label{g9}
\end{multline}
Перейдем теперь к оценке интегралов. Из~(\ref{g7})--(\ref{g9}) следует, 
что $I_1 \leq I_{11} +I_{12}$, где

\noindent
\begin{multline*}
I_{11}  =  \int\limits_{|t|<\lambda\sqrt{m}} \fr{|S_{1,m}(t)|}{|t|} \,dt \leq{}\\
{}\leq \int\limits_{|t| \leq \lambda\sqrt{m}} \fr{|t|^5}{6m^2(1-\lambda)} \,e^{-at^2/2}\,dt \leq{}\\
{}\leq  \fr{4}{3m^2(1-\lambda)a^3}\int\limits_0^\infty u^2e^{-u}\,du
= \fr{8}{3m^2(1-\lambda)a^3} 
%\label{g102}
\end{multline*}
с  $a=1 -  \lambda^2/2 -   \lambda^4/[3(1-\lambda)]$,
и, поскольку $E(Y^8)=$\linebreak $=105$, $E(Y^6)=15$ и $E(Y^4)=3$ для стандартной нормальной случайной величины~$Y$,
\begin{multline*} %\label{g103}
I_{12}  =  \int\limits_{|t|<\lambda\sqrt{m}} \fr{|S_{2,m}(t)|}{|t|} \,\,dt\leq{}\\[3pt]
{} \leq \int\limits_{|t| < \lambda\sqrt{m}} e^{-t^2/2}\left(
\fr{|t|^8}{162m^{3/2}}+\fr{|t|^6}{12m^{3/2}}+{}\right. \\[3pt]
\left.{}+\fr{ |t|^4}{5m^{3/2}}+\fr{ |t|^8}{20m^{5/2}}+\fr{|t|^7}{32m^2}\, e^{\lambda^2 t^2 /4}\,\right)\,dt \leq{}\\[3pt]
{}\leq \fr{105 \sqrt{2\pi}}{162m^{3/2}}+\fr{15\sqrt{2\pi}}{12m^{3/2}} +
\fr{3\sqrt{2\pi}}{5m^{3/2}} +\fr{105\sqrt{2\pi}}{20m^{5/2} }+
\fr{3}{b^4 m^2}
\end{multline*}
с $b=b(\lambda)=1- \lambda^2/2$.


В приведенных выше оценках для интегралов   $I_{11}$ и~$I_{12}$ верхний предел $\lambda\sqrt{m}$ 
был заменен бесконечностью. Если этого не делать, то можно получить более точные, но вместе с тем 
более громоздкие оценки:
из формулы~7.1.13 в~\cite{8-cr}   вытекает, что
\begin{equation}
\fr{2e^{-r^2/2}}{r + \sqrt{r^2 + 4}} \leq \int\limits_r^{\infty} e^{-t^2/2}\,dt \leq 
\fr{2 e^{-r^2/2}}{r + \sqrt{r^2 + 8/\pi}}\,.
\label{g102a}
\end{equation}
Используя интегрирование по частям для   $k=4, 3, 2$  и нижнюю оценку из~(\ref{g102a}) для $k=0$, находим
\begin{multline*}
  \int\limits_0^{\lambda\sqrt{m}}t^{2k}\,e^{-t^2/2}\,dt =\int\limits_0^{\infty}t^{2k}e^{-t^2/2}\,dt - {}\\
  {}-
  \int\limits_{\lambda\sqrt{m}}^{\infty}t^{2k}e^{-t^2/2}\,dt \leq{}\\
{}\leq  \sqrt{\fr{\pi}{2}}(2k-1)!!-U_{2k}\left(\lambda\,\sqrt{\fr{n}{2}}\,\right).
\end{multline*}
Аналогично с помощью интегрирования по частям для   $k=3$  и $k=2$ находим, что
\begin{multline*}
\int\limits_0^{\lambda\sqrt{m}} t^{2k+1}e^{-ct^2/2}\,dt =
\int\limits_0^{c\lambda^2 m/2}u^k e^{-t^2/2}\,dt\leq{}\\
{}\leq\fr{2^k}{c^{k+1}}\left(k! - U_{2k+1}\fr{c\lambda^2m}{2}\right)\,,
\end{multline*}
где функции   $U_4(\ldots)- U_8(\ldots)$ 
определены перед формулировкой теоремы~3. Отрицательные члены оказывают заметное влияние на величины 
оценок, когда   $m = n/2 $ мало. При больших значениях~$n$ этим влиянием можно пренебречь.  

Для $I_2$ воспользуемся оценкой, полученной в~\cite{1-cr}    с 
$|f_n(t)| = (1+ t^2/m)^{-m/2}$:
\begin{multline*}
I_2 \leq \int\limits_{|t|\geq \lambda\sqrt{m}} \fr{1}{|t|} \,\fr{1}{\left(1+t^2/m\right)^{m/2}}\,dt
={}\\
{}=  \int\limits_{u\geq \lambda^2} \fr{du}{u\left(1+u \right)^{m/2}}\leq
 \fr{1+\lambda^2}{\lambda^2}\int\limits_{u\geq \lambda^2} \fr{du}{(1+u )^{1+m/2}}={}\\
 {}= \fr{1+\lambda^2}{\lambda^2m/2}\left(1+\lambda^2 \right)^{-m/2}\,.
\end{multline*}
Для оценки   $I_{31}$ воспользуемся формулой~5.1.20 из~[10]
\begin{multline*} 
\int\limits_z^{\infty} \fr{e^{-t^2/2}}{t}\,dt= \fr{1}{2}\int\limits_{z^2/2}^{\infty}\fr{e^{-u}}{u}\, du \leq{}\\
{}\leq
\fr{1}{2}\, e^{-z^2/2}\,\ln\left(1+\fr{2}{z^2}\right) \,,
\end{multline*}
интегрированием по частям и верхней оценкой в~(\ref{g102a}). Тогда получаем
\begin{multline*} 
I_{31} \leq 2\int\limits_{\lambda\sqrt{m}}^{\infty} e^{-t^2/2}
\left(\fr{1}{t} + \fr{t^2}{3\sqrt{m}} \right)\, dt \leq{}\\
{} \leq \left( 
\vphantom{\fr{4}{\sqrt{\lambda^2}}}
\ln\left(1+\fr{2}{\lambda^2m}\right)+ {}\right.\\
\left.{}+
\fr{2\lambda}{3}+\fr{4}{3\lambda m(1+\sqrt{1+8/(\pi \lambda^2 m)})}\right) e^{-\lambda^2m/2}\,;
\end{multline*}

\noindent
\begin{multline*}
I_{32} \leq  2\int\limits_{\lambda\sqrt{m}}^{\infty} e^{-t^2/2}
\left(\fr{t^5}{18m} + \fr{t^3}{4m}\right)\, dt
 \leq{} \\
{} \leq \left( \fr{\lambda^4m^2 + 4\lambda^2 m +8}{9m} + 
\fr{\lambda^2m + 2}{2m}\right)e^{-\lambda^2m/2} \,.
\end{multline*}
Объединение оценок для   $I_1, I_2$, $I_{31}$ и~$I_{32}$ завершает доказательство теорем~2 и~3.

{\small\frenchspacing
{%\baselineskip=10.8pt
%\addcontentsline{toc}{section}{Литература}
\begin{thebibliography}{99}

\bibitem{1-cr}
\Au{Ульянов  В.\,В., Кристоф~Г., Фуджикоши~Я.}  О~приближениях
преобразований хи-квадрат распределений в статистических приложениях~//
Сибирский математический журнал, 2006. Т.~47. №\, 6. С.~1401--1413.

\bibitem{9-cr} %2
\Au{Sezgin A., Oechtering T.\,J.} 
Complete characterization of the
equivalent MIMO channel for quasi-orthogonal space-time codes~//
IEEE Transactions on Information Theory, 2008. Vol.~54. No.\,7. P.~3315--3327.


\bibitem{2-cr} %3
\Au{Fujikoshi Y.,   Ulyanov~V.\,V.,   Shimizu~R.} Multivariate
statistics: High-dimensional and large-sample approximations.~---
Hoboken, N.J.: John Wiley and Sons, 2010.

\bibitem{10-cr} %4
\Au{Hawwar Y., Reza~A.} Spatially adaptive multiplicative noise
image denoising technique~// IEEE Transactions on Image Processing,
2002. Vol.~11. No.\,12. P.~1397--1404.

\bibitem{3-cr} %5
\Au{Королев В.\,Ю., Шевцова И.\,Г.} Уточнение неравенства
Берри--Эссеена с приложениями к пуассоновским и смешанным
пуассоновским случайным суммам~// Обозрение прикладной и
промышленной математики, 2010. Т.~17. Вып.~1. С.~25--56.

\bibitem{4-cr} %6
\emph{Тюрин И.\,С.}  Уточнение верхних оценок констант в теореме Ляпунова~// УМН, 2010. 
Т.~65. Вып.~3(393). С.~201--202.

\bibitem{5-cr} %7
\Au{Кавагучи Ю., Ульянов В.\,В., Фуджикоши~Я.} Приближения для
статистик, описывающих геометрические свойства данных большой
размерности, с оценками ошибок~// Информатика и её применения, 2010.
Т.~4. Вып.~1. С.~22--27.

\bibitem{6-cr} %8
\emph{Dobric V., Ghosh B.\,K.} 
Some analogs of the Berry--Esseen bounds for first-order
Chebyshev--Edgeworth expansions~// Statist. Decisions, 1996. Vol.~14. No.\,4. P.~383--404.

\bibitem{7-cr} %9
\emph{Christoph G., Ulyanov~V.}
Bounds for $L_1$-approximation of chi-squared-density by a first order Chebyshev--Edgeworth-expansion~//
Int.\ J.~Communications in Dependability and Quality Management, 2006. Vol.~9. No.\,1. P.~12--16.

 \label{end\stat}

\bibitem{8-cr} %10
Справочник по специальным функциям~/ Под ред. М.~Абрамовица, И.~Стиган.~--- М.: Наука, 1979.


 \end{thebibliography}
}
}


\end{multicols}      %3
\def\stat{kor-kor}



\def\tit{МОДИФИЦИРОВАННЫЙ СЕТОЧНЫЙ МЕТОД РАЗДЕЛЕНИЯ ДИСПЕРСИОННО-СДВИГОВЫХ
СМЕСЕЙ НОРМАЛЬНЫХ ЗАКОНОВ$^*$}



\def\titkol{Модифицированный сеточный метод разделения дисперсионно-сдвиговых
смесей нормальных законов}

\def\aut{В.\,Ю.~Королев$^1$,  А.\,Ю.~Корчагин$^2$}

\def\autkol{В.\,Ю.~Королев,  А.\,Ю.~Корчагин}

\titel{\tit}{\aut}{\autkol}{\titkol}

{\renewcommand{\thefootnote}{\fnsymbol{footnote}} \footnotetext[1]
{Работа поддержана Российским научным фондом (проект 14-11-00364).}}


\renewcommand{\thefootnote}{\arabic{footnote}}
\footnotetext[1]{Факультет
вычислительной математики и кибернетики Московского государственного
университета им.\ М.\,В.~Ломоносова; Институт проблем информатики
Российской академии наук; victoryukorolev@yandex.ru}
\footnotetext[2]{Факультет вычислительной математики и кибернетики
Московского государственного университета им.\ М.\,В.~Ломоносова;
sasha.korchagin@gmail.com}

%\vspace*{2pt}



\Abst{Описывается модифицированный двухэтапный
сеточный метод разделения дис\-пер\-си\-он\-но-сдви\-го\-вых смесей нормальных
законов, представляющий собой альтернативу чистому ЕМ (expectation-maximization)
ал\-го\-рит\-му. На
первом этапе этого алгоритма строится дискретная аппроксимация для
смешивающего распределения, на втором этапе подбирается абсолютно
непрерывное распределение из заранее заданного семейства, например,
обобщенных обратных гауссовских законов, ближайшее к~дискретному
распределению, полученному на первом этапе. Обсуждаются вопросы
сходимости этого двухэтапного алгоритма. Доказана монотонность
сеточного итерационного метода, используемого на первом этапе.
Подробно обсуждается вопрос оптимального выбора параметров метода,
прежде всего сетки, накидываемой на носитель смешивающего
распределения. С~этой целью предложены статистические оценки
квантилей смешивающего распределения. Эффективность метода
иллюстрируется примерами конкретных вычислений оценок параметров
обобщенных гиперболических распределений.}

\KW{смесь распределений вероятностей;
дис\-пер\-си\-он\-но-сдви\-го\-вая смесь нормальных законов; обобщенное
гиперболическое распределение; ЕМ-ал\-го\-ритм; сеточный метод
разделения смесей}

\vspace*{1pt}

%\vspace*{2pt}

\DOI{10.14357/19922264140402}


\vskip 12pt plus 9pt minus 6pt

\thispagestyle{headings}

\begin{multicols}{2}

\label{st\stat}

\section{Введение}

При {\it практическом} решении задачи моделирования и исследования
волатильности (изменчивости) хаотических стохастических процессов
ключевым этапом является статистическое разделение смесей
вероятностных распределений. Задача разделения смесей~---
статистического оценивания параметров смесей вероятностных
распределений~--- в~деталях разобрана, например, в~книге~\cite{k2011}.

Для решения задачи разделения смесей вероятностных распределений
традиционно используются итерационные процедуры типа ЕМ-ал\-го\-рит\-ма.
К~сожалению, классический ЕМ-ал\-го\-ритм обладает рядом серьезных
недостатков при его применении к~смесям нормальных законов, а~именно:
он демонстрирует крайнюю неустойчивость по отношению к~исходным
данным и~начальным приближениям.

Для преодоления этих недостатков
предложено много модификаций ЕМ-ал\-го\-рит\-ма (см., например,~\cite{k2011}).
Вместе с тем в~указанной книге предложен и~исследован
принципиально новый~--- сеточный~--- метод приближенного решения
задачи разделения смесей. В~работе~\cite{n2013} подробно исследованы
вопросы сходимости сеточных методов разделения смесей.

В соответствии с подходом к~статистическому анализу хаотических
стохастических процессов, в~частности к~решению задачи декомпозиции
волатильности таких процессов, развитом в~книге~\cite{k2011},
в~общем случае на практике приходится решать задачу разделения
конечных смесей нормальных законов с~произвольно большим числом
неизвестных параметров (параметров компонент и~их весов).
И~хотя в~большинстве приложений возникают смеси не более чем с~пятью--семью
компонентами, даже при использовании таких смесей, скажем, в~задачах
анализа и~прогнозирования финансовых рисков приходится моделировать
траекторию движения точки в~пространствах, размерность которых
соответственно лежит в~пределах от~14 (для пятикомпонентных смесей)
до~20 (для семикомпонентных смесей), что существенно увеличивает
вычислительные и~временн$\acute{\mbox{ы}}$е ресурсы, необходимые для практического
решения указанных задач.

Поскольку во многих ситуациях (например,
при прогнозировании на основе высокочастотных данных) эти задачи
необходимо решать в~режиме, близком к~реальному времени, для
создания эффективных методов статистического анализа на основе
смешанных моделей на первый план выходит проб\-ле\-ма снижения
размерности решаемой задачи, т.\,е.\ параметрического пространства.

Одним из возможных подходов к~снижению размерности является
априорное сужение классов допусти\-мых смесей. К~примеру, при решении
многих задач, связанных с~анализом процессов атмосферной или
плазменной турбулентности, а~так\-же процессов, описывающих эволюцию
различных финансовых индексов, высочайшую адекватность
продемонстрировали модели, основанные на дис\-пер\-си\-он\-но-сдви\-го\-вых
смесях нормальных законов. Класс таких смесей очень обширен
и,~в~част\-ности, включает в~себя обобщенные гиперболические распределения,
которые были введены О.-Е.~Барн\-дорфф-Ниль\-се\-ном в~1977--1978~гг.\ как
класс специальных сдвиг-мас\-штаб\-ных смесей нормальных законов~\cite{BN1977, BN1978}.
Пусть $\alpha\hm\in\r$, $\beta\hm\in\r$. Если
функцию распределения обобщенного гиперболического закона
с~параметрами~$\alpha$, $\beta$, $\nu$, $\mu$, $\lambda$ обозначить
$P_{GH}(x;\alpha,\beta,\nu,\mu,\lambda)$, то по определению
\begin{multline}
P_{GH}(x;\alpha,\beta,\nu,\mu,\lambda)={}\\
{}=
\int\limits_{0}^{\infty}\Phi\left(\fr{x-\beta-\alpha
z}{\sqrt{z}}\right)\,p_{GIG}(z;\nu,\mu,\lambda)\,dz\,,\\
x\in\r\,,
\label{e1-kor}
\end{multline}
где $\Phi(x)$~--- стандартная нормальная функция распределения:
$$
\Phi(x)=\int\limits_{-\infty}^{x}\varphi(z)\,dz\,,\enskip
\varphi(x)=\fr{1}{\sqrt{2\pi}}e^{-x^2/2}\,,\enskip  x\in\mathbb{R}\,;
$$
$p_{GIG}(x;\nu,\mu,\lambda)$~--- плот\-ность обобщенного обратного
гауссовского распределения:
\begin{multline*}
p_{GIG}(x;\nu,\mu,\lambda)={}\\
{}=\fr{\lambda^{\nu/2}}{2\mu^{\nu/2}
K_{\nu}\left(\sqrt{\mu\lambda}\right)}\,
x^{\nu-1}\exp\left\{-\fr{1}{2}\left(\fr{\mu}{x}+\lambda
x\right)\right\}\,,\\ x>0\,.
\end{multline*}
Здесь $\nu\in\r$;
$$
\begin{array}{lll}
\mu>0\,, & \lambda\geqslant0\,, & \mbox{если }\nu<0\,;\\[6pt]
\mu>0\,, & \lambda>0\,, & \mbox{если }\nu=0\,;\\[6pt]
\mu\geqslant0\,, & \lambda>0\,, & \mbox{если }\nu>0\,;
\end{array}
$$
$K_{\nu}(z)$~--- модифицированная бесселева функция третьего рода
порядка~$\nu$:

\noindent
\begin{multline*}
K_{\nu}(z)=\fr{1}{2}\int\limits_{0}^{\infty}y^{\nu-1}\exp
\left\{-\fr{z}{2}\left(y+\fr{1}{y}\right)\right\}\,dy\,,\\
z\in\mathbb{C}\,,\enskip \mathrm{Re}\,z>0\,.
\end{multline*}
Обратим внимание, что в~(1) смешивание происходит одновременно и~по
параметру сдвига, и~по параметру масштаба, но так как эти параметры
в~(1)  связаны жесткой зависимостью, так что параметр сдвига
смешиваемого распределения пропорционален его дисперсии, то
фактически смесь~(1) является {\it однопараметрической} и~поэтому
называется {\it дис\-пер\-си\-он\-но-сдви\-го\-вой} (см., например,~\cite{BN1982}).

Другим примером дис\-пер\-си\-он\-но-сдви\-го\-вых смесей нормальных законов
являются обобщенные дисперсионные гам\-ма-рас\-пре\-де\-ле\-ния, в~которых
смешивающими являются обобщенные гам\-ма-рас\-пре\-де\-ле\-ния~\cite{ks2012, zk2013}.

В указанных семействах смесей число неизвестных параметров равно
пяти или шести (если\linebreak учитывать неслучайный сдвиг). Вместе
с~тем у~подоб\-ных моделей имеются довольно серьезные тео\-ре\-ти\-че\-ские
обоснования: в~работах~\cite{zk2013, k2013} показано, что указанные
модели являются асимптотическими аппроксимациями в~простой
предельной схеме случайного суммирования и~потому могут успешно
применяться для анализа процессов типа остановленных случайных
блужданий. Эти выводы подтверждены статистическим анализом
вы\-со\-ко\-час\-тот\-ных финансовых данных, в~результате которого выявлен
синхронизированный характер изменения интенсивностей потоков заявок
в~сис\-те\-мах электронных торгов, что естественно приводит к~синхронизированному
поведению па\-ра\-мет\-ров сдвига и~диффузии в~соответствующих моделях вида смесей
нормальных законов~\cite{kckg2013}.

\section{Описание моди\-фи\-ци\-ро\-ван\-но\-го
сеточного ме\-то\-да разделения дисперсионно-сдвиговых смесей
нормальных законов и~его свойства}

Оказывается, что сеточные методы разделения смесей довольно
эффективны не только при разделении конечных смесей нормальных
законов, но и~при разделении произвольных дис\-пер\-си\-он\-но-сдви\-го\-вых
смесей нормальных законов. Поясним сказанное на примере задачи
оценивания па\-ра\-мет\-ров обобщенных гиперболических распределений.

Для решения задачи оценивания параметров обобщенных гиперболических
распределений традиционно используется метод, предложенный в~статье~\cite{p2004}
и~по сути являющийся классическим ЕМ-ал\-го\-рит\-мом,
приспособленным к~конкретной задаче, и,~соответственно, наследующий
присущие ЕМ-ал\-го\-рит\-мам недостатки.

Рассмотрим следующий альтернативный двухэтапный метод. На первом
этапе на поло\-жи\-тельной полупрямой выделим основную часть носителя
смешивающего распределения, т.\,е.\ \mbox{ограниченный} интервал,
вероятность которого, вычисленная в~соответствии со смешивающим
распределением, практически равна единице. На этот интервал накинем
конечную сетку, содержащую, возможно, очень много {\it известных}
узлов $u_1,\ldots,u_K$. Считая параметр сдвига~$\beta$ равным нулю,
приблизим искомое обобщенное гиперболическое распределение конечной
смесью нормальных законов:

\noindent
\begin{multline}
P_{GH}(x;\,\alpha,0,\nu,\mu,\lambda)\approx{}\\
{}\approx \sum\limits_{i=1}^K
p_i\Phi\left(\fr{x-\alpha u_i}{\sqrt{u_i}}\right)\,,\enskip
x\in\mathbb{R}\,.\label{e2-kor}
\end{multline}
В смеси, стоящей в~правой части соотношения~(2), неизвестными
являются только параметры $p_1,\ldots,p_{K-1}$ и~$\alpha$. Пусть
$x_1,\ldots,x_n$~--- анализируемая выборка значений случайной
величины с~оцениваемым обобщенным гиперболическим распределением.
Итерационный процесс, определяющий сеточный ЕМ-ал\-го\-ритм для данной
задачи, задается следующим образом. Пусть
$p_1^{(m)},\ldots,p_{K-1}^{(m)}$ и~$\alpha^{(m)}$~--- оценки параметров
$p_1,\ldots,p_{K-1}$ и~$\alpha$ на $m$-й итерации,
$p_K^{(m)}\hm=1\hm-p_1^{(m)}-\cdots-p_{K-1}^{(m)}$. Обозначим

\noindent
\begin{align*}
\varphi_{ij}^{(m)}&=\fr{1}{\sqrt{u_i}}\varphi\left(\fr{x_j-\alpha^{(m)}u_i}{\sqrt{u_i}}\right)\,;
\\
g_{ij}^{(m)}&=\fr{p_i^{(m)}\varphi_{ij}^{(m)}}{\sum\limits_{r=1}^K
p_r^{(m)}\varphi_{rj}^{(m)}}\,,\\
&\hspace*{14mm}i=1,\ldots,K\,;\enskip j=1,\ldots,n\,.
\end{align*}
Тогда, используя стандартные рассуждения, определяющие
вычислительные формулы EM-ал\-го\-рит\-ма для параметров конечной смеси
нормальных законов (см, например,~[1, разд.~5.3.7--5.3.8]),
следует положить

\noindent
\begin{equation}
p_i^{(m+1)}=\fr{1}{n}\sum\limits_{j=1}^n g_{ij}^{(m)}\,, \enskip
i=1,\ldots,K\,.\label{e3-kor}
\end{equation}
Обозначим $\overline{x}=(1/n)\sum\limits_{j=1}^nx_j$. Используя
соотношение~(5.3.24) в~\cite{k2011}, с~учетом очевидного равенства
$\sum\limits_{i=1}^K g_{ij}^{(m)}\hm=1$ можно заметить, что уточненная
оценка параметра~$\alpha$ имеет вид:

\columnbreak

\noindent
\begin{equation}
\alpha^{(m+1)}=\fr{\overline{x}}{\sum\limits_{i=1}^K u_ip_i^{(m+1)}}\,,
\label{e4-kor}
\end{equation}
т.\,е.\ равна отношению генерального выборочного среднего и~текущего
эмпирического среднего смешивающего распределения, что вполне
согласуется с~тем, что в~соответствии с~приводимым ниже соотношением~(\ref{e5-kor})
в~данном случае ${\sf E}X\hm=\alpha{\sf E}U$.

В силу монотонности классического ЕМ-ал\-го\-рит\-ма справедливо следующее
утверждение.

\smallskip

\noindent
\textbf{Теорема~1.} {\it Пусть узлы $u_1,\ldots,u_K$ сетки различны,
неотрицательны и~известны. Тогда итерационный процесс $(3)$--$(4)$
является монотонным, т.\,е.\ каждая его итерация не уменьшает
целевую сеточную функцию правдоподобия}
\begin{multline*}
L(p_1,\ldots,p_K,\alpha;x_1,\ldots,x_n)={}\\
{}=
\prod\nolimits_{j=1}^n\left[\sum\nolimits_{i=1}^K
\fr{p_i}{\sqrt{u_i}}\,\varphi\left(\fr{x_j-\alpha^{(m)}u_i}{\sqrt{u_i}}\right)\right].
\end{multline*}

\smallskip

\noindent
\textbf{Замечание~1.} В~разд.~5.7.4 книги~\cite{k2011} показано, что
при каждом фиксированном значении параметра~$\alpha$ сеточная
функция правдоподобия\linebreak
$L(p_1,\ldots,p_{K-1},\alpha;\,x_1,\ldots,x_n)$ вогнута по
аргументам $p_1,\ldots,p_{K-1}$. Поэтому на каждом шаге
итерационного процесса вместо соотношения~(3) можно\linebreak использо\-вать
любой более быстрый алгоритм максимизации функции
$L(p_1,\ldots,p_{K-1},\alpha^{(m)};\,x_1,\ldots$\linebreak $\ldots,x_n)$ по переменным
$p_1,\ldots,p_{K-1}$. Например, оценки весов $p_1,\ldots,p_K$ можно
искать методом условного градиента~\cite{k2011, kn2010}.

\smallskip

Таким образом, на первом этапе получаются оценки параметра~$\alpha$
и~весов всех узлов~$u_i$ конечной сетки, накинутой на носитель
смешивающего обобщенного обратного гауссовского распределения
$P_{\mathrm{GIG}}(z;\,\nu,\mu,\lambda)$.

На втором этапе остается применить ка\-кой-ли\-бо стандартный метод
подгонки обобщенного обратного гауссовского распределения
$P_{\mathrm{GIG}}(z;\,\nu,\mu,\lambda)$ к~эмпирическим данным типа
гистограммы $(u_1, p_1),\ldots, (u_K, p_K)$. Например, параметры~$\nu$,
$\mu$ и~$\lambda$ можно оценить, минимизируя соответствующую
статистику хи-квад\-рат. Или же, например, можно решить задачу
наименьших квад\-ратов:
\begin{multline*}
(\nu^*,\mu^*,\lambda^*)={}\\
{}=\arg\min\limits_{\nu,\mu,\lambda}\sum\limits_{i=1}^K
\left[p_i- \!\!\!\!\!
\int\limits_{(1/2)\left(u_{i-1}+u_i\right)}^{(1/2)(u_i+u_{i+1})}\!\!\!\!\!\!\!\!\!\!\!\!\!\!\!
p_{GIG}(u;\,\nu,\mu,\lambda)\,du\right]^2,
\end{multline*}
где $u_0=0$; $u_{K+1}\hm=\infty$.

На практике хорошие результаты показал подход с решением задачи
наименьших квадратов. Для поиска параметров использовался алгоритм
ns2sol, описанный в~книге~\cite{DSch1983}. Указанный алгоритм
доступен во многих статистических пакетах, отличается высоким
быстродействием и~возможностью при желании задавать разумные
интервалы для поиска параметров.

%\vspace*{-9pt}

\section{О практическом выборе сетки
на~первом этапе моди\-фи\-ци\-ро\-ван\-но\-го
сеточного метода разделения дисперсионно-сдвиговых смесей нормальных
законов}

Естественно, что при использовании указанного двухэтапного метода
в~динамическом режиме крайне важным становится вопрос о~выборе
наиболее эффективных и~быстродействующих численных процедур и~их
параметров. В~частности, исключительную важность приобретает
правильный выбор сетки на первом этапе. Рассмотрим этот вопрос
подробнее.

Формально рассматриваемая задача выглядит так: по наблюдаемым
значениям $x_1,\ldots,x_n$ требуется построить статистическую оценку
верхней границы квантилей заданного порядка сме\-ши\-ва\-юще\-го закона так,
чтобы как можно точнее оценить носитель смешивающего распределения.

В дальнейшем будем считать, что $x_1,\ldots,x_n$~--- независимые
реализации случайной величины $X\hm=Y\sqrt{U}+\alpha U$, где $Y$~---
случайная величина со стандартным нормальным распределением, а~$U$~---
независимая от нее случайная величина с~обобщенным обратным
гауссовским распределением. Тогда, очевидно, распределение случайной
величины~$X$ имеет вид~(1). Предположим, что у~случайной величины~$U$
существуют моменты первых двух порядков. Тогда, как несложно видеть,
\begin{equation}
{\sf E}X={\sf E}Y\cdot{\sf E}\sqrt{U}+\alpha{\sf E}U=\alpha{\sf
E}U\,.\label{e5-kor}
\end{equation}
При этом по усиленному закону больших чисел с~вероятностью единица
$\overline x\hm\longrightarrow {\sf E}X$ $(n\hm\to\infty)$, так что при
больших~$n$ справедливо приближенное равенство ${\sf E}X\hm\approx\overline x$
и~с учетом~(\ref{e5-kor})
\begin{equation}
{\sf E}U\approx\fr{\overline x}{\alpha}\,.\label{e6-kor}
\end{equation}
Далее, очевидно,

\columnbreak

\noindent
\begin{multline}
{\sf E}X^2={\sf E}Y^2\cdot{\sf E}U+2\alpha{\sf E}X\cdot{\sf E}U^{3/2}+{}\\
{}+
\alpha^2{\sf E}U^2={\sf E}U+\alpha^2{\sf E}U^2\,.
\label{e7-kor}
\end{multline}

\noindent
Поэтому, обозначив
$$
m^2=\fr{1}{n}\sum\limits_{i=1}^nx_i^2\,,
$$
получаем приближенное равенство ${\sf E}X^2\hm\approx m^2$, так что
с~учетом~(\ref{e6-kor}) и~(\ref{e7-kor}) имеем:
\begin{equation}
{\sf E}U^2\approx\fr{1}{\alpha^2}\left(m^2-\fr{\overline
x}{\alpha}\right)\,.\label{e8-kor}
\end{equation}
Если параметр~$\alpha$ известен, то для определения верхней границы~$u^*$
сетки, накидываемой на носитель распределения случайной
величины~$U$, можно задать малое положительное число~$\varepsilon$
и~воспользоваться требованием
\begin{equation}
{\sf P}(U\geqslant u^*)\leqslant\varepsilon\,.\label{e9-kor}
\end{equation}
А~для гарантированного выполнения требования~(\ref{e9-kor}) можно использовать
неравенство Маркова:
$$
{\sf P}(U\geqslant u^*)\leqslant\fr{{\sf E}U^2}{(u^*)^2}\leqslant \varepsilon\,,
$$
откуда с учетом~(\ref{e8-kor})
$$
(u^*)^2\geqslant\fr{{\sf E}U^2}{\varepsilon}\approx
\fr{1}{\alpha^2\varepsilon}\left( m^2-\fr{\overline x}{\alpha}\right)
$$
или
\begin{equation}
u^*\approx\fr{1}{\alpha\sqrt{\varepsilon}}\sqrt{m^2-
\fr{\overline x}{\alpha}}\,.\label{e10-kor}
\end{equation}

\begin{figure*}[b] %fig1
\vspace*{1pt}
 \begin{center}
 \mbox{%
 \epsfxsize=161.718mm
 \epsfbox{kor-1.eps}
 }
 \end{center}
 \vspace*{-9pt}
\Caption{Примеры применения модифицированного двухэтапного сеточного
ЕМ-ал\-го\-рит\-ма для подгонки обобщенного гиперболического распределения
к искусственным данным, $\beta\hm=0$: (\textit{a})~$n\hm=1000$, $\alpha\hm=0{,}3$,
$\nu\hm=1{,}3$, $\mu\hm=1{,}6$, $\lambda\hm=0{,}2$;
(\textit{б})~$n\hm=1000$, $\alpha\hm=0{,}5$, $\nu\hm=1$, $\mu\hm=1$,
$\lambda\hm=3$;
(\textit{в})~$n\hm=1000$, $\alpha\hm=3$,
 $\nu\hm=1{,}3$, $\mu\hm=1{,}6$, $\lambda\hm=2$;
(\textit{г})~$n\hm=10\,000$,
$\alpha\hm=0{,}3$, $\nu\hm=1{,}3$, $\mu\hm=1{,}6$, $\lambda\hm=0{,}2$}
\end{figure*}


Если же параметр~$\alpha$, определяющий асим\-мет\-рию распределения
случайной величины~$X$, неизвестен, то можно воспользоваться
следующими рассуждениями. Обозначим
$$
q_n=\fr{1}{n}\sum\limits_{i=1}^n{\bf 1}(x_i<0)\,,
$$
где ${\bf 1}(A)$~--- индикаторная функция множества (события)~$A$.
При этом по усиленному закону больших чисел с~вероятностью единица
$q_n\hm\longrightarrow {\sf P}(X\hm<0)$ $(n\hm\to\infty)$, так что при
больших~$n$ справедливо приближенное равенство
\begin{equation}
q_n\approx{\sf P}(X<0)\,.\label{e11-kor}
\end{equation}
Но
\begin{multline}
{\sf P}(X<0)=\int\limits_{0}^{\infty}\Phi
\left(-\alpha\sqrt{u}\right) p_{\mathrm{GIG}}(u;\nu,\mu,\lambda)\,du={}\\
{}=
{\sf E}\Phi\left(-\alpha\sqrt{U}\right)\,.\label{e12-kor}
\end{multline}

\pagebreak

\noindent
Предположим сначала, что $q_n\hm<1/2$. Если~$n$ достаточно велико,
то можно с~большой степенью
 уверенности утверж\-дать, что тогда
$\overline x\hm>0$ и~$-\alpha\hm<0$, т.\,е.
 $\alpha\hm>0$ и,~стало быть, на
положительной полуоси значений аргумента~$u$ функция $\Phi(\alpha u)$
вогнута, т.\,е.\ выпукла вверх. Тогда из~(\ref{e11-kor}) и~(\ref{e12-kor}), дважды
применяя неравенство Иенсена, в~силу монотонности функции~$\Phi$
получаем:
\begin{multline}
1-q_n\approx 1-{\sf E}\Phi\left(-\alpha\sqrt{U}\right)=
          {\sf E}\Phi\left(\alpha\sqrt{U}\right)\leqslant{}\\
          {}\leqslant\Phi
          \left(\alpha{\sf E}\sqrt{U}\right)\leqslant
          \Phi\left(\alpha\sqrt{{\sf E}U}\right)\,.\label{e13-kor}
\end{multline}
Если теперь для $t\hm\in(0,1)$ символом~$v_t$ обозначить $t$-кван\-тиль
стандартного нормального закона, то из~(\ref{e13-kor}) и~(\ref{e6-kor}) вытекает
<<приближенное неравенство>>
$$
v_{1-q_n}\hm\leqslant \alpha\sqrt{{\sf E}U}\,,
$$
т.\,е.
$$
\alpha\geqslant\fr{v_{1-q_n}}{\sqrt{{\sf E}U}}\approx
\fr{v_{1-q_n}\sqrt{\alpha}}{\sqrt{\overline x}}\,,
$$
откуда получаем, что при достаточно больших~$n$
\begin{equation}
\alpha\geqslant\fr{v_{1-q_n}^2}{\overline x}\,.\label{e14-kor}
\end{equation}
Если теперь задать малое положительное число~$\varepsilon$, то
для определения верхней границы~$u^*$ сетки, накидываемой на
носитель распределения случайной величины~$U$, можно воспользоваться
требованием~(\ref{e9-kor}), для гарантированного выполнения которого
с~учетом~(\ref{e6-kor}) и~(\ref{e14-kor}) можно использовать неравенство Маркова:
$$
{\sf P}(U\geqslant u^*)\leqslant \fr{{\sf E}U}{u^*}\approx\fr{\overline
x}{\alpha u^*}\leqslant \fr{(\overline x)^2}{v_{1-q_n}^2 u^*}\leqslant
\varepsilon\,,
$$
откуда окончательно вытекает оценка
\begin{equation}
u^*\approx\fr{(\overline x)^2}{v_{1-q_n}^2 \varepsilon}\,.\label{e15-kor}
\end{equation}

\begin{figure*}[b] %fig2
\vspace*{18pt}
 \begin{center}
 \mbox{%
 \epsfxsize=162.433mm
 \epsfbox{kor-3.eps}
 }
 \end{center}
 \vspace*{-9pt}
\Caption{Примеры применения модифицированного двухэтапного
сеточного ЕМ-ал\-го\-рит\-ма для подгонки обобщенного гиперболического
распределения к~искусственным данным, $n=10\,000$, $\beta\hm=0$:
(\textit{а})~$\alpha\hm=0{,}3$,
$\nu\hm=2$, $\mu\hm=2$, $\lambda\hm=2{,}5$;
(\textit{б})~$\alpha\hm=0{,}5$,  $\nu\hm=1$, $\mu\hm=1$, $\lambda\hm=3$;
(\textit{в})~$\alpha\hm=0{,}8$,
$\nu\hm=1{,}3$, $\mu\hm=1{,}6$, $\lambda\hm=2$;
(\textit{г})~$\alpha\hm=1{,}3$, $\nu\hm=2$, $\mu\hm=2$, $\lambda\hm=2{,}5$}
\end{figure*}



В случае $q_n\hm\geqslant1/2$, если $n$ достаточно велико, то можно
с~большой степенью уверенности утверж\-дать, что $\overline x\hm\leqslant 0$
и~$-\alpha\hm\geqslant 0$, т.\,е.\ на положительной\linebreak\vspace*{-12pt}

\pagebreak

%\end{multicols}


%\begin{multicols}{2}

\noindent
 полуоси значений аргумента~$u$
функция $\Phi(-\alpha u)$ вогнута, т.\,е.\ выпукла вверх. Тогда
из~(\ref{e11-kor}) и~(\ref{e12-kor}), дважды применяя неравенство Иенсена, в~силу
монотонности функции~$\Phi$ получаем
$$
q_n\approx {\sf E}\Phi\left(-\alpha\sqrt{U}\right)\leqslant
\Phi\left(-\alpha\sqrt{{\sf E}U}\right)\,,
$$
откуда вытекает <<приближенное неравенство>> $v_{q_n}\hm \leqslant
-\alpha\sqrt{{\sf E}U}$,
т.\,е.
$$
-\alpha\geqslant\fr{v_{q_n}}{\sqrt{{\sf E}U}}\approx
\fr{v_{q_n}\sqrt{|\alpha|}}{\sqrt{|\overline x|}}
$$
и при достаточно больших~$n$
\begin{equation}
|\alpha|\geqslant\fr{v_{q_n}^2}{|\overline x|}\,.\label{e16-kor}
\end{equation}
Для определения верхней границы~$u^*$ сетки, накидываемой на
носитель распределения случайной величины~$U$, снова зададим малое
положительное число~$\varepsilon$ и~потребуем, чтобы было
справедливо условие~(\ref{e9-kor}), для гарантированного выполнения которого
с~учетом~(\ref{e6-kor}) и~(\ref{e16-kor}) используем неравенство Маркова и~тот факт, что
$\mathrm{sign}\, \overline x\hm=\mathrm{sign}\,\alpha$ при достаточно
больших~$n$:
\begin{multline}
{\sf P}(U\geqslant u^*)\leqslant \fr{{\sf E}U}{u^*}\approx
\fr{\overline x}{\alpha u^*}=
\fr{|\overline x|}{|\alpha| u^*} \leqslant{}\\
{}\leqslant
\fr{(\overline x)^2}{v_{q_n}^2 u^*}\leqslant
\varepsilon\,.\label{e17-kor}
\end{multline}
В силу симметричности нормального распределения $v_{t}\hm=-v_{1-t}$ для
любого $t\hm\in(0,1)$, поэтому $v_{q_n}^2\hm=v_{1-q_n}^2$ и~в~случае
$q_n\hm\geqslant1/2$ соотношение~(\ref{e17-kor}) снова приводит к~оценке~(\ref{e15-kor}).

Справедливости ради необходимо отметить, что оценки~(\ref{e10-kor}) и~(\ref{e15-kor})
являются завышенными, но они гарантируют, что
$(1-\varepsilon)$-почти-весь носитель распределения случайной
величины~$U$ будет лежать внутри интервала $[0, u^*]$.

\section{Результаты численных экспериментов}

Приводимые в~данном разделе графики иллюстрируют качество работы
модифицированного сеточного метода разделения дис\-пер\-си\-он\-но-сдви\-го\-вых
смесей нормальных законов на примере его\linebreak применения к~оцениванию
параметров обоб\-щенных гиперболических распределений с~ис\-поль\-зованием
указанного алгоритма выбора сетки\linebreak с~умеренным чис\-лом узлов $K\hm=40$.
Для вы\-чис\-ле\-ний использовались искусственно сгенерированные выборки
объемов $n\hm=1000$ и~$n\hm=10\,000$ с~разными наборами параметров, значения
которых указаны на рисунках. На рис.~1 и~2 изображены гистограммы
(серые столбики) и~графики
истинной плот\-ности (штриховые линии), промежуточной
оценки, полученной сеточным ЕМ-ал\-го\-рит\-мом (пунктирные линии)
и~итоговой оценки (непрерывные линии). На рис.~1 и~2 так\-же указаны
значения полученных оценок параметров. Как видно из приводимых
рисунков, параметры~$\alpha$ оцениваются очень точно. Точность
оценок остальных параметров удовлетворительная и~может быть повышена
за счет использования более частых сеток и~более чувствительных
критериев остановки ЕМ-ал\-го\-рит\-ма на первом этапе. Следует отметить,
что даже в~тех случаях, в~которых наблюдаются заметные расхождения
оценок параметров и~их точных значений, оценки самих плотностей
довольно \mbox{точны}.




{\small\frenchspacing
 {%\baselineskip=10.8pt
 \addcontentsline{toc}{section}{References}
 \begin{thebibliography}{99}
\bibitem{k2011}
\Au{Королев В.\,Ю.} Ве\-ро\-ят\-но\-ст\-но-ста\-ти\-сти\-че\-ские методы
декомпозиции волатильности хаотических процессов.~--- М.: Изд-во
Московского ун-та, 2011.

\bibitem{n2013}
\Au{Назаров А.\,Л.} Приближенные методы разделения смесей
вероятностных распределений: Дисс.\ \ldots\  канд. физ.-мат. наук.~--- М.:
МГУ им.\ М.\,В.~Ломоносова, 2013.

\bibitem{BN1977}
\Au{Barndorff-Nielsen~O.-E.} Exponentially decreasing distributions
for the logarithm of particle size~// Proc. Roy. Soc. Lond.~A,
1977. Vol.~353. P.~401--419.

\bibitem{BN1978}
\Au{Barndorff-Nielsen~O.-E.} Hyperbolic distributions and
distributions of hyperbolae~// Scand. J. Statist., 1978. Vol.~5.
P.~151--157.

\bibitem{BN1982}
\Au{Barndorff-Nielsen~O.-E., Kent~J., S\!{\!\ptb{\!\o}}\,rensen~M.} Normal
variance-mean mixtures and $z$-distributions~// Int. Statist. Rev.,
1982. Vol.~50. No.\,2. P.~145--159.

\bibitem{ks2012}
\Aue{Королев В.\,Ю., Соколов И.\,А.} Скошенные распределения
Стьюдента, дисперсионные гам\-ма-рас\-пре\-де\-ле\-ния и~их обобщения как
асимптотические аппроксимации~// Информатика и~её применения, 2012.
Т.~6. Вып.~1. С.~2--10.

\bibitem{zk2013}
\Au{Закс Л.\,М., Королев В.\,Ю.} Обобщенные дисперсионные
гам\-ма-рас\-пре\-де\-ле\-ния как предельные для случайных сумм~// Информатика
и её применения, 2013. Т.~7. Вып.~1. С.~105--115.

\bibitem{k2013}
\Au{Королев В.\,Ю.} Обобщенные гиперболические
распределения как предельные для случайных сумм~// Тео\-рия
вероятностей и~ее применения, 2013. Т.~58. Вып.~1. С.~117--132.

\bibitem{kckg2013}
\Au{Королев В.\,Ю., Черток А.\,В., Корчагин~А.\,Ю.,
Горшенин~А.\,К.} Ве\-ро\-ят\-но\-ст\-но-ста\-ти\-сти\-че\-ское моделирование
информационных потоков в~сложных финансовых системах на основе
высокочастотных данных~// Информатика и~её применения, 2013. Т.~7.
Вып.~1. С.~12--21.

\bibitem{p2004}
\Au{Protassov R.\,S.} EM-based maximum likelihood parameter
estimation for a~multivariate generalized hyperbolic distribution
with fixed~$\lambda$~// Statistics Computing, 2004. Vol.~14.
P.~67--77.

\bibitem{kn2010}
\Au{Королев В.\,Ю., Назаров А.\,Л.} Разделение смесей
вероятностных распределений при помощи сеточных методов моментов и~максимального правдоподобия~//
Автоматика и~телемеханика, 2010. Вып.~3. С.~98--116.

\bibitem{DSch1983}
\Au{Dennis J.\,E., Schnabel R.\,B.} Numerical methods for
unconstrained optimization and nonlinear equations.~--- Englewood
Cliffs: Prentice-Hall, 1983. 378~p.
 \end{thebibliography}

 }
 }

\end{multicols}

\vspace*{-6pt}

\hfill{\small\textit{Поступила в редакцию 01.10.14}}

\newpage

%\vspace*{12pt}

%\hrule

%\vspace*{2pt}

%\hrule

%\vspace*{12pt}

\def\tit{A MODIFIED GRID METHOD FOR~STATISTICAL SEPARATION
OF~NORMAL VARIANCE-MEAN MIXTURES}

\def\titkol{A modified grid method for statistical separation
of~normal variance-mean mixtures}

\def\aut{V.\,Yu.~Korolev$^{1,2}$ and~A.\,Yu.~Korchagin$^1$}

\def\autkol{V.\,Yu.~Korolev and~A.\,Yu.~Korchagin}

\titel{\tit}{\aut}{\autkol}{\titkol}

\vspace*{-9pt}


\noindent
$^1$Faculty of Computational Mathematics and Cybernetics,
M.\,V.~Lomonosov Moscow State University,\linebreak
$\hphantom{^1}$1-52 Leninskiye Gory, GSP-1, Moscow 119991, Russian Federation


\noindent
$^2$Institute of Informatics Problems, Russian Academy of Sciences,
44-2~Vavilov Str., Moscow 119333, Russian\linebreak
$\hphantom{^1}$Federation

\def\leftfootline{\small{\textbf{\thepage}
\hfill INFORMATIKA I EE PRIMENENIYA~--- INFORMATICS AND
APPLICATIONS\ \ \ 2014\ \ \ volume~8\ \ \ issue\ 4}
}%
 \def\rightfootline{\small{INFORMATIKA I EE PRIMENENIYA~---
INFORMATICS AND APPLICATIONS\ \ \ 2014\ \ \ volume~8\ \ \ issue\ 4
\hfill \textbf{\thepage}}}

\vspace*{3pt}

\Abste{A~modified two-stage grid method for
statistical separation of normal variance-mean mixtures is described
as an alternative to a pure EM (expectation-maximization) algorithm.
At the first stage of this
algorithm, a~discrete approximation is constructed to the mixing
distribution. At the second stage, the obtained discrete
distribution is approximated by an absolutely continuous
distribution from a~predetermined family, say, by a generalized
inverse Gaussian distribution. The convergence of this two-stage
procedure is discussed. The monotonicity of the grid procedure used
at the first stage is proved. The problem of the optimal choice of
the parameters of the method is discussed in detail. First of all,
the problem of the optimal choice of the grid thrown on the support
of the mixing distribution is considered. Statistical estimators are
proposed for the quantiles of the mixing law. The efficiency of the
method is illustrated by examples of its application to the
estimation of the parameters of generalized hyperbolic
distributions.}

\smallskip

\KWE{mixture of probability distributions; normal
variance-mean mixture; generalized hyperbolic distribution;
EM-algorithm; grid method of separation of mixtures}

\DOI{10.14357/19922264140402}

\Ack
\noindent
The research was supported by the Russian Science Foundation (project 14-11-00364).

%\vspace*{3pt}

  \begin{multicols}{2}

\renewcommand{\bibname}{\protect\rmfamily References}
%\renewcommand{\bibname}{\large\protect\rm References}



{\small\frenchspacing
 {%\baselineskip=10.8pt
 \addcontentsline{toc}{section}{References}
 \begin{thebibliography}{99}
 \bibitem{k2011eng}
 \Aue{Korolev, V.\,Yu.} 2011.
\textit{Veroyatnostno-statisticheskie metody dekompozitsii
volatil'nosti khaoticheskikh protsessov}
[Probabilistic and statistical methods for the decomposition of volatility
of chaotic processes].
Moscow: Moscow University Press. 510~p.

\bibitem{n2013eng}
\Aue{Nazarov, A.\,L.} 2013.
{Priblizhennye metody razdeleniya smesey veroyatnostnykh raspredeleniy}
[Approximate methods for the decomposition of volatility of chaotic processes].
Ph.D. Thesis. Moscow: Moscow State University.

\bibitem{BN1977eng}
\Aue{Barndorff-Nielsen, O.\,E.} 1977.
Exponentially decreasing distributions for the logarithm of particle size.
\textit{Proc. Roy. Soc. Lond. A} 353:401--419.

\bibitem{BN1978eng}
\Aue{Barndorff-Nielsen, O.\,E.} 1978.
Hyperbolic distributions and distributions of hyperbolae.
\textit{Scand. J. Statist.} 5:151--157.

\bibitem{BN1982eng}
\Aue{Barndorff-Nielsen, O.\,E., J.~Kent, and M.~S\!{\ptb{\o}}rensen}. 1982.
Normal variance-mean mixtures and $z$-distributions.
\textit{Int. Statist. Rev.} 50(2):145--159.

\bibitem{ks2012eng}
\Aue{Korolev, V.\,Yu., and I.\,A. Sokolov}. 2012.
{Skoshennye raspredeleniya St'yudenta, dispersionnye
gam\-ma-ras\-pre\-de\-le\-niya i~ikh obobshcheniya kak asimptoticheskie
approksimatsii}
[Skewed Student's distributions, variance gamma distributions, and their
generalizations as asymptotic approximations].
\textit{Informatika i ee Primeneniya}~--- \textit{Inform. Appl.} 6(1):2--10.

\bibitem{zk2013eng}
\Aue{Korolev, V.\,Yu., and L.\,M.~Zaks}. 2013.
{Obobshchennye dispersionnye gam\-ma-ras\-pre\-de\-le\-niya kak
predel'nye dlya sluchaynykh summ}
[Generalized variance gamma distributions as limiting for random sums].
\textit{Informatika i ee Primeneniya}~--- \textit{Inform. Appl.} 7(1):105--115.

\bibitem{k2013eng} \Aue{Korolev, V.\,Yu.} 2013.
{Obobshchennye giperbolicheskie raspredeleniya kak predel'nye dlya sluchaynykh summ}
[Generalized hyperbolic distributions as limiting for random sums]
\textit{Theory Probab. Appl.} 58(1):117--132.

\bibitem{kckg2013eng}
\Aue{Korolev, V.\,Yu., A.\,V. Chertok, A.\,Yu.~Korchagin, and A.\,K.~Gorshenin}.
2013. {Ve\-ro\-yat\-no\-st\-no-sta\-ti\-sti\-che\-skoe
mo\-de\-li\-ro\-va\-nie informatsionnykh potokov v~slozhnykh finansovykh sistemakh
na osnove vysokochastotnykh dannykh}
[Probability and statistical modeling of information flows in complex
financial systems from high-frequency data].
\textit{Informatika i~ee Primeneniya}~--- \textit{Inform.  Appl.} 7(1):12--21.

\bibitem{p2004eng-1}
\Aue{Protassov, R.\,S.} 2004.
EM-based maximum likelihood parameter estimation for a multivariate
generalized hyperbolic distribution with fixed~$\lambda$.
\textit{Statistics Computing} 14:67--77.

\bibitem{kn2010eng-1}
\Aue{Korolev, V.\,Yu., and A.\,L.~Nazarov}. 2010.
{Razdelenie smesey veroyatnostnykh raspredeleniy pri pomoshchi
setochnykh metodov momentov i~maksimal'nogo pravdopodobiya}
[Separation of mixtures using grid moment-based methods and maximum likelihood].
\textit{Avtomatika i~Telemekhanika} [Automatics and Telemechanics] 3:98--116.

\bibitem{DSch1983eng}
\Aue{Dennis, J.\,E., and R.\,B.~Schnabel}. 1983.
\textit{Numerical methods for unconstrained optimization and nonlinear equations}.
Englewood Cliffs: Prentice-Hall. 378~p.


\end{thebibliography}

 }
 }

\end{multicols}

\vspace*{-6pt}

\hfill{\small\textit{Received October 01, 2014}}

\vspace*{-18pt}

\Contr

\noindent
\textbf{Korolev Victor Yu.} (b.\ 1954)~---
Doctor of Science in physics and mathematics, professor,
Department of Mathematical Statistics, Faculty of Computational Mathematics
and Cybernetics, M.\,V.~Lomonosov Moscow State University,
1-52 Leninskiye Gory, GSP-1, Moscow 119991, Russian Federation;
leading scientist, Institute of Informatics Problems,
Russian Academy of Sciences, 44-2~Vavilov Str., Moscow 119333, Russian
Federation; victoryukorolev@yandex.ru

\vspace*{3pt}

\noindent
\textbf{Korchagin Alexander Yu.} (b.\ 1989)~---
PhD student, Faculty of Computational Mathematics and Cybernetics,
M.\,V.~Lomonosov Moscow State University,
1-52 Leninskiye Gory, GSP-1, Moscow 119991, Russian Federation;
sasha.korchagin@gmail.com


\label{end\stat}

\renewcommand{\bibname}{\protect\rm Литература} %4

\newcommand{\bet}{\beta_{2+\delta}}
\newcommand{\lowaexK}{{\underline{K_{\textsc{ап}}}}}% lower asymptotically exact constant
\newcommand{\lowaex}{{\underline{C_{\textsc{ап}}}}}% lower asymptotically exact constant
\newcommand{\upaex}{\overline{C}_{\textsc{ап}}} % upper asymptotically exact constant

\def\stat{nefedova}
\label{nefedova1}

\def\tit{О ТОЧНОСТИ НОРМАЛЬНОЙ АППРОКСИМАЦИИ ДЛЯ~РАСПРЕДЕЛЕНИЙ ПУАССОНОВСКИХ СЛУЧАЙНЫХ
СУММ$^*$}

\def\titkol{О точности нормальной аппроксимации для распределений пуассоновских случайных
сумм}

\def\autkol{Ю.\,С.~Нефедова, И.\,Г.~Шевцова}
\def\aut{Ю.\,С.~Нефедова$^1$, И.\,Г.~Шевцова$^2$}

\titel{\tit}{\aut}{\autkol}{\titkol}

{\renewcommand{\thefootnote}{\fnsymbol{footnote}}\footnotetext[1]
{Работа поддержана Российским фондом фундаментальных
исследований (проекты 08-01-00563, 08-01-00567, 08-07-00152 и
09-07-12032-офи-м), а также Министерством образования и науки РФ
(грант МК-581.2010.1, государственные контракты П1181, П958, П779 и
16.740.11.0133 в рамках ФЦП <<На\-уч\-ные и на\-уч\-но-пе\-да\-го\-ги\-че\-ские кадры
инновационной России на 2009--2013~годы>>).}}

\renewcommand{\thefootnote}{\arabic{footnote}}
\footnotetext[1]{Московский
государственный университет им.\ М.\,В.~Ломоносова, факультет
вычислительной математики и кибернетики, julia\_n@inbox.ru}
\footnotetext[2]{Московский государственный университет им.\
М.\,В.~Ломоносова, факультет вычислительной математики и кибернетики,
ishevtsova@cs.msu.su}


\Abst{Построены двусторонние оценки для константы в
неравенстве Бер\-ри--Эс\-се\-ена для пуассоновских случайных сумм
независимых одинаково распределенных случайных величин с конечными
моментами порядка $2+\delta,$ где $\delta\in (0\,,1].$ Нижние оценки
получены впервые. Для случая $0<\delta<1$ уточнены верхние оценки
и доказаны неравномерные оценки.}

\KW{центральная предельная теорема;
пуассоновские случайные суммы; неравенство Берри--Эссеена;
абсолютная постоянная; неравномерные оценки}

      \vskip 20pt plus 9pt minus 6pt

      \thispagestyle{headings}

      \begin{multicols}{2}
      
            \label{st\stat}

  \section{Введение}
  
  Пусть $X_1, X_2, \ldots$~--- последовательность независимых
одинаково распределенных случайных величин таких, что
\begin{equation}
{\e}X_1\equiv\mu\,, \enskip {\D}X_1 \equiv \sigma^2>0\,, \enskip
{\e}|X_1|^{3} \equiv \beta_{3}<\infty\,. \label{e1-shev}
\end{equation}
Обозначим $\F_3$ класс всех функций распределения $F$ случайной
величины $X_1$, для которых справедливы условия (1). Пусть
$N_\lambda$ -- случайная величина, имеющая распределение Пуассона с
параметром $\lambda>0.$ Предположим, что при каждом $\lambda>0$
случайные величины $N_\lambda, X_1, X_2,\ldots$ независимы.
  Рассмотрим пуассоновскую случайную сумму
$$
S_{\lambda} = X_1+\cdots+ X_{N_\lambda}\,.
$$
Для определенности полагаем, что $S_\lambda = 0$ при $N_\lambda =
0.$ Несложно видеть, что
$$
{\e}S_\lambda = \lambda\mu\,, \quad {\D}S_\lambda = \lambda(\mu^2+\sigma^2)\,.
$$
Функцию распределения стандартизованной пуассоновской случайной
суммы
$$
\widetilde{S}_\lambda \equiv\fr{S_\lambda-\lambda\mu}{\sqrt{\lambda(\mu^2+\sigma^2)}}
$$
обозначим $F_\lambda(x).$ Всюду далее для определенности полагаем,
что функция распределения непрерывна слева.

\columnbreak

Задаче изучения точности нормальной ап\-прок\-си\-ма\-ции для
распределений пуассоновских слу\-чайных сумм, так называемых
обобщенных пуассоновских распределений, посвящена обширная\linebreak
литература (см., например, библиографию в книгах~\cite{BeningKorolev2002, KBS2007}). Большой интерес к данной задаче
обуслов\-лен тем, что пуассоновские случайные суммы являются
<<накопленными>> значениями маркированного пуассоновского
процесса, который, как отмечено в указанных книгах, может быть
интерпретирован как абсолютно хаотическое случайное блуждание с
дискретным временем. Подобные модели традиционно широко
используются в теории массового обслуживания при анализе
информационных и телекоммуникационных систем, в теории управ\-ле\-ния
запасами, страховой математике и других областях.

Известно, что для приведенных выше условий~(\ref{e1-shev}) на моменты
случайной величины~$X_1$ справедливо неравенство Бер\-ри--Эс\-се\-ена
для пуассоновских случайных сумм: существует абсолютная
положительная и конечная постоянная~$C$ такая, что
\begin{equation}
\rho(F_\lambda,\Phi) \equiv \sup_x\left|F_\lambda(x) -
\Phi(x)\right|\le C  L_\lambda^3\,, 
\label{e2-shev}
\end{equation}
где $\Phi(\cdot)$~--- функция стандартного нормального распределения; $L_\lambda^3$~---
нецентральная ляпуновская дробь:
$$
L_\lambda^3=\fr{\beta_3}{(\mu^2+\sigma^2)^{3/2}\sqrt{\lambda}}\,.
$$


  Неравенство~(\ref{e2-shev}) имеет интересную историю. По-видимому, впервые
это неравенство было доказано в работе~\cite{R1972} и опубликовано
в статье~\cite{R1976} с $C =2{{,}}23$ (диссертация~\cite{R1972} не
опубликована, в то время как в статье~\cite{R1976} не было
приведено доказательство этого результата). Позднее с
использованием традиционной техники, основанной на неравенстве
сглаживания Эссеена, эта оценка была доказана в работе~\cite{vonChossyRappl1983} с $C = 2{,}21.$

  В работе~\cite{Michel1993} с использованием свойства безграничной
делимости обобщенных пуассоновских распределений и оценки абсолютной
постоянной в классическом неравенстве Бер\-ри--Эс\-се\-ена для сумм
неслучайного числа независимых случайных величин, полученной Ван
Биком~\cite{vanBeek1972}, было показано, что в~(\ref{e2-shev}) $C\le 0{,}8.$
Не будучи знакомыми с этой работой Михеля, авторы статьи~\cite{BeningKorolevShorgin1997}, 
применив уточненное неравенство
сглаживания Эссеена, получили оценку $C\le 1{,}99$. Из метода
доказательства, использованного в работе Михеля, вытекает, что если
для абсолютной постоянной~$C_0$ в классическом неравенстве
Бер\-ри--Эс\-се\-ена известна оценка $C_0\le M,$ то неравенство~(\ref{e2-shev})
справедливо с $C=M$. На это обстоятельство также обратили внимание
авторы работы~\cite{KorolevShorgin1997}, в которой независимо от~\cite{Michel1993} 
получен тот же результат, но с другой, лучшей на
тот момент времени, текущей оценкой $M=0{,}7655.$

  Как показано в работах~\cite{KSOPPM2010, KSSAJ2010}, наилучшая на
сегодняшний день оценка абсолютной постоянной в классическом
неравенстве Бер\-ри--Эс\-се\-ена имеет вид $C_0\le 0{,}4784.$ Поэтому,
следуя логике авторов работ~\cite{Michel1993, KorolevShorgin1997}, 
можно заключить, что неравенство~(\ref{e2-shev}) справедливо с $C = 0{,}4784.$

  Однако в тех же работах~[10, 11] показано, что на самом деле привязка оценки
константы~$C$ в~(\ref{e2-shev}) к оценке абсолютной постоянной в
классическом неравенстве Бер\-ри--Эс\-се\-ена $C_0$ менее жесткая. 
А~именно, несмотря на то что, как уже говорилось, наилучшая на
сегодняшний день верхняя граница для $C_0$ равна $0{,}4784,$
неравенство~(\ref{e2-shev}) справедливо с $C=0{,}3041$~\cite{KSOPPM2010, KSSAJ2010}.

  Тем не менее, несмотря на более чем тридцатилетнюю историю
существования неравенства~(\ref{e2-shev}), нижние оценки для~$C$ пока
найдены не были, и в этой работе приводятся впервые.
  Кроме того, будут построены неравномерные оценки, уточ\-ня\-ющие оценки
Михеля~\cite{Michel1993} для случая существования третьих
моментов слагаемых, и впервые приведены неравномерные оценки для
случая существования моментов порядка, меньшего трех. При этом
попутно уточняются приведенные в работах~\cite{Michel1981, PaditzTysiak1990} константы в неравномерных оценках точности
нормальной аппроксимации для распределений сумм детерминированного
числа слагаемых.

  \section{Нижняя оценка для~абсолютной константы в~неравенстве
Берри--Эссеена для~пуассоновских случайных сумм}

  В терминах, введенных в работе~\cite{S2010}, определим верхнюю
асимптотически правильную постоянную
$$
\upaex = \limsup\limits_{\lambda\rightarrow\infty}\sup_{F\in\F_3}
\fr{\rho(F_\lambda,\Phi)}{L_\lambda^3}\,.
$$
Символом $\stackrel{d}{=}$ будет обозначаться
совпадение распределений.

  \medskip
  
  \noindent
  \textbf{Теорема 1.} \textit{ Для константы $C$ в неравенстве}~(\ref{e2-shev})
\textit{справедлива оценка} $$C\ge \upaex \ge
\fr{1}{2}\sup\limits_{\gamma>0}\sqrt{\gamma}\,e^{-\gamma}I_0(\gamma)
= 0{,}2344\ldots,$$ \textit{где $I_0(\gamma)$~--- модифицированная
функция Бесселя нулевого порядка}
$$
I_0(\gamma) =
\sum\limits_{k=0}^\infty\fr{({\gamma}/{2})^{2k}}{(k!)^2}\,.
$$

  \medskip

\noindent
 Д\,о\,к\,а\,з\,а\,т\,е\,л\,ь\,с\,т\,в\,о\,.\ Рассмотрим случайную величину $X_1$:
\begin{align*}
{\sf P}(X_1=-1) &={\sf P} (X_1=1)=p\,,\\
{\sf P}(X_1=0)&=1-2p\,,\qquad 0<p\le \fr{1}{2}\,.
\end{align*}
Легко видеть, что
\begin{gather*}
\mu\equiv {\e}X_1 = 0\,, \enskip \sigma^2\equiv{\D}X_1 = 2p\,, \enskip
\beta_3 \equiv{\e}|X_1|^3 = 2p\,,
\\
L_\lambda^3 \equiv\fr{\beta_3}{\sigma^3\sqrt{\lambda}} =
\fr{1}{\sqrt{2p\lambda}}\,.
\end{gather*}
  Очевидно, что для верхней асимптотически правильной постоянной
$\upaex$ справедлива оценка
\begin{equation}
\upaex\ge
\limsup\limits_{\lambda\rightarrow\infty}\sup\limits_{0<p\le 1/2}\sqrt{2\lambda
p}\,\rho(F_\lambda,\Phi)\,. 
\label{e3-shev}
\end{equation}
  Функцию распределения~$F_{\lambda}(x)$ пуассоновской случайной
суммы~$\widetilde{S}_\lambda$ с использованием формулы полной
вероятности можно представить в виде:
$$
F_{\lambda}(x) = \sum\limits_{n=0}^\infty \fr{\lambda^n
e^{-\lambda}}{n!} \,F_n(x)\,,
$$
где $F_n(x)$~--- функция распределения $S_n = (X_1\hm + \cdots+
X_n)/\sqrt{2\lambda p},$ а $F_0(x)$~--- функция распределения с
единичным скачком в нуле.
  В силу симметрии рассматриваемого трехточечного распределения~$X_1$ 
  справедливо соотношение $S_n \stackrel{d}{=} -S_n$, а
следовательно, для всех $n\ge1$
$$
F_n(0) = {\p}(S_n<0) = \fr{1 - {\p}(S_n=0)}{2}\,,\quad F_0(0) =
0\,.$$
  Учитывая вышесказанное, для $\rho(F_\lambda,\Phi)$ получаем:
\begin{multline*}
\rho(F_\lambda,\Phi) \equiv \sup\limits_x \left|F_\lambda(x) -
\Phi(x)\right|\ge{}\\
{}\ge \left|F_\lambda(0) - \Phi(0)\right| =
\left\vert\sum\limits_{n=0}^\infty \fr{\lambda^n e^{-\lambda}}{n!}
F_n(0) - \fr{1}{2}\right\vert = {}\\
{}
= \left\vert\sum\limits_{n=0}^\infty \fr{\lambda^n e^{-\lambda}}{n!}
\left(F_n(0) - \fr{1}{2}\right)\right\vert
={}\\
{}=\bigg|e^{-\lambda}\left(F_0(0) -
\fr{1}{2}\right)+\sum\limits_{n=1}^\infty \fr{\lambda^n
e^{-\lambda}}{n!}\left( F_n(0) - \fr{1}{2}\right)\bigg| = {}\\
{}
= \bigg|\fr{-e^{-\lambda}}{2}-\sum\limits_{n=1}^\infty
\fr{\lambda^n e^{-\lambda}}{n!}\frac{{\p}( S_n=0)}{2}\bigg|{}=\\
{}=\fr{1}{2}\bigg(e^{-\lambda}+\sum\limits_{n=1}^\infty
\fr{\lambda^n e^{-\lambda}}{n!}{\p}( S_n=0)\bigg) = %{}\\
%{}=
\fr{1}{2}\bigg(e^{-\lambda}+{}\\
{}+\sum\limits_{n=1}^\infty
\fr{\lambda^n e^{-\lambda}}{n!}\sum_{k=0}^{\lfloor
n/2\rfloor}\fr{n!}{k!k!(n-2k)!}p^{2k}(1-2p)^{n-2k}\bigg) = {}\\
{}= 
\fr{1}{2}\bigg(e^{-\lambda}+\sum\limits_{n=1}^\infty
\fr{\lambda^n e^{-\lambda}}{n!}(1-2p)^n+{}\\
{}+\sum\limits_{n=2}^\infty
\fr{\lambda^n e^{-\lambda}}{(n-2)!}\,\fr{p^2(1-2p)^{n-2}}{1!1!}+{}\\
{}+
\sum\limits_{n=4}^\infty \fr{\lambda^n
e^{-\lambda}}{(n-4)!}\,\fr{p^4(1-2p)^{n-4}}{2!2!} + \ldots\bigg)
={}\\
{}
= \fr{e^{-\lambda}}{2}\bigg(\sum\limits_{n=0}^\infty
\fr{(\lambda(1-2p))^n }{n!}+{}\\
{}+\fr{(\lambda
p)}{1!1!}^2\sum\limits_{n=2}^\infty
\fr{(\lambda(1-2p))^{n-2}}{(n-2)!} +{}\\
{}
+ \fr{(\lambda p)^4}{2!2!}\sum\limits_{n=4}^\infty
\fr{(\lambda(1-2p))^{n-4} }{(n-4)!} + \ldots\bigg) ={}\\
{}=\fr{e^{-\lambda}}{2}\sum_{k=0}^\infty \frac{(\lambda p)^{2k}}{(k!)^2}e^{\lambda(1-2p)} = 
\fr{1}{2}\,e^{-2\lambda p}\sum_{k=0}^\infty 
\fr{(\lambda p)^{2k}}{(k!)^2}\,.
\end{multline*}
  Следовательно, можно продолжить цепочку неравенств~(\ref{e3-shev}) следующим
образом:

\noindent
\begin{multline*}
\upaex\ge \limsup\limits_{\lambda\rightarrow\infty}\sup\limits_{0<p\le {1}/{2}}
\sqrt{2\lambda p}\,\rho(F_\lambda,\Phi)\ge{}\\
{}\ge
 \limsup\limits_{\lambda\rightarrow\infty}\sup\limits_{0<p\le 1/2}
\fr{1}{2}\,e^{-2\lambda p}\sqrt{2\lambda p}\,\sum_{k=0}^\infty \fr{(\lambda p)^{2k}}{(k!)^2}\,.
\end{multline*}
Пусть $p = {\gamma}/({2\lambda})$, $0<\gamma\le \lambda$, тогда
окончательно получаем
\begin{multline*}
\upaex\ge
\limsup\limits_{\lambda\rightarrow\infty}\sup_{0<\gamma\le
\lambda}\fr{1}{2}\,e^{-\gamma}\sqrt{\gamma}\,\sum_{k=0}^\infty
\fr{({\gamma}/{2})^{2k}}{(k!)^2} ={}\\
{}=
\fr{1}{2}\,\sup\limits_{\gamma>0}\sqrt{\gamma}\,e^{-\gamma}I_0(\gamma)
= 0{,}2344\ldots\,,
\end{multline*}
причем супремум достигается в точке $\gamma\approx0{,}79.$ Тео\-ре\-ма
доказана.


\section{Аналог неравенства Берри--Эссеена для~пуассоновских случайных сумм слагаемых с моментами порядка 
$2+\delta$}

Теперь предположим, что последовательность независимых одинаково
распределенных случайных величин $X_1, X_2, \ldots$ удовлетворяет
следующим моментным условиям:
$$
\left.
\begin{array}{c}
{\e}X_1\equiv\mu\,, \enskip {\D}X_1 \equiv \sigma^2>0\,; 
\\[9pt]
{\e}|X_1|^{2+\delta} \equiv \beta_{2+\delta}<\infty\!
\end{array}
\right \}
\eqno(1^\ast)
$$
с некоторым $\delta\in(0,1).$ Обозначим $\F_{2+\delta}$ множество
всех функций распределения~$F$ случайной величины~$X_1$,
удовлетворяющих условиям~(1$^*$).  Нецентральная ляпуновская дробь в
этом случае определяется выражением:
$$
L_\lambda^{2+\delta}=\fr{\bet}{(\mu^2+\sigma^2)^{1+\delta/2}
\lambda^{\delta/2}}\,.
$$

  \subsection{Вспомогательные результаты}
  
  \noindent
  \textbf{Лемма 1.} \textit{Пусть $\xi_1, \xi_2,\ldots$~--- независимые
одинаково распределенные случайные величины с нулевым средним,
единичной дисперсией и конечным абсолютным моментом порядка
$2+\delta.$ Тогда существует конечная положительная абсолютная
постоянная~$C(\delta)$ такая, что для всех $n\ge1$ справедливо
неравенство:}
\begin{multline*}
\sup_x\left|{\p}\left(\fr{\xi_1+\cdots+\xi_n}{\sqrt{n}}<x\right)
- \Phi(x)\right|\le{}\\
{}\le
\fr{C(\delta)({\e}|\xi_1|^{2+\delta}+1)}{n^{\delta/2}}\,,
\end{multline*}
%\pagebreak

%\vspace*{9pt}

\noindent
\begin{center}
\parbox{50mm}{{{\tablename~1}\ \ \small{Двусторонние оценки $C(\delta)$ для некоторых
$\delta$}}

}

\vspace*{6pt}

{\small 
\tabcolsep=12pt
\begin{tabular}{|c|c|c|}
  \hline 
  $\delta$ & $C(\delta)\le$ & $C(\delta)\ge$ \\
\hline
1{,}0 & 0{,}3041 & 0{,}2344\\
0{,}9 & 0{,}3089 & 0{,}2383\\
0{,}8 & 0{,}3187 & 0{,}2446\\
0{,}7 & 0{,}3334 & 0{,}2534\\
0{,}6 & 0{,}3538 & 0{,}2651\\
0{,}5 & 0{,}3775 & 0{,}2803\\
0{,}4 & 0{,}4080 & 0{,}3000\\
0{,}3 & 0{,}4450 & 0{,}3257\\
0{,}2 & 0{,}4901 & 0{,}3603\\
0{,}1 & 0{,}5451 & 0{,}4097\\
\hline
\end{tabular}
}
\end{center}
\vspace*{9pt}

\bigskip
\addtocounter{table}{1}

\noindent
\textit{причем для константы $C(\delta)$ справедливы верхние оценки,
приведенные в табл.}~1.



  \medskip

\noindent
Д\,о\,к\,а\,з\,а\,т\,е\,л\,ь\,с\,т\,в\,о\ содержится в работе~\cite{GS2010} для случая 
$\delta\in(0,1)$ и~\cite{KSOPPM2010,KSSAJ2010} для $\delta=1$.
  %\medskip
  %Значения верхних оценок константы $C(\delta)$, полученные в работе
%\cite{GS2010} для некоторых $\delta$, приведены в таблице 1 в
%третьей строке.

  \medskip

  Обозначим $$\nu = \fr{\lambda}{n}\,.$$

  \medskip

\noindent
\textbf{Лемма 2} (см.~\cite{KBS2007, S2007}). \textit{При условиях
$(1^\ast)$ для любого натурального $n\ge1$}
$$
\widetilde{S}_\lambda \stackrel{d}{=}
\fr{1}{\sqrt{n}}\,\sum_{k=1}^n Z_{\nu,k}\,,
$$
\textit{где при каждом $n$ случайные величины $Z_{\nu,1},\ldots,
Z_{\nu,n}$ независимы и одинаково распределены. Более того, $\e
Z_{\nu,1}=0$, $\e Z_{\nu,1}^2=1$ и при всех $n\ge\lambda$}
\begin{equation}
{\e}|Z_{\nu,1}|^{2+\delta}\le
\fr{\beta_{2+\delta}(1+40\nu)}{(\mu^2+\sigma^2)^{1+\delta/2}}
\left(\fr{n}{\lambda}\right)^{\delta/2}\,. 
\label{e4-shev}
\end{equation}

  \subsection{Верхние оценки}
  
  В данном разделе доказывается аналог неравенства Бер\-ри--Эс\-се\-ена для
пуассоновских случайных сумм слагаемых с моментами порядка
$2+\delta$. Аналогичный результат получен в работе~\cite{S2007},
однако он справедлив лишь в асимптотическом смысле, когда
ляпуновская дробь $L_\lambda^{2+\delta}$ бесконечно мала. А~именно,
в~\cite{S2007} была найдена мажоранта не {\it абсолютной}, но {\it
асимптотически правильной} константы:
$$
\limsup\limits_{\ell\rightarrow0}\sup_{\lambda,F\colon
L_\lambda^{2+\delta}=\ell} \fr{\rho(F_\lambda,\Phi)}{\ell}\,.
$$
Следующее утверждение исправляет неточность, допущенную в одном из
результатов упомянутой работы и повлиявшую на окончательный
результат.

  \medskip

\noindent
\textbf{Теорема 2.} \textit{При условиях $(1^\ast)$ для любого
$\lambda>0$ справедливо неравенство:}
\begin{multline}
\rho(F_\lambda,\Phi)\equiv \sup_x|F_\lambda(x) -
\Phi(x)|\le{}\\
{}\le
\fr{C(\delta)\beta_{2+\delta}}
{(\mu^2+\sigma^2)^{1+\delta/2}\lambda^{\delta/2}}\,, 
\label{e5-shev}
\end{multline}
\textit{где $C(\delta)$ та же, что и в лемме~$1$.}

  \medskip

  \noindent
  Д\,о\,к\,а\,з\,а\,т\,е\,л\,ь\,с\,т\,в\,о\,.\ Из леммы~2 вытекает,
что для любого целого $n\ge1$
$$
\rho(F_\lambda,\Phi) =
\sup_x\bigg|{\p}\bigg(\fr{1}{\sqrt{n}}\,\sum_{k=1}^n
Z_{\nu,k}<x\bigg)-\Phi(x)\bigg|\,.
$$ 
Следовательно, по лемме~1 для произвольного целого $n\ge1$ имеем:
\begin{equation}
\rho(F_\lambda,\Phi) \le
C(\delta)\fr{{\e}|Z_{\nu,1}|^{2+\delta}}
{n^{\delta/2}}+\fr{C(\delta)}{n^{\delta/2}}\,. \label{e6-shev}
\end{equation} 
Пусть теперь $n\ge\lambda$. Тогда, используя оценку~(\ref{e4-shev}), 
в продолжение~(\ref{e6-shev}) получаем неравенство:
$$
\rho(F_\lambda,\Phi)\le C(\delta) 
\fr{\beta_{2+\delta}(1+40{\lambda}/{n})}{(\mu^2+\sigma^2)^{1+\delta/2}\lambda^{\delta/2}}+
\fr{C(\delta)}{n^{\delta/2}}\,.
$$
Так как здесь $n\ge\lambda$ произвольно, устремляя
$n\rightarrow\infty,$ окончательно получаем:
\begin{multline*}
\rho(F_\lambda,\Phi)\le \lim_{n\rightarrow\infty}\left[C(\delta)
\fr{\beta_{2+\delta}(1+40{\lambda}/n)}{(\mu^2+\sigma^2)^{1+\delta/2}\lambda^{\delta/2}}+{}\right.\\
\left.{}+
\fr{C(\delta)}{n^{\delta/2}}\right] = 
%{}=
\fr{C(\delta)\beta_{2+\delta}}{(\mu^2+\sigma^2)^{1+\delta/2}\lambda^{\delta/2}}\,,
\end{multline*}
что и требовалось доказать.

  \subsection{Нижние оценки}

  В терминах, введенных в работе~\cite{S2010}, определим верхнюю
асимптотически правильную постоянную для $\delta\in (0,1)$:
$$
\upaex(\delta) = \limsup\limits_{\lambda\rightarrow\infty}
\sup_{F\in\F_{2+\delta}}\fr{\rho(F_\lambda,\Phi)}{L_\lambda^{2+\delta}}\,.
$$

  \medskip

\noindent
  \textbf{Теорема 3.} \textit{ Для константы $C(\delta)$ в
неравенстве~$(5)$ справедлива нижняя оценка}
$$C(\delta)\ge \upaex(\delta) \ge
\fr{1}{2}\,\sup\limits_{\gamma>0}\gamma^{\delta/2}e^{-\gamma}I_0(\gamma)\,,
$$
\textit{где $I_0(\gamma)$~--- модифицированная функция Бесселя}
$$
I_0(\gamma) = \sum\limits_{k=0}^\infty\fr{(\gamma/2)^{2k}}{(k!)^2}\,.
$$

  \medskip

\noindent
Д\,о\,к\,а\,з\,а\,т\,е\,л\,ь\,с\,т\,в\,о\,\ аналогично доказательству тео\-ре\-мы~1.

  \medskip

  Конкретные значения минорант~$C(\delta)$ для некоторых~$\delta$
приведены в табл.~1 в третьем столбце.


\section{Неравномерные оценки точности нормальной аппроксимации
для~обобщенных пуассоновских~распределений}

Данный раздел предварим одним вспомогательным утверждением,
устанавливающим неравномерную оценку скорости сходимости для сумм
детерминированного числа случайных слагаемых.

Пусть $\delta\in(0,\,1]$. Обозначим $S_n = X_1+\cdots+X_n,$
\begin{multline*}
G_n(x) = {\p}\left(\fr{S_n-n\mu}{\sigma\sqrt{n}}<x\right)=
F^{*n}(n\mu+\sigma x\sqrt{n})\,,\\ x\in \mathbb{R}\,.
\end{multline*}

  \medskip

\noindent
  \textbf{Лемма 3.} \textit{Пусть выполнены условия $(1^\ast)$.
Тогда для произвольного целого $n\ge1$ справедлива неравномерная
оценка:}
$$ %\begin{multline*}
\left|G_n(x) - \Phi(x)\right|\le
\fr{K(\delta)}{n^{\delta/2}}\fr{{\e}|X_1 -
\mu|^{2+\delta}}{\sigma^{2+\delta}(1+|x|^{2+\delta})}\,,\enskip
x\in\r\,,
$$ %\end{multline*} 
\textit{где $K(\delta)$ зависит только от $\delta$.}

  \smallskip

  Данное утверждение доказано Нагаевым~\cite{N1965} для случая
$\delta=1$ и Бикялисом~\cite{B1966} для случая произвольного
$\delta\in(0,1]$.

  \medskip

  Используя метод вычисления абсолютной константы в
неравномерном аналоге неравенства Бер\-ри--Эс\-се\-ена (неравенстве
На\-га\-ева--Би\-кя\-ли\-са, см.\ лемму~3), описанный
Падицем~\cite{Paditz1989}, с учетом новых оценок для
абсолютной константы~$C_0(\delta)$ в классическом неравенстве
Бер\-ри--Эс\-се\-ена, полученных в  работах~\cite{KSOPPM2010, KSSAJ2010}
для $\delta=1$ ($C_0(1)\le0{,}4784$) и в~\cite{GS2010} для
$0<\delta<1$ ($C_0(\delta)$ в данной статье не приводится), можно
уточнить верхние оценки константы $K(\delta)$ в
 лемме~3,
полученные в работе~\cite{PaditzTysiak1990} и процитированные в
статье~\cite{Paditz1996} и книге~\cite{KBS2007}. Полученные таким\linebreak
\noindent
\begin{center}
\parbox{50mm}{{{\tablename~2}\ \ \small{Двусторонние оценки $K(\delta)$ для некоторых
$\delta$}}

}

\vspace*{6pt}

{\small 
\tabcolsep=12pt
\begin{tabular}{|c|c|c|}
\hline 
  $\delta$ & $K(\delta)\le$ & $K(\delta)\ge$ \\
\hline
1{,}0 & 25{,}7984 & 0{,}0177\\
0{,}9 & 24{,}2210 & 0{,}0198\\
0{,}8 & 22{,}4063 & 0{,}0223\\
0{,}7 & 20{,}6726 & 0{,}0253\\
0{,}6 & 19{,}0089 & 0{,}0290\\
0{,}5 & 17{,}3674 & 0{,}0334\\
0{,}4 & 15{,}6802 & 0{,}0390\\
0{,}3 & 14{,}0732 & 0{,}0459\\
0{,}2 & 12{,}6421 & 0{,}0550\\
0{,}1 & 11{,}3653 & 0{,}0674\\
\hline
\end{tabular}
}
\end{center}
\vspace*{-6pt}
\columnbreak


%\addtocounter{table}{1}

\noindent
образом верхние оценки константы~$K(\delta)$ приведены в табл.~2
для некоторых~$\delta$.


  \medskip

\noindent
\textbf{Замечание~1.} Оценка $K(1)\le 25{,}80$, приведенная в
табл.~2, уточняет оценку $K(1)\le 30{,}54$ из работы~\cite{Michel1981}.

\noindent
\textbf{Замечание~2.} Нижнюю оценку для константы $K(\delta)$ легко
получить из нижней оценки для $\sup\limits_x|G_n(x)-\Phi(x)|$, доказанной
одним из авторов данной работы в~\cite{S2010} и использованной для
построения миноранты нижней асимптотически\linebreak правильной постоянной в
оценках равномерной мет\-ри\-ки (неравенстве Бер\-ри--Эс\-се\-ена). Введем
нижнюю асимптотически правильную постоянную в неравномерной оценке
(неравенстве На\-га\-ева--Би\-кя\-лиса):
\begin{multline*}
\lowaexK(\delta)= \limsup_{\ell\to0}
\limsup_{n\to\infty}\sup_{F\in\F_{2+\delta}} \sup_{x\in\mathbb{R}}
(1+|x|^{2+\delta})\times{}\\
\times\fr{\vert F^{*n}(n\mu+\sigma x\sqrt{n})-\Phi(x)\vert }{\ell}\,,
\end{multline*}
где супремум берется по всем $F\in\F_{2+\delta}$ с фиксированным
значением (центральной) ляпуновской дроби
$$
\fr{\e|X_1-\mu|^{2+\delta}}{\sigma^{2+\delta}n^{\delta/2}}=\ell\,.
$$
Тогда для константы $K(\delta)$ получаем:
\begin{multline*}
K(\delta)\ge \lowaexK(\delta)\ge{}\\
{}\ge \limsup_{\ell\to0}
\limsup_{n\to\infty}\sup_{F\in\F_{2+\delta}} \sup_{x\in\mathbb{R}}
\fr{1}{\ell}\times{}\\
{}\times{|F^{*n}(n\mu+\sigma x\sqrt{n})-\Phi(x)|} \ge{}\\
{}
\ge\sup_{h\ge0,\,s>0} 
\left(
\fr{4}{\sqrt{2+s^2}}\,\exp \left
\{-\fr{h^2}{2(2+s^2)}\right\}+{}\right.\\
\left.{}+
\fr{h^2+s^2}{\sqrt{2}}-2\sqrt{2}\right)\Bigg / 
\left(\vphantom{\fr{h^2}{2s^2}}
8\varkappa_{2+\delta}\,
s^{2+\delta}\times{}\right.\\
\left.{}\times e^{-h^2/(2s^2)}
{_1F_1}\left(\fr{3+\delta}{2},\fr{1}{2},\fr{h^2}{2s^2}\right)\right)\,,
\end{multline*}
где ${_1F_1}$~--- обобщенная гипергеометрическая функция
(вырожденная функция Мейера):
\begin{multline*}
{_1F_1}(a,b,z)=\fr{\Gamma(b)}{\Gamma(a)}\,\sum_{k=0}^\infty
\fr{\Gamma(a+k)}{\Gamma(b+k)}\fr{z^k}{k!}\,,\\ z>0\,,\quad
{_1F_1}(a,b,0)=1\,;
\end{multline*}

\noindent
$$
\varkappa_{2+\delta}\equiv
\int\limits_{-\infty}^\infty|x|^{2+\delta}\,d\Phi(x)=
(1+\delta)\Gamma\left(\fr{1+\delta}2\right)
\fr{2^{\delta/2}}{\sqrt{\pi}}\,;
$$
$\Gamma(\cdot)$~--- Эйлерова гам\-ма-функ\-ция. Конкретные значения
миноранты введенной нижней асимптотически правильной постоянной
$\lowaexK(\delta)$, а следовательно, и константы~$K(\delta)$, для
некоторых~$\delta$ приведены в табл.~2.

%\pagebreak

Вытекающее из замечания~2 утверждение о положительности нижней
асимптотически правильной постоянной $\lowaexK(\delta)$
представляет особый интерес при $0<\delta<1$, поскольку для этого
случая в работе~\cite{OsipovPetrov1967} было показано, что для
любой функции распределения $F\in\F_{2+\delta}$ найдется
ограниченная убывающая функция $\psi(u)$, $u>0$, с пределом
$\lim\limits_{u\to\infty}\psi(u)=0$ и такая, что
$$
|G_n(x)-\Phi(x)|\le
\fr{\psi(\sqrt{n}(1+|x|))}{n^{\delta/2}(1+|x|^{2+\delta})}\,,\quad
x\in\mathbb{R}\,,\ n\ge1\,,
$$
т.\,е.\ для каждого {\it фиксированного} распределения
$F\in\F_{2+\delta}$ величина $|G_n(x)-\Phi(x)|$ убывает быст\-рее,
чем $n^{\delta/2}(1+|x|^{2+\delta})$, что ставит под сомнение
<<правильность>> (точность) оценок типа неравенства
На\-га\-ева--Би\-кя\-ли\-са. Положительность же нижней асимптотической
постоянной $\lowaexK(\delta)$ означает, что неравенство
На\-га\-ева--Би\-кя\-ли\-са устанавливает <<правильный>> порядок, понимаемый
в {\it равномерном} смысле.

\smallskip

\noindent
\textbf{Теорема 4.} \textit{При условиях $(1^\ast)$ для любого
$\lambda>0$ справедливо неравенство:}
$$
|F_\lambda(x) - \Phi(x)|\le \fr{K(\delta)
L_\lambda^{2+\delta}}{1+|x|^{2+\delta}}\,,\enskip x\in\r\,,
$$
\textit{где $K(\delta)$ та же, что и в лемме~$3$.}

  \medskip

\noindent
Д\,о\,к\,а\,з\,а\,т\,е\,л\,ь\,с\,т\,в\,о\,.\ Из леммы~2 вытекает,
что для любого целого $n\ge1$
$$ %\begin{multline*}
|F_\lambda(x) - \Phi(x)| =
\left\vert{\p}\left(\fr{1}{\sqrt{n}}\,\sum_{k=1}^n
Z_{\nu,k}<x\right)-\Phi(x)\right|\,.
$$ 
Следовательно, по лемме~3 для произвольного целого $n\ge1$ имеем:
\begin{equation}
|F_\lambda(x) - \Phi(x)| \le
\fr{K(\delta)}{n^{\delta/2}}\,\fr{{\e}|Z_{\nu,1}|^{2+\delta}}{1+|x|^{2+\delta}}\,.
\label{e7-shev}
\end{equation}
Пусть теперь $n\ge\lambda.$ Тогда, используя оценку~(\ref{e4-shev}), 
в продолжение~(\ref{e7-shev}) получаем неравенство
$$ %\begin{multline*}
|F_\lambda(x) - \Phi(x)|
\le \fr{K(\delta)}{\lambda^{\delta/2}}\, 
\fr{\beta_{2+\delta}(1+40{\lambda}/n)}{(\mu^2+\sigma^2)^{1+\delta/2}(1+|x|^{2+\delta})}\,.
$$ %\end{multline*}
Так как здесь $n\ge\lambda$  произвольно, устремляя
$n\rightarrow\infty,$ окончательно получаем

\noindent
\begin{multline*}
|F_\lambda(x) - \Phi(x)|\le {}\\
{}\le \lim_{n\rightarrow\infty}\left[
\fr{K(\delta)}{\lambda^{\delta/2}}\, 
\fr{\beta_{2+\delta}(1+40{\lambda}/{n})}{(\mu^2+\sigma^2)^{1+\delta/2}(1+|x|^{2+\delta})}\right] ={}\\
{}
=\fr{K(\delta)}{\lambda^{\delta/2}}\,\fr{\beta_{2+\delta}}{(\mu^2+\sigma^2)^{1+\delta/2}(1+|x|^{2+\delta})}\,.
\end{multline*}

    \medskip

  Аналогично доказывается и более общее утверж\-де\-ние.

  \medskip

\noindent
\textbf{Теорема~5.} \textit{Предположим, что существует функция
$Q(x)$ такая, что для всех $x\in\mathbb{R}$}
\begin{multline*}
\left|F_n(x) - \Phi(x)\right| =\left| {\p}\left(\fr{S_n-n\mu}{\sigma\sqrt{n}}<x\right)- \Phi(x)\right|\le {}\\
{}\le
\fr{Q(x)}{n^{\delta/2}}\, \fr{{\e}|X_1 - \mu|^{2+\delta}}{\sigma^{2+\delta}}\,.
\end{multline*}
\textit{Тогда верна оценка:}
\begin{multline*}
\left|F_\lambda(x) - \Phi(x)\right| ={}\\
{}=\left|
{\p}\left(\fr{S_\lambda-\lambda\mu}{\sqrt{\lambda(\mu^2+\sigma^2)}}<x\right)-
\Phi(x)\right|\le{}\\
{}\le \fr{Q(x)}{\lambda^{\delta/2}}\,
\fr{{\e}|X_1|^{2+\delta}}{(\mu^2+\sigma^2)^{1+\delta/2}}\,,\enskip
x\in\mathbb{R}\,,
\end{multline*}
\textit{с той же самой $Q(x)$.}

  \bigskip
  В заключение авторы выражают признательность В.\,Ю.~Королеву за 
  стимулирующие дискуссии и постоянное внимание к работе.

{\small\frenchspacing
{%\baselineskip=10.8pt
\addcontentsline{toc}{section}{Литература}
\begin{thebibliography}{99}

\bibitem{BeningKorolev2002} %1
\Au{Bening V., Korolev V.}  Generalized Poisson models and their
applications in insurance and finance.~--- Utrecht: VSP, 2002.

  \bibitem{KBS2007} %2
\Au{Королев В.\,Ю., Бенинг В.\,Е., Шоргин~С.\,Я.} Математические
основы теории риска.~--- М.: Физматлит, 2007.

  \bibitem{R1972} %3
\Au{Ротарь Г.\,В.} Некоторые задачи планирования резерва. Дис.\ \ldots канд. физ.-мат. наук.~--- М.: Центральный
эко\-но\-ми\-ко-ма\-те\-ма\-ти\-че\-ский институт, 1972.

  \bibitem{R1976} %4
\Au{Ротарь Г.\,В.} Об одной задаче управления резервами~//
Эко\-но\-ми\-ко-ма\-те\-ма\-ти\-че\-ские методы, 1976. Т.~12. Вып.~4. С.~733--739.

  \bibitem{vonChossyRappl1983} %5
\Au{Von Chossy R., Rappl G.} Some approximation methods for the
distribution of random sums~// Insurance: Mathematics and Economics,
1983. Vol.~2. No.\,1. P.~251--270.

  \bibitem{Michel1993} %6
\textit{Michel R.} On Berry--Esseen results for the compound Poisson
distribution~// Insurance: Mathematics and Economics, 1993. Vol.~13. No.\,1. P.~35--37.

  \bibitem{vanBeek1972} %7
\Au{Van Beek P.} An application of Fourier methods to the
problem of sharpening the Berry--Esseen inequality~// Z.~Wahrsch.
verw. Geb., 1972. Bd.~23. S.~187--196.

\bibitem{BeningKorolevShorgin1997} %8
\Au{Bening V.\,E., Korolev~V.\,Yu., Shorgin~S.\,Ya.} On
approximations to generalized Poisson distribution~// J.\
Math. Sci., 1997. Vol.~83. No.\,3. P.~360--367.

  \bibitem{KorolevShorgin1997} %9
\Au{Korolev V.\,Yu., Shorgin S.\,Ya.} On the absolute constant in
the remainder term estimate in the central limit theorem for Poisson
random sums~// Probabilistic Methods in Discrete Mathematic:
4th International Petrozavodsk Conference Proceedings.~---
Utrecht: VSP, 1997. P.~305--308.

  \bibitem{KSOPPM2010} %10
\Au{Королев В.\,Ю., Шевцова И.\,Г.} Уточнение неравенства
Бер\-ри--Эс\-се\-ена с приложениями к пуассоновским и смешанным
пуассоновским случайным суммам~// Обозрение прикладной и
промышленной математики, 2010. Т.~17. Вып.~1. С.~25--56.

  \bibitem{KSSAJ2010} %11
\Au{Korolev V.\,Yu., Shevtsova~I.\,G.} An improvement of the
Berry--Esseen inequality with applications to Poisson and mixed
Poisson random sums~// Scandinavian Actuarial J., 2011 (in
press). Online first:
{\sf http://www.informaworld.com/10.1080/ 03461238.2010.485370}.

  \bibitem{Michel1981} %12
\Au{Michel R.} On the constant in the nonuniform version of the
Berry--Esseen theorem~// Z. Wahrsch. verw. Geb., 1981. Bd.~55. S.~109--117.

  \bibitem{PaditzTysiak1990} %13
\Au{Paditz L., Tysiak W.} Quantitative Auswertung einer
ungleichm$\ddot{\mbox{a}}${\ss}igen Fehlerabschh$\ddot{\mbox{a}}$tzung im zentralen
Grenwertsatz~// Mathematiker-Kongre{\ss} der DDR.
Vortragsausz$\ddot{\mbox{u}}$ge III.~--- Dresden, 1990. S.~153.

\columnbreak

  \bibitem{S2010} %14
\Au{Шевцова И.\,Г.} Об асимптотически правильных постоянных в
неравенстве Бер\-ри--Эс\-се\-ена--Ка\-ца~// Теория вероятностей и ее
применения, 2010. Вып.~2. С.~271--304.

  \bibitem{GS2010} %15
\Au{Григорьева М.\,Е., Шевцова И.\,Г.} Уточнение неравенства
Ка\-ца--Бер\-ри--Эс\-се\-ена~// Информатика и её применения, 2010. Т.~4.
Вып.~2. С.~78--85.

  \bibitem{S2007} %16
\Au{Шевцова И.\,Г.} О точности нормальной аппроксимации для
распределений пуассоновских случайных сумм~// Обозрение промышленной
и прикладной математики, 2007. Т.~14. Вып.~1. С.~3--28.

  \bibitem{N1965} %17
\Au{Нагаев С.\,В.} Некоторые предельные теоремы для больших
уклонений~// Теория вероятностей и ее применения, 1965. Т.~10. Вып.~2. С.~231--254.

  \bibitem{B1966} %18
\Au{Бикялис А.} Оценки остаточного члена в центральной
предельной теореме // Литовский математический сборник, 1966. Т.~6.
№\,3. С.~323--346.

  \bibitem{Paditz1989} %19
\Au{Paditz L.} On the analytical structure of the constant in
the nonuniform version of the Esseen inequality~// Statistics.~---
Berlin: Akademie-Verlag, 1989. Vol.~20. No.\,3. P.~453--464.

  \bibitem{Paditz1996} %20
\Au{Paditz L.} On the error-bound in the the nonuniform version
of Esseen's inequality in the $L_p$-metric~// Statistics.~--- Berlin: Akademie-Verlag, 1996. Vol.~27.
No.\,3. P.~379--394.

 \label{end\stat}
 \label{end-nefedova}

\bibitem{OsipovPetrov1967} %21
\Au{Осипов Л.\,В., Петров В.\,В.} Об оценке остаточного члена в
центральной предельной теореме~// Теория вероятностей и ее
применения, 1967. Т.~12. Вып.~2. С.~322--329.

 \end{thebibliography}
}
}


\end{multicols}    %5
\def\stat{chubich}

\def\tit{ИНФОРМАЦИОННАЯ ТЕХНОЛОГИЯ АКТИВНОЙ  ПАРАМЕТРИЧЕСКОЙ ИДЕНТИФИКАЦИИ
СТОХАСТИЧЕСКИХ КВАЗИЛИНЕЙНЫХ ДИСКРЕТНЫХ  СИСТЕМ$^*$}

\def\titkol{Информационная технология активной  параметрической идентификации
%стохастических квазилинейных дискретных  систем
}

\def\autkol{В.\,М.~Чубич}
\def\aut{В.\,М.~Чубич$^1$}

\titel{\tit}{\aut}{\autkol}{\titkol}

{\renewcommand{\thefootnote}{\fnsymbol{footnote}}\footnotetext[1]
{Работа выполнена  при поддержке Федерального агентства по образованию в рамках ФЦП
<<Научные и на\-уч\-но-пе\-да\-го\-ги\-че\-ские кадры инновационной России>> на
  2009--2013~гг.\ (гос. контракт №\,П2365 от 18.11.2009~г.).}}

\renewcommand{\thefootnote}{\arabic{footnote}}
\footnotetext[1]{Новосибирский государственный технический университет, chubich\_62@ngs.ru}


\vspace*{6pt}

  \Abst{Впервые рассмотрены теоретические и прикладные аспекты проблемы активной
па\-ра\-мет\-ри\-че\-ской идентификации гауссовских нелинейных дискретных сис\-тем. Рассмотрен
случай вхождения подлежащих оцениванию параметров в уравнения состояния и
наблюдения, начальные условия и ковариационные матрицы помех динамики и ошибок
измерений. Приведены оригинальные результаты. Рассмотрен пример оптимального
оценивания параметров одной модельной структуры.}

\vspace*{2pt}

  \KW{оценивание параметров; метод максимального правдоподобия; планирование
оптимальных входных сигналов; информационная матрица Фишера; критерий
оптимальности}

      \vskip 14pt plus 9pt minus 6pt

      \thispagestyle{headings}

      \begin{multicols}{2}

            \label{st\stat}

\section{Введение}

  Проблема идентификации~--- одна из основных проблем теории и практики
автоматического управ\-ле\-ния. Результаты решения задачи идентификации могут
использоваться при проектировании различных систем управления
подвижными и технологическими объектами, при построении прогнозирующих
моделей, при конструировании следящих и измерительных систем.

  По способу проведения эксперимента существующие методы идентификации
можно разделить на пассивные и активные. При пассивной идентификации для
построения математической модели используются реально действующие в
системе сигналы и тем самым нормальный режим эксплуатации не нарушается.
Методы пассивной идентификации достаточно полно описаны, например,
  в~[1--3]. Активная идентификация, напротив, предполагает нарушение
технологического режима и подачу на вход изучаемой системы специальным
образом синтезированного сигнала. Его находят в результате решения
экстремальной задачи для некоторого предварительно выбранного
функционала от информационной (или дисперсионной) матрицы вектора
оцениваемых параметров. Трудности, связанные с необходимостью нарушения
технологического режима, окупаются повышением эффективности и
корректности проводимых исследований. Это обусловлено самой идеологией
активной идентификации, базирующейся на сочетании традиционных приемов
параметрического оценивания с концепцией планирования эксперимента [4--7].

  Более определенно процедура активной параметрической идентификации
систем с предварительно выбранной модельной структурой предполагает
выполнение следующих этапов:
  \begin{enumerate}[1.]
\item Вычисление оценок параметров по измерительным данным,
соответствующим выбранному пробному сигналу.
\item Синтез с учетом полученных на первом этапе оценок оптимального в
соответствии с некоторым критерием сигнала.
\item Пересчет оценок неизвестных параметров по измерительным данным,
соответствующим полученному на втором этапе сигналу.
\end{enumerate}

  Целесообразность применения концепции активной идентификации при
построении ма\-те\-ма\-тических моделей стохастических стационарных линейных
дискретных и непрерывно-дискретных\linebreak сис\-тем показана в~[8--10]. Рецензия на
монографию~\cite{10-c} помещена в~\cite{11-c}. В~данной статье приведены
результаты дальнейших исследований автора в рамках указанной проблемы
применительно к стохастическим нелинейным дискретным системам.

\section{Постановка задачи}

  Рассмотрим следующую модель управляемой, наблюдаемой,
идентифицируемой динамической системы в пространстве состояний:

\noindent
  \begin{align}
  x(k+1)&=f[x(k),u(k),k]+\Gamma(k) w(k)\,,
  \label{e1-c}\\
  y(k+1) & =h[x(k+1),k+1]+v(k+1)\,,\notag\\
  &\hspace*{13mm} k=0, 1, \ldots , N-1\,,
  \label{e2-c}
  \end{align}
где $x(k)$~--- $n$-век\-тор состояния; $u(k)$~--- детерминированный
$r$-век\-тор управления (входа); $w(k)$~--- $p$-век\-тор возмущения;
$y(k+1)$~--- $m$-век\-тор измерения (выхода); $v(k+1)$~--- $m$-век\-тор
ошибки измерения.

  Предположим, что век\-тор-функ\-ции $f[x(k),u(k),k]$ и $h[x(k+1),k+1]$
непрерывны и дифференцируемы по~$x(k)$, $u(k)$ и $x(k+1)$ соответственно;
случайные векторы~$w(k)$ и~$v(k+1)$ образуют стационарные белые
гауссовские последовательности, для которых $E[w(k)]=0$,
$E[w(k)w^{\mathrm{T}}(i)]\hm =Q\delta_{ki}$, $E[v(k+1)]=0$, $E[v(k+1)v^{\mathrm{T}} (i+1)] R\delta_{ki}$,
$E[v(k+1) w^{\mathrm{T}}(i)=0$, $k,i=0, 1, \ldots , N-1$ (здесь и далее $E[\cdot]$~--- оператор
математического ожидания, $\delta_{ki}$~--- символ Кронекера); начальное
состояние~$x(0)$ имеет нормальное распределение с параметрами
$E[x(0)]=\overline{x}(0)$, $E\{[x(0)-\overline{x}(0)][x(0)-
\overline{x}(0)]^{\mathrm{T}}\}=P(0)$ и не коррелирует с~$w(k)$ и~$v(k+1)$ при любых
значениях переменной~$k$; неизвестные параметры сведены в вектор $\Theta =
(\theta_1,\theta_2,\ldots , \theta_s)$, включающий в себя элементы
  век\-тор-функ\-ций $f[x(k), u(k),k]$, $h[x(k+1), k+1]$, мат\-риц $\Gamma(k)$,
$Q$, $R$, $P(0)$ и вектора~$\overline{x}(0)$ в различных комбинациях.

  Необходимо для математической модели~(1), (2) с учетом высказанных
априорных предположений разработать процедуру активной параметрической
идентификации и исследовать ее эффективность. В~такой математической
постановке задача рассматривается и решается впервые.

\vspace*{-4pt}

\section{Линеаризация модели}

\vspace*{-2pt}

  Считая значение вектора неизвестных параметров~$\Theta$ фиксированным,
выполним линеаризацию во временной области нелинейной модели~(1), (2)
относительно номинальной траектории $\{x_H(k+1),\ k=0, 1, \ldots , N-1\}$, для
которой
  \begin{equation}
  \left.
  \begin{array}{rl}
  x_H(k+1) & =f[x_H(k), u_H(k),k]\,,\\[3pt]
  &\hspace*{7mm} k=0,1, \ldots , N-1\,;\\[3pt]
  x_H(0) & =\overline{x}(0)\,.
  \end{array}
  \right \}
  \label{e3-c}
  \end{equation}

  Разложив для каждого~$k$ век\-тор-функ\-ции $f[x(k), u(k), k]$ и $h[x(k+1),
k+1]$ в ряды Тейлора в окрестностях точек $[x_H(k), u_H(k)]$ и $x_H(k+1)$
соответственно и отбросив члены второго и более высоких порядков, запишем
уравнения линеаризованной модели

\noindent
  \begin{multline*}
  x(k+1)=f\left[x_H(k),u_H(k),k\right]+{}\\
  {}+\fr{\partial f[x_H(k),u_H(k)k]}{\partial
x(k)}\left[x(k)-x_H(k)\right]+{}\hspace*{5mm}
\end{multline*}
  \begin{multline}
  {}+\fr{\partial f[x_H(k),u_H(k)k]}{\partial u(k)}\left[ u(k)-
u_H(k)\right]+{}\\
{}+\Gamma(k) w(k)\,;
\label{e4-c}
  \end{multline}
  
  \vspace*{-12pt}

  \noindent
  \begin{multline}
  y(k+1) =h\left[x_H(k+1),k+1\right]+{}\\
  {}+
  \fr{\partial h[x_H(k+1),k+1]}{\partial x(k+1)}\left[ x(k+1)-
x_H(k+1)\right]+{}\\
{}+v(k+1)\,,\label{e5-c}
  \end{multline}
для которой и будем решать поставленную задачу. С~учетом обозначений
\begin{multline}
a(k)=f\left[ x_H(k),u_H(k),k\right] -{}\\
{}-\fr{\partial f [x_H(k),u_H(k),k]}{\partial
x(k)}\,x_H(k)+{}\\
{}+\fr{\partial f[x_H(k),u_H(k),k]}{\partial u(k)}\left [ u(k)-u_H(k)\right]\,;
\label{e6-c}
\end{multline}

\vspace*{-8pt}

\noindent
\begin{equation}
\Phi(k)=\fr{\partial f[x_H(k),u_H(k),k]}{\partial x(k)}\,;
\label{e7-c}
\end{equation}

\vspace*{-14pt}

\noindent
\begin{multline}
A(k+1)=h\left[ x_H(k+1),k+1\right]-{}\\
{}-\fr{\partial h [x_H(k+1),k+1]}{\partial
x(k+1)}\,x_H(k+1)\,;
\label{e8-c}
\end{multline}

\vspace*{-8pt}

\noindent
\begin{equation}
H(k+1)=\fr{\partial h[x_H(k+1),k+1]}{\partial x(k+1)}
\label{e9-c}
\end{equation}
соотношения~(4), (5) определяют модель гауссовской линейной
нестационарной системы, описывающейся уравнениями:
\begin{align}
x(k+1) &= a(k)+\Phi(k)x(k)+\Gamma(k) w(k)\,;\label{e10-c}\\[4pt]
y(k+1) &=A(k+1) +H(k+1) x(k+1)+v(k+1)\,,\notag\\
&\hspace*{20mm} k=0, 1, \ldots , N-1\,.\label{e11-c}
\end{align}

  Заметим, что для нелинейностей, имеющих характеристики с угловыми
точками и разрывами, можно воспользоваться методом статистической
линеаризации~\cite{12-c, 13-c}.

\section{Оценивание неизвестных параметров}

  Оценивание неизвестных параметров математической модели
осуществляется по данным наблюдений $\Xi$ в соответствии с критерием
идентификации~$\chi$. Сбор числовых данных происходит в процессе
проведения идентификационных экспериментов, которые выполняются по
некоторому плану~$\xi_v$.

  Предположим, что экспериментатор может произвести $v$ запусков
системы, причем сигнал~$U_1$ он
 подает на вход системы $k_1$~раз, сигнал
$U_2$~--- $k_2$~раз\linebreak\vspace*{-12pt}

\pagebreak

\noindent
 и~т.\,д., наконец, сигнал $U_q$~--- $k_q$~раз. В~этом
случае дискретный (точный) нормированный план эксперимента~$\xi_v$
представляет собой совокупность точек $U_1, U_2, \ldots , U_q$, называемых
спектром плана, и соответствующих им долей повторных запусков:
  $$
  \xi_v=\left \{
  \begin{matrix}
  U_1, U_2, \ldots , U_q\\
  \fr{k_1}{v}, \fr{k_2}{v}, \ldots ,\fr{k_q}{v}
  \end{matrix}
  \right \}\,,
  \enskip U_i\in \Omega_U,\ i=1, 2, \ldots ,q\,.
  $$
Здесь $U_i^{\mathrm{T}}=\{[u^i(0)]^{\mathrm{T}}, [u^i(1)]^{\mathrm{T}}, \ldots , [u^i(N-1)]^{\mathrm{T}}\}$, $i=1, 2, \ldots , q$;
$\Omega_U\subset R^{Nr}$ задает ограничения на условия проведения
эксперимента.

  Обозначим через $Y_{i,j}$ $j$-ю реализацию выходного сигнала ($j=1, 2,
\ldots , k_i$), соответствующую $i$-му\linebreak входному сигналу~$U_i$
  ($i=1, 2, \ldots ,q$). Тогда в результате проведения по плану
$\xi_v$~идентификационных экспериментов будет сформировано множество:
  \begin{multline*}
  \Xi = \left\{ (U_i, Y_{i,j}),\ j=1, 2, \ldots ,
  k_i, \ i=1, 2, \ldots , q\right \}\,,\\
\sum\limits_{i=1}^q k_i=v\,.
  \end{multline*}
Уточним структуру $Y_{i,j}$:
\begin{multline*}
Y_{i,j}^{\mathrm{T}} =\left \{ \left[y^{i,j}(1)\right]^{\mathrm{T}}, \left [y^{i,j}(2)\right]^{\mathrm{T}}, \ldots , \left[
y^{i,j}(N)\right ]^{\mathrm{T}}\right \}\,,\\
j=1, 2,\ldots , k_i\,,\ i=1, 2, \ldots , q\,,
\end{multline*}
и заметим, что в случае пассивной параметрической идентификации, как
правило, $q=v=1$.

  Априорные предположения, высказанные при постановке задачи, и
выполненная линеаризация моделей состояния и наблюдения относительно
опорной траектории~(3) позволяют воспользоваться для оценивания
неизвестных параметров методом максимального правдоподобия (ММП).
В~соответствии с этим методом необходимо найти такие значения параметров
$\hat\Theta $, для которых
  $$
  \hat\Theta  =\mathrm{arg}\,\min_{\Theta\in \Omega_\Theta}\left[ \chi(\Theta,
\Xi)\right] =\mathrm{arg}\,\min_{\Theta\in\Omega_\Theta} \left[-\ln L(\Theta,
\Xi)\right]\,,
  $$
где $\ln L(\Theta, \Xi)$~--- логарифмическая функция правдоподобия.

  Согласно~\cite{14-c, 15-c} в нашем случае критерий идентификации имеет
следующий вид:
  \begin{multline*}
  \chi(\Theta, \Xi) =\fr{Nmv}{2}\,\ln 2\pi+{}\\
  {}+\fr{1}{2}\sum\limits_{i=1}^q
\sum\limits_{j=1}^{k_i}\sum\limits_{k=1}^{N-1}\left [
\varepsilon^{i,j}(k+1)\right]^{\mathrm{T}} \left[B^i(k+1)\right]^{-1}\times{}\\
{}\times \left[
\varepsilon^{i,j}(k+1)\right] +\fr{1}{2}\,\sum\limits_{i=1}^q
k_i\sum\limits_{k=0}^{N-1}\ln \mathrm{det}\, B^i(k+1)\,,
  \end{multline*}
где
$\varepsilon^{i,j}(k+1) =y^{i,j}(k+1)+\hat{y}^{i,j}(k+1\vert k)$,
а $\hat{y}^{i,j}(k+1\vert k)$ и $B^i(k+1)$ определяются по соответствующим
ре-\linebreak\vspace*{-12pt}
\columnbreak

\noindent
куррентным уравнениям дискретного фильтра Калмана (см.,
например,~\cite{16-c}):
\begin{align}
\hat{x}^{i,j}(k+1\vert k) &=\Phi^i(k) \hat{x}^{i,j}(k\vert k)+a^i(k)\,;\notag\\
P^i (k+1\vert k)&={}\notag\\
&\hspace*{-16mm}{}=\Phi^i (k) P^i (k\vert k)[\Phi^i (k) ]^{\mathrm{T}}+
\Gamma(k) Q \Gamma^{\mathrm{T}}(k)\,;\label{e12-c}\\
\hat{y}^{i,j}(k+1\vert k) &={}\notag\\
&\hspace*{-5mm}{}=H^i(k+1) \hat{x}^{i,j}(k+1\vert k)+A^i(k+1)\,;\notag\\
B^i(k+1) &={}\notag\\
&\hspace*{-22mm}{}=H^i(k+1)P^i(k+1\vert k)\left[H^i(k+1)\right]^{\mathrm{T}}+R\,;\label{e13-c}\\
K^i(k+1) & ={}\notag\\
&\hspace*{-22mm}{}= P^i(k+1\vert k) \left[ H^i(k+1)\right]^{\mathrm{T}}\left[B^i(k+1)\right]^{-1}\,;\label{e14-c}\\
\hat{x}^{i,j}(k+1\vert k+1) & ={}\notag\\
&\hspace*{-10mm}{}=\hat{x}^{i,j}(k+1\vert  k)+K^i(k+1)\varepsilon^{i,j}(k+1)\,;\notag\\
P^i(k+1 \vert k+1) &={}\notag\\
&\hspace*{-16mm}{}=\left[ I-K^i(k+1)H^i(k+1)\right] P^i(k+1\vert k)\label{e15-c}
\end{align}
для $k=0, 1, \ldots , N-1$, $j=1, 2, \ldots , k_i$, $i=1, 2, \ldots , q$ с начальными
условиями $\hat{x}^{i,j}(0\vert 0)=\overline{x}(0)$, $P(0\vert 0)\hm=P(0)$.

  Для нахождения условного минимума $\chi(\Theta, \Xi)$ воспользуемся
методом проекции градиента~\cite{17-c, 18-c}, учитывая, что
  \begin{multline*}
  \fr{\partial \chi(\theta)}{\partial \theta_l} = \sum\limits_{i=1}^q
\sum\limits_{j=1}^{k_i}\sum\limits_{k=0}^{N-1} \left[ \fr{\partial
\varepsilon^{i,j}(k+1)}{\partial\theta_l}\right]^{\mathrm{T}} \times{}\\
{}\times \left[ B^i(k+1)\right ]^{-
1}\left[\varepsilon^{i,j}(k+1)\right]-{}\\
  {}- \fr{1}{2}\sum\limits_{i=1}^q
\sum\limits_{j=1}^{k_i}\sum\limits_{k=0}^{N-1} \left[
\varepsilon^{i,j}(k+1)\right]^{\mathrm{T}} \left[B^i(k+1)\right]^{-1}\times{}\\
{}\times
\fr{\partial
B^i(k+1)}{\partial\theta_l}\left[ B^i(k+1)\right]^{-1}\varepsilon^{i,j}(k+1)+{}\\
  {}+\fr{1}{2}\,\sum\limits_{i=1}^q k_i \sum\limits_{k=0}^{N-1} Sp \left[\left[
B^i(k+1)\right]^{-1}\fr{\partial B^i(k+1)}{\partial\theta_l}\right]\,,\\
 l=1, 2, \ldots ,s\,.
  \end{multline*}
  
\vspace*{-8pt}

\section{Планирование оптимальных входных сигналов}

\vspace*{-2pt}

     Оптимальный выбор входных сигналов позволяет экспериментатору при
заданном чис\-ле запусков системы подготовить наиболее информативные
данные наблюдений, использующиеся для на\-хож\-де\-ния оценок неизвестных
параметров. Отметим, что свобода в выборе входных характеристик
суще-\linebreak\vspace*{-12pt}
\pagebreak

\noindent
ственно различается в зависимости от приложений. В~экономических и
экологических системах у экспериментатора нет возможности воздействовать
на систему с целью проведения идентификационных экспериментов, в то время
как в лабораторных условиях и на стадиях разработки нового оборудования
выбор входных величин имеет лишь амплитудные и мощностные ограничения.

  Предварим рассмотрение алгоритмов синтеза оптимальных входных
сигналов изложением некоторых основополагающих понятий и результатов
теории планирования эксперимента для нашего случая.

  Под непрерывным нормированным планом~$\xi$ условимся понимать
совокупность величин:
  \begin{multline}
  \xi =
    \left \{
  \begin{matrix}
  U_1, U_2, \ldots , U_q\\[3pt]
  p_1, p_2, \ldots , p_q
  \end{matrix}
  \right \}\,, \enskip p_i\geq 0\,,\enskip
  \sum\limits_{i=1}^q p_i=1\,,\\
  U_i\in \Omega_U\,,\enskip   i=1, 2, \ldots , q\,.
    \label{e16-c}
\end{multline}
Здесь $U_i^{\mathrm{T}}=\{[u^i(0)]^{\mathrm{T}}, [u^i(1)]^{\mathrm{T}}, \ldots , [u^i(N-1)]^{\mathrm{T}}\}$, $i=1, 2, \ldots ,q$,
по аналогии с дискретным планом~$\xi_v$, но веса~$p_i$ могут принимать
любые значения в диапазоне от~0 до~1, в том числе и иррациональные.
Множество планирования~$\Omega_U$ определяется ограничениями на
условия проведения эксперимента.

  Для плана~(\ref{e16-c}) нормированная информационная матрица~$M(\xi)$
определяется соотношением:
  \begin{equation}
  M(\xi)=\sum\limits_{i=1}^q p_i M(U_i;\theta)\,,\label{e17-c}
  \end{equation}
в котором информационные матрицы Фишера (ИМФ) одноточечных планов
$$
M(U;\theta)=- \underset{Y}{E} \left [ \fr{\partial^2\ln L(\Theta,
\Xi)}{\partial\theta\partial\theta^{\mathrm{T}}}\right]
$$
зависят от неизвестных параметров~$\Theta$, что позволяет в дальнейшем
говорить только о локально-оп\-ти\-маль\-ном планировании. В~\cite{19-c}
приводятся выражение и алгоритм вычисления информационных матриц
Фишера $M(U;\Theta)$ для модели~(\ref{e10-c}), (\ref{e11-c}).

  В~разд.~4 были рассмотрены вопросы, связанные с оцениванием
неизвестных параметров мо\-делей стохастических нелинейных дискретных
сис\-тем. Качество оценивания параметров можно\linebreak повысить за счет построения
плана эксперимента, оптимизирующего некоторый выпуклый функционал~$X$
от информационной матрицы $M(\xi)$, решив экстремальную задачу
  \begin{equation}
  \xi^* =\mathrm{arg}\,\min_{\xi\in\Omega_\xi} X[M(\xi)]\,.
  \label{e18-c}
  \end{equation}
Для критерия $D$-оп\-ти\-маль\-ности $X[M(\xi)] =$\linebreak
$=-\ln\,\mathrm{det}\,M(\xi)$,
для критерия $A$-оп\-ти\-маль\-ности $X[M(\xi)]\hm =-Sp M^{-1}(\xi)$.

  Планирование эксперимента определенным образом воздействует на
нижнюю границу неравенства Рао--Кра\-ме\-ра~\cite{20-c}: для
  $D$-оп\-ти\-маль\-но\-го плана минимизируется объем, для
  $A$-оп\-ти\-маль\-ного плана~--- сумма квадратов длин осей эллипсоида
рассеяния оценок параметров.

\vspace*{-6pt}

\section{Алгоритмы численного построения оптимальных планов}

\vspace*{-3pt}

  В соответствии с~\cite{21-c} экстремальная задача поиска минимума
$X[M(\xi)]$ может быть сведена к конечномерной задаче с размерностью
пространства варьируемых переменных не более чем $(Nr+1)(s(s+1)/2+1)$. Ее
решение можно осуществить с помощью общих методов численного поиска
экстремума. При этом возможны два подхода. 
Первый из них (прямой)
предполагает поиск минимума функционала $X[M(\xi)]$ в про\-стран\-ст\-ве
элементов информационной матрицы при ограни-\linebreak чениях $M\in\Omega_M$, где
$\Omega_M=\{M(\xi)\vert \xi\in\Omega_\xi\}$~--- мно-\linebreak жество информационных
матриц. Характерной осо\-бен\-ностью этого подхода является большая
размерность экстремальной задачи. Поскольку $X[M(\xi)]$~--- выпуклый
функционал, здесь имеет мес\-то задача выпуклого программирования, для
решения которой предлагается следующий алгоритм.

\vspace*{-4pt}

\subsection{Прямая градиентная процедура построения непрерывных
оптимальных планов}

\smallskip

\textbf{Шаг~1.} Зададим начальный невырожденный план:

\noindent
\begin{multline*}
\xi_0=\left \{
\begin{matrix}
U_1^0, U_2^0, \ldots , U_q^0,\\[3pt]
p_1^0, p_2^0, \ldots , p_q^0
\end{matrix}
\right \}\,,\ U_i^0\in \Omega_U\,,\
p_i^0=\fr{1}{q}\,,\\ i=1, 2, \ldots ,q\,,
\end{multline*}
в котором $q=s(s+1)/2+1$. Вычислим информационные матрицы $M(U_i^0)$
одноточечных планов для $i=1, 2, \ldots , q$ и по формуле~(\ref{e17-c})
информационную матрицу всего плана~$\xi_0$. Положим $l=0$.

\smallskip

\textbf{Шаг~2.} Считая веса $p_1^l, p_2^l, \ldots , p_q^l$ фиксированными, решим
задачу
$$
X[M(\xi_1)]\rightarrow \min_{U_1^l, \ldots , U_q^l}\,, U_i^l\in \Omega_U\,,\
i=1, 2, \ldots , q\,,
$$
методом проекции градиента:
$$
\tilde{U}^{l+1}=\pi_{\Omega_{\tilde{U}}}\left\{ \tilde{U}^l-\rho_l^\prime
\nabla_{\tilde{U}} X[M(\xi_l)]\right \}\,,
$$


\noindent
где $\tilde{U^{\mathrm{T}}}=(U_1^{\mathrm{T}}, U_2^{\mathrm{T}}, \ldots , U_q^{\mathrm{T}})$;
$\pi_{\Omega_{\tilde{U}}}\{\cdot\}$~--- проекция точ\-ки\footnote{Для сигналов,
ограниченных по мощности или по амплитуде, когда $\Omega_{\tilde{U}}$~---
соответственно координатный шар или параллелепипед, известны явные выражения для
проекции~\cite{18-c}.} на множество $\Omega_{\tilde{U}}$; $\rho_l^\prime \geq
0$~--- длина шага.

  Далее составим план:
  $$
  \tilde{\xi}_l=\left \{
  \begin{matrix}
  U_1^{l+1}, U_2^{l+1}, \ldots , U_q^{l+1}\\[3pt]
  p_1^l, p_2^l, \ldots , p_q^l
  \end{matrix}
  \right \}\,,
  $$
где $U_i^{l+1}$~--- точки, найденные на шаге~2. 

Вычислим $M(U_i^{l+1})$, $i=1, 2, \ldots ,q$.

  \smallskip

\textbf{Шаг~3.} Зафиксировав точки спектра полученного плана, решим задачу
\begin{multline*}
  X[M(\tilde{\xi}_l)] \rightarrow \min_{p_1^l, p_2^l, \ldots , p_q^l}\,\enskip
\sum\limits_{i=1}^q p_i^l=1\,,\ p_i^l\geq 0\,,\\ i=1, 2, \ldots ,q\,,
\end{multline*}
методом проекции градиента Розена~\cite{17-c}:
$$
\tilde{p}^{l+1} =\tilde{p}^l-\rho_l^{\prime\prime} P\nabla_{\tilde{p}}
X[M(\tilde{\xi}_l)]\,,
$$
где $\tilde{p}=(p_1, p_2, \ldots , p_q)$, $\rho_l^{\prime\prime}\geq 0$~--- длина
шага, $P$~--- мат\-ри\-ца оператора проектирования. Составим план:
$$
\xi_{l+1}=\left \{
\begin{matrix}
U_1^{l+1}, U_2^{l+1}, \ldots , U_q^{l+1}\\[3pt]
p_1^{l+1}, p_2^{l+1}, \ldots ,  p_q^{l+1}
\end{matrix}
\right \}\,.
$$

  \smallskip

\textbf{Шаг~4.} Если выполняется неравенство
  $$
  \sum\limits^q_{i=1}\left[ \Vert U_i^{l+1}-U_i^l\Vert^2+\left( p_i^{l+1}-
p_i^l\right)^2\right]\leq \delta\,,
  $$
где $\delta$~--- малое положительное число, перейдем к шагу~5. В~противном
случае для $l=l+1$ повторим шаги~2 и~3.

  \smallskip

\textbf{Шаг~5.} Проверим необходимое условие оптимальности плана:
  $$
  \left\vert \mu \left( U_i^{l+1},\xi_{l+1}\right) -\eta\right\vert \leq \delta\,,\enskip
i=1,2, \ldots , q\,.
  $$
Если оно выполняется, закончим процесс. В~противном случае повторим все
сначала, скорректировав начальное приближение~$\xi_0$.

  Значения параметров $X[M(\xi)]$, $\mu(U,\xi)$, $\eta$ прямой градиентной
процедуры для критериев $D$- и $A$-оп\-ти\-маль\-ности определяем по
табл.~1.

  \noindent
\begin{center} %tabl1
\parbox{80mm}{{\tablename~1}\ \ \small{Соответствие значений параметров $X[M(\xi)]$, $\mu(U,\xi)$ и $\eta$
  критериям оптимальности}}
%\begin{center} %fig1

\vspace*{2ex}
\tabcolsep=4.3pt
{\small
\begin{tabular}{|c|c|c|c|}
  \hline
\tabcolsep=0pt\begin{tabular}{c}Кри-\\ терий\end{tabular}&$X[M(\xi)]$ &$\mu(U,\xi)$ &$\eta$\\
\hline
&&&\\[-8pt]
$D$ &$-\ln\,\mathrm{det}\,M(\xi)$ &$Sp[M^{-1}(\xi)M(U)]$&$s$\\
$A$ &$SpM^{-1}(\xi)$ &$Sp[M^{-2}(\xi)M(U)]$ &$SpM^{-1}(\xi)$ \\
\hline
\end{tabular}
}
\end{center}
%\vspace*{12pt}
%\begin{center}
%\end{center}
\vspace*{9pt}

\smallskip
\addtocounter{table}{1}


Приведенный алгоритм требует вычисления градиентов
\begin{align*}
\nabla_{\tilde{U}} X[M(\xi)]&=\left\Vert \fr{\partial X[M(\xi)]}{\partial
u_j^{(i)}(t)}\right\Vert\,,\enskip i=1, \ldots ,q\,,\\
&\hspace*{12mm}t=0, \ldots , N-1\,,\enskip j=1, \ldots ,r\,;\\
\nabla_{\tilde{p}} X[M(\xi)] &=\left\Vert \fr{\partial X [M(\xi)]}{\partial
p_i}\right\Vert\,,\enskip i=1, \ldots ,q\,.
\end{align*}

  Начнем с критерия $D$-оп\-ти\-маль\-ности. Для него получаем
  \begin{multline*}
  \fr{\partial X[M(\xi)]}{\partial u_j^{(i)}(t)}=\fr{\partial [-
\ln\,\mathrm{det}\,M(\xi)]}{\partial u_j^{(i)}(t)}={}\\
{}=
  -Sp\left[ M^{-1}(\xi)\fr{\partial M(\xi)}{\partial u_j^{(i)}(t)}\right] ={}\\
  {}=-p_i
Sp\left[M^{-1}(\xi)\fr{\partial M(U_i;\theta)}{\partial u_j^{(i)}(t)}\right]\,,\\
  i=1, \ldots ,q\,,\ t=0, \ldots , N-1\,,\ j=1, \ldots ,r\,.
  \end{multline*}
Для вычисления производных
$$
\fr{\partial M(U;\theta)}{\partial u_j(t)}=
\left\Vert \fr{\partial M_{\alpha\beta}(U;\theta)}{\partial u_j(t)}\right\Vert\,,\enskip
\alpha, \beta=1, \ldots , s\,,
$$
воспользуемся тем, что ИМФ можно представить в виде суммы двух слагаемых,
одно из которых зависит от входного сигнала, а другое~--- нет (данный факт
вытекает из материалов~\cite{19-c}):
$$
M_{\alpha\beta}(U;\Theta)=W_{\alpha\beta}(U;\Theta)+V_{\alpha\beta}(\Theta)\,.
$$

\noindent
Здесь

\end{multicols}

\hrule


\begin{multline*}
W_{\alpha\beta}(U;\Theta)=
\sum\limits_{k=0}^{N-1}\left \{
Sp\left [
\fr{\partial H(k+1)}{\partial
\theta_\alpha}\,C_0\overline{x}_A(k+1)\overline{x}_A^{\mathrm{T}}(k+1)C_0^{\mathrm{T}}\fr{\partial
H^{\mathrm{T}}(k+1)}{\partial \theta_\beta}\,B^{-1}(k+1)\right ]+{}\right.\\
{}+Sp\left[\fr{\partial H(k+1)}{\partial\theta_\alpha}\,C_0\overline{x}_A(k+1)
\overline{x}_A^{\mathrm{T}}(k+1)C_{\beta}^{\mathrm{T}} H^{\mathrm{T}}(k+1) B^{-1}(k+1)\right]+{}
\end{multline*}

\noindent
\begin{multline*}
{}+ Sp\left[\fr{\partial
H(k+1)}{\partial\theta_\alpha}\,C_0\overline{x}_A(k+1)\fr{\partial
A^{\mathrm{T}}(k+1)}{\partial\theta_\beta} B^{-1}(k+1)\right]+{}\\
{}+
Sp\left[H(k+1)C_\alpha\overline{x}_A(k+1)\overline{x}_A^{\mathrm{T}}(k+1)C_0^{\mathrm{T}}\fr{\partial
H^{\mathrm{T}}(k+1)}{\partial\theta_\beta} B^{-1}(k+1)\right]+{}\\
{}+Sp\left[ H(k+1)C_\alpha\overline{x}_A(k+1)\overline{x}_A^{\mathrm{T}}(k+1) C_\beta^{\mathrm{T}}
H^{\mathrm{T}}(k+1) B^{-1}(k+1)\right]+{}\\
{}+Sp\left[ H(k+1)C_\alpha \overline{x}_A(k+1)\fr{\partial
A^{\mathrm{T}}(k+1)}{\partial\theta_\beta}\,B^{-1}(k+1)\right]+{}\\
{}+Sp\left[ \fr{\partial
A(k+1)}{\partial\theta_\alpha}\,\overline{x}_A^{\mathrm{T}}(k+1)C_0^{\mathrm{T}}\fr{\partial
H^{\mathrm{T}}(k+1)}{\partial \theta_\beta}\,B^{-1}(k+1)\right]+{}\\
\left.{}+ Sp\left[ \fr{\partial
A(k+1)}{\partial\theta_\alpha}\,\overline{x}_A^{\mathrm{T}}(k+1)C_\beta^{\mathrm{T}} H^{\mathrm{T}} (k+1) B^{-1}(k+1)
\right]\right\}\,,\ \alpha,\beta=1, 2, \ldots , s\,.
\end{multline*}
 В соответствии с указанным разложением получаем, что
  \begin{multline}
  \fr{\partial M_{\alpha\beta}(U;\Theta)}{\partial u_j(t)}=\fr{\partial
W_{\alpha\beta}(U;\Theta)}{\partial u_j(t)}={}\\
  {}=\sum\limits_{k=0}^{N-1}\!\left \{\!Sp\left[\fr{\partial
H(k+1)}{\partial\theta_\alpha}\,C_0\left(\fr{\partial\overline{x}_A(k+1)}{\partial
u_j(t)}\,\overline{x}_A^{\mathrm{T}}(k+1)+\overline{x}_A(k+1)\fr{\partial\overline{x}_A^{\mathrm{T}}(k
+1)}{\partial u_j(t)}\right) C_0^{\mathrm{T}}\fr{\partial H^{\mathrm{T}}(k+1)}{\partial\theta_\beta}\,B^{-
1}(k+1)\right]+{}\right.\\
  {}+Sp\left[\fr{\partial
H(k+1)}{\partial\theta_\alpha}\,C_0\left(\fr{\partial\overline{x}_A(k+1)}{\partial
u_j(t)}\,\overline{x}_A^{\mathrm{T}}(k+1)+\overline{x}_A(k+1\vert
k)\fr{\partial\overline{x}_A^{\mathrm{T}}(k+1)}{\partial u_j(t)}\right)C_\beta^{\mathrm{T}} H^{\mathrm{T}}(k+1)B^{-
1}(k+1)\right]+{}\\
  {}+Sp\left[\fr{\partial
H(k+1)}{\partial\theta_\alpha}\,C_0\fr{\partial\overline{x}_A(k+1)}{\partial
u_j(t)}\,\fr{\partial A^{\mathrm{T}}(k+1)}{\partial\theta_\beta}\,B^{-1}(k+1)\right]+{}\\
  {}+ Sp\left[ H(k+1) C_\alpha\left( \fr{\partial\overline{x}_A(k+1)}{\partial
u_j(t)}\,\overline{x}_A^{\mathrm{T}}(k+1)+\overline{x}_A(k+1)\fr{\partial
\overline{x}^{\mathrm{T}}_A(k+1)}{\partial u_j(t)}\right) C_0^{\mathrm{T}}\fr{\partial
H^{\mathrm{T}}(k+1)}{\partial\theta_\beta}\,B^{-1}(k+1)\right]+{}\\
  {}+ Sp\left[ H(k+1)C_\alpha \left( \fr{\partial \overline{x}_A(k+1)}{\partial
u_j(t)}\,\overline{x}_A^{\mathrm{T}}(k+1)+\overline{x}_A(k+1)\fr{\partial\overline{x}_A^{\mathrm{T}}(k
+1)}{\partial u_j(t)}\right)C_\beta^{\mathrm{T}} H^{\mathrm{T}}(k+1) B^{-1}(k+1)\right]+{}\\
  {}+ Sp\left[ H(k+1) C_\alpha \fr{\partial\overline{x}_A(k+1)}{\partial
u_j(t)}\,\fr{\partial A^{\mathrm{T}}(k+1)}{\partial\theta_\beta}\,B^{-1}(k+1)\right]+{}\\
  {}+ Sp\left[ \fr{\partial
A(k+1)}{\partial\theta_\alpha}\,\fr{\partial\overline{x}_A^{\mathrm{T}}(k+1)}{\partial
u_j(t)}\,C_0^{\mathrm{T}}\fr{\partial H^{\mathrm{T}}(k+1)}{\partial\theta_\beta}\,B^{-1}(k+1)\right]+{}\\
  \left.{}+Sp\left[ \fr{\partial
A(k+1)}{\partial\theta_\alpha}\,\fr{\partial\overline{x}_A^{\mathrm{T}}(k+1)}{\partial
u_j(t)}\,C_\beta^{\mathrm{T}} H^{\mathrm{T}}(k+1)B^{-1}(k+1)\right]\right\}\,.
  \label{e19-c}
  \end{multline}
  

  Алгоритм вычисления производных от ИМФ по компонентам входного
сигнала $\partial M(U;\Theta)/\partial u_j(t)$ для заданных значений $U,\Theta$
при фиксированных $j,t$ может быть следующим\footnote{Приводится здесь
впервые. Разработан на основе материалов~\cite{19-c}.}:

\textbf{Шаг~1.} Задать $Q$, $R$, $\overline{x}(0)$, $P(0)$,
$\{\partial\overline{x}(0)/\partial\theta_i$, $i\hm=1, 2, \ldots , s\}$.

\textbf{Шаг~2.} Положить $\partial M(U;\Theta)/\partial u_j(t)=0$; $k=0$;
$x_H(k)=\overline{x}(0)$; $\partial
x_H(k)/\partial\theta_i=\partial\overline{x}(0)/\partial\theta_i$, $i=1, 2, \ldots , s$;
$P(k\vert k)=P(0)$.

%\columnbreak

\textbf{Шаг~3.} Определив $u_H(k)$, найти $a(k)$ по формуле~(\ref{e6-c}) и
$\{\partial a(k)/\partial\theta_i, \ i=1, 2, \ldots ,s\}$ по формуле:
  \begin{multline*}
  \fr{\partial a(k)}{\partial\theta_i}=\fr{\partial
f[x_H(k),u_H(k),k]}{\partial\theta_i}-
\fr{\partial^2 f[x_H(k),
u_H(k),k]}{\partial\theta_i\partial x(k)}\,x_H(k)
-\fr{\partial f[x_H(k), u_H(k),k]}{\partial x(k)}\,\fr{\partial
x_H(k)}{\partial\theta_i}+{}\\
{}+\fr{\partial^2 f[x_H(k), u_H(k), k]}{\partial\theta_i \partial
u(k)}\left[ u(k)-u_H(k)\right]\,.
  \end{multline*}
  
 % \pagebreak
При помощи выражения~(\ref{e7-c}) получить $\Phi(k)$ и
$\{\partial\Phi(k)/\partial\theta_i,\ i=1, 2, \ldots ,s\}$.

\textbf{Шаг~4.} Если $k=0$, вычислить:
  $$
  \overline{x}_A(k+1)=
  \begin{bmatrix}
  \Phi(0)\overline{x}(0)+a(0)\\[3pt]
  \fr{\partial\Phi(0)}{\partial\theta_1}\,\overline{x}(0)+\Phi(0)\fr{\partial\overline{x}(
0)}{\partial\theta_1}+\fr{\partial a(0)}{\partial\theta_1}\\[3pt]
  \ldots\\[3pt]
  \fr{\partial\Phi(0)}{\partial\theta_s}\,\overline{x}(0)+\Phi(0)\fr{\partial\overline{x
}(0)}{\partial\theta_s}+\fr{\partial a(0)}{\partial\theta_s}
  \end{bmatrix}
  $$
и перейти к шагу~8.

\textbf{Шаг~5.} Найти $\tilde{K}(k)$ по формуле:
  $$
  \tilde{K}(k)=\Phi(k) K(k)\,.
  $$

\textbf{Шаг~6.} Сформировать матрицы $\Phi_A(k)$, $a_A(k)$ в соответствии с
равенствами:
  $$
  \Phi_A(k) =
  \begin{bmatrix}
  \Phi(k) & 0 & \ldots & 0\\[3pt]
  \fr{\partial\Phi(k)}{\partial\theta_1} -\tilde{K}(k)\fr{\partial
H(k)}{\partial\theta_1} & \Phi(k)-\tilde{K}(k)H(k) &\ldots & 0\\[3pt]
  \ldots &\ldots &\ldots &\ldots\\[3pt]
  \fr{\partial\Phi(k)}{\partial\theta_s}-\tilde{K}(k)\fr{\partial H(k)}{\partial
\theta_s} & 0 &\ldots & \Phi(k)-\tilde{K}(k)H(k)
  \end{bmatrix}\,;
  $$
  $$
  a_A(k)=
  \begin{bmatrix}
  a(k)\\[3pt]
  \fr{\partial a(k)}{\partial\theta_1}-\tilde{K}(k)\fr{\partial A(k)}{\partial\theta_1}\\[3pt]
  \ldots\\[3pt]
  \fr{\partial a(k)}{\partial\theta_s}-\tilde{K}(k)\fr{\partial A(k)}{\partial\theta_s}
  \end{bmatrix}\,.
  $$
  
  \vspace*{6pt}
  
    \hrule
  
  \smallskip

  \begin{multicols}{2}
  
 

  \textbf{Шаг~7.} Вычислить $\overline{x}_A(k+1)$ по формуле:
  $$
  \overline{x}_A(k+1)=\Phi_A(k)\overline{x}_A(k)+a_A(k)\,.
  $$

\textbf{Шаг 8.} Найти $x_H(k+1)$ и $\{\partial x_H(k+1)/\partial\theta_i$,\ $i=1, 2, \ldots ,
s\}$ при помощи выражения~(\ref{e3-c}).
Вы\-чис\-лить $\{\partial A(k+1)/\partial\theta_i, \ i=1, 2, \ldots ,s\}$,
воспользовавшись равенством~(\ref{e8-c}). Определить $H(k+1)$ по
формуле~(\ref{e9-c}) и $\{\partial H(k+1)/\partial\theta_i,\ i=1, 2, \ldots , s\}$.

\smallskip
\textbf{Шаг~9.} Сформировать матрицу~$\Gamma(k)$ и найти $P(k+1\vert k)$,
$B(k+1)$, $K(k+1)$, $P(k+1\vert k+1)$, используя
  выражения~(\ref{e12-c})--(\ref{e15-c}).

\smallskip
\textbf{Шаг~10.} Определить $\partial a(k)/\partial u_j(t)$ и $\{\partial^2
a(k)/(\partial\theta_i \partial u_j(t))$, $i=1, 2, \ldots ,s\}$ по формулам:
  \begin{align*}
  \fr{\partial a(k)}{\partial u_j(t)} &=\fr{\partial f [x_H(k),u_H(k),k]}{\partial
u(k)}\,\fr{\partial u(k)}{\partial u_j(t)}\,;\\[6pt]
  \fr{\partial^2 a(k)}{\partial\theta_i\partial u_j(t)} &=\fr{\partial^2 f[x_H(k),
u_H(k),k]}{\partial\theta_i \partial u(k)}\,\fr{\partial u(k)}{\partial u_j(t)}\,.
  \end{align*}
  \columnbreak

\textbf{Шаг 11.} Если $k=0$, вычислить $\partial\overline{x}_A(k+1)/\partial u_j(t)$ по
  формуле:
    $$
  \fr{\partial\overline{x}_A(k+1)}{\partial u_j(t)} =
  \begin{cases}
    \begin{bmatrix}
  \fr{\partial a(0)}{\partial u_j(t)}\\[3pt]
  \fr{\partial^2 a(0)}{\partial\theta_1 \partial u_j(t)}\\[3pt]
  \ldots\\[3pt]
  \fr{\partial^2 a(0)}{\partial\theta_s\partial u_j(t)}
  \end{bmatrix}\,, & \mbox{если}\ t=0\,;\\[6pt]
  0\,, & \mbox{если}\ t\not=0
  \end{cases}
  $$
и перейти к шагу~14.

\textbf{Шаг~12.} Сформировать вектор $\partial a_A(k)/\partial u_j(t)$ в соответствии с
равенством:
  $$
  \fr{\partial a_A(k)}{\partial u_j(t)}=
  \begin{bmatrix}
  \fr{\partial a(k)}{\partial u_j(t)}\\[3pt]
  \fr{\partial^2 a(k)}{\partial\theta_1 \partial u_j(t)}
  \\[3pt]
  \ldots\\[3pt]
  \fr{\partial^2 a(k)}{\partial\theta_s \partial u_j(t)}
  \end{bmatrix}\,.
  $$
  
  \pagebreak

\textbf{Шаг~13.} Вычислить $\partial\overline{x}_A(k+1)/\partial u_j(t)$ по формуле:
  $$
  \fr{\partial\overline{x}_A(k+1)}{\partial u_j(t)}=
  \begin{cases}
  \Phi_A(k)\fr{\partial\overline{x}_A(k)}{\partial u_j(t)}+\fr{\partial
a_A(k)}{\partial u_j(t)}\,, & t\leq k\,;\\
  0\,, &\hspace*{-30mm} \mbox{в противном случае}\,.
  \end{cases}
  $$

\smallskip
\textbf{Шаг~14.} Используя выражение~(\ref{e19-c}), получить приращение $\Delta
(\partial M(U;\Theta)/\partial u_j(t))$, отвечающее текущему значению~$k$.

\smallskip
\textbf{Шаг 15.} Положить
  $$
  \fr{\partial M(U;\Theta)}{\partial u_j(t)} =\fr{\partial M(U;\Theta)}{\partial
u_j(t)}+\Delta\fr{\partial M(U;\Theta)}{\partial u_j(t)}\,.
  $$

\smallskip
\textbf{Шаг~16.} Увеличить $k$ на единицу. Если $k\leq N-1$, перейти к шагу~3.
В~противном случае закончить процесс.

  \bigskip

  Получить выражение для градиента по весам не составляет особого труда,
поскольку
  \begin{multline*}
  \fr{\partial X[M(\xi)]}{\partial p_i}=\fr{\partial [-\ln\,\mathrm{det}\,M(\xi)]}{\partial p_i}={}\\
{}= -Sp\left[ M^{-1}(\xi)\fr{\partial
M(\xi)}{\partial p_i}\right]={}\\
  {}=-Sp\left[ M^{-1}(\xi) M(U_i;\theta)\right]\,,\ i=1,\ldots ,q\,.
  \end{multline*}

  Перейдем к критерию $A$-оп\-ти\-маль\-ности. В~этом случае
  \begin{multline*}
  \fr{\partial X[M(\xi)]}{\partial u_j^{(i)}(t)}=\fr{\partial [Sp M^{-1}(\xi)]}{\partial
u_j^{(i)}(t)}=
Sp\left[\fr{\partial M^{-1}(\xi)}{\partial u_j^{(i)}(t)}\right]={}\\
  {}=-Sp \left [ M^{-1}(\xi)\fr{\partial M(\xi)}{\partial u_j^{(i)}(t)}\,M^{-1}(\xi)\right]={}\\
  {}= -p_i Sp\left[ M^{-1}(\xi)\fr{\partial M(U_i;\theta)}{\partial
u_j^{(i)}(t)}\, M^{-1}(\xi)\right]={}\\
  {}= -p_i Sp \left [ M^{-2}(\xi)\fr{\partial M(U_i;\theta)}{\partial
u_j^{(i)}(t)}\right ]\,,\\
  i=1, \ldots , q\,, \ t=0, \ldots , N-1\,,\ j=1, \ldots ,r\,;
  \end{multline*}
  
  \vspace*{-6pt}

\noindent
\begin{multline*}
  \fr{\partial X[M(\xi)]}{\partial p_i}=\fr{\partial Sp M^{-1}(\xi)]}{\partial
p_i}=Sp\left[\fr{\partial M^{-1}(\xi)}{\partial p_i}\right]={}\\
{}=-Sp \left [ M^{-1}(\xi)\fr{\partial M(\xi)}{\partial p_i}\,M^{-1}(\xi)\right]={}\\
{}= -Sp\left[ M^{-1}(\xi) M(U_i;\theta) M^{-1}(\xi)\right]={}\\
{}=
-Sp\left[ M^{-2}(\xi) M(U_i;\theta)\right]\,, \enskip i=1, 2, \ldots , q\,.
\end{multline*}

  Другой подход (его называют двойственным) к решению оптимизационной
задачи~(\ref{e18-c}) основан на теореме эквивалентности из~\cite{21-c} и
заключается в минимизации $X[M(\xi)]$ по набору аргументов
$\{U_i,p_i\}_{i=1}^q$ при ограничениях $U_i\in\Omega_U$, $p_i\geq 0$,
$\sum\limits_{i=1}^q p_i=1$. В~этом случае рассматриваемая задача уже не
является задачей выпуклого программирования, но размерность вектора
варьируемых параметров может оказаться значительно меньше, чем при
прямом подходе.

\subsection{Двойственная градиентная процедура построения
непрерывных оптимальных планов$^1$}

\renewcommand{\thefootnote}{\arabic{footnote}}
\footnotetext[1]{Соответствие значений параметров $X[M(\xi)]$, $\mu(U,\xi)$, $\eta$
двойственной процедуры критериям $A$- и $D$-оп\-ти\-маль\-ности также см.\ по табл.~1.}

\smallskip\textbf{Шаг~1.} Зададим начальный невырожденный план~$\xi_0$ и по
формуле~(\ref{e17-c}) вычислим нормированную матрицу $M(\xi_0)$ плана.
Положим $l=0$.

\smallskip\textbf{Шаг~2.} Найдем локальный максимум
  $$
  U^l=\mathrm{arg}\,\max_{U\in \Omega_U} \mu\left ( U,\xi_l\right)
  $$
методом проекции градиента. Если окажется, что
$\left\vert \mu\left( U^l,\xi_l\right)-\eta\right\vert \leq\delta$,
закончим процесс.
  Если
$
  \mu\left( U^l,\xi_l\right)>\eta$, перейдем к шагу~3. В~противном случае будем искать новый локальный
максимум.

\smallskip\textbf{Шаг~3.} Вычислим $\tau_l$ по формуле
  \begin{equation*}
  \tau_l =\mathrm{arg}\,\min_{0\leq \tau\leq 1} X\left[
M\left(\xi^\tau_{l+1}\right)\right]\,.
\end{equation*}
Здесь
\begin{equation*}
  \xi^\tau_{l+1} =\left(1-\tau\right)\xi_l+\tau\xi\left(U^l\right)\,,
  \end{equation*}
где $\xi(U^l)$~--- одноточечный план, размещенный в точке~$U^l$.

\smallskip\textbf{Шаг~4.} Составим план
  $$
  \xi_{l+1}=\left (1-\tau_l\right)\xi_l+\tau_l\xi\left(U^l\right)\,,
  $$
произведем его <<очистку>> в соответствии с рекомендациями из~\cite{4-c},
положим $l=l+1$ и перейдем к шагу~2.

  Приведенный алгоритм требует вычисления градиента
  \begin{multline*}
  \nabla_U \mu\left(U,\xi\right) =\left\Vert \fr{\partial\mu(U,\xi)}{\partial
u_j(t)}\right\Vert\,,\enskip t=0,1, \ldots ,N-1\,,\\  j=1,2, \ldots ,r\,.
  \end{multline*}

  Для критерия $D$-опти\-маль\-ности получаем:
  \begin{multline*}
  \fr{\partial\mu(U,\xi)}{\partial u_j(t)} =\fr{\partial
  Sp[M^{-1}(\xi)M(U)]}{\partial u_j(t)}= {}\\[6pt]
  {}=Sp\left[ M^{-1}(\xi)\fr{\partial
M(U)}{\partial u_j(t)}\right]\,, \enskip t=0,1, \ldots ,N-1\,, \\[6pt]
j=1, \ldots ,r\,.
  \end{multline*}

  В случае критерия $A$-опти\-маль\-ности
    \begin{multline*}
  \fr{\partial\mu(U,\xi)}{\partial u_j(t)} =\fr{\partial
  Sp[M^{-2}(\xi)M(U)]}{\partial u_j(t)}={}\\[6pt]
  {}=Sp\left[ M^{-2}(\xi)\fr{\partial
M(U)}{\partial u_j(t)}\right]\,, \enskip t=0,1, \ldots ,N-1\,, \\ j=1, \ldots ,r\,.
    \end{multline*}

   Итак, рассмотрены две принципиально разные градиентные процедуры
построения непрерыв-\linebreak ных оптимальных планов. До решения конкретной задачи
не представляется возможным судить, эффективность какой из них окажется
выше. Проведенный анализ позволил организовать вычисление градиентов
функционалов $X[M(\xi)]$ и $\mu[U,\xi]$ по рекуррентным аналитическим
формулам. Основу алгоритмов вычисления указанных градиентов составили
алгоритмы нахождения информационной матрицы одноточечного плана
$M(U;\Theta)$ из~\cite{19-c} и ее производных $\partial M(U;\Theta)/\partial
u_j(t)$.

  Практическое применение построенного непрерывного оптимального плана
\begin{multline*}
  \xi^*=\left\{
  \begin{matrix}
  U_1^*, U_2^*, \ldots , U_q^*\\[6pt]
  p_1^*, p_2^*, \ldots , p_q^*
  \end{matrix}
  \right \}\,,\enskip \sum\limits_{i=1}^q p_i^*=1\,,\ p_i^*\geq 0\,,\\ U_i^*\in
\Omega_U\,,\ i=1, 2, \ldots , q\,,
  \end{multline*}
затруднено тем обстоятельством, что веса~$p_i^*$ представляют собой, вообще
говоря, произвольные вещественные числа, заключенные в интервале от нуля
до единицы. Возможный алгоритм <<округления>> непрерывного плана до
дискретного изложен в~\cite{6-c}.

     Разработанный в рамках системы MATLAB программный комплекс
включает в себя модули, отвечающие за вычисление информационной матрицы
и ее производных по компонентам входного сигнала, нахождение
     ММП-оце\-нок неизвестных параметров моделей гауссовских
нелинейных дискретных систем, синтез $A$- и $D$-оп\-ти\-маль\-ных входных
сигналов (реализованы прямая и двойственная градиентные процедуры).

\section{Пример активной идентификации динамической~системы}

  Рассмотрим следующую модель стохастической нелинейной дискретной
системы:
  \begin{align}
  x(k+1) &=\left(1-\fr{\theta_2}{\theta_1}\right) x(k)+{}\notag\\
&\hspace*{-15mm}  {}+\fr{0{,}01}{\theta_1}\left[
u(k)-x(k)\right] e^{0{,}25[u(k)-x(k)]}+\fr{0{,}1}{\theta_1}\,w(k)\,;\label{e20-c}\\
  y(k+1) &= x(k+1)+v(k+1)\,, \notag\\
  &\hspace*{18mm}k=0,1,\ldots , N-1\,,
  \label{e21-c}
  \end{align}
где $\theta_1$, $\theta_2$~--- неизвестные параметры системы, причем $2\leq
\theta_1\leq 10$, $0{,}05\leq \theta_2\leq 2$.

  Будем считать, что выполнены все априорные предположения из разд.~2,
причем
  \begin{align*}
  E\left[ w(k) w(i)\right] & = 0{,}6\delta_{ki}=Q\delta_{ki}\,;\\
  E\left[ v(k+1) v(i+1)\right] &=0{,}3\delta_{ki}=R\delta_{ki}\,;\\
  E\left[x(0)\right] &=0=\overline{x}(0)\,;\\
  E\left\{\left[ x(0)-\overline{x}(0)\right]\left[ x(0)-
\overline{x}(0)\right]\right\}&=0{,}01=P(0)\,.
  \end{align*}

  Выполнив временную линеаризацию модели~(\ref{e20-c}), (\ref{e21-c})
относительно номинальной траектории
\begin{equation}
\left.
\begin{array}{rl}
x_H(k+1)  &=\left(1-\fr{\theta_2}{\theta_1}\right) x_H(k)+{}\\[6pt]
&\hspace*{-15mm}{}+\fr{0{,}01}{\theta_1}\left[
u_H(k)-x_H(k)\right] e^{0{,}25[u_H(k)-x_H(k)]}\,,\\[6pt] 
&\hspace*{15mm}k=0,1, \ldots ,N-1\,;\\[9pt]
x_H(0)&=0\,,
\end{array}
\right \}
\label{e22-c}
\end{equation}
получим линеаризованную модель вида~(\ref{e10-c}), (\ref{e11-c}), в которой
\begin{multline*}
a(k)=\fr{0{,}01}{\theta_1}\left \{
\vphantom{\left[ u_H(k)-x_H(k)\right]^2}
\vphantom{[x_H(k)]^2}
 \left[ 1+0{,}25(u_H(k)-x_H(k))\right] u(k)-
{}\right.\\
\left.{}-0{,}25\left[ u_H(k)-x_H(k)\right]^2\right \} e^{0{,}25 [u_H(k)-x_H(k)]}\,;
\end{multline*}

\vspace*{-12pt}

\noindent
\begin{multline*}
\Phi(k)=1-\fr{\theta_2}{\theta_1}-\fr{0{,}01}{\theta_1}\left\{1+0{,}25\left[u_H(k)-{}\right.\right.\\
\left.\left.{}-x_H(k)\right]\right\} e^{0{,}25[u_H(k)-x_H(k)]}\,;
\end{multline*}

\vspace*{-6pt}

\noindent
\begin{equation*}
\Gamma(k) = \fr{0{,}1}{\theta_1}\,; \enskip A(k+1)= 0\,;\enskip
H(k+1)=1\,.
\end{equation*}

  Таким образом, необходимо оценить па\-ра\-мет\-ры~$\theta_1$, $\theta_2$,
входящие в~$a(k)$, $\Phi(k)$ и~$\Gamma(k)$.

  Считаем, что для номинальной траектории~(\ref{e22-c}) $u_H(k)=u(k)$, $k=0,
1, \ldots ,N-1$, обеспечивается наилучшее приближение построенной
линеаризованной модели к своему нелинейному аналогу.

  Выберем область планирования
  $
  \Omega_U=\{U\hm \in R^N\vert 10\leq u(k)\leq 15\,, k=0,1, \ldots ,N-1\}$ и критерий
$A$-оп\-ти\-маль\-ности.

  Для того чтобы ослабить зависимость результатов оценивания от
выборочных данных, произведем пять независимых запусков системы и
усредним полученные оценки неизвестных параметров. Реализации выходных
сигналов получим компьютерным моделированием при истинных значениях
параметров $\theta_1^*=4$, $\theta_2^*=0{,}5$ и $N=31$.

  О качестве идентификации в пространстве параметров и в пространстве
откликов будем судить соответственно по значениям
коэффициентов~$k_\theta$ и~$k_Y$, вычисляющихся по следующим
формулам:

\noindent
  \begin{align*}
  k_\theta &= \fr{\Vert \theta^* -\hat{\theta}_{\mathrm{ср}}\Vert}{\Vert \theta^* -
\hat{\theta}^*_{\mathrm{ср}}\Vert } ={}\\[3pt]
&\hspace*{12mm}{}=\sqrt{\fr{(\theta_1^*-
\hat{\theta}_{1\mathrm{ср}})^2+(\theta_2^*-
\hat{\theta}_{2\mathrm{ср}})^2}{(\theta_1^*-
\hat{\theta}_{1\mathrm{ср}}^*)^2+(\theta_2^*-\hat{\theta}_{2\mathrm{ср}}^*)^2}}\,;\\[6pt]
  k_Y &=\fr{\Vert Y_{\mathrm{ср}}-\hat{Y}_{\mathrm{ср}}\Vert}{\Vert
Y_{\mathrm{ср}}-\hat{Y}^*_{\mathrm{ср}}\Vert }={}\\[3pt]
&\hspace*{-1mm}{}=\sqrt{\fr{\sum_{k=0}^{N-1}
(y_{\mathrm{ср}}(k+1)-\hat{y}_{\mathrm{ср}}(k+1\vert k+1))^2}{\sum_{k=0}^{N-1}
(y_{\mathrm{ср}}(k+1)-\hat{y}^*_{\mathrm{ср}}(k+1\vert k+1))^2}}\,,
  \end{align*}
  
  \end{multicols}
  
    \begin{table}[b]\small %tabl2
    \vspace*{-24pt}
  \begin{center}
  \Caption{Результат выполнения процедуры активной идентификации
  \label{t2-c}}
  \vspace*{2ex}

  \begin{tabular}{|p{90mm}|c|c|c|}
  \hline
\multicolumn{1}{|c|}{\raisebox{-6pt}[0pt][0pt]{Входной сигнал}}& 
\raisebox{-6pt}[0pt][0pt]{\tabcolsep=0pt\begin{tabular}{c}Номер запуска\\ системы\end{tabular}}&
\multicolumn{2}{c|}{\tabcolsep=0pt\begin{tabular}{c}Значения оценок\\ параметров\end{tabular}}\\
\cline{3-4}
&&&\\[-8pt]
&&$\hat{\theta}_1$&$\hat{\theta}_2$\\
\hline
\multicolumn{1}{|c|}{Исходный }& &&\\
%&&&\\[-9pt]
\hspace*{0pt}{\raisebox{-52mm}{
\epsfxsize=77.97mm
\epsfbox{cht-1.eps}
}}
&1&\,6,6782\,&0,4520\\[-135pt]
\cline{2-4}
&&&\\[-2pt]
&2&6,1915&0,4014\\[7pt]
\cline{2-4}
&&&\\[-2pt]
&3&3,2376&0,5220\\[7pt]
\cline{2-4}
&&&\\[-2pt]
&4&3,8666&0,5807\\[7pt]
\cline{2-4}
&&&\\[-2pt]
&5&4,7786&0,5493\\[7pt]
\cline{2-4}
&\tabcolsep=0pt\begin{tabular}{c}Средние\\ значения\\ по запускам\end{tabular}&4,9505&0,5011\\
\hline
\multicolumn{1}{|c|}{Синтезированный }& &&\\
%&&&\\[-9pt]
\hspace*{0pt}{\raisebox{-52mm}{
\epsfxsize=77.97mm
\epsfbox{cht-2.eps}
}}&1&4,7958&0,4888\\[-135pt]
\cline{2-4}
&&&\\[-2pt]
&2&4,5351&0,4962\\[7pt]
\cline{2-4}
&&&\\[-2pt]
&3&3,6677&0,4967\\[7pt]
\cline{2-4}
&&&\\[-2pt]
&4&3,2482&0,4975\\[7pt]
\cline{2-4}
&&&\\[-2pt]
&5&4,8276&0,5340\\[7pt]
\cline{2-4}
&\tabcolsep=0pt\begin{tabular}{c}Средние\\ значения\\ по запускам\end{tabular}&4,2151&0,5026\\
\hline
\end{tabular}
\end{center}
\end{table}

\addtocounter{figure}{1}
\begin{figure}[b] %fig2
\vspace*{1pt}
\begin{center}
\mbox{%
\epsfxsize=163.409mm
\epsfbox{chu-2.eps}
}
\end{center}
\vspace*{-6pt}
\begin{minipage}[t]{79mm}
\Caption{Графическое представление $Y_{\mathrm{ср}}(k+1)$~(\textit{1}) и
$\hat{Y}_{\mathrm{ср}}(k+1|k+1)$~(\textit{2}) при $U$, изображенном на рис.~1
\label{f2-c}}
%\end{figure*}
\end{minipage}
\hfill
%\begin{figure*} %fig3
\vspace*{-6pt}
\begin{minipage}[t]{79mm}
  \Caption{Графическое представление  $Y_{\mathrm{ср}}(k+1)$~(\textit{1}) и  
  $\hat{Y}^*_{\mathrm{ср}}(k+1\vert k+1)$~(\textit{2})
при $U$, изображенном на рис.~1
  \label{f3-c}}
  \end{minipage}
  \end{figure}



\begin{multicols}{2}


\begin{center} %fig1
\vspace*{3pt}
\mbox{%
\epsfxsize=77.97mm
\epsfbox{chu-1.eps}
}
\end{center}
\vspace*{6pt}
%\begin{center}
{{\figurename~1}\ \ \small{Тестовый сигнал~$U$ для анализа
 качества прогнозирования на основе результатов из табл.~2}}
%\end{center}
%\vspace*{9pt}

\bigskip
\addtocounter{figure}{1}


\noindent
где $\theta^*$~--- вектор истинных значений параметров;
$\hat{\theta}_{\mathrm{ср}}$~--- вектор усредненных оценок неизвестных
зна-чений параметров по исходному входному сигналу;
$\hat{\theta}_{\mathrm{ср}}^*$~--- вектор усредненных оценок неизвестных
значений параметров по синтезированному входному сигналу;
$Y_{\mathrm{ср}}=\{y_{\mathrm{ср}}(k+1),\ k=0,1, \ldots , N-1\}$,
$\hat{Y}_{\mathrm{ср}} =\{y_{\mathrm{ср}}(k+1\vert k+1),\ k=0,1,\ldots , N-1\}$,
$\hat{Y}^*_{\mathrm{ср}}=\{\hat{y}_{\mathrm{ср}}^*(k+1|k+1),\ k=0,1, \ldots , N-1\}$~---
усредненные по всем запускам последовательности измерений для
вектора~$\theta$, равного~$\theta^*$, $\hat{\theta}_{\mathrm{ср}}$,
$\hat{\theta}_{\mathrm{ср}}^*$ соответственно, при некотором выбранном
допустимом входном сигнале $U\in \Omega_U$.

  Результаты выполнения процедуры активной параметрической
идентификации представлены в табл.~2 (оптимальный план получился
одноточечным).


  Данные, приведенные в табл.~2, показывают, что коэффициент
$k_\theta\approx 4{,}4$. В~пространстве откликов при псевдослучайном
входном сигнале~$U$, приведенном на рис.~1, $k_Y\approx 1{,}3$.
Последовательности $Y_{\mathrm{ср}}$,  $\hat{Y}_{\mathrm{ср}}$,
$\hat{Y}^*_{\mathrm{ср}}$ изображены на рис.~2 и~3.




\section{Заключение}

  Дано систематическое изложение наиболее существенных для практики
вопросов теории и техники активной параметрической идентификации
гауссовских нелинейных дискретных систем. Рассмотрен случай вхождения
неизвестных параметров в уравнения состояния и наблюдения, начальные
условия и ковариационные матрицы помех динамики и ошибок измерений.
Приведен разработанный алгоритм для вычисления производных от
информационной матрицы по компонентам входного сигнала, позволяющий
использовать при синтезе входного сигнала метод проектирования градиента и
в результате сократить время поиска оптимального плана эксперимента.
Приведены оригинальные градиентные алгоритмы активной идентификации,
позволяющие решать задачи оптимального оценивания параметров методом
максимального правдоподобия с привлечением прямой и двойственной
процедур синтеза $A$- и $D$-оп\-ти\-маль\-ных входных\linebreak сигналов. На примере
одной модельной струк\-ту\-ры продемонстрирована эффективность и
це\-ле\-со\-образ\-ность применения концепции активной па\-ра\-мет\-ри\-че\-ской
идентификации при построении\linebreak моделей нелинейных систем.

{\small\frenchspacing
{%\baselineskip=10.8pt
\addcontentsline{toc}{section}{Литература}
\begin{thebibliography}{99}

\bibitem{1-c}
\Au{Сейдж Э.\,П., Мелса Дж.\,Л.}
Идентификация систем управления.~--- М.: Наука, 1974.

\bibitem{2-c}
\Au{Эйкхофф П.}
Основы идентификации систем управ\-ле\-ния. Оценивание параметров и
состояния.~--- М.: Мир, 1975.

\bibitem{3-c}
\Au{Гроп Д.} Методы идентификации систем.~--- М.: Мир, 1979.

\bibitem{4-c}
\Au{Федоров В.\,В.}
Теория оптимального эксперимента (планирование регрессионных
экспериментов).~--- М.: Наука, 1971.

\bibitem{5-c}
\Au{Денисов В.\,И.}
Математическое обеспечение системы ЭВМ~--- экспериментатор.~--- М.:
Наука, 1977.

\bibitem{7-c}
\Au{Горский В.\,Г., Адлер Ю.\,П., Талалай~А.\,М.}
Планирование промышленных экспериментов (модели динамики).~--- М.:
Металлургия, 1978.

\bibitem{6-c}
\Au{Ермаков С.\,М., Жиглявский А.\,А.}
Математическая теория оптимального эксперимента.~--- М.: Наука, 1987.


\bibitem{8-c}
\Au{Денисов В.\,И., Чубич В.\,М., Черникова~О.\,С.}
Активная идентификация стохастических линейных дискретных систем во
временной области~// Сиб. журн. индустр. матем., 2003. Т.~6. №\,3(15).
С.~70--87.

\bibitem{9-c}
\Au{Денисов В.\,И., Чубич В.\,М., Черникова~О.\,С.}
Активная параметрическая идентификация стохастических линейных
дискретных систем в частотной области~// Сиб. журн. индустр. матем., 2007.
Т.~10. №\,1(29). С.~70--89.

\bibitem{10-c}
\Au{Денисов В.\,И., Чубич В.\,М., Черникова~О.\,С., Бобылева~Д.\,И.}
Активная параметрическая идентификация стохастических линейных
сис\-тем.~--- Новосибирск: НГТУ, 2009.

\bibitem{11-c}
\Au{Синицын И.\,Н.}
Рецензия на книгу В.\,И.~Денисова, В.\,М.~Чубича, О.\,С.~Черниковой,
Д.\,И.~Бобылевой <<Активная параметрическая идентификация
стохастических линейных систем>>~// Системы высокой доступности, 2009.
№\,3. С.~56.

\bibitem{12-c}
\Au{Казаков И.\,Е.}
Статистические методы проектирования систем управления.~--- М.:
Машиностроение, 1969.

\bibitem{13-c}
\Au{Пугачев В.\,С., Казаков И.\,Е., Евланов Л.\,Г.}
Основы статистической теории автоматических систем.~--- М.:
Машиностроение, 1974.

\bibitem{14-c}
\Au{Gupta N.\,K., Mehra R.\,K.}
Computational aspects of maximum likelihood estimation and reduction in sensitivity
function calculations~// IEEE Trans. Automat. Control, 1974. Vol.~19. No.\,6.
P.~774--783.

\bibitem{15-c}
\Au{$\acute{\mbox{А}}$str$\Ddot{\mbox{o}}$m K.\,J.}
Maximum likelihood and prediction errors methods~// Automatica, 1980,
Vol.~16. Р.~551--574.

\bibitem{16-c}
\Au{Огарков М.\,А.} Методы статистического оценивания параметров
случайных процессов.~--- М.: Энергоатомиздат, 1980.

\bibitem{17-c}
\Au{Базара М., Шетти К.}
Нелинейное программирование.~--- М.: Мир, 1982.

\bibitem{18-c}
\Au{Сухарев А.\,Г., Тимохов~В.\,В., Федоров~В.\,В.}
Курс методов оптимизации.~--- М.: Наука, 1986.

\bibitem{19-c}
\Au{Чубич В.\,М.}
Вычисление информационной матрицы Фишера в задаче активной
параметрической идентификации стохастических нелинейных дискретных
систем~// Науч. вест. НГТУ, 2009. №\,1(34). С.~23--40.

\bibitem{20-c}
\Au{Льюнг Л.}
Идентификация систем: Теория для пользователя.~--- М.: Наука, 1991.

 \label{end\stat}

\bibitem{21-c}
\Au{Mehra R.\,K.}
Optimal input signals for parameter estimation in dynamic systems~--- survey and
new results~// IEEE Trans. Automat. Control, 1974. Vol.~19. No.\,6.
P.~753--768.
 \end{thebibliography}
}
}


\end{multicols} %6
\def\stat{chirkunov}

\def\tit{АГЕНТНОЕ МОДЕЛИРОВАНИЕ РАЗВИТИЯ ТЕРРИТОРИАЛЬНОЙ 
СИСТЕМЫ}

\def\titkol{Агентное моделирование развития территориальной 
системы}

\def\autkol{К.\,С.~Чиркунов}
\def\aut{К.\,С.~Чиркунов$^1$}

\titel{\tit}{\aut}{\autkol}{\titkol}

%{\renewcommand{\thefootnote}{\fnsymbol{footnote}}\footnotetext[1]
%{Работа выполнена при поддержке гранта РФФИ 10-07-00433 и ФЦП <<Научные и на\-уч\-но-пе\-да\-го\-ги\-че\-ские 
%кадры инновационной России>> на 2009--2013~годы. Статья подготовлена по результатам работы секции 
%<<Биометрия>> 20-й Международной конференции по компьютерной графике и зрению Графикон-2010, 
%С.-Петербург, 20--24~сентября 2010~г.}}

\renewcommand{\thefootnote}{\arabic{footnote}}
\footnotetext[1]{Институт систем информатики им.\ А.\,П.~Ершова СО РАН, cyril.chirkunov@computer.org}


\Abst{Рассмотрена агентная система, имитирующая развитие экономики страны 
(строительство новых производств, повышение общего уровня доходов) на базе модели 
территориальной системы. Элементы модели представлены в виде автономных единиц, 
способных к взаимодействию друг с другом,~--- агентов.}

\KW{агентные алгоритмы переговоров; территориальная система; моделирование; 
экономическое районирование}

      \vskip 14pt plus 9pt minus 6pt

      \thispagestyle{headings}

      \begin{multicols}{2}
      
            \label{st\stat}

\section{Введение}

  Для моделирования процессов формирования и развития территориальных 
образований изначально применялись методики <<сверху вниз>> с 
последующей корректировкой решений <<снизу вверх>>, которые 
предполагали организацию вычислений с выделенным центром. Такой подход 
использовали А.\,Г.~Аганбегян, К.\,А.~Багриновский, А.\,Г.~Гранберг (об этом 
упоминается в работе~[1]). Новые подходы с использованием агентов 
(М.~Вулдридж, Й.~Шоэм, К.~Ли\-тон-Бра\-ун и~др.) позволяют описать 
территориальную систему  как совокупность взаимодействующих между собой 
агентов~[2, 3]. Решения на глобальном уровне (на уровне всей системы) 
возникают как результат локальных коопераций ее элементов. 
  
  Для чего нужно развивать новые подходы при уже неоднократно успешно 
зарекомендовавшем себя на практике существующем аппарате моделирования 
развития территориальных систем?
  
  Ответ на этот вопрос можно найти, например, в работе известного 
американского философа науки Пола Фейерабенда~[4]. По его мнению, ряд 
наиболее важных формальных свойств теории обнаруживается благодаря 
контрасту, а не анализу. Ученый, желающий более глубоко уяснить свои 
концепции, должен вводить новые.
  
  Не вдаваясь, подобно Полу Фейерабенду, в крайний релятивизм, примем его 
утверждение о том, что всякая устоявшаяся методология~--- даже наиболее 
очевидная~--- имеет свои пределы. Отступим от нескольких догматов, которые 
были <<провозглашены>> в рамках традиционного моделирования 
территориальной системы нашей страны, и предпримем попытку смоделировать 
систему как совокупность взаимодействующих агентов при отсутствии 
вертикальных иерархических связей на верхнем уровне. При этом сохраним ряд 
изначально существовавших в этой области понятий. 
  
  В процессе построения агентной модели необходимо решить целый ряд 
задач: определить структуру агентов и их коммуникации, описать сре\-ду
системы, составить алгоритмы коопера-\linebreak тивных взаимодействий (в том чис\-ле 
алгоритм\linebreak определения набора новых производств, алгоритм установления 
межрайонных, внешних то\-вар\-но-ма\-те\-ри\-а\-льных связей и~т.\,д.), 
привести обоснования\linebreak использованных решений. Все эти вопросы так или 
иначе должны быть затронуты в работе.

\section{Территориальная система как~совокупность агентов 
экономических районов}
  
  В данном разделе опишем общие базовые понятия и простую схему агентной 
модели территориальной системы.

\subsection{ Что такое агент?}

  В этом подразделе следовало бы дать четкую и ясную формулировку понятия 
<<агент>> и перейти к следующему. Однако не все так просто. Известно по 
меньшей мере шесть различных определений этого понятия в работах, 
посвященных агентному моделированию, из которых ни одно \mbox{нельзя} назвать 
общим, полным и ясным. Наиболее важным свойством агента можно считать 
автономность, иногда также отмечают обучаемость (способность к адаптации),  
но зачастую это свойство опускается. Агент действует в некоторой среде и 
обладает спо\-соб\-ностью получать информацию об объектах, которые в ней 
находятся. Будем считать, что среда является полностью наблюдаемой, т.\,е.\ 
агент может получить информацию обо всех объектах, которые в ней находятся 
(более формально составляющие среды будут описаны далее). Также агент 
может оказывать воздействие на среду и изменять ее состояние. Надо сказать, 
что среда в данной модели не является детерминированной для одного агента, 
т.\,е.\ изначально она не гарантирует только один исход после любого действия 
агента. Однако агенты могут договариваться друг с другом и, как следствие, 
обеспечивать детерминированность среды для определенного набора действий. 
В~данной работе будем предполагать, что агенты действуют одновременно, а 
среда является динамичной и дискретной. 
  
  Отдельно выделяют характеристики интеллектуальных агентов (в некоторых 
работах~--- интеллектуальных в слабом смысле).
  \begin{enumerate}
\item Реактивность~--- способность своевременно реагировать на воспринятые 
изменения среды.
\item Проактивность~--- способность проявлять инициативу для достижения 
своих целей.
\item Социальные навыки~--- способность к взаимодействию с другими 
агентами <<ради дела>>. 
\end{enumerate}

  Будем считать, что агенты предлагаемой модели обладают сле\-ду\-ющи\-ми 
характеристиками: они способны оперативно реагировать 
на изменения внешней конъюнктуры, могут выступать инициаторами 
переговоров с другими агентами и взаимодействовать с другими агентами ради 
общей и собственной выгоды. 
  
  Каждый агент имеет функцию полезности, которая в численном выражении 
показывает, насколько хорошо <<живется>> агенту в системе. Агент прилагает 
все усилия, чтобы повысить ее значение, но при этом старается увеличить и 
полезность общественную, которая задается как сумма значений функции 
полезности всех агентов. В~качестве значений функции полезности будет 
выступать доход, получаемый агентом в сис\-теме.
  
  Популяция взаимодействующих агентов совместно со средой образует 
\textit{многоагентную систему} (МАС).

\subsection{Схема модели}

  Теперь покажем, каким образом можно представить территориальную 
систему в виде МАС.
  
  На рис.~1 показана простая схема такой модели. Агенты, представляющие 
экономические районы (агенты ЭР), потребляют ресурсы, которые\linebreak\vspace*{-12pt}
\begin{center} %fig1
\vspace*{6pt}
\mbox{%
\epsfxsize=79mm
\epsfbox{chi-1.eps}
}
\end{center}
\vspace*{12pt}
%\begin{center}
{{\figurename~1}\ \ \small{Взаимодействие популяции агентов ЭР и внешней среды}}
%\end{center}
\vspace*{9pt}

\bigskip
\addtocounter{figure}{1}

\begin{center} %fig2
\vspace*{6pt}
\mbox{%
\epsfxsize=79mm
\epsfbox{chi-2.eps}
}
\end{center}
\vspace*{12pt}
%\begin{center}
{{\figurename~2}\ \ \small{Схема устройства модели территориальной сис\-те\-мы}}
%\end{center}
\vspace*{9pt}

\bigskip
\addtocounter{figure}{1}
     
     \noindent 
предо\-став\-ля\-ет им внешняя среда (т.\,е.\ территориальная система поглощает 
импортную продукцию), и, в свою очередь, генерируют товарную массу для 
внешней среды (т.\,е.\ система экспортирует на внешние рынки произведенную 
на своей территории продукцию).

 
     
  На схеме, приведенной на рис.~2, территориальная система описывается 
посредством активных сущностей (агентов) и пассивных сущностей (объектов). 
Агенты обозначаются кругами, объекты, которые они могут использовать, 
изображены в виде окружающих их квадратов. Агенты верхнего уровня 
иерархии (большие кружки) содержат в себе популяцию агентов и объекты 
нижнего уровня (маленькие кружки и квадраты). Пунктиром обозначена 
граница популяции агентов. Как и на рис.~1, стрелками показаны ма\-те\-ри\-аль\-но-то\-вар\-ные 
связи. Однако здесь уже видно, что агенты взаимодействуют и друг с 
другом: помимо внешних связей (жирные линии со стрелками), направленных 
либо наружу, либо внутрь популяции, появились и внутренние (линии со 
стрелками средней толщины и тонкие). Большими квадратами обозначены 
ареалы районов, малыми~--- площадки для размещения тер\-ри\-то\-ри\-аль\-но-про\-из\-вод\-ст\-вен\-ных 
комплексов (ТПК). Большие круги~--- агенты ЭР, 
генерирующие товарный поток в рамках своей межрайонной специализации и 
поглощающие материалы и продукцию, необходимую для производства и 
других хозяйственных нужд.
  
  
  Иными словами, если агент ЭР$_1$ нуждается в какой-либо продукции из 
межрайонных отраслевых специализаций ЭР$_2$, то он <<договаривается>> с 
агентом ЭР$_2$ и между ними устанавливается то\-вар\-но-ма\-те\-ри\-а\-ль\-ная 
связь. Маленькие кружки показывают агентов ТПК, отношения между 
которыми устанавливаются по такому же принципу, как и между агентами ЭР, 
но только внутри их родителя. Детальное рассмотрение агентов ТПК выходит 
за рамки данной работы.
  
  Можно заметить, что некоторые квадраты-пло\-щад\-ки являются вакантными. 
Это значит, что на данном месте еще не размещен ТПК. Тер\-ри\-то\-ри\-аль\-но-про\-из\-вод\-ст\-вен\-ные 
комплексы могут  динамически формироваться внутри незанятых площадок в ходе процесса 
симуляции. 
  
  При описании модели использовалось пред\-став\-ле\-ние о 
  тер\-ри\-то\-ри\-аль\-но-про\-из\-вод\-ст\-вен\-ном комплексе, которое 
содержится в работах Н.\,Н.~Колосовского. 
  
  \medskip
  
  \noindent
  \textbf{Определение~1.}
   \textit{Тер\-ри\-то\-ри\-аль\-но-про\-из\-вод\-ст\-вен\-ный комплекс~--- 
совокупность расположенных рядом друг с другом взаимосвязанных 
производств.}

  \smallskip
  
  Понятие было введено в экономическую географию в 1940-х~гг. В~исходном 
определении речь шла о взаимосвязанных и взаимообусловленных 
производствах, от размещения которых на определенной территории 
достигается дополнительный экономический эффект за счет использования 
общей инфраструктуры, кадровой базы, мощностей и~т.\,д.
  
  \smallskip
  
  \noindent
  \textbf{Определение 2.} \textit{Экономический район~--- это 
территориально связанные части единого народного хозяйства страны с 
различной специализацией, постоянным обменом производимых товаров и 
другими экономическими отношениями.}

  \smallskip
  
  Более детально о модели территориальной системы будет рассказано 
в следующем разделе.

\section{Математическое и~алгоритмическое описание задачи 
развития территориальной системы}
  
  Введем динамическое (т.\,е.\ меняющееся в ходе исполнения) множество 
товарных рынков~$M$, которое будет иметь следующую структуру: 
  \begin{equation*}
  M=M^{\mathrm{EX}}\cup M^{\mathrm{NR}}\cup M^I\,,
  \end{equation*}
  где $M^{\mathrm{EX}}$~--- множество товарных рынков за пределами территории 
страны, внешние рынки;
  $M^{\mathrm{NR}}$~--- множество межрайонных товарных рынков, описывающее 
потребности районов в тех или иных видах продукции, ресурсов и~т.\,д.;
  $M^I$~--- множество внутренних районных товарных рынков, описывающее 
потребности, которые могут быть удовлетворены только в рамках района. 
  
  Элемент $m\in M$ имеет следующую структуру: $m=\langle \mathrm{pr}, d\rangle$, где 
$\mathrm{pr}\in \mathrm{PR}$~--- это товар из множества возможных товаров, $d\in C$ ($C\in 
\mathcal{R}$)~--- денежная величина объема рынка. 
  
  Обозначим через TS множество территориальных систем: 
  $$
\mathrm{TS}=\{\mathrm{ts}\vert \mathrm{ts}~\mbox{---~территориальная~система}\}\,.
  $$
  
  Введем множество ER экономических районов, которое формально можно 
задать так: 
  $\mathrm{ER}=$\linebreak $=\{\mathrm{er} \vert \mathrm{er}~\mbox{---~экономический~район}\}$.
  
  Будем считать, что все территориально-производственные комплексы 
являются элементами одного и того же множества~TPC. 
  
  Далее опишем функцию развития территориальной системы.
  
  Функция~$U$ действует на множестве~TPC, и ее аргументы и значения~--- 
элементы этого множества: $U: \mathrm{TPC}\rightarrow \mathrm{TPC}$. 
  
  Доход, получаемый системой, будем обозначать через~$R$, где $R$~--- это 
функция, определенная на множестве территориальных систем~TS и 
отображающая его на множество действительных чисел~$\mathcal{R}$. 
В~качестве аргумента функции выступает произвольная территориальная 
система, а ее значением служит действительное число.
  
  \medskip
  
  \noindent
  \textbf{Определение 3.} $U$~--- \textit{программа развития для $\mathrm{ts}\in \mathrm{TS}$, 
если $R(U(\mathrm{ts}))>R(\mathrm{ts})$.}
  
  \smallskip
  
  \noindent
  \textbf{Определение 4.} \textit{Будем говорить, что $U^*$~--- оптимальная 
программа развития для $\mathrm{ts}\in \mathrm{TS}$ среди всех программ развития~$U$ для 
$\mathrm{ts}\in \mathrm{TS}$, если $R(U^*(\mathrm{ts}))\leq R(U(\mathrm{ts}))$ для любой такой~$U$.}
  
  \smallskip
  
  Территориальную систему $\mathrm{ts}\in \mathrm{TS}$ представим в виде структуры $\mathrm{ts}=\langle 
\mathrm{ER}_0, L_{\mathrm{TS}}, \mathrm{env}\rangle$.
  
  Итак, территориальная система описывается при помощи входящих в нее 
экономических районов ($\mathrm{ER}_0\subseteq \mathrm{ER}$), связей между ними (которые 
задаются через функцию $L_{\mathrm{TS}}: \mathrm{ER}\times \mathrm{ER}\rightarrow$\linebreak $\rightarrow  M^{\mathrm{NR}}$, 
действующую на пару экономических\linebreak районов и отобра\-жа\-ющую ее на 
подмножество множества межрайонных товарных рынков) и структурой 
внешней среды.
  
  Структура внешней среды $\mathrm{env}\in \mathrm{ENV}$ (где ENV~--- множество всех 
сред) специфицируется как $\mathrm{env}=\{S^{\mathrm{EX}}, M^{\mathrm{EX}}\}$, где $S^{\mathrm{EX}}$~--- 
множество внешних ресурсов (товарных, сырьевых и~т.\,д.), $M^{\mathrm{EX}}$~--- 
множество внешних рынков.
  
  Далее определим структуру самого экономического района.

Элемент $\mathrm{er}\in \mathrm{ER}$ можно представить в виде $\mathrm{er}=$\linebreak $=\langle \mathrm{TPC}_0, \mathrm{SQ}, 
L_{\mathrm{TPC}_0}\rangle$, где $\mathrm{TPC}_0\subset \mathrm{TPC}$~--- множество 
тер\-ри\-то\-ри\-аль\-но-про\-из\-вод\-ст\-вен\-ных комплексов, SQ~---
множество площадок, на которых можно разместить ТПК, $L_{\mathrm{TPC}_0}$~--- это 
функция ма\-те\-ри\-аль\-но-то\-вар\-ных связей между ТПК, $L_{\mathrm{TPC}_0}: 
\mathrm{TPC}_0\times$\linebreak $\times \mathrm{TPC}_0\rightarrow M^I$. Она отображает два 
ТПК на некоторое подмножество множества 
внутрирайонных рынков.
  
  Теперь рассмотрим, что в себя включает элемент $\mathrm{sq}\in \mathrm{SQ}$. Он имеет 
структуру $\mathrm{sq}=\langle G,\mathrm{NR},\mathrm{NP}\rangle$, где участок двухмерного пространства 
$G\subset \mathcal{R}^2$ задает локацию площадки, $\mathrm{NR}\subseteq S^I$ 
  ($S^I$~--- мно-\linebreak жество внутрирайонных ресурсов) обозначает\linebreak име\-юще\-еся на 
площадке множество природных ресурсов, $\mathrm{NP}\subseteq \mathrm{LP}^I$ ($\mathrm{LP}^I$~--- 
внутрирайонные трудовые ресурсы) обозначает имеющиеся на площадке 
трудовые ресурсы.
  
  Опишем теперь активные сущности системы~--- агентов.
  
  \smallskip
  
  \textbf{Агенты экономических районов.} Такие агенты имеют 
собственное внутреннее состояние, состоящее из элемента $\mathrm{er}\in \mathrm{ER}_0$, могут 
взаимодействовать как друг с другом, так и со средой, в которую помещена 
система. Внешнее состояние агента описывается набором производственных 
специализаций экономического района в виде $\langle f_1, \ldots , f_n\rangle$, 
где $f_i$ является элементом множества производственных специализаций 
$f_i\in F$, $i\in \{1, \ldots , n\}$. В~рамках данной модели предполагается, что 
производственные специализации ЭР образуются за счет 
внешних специализаций ТПК, находящихся на террито-\linebreak рии ЭР.
  
  В модели также используются агенты ТПК, но их описание выходит за рамки 
данной работы.
  
  Будем считать, что любая производственная специализация~$f$ 
соответствует некоторому элементу из множества товаров $\mathrm{pr}\in \mathrm{PR}$.

\subsection{Переговоры о специализации}

  В процессе функционирования территориальной системы агенты 
экономических районов договариваются друг с другом о наборе специализаций 
системы. Каждый агент ЭР способен предложить только какой-то 
ограниченный спектр специализаций, в то время как все множество имеющихся 
в сис\-те\-ме специализаций может быть шире. Более того, каждый район может 
делать лишь частичный вклад в конкретную специализацию сис\-те\-мы. 
Например, один экономический район~$\mathrm{er}_1$ производит 60\%, а другой 
экономический район $\mathrm{er}_2$ производит 40\% от всего объема 
продукции~pr, поставляемой на внешний рынок. Однако для этого случая 
трудоемкость реализации переговоров даже для десяти районов и нескольких 
десятков специализаций является весьма впечатляющей, поэтому в данной 
работе будем исходить из предположения, что каждый экономический район в 
рамках специализации производит 100\% требуемого товара.
  
  Алгоритм ведения переговоров между агентами состоит из двух шагов:
  \begin{enumerate}[(1)]
\item генерация предложений; 
\item выбор единственного предложения из уже имеющихся.
\end{enumerate}

  В зависимости от выбранного метода реализации шаги могут повторяться по 
нескольку раз. Например, это может произойти, когда агенты на\linebreak каж\-дой 
итерации алгоритма получают новую дополнительную информацию, которая 
влияет на состав генерируемых ими предложений. 

Представим сделку~$\lambda$ между агентами ЭР в следующем виде:
$$
\lambda\! =\! \left \{
\underbrace{f_1^1, \ldots , f^1_{n_1}}_{\mbox{\footnotesize\textit{набор специализаций~} 
}\mathrm{er}_1}\!,\ldots , \underbrace{f_1^{\vert \mathrm{ER}_0\vert}, \ldots , f_{n_{\vert 
\mathrm{ER}_0\vert}}^{\vert \mathrm{ER}_0\vert}}_{\mbox{\footnotesize\textit{набор специализаций~}}\mathrm{er}_2}
\right \},
$$
где $f_j^i$~--- производственная специализация $\mathrm{er}_i$ из набора, 
предлагаемого агентом экономического района~$er_i$. 
  
  Каждый экономический район может иметь несколько возможных наборов 
собственных специализаций

\noindent
  $$
  \left \{
  \begin{matrix}
  \{f_{1,1}, \ldots , f_{1,n_{j_1}}\}\,,\\
  \{f_{2,1}, \ldots , f_{2,n_{j_2}}\}\,,\\
  \ldots\,,\\
  \{f_{m,1}, \ldots , f_{m,n_{j_m}}\}\,,\\
  \{\}
  \end{matrix}
  \right \}\,.
  $$
  
  Заметим, что это множество содержит в качестве элемента также и пустое 
множество.
  
  Все множество сделок~$\Lambda$ описывается декартовым произведением 
множеств возможных наборов собственных специализаций экономических 
районов $\mathrm{er}\in \mathrm{ER}_0$:
\begin{multline*}
  \Lambda = \left \{
  \begin{matrix}
  \{f^1_{1,1}, \ldots , f^1_{1,n_{j_{1,1}}}\}\,,\\
  \{f^1_{2,1}, \ldots , f^1_{2,n_{j_{2,1}}}\}\,,\\
  \ldots\\
  \{f^1_{m_1,1}, \ldots , f^1_{m_1,n_{j_{1,m_1}}}\}\,,
  \end{matrix}
  \right \}
  \times\ldots\\
  {}\ldots\times
  \left \{
  \begin{matrix}
  \{ f^{\vert \mathrm{ER}\vert}_{1,1},\ldots , f_{1,n_{j_{1,1}}}^{\vert \mathrm{ER}\vert}\}\\
  \{ f^{\vert \mathrm{ER}\vert}_{2,1},\ldots , f_{2,n_{j_{2,1}}}^{\vert \mathrm{ER}\vert}\}\\
  \ldots\\
  \{ f^{\vert \mathrm{ER}\vert}_{m_{\vert \mathrm{ER}\vert }},\ldots , f_{m_{\vert 
\mathrm{ER}\vert},n_{j_{\vert \mathrm{ER}\vert,m_{\vert \mathrm{ER} \vert}}}}^{\vert \mathrm{ER}\vert}\}\\
  \end{matrix}
  \right \}
  \end{multline*}
  
  Теперь попытаемся оценить трудоемкость построения такого множества. Это 
важно ввиду следующего. Предположим, имеется некоторое множество сделок 
и необходимо выбрать такую из них, при которой территориальная система 
получит наибольший доход среди всех остальных сделок.
  
  Так как элементы множества не упорядочены, задача выбора сделки (задача 
поиска в неупорядоченном массиве) в худшем случае будет иметь 
трудоемкость, пропорциональную числу элементов этого множества.
  
  В данной работе будем исходить из предположения, что один экономический 
район производит товаров в рамках своей специализации вполне достаточно, 
чтобы полностью покрыть потребность в них на внешнем рынке, а также 
считаем, что чис\-ло новых возможных специализаций для района и размер 
предлагаемого набора специализаций для развития невелики: 3 и~2 
соответственно. Также учитываем то, что состав специализаций района может 
иметь переменную длину от~0 до~2 (т.\,е.\ множество предложений может 
содержать пустое множество). Однако полное число возможных специализаций 
для территориальной системы может быть и больше. Число экономических 
районов равно~12 (в соответствии с делением, принятым в~[5]). При 
таких допущениях подсчитаем максимальное число сделок: 
  $$
  (C_3^2+C_3^1+1)^{12}=13841287201\approx 1{,}4\cdot 10^{10}\,.
  $$
%\smallskip

\noindent
\textbf{Замечание:} можно уменьшить размер получаемого множества путем 
введения ограничения на набор специализаций для системы, т.\,е.\ он не должен 
содержать повторяющихся позиций, так как одинаковые специализации 
приводят к конкуренции, что, в свою очередь, влечет снижение доходов агентов 
и, следовательно, всей системы.

\medskip

\noindent
\textbf{Алгоритм 1.} Выбор набора специализаций территориальной системы.
\begin{enumerate}[1.]
\item Выбирается агент, принимающий предложения от других агентов,~--- 
посредник. Например, это может быть агент, соответствующий 
экономическому району с номером~1. 

\smallskip

\noindent
\textbf{Замечание:} экономические районы можно упорядочить по названиям в 
лексикографическом порядке, а затем присвоить им порядковые номера.
\item Агенты ЭР формируют наборы возможных специализаций и посылают их 
аген\-ту-по\-сред\-ни\-ку.
\item Агент-посредник на основе полученной информации формирует 
множество сделок и выбирает ту, которая принесет наибольший доход 
территориальной системе, с помощью функции~$R$.
\item Окончательная сделка рассылается аген\-том-по\-сред\-ни\-ком всем 
остальным агентам.
\item Агенты из полученного сообщения извлекают свое множество 
специализаций.
\end{enumerate}

\medskip

\noindent
\textbf{Примечание 1.} Можно преобразовать данный ал\-горитм в полностью 
децентрализованный, если\linebreak обязать агентов рассылать свои предложения друг 
другу. Каждый агент, зная функцию~$R$, может самостоятельно сформировать 
множество сделок и выбрать среди них наиболее полезную для популяции, 
после чего определить свой состав производственных специализаций. Можно 
утверждать, что все агенты получат в итоге одну и ту же сделку, если задать 
дополнительное условие: в случае, если доход от разных сделок одинаковый, то 
предпочтение отдается той, которая лексикографически меньше всех других.
\smallskip

\noindent
\textbf{Примечание 2.} Для получения приемлемой сделки также можно 
использовать \textit{монотонный протокол уступок и стратегию Жозена}, 
описанные в книге Вулдриджа, посвященной многоагентному 
моделированию~[2]. Такое решение не гарантирует достижения глобального 
максимума на множестве сделок, однако с помощью него возможно заключение 
Па\-ре\-то-оп\-ти\-маль\-ных сделок.

\subsection{Функции развития и алгоритм выбора специализаций района}

  Теперь перейдем к вопросу формирования предложений в рамках 
экономического района $\mathrm{er}\in \mathrm{ER}$.
  
  Будем считать, что каждый экономический район $\mathrm{er}\in \mathrm{ER}$ уже имеет 
некоторую совокупность отраслей специализации $\{f_1, \ldots , f_n\}$, которая 
формируется за счет входящих в его состав ТПК.
  
  Введем семейство функций $CH: 2^F\rightarrow 2^F$ (где~$F$, как уже 
упоминалось, является множеством производств по специализациям):
  \begin{enumerate}[1)]
\item CH$_0$~--- пустая функция, которая отображает аргумент сам на себя;
\item CH$_1$~--- функция, которая <<ликвидирует>> производства с 
отрицательной прибылью.
Предположим, что имеется совокупность про\-извод\-ственных отраслей 
экономического \mbox{района} $\mathrm{er}\in$\linebreak $\in  \mathrm{ER}_0$ представленная в виде $\{f_1,f_2\}$, 
причем~$f_2$ является убыточной. Тогда CH$_1(\{f_1,f_2\})=$\linebreak $=\{f_1\}$. Если 
убыточным окажется и производство~$f_1$, то значением функции станет 
пустое множество.
\item CH$_2$~--- функция, отвечающая за применение новых технологий. Она не 
изменяет размер множества отраслей специализации экономического района 
$\mathrm{er}\in \mathrm{ER}$, однако способствует повышению доходности отрасли. Пусть 
имеются новые технологии для~$f_1$. Тогда 
CH$_2(\{f_1,f_2\})=\{f_1^\prime,f_2\}$.
\item CH$_3$~--- функция, отвечающая за открытие новой производственной 
специализации или освоение нового ресурса. Пусть $\{f_1\}$~--- имеющийся 
набор специализаций для экономического района $\mathrm{er}\in \mathrm{ER}$ и появилась 
возможность открытия новой специализации~$f_2$. Тогда $\{f_1,f_2\}$ будет 
принадлежать к множеству об\-ласти значений данной функции. Заметим, что 
вариантов открытия новой производственной специализации может быть 
несколько, соответственно, для одного аргумента может быть несколько 
значений функции. В~данной работе будем рассматривать только случай, когда 
функция либо не определена (как для данного аргумента, так и вообще), либо 
имеет единственное значение. В нашем случае CH$_3(\{f_1\})=\{f_1,f_2\}$.
\end{enumerate}


\noindent
\textbf{Замечание:} Функции CH$_1$, CH$_2$, CH$_3$ заданы лишь частично, 
т.\,е.\ они определены не на всем множестве~$2^F$. Доопределим их функцией 
CH$_0$, которая определена на всем множестве~$2^F$. 
  
  \smallskip
  
  Теперь опишем простой алгоритм принятия решения о применении той или 
иной функции из семейства~CH (он используется агентами ЭР для 
формирования наборов специализаций при проведении переговоров). Для этого 
введем дополнительную функцию дохода экономического района для 
заданного набора специализаций (которая определяется аналогично функции 
дохода для территориальных систем).
  Функция $R: \mathrm{ER}\times F\rightarrow \mathcal{R}$ преобразует экономический 
район с заданным набором отраслей в величину дохода, который будет иметь 
район с такими производствами.

\medskip

\noindent
\textbf{Алгоритм 2.} Выбор программы развития района.

\textit{Входные данные}: er и $\{f_1, \ldots ,f_n\}$ (экономический район~$er$ 
с набором специализаций).
  
  Последовательно применяем функции CH$_0$, CH$_1$, CH$_2$, CH$_3$ к 
одному и тому же набору $\{f_1, \ldots ,f_n\}$ и среди результатов выбираем те, 
при которых функция~$R$ принимает наибольшее значение. 

\textit{Выходные данные}: результаты в виде $\{f_1^*, \ldots , f_n^*\}$ (набор 
специализаций, оптимальный по доходу на множестве результатов применения 
функций из семейства~CH от аргумента $\{f_1, \ldots , f_n\}$ для 
района~er).

Очевидно, $R(\mathrm{er},\{f_1, \ldots , f_n\})\leq R(\mathrm{er},\{f_1^*, \ldots$\linebreak $\ldots , f_n^*\})$.

\medskip

\noindent
\textbf{Замечание 1.} В~силу того что функций всего четыре, можно 
использовать все получившиеся результаты (а не только оптимальные) для 
дальнейшего выбора набора специализаций территориальной системы. Это 
увеличивает пространство возможных решений на уровне системы. 
  
  (Правильнее будет сказать, что число классов функций заданного вида равно 
четырем, однако в данной работе считается, что каждый из классов содержит 
только один элемент.)
  \medskip
  
  Применение функции CH$_3$ на внутрирайонном уровне может приводить к 
тому, что в er возникнет потребность в определенном товаре, что может 
привести к появлению межрайонного рынка $m\in M^{\mathrm{NR}}$ так же, как и к 
увеличению числа внешних связей территориальной системы.

\section{Заключение}

  В работе описан один из возможных подходов к моделированию развития 
территориальных систем. Были освещены далеко не все аспекты, а лишь самые 
ключевые моменты: представление территориальной системы с помощью 
взаимодействующих агентов, выбор программы развития\linebreak экономического 
района, формирование набора специализаций территориальной системы путем 
переговоров. В~написании статьи очень помогли работы М.\,К.~Бандмана, 
М.~Вулдриджа, Ф.~Котлера, Н.\,Н.~Колосовского 
и~др.~\cite{1-chu, 2-chu, 6-chu, 7-chu}. 
  
  В дальнейшем планируется подробно остановиться на алгоритмах 
модернизации и формирования ТПК в пределах экономического района в 
рамках описанной формальной модели.

{\small\frenchspacing
{%\baselineskip=10.8pt
\addcontentsline{toc}{section}{Литература}
\begin{thebibliography}{9}

\bibitem{1-chu}
\Au{Бандман М.\,К., Бурматова О.\,П., Воробьева~В.\,В.}
Моделирование формирования 
тер\-ри\-то\-ри\-аль\-но-про\-из\-вод\-ст\-вен\-ных комплексов.~--- 
Новосибирск: Наука, 1976.

\bibitem{2-chu}
\Au{Wooldridge M.}
An introduction to multiagent systems.~--- New York: John Wiley \& Sons, 2002.

\bibitem{3-chu}
\Au{Shoham Y., Leyton-Brown~K.}
Multiagent systems: Algorithmic, game-theoretic, and logical foundations.~--- 
Cambridge University Press, 2009.

\bibitem{4-chu}
\Au{Фейерабенд П.}
Против метода. Очерк анархистской тео\-рии познания~/ Пер. с англ. 
А.\,Л.~Никифорова.~--- М.: АСТ, 2007. 

\bibitem{5-chu}
Общероссийский классификатор экономических регионов ОК 024-95 (ОКЭР). 
Утвержден постановлением Госстандарта РФ от 27~декабря 1995~г. №\,640, в 
ред. изменения №\,1, ноябрь 1998~г., с изм. и доп. №\,2/99, №\,3/2000, 
№\,4/2001, №\,5/2001.

\bibitem{6-chu}
\Au{Котлер Ф.}
Маркетинг менеджмент.~--- СПб.: Питер, 2001.

 \label{end\stat}

\bibitem{7-chu}
\Au{Колосовский Н.\,Н.}
Теория экономического 
районирования.~--- М.: Мысль, 1969.


 \end{thebibliography}
}
}


\end{multicols}   %7
\def\stat{dem}

\def\tit{О ДВУХ МОДЕЛЯХ РАСПРЕДЕЛЕНИЯ РЕСУРСОВ 
ПРИ~ОРГАНИЗАЦИИ ИНВЕСТИЦИОННЫХ ПРОЦЕССОВ}

\def\titkol{О двух моделях распределения ресурсов 
при~организации инвестиционных процессов}

\def\autkol{П.\,В.~Демин}
\def\aut{П.\,В.~Демин$^1$}

\titel{\tit}{\aut}{\autkol}{\titkol}

%{\renewcommand{\thefootnote}{\fnsymbol{footnote}}\footnotetext[1]
%{Работа выполнена при поддержке ФЦП <<Научные и научно-педагогические кадры инновационной России>> 
%на 2009--2013~гг.}}

\renewcommand{\thefootnote}{\arabic{footnote}}
\footnotetext[1]{Государственное образовательное учреждение <<Московская академия рынка труда и 
информационных технологий>>, pdemin@mail.ru}

\vspace*{-12pt}
 
      
\Abst{Приводятся два примера решения проблем, возникающих 
при организации инвестиционных процессов, связанных с модернизацией экономики. 
Первый относится к проблеме, возникающей в банке при решении вопроса выбора для 
финансирования из некоторого множества проектов. Второй~--- к решению задачи 
распределения инвестиционного ресурса между предприятиями, входящими в состав 
холдинга.}

\vspace*{-2pt}
      
      \KW{инвестиционный процесс; инновация; банковское финансирование; ресурсы; 
холдинг}

      \vskip 8pt plus 9pt minus 6pt

      \thispagestyle{headings}

      \begin{multicols}{2}
      
            \label{st\stat}


\section{Введение}

В данной работе приводятся два примера решения проблем, возникающих 
при организации инвестиционных процессов, связанных с модернизацией 
экономики. Первый относится к проблеме, возникающей в банке при 
решении вопроса выбора для финансирования из некоторого множества 
проектов. Второй~--- к решению задачи распределения инвестиционного 
ресурса между предприятиями, входящими в состав холдинга. В~первом 
случае проблему удается свести к оптимальной дискретной задаче, в которой 
оптимизируется суммарный доход, полученный банком за рассматриваемый 
период времени. В~качестве ограничений рассматриваются лимиты на 
доступные ресурсы в каждый из рассматриваемых моментов времени. Во 
втором случае решение сводится к построению многоэтапной модели 
распределения инвестиционных ресурсов. Обе задачи решаются при наличии 
как в банке, так и в холдинге базы первичных данных о предприятиях в 
целом, позволяющих рассчитывать необходимые для решения параметры и 
функции. Это объединяет оба примера. В~силу сложности и размерности 
возникающих оптимизационных задач для решения используются элементы 
эвристических методик.

\section{Модель выбора в банке проектов для финансирования}
     
     Важнейшим условием реализации инновационного проекта является 
наличие ресурсного обеспечения. Многое в процессе поиска и 
предоставления необходимых ресурсов зависит от банка, клиентом которого 
является субъект, планирующий эту реализацию. Рассмотрим модель, 
позволяющую банку выбрать вариант финансирования подмножества 
проектов из некоторого множества ин\-вес\-ти\-ци\-он\-ных проектов. Учитывая, что 
такие проекты, как правило, отличаются большими сроками реализации и 
объемами финансирования (медленные процессы), задача решается на фоне 
текущей кредитной деятельности банка (быстрые процессы). Общие 
характеристики текущей деятельности банка входят параметрами в модель 
выбора подмножества инвестиционных проектов.

\subsection{Формальная постановка} %2.1
     
     При формализации задачи важным является понятие варианта 
финансирования инвестиционного проекта. Пусть~$I$~--- множество 
проектов, которые находятся в портфеле у банка и из числа которых банк 
может выбирать проекты для финансирования в соответствии с наличием 
ресурсов и представлениями о целесообразности их выделения. Введем 
переменные~$x_{iq}$, где $q\in Q_i$, $i\in I$, $Q_i$~--- множество вариантов 
финансирования $i$-го  проекта. Варианты могут различаться, например, 
сроками начала финансирования или объемами ресурсов, необходимых в 
конкретные моменты времени. Будем полагать, что
     $$
     x_{iq}= 
     \begin{cases}
     1, & \mbox{если проект}\ i\in I \mbox{~~финансируется}\\
     & \mbox{по варианту}\ q\in  Q_i\,;\\
     0 & \mbox{иначе.}
     \end{cases}
     $$
     

Обозначим
\begin{gather*}
x=\left \{ x_{iq}/x_{iq} =0, 1, \enskip q\in Q_i, \ i\in I\right \}\,,\\
x\in X=\left \{ x/ \sum\limits_q x_{iq}\leq 1, \enskip i\in I\right \}\,.
\end{gather*}

     
     Введем величины~$a_{iqt}$, $b_{iqt}$, где $a_{iqt}$~--- объем 
ресурсов, необходимых для финансирования проекта $i\in I$ в момент $t\in 
[0,\,T]$, если финансирование происходит по варианту $q\in Q_i$, 
$b_{iqt}$~--- объем ресурсов, который возвращается банку в момент $t\in 
[0,\,T]$ при финансировании проекта по варианту $q\in Q_i$.

Будем также считать, что текущая деятельность банка заключается в 
обслуживании кредитных и депозитных договоров и ее можно описать 
сле\-ду\-ющи\-ми зависимостями ($t\in [0,\,T]$):
\begin{description}
%\smallskip
%\noindent
\item[\,] $z(t)$~--- средний суммарный остаток на расчетных счетах банка в момент 
времени~$t$; 
\item[\,]
%\noindent
$d(t)$~--- средний суммарный остаток на депозитных счетах в момент 
времени~$t$;
\item[\,]
%\noindent
$S(t)$~--- средняя суммарная величина выплат процентов по депозитным 
счетам в момент времени~$t$; 
\item[\,]
%\noindent
$A(t)$~--- величина предоставляемых в момент времени~$t$ кредитных 
ресурсов;
\item[\,]
%\noindent
$B(t)$~--- величина возвращаемых в момент времени~$t$ кредитных 
ресурсов.
\end{description}
%\smallskip
     
     Тогда задачу выбора вариантов финансирования инвестиционных 
проектов можно записать в виде
     \begin{equation}
     \underset{x\in X}{\max} \sum\limits_i \sum\limits_q \left[ \sum\limits_t 
b_{iqt}-a_i^0\right ] x_{iq}\,,\label{e2.1.1d}
     \end{equation}
     
     \vspace*{-12pt}
     
     \noindent
     \begin{multline}
     \sum\limits_{\tau=1}^t\left[\sum\limits_i\sum\limits_q a_{iq\tau} 
x_{iq}+A(\tau)\right] \leq z(t)+d(t)+{}\\
\!\!{}+\sum\limits_{\tau=1}^t\left[ B(\tau)-S(\tau)+\sum\limits_i\sum\limits_q 
b_{iq\tau} x_{iq}\right ]\!\!,\ t\in [0,\,T],\!\!\!
     \label{e2.1.2d}
     \end{multline}
где $a_i^0$~--- суммарная величина ресурсов, необходимых для 
финансирования $i$-го проекта.
     
     Задача~(\ref{e2.1.1d}), (\ref{e2.1.2d}) имеет прозрачный физический 
смысл: для финансирования выбираются проекты, которые приносят 
максимальный доход банку, при этом сумма размещенных активов в каждый 
момент времени не должна превышать текущих пассивов. (Здесь для 
простоты изложения будем считать, что собственные средства банка 
полностью расходуются на предоставление краткосрочных кредитов). 

\subsection{Построение решения поставленной задачи} %2.2
     
     Задачу~(\ref{e2.1.1d}), (\ref{e2.1.2d}) запишем в виде:
     \begin{equation}
     \underset{x\in X}{\max} \left \{f(x)\vert g(x)\geq 0\right \}\,,
     \label{e2.2.1d}
     \end{equation}
где $f(x)$~--- целевая функция, $g(x)\geq 0$~--- вектор ограничений. Пусть 
$L(x,y)=f(x)+y\times g(x)$~--- функция Лагранжа для задачи~(\ref{e2.2.1d}). 
Известно, что $\underset{x\in X}{\max}\,\underset{y\geq 0}{\min} L(x,y) \leq 
\underset{y\geq 0}{\min}\,\underset{x\in X}{\max} L(x,y)=\omega^*$.
     
     Нахождение $\underset{x\in X}{\max}\,\underset{y\geq 0}{\min} L(x,y)$ 
соответствует решению исходной задачи~(\ref{e2.2.1d}). Задачу 
$\underset{y\geq 0}{\min}\,\underset{x\in X}{\max} L(x,y)$ называют 
двойственной. Решение двойственной задачи позволяет получить оценку 
сверху~$\omega^*$ для значения исходной задачи и служит эвристическим 
способом построения допустимого решения для исходной задачи. 
Обоснование метода решения двойственной задачи содержится 
     в~\cite{1-dem} . Его идея заключается в следующем. Можно показать, 
что
\vspace*{-8pt}

\noindent
     \begin{multline}
     \omega^*=\underset{\omega,y}{\min}\left\{\omega\vert\omega\geq 
f(x)+y\times g(x)\,,\right. \\[-2pt]
\left. x\in X_y\,, \ y\geq o\right \}\,,
     \label{e2.2.2d}
     \end{multline}
где $X_y=\mathrm{Arg}\, \underset{x}{\max}\left\{ L(x,y)\vert y\geq 0,\ x\in X\right \}$, 
т.\,е.\ множество решений, доставляющих максимум функции Лагранжа при 
всех возможных неотрицательных значениях вектора двойственных 
переменных. Задачу~(\ref{e2.2.2d}) можно решать методами линейного 
программирования, итерационно добавляя существенные ограничения и 
отбрасывая несущественные. При этом из построенных элементов 
множества~$X_y$, допустимых для исходной задачи, выбирается такой, 
который максимизирует целевую функцию исходной задачи. Несмотря на то,
что строго доказать существование допустимого для исходной задачи 
элемента множества~$X_y$ удалось только для одного ограничения, 
реальные вычисления показывают, что такой элемент получается 
практически всегда.

\vspace*{-5pt}

\section{Модель распределения ресурсов между предприятиями 
холдинга}

\vspace*{-3pt}
     
     В период инновационного развития экономики в крупных холдингах 
возникает задача распределения ресурсов развития между предприятиями с 
таким расчетом, чтобы получить максимальный желаемый эффект. 
Рассмотрим многоэтапную модель распределения ресурсов, влияющих на 
инновационное развитие предприятий, входящих в холдинг. Решения о 
распределении ресурсов развития принимаются (как это обычно и 
происходит в жизни) в дискретные моменты времени. Предложенный 
алгоритм оптимизации не претендует на глобальность, но прозрачен и прост 
в реализации. Его можно рассматривать как инструмент при скользящем 
планировании распределения ресурсов, когда решения принимаются на шаг 
вперед.
     
     
     \vspace*{-4pt}
     
\subsection{Описание холдинговой структуры} %3.1

\vspace*{-1pt}
     
     Рассмотрим организационную структуру холдингового типа. 
Предприятия связаны холдинго-\linebreak\vspace*{-12pt}
\pagebreak

\noindent
выми отношениями, позволяющими одному 
из них (головной компании) определять решения, принимаемые другими 
участниками холдинга. 
     
     Головная компания осуществляет: централизованное управление 
активами холдинга; управление корпоративной стратегией; мониторинг 
процессов, проходящих в холдинге в необходимом временном режиме; 
контроль интегральных и мониторинг частных целевых показателей 
     биз\-нес-планов.
     
\vspace*{-3pt}

\subsection{Модель холдинговой системы} %3.2

%\vspace*{-1pt}

     
     Рассматриваемая система является иерархической, с вертикальными 
связями. В~системе имеется управляющий центр и элементы более низкого 
уровня. 
     
     Пусть в состав холдинговой структуры входит $N$~предприятий, 
     $t$~--- номер дискретного интервала времени, на котором 
рассматривается деятельность холдинга, всего рассматривается 
$T$~интервалов времени. Предположим, что для $n$-го элемента 
холдинговой системы можно построить функцию~$F_n(z_{nt})$, где 
$F_n(\cdot)$~--- объем выпущенной продукции (в денежном выражении) на 
конец рассматриваемого периода; $z_{nt}$~--- объем средств, направляемых 
на улучшение и модернизацию $n$-го производства в интервале 
времени~$t$. Обозначим $z=\{ z_t\vert t, \ldots , T\}$, $z_t=\{z_{nt}\vert 
n=1,\ldots ,N\}$.
     
     Целью управляющего центра является рост консолидированной 
прибыли холдинга на протяжении долгосрочного периода. В~качестве 
критерия развития холдинга примем функционал $\sum\limits_n b_n 
F_n(z_{nT})$. Таким образом, рассматривается задача нахождения 

\noindent
     \begin{equation}
     \underset{z}{\max} \sum\limits_n b_n F_n(z_{nT})
     \label{e3.2.1d}
     \end{equation}
при выполнении ограничений

\vspace*{-6pt}

\noindent
\begin{multline}
\sum\limits_n z_{nt}\leq \lambda \sum\limits_n F_n(z_{n\,t-1})\,,\\[-6pt]
 t=1,2\ldots 
, T\,, \ z\geq 0\,.
\label{e3.2.2d}
\end{multline}
     
     Параметр~$\lambda$ считается заданным центром. Предполагается, что 
для $t=0$ известен начальный распределяемый инвестиционный ресурс. 
     
     Вместо решения задачи~(\ref{e3.2.1d}), (\ref{e3.2.2d}) будем для 
каж\-до\-го~$t$, начиная с~1, последовательно решать оптимальные задачи 

\vspace*{-6pt}

\noindent
     \begin{multline}
     \underset{z_t}{\max}\left \{ \sum\limits_n b_n F_n(z_{nt})\left\vert 
\sum\limits_n z_{nt}\leq \lambda \sum\limits_n F_n(z_{n\,t-1})\right.\right.\,,\\[-6pt]
\left. \vphantom{\sum\limits_n} z_t\geq 0
\right \}
\,.
     \label{e3.2.3d}
     \end{multline}

Оптимизацию по $z_t$ для каждого~$t$ можно проводить с помощью 
процедуры динамического программирования~[2]. Не вдаваясь в 
подроб-\linebreak\vspace*{-12pt}
\columnbreak

\noindent
ности этой хорошо известной процедуры, отметим только, что в ходе 
ее реализации при перемещении от~$N$ до~1 строятся функции 
$\varphi_n(c)=$\linebreak $=\underset{z_t}{\max} \left \{\sum\limits_{k=n}^N b_k 
F_k(z_{kt})\left\vert \sum\limits_{k=n}^N z_{kt}\right. \leq c,\ c\geq o\right \}$, 
удовле\-творяющие рекуррентному соотношению $\varphi_{n-
1}(c)\!=\!\underset{z_{n-1},t}{\max}\!\left\{\varphi_n(c-z_{n-1,t})+b_{n-1}F_{n-
1}(z_{n-1,t})\right \}$. Затем при перемещении в обратном направлении 
получают оптимальное решение задачи~(\ref{e3.2.3d}), т.\,е.\ находят 
распределение инвестиций по предприятиям, входящим в холдинг.

\vspace*{-9pt}

\section{Заключение}

\vspace*{-6pt}
     
     В работе рассмотрены две модели, связанные с организацией 
финансирования инвестиционных процессов. Одна модель описывает 
процесс выбора проектов для финансирования в банке. Вторая~--- процесс 
распределения инвестиционных ресурсов между предприятиями, входящими 
в состав холдинга. 
     
     Задачи, встающие при организации инвестиционных процессов, 
подобные рассмотренным выше, возникают и в банках, и в крупных 
холдингах повсеместно. Если реализовать приведенные модели в 
соответствующих интерактивных системах для организации финансирования 
инвестиционных процессов, то практическая польза окажется существенно 
выше, чем при простом поиске оптимальных решений поставленных задач. 
Например,\linebreak кредитное управление банка чрезвычайно заин\-тересовано в 
оценке дохода ($\omega^*$), который можно получить при имеющемся 
портфеле проектов\linebreak и прогнозе доступных ресурсов. Более того, 
эффективность решений по привлечению ресурсов\linebreak банком может быть 
существенно повышена, если заранее указать моменты времени, в которые 
намечается избыток или недостаток ресурсов, а это модель позволяет 
сделать. 
     
     Несмотря на востребованность, постановочные работы подобного 
плана практически отсутствуют. В~какой-то мере это объясняется тем, что 
необходимые для моделей параметрические зависимости рассчитываются 
при наличии в организации базы первичных данных по организации в целом.

\vspace*{-14pt}

{\small\frenchspacing
{%\baselineskip=10.8pt
\addcontentsline{toc}{section}{Литература}
\begin{thebibliography}{9}

\vspace*{-2pt}

\bibitem{1-dem}
\Au{Демин В.\,К., Малашенко~Ю.\,Е.}
Получение оценочных решений для задач оптимального резервирования~// Известия АН 
СССР, Техническая кибернетика, 1974. №\,1. С.~112--117.


 \label{end\stat}

\bibitem{2-dem}
\Au{Замков О.\,И., Черемных~Ю.\,А., Толстопятенко~А.\,В.}
Математические методы в экономике: Учебник.~--- 4-е изд., стереотип.~--- М.: Дело и сервис, 
2004. 368~с.
 \end{thebibliography}
}
}


\end{multicols}   %8
\def\stat{pavel}


\def\tit{ПОИСК И АНАЛИЗ КЛЮЧЕВЫХ ТОЧЕК РАДУЖНОЙ ОБОЛОЧКИ ГЛАЗА МЕТОДОМ 
ПРЕОБРАЗОВАНИЯ ЭРМИТА$^*$}
\def\titkol{Поиск и анализ ключевых точек радужной оболочки глаза методом 
преобразования Эрмита}

\def\autkol{Е.\,А.~Павельева, А.\,С.~Крылов}
\def\aut{Е.\,А.~Павельева$^1$, А.\,С.~Крылов$^2$}

\titel{\tit}{\aut}{\autkol}{\titkol}

{\renewcommand{\thefootnote}{\fnsymbol{footnote}}\footnotetext[1]
{Работа выполнена при финансовой поддержке РФФИ (проект №\,10-07-00433-а).}}

\renewcommand{\thefootnote}{\arabic{footnote}}
\footnotetext[1]{Московский государственный университет им.\ М.\,В.~Ломоносова,
факультет вычислительной математики и кибернетики, 
 paveljeva@yandex.ru}
\footnotetext[2]{Московский государственный университет 
им.\ М.\,В.~Ломоносова, факультет вычислительной математики и кибернетики,
kryl@cs.msu.ru}

\vspace*{-6pt}

\Abst{Предложен алгоритм поиска ключевых точек для распознавания человека по 
радужной оболочке глаза, основанный на локальном преобразовании Эрмита. 
Использование для распознавания только ключевых точек радужной оболочки позволяет 
хранить небольшой объем информации при достаточно хорошем качестве распознавания.}
\vspace*{1pt}

\KW{биометрия; радужная оболочка глаза; преобразование Эрмита; ключевые точки}

   \vskip 18pt plus 9pt minus 6pt

      \thispagestyle{headings}
      
       \vspace*{6pt}

      \begin{multicols}{2}

      \label{st\stat}
      
  

\section{Введение}

Преобразование Эрмита~[1] является известным методом, применяющимся для решения 
биометрических задач~[2--4]. Этот локальный метод основан на вычислении сверток функции 
интенсивности изображения с функциями преобразования Эрмита в каждой точке изображения. 
При этом в работе~[5] показано, что для задачи распознавания по радужной оболочке глаза 
наиболее информативными являются свертки с двумерной функцией 
пре\-об\-разо\-ва\-ния Эрмита~$\varphi_{1,0}$. Также широко известным методом в обработке сигналов является метод 
моментов Гаусса--Эрмита~\cite{2pav}, эквивалентный преобразованию Эрмита с 
точностью до знаков сверток с нечетными функциями преобразования Эрмита. При этом для 
решения задач распознавания формируются бинарные матрицы, составленные из знаков сверток 
в каждой точке, которые затем сравниваются с матрицами из базы данных.
{ %\looseness=1

}

В данной работе на основе преобразования Эрмита предложен метод нахождения ключевых 
точек текстуры радужной оболочки. Эти точки соответствуют наиболее значимым экстремумам 
свертки функции интенсивности изображения с функцией~$\varphi_{1,0}$. 

В разд.~2 дается описание преобразования Эрмита. В разд.~3 приведены некоторые детали 
использованного метода предобработки изображений радужной оболочки. Раздел~4 описывает 
алгоритм нахождения ключевых точек радужной оболочки, приведены результаты 
экспериментов на базе данных CASIA-IrisV3~\cite{6pav}. 

\section{Преобразование Эрмита}

Функции Эрмита определяются как
$$
\psi_n (x) = \fr{(-1)^n e^{-x^2/2}}{\sqrt{2^n n!\sqrt{\pi}}}\,H_n(x)\,,
$$
где $H_n(x)$~--- полиномы Эрмита:
\begin{gather*}
H_0(x) =1\,,\quad H_1(x) =2x\,,\\
H_n(x) =2x H_{n-1}(x) -2(n-1)H_{n-2}(x)\,.
\end{gather*}
      
Функции Эрмита являются собственными функциями преобразования Фурье и образуют полную 
ортонормированную систему функций в пространстве~$L_2(-\infty,\,\infty)$. 

Функции преобразования Эрмита (рис.~1) связаны с функциями Эрмита соотношением
$$
\varphi_n(x) =\psi_0(x)\psi_n(x) =\fr{(-1)^n e^{-x^2}}{\sqrt{2^n n!\pi}}\,H_n(x)\,.
$$
При вычислениях они являются одновременно локализованными в координатном и частотном 
пространствах. Так как~$\psi_n$~--- ортонормированная сис\-те\-ма функций, то
$$
\int\limits_{-\infty}^\infty \varphi_n(x)\,dx =\int\limits_{-\infty}^\infty \psi_0(x)\psi_n(x)\,dx =0\quad 
\forall n>0\,,
$$ 
т.\,е.\ функции~$\varphi_n(x)$ имеют среднее нулевое значение для номеров $n>0$. 
Это свойство очень важно\linebreak\vspace*{-12pt}
\pagebreak

\noindent
\begin{center} %fig1
\vspace*{3pt}
\mbox{%
\epsfxsize=77.701mm %78.39mm
\epsfbox{pav-1.eps}
}
%\end{center}
%\vspace*{6pt}

{{\figurename~1}\ \ \small{Примеры функций преобразования Эрмита}}
\end{center}
\vspace*{-6pt}


\bigskip
\addtocounter{figure}{1}


\noindent
 для методов, исполь\-зу\-ющих знаки сверток с такими функциями. 

Двумерные функции преобразования Эрмита являются произведением одномерных функций:
%\noindent
$$
\varphi_{n,m}(x,y) =\varphi_n(x)\varphi_m(y)\,.
$$


Преобразование Эрмита для изображения опреде\-ляется в каждой точке~$(x_0,y_0)$ значениями 
сверток функции интенсивности изображения~$I(x,y)$ с функциями преобразования %\linebreak 
Эрмита~$\varphi_{m,n}(x,y)$ для выбранного конечного набора индек\-сов~$(m,n)$~[1,~4]:
\begin{multline*}
M_{m,n}(x_0,y_0) =(I(x,y) * \varphi_{m,n}(x,y))(x_0,y_0)={}\\
{}= \iint\limits_G I(x,y) \varphi_{m,n}(x_0-x,y_0-y)\,dxdy\,,
\end{multline*}
где $G$~--- область сосредоточения функции~$\varphi_{m,n}$.



\section{Предобработка изображений и~контроль наличия века в~области 
параметризации}

Алгоритм нахождения радужной оболочки на изображении глаза описан в~\cite{5pav, 7pav} и 
основывается на поиске максимального скачка средней интенсивности вдоль круговых контуров 
изображения. После локализации радужная оболочка глаза переводится в прямоугольное 
нормализованное изобра-\linebreak\vspace*{-12pt}


\noindent
\begin{center} %fig2
\vspace*{6pt}
\mbox{%
\epsfxsize=78.847mm %78.39mm
\epsfbox{pav-2.eps}
}
%\end{center}
\vspace*{6pt}

{{\figurename~2}\ \ \small{Нормализация радужной оболочки}}
\end{center}
%\vspace*{-6pt}


\bigskip
\addtocounter{figure}{1}


\noindent
жение. Для дальнейшей параметризации в работе используется только 
область, включающая правую верхнюю четверть нормализованного изображения, на которую, 
как правило, не попадают ресницы и веки~\cite{5pav} (рис.~2).


Тем не менее для определения наличия века в этой области параметризации используется 
специальный алгоритм. 
Ищется максимум вертикальной производной яркости изображения 
$$
\underset{y}{\max}\sum\limits_x\left\vert \fr{\partial I(x,y)}{\partial y}\right\vert
$$
в области $[x_p-r/2,\,x_p+r/2][y_p+r,\,y_p+(R+r)/2]$, выделенной на рис.~3. 
Здесь~$(x_p,y_p)$~--- центр зрачка,
 $r$ и $R$~--- радиусы границ радужной оболочки. Если это 
значение больше порогового, то считается, что нижнее веко попало в область параметризации.

\begin{center} %fig3
\vspace*{6pt}
\mbox{%
\epsfxsize=80.281mm 
\epsfbox{pav-3.eps}
}
%\end{center}
%\vspace*{1pt}

{{\figurename~3}\ \ \small{Изображения $I(x,y)$ и $I_y^\prime(x,y)$}}
\end{center}
%\vspace*{-6pt}

%\bigskip
\addtocounter{figure}{1}

 
\section{Метод ключевых точек параметризации радужной оболочки}

В работе~\cite{5pav} показано, что наиболее информативными номерами двумерных функций 
преобразования Эрмита~$\varphi_{m,n}(x,y)$ для задачи идентификации по радужной оболочке 
являются номера~(1,\,0), (1,\,1), (2,\,0) в указанном порядке. Поэтому для параметризации данных 
радужной оболочки в данной работе была выбрана функция пре\-обра\-зо\-ва\-ния Эрмита~$\varphi_{1,0}(x,y)$. 

Рассмотрим в каждой точке области парамет\-ризации величину $F_1=M_{1,0}$. В~качестве кода 
радужной оболочки (ключевых точек) рассматри-\linebreak\vspace*{-12pt}
\pagebreak

\noindent
\begin{center} %fig4
%\vspace*{6pt}
\mbox{%
\epsfxsize=80mm %78.39mm
\epsfbox{pav-4.eps}
}
\end{center}
%\vspace*{-1pt}
{{\figurename~4}\ \ \small{Область параметризации радужной оболочки с кодом радужной оболочки}}

\vspace*{18pt}

 
\begin{center}
%\vspace*{12pt}
\mbox{%
\epsfxsize=80mm %78.39mm
\epsfbox{pav-5.eps}
}
\end{center}
%\vspace*{-1pt}
{{\figurename~5}\ \ \small{Примеры работы алгоритма выделения ключевых точек для изображений с наложением века и бликов на область 
параметризации}}
%\end{center}
%\vspace*{12pt}


\bigskip
\medskip
\addtocounter{figure}{2}

\noindent
ваются~$N$ ($N = 50$, 100, 150, 200) точек, 
разбитых на две группы: $N/2$ точек с максимальными значениями~$F_1$, удаленных друг от 
друга не менее чем на 2~пикселя, и аналогично~$N/2$~--- с минимальными значениями~$F_1$. 
Пример кода радужной оболочки для $N = 150$ ключевых точек приведен на рис.~4. 
Черными точками обозначены ключевые точки с максимальными значениями~$F_1$, белыми~--- 
с минимальными. На рис.~4 также обозначена граница возможных значений точек кода, 
отстоящая от краев области параметризации на полуширину области сосредоточения 
функции~$\varphi_1$.


Отметим, что этот метод эффективен только для изображений без наложения века на область 
параметризации, так как в области века и бликов большинство детектируемых точек не является 
точками радужной оболочки (рис.~5). 


При идентификации по радужной оболочке используются матрицы сравнения ключевых точек.
Для построения матрицы сравнения область пара\-мет\-ризации разбивается на непересекающиеся 
блоки размера $3\times 3$. Если в блок не попадает ни одной точки кода, то ему соответствует 
значение~0. Если в блок попадает хотя бы одна черная (белая) точка кода, то соответствующее 
значение матрицы сравнения равняется~1~($-1$), если и черная, и белая, то значение равняется 2. 
При сравнении двух матриц считается, что соответствующие блоки равны, если в них попали 
точки одного цвета (табл.~1).


Чтобы алгоритм был устойчив к поворотам глаза (поворот глаза соответствует циклическому 
сдвигу всего нормализованного изображения), сравниваются матрицы точек уменьшенного 
размера. Границы такой урезанной матрицы показаны на рис.~6 и сдвигаются у 
исследуемого изображения в обе стороны до границ возможных значений точек ко-\linebreak\vspace*{-12pt}
\columnbreak

%\bigskip

\noindent
\begin{center}
\noindent
\parbox{51mm}{{\tablename~1}\ \ \small{Сопоставление значений матриц сравнения: <<$+$>> означает, что блоки считаются 
равными, <<$-$>>~--- различными}}
\end{center}
%\vspace*{2ex}

\begin{center}
\tabcolsep=9pt
\begin{tabular}{|c|c|c|c|c|}
\hline
&0&1&$-1$&2\\
\hline
0&$+$&$-$&$-$&$-$\\
1&$-$&$+$&$-$&$+$\\
$-1$\hphantom{$-$}&$-$&$-$&$+$&$+$\\
2&$-$&$+$&$+$&$+$\\
\hline
\end{tabular}
\end{center}
\vspace*{12pt}

%\bigskip
\addtocounter{table}{1}

\begin{center} %fig6
\vspace*{6pt}
\mbox{%
\epsfxsize=80mm %78.39mm
\epsfbox{pav-6.eps}
}
\end{center}
%\vspace*{6pt}
{{\figurename~6}\ \ \small{Области сравнения матриц кодов радужных оболочек}}
%\end{center}
%\vspace*{-6pt}


\bigskip
\addtocounter{figure}{1}


\noindent
да. В~данной 
работе учитываются углы поворота от~$-10^\circ$ до ~$10^\circ$.



Алгоритм параметризации радужной оболочки по ключевым точкам протестирован на базе 
данных  CASIA-IrisV3~\cite{6pav}, содержащей 2655~изображений глаз. Результаты работы 
алгоритма приведены в табл.~2 и на рис.~7. Здесь CRR (Correct Recognition Rate)~--- вероятность верного распознавания.

\bigskip
%\vspace*{3pt}

%\begin{center}
\noindent
{{\tablename~2}\ \ \small{Результаты работы алгоритма ключевых точек}}
%\end{center}
%\vspace*{2ex}

{\small 
\begin{center}
\tabcolsep=8pt
\begin{tabular}{|c|c|c|c|}
\hline
\tabcolsep=0pt\begin{tabular}{c}Число\\ ключевых\\ точек $N$\end{tabular}&
\tabcolsep=0pt\begin{tabular}{c}Число\\ неверных\\ ближайших\\ изображений\\ (из 2655)\end{tabular}&
\tabcolsep=0pt\begin{tabular}{c}Неверные\\ из-за\\ наложения\\ века\\ и бликов\end{tabular}&
\tabcolsep=0pt\begin{tabular}{c}CRR,\\ \%\end{tabular}\\
\hline
200&$23 = 18 + 5$&18&99.81\\
150&$36 = 30 + 6$&30&99.77\\
100&\hphantom{9}$58 = 41 + 17$&41&99.35\\
\hphantom{9}50&$135 = 65 + 70$&65&97.36\\
\hline
\end{tabular}
\end{center}
}
\vspace*{12pt}


%\bigskip
\addtocounter{table}{1}


\begin{center} %fig7
\vspace*{6pt}
\mbox{%
\epsfxsize=74.136mm 
\epsfbox{pav-7.eps}
}
\end{center}
\vspace*{6pt}
{{\figurename~7}\ \ \small{График зависимости CRR от числа взятых ключевых точек}}
%\end{center}
%\vspace*{6pt}


%\bigskip
\addtocounter{figure}{1}

%\noindent


\section{Заключение}

В работе предложен алгоритм идентификации человека, использующий ключевые точки 
радужной оболочки глаза, найденные локальным методом преобразования Эрмита. Этот 
алгоритм позволяет получать достаточно хорошие результаты распознавания даже при 
небольшом объеме хранимой информации. Он достаточно перспективен для использования в 
мультибиометрических системах распознавания.

\bigskip
Работа выполнена при поддержке ФЦП <<Научные и научно-педагогические кадры 
инновационной России>> на 2009--2013~гг.

{\small\frenchspacing
{%\baselineskip=10.8pt
\addcontentsline{toc}{section}{Литература}
\begin{thebibliography}{9}

\bibitem{1pav}
\Au{Martens J.\,B.}
The Hermite transform-theory~// IEEE Transactions on Acoustics, Speech, and Signal Processing, 1990. 
Vol.~38. No.\,9. P.~1595--1606.

\bibitem{2pav}
\Au{Ma~L., Tan~T., Zhang~D., Wang~Y.}
Local intensity variation analysis for iris recognition~// Pattern Recognition, 2004. Vol.~37. No.\,6. 
P.~1287--1298.

\bibitem{3pav}
\Au{Wang L., Dai~M.}
Extraction of singular points in fingerprints by the distribution of Gaussian--Hermite moment~// IEEE 
1st Conference (International) DFMA Proceedings, 2005. P.~206--209.

\bibitem{4pav}
\Au{Estudillo-Romero~A., Escalante-Ramirez~B.}
The Hermite transform: An alternative image representation model for iris recognition~// LNCS, 2008. 
No.\,5197. P.~86--93.


\bibitem{5pav}
\Au{Павельева Е.\,А., Крылов~А.\,С., Ушмаев~О.\,С.}
Развитие информационной технологии идентификации человека по радужной оболочке глаза на 
основе преобразования Эрмита~// Системы высокой доступности, 2009. №\,1. С.~36--42.

\bibitem{6pav}
База данных CASIA-IrisV3. {\sf  http://www.cbsr.ia.ac.cn/ IrisDatabase.htm}.

\label{end\stat}

\bibitem{7pav}
\Au{Krylov A.\,S., Pavelyeva~E.\,A.}
Iris data parametrization by Hermite projection method~// GraphiCon'2007 Conference Proceedings, 2007. P.~147--149. 
 \end{thebibliography}
}
}
\end{multicols}


         %9
\def\stat{basha}

\def\tit{АЛГОРИТМ АВТОМАТИЧЕСКОГО ВЫДЕЛЕНИЯ ЛИЦА 
НА~ТЕРМОГРАФИЧЕСКИХ ИЗОБРАЖЕНИЯХ$^*$}

\def\titkol{Алгоритм автоматического выделения лица 
на~термографических изображениях}

\def\autkol{Н.\,С.~Баша, Л.\,А.~Шульга}
\def\aut{Н.\,С.~Баша$^1$, Л.\,А.~Шульга$^2$}

\titel{\tit}{\aut}{\autkol}{\titkol}

{\renewcommand{\thefootnote}{\fnsymbol{footnote}}\footnotetext[1]
{Статья подготовлена по результатам работы секции 
<<Биометрия>> 20-й Международной конференции по компьютерной графике и зрению Графикон-2010, 
г.~С.-Петербург, 20--24~сентября 2010~г.}}

\renewcommand{\thefootnote}{\arabic{footnote}}
\footnotetext[1]{Научно-исследовательский институт прикладной акустики, Международный университет природы, 
общества и человека <<Дубна>>, natalia.basha@niipa.ru}
\footnotetext[2]{Научно-исследовательский институт прикладной акустики, luda.shulga@niipa.ru}

\vspace*{-6pt}

\Abst{Представлен подход к исследованию термографических изображений человека 
для задач интеллектуального видеонаблюдения. Предложен алгоритм автоматического 
выделения лица в инфракрасном (ИК) спектре излучения, приведены результаты его работы и 
проведен анализ эффективности на базе данных, состоящей из 103 термографических портретов 
15~человек разного пола, возраста и телосложения, сделанных в различных условиях 
окружающей среды.}

\vspace*{-4pt}

\KW{распознавание образов; анализ изображений; системы видеонаблюдения; термография; 
детекция лица}

%            \vspace*{-4pt}

      \vskip 10pt plus 9pt minus 6pt

      \thispagestyle{headings}

      \begin{multicols}{2}
      
            \label{st\stat}
            


\section{Введение}
  
  В настоящее время для задач видеонаблюдения все чаще применяются 
устройства, позволяющие улавливать ИК излучение объекта и получать 
его температурные карты (термограммы). Для получения термографических 
изображений используются специальные тепловизионные камеры (тепловизоры 
или термографы). Существенными преимуществами их использования по 
сравнению с видеокамерами являются:
  \begin{itemize}
\item нечувствительность к освещенности объекта и способность работать в 
полной темноте;
\item способность давать вполне приемлемое для опознавания изображение даже 
при значительном удалении от человека;
\item нечувствительность к внешней маскировке (например, элементам макияжа).
\end{itemize}

  Эти отличительные черты позволяют применять термографию в тех условиях, 
когда получение изображения с видеокамеры недостаточно для реализации 
поставленных целей~[1, 2].
  
  Работы, связанные с задачами распознавания лиц в  
  ИК-диа\-па\-зо\-не, ведутся последние 10~лет и решаются с помощью 
высокочувствительных видеокамер, работающих в отра\-жен\-ном ИК-диа\-па\-зо\-не. 
Возможность применять тепловизионные камеры для данного рода исследований 
появилась недавно. 
  
  Информационными признаками в термографии служат подкожные рисунки 
артерий и вен, которые уникальны и неизменны для каждого человека, так как 
сосудистый рисунок не зависит от температуры лица, пластических операций и 
фактора старения человека.

\vspace*{-6pt}

\section{Методы исследования}

\vspace*{-2pt}

\subsection{База данных}

  Для апробации предложенного метода детекции лица была собрана база 
термографических изображений. Все изображения, приведенные в статье (рис.~1), 
получены термографом <<ИРТИС-2000МЕ>> и представляют собой матрицу 
температур размером $240\times 320$, снятых в спектральном диапазоне 3--5~мкм, с точностью изменения температур 0,01~$^\circ$C. База 
термографических данных состоит из\linebreak
 \begin{center} %fig1
\vspace*{2pt}
\mbox{%
\epsfxsize=78mm
\epsfbox{bas-1.eps}
}
\end{center}
\vspace*{2pt}
%\begin{center}
{{\figurename~1}\ \ \small{Изображение лица, полученное термографом <<ИРТИС-2000МЕ>>: 
(\textit{а})~двумерное изображение термограммы лица; (\textit{б})~трехмерное отобра\-же\-ние 
термограммы лица}}
%\end{center}
%\vspace*{9pt}

%\bigskip
\addtocounter{figure}{1}

    \begin{figure*} %fig2
    \vspace*{1pt}
\begin{center}
\mbox{%
\epsfxsize=161.999mm
\epsfbox{bas-2.eps}
}
\end{center}
\vspace*{-6pt}
\Caption{Изображения лица, полученное термографом: (\textit{а})~гипертермия области 
внутренних углов глаз (норма); (\textit{б})~симметричная гипертермия в области ключиц, на 
границе с одеждой; (\textit{в})~гипертермия в области носа и рта; (\textit{г})~гипертермия в 
области лба, частично разрываемая волосами
  \label{f2-ba}}
  \end{figure*}
  
\noindent
 103~снимков 15~людей разного возраста, 
пола и телосложения. Термограммы получены в разное время суток, при разных 
условиях окружающей среды (в помещении и на улице). Сбор данных проводился 
еженедельно в течение 4~месяцев.


\subsection{Алгоритм автоматического поиска лиц на термографических 
изображениях}
  
  Проведено исследование физиологических особенностей лица человека с целью 
выделения универсальной и стабильной температурной области, которую можно 
использовать как область привязки при выделении области лица.



В результате исследования было выяснено, что внутренний угол глаза здорового 
человека дает наиболее интенсивный отклик в ИК-диапазоне (рис.~1 
и~\ref{f2-ba},\,\textit{а}). Экспериментально были замечены некоторые 
отклонения от общей тенденции. Основные причины: симметричные зоны, 
связанные с гипертермией на границе открытых и закрытых участков тела 
(рис.~\ref{f2-ba},\,\textit{б}), с воспалительными процессами 
(рис.~\ref{f2-ba},\,\textit{в}), с вегетососудистой дистонией (рис.~\ref{f2-ba},\,\textit{г}). 
Результаты наблюдений были учтены при разработке алгоритма 
детекции лица. Поэтому в качестве опорных точек области интереса были 
выбраны точки внутренних углов глаз (рис.~3,\,\textit{а}). 
Использование данных меток эффективно потому, что с их помощью по 
коэффициентам про\-пор\-ци\-о\-наль\-ности на лице можно вычислить размеры самого 
лица (рис.~3,\,\textit{б}) и расположение основных деталей~[3].

 
%\noindent

  Рассмотрим алгоритм автоматического выделения области лица, базирующийся 
на детекции внут\-рен\-них углов глаз. Анализируя тепловые профили, 
соответствующие зоне внутренних углов глаз, было установлено, что данные 
точки на профиле\linebreak
 \begin{center} %fig3
\vspace*{1pt}
\mbox{%
\epsfxsize=80.23mm
\epsfbox{bas-3.eps}
}
\end{center}
\vspace*{4pt}
%\begin{center}
{{\figurename~3}\ \ \small{Алгоритм автоматической детекции лица: (\textit{а})~выставление меток, 
соответствующих внутренним углам глаз; (\textit{б})~наложение маски коэффициентов 
пропорциональности лица человека; (\textit{в})~результат детекции лица}}
%\end{center}
%\vspace*{9pt}

\bigskip
\addtocounter{figure}{1}

\noindent
 пред\-став\-ля\-ют\-ся в виде двух четко выраженных пиков, 
симметричных относительно серединной линии лица. Тепловой профиль по 
строке, соответствующей зоне внутренних углов глаз, приведен на рис.~4. 




  Разработанный алгоритм выделения области лица состоит из нескольких 
этапов.
  
  \medskip
  
  \noindent
  {\bfseries\textit{Первый этап: пороговая фильтрация}}
  \smallskip
  
  Первым этапом является предварительная обработка изображения методом 
пороговой фильтрации с целью выделения из окружающей среды объекта, 
который может оказаться человеком. Значение порога было установлено 
экспериментально на уровне $\min + (\max - \min)/3$, где $\min$ и $\max$~--- 
соответственно минимальное и максимальное значения температуры на термограмме. Для 
человека это значение порога соответствует температуре 
кожи и не включает одежду и волосы. Для каждой строки изображения находится 
левая и правая граница\linebreak\vspace*{-12pt}
\pagebreak

\end{multicols}

\begin{figure} %fig4
  \vspace*{1pt}
\begin{center}
\mbox{%
\epsfxsize=159.968mm
\epsfbox{bas-4.eps}
}
\end{center}
\vspace*{-6pt}
\Caption{Тепловой профиль по строке, соответствующей зоне внутренних углов глаз: (\textit{а})~часть 
исходного изображения термограммы лица (линией выделена строка, содержащая 
внутренние углы глаз); (\textit{б})~соответствующий ей тепловой профиль
\label{f4-ba}}
\end{figure} 

\begin{figure} %fig5
  \vspace*{3pt}
\begin{center}
\mbox{%
\epsfxsize=123.914mm
\epsfbox{bas-5.eps}
}
\end{center}
\vspace*{-6pt}
  \Caption{Процесс выделения лица: (\textit{а})~этап~I: пороговая фильтрация и определение 
верхней границы лица; (\textit{б})~этап~II: определение моды правой и левой границ лица; 
(\textit{в})~этап~III: выделение и корректировка линии глаз
  \label{f5-ba}}
  \vspace*{3pt}
  \end{figure}

\begin{multicols}{2}


\noindent
 области предполагаемого лица, а также точка~$Y_t$ для 
всего изображения, соответствующая верхней точке потенциальной области лица 
(рис.~\ref{f5-ba},\,\textit{а}).
  
  \medskip

  \noindent
    {\bfseries\textit{Второй этап: вычисление интервала для нахождения линии 
глаз}}
  
  \smallskip
  
  На втором этапе вычисляется мода для левой и правой границы 
предполагаемой области лица ($X_l$ и $X_r$ соответственно), мода для средней 
линии об\-ласти~$X_{mid}$ (как мода середины отрезка между правой и левой 
границей каждой конкретной строки выделенной области изображения) и мода 
ширины области~$W$ (как мода разности между правой и левой границей каждой 
конкретной строки выделенной области изображения, рис.~\ref{f5-ba},\,\textit{б}). 
Экспериментально было установлено, что значение моды ширины области 
предполагаемого лица соответствует ширине реального лица на уровне глаз.
  
  Исходя из пропорций лица, значения верхней точки головы и значения средней 
ширины рассчитывается теоретическое значение линии глаз
  $$
  Y_{\mathrm{eye}}= Y_t+\mathrm{round}\,\left(0{,}68\times W\right)\,.
  $$
  
%  \end{multicols}
  

  
  Экспериментально установлено, что для на\-хож\-де\-ния реального уровня глаз 
необходимо задать окрестность~$b$. Таким образом, реальный уровень глаз 
будет находиться в диапазоне (рис.~\ref{f5-ba},\,\textit{в})
  \begin{equation}
  \left [ Y_{\mathrm{eye}}-b;\, Y_{\mathrm{eye}}+b\right]\,.
  \label{e1-ba}
  \end{equation}
  

 
%  \smallskip


    \noindent
    {\bfseries\textit{Третий этап: поиск линии глаз}}
  \smallskip
  
  Как было установлено выше, глаза являются зонами гипертермии. Поэтому на 
выделенном интервале~(1) ведется поиск значения максимума 
температуры~$T_{\max}$. Затем построчно осуществляется проход маской с двумя 
пиками (обозначаемыми~$P_1$ и~$P_2$) и впадиной ($H$) всего выделенного 
интервала. Среди строк, удовлетворяющих условиям маски, в качестве линии глаз 
выбирается строка~$y_0$, где (см.\ рис.~\ref{f4-ba}):
  \begin{itemize}
\item значение одного из пиков равно максимуму на интервале

\noindent
$$
t_{P_1}=T_{\max}\parallel t_{P_2} -T_{\max}\,;
$$
\item пики симметрично расположены относительно средней линии

\noindent
$$
x_{P_1}-W=W-x_{P_2}\,;
$$
\item пики симметрично расположены относительно соответствующих 
границ лица

\noindent
$$
x_{P_1}-X_l=X_r-x_{P_2}\,;
$$
\item впадина расположена на средней линии

\noindent
$$
x_H=X_{\mathrm{mid}}\,;
$$
\item разница температур между пиком и впадиной больше порогового 
значения (экспериментально было установлено пороговое значение, равное 
0,5~$^\circ$C):

\noindent
$$
T_{\max}-t_H>0{,}5~^\circ\mathrm{C}\,,
$$
где $x_{P_1}$, $x_H$ и $x_{P_2}$~--- значения абсцисс точек $P_1$, $H$ и $P_2$, а
$t_{P_1}$, $t_H$ и~$t_{P_2}$~--- значения температур в точках ($x_{P_1}, y_0)$, ($x_H, y_0)$ и
($x_{P_2}, y_0)$ соответственно.
\end{itemize}

\medskip

  \noindent
    {\bfseries\textit{Четвертый этап: определение области лица}}
 \smallskip
  
    На изображении расставляются маркеры глаз (соответствующие выбранным 
пикам~$P_1$ и~$P_2$). Вычисляется ширина области предполагаемого лица
на уровне глаз и по ней рассчитывается верхний
 угол прямоугольника, заключающего потенциальное лицо, его длина и ширина. Выделенная таким 
образом область принимается за лицо.

\setcounter{figure}{6}
\begin{figure*} %[b] %fig7
    \vspace*{1pt}
\begin{center}
\mbox{%
\epsfxsize=162.7mm
\epsfbox{bas-7.eps}
}
\end{center}
\vspace*{-6pt}
\Caption{Результат работы алгоритма автоматического выделения области лица: 
(\textit{а})~термограмма; (\textit{б})~соответствующее ей изображение в видимом диапазоне; 
(\textit{в})~результат работы алгоритма выделения лица
\label{f7-ba}}
\end{figure*}
  
  
  Предложенный метод автоматической детекции области лица проверен на базе, 
состоящей из 103~изображений 15~человек. Результат выделения лица 
продемонстрирован на рис.~6. По данной выборке доля верного 
выделения лиц составила 98\%. Анализ ошибочно детектированных изображений 
показал, что ошибка связана с размытостью исходной термограммы (по причине 
движения объекта во время съемки). В~экспериментальных исследованиях 
установлено, что на качество детекции лица конкретного человека не влияют ни 
изменение\linebreak\vspace*{-16pt}

\columnbreak

%\vspace*{-12pt}
    \begin{center} %fig6
\vspace*{1pt}
\mbox{%
\epsfxsize=79mm
\epsfbox{bas-6.eps}
}
\end{center}
%\vspace*{2pt}
%\begin{center}
{{\figurename~6}\ \ \small{Результат работы алгоритма автоматического выделения области лица}}
%\end{center}
%\vspace*{9pt}

\medskip
%\addtocounter{figure}{1}

\addtocounter{figure}{1}

%\pagebreak

\noindent

\noindent
прически и волосяного покрова лица (наличия и отсутствия бороды, 
усов), ни температурные условия сбора изображений. Сбор проводился в 
лабораторных условиях и на открытом воздухе в зимнее время.
В обоих случаях выделение области 
лица проходило успешно. 

  
  
  Проведена временн$\acute{\mbox{а}}$я оценка работы алгоритма: пакетная 
обработка 103~изображений составила 157~с; таким образом, среднее время 
выделения одного лица по термограмме составляет 1,52~с (в среде MatLab).
  
  С целью расширения класса термограмм, на которых применим разработанный 
метод выделения лица, приведем результаты детекции для термограмм
различных объектов. 
  
  
  На рис.~\ref{f7-ba},\,\textit{а} представлен термографический снимок, на 
котором лицо конкурирует по температуре с другим объектом. Для обработки 
этого термографического снимка сначала производилась операция 
сегментирования областей, похожих на лицо (выделены 2~области: <<лицо>> и 
<<чайник>>, находящиеся в одном температурном диапазоне и схожие по 
площади). Далее применялся разработанный алгоритм выделения лица. Объект 
<<чайник>> был отбракован на третьем этапе: не были найдены маркеры линии глаз. На рис.~\ref{f7-ba},\,\textit{в} 
представлен результат детекции лица.

\vspace*{-6pt}

\section{Выводы}

  Предложен алгоритм автоматического выделения лица на термографических 
изображениях. Описанный алгоритм базируется на поиске внут\-рен\-них углов глаз, 
которые являются стабильной гипертермической областью на лице человека. 
Высокий показатель правильной детекции лица (98\%) подтвердил эффективность 
представленного алгоритма.
  
  Стоит отметить, что данный алгоритм можно использовать как алгоритм 
автоматического определения присутствия человека в кадре, например для 
систем контроля доступа и охраны периметра.
  
  В~дальнейшем рассмотренный подход можно применять для идентификации 
личности в интеллектуальных системах видеонаблюдения.

\vspace*{-6pt}

{\small\frenchspacing
{%\baselineskip=10.8pt
\addcontentsline{toc}{section}{Литература}
\begin{thebibliography}{9}

\bibitem{1-ba}
\Au{Evans~D.}
Infrared facial recognition technology being pushed toward emerging applications~// 
Proc. SPIE, 1997. Vol.~2962. P.~276--286.

\bibitem{2-ba}
\Au{Иваницкий Г.\,Р.}
Современное матричное тепловидение в биомедицине~// Успехи физических 
наук, 2006. Т.~176. №\,12. С.~1293--1320.

 \label{end\stat}

\bibitem{3-ba}
\Au{Куприянов В.\,В., Стовичек~Г.\,В.}
Лицо человека: Анатомия, мимика.~--- М.: Медицина, 1988.
 \end{thebibliography}
}
}


\end{multicols}   %10
\def\stat{kudr}

\def\tit{ПРИБЛИЖЕННЫЕ МЕТОДЫ РЕШЕНИЯ ЗАДАЧИ ДИАГНОСТИКИ ПЛОСКИМ 
ЗОНДОМ СИЛЬНОИОНИЗОВАННОЙ ПЛАЗМЫ С~УЧЕТОМ КУЛОНОВСКИХ 
СТОЛКНОВЕНИЙ}

\def\titkol{Приближенные методы решения задачи диагностики плоским 
зондом сильноионизованной плазмы} %с~учетом Кулоновских  столкновений}

\def\autkol{И.\,А.~Кудрявцева, А.\,В.~Пантелеев}
\def\aut{И.\,А.~Кудрявцева$^1$, А.\,В.~Пантелеев$^2$}

\titel{\tit}{\aut}{\autkol}{\titkol}

%{\renewcommand{\thefootnote}{\fnsymbol{footnote}}\footnotetext[1]
%{Работа поддержана Российским фондом фундаментальных исследований
%(проекты 11-01-00515а и 11-07-00112а), а также Министерством
%образования и науки РФ в рамках ФЦП <<Научные и
%научно-педагогические кадры инновационной России на 2009--2013~годы>>.}}


\renewcommand{\thefootnote}{\arabic{footnote}}
\footnotetext[1]{Московский авиационный институт, irina.home.mail@mail.ru}
\footnotetext[2]{Московский авиационный институт, avpanteleev@inbox.ru}

\vspace*{-2pt}

\Abst{Сформирована математическая модель, описывающая динамику сильноионизованной 
плазмы с учетом столкновений заряженных частиц вблизи плоского зонда. Модель включает уравнение 
Фоккера--Планка и уравнение Пуассона. Предложено два подхода к решению задачи: на основе метода 
статистических испытаний Мон\-те-Кар\-ло и на основе композиции метода крупных частиц и метода 
расщепления.} 

\vspace*{-2pt}

\KW{телекоммуникационные системы; метод Монте-Карло; метод крупных частиц; метод 
расщепления; зонд; уравнение Фоккера--Планка; уравнение Пуассона} 

\vspace*{-4pt}

 \vskip 8pt plus 9pt minus 6pt

      \thispagestyle{headings}

      \begin{multicols}{2}
      
            \label{st\stat}

\section{Введение}

В настоящее время в области телекоммуникаций все более востребованными становятся 
информационные технологии, основанные на использовании математических моделей и численных 
методов физики плазмы. Поэтому особенно актуальным является решение разнообразных задач анализа 
поведения плазмы, включающих в себя формирование новых моделей и методов их исследования. 
Помимо этого, в разработке телекоммуникационного оборудования эффективно используются 
собственно физические свойства плазмы. В~частности, изготовлена антенна, работа которой основана 
на газовом разряде низкотемпературной плазмы~[1], интенсивно ведутся разработки по созданию и 
усовершенствованию источников бесперебойного питания на основе плазменных элементов~[2, 3]. 
      
      Одним из наиболее перспективных направлений для построения систем оптической 
беспроводной связи является использование лазеров~\cite{4-k, 5-k}. В~этой связи большое внимание 
уделяется использованию плазмы при разработке импульсных сильноточных коммутаторов~\cite{6-k}, 
так как практическое применение подобных разработок требует повышения уровня надежности и 
быстродействия лазерных систем.
      
      Исследования низкотемпературной плазмы также связаны с разработками в области дальней 
космической связи, так как моделирование процессов взаимодействия заряженного тела с верхними 
слоями атмосферы позволяет предлагать способы улучшения существующих систем радиосвязи с 
космическими летательными аппаратами~\cite{7-k}. 
      
      Наряду с этим актуальными также являются задачи диагностики плазмы, поскольку перспективы 
ее использования в области телекоммуникаций после более полного изучения физических свойств 
могут значительно расшириться. 

Для диагностики плазмы применяют зондовые методы исследования~[8--11]. Эти методы относятся к 
классу контактных методов; как следствие, возникает сложность в исследовании пристеночной области 
вблизи зонда, которая характеризуется достаточно сложным распределением потенциала и функциями 
распределения, отличными от максвелловских. 

Данная работа посвящена исследованию переходного режима обтекания заряженного тела плазмой. Для 
переходного режима выполняется следующее условие: длина свободного пробега иона до столкновения 
с нейтральным атомом или другим ионом невелика по сравнению с характерными размерами тела. 
В~этом случае возникает необходимость учета столкновений заряженных частиц с нейтральными 
атомами и кулоновских столкновений. В~работах~\cite{10-k, 11-k} подробно рассмотрена модель с 
учетом столкновений заряженных частиц с нейтральными атомами. В~настоящей статье представлена 
теоретическая модель, описывающая влияния ион-ионных и ион-элек\-т\-рон\-ных столкновений на 
измеряемые характеристики плазмы, что ранее детально не исследовалось.
      
      В~рамках данной работы предлагается модель, описывающая динамику сильноионизованной 
плазмы с учетом кулоновских столкновений. Эта модель учитывает такие процессы взаимодействия, 
как перенос частиц и столкновения между заряженными частицами типа <<ион--ион>> и 
      <<ион--электрон>> под влиянием макроскопического электрического поля. Перечисленные 
процессы описываются самосогласованной системой уравнений, включающей уравнение 
      Фок\-ке\-ра--План\-ка и уравнение Пуассона~[12].
      
      Вычислительная модель задачи строится на основе двух методов: метода статистических 
испытаний Мон\-те-Кар\-ло и композиции метода крупных частиц и метода расщепления. Приведены 
результаты численного моделирования, полученные с использованием вышеперечисленных методов.

\vspace*{-4pt}

\section{Постановка задачи}

\vspace*{-2pt}

Рассматривается следующая физическая постановка зондовой задачи~[11]. В~невозмущенную 
бесконечно протяженную плазму, состоящую из электронов и однозарядных ионов, внесена большая\linebreak 
заряженная до потенциала $\varphi_p$ плоскость. Плоскость, расположенная поперек потока плазмы, 
является идеально поглощающей для электронов. Ионы при ударе о плоскость нейтрализуются. 
Предполагается, что частицы в плазме движутся под действием внешнего электрического поля, 
магнитное поле отсутствует. Концентрации ионов $n_{i\infty}$ и электронов $n_{e\infty}$, а также 
температуры данных час\-тиц~$T_{i\infty}$ 
и~$T_{e\infty}$ в невозмущенной плазме заданы. За начальные 
функции распределения обоих типов час\-тиц принимаются функции распределения Максвелла. 
      
      Требуется с учетом столкновений между заряженными частицами найти напряженность 
самосогласованного электрического поля $\vec{E}(\vec{r},t)$, функции распределения однозарядных 
ионов $f_i(\vec{r}, \vec{v}, t)$ и электронов $f_e(\vec{r}, \vec{v}, t)$, 
а также их моменты (плотности 
токов ионов и электронов  $j_i(\vec{r},t)\hm
=q\int f_i(\vec{r}, \vec{v}, t)\vec{v}\,d\vec{v}$, $j_e(\vec{r},t) 
\hm={\sf e}\int f_e(\vec{r},\vec{v},t)\vec{v}\,d\vec{v}$, где $q=Z_i{\sf e}$, $Z_i=1$~--- заряд иона, ${\sf 
e}$~--- заряд электрона; концентрации ионов и электронов $n_i(\vec{r},t)\hm=\int 
f_i(\vec{r},\vec{v},t)\,d\vec{v}$, $n_e(\vec{r},t)\hm=\int f_e(\vec{r},\vec{v}, t)\,d\vec{v}$). 
Поведение частиц во 
времени~$t$ характеризуется ра\-ди\-ус-век\-то\-ром~$\vec{r}$ и вектором скорости~$\vec{v}$.
      
      Математическая модель, соответствующая данной физической постановке задачи, имеет 
вид~\cite{11-k, 13-k}:

\noindent
      \begin{equation}
      \left.
      \begin{array}{c}
      \fr{\partial f_\alpha (\vec{r},\vec{v},t)}{\partial t}+
      \vec{v}\fr{\partial f_\alpha (\vec{r},\vec{v},t)}{ 
\partial \vec{r}}+
\fr{\vec{F}_\alpha(\vec{r},t)}{m_\alpha}\times{}\\[4pt]
{}\times\fr{\partial f_\alpha(\vec{r},\vec{v},t)}{ \partial 
\vec{v}}=
\left(\fr{\partial f_\alpha(\vec{r},\vec{v},t)}{ \partial t}\right)_{\mathrm{с}}+S_\alpha 
(\vec{r},\vec{v},t)\,;\\[6pt]
      \Delta\varphi(\vec{r},t)=-\fr{{\sf e}}{\varepsilon_0}\left( n_i(\vec{r},t)-n_e(\vec{r},t)\right)\,;\\[6pt]
      \vec{E}(\vec{r},t)=-\nabla \varphi(\vec{r},t)\,.
      \end{array}\!\!
      \right\}\!\!
      \label{e1-k}
      \end{equation}
Здесь первое уравнение~--- уравнение Фок\-ке\-ра--План\-ка для частиц сорта~$\alpha$ ($\alpha=i,e$), 
второе~--- уравнение Пуассона для самосогласованного электрического поля; 
$f_\alpha(\vec{r},\vec{v},t)$~--- функция\linebreak
распределения час\-тиц сорта~$\alpha$; $(\partial 
f_\alpha(\vec{r},\vec{v},t)/\partial t)_{\mathrm{с}}$~--- 
оператор столкновений Фок\-ке\-ра--План\-ка; 
функция~$S_\alpha(\vec{r},\vec{v},t)$ описывает источники или стоки\linebreak
 час\-тиц; 
$\vec{F}_\alpha(\vec{r},t)=q_\alpha\vec{E}(\vec{r},t)$, где $\vec{E}(\vec{r},t)$~--- напряженность 
самосогласованного электрического поля, 
$$
q_\alpha =
\begin{cases}
-{\sf e}\,, & \alpha=e\,,\\
{\sf e}\,, & \alpha=i\,;
\end{cases}
$$
$\varphi(\vec{r},t)$~--- потенциал самосогласованного электрического поля; $n_\alpha(\vec{r},t)$ ($\alpha 
\hm=i,e$)~--- концентрация частиц сорта~$\alpha$; $m_\alpha$~--- масса частицы сорта~$\alpha$; 
$\varepsilon_0$~--- электрическая постоянная. 

Оператор столкновений Фок\-ке\-ра--План\-ка имеет вид~\cite{13-k, 14-k}
\begin{multline*}
\fr{1}{\Gamma_\alpha}\left( \fr{\partial f_\alpha}{\partial t}\right)_{\mathrm{с}} 
=\fr{1}{2}\,\nabla_v\nabla_v:\left(f_\alpha\nabla_v\nabla_vg_\alpha(\vec{r},\vec{v},t)\right)-{}\\
{}-
\nabla_v\cdot\left(f_\alpha\nabla_v h_\alpha\right)\,,
\end{multline*}
где $\nabla_v\nabla_v g_\alpha(\vec{r},\vec{v},t)$~--- ковариантная тензорная производная второго ранга, 
знак двоеточия ($:$) обозначает операцию двойного суммирования:
\begin{gather*}
\Gamma_\alpha=\fr{Z_\alpha^4 {\sf e}^4}{4\pi \varepsilon_0^2 m^2_\alpha}\,\ln D_\alpha\,;
\\
D_\alpha =\fr{12\pi\varepsilon_0 kT_{\alpha\infty}}{Z_\alpha^2 {\sf e}^2}\left( \fr{\varepsilon_0 k 
T_{e\infty}}{n_{e\infty} {\sf e}^2}\right)^{1/2}\,;\\
g_\alpha (\vec{r},\vec{v},t)=\sum\limits_{b=i,e}\left( \fr{Z_b}{Z_\alpha}\right) \int f_b 
(\vec{r},{\vec{v}}^{\,\prime},t)\left\vert \vec{v}-{\vec{v}}^{\,\prime}\right\vert\,d\vec{v}^{\,\prime}\,;\\
h_\alpha (\vec{r},\vec{v},t)=\sum\limits_{b=i,e} \fr{m_\alpha+m_b}{m_b} 
\left(\fr{Z_b}{Z_\alpha}\right)
\int
\fr{f_b(\vec{r},{\vec{v}}^{\,\prime}, t)}{\vert \vec{v}-{\vec{v}}^{\,\prime}\vert}
\,d{\vec{v}}^{\,\prime}\,;\\
Z_\alpha =1\,, \quad \alpha=i,e\,.
\end{gather*}
 
К системе уравнений~(\ref{e1-k}) необходимо добавить начальные и краевые условия:
\begin{equation}
\!\left.
\begin{array}{rrl}
t=0:\ & f_\alpha(\vec{r},\vec{v},0)&=f_\alpha^{\mathrm{maksv}}\,,\enskip \alpha=i,e;\\[9pt]
\vec{r}\in \Omega_p:\ & f_\alpha(\vec{r},\vec{v},t)\big\vert_{\vec{r}\in\Omega_p}&=0\,,\enskip \alpha=i,e\,;\\[9pt]
&\varphi(\vec{r},t)\big\vert_{\vec{r}\in\Omega_p}&=\varphi_p\,;\\[9pt]
\vec{r}\in\Omega_\infty:\ & 
f_\alpha(\vec{r},\vec{v},t)\big\vert_{\vec{r}\in\Omega_\infty}&= %{}\\[9pt]
f_\alpha^{\mathrm{maksv}}\,,\enskip \alpha=i,e\,;\\[9pt]
&\varphi(\vec{r},t)\big\vert_{\vec{r}\in\Omega_\infty}&=0\,,
\end{array}\!\!
\right\}\!\!\!\!
\label{e2-k}
\end{equation}
    где 
    
    \noindent
    \begin{multline*}
    f_\alpha^{\mathrm{maksv}}=n_{\alpha\infty}\left(\fr{m_\alpha}{2k\pi T_{\alpha\infty}}\right)^{3/2}\times{}\\
    {}\times
    \exp\left( -
\fr{m_\alpha}{2kT_{\alpha\infty}}\left\vert\vec{v}-\vec{v}_\infty\right\vert^2\right)\,,
\enskip \alpha=i, e\,;
\end{multline*} 
$\Omega_p$ и $\Omega_\infty$~--- множество радиус-векторов час\-тиц, концы которых принадлежат плоскости зонда и 
границе возмущенной зоны соответственно.

Для решения поставленной задачи введем декартову систему координат таким образом, чтобы 
заряженная плоскость совпала с плоскостью~$0xz$. Тогда положение частицы в пространстве будет 
определяться координатами $x,y,z$, а скорость~--- координатами $v_x, v_y, v_z$. В~силу того что 
плоскость является бесконечно большой в сравнении с характерным размером задачи, функции 
распределения частиц будут зависеть только от переменных $y, v_y, t$.

Поставленную задачу предлагается решать независимо двумя методами. Первый метод основывается на 
методе статистических испытаний Мон\-те-Кар\-ло, второй метод является композицией метода 
расщепления и метода крупных частиц.

\section{Применение метода Монте-Карло}

Запишем самосогласованную систему уравнений~(\ref{e1-k}) и~(\ref{e2-k}) в декартовой системе 
координат с учетом сделанных предположений:
\begin{equation}
\left.
\begin{array}{l}
\fr{\partial f_\alpha}{\partial t}+
v_y\fr{\partial f_\alpha}{\partial y}+\fr{F_y^\alpha}{m_\alpha}\,\fr{\partial 
f_\alpha}{\partial v_y}=\fr{1}{2}\,\fr{\partial^2 }{\partial [v_y]^2}\times{}\\
{}\times \left( 
f_\alpha\fr{\partial^2 g_\alpha  }{\partial [v_y]^2}\right) -
\fr{\partial}{\partial v_y}\left( f_\alpha\fr{\partial h_\alpha}{\partial v_y}\right)\,,
\enskip \alpha=i,e\,;\\[6pt]
    \fr{\partial^2\varphi}{\partial y^2} =-\fr{{\sf e}}{\varepsilon_0}\left(n_i-n_e\right)\,;
    \enskip E_y=-
\fr{\partial\varphi}{\partial y}\,;\\[6pt]
\hspace*{3.1mm}    t=0:\  \hspace*{2.6mm}f_\alpha(y,v_y,0)=f_\alpha^{\mathrm{maksv}}\,,\ \alpha=i,e\,;\\[9pt]
\hspace*{2.9mm} y=0:\ \hspace*{2.8mm}f_\alpha(0,v_y,t)=0\,,\ \alpha=i,e\,;\\[9pt]
\hspace*{24.3mm}\varphi(0,t)=\varphi_p\,;\\[9pt]
y=y_\infty:\ f_\alpha(y_\infty, v_y, t)=f_\alpha^{\mathrm{maksv}}\,,\ \alpha=i,e\,;\\[9pt]
\hspace*{21.5mm}\varphi(y_\infty, t)=0\,.
\end{array}
\right \}
\label{e3-k}
\end{equation}

В полученной системе уравнений~(\ref{e3-k}) перейдем к безразмерным величинам, применив 
соотношение $X=M_X \hat{X}$, где $M_X$~--- масштаб размерной величины~$X$, $\hat{X}$~--- 
безразмерная величина~$X$. В~качестве используемых масштабов были взяты следующие: радиус 
Дебая, скорость теплового движения частиц, концентрация частиц в невозмущенной плазме, потенциал, 
возникающий при разделении зарядов в дебаевской сфере, и производные от них величины.

Система безразмерных уравнений имеет следующий вид:
%\noindent
\begin{equation}
\left.
\begin{array}{l}
\fr{\partial 
\hat{f}_\alpha}{\partial\hat{t}}+A_\alpha\fr{\partial\hat{f}_\alpha}{\partial\hat{y}}+
B_\alpha\hat{E}_y\fr{\partial\hat{f}_\alpha}{\partial \hat{v}_y}={}\\
\!{}=
\fr{\partial^2}{\partial[\hat{v}_y]^2}\left(D_\alpha 
\hat{f}_\alpha\right)-\fr{\partial}{\partial\hat{v}_y}\left(K_\alpha \hat{f}_\alpha\right),\enskip 
\alpha=i,e;\\[9pt]
\fr{\partial^2\hat{\varphi}}{\partial\hat{y}^2}=-\left(\hat{n}_i-\hat{n}_e\right)\,;\enskip \hat{e}_y=-
\fr{\partial\hat\varphi}{\partial\hat{y}}\,;\\[9pt]
\hspace*{3.1mm}\hat{t}=0:\ \hspace*{2.6mm}\hat{f}_\alpha(\hat{y},\hat{v}_y,0)=\hat{f}_\alpha^{\mathrm{maksv}}\,,\enskip \alpha-i,e\,;\\[9pt]
\hspace*{2.9mm}\hat{y}=0:\ \hspace*{2.8mm}\hat{f}_\alpha(0,\hat{v}_y,\hat{t})=0\,,\enskip \alpha=i,e\,;\\[9pt]
\hspace*{24.3mm}\hat\varphi(0,\hat{t})=\hat{\varphi}_p\,;\\[9pt]
\hat{y}=\hat{y}_\infty:\ \hat{f}_\alpha(\hat{y}_\infty, \hat{v}_y, \hat{t})=\hat{f}^{\mathrm{maksv}}_\alpha\,,\enskip 
\alpha=i,e\,;\\[9pt]
\hspace*{21.5mm}\hat\varphi(\hat{y}_\infty,\hat{t})=0\,.
\end{array}
\right\}
\label{e4-k}
\end{equation}
Здесь 

\vspace*{-2pt}

\noindent
\begin{gather*}
A_\alpha=\sqrt{\delta_\alpha }\,\hat{v}_y\,;\enskip 
B_\alpha=\sqrt{\delta_\alpha}\,\fr{z_\alpha}{2\varepsilon_\alpha}\,;\\
\delta_\alpha=\fr{\varepsilon_\alpha}{\mu_\alpha}\,;\enskip 
\varepsilon_\alpha=\fr{T_{\alpha\infty}}{T_{i\infty}}\,;\\
\mu_\alpha=\fr{m_\alpha}{m_i}\,;\enskip 
D_\alpha=A_g^\alpha\fr{\partial^2\hat{g}_\alpha}{\partial  [\hat{v}_y]^2}\,;\\
K_\alpha=A_h^\alpha \fr{\partial \hat{h}_\alpha}{\partial \hat{v}_y}\,,\enskip \alpha=i,e\,,
\end{gather*}
где $A_g^\alpha$ и $A_h^\alpha$~--- коэффициенты, определяемые характерными параметрами 
задачи~\cite{15-k}.

Поиск решения самосогласованной системы уравнений~(\ref{e4-k}) осуществляется по следующей 
схе-\linebreak ме. Вначале находятся значения напряженности\linebreak
 электрического поля по значениям потенциала, 
полученным из граничной задачи для уравнения Пуассона. Далее, используя найденные значения 
напряженности, решается уравнение Фок\-ке\-ра--План\-ка путем перехода к стохастическому 
дифференциальному уравнению (СДУ) Ито:

\noindent
\begin{multline*}
d\Theta_\alpha(\hat{t}) = a_\alpha \left(\hat{t},\Theta_\alpha(\hat{t})\right)+{}\\
{}+\sigma\left(
\hat{t},\Theta_\alpha(\hat{t})\right)\,dW(\hat{t})\,,\quad \alpha=i,e\,,
%\label{e5-k}
\end{multline*}
где 

\noindent
\begin{align*}
\Theta_\alpha(\hat{t})&=\begin{bmatrix}
\hat{y}(\hat{t})\\ \hat{v}_y(\hat{t})
\end{bmatrix}\,;\\
a_\alpha\left(\hat{t},\Theta_\alpha(\hat{t})\right)&=\begin{bmatrix}
-A_\alpha\\ -K_\alpha -B_\alpha \hat{E}_y
\end{bmatrix}\,;\\
\sigma_\alpha\left(\hat{t},\Theta_\alpha(\hat{t})\right)\sigma_\alpha^{\mathrm{T}}\left( 
\hat{t},\Theta_\alpha(\hat{t})\right)&=D_\alpha\,,\enskip \alpha=i,e\,;
\end{align*} 
$W(\hat{t})$~--- стандартный винеровский случайный процесс.
\pagebreak

Для нахождения значений вектора состояния~$\Theta_\alpha(\hat{t})$ применим явную разностную 
схему стохастического метода Эйлера~\cite{16-k}:
\begin{multline*}
\Theta_\alpha^{n+1}=\Theta_\alpha^n +h_\tau a_\alpha \left( \hat{t}_n, \Theta_\alpha^n\right)+\sigma_\alpha 
\left( \hat{t}_n, \Theta_\alpha^n\right)\Delta W_n\,,\\ 
n=0,\ldots , N\,,\ \alpha=i,e\,,
%\label{e6-k}
\end{multline*}
где $\Theta_\alpha^n$, $n=0,\ldots , N$,~--- приближенное значение вектора 
состояния~$\Theta_\alpha(\hat{t})$, $\alpha=i,e$, в момент времени $\hat{t}\hm=\hat{t}_n$, 
$\hat{t}_n\hm=n h_\tau$, $n=0,\ldots , N$; $h_\tau$~--- достаточно малый шаг интегрирования; $\Delta 
W_n$, $n=0,\ldots ,N$,~--- величина приращения винеровского процесса~$W(\hat{t})$ на отрезке $\left[ 
\hat{t}_n,\,\hat{t}_{n+1}\right]$, по определению независимая от~$\Theta_\alpha^0$, 
$\Delta W_0,\ldots , 
\Delta W_{n-1}$: $\Delta W_n\hm=W(\hat{t}_{n-1})\hm-W(\hat{t}_n)$; $\Delta W_n\hm\sim N(0,\,h_\tau)$, 
т.\,е.\ $\Delta W_n$ представляют собой гауссовские случайные величины с нулевыми математическими 
ожиданиями и дисперсиями, равными шагу интегрирования; $\Theta_\alpha^0$~--- значение вектора 
состояния $\Theta_\alpha(\hat{t})$, $\alpha\hm=i,e$, в момент времени $\hat{t}=0$, 
$\Theta_\alpha^0\hm\sim \hat{f}_\alpha^{\mathrm{maksv}}$. 

Частные производные $\partial^2\hat{g}_\alpha/\partial[\hat{v}_y]^2$ и $\partial \hat{h}_\alpha/\partial 
\hat{v}_y$, являющиеся составляющими матрицы $\sigma_\alpha (\hat{t}_n, 
\Theta_\alpha^n)\sigma_\alpha^{\mathrm{T}}(\hat{t}_n,\Theta_\alpha^n)$ и вектора $a_\alpha(\hat{t}_n, 
\Theta_\alpha^n)$ соответственно, аппроксимируются со вторым порядком точности на трехточечном 
шаблоне на основе значений~$\hat{g}_\alpha$ и~$\hat{h}_\alpha$~\cite{17-k}.
      
      В выражения для функций~$\hat{g}_\alpha$ и~$\hat{h}_\alpha$ входят интегралы, которые 
вычисляются методом Мон\-те-Кар\-ло с использованием набора значений скоростной компоненты 
вектора состояния~$\hat{v}_y$, полученных из решения СДУ Ито:
      \begin{equation*}
      \int \hat{f}_\alpha \left\vert \hat{v}_y-
\hat{v}_y^\prime\right\vert\,dv_y^\prime=M\left(\zeta\left(\hat{V}_y\right)\right)\,,
\end{equation*}
где
$$
      \zeta\left(\hat{V}_y\right)=\left\vert \hat{v}_y-\hat{V}_y\right\vert\,,\enskip \hat{V}_y\sim 
\hat{f}_\alpha\,.
  $$
      
      Для вычисления напряженности самосогласованного электрического поля $\hat{E}_y=-
\partial\hat{\varphi}/\partial\hat{y}$, входящей в вектор $a_\alpha(\hat{t}_n, \Theta_\alpha^n)$, необходимо 
аналогично аппроксимировать со вторым порядком точности производную 
$\partial\hat{\varphi}/\partial\hat{y}$ на трехточечном шаблоне с использованием значений 
потенциала~$\hat{\varphi}$~\cite{17-k}. Значения потенциала~$\hat\varphi$ находятся из решения 
уравнения Пуассона. 
      
      Граничную задачу для уравнения Пуассона 
      \begin{align*}
      \fr{\partial^2 \hat\varphi}{\partial \hat{y}^2} & = -\left(\hat{n}_i-\hat{n}_e\right)\,;\\
      \hat{\varphi}\big|_{\hat{y}=0} &=\hat{\varphi}_p\,;\\
      \hat{\varphi}\big|_{\hat{y}_\infty=0} &=0
      \end{align*}
    предлагается решать путем перехода к конечно-разностной системе с последующим ее решением 
методом прогонки~\cite{17-k}:

\noindent
\begin{gather*}
\hat{\varphi}^n_{l-1}+2\hat{\varphi}_l^n+\hat{\varphi}^n_{l+1}=
h_y\hat{\delta}_l^n\,,\enskip l=1,\ldots , 
N_y\,;\\
\hat{\delta}_l^n=-\left( \hat{n}^n_{i,l}-\hat{n}^n_{e,l}\right)\,;\enskip 
\hat{\varphi}_0=\hat{\varphi}_p\,;\enskip \hat{\varphi}_{N_y}=0\,,
\end{gather*}
где $N_y$~--- число шагов по переменной~$\hat{y}$, $h_y$~--- величина шагов разбиения по~$\hat{y}$. 
      
      Концентрации $\hat{n}_\alpha$, $\alpha=i,e$, и плотности токов частиц на зонд~$\hat{f}_\alpha$, 
$\alpha=i,e$, вычисляются согласно описанному выше методу Мон\-те-Карло.

\section{Применение метода расщепления и~метода крупных~частиц}

Решение задачи в данном случае предлагается начать с записи правой части уравнения 
Фок\-ке\-ра--План\-ка в декартовой системе координат в виде:
$$
\mathbf{Q} f_\alpha = \fr{1}{2}\,\fr{\partial^2 f_\alpha}{\partial [v_y]^2}\,\fr{\partial^2 g_\alpha}{\partial 
[v_y]^2}+\fr{\partial f_\alpha}{\partial v_y}\,\fr{\partial C_\alpha}{\partial v_y}+H_\alpha\,,\enskip 
\alpha=i,e\,,
$$  
где 
\begin{align*}
C_\alpha(\vec{r},\vec{v},t)&=
\begin{cases}
\fr{1-\gamma}{Z_i^2}\int\fr{f_e(\vec{r},{\vec{v}}^{\,\prime},t)}{|\vec{v}-{\vec{v}}^{\,\prime} |}\,d{\vec{v}}^{\,\prime}\,, 
&\alpha=i\,;\\[9pt]
\fr{Z_i^2(\gamma-1)}{\gamma}\int \fr{f_i(\vec{r},{\vec{v}}^{\,\prime}, t)}
{|\vec{v}-{\vec{v}}^{\,\prime} 
|}\,d{\vec{v}}^{\,\prime}\,, &\alpha=e\,;
\end{cases} 
\\
H_\alpha&=
\begin{cases}
4\pi \left( \fr{\gamma f_e}{Z_i^2}+f_i\right)f_i\,, & \alpha=i\,;\\[9pt]
4\pi\left(\fr{Z_i^2 f_i}{\gamma}+f_e\right)f_e\,, &\alpha=e\,.
\end{cases}
\end{align*}
Тогда при переходе к безразмерным величинам (см.\ разд.~3) система~(\ref{e1-k}) запишется 
следующим образом:
      \begin{equation}
      \left.
\!\!\begin{array}{l}
      \fr{\partial 
\hat{f}_\alpha}{\partial\hat{t}}+A_\alpha\fr{\partial\hat{f}_\alpha}{\partial\hat{y}}+
B_\alpha  \hat{E}_y
\fr{\partial\hat{f}_\alpha}{\partial\hat{v}_\alpha}=\tilde{\mathbf{Q}}\hat{f}_\alpha\,,\enskip 
\alpha=i,e;\\[9pt]
      \fr{\partial^2\hat{\varphi}}{\partial\hat{y}^2}=-\left( \hat{n}_i-\hat{n}_e\right)\,,\enskip \hat{E}_y=-
\fr{\partial\hat\varphi}{\partial\hat{y}}\,,\\[9pt]
\hspace*{3.1mm}\hat{t}=0:\ \hspace*{2.6mm}\hat{f}_\alpha(\hat{y},\hat{v}_y, 0)=\hat{f}_\alpha^{\mathrm{maksv}}\,,\enskip \alpha=i,e\,,\\[9pt]
\hspace*{2.9mm} \hat{y}=0:\ \hspace*{2.8mm}\hat{f}_\alpha(0,\hat{v}_y,\hat{t})=0\,,\enskip \alpha=i,e\,;\\[9pt]
\hspace*{24.3mm}\hat\varphi(0,\hat{t})=\hat{\varphi}_p\,;\\[9pt]
      \hat{y}=\hat{y}_\infty:\ \hat{f}_\alpha(\hat{y}_\infty, 
\hat{v}_y,\hat{t})=\hat{f}_\alpha^{\mathrm{maksv}}\,,\enskip \alpha=i,e\,;\\[9pt]
\hspace*{21.5mm}\hat{\varphi}(\hat{y}_\infty,\hat{t})=0\,,\\[9pt]
    \end{array}
\right\}\!\!
\label{e7-k}
\end{equation}
где 
\begin{gather*}
\tilde{\mathbf{Q}} \hat{f}_\alpha=D_\alpha\fr{\partial^2\hat{f}_\alpha}{\partial 
[\hat{v}_y]^2}+K_\alpha\fr{\partial\hat{f}_\alpha}{\partial\hat{v}_y}+H_\alpha\,;\\
D_\alpha=A_g^\alpha\fr{\partial^2\hat{g}_\alpha}{\partial [\hat{v}_y]^2}\,;\enskip 
K_\alpha=A_h^\alpha \fr{\partial \hat{h}_\alpha}{\partial\hat{v}_y}\,,\ \alpha=i,e\,.
\end{gather*}

Для решения системы уравнений~(\ref{e7-k}) применяется модификация метода 
расщепления~\cite{17-k}, согласно которой исходная задача разбивается на две вспомогательные. Такое 
разбиение можно осуществить, переписав уравнение Фок\-ке\-ра--План\-ка в следующем виде:
$$
\fr{\partial\hat{f}_\alpha}{\partial\hat{t}} =
\tilde{\mathbf{Q}}_1\hat{f}_\alpha+\tilde{\mathbf{Q}}_2\hat{f}_\alpha\,,
$$
где 
\begin{align*}
\tilde{\mathbf{Q}}_1\hat{f}_\alpha &=-
\left(A_\alpha\fr{\partial\hat{f}_\alpha}{\partial\hat{y}}+
B_\alpha\fr{\partial\hat{f}_\alpha}{\partial\hat{y}}
\right)\,;\\
\tilde{\mathbf{Q}}_2\hat{f}_\alpha 
&=\left(D_\alpha\fr{\partial^2\hat{f}_\alpha}{\partial[\hat{v}_y]^2}+K_\alpha\fr{\partial 
\hat{f}_\alpha}{\partial\hat{v}_y}+H_\alpha\right)\,.
\end{align*}

      Правая часть уравнения Фок\-ке\-ра--План\-ка представляет собой сумму двух операторов, 
первый из которых отвечает за перенос частиц, второй~--- за столкновения заряженных частиц. 
В~результате образуются следующие задачи, которые решаются последовательно:
      \begin{itemize}
\item первая задача:
\begin{align*}
&\fr{\partial w_\alpha(\hat{y},\hat{v}_y,\hat{t})}{\partial\hat{t}} =\mathbf{Q}_1 
w_\alpha(\hat{y},\hat{v}_y,\hat{t})\,,\enskip \alpha=i,e\,;\\[9pt]
&\fr{\partial^2\hat\varphi}{\partial\hat{y}^2}=-\left(\hat{n}_i-\hat{n}_e\right)\,;\enskip
\hat{E}_y=-
\fr{\partial\hat\varphi}{\partial\hat{y}}\,;\\[9pt]
&w_\alpha(\hat{y},\hat{v}_y,\hat{t}^n)=\hat{f}_\alpha(\hat{y},\hat{v}_y,\hat{t}^n)\,,\enskip n=0,\ldots ,N-
1\,;\\[9pt]
&\hspace{2.9mm}\hat{y}=0:\ \hspace*{2.9mm}w_\alpha(0,\hat{v}_y,\hat{t})=0\,,\enskip \alpha=i,e\,;\\[9pt]
&\hspace*{25.1mm}\hat\varphi(0,\hat{t})=\hat{\varphi}_p\,;\\[9pt]
&\hat{y}=\hat{y}_\infty:\ w_\alpha(\hat{y}_\infty, \hat{v}_y, \hat{t})=
\hat{f}_\alpha^{\mathrm{maksv}}\,,\enskip 
\alpha=i,e\,;\\[9pt]
&\hspace*{22.5mm}\hat\varphi(\hat{y}_\infty,\hat{t})=0\,;
\end{align*}
\item вторая задача:
\begin{align*}
\!\!\!\!\!\!\!\fr{\partial s_\alpha(\hat{y},\hat{v}_y,\hat{t})}{\partial \hat{t}} &=\mathbf{Q}_2 
s_\alpha(\hat{y},\hat{v}_y,\hat{t})\,, & \alpha&=i,e\,;\\
\!\!\!\!\!\!\!s_\alpha (\hat{y},\hat{v}_y,\hat{t}^n) &=w_\alpha (\hat{y},\hat{v}_y, \hat{t}^{n+1}),& n&=0,\ldots ,N-
1.
\end{align*}
\end{itemize}

Первая задача представляет собой систему безразмерных уравнений Вла\-со\-ва--Пуас\-со\-на. Для ее 
решения применяется метод крупных частиц~\cite{18-k}. Согласно этому методу решение задачи 
осуществляется путем расщепления на два этапа: на первом этапе не учитываются конвективные члены 
и решение получается обычным интегрированием на неподвижной эйлеровой сетке, а на втором этапе 
рассматривается система, которая описывает перенос частиц в лагранжевой системе координат. Кроме 
того, на первом этапе необходимо решить уравнение Пуассона для получения значений потенциала 
самосогласованного электрического поля. Для этого применяется метод, описанный в разд.~3. 

Вторая задача решается путем перехода к ко\-неч\-но-раз\-ност\-ной сис\-те\-ме. При этом частные 
производные $\partial^2\hat{g}_\alpha/\partial[\hat{v}_y]^2$ и $\partial\hat{h}_\alpha/\partial\hat{v}_y$ 
аппроксимируются со вторым порядком точности с использованием трехточечного шаблона, а 
производная $\partial s_\alpha/\partial\hat{t}$ аппроксимируется на двухточечном шаблоне с первым 
порядком точности~\cite{16-k}. К~полученной системе разностных уравнений предлагается применить 
один из классических методов решения систем линейных уравнений, например метод 
Гаусса~\cite{19-k}.
      
      Решением первой задачи является функция $w_\alpha(\hat{y}, \hat{v}_y, \hat{t}^n)$, 
$n\hm=0,\ldots ,N$, , которая дает начальное условие для второй задачи. Решая вторую задачу, находим 
функцию $s_\alpha(\hat{y},\hat{v}_y,\hat{t}^n)\hm=\hat{f}_\alpha(\hat{y},\hat{v}_y,\hat{t}^n)$, 
$n=1,\ldots ,N$, $\alpha=i,e$, которая определяет решение $\hat{f}_\alpha(\hat{y},\hat{v}_y,\hat{t}^n)$, 
$\alpha=i,e$, исходной системы~(\ref{e7-k}) для рассматриваемых моментов времени $n=1,\ldots ,N$.

Моменты функций распределения $\hat{f}_\alpha$, $\alpha=i,e$, находятся с помощью методов 
численного интегрирования, например метода трапеций~\cite{19-k}.

\section{Результаты численного моделирования}

Для двух описанных выше методов реализованы две отдельные программы в среде {Matlab~7.0}. 
Эти программы позволяют по заданным значениям концентраций и температур частиц $n_{i\infty}$, 
$n_{e\infty}$, $T_{i\infty}$ и~$T_{e\infty}$ в невозмущенной плазме, а также потенциала~$\varphi_p$, 
подаваемого на зонд, изучить эволюцию во времени плотностей тока частиц~$j_i$ и~$j_e$, концентраций 
частиц~$n_i$  и~$n_e$ в произвольной точке пространства в возмущенной зоне, а также динамику 
изменения напряженности~$E_y$ самосогласованного электрического поля во времени и пространстве.

С использованием разработанных программ проведены серии расчетных экспериментов, в которых 
значение концентраций варьировалось в пределах $n_{i\infty} \hm = n_{e\infty}\hm =10^{18}\div 
10^{22}$~м$^{-3}$. Значение температур было выбрано неизменным и равным $T_{i\infty}\hm = 
T_{e\infty}\hm=3000$~K, а значения потенциала, подаваемого на зонд, изменялись в пределах 
$\varphi_p\hm=0\div 2{,}6$~В.

На рис.~1  и~2 приведены графики изменения напряженности самосогласованного электрического
 поля (см.\ рис.~1) и плотности токов ионов (см.\linebreak\vspace*{-12pt}

\pagebreak

\end{multicols}

\begin{figure} %fig1
\vspace*{1pt}
\begin{center}
\mbox{%
\epsfxsize=162.594mm
\epsfbox{kud-1.eps}
}
\end{center}
\vspace*{-9pt}
\Caption{Динамика изменения плотности тока ионов во времени в фиксированной точке возмущенной 
зоны для значений потенциала: \textit{1}~--- $\varphi_p=-6$; 
\textit{2}~--- $\varphi_p=-16$; \textit{3}~--- $\varphi_p=- 30$ 
в случае применения методов Монте-Карло~(\textit{а}) 
и крупных частиц~(\textit{б})}
\end{figure}

\begin{figure} %fig2
\vspace*{1pt}
\begin{center}
\mbox{%
\epsfxsize=162.713mm
\epsfbox{kud-2.eps}
}
\end{center}
\vspace*{-9pt}
\Caption{Динамика изменения напряженности электрического поля во времени в фиксированной точке 
возмущенной зоны для значений потенциала: 
\textit{1}~--- $\varphi_p=-6$; \textit{2}~--- $\varphi_p=-16$; 
\textit{3}~--- $\varphi_p=-30$ в случае применения методов Монте-Карло~(\textit{а}) и
крупных частиц~(\textit{б})
}
\end{figure}

\begin{multicols}{2}

\noindent
 рис.~2) во времени в фиксированной точке пространства 
возмущенной зоны в случае применения обоих разработанных алгоритмов.


На основании полученных результатов можно отметить похожее поведение зависимостей 
напряженности электрического поля и плотности тока от времени в двух рассматриваемых случаях. 
Графики кривых сначала убывают, затем начинают возрастать, выходя в некоторый момент 
времени~$t^\prime$ (момент установления) на стационарные значения. 

Одинаковое поведение 
напряженности и плот\-ности тока можно объяснить из следующих соображений: плотность тока ионов в 
данной области пространства равна произведению концентрации ионов на их направленную скорость и 
на заряд иона. Скорость ионов, в свою очередь, зависит от заряда, массы и напряженности 
электрического поля. 
%\columnbreak

При внесении в плазму отрицательно заряженного зонда возникает электрическое поле, которое 
нарушает квазинейтральность плазмы. Для того чтобы компенсировать действие внешнего 
электрического поля, ионы устремляются к зонду, а электроны~--- от зонда. Это приводит к дисбалансу 
концентраций вблизи зонда и, как следствие, к увеличению разности потенциалов; график 
напряженности электрического поля убывает. Вскоре разделение зарядов компенсирует внешнее 
электрическое поле; график выходит на стационарное значение. 

Также можно отметить, что значения 
напряженности электрического поля и плотности тока частиц на зонд в момент установления для двух 
методов совпадают. 

Момент установления~$t^\prime$ зависит от при\-ме\-ня\-емо\-го метода решения. В~случае метода 
Мон\-те-Кар\-ло $t^\prime=3{,}5\div 4$~ед., а для метода крупных частиц совместно с методом 
расщепления $t^\prime\hm=5\div 5{,}5$~ед. Используя ко\-неч\-но-раз\-ност\-ный метод, можно 
получить динамику изменения функций распределения частиц~$f_\alpha$, $\alpha=i,e$, во времени и 
пространстве. Функции распределения позволяют наглядно представить влияние на картину 
распределения частиц вблизи зонда самой поверхности зонда и электрического поля.

\section{Заключение}
      
      В работе найдено решение задачи диагностики плоским зондом сильноионизованной плазмы с 
учетом столкновений заряженных частиц. Разработана математическая модель исследуемого явления, 
описываемая уравнениями Фок\-ке\-ра--План\-ка и Пуассона. Решение получено двумя методами:\linebreak 
статистическим и ко\-неч\-но-раз\-ност\-ным на основе\linebreak сформированных алгоритмов. Приведены 
резуль-\linebreak таты численного моделирования при различных\linebreak характерных параметрах задачи.
 Из  проведенных 
вычислительных экспериментов вытекает, что искомые величины: напряженность 
электрического поля, плотности токов частиц на зонд, концентрации частиц вблизи зонда~--- как по 
характеру зависимости, так и по числовым значениям совпадают. При применении метода 
      Мон\-те-Кар\-ло момент установления наступает быстрее по сравнению с конечно-разностным 
методом, однако конечно-разностный метод позволяет получить более наглядные результаты.

{\small\frenchspacing
{%\baselineskip=10.8pt
\addcontentsline{toc}{section}{Литература}
\begin{thebibliography}{99}

\bibitem{1-k}
\Au{Alexeff I., Anderson T.}
Experimental and theoretical results with plasma antenna~// IEEE Trans. Plasma Sci., 2006. Vol.~34. 
No.\,2. P.~166--172.

\bibitem{2-k}
\Au{Сысун В.\,И.}
Сильноионизованная низкотемпературная плазма в приборах электронной техники: Методы 
исследования, свойства, применение. Дисс. \ldots д-ра физ.-мат. наук в форме науч. докл.: 
01.04.08.~--- Пет\-ро\-за\-водск, 1996.

\bibitem{3-k}
\Au{Тухас В.\,А.}
Методология создания средств измерений и испытаний на устойчивость к кондуктивным помехам~// 
Мат-лы VI Междунар. симп. по электромагнитной совместимости и 
электромагнитной экологии.~--- СПб., 2005. С.~231--234.

\bibitem{4-k}
\Au{Гудзенко Л.\,И., Яковленко С.\,И.}
Плазменные лазеры.~--- М.: Атомиздат, 1978.  256~с.

\bibitem{5-k}
\Au{Звелто О.}
Принципы лазеров.~--- М.: Мир, 1990.  560~с.

\bibitem{6-k}
\Au{Сысун В.\,И., Хромой Ю.\,Д.}
Расширение канала мощного импульсного разряда в парах ртути~// Электронная техника, 1974. 
Сер.~4. Вып.~10. С.~80--85. 

\bibitem{7-k}
\Au{Винклер Дж.\,Р.}
Искусственные пучки частиц в космической плазме.~--- М.: Мир, 1985.  451~с.

\bibitem{8-k}
\Au{Bernstein I.\,B., Rabinowitz I.\,N.}
Theory of electrostatic probes in low-density plasma~// Phys. Fluids, 1959. Vol.~2. No.\,2. P.~112--121. 

\bibitem{9-k}
\Au{Альперт Я.\,Л., Гуревич А.\,В., Питаевский~Л.\,П.}
Искусственные спутники в разреженной плазме.~--- М.: Наука, 1964.  282~с.

\bibitem{10-k}
\Au{Чан П., Тэлбот Л., Турян~К.}
Электрические зонды в неподвижной и движущейся плазме.~--- М.: Мир, 1978.  202~с.

\bibitem{11-k}
\Au{Алексеев Б.\,В., Котельников В.\,А.}
Зондовый метод диагностики плазмы.~--- М.: Энергоатомиздат, 1989.  240~с.

\bibitem{12-k}
\Au{Пантелеев А.\,В., Кудрявцева И.\,А.}
Формирование математической модели двухкомпонентной плазмы с учетом столкновений 
заряженных частиц в случае плоского зонда~// Теоретические вопросы вычислительной техники и 
программного обеспечения: Межвузовский сб. научн. тр.~--- М.: МИРЭА, 2006. С.~11--21.

\bibitem{13-k}
\Au{Олдер Б.}
Вычислительные методы в физике плазмы.~--- М.: Мир, 1974.  111~с.

\bibitem{14-k}
\Au{Montgomery D.\,C., Tidman D.\,A.}
Plasma kinetic theory.~--- New York, 1964. 

\bibitem{15-k}
\Au{Кудрявцева И.\,А., Пантелеев А.\,В.}
Применение метода Мон\-те-Кар\-ло для анализа поведения двухкомпонентной плазмы с учетом 
столкновений между заряженными частицами~// Теоретические вопросы\linebreak
вычислительной техники и 
программного обеспечения: Межвузовский сб. научн. тр.~--- М.: МИРЭА, 2008. С.~122--128. 

\bibitem{16-k}
\Au{Семенов В.\,В., Пантелеев А.\,В., Руденко~Е.\,А., Бор\-та\-ков\-ский~А.\,С.}
Методы описания, анализа и синтеза нелинейных систем управления.~--- М.: МАИ, 1993.  312~с.

\bibitem{17-k}
\Au{Киреев В.\,И., Пантелеев А.\,В.}
Численные методы в примерах и задачах.~--- М.: Высшая школа, 2006.  480~с.

\bibitem{18-k}
\Au{Белоцерковский О.\,М., Давыдов~Ю.\,М.}
Метод крупных частиц в газовой динамике. Вычислительный эксперимент.~--- М.: Наука, 
Физматгиз, 1982.

\label{end\stat}

\bibitem{19-k}
\Au{Вержбицкий В.\,М.}
Основы численных методов.~--- М.: Высшая школа, 2002.  840~с.
 \end{thebibliography}
}
}


\end{multicols}        %11



%   { %\Large  
   { %\baselineskip=16.6pt
   
   \vspace*{-48pt}
   \begin{center}\LARGE
   \textit{Предисловие}
   \end{center}
   
   %\vspace*{2.5mm}
   
   \vspace*{25mm}
   
   \thispagestyle{empty}
   
   { %\small 

    
Вниманию читателей журнала <<Информатика и её применения>> предлагается 
очередной тематический выпуск <<Вероятностно-статистические методы и 
задачи информатики и информационных технологий>>. Предыдущие тематические 
выпуски журнала по данному направлению вышли в 2008~г.\ (т.~2, вып.~2), 
в 2009~г.\ (т.~3, вып.~3) и в 2010~г.\ (т.~4, вып.~2). 

Статьи, собранные в данном журнале, посвящены разработке новых вероятностно-статистических 
методов, ориентированных на применение к решению конкретных задач информатики и информационных 
технологий, а также~--- в ряде случаев~--- и других прикладных задач. Проблематика, охватываемая 
публикуемыми работами, развивается в рамках научного сотрудничества между Институтом проблем 
информатики Российской академии наук (ИПИ РАН) и Факультетом вычислительной математики и 
кибернетики Московского государственного университета им.\ М.\,В.~Ломоносова в ходе работ 
над совместными научными проектами (в том числе в рамках функционирования 
Научно-образовательного центра <<Вероятностно-статистические методы анализа рисков>>). 
Многие из авторов статей, включенных в данный номер журнала, являются активными участниками 
традиционного международного семинара по проблемам устойчивости стохастических моделей, 
руководимого В.\,М.~Золотаревым и В.\,Ю.~Королевым; регулярные сессии этого семинара 
проводятся под эгидой МГУ и ИПИ РАН (в 2011~г.\ указанный семинар проводится в октябре 
в Калининградской области РФ). 

Наряду с представителями ИПИ РАН и МГУ в число авторов данного выпуска журнала входят 
ученые из Научно-исследовательского института системных исследований РАН, Института 
проблем технологии микроэлектроники и особочистых материалов РАН, Института 
прикладных математических исследований Карельского НЦ РАН, Московского 
авиационного института, Вологодского государственного педагогического университета, 
НИИММ им.\ Н.\,Г.~Чеботарева, Казанского государственного университета, Дебреценского 
университета (Венгрия).

Несколько статей выпуска посвящено разработке и применению стохастических методов и 
информационных технологий для решения различных прикладных задач. В~работе В.\,Г.~Ушакова 
и О.\,В.~Шестакова рассмотрена задача определения вероятностных характеристик случайных 
функций по распределениям интегральных преобразований, возникающих в задачах эмиссионной 
томографии. В~статье Д.\,О.~Яковенко и М.\,А.~Целищева рассмотрены некоторые вопросы 
математической теории риска и предложен новый подход к диверсификации инвестиционных 
портфелей. Работа И.\,А.~Кудрявцевой и А.\,В.~Пантелеева посвящена построению и 
исследованию математической модели, описывающей динамику сильноионизованной плазмы. 
В~статье П.\,П.~Кольцова изучается качество работы ряда алгоритмов сегментации изображений. 
Статья А.\,Н.~Чупрунова и И.~Фазекаша посвящена вероятностному анализу числа без\-оши\-бочных 
блоков при помехоустойчивом кодировании; получены усиленные законы больших чисел для указанных 
величин.

В данном выпуске традиционно присутствует тематика, весьма активно разрабатываемая в течение 
многих лет специалистами ИПИ РАН и МГУ,~--- методы моделирования и управления для 
информационно-телекоммуникационных и вычислительных систем, в частности методы 
теории массового обслуживания. В~статье А.\,И.~Зейфмана с соавторами рассматриваются 
модели обслуживания, описываемые марковскими цепями с непрерывным временем в случае 
наличия катастроф. В~работе М.\,М.~Лери и И.\,А.~Чеплюковой рассматриваются случайные 
графы Интернет-типа, т.\,е.\ графы, степени вершин которых имеют степенные распределения; 
такие задачи находят применение при исследовании глобальных сетей передачи данных. 
Работа Р.\,В.~Разумчика посвящена исследованию систем массового обслуживания специального 
вида~--- с отрицательными заявками и хранением вытесненных заявок.

Ряд статей посвящен развитию перспективных теоретических 
вероятностно-статистических методов, которые находят широкое применение в различных 
задачах информатики и информационных технологий. В~работе В.\,Е.~Бенинга, А.\,К.~Горшенина 
и В.\,Ю.~Королева рассмотрена задача статистической проверки гипотез о числе компонент 
смеси вероятностных распределений, приводится конструкция асимптотически наиболее мощного 
критерия. Результаты этой работы найдут применение в ряде прикладных задач, использующих 
математическую модель смеси вероятностных распределений (в информатике, моделировании 
финансовых рынков, физике турбулентной плазмы и~т.\,д.). В~статье В.\,Ю.~Королева, 
И.\,Г.~Шевцовой и С.\,Я.~Шоргина строится новая, улучшенная оценка точности нормальной 
аппроксимации для пуассоновских случайных сумм; как известно, указанные случайные суммы 
широко используются в качестве моделей многих реальных объектов, в том числе в информатике, 
физике и других прикладных областях. Работа В.\,Г.~Ушакова и Н.\,Г.~Ушакова посвящена 
исследованию ядерной оценки плотности распределения; эти результаты могут применяться, 
в част\-ности, при анализе трафика в телекоммуникационных системах. Серьезные приложения 
в статистике могут получить результаты работы О.\,В.~Шестакова, в которой доказаны оценки 
скорости сходимости распределения выборочного абсолютного медианного отклонения к нормальному 
закону. 

\smallskip

Редакционная коллегия журнала выражает надежду, что данный тематический  выпуск 
будет интересен специалистам в области теории вероятностей и математической статистики 
и их применения к решению задач информатики и информационных технологий.
     
     %\vfill 
     \vspace*{20mm}
     \noindent
     Заместитель главного редактора журнала <<Информатика и её 
применения>>,\\
     директор ИПИ РАН, академик  \hfill
     \textit{И.\,А.~Соколов}\\
     
     \noindent
     Редактор-составитель тематического выпуска,\\
     профессор кафедры математической статистики факультета\\
      вычислительной математики и кибернетики МГУ им.\ М.\,В.~Ломоносова,\\
     ведущий научный сотрудник ИПИ РАН,\\ 
доктор физико-математических наук \hfill
      \textit{В.\,Ю.~Королев}
     
     } }
     }

%%%%%%%%%%%%%%%%%%%%%%%%%%%%%%%%%%%%%%%%%%%%%%%


                       
%\end{document}

%\def\stat{rez}
{%\hrule\par
%\vskip 7pt % 7pt
\raggedleft\Large \bf%\baselineskip=3.2ex
Р\,Е\,Ц\,Е\,Н\,З\,И\,И \vskip 17pt
    \hrule
    \par
\vskip 6pt plus 6pt minus 3pt }

%\thispagestyle{headings} %с верхним колонтитулом
%\thispagestyle{myheadings} %с нижним колонтитулом, но в верхнем РЕЦЕНЗИИ

\def\tit{НОВАЯ КНИГА И.\,Н.~СИНИЦЫНА, А.\,С.~ШАЛАМОВА <<ЛЕКЦИИ ПО ТЕОРИИ 
ИНТЕГРИРОВАННОЙ ЛОГИСТИЧЕСКОЙ ПОДДЕРЖКИ>> (М.: ТОРУС ПРЕСС, 2012. 624~с.)}

%1
\def\aut{Д.ф.-м.н., профессор С.\,Я.~Шоргин}

\def\auf{\ }

\def\leftkol{\ % РЕЦЕНЗИИ
}

\def\rightkol{ \ } 

%\def\leftkol{\ } % ENGLISH ABSTRACTS}

%\def\rightkol{\ } %ENGLISH ABSTRACTS}

%\def\leftkol{РЕЦЕНЗИИ}

%\def\rightkol{РЕЦЕНЗИИ}

\titele{\tit}{\aut}{\auf}{\leftkol}{\rightkol}
\vspace*{-18pt}


     \label{st\stat}

     \begin{multicols}{2}
     {\small
     {\baselineskip=10.1pt
     

      В книге представлено системное изложение теоретических основ одного из новейших 
направлений в \mbox{об\-ласти} экономики послепродажного обслуживания изделий наукоемкой 
продукции (ИНП) длительного пользования~--- интегрированной логистической поддержки
(ИЛП). 
{\looseness=1

}

Приведены также результаты новых работ, выполненных в Институте проблем информатики 
Российской академии наук в рамках научного направления <<Информационные технологии и 
анализ сложных сис\-тем>>.
 {%\looseness=1

}
     
      Излагаемые в книге научные подходы позво\-ляют карди\-наль\-но реформировать 
существующие системы производства и эксплуатации ИНП путем создания и внед\-ре\-ния 
методов рационального и оптимального управ\-ле\-ния процессами расходования 
вре\-мен\-н$\acute{\mbox{ы}}$х, 
мате\-ри\-аль\-ных, трудовых и других ресурсов на всех стадиях жизненного цикла изделий (ЖЦИ) по 
критериям экономической целесообразности и эф\-фек\-тив\-ности.
  {\looseness=1

}
    
      В книге приведен краткий обзор причин возник\-новения и
      развития CALS-методологии как основы 
современных международных стандартов по созданию и функционированию глобальных 
ин\-фор\-ма\-ци\-он\-но-ком\-му\-ни\-ка\-ци\-он\-ных систем, ее ключевых возможностей и эффективности 
результатов ее использования. 
Авторы %\linebreak 
предлагают ряд научных обоснований для разработки 
единой теории проектирования и управления систем ИЛП для полноценного использования 
преимуществ %\linebreak
 суще\-ст\-ву\-ющей методологии, определяют \mbox{общую} структурную схему 
комплексной системы <<ИНП-СППО>> и необходимость разработки для ее описания 
гибридных стохастических моделей.
{%\looseness=1

}

%\columnbreak
      
      Книга состоит из пяти частей, где последовательно излагается материал по каждой из 
следующих тем: <<Интегрированная логистическая поддержка>>, <<Теория гибридных 
стохастических систем и компьютерная поддержка исследований и разработок>>, <<Основы 
математического моделирования, анализа и синтеза систем послепродажного обслуживания>>, 
<<Определение и анализ показателей экспортного потенциала ИНП при проектировании>>, 
<<Задачи управления поддержкой послепродажного обслуживания>>, а также 
<<Моделирование инвестиционных процессов ИЛП в условиях неравновесных финансовых 
рынков>>. 
   
      В конце каждой главы приведены выводы и даны вопросы и задания для 
самоконтроля. В~приложениях содержатся основные определения по программам работ по 
анализу ИЛП, логистическим базам данных и компьютерным решениям, эквивалентной статистической 
линеаризации нелинейных преобразований ИЛП, справочный материал, а также развернутые 
уравнения для вероятностных характеристик.


      \def\leftkol{РЕЦЕНЗИИ}

\def\rightkol{РЕЦЕНЗИИ} 

      
      Книга заинтересует широкий круг специалистов и может быть использована научными 
проектными организациями в сфере промышленного производства ИНП. Большое количество 
иллюстраций, примеров и вопросов, обращенных к читателю, позволяет использовать книгу 
также в качестве учебного пособия для студентов и аспирантов машиностроительных, 
транспортных и~других специальностей, а также для самостоятельного изучения. 
{%\looseness=-1

}

Книга 
представляет несомненный интерес для специалистов и студентов в области прикладной 
математики и информатики.
    

}

}
\end{multicols}

%\newpage

\include{obchak}



\def\stat{authorsrus}
{%\hrule\par
%\vskip 7pt % 7pt
\raggedleft\Large \bf%\baselineskip=3.2ex
О\,Б\ \ А\,В\,Т\,О\,Р\,А\,Х \vskip 17pt
    \hrule
    \par
\vskip 21pt plus 8pt minus 4pt }


\def\tit{\ }

\def\aut{\ }

\def\auf{\ }

\def\leftkol{\ } % ENGLISH ABSTRACTS}

\def\rightkol{ОБ АВТОРАХ} %ENGLISH ABSTRACTS}

\titele{\tit}{\aut}{\auf}{\leftkol}{\rightkol}
      
            \label{st\stat}



\vspace*{24pt}

\begin{multicols}{2}




\noindent
\textbf{Архипов Олег Петрович} (р.\ 1948)~---
кандидат технических наук, директор Орловского филиала Института проб\-лем информатики
Российской академии наук
%302025, г.Орел, Московское шоссе, д.137

\vspace*{3pt}

\noindent
\textbf{Бирюкова Татьяна Константиновна} (р.\ 1968)~---
кандидат фи\-зи\-ко-ма\-те\-ма\-ти\-че\-ских наук, старший научный сотрудник Института проб\-лем информатики
Российской академии наук

\vspace*{3pt}

\noindent 
\textbf{Бобков  Сергей Геннадьевич} (р.\ 1955)~---
доктор технических наук,  заведующий отделением На\-уч\-но-ис\-сле\-до\-ва\-тель\-ско\-го 
института системных исследований Российской академии наук
%117218, Москва, Нахимовский просп., 36, к.1 

\vspace*{3pt}

\noindent \textbf{Васильев Николай Семенович} (р.\ 1952)~--- доктор 
фи\-зи\-ко-ма\-те\-ма\-ти\-че\-ских наук, профессор, 
МГТУ им.\ Н.\,Э.~Баумана 
%, Москва 105005, 2-я Бауманская ул., д.~5,

\vspace*{3pt}

\noindent
\textbf{Гершкович Максим Михайлович} (р.\ 1968)~---
старший научный сотрудник Института проб\-лем информатики
Российской академии наук

\vspace*{3pt}

\noindent 
\textbf{Дьяченко Юрий Георгиевич} (р.\ 1958)~--- кандидат технических наук, 
старший научный сотрудник Института проб\-лем информатики
Российской академии наук

\vspace*{3pt}

\noindent 
\textbf{Ерошенко Александр Андреевич} (р.\ 1989)~--- аспирант кафедры 
математической статистики факультета вычисли\-тельной математики и кибернетики 
Московского государственного университета им.\ М.\,В.~Ломоносова
%119991, Москва ГСП-1, Ленинские горы, д.\ 1, стр. 52

\vspace*{3pt}
 
\noindent 
\textbf{Захаров Виктор Николаевич} (р.\ 1948)~--- 
доктор технических наук, доцент, ученый секретарь Института проб\-лем информатики
Российской академии наук

\vspace*{3pt}

\noindent
\textbf{Зейфман Александр Израилевич} (р.\ 1954)~---
доктор фи\-зи\-ко-ма\-те\-ма\-ти\-че\-ских наук, профессор, 
заведующий кафедрой Вологодского государственного университета; 
старший научный сотрудник Института проб\-лем информатики
Российской академии наук; главный научный сотрудник ИСЭРТ Российской академии наук

\vspace*{3pt}

\noindent
\textbf{Зыкин Сергей Владимирович} (р.\ 1959)~--- 
доктор технических наук, профессор, заведующий лабораторией Института математики 
им.\ С.\,Л.~Соболева Сибирского отделения Российской академии наук, Новосибирск 
%630090, пр.\ ак.\ Коптюга, 4 

\vspace*{4pt}

\noindent
\textbf{Киреев Владимир Иванович} (р.\ 1938)~---
доктор фи\-зи\-ко-ма\-те\-ма\-ти\-че\-ских наук, профессор Московского 
государственного горного университета
%Адрес: Россия, 119991, г. Москва, Ленинский проспект, д. 6

%\columnbreak

\vspace*{4pt}

\noindent
\textbf{Козеренко Елена Борисовна} (р.\ 1959)~---
кандидат филологических наук, заведующая лабораторией Института проб\-лем информатики
Российской академии наук

\vspace*{4pt}

\noindent
\textbf{Королев Виктор Юрьевич} (р.\ 1954)~--- доктор
фи\-зи\-ко-ма\-те\-ма\-ти\-че\-ских наук, профессор кафедры математической 
статистики факультета вычисли\-тельной математики и кибернетики 
Московского государственного университета; 
ведущий научный сотрудник Института проб\-лем информатики
Российской академии наук

\vspace*{4pt}

\noindent
\textbf{Коротышева Анна Владимировна} (р.\ 1988)~---
старший преподаватель Вологодского государственного университета

\vspace*{4pt}

\noindent 
\textbf{Кун Де Турк} (р.\ 1981)~--- научный сотрудник 
исследовательской группы SMACS факультета телекоммуникаций и обработки информации
Университета Гента, Бельгия
%В-9000 Гент, Бельгия

\vspace*{4pt}

\noindent
\textbf{Лупенцов Олег Сергеевич} (р.\ 1986)~---
аспирант Омского государственного института сервиса
%Омск 644043, ул.\ Певцова 13

\vspace*{4pt}

\noindent
\textbf{Лучко Олег Николаевич} (р.\ 1961)~---
кандидат педагогических наук, профессор, заведующий кафедрой 
Омского государственного института сервиса
%Омск 644043, ул.\ Певцова 13

\vspace*{4pt}

\noindent
\textbf{Малашенко Юрий Евгеньевич} (р.\ 1946)~---
доктор фи\-зи\-ко-ма\-те\-ма\-ти\-че\-ских наук, заведующий сектором 
Вычислительного центра им.\ А.\,А.~Дородницына Российской академии наук
%Адрес: 119333, Москва, ул. Вавилова, 40,

\vspace*{4pt}

\noindent
\textbf{Маньяков Юрий Анатольевич} (р.\ 1984)~---
кандидат технических наук, научный сотрудник Орловского филиала Института проб\-лем информатики
Российской академии наук
%302025, г.Орел, Московское шоссе, д.137

\vspace*{4pt}

\noindent
\textbf{Маренко Валентина Афанасьевна} (р.\ 1951)~---
кандидат технических наук, доцент, старший научный сотрудник 
Института математики им.\ С.\,Л.~Соболева Сибирского отделения Российской академии наук
%Новосибирск 630090, пр. ак. Коптюга, 4 

\vspace*{3pt}

\noindent 
\textbf{Морозов Евсей Викторович} (р.\ 1947)~--- доктор 
фи\-зи\-ко-ма\-те\-ма\-ти\-че\-ских, профессор, ведущий научный сотрудник 
Института прикладных математических исследований Карельского научного центра Российской
академии наук; 
%%185910 Россия, Республика Карелия, г.\ Петрозаводск, ул.\ Пушкинская, 11
профессор Петрозаводского государственного университета, Петрозаводск
%185910 Россия, Республика Карелия, г.\ Петрозаводск, пр.\ Ленина, 33

%\pagebreak

\vspace*{3pt}

\noindent
\textbf{Назарова Ирина Александровна} (р.\ 1966)~---
кандидат фи\-зи\-ко-ма\-те\-ма\-ти\-че\-ских наук, 
научный сотрудник Вычислительного центра им.\ А.\,А.~Дородницына Российской академии наук 
%Адрес: 119333, Москва, ул. Вавилова, 40

\vspace*{3pt}

\noindent
\textbf{Павлов Игорь Валерианович} (р.\ 1945)~--- 
доктор фи\-зи\-ко-ма\-те\-ма\-ти\-че\-ских наук, профессор МГТУ им.\ Н.\,Э.~Баумана 
%Москва 105005, 2-я Бауманская ул., д.~5 

%\pagebreak

\vspace*{3pt}

\noindent 
\textbf{Потахина Любовь Викторовна} (р.\ 1989)~--- аспирантка
Института прикладных математических исследований Карельского научного центра
Российской академии наук; 
%%185910 Россия, Республика Карелия, г.\ Петрозаводск, ул.\ Пушкинская, 11
инженер Петрозаводского государственного университета, Петрозаводск
%185910 Россия, Республика Карелия, г.\ Петрозаводск, пр.\ Ленина, 33

\vspace*{3pt}

\noindent 
\textbf{Рождественский Юрий Владимирович} (р.\ 1952)~--- 
кандидат технических наук, заведующий сектором Института проб\-лем информатики
Российской академии наук

\vspace*{3pt}

\noindent 
\textbf{Синицын Игорь Николаевич} (р.\ 1940)~--- доктор технических наук,
профессор, заслуженный деятель\linebreak\vspace*{-12pt}

\columnbreak

\noindent
 науки РФ, заведующий отделом Института проб\-лем информатики
Российской академии наук

\vspace*{7pt}


\noindent
\textbf{Сиротинин Денис Олегович} (р.\ 1984)~---
кандидат технических наук, научный сотрудник Орловского филиала Института проб\-лем информатики
Российской академии наук
%302025, г.Орел, Московское шоссе, д.137

\vspace*{7pt}

%\columnbreak

\noindent 
\textbf{Соколов  Игорь Анатольевич} (р.\ 1954)~--- академик (действительный член) Российской 
академии наук, доктор технических наук, директор Института проб\-лем информатики
Российской академии наук

\vspace*{7pt}

\noindent
\textbf{Степченков Юрий Афанасьевич} (р.\ 1951)~---
кандидат технических наук, заведующий отделом Института проб\-лем информатики
Российской академии наук

\vspace*{7pt}

\noindent
\textbf{Сурков Алексей Викторович} (р.\ 1978)~--- 
старший научный сотрудник На\-уч\-но-ис\-сле\-до\-ва\-тель\-ско\-го 
института системных исследований Российской академии наук
%117218, Москва, Нахимовский просп., 36, к.1 

\vspace*{7pt}

\noindent 
\textbf{Шестаков Олег Владимирович} (р.\ 1976)~--- доктор 
фи\-зи\-ко-ма\-те\-ма\-ти\-че\-ских, доцент кафедры математической статистики 
факультета вычисли\-тельной математики и кибернетики Московского 
государственного университета им.\ М.\,В.~Ломоносова; 
%119991, Москва ГСП-1, Ленинские горы, д.\ 1, стр. 52
старший научный сотрудник Института проб\-лем информатики
Российской академии наук
%, Москва 119333, ул. Вавилова, д.~44, корп.~2

\vspace*{7pt}

\noindent 
\textbf{Шоргин Сергей Яковлевич} (р.\ 1952.)~--- доктор
фи\-зи\-ко-ма\-те\-ма\-ти\-че\-ских наук, профессор, заместитель директора Института 
проб\-лем информатики Российской академии наук





%%%%%%%%%%%%%%%%%%%%%%%%%%%%%%%%%%%%%%%%%%%%%%%%%%%%%%%%%%%%%%%%%%%%%%%%%%%%%%%




%\def\rightkol{ОБ АВТОРАХ}
%\def\leftkol{ОБ АВТОРАХ}

 \label{end\stat}





%\def\leftfootline{\small{\textbf{\thepage}
%\hfill ИНФОРМАТИКА И ЕЁ ПРИМЕНЕНИЯ\ \ \ том~7\ \ \ выпуск~1\ \ \ 2013}
%}%
% \def\rightfootline{\small{ИНФОРМАТИКА И ЕЁ ПРИМЕНЕНИЯ\ \ \ том~7\ \ \ выпуск~1\ \ \ 2013
%\hfill \textbf{\thepage}}}


%\thispagestyle{myheadings}



\end{multicols}

\newpage


%\vspace*{-48pt}
\begin{center}\LARGE
\textit{About Authors}
\end{center}

\thispagestyle{empty}
\def\tit{\ }

\def\aut{\ }

\def\auf{\ }


\def\leftkol{ABOUT AUTHORS}

\def\rightkol{ABOUT AUTHORS}

\vspace*{-18pt}

\titele{\tit}{\aut}{\auf}{\leftkol}{\rightkol}

%\vspace*{36pt}

\def\rightmark{{\noindent\hbox to \textwidth{\hfill\small ABOUT AUTHORS
%\hfill \large\bf\thepage
}}}
\def\leftmark{{\noindent\parbox{\textwidth}{
%\raggedleft\large\bf\thepage \hfill
\small\textrm{ABOUT AUTHORS}\hfill}}}


\def\leftfootline{\small{\textbf{\thepage}
\hfill ИНФОРМАТИКА И ЕЁ ПРИМЕНЕНИЯ\ \ \ том~6\ \ \ выпуск~2\ \ \ 2012}
}%
 \def\rightfootline{\small{ИНФОРМАТИКА И ЕЁ ПРИМЕНЕНИЯ\ \ \ том~6\ \ \ выпуск~2\ \ \ 2012
\hfill \textbf{\thepage}}}


\begin{multicols}{2}

\noindent
\textbf{Agalarov Yaver M.} (b.\ 1952)~--- Candidate of Science (PhD)
in technology, 
leading scientist, Institute of Informatics Problems, Russian Academy of Sciences

\vspace*{5pt}


  \noindent
\textbf{Bosov Alexey V.} (b.\ 1969)~--- Doctor of Science in technology, Head of
Laboratory, Institute of Informatics Problems, Russian Academy of Sciences

\vspace*{5pt}


\noindent
\textbf{Dulin Sergey K.} (b.\ 1950)~--- Doctor of Science in technology, 
professor, senior scientist, Institute of Informatics Problems, Russian Academy of Sciences

\vspace*{5pt}

\noindent
\textbf{Gorshenin Andrey K.}~--- (b.\ 1986)~--- Candidate of Science (PhD)
in physics and mathematics,
senior scientist, Institute of Informatics Problems, Russian Academy of Sciences

\vspace*{5pt}

\noindent
\textbf{Kalenov Nikolay E.}  (b.\ 1945)~--- Doctor of Science in technology,
professor, Director, Library for Natural Sciences,  Russian Academy of Sciences 

\vspace*{5pt}

\noindent
\textbf{Kalinichenko Leonid A.} (b.\ 1937)~--- Doctor of Science in physics and mathematics, 
professor, Honored scientist of RF, 
Head of Laboratory, Institute of Informatics Problems, Russian Academy of Sciences 

\vspace*{5pt}

\noindent
\textbf{Karpov Alexey A.} (b.\ 1978)~--- Candidate of Science (PhD) in technology, 
senior scientist, St.\ Petersburg Institute for
Informatics and Automation,  Russian Academy of Sciences

\vspace*{5pt}

\noindent
\textbf{Kuznetsov Igor P.} (b.\ 1938)~--- Doctor of Science in technology, 
professor, principal scientist, Institute of Informatics Problems, Russian Academy of Sciences

\vspace*{5pt}


\noindent
\textbf{Markova Natalia A.} (b.\ 1950)~--- Candidate of Science (PhD) in
physics and mathematics, leading scientist,  
Institute of Informatics Problems, Russian Academy of Sciences

\vspace*{5pt}

\noindent
\textbf{Nikolaev Andrey V.} (b.\ 1985)~--- Candidate of Science (PhD) in technology, 
senior lecturer, Tchaikovsky Technological Institute, Branch of the Izhevsk State Technical 
University

\vspace*{6pt}

\noindent
\textbf{Pavlov Igor V.} (b.\ 1945)~---  Doctor of Science in physics and mathematics,
professor, Bauman Moscow State Technical University

\vspace*{6pt}

%\columnbreak

\noindent
\textbf{Rozenberg Igor N.} (b.\ 1965)~--- Doctor of Science in technology, 
First Deputy Director General, Research \& Design Institute for Information 
Technology, Signalling and Telecommunications on Railway Transport (JSC NIIAS)

\vspace*{6pt}


\noindent
\textbf{Semenov Konstantin K.} (b.\ 1986)~--- MPhil, 
associate professor, St.\ Petersburg State Polytechnical University

\vspace*{6pt}

\noindent
\textbf{Sharnin Mikhail M.} (b.\ 1959)~--- Candidate of Science (PhD) 
in technology, senior scientist, Institute of Informatics Problems, Russian Academy of Sciences

\vspace*{6pt}

\noindent 
\textbf{Shestakov Oleg V.} (b.\ 1976)~--- Candidate of Science (PhD) in physics and mathematics,
associate professor, Department of Mathematical Statistics, Faculty of Computational Mathematics and Cybernetics,
M.\,V.~Lomonosov Moscow State University; senior scientist, Institute of Informatics Problems, 
Russian Academy of Sciences

\vspace*{6pt}

\noindent
\textbf{Stupnikov Sergey A.} (b.\ 1978)~--- Candidate of Science (PhD) in technology, 
senior scientist, Institute of Informatics Problems, Russian Academy of Sciences 

\vspace*{6pt}

\noindent
\textbf{Umansky Vladimir I.} (b.\ 1954)~--- Candidate of Science (PhD) in technology, 
Director General, ``IntechGeoTrans'' Closed Joint Stock Company

\vspace*{6pt}

\noindent
\textbf{Zhevnerchuk Dmitry V.} (b.\ 1978)~--- Candidate of Science (PhD) in technology, 
associate professor, Tchaikovsky Technological Institute, Branch of the Izhevsk State 
Technical University

%\vspace*{6pt}

\def\leftfootline{\small{\textbf{\thepage}
\hfill ИНФОРМАТИКА И ЕЁ ПРИМЕНЕНИЯ\ \ \ том~6\ \ \ выпуск~2\ \ \ 2012}
}%
 \def\rightfootline{\small{ИНФОРМАТИКА И ЕЁ ПРИМЕНЕНИЯ\ \ \ том~6\ \ \ выпуск~2\ \ \ 2012
\hfill \textbf{\thepage}}}



%\thispagestyle{myheadings}

\end{multicols}
\newpage

%\def\stat{cont}
{%\hrule\par
%\vskip 7pt % 7pt
\raggedleft\Large \bf%\baselineskip=3.2ex
А\,В\,Т\,О\,Р\,С\,К\,И\,Й\ \ У\,К\,А\,З\,А\,Т\,Е\,Л\,Ь\ \ З\,А\ \ 2\,0\,1\,0 г. \vskip 17pt
    \hrule
    \par
\vskip 21pt plus 6pt minus 3pt }

\label{st\stat}

\def\tit{\ }

\def\aut{\ }
\def\auf{\ }

\def\leftkol{\ } % ENGLISH ABSTRACTS}

\def\rightkol{\ } %АВТОРСКИЙ УКАЗАТЕЛЬ ЗА 2010 г.} %ENGLISH ABSTRACTS}

\titele{\tit}{\aut}{\auf}{\leftkol}{\rightkol}

\vspace*{-12pt}

{\tabcolsep=3pt
\begin{tabular}{p{388pt}rr}
&\textbf{Выпуск} & \textbf{Стр.}\\[6pt]
\hangindent=23pt\noindent\textbf{Арутюнян~А.\,Р.} Моделирование влияния деформаций отпечатков пальцев на 
точность\linebreak
\vspace*{-12pt}\\
\hspace*{23pt}дактилоскопической идентификации$\dotfill$&1&51\\
\hangindent=23pt\noindent\textbf{Архипов~О.\,П., Зыкова~З.\,П.} Интеграция гетерогенной информации о цветных 
пикселях\linebreak
\vspace*{-12pt}\\
\hspace*{23pt}и их цветовосприятии$\dotfill$&4&15\\
\hangindent=23pt\noindent\textbf{Баранов~С.\,И., Френкель~С.\,Л., Захаров~В.\,Н.} Полуформальная верификация 
цифрового устройства с конвейером, основанная на использовании алгоритмических машин\linebreak
\vspace*{-12pt}\\
\hspace*{23pt}состояния$\dotfill$&4&49\\
\textbf{Бекетова~И.\,В.} см.~Каратеев~С.\,Л.&&\\
\textbf{Белоусов~В.\,В.} см.~Синицын~И.\,Н.&&\\
\hangindent=23pt\noindent\textbf{Бенинг~В.\,Е., Королев~Р.\,А.} О предельном поведении мощностей критериев в 
случае\linebreak
\vspace*{-12pt}\\
\hspace*{23pt}распределения Лапласа$\dotfill$&2&63\\
\hangindent=23pt\noindent\textbf{Бенинг~В.\,Е., Сипина~А.\,В.} Асимптотическое разложение для мощности 
критерия,\linebreak
\vspace*{-12pt}\\
\hspace*{23pt}основанного на выборочной медиане, в случае распределения Лапласа$\dotfill$&1&18\\
\textbf{Бондаренко~А.\,В.} см.~Каратеев~С.\,Л.&&\\
\hangindent=23pt\noindent\textbf{Бородина~А.\,В., Морозов~Е.\,В.} Об оценивании асимптотики вероятности 
большого\linebreak
\vspace*{-12pt}\\
\hspace*{23pt}уклонения стационарной регенеративной очереди с одним прибором$\dotfill$&3&29\\
\hangindent=23pt\noindent\textbf{Бунтман~Н.\,В., Минель~Ж.-Л., Ле~Пезан~Д., Зацман~И.\,М.} Типология и 
компьютерное\linebreak
\vspace*{-12pt}\\
\hspace*{23pt}моделирование трудностей перевода$\dotfill$&3&77\\
\textbf{Визильтер~Ю.\,В.} см.~Каратеев~С.\,Л.&&\\
\hangindent=23pt\noindent\textbf{Гавриленко~С.\,В.} Оценки скорости сходимости распределений случайных сумм с 
безгранично делимыми индексами к нормальному закону$\dotfill$&4&81\\
\hangindent=23pt\noindent\textbf{Григорьева~М.\,Е., Шевцова~И.\,Г.} Уточнение неравенства 
Каца--Берри--Эссеена$\dotfill$&2&75\\
\hangindent=23pt\noindent\textbf{Грушо~А.\,А., Грушо~Н.\,А., Тимонина~Е.\,Е.} Поиск конфликтов в политиках 
безопасности: модель случайных графов$\dotfill$&3&38\\
\textbf{Грушо~Н.\,А.} см.~Грушо~А.\,А.&&\\
\hangindent=23pt\noindent\textbf{Гудков~В.\,Ю.} Математические модели изображения отпечатка пальца на основе 
описания линий$\dotfill$&1&58\\
\textbf{Гуртов~А.\,В.} см.~Лукьяненко~А.\,С.&&\\
\textbf{Желтов~С.\,Ю.} см.~Каратеев~С.\,Л.&&\\
\hangindent=23pt\noindent\textbf{Захаров~А.\,А., Серебряков~В.\,А.} Система управления электронной библиотекой 
LibMeta$\dotfill$&4&2\\
\textbf{Захаров~В.\,Н.} см.~Баранов~С.\,И.&&\\
\textbf{Захарова~Т.\,В.} см.~Матвеева~С.\,С.&&\\
\hangindent=23pt\noindent\textbf{Зацаринный~А.\,А., Чупраков~К.\,Г.} Некоторые аспекты выбора технологии для 
постро-\linebreak
\vspace*{-12pt}\\
\hspace*{23pt}ения систем отображения информации ситуационного центра$\dotfill$&3&59\\
\textbf{Зацман~И.\,М.} см.~Бунтман~Н.\,В.&&\\
\hangindent=23pt\noindent\textbf{Зейфман~А.\,И., Коротышева~А.\,В., Сатин~Я.\,А., Шоргин~С.\,Я.} Об 
устойчивости нестаци-\linebreak
\vspace*{-12pt}\\
\hspace*{23pt}онарных систем обслуживания с катастрофами$\dotfill$&3&9\\
\textbf{Зыкова~З.\,П.} см.~Архипов~О.\,П.&&\\
\hangindent=23pt\noindent\textbf{Илюшин~Г.\,Я., Соколов~И.\,А.} Организация управляемого доступа пользователей 
к\linebreak
\vspace*{-12pt}\\
\hspace*{23pt}разнородным ведомственным информационным ресурсам$\dotfill$&1&24\\
\hangindent=23pt\noindent\textbf{Кавагучи~Ю., Ульянов~В.\,В., Фуджикоши~Я.} Приближения для статистик, 
описывающих\linebreak
\vspace*{-12pt}\\
\hspace*{23pt}геометрические свойства данных большой размерности, с оценками 
ошибок$\dotfill$&1&12\\
\hangindent=23pt\noindent\textbf{Каратеев~С.\,Л., Бекетова~И.\,В., Ососков~М.\,В., Князь~В.\,А., 
Визильтер~Ю.\,В., Бондаренко~А.\,В., Желтов~С.\,Ю.} Автоматизированный контроль 
качества цифровых\linebreak
\vspace*{-12pt}\\
\hspace*{23pt}изображений для персональных документов$\dotfill$&1&65\\
\end{tabular}
}

\pagebreak

\def\leftkol{АВТОРСКИЙ УКАЗАТЕЛЬ ЗА 2010 г.} % ENGLISH ABSTRACTS}

\def\rightkol{АВТОРСКИЙ УКАЗАТЕЛЬ ЗА 2010 г.} %ENGLISH ABSTRACTS}

{\tabcolsep=3pt
\begin{tabular}{p{388pt}rr}
&\textbf{Выпуск} & \textbf{Стр.}\\[3pt]
\hangindent=23pt\noindent\textbf{Козеренко~Е.\,Б.} Лингвистические фильтры в статистических моделях машинного\linebreak
\vspace*{-12pt}\\
\hspace*{23pt}перевода$\dotfill$&2&83\\
\hangindent=23pt\noindent\textbf{Козеренко~Е.\,Б., Кузнецов~И.\,П.} Когнитивно-лингвистические представления в 
систе-\linebreak
\vspace*{-12pt}\\
\hspace*{23pt}мах обработки текстов$\dotfill$&3&69\\
\textbf{Князь~В.\,А.} см.~Каратеев~С.\,Л.&&\\
\hangindent=23pt\noindent\textbf{Колесников~А.\,В., Солдатов~С.\,А.} Алгоритм координации для гибридной 
интеллектуальной системы решения сложной задачи оперативно-производственного\linebreak
\vspace*{-12pt}\\
\hspace*{23pt}планирования$\dotfill$&4&61\\
\hangindent=23pt\noindent\textbf{Коновалов~М.\,Г.} О планировании потоков в системах вычислительных 
ресурсов$\dotfill$&2&3\\
\textbf{Конушин~А.\,С.} см.~Конушин~В.\,С.&&\\
\hangindent=23pt\noindent\textbf{Конушин~В.\,С., Кривовязь~Г.\,Р., Конушин~А.\,С.} Алгоритм распознавания людей 
в видео-\linebreak
\vspace*{-12pt}\\
\hspace*{23pt}последовательности по одежде$\dotfill$&1&74\\
\textbf{Корепанов~Э.\, Р.} см.~Синицын~И.\,Н.&&\\
\textbf{Королев~В.\,Ю.} см.~Соколов~И.\,А.&&\\
\textbf{Королев~Р.\,А.} см.~Бенинг~В.\,Е.&&\\
\textbf{Коротышева~А.\,В.} см.~Зейфман~А.\,И.&&\\
\hangindent=23pt\noindent\textbf{Кривенко~М.\,П.} Непараметрическое оценивание элементов байесовского 
клас\-си-\linebreak
\vspace*{-12pt}\\
\hspace*{23pt}фикатора$\dotfill$&2&13\\
\textbf{Кривовязь~Г.\,Р.} см.~Конушин~В.\,С.&&\\
\textbf{Крылов~А.\,С.} см.~Павельева~Е.\,А.&&\\
\hangindent=23pt\noindent\textbf{Крылов~В.\,А.} Моделирование и классификация многоканальных дистанционных\linebreak
\vspace*{-12pt}\\
\hspace*{23pt}изображений с использованием копул$\dotfill$&4&34\\
\hangindent=23pt\noindent\textbf{Крючин~О.\,В.} Разработка параллельных эвристических алгоритмов подбора 
весовых\linebreak
\vspace*{-12pt}\\
\hspace*{23pt}коэффициентов искусственной нейтронной сети$\dotfill$&2&53\\
\hangindent=23pt\noindent\textbf{Кудрявцев~А.\,А., Шоргин~С.\,Я.} Байесовские модели массового обслуживания и 
надеж-\linebreak
\vspace*{-12pt}\\
\hspace*{23pt}ности: характеристики среднего числа заявок в системе $M\vert M \vert 1\vert 
\infty$$\dotfill$&3&16\\
\hangindent=23pt\noindent\textbf{Кузнецов~А.\,А.} Связь между временными и структурно-топологическими 
характери-\linebreak
\vspace*{-12pt}\\
\hspace*{23pt}стиками диаграмм ритма сердца здоровых людей$\dotfill$&4&39\\
\textbf{Кузнецов~И.\,П.} см.~Козеренко~Е.\,Б.&&\\
\textbf{Ле~Пезан~Д.} см.~Бунтман~Н.\,В.&&\\
\hangindent=23pt\noindent\textbf{Лукьяненко~А.\,С., Морозов~Е.\,В., Гуртов~А.\,В.} Анализ сетевого протокола с общей 
функ-\linebreak
\vspace*{-12pt}\\
\hspace*{23pt}цией расширения окна передачи сообщения при конфликтах$\dotfill$&2&46\\
\hangindent=23pt\noindent\textbf{Лямин~О.\,О.} О предельном поведении мощностей критериев в случае обобщенного\linebreak
\vspace*{-12pt}\\
\hspace*{23pt}распределения Лапласа$\dotfill$&3&47\\
\hangindent=23pt\noindent\textbf{Маркин~А.\,В., Шестаков~О.\,В.} Асимптотики оценки риска при пороговой 
обработке\linebreak
\vspace*{-12pt}\\
\hspace*{23pt}вейвлет-вейглет коэффициентов в задаче томографии$\dotfill$&2&36\\
\hangindent=23pt\noindent\textbf{Матвеева~С.\,С., Захарова~Т.\,В.} Сети массового обслуживания с наименьшей 
длиной\linebreak
\vspace*{-12pt}\\
\hspace*{23pt}очереди$\dotfill$&3&22\\
\hangindent=23pt\noindent\textbf{Матюшенко~С.\,И.} Стационарные характеристики двухканальной системы 
обслужива-\linebreak
\vspace*{-12pt}\\
\hspace*{23pt}ния с переупорядочиванием заявок и распределениями фазового типа$\dotfill$&4&68\\
\textbf{Минель~Ж.-Л.} см.~Бунтман~Н.\,В.&&\\
\textbf{Морозов~Е.\,В.} см.~Бородина~А.\,В.&&\\
\textbf{Морозов~Е.\,В.} см.~Лукьяненко~А.\,С.&&\\
\textbf{Ососков~М.\,В.} см.~Каратеев~С.\,Л.&&\\
\hangindent=23pt\noindent\textbf{Павельева~Е.\,А., Крылов~А.\,С.} Поиск и анализ ключевых точек радужной 
оболочки\linebreak
\vspace*{-12pt}\\
\hspace*{23pt}глаза методом преобразования Эрмита$\dotfill$&1&79\\
\textbf{Печинкин~А.\,В.} см.~Френкель~С.\,Л.,&&\\
\hangindent=23pt\noindent\textbf{Протасов~В.\,И.} Составление субъективного портрета с использованием 
эволюционно-\linebreak
\vspace*{-12pt}\\
\hspace*{23pt}го морфинга и квалиметрия метода$\dotfill$&1&83\\
\hangindent=23pt\noindent\textbf{Рудаков~К.\,В., Торшин~И.\,Ю.} Вопросы разрешимости задачи распознавания 
вторичной\linebreak
\vspace*{-12pt}\\
\hspace*{23pt}структуры белка$\dotfill$&2&25\\
\textbf{Сатин~Я.\,А.} см.~Зейфман~А.\,И.&&\\
\hangindent=23pt\noindent\textbf{Сейфуль-Мулюков~Р.\,Б.} Нефть как носитель информации о своем 
происхождении,\linebreak
\vspace*{-12pt}\\
\hspace*{23pt}структуре и эволюции$\dotfill$&1&41\\
\end{tabular}
}

{\tabcolsep=3pt
\begin{tabular}{p{388pt}rr}
&\textbf{Выпуск} & \textbf{Стр.}\\[6pt]
\textbf{Семендяев~Н.\,Н.} см.~Синицын~И.\,Н.&&\\
\textbf{Серебряков~В.\,А.} см.~Захаров~А.\,А.&&\\
\textbf{Синицын~В.\,И.} см.~Синицын~И.\,Н.&&\\
\hangindent=23pt\noindent\textbf{Синицын~И.\,Н., Синицын~В.\,И., Корепанов~Э.\, Р., Белоусов~В.\,В., 
Семендяев~Н.\,Н.} Оперативное построение информационных моделей движения полюса 
Земли\linebreak
\vspace*{-12pt}\\
\hspace*{23pt}методами линейных и линеаризованных фильтров$\dotfill$&1&2\\
\textbf{Сипина~А.\,В.} см.~Бенинг~В.\,Е.&&\\
\hangindent=23pt\noindent\textbf{Соколов~И.\,А.} О работах заслуженного деятеля науки Российской Федерации 
И.\,Н.~Синицына в области информационных технологий и автоматизации (к 70-летию\linebreak
\vspace*{-12pt}\\
\hspace*{23pt}со дня рождения)$\dotfill$&3&84\\
\textbf{Соколов~И.\,А.} см.~Илюшин~Г.\,Я.&&\\
\hangindent=23pt\noindent\textbf{Соколов~И.\,А., Королев~В.\,Ю.} Предисловие$\dotfill$&2&2\\
\textbf{Солдатов~С.\,А.} см.~Колесников~А.\,В.&&\\
\hangindent=23pt\noindent\textbf{Степанов~С.\,Ю.} Использование координатного метода фрагментации 
коммутаторной\linebreak
\vspace*{-12pt}\\
\hspace*{23pt}нейронной сети для сокращения трафика$\dotfill$&2&57\\
\textbf{Тимонина~Е.\,Е.} см.~Грушо~А.\,А.&&\\
\textbf{Торшин~И.\,Ю.} см.~Рудаков~К.\,В.&&\\
\textbf{Ульянов~В.\,В.} см.~Кавагучи~Ю.&&\\
\textbf{Фазекаш~И.} см.~Чупрунов~А.\,Н.&&\\
\textbf{Френкель~С.\,Л.} см.~Баранов~С.\,И.&&\\
\hangindent=23pt\noindent\textbf{Френкель~С.\,Л., Печинкин~А.\,В.} Оценка времени самовосстановления в 
цифровых\linebreak
\vspace*{-12pt}\\
\hspace*{23pt}системах после сбоев, вызываемых переходными помехами$\dotfill$&3&2\\
\textbf{Фуджикоши~Я.} см.~Кавагучи~Ю.&&\\
\hangindent=23pt\noindent\textbf{Цискаридзе~А.\,К.} Математическая модель и метод восстановления позы человека 
по\linebreak
\vspace*{-12pt}\\
\hspace*{23pt}стереопаре силуэтных изображений$\dotfill$&4&27\\
\hangindent=23pt\noindent\textbf{Чупраков~К.\,Г.} К вопросу о размещении коллективных средств отображения в 
ситуа-\linebreak
\vspace*{-12pt}\\
\hspace*{23pt}ционном зале с заданными параметрами$\dotfill$&4&89\\
\textbf{Чупраков~К.\,Г.} см.~Зацаринный~А.\,А.&&\\
\hangindent=23pt\noindent\textbf{Чупрунов~А.\,Н., Фазекаш~И.} Законы повторного логарифма для числа 
безошибочных\linebreak
\vspace*{-12pt}\\
\hspace*{23pt}блоков при помехоустойчивом кодировании$\dotfill$&3&42\\
\textbf{Шевцова~И.\,Г.} см.~Григорьева~М.\,Е.&&\\
\hangindent=23pt\noindent\textbf{Шестаков~О.\,В.} Аппроксимация распределения оценки риска пороговой 
обработки вейвлет-коэффициентов нормальным распределением при использовании 
выбо-\linebreak
\vspace*{-12pt}\\
\hspace*{23pt}рочной дисперсии$\dotfill$&4&73\\
\textbf{Шестаков~О.\,В.} см.~Маркин~А.\,В.&&\\
\textbf{Шоргин~С.\,Я.} см.~Зейфман~А.\,И.&&\\
\textbf{Шоргин~С.\,Я.} см.~Кудрявцев~А.\,А.&&\\
\end{tabular}
}

%\thispagestyle{myheadings}
\def\leftfootline{\small{\textbf{\thepage}
\hfill ИНФОРМАТИКА И ЕЁ ПРИМЕНЕНИЯ\ \ \ том~4\ \ \ выпуск~4\ \ \ 2010}
}%
 \def\rightfootline{\small{ИНФОРМАТИКА И ЕЁ ПРИМЕНЕНИЯ\ \ \ том~4\ \ \ выпуск~4\ \ \ 2010
 \hfill \textbf{\thepage}}}
 \label{end\stat}

%
%Том 10 Выпуск 1-4 Год 2016

\def\stat{cont-e}
{%\hrule\par
%\vskip 7pt % 7pt
\raggedleft\Large \bf%\baselineskip=3.2ex
2\,0\,1\,6\ \ A\,U\,T\,H\,O\,R\ \ I\,N\,D\,E\,X \vskip 17pt
 \hrule
 \par
\vskip 21pt plus 6pt minus 3pt }

\label{st\stat}

\def\tit{\ }

\def\aut{\ }
\def\auf{\ }

\def\leftkol{\ } %2016 AUTHOR INDEX} % ENGLISH ABSTRACTS}

\def\rightkol{\ } %2016 AUTHOR INDEX} %ENGLISH ABSTRACTS}

\titele{\tit}{\aut}{\auf}{\leftkol}{\rightkol}

\def\leftfootline{\small{\textbf{\thepage}
\hfill INFORMATIKA I EE PRIMENENIYA~--- INFORMATICS AND APPLICATIONS\ \ \ 2016\
\ \ volume~10\ \ \ issue\ 4}
}%
 \def\rightfootline{\small{INFORMATIKA I EE PRIMENENIYA~--- INFORMATICS AND APPLICATIONS\ \ \ 2016\ \ \ volume~10\ \ \ issue\ 4
\hfill \textbf{\thepage}}}

\vspace*{-12pt}
\vspace*{-18pt}

{\tabcolsep=2.8pt
\begin{tabular}{p{382pt}cc}
&\textbf{Issue} & \textbf{Page}\\[6pt]
\Avtors{Agalarov~M.\,Ya.} see~Agalarov~Ya.\,M.&&\\
\Avtors{Agalarov~Ya.\,M., Agalarov~M.\,Ya., and
Shorgin~V.\,S.} About the optimal threshold of queue\linebreak
\\[-12pt]
\hspace*{23pt}length in a~particular problem of profit maximization
in the $M/G/1$ queuing system&2&70--79\\
\Avtors{Alexeyevsky~D.\,A.} BioNLP ontology extraction from 
a~restricted language corpus with\linebreak
\\[-12pt]
\hspace*{23pt}context-free grammars&1&119--128\\
\Avtors{Andreev~S.\,D.} see~Gaidamaka~Yu.\,V.&&\\
\Avtors{Andreev~S.\,D.} see~Ometov~A.\,Ya.&&\\
\Avtors{Arkhipov~O.\,P., Arkhipov~P.\,O., and Sidorkin~I.\,I.} The
option to create a~local coordinate\linebreak
\\[-12pt]
\hspace*{23pt}system for synchronization of selected images&3&91--97\\
\Avtors{Arkhipov~P.\,O.} see~Arkhipov~O.\,P.&&\\
\Avtors{Belousov~V.\,V.} see~Shnurkov~P.\,V.&&\\
\Avtors{Belousov~V.\,V.} see~Shnurkov~P.\,V.&&\\
\Avtors{Bening~V.\,E.} Calculation of~the~asymptotic deficiency
of~some statistical procedures based\linebreak
\\[-12pt]
\hspace*{23pt}on~samples with~random sizes&4&34--45\\
\Avtors{Borisov~A.\,V., Bosov~A.\,V., and Miller~G.\,B.} Modeling and
monitoring of VoIP connection&2&\hphantom{1}2--13\\
\Avtors{Bosov~A.\,V.} see~Borisov~A.\,V.&&\\
\Avtors{Briukhov~D.\,O.} see~Stupnikov~S.\,A.&&\\
\Avtors{Callaos~N.\,K.\ and Seyful-Mulyukov~R.\,B.} Complexity and
its information content&1&129--139\\
\Avtors{Chertok~A.\,V., Kadaner~A.\,I., Khazeeva~G.\,T., and
Sokolov~I.\,A.} Regime switching detection\linebreak
\\[-12pt]
\hspace*{23pt}for~the~Levy driven
Ornstein--Uhlenbeck process using CUSUM methods&4&46--56\\
\Avtors{Chichagov~V.\,V.} Asymptotic expansions of mean absolute
error of uniformly minimum variance unbiased and maximum likelihood
estimators on the one-parameter exponential\linebreak
\\[-12pt]
\hspace*{23pt}family model of lattice distributions&3&66--76\\
\Avtors{Danishevsky~V.\,I.} see~Kolesnikov A.\,V.&&\\
\Avtors{Fazliev~A.\,Z.} see~Kalinichenko~L.\,A.&&\\
\Avtors{Fedoseev~A.\,A.} What is behind the concept of ``knowledge in
small packages''&3&105--110\\
\Avtors{Gaidamaka~Yu.\,V., Andreev~S.\,D., Sopin~E.\,S.,
Samouylov~K.\,E., and Shorgin~S.\,Ya.} Interference analysis
of~the~device-to-device communications model with~regard to~a~signal\linebreak
\\[-12pt]
\hspace*{23pt}propagation environment&4&\hphantom{1}2--10\\
\Avtors{Gasilov~A.\,V.} see~Yakovlev~O.\,A.&&\\
\Avtors{Goncharov~A.\,V.\ and Strijov~V.\,V.} Metric time series
classification using weighted dynamic\linebreak
\\[-12pt]
\hspace*{23pt}warping relative to centroids of classes&2&36--47\\
\Avtors{Gordov~E.\,P.} see~Kalinichenko~L.\,A.&&\\
\Avtors{Gorshenin~A.\,K.} Concept of online service for stochastic
modeling of real processes&1&72--81\\
\Avtors{Gorshenin~A.\,K.} see~Shnurkov~P.\,V.&&\\
\Avtors{Gorshenin~A.\,K.} see~Shnurkov~P.\,V.&&\\
\Avtors{Grusho~A.\,A., Grusho~N.\,A., Zabezhailo~M.\,I., and
Timonina~E.\,E.} Integration of statistical and\linebreak
\\[-12pt]
\hspace*{23pt}deterministic methods for
analysis of information security&3&2--8\\
\Avtors{Grusho~A.\,A., Zabezhailo~M.\,I., and Zatsarinny~A.\,A.} On
the advanced procedure to reduce\linebreak
\\[-12pt]
\hspace*{23pt}calculation of Galois closures&4&\hphantom{1}96--104\\
\Avtors{Grusho~N.\,A.} see~Grusho~A.\,A.&&\\
\Avtors{Havanskov~V.\,A.} see~Minin~V.\,A.&&\\
\Avtors{Inkova~O.\,Yu.} see~Zatsman~I.\,M.&&\\
\Avtors{Isachenko~R.\,V.\ and Strijov~V.\,V.} Metric learning in
multiclass time series classification\linebreak
\\[-12pt]
\hspace*{23pt}problem&2&48--57\\
\end{tabular}
}
\pagebreak

\def\leftfootline{\small{\textbf{\thepage}
\hfill INFORMATIKA I EE PRIMENENIYA~--- INFORMATICS AND APPLICATIONS\ \ \ 2016\
\ \ volume~10\ \ \ issue\ 4}
}%
 \def\rightfootline{\small{INFORMATIKA I EE PRIMENENIYA~---
INFORMATICS AND APPLICATIONS\ \ \ 2016\ \ \ volume~10\ \ \ issue\ 4
\hfill \textbf{\thepage}}}

\def\leftkol{2016 AUTHOR INDEX} % ENGLISH ABSTRACTS}

\def\rightkol{2016 AUTHOR INDEX} %ENGLISH ABSTRACTS}


{\tabcolsep=2.83pt
\begin{tabular}{p{382pt}cc}
&\textbf{Issue} & \textbf{Page}\\[6pt]
\Avtors{Kadaner~A.\,I.} see~Chertok~A.\,V.&&\\[.255pt]
\Avtors{Kalinichenko~L.\,A., Volnova~A.\,A., Gordov~E.\,P.,
Kiselyova~N.\,N., Kovaleva~D.\,A., Malkov~O.\,Yu., Okladnikov~I.\,G.,
Podkolodnyy~N.\,L., Pozanenko~A.\,S., Ponomareva~N.\,V.,
Stupnikov~S.\,A.,} \textbf{and Fazliev~A.\,Z.} Data access challenges for data
intensive\linebreak
\\[-12pt]
\hspace*{23pt}research in Russia&1& 2--22\\[.255pt]
\Avtors{Karasikov~M.\,E.\ and Strijov~V.\,V.} Feature-based
time-series classification&4&121--131\\[.255pt]
\Avtors{Khazeeva~G.\,T.} see~Chertok~A.\,V.&&\\[.255pt]
\Avtors{Khokhlov~Yu.\,S.} Multivariate fractional Levy motion and its
applications&2&\hphantom{1}98--106\\[.255pt]
\Avtors{Kirikov~I.\,A., Kolesnikov~A.\,V., Listopad~S.\,V., and
Rumovskaya~S.\,B.} Fine-grained hybrid\linebreak
\\[-12pt]
\hspace*{23pt}intelligent systems. Part 2:
Bidirectional hybridization&1&\hphantom{1}96--105\\[.255pt]
\Avtors{Kirikov~I.\,A., Kolesnikov~A.\,V., Listopad~S.\,V., and
Rumovskaya~S.\,B.} ``Virtual council''~---\linebreak
\\[-12pt]
\hspace*{23pt}source environment
supporting complex diagnostic decision making&3&81--90\\[.255pt]
\Avtors{Kiselyova~N.\,N.} see~Kalinichenko~L.\,A.&&\\[.255pt]
\Avtors{Kolesnikov A.\,V., Listopad~S.\,V., Rumovskaya~S.\,B., and
Danishevsky~V.\,I.} Informal axiomatic\linebreak
\\[-12pt]
\hspace*{23pt}theory of~the~role visual models&4&114--120\\[.255pt]
\Avtors{Kolesnikov~A.\,V.} see~Kirikov~I.\,A.&&\\[.255pt]
\Avtors{Kolesnikov~A.\,V.} see~Kirikov~I.\,A.&&\\[.255pt]
\Avtors{Kolin~K.\,K.} Humanitarian aspects of information
security&3&111--121\\[.255pt]
\Avtors{Konovalov~M.\,G.\ and Razumchik~R.\,V.} Dispatching
to~two parallel nonobservable queues using\linebreak
\\[-12pt]
\hspace*{23pt}only static
information&4&57--67\\[.255pt]
\Avtors{Korchagin~A.\,Yu.} see~Korolev~V.\,Yu.&&\\[.255pt]
\Avtors{Korchagin~A.\,Yu.} see~Korolev~V.\,Yu.&&\\[.255pt]
\Avtors{Korepanov~E.\,R.} see~Sinitsyn~I.\,N.&&\\[.255pt]
\Avtors{Korepanov~E.\,R.} see~Sinitsyn~I.\,N.&&\\[.255pt]
\Avtors{Korolev~V.\,Yu., Korchagin~A.\,Yu., and Zeifman~A.\,I.} The
Poisson theorem for Bernoulli trials\linebreak
\\[-12pt]
\hspace*{23pt}with~a~random probability
of~success and~a~discrete analog of~the~Weibull distribution&4&11--20\\[.255pt]
\Avtors{Korolev~V.\,Yu., Zeifman~A.\,I., and Korchagin~A.\,Yu.}
Asymmetric Linnik distributions as~limit\linebreak
\\[-12pt]
\hspace*{23pt}laws for~random sums
of~independent random variables with~finite variances&4&21--33\\[.255pt]
\Avtors{Koucheryavy~E.\,A.} see~Ometov~A.\,Ya.&&\\[.255pt]
\Avtors{Kovaleva~D.\,A.} see~Kalinichenko~L.\,A.&&\\[.255pt]
\Avtors{Kovalyov~S.\,P.} Metaprogramming to increase
manufacturability of large-scale software-\linebreak
\\[-12pt]
\hspace*{23pt}intensive systems&1&56--66\\[.255pt]
\Avtors{Krivenko~M.\,P.} Significance tests of feature selection for
classification&3&32--40\\[.255pt]
\Avtors{Kruzhkov~M.\,G.} see~Zalizniak~Anna~A.&&\\[.255pt]
\Avtors{Kruzhkov~M.\,G.} see~Zatsman~I.\,M.&&\\[.255pt]
\Avtors{Kudryavtsev~A.\,A.} Bayesian queueing and reliability models:
\textit{A~priori} distributions with\linebreak
\\[-12pt]
\hspace*{23pt}compact support&1&67--71\\[.255pt]
\Avtors{Kudryavtsev~A.\,A.} Characteristics dependent on the balance
coefficient in Bayesian models\linebreak
\\[-12pt]
\hspace*{23pt}with compact support of \textit{a priori}
distributions&3&77--80\\[.255pt]
\Avtors{Kudryavtsev~A.\,A.\ and Palionnaia~S.\,I.} Bayesian recurrent
model of reliability growth:\linebreak
\\[-12pt]
\hspace*{23pt}Parabolic distribution of parameters&2&80--83\\[.255pt]
\Avtors{Kudryavtsev~A.\,A.\ and Titova~A.\,I.} Bayesian queuing
and~reliability models: Degenerate-\linebreak
\\[-12pt]
\hspace*{23pt}Weibull case&4&68--71\\[.255pt]
\Avtors{Leontyev~N.\,D.\ and Ushakov~V.\,G.} Analysis of a queueing
system with autoregressive arrivals\linebreak
\\[-12pt]
\hspace*{23pt}and nonpreemptive priority&3&15--22\\[.255pt]
\Avtors{Listopad~S.\,V.} see~Kirikov~I.\,A.&&\\[.255pt]
\Avtors{Listopad~S.\,V.} see~Kirikov~I.\,A.&&\\[.255pt]
\Avtors{Listopad~S.\,V.} see~Kolesnikov A.\,V.&&\\[.255pt]
\Avtors{Malkov~O.\,Yu.} see~Kalinichenko~L.\,A.&&\\[.255pt]
\Avtors{Markov~A.\,S., Monakhov~M.\,M., and
Ulyanov~V.\,V.} Generalized Cornish--Fisher expansions\linebreak
\\[-12pt]
\hspace*{23pt}for distributions of statistics based on samples
of random size&2&84--91\\[.255pt]
\Avtors{Melnikov~A.\,K.\ and Ronzhin~A.\,F.} Generalized statistical
method of~text analysis based\linebreak
\\[-12pt]
\hspace*{23pt}on~calculation of~probability distributions
of~statistical values&4&89--95\\
\end{tabular}
}
\pagebreak

\def\leftfootline{\small{\textbf{\thepage}
\hfill INFORMATIKA I EE PRIMENENIYA~--- INFORMATICS AND APPLICATIONS\ \ \ 2016\
\ \ volume~10\ \ \ issue\ 4}
}%
 \def\rightfootline{\small{INFORMATIKA I EE PRIMENENIYA~---
INFORMATICS AND APPLICATIONS\ \ \ 2016\ \ \ volume~10\ \ \ issue\ 4
\hfill \textbf{\thepage}}}

\def\leftkol{2016 AUTHOR INDEX} % ENGLISH ABSTRACTS}

\def\rightkol{2016 AUTHOR INDEX} %ENGLISH ABSTRACTS}


{\tabcolsep=3pt
\begin{tabular}{p{381pt}cc}
&\textbf{Issue} & \textbf{Page}\\[6pt]
\Avtors{Meykhanadzhyan~L.\,A.} Stationary characteristics of the finite
capacity queueing system with\linebreak
\\[-12pt]
\hspace*{23pt}inverse service order and generalized
probabilistic priority&2&123--131\\[.23pt]
\Avtors{Miller~G.\,B.} see~Borisov~A.\,V.&&\\[.23pt]
\Avtors{Minin~V.\,A., Zatsman~I.\,M., Havanskov~V.\,A., and
Shubnikov~S.\,K.} Intensity of citation of scientific publications in
inventions on information and computer technologies patented\linebreak
\\[-12pt]
\hspace*{23pt}in Russia by domestic and foreign applicants&2&107--122\\[.23pt]
\Avtors{Monakhov~M.\,M.} see~Markov~A.\,S.&&\\[.23pt]
\Avtors{Naumov~V.\,A.\ and Samouylov~K.\,E.} On relationship
between queuing systems with resources\linebreak
\\[-12pt]
\hspace*{23pt}and Erlang networks&3&\hphantom{1}9--14\\[.23pt]
\Avtors{Okladnikov~I.\,G.} see~Kalinichenko~L.\,A.&&\\[.23pt]
\Avtors{Ometov~A.\,Ya., Andreev~S.\,D., Turlikov~A.\,M., and
Koucheryavy~E.\,A.} Performance analysis of\linebreak
\\[-12pt]
\hspace*{23pt}a wireless data
aggregation system with contention for contemporary sensor
networks&3&23--31\\[.23pt]
\Avtors{Palionnaia~S.\,I.} see~Kudryavtsev~A.\,A.&&\\[.23pt]
\Avtors{Podkolodnyy~N.\,L.} see~Kalinichenko~L.\,A.&&\\[.23pt]
\Avtors{Ponomareva~N.\,V.} see~Kalinichenko~L.\,A.&&\\[.23pt]
\Avtors{Popkova~N.\,A.} see~Zatsman~I.\,M.&&\\[.23pt]
\Avtors{Pozanenko~A.\,S.} see~Kalinichenko~L.\,A.&&\\[.23pt]
\Avtors{Razumchik~R.\,V.} see~Konovalov~M.\,G.&&\\[.23pt]
\Avtors{Ronzhin~A.\,F.} see~Melnikov~A.\,K.&&\\[.23pt]
\Avtors{Rumovskaya~S.\,B.} see~Kirikov~I.\,A.&&\\[.23pt]
\Avtors{Rumovskaya~S.\,B.} see~Kirikov~I.\,A.&&\\[.23pt]
\Avtors{Rumovskaya~S.\,B.} see~Kolesnikov A.\,V.&&\\[.23pt]
\Avtors{Samouylov~K.\,E.} see~Gaidamaka~Yu.\,V.&&\\[.23pt]
\Avtors{Samouylov~K.\,E.} see~Naumov~V.\,A.&&\\[.23pt]
\Avtors{Serebryanskii~S.\,M.} see~Tyrsin~A.\,N.&&\\[.23pt]
\Avtors{Seyful-Mulyukov~R.\,B.} see~Callaos~N.\,K.&&\\[.23pt]
\Avtors{Shestakov~O.\,V.} Statistical properties of the denoising method
based on the stabilized hard\linebreak
\\[-12pt]
\hspace*{23pt}thresholding&2&65--69\\[.23pt]
\Avtors{Shestakov~O.\,V.} The strong law of large numbers for the risk
estimate in the problem of\linebreak
\\[-12pt]
\hspace*{23pt}tomographic image reconstruction from
projections with a correlated noise&3&41--45\\[.23pt]
\Avtors{Shestakov~O.\,V.} see~Zakharova~T.\,V.&&\\[.23pt]
\Avtors{Shnurkov~P.\,V., Gorshenin~A.\,K., and Belousov~V.\,V.}
Analytical solution of~the~optimal control\linebreak
\\[-12pt]
\hspace*{23pt}task of~a~semi-Markov
process with~finite set of~states&4&72--88\\[.23pt]
\Avtors{Shnurkov~P.\,V., Zasypko~V.\,V., Belousov~V.\,V., and
Gorshenin~A.\,K.} Development of the algorithm of numerical solution
of the optimal investment control problem\linebreak
\\[-12pt]
\hspace*{23pt}in the closed dynamical model of three-sector economy&1&82--95\\[.23pt]
\Avtors{Shorgin~S.\,Ya.} see~Gaidamaka~Yu.\,V.&&\\[.23pt]
\Avtors{Shorgin~V.\,S.} see~Agalarov~Ya.\,M.&&\\[.23pt]
\Avtors{Shubnikov~S.\,K.} see~Minin~V.\,A.&&\\[.23pt]
\Avtors{Sidorkin~I.\,I.} see~Arkhipov~O.\,P.&&\\[.23pt]
\Avtors{Sinitsyn~I.\,N.} Analytical modeling of processes in stochastic
systems with complex fractional\linebreak
\\[-12pt]
\hspace*{23pt}order Bessel nonlinearities&3&55--65\\[.23pt]
\Avtors{Sinitsyn~I.\,N.} Orthogonal supoptimal filters for nonlinear
stochastic systems on manifolds&1&34--44\\[.23pt]
\Avtors{Sinitsyn~I.\,N.\ and Korepanov~E.\,R.} Normal Pugachev
conditionally-optimal filters and extra-\linebreak
\\[-12pt]
\hspace*{23pt}polators for state linear stochastic systems&2&14--23\\[.23pt]
\Avtors{Sinitsyn~I.\,N.\ and Sinitsyn~V.\,I.} Analytical modeling of
distributions in stochastic systems on\linebreak
\\[-12pt]
\hspace*{23pt}manifolds based on ellipsoidal approximation&1&45--55\\[.23pt]
\Avtors{Sinitsyn~I.\,N., Sinitsyn~V.\,I., and
Korepanov~E.\,R.} Ellipsoidal suboptimal filters for nonlinear\linebreak
\\[-12pt]
\hspace*{23pt}stochastic systems on manifolds&2&24--35\\[.23pt]
\Avtors{Sinitsyn~V.\,I.} see~Sinitsyn~I.\,N.&&\\[.23pt]
\Avtors{Sinitsyn~V.\,I.} see~Sinitsyn~I.\,N.&&\\[.23pt]
\Avtors{Skvortsov~N.\,A.} see~Stupnikov~S.\,A.&&\\[.23pt]
\Avtors{Sokolov~I.\,A.} see~Chertok~A.\,V.&&\\
\end{tabular}
}
\pagebreak

\def\leftfootline{\small{\textbf{\thepage}
\hfill INFORMATIKA I EE PRIMENENIYA~--- INFORMATICS AND APPLICATIONS\ \ \ 2016\
\ \ volume~10\ \ \ issue\ 4}
}%
 \def\rightfootline{\small{INFORMATIKA I EE PRIMENENIYA~---
INFORMATICS AND APPLICATIONS\ \ \ 2016\ \ \ volume~10\ \ \ issue\ 4
\hfill \textbf{\thepage}}}

\def\leftkol{2016 AUTHOR INDEX} % ENGLISH ABSTRACTS}

\def\rightkol{2016 AUTHOR INDEX} %ENGLISH ABSTRACTS}


{\tabcolsep=3pt
\begin{tabular}{p{382pt}cc}
&\textbf{Issue} & \textbf{Page}\\[6pt]
\Avtors{Sopin~E.\,S.} see~Gaidamaka~Yu.\,V.&&\\
\Avtors{Strijov~V.\,V.} see~Goncharov~A.\,V.&&\\
\Avtors{Strijov~V.\,V.} see~Isachenko~R.\,V.&&\\
\Avtors{Strijov~V.\,V.} see~Karasikov~M.\,E.&&\\
\Avtors{Stupnikov~S.\,A., Briukhov~D.\,O., and Skvortsov~N.\,A.}
Co-lending systemic risk analysis over\linebreak
\\[-12pt]
\hspace*{23pt}heterogeneous data collections&1&23--33\\
\Avtors{Stupnikov~S.\,A.} see~Kalinichenko~L.\,A.&&\\
\Avtors{Suchkov~A.\,P.} see~Zatsarinny~A.\,A.&&\\
\Avtors{Timonina~E.\,E.} see~Grusho~A.\,A.&&\\
\Avtors{Titova~A.\,I.} see~Kudryavtsev~A.\,A.&&\\
\Avtors{Turlikov~A.\,M.} see~Ometov~A.\,Ya.&&\\
\Avtors{Tyrsin~A.\,N.\ and Serebryanskii~S.\,M.} Recognition of
dependences on the basis of inverse\linebreak
\\[-12pt]
\hspace*{23pt}mapping&2&58--64\\
\Avtors{Ulyanov~V.\,V.} see~Markov~A.\,S.&&\\
\Avtors{Ushakov~V.\,G.} Queueing system with working vacations and
hyperexponential input stream&2&92--97\\
\Avtors{Ushakov~V.\,G.} see~Leontyev~N.\,D.&&\\
\Avtors{Volnova~A.\,A.} see~Kalinichenko~L.\,A.&&\\
\Avtors{Yakovlev~O.\,A.\ and Gasilov~A.\,V.} Speeded-up stereo
matching using geodesic support weights&3&\hphantom{1}98--104\\
\Avtors{Zabezhailo~M.\,I.} see~Grusho~A.\,A.&&\\
\Avtors{Zabezhailo~M.\,I.} see~Grusho~A.\,A.&&\\
\Avtors{Zakharova~T.\,V.\ and Shestakov~O.\,V.} Precision analysis of
wavelet processing of aerodynamic\linebreak
\\[-12pt]
\hspace*{23pt}flow patterns&3&46--54\\
\Avtors{Zalizniak~Anna~A.\ and Kruzhkov~M.\,G.} Database
of~Russian impersonal verbal constructions&4&132--141\\
\Avtors{Zasypko~V.\,V.} see~Shnurkov~P.\,V.&&\\
\Avtors{Zatsarinny~A.\,A.\ and Suchkov~A.\,P.} Systems engineering
approaches to~the~establishment of\linebreak
\\[-12pt]
\hspace*{23pt}a~system for~decision support based
on~situational analysis&4&105--113\\
\Avtors{Zatsarinny~A.\,A.} see~Grusho~A.\,A.&&\\
\Avtors{Zatsman~I.\,M., Inkova~O.\,Yu., Kruzhkov~M.\,G., and
Popkova~N.\,A.} Representation of cross-\linebreak
\\[-12pt]
\hspace*{23pt}lingual knowledge about
connectors in supracorpora databases&1&106--118\\
\Avtors{Zatsman~I.\,M.} see~Minin~V.\,A.&&\\
\Avtors{Zeifman~A.\,I.} see~Korolev~V.\,Yu.&&\\
\Avtors{Zeifman~A.\,I.} see~Korolev~V.\,Yu.&&\\
\end{tabular}
}

%\thispagestyle{myheadings}
\def\leftfootline{\small{\textbf{\thepage}
\hfill INFORMATIKA I EE PRIMENENIYA~--- INFORMATICS AND APPLICATIONS\ \ \ 2016\
\ \ volume~10\ \ \ issue\ 4}
}%
 \def\rightfootline{\small{INFORMATIKA I EE PRIMENENIYA~---
INFORMATICS AND APPLICATIONS\ \ \ 2016\ \ \ volume~10\ \ \ issue\ 4
\hfill \textbf{\thepage}}}

 \label{end\stat}

\newpage


\vspace*{-60pt} {\small
{\baselineskip=9.1pt
\section*{Правила подготовки рукописей статей для публикации в журнале
<<Информатика и её применения>>}

\thispagestyle{empty}

 Журнал <<Информатика и её применения>> публикует
теоретические, обзорные и дискуссионные статьи, посвященные научным
исследованиям и разработкам в области информатики и ее приложений. Журнал
издается на русском языке. По специальному решению редколлегии отдельные статьи,
в виде исключения, могут печататься на английском языке.
Тематика журнала охватывает следующие направления:
\begin{itemize}
\item теоретические основы информатики; %\\[-13.5pt]
\item математические методы исследования сложных систем и процессов; %\\[-13.5pt]
\item информационные системы и сети; %\\[-13.5pt]
\item информационные технологии; %\\[-13.5pt]
\item архитектура и программное
обеспечение вычислительных комплексов и сетей.
\end{itemize}
\begin{enumerate}
\item В журнале печатаются результаты, ранее не
опубликованные и не предназначенные к одновременной публикации в других
изданиях. Публикация не должна нарушать закон об авторских правах. Направляя
свою рукопись в редакцию, авторы автоматически передают учредителям и
редколлегии неисключительные права на издание данной статьи на русском языке и
на ее распространение в России и за рубежом. При этом за авторами сохраняются
все права как собственников данной рукописи. В связи с этим авторами должно
быть представлено в редакцию письмо в следующей форме:
Соглашение о передаче права на публикацию:

\textit{<<Мы, нижеподписавшиеся, авторы рукописи <<$\qquad\qquad$>>, передаем
учредителям и редколлегии журнала <<Информатика и её применения>>
неисключительное право опубликовать данную рукопись статьи на русском языке как
в печатной, так и в электронной версиях журнала. Мы подтверждаем, что данная
публикация не нарушает авторского права других лиц или организаций. Подписи
авторов: (ф.\,и.\,о., дата, адрес)>>.}

Указанное соглашение может быть представлено 
как в бумажном виде, так и в виде отсканированной копии (с подписями авторов).


Редколлегия вправе запросить у авторов экспертное заключение о возможности
опубликования представленной статьи в открытой печати. %\\[-13.5pt]
\item Статья
подписывается всеми авторами. На отдельном листе представляются данные автора
(или всех авторов): фамилия, полные имя и отчество, телефон, факс, e-mail,
почтовый адрес. Если работа выполнена несколькими авторами, указывается фамилия
одного из них, ответственного за переписку с редакцией. %\\[-13.5pt]
\item Редакция журнала
осуществляет самостоятельную экспертизу присланных статей. Возвращение рукописи
на доработку не означает, что статья уже принята к печати. Доработанный вариант
с ответом на замечания рецензента необходимо прислать в редакцию. %\\[-13.5pt]
\item Решение
редакционной коллегии о принятии статьи к печати или ее отклонении сообщается
авторам. Редколлегия не обязуется направлять рецензию авторам отклоненной
статьи. %\\[-13.5pt]
\item Корректура статей высылается авторам для просмотра. Редакция
просит авторов присылать свои замечания в кратчайшие сроки. %\\[-13.5pt]
\item При
подготовке рукописи в MS Word рекомендуется использовать следующие настройки.
Параметры страницы: формат~--- А4; ориентация~--- книжная; поля (см): внутри~---
2,5, снаружи~--- 1,5, сверху~--- 2, снизу~--- 2, от края до нижнего
колонтитула~--- 1,3. Основной текст: стиль~--- <<Обычный>>: шрифт Times New
Roman, размер 14~пунктов, абзацный отступ~--- 0,5~см, 1,5 интервала,
выравнивание~--- по ширине. Рекомендуемый объем рукописи~--- не свыше
25~страниц указанного формата. Ознакомиться с шаблонами, содержащими примеры
оформления, можно по адресу в Интернете:
\textsf{http://www.ipiran.ru/journal/template.doc}.
\item К рукописи, предоставляемой в 2-х
экземплярах, обязательно прилагается электронная версия статьи (как правило, в
форматах MS WORD (.doc) или \LaTeX\ (.tex), а также~--- дополнительно~--- в
формате .pdf) на дискете, лазерном диске или по электронной почте. Сокращения
слов, кроме стандартных, не применяются. Все страницы рукописи должны быть
пронумерованы. %\\[-13.5pt]
\item Статья должна содержать следующую информацию на русском и
английском языках: название, Ф.И.О. авторов, места работы авторов и их
электронные адреса, подробные сведения об авторах, оформленные в соответствии с форматом, 
определяемым файлами {\sf http://www.ipiran.ru/journal/issues/2011\_05\_01/authors.asp} и 
{\sf http://www.ipiran.ru/journal/issues/2011\_01\_eng/authors.asp},
аннотация (не более 100~слов), ключевые слова. Ссылки на
литературу в тексте статьи нумеруются (в квадратных скобках) и располагаются в
порядке их первого упоминания. В~списке литературы не должно быть позиций, на которые нет ссылки в тексте статьи.
Все фамилии авторов, заглавия статей, названия
книг, конференций и~т.\,п.\ даются на языке оригинала, если этот язык
использует кириллический или латинский алфавит. %\\[-13.5pt]
\item Присланные в редакцию материалы авторам не возвращаются.
\item При отправке файлов по электронной
почте просим придерживаться следующих правил:
\begin{itemize}
\item указывать в поле subject (тема) название журнала и фамилию автора; %\\[-13.5pt]
\item использовать attach (присоединение); %\\[-13.5pt]
\item в случае больших объемов информации возможно
использование общеизвестных архиваторов (ZIP, RAR); %\\[-13.5pt]
\item в состав электронной версии статьи должны входить: файл, содержащий текст статьи, и файл(ы),
содержащий(е) иллюстрации. %\\[-13.5pt]
\end{itemize}
\item Журнал <<Информатика и её применения>> является некоммерческим изданием. 
Плата за публикацию с авторов не взимается, гонорар авторам не выплачивается.
\end{enumerate}
\thispagestyle{empty}
\textbf{Адрес редакции:} Москва 119333,
ул.~Вавилова, д.~44, корп.~2, ИПИ РАН\\
\hphantom{\textbf{Адрес редакции:} }Тел.: +7 (499) 135-86-92\ \
Факс:  +7 (495) 930-45-05\ \  E-mail:   rust@ipiran.ru }
}

\end{document}


%\tableofcontents

%\end{document}





%\def\stat{cont}
{%\hrule\par
%\vskip 7pt % 7pt
\raggedleft\Large \bf%\baselineskip=3.2ex
А\,В\,Т\,О\,Р\,С\,К\,И\,Й\ \ У\,К\,А\,З\,А\,Т\,Е\,Л\,Ь\ \ З\,А\ \ 2\,0\,0\,7 г. \vskip 17pt
    \hrule
    \par
\vskip 21pt plus 6pt minus 3pt }

\label{st\stat}

\def\tit{\ }

\def\aut{\ }
\def\auf{\ }

\def\leftkol{\ } % ENGLISH ABSTRACTS}

\def\rightkol{\ } %ENGLISH ABSTRACTS}

\titele{\tit}{\aut}{\auf}{\leftkol}{\rightkol}


\contentsline {chapter}{\ }{Выпуск \quad Стр.} 
\contentsline {section}{\textbf{Батракова Д.\,А., Королев В.\,Ю., Шоргин С.\,Я.}\ \ Новый метод вероятностно-ста\-ти\-сти\-че\-ско\-го анализа информационных потоков в\nobreakspace {}телекоммуникационных сетях}{\qquad 1 \qquad 40} 
\contentsline {section}{\textbf{Борисов А.\,В.}\ \ Байесовское оценивание в системах наблюдения с\nobreakspace {}марковскими скачкообразными процессами: игровой подход}{\qquad 2 \qquad 65}
\contentsline {section}{\textbf{Босов А.\,В., Иванов А.\,В.}\ \ Программная инфраструктура информационного Web-пор\-тала}{\qquad 2 \qquad 50}
\contentsline {section}{\textbf{Захаров В.\,Н., Калиниченко Л.\,А., Соколов И.\,А., Ступников С.\,А.}\ \ Конструирование канонических информационных моделей для интегрированных информационных систем}{\qquad 2 \qquad 15}
\contentsline {section}{\textbf{Захаров В.\,Н., Козмидиади В.\,А.}\ \ Средства обеспечения отказоустойчивости при\-ло\-жений}{\qquad 1 \qquad 14} 
\contentsline {section}{\textbf{Иванов А.\,В.}\ \ см. Босов А.\,В.\hfill\hfill\hfill\hfill\hfill\hfill\hfill\hfill\hfill\hfill\hfill\hfill\hfill\hfill\hfill\hfill\hfill\hfill\hfill\hfill\hfill\hfill\hfill\hfill\hfill\hfill\hfill\hfill\hfill\hfill\hfill\hfill\hfill\hfill\hfill}{\ }
\contentsline {section}{\textbf{Ильин В.\,Д., Соколов И.\,А.}\ \ Символьная модель системы знаний информатики в\nobreakspace {}че\-ло\-ве\-ко-автоматной среде}{\qquad 1 \qquad 66} 
\contentsline {section}{\textbf{Калиниченко Л.\,А.}\ \ см. Захаров В.\,Н.\hfill\hfill\hfill\hfill\hfill\hfill\hfill\hfill\hfill\hfill\hfill\hfill\hfill\hfill\hfill\hfill\hfill\hfill\hfill\hfill\hfill\hfill\hfill\hfill\hfill\hfill\hfill\hfill\hfill\hfill\hfill\hfill\hfill\hfill\hfill}{\ }
\contentsline {section}{\textbf{Козеренко Е.\,Б.}\ \ Лингвистическое моделирование для систем машинного перевода и обработки знаний}{\qquad 1 \qquad 54} 
\contentsline {section}{\textbf{Козмидиади В.\,А.}\ \ см. Захаров В.\,Н.\hfill\hfill\hfill\hfill\hfill\hfill\hfill\hfill\hfill\hfill\hfill\hfill\hfill\hfill\hfill\hfill\hfill\hfill\hfill\hfill\hfill\hfill\hfill\hfill\hfill\hfill\hfill\hfill\hfill\hfill\hfill\hfill\hfill\hfill\hfill }{\ } 
\contentsline {section}{\textbf{Королев В.\,Ю.}\ \ см. Батракова Д.\,А.\hfill\hfill\hfill\hfill\hfill\hfill\hfill\hfill\hfill\hfill\hfill\hfill\hfill\hfill\hfill\hfill\hfill\hfill\hfill\hfill\hfill\hfill\hfill\hfill\hfill\hfill\hfill\hfill\hfill\hfill\hfill\hfill\hfill\hfill\hfill}{\ } 
\contentsline {section}{\textbf{Кудрявцев А.\,А., Шоргин С.\,Я.}\ \ Байесовский подход к\nobreakspace {}анализу систем массового обслуживания и\nobreakspace {}показателей надежности}{\qquad 2 \qquad 76}
\contentsline {section}{\textbf{Печинкин А.\,В., Соколов И.\,А., Чаплыгин В.\,В.}\ \ Многолинейная система массового обслуживания с конечным накопителем и ненадежными приборами}{\qquad 1 \qquad 27} 
\contentsline {section}{\textbf{Печинкин А.\,В., Соколов И.\,А., Чаплыгин В.\,В.}\ \ Стационарные характеристики многолинейной\nobreakspace {}системы массового обслуживания с\nobreakspace {}одновременными отказами приборов}{\qquad 2 \qquad 39}
\contentsline {section}{\textbf{Синицын И.\,Н.}\ \ Корреляционные методы построения аналитических информационных моделей флуктуаций полюса Земли по априорным данным}{\qquad 2 \qquad \hphantom{9}2}
\contentsline {section}{\textbf{Синицын И.\,Н.}\ \ Развитие теории фильтров Пугачева для оперативной обработки информации в стохастических системах}{{\qquad 1 \qquad \hphantom{9}3}} 
\contentsline {section}{\textbf{Соколов И.\,А.}\ \ см. Захаров В.\,Н.\hfill\hfill\hfill\hfill\hfill\hfill\hfill\hfill\hfill\hfill\hfill\hfill\hfill\hfill\hfill\hfill\hfill\hfill\hfill\hfill\hfill\hfill\hfill\hfill\hfill\hfill\hfill\hfill\hfill\hfill\hfill\hfill\hfill\hfill\hfill}{\ }
\contentsline {section}{\textbf{Соколов И.\,А.}\ \ см. Ильин В.\,Д.\hfill\hfill\hfill\hfill\hfill\hfill\hfill\hfill\hfill\hfill\hfill\hfill\hfill\hfill\hfill\hfill\hfill\hfill\hfill\hfill\hfill\hfill\hfill\hfill\hfill\hfill\hfill\hfill\hfill\hfill\hfill\hfill\hfill\hfill\hfill}{\ } 
\contentsline {section}{\textbf{Соколов И.\,А.}\ \ см. Печинкин А.\,В.\hfill\hfill\hfill\hfill\hfill\hfill\hfill\hfill\hfill\hfill\hfill\hfill\hfill\hfill\hfill\hfill\hfill\hfill\hfill\hfill\hfill\hfill\hfill\hfill\hfill\hfill\hfill\hfill\hfill\hfill\hfill\hfill\hfill\hfill\hfill}{\ } 
\contentsline {section}{\textbf{Соколов И.\,А.}\ \ см. Печинкин А.\,В.\hfill\hfill\hfill\hfill\hfill\hfill\hfill\hfill\hfill\hfill\hfill\hfill\hfill\hfill\hfill\hfill\hfill\hfill\hfill\hfill\hfill\hfill\hfill\hfill\hfill\hfill\hfill\hfill\hfill\hfill\hfill\hfill\hfill\hfill\hfill}{\ }
\contentsline {section}{\textbf{Ступников С.\,А.}\ \ см. Захаров В.\,Н.\hfill\hfill\hfill\hfill\hfill\hfill\hfill\hfill\hfill\hfill\hfill\hfill\hfill\hfill\hfill\hfill\hfill\hfill\hfill\hfill\hfill\hfill\hfill\hfill\hfill\hfill\hfill\hfill\hfill\hfill\hfill\hfill\hfill\hfill\hfill}{\ }
\contentsline {section}{\textbf{Чаплыгин В.\,В.}\ \ см. Печинкин А.\,В.\hfill\hfill\hfill\hfill\hfill\hfill\hfill\hfill\hfill\hfill\hfill\hfill\hfill\hfill\hfill\hfill\hfill\hfill\hfill\hfill\hfill\hfill\hfill\hfill\hfill\hfill\hfill\hfill\hfill\hfill\hfill\hfill\hfill\hfill\hfill}{\ } 
\contentsline {section}{\textbf{Чаплыгин В.\,В.}\ \ см. Печинкин А.\,В.\hfill\hfill\hfill\hfill\hfill\hfill\hfill\hfill\hfill\hfill\hfill\hfill\hfill\hfill\hfill\hfill\hfill\hfill\hfill\hfill\hfill\hfill\hfill\hfill\hfill\hfill\hfill\hfill\hfill\hfill\hfill\hfill\hfill\hfill\hfill}{\ }
\contentsline {section}{\textbf{Шоргин С.\,Я.}\ \ см. Батракова Д.\,А.\hfill\hfill\hfill\hfill\hfill\hfill\hfill\hfill\hfill\hfill\hfill\hfill\hfill\hfill\hfill\hfill\hfill\hfill\hfill\hfill\hfill\hfill\hfill\hfill\hfill\hfill\hfill\hfill\hfill\hfill\hfill\hfill\hfill\hfill\hfill}{\ } 
\contentsline {section}{\textbf{Шоргин С.\,Я.}\ \ см. Кудрявцев А.\,А.\hfill\hfill\hfill\hfill\hfill\hfill\hfill\hfill\hfill\hfill\hfill\hfill\hfill\hfill\hfill\hfill\hfill\hfill\hfill\hfill\hfill\hfill\hfill\hfill\hfill\hfill\hfill\hfill\hfill\hfill\hfill\hfill\hfill\hfill\hfill}{\ }
%\thispagestyle{myheadings}
\def\leftfootline{\small{\textbf{\thepage}
\hfill ИНФОРМАТИКА И ЕЁ ПРИМЕНЕНИЯ\ \ \ том~1\ \ \ выпуск~2\ \ \ 2007}
}%
 \def\rightfootline{\small{ИНФОРМАТИКА И ЕЁ ПРИМЕНЕНИЯ\ \ \ том~1\ \ \ выпуск~2\ \ \ 2007
 \hfill \textbf{\thepage}}}
 \label{end\stat}

%\def\stat{cont-e}
{%\hrule\par
%\vskip 7pt % 7pt
\raggedleft\Large \bf%\baselineskip=3.2ex
2\,0\,0\,7\ \ A\,U\,T\,H\,O\,R\ \ I\,N\,D\,E\,X \vskip 17pt
    \hrule
    \par
\vskip 21pt plus 6pt minus 3pt }

\label{st\stat}

\def\tit{\ }

\def\aut{\ }
\def\auf{\ }

\def\leftkol{\ } % ENGLISH ABSTRACTS}

\def\rightkol{\ } %ENGLISH ABSTRACTS}

\titele{\tit}{\aut}{\auf}{\leftkol}{\rightkol}


\contentsline {chapter}{\ }{Issue \quad Page} 
\contentsline {subsection}{\textbf{Batrakova D.\,A., Korolev V.\,Yu., Shorgin S.\,Ya.}\ \ A New Method for the Probabilistic and Statistical Analysis of Information Flows in Telecommunication Networks}{\qquad 1 \qquad 40} 
\contentsline {subsection}{\textbf{Borisov A.\,V.}\ \ Bayesian Estimation in\nobreakspace {}Observation Systems with\nobreakspace {}Markov Jump Processes: Game-Theoretic Approach}{\qquad 2 \qquad 65} 
\contentsline {subsection}{\textbf{Bosov A.\,V., Ivanov A.\,V.}\ \ Linguistic Simulation for Machine Translation and Knowledge Management Systems}{\qquad 2 \qquad 50} 
\contentsline {subsection}{\textbf{Chaplygin V.\,V.} see Pechinkin A.\,V.\hfill\hfill\hfill\hfill\hfill\hfill\hfill\hfill\hfill\hfill\hfill\hfill\hfill\hfill\hfill\hfill\hfill\hfill\hfill\hfill\hfill\hfill\hfill\hfill\hfill\hfill\hfill\hfill\hfill\hfill\hfill\hfill\hfill\hfill\hfill}{\ }
\contentsline {subsection}{\textbf{Chaplygin V.\,V.} see Pechinkin A.\,V.\hfill\hfill\hfill\hfill\hfill\hfill\hfill\hfill\hfill\hfill\hfill\hfill\hfill\hfill\hfill\hfill\hfill\hfill\hfill\hfill\hfill\hfill\hfill\hfill\hfill\hfill\hfill\hfill\hfill\hfill\hfill\hfill\hfill\hfill\hfill}{\ }
\contentsline {subsection}{\textbf{Ilyin V.\,D., Sokolov I.\,A.}\ \ The Symbol Model of Informatics Knowledge System in Human-Automaton Environment}{\qquad 1 \qquad 66} 
\contentsline {subsection}{\textbf{Ivanov A.\,V.} see Bosov A.\,V.\hfill\hfill\hfill\hfill\hfill\hfill\hfill\hfill\hfill\hfill\hfill\hfill\hfill\hfill\hfill\hfill\hfill\hfill\hfill\hfill\hfill\hfill\hfill\hfill\hfill\hfill\hfill\hfill\hfill\hfill\hfill\hfill\hfill\hfill\hfill}{\ }
\contentsline {subsection}{\textbf{Kalinichenko L.\,A.} see Zakharov V.\,N.\hfill\hfill\hfill\hfill\hfill\hfill\hfill\hfill\hfill\hfill\hfill\hfill\hfill\hfill\hfill\hfill\hfill\hfill\hfill\hfill\hfill\hfill\hfill\hfill\hfill\hfill\hfill\hfill\hfill\hfill\hfill\hfill\hfill\hfill\hfill}{\ }
\contentsline {subsection}{\textbf{Korolev V.\,Yu.} see Batrakova D.\,A.\hfill\hfill\hfill\hfill\hfill\hfill\hfill\hfill\hfill\hfill\hfill\hfill\hfill\hfill\hfill\hfill\hfill\hfill\hfill\hfill\hfill\hfill\hfill\hfill\hfill\hfill\hfill\hfill\hfill\hfill\hfill\hfill\hfill\hfill\hfill}{\ }
\contentsline {subsection}{\textbf{Kozerenko E.\,B.}\ \ Linguistic Simulation for Machine Translation and Knowledge Management Systems}{\qquad 1 \qquad 54} 
\contentsline {subsection}{\textbf{Kozmidiady V.\,A.} see Zakharov V.\,N.\hfill\hfill\hfill\hfill\hfill\hfill\hfill\hfill\hfill\hfill\hfill\hfill\hfill\hfill\hfill\hfill\hfill\hfill\hfill\hfill\hfill\hfill\hfill\hfill\hfill\hfill\hfill\hfill\hfill\hfill\hfill\hfill\hfill\hfill\hfill}{\ }
\contentsline {subsection}{\textbf{Kudryavtsev A.\,A., Shorgin S.\,Ya.}\ \ Bayesian Approach to Queueing Systems and Reliability Characteristics}{\qquad 2 \qquad 76} 
\contentsline {subsection}{\textbf{Pechinkin A.\,V., Sokolov I.\,A., Chaplygin V.\,V.}\ \ Multichannel Queuing System with Finite Buffer and Unreliable Servers}{\qquad 1 \qquad 27} 
\contentsline {subsection}{\textbf{Pechinkin A.\,V., Sokolov I.\,A., Chaplygin V.\,V.}\ \ Stationary Characteristics of a Multichannel Queueing System with\nobreakspace {}Simultaneous Refusals of Servers}{\qquad 2 \qquad 39} 
\contentsline {subsection}{\textbf{Shorgin S.\,Ya.} see Batrakova D.\,A.\hfill\hfill\hfill\hfill\hfill\hfill\hfill\hfill\hfill\hfill\hfill\hfill\hfill\hfill\hfill\hfill\hfill\hfill\hfill\hfill\hfill\hfill\hfill\hfill\hfill\hfill\hfill\hfill\hfill\hfill\hfill\hfill\hfill\hfill\hfill}{\ }
\contentsline {subsection}{\textbf{Shorgin S.\,Ya.} see Kudryavtsev A.\,A.\hfill\hfill\hfill\hfill\hfill\hfill\hfill\hfill\hfill\hfill\hfill\hfill\hfill\hfill\hfill\hfill\hfill\hfill\hfill\hfill\hfill\hfill\hfill\hfill\hfill\hfill\hfill\hfill\hfill\hfill\hfill\hfill\hfill\hfill\hfill}{\ }
\contentsline {subsection}{\textbf{Sinitsyn I.\,N.}\ \ Correlational Methods for Analytical Informational Models of the Earth Pole Fluctuations Design Based on a priori Data}{\qquad 2 \qquad \hphantom{9}2}
\contentsline {subsection}{\textbf{Sinitsyn I.\,N.}\ \ Development of Pugachev Filtering for Stochastic Systems}{\qquad 1 \qquad \hphantom{9}3}
\contentsline {subsection}{\textbf{Sokolov I.\,A.} see Ilyin V.\,D.\hfill\hfill\hfill\hfill\hfill\hfill\hfill\hfill\hfill\hfill\hfill\hfill\hfill\hfill\hfill\hfill\hfill\hfill\hfill\hfill\hfill\hfill\hfill\hfill\hfill\hfill\hfill\hfill\hfill\hfill\hfill\hfill\hfill\hfill\hfill}{\ }
\contentsline {subsection}{\textbf{Sokolov I.\,A.} see Pechinkin A.\,V.\hfill\hfill\hfill\hfill\hfill\hfill\hfill\hfill\hfill\hfill\hfill\hfill\hfill\hfill\hfill\hfill\hfill\hfill\hfill\hfill\hfill\hfill\hfill\hfill\hfill\hfill\hfill\hfill\hfill\hfill\hfill\hfill\hfill\hfill\hfill}{\ }
\contentsline {subsection}{\textbf{Sokolov I.\,A.} see Pechinkin A.\,V.\hfill\hfill\hfill\hfill\hfill\hfill\hfill\hfill\hfill\hfill\hfill\hfill\hfill\hfill\hfill\hfill\hfill\hfill\hfill\hfill\hfill\hfill\hfill\hfill\hfill\hfill\hfill\hfill\hfill\hfill\hfill\hfill\hfill\hfill\hfill}{\ }
\contentsline {subsection}{\textbf{Sokolov I.\,A.} see Zakharov V.\,N.\hfill\hfill\hfill\hfill\hfill\hfill\hfill\hfill\hfill\hfill\hfill\hfill\hfill\hfill\hfill\hfill\hfill\hfill\hfill\hfill\hfill\hfill\hfill\hfill\hfill\hfill\hfill\hfill\hfill\hfill\hfill\hfill\hfill\hfill\hfill}{\ }
\contentsline {subsection}{\textbf{Stupnikov S.\,A.} see Zakharov V.\,N.\hfill\hfill\hfill\hfill\hfill\hfill\hfill\hfill\hfill\hfill\hfill\hfill\hfill\hfill\hfill\hfill\hfill\hfill\hfill\hfill\hfill\hfill\hfill\hfill\hfill\hfill\hfill\hfill\hfill\hfill\hfill\hfill\hfill\hfill\hfill}{\ }
\contentsline {subsection}{\textbf{Zakharov V.\,N., Kalinichenko L.\,A., Sokolov I.\,A., Stupnikov S.\,A.}\ \ Development of Canonical Information Models for Integrated Information Systems}{\qquad 2 \qquad 15} 
\contentsline {subsection}{\textbf{Zakharov V.\,N., Kozmidiady V.\,A.}\ \ Means Providing Applications Fault Tolerance}{\qquad 1 \qquad 14} 
\def\leftfootline{\small{\textbf{\thepage}
\hfill ИНФОРМАТИКА И ЕЁ ПРИМЕНЕНИЯ\ \ \ том~1\ \ \ выпуск~2\ \ \ 2007}
}%
 \def\rightfootline{\small{ИНФОРМАТИКА И ЕЁ ПРИМЕНЕНИЯ\ \ \ том~1\ \ \ выпуск~2\ \ \ 2007
 \hfill \textbf{\thepage}}}
 \label{end\stat}


%\tableofcontents


\end{document}