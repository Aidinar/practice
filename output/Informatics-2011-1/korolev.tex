\def\stat{korolev}

\def\tit{УСТОЙЧИВОСТЬ КОНЕЧНЫХ СМЕСЕЙ ОБОБЩЕННЫХ ГАММА-РАСПРЕДЕЛЕНИЙ
ОТНОСИТЕЛЬНО ВОЗМУЩЕНИЙ ПАРАМЕТРОВ$^*$}

\def\titkol{Устойчивость конечных смесей обобщенных гамма-распределений
относительно возмущений параметров}

\def\autkol{В.\,Ю.~Королев, В.\,А.~Крылов, В.\,Ю.~Кузьмин}
\def\aut{В.\,Ю.~Королев$^1$, В.\,А.~Крылов$^2$, В.\,Ю.~Кузьмин$^3$}

\titel{\tit}{\aut}{\autkol}{\titkol}

{\renewcommand{\thefootnote}{\fnsymbol{footnote}}\footnotetext[1]
{Работа поддержана
Российским фондом фундаментальных исследований (проекты
08-01-00563, 08-01-00567, 08-07-00152, 09-07-12032-офи-м,
11-07-00112 и 11-01-00515), а также Министерством образования и
науки РФ в рамках ФЦП <<Научные и научно-педагогические кадры
инновационной России на 2009--2013~гг>>.}}

\renewcommand{\thefootnote}{\arabic{footnote}}
\footnotetext[1]{Московский государственный университет им.\ М.\,В.~Ломоносова,
факультет вычислительной математики и кибернетики; Институт проблем
информатики Российской академии наук, vkorolev@comtv.ru}
\footnotetext[2]{Московский государственный университет им.\ М.\,В.~Ломоносова, факультет вычислительной математики и кибернетики,
vkrylov@cs.msu.ru}
\footnotetext[2]{Московский
государственный университет им.\ М.\,В.~Ломоносова, факультет
вычислительной математики и кибернетики, silencershade@gmail.com}

\vspace*{-6pt}

\Abst{На примере модели <<контаминации>> в терминах
равномерного расстояния и метрики Леви получены оценки
устойчивости конечных смесей обобщенных гам\-ма-рас\-пре\-де\-ле\-ний по
отношению к малым возмущениям параметров.}

\vspace*{-6pt}

\KW{обобщенное гамма-распределение; конечная
смесь; мет\-ри\-ка Леви}

      \vskip 8pt plus 9pt minus 6pt

      \thispagestyle{headings}

      \begin{multicols}{2}
      
            \label{st\stat}

\section{Введение}

В работе~\cite{BKSh} был предложен новый метод анализа поведения
хаотических информационных потоков в сложных телекоммуникационных
сетях, основанный на стохастической модели телекоммуникационной
сети, имеющей вид суперпозиции некоторых простых структур.  
В~качестве итоговой модели распределения времени выполнения
(обработки) запроса сетью в~\cite{BKSh} рассматривается смесь
гам\-ма-рас\-пре\-де\-ле\-ний или смесь распределений Вейбулла. Полученные в
указанной работе результаты могут быть достаточно просто
интерпретированы. Число компонент смеси символизирует число
типичных параллельных или последовательных структур. Значения
параметров составляющих смесь гам\-ма-рас\-пре\-де\-ле\-ний показывают
<<степень параллелизма>> соответствующей структуры: чем ближе
параметр формы к единице, тем выше эта <<степень>>. И~наоборот,
чем дальше значение параметра формы от единицы, тем больше
последовательных операций выполняется в соответствующем блоке.
Веса компонент характеризуют примерную долю использования ресурсов
для сообщений, соответствующих каждому распределению входных
данных. Предложенный метод позволяет анализировать информационные
потоки в инфотелекоммуникационных системах и, в част\-ности,
выявлять (статистически оценивать) стохастическую структуру
сложных прог\-рам\-мно-вы\-чис\-ли\-тель\-ных и телекоммуникационных сетей
даже в том случае, когда их реальная топология неизвестна. Главную
роль в указанном методе анализа стохастической структуры
информационных потоков играет зада-\linebreak ча динамического статистического
разделения\linebreak смеси, т.\,е.\ задача статистического оценивания
па\-ра\-мет\-ров смеси. Для решения этой задачи в~\cite{BKSh}, следуя
традиции, использован ЕМ-ал\-го\-ритм. Однако примене\-ние предложенного
метода к реальным статистическим данным выявило его заметную
неустойчивость по исходным данным. В~работе~\cite{KN} предложен
альтернативный <<сеточный>> метод решения задачи разделения смесей
вероятностных распределений, основанный на подмене исходной смеси
смесью с намного б$\acute{\mbox{о}}$льшим числом компонент с {\it известными}
параметрами, в которой неизвестными параметрами являются только
веса компонент. При этом для обоснования возможности подмены
исходной смеси новой необходимо, чтобы смешиваемые распределения
обладали свойством устойчивости к малым возмущениям параметров.

В данной заметке будет исследована устойчивость конечных смесей
обобщенных гамма-рас\-преде\-ле\-ний (ОГ-рас\-пре\-де\-ле\-ний)
относительно возмущений параметров. ОГ-рас\-пре\-де\-ле\-ния были\linebreak
введены в 1962~г.\ в работе~\cite{St} в качестве семейства
вероятностных моделей, включающего в себя одновременно
гам\-ма-рас\-пре\-де\-ле\-ния и распределения Вейбулла. Напомним выражение
для функции ОГ-рас\-пре\-де\-ле\-ния:

\vspace*{-6pt}

\noindent
\begin{equation}
{\mathcal G}_{\beta,\alpha,\sigma}(x) =
\fr{\gamma\left(\alpha,\left[{x}/{\sigma}\right]^{\beta}\right)}{\Gamma(\alpha)}\,,
\quad x\geqslant 0\,, \label{GGDG}
\end{equation}
%\pagebreak

\noindent
с параметрами $\beta,\alpha,\sigma\in{\mathbb R}^+$, где
$\Gamma(\alpha)$~--- эйлерова гамма-функция:
$$
\Gamma(\alpha)=\int\limits_0^{\infty}x^{\alpha-1}e^{-x}\,dx\,;
$$
$\gamma(\alpha,x)$~--- (нижняя) неполная гам\-ма-функ\-ция:
$$
\gamma(\alpha,x)=\int\limits_0^xz^{\alpha-1}e^{-z}\,dz\,.
$$
 Семейство ОГ-рас\-пре\-де\-ле\-ний
включает в себя гам\-ма-\-рас\-пре\-де\-ле\-ние и его частные случаи~---
по-\linebreak ка\-за\-тель\-ное распределение, распределения Эр-\linebreak ланга и хи-\-квад\-рат,~--- 
рас\-пре\-де\-ле\-ние Накагами,\linebreak рас\-пре\-деле\-ние максимума стандартного
винеровского процесса на интервале $[0,1]$ (распределение модуля
стандартной нормальной случайной величины или <<полунормальное>>
распределение), обратное гам\-ма-\-рас\-пре\-де\-ле\-ние, хи-рас\-пре\-де\-ле-\linebreak ние,
распределения Рэлея, Максвелла, Вейбулла и\linebreak Леви. Предельным
случаем обобщенного гам\-ма-\linebreak рас\-пре\-де\-ле\-ния при $\alpha\to\infty$
является логнормальное распределение. Как уже было упомянуто выше,
ОГ-рас\-пре\-де\-ле\-ния находят применение в задаче анализа
стохастической структуры информационных потоков. Это семейство
также широко используется в прикладных задачах, связанных с
вычислением надежностных характеристик~\cite{7, 25},
прогнозированием продолжительности лечения и затратами на
медицинское обслуживание~\cite{25, 6}, расчетами инженерных рисков
и рисков катастроф (землетрясений и наводнений)~\cite{26},
обработкой изображений~\cite{28, 29} и дистанционным зондированием~\cite{30, 31}. 
ОГ-рас\-пре\-де\-ле\-ния также используются в качестве
моделей распределения доходов~\cite{8}.

\section{Постановка задачи}

В данной заметке устойчивость конечных смесей ОГ-рас\-пре\-де\-ле\-ний по
отношению к малым возму\-ще\-ни\-ям параметров будет рассмотрена на
примере простейшей модели загрязнения (<<контаминации>>),
предложенной Дж.~Тьюки~\cite{Tukey1960} для нормальных
распределений (см.\ также~\cite{K2006}). Согласно этой модели
вместо <<чистого>> ОГ-рас\-пре\-де\-ле\-ния, будем рассматривать смеси
ОГ-рас\-пре\-де\-ле\-ний. Так, контаминационная модель по параметру~$\beta$ имеет следующий вид:
\begin{equation}
F_{p,\beta_1,\beta_2} = p {\mathcal G}_{\beta_1,1,1}(x) +
(1-p){\mathcal G}_{\beta_2,1,1}(x)\,, \label{Fmix}
\end{equation}
где $p\in(0,1]$ и $0 < \beta_1,\beta_2 \leqslant M_{\beta}$ для
некоторой положительной константы~$M_{\beta}$. Предполагается, что\linebreak
${\mathcal G}_{\beta_1,1,1}(x)$~--- <<истинное>> распределение
наблюдений. Если выборка представляет собой набор независимых
реализаций случайной величины с функцией распределения
$F_{p,\beta_1,\beta_2}$, то компонента\linebreak смеси ${\mathcal
G}_{\beta_2,1,1}(x)$ соответствует <<загрязняющим>> (или
нетипичным) наблюдениям. Задачу устойчивости сформулируем в
следующем виде: установить зависимость между <<близостью>> функций
распределения $F_{p,\beta_1,\beta_2}$ и ${\mathcal
G}_{\beta_1,1,1}(x)$ и близостью значений па\-ра\-мет\-ров $\beta_1$ и
$\beta_2$ (близость па\-ра\-мет\-ра~$p$ к единице). Аналогично
рассмотрим модели контаминации по параметрам~$\alpha$ и~$\sigma$.
Подобная задача была исследована для масштабных и сдвиговых смесей
нормальных распределений в~\cite{Korolev2007s}. Заметим, что
ограничение, предусмотренное моделью~(\ref{Fmix}), а именно
фиксация значений $\alpha=1$ и $\sigma=1$, введено с целью
упрощения формулировок последующих утверждений. Аналогичные
утверждения могут быть получены и в более общем предположении
$\alpha=\alpha_0$ и $\sigma=\sigma_0$.

Как отмечено в~\cite{Korolev2007s}, для оценки <<близости>>
функций распределения в данной задаче целесообразным
представляется использовать мет\-ри\-ки, характеризующие слабую
сходимость, т.\,е., например, мет\-ри\-ку Леви или сглаженную
равномерную мет\-ри\-ку. В~данном случае будем использовать
равномерную мет\-ри\-ку~$\rho$, определяемую как
$$
\rho(X,Y) \equiv \rho(F,G) = \sup\limits_{x}|F(x)-G(x)|
$$
для двух случайных величин~$X$ и~$Y$, имеющих функции
распределения~$F(x)$ и~$G(x)$ соответственно. На основе
результатов для равномерной метрики будут также получены
результаты в терминах метрики Леви, которая определяется следующим
образом:
\begin{multline*}
L(F, G) = \inf\{r: G(x-r)-r \leqslant F(x) \leqslant{}\\
{}\leqslant G(x+r)+r\,, \ \mbox{для всех}\ x\in{\mathbb R}\}\,.
\end{multline*}
Расстояние Леви допускает наглядную геометрическую интерпретацию:
оно равно длине стороны наибольшего квадрата, который можно
вписать между графиками функций распределения $F(x)$ и~$G(x)$ так,
чтобы стороны его были параллельны координатным осям.

\section{Прямая задача устойчивости}

Пусть $X_i$ и $Y_i$~--- независимые случайные величины (с.в.) и
пусть, более того, $P(Y_i\geqslant0)=1,\; i=1,2$. Очевидно, что
если рассматривать дискретные с.в.\ $Y_i$, то $X_iY_i$ будут
представлять собой конечные масштабные смеси. {\it Прямая} задача
устойчивости состоит в оценке величины
$$
\rho(X_1Y_1, X_2Y_2)
$$
через $\rho(X_1,X_2)$ и $\rho(Y_1,Y_2)$. Следующее утверждение
дает решение этой задачи.

\medskip

\noindent
\textbf{Утверждение (\cite{Korolev2007s}, следствие 3.3.6)}.
\textit{Пусть $X_i$ и $Y_i$ независимые с.в., причем $P(Y_i>0)=1,\;
i=1,2$. Тогда}
$$
\rho(X_1Y_1, X_2Y_2) \leqslant \rho(X_1,X_2) + \rho(Y_1,Y_2)\,.
$$

\smallskip

Данное утверждение дает простую и наглядную оценку близости
масштабных смесей в терминах расстояний между соответствующими
смешиваемыми и смешивающими распределениями.

Основная идея регуляризации общей некорректно поставленной задачи
разделения смеси (на практике не известной точно) законов
распределения основана на замене <<истинного>> смешивающего закона
его конечной дискретной аппроксимацией. Рассмотрим некоторые
вопросы, связанные с оцениванием потери точности, сопутствующей
такой подмене, в случае масштабных смесей ОГ-рас\-пре\-де\-лений.

Из приведенного утверждения вытекает, что для любых
неотрицательных с.в.\ $Y_1$ и~$Y_2$ справедлива оценка
\begin{multline*}
\sup\limits_{x>0} \left\vert {\mathbb E}{\mathcal
G}_{\beta,\alpha,\sigma}(x\sqrt{Y_1})-{\mathbb E}{\mathcal
G}_{\beta,\alpha,\sigma}(x\sqrt{Y_2}) \right\vert \leqslant {}\\
{}\leqslant
\sup\limits_{y>0} \left\vert P(Y_1<y) - P(Y_2<y) \right\vert\,.
\end{multline*}

Таким образом, если известна оценка равномерного расстояния между
<<истинным>> смешивающим распределением и его аппроксимацией, то
равномерное расстояние между <<истинной>> и приближенной смесями
не превосходит этой оценки.

Для масштабного параметра $\sigma$ ОГ-рас\-пре\-де\-ле\-ния можно получить
неравенство для решения прямой задачи устойчивости в терминах
метрики Леви.

Заметим сперва, что масштабный параметр $\sigma$ обладает
следующим очевидным свойством:
$$
{\mathcal G}_{\beta,\alpha,\sigma}(x) = {\mathcal
G}_{\beta,\alpha,1}(\sigma^{-1} x)\,.
$$
Поэтому, если обозначить через $U_{\sigma_1,\sigma_2}$ с.в.,
принимающую значения~1 и $\sigma_1/\sigma_2$ с вероятностями~$p$ и
$(1-p)$ соответственно, то с.в.\
$$\hat F_{\sigma_1,\sigma_2} = p {\mathcal G}_{\beta,\alpha,\sigma_1}(x) + 
(1-p){\mathcal G}_{\beta,\alpha,\sigma_2}(x)$$
распределена как произведение независимых c.в.\
$U_{\sigma_1,\sigma_2}$ и с. в. с ОГ-рас\-пре\-де\-ле\-ни\-ем с параметрами
$\{\beta,\alpha,\sigma_1\}$.

\medskip

\noindent
\textbf{Утверждение 1.}
\textit{Пусть $p\in(0,1]$, $\beta,\alpha,\sigma_1,\sigma_2 \in$\linebreak $\in
(0,+\infty)$. Тогда}
\begin{multline}
L(\hat F_{\sigma_1,\sigma_2}, {\mathcal G}_{\beta,\alpha,\sigma_1}
) \leqslant{}\\
{}\leqslant
2\left(1+\fr{\beta\alpha^{\alpha}\exp[-\alpha]}{\Gamma(\alpha)}\right)
L( U_{\sigma_1,\sigma_2},1)\,. 
\label{stab7a}
\end{multline}


\smallskip


\noindent
Д\,о\,к\,а\,з\,а\,т\,е\,л\,ь\,с\,т\,в\,о\,.\
Для доказательства воспользуемся доказанным в~\cite{Hall1979}
(теорема~4) утверждением: для любой с.в.~$X$ с плотностью~$p(x)$
и неотрицательной с.в.~$W$, удовлетворяющей ${\mathbb
E}W<\infty$, имеет место
$$
\rho (XW,X) \leqslant 2\left(1+\max\limits_{x>0} [xp(x)]\right) L(W,1)\,.
$$
Следовательно, так как очевидно, что $L(X,Y) \leqslant$\linebreak $\leqslant \rho (X,Y)$, имеем
\begin{multline}
L({\mathcal G}_{\beta,\alpha,\sigma_1}(x),{\mathbb E}{\mathcal
G}_{\beta,\alpha,\sigma_1}(xU)) \leqslant{}\\
{}\leqslant
2\left(1+\max\limits_{x>0} [xp(x)]\right) L(W,1)\,. \label{stab7a1}
\end{multline}
Взяв в качестве $X$ с.в.\ с функцией распределения ${\mathcal
G}_{\beta,\alpha,\sigma_1}$, находим $\max\limits_{x>0} [xp(x)]$,
где
$$
xp(x)=
\fr{\beta}{\sigma_1^{\alpha\beta}\Gamma(\alpha)}\,x^{\alpha\beta}\exp\left[-\left(
\fr{x}{\sigma_1}\right)^\beta\right]\,.
$$
Для этого находим
$$
[xp(x)]^{\prime} = \fr{\beta^2}{\sigma_1^{\alpha\beta}\Gamma(\alpha)}
\exp\left[-\left(\fr{x}{\sigma_1}\right)^\beta\right] \left(
\alpha - x^{\beta}\sigma_1^{-\beta}\right)\,.
$$
Поэтому функция $xp(x)$ имеет глобальный максимум в точке $x_0 =
\sigma_1\alpha^{1/\beta}$ и
$$
\max\limits_{x>0} [xp(x)] = xp(x)\;\biggl|_{x=x_0}\; =
\fr{\beta\alpha^{\alpha}\exp[-\alpha]}{\Gamma(\alpha)}\,.
$$
Наконец, подставив последнее уравнение в~(\ref{stab7a1}) и заменив
$W= U_{\sigma_1,\sigma_2}$, приходим к~(\ref{stab7a}). \hfill$\square$


\section{Обратная задача устойчивости: равномерная метрика}

В реальности намного больший интерес представляет решение задачи,
являющейся в определенном смысле обратной к только что
рассмотренной. {\it Обратная} задача состоит в следующем: по
заданному расстоянию (равномерному, Леви) между смесями найти
оценку расстояния между <<истинным>> и статистически оцененным
смешивающими распределениями. Решению данной задачи в терминах
равномерной метрики и метрики Леви посвящены данный и следующий
разделы. Обратная задача оценивания устойчивости ОГ-рас\-пре\-де\-ле\-ния
по отношению к <<загрязнению>> может быть сформулирована следующим
образом. Пусть для произвольно малого $\varepsilon>0$ выполнено
$\rho(F_{p,\beta_1,\beta_2}, {\mathcal G}_{\beta_1,1,1} )
\leqslant \varepsilon.$ Насколько близки к нулю будут $(1-p)$ и
$|\beta_1-\beta_2|$? Ответ на этот вопрос дает следующее
утверждение.

\medskip

\noindent
\textbf{Теорема 1.} \textit{Пусть $p\in(0,1]$, $\beta_1,\beta_2\in (0, M_{\beta}]$ для
некоторой положительной константы $M_{\beta}$ и для некоторого
$\varepsilon>0$ имеет место неравенство}
\begin{equation}
\rho(F_{p,\beta_1,\beta_2}, {\mathcal G}_{\beta_1,1,1} ) \leqslant
\varepsilon\,. \label{rho1}
\end{equation}
\textit{Тогда для параметров $p,\beta_1,\beta_2$ выполнено соотношение}
\begin{equation}
(1-p) |\beta_1-\beta_2| \leqslant
\fr{\exp[2^{M_{\beta}}]}{2^{M_{\beta}} \ln2}\,\varepsilon\,.
\label{stat1}
\end{equation}


\smallskip

Д\,о\,к\,а\,з\,а\,т\,е\,л\,ь\,с\,т\,в\,о\,.\
Несложно заметить, что
\begin{multline}
\rho(F_{p,\beta_1,\beta_2}, {\mathcal G}_{\beta_1,1,1} ) ={}\\
{}=
(1-p)\sup\limits_{x}|{\mathcal G}_{\beta_1,1,1}(x) - {\mathcal
G}_{\beta_2,1,1}(x)| \geqslant{}\\
{}\geqslant
 (1-p) |{\mathcal
G}_{\beta_1,1,1}(2) - {\mathcal G}_{\beta_2,1,1}(2)|\,.
\label{stat1_0}
\end{multline}
По формуле Лагранжа получаем
\begin{multline}
|{\mathcal G}_{\beta_1,1,1}(2) - {\mathcal G}_{\beta_2,1,1}(2)| ={}\\
{}=
|\beta_1-\beta_2| \left.\fr{d{\mathcal
G}_{\beta,1,1}(2)}{d\beta}\right|_{\beta = \bar\beta}\,,
\label{stat1_1}
\end{multline}
где $\bar\beta \in [\beta_1,\beta_2] \in (0, M_{\beta}]$.
Установим нижнюю границу для производной в правой
части~(\ref{stat1_1}):
\begin{multline*}
\left.\fr{d{\mathcal G}_{\beta,1,1}(2)}{d\beta}\right|_{\beta =
\bar\beta} = \fr{1}{\Gamma(1)}
\left[\fr{d}{d\beta}\gamma(1,2^{\beta})\right]_{\beta =\bar\beta} ={}\\
{}= \left[\fr{d}{d\beta} \int\limits_0^{2^{\beta}} \exp[-t]\, dt \right]_{\beta = \bar\beta} = {}\\
{}=
\left[\fr{d}{d\beta} \left(-\exp[-2^{\beta}]+1\right)
\right]_{\beta = \bar\beta} ={}\\
{}=
\left[2^{\beta} \ln 2 \exp[-2^{\beta}] \right]_{\beta = \bar\beta} \geqslant 
2^{M_{\beta}} \ln2 \exp[-2^{M_{\beta}}]\,.
\end{multline*}
Последнее неравенство имеет место, так как функция $2^{\beta} \ln
2 \exp[-2^{\beta}]$ убывает при любых $\beta>0$ и $\bar\beta\in(0,
M_{\beta}]$.

Объединив последнее неравенство с~(\ref{stat1_0}), получаем
\begin{multline*}
\rho(F_{p,\beta_1,\beta_2}, {\mathcal G}_{\beta_1,1,1} ) \geqslant{}\\
{}\geqslant
(1-p) |\beta_1-\beta_2| 2^{M_{\beta}} \ln2 \exp[-2^{M_{\beta}}]\,.
\end{multline*}

Следовательно, если выполнено~(\ref{rho1}), то
$$
(1-p) |\beta_1-\beta_2| \leqslant
\fr{\exp[2^{M_{\beta}}]\varepsilon}{2^{M_{\beta}} \ln2}\,.~~~~~~~~~~~~~\square
$$

%\smallskip

Таким образом, если <<загрязненная>> смесь $F_{p,\beta_1,\beta_2}$
близка к ОГ-рас\-пре\-де\-ле\-нию ${\mathcal G}_{\beta_1,1,1}(x)$ в смысле
равномерного расстояния, то и величины $(1-p)$ и
$|\beta_1-\beta_2|$ должны быть близки к нулю согласно
соотношению~(\ref{stat1}).


Зафиксируем аналогичное утверждение для параметра формы $\alpha$.
Для этого рассмотрим контаминационную модель Тьюки по $\alpha$:
\begin{equation*}
F_{p,\alpha_1,\alpha_2} = p {\mathcal G}_{1,\alpha_1,1}(x) +
(1-p){\mathcal G}_{1,\alpha_2,1}(x)\,. 
%\label{Fmix1}
\end{equation*}

\smallskip

\noindent
\textbf{Теорема 2.} \textit{Пусть $p\in(0,1]$, $\alpha_1,\alpha_2 \in [m_{\alpha},
M_{\alpha}]$ для некоторых $M_{\alpha}>m_{\alpha}>0$ и для
некоторого $\varepsilon>0$ имеет место}
\begin{equation}
\rho(F_{p,\alpha_1,\alpha_2}, {\mathcal G}_{1,\alpha_1,1} )
\leqslant \varepsilon\,. \label{rho2}
\end{equation}
\textit{Тогда для параметров $p,\alpha_1,\alpha_2$ выполнено соотношение}
\begin{equation*}
(1-p) |\alpha_1-\alpha_2| \leqslant \fr{ \max
[\Gamma^2(m_{\alpha}),\Gamma^2(M_{\alpha})]}{\gamma(1,M_{\alpha})\Gamma(e,m_{\alpha})}
\, \varepsilon\,, %\label{stat2}
\end{equation*}
\textit{где через $\gamma(s,x)$ и $\Gamma(s,x)$ обозначены нижняя и
верхняя неполная гамма-функция.}

\smallskip

\noindent
Д\,о\,к\,а\,з\,а\,т\,е\,л\,ь\,с\,т\,в\,о\,.\
Аналогично~(\ref{stat1_0})--(\ref{stat1_1}) имеем
\begin{multline}
\hspace*{-2pt}\!\rho(F_{p,\alpha_1,\alpha_2}, {\mathcal G}_{1,\alpha_1,1} )
\!\geqslant \!(1-p) |{\mathcal G}_{1,\alpha_1,1}(1) - {\mathcal
G}_{1,\alpha_2,1}(1)| ={}\\
{}= |\alpha_1-\alpha_2| \left.\fr{d{\mathcal
G}_{1,\alpha,1}(1)}{d\alpha}\right|_{\alpha = \bar\alpha}\,,
\label{stat2_1}
\end{multline}
где $\bar\alpha \in [\alpha_1,\alpha_2] \in (m_{\alpha},
M_{\alpha}]$. Установим нижнюю границу для производной в правой
части~(\ref{stat2_1}):
\begin{multline*}
\left.\fr{d{\mathcal G}_{1,\alpha,1}(1)}{d\alpha}\right|_{\alpha
= \bar\alpha} = \left[\fr{d}{d\alpha} \,
\fr{\gamma(\alpha,1)}{\Gamma(\alpha)} \right]_{\alpha =\bar\alpha} = {}\\
{}=
\left[\fr{d}{d\alpha} \, \fr{\int_0^1 t^{\alpha-1}\exp[-t]\,dt}
{\int_0^{\infty} t^{\alpha-1}\exp[-t]dt} \right]_{\alpha = \bar\alpha} ={} \\
\!\!\!\!\!{}= \Gamma^{-2}(\bar\alpha)  \left\vert  \left[
\int\limits_0^{\infty}\!\! t^{\alpha-1}\exp[-t]\,dt\! \!\int\limits_0^1\! \!t^{\alpha-1}\ln t \exp[-t]\,dt-{}\right.\right. \\
\left.\left.{}-\int\limits_0^{\infty} t^{\alpha-1}\ln t\exp[-t]dt
\int\limits_0^1 t^{\alpha-1}\exp[-t]dt \right]_{\alpha =
\bar\alpha} \right\vert\,.
\end{multline*}
Разбивая интегралы по $[0,\infty)$ на сумму интегралов по $[0,1]$
и $[1,\infty)$, получаем

\noindent
\begin{multline*}
\left.\fr{d{\mathcal G}_{1,\alpha,1}(1)}{d\alpha}\right|_{\alpha
= \bar\alpha} ={}\\
{}= \Gamma^{-2}(\bar\alpha)  \Biggl|  \Biggl[
\int\limits_1^{\infty} t^{\alpha-1}\exp[-t]\,dt \int\limits_0^1 t^{\alpha-1}\ln t \exp[-t]\,dt -{} \\
{}-\int\limits_1^{\infty} t^{\alpha-1}\ln t\exp[-t]\,dt
\int\limits_0^1 t^{\alpha-1}\exp[-t]\,dt \Biggl]_{\alpha =
\bar\alpha} \Biggl|\,.
\end{multline*}
Воспользуемся теперь свойствами гам\-ма-функ\-ции на положительной
полуоси~\cite{Abram}: $\Gamma(x)$ строго убывает от $+\infty$ в
нуле до точки $x_1$ ($x_1\approx 1{,}4616$ и
$\Gamma(x_1)\approx 0{,}8856$) и строго возрастает далее до $+\infty$
при $x\to\infty$. Следовательно, при $\bar\alpha \in [m_{\alpha},
M_{\alpha}]$ имеем
$\Gamma(\bar\alpha)\leqslant\max [\Gamma(m_{\alpha}),\Gamma(M_{\alpha})]$.
В последнем выражении первое и второе слагаемое отрицательны, что
позволяет раскрыть модуль и прийти к неравенству
\begin{multline*}
\left.\fr{d{\mathcal G}_{1,\alpha,1}(1)}{d\alpha}\right|_{\alpha
= \bar\alpha} \geqslant
\max^{-2}[\Gamma(m_{\alpha}),\Gamma(M_{\alpha})]\times{}\\
{}\times \Biggl[
\int\limits_1^{\infty} t^{\alpha-1}\exp[-t]\,dt \int\limits_0^1 t^{\alpha-1} |\ln t| \exp[-t]\,dt +{} \\
{}+\int\limits_1^{\infty} t^{\alpha-1}\ln t\exp[-t]\,dt
\int\limits_0^1 t^{\alpha-1}\exp[-t]dt \Biggl]_{\alpha =
\bar\alpha}\,.
\end{multline*}
В последнем выражении все четыре интеграла положительны. Отбросив
первое из слагаемых и сузив один из интервалов интегрирования,
приходим к
\begin{multline*}
\left.\fr{d{\mathcal G}_{1,\alpha,1}(1)}{d\alpha}\right|_{\alpha
= \bar\alpha} \geqslant
\max^{-2}[\Gamma(m_{\alpha}),\Gamma(M_{\alpha})] \times{}\\
{}\times \left[
\int\limits_e^{\infty} t^{\alpha-1}\ln t\exp[-t]dt \int\limits_0^1
t^{\alpha-1}\exp[-t]dt \right]_{\alpha = \bar\alpha}\,.
\end{multline*}
Теперь в первом интеграле отбросим сомножитель $\ln t>1$:
\begin{multline*}
\left.\fr{d{\mathcal G}_{1,\alpha,1}(1)}{d\alpha}\right|_{\alpha
= \bar\alpha} \geqslant
\text{max}^{-2}[\Gamma(m_{\alpha}),\Gamma(M_{\alpha})]\times{}\\
{}\times \left[
\int\limits_e^{\infty} t^{\alpha-1}\exp[-t]dt \int\limits_0^1
t^{\alpha-1}\exp[-t]dt \right]_{\alpha = \bar\alpha}\,.
%\label{stat2_2}
\end{multline*}
Интегралы, стоящие в правой части~(\ref{stat2_1}), пред\-став\-ля\-ют
собой верхнюю и нижнюю неполные гам\-ма-функ\-ции соответственно.
Исследуем свойства этих функций. Обозначим 
$$
g_1(\alpha) = \int\limits_e^{\infty} t^{\alpha-1}\exp[-t]\,dt\,.
$$ 
%
Дифференцируя, получаем
$$
g_1^{'}(\alpha) = \int\limits_e^{\infty} t^{\alpha-1}\ln t\exp[-t]\,dt\,.
$$
%
Несложно заметить, что подынтегральное выражение положительно на
интервале $t\in(e,\infty)$ при любых значениях $\alpha$.
Следовательно, $g_1(\alpha)$ строго возрастает. Аналогично можно
убедиться в том, что 
$$
g_2(\alpha) = \int\limits_0^1 t^{\alpha-1}\exp[-t]\,dt$$ 
представляет собой строго убывающую функцию от~$\alpha$.

Таким образом,
\begin{multline*}
\left.\fr{d{\mathcal
G}_{1,\alpha,1}(1)}{d\alpha}\right|_{\alpha = \bar\alpha}
\geqslant {}\\
{}\geqslant
\gamma(1,M_{\alpha})\Gamma(e,m_{\alpha})
\max^{-2}[\Gamma(m_{\alpha}),\Gamma(M_{\alpha})]\,.
\end{multline*}

Объединив последнее неравенство с~(\ref{stat2_1}), получаем
\begin{multline*}
\rho(F_{p,\alpha_1,\alpha_2}, {\mathcal G}_{1,\alpha_1,1} )
\geqslant{}\\
{}\geqslant (1-p) |\alpha_1-\alpha_2| 
\fr{\gamma(1,M_{\alpha})\Gamma(e,m_{\alpha})}{\max\left[\Gamma^2(m_{\alpha}),\Gamma^2(M_{\alpha})\right]}\,.
\end{multline*}

Следовательно, если выполнено~(\ref{rho2}), то
$$
(1-p) |\alpha_1-\alpha_2| \leqslant
\fr{\max\left[\Gamma^2(m_{\alpha}),\Gamma^2(M_{\alpha})\right]}{\gamma(1,M_{\alpha})\Gamma(e,m_{\alpha})}
\, \varepsilon\,.~~~\square
$$

\smallskip

Установим теперь аналогичное утверждение для масштабного параметра
$\sigma$. Для этого рассмотрим контаминационную модель Тьюки по
$\sigma$:
\begin{equation*}
F_{p,\sigma_1,\sigma_2} = p {\mathcal G}_{1,1,\sigma_1}(x) + (1-p){\mathcal G}_{1,1,\sigma_2}(x)\,.
%\label{Fmix2}
\end{equation*}

\smallskip

\noindent
\textbf{Теорема 3.} \textit{Пусть $p\in(0,1]$, $\sigma_1,\sigma_2 \in (0,+\infty)$ и для
некоторого $\varepsilon>0$ имеет место неравенство}
\begin{equation}
\rho(F_{p,\sigma_1,\sigma_2}, {\mathcal G}_{1,1,\sigma_1} ) \leqslant \varepsilon\,.
\label{rho3}
\end{equation}
\textit{Тогда для параметров $p,\sigma_1,\sigma_2$ выполнено соотношение}
\begin{equation*}
(1-p) |\sigma_1-\sigma_2| < 1.85 \; \varepsilon\,.
%\label{stat3}
\end{equation*}

\smallskip

\noindent
Д\,о\,к\,а\,з\,а\,т\,е\,л\,ь\,с\,т\,в\,о\,.\
Аналогично~(\ref{stat1_0}) и~(\ref{stat1_1}) имеем
\begin{multline}
\hspace*{-1.13pt}\!\!\!\rho(F_{p,\sigma_1,\sigma_2}, {\mathcal G}_{1,1,\sigma_1} )
\!\geqslant\! (1-p) \left|{\mathcal G}_{1,1,\sigma_1}(1) - {\mathcal
G}_{1,1,\sigma_2}(1)\right| ={}\\
{}= \left|\sigma_1-\sigma_2\right| \left.\fr{d{\mathcal
G}_{1,1,\sigma}(1)}{d\sigma}\right|_{\sigma = \bar\sigma}
\label{stat3_1}
\end{multline}
для некоторого $\bar\sigma$.
Установим нижнюю границу для производной в правой части~(\ref{stat3_1}):
\begin{multline*}
\left.\fr{d{\mathcal G}_{1,1,\sigma}(1)}{d\sigma}\right|_{\sigma = \bar\sigma}\! = 
\fr{1}{\Gamma(1)}
\left[\fr{d}{d\sigma}\gamma(1,\sigma^{-1})\right]_{\sigma =\bar\sigma}\!={}\\
{}= \left[\fr{d}{d\sigma} \int\limits_0^{\sigma^{-1}} \exp[-t]\, dt \right]_{\sigma = \bar\sigma} ={}\\
\!{}= 
\left[\fr{d}{d\sigma} \left(-\exp[-\sigma^{-1}]+1\right)
\right]_{\sigma = \bar\sigma}\!\! =
\left[\fr{\exp[-\sigma^{-1}]}{\sigma^2} \right]_{\sigma =\bar\sigma}\,.\hspace*{-4.94658pt}
\end{multline*}
Функция, стоящая в квадратных скобках в последнем выражении,
принимает максимальное значение при $\sigma=0{,}5$. Следовательно,
$$
\left.\fr{d{\mathcal G}_{1,1,\sigma}(1)}{d\sigma}\right|_{\sigma
= \bar\sigma} \geqslant \fr{4}{e^2}\,.
$$

Объединив последнее неравенство с~(\ref{stat3_1}), получаем
$$
\rho(F_{p,\sigma_1,\sigma_2}, {\mathcal G}_{1,1,\sigma_1} ) \geqslant (1-p) |\sigma_1-\sigma_2| \fr{4}{e^2}\,.
$$

Следовательно, если выполнено~(\ref{rho3}), то
$$
(1-p) |\sigma_1-\sigma_2| \leqslant \fr{e^2}{4}\,\varepsilon < 1{,}85  \varepsilon\,.~~~~~~\square
$$


Рассмотрим далее один полезный вид обобщения приведенных выше
теорем. Так, рассмотрим две контаминационные смеси с различными
смешивающими параметрами $p$ и~$q$:
\begin{align*}
F_{p,\beta_1,\beta_2} &= p {\mathcal G}_{\beta_1,1,1}(x) + (1-p){\mathcal G}_{\beta_2,1,1}(x)\,,\\
F_{q,\beta_1,\beta_2} &= q {\mathcal G}_{\beta_1,1,1}(x) +
(1-q){\mathcal G}_{\beta_2,1,1}(x)\,,
\end{align*}
где $p, q\in[0,1]$ и $\beta_1,\beta_2\in (0, M_{\beta}]$.
Очевидно, что
\begin{multline*}
\rho(F_{p,\beta_1,\beta_2},F_{q,\beta_1,\beta_2}) ={}\\
{}= |p-q|
\sup\limits_x |{\mathcal G}_{\beta_1,1,1}(x) - {\mathcal
G}_{\beta_2,1,1}(x)|\,.
\end{multline*}
Поэтому, повторяя логику доказательства
теорем~1--3, можно получить следующие результаты.

\medskip

\noindent
\textbf{Утверждение 2.}
\textit{Пусть $p, q\in[0,1]$, $\beta_1,\beta_2\in (0, M_{\beta}]$ для
некоторой положительной константы $M_{\beta}$ и для некоторого
$\varepsilon>0$ имеет место неравенство}
$$
\rho(F_{p,\beta_1,\beta_2},F_{q,\beta_1,\beta_2}) \leqslant
\varepsilon\,.
$$
\textit{Тогда для параметров $p, q, \beta_1,\beta_2$ выполнено соотношение}
$$
|p-q| |\beta_1-\beta_2| \leqslant
\fr{\exp[2^{M_{\beta}}]}{2^{M_{\beta}} \ln2}\,\varepsilon\,.
$$

\medskip

\noindent
\textbf{Утверждение 3.}
\textit{Пусть $p,q \in[0,1]$, $\alpha_1,\alpha_2 \in [m_{\alpha},
M_{\alpha}]$ для некоторых $M_{\alpha}>m_{\alpha}>0$ и для
некоторого $\varepsilon>0$ имеет место неравенство}
$$
\rho(F_{p,\alpha_1,\alpha_2}, F_{q,\alpha_1,\alpha_2} ) \leqslant
\varepsilon\,.
$$
\textit{Тогда для параметров $p,q,\alpha_1,\alpha_2$ выполнено соотношение}
$$
|p-q| |\alpha_1-\alpha_2| \leqslant \fr{ \max
[\Gamma^2(m_{\alpha}),\Gamma^2(M_{\alpha})]}{\gamma(1,M_{\alpha})\Gamma(e,m_{\alpha})}
\, \varepsilon\,,
$$
\textit{где через $\gamma(s,x)$ и $\Gamma(s,x)$ обозначены нижняя и
верхняя неполная гамма-функция.}

\medskip

\noindent
\textbf{Утверждение 4.}
\textit{Пусть $p,q\in[0,1]$, $\sigma_1,\sigma_2 \in (0,+\infty)$ и для
некоторого $\varepsilon>0$ имеет место неравенство}
$$
\rho(F_{p,\sigma_1,\sigma_2}, F_{q,\sigma_1,\sigma_2} ) \leqslant \varepsilon\,.
$$
\textit{Тогда для параметров $p,q,\sigma_1,\sigma_2$ выполнено соотношение}
$$
|p-q| |\sigma_1-\sigma_2| < 1{,}85 \varepsilon\,.
$$

%\medskip

\section{Обратная задача устойчивости: метрика Леви}

Утверждения, подобные установленным в
теоремах~1--3, можно получить также и в
терминах метрики Леви $L$. Для этого достаточно воспользоваться
неравенством (\cite{Zolotarev1986}, с. 107)
$$
\rho(F,G) \leqslant (1+\sup\limits_x p_X(x)) L(F,G)\,,
$$
где $F(x)$, $G(x)$~--- функции распределения с.в.\ $X$ и $Y$, а
$p_X$~--- плотность (ограниченная) величины~$X$.

Таким образом, чтобы перейти к метрике Леви в
теоремах~1--3, необходимо определить значение
$\sup\limits_x p_{X_i}(x)$ для $X_1\sim{\mathcal G}_{\beta,1,1}$,
$X_2\sim{\mathcal G}_{1,\alpha,1}$ и $X_3\sim{\mathcal
G}_{1,1,\sigma}$, что и будет сделано в следующих трех
утверждениях.

\medskip

\noindent
\textbf{Утверждение 5.}
\textit{Если $v\in[1,+\infty)$, то}
$$
\sup\limits_x \fr{d}{dx}\,{\mathcal G}_{\beta,1,1}(x) \leqslant 1\,.
$$


\smallskip

\noindent
Д\,о\,к\,а\,з\,а\,т\,е\,л\,ь\,с\,т\,в\,о\,.\
Найдем выражение для плотности с параметрами $(\beta,1,1)$:
$$
\fr{d}{dx}{\mathcal G}_{\beta,1,1}(x) = \beta x^{\beta-1}\exp
\left[-x^{\beta}\right]\,.
$$
Исследуем поведение данной функции:
$$
\fr{d^2}{dx^2}\,{\mathcal G}_{\beta,1,1}(x) =
\beta\exp[-x^{\beta}]x^{\beta-2}\left[\beta-1-x^{\beta}\right]\,.
$$
Следовательно, так как $\beta\geqslant1$,
\begin{multline*}
\sup\limits_x \frac{d}{dx}{\mathcal G}_{\beta,1,1}(x) = {\mathcal
G}_{\beta,1,1}( (\beta-1)^{{1}/{\beta}} ) ={}\\
{}=
\beta(\beta-1)^{(\beta-1)/\beta}\exp[1-\beta]\,.
\end{multline*}
Найдем теперь максимум данного выражения при $\beta\geqslant1$.
Для этого вычислим
\begin{multline*}
\fr{d}{d\beta} \ln {\mathcal
G}_{\beta,1,1}((\beta-1)^{{1}/{\beta}}) ={}\\
{}=
\beta^{-2}\left[-\beta^3 + \beta^2 + 2\beta + \ln(\beta-1)\right]\,.
\end{multline*}
Выражение в скобках обращается в нуль в точках
$\beta_1\in(1{,}4,\,1{,}5)$ и $\beta_2=2$, при этом $\beta_1$~---
локальный минимум, а $\beta_2$~--- локальный максимум функции
${\mathcal G}_{\beta,1,1}( (\beta-1)^{1/\beta} )$. Более
того, данная функция убывает на отрезке $[1,\,\beta_1]$. Поэтому
\begin{multline*}
\max\limits_{\beta\in[1,+\infty)} {\mathcal
G}_{\beta,1,1}((\beta-1)^{{1}/{\beta}}) ={}\\
{}=
\max\limits_{\beta\in\{1,2\}} {\mathcal
G}_{\beta,1,1}((\beta-1)^{{1}/{\beta}}) ={}\\
{}= \left.{\mathcal
G}_{\beta,1,1}((\beta-1)^{{1}/{\beta}})\right|_{\beta=1} = 1\,.~\square
\end{multline*}


\noindent
\textbf{Замечание.}
При $\beta\in(0,1)$ $\sup\limits_x (d/dx){\mathcal
G}_{\beta,1,1}(x) = +\infty$.

\smallskip

\medskip

\noindent
\textbf{Утверждение 6.}
\textit{Если $\alpha\in[1,M_{\alpha}]$ для некоторого положительного
$M_{\alpha}$, то}
$$
\sup\limits_x \frac{d}{dx}{\mathcal G}_{1,\alpha,1}(x) \leqslant
\max\left[ 1,
\left(\fr{M_{\alpha}-1}{e}\right)^{M_{\alpha}-1}\right]\,.
$$

\smallskip

\noindent
Д\,о\,к\,а\,з\,а\,т\,е\,л\,ь\,с\,т\,в\,о\,.\ 
Найдем выражение для плотности с параметрами $(1,\alpha,1)$:
$$
\fr{d}{dx}{\mathcal G}_{1,\alpha,1}(x) = x^{\alpha-1}\exp [-x]\,.
$$
Исследуем поведение данной функции:
$$
\fr{d^2}{dx^2}\,{\mathcal G}_{1,\alpha,1}(x) =
x^{\alpha-2}\exp[-x]\left(\alpha-1-x\right)\,.
$$
Следовательно, так как $\alpha\geqslant1$,
$$
\sup\limits_x \fr{d}{dx}\,{\mathcal G}_{1,\alpha,1}(x) = {\mathcal
G}_{1,\alpha,1}( \alpha-1 ) =
\left(\fr{\alpha-1}{e}\right)^{\alpha-1}\,.
$$
Данная функция убывает на отрезке $[1,\,1/e+1]$ и далее
возрастает. Поэтому
\begin{multline*}
\max\limits_{\beta\in[1,M_{\beta}]} {\mathcal
G}_{1,\alpha,1}(\alpha-1) = \max\limits_{\beta\in\{1,M_{\alpha}\}}
{\mathcal G}_{1,\alpha,1}(\alpha-1) = {}\\
{}=\max \left[ 1,
\left(\fr{M_{\alpha}-1}{e}\right)^{M_{\alpha}-1}\right]\,.~~~\square
\end{multline*}


\noindent
\textbf{Замечание.} 
При $\alpha\in(0,1)$ $\sup\limits_x \frac{d}{dx}{\mathcal
G}_{1,\alpha,1}(x) = +\infty$.

\medskip

\medskip

\noindent
\textbf{Утверждение 7.}
\textit{Если $\sigma\in[m_{\sigma},+\infty]$ для некоторого положительного
$m_{\sigma}$, то}
$$
\sup\limits_x \fr{d}{dx}{\mathcal G}_{1,1,\sigma}(x) \leqslant
m_{\sigma}^{-1}\,.
$$

\smallskip

\noindent
Д\,о\,к\,а\,з\,а\,т\,е\,л\,ь\,с\,т\,в\,о\,.\
Найдем выражение для плотности с параметрами $(1,1,\sigma)$:
$$
\fr{d}{dx}{\mathcal G}_{1,1,\sigma}(x) =
{\sigma}^{-1}\exp[-x{\sigma}^{-1}]\,.
$$
Очевидно, что
$$
\sup\limits_x \fr{d}{dx}{\mathcal G}_{1,1,\sigma}(x) = {\mathcal
G}_{1,1,\sigma}( 0 ) = {\sigma}^{-1} \leqslant m_{\sigma}^{-1}\,.~~~~~\square
$$

\smallskip

Заметим теперь, что если определить с.в.\ $U_{p,\beta_1,\beta_2}$,
$p\in[0,\,1]$, как принимающую значения $\beta_1$ с вероятностью~$p$
и $\beta_2$ с вероятностью $1-p$, то из определения метрики Леви
следует, что
$$
L(U_{p,\beta_1,\beta_2},\beta_1) = \min[(1-p), |\beta_1-\beta_2|]\,.
$$
И, следовательно,
\begin{multline*}
L^2(U_{p,\beta_1,\beta_2},\beta_1) = {}\\
{}=\mathrm{min}^2\left[(1-p),
|\beta_1-\beta_2|\right] \leqslant (1-p) |\beta_1-\beta_2|\,.
\end{multline*}
В объединении с теоремами~1--3 и
утверждениями~5--7 это приводит к следующим теоремам.

\medskip

\noindent
\textbf{Теорема 4.}
\textit{Пусть $p\in(0,1]$, $\beta_1,\beta_2\in [1,\, M_{\beta}]$ для
некоторой положительной константы $M_{\beta}$ и для некоторого
$\varepsilon>0$ имеет место неравенство}
$$
L(F_{p,\beta_1,\beta_2}, {\mathcal G}_{\beta_1,1,1} ) \leqslant
\varepsilon\,.
$$
\textit{Тогда}
$$
L^2(U_{p,\beta_1,\beta_2},\beta_1) \leqslant
\fr{\exp[2^{M_{\beta}}]}{2^{M_{\beta}+1} \ln2}\,\varepsilon\,.
$$

\smallskip

\noindent
\textbf{Теорема 5.}
\textit{
Пусть $p\in(0\,,1]$, $\alpha_1,\alpha_2 \in [m_{\alpha},
M_{\alpha}]$ для некоторых $M_{\alpha}>m_{\alpha}\geqslant1$ и для
некоторого $\varepsilon>0$ имеет место неравенство}
$$
L(F_{p,\alpha_1,\alpha_2}, {\mathcal G}_{1,\alpha_1,1} ) \leqslant
\varepsilon\,.
$$
\textit{Тогда}
\begin{multline*}
L^2(U_{p,\alpha_1,\alpha_2},\alpha_1) 
\leqslant{}\\
{}\leqslant
 \max
[\Gamma^2(m_{\alpha}),\Gamma^2(M_{\alpha})]\Big /\!\!
\left.\left(\gamma(1,M_{\alpha})\Gamma(e,m_{\alpha})
\left(\vphantom{\Gamma^2\Big /}
1+{}\right.\right.\right. %\hspace*{-2.88528pt}
\\
\left.\left.{}+\max\left[ 1,\,
\left((M_{\alpha}-1)/e\right)^{M_{\alpha}-1}\right]\right)\right)
\, \varepsilon\,,
\end{multline*}
\textit{где через $\gamma(s,x)$ и $\Gamma(s,x)$ обозначены нижняя и
верхняя неполная гам\-ма-функ\-ция.} 

\medskip

\noindent
\textbf{Теорема 6.}
\textit{Пусть $p\in(0,\,1]$, $\sigma_1,\sigma_2 \in [m_{\sigma},+\infty)$
для некоторой константы $m_{\sigma}>0$ и для некоторого
$\varepsilon>0$ имеет место неравенство}
$$
L(F_{p,\sigma_1,\sigma_2}, {\mathcal G}_{1,1,\sigma_1} ) \leqslant \varepsilon.
$$
\textit{Тогда}
$$
L^2(U_{p,\sigma_1,\sigma_2},\sigma_1) < 1.85 \; \frac{m_{\sigma}}{m_{\sigma}+1} \, \varepsilon\,.
$$

{\small\frenchspacing
{%\baselineskip=10.8pt
%\addcontentsline{toc}{section}{Литература}
\begin{thebibliography}{99}

\bibitem{BKSh} %1
\Au{Батракова Д.\,А., Королев~В.\,Ю., Шоргин~С.\,Я.}
Новый метод ве\-ро\-ят\-но\-ст\-но-ста\-ти\-сти\-че\-ско\-го анализа информационных
потоков в телекоммуникационных сетях~// Информатика и её
применения, 2007. Т.~1. Вып.~1. С.~40--53.

\bibitem{KN} %2
\Au{Назаров А.\,Л., Королев В.\,Ю.} Разделение смесей вероятностных
распределений при помощи сеточных методов моментов и максимального
правдоподобия~// Автоматика и телемеханика, 2010. Вып.~3. С.~98--116.

\bibitem{St} %3
\Au{Stacy  E.\,W.} A generalization of the gamma
distribution~// Ann. Math. Statistics, 1962. Vol.~33. P.~1187--1192.

\bibitem{7} %4
\Au{Van Parr~B., Webster J.} A method for discriminating
between failure density function used in reliability predictions~// 
Technometrics, 1965. Vol.~7. P.~1--10.

\bibitem{25} %5
\Au{Farewell V., Prentice R.} A study of distributional
shape in life testing~// Technometrics, 1977. Vol.~19. P.~69--76.

\bibitem{6} %6
\Au{Basu A., Manning W.\,G.} 
Issues for the next generation
of health care analyses~// Medical Care, 2009. Vol.~47. P.~109--114.

\bibitem{26}  %7
\Au{Pham T., Almhana J.} The generalized gamma
distribution: its hazard rate and stress-strength model~// IEEE
Transactions Reliability, 1995. Vol.~44. P.~392--397.

\bibitem{28} %8
\Au{Chang J.\,H., Shin J.\,W., Kim~N.\,S., Mitra~S.\,K.}
Image probability distribution based on generalized gamma function~// 
IEEE Signal Proc.\ Lett., 2005. Vol.~12(4). P.~325--328.


\bibitem{29} %9
\Au{Shin J.\,W., Chang J.\,H., Kim~N.\,S.} 
Statistical modeling of speech signals based on generalized gamma function~//
IEEE Signal Processing Letters, 2005. Vol.~12(3). P.~258--261.

\bibitem{30}  %10
\Au{Li H.-C., Hong W., Wu~Y.-R.} Generalized gamma
distribution with MoLC estimation for statistical modeling of SAR
images~// Asian and Pacific Conference on SAR Proccedings.~---
Huangshan, China, 2007. P.~525--528.

\bibitem{31} %11
\Au{Li H.-C., Hong W., Wu~Y.-R., Fan~P.-Z.} An efficient
and flexible statistical model based on generalized gamma
distribution for amplitude SAR images~// IEEE Transactions on
Geosci. Remote Sens., 2010. Vol.~48. P.~2711--2722.

\bibitem{8} %12
\Au{Kleiber C., Kotz S.} 
Statistical size distributions in economics and actuarial
sciences.~--- New York: Wiley, 2003.


%\bibitem{27} 
%\Au{Lienhard J.\,H., Meyer~P.\,L.} A physical basis for the
%generalized gamma distribution~// Quarterly Applied
%Mathematics, 1967. Vol.~25(3). P.~330--334.


\bibitem{Tukey1960} %13
\Au{Tukey J.\,W.} A survey of sampling from contaminated
distributions~// Contributions to probability and
statistics. Essays in honor of Harold Hotelling~/ Eds.\ I.~Olkin, S.~G.~Ghurye, W.~Hoeffding,
W.~G.~Madow, H.~B.~Mann.~--- Stanford: Stanford University Press, 1960. P.~448--485.

\bibitem{K2006}  %14
\Au{Королев В.\,Ю.} Теория вероятностей и математическая
статистика.~--- М.: Проспект, 2006.


\bibitem{Korolev2007s}  %15
\Au{Королев В.\,Ю.} Вероятностно-ста\-ти\-сти\-че\-ский анализ хаотических
процессов с помощью смешанных гауссовских моделей. Декомпозиция
волатильности финансовых индексов и турбулентной плазмы.~--- М.: ИПИ
РАН, 2007. 363~с.

\bibitem{Hall1979}  %16
\Au{Hall P.} On measures of
the distance of a mixture from its parent distribution~//
Stochastic Proc. Appl., 1979. Vol.~8. P.~357--365.

\bibitem{Abram}  %17
Справочник
по специальным функциям с формулами, графиками и математическими
таблицами~/ Под ред.\ М.~Абрамовиц, И.~Стиган.~--- М.: Наука, 1979.


 \label{end\stat}
 
 \bibitem{Zolotarev1986}  %18
\Au{Золотарев В.\,М.} Современная теория суммирования независимых
случайных величин.~--- М.: Наука, 1986.


 \end{thebibliography}
}
}


\end{multicols}  