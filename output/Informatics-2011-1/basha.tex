\def\stat{basha}

\def\tit{АЛГОРИТМ АВТОМАТИЧЕСКОГО ВЫДЕЛЕНИЯ ЛИЦА 
НА~ТЕРМОГРАФИЧЕСКИХ ИЗОБРАЖЕНИЯХ$^*$}

\def\titkol{Алгоритм автоматического выделения лица 
на~термографических изображениях}

\def\autkol{Н.\,С.~Баша, Л.\,А.~Шульга}
\def\aut{Н.\,С.~Баша$^1$, Л.\,А.~Шульга$^2$}

\titel{\tit}{\aut}{\autkol}{\titkol}

{\renewcommand{\thefootnote}{\fnsymbol{footnote}}\footnotetext[1]
{Статья подготовлена по результатам работы секции 
<<Биометрия>> 20-й Международной конференции по компьютерной графике и зрению Графикон-2010, 
г.~С.-Петербург, 20--24~сентября 2010~г.}}

\renewcommand{\thefootnote}{\arabic{footnote}}
\footnotetext[1]{Научно-исследовательский институт прикладной акустики, Международный университет природы, 
общества и человека <<Дубна>>, natalia.basha@niipa.ru}
\footnotetext[2]{Научно-исследовательский институт прикладной акустики, luda.shulga@niipa.ru}

\vspace*{-6pt}

\Abst{Представлен подход к исследованию термографических изображений человека 
для задач интеллектуального видеонаблюдения. Предложен алгоритм автоматического 
выделения лица в инфракрасном (ИК) спектре излучения, приведены результаты его работы и 
проведен анализ эффективности на базе данных, состоящей из 103 термографических портретов 
15~человек разного пола, возраста и телосложения, сделанных в различных условиях 
окружающей среды.}

\vspace*{-4pt}

\KW{распознавание образов; анализ изображений; системы видеонаблюдения; термография; 
детекция лица}

%            \vspace*{-4pt}

      \vskip 10pt plus 9pt minus 6pt

      \thispagestyle{headings}

      \begin{multicols}{2}
      
            \label{st\stat}
            


\section{Введение}
  
  В настоящее время для задач видеонаблюдения все чаще применяются 
устройства, позволяющие улавливать ИК излучение объекта и получать 
его температурные карты (термограммы). Для получения термографических 
изображений используются специальные тепловизионные камеры (тепловизоры 
или термографы). Существенными преимуществами их использования по 
сравнению с видеокамерами являются:
  \begin{itemize}
\item нечувствительность к освещенности объекта и способность работать в 
полной темноте;
\item способность давать вполне приемлемое для опознавания изображение даже 
при значительном удалении от человека;
\item нечувствительность к внешней маскировке (например, элементам макияжа).
\end{itemize}

  Эти отличительные черты позволяют применять термографию в тех условиях, 
когда получение изображения с видеокамеры недостаточно для реализации 
поставленных целей~[1, 2].
  
  Работы, связанные с задачами распознавания лиц в  
  ИК-диа\-па\-зо\-не, ведутся последние 10~лет и решаются с помощью 
высокочувствительных видеокамер, работающих в отра\-жен\-ном ИК-диа\-па\-зо\-не. 
Возможность применять тепловизионные камеры для данного рода исследований 
появилась недавно. 
  
  Информационными признаками в термографии служат подкожные рисунки 
артерий и вен, которые уникальны и неизменны для каждого человека, так как 
сосудистый рисунок не зависит от температуры лица, пластических операций и 
фактора старения человека.

\vspace*{-6pt}

\section{Методы исследования}

\vspace*{-2pt}

\subsection{База данных}

  Для апробации предложенного метода детекции лица была собрана база 
термографических изображений. Все изображения, приведенные в статье (рис.~1), 
получены термографом <<ИРТИС-2000МЕ>> и представляют собой матрицу 
температур размером $240\times 320$, снятых в спектральном диапазоне 3--5~мкм, с точностью изменения температур 0,01~$^\circ$C. База 
термографических данных состоит из\linebreak
 \begin{center} %fig1
\vspace*{2pt}
\mbox{%
\epsfxsize=78mm
\epsfbox{bas-1.eps}
}
\end{center}
\vspace*{2pt}
%\begin{center}
{{\figurename~1}\ \ \small{Изображение лица, полученное термографом <<ИРТИС-2000МЕ>>: 
(\textit{а})~двумерное изображение термограммы лица; (\textit{б})~трехмерное отобра\-же\-ние 
термограммы лица}}
%\end{center}
%\vspace*{9pt}

%\bigskip
\addtocounter{figure}{1}

    \begin{figure*} %fig2
    \vspace*{1pt}
\begin{center}
\mbox{%
\epsfxsize=161.999mm
\epsfbox{bas-2.eps}
}
\end{center}
\vspace*{-6pt}
\Caption{Изображения лица, полученное термографом: (\textit{а})~гипертермия области 
внутренних углов глаз (норма); (\textit{б})~симметричная гипертермия в области ключиц, на 
границе с одеждой; (\textit{в})~гипертермия в области носа и рта; (\textit{г})~гипертермия в 
области лба, частично разрываемая волосами
  \label{f2-ba}}
  \end{figure*}
  
\noindent
 103~снимков 15~людей разного возраста, 
пола и телосложения. Термограммы получены в разное время суток, при разных 
условиях окружающей среды (в помещении и на улице). Сбор данных проводился 
еженедельно в течение 4~месяцев.


\subsection{Алгоритм автоматического поиска лиц на термографических 
изображениях}
  
  Проведено исследование физиологических особенностей лица человека с целью 
выделения универсальной и стабильной температурной области, которую можно 
использовать как область привязки при выделении области лица.



В результате исследования было выяснено, что внутренний угол глаза здорового 
человека дает наиболее интенсивный отклик в ИК-диапазоне (рис.~1 
и~\ref{f2-ba},\,\textit{а}). Экспериментально были замечены некоторые 
отклонения от общей тенденции. Основные причины: симметричные зоны, 
связанные с гипертермией на границе открытых и закрытых участков тела 
(рис.~\ref{f2-ba},\,\textit{б}), с воспалительными процессами 
(рис.~\ref{f2-ba},\,\textit{в}), с вегетососудистой дистонией (рис.~\ref{f2-ba},\,\textit{г}). 
Результаты наблюдений были учтены при разработке алгоритма 
детекции лица. Поэтому в качестве опорных точек области интереса были 
выбраны точки внутренних углов глаз (рис.~3,\,\textit{а}). 
Использование данных меток эффективно потому, что с их помощью по 
коэффициентам про\-пор\-ци\-о\-наль\-ности на лице можно вычислить размеры самого 
лица (рис.~3,\,\textit{б}) и расположение основных деталей~[3].

 
%\noindent

  Рассмотрим алгоритм автоматического выделения области лица, базирующийся 
на детекции внут\-рен\-них углов глаз. Анализируя тепловые профили, 
соответствующие зоне внутренних углов глаз, было установлено, что данные 
точки на профиле\linebreak
 \begin{center} %fig3
\vspace*{1pt}
\mbox{%
\epsfxsize=80.23mm
\epsfbox{bas-3.eps}
}
\end{center}
\vspace*{4pt}
%\begin{center}
{{\figurename~3}\ \ \small{Алгоритм автоматической детекции лица: (\textit{а})~выставление меток, 
соответствующих внутренним углам глаз; (\textit{б})~наложение маски коэффициентов 
пропорциональности лица человека; (\textit{в})~результат детекции лица}}
%\end{center}
%\vspace*{9pt}

\bigskip
\addtocounter{figure}{1}

\noindent
 пред\-став\-ля\-ют\-ся в виде двух четко выраженных пиков, 
симметричных относительно серединной линии лица. Тепловой профиль по 
строке, соответствующей зоне внутренних углов глаз, приведен на рис.~4. 




  Разработанный алгоритм выделения области лица состоит из нескольких 
этапов.
  
  \medskip
  
  \noindent
  {\bfseries\textit{Первый этап: пороговая фильтрация}}
  \smallskip
  
  Первым этапом является предварительная обработка изображения методом 
пороговой фильтрации с целью выделения из окружающей среды объекта, 
который может оказаться человеком. Значение порога было установлено 
экспериментально на уровне $\min + (\max - \min)/3$, где $\min$ и $\max$~--- 
соответственно минимальное и максимальное значения температуры на термограмме. Для 
человека это значение порога соответствует температуре 
кожи и не включает одежду и волосы. Для каждой строки изображения находится 
левая и правая граница\linebreak\vspace*{-12pt}
\pagebreak

\end{multicols}

\begin{figure} %fig4
  \vspace*{1pt}
\begin{center}
\mbox{%
\epsfxsize=159.968mm
\epsfbox{bas-4.eps}
}
\end{center}
\vspace*{-6pt}
\Caption{Тепловой профиль по строке, соответствующей зоне внутренних углов глаз: (\textit{а})~часть 
исходного изображения термограммы лица (линией выделена строка, содержащая 
внутренние углы глаз); (\textit{б})~соответствующий ей тепловой профиль
\label{f4-ba}}
\end{figure} 

\begin{figure} %fig5
  \vspace*{3pt}
\begin{center}
\mbox{%
\epsfxsize=123.914mm
\epsfbox{bas-5.eps}
}
\end{center}
\vspace*{-6pt}
  \Caption{Процесс выделения лица: (\textit{а})~этап~I: пороговая фильтрация и определение 
верхней границы лица; (\textit{б})~этап~II: определение моды правой и левой границ лица; 
(\textit{в})~этап~III: выделение и корректировка линии глаз
  \label{f5-ba}}
  \vspace*{3pt}
  \end{figure}

\begin{multicols}{2}


\noindent
 области предполагаемого лица, а также точка~$Y_t$ для 
всего изображения, соответствующая верхней точке потенциальной области лица 
(рис.~\ref{f5-ba},\,\textit{а}).
  
  \medskip

  \noindent
    {\bfseries\textit{Второй этап: вычисление интервала для нахождения линии 
глаз}}
  
  \smallskip
  
  На втором этапе вычисляется мода для левой и правой границы 
предполагаемой области лица ($X_l$ и $X_r$ соответственно), мода для средней 
линии об\-ласти~$X_{mid}$ (как мода середины отрезка между правой и левой 
границей каждой конкретной строки выделенной области изображения) и мода 
ширины области~$W$ (как мода разности между правой и левой границей каждой 
конкретной строки выделенной области изображения, рис.~\ref{f5-ba},\,\textit{б}). 
Экспериментально было установлено, что значение моды ширины области 
предполагаемого лица соответствует ширине реального лица на уровне глаз.
  
  Исходя из пропорций лица, значения верхней точки головы и значения средней 
ширины рассчитывается теоретическое значение линии глаз
  $$
  Y_{\mathrm{eye}}= Y_t+\mathrm{round}\,\left(0{,}68\times W\right)\,.
  $$
  
%  \end{multicols}
  

  
  Экспериментально установлено, что для на\-хож\-де\-ния реального уровня глаз 
необходимо задать окрестность~$b$. Таким образом, реальный уровень глаз 
будет находиться в диапазоне (рис.~\ref{f5-ba},\,\textit{в})
  \begin{equation}
  \left [ Y_{\mathrm{eye}}-b;\, Y_{\mathrm{eye}}+b\right]\,.
  \label{e1-ba}
  \end{equation}
  

 
%  \smallskip


    \noindent
    {\bfseries\textit{Третий этап: поиск линии глаз}}
  \smallskip
  
  Как было установлено выше, глаза являются зонами гипертермии. Поэтому на 
выделенном интервале~(1) ведется поиск значения максимума 
температуры~$T_{\max}$. Затем построчно осуществляется проход маской с двумя 
пиками (обозначаемыми~$P_1$ и~$P_2$) и впадиной ($H$) всего выделенного 
интервала. Среди строк, удовлетворяющих условиям маски, в качестве линии глаз 
выбирается строка~$y_0$, где (см.\ рис.~\ref{f4-ba}):
  \begin{itemize}
\item значение одного из пиков равно максимуму на интервале

\noindent
$$
t_{P_1}=T_{\max}\parallel t_{P_2} -T_{\max}\,;
$$
\item пики симметрично расположены относительно средней линии

\noindent
$$
x_{P_1}-W=W-x_{P_2}\,;
$$
\item пики симметрично расположены относительно соответствующих 
границ лица

\noindent
$$
x_{P_1}-X_l=X_r-x_{P_2}\,;
$$
\item впадина расположена на средней линии

\noindent
$$
x_H=X_{\mathrm{mid}}\,;
$$
\item разница температур между пиком и впадиной больше порогового 
значения (экспериментально было установлено пороговое значение, равное 
0,5~$^\circ$C):

\noindent
$$
T_{\max}-t_H>0{,}5~^\circ\mathrm{C}\,,
$$
где $x_{P_1}$, $x_H$ и $x_{P_2}$~--- значения абсцисс точек $P_1$, $H$ и $P_2$, а
$t_{P_1}$, $t_H$ и~$t_{P_2}$~--- значения температур в точках ($x_{P_1}, y_0)$, ($x_H, y_0)$ и
($x_{P_2}, y_0)$ соответственно.
\end{itemize}

\medskip

  \noindent
    {\bfseries\textit{Четвертый этап: определение области лица}}
 \smallskip
  
    На изображении расставляются маркеры глаз (соответствующие выбранным 
пикам~$P_1$ и~$P_2$). Вычисляется ширина области предполагаемого лица
на уровне глаз и по ней рассчитывается верхний
 угол прямоугольника, заключающего потенциальное лицо, его длина и ширина. Выделенная таким 
образом область принимается за лицо.

\setcounter{figure}{6}
\begin{figure*} %[b] %fig7
    \vspace*{1pt}
\begin{center}
\mbox{%
\epsfxsize=162.7mm
\epsfbox{bas-7.eps}
}
\end{center}
\vspace*{-6pt}
\Caption{Результат работы алгоритма автоматического выделения области лица: 
(\textit{а})~термограмма; (\textit{б})~соответствующее ей изображение в видимом диапазоне; 
(\textit{в})~результат работы алгоритма выделения лица
\label{f7-ba}}
\end{figure*}
  
  
  Предложенный метод автоматической детекции области лица проверен на базе, 
состоящей из 103~изображений 15~человек. Результат выделения лица 
продемонстрирован на рис.~6. По данной выборке доля верного 
выделения лиц составила 98\%. Анализ ошибочно детектированных изображений 
показал, что ошибка связана с размытостью исходной термограммы (по причине 
движения объекта во время съемки). В~экспериментальных исследованиях 
установлено, что на качество детекции лица конкретного человека не влияют ни 
изменение\linebreak\vspace*{-16pt}

\columnbreak

%\vspace*{-12pt}
    \begin{center} %fig6
\vspace*{1pt}
\mbox{%
\epsfxsize=79mm
\epsfbox{bas-6.eps}
}
\end{center}
%\vspace*{2pt}
%\begin{center}
{{\figurename~6}\ \ \small{Результат работы алгоритма автоматического выделения области лица}}
%\end{center}
%\vspace*{9pt}

\medskip
%\addtocounter{figure}{1}

\addtocounter{figure}{1}

%\pagebreak

\noindent

\noindent
прически и волосяного покрова лица (наличия и отсутствия бороды, 
усов), ни температурные условия сбора изображений. Сбор проводился в 
лабораторных условиях и на открытом воздухе в зимнее время.
В обоих случаях выделение области 
лица проходило успешно. 

  
  
  Проведена временн$\acute{\mbox{а}}$я оценка работы алгоритма: пакетная 
обработка 103~изображений составила 157~с; таким образом, среднее время 
выделения одного лица по термограмме составляет 1,52~с (в среде MatLab).
  
  С целью расширения класса термограмм, на которых применим разработанный 
метод выделения лица, приведем результаты детекции для термограмм
различных объектов. 
  
  
  На рис.~\ref{f7-ba},\,\textit{а} представлен термографический снимок, на 
котором лицо конкурирует по температуре с другим объектом. Для обработки 
этого термографического снимка сначала производилась операция 
сегментирования областей, похожих на лицо (выделены 2~области: <<лицо>> и 
<<чайник>>, находящиеся в одном температурном диапазоне и схожие по 
площади). Далее применялся разработанный алгоритм выделения лица. Объект 
<<чайник>> был отбракован на третьем этапе: не были найдены маркеры линии глаз. На рис.~\ref{f7-ba},\,\textit{в} 
представлен результат детекции лица.

\vspace*{-6pt}

\section{Выводы}

  Предложен алгоритм автоматического выделения лица на термографических 
изображениях. Описанный алгоритм базируется на поиске внут\-рен\-них углов глаз, 
которые являются стабильной гипертермической областью на лице человека. 
Высокий показатель правильной детекции лица (98\%) подтвердил эффективность 
представленного алгоритма.
  
  Стоит отметить, что данный алгоритм можно использовать как алгоритм 
автоматического определения присутствия человека в кадре, например для 
систем контроля доступа и охраны периметра.
  
  В~дальнейшем рассмотренный подход можно применять для идентификации 
личности в интеллектуальных системах видеонаблюдения.

\vspace*{-6pt}

{\small\frenchspacing
{%\baselineskip=10.8pt
\addcontentsline{toc}{section}{Литература}
\begin{thebibliography}{9}

\bibitem{1-ba}
\Au{Evans~D.}
Infrared facial recognition technology being pushed toward emerging applications~// 
Proc. SPIE, 1997. Vol.~2962. P.~276--286.

\bibitem{2-ba}
\Au{Иваницкий Г.\,Р.}
Современное матричное тепловидение в биомедицине~// Успехи физических 
наук, 2006. Т.~176. №\,12. С.~1293--1320.

 \label{end\stat}

\bibitem{3-ba}
\Au{Куприянов В.\,В., Стовичек~Г.\,В.}
Лицо человека: Анатомия, мимика.~--- М.: Медицина, 1988.
 \end{thebibliography}
}
}


\end{multicols}  