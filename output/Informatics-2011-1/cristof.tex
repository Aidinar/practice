\def\stat{ul}

\def\tit{О ТОЧНОСТИ ПРИБЛИЖЕНИЙ НОРМИРОВАННЫХ ХИ-КВАДРАТ РАСПРЕДЕЛЕНИЙ   
   АСИМПТОТИЧЕСКИМИ РАЗЛОЖЕНИЯМИ ЭДЖВОРТА--ЧЕБЫШЕВА$^*$}

\def\titkol{О точности приближений нормированных хи-квадрат распределений   
%асимптотическими разложениями Эджворта--Чебышева
}

\def\autkol{Г.~Кристоф, В.\,В.~Ульянов}
\def\aut{Г.~Кристоф$^1$, В.\,В.~Ульянов$^2$}

\titel{\tit}{\aut}{\autkol}{\titkol}

{\renewcommand{\thefootnote}{\fnsymbol{footnote}}\footnotetext[1]
{Исследования выполнены при частичной поддержке РФФИ, гранты 08-01-00567, 09-01-12180.}}

\renewcommand{\thefootnote}{\arabic{footnote}}
\footnotetext[1]{Магдебургский университет, факультет математики,  gerd.christoph@ovgu.de}
\footnotetext[2]{Московский государственный университет им.\ М.\,В.~Ломоносова, 
факультет вычислительной математики и кибернетики, vulyan@gmail.com}

\Abst{Рассмотрено распределение нормированной хи-квад\-рат случайной величины с  $n$~степенями свободы. 
С~помощью разложений Эджвор\-та--Че\-бы\-ше\-ва построены вычислимые оценки приближений этого распределения 
различного порядка: $O(n^{-1/2})$, $O(n^{-1})$ и $O(n^{-3/2})$. Результаты такого типа полезны в приложениях, 
в частности при анализе свойств статистик отношения правдоподобия.
}

\KW{асимптотические разложения; оценки погрешности; хи-квадрат распределение}

      \vskip 14pt plus 9pt minus 6pt

      \thispagestyle{headings}

      \begin{multicols}{2}
      
            \label{st\stat}

\section{Введение}

Пусть ${\cal X}_n^2$ есть случайная величина, имеющая хи-квад\-рат функцию распределения   $G_n(x)$ с 
$n$~степенями свободы  и функцию плотности
\begin{equation*}
%\label{g0}
p_{{\cal X}_n^2}\,(x) =
\fr{1}{2^{n/2}\,\Gamma(n/2)} \, x^{-1+n/2}  e^{-x/2}\,I_{(0,\,\infty)}(x)\,,
\end{equation*}
где $I_A(x)$ обозначает индикаторную функцию множества~$A$. Отметим, что
$ E({\cal X}_n^2) = n $,  $ Var({\cal X}_n^2) =2\,n$.

\smallskip

В прикладных исследованиях   хи-квад\-рат распределение возникает часто. 
Например, распределение нормированной хи-квад\-рат статистики
\begin{equation*}
T_1 = {\cal X}^2_n - n \log \fr{{\cal X}^2_n}{n} - n
%\label{chi1}
\end{equation*}
появляется в критериях отношения правдоподобия при проверке нулевой гипотезы  о равенстве дисперсии~$\sigma^2$ 
заданному значению, когда выборка объема $n+1$  берется из нормального распределения  $N(\mu, \sigma^2)$, 
см.\ также соответствующие многомерные обобщения в~[1]. Другие примеры использования хи-квад\-рат 
случайных величин в приложениях, в частности  в информатике, см. в~\cite{9-cr}; в гл.~5, 13, 15, 16 в~\cite{2-cr}; 
а также в~\cite{10-cr}, где описан новый подход по <<очистке>> изображений. При этом оказывается, 
что уже при  малом числе степеней свободы хи-квад\-рат распределение может быть успешно заменено 
нормальным распределением. В настоящей работе найдены оценки погрешности такой замены. 
В~[1] доказано, что
\begin{equation}
\label{g1}
\sup_x\,|P (T_1 \leq x) - G_1 (x)| \leq C(n)\,,
\end{equation}
где  $C(n)$ есть вычислимая величина, при этом  $C(n) = O(n^{-1})$ при
$n \rightarrow\infty$. В лемме~3 в~\cite{1-cr}
найдена оценка сверху для приближения функции распределения нормированной случайной величины ${\cal X}_n^2$ с 
помощью разложения Эджворта--Чебышева первого порядка. Отметим, что эта оценка вносит заметный вклад в величину
 $C(n)$  в правой части~(\ref{g1}).
 
 \smallskip
 
Цель настоящей работы~--- уточнить эту оценку, используя  разложения Эджворта--Чебышева второго порядка.
\smallskip

Пусть $X= ({\cal X}_1^2 - 1)/\sqrt{2}$ есть нормированная хи-квадрат случайная величина 
с одной степенью свободы и   $X, X_1, X_2,\ldots$~--- независимые одинаково распределенные 
случайные величины. Тогда нормирование  ${\cal X}_n^2$ приводит к случайной величине
\begin{multline*}
V_n =\,(2\,n)^{-1/2}\left({\cal X}_n^2-n\right)= {}\\
{}=n^{-1/2}\left(X_1 + X_2 + \ldots + X_n\right)\,. 
\end{multline*}
В силу центральной предельной теоремы функция распределения
$$ 
F_n(x) = P(V_n \leq x)=P\left({\cal X}_n^2-n \leq \sqrt{2\,n}\,x\right)
$$
стремится к нормальному закону   
$$ 
\Phi(x) =  (2\pi)^{-1/2} \int\limits_{- \infty}^x e^{-u^2/2}\,du 
$$
при $n \rightarrow~\infty$.

Кроме этого, из неравенства Берри--Эссеена находим

\noindent
\begin{equation}
\label{g1a}
\sup\nolimits_x |F_n(x) - \Phi(x)| \leq \fr{C\,E|X|^3 }{ \sqrt{n}} \leq \fr{1{,}4720}{\sqrt{n}}\,,
\end{equation}
где $E|X|^3 = 3{,}0729\ldots$ и $C = 0{,}478\ldots$ (см., например,~\cite{3-cr, 4-cr}).
%
Если использовать не только информацию о третьем абсолютном моменте распределения~$V_n$, 
но и другие свойства   нормированной величины  ${\cal X}_n^2$, то можно получить оценку, 
более точную, чем~(\ref{g1a}). В~\cite{5-cr} доказано, что
при всех $\lambda \in (0,\sqrt{3}-1)$  и целых $n\geq 1$
$$\sup\nolimits_x |F_n(x) - \Phi(x)| \leq D(\lambda, n)\,,
$$
где

\noindent
\begin{multline} 
D(\lambda, n) = \fr{1}{3\,\sqrt{\pi\,n}} +
\fr{2}{\pi n}\left(
\vphantom{\fr{(1+\lambda^2)^{1-n/4}}{\lambda^2}}
\fr{2(1-\lambda)}{(2-2\lambda - \lambda^2)^2} +{}\right.\\
\left.{}+
\fr{(1+\lambda^2)^{1-n/4}}{\lambda^2} +
\fr{1}{\lambda^2}e^{-\lambda^2 n/4}
\right)\,.
\label{g2}
\end{multline}
Поскольку   $E(X^3) = 2\sqrt{2}$ и $E(X^4)=15$, используя
разложения Эджворта--Чебышева первого и второго порядков
\begin{equation}
\label{g2a} 
\Phi_{1,n}(x) = \Phi (x) - \fr{\sqrt{2}\,(x^2-1)}{3 \sqrt{n}}
\,\fr{1}{\sqrt{2\pi}}\, e^{-x^2/2}\,;
\end{equation}

\vspace*{-12pt}

\noindent
\begin{multline}
\Phi_{2,n}(x) = \Phi (x) -\fr{e^{-x^2/2}}{\sqrt{2\,\pi}}
\left( \fr{\sqrt{2}\left(x^2-1\right)}{3\sqrt{n}}+{}\right.\\
\left.{}+
   \fr{x^5 -10x^3+15x}{9n}+\fr{x^3-3x}{2n}\right)\,,
   \label{g2b}
\end{multline}
получаем

\noindent
\begin{align*}
 F_n(x) = \Phi_{1,n}(x) + O(n ^{-1})\,;\\
 F_n(x) = \Phi_{2,n}(x) + O(n^{-3/2})
\end{align*}
 при $n \to \infty$.
В~\cite{6-cr}  доказано следующее неравенство (см.\ пример~3 в~\cite{6-cr} при  $a=b=1/2$):
\begin{multline}
\sup\nolimits_x\, |F_n(x) - \Phi_{1,n}(x)|
\leq D_1^*(n)=\fr{1{,}9}{n} \left( \fr{n}{n-1}\right)^2
+{}\\
{}+\fr{15{,}59}{n}\, 0{,}9906^n+ C(n) 0{,}9894^n
\label{g3} 
\end{multline}
с
$C(n) = 15{,}21/(n-4)$, для $n > 32$ и $C(n) = 0{,}5271$ для $4 \leq n \leq 32.$
%
Аналогичные оценки в   $L^1$-норме для разности плотности   $p_{V_n} (x)$ и
$d\Phi_{1,n}(x)/dx$ получены в~\cite{7-cr}.
%
Приведем исправленный вариант леммы~3 из~\cite{1-cr} в виде теоремы.

\medskip

\noindent
\textbf{Теорема~1.}
\textit{Для всех $\lambda \in (0,\sqrt{3}-1)$  и целых $n\geq 1$ справедливо неравенство
$$\sup\nolimits_x |F_n(x) - \Phi_{1,n}(x)| \leq D_1(\lambda, n)\,, $$
где}
\columnbreak

\noindent
\begin{multline}
D_1(\lambda, n) = \fr{2}{\pi n}\left(
\vphantom{\fr{(1+\lambda^2)^{1-n/4}}{\lambda^2}}
\fr{4}{9} +
\fr{2(1-\lambda)}{(2-2\lambda - \lambda^2)^2} +{}\right.\\
\!\left.{}+\fr{(1+\lambda^2)^{1-n/4}}{\lambda^2} +
\fr{3+\lambda}{3\lambda^2}\,e^{-\lambda^2(3-\lambda)n/(12+4\lambda)}
\right).\!\!
\label{g4}
\end{multline}

\smallskip

Отметим, что первое слагаемое в скобках в правой части~(\ref{g4}) есть 
$4/9$, а не $1/9$, как было напечатано в~\cite{1-cr}. Это  вытекает из оценки 
$ I_{11} \leq 16/(9n)$ на с.~1160. В~связи с этим ко всем значениям в таблице для~ $D(n)$ 
на с.~1155 в~\cite{1-cr} следует прибавить  $2/(3\pi n)$.

\section{Основные результаты и их обсуждение}

Сначала приведем две оценки точности приближений для распределения нормированной хи-квад\-рат 
случайной величины ${\cal X}_n^2$, полученных с помощью разложения Эджворта--Чебышева второго порядка.

\medskip

\noindent
\textbf{Теорема~2.} \textit{При всех $\lambda \in (0,\,0{,}75)$  и целых $n\geq 1$ имеем
$$
\sup\nolimits_x \left\vert F_n(x) - \Phi_{2,n}(x)\right\vert \leq D_2(\lambda, n)\,, 
$$
где с $a=a(\lambda)=1-\lambda^2/2-\lambda^4/(3-3\lambda)$ и $b=b(\lambda)=$\linebreak $=1-\lambda^2/2$,}

\noindent
\begin{multline*}
D_2(\lambda, n)  = \fr{1}{\sqrt{\pi}}\left(\fr{1349}{270n^{3/2}}+\fr{21}{n^{5/2}}\right)+{}\\
{}+
\fr{1}{2\pi}\left[\fr{32}{(3-3\lambda)a^3n^2} +\fr{12}{b^4n^2}+{}\right.\\
\left. {}+\fr{4}{\lambda^2n} (1+\lambda^2)^{1-n/4}+
\left( \ln\left(1+\fr{4}{\lambda^2n}\right)+ \fr{2\lambda}{3}+{}\right.\right.\\
\left. \left.{}+\fr{8}{3\lambda n(1+\sqrt{1+16/(\pi \lambda^2 n)})}
+\fr{\lambda^4 n}{18}+{}\right.\right.\\
\left. \left. {}+ \fr{17\lambda^2}{18} +\fr{34}{9\,n}\right)e^{-\lambda^2 n/4}\right]
\,.
%\label{g5111}
\end{multline*}



Значения для $D_2(\lambda,n)$ при различных значениях  $\lambda$ и~$n$, вычисленные в    MAPLE,
представлены в табл.~1.

Результат теоремы~2 можно улучшить, построив более точные, но и более громоздкие оценки. Определим функции
\begin{align*}
  U_8(A)&= \Bigg(A^7+7A^5+35A^3+ 105A + {}\\
&{}+\fr{210}{A+\sqrt{A^2+4}}\Bigg)e^{-A^2/2};\\
  U_7(B)&= \left(B^3+3B^2+6B+6\right)e^{-B};
\end{align*}


  \noindent
\begin{center} %tabl1
\parbox{56mm}{{\tablename~1}\ \ \small{Значения для $D_2(\lambda,n)$ 
при различных значениях  $\lambda$ и~$n$, вычисленные в    MAPLE}}
%\begin{center} %fig1

\vspace*{2ex}
\tabcolsep=14.6pt
{\small
\begin{tabular}{|c|c|c|}
\hline
$n$   & $\lambda$& $D_2(\lambda,n)$\\
\hline
\hphantom{9}50 &0,60 & 0,01862\\
100  & 0,54& 0,00423\\
200  & 0,44 & 0,00121\\
300&   0,37 & 0,00062\\
500&  0,30 & 0,00028\\
1000\hphantom{9} & 0,23 & 0,00009\\
\hline 
\end{tabular}}
\end{center}
%\vspace*{12pt}
%\begin{center}
%\end{center}
\vspace*{9pt}

\smallskip
\addtocounter{table}{1}

\noindent
  \begin{align*}
  U_6(A)&= \Bigg(\!A^5+5A^3+ 15A + \fr{30}{A+\sqrt{A^2+4}}\!\Bigg)e^{-A^2/2};\\
U_5(B)& = \left(B^2+2B+2\right)e^{-B}\,;\\
U_4(A)&= \left(A^3+ 3A + \fr{6}{A+\sqrt{A^2+4}}\right)e^{-A^2/2} \,.
\end{align*}

%\vspace*{6pt}




\noindent
\textbf{Теорема 3.} \textit{Пусть $a=a(\lambda) $ и $b=b(\lambda) $  те же, что в теореме~2. 
При всех $\lambda \in (0,0.75)$  и целых $n\geq 1$ имеем
$$
\sup\nolimits_x \left\vert F_n(x) - \Phi_{2,n}(x)\right\vert \leq D_2^*(\lambda, n)\,, 
$$
где
\begin{multline*}
D_2^*(\lambda, n) = \fr{2{,}8189}{n^{3/2}}+\fr{1{,}6977}{ (1-\lambda)a^3n^2} +
\fr{1{,}9099}{b^4n^2}+{}\\
{}+\fr{11{,}8480}{n^{5/2}} -U(\lambda,n)
+\fr{0{,}6366}{\lambda^2 n} (1+\lambda^2)^{1-n/4}
+{}\\
{}+\left(\fr{0{,}6366}{\lambda^2 n}+
\fr{0{,}2122}{\lambda \,n}+
\fr{0{,}6012 }{n}+ 0{,}1061\lambda + {}\right.\\
\left.{}+0{,}1503 \lambda^2 +0{,}0088 \lambda^4 n
\vphantom{\fr{0{,}}{0{,}}}\right)e^{-\lambda^2 n/4}
\end{multline*}
и функция $U(\lambda,n)$ определяется формулой}
\begin{multline*} %\label{g5a}
 U(\lambda,n)  = \fr{\sqrt{2}}{\pi n^{3/2}}\left[ 
 \vphantom{\left(\fr{1}{2}\right)^1}
 \left(\fr{1}{81}\,U_8\left(\lambda \sqrt{\fr{n}{2}}\right) + {}\right.\right.\\
\left. {}+
 \fr{1}{6}\, U_6\left(\lambda\,\sqrt{\fr{n}{2}}\right) + \fr{2}{5}\, U_4\left(\lambda \sqrt{\fr{n}{2}}\right) \right)+{}\\
{} + \left(\fr{4}{3(1-\lambda)a^3}\, U_5\left(a\lambda^2 \fr{n}{4}\right) + {}\right.\\
{}+\left.\left.\fr{1}{2b^4}\, 
U_7\left(b \lambda^2 \fr{n}{4}\right)\right)\!\left(\fr{2}{n}\right)^{1/2} 
+ \fr{1}{5 n}\, U_8\left(\lambda\,\sqrt{\fr{n}{2}}\right) \!\right].
\end{multline*}

Значения для $D_2^*(\lambda,n)$ при различных значениях  $\lambda$ и~$n$, вычисленные в    MAPLE,
представлены в табл.~2.

%\medskip


Теперь можно оценить   $\sup\nolimits_x \left\vert F_n(x) - \Phi_{1,n}(x)\right\vert$ и 
$\sup\nolimits_x \left\vert F_n(x) - \Phi(x)\right\vert$ с помощью теорем~2 и~3.   

Пусть $Q_2(x)/n$ есть член порядка   $1/n$ в разложении Эджворта--Чебышева (\ref{g2b}). Тогда

\columnbreak


\noindent
\begin{center}
\parbox{56mm}{{\tablename~2}\ \ \small{Значения для 
$D_2^*(\lambda,n)$ при различных значениях  $\lambda$ и~$n$, вычисленные в    MAPLE}}
%\begin{center} %fig1

\vspace*{2ex}
\tabcolsep=15.6pt
{\small
\begin{tabular}{|c|c|c|}
\hline
$n$    & $\lambda$ & $D_2^*(\lambda,n)$\\
\hline
\hphantom{9}20   & 0,69 & 0,06698\\
\hphantom{9}30   &0,62 & 0,03981\\
\hphantom{9}40    &0,61 & 0,02030\\
\hphantom{9}50  & 0,59& 0,01353\\
100   & 0,51 & 0,00389\\
150 & 0,45 & 0,00193\\
\hline 
\end{tabular}
}
\end{center}
%\vspace*{9pt}

\smallskip
\addtocounter{table}{1}


%\vspace*{-6pt}

\noindent
\begin{multline*} 
\sup\nolimits_x|Q_2(x)|=\sup\nolimits_x \left|\fr{e^{-x^2/2}}{\sqrt{2\pi}}
  \left( \fr{x^5 -10x^3+15x}{9}+{}\right.\right.\\
\left.\left.  {}+\fr{x^3-3x}{2}\right)\right|\leq 0{,}1269\,.
  \end{multline*}
Заметим, что $Q_2(x)$ принимает экстремальные значения при   $x=\pm 1{,}43$.

\smallskip

\noindent
\textbf{Следствие 1.} \textit{При всех $\lambda \in (0,3/4)$  и целых $n\geq 1$ имеем}

\vspace*{-3pt}

\noindent
\begin{multline}
\!\! \!\!\!\sup\nolimits_x \left\vert F_n(x) - \Phi_{1,n}(x)\right\vert
  \leq \sup_x \left\vert Q_2(x)\right\vert  \fr{1}{n} + 
  D_2(\lambda,n) \leq {}\\
  {}\leq \fr{0{,}12690}{n} +D_2(\lambda,n)=D_1^{**}(\lambda,n)\,.
  \label{g6a}
\end{multline}

%\vspace*{-3pt}

\noindent
\begin{center}
\parbox{80mm}{{\tablename~3}\ \ \small{Значения для $D^*_1(n)$, $D_1(\lambda,n)$ и $D_1^{**}(\lambda,n)$ 
при различных значениях  $\lambda$  и~$n$, вычисленные в    MAPLE}}
%\begin{center} %fig1

\vspace*{2ex}
%\tabcolsep=5pt
{\small
\begin{tabular}{|c|c||c|c||c|c|}
\hline
$n$ & $D^*_1(n)$ & $\lambda$ &   $D_1(\lambda,n)$ & $\lambda$ &   $D_1^{**}(\lambda,n)$\\
\hline
\hphantom{9}50   & 0,42808 & 0,50 & 0,03858 & 0,60 & 0,02115\\
100   & 0,13460 & 0,45 & 0,01275 & 0,54 & 0,00550\\
150  & 0,05911 & 0,42 & 0,00712 & 0,48 & \hphantom{9}0,002839\\
200  & 0,03059 & 0,40 & 0,00501 & 0,44 & 0,00185\\
300& 0,01153 & 0,31 & 0,00281 & 0,38 & 0,00105\\
500 & 0,00424 & 0,25 & 0,00153 & 0,31 & 0,00053\\
\hline 
\end{tabular}
}
\end{center}
%\vspace*{9pt}

\smallskip
\addtocounter{table}{1}


В табл.~3 для различных значений $n$ и $\lambda$ и в случае приближения~$F_n(x)$  
разложением первого порядка $\Phi_{1,n}(x)$ приведены  величины погрешностей   $D^*_1(n)$ 
из неравенства~(\ref{g3}), см.~\cite{6-cr}; величины   $D_1(\lambda,n)$ из~(\ref{g4}), см.~\cite{1-cr}, 
и $D_1^{**}(\lambda,n)$ из неравенства~(\ref{g6a}), доказанного в настоящей работе.



Пусть $Q_1(x)$ есть член порядка   $1/\sqrt{n}$ в разложениях Эджворта--Чебышева~(\ref{g2a})  
или~(\ref{g2b}). Тогда

\noindent
\begin{multline*} 
\sup\nolimits_x|Q_1(x)|=\sup\nolimits_x \left|\fr{e^{-x^2/2}}{\sqrt{2\pi}}\,
  \fr{\sqrt{2}\left(x^2-1\right)}{3 \sqrt{n}}\right|={}\\
  {}= \fr{1}{3\sqrt{\pi n}}\leq \fr{0{,}18806}{\sqrt{n}}\,.
  \end{multline*}
Заметим, что $Q_1(x)$ достигает экстремального значения при   $x=0$.

\medskip

\noindent
\textbf{Следствие 2.} \textit{При всех $\lambda \in (0,3/4)$  и целых $n\geq 1$ имеем}
\begin{multline}
\sup\nolimits_x \left\vert F_n(x) - \Phi(x)\right\vert \leq  \fr{0{,}18806}{\sqrt{n}}+
\fr{0{,}12690}{n} +{}\\
{}+D_2^*(\lambda,n)= D_0(\lambda,n)\,.
\label{g6b}
\end{multline}

\noindent
\begin{center}
\parbox{80mm}{{\tablename~4}\ \ \small{Значения для БЭ-оценки, 
$D(\lambda,n)$ и $D_0(\lambda,n)$ при различных значениях~ $\lambda$ и~$n$, 
вычисленные в    MAPLE}}
%\begin{center} %fig1

\vspace*{2ex}
\tabcolsep=4.5pt
{\small 
\begin{tabular}{|c|c||c|c||c|c|}
\hline
$n$    &БЭ-оценка & $\lambda$ &   $D(\lambda,n)$ & $\lambda$ & $D_0(\lambda,n)$\\
\hline
\hphantom{9}50   & 0,20818 & 0,483 & 0,05498 & 0,57 & 0,04266\\
100  & 0,14720 & 0,416 & 0,02763 & 0,49 & 0,02397\\
300 & 0,08499 & 0,295 & 0,1267\hphantom{9} & 0,37 & 0,01190\\
500 & 0,06583 & 0,244 & 0,00935 & 0,30 & 0,00894\\
1000\hphantom{9} & 0,04655 & 0,196 & 0,00638 & 0,23 & 0,00628\\
\hline 
\end{tabular}
}
\end{center}
\vspace*{9pt}

\smallskip
\addtocounter{table}{1}

В табл.~4 для различных значений~$n$ и~$\lambda$ и в случае приближения $F_n(x)$  
стандартной нормальной функцией распределения   $\Phi(x)$ приведены  величины 
погрешностей, даваемые неравенством Берри--Эссеена (БЭ-оцен\-ка)~(\ref{g1a}), 
а также значения $D(\lambda,n)$, определенного в~(\ref{g2}), см.~\cite{5-cr}, 
и значения $D_0(\lambda,n)$ из неравенства~(\ref{g6b}), доказанного в настоящей работе.


Значения параметра  $\lambda$, появляющегося в выражениях для   $D_2(\lambda, n)$, 
$D_1(\lambda, n)$ и $D_0(\lambda, n)$, следует выбирать в зависимости от размера выборки~$n$.

\section{Доказательство теорем~2 и~3}

Пусть $f_n(t)$ и $g^*_n(t)$ есть характеристическая функция распределения   $F_n(x)$ и преобразование Фурье--Стил\-тье\-са 
функции   $\Phi_{2,n}(x)$ соответственно. Тогда, обозначая   $m=n/2$, имеем
\begin{align*}
 f_n(t) &= e^{-it\sqrt{m}}\left(1-\fr{i\,t}{\sqrt{m}}\right)^{-m}\,;
\\
\quad g^*_n(t) &=
\left(1+\fr{(i\,t)^3}{3\sqrt{m}}+\fr{(it)^6}{18m}+\fr{(it)^4}{4m} \right) e^{-t^2/2}\,. 
\end{align*}
Поскольку функции $F_n(x)$ и $\Phi_{2,n}(x)$ имеют непрерывные производные, воспользуемся 
формулой обращения для характеристических функций
\begin{multline*}
\left| F_n(x) - \Phi_{2,n}(x) \right| ={}\\
{}=  \fr{ 1}{ 2\pi}\,\left|
\int\limits_{-\infty}^\infty e^{-\,itx}\fr{f_n(t)\,-\,g^*_n(t)}{-\,it}\,dt \right| \leq{}\\
{}\leq  \fr{ 1}{ 2\pi}\left( I_1+I_2+I_{31}+I_{32}\right)\,, 
\end{multline*}
где
\begin{align*}
 I_1 &= \int\limits_{|t|<\lambda\,\sqrt{m}} \fr{1}{|t|}
\left|\vphantom{\fr{(it)^3}{\sqrt{m}}}
f_n(t)-{}\right.\\
&\left.{}-e^{-t^2/2}\left(1+\fr{(it)^3}{3\sqrt{m}}+
\fr{(it)^6}{18m}+\fr{(it)^4}{4m} \right) \right| \, dt\,;\\
 I_2&=\int\limits_{|t|\geq \lambda \sqrt{m}} \fr{|f_n(t)|}{|t|} \, dt \,;\\
  I_{31} &= \int\limits_{|t|\geq\lambda \sqrt{m}} \fr{e^{-t^2/2}}{|t|}
  \left|\left(1+\fr{(it)^3}{3\sqrt{m}}\right)\right| \,dt\,;\\
 I_{32}&= \int\limits_{|t|\geq\lambda \sqrt{m}} \fr{e^{-t^2/2}}{|t|}
  \left|\left(\fr{(it)^6}{18m}+\fr{(it)^4}{4m} \right) \right|\, dt\,.
  \end{align*}

\medskip


Сначала оценим   $|f_n(t)\,-\,g^*_n(t)|$ для $|t| \le \lambda\,\sqrt{m}$, когда    $\lambda < 3/4$ и $m=n/2$.
Используя ту ветвь комплексной функции    $\tau(z)=\ln(1 - z)$, для которой
  $\tau(0) =0$, запишем
\begin{multline}
f_n(t)  =  \exp\left\{-it\sqrt{m}-m\ln\left(1-\fr{it}{\sqrt{m}}\right)\right\}={}\\
{}= \exp\left\{\!
-\fr{t^2}{2} +\fr{(it)^3}{3\sqrt{m}}+\fr{(it)^4}{4m}+ %{}\right.\\
%\left.{} +
\fr{(it)^5}{5m^{3/2}}+mR_m(t)\!\right\}={}\hspace{-1.603pt}\\
{} = \exp\left\{-\fr{t^2}{2} +\fr{(it)^3}{3\sqrt{m}}+\fr{(it)^4}{4m}+
\fr{(it)^5}{5m^{3/2}}\right\}+S_{1,m}(t)={}\\
{} =  e^{-t^2/2}\left(1+\fr{(it)^3}{3\sqrt{m}}+\fr{(it)^6}{18m}+\fr{(it)^4}{4m} \right)+{}\\
{} +S_{2,m}(t)+S_{1,m}(t)\,,
\label{g7}
\end{multline}
где
\begin{multline*}
 R_m(t) =  -\ln\left(1-\fr{ it}{\sqrt{m}}\right) -\fr{ it}{ \sqrt{m}}-{}\\
 {}-\fr{(it)^2}{ 2m} -
\fr{ (it)^3}{ 3m^{3/2}}-\fr{ (it)^4}{ 4m^{2}}-\fr{ (it)^5}{ 5m^{5/2}}\,;
\end{multline*}

\vspace*{-12pt}

\begin{multline*}
S_{1,m}(t) = \exp\left\{-\fr{t^2}{2} +\fr{(it)^3}{3\sqrt{m}}+
\fr{(it)^4}{4m}+{}\right.\\
\left.{}+\fr{(i\,t)^5}{5m^{3/2}}\right\}\left(\exp\{mR_m(t)\}-1\right)\,;
\end{multline*} 

\vspace*{-12pt}

\noindent
\begin{multline*}
 S_{2,m}(t)=e^{-t^2/2}\left(\!\exp\!\left\{\fr{(it)^3}{3\sqrt{m}}+\fr{(it)^4}{4m} +
 \fr{ (it)^5}{ 5m^{3/2}}\right\}-{}\right.\hspace*{-0.5347pt}\\
\left.{}-1-\fr{(it)^3}{3\sqrt{m}}-\fr{(it)^6}{18m}-\fr{(it)^4}{4m}\right)\,.
\end{multline*}

Используя 
$$
\left| -\ln(1-z) -z -\fr{z^2}{2} - \fr{z^3}{3} - \fr{z^4}{4} - \fr{z^5}{5} \right| \leq \fr{|z|^6}{6(1-|z|)}
$$ 
при 
$|z| < 1$, находим, что для  $|t| < \lambda \sqrt{m}$ справедливы неравенства:

\noindent
$$
m|R_m(t)|\leq \fr{  |t|^6}{  6m^2(1-|t|/\sqrt{m})}
\leq \fr{t^2\lambda^4}{6(1- \lambda)}\,,
$$
и, обозначая 

\noindent
$$
a=a(\lambda)=1-\fr{\lambda^2}{2} - \fr{\lambda^4}{3(1-\lambda)}\,,
$$ 
получаем:

\noindent
\begin{multline}
\left|S_{1,m}(t)\right| \leq  e^{-t^2/2+t^4/(4m)} \left|e^{m R_m(t)}-1\right| \leq {}\\
{} \leq  e^{-t^2/2+t^2\lambda^2/4} m|R_m(t)| e^{m |R_m(t)|} 
\leq{}\\
{}\leq  \fr{|t|^6 e^{-at^2/2}}{6m^2(1-\lambda)} \,.
\label{g8}
\end{multline}
%
 Для оценки $S_{2,m}(t)$ положим:
 $$
 u= \fr{(it)^3}{3\sqrt{m}}\,;\enskip
w= \fr{(it)^4}{4\,m}\,; \enskip z=\fr{(it)^5}{5m^{3/2}}\,.
$$  
Тогда имеем:

\noindent
\begin{gather*}
|e^u|= 1\,;\enskip \left|e^u-1-u-\fr{u^2}{2}\right|\leq \fr{u^3}{6}\,;\enskip\\[-1pt]
|e^w-1-w|\leq \fr{w^2e^{|w|}}{2}\,;\enskip |e^z|=1\,;\enskip |e^z - 1|\leq |z|\,;
\end{gather*}

\vspace*{-12pt}

\noindent
\begin{multline}
\big|S_{2,m}(t)\big|  ={}\\[-2pt]
{}=  e^{-t^2/2}\left(\left|e^u e^w e^z- 1 - u - \fr{u^2}{2} - w \right|\right)\leq{}\\[-1pt]
{} \leq e^{-t^2/2}\left(
\left|e^u (1+w) e^z- 1 - u - \fr{u^2}{2} - w \right| +{} \right.\\[-1pt]
\left.{} + \left|e^u(e^w - 1 -w)e^z\right| \vphantom{\fr{u^2}{2}}\right)\leq{}\\[-1pt]
{} \leq  e^{-t^2/2}\left(
\vphantom{\fr{|w|^2 e^{|w|}}{2}}\left|e^u(1+w) - 1 - u - \fr{u^2}{2} - w \right| + {}\right.\\[-1pt]
\left.{} + \left|e^u (1 + w)(e^z -1)\right|  +\fr{|w|^2e^{|w|}}{2}
\right)\leq{}\\[-1pt]
{} \leq  e^{-t^2/2}\left( 
\vphantom{\fr{|w|^2 e^{|w|}}{2}}
\left| e^u -1 - u - \fr{u^2}{2}\right| + |w|\cdot |(e^u - 1)| +{} \right.\\[-1pt]
\left.{} + (1+|w|)|z| + \fr{|w|^2 e^{|w|}}{2}\right)\leq{}\\[-1pt]
{} \leq  e^{-t^2/2}\left( \vphantom{\fr{|u|^3 e^{|w|}}{6}}
\fr{|u|^3}{6}+|w|\cdot |u|+ |z| +{}\right.\\[-1pt]
\left.{}+ |w||z| +\fr{|w|^2 e^{|w|}}{2}\right)\leq{}\\[-1pt]
{} \leq  e^{-t^2/2}\left(\fr{|t|^9}{162m^{3/2}}+\fr{|t|^7}{12m^{3/2}}+\fr{ |t|^5}{5m^{3/2}}+{}\right.\\[-1pt]
\left.{} + \fr{ |t|^9}{20m^{5/2}}+\fr{|t|^8}{32m^2}\, e^{\lambda^2 t^2 /4}\right)\,.
\label{g9}
\end{multline}
Перейдем теперь к оценке интегралов. Из~(\ref{g7})--(\ref{g9}) следует, 
что $I_1 \leq I_{11} +I_{12}$, где

\noindent
\begin{multline*}
I_{11}  =  \int\limits_{|t|<\lambda\sqrt{m}} \fr{|S_{1,m}(t)|}{|t|} \,dt \leq{}\\
{}\leq \int\limits_{|t| \leq \lambda\sqrt{m}} \fr{|t|^5}{6m^2(1-\lambda)} \,e^{-at^2/2}\,dt \leq{}\\
{}\leq  \fr{4}{3m^2(1-\lambda)a^3}\int\limits_0^\infty u^2e^{-u}\,du
= \fr{8}{3m^2(1-\lambda)a^3} 
%\label{g102}
\end{multline*}
с  $a=1 -  \lambda^2/2 -   \lambda^4/[3(1-\lambda)]$,
и, поскольку $E(Y^8)=$\linebreak $=105$, $E(Y^6)=15$ и $E(Y^4)=3$ для стандартной нормальной случайной величины~$Y$,
\begin{multline*} %\label{g103}
I_{12}  =  \int\limits_{|t|<\lambda\sqrt{m}} \fr{|S_{2,m}(t)|}{|t|} \,\,dt\leq{}\\[3pt]
{} \leq \int\limits_{|t| < \lambda\sqrt{m}} e^{-t^2/2}\left(
\fr{|t|^8}{162m^{3/2}}+\fr{|t|^6}{12m^{3/2}}+{}\right. \\[3pt]
\left.{}+\fr{ |t|^4}{5m^{3/2}}+\fr{ |t|^8}{20m^{5/2}}+\fr{|t|^7}{32m^2}\, e^{\lambda^2 t^2 /4}\,\right)\,dt \leq{}\\[3pt]
{}\leq \fr{105 \sqrt{2\pi}}{162m^{3/2}}+\fr{15\sqrt{2\pi}}{12m^{3/2}} +
\fr{3\sqrt{2\pi}}{5m^{3/2}} +\fr{105\sqrt{2\pi}}{20m^{5/2} }+
\fr{3}{b^4 m^2}
\end{multline*}
с $b=b(\lambda)=1- \lambda^2/2$.


В приведенных выше оценках для интегралов   $I_{11}$ и~$I_{12}$ верхний предел $\lambda\sqrt{m}$ 
был заменен бесконечностью. Если этого не делать, то можно получить более точные, но вместе с тем 
более громоздкие оценки:
из формулы~7.1.13 в~\cite{8-cr}   вытекает, что
\begin{equation}
\fr{2e^{-r^2/2}}{r + \sqrt{r^2 + 4}} \leq \int\limits_r^{\infty} e^{-t^2/2}\,dt \leq 
\fr{2 e^{-r^2/2}}{r + \sqrt{r^2 + 8/\pi}}\,.
\label{g102a}
\end{equation}
Используя интегрирование по частям для   $k=4, 3, 2$  и нижнюю оценку из~(\ref{g102a}) для $k=0$, находим
\begin{multline*}
  \int\limits_0^{\lambda\sqrt{m}}t^{2k}\,e^{-t^2/2}\,dt =\int\limits_0^{\infty}t^{2k}e^{-t^2/2}\,dt - {}\\
  {}-
  \int\limits_{\lambda\sqrt{m}}^{\infty}t^{2k}e^{-t^2/2}\,dt \leq{}\\
{}\leq  \sqrt{\fr{\pi}{2}}(2k-1)!!-U_{2k}\left(\lambda\,\sqrt{\fr{n}{2}}\,\right).
\end{multline*}
Аналогично с помощью интегрирования по частям для   $k=3$  и $k=2$ находим, что
\begin{multline*}
\int\limits_0^{\lambda\sqrt{m}} t^{2k+1}e^{-ct^2/2}\,dt =
\int\limits_0^{c\lambda^2 m/2}u^k e^{-t^2/2}\,dt\leq{}\\
{}\leq\fr{2^k}{c^{k+1}}\left(k! - U_{2k+1}\fr{c\lambda^2m}{2}\right)\,,
\end{multline*}
где функции   $U_4(\ldots)- U_8(\ldots)$ 
определены перед формулировкой теоремы~3. Отрицательные члены оказывают заметное влияние на величины 
оценок, когда   $m = n/2 $ мало. При больших значениях~$n$ этим влиянием можно пренебречь.  

Для $I_2$ воспользуемся оценкой, полученной в~\cite{1-cr}    с 
$|f_n(t)| = (1+ t^2/m)^{-m/2}$:
\begin{multline*}
I_2 \leq \int\limits_{|t|\geq \lambda\sqrt{m}} \fr{1}{|t|} \,\fr{1}{\left(1+t^2/m\right)^{m/2}}\,dt
={}\\
{}=  \int\limits_{u\geq \lambda^2} \fr{du}{u\left(1+u \right)^{m/2}}\leq
 \fr{1+\lambda^2}{\lambda^2}\int\limits_{u\geq \lambda^2} \fr{du}{(1+u )^{1+m/2}}={}\\
 {}= \fr{1+\lambda^2}{\lambda^2m/2}\left(1+\lambda^2 \right)^{-m/2}\,.
\end{multline*}
Для оценки   $I_{31}$ воспользуемся формулой~5.1.20 из~[10]
\begin{multline*} 
\int\limits_z^{\infty} \fr{e^{-t^2/2}}{t}\,dt= \fr{1}{2}\int\limits_{z^2/2}^{\infty}\fr{e^{-u}}{u}\, du \leq{}\\
{}\leq
\fr{1}{2}\, e^{-z^2/2}\,\ln\left(1+\fr{2}{z^2}\right) \,,
\end{multline*}
интегрированием по частям и верхней оценкой в~(\ref{g102a}). Тогда получаем
\begin{multline*} 
I_{31} \leq 2\int\limits_{\lambda\sqrt{m}}^{\infty} e^{-t^2/2}
\left(\fr{1}{t} + \fr{t^2}{3\sqrt{m}} \right)\, dt \leq{}\\
{} \leq \left( 
\vphantom{\fr{4}{\sqrt{\lambda^2}}}
\ln\left(1+\fr{2}{\lambda^2m}\right)+ {}\right.\\
\left.{}+
\fr{2\lambda}{3}+\fr{4}{3\lambda m(1+\sqrt{1+8/(\pi \lambda^2 m)})}\right) e^{-\lambda^2m/2}\,;
\end{multline*}

\noindent
\begin{multline*}
I_{32} \leq  2\int\limits_{\lambda\sqrt{m}}^{\infty} e^{-t^2/2}
\left(\fr{t^5}{18m} + \fr{t^3}{4m}\right)\, dt
 \leq{} \\
{} \leq \left( \fr{\lambda^4m^2 + 4\lambda^2 m +8}{9m} + 
\fr{\lambda^2m + 2}{2m}\right)e^{-\lambda^2m/2} \,.
\end{multline*}
Объединение оценок для   $I_1, I_2$, $I_{31}$ и~$I_{32}$ завершает доказательство теорем~2 и~3.

{\small\frenchspacing
{%\baselineskip=10.8pt
%\addcontentsline{toc}{section}{Литература}
\begin{thebibliography}{99}

\bibitem{1-cr}
\Au{Ульянов  В.\,В., Кристоф~Г., Фуджикоши~Я.}  О~приближениях
преобразований хи-квадрат распределений в статистических приложениях~//
Сибирский математический журнал, 2006. Т.~47. №\, 6. С.~1401--1413.

\bibitem{9-cr} %2
\Au{Sezgin A., Oechtering T.\,J.} 
Complete characterization of the
equivalent MIMO channel for quasi-orthogonal space-time codes~//
IEEE Transactions on Information Theory, 2008. Vol.~54. No.\,7. P.~3315--3327.


\bibitem{2-cr} %3
\Au{Fujikoshi Y.,   Ulyanov~V.\,V.,   Shimizu~R.} Multivariate
statistics: High-dimensional and large-sample approximations.~---
Hoboken, N.J.: John Wiley and Sons, 2010.

\bibitem{10-cr} %4
\Au{Hawwar Y., Reza~A.} Spatially adaptive multiplicative noise
image denoising technique~// IEEE Transactions on Image Processing,
2002. Vol.~11. No.\,12. P.~1397--1404.

\bibitem{3-cr} %5
\Au{Королев В.\,Ю., Шевцова И.\,Г.} Уточнение неравенства
Берри--Эссеена с приложениями к пуассоновским и смешанным
пуассоновским случайным суммам~// Обозрение прикладной и
промышленной математики, 2010. Т.~17. Вып.~1. С.~25--56.

\bibitem{4-cr} %6
\emph{Тюрин И.\,С.}  Уточнение верхних оценок констант в теореме Ляпунова~// УМН, 2010. 
Т.~65. Вып.~3(393). С.~201--202.

\bibitem{5-cr} %7
\Au{Кавагучи Ю., Ульянов В.\,В., Фуджикоши~Я.} Приближения для
статистик, описывающих геометрические свойства данных большой
размерности, с оценками ошибок~// Информатика и её применения, 2010.
Т.~4. Вып.~1. С.~22--27.

\bibitem{6-cr} %8
\emph{Dobric V., Ghosh B.\,K.} 
Some analogs of the Berry--Esseen bounds for first-order
Chebyshev--Edgeworth expansions~// Statist. Decisions, 1996. Vol.~14. No.\,4. P.~383--404.

\bibitem{7-cr} %9
\emph{Christoph G., Ulyanov~V.}
Bounds for $L_1$-approximation of chi-squared-density by a first order Chebyshev--Edgeworth-expansion~//
Int.\ J.~Communications in Dependability and Quality Management, 2006. Vol.~9. No.\,1. P.~12--16.

 \label{end\stat}

\bibitem{8-cr} %10
Справочник по специальным функциям~/ Под ред. М.~Абрамовица, И.~Стиган.~--- М.: Наука, 1979.


 \end{thebibliography}
}
}


\end{multicols}  