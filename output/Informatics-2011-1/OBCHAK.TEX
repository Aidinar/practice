\def\stat{abstr}
{%\hrule\par
%\vskip 7pt % 7pt
\raggedleft\Large \bf%\baselineskip=3.2ex
A\,B\,S\,T\,R\,A\,C\,T\,S \vskip 17pt
    \hrule
    \par
\vskip 21pt plus 6pt minus 3pt }

\label{st\stat}

%1
\def\tit{TWO-PRIORITY SYSTEM WITH RESERVATION OF CHANNELS AND~MARKOV INPUT FLOW}


\def\aut{A.\,V.~Pechinkin}
\def\auf{IPI RAN, apechinkin@ipiran.ru}

\def\leftkol{\ } % ENGLISH ABSTRACTS}

\def\rightkol{\ } %ENGLISH ABSTRACTS}

\titele{\tit}{\aut}{\auf}{\leftkol}{\rightkol}

%\vspace*{-2pt}

\noindent
The two-priority queueing system with trunk reservation and 
Markovian input flow of customers is considered. It is supposed that customers 
of each priority have two different phase-type distributions of their service 
times. Trunk reservation means that there are some channels which can be used 
only by customers of high priority.  On the basis of 
researching the special general model, analytic relations that permit 
to calculate the main 
stationary characteristics of the queueing system were obtained.


%\vspace*{-5pt}

\KWN{queueing system; relative priority; reserving the channels}

%\pagebreak

% \thispagestyle{headings}

\vskip 14pt plus 6pt minus 3pt

%2
\def\tit{IMPROVEMENTS OF THE NONUNIFORM ESTIMATE FOR CONVERGENCE OF~DISTRIBUTIONS
OF POISSON RANDOM SUMS TO~THE~NORMAL DISTRIBUTION}

\def\aut{S.\,V.~Gavrilenko}

\def\auf{Department of Mathematical Statistics, Faculty of Computational Mathematics and
Cybernetics, M.\,V.~Lomonosov Moscow State University, gavrilenko.cmc@gmail.com}


\titele{\tit}{\aut}{\auf}{\leftkol}{\rightkol}

%\vspace*{-2pt}

\noindent
The nonuniform estimates for convergence rate in the
 central limit theorem have been built. Using these structural
 improvements, it is shown that absolute constant in the
 nonuniform estimate for convergence rate in the central limit
 theorem for Poisson random sums is strictly less than similar
 constant in the nonuniform estimate for convergence rate in the classical central limit
 theorem and, assuming finite third moment, it does not exceed
 22.7707. As a result, nonuniform estimates for convergence
 rate of the mixed Poisson, particularly, negative
 binomial, random sums have been built.

%\vspace*{-5pt}

\KWN{central limit theorem; convergence rate;
nonuniform estimate; absolute constant; Poisson random sum; mixed
Poisson distribution}
%\pagebreak


\vskip 14pt plus 6pt minus 3pt


%3
\def\tit{ON ACCURACY OF APPROXIMATIONS FOR STANDARDIZED CHI-SQUARED DISTRIBUTIONS BY~EDGEWORTH--CHEBYSHEV EXPANSIONS}

\def\aut{G.~Cristoph$^1$ and V.\,V.~Ulyanov$^2$}

\def\auf{$^1$Institute for Mathematical Stochastics, Faculty of Mathematics, University of Magdeburg, Magdeburg, Germany,\\
$\hphantom{^1}$gerd.christoph@ovgu.de\\[1pt]
$^2$Department of Mathematical Statistics, Faculty of Computational Mathematics and
Cybernetics,\\
$\hphantom{^1}$M.\,V.~Lomonosov Moscow State University, vulyan@gmail.com}

\titele{\tit}{\aut}{\auf}{\leftkol}{\rightkol}

%\vspace*{-2pt}

\noindent
A standardized chi-squared random variable $V_n$ with $n$ degrees of freedom is considered
with Edgeworth--Chebyshev expansions, the computable error bounds of orders 
$O(n^{-1/2})$, $O(n^{-1})$, and  $O(n^{-3/2})$ for approximations of $V_n$ have been obtained. 
The results are useful in applications, in particular, in analysis of statistics ratio of
probability features.

%\vspace*{-5pt}
\KWN{asymptotic expansions; error bounds; chi-squared distribution}
\pagebreak

% \vskip 14pt plus 6pt minus 3pt

\def\leftkol{ENGLISH ABSTRACTS}

\def\rightkol{ENGLISH ABSTRACTS}

 
 %4
\def\tit{STABILITY OF FINITE MIXTURES OF GENERALIZED GAMMA-DISTRIBUTIONS
WITH~RESPECT TO~DISTURBANCE OF~PARAMETERS}

\def\aut{V.\,Yu.~Korolev$^1$, V.\,A.~Krylov$^2$, and V.\,Yu.~Kuz'min$^3$}

\def\auf{$^1$Department of Mathematical Statistics, 
Faculty of Computational Mathematics and Cybernetics,\\ 
$\hphantom{^1}$M.\,V.~Lomonosov Moscow State University;
Institute of Informatics Problems, Russian Academy of Sciences,\\
$\hphantom{^1}$vkorolev@comtv.ru\\[1pt]
$^2$Department of Mathematical Statistics, 
Faculty of Computational Mathematics and Cybernetics,\\ 
$\hphantom{^1}$M.\,V.~Lomonosov Moscow State University,
vkrylov@cs.msu.ru\\[1pt]
$^3$Department of Mathematical Statistics, 
Faculty of Computational Mathematics and Cybernetics,\\ 
$\hphantom{^1}$M.\,V.~Lomonosov Moscow State University, silencershade@gmail.com}


\def\leftkol{ENGLISH ABSTRACTS}

\def\rightkol{ENGLISH ABSTRACTS}

\titele{\tit}{\aut}{\auf}{\leftkol}{\rightkol}


\noindent
Using the example of the ``contamination'' model, in terms of uniform
distance and Levy metric, estimates of the stability of finite mixtures of
generalized gamma-distributions against small disturbance of parameters were obtained.

%\vspace*{-5pt}

\KWN{generalized gamma-distribution; finite mixture; Levy metric}
 \vskip 14pt plus 6pt minus 3pt

%5
\def\tit{ON THE ACCURACY OF THE NORMAL APPROXIMATION TO~DISTRIBUTIONS OF~POISSON RANDOM SUMS}

\def\aut{Yu.\,S.~Nefedova$^1$ and I.\,G.~Shevtsova$^2$}

\def\auf{$^1$Department of Mathematical Statistics, 
Faculty of Computational Mathematics and Cybernetics,\\ 
$\hphantom{^1}$M.\,V.~Lomonosov Moscow State University, julia\_n@inbox.ru\\[1pt]
$^2$Department of Mathematical Statistics, 
Faculty of Computational Mathematics and Cybernetics,\\ 
$\hphantom{^1}$M.\,V.~Lomonosov Moscow State University, ishevtsova@cs.msu.su}

\titele{\tit}{\aut}{\auf}{\leftkol}{\rightkol}

%\vspace*{-2pt}

\noindent
Two-sided bounds were constructed for the constant in the Berry--Esseen 
inequality for Poisson random sums of independent identically distributed 
random variables with finite absolute moments of order $2+\delta$ with $\delta\in(0,1]$. 
The lower bounds were obtained for the first time. For the case $0<\delta<1$, 
the upper bounds were sharpened, and the nonuniform estimates were proved.


\KWN{central limit theorem; Poisson random sums; Berry--Esseen inequality; absolute constant; nonuniform estimate}

 \vskip 14pt plus 6pt minus 3pt

%6
\def\tit{INFORMATION TECHNOLOGY OF ACTIVE PARAMETRIC IDENTIFICATION OF~STOCHASTIC
QUASI-LINEAR DISCRETE SYSTEMS}

\def\aut{V.\,M.~Chubich}
\def\auf{Department of Applied Mathematics, Faculty of
Applied Mathematics and Computer Sciences,\\  
Novosibirsk State Technical University, chubich\_62@ngs.ru}

\titele{\tit}{\aut}{\auf}{\leftkol}{\rightkol}

%\vspace*{-2pt}

\noindent
Some theoretical and applied aspects of the active parametric identification of the Gaussian nonlinear
discrete systems are considered for the first time. The original results are obtained for the case when the
parameters of the mathematical models to be estimated appear in the state and observation equations;
 the initial conditions and covariance matrices of the dynamic noise and
measurements errors were considered. An example of optimal parameter estimation of one model structure is shown.


\KWN{parameter estimation; maximum likelihood method; optimal input
signal design; Fisher information matrix; optimality criterion}
%\pagebreak

\vskip 14pt plus 6pt minus 3pt

%7
\def\tit{AGENT MODELING OF TERRITORIAL SYSTEM DEVELOPMENT}

\def\aut{K.\,S.~Chirkunov}

\def\auf{A.\,P.~Ershov Institute of Informatics Systems, cyril.chirkunov@computer.org}


\def\leftkol{ENGLISH ABSTRACTS}

\def\rightkol{ENGLISH ABSTRACTS}

\titele{\tit}{\aut}{\auf}{\leftkol}{\rightkol}

%\vspace*{-2pt}

\noindent
An agent system that reflects development of the
country economy (building of new factories, raising of the overall level of income) based on 
the model of territorial system is considered. The elements of the model are
presented in the form of agents, autonomous units capable to interact with each other. 


%\vspace*{-6pt}

\KWN{agent negotiation algorithms; territorial system; simulation; economic zoning}

%\vskip 18pt plus 6pt minus 3pt

 \vskip 14pt plus 6pt minus 3pt

%8
\def\tit{TWO MODELS OF RESOURCE ALLOCATION UNDER~THE~ORGANIZATION OF~INVESTMENT PROCESSES}

\def\aut{P.\,V.~Demin}

\def\auf{State Educational Institution ``Moscow Academy of the Labor
Market and Information Technology,''\\
pdemin@mail.ru}


\titele{\tit}{\aut}{\auf}{\leftkol}{\rightkol}

%\vspace*{-2pt}

\noindent 
Two examples of solving the problems
that arise during the organization of the investment processes which are associated
with the modernization of economy are considered. The first one relates with
the problems while choosing for financing a project among the variety.
The second one relates with the problem of distribution of the investment resources
among the enterprises that are a part of the holding.

\KWN{investment process; innovation; bank financing;
resources; holding}

  \vskip 14pt plus 6pt minus 3pt

%9
\def\tit{IRIS IMAGES COMPARISON ALGORITHM BASED ON IRIS KEY POINTS}

\def\aut{E.~Pavelyeva$^1$ and A.~Krylov$^2$}

\def\auf{$^1$Faculty of Computational Mathematics and Cybernetics, M.\,V.~Lomonosov Moscow State University,\\
$\hphantom{^1}$paveljeva@yandex.ru\\[1pt]
$^2$Laboratory of Mathematical Methods of Image Processing, 
Faculty of Computational Mathematics and\\
$\hphantom{^1}$Cybernetics, M.\,V.~Lomonosov Moscow State University, kryl@cs.msu.ru}

\titele{\tit}{\aut}{\auf}{\leftkol}{\rightkol}

%\vspace*{-2pt}

\noindent
Iris images comparison algorithm using iris key points is proposed. 
As the iris key points,  the points with the maximal convolution values in the 
Hermite transform are taken. Only iris regions free of glares, eyelashes, and eyelids are analyzed 
to increase the effectiveness of the algorithm. The iris rotation angle estimation method 
is also proposed. The proposed algorithm was tested with the public iris images database 
CASIA-IrisV3.


\KWN{biometrics; iris recognition; Hermite transform; iris key points}



\vskip 14pt plus 6pt minus 3pt


%10
\def\tit{ALGORITHM OF AUTOMATIC FACE DETECTION IN~THERMAL IMAGES
}

\def\aut{N.~Basha$^1$ and L.~Shulga$^2$}

\def\auf{$^1$Institute of Applied Acoustics, IUNSM ``Dubna,'' natalia.basha@niipa.ru \\[1pt]
$^2$Institute of Applied Acoustics, luda.shulga@niipa.ru}

\titele{\tit}{\aut}{\auf}{\leftkol}{\rightkol}

%\vspace*{-6pt}

\noindent
An approach for thermal images analysis of intelligent video surveillance systems
has been suggested. An automatic face detection 
algorithm in thermal images is proposed. In order to evaluate the performance of the algorithm,
 the experiments were carried out on the private database of 103~thermal images of 15~subjects 
 obtained under different environment conditions. The experimental results reveal good performance 
 of the proposed algorithm.

%\vspace*{-2pt}

\KWN{pattern recognition; image analysis; video surveillance; thermography; face detection}
%\pagebreak

\vskip 14pt plus 6pt minus 3pt


%11
\def\tit{ON A REFINEMENT OF CERTAIN RESULTS FOR A BAYESIAN QUEUING MODEL}

\def\aut{A.\,A.~Kudriavtsev$^1$ and S.\,Ya.~Shorgin$^2$}

\def\auf{$^1$Faculty of Computational Mathematics and Cybernetics, M.\,V.~Lomonosov Moscow State University,\\
$\hphantom{^1}$nubigena@hotmail.com\\[1pt]
$^2$IPI RAN, sshorgin@ipiran.ru}


%\def\leftkol{ENGLISH ABSTRACTS}

%\def\rightkol{ENGLISH ABSTRACTS}

\titele{\tit}{\aut}{\auf}{\leftkol}{\rightkol}

%\vspace*{-2pt}
\noindent
The paper relates to an essential refinement of published earlier formula
for distribution of nonloss probability in a Bayesian reliability model. The paper includes the correct formula for the
probability density and improved formulae of the first two moments of mentioned distribution.

%\vspace*{-5pt}

\KWN{Bayesian approach; queueing systems; reliability;
mixed distributions; modeling}


%\pagebreak
 \label{end\stat}
 