
\newcommand{\bet}{\beta_{2+\delta}}
\newcommand{\lowaexK}{{\underline{K_{\textsc{ап}}}}}% lower asymptotically exact constant
\newcommand{\lowaex}{{\underline{C_{\textsc{ап}}}}}% lower asymptotically exact constant
\newcommand{\upaex}{\overline{C}_{\textsc{ап}}} % upper asymptotically exact constant

\def\stat{nefedova}
\label{nefedova1}

\def\tit{О ТОЧНОСТИ НОРМАЛЬНОЙ АППРОКСИМАЦИИ ДЛЯ~РАСПРЕДЕЛЕНИЙ ПУАССОНОВСКИХ СЛУЧАЙНЫХ
СУММ$^*$}

\def\titkol{О точности нормальной аппроксимации для распределений пуассоновских случайных
сумм}

\def\autkol{Ю.\,С.~Нефедова, И.\,Г.~Шевцова}
\def\aut{Ю.\,С.~Нефедова$^1$, И.\,Г.~Шевцова$^2$}

\titel{\tit}{\aut}{\autkol}{\titkol}

{\renewcommand{\thefootnote}{\fnsymbol{footnote}}\footnotetext[1]
{Работа поддержана Российским фондом фундаментальных
исследований (проекты 08-01-00563, 08-01-00567, 08-07-00152 и
09-07-12032-офи-м), а также Министерством образования и науки РФ
(грант МК-581.2010.1, государственные контракты П1181, П958, П779 и
16.740.11.0133 в рамках ФЦП <<На\-уч\-ные и на\-уч\-но-пе\-да\-го\-ги\-че\-ские кадры
инновационной России на 2009--2013~годы>>).}}

\renewcommand{\thefootnote}{\arabic{footnote}}
\footnotetext[1]{Московский
государственный университет им.\ М.\,В.~Ломоносова, факультет
вычислительной математики и кибернетики, julia\_n@inbox.ru}
\footnotetext[2]{Московский государственный университет им.\
М.\,В.~Ломоносова, факультет вычислительной математики и кибернетики,
ishevtsova@cs.msu.su}


\Abst{Построены двусторонние оценки для константы в
неравенстве Бер\-ри--Эс\-се\-ена для пуассоновских случайных сумм
независимых одинаково распределенных случайных величин с конечными
моментами порядка $2+\delta,$ где $\delta\in (0\,,1].$ Нижние оценки
получены впервые. Для случая $0<\delta<1$ уточнены верхние оценки
и доказаны неравномерные оценки.}

\KW{центральная предельная теорема;
пуассоновские случайные суммы; неравенство Берри--Эссеена;
абсолютная постоянная; неравномерные оценки}

      \vskip 20pt plus 9pt minus 6pt

      \thispagestyle{headings}

      \begin{multicols}{2}
      
            \label{st\stat}

  \section{Введение}
  
  Пусть $X_1, X_2, \ldots$~--- последовательность независимых
одинаково распределенных случайных величин таких, что
\begin{equation}
{\e}X_1\equiv\mu\,, \enskip {\D}X_1 \equiv \sigma^2>0\,, \enskip
{\e}|X_1|^{3} \equiv \beta_{3}<\infty\,. \label{e1-shev}
\end{equation}
Обозначим $\F_3$ класс всех функций распределения $F$ случайной
величины $X_1$, для которых справедливы условия (1). Пусть
$N_\lambda$ -- случайная величина, имеющая распределение Пуассона с
параметром $\lambda>0.$ Предположим, что при каждом $\lambda>0$
случайные величины $N_\lambda, X_1, X_2,\ldots$ независимы.
  Рассмотрим пуассоновскую случайную сумму
$$
S_{\lambda} = X_1+\cdots+ X_{N_\lambda}\,.
$$
Для определенности полагаем, что $S_\lambda = 0$ при $N_\lambda =
0.$ Несложно видеть, что
$$
{\e}S_\lambda = \lambda\mu\,, \quad {\D}S_\lambda = \lambda(\mu^2+\sigma^2)\,.
$$
Функцию распределения стандартизованной пуассоновской случайной
суммы
$$
\widetilde{S}_\lambda \equiv\fr{S_\lambda-\lambda\mu}{\sqrt{\lambda(\mu^2+\sigma^2)}}
$$
обозначим $F_\lambda(x).$ Всюду далее для определенности полагаем,
что функция распределения непрерывна слева.

\columnbreak

Задаче изучения точности нормальной ап\-прок\-си\-ма\-ции для
распределений пуассоновских слу\-чайных сумм, так называемых
обобщенных пуассоновских распределений, посвящена обширная\linebreak
литература (см., например, библиографию в книгах~\cite{BeningKorolev2002, KBS2007}). Большой интерес к данной задаче
обуслов\-лен тем, что пуассоновские случайные суммы являются
<<накопленными>> значениями маркированного пуассоновского
процесса, который, как отмечено в указанных книгах, может быть
интерпретирован как абсолютно хаотическое случайное блуждание с
дискретным временем. Подобные модели традиционно широко
используются в теории массового обслуживания при анализе
информационных и телекоммуникационных систем, в теории управ\-ле\-ния
запасами, страховой математике и других областях.

Известно, что для приведенных выше условий~(\ref{e1-shev}) на моменты
случайной величины~$X_1$ справедливо неравенство Бер\-ри--Эс\-се\-ена
для пуассоновских случайных сумм: существует абсолютная
положительная и конечная постоянная~$C$ такая, что
\begin{equation}
\rho(F_\lambda,\Phi) \equiv \sup_x\left|F_\lambda(x) -
\Phi(x)\right|\le C  L_\lambda^3\,, 
\label{e2-shev}
\end{equation}
где $\Phi(\cdot)$~--- функция стандартного нормального распределения; $L_\lambda^3$~---
нецентральная ляпуновская дробь:
$$
L_\lambda^3=\fr{\beta_3}{(\mu^2+\sigma^2)^{3/2}\sqrt{\lambda}}\,.
$$


  Неравенство~(\ref{e2-shev}) имеет интересную историю. По-видимому, впервые
это неравенство было доказано в работе~\cite{R1972} и опубликовано
в статье~\cite{R1976} с $C =2{{,}}23$ (диссертация~\cite{R1972} не
опубликована, в то время как в статье~\cite{R1976} не было
приведено доказательство этого результата). Позднее с
использованием традиционной техники, основанной на неравенстве
сглаживания Эссеена, эта оценка была доказана в работе~\cite{vonChossyRappl1983} с $C = 2{,}21.$

  В работе~\cite{Michel1993} с использованием свойства безграничной
делимости обобщенных пуассоновских распределений и оценки абсолютной
постоянной в классическом неравенстве Бер\-ри--Эс\-се\-ена для сумм
неслучайного числа независимых случайных величин, полученной Ван
Биком~\cite{vanBeek1972}, было показано, что в~(\ref{e2-shev}) $C\le 0{,}8.$
Не будучи знакомыми с этой работой Михеля, авторы статьи~\cite{BeningKorolevShorgin1997}, 
применив уточненное неравенство
сглаживания Эссеена, получили оценку $C\le 1{,}99$. Из метода
доказательства, использованного в работе Михеля, вытекает, что если
для абсолютной постоянной~$C_0$ в классическом неравенстве
Бер\-ри--Эс\-се\-ена известна оценка $C_0\le M,$ то неравенство~(\ref{e2-shev})
справедливо с $C=M$. На это обстоятельство также обратили внимание
авторы работы~\cite{KorolevShorgin1997}, в которой независимо от~\cite{Michel1993} 
получен тот же результат, но с другой, лучшей на
тот момент времени, текущей оценкой $M=0{,}7655.$

  Как показано в работах~\cite{KSOPPM2010, KSSAJ2010}, наилучшая на
сегодняшний день оценка абсолютной постоянной в классическом
неравенстве Бер\-ри--Эс\-се\-ена имеет вид $C_0\le 0{,}4784.$ Поэтому,
следуя логике авторов работ~\cite{Michel1993, KorolevShorgin1997}, 
можно заключить, что неравенство~(\ref{e2-shev}) справедливо с $C = 0{,}4784.$

  Однако в тех же работах~[10, 11] показано, что на самом деле привязка оценки
константы~$C$ в~(\ref{e2-shev}) к оценке абсолютной постоянной в
классическом неравенстве Бер\-ри--Эс\-се\-ена $C_0$ менее жесткая. 
А~именно, несмотря на то что, как уже говорилось, наилучшая на
сегодняшний день верхняя граница для $C_0$ равна $0{,}4784,$
неравенство~(\ref{e2-shev}) справедливо с $C=0{,}3041$~\cite{KSOPPM2010, KSSAJ2010}.

  Тем не менее, несмотря на более чем тридцатилетнюю историю
существования неравенства~(\ref{e2-shev}), нижние оценки для~$C$ пока
найдены не были, и в этой работе приводятся впервые.
  Кроме того, будут построены неравномерные оценки, уточ\-ня\-ющие оценки
Михеля~\cite{Michel1993} для случая существования третьих
моментов слагаемых, и впервые приведены неравномерные оценки для
случая существования моментов порядка, меньшего трех. При этом
попутно уточняются приведенные в работах~\cite{Michel1981, PaditzTysiak1990} константы в неравномерных оценках точности
нормальной аппроксимации для распределений сумм детерминированного
числа слагаемых.

  \section{Нижняя оценка для~абсолютной константы в~неравенстве
Берри--Эссеена для~пуассоновских случайных сумм}

  В терминах, введенных в работе~\cite{S2010}, определим верхнюю
асимптотически правильную постоянную
$$
\upaex = \limsup\limits_{\lambda\rightarrow\infty}\sup_{F\in\F_3}
\fr{\rho(F_\lambda,\Phi)}{L_\lambda^3}\,.
$$
Символом $\stackrel{d}{=}$ будет обозначаться
совпадение распределений.

  \medskip
  
  \noindent
  \textbf{Теорема 1.} \textit{ Для константы $C$ в неравенстве}~(\ref{e2-shev})
\textit{справедлива оценка} $$C\ge \upaex \ge
\fr{1}{2}\sup\limits_{\gamma>0}\sqrt{\gamma}\,e^{-\gamma}I_0(\gamma)
= 0{,}2344\ldots,$$ \textit{где $I_0(\gamma)$~--- модифицированная
функция Бесселя нулевого порядка}
$$
I_0(\gamma) =
\sum\limits_{k=0}^\infty\fr{({\gamma}/{2})^{2k}}{(k!)^2}\,.
$$

  \medskip

\noindent
 Д\,о\,к\,а\,з\,а\,т\,е\,л\,ь\,с\,т\,в\,о\,.\ Рассмотрим случайную величину $X_1$:
\begin{align*}
{\sf P}(X_1=-1) &={\sf P} (X_1=1)=p\,,\\
{\sf P}(X_1=0)&=1-2p\,,\qquad 0<p\le \fr{1}{2}\,.
\end{align*}
Легко видеть, что
\begin{gather*}
\mu\equiv {\e}X_1 = 0\,, \enskip \sigma^2\equiv{\D}X_1 = 2p\,, \enskip
\beta_3 \equiv{\e}|X_1|^3 = 2p\,,
\\
L_\lambda^3 \equiv\fr{\beta_3}{\sigma^3\sqrt{\lambda}} =
\fr{1}{\sqrt{2p\lambda}}\,.
\end{gather*}
  Очевидно, что для верхней асимптотически правильной постоянной
$\upaex$ справедлива оценка
\begin{equation}
\upaex\ge
\limsup\limits_{\lambda\rightarrow\infty}\sup\limits_{0<p\le 1/2}\sqrt{2\lambda
p}\,\rho(F_\lambda,\Phi)\,. 
\label{e3-shev}
\end{equation}
  Функцию распределения~$F_{\lambda}(x)$ пуассоновской случайной
суммы~$\widetilde{S}_\lambda$ с использованием формулы полной
вероятности можно представить в виде:
$$
F_{\lambda}(x) = \sum\limits_{n=0}^\infty \fr{\lambda^n
e^{-\lambda}}{n!} \,F_n(x)\,,
$$
где $F_n(x)$~--- функция распределения $S_n = (X_1\hm + \cdots+
X_n)/\sqrt{2\lambda p},$ а $F_0(x)$~--- функция распределения с
единичным скачком в нуле.
  В силу симметрии рассматриваемого трехточечного распределения~$X_1$ 
  справедливо соотношение $S_n \stackrel{d}{=} -S_n$, а
следовательно, для всех $n\ge1$
$$
F_n(0) = {\p}(S_n<0) = \fr{1 - {\p}(S_n=0)}{2}\,,\quad F_0(0) =
0\,.$$
  Учитывая вышесказанное, для $\rho(F_\lambda,\Phi)$ получаем:
\begin{multline*}
\rho(F_\lambda,\Phi) \equiv \sup\limits_x \left|F_\lambda(x) -
\Phi(x)\right|\ge{}\\
{}\ge \left|F_\lambda(0) - \Phi(0)\right| =
\left\vert\sum\limits_{n=0}^\infty \fr{\lambda^n e^{-\lambda}}{n!}
F_n(0) - \fr{1}{2}\right\vert = {}\\
{}
= \left\vert\sum\limits_{n=0}^\infty \fr{\lambda^n e^{-\lambda}}{n!}
\left(F_n(0) - \fr{1}{2}\right)\right\vert
={}\\
{}=\bigg|e^{-\lambda}\left(F_0(0) -
\fr{1}{2}\right)+\sum\limits_{n=1}^\infty \fr{\lambda^n
e^{-\lambda}}{n!}\left( F_n(0) - \fr{1}{2}\right)\bigg| = {}\\
{}
= \bigg|\fr{-e^{-\lambda}}{2}-\sum\limits_{n=1}^\infty
\fr{\lambda^n e^{-\lambda}}{n!}\frac{{\p}( S_n=0)}{2}\bigg|{}=\\
{}=\fr{1}{2}\bigg(e^{-\lambda}+\sum\limits_{n=1}^\infty
\fr{\lambda^n e^{-\lambda}}{n!}{\p}( S_n=0)\bigg) = %{}\\
%{}=
\fr{1}{2}\bigg(e^{-\lambda}+{}\\
{}+\sum\limits_{n=1}^\infty
\fr{\lambda^n e^{-\lambda}}{n!}\sum_{k=0}^{\lfloor
n/2\rfloor}\fr{n!}{k!k!(n-2k)!}p^{2k}(1-2p)^{n-2k}\bigg) = {}\\
{}= 
\fr{1}{2}\bigg(e^{-\lambda}+\sum\limits_{n=1}^\infty
\fr{\lambda^n e^{-\lambda}}{n!}(1-2p)^n+{}\\
{}+\sum\limits_{n=2}^\infty
\fr{\lambda^n e^{-\lambda}}{(n-2)!}\,\fr{p^2(1-2p)^{n-2}}{1!1!}+{}\\
{}+
\sum\limits_{n=4}^\infty \fr{\lambda^n
e^{-\lambda}}{(n-4)!}\,\fr{p^4(1-2p)^{n-4}}{2!2!} + \ldots\bigg)
={}\\
{}
= \fr{e^{-\lambda}}{2}\bigg(\sum\limits_{n=0}^\infty
\fr{(\lambda(1-2p))^n }{n!}+{}\\
{}+\fr{(\lambda
p)}{1!1!}^2\sum\limits_{n=2}^\infty
\fr{(\lambda(1-2p))^{n-2}}{(n-2)!} +{}\\
{}
+ \fr{(\lambda p)^4}{2!2!}\sum\limits_{n=4}^\infty
\fr{(\lambda(1-2p))^{n-4} }{(n-4)!} + \ldots\bigg) ={}\\
{}=\fr{e^{-\lambda}}{2}\sum_{k=0}^\infty \frac{(\lambda p)^{2k}}{(k!)^2}e^{\lambda(1-2p)} = 
\fr{1}{2}\,e^{-2\lambda p}\sum_{k=0}^\infty 
\fr{(\lambda p)^{2k}}{(k!)^2}\,.
\end{multline*}
  Следовательно, можно продолжить цепочку неравенств~(\ref{e3-shev}) следующим
образом:

\noindent
\begin{multline*}
\upaex\ge \limsup\limits_{\lambda\rightarrow\infty}\sup\limits_{0<p\le {1}/{2}}
\sqrt{2\lambda p}\,\rho(F_\lambda,\Phi)\ge{}\\
{}\ge
 \limsup\limits_{\lambda\rightarrow\infty}\sup\limits_{0<p\le 1/2}
\fr{1}{2}\,e^{-2\lambda p}\sqrt{2\lambda p}\,\sum_{k=0}^\infty \fr{(\lambda p)^{2k}}{(k!)^2}\,.
\end{multline*}
Пусть $p = {\gamma}/({2\lambda})$, $0<\gamma\le \lambda$, тогда
окончательно получаем
\begin{multline*}
\upaex\ge
\limsup\limits_{\lambda\rightarrow\infty}\sup_{0<\gamma\le
\lambda}\fr{1}{2}\,e^{-\gamma}\sqrt{\gamma}\,\sum_{k=0}^\infty
\fr{({\gamma}/{2})^{2k}}{(k!)^2} ={}\\
{}=
\fr{1}{2}\,\sup\limits_{\gamma>0}\sqrt{\gamma}\,e^{-\gamma}I_0(\gamma)
= 0{,}2344\ldots\,,
\end{multline*}
причем супремум достигается в точке $\gamma\approx0{,}79.$ Тео\-ре\-ма
доказана.


\section{Аналог неравенства Берри--Эссеена для~пуассоновских случайных сумм слагаемых с моментами порядка 
$2+\delta$}

Теперь предположим, что последовательность независимых одинаково
распределенных случайных величин $X_1, X_2, \ldots$ удовлетворяет
следующим моментным условиям:
$$
\left.
\begin{array}{c}
{\e}X_1\equiv\mu\,, \enskip {\D}X_1 \equiv \sigma^2>0\,; 
\\[9pt]
{\e}|X_1|^{2+\delta} \equiv \beta_{2+\delta}<\infty\!
\end{array}
\right \}
\eqno(1^\ast)
$$
с некоторым $\delta\in(0,1).$ Обозначим $\F_{2+\delta}$ множество
всех функций распределения~$F$ случайной величины~$X_1$,
удовлетворяющих условиям~(1$^*$).  Нецентральная ляпуновская дробь в
этом случае определяется выражением:
$$
L_\lambda^{2+\delta}=\fr{\bet}{(\mu^2+\sigma^2)^{1+\delta/2}
\lambda^{\delta/2}}\,.
$$

  \subsection{Вспомогательные результаты}
  
  \noindent
  \textbf{Лемма 1.} \textit{Пусть $\xi_1, \xi_2,\ldots$~--- независимые
одинаково распределенные случайные величины с нулевым средним,
единичной дисперсией и конечным абсолютным моментом порядка
$2+\delta.$ Тогда существует конечная положительная абсолютная
постоянная~$C(\delta)$ такая, что для всех $n\ge1$ справедливо
неравенство:}
\begin{multline*}
\sup_x\left|{\p}\left(\fr{\xi_1+\cdots+\xi_n}{\sqrt{n}}<x\right)
- \Phi(x)\right|\le{}\\
{}\le
\fr{C(\delta)({\e}|\xi_1|^{2+\delta}+1)}{n^{\delta/2}}\,,
\end{multline*}
%\pagebreak

%\vspace*{9pt}

\noindent
\begin{center}
\parbox{50mm}{{{\tablename~1}\ \ \small{Двусторонние оценки $C(\delta)$ для некоторых
$\delta$}}

}

\vspace*{6pt}

{\small 
\tabcolsep=12pt
\begin{tabular}{|c|c|c|}
  \hline 
  $\delta$ & $C(\delta)\le$ & $C(\delta)\ge$ \\
\hline
1{,}0 & 0{,}3041 & 0{,}2344\\
0{,}9 & 0{,}3089 & 0{,}2383\\
0{,}8 & 0{,}3187 & 0{,}2446\\
0{,}7 & 0{,}3334 & 0{,}2534\\
0{,}6 & 0{,}3538 & 0{,}2651\\
0{,}5 & 0{,}3775 & 0{,}2803\\
0{,}4 & 0{,}4080 & 0{,}3000\\
0{,}3 & 0{,}4450 & 0{,}3257\\
0{,}2 & 0{,}4901 & 0{,}3603\\
0{,}1 & 0{,}5451 & 0{,}4097\\
\hline
\end{tabular}
}
\end{center}
\vspace*{9pt}

\bigskip
\addtocounter{table}{1}

\noindent
\textit{причем для константы $C(\delta)$ справедливы верхние оценки,
приведенные в табл.}~1.



  \medskip

\noindent
Д\,о\,к\,а\,з\,а\,т\,е\,л\,ь\,с\,т\,в\,о\ содержится в работе~\cite{GS2010} для случая 
$\delta\in(0,1)$ и~\cite{KSOPPM2010,KSSAJ2010} для $\delta=1$.
  %\medskip
  %Значения верхних оценок константы $C(\delta)$, полученные в работе
%\cite{GS2010} для некоторых $\delta$, приведены в таблице 1 в
%третьей строке.

  \medskip

  Обозначим $$\nu = \fr{\lambda}{n}\,.$$

  \medskip

\noindent
\textbf{Лемма 2} (см.~\cite{KBS2007, S2007}). \textit{При условиях
$(1^\ast)$ для любого натурального $n\ge1$}
$$
\widetilde{S}_\lambda \stackrel{d}{=}
\fr{1}{\sqrt{n}}\,\sum_{k=1}^n Z_{\nu,k}\,,
$$
\textit{где при каждом $n$ случайные величины $Z_{\nu,1},\ldots,
Z_{\nu,n}$ независимы и одинаково распределены. Более того, $\e
Z_{\nu,1}=0$, $\e Z_{\nu,1}^2=1$ и при всех $n\ge\lambda$}
\begin{equation}
{\e}|Z_{\nu,1}|^{2+\delta}\le
\fr{\beta_{2+\delta}(1+40\nu)}{(\mu^2+\sigma^2)^{1+\delta/2}}
\left(\fr{n}{\lambda}\right)^{\delta/2}\,. 
\label{e4-shev}
\end{equation}

  \subsection{Верхние оценки}
  
  В данном разделе доказывается аналог неравенства Бер\-ри--Эс\-се\-ена для
пуассоновских случайных сумм слагаемых с моментами порядка
$2+\delta$. Аналогичный результат получен в работе~\cite{S2007},
однако он справедлив лишь в асимптотическом смысле, когда
ляпуновская дробь $L_\lambda^{2+\delta}$ бесконечно мала. А~именно,
в~\cite{S2007} была найдена мажоранта не {\it абсолютной}, но {\it
асимптотически правильной} константы:
$$
\limsup\limits_{\ell\rightarrow0}\sup_{\lambda,F\colon
L_\lambda^{2+\delta}=\ell} \fr{\rho(F_\lambda,\Phi)}{\ell}\,.
$$
Следующее утверждение исправляет неточность, допущенную в одном из
результатов упомянутой работы и повлиявшую на окончательный
результат.

  \medskip

\noindent
\textbf{Теорема 2.} \textit{При условиях $(1^\ast)$ для любого
$\lambda>0$ справедливо неравенство:}
\begin{multline}
\rho(F_\lambda,\Phi)\equiv \sup_x|F_\lambda(x) -
\Phi(x)|\le{}\\
{}\le
\fr{C(\delta)\beta_{2+\delta}}
{(\mu^2+\sigma^2)^{1+\delta/2}\lambda^{\delta/2}}\,, 
\label{e5-shev}
\end{multline}
\textit{где $C(\delta)$ та же, что и в лемме~$1$.}

  \medskip

  \noindent
  Д\,о\,к\,а\,з\,а\,т\,е\,л\,ь\,с\,т\,в\,о\,.\ Из леммы~2 вытекает,
что для любого целого $n\ge1$
$$
\rho(F_\lambda,\Phi) =
\sup_x\bigg|{\p}\bigg(\fr{1}{\sqrt{n}}\,\sum_{k=1}^n
Z_{\nu,k}<x\bigg)-\Phi(x)\bigg|\,.
$$ 
Следовательно, по лемме~1 для произвольного целого $n\ge1$ имеем:
\begin{equation}
\rho(F_\lambda,\Phi) \le
C(\delta)\fr{{\e}|Z_{\nu,1}|^{2+\delta}}
{n^{\delta/2}}+\fr{C(\delta)}{n^{\delta/2}}\,. \label{e6-shev}
\end{equation} 
Пусть теперь $n\ge\lambda$. Тогда, используя оценку~(\ref{e4-shev}), 
в продолжение~(\ref{e6-shev}) получаем неравенство:
$$
\rho(F_\lambda,\Phi)\le C(\delta) 
\fr{\beta_{2+\delta}(1+40{\lambda}/{n})}{(\mu^2+\sigma^2)^{1+\delta/2}\lambda^{\delta/2}}+
\fr{C(\delta)}{n^{\delta/2}}\,.
$$
Так как здесь $n\ge\lambda$ произвольно, устремляя
$n\rightarrow\infty,$ окончательно получаем:
\begin{multline*}
\rho(F_\lambda,\Phi)\le \lim_{n\rightarrow\infty}\left[C(\delta)
\fr{\beta_{2+\delta}(1+40{\lambda}/n)}{(\mu^2+\sigma^2)^{1+\delta/2}\lambda^{\delta/2}}+{}\right.\\
\left.{}+
\fr{C(\delta)}{n^{\delta/2}}\right] = 
%{}=
\fr{C(\delta)\beta_{2+\delta}}{(\mu^2+\sigma^2)^{1+\delta/2}\lambda^{\delta/2}}\,,
\end{multline*}
что и требовалось доказать.

  \subsection{Нижние оценки}

  В терминах, введенных в работе~\cite{S2010}, определим верхнюю
асимптотически правильную постоянную для $\delta\in (0,1)$:
$$
\upaex(\delta) = \limsup\limits_{\lambda\rightarrow\infty}
\sup_{F\in\F_{2+\delta}}\fr{\rho(F_\lambda,\Phi)}{L_\lambda^{2+\delta}}\,.
$$

  \medskip

\noindent
  \textbf{Теорема 3.} \textit{ Для константы $C(\delta)$ в
неравенстве~$(5)$ справедлива нижняя оценка}
$$C(\delta)\ge \upaex(\delta) \ge
\fr{1}{2}\,\sup\limits_{\gamma>0}\gamma^{\delta/2}e^{-\gamma}I_0(\gamma)\,,
$$
\textit{где $I_0(\gamma)$~--- модифицированная функция Бесселя}
$$
I_0(\gamma) = \sum\limits_{k=0}^\infty\fr{(\gamma/2)^{2k}}{(k!)^2}\,.
$$

  \medskip

\noindent
Д\,о\,к\,а\,з\,а\,т\,е\,л\,ь\,с\,т\,в\,о\,\ аналогично доказательству тео\-ре\-мы~1.

  \medskip

  Конкретные значения минорант~$C(\delta)$ для некоторых~$\delta$
приведены в табл.~1 в третьем столбце.


\section{Неравномерные оценки точности нормальной аппроксимации
для~обобщенных пуассоновских~распределений}

Данный раздел предварим одним вспомогательным утверждением,
устанавливающим неравномерную оценку скорости сходимости для сумм
детерминированного числа случайных слагаемых.

Пусть $\delta\in(0,\,1]$. Обозначим $S_n = X_1+\cdots+X_n,$
\begin{multline*}
G_n(x) = {\p}\left(\fr{S_n-n\mu}{\sigma\sqrt{n}}<x\right)=
F^{*n}(n\mu+\sigma x\sqrt{n})\,,\\ x\in \mathbb{R}\,.
\end{multline*}

  \medskip

\noindent
  \textbf{Лемма 3.} \textit{Пусть выполнены условия $(1^\ast)$.
Тогда для произвольного целого $n\ge1$ справедлива неравномерная
оценка:}
$$ %\begin{multline*}
\left|G_n(x) - \Phi(x)\right|\le
\fr{K(\delta)}{n^{\delta/2}}\fr{{\e}|X_1 -
\mu|^{2+\delta}}{\sigma^{2+\delta}(1+|x|^{2+\delta})}\,,\enskip
x\in\r\,,
$$ %\end{multline*} 
\textit{где $K(\delta)$ зависит только от $\delta$.}

  \smallskip

  Данное утверждение доказано Нагаевым~\cite{N1965} для случая
$\delta=1$ и Бикялисом~\cite{B1966} для случая произвольного
$\delta\in(0,1]$.

  \medskip

  Используя метод вычисления абсолютной константы в
неравномерном аналоге неравенства Бер\-ри--Эс\-се\-ена (неравенстве
На\-га\-ева--Би\-кя\-ли\-са, см.\ лемму~3), описанный
Падицем~\cite{Paditz1989}, с учетом новых оценок для
абсолютной константы~$C_0(\delta)$ в классическом неравенстве
Бер\-ри--Эс\-се\-ена, полученных в  работах~\cite{KSOPPM2010, KSSAJ2010}
для $\delta=1$ ($C_0(1)\le0{,}4784$) и в~\cite{GS2010} для
$0<\delta<1$ ($C_0(\delta)$ в данной статье не приводится), можно
уточнить верхние оценки константы $K(\delta)$ в
 лемме~3,
полученные в работе~\cite{PaditzTysiak1990} и процитированные в
статье~\cite{Paditz1996} и книге~\cite{KBS2007}. Полученные таким\linebreak
\noindent
\begin{center}
\parbox{50mm}{{{\tablename~2}\ \ \small{Двусторонние оценки $K(\delta)$ для некоторых
$\delta$}}

}

\vspace*{6pt}

{\small 
\tabcolsep=12pt
\begin{tabular}{|c|c|c|}
\hline 
  $\delta$ & $K(\delta)\le$ & $K(\delta)\ge$ \\
\hline
1{,}0 & 25{,}7984 & 0{,}0177\\
0{,}9 & 24{,}2210 & 0{,}0198\\
0{,}8 & 22{,}4063 & 0{,}0223\\
0{,}7 & 20{,}6726 & 0{,}0253\\
0{,}6 & 19{,}0089 & 0{,}0290\\
0{,}5 & 17{,}3674 & 0{,}0334\\
0{,}4 & 15{,}6802 & 0{,}0390\\
0{,}3 & 14{,}0732 & 0{,}0459\\
0{,}2 & 12{,}6421 & 0{,}0550\\
0{,}1 & 11{,}3653 & 0{,}0674\\
\hline
\end{tabular}
}
\end{center}
\vspace*{-6pt}
\columnbreak


%\addtocounter{table}{1}

\noindent
образом верхние оценки константы~$K(\delta)$ приведены в табл.~2
для некоторых~$\delta$.


  \medskip

\noindent
\textbf{Замечание~1.} Оценка $K(1)\le 25{,}80$, приведенная в
табл.~2, уточняет оценку $K(1)\le 30{,}54$ из работы~\cite{Michel1981}.

\noindent
\textbf{Замечание~2.} Нижнюю оценку для константы $K(\delta)$ легко
получить из нижней оценки для $\sup\limits_x|G_n(x)-\Phi(x)|$, доказанной
одним из авторов данной работы в~\cite{S2010} и использованной для
построения миноранты нижней асимптотически\linebreak правильной постоянной в
оценках равномерной мет\-ри\-ки (неравенстве Бер\-ри--Эс\-се\-ена). Введем
нижнюю асимптотически правильную постоянную в неравномерной оценке
(неравенстве На\-га\-ева--Би\-кя\-лиса):
\begin{multline*}
\lowaexK(\delta)= \limsup_{\ell\to0}
\limsup_{n\to\infty}\sup_{F\in\F_{2+\delta}} \sup_{x\in\mathbb{R}}
(1+|x|^{2+\delta})\times{}\\
\times\fr{\vert F^{*n}(n\mu+\sigma x\sqrt{n})-\Phi(x)\vert }{\ell}\,,
\end{multline*}
где супремум берется по всем $F\in\F_{2+\delta}$ с фиксированным
значением (центральной) ляпуновской дроби
$$
\fr{\e|X_1-\mu|^{2+\delta}}{\sigma^{2+\delta}n^{\delta/2}}=\ell\,.
$$
Тогда для константы $K(\delta)$ получаем:
\begin{multline*}
K(\delta)\ge \lowaexK(\delta)\ge{}\\
{}\ge \limsup_{\ell\to0}
\limsup_{n\to\infty}\sup_{F\in\F_{2+\delta}} \sup_{x\in\mathbb{R}}
\fr{1}{\ell}\times{}\\
{}\times{|F^{*n}(n\mu+\sigma x\sqrt{n})-\Phi(x)|} \ge{}\\
{}
\ge\sup_{h\ge0,\,s>0} 
\left(
\fr{4}{\sqrt{2+s^2}}\,\exp \left
\{-\fr{h^2}{2(2+s^2)}\right\}+{}\right.\\
\left.{}+
\fr{h^2+s^2}{\sqrt{2}}-2\sqrt{2}\right)\Bigg / 
\left(\vphantom{\fr{h^2}{2s^2}}
8\varkappa_{2+\delta}\,
s^{2+\delta}\times{}\right.\\
\left.{}\times e^{-h^2/(2s^2)}
{_1F_1}\left(\fr{3+\delta}{2},\fr{1}{2},\fr{h^2}{2s^2}\right)\right)\,,
\end{multline*}
где ${_1F_1}$~--- обобщенная гипергеометрическая функция
(вырожденная функция Мейера):
\begin{multline*}
{_1F_1}(a,b,z)=\fr{\Gamma(b)}{\Gamma(a)}\,\sum_{k=0}^\infty
\fr{\Gamma(a+k)}{\Gamma(b+k)}\fr{z^k}{k!}\,,\\ z>0\,,\quad
{_1F_1}(a,b,0)=1\,;
\end{multline*}

\noindent
$$
\varkappa_{2+\delta}\equiv
\int\limits_{-\infty}^\infty|x|^{2+\delta}\,d\Phi(x)=
(1+\delta)\Gamma\left(\fr{1+\delta}2\right)
\fr{2^{\delta/2}}{\sqrt{\pi}}\,;
$$
$\Gamma(\cdot)$~--- Эйлерова гам\-ма-функ\-ция. Конкретные значения
миноранты введенной нижней асимптотически правильной постоянной
$\lowaexK(\delta)$, а следовательно, и константы~$K(\delta)$, для
некоторых~$\delta$ приведены в табл.~2.

%\pagebreak

Вытекающее из замечания~2 утверждение о положительности нижней
асимптотически правильной постоянной $\lowaexK(\delta)$
представляет особый интерес при $0<\delta<1$, поскольку для этого
случая в работе~\cite{OsipovPetrov1967} было показано, что для
любой функции распределения $F\in\F_{2+\delta}$ найдется
ограниченная убывающая функция $\psi(u)$, $u>0$, с пределом
$\lim\limits_{u\to\infty}\psi(u)=0$ и такая, что
$$
|G_n(x)-\Phi(x)|\le
\fr{\psi(\sqrt{n}(1+|x|))}{n^{\delta/2}(1+|x|^{2+\delta})}\,,\quad
x\in\mathbb{R}\,,\ n\ge1\,,
$$
т.\,е.\ для каждого {\it фиксированного} распределения
$F\in\F_{2+\delta}$ величина $|G_n(x)-\Phi(x)|$ убывает быст\-рее,
чем $n^{\delta/2}(1+|x|^{2+\delta})$, что ставит под сомнение
<<правильность>> (точность) оценок типа неравенства
На\-га\-ева--Би\-кя\-ли\-са. Положительность же нижней асимптотической
постоянной $\lowaexK(\delta)$ означает, что неравенство
На\-га\-ева--Би\-кя\-ли\-са устанавливает <<правильный>> порядок, понимаемый
в {\it равномерном} смысле.

\smallskip

\noindent
\textbf{Теорема 4.} \textit{При условиях $(1^\ast)$ для любого
$\lambda>0$ справедливо неравенство:}
$$
|F_\lambda(x) - \Phi(x)|\le \fr{K(\delta)
L_\lambda^{2+\delta}}{1+|x|^{2+\delta}}\,,\enskip x\in\r\,,
$$
\textit{где $K(\delta)$ та же, что и в лемме~$3$.}

  \medskip

\noindent
Д\,о\,к\,а\,з\,а\,т\,е\,л\,ь\,с\,т\,в\,о\,.\ Из леммы~2 вытекает,
что для любого целого $n\ge1$
$$ %\begin{multline*}
|F_\lambda(x) - \Phi(x)| =
\left\vert{\p}\left(\fr{1}{\sqrt{n}}\,\sum_{k=1}^n
Z_{\nu,k}<x\right)-\Phi(x)\right|\,.
$$ 
Следовательно, по лемме~3 для произвольного целого $n\ge1$ имеем:
\begin{equation}
|F_\lambda(x) - \Phi(x)| \le
\fr{K(\delta)}{n^{\delta/2}}\,\fr{{\e}|Z_{\nu,1}|^{2+\delta}}{1+|x|^{2+\delta}}\,.
\label{e7-shev}
\end{equation}
Пусть теперь $n\ge\lambda.$ Тогда, используя оценку~(\ref{e4-shev}), 
в продолжение~(\ref{e7-shev}) получаем неравенство
$$ %\begin{multline*}
|F_\lambda(x) - \Phi(x)|
\le \fr{K(\delta)}{\lambda^{\delta/2}}\, 
\fr{\beta_{2+\delta}(1+40{\lambda}/n)}{(\mu^2+\sigma^2)^{1+\delta/2}(1+|x|^{2+\delta})}\,.
$$ %\end{multline*}
Так как здесь $n\ge\lambda$  произвольно, устремляя
$n\rightarrow\infty,$ окончательно получаем

\noindent
\begin{multline*}
|F_\lambda(x) - \Phi(x)|\le {}\\
{}\le \lim_{n\rightarrow\infty}\left[
\fr{K(\delta)}{\lambda^{\delta/2}}\, 
\fr{\beta_{2+\delta}(1+40{\lambda}/{n})}{(\mu^2+\sigma^2)^{1+\delta/2}(1+|x|^{2+\delta})}\right] ={}\\
{}
=\fr{K(\delta)}{\lambda^{\delta/2}}\,\fr{\beta_{2+\delta}}{(\mu^2+\sigma^2)^{1+\delta/2}(1+|x|^{2+\delta})}\,.
\end{multline*}

    \medskip

  Аналогично доказывается и более общее утверж\-де\-ние.

  \medskip

\noindent
\textbf{Теорема~5.} \textit{Предположим, что существует функция
$Q(x)$ такая, что для всех $x\in\mathbb{R}$}
\begin{multline*}
\left|F_n(x) - \Phi(x)\right| =\left| {\p}\left(\fr{S_n-n\mu}{\sigma\sqrt{n}}<x\right)- \Phi(x)\right|\le {}\\
{}\le
\fr{Q(x)}{n^{\delta/2}}\, \fr{{\e}|X_1 - \mu|^{2+\delta}}{\sigma^{2+\delta}}\,.
\end{multline*}
\textit{Тогда верна оценка:}
\begin{multline*}
\left|F_\lambda(x) - \Phi(x)\right| ={}\\
{}=\left|
{\p}\left(\fr{S_\lambda-\lambda\mu}{\sqrt{\lambda(\mu^2+\sigma^2)}}<x\right)-
\Phi(x)\right|\le{}\\
{}\le \fr{Q(x)}{\lambda^{\delta/2}}\,
\fr{{\e}|X_1|^{2+\delta}}{(\mu^2+\sigma^2)^{1+\delta/2}}\,,\enskip
x\in\mathbb{R}\,,
\end{multline*}
\textit{с той же самой $Q(x)$.}

  \bigskip
  В заключение авторы выражают признательность В.\,Ю.~Королеву за 
  стимулирующие дискуссии и постоянное внимание к работе.

{\small\frenchspacing
{%\baselineskip=10.8pt
\addcontentsline{toc}{section}{Литература}
\begin{thebibliography}{99}

\bibitem{BeningKorolev2002} %1
\Au{Bening V., Korolev V.}  Generalized Poisson models and their
applications in insurance and finance.~--- Utrecht: VSP, 2002.

  \bibitem{KBS2007} %2
\Au{Королев В.\,Ю., Бенинг В.\,Е., Шоргин~С.\,Я.} Математические
основы теории риска.~--- М.: Физматлит, 2007.

  \bibitem{R1972} %3
\Au{Ротарь Г.\,В.} Некоторые задачи планирования резерва. Дис.\ \ldots канд. физ.-мат. наук.~--- М.: Центральный
эко\-но\-ми\-ко-ма\-те\-ма\-ти\-че\-ский институт, 1972.

  \bibitem{R1976} %4
\Au{Ротарь Г.\,В.} Об одной задаче управления резервами~//
Эко\-но\-ми\-ко-ма\-те\-ма\-ти\-че\-ские методы, 1976. Т.~12. Вып.~4. С.~733--739.

  \bibitem{vonChossyRappl1983} %5
\Au{Von Chossy R., Rappl G.} Some approximation methods for the
distribution of random sums~// Insurance: Mathematics and Economics,
1983. Vol.~2. No.\,1. P.~251--270.

  \bibitem{Michel1993} %6
\textit{Michel R.} On Berry--Esseen results for the compound Poisson
distribution~// Insurance: Mathematics and Economics, 1993. Vol.~13. No.\,1. P.~35--37.

  \bibitem{vanBeek1972} %7
\Au{Van Beek P.} An application of Fourier methods to the
problem of sharpening the Berry--Esseen inequality~// Z.~Wahrsch.
verw. Geb., 1972. Bd.~23. S.~187--196.

\bibitem{BeningKorolevShorgin1997} %8
\Au{Bening V.\,E., Korolev~V.\,Yu., Shorgin~S.\,Ya.} On
approximations to generalized Poisson distribution~// J.\
Math. Sci., 1997. Vol.~83. No.\,3. P.~360--367.

  \bibitem{KorolevShorgin1997} %9
\Au{Korolev V.\,Yu., Shorgin S.\,Ya.} On the absolute constant in
the remainder term estimate in the central limit theorem for Poisson
random sums~// Probabilistic Methods in Discrete Mathematic:
4th International Petrozavodsk Conference Proceedings.~---
Utrecht: VSP, 1997. P.~305--308.

  \bibitem{KSOPPM2010} %10
\Au{Королев В.\,Ю., Шевцова И.\,Г.} Уточнение неравенства
Бер\-ри--Эс\-се\-ена с приложениями к пуассоновским и смешанным
пуассоновским случайным суммам~// Обозрение прикладной и
промышленной математики, 2010. Т.~17. Вып.~1. С.~25--56.

  \bibitem{KSSAJ2010} %11
\Au{Korolev V.\,Yu., Shevtsova~I.\,G.} An improvement of the
Berry--Esseen inequality with applications to Poisson and mixed
Poisson random sums~// Scandinavian Actuarial J., 2011 (in
press). Online first:
{\sf http://www.informaworld.com/10.1080/ 03461238.2010.485370}.

  \bibitem{Michel1981} %12
\Au{Michel R.} On the constant in the nonuniform version of the
Berry--Esseen theorem~// Z. Wahrsch. verw. Geb., 1981. Bd.~55. S.~109--117.

  \bibitem{PaditzTysiak1990} %13
\Au{Paditz L., Tysiak W.} Quantitative Auswertung einer
ungleichm$\ddot{\mbox{a}}${\ss}igen Fehlerabschh$\ddot{\mbox{a}}$tzung im zentralen
Grenwertsatz~// Mathematiker-Kongre{\ss} der DDR.
Vortragsausz$\ddot{\mbox{u}}$ge III.~--- Dresden, 1990. S.~153.

\columnbreak

  \bibitem{S2010} %14
\Au{Шевцова И.\,Г.} Об асимптотически правильных постоянных в
неравенстве Бер\-ри--Эс\-се\-ена--Ка\-ца~// Теория вероятностей и ее
применения, 2010. Вып.~2. С.~271--304.

  \bibitem{GS2010} %15
\Au{Григорьева М.\,Е., Шевцова И.\,Г.} Уточнение неравенства
Ка\-ца--Бер\-ри--Эс\-се\-ена~// Информатика и её применения, 2010. Т.~4.
Вып.~2. С.~78--85.

  \bibitem{S2007} %16
\Au{Шевцова И.\,Г.} О точности нормальной аппроксимации для
распределений пуассоновских случайных сумм~// Обозрение промышленной
и прикладной математики, 2007. Т.~14. Вып.~1. С.~3--28.

  \bibitem{N1965} %17
\Au{Нагаев С.\,В.} Некоторые предельные теоремы для больших
уклонений~// Теория вероятностей и ее применения, 1965. Т.~10. Вып.~2. С.~231--254.

  \bibitem{B1966} %18
\Au{Бикялис А.} Оценки остаточного члена в центральной
предельной теореме // Литовский математический сборник, 1966. Т.~6.
№\,3. С.~323--346.

  \bibitem{Paditz1989} %19
\Au{Paditz L.} On the analytical structure of the constant in
the nonuniform version of the Esseen inequality~// Statistics.~---
Berlin: Akademie-Verlag, 1989. Vol.~20. No.\,3. P.~453--464.

  \bibitem{Paditz1996} %20
\Au{Paditz L.} On the error-bound in the the nonuniform version
of Esseen's inequality in the $L_p$-metric~// Statistics.~--- Berlin: Akademie-Verlag, 1996. Vol.~27.
No.\,3. P.~379--394.

 \label{end\stat}
 \label{end-nefedova}

\bibitem{OsipovPetrov1967} %21
\Au{Осипов Л.\,В., Петров В.\,В.} Об оценке остаточного члена в
центральной предельной теореме~// Теория вероятностей и ее
применения, 1967. Т.~12. Вып.~2. С.~322--329.

 \end{thebibliography}
}
}


\end{multicols}  