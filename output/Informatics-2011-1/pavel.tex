\def\stat{pavel}

\def\tit{АЛГОРИТМ СРАВНЕНИЯ ИЗОБРАЖЕНИЙ РАДУЖНОЙ ОБОЛОЧКИ 
ГЛАЗА НА~ОСНОВЕ КЛЮЧЕВЫХ ТОЧЕК$^*$}

\def\titkol{Алгоритм сравнения изображений радужной оболочки 
глаза на основе ключевых точек}

\def\autkol{Е.\,А.~Павельева, А.\,С.~Крылов}
\def\aut{Е.\,А.~Павельева$^1$, А.\,С.~Крылов$^2$}

\titel{\tit}{\aut}{\autkol}{\titkol}

{\renewcommand{\thefootnote}{\fnsymbol{footnote}}\footnotetext[1]
{Работа выполнена при поддержке гранта РФФИ 10-07-00433 и ФЦП <<Научные и на\-уч\-но-пе\-да\-го\-ги\-че\-ские 
кадры инновационной России>> на 2009--2013~гг. Статья подготовлена по результатам работы секции 
<<Биометрия>> 20-й Международной конференции по компьютерной графике и зрению Графикон-2010, 
г.~С.-Петербург, 20--24~сентября 2010~г.}}

\renewcommand{\thefootnote}{\arabic{footnote}}
\footnotetext[1]{Московский государственный университет им.\ М.\,В.~Ломоносова, факультет вычислительной математики и 
кибернетики, paveljeva@yandex.ru}
\footnotetext[2]{Московский государственный университет им.\ М.\,В.~Ломоносова, факультет вычислительной математики и кибернетики, 
kryl@cs.msu.ru}


\Abst{Предложен алгоритм сравнения изображений радужных оболочек глаза на основе 
сравнения ключевых точек изображений. Ключевые точки определяются как точки с 
наибольшим значением свертки с функциями преобразования Эрмита. Для повышения 
эффективности алгоритма предложено исследовать только те области изображения, которые 
свободны от век, ресниц и бликов, а также учитывать возможность поворота глаза. 
Предложенный алгоритм протестирован на общедоступной базе изображений радужных 
оболочек глаз CASIA-IrisV3.}

\KW{биометрическая идентификация; радужная оболочка глаза; преобразование Эрмита; 
ключевые точки}


      \vskip 14pt plus 9pt minus 6pt

      \thispagestyle{headings}

      \begin{multicols}{2}
      
            \label{st\stat}
            
\section{Введение}

Извлечение информативных признаков изобра-\linebreak жений радужной оболочки глаза 
состоит из двух\linebreak этапов: сегментации (определения областей радуж-\linebreak ной оболочки на 
изображении) и параметризации\linebreak (вычисления информативных признаков). На 
этапе\linebreak сегментации~[1] на изображении выделяются об-\linebreak ласти радужной оболочки, 
не закрытые ресницами,\linebreak веками и бликами (рис.~1). Эти области радужной 
оболочки  называются маской радужной оболочки.
 На этапе параметризации~[2, 3] 
извлекаются ин-\linebreak
\begin{center} %fig1
%\vspace*{6pt}
\mbox{%
\epsfxsize=78mm
\epsfbox{pav-1.eps}
}
\end{center}
\vspace*{6pt}
%\begin{center}
{{\figurename~1}\ \ \small{Примеры изображений радужных оболочек глаз и их масок}}
%\end{center}
%\vspace*{9pt}

%\bigskip
\addtocounter{figure}{1}

\noindent
формативные признаки (параметры) изображения радужной 
оболочки для областей маски радужной оболочки. По этим параметрам 
происходит сравнение и идентификация радужных оболочек. 



В работе~[4] предложен метод параметризации радужных оболочек на основе 
выделения ключевых
 точек радужной оболочки методом преобразования
 Эрмита~[5]. Этот метод показал хорошие результаты
  на базе данных 
CASIA-IrisV3~[6], но, тем не менее,\linebreak
\begin{center} %fig2
\vspace*{6pt}
\mbox{%
\epsfxsize=78mm
\epsfbox{pav-2.eps}
}
\end{center}
\vspace*{6pt}
%\begin{center}
{{\figurename~2}\ \ \small{Примеры изображений глаз из базы данных\protect\linebreak CASIA}}
%\end{center}
%\vspace*{9pt}

%\bigskip
\addtocounter{figure}{1}

\noindent
 0,23\% глаз были распознаны неверно. 
Примеры глаз из базы данных CASIA показаны на рис.~2.

В данной работе предложена модификация метода ключевых точек 
идентификации человека по радужной оболочке глаза. Для каждого изображения 
из базы данных автоматически строится маска, определяющая области радужной 
оболочки без посторонней информации~--- век, ресниц и бликов. Для пары глаз их 
общей маской называется пересечение индивидуальных масок. При сравнении 
двух радужных оболочек ключевые точки ищутся только внутри областей их 
общей маски.

\vspace*{-3pt}

\section{Постановка задачи}

Общая схема алгоритма распознавания человека по ключевым точкам, 
предложенная в~[4] и используемая в данной работе, заключается в следующем.

После выделения (локализации) радужная оболочка глаза переводится в 
прямоугольное нормализованное изображение (рис.~3). Далее в об\-ласти 
параметризации нормализованного изображения\linebreak находятся ключевые точки 
радужной оболочки методом преобразования Эрмита. Для определения обладателя 
входной радужной оболочки в базе данных происходит сравнение матриц 
ключевых точек~[4].

Ранее в качестве области параметризации рассматривалась только правая верхняя 
четверть нормализованного изображения (см.\ рис.~3). В~данной работе анализируется 
все нормализованное изображение, из которого выделяется область 
параметризации (маска).

При сравнении изображений радужных оболочек на изображениях не должны 
присутствовать  области век, бликов и ресниц. В~противном случае большинство 
ключевых точек будут соответство-\linebreak
\begin{center} %fig3
\vspace*{3pt}
\mbox{%
\epsfxsize=78mm
\epsfbox{pav-3.eps}
}
\end{center}
\vspace*{3pt}
%\begin{center}
{{\figurename~3}\ \ \small{Локализация и нормализация радужной оболочки}}
%\end{center}
%\vspace*{6pt}

\begin{center} %fig4
\vspace*{6pt}
\mbox{%
\epsfxsize=78mm
\epsfbox{pav-4.eps}
}
\end{center}
%\vspace*{3pt}
%\begin{center}
{{\figurename~4}\ \ \small{Нормализованное изображение глаза, на область параметризации (пунктирная область) 
которого попали веко и блики}}
%\end{center}
%\vspace*{9pt}


%\bigskip
\addtocounter{figure}{2}

\noindent
вать этим областям (рис.~4), так как в них 
происходит наибольший скачок интенсивности изобра\-же\-ния, характеризуемый 
сверткой с выбранной в~[4] функцией преобразования Эрмита~$\varphi_{1,0}$ 
(рис.~5).
Поэтому в данной работе строится маска изображения, закрывающая области 
бликов, век и ресниц, попадающих на радужную оболочку.

\begin{center} %fig5
\vspace*{6pt}
\mbox{%
\epsfxsize=56mm
\epsfbox{pav-5.eps}
}
\end{center}
\vspace*{3pt}
%\begin{center}
{{\figurename~5}\ \ \small{Функция преобразования Эрмита $\varphi_{1,0}(x,y)\hm = -(\sqrt{2}/\pi) x e^{-(x^2+y^2)} $}}
%\end{center}
\vspace*{9pt}


%\bigskip
\addtocounter{figure}{1}

\section{Построение маски изображения}

Маска изображения радужной оболочки mask в точке $(i,j)$ равна~1, если точка 
не закрыта веком, ресницами или бликами. В~противном случае значение маски в 
данной точке равно~0. Для на\-хож\-де\-ния бликов, век и ресниц используется 
детектор границ Канни~[7] (Canny Edge detector).


Бликами считаются точки исходного изображения с интенсивностью выше~245, 
лежащие на определенных методом Канни границах или примыкающие к ним. 
Метод Канни применяется в работе дважды: сначала с порогами $\mathrm{TL} = 0$, $\mathrm{TH} = 
30$ для сильно размытого Гауссом с $\sigma = 5$ изображе-\linebreak

\begin{center} %fig6
%\vspace*{6pt}
\mbox{%
\epsfxsize=78mm
\epsfbox{pav-6.eps}
}
\end{center}
\vspace*{3pt}
%\begin{center}
{{\figurename~6}\ \ \small{Исходное изображение~(\textit{а}); изображение с наложением маски~(\textit{б}); 
маска на нормализованном изображении~(\textit{в}). Черные области~--- точки, не 
принадлежащие маске}}
%\end{center}
%\vspace*{9pt}


\bigskip
\addtocounter{figure}{1}


\noindent
ния (выделяются 
наиболее четкие границы); затем с параметрами $\sigma = 1$, $\mathrm{TL} = 0$, $\mathrm{TH} = 50$. 
При втором применении метода рассматриваются лишь те границы, которые 
примыкают к выделенным на предыдущем этапе четким границам.

Результат работы алгоритма выделения маски изображения показан на рис.~6.



\section{Определение угла поворота глаза методом полярного 
преобразования Эрмита}

Если одно изображение повернуто на угол~$\varphi$ относительно другого 
(рис.~7), то при их сравнении необходимо учитывать угол поворота. Для этого 
нормализованное изображение перед параметризацией необходимо циклически 
сдвинуть (что соответствует повороту исходного изображения) на число пикселей, 
соответствующее углу поворота.

\begin{center} %fig7
\vspace*{6pt}
\mbox{%
\epsfxsize=78mm
\epsfbox{pav-7.eps}
}
\end{center}
%\vspace*{3pt}
%\begin{center}
{{\figurename~7}\ \ \small{Нормализованные изображения двух радужных оболочек до поворота}}
%\end{center}
%\vspace*{9pt}


\bigskip
\addtocounter{figure}{1}

Для определения угла поворота глаза в работе используется метод полярного 
преобразования Эрмита.

Пусть $l_{i,j}$~--- декартовы коэффициенты Эрмита~[8, 9] для изображения, т.\,е.\ 
коэффициенты\linebreak Фурье разложения изображения в ряд по функциям Эрмита 
(собственным функциям преобразования Фурье):
$$
l_{i,j}=\int\limits_{-\infty}^\infty dy \int\limits_{-\infty}^\infty 
I(x,y)\psi_{i,j}(x,y)\,dx\,.
$$
Полярные коэффициенты Эрмита $l^p_{n-k,k}$ вы\-чис\-ля\-ют\-ся через декартовы 
коэффициенты Эрмита~$l_{n-m,m}$ по формулам:
\begin{align*}
l^p_{n-k,k} &=\sum\limits_{m=0}^n G_n^{\widetilde{cp}}(m,k) l_{n-m,m}\,;\\
G_n^{\widetilde{cp}} (m,k) &=\fr{1}{2\pi}\int\limits_0^{2\pi} \alpha^c_{n-m,m}
(\omega) \tilde{\alpha}_{n-k,k}^p (\omega)\,d\omega\,;
\end{align*}

\noindent
\begin{align*}
\alpha^c_{n-m,m}(\omega)&=\sqrt{\fr{n!}{(n-m)!m!}}\,\cos^{n-m}\omega\sin^m\omega\,,\\
&\hspace*{20mm} m=0, \ldots ,n\,;
\end{align*}

\noindent
$$
\tilde{\alpha}^p_{n-k,k}(\omega) =
\begin{cases}
\sqrt{\fr{2^n(n-k)!k!}{n!}}\,\sqrt{2}\,\cos(n-2k)\omega\,, &\\
& \hspace*{-45pt}0\leq k<\fr{n}{2}\,;\\
\sqrt{\fr{2^n(n-k)!k!}{n!}}\,; &\hspace*{-45pt} k=n-k\,;\\
\sqrt{\fr{2^n(n-k)!k!}{n!}}\,\sqrt{2}\sin(n-2k)\omega, &\\
& \hspace*{-45pt}\fr{n}{2}<k\leq n
\end{cases}
$$
и обладают следующим свойством: при повороте исходного изображения на 
угол~$\varphi$ полярные коэффициенты пересчитываются по формуле:

\noindent
\begin{multline*}
\begin{bmatrix}
l^p_{n-m,m}(\varphi)\\[6pt]
l^p_{m,n-m}(\varphi)
\end{bmatrix}={}\\
{}=
\begin{bmatrix}
\cos(n-2m)\varphi & \sin(n-2m)\varphi\\
-\sin(n-2m)\varphi & \cos(n-2m)\varphi
\end{bmatrix}
\cdot 
\begin{bmatrix}
l^p_{n-m,m}\\
l^p_{m,n-m}
\end{bmatrix}\,.
\end{multline*}
\begin{center} %fig8
\vspace*{-3pt}
\mbox{%
\epsfxsize=76.075mm
\epsfbox{pav-8.eps}
}
\end{center}
\vspace*{-6pt}
%\begin{center}
{{\figurename~8}\ \ \small{Исходное изображение~(\textit{а}) и получение нормализованного изображения~(\textit{б}) и~(\textit{в})}}
%\end{center}
%\vspace*{9pt}

\setcounter{figure}{9}
\begin{figure*} %fig10
\vspace*{1pt}
\begin{center}
\mbox{%
\epsfxsize=112.131mm
\epsfbox{pav-10.eps}
}
\end{center}
\vspace*{-6pt}
\Caption{Общая маска двух изображений (маска обозначена белым цветом) и сужение общей 
маски
\label{f10-pa}}
\end{figure*}



%\bigskip
%\addtocounter{figure}{1}


\begin{center} %fig9
%\vspace*{-3pt}
\mbox{%
\epsfxsize=78mm
\epsfbox{pav-9.eps}
}
\end{center}
\vspace*{-6pt}
%\begin{center}
{{\figurename~9}\ \ \small{Нормализованные изображения после поворота}}
%\end{center}
%\vspace*{9pt}
\pagebreak

%\bigskip
%\addtocounter{figure}{1}


Таким образом, для вычисления полярных коэффициентов Эрмита для 
повернутого на заданный угол~$\varphi$ изображения достаточно умножить 
коэффициенты исходного изображения (без поворота) на матрицу поворота.

Для того чтобы вычислять угол поворота одного глаза относительно другого, 
изображения глаз приводятся к нормализованному виду~--- квадрату 
фиксированного размера $N\times N$. Для этого сначала зрачок переводится в 
центр оболочки и приводится к нормализованному размеру ($r^\prime =N/3$). 
Далее радужная оболочка приводится к фиксированному размеру шириной в 
64~пикселя, и нормализованным изображением считается часть изображения, 
попавшая в квадрат $N\times N$ (рис.~8,\,\textit{б}). Для определения угла 
поворота в нормализованном изображении рассматривается только кольцевая 
область радужной оболочки (рис.~8,\,\textit{в}).


Далее считаются полярные коэффициенты Эрмита $l_{n-i,i}$ для $n = 0, 1, \ldots , 
63$, $i = 0, 1, \ldots , n$ для изображения и полярные коэффициенты для этого же 
изображения, повернутого на углы $\varphi=\pm2^\circ$, $\pm4^\circ$, \ldots , $\pm 
20^\circ$. Все эти коэффициенты сравниваются с полярными коэффициентами 
второго изобра\-же\-ния по метрике суммы квадратов отклонения коэффициентов. 
Угол, соответствующий минимальному отклонению коэффициентов, считается 
углом поворота между изображениями глаз.
Пример работы алгоритма 
определения угла поворота показан на рис.~9. 


После коррекции поворота двух изображений глаз составляется общая маска этих 
глаз, явля\-юща\-я\-ся пересечением масок изображений (рис.~10). 


\section{Алгоритм нахождения ключевых точек изображений радужных 
оболочек}

Для нахождения ключевых точек гистограмма нормализованного изображения 
внутри об\-ласти общей маски приводится к эквализованному виду (рис.~11), и на 
полученном изображении ищутся ключевые точки. Для этого в каждой точке 
маскированного изображения вычисляется величина $F(x_0, y_0)=(I(x,y) 
\varphi_{1,0}(x,y))(x_0,y_0)$, где $\varphi$~--- функция преобразования Эрмита. 
В~качестве кода радужной оболочки (ключевых точек) рассматривается~$N$ ($N = 
300$) точек, разбитых на две группы: $N/2$~точек с максимальными 
значениями~$F$ (черные точки на рис.~12), удаленных друг от друга не менее чем 
на 2~пикселя, и аналогично $N/2$~--- с минимальными значениями~$F$ (белые 
точки). Выбор данного числа ключевых точек будет обоснован ниже.


\begin{center} %fig11
\vspace*{6pt}
\mbox{%
\epsfxsize=78mm
\epsfbox{pav-11.eps}
}
\end{center}
%\vspace*{3pt}
%\begin{center}
{{\figurename~11}\ \ \small{Нормализованное изображение с наложением общей маски до и после эквализации 
гистограммы}}
%\end{center}
%\vspace*{9pt}


\bigskip
\addtocounter{figure}{1}


\begin{center} %fig12
%\vspace*{-3pt}
\mbox{%
\epsfxsize=78mm
\epsfbox{pav-12.eps}
}
\end{center}
%\vspace*{3pt}
%\begin{center}
{{\figurename~12}\ \ \small{Выделение ключевых точек на изображении с наложением общей маски}}
%\end{center}
%\vspace*{9pt}


%\bigskip
\addtocounter{figure}{1}



\subsection{Определение оптимального числа ключевых точек 
для~параметризации}

В работе~[4] было показано, что в случае поиска ключевых точек только в правой 
верхней четверти нормализованного изображения оптимальным числом является 
150~ключевых точек (рис.~13).

На основе этого факта определяется оптимальное число ключевых точек для 
параметризации маскированного изображения. Поскольку ключевые точки 
разбиваются на две группы (половина точек с максимальными значениями 
свертки, половина~--- с минимальными), то дальнейшие действия производятся с 
каждой из групп по отдельности.

Пусть для изображения $i$ максимальное значение свертки при выделении 
ключевых точек равно $M_i$, а значение свертки для последней взятой ключевой 
точки из данной группы равно $M_i/k_i$.\linebreak\vspace*{-12pt}

\pagebreak

\noindent
\begin{center} %fig13
\vspace*{1pt}
\mbox{%
\epsfxsize=77.878mm
\epsfbox{pav-13.eps}
}
\end{center}
%\vspace*{3pt}
%\begin{center}
{{\figurename~13}\ \ \small{График зависимости вероятности верного распознавания от числа взятых ключевых точек }}
%\end{center}
\vspace*{3pt}


\medskip
\addtocounter{figure}{1}


\noindent
 Усреднив значения~$k_i$ по всей базе 
данных, получаем значение $k_{\mathrm{mean}}=2{,}08$. Соответственно, в данной работе 
для изображения~$i$ ищется $N_i$~ключевых точек, которые попадают в 
диапазон $[M_i/k_{\mathrm{mean}},\,M_i]$. Усреднив это значение по базе данных, 
получаем $N_{\mathrm{mean}}=300$.

\vspace*{-4pt}

\subsection{Результаты}

Для оценки эффективности предложенного метода ключевых точек 
протестирована часть базы 
 данных, содержащая все изображения глаз, которые в 
работе~[4] (с использованием фиксированной области параметризации~--- верхней 
правой четверти изображения радужной оболочки глаза) давали ошибку 
распознавания. При параметризации радужных оболочек с учетом маски все эти 
глаза распознаются верно. Таким образом, применение метода ключевых точек в 
задаче распознавания человека по радужной оболочке глаза позволяет получить 
100\%-ное распознавание на базе данных CASIA-IrisV3.

\vspace*{-5pt}

\section{Заключение}

В~работе предложен алгоритм сравнения радужных оболочек глаза, полученный 
путем модификации ранее предложенного алгоритма, основанного на нахождении 
ключевых точек методом преобразования Эрмита. В~новом алгоритме строится 
маска областей изображения, свободных от век, ресниц и бликов, с учетом 
возможности поворота глаза. Предложенный метод распознавания использует 
оптимальное число ключевых точек и позволяет достичь безошибочного 
распознавания на базе данных CASIA-IrisV3.

{\small\frenchspacing
{%\baselineskip=10.8pt
\addcontentsline{toc}{section}{Литература}
\begin{thebibliography}{9}

\bibitem{1-pa}
\Au{Proenca H.}
Iris recognition: On the segmentation of degraded images acquired in the visible 
wavelength~// IEEE Transaction on Pattern Analysis and Machine Intelligence, 
2010. Vol.~32. No.\,8. P.~1502--1516.

\bibitem{2-pa}
\Au{Daugman J.} How iris recognition works~// IEEE Transactions on Circuits and 
Systems for Video Technology, 2004. Vol.~14. No.\,1. P.~21--30. 

\bibitem{3-pa}
\Au{Hollingsworth K., Bowyer~K., Flynn~P.}
The best bits in an iris code~// IEEE Transaction on Pattern Analysis and Machine 
Intelligence, 2009. Vol.~31. No.\,6. P.~964--973. 

\bibitem{4-pa}
\Au{Павельева Е.\,А., Крылов А.\,С.}
Поиск и анализ ключевых точек радужной оболочки глаза методом 
преобразования Эрмита~// Информатика и её применения, 2010. Т.~4. Вып.~1. 
С.~79--82.

\bibitem{5-pa}
\Au{Martens J.-B.}
The Hermite transform-theory~// IEEE Transactions on Acoustics, Speech, and Signal 
Processing, 1990. Vol.~38. No.\,9. P.~1595--1606.

\bibitem{6-pa}
База данных CASIA-IrisV3. {\sf http://www.cbsr.ia.ac.cn/ IrisDatabase.htm.}

\bibitem{7-pa}
\Au{Canny J.}
A computational approach to edge detection~// IEEE Transaction on Pattern Analysis 
and Machine Intelligence, 1986. Vol.~8. P.~34--43.

\bibitem{8-pa}
\Au{Martens J.-B.}
Local orientation analysis in images by means of the Hermite transform~// IEEE 
Transactions on Image Processing, 1997. Vol.~6. No.\,8. P.~1103--1116.


 \label{end\stat}
 
\bibitem{9-pa}
\Au{Kutovoi A.\,V., Krylov A.\,S.}
A new method for texture-based image analysis~// GraphiCon'2006: Conference 
Proceedings.~--- Novosibirsk, 2006. P.~235--238.
 \end{thebibliography}
}
}


\end{multicols}  

 
 
 
 