
\def\stat{pechinkin}

\def\tit{ДВУХПРИОРИТЕТНАЯ СИСТЕМА С РЕЗЕРВИРОВАНИЕМ КАНАЛОВ
И МАРКОВСКИМ ВХОДЯЩИМ ПОТОКОМ}

\def\titkol{Двухприоритетная система с резервированием каналов
и марковским входящим потоком}

\def\autkol{А.\,В.~Печинкин}
\def\aut{А.\,В.~Печинкин$^1$}

\titel{\tit}{\aut}{\autkol}{\titkol}

{\renewcommand{\thefootnote}{\fnsymbol{footnote}}\footnotetext[1]
{Работа выполнена при
поддержке РФФИ (грант \No\ 11-07-00112).}}

\renewcommand{\thefootnote}{\arabic{footnote}}
\footnotetext[1]{Институт проблем информатики
Российской академии наук, apechinkin@ipiran.ru}

\vspace*{-6pt}

\Abst{Рассматривается двухприоритетная система с
резервированием каналов пучка (СРКП), марковским входящим потоком и
различными распределениями фазового типа времен обслуживания заявок
каждого приоритета. Резервирование каналов пучка означает наличие
некоторого числа каналов, которые могут быть заняты только
приоритетными заявками. На основе исследования базовой модели
получены аналитические соотношения, позволяющие вычислять основные
стационарные показатели функционирования этой системы.}

\KW{система массового обслуживания; относительный
приоритет; резервирование каналов пучка}

\vspace*{-4pt}

      \vskip 8pt plus 9pt minus 6pt

      \thispagestyle{headings}

      \begin{multicols}{2}
      
            \label{st\stat}

\section{Введение}

В современных информационно-те\-ле\-ком\-му\-ни\-ка\-ци\-он\-ных сис\-те\-мах (ИТС)
часто приходится сталкиваться с ситуацией,
когда требования разных приоритетов обслуживаются каналами
одного пучка.
При этом, как правило, требования более высоких приоритетов
должны обслуживаться с более высоким качеством (под качеством
обслуживания в зави\-си\-мости от типа системы понимается либо
вероятность потери требования, либо среднее время задержки
требования, либо другая подобная характеристика), причем
прерывание обслуживания даже менее приоритетных требований не
допускается.
Очевидно, что применение в данном случае полнодоступных схем
(даже с учетом приоритета при выборе требования из очереди)
неоправданно завышает качество обслуживания менее приоритетных
требований и, как следствие, ведет к увеличению такой
экономической характеристики пучка, как его канальная емкость.

Один из возможных выходов из создавшейся ситуации состоит в
использовании простой схемы, получившей в литературе
название системы с резервированием каналов пучка.
При этой схеме для требований $i$-го приоритета имеется
свой порог~$n_i$ и требования данного приоритета принимаются к
обслуживанию только в том случае, когда число занятых
каналов меньше~$n_i$.

По-видимому, впервые СРКП была описана P.\,J.~Burke (фирма Bell,
США) в 1961~г.\ для обслуживания двухприоритетного пучка и
впоследствии использована в ряде систем связи, в частности в системе
RITA (Франция)~[1, 2]. В~своем первоначальном виде (без очередей)
эта система нашла также применение при создании методов
динамического управления потоками с ограничением нагрузки в сетях с
коммутацией каналов. Так, в~[3, 4] с целью ограничения использования
отдельных направлений предлагалось резервировать в каждом
направлении определенные каналы, по которым передача транзитной
нагрузки не разрешалась. В~\cite{8-p} была представлена улучшенная
динамическая стратегия, особенность которой состояла в том, что на
каждом направлении резервировалось лишь число каналов, а не
конкретные каналы, по которым запрещалось устанавливать транзитные
соединения. Именно эта модификация СРКП и рассматривается в
настоящей статье.

Вместе с практической реализацией СРКП началось ее исследование.
В~\cite{4-p} система изучалась для двухприоритетного трафика при
отсутствии очередей.
Также для двухприоритетного потока в~\cite{9-p} ана-\linebreak лизировались
четыре варианта СРКП (варианты отличались наличием или отсутствием
бесконечной очереди для приоритетных и неприоритетных требований).
Характеристики двухприоритетной СРКП для случая различных
интенсивностей обслуживания требований первого и второго
приоритетов рассчитывались в~\cite{10-p}.
В~\cite{11-p} система с резервированием анализировалась
применительно к телефонным сетям, в которых каналы пучка
использовались для передачи телефонных разговоров и данных.
В~\cite{12-p} был предложен метод приближенной оценки качества
обслуживания в подобных системах на основе введенного там понятия
эквивалентного трафика.
Наконец, в~\cite{13-p} была предложена методика расчета СРКП с
произвольным числом приоритетов, пуассоновскими входящими потоками
и одинаковым экспоненциальным обслуживанием требований всех
приоритетов.
Эта методика позволила получить в~\cite{14-p} легко реализуемые
алгоритмы вычисления основных стационарных характеристик
двухприоритетной СРКП с марковским входящим потоком и одинаковым
экспоненциальным обслуживанием требований всех приоритетов.

Очевидно, однако, что  предположения об экспоненциальности и
одинаковой распределенности времен обслуживания заявок различных
приоритетов существенно ограничивают сферу применения результатов~\cite{13-p, 14-p}.
Поэтому в настоящей статье продолжено изучение вариантов СРКП.
Рас\-смот\-ре\-на двухприоритетная СРКП с марковским входящим потоком и
фазовыми распределениями времен обслуживания заявок различных
приоритетов, причем с разными определяющими параметрами.
При этом сначала анализируется более общая модель, которая далее
используется как базовая для анализа двухприоритетной СРКП.


\section{Описание общей системы}

Рассмотрим систему массового обслуживания (СМО) марковского типа с входящим потоком заявок двух
типов (далее заявки первого типа будем называть приоритетными,
а второго~--- неприоритетными) и относительным приоритетом,
описываемую следующим образом.

Имеются неотрицательные целые числа $n_0\ge 1$,\ \ $n_1\ge n_0$ и
$n_2\ge 0$.
Эти числа интерпретируются так:
\begin{description}
\item[\,] $n_0$~--- число приборов, доступных всем заявкам (иными
словами, если занято менее $n_0$ приборов, то любая
поступающая заявка немедленно начинает обслуживаться);
\item[\,]
$n_1$~--- максимальное суммарное число неприоритетных заявок,
находящихся на приборах, и приоритетных заявок, находящихся
в системе;
\item[\,]
$n_2$~--- число мест ожидания для неприоритетных заявок,
т.\,е.\ максимальное число не\-при\-о\-ри\-тет\-ных заявок,
которые могут находиться в системе дополнительно к
заявкам на приборах.
\end{description}

Далее в этой статье будем предполагать, что $n_1>n_0$ и $n_2\ge 1$,
поскольку в случае $n_1=n_0$ и/или $n_2=0$ некоторые уравнения будут
иметь несколько другой (более простой) вид.

Процесс поступления и обслуживания заявок определяется матрицами
$\Lambda^{(u)}_n$,\ $n=\overline{0,n_1}$,\  $u=1,2$,\ 
$M_n$,\  $n=\overline{1,n_1}$,\ 
$N_n$,\  $n=\overline{0,n_1}$, и
$\Omega$ (размеры мат\-риц определяются далее и задаются числами
$I_n$,\  $n=\overline{0,n_1}$, которые будем называть
числами фаз процесса по\-ступ\-ле\-ния-об\-слу\-жи\-ва\-ния заявок
в слоях~$n$, или просто числами фаз) и протекает таким образом.

Если в системе находится~$n$,\  $n=\overline{0,n_0-1}$,
заявок (далее будем говорить, что процесс по\-ступ\-ле\-ния-об\-слу\-жи\-ва\-ния
находится в слое~$n$) и фаза равна~$i$,\  $i=\overline{1,I_n}$,
то с интенсивностью $(\Lambda^{(u)}_n)_{ij}$,\  $u=1,2$,\ 
$j=\overline{1,I_{n+1}}$, в систему поступает новая заявка $u$-го типа
(процесс по\-ступ\-ле\-ния-об\-слу\-жи\-ва\-ния переходит в слой $n+1$), которая
тут же начинает обслуживаться,  и фаза становится равной~$j$.
Кроме того, с интенсивностью $(N_n)_{ij}$,\  $j=\overline{1,I_n}$,\ 
$j\ne i$, в систему не поступают
заявки и не заканчивается обслуживание ни одной из $n$ заявок
(процесс поступления-обслуживания остается в том же слое~$n$),
но фаза становится равной~$j$.
Наконец, если дополнительно $n\ge 1$, то с интенсивностью
$(M_n)_{ij}$,\  $j=\overline{1,I_{n-1}}$, заканчивается
обслуживание одной из $n$ заявок (это может быть заявка любого
типа), процесс поступления-обслуживания переходит в слой $n-1$
и фаза становится равной~$j$.

При $n=\overline{n_0+1,n_1-1}$ будем говорить, что процесс
поступления-обслуживания находится в слое~$n$ в том случае, когда
суммарное число неприоритетных заявок, находящихся на приборах, и
приоритетных заявок, находящихся в системе, равно~$n$
и, возможно, имеется еще ка\-кое-то число (но не более~$n_2$)
неприоритетных заявок в очереди.
При этом если фаза процесса поступления-обслуживания равна
$i$,\  $i=\overline{1,I_n}$, то с интенсивностью
$(\Lambda^{(1)}_n)_{ij}$,\  $j=\overline{1,I_{n+1}}$,
в систему поступает приоритетная заявка, причем, как и прежде,
процесс по\-ступ\-ле\-ния-об\-слу\-жи\-ва\-ния переходит в слой $n+1$
и фаза становится равной~$j$.
С~интенсивностью $(N_n)_{ij}$,\  $j=\overline{1,I_n}$,\  $j\ne i$,
в систему не поступают заявки и не заканчивается обслуживание
заявок (процесс по\-ступ\-ле\-ния-об\-слу\-живания остается в том же
слое $n$), но с $i$-й на $j$-ю изменяется фаза, а с интенсивностью
$(M_n)_{ij}$,\  $j=\overline{1,I_{n-1}}$, заканчивается обслуживание
одной из заявок (процесс по\-ступ\-ле\-ния-об\-слу\-жи\-ва\-ния переходит в слой
$n-1$) и фаза становится равной~$j$.
Однако, если в систему поступает неприоритетная заявка
(с интенсивностью $(\Lambda^{(2)}_n)_{ij}$,\  $j=\overline{1,I_n}$),
то слой $n$ не меняется, а фаза становится равной~$j$.
Поступающая неприоритетная заявка либо становится в очередь
неприоритетных заявок (если там имеются свободные места),
либо теряется.

Единственным отличием случая $n=n_1$ от\linebreak предыду\-ще\-го является
то, что поступающая приоритетная заявка теряется, изменяя фазу в
том же слое.
Естественно, матрица $\Lambda^{(1)}_{n_1}$ в этом случае
является квадратной порядка $I_{n_1}$.

Обратимся к последнему случаю: $n=n_0$.
Здесь возможны два варианта, связанных с окончанием обслуживания.

Первый вариант появляется, если в системе в очереди отсутствуют
неприоритетные заявки.
Тогда, как и раньше, с интенсивностью $(M_{n_0})_{ij}$,\ 
$i=\overline{1,I_{n_0}}$,\  $j=\overline{1,I_{n_0-1}}$, заканчивается
обслуживание одной из заявок, процесс по\-ступ\-ле\-ния-об\-слу\-жи\-ва\-ния
переходит в слой $n_0-1$ и фаза становится равной~$j$.
Прежний смысл имеют также матрицы $\Lambda^{(u)}_{n_0}$,\ $u=1,2$,
и $N_{n_0}$.

Если же в системе в очереди имеются неприоритетные заявки
(второй вариант), то с интенсивностью $(M_{n_0})_{ij}$,\
$i=\overline{1,I_{n_0}}$,\ $j=\overline{1,I_{n_0-1}}$,\linebreak
заканчивается обслуживание заявки, но на освободившийся прибор
мгновенно поступает заявка из очереди неприоритетных заявок,
с вероятностью $\Omega_{jl}$,\ $l=\overline{1,I_{n_0}}$,
меняя фазу с $j$-й на $l$-ю и возвращая процесс
по\-ступ\-ле\-ния-об\-слу\-жи\-ва\-ния в слой~$n_0$.
В этом варианте формально изменение фазы без изменения слоя
определяется матрицей интенсивностей $N^*_{n_0}=N_{n_0}+M_{n_0}\Omega$,
а матрица $M^*_{n_0}=0$.
Матрицы $\Lambda^{(u)}_{n_0}$ имеют прежний смысл.

Как обычно, диагональные элементы матриц $N_n$,\ $n=\overline{0,n_1}$,
определяются таким образом, чтобы
\begin{gather*}
N_0 \vec1 + \Lambda^{(1)}_0 \vec1 + \Lambda^{(2)}_0 \vec1
= \vec 0\,,\\
N_n \vec1 + \Lambda^{(1)}_n \vec1 +
\Lambda^{(2)}_n \vec1 + M_n \vec1
=
\vec 0\,,
\ \ n=\overline{1,n_1}\,.
\end{gather*}
Здесь и далее через $\vec1$ обозначается век\-тор-стол\-бец $(1,\ldots,1)^T$,
размерность которого определяется размером умножаемой на
него слева матрицы.

В тех случаях, когда суммирование возможно, положим
\begin{align*}
N^{(0)}_n &= N_n + \Lambda^{(1)}_n + \Lambda^{(2)}_n\,,\\ 
N^{(1)}_n &= N_n + \Lambda^{(2)}_n\,,\\
N^{(2)}_n &= N_n + \Lambda^{(1)}_n\,.
\end{align*}

Очевидно, что функционирование рас\-смат\-риваемой системы можно
описать марковским\linebreak процессом с конечным числом состояний.
В~даль-\linebreak нейшем, не останавливаясь на деталях, будем\linebreak предполагать
выполненным естественное условие: этот процесс является
неприводимым.

% 3
\section{Вспомогательные матрицы}

Пусть $\nu(t)$ обозначает слой, в котором находится процесс
по\-ступ\-ле\-ния-об\-слу\-жи\-ва\-ния в момент~$t$ (т.\,е.\ $\nu(t)$ равно
при $\nu(t)\le n_0$ числу заявок в системе, или, что
то же самое, числу заявок на приборе, а при $n_0<\nu(t)\le n_1$
сумме чисел неприоритетных заявок, находящихся на приборах,
и приоритетных заявок, находящихся в системе).

Предположим, что в начальный момент значение процесса~$\nu(t)$
равнялось~$n$,\ $n=\overline{n_0,n_1}$, отсутствовала
очередь неприоритетных заявок и фаза была~$i$.
Введем матрицы $F_k(n)$,\  $k=\overline{0,n_2-1}$, и
$\tilde F_k(n)$,\ $k=\overline{0,n_2}$.
Элементы $(F_k(n))_{ij}$ и $(\tilde F_k(n))_{ij}$,\ 
$i=\overline{1,I_{n}}$,\  $j=\overline{1,I_{n-1}}$,
мат\-риц~$F_k(n)$ и~$\tilde F_k(n)$ представляют собой
вероятности того, что непосредственно после момента,
когда впервые значение процесса~$\nu(t)$ станет равным $n-1$
(в случае $n=n_0$ таковым может быть также момент,
когда на прибор впервые поступит неприоритетная заявка
из очереди неприоритетных заявок, однако эта неприоритетная
заявка пока еще не учитывается как поступившая на прибор),
фаза будет~$j$ и в системе в очереди неприоритетных заявок
будет находиться ровно~$k$ и, соответственно, не менее
$k$~за\-явок.

Следующие соотношения позволяют рекуррентно по~$n$ и~$k$, начиная
с $n=n_1$ и $k=0$, вычислять мат\-ри\-цы $F_k(n)$ и $\tilde F_k(n)$:
\begin{align*}
F_0(n_1) &= (-N^{(2)}_{n_1})^{-1} M_{n_1}\,;
\\
\tilde F_0(n_1) & = (-N^{(0)}_{n_1})^{-1} M_{n_1}\,;
\\
F_k(n_1) & = (-N^{(2)}_{n_1})^{-1} \Lambda^{(2)}_{n_1}
F_{k-1}(n_1)\,, \enskip k=\overline{1,n_2-1}\,;
\\
\tilde F_k(n_1) & = (-N^{(2)}_{n_1})^{-1} \Lambda^{(2)}_{n_1}
\tilde F_{k-1}(n_1)\,, \enskip  k=\overline{1,n_2}\,;
\\
F_0(n) & = -N_{n}^{-1} \left[ \Lambda^{(1)}_{n} F_0(n+1) F_0(n) + M_{n} \right]\,,\\
&\hspace*{40mm} n=\overline{n_0,n_1-1}\,;
\\
\tilde F_0(n) & =
(-N^{(1)}_{n})^{-1} \left[ \Lambda^{(1)}_{n}
\tilde F_0(n+1) \tilde F_0(n) + M_{n} \right]\,,\\
&\hspace*{40mm} n=\overline{n_0,n_1-1}\,;
\\
F_k(n) &=
-N_{n}^{-1} \left[ \Lambda^{(1)}_{n}
\sum\limits_{i=0}^{k} F_i(n+1) F_{k-i}(n)
+{}\right.\\
&\hspace*{-8mm}\left.{}+ \Lambda^{(2)}_{n} F_{k-1}(n)
\vphantom{\sum\limits_{i}^k}\right]\,, \  k=\overline{1,n_2-1}\,,  \ n=\overline{n_0,n_1-1}\,;
\\
\tilde F_k(n) &=
-N_{n}^{-1} \left[
\Lambda^{(1)}_{n}
\left( \sum\limits_{i=0}^{k-1} F_i(n+1) \tilde F_{k-i}(n)
+{}\right.\right.\\
&\left.\left.{}+ \tilde F_k(n+1) \tilde F_0(n)
\vphantom{\sum\limits_{i=0}^k}
\right)
+
\Lambda^{(2)}_{n} \tilde F_{k-1}(n)
\vphantom{\sum\limits_{i=0}^k}
\right]\,,\\
&  \hspace*{15mm}k=\overline{1,n_2}\,,
 \enskip n=\overline{n_0,n_1-1}\,.
\end{align*}

Положим $F_k = F_k(n_0)$,\  $k=\overline{0,n_2-1}$, и
$\tilde F_k=$\linebreak $=\tilde F_k(n_0)$,\ $k=\overline{0,n_2}$.

Далее понадобятся также некоторые характеристики, связанные со
средними временами пребывания системы в определенных состояниях.

Предположим, что в начальный момент значение процесса $\nu(t)$
равнялось~$n$,\ \ $n=\overline{n_0,n_1}$, и фаза была $i$,\ 
$i=\overline{1,I_{n}}$.
Обозначим через $S_k(n)$,\ $k=\overline{n,n_1}$,
матрицу, элементом $(S_k(n))_{ij}$,\  $j=\overline{1,I_{k}}$,
которой является среднее время, проведенное сис\-те\-мой в состояниях,
когда значение процесса~$\nu(t)$ равнялось~$k$ и фаза
была~$j$, до того момента, когда
впервые значение процесса~$\nu(t)$ стало равным $n-1$
(как и прежде, в случае $n=n_0$ таковым может быть
также момент, когда на прибор впервые поступит неприоритетная
заявка из очереди неприоритетных заявок).

Матрицы $S_k(n)$ удовлетворяют соотношениям
\begin{align}
\label{s1}
S_{n_1}(n_1) &= (-N^{(0)}_{n_1})^{-1}\,;
\\
\label{s2}
S_n(n) &= (-N^{(1)}_{n})^{-1}
\left[ E + \Lambda^{(1)}_{n} \tilde F_0(n+1) S_n(n)
\right]\,,\notag\\
&\hspace*{28mm}n=\overline{n_0,n_1-1}\,;
\\
\label{s3}
S_k(n) &  = (-N^{(1)}_{n})^{-1} \Lambda^{(1)}_{n}
\left[\vphantom{\tilde F_0(n+1)}
S_{k}(n+1) +{}\right.\notag\\
&\left.{}+ \tilde F_0(n+1) S_k(n)
\right]\,,
\ \ n = \overline{n_0,n_1-1}\,,\notag\\
&\hspace*{31mm} k = \overline{n+1,n_1}\,.
\end{align}
Соотношения~(\ref{s1})--(\ref{s3}) дают возможность применить для
нахождения матриц $S_k(n)$ рекуррентную по~$n$ и~$k$ процедуру
начиная с $n=n_1$ и $k=n_1$.

Положим $S_k = S_k(n_0)$,\ \ $k=\overline{n_0,n_1}$.

Предположим теперь, что в начальный момент значение процесса
$\nu(t)$ равнялось~$n$,\  $n=\overline{n_0,n_1}$,
отсутствовала очередь неприоритетных заявок и фаза была~$i$.
Обозначим через $\vec t_k(n)$,\  $k=\overline{0,n_2-1}$,
и $\vec T_k(n)$,\  $k=\overline{0,n_2}$,
вектор-столбцы, координатами $(\vec t_k(n))_i$ и
$(\vec T_k(n))_i$,\ $i=\overline{1,I_{n}}$,
которых являются средние времена, проведенные системой в
состояниях, когда в очереди неприоритетных
заявок имелось ровно~$k$ и не менее $k$~зая\-вок (без учета фазы), до того момента, когда впервые
значение процесса~$\nu(t)$ стало равным $n-1$
(с прежним замечанием относительно случая $n=n_0$).

Векторы $\vec t_k(n)$ и $\vec T_k(n)$ определяются рекуррентными
формулами
\begin{align}
\label{t1}
\vec t_0(n_1) & =
(-N^{(2)}_{n_1})^{-1} \vec 1\,;                
\\
\label{t2}
\vec T_0(n_1) &=
(-N^{(0)}_{n_1})^{-1} \vec 1\,;                 
\\
\label{t3}
\vec t_k(n_1) &=
(-N^{(2)}_{n_1})^{-1} \Lambda^{(2)}_{n_1}
\vec t_{k-1}(n_1)\,,\notag\\
&\hspace*{28mm} k=\overline{1,n_2-1}\,;                    \\
\label{t4}
\vec T_k(n_1) &=
(-N^{(2)}_{n_1})^{-1} \Lambda^{(2)}_{n_1}
\vec T_{k-1}(n_1)\,,\notag\\
&\hspace*{28mm} k=\overline{1,n_2}\,; \\
\label{t5}
\vec t_0(n) &= -N_{n}^{-1} 
\left[
\vec1 + \Lambda^{(1)}_{n} \vec t_0(n+1) +{}\right.\notag\\
&\hspace*{-8mm}\left.{}+ \Lambda^{(1)}_{n} F_0(n+1) \vec t_0(n)
\right]\,,
\ \ n=\overline{n_0,n_1-1}\,;
\\
\label{t6}
\vec T_0(n) &=
(-N^{(1)}_{n})^{-1}
\left[
\vec1 + \Lambda^{(1)}_{n} \vec T_0(n+1) +{}\right.\notag\\
&\hspace*{-8mm}\left.{}+ \Lambda^{(1)}_{n} \tilde F_0(n+1) \vec T_0(n)
\right]\,,
\ \ n=\overline{n_0,n_1-1}\,; 
\end{align}
\begin{align}
\label{t7}
\vec t_k(n) &=
-N_{n}^{-1}
\left[
\Lambda^{(1)}_{n} 
\left(
\vphantom{\sum\limits_i^k}
\vec t_k(n+1) +{}\right.\right.\notag\\
&\left.\left.{}+ \sum\limits_{i=0}^{k} F_i(n+1) \vec t_{k-i}(n)
\right)
+
\Lambda^{(2)}_{n} \vec t_{k-1}(n)
\right]\,,\notag\\
&\hspace*{5mm} k=\overline{1,n_2-1}\,,
\ \ n=\overline{n_0,n_1-1}\,;               
\\
\label{t8}
\vec T_k(n) &=
-N_{n}^{-1}
\left[
\Lambda^{(1)}_{n}
\left(
\vphantom{\sum\limits_i^k}
\vec T_k(n+1) +{}\right.\right.\notag\\
&\hspace*{-2mm}\left.{}+ \sum\limits_{i=0}^{k-1} F_i(n+1) \vec T_{k-i}(n) +
\tilde F_k(n+1) \vec T_0(n)
\right)+{}\notag\\
&\hspace*{-10mm}\left.{}+ \Lambda^{(2)}_{n} \vec T_{k-1}(n)
\vphantom{\sum\limits_i^k}
\right]\,,
\ \ k=\overline{1,n_2}\,,
\ \ n=\overline{n_0,n_1-1}\,.                    
\end{align}

Введем обозначения
$\vec t_k=\vec t_k(n_0)$,\  $k=\overline{0,n_2-1}$, и
$\vec T_k=\vec T_k(n_0)$,\  $k=\overline{0,n_2}$.


Наконец, предполагая, что в начальный момент значение процесса $\nu(t)$
равнялось $n$,\  $n=\overline{n_0,n_1}$, число неприоритетных
заявок в очереди было~$m$,\ $m=\overline{0,n_2}$, и фаза была~$i$,\ 
$i=\overline{1,I_{n}}$,
обозначим через $S_{kl}(n,m)$,\  $k=\overline{n,n_1}$,\ \
$l=\overline{m,n_2}$, мат\-ри\-цу, элементом $(S_{kl}(n,m))_{ij}$,\ 
$j=\overline{1,I_{k}}$, которой является среднее время,
проведенное сис\-те\-мой в со\-сто\-яни\-ях, когда значение процесса
$\nu(t)$ равнялось~$k$, число неприоритетных заявок в очереди
было~$l$ и фаза была~$j$, до того момента, когда впервые значение
процесса~$\nu(t)$ стало равным $n-1$ (как и прежде, в случае
$n=n_0$ таковым может быть также момент, когда на прибор
впервые поступит неприоритетная заявка из очереди неприоритетных
заявок).

Матрицы $S_{kl}(n,m)$ удовлетворяют соотношениям:
\begin{align}
\label{2m1}
S_{n_1n_2}(n_1,n_2) &=
(-N^{(0)}_{n_1})^{-1}\,;
\\
S_{nn_2}(n,n_2) &=
(-N^{(1)}_{n})^{-1}
\left[\vphantom{\tilde F_0(n+1)}
E +{}\right.\notag\\
&\hspace*{-23mm}\left.{}+ \Lambda^{(1)}_{n} \tilde F_0(n+1) S_{nn_2}(n,n_2)
\right]\!,
\ n=\overline{n_0,n_1-1}\,;\!\label{2m2}\\
S_{kn_2}(n,n_2) &=
(-N^{(1)}_{n})^{-1} \Lambda^{(1)}_{n}
\left[\vphantom{\tilde F_0(n+1)}
S_{kn_2}(n+1,n_2) +{}\notag\right.\\
&\hspace*{-15mm}\left.{}+ \tilde F_0(n+1) S_{kn_2}(n,n_2)
\right]\,,
 \ n=\overline{n_0,n_1-1}\,,\notag\\
& \hspace*{28mm} k=\overline{n+1,n_1}\,;
 \label{2m3}\\
 S_{n_1m}(n_1,m) &=
(-N^{(2)}_{n_1})^{-1}\,,
\ \ m=\overline{0,n_2-1}\,;
\label{2m4}\\
\label{2m5}
S_{n_1l}(n_1,m) & =
(-N^{(2)}_{n_1})^{-1} \Lambda^{(2)}_{n} S_{n_1l}(n_1,m+1)\,,\notag\\
& m=\overline{0,n_2-1}\,,
\enskip l=\overline{m+1,n_2}\,;\\
\label{2m6}
S_{nm}(n,m) &=\notag\\
&\hspace*{-5.5mm}{}=-N_{n}^{-1} 
\left[
E + \Lambda^{(1)}_{n} F_0(n+1) S_{nm}(n,m)
\right]\,,\notag\\
&  n=\overline{n_0,n_1-1}\,,\ m=\overline{0,n_2-1}\,;
\end{align}
\begin{align}
\label{2m7}
S_{km}(n,m) & =
-N_{n}^{-1} \Lambda^{(1)}_{n}
\left[
S_{km}(n+1,m) +{}\right.\notag\\
&\hspace*{-10mm}\left.{}+ F_0(n+1) S_{km}(n,m)
\right]
\,,
\ \ n=\overline{n_0,n_1-1}\,,\notag\\
&m=\overline{0,n_2-1}\,,
\ \ k=\overline{n+1,n_1}\,;
\\
\label{2m8}
S_{nl}(n,m) &=
-N_{n}^{-1} 
\left(\vphantom{\sum\limits_{i=0}^{l-m}}
\Lambda^{(2)}_{n} S_{nl}(n,m+1) +{}\right.\notag\\
&\hspace*{-20mm}\left.{}+ \Lambda^{(1)}_{n} \sum_{i=0}^{l-m} F_i(n+1) S_{nl}(n,m+i)
\right)\!,\ n=\overline{n_0,n_1-1}\,,\notag\\
& m=\overline{0,n_2-1}\,,
\enskip l=\overline{m+1,n_2-1}\,;
\\
\label{2m9}
S_{nn_2}(n,m) &=
-N_{n}^{-1}
\left(
\vphantom{\sum\limits_{i=0}^{l-m}}
\Lambda^{(2)}_{n} S_{nn_2}(n,m+1) +{}\right.\notag\\
&\hspace*{-15mm}{}+ \Lambda^{(1)}_{n}
\left[
\sum_{i=0}^{n_2-m-1} F_i(n+1) S_{nn_2}(n,m+i) +{}\right.\notag\\
&\left.\left.{}+\tilde F_{n_2-m}(n+1) S_{nn_2}(n,n_2)
\vphantom{\sum\limits_i^l}
\right]
\right)
\,,\notag\\
&\hspace*{5mm}n=\overline{n_0,n_1-1}\,,
\enskip  m=\overline{0,n_2-1}\,;
\\
\label{2m10}
S_{kl}(n,m) &=\notag\\
&\hspace*{-18mm}{}=-N_{n}^{-1}\!
\left( \vphantom{\sum\limits_i^l}
\Lambda^{(2)}_{n} S_{kl}(n,m+1) +\right.
\Lambda^{(1)}_{n}\!
\left[\vphantom{\sum\limits_i^k}
S_{kl}(n+1,m) +{}\right.\notag\\
&\left.\left.{}+ \sum_{i=0}^{l-m} F_i(n+1) S_{kl}(n,m+i)
\right ]
\right )
\,,\notag\\
&n=\overline{n_0,n_1-1}\,,
\enskip \ m=\overline{0,n_2-1}\,,\notag\\
&k=\overline{n+1,n_1}\,,
\enskip l=\overline{m+1,n_2-1}\,;
\\
\label{2m11}
S_{kn_2}(n,m) &=
-N_{n}^{-1}
\left(
\Lambda^{(2)}_{n} S_{kn_2}(n,m+1) +{}\right.\notag\\
&{}+ \Lambda^{(1)}_{n}
\left[ \vphantom{\tilde{F}_{n_2}}
S_{kn_2}(n+1,m) +{}\right.\notag\\
&{}+ \sum_{i=0}^{n_2-m-1} F_i(n+1) S_{kn_2}(n,m+i) +{}\notag\\
&\left.\left.{}+
\tilde F_{n_2-m}(n+1) S_{kn_2}(n,n_2)
\right]
\right)
\,,\notag\\
& n=\overline{n_0,n_1-1}\,,
\enskip m=\overline{0,n_2-1}\,,\notag\\
&\hspace*{15mm} k=\overline{n+1,n_1}\,.
\end{align}
Соотношения (\ref{2m1})--(\ref{2m11}) позволяют вычислять
эти матрицы рекуррентно по~$n$ и~$m$, начиная
с $n=n_1$ и $k=n_2$, и рекуррентно по~$k$ и~$l$, начиная
с $k=n_1$ и $l=n_2$.

Положим $S_{kl}(m) = S_{kl}(n_0,m)$,\  $m=\overline{0,n_2-1}$,\ 
$k=\overline{n_0,n_1}$,\ $l=\overline{0,n_2}$.

Отметим, что матрицы $S_k(n)$ и векторы $\vec t_k(n)$ и $\vec T_k(n)$
могут быть легко найдены из матриц $S_{kl}(n,m)$.
Однако вычисление матриц $S_{kl}(n,m)$ требует весьма больших объемов
памяти и вычислений, что делает во многих случаях такой способ
неприемлемым.

\section{Вложенная цепь Маркова}

Вложенную цепь Маркова для данной системы можно ввести различными
способами.
Здесь удобно определить ее таким образом.

Рассмотрим следующие моменты:
\begin{itemize}
\item моменты освобождения каких-либо приборов или поступления
заявок в систему в случае, когда перед этими моментами было
занято менее $n_0$ приборов;
\item
моменты освобождения каких-либо приборов в случае, когда
перед этими моментами было занято ровно $n_0$ приборов.
\end{itemize}

Последовательность таких моментов обозначим через
$\{\tau_l$,\ \ $l\ge 1\}$.

Отметим, что непосредственно после момента второго типа либо в
системе остается ровно $n_0-1$ заявок (все они обслуживаются
на приборах), либо на прибор поступает неприоритетная заявка
из очереди.

Множество состояний вложенной цепи Маркова представляет собой
пару $(i,m)$,\  $m=$\linebreak $=\overline{0,n_0+n_2-1}$.
Здесь $i$~--- фаза (процесса по\-ступ\-ле\-ния-об\-слу\-жи\-ва\-ния)
соответствующего слоя.
Более хитро определяется второй компонент $m$ множества
состояний.
А~именно, если $m=\overline{0,n_0-1}$, то, как обычно,
$m$~--- число заявок в системе (оно же номер слоя, или число
занятых приборов).
Однако если $m=\overline{n_0,n_0+n_2-1}$,
то число занятых приборов равно $n_0$, а $m-n_0$~---
число неприоритетных заявок в очереди.
Саму цепь Маркова образуют фаза и число занятых приборов
(номер слоя) или число занятых приборов (номер слоя) плюс
число неприоритетных заявок в очереди непосредственно после
моментов~$\tau_l$.

Матрицу переходных вероятностей вложенной цепи Маркова будем
обозначать через
$P = (P_{m_1m_2})$,\ $m_1,m_2=\overline{0,n_0+n_2-1}$.
Отметим, что в силу принятого соглашения элементы $P_{m_1m_2}$
мат\-ри\-цы~$P$ сами являются матрицами размеров,
определяемых числами фаз соответствующих слоев.

Матрицы $P_{m_1m_2}$ определяются формулами:
\begin{align*}
P_{m,m-1} &= - N_m^{-1} M_m\,,
\ \ m=\overline{1,n_0-1}\,;
\\
P_{m,m+1} &= - N_m^{-1}
(\Lambda^{(1)}_m+\Lambda^{(2)}_m)\,,
\ \ m=\overline{0,n_0-1}\,;
\\
%%%%%%%%%%%%%%%%%%%%%%%%%%%%%%%%%%%%%%%
P_{m_1m_2} &=
F_{m_2-m_1+1} \Omega\,,
\ \ m_1=\overline{n_0,n_0+n_2-1}\,,\notag\\
& \hspace*{-3mm}m_2=\overline{m_1-1,n_0+n_2-2}\,,
\enskip m_2\ne n_0-1\,;
\\
P_{n_0,n_0-1} &= F_0\,;
\\
P_{m,n_0+n_2-1} &=
\tilde F_{n_0+n_2-m} \Omega\,,
\ \ m=\overline{n_0,n_0+n_2-1}\,.
\end{align*}
Остальные матрицы $P_{m_1m_2}$ являются нулевыми.

Обозначим через $\vec p^{\,*}$ век\-тор-стро\-ку стационарных
вероятностей вложенной цепи Маркова. Ес\-тественно, координаты
$\vec p^{\,*}_m$,\ $m=\overline{0,n_0+n_2-1}$,\linebreak вектора
$\vec p^{\,*}$ сами являются вектор-строками с координатами
$(\vec p^{\,*}_m)_i$, где индекс~$i$ означает фазу.

Вектор $\vec p^{\,*}$ удовлетворяет системе уравнений равновесия
(СУР) $\vec p^{\,*} = \vec p^{\,*} P$
с условием нормировки
$$
\sum\limits_{m=0}^{n_0+n_2-1} \vec p^{\,*}_m \vec1 = 1\,.
$$

Для решения СУР удобно применить алгоритм, основанный на
последовательном упрощении цепи Маркова с помощью исключения
множеств состояний (см.~\cite{15-p}, с.~22).

\section{Стационарные характеристики системы}

В этом разделе будут найдены основные стационарные характеристики
функционирования общей системы.

Предположим, что в начальный момент времени вложенная цепь
Маркова находилась в состоянии
$(i,n)$,\  $n=\overline{0,n_0+n_2-1}$.
Обозначим через $\vec m_n$ век\-тор-стол\-бец, координатой $(\vec m_n)_i$
которого является среднее время до следующего момента
изменения состояния вложенной цепи Маркова.
Тогда
\begin{align*}
\vec m_n &= - N_n^{-1} \vec1\,,
\ \ n=\overline{0,n_0-1}\,;
\\
\vec m_n &= \vec T_0\,,
\ \ n=\overline{n_0,n_0+n_2-1}\,.
\end{align*}

Среднее время $\overline m$ между соседними моментами
изменения вложенной цепи Маркова системы, функционирующей
в стационарном режиме, равно
$$
\overline m = \sum\limits_{n=0}^{n_0+n_2-1} \vec p^{\,*}_n \vec m_n\,.
$$

Введем следующие обозначения:
\begin{description}
\item[\,] $\vec p_n$,\  $n=\overline{0,n_1}$,~---
век\-тор-стро\-ка с координатами $(\vec p_n)_i$,\ \ $i=\overline{1,I_n}$,
представляющими собой стационарные вероятности того, что
значение процесса $\nu(t)$ равно~$n$ и фаза равна~$i$;
\item[\,] 
$p^{(2)}_n$,\  $n=\overline{0,n_2}$,~---
стационарная вероятность того, что в системе в очереди
неприоритетных заявок имеется $n$~заявок (фаза не учитывается);
\item[\,] 
$\vec p_{nm}$,\  $n=\overline{0,n_1}$,\ 
$m=\overline{0,n_2}$,~---
век\-тор-стро\-ка с координатами $(\vec p_{nm})_i$,\  $i=\overline{1,I_n}$,
представляющими собой стационарные вероятности того, что
значение процесса $\nu(t)$ равно~$n$, в очереди неприоритетных
заявок находится $m$ заявок и фаза равна~$i$.
\end{description}

Тогда
\begin{align*}
 \vec p_n &= - \fr{1}{\overline m}\, \vec p^{\,*}_n N_n^{-1}\,,
\ \ n=\overline{0,n_0-1}\,;\\
\vec p_n &= \fr{1}{\overline m}\, \sum\limits_{k=n_0}^{n_0+n_2-1} \vec p^{\,*}_k S_{n}\,,
\ \ n=\overline{n_0,n_1}\,;\\
p^{(2)}_0 &= \fr{1}{\overline m}\, \left(
-
\sum\limits_{k=0}^{n_0-1} \vec p^{\,*}_k N_k^{-1} \vec 1 +
\vec p^{\,*}_{n_0} \vec t_0
\right)\,;
\\
p^{(2)}_n & = \fr{1}{\overline m}\,
\sum\limits_{k=0}^{n} \vec p^{\,*}_{n_0+k} \vec t_{n-k}\,,
\ \ n=\overline{1,n_2-1}\,;
\\
p^{(2)}_{n_2} & = \fr{1}{\overline m}\,
\sum\limits_{k=0}^{n_2-1} \vec p^{\,*}_{n_0+k} \vec T_{n_2-k}\,;
\\
\vec p_{n0} & = \vec p_n, \ \ n=\overline{0,n_0-1}\,;
\\
\vec p_{nm} & = \vec 0\,,
\ \ n=\overline{0,n_0-1}\,,
\ \ m=\overline{1,n_2}\,;
\\
\vec p_{nm} & =
\fr{1}{\overline m}\, \sum\limits_{k=0}^{m} \vec p^{\,*}_{n_0+k} S_{nm}(m-k)\,,
\ \ n=\overline{n_0,n_1}\,,\notag\\
&\hspace*{35mm} m=\overline{0,n_2-1}\,;
\\
\vec p_{nn_2} & = \fr{1}{ \overline m}\,
\sum\limits_{k=0}^{n_2-1} \vec p^{\,*}_{n_0+k} S_{nm}(m-k)\,,
\ \ n=\overline{n_0,n_1}\,.
\end{align*}

Интенсивность поступления приоритетных заявок в стационарном
режиме функционирования определяется формулой
$$
\lambda^{(1)}
=
\sum\limits_{n=0}^{n_1} \vec p_{n} \Lambda_n^{(1)} \vec 1\,,
$$
а неприоритетных~--- формулой
$$
\lambda^{(2)} = \sum\limits_{n=0}^{n_1} \vec p_{n} \Lambda_n^{(2)} \vec 1\,.
$$

Стационарная вероятность потери приоритетной заявки определяется
выражением
$$
p^{(1)}_{\mathrm{loss}} = \fr{1}{\lambda^{(1)}}\,
\vec p_{n_1} \Lambda_{n_1}^{(1)} \vec 1\,,
$$
а неприоритетной --- выражением
$$
p^{(2)}_{\mathrm{loss}} = \fr{1}{\lambda^{(2)}} \sum_{n=n_0}^{n_1} \vec p_{n,n_2} \Lambda_{n}^{(2)} \vec 1\,.
$$

Пусть в начальный момент значение процесса $\nu(t)$ равнялось~$n$,\  
$n=\overline{n_0,n_1}$, и фаза была~$i$,\ $i=\overline{1,I_{n}}$.
Обозначим через $W(x;n)$ мат\-ри\-цу, элементом
$W_{ij}(x;n)$,\ $j=\overline{1,I_{n-1}}$, которой является
вероятность того, что время до первого момента, когда
значение процесса $\nu(t)$ станет равным $n-1$, будет меньше~$x$
и фаза в этот момент будет~$j$.
Положим
$$
w(\omega;n) = \int\limits_0^\infty e^{-\omega x}\, dW(x;n)
\ \ n=\overline{n_0,n_1}\,.
$$
%%%%%%%%%%%%%%%%%%%%%%%%%%%
Тогда
$$
w(\omega;n_1) = (\omega E - N_{n_1}^{(0)})^{-1} M_{n_1}\,;
$$

\vspace*{-12pt}

\noindent
\begin{multline*}
w(\omega;n) = (\omega E - N_{n}^{(1)})^{-1}
\left[\vphantom{\Lambda_n^{(1)}}
M_{n} +{}\right.\\
\left.{}+ \Lambda_n^{(1)} w(\omega;n+1) w(\omega;n)\right]\,,\enskip
n=\overline{n_0,n_1-1}\,.
\end{multline*}
Введем обозначения

\noindent
\begin{gather*}
\tilde w(\omega;n) = w(\omega;n) \cdots w(\omega;n_0)\,,
\ \ n=\overline{n_0,n_1}\,;
\\
w^*(\omega) = w(\omega;n_0)\,.
\end{gather*}

Преобразование Лап\-ла\-са--Стилтье\-са (ПЛС) стационарного распределения
времени ожидания начала обслуживания принятой в систему неприоритетной
заявки (предполагается, что неприоритетные заявки
обслуживаются в порядке поступления)
определяется формулой:

\noindent
\begin{multline*}
w^{(2)}(\omega) = \fr{1}{\lambda^{(2)}(1-p^{(2)}_{\mathrm{loss}})
}
\left(
\sum_{n=0}^{n_0-1} \vec p_{n} \Lambda_n^{(2)} \vec 1 +{}\right.\\
\left.{}+
\sum_{n=n_0}^{n_1} \sum_{m=0}^{n_2-1}
\vec p_{nm} \Lambda_n^{(2)} \tilde w(\omega;n) [\Omega\, w^*(\omega)]^m
\vec 1 \right)\,.
\end{multline*}

Стационарное распределение времени ожи\-дания
начала обслуживания принятой в систему\linebreak
приоритетной заявки (в предположении, что и приоритетные заявки
обслуживаются в порядке поступления)
здесь будет вычислено для несколько упрощенного
варианта общей системы.
А~именно, будем предполагать, что общее число приборов в системе равно
$n^*$,\ $n_0 \le n^* \le n_1$.
Тогда если суммарное число $\nu(t)$ неприоритетных заявок на приборах
и приоритетных заявок в системе в момент~$t$ меньше $n^*$, то
поступающая в систему в этот момент приоритетная заявка попадает на
прибор, а если $\nu(t)$ не меньше $n^*$, то становится в очередь и
ждет, пока не обслужатся $\nu(t)-n^*+1$ заявок.
Кроме того, предполагается, что $I_n = \tilde I$,\ 
$\Lambda_n^{(u)}=\tilde \Lambda^{(u)}$,\  $u=1,2$,\  $M_n=\tilde M$
и $N_n=\tilde N$ для всех $n$,\  $n^*\le n \le n_1$.
Тогда ПЛС стационарного распределения времени ожидания принятой к
обслуживанию приоритетной заявки определяется формулой:

\noindent
\begin{multline*}
w^{(1)}(\omega) = \fr{1}{\lambda^{(1)} (1-p^{(1)}_{\mathrm{loss}})}
\left[ 
\sum_{n=0}^{n^*-1} \vec p_{n}
\Lambda_n^{(1)} \vec 1 +{}\right.\\
\left.{}+ \sum_{n=n^*}^{n_1-1} \vec p_{n} \tilde \Lambda^{(1)}
\left[(\omega E - \tilde N^{(0)})^{-1} \tilde M\right]^{n-n^*+1}
\vec 1 \right]\,,
\end{multline*}
где
$\tilde N^{(0)}=\tilde N+\tilde \Lambda^{(1)}+\tilde \Lambda^{(2)}$.

\columnbreak

Отметим, что приведенное выше описание общей системы является
недостаточным для вы\-чис\-ле\-ния распределений, связанных с временами
обслуживания на приборах и пребывания заявок в системе.

\vspace*{-12pt}

\section{Двухприоритетная СРКП}

\vspace*{-12pt}

Рассмотрим теперь двухприоритетную СРКП с марковским входящим потоком, фазовыми распределениями времен
обслуживания заявок каждого типа и относительным приоритетом.
Эта система описывается следующим образом.

В систему поступает марковский поток заявок двух типов,
определяемый квадратными матрицами $\Lambda^{(u)}$,\ \ $u=1,2$,
и $N$ порядка $J$, представляющими собой матрицы
интенсивностей переходов процесса генерации заявок с
поступлением заявки $u$-го типа и без поступления заявок.

Время обслуживания заявки $u$-го типа имеет {\it PH}-рас\-пре\-де\-ле\-ние
с параметрами $(\vec b^{(u)},M^{(u)})$, где $\vec b^{(u)}$ и
$M^{(u)}$~--- век\-тор-стро\-ка размерности $J_u$ и мат\-ри\-ца того же порядка.

В системе имеется $n^*$ приборов. Однако заявки второго типа (неприоритетные) принимаются к
обслуживанию только в том случае, если число занятых приборов меньше $n_0\le n^*$;
в противном случае они становятся в очередь неприоритетных заявок с числом мест ожидания $n_2$.
Заявки первого типа (приоритетные) поступают на обслуживание, если имеется
хотя бы один свободный прибор, а иначе становятся в очередь
приоритетных заявок с числом мест ожидания $n_1-n^*$.
Заявка $u$-го типа, поступившая при отсутствии свободных мест
ожидания в соответствующей очереди, теряется.
Прерывание обслуживания заявки любого типа не допускается.

Естественно, при $n^*=n_0$ неприоритетные заявки могут
поступать из очереди на освободившийся прибор только в том
случае, когда нет заявок в очереди приоритетных заявок, т.\,е.\
имеет место стандартная СМО $MAP/PH_2/n_0/(n_1-n_0,n_2)$ с
относительным приоритетом.

Покажем, как двухприоритетную СРКП можно привести к общей
рассматриваемой марковской модели (упрощенный вариант).
При этом сразу же заметим, что предлагаемый способ не
является наилучшим, поскольку приводит к матрицам большой
размерности.
Применение более рациональных способов, основанных на
фиксации только числа заявок, обслуживаемых на разных
фазах (см., например,~\cite{16-p}), технически более сложно, и
такие способы здесь не описываются.

\pagebreak

Введем обозначения:
\begin{description}
\item[\,] $E_J$~--- единичная матрица порядка $J$;
\item[\,] 
$E^{(n)}$~--- единичная матрица порядка $(J_1+J_2)^n$;
\item[\,] 
$E_1^{(n)}$~--- единичная матрица порядка $J_1^n$;
\item[\,] 
$\vec\beta_1 = (b^{(1)}_1,\ldots,b^{(1)}_{J_1},0,\ldots,0)$~---
век\-тор-стро\-ка порядка $J_1+J_2$;
\item[\,] 
$\vec\beta_2=(0,\ldots,0,b^{(2)}_1,\ldots,b^{(2)}_{J_2})$~---
вектор-строка порядка $J_1+J_2$;
\item[\,] 
$\vec\beta=(b^{(1)}_1,\ldots,b^{(1)}_{J_1},
b^{(2)}_1,\ldots,b^{(2)}_{J_2})$~--- век\-тор-стро\-ка порядка $J_1+J_2$;
\item[\,] 
$ M= \begin{pmatrix}
M^{(1)}   &     0       \\
   0      &   M^{(2)}   
   \end{pmatrix}\,.
$
\end{description}

Положим
$$
I_n
=
\begin{cases}
J (J_1+J_2)^{n_0} J_1^{n^*-n_0}\,,
                          &n=\overline{n^*+1,n_1}\,,        \\
J (J_1+J_2)^{n_0} J_1^{n-n_0}\,, &n=\overline{n_0+1,n^*}\,,        \\
J (J_1+J_2)^n\,,            &n=\overline{0,n_0}\,;
\end{cases}
$$
$$
\mu_i^{(u)} = - \sum\limits_{j=1}^{J_u} M^{(u)}_{ij}\,;\enskip 
\vec\mu^{(1)} = (\mu_1^{(1)},\ldots,\mu_{J_1}^{(1)})^T\,;
$$
$$
\vec\mu^{(2)} = (\mu_1^{(2)},\ldots,\mu_{J_2}^{(2)})^T\,;
$$
$$
\vec\mu = (\mu_1^{(1)},\ldots,\mu_{J_1}^{(1)}\,,
\mu_1^{(2)},\ldots,\mu_{J_2}^{(2)})^T\,;
$$
$$
\Lambda^{(1)}_n
= \begin{cases}
\Lambda^{(1)} \otimes E_1^{(n^*-n_0)} \otimes E^{(n_0)}\,, &\hspace*{-12pt}n=\overline{n^*,n_1}\,, \\
\Lambda^{(1)} \otimes \vec b^{(1)} \otimes E_1^{(n-n_0)} \otimes E^{(n_0)}\,, &\\
&\hspace*{-26pt}n=\overline{n_0,n^*-1}\,, \\
\Lambda^{(1)} \otimes \vec\beta_1 \otimes E^{(n)}\,, &\hspace*{-22pt}n=\overline{0,n_0-1}\,; 
\end{cases}
$$
$$
\Lambda^{(2)}_n = 
\begin{cases}
\Lambda^{(2)}\otimes E_1^{(n^*-n)} \otimes E^{(n_0)}\,,
 &n=\overline{n^*,n_1}\,, \\
\Lambda^{(2)} \otimes E_1^{(n-n_0)} \otimes E^{(n_0)}\,,
&n=\overline{n_0,n^*-1}\,, \\
\Lambda^{(2)} \otimes E^{(n)} \otimes \vec\beta_2\,, &n=\overline{0,n_0-1}\,; 
\end{cases}
$$
$$
M_n = \begin{cases}
E_J \otimes \vec b^{(1)} \otimes
\left(
\sum\limits_{k=0}^{n^*-n_0-1}
E_1^{(k)} \otimes
\vec\mu^{(1)} \otimes\right.{}\\
{}\otimes
E_1^{(n^*-n_0-k-1)} \otimes E^{(n_0)}
+
E_1^{(n^*-n_0)} \otimes{}\\
\left.{}\otimes
\sum\limits_{k=0}^{n_0-1}
E^{(k)} \otimes
\vec\mu \otimes E^{(n_0-k-1)}
\vphantom{\sum\limits_{k=0}^{n^*-n_0-1}}
\right),\\
\hspace*{35mm}n=\overline{n^*+1,n_1}\,, \\
E_J\! \otimes
\!\left(
\sum\limits_{k=0}^{n-n_0-1}\!\!\!
E_1^{(k)} \!\otimes\!
\vec\mu^{(1)} \otimes
E_1^{(n-n_0-k-1)}\! \otimes\right.\\
\hspace*{3mm}{}\otimes E^{(n_0)}+  
E_1^{(n-n_0)} \otimes
\sum\limits_{k=0}^{n_0-1}
E^{(k)} \otimes
\vec\mu \otimes{}\\
\hspace*{15mm}\left.{}\otimes E^{(n_0-k-1)}
\vphantom{\sum\limits_{k=0}^{n_0-1}}\right),\, 
          n=\overline{n_0+1,n^*}\,, \\
E_J \otimes
\sum\limits_{k=0}^{n-1}
E^{(k)} \otimes
\vec\mu \otimes E^{(n-k-1)},\
                    n=\overline{1,n_0}\,; 
\end{cases}
$$

\noindent
$$
N_n =
\begin{cases}
N \otimes E_1^{(n^*-n_0)}
\otimes E^{(n_0)} +{}\\[2pt]
\hspace*{3mm}{}+
E_J \otimes
\bigg(
\sum\limits_{k=0}^{n^*-n_0-1}
E_1^{(k)} \otimes
M^{(1)}
\otimes{}\\[2pt]
\hspace*{4.5mm}{}\otimes E_1^{(n^*-n_0-k-1)}
\otimes E^{(n_0)}
 +
E_1^{(n^*-n_0)} \otimes{}\\[2pt]
\hspace*{9mm}{}\otimes
\sum\limits_{k=0}^{n_0-1}
E^{(k)} \otimes
M \otimes E^{(n_0-k-1)}
\bigg),\\[3pt]
          \hspace*{40mm}n=\overline{n^*+1,n_1}\,,    \\[6pt]
N \otimes E_1^{(n-n_0)}
\otimes E^{(n_0)} +{}\\[2pt]
\hspace*{5mm}{}+
E_J \otimes
\bigg(
\sum\limits_{k=0}^{n-n_0-1}
E_1^{(k)} \otimes
M^{(1)} \otimes{}\\[2pt]
\hspace*{7mm}{}\otimes
E_1^{(n-n_0-k-1)} \otimes E^{(n_0)}
+
E_1^{(n-n_0)} \otimes{}\\[2pt]
\hspace*{9mm}{}\otimes
\sum\limits_{k=0}^{n_0-1}
E^{(k)} \otimes
M \otimes E^{(n_0-k-1)}
\bigg)\,,\\[3pt]
            \hspace*{40mm}n=\overline{n_0+1,n^*}\,,  \\[6pt]
N \otimes E^{(n)} +
E_J \otimes
\sum\limits_{k=0}^{n-1}
E^{(k)} \otimes
M \otimes{}\\[2pt]
\hspace*{5mm}{}\otimes E^{(n-k-1)}\,, \hspace*{15mm}n=\overline{1,n_0}\,,     \\[6pt]
N,\hspace{36mm} n=0\,;                   
                     \end{cases}
$$

\noindent
$$
\Omega = E_J \otimes E^{(n_0-1)} \otimes \vec\beta_2\,.
$$

Определив таким образом исходные
параметры рассмотренной в предыдущих разделах общей
марковской системы, приходим к двухприоритетной СРКП,
описанной в настоящем разделе.
При этом принята следующая нумерация приборов: прибору,
на который поступает новая приоритетная заявка,
присваивается первый номер (номера остальных приборов
увеличиваются на единицу), а неприоритетная заявка~---
последний номер.
Ясно, что если число~$n$ обслуживаемых заявок больше
$n_0$, то на приборах с номерами от одного до $n-n_0$
обязательно находятся приоритетные заявки.

Кратко остановимся на тех показателях функционирования СРКП,
которые можно вычислить с помощью более простых, чем для
общей системы, формул.

Прежде всего, стационарная вероятность
$p^{(1)}_n$,\ $n=\overline{0,n_1-n^*}$, того, что в системе в
очереди приоритетных заявок имеется $n$~заявок (без учета фазы),
определяется формулами:
\begin{align*}
p^{(1)}_0 &= \sum\limits_{k=0}^{n^*} \vec p_k \vec 1\,;
\\
p^{(1)}_n &= \vec p_{n^*+n} \vec 1\,,
\ \ n=\overline{1,n_1-n^*}\,.
\end{align*}

Далее обозначим через $\vec\pi$ век\-тор-стро\-ку стационарных
вероятностей состояний процесса генерации заявок.
Для $\vec\pi$ справедлива СУР
$$
\vec\pi (\Lambda^{(1)}+\Lambda^{(2)}+N)=\vec {0}
$$
с условием нормировки
$\vec\pi \vec1 =1.$
Тогда стационарную интенсивность поступления заявок $u$-го типа
можно записать в форме:
$$
\lambda^{(u)} = \vec\pi \Lambda^{(u)} \vec1\,,
\ \ u=1,2\,.
$$

Введем матрицы $t_k(n)$,\  $k=\overline{1,n_2-1}$,\ \
$n=\overline{n_0,n_1}$, и $T_k(n)$,\  $k=\overline{1,n_2}$,\ 
$n=\overline{n_0,n_1}$, размерности $I_{n_0+n} \times J$,
отличающиеся от векторов $\vec t_k(n)$ и $\vec T_k(n)$,
определенных в разд.~2, только тем, что второй
индекс~$j$ элементов $(t_k(n))_{ij}$ и $(T_k(n))_{ij}$,\
$i=\overline{1,I_{n_0+n}}$,\ $j=\overline{1,J}$,
этих матриц фиксирует состояние процесса генерации.
Для матриц $t_k(n)$ и $T_k(n)$ справедливы формулы:
\begin{align*}
t_0(n_1) &= (-N^{(2)}_{n_1})^{-1} (E_J\otimes \vec 1)\,;
\\[4pt]
T_0(n_1) & = (-N^{(0)}_{n_1})^{-1} (E_J\otimes \vec 1)\,;
\\[4pt]
t_k(n_1) & =
(-N^{(2)}_{n_1})^{-1} \Lambda^{(2)}_{n_1}
t_{k-1}(n_1)\,,
\ \ k=\overline{1,n_2-1}\,;
\\[4pt]
T_k(n_1) & =
(-N^{(2)}_{n_1})^{-1} \Lambda^{(2)}_{n_1}
T_{k-1}(n_1)\,,
\ \ k=\overline{1,n_2}\,;
\\[4pt]
t_0(n) & = -N_{n}^{-1}
\big[
(E_J\otimes \vec1) +
\Lambda^{(1)}_{n} t_0(n+1) +{}\\
&\hspace*{5mm}{}+
\Lambda^{(1)}_{n} F_0(n+1) t_0(n)
\big]\,,
\ \ n=\overline{n_0,n_1-1}\,;
\\[4pt]
T_0(n) & =
(-N^{(1)}_{n})^{-1}
\big[
(E_J \otimes \vec 1) +
\Lambda^{(1)}_{n}
T_0(n+1) +{}\\
{}+
&\hspace*{10mm}\Lambda^{(1)}_{n} \tilde F_0(n+1) T_0(n)
\big]\,,
\ \ n=\overline{n_0,n_1-1}\,;
\\[4pt]
t_k(n) & =
-N_{n}^{-1}
\Bigg[
\Lambda^{(1)}_{n}
\Bigg(
t_k(n+1)
+{}\\
&{}+
\sum\limits_{i=0}^{k}
F_i(n+1) t_{k-i}(n)
\Bigg)
+
\Lambda^{(2)}_{n} t_{k-1}(n)
\Bigg]\,,\\
& k=\overline{1,n_2-1}\,,
\enskip n=\overline{n_0,n_1-1}\,;
\\[4pt]
T_k(n) & =
-N_{n}^{-1}
\Bigg[
\Lambda^{(1)}_{n}
\Bigg(
T_k(n+1)
+{}\\
&{}+
\sum\limits_{i=0}^{k-1}
F_i(n+1) T_{k-i}(n)
+
\tilde F_k(n+1) T_0(n)
\Bigg)
+{}\\
&{}+
\Lambda^{(2)}_{n} T_{k-1}(n)
\Bigg]\,,
\ \ k=\overline{1,n_2}\,,
\ \ n=\overline{n_0,n_1-1}\,.
\end{align*}
Эти формулы фактически не отличаются от формул (\ref{t1})--(\ref{t8}).

Вводя обозначения $t_k=t_k(0)$,\  $T_k=T_k(0)$,
получаем для вектор-строк $\vec p^{\,(2)}_n$,\ \ $n=\overline{0,n_2}$,
стационарных вероятностей того, что в системе в очереди
неприоритетных заявок имеется $n$~заявок (теперь уже
с учетом фазы генерации), соотношения
\begin{equation*}
\vec p^{\,(2)}_0 = \fr{1}{\overline m}
\bigg(
-
\sum\limits_{k=0}^{n_0-1}
\vec p^{\,*}_k N_k^{-1} (E_J\otimes\vec 1)
+
\vec p^{\,*}_{n_0} t_0
\bigg)\,;
\end{equation*}
\begin{align*}
\vec p^{\,(2)}_n &= \fr{1}{\overline m}\,\sum\limits_{k=0}^{n}
\vec p^{\,*}_{n_0+k} t_{n-k}\,,
\ \ n=\overline{1,n_2-1}\,;
\\
\vec p^{\,(2)}_{n_2} &=
\fr{1}{\overline m}\,\sum\limits_{k=0}^{n_2-1} \vec p^{\,*}_{n_0+k} T_{n_2-k}\,.
\end{align*}
Стационарная вероятность потери неприоритетной заявки
определяется формулой
$$
p^{(2)}_{\mathrm{loss}} = \fr{1}{\lambda^{(2)}}\,
\vec p_{n_2}^{\,(2)} \Lambda^{(2)} \vec 1\,.
$$

Наконец, учитывая, что ПЛС $\beta^{(1)}(\omega)$ и
$\beta^{(2)}(\omega)$ времен обслуживания приоритетной и
неприоритетной заявок задаются выражениями
$\beta^{(1)}(\omega) =$\linebreak $= \vec b^{(1)} (\omega E - M^{(1)})^{-1} \vec\mu^{(1)}$
и
$\beta^{(2)}(\omega) = \vec b^{(2)} (\omega E -$\linebreak $- M^{(2)})^{-1} \vec\mu^{(2)}$,
получаем для ПЛС $v^{(1)}(\omega)$ и $v^{(2)}(\omega)$\linebreak стационарных
распределений $V^{(1)}(x)$ и $V^{(2)}(x)$ времен пребывания в системе
принятых к обслуживанию приоритетной и неприоритетной заявок следующие
формулы:
\begin{align*}
v^{(1)}(\omega) &=
w^{(1)}(\omega) \beta^{(1)}(\omega)\,;
\\
v^{(2)}(\omega) &=
w^{(2)}(\omega) \beta^{(2)}(\omega)\,.
\end{align*}

\section{Заключение}

Таким образом, в настоящей статье получены аналитические соотношения
для вычисления основных стационарных показателей функционирования
двухприоритетной СРКП и ее обобщения.
При этом соотношения для стационарных распределений очередей могут
быть использованы для непосредственного вычисления этих характеристик,
в то время как с помощью полученных в терминах ПЛС соотношений
для стационарных временных характеристик можно находить только
моменты.

Отметим, что стационарные распределения времен пребывания в системе
приоритетных и неприоритетных заявок можно получать в виде
степенных матричных рядов, однако такой способ приводит к очень
громоздким вычислениями, особенно для неприоритетных заявок.

{\small\frenchspacing
{%\baselineskip=10.8pt
\addcontentsline{toc}{section}{Литература}
\begin{thebibliography}{99}

\bibitem{4-p}
\Au{Grandjean C.\,H.} Traffic calculations in suturation routing
with priorities~// Electr.\ Commun., 1974. Vol.~49. No.\,1. P.~72--79.

\bibitem{5-p}
\Au{Людвиг Г., Рой Р.} Ограничения для сетей с волновым поиском
сетей~// Тр.\ ин-та инженеров электроники и радиоэлектроники, 1977.
Т.~65. №\,9. С.~154--165.

\bibitem{6-p}
\Au{Weber J.\,H.}
Some traffic characteristics of communications networks with
automatic alternate routing~//
Bell System Techn.\ J., 1962. March. P.~1201--1247.

\bibitem{7-p}
\Au{Weber J.\,H.}
Simulation study of routing and control in communications networks~//
Bell System Techn.\ J., 1964. Nov. P.~2639--2676.

\bibitem{8-p}
\Au{Grandjean C.\,H.} Call routing strategies in telecommunications
networks~// Electr.\ Commun., 1967. Vol.~42. No.\,3. P.~380--391.

\bibitem{9-p}
\Au{Джейсуол Н.}
Очереди с приоритетами.~--- М.: Мир, 1973.

\bibitem{10-p}
{\it Esoqbue A.\,O., Singh A.\,J.} A stochastic model for a optimal
priority bed distribution in a hospital~// Oper.\ Res., 1976. No.\,24. P.~884--889.

\bibitem{11-p}
\Au{Otterman J.}
Grande of service direct traffic mixed with store-and-forward traffic~//
Bell System Techn.\ J., 1962. Apr. P.~1415--1437.

\bibitem{12-p}
\Au{Liu F.\,K.} A combined delay and loss system with priority~//
ICC, 1973. Vol.~39. No.\,7. P.~39-7--39-13.

\bibitem{13-p}
\Au{Печинкин А.\,В., Федоров В.\,М.}
Методика расчета многоканальной системы приоритетного обслуживания
с резервированием каналов~//
Системное моделирование. Вып.~15.~--- Новосибирск: ВЦ СО АН СССР, 1990.

\bibitem{14-p}
\Au{Бурыгин С.\,В., Глазунов А.\,С., Печинкин~А.\,В.}
Система приоритетного обслуживания с резервированием каналов
и марковским входящим потоком~//
Вестник Российского ун-та дружбы народов.
Сер.\ Прикладная математика и информатика, 2001. №\,1. С.~80--89.

\bibitem{15-p}
\Au{Bocharov P.\,P., D'Apice C., Pechinkin~A.\,V., Salerno~S.} 
Queueing theory.~--- Utrecht--Boston: VSP, 2004.

\label{end\stat}

\bibitem{16-p}
\Au{Печинкин А.\,В., Чаплыгин В.\,В.}
Стационарные характеристики системы массового обслуживания
$SM/MSP/n/r$~// Автоматика и телемеханика, 2004. №\,9. С.~85--100.

 \end{thebibliography}
}
}


\end{multicols}  