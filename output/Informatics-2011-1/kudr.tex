

\def\stat{kudr}

\def\tit{ОБ УТОЧНЕНИИ НЕКОТОРЫХ РЕЗУЛЬТАТОВ ДЛЯ ОДНОЙ БАЙЕСОВСКОЙ
МОДЕЛИ МАССОВОГО ОБСЛУЖИВАНИЯ$^*$}

\def\titkol{Об уточнении некоторых результатов для одной байесовской
модели массового обслуживания}

\def\autkol{А.\,А.~Кудрявцев, С.\,Я.~Шоргин}
\def\aut{А.\,А.~Кудрявцев$^1$, С.\,Я.~Шоргин$^2$}

\titel{\tit}{\aut}{\autkol}{\titkol}

{\renewcommand{\thefootnote}{\fnsymbol{footnote}}\footnotetext[1]
{Работа выполнена при
поддержке РФФИ, проекты 11-01-00515 и  11-07-00112.}}

\renewcommand{\thefootnote}{\arabic{footnote}}
\footnotetext[1]{Московский государственный университет им.\ М.\,В.~Ломоносова, 
факультет вычислительной математики и кибернетики,
nubigena@hotmail.com}
\footnotetext[2]{ИПИ РАН, sshorgin@ipiran.ru}

\vspace*{6pt}

\Abst{Данная работа посвящена одному принципиальному
уточнению опубликованных в работе~[1] характеристик распределения
вероятности непотери вызова в рамках байесовской модели массового
обслуживания и надежности. Приводится исправленная формула для
соответствующей плотности и выражения для первых двух моментов в
несколько упрощенном виде.}

\KW{байесовский подход; системы массового обслуживания; надежность; смешанные
распределения; моделирование}

      \vskip 20pt plus 9pt minus 6pt

      \thispagestyle{headings}

      \begin{multicols}{2}
      
            \label{st\stat}

В работе~[1] в рамках байесовского подхода рассматривается система
$M|M|1|0$, в которой интенсивность входящего потока~$\lambda$ имеет
вырожденное распределение, а интенсивность обслуживания $\mu$~---
распределение Эрланга с параметрами~$n$ и~$\alpha$.

Одной из задач для такой модели является нахождение основных вероятностных характеристик вероятности не потерять вызов
$$
\pi=\fr{1}{1+\rho}\,,
$$
где $\rho$~--- коэффициент загрузки системы.

В работе~[1] показано, что функция распределения случайной величины~$\pi$ имеет вид
\begin{equation*}
F_\pi(x)=1-e^{-{\alpha\lambda
x}/(1-x)}\sum_{k=0}^{n-1}\fr{(\alpha\lambda)^k x^k}{(1-x)^k k!}\,,\enskip x\in(0,\ 1)\,.
\end{equation*}

В этом случае плотность~$\pi$, как легко видеть, вычисляется по формуле, отличной от приведенной в работе~[1], а
именно:
\begin{multline*}
\hspace*{-5.127pt}f_\pi(x)=e^{-{\alpha\lambda x}/{(1-x)}}\sum_{k=0}^{n-1}\fr{(\alpha\lambda)^k x^{k-1}(\alpha\lambda x+k
x-k)}{k!(1-x)^{k+2}}\,,\\
 x\in(0,\ 1)\,.
 \end{multline*}

Приведем также явные выражения для первых двух моментов~$\pi$ в менее громоздком виде по сравнению с результатами,
опубликованными в~[1]. Для этого воспользуемся формулой~3.353.5 из~[2].

Имеем
\begin{multline*}
\e\pi={}\\
{}=\il{0}{1}e^{-{\alpha\lambda x}/(1-x)} \sum_{k=0}^{n-1}\fr{(\alpha\lambda)^k x^k(\alpha\lambda x +k x
-k)}{k! (1-x)^{k+2}}\, dx={}\\
{}=\il{0}{\infty}e^{-z} \sum_{k=0}^{n-1}\fr{z^k(z-k)}{k!(\alpha\lambda+z)}\, dz={}\\
{}=
\sum_{k=0}^{n-1}\fr{1}{k!}\left[\il{0}{\infty}\fr{e^{-z}z^{k+1}}{\alpha\lambda+z}\,
dz-k\il{0}{\infty}\fr{e^{-z}z^k}{\alpha\lambda+z}\, dz\right]\,.
\end{multline*}

Для упрощения дальнейших выкладок введем следующее обозначение.
Пусть $Ei(x)$~--- интегральная показательная функция
$$
Ei(x)=-\il{-x}{\infty}\fr{e^{-t}}{t}\,dt\,.
$$

Используя~[2, формула~3.353.5], получаем
\begin{multline*}
\e\pi=
\sum_{k=0}^{n-1}\fr{1}{k!}\left[
\vphantom{\sum_{m=1}^{k+1}}
(-1)^k(\alpha\lambda)^{k+1}e^{\alpha\lambda}Ei(-\alpha\lambda)+{}\right.\\
{}+\sum_{m=1}^{k+1}
(m-1)!(-1)^{k-m+1}(\alpha\lambda)^{k-m+1}-{}\\
{}-k(-1)^{k-1}(\alpha\lambda)^ke^{\alpha\lambda}Ei(-\alpha\lambda)-{}\\
\left.{}-k\sum_{m=1}^k(m-1)!(-1)^{k-m}(\alpha\lambda)^{k-m}
\right]={}
\end{multline*}
\begin{multline*}
{}
=n+\sum_{k=0}^{n-1}\fr{(-\alpha\lambda)^k(\alpha\lambda+k)}{k!}\left[\vphantom{\sum\limits_m^k}e^{\alpha\lambda}Ei(-\alpha\lambda)-{}\right.\\
{}- \left.
\sum_{m=1}^{k}\fr{(m-1)!}{(-\alpha\lambda)^m}\right]\,.
\end{multline*}

Для второго момента $\pi$ имеем аналогично
\begin{multline*}
\e\pi^2={}\\
{}=\!\il{0}{1}\!\!e^{-{\alpha\lambda x}/(1-x)}\sum\limits_{k=0}^{n-1}\fr{(\alpha\lambda)^k x^{k+1}(\alpha\lambda
x+kx-k)}{k!(1-x)^{k+2}}\, dx={}\\
{}=\sum\limits_{k=0}^{n-1}\fr{1}{k!}\left[\il{0}{\infty}\fr{e^{-z}z^{k+2}}{(\alpha\lambda+z)^2}\,
dz-k\il{0}{\infty}\fr{e^{-z}z^{k+1}}{(\alpha\lambda+z)^2}\, dz\right]={}\\
{}=\alpha\lambda n+\fr{n^2+n}{2}+{}\hspace*{10mm}
\end{multline*}
\begin{multline*}
\hspace*{5mm}{}+\sum\limits_{k=0}^{n-1}\fr{(-\alpha\lambda)^k((\alpha\lambda+k+1)^2-k-1)}{k!}\times{}\\
{}\times\left[e^{\alpha\lambda}Ei(-\alpha\lambda)-\sum\limits_{m=1}^k\fr{(m-1)!}{(-\alpha\lambda)^m}\right]\,.
\end{multline*}

{\small\frenchspacing
{%\baselineskip=10.8pt
%\addcontentsline{toc}{section}{Литература}
\begin{thebibliography}{9}

\bibitem{KuSh08}
\Au{Кудрявцев А.\,А., Шоргин~С.\,Я.}
Байесовский подход к анализу систем массового обслуживания и показателей надежности~// Информатика и её 
применения, 2007. Т.~1. Вып.~2. С.~76--82.

 \label{end\stat}
 

\bibitem{GR71}
\Au{Градштейн И.\,С., Рыжик~И.\,М.}
Таблицы интегралов, сумм, рядов и произведений.~--- М.: Наука, 1971. 1108~с.
 \end{thebibliography}
}
}


\end{multicols}  