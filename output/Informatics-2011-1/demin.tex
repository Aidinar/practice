\def\stat{dem}

\def\tit{О ДВУХ МОДЕЛЯХ РАСПРЕДЕЛЕНИЯ РЕСУРСОВ 
ПРИ~ОРГАНИЗАЦИИ ИНВЕСТИЦИОННЫХ ПРОЦЕССОВ}

\def\titkol{О двух моделях распределения ресурсов 
при~организации инвестиционных процессов}

\def\autkol{П.\,В.~Демин}
\def\aut{П.\,В.~Демин$^1$}

\titel{\tit}{\aut}{\autkol}{\titkol}

%{\renewcommand{\thefootnote}{\fnsymbol{footnote}}\footnotetext[1]
%{Работа выполнена при поддержке ФЦП <<Научные и научно-педагогические кадры инновационной России>> 
%на 2009--2013~гг.}}

\renewcommand{\thefootnote}{\arabic{footnote}}
\footnotetext[1]{Государственное образовательное учреждение <<Московская академия рынка труда и 
информационных технологий>>, pdemin@mail.ru}

\vspace*{-12pt}
 
      
\Abst{Приводятся два примера решения проблем, возникающих 
при организации инвестиционных процессов, связанных с модернизацией экономики. 
Первый относится к проблеме, возникающей в банке при решении вопроса выбора для 
финансирования из некоторого множества проектов. Второй~--- к решению задачи 
распределения инвестиционного ресурса между предприятиями, входящими в состав 
холдинга.}

\vspace*{-2pt}
      
      \KW{инвестиционный процесс; инновация; банковское финансирование; ресурсы; 
холдинг}

      \vskip 8pt plus 9pt minus 6pt

      \thispagestyle{headings}

      \begin{multicols}{2}
      
            \label{st\stat}


\section{Введение}

В данной работе приводятся два примера решения проблем, возникающих 
при организации инвестиционных процессов, связанных с модернизацией 
экономики. Первый относится к проблеме, возникающей в банке при 
решении вопроса выбора для финансирования из некоторого множества 
проектов. Второй~--- к решению задачи распределения инвестиционного 
ресурса между предприятиями, входящими в состав холдинга. В~первом 
случае проблему удается свести к оптимальной дискретной задаче, в которой 
оптимизируется суммарный доход, полученный банком за рассматриваемый 
период времени. В~качестве ограничений рассматриваются лимиты на 
доступные ресурсы в каждый из рассматриваемых моментов времени. Во 
втором случае решение сводится к построению многоэтапной модели 
распределения инвестиционных ресурсов. Обе задачи решаются при наличии 
как в банке, так и в холдинге базы первичных данных о предприятиях в 
целом, позволяющих рассчитывать необходимые для решения параметры и 
функции. Это объединяет оба примера. В~силу сложности и размерности 
возникающих оптимизационных задач для решения используются элементы 
эвристических методик.

\section{Модель выбора в банке проектов для финансирования}
     
     Важнейшим условием реализации инновационного проекта является 
наличие ресурсного обеспечения. Многое в процессе поиска и 
предоставления необходимых ресурсов зависит от банка, клиентом которого 
является субъект, планирующий эту реализацию. Рассмотрим модель, 
позволяющую банку выбрать вариант финансирования подмножества 
проектов из некоторого множества ин\-вес\-ти\-ци\-он\-ных проектов. Учитывая, что 
такие проекты, как правило, отличаются большими сроками реализации и 
объемами финансирования (медленные процессы), задача решается на фоне 
текущей кредитной деятельности банка (быстрые процессы). Общие 
характеристики текущей деятельности банка входят параметрами в модель 
выбора подмножества инвестиционных проектов.

\subsection{Формальная постановка} %2.1
     
     При формализации задачи важным является понятие варианта 
финансирования инвестиционного проекта. Пусть~$I$~--- множество 
проектов, которые находятся в портфеле у банка и из числа которых банк 
может выбирать проекты для финансирования в соответствии с наличием 
ресурсов и представлениями о целесообразности их выделения. Введем 
переменные~$x_{iq}$, где $q\in Q_i$, $i\in I$, $Q_i$~--- множество вариантов 
финансирования $i$-го  проекта. Варианты могут различаться, например, 
сроками начала финансирования или объемами ресурсов, необходимых в 
конкретные моменты времени. Будем полагать, что
     $$
     x_{iq}= 
     \begin{cases}
     1, & \mbox{если проект}\ i\in I \mbox{~~финансируется}\\
     & \mbox{по варианту}\ q\in  Q_i\,;\\
     0 & \mbox{иначе.}
     \end{cases}
     $$
     

Обозначим
\begin{gather*}
x=\left \{ x_{iq}/x_{iq} =0, 1, \enskip q\in Q_i, \ i\in I\right \}\,,\\
x\in X=\left \{ x/ \sum\limits_q x_{iq}\leq 1, \enskip i\in I\right \}\,.
\end{gather*}

     
     Введем величины~$a_{iqt}$, $b_{iqt}$, где $a_{iqt}$~--- объем 
ресурсов, необходимых для финансирования проекта $i\in I$ в момент $t\in 
[0,\,T]$, если финансирование происходит по варианту $q\in Q_i$, 
$b_{iqt}$~--- объем ресурсов, который возвращается банку в момент $t\in 
[0,\,T]$ при финансировании проекта по варианту $q\in Q_i$.

Будем также считать, что текущая деятельность банка заключается в 
обслуживании кредитных и депозитных договоров и ее можно описать 
сле\-ду\-ющи\-ми зависимостями ($t\in [0,\,T]$):
\begin{description}
%\smallskip
%\noindent
\item[\,] $z(t)$~--- средний суммарный остаток на расчетных счетах банка в момент 
времени~$t$; 
\item[\,]
%\noindent
$d(t)$~--- средний суммарный остаток на депозитных счетах в момент 
времени~$t$;
\item[\,]
%\noindent
$S(t)$~--- средняя суммарная величина выплат процентов по депозитным 
счетам в момент времени~$t$; 
\item[\,]
%\noindent
$A(t)$~--- величина предоставляемых в момент времени~$t$ кредитных 
ресурсов;
\item[\,]
%\noindent
$B(t)$~--- величина возвращаемых в момент времени~$t$ кредитных 
ресурсов.
\end{description}
%\smallskip
     
     Тогда задачу выбора вариантов финансирования инвестиционных 
проектов можно записать в виде
     \begin{equation}
     \underset{x\in X}{\max} \sum\limits_i \sum\limits_q \left[ \sum\limits_t 
b_{iqt}-a_i^0\right ] x_{iq}\,,\label{e2.1.1d}
     \end{equation}
     
     \vspace*{-12pt}
     
     \noindent
     \begin{multline}
     \sum\limits_{\tau=1}^t\left[\sum\limits_i\sum\limits_q a_{iq\tau} 
x_{iq}+A(\tau)\right] \leq z(t)+d(t)+{}\\
\!\!{}+\sum\limits_{\tau=1}^t\left[ B(\tau)-S(\tau)+\sum\limits_i\sum\limits_q 
b_{iq\tau} x_{iq}\right ]\!\!,\ t\in [0,\,T],\!\!\!
     \label{e2.1.2d}
     \end{multline}
где $a_i^0$~--- суммарная величина ресурсов, необходимых для 
финансирования $i$-го проекта.
     
     Задача~(\ref{e2.1.1d}), (\ref{e2.1.2d}) имеет прозрачный физический 
смысл: для финансирования выбираются проекты, которые приносят 
максимальный доход банку, при этом сумма размещенных активов в каждый 
момент времени не должна превышать текущих пассивов. (Здесь для 
простоты изложения будем считать, что собственные средства банка 
полностью расходуются на предоставление краткосрочных кредитов). 

\subsection{Построение решения поставленной задачи} %2.2
     
     Задачу~(\ref{e2.1.1d}), (\ref{e2.1.2d}) запишем в виде:
     \begin{equation}
     \underset{x\in X}{\max} \left \{f(x)\vert g(x)\geq 0\right \}\,,
     \label{e2.2.1d}
     \end{equation}
где $f(x)$~--- целевая функция, $g(x)\geq 0$~--- вектор ограничений. Пусть 
$L(x,y)=f(x)+y\times g(x)$~--- функция Лагранжа для задачи~(\ref{e2.2.1d}). 
Известно, что $\underset{x\in X}{\max}\,\underset{y\geq 0}{\min} L(x,y) \leq 
\underset{y\geq 0}{\min}\,\underset{x\in X}{\max} L(x,y)=\omega^*$.
     
     Нахождение $\underset{x\in X}{\max}\,\underset{y\geq 0}{\min} L(x,y)$ 
соответствует решению исходной задачи~(\ref{e2.2.1d}). Задачу 
$\underset{y\geq 0}{\min}\,\underset{x\in X}{\max} L(x,y)$ называют 
двойственной. Решение двойственной задачи позволяет получить оценку 
сверху~$\omega^*$ для значения исходной задачи и служит эвристическим 
способом построения допустимого решения для исходной задачи. 
Обоснование метода решения двойственной задачи содержится 
     в~\cite{1-dem} . Его идея заключается в следующем. Можно показать, 
что
\vspace*{-8pt}

\noindent
     \begin{multline}
     \omega^*=\underset{\omega,y}{\min}\left\{\omega\vert\omega\geq 
f(x)+y\times g(x)\,,\right. \\[-2pt]
\left. x\in X_y\,, \ y\geq o\right \}\,,
     \label{e2.2.2d}
     \end{multline}
где $X_y=\mathrm{Arg}\, \underset{x}{\max}\left\{ L(x,y)\vert y\geq 0,\ x\in X\right \}$, 
т.\,е.\ множество решений, доставляющих максимум функции Лагранжа при 
всех возможных неотрицательных значениях вектора двойственных 
переменных. Задачу~(\ref{e2.2.2d}) можно решать методами линейного 
программирования, итерационно добавляя существенные ограничения и 
отбрасывая несущественные. При этом из построенных элементов 
множества~$X_y$, допустимых для исходной задачи, выбирается такой, 
который максимизирует целевую функцию исходной задачи. Несмотря на то,
что строго доказать существование допустимого для исходной задачи 
элемента множества~$X_y$ удалось только для одного ограничения, 
реальные вычисления показывают, что такой элемент получается 
практически всегда.

\vspace*{-5pt}

\section{Модель распределения ресурсов между предприятиями 
холдинга}

\vspace*{-3pt}
     
     В период инновационного развития экономики в крупных холдингах 
возникает задача распределения ресурсов развития между предприятиями с 
таким расчетом, чтобы получить максимальный желаемый эффект. 
Рассмотрим многоэтапную модель распределения ресурсов, влияющих на 
инновационное развитие предприятий, входящих в холдинг. Решения о 
распределении ресурсов развития принимаются (как это обычно и 
происходит в жизни) в дискретные моменты времени. Предложенный 
алгоритм оптимизации не претендует на глобальность, но прозрачен и прост 
в реализации. Его можно рассматривать как инструмент при скользящем 
планировании распределения ресурсов, когда решения принимаются на шаг 
вперед.
     
     
     \vspace*{-4pt}
     
\subsection{Описание холдинговой структуры} %3.1

\vspace*{-1pt}
     
     Рассмотрим организационную структуру холдингового типа. 
Предприятия связаны холдинго-\linebreak\vspace*{-12pt}
\pagebreak

\noindent
выми отношениями, позволяющими одному 
из них (головной компании) определять решения, принимаемые другими 
участниками холдинга. 
     
     Головная компания осуществляет: централизованное управление 
активами холдинга; управление корпоративной стратегией; мониторинг 
процессов, проходящих в холдинге в необходимом временном режиме; 
контроль интегральных и мониторинг частных целевых показателей 
     биз\-нес-планов.
     
\vspace*{-3pt}

\subsection{Модель холдинговой системы} %3.2

%\vspace*{-1pt}

     
     Рассматриваемая система является иерархической, с вертикальными 
связями. В~системе имеется управляющий центр и элементы более низкого 
уровня. 
     
     Пусть в состав холдинговой структуры входит $N$~предприятий, 
     $t$~--- номер дискретного интервала времени, на котором 
рассматривается деятельность холдинга, всего рассматривается 
$T$~интервалов времени. Предположим, что для $n$-го элемента 
холдинговой системы можно построить функцию~$F_n(z_{nt})$, где 
$F_n(\cdot)$~--- объем выпущенной продукции (в денежном выражении) на 
конец рассматриваемого периода; $z_{nt}$~--- объем средств, направляемых 
на улучшение и модернизацию $n$-го производства в интервале 
времени~$t$. Обозначим $z=\{ z_t\vert t, \ldots , T\}$, $z_t=\{z_{nt}\vert 
n=1,\ldots ,N\}$.
     
     Целью управляющего центра является рост консолидированной 
прибыли холдинга на протяжении долгосрочного периода. В~качестве 
критерия развития холдинга примем функционал $\sum\limits_n b_n 
F_n(z_{nT})$. Таким образом, рассматривается задача нахождения 

\noindent
     \begin{equation}
     \underset{z}{\max} \sum\limits_n b_n F_n(z_{nT})
     \label{e3.2.1d}
     \end{equation}
при выполнении ограничений

\vspace*{-6pt}

\noindent
\begin{multline}
\sum\limits_n z_{nt}\leq \lambda \sum\limits_n F_n(z_{n\,t-1})\,,\\[-6pt]
 t=1,2\ldots 
, T\,, \ z\geq 0\,.
\label{e3.2.2d}
\end{multline}
     
     Параметр~$\lambda$ считается заданным центром. Предполагается, что 
для $t=0$ известен начальный распределяемый инвестиционный ресурс. 
     
     Вместо решения задачи~(\ref{e3.2.1d}), (\ref{e3.2.2d}) будем для 
каж\-до\-го~$t$, начиная с~1, последовательно решать оптимальные задачи 

\vspace*{-6pt}

\noindent
     \begin{multline}
     \underset{z_t}{\max}\left \{ \sum\limits_n b_n F_n(z_{nt})\left\vert 
\sum\limits_n z_{nt}\leq \lambda \sum\limits_n F_n(z_{n\,t-1})\right.\right.\,,\\[-6pt]
\left. \vphantom{\sum\limits_n} z_t\geq 0
\right \}
\,.
     \label{e3.2.3d}
     \end{multline}

Оптимизацию по $z_t$ для каждого~$t$ можно проводить с помощью 
процедуры динамического программирования~[2]. Не вдаваясь в 
подроб-\linebreak\vspace*{-12pt}
\columnbreak

\noindent
ности этой хорошо известной процедуры, отметим только, что в ходе 
ее реализации при перемещении от~$N$ до~1 строятся функции 
$\varphi_n(c)=$\linebreak $=\underset{z_t}{\max} \left \{\sum\limits_{k=n}^N b_k 
F_k(z_{kt})\left\vert \sum\limits_{k=n}^N z_{kt}\right. \leq c,\ c\geq o\right \}$, 
удовле\-творяющие рекуррентному соотношению $\varphi_{n-
1}(c)\!=\!\underset{z_{n-1},t}{\max}\!\left\{\varphi_n(c-z_{n-1,t})+b_{n-1}F_{n-
1}(z_{n-1,t})\right \}$. Затем при перемещении в обратном направлении 
получают оптимальное решение задачи~(\ref{e3.2.3d}), т.\,е.\ находят 
распределение инвестиций по предприятиям, входящим в холдинг.

\vspace*{-9pt}

\section{Заключение}

\vspace*{-6pt}
     
     В работе рассмотрены две модели, связанные с организацией 
финансирования инвестиционных процессов. Одна модель описывает 
процесс выбора проектов для финансирования в банке. Вторая~--- процесс 
распределения инвестиционных ресурсов между предприятиями, входящими 
в состав холдинга. 
     
     Задачи, встающие при организации инвестиционных процессов, 
подобные рассмотренным выше, возникают и в банках, и в крупных 
холдингах повсеместно. Если реализовать приведенные модели в 
соответствующих интерактивных системах для организации финансирования 
инвестиционных процессов, то практическая польза окажется существенно 
выше, чем при простом поиске оптимальных решений поставленных задач. 
Например,\linebreak кредитное управление банка чрезвычайно заин\-тересовано в 
оценке дохода ($\omega^*$), который можно получить при имеющемся 
портфеле проектов\linebreak и прогнозе доступных ресурсов. Более того, 
эффективность решений по привлечению ресурсов\linebreak банком может быть 
существенно повышена, если заранее указать моменты времени, в которые 
намечается избыток или недостаток ресурсов, а это модель позволяет 
сделать. 
     
     Несмотря на востребованность, постановочные работы подобного 
плана практически отсутствуют. В~какой-то мере это объясняется тем, что 
необходимые для моделей параметрические зависимости рассчитываются 
при наличии в организации базы первичных данных по организации в целом.

\vspace*{-14pt}

{\small\frenchspacing
{%\baselineskip=10.8pt
\addcontentsline{toc}{section}{Литература}
\begin{thebibliography}{9}

\vspace*{-2pt}

\bibitem{1-dem}
\Au{Демин В.\,К., Малашенко~Ю.\,Е.}
Получение оценочных решений для задач оптимального резервирования~// Известия АН 
СССР, Техническая кибернетика, 1974. №\,1. С.~112--117.


 \label{end\stat}

\bibitem{2-dem}
\Au{Замков О.\,И., Черемных~Ю.\,А., Толстопятенко~А.\,В.}
Математические методы в экономике: Учебник.~--- 4-е изд., стереотип.~--- М.: Дело и сервис, 
2004. 368~с.
 \end{thebibliography}
}
}


\end{multicols}  