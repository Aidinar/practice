\newcommand{\R}{\mathbb R}
\newcommand{\dd}{3}
\renewcommand{\d}{1}
\newcommand{\betr}{\beta_{3}}
\newcommand{\Lo}{\fr{\betr}{\sqrt{n}}}
\newcommand{\Ll}{\fr{\betr+1}{\sqrt{n}}}
\newcommand{\pto}{\stackrel{P}{\longrightarrow}}

\def\stat{gavr}

\def\tit{УТОЧНЕНИЕ НЕРАВНОМЕРНОЙ ОЦЕНКИ СКОРОСТИ СХОДИМОСТИ
РАСПРЕДЕЛЕНИЙ ПУАССОНОВСКИХ СЛУЧАЙНЫХ СУММ К НОРМАЛЬНОМУ
ЗАКОНУ$^*$}

\def\titkol{Уточнение неравномерной оценки скорости сходимости
распределений пуассоновских случайных сумм % к нормальному закону
}

\def\autkol{С.\,В.~Гавриленко}
\def\aut{С.\,В.~Гавриленко$^1$}

\titel{\tit}{\aut}{\autkol}{\titkol}

{\renewcommand{\thefootnote}{\fnsymbol{footnote}}\footnotetext[1]
{Работа выполнена при поддержке Министерства
образования и науки (государственный контракт 16.740.11.0133 от
02.09.2010).}}

\renewcommand{\thefootnote}{\arabic{footnote}}
\footnotetext[1]{Московский государственный университет им.\ М.\,В.~Ломоносова, 
факультет вычислительной математики и кибернетики,
gavrilenko.cmc@gmail.com}


\Abst{Строятся неравномерные оценки скорости сходимости
в классической центральной предельной теореме с уточненной
структурой. С~помощью этих структурных уточнений показано, что
абсолютная константа в неравномерной оценке скорости сходимости в
центральной предельной теореме (ЦПТ) для пуассоновских случайных сумм
строго меньше, чем аналогичная константа в неравномерной оценке
скорости сходимости в классической ЦПТ,
и при условии существования третьих моментов слагаемых не
превосходит 22,7707. В~качестве следствия построены неравномерные
оценки скорости сходимости смешанных пуассоновских, в частности
отрицательных биномиальных случайных сумм.}

\KW{центральная предельная теорема; скорость
сходимости; неравномерная оценка; абсолютная константа;
пуассоновская случайная сумма; смешанное пуассоновское
распределение}

      \vskip 14pt plus 9pt minus 6pt

      \thispagestyle{headings}

      \begin{multicols}{2}
      
            \label{st\stat}

\section{Введение}

Пусть $X_1,X_2,\ldots$~--- последовательность независимых одинаково
распределенных случайных величин таких, что ${\sf E}X_1=0$, ${\sf
E}X_1^2=1$, ${\sf E}|X_1|^{\dd}=\betr<\infty$. Положим $F_n(x)={\sf
P}\big(X_1+\ldots+X_n<x \sqrt{n}\big)$. Пусть $\Phi(x)$~--- стандартная нормальная функция распределения, т.\,е.\
$$
\Phi(x) = \fr{1}{\sqrt{2\pi}}\int_{-\infty}^x e^{-t^2/2}\,dt\,.
$$

Известно, что при указанных условиях существуют абсолютные
положительные конечные константы $C_0$ и $C_1$ такие, что~[1, 2]
\begin{equation}
\sup_x|F_n (x)-\Phi(x)|\le C_0 \Lo
\label{e1gv}
\end{equation}
 и~[3, 4]:
\begin{multline}
\sup_x\left|F_n(x)-\Phi(x)\right|\le C_1\Ll={}\\
{}=
C_1\left(1+\fr{1}{\betr}\right)\Lo\label{e2gv}\,.
\end{multline}
 Для констант $C_0$ и $C_1$ известны следующие численные
оценки~[4, 5]:
\begin{equation*}
0{,}4097\approx\fr{\sqrt{10}+3}{6\sqrt{2\pi}}\le C_0\le
0{,}4784
%\label{e3gv}
\end{equation*}
и~[3, 4]:
\begin{equation*}
0{,}2659\approx\fr{2}{3\sqrt{2\pi}}\le C_1\le 0{,}3041\,.
%\label{e4gv}
\end{equation*}
При этом, поскольку всегда $\betr\ge1$, при больших значениях
$\betr$ оценка~(\ref{e2gv}) точнее, чем~(\ref{e1gv}), за счет меньших значений
абсолютных констант.

Оценка скорости сходимости $F_n(x)$ к $\Phi(x)$,
устанавливаемая неравенствами~(\ref{e1gv}) и~(\ref{e2gv}), \textit{равномерна} по~$x$.
Но поскольку и $F_n(x)$, и $\Phi(x)$~--- функции распределения, то
должно выполняться соотношение $|F_n(x)-\Phi(x)|\rightarrow 0$ при
$|x|\to0$. Это обстоятельство не учитывается в равномерных
оценках. Вместе с тем точность нормальной аппроксимации для
функции распределения сумм случайных величин именно при больших
значениях аргумента представляет особый интерес, например, при
вычислении рисков критически больших потерь. В~данной статье будут
рассмотрены неравномерные оценки скорости сходимости в центральной
предельной теореме.

По-видимому, исторически первая оценка такого рода была получена в
работе~\cite{Meshalkin}, где для $\delta=1$, то есть для случая
существования третьего момента слагаемых, было доказано
существование конечной положительной абсолютной постоянной~$A$
такой, что для любого $x\in\R$ справедливо неравенство
$$
(1+x^2)|F_n(x)-\Phi(x)|\le A\fr{\beta_3}{\sqrt{n}}\,.
$$

Этот результат был усилен в работе~\cite{Nagaev}, где было
показано, что существует такое положительное конечное число~$C$,
что
\begin{equation}
\sup_x\left(1+|x|^{\dd}\right)\left|F_n(x)-\Phi(x)\right|\le
C\Lo\,.\label{e5gv}
\end{equation}
При этом для рассматриваемых условий на моменты слагаемых порядок
оценки~(\ref{e5gv}) по $x$ неулучшаем без дополнительных предположений.

Что касается значения абсолютной константы~$C$ в~(\ref{e5gv}), то в работе~\cite{Mich81} 
было показано, что $C\le C_0+8(1+e)$, что с учетом
оценки $C_0\le 0{,}4784$, полученной в~\cite{KorolevBEs, KorSchev}, влечет оценку $C\le
30{,}2247$. Недавно эта оценка была уточнена в работе~\cite{Nefedova},
где было показано, что $C \le 25{,}7984$.

В данной работе с помощью модификации метода Л.~Падитца~\cite{Paditz89} будут построены альтернативные неравномерные
оценки скорости сходимости в центральной предельной теореме,
имеющие структуру, аналогичную неравенству~(\ref{e2gv}). Полученные оценки
затем будут использованы для уточнения абсолютной константы в
аналоге неравенства~(\ref{e5gv}) для пуассоновских случайных сумм.

Согласно результатам работы~\cite{Mich93}, в качестве абсолютной
константы в неравномерной оценке скорости сходимости в центральной
предельной тео\-ре\-ме для пуассоновских случайных сумм можно брать
абсолютную константу~$C$ из неравенства~(\ref{e5gv}). В~предлагаемой статье
с использованием упомянутых выше структурных уточнений неравенства~(\ref{e5gv})\linebreak 
будет показано, что на самом деле абсолютная константа в
неравномерной оценке ско\-рости схо\-ди\-мости в центральной предельной
теореме для пуассоновских случайных сумм строго меньше, чем\linebreak
аналогичная константа в неравенстве~(\ref{e5gv}), и не превосходит
22,7707. С~помощью этих результатов затем будут построены
неравномерные оценки скорости сходимости распределений смешанных
пуассоновских, в частности отрицательных биномиальных, случайных
сумм.

\section{Неравномерные оценки скорости сходимости в~центральной предельной теореме с~уточненной структурой}

Идея, лежащая в основе метода построения неравномерных оценок
точности нормальной аппроксимации для распределений сумм независимых
случайных величин, описанного в работе~\cite{Paditz89}, заключается
в подходящем разбиении вещественной прямой на зоны <<малых>>,
<<умеренных>> и <<больших>> значений~$x$. Традиционно используются
разбиения следующего вида:
\begin{enumerate}[$i$]
\item
<<малые>> значения $x$: $0\le x^2\le K^2$;
\item
<<умеренные>> значения $x$: $K^2\le x^2\le c_n(x;a,b)$;
\item
<<большие>> значения $x$: $c_n(x;a,b)\le x^2<\infty$,
\end{enumerate}
где $K>0$, $a>0$, $b>1$~--- вспомогательные свободные
параметры,
$$
c_n(x;a,b)=2b\left[\log|x|^{\dd}-\log\left(a\Lo\right)\right]
$$
(см., в частности,~\cite{Paditz89, Rychlik}).

\subsection{Случаи \boldmath{$i$} и \boldmath{$iii$}, то есть <<малые>> и~<<большие>> значения \boldmath{$x$}}

В случае~$i$, т.\,е.\ для $0\le |x|\le K$, в соответствии с
неравенством~(\ref{e2gv}) имеем
\begin{multline}
|x|^{\dd}\left|F_n(x)-\Phi(x)\right|\le
C_1K^{\dd}\Ll={}\\
{}=C_1K^{\dd}\Lo+C_1K^{\dd}\fr{1}{n^{\d/2}}\,. 
\label{e6gv}
\end{multline}


В случае же $iii$, т.\,е.\ для
$x^2\in[2b(\log|x|^{\dd}\hm-\log(a\beta_3/\sqrt{n})),\infty]$, в
соответствии с работами~\cite{Paditz81, Tysiak} справедлив следующий результат.
Обозначим
$$
P(a,b,K)=(2b)^{\dd}+a\exp\left\{\fr{(2b)^{\dd}}{a}-\fr{(b-1)K^2}{2b}\right\}\,.
$$

\smallskip

\noindent
\textbf{Лемма 2.1}. \textit{Предположим, что $x^2\ge c_n(x;a,b)\ge$\linebreak
$\ge K^2\ge (2\pi)^{-1}$. Тогда для любого $n\ge1$}
\begin{equation*}
|x|^{\dd}\left|F_n(x)-\Phi(x)\right|\le P(a,b,K)\Lo\,.
%\label{e7gv}
\end{equation*}

\smallskip

\noindent
Д\,о\,к\,а\,з\,а\,т\,е\,л\,ь\,с\,т\,в\,о\,\ см.\ в работах~\cite{Paditz81, Tysiak}.

\subsection{Случай \boldmath{$ii$}, то есть <<умеренные>> значения~\boldmath{$x$}}

Рассмотрение этого случая базируется на следующем фундаментальном
неравенстве (см.~\cite{Mich81} и~\cite{Tysiak}), в котором без
ограничения общности $x\ge0$:
\begin{multline}
\left|F_n(x)-\Phi(x)\right|\le n{\sf P}(|X_1|>y)
+
\left|
\vphantom{\fr{1}{2}}
f^n(h)-{}\right.\\
\left.{}-\exp\left\{\fr{1}{2}\,h^2n\right\}\right|
\exp\left\{-hx\sqrt{n}\right\}{\sf P}\left(S_n^*>x\sqrt{n}\right)+{}\\
{}+
2\exp\left\{\fr{1}{2}\,h^2n-hx\sqrt{n}\right\}\cdot\sup_{u\ge x}
\left|{\sf P}\left(S_n^*<u\sqrt{n}\right)-{}\right.\\
\left.{}-\Phi\left(u-h\sqrt{n}\right)\right|\,,
\label{e8gv}
\end{multline}
где $f(h)={\sf E}\exp\left\{h\overline{X}_1\right\}$,
$\overline{X}_1=X_1\I\{|X_1|<y\}$, $S_n^*=X_1^*+\ldots+X_n^*$~---
сумма независимых одинаково распределенных случайных величин с
общей функцией распределения
\begin{gather*}
{\sf P}(X_1^*<u)=\fr{1}{f(h)}\int\limits_{-\infty}^{u}e^{ht}\,d{\sf
P}\left(\overline{X}_1<t\right)\,,\\
y=\gamma x\sqrt{n}\,,\enskip h=\fr{(1-\gamma)x}{\sqrt{n}}\,,\enskip \gamma\in\left(0,\fr{1}{2}\right)\,.
\end{gather*}

Для начала сформулируем два утверждения, которые будут
неоднократно использоваться в дальнейшем. Во-первых, если $x^2\le
c_n(x;a,b)$, то (см.~$ii$)
\begin{equation}
\Lo\le\fr{|x|^{\dd}}{a}\,\exp\left\{-\fr{x^2}{2b}\right\}\,.\label{e9gv}
\end{equation}
Во-вторых, если $x^2\ge K^2$, то (см.~\cite{Paditz81})
\begin{equation}
x^r\exp\left\{-sx^2\right\}\le K^r\exp\left\{-sK^2\right\}\label{e10gv}
\end{equation}
при $x\ge\sqrt{r/(2s)}$ ($r>0$, $s>0$) или $r\le 0$.

Чтобы оценить выражение
$$
I_1=\left|f^n(h)-\exp\left\{\fr{1}{2}h^2n\right\}\right|
\exp\left\{-hx\sqrt{n}\right\}\,,
$$
воспользуемся результатом из~\cite{Tysiak}, согласно
которому
\begin{multline*}
I_1\le
\max\left\{
\vphantom{\left(1-\fr{\betr e^{hy}}{n^{\d/2}(\gamma
x)^{\dd}}\right)^{-1}}
\exp\left\{\fr{1}{2}\,h^2n-hx \sqrt{n}+
\fr{\betr e^{hy}}{n^{\d/2}(\gamma x)^{\dd}}\right\}\times\right.{}\\
\times \fr{\betr
e^{hy}}{n^{\d/2}(\gamma x)^{\dd}}\,, n\exp\left\{\fr{1}{2}h^2n-hx\sqrt{n}\right\}\times{}\\
\left.{}\times
\left(\fr{h^4}{4}+\fr{\betr
e^{hy}}{y^{\dd}}\right)\left(1-\fr{\betr e^{hy}}{n^{\d/2}(\gamma
x)^{\dd}}\right)^{-1}
\vphantom{}
\right\}.
\end{multline*}
Из~(\ref{e9gv}) вытекает, что
\begin{multline*}
1-\fr{\betr e^{hy}}{n^{\d/2}(\gamma x)^{\dd}}\ge{}\\
{}\ge 1-\fr{1}{a
\gamma^{\dd}}\exp\left\{\left(\gamma(1-\gamma)-\fr{1}{2b}\right)x^2\right\}\equiv
A_1(x)\,.
\end{multline*}
Здесь и далее символами $A(x)$, $A_1(x)$, $A_2(x)$,\ldots будут
обозначаться положительные функции аргумента $x$, а также
зависящие от параметров $a$, $b$, $\gamma$. Принимая во внимание
оценку
$$
n^{-1/2}\le\left(\Lo\right)^{1/3}
$$
и неравенство~(\ref{e9gv}), получим
\begin{multline*}
n\left(\fr{h^4}{4}+\fr{\betr e^{hy}}{y^{\dd}}\right)\le{}\\
\!{}\le
\fr{\betr}{n^{\d/2}|x|^{\dd}}\!\left[\fr{(1-\gamma)^4x^8}{4a^{1/3}}\,
\exp\!\left\{-\fr{x^2}{6b}\right\}+\fr{e^{\gamma(1-\gamma)x^2}}{\gamma^{\dd}}\right].
\end{multline*}
С учетом неравенства
$$
\exp\{1-A_1(x)\}\le\fr{1}{A_1(x)}\,,\enskip A_1(x)>0\,,
$$
также получаем
\begin{multline*}
\exp\left\{\fr{1}{2}\,h^2n-hx\sqrt{n}+ \fr{\betr
e^{hy}}{n^{1/2}(\gamma x)^{\dd}}\right\}\le{}\\
{}\le
\exp\left\{\fr{1}{2}\,h^2n-hx\sqrt{n}+1-A_1(x)\right\}\le{}\\
{}\le
\fr{1}{A_1(x)}\,\exp\left\{\fr{1}{2}h^2n-hx\sqrt{n}\right\}\,.
\end{multline*}
Наконец, если $\gamma(1-\gamma)-1/(2b)<0$, т.\,е.\
$b<$\linebreak $<[2\gamma(1-\gamma)]^{-1}$, то
\begin{equation}
I_1\le\fr{A_2(x)\betr}{n^{\d/2}A_1(K)|x|^{\dd}}\,,\label{e11gv}
\end{equation}
где $A_1(K)>0$ и
\begin{multline*}
A_2(x)=\fr{(1-\gamma)^4x^8}{4a^{1/3}}
\exp\left\{-\fr{x^2}{2}\left[\fr{1}{3b}+1-\gamma^2\right]\right\}+{}\\
{}+
\fr{1}{\gamma^{\dd}}\,\exp\left\{-\fr{x^2}{2}(1-\gamma)^2\right\}\,.
\end{multline*}
Для удобства дальнейших ссылок заметим, что с учетом~(\ref{e9gv})
$$
f(h)\ge 1-\fr{h\betr}{\gamma^{2}}\ge
1-\fr{x^2(1-\gamma)}{a\gamma^{2}}\,\exp\left\{-\fr{x^2}{2b}\right\}\equiv
A_3(x)
$$
(см.\ соотношение~(4.18) в \cite{Tysiak}).

Теперь оценим ${\sf E}\big(X_1^*\big)^2$. В~соответствии с~\cite{Tysiak} получаем, что если $K\ge\sqrt{2b}$ 
(см.~(\ref{e10gv}), то
\begin{equation}
{\sf E}\left(X_1^*\right)^2\le \fr{1}{A_3^{2}(K)}\left(h+\fr{\betr
e^{hy}}{y^{3}}\right)^2\!\le A_4(x)\Lo,\!\label{e12gv}
\end{equation}
где
\begin{multline*}
A_4(x)=\fr{1}{A_3^2(x)x^{\d}a^{1/3}}\,\exp\left\{-\fr{x^2}{6b}\right\}
\left[
\vphantom{\fr{1}{6}}
x^2(1-\gamma)+{}\right.\\
\left.{}+\fr{1}{\gamma^{2}a^{\d/3}}\exp\left\{\left(\gamma(1-\gamma)-
\fr{\d}{6b}\right)x^2\right\}
\right]^2\,.
\end{multline*}
Далее, согласно~\cite{Tysiak}

\noindent
\begin{multline*}
f^{-1}(h)\ge 2-f(h)\ge 1-\fr{h^2}{2}-\fr{\betr
e^{hy}}{y^{3}}\ge{}\\
{}\ge 1-A_5(x)\Lo\,,
\end{multline*}
где

\noindent
\begin{multline*}
A_5(x)=\fr{(1-\gamma)^2x^{3}}{2a^{1/3}}\,\exp\left\{-\fr{x^2}{6b}\right\}+{}\\
{}+
\fr{1}{\gamma^{3}a^{2/3}x}\,\exp\left\{\left(\gamma(1-\gamma)-
\fr{1}{3b}\right)x^2\right\}\,,
\end{multline*}
а также
$$
f(h){\sf E}\left(X_1^*\right)^2\ge 1-\betr\max\{\gamma^{-\d},\,h\}\,.
$$
Таким образом,
\begin{multline*}
B_n^2(h)\equiv {\sf D}S_n^*= n\left[{\sf
E}\left(X_1^*\right)^2-\left({\sf E}X_1^*\right)^2\right]\ge{}\\
{}\ge
n\left[f^{-1}(h)\left(1-\betr\max\{\gamma^{-\d},\,h\}\right)\right]-{}\\
{}- n\left({\sf
E}X_1^*\right)^2 \ge
n\left(1-\fr{A_5(x)\betr}{n^{\d/2}}\right)-{}\\
{}-\fr{n^{\d/2}\betr}{A_3(K)\gamma
x}\,\max\left\{1,\,\gamma(1-\gamma)x^2\right\}-{}\\
{}-A_4(x)n^{\d/2}\betr=n\left[1-A_6(x)\Lo\right]\,,
\end{multline*}
где
$$
A_6(x)=A_4(x)+A_5(x)+\fr{\max\{1,\,\gamma(1-\gamma)x^2\}}{A_3(K)\gamma
x}\,.
$$
С учетом~(\ref{e9gv}) имеем
\begin{multline}
B_n^2(h)\ge
n\left(1-\fr{x^{3}A_6(x)}{a}\exp\left\{-\fr{x^2}{2b}\right\}\right)\equiv{}\\
{}\equiv 
nA_7(x)\,;
\label{e13gv}
\end{multline}

\vspace*{-6pt}

\noindent
\begin{equation}
\fr{n-B_n^2(h)}{B_n^2(h)}\le\fr{A_6(x)}{A_7(x)}\Lo\,.\label{e14gv}
\end{equation}
Справедливы неравенства
\begin{multline*}
{\sf E}|X_1^*-{\sf E}X_1^*|^3\le{}\\
{}\le {\sf E}|X_1^*|^3+3{\sf
E}\left(X_1^*\right)^2|{\sf E}X_1^*|
+{\sf E}|X_1^*|\left({\sf E}X_1^*\right)^2\,;
\end{multline*}

\noindent
$$
{\sf
E}|X_1^*|^3\le\fr{e^{hy}\betr}{A_3(K)}\le\fr{1}{A_3(K)}\exp\{\gamma(1-\gamma)x^2\}\betr\,;
$$


\noindent
\begin{multline*}
3{\sf E}\left(X_1^*\right)^2\left|{\sf
E}X_1^*\right|\le{}\\
{}\le\fr{3\betr}{A_3^2(K)x}\left[\fr{(1-\gamma)x^3}{a^{1/3}}\,
\exp\left\{-\fr{x^2}{6b}\right\}+{}\right.\\
{}+
\fr{x+x^3\gamma(1-\gamma){\gamma^{2}a^{2/3}}}{\exp}\left\{\left(\gamma(1-\gamma)-
\fr{1}{3b}\right)x^2\right\}+{}\\
\left.{}+
\fr{x}{\gamma^{3}a}\exp\left\{\left(2\gamma(1-\gamma)-\fr{1}{2b}\right)x^2\right\}\right]
\equiv A_8(x)\betr\,;
\end{multline*}

\noindent
\begin{multline*}
{\sf E}|X_1^*|\left({\sf E}X_1^*\right)^2\le\sqrt{{\sf
E}\left(X_1^*\right)^2}\left({\sf E}X_1^*\right)^2\le{}\\
{}\le
\fr{\betr }{A_3^{5/2}(K)a^{2/3}}\,
\exp\left\{\left(\fr{5}{2}\gamma(1-\gamma)-\fr{3}{4b}\right)x^2\right\}\times{}\\
{}\times
\left[\fr{1}{\gamma
a^{1/3}}+\exp\left\{\left(\fr{1}{6b}-\gamma(1-\gamma)\right)x^2\right\}\right]^{1/2}\times{}\\
{}
\times\left[\fr{1}{\gamma^{2}a^{1/3}}+\right.\\
\left.{}+x^2(1-\gamma)\,\exp\left\{\left(\fr{1}{6b}-
\gamma(1-\gamma)\right)x^2\right\}\right]^2\equiv{}\\
{}\equiv A_9(x)\betr\,.
\end{multline*}
Таким образом,

\noindent
\begin{equation}
{\sf E}|X_1^*-{\sf E}X_1^*|^3\le A_{10}(x)\betr\,,\label{e15gv}
\end{equation}
где

\noindent
$$
A_{10}(x)=\fr{1}{A_3(K)}\,\exp\{\gamma(1-\gamma)x^2\}+A_8(x)+A_9(x)\,.
$$
Следовательно, в соответствии с неравенством~(\ref{e2gv}), учитывая~(\ref{e15gv}) и~(\ref{e13gv}), имеем

\noindent
\begin{multline}
\left|{\sf P}\left(\fr{S_n^*-{\sf E}S_n^*}{\sqrt{{\sf
D}S_n^*}}<u\right)-\Phi(u)\right|\le{}\\
{}\le 0{,}3041\fr{{\sf E}|X_1^*-{\sf
E}X_1^*|^3+1}{\sqrt{n}\big({\sf D}X_1^*\big)^{3/2}}\le{}\\
{}
\le 0{,}3041
\fr{A_{10}(x)\betr}{A_7^{3/2}(x)n^{\d/2}}+\fr{0{,}3041}{A_7^{3/2}(x)\sqrt{n}}\,.\label{e16gv}
\end{multline}
Далее в предположении, что

\noindent
\begin{multline*}
A(x)\equiv\fr{1}{2\gamma
a^{1/3}}\,\exp\left\{-\fr{x^2}{6b}\right\}+{}\\
{}+
\fr{1}{\gamma^4(1-\gamma)^2x^4a^{2/3}}\,\exp\left\{\!\left(\gamma(1-\gamma)-
\fr{1}{3b}\right)x^2\!\right\}\le\fr{1}{6}\hspace*{-2.1413pt}
\end{multline*}
(см.\ выражение для $g_{19}(x)$ в~\cite{Tysiak}), имеем
\begin{multline}
\left|{\sf
E}S_n^*-hB_n^*\right|\le\fr{\betr}{A_3(K)\gamma^{2}x^{2}}\left[
\vphantom{\fr{x^2}{3}}\exp\left\{\gamma(1-\gamma)x^2\right\}+{}\right.\\
\left.{}+
\fr{(1-\gamma)^2x^4}{a^{2/3}}\,\exp\left\{-\fr{x^2}{3b}\right\}\right]\equiv
A_{11}(x)\betr\,.\label{e17gv}
\end{multline}
Наряду с~(\ref{e14gv}) имеет место оценка
\begin{equation*}
B_n^2(h)-n\le
n\left[\fr{e^{hy}\betr}{f(h)y}+\fr{1}{f(h)}-1\right]\le{}\hspace*{8mm}
\end{equation*}
\pagebreak

\noindent
\begin{multline*}
{}\le 
\fr{n^{\d/2}\betr}{A_3(K)\gamma
x}\left[\exp\left\{\gamma(1-\gamma)x^2\right\}+{}\right.\\
\left.{}+\fr{(1-\gamma)x^2}{\gamma
a^{2/3}}\exp\left\{-\fr{x^2}{3b}\right\}\right]\equiv
A_{12}(x)n^{\d/2}\betr\,.
\end{multline*}
Таким образом,
\begin{multline}
\left|B_n^2(h)-n\right|\max\left\{\fr{1}{n},\,\fr{1}{B_n^2(h)}\right\}\le{}\\
{}\le \max\left\{A_{12}(x),\,\fr{A_6(x)}{A_7(x)}\right\}\Lo\,.\label{e18gv}
\end{multline}
Теперь можно приступить к оцениванию величин
(см.~(\ref{e8gv})):
\begin{align*}
I_2&\equiv {\sf P}\left(S_n^*>x\sqrt{n}\right) \  \mbox{и}\ \\
I_3&\equiv \sup_{u\ge x}\left|{\sf
P}\left(S_n^*<u\sqrt{n}\right)-\Phi\left(u-h\sqrt{n}\right)\right|
\end{align*}


Наряду с~(\ref{e13gv}) понадобится верхняя оценка для $B_n^2(h)$, которую
получим с учетом тождества $(1-z)^{-1}=z(1-z)^{-1}+1$:
\begin{multline*}
B_n^2(h)\le n{\sf E}\left(X_1^*\right)^2\le{}\\
{}\le
 n\left(1+\fr{e^{hy}\betr}{y}\right)\left(1-\fr{h\betr}{y^{2}}\right)^{-1}={}\\
{} =
n\left(1+\fr{e^{hy}\betr}{y}\right)\left[\fr{h\betr}{y^{2}}\left(1-\fr{h\betr}{y^{2}}\right)^{-1}+1\right]\le{}\\
{}\le
\fr{n}{A_3(K)}\left[\fr{h\betr}{y^{2}}+
\fr{he^{hy}\beta^2_{\dd}}{y^{3}}+A_3(K)\left(\!1+\fr{e^{hy}\betr}{y}\!\right)\right]
\equiv{}\hspace*{-0.7442pt}\\
{}\equiv n A_{13}(x)\,,
\end{multline*}
где
\begin{multline*}
A_{13}(x)=1+(\gamma x)^2\left(1-A_1(K)\right)+{}\\
{}
+\fr{\left(1-A_3(K)\right)x^2}{aA_3(K)}\left[a^{\d/3}\exp\left\{-\fr{x^2}{3b}\right\}+{}\right.\\
\left.{}+\fr{1}{\gamma}\,\exp\left\{\left(\gamma(1-\gamma)-\fr{1}{2b}\right)x^2\right\}\right]\,.
\end{multline*}
Найдем верхнюю оценку для ${\sf E}S_n^*-x\sqrt{n}$ (см.~(\ref{e17gv}):
\begin{multline*}
{\sf E}S_n^*-x\sqrt{n}={\sf
E}S_n^*-nh+nh-x\sqrt{n}\le{}\\
{}\le
\sqrt{n}\left(\fr{A_{11}(x)x^{\dd}}{a}\exp\left\{-\fr{x^2}{2b}\right\}-\gamma
x\right)\,,
\end{multline*}
т.\,е.\
\begin{multline}
\fr{x\sqrt{n}-{\sf E} S_n^*}{B_n(h)}\ge
\fr{1}{\sqrt{A_{13}(x)}}\left(\vphantom{\fr{x^3}{a}}\gamma x-{}\right.\\
\left.{}-\fr{x^{\dd}A_{11}(x)}{a}\exp\left\{-\fr{x^2}{2b}\right\}\right)\equiv
A_{14}(x)\,.\label{e19gv}
\end{multline}
Отсюда с учетом~(\ref{e1gv}), (\ref{e2gv}), (\ref{e9gv}) и (\ref{e16gv}) при
$A_{14}(x)\ge1$ получаем
\begin{multline}
I_2\le\left|{\sf
P}\left(S_n^*<x\sqrt{n}\right)-\Phi\left(\fr{x\sqrt{n}-{\sf
E}S_n^*}{\sqrt{{\sf
D}S_n^*}}\right)\right|+{}\\
{}+\Phi\left(-\fr{x\sqrt{n}-{\sf
E}S_n^*}{\sqrt{{\sf D}S_n^*}}\right)\le{}\\
{}\le 
0{,}3041\fr{A_{10}(x)x^{\dd}}{aA_7^{3/2}(x)}\exp\left\{-\fr{x^2}{2b}\right\}+{}\\
{}+
\fr{0{,}3041}{A_7^{3/2}(x)\sqrt{n}}+
\fr{1}{\sqrt{2\pi}A_{14}(x)}\exp\left\{-\fr{A_{14}^2(x)}{2}\right\}\equiv{}\\
{}
\equiv A_{15}(x)+\fr{0{,}3041}{A_7^{3/2}(x)\sqrt{n}}\,.\label{e20gv}
\end{multline}

Чтобы оценить $I_3$, воспользуемся соотношениями~(\ref{e13gv}),
(\ref{e16gv})--(\ref{e19gv}) и~(\ref{e2gv}) и получим
\begin{multline}
I_3=\sup_{u\ge x}\bigg|{\sf P}\left(\fr{S_n^*-{\sf
E}S_n^*}{B_n(h)}<\fr{u\sqrt{n}-{\sf
E}S_n^*}{B_n(h)}\right)-{}\\
{}-\Phi\left(\fr{u\sqrt{n}-{\sf
E}S_n^*}{B_n(h)}\right)+\Phi\left(\fr{u\sqrt{n}-{\sf
E}S_n^*}{B_n(h)}\right)-{}\\
{}-
\Phi\left(\fr{u\sqrt{n}-{\sf
E}S_n^*}{\sqrt{n}}\right)+\Phi\left(\fr{u\sqrt{n}-{\sf
E}S_n^*}{\sqrt{n}}\right)-{}\\
{}-\Phi\left(u-h\sqrt{n}\right)\bigg|\le{}\\
{}\le
\sup_{v\ge(x\sqrt{n}-{\sf E}S_n^*)/B_n(h)}\bigg[\bigg|{\sf
P}\bigg(\fr{S_n^*-{\sf
E}S_n^*}{B_n(h)}<v\bigg)-{}\\
{}-
\Phi(v)\bigg|+\bigg|\Phi(v)-\Phi\bigg(v\fr{B_n(h)}{\sqrt{n}}\bigg)
\bigg|\bigg]+{}\\
{}
+\sup_{u\ge x}\left|\Phi\left(u-{\sf
E}S_n^*/\sqrt{n}\right)-\Phi\left(u-h\sqrt{n}\right)\right|\le{}\\
{}\le
0{,}3041\fr{A_{10}(x)\betr}{A_7^{3/2}(x)n^{\d/2}}+\fr{0{,}3041}{A_7^{3/2}(x)\sqrt{n}}+{}\\
{}
+\fr{1}{\sqrt{8\pi}}\left|\fr{B_n^2(h)}{n}-1\right|\max\left\{1,\,\fr{n}{B_n^2(h)}\right\}\times{}\\
{}\times
\sup\left\{\vphantom{\fr{B_h(h)}{B_h(h)}}
|s|e^{-s^2/2}:\right.\\
\left.s\ge\fr{x\sqrt{n}-{\sf
E}S_n^*}{B_n(h)}\min\left\{1,\,\fr{B_n(h)}{\sqrt{n}}\right\}\right\}+{}\\
{}+
\fr{1}{\sqrt{2\pi}}\bigg|\fr{{\sf
E}S_n^*-nh}{\sqrt{n}}\bigg|
\sup\bigg\{e^{-s^2/2}:\\
s\ge\min\Big\{x-\fr{{\sf
E}S_n^*}{\sqrt{n}},\,x-h\sqrt{n}\Big\}\bigg\}\le{}\\
{}\le
A_{16}(x)\Lo+\fr{0{,}3041}{A_7^{3/2}(x)\sqrt{n}}\,,\label{e21gv}
\end{multline}
где
\begin{multline*}
A_{16}(x)=0{,}3041\fr{A_{10}(x)}{A_7^{3/2}(x)}+{}\\
{}+\fr{A_{14}(x)}{\sqrt{8\pi}}\max\left\{A_{12}(x),\,\fr{A_6(x)}{A_7(x)}\right\}
\exp\left\{-\fr{A_{14}^2(x)}{2}\right\}+{}\\
{}+
\fr{A_{11}(x)}{\sqrt{2\pi}}\exp\!\bigg\{\!-\fr{1}{2}\Big[\gamma
x-\fr{A_{11}(x)|x|^{\dd}}{a}\exp\left\{-\fr{x^2}{2b}\right\}\Big]^2\!\bigg\}.\hspace*{-1.467pt}
\end{multline*}

Соотношение~(\ref{e21gv}) справедливо в предположении
$A_{14}(x)\ge 1$, так что
$$
\gamma
x-\fr{A_{11}(x)|x|^{\dd}}{a}\,\exp\left\{-\fr{x^2}{2b}\right\}\ge 0\,.
$$
Первое слагаемое в правой части~(\ref{e8gv}) оценим с помощью неравенства
Маркова:
\begin{equation}
n{\sf P}(|X_1|>y)\le\fr{\betr}{\gamma^{\dd}
|x|^{\dd}n^{\d/2}}\,.\label{e22gv}
\end{equation}
В итоге из~(\ref{e8gv}) с учетом~(\ref{e11gv}), (\ref{e20gv})--(\ref{e22gv}) мы
получаем: для каждого~$x$ из рассматриваемого диапазона~ii
справедливо неравенство
\begin{multline}
\left|F_n(x)-\Phi(x)\right|\le\fr{\betr}{\gamma^{\dd}|x|^{\dd}n^{\d/2}}+I_1
I_2+{}\\
{}+2I_3\exp\left\{-(1-\gamma^2)\fr{x^2}{2}\right\}\le{}\\
{}\le 
Q_n(x;\,a,b,\gamma,K)\fr{\betr}{|x|^{\dd}n^{\d/2}}+{}\\
{}+
0{,}6082\fr{\exp\left\{-(1-\gamma^2)x^2/2\right\}}{A_7^{3/2}(x)\sqrt{n}}\,,\label{e23gv}
\end{multline}
где
\begin{multline*}
Q_n(x;\,a,b,\gamma,K)=\fr{1}{\gamma^{\dd}}+{}\\
{}+
\fr{A_2(x)}{A_1(K)}\left[1+A_{15}(x)+
\fr{0{,}3041}{A_7^{3/2}(x)\sqrt{n}}\right]+{}\\
{}+2A_{16}(x)|x|^{\dd}\exp\left\{-(1-\gamma^2)\fr{x^2}{2}\right\}
\end{multline*}
%$\big 
(отметим, что
$ %\begin{multline*}
\lim_{n\to\infty}Q_n(x;\,a,b,\gamma,K)={1}/{\gamma^{\dd}}+$\linebreak $+
{A_2(x)\left(1+A_{15}(x)\right)}/{A_1(K)}+2A_{16}(x)|x|^{\dd}\exp\{-(1-$\linebreak 
$-\gamma^2){x^2}/{2}\}$).
%\end{multline*}

Обозначим
$$
R(x;\,a,b,\gamma,K)=0{,}6082\fr{|x|^{\dd}\exp\left\{-(1-\gamma^2){x^2}/{2}\right\}}{A_7^{3/2}(x)}\,.
$$
Тогда справедливо следующее утверждение.

\columnbreak

\noindent
\textbf{Лемма 2.2} \textit{Предположим, что $K^2\le x^2\le
c_n(x;\,a,b)$,}

\noindent
\begin{multline*}
K^2\ge\fr{1}{2\pi}\,,\enskip  0<\gamma<\fr{1}{2}\,,\enskip  a>0\,,\\
1<b<\min\left\{\fr{1}{2\gamma(1-\gamma)},\,\fr{K^2}{2}\right\}
\end{multline*}
\textit{и}

\noindent
\begin{multline*}
A(x)\le\fr{1}{6}\,, \enskip A_1(K)>0\,,\enskip  A_3(K)>0\,,\\
 A_7(x)>0\,,\enskip 
A_{14}(x)\ge1\,.
\end{multline*}
\textit{Тогда для всех $n\ge1$}

\noindent
\begin{multline*}
|x|^{\dd}\left|F_n(x)-\Phi(x)\right|
\le
Q_n(x;\,a,b,\gamma,K)\Lo+{}\\
{}+R(x;\,a,b,\gamma,K)\frac{1}{\sqrt{n}}\,.
\end{multline*}

\smallskip

Приведем две мажоранты функции $Q_n(x;\,a,b,\gamma,K)$,
не зависящие от~$n$.

Во-первых, очевидно, что
\begin{multline}
Q_n(x;\,a,b,\gamma,K)\le Q_1(x;\,a,b,\gamma,K)={}\\
{}
=\fr{1}{\gamma^{\dd}}+\fr{A_2(x)}{A_1(K)}\left[1+A_{15}(x)+
\fr{0{,}3041}{A_7^{3/2}(x)}\right]+{}\\
{}+2A_{16}(x)|x|^{\dd}\exp\left\{-(1-\gamma^2)\fr{x^2}{2}\right\}\,.\label{e24gv}
\end{multline}


Во-вторых, с учетом того, что в сделанных предположениях о
моментах случайной величины~$X_1$ всегда $\betr\ge1$, из
соотношения~(\ref{e9gv}) вытекает неравенство
$$
\fr{1}{\sqrt{n}}\le\fr{|x|^{\dd}}{a}\exp\left\{-\fr{x^2}{2b}\right\}\,.
$$
Поэтому
\begin{multline}
Q_n(x;\,a,b,\gamma,K)\le Q'(x;\,a,b,\gamma,K)\equiv{}\\
{}\equiv
\fr{1}{\gamma^{\dd}}+\fr{A_2(x)}{A_1(K)}\bigg[1+{}\\
{}+A_{15}(x)+
\fr{0{,}3041}{A_7^{3/2}(x)}\bigg(\fr{|x|^{\dd}}{a}\exp\left\{-\fr{x^2}{2b}\right\}\bigg)\bigg]
+{}\\
{}+
2A_{16}(x)|x|^{\dd}\exp\left\{-(1-\gamma^2)\fr{x^2}{2}\right\}\,.\label{e25gv}
\end{multline}

Наконец, пытаясь ограничить $Q_n(x;\,a,b,\gamma,K)$, вместо
неравенства~(\ref{e2gv}) при оценивании величины~$I_2$ можно
воспользоваться неравенством~(\ref{e1gv}) с наилучшей известной на
сегодняшний день оценкой константы $C_0$: $C_0 \le 0{,}4784$~\cite{KorolevBEs}. 
Тогда из~(\ref{e1gv}), (\ref{e9gv}) и~(\ref{e16gv}) при $A_{14}(x)\ge1$
будет следовать оценка

\noindent
\begin{multline*}
I_2\le\bigg|{\sf
P}\big(S_n^*<x\sqrt{n}\big)-\Phi\bigg(\fr{x\sqrt{n}-{\sf
E}S_n^*}{\sqrt{{\sf
D}S_n^*}}\bigg)\bigg|+{}\\
{}+\Phi\bigg(\!-\fr{x\sqrt{n}-{\sf
E}S_n^*}{\sqrt{{\sf D}S_n^*}}\!\bigg)\le
0{,}4784\fr{A_{10}(x)x^{\dd}}{aA_7^{3/2}(x)}\exp\Big\{-\frac{x^2}{2b}\Big\}+{}\\
{}+
\fr{1}{\sqrt{2\pi}A_{14}(x)}\exp\left\{-\fr{A_{14}^2(x)}{2}\right\}\equiv
A'_{15}(x)\,.
\end{multline*}
При этом в~(\ref{e23gv}) и последующих выкладках и вычислениях вместо
$Q_n(x;\,a,b,\gamma,K)$ cоответственно следует использовать
величину

\noindent
\begin{multline}
Q''(x;\,a,b,\gamma,K)=\fr{1}{\gamma^{\dd}}+\fr{A_2(x)}{A_1(K)}\left[1+A'_{15}(x)\right]
+{}\\
{}+2A_{16}(x)|x|^{\dd}\exp\left\{-(1-\gamma^2)\fr{x^2}{2}\right\}\,.\label{e26gv}
\end{multline}

Положим

\noindent
\begin{multline*}
Q(x;\,a,b,\gamma,K)=\min\left\{Q_1(x;\,a,b,\gamma,K),\right.\\
\left.Q'(x;\,a,b,\gamma,K),\,Q''(x;\,a,b,\gamma,K)\right\}\,.
\end{multline*}

%\medskip

\noindent
\textbf{Следствие 2.1}. \textit{В условиях леммы~$2.2$ для всех $n\ge1$
справедливо неравенство}

\noindent
\begin{multline*}
|x|^{\dd}\left|F_n(x)-\Phi(x)\right|\le
Q(x;\,a,b,\gamma,K)\Lo+{}\\
{}+R(x;\,\d,a,b,\gamma,K)\fr{1}{\sqrt{n}}\,.
\end{multline*}

\subsection{Неравномерные оценки скорости сходимости в центральной предельной теореме}

Положим

\noindent
\begin{align*}
U_n&=\min_{a,b,\gamma,K}\max\Big\{C_1
K^{\dd}\,,\\
&\hspace*{-8mm}\max_{K\le|x|\le\sqrt{c_n(x;\,a,b)}}Q_n(x;\,a,b,\gamma,K),\,
P(a,b,K)\Big\}\,;\\
U&=\min_{a,b,\gamma,K}\max\Big\{C_1
K^{\dd}\,,\\
&\hspace*{-8mm}\max_{K\le|x|\le\sqrt{c_n(x;\,a,b)}}Q(x;\,a,b,\gamma,K),\,
P(a,b,K)\Big\}\,,
\end{align*}
где минимум берется по множеству значений вспомогательных
параметров, описанному в формулировке леммы~2. Значения параметров
$a$, $b$, $\gamma$, $K$, доставляющие вышеуказанные минимумы,
обозначим соответственно $a^{(n)}_0$, $b^{(n)}_0$,
$\gamma^{(n)}_0$, $K^{(n)}_0$ и $a_0$, $b_0$, $\gamma_0$, $K_0$.
Положим

\noindent
\begin{multline*}
R^{(n)}_0={}\hspace*{60mm}\\
=\!\!\!\max_{K^{(n)}_0\le|x|\le\sqrt{c_n(x;\,a^{(n)}_0,b^{(n)}_0)}}R(x;a^{(n)}_0,b^{(n)}_0,\gamma^{(n)}_0,K^{(n)}_0);\hspace*{-2.76pt}
\end{multline*}
\begin{gather*}
R_0=\max_{K_0\le|x|\le\sqrt{c_n(x;\,a_0,b_0)}}R(x;a_0,b_0,\gamma_0,K_0)\,;
\\
D_n=U_n+C_1,\enskip D=U+C_1\,;
\\
V_n=\max\left\{R^{(n)}_0,\,C_1(K^{(n)}_0)^{\dd}\right\}+C_1\,;
\\
V(n)=\max\left\{R_0,\,C_1 K_0^{\dd}\right\}+C_1.
\end{gather*}
Очевидно, что $V_n\le V_1$ и $V(n)\le V(1)\equiv V$.

Из соотношения~(\ref{e6gv}), лемм~2.1 и~2.2 вытекает следующее утверждение.

\medskip

\noindent
\textbf{Теорема 2.1.} \textit{Для всех $x\in\R$ и всех $n\ge1$
справедливы неравенства:}
\begin{align*}
\left(1+|x|^{\dd}\right)\left|F_n(x)-\Phi(x)\right|&\le
D_n\Lo+\fr{V_n}{n^{\d/2}}\,;
\\
\left(1+|x|^{\dd}\right)\left|F_n(x)-\Phi(x)\right|&\le
D\Lo+\fr{V(n)}{n^{\d/2}}\,.
\end{align*}

\medskip

\noindent
\textbf{Следствие 2.2} \textit{Для всех $x\in\R$ и всех $n\ge1$
справедливо неравенство}
$$
\left(1+|x|^{\dd}\right)\left|F_n(x)-\Phi(x)\right|\le
D\Lo+\fr{V}{n^{\d/2}}\,.
$$

\medskip

\noindent
\textbf{Замечание 2.1} На практике процедуру поиска оптимальных
значений $a$, $b$, $\gamma$, $K$ можно организовать следующим
образом. Поскольку выражение $C_1(1)K^{\dd}$ не зависит от $a$,
$b$, $\gamma$, сначала можно найти значение $K^*=K^*(a,b)$ из
условия
$$
C_1(K^*)^{\dd}=P(a,b,K^*)\,,
$$
затем найти пару $(a^*,b^*)$ из условия
$$
(a^*,b^*)=\mathrm{arg}\,\min_{a,b}P\left(a,b,K^*(a,b)\right)\,,
$$
а затем провести оптимизацию
\begin{align*}
&\hspace*{-5mm}\max_{
K^*(a^*,b^*)\le|x|\le\sqrt{c_n(x;\,a^*,b^*)}}Q_n(x;\,a^*,b^*,\gamma,\\
&\hspace*{35mm}K^*(a^*,b^*)) \longrightarrow\min_{\gamma}\,,
\\
&\hspace*{-5mm}\max_{
K^*(a^*,b^*)\le|x|\le\sqrt{c_n(x;\,a^*,b^*)}}Q(x;\,a^*,b^*,\gamma, \\
&\hspace*{35mm}K^*(a^*,b^*)) \longrightarrow\min_{\gamma}\,,
\\
&\hspace*{-5mm}\max_{
K^*(a^*,b^*)\le|x|\le\sqrt{c_n(x;\,a^*,b^*)}}\bar{Q}\left(x;\,a^*,b^*,\gamma,\right.\\
&\hspace*{35mm}\left.K^*(a^*,b^*)\right) \longrightarrow\min_{\gamma}\,.
\end{align*}
При этом вычисления показывают, что все три вышеперечисленных
максимума достигаются в точке $x = K^*(a^*,b^*).$

\smallskip

\noindent
\textbf{Замечание 2.2} Конкретные вычисления показали, что
значения функций $Q(x,a,b,\gamma,K)$ и $R(x,a,b,\gamma,K)$
оказываются строго меньшими, чем значения
$P(a,b,\gamma,K)$ и $C_1K^3,$ поэтому максимум в выражениях для
$U$ и $V(n)$ определяется лишь значениями функций
$P(a,b,\gamma,K)$ и $C_1K^3,$ а следовательно, $D$ и $V$
совпадают, причем оптимальное значение $D$ не превосходит~22,7707.
Таким образом, приведенная в следствии~2.2 оценка имеет структуру,
аналогичную оценке~(\ref{e2gv}), и справедливо следующее
утверждение.

\smallskip

\noindent
\textbf{Следствие 2.2\boldmath{$'$}} \textit{Для всех $x\in\R$ и всех $n\ge1$
справедливо неравенство}

\noindent
$$
(1+|x|^3)\left|F_n(x)-\Phi(x)\right|\leqslant
22{,}7707\fr{\beta_3+1}{n^{1/2}}\,.
$$

\smallskip

Следствие 2.2$'$ позволяет уточнить константу в неравномерной
оценке скорости сходимости в ЦПТ для пуассоновских случайных сумм,
чему посвящен следующий раздел.

\vspace*{-3pt}

\section{Неравномерная оценка скорости сходимости в~центральной предельной
теореме для~пуассоновских случайных сумм}

Пусть теперь $X_1, X_2, \ldots$~--- последовательность независимых
одинаково распределенных случайных величин таких, что
\begin{equation}
{\sf E}X_1\equiv\mu\,, \ \ {\sf D}X_1 \equiv \sigma^2>0\,,\ \ {\sf
E}|X_1|^{\dd} \equiv \betr<\infty\,. \label{e27gv}
\end{equation}
Пусть $N_\lambda$~--- случайная величина, имеющая распределение
Пуассона с параметром $\lambda>0.$ Предположим, что при каждом~$\lambda>0$ 
случайные величины $N_\lambda, X_1, X_2,\ldots$
независимы.

Рассмотрим пуассоновскую случайную сумму
$$
S_{\lambda} = X_1+\cdots+ X_{N_\lambda}\,.
$$
Для определенности полагаем, что  $S_\lambda = 0$ при $N_\lambda =
0.$ Несложно видеть, что в рас\-смат\-ри\-ва\-емых условиях на моменты
случайной величины~$X_1$ справедливы соотношения
$$
{\sf E}S_\lambda=\lambda\mu\,, \enskip {\sf
D}S_\lambda=\lambda(\mu^2+\sigma^2)\,.
$$
Функцию распределения стандартизованной пуассоновской случайной
суммы
$$
\widetilde{S}_\lambda
\equiv\fr{S_\lambda-\lambda\mu}{\sqrt{\lambda(\mu^2+\sigma^2)}}
$$
обозначим~$F_\lambda(x)$.

\columnbreak

Пуассоновские случайные суммы~$S_{\lambda}$ являются очень
популярными математическими моделями многих объектов и процессов в
самых разных областях, в том числе в страховании, где они
используются для описания суммы страховых требований, поступивших
в течение определенного периода времени, в теории управления
запасами, где они описывают суммарные заявки на продукт. При
анализе информационных систем также традиционным предположением
является пуассоновский характер потока заявок (клиентов,
требований, задач, сообщений), так что суммарные характеристики
заявок в информационных системах имеют вид пуассоновских случайных
сумм. Задаче изуче\-ния точ\-ности нормальной аппроксимации для
распределений пуассоновских случайных сумм~--- так называемых
обобщенных пуассоновских распределений~--- посвящена обширная
литература (см., например, библиографию в книгах~\cite{KorBenShorg} и~\cite{BenKor2002}).

В работе~\cite{Mich93} показано, что для любых $x\in\R$
и любых $n\ge1$ справедливо неравенство
\begin{multline}
\left(1+|x|^{3}\right)\left|F_{\lambda}(x)-\Phi(x)\right|\le{}\\
{}\le
C\cdot\fr{\betr}{\lambda^{\delta/2}(\mu^2+\sigma^2)^{1+\delta/2}}\,,\label{e28gv}
\end{multline}
где $C$~--- та же константа, что и в <<классической>> оценке~(\ref{e5gv}). 
В~данном разделе будет показано, что на самом деле неравенство~(\ref{e28gv})
справедливо с заменой~$C$ на~$D$ (см.\ следствие~2.2$'$). Для этого
понадобятся некоторые вспомогательные утверждения.

Обозначим
$$
\nu = \fr{\lambda}{n}\,.
$$

\medskip

\noindent
\textbf{Лемма 3.1.} \textit{Распределение пуассоновской случайной суммы
$S_\lambda$ совпадает с распределением суммы $n$ независимых
одинаково распределенных случайных величин, каким бы ни было
натуральное число $n\geq1:$}
$$
X_1+\cdots+X_{N_\lambda} \stackrel{d}{=} Y_{\nu,1}+\cdots+Y_{\nu,n}\,,
$$
\textit{где при каждом $n$ случайные величины $Y_{\nu,1},\ldots,
Y_{\nu,n}$ независимы и одинаково распределены. При этом если
случайная величина $X_1$ удовлетворяет условиям}~(\ref{e27gv}), \textit{то для
моментов случайной величины $Y_{\nu,1}$ имеют мес\-то соотношения:}
\begin{gather*}
{\sf E}Y_{\nu,1} = \mu\nu\,; \quad {\sf D}Y_{\nu,1} =
(\mu^2+\sigma^2)\nu\,;\\
{\sf E}|Y_{\nu,1} - \mu\nu|^{3}\leq \nu\beta_{3}(1+40\nu) \,, \quad
n\geq\lambda\,.
\end{gather*}

\medskip

Д\,о\,к\,а\,з\,а\,т\,е\,л\,ь\,с\,т\,в\,о\,\ см.\ в работе~\cite{Shev2007}.

\medskip

\noindent
\textbf{Следствие 3.1} \textit{Если выполнены условия $(27),$ то для
любого $n=1,2, \ldots$}
$$
\widetilde{S}_\lambda \stackrel{d}{=}
\fr{1}{\sqrt{n}}\sum_{k=1}^n Z_{\nu,k}\,,
$$
\textit{где при каждом $n$ случайные величины $Z_{\nu,1},\ldots,
Z_{\nu,n}$ независимы и одинаково распределены. Более того, эти
случайные величины имеют нулевое среднее и единичную дисперсию и
при всех $n\geq\lambda$}
\begin{equation}
{\sf E}|Z_{\nu,1}|^{3}\le
\fr{\beta_{3}(1+40\nu)}{(\mu^2+\sigma^2)^{3/2}}\left(\fr{n}{\lambda}\right)^{1/2}\,.
\label{e29gv}
\end{equation}

\medskip

Следующее утверждение представляет собой основной результат данной
статьи.

\medskip

\noindent
\textbf{Теорема 3.1} \textit{При условиях}~(\ref{e27gv}) \textit{для любого $\lambda>0$
справедливо неравенство}
$$
\sup_{x\in\R}(1+|x|^{3})\left|F_\lambda(x) - \Phi(x)\right|\leq
\fr{D\betr}{\lambda^{1/2}(\mu^2+\sigma^2)^{3/2}}\,,
$$
\textit{где константа~$D$ та же, что и в следствии~$2.2'$,
т.\,е.\ $D\le22.7707$.}

\medskip

\noindent
Д\,о\,к\,а\,з\,а\,т\,е\,л\,ь\,с\,т\,в\,о\,.\ Из леммы~3.1 и следствия~3.1
вытекает, что для любого целого $n\geq1$
$$
|F_\lambda(x) - \Phi(x)| = \left|{\sf
P}\left(\fr{1}{\sqrt{n}}\sum_{k=1}^n
Z_{\nu,k}<x\right)-\Phi(x)\right|\,.
$$
Следовательно, в силу следствия~2.2 для произвольного целого
$n\geq1$ при каждом фиксированном $x\in\R$ имеем
\begin{equation}
\left(1+|x|^{3}\right)\left|F_\lambda(x) - \Phi(x)\right| \leq D\fr{{\sf
E}|Z_{\nu,1}|^{3}}{n^{1/2}}+\fr{V}{n^{\d/2}}\,. \label{e30gv}
\end{equation}
Поскольку в~(\ref{e30gv}) $n$ произвольно, можно считать, что
$n\geq\lambda.$ Тогда, используя оценку~(\ref{e29gv}), в продолжение~(\ref{e30gv})
получаем неравенство
\begin{multline*}
\left(1+|x|^{3}\right)\left|F_\lambda(x) - \Phi(x)\right|\leq{}\\
{}\leq
\fr{D\betr}{\lambda^{1/2}(\mu^2+\sigma^2)^{3/2}}
\left(1+40\fr{\lambda}{n}\right)+\fr{V}{n^{\d/2}}\,.
\end{multline*}
Так как здесь $n\geq\lambda$ произвольно, устремляя
$n\rightarrow\infty,$ окончательно получаем
\begin{multline*}
\left(1+|x|^{3}\right)\left|F_\lambda(x) - \Phi(x)\right|\leq{}\\
{}\leq
\lim_{n\rightarrow\infty}\left[\fr{D\betr}{\lambda^{1/2}(\mu^2+\sigma^2)^{3/2}}
\left(1+40\fr{\lambda}{n}\right)+\fr{V}{n^{\d/2}}\right]={}\\
{}=
\fr{D\betr}{\lambda^{1/2}(\mu^2+\sigma^2)^{3/2}}\,,
\end{multline*}
что и требовалось доказать.

\section{Неравномерные оценки скорости сходимости в~предельных
теоремах для~смешанных пуассоновских случайных сумм}

Пусть $\Lambda_t$~--- положительная случайная величина, функция
распределения $G_t(x)={\sf P}(\Lambda_t<x)$ которой зависит от
некоторого параметра $t>0$. Под смешанным пуассоновским
распределением со структурным распределением~$G_t$ будем
подразумевать распределение случайной величины~$N(t)$, принимающей
целые неотрицательные значения с вероятностями

\noindent
$$
{\sf P}\left(N(t)=k\right)=\fr{1}{k!}\int\limits_{0}^{\infty}
e^{-\lambda}\lambda^kdG_t(\lambda),\enskip k=0,1,2,\ldots
$$
Известно несколько конкретных примеров смешанных пуассоновских
распределений, наиболее широко используемым среди которых, пожалуй,
является отрицательное биномиальное распределение (это распределение
было использовано в виде смешанного пуассоновского еще в работе~\cite{Greenwood1920} 
для\linebreak моделирования частоты несчастных
случаев на\linebreak производстве). Отрицательное биномиальное распределение
порождается структурным гам\-ма-рас\-пре\-де\-ле\-ни\-ем. Другими примерами
смешанных пуассоновских распределений являются распределение
Делапорте, порождаемое сдвинутым гам\-ма-струк\-тур\-ным распределением~\cite{Delaporte}, 
распределение Зихеля, порожденное
обратным нормальным структурным распределением~\cite{Holla}--\cite{Willmot}, обобщенное
распределение Варинга~\cite{Irwin, Seal}. Свойства смешанных пуассоновских
распределений подробно описаны в книгах~\cite{Grandell} и~\cite{BenKor2002}.

Пусть, как и ранее, $X_1,X_2,\ldots$~--- независимые одинаково
распределенные случайные величины. Предположим, что случайные
величины $N(t),X_1,X_2,\ldots$ независимы при каждом $t>0$. Положим

\noindent
$$
S(t)=X_1+ \ldots +X_{N(t)}
$$
(как и ранее, для определенности будем считать, что $S(t)=0$, если
$N(t)=0$). Случайную величину~$S(t)$ будем называть смешанной
пуассоновской случайной суммой, а ее распределение~--- обобщенным
смешанным пуассоновским.

Асимптотическое поведение смешанных пуассоновских случайных сумм~$S(t)$, когда $N(t)$ 
в определенном смысле неограниченно
возрастает, принципиально различно в зависимости от того, равно
нулю математическое ожидание~$\mu$ слагаемых или нет.

Сходимость по распределению и по вероятности будет обозначаться
символами $\Longrightarrow$ и~$\pto$ соответственно.

В этом разделе сосредоточимся на ситуации, когда ${\sf E}X_1=0$. 
В~таком случае предельными распределениями для стандартизованных
смешанных пуассоновских случайных сумм являются масштабные смеси
нормальных законов. Не ограничивая общности, будем считать, что
${\sf D}X_1=1$.

\smallskip

\noindent
\textbf{Теорема 4.1} \cite{Korolev96, BenKor2002}. \textit{Предположим, что
$\Lambda_t\pto\infty$ при $t\to\infty$. Тогда для положительной
неограниченно возрастающей функции~$d(t)$ существует функция
распределения $H(x)$ такая, что}
$$
{\sf P}\left(\fr{S(t)}{\sqrt{d(t)}}<x\right)\Longrightarrow H(x)\
\enskip (t\to\infty)\,,
$$
\textit{в том и только в том случае, когда существует функция
распределения~$G(x)$ такая, что при той же функции~$d(t)$}
\begin{equation}
G_t\left(xd(t)\right)\Longrightarrow G(x) \enskip (t\to\infty)\label{e31gv}
\end{equation}
\textit{и}
$$
H(x)=\int\limits_{0}^{\infty}\Phi\left(\frac{x}{\sqrt{y}}\right)dG(y)\,,\enskip
x\in\R\,.
$$

\smallskip

Легко видеть, что распределение смешанной пуассоновской случайной
суммы~$S(t)$ можно записать в виде
\begin{multline}
{\sf P}(S(t)<x)=\int_{0}^{\infty}{\sf
P}\left(\sum_{j=1}^{N_{\lambda}}X_j<x\right)dG_t(\lambda)\,,\\
x\in\R\,,
\label{e32gv}
\end{multline}
где $N_\lambda$~--- пуассоновская случайная величина с параметром
$\lambda>0$, такая что при каждом $\lambda>0$ случайные величины
$N_\lambda,X_1,X_2,\ldots$ независимы. (Следует заметить, что запись
приводимых соотношений в терминах случайных величин использована
лишь для удобства и наглядности. На самом деле речь идет о
соответствующих соотношениях для распределений, но такая форма
записи оказывается более громоздкой. Поэтому предположение о
существовании вероятностного пространства, на котором определены
упомянутые выше случайные величины с указанными свойствами, ни в
коей мере не ограничивает общности.) Пусть
\begin{equation}
{\sf E}X_1=0\,,\ \ {\sf E}X_1^2=1\,,\ \ \beta^3={\sf
E}|X_1|^3<\infty\label{e33gv}
\end{equation}
и $d(t)$, $t>0$,~--- некоторая положительная неограниченно
возрастающая функция. Равномерные оценки скорости сходимости в
теореме~4.1 приведены в работах~\cite{KorolevBEs, KorSchev, Gavrilenko} (см.\ также~\cite{KorBenShorg}). 
В~данном разделе будут приведены неравномерные
оценки скорости сходимости в теореме~4.1 и ее частных случаях.

Пусть $G(x)$~--- функция распределения такая, что $G(0)=0$. Если
выполнено условие~(\ref{e31gv}), то в соответствии с теоремой~4.1
обобщенные смешанные пуассоновские распределения случайной
величины~$S(t)$, нормированной квадратным корнем из функции~$d(t)$, 
сходятся к масштабным смесям нормальных законов, в которых
смешивающим распределением является~$G(x)$. Обозначим
\begin{gather*}
\Delta_t(x)=\left|{\sf P}\left(\fr{S(t)}{\sqrt{d(t)}}<x\right)-
\int\limits_{0}^{\infty}
\Phi\left(\fr{x}{\sqrt{\lambda}}\right)dG(\lambda)\right|\,,\\
\delta_t(x)=G_t\left(d(t)x\right)-G(x)\,.
\end{gather*}

\smallskip

\noindent
\textbf{Теорема 4.2.} \textit{Предположим, что выполнены условия}~(\ref{e28gv}).
\textit{Тогда при каждом $t>0$ при любом $x\in\R$ имеет место оценка}
\begin{multline}
\Delta_t(x)\le 22{,}7707\fr{\beta_3}{\sqrt{d(t)}}\,{\sf
E}\left\{\fr{\Lambda_t}{d(t)}\left[\left(\fr{\Lambda_t}{d(t)}\right)^{3/2}+{}\right.\right.\\
\left.\left.{}+|x|^3
\vphantom{\fr{\Lambda_t}{d(t)}}\right]^{-1}\right\}+
\int\limits_0^{\infty}|\delta_t(\lambda)|\,d_{\lambda}\Phi\left(\fr{x}{\sqrt{\lambda}}\right)\,.\label{e34gv}
\end{multline}

\smallskip

\noindent
Д\,о\,к\,а\,з\,а\,т\,е\,л\,ь\,с\,т\,в\,о\,.\ Идея доказательства аналогична идее
доказательства равномерной оценки в работе~\cite{Gavrilenko} и основана на использовании
представления~(\ref{e32gv}). Имеем
\begin{multline*}
\Delta_t(x)\le\int\limits_0^{\infty}\left|{\sf
P}\left(\fr{1}{\sqrt{\lambda d(t)}}\sum_{j=1}^{N_{\lambda
d(t)}}X_j<\fr{x}{\sqrt{\lambda}}\right)-{}\right.\\
\left.{}-\Phi\left(\fr{x}{\sqrt{\lambda}}\right)\right|\,dG_t\left(\lambda
d(t)\right)+
\left|\int\limits_0^{\infty}\Phi\left(\fr{x}{\sqrt{\lambda}}\right)\,d\delta_t(\lambda)\right|\equiv{}\\
{}\equiv
I_1(t;x)+I_2(t;x)\,.
\end{multline*}
Подынтегральное выражение в $I_1(t;x)$ оценим с помощью теоремы~3.1 и получим
\begin{multline}
I_1(t;x)\le{}\\
{}\le
D\beta_3\int\limits_0^{\infty}\fr{\lambda^{3/2}}{\sqrt{\lambda
d(t)}\left(\lambda^{3/2}+|x|^3\right)}\,dG_t\left(\lambda d(t)\right)={}\\
{}=
\fr{D\beta_3}{\sqrt{d(t)}}{\sf E}\left\{
\fr{\Lambda_t}{d(t)}\left[\left(\fr{\Lambda_t}{d(t)}\right)^{3/2}+|x|^3\right]^{-1}\right\}\,.\label{e35gv}
\end{multline}
Интегрируя по частям, получаем
\begin{equation}
I_2(t;x)\le\int\limits_0^{\infty}|\delta_t(\lambda)|\,d_{\lambda}\Phi\left(\fr{x}{\sqrt{\lambda}}\right)\,.\label{e36gv}
\end{equation}
Требуемое утверждение вытекает из~(\ref{e35gv}) и~(\ref{e36gv}). Тео\-ре\-ма доказана.

\smallskip

В качестве примера применения теоремы~4.2 рассмотрим случай, когда
при каждом $t>0$ случайная величина $\Lambda_t$ имеет
гам\-ма-рас\-пре\-де\-ле\-ние. Этот случай представляет особый интерес с
точки зрения его применения в финансовой мате-\linebreak матике для
асимптотического обоснования адекватности таких популярных моделей
эволюции\linebreak финансовых индексов, как дисперсионные гам\-ма-про\-цес\-сы Леви
(variance-gamma L$\acute{\mbox{e}}$vy processes, VG-processes)~\cite{Madan} или
двусторонние гам\-ма-про\-цес\-сы~\cite{Carr} (также см.~\cite{Korolev2010}).

Смешанное пуассоновское распределение со структурным
гам\-ма-рас\-пре\-де\-ле\-ни\-ем является не чем иным, как отрицательным
биномиальным распределением. Убедимся в этом. Плотность
гам\-ма-рас\-пре\-де\-ле\-ния с параметром формы $r>0$ и параметром масштаба
$\sigma>0$, как известно, имеет вид
$$
g_{r,\si}(x)=\fr{\si^r}{\Gamma(r)}e^{-\si x}x^{r-1}\,,\enskip x>0\,.
$$
Таким образом, смешанное пуассоновское распределение со
структурным гам\-ма-рас\-пре\-де\-ле\-ни\-ем имеет характеристическую функцию
\begin{multline*}
\psi(z)=\int\limits_{0}^{\infty}\exp\left\{y(e^{iz}-1)\right\}\fr{\si^r}{\Gamma(r)}
e^{-\si y}y^{r-1}\,dy={}\\
{}=
\fr{\si^r}{\Gamma(r)}\int\limits_{0}^{\infty}\exp\left\{-\si
y\left(1+ \fr{1-e^{iz}}{\si}\right)\right\}y^{r-1}\,dy={}\\
{}=
\left(1+\fr{1-e^{iz}}{\si}\right)^{-r}\,.
\end{multline*}
Вводя новую параметризацию
$$
\si=\fr{p}{1-p}\  \ \left(p=\fr{\si}{1+\si}\right)\,,\ \ 
p\in(0,1)\,,
$$
окончательно получаем
$$
\psi(z)=\left(\fr{p}{1-(1-p)e^{iz}}\right)^r\,,\enskip z\in\R\,,
$$
что совпадает с характеристической функцией отрицательного
биномиального распределения с параметрами $r>0$ и $p\in(0,1)$.

Функцию гамма-распределения с параметром масштаба $\si$ и
параметром формы $r$ обозначим $G_{r,\si}(x)$. Несложно убедиться,
что
\begin{equation}
G_{r,\si}(x)\equiv G_{r,1}(\si x)\,.\label{e37gv}
\end{equation}
Случайную величину с функцией распределения $G_{r,\si}(x)$
обозначим $U(r,\si)$. Хорошо известно, что
$$
{\sf E}U(r,\si)=\fr{r}{\si}\,.
$$
Зафиксируем параметр $r$ и в качестве структурной случайной
величины~$\Lambda_t$ возьмем $U(r,\si)$, предполагая, что
$t=\si^{-1}$:
$$
\Lambda_t=U(r,t^{-1})\,.
$$
В качестве функции~$d(t)$ возьмем $d(t)\equiv{\sf E}\Lambda_t=$\linebreak $={\sf
E}U(r,t^{-1})$. Очевидно, что ${\sf E}U(r,t^{-1})=rt$. Тогда с
учетом~(\ref{e37gv}) будем иметь
\begin{multline*}
G_t\left(xd(t)\right)={\sf P}\left(U(r,t^{-1})<xrt\right)={}\\
{}={\sf P}\left(U(r,1)<xr\right)={\sf P}\left(U(r,r)<x\right)=G_{r,r}(x)\,.
\end{multline*}
Заметим, что функция распределения в правой час\-ти последнего
соотношения не зависит от~$t$. Поэтому указанный выбор функции
$d(t)$ тривиальным образом гарантирует выполнение условия~(\ref{e31gv})
теоремы~4.1. Более того, в таком случае $\delta_t(x)=0$ для всех
$t>0$ и $x\in\R$.

При этом случайная величина~$N(t)$ имеет отрицательное
биномиальное распределение с па\-ра\-мет\-ра\-ми~$r$ и $p=(t+1)^{-1}$:
\begin{multline}
{\sf P}\left(N(t)=k\right)=C_{r+k-1}^{k}p^r(1-p)^k
={}\\
{}=C_{r+k-1}^{k}\fr{t^k}{(1+t)^{r+k}}\,,\enskip k=0,1,2,\ldots
\label{e38gv}
\end{multline}
Здесь для нецелых $r$ величина $C_{r+k-1}^{k}$ определяется как
$$
C_{r+k-1}^{k} = \fr{\Gamma(r+k)}{k!\Gamma(r)}\,.
$$

Итак, в рассматриваемом случае $G_t\big(xd(t)\big)\equiv$\linebreak $\equiv
G_{r,r}(x)$, $d(t)\equiv rt$ и второе слагаемое в правой час\-ти~(\ref{e34gv}) 
равно нулю. Вычислим первое слагаемое в правой части~(\ref{e34gv}) для
рассматриваемой ситуации в предположении, что
\begin{equation}
r>\fr{1}{2}\,.\label{e39gv}
\end{equation}
Имеем
\begin{multline}
\fr{1}{\sqrt{d(t)}}\,{\sf
E}\left\{\fr{\Lambda_t}{d(t)}\left[\left(\fr{\Lambda_t}{d(t)}\right)^{3/2}+|x|^3\right]^{-1}\right\}={}\\
{}=
\fr{1}{\sqrt{rt}}\int\limits_0^{\infty}\fr{y}{y^{3/2}+|x|^3}\,dG_{r,r}(x)={}\\
{}=
\fr{r^{r-1/2}}{\sqrt{t}\Gamma(r)}\int\limits_0^{\infty}\fr{y^re^{-ry}\,dy}{y^{3/2}+|x|^3}\,.
\label{e40gv}
\end{multline}

Приведем несколько оценок для правой час\-ти~(\ref{e40gv}). Во-пер\-вых,
несложно убедиться, что при $y>0$
$$
\sup_x\fr{1+|x|^3}{y^{3/2}+|x|^3}=\max\{1,\,y^{-3/2}\}\,.
$$
Поэтому
\begin{multline*}
J_r(x)\equiv\int\limits_0^{\infty}\fr{y^re^{-ry}dy}{y^{3/2}+|x|^3}\le{}\\
{}\le
\fr{1}{1+|x|^3}
\int\limits_0^{\infty}y^re^{-ry}\sup_x\left\{\fr{1+|x|^3}{y^{3/2}+|x|^3}\right\}\,dy\le{}\\
{}\le
\fr{1}{1+|x|^3}\left[\int\limits_0^1y^{r-3/2}e^{-ry}\,dy+\int\limits_1^{\infty}y^re^{-ry}\,dy\right]
={}\\
{}=
\fr{1}{1+|x|^3}\left[\fr{\gamma_r(r-{1}/{2})}{r^{r-1/2}}+\fr{\Gamma(r+1)-\gamma_r(r+1)}{r^{r+1}}\right]\,,
\end{multline*}
где $\gamma_x(\alpha)$~--- неполная гам\-ма-функ\-ция,
$$
\gamma_x(\alpha)=\int\limits_0^x e^{-z}z^{\alpha-1}\,dz\,,\ \ \
\alpha>0\,,\ x\ge0\,.
$$
Следовательно,
\begin{multline*}
\fr{r^{r-1/2}}{\sqrt{t}\Gamma(r)}J_r(x)\le\fr{1}{\sqrt{t}(1+|x|^3)}\left[
\fr{\gamma_r(r-{1}/{2})}{\Gamma(r)}+{}\right.\\
\left.{}+
\fr{1}{\sqrt{r}}-\fr{\gamma_r(r+1)}{\Gamma(r+1)\sqrt{r}}\right]\,.
\end{multline*}
Таким образом, справедливо

\smallskip

\noindent
\textbf{Следствие 4.1.} \textit{Пусть случайная величина $N(t)$ имеет
отрицательное биномиальное распределение}~(\ref{e38gv}), $t>0$.
\textit{Предположим, что выполнены условия}~(\ref{e33gv}) \textit{и}~(\ref{e39gv}). \textit{Тогда для
любых $t>0$ и $x\in\R$ справедлива оценка}

\noindent
\begin{multline*}
\left|{\sf P}(S(t)<x\sqrt{rt})-\int\limits_{0}^{\infty}
\Phi\left(\fr{x}{\sqrt{y}}\right)dG_{r,r}(y)\right|\le{}\\
{}\le \fr{22{,}7707\beta_3}{\sqrt{t}(1+|x|^3)}
K_1(r)\,,
\end{multline*}
\textit{где}
$$
K_1(r)=\fr{\gamma_r(r-{1}/{2})}{\Gamma(r)}+
\fr{1}{\sqrt{r}}-\fr{\gamma_r(r+1)}{\Gamma(r+1)\sqrt{r}}\,.
$$

\smallskip

С другой стороны, очевидно, что
$\gamma_x(\alpha)\le\Gamma(\alpha)$ при любых $x\ge0$ и
$\alpha>0$. Поэтому величину $K_1(r)$ можно оценить выражением,
содержащим только <<полные>> гам\-ма-функ\-ции:
\begin{equation}
K_1(r)\le \fr{\Gamma(r-{1}/{2})}{\Gamma(r)}+
\fr{1}{\sqrt{r}}\,.\label{e41gv}
\end{equation}

\smallskip

При $r=1$ случайная величина $\Lambda_t=U(1,t^{-1})$ имеет
показательное распределение с параметром $\sigma=1/t$.
Следовательно, как легко убедиться, смешанная пуассоновская
случайная величина~$N(t)$ с таким структурным распределением имеет
геометрическое распределение с параметром
$p=(t+1)^{-1}$. При этом предельное (при 
$t\rightarrow \infty$) распределение для стандартизованной
геометрической суммы~$S(t)$ является распределение Лапласа с
плотностью
$$
l(x)=\fr{1}{\sqrt{2}}\,e^{-\sqrt{2}\,|x|},\enskip x\in\R\,.
$$
Функцию распределения, соответствующую плотности $l(x)$, обозначим~$L(x)$:
$$
L(x)=\begin{cases}
\fr{1}{2}\,e^{\sqrt{2}x}\,, & \mbox{если}\ \ x<0\,,\\
1-\fr{1}{2}\,e^{-\sqrt{2}x}\,, & \mbox{если}\ \ x\ge0\,.
\end{cases}
$$
Из следствия~4.1 при этом получаем

\smallskip

\noindent
\textbf{Следствие 4.2.} \textit{Пусть случайная величина $N(t)$ имеет
геометрическое распределение с параметром $p=(1+t)^{-1}$, $t>0$.
Предположим, что выполнены условия}~(\ref{e33gv}). \textit{Тогда для любых $t>0$ и
$x\in\R$ справедлива оценка}
$$
\left|{\sf P}(S(t)<x\sqrt{t})-L(x)\right|\le
\fr{50{,}7652\beta_3}{\sqrt{t}\left(1+|x|^3\right)}\,.
$$

\smallskip

Заметим, что оценка, приведенная в следствии~4.2, точнее
равномерной оценки, приведенной в работах~\cite{KorolevBEs} и~\cite{KorSchev}, для
$|x|>4{,}5334$.

Заметим также, что использование оценки~(\ref{e41gv}) для $K_1(r)$
увеличивает абсолютную константу в следствии~4.2 до 63,1308~---
такова цена упрощения.

{\small\frenchspacing
{%\baselineskip=10.8pt
%\addcontentsline{toc}{section}{Литература}
\begin{thebibliography}{99}


\bibitem{Esseen42} 
\Au{Esseen C.\,G.} On the Liapunoff limit of error in
the theory of probability~// Ark. Mat. Astron. Fys., 1942. Vol.~A28.
No.\,9. P.~1--19.

\bibitem{Berry41} 
\Au{Berry A.\,C. } The accuracy of the Gaussian
approximation to the sum of the distributed random variables~// J.~Theor. Probab., 1994. Vol.~2. No.\,2. P.~211--224.

\bibitem{KorolevVOz} 
\Au{Королев В.\,Ю., Шевцова И.\,Г.} Уточнение верхней оценки
абсолютной постоянной в неравенстве Берри--Эссеена для смешанных
пуассоновских случайных сумм~// Докл. РАН,
2010. Т.~431. Вып.~1. С.~16--19.

\bibitem{KorolevBEs} 
\Au{Королев В.\,Ю., Шевцова И.\,Г.} Уточнение неравенства
Берри--Эссеена с приложениями к пуассоновским и смешанным
пуассоновским случайным суммам~// Обозрение прикладной и
промышленной математики, 2010. Т.~17. Вып.~1. С.~25--56.

\bibitem{Ess} \Au{Esseen C.\,G.} 
A moment inequality with an application to
the central limit theorem~// Skand. Aktuarrietidskr, 1956. Vol.~39.
P.~160--170.

\bibitem{Meshalkin} \Au{Мешалкин Л.\,Д., Рогозин Б.\,А.} Оценка расстояния между
функциями распределения по близости их характеристических функций
и ее применение к центральной предельной теореме~// Предельные
теоремы теории вероятностей.~--- Ташкент: АН УзССР, 1963. С.~40--55.

\bibitem{Nagaev} \Au{Нагаев С.\,В.} Некоторые предельные теоремы для больших
уклонений~// Теория вероятностей и ее применения, 1965. Т.~10.
Вып.~2. С.~231--254.

\bibitem{Mich81} \Au{Michel R.} On the constant in the nonuniform version of the
Berry--Esseen theorem~// Z.~Wahrsch. verw. Geb., 1981. Bd.~55. P.~109--117.

\bibitem{KorSchev} \Au{Korolev  V., Shevtsova~I.} An improvement of the
Berry--Esseen inequality with applications to Poisson and mixed
Poisson random sums~// Scandinavian Actuarial J., 2010.
{\sf http://www.informaworld.com/10.1080/03461238.\newline 2010.485370}.

\bibitem{Nefedova} \Au{Нефедова Ю.\,С., Шевцова И.\,Г.} О~точности нормальной
аппроксимации для распределений пуассоновских случайных сумм~//
Информатика и её применения, 2011. Т.~5. Вып.~1. С.~\pageref{nefedova1}--\pageref{end-nefedova}.

\bibitem{Paditz89} \Au{Paditz L.} On the analytical structure of the constant in
the nonuniform version of the Esseen inequality~// Statistics, 1989. Vol.~20. No.~3. P.~453--464.

\bibitem{Mich93} \Au{Michel R.} On Berry--Esseen results for the compound
Poisson distribution~// Insurance: Mathematics and Economics,
1993. Vol.~13. No.\,1. P.~35--37.

\bibitem{Rychlik} \Au{Rychlik Z.} Nonuniform central limit bounds and their
applications~// Теория вероятностей и ее применения, 1983. T.~28.
Вып.~3. С.~646--652.

\bibitem{Paditz81} \Au{Paditz L.} Einseitige Fehlerabsch$\ddot{\mbox{a}}$tzungen im zentralen
Grenzwertsatz~// Math. Operationsforsch. und Statist., ser.
Statist., 1981. Bd.~12. P.~587--604.

\bibitem{Tysiak} \Au{Tysiak W.} Gleichm$\ddot{\mbox{a}}${\ss}ige und
nicht-gleichm$\ddot{\mbox{a}}${\ss}ige Berry--Esseen-Absch$\ddot{\mbox{a}}$tzungen.
Dissertation.~--- Wuppertal, 1983.

\bibitem{KorBenShorg} \Au{Королев В.\,Ю., Бенинг В.\,Е., Шоргин С.\,Я.} 
Математические основы теории риска.~--- М.: Физматлит, 2007.

\bibitem{BenKor2002} \Au{Bening V., Korolev V.}  Generalized Poisson models
and their applications in insurance and finance.~--- Utrecht: VSP, 2002.

\bibitem{Shev2007} \Au{Шевцова И.\,Г.} О~точности нормальной аппроксимации для
распределений пуассоновских случайных сумм~// Обозрение
промышленной и прикладной математики, 2007. Т.~14. Вып.~1. С.~3--28.

\bibitem{Greenwood1920} \Au{Greenwood M., Yule G.\,U.} An inquiry into the nature of
frequency-distributions of multiple happenings, etc.~// J.~Roy.
Statist. Soc., 1920. Vol.~83. P.~255--279.

\bibitem{Delaporte} \Au{Delaporte P.} Un probl$\acute{\mbox{e}}$me de tarification de l'assurance
accidents d'automobile examin$\acute{\mbox{e}}$ par la statistique math$\acute{\mbox{e}}$matique~// 
Trans. 16th  Congress (International) of Actuaries.~--- Brussels,
1960. Vol.~2. P.~121--135.

\bibitem{Holla} \Au{Holla M. S.} On a Poisson-inverse Gaussian distribution~//
Metrika, 1967. Vol.~11. P.~115--121.

\bibitem{Sichel} \Au{Sichel H. S.} On a family of discrete distributions
particular suited to represent long tailed frequency data~// 
3rd Symposium on Mathematical Statistics Proceedings~/ Ed.\ N.\,F.~Laubscher.~---
Pretoria: CSIR, 1971. P.~51--97.

\bibitem{Willmot} \Au{Willmot G.\,E.} The Poisson-inverse Gaussian distribution
as an alternative to the negative binomial~// Scandinavian Actuar.~J., 1987. P.~113--127.

\bibitem{Irwin} \Au{Irwin J.\,O.} The generalized Waring distribution applied to
accident theory~// J.~Royal Statist. Soc., Ser. A, 1968. Vol.~130. P.~205--225.

\bibitem{Seal} \Au{Seal H.} Survival probabilities. The goal of risk
theory.~--- Chichester\,--\,New York\,--\,Brisbane\,--\,Toronto: Wiley, 1978.

\bibitem{Grandell} \Au{Grandell J.} Mixed Poisson processes.~--- London:
Chapman and Hall, 1997.

\bibitem{Korolev96} \Au{Korolev V.\,Yu.} A general theorem on the limit behavior of
superpositions of independent random processes with applications
to Cox processes~// J. Math. Sci., 1996. Vol.~81. No.\,5. P.~2951--2956.

\bibitem{Gavrilenko} \Au{Гавриленко С.\,В., Королев В.\,Ю.} Оценки скорости
сходимости смешанных пуассоновских случайных сумм~// 
Системы и средства информатики. Спец. вып. Математические модели в информационных технологиях.~--- М.: ИПИ РАН,
2006. С.~248--257.

\bibitem{Madan} \Au{Madan D.\,B.,  Seneta E.} The variance gamma ($V.G.$)
model for share market return~// J. Business, 1990. Vol.~63. P.~511--524.

\bibitem{Carr} \Au{Carr P.\,P., Madan D.\,B., Chang~E.\,C.} The variance gamma
process and option pricing~// Eur. Finance Rev., 1998. Vol.~2. P.~79--105.

 \label{end\stat}

\bibitem{Korolev2010} \Au{Королев В.\,Ю.} Ве\-ро\-ят\-ност\-но-ста\-ти\-сти\-че\-ские методы
декомпозиции волатильности хаотических процессов.~--- М.: МГУ, 2010.
 \end{thebibliography}
}
}


\end{multicols}  