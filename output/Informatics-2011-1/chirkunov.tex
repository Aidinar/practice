\def\stat{chirkunov}

\def\tit{АГЕНТНОЕ МОДЕЛИРОВАНИЕ РАЗВИТИЯ ТЕРРИТОРИАЛЬНОЙ 
СИСТЕМЫ}

\def\titkol{Агентное моделирование развития территориальной 
системы}

\def\autkol{К.\,С.~Чиркунов}
\def\aut{К.\,С.~Чиркунов$^1$}

\titel{\tit}{\aut}{\autkol}{\titkol}

%{\renewcommand{\thefootnote}{\fnsymbol{footnote}}\footnotetext[1]
%{Работа выполнена при поддержке гранта РФФИ 10-07-00433 и ФЦП <<Научные и на\-уч\-но-пе\-да\-го\-ги\-че\-ские 
%кадры инновационной России>> на 2009--2013~годы. Статья подготовлена по результатам работы секции 
%<<Биометрия>> 20-й Международной конференции по компьютерной графике и зрению Графикон-2010, 
%С.-Петербург, 20--24~сентября 2010~г.}}

\renewcommand{\thefootnote}{\arabic{footnote}}
\footnotetext[1]{Институт систем информатики им.\ А.\,П.~Ершова СО РАН, cyril.chirkunov@computer.org}


\Abst{Рассмотрена агентная система, имитирующая развитие экономики страны 
(строительство новых производств, повышение общего уровня доходов) на базе модели 
территориальной системы. Элементы модели представлены в виде автономных единиц, 
способных к взаимодействию друг с другом,~--- агентов.}

\KW{агентные алгоритмы переговоров; территориальная система; моделирование; 
экономическое районирование}

      \vskip 14pt plus 9pt minus 6pt

      \thispagestyle{headings}

      \begin{multicols}{2}
      
            \label{st\stat}

\section{Введение}

  Для моделирования процессов формирования и развития территориальных 
образований изначально применялись методики <<сверху вниз>> с 
последующей корректировкой решений <<снизу вверх>>, которые 
предполагали организацию вычислений с выделенным центром. Такой подход 
использовали А.\,Г.~Аганбегян, К.\,А.~Багриновский, А.\,Г.~Гранберг (об этом 
упоминается в работе~[1]). Новые подходы с использованием агентов 
(М.~Вулдридж, Й.~Шоэм, К.~Ли\-тон-Бра\-ун и~др.) позволяют описать 
территориальную систему  как совокупность взаимодействующих между собой 
агентов~[2, 3]. Решения на глобальном уровне (на уровне всей системы) 
возникают как результат локальных коопераций ее элементов. 
  
  Для чего нужно развивать новые подходы при уже неоднократно успешно 
зарекомендовавшем себя на практике существующем аппарате моделирования 
развития территориальных систем?
  
  Ответ на этот вопрос можно найти, например, в работе известного 
американского философа науки Пола Фейерабенда~[4]. По его мнению, ряд 
наиболее важных формальных свойств теории обнаруживается благодаря 
контрасту, а не анализу. Ученый, желающий более глубоко уяснить свои 
концепции, должен вводить новые.
  
  Не вдаваясь, подобно Полу Фейерабенду, в крайний релятивизм, примем его 
утверждение о том, что всякая устоявшаяся методология~--- даже наиболее 
очевидная~--- имеет свои пределы. Отступим от нескольких догматов, которые 
были <<провозглашены>> в рамках традиционного моделирования 
территориальной системы нашей страны, и предпримем попытку смоделировать 
систему как совокупность взаимодействующих агентов при отсутствии 
вертикальных иерархических связей на верхнем уровне. При этом сохраним ряд 
изначально существовавших в этой области понятий. 
  
  В процессе построения агентной модели необходимо решить целый ряд 
задач: определить структуру агентов и их коммуникации, описать сре\-ду
системы, составить алгоритмы коопера-\linebreak тивных взаимодействий (в том чис\-ле 
алгоритм\linebreak определения набора новых производств, алгоритм установления 
межрайонных, внешних то\-вар\-но-ма\-те\-ри\-а\-льных связей и~т.\,д.), 
привести обоснования\linebreak использованных решений. Все эти вопросы так или 
иначе должны быть затронуты в работе.

\section{Территориальная система как~совокупность агентов 
экономических районов}
  
  В данном разделе опишем общие базовые понятия и простую схему агентной 
модели территориальной системы.

\subsection{ Что такое агент?}

  В этом подразделе следовало бы дать четкую и ясную формулировку понятия 
<<агент>> и перейти к следующему. Однако не все так просто. Известно по 
меньшей мере шесть различных определений этого понятия в работах, 
посвященных агентному моделированию, из которых ни одно \mbox{нельзя} назвать 
общим, полным и ясным. Наиболее важным свойством агента можно считать 
автономность, иногда также отмечают обучаемость (способность к адаптации),  
но зачастую это свойство опускается. Агент действует в некоторой среде и 
обладает спо\-соб\-ностью получать информацию об объектах, которые в ней 
находятся. Будем считать, что среда является полностью наблюдаемой, т.\,е.\ 
агент может получить информацию обо всех объектах, которые в ней находятся 
(более формально составляющие среды будут описаны далее). Также агент 
может оказывать воздействие на среду и изменять ее состояние. Надо сказать, 
что среда в данной модели не является детерминированной для одного агента, 
т.\,е.\ изначально она не гарантирует только один исход после любого действия 
агента. Однако агенты могут договариваться друг с другом и, как следствие, 
обеспечивать детерминированность среды для определенного набора действий. 
В~данной работе будем предполагать, что агенты действуют одновременно, а 
среда является динамичной и дискретной. 
  
  Отдельно выделяют характеристики интеллектуальных агентов (в некоторых 
работах~--- интеллектуальных в слабом смысле).
  \begin{enumerate}
\item Реактивность~--- способность своевременно реагировать на воспринятые 
изменения среды.
\item Проактивность~--- способность проявлять инициативу для достижения 
своих целей.
\item Социальные навыки~--- способность к взаимодействию с другими 
агентами <<ради дела>>. 
\end{enumerate}

  Будем считать, что агенты предлагаемой модели обладают сле\-ду\-ющи\-ми 
характеристиками: они способны оперативно реагировать 
на изменения внешней конъюнктуры, могут выступать инициаторами 
переговоров с другими агентами и взаимодействовать с другими агентами ради 
общей и собственной выгоды. 
  
  Каждый агент имеет функцию полезности, которая в численном выражении 
показывает, насколько хорошо <<живется>> агенту в системе. Агент прилагает 
все усилия, чтобы повысить ее значение, но при этом старается увеличить и 
полезность общественную, которая задается как сумма значений функции 
полезности всех агентов. В~качестве значений функции полезности будет 
выступать доход, получаемый агентом в сис\-теме.
  
  Популяция взаимодействующих агентов совместно со средой образует 
\textit{многоагентную систему} (МАС).

\subsection{Схема модели}

  Теперь покажем, каким образом можно представить территориальную 
систему в виде МАС.
  
  На рис.~1 показана простая схема такой модели. Агенты, представляющие 
экономические районы (агенты ЭР), потребляют ресурсы, которые\linebreak\vspace*{-12pt}
\begin{center} %fig1
\vspace*{6pt}
\mbox{%
\epsfxsize=79mm
\epsfbox{chi-1.eps}
}
\end{center}
\vspace*{12pt}
%\begin{center}
{{\figurename~1}\ \ \small{Взаимодействие популяции агентов ЭР и внешней среды}}
%\end{center}
\vspace*{9pt}

\bigskip
\addtocounter{figure}{1}

\begin{center} %fig2
\vspace*{6pt}
\mbox{%
\epsfxsize=79mm
\epsfbox{chi-2.eps}
}
\end{center}
\vspace*{12pt}
%\begin{center}
{{\figurename~2}\ \ \small{Схема устройства модели территориальной сис\-те\-мы}}
%\end{center}
\vspace*{9pt}

\bigskip
\addtocounter{figure}{1}
     
     \noindent 
предо\-став\-ля\-ет им внешняя среда (т.\,е.\ территориальная система поглощает 
импортную продукцию), и, в свою очередь, генерируют товарную массу для 
внешней среды (т.\,е.\ система экспортирует на внешние рынки произведенную 
на своей территории продукцию).

 
     
  На схеме, приведенной на рис.~2, территориальная система описывается 
посредством активных сущностей (агентов) и пассивных сущностей (объектов). 
Агенты обозначаются кругами, объекты, которые они могут использовать, 
изображены в виде окружающих их квадратов. Агенты верхнего уровня 
иерархии (большие кружки) содержат в себе популяцию агентов и объекты 
нижнего уровня (маленькие кружки и квадраты). Пунктиром обозначена 
граница популяции агентов. Как и на рис.~1, стрелками показаны ма\-те\-ри\-аль\-но-то\-вар\-ные 
связи. Однако здесь уже видно, что агенты взаимодействуют и друг с 
другом: помимо внешних связей (жирные линии со стрелками), направленных 
либо наружу, либо внутрь популяции, появились и внутренние (линии со 
стрелками средней толщины и тонкие). Большими квадратами обозначены 
ареалы районов, малыми~--- площадки для размещения тер\-ри\-то\-ри\-аль\-но-про\-из\-вод\-ст\-вен\-ных 
комплексов (ТПК). Большие круги~--- агенты ЭР, 
генерирующие товарный поток в рамках своей межрайонной специализации и 
поглощающие материалы и продукцию, необходимую для производства и 
других хозяйственных нужд.
  
  
  Иными словами, если агент ЭР$_1$ нуждается в какой-либо продукции из 
межрайонных отраслевых специализаций ЭР$_2$, то он <<договаривается>> с 
агентом ЭР$_2$ и между ними устанавливается то\-вар\-но-ма\-те\-ри\-а\-ль\-ная 
связь. Маленькие кружки показывают агентов ТПК, отношения между 
которыми устанавливаются по такому же принципу, как и между агентами ЭР, 
но только внутри их родителя. Детальное рассмотрение агентов ТПК выходит 
за рамки данной работы.
  
  Можно заметить, что некоторые квадраты-пло\-щад\-ки являются вакантными. 
Это значит, что на данном месте еще не размещен ТПК. Тер\-ри\-то\-ри\-аль\-но-про\-из\-вод\-ст\-вен\-ные 
комплексы могут  динамически формироваться внутри незанятых площадок в ходе процесса 
симуляции. 
  
  При описании модели использовалось пред\-став\-ле\-ние о 
  тер\-ри\-то\-ри\-аль\-но-про\-из\-вод\-ст\-вен\-ном комплексе, которое 
содержится в работах Н.\,Н.~Колосовского. 
  
  \medskip
  
  \noindent
  \textbf{Определение~1.}
   \textit{Тер\-ри\-то\-ри\-аль\-но-про\-из\-вод\-ст\-вен\-ный комплекс~--- 
совокупность расположенных рядом друг с другом взаимосвязанных 
производств.}

  \smallskip
  
  Понятие было введено в экономическую географию в 1940-х~гг. В~исходном 
определении речь шла о взаимосвязанных и взаимообусловленных 
производствах, от размещения которых на определенной территории 
достигается дополнительный экономический эффект за счет использования 
общей инфраструктуры, кадровой базы, мощностей и~т.\,д.
  
  \smallskip
  
  \noindent
  \textbf{Определение 2.} \textit{Экономический район~--- это 
территориально связанные части единого народного хозяйства страны с 
различной специализацией, постоянным обменом производимых товаров и 
другими экономическими отношениями.}

  \smallskip
  
  Более детально о модели территориальной системы будет рассказано 
в следующем разделе.

\section{Математическое и~алгоритмическое описание задачи 
развития территориальной системы}
  
  Введем динамическое (т.\,е.\ меняющееся в ходе исполнения) множество 
товарных рынков~$M$, которое будет иметь следующую структуру: 
  \begin{equation*}
  M=M^{\mathrm{EX}}\cup M^{\mathrm{NR}}\cup M^I\,,
  \end{equation*}
  где $M^{\mathrm{EX}}$~--- множество товарных рынков за пределами территории 
страны, внешние рынки;
  $M^{\mathrm{NR}}$~--- множество межрайонных товарных рынков, описывающее 
потребности районов в тех или иных видах продукции, ресурсов и~т.\,д.;
  $M^I$~--- множество внутренних районных товарных рынков, описывающее 
потребности, которые могут быть удовлетворены только в рамках района. 
  
  Элемент $m\in M$ имеет следующую структуру: $m=\langle \mathrm{pr}, d\rangle$, где 
$\mathrm{pr}\in \mathrm{PR}$~--- это товар из множества возможных товаров, $d\in C$ ($C\in 
\mathcal{R}$)~--- денежная величина объема рынка. 
  
  Обозначим через TS множество территориальных систем: 
  $$
\mathrm{TS}=\{\mathrm{ts}\vert \mathrm{ts}~\mbox{---~территориальная~система}\}\,.
  $$
  
  Введем множество ER экономических районов, которое формально можно 
задать так: 
  $\mathrm{ER}=$\linebreak $=\{\mathrm{er} \vert \mathrm{er}~\mbox{---~экономический~район}\}$.
  
  Будем считать, что все территориально-производственные комплексы 
являются элементами одного и того же множества~TPC. 
  
  Далее опишем функцию развития территориальной системы.
  
  Функция~$U$ действует на множестве~TPC, и ее аргументы и значения~--- 
элементы этого множества: $U: \mathrm{TPC}\rightarrow \mathrm{TPC}$. 
  
  Доход, получаемый системой, будем обозначать через~$R$, где $R$~--- это 
функция, определенная на множестве территориальных систем~TS и 
отображающая его на множество действительных чисел~$\mathcal{R}$. 
В~качестве аргумента функции выступает произвольная территориальная 
система, а ее значением служит действительное число.
  
  \medskip
  
  \noindent
  \textbf{Определение 3.} $U$~--- \textit{программа развития для $\mathrm{ts}\in \mathrm{TS}$, 
если $R(U(\mathrm{ts}))>R(\mathrm{ts})$.}
  
  \smallskip
  
  \noindent
  \textbf{Определение 4.} \textit{Будем говорить, что $U^*$~--- оптимальная 
программа развития для $\mathrm{ts}\in \mathrm{TS}$ среди всех программ развития~$U$ для 
$\mathrm{ts}\in \mathrm{TS}$, если $R(U^*(\mathrm{ts}))\leq R(U(\mathrm{ts}))$ для любой такой~$U$.}
  
  \smallskip
  
  Территориальную систему $\mathrm{ts}\in \mathrm{TS}$ представим в виде структуры $\mathrm{ts}=\langle 
\mathrm{ER}_0, L_{\mathrm{TS}}, \mathrm{env}\rangle$.
  
  Итак, территориальная система описывается при помощи входящих в нее 
экономических районов ($\mathrm{ER}_0\subseteq \mathrm{ER}$), связей между ними (которые 
задаются через функцию $L_{\mathrm{TS}}: \mathrm{ER}\times \mathrm{ER}\rightarrow$\linebreak $\rightarrow  M^{\mathrm{NR}}$, 
действующую на пару экономических\linebreak районов и отобра\-жа\-ющую ее на 
подмножество множества межрайонных товарных рынков) и структурой 
внешней среды.
  
  Структура внешней среды $\mathrm{env}\in \mathrm{ENV}$ (где ENV~--- множество всех 
сред) специфицируется как $\mathrm{env}=\{S^{\mathrm{EX}}, M^{\mathrm{EX}}\}$, где $S^{\mathrm{EX}}$~--- 
множество внешних ресурсов (товарных, сырьевых и~т.\,д.), $M^{\mathrm{EX}}$~--- 
множество внешних рынков.
  
  Далее определим структуру самого экономического района.

Элемент $\mathrm{er}\in \mathrm{ER}$ можно представить в виде $\mathrm{er}=$\linebreak $=\langle \mathrm{TPC}_0, \mathrm{SQ}, 
L_{\mathrm{TPC}_0}\rangle$, где $\mathrm{TPC}_0\subset \mathrm{TPC}$~--- множество 
тер\-ри\-то\-ри\-аль\-но-про\-из\-вод\-ст\-вен\-ных комплексов, SQ~---
множество площадок, на которых можно разместить ТПК, $L_{\mathrm{TPC}_0}$~--- это 
функция ма\-те\-ри\-аль\-но-то\-вар\-ных связей между ТПК, $L_{\mathrm{TPC}_0}: 
\mathrm{TPC}_0\times$\linebreak $\times \mathrm{TPC}_0\rightarrow M^I$. Она отображает два 
ТПК на некоторое подмножество множества 
внутрирайонных рынков.
  
  Теперь рассмотрим, что в себя включает элемент $\mathrm{sq}\in \mathrm{SQ}$. Он имеет 
структуру $\mathrm{sq}=\langle G,\mathrm{NR},\mathrm{NP}\rangle$, где участок двухмерного пространства 
$G\subset \mathcal{R}^2$ задает локацию площадки, $\mathrm{NR}\subseteq S^I$ 
  ($S^I$~--- мно-\linebreak жество внутрирайонных ресурсов) обозначает\linebreak име\-юще\-еся на 
площадке множество природных ресурсов, $\mathrm{NP}\subseteq \mathrm{LP}^I$ ($\mathrm{LP}^I$~--- 
внутрирайонные трудовые ресурсы) обозначает имеющиеся на площадке 
трудовые ресурсы.
  
  Опишем теперь активные сущности системы~--- агентов.
  
  \smallskip
  
  \textbf{Агенты экономических районов.} Такие агенты имеют 
собственное внутреннее состояние, состоящее из элемента $\mathrm{er}\in \mathrm{ER}_0$, могут 
взаимодействовать как друг с другом, так и со средой, в которую помещена 
система. Внешнее состояние агента описывается набором производственных 
специализаций экономического района в виде $\langle f_1, \ldots , f_n\rangle$, 
где $f_i$ является элементом множества производственных специализаций 
$f_i\in F$, $i\in \{1, \ldots , n\}$. В~рамках данной модели предполагается, что 
производственные специализации ЭР образуются за счет 
внешних специализаций ТПК, находящихся на террито-\linebreak рии ЭР.
  
  В модели также используются агенты ТПК, но их описание выходит за рамки 
данной работы.
  
  Будем считать, что любая производственная специализация~$f$ 
соответствует некоторому элементу из множества товаров $\mathrm{pr}\in \mathrm{PR}$.

\subsection{Переговоры о специализации}

  В процессе функционирования территориальной системы агенты 
экономических районов договариваются друг с другом о наборе специализаций 
системы. Каждый агент ЭР способен предложить только какой-то 
ограниченный спектр специализаций, в то время как все множество имеющихся 
в сис\-те\-ме специализаций может быть шире. Более того, каждый район может 
делать лишь частичный вклад в конкретную специализацию сис\-те\-мы. 
Например, один экономический район~$\mathrm{er}_1$ производит 60\%, а другой 
экономический район $\mathrm{er}_2$ производит 40\% от всего объема 
продукции~pr, поставляемой на внешний рынок. Однако для этого случая 
трудоемкость реализации переговоров даже для десяти районов и нескольких 
десятков специализаций является весьма впечатляющей, поэтому в данной 
работе будем исходить из предположения, что каждый экономический район в 
рамках специализации производит 100\% требуемого товара.
  
  Алгоритм ведения переговоров между агентами состоит из двух шагов:
  \begin{enumerate}[(1)]
\item генерация предложений; 
\item выбор единственного предложения из уже имеющихся.
\end{enumerate}

  В зависимости от выбранного метода реализации шаги могут повторяться по 
нескольку раз. Например, это может произойти, когда агенты на\linebreak каж\-дой 
итерации алгоритма получают новую дополнительную информацию, которая 
влияет на состав генерируемых ими предложений. 

Представим сделку~$\lambda$ между агентами ЭР в следующем виде:
$$
\lambda\! =\! \left \{
\underbrace{f_1^1, \ldots , f^1_{n_1}}_{\mbox{\footnotesize\textit{набор специализаций~} 
}\mathrm{er}_1}\!,\ldots , \underbrace{f_1^{\vert \mathrm{ER}_0\vert}, \ldots , f_{n_{\vert 
\mathrm{ER}_0\vert}}^{\vert \mathrm{ER}_0\vert}}_{\mbox{\footnotesize\textit{набор специализаций~}}\mathrm{er}_2}
\right \},
$$
где $f_j^i$~--- производственная специализация $\mathrm{er}_i$ из набора, 
предлагаемого агентом экономического района~$er_i$. 
  
  Каждый экономический район может иметь несколько возможных наборов 
собственных специализаций

\noindent
  $$
  \left \{
  \begin{matrix}
  \{f_{1,1}, \ldots , f_{1,n_{j_1}}\}\,,\\
  \{f_{2,1}, \ldots , f_{2,n_{j_2}}\}\,,\\
  \ldots\,,\\
  \{f_{m,1}, \ldots , f_{m,n_{j_m}}\}\,,\\
  \{\}
  \end{matrix}
  \right \}\,.
  $$
  
  Заметим, что это множество содержит в качестве элемента также и пустое 
множество.
  
  Все множество сделок~$\Lambda$ описывается декартовым произведением 
множеств возможных наборов собственных специализаций экономических 
районов $\mathrm{er}\in \mathrm{ER}_0$:
\begin{multline*}
  \Lambda = \left \{
  \begin{matrix}
  \{f^1_{1,1}, \ldots , f^1_{1,n_{j_{1,1}}}\}\,,\\
  \{f^1_{2,1}, \ldots , f^1_{2,n_{j_{2,1}}}\}\,,\\
  \ldots\\
  \{f^1_{m_1,1}, \ldots , f^1_{m_1,n_{j_{1,m_1}}}\}\,,
  \end{matrix}
  \right \}
  \times\ldots\\
  {}\ldots\times
  \left \{
  \begin{matrix}
  \{ f^{\vert \mathrm{ER}\vert}_{1,1},\ldots , f_{1,n_{j_{1,1}}}^{\vert \mathrm{ER}\vert}\}\\
  \{ f^{\vert \mathrm{ER}\vert}_{2,1},\ldots , f_{2,n_{j_{2,1}}}^{\vert \mathrm{ER}\vert}\}\\
  \ldots\\
  \{ f^{\vert \mathrm{ER}\vert}_{m_{\vert \mathrm{ER}\vert }},\ldots , f_{m_{\vert 
\mathrm{ER}\vert},n_{j_{\vert \mathrm{ER}\vert,m_{\vert \mathrm{ER} \vert}}}}^{\vert \mathrm{ER}\vert}\}\\
  \end{matrix}
  \right \}
  \end{multline*}
  
  Теперь попытаемся оценить трудоемкость построения такого множества. Это 
важно ввиду следующего. Предположим, имеется некоторое множество сделок 
и необходимо выбрать такую из них, при которой территориальная система 
получит наибольший доход среди всех остальных сделок.
  
  Так как элементы множества не упорядочены, задача выбора сделки (задача 
поиска в неупорядоченном массиве) в худшем случае будет иметь 
трудоемкость, пропорциональную числу элементов этого множества.
  
  В данной работе будем исходить из предположения, что один экономический 
район производит товаров в рамках своей специализации вполне достаточно, 
чтобы полностью покрыть потребность в них на внешнем рынке, а также 
считаем, что чис\-ло новых возможных специализаций для района и размер 
предлагаемого набора специализаций для развития невелики: 3 и~2 
соответственно. Также учитываем то, что состав специализаций района может 
иметь переменную длину от~0 до~2 (т.\,е.\ множество предложений может 
содержать пустое множество). Однако полное число возможных специализаций 
для территориальной системы может быть и больше. Число экономических 
районов равно~12 (в соответствии с делением, принятым в~[5]). При 
таких допущениях подсчитаем максимальное число сделок: 
  $$
  (C_3^2+C_3^1+1)^{12}=13841287201\approx 1{,}4\cdot 10^{10}\,.
  $$
%\smallskip

\noindent
\textbf{Замечание:} можно уменьшить размер получаемого множества путем 
введения ограничения на набор специализаций для системы, т.\,е.\ он не должен 
содержать повторяющихся позиций, так как одинаковые специализации 
приводят к конкуренции, что, в свою очередь, влечет снижение доходов агентов 
и, следовательно, всей системы.

\medskip

\noindent
\textbf{Алгоритм 1.} Выбор набора специализаций территориальной системы.
\begin{enumerate}[1.]
\item Выбирается агент, принимающий предложения от других агентов,~--- 
посредник. Например, это может быть агент, соответствующий 
экономическому району с номером~1. 

\smallskip

\noindent
\textbf{Замечание:} экономические районы можно упорядочить по названиям в 
лексикографическом порядке, а затем присвоить им порядковые номера.
\item Агенты ЭР формируют наборы возможных специализаций и посылают их 
аген\-ту-по\-сред\-ни\-ку.
\item Агент-посредник на основе полученной информации формирует 
множество сделок и выбирает ту, которая принесет наибольший доход 
территориальной системе, с помощью функции~$R$.
\item Окончательная сделка рассылается аген\-том-по\-сред\-ни\-ком всем 
остальным агентам.
\item Агенты из полученного сообщения извлекают свое множество 
специализаций.
\end{enumerate}

\medskip

\noindent
\textbf{Примечание 1.} Можно преобразовать данный ал\-горитм в полностью 
децентрализованный, если\linebreak обязать агентов рассылать свои предложения друг 
другу. Каждый агент, зная функцию~$R$, может самостоятельно сформировать 
множество сделок и выбрать среди них наиболее полезную для популяции, 
после чего определить свой состав производственных специализаций. Можно 
утверждать, что все агенты получат в итоге одну и ту же сделку, если задать 
дополнительное условие: в случае, если доход от разных сделок одинаковый, то 
предпочтение отдается той, которая лексикографически меньше всех других.
\smallskip

\noindent
\textbf{Примечание 2.} Для получения приемлемой сделки также можно 
использовать \textit{монотонный протокол уступок и стратегию Жозена}, 
описанные в книге Вулдриджа, посвященной многоагентному 
моделированию~[2]. Такое решение не гарантирует достижения глобального 
максимума на множестве сделок, однако с помощью него возможно заключение 
Па\-ре\-то-оп\-ти\-маль\-ных сделок.

\subsection{Функции развития и алгоритм выбора специализаций района}

  Теперь перейдем к вопросу формирования предложений в рамках 
экономического района $\mathrm{er}\in \mathrm{ER}$.
  
  Будем считать, что каждый экономический район $\mathrm{er}\in \mathrm{ER}$ уже имеет 
некоторую совокупность отраслей специализации $\{f_1, \ldots , f_n\}$, которая 
формируется за счет входящих в его состав ТПК.
  
  Введем семейство функций $CH: 2^F\rightarrow 2^F$ (где~$F$, как уже 
упоминалось, является множеством производств по специализациям):
  \begin{enumerate}[1)]
\item CH$_0$~--- пустая функция, которая отображает аргумент сам на себя;
\item CH$_1$~--- функция, которая <<ликвидирует>> производства с 
отрицательной прибылью.
Предположим, что имеется совокупность про\-извод\-ственных отраслей 
экономического \mbox{района} $\mathrm{er}\in$\linebreak $\in  \mathrm{ER}_0$ представленная в виде $\{f_1,f_2\}$, 
причем~$f_2$ является убыточной. Тогда CH$_1(\{f_1,f_2\})=$\linebreak $=\{f_1\}$. Если 
убыточным окажется и производство~$f_1$, то значением функции станет 
пустое множество.
\item CH$_2$~--- функция, отвечающая за применение новых технологий. Она не 
изменяет размер множества отраслей специализации экономического района 
$\mathrm{er}\in \mathrm{ER}$, однако способствует повышению доходности отрасли. Пусть 
имеются новые технологии для~$f_1$. Тогда 
CH$_2(\{f_1,f_2\})=\{f_1^\prime,f_2\}$.
\item CH$_3$~--- функция, отвечающая за открытие новой производственной 
специализации или освоение нового ресурса. Пусть $\{f_1\}$~--- имеющийся 
набор специализаций для экономического района $\mathrm{er}\in \mathrm{ER}$ и появилась 
возможность открытия новой специализации~$f_2$. Тогда $\{f_1,f_2\}$ будет 
принадлежать к множеству об\-ласти значений данной функции. Заметим, что 
вариантов открытия новой производственной специализации может быть 
несколько, соответственно, для одного аргумента может быть несколько 
значений функции. В~данной работе будем рассматривать только случай, когда 
функция либо не определена (как для данного аргумента, так и вообще), либо 
имеет единственное значение. В нашем случае CH$_3(\{f_1\})=\{f_1,f_2\}$.
\end{enumerate}


\noindent
\textbf{Замечание:} Функции CH$_1$, CH$_2$, CH$_3$ заданы лишь частично, 
т.\,е.\ они определены не на всем множестве~$2^F$. Доопределим их функцией 
CH$_0$, которая определена на всем множестве~$2^F$. 
  
  \smallskip
  
  Теперь опишем простой алгоритм принятия решения о применении той или 
иной функции из семейства~CH (он используется агентами ЭР для 
формирования наборов специализаций при проведении переговоров). Для этого 
введем дополнительную функцию дохода экономического района для 
заданного набора специализаций (которая определяется аналогично функции 
дохода для территориальных систем).
  Функция $R: \mathrm{ER}\times F\rightarrow \mathcal{R}$ преобразует экономический 
район с заданным набором отраслей в величину дохода, который будет иметь 
район с такими производствами.

\medskip

\noindent
\textbf{Алгоритм 2.} Выбор программы развития района.

\textit{Входные данные}: er и $\{f_1, \ldots ,f_n\}$ (экономический район~$er$ 
с набором специализаций).
  
  Последовательно применяем функции CH$_0$, CH$_1$, CH$_2$, CH$_3$ к 
одному и тому же набору $\{f_1, \ldots ,f_n\}$ и среди результатов выбираем те, 
при которых функция~$R$ принимает наибольшее значение. 

\textit{Выходные данные}: результаты в виде $\{f_1^*, \ldots , f_n^*\}$ (набор 
специализаций, оптимальный по доходу на множестве результатов применения 
функций из семейства~CH от аргумента $\{f_1, \ldots , f_n\}$ для 
района~er).

Очевидно, $R(\mathrm{er},\{f_1, \ldots , f_n\})\leq R(\mathrm{er},\{f_1^*, \ldots$\linebreak $\ldots , f_n^*\})$.

\medskip

\noindent
\textbf{Замечание 1.} В~силу того что функций всего четыре, можно 
использовать все получившиеся результаты (а не только оптимальные) для 
дальнейшего выбора набора специализаций территориальной системы. Это 
увеличивает пространство возможных решений на уровне системы. 
  
  (Правильнее будет сказать, что число классов функций заданного вида равно 
четырем, однако в данной работе считается, что каждый из классов содержит 
только один элемент.)
  \medskip
  
  Применение функции CH$_3$ на внутрирайонном уровне может приводить к 
тому, что в er возникнет потребность в определенном товаре, что может 
привести к появлению межрайонного рынка $m\in M^{\mathrm{NR}}$ так же, как и к 
увеличению числа внешних связей территориальной системы.

\section{Заключение}

  В работе описан один из возможных подходов к моделированию развития 
территориальных систем. Были освещены далеко не все аспекты, а лишь самые 
ключевые моменты: представление территориальной системы с помощью 
взаимодействующих агентов, выбор программы развития\linebreak экономического 
района, формирование набора специализаций территориальной системы путем 
переговоров. В~написании статьи очень помогли работы М.\,К.~Бандмана, 
М.~Вулдриджа, Ф.~Котлера, Н.\,Н.~Колосовского 
и~др.~\cite{1-chu, 2-chu, 6-chu, 7-chu}. 
  
  В дальнейшем планируется подробно остановиться на алгоритмах 
модернизации и формирования ТПК в пределах экономического района в 
рамках описанной формальной модели.

{\small\frenchspacing
{%\baselineskip=10.8pt
\addcontentsline{toc}{section}{Литература}
\begin{thebibliography}{9}

\bibitem{1-chu}
\Au{Бандман М.\,К., Бурматова О.\,П., Воробьева~В.\,В.}
Моделирование формирования 
тер\-ри\-то\-ри\-аль\-но-про\-из\-вод\-ст\-вен\-ных комплексов.~--- 
Новосибирск: Наука, 1976.

\bibitem{2-chu}
\Au{Wooldridge M.}
An introduction to multiagent systems.~--- New York: John Wiley \& Sons, 2002.

\bibitem{3-chu}
\Au{Shoham Y., Leyton-Brown~K.}
Multiagent systems: Algorithmic, game-theoretic, and logical foundations.~--- 
Cambridge University Press, 2009.

\bibitem{4-chu}
\Au{Фейерабенд П.}
Против метода. Очерк анархистской тео\-рии познания~/ Пер. с англ. 
А.\,Л.~Никифорова.~--- М.: АСТ, 2007. 

\bibitem{5-chu}
Общероссийский классификатор экономических регионов ОК 024-95 (ОКЭР). 
Утвержден постановлением Госстандарта РФ от 27~декабря 1995~г. №\,640, в 
ред. изменения №\,1, ноябрь 1998~г., с изм. и доп. №\,2/99, №\,3/2000, 
№\,4/2001, №\,5/2001.

\bibitem{6-chu}
\Au{Котлер Ф.}
Маркетинг менеджмент.~--- СПб.: Питер, 2001.

 \label{end\stat}

\bibitem{7-chu}
\Au{Колосовский Н.\,Н.}
Теория экономического 
районирования.~--- М.: Мысль, 1969.


 \end{thebibliography}
}
}


\end{multicols}  