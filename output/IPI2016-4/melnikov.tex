\def\stat{melnikov}

\def\tit{ОБОБЩЕННЫЙ СТАТИСТИЧЕСКИЙ МЕТОД АНАЛИЗА ТЕКСТОВ, ОСНОВАННЫЙ 
НА~РАСЧЕТЕ РАСПРЕДЕЛЕНИЙ ВЕРОЯТНОСТЕЙ ЗНАЧЕНИЙ СТАТИСТИК}

\def\titkol{Обобщенный статистический метод анализа текстов, основанный на 
расчете распределений вероятностей %значений 
статистик}

\def\aut{А.\,К.~Мельников$^1$, А.\,Ф.~Ронжин$^2$}

\def\autkol{А.\,К.~Мельников, А.\,Ф.~Ронжин}

\titel{\tit}{\aut}{\autkol}{\titkol}

\index{Мельников А.\,К.}
\index{Ронжин А.\,Ф.}
\index{Melnikov A.\,K.}
\index{Ronzhin A.\,F.}


%{\renewcommand{\thefootnote}{\fnsymbol{footnote}} \footnotetext[1]
%{Работа выполнена при финансовой поддержке РФФИ (проект 16-37-00485).}}


\renewcommand{\thefootnote}{\arabic{footnote}}
\footnotetext[1]{НТЦ ЗАО <<ИнформИнвестГрупп>>, \mbox{ak@iigroup.ru}}
\footnotetext[2]{Институт точной механики и~вычислительной техники 
им.\ С.\,А.~Лебедева РАН, \mbox{raf@zao-zt.ru}}

  

     \Abst{Рассмотрены возможности применения точных и~предельных 
распределений вероятностей значений статистик для построения критериев согласия 
в~рамках анализа текстовой информации. Исследованы значения граничных 
параметров, для которых возможен расчет точных распределений на современном 
этапе. Разработан обобщенный статистический метод анализа (ОСМА) текстов, применимый 
при широком спектре значений параметров текстов.}
     
     \KW{вероятность; точное распределение; предельное распределение; статистика; 
критерий; частота; сложность алгоритма; производительность многопроцессорной 
вычислительной системы; метод анализа}

\DOI{10.14357/19922264160409} 


\vskip 10pt plus 9pt minus 6pt

\thispagestyle{headings}

\begin{multicols}{2}

\label{st\stat}
   
\section{Введение}

    Различные задачи автоматической обработки результатов наблюдений 
сводятся к~обработке потоков данных, которые далее будем называть 
текстами.
    
    Тексты представляют собой последовательности знаков длины~$n$ 
в~некотором алфавите мощности~$N$. В~данных имеются тексты, 
которые являются реализациями случайных последовательностей 
с~равномерным распределением на множестве всех своих значений, или, 
другими словами, случайными выборками длины~$n$ из равновероятного 
распределения на алфавите мощности~$N$, и~тексты с~уникальными 
свойствами. Такие тексты, во-пер\-вых, не являются выбором из 
равномерного распределения; во-вто\-рых, обладают структурными 
зависимостями внутри последовательности.
    
     Алгоритмы определения структурных зависимостей сложны 
и~требуют значительного вычислительного ресурса, поэтому предлагается 
двухэтапная процедура. На первом этапе выбираются самые 
неравновероятные тексты, а~затем применяются алгоритмы поиска 
структурных зависимостей.
    
    В ряде случаев, когда данные образуют непрерывный поток 
и~имеются естественные ограничения на ресурс памяти, нет возможности 
накопить тексты, выбрать из них самые неравновероятные и~обработать их 
по алгоритмам поиска структурных зависимостей. Тогда по критериям 
согласия с~равновероятным распределением отсеивают тексты без 
уникальных свойств и~к оставшимся текстам применяют алгоритмы поиска 
структурных зависимостей. В~этом случае размер критерия согласия 
определяет долю ложно принятых текстов без уникальных свойств или 
объем лишней работы по поиску текстов с~уникальными свойствами. Для 
вычисления размера критерия согласия необходимо знать вероятностные 
распределения при\-ме\-ня\-емых в~критериях согласия статистик. 
    
    Целью данной работы является описание границ, для которых 
возможно вычисление точных значений вероятностных распределений на 
современных многопроцессорных вычислительных сис\-те\-мах~[1], 
и~построение ОСМА текстов, применимого для всех значений их параметров.
    
\section{Статистический анализ текстов, критерии согласия}

    Сосредоточимся на первом этапе анализа текстов. 
    
    Задача состоит в~том, чтобы относительно $\nu$-го текста $T_n(\nu)$ 
длины~$n$
    $$
T_n(\nu) = \{t_1(\nu), \ldots, t_n(\nu)\}
$$
проверить простую гипотезу~$H_0$, состоящую в~том, что все знаки 
текста (испытания) $t_i(\nu)$, $i \hm= 1, \ldots, n$, принимают значения из 
множества (алфавита) $A_N \hm=\{a_1, \ldots, a_N\}$ мощности~$N$ 
c~равными вероятностями ${\sf P}\{t_i(\nu) \hm= a_j\} \hm= 1/N$ для $i \hm= 1, 
\ldots, n$, $j \hm= 1, \ldots, N$, т.\,е.\ наблюдается выборка из 
равновероятного полиномиального распределения.
    
    В качестве альтернативы выступает сложная гипотеза~$H$, которая 
заключается в~том, что текст $T_n(\nu)$ не является выборкой из 
равновероятного полиномиального распределения в~том смысле, что 
существует такое~$j$, что ${\sf P}\{t_i(\nu) \hm= a_j\} \not= 1/N \vert  i \hm= 1, \ldots, 
n$.
    
    Будем отвергать гипотезу~$H_0$ при $S_n(T_n(\nu)) \hm\geq c$, 
где~$S_n$~--- функция от текста~$T_n(\nu)$ (статистика), а~константа~$c$ 
выбирается исходя из заданного размера критерия~$\alpha$: ${\sf P}\{S_n 
\hm\geq c\} \hm= \alpha$.
    
\section{Расчет точных распределений значений статистик}

    Не ограничивая общности приводимых рас\-суж\-де\-ний, в~качестве 
статистики~$S_n$ критерия будем рассматривать введенную 
более~100~лет назад Карлом Пирсоном~[2, 3] статистику хи-квад\-рат~$\chi_n$:
    $$
    \chi_n=\sum\limits_{i=1}^N \fr{(h_i-np_i)^2}{np_i}\,,
    $$
где $h_i$~--- частота встречаемости знака (исхода)~$a_i$ 
в~первых~$n$~испытаниях; $n$~--- длина текста (объем выборки); $N$~--- 
число исходов полиномиальной схемы (мощность алфавита~--- $A_N$); 
$p_i$~--- вероятность $a_i$-го исхода.

    Для расчета точного распределения вероятности ${\sf P}\{S_n\hm\geq c\}$ 
методом <<полного перебора>> необходимо проделать следующие шаги.
    \begin{description}
    \item[Шаг~1:] cгенерировать множество $\overline{\sf T}(n, A_N)$ всех 
неповторяющихся текстов длины~$n$, состоящих только из знаков, 
принадлежащих алфавиту $A_N\hm = \{a_1,\ldots, a_N\}$ мощности~$N$:
$$
\overline{\sf T}(n,A_N) = \left\{T_n(\nu)\vert  \nu= 1, \ldots, N^n\right\}\,.
$$
    \item[Шаг~2:] получить множество $H(n,N)$ массивов 
час\-тот встречаемости $h(\nu)$ знаков алфавита~$A_N$ в~текстах $T_n (\nu)$:
    \begin{multline*}
    H(n, N) ={}\\
    {}= \left\{h(\nu)= \left(h_1(\nu), \ldots, h_N 
(\nu)\right)\vert  \nu= 1, \ldots, N^n\right\}\,,\hspace*{-4.2pt}
    \end{multline*}
где $h_i(\nu)$~--- частота встречаемости знака~$a_i$ алфавита~$A_N$ 
в~$\nu$-м тексте $T_n(\nu)$ массива текстов~$H(n, N)$.
    \item[Шаг~3:] получить множество значений статистик 
{\textbf{\ptb{\v{S}}}}$_n(N)$ на множестве массивов частот встре\-ча\-емости знаков 
$H(n, N)$:
    $$
    \mbox{{\textbf{\ptb{\v{S}}}}}_n(N)= \left\{S_n\left(h(\nu)\right) \vert \nu= 1, \ldots, N^n\right\}\,,
    $$
применяя формулу выбранной статистики~$S_n$ и~учитывая 
равновероятность исходов полиномиальной схемы. 
    \item[Шаг~4:] получить распределение вероятностей значений 
статистики~$S_n$~--- ${\sf P}_T\{S_n \hm\geq c\}$. Для этого надо провести 
маркировку множества значений статистик~{\!\textbf{\ptb{\v{S}}}}$_n(N)$ в~интервалы 
значений $[c, +\infty)$, где, например, $c\hm = 1(1)100$.
    \end{description}
    
%    \vspace*{3pt}

{\small
 \begin{center} 
\parbox{64mm}{{{\tablename~1}\ \ \small{Максимальные значения па\-ра\-мет\-ров рассчитанных точных распределений 
статистик $S_n$}}
}

\vspace*{6pt}

 %
% \tabcolsep=7pt
 \begin{tabular}{|c|c|}
\hline
\tabcolsep=0pt\begin{tabular}{c}Объем выборки\\ (длина текста)  $n$\end{tabular}&\tabcolsep=0pt\begin{tabular}{c}Число исходов\\
 полиномиальной схемы\\ 
(мощность алфавита) $N$\end{tabular}\\
\hline
50& 2\\
30&3\\
27&4\\
22&5\\
19&6\\
16&7\\
15&8\\
14&9\\
11&10\hphantom{9}\\
\hphantom{9}8&16\hphantom{9}\\
\hphantom{9}6&26\hphantom{9}\\
\hphantom{9}4& 32\hphantom{9}\\
\hline
\end{tabular}
\end{center}
\vspace*{-9pt}
}

   

{ \begin{center}  %fig1
% \vspace*{6pt}
 \mbox{%
\epsfxsize=75.812mm
\epsfbox{mel-1.eps}
}
\end{center}

\vspace*{-5pt}

\noindent
{{\figurename~1}\ \ \small{Диаграмма параметров рассчитанных точных распределений 
статистик~$S_n$}}
}

\vspace*{8pt}
    
    Методом <<полного перебора>> в~XX~в.\ были получены таблицы 
точного распределения статистики $S_n\hm = \chi_n$ для~$n$ и~$N$, 
максимальные значения которых приведены в~табл.~1 и~на рис.~1.




    Необходимо указать, что, применяя некоторые свойства статистик 
(симметричность и~пр.)\ и~рекуррентные соотношения, математикам 
прошлых веков удалось рассчитать таблицы точных распределений,  
точ\-ность которых~10$^{-5}$, для следующих критических значений 
параметров: $N\hm= 2(1)4$, $n\hm=$\linebreak\vspace*{-12pt}



\pagebreak


%\vspace*{12pt}



\addtocounter{figure}{1}
\addtocounter{table}{1}

\noindent
$=26(1)50$; $N\hm= 5$, $n\hm= 26(1)40$; 
$N\hm=6(1)10$, $n\hm= 26(1)30$; $N\hm=10$, $n\hm=1(1)30$; $N\hm=16$, 
$n\hm=1(1)30$; $N\hm=26$, $n=1(1)30$; $N\hm=32$, $n\hm=1(1)30$.
    
    Отметим некоторую универсальность алгоритма расчета точных 
распределений методом <<полного перебора>>. 
    
    Алгоритм может применяться <<последовательно>> 
и~<<параллельно>>:
    \begin{itemize}
    \item при <<последовательном>> применении алгоритма вначале 
генерируем все множество текстов $\overline{\sf T}(n, A_N)$ (шаг~1), затем 
получаем все множество векторов частот встречаемости 
$H(n, N)$ (шаг~2), далее вычисляем все множество 
значений статистик $\mbox{{\textbf{\ptb{\v{S}}}}}_n(N)$ (шаг~3) и, обрабатывая все значения 
множества~$\mbox{{\textbf{\ptb{\v{S}}}}}_n(N)$ (шаг~4), получаем точное распределение 
вероятностей значений статистики~$S_n$~--- ${\sf P}\{S_n \hm\geq c\}$;
    \item при <<параллельном>> применении алгоритма на шаге~1 
выполняем генерацию только одного\linebreak текста~$T_n(\nu)$ из 
множества~$\overline{\sf T}(n, A_N)$ всех непо\-вторяющихся текстов, далее 
для сгенерированного текста вычисляем вектор встре\-ча\-емости 
знаков~$h(\nu)$ (шаг~2) и~значение статистики~$S_n(h(\nu))$ (шаг~3), 
полученное значение статистики заносим в~массив интервалов значений 
$[c, +\infty)$ (шаг~4), после чего возвращаемся к~генерации следующего 
неповторяющегося текста на шаге~1; алгоритм заканчивает свою работу 
тогда, когда на шаге~1 исчерпаны все неповторяющиеся тексты 
рассматриваемой длины.
    \end{itemize}
    
    <<Последовательное>> применение алгоритма требует достаточных 
ресурсов памяти для хранения множеств, <<параллельное>> применение 
не требует больших ресурсов памяти для хранения всех элементов 
множеств и~позволяет распараллеливать выполнение алгоритма по 
данным, что дает возможность применять для его выполнения кластерные 
многопроцессорные вычислительные системы.
    
    Предположим, что в~течение~30~дней (1~месяц) доступен 
вычислительный ресурс производительностью~10$^{15}$ операций 
в~секунду, т.\,е.\ за отведенное время можно произвести для решения 
задачи расчета точных распределений $2{,}59\cdot10^{21}$ операций:
    \begin{multline*}
10^{15}~\mbox{оп./с}\cdot (30~\mbox{дн.} \cdot 24~\mbox{ч}\cdot 
60~\mbox{мин}\cdot 60~\mbox{с})~\mbox{с} 
={}\\
{}=2{,}59\cdot10^{21}~\mbox{оп.}
\end{multline*}
    
    Предположим также, что для расчета одного значения статистики 
$S_n(h(\nu))$ и~маркировки его значения в~интервале $[c,+\infty)$ 
необходимо про\-из\-вести $(5n \hm+ 3N \hm+ 50)$ операций, где
    $5n$~--- число операций на маркировку текста длины~$n$;
    $3N$~--- число\linebreak\vspace*{-12pt}
    
    \columnbreak
    
    {\small
 \begin{center} 
\parbox{66mm}{{{\tablename~2}\ \ \small{Максимальные значения па\-ра\-мет\-ров, для которых на современном этапе 
могут быть рассчитаны точные распределений статистик}}
}

\vspace*{6pt}

 %
% \tabcolsep=7pt
 \begin{tabular}{|c|c|}
\hline
\tabcolsep=0pt\begin{tabular}{c}Объем выборки\\ (длина текста)  $n$\end{tabular}&\tabcolsep=0pt\begin{tabular}{c}Число исходов\\
 полиномиальной схемы\\ 
(мощность алфавита) $N$\end{tabular}\\
\hline
58&\hphantom{9}2\\
29&\hphantom{9}4\\
19&\hphantom{9}8\\
17&10\\
12&26\\
11&36\\
\hphantom{9}9&64\\
\hphantom{9}8&128\hphantom{9}\\
\hphantom{9}7& 256\hphantom{9}\\
\hline
\end{tabular}
\end{center}
\vspace*{6pt}
}

   

{ \begin{center}  %fig2
% \vspace*{6pt}
 \mbox{%
\epsfxsize=78.132mm
\epsfbox{mel-2.eps}
}
\end{center}

\vspace*{-3pt}

\noindent
{{\figurename~2}\ \ \small{Диаграмма параметров, для которых на современном этапе могут быть 
рассчитаны точные распределения статистик: \textit{1}~--- область возможного 
расчета точных распределений на современном этапе;
\textit{2}~--- область рассчитанных точных распределений}}
}



\vspace*{18pt}



\addtocounter{figure}{1}
\addtocounter{table}{1}

    
    
    \noindent
     операций на вычисление статистики~$S_n$ на 
равновероятном распределении с~$N$~исходами;
    50~--- число вспомогательных операций.
 Тогда параметры~$n$ и~$N$ должны удовлетворять следующему 
соотношению: $N^n\hm\leq2{,}59\cdot10^{21}/(5n \hm+ 3N \hm+ 50)$.
 
    Используя указанные выше предположения, были рассчитаны 
максимальные значения па\-ра\-мет\-ров, для которых на современном этапе за 
приемлемое время (1~мес.)\ могут быть рассчитаны точные 
распределения статистик. Значения параметров приведены в~табл.~2 и~на 
рис.~2.


    Диаграмма на рис.~2 определяет область~$O_1$, в~которой находятся 
параметры текстов $(n, N)$ множеств~$\overline{\sf T}(n, A_N)$, для 
которых на современном этапе могут быть рассчитаны точные 
распределения статистик~$S_n$. Линию, ограничивающую эту область 
сверху, обозначим через~$G_1$.
     
    Определим множество $M_1\overline{\sf T}(n, A_N)$ как множество текстов, 
являющееся объединением непересе-\linebreak\vspace*{-12pt}

\pagebreak

{ \begin{center}  %fig3
 \vspace*{-6pt}
 \mbox{%
\epsfxsize=78.137mm
\epsfbox{mel-3.eps}
}
\end{center}

\vspace*{-3pt}

\noindent
{{\figurename~3}\ \ \small{Диаграмма параметров точных распределений статистик}}
}



\vspace*{12pt}



\addtocounter{figure}{1}


\noindent
кающихся подмножеств~$\overline{\sf 
T}(j, A_k)$ c параметрами~$j$ и~$k$ из области~$O_1$ на рис.~2:
    \begin{multline*}
    M_1\overline{\sf T}(n,A_N) =\left\{ \cup \overline{\sf T} (j,A_k)\left\vert 
j=1,\ldots, n\,;\right.\right.\\ 
\left.\left.k=1,\ldots, N \right\vert (n,N)\subset O_1\right\}\,.
    \end{multline*}
    
    Анализируя значения параметров на рис.~2, можно сделать вывод 
о~том, что значительный рост производительности вычислительных 
средств не дает сколь\-ко-ни\-будь заметного увеличения значений 
параметров~$n$ и~$N$ в~решении задачи расчета точных распределений 
и,~соответственно, эффективного решения задач анализа текстов.
    
    Назовем область~$O_1$ областью точных распределений. 
Соответствующая диаграмма приведена на рис.~3.



\section{Расчет {\boldmath{$\Delta$}}-точных распределений 
значений статистик}

    Для преодоления возникшего барьера и~создания возможности 
анализа длинных текстов в~алфавитах большой мощности был разработан 
метод расчета $\Delta$-точ\-ных  
распределений~${\sf P}_\Delta\{S_n\hm\geq c\}$~--- распределений, отличающихся 
от точных распределений не более чем на заранее заданную 
величину~$\Delta$.
    
    Для $\Delta=10^{-5}$ по методу расчета $\Delta$-точ\-ных 
распределений были проведены расчеты~${\sf P}_\Delta\{S_n\hm\geq c\}$ для 
параметров~$n$ и~$N$, значения которых приведены в~табл.~3 и~на 
рис.~4.
    


    Диаграмма на рис.~4 определяет линию, ограничивающую область 
возможности расчета $\Delta$-точ\-ных распределений сверху~--- $G_2$. 
Область, заключенную между границами~$G_1$ и~$G_2$, в~которой 
находятся параметры текстов $(n, N)$ множеств~$\overline{\sf T}(n,A_N)$,\linebreak\vspace*{-12pt}

\columnbreak

\noindent
 {\small
 \begin{center} 
\parbox{66mm}{{{\tablename~3}\ \ \small{Значения параметров, для которых проведены расчеты $\Delta$-точ\-ных 
распределений статистик~$S_n$ для $\Delta\hm=10^{-5}$}}
}

\vspace*{6pt}

 %
% \tabcolsep=7pt
 \begin{tabular}{|c|c|}
\hline
\tabcolsep=0pt\begin{tabular}{c}Объем выборки\\ (длина текста)  $n$\end{tabular}&\tabcolsep=0pt\begin{tabular}{c}Число исходов\\
 полиномиальной схемы\\ 
(мощность алфавита) $N$\end{tabular}\\
\hline
150\hphantom{9}&\hphantom{9}2\\
80&\hphantom{9}4\\
70&\hphantom{9}8\\
60&10\\
50&26\\
50&32\\
45&128\hphantom{9}\\
40&256\hphantom{9}\\
\hline
\end{tabular}
\end{center}
\vspace*{6pt}
}

   

{ \begin{center}  %fig4
% \vspace*{6pt}
 \mbox{%
\epsfxsize=79.137mm
\epsfbox{mel-4.eps}
}
\end{center}

\vspace*{-3pt}

\noindent
{{\figurename~4}\ \ \small{Значения параметров, для которых рассчитаны $\Delta$-точ\-ные 
распределения~$S_n$:
\textit{1}~--- область $\Delta$-точ\-ных распределений~$O_2$;
\textit{2}~--- область точных распределений~$O_1$}}
}



\vspace*{18pt}



\addtocounter{figure}{1}
\addtocounter{table}{1}

\noindent 
для которых рассчитаны $\Delta$-точ\-ные распределения 
статистик~$S_n$, определим как~$O_2$.

         
    По аналогии с~$M_1\overline{\sf T}(n, A_N)$ определим 
множество~$M_2\overline{\sf T}(n, A_N)$ как множество текстов, явля\-юще\-еся 
объединением непересекающихся подмножеств~$\overline{\sf T}(j, A_k)$ 
c~параметрами~$j$ и~$k$ из области~$O_2$ на рис.~4:
   \begin{multline*}
M_2(n, A_N)=\left\{\cup \overline{\sf T}(j, A_k) \left\vert j = 1, \ldots, n;\right.\right.\\ 
\left.\left.k = 1,\ldots, N \right\vert (n, N) \subset O_2\right\}\,.
\end{multline*}

    Сложность\ алгоритма расчета $\Delta$-точ\-ных рас\-пределений 
статистик~${\sf P}_\Delta\{S_n\hm\geq c\}$~--- $C({\sf P}_\Delta\{S_n\hm\geq c\})$~--- 
определяется как чис\-ло целочисленных решений системы уравнений:
    \begin{align*}
    \mu_0+\mu_1+\cdots +\mu_k &=N\,;\\
    1\mu_1+2\mu_2+\cdots + k \mu_k&=n\,,
    \end{align*}
в которой $\mu_\nu$~--- число значений частот исходов~$h_i$, 
равных~$\nu$. 
 
 Вектор $(\mu_0, \ldots, \mu_N)$ получил название вектора вторых 
маркировок. 
 
 Значение $k$ подбирается из условия ${\sf P}\{M_n\hm>k\}\hm\leq\Delta$, 
в~котором~$M_n$~--- статистика максимальной частоты: $M_n\hm = 
\max\limits_{i=1}^N h_i$. 
 
 Вероятность ${\sf P}\{M_n \hm> k\}$ вычисляется с~помощью рекуррентной 
формулы, предложенной Б.\,И.~Селивановым:
 $$
{\sf P}\left\{M_{n+1} < m\right\} =\sum\limits_{\nu=0}^n \begin{pmatrix}
n\\ \nu\end{pmatrix}  P\left\{ M_{n-\nu} < m\right\}  d_{\nu+1}^{(m)}  
\fr{1}{N^\nu}
$$
с начальным условием ${\sf P}\{M_0\hm<m\} \hm= 1$, 
коэффициенты~$d_{\nu+1}^{(m)}$ вычисляются по рекуррентной формуле 
$$
d_{h+1}^{(m)}= -\sum\limits_{\nu=1}^{m-1}\begin{pmatrix}
n\\ \nu\end{pmatrix} d^{(m)}_{n-\nu+1}
$$
с начальными условиями $d_1^{(m)}\hm= 1$, $d_2^{(m)}\hm=\cdots $\linebreak
$\cdots = 
d^{(m)}_{n-1}\hm= 0$, $d_m^{(m)}\hm=-1$, $d^{(m)}_{m+1}\hm= m$.

    Сложность алгоритма расчета $\Delta$-точ\-ных рас\-пределений 
статистик $C({\sf P}_\Delta\{S_n\hm\geq c\})$ может быть оценена сверху как
    $$
    C({\sf P}_\Delta\{S_n\hm\geq c\})\leq \begin{pmatrix}
    \vert n-N\vert+k\\ 
    \vert n-N\vert
    \end{pmatrix}\,,
    $$
    где $\vert x\vert$ обозначает модуль числа~$x$.
    
\section{Предельные распределения значений статистик} 
    
    Одним из замечательных свойств статистики~$\chi_n$ является то, что 
при $m\hm =\min\limits_{i=1}^N np_i\hm\to\infty$ ее предельное распределение 
не зависит от $p_i, \ldots, p_N$ и~совпадает  
с~$\chi^2$-рас\-пре\-де\-ле\-ни\-ем с~($N\hm-1$) степенью  
свободы~\cite{4-mel}. 
    
    Вопрос о том, начиная с~какого~$m$ можно пользоваться предельным 
распределением, задавали себе многие авторы, и~не всегда их мнения 
совпадали. Фишер~\cite{5-mel} рекомендует ограничение $m\hm\geq5$, 
Крамер~\cite{4-mel}~--- $m\hm\geq30$, Кендалл и~Стьюарт~\cite{6-mel}~--- 
$m\hm\geq20$.
    
    Следует напомнить, что теоретическая оценка остаточного члена 
в~предельной теореме, приведенная в~\cite{7-mel}, имеет вид~$O(n^{-(N-1)/N})$ и~мало что дает для практического использования. Поэтому расчет 
непредельных распределений значений статистик для анализа тексов еще 
долго будет оставаться актуальной проблемой для специалистов в~области 
математической статистики, вычислительной математики 
и~программирования.

{ \begin{center}  %fig5
% \vspace*{6pt}
 \mbox{%
\epsfxsize=79.137mm
\epsfbox{mel-5.eps}
}
\end{center}

\vspace*{-9pt}

\noindent
{{\figurename~5}\ \ \small{Область предельных распределений, ограниченная снизу~$G_2$:
     \textit{1}~--- область предельных распределений;
     \textit{2}~--- область $\Delta$-точных распределений~$O_2$;
     \textit{3}~--- область точных распределений~$O_1$}}
}



\vspace*{10pt}



\addtocounter{figure}{1}

        
    Областью предельных распределений назовем область параметров 
$(n, N)$, для которой при современной производительности 
вычислительных средств невозможно рассчитать точные  
и~$\Delta$-точ\-ные распределения. Эта область ограничена снизу~$G_2$. 
Отметим, что граница~$G_2$ рассчитанных $\Delta$-точ\-ных 
распределений на рис.~4 совпадает с~границей~$G_2$ области предельных 
распределений на рис.~5.

\vspace*{-9pt}
    

\section{Обобщенный статистический метод анализа текстов}

\vspace*{-1pt}

    После проведенного анализа возможностей расче\-та и~применения 
точных, $\Delta$-точ\-ных и~предельных распределений вероятностей 
значений статистки~$S_n$ для построения статистического критерия 
согласия с~равновероятным распределением, используемого для отбора 
текстов без уникальных свойств, построим обобщенный метод анализа 
текстов без ограничений на его параметры, такие как длина и~мощность 
алфавита входящих в~него знаков.
{\looseness=-1

}
    
    Пусть необходимо проанализировать текст $T_i(n, A_N)$ длины~$n$ 
из алфавита~$A_N$ мощности~$N$\linebreak и~определить, обладает ли он 
уникальными свойствами. Тогда его стоит дальше исследовать с~по\-мощью 
трудоемких алгоритмов выявления структурных зависимостей. Либо 
данный текст не\linebreak обладает уникальными свойствами и~не представляет 
дальнейшего интереса. 
    
     Обобщенный статистический метод анализа текстов~$T_i(n, A_N)$ 
с~выводом $\mathrm{ОСМА}\{T_i(n, A_N)\}$ будем строить следующим 
образом:
     \begin{enumerate}[(1)]
     \item в~зависимости от значений параметров $(n, N)$ выберем 
распределение ${\sf P}\{S_n \hm\geq c\}$:

\noindent
     \begin{multline*}
     {\sf P}\{S_n \hm\geq c\}={}\\
\hspace*{-3.8mm} {}= \begin{cases}
     {\sf P}_T\{S_n\geq c\}\,, & \hspace*{-39mm}\mbox{если\ } (n, N) \subset O_1,\\ 
&\hspace*{-39mm}\mbox{ограниченной } G_1\ \mbox{сверху};\\
{\sf P}_\Delta\{Sn\geq c\}\,, & \hspace*{-39mm}\mbox{если}\ (n, N) \subset O_2,\\
&\hspace*{-109pt}\mbox{ограниченной } G_1\  \mbox{снизу} \\
&\hspace*{-109pt}\mbox{и\ } G_2\ \mbox{сверху};\\
  {\sf P}\{\chi^2\geq c\}\ \mbox{с}\  (N-1)\ \mbox{степенью\ свободы} &\\
& \hspace*{-39mm}\mbox{в\ противном\ случае};
  \end{cases}
  \end{multline*}
     \item в~соответствии с~выбранным распределение ${\sf P}\{S_n\hm\geq c\}$ 
и~задаваемым размером критерия~$\alpha$ получаем разделяющую 
константу~$c$: ${\sf P}\{S_n\hm\geq c\} \hm= \alpha$;
     \item рассчитываем значение статистики $S_n(T_i(n, A_N))$ от 
анализируемого текс\-та~$T_i(n, A_N)$;
     \item сравниваем значение статистики $S_n(T_i(n, A_N))$ 
с~разделяющей константой~$c$ и~делаем вывод 
($\mathrm{ОСМА}\{T_i(n, A_N)\}$) о~наличии или отсутствии у текста 
$T_i(n, A_N)$ уникальных свойств:
     \begin{itemize}
\item    если $S_n(T_i(n, A_N)) \hm\geq c$, то 
$\mathrm{ОСМА}\{T_i(n, A_N)\}\hm = \mbox{текст}\ T_i(n, A_N)$ обладает 
уникальными свойствами;
  \item  
    если $S_n(T_i(n, A_N)) 
    \hm < c$, то $\mathrm{ОСМА}\{T_i(n, A_N)\} \hm= \mbox{текст}\ 
T_i(n, A_N)$ не обладает уникальными свойствами.
\end{itemize}
    \end{enumerate}
    
    Построение обобщенного метода анализа текстов окончено.
    
    Необходимо отметить, что значения~$G_1$ и~$G_2$ по построению 
определяются алгоритмами расчета точных и~$\Delta$-точ\-ных 
распределений и~производительностью имеющихся на момент 
времени~$\tau$ вычислительных средств~$p(\tau)$, а следовательно, могут 
быть определены как функции от~$\tau$: $G_1(\tau)$ и~$G_2(\tau)$.
    
    Придерживаясь закона Мура, считаем функцию~$p(\tau)$ 
возрастающей, т.\,е.\ при $\tau_1\hm<\tau_2$ $p(\tau_1)\hm \leq p(\tau_2)$. 
Тогда функции~$G_1(\tau)$ и~$G_2(\tau)$ должны обладать сле\-ду\-ющи\-ми 
свойствами:
    \begin{itemize}
    \item  $G_1(\tau) < G_2(\tau)$ по построению;
    \item  если $\tau_1\hm<\tau_2$, то  $G_1(\tau_1)\hm\leq G_1(\tau_2)$;
    \item если $\tau_1\hm<\tau_2$, то $G_2(\tau_1)\hm\leq G_2(\tau_2)$.
    \end{itemize}
    
    Увеличение значений~$G_2(\tau)$~ при росте~$p(\tau)$ поз\-во\-ля\-ет при 
использовании ОСМА для большего чис\-ла значений параметров $(n, N)$ 
применять \mbox{$\Delta$-точ}\-ные распределения вместо предельных, что 
в~несколько раз уменьшает чис\-ло лож\-но отобранных как обладающих 
уникальными свойствами текс\-тов (ошибка первого рода).
    
    Увеличение значений $G_1(\tau)$ при росте~$p(\tau)$ позволяет при 
применении ОСМА для большего чис\-ла значений параметров $(n, N)$ 
применить точ\-ные распределения вместо $\Delta$-точ\-ных, что упрощает 
процедуру проведения анализа текс\-та и~уменьшает его трудоемкость.
    
    Таким образом, предложенный обощенный метод анализа текстов 
ОСМА при росте производительности~$p(\tau)$ вычислительных средств 
позволяет увеличить точность анализа и~снизить его трудоемкость.
    
    
{\small\frenchspacing
 {%\baselineskip=10.8pt
 \addcontentsline{toc}{section}{References}
 \begin{thebibliography}{9}
    \bibitem{1-mel}
    \Au{Каляев В.\,А., Левин~И.\,И., Семерников~Е.\,А., Шмойлов~В.\,И.} 
Реконфигурируемые мультиконвейерные вычислительные структуры.~--- 
Ростов-на-Дону: ЮНЦ РАН, 2008. 397~с.
    \bibitem{2-mel}
    \Au{Pearson K.} On the criterion that a given system of deviations from 
the probable in the case of a~correlated system of variables is such that it can be 
reasonably supposed to have arisen from random sampling~// Philos. 
Mag. Ser.~5, 1900. Vol.~50. No.\,302. P.~157--175.
    \bibitem{3-mel}
    \Au{Smith~P.\,F., Rae~D.\,S., Manderscheid~R.\,W., Silbergeld~S.} Exact 
and approximate distributions of the chi-squared statistic for equiprobability~// 
Commun. Stat., 1979. Vol.~8. No.\,2. P.~131--149.
%    \bibitem{3-mel}
 %   \Au{Neyman J., Pearson~E.\,S.} On the use and interpretation of certain 
%test criteria for purposes of statistical inference~// Biometrika, 1928. Vol.~20A. 
%No.\,1/2. P.~175--240, 264--299.
    \bibitem{4-mel}
    \Au{Крамер Г.} Математические методы статистики~/
    Пер. с~англ.~--- М.: Мир, 
1975. 648~с. (\Au{Cramer~G.} {Mathematical methods of statistics}.~--- 
Princeton, NJ, USA: Princeton University Press, 1946. 592~p.)
    \bibitem{5-mel}
    \Au{Фишер~Р.\,А.} Статистические методы для исследователей~/
    Пер. с~англ.~--- 
М.: Госстатиздат, 1958. 73~с.
(\Au{Fisher~R.\,A.} {Statistical methods for research 
workers}.~--- 12th ed.~--- Edinburgh: Oliver and Boyd, 1954. 356~p.)
    \bibitem{6-mel}
    \Au{Кендалл М.\,Г., Стьюарт~А.} Теория распределений.~--- М.: 
Наука, 1966. 302~с.
\bibitem{7-mel}
\Au{Hutchinson T.\,P.} 1979. The validity of the chi-squared test when expected
frequencis are small: A~list of recent research references~//
{Commun. Stat.~A Theor.} Vol.~8. No.\,4. P.~327--335.
 \end{thebibliography}

 }
 }

\end{multicols}

\vspace*{-6pt}

\hfill{\small\textit{Поступила в~редакцию 02.02.16}}

%\vspace*{8pt}

\newpage

\vspace*{-28pt}

%\hrule

%\vspace*{2pt}

%\hrule

%\vspace*{8pt}


\def\tit{GENERALIZED STATISTICAL METHOD OF~TEXT 
ANALYSIS\\ BASED ON~CALCULATION OF~PROBABILITY 
DISTRIBUTIONS OF~STATISTICAL VALUES}

\def\titkol{Generalized statistical method of~text 
analysis based on~calculation of~probability 
distributions of~statistical values}

\def\aut{A.\,K.~Melnikov$^1$ and A.\,F.~Ronzhin$^2$}

\def\autkol{A.\,K.~Melnikov and A.\,F.~Ronzhin}

\titel{\tit}{\aut}{\autkol}{\titkol}

\vspace*{-9pt}

 
\noindent
$^1$STC CLSC ``InformInvestGroup;'' 125, Bld.~17~Varshavskoye Shosse, Moscow 117587, 
Russian Federation 

\noindent
$^2$S.\,A.~Lebedev Institute of Precision Mechanics and Computer Engineering of 
the Russian Academy of Sciences,\linebreak
$\hphantom{^1}$51~Leninsky  Prosp., Moscow 119991, Russian Federation



\def\leftfootline{\small{\textbf{\thepage}
\hfill INFORMATIKA I EE PRIMENENIYA~--- INFORMATICS AND
APPLICATIONS\ \ \ 2016\ \ \ volume~10\ \ \ issue\ 4}
}%
 \def\rightfootline{\small{INFORMATIKA I EE PRIMENENIYA~---
INFORMATICS AND APPLICATIONS\ \ \ 2016\ \ \ volume~10\ \ \ issue\ 4
\hfill \textbf{\thepage}}}

\vspace*{3pt}
    
    
\Abste{A lot of data streams are a~mixture of random and unique 
data. One of the properties of unique data is the nonuniform 
distribution of probability of encountering the data on the set of the 
values. The procedure of two steps is implemented for distinguishing 
unique data. On the first step of candidate selection, the criterion of 
consensus with the uniform distribution is implemented. On the 
second step, resource-intensive calculation in a~condition of 
indeterminacy is performed in order to check other unique attributes 
of the candidates. The choice of the size of the criterion depends on 
the amount of resources given for the second step. The accuracy of 
calculation determines the quantity of overhead of the second term for 
processing random data and, therefore, a~part of unique data loss. The 
paper analyzes the values of boundary parameters for which at the current 
level of computer technology, one can calculate the exact distribution. 
A~generalized statistical method of text analysis, which can be used 
for a~wide spectrum of text parameters, is developed.}

\KWE{probability; exact distribution; limit distribution; statistics; 
criterion; frequency; algorithm complexity; performance of 
multiprocessor computer system; analysis method}
    
\DOI{10.14357/19922264160409} 

%\vspace*{-9pt}

%\Ack
%\noindent




%\vspace*{3pt}

  \begin{multicols}{2}

\renewcommand{\bibname}{\protect\rmfamily References}
%\renewcommand{\bibname}{\large\protect\rm References}

{\small\frenchspacing
 {%\baselineskip=10.8pt
 \addcontentsline{toc}{section}{References}
 \begin{thebibliography}{9}
    \bibitem{1-mel-1}
    \Aue{Kalyaev, I.\,A., I.\,I.~Levin, E.\,A.~Semernikov, and 
V.\,I.~Shmoylov}. 2008. \textit{Rekonfiguriruemye mul'tikonveyernye 
vychislitel'nye struktury} [Reconfigurable multiconference computational 
patterns]. Rostov-on-Don: YuNTs RAN. 397~p.
    \bibitem{2-mel-1}
    \Aue{Pearson, K.} 1900. On the criterion that a given system of deviations 
from the probable in the case of a correlated system of variables is such that it 
can be reasonably supposed to have arisen from random sampling. 
\textit{Philos. Mag. Ser.~5} 50(302):157--175.
    \bibitem{3-mel-1}
    \Aue{Smith, P.\,F., D.\,S.~Rae, R.\,W.~Manderscheid, and S.~Silbergeld}. 
1979. Exact and approximate distributions of the chi-squared statistic for 
equiprobability. \textit{Commun. Stat.} 8(2):131--149.
    %\bibitem{4-mel-1}
%    \Aue{Neyman, J., and E.\,S.~Pearson}. 1928. On the use and interpretation 
%of certain test criteria for purposes of statistical inference. \textit{Biometrika} 
%20A(1-2):175--240, 264--299.
    \bibitem{4-mel-1}
    \Aue{Cramer, G.} 1946. \textit{Mathematical methods of statistics}. 
Princeton, NJ: Princeton University Press. 592~p.
    \bibitem{5-mel-1}
    \Aue{Fisher, R.\,A.} 1954. \textit{Statistical methods for research 
workers}. 12th ed. Edinburgh: Oliver and Boyd. 356~p.
    \bibitem{6-mel-1}
    \Aue{Kendall, M.\,G., and A.~Stuart.} 1967. \textit{The advanced theory 
of statistics. Vol.~1: Distribution theory}. 3rd ed. London: Charles Griffin 
Co. 439~p.
\bibitem{7-mel-1}
\Aue{Hutchinson, T.\,P.} 1979. The validity of the chi-squared test when expected
frequencis are small: A~list of recent research refernces.
\textit{Commun. Stat.~A Theor.} 8(4):327--335.
\end{thebibliography}

 }
 }

\end{multicols}

\vspace*{-3pt}

\hfill{\small\textit{Received February 2, 2016}}

\Contr

\noindent
\textbf{Melnikov Andrey K.} (b.\ 1956)~--- Candidate of Science (PhD) in technology, associate 
professor, principal scientist, STC CLSC ``InformInvestGroup;'' 125, Bld.~17~Varshavskoye 
Shosse, Moscow 117587, Russian Federation; \mbox{ak@iigroup.ru}

\vspace*{3pt}

\noindent
\textbf{Ronzhin Alexander F.} (b.\ 1954)~--- Doctor of Science in physics and mathematics, 
professor, S.\,A.~Lebedev Institute of Precision Mechanics and Computer Engineering of 
the Russian Academy of Sciences, 51~Leninsky 
Prosp., Moscow 119991, Russian Federation; {\mbox{raf@zao-zt.ru}
\label{end\stat}


\renewcommand{\bibname}{\protect\rm Литература} 