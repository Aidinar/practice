
%Том 10 Выпуск 1-4 Год 2016

\def\stat{cont}
{%\hrule\par
%\vskip 7pt % 7pt
\raggedleft\Large \bf%\baselineskip=3.2ex
А\,В\,Т\,О\,Р\,С\,К\,И\,Й\ \ У\,К\,А\,З\,А\,Т\,Е\,Л\,Ь\ \ З\,А\ \ 2\,0\,1\,6 г. \vskip 17pt
 \hrule
 \par
\vskip 21pt plus 6pt minus 3pt }

\label{st\stat}

\def\tit{\ }

\def\aut{\ }
\def\auf{\ }

\def\leftkol{\ } % ENGLISH ABSTRACTS}

\def\rightkol{\ } %АВТОРСКИЙ УКАЗАТЕЛЬ ЗА 2016 г.} %ENGLISH ABSTRACTS}

\titele{\tit}{\aut}{\auf}{\leftkol}{\rightkol}

\vspace*{-12pt}
\vspace*{-36pt}

{\tabcolsep=3pt
\begin{tabular}{p{372pt}cc}
&\textbf{Вып.} & \textbf{Стр.}\\[6pt]
\Avtors{Агаларов~М.\,Я.} см.~Агаларов~Я.\,М.&&\\
\Avtors{Агаларов~Я.\,М., Агаларов~М.\,Я., Шоргин~В.\,С.} Об
оптимальном пороговом значении длины очереди в одной задаче
максимизации дохода системы массового\linebreak
\\[-12pt]
\hspace*{23pt}обслуживания типа $M/G/1$&2&70--79\\
\Avtors{Алексеевский~Д.\,А.} Применение контекстно-свободных
грамматик для извлечения\linebreak
\\[-12pt]
\hspace*{23pt}онтологии из текстов коротких описаний
статей биологической тематики&1&119--128\\
\Avtors{Андреев~С.\,Д.} см.~Гайдамака~Ю.\,В.&&\\
\Avtors{Андреев~С.\,Д.} см.~Омётов~А.\,Я.&&\\
\Avtors{Архипов~О.\,П., Архипов~П.\,О., Сидоркин~И.\,И.} Вариант
создания локальной системы\linebreak
\\[-12pt]
\hspace*{23pt}координат для синхронизации
изображений выбранных снимков&3&91--97\\
\Avtors{Архипов~П.\,О.} см.~Архипов~О.\,П.&&\\
\Avtors{Белоусов~В.\,В.} см.~Шнурков~П.\,В.&&\\
\Avtors{Белоусов~В.\,В.} см.~Шнурков~П.\,В.&&\\
\Avtors{Бенинг~В.\,Е.} Вычисление асимптотического дефекта
некоторых статистических\linebreak
\\[-12pt]
\hspace*{23pt}процедур, основанных на выборках
случайного объема&4&34--45\\
\Avtors{Борисов~А.\,В., Босов~А.\,В.,
Миллер~Г.\,Б.} Моделирование и мониторинг состояния\linebreak
\\[-12pt]
\hspace*{23pt}VoIP-соединения&2&\hphantom{1}2--13\\
\Avtors{Босов~А.\,В.} см.~Борисов~А.\,В.&&\\
\Avtors{Брюхов~Д.\,О.} см.~Ступников~С.\,А.&&\\
\Avtors{Вольнова~А.\,А.} см.~Калиниченко~Л.\,А.&&\\
\Avtors{Гайдамака~Ю.\,В., Андреев~С.\,Д., Сопин~Э.\,С.,
Самуйлов~К.\,Е., Шоргин~С.\,Я.} Анализ характеристик
интерференции в~модели взаимодействия устройств\linebreak
\\[-12pt]
\hspace*{23pt}с~учетом среды
распространения сигнала&4&\hphantom{1}2--10\\
\Avtors{Гасилов~А.\,В.} см.~Яковлев~О.\,А.&&\\
\Avtors{Гончаров~А.\,В., Стрижов~В.\,В.} Метрическая
классификация временных рядов со\linebreak
\\[-12pt]
\hspace*{23pt}взвешенным выравниванием
относительно центроидов классов&2&36--47\\
\Avtors{Гордов~Е.\,П.} см.~Калиниченко~Л.\,А.&&\\
\Avtors{Горшенин~А.\,К.} Концепция онлайн-комплекса для
стохастического моделирования\linebreak
\\[-12pt]
\hspace*{23pt}реальных процессов&1&72--81\\
\Avtors{Горшенин~А.\,К.} см.~Шнурков~П.\,В.&&\\
\Avtors{Горшенин~А.\,К.} см.~Шнурков~П.\,В.&&\\
\Avtors{Грушо~А.\,А., Грушо~Н.\,А., Забежайло~М.\,И.,
Тимонина~Е.\,Е.} Интеграция ста\-тис\-ти-\linebreak
\\[-12pt]
\hspace*{23pt}ческих и детерминистских
методов анализа информационной безопасности&3&2--8\\
\Avtors{Грушо~А.\,А., Забежайло~М.\,И., Зацаринный~А.\,А.} Об
одном способе сокращения\linebreak\\[-12pt]
\hspace*{23pt}вы\-чис\-ле\-ний при формировании замыканий Галуа&4&\hphantom{1}96--104\\
\Avtors{Грушо~Н.\,А.} см.~Грушо~А.\,А.&&\\
\Avtors{Данишевский~В.\,И.} см.~Колесников~А.\,В.&&\\
\Avtors{Забежайло~М.\,И.} см.~Грушо~А.\,А.&&\\
\Avtors{Забежайло~М.\,И.} см.~Грушо~А.\,А.&&\\
\Avtors{Зализняк~Анна~А., Кружков~М.\,Г.} База данных
безличных глагольных конструкций\linebreak
\\[-12pt]
\hspace*{23pt}русского языка&4&132--141\\
\Avtors{Засыпко~В.\,В.} см.~Шнурков~П.\,В.&&\\
\Avtors{Захарова~Т.\,В., Шестаков~О.\,В.} Анализ точности
вейвлет-обработки аэродина-\linebreak
\\[-12pt]
\hspace*{23pt}мических картин обтекания&3&46--54\\
\Avtors{Зацаринный~А.\,А., Сучков~А.\,П.} Системотехнические
подходы к~созданию системы\linebreak
\\[-12pt]
\hspace*{23pt}поддержки принятия решений
на~основе ситуационного анализа&4&105--113\\
\Avtors{Зацаринный~А.\,А.} см.~Грушо~А.\,А.&&
\end{tabular}
}

\pagebreak

\def\leftkol{АВТОРСКИЙ УКАЗАТЕЛЬ ЗА 2016 г.} % ENGLISH ABSTRACTS}

\def\rightkol{АВТОРСКИЙ УКАЗАТЕЛЬ ЗА 2016 г.} %ENGLISH ABSTRACTS}

%\thispagestyle{myheadings}
\def\leftfootline{\small{\textbf{\thepage}
\hfill ИНФОРМАТИКА И ЕЁ ПРИМЕНЕНИЯ\ \ \ том~10\ \ \ выпуск~4\ \ \ 2016}
}%
 \def\rightfootline{\small{ИНФОРМАТИКА И ЕЁ ПРИМЕНЕНИЯ\ \ \ том~10\ \ \ выпуск~4\ \ \ 2016
 \hfill \textbf{\thepage}}}


{\tabcolsep=3pt
\begin{tabular}{p{373pt}cc}
&\textbf{Вып.} & \textbf{Стр.}\\[2.3pt]
\Avtors{Зацман~И.\,М., Инькова~О.\,Ю., Кружков~М.\,Г.,
Попкова~Н.\,А.} Представление кросс-\linebreak
\\[-12pt]
\hspace*{23pt}языковых знаний о
коннекторах в надкорпусных базах данных&1&106--118\\[.4pt]
\Avtors{Зацман~И.\,М.} см.~Минин~В.\,А.&&\\[.4pt]
\Avtors{Зейфман~А.\,И.} см.~Королев~В.\,Ю.&&\\[.4pt]
\Avtors{Зейфман~А.\,И.} см.~Королев~В.\,Ю.&&\\[.4pt]
\Avtors{Инькова~О.\,Ю.} см.~Зацман~И.\,М.&&\\[.4pt]
\Avtors{Исаченко~Р.\,В., Стрижов~В.\,В.} Метрическое обучение в
задачах мультиклассовой\linebreak
\\[-12pt]
\hspace*{23pt}классификации временных рядов&2&48--57\\[.4pt]
\Avtors{Каданер~А.\,И.} см.~Черток~А.\,В.&&\\[.4pt]
\Avtors{Калиниченко~Л.\,А., Вольнова~А.\,А., Гордов~Е.\,П.,
Киселева~Н.\,Н., Ковалева~Д.\,А., Малков~О.\,Ю.,
Окладников~И.\,Г., Подколодный~Н.\,Л., Позаненко~А.\,С.,
Пономарева~Н.\,В., Ступников~С.\,А., Фазлиев~А.\,З.} Проблемы
доступа к данным в исследо-\linebreak
\\[-12pt]
\hspace*{23pt}ваниях с интенсивным использованием
данных в России&1&\hphantom{1}2--22\\[.4pt]
\Avtors{Каллаос~Н.\,К., Сейфуль-Мулюков~Р.\,Б.} Сложность и~ее
информационное содер-\linebreak
\\[-12pt]
\hspace*{23pt}жание&1&129--139\\[.4pt]
\Avtors{Карасиков~М.\,Е., Стрижов~В.\,В.} Классификация
временных рядов в~пространстве\linebreak
\\[-12pt]
\hspace*{23pt}параметров порождающих
моделей&4&121--131\\[.4pt]
\Avtors{Кириков~И.\,А., Колесников~А.\,В., Листопад~С.\,В.,
Румовская~С.\,Б.} <<Виртуальный консилиум>>~---
инструментальная среда поддержки принятия сложных
диа-\linebreak
\\[-12pt]
\hspace*{23pt}гностических решений&3&81--90\\[.4pt]
\Avtors{Кириков~И.\,А., Колесников~А.\,В., Листопад~С.\,В.,
Румовская~С.\,Б.} Мелкозернистые\linebreak
\\[-12pt]
\hspace*{23pt}гибридные интеллектуальные
системы. Часть 2: Двунаправленная гибриди-\linebreak
\\[-12pt]
\hspace*{23pt}зация&1&\hphantom{1}96--105\\[.4pt]
\Avtors{Киселева~Н.\,Н.} см.~Калиниченко~Л.\,А.&&\\[.4pt]
\Avtors{Ковалева~Д.\,А.} см.~Калиниченко~Л.\,А.&&\\[.4pt]
\Avtors{Ковалёв~С.\,П.} Применение метапрограммирования для
повышения технологич-\linebreak
\\[-12pt]
\hspace*{23pt}ности больших автоматизированных систем&1&56--66\\[.4pt]
\Avtors{Колесников~А.\,В., Листопад~С.\,В., Румовская~С.\,Б.,
Данишевский~В.\,И.} Неформаль-\linebreak
\\[-12pt]
\hspace*{23pt}ная аксиоматическая теория
ролевых визуальных моделей&4&114--120\\[.4pt]
\Avtors{Колесников~А.\,В.} см.~Кириков~И.\,А.&&\\[.4pt]
\Avtors{Колесников~А.\,В.} см.~Кириков~И.\,А.&&\\[.4pt]
\Avtors{Колин~К.\,К.} Гуманитарные аспекты проблемы
информационной безопасности&3&111--121\\[.4pt]
\Avtors{Коновалов М.\,Г., Разумчик~Р.\,В.} О размещении заданий
на двух серверах при неполном наблюдении
\\[-12pt]
\hspace*{23pt}&4&57--67\\[.4pt]
\Avtors{Корепанов~Э.\,Р.} см.~Синицын~И.\,Н.&&\\[.4pt]
\Avtors{Корепанов~Э.\,Р.} см.~Синицын~И.\,Н.&&\\[.4pt]
\Avtors{Королев~В.\,Ю., Зейфман~А.\,И., Корчагин~А.\,Ю.}
Несимметричные распределения Линника как предельные законы для
случайных сумм независимых случай-\linebreak
\\[-12pt]
\hspace*{23pt}ных величин с~конечными дисперсиями&4&21--33\\[.4pt]
\Avtors{Королев~В.\,Ю., Корчагин~А.\,Ю., Зейфман~А.\,И.}
Теорема Пуассона для схемы испытаний Бернулли со~случайной
вероятностью успеха и~дискретный\linebreak
\\[-12pt]
\hspace*{23pt}аналог распределения Вейбулла&4&11--20\\[.4pt]
\Avtors{Корчагин~А.\,Ю.} см.~Королев~В.\,Ю.&&\\[.4pt]
\Avtors{Корчагин~А.\,Ю.} см.~Королев~В.\,Ю.&&\\[.4pt]
\Avtors{Кривенко~М.\,П.} Критерии значимости отбора признаков
классификации&3&32--40\\[.4pt]
\Avtors{Кружков~М.\,Г.} см.~Зализняк~Анна~А.&&\\[.4pt]
\Avtors{Кружков~М.\,Г.} см.~Зацман~И.\,М.&&\\[.4pt]
\Avtors{Кудрявцев~А.\,А.} Байесовские модели массового
обслуживания и надежности:\linebreak
\\[-12pt]
\hspace*{23pt}априорные распределения
с компактным носителем
&1&67--71\\[.4pt]
\Avtors{Кудрявцев~А.\,А.} Зависимые от коэффициента баланса
характеристики в байесовских\linebreak
\\[-12pt]
\hspace*{23pt}моделях с компактным носителем
априорных распределений&3&77--80\\[.4pt]
\Avtors{Кудрявцев~А.\,А., Палионная~С.\,И.} Байесовская
рекуррентная модель роста надеж-\linebreak
\\[-12pt]
\hspace*{23pt}ности: параболическое распределение параметров&2&80--83\\
\end{tabular}
}

\pagebreak

\def\leftkol{АВТОРСКИЙ УКАЗАТЕЛЬ ЗА 2016 г.} % ENGLISH ABSTRACTS}

\def\rightkol{АВТОРСКИЙ УКАЗАТЕЛЬ ЗА 2016 г.} %ENGLISH ABSTRACTS}

%\thispagestyle{myheadings}
\def\leftfootline{\small{\textbf{\thepage}
\hfill ИНФОРМАТИКА И ЕЁ ПРИМЕНЕНИЯ\ \ \ том~10\ \ \ выпуск~4\ \ \ 2016}
}%
 \def\rightfootline{\small{ИНФОРМАТИКА И ЕЁ ПРИМЕНЕНИЯ\ \ \ том~10\ \ \ выпуск~4\ \ \ 2016
 \hfill \textbf{\thepage}}}


{\tabcolsep=3pt
\begin{tabular}{p{373pt}cc}
&\textbf{Вып.} & \textbf{Стр.}\\[2.3pt]
\Avtors{Кудрявцев~А.\,А., Титова~А.\,И.} Байесовские модели
массового обслуживания и~на-\linebreak
\\[-12pt]
\hspace*{23pt}дежности: вырожденно-вейбулловский случай&4&68--71\\[.36pt]
\Avtors{Кучерявый~Е.\,А.} см.~Омётов~А.\,Я.&&\\[.36pt]
\Avtors{Леонтьев~Н.\,Д., Ушаков~В.\,Г.} Анализ системы
обслуживания с входящим потоком\linebreak
\\[-12pt]
\hspace*{23pt}авторегрессионного типа и
относительным приоритетом&3&15--22\\[.36pt]
\Avtors{Листопад~С.\,В.} см.~Кириков~И.\,А.&&\\[.36pt]
\Avtors{Листопад~С.\,В.} см.~Кириков~И.\,А.&&\\[.36pt]
\Avtors{Листопад~С.\,В.} см.~Колесников~А.\,В.&&\\[.36pt]
\Avtors{Малков~О.\,Ю.} см.~Калиниченко~Л.\,А.&&\\[.36pt]
\Avtors{Марков~А.\,С., Монахов~М.\,М.,
Ульянов~В.\,В.} Разложения типа Корниша--Фишера\linebreak
\\[-12pt]
\hspace*{23pt}для распределений статистик, построенных по выборкам случайного
размера&2&84--91\\[.36pt]
\Avtors{Мейханаджян~Л.\,А.} Стационарные вероятности
состояний в системе обслуживания конечной емкости с
инверсионным порядком обслуживания и обобщенным\linebreak
\\[-12pt]
\hspace*{23pt}вероятностным приоритетом&2&123--131\\[.36pt]
\Avtors{Мельников~А.\,К., Ронжин~А.\,Ф.} Обобщенный
статистический метод анализа тек-\linebreak
\\[-12pt]
\hspace*{23pt}стов, основанный на~расчете
распределений вероятностей значений статистик&4&89--95\\[.36pt]
\Avtors{Миллер~Г.\,Б.} см.~Борисов~А.\,В.&&\\[.36pt]
\Avtors{Минин~В.\,А., Зацман~И.\,М., Хавансков~В.\,А.,
Шубников~С.\,К.} Интенсивность цитирования научных
публикаций в изобретениях по ин\-фор\-ма\-ци\-он\-но-ком\-пью\-тер\-ным
технологиям, патентуемых в России отечественными и зарубеж-\linebreak
\\[-12pt]
\hspace*{23pt}ными заявителями&2&107--122\\[.36pt]
\Avtors{Монахов~М.\,М.} см.~Марков~А.\,С.&&\\[.36pt]
\Avtors{Наумов~В.\,А., Самуйлов~К.\,Е.} О связи ресурсных систем
массового обслуживания\linebreak
\\[-12pt]
\hspace*{23pt}с сетями Эрланга&3&\hphantom{1}9--14\\[.36pt]
\Avtors{Окладников~И.\,Г.} см.~Калиниченко~Л.\,А.&&\\[.36pt]
\Avtors{Омётов~А.\,Я., Андреев~С.\,Д., Тюрликов~А.\,М.,
Кучерявый~Е.\,А.} Анализ производительности беспроводной
системы агрегации данных с состязанием для\linebreak
\\[-12pt]
\hspace*{23pt}современных сенсорных сетей&3&23--31\\[.36pt]
\Avtors{Палионная~С.\,И.} см.~Кудрявцев~А.\,А.&&\\[.36pt]
\Avtors{Подколодный~Н.\,Л.} см.~Калиниченко~Л.\,А.&&\\[.36pt]
\Avtors{Позаненко~А.\,С.} см.~Калиниченко~Л.\,А.&&\\[.36pt]
\Avtors{Пономарева~Н.\,В.} см.~Калиниченко~Л.\,А.&&\\[.36pt]
\Avtors{Попкова~Н.\,А.} см.~Зацман~И.\,М.&&\\[.36pt]
\Avtors{Разумчик~Р.\,В.} см.~Коновалов М.\,Г.&&\\[.36pt]
\Avtors{Ронжин~А.\,Ф.} см.~Мельников~А.\,К.&&\\[.36pt]
\Avtors{Румовская~С.\,Б.} см.~Кириков~И.\,А.&&\\[.36pt]
\Avtors{Румовская~С.\,Б.} см.~Кириков~И.\,А.&&\\[.36pt]
\Avtors{Румовская~С.\,Б.} см.~Колесников~А.\,В.&&\\[.36pt]
\Avtors{Самуйлов~К.\,Е.} см.~Гайдамака~Ю.\,В.&&\\[.36pt]
\Avtors{Самуйлов~К.\,Е.} см.~Наумов~В.\,А.&&\\[.36pt]
\Avtors{Сейфуль-Мулюков~Р.\,Б.} см.~Каллаос~Н.\,К.&&\\[.36pt]
\Avtors{Серебрянский~С.\,М.} см.~Тырсин~А.\,Н.&&\\[.36pt]
\Avtors{Сидоркин~И.\,И.} см.~Архипов~О.\,П.&&\\[.36pt]
\Avtors{Синицын~В.\,И.} см.~Синицын~И.\,Н.&&\\[.36pt]
\Avtors{Синицын~В.\,И.} см.~Синицын~И.\,Н.&&\\[.36pt]
\Avtors{Синицын~И.\,Н.} Аналитическое моделирование
нормальных процессов в стохастических системах со сложными
бесселевыми нелинейностями дробного\linebreak
\\[-12pt]
\hspace*{23pt}порядка&3&55--65\\[.36pt]
\Avtors{Синицын~И.\,Н.} Ортогональные субоптимальные фильтры
для нелинейных стоха-\linebreak
\\[-12pt]
\hspace*{23pt}стических систем на многообразиях&1&34--44\\[.36pt]
\Avtors{Синицын~И.\,Н., Корепанов~Э.\,Р.} Нормальныe
условно-оптимальные фильтры и~экстраполяторы Пугачёва для
стохастических систем, линейных относительно\linebreak
\\[-12pt]
\hspace*{23pt}состояния&2&14--23
\end{tabular}
}

\pagebreak

\def\leftkol{АВТОРСКИЙ УКАЗАТЕЛЬ ЗА 2016 г.} % ENGLISH ABSTRACTS}

\def\rightkol{АВТОРСКИЙ УКАЗАТЕЛЬ ЗА 2016 г.} %ENGLISH ABSTRACTS}

%\thispagestyle{myheadings}
\def\leftfootline{\small{\textbf{\thepage}
\hfill ИНФОРМАТИКА И ЕЁ ПРИМЕНЕНИЯ\ \ \ том~10\ \ \ выпуск~4\ \ \ 2016}
}%
 \def\rightfootline{\small{ИНФОРМАТИКА И ЕЁ ПРИМЕНЕНИЯ\ \ \ том~10\ \ \ выпуск~4\ \ \ 2016
 \hfill \textbf{\thepage}}}


{\tabcolsep=3pt
\begin{tabular}{p{373pt}cc}
&\textbf{Вып.} & \textbf{Стр.}\\[2.3pt]
\Avtors{Синицын~И.\,Н., Синицын~В.\,И.} Аналитическое
моделирование распределений в~нелинейных стохастических
системах на многообразиях методом эллипсои-\linebreak
\\[-12pt]
\hspace*{23pt}дальной аппроксимации&1&45--55\\
\Avtors{Синицын~И.\,Н., Синицын~В.\,И.,
Корепанов~Э.\,Р.} Эллипсоидальные субоптимальные\linebreak
\\[-12pt]
\hspace*{23pt}фильтры для
нелинейных стохастических систем на многообразиях&2&24--35\\
\Avtors{Скворцов~Н.\,А.} см.~Ступников~С.\,А.&&\\
\Avtors{Соколов~И.\,А.} см.~Черток~А.\,В.&&\\
\Avtors{Сопин~Э.\,С.} см.~Гайдамака~Ю.\,В.&&\\
\Avtors{Стрижов~В.\,В.} см.~Гончаров~А.\,В.&&\\
\Avtors{Стрижов~В.\,В.} см.~Исаченко~Р.\,В.&&\\
\Avtors{Стрижов~В.\,В.} см.~Карасиков~М.\,Е.&&\\
\Avtors{Ступников~С.\,А., Брюхов~Д.\,О., Скворцов~Н.\,А.} Анализ
системного риска совмест-\linebreak
\\[-12pt]
\hspace*{23pt}ного кредитования над неоднородными
коллекциями данных&1&23--33\\
\Avtors{Ступников~С.\,А.} см.~Калиниченко~Л.\,А.&&\\
\Avtors{Сучков~А.\,П.} см.~Зацаринный~А.\,А.&&\\
\Avtors{Тимонина~Е.\,Е.} см.~Грушо~А.\,А.&&\\
\Avtors{Титова~А.\,И.} см.~Кудрявцев~А.\,А.&&\\
\Avtors{Тырсин~А.\,Н., Серебрянский~С.\,М.} Распознавание
зависимостей на основе обрат-\linebreak
\\[-12pt]
\hspace*{23pt}ного отображения&2&58--64\\
\Avtors{Тюрликов~А.\,М.} см.~Омётов~А.\,Я.&&\\
\Avtors{Ульянов~В.\,В.} см.~Марков~А.\,С.&&\\
\Avtors{Ушаков~В.\,Г.} Система обслуживания 
с~гиперэкспоненциальным входящим потоком\linebreak
\\[-12pt]
\hspace*{23pt}и~профилактиками
прибора&2&92--97\\
\Avtors{Ушаков~В.\,Г.} см.~Леонтьев~Н.\,Д.&&\\
\Avtors{Фазлиев~А.\,З.} см.~Калиниченко~Л.\,А.&&\\
\Avtors{Федосеев~А.\,А.} К вопросу об уменьшении объема порций
учебного материала при\linebreak
\\[-12pt]
\hspace*{23pt}электронном обучении&3&105--110\\
\Avtors{Хавансков~В.\,А.} см.~Минин~В.\,А.&&\\
\Avtors{Хазеева~Г.\,Т.} см.~Черток~А.\,В.&&\\
\Avtors{Хохлов~Ю.\,С.} Многомерное дробное движение Леви и его
приложения&2&\hphantom{1}98--106\\
\Avtors{Черток~А.\,В., Каданер~А.\,И., Хазеева~Г.\,Т.,
Соколов~И.\,А.} Метод кумулятивных сумм для~поиска смены
режима в~процессе Орнштейна--Уленбека на~основе\linebreak
\\[-12pt]
\hspace*{23pt}процесса Леви&4&46--56\\
\Avtors{Чичагов~В.\,В.} Асимптотические разложения средней
абсолютной ошибки несмещенной оценки с равномерно
минимальной дисперсией и оценки максимального правдоподобия в
модели однопараметрического экспоненциального\linebreak
\\[-12pt]
\hspace*{23pt}семейства 
решетчатых распределений&3&66--76\\
\Avtors{Шестаков~О.\,В.} Статистические свойства метода
подавления шума, основанного на\linebreak
\\[-12pt]
\hspace*{23pt}стабилизированной жесткой
пороговой обработке&2&65--69\\
\Avtors{Шестаков~О.\,В.} Усиленный закон больших чисел для
оценки риска в задаче реконструкции томографических изображений
из проекций с коррелированным\linebreak
\\[-12pt]
\hspace*{23pt}шумом&3&41--45\\
\Avtors{Шестаков~О.\,В.} см.~Захарова~Т.\,В.&&\\
\Avtors{Шнурков~П.\,В., Горшенин~А.\,К., Белоусов~В.\,В.}
Аналитическое решение задачи оптимального управления
полумарковским процессом с~конечным множе-\linebreak
\\[-12pt]
\hspace*{23pt}ством состояний&4&72--88\\
\Avtors{Шнурков~П.\,В., Засыпко~В.\,В., Белоусов~В.\,В.,
Горшенин~А.\,К.} Разработка алгоритма численного решения
задачи оптимального управления инвестициями\linebreak
\\[-12pt]
\hspace*{23pt}в закрытой 
динамической модели трехсекторной экономики&1&82--95\\
\Avtors{Шоргин~В.\,С.} см.~Агаларов~Я.\,М.&&\\
\Avtors{Шоргин~С.\,Я.} см.~Гайдамака~Ю.\,В.&&\\
\Avtors{Шубников~С.\,К.} см.~Минин~В.\,А.&&\\
\Avtors{Яковлев~О.\,А., Гасилов~А.\,В.} Ускоренный алгоритм
стереосопоставления на основе\linebreak
\\[-12pt]
\hspace*{23pt}геодезических вспомогательных
коэффициентов&3&\hphantom{1}98--104
\end{tabular}
}

%\thispagestyle{myheadings}
\def\leftfootline{\small{\textbf{\thepage}
\hfill ИНФОРМАТИКА И ЕЁ ПРИМЕНЕНИЯ\ \ \ том~10\ \ \ выпуск~4\ \ \ 2016}
}%
 \def\rightfootline{\small{ИНФОРМАТИКА И ЕЁ ПРИМЕНЕНИЯ\ \ \ том~10\ \ \ выпуск~4\ \ \ 2016
 \hfill \textbf{\thepage}}}

 \label{end\stat}