%\newcommand{\tod}{\stackrel{d}{\longrightarrow}}

\def\stat{korr+kor}

\def\tit{ТЕОРЕМА ПУАССОНА ДЛЯ СХЕМЫ ИСПЫТАНИЙ БЕРНУЛЛИ\\ СО~СЛУЧАЙНОЙ
ВЕРОЯТНОСТЬЮ УСПЕХА\\ И~ДИСКРЕТНЫЙ АНАЛОГ РАСПРЕДЕЛЕНИЯ
ВЕЙБУЛЛА$^*$}

\def\titkol{Теорема Пуассона для схемы испытаний Бернулли со случайной
вероятностью успеха} % и~дискретный аналог распределения Вейбулла}

\def\aut{В.\,Ю.~Королев$^1$, А.\,Ю.~Корчагин$^2$, А.\,И.~Зейфман$^3$}

\def\autkol{В.\,Ю.~Королев, А.\,Ю.~Корчагин, А.\,И.~Зейфман}

\titel{\tit}{\aut}{\autkol}{\titkol}

\index{Королев В.\,Ю.}
\index{Зейфман А.\,И.}
\index{Корчагин А.\,Ю.}
\index{Korolev V.\,Yu.}
\index{Zeifman A.\,I.}
\index{Korchagin A.\,Yu.}


{\renewcommand{\thefootnote}{\fnsymbol{footnote}} \footnotetext[1]
{Работа выполнена при поддержке Российского научного
фонда (проект 14-11-00397).}}


\renewcommand{\thefootnote}{\arabic{footnote}}
\footnotetext[1]{Факультет вычислительной математики и~кибернетики 
Московского государственного 
университета им.\ М.\,В.~Ломоносова; Институт проб\-лем информатики Федерального 
исследовательского центра <<Информатика и~управ\-ле\-ние>> Российской академии наук, 
\mbox{vkorolev@cs.msu.ru}}
\footnotetext[2]{Факультет вычислительной математики и~кибернетики 
Московского государственного университета им.\ М.\,В.~Ломоносова; Институт проб\-лем информатики Федерального 
исследовательского центра <<Информатика и~управ\-ле\-ние>> Российской академии наук,
\mbox{sasha.korchagin@gmail.com}}
\footnotetext[3]{Вологодский государственный университет; Институт проб\-лем 
информатики Федерального исследовательского центра <<Информатика и~управ\-ле\-ние>>
 Российской академии наук; Институт со\-ци\-аль\-но-эко\-но\-ми\-че\-ско\-го 
 развития территорий 
 Российской академии наук, \mbox{a\_zeifman@mail.ru}}
 
 \vspace*{-6pt}

\Abst{Рассматривается задача, связанная с~испытаниями
Бернулли со случайной вероятностью успеха. Сначала в~результате
<<предварительного>> эксперимента определяется значение случайной
величины $\pi\hm\in(0,1)$, которое принимается в~качестве вероятности
успеха в~испытаниях Бернулли. Затем случайная величина $N$
определяется как число успехов в~$k\hm\in\mathbb{N}$ испытаниях
Бернулли с~так определенной ве\-ро\-ят\-ностью успеха~$\pi$. Распределение
случайной величины $N$ называется $\pi$-сме\-шан\-ным биномиальным. 
В~рамках такой схемы испытаний Бернулли со случайной вероятностью
успеха формулируется <<случайный>> аналог классической теоремы
Пуассона для $\pi$-сме\-шан\-ных биномиальных распределений, в~котором
предельным законом оказывается смешанное пуассоновское
распределение. Особое внимание уделено случаю, в~котором смешивающим
распределением является распределение Вейбулла. Соответствующее
смешанное пуассоновское распределение~--- пуас\-сон-вей\-бул\-лов\-ское
распределение~--- предложено в~качестве дискретного аналога
распределения Вейбулла. Обсуждаются некоторые свойства
пу\-ас\-сон-вей\-бул\-лов\-ско\-го распределения. В~частности, показано, что это
распределение является смешанным геометрическим. Предложен
двухэтапный сеточный алгоритм оценивания параметров смешанных
пуассоновских распределений и,~в~част\-ности, пуас\-сон-вей\-бул\-лов\-ско\-го
распределения. Построены статистические оценки верхней границы
сетки. Приведены примеры вычислений по предложенному алгоритму.}

\KW{испытания Бернулли со случайной вероятностью
успеха; смешанное биномиальное распределение; теорема Пуассона;
смешанное пуассоновское распределение; распределение Вейбулла;
пу\-ас\-сон-вей\-бул\-лов\-ское распределение; смешанное геометрическое
распределение; ЕМ-ал\-го\-ритм}

\DOI{10.14357/19922264160402} 

\vspace*{-6pt}


\vskip 8pt plus 9pt minus 6pt

\thispagestyle{headings}

\begin{multicols}{2}

\label{st\stat}



\section{Введение}

Исследование, некоторые результаты которого излагаются в~данной
статье, мотивировано несколькими обстоятельствами. Во-пер\-вых, 
в~последнее время получил серьезное развитие метод прогнозирования
временн$\acute{\mbox{ы}}$х характеристик катастроф в~неоднородных потоках
экстремальных событий (см., например,~\cite{GrigoryevaKorolevSokolov2013}).
 Этот метод можно считать
глубокой модернизацией метода превышений порога (POT-method, POT~---
Peaks Over Threshold). В~рамках этого метода исходный ряд
(маркированный точечный процесс) прореживается таким образом, что
все наблюдения (точки), <<марки>> которых меньше указанного порога,
выбрасываются. При этом, как правило, величина порога определяется
статистически, и~потому вероятность, с~которой очередное
наблюдение отбрасывается, является случайной.

Во-вторых, схема простого прореживания процессов восстановления
(см., например,~\cite{Renyi1956-k, Mogyorodi1971}) предос\-тав\-ля\-ет
естественный подход к~определению \textit{редкого} события, в~рамках
которого это понятие\linebreak связыва\-ется с~хорошо известной конструкцией
пуассоновского процесса, 
характеризующегося показательностью распределения интервалов времени
между событиями (<<восстановлениями>>) в~классе процессов восстановления. Было бы желательно иметь
столь же простой и~основанный на прореживании подход к~конструкции
смешанного пуассоновского процесса. В~рамках такого подхода,
построив смешанный пуассоновский процесс как асимптотическую
аппроксимацию для дважды стохастически прореженных процессов
восстановления, можно надеяться получить дополнительное понимание
структуры многих смежных математических моделей, в~частности
популярных ныне байесовских моделей и~методов, и~более осмысленно
подойти в~выбору соответствующего смешивающего (<<априорного>>)
распределения.

В дальнейшем будет удобнее вести изложение не в~терминах
распределений, а~в~терминах случайных величин (с.в.)\ (предполагая,
что все они заданы на одном вероятностном пространстве
$(\Omega,\mathfrak{A}, {\sf P})$).

Символы $\eqd$ и~$\Longrightarrow$ будут соответственно обозначать
совпадение распределений и~сходимость по распределению.

Функция распределения (ф.р.)\ и~плотность строго устойчивого
распределения с~характеристическим показателем~$\alpha$ и~параметром
формы~$\theta$, определяемого характеристической функцией
$$
\mathfrak{f}_{\alpha,\theta}(t)=
\exp\left\{-|t|^{\alpha}\exp
\left\{-\fr{1}{2}\,i\pi\theta\alpha\mathrm{sign}\,t\right\}\right\},\enskip
t\in\mathbb{R}\,,
$$
где $0<\alpha\le2$, $|\theta|\hm\le\min\{1,{2}/{\alpha}-1\}$, будут
соответственно обозначаться $G_{\alpha,\theta}(x)$ 
и~$g_{\alpha,\theta}(x)$ (см., например,~\cite{Zolotarev1983-k}). Любую
с.в.\ с~ф.р.~$G_{\alpha,\theta}(x)$ будем обозначать~$S_{\alpha,\theta}$.

Симметричным строго устойчивым распределениям соответствует значение
$\theta\hm=0$ и~х.ф.~$\mathfrak{f}_{\alpha,0}(t)\hm=e^{-|t|^{\alpha}}$,
$t\hm\in\mathbb{R}$. Односторонним строго устойчивым законам, сосредоточенным
на неотрицательной полуоси, соответствуют значения $\theta=1$ и
$0\hm<\alpha\hm\le1$. Пары $\alpha\hm=1$, $\theta\hm=\pm1$ отвечают
распределениям, вырожденным в~$\pm1$ соответственно. Остальные
устойчивые распределения абсолютно непрерывны. Явные выражения
устойчивых плотностей в~терминах элементарных функций отсутствуют за
четырьмя исключениями (нормальный закон ($\alpha\hm=2$, $\theta\hm=0$),
распределение Коши ($\alpha\hm=1$, $\theta\hm=0$), распределение Леви
($\alpha\hm=1/2$, $\theta\hm=1$) и~распределение, симметричное 
к~распределению Леви ($\alpha\hm=1/2$, $\theta\hm=-1$)). Выражения
устойчивых плотностей в~терминах функций Фокса (обобщенных
$G$-функ\-ций Мейера) можно найти в~\cite{Schneider1986-k, UchaikinZolotarev1999-k}.

\section{Смешанные биномиальные распределения и~их~асимп\-то\-ти\-чес\-кое поведение}

Рассмотрим задачу, связанную с~испытаниями Бернулли со случайной
вероятностью успеха. Сначала в~результате <<предварительного>>
эксперимента определяется значение с.в.\ $\pi\hm\in(0,1)$. Это
значение принимается в~качестве вероятности успеха в~испытаниях
Бернулли. Затем случайная величина $N$ определяется как число
успехов в~$k\hm\in\mathbb{N}$ испытаниях Бернулли с~так определенной
вероятностью успеха~$\pi$. Чтобы описать бесконечную малость
вероятности успеха~$\pi$, снабдим последнюю и~(для общности)
параметр~$k$, а также, соответственно, с.в.~$N$ <<бесконечно
большим>> индексом~$n$, позволяющим проследить сходимость
последовательности с.в.\ $\pi\hm=\pi_n$ к~нулю при $n\hm\to\infty$. В~свою
очередь, бесконечная малость~$\pi_n$ означает, что успехи являются
редкими событиями в~рамках рассматриваемой последовательности
испытаний Бернулли.

В рамках схемы испытаний Бернулли со случайной вероятностью успеха,
описанной выше, можно сформулировать и~доказать <<случайный>> аналог
классической теоремы Пуассона (так называемого <<закона малых
чисел>>) для \textit{$\pi_n$-сме\-шан\-ных биномиальных распределений} со
случайной ве\-ро\-ят\-ностью успеха и~неограниченно возрастающим
це\-ло\-чис\-лен\-ным параметром~$k_n$ (<<чис\-лом испытаний>>). В~известных
вариантах <<случайного>> аналога тео\-ре\-мы Пуассона (см., 
к~примеру,~\cite{KorolevBeningShorgin2011}), наоборот, случайным считалось число
испытаний, а~вероятность успеха оставалась неслучайной.

Пусть $k_n\hm\in\mathbb{N}$, $k=1,2,\ldots$ Будем говорить, что с.в.~$Q_n$ 
имеет \textit{$\pi_n$-сме\-шан\-ное биномиальное распределение} 
с~параметром~$k_n$, если

\vspace*{-2pt} 

\noindent
\begin{multline}
 {\sf P}\left(Q_n=j\right)=C_{k_n}^j\int\limits_{0}^{1}z^k(1-z)^{k_n-j}\,
d{\sf P}\left(\pi_n<z\right)\,,\\ j=0,1,\ldots,k_n\,.
\label{e1-kk}
\end{multline}

\vspace*{-2pt}

\noindent
Для $x\in\mathbb{R}$ обозначим $B_n(x)\hm={\sf P}(Q_n\hm<x)$. Пусть $N$~---
положительная с.в. Смешанная пуассоновская ф.р.\ со структурной с.в.~$N$ 
(по терминологии, принятой в~\cite{Grandell1997}) будет
обозначаться~$\Pi^{(N)}(x)$:
$$
\Pi^{(N)}(x+0)=\sum\limits_{j=0}^{[x]}\fr{1}{j!}\int\limits_{0}^{\infty}e^{-z}z^j\,
d{\sf P}(N<z)\,,\enskip x\in\mathbb{R}\,.
$$
В~\cite{KorolevKorchaginZeifman2017} доказана следующая теорема.

\smallskip

\noindent
\textbf{Теорема~1.}\ %\cite{KorolevKorchaginZeifman2017}. 
\textit{Пусть
$(k_n)_{n\ge1}$~--- неограниченно возрас-\linebreak тающая последовательность
натуральных чисел. Пусть $Q_n$~--- с.в.\ с~$\pi_n$-сме\-шан\-ным
биномиальным распределением $(1)$ с~целочисленным параметром~$k_n$ 
и~ф.р.~$B_n(x)$. Предположим, что в}~(\ref{e1-kk}) \textit{с.в.~$\pi_n$ бесконечно
малы в~том смысле, что существует с.в.~$N$ такая, что ${\sf P}
(0\hm< N\hm<\infty)\hm=1$ и~выполнено условие $k_n\pi_n\hm\Longrightarrow N$ при
$n\hm\to\infty$. Тогда} $B_n(x)\hm\Longrightarrow \Pi^{(N)}(x)$
$(n\hm\to\infty)$.

\smallskip

Необходимо отметить, что если прореживание процесса происходит по
<<независимой>> схе-\linebreak\vspace*{-12pt}

\pagebreak

\noindent
ме, в~которой имеется двойной массив
$\{\pi_{n,j}\}_{j\ge1}$,
 $n\hm=1,2,\ldots$, независимых в~каждой серии с.в., 
причем $\pi_{n,j}\eqd\pi_n$, $j\hm=1,2\ldots$, при каждом $n\hm\ge1$, так
что $j$-я точка исходного процесса удаляется с~ве\-ро\-ят\-ностью
$1\hm-\pi_{n,j}$, то, как несложно убедиться, процесс прореживания
сводится к~классическому варианту, в~котором предельный процесс
является <<чистым>> пуассоновским.

\section{Пуассон-вейбулловское распределение}

Приведем два хорошо известных примера смешанных пуассоновских
распределений. Во-пер\-вых, это \textit{геометрическое распределение} как
дискретный\linebreak аналог непрерывного показательного распределения, который
получается, если в~смешанной пу\-ассоновской модели <<случайный>>
параметр пу\-ас\-соновского распределения имеет показательное
распреде\-ление.

Во-вторых, это \textit{отрицательное биномиальное\linebreak распределение} как
дискретный аналог непрерывного гам\-ма-рас\-пре\-де\-ле\-ния, который
получается, если в~смешанной пуассоновской модели <<случайный>>
параметр пуассоновского распределения имеет гам\-ма-рас\-пре\-де\-ле\-ние.

Пусть $W_{\gamma, \mu}$~--- с.в., имеющая распределение Вейбулла 
с~параметром масштаба $\mu\hm>0$ и~параметром формы $\gamma\hm>0$: 
${\sf P}(W_{\gamma,\mu}\hm<x)\hm=1\hm-\exp\{-\mu x^{\gamma}\}$, $x\hm\ge0$, 
и~${\sf P}(W_{\gamma,\mu}\hm<0)\hm=0$. Распределение Вейбулла играет важную роль
во многих прикладных задачах в~качестве популярной и~адекватной
модели распределения времени жизни или безотказной работы. Известен
дискретный аналог распределения Вейбулла, который получается
формальным <<квантованием>> непрерывного распределения Вейбулла:
если классический дискретный аналог с.в.~$W_{\gamma,\mu}$ 
с~распределением Вейбулла обозначить~$\widetilde W_{\gamma,\mu}$, то
обычно полагают
$$
{\sf P}\left(\widetilde W_{\gamma,\mu}=k\right)=e^{-\mu k^{\gamma}}-
e^{-\mu(k+1)^{\gamma}}\,,\enskip k=0,1,\ldots
$$
(см., например,~\cite{NakagawaOsaki1975}). У~такого формального
подхода есть существенный недостаток: крайне сложно (если вообще
возможно) сформулировать предельную теорему в~бо\-лее-ме\-нее простой
предельной схеме (например, суммирования или взятия экстремумов с.в.), 
в~которой такое распределение было бы предельным. А~стало быть,
весьма проблематичным становится теоретическое обоснование
воз\-мож\-ности использования такого распределения в~качестве
асимптотической аппроксимации.

Используя аппарат смешанных пуассоновских распределений (являющихся
предельными, например, в~теореме~1 и~сходных с~ней), можно предложить
альтернативный аналог дискретного распределения Вейбулла как
пуас\-сон-вей\-бул\-лов\-ское распределение, т.\,е.\ смешанное пуассоновское
распределение, в~котором смешивание происходит по распределению
Вейбулла.

Рассмотрим с.в.~$V_{\gamma,\mu}$, имеющую смешанное пуассоновское
распределение вида

\vspace*{-2pt}

\noindent
\begin{multline*}
{\sf P}(V_{\gamma,\mu}=k)=\fr{1}{k!}\int\limits_{0}^{\infty}e^{-z}z^kd{\sf P}
\left(W_{\gamma,\mu}<z\right)={}\\
{}=
\fr{\mu\gamma}{k!}\!\int\limits_{0}^{\infty}\!z^{k+\gamma-1}\exp\left\{-\left(z+\mu 
z^{\gamma}\right)\right\}\,dz\,,\ k=0,1,2,\ldots\hspace*{-6.64555pt}
\end{multline*}

\vspace*{-2pt}

\noindent
Такое распределение будем называть \textit{пуас\-сон-вей\-бул\-лов\-ским}.
Возможность считать пуас\-сон-вей\-буллов\-ское распределение дискретным
аналогом распределения Вейбулла обусловлена не только формальным
сходством с~отрицательным биномиальным или геометрическим
распределениями, являющимися смешанными пуассоновскими
распределениями, в~которых смешивание происходит именно по тем
непрерывным распределениям (соответственно гамма- и~показательному),
дискретными аналогами которых они являются. Дополнительным
аргументом можно считать следующее асимптотическое свойство
пуас\-сон-вей\-бул\-лов\-ско\-го распределения.

\smallskip

\noindent
\textbf{Теорема~2}.\ \textit{Справедливо следующее асимптотическое
соотношение: при} $\mu\hm\to 0$
$$
{\sf P}\left(\mu^{1/\gamma}V_{\gamma,\mu}<x\right)\Longrightarrow 
{\sf P}\left(W_{\gamma,1}<x\right)\,.
$$

\smallskip

\noindent
Д\,о\,к\,а\,з\,а\,т\,е\,л\,ь\,с\,т\,в\,о\ этого утверждения основано на том, что,
во-пер\-вых, $V_{\gamma,\mu}\eqd P_1(W_{\gamma,\mu})$, где $P_1(t)$,
$t\hm\ge0$,~--- стандартный пуассоновский процесс, независимый от 
с.в.~$W_{\gamma,\mu}$, во-вто\-рых, $W_{\gamma,\mu}\eqd \mu^{-1/\gamma}W_{\gamma,1}$ 
и,~в-треть\-их, $z^{-1}P_1(z\Lambda)\hm\Longrightarrow \Lambda$ при $z\hm\to\infty$, где
$\Lambda$~--- независимая от $P_1(t)$ неотрицательная с.в.\ (см.,
например,~\cite{KorolevBeningShorgin2011}).

\smallskip

Особый интерес представляет случай $0\hm<\gamma\hm\le1$. Распределения
Вейбулла с~такими па\-ра\-мет\-ра\-ми формы называются \textit{растянутыми
пока\-зательными}\linebreak (stretched exponential)~[11--13]. 
Они занимают промежуточное
место между распределениями с~хвос\-та\-ми, убывающими степенн$\acute{\mbox{ы}}$м
образом, и~рас-\linebreak пределениями с~экспоненциально убывающими хвос\-та\-ми,
играющими важную роль при математическом моделировании явлений 
и~процессов на финансовых рынках. Оказывается, что
пу\-ас\-сон-вей\-бул\-лов\-ское распределение с.в.~$V_{\gamma,\mu}$ 
с~$0\hm<\gamma\hm\le1$ является смешанным геометрическим, т.\,е.\ описывает
распределение числа испытаний до первого успеха в~описанной выше
схеме испытаний Бернулли со случайной вероятностью успеха, име\-ющей
специальное распределение. Для простоты без ограничения общности 
в~следующем утверждении считаем, что $\mu\hm=1$.

\smallskip

\noindent
\textbf{Теорема~ 3.}\ \textit{Пуас\-сон-вей\-бул\-лов\-ское распределение 
с.в.~$V_{\gamma,1}$ с~$0\hm<\gamma\hm\le1$ является смешанным гео\-мет\-ри\-че\-ским}:
$$
{\sf P}(V_{\gamma,1}=k)= 
\int\limits_{0}^{1}z(1-z)^kp_{\gamma,1}(z)dz,\ \ \ \ k=0,1,\ldots,
$$
\textit{где плотность $p_{\gamma,1}(z)$ имеет вид}:
$$
p_{\gamma,1}(z)=\fr{1}{(1-z)^{2}}\,g_{\gamma,1}\left(\fr{z}{1-z}\right)\,,\enskip
0\le z\le1\,.
$$

%\smallskip

\noindent
Д\,о\,к\,а\,з\,а\,т\,е\,л\,ь\,с\,т\,в\,о\,.\ 
В~работе~\cite{Korolev2016Weibull} показано,
что если $0\hm<\gamma\hm\le1$, то $W_{\gamma,1}\eqd W_{1,1}S_{\gamma,1}^{-1}$, 
где с.в.~$W_{1,1}$ и~$S_{\gamma,1}$
независимы. Тогда для $k\hm\in\{0\}\cup\mathbb{N}$
\begin{multline*}
{\sf P}\left(V_{\gamma,1}=k\right)={\sf P}
\left(P_1(W_{\gamma,1})=k\right)={}\\[1pt]
{}=\int\limits_{0}^{\infty}e^{-\lambda}
\fr{\lambda^k}{k!}\left(\int\limits_{0}^{\infty} ze^{-\lambda z}
g_{\gamma,1}(z)\,dz\right)\,d\lambda={}
\\[1pt]
{}=\fr{1}{k!}\int\limits_{0}^{\infty}
zg_{\gamma,1}(z)\left(
\int\limits_{0}^{\infty}e^{-\lambda(z+1)}\lambda^k\,d\lambda\right)\,dz={}\\[1pt]
{}=
\fr{\Gamma(k+1)}{k!}\int\limits_{0}^{\infty}\fr{zg_{\gamma,1}(z)}{(z+1)^{k+1}}\,dz={}
\\[1pt]
{}=\int\limits_{0}^{\infty}\fr{z}{z+1}\left(1-\fr{z}{z+1}\right)^kg_{\gamma,1}(z)\,dz={}\\[1pt]
{}=
\int\limits_{0}^{1}z(1-z)^k g_{\gamma,1}\left(
\fr{z}{1-z}\right)\fr{dz}{(1-z)^{2}}={}\\[1pt]
{}=\int\limits_{0}^{1}z(1-z)^kp_{\gamma,1}(z)\,dz\,,
\end{multline*}
что и~требовалось доказать.

\smallskip

Как известно, производящая функция моментов с.в.~$W_{\gamma,\mu}$
имеет вид:
\begin{multline*}
\hspace*{-1.18353pt}\Psi_{\gamma,\mu}(t)={\sf E}
\exp\left\{tW_{\gamma,\mu}\right\}=
\sum\limits_{n=0}^{\infty}\fr{t^n}{\mu^{n/\gamma}n!}\,\Gamma
\left(1+\fr{n}{\gamma}\right)\,,\\[1pt]
t\in\mathbb{R}.
\vspace*{-12pt}
\end{multline*}


\noindent
В~общем случае производящая функция $\psi_{\gamma,\mu}(s)\hm \equiv 
{\sf E}s^{V_{\gamma,\mu}}$ пуас\-сон-вей\-бул\-лов\-ско\-го распределения имеет вид:

\noindent
\begin{multline*}
\psi_{\gamma,\mu}(s)=\sum\limits_{k=0}^{\infty}s^k{\sf P}
\left(V_{\gamma,\mu}=k\right)={}\\[1pt]
{}=\sum\limits_{k=0}^{\infty}
\fr{s^k}{k!}\int\limits_{0}^{\infty}e^{-z}z^k\,d{\sf P}
\left(W_{\gamma,\mu}<z\right)={}\\[1pt]
{}=\int\limits_{0}^{\infty}
e^{-z}\left[\sum\limits_{k=0}^{\infty}\fr{(sz)^k}{k!}\right]\,d{\sf P}
\left(W_{\gamma,\mu}<z\right)= {}\\[1pt]
{}=\int\limits_{0}^{\infty}e^{z(s-1)}\,d{\sf P}(W_{\gamma,\mu}<z)={\sf E}
\exp\left\{(s-1)W_{\gamma,\mu}\right\}={}\\[1pt]
{}=
\Psi_{\gamma,\mu}(s-1)=\sum\limits_{n=0}^{\infty}\fr{(s-1)^n}{\mu^{n/\gamma}n!}\,
\Gamma\left(1+\fr{n}{\gamma}\right)\,,\\[1pt]
 s\in[0,1]\,.
\end{multline*}
Отсюда имеем:
\begin{multline*}
{\sf P}\left(V_{\gamma,\mu}=0\right)=
\psi_{\gamma,\mu}(s)|_{s=0}=\Psi_{\gamma,\mu}(-1)={}\\[1pt]
{}=
\sum\limits_{n=0}^{\infty}\fr{\Gamma(1+{n}/{\gamma})(-1)^n}{\mu^{n/\gamma}n!}\,;
\end{multline*}

\vspace*{-12pt}

\begin{multline*}
{\sf P}\left(V_{\gamma,\mu}=1\right)=
\fr{d\psi_{\gamma,\mu}(s)}{ds}\Big|_{s=0}={}\\[1pt]
{}=
\sum\limits_{n=0}^{\infty}\fr{\Gamma(1+{n}/{\gamma})}{\mu^{n/\gamma}n!}\,
\fr{d}{ds}\left(s-1\right)^n\Big|_{s=0}={}\\[1pt]
{}=
\sum\limits_{n=1}^{\infty}\fr{\Gamma(1+{n}/{\gamma})(-1)^{n-1}}{\mu^{n/\gamma}(n-1)!}\,;
\end{multline*}

\vspace*{-12pt}

\begin{multline*}
{\sf P}\left(V_{\gamma,\mu}=2\right)=
\fr{d^2\psi_{\gamma,\mu}(s)}{2ds^2}\Big|_{s=0}={}\\[1pt]
{}=
\fr{1}{2}\sum\limits_{n=0}^{\infty}\fr{\Gamma(1+{n}/{\gamma})}
{\mu^{n/\gamma}n!}\,\fr{d^2}{ds^2}\left(s-1\right)^n\Big|_{s=0}={}\\[1pt]
{}=
\fr{1}{2}\sum\limits_{n=2}^{\infty}\fr{\Gamma(1+{n}/{\gamma})(-1)^{n-2}}
{\mu^{n/\gamma}(n-2)!};\ldots;
\end{multline*}

\vspace*{-12pt}

\begin{multline*}
{\sf P}(V_{\gamma,\mu}=k)=\fr{d^k\psi_{\gamma,\mu}(s)}{k!ds^k}\Big|_{s=0}={}\\[1pt]
{}=
\fr{1}{k!}\sum\limits_{n=k}^{\infty}\fr{\Gamma(1+{n}/{\gamma})(-1)^{n-k}}
{\mu^{n/\gamma}(n-k)!}\,,\enskip k=3,4,\ldots
\end{multline*}
Как видно, это распределение весьма громоздко и~<<прямое>>
оценивание его параметров представляет нетривиальную задачу.

\section{Двухэтапный сеточный ЕМ-алгоритм для~оценивания параметров 
смешанных пуассоновских
распределений и,~в~частности, параметров пуассон-вейбулловского
рас\-пре\-де\-ле\-ния}

По сути оценивание параметров смешанных пуассоновских моделей
сводится к~оцениванию смешивающего распределения. Традиционно с~этой
целью используется классический ЕМ (expectation-maximization) ал\-го\-ритм~\cite{Karlis2005}.
Однако иногда, в~част\-ности в~пуас\-сон-вей\-бул\-лов\-ском \mbox{случае},
классический ЕМ-ал\-го\-ритм оказывается менее эффективным, чем
альтернативный двухэтапный сеточный ЕМ-ал\-го\-ритм оценивания
параметров смешанных пуассоновских распределений.

Следует отметить, что сеточные методы разделения смесей довольно
эффективны не только при оценивании параметров смешанных
пуассоновских распределений, но и~при разделении конечных или
произвольных дис\-пер\-си\-он\-но-сдви\-го\-вых смесей нормальных 
законов~\cite{KorolevKorchagin2014}.

Рассмотрим следующий двухэтапный метод разделения смешанных
пуассоновских распределений на примере оценивания параметров~$\gamma$, 
$\mu$ пу\-ас\-сон-вей\-бул\-лов\-ско\-го распределения
$\Pi^{(N)}(x)\hm=\Pi^{(W_{\gamma,\mu})}(x)$.

На первом этапе на положительной полупрямой выделим основную часть
носителя смешивающего распределения, т.\,е.\ ограниченный интервал,
вероятность которого, вычисленная в~соответствии со смешивающим
распределением, практически равна единице. На этот интервал накинем
конечную сетку, содержащую (возможно, очень большое чис\-ло)
$K\hm\in\mathbb{N}$ \textit{известных} узлов $\lambda_1,\ldots,\lambda_K$.
Приблизим искомое смешанное пуассоновское распределение конечной
смесью пуассоновских законов:
\begin{equation}
\Pi^{(W_{\gamma,\mu})}(x+0)\approx\sum\limits_{j=0}^{[x]}
\fr{1}{j!}\sum\limits_{i=1}^K p_i e^{-\lambda_i}\lambda_i^j\,,\enskip
x\in\mathbb{R}\,.
\label{e2-kk}
\end{equation}
В смеси, стоящей в~правой части соотношения~(\ref{e2-kk}), неизвестными
являются только па\-ра\-мет\-ры $p_1,\ldots,p_{K}$. Пусть $x_1,\ldots,x_n$~---
анализируемая выборка значений случайной величины с~оце\-ни\-ва\-емым
смешанным пуассоновским распределением. Итерационный процесс,
определяющий сеточный ЕМ-ал\-го\-ритм для данной задачи, задается
следующим образом. Пусть $p_1^{(m)},\ldots,p_{K-1}^{(m)}$~--- оценки
па\-ра\-мет\-ров $p_1,\ldots,p_{K-1}$ на $m$-й итерации,
$p_K^{(m)}\hm=1\hm-p_1^{(m)}-\cdots -p_{K-1}^{(m)}$. Для $i\hm=1,\ldots,K$,
$j\hm=1,\ldots,n$ обозначим $\phi_{ij}\hm=e^{-\lambda_i+x_j\ln \lambda_i}$.
Тогда, используя стандартные рассуждения, определяющие
вычислительные формулы EM-ал\-го\-рит\-ма для параметров конечной смеси
вероятностных распределений (см, например,~[17,
разд.~5.3.7--5.3.8]), следует положить:

\vspace*{-2pt}

\noindent
\begin{multline}
p_i^{(m+1)}=\fr{p_i^{(m)}}{n}\sum\limits_{j=1}^n
\fr{\phi_{ij}}{\sum\nolimits_{r=1}^Kp_r^{(m)}\phi_{rj}}\,,
\\
 i=1,\ldots,K\,.
\label{e3-kk}
\end{multline}
Как видим, итерационный процесс, задаваемый соотношением~(\ref{e3-kk}), очень
прост. В~силу монотонности классического ЕМ-алгоритма справедливо
сле\-ду\-ющее утверждение.

\smallskip

\noindent
\textbf{Теорема~4.}\ \textit{Пусть узлы $\lambda_1,\ldots,\lambda_K$ сетки
различны, неотрицательны и~известны. Тогда итерационный процесс}~(\ref{e3-kk}) 
\textit{является монотонным, т.\,е.\ каждая его итерация не уменьшает
целевую сеточную функцию правдоподобия}


\noindent
$$
L\left(p_1,\ldots,p_K;x_1,\ldots,x_n\right)=\prod\limits_{j=1}^n\left[
\sum\limits_{i=1}^K
p_i\phi_{ij}\right]\,.
$$

%\smallskip

Заметим, что, как показано в~[17, разд.~5.7.4],
сеточная функция правдоподобия $L(p_1,\ldots,p_{K};\,x_1,\ldots,x_n)$
вогнута по аргументам $p_1,\ldots,p_{K}$. Поэтому на каждом шаге
итерационного процесса вместо соотношения~(\ref{e3-kk}) можно использовать
любой более быстрый алгоритм максимизации функции
$L(p_1,\ldots,p_{K};\,x_1,\ldots,x_n)$ по переменным $p_1,\ldots,p_{K}$.
Например, оценки весов $p_1,\ldots,p_K$ можно искать методом условного
градиента~\cite{Korolev2011, KorolevNazarov2010}.

\smallskip

Таким образом, на первом этапе получаются оценки весов~$p_i$ всех
узлов~$\lambda_i$, $i\hm=1,\ldots,K$, конечной сетки, накинутой на
носитель смешивающего распределения.

На втором этапе остается применить ка\-кой-ли\-бо стандартный метод
подгонки распределения Вейбулла ${\sf P}(W_{\gamma,\mu}\hm<x)$ 
к~эмпирическим данным типа гистограммы $(\lambda_1, p_1),\ldots,
(\lambda_K, p_K)$, полученным на первом этапе. Например, параметры~$\gamma$ 
и~$\mu$ можно оценить, минимизируя соответствующую
статистику хи-квад\-рат. Или же, например, можно решить задачу
наименьших квадратов:

\vspace*{-2pt}

\noindent
\begin{multline*}
(\gamma^*,\mu^*)={}\\
{}=\arg\min_{\gamma,\mu}\sum_{i=1}^K\left[p_i-
\exp\left\{-\fr{\mu}{2}\left(\lambda_{i-1}+\lambda_i\right)^{\gamma}\right\}+{}\right.\\
\left.{}+
\exp\left\{-\fr{\mu}{2}\left(\lambda_i+\lambda_{i+1}\right)^{\gamma}\right\}\right]^2\,,
\end{multline*}
где $\lambda_0\hm=0$, $\lambda_{K+1}\hm=\infty$.

\pagebreak

На практике хорошие результаты показал подход с~решением задачи
наименьших квадратов. Для поиска параметров использовался алгоритм
{\sf ns2sol}, описанный в~книге~\cite{DSch1983}. Указанный алгоритм
доступен во многих статистических пакетах, отличается высоким
быстродействием и~возможностью при желании задавать разумные
интервалы для поиска параметров.

Также хорошие результаты показал метод поиска наилучшего
распределения в~смысле минимизации расстояния Куль\-ба\-ка--Лейб\-ле\-ра,
который в~данном случае эквивалентен максимизации правдоподобия
полученной гистограммы в~классе распределений Вейбулла.

При фиксированной сетке двухэтапный метод дает лишь приближенные
оценки параметров смешанных пуассоновских распределений, причем
точность приближения зависит от успешного выбора сетки. Некоторые
аспекты этого выбора будут рассмотрены в~следующем разделе. Говорить
о~состоятельности получаемых оценок при фиксированной сетке нельзя.
Но если объем выборки неограниченно возрастает и~вместе с~ним
согласованно увеличивается число узлов, то вопрос о~состоятельности
получаемых оценок приобретает смысл и~будет рассмотрен в~одной из
следующих публикаций.

\section{О~практическом выборе сетки на~первом этапе двухэтапного
сеточного ЕМ-алгоритма для~разделения смесей пуас\-со\-нов\-ских
распределений}

Естественно, что при использовании указанного двухэтапного метода 
в~динамическом режиме крайне важным становится вопрос о~выборе
наиболее эффективных и~быстродействующих численных процедур и~их
параметров. В~част\-ности, исключительную важность приобретает
правильный выбор границ сетки на первом этапе. Рас\-смот\-рим этот
вопрос подробнее.

Формально рассматриваемая задача выглядит так: по наблюдаемым
значениям $x_1,\ldots,x_n$ требуется построить статистическую оценку
верхней границы квантилей заданного порядка сме\-ши\-ва\-юще\-го закона так,
чтобы как можно точнее оценить носитель смешивающего распределения.

В дальнейшем будем считать, что $x_1,\ldots,x_n$~--- независимые
реализации с.в.\  $X\hm\eqd P_1(\Lambda)$, где $P_1(t)$~--- стандартный
пуассоновский процесс, $t\hm\ge0$, $\Lambda$~--- независимая от~$P_1(t)$
неотрицательная с.в. В~случае пу\-ас\-сон-вей\-бул\-лов\-ско\-го распределения
$X\hm\eqd V_{\gamma,\mu}$, $\Lambda\hm\eqd W_{\gamma,\mu}$.

Сначала рассмотрим более общий случай и~предположим, что ${\sf E}
\Lambda^2\hm<\infty$. Тогда ${\sf E}X\hm={\sf E}\Lambda$, ${\sf D}X\hm=
{\sf E}\Lambda\hm+{\sf D}\Lambda\hm={\sf E}X\hm+{\sf D}\Lambda$. Следовательно,
${\sf D}\Lambda\hm={\sf D}X\hm-{\sf E}X$. Но ${\sf D}\Lambda\hm={\sf E}
\Lambda^2\hm-({\sf E}\Lambda)^2\hm={\sf E}\Lambda^2\hm-({\sf E}X)^2$.
Поэтому ${\sf E}\Lambda^2\hm={\sf D}\Lambda\hm+({\sf E}X)^2\hm={\sf D}X\hm-
{\sf E}X\hm+({\sf E}X)^2\hm={\sf E}X^2\hm-{\sf E}X$. Таким образом, для
$\lambda\hm>0$ по неравенству Маркова имеем:
\begin{equation}
{\sf P}(\Lambda\ge\lambda)\le\frac{{\sf
E}\Lambda^2}{\lambda^2}=\frac{{\sf E}X^2-{\sf
E}X}{\lambda^2}.\label{e4-kk}
\end{equation}
Отсюда, задав произвольно малое $\varepsilon\hm>0$, можно \mbox{найти}
приближенную верхнюю оценку 
$(1\hm-\varepsilon)$-кван\-ти\-ли~$\lambda^{(1-\varepsilon)}$ 
с.в.~$\Lambda$. С~этой целью положим:
$$
\lambda_{\varepsilon}=\sqrt{\fr{{\sf E}X^2-{\sf E}X}{\varepsilon}}\,.
$$

\begin{figure*}[b] %fig1
\vspace*{6pt}
\begin{center}
\mbox{%
\epsfxsize=163.367mm
\epsfbox{kor-1.eps}
}
\end{center}
\vspace*{-9pt}
  \Caption{Графики <<истинной>> смешивающей плотности с~$\mu\hm=1$ 
  и~$\gamma\hm=1/2$ (\textit{1}) и~ее оценки, полученной
двухэтапным сеточным методом~(\textit{2})~(\textit{а}) и~графики
соответствующего <<истинного>> пу\-ас\-сон-вей\-бул\-лов\-ско\-го 
распределения~(\textit{3}) и~его статистической оценки~(\textit{4})~(\textit{б}). Объем выборки $n\hm=500$}
%\end{figure*}
%\begin{figure*} %fig2
\vspace*{12pt}
\begin{center}
\mbox{%
\epsfxsize=162.372mm
\epsfbox{kor-2.eps}
}
\end{center}
\vspace*{-9pt}
   \Caption{Графики <<истинной>> смешивающей плотности с~$\mu\hm=2$ 
   и~$\gamma\hm=1/2$ (\textit{1}) и~ее оценки, полученной
двухэтапным сеточным методом~(\textit{2})~(\textit{а}) и~графики
соответствующего <<истинного>> пу\-ас\-сон-вей\-бул\-лов\-ско\-го 
распределения~(\textit{3}) и~его статистической оценки~(\textit{4})~(\textit{б}). Объем выборки
$n\hm=500$}
\end{figure*}


\noindent
Тогда из~(\ref{e4-kk}) вытекает, что ${\sf P}
(\Lambda\hm\ge\lambda_{\varepsilon})\hm\le\varepsilon$, т.\,е.\ можно
положить $\lambda_K\hm=\lambda^{(1-\varepsilon)}$, причем
$$
\lambda^{(1-\varepsilon)}\le\lambda_{\varepsilon}=\sqrt{\fr{{\sf E}
X^2-{\sf E}X}
{\varepsilon}}\approx\sqrt{\fr{1}{n\varepsilon}\sum\limits_{j=1}^nx_j(x_j-1)}\,.
$$

Теперь отдельно рассмотрим случай, когда ${\sf E}e^X\hm<\infty$. Этот
случай, в~част\-ности, имеет место для пу\-ас\-сон-вей\-бул\-лов\-ско\-го
распределения $X\hm\eqd V_{\gamma,\mu}$, $\Lambda\hm\eqd W_{\gamma,\mu}$,
если $\gamma\hm>1$ или $\gamma\hm=1$ и~$\mu\hm>e\hm-1$. Имеем
\begin{multline*}
{\sf E}e^X=\sum\limits_{k=0}^{\infty}\fr{e^k}{k!}
\int\limits_{0}^{\infty}e^{-\lambda}\lambda^kd{\sf P}
(\Lambda<\lambda)={}\\
{}=\int\limits_{0}^{\infty}e^{-\lambda}
\left(\sum\limits_{k=0}^{\infty}\fr{(e\lambda)^k}{k!}\right)\,d{\sf P}
(\Lambda<\lambda)={}
\\
{}=
\int\limits_{0}^{\infty}e^{\lambda(e-1)}\,d{\sf P}(\Lambda<\lambda)=
{\sf E}e^{\Lambda(e-1)}\,.
\end{multline*}
Тогда
\begin{multline}
{\sf P}(\Lambda\ge\lambda)={\sf P}
\left(\Lambda(e-1)\ge\lambda(e-1)\right)\le{}\\
{}\le \fr{{\sf E}
e^{\Lambda(e-1)}}{e^{\lambda(e-1)}}=\fr{{\sf E} e^X}{e^{\lambda(e-1)}}\,.
\label{e5-kk}
\end{multline}
Для произвольно малого положительного~$\varepsilon$ положим:
$$
\lambda_{\varepsilon}=\fr{\ln{\sf E}e^X-\ln\varepsilon}{e-1}\,.
$$
Тогда из~(\ref{e5-kk}) вытекает, что ${\sf P}
(\Lambda\hm\ge\lambda_{\varepsilon})\hm\le\varepsilon$, т.\,е.\ можно
положить $\lambda_K\hm=\lambda^{(1-\varepsilon)}$, причем
\begin{multline*}
\lambda_K=\lambda^{(1-\varepsilon)}\le\lambda_{\varepsilon}=\fr{\ln{\sf E}
e^X-\ln\varepsilon}{e-1}\approx{}\\
{}\approx \fr{1}{e-1}\ln\left( 
\fr{1}{n\varepsilon}\sum\limits_{j=1}^ne^{x_j}\right)\,.
\end{multline*}

\vspace*{-9pt}

\section{Примеры}

\vspace*{-2pt}

В этом разделе приведены результаты тестовых вычислений по
описанному выше алгоритму (рис.~1--4). Моделировались искусственные выборки из
пу\-ас\-сон-вей\-бул\-лов\-ско\-го распределения, к~которым применялся описанный
выше двухэтап-\linebreak\vspace*{-10pt}

\columnbreak

\noindent
ный метод. Необходимо отметить, что во всех случаях
размер сетки был относительно небольшим ($K\hm=15$), тем не менее
достигнута приемлемая точность. В~каждой паре рисунков слева~---
графики <<истинной>> смешивающей плотности и~ее оценки, полученной
сеточным методом, справа~--- графики соответствующего <<истинного>>
пуас\-сон-вей\-бул\-лов\-ско\-го распределения и~его статистической оценки.

%\bigskip

Авторы выражают благодарность И.\,Г.~Шевцовой и~А.\,В.~Дорофеевой 
за участие в~работе по тес\-ти\-ро\-ва\-нию алгоритма, проведению вычислений\linebreak
 и~построению 
графиков, проведенную в~рамках выполнения проекта №\,14-11-00397 
Российского научного фонда.

\pagebreak




\end{multicols}

\begin{figure*} %fig3
\vspace*{1pt}
\begin{center}
\mbox{%
\epsfxsize=163.372mm
\epsfbox{kor-3.eps}
}
\end{center}
\vspace*{-9pt}
    \Caption{Графики <<истинной>> смешивающей плотности 
    с~$\mu\hm=0{,}2$ и~$\gamma\hm=2$ (\textit{1}) и~ее оценки, полученной
двухэтапным сеточным методом~(\textit{2}) и~графики
соответствующего <<истинного>> пу\-ас\-сон-вей\-бул\-лов\-ско\-го распределения
(\textit{3}) и~его статистической оценки~(\textit{4})~(\textit{б}). Объем выборки
$n\hm=1000$}
%\end{figure*}
%\begin{figure*} %fig4
\vspace*{12pt}
\begin{center}
\mbox{%
\epsfxsize=163.472mm
\epsfbox{kor-4.eps}
}
\end{center}
\vspace*{-9pt}
    \Caption{Графики <<истинной>> смешивающей плотности 
    с~$\mu=2$ и~$\gamma=1/2$~(\textit{1}) и~ее оценки, полученной
двухэтапным сеточным методом~(\textit{2}) и~графики
соответствующего <<истинного>> пу\-ас\-сон-вей\-бул\-лов\-ско\-го 
распределения~(\textit{3}) и~его статистической оценки~(\textit{4})~(\textit{б}). Объем выборки
$n\hm=1000$}
\end{figure*}

\begin{multicols}{2}


{\small\frenchspacing
 {%\baselineskip=10.8pt
 \addcontentsline{toc}{section}{References}
 \begin{thebibliography}{99}
    
    \bibitem{GrigoryevaKorolevSokolov2013}
\Au{Григорьева М.\,Е., Королев В.\,Ю., Соколов~И.\,А.} Предель\-ная теорема
    для геометрических сумм неза\-ви\-симых неодинаково распределенных
    случайных ве\-личин и~ее применение к~прогнозированию вероятности
    катастроф в~неоднородных потоках экстремальных событий~//
    Информатика и~её применения, 2013. Т.~7. Вып.~4. С.~11--19.
    
    \bibitem{Renyi1956-k}
\Au{R$\acute{\mbox{e}}$nyi A.} A~Poisson-folyamat egy jellemzese~// Maguar Tud. Acad.
    Mat. Int. Kozl., 1956. Vol.~1. P.~519--527.
    
    \bibitem{Mogyorodi1971}
\Au{Mogyorodi J.} Some notes on thinning recurrent flows~// 
Litovsky Math. Sbornik, 1971. Vol.~11. P.~303--315.
    
    \bibitem{Zolotarev1983-k}
\Au{Золотарев В.\,М.} Одномерные устойчивые распределения.~--- М.: Наука, 1983. 304~с.
    
    \bibitem{Schneider1986-k}
\Au{Schneider W.\,R.} Stable distributions: Fox
    function representationand generalization~// 
    Stochastic processes in classical and quantum systems~/ 
    Eds. S.~Albeverio, G.~Casati, D.~Merlini.~--- Berlin: Springer, 1986. P.~497--511.
    
    \bibitem{UchaikinZolotarev1999-k} 
    \Au{Uchaikin V.\,V., Zolotarev~V.\,M.}
    Chance and stability.~--- Utrecht: VSP, 1999. 570~p.
    
    \bibitem{KorolevBeningShorgin2011} %%%% оставить
\Au{Королев В.\,Ю., Бенинг~В.\,Е., Шоргин~С.\,Я.} Математические
        основы теории риска.~--- 2-е изд.~--- М.: Физматлит, 2011. 591~с.
    
    \bibitem{Grandell1997} %%%% оставить
    \Au{Grandell J.} Mixed Poisson processes.~--- London: Chapman and Hall,
    1997. 268~p.
    
    \bibitem{KorolevKorchaginZeifman2017} %%%%% оставить
\Au{Korolev V.\,Yu., Korchagin~A.\,Yu., Zeifman~A.\,I.} On doubly
    stochastic rarefaction of renewal processes~// 14th 
     Conference (International) of Numerical Analysis and Applied
    Mathematics Proceedings.~---
    American Institute of Physics Proceedings, 2017 (in press).
    
    \bibitem{NakagawaOsaki1975} %%%% оставить
    \Au{Nakagawa T., Osaki~Sh.} The discrete Weibull distribution~// IEEE
    Trans. Reliab., 1975. Vol.~24. P.~300--301.
    
    \bibitem{LaherrereSornette1998}
    \Au{\mbox{Laherr{\!\!\ptb{\`{e}}}re}~J., Sornette D.}
    Stretched exponential distributions in nature and economy: ``Fat
    tails'' with characteristic scales~// Eur. Phys.~J.~B,
    1998. Vol.~2. P.~525--539.
    
    \bibitem{Sornette_et_al2005} 
    \Au{Malevergne Y., Pisarenko~V., Sornette~D.}
    Empirical distributions of stock returns: Between the
    stretched exponential and the power law?~// Quant. Financ.,
    2005. Vol.~5. P.~379--401.
    
    \bibitem{Sornette_et_al2006} 
    \Au{Malevergne Y., Pisarenko~V., Sornette~D.}
    On the power of generalized extreme value (GEV) and
    generalized Pareto distribution (GDP) estimators for empirical
    distributions of stock returns~// Appl. Financ. Econ., 2006.
    Vol.~16. P.~271--289.
    
    \bibitem{Korolev2016Weibull}
\Au{Korolev V.\,Yu.} Product representations for random variables with
    the Weibull distributions and their applications~// J.~Math. Sci., 
    2016. Vol.~218. No.\,3. P.~298--313.
    
    \bibitem{Karlis2005} %%%% оставить
\Au{Karlis D.} An EM algorithm for mixed Poisson distributions~// ASTIN
    Bull., 2005. Vol.~35. P.~3--24.
    
    \bibitem{KorolevKorchagin2014} %%%%% оставить
    \Au{Королев В.\,Ю., Корчагин~А.\,Ю.} Модифицированный сеточный метод
    разделения дис\-пер\-си\-он\-но-сдви\-го\-вых смесей нормальных законов~//
    Информатика и~её применения, 2014. Т.~8. Вып.~4. С.~11--19.
    
    \bibitem{Korolev2011} %%%% оставить
\Au{Королев В.\,Ю.} Вероятностно-статистические методы декомпозиции
    волатильности хаотических процессов.~--- М.: Изд-во Московского
    ун-та, 2011. 510~с.
    
    \bibitem{KorolevNazarov2010} %%%% оставить
\Au{Королев В.\,Ю., Назаров~А.\,Л.} Разделение смесей вероятностных
    распределений при помощи сеточных методов моментов и~максимального
    правдоподобия~// Автоматика и~телемеханика, 2010. Вып.~3. С.~98--116.
    
    \bibitem{DSch1983} %%%% оставить
\Au{Dennis J.\,E., Schnabel~R.\,B.} Numerical methods for unconstrained
    optimization and nonlinear equations.~--- Englewood Cliffs:
    Prentice-Hall, 1983. 375~p.
 \end{thebibliography}

 }
 }

\end{multicols}

\vspace*{-3pt}

\hfill{\small\textit{Поступила в~редакцию 15.10.16}}

\vspace*{8pt}

%\newpage

%\vspace*{-24pt}

\hrule

\vspace*{2pt}

\hrule

%\vspace*{8pt}


\def\tit{THE POISSON THEOREM FOR BERNOULLI TRIALS WITH~A~RANDOM
PROBABILITY OF~SUCCESS\\ AND~A~DISCRETE ANALOG OF~THE~WEIBULL
DISTRIBUTION}

\def\titkol{The Poisson theorem for Bernoulli trials with a random
probability of success and~a~discrete analog of the Weibull
distribution}

\def\aut{V.\,Yu.~Korolev$^{1,2}$, A.\,Yu.~Korchagin$^{1,2}$, and~A.\,I.~Zeifman$^{2,3,4}$}

\def\autkol{V.\,Yu.~Korolev, A.\,Yu.~Korchagin, and~A.\,I.~Zeifman}

\titel{\tit}{\aut}{\autkol}{\titkol}

\vspace*{-9pt}


    
\noindent
  
\noindent
$^1$Faculty of Computational Mathematics and Cybernetics, 
M.\,V.~Lomonosov Moscow State University, 
1-52~Lenin-\linebreak
$\hphantom{^1}$skiye Gory, GSP-1, Moscow 119991, Russian Federation

\noindent
$^2$Institute of Informatics Problems, Federal Research Center 
``Computer Science and Control'' of the Russian\linebreak
$\hphantom{^1}$Academy of Sciences, 44-2~Vavilov Str., 
Moscow 119333,  Russian Federation

\noindent
$^3$Vologda State University, 15~Lenin Str., Vologda 160000, Russian Federation

\noindent
$^4$ISEDT RAS, 56-A~Gorky Str., Vologda 16001, Russian Federation



\def\leftfootline{\small{\textbf{\thepage}
\hfill INFORMATIKA I EE PRIMENENIYA~--- INFORMATICS AND
APPLICATIONS\ \ \ 2016\ \ \ volume~10\ \ \ issue\ 4}
}%
 \def\rightfootline{\small{INFORMATIKA I EE PRIMENENIYA~---
INFORMATICS AND APPLICATIONS\ \ \ 2016\ \ \ volume~10\ \ \ issue\ 4
\hfill \textbf{\thepage}}}

\vspace*{3pt}    


\Abste{A problem related to the Bernoulli trials with 
a~random probability of success is considered. First, as a~result of
the preliminary experiment, the value of the random variable
$\pi\in(0,1)$ is determined that is taken as the probability of
success in the Bernoulli trials. Then, the random variable $N$ is
determined as the number of successes in $k\in\mathbb{N}$ Bernoulli
trials with the so determined success probability~$\pi$. The
distribution of the random variable~$N$ is called $\pi$-mixed
binomial. Within the framework of these Bernoulli trials with the
random probability of success, a~``random'' analog of the classical
Poisson theorem is formulated for the $\pi$-mixed binomial distributions, in which
the limit distribution turns out to be the mixed Poisson distribution. Special
attention is paid to the case where mixing is performed with
respect to the Weibull distribution. The corresponding mixed Poisson
distribution called Poisson--Weibull law is proposed as a~discrete
analog of the Weibull distribution. Some properties of the
Poisson--Weibull distribution are discussed. In particular, it is
shown that this distribution can be represented as the mixed geometric
distribution. A~two-stage grid algorithm is proposed for 
estimation of parameters of mixed Poisson distributions and, in
particular, of the Poisson--Weibull distribution. Statistical estimators
for the upper bound of the grid are constructed. The examples of
practical computations performed by the proposed algorithm are presented.}


\KWE{Bernoulli trials with a random probability of
success; mixed binomial distribution; Poisson theorem; mixed Poisson
distribution; Weibull distribution; Poisson--Weibull distribution;
mixed geometric distribution; EM-algorithm}

\DOI{10.14357/19922264160402} 

%\vspace*{-9pt}

\Ack
    \noindent
This work was financially supported by the Russian Science Foundation 
(grant No.\ 14-11-00397).



%\vspace*{3pt}

  \begin{multicols}{2}

\renewcommand{\bibname}{\protect\rmfamily References}
%\renewcommand{\bibname}{\large\protect\rm References}

{\small\frenchspacing
 {%\baselineskip=10.8pt
 \addcontentsline{toc}{section}{References}
 \begin{thebibliography}{99}

    \bibitem{GrigoryevaKorolevSokolov2013_eng} %%%% оставить
    \Aue{Grigoryeva, M.\,E., V.\,Yu.~Korolev, and I.\,A.~Sokolov}. 2013.
    Predel'naya teorema dlya geometricheskikh summ ne\-za\-vi\-si\-mykh neodinakovo
    raspredelennykh sluchaynykh velichin i~ee primenenie k~progrozirovaniyu
    veroyatnosti ka\-tast\-rof v~neodnorodnykh potokakh ekstremal'nykh sobytiy
    [A~limit theorem for geometric sums of independent nonidentically
    distributed random variables and its application to the prediction of the probabilities of
    catastrophes in
    nonhomogeneous flows of extremal events].
    \textit{Informatika i~ee Primeneniya~--- Inform. Appl.} 7(4):11--19.
    
    \bibitem{Renyi1956_eng-k} %%%%% оставить
    \Aue{R$\acute{\mbox{e}}$nyi,~A.} 1956.
    A~Poisson-folyamat egy jellemzese. \textit{Maguar Tud. Acad. Mat. Int. Kozl.} 1:519--527.
    
    \bibitem{Mogyorodi1971_eng} %%%%% оставить
    \Aue{Mogyorodi, J.} 1971.
    Some notes on thinning recurrent flows. \textit{Litovsky Math. Sbornik} 11:303--315.
    
    \bibitem{Zolotarev1983_eng-k} 
    \Aue{Zolotarev, V.\,M.} 1983.
    \textit{Odnomernye ustoychivye raspredeleniya} 
    [One-dimensional stable distributions]. Moscow: Nauka. 304~p.
    
    
    \bibitem{Schneider1986_eng-k} 
    \Aue{Schneider, W.\,R.} 1986.
    Stable distributions: Fox function representationand generalization. 
    \textit{Stochastic processes in classical and quantum systems}. 
    Eds.\ S.~Albeverio, G.~Casati, and D.~Merlini. Berlin: Springer. 497--511.
    
    \bibitem{UchaikinZolotarev1999_eng-k} 
    \Aue{Uchaikin, V.\,V., and V.\,M.~Zolotarev}. 1999.
    \textit{Chance and stability.} Utrecht: VSP. 570~p.
    
    \bibitem{KorolevBeningShorgin2011_eng} %%%% оставить
\Aue{Korolev, V.\,Yu., V.\,E.~Bening, and S.\,Ya.~Shorgin}. 2011.
    \textit{Matematicheskie osnovy teorii riska} 
    [Mathematical fundamentals of risk theory]. 2nd ed. Moscow: Fizmatlit. 591~p.
    
    \bibitem{Grandell1997_eng} %%%% оставить
\Aue{Grandell, J.} 1997.
    \textit{Mixed Poisson processes.} London: Chapman and Hall. 268~p.
    
    \bibitem{KorolevKorchaginZeifman2017_eng} %%%%% оставить
\Aue{Korolev, V.\,Yu., A.\,Yu.~Korchagin, and A.\,I.~Zeifman}. 2017 (in press).
    On doubly stochastic rarefaction of renewal processes. \textit{14th 
     Conference (International) of Numerical Analysis and Applied
        Mathematics Proceedings}.
    American Institute of Physics Proceedings. 
    
    \bibitem{NakagawaOsaki1975_eng} %%%% оставить
    \Aue{Nakagawa, T., and Sh.~Osaki}. 1975.
    The discrete Weibull distribution. \textit{IEEE Trans. Reliab.} 24:300--301.
    
    \bibitem{LaherrereSornette1998_eng}
    \Aue{\mbox{Laherr{\!\ptb{\`{e}}}re}, J., and D.~Sornette}. 1998.
    Stretched exponential distributions in nature and economy: ``Fat
    tails'' with characteristic scales. \textit{Eur. Phys. J.~B} 2:525--539.
    
    \bibitem{Sornette_et_al2005_eng}
    \Aue{Malevergne, Y., V.~Pisarenko, and D.~Sornette}. 2005.
    Empirical distributions of stock returns: Between the
    stretched exponential and the power law? \textit{Quant. Financ.} 5:379--401.
    
    \bibitem{Sornette_et_al2006_eng}
    \Aue{Malevergne, Y., V.~Pisarenko, and D.~Sornette}. 2006.
    On the power of generalized extreme value (GEV) and
    generalized Pareto distribution (GDP) estimators for empirical
    distributions of stock returns. \textit{Appl. Financ. Econ.} 16:271--289.
    
    \bibitem{Korolev2016Weibull_eng} 
    \Aue{Korolev, V.\,Yu.} 2016.
    Product representations for random variables with the Weibull distributions 
    and their applications. \textit{J.~Math. Sci.} 218(3):298--313.
    
    \bibitem{Karlis2005_eng} %%%% оставить
    \Aue{Karlis, D.} 2005.
    An EM algorithm for mixed Poisson distributions. 
    \textit{ASTIN Bull.} 35:3--24.
    
    \bibitem{KorolevKorchagin2014_eng} %%%%% оставить
    \Aue{Korolev, V.\,Yu., and A.\,Yu.~Korchagin}. 2014.
    Modi\-fi\-tsi\-ro\-van\-nyy setochnyy metod razdeleniya dispersionno-sdvigovykh smesey
    normal'nykh zakonov [Modified grid method for decomposition of 
    mean-variance normal mixtures].
    \textit{Informatika i~ee Primeneniya--- Inform. Appl.} 8(4):11--19.
    
    \bibitem{Korolev2011_eng} %%%% оставить
    \Aue{Korolev, V.\,Yu.} 2011.
    \textit{Veroyatnostno-statisticheskie metody dekompozitsii volatil'nosti
        khaoticheskikh protsessov} [Probablity-based method for volatility
        decomposition of chaotic processes]. Moscow: Moscow University Press. 510~p.
    
    \bibitem{KorolevNazarov2010_eng} %%%% оставить
   \Aue{Korolev, V.\,Yu., and A.\,L.~Nazarov}. 2010.
    %Razdeleniye smesei veroyatnostnih raspredeleniy pri pomoshi setochnyh
%    metodov momentov i~maksimalnogo pravdopodobiya [
Separating mixtures of probability distributions with the 
grid maximum likelihood method].
    \textit{Avtomat. Rem. Contr.} 71(3):455--472.
    
    \bibitem{DSch1983_eng} %%%% оставить
    \Aue{Dennis, J.\,E., and R.\,B.~Schnabel}. 1983.
    \textit{Numerical methods for unconstrained optimization and nonlinear equations.} 
    Englewood Cliffs:     Prentice-Hall. 375~p.
    \end{thebibliography}

 }
 }

\end{multicols}

\vspace*{-3pt}

\hfill{\small\textit{Received October 15, 2016}}

\Contr


\noindent
\textbf{Korolev Victor Yu.} (b.\ 1954)~--- Doctor of Science in physics and mathematics, professor, 
Head of the Department of Mathematical Statistics, 
Faculty of Computational Mathematics and Cybernetics, 
M.\,V.~Lomonosov Moscow State University, 
1-52~Leninskiye Gory, GSP-1, Moscow 119991, Russian Federation; 
leading scientist, Institute of Informatics Problems, Federal Research Center 
``Computer Science and Control'' of the Russian Academy of Sciences, 44-2~Vavilov Str., 
Moscow 119333,  Russian Federation; \mbox{vkorolev@cs.msu.su} 

 \vspace*{3pt}

\noindent
\textbf{Korchagin Alexander Yu.} (b.\ 1989)~---
junior scientist, Faculty of Computational Mathematics and Cybernetics, 
M.\,V.~Lomonosov Moscow State University, 1-52~Leninskiye Gory, GSP-1, Moscow 119991, 
Russian Federation; Institute of Informatics Problems, Federal Research Center 
``Computer Science and Control'' of the Russian Academy of Sciences, 44-2~Vavilov Str., 
Moscow 119333, Russian Federation; \mbox{sasha.korchagin@gmail.com}

\vspace*{3pt}

\noindent
\textbf{Zeifman Alexander I.} (b.\ 1954)~---
Doctor of Science in physics and mathematics, professor, Head of Department, 
Vologda State University, 15~Lenin Str., Vologda 160000, Russian Federation; 
senior scientist, Institute of Informatics Problems, Federal Research Center 
``Computer Science and Control'' of the Russian Academy of Sciences, 44-2~Vavilov Str., 
Moscow 119333, Russian Federation; principal scientist, 
ISEDT RAS, 56-A~Gorky Str., Vologda 16001, Russian Federation; a\_zeifman@mail.ru
\label{end\stat}


\renewcommand{\bibname}{\protect\rm Литература} 