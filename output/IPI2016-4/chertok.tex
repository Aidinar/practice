\def\stat{chertok}

\def\tit{МЕТОД КУМУЛЯТИВНЫХ СУММ ДЛЯ~ПОИСКА СМЕНЫ РЕЖИМА В~ПРОЦЕССЕ 
ОРНШТЕЙНА--УЛЕНБЕКА\\ НА~ОСНОВЕ ПРОЦЕССА ЛЕВИ$^*$}

\def\titkol{Метод кумулятивных сумм для поиска смены режима в~процессе 
Орнштейна--Уленбека на основе процесса Леви}

\def\aut{А.\,В.~Черток$^1$, А.\,И.~Каданер$^2$, Г.\,Т.~Хазеева$^3$, И.\,А.~Соколов$^4$}

\def\autkol{А.\,В.~Черток, А.\,И.~Каданер, Г.\,Т.~Хазеева, И.\,А.~Соколов}

\titel{\tit}{\aut}{\autkol}{\titkol}

\index{Черток А.\,В.}
\index{Каданер А.\,И.}
\index{Хазеева Г.\,Т.}
\index{Соколов И.\,А.}
\index{Chertok A.\,V.}
\index{Kadaner A.\,I.}
\index{Khazeeva G.\,T.} 
\index{Sokolov I.\,A.}


{\renewcommand{\thefootnote}{\fnsymbol{footnote}} \footnotetext[1]
{Работа выполнена при частичной 
финансовой поддержке РФФИ (проект 14-07-00041).}}


\renewcommand{\thefootnote}{\arabic{footnote}}
\footnotetext[1]{Факультет вычислительной математики и~кибернетики 
Московского государственного университета им.\ М.\,В.~Ломоносова; Сбербанк России, 
\mbox{avchertok.sbt@sberbank.ru}}
\footnotetext[2]{Механико-математический 
факультет Московского государственного университета им.\ М.\,В.~Ломоносова; 
Сбербанк России, \mbox{aikadaner.sbt@sberbank.ru}}
\footnotetext[3]{Факультет вычислительной математики и~кибернетики 
Московского государственного университета им.~М.\,В.~Ломоносова, 
\mbox{gelana.khazeyeva@gmail.com}}
\footnotetext[4]{Институт проб\-лем информатики Федерального 
исследовательского центра <<Информатика и~управ\-ле\-ние>> Российской академии наук, 
\mbox{isokolov@ipiran.ru}}

\vspace*{-3pt}

\Abst{Рассматривается процесс Орн\-штей\-на--Улен\-бе\-ка (ОУ) с~трендом 
на основе процесса Леви для моделирования финансовых временных рядов. 
Продемонстрировано, что использование процесса Леви в~основе процесса 
ОУ дает больше гибкости для описания финансовых 
временных рядов по 
сравнению с~классической гауссовой моделью. В~частности, процесс Леви позволяет 
моделировать остатки с~тяжелыми хвостами, что является  распространенным 
свойством реальных временных рядов. Приводятся эффективные решения для 
оценивания параметров модели с~использованием таких методов, как OLS (ordinary least squares)
и~RLS (regularized least squares). 
Решается задача поиска моментов смены режима в~модели при условии поступления 
данных в~режиме реального времени. Приведен алгоритм, основанный на  
CUSUM (CUmulative SUM) ме\-то\-дах,  способный последовательно обрабатывать смены режима и~поддерживать 
параметры модели актуальными для каждого момента времени.  Решение задачи поиска 
разладки модели и~соответствующих смен режима имеет важное прикладное значение, 
поскольку в~большинстве случаев параметры моделей, описывающих динамику реальных 
систем, меняются во времени под действием внешних факторов.}

\KW{случайные процессы; процессы со свойством возвратности к~среднему; 
процесс Орн\-штей\-на--Улен\-бе\-ка, управляемый процессом Леви; процесс 
Орн\-штей\-на--Улен\-бе\-ка с~трендом; смена режима; CUSUM-ал\-го\-ритмы}

\DOI{10.14357/19922264160405}

\vspace*{-3pt} 


\vskip 10pt plus 9pt minus 6pt

\thispagestyle{headings}

\begin{multicols}{2}

\label{st\stat}

\section{Введение}

Процессы со свойством возвратности к~среднему играют важную роль в~моделировании 
динамики явлений из различных областей человеческой деятельности.  В~частности, 
эти процессы привлекательны для моделирования различных явлений в~эконометрике, 
таких как процентные ставки, курсы обмена валют и~цены на сырьевые товары, где 
свойство возвратности к~среднему имеет фундаментальную природу. 

В~работе~\cite{brigo2007} рассмотрено несколько видов случайных процессов со свойством 
возвратности к~среднему и~описаны их основные характеристики.
В~настоящей статье в~качестве такого процесса рассматривается процесс 
ОУ, управляемый процессом Леви. 

Классическая версия 
процесса была 
впервые представлена в~совместной работе голландских физиков Л.\,С.~Орнштейна 
и~Дж.\,Е.~Уленбека~\cite{ou1930} в~качестве модели, которая способна описать данные 
с~гауссовской и~диффузионной структурой. В~экономике же классический процесс 
ОУ известен как модель Васичека благодаря фундаментальной 
работе~\cite{vasicek1977}, где автор предлагает использовать ее для 
моделирования временн$\acute{\mbox{о}}$го ряда процентной ставки. Ее основной недостаток 
заключается в~том, что существует ненулевая вероятность появления отрицательных 
значений, нереалистичных для экономических процессов. Для решения данной 
проблемы позднее была разработана экспоненциальная модель Васичека, а~также 
модель процесса Кок\-са--Ин\-гер\-сол\-ла--Рос\-са, также называемая <<мо\-делью 
с~квад\-рат\-ным корнем>>, в~которой процентная ставка принимает только 
неотрицательные значения и~имеет гам\-ма-рас\-пре\-де\-ле\-ние~\cite{cox1985}.

        \begin{figure*} %fig1
        \vspace*{1pt}
\begin{center}
\mbox{%
\epsfxsize=109.749mm
\epsfbox{che-1.eps}
}
\end{center}
\vspace*{-9pt}
 \Caption{График соотношения цен для фьючерсов компаний 
<<Лукойл>> и~<<Роснефть>>}
                \label{rtsmixpic}
        \end{figure*}


В настоящей статье подтверждается тот факт, что предположение нормальности 
в~классической версии процесса ОУ не описывает реальную структуру 
данных, и~поэтому рассматривается обоб\-щение классического процесса~--- процесс 
ОУ,\linebreak управ\-ля\-емый процессом Леви. Некоторые его модификации 
изучены в~работе~\cite{GarOlk2000}. Предложено рас\-смат\-ри\-вать нормальный обратный 
гауссовский и~дис\-пер\-сионный гам\-ма-про\-цесс для описания динамики его остатков. 
Распределения прираще\-ний этих процессов имеют хвосты тяжелее, чем у~нормального 
распределения, что часто встречается в~реальных данных. 

Дополнительная мотивация 
в~использовании именно этих распределений исходит из приложений в~финансах. 
Например, дисперсионное гам\-ма-рас\-пре\-де\-ле\-ние используется для моделирования цен 
акций, как это делается в~работе~\cite{Fin2009}, а~нормальное обратное 
гауссовское распределение хорошо описывает логарифмические приращения цен 
активов, например в~работе А.\,В.~Кузьминой~\cite{Kuzmina2011} это подтверждается 
на примере данных о~цене фьючерса RTS.

Для более общего механизма построения моделей к~классической модели ОУ 
добавляется линейная составляющая, или тренд. Такой подход позволяет 
моделировать большее число явлений, не выходя за рамки одной модели.

Как известно, финансовые рынки являются динамическими и~нестационарными 
системами. Поэтому отношения, связывающие различные факторы рынка, склонны 
меняться во времени. Пример данного явления продемонстрирован на рис.~\ref{rtsmixpic}. 
По оси~$x$ отложены цены фьючерса на акции компании 
<<Роснефть>> (ROSN), а~по оси~$y$~--- цены фьючерса на акции компании <<Лукойл>> 
(LKOH) с~08.01.2013 по~28.10.2016. Видно, что параметры этой зависимости 
являются также изменяющимися во времени на протяжении дня, так как можно 
отчетливо выделить области, где точки группируются в~окрестностях прямых 
с~разными параметрами.

Все это ставит задачу определения моментов, в~которые предложенная для описания 
данных модель с~определенными параметрами перестает работать, после чего процесс 
начинает следовать той же самой модели, но уже с~другими параметрами. В~данной 
статье эта проблема решена для модели\linebreak
 ОУ. Более того, предлагается процедура 
оценивания параметров  и~обнаружения смен режима в~реаль\-ном времени 
с~использованием RLS или рекурсивного метода наименьших квадратов для оценивания 
параметров, а~также алгоритм, основанный на CUSUM-про\-це\-ду\-рах для обнаружения 
смен режима. В~конце статьи предложенная процедура применяется на различных 
данных.


\section{Моделирование временного ряда}

        \subsection {Одномерный процесс Орнштейна--Уленбека}

       Процесс ОУ с~трендом, управляемый процессом Леви, 
определяется как решение стохастического дифференциального уравнения  (СДУ):       
\begin{align*}
d\left(X_t -\mu -  \nu t\right)& = -\alpha\left(X_t - \mu - \nu 
t\right) dt +  dL_{\lambda t}\,,\\ 
&\hspace{45mm}\forall\  t>0\,;  \\
X(0) &= X_0\,,  
       \end{align*}
         где $\alpha, \mu \in \mathbb{R}$; $L_t$~--- процесс Леви; $X_0$~--- 
некоторая случайная величина, независимая от~$\{L_t\}$; $\nu$ определяет 
постоянный на всем промежутке времени линейный тренд. Параметр~$\mu$ здесь 
означает долгосрочное среднее, а~$\alpha$ определяет скорость стремления 
процесса возвращаться к~своему среднему~--- тренду.

Как показано в~\cite{Protter}, данное СДУ имеет сле\-ду\-ющее решение:
\begin{multline*}
X(t) =\nu t + \mu + \exp\left(-\alpha t\right) \times{}\\
{}\times
\left(
\left(X_0 - \mu \right)+ \int\limits_0^t\exp(\alpha  s)\,dL_{\lambda s}\right)\,, 
\quad X_0 = X(0)\,,
\end{multline*}
или
\begin{multline}
X(t + \tau) =\mu +\nu  (t + \tau) +   \exp
\left(-\alpha \tau\right)\times{}\\
\!\!{}\times\! \left(\!\!
(X(t) - \mu - \nu  t) + \exp(- \alpha t)\!\int\limits_t^{t + \tau}\!\!
\exp(\alpha s)\,dL_{\lambda s}\!\right).\!\!\!\!
\label{explicit_ou}
\end{multline}
Отсюда, в~частности, следует, что $X_t$~--- марковский процесс. Еще стоит 
заметить, что данное решение единственно с~точностью до неотличимости~\cite{sato}. 
Для более подробного рассмотрения процессов Леви см.~[8, 10].

Для удобства обозначим через $Y_t\hm = X_t \hm-\mu\hm- \nu t$ соответствующий приведенный 
процесс ОУ без тренда и~имеющий нулевое среднее.

Свойство возвратности  процесса~$Y_t$ к~нулевому уровню  при $\alpha \hm> 0$ может 
быть получено из~(\ref{explicit_ou}):  если~$Y_t$ стал больше~0 в~момент 
времени~$t$, то коэффициент при~$dt$ отрицательный и~$Y_t$ будет стремиться 
немедленно вернуться к~0; аналогично происходит, если случайный процесс 
становится меньше~0.

\subsubsection{Авторегрессия и~оценка параметров}

Пусть $ X ^ * = {\left(X^*_{t_i}\right)}_ {i = 1, \ldots, N} $~--- 
наблюдения с~интервалом 
$\Delta \hm=  1$ процесса, описываемого определенной выше моделью ОУ с~трендом. 
В~дискретном случае уравнение процесса~(\ref{explicit_ou}) выглядит следующим 
образом:
        \begin{multline}
         \label{OUtrend_d}
        X_{i+1} = \mu + \mu_0\left(1 - e^{-\lambda}\right) + 
        \mu \left(1 - e^{-\lambda}\right)i+ {}\\
        {}+e^{-\lambda } X_{i} + l_i\,,
        \end{multline}
        где $l_i $~--- некоторая случайная величина с~нулевым средним.

        Соотношение~(\ref{OUtrend_d}) описывается регрессионной моделью. Запишем 
его в~виде:
        \begin{equation*} 
%        \label{OUtrend_regr}
        X_{i+1} = c + b t_i + a X_{i} + l_i\,.
        \end{equation*}

        Чтобы оценить параметры $a$, $b$ и~$c$ регрессии, можно воспользоваться 
методом наименьших квад\-ра\-тов и~получить оценки~$\hat{a}$, $\hat{b}$ и~$\hat{c}.$ 
Тогда параметры исходного процесса ОУ с~трендом можно получить 
из соотношений:
        \begin{equation*}
                \hat{\lambda} = -\fr{\ln\hat{a}}{\tau}\,;\quad
                \hat{\mu} = \fr{\hat{b}}{1-\hat{a}}\,; \quad
                \hat{\mu}_0 = \fr{\hat{c} - \hat{\mu} \tau}{1 - \hat{a}}\,.
        \end{equation*}

        Из независимости приращений также можно явно посчитать логарифмическую 
функцию правдоподобия: 
\begin{multline*}
L\left(X^*, \theta\right) =  \sum\limits_{k = 2}^n \ln 
f_{Y_i |Y_{i -1}}(X_i , \theta) = {}\\
{}=  \sum\limits_{k = 2}^n \ln 
f\left(Y_i - a Y_{i - 1} - c,\theta\right)\,,
\end{multline*}
где $\theta$~--- параметры модели.

\subsubsection{Симуляция}

     Используя соотношения авторегрессионного вида процесса ОУ, можно 
смоделировать процесс ОУ итеративно, задав некоторую начальную 
точку~$X_0$. На рис.~2 проиллюстрирован построенный 
итеративно  процесс ОУ с~положительным трендом.

{ \begin{center}  %fig2
 \vspace*{6pt}
 \mbox{%
\epsfxsize=77.781mm
\epsfbox{che-2.eps}
}
\end{center}

%\vspace*{-3pt}


\noindent
{{\figurename~2}\ \ \small{Пример процесса ОУ с~трендом ($\alpha\hm=0{,}5$, 
$\nu\hm=0{,}1$, $\mu_0\hm=0$ и~$\sigma\hm=1$)}}
}

\addtocounter{figure}{1}

\begin{table*}[b]\small
\begin{center}
\Caption{Характеристики дисперсионного гам\-ма-про\-цес\-са $V \hm= (V_t)_{t \geqslant 0}$}
\label{table1}
\vspace*{2ex}

                \begin{tabular}{|c|c|c|c|}
                        \hline
                       \tabcolsep=0pt\begin{tabular}{c}
                        Математическое\\ ожидание\end{tabular} &                         Дисперсия &
                                               Асимметрия&                         Эксцесс\\
                                               \hline
                                               &&&\\[-9pt]
                        $\theta t$  & $\left(\sigma^2 + \nu \theta^2\right) t$ 
 & $\displaystyle
                        \fr{\theta \nu \left(3 \sigma^ 2 + 2 \nu 
\theta^2\right)}{t^{{1}/{2}}} \left(\sigma^2 + \nu \theta ^2\right)^{{3/}{2}}$ 
 & $\displaystyle 3 \left( 1 + \fr{2 \nu}{t} - \nu \theta^4 
t \left(\sigma^2 + \nu \theta^2\right)^{-2} \right) $ \\
                        \hline
                \end{tabular}
        \end{center}
%\end{table*}
%\begin{table*}\small
\begin{center}
\Caption{Характеристики нормального обратного гауссовского распределения}
\label{table2}
\vspace*{2ex}

                \begin{tabular}{|c|c|c|c|}
                        \hline
                        Математическое ожидание & 
                                                Дисперсия &
                                                                        Асимметрия &
                                    Эксцесс \\
 \hline
 &&&\\[-9pt]
 $\mu + \delta \tau$ &
 $\displaystyle\fr{\delta^2(1 + \tau^2)}{\xi}$ 
& $\displaystyle\fr{3}{\tau \sqrt{\xi (1 + \tau^2)}}$ 
& $\displaystyle\fr{3}{\xi} \left(  1 + 4 \fr{\tau^2}
                        {1 + \tau^2}  \right)$  \\[8pt]
                        \hline
                \end{tabular}
        \end{center}
\end{table*}
        

\subsection{Частные случаи моделирования остатков процесса Орнштейна--Уленбека}

В работе~\cite{taufer} авторы приводят быстрые и~эффективные методики для 
симуляции различных ОУ-про\-цес\-сов, управляемых процессом Леви, а~также  
описание множества различных частных его случаев. Будем рассматривать три типа 
процессов Леви: винеровский процесс, дисперсионный гам\-ма-про\-цесс (VG), а~также 
нормальный обратный гауссовский процесс (NIG).  Оба последних процесса 
моделируют тяжелые хвосты и~принадлежат классу обобщенных гиперболических 
распределений. Они часто применяются в~финансах и~эконометрике (для 
VG см.~[12,  13]), для NIG см.~[10, 14--16]).

\subsubsection {Дисперсионный гамма-процесс}

\noindent
\textbf{Определение 2.1.}\
Случайная величина~$\xi$ имеет дисперсионное гам\-ма-рас\-пре\-де\-ле\-ние, 
если ее плотность распределения имеет вид:
\begin{multline} \label{vgpdf}
 f_\xi(x) = \int\limits_{0}^{\infty} \fr{1}{\sigma  \sqrt{2 \pi g}} 
 \exp \left( - \fr{(x - \theta g)^2}{2 \sigma^2 g}  \right) \times{}\\
 {}\times
\fr{g ^{{1}/{\nu} - 1} \exp \left( - {g}/{\nu} \right)}
{\nu  ^{{1}/{\nu}} \Gamma \left({1}/{\nu}\right) }\, dg\,,\enskip x \in \mathbb{R},
                \end{multline}
                где $\Gamma (x)$, $x\hm>0$,~--- гам\-ма-функ\-ция, 
                а~$\sigma \hm> 0$, $\nu \hm> 0$,  
$\theta \hm\in \mathbb{R}$.


        Обозначение: $\xi \sim V(\sigma, \nu, \theta)$.

\smallskip

\noindent
\textbf{Определение~2.2.}\
        Случайный  процесс  $V \hm= (V_t)_{t \hm\geqslant 0} $  с~ параметрами        
$\sigma\hm >0$, $\nu \hm> 0$, $\theta \hm\in \mathbb{R} $, заданный на вероятностном 
пространстве $ (\Omega, F, \mathbb{P}) $ со
        значениями в~$ \mathbb{R}$, называется дисперсионным гам\-ма-про\-цес\-сом, 
если $V_0\overset{\mathrm{p.n.}}{=} 0$, $V$ имеет независимые приращения и~для любых 
$s \hm\geqslant 0$, $t \hm\geqslant 0$ 
$V$ имеет стационарные приращения с~дисперсионным 
гам\-ма-рас\-пре\-де\-ле\-ни\-ем~(\ref{vgpdf}) с~параметрами $\sigma \sqrt{t}\hm > 0$, 
$\nu/t \hm> 0$ и~$t \theta\hm > 0$,~т.\,е.\
$$
V_{t+s} - V_s \overset{\mathrm{d}}{=} V_t - V_0 \sim 
V\left(\sigma \sqrt{t}, \fr{\nu}{t}, t \theta\right) \,.
$$

\smallskip

        Характеристики дисперсионного гам\-ма-про\-цес\-са $V \hm= (V_t)_{t \geqslant 0}$ 
с параметрами~$\sigma$, $\nu$ и~$\theta$ представлены в~табл.~1.


       
        В работе~\cite{madancarr} показано, что плотность дисперсионного 
        гамма-процесса  $V \hm= (V_t)_{t \geqslant 0}$ выражается аналитически с~использованием 
модифицированной функции Бесселя второго рода с~индексом~$\nu$.

        \subsubsection {Нормальный обратный гауссовский процесс}

\noindent
\textbf{Определение 2.3.}\  Случайная величина~$\eta$ имеет 
нормальное обратное гауссовское распределение с~параметрами~$\alpha$, $\beta$, 
$\delta$ и~$\mu$ ($\eta \hm\sim \mathrm{NIG}\,(\alpha, \beta, \delta, \mu)$), если ее плотность 
распределения имеет вид:
                \begin{multline*} 
%                \label{nigpdf}
                   \hspace*{-3mm}f_\eta(x, \alpha, \beta, \delta, \mu) = \fr{\alpha 
\delta}{\pi} \exp \left(\delta \sqrt{\alpha^2 - \beta^2} + \beta (x - \mu)\right) 
\times{}\\
{}\times \fr{K_1\left(\alpha \sqrt{\delta^2 + (x - \mu)^2}\right)}{\sqrt{\delta ^2 + (x - 
\mu)^2}}\,,
                \end{multline*}
                где $K_1(z) = (1/2) \int\nolimits_{0}^{\infty} \exp 
                (-({1}/{2}) z (u \hm+ u^{-1}))\,du$, $z\hm>0$,~--- 
                модифицированная функция Бесселя 
второго рода с~индексом~1, $\alpha \hm> 0$, $-\alpha \hm< \beta \hm< \alpha$, 
$\delta \hm> 0$,  $\mu \hm\in  \mathbb{R}$, $x\hm>0$.

                Параметры $\alpha$, $\beta$, $\delta$ и~$\mu$ являются параметрами 
формы, асимметрии, масштаба и~расположения соответственно. 

\smallskip

\noindent
\textbf{Определение 2.4.}\
                Случайный процесс $N \hm= (N_t)_{t \geqslant 0}$ с~параметрами 
$\alpha$, $\beta$, $\delta$ и~$\mu$, заданный на вероятностном пространстве $(\Omega, 
F, P)$ со значениями  в~$\mathbb{R}$,\linebreak
 называется нормальным обратным гауссовским 
процессом, если $N_0 \overset{\mathrm{p.n.}}{=} 0$, $N$ имеет независимые приращения 
и~для любых $s \hm\geqslant 0$, $t \hm\geqslant 0$ $N$ имеет стационарные приращения 
с~нормальным обратным гауссовским распределением:
$$
    N_{t+s} - N_s  \overset{\mathrm{d}}{=} N_t - N_0 \sim 
\mathrm{NIG}\,( \alpha, \beta, \delta t, \mu t) 
  $$
        с~параметрами $\alpha \hm> 0$, 
        $-\alpha \hm< \beta \hm< \alpha$, $\delta t\hm > 0$ 
и~$\mu t \hm\in  \mathbb{R}$.

\smallskip

    Плотность нормального обратного гауссовского распределения может быть 
представлена в~аналитической форме.

        Характеристики нормального обратного гауссовского распределения 
представлены в~табл.~\ref{table2}.



\section {Оценивание параметров и~поиск смен режима в~реальном времени}

В~данной разделе рассмотрено моделирование и~описание данных при условии их 
поступления в~режиме реального времени, когда значения выборки данных поступают 
одно за другим. Специфика данной задачи заключается в~высокой скорости 
поступления данных в~ее приложениях и~их большом объеме, поэтому любые 
приводимые алгоритмы должны быть достаточно быстрыми и~эффективно использовать 
компьютерную память.

\subsection{Оценивание параметров}

Без каких-ли\-бо ограничений на компьютерные мощности самым очевидным решением для 
оценивания параметров было бы на каждом шаге использовать метод наименьших 
квадратов (OLS). Чтобы удовлетворить необходимость обрабатывать потоковые 
данные, воспользуемся рекурсивным методом наименьших квадратов (RLS). Данный 
алгоритм на каждом шаге обновляет рекурсивно оценку параметра~$\theta$ 
и~ковариационную матрицу~$X^{\mathrm{T}} X$  вмес\-то того, чтобы насчитываться с~нуля каждый 
раз. Данный алгоритм и~его реализация хорошо известны и~могут быть найдены 
в~\cite{haykin}.

\subsection{Постановка простейшей смены режима}

        В реальной жизни некоторые явления могут быть связаны отношениями, 
например линейными, параметры которых изменяются во времени. Самым простым 
подобным примером является процесс, описываемый следующим образом:      
 \begin{equation*}
        M_t= 
                        \begin{cases}
                M_t^1\,, &\ t \leqslant t^*\,; \\
                M_t^2\,, &\ t > t^*\,,
                \end{cases}
        \end{equation*}
        где $t \in [1,\ldots,T]$ обозначает время; $ t^*$~--- критическое 
значение внешней переменной~$t$, или момент смены режима (change point, regime 
switch);  $M^{1,2}$~--- это две различные модели, соответствующие разным 
временным промежуткам: до и~после.
        В общем случае нельзя точно определить значение~$t^*$. Задача состоит 
        в~том, чтобы наилучшим образом оценить ее значение, имея на входе выборку 
наблюдений, при условии что на данном временн$\acute{\mbox{о}}$м промежутке произошла ровно одна 
смена режима (рис.~3), а~также оценить параметры старой и~новой модели.   
В~данном случае 
рассматривается модель ОУ и~исследуемый процесс выглядит 
следующим образом:

\columnbreak

\noindent
 \begin{center}  %fig1
 \vspace*{-2pt}
 \mbox{%
\epsfxsize=77.781mm
\epsfbox{che-3.eps}
}

\vspace*{3pt}

\noindent
{{\figurename~3}\ \ \small{Пример смены режима}}
\end{center}



 \vspace*{6pt}



\addtocounter{figure}{1}


\noindent
\begin {equation*}
        X_t =                \begin{cases}
                \mathrm{OU}_t^1\,, &\ t \leqslant t^*\,; \\
                \mathrm{OU}_t^2\,, &\ t > t^*\,,
                \end{cases}
        \end {equation*}
        где $\mathrm{OU}^i$~--- процессы ОУ, описанные выше.

        

\subsection{Постановка задачи для~потоковых данных}

            В случае потока данных значения выборки $X_1,X_2,\ldots,X_n, \ldots$ 
поступают последовательно. В~этом случае нет предпосылок для того, чтобы 
в~ка\-кой-то момент произошла смена режима, а~сами смены могут происходить 
последовательно много раз. Задача состоит в~том, чтобы последовательно их 
обнаруживать и~предоставлять оценку для параметров новой модели. Эффективность 
методов определяется тем, как часто метод ошибочно определяет режимы и~как 
быстро он способен обнаруживать смену режима.

\subsection{Решение задачи}

В современной литературе можно найти множество методов  для определения смен 
режима. Основным применяемым подходом является по\-стро\-ение по наблюдаемой системе  
некоторого детектора (change-detector), который сигнализирует, когда параметры 
модели перестают соответствовать выборке наблюдений и~предположительно сменился 
режим. Одним из таких методов  является CUSUM-тест, или метод кумулятивных сумм, 
который рассматривается в~данной работе для обнаружения смен режима. Стоит 
отметить другие известные методы, такие как метод GLT (generalized likehood 
test)~\cite{appel} и~MLT (marginalized likelihood text)~\cite{gustafsson}.

\subsubsection{Краткое описание CUSUM-методов}

В данной статье будут рассмотрены два базовых CUSUM-ме\-то\-да: CUSUM-ме\-тод, 
основанный на максимизации правдоподобия выборки наблюдений, и~CUSUM-ме\-тод, 
определяющий смену среднего значения выборки наблюдений.

\subsubsection*{CUSUM-метод максимизации правдоподобия}

Пусть есть некоторый поток данных $X^* \hm= X^*_1,\ldots X^*_n,\ldots $, 
элементы которого 
являются выборкой независимых одинаково распределенных случайных величин. 
Обозначим плотность соответствующей случайной величины через $p_\theta(x).$ 
Предполагается, что в~ка\-кой-то момент времени~$r^*$ может произойти смена режима 
модели. Это означает, что до~$r^*$ в~модели действует параметр~$\theta_0$, 
а~после~---~$\theta_1$. Введем соответствующие гипотезы: гипотезу <<неизменности>> 
модели~$H_0$ (разладки не произошло) и~гипотезу~$H_1$ об <<одноразовой 
разладке>>. Одним из самым известных и~простых методов для нахождения разладки 
модели является тест отношения правдоподобий~\cite{kay}.

\noindent
\textbf{Алгоритм 3.1.}\ %\begin{algorithm}
 Определим логарифмическое отношение правдоподобий для моделей~$H_0$ и~$H_1$: 
\begin{equation*}
\mathrm{LLR} \left(X, r^*\right) = 
\ln\fr{p_{H_1}(X)}{p_{H_0}(X)}\,.
\end{equation*} 
Тогда  принимается гипотеза~$H_1$, если $\mathrm{LLR} \hm>h$, где параметр~$h$  
отвечает за чувствительность 
алгоритма: чем меньше~$h$, тем быстрее будет происходить обнаружение разладки, 
но при этом тем больше будет срабатывать ложных сигналов.

\smallskip

К~сожалению,  в~данном случае неизвестно значение~$r^*$ и~поэтому явно посчитать 
$p_{H_1}(X)$ не представляется возможным. Данную проблему можно решить, перебрав 
все значения~$r^*$ и~взяв то, которому соответствует максимальное значение~$\mathrm{LLR}$. 
Данный метод называется обобщенным методом максимального 
правдоподобия (GLT).


\smallskip

\noindent
\textbf{Алгоритм 3.2.} %\begin{algorithm}\label{algo2}
Определим обобщенное логарифмическое отношение 
правдоподобий для выборки размера~$N$: 
\begin{multline*}
\mathrm{GLLR}(X) =\max\limits_{1   \leqslant r^* \leqslant N } 
\mathrm{LLR}(X, r^*)  ={}\\
{}=\max\limits_{1  \leqslant r^* \leqslant N } 
\ln\fr{p_{H_1}(X)}{p_{H_0}(X)}  =\max\limits_{1  \leqslant r^* \leqslant N } 
\sum\limits_{i = r^*}^N \ln \fr{p_{\theta_1}(X_i)}{p_{\theta_0}(X_i)}\,. 
%\label{gllr} 
\end{multline*}
Тогда на каждом шаге~$n$ принимается гипотеза~$H_1$  против гипотезы~$H_0$, если 
$\mathrm{GLLR}(X)\hm>h.$

\smallskip

Введем  кумулятивную сумму точечных отношений правдоподобий:
\begin{equation*}
S(n) = \sum\limits_{i = 1}^n \ln  
\fr{p_{\theta_1}(X_i)}{p_{\theta_0}(X_i)}\,. 
\end{equation*}
Тогда 
\begin{align*}
\mathrm{LLR} (X, r^*) &= S(N) - S(r^*)\,; \\
\mathrm{GLLR}(X, N) &= S(N) - \min\limits_{1  \leqslant r^* \leqslant N  }S(r^*)\,,
\end{align*}
где
$$
\hat{r}^* = \mathop{\mathrm{argmin}}\limits_{1  \leqslant r^* \leqslant N}S(r^*)\,.
$$
Заметим, что для того чтобы использовать алгоритм~3.2, достаточно 
считать кумулятивную сумму~$S$. Более того, так как уровень~$h$ берется 
положительным, вместо того, чтобы явно насчитывать значение~$\mathrm{GLLR}$ на 
каждом шаге, достаточно рекурсивно считать функцию 
\begin{equation*}
 G(N)  =\max\left(G(N- 1) +\ln \fr{p_{\theta_1}(X_N)}{p_{\theta_0}(X_N)}\,, 
\;0\right),
\end{equation*}
которая совпадает с~$\mathrm{GLLR}$ там, где последняя положительна. В~этом 
и~заключается метод кумулятивных сумм. Из вышесказанного вытекает следующий 
алгоритм, эквивалентный алгоритму~3.2:

\smallskip

\noindent
\textbf{Алгоритм 3.3.} %\begin{algorithm}
На каждом шаге~$N$ принимается гипотеза~$H_1$ против гипотезы~$H_0$, если 
$G(N)\hm>h.$


\smallskip

На практике значение~$\theta_0$ можно оценить, а~значение~$\theta_1$ неизвестно. 
Поэтому  берут $\theta_1 \hm= \theta_0 \hm+ \delta$, где~$\delta$~--- минимальная 
величина, изменение которой хотят детектировать.

\subsubsection*{CUSUM-метод определения смены среднего выборки}

%\smallskip

\noindent
\textbf{Алгоритм 3.4.} %\begin{algorithm}
Определим рекурсивно 
\begin{equation*} 
S_n^+=\max\left (S_{n- 1}^+ + 
\fr{X_N - \mu_0}{\sigma} - k, \;0\right)\,, 
\end{equation*}
где $\mu_0$~--- среднее текущей модели; $\sigma$~--- среднее текущей выборки; $k$~--- 
уровень чувствительности метода к~разбросам. Тогда считается, что принимается 
гипотеза~$H_1$, если $S^+_n \hm> h.$

\smallskip

Подробнее с~алгоритмом можно 
ознакомиться, например, в~\cite{page61}.

\begin{figure*}[b] %fig4
 \vspace*{1pt}
 \begin{minipage}[t]{79mm}
\begin{center}
\mbox{%
\epsfxsize=77.835mm
\epsfbox{che-5.eps}
}
\end{center}
\vspace*{-9pt}
  \Caption{Частичная автокорреляционная функция}\label{pacfpic}
  \end{minipage}
%\end{figure*}
\hfill
%\begin{figure*} %fig5
        \vspace*{1pt}
         \begin{minipage}[t]{79mm}
\begin{center}
\mbox{%
\epsfxsize=78.035mm
\epsfbox{che-4.eps}
}
\end{center}
\vspace*{-9pt}
  \Caption{Автокорреляционная функция}\label{acfpic}
    \end{minipage}
\end{figure*}

\subsection{Общий алгоритм для процесса Орнштейна--Уленбека}

Как было замечено ранее, приведенный процесс ОУ с~трендом обладает 
авторегрессионным свойством AR(1). Поэтому можно применять CUSUM-ме\-тод 
максимального правдоподобия для приращений~$l_i$ данного процесса. Данный метод 
будем\linebreak
 применять для детектирования смены во\-ла\-тиль\-ности модели, в~то время как 
для обнаружения смены среднего, или тренда, будем применять CUSUM-ме\-тод поиска 
смены среднего. Применить первый алгоритм для детектирования среднего 
оказывается сложно из-за большого числа параметров, которые нельзя адекватно 
оценить, в~частности па\-ра\-мет\-ров~$\mu_0$ и~$\nu$.

Таким образом, общий алгоритм следующий:

\smallskip

\noindent
\textbf{Алгоритм 3.5.} %%rithm}

\noindent
\begin{enumerate}[1.]
\item На каждом шаге оцениваем наиболее вероятные параметры выборки~$\theta_0$ 
для выбранной модели с~помощью метода RLS.
\item На каждом шаге считаем значения детекторов CUSUM смены волатильности 
и~смены тренда.
\item В случае, когда детекторы сигнализируют\linebreak о~смене режима, проходимся общим 
методом обобщенного отношения правдоподобий (GLT) по выборке и~находим наиболее 
вероятную точку смены режима~$r^*$ с~оценкой параметров~$\theta_1$. Далее 
исключаем из выборки все ее элементы до~$r^*$ и~продолжаем процедуру алгоритма 
с~$\theta_0:=\theta_1$.
\end{enumerate}


\section{Анализ данных}

       В этом разделе описывается моделирование процессом ОУ 
реальных финансовых данных. В~качестве данных были выбраны секундные данные по 
ценам фьючерсов на индекс RTS ($x_t$) и~на акции компании <<Газпром>> ($y_t$) 
с~MOEX за~07.10.2014. Если наблюдения сделаны через равные промежутки времени, то 
можно рассматривать их как временной ряд.
Предполагается, что разность $z_t \hm= x_t \hm - 6  y_t$ обладает свойством 
стационарности и~может быть описана с~помощью процесса ОУ. Для 
того чтобы ряд имел свойство авторегрессии, вычитаем из ряда его скользящее 
среднее с~периодом 5~мин.

Тест Дики--Фуллера (с уровнем зна\-чи\-мости 0,05) подтверждает предположение 
о~стационар\-ности: значение статистики Ди\-ки--Фул\-ле\-ра: $-25{,}374$; $p$-зна\-че\-ние: 
0,001. Тест  отвергает нулевую гипотезу о существовании единичного корня 
с~уровнем значимости~0,05, что подтверждает стационарность данного ряда.  Для 
проверки наличия свойства AR(1) проанализируем вид автокорреляционной (ACF) 
и~частичной автокорреляционной (PACF) функций.

         Для модели AR(1) характерен следующий вид автокорреляционных 
         и~частичных автокорреляционных функций: график ACF экспоненциально убывает, 
         а~график PACF имеет пик при значении сдвига, равном~1, и~практически равен~0 при 
значениях сдвига более высокого порядка.

       
       На рис.~\ref{pacfpic} изображен график PACF для рас\-смат\-ри\-ва\-емых данных. 
Заметно, что он имеет пик при сдвиге~1 и~практически равен нулю для сдвигов 
более высокого порядка.
        На рис.~\ref{acfpic} ACF для исходных данных убывает экспоненциально.
        Такое поведение графиков ACF и~PACF соответствует модели AR(1).

        Теперь можно говорить о том, что данные представляют собой процесс 
ОУ, и~перейти к~оценке его параметров.

        Для начала оценим параметры~$\theta$ и~$\alpha$ с~по\-мощью метода 
наименьших квадратов, получаем оценки параметров $\hat \theta \hm= 0$ и~$\hat \alpha 
\hm= 0{,}7506$.


        Для того чтобы оценить качество полученных оценок для данного процесса, 
построим QQ-плот для оцененных параметров нормального распределения для остатков 
(рис.~\ref{qqplot}). Это график, где по оси~$x$~--- квантили теоретического 
распределения, а~по оси~$y$~--- эмпирические квантили данных. Если теоретическое 
распределение хорошо описывает\linebreak\vspace*{-12pt}

\pagebreak

\end{multicols}

\begin{figure*} %fig6
        \vspace*{1pt}
\begin{center}
\mbox{%
\epsfxsize=161.601mm
\epsfbox{che-6.eps}
}
\end{center}
\vspace*{-11pt}
\Caption{Графики QQ-plot для оценивания параметров}\label{qqplot}
\vspace*{-3pt}
\end{figure*}

\begin{multicols}{2}

\noindent
 реальные данные, то график <<кван\-тиль--кван\-тиль>> 
близок к~прямой $y \hm= x$.


        Видно, что нормальное распределение не очень\linebreak хорошо описывает 
распределение остатков. По\-пробуем вместо нормального распределение\linebreak использовать 
распределение с~более тяжелыми хвостами, например дисперсионное 
гам\-ма-рас\-пре\-де\-ле\-ние и~нормальное обратное гауссовское распределение. Для оценки 
параметров этих распределений используем метод максимального правдоподобия, 
описанный в~п.~3.2.3.

       

        Анализируя графики QQ-plot (см.\ рис.~\ref{qqplot}) для оцененных параметров 
дисперсионного гамма- и~нормального обратного гауссовского распределений для 
остатков, можно прийти к~выводу, что дисперсионное гамма- и~нормальное обратное 
гауссовское распределение лучше описывают структуру независимых приращений 
в~процессе~ОУ.

        Для того чтобы оценить качество полученных результатов, применим критерий 
согласия Колмогорова для новой выборки данных. Результат подсчета статистики 
представлен в~табл.~3.



        По результатам, представленным в~табл.~3, \mbox{можно} заключить, 
что гипотеза о нормальном рас\-пре\-делении остатков отвергнута при уровне 
зна\-чи\-мости~0,01, гипотеза о~дисперсионном гам\-ма-рас\-пре\-де\-ле\-нии остатков 
и~нормальном обратном гауссовском распределении остатков принята при уровне 
значимости~0,01.

\vspace*{12pt}

\noindent
{{\tablename~3}\ \ \small{Оценки параметров с~результатом критерия Колмогорова}}

\vspace*{1pt}

{\small
 \begin{center}  %
\tabcolsep=3pt
                        \begin{tabular}{|c|c|c|c|c|}
                                \hline
                                \multicolumn{3}{|c|} {Оценка параметра} &
\tabcolsep=0pt\begin{tabular}{c} Значение\\ статистики \end{tabular}&  
$p$-значение  \\
                                \hline
\multicolumn{1}{|c|}{\raisebox{-6pt}[0pt][0pt]{$N(\mu, \sigma^2)$}}
                                & $\hat \mu$ & 0 &  &   \\ 
                              %  \cline{2-3}
                                & $\hat \sigma$ & 
3,6885 &
\multicolumn{1}{c|}{\raisebox{6pt}[0pt][0pt]{ 0,087731}} & 
\multicolumn{1}{c|}{\raisebox{6pt}[0pt][0pt]{$1{,}6679\cdot 10^{-12}$}}\\
                                \hline
                                & $\hat \sigma$ & 3,6712 &  &   \\ 
                                %\cline{2-3}
$\mathrm{VG}\,(\sigma, \nu, \theta)$& $\hat \nu$ & 1,5226 & 0,026806 &   0,14799  \\ 
%\cline{2-3}
                                & $\hat \theta$ & 0,0379 &  & 
\\
                                \hline
\multicolumn{1}{|c|}{\raisebox{-18pt}[0pt][0pt]{$\mathrm{NIG}\,(\theta, \xi, \delta, \mu)$}}
                                & $\hat \theta$ & 0,0592 &  &   \\ 
                                %\cline{2-3}
                                & $\hat \xi$ & 0,3714 &  &   \\ 
                                %\cline{2-3}
                                & $\hat \delta$ & 2,2690 &  &   \\ 
                                %\cline{2-3}
                                & $\hat \mu$ & $-$0,0963 & 
 \multicolumn{1}{c|}{\raisebox{18pt}[0pt][0pt]{0,034654}} & 
 \multicolumn{1}{c|}{\raisebox{18pt}[0pt][0pt]{0,025953}}  
\\
                                \hline
                        \end{tabular}
                        \vspace*{3pt}
\end{center}
}

\pagebreak

\end{multicols}

 \begin{figure*} %fig7
         \vspace*{1pt}
         \begin{minipage}[t]{80mm}
\begin{center}
\mbox{%
\epsfxsize=78.067mm
\epsfbox{che-7.eps}
}
\end{center}
\vspace*{-9pt}
        \Caption{Пример применения CUSUM-алгоритма}
        \label{cusum_vola}
%        \end{figure*}
\end{minipage}
\hfill
%        \begin{figure*} %fig8[H]
                 \vspace*{1pt}
                          \begin{minipage}[t]{80mm}
\begin{center}
\mbox{%
\epsfxsize=78.067mm
\epsfbox{che-8.eps}
}
\end{center}
\vspace*{-9pt}
               \Caption{Пример применения CUSUM-алгоритма}
        \label{cusum_mean}
        \end{minipage}
        \end{figure*}

\begin{multicols}{2}

        Полученный результат говорит о том, что структура реальных данных 
сложнее, чем может описать классический процесс ОУ. 
Целесообразнее использовать обобщенный процесс ОУ, где процесс 
броуновского движения заменен на процесс Леви.

        \subsection{Применение алгоритма для~детектирования изменения 
волатильности}

       Будем рассматривать гауссовский процесс ОУ. Для этого 
были сгенерированы две выборки процесса размера~100 с~$\sigma_1\hm=1$ и~$\sigma_2\hm=3.$ 
Для CUSUM-тес\-та будем брать $\theta_1\hm=2$, т.\,е.\ $\delta\hm=1$. Уровень $h\hm=45.$ 
Результат применения алгоритма проиллюстрирован на рис.~7. На 
рис.~7,\,\textit{а} изображена выборка сгенерированного процесса ОУ со сменой 
режима, в~то время как на рис.~7,\,\textit{б} построено значение CUSUM-де\-тек\-то\-ра. Сплошная 
вертикальная линяя обозначает фактическую смену режима, а~пунктирная~--- время ее 
обнаружения. Заметим, что смена режима могла бы быть обнаружена быстрее при 
другом выборе уровня~$h$.

        \subsection{Применение алгоритма для~обнаружения тренда}

        Для обнаружения тренда также были сгенерированы две выборки гауссовского 
процесса ОУ, которые потом были склеены. Параметры процессов 
следующие: $\alpha_0\hm=0{,}5$, $\nu_0\hm=0$, $\mu_0^0\hm=0$, 
$\sigma_0\hm=0{,}6$, $\alpha_1\hm=0{,}5$, 
$\nu_1\hm=0{,}05$, $\mu_0^1\hm=0$ и~$\sigma_1\hm=0{,}6$. 
Аналогично построена выборка и~значения 
CUSUM-де\-тек\-то\-ра. Уровень $h\hm=13.$ Алгоритм успешно определил смену режима 
(рис.~\ref{cusum_mean}).
       



\section {Заключение}

В статье рассмотрен процесс ОУ с~трендом, управ\-ля\-емый процессом 
Леви, для описания финансовых временн$\acute{\mbox{ы}}$х рядов.
На реальных данных было показано, что дисперсионный гамма- и~нормальный обратный 
гауссовский процессы в~качестве процесса Леви способны гораздо точнее описывать 
реальные явления. Также были рас\-смот\-ре\-ны проб\-ле\-мы разладки модели и~поиска смены 
режима в~реальном времени. Была представлена процедура обнаружения разладки 
модели, а~также определения параметров новой модели. Данный алгоритм способен 
детектировать многократные смены режима последовательно, сохраняя текущую модель 
актуальной для текущего потока данных.

{\small\frenchspacing
 {%\baselineskip=10.8pt
 \addcontentsline{toc}{section}{References}
 \begin{thebibliography}{99}
\bibitem{brigo2007}
    \Au{Brigo D., Dalessandro~A., Neugebauer~M., Triki~F.} A~stochastic 
processes toolkit for risk management.~--- London: King's College 
London, November 2007.  Working paper. 48~p.


\bibitem{ou1930}
\Au{Ornstein L.\,S., Uhlenbeck~G.\,E.} On the theory of the Brownian motion~// 
Phys. Rev., 1930. Vol.~36. No.\,5. P.~823.
    
    \bibitem{vasicek1977}
    \Au{Vasicek O.} An equilibrium characterization of the term structure~// 
J.~Financ. Econ., 1977. Vol.~5. P.~177.

\bibitem{cox1985}
        \Au{Cox J.\,C., Ingersoll E., Jr., Ross~S.\,A.} A~theory of the term 
structure of interest rates~//  Econometrica, 1985. Vol.~53. No.\,2.  P.~385--407.



\bibitem{GarOlk2000}
    \Au{Garbaczewski P., Olkiewicz~R.} Ornstein--Uhlenbeck--Cauchy process~// 
J.~Math. Phys., 2000. Vol.~41. P.~6843.

\bibitem{Fin2009}
    \Au{Finlay R.} The variance gamma (VG) model with long range dependence: 
A~model for financial data incorporating long range dependence in squared 
returns.~--- Sydney, Australia: University of Sydney, School of 
Mathematics and Statistics, 2009. PhD Thesis. 144~p.

\bibitem{Kuzmina2011}
    \Au{Кузьмина А.\,В.} Моделирование нормального обратного гауссовского 
процесса и~оценивание его параметров~// Информатика, 2011. №\,2. С.~133--136.

    \bibitem{Protter}
    \Au{Protter P.} Stochastic integration and differential equations.~--- 
Heidelberg: Springer-Verlag, 1990. 415~p.
    
           \bibitem{sato}
    \Au{Sato K.\,I.} L$\acute{\mbox{e}}$vy processes and 
    infinitely divisible distributions.~--- Cambridge: Cambridge University Press, 1999.
    500~p.
    
        \bibitem{nielsen}
\Au{Barndorff-Nielsen O.\,E., Shephard~N.} Non-Gaussian Ornstein--Uhlenbeck-based 
models and some of their uses in financial economics~// 
J.~Roy. Stat. Soc. B, 2001. Vol.~63. P.~167--241.
    
    \bibitem{taufer}
\Au{Taufer E., Leonenko~N.} Simulation of L$\acute{\mbox{e}}$vy-driven 
Ornstein--Uhlenbeck processes with given marginal distribution~// 
Comput. Stat. Data An., 2008. Vol.~53.  P.~2427--2437.

\bibitem{madanseneta}
    \Au{Madan D.\,B., Seneta~E.} The VG model for share market returns~// 
    J.~Bus., 1990. Vol.~63. P.~511--524.

\bibitem{madancarr}
\Au{Madan D.\,B., Carr P.\,P., Chang~E.\,C.} The variance gamma 
process and option pricing~// Eur. Finance Rev., 1998. Vol.~2. P.~79--105.

\bibitem{nielsen2}
 \Au{Barndorff-Nielsen O.\,E.} Normal inverse Gaussian distributions and 
stochastic volatility modelling~// Scand. J.~Stat., 1997. Vol.~24. No.\,1. P.~1--13.
    
   
    
    \bibitem{rydberg}
\Au{Rydberg H.} The Normal inverse Gaussian L$\acute{\mbox{e}}$vy process: Simulation 
and approximation~// Commun. Stat. Stochastic Models, 1997. 
Vol.~13. No.\,4. P.~887--910.

 \bibitem{nielsen3}
   \Au{Barndorff-Nielsen O.\,E.} Processes of normal inverse Gaussian type~// 
Financ. Stoch., 1998. Vol.~2. P.~41--68.

\bibitem{haykin}
    \Au{Haykin S.} Adaptive filter theory.~--- 3rd ed.~--- Upper Saddle River, NJ, USA: 
Prentice Hall, 1996. 989~p.

\bibitem{appel}
\Au{Appel U., Brandt~A.\,V.} Adaptive sequential segmentation of 
piecewise stationary time series~// Inform. Sci., 1983. Vol.~29. P.~27--56.

\bibitem{gustafsson}
\Au{Gustafsson F.} The marginalized likelihood ratio test for detecting 
abrupt changes~// IEEE Trans. Automat. Contr., 1996. Vol.~41. P.~66--78.
    
    \bibitem{kay}
    \Au{Kay S.} Fundamentals of statistical signal processing. Vol.~I. 
Estimation theory.~--- Upper Saddle River, NJ, USA: Prentice Hall, 1993. 625~p.

\bibitem{page61}
   \Au{Page E.\,S.} Cumulative sum control chart~// Technometrics, 1961. 
Vol.~3. P.~1--9.
 \end{thebibliography}

 }
 }

\end{multicols}

\vspace*{-6pt}

\hfill{\small\textit{Поступила в~редакцию 20.10.16}}

\vspace*{8pt}

%\newpage

%\vspace*{-24pt}

\hrule

\vspace*{2pt}

\hrule

\vspace*{8pt}


\def\tit{REGIME SWITCHING DETECTION FOR~THE~LEVY DRIVEN ORNSTEIN--UHLENBECK PROCESS 
USING CUSUM METHODS}

\def\titkol{Regime switching detection for the Levy driven Ornstein--Uhlenbeck process 
using CUSUM methods}

\def\aut{A.\,V.~Chertok$^{1,2}$, A.\,I.~Kadaner$^{2,3}$, G.\,T.~Khazeeva$^1$, 
and~I.\,A.~Sokolov$^4$}

\def\autkol{A.\,V.~Chertok, A.\,I.~Kadaner, G.\,T.~Khazeyeva, 
and~I.\,A.~Sokolov}

\titel{\tit}{\aut}{\autkol}{\titkol}

\vspace*{-9pt}

\noindent
$^1$Faculty of Computational Mathematics and Cybernetics, 
M.\,V.~Lomonosov Moscow State University, 1-52~Lenin-\linebreak
$\hphantom{^1}$skiye Gory, GSP-1, 
Moscow 119991, Russian Federation

\noindent
$^2$Sberbank of Russia, 19~Vavilov Str., Moscow 117999, Russian Federation

\noindent
$^3$Faculty of Mechanics and Mathematics, 
M.\,V.~Lomonosov Moscow State University, Main Building, Leninskiye\linebreak
$\hphantom{^1}$Gory, 
GSP-1, Moscow 119991, Russian Federation

\noindent
$^4$Federal Research Center ``Computer Science and Control'' 
of the Russian Academy of Sciences, 44-2~Vavilov\linebreak 
$\hphantom{^1}$Str., Moscow 119333, 
Russian Federation


\def\leftfootline{\small{\textbf{\thepage}
\hfill INFORMATIKA I EE PRIMENENIYA~--- INFORMATICS AND
APPLICATIONS\ \ \ 2016\ \ \ volume~10\ \ \ issue\ 4}
}%
 \def\rightfootline{\small{INFORMATIKA I EE PRIMENENIYA~---
INFORMATICS AND APPLICATIONS\ \ \ 2016\ \ \ volume~10\ \ \ issue\ 4
\hfill \textbf{\thepage}}}

\vspace*{3pt}



\Abste{The article considers using a trending Ornstein--Uhlenbeck process, driven 
by a~Levy process, for modeling financial time series. The authors demonstrate 
that the Levy driven model gives more flexibility to describe financial time series 
than the simple classical model. In particular, the Levy driven model allows 
modeling distributions with heavy tails, which is a~common property of time series 
in real applications. The authors describe efficient methods for estimating model 
parameters using such methods as OLS (ordinary least squares)
and RLS (regularized least squares). The article also solves the regime 
switching problem in a~real time data stream. The authors built an algorithm based 
on CUSUM (CUmulative SUM) methods that is capable of determining regime switches consecutively as 
they happen online and keep model parameters up to date. Solution of the regime 
switching problem is important in real applications, since the dynamics of real 
systems tend to change over time under the influence of external factors.} 

\KWE{random process; mean-reverting process; Ornstein--Uhlenbeck process driven 
by Levy process; trending Ornstein--Uhlenbeck process; regime switch; 
change point detection; CUSUM algorithm}



\DOI{10.14357/19922264160405} 

\vspace*{-16pt}

\Ack
\noindent
The research was partially supported by the Russian Foundation for Basic Research 
(project 14-07-00041).



%\vspace*{3pt}

  \begin{multicols}{2}

\renewcommand{\bibname}{\protect\rmfamily References}
%\renewcommand{\bibname}{\large\protect\rm References}

{\small\frenchspacing
 {%\baselineskip=10.8pt
 \addcontentsline{toc}{section}{References}
 \begin{thebibliography}{99}

\bibitem{1-ch-1}
\Aue{Brigo, D., A.~Dalessandro, M.~Neugebauer, and F.~Triki}. 2007. 
{A~stochastic processes toolkit for risk management}. 
London: King's College London.  Working paper. 48~p.
\bibitem{2-ch-1}
\Aue{Ornstein, L.\,S., and G.\,E.~Uhlenbeck}. 1930. On the theory of the Brownian motion. 
\textit{Phys. Rev.} 36(5):823.
\bibitem{3-ch-1}
\Aue{Vasicek, O.} 1977. An equilibrium characterization of the term structure. 
\textit{J.~Financ. Econ.} 5(2):177--188.
\bibitem{4-ch-1}
\Aue{Cox, J.\,C., E.~Ingersoll, Jr., and S.\,A.~Ross}. 1985. 
A~theory of the term structure of interest rates. \textit{Econometrica} 53(2):385--407.
\bibitem{5-ch-1}
\Aue{Garbaczewski, P., and R.~Olkiewicz}. 2000. Ornstein--Uhlenbeck--Cauchy process. 
\textit{J.~Math. Phys.} 41:6843--6860.
\bibitem{6-ch-1}
\Aue{Finlay, R.} 2009. The variance gamma (VG) model with long range dependence: 
A~model for financial data incorporating long range dependence in squared returns.
Sydney, Australia: University of Sydney, School of Mathematics and Statistics. 
 PhD Thesis. 144~p.
\bibitem{7-ch-1}
\Aue{Kuzmina, A.\,V.} 2011. Modelirovanie normal'nogo obratnogo gaussovskogo 
protsessa i~otsenivanie ego papametrov [Normal inverse Gaussian distribution 
modeling and its parameters estimation]. 
\textit{Vestnik Belorusskogo gosudarstvennogo universiteta. Ser.~1: Fizika. Matematika. 
Informatika} [Herald of the Belarusian State University. Ser.~1: 
Physics. Mathematics. Informatics] 2:133--136. 
\bibitem{8-ch-1}
\Aue{Protter, P.} 1990. 
\textit{Stochastic integration and differential equations.}  
Heidelberg: Springer-Verlag. 415~p.
\bibitem{9-ch-1}
\Aue{Sato, K.\,I.} 1999. \textit{L$\acute{\mbox{e}}$vy processes and infinitely divisible 
distributions.}  Cambridge: Cambridge University Press. 500~p.
\bibitem{10-ch-1}
\Aue{Barndorff-Nielsen, O.\,E., and N.~Shephard}. 2001. Non-Gaussian 
Ornstein--Uhlenbeck-based models and some of their uses in financial economics. 
\textit{J.~Roy. Stat. Soc.~B} 63:167--241.
\bibitem{11-ch-1}
\Aue{Taufer, E., and N.~Leonenko.} 2008. Simulation of L$\acute{\mbox{e}}$vy-driven 
Ornstein--Uhlenbeck processes with given marginal distribution. 
\textit{Comput. Stat. Data An.} 53:2427--2437.
\bibitem{12-ch-1}
\Aue{Madan, D.\,B., and E.~Seneta}. 1990. 
The VG model for share market returns. \textit{J.~Bus.} 63:511--524.
\bibitem{13-ch-1}
\Aue{Madan, D.\,B, P.\,P.~Carr, and E.\,C.~Chang}. 1998. 
The variance gamma process and option pricing. \textit{Eur. Finance Rev.} 2:79--105.
\bibitem{14-ch-1}
\Aue{Barndorff-Nielsen, O.\,E.} 1997. 
Normal inverse Gaussian distributions and stochastic volatility modeling. 
\textit{Scand. J.~Stat.} 24(1):1--13.

\bibitem{16-ch-1}
\Aue{Rydberg, H.} 1997. The normal inverse Gaussian L$\acute{\mbox{e}}$vy process: 
Simulation and approximation. \textit{Commun. Stat. Stochastic Models} 
13(4):887--910.
\bibitem{15-ch-1}
\Aue{Barndorff-Nielsen, O.\,E.} 1998. Processes of normal inverse Gaussian type. 
\textit{Financ.  Stoch.} 2:41--68.
\bibitem{17-ch-1}
\Aue{Haykin, S.} 1996. \textit{Adaptive filter theory}. 3rd ed. 
Upper Saddle River, NJ: Prentice Hall. 989~p.
\bibitem{18-ch-1}
\Aue{Appel, U., and A.\,V.~Brandt}. 1983. 
Adaptive sequential segmentation of piecewise stationary time series.  
\textit{Inform. Sci.} 29:27--56.
\bibitem{19-ch-1}
\Aue{Gustafsson, F.} 1996. The marginalized likelihood ratio test for 
detecting abrupt changes. \textit{IEEE Trans. Automat. Contr.} 41:66--78.
\bibitem{20-ch-1}
\Aue{Kay, S.} 1993. \textit{Fundamentals of statistical signal processing. Vol.~I. 
Estimation theory}.  Upper Saddle River, NJ: Prentice Hall. 625~p.
\bibitem{21-ch-1}
\Aue{Page, E.\,S.} 1961. Cumulative sum control chart. 
\textit{Technometrics} 3:1--9.
\end{thebibliography}

 }
 }

\end{multicols}

\vspace*{-6pt}

\hfill{\small\textit{Received October 20, 2016}}

\vspace*{-18pt}

\Contr

\vspace*{-2pt}

\noindent
\textbf{Chertok Andrey V.} (b.\ 1987)~--- 
junior scientist, Faculty of Computational Mathematics and Cybernetics, 
M.\,V.~Lo\-monosov Moscow State University, 1-52~Leninskiye Gory, GSP-1, Moscow 119991, 
Russian Federation; Head of R\&D, Data Science, Sberbank of Russia, 19~Vavilov Str.,
Moscow 117999, Russian Federation; \mbox{avchertok.sbt@sberbank.ru}

 \vspace*{1pt}
 
\noindent
\textbf{Kadaner Arsenii I.} (b.\ 1995)~--- 
student, Faculty of Mechanics and Mathematics, 
M.\,V.~Lomonosov Moscow State University, 
Main Building, Leninskiye Gory, GSP-1, Moscow 119991, Russian Federation; 
data scientist, Sberbank of Russia, 19~Vavilov Str., Moscow 117999, 
Russian Federation; \mbox{aikadaner.sbt@sberbank.ru}

  \vspace*{1pt}
 
\noindent
\textbf{Khazeeva Gelana T.} (b.\ 1993)~---
 student,  Faculty of Computational Mathematics and Cybernetics, M.\,V.~Lo\-monosov 
 Moscow State University, 1-52~Leninskiye Gory, GSP-1, Moscow 119991, 
 Russian Federation; \mbox{gelana.khazeyeva@gmail.com} 

 
 \vspace*{1pt}
 
\noindent
\textbf{Sokolov Igor A.} (b.\ 1954)~---
Academician of the Russian Academy of Sciences, Doctor of Science in technology, 
Director, Federal Research Center ``Computer Science and Control'' of 
the Russian Academy of Sciences, 44-2~Vavilov Str., Moscow 119333, Russian Federation; 
\mbox{isokolov@ipiran.ru}
\label{end\stat}


\renewcommand{\bibname}{\protect\rm Литература} 