\def\stat{grusho}

\def\tit{ОБ ОДНОМ СПОСОБЕ СОКРАЩЕНИЯ ВЫЧИСЛЕНИЙ ПРИ~ФОРМИРОВАНИИ 
ЗАМЫКАНИЙ ГАЛУА$^*$\\[-3pt]}

\def\titkol{Об одном способе сокращения вычислений при формировании 
замыканий Галуа}

\def\aut{А.\,А. Грушо$^1$, М.\,И. Забежайло$^2$, А.\,А.~Зацаринный$^3$\\[-11pt]}

\def\autkol{А.\,А. Грушо, М.\,И. Забежайло, А.\,А.~Зацаринный}

\titel{\tit}{\aut}{\autkol}{\titkol}

\index{Грушо А.\,А.}
\index{Забежайло М.\,И.}
\index{Зацаринный А.\,А.}
\index{Grusho A.\,A.}
\index{Zabezhailo M.\,I.}
\index{Zatsarinny A.\,A.}


{\renewcommand{\thefootnote}{\fnsymbol{footnote}} \footnotetext[1]
{Работа поддержана РФФИ (проект 15-29-07981).}}


\renewcommand{\thefootnote}{\arabic{footnote}}
\footnotetext[1]{Институт проблем информатики Федерального исследовательского центра <<Информатика и~управление>> 
Российской академии наук, \mbox{grusho@yandex.ru}}
\footnotetext[2]{Институт проблем информатики Федерального исследовательского центра <<Информатика и~управление>> 
Российской академии наук, \mbox{m.zabezhailo@yandex.ru}}
\footnotetext[3]{Институт проблем информатики Федерального исследовательского центра <<Информатика и~управление>> 
Российской академии наук, \mbox{alex250451@mail.ru}}

 \vspace*{-12pt}
     
  \Abst{Уточнение понятия сходства как алгебраической операции~--- процедурная 
основа многих современных методов интеллектуального анализа данных (ИАД). Однако в~ряде 
важных приложений (в~частности, при управлении информационными потоками 
в~компьютерных сетях, обеспечении информационной безопасности (ИБ) в~облачных средах 
и~др.)\ применение подобной математической техники оказывается ограниченным 
объемами вычислений при обработке больших объемов данных в~режиме реального 
времени. Пример~--- многократные вычисления пересечений множеств булевских строк 
большой длины в~процессе поиска неподвижных точек замыканий Галуа. В~статье 
предложен алгоритм формирования замыканий Галуа, позволяющий ускорить проверку 
замкнутости множеств. Приведены примеры использования предлагаемой техники в~ряде 
задач управления потоками сообщений в~компьютерных сетях и~контроля содержимого 
пересылаемых пакетов.}
  
\KW{интеллектуальный анализ данных; сходство как алгебраическая операция; методы 
сокращения перебора; скорость управления и~безопасность информационных потоков 
в~компьютерных сетях}

\DOI{10.14357/19922264160410} 

\vspace*{-4pt}


\vskip 10pt plus 9pt minus 6pt

\thispagestyle{headings}

\begin{multicols}{2}

\label{st\stat}

\section{Общее представление о~рассматриваемой проблеме}

\vspace*{-5pt}

    Искусственный интеллект как область исследований сегодня наиболее 
ярким образом представлен разнообразием математических моделей, 
методов и~компьютерных систем так называемого \textit{интеллектуального 
анализа данных}. В~компьютерном анализе данных как целом ИАД 
можно вы-\linebreak делить <<инструментально>> как область анализа\linebreak данных, 
выполняемого средствами \textit{интеллектуальных компьютерных систем}. 
Основная отличительная особенность таких систем~--- использование 
\textit{формальных моделей рассуждений}, в~том числе\linebreak корректных 
\textit{формализаций эвристик} применяемых специализирующимися на 
анализе данных экспертами (например, выделение областей математически 
корректного использования соответствующих эвристик и~контроль 
<<невыхода>> за границы таких областей в~процессе ИАД, выполняемого 
реализующими эти эвристики <<инструментальными>> средствами).
    
    Характерным для ИАД стало оценивание целевых параметров по 
косвенным признакам (см., например,~[1--4]). При этом фактически постулируется 
принцип: \textit{сходство в~описаниях} прецедентов \textit{влечет сходство 
в~описаниях анализируемых} (по <<косвенным>> данным) \textit{целевых 
свойств} изучаемых прецедентов. Так в~центре внимания оказались 
формализация и~анализ сходства как математически корректной 
конструкции. В~рамках ИАД предложены и~активно развиваются различные 
варианты уточнения понятия сходства~--- метрики, меры близости, 
формализация сходства как алгебраической операции и~др.
    
    При вовлечении в~ИАД данных комплексной <<природы>> (в~том 
числе в~работе с~нечисловыми объектами, свойства которых определяются 
как\linebreak особенностями их внутренней структуры, так и~принимающими 
чис\-ло\-вые значения\footnote[4]{Например, это могут быть физиологически активные 
химические соединения, целевые (в частности~--- \textit{лекарственные}) свойства которых 
определяются как их структурными особенностями, так и~целым рядом дополнительных 
па\-ра\-мет\-ров, описываемых определенными числовыми значениями (в~том числе  дозами, 
концентрациями, температурами и~т.\,п.).} па\-ра\-мет\-рами) достаточно эффективным 
оказывается использование уточнения сходства как \textit{алгебраической 
операции}. Ключевым элементом задействованной здесь формальной 
математической конструкции оказывается поиск \textit{всех тех} 
и~\textit{только тех \mbox{объектов}}, которые удовлетворяют заданному набору 
условий, т.\,е.\ использование специального \textit{оператора замыкания} 
и~поиска неподвижных точек некоторой специальной версии замыкания 
Галуа~[5]. Своего рода <<платой>> за богатые выразительные возможности 
применяемой математической техники оказывается необходимость 
оперировать достаточно сложными комбинаторными объектами~--- 
по\-рож\-да\-емы\-ми в~процессе вычисления сходств диаграммами взаимной 
вложимости классов сходства и~классов эквивалентности. Именно с~этим 
связан пристальный интерес специалистов к~любым возможностям 
<<\textit{ускорения}>> вычислений (за счет тех или иных \textit{сокращений} их 
\textit{объемов}) при многократно выполняемом в~процессе ИАД поиске 
сходств.
    
    Процедурные конструкции рассматриваемого типа демонстрируют свою 
эффективность и~в~задачах обеспечения 
ИБ. Так, например, в~ряде технологий DPI (Deep Packet Inspection), IDS/IPS\linebreak 
(Intrusion Detection/Protection Systems) и~др.\ критически важным 
компонентом оказывается выделение эмпирических (извлекаемых в~процессе 
ИАД из описаний инцидентов) зависимостей, которые неявным образом 
представлены в~накапливаемых данных. Цель порождения таких 
зависимостей~--- <<обучение>> НОРМЕ и~идентификация АНОМАЛИЙ 
в~поведении объекта зашиты. Выделяемые в~процессе обучения комбинации 
признаков дают\linebreak возможность оперативно опознавать <<аналогичные>> 
явления (АНОМАЛИИ) во вновь анализиру\-емых данных. При этом скорость 
анализа данных\linebreak в~задачах обеспечения ИБ (например, облачных сред или же 
обработки так называемых Big Data) оказывается критически важным 
фактором успешной работы соответствующих средств защиты. 
В~практически значимых приложениях в~первую очередь именно скоростью 
ведения ИАД определяются возможности обрабатывать в~режиме реального 
времени огромные коммутационные таблицы\linebreak в~устройствах управления 
трафиком в~ком\-пью\-тер\-ных сетях, все более объемные реестры %\linebreak 
сигнатур вирусов и~т.\,п. Представленная  
<<\textit{ин\-тер\-по\-ля\-ци\-он\-но-экс\-тра\-по\-ля\-ци\-он\-ная}>> 
(порождения и~<<переноса>> эмпирических зависимостей на описания 
новых ИБ-ин\-ци\-ден\-тов) техника <<диагностики>> АНОМАЛИЙ 
фактически и~есть один из механизмов ИАД, привнесенных в~проблематику 
ИБ исследованиями в~об\-ласти искусственного интеллекта (в~том числе 
разработки и~экспериментального исследования интеллектуальных 
компьютерных систем).
{\looseness=-1

}

\vspace*{-8pt}
    
\section{Базовые понятия и~определения}

\vspace*{-3pt}
    
    Пусть заданы два множества: $\mathbf{U} \hm= \{a_1, a_2, \ldots, 
a_n\}$ и~$\boldsymbol{\Omega}\hm= \{O_1, O_2, \ldots, O_m\}\subseteq 2^{\mathbf{U}}\backslash 
\varnothing$, 
 первое из которых будем называть исходным алфавитом 
(множеством образующих элементов для анализируемых прецедентов 
из~$\Omega$), а~второе~--- множеством описаний прецедентов (непустых 
множеств образующих), построенных над универсумом~$\mathbf{U}$, т.\,е.\ 
над множеством объектов, построенных из образующих $a_1, a_2, \ldots, 
a_n$.
    
    Располагая множествами $\mathbf{U}$ и~$\boldsymbol{\Omega}$, определим два 
отображения~$f$ и~$\varphi$:
       \begin{itemize}
   \item 
$\forall\ \xi \in \mathbf{2}^{\mathbf{U}}$ (т.\,е.\ для каждого $\xi$~--- 
подмножества образующих из множества~$\mathbf{U}$)
$f(\xi)$\;=\;\{множество всех таких~$O_i$ из~$\boldsymbol{\Omega}$, что $\forall\ 
a_j\hm\in \xi$ имеет место $a_j\hm\in O_i$ для каждого из этих 
выбранных~$O_i$ (т.\,е.\ это множество всех прецедентов 
из~$\boldsymbol{\Omega}$, в~которые все~$a_j$ из заданного~$\xi$ входят 
одновременно)\};\\[-15pt]

\item
$\forall\ \zeta \hm\in \mathbf{2}^{\boldsymbol{\Omega}}$ (т.\,е.\ для каждого~$\zeta$~--- 
подмножества прецедентов из множества~$\boldsymbol{\Omega}$)
$\varphi(\zeta)$\;=\;\{множество всех таких~$a_j$ из~$\mathbf{U}$, что 
$\forall\ O_i\hm\in \zeta$ имеет место $a_j\hm\in O_i$ для каждого из 
этих выбранных~$a_j$ (т.\,е.\ это множество всех таких~$a_j$ 
из~$\mathbf{U}$, которые во все прецеденты из~$\zeta$ входят 
одновременно)\}, \\[-15pt]
\end{itemize}
где
$\mathbf{2}^{\mathbf{U}}$~--- множество всех подмножеств 
множества~$\mathbf{U}$ (исходного алфавита), а~$\mathbf{2}^{\boldsymbol{\Omega}}$~--- 
множество всех подмножеств множества~$\boldsymbol{\Omega}$ (множества описаний 
прецедентов).

    Можно показать, что пара отображений $\langle f,\varphi\rangle$ 
представляет собой соответствие Галуа (см., например,~[5]), а~их 
произведения $f(\varphi(\zeta))$  и~$\varphi (f(\zeta))$~--- соответствующие 
замыкания Галуа (см., например,~[5]), которые будем обозначать как 
$[\_]_{\mathbf{U},\boldsymbol{\Omega}}$ и~$[\_]_{\boldsymbol{\Omega},\mathbf{U}}$. 
    
    \smallskip
    
    \noindent
    \textbf{Определение~1.}\ Неподвижными точками замыкания Галуа 
$[\_]_{\mathbf{U},\boldsymbol{\Omega}}$ будем называть все такие 
$[{X}]_{\mathbf{U},\boldsymbol{\Omega}}$, что 
$[{X}]_{\mathbf{U},\boldsymbol{\Omega}} \hm= {X}$.
    
    Посредством $\mathbf{GC}_{f,\varphi} (\Omega)$ 
    и~$\mathbf{GC}_{\varphi,f} (\mathbf{U})$ будем обозначать множества 
неподвижных точек соответствующих замыканий Галуа: 
$\mathbf{GC}_{f,\varphi} (\boldsymbol{\Omega}) \hm= 
\{\zeta\hm\in \mathbf{2}^{\boldsymbol{\Omega}}$ таких, 
что $f(\varphi (\zeta)) \hm= \zeta\}$ и~$\mathbf{GC}_{\varphi,f} 
(\mathbf{U}) 
\hm= \{\xi\hm\in \mathbf{2}^{\mathbf{U}}$ таких, что $\varphi (f(\xi)) \hm= 
\zeta\}$. Каждое из этих множеств можно рассматривать как частично 
упорядоченное в~соответствии со взаимным вложением как 
соответствующих множеств прецедентов (подмножеств для~$\boldsymbol{\Omega}$), так 
и~подмножеств образующих из~$\mathbf{U}$.
    
    Таким образом, для заданного множества образующих 
$\mathbf{u}\hm= \{u_1, u_2, \ldots, u_s\}$, где $\mathbf{u}\subseteq 
\mathbf{U}$, проверку его замкнутости (относительно заданного 
множества прецедентов~$\boldsymbol{\Omega}$, представленных 
в~алфавите~$\mathbf{U}$) может выполнить следующий
    
    \smallskip
    
    \noindent
    \textbf{Алгоритм~1}
        \begin{description}
\item[Входные данные:] множества $\mathbf{U}$, $\Omega$ 
и~$\mathbf{u}$.

\item[ШАГ~1.] Для каждого~$i$ от~1 до~$s$ ($s\hm\leq n$) в~множестве~$\Omega$ 
описаний прецедентов выделяется подмножество~$\Omega_i$ таких, что 
образующая~$u_i$ входит в~описание каждого из прецедентов~$O_j$, 
входящих в~соответствующее подмножество~$\Omega_i$.

\item[ШАГ~2.] Вычисляется $\mathbf{u}^*$~--- общая часть (пересечение 
соответствующих множеств образующих) всех~$\Omega_i$ (по~$i$ от~1 
до~$s\hm\leq n$).

\item[ШАГ~3.] Проверяется выполнимость равенства $\mathbf{u}^*\hm= \mathbf{u}$. При 
положительном исходе~$\mathbf{u}$ замкнуто, в~противном случае~--- 
нет.

\item[ШАГ~4.] СТОП.
\end{description}

\smallskip

    К сожалению, для больших значений па\-ра\-мет\-ра~$m$ (числа описаний 
прецедентов в~множестве~${\boldsymbol{\Omega}}$) при <<массовой>> проверке 
замкнутости большого числа <<тестируемых>> множеств~$\mathbf{u}$
(важный для приложений пример такой ситуации дает характерная для 
управления потоками данных в~компьютерных сетях задача поиска 
заголовка пакета в~таблице коммутации, где необходимо в~режиме реального 
времени оперировать таб\-ли\-ца\-ми коммутации, содержащими десятки, 
а~в~ряде случаев и~сотни тысяч строк) алгоритм~1 оказывается весьма 
ресурсоемким (в~част\-ности, шаг~2 требует массовых операций с~булевскими 
последовательностями длины~$m$). Этим обусловлен интерес к~построению 
тех или иных сравнительно (с~<<прямой>> проверкой) быст\-ро вычислимых 
локальных условий (где, например, было бы достаточно выявления 
совместного вхож\-де\-ния пары образующих~$u_i$, $u_j$ из 
алфавита~$\mathbf{U}$, <<непопадание>> одной из которых 
в~порождаемое замыкание~$\mathbf{u}^*$ сигнализировало бы 
о~незамкнутости текущего тестируемого множества 
образующих~$\mathbf{u}$). Один из вариантов подобного достаточного 
условия незамкнутости дает
    
    \smallskip
    
    \noindent
    \textbf{Утверждение~1.}\ Для того чтобы заданное подмножество 
$\mathbf{u}\hm= \{u_1, u_2, \ldots, u_s\}$ образующих из 
алфавита~$\mathbf{U}$ было незамкнуто относительно множества 
описаний прецедентов~${\boldsymbol{\Omega}}$, достаточно, чтобы для замыканий 
одноэлементных подмножеств~$\mathbf{u}$ выполнялось строгое 
включение: 
    $$
\mathbf{u}= \{u_1, u_2, \ldots, u_s\} \subset 
\mathop{\bigcup}\limits_{i=1}^s \left[ 
\left\{ u_i\right\}\right]_{\mathbf{U},{\boldsymbol{\Omega}}}\,.
    $$
  
  К сожалению, представленное утверждением~1 условие незамкнутости не 
является необходимым, в~чем нетрудно убедиться, принимая во внимание
  
  \smallskip
  
  \noindent
  \textbf{Пример~1.}\ Рассмотрим множество булевских векторов длины~4: 
$$
\mathbf{B} = \{\langle0000\rangle, \langle0001\rangle, \ldots, 
\langle1110\rangle, \langle1111\rangle\}\,.
$$
 Положим $\mathbf{U}
\hm=  \{a^0_1, 
a^1_1, a^0_2, a^1_2, a^0_3, a^1_3, a^0_4, a^1_4\}$, а~множество описаний прецедентов 
${\boldsymbol{\Omega}}\hm= \{O_1, O_2,\ldots$\linebreak $\ldots, O_{15}\}$ сформируем из 
первых~15~элементов\linebreak множества~$\mathbf{B}$ по следующему правилу: 
в~каждое множество $O_j \hm= \{a_{j1}, a_{j2}, a_{j3}, a_{j4}\}$ в~качестве 
элемента~$a_{jr}$ берется~$a^0_r$, если в~$i$-м элементе 
множества~$\mathbf{B}$ на позиции номер~$r$ стоит~0, в~противном 
случае (когда в~этой позиции в~$i$-м элементе множества~$\mathbf{B}$
находится единица) на соответствующую позицию в~$O_j$ берем 
образующую~$a^1_r$. Несложно убедиться, что для последнего, 16-го 
элемента $\langle1111\rangle$ множества~$\mathbf{B}$ соответствующее 
множество $O_{16}\hm=  \{a^1_1, a^1_2, a^1_3, a^1_4\}$, не входящее в~${\boldsymbol{\Omega}}$, 
тем не менее, удовлетворяет условию:
  $$
  O_{16}= \left\{ a^1_1, a^1_2, a^1_3, a^1_4\right\} = \mathop{\bigcup}\limits_{i=1}^4 \left[ 
\left\{a^1_i\right\}\right]_{\mathbf{U},{\boldsymbol{\Omega}}}\,.
  $$
  
\section{Процедура формирования замыкания и~ее~свойства}

    Пусть заданы: множество примеров (описаний прецедентов) 
${\boldsymbol{\Omega}}\hm= \{O_1, O_2, \ldots, O_m\}$, построенных на алфавите 
$\mathbf{U}\hm= \{a_1,a_2, \ldots, a_n\}$, и~некоторое непустое 
подмножество $\mathbf{u}\hm= \{a(u)_1, a(u)_2, \ldots, a(u)_k\}$ 
исходного~$\mathbf{U}$. Пусть~$\alpha$, $\alpha_1$ и~$\alpha_2$~---
перестановки индексов $\{1, 2, \ldots, s\}$, а~$\mathbf{u}_\alpha\hm= \langle 
u_{i1}, u_{i2}, \ldots, u_{is}\rangle$~--- упорядочение входящих 
в~множество~$\mathbf{U}$ обра\-зу\-ющих в~соответствии с~их перечислением 
в~$\alpha$. 
    \smallskip
    
\noindent
\textbf{Алгоритм~2}
\begin{description}
\item[Входные данные:] множества $\mathbf{U}$, ${\boldsymbol{\Omega}}$ 
и~$\mathbf{u}$, 
а~также перестановка~$\alpha$, определяющая порядок рас\-смот\-ре\-ния 
образующих в~тестовом множестве~$\mathbf{u}$.

\item[ШАГ~1.] Выбираем в~текущем~$\mathbf{u}$ в~соответствии с~$\alpha$ 
первый элемент~---~$u_1$.

\item[ШАГ~2.] Строим по $u_1$, ${\boldsymbol{\Omega}}$ и~$\mathbf{U}$ множество 
$[\{u_1\}]_{\mathbf{U},{\boldsymbol{\Omega}}}$.

\item[ШАГ~3.] Выделяем в~${\boldsymbol{\Omega}}$ подмножество~${\boldsymbol{\Omega}} 
(u_1)$ всех~$O_{u1}$, 
содержащих $[\{u_1\}]_{\mathbf{U},{\boldsymbol{\Omega}}}$.

\item[ШАГ~4.] Удаляем из каждого входящего в~${\boldsymbol{\Omega}} (u_1)$ примера 
образующие из $[\{u_1\}]_{\mathbf{U},{\boldsymbol{\Omega}}}$ и~формируем (этими 
удалениями) новое~--- модифицированное~--- множество~${\boldsymbol{\Omega}}^*(u_1)$. 
Получаем <<усечен\-ное>> множество примеров~${\boldsymbol{\Omega}}^*(u_1)$ 
и~использованный при его формировании <<усеченный>> 
алфавит~$\mathbf{U}(u_1)$.

\item[ШАГ~5.] Удаляем из текущего~$\mathbf{u}$ все образующие, вошедшие 
в~$[\{u_1\}]_{\mathbf{U},{\boldsymbol{\Omega}}}$.

\item[ШАГ~6.] Если в~текущем~$\mathbf{u}$ уже нет (неудаленных) образующих, 
переходим на шаг~7. Иначе заменяем текущее~${\boldsymbol{\Omega}}$ на вновь 
построенное множество~${\boldsymbol{\Omega}}^*(u_1)$, а~текущее~$\mathbf{U}$~--- на 
<<усеченный>> алфавит, сформированный на шаге~4, и,~наконец, заменяем 
имевшееся на шаге~1 множество~$\mathbf{u}$~--- на его текущую (см.\ 
шаг~5) <<усеченную>> версию, после чего переходим на шаг~1.

\item[ШАГ 7.] СТОП.
\end{description}

    Будем обозначать последовательность $u_1, u_2, \ldots, u_l$ (где $l \hm\leq 
k$) формирующих $\mathbf{u}$ элементов исходного~$\mathbf{U}$, 
удаляемых на каждом из шагов предпринимаемой рекурсии, как 
$T_\alpha(\mathbf{u},{\boldsymbol{\Omega}},\mathbf{U})$ и~называть ее задаваемой 
упорядочением $\alpha$ \textit{траекторией} <<\textit{усечения}>> 
проверяемого на замкнутость множества~$\mathbf{u}$.
    
    Сократить объем вычислений алгоритма~2 можно, выбирая при 
порождении <<сужений>>~${\boldsymbol{\Omega}}^*$ из имеющихся вариантов тот, где 
минимизируется размер~${\boldsymbol{\Omega}}^*$. Для этого достаточно рассмотреть на 
каждом шаге <<сужений>> замыкания всех одноэлементных подмножеств 
текущего алфавита. Максимальным по размеру элементам такого множества 
замыканий соответствуют минимальные по размеру подмножества исходного 
множества~${\boldsymbol{\Omega}}$. 
    
    \medskip
    
\noindent
\textbf{Алгоритм~3}

\begin{description}
\item[Входные данные:] множества $\mathbf{U}$, ${\boldsymbol{\Omega}}$ и~$\mathbf{u}$.

\noindent
ШАГ~1. Строим по ${\boldsymbol{\Omega}}$ и~$U$ множество замыканий всех 
одноэлементных подмножеств алфавита~$U$: 
$\mathbf{B}\_{\mathbf{GC}}_{s^*g}
(\mathbf{U},{\boldsymbol{\Omega}})\hm = \{[\{a_1\}], 
[\{a_2\}], \ldots, [\{a_n\}]\}$.

\item[ШАГ~2.] Выбираем в~$\mathbf{u}$ образующую (обозначим ее~$u_1$), 
которой соответствует максимальный элемент 
в~$\mathbf{B}\_\mathbf{GC}_{s^*g}(\mathbf{U},{\boldsymbol{\Omega}})$.

\item[ШАГ~3.] Выделяем в~${\boldsymbol{\Omega}}$ 
подмножество ${\boldsymbol{\Omega}} (u_1)$ всех~$O_{u1}$, 
содержащих $[u_1]_{\mathbf{U},{\boldsymbol{\Omega}}}$.

\item[ШАГ~4.] Удаляем из каждого входящего в~${\boldsymbol{\Omega}} (u_1)$ примера 
образующие из~$[u_1]_{\mathbf{U},{\boldsymbol{\Omega}}}$, формируя модифицированное 
множество примеров~${\boldsymbol{\Omega}}^*(u_1)$. Результат~--- <<усеченное>> 
множество примеров ${\boldsymbol{\Omega}}^*(u_1)$ и~<<усеченный>> алфавит 
$\mathbf{U}(u_1)$.

\item[ШАГ~5.] Удаляем из текущего~$\mathbf{u}$ все образующие, вошедшие 
в~$[u_1]_{\mathbf{U},{\boldsymbol{\Omega}}}$.

\item[ШАГ~6.] Если в~текущей версии множества~$\mathbf{u}$ больше не осталось 
еще не удаленных образующих, переходим на шаг~7. В~противном случае 
заменяем текущее~${\boldsymbol{\Omega}}$ на вновь построенное множество 
${\boldsymbol{\Omega}}^*(u_1)$, а~текущее множество~$\mathbf{U}$~--- на 
<<усеченный>> алфавит, сформированный на шаге~4, после чего переходим 
на шаг~1 (т.\,е.\ возвращаемся к~построению 
$\mathbf{B}\_\mathbf{GC}_{s^*g}$, но теперь уже на <<усеченных>> 
${\boldsymbol{\Omega}}^*(u^\prime)$ и~$\mathbf{U}(u^\prime)$, т.\,е.\ переходим 
к~построению множества замыканий всех одноэлементных подмножеств 
<<усеченного>> алфавита~$\mathbf{U}(u^\prime)$ относительно 
<<усеченного>> множества примеров~${\boldsymbol{\Omega}}^*(u^\prime)$, где текущее 
значение~$u^\prime$ получено удалением из текущего~$\mathbf{u}$ всех 
образующих из текущего $[u_1]_{\mathbf{U},{\boldsymbol{\Omega}}})$. 

\item[ШАГ~7.] СТОП.

\end{description}


    Возможности представления средствами алгоритмов~2 и~3 
<<составных>> (многоэлементных) неподвижных точек рассматриваемого 
замыкания Галуа с~помощью замыканий одноэлементных подмножеств 
алфавита~$\mathbf{U}$ демонстрирует 
    
    \smallskip
    
    \noindent
    \textbf{Утверждение~2}\ (\textit{Теорема о~представлении}). Множество 
образующих~$\mathbf{u}$ замкнуто относительно множества 
примеров~${\boldsymbol{\Omega}}$ в~алфавите~$\mathbf{U}$ тогда и~только тогда, когда 
объединение всех замыканий $[u^\prime]_{\mathbf{U}^*,{\boldsymbol{\Omega}}^*}$ вдоль 
сформированной траектории $T(\mathbf{u}, {\boldsymbol{\Omega}}, \mathbf{U})$ 
совпадает с~$\mathbf{u}$: 
    $$
    [\mathbf{u}]_{\mathbf{U},{\boldsymbol{\Omega}}} =
    \mathbf{u}
    $$ 
тогда и только тогда, когда
$$ 
\mathbf{Un}(T) = \left ( 
\mathop{\bigcup}\limits_{u^\prime_i\in T(\mathbf{u},{\boldsymbol{\Omega}}, \mathbf{U})} \left[ 
u^\prime_i\right]\right) = 
\mathbf{u}\,.
    $$

\noindent
Д\,о\,к\,а\,з\,а\,т\,е\,л\,ь\,с\,т\,в\,о\,.\ \ Из-за ограниченности объема данной 
публикации приведем лишь ключевые элементы используемой при 
доказательстве утверж\-де\-ния~2 системы аргументов.
\begin{enumerate}[(i)]
\item Можно показать, что (вследствие ассоциативности операции 
пересечения множеств) для любых упорядочений~$\alpha_1$ и~$\alpha_2$ 
образу\-ющих из~$\mathbf{u}$ в~случае, когда 
замыкание~$\mathbf{u}$
относительно~$\mathbf{U}$ и~${\boldsymbol{\Omega}}$ не пусто 
(т.\,е.\ $\varnothing\not= 
[\mathbf{u}]_{\mathbf{U},{\boldsymbol{\Omega}}} 
\hm\subseteq \mathbf{U}$), результаты 
вычисления алгоритмом~2 по каждому из этих упорядочений~--- по 
порождаемым ими траекториям $T_1(\mathbf{u}, 
{\boldsymbol{\Omega}}, \mathbf{U})$ 
и~$T_2(\mathbf{u}, {\boldsymbol{\Omega}}, 
\mathbf{U})$), совпадают.
\item Если тестовое~$\mathbf{u}$ не входит ни в~один из элементов 
исходного~${\boldsymbol{\Omega}}$, то~$\mathbf{u}$ не замкнуто и~его нет 
в~множестве $\mathbf{Dom}({\boldsymbol{\Omega}})$ всех пересечений примеров 
из~${\boldsymbol{\Omega}}$. При этом найдется $u_0\hm\in \mathbf{u}$ такая, что 
в~соответствующем ей <<усечении>>~${\boldsymbol{\Omega}}^*$ исходного~${\boldsymbol{\Omega}}$ 
замыкание~$u_0$ окажется пустым множеством: 
$[\{u_0\}]_{\mathbf{U}^*, {\boldsymbol{\Omega}}^*}\hm=\varnothing$.
\item Если тестовое~$\mathbf{u}$ строго вкладывается ровно в~один 
(пусть это будет~$O_{\mathbf{u}}$) из элементов исходного~${\boldsymbol{\Omega}}$ (т.\,е.\ 
$\mathbf{u}\hm\subset O_{\mathbf{u}}$), то $\mathbf{u}$ не замкнуто 
и~$[\mathbf{u}]_{\mathbf{U},{\boldsymbol{\Omega}}} \hm= O_{\mathbf{u}} \hm= \mathbf{Un}(T)$. При 
совпадении~$\mathbf{u}$ ровно с~одним (пусть это будет~$O_{\mathbf{u}}$) из 
элементов исходного множества примеров~${\boldsymbol{\Omega}}$ 
данное~$\mathbf{u}$ замкнуто, 
а~$[\mathbf{u}]_{\mathbf{U},{\boldsymbol{\Omega}}}\hm 
= O_{\mathbf{u}}$.
\item Если тестовое~$\mathbf{u}$ вкладывается по крайней мере 
в~два элемента исходного~${\boldsymbol{\Omega}}$, то 
$[\mathbf{u}]_{\mathbf{U},{\boldsymbol{\Omega}}}$ лежит в~множестве 
$\mathbf{Dom}({\boldsymbol{\Omega}})$ всех пересечений примеров из~${\boldsymbol{\Omega}}$.
\item Четыре возможные ситуации описывают взаимосвязи 
заданных~$\mathbf{u}$ и~${\boldsymbol{\Omega}}$:
\begin{enumerate}[(1)]
\item $\mathbf{u}$ не входит ни в~один из элементов~${\boldsymbol{\Omega}}$.

Следуя~(ii), в~данном случае $\mathbf{u}$ не замк\-ну\-то, а~по~(iii) 
имеет место строгое вложение $\mathbf{Un}(T) \hm\subset \mathbf{u}$, 
т.\,е.\ условия $([\mathbf{u}]_{\mathbf{U},{\boldsymbol{\Omega}}} \not= 
\mathbf{u}) \& 
(\mathbf{Un}(T) \hm= \mathbf{u})$ и
$([\mathbf{u}]_{\mathbf{U},{\boldsymbol{\Omega}}}\hm = 
\mathbf{u}) \& (\mathbf{Un}(T) 
\not= \mathbf{u})$ в~данном случае выполняться не могут;
    \item  $\mathbf{u}$ как собственное подмножество входит ровно 
в~один из элементов~${\boldsymbol{\Omega}}$.
    
Следуя~(iii), имеем $\mathbf{Un}(T)\hm= O_{\mathbf{u}}$, т.\,е.\ условия 
$([\mathbf{u}]_{\mathbf{U},{\boldsymbol{\Omega}}} \not= 
\mathbf{u}) \& 
(\mathbf{Un}(T)\hm = \mathbf{u})$ 
и~$([\mathbf{u}]_{\mathbf{U},{\boldsymbol{\Omega}}}\hm = 
\mathbf{u}) \& 
(\mathbf{Un}(T)\not= \mathbf{u})$ в~данном случае также выполняться не 
могут;
    \item  $\mathbf{u}$ входит ровно в~один из элементов~${\boldsymbol{\Omega}}$, 
совпадая с~этим примером.
    
Следуя~(iv), множество $\mathbf{u}$ замкнуто: 
$[\mathbf{u}]_{\mathbf{U},{\boldsymbol{\Omega}}} \hm= O_{\mathbf{u}}$. По аналогии с~(iii) 
нетрудно убедиться, что в~данном случае также имеет место равенство 
$\mathbf{Un}(T)\hm= O_{\mathbf{u}}$, т.\,е.\ формулируемый утверждением~2 критерий 
выполнен; 
    \item  $\mathbf{u}$ входит по крайней мере в~пару 
элементов~${\boldsymbol{\Omega}}$.

С~учетом~(iv) и~алгоритма~1 здесь $[\mathbf{u}]_{\mathbf{U},{\boldsymbol{\Omega}}}$ 
принадлежит множеству $\mathbf{Dom}({\boldsymbol{\Omega}})$.
    \end{enumerate}
    \end{enumerate}

\noindent
$\Rightarrow$. Пусть $\mathbf{u}$ замкнуто, тогда 
$\mathbf{u}\hm= 
\mathop{\cup}\limits_{i=1}^k [\{a_i\}]$ и~каж\-дая~$u_i$ из $\mathbf{u}$ входит 
в~результирующее объединение $\mathbf{Un}(T)$; кроме того, каждое 
$[\{a_i\}]$ из множества $\{[\{u_1\}], [\{u_2\}], \ldots, [\{u_k\}]\}$ не выходит за 
пределы множества~$\mathbf{u}$. Таким образом, выполняется 
и~равенство $\mathbf{u}\hm= \mathbf{Un}(T)$. 

    Пусть, наоборот, $\mathbf{u}$ не замкнуто, тогда в~данном случае 
найдутся $u_{01}\hm\in\mathbf{u}$ и~$u_{02}\hm\in 
([\mathbf{u}]_{\mathbf{U},{\boldsymbol{\Omega}}}\backslash 
\mathbf{u})$ такие, что 
$u_{02}\hm\in [\{u_{01}\}]_{\mathbf{U},{\boldsymbol{\Omega}}}$, иначе 
$\mathbf{u}$
здесь не может быть незамкнутым. Рассмотрим замыкание 
$[\{u_{01}\}]_{\mathbf{U}^*,{\boldsymbol{\Omega}}^*}$ в~соответствующем 
<<усечении>>~${\boldsymbol{\Omega}}^*$ исходного~${\boldsymbol{\Omega}}$. Как следствие~(i) имеем 
$u_{02}\hm\in [\{u_{01}\}]_{\mathbf{U}^*,{\boldsymbol{\Omega}}^*}$, т.\,е.\ 
$\mathbf{u}\subset \mathbf{Un}(T)$, так как $\mathbf{u}$ не 
содержит~$u_{02}$, а~$\mathbf{Un}(T)$, наоборот, содержит~$u_{02}$. Итак, 
в~рассматриваемой ситуации формула 
$([\mathbf{u}]_{\mathbf{U},{\boldsymbol{\Omega}}}\not= 
\mathbf{u})$ \& 
$(\mathbf{Un}(T)\hm=\mathbf{u})$ не выполнима.
    
\noindent
$\Leftarrow$. Пусть $\mathbf{u}\subset \mathbf{Un}(T)$, т.\,е.\ 
в~$\mathbf{Un}(T)$ содержится некоторая образующая~$u_{02}$, которая 
не входит в~$\mathbf{u}$: $u_{02}\hm\notin \mathbf{u}$. Тогда среди 
образующих~$u^\prime_i$, замыкаемых в~<<усечениях>>~${\boldsymbol{\Omega}}^*$ 
исходного~${\boldsymbol{\Omega}}$, найдется такая~$u_{01}$, что $u_{02}\hm\in 
[\{u_{01}\}]_{\mathbf{U}^*,{\boldsymbol{\Omega}}^*}$~--- ее замыкание относительно 
соответствующего~${\boldsymbol{\Omega}}^*$ содержит~$u_{02}$. Тогда, следуя~(i), 
в~множестве~$\mathbf{Dom}({\boldsymbol{\Omega}})$ можно указать 
некоторое~$\mathbf{u}^*$, которое замкнуто относительно 
исходного~${\boldsymbol{\Omega}}$ и~в~которое входят не только все элементы 
из~$\mathbf{u}$, но также дополнительно и~$u_{02}$. т.\,е.\ $\mathbf{u}$ 
в~данном случае не замкнуто (этому <<мешает>>~$u_{02}$), а~формула 
$(\mathbf{Un}(T)\not= \mathbf{u})$ \& 
$([\mathbf{u}]_{\mathbf{U},{\boldsymbol{\Omega}}}=\mathbf{u})$ не выполнима.

\section{Примеры использования предлагаемого подхода}

\vspace*{-2pt}

    Рассмотрим несколько примеров использования предложенной техники 
при решении задач\linebreak управления информационными потоками и~обеспечения 
ИБ в~компьютерных сетях. Контроль\linebreak 
и~управ\-ле\-ние информационными потоками в~об\-лач\-ной среде~--- важные 
механизмы ИБ~[6]. Клю\-чевая проблема 
в~реализации таких механизмов~--- скорость коммутации пакетов 
в~соответствии\linebreak с~заданными правилами. Критичная <<техническая>>
характеристика здесь~--- это скорость поиска в~соответствующей таблице 
коммутации (КТ) заголовка пакета, поступающего на вход коммутатора. 
С~формальной точки зрения это скорость поиска предъявляемого КТ 
заголовка пакета (булевского вектора фиксированной длины) в~множестве 
строк этой таблицы (в~реальных приложениях содержащем десятки, а~иногда и~сотни тысяч заголовков). Массовый приход пакетов и~необходимость 
обрабатывать их в~режиме про\-цес\-сно-ре\-аль\-но\-го времени (в~том числе 
в~условиях внесения динамических изменений в~текущее состояние КТ) 
делают эту область весьма чувствительной к~любым возможностям сократить 
трудоемкость соответствующих процедур проверки.
    
    Примеры 2 и~3 демонстрируют возможности предлагаемой процедурной 
схемы. При этом каж\-дый заголовок входного пакета (булевский вектор 
фиксированной длины), так же как и~каждая строка соответствующей КТ по 
схеме из примера~1, могут быть взаимно однозначным образом 
представлены в~виде множеств процедурой кодирования~0 и~1 
в~соответствующих разрядах отдельными переменными (образующими 
соответствующего алфавита).
    
    \smallskip
    
    \noindent
    \textbf{Пример~2.}\ Пусть дано множество примеров ${\boldsymbol{\Omega}}\hm = \{O_1, 
O_2, \ldots, O_8\}$ (табл.~1). Требуется проверить, имеется ли 
в~${\boldsymbol{\Omega}}$ пример~$O_9$, которому может быть сопоставлен вектор 
(заголовок пакета) $\langle01110\rangle$? Для этого на первом шаге выберем 
столбец для <<усечения>> табл.~1, подсчитав число нулей и~единиц в~ее 
столбцах (табл.~2). Представляется естественным выбрать столбец номер~3 
(на котором может быть выделено минимальное число строк табл.~1, 
имеющих тот же показатель в~столбце номер~3, что и~для~$O_9$). Таким 
образом, оказывается, что исходное множество~${\boldsymbol{\Omega}}$ усекается до 
подмножества ${\boldsymbol{\Omega}}^*(u_3)\hm=\{O_3,O_5\}$ (табл.~3), где 
замыкание~$[\{u_3\}]_{\mathbf{U},{\boldsymbol{\Omega}}^*(u_3)}$ есть множество $\{u_3, u_5\}$, 
что исключает попадание~$O_9$ в~${\boldsymbol{\Omega}}$, так как  
в~$O_9$ единица в~столбце~3,  
а~в~пятом~---~0.

     \noindent
     \textbf{Пример~3.}\ Пусть множество примеров (описаний 
прецедентов) ${\boldsymbol{\Omega}}\hm= \{O_1, O_2, \ldots, O_8\}$ представлено\linebreak\vspace*{-12pt}

\pagebreak

\noindent
{\small
 \begin{center}  %
 \parbox{46mm}{{\tablename~1}\ \ \small{Описания объектов (примеры~2 и~3)}}
 
 \vspace*{2ex}
  
     \tabcolsep=8pt
     \begin{tabular}{|c|c|c|c|c|c|}
     \hline
     $O$ &\multicolumn{5}{c|}{Описание}\\
     \hline
      $O_1$&0&1&0&0&1\\
$O_2$&1&1&0&1&0\\
     $O_3$&0&1&1&0&1\\
      $O_4$&1&1&0&1&1\\
      $O_5$&0&0&1&0&1\\
$O_6$&1&1&0&0&0\\
      $O_7$&1&0&0&1&0\\
      $O_8$&0&1&0&1&1\\
      \hline
      \end{tabular}
\end{center}
}

\vspace*{3pt}

\noindent
{\small
 \begin{center}  %
 \parbox{58mm}{{\tablename~2}\ \ \small{Число нулей и~единиц в~столбцах (пример~2)}}

\vspace*{2ex}

    \begin{tabular}{|c|c|c|c|c|c|}
\hline
Параметр& \multicolumn{5}{c|}{Количество в столбцах}\\
\hline
0&\ \ 4\ \ &\ \ 2\ \ &\ \ 6\ \ &\ \ 4\ \ &\,3\,\\
1&\ \ 4\ \ &\ \ 6\ \ &\ \ \cellcolor[gray]{.6}2\ \ &\ \ 4\ \ &\,5\,\\
$\langle$01110$\rangle$&\ 4\ &\ 6\ &\ \cellcolor[gray]{.6}2\ &\ 4\ &3\\
\hline
\end{tabular}
\end{center}
}

\vspace*{3pt}

\noindent
{\small
 \begin{center}  %
 \parbox{46mm}{{\tablename~3}\ \ \small{Результат усечения (пример~2)}}

\vspace*{2ex}

      \begin{tabular}{|c|c|c|c|c|c|}
     \hline
     Столбец &\multicolumn{5}{c|}{Описание}\\
     \hline
3&0&1&\cellcolor[gray]{.6}1&0&1\\
      5&0&0&\cellcolor[gray]{.6}1&0&1\\
      \hline
      \end{tabular}
\end{center}
}




\vspace*{9pt}

        
     
     \smallskip
     
     \noindent
 табл.~1. 
Требуется проверить, имеется ли в~${\boldsymbol{\Omega}}$ пример~$O_{10}$, которому 
может быть сопоставлен вектор (заголовок пакета) $\langle01011\rangle$.
     
Для этого, подсчитав число нулей и~единиц в~столбцах табл.~1 и~получив 
для~$O_{10}$ набор оценок (табл.~4 и~5), выберем первый столбец 
и~построим множество ${\boldsymbol{\Omega}}^*(u_1)\hm= \{O_1, O_3, O_5, O_8\}$ 
(табл.~6), после чего подсчитаем требуемые~0--1 показатели для табл.~6 
(табл.~7) и~далее по столбцу номер~3 (с~учетом корректности на~$O_{10}$ 
замыкания $[\{u_1\}]_{\mathbf{U},{\boldsymbol{\Omega}}^*(u_1)}\hm = \{u_1, u_5\})$ перейдем 
к~новому
${\boldsymbol{\Omega}}^{**}(u_1,u_3)\hm = \{O_1, O_8\}$ (табл.~8), где далее, 
например по единственной единице  в~четвертой колонке (напомним, что 
примеру~$O_{10}$ сопоставлен вектор
{\tabcolsep=0pt\begin{tabular}{ccc}
$\langle 010$&\cellcolor[gray]{.6}\textbf{1}&$1\rangle$\end{tabular}}), 
получаем $O_{10}\hm=O_8$, т.\,е.\ <<тестируемый>> пример~$O_{10}$ 
присутствует в~исходном~${\boldsymbol{\Omega}}$.




     
    Еще одна интересная область приложения об\-суж\-да\-емо\-го подхода~--- 
задачи анализа содержимого
  информационных потоков в~компьютерных 
сетях (см., например, уже упомянутую выше во введении проблематику DPI, 
IDS/IPS). Здесь одной из базовых процедур предпринимаемого 
интеллектуального анализа данных оказывается проверка вложимости тех 
или иных эталонных объектов в~пересылаемые пакеты данных, которая 
может быть формализована, в~частности, как задача проверки
 замкнутости 
предлагаемого для анализа множества
 <<структурных компонентов>> 
(например, фрагмен-\linebreak\vspace*{-12pt}

\columnbreak

\noindent
{\small
 \begin{center}  %
 \parbox{58mm}{{\tablename~4}\ \ \small{Число нулей и~единиц в~столбцах (пример~3)}}

\vspace*{2ex}

      \begin{tabular}{|c|c|c|c|c|c|}
\hline
Параметр& \multicolumn{5}{c|}{Количество в столбцах}\\
\hline
0&\ \ 2\ \ &\ \ 1\ \ &\ \ \cellcolor[gray]{.6}0\ \ &\ \  2\ \ &\,\cellcolor[gray]{.6}0\, \\
1&\ \ 0\ \ &\ \ 1\ \ &\ \ \cellcolor[gray]{.6}2\ \  &\ \ 0\ \ &\,\cellcolor[gray]{.6}2\, \\
\hline
\end{tabular}
\end{center}
}

\vspace*{3pt}

\noindent
{\small
 \begin{center}  %
 \parbox{50mm}{{\tablename~5}\ \ \small{Оценки для примера $O_{10}$}}

\vspace*{2ex}

      \begin{tabular}{|c|c|c|c|c|c|}
\hline
Заголовок пакета& \multicolumn{5}{c|}{Оценки}\\
\hline
$\langle$01011$\rangle$ & \cellcolor[gray]{.6}4&6&6&4&5\\
\hline
\end{tabular}
\end{center}
}

\vspace*{3pt}

\noindent
{\small
 \begin{center}  %
 \parbox{50mm}{{\tablename~6}\ \ \small{Усеченное множество ${\boldsymbol{\Omega}}^*(u_1)\hm = \{O_1, O_3, O_5, O_8\}$}}

\vspace*{2ex}

      \tabcolsep=9pt
      \begin{tabular}{|c|c|c|c|c|c|}
\hline
$O$ &\multicolumn{5}{c|}{Описание}\\
\hline
$O_1$&\cellcolor[gray]{.6}0&1&0&0&1\\
$O_3$&\cellcolor[gray]{.6}0&1&1&0&1\\
$O_5$&\cellcolor[gray]{.6}0&0&1&0&1\\
$O_8$& \cellcolor[gray]{.6}0&1&0&1&1\\
\hline
\end{tabular}
\end{center}
}

\vspace*{3pt}


\noindent
{\small
 \begin{center}  %
 \parbox{54mm}{{\tablename~7}\ \ \small{Показатели 0--1 для ${\boldsymbol{\Omega}}^*(u_1)$}}

\vspace*{2ex}

      \begin{tabular}{|c|c|c|c|c|c|}
\hline
Параметр& \multicolumn{5}{c|}{Количество в столбцах}\\
\hline
0&\ 4\ &\ 1\ &\ 2\ &\ 3\ &0\\
1&0&3&2&10\hphantom{9}&4\\
\hline
\end{tabular}
\end{center}
}

\vspace*{3pt}


\noindent
{\small
 \begin{center}  %
 \parbox{46mm}{{\tablename~8}\ \ \small{Усеченное множество ${\boldsymbol{\Omega}}^{**}(u_1,u_3) = \{O_1, O_8\}$}}

\vspace*{2ex}

      \begin{tabular}{|c|c|c|c|c|c|}
\hline
Параметр& \multicolumn{5}{c|}{Описание}\\
\hline
1&0 & 1&\cellcolor[gray]{.6}0&0&1\\
8&0&1&\cellcolor[gray]{.6}0&1&1\\
\hline
\end{tabular}
\end{center}
}

\vspace*{3pt}



\noindent
тов исходного кода) относительно заданного множества 
прецедентов (т.\,е.\ как задача проверки:\linebreak
 является ли заданное множество 
<<структурных фрагментов>> множеством \textit{всех} тех и~\textit{только} тех 
компонентов, которые являются общими для того или иного подмножества 
описаний прецедентов). Примеры~4 и~5 демонстрируют возможности 
пред\-ла\-га\-емой процедурной конструкции при решении задач этого типа.
    
    \smallskip
    
    \noindent
    \textbf{Пример~4.}\ Пусть множество ${\boldsymbol{\Omega}}\hm = \{O_1, O_2, \ldots, 
O_8\}$ для алфавита $\mathbf{U}\hm = \{a_1, a_2, \ldots, a_9\}$ 
сформировано так, как показано в~табл.~9.
Проверим, является
 ли замкнутым относительно ${\boldsymbol{\Omega}}$ множество 
образующих $\mathbf{u}\hm= \{u_1, u_2, u_3\}\hm = \{a_3, a_4, a_5\}$. 
Начнем с~выбора образующей из~$\mathbf{U}$, по которой будет 
проведено <<усечение>> множества~${\boldsymbol{\Omega}}$ (табл.~10). Имея одинаковые 
показатели во всех трех рас\-смат\-ри\-ва\-емых столбцах, выберем для 
дальнейших расчетов столбец~$a_3$ (учитывая, что образующие~$a_1$, $a_2$ 
и~$a_6$--$a_9$ следует исключить из рас\-смот\-ре\-ния как не <<участвующие>> 
в~формировании множества~$\mathbf{u}$) (табл.~11). Принимая во 
внимание, что на <<усе-\linebreak\vspace*{-12pt}

\pagebreak

\noindent
{\small
 \begin{center}  %
{{\tablename~9}\ \ \small{Описания объектов (примеры~4 и~5)}}

\vspace*{2ex}

      \begin{tabular}{|c|c|c|c|c|c|c|c|c|c|}
\hline
$O$&$a_1$&$a_2$&$a_3$&$a_4$&$a_5$&$a_6$&$a_7$&$a_8$&$a_9$\\
\hline
$O_1$&&&&&&Х&Х&&\\
$O_2$&&&&&&Х&Х&Х&\\
$O_3$&&&&&&Х&Х&Х&Х\\
$O_4$&Х&Х&Х&Х&&Х&Х&Х&Х\\
$O_5$&Х&Х&&Х&Х&&&&\\
$O_6$&Х&Х&Х&Х&Х&&&&\\
$O_7$&Х&Х&Х&&Х&&&&\\
$O_8$&Х&&Х&Х&Х&&&&\\
\hline
\end{tabular}
\end{center}
}

\vspace*{6pt}

\noindent
{\small


 \begin{center}  %
\parbox{76mm}{{\tablename~10}\ \ \small{Выбор образующей для усечения ${\boldsymbol{\Omega}}$ (пример~4)}}

\vspace*{2ex}
     \tabcolsep=4pt
      \begin{tabular}{|c|c|c|c|c|c|c|c|c|c|}
\hline
Множество&$a_1$&$a_2$&$a_3$&$a_4$&$a_5$&$a_6$&$a_7$&$a_8$&$a_9$\\
\hline
$\{a_3, a_4, a_5\}$&---&---&4&4&4&---&---&---&---\\
\hline
\end{tabular}
\end{center}
}

\vspace*{6pt}

\noindent
{\small
 \begin{center}  %
{{\tablename~11}\ \ \small{Усеченное множество ${\boldsymbol{\Omega}}^*= \{O_4, O_6, O_7, O_8\}$}}

\vspace*{2ex}

      \tabcolsep=6.4pt
      \begin{tabular}{|c|c|c|c|c|c|c|c|c|c|}
\hline
$O$&\cellcolor[gray]{.6}$a_1$&$a_2$&\cellcolor[gray]{.6}$a_3$&$a_4$&$a_5$&$a_6$&$a_7$&$a_8$&$a_9$\\
\hline
$O_4$&\cellcolor[gray]{.6}Х&Х&\cellcolor[gray]{.6}Х&Х&&Х&Х&Х&Х\\
$O_6$&\cellcolor[gray]{.6}Х&Х&\cellcolor[gray]{.6}Х&Х&Х&&&&\\
$O_7$&\cellcolor[gray]{.6}Х&Х&\cellcolor[gray]{.6}Х&&Х&&&&\\
$O_8$&\cellcolor[gray]{.6}Х&&\cellcolor[gray]{.6}Х&Х&Х&&&&\\
\hline
\end{tabular}
\end{center}
}

\vspace*{12pt}


\noindent
ченном>> множестве ${\boldsymbol{\Omega}}^*(u_1)\hm = \{O_4, O_6, 
O_7, O_8\}$ замыкание $[\{u_1\}]_{\mathbf{U},{\boldsymbol{\Omega}}^*(u_1)}\hm = \{a_1, a_3\}$ 
и~то, что~$a_1$ не входит в~$\mathbf{u}$, получаем, что~$\mathbf{u}$ 
не является замкнутым относительно исходного~${\boldsymbol{\Omega}}$.
     
     \smallskip
     
     \noindent
     \textbf{Пример~5.}\ Пусть, как и~в~примере~4, множества 
${\boldsymbol{\Omega}}\hm= \{O_1, O_2, \ldots, O_8\}$ и~$\mathbf{U}\hm=\{a_1, a_2, \ldots, 
a_9\}$ представлены табл.~9. Будет ли $\mathbf{u}\hm= \{u_1, u_2, u_3, 
u_4\}\hm=\{a_1, a_2, a_3, a_4\}$ замкнутым относительно~${\boldsymbol{\Omega}}$? Оценим 
встречаемость образующих из~$\mathbf{u}$ в~описаниях прецедентов 
из~${\boldsymbol{\Omega}}$ (табл.~12). Проведем <<усечение>> ${\boldsymbol{\Omega}}$ по~$a_2$ 
(табл.~13) и~убедимся, что на <<сужен\-ном>> множестве ${\boldsymbol{\Omega}}^*(u_2)\hm= 
\{O_4, O_5, O_6, O_7\}$ происходит замыкание 
$[\{u_2\}]_{\mathbf{U},{\boldsymbol{\Omega}}^*(u_2)}\hm = \{a_1, 
a_2\}$ при выполнении включения $a_1\hm\in\{a_1, a_2, a_3, a_4\} \hm= 
\mathbf{u}$. <<Сузим>> ${\boldsymbol{\Omega}}^*(u_2)$ по~$a_3$ (табл.~14), получив 
${\boldsymbol{\Omega}}^{**}(u_2,u_3)\hm=\{O_4, O_6, O_7\}$, и~(по~$a_4$)~--- множество 
${\boldsymbol{\Omega}}^{***}(u_2, u_3, u_4)\hm = \{O_4, O_6\}$, порождающее (сходством 
примеров~$O_4$ и~$O_6$ из исходного~${\boldsymbol{\Omega}}$~--- табл.~15) замкнутое 
подмножество $\mathbf{u}\hm = \{u_1, u_2, u_3, u_4\}\hm = \{a_1, a_2, a_3, 
a_4\}$.
     
\vspace*{6pt}

\noindent
{\small
 \begin{center}  %
{{\tablename~12}\ \ \small{Показатели 0--1 для ${\boldsymbol{\Omega}}$ (пример~5)}}

\vspace*{2ex}

\tabcolsep=4pt
      \begin{tabular}{|c|c|c|c|c|c|c|c|c|c|}
\hline
Множество&$a_1$&\cellcolor[gray]{.6}$a_2$&$a_3$&$a_4$&$a_5$&$a_6$&$a_7$&$a_8$&$a_9$\\
\hline
$\{a_1,a_2,a_3,a_4\}$&5&\cellcolor[gray]{.6}4&4&4&---&---&---&---&---\\
\hline
\end{tabular}
\end{center}
}


\vspace*{3pt}

\noindent
{\small
 \begin{center}  %
{{\tablename~13}\ \ \small{Усечение ${\boldsymbol{\Omega}}$ по $a_2$}}

\vspace*{2ex}

    \begin{tabular}{|c|c|c|c|c|c|c|c|c|c|}
\hline
$O$&\cellcolor[gray]{.6}$a_1$&\cellcolor[gray]{.6}$a_2$&$a_3$&$a_4$&$a_5$&$a_6$&$a_7$&$a_8$&$a_9$\\
\hline
$O_4$&\cellcolor[gray]{.6}Х&\cellcolor[gray]{.6}Х&Х&Х&&Х&Х&Х&Х\\
$O_5$&\cellcolor[gray]{.6}Х&\cellcolor[gray]{.6}Х&&Х&Х&&&&\\
$O_6$&\cellcolor[gray]{.6}Х&\cellcolor[gray]{.6}Х&Х&Х&Х&&&&\\
$O_7$&\cellcolor[gray]{.6}Х&\cellcolor[gray]{.6}Х&Х&&Х&&&&\\
\hline
\end{tabular}
\end{center}
}

\columnbreak



\noindent
{\small
 \begin{center}  %
{{\tablename~14}\ \ \small{Показатели 0--1 для ${\boldsymbol{\Omega}}^*(u_2)$}}

\vspace*{2ex}

\tabcolsep=4pt
    \begin{tabular}{|c|c|c|c|c|c|c|c|c|c|}
\hline
Множество&$a_1$&$a_2$&\cellcolor[gray]{.6}$a_3$&$a_4$&$a_5$&$a_6$&$a_7$&$a_8$&$a_9$\\
\hline
$\{a_1,a_2,a_3,a_4\}$&---&---&\cellcolor[gray]{.6}4&4&4&---&---&---&---\\
\hline
\end{tabular}
\end{center}
}

\vspace*{6pt}


\noindent
{\small
 \begin{center}  %
\parbox{78mm}{{\tablename~15}\ \ \small{Усеченное множество ${\boldsymbol{\Omega}}^{***}
(u_2, u_3, u_4)\hm = \{O_4, O_6\}$}}

\vspace*{2ex}

   \tabcolsep=6.3pt
    \begin{tabular}{|c|c|c|c|c|c|c|c|c|c|}
\hline
$O$&\cellcolor[gray]{.6}$a_1$&\cellcolor[gray]{.6}$a_2$&
\cellcolor[gray]{.6}$a_3$&\cellcolor[gray]{.6}$a_4$&$a_5$&$a_6$&$a_7$&$a_8$&$a_9$\\
\hline
$O_4$&\cellcolor[gray]{.6}Х&\cellcolor[gray]{.6}Х&\cellcolor[gray]{.6}Х&\cellcolor[gray]{.6}Х&&Х&Х&Х&Х\\
$O_6$&\cellcolor[gray]{.6}Х&\cellcolor[gray]{.6}Х&\cellcolor[gray]{.6}Х&\cellcolor[gray]{.6}Х&Х&&&&\\
\hline
\end{tabular}
\end{center}
}

\vspace*{-3pt}



\section{Заключение}

\vspace*{-3pt}

    Завершая обсуждение, обратим внимание на некоторые особенности 
представленной процедурной конструкции. Так, сравнивая <<прямой>> (см.\ 
алгоритм~1) и~оптимизированный (см.\ алгоритм~3)\linebreak способы проверки 
замкнутости, отметим возможности получить отрицательный ответ (т.\,е.\ 
продемонстрировать незамкнутость тестируемого множества образующих) 
в~том числе и~\textit{до полного\linebreak
 завершения всей процедуры} (например, 
в~ситуации, когда замыкание по текущему столбцу (обра\-зу\-ющей) множества 
примеров выводит получаемый результат за пределы тестируемого 
множества). Сокращение объемов вычислений дает также переход от 
исходного множества описаний прецедентов к~его все более <<глубоко>> 
усекаемым подмножествам. 
    
    При проверках встречаемости заголовка пакета в~таблице коммутации 
КТ известен (см., например,~\cite{7-zab}) быстрый алгоритм, использующий 
лексикографическое упорядочение строк КТ и~дальнейшее преобразование 
этого упорядочения в~дерево, которое позволяет диагностировать разряд за 
разрядом булевский вектор заголовка текущего анализируемого пакета 
в~процессе вычислений, объем которых линейно зависит от длины такого 
заголовка. Прямой перенос этой технологии на проверку замк\-ну\-тости 
произвольных подмножеств исходного алфавита оказывается 
затруднительным. Это дополнительный аргумент в~пользу предложенной 
здесь процедурной конструкции.
    
    В условиях жестких ограничений реального времени выполнения ИАД 
предлагаемая техника формирования замыканий Галуа легко может быть 
реализована в~режиме параллельных вычислений: в~анализе замкнутости 
сразу нескольких множеств\linebreak
 образующих каждое из них может быть 
рас\-смотре\-но в~рамках отдельного вычислительного\linebreak процесса (оперирующего 
копией исходного множества описаний прецедентов и~собственно 
тестируемым множеством образующих). Более того, представляющая 
исходные данные таблица [{\small ОБЪЕКТЫ}\;$\times$\;{\small ОБРАЗУЮЩИЕ}], как 
несложно показать, может быть разбита на <<клетки>> по подмножествам 
строк и~подмножествам столбцов так, что каждая из <<клеток>> может 
обрабатываться независимо от других. При этом финальный результат ИАД 
(например, наличие заголовка входного пакета в~таблице КТ или же 
замкнутость <<тестового>> множества образующих) будет 
характеризоваться достаточно простыми комбинациями результатов 
вычислений в~порожденных <<клетках>> (например, наличием хотя бы 
одной <<клетки>>, где зафиксирован эффект замкнутости соответствующего 
множества образующих; совместным рассмотрением всех демонстрирующих 
эффект замкнутости клеток и~т.\,п.). Число <<параллельных>> процессов 
такого вида будет определяться лишь характеристиками соответствующей 
вычислительной установки. Дополнительные возможности для сокращения 
объемов перебора здесь можно получить, используя предложенную 
в~работе~\cite{8-zab} технику декомпозиции исходных описаний 
прецедентов на самостоятельно обрабатываемые псевдодеревья (диаграммы 
вложимости замыканий Галуа).
    
    Наконец, при переходе к~<<усеченным>> подмножествам исходного 
множества примеров можно останавливать дальнейшие шаги процедуры 
<<усечения>>, получив текущее подмножество такого размера, что его 
можно полностью <<упаковать>> в~то\linebreak или иное схемное решение (ТСАМ  (ternary content-addressable
memory)
или, например, FPGA (field-programmable gate array)
соответствующего размера), где дальнейшие 
вычисления могут быть осуществле\-ны аппаратными средствами за один 
<<такт>> работы такого устрой\-ства.

    
    Рассмотренные особенности предложенной технологии позволяют 
говорить о~ней как о~еще\linebreak одном инструментальном основании для раз\-ра\-ботки 
активно развиваемых в~настоящее время\linebreak специализированных компьютерных 
ре\-ше\-ний-<<ак\-се\-ле\-ра\-то\-ров>> (см., например, обзор~\cite{9-zab}), которые 
расширяют спектр возможностей увеличения производительности 
проб\-лем\-но-ори\-ен\-ти\-ро\-ван\-ных прикладных систем ИАД.

{\small\frenchspacing
 {%\baselineskip=10.8pt
 \addcontentsline{toc}{section}{References}
 \begin{thebibliography}{9}
 \bibitem{3-zab} %1
  \Au{Финн В.\,К.} О~ма\-шин\-но-ори\-ен\-ти\-ро\-ван\-ной формализации 
правдоподобных рассуждений в~стиле Ф.~Бэкона--Д.\,С.~Милля~// 
Семиотика и~информатика, 1983. Вып.~20. С.~35--101.
  \bibitem{1-zab} %2
  \Au{McCarthy J.} Phenomenal data mining: From data to phenomena.~--- 
Stanford, CA, USA: Computer Science Department, Stanford University, 2000. 
{\sf http://jmc.\linebreak stanford.edu/articles/phenomenal.html}.
  \bibitem{2-zab} %3
  \Au{McCarthy J.} John McCarthy's home page. {\sf  
http://www-formal.stanford.edu/jmc}.
  
  \bibitem{4-zab} %4
  \Au{Финн В.\,К.} Индуктивные методы Д.\,С.~Милля в~сис\-те\-мах 
искусственного интеллекта~// Искусственный интеллект и~принятие 
решений, 2010. Ч.~I. №\,3. С.~3--21; Ч.~II. №\,4. С.~14--40.
  \bibitem{5-zab}
  \Au{Кон П.\,М.} Универсальная алгебра~/
  Пер. с англ.~--- М.: Мир, 1968. 359~с.
  (\Au{Cohn~P.\,M.}  {Universal algebra}.~---  New York, NY, USA: 
Harper and Row, 1965. 333~p.)
  \bibitem{6-zab}
  \Au{Грушо А.\,А., Забежайло~М.\,И., Зацаринный~А.\,А.} Контроль 
и~управление информационными потоками в~облачной среде~// 
Информатика и~её применения, 2015. Т.~9. Вып.~4. С.~95--101.
  \bibitem{7-zab}
  \Au{Кнут Д.} Искусство программирования. Т.~3. Сортировка и~поиск~/
  Пер. с англ.~--- М: Вильямс, 2007. 832~с.
(\Au{Knuth~D.\,E.} 
  {The art of computer programming. 3:~Sorting and searching}.~--- 
  2nd ed.~--- Reading, MA, USA: Addison-Wesley,  1998. 780~p.)
  \bibitem{8-zab}
  \Au{Забежайло М.\,И.} О~некоторых возможностях управ\-ле\-ния 
перебором в~ДСМ-ме\-то\-де~// Искусственный интеллект и~принятие 
решений, 2014. Ч.~I. №\,1. С.~95--110; Ч.~II. №\,3. С.~3--21.
  \bibitem{9-zab}
  \Au{Грушо А.\,А., Забежайло~М.\,И., Зацаринный~А.\,А., 
Писковский~В.\,О., Борохов~С.\,В.} О~возможностях приложений 
интеллектуального анализа данных в~задачах обеспечения информационной 
безопасности облачных сред~// НТИ. Сер.~2: Информационные процессы 
и~системы, 2015. №\,11. С.~1--11.
 \end{thebibliography}

 }
 }

\end{multicols}

\vspace*{-6pt}

\hfill{\small\textit{Поступила в~редакцию 15.10.16}}

\vspace*{8pt}

%\newpage

%\vspace*{-24pt}

\hrule

\vspace*{2pt}

\hrule

%\vspace*{8pt}


\def\tit{ON~THE~ADVANCED PROCEDURE
TO~REDUCE CALCULATION OF~GALOIS CLOSURES}

\def\titkol{On~the~advanced procedure
to~reduce calculation of~Galois closures}

\def\aut{A.\,A.~Grusho, M.\,I.~Zabezhailo, and~A.\,A.~Zatsarinny}

\def\autkol{A.\,A.~Grusho, M.\,I.~Zabezhailo, and~A.\,A.~Zatsarinny}

\titel{\tit}{\aut}{\autkol}{\titkol}

\vspace*{-9pt}

\noindent
Institute of Informatics Problems, Federal Research Center ``Computer Sciences and Control'' of the 
Russian Academy of Sciences, 44-2~Vavilov Str., Moscow 119333, Russian Federation



\def\leftfootline{\small{\textbf{\thepage}
\hfill INFORMATIKA I EE PRIMENENIYA~--- INFORMATICS AND
APPLICATIONS\ \ \ 2016\ \ \ volume~10\ \ \ issue\ 4}
}%
 \def\rightfootline{\small{INFORMATIKA I EE PRIMENENIYA~---
INFORMATICS AND APPLICATIONS\ \ \ 2016\ \ \ volume~10\ \ \ issue\ 4
\hfill \textbf{\thepage}}}

\vspace*{3pt}


\Abste{Formalization of similarity by algebraic operation is used as a key element 
of many modern intelligent data analysis methods. Nevertheless, in some important 
cases (e.\,g., in computer network traffic control, network\linebreak\vspace*{-12pt}}

\Abstend{security policy control in 
cloud computing environment, and some other), direct implementation of this 
technique is limited by the necessity to process huge amount of data in the real 
time mode. For example, it is necessary to intersect elements of a~large set of 
Boolean vectors of large length to find fixed points of so-called Galois closure. An 
advanced algorithm of Galois closure formation is introduced. The algorithm is 
used to optimize the process of set closeness checking. Some examples of 
applications of the presented technique in computer network traffic control and 
deep packet inspection are discussed.}

\KWE{intelligent data analysis; object similarity formalized as algebraic 
operation; combinatorial search optimization; header analysis; traffic control in 
computer networks; information security in cloud computing environment}

\DOI{10.14357/19922264160410} 

\vspace*{-12pt}

\Ack
\noindent
The paper was supported by the Russian Foundation for Basic Research (project 15-29-07981).



%\vspace*{12pt}

  \begin{multicols}{2}

\renewcommand{\bibname}{\protect\rmfamily References}
%\renewcommand{\bibname}{\large\protect\rm References}

{\small\frenchspacing
 {%\baselineskip=10.8pt
 \addcontentsline{toc}{section}{References}
 \begin{thebibliography}{9}
 
 \bibitem{3-zab-1} %1
  \Aue{Finn, V.\,K.} 1983. O~mashinno-orientirovannoy for\-ma\-li\-za\-tsii 
pravdopodobnykh rassuzhdeniy v~stile F.~Bacona--J.\,S.~Milla [On the  
computer-oriented formalization of plausible reasoning in the style of  
F.~Bacon\,--\,J.\,S.~Mill]. \textit{Semiotika i~informatika} [Semiotics and 
Informatics] 20:35--101.


  \bibitem{1-zab-1} %2
  \Aue{McCarthy, J.} 2000. \textit{Phenomenal data mining: From data to 
phenomena}.  Stanford, CA: Computer Science Department, Stanford University.  
 Available at {\sf 
http://jmc.\linebreak stanford.edu/articles/phenomenal.html} (accessed October~15, 2016). 
  \bibitem{2-zab-1} %3
  \Aue{McCarthy, J.} John McCarthy's home page. Available at {\sf 
   http://www-formal.stanford.edu/jmc/} (accessed October~15, 2016). 
  
  \bibitem{4-zab-1} %4
  \Aue{Finn, V.\,K.} 2011--2012. J.\,S.~Mill's inductive methods in artificial 
intelligence systems. \textit{Scientific and Technical Information Processing}. 
Part~I. 2011. 38(6):385--402; Part~II. 2012. 39(5):241--261.
  \bibitem{5-zab-1}
  \Aue{Cohn, P.\,M.} 1965. \textit{Universal algebra}.  New York, NY: 
Harper and Row. 333~p.
  \bibitem{6-zab-1}
  \Aue{Grusho, A.\,A., M.\,I.~Zabezhailo, and A.\,A.~Zatsarinny}. 2015. Kontrol'
i~upravlenie informatsionnymi potokami v~oblachnoy srede [Information flow 
monitoring and control in the cloud computing environment]. \textit{Informatika 
i~ee Primeneniya~--- Inform. Appl.} 9(4):95--101.
  
  \bibitem{7-zab-1}
  \Aue{Knuth, D.\,E.} 
  1998. \textit{The art of computer programming. 3:~Sorting and searching}. 
  2nd ed. Reading, MA: Addison-Wesley. 780~p.
  \bibitem{8-zab-1}
  \Aue{Zabezhailo, M.\,I.} 2014. O~nekotorykh vozmozhnostyakh upravleniya 
pereborom v~DSM-metode [On some new possibilities to control computational 
complexity of hypotheses generation in JSM-method].  Iskusstvennyy intellekt 
i~prinyatie resheniy [Artificial Intelligence and Decision Making]. Part~I:  
1:95--110; Part~II: 3:3--21.
  \bibitem{9-zab-1}
  \Aue{Grusho, A.\,A., M.\,I.~Zabezhailo, A.\,A.~Zatsarinnyy, V.\,O.~Piskovski, 
and S.\,V.~Borokhov}. 2015. O~vozmozhnostyakh prilozheniy intellektual'nogo 
analiza dannykh v~zadachakh obespecheniya informatsionnoy bez\-opas\-nosti 
oblachnykh sred [On the applications of intelligent data analysis in information 
security of cloud computing environment]. \textit{Nauchno-tehnicheskaya 
inormatsiya}  [Scientific and Technical Information]. Ser.~2. 11:1--11. 
\end{thebibliography}

 }
 }

\end{multicols}

\vspace*{-9pt}

\hfill{\small\textit{Received October 15, 2016}}

\vspace*{-6pt}

\Contr

\noindent
\textbf{Grusho Alexander A.} (b.\ 1946)~--- Doctor of Science in physics and mathematics, professor, 
Head of Laboratory, Institute of Informatics Problems, Federal Research Center ``Computer Sciences 
and Control'' of the Russian Academy of Sciences, 44-2~Vavilov Str., Moscow 119333, Russian 
Federation; \mbox{grusho@yandex.ru} 

\vspace*{3pt}

\noindent
\textbf{Zabezhailo Michael I.} (b.\ 1956)~--- Candidate of Science (PhD) in physics and mathematics, 
associate professor, Head of laboratory, Institute of Informatics Problems, Federal Research Center 
``Computer Sciences and Control'' of the Russian Academy of Sciences, 44-2~Vavilov Str., Moscow 
119333, Russian Federation; \mbox{m.zabezhailo@yandex.ru} 

\vspace*{3pt}

\noindent
\textbf{Zatsarinn Alexander A.} (b.\ 1951)~--- Doctor of Science in technology, professor, Deputy 
Director, Federal Research Center ``Computer Science and Control'' of the Russian Academy of 
Sciences, 44-2~Vavilov Str., Moscow 119333, Russian Federation; \mbox{azatsarinny@ipiran.ru}

\label{end\stat}


\renewcommand{\bibname}{\protect\rm Литература} 