\def\stat{zatsar}

\def\tit{СИСТЕМОТЕХНИЧЕСКИЕ ПОДХОДЫ К~СОЗДАНИЮ\\ 
СИСТЕМЫ ПОДДЕРЖКИ ПРИНЯТИЯ РЕШЕНИЙ\\ НА~ОСНОВЕ 
СИТУАЦИОННОГО АНАЛИЗА}

\def\titkol{Системотехнические подходы к~созданию 
системы поддержки принятия решений на~основе 
ситуационного анализа}

\def\aut{А.\,А.~Зацаринный$^1$, А.\,П.~Сучков$^2$}

\def\autkol{А.\,А.~Зацаринный, А.\,П.~Сучков}

\titel{\tit}{\aut}{\autkol}{\titkol}

\index{Зацаринный А.\,А.}
\index{Сучков А.\,П.}
\index{Zatsarinny A.\,A.}
\index{Suchkov A.\,P.}


%{\renewcommand{\thefootnote}{\fnsymbol{footnote}} \footnotetext[1]
%{Работа выполнена при финансовой поддержке РФФИ (проект 16-37-00485).}}


\renewcommand{\thefootnote}{\arabic{footnote}}
\footnotetext[1]{Институт проблем информатики Федерального исследовательского центра 
<<Информатика и~управ\-ле\-ние>> Российской академии наук, \mbox{AZatsarinny@ipiran.ru}}
\footnotetext[2]{Институт проблем информатики Федерального исследовательского центра 
<<Информатика и~управ\-ле\-ние>> Российской академии наук, \mbox{Asuchkov@ipiran.ru}}

      

\Abst{Обсуждаются вопросы создания сис\-тем поддержки принятия решений 
(СППР) на основе ситуационного анализа текущей и~прогнозируемой обстановки 
в~контролируемом пространстве органа управления. Как правило, такие сис\-те\-мы 
управления в~режиме реального времени опираются на ситуационные центры (СЦ)~--- 
совокупность информационных, программных и~аппаратных средств, а также 
обслуживающего персонала, реализующих информационные технологии по мониторингу 
обстановки, ее ситуационному анализу для выработки решений и~алгоритмов применения 
управляющих воздействий. Рассмотрены содержательные характеристики составляющих 
частей СППР, реализующих полный цикл управления от целеполагания до контроля 
исполнения принимаемых решений. Отмечается, что реализация СППР зависит от уровня 
сис\-те\-мы управ\-ле\-ния~--- стратегического, оперативного, тактического, базового, приводятся 
функциональные особенности и~способы анализа обстановки на различных уровнях 
сис\-те\-мы управ\-ления.}

\KW{ситуационный анализ; сис\-те\-ма поддержки принятия решений; сис\-те\-ма управ\-ле\-ния; 
ситуационный центр}

\DOI{10.14357/19922264160411} 


\vskip 10pt plus 9pt minus 6pt

\thispagestyle{headings}

\begin{multicols}{2}

\label{st\stat}

\section{Введение}

     В Стратегии национальной безопасности Российской Федерации 
(утверждена Указом Президента Российской Федерации от~31~декабря 
2015~г. №\,683)~[1] определено, что информационную основу реализации 
Стратегии составляет федеральная информационная сис\-те\-ма стратегического 
планирования, включающая в~себя информационные ресурсы органов 
государственной власти и~органов местного самоуправления, сис\-те\-мы 
распределенных СЦ и~государственных научных 
организаций. В~рамках такой сис\-те\-мы должна быть реализована поддержка 
управленческих решений в~интересах центральных органов исполнительной 
власти на основе организации взаимодействия региональных 
и~ведомственных СЦ, а~также других информационных 
сис\-тем. Для эффективного решения этой задачи необходимо создание СППР 
в~со\-ста\-ве СЦ и~придания им принципиально новых качеств. 
     
     В связи с~этим целью статьи является обоснование сис\-те\-мо\-тех\-ни\-че\-ских 
и~методических подхо\-дов к~структурному и~функциональному составу\linebreak 
СППР и~ее месту в~составе СЦ, обеспечивающих 
информационно-аналитическую поддержку принятия управленческих 
решений в~рамках государственного управления, стратегического 
планирования и~мониторинга реализации документов стратегического 
планирования в~Российской Феде-\linebreak рации. 

\vspace*{-6pt}
     
\section{Базовые понятия }

\vspace*{-2pt}

    При рассмотрении сис\-тем\-ных и~методических вопросов создания СППР, 
основанных на ситуационном анализе, в~статье используется ряд базовых 
понятий: событие, обстановка, ситуация, угроза, управление, цели 
управления и~др.~[2]. 
    
    \textit{Ситуация} определяется состоянием взаимосвязанных 
\textit{элементов обстановки} в~контролируемом пространстве; изменения 
обстановки определя-\linebreak ются \textit{событиями}, образующими некоторые 
разворачивающиеся во времени наблюдаемые и~ре\-гист\-ри\-ру\-емые потоки. При 
этом под \textit{управлением}\linebreak понимается \textbf{целенаправленное} 
воздействие органа управления на подчиненные ему или взаимодействующие 
элементы обстановки (ресурсы). 
    
    Совокупность ситуаций в~сис\-те\-ме управ\-ле\-ния распадается на текущие, 
прогнозируемые и~целевые ситуации. При этом текущие ситуации являются 
результатом наблюдения и~регистрации событий, прогнозируемые 
определяются методами ситуационного анализа, а целевые отражают 
краткосрочные, среднесрочные и~долгосрочные цели управления. Последнее 
немаловажно, так как зачастую ситуационный анализ понимается как 
обеспечение реакций сис\-те\-мы управ\-ле\-ния на чрезвычайные ситуации после 
того, как они сложились. Однако теория ситуационного подхода 
предполагает учет <<планируемой и~прогнозируемой обстановки>>, 
отражающей стратегические, тактические и~оперативные \textit{цели 
управления}, а~также учет факторов самоорганизации управляющего 
сегмента сис\-те\-мы, определяющих стимулы для достижения этих 
целей~[2,~3]. Под \textit{угрозой} в~процессах управления понимается 
ситуация или совокупность ситуаций, развитие которых противоречит целям 
управления и~отдаляет текущее состояние от целевого.
    
    В конце 1970-х~гг.\ была создана модель сис\-те\-мы управ\-ле\-ния  
<<наблю\-де\-ние--ори\-ен\-ти\-ро\-ва\-ние--ре\-ше\-ние--дей\-ст\-вие>> 
(НОРД) для принятия решений при ведении боевых действий~[4, 5]. 
В~настоящее время эта модель активно используется во многих сис\-те\-мах 
управ\-ле\-ния разных отраслей~[6]. В~рамках ситуационного подхода 
к~управлению предложена модифицированная модель, включающая 
дополнительную стадию управляющего цикла~--- целеполагание~[7].
    
    \textbf{Целеполагание} (стадия~Ц)~--- формализованное представление 
целевых показателей, установление количественных 
и~временн$\acute{\mbox{ы}}$х критериев их достижения.
    
    \textbf{Мониторинг} (стадия~М)~--- это процесс сбора информации об 
окружающей среде в~контролируемом пространстве, включая состояние 
целевых показателей. Стадия М также принимает внутренние инструкции от 
стадии анализа (А), так же как и~поддержку от процессов~Р и~Д. 
    
    \textbf{Анализ} (стадия~А)~--- оценка ситуации (типовая, нетиповая), 
анализ существующего опыта, пополнение опыта, обеспечивает внутреннюю 
поддержку~М (корректировка фильтров).
    
    \textbf{Решение} (стадия~Р)~--- это процесс осуществления выбора 
среди гипотез о состоянии окружающей среды и~возможной реакции на него. 
Процесс~Р руководствуется прямой внутренней связью с~процессом~А 
и~обеспечивает внутреннюю поддержку процесса~М, возможна 
корректировка целевых показателей (стадия~Ц).
    
    \textbf{Действие} (стадия~Д)~--- это процесс выполнения выбранной 
реакции путем взаимодействия с~окружающей средой. Действие принимает 
внутренние руководства от процесса~А, также оно напрямую связано с~Р. 
Оно обеспечивает внутреннюю поддержку~Ц и~М.
    
    Особенности реализации цикла управления в~сис\-те\-ме, реализующей 
процессы стратегического планирования и~управления, заключаются в~том, 
что содержательно стадии~Ц, А и~Р реализуются непосредственно высшими 
органами исполнительной власти. Это означает осуществление сле\-ду\-ющих 
основных функций:
    \begin{itemize}
\item  доведение до подчиненных органов данных целеполагания 
и~стратегического планирования на основе их формализации 
и~регламентации обмена (стадия~Ц);
\item регламентированный сбор данных о состоянии целевых показателей от 
органов испол\-нительной власти и~об обстановке в~конт\-ро\-ли\-ру\-емом 
пространстве по определенному\linebreak регламенту и~в~режиме реального времени 
(стадия~М);
\item обмен аналитическими данными участников\linebreak стратегического 
планирования по целеполаганию, прогнозированию, планированию 
и~программированию~--- федеральных органов исполнительной власти 
(ФОИВ), субъектов Россий\-ской Федерации и~муниципальных образований, 
отраслей экономики и~сфер государственного и~муниципального управления 
(стадия~А);
\item  доведение до подчиненных органов принимаемых решений по 
применению сил и~средств и,~возможно, по корректировке стратегических 
планов с~целью достижения поставленных стратегических целей (стадия~Р) 
и~контроль исполнения решений (стратегических планов) на основе 
докладов (стадия~Д).
    \end{itemize}
    
    На тактическом и~базовом уровнях управления осуществляются,  
во-пер\-вых, реализация функ-\linebreak ций мониторинга контролируемого 
пространства и~организа\-ции учета контролируемых объектов (стадия~М),  
во-вто\-рых, специальный анализ фактографических данных о конкретных 
элементах обстановки, формализованных в~виде семантической сети, 
позволяющий выявлять неочевидные связи между элементами обстановки, 
определять схожие про\-стран\-ст\-вен\-но-со\-бы\-тий\-ные ситуации, выявлять 
ассоциативные связи и~закономерности с~\mbox{целью} поддержки процессов 
принятия решений (стадия~А), в-треть\-их, процессы принятия решений 
по планированию применения сил и~средств на период времени и~по 
складывающейся обстановке в~соответствии с~указаниями вышестоящих 
органов (стадии~Р и~Д).

\begin{figure*}[b] %fig1
\vspace*{1pt}
\begin{center}
\mbox{%
\epsfxsize=160.901mm
\epsfbox{zac-1.eps}
}
\end{center}
\vspace*{-9pt}
\Caption{Обобщенная функциональная структура СЦ}
\end{figure*}
    

\section{Ситуационный центр как составляющая современной системы 
управления}
    
    Определим СЦ сис\-те\-мы управ\-ле\-ния как совокупность 
информационных, программных и~аппаратных средств, а~также 
обслуживающего персонала, реализующих информационные технологии\linebreak по 
мониторингу обстановки, ее ситуационному анализу для выработки решений 
и~алгоритмов применения управляющих воздействий с~\mbox{целью} эффективной 
реализации функций управления и~минимизации ущерба от угроз в~зоне 
ответствен\-ности\linebreak органа управ\-ле\-ния, доведения их до объектов управ\-ле\-ния 
и~контроля исполнения,
    
    По сути дела, СЦ является составной частью сис\-те\-мы 
управ\-ле\-ния, осуществляющей автоматизацию ряда функций всего органа 
управления и~отдельных должностных лиц.
    
    Исходя из накопленного в~Институте проблем информатики РАН опыта 
разработки крупных информационных сис\-тем в~интересах органов 
государственной власти, в~организационной структуре СЦ можно выделить 
четыре основных функциональных сегмента (рис.~1)~\cite{8-zat}:
    \begin{enumerate}[(1)]
\item сегмент руководства (лиц, принимающих решения, ЛПР); 
\item сегмент мониторинга состояния контролируемых объектов 
и~окружающей среды и~сбора информации; 
\item сегмент ситуационного анализа и~сис\-те\-ма\-ти\-за\-ции информации;
\item сегмент администрирования и~эксплуатации.
\end{enumerate}
    При этом СППР базируется на ресурсах всех четырех сегментов. Вместе 
с тем центральным звеном СЦ и~его СППР, обеспечивающим реализацию 
основной функции сис\-те\-мы управ\-ле\-ния по эффективному управлению 
силами и~средствами, является \textit{сегмент ситуационного анализа 
и~сис\-те\-ма\-ти\-за\-ции информации}. Он должен обеспечивать реализацию 
следующих функций:
    \begin{itemize}
\item возможность визуализации результатов анализа обстановки на 
индивидуальных и~коллективных средствах отображения;
\item во взаимодействии с~сегментом мониторинга получение данных 
о~состоянии обстановки от собственных (субъективных и~объективных 
средств наблюдения и~контроля) и~внешних по отношению к~сис\-те\-ме 
источников информации (ведомственных, межведомственных, 
международных, независимых и~др.);
\item извлечение фактов, структуризация и~формализация разнородных 
данных о~значимых событиях в~соответствии с~выбранной информационной 
моделью предметной области;
\item формирование хранилищ ситуационных данных;
\item формирование способов визуализации агрегированных данных 
о~складывающейся обстановке для ЛПР и~оперативного состава;
\item формирование отчетности и~служебной документации;
\item расчет первичных и~интегральных показателей обстановки, а~также 
статистическая оценка характеристик ненаблюдаемых элементов обстановки;
\item решение задач перспективного планирования, контроль исполнения 
решений по планированию;
\item выявление значимых ситуаций, их ранжирование по степени 
важности, видам и~типам, формирование текущего перечня 
аналитических задач по складывающейся обстановке и~по поручениям 
руководства;
\item  выработка вариантов решений по применению управляющих 
воздействий для достижения целевых ситуаций, формирование спо\-собов 
наглядного представления вариантов\linebreak реше\-ния для ЛПР (оперативное 
планирование);
\item прогнозирование развития обстановки и~процесса реализации целей 
сис\-те\-мы управ\-ле\-ния на основе сформированных ситуационных моделей 
и~моделей угроз, в~том числе и~с~учетом применения выработанных 
вариантов решений;
\item обеспечение процессов принятия решений комплексом  
ин\-фор\-ма\-ци\-он\-но-рас\-чет\-ных задач (ИРЗ).
    \end{itemize}
    
    Наряду с~перечисленными в~СППР СЦ реализуются важнейшие функции 
администрирования аналитической под\-сис\-те\-мы~СЦ:
    \begin{itemize}
\item формирование и~корректировка сис\-те\-мы целей управ\-ления;
\item формирование, настройка и~корректировка сис\-те\-мы моделей целей 
управления, обстановки, ситуаций и~угроз;
\item формирование, настройка и~корректировка сис\-те\-мы расчетных 
показателей, характеризующих обстановку и~ее элементы;
\item формирование, настройка и~корректировка сис\-те\-мы критериев, 
пороговых значений, эвристик, параметров расчетных алгоритмов.
\end{itemize}

\section{Целеполагание~--- определение целей системы управления}

    Под \textit{целью ситуационного анализа} предлагается понимать 
поддержку процессов принятия решений для достижения поставленных 
целей путем применения доступных в~сис\-те\-ме управ\-ле\-ния сил и~средств 
(ресурсов).
    
    Целесообразность деятельности сис\-те\-мы управ\-ле\-ния определяется 
иерархической сис\-те\-мой целей\linebreak (подцелей). Для ФОИВ она задается 
законодательно, а также при определении приоритетов в~орга\-низации 
деятельности сис\-те\-мы управ\-ле\-ния первым\linebreak лицом (руководителем). 
Формирование сис\-те\-мы целей сопровождается формированием сис\-те\-мы 
показателей реализации целей (подцелей) и~критериев достижения целей. 
Показатели являются вычисляемыми величинами как функции обстановки 
или экспертно оцениваемые параметры. Критерии достижения обычно 
формулируются как некие пороговые плановые значения на временн$\acute{\mbox{о}}$й 
шкале.
    
    Эффективность сис\-те\-мы управ\-ле\-ния в~каждый момент времени 
определяется, во-пер\-вых, степенью достижения пороговых значений 
планируемых целевых показателей, во-вто\-рых, объемом затрачиваемых 
ресурсов на единицу оптимизируемого целевого показателя.
    
    Цели управления формируются на основании сис\-тем\-но\-го анализа  
нор\-ма\-тив\-но-пра\-во\-вых основ функционирования сис\-те\-мы управ\-ле\-ния. 
Цели управления образуют дерево целей, детализация которого (число 
уровней) определяется воз\-мож\-ностью декомпозиции конкретной цели на 
значимые подцели. Цели и~подцели должны обладать индикаторами 
состояния (как правило, \%) и~весовыми коэффициентами доли подцели 
в~реализации всей цели. Цели могут включать ориентиры развития сис\-те\-мы 
управления, установленные первым лицом.

\begin{figure*} %fig2
\vspace*{1pt}
\begin{center}
\mbox{%
\epsfxsize=165.008mm
\epsfbox{zac-2.eps}
}
\end{center}
\vspace*{-9pt}
\Caption{Обобщенная структура сис\-те\-мы целей}
\end{figure*}
    
    Выбор структуры сис\-те\-мы целей предлагается осуществлять с~учетом 
следующих соображений.
    \begin{enumerate}[1.]
    \item Цели управления сложной управляющей сис\-те\-мой определяются 
нор\-ма\-тив\-но-пра\-во\-вы\-ми документа\-ми, регламентирующими ее 
функционирование, и, как правило, образуют \textbf{иерархическую 
структуру} в~соответствии со структурой направлений деятельности 
(рис.~2).
    \item Ситуационный подход к~управлению предполагает реагирование на 
складывающуюся обстановку в~режиме реального времени. В~силу этого, 
помимо фиксированных целей в~сис\-те\-ме управ\-ле\-ния необходим механизм 
формирования \textbf{динамических целей}, отражающих процесс 
нормализации складывающихся чрезвычайных ситуаций и~присутствующих 
в~сис\-те\-ме целеполагания на период существования ситуации.
    \item В~концепции <<управления по целям>> эффективность 
целеполагания проверяется по критериям SMART~\cite{9-zat}: цель должна 
быть конкретная, измеримая (подразумевает количественную измеримость 
результата), достижимая (должна быть выполнимой), реалистичная 
(достижение цели должно быть обеспечено ресурсами), привязанная  
к~точ\-ке/ин\-тер\-ва\-лу времени.
    \end{enumerate}
    
    Данный подход накладывает \textbf{требования на атрибуты целей} 
в~части формирования количественных характеристик их достижения, 
плановых характеристик, критериев достижения (см.\ рис.~2). 
    


    Основные атрибуты цели:\\[-14pt]
    \begin{itemize}
\item описание~--- дает определение и~конкретизацию цели;\\[-14pt]
\item весовой коэффициент~--- определяет вклад подцели 
в~вышестоящую цель;\\[-14pt]
\item индикатор~--- задает количественный показатель достижения 
результата;\\[-14pt]
\item критерий~--- задает способ определения достижения результата 
с~помощью индикатора;\\[-14pt]
\item план~--- определяет количественные значения критерия 
достижения цели и~требуемые вре\-мен\-н$\acute{\mbox{ы}}$е параметры.
\end{itemize}

\vspace*{-9pt}

\section{Анализ обстановки и~выработка вариантов решений}

\vspace*{-2pt}

\subsection{Мониторинг обстановки}

\vspace*{-1pt}

     В процессе мониторинга контролируемых элементов обстановки 
осуществляются (рис.~3):
     \begin{itemize}
\item сбор данных о состоянии контролируемых объектов, анализ 
неструктурированной информации с~целью извлечения фактов и~знаний; 
\item постановка объектов на контроль (оператор, автоматически); 
\item отображение контролируемых объектов по шкале состояний и~по 
критериям~--- соотношение текущего или прогнозируемого значения 
индикатора (интегрального показателя) и~сис\-те\-мы порогов, обеспечивающих 
градацию состояния (<<типовое>>, <<чрезвычайное>>, <<критическое>> 
или другие подобные).
\end{itemize}

    По данным мониторинга контролируемых элементов обстановки из 
различных источников формируется \textit{хранилище} СППР, которое 
пред\-став\-ляет собой совокупность взаимоувязанных на\linebreak основе единого 
информационного и~лингвистического обеспечения баз данных (БД): 
обстановки (события, ситуации, элементы окружающей\linebreak среды), сил и~средств 
(свои силы и~средства, противодействующие силы и~средства, так\-ти\-ко-тех\-ни\-че\-ские
характеристики), целевых 
показателей (первичные показатели, интегральные показатели,\linebreak индикаторы, 
критерии), типовых решений (типовые решения, конкретные решения), 
ретроспективная (нормализованные исторические данные, архив 
обстановки), нормативных документов, биб\-лио\-те\-ка математических моделей.

\vspace*{-6pt}

\subsection{Поддержка процесса принятия решений}

\vspace*{-2pt}

    На основе мониторинга текущей обстановки и~поступления событийной 
информации в~хранилище осуществляется расчет заданных в~сис\-те\-ме 
первичных и~интегральных показателей обстановки и~целевых показателей 
в~двух режимах: по регламенту (с~определенной периодичностью) и~по 
запросу пользователя с~использованием блоков расчетов, блока первичного, 
краткосрочного, среднесрочного и~долгосрочного анализа, блока 
визуализации  и~блока поддержки принятия решений (рис.~4).\linebreak\vspace*{-12pt}


\pagebreak

\end{multicols}
\begin{figure*} %fig3
\vspace*{1pt}
\begin{center}
\mbox{%
\epsfxsize=157.334mm
\epsfbox{zac-3.eps}
}
\end{center}
\vspace*{-6pt}
\Caption{Мониторинг обстановки}
\vspace*{6pt}
\end{figure*}

\begin{multicols}{2}




    
    При этом реализуются следующие функции.
    \begin{enumerate}[1.]
\item  Создание (привязка существующих) динамических моделей 
обстановки:
\begin{itemize}
\item моделей <<нормальной>> обстановки;
\item моделей для прогноза обстановки;
\item моделей для анализа трендов, циклов, аномалий обстановки.
\end{itemize}

    \item Проведение оперативного анализа текущей обстановки 
с~использованием математических методов (см.\ рис.~4):
\begin{itemize}
\item анализ отклонения от <<нормальной>> текущей обстановки;
\item прогноз развития обстановки;
\item анализ трендов, циклов, аномалий обстановки;
\item выявление и~идентификация значимых ситуаций 
на основе выявления типовых кон-\linebreak\vspace*{-12pt}

\columnbreak

\noindent
фигураций событий 
и~правил идентификации, идентификация типа ситуации, 
фор\-ми\-ро\-ва\-ние неотложных целей.\\[-7.5pt]
\end{itemize}
    \item Визуализация и~индикация состояний контролируемых объектов 
    с~использованием полученных результатов анализа (наглядное пред\-став\-ле\-ние 
текущей с~индикацией ситуаций,\linebreak требующих принятия решения или 
применения типовых решений).\\[-6pt]
    \item Выработка вариантов решений по складыва\-ющейся обстановке 
(решение содержит динамическую цель, перечень подцелей (с~весами~--- 
доли подцели в~реализации всей цели),\linebreak сроки достижения подцелей, 
ответственных, совокупность типовых уведомлений и~рапортов):
\begin{itemize}
\item применение типовых решений по типовым ситуациям (привязка их 
к~реальной обстановке);
\end{itemize}
\end{enumerate}



\pagebreak

\end{multicols}

\begin{figure*} %fig4
\vspace*{1pt}
\begin{center}
\mbox{%
\epsfxsize=164.07mm
\epsfbox{zac-4.eps}
}
\end{center}
\vspace*{-11pt}
\Caption{Структура блока принятия решений}
\vspace*{-3pt}
\end{figure*}

\begin{multicols}{2}

\noindent
\begin{enumerate}
\item[\ ]
\vspace*{-13pt}
\begin{itemize}
\item выработка вариантов решения экспертным путем в~случае критических 
и чрезвычайных ситуаций;\\[-15pt]
\item анализ развития обстановки с~учетом вариантов решений (прогноз 
благоприятного и~неблагоприятного развития обстановки, расчет 
вероятностей выполнения задач, оценка вариантов решений).
\end{itemize}
\end{enumerate}

\vspace*{-6pt}

    \subsection{Реализация решений }
    
    \vspace*{-2pt}
    
    На данной стадии осуществляется мониторинг процессов реализации 
решений по краткосрочным, среднесрочным и~долгосрочным планам 
(решение содержит цель, перечень подцелей (с~весами~--- доли подцели 
в~реализации всей цели), сроки достижения подцелей, ответственных, виды 
отчетности):
\begin{itemize}
\item сбор информации по ходу выполнения плана (отчетность), 
визуализация хода исполнения, контроль исполнения;\\[-15pt]
\item сравнительный анализ показателей плана по целям и~подцелям 
и~текущей обстановки, включая расчет степени реализации плана 
и~прогнозирование возможности реализации плана;\\[-15pt]
\item реализация обратной связи по уточнению решения по планированию 
с~целью обеспечения выполнения плана;\\[-15pt]
\item доведение уточненного решения (уведомления) и~контроль исполнения.
\end{itemize}

    Мониторинг реализации решений по ситуациям (решение содержит 
динамическую цель, перечень подцелей (с~весами~--- доли подцели 
в~реализации всей цели), сроки достижения подцелей, ответственных, 
совокупность типовых уведомлений и~рапортов): 
    \begin{itemize}
\item сбор информации по ходу выполнения решения (рапорты), 
визуализация хода исполнения, контроль исполнения;
\item сравнительный анализ показателей по целям и~подцелям и~текущей 
обстановки, включая расчет степени реализации решения и~прогнозирование 
возможности реализации решения;
\item реализация обратной связи по уточнению решения по ситуации с~целью 
обеспечения выполнения плана.
\item доведение уточненного решения (уведомления) и~контроль исполнения.
\end{itemize}

\vspace*{-6pt}

\section{Заключение}

\noindent
\begin{enumerate}[1.]
\item В современных условиях развития информационных сис\-тем особую 
значимость приобретает актуальность исследования сис\-те\-мо\-тех\-ни\-че\-ских 
и~технологических вопросов создания в~составе СЦ
СППР.
\item Важнейшей методологической и~концептуальной основой СППР 
является полнофункциональный цикл управления, включающий стадии 
целеполагания, мониторинга обстановки, анализа обстановки, выработки 
вариантов решений и~их реализации.
\item В СППР реализуются следующие функциональные задачи:
\begin{itemize}
\item мониторинг контролируемых элементов обстановки;
\item расчет характеристик событийной информации (первичные 
и~интегральные показатели текущей обстановки и~состояния 
целей);
\item визуализация текущего состояния обстановки;
\item визуализация текущего состояния индикаторов целей;
\item блок анализа и~принятия решений.
\item мониторинг контролируемых решений;
\item формирование документов и~отчетов.
\end{itemize}
\item Важнейшим сис\-те\-мо\-обра\-зу\-ющим компонентом СППР является 
хранилище, формируемое в~автоматизированном режиме из различных 
источников в~виде совокупности взаимоувязанных на основе единого 
информационного и~лингвистического обеспечения БД (о~событиях, 
силах и~средствах, целевых показателях и~критериях, типовых решений, 
ретроспективной информации, нормативных документов, математических 
моделей).
\item Предложенные в~статье сис\-те\-мо\-тех\-ни\-че\-ские подходы и~решения 
апробированы в~рамках нескольких проектов по созданию крупных 
территориально распределенных  
ин\-фор\-ма\-ци\-он\-но-ана\-ли\-ти\-че\-ских сис\-тем специального 
назначения.
\end{enumerate}

\vspace*{-6pt}

{\small\frenchspacing
 {%\baselineskip=10.8pt
 \addcontentsline{toc}{section}{References}
 \begin{thebibliography}{9}

\bibitem{1-zat}
Стратегия национальной безопасности Российской Федерации. Утверждена Указом 
Президента Российской Федерации от 31~декабря 2015~г. №\,683. 
\bibitem{2-zat}
\Au{Зацаринный А.\,А., Сучков А.\,П.} Некоторые подходы к~ситуационному анализу 
потоков событий~// Открытое образование, 2012. №\,1. С.~39--45.
\bibitem{3-zat}
\Au{Бир С.\,Э.} Мозг фирмы~/
Пер. с~англ.~--- М.: Радио и~связь, 1993. 416~с.
(\Au{Beer~S.}  {Brain of the firm}.~--- Allen Lane, The Penguin Press, London; Herder 
and Herder, USA, 1972. 416~p.)

\bibitem{5-zat}%4
\Au{Grant Т., Kooter В.} Comparing OODA \& other models as operational view~C2 
architecture~// 10th Command and Control Research Technology Symposium (International) 
Proceedings.~--- McLean, VA, USA, 2005.
\bibitem{4-zat} %5
\Au{Ивлев А.\,А.} Основы теории Бойда. Направления развития, применения 
и~реализации.~--- SlideShare, 2008. 64~с. {\sf  
http://www.slideshare.net/defensenetwork/ss-10380168}.
\bibitem{6-zat}
\Au{Босов А.\,В., Зацаринный А.\,А., Сучков~А.\,П.} Некоторые общие подходы 
к~формированию функциональных требований к~ситуационным центрам и~их 
реализации~// Системы и~средства информатики, 2010. Вып.~20. №\,3. С.~98--125.
\bibitem{7-zat}
\Au{Сучков А.\,П.} Формирование сис\-те\-мы целей для ситуационного управ\-ле\-ния~// 
Сис\-те\-мы и~средства информатики, 2013. Т.~23. №\,2. С.~171--182.
\bibitem{8-zat}
\Au{Зацаринный А.\,А., Сучков~А.\,П., Козлов~С.\,В.} Особенности проектирования 
и~функционирования сис\-те\-мы ситуационных центров~// Системы высокой доступности, 
2012. Т.~8. №\,1. С.~12--21.
\bibitem{9-zat}
\Au{Doran G.\,T.} There's a~S.M.A.R.T.\ way to write management's goals and objectives~// 
Manag. Rev., 1981. Vol.~70. Iss.~11. P.~35--36.
 \end{thebibliography}

 }
 }

\end{multicols}

\vspace*{-6pt}

\hfill{\small\textit{Поступила в~редакцию 23.08.16}}

%\vspace*{8pt}

\newpage

\vspace*{-24pt}

%\hrule

%\vspace*{2pt}

%\hrule

%\vspace*{8pt}


\def\tit{SYSTEMS ENGINEERING APPROACHES TO~THE~ESTABLISHMENT 
OF~A~SYSTEM FOR~DECISION SUPPORT BASED ON~SITUATIONAL ANALYSIS}

\def\titkol{Systems engineering approaches to~the~establishment 
of~a~system for~decision support based on~situational analysis}

\def\aut{A.\,A.~Zatsarinny and A.\,P.~Suchkov}

\def\autkol{A.\,A.~Zatsarinny and A.\,P.~Suchkov}

\titel{\tit}{\aut}{\autkol}{\titkol}

\vspace*{-9pt}


\noindent
Institute of Informatics Problems, 
Federal Research Center ``Computer Sciences and Control'' of the 
Russian Academy of Sciences, 44-2~Vavilov Str., Moscow 119333, 
Russian Federation



\def\leftfootline{\small{\textbf{\thepage}
\hfill INFORMATIKA I EE PRIMENENIYA~--- INFORMATICS AND
APPLICATIONS\ \ \ 2016\ \ \ volume~10\ \ \ issue\ 4}
}%
 \def\rightfootline{\small{INFORMATIKA I EE PRIMENENIYA~---
INFORMATICS AND APPLICATIONS\ \ \ 2016\ \ \ volume~10\ \ \ issue\ 4
\hfill \textbf{\thepage}}}

\vspace*{3pt}

 
\Abste{The article discusses the issues of decision-making support systems (DMSS) 
creation based on the situational analysis of the current and projected situation in the 
controlled space. Typically, such control systems in real time are based on situational 
centers, which are sets of information, software, hardware, and staff implementing 
information technology to monitor the situation and its situational analysis to develop 
solutions and algorithms application of control actions. The paper considers 
characteristics of the DMSS components, implementing the full management cycle from 
goal setting to execution control decisions. It is noted that the implementation of the 
decision support system depends on the level of management~--- strategic, operational, tactical, basic, and 
functional features and methods of analysis of the situation at different levels of the 
control system.}

\KWE{situational analysis; system of decision-making process support; management 
system; situational center}

\DOI{10.14357/19922264160411} 

%\vspace*{-9pt}

%\Ack
%\noindent


%\vspace*{3pt}

  \begin{multicols}{2}

\renewcommand{\bibname}{\protect\rmfamily References}
%\renewcommand{\bibname}{\large\protect\rm References}

{\small\frenchspacing
 {%\baselineskip=10.8pt
 \addcontentsline{toc}{section}{References}
 \begin{thebibliography}{9}

\bibitem{1-zat-1}
Strategiya natsional'noy bezopasnosti Rossiyskoy Fe\-de\-ra\-tsii [The National Security Strategy of 
the Russian Federation]. Approved by the Decree of the President of the Russian Federation 
No.\,683, 31.12.2015.
\bibitem{2-zat-1}
\Aue{Zatsarinny, A.\,A., and A.\,P.~Suchkov.} 2012. Nekotorye podkhody k~situatsionnomu 
analizu potokov sobytiy [Some approaches to the situational analysis of the flows of events]. 
\textit{Otkrytoe obrazovanie} [Open Education] 1:39--45.
\bibitem{3-zat-1}
\Aue{Beer, S.} 1972. \textit{Brain of the firm}. Allen Lane, The Penguin Press, London; Herder 
and Herder, USA. 416~p. 

\bibitem{5-zat-1}
\Aue{Grant, Т., and B. Кoote.} 2005. Comparing OODA \& other models as operational view C2 
architecture. \textit{10th Command and Control Research Technology Symposium 
(International) Proceedings}. McLean, VA. USA. 
\bibitem{4-zat-1}
\Aue{Ivlev, A.\,A.} 2008. \textit{Osnovy teorii Boyda. Napravleniya razvitiya, primeneniya 
i~realizatsii} [Fundamentals of the theory of Boyd. Areas of development, application, and 
implementation]. SlideShare. Available at: {\sf http://www.slideshare.net/defensenetwork/ss-10380168} (accessed  October~29, 2016).
\bibitem{6-zat-1}
\Aue{Bosov, A.\,V., A.\,A.~Zatsarinny, A.\,P.~Suchkov}. 2010. Nekotorye obshchie podkhody 
k~formirovaniyu funktsional'nykh trebovaniy k~situatsionnym tsentram i~ikh realizatsii [Some 
common approaches to the formation of functional requirements for situation centers and their 
implementation]. \textit{Sistemy i~Sredstva Informatiki~--- Systems and Means of Informatics} 
20(3):98--125.
\bibitem{7-zat-1}
\Aue{Suchkov, A.\,P.} 2013. Formirovanie sistemy tseley dlya si\-tu\-a\-tsi\-on\-no\-go upravleniya 
[The formation of the objective system to situational management]. \textit{Sistemy i~Sredstva 
Informatiki~--- Systems and Means of Informatics} 23(2):171--182.
\bibitem{8-zat-1}
\Aue{Zatsarinny, A.\,A., A.\,P.~Suchkov, and S.\,V.~Kozlov}. 2012. Osobennosti proektirovaniya 
i~funktsionirovaniya sistemy situatsionnykh tsentrov [Features of the design and functioning of 
the situational centers ]. \textit{Sistemy Vysokoy Dostupnosti} [High Availability Systems]  
8(1):12--21.
\bibitem{9-zat-1}
\Aue{Doran, G.\,T.} 1981. There's a~S.M.A.R.T. way to write management's goals and 
objectives. \textit{Manag. Rev.} 70(11):35--36.
\end{thebibliography}

 }
 }

\end{multicols}

\vspace*{-6pt}

\hfill{\small\textit{Received August 23, 2016}}

\vspace*{-12pt}

\Contr

\noindent
\textbf{Zatsarinny Alexander A.} (b.\ 1951)~--- Doctor of Science in technology, 
professor, 
Deputy Director, Federal Research Center ``Computer Sciences and Control'' of the 
Russian Academy of Sciences, 44-2~Vavilov Str., Moscow 119333, Russian Federation; 
\mbox{AZatsarinny@ipiran.ru}


\vspace*{3pt}


\noindent
\textbf{Suchkov Alexander P.} (b.\ 1954)~--- Doctor of Science in technology, 
leading scientist, Institute of Informatics Problems, Federal Research Center 
``Computer Science and Control'' of the 
Russian Academy of Sciences, 44-2~Vavilov Str., Moscow 119333, 
Russian Federation; \mbox{Asuchkov@ipiran.ru}

 


\label{end\stat}


\renewcommand{\bibname}{\protect\rm Литература} 