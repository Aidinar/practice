
\def\stat{shnurkov}

\def\tit{АНАЛИТИЧЕСКОЕ РЕШЕНИЕ ЗАДАЧИ ОПТИМАЛЬНОГО УПРАВЛЕНИЯ ПОЛУМАРКОВСКИМ ПРОЦЕССОМ\\ 
С~КОНЕЧНЫМ МНОЖЕСТВОМ СОСТОЯНИЙ$^*$}

\def\titkol{Аналитическое решение задачи оптимального управления полумарковским 
процессом} %с~конечным множеством состояний}

\def\aut{П.\,В.~Шнурков$^1$, А.\,К.~Горшенин$^2$, В.\,В.~Белоусов$^3$}

\def\autkol{П.\,В.~Шнурков, А.\,К.~Горшенин, В.\,В.~Белоусов}

\titel{\tit}{\aut}{\autkol}{\titkol}

\index{Шнурков П.\,В.}
\index{Горшенин А.\,К.}
\index{Белоусов В.\,В.}
\index{Shnurkov P.\,V.}
\index{Gorshenin A.\,K.}
\index{Belousov V.\,V.}


{\renewcommand{\thefootnote}{\fnsymbol{footnote}} \footnotetext[1]
{Работа выполнена при частичной поддержке РФФИ (проект 15-07-05316).}}


\renewcommand{\thefootnote}{\arabic{footnote}}
\footnotetext[1]{Национальный исследовательский университет <<Высшая школа экономики>>, 
\mbox{pshnurkov@hse.ru}}
\footnotetext[2]{Институт проблем информатики Федерального исследовательского центра <<Информатика 
и~управ\-ле\-ние>> Российской академии наук, \mbox{agorshenin@frccsc.ru}}
\footnotetext[3]{Институт проблем информатики Федерального исследовательского центра <<Информатика 
и~управление>> Российской академии наук, \mbox{vbelousov@ipiran.ru}}

%\vspace*{-6pt}

\Abst{Настоящее исследование посвящено теоретическому обоснованию нового метода 
нахождения оптимальной стратегии управления полумарковским процессом с~конечным 
множеством состояний. Рассматриваются марковские рандомизированные стратегии 
управления, определяемые конечным набором вероятностных мер, соответствующих 
каждому состоянию. Характеристикой качества управления служит стационарный 
стоимостной показатель. Данный показатель представляет собой дроб\-но-ли\-ней\-ный 
интегральный функционал от набора вероятностных мер, задающих стратегию управления. 
Для этого функционала известны явные аналитические представления подынтегральных 
функций числителя и~знаменателя. Дальнейшие результаты основываются на новой 
усиленной и~обобщенной форме теоремы об экстремуме дроб\-но-ли\-ней\-но\-го интегрального 
функционала. Доказывается, что проблемы существования оптимальной стратегии управления 
полумарковским процессом и~ее нахождения сводятся к~задаче численного исследования 
на глобальный экстремум заданной функции от конечного числа вещественных переменных.}

\KW{оптимальное управление полумарковским процессом; стационарный стоимостной 
показатель качества управления; дроб\-но-ли\-ней\-ный интегральный функционал}

\DOI{10.14357/19922264160408} 

\vspace*{9pt}


\vskip 10pt plus 9pt minus 6pt

\thispagestyle{headings}

\begin{multicols}{2}

\label{st\stat}

\section{Введение}

Теория оптимального управления марковскими и~полумарковскими случайными 
процессами интенсивно развивается с~начала 1960-х~гг. Еще в~первых 
основополагающих исследованиях рассматривались не только проблемы существования 
оптимальных стратегий управления, но и~способы нахождения этих стратегий. 

Для решения таких проблем, имеющих алгоритмическое содержание, использовались 
открытые незадолго до этого мощные методы прикладной математики: линейное 
программирование и~динамическое программирование. Отметим, прежде всего, 
классическую работу Р.~Ховарда~\cite{1}, в~которой метод динамического 
программирования был применен для решения проблемы оптимального управления 
марковским процессом с~непрерывным временем. В~дальнейшем В.\,В.~Рыков~\cite{2} 
доказал, что для аналогичной модели управления марковским процессом с~учетом 
переоценки оптимальной стратегией также является стационарная.

Важную роль в~развитии теории управления случайными процессами сыграла работа 
В.~Джевелла~\cite{3}, в~которой были впервые рассмотрены полумарковские модели 
управления для вариантов с~переоценкой и~без переоценки. Данная работа была 
переведена на русский язык и~послужила основой для многих последующих работ 
отечественных и~зарубежных специалистов. В~частности, Б.~Фокс показал~\cite{4}, 
что оптимальной стратегией управления полумарковским процессом в~варианте без 
переоценки является стационарная; аналогичные результаты были получены Э.~Денардо 
и~для варианта с~переоценкой~\cite{5}.

Среди последующих исследований алгоритмической направленности отметим работы 
Р.~Ховарда~\cite{6}, Б.~Фокса~\cite{4}, а также С.~Осаки и~Х.~Майна~\cite{7}. 
В~этих работах для нахождения оптимальных стратегий управления полумарковскими 
процессами использовался метод линейного программирования.

В 1970~г.\ была опубликована фундаментальная монография Х.~Майна и~С.~Осаки~\cite{8}, 
переведенная на русский язык в~1977~г., в~которой были систе\-ма\-ти\-зи\-ро\-ва\-ны и~изложены 
основные результаты по теории оптимального управления марковскими и~полумарковскими 
случайными процессами. Фактически данная книга стала итогом исследований по проблемам 
стохастического управления\linebreak
 за~10~лет. Отметим, что в~этой монографии рас\-смат\-ри\-ва\-лись 
марковские и~полумарковские модели управления с~конечными множествами состояний 
и~допустимых решений, принимаемых \mbox{в~каждом} состоянии. Были получены принципиальные 
тео\-ре\-ти\-че\-ские результаты, заключающиеся в~том, что оптимальные стратегии управ\-ле\-ния 
для основных видов рас\-смат\-ри\-ва\-емых моделей с~переоценкой и~без переоценки являются 
детерминированными и~стационарными. Были разработаны и~обоснованы процедуры нахождения 
оптимальных стратегий управления. В~частности, для модели управления полумарковским 
процессом без переоценки, когда множество со\-сто\-яний образует один эргодический класс, 
а~показатель качества управления пред\-став\-ля\-ет собой стационарный средний удельный 
доход (см.~[8, гл.~5, п.~5.5]), процедура поиска оптимальной рандомизированной 
стратегии осуществлялась методом линейного программирования. Обратим особое внимание 
на данный результат, поскольку аналогичная модель управления полумарковским 
процессом будет рассмотрена в~настоящей работе.

Принципиальную роль в~развитии теории стохастического управления сыграла 
монография И.\,И.~Гихмана и~А.\,В.~Скорохода~\cite{9}. В~этой книге были впервые 
систематически изложены основы теории оптимального управления случайными процессами 
с~дискретным и~непрерывным временем, включая теорию управления процессами, которые 
описываются стохастическими дифференциальными уравнениями. Отдельно были рас\-смот\-ре\-ны 
проблемы управления марковскими процессами с~дискретным временем и~скачкообразными 
марковскими процессами с~непрерывным временем. Роли множеств состояний и~допустимых 
управ\-ле\-ний играли пространства весьма общей структуры. Для широких классов функционалов 
качества управ\-ле\-ния (так называемых эволюционных функционалов в~марковских моделях 
с~дискретным временем и~интегральных функционалов накопления в~марковских моделях 
с~непрерывным временем) были доказаны теоремы о~существовании и~формах пред\-став\-ле\-ния 
оптимальных стратегий управ\-ле\-ния. Было установлено, что для однородных марковских 
моделей оптимальные стратегии управ\-ле\-ния существуют, являются стационарными 
и~детерминированными. Иначе говоря, такие стратегии задаются детерминированными 
функциями, аргументом которых является со\-сто\-яние сис\-те\-мы в~момент принятия решения, 
и~не зависящими от самого момента принятия решения. Что же касается важного вопроса 
о~формах представления этих функций, то их можно охарактеризовать следующим образом. 
Были найдены функциональные уравнения, осложненные условием экстремума, которым 
удовле\-тво\-ря\-ют упомянутые функции. По существу эти соотношения пред\-став\-ля\-ют собой 
уравнения Беллмана для соответствующих динамических стохастических моделей.

Особо отметим, что в~монографии~\cite{9} не рас\-смат\-ри\-ва\-лись проблемы управления 
полумарковскими процессами. Однако дальнейшее развитие общей теории управления 
такими процессами шло по пути, идейно намеченному в~указанной книге.

В последующие годы развитие теории управ\-ле\-ния полумарковскими процессами 
осуществля-\linebreak лось по направлению усложнения моделей и~обобщения исходных предположений. 
Например,\linebreak в~работах~\cite{10, 11} рассмотрены управляемые по\-лумарковские процессы при 
весьма общих предположениях относительно характера пространств состояний и~управлений. 
Проблемы управления исследовались по отношению к~различным видам целевых показателей, 
обобщающих упомянутый выше стационарный показатель средней удельной прибыли. В~этих 
работах доказывается, что оптимальная стратегия управления по отношению к~каж\-до\-му из 
показателей существует и~является одной и~той же стационарной детерминированной 
стратегией, определяемой некоторой функцией, заданной на множестве со\-сто\-яний процесса. 
Об этой функции известно лишь то, что она удовлетворяет некоторому интегральному 
уравнению, которое по содержанию пред\-став\-ля\-ет собой уравнение Бел\-лма\-на для 
соответствующей задачи управ\-ления.

Среди исследований, предшествовавших настоящему, отметим работу 
В.\,А.~Каштанова~[12, гл. 13]. В этом разделе коллективной монографии~\cite{12} 
автором была рассмотрена проблема оптимального управления полумарковским 
процессом с~конечным множеством состояний и~множеством возможных решений, 
которое представляет собой произвольный интервал множества вещественных чисел. 
Модель относится к~виду моделей без переоценки, показателем качества управления 
служит стационарное значение среднего удельного дохода, определяемое аналогично 
классическим работам~\cite{3, 8}. Рандомизированное управление в~каждом состоянии 
определяется в~соответствии с~вероятностным распределением, совокупность которых 
задает\linebreak
 стратегию управления. В.\,А.~Каш\-та\-но\-вым было\linebreak сформулировано утверждение о том, 
что стацио\-нарное значение среднего удельного дохода представляет собой 
дроб\-но-ли\-ней\-ный интегральный функционал от набора вероятностных распределений, 
образующих стратегию управления. При этом\linebreak ранее~[12, гл.~10; 13] было уста\-нов\-ле\-но, 
что дроб\-но-ли\-ней\-ный функционал достигает экстремума на вырожденных распределениях. 
Отсюда естест-\linebreak венно следует, что оптимальная стратегия управ\-ле-ния является 
детерминированной и~должна\linebreak определяться точкой экстремума функции, представляющей 
собой отношение подынтегральных функций чис\-ли\-те\-ля и~знаменателя данного 
дроб\-но-ли\-ней\-но\-го функционала. Однако в~\cite{12} не были получены явные 
представления для указан-\linebreak ных функций. Кроме того, приведенный в~гл.~10 
монографии~\cite{12} вариант теоремы об экстремуме дроб\-но-ли\-ней\-но\-го 
интегрального функционала требовал проверки выполнения условия существования 
этого экстремума. Такие условия указаны не были. В~связи с~этими обстоятельствами 
использовать полученные в~\cite{12} результаты для доказательства существования 
оптимальной детерминированной стратегии управ\-ле\-ния полумарковским процессом и~для 
строгого обоснования способа нахождения такой стратегии оказалось невозможным.

Настоящее исследование посвящено теоретическому обоснованию нового метода 
нахождения\linebreak оптимальной стратегии управления полумарковским процессом с~конечным 
множеством со\-сто\-яний. Рассматриваются марковские рандомизи\-рованные стратегии 
управления, определяемые конеч\-ным набором вероятностных мер, соответствующих 
каждому состоянию. Показателем качества управления служит уже упоминавшийся 
классический  показатель~--- стационарное значение средней удельной прибыли. 
Доказано, что этот показатель представляет собой дроб\-но-ли\-ней\-ный интегральный 
функционал от набора вероятностных мер, задающих стратегию управления. При этом, 
в~отличие от~\cite{12}, получены явные аналитические представления для подынтегральных 
функций числителя и~знаменателя этого дроб\-но-ли\-ней\-но\-го\linebreak
 функционала. Дальнейшие 
результаты основываются на новой усиленной и~обобщенной форме\linebreak
 теоремы об экстремуме 
дроб\-но-ли\-ней\-но\-го интегрального функционала, впервые опубликованной 
в~работе П.\,В.~Шнуркова~\cite{14}. Согласно\linebreak
 утверж\-де\-нию этой теоремы, если 
существует глобальный экстремум так называемой основной функции дроб\-но-ли\-ней\-но\-го 
функционала, которая пред\-став\-ля\-ет собой отношение подынтегральных функций чис\-ли\-те\-ля 
и~знаменателя, то существует безусловный экстремум самого дроб\-но-ли\-ней\-но\-го 
функционала, который достигается на наборе вырожденных вероятностных распределений, 
сосредоточенных в~точке глобального экстремума. В~этом случае оптимальная стратегия 
управ\-ле\-ния существует, является стационарной и~детерминированной и~определяется точкой, 
в~которой основная\linebreak функция достигает глобального экстремума. Таким\linebreak образом, проблемы 
существования оптимальной стратегии управ\-ле\-ния полумарковским процессом и~ее 
нахождения сводятся к~задаче чис\-лен\-но\-го исследования на глобальный экстремум 
заданной функции от конечного чис\-ла вещественных переменных.

\section{Общее описание модели управления полумарковским случайным процессом}

Построим модель управления полумарковским случайным процессом, следуя общему 
подходу, принятому в~классических работах~\cite{3, 8}. Пусть $\xi(t)$~--- 
случайный полумарковский процесс с~конечным множеством состояний
$X\hm=\{1,2,\ldots, N\}$, $N\hm< \infty$. Обозначим через~$t_n$, $n=0,1,2,\ldots$, 
$t_0\hm=0$, случайные моменты изменения состояний данного процесса, 
$\theta_n\hm=t_{n+1}-t_n$, $n\hm=0,1,2,\ldots$, $\xi_n\hm=\xi(t_n)\hm=\xi(t_n+0)$, 
$n\hm=0,1,2,\ldots$ (предполагается, что траектории процесса~$\xi(t)$ 
непрерывны справа). Случайная последовательность~$\{\xi_n\}$
образует цепь Маркова, вложенную в~полумарковский процесс~$\xi(t)$.
Зададим набор измеримых пространств\linebreak $(U_1, \mathscr{B}_1), 
(U_2, \mathscr{B}_2), \ldots, (U_N, \mathscr{B}_N)$, где $U_i$~--- 
множество возможных допустимых управ\-ле\-ний, $\mathscr{B}_i$~--- $\sigma$-ал\-геб\-ра 
подмножеств множества~$U_i$, вклю\-ча\-ющая в~себя все одноточечные подмножества\linebreak  
множества~$U_i$, т.\,е.\ если $u_i \hm\in U_i$, то $\{u_i\} \hm\in \mathscr{B}_i$, 
$i\hm=1,2,\ldots, N$.
Пусть $\Gamma_i$~--- некоторое множество всевозможных вероятностных мер $\Psi_i 
\hm \in \Gamma_i$, заданных на $\sigma$-ал\-геб\-ре~$\mathscr{B}_i$, $i\hm=1,2,\ldots,N$.

Поскольку идейное содержание и~свойства вероятностных мер существенно используются 
в~данной работе, укажем на некоторые фундаментальные издания, в~которых 
изложена соответствующая тео\-рия. Понятие и~основные свойства вероятностной 
меры определены и~подробно проанализированы в~книге А.\,Н.~Ширяева~\cite[гл.~II]{15}. 
Глубокое изложение основ теории вероятностных мер имеется также в~книге 
А.\,А.~Боровкова~\cite{16}. Заметим попутно, что в~книге~\cite{16} имеются разделы, 
посвященные изложению основ теории полумарковских и~регенерирующих случайных процессов. 
Из зарубежных изданий отметим фундаментальную работу П.~Хеннекена и~А.~Тортра~\cite{17}, 
основная часть которой посвящена изложению математических основ теории вероятностей.

Введем специальное понятие вырожденной вероятностной меры, которое будет часто 
использоваться в~дальнейшем. Пусть $(U_0, \mathscr{B}_0)$~--- некоторое измеримое 
пространство, $\mathscr{B}_0$~--- $\sigma$-ал\-геб\-ра подмножеств множества~$U_0$, 
включающая в~себя все одноточечные подмножества этого множества.

\medskip

\noindent
\textbf{Определение 1.}\ Вероятностная мера~$\Psi^*$, заданная 
на $\sigma$-ал\-геб\-рe~$\mathscr{B}_0$, называется вырожденной, если существует 
такой элемент $u^* \hm\in U_0$, для которого выполняются условия $\Psi^*(\{u^*\})\hm=
1$, $\Psi^*(U_0 \setminus \{u^*\})\hm=0$, где $\{u^*\}=u^*$~--- 
множество, состоящее из единственной точки $u^* \hm\in U_0$. Соответствующая 
точка $u^* \hm\in U_0$ будет называться точкой сосредоточения вырожденной 
вероятностной меры~$\Psi^*$.
Таким образом, всякая вырожденная вероятностная мера~$\Psi^*$ определяется 
своей точкой сосредоточения~$u^*$. В~дальнейшем будем использовать 
обозначение~$\Psi_{u^*}^{*}$, имея в~виду, что вырожденная вероятностная мера~$\Psi^*$ 
сосредоточена в~точке~$u^*$.
Отметим также, что вырожденная вероятностная мера~$\Psi_{u^*}^{*}$ соответствует 
детерминированной величине, которая принимает фиксированное значение $u\hm=u^*$ 
с~вероятностью, равной единице.

\medskip

Обозначим через $\Gamma_0$ множество всех  вероятностных мер, заданных 
на измеримом пространстве ($U_0, \mathscr{B}_0$), а через~$\Gamma_0^*$~--- 
множество всех вырожденных вероятностных мер, заданных на этом пространстве, 
$\Gamma_0^*\hm\in \Gamma_0$. Аналогичные обозначения будут использоваться 
и~в~дальнейшем. Заметим, что множество~$\Gamma_0^*$ находится во взаимно
 однозначном соответствии с~множеством точек сосредоточения вырожденных 
 вероятностных мер, т.\,е.\ с~множеством~$U_0$.

Пусть $\Gamma_i^{*}$~--- множество всех вырожденных мер, заданных на 
$\sigma$-ал\-геб\-ре~$\mathscr{B}_i$, $\Gamma_i^{*}\hm\subset \Gamma_i$.
Произвольная вероятностная мера~$\Psi_i$ описывает случайную величину, 
принимающую значения в~$U_i$, а вырожденная мера~$\Psi_i^*$, сосредоточенная 
в~точке~$u_i^*$, соответствует детерминированной величине $u_i^*\hm\in U_i$.
Предполагается, что соответствующие конструкции определены на всех измеримых 
пространствах управлений $(U_1, \mathscr{B}_1), (U_2, \mathscr{B}_2), \ldots, 
(U_N,\mathscr{B}_N)$.

Предположим, что управления случайным полумарковским процессом~$\xi(t)$ 
осуществляются в~моменты времени~$t_n,$ $n\hm=0,1,2,\ldots,$
непосредственно после изменения состояния процесса. Если\linebreak 
$\xi_n\hm=\xi(t_n)\hm=i \hm\in X$, то значение управления представляет 
собой случайную величину~$u_n$, принимающую значения в~множестве допустимых 
управ\-ле\-ний~$U_i$ и~описываемую вероятностной мерой (распределе\-ни\-ем 
вероятностей) $\Psi_i \hm\in \Gamma_i$.
Будем предполагать, что при фиксированном условии $\xi_n\hm=\xi(t_n)=i$ 
управ\-ле\-ние определяется независимо от прошлого поведения процесса~$\xi(t)$ 
и~вероятностная мера~$\Psi_i$,
описывающая стохастическое управление~$u_n$, зависит только от состояния $i\hm\in X$.
Тогда выбор управ\-ле\-ний в~моменты изменения состояний $\{t_n, n\hm=0,1,2,\ldots \}$ 
описывается набором вероятностных мер (распределений вероятностей) 
$(\Psi_1, \Psi_2,\ldots, \Psi_N)$, 
$\Psi_i \hm\in \Gamma_i$, $i\hm=1,2,\ldots,N$.
Назовем любой такой набор стратегией управ\-ле\-ния полумарковским процессом~$\xi(t)$. 
По своим свойствам такая стратегия является марковской, однородной 
и~рандомизированной.

Следуя классической монографии П.~Халмоша~\cite[гл.~VII]{18}, 
рассмотрим декартово произведение пространств $U\hm=U_1\times U_2\times \cdots\times U_N$ 
и~соответствующих $\sigma$-ал\-гебр $\mathscr{B}\hm=\mathscr{B}_1\times \mathscr{B}_2
\times \cdots \times\mathscr{B}_N$. Обозначим через $\Psi\hm=\Psi_1\times \Psi_2\times \cdots
\times \Psi_N$ вероятностную меру на~$(U,\mathscr{B})$, определяемую как 
произведение мер $\Psi_1,\Psi_2, \ldots , \Psi_N$, где $\Psi_i \hm\in \Gamma_i$, 
$i\hm=1,2,\ldots,N$. Обозначим также через~$\Gamma$ множество вероятностных мер~$\Psi$, 
заданных на~$(U,\mathscr{B})$, которые пред\-став\-ля\-ют собой произведение 
мер $\Psi_1,\Psi_2, \ldots , \Psi_N$, где $\Psi_i \hm\in \Gamma_i$, $i\hm=1,2,\ldots,N$.
Множество~$\Gamma$ можно отож\-де\-ст\-вить с~множеством всех стратегий управ\-ле\-ния 
полумарковским процессом~$\xi(t)$.

Проблема оптимального управления полумар\-ковским процессом~$\xi(t)$ будет в~дальнейшем 
сформулирована в~виде задачи безусловного экстремума некоторого функционала 
$I(\Psi)\hm=I(\Psi_1,\Psi_2, \ldots , \Psi_N)$, заданного на множестве 
допустимых стратегий управления. Содержание показателя качества управления~$I(\Psi)$, 
аналитическое представление для него, а~также описание множества допустимых 
стратегий управления будут приведены в~последующих разделах данной работы.

Для получения дальнейших результатов потребуются различные вероятностные 
характеристики управляемого полумарковского процесса~$\xi(t)$. Как известно из
 общей теории полумарковских процессов~\cite{19, 20}, 
 основной вероятностной характеристикой такого процесса является так называемая 
 полумарковская функция. Определим эту функцию для процесса с~управлением 
 (см.~\cite[гл.~5]{8}):
\begin{multline}
Q_{ij}(t,u)=
{\sf P}\left(\xi_{n+1}=j,\theta_n<t \mid \xi_n=i, u_n=u\right)\,,\\
t\in [0,\infty)\,,\ u\in U_i\,;\ i,j\in X=\{1,2,\ldots,N\}\,. \label{e1}
\end{multline}
Используя полумарковские функции, можно получить вероятности перехода 
управляемой цепи Маркова~$\{\xi_n\}$:
\begin{multline}
p_{ij}(u)={\sf P}\left(\xi_{n+1}=j \mid \xi_n=i, u_n=u\right)= {}\\
{}=
\lim\limits_{t\rightarrow\infty}Q_{ij}(t,u)\,,\enskip
u\in U_i\,;\enskip i,j\in X\,, 
\label{e2}
\end{multline}
а также функции распределения длительностей пребывания полумарковского 
процесса~$\xi(t)$ в~соответствующих состояниях:

\noindent
\begin{multline}
H_{i}(t,u)={\sf P}\left(\theta_n<t \mid \xi_n=i, u_n=u\right)={}\\
{}=
\sum\limits_{j\in X}Q_{ij}(t,u)\,,\enskip
t\in [0,\infty)\,,\  u\in U_i\,; \  i\in X\,. 
\label{e3}
\end{multline}

Обозначим через
\begin{multline}
T_{i}(u)=\mathbf{E}\left[\theta_n \mid \xi_n=i, u_n=u\right]={}\\
{}=
\int\limits_0^{\infty}\left[1-H_i(t,u)\right]\,dt\,,\enskip
u\in U_i\,,\ i\in X\,, 
\label{e4}
\end{multline}
математические ожидания длительностей пребывания полумарковского процесса~$\xi(t)$ 
в~каждом из состояний.

Введенные выше характеристики~(1)--(4) определены для случая, когда 
в~момент изменения состояния~$t_n$ процесс оказывается в~состоянии~$i$ 
и~принимается решение $u\hm\in U_i$. При заданной стратегии управления 
$\Psi\hm=\left(\Psi_1,\Psi_2, \ldots , \Psi_N\right)$ можно записать 
соответствующие вероятностные характеристики без условия на управление, а~именно:
\begin{multline*}
Q_{ij}(t)={\sf P}\left(\xi_{n+1}=j,\theta_n<t \mid \xi_n=i\right)={}\\
{}=
\int\limits_{U_i}Q_{ij}(t,u) \,d\Psi_i(u)\,,\enskip 
t\in [0,\infty)\,,\ i,j\in X\,; %\label{e5}
\end{multline*}

\vspace*{-12pt}

\noindent
\begin{multline}
p_{ij}={\sf P}\left(\xi_{n+1}=j \mid \xi_n=i\right)=
\int\limits_{U_i}p_{ij}(u)\, d\Psi_i(u)\,,\\  
i,j\in X\,; 
\label{e6}
\end{multline}

\vspace*{-9pt}

\noindent
\begin{equation}
T_{i}=\mathbf{E}\left[\theta_n \mid \xi_n=i\right]=
\int\limits_{U_i}T_{i}(u)\,d\Psi_i(u)\,,\enskip i\in X\,. 
\label{e7}
\end{equation}
В дальнейшем будем предполагать, что для рас\-смат\-ри\-ва\-емой 
полумарковской модели заданы вероятностные характеристики 
$p_{ij}(u)$, $u\hm\in U_i$, $i,j\hm\in X$, и~$T_i(u)$, $u\hm\in U_i$, $i\hm\in X$, 
определяемые соотношениями~(\ref{e2}) и~(\ref{e4}). 
Для фиксированной стратегии управления $\Psi\hm=(\Psi_1, \Psi_2,\ldots, \Psi_N)$ 
соответствующие вероятностные характеристики~$p_{ij}$ и~ $T_i$, $i,j\hm\in X,$ 
определены равенствами~(\ref{e6}) и~(\ref{e7}) без условий на управление.

\section{Стационарный стоимостной показатель качества управления}

Определим некоторый стоимостной аддитивный функционал, связанный 
с~рассматриваемым полумарковским процессом~$\xi(t)$. По содержанию этот функционал 
представляет собой случайный\linebreak доход или прибыль, накопленную за период времени $[0,t]$. 
Определения такого функционала приведены в~основополагающих работах~[3; 8, гл.~5].\linebreak 
Обозначим через $\widetilde{v}(t)$, $t\hm\geq 0$, значение этого аддитивного 
функционала в~момент времени~$t$; $\widetilde{v}_n\hm=\widetilde{v}(t_n\hm+0)$~--- 
соответствующее значение непосредственно после очередного момента изменения\linebreak 
состояния~$t_n$, $n\hm=0,1,2,\ldots$; $\widetilde{v}_0\hm=v_0$~--- 
заданное начальное значение в~момент $t\hm=0$. Рассмотрим величину
\begin{multline}
d_i(u)=\mathbf{E}\left[\widetilde{v}_{n+1}-\widetilde{v}_n \mid \xi_n=i\,, 
u_n=u\right]\,,\\
u\in U_i\,, \enskip i\in X\,, \label{e8}
\end{multline}
представляющую собой математическое ожидание приращения стоимостного 
аддитивного функционала за период времени между последовательными 
изменениями состояния полумарковского процесса~$\xi(t)$. Тогда соответствующее 
математическое ожидание, вычисляемое без условия на решение, 
принимаемое в~момент времени~$t_n$, представляется в~виде:
\begin{equation*}
d_i=\mathbf{E}\left[\widetilde{v}_{n+1}-\widetilde{v}_n \mid \xi_n=i\right]=
\!\int\limits_{U_i}\!d_i(u)\,d\Psi_i(u)\,,\ i\in X \,. %\label{e9}
\end{equation*}

Предположим, что для заданной стратегии управ\-ле\-ния 
$\Psi\hm=(\Psi_1,\Psi_2,\ldots,\Psi_N)$ вложенная цепь Маркова~$\{\xi_n\}$ 
имеет ровно один класс возвратных положительных состояний (по терминологии, 
принятой в~\cite{8}, такое множество состояний называется эргодическим классом). 
Как известно~\cite[гл.~VIII]{15}, данное условие является необходимым 
и~достаточным для существования единственного\linebreak стационарного распределения. 
Обозначим это стационарное распределение цепи Маркова~$\{\xi_n\}$ через 
$\pi\hm=(\pi_1, \pi_2,\ldots, \pi_N)$. Заметим, что данное\linebreak распределение зависит  
от стратегии управления $\Psi\hm=(\Psi_1,\Psi_2,\ldots,\Psi_N)$. При указанном 
условии имеет место следующий результат, называемый эргодической теоремой 
для аддитивного стоимостного функционала:
\begin{equation}
I=\lim\limits_{t\rightarrow\infty}\fr{\mathbf{E}\widetilde{v}(t)}{t}=
\fr{\sum\nolimits_{i=1}^N d_i\pi_i}{\sum\nolimits_{i=1}^N T_i\pi_i}\,. 
\label{e10}
\end{equation}

Соотношение~(\ref{e10}) доказано в~работе~\cite[гл.~5]{8}. Заметим, что аналогичные 
результаты имеют мес\-то для гораздо более общих полумарковских моделей~\cite{10, 11}.

По своему прикладному содержанию величина, определяемая соотношением~(\ref{e10}), 
представляет собой
среднюю удельную прибыль, связанную с~эволюцией системы в~стационарном
режиме. Кроме того, величина~$I$ представляет собой функционал от
набора вероятностных распределений~$\Psi_{i}$, $i\hm\in\lbrace 1,\ldots
,N\rbrace $, определяющих стратегию управле-\linebreak\vspace*{-12pt}

\pagebreak

\noindent
ния системой. 
В~дальнейшем будем рассматривать стационарный стоимостной функционал 
$I\hm=I(\Psi_{1},\Psi_{2},\ldots , \Psi_{N})$ как
показатель качества управ\-ле\-ния системой и~построенным полумарковским
процессом~$\xi (t)$.

\section{Представление стационарного показателя в~форме
дробно-линейного интегрального функционала}

В данном разделе будет приведено утверждение об аналитическом
представлении стационарного стоимостного функционала~(\ref{e10}), 
служащего критерием качества управления в~рассматриваемой задаче управления 
полумарковским процессом.

\smallskip

\noindent
\textbf{Теорема 1.} \textit{Стационарный стоимостной показатель, 
определяемый равенством}~(\ref{e10}), \textit{представляет собой дроб\-но-ли\-ней\-ный
функционал от вероятностных распределений~$\Psi_{i}(u_{i})$,
$i\hm\in\{1,\dots,N\}$. Данный функционал задается
аналитически следующей формулой:}
\begin{multline}
I=I(\Psi_{1},\ldots, \Psi_{N})={}\\
\hspace*{-2mm}{}=\!
\fr{\int\nolimits_{U_1}\!{\cdots\! 
\int\nolimits_{U_N}\!{A(u_{1},\ldots ,u_{N})d\Psi_{1}(u_{1})\cdots
\,d\Psi_{N}(u_{N})}}}{\int\nolimits_{U_1}{\!\cdots\! \int\nolimits_{U_N}\!{B(u_{1},\ldots ,u_{N})
\,d\Psi_{1}(u_{1})\ldots
d\Psi_{N}(u_{N})}}},\!\!\! \label{e11}
\end{multline}
\textit{где подынтегральные функции числителя и~знаменателя выражаются
соотношениями}:
\begin{align}
A(u_{1},\ldots
,u_{N})&={}\notag\\
&\hspace*{-20mm}{}=\sum\limits_{i=1}^{N}{d_{i}(u_{i})}{\widehat{D}}^{(i)}(u_{1}, \ldots
,u_{i-1},u_{i+1}, \ldots , u_{N})\,;  \label{e12}\\
 B(u_{1},\ldots
,u_{N})&={}\notag\\
&\hspace*{-20mm}{}=\sum\limits_{i=1}^{N}{T_{i}(u_{i})}{\widehat{D}}^{(i)}(u_{1}, \ldots
,u_{i-1},u_{i+1}, \ldots , u_{N})\,.  \label{e13}
\end{align}
\textit{В свою очередь, функции} ${\widehat{D}}^{(i)}(u_{1}, \ldots
,u_{i-1},u_{i+1}, \ldots$\linebreak $\ldots , u_{N})$, $i\hm\in\{1,\dots,N\}$, 
\textit{входящие в~правые части формул}~(\ref{e12}) и~(\ref{e13}), 
\textit{определяются следующим образом:}

\noindent
\begin{multline}
{\widehat{D}}^{(i)}(u_{1}, \ldots ,u_{i-1},u_{i+1}, \ldots , u_{N})={}
\\
{}=(-1)^{N+i+2}\sum\limits_{\alpha ^{(N),i}}{(-1)}^{\delta (\alpha
^{(N),i}) }{\widehat{D}}_{0}^{(i)}\left(\alpha ^{(N),i},u_{1}, \ldots\right.\\
\left.\ldots , u_{i-1},u_{i+1}, \ldots , u_{N}\right)\,. \label{e14}
\end{multline}
\textit{Здесь} $\alpha ^{(N),i}=(\alpha _{1}, \ldots , \alpha _{i-1},\alpha_{i+1}, \ldots , 
\alpha _{N})$~\textit{--- произвольная
перестановка чисел }$(1, \ldots , i-1, i+1, \ldots , N)$;
$\delta
(\alpha ^{(N),i})$~\textit{--- число инверсий в~перестановке} 
$\alpha ^{(N),i}$;

\noindent
\begin{multline}
{\widehat{D}}_{0}^{(i)}(\alpha ^{(N),i},u_{1}, \ldots ,u_{i-1},u_{i+1},
\ldots , u_{N})={}\\
{} ={\widetilde{p}}_{1,\alpha _{1}}\left(u_{1}\right)\cdots {\widetilde{p}}_{i-1,\alpha
_{i-1}}\left(u_{i-1}\right){\widetilde{p}}_{i+1,\alpha _{i+1}}\left(u_{i+1}\right)\cdots\\
\cdots
{\widetilde{p}}_{N,\alpha _{N}}\left(u_{N}\right)\,, 
\label{e15}
\end{multline}
где
\begin{multline}
 {\widetilde{p}}_{k,\alpha _{k}}(u_{k})=
\begin{cases}
p_{kk}(u_{k})-1,\  & \alpha _{k}=k\,; \\
p_{k,\alpha _{k}}(u_{k}),\  & \alpha _{k}\ne k, \\
\end{cases}\\
 k=1, \ldots , i-1, i+1, \ldots ,N\,. \label{e16}
 \end{multline}
\textit{Функции $p_{ij}(u_i)$, $T_{i}(u_{i})$ и~$d_{i}(u_{i})$,
$u_i\hm\in U_i$, $i,j\hm\in \{1,2,\ldots,N\}$, 
входящие в~соотношения}~(\ref{e12})--(\ref{e16}), 
\textit{определяются равенствами}~(\ref{e2}), (\ref{e4}) \textit{и}~(\ref{e8}) \textit{соответственно.}

\smallskip

\noindent
Д\,о\,к\,а\,з\,а\,т\,е\,л\,ь\,с\,т\,в\,о\ теоремы~1 
в~весьма сжатой форме приведено в~работе~\cite{21}. Читателю, интересующемуся 
более подробным обоснованием данного результата, порекомендуем обратиться к~тексту 
кандидатской диссертации А.\,В.~Иванова~\cite[гл.~3]{22}.

\smallskip

Итак, теорема~1 позволяет получить явное аналитическое представление 
для стационарного стоимостного показателя вида~(\ref{e10}) в~форме 
дроб\-но-ли\-ней\-но\-го интегрального функционала от набора\linebreak вероятностных мер 
$\Psi\hm=(\Psi_{1},\Psi_{2},\ldots , \Psi_{N})$, за\-да\-ющих стратегию управления 
полумарковским процессом~$\xi(t)$. При этом подынтегральные функции числителя 
и~знаменателя задаются формулами~(\ref{e12}), (\ref{e13}) 
и~вспомогательными равенствами~(\ref{e14})--(\ref{e16}). Таким образом, функция
\begin{equation}
C\left(u_1, u_2,\ldots, u_N\right)=\fr{A(u_1, u_2,\ldots, u_N)}{B(u_1, u_2,\ldots, u_N)}\,,
\label{e17}
\end{equation}
которая в~дальнейшем будет называться основной функцией дроб\-но-ли\-ней\-но\-го 
интегрального функционала~(\ref{e11}) и~которая будет играть важную роль 
в~дальнейшем исследовании, также явно определяется формулами~(\ref{e17}), 
(\ref{e12}), (\ref{e13}).

\section{Формальная постановка оптимизационной задачи 
и~условия существования оптимальной стратегии управления полумарковским процессом}

Будем рассматривать проблему управления полумарковским процессом~$\xi(t)$ в~форме 
экстремальной задачи
\begin{multline}
I(\Psi)=I\left(\Psi_1, \Psi_2,\ldots,\Psi_N\right)\rightarrow \mathrm{extr}\,,
\\
\Psi=\left(\Psi_1, \Psi_2,\ldots,\Psi_N\right)\in\Gamma\,. \label{e18}
\end{multline}
При этом показатель качества управления~$I(\Psi)$ представляет собой 
дроб\-но-ли\-ней\-ный интегральный функционал вида~(\ref{e11}).

Для решения экстремальной задачи~(\ref{e18}) воспользуемся некоторым утверждением 
об экстремуме дроб\-но-ли\-ней\-но\-го интегрального функционала. Прежде 
чем сформулировать данное утверждение, отметим, что в~теории оптимизации 
хорошо известны задачи, в~которых целевая функция представляет собой 
отношение двух линейных отображений, а имеющиеся ограничения также линейны. 
Такой раздел называется дроб\-но-ли\-ней\-ным программированием. Основные
 теоретические результаты данного направления изложены в~работе~\cite{23},
  там же приведена подробная библиография. В~дальнейшем потребуется некоторый 
  специальный результат о безусловном экстремуме дроб\-но-ли\-ней\-но\-го 
  интегрального функционала вида~(\ref{e11}), который был впервые сформулирован 
  в~работе~\cite{14}. Заметим, что для использования этого результата необходимо, 
  чтобы выполнялись некоторые предварительные условия, которые в~данном случае 
  можно сформулировать следующим образом:
\begin{enumerate}[1.]
\item Интегральные выражения
\begin{align*}
I_1(\Psi)&=I_1\left(\Psi_1,\Psi_2,\ldots,\Psi_N\right)={}&\\
&\hspace*{-13mm}{}=\int\limits_{U_1}\!\cdots\!
\int\limits_{U_N}\!\!A\left(u_1,\ldots ,u_N\right)\,
d\Psi_1\left(u_1\right) %d\Psi_2\left(u_2\right)
\cdots
 d\Psi_N\left(u_N\right)\,;
\\
I_2(\Psi)&=I_2\left(\Psi_1,\Psi_2,\ldots,\Psi_N\right)={}&\\
&\hspace*{-13mm}{}=\int\limits_{U_1}\!\cdots\!\int\limits_{U_N}\!\!
B\left(u_1,\ldots,u_N\right)\,
d\Psi_1\left(u_1\right)% d\Psi_2\left(u_2\right)\cdots\\
\cdots d\Psi_N\left(u_N\right)
\end{align*}
определены для всех стратегий управления $\Psi\hm=(\Psi_1, \ldots,\Psi_N)
\hm\in \Gamma$.

\item Функционал $I_2(\Psi)=I_2(\Psi_1, \ldots,\Psi_N)\hm\neq 0$ 
для всех $\Psi\hm=(\Psi_1, \ldots,\Psi_N)\hm\in \Gamma$.

\item Множество $\Gamma$ включает в~себя множество всех вырожденных 
вероятностных мер: $\Gamma^* \hm\subset \Gamma$.
\end{enumerate}

Сделаем несколько важных замечаний по поводу введенных предварительных условий.

\smallskip

\noindent
\textbf{Замечание~1.}\ Из условия~2 следует, что функция $B(u_1, u_2,\ldots, u_N)$ 
не может принимать значения разных знаков. С~учетом условия~3 
получаем, что указанная функция должна обладать \mbox{свойством} строгой 
знакопостоянности на всем множестве~$U$. С~другой стороны, если выполняется 
условие строгой знакопостоянности функции $B(u_1, u_2,\ldots, u_N), 
(u_1, u_2,\ldots, u_N)\hm\in U$, то условие~2 выполняется автоматически.

\smallskip

\noindent
\textbf{Замечание~2.}\ Если рассматривать в~качестве целевого функционала 
$I(\Psi_1, \Psi_2,\ldots,\Psi_N)$ экстремальной задачи~(\ref{e18}) 
стационарный стоимостной пока\-затель~(\ref{e10}), то функция $B(u_1,u_2,\ldots,u_N)$ 
имеет\linebreak следующее теоретическое содержание. Данная функция представляет собой условное 
математическое ожидание длительности периода времени между соседними моментами 
изменения со\-сто\-яния полумарковского процесса~$\xi(t)$ при условии, что стратегия 
его управ\-ле\-ния является детерминированной и~задается набором значений аргументов 
$(u_1,u_2,\ldots,u_N)$. Тогда условие строгой положительности функции 
$B(u_1,u_2,\ldots,u_N)$ при всех $(u_1,u_2,\ldots,u_N)\hm\in U$ является естественным 
и~фактически означает, что при любой заданной детерминированной стратегии 
управ\-ле\-ния процесс~$\xi(t)$ не имеет мгновенных со\-сто\-яний, длительность пребывания 
в~которых равна нулю.

\smallskip

\noindent
\textbf{Замечание~3.}\ Сделаем некоторые замечания, связан\-ные с~подынтегральной 
функцией числителя дроб\-но-ли\-ней\-но\-го интегрального функционала~(\ref{e11}). 
Как и~ранее, будем рассматривать в~качестве целевого функционала $I(\Psi_1, \Psi_2,\ldots,\Psi_N)$\linebreak 
экстремальной задачи~(\ref{e18}) стационарный стоимостной показатель~(\ref{e10}). 
Тогда для любого фиксированного набора значений аргументов $(u_1,u_2,\ldots,u_N)\hm\in U$ 
значение функции $A(u_1,u_2,\ldots\linebreak \ldots,u_N)$ представляет собой условное математическое
 ожидание приращения рассматриваемого стоимостного функционала, 
 происшедшее за время пребывания полумарковского процесса~$\xi(t)$ в~некотором 
 фиксированном  состоянии при условии, что стратегия управления является 
 детерминированной и~задается указанным набором $(u_1,u_2,\ldots,u_N)\hm\in U$. 
 Отметим, что в~теореме об экстремуме дроб\-но-ли\-ней\-но\-го интегрального 
 функционала, доказанной в~работе~\cite[гл.~10]{12}, 
 на подынтегральную функцию числителя накладываются условия ограниченности на 
 всем множестве значений аргумента. Для многих математических моделей и~связанных 
 с~ними задач оптимального управления такое условие является излишне ограничительным. 
 В~качестве примера можно привести модели оптимального управления запасом непрерывного 
 продукта, рассмотренные в~работах~\cite{27, 28}. 
 В~настоящем исследовании на функцию $A(u_1,u_2,\ldots,u_N)$ накладывается только 
 условие интегрируемости по любому заданному набору вероятностных мер 
 $\Psi\hm=(\Psi_1, \Psi_2,\ldots,\Psi_N)$, образующему стратегию управления 
 полумарковским процессом~$\xi(t)$ (условие~1 системы предварительных условий).

\smallskip

\noindent
\textbf{Замечание~4.} Условия~1--3 являются необходимыми для корректной 
постановки задачи безусловного экстремума дроб\-но-ли\-ней\-но\-го интегрального 
функционала. Если этот функционал служит показателем качества в~задаче оптимального 
управления случайным процессом, то необходимо добавить к~этим условиям дополнительное, 
связанное с~некоторой регулярностью самого управляемого процесса, а~именно: некоторый 
содержательный показатель, связанный с~поведением этого процесса, должен существовать 
и~быть представимым в~виде дроб\-но-ли\-ней\-но\-го интегрального функционала. 
Если потребовать, чтобы выполнялось эргодическое соотношение~(\ref{e10}), 
то можно использовать\linebreak теорему~1 и~сформулировать задачу оптимального управ\-ле\-ния 
в~виде~(\ref{e18}) для дроб\-но-ли\-ней\-но\-го\linebreak интегрального функционала~(\ref{e11}). 
Таким образом, необходимо ввести условие, обеспечивающее существование единственного 
стационарного распределения вложенной цепи Маркова и~выполнение\linebreak соотношения~(\ref{e10}). 
По аналогии с~[8, гл.~5] сформулируем это дополнительное условие в~следующем виде:
\begin{enumerate}
\setcounter{enumi}{3}
\item Для любой рассматриваемой стратегии управ\-ле\-ния $\Psi\hm=
(\Psi_1, \Psi_2,\ldots,\Psi_N)\hm\in \Gamma$ вложенная цепь Маркова 
полумарковского процесса $\xi(t)$ имеет ровно один класс возвратных 
положительных состояний.
\end{enumerate}

Теперь определим понятие допустимой стратегии управления полумарковским процессом 
с~конечным множеством состояний.

\smallskip

\noindent
\textbf{Определение~2.}\ Назовем стратегию управления 
$\Psi\hm=(\Psi_1, \Psi_2,\ldots,\Psi_N)$ 
допустимой в~данной задаче, если она удовлетворяет условиям~1--4.


\smallskip

\noindent
\textbf{Замечание~5.}\ Как следует из замечания~1, если потребовать, 
чтобы функция $B(u_1, u_2,\ldots,u_N)$ являлась строго знакопостоянной при 
всех $(u_1, u_2,\ldots,u_N)\hm\in U$, то можно считать допустимыми стратегии 
$(\Psi_1, \Psi_2,\ldots,\Psi_N)$, удовлетворяющие условиям~1, 3, 4. С~учетом замечания~2 
о~естественном характере условия строгой знакопостоянности функции $B(u_1,u_2,\ldots,u_N)$ 
при всех значениях аргументов $(u_1, u_2,\ldots,u_N)\hm\in U$ будем требовать 
выполнения этого условия в~формулировке приводимой в~дальнейшем основной 
теоремы об оптимальной стратегии управления полумарковским процессом.

\smallskip

\noindent
\textbf{Замечание~6.}\ Ниже будет сформулирована и~доказана основная 
теорема об оптимальной стра\-тегии управления полумарковским процессом с~конеч\-ным 
множеством состояний. Будем формулировать эту теорему по отношению к~экстремальной 
задаче~(\ref{e18}), в~которой целевой функционал $I(\Psi_1, \Psi_2,\ldots,\Psi_N)$ 
имеет вид дроб\-но-ли\-ней\-но\-го интегрального функционала. 
Это обстоятельство связано с~тем, что целевой функционал в~задаче 
оптимального управления необязательно должен иметь характер стационарного 
стоимостного показателя вида~(\ref{e10}). В~частности, еще в~1983~г.\ П.\,В.~Шнурковым 
было установлено~\cite{24}, что ряд показателей, связанных 
с~временем пребывания управляемого полумарковского процесса в~заданном конечном 
подмножестве состояний, имеет структуру дроб\-но-ли\-ней\-но\-го интегрального 
функционала от набора вероятностных мер, определяющих стратегию управления. 
Таким образом, рассматриваемая задача управления имеет более общий характер, 
чем задача, в~которой целевой функционал представляет собой стационарный 
стоимостной показатель вида~(\ref{e10}).






\smallskip

\noindent
\textbf{Замечание~7.}\ Если рассматривать задачу оптимального управления 
полумарковским процессом, в~кото\-рой целевой функционал не совпадает 
со стационарным стоимостным показателем~(\ref{e10}), то возможно, что могут 
потребоваться другие дополнительные условия, обеспечивающие существование этого 
показателя и~его представление в~форме~(\ref{e11}). В~связи с~этим в~формулировке 
основной теоремы будем использовать термин допустимые стратегии в~широком смысле, 
имея в~виду выполнение всех необходимых условий для каждого рассмат\-ри\-ва\-емо\-го 
показателя качества управления.

\smallskip


\noindent
\textbf{Замечание 8.} Множество допустимых стратегий может 
не совпадать с~множеством всех возможных стратегий управления. 
В~частности, допустимые стратегии могут состоять только из дискретных вероятностных 
мер $\Psi_1, \Psi_2,\ldots,\Psi_N$, т.\,е.\ таких, которые сосредоточены на дискретных 
множествах точек пространств $U_1, U_2,\ldots,U_N$.

\section{Теоретическое решение задачи оптимального управления}

Перейдем к~формулировке и~доказательству тео\-ре\-мы об 
оптимальной стратегии управ\-ле\-ния полумарковским процессом с~конечным 
множеством состояний.

\smallskip

\noindent
\textbf{Теорема~2.} \textit{Рассмотрим проблему оптимального управ\-ле\-ния 
полумарковским процессом~$\xi(t)$ в~виде экстремальной задачи}~(\ref{e18}), 
\textit{определенной на множестве допустимых стратегий $\Gamma$, 
для дроб\-но-ли\-ней\-но\-го 
функционала}~(\ref{e11}). \textit{Пусть функция $B(u_1,u_2,\ldots,u_N)$, 
входящая в~определение функционала}~(\ref{e11}),
\textit{является строго знакопостоянной (строго положительной или строго отрицательной) 
при всех значениях аргументов $(u_1,u_2,\ldots,u_N)\hm\in U$.
Тогда справедливы сле\-ду\-ющие утверждения}:
\begin{enumerate}[1.]
\item \textit{Если функция} $C(u_1,u_2,\ldots,u_N)\hm=A(u_1,u_2,\ldots$\linebreak
$\ldots,u_N)/{B(u_1,u_2,\ldots,u_N)}$ 
\textit{ограничена сверху или снизу и~достигает глобального экст\-ре\-му\-ма на множестве
$U\hm=U_1\times U_2\times \cdots \times U_N$ (максимума или минимума), 
то оптимальная стратегия управления полумарковским процессом~$\xi(t)$ существует, 
является детерминированной и~определяется
вырожденной вероятностной мерой $\Psi^*\hm\in \Gamma^*$, сосредоточенной в~точке, 
в~которой достига\-ет соответствующего экстремума функция $C(u_1,u_2,\ldots,u_N)$,
и~при этом выполняются соотношения}:
\begin{multline}  %{\substack{{i=\overline{1,n}}\\ {j=\overline{1,l}}}}
\max\limits_{\Psi \in \Gamma} I(\Psi)=
\max\limits_{\substack{{\Psi_i \in \Gamma_i\,,}\\ 
{i=\overline{1,N}}}}
I\left(\Psi_1,\Psi_2,\ldots,\Psi_N\right)={}\\
{}=
\max\limits_{\substack{{\Psi_i^* \in \Gamma_i^*,}\\ 
{i=\overline{1,N}}}}
 I\left(\Psi_1^*,\Psi_2^*,\ldots,\Psi_N^*\right)={}\\
{}=\max\limits_{(u_1,u_2,\ldots,u_N)\in U}\fr{A(u_1,u_2,\ldots,u_N)}
{B(u_1,u_2,\ldots,u_N)}\,; \label{e19}
\end{multline}

\vspace*{-12pt}

\noindent
\begin{multline*}
\min\limits_{\Psi \in \Gamma} I(\Psi)=
\min\limits_{\substack{{\Psi_i \in \Gamma_i\,,}\\ 
{i=\overline{1,N}}}} I\left(\Psi_1,\Psi_2,\ldots,\Psi_N\right)={}\\
{}=
\min\limits_{\substack{{\Psi_i^* \in \Gamma_i^*,}\\ 
{i=\overline{1,N}}}}
I\left(\Psi_1^*,\Psi_2^*,\ldots,\Psi_N^*\right)={}\\
{}=\min\limits_{(u_1,u_2,\ldots,u_N)\in U}\fr{A(u_1,u_2,\ldots,u_N)}
{B(u_1,u_2,\ldots,u_N)}\,. %\label{e20}
\end{multline*}
\item \textit{Если функция $C(u_1,u_2,\ldots,u_N)\hm=
{A(u_1,u_2,\ldots,u_N)}/{B(u_1,u_2,\ldots,u_N)}$ ограничена сверху или снизу, 
но не достигает глобального экстремума на множестве $U\hm=U_1\times U_2\times\cdots
\times U_N$,
то для любого $\varepsilon\hm > 0$ можно выбрать $\varepsilon$-оп\-ти\-маль\-ную 
детерминированную стратегию управления полумарковским процессом~$\xi(t)$, 
которая определяется вырожденной
вероятностной мерой $\Psi^{*(+)}(\varepsilon)\hm\in \Gamma^*$ или вырожденной
вероятностной мерой $\Psi^{*(-)}(\varepsilon)\hm\in \Gamma^*$, в~зависимости от 
вида экстремума (максимума или минимума) в~задаче}~(\ref{e18}). 
\textit{При этом вероятностная мера $\Psi^{*(+)}(\varepsilon)\hm\in \Gamma^*$ может быть 
сосредоточена в~любой точке $\left(u_1^{(+)}(\varepsilon),u_2^{(+)}(\varepsilon),\ldots,
u_N^{(+)}(\varepsilon)\right)$, удовлетворяющей соотношению}:
\begin{multline}
\sup\limits_{(u_1,u_2,\ldots,u_N) \in U}
\fr{A(u_1,u_2,\ldots,u_N)}{B(u_1,u_2,\ldots,u_N)}-\varepsilon <{}\\
{}<
\fr{A\left(u_1^{(+)}(\varepsilon),u_2^{(+)}(\varepsilon),\ldots,u_N^{(+)}
(\varepsilon)\right)}
{B\left(u_1^{(+)}(\varepsilon),u_2^{(+)}(\varepsilon),\ldots,u_N^{(+)}
(\varepsilon)\right)}<{}\\
{}<\sup\limits_{(u_1,u_2,\ldots,u_N) \in U}
\fr{A(u_1,u_2,\ldots,u_N)}{B(u_1,u_2,\ldots,u_N)}<\infty\,, 
\label{e21}
\end{multline}
\textit{если функция $C(u_1,u_2,\ldots,u_N)$ ограничена сверху 
и~экстремальная задача}~(\ref{e18}) 
\textit{представляет собой задачу на максимум. Аналогично вероятностная мера 
$\Psi^{*(-)}(\varepsilon)\hm\in \Gamma^*$ может быть сосредоточена в~любой точке 
$\left(u_1^{(-)}(\varepsilon),u_2^{(-)}(\varepsilon),\ldots,u_N^{(-)}(\varepsilon)
\right)$, удовлетворяющей соотношению}:

\noindent
\begin{multline*}
-\infty<\inf\limits_{(u_1,u_2,\ldots,u_N) \in U}\fr{A(u_1,u_2,\ldots,u_N)}
{B(u_1,u_2,\ldots,u_N)} <{}\\
{}<
\fr{A\left(u_1^{(-)}(\varepsilon),u_2^{(-)}
(\varepsilon),\ldots,u_N^{(-)}(\varepsilon)\right)}
{B\left(u_1^{(-)}(\varepsilon),u_2^{(-)}(\varepsilon),\ldots,
u_N^{(-)}(\varepsilon)\right)}<{}\\
{}<\inf\limits_{(u_1,u_2,\ldots,u_N) \in U}
\fr{A(u_1,u_2,\ldots,u_N)}{B(u_1,u_2,\ldots,u_N)}+\varepsilon\,, 
%\label{e22}
\end{multline*}
\textit{если функция $C(u_1,u_2,\ldots,u_N)$ ограничена снизу и~экстремальная 
задача}~(\ref{e18})  \textit{представляет собой задачу на минимум}.
\item \textit{Если функция $C(u_1,u_2,\ldots,u_N)\hm=
{A(u_1,u_2,\ldots,u_N)}/{B(u_1,u_2,\ldots,u_N)}$ не ограничена сверху 
или снизу, то оптимальной стратегии управления в~смысле
соответствующей экстремальной задачи не существует. 
При этом найдется такая последовательность вырожденных вероятностных
мер~$\Psi^{*(+)}(n)$, сосредоточенных в~точках 
$\left(u_1^{(+)}(n),u_2^{(+)}(n),\ldots,u_N^{(+)}(n)\right)$, $n\hm=1,2,\dots $, 
для которых выполняется соотношение}:
\begin{multline*}
I\left(\Psi^*(n)\right)={}\\
{}=
I\left(\Psi_1^{*(+)}(n),\Psi_2^{*(+)}(n),\ldots,\Psi_N^{*(+)}(n)\right)={}\\
{}=\fr{A\left(u_1^{(+)}(n),u_2^{(+)}(n),\ldots,u_N^{(+)}(n)\right)}
{B\left(u_1^{(+)}(n),u_2^{(+)}(n),\ldots,u_N^{(+)}(n)\right)}\to 
\infty\\
\mbox{при}\ n\to\infty\,, 
%\label{e23}
\end{multline*}
\textit{если функция $C(u_1,u_2,\ldots,u_N)$ не ограничена сверху. 
Аналогично найдется такая последовательность вырожденных вероятностных
мер~$\Psi^{*(-)}(n)$, сосредоточенных в~точках 
$\left(u_1^{(-)}(n),u_2^{(-)}(n),\ldots,u_N^{(-)}(n)\right)$, 
$n\hm=1,2,\dots $, для которых выполняется соотношение}:
\begin{multline*}
I\left(\Psi^{*(-)}(n)\right)={}\\
{}= I
\left(\Psi_1^{*(-)}(n),\Psi_2^{*(-)}(n),\ldots,\Psi_N^{*(-)}(n)\right)={}\\
{}=\fr{A\left(u_1^{(-)}(n),u_2^{(-)}(n),\ldots,u_N^{(-)}(n)\right)}
{B\left(u_1^{(-)}(n),u_2^{(-)}(n),\ldots,u_N^{(-)}(n)\right)}\to 
-\infty\\
\mbox{при}~~n\to\infty\,,  
%\label{e24}
\end{multline*}
\textit{если функция $C(u_1,u_2,\ldots,u_N)$ не ограничена \mbox{снизу}}.
\end{enumerate}
\textit{При этом сформулированные утверждения каждого пункта теоремы~$2$ 
могут выполняться как по отдельности, для одного из двух
видов экстремума, так и~совместно, для обоих видов экстремума.}

\smallskip

Прежде чем непосредственно доказывать теорему~2, докажем некоторые 
вспомогательные утверждения.

\smallskip

\noindent
\textbf{Лемма~1.}\ 
\textit{Рассмотрим дроб\-но-ли\-ней\-ный интегральный функционал 
$I(\Psi_1, \Psi_2,\ldots, \Psi_N)$ вида}~(\ref{e11}), 
\textit{заданный на некотором множестве наборов вероятностных мер 
$\Psi\hm=(\Psi_1, \Psi_2,\ldots, \Psi_N)\hm \in \Gamma$. Предположим, что на 
множестве~$\Gamma$ выполняется условие~$1$ из набора предварительных условий 
и~функция $B(u_1, u_2,\ldots, u_N)$  обладает свойством строгой знакопостоянности 
при всех $(u_1, u_2,\ldots, u_N) \hm\in U$. Тогда справедливы следующие утверждения}:
\begin{enumerate}[1.]
\item \textit{Если основная функция 
$C(u_1, u_2,\ldots, u_N)\hm={A(u_1, u_2,\ldots, u_N)}/{B(u_1, u_2,\ldots, u_N)}$ 
ограничена сверху, т.\,е.\ выполняется условие}
\begin{multline}
C\left(u_1, u_2,\ldots, u_N\right)=
\fr{A(u_1, u_2,\ldots, u_N)}{B(u_1, u_2,\ldots, u_N)}\leq {}\\
{}\leq
c_0^{(+)}<\infty \,, \enskip \left(u_1, u_2,\ldots, u_N\right) \in U\,, \label{e25}
\end{multline}
\textit{то имеет место неравенство}:
\begin{equation}
I\left(\Psi_1, \Psi_2,\ldots, \Psi_N\right)\leq c_0^{(+)} 
\label{e26}
\end{equation}
\textit{для всех} $(\Psi_1, \Psi_2,\ldots, \Psi_N) \in \Gamma$.
\item \textit{Если основная функция 
$C(u_1, u_2,\ldots, u_N)\hm={A(u_1, u_2,\ldots, u_N)}/{B(u_1, u_2,\ldots, u_N)}$ 
ограничена снизу, т.\,е.\ выполняется условие}
\begin{multline*}
C\left(u_1, u_2,\ldots, u_N\right)=\fr{A(u_1, u_2,\ldots, u_N)}{B(u_1, u_2,\ldots, 
u_N)}\geq{}\\
{}\geq c_0^{(-)}>-\infty \,, 
\left(u_1, u_2,\ldots, u_N\right) \in U\,, 
%\label{e27}
\end{multline*}
\textit{то имеет место неравенство}:
\begin{equation*}
I\left(\Psi_1, \Psi_2,\ldots, \Psi_N\right)\geq c_0^{(-)} 
%\label{e28}
\end{equation*}
\textit{для всех} $(\Psi_1, \Psi_2,\ldots, \Psi_N) \hm\in \Gamma$.
\end{enumerate}

\noindent
Д\,о\,к\,а\,з\,а\,т\,е\,л\,ь\,с\,т\,в\,о\ \ леммы~1.\ 
Докажем первое утверждение леммы. Предположим сначала, 
что функция $B(u_1, u_2,\ldots,  u_N)$ строго положительна:
\begin{equation}
B\left(u_1, u_2,\ldots, u_N\right)>0\,,\enskip
\left(u_1, u_2,\ldots, u_N\right)\in U\,. \label{e29}
\end{equation}
Заметим, что в~таком случае по свойству интеграла~\cite[гл.~V]{18}
\begin{multline}
\hspace*{-2mm}\int\limits_{U_1}\!\!\cdots\! \!\int\limits_{U_N}\!\!B(u_1, \ldots,u_N) \,
d\Psi_1(u_1)%d\Psi_2(u_2)\cdots\\
\cdots d\Psi_N(u_N)>0 \!\!\!\!\label{e30}
\end{multline}
для любого фиксированного набора $\Psi\hm=(\Psi_1, \ldots, \Psi_N)\hm\in \Gamma$.
Из неравенства~(\ref{e25}) с~уче\-том~(\ref{e29}) получаем:
\begin{multline}
\hspace*{-4mm}A\left(u_1,\ldots, u_N\right)\leq{}\\
\hspace*{-4mm}{}\leq c_0^{(+)} B\left(u_1, \ldots, u_N\right)\,, 
\left(u_1, \ldots, u_N\right)\in U\,. \label{e31}
\end{multline}
В свою очередь, из неравенства~(\ref{e31}) и~свойств интеграла следует:
\begin{multline}
\int\limits_{U_1}\!\!\cdots\! \!\int\limits_{U_N}\!\!A(u_1,\ldots, u_N) \,
d\Psi_1\left(u_1\right)%d\Psi_2\left(u_2\right)\cdots\\
\cdots d\Psi_N\left(u_N\right)\leq\\
\hspace*{-24pt}\leq 
c_0^{(+)}\!\!\int\limits_{U_1}\!\!\cdots\!\! \int\limits_{U_N}\!\!\!B\!\left(u_1,\ldots, u_N\right)
 d\Psi_1\!\left(u_1\right)\!%d\Psi_2\left(u_2\right)\cdots\\
 \cdots d\Psi_N\!\left(u_N\right)\!\! 
 \label{e32}
\end{multline}
для любого фиксированного набора $\Psi\hm=(\Psi_1, \ldots, \Psi_N)\hm\in \Gamma$. 
Но тогда из~(\ref{e32}) с~учетом~(\ref{e30}) получаем:
\begin{multline}
I(\Psi_1, \ldots, \Psi_N)={}\\
{}=
\fr{\int\nolimits_{U_1}\!\cdots\! \int\nolimits_{U_N}\!\!A\left(u_1, \ldots, u_N\right)\,
 d\Psi_1(u_1)\cdots d\Psi_N(u_N)}{
\int\nolimits_{U_1}\!\cdots\! \int\nolimits_{U_N}\!\!B\left(u_1, \ldots, u_N\right)\,
 d\Psi_1(u_1)
 \cdots d\Psi_N(u_N)}\leq{}\\
 {}\leq c_0^{(+)} 
 \label{e33}
\end{multline}
для любого фиксированного набора $(\Psi_1, \ldots\linebreak\ldots, \Psi_N)\hm\in \Gamma$.

Предположим теперь, что функция $B(u_1,\ldots, u_N)$ строго отрицательна:
\begin{equation}
B(u_1,\ldots, u_N)<0 \quad \left(u_1, \ldots, u_N\right)\in U\,. 
\label{e34}
\end{equation}
Тогда
\begin{multline}
\hspace*{-6pt}\int\limits_{U_1}\!\!\cdots\!\! \int\limits_{U_N}\!\!B\!\left(u_1,\ldots, u_N\right)\!
 d\Psi_1(u_1) \cdots d\Psi_N(u_N)<0 \!\!\!
 \label{e35}
\end{multline}
для любого фиксированного набора $(\Psi_1, \ldots\linebreak \ldots, \Psi_N)\hm\in \Gamma$.

Как и~ранее, будем исходить из неравенства~(\ref{e25}). 
При выполнении условий~(\ref{e34}) и~(\ref{e35}) характер неравенств~(\ref{e31}) 
и~(\ref{e32}) меняется на противоположный, но характер неравенства~(\ref{e33}) 
остается неизменным. Таким образом, для любой функции 
$B(u_1, u_2,\ldots, u_N)$, обладающей свойством строгой знакопостоянности, 
из условия~(\ref{e25}) следует выполнение неравенства~(\ref{e33}), 
которое совпадает с~(\ref{e26}). Первое утверждение леммы~1 доказано. 
Второе утверждение доказывается аналогично. Лемма~1 доказана.

\smallskip

\noindent
\textbf{Лемма 2.} \textit{Рассмотрим дроб\-но-ли\-ней\-ный интегральный функционал 
$I(\Psi_1, \Psi_2,\ldots, \Psi_N)$ вида}~(\ref{e11}), 
\textit{заданный на некотором множестве наборов вероятностных мер 
$\Psi\hm=(\Psi_1, \Psi_2,\ldots, \Psi_N)\hm\in \Gamma$. Предпо\-ложим, что на 
множестве~$\Gamma$ выполняется условие~$1$ из набора предварительных условий 
и~функция $B(u_1, u_2,\ldots, u_N)$ обладает свойством строгой знакопостоянности 
при всех $(u_1, u_2,\ldots, u_N)\hm\in U$. Тогда справедливы следующие утверждения}:
\begin{enumerate}[1.]
\item \textit{Если основная функция $C(u_1, u_2,\ldots, u_N)\hm=
{A(u_1, u_2,\ldots, u_N)}/{B(u_1, u_2,\ldots, u_N)}$ ограничена сверху, 
но не достигает своего максимального 
значения, то имеет место неравенство}:
\begin{multline}
I\left(\Psi_1, \Psi_2,\ldots, \Psi_N\right)<{}\\
{}< \sup\limits_{(u_1, u_2,\ldots, u_N)\in U}
 C\left(u_1, u_2,\ldots, u_N\right)<\infty \label{e36}
\end{multline}
\textit{для всех} $(\Psi_1, \Psi_2,\ldots, \Psi_N)\in \Gamma$.
\item \textit{Если основная функция $C(u_1, u_2,\ldots, u_N)\hm=
{A(u_1, u_2,\ldots, u_N)}/{B(u_1, u_2,\ldots, u_N)}$ ограничена снизу, 
но не достигает своего минимального значения, то имеет место неравенство}:
\begin{multline*}
I\left(\Psi_1, \Psi_2,\ldots, \Psi_N\right)>{}\\
{}> \inf\limits_{(u_1, u_2,\ldots, u_N)\in U} 
C\left(u_1, u_2,\ldots, u_N\right)>-\infty 
%\label{e37}
\end{multline*}
\textit{для всех} $(\Psi_1, \Psi_2,\ldots, \Psi_N)\hm\in \Gamma$.
\end{enumerate}

\noindent
Д\,о\,к\,а\,з\,а\,т\,е\,л\,ь\,с\,т\,в\,о\ \ леммы~2. 
Докажем первое утверждение леммы. Поскольку множество значений 
основной функции $C(u_1, u_2,\ldots, u_N)$ ограничено сверху, оно имеет конечную 
верхнюю грань:
$$
\exists \sup\limits_{(u_1, u_2,\ldots, u_N)\in U} 
C\left(u_1, u_2,\ldots, u_N\right)<\infty
$$
(см.~\cite[гл.~1, \S3, п.~3.4, теорема~1]{25}).

По условию функция $C(u_1, u_2,\ldots, u_N)$ не достигает своего максимального 
значения. Следовательно, выполняется неравенство:
\begin{multline}
C(u_1, u_2,\ldots, u_N)=\fr{A(u_1, u_2,\ldots, u_N)}{B(u_1, u_2,\ldots, u_N)}<{}\\
{}< 
\sup\limits_{(u_1, u_2,\ldots, u_N)\in U} C(u_1, u_2,\ldots, u_N)<\infty\,, 
\\
\left(u_1, u_2,\ldots, u_N\right)\in U\,.
\label{e38}
\end{multline}
Взяв за основу строгое неравенство~(\ref{e38}), проведем рассуждения, аналогичные тем, 
которые были проведены в~лемме~1 по отношению к~неравенству~(\ref{e25}). 
В~результате получим строгое неравенство~(\ref{e36}).

Второе утверждение леммы~2 доказывается аналогично. Лемма~2 доказана.

\noindent
Д\,о\,к\,а\,з\,а\,т\,е\,л\,ь\,с\,т\,в\,о\ 
\ теоремы~2.
Начнем с~доказательства утверждения~1. Предположим сначала, что основная 
функция $C(u_1, u_2,\ldots, u_N)={A(u_1, u_2,\ldots, u_N)}/{B(u_1, u_2,\ldots, u_N)}$ 
ограничена сверху и~достигает глобального максимума на множестве~$U$ 
в~некоторой точке $u^{(+)}\hm=\left(u^{(+)}_1,u^{(+)}_2,\ldots,u^{(+)}_N\right)\hm\in U$,
а~именно:
\begin{multline*}
\max\limits_{(u_1, u_2,\ldots, u_N)\in U} C\left(u_1, u_2,\ldots, u_N\right) = {}\\
{}=
C\left(u^{(+)}_1,u^{(+)}_2,\ldots,u^{(+)}_N\right)<\infty\,.
\end{multline*}
Тогда выполняется соотношение:
\begin{multline}
C(u_1, u_2,\ldots, u_N)=\fr{A(u_1, u_2,\ldots, u_N)}{B(u_1, u_2,\ldots, u_N)}
\leq{}\\
{}\leq C\left(u^{(+)}_1,u^{(+)}_2,\ldots,u^{(+)}_N\right)<\infty\,, 
\\
\left(u_1, u_2,\ldots, u_N\right)\in U\,.
\label{e39}
\end{multline}
Условия леммы~1 выполнены, и~можно воспользоваться ее утверждениями. 
Согласно первому из них, если выполняется неравенство~(\ref{e39}), 
то имеет место соотношение:
\begin{equation*}
I(\Psi_1, \Psi_2,\ldots, \Psi_N)\leq 
C\left(u^{(+)}_1,u^{(+)}_2,\ldots,u^{(+)}_N\right)<\infty 
%\label{e40}
\end{equation*}
для всех стратегий управления $\Psi\hm=(\Psi_1, \Psi_2,\ldots\linebreak
\ldots, \Psi_N)\hm\in \Gamma$.

Таким образом, множество значений дроб\-но-ли\-ней\-но\-го интегрального 
функционала $I(\Psi_1, \Psi_2,\ldots, \Psi_N)$ ограничено сверху при всех 
$\Psi\hm=(\Psi_1, \Psi_2,\ldots, \Psi_N)\hm\in \Gamma$. Тогда существует верхняя 
грань этого множества и~выполняется неравенство:
\begin{multline}
\sup\limits_{(\Psi_1, \Psi_2,\ldots, \Psi_N)\in \Gamma} 
I\left(\Psi_1, \Psi_2,\ldots, \Psi_N\right)\leq {}\\
{}\leq
C\left(u^{(+)}_1,u^{(+)}_2,\ldots,u^{(+)}_N\right). \label{e41}
\end{multline}
Рассмотрим детерминированную стратегию управ\-ле\-ния 
$\Psi^{*(+)}\hm=\left(\Psi_1^{*(+)}, \Psi_2^{*(+)},\ldots, \Psi_N^{*(+)}\right)$, 
в~которой каждая вероятностная мера~$\Psi_i^{*(+)}$ является вы\-рож\-ден\-ной 
и~сосредоточена в~точке $u_i^{(+)}$, $i\hm=\overline{1, N}$.
По свойству интеграла
\begin{multline}
I\left(\Psi_1^{*(+)}, \Psi_2^{*(+)},\ldots ,\Psi_N^{*(+)}\right)={}\\
{}=
C\left(u^{(+)}_1,u^{(+)}_2,\ldots,u^{(+)}_N\right). \label{e42}
\end{multline}
Из соотношений~(\ref{e41}) и~(\ref{e42}) получаем:
\begin{multline}
\sup\limits_{(\Psi_1, \Psi_2,\ldots, \Psi_N)\in \Gamma} 
I\left(\Psi_1, \Psi_2,\ldots, \Psi_N\right)\leq{}\\
{}\leq
 I\left(\Psi_1^{*(+)}, 
\Psi_2^{*(+)},\ldots, \Psi_N^{*(+)}\right). \label{e43}
\end{multline}
Заметим дополнительно, что выполняются отношения принадлежности:
\begin{equation}
\Psi^{*(+)}=\left(\Psi_1^{*(+)}, \Psi_2^{*(+)},\ldots, \Psi_N^{*(+)}\right) 
\in \Gamma^* \subset \Gamma\,. \label{e44}
\end{equation}
Из~(\ref{e44}) и~свойства верхней грани следует:
\begin{multline}
\sup\limits_{\left(\Psi_1^{*}, \Psi_2^{*},\ldots, \Psi_N^{*}\right) \in \Gamma^*} 
I\left(\Psi_1^{*}, \Psi_2^{*},\ldots, \Psi_N^{*}\right)\leq {}\\
{}\leq
\sup\limits_{\left(\Psi_1, \Psi_2,\ldots, \Psi_N\right) 
\in \Gamma} I\left(\Psi_1, \Psi_2,\ldots, \Psi_N\right)\,. 
\label{e45}
\end{multline}
Объединяя~(\ref{e42}), (\ref{e43}) и~(\ref{e45}), получаем соотношение:
\begin{multline}
\sup\limits_{\left(\Psi_1^{*}, \Psi_2^{*},\ldots, \Psi_N^{*}\right) 
\in \Gamma^*} I\left(\Psi_1^{*}, \Psi_2^{*},\ldots, 
\Psi_N^{*}\right)\leq{}\\
{}\leq \sup\limits_{\left(\Psi_1, \Psi_2,\ldots, \Psi_N\right) 
\in \Gamma} I\left(\Psi_1, \Psi_2,\ldots, \Psi_N\right)\leq{}\\
{}\leq I\left(\Psi_1^{*(+)}, \Psi_2^{*(+)},\ldots, \Psi_N^{*(+)}\right)={}\\
{}=
\fr{A\left(u^{(+)}_1,u^{(+)}_2,\ldots,u^{(+)}_N\right)}{B\left(u^{(+)}_1,u^{(+)}_2,
\ldots,u^{(+)}_N\right)}\,.
 \label{e46}
\end{multline}
Из соотношения~(\ref{e46}) с~учетом~(\ref{e44}) получаем, что максимум 
функционала $I(\Psi_1, \Psi_2,\ldots, \Psi_N)$ на множестве допустимых стратегий 
$\Psi\hm=(\Psi_1, \Psi_2,\ldots, \Psi_N)\hm\in \Gamma$ существует и~достигается 
на детерминированной стратегии $\left(\Psi_1^{*(+)}, \Psi_2^{*(+)},\ldots, 
\Psi_N^{*(+)}\right)$.

Кроме того, выполняются соотношения~(\ref{e19}). Таким образом, утверждение~1 
в~случае, когда основная функция $C(u_1, u_2,\ldots, u_N)$ достигает глобального 
максимума, доказано. Соответствующее утверждение в~случае, когда основная функция 
$C(u_1, u_2,\ldots, u_N)$ достигает глобального минимума, доказывается аналогично. 
При этом используется второе утверждение леммы~1.

\smallskip

Перейдем к~доказательству второго утверждения теоремы~2. Предположим, что основная 
функция $C(u_1, u_2,\ldots, u_N)\hm=A(u_1, u_2,\ldots$\linebreak
$\ldots, u_N)/{B(u_1, u_2,\ldots, u_N)}$ 
ограничена сверху, но не достигает глобального максимума на множестве 
$U \hm= U_1 \times U_2 \times \cdots \times U_N$. Тогда множество значений 
основной функции имеет конечную верхнюю грань:

\noindent
\begin{multline*}
C\left(u_1, u_2,\ldots, u_N\right)=\fr{A(u_1, u_2,\ldots, u_N)}
{B(u_1, u_2,\ldots, u_N)}<{}\\
{}<
\sup\limits_{(u_1, u_2,\ldots, u_N)\in U} \fr{A(u_1, u_2,\ldots, u_N)}
{B(u_1, u_2,\ldots, u_N)}<\infty\,, 
\\
\left(u_1, u_2,\ldots, u_N\right)\in U\,.
%\label{e47}
\end{multline*}
По определению верхней грани для любого фиксированного $\varepsilon \hm>0$ 
существует точка $(u_1^{(+)}(\varepsilon), u_2^{(+)}(\varepsilon),\ldots, 
u_N^{(+)}(\varepsilon))$ такая, что выполняется двойное неравенство~(\ref{e21}) 
(см.~\cite[гл.~1, \S\,3, п.~3.4]{25}). Иначе говоря, значение основной функции 
в~указанной точке лежит в~левой \mbox{$\varepsilon$-окрест}\-ности верхней грани. 
Рассмотрим детерминированную стратегию управления 
$\Psi^{*(+)}(\varepsilon)\hm=\!\left(\Psi_1^{*(+)}(\varepsilon), 
\Psi_2^{*(+)}(\varepsilon),\ldots, \Psi_N^{*(+)}(\varepsilon)\!\right)$, компонентами\linebreak 
которой являются вырожденные вероятностные меры $\Psi_1^{*(+)}(\varepsilon), 
\Psi_2^{*(+)}(\varepsilon),\ldots, \Psi_N^{*(+)}(\varepsilon)$, причем вырожденная 
мера~$\Psi_i^{*(+)}(\varepsilon)$ сосредоточена в~точке~$u_i^{(+)}(\varepsilon)$,
$i\hm=1,2,\ldots,N$.

По свойству интеграла
\begin{multline}
I\left(\Psi_1^{*(+)}(\varepsilon), \Psi_2^{*(+)}(\varepsilon),\ldots,
 \Psi_N^{*(+)}(\varepsilon)\right)={}\\
 {}=
 C\left(u_1^{(+)}(\varepsilon), u_2^{(+)}(\varepsilon),\ldots, 
 u_N^{(+)}(\varepsilon)\right)\,. 
 \label{e48}
\end{multline}
Из соотношения~(\ref{e48}) с~учетом указанного свойства основной функции получаем:
\begin{multline}
\sup\limits_{(u_1, u_2,\ldots, u_N)\in U} \fr{A(u_1, u_2,\ldots, u_N)}
{B(u_1, u_2,\ldots, u_N)}-\varepsilon<{}\\
{}< I\left(\Psi_1^{*(+)}(\varepsilon), 
\Psi_2^{*(+)}(\varepsilon),\ldots, \Psi_N^{*(+)}(\varepsilon)\right)<{}
\\
{}< \sup\limits_{(u_1, u_2,\ldots, u_N)\in U} \fr{A(u_1, u_2,\ldots, u_N)}
{B(u_1, u_2,\ldots, u_N)}<\infty\,. 
\label{e49}
\end{multline}
Заметим также, что в~рассматриваемом случае выполнены условия леммы~2. 
Воспользуемся первым утверждением этой леммы, а~именно соотношением~(\ref{e36}):
\begin{multline}
I(\Psi_1, \Psi_2,\ldots, \Psi_N)< {}\\
{}<\sup\limits_{(u_1, u_2,\ldots, u_N)
\in U} \fr{A(u_1, u_2,\ldots, u_N)}{B(u_1, u_2,\ldots, u_N)}<\infty 
\label{e50}
\end{multline}
для всех $(\Psi_1, \Psi_2,\ldots, \Psi_N)\in\Gamma$.

Из соотношений~(\ref{e49}) и~(\ref{e50}) следует, что детерминированная стратегия 
$\Psi^{*(+)}(\varepsilon)\hm=\left(\Psi_1^{*(+)}(\varepsilon), \Psi_2^{*(+)}(\varepsilon),
\ldots, \Psi_N^{*(+)}(\varepsilon)\right)$, опре\-де\-ля\-емая набором вырожденных 
вероятностных мер, сосредоточенных в~соответствующих точках 
$\left(u_1^{(+)}(\varepsilon), u_2^{(+)}(\varepsilon),\ldots, 
u_N^{(+)}(\varepsilon)\right)$, является $\varepsilon$-оп\-ти\-маль\-ной. 
Вторая часть утверждения~2 теоремы~2, связанная со свойствами нижней грани, 
доказывается аналогично.

Докажем третье утверждение теоремы~2. Предположим, что множество значений 
основной функции $C(u_1, u_2,\ldots, u_N)\hm=
A(u_1, u_2,\ldots$\linebreak $\ldots, u_N)/{B(u_1, u_2,\ldots, u_N)}$
не является ограниченным сверху на множестве $U\hm=U_1\times U_2 \times \cdots $\linebreak
$\cdots \times U_N$.
Тогда существует последовательность\linebreak точек $\left(u_1^{(+)}(n), u_2^{(+)}(n),
\ldots,u_N^{(+)}(n)\right)\hm\in U$, $n\hm=1,2,\ldots$, для которой
\begin{multline}
C\left(u_1^{(+)}(n), u_2^{(+)}(n),\ldots,u_N^{(+)}(n)\right)={}\\
{}=
\fr{A\left(u_1^{(+)}(n), u_2^{(+)}(n),\ldots,u_N^{(+)}(n)\right)}
{B\left(u_1^{(+)}(n), u_2^{(+)}(n),\ldots,u_N^{(+)}(n)\right)}
\longrightarrow \infty \,,\\
n\rightarrow \infty\,.
\label{e51}
\end{multline}
Зафиксируем некоторую последовательность точек $\left(u_1^{(+)}(n), u_2^{(+)}(n),
\ldots,u_N^{(+)}(n)\right)\hm\in U$, $n\hm=1,2,\ldots$, обладающих указанным свойством, 
и~рассмотрим последовательность детерминированных  стратегий управления 
$\Psi^{*(+)}(n)\hm=\left(\Psi_1^{*(+)}(n), \Psi_2^{*(+)}(n),\ldots, 
\Psi_N^{*(+)}(n)\right)$, $n\hm=1,2,\ldots$, определяемых набором вырожденных 
вероятностных мер, сосредоточенных в~соответствующих точках 
$\left(u_1^{(+)}(n), u_2^{(+)}(n),\ldots,u_N^{(+)}(n)\right)$, $n\hm=1,2,\ldots$ 
По свойству интеграла для любого фиксированного значения $n=1,2,\ldots$ 
выполняется равенство:
\begin{multline}
I \left(\Psi^{*(+)}(n)\right)={}\\
{}=I\left(\Psi_1^{*(+)}(n), \Psi_2^{*(+)}(n),\ldots,
 \Psi_N^{*(+)}(n)\right)={}\\
{}=\fr{A\left(u_1^{(+)}(n), u_2^{(+)}(n),\ldots,u_N^{(+)}(n)\right)}
{B\left(u_1^{(+)}(n), u_2^{(+)}(n),\ldots,u_N^{(+)}(n)\right)}\,. 
\label{e52}
\end{multline}
Из соотношений~(\ref{e51}) и~(\ref{e52}) следует, что
\begin{multline}
I\left(\Psi^{*(+)}(n)\right)={}\\
{}=I\left(\Psi_1^{*(+)}(n), \Psi_2^{*(+)}(n),\ldots, 
\Psi_N^{*(+)}(n)\right)\longrightarrow\infty\,,\\ 
n \rightarrow\infty\,.
 \label{e53}
\end{multline}
Соотношение~(\ref{e53}) означает, что множество значе\-ний дроб\-но-ли\-ней\-но\-го 
интегрального функциона\-ла $I(\Psi_1, \Psi_2,\ldots, \Psi_N)$ вида~(\ref{e11}) 
не ограничено сверху\linebreak на множестве наборов вырожденных вероятностных мер 
$\left(\Psi_1^{*(+)}(n), \Psi_2^{*(+)}(n),\ldots, \Psi_N^{*(+)}(n)\right)\hm\in\Gamma^*$, 
а~следовательно, и~на более широком\linebreak множестве наборов вероятностных 
мер $(\Psi_1, \Psi_2,\ldots$\linebreak $\ldots, \Psi_N)\hm\in\Gamma$. В~таком случае решения экстремальной 
задачи~(\ref{e18}) в~форме задачи на максимум не существует. Соответствующее утвержде\-ние 
для варианта, когда множество значений основной функции $C(u_1, u_2,\ldots,u_N)
\hm=A(u_1, u_2,\ldots$\linebreak $\ldots,u_N)/{B(u_1, u_2,\ldots,u_N)}$ 
не является ограниченным снизу, доказывается аналогично. Третье утверж\-де\-ние теоремы~2 
доказано. Тем самым тео\-ре\-ма~2 доказана полностью.

\smallskip

Применим теорему~2 для решения поставленной задачи оптимального управления. 
Из утверждения этой теоремы следует, что для доказательства су-\linebreak ществования 
оптимального управ\-ле\-ния и~его нахождения необходимо исследовать на 
глобальный экстремум основную функцию дроб\-но-ли\-ней\-но\-го интегрального 
функционала $C(u_1,u_2,\ldots,u_N)$, определяемую формулой~(\ref{e17}) с~учетом 
равенств~(\ref{e12})--(\ref{e16}). В~некоторых случаях, например когда основной 
процесс~$\xi(t)$ является регенерирующим, а~стоимостные характеристики 
модели задаются линейными функциями, такое исследование можно провести 
аналитически. Однако для подавляющего большинства полумарковских моделей 
для этого необходимо использовать численные методы.

\section{Заключение}

В заключительной части работы приведем \mbox{краткое} описание теоретической 
основы метода решения задачи оптимального управления полумарковским 
процессом с~конечным множеством состояний.

\begin{enumerate}[1.]
\item Исходная проблема оптимального управления формулируется в~виде 
экстремальной задачи~(\ref{e18}). Целевым показателем качества управ\-ле\-ния в~данной задаче 
служит величина~(\ref{e10}), которая имеет характер средней удельной прибыли.
\item Доказывается, что стационарный показатель~(\ref{e10}) представим в~виде 
дроб\-но-ли\-ней\-но\-го интегрального функционала~(\ref{e11}), для которого явно 
определяются подынтегральные функции числителя и~знаменателя, а~следовательно, 
и~основная функция данного функционала.
\item Используется теорема об экстремуме дроб\-но-ли\-ней\-но\-го интегрального 
функционала. На основании утверждений этой теоремы уста\-нав\-ли\-ва\-ет\-ся, что 
исходная задача оптимального управления сводится к~исследованию на глобальный 
экстремум основной функции этого функционала, для которой получено явное 
аналитическое представление.
\end{enumerate}

Заметим, что такое исследование задач оптимального управления 
стохастическими системами фактически уже было проведено в~ряде работ П.\,В.~Шнуркова 
и~его соавторов. В~частности, в~работе~\cite{26} была рассмотрена модель 
управления для обрывающегося процесса восстановления, описывающего функционирование 
некоторой технической системы. Задача управления решалась для различных показателей 
эффективности и~надежности этой системы, имеющих структуру дроб\-но-ли\-ней\-но\-го 
интегрального функционала.

В работах~\cite{27, 28} рассматривались модели регенерирующих процессов 
для исследования сис\-тем управления запасами. Различные показатели качества 
управления были представлены в~форме дроб\-но-ли\-ней\-ных интегральных функционалов. 
Основные функции этих функционалов были найде\-ны в~явной форме и~исследовались 
на глобальный экстремум. В~работах~\cite{21,29} рассматривалась достаточно 
сложная полумарковская модель с~конечным множеством состояний, описывающая 
сис\-те\-му управления запасом непрерывного продукта. Показатели качества управления в~этой 
модели также имели структуру дроб\-но-ли\-ней\-ных интегральных функционалов, 
для основных функций которых были найдены явные аналитические представления. 
Упомянем также работы~\cite{30, 31}, в~которых была исследована полумарковская 
модель с~дис\-крет\-но-не\-пре\-рыв\-ным фазовым пространством. Показатели 
качества управления в~этой  модели были найдены в~явной форме как функции от 
двух непрерывных параметров управления.

Фактически во всех упомянутых работах уже был использован метод решения задачи 
оптимального управления регенерирующим или полумарковским случайным процессом, 
основанный на исследовании экстремальных свойств основной функции соответствующего 
дроб\-но-ли\-ней\-но\-го интегрального функционала. Из соображений, изложенных 
во\linebreak введении, следует, что в~период написания и~пуб\-ли\-кации этих работ данный метод 
не имел стро\-гого обоснования. Однако после публикации\linebreak работы~\cite{14} и~настоящего 
исследования можно утверж\-дать, что полученные в~них результаты полностью теоретически 
обоснованы.

Таким образом, изложенный выше метод решения проблемы оптимального управления 
полумарковскими процессами с~конечными множествами состояний может быть успешно 
реализован для многих задач, рассматриваемых в~различных областях прикладной 
теории вероятностей.

Практическая реализация численной процедуры поиска оптимального решения на примере\linebreak 
полумарковской модели управления запасом непрерывного продукта (подробнее 
см.~\cite{21, 29}), ба\-зи\-ру\-юща\-яся на изложенных выше результатах (в~частности, 
теореме~1), была осуществлена А.\,К.~Горшениным и~соавторами 
в~статье~\cite{Gorshenin2015}. Коротко опишем наиболее важные аспекты этой работы.

Для решения поставленной задачи опти\-мального управления была создана 
специальная программа \verb"Inventory" на встроенном языке программирования 
пакета \verb"MATLAB", ее возможности\linebreak кратко представле\-ны в~упомянутой ранее 
\mbox{статье}~\cite{Gorshenin2015}. В~программе \verb"Inventory" реализованы функции 
для оценивания через заданные исходные параметры вероятностных и~стоимостных 
характеристик модели, которые в~дальнейшем используются для поиска значений 
основной функции дроб\-но-ли\-ней\-но\-го функционала~(\ref{e17}). Точка глобального 
экстремума этой функции и~определяет оптимальное управление.

В качестве начальных данных необходимо задание следующих параметров:
\begin{itemize}
\item спрос и~вместимость склада;
\item разбиение множества значений объема запаса;
\item вероятностные характеристики, описывающие модель пополнения запаса;
\item условные математические ожидания длительностей задержек пополнения запаса;
\item функции для характеризации затрат и~доходов.
\end{itemize}

По итогам работы программы \verb"Inventory" ряд вспомогательных функций 
представляется в~аналитической форме (в частности, с~использованием аппарата 
символьных вычислений  \verb"Symbolic Toolbox"\linebreak пакета \verb"MATLAB"), выводится 
точка глобального экстремума функции нескольких вещественных переменных~(\ref{e17}), 
найденная с~помощью применения численных и~при\-бли\-жен\-но-ана\-ли\-ти\-че\-ских\linebreak 
аппроксимаций. 
Также формируются графики оценок значений ве\-ро\-ят\-ност\-но-сто\-и\-мост\-ных 
характеристик 
и~основной функции дроб\-но-ли\-ней\-но\-го функционала~(\ref{e17}), либо трехмерных 
сечений в~случае наличия более трех параметров управления (переменных).

Функциональность пакета \verb"Inventory" может быть расширена для практической 
реализации метода решения задачи поиска оптимального управ\-ле\-ния полумарковскими 
процессами с~конечными множествами состояний, рассмотренного в~данной статье.


 {\small\frenchspacing
 {%\baselineskip=10.8pt
 \addcontentsline{toc}{section}{References}
 \begin{thebibliography}{99}
 \bibitem{1}
\Au{Ховард Р.} Динамическое программирование и~марковские процессы~/ 
Пер. с~англ.~--- М.: Сов. радио, 1964. 189~с.
(\Au{Howard~R.\,A.} Dynamic programming and Markov processes.~--- 
Cambridge, MA, USA: MIT Press, 1960. 136~p.)
\bibitem{2} 
\Au{Рыков В.\,В.} Управляемые марковские процессы с~конечными пространствами 
состояний и~управлений~// Теория вероятностей и~ее применения, 1966. Т.~11. 
Вып.~2. С.~343--351.
\bibitem{3} 
\Au{Джевелл В.} Управляемые полумарковские процессы~// Кибернетич. сборник.~--- 
М.: Мир, 1967. Вып.~4. С.~97--134.
%{\em Jewell W.\,S.} Markov-renewal programming~// Operations Research, 1963. Vol.~11. P.~938--971.
\bibitem{4} 
\Au{Fox B.} Markov renewal programming by linear fractional programming~// 
SIAM J.~Appl. Math., 1966. Vol.~14. P.~1418--1432.
\bibitem{5} 
\Au{Denardo E.\,V.} Contraction mappings in the theory underlying dinamic programming~// 
SIAM Rev., 1967. Vol.~9. P.~165--177.

\bibitem{6} 
\Au{Howard R.\,A.} Research in semi-Markovian decision structures~// 
J.~Oper. Res. Soc. Japan, 1963. Vol.~6. P.~163--199.
\bibitem{7} 
\Au{Osaki S., Mine H.} Linear programming algorithms for Markovian decision processes~//
 J.~Math. Anal. Appl., 1968. Vol.~22. P.~356--381.
\bibitem{8} 
\Au{Майн Х., Осаки С.} Марковские процессы принятия решений~/ Пер. с~англ.~--- 
М.: Наука, 1977. 176~с.
(\Au{Mine~H., Osaki~S.} 
Markovian decision processes.~--- New York, NY, USA: 
American Elsevier Publishing Co., 1970. 142~p.)
\bibitem{9} 
\Au{Гихман И.\,И., Скороход А.\,В.} Управляемые случайные процессы.~--- 
Киев: Наукова думка, 1977. 251~с.
\bibitem{10} 
\Au{Luque-Vasquez F., Herndndez-Lerma~О.} Semi-Markov control models with average costs~// 
Appl. Math., 1999. Vol.~26. No.\,3. P.~315--331.
\bibitem{11} 
\Au{Vega-Amaya O., Luque-Vasquez~F.} Sample-path average cost optimality for 
semi-Markov control processes on Borel spaces: Unbounded costs and mean holding times~// 
Appl. Math., 2000. Vol.~27. No.\,3. P.~343--367.
\bibitem{12} 
Вопросы математической теории надежности~/ Под ред. Б.\,В. Гнеденко.~--- 
М.: Радио и~связь, 1983. 376~с.
\bibitem{13} 
\Au{Барзилович Е.\,Ю., Каштанов~В.\,А.} Некоторые математические вопросы теории 
обслуживания сложных систем.~---  М.: Сов. радио, 1971. 272~с.
\bibitem{14} 
\Au{Шнурков П.\,В.} О~решении проблемы безусловного экстремума для 
дроб\-но-ли\-ней\-но\-го интегрального функционала на множестве вероятностных мер~// 
Докл. РАН. Сер. Математика, 2016. Т.~470. №\,4. C.~387--392.
\bibitem{15} 
\Au{Ширяев А.\,Н.}  Вероятность.~--- М.:~МЦНМО, 2011. Кн.~1. 552~с. Кн.~2. 968~с.
\bibitem{16} 
\Au{Боровков А.\,А.} Теория вероятностей.~--- М.: Либроком, 2009. 656~c.
\bibitem{17} 
\Au{Хеннекен П.\,Л., Тортра А.} Теория вероятностей 
и~некоторые ее приложения.~--- М.: Наука, 1974. 472~c.
\bibitem{18} 
\Au{Халмош П.} Теория меры~/ Пер. с~англ.~--- М.: ИЛ, 1953. 282~c.
(\Au{Halmos~P.} Measure theory.~--- Litton Educational Publishing, Inc. 1950. 304~p.)
\bibitem{19} 
\Au{Королюк В.\,С., Турбин~А.\,Ф.} Полумарковские процессы и~их приложения.~--- 
Киев:~Наукова думка, 1976. 184~с.
\bibitem{20} 
\Au{Janssen J., Manca R.} Applied semi-Markov processes.~--- New York,
NY, USA: Springer, 2006. 309~p.
\bibitem{21} 
\Au{Шнурков П.\,В., Иванов~А.\,В.} Анализ дискретной полумарковской модели
 управления запасом непрерывного продукта при периодическом прекращении потребления~// 
 Дискретная математика, 2014. Т.~26. Вып.~1. С.~143--154.
\bibitem{22} 
\Au{Иванов~А.\,В.} Анализ дискретной полумарковской модели
 управления запасом непрерывного продукта при периодическом прекращении 
 потребления.~--- М.: НИУ ВШЭ, 2014.  Дисс.\ \ldots\ канд. физ.-мат. наук. 120~с.
\bibitem{23}  %23
\Au{Bajalinov~E.\,B.} Linear-fractional programming. 
Theory, methods, applications and software.~--- 
Boston/\linebreak Dordrecht/London: Kluwer Academic Publs., 2003. 423~p.

\bibitem{27} %27
\Au{Шнурков П.\,В., Мельников~Р.\,В.} Оптимальное управление запасом 
непрерывного продукта в~модели регенерации~// Обозрение прикладной 
и~промышленной математики, 2006. Т.~13. Вып.~3. С.~434--452.
\bibitem{28} 
\Au{Шнурков П.\,В., Мельников~Р.\,В.} 
Исследование проб\-ле\-мы управления запасом непрерывного продукта при детерминированной 
задержке поставки~// Автоматика и~телемеханика, 2008. Т.~10. С.~93--113.


\bibitem{24}  %26
\Au{Шнурков П.\,В.} Методы исследования задач оптимального обслуживания 
в~математической теории надежности.~--- 
М.: МИЭМ, 1983.  Дисс.\ \ldots\ канд. физ.-мат. наук.

 \bibitem{25}  %25
\Au{Кудрявцев Л.\,Д.} Курс математического анализа. Т.~1.~--- 
М.: Дрофа, 2006. 704~с.

\bibitem{26} %24
\Au{Шнурков П.\,В.} Оптимальное обслуживание на периоде 
до первого отказа системы~// Применение аналитических методов в~вероятностных
 задачах.~--- Киев: Институт математики АН УССР, 1986. С.~121--129.

\bibitem{29} 
\Au{Шнурков П.\,В., Иванов~А.\,В.} Исследование задачи оптимизации в~дискретной 
полумарковской модели управления непрерывным запасом~// Вестник МГТУ им.\ 
Н.\,Э. Баумана. Сер.\ Естественные науки, 2013. Т.~3. Вып.~50. С.~62--87.
\bibitem{30} 
\Au{Shnourkoff P.\,V.} The two-element system with one 
restoring device optimum maintenance~// Stoch. Anal. Appl., 1997. 
Vol.~15. No.\,5. P.~823--837.
\bibitem{31} 
\Au{Shnourkoff P.\,V.} The two-element system optimum maintenance tills the first fail~// 
Stoch. Anal. Appl., 2001. Vol.~19. No.\,6. P.~1005--1024.
\bibitem{Gorshenin2015} 
\Au{Gorshenin~A.\,K., Belousov~V.\,V., Shnourkoff~P.\,V.,
Ivanov~A.\,V.} Numerical research of the optimal control problem in the semi-Markov 
inventory model~// AIP Conference Proceedings, 2015. Vol.~1648. {250007}. 4~p.
%\bibitem{33} {\em Горшенин А.\,К., Белоусов В.\,В., Шнурков П.\,В.} 2016. Система управления запасами на основе стохастических полумарковских моделей. Свидетельство о государственной регистрации программы для ЭВМ \textnumero 2016619021.
 \end{thebibliography}

 }
 }

\end{multicols}

\vspace*{-6pt}

\hfill{\small\textit{Поступила в~редакцию 15.07.16}}

%\vspace*{8pt}

\newpage

\vspace*{-24pt}

%\hrule

%\vspace*{2pt}

%\hrule

%\vspace*{8pt}


\def\tit{ANALYTICAL SOLUTION OF~THE~OPTIMAL CONTROL TASK OF~A~SEMI-MARKOV 
PROCESS WITH~FINITE SET OF~STATES}

\def\titkol{Analytical solution of~the~optimal control task of~a~semi-Markov 
process with~finite set of~states}

\def\aut{P.\,V.~Shnurkov$^{1}$, A.\,K.~Gorshenin$^{2}$, and~V.\,V.~Belousov$^{2}$}

\def\autkol{P.\,V.~Shnurkov, A.\,K.~Gorshenin, and~V.\,V.~Belousov}

\titel{\tit}{\aut}{\autkol}{\titkol}

\vspace*{-9pt}


    
\noindent
$^1$National Research University Higher School of Economics, 34~Tallinskaya Str., 
Moscow, 123458, Russian\linebreak
$\hphantom{^9}$Federation

\noindent
$^2$Institute of Informatics Problems, Federal Research Center 
``Computer Science and Control'' of the Russian\linebreak
$\hphantom{^9}$Academy of Sciences, 44-2~Vavilova Str., 
Moscow 119333, Russian Federation



\def\leftfootline{\small{\textbf{\thepage}
\hfill INFORMATIKA I EE PRIMENENIYA~--- INFORMATICS AND
APPLICATIONS\ \ \ 2016\ \ \ volume~10\ \ \ issue\ 4}
}%
 \def\rightfootline{\small{INFORMATIKA I EE PRIMENENIYA~---
INFORMATICS AND APPLICATIONS\ \ \ 2016\ \ \ volume~10\ \ \ issue\ 4
\hfill \textbf{\thepage}}}

\vspace*{3pt}


\Abste{The theoretical verification of the new method of finding 
the optimal strategy of control of a~semi-Markov process with finite set of states is 
presented. The paper considers Markov randomized strategies of control, determined by 
a~finite collection of probability measures, corresponding to each state. The quality 
characteristic is the stationary cost index. This index is a~linear-fractional integral 
functional, depending on collection of probability measures, giving the strategy of control. 
Explicit analytical forms of integrands of numerator and denominator of this 
linear-fractional integral functional are known. The basis of consequent results is 
the new generalized and strengthened form of the theorem about an extremum of 
a~linear-fractional integral functional. It is proved that problems of existence 
of an optimal control strategy of a~semi-Markov process and finding this strategy 
can be reduced to the task of numerical analysis of global extremum for 
the given function, depending on finite number of real arguments.}

\KWE{optimal control of a~semi-Markov process; stationary cost index of quality control; 
linear-fractional integral functional}




\DOI{10.14357/19922264160408} 

\vspace*{-16pt}

\Ack
\noindent
The research was partially supported by the Russian Foundation 
for Basic Research (project 15-07-05316).



%\vspace*{3pt}

  \begin{multicols}{2}

\renewcommand{\bibname}{\protect\rmfamily References}
%\renewcommand{\bibname}{\large\protect\rm References}

{\small\frenchspacing
 {%\baselineskip=10.8pt
 \addcontentsline{toc}{section}{References}
 \begin{thebibliography}{99}
\bibitem{1-1}
\Aue{Howard,~R.\,A.} 1960. \textit{Dynamic programming and Markov processes}. 
Cambridge, MA: MIT Press. 136~p.
\bibitem{2-1}
\Aue{Rykov,~V.\,V.} 1966. Upravlyaemye markovskie protsessy 
s~konechnymi prostranstvami sostoyaniy i~upravleniy 
[Controlled Markov processes with finite spaces of states and controls ]. 
\textit{Teoriya veroyatnostey i~ee primeneniya} 
[Theory of Probability and Its Applications] 11(2):343--351.
\bibitem{3-1}
\Aue{Jewell,~W.\,S.} 1963. Markov-renewal programming. 
\textit{Oper. Res.} 11:938--971.
\bibitem{4-1}
\Aue{Fox,~B.} 1966. Markov renewal programming by linear fractional programming. 
\textit{SIAM J.~Appl. Math.} 14:1418--1432.
\bibitem{5-1}
\Aue{Denardo, E.\,V.} 1967. Contraction mappings in the theory underlying dinamic 
programming. \textit{SIAM Rev.} 9:165--177.
\bibitem{6-1}
\Aue{Howard,~R.\,A.} 1963. Research in semi-Markovian decision structures. 
\textit{J.~Oper. Res. Soc. Japan} 6:163--199.
\bibitem{7-1}
\Aue{Osaki,~S., and H.~Mine.} 1968. Linear programming algorithms 
for Markovian decision processes. \textit{J.~Math. Anal. Appl.} 22:356--381.
\bibitem{8-1}
\Aue{Mine,~H., and S.~Osaki.} 1970. 
\textit{Markovian decision processes}. New York, NY: Elsevier. 142~p.
\bibitem{9-1}
\Aue{Gikhman,~I.\,I., and A.\,V.~Skorokhod.} 1977. 
\textit{Upravlyaemye sluchaynye protsessy} 
[Controlled random processes]. Kiev: Naukova Dumka. 251~p.
\bibitem{10-1}
\Aue{Luque-Vasquez,~F., and О.~Herndndez-Lerma.} 1999. 
Semi-Markov control models with average costs. \textit{Appl. Math.} 26(3):315--331.
\bibitem{11-1}
\Aue{Vega-Amaya,~O., and  F.~Luque-Vasquez.} 2000.  
Sample-path average cost optimality for semi-Markov control processes on Borel spaces: 
Unbounded costs and mean holding times. \textit{Appl. Math.} 27(3):343--367.
\bibitem{12-1}
Gnedenko,~B.~V., ed. 1983. 
\textit{Voprosy matematicheskoy teorii nadezhnosti} 
[Problems of the mathematical theory of reliability].  Moscow: Radio i~svyaz'. 376~p.
\bibitem{13-1}
\Aue{Barzilovich,~E.\,Yu., and V.\,A.~Kashtanov.} 1971. 
\textit{Nekotorye matematicheskie voprosy teorii obsluzhivaniya slozhnykh sistem}  
[Some mathematical questions in theory of complex systems maintenance]. 
Moscow: Sovetskoe radio. 272~p.
\bibitem{14-1}
\Aue{Shnurkov,~P.\,V.} 2016. Solution of the unconditional extremum problem for 
a~linear-fractional 
integral functional on a~set of probability measures. 
\textit{Dokl. Math.} 94(2):550--554.
\bibitem{15-1} %14
\Aue{Shiryaev,~A.\,N.} 2016. 
\textit{Probability-1}. Graduate texts in mathematics ser.
New York, NY: Springer. Vol.~95. 503~p.;
2017. \textit{Probability-2.} Vol.~900. 500~p.
\bibitem{16-1}
\Aue{Borovkov,~А.\,А.} 2009. 
\textit{Teoriya veroyatnostey} [Probability theory]. Moscow: Librokom. 656~p.
\bibitem{17-1}
\Aue{Khenneken,~P.\,L., and A.~Tortra.} 1974. 
\textit{Teoriya veroyatnostey i~nekotorye ee prilozheniya} 
[Probability theory and some of its applications]. Moscow: Nauka. 472~p.
\bibitem{18-1}
\Aue{Halmos,~P.} 1950. \textit{Measure theory}. Litton Educational Publishing. 304~p.
\bibitem{19-1}
\Aue{Korolyuk, V.\,S., and A.\,F.~Turbin.} 1976. 
\textit{Polumarkovskie protsessy i~ikh prilozheniya} 
[Semi-Markov processes and their applications]. Kiev: Naukova Dumka. 184~p.
\bibitem{20-1}
\Aue{Janssen,~J., and R.~Manca.} 2006. 
\textit{Applied semi-Markov processes}. New York, NY: Springer. 309~p.
\bibitem{21-1}
\Aue{Shnurkov,~P.\,V, and A.\,V~Ivanov.} 2015. Analysis of a~discrete semi-Markov model of continuous inventory 
control with periodic interruptions of consumption. 
\textit{Discrete Math. \mbox{Appl}.} 25(1):59--67.
\bibitem{22-1} %21
\Aue{Ivanov,~A.\,V.} 2014. Analiz diskretnoy polumarkovskoy modeli upravleniya 
zapasom nepreryvnogo produkta pri periodicheskom prekrashchenii potrebleniya 
[Analysis of a~discrete semi-Markov control model of continuous product inventory 
in a~periodic cessation of consumption].  
Moscow: Natsional'nyy Issledovatel'skiy Universitet ``Vysshaya Shkola Ekonomiki.'' 
PhD Thesis. 120~p.
\bibitem{23-1} %22
\Aue{Bajalinov,~E.\,B.} 2003. 
\textit{Linear-fractional programming. Theory, methods, applications and software}. 
Boston/\linebreak Dordrecht/London: Kluwer Academic Publs. 423~p.
\bibitem{26-1} %24
\Aue{Shnurkov,~P.\,V., and R.\,V.~Mel'nikov.} 2006. Optimal'noe upravlenie 
zapasom nepreryvnogo produkta v modeli regeneratsii [Optimal control of 
a~continuous product inventory in the regeneration model]. 
\textit{Obozrenie prikladnoy i~promyshlennoy matematiki} [Rev. Appl. Ind. Math.]
13(3):434--452.

\bibitem{25-1} %25
\Aue{Shnurkov,~P.\,V., and R.\,V.~Mel'nikov.} 2008. 
Analysis of the problem of continuous-product inventory control under deterministic 
lead time. \textit{Automat. Rem. Contr.} 69(10):1734--1751.

\columnbreak

\bibitem{24-1} %26
\Aue{Shnurkov,~P.\,V.} 1983. Metody issledovaniya zadach optimal'nogo obsluzhivaniya 
v~matematicheskoy teorii nadezhnosti [Research methods of optimal service problems 
in the mathematical theory of reliability].  
Moscow: Moskovskiy Institut Elektronnogo Mashinostroeniya.  PhD Thesis. 


\bibitem{27-1} %27
\Aue{Kudryavtsev,~L.\,D.} 2006. 
\textit{Kurs matematicheskogo analiza} 
[A~course of mathematical analysis]. Vol.~1. Moscow: Drofa. 704~p.

\bibitem{28-1}
\Aue{Shnurkov,~P.\,V.} 1986. Optimal'noe obsluzhivanie na periode do 
pervogo otkaza sistemy [The optimum service period until the first system failure]. 
\textit{Primenenie analiticheskikh metodov v~veroyatnostnykh zadachakh} 
[The application of analytical methods in probabilistic tasks]. Kiev:
Institute of Mathematics of the Academy of Sciences of the USSR. 121--129.

\bibitem{29-1}
\Aue{Shnurkov,~P.\,V., and A.\,V.~Ivanov.} 2013. Issledovanie zadachi optimizatsii 
v~diskretnoy polumarkovskoy modeli upravleniya nepreryvnym zapasom 
[Study of the optimization problem in discrete semi-Markov model of continuous 
inventory control]. \textit{Vestnik MGTU im.\ N.\,E.~Baumana. Ser. 
Estestvennye nauki} [Vestnik of MSTU named after N.\,E.~Bauman. Ser. Natural sciences] 
3(50):62--87.
\bibitem{30-1}
\Aue{Shnourkoff,~P.\,V.} 1997. The two-element system with one restoring device 
optimum maintenance.  \textit{Stoch. Anal. Appl.} 15(5):823--837.
\bibitem{31-1}
\Aue{Shnourkoff,~P.\,V.} 2001. The two-element system optimum maintenance tills 
the first fail. \textit{Stoch. Anal. Appl.} 19(6):1005--1024.
\bibitem{32-1}
\Aue{Gorshenin,~A.\,K., V.\,V.~Belousov, P.\,V.~Shnourkoff, and A.\,V.~Ivanov.}
2015. Numerical research of the optimal control problem in the semi-Markov 
inventory model. \textit{AIP Conference Proceedings} 1648:250007.
\end{thebibliography}

 }
 }

\end{multicols}

\vspace*{-3pt}

\hfill{\small\textit{Received July 15, 2016}}

\Contr

\noindent
\textbf{Shnurkov Peter V.} (b.\ 1953)~---
 Candidate of Science (PhD) in physics and mathematics, 
 associate professor, National Research University Higher School of Economics, 
 34~Tallinskaya Str., Moscow 123458, Russian Federation; \mbox{pshnurkov@hse.ru} 
 
 \vspace*{3pt}
 
 \noindent
\textbf{Gorshenin Andrey K.}  (b.\ 1986)~---
Candidate of Science (PhD) in physics and mathematics, leading scientist, 
Institute of Informatics Problems, Federal Research Center ``Computer Science 
and Control'' of the Russian Academy of Sciences, 44-2~Vavilov Str., Moscow 119333, 
Russian Federation; associate professor, Federal State Budget Educational 
Institution of Higher Education ``Moscow Technological University,'' 
78~Vernadskogo Ave., Moscow 119454, Russian Federation;
\mbox{agorshenin@frccsc.ru}

\vspace*{3pt}

\noindent
\textbf{Belousov Vasiliy V.} (b.\ 1977)~---
Candidate of Science (PhD) in technology, senior scientist, Institute of 
Informatics Problems, Federal Research Center ``Computer Science and Control'' 
of the Russian Academy of Sciences, 44-2~Vavilov Str., Moscow 119333, Russian 
Federation; \mbox{VBelousov@ipiran.ru}
\label{end\stat}


\renewcommand{\bibname}{\protect\rm Литература} 