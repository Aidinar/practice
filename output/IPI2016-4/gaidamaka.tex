\def\stat{gaidamaka}

\def\tit{АНАЛИЗ ХАРАКТЕРИСТИК ИНТЕРФЕРЕНЦИИ\\ В~МОДЕЛИ ВЗАИМОДЕЙСТВИЯ 
УСТРОЙСТВ\\ С~УЧЕТОМ СРЕДЫ РАСПРОСТРАНЕНИЯ СИГНАЛА$^*$}

\def\titkol{Анализ характеристик интерференции в~модели взаимодействия 
устройств с~учетом среды распространения сигнала}

\def\aut{Ю.\,В.~Гайдамака$^1$, С.\,Д.~Андреев$^2$, Э.\,С.~Сопин$^3$, 
К.\,Е.~Самуйлов$^4$, С.\,Я.~Шоргин$^5$}

\def\autkol{Ю.\,В.~Гайдамака, С.\,Д.~Андреев, Э.\,С.~Сопин и др.} 
%К.\,Е.~Самуйлов, С.\,Я.~Шоргин}

\titel{\tit}{\aut}{\autkol}{\titkol}

\index{Гайдамака Ю.\,В.}
\index{Андреев С.\,Д.}
\index{Сопин Э.\,С.} 
\index{Самуйлов К.\,Е.}
\index{Шоргин С.\,Я.}
\index{Gaidamaka Yu.\,V.}
\index{Andreev S.\,D.}
\index{Sopin E.\,S.} 
\index{Samouylov K.\,E.}
\index{Shorgin S.\,Ya.}


{\renewcommand{\thefootnote}{\fnsymbol{footnote}} \footnotetext[1]
{Исследование выполнено при финансовой поддержке Российского научного фонда 
(проект 16-11-10227).}}


\renewcommand{\thefootnote}{\arabic{footnote}}
\footnotetext[1]{Российский университет дружбы народов; Институт проблем информатики Федерального 
исследовательского центра <<Информатика и~управ\-ле\-ние>> Российской академии наук, 
\mbox{ygaidamaka@sci.pfu.edu.ru}}
\footnotetext[2]{Российский университет дружбы народов; Институт проблем информатики Федерального 
исследовательского центра <<Информатика и~управ\-ле\-ние>> Российской академии наук, 
\mbox{serge.andreev@gmail.com}}
\footnotetext[3]{Российский университет дружбы народов; Институт проблем информатики Федерального 
исследовательского центра <<Информатика и~управ\-ле\-ние>> Российской академии наук, 
\mbox{eduard.sopin@gmail.com}}
\footnotetext[4]{Российский университет дружбы народов; Институт проб\-лем информатики Федерального 
исследовательского центра <<Информатика и~управ\-ле\-ние>> Российской академии наук, 
\mbox{ksam@sci.pfu.edu.ru}}
\footnotetext[5]{Институт проблем информатики Федерального исследовательского центра <<Информатика и~управ\-ле\-ние>> Российской 
академии наук, \mbox{sshorgin@ipiran.ru}}

\vspace*{-6pt}

\Abst{Современные беспроводные сети четвертого (4G) и~пятого (5G) поколения 
допускают для обеспечения связи внутри зданий размещение отдельной точки 
беспроводного доступа в~каждом помещении здания. В~статье проведен анализ 
отношения сигнал/интерференция для беспроводных систем, работающих в~смежных 
помещениях, разделенных стенами из различных материалов, с~учетом потери мощности 
при прохождении сигнала сквозь различные среды распространения. Предложен метод 
оценки плотности распределения случайной величины отношения сигнал/интерференция 
путем аппроксимации мощности интерферирующего сигнала несколькими известными 
распределениями. Численный анализ показал, что эффективность таких систем 
характеризуется как параметром распространения сигнала в~пространстве, так 
и~параметром затухания сигнала, зависящим от материала, из которого изготовлены 
стены. Получены результаты для различных материалов и~толщины стен.}

\KW{беспроводная сеть; отношение сигнал/интерференция; прямое взаимодействие 
устройств; распространение сигнала; среда распространения}

\DOI{10.14357/19922264160401} 


\vspace*{-6pt}

\vskip 10pt plus 9pt minus 6pt

\thispagestyle{headings}

\begin{multicols}{2}

\label{st\stat}

\section{Введение}

  Прогнозируемое увеличение объема трафика в~беспроводных сетях создает 
высокие требования к~пятому поколению мобильных систем~[1, 2]. 
Согласно ежегодному отчету компании Cisco <<Наглядный индекс развития 
сетевых технологий>>~\cite{1-gai},\linebreak ежегод\-ный прирост трафика, 
созда\-ва\-емый мобильными беспроводными устройствами, со\-став\-ля\-ет~53\% 
и~по сравнению с~6,2~экзабайт в~месяц ($6{,}2\cdot 10^{18}$) в~2016~г.\ 
достигнет~30,6~экзабайт в~месяц к~2020~г. 

Одной из причин является рост 
числа пользовательских мобильных устройств~--- смартфонов, 
коммуникаторов, планшетов, носимых (wearable) устройств, таких как умные 
очки, браслеты, датчики слежения за физическим состоянием. Кроме того, 
прирост трафика связан с~активным продвижением концепции Интернета 
вещей~[3], которая предусматривает автоматическое или с~минимальным 
вовлечением человека производство и~передачу данных межмашинного 
взаимодействия (machine-to-machine, M2M), собранных системами 
телеметрии и~телеконтроля в~жи\-лищ\-но-ком\-му\-наль\-ном хозяйстве (умный 
дом, умный город), сельском хозяйстве (сенсоры контроля технологических 
параметров в~агрономии, датчики отслеживания и~мониторинга 
в~животноводстве), в~интеллектуальных системах без\-опас\-ности. По данным 
текущего отчета компании Ericsson <<Ericsson Mobility Report>>~\cite{2-gai}, 
в котором исследуются различные аспекты мобильной индустрии, на 
июнь~2016~г.\ на~5~млрд чел., проживающих в~зоне действия 
беспроводных сетей, приходится~7,4~млрд мобильных  
подключений~\cite{2-gai}, а~к~2020~г.\ эти цифры вырастут до~7,8~млрд 
и~11,6~млрд соответственно~\cite{1-gai}. 

Таким образом, уже сегодня во 
многих странах число абонентов мобильной связи превышает численность 
населения, на душу населения приходится в~среднем~1,5~мобильных 
устройства. 
  
  Беспроводные сети последующих поколений должны не только покрывать 
огромные территории с~многомиллиардным населением, но и~быть способны 
решить задачи со\-ци\-аль\-но-тех\-но\-ло\-ги\-че\-ской эволюции 
следующего десятилетия, обеспечить разнообразие и~масштабируемость 
услуг для пользователей. Сегодня действуют сети четвертого поколения 4G 
(International Mobile Telecommunications Advanced, IMT-Advanced), 
работающие на технологиях LTE Advanced (LTE-A) и~WiMAX~2  
(WMAN-Advanced), в~России сотовые операторы осуществляют 
пользовательский доступ в~Интернет на скорости передачи данных 
до~300~Мбит/с. Коммерческий запуск сетей стандарта 5G прогнозируется 
в~2020~г. В~2016~г.\ тестовые пилотные испытания мобильной сети пятого 
поколения на оборудовании компании Nokia показали пиковые скорости 
свыше~30~Гбит/с и~более~1~млн одновременных подключений 
в~соте~\cite{4-gai}, а~в~России в~сети компании <<Мегафон>> на этом 
оборудовании удалось добиться скорости на уровне~5~Гбит/с~\cite{5-gai}. 
Стандартизацией сетей пятого поколения 5G занимается созданная в~2015~г. 
Оперативная группа по сетевым аспектам IMT-2020, которая опубликовала 
так называемую <<дорожную карту>> развития мобильной сети 5G, где 
обозначена задача разработки точных и~подробных требований к~новой сети 
для достижения сверхвысокой скорости передачи, ультранизкой задержки, 
высокой энергоэффективности и~экологической безопасности. 
  
  Эти требования охватывают технологии радиоинтерфейсов, входящие 
в~семейство IMT-2020, кото\-рые стали кандидатами для стандарта сетей 
пято\-го поколения, в~том числе новые методы модуляции, кодирования, 
применение адаптивных антенных решеток с~MIMO (Multiple Input Multiple 
Output) с~узкой направленностью сигнала, а~также ряд сетевых решений, 
обеспечивающих значительное увеличение производительности, включая 
использование малых (микро/пико/фемто) сот~\cite{6-gai}, ретрансляции 
через клиента (client-relays)~\cite{7-gai}, прямые соединения между 
конечными устройствами (device-to-device, D2D)~\cite{8-gai}. Эти 
технологии основываются на повторном пространственном использовании 
частот, механизмы которого учитывают возникающую при этом 
повышенную интерфе-\linebreak ренцию. Особенно заметно интерференция влияет на
производительность беспроводной сети в~гетерогенных сотовых сетях, где 
одновременное применение нескольких технологий беспроводного\linebreak
 доступа 
сочетается с~высокой плотностью пользователей. Универсальной метрикой 
определения\linebreak производительности беспроводных систем явля\-ется отношение 
сигнал/интерференция (SIR,\linebreak Signal-to-Interference Ratio) в~канале от 
передатчика к~приемнику радиосигнала~\cite{9-gai}, при этом 
интерференция зависит от мощностей передатчиков, расстояний между 
интерферирующими устройствами и~от среды распространения сигнала. 
Отношение сиг\-нал/ин\-тер\-фе\-рен\-ция влияет на показатели качества 
передачи в~радиоканале, например коэффициент ошибок на бит (bit error 
rate, BER), максимальную пропускную способность и~спектральную 
эффективность канала связи, которые для услуг сетей последующих 
поколений определены международными стандартами.
  
  Таким образом, задача нахождения отношения  
сиг\-нал/ин\-тер\-фе\-рен\-ция для типичных сетевых конфигураций имеет 
особое значение, поскольку ее решение позволяет оценить применимость 
сценариев взаимодействия устройств в~современных и~перспективных 
беспроводных сетях. При анализе интерференции для моделирования 
расположения мобильных устройств традиционно применяются 
пространственные точечные процессы~\cite{10-gai}, которые позволяют для 
устрой\-ст\-ва-при\-ем\-ни\-ка в~одной из точек процесса оценить искомую 
метрику~--- отношение сиг\-нал/ин\-тер\-фе\-рен\-ция~--- с~учетом 
интерференции от остальных устройств-пе\-ре\-дат\-чи\-ков~--- соседних 
точек процесса. 

Так, в~\cite{11-gai} был исследован пространственный 
пуассоновский точечный процесс (Poisson Point Process, PPP), для которого 
известна формула рассто\-яния до $n$-бли\-жай\-ше\-го соседа, а~также\linebreak 
процесс твердого ядра Маттерна (Matern Hardcore\linebreak Process, MHC), который 
позволяет учесть ограни\-че\-ния на расстояние внутри пары  
при\-ем\-ник--пе\-ре\-дат\-чик. Для этих процессов показана сложность\linebreak 
получения распределения величины \mbox{отношения}  
сиг\-нал/ин\-тер\-фе\-рен\-ция в~замкнутой аналитической форме, поэтому 
актуальной является задача подбора аппроксимации для нахождения 
отношения сиг\-нал/ин\-тер\-фе\-рен\-ция. 

 
  
  В статье проблема нахождения отношения сиг\-нал/ин\-тер\-фе\-рен\-ция 
исследуется для случая прямого взаимодействия устройств  
(D2D-вза\-имо\-дей\-ст\-вия) внутри помещений, например в~офисных\linebreak 
зданиях или торгово-развлекательных центрах, где для обеспечения 
покрытия на каждом этаже размещается несколько точек беспроводного 
доступа (Access Point, AP), имеющих относительно небольшую зону 
покрытия. При этом предполагается, что соседние точки располагаются 
в~смежных помещениях. Особенностью этой задачи является фиксированное 
расположение передатчиков, что не позволяет при анализе интерференции 
воспользоваться в~полной мере аппаратом стохастической гео\-мет\-рии~--- 
дисциплины, изучающей взаимоотношения между гео\-мет\-ри\-ей и~теорией 
вероятностей, которая развилась из классической интегральной гео\-мет\-рии 
и~задач о~геометрической вероятности с~привнесением идей и~методов 
теории случайных процессов, в~особенности тео\-рии точечных процессов. Для 
исследованного случая аналитические формулы, учитывающие 
интерференцию от нескольких передатчиков из смежных помещений, 
получены в~предположении об идентичности основных характеристик 
интерференции для всех источников и~являются развитием результатов для 
случая одного интерферирующего источника, известных из~\cite{12-gai}. 

Также получен метод оценки плотности распределения случайной величины (с.в.)\ отношения  
сиг\-нал/ин\-тер\-фе\-рен\-ция с~по\-мощью аппроксимации\linebreak мощности 
интерферирующего сигнала несколькими классическими распределениями, 
который учитывает потерю мощности при прохождении сигнала сквозь 
различные среды распространения\linebreak (пространство внутри помещений, стены, 
межэтажные перекрытия). 
%
Проведено сравнение\linebreak аппроксимации для случая 
нормального и~гиперэкспоненциального распределений с~разработанной 
имитационной моделью. Приведена оценка отношения 
сиг\-нал/ин\-тер\-фе\-рен\-ция для различных материалов стен.

\section{Сценарий взаимодействия беспроводных устройств}

  Исследуется сценарий прямого взаимодействия устройств внутри смежных 
помещений, имеющих прямоугольную или квадратную форму (рис.~1). 
  

  
  Построение модели проведем в~обозначениях статьи~\cite{12-gai}. 
В~каждом помещении имеется собственная точка доступа~Rx$_i$, 
$i\hm=\overline{0,4}$, которая размещена в~геометрическом центре 
помещения. Техника повторного пространственного использования
 частот 
предполагает назначение одного и~того же радиоканала для пар  
при\-ем\-ник--пе\-ре\-дат\-чик, нахо-\linebreak\vspace*{-12pt}
\noindent
 \begin{center}  %fig1
 \vspace*{12pt}
 \mbox{%
\epsfxsize=75.343mm 
\epsfbox{gai-1.eps}
}
\end{center}

%\vspace*{3pt}

\noindent
{{\figurename~1}\ \ \small{Схема взаимодействия устройств в~смежных помещениях}}


% \vspace*{6pt}



\addtocounter{figure}{1}


\noindent
дящихся в~смежных помещениях. 
Исследуется интерференция на приемнике (точке доступа)~Rx$_0$ для 
восходящего радиоканала (uplink) в~центральном помещении. 
Предполагается, что в~каждом из четырех смежных помещений имеется один 
интерферирующий передатчик (мобильное устройство пользователя)~Tx$_i$, 
$i\hm=\overline{1,4}$. При этом каждая координата передатчиков имеет 
равномерное распределение по длине стороны прямоугольника. Расстояние 
между передатчиком~Tx$_0$ и~приемником~Rx$_0$ в~центральном 
помещении обозначим~$R_0$. Мощность интерферирующего сигнала от 
передатчика~Tx$_i$ зависит от расстояния между передатчиком~Tx$_i$ 
и~приемником~Rx$_0$, которое обозначим~$D_i$, $i\hm=\overline{1,4}$. 
Затухание сигнала при прохождении сквозь различные среды учитывается 
в~модели с~помощью поправочных коэффициентов~$B_i$. При построении 
модели использован подход~\cite{13-gai}~--- при оценке интерференции 
учитывается только вклад слагаемых, соответствующих устройствам, 
расположенным в~помещениях, смежных с~центральным, и~пренебрегается 
интерферирующим сигналом от остальных устройств, затухающим при 
прохождении через два и~более экрана (стены, перегородки). 

\section{Модель системы и~метод анализа показателей качества}

  Для сценария с~четырьмя интерферирующими устройствами, 
расположенными в~помещениях, смежных с~центральным, как показано на 
рис.~1, величина~SIR отношения сиг\-нал/ин\-тер\-фе\-рен\-ция может 
быть рассчитана по формуле:
  \begin{equation}
  \mathrm{SIR}= \fr{S(R_0)}{\sum\nolimits_{i=1}^4 B_i I_i(D_i)}\,,
  \label{e1-gai}
\end{equation}
где $S(R_0)$~--- мощность <<полезного>> сигнала от передатчика 
в~центральном помещении, $\sum\nolimits_{i=1}^4 B_iI_i(D_i)$~--- мощность 
суммарного интерферирующего сигнала от других передатчиков.
%
При этом величины $S(R_0)$ и~$I_i(D_i)$ определяются сходным образом:
\begin{align}
S(R_0)&= g_o R_0^{-\gamma_0}\,;\label{e2-gai}\\
\sum\limits_{i=1}^4 B_i I_i(D_i) &= \sum\limits_{i=1}^4 B_i g_i  
D_i^{-\gamma_i}\,.\label{e3-gai}
\end{align}
  Здесь $g_i$~--- базовые мощности сигнала передатчика в~центральном 
помещении ($i\hm=0$) и~сигналов интерферирующих передатчиков 
в~смежных помещениях ($i\hm=\overline{1,4}$); $\gamma_i$~--- 
коэффициент затухания
 (path loss exponent) сигнала в~центральном 
($i\hm=0$)
 и~смежных ($i\hm=\overline{1,4}$) помещениях; $B_i$~--- 
коэффициент потерь сигнала при прохождении сквозь стены, 
$i\hm=\overline{1,4}$.
  
  Дальнейший анализ проводим для квадратных помещений со стороной 
$c\hm= a_j\hm= b_j$, $j\hm=\overline{1,3}$, в~предположении о равенстве 
мощностей сигналов передатчиков $g_i\hm=g$, одинаковых коэффициентах 
затухания $\gamma_i\hm=\gamma$ во всех помещениях, 
$i\hm=\overline{0,4}$, а~также одинаковых коэффициентах потерь сигнала 
при прохождении сквозь стены $B_i\hm=B$, $i\hm=\overline{1,4}$. Введем 
случайные величины $\xi\hm= \mathrm{SIR}$, $\eta_1\hm= R^{-\gamma}$ и~$\eta_2\hm= 
D^{-\gamma}$. Из~\cite{14-gai} известна функция плотности 
$W_{\eta_2}(y_2)$ с.в.\ $\eta_2\hm= D^{-\gamma}$:

\noindent
  \begin{multline}
  W_{\eta_2} (y_2)= \left( \fr{2}{\gamma c^2}\,y_2^{-2/\gamma-1}\right) \times{}\\
\times{}\begin{cases}
  0, &\hspace*{-5mm}y_2\geq \left(\fr{c}{2}\right)^{\gamma}\,;\\[6pt]
  \arcsin \left[ \fr{c}{2y_2^{-1/\gamma}}\right]- {}\\[14pt]
  &\hspace*{-40mm}{}-\arcsin \left[ \fr{\sqrt{-
9c^2+4y_2^{-2/\gamma}}}{2y_2^{-1/\gamma}}\right]\,,\\[14pt]
 &\hspace*{-30mm}  \left( 
c\sqrt{\fr{5}{2}}\right)^{-\gamma}<y_2 \leq \left( \fr{3c}{2}\right)^{-\gamma}\,;\\[14pt]
  \arcsin\left[\fr{\sqrt{-c^2+4y_2^{-2/\gamma}}}{2y_2^{-1/\gamma}}\right]\,, \\[6pt]
  &\hspace*{-28mm} 
  \left( \fr{c}{\sqrt{2}}\right)^{-\gamma}< y_2\leq \left( \fr{c}{2}\right)^{-
\gamma}\,;\\[14pt]
  \arcsin \left[ \fr{c}{2y_2^{-1/\gamma}}\right]\,,\\[6pt]
   &\hspace*{-30mm}  \left( \fr{3c}{2} 
\right)^{-\gamma} <  y_2\leq \left( \fr{c}{\sqrt{2}}\right)^{-\gamma}\,.
  \end{cases}
  \label{e4-gai}
  \end{multline}
  
  Формула~(\ref{e4-gai}) служит для оценки характеристик интерференции 
одного из слагаемых в~знаменателе выражения~(\ref{e1-gai}) для отношения 
сиг\-нал/ин\-тер\-фе\-рен\-ция. Для оценки основного исследуемого 
показателя, а~именно отношения сиг\-нал/ин\-тер\-фе\-рен\-ция, необходимо 
для заданных распределений с.в.\ расстояния~$R_0$ от приемника~Rx$_0$ 
до передатчика~Tx$_0$ и~с.в.\ расстояний~$D_i$ от приемника~Rx$_0$ до 
передатчика~Tx$_i$, $i\hm=\overline{1,4}$, найти распределение с.в.\ 
$S(R_0)$ по формуле~(\ref{e2-gai}) и~$I_i(D_i)$ по формуле~(\ref{e3-gai}), 
а~затем получить совместное распределение с.в.~SIR по  
формуле~(\ref{e1-gai}).
  
\section{Аппроксимация нормальным распределением}

  Особенность рассматриваемой задачи заключается в~том, что плотность 
суммарного интерферирующего сигнала~(\ref{e3-gai}) представляет собою 
свертку выражений вида~(\ref{e4-gai}). Исследование ее в~аналитическом 
виде нецелесообразно, поэтому далее в~статье решается задача нахождения 
аппроксимирующего распределения для суммарного интерфери\-ру\-юще\-го 
сигнала. Сделаем упрощающее предположение о том, что основные 
показатели интерференции идентичны для всех слагаемых в~(\ref{e3-gai}). 
Тогда плотность распределения суммарного интерфе\-ри\-ру\-юще\-го сигнала 
можно аппроксимировать нормальным законом  
$N(\tilde{\mu},\tilde{\sigma})$, параметры которого в~предположении о том, 
что материал и~толщина стен одинаковы, имеют вид:  
$$
\tilde{\mu}= 
4B\mu(D)\,;\enskip  \tilde{\sigma}^2= 4B\sigma^2(D)\,,
$$
 где $\mu(D)$  и~$\sigma^2(D)$ определяются из~(\ref{e4-gai}). 
  
  В сделанных предположениях плотность распределения $W_{\eta_1}(y_1)$ 
мощности полезного сигнала определяется формулой:
  \begin{multline}
  W_{\eta_1}(y_1)= \left( \fr{2}{\gamma c^2}\,y_1^{-2/\gamma-1}\right)\times{}\\
  {}\times
  \begin{cases}
  \pi\,, &\hspace*{-10mm} \left(\fr{c}{2}\right)^{-\gamma}<y_1<\infty\,;\\[3pt]
  2\left( \arcsin \left[ \fr{c}{2y_1^{-1/\gamma}}\right] -{}\right. & \\[12pt]
  &\left.\hspace*{-30mm}{}-\arcsin\left[ \fr{\sqrt{-c^2 
+ 4y_1^{-2/\gamma}}} {2y_1^{-1/\gamma}} \right] \right)\,,\\[3pt]
&\hspace*{-20mm} \left( \fr{c} 
{\sqrt{2}}\right)^{-\gamma} <y_1\leq \left( \fr{c}{2}\right)^{-\gamma}\,.
  \end{cases}
  \label{e5-gai}
  \end{multline}
  
  Воспользовавшись известным из~\cite{15-gai} преобразованием 
с.в., получим плотность распределения с.в.\ $\xi\hm= \mathrm{SIR}$ как 
совместную плотность мощности полезного сигнала и~суммы мощностей 
интерферирующих сигналов. Учитывая вид~(\ref{e1-gai}) с.в.~$\xi$, введем 
вспомогательные переменные $y_3\hm= f(x_3, x_4)\hm= x_3/x_4$ 
и~$y_4\hm= x_4$, а~также обратное преобразование $x_3\hm= 
\varphi(y_3,y_4)\hm= y_3y_4$, заметив, что $\partial \varphi(y_3,y_4)/\partial 
y_3\hm= y_4$. Искомую функцию получим, используя соотношение: 
  \begin{multline}
  W_{\eta_1,\eta_2}(y_3,y_4) ={}\\
  {}=w_{\chi_1,\chi_2} \left( \varphi\left(y_3, 
y_4\right),y_4\right) \left\vert \fr{\partial \varphi(y_3,y_4)} {\partial 
y_3}\right\vert\,,
  \label{e6-gai}
\end{multline}
где $w_{\chi_1,\chi_2}(y_3,y_4)$~--- совместная плотность  
с.в.~$R_0^{-\gamma}$ и~$\sum\nolimits_{i=1}^4 D_i^{-\gamma_i}$. После 
интегрирования~(\ref{e6-gai}) по~$y_4$ получаем плотность с.в.~$\xi$ 
в~виде: 

\noindent
\begin{equation*}
W_{\xi}(y_3) =\int\limits_{\mathbf{Y}} w_{\chi_1,\chi_2}\left( \varphi\left( 
y_3,y_4\right),y_4\right)  \left\vert \fr{\partial\varphi(y_3,y_4)}{\partial 
y_3}\right\vert \, dy_4\,,
%\label{e7-gai}
\end{equation*}
где $\mathbf{Y}$~--- область значений переменной~$y_4$.

Вследствие ограничений выражений~(\ref{e4-gai}) и~(\ref{e5-gai}) 
множество~$\mathbf{Y}$ имеет вид: 
\begin{multline*}
\hspace*{-4mm}\mathbf{Y}=\left\{ \left( \fr{c}{\sqrt{2}}\right)^{-\gamma} <y_1 \leq 
\left( \fr{c}{2}\right)^{-\gamma},\,
-\infty <y_2<\infty\right\}\cup{}\\
{}\cup \left\{ \left( \fr{c}{2}\right)^{-\gamma}< y_1<\infty,\,
-\infty<y_2<\infty\right\}\,.
%\label{e8-gai}
\end{multline*}
Таким образом, 
\begin{align*}
\mathbf{Y} &= \mathrm{Y}^1\cup \mathrm{Y}^2\,;\\[6pt]
\mathrm{Y}_1 &={}\\[1pt]
&\hspace*{-7mm}{}= \left\{ y_3<0,\ \fr{e^{\gamma\ln 2-\gamma \ln c}}{y_3}<
y_4<\fr{e^{(1/2)\gamma\ln 2- \gamma \ln c}}{y_3}\right\} \cup{}\\[6pt]
&\hspace*{-7mm}{}\cup \left\{ y_3>0\,,\
\fr{e^{(1/2)\gamma\ln 2-\gamma\ln c}}{y_3}<y_4<\fr{e^{\gamma\ln 2-\gamma \ln c}}{y_3}\right\}\,;\\[6pt]
\mathrm{Y}_2 &= \left\{ y_3<0\,,\ y_4<\fr{e^{\gamma\ln 2-\gamma \ln c}}{y_3}\right\}
 \cup {}\\[6pt]
 &\hspace*{22.5mm}{}\cup  \left\{ y_3>0\,,\ y_4>\fr{e^{\gamma\ln 2- \gamma\ln c}}{y_3}\right\}\,.
 % \label{e9-gai}
 \end{align*}

%\smallskip

\noindent
\textbf{Утверждение.} Плотность~$W_\xi(y_3)$ с.в.~$\xi$ отношения 
сигнал/интерференция имеет вид:
\begin{multline*}
W_\xi (y_3) = {}\\
{}=\begin{cases}
\displaystyle \,\int\limits_{M_1} I_1(y_3, y_4)\,dy_4+\int\limits_{M_3} I_1(y_3,y_4)\,dy_4\,, &\ y_3<0\,;\\[16pt]
\displaystyle \,\int\limits_{M_2} I_2(y_3, y_4)\,dy_4+\int\limits_{M_4} I_2(y_3,y_4)\,dy_4\,, &\ y_3\geq0
\end{cases}\hspace*{-3.75pt}
%\label{e10-gai}
\end{multline*}
с областями интегрирования

\noindent
\begin{align*}
M_1&={}\\
&\hspace*{-15pt}{}=\left\{ (y_3,y_4):\ \fr{e^{\gamma\ln 2-\gamma \ln c}}{y_3}<y_4<
\fr{e^{(1/2)\gamma\ln 2-\gamma\ln c}}{y_3}\right\}\,;
\end{align*}

\noindent
\begin{align*}
M_2&={}\\
&\hspace*{-15pt}{}=\left\{ (y_3,y_4):\ \fr{e^{(1/2)\gamma\ln 2-\gamma \ln c}}{y_3}<y_4<
\fr{e^{\gamma\ln 2-\gamma\ln c}}{y_3}\right\}\,;\\
M_3&=\left\{ (y_3,y_4):\ y_4<\fr{e^{\gamma\ln 2-\gamma \ln c}}{y_3}\right\}\,;\\
M_4&=\left\{ (y_3,y_4):\ y_4>\fr{e^{\gamma\ln 2-\gamma \ln c}}{y_3}\right\}
\end{align*}
и подынтегральными выражениями
\begin{align*}
I_1(y_3,y_4)&=
\left( \vphantom{\left[ \fr{\sqrt{-c^2+4(y_3 y_4)^{-2/\gamma}}}
{2(y_3 y_4)^{-1/\gamma}}\right]}
\arcsin\left[
\fr{c}{2(y_3 y_4)^{-1/\gamma}}
\right] -{}\right.\\
&\left.\hspace*{-15mm}{}-\arcsin\left[ \fr{\sqrt{-c^2+4(y_3 y_4)^{-2/\gamma}}}
{2(y_3 y_4)^{-1/\gamma}}\right]\right) \Bigg/ \\
&\hspace*{-10mm}\left(\fr{(4y_4/(\gamma c^2))\left(y_3 y_4\right)^{-2/\gamma-1}}
{\tilde{\sigma}\sqrt{2\pi}}\,e^{(y_4-\tilde{\mu})^2/(2\tilde{\sigma}^2)}\right)\,;\\
I_2(y_3,y_4) &= \fr{2\sqrt{2\pi^3}y_4 \tilde{\sigma}(y_3 y_4)^{-2/\gamma-1}}
{\gamma c^2 e^{-(y_4-\tilde{\mu}^2)/(2\tilde{\sigma}^2)}}\,.
\end{align*}

  
  Точность предложенной аппроксимации проверена с~по\-мощью 
имитационного моделирования методом Мон\-те-Кар\-ло случайных 
переменных, а~именно координат расположения устройств в~помещениях, 
с~последующей оценкой характеристик исследуемого показателя~--- 
отношения сиг\-нал/ин\-тер\-фе\-рен\-ция. Разработанная в~\cite{13-gai} 
имитационная программа позволяет варьировать входные па\-ра\-мет\-ры модели, 
представляющие интерес для данного исследования, в~том числе размеры 
помещений, коэффициенты потерь и~коэффициенты затухания, число 
экспериментов и~час\-то\-ту дискретизации при построении гистограммы 
плотностей распределения основных случайных величин. Выходными 
данными имитационной программы являются функции эмпирической 
плотности, значения среднего и~среднеквадратического отклонения, а~также 
оценки квантилей основного исследуемого показателя~--- отношения  
сиг\-нал/ин\-тер\-фе\-рен\-ция.

\begin{figure*} %fig2
\vspace*{1pt}
\begin{minipage}[t]{78mm}
\begin{center}
\mbox{%
\epsfxsize=70.373mm
\epsfbox{gai-2.eps}
}
\end{center}
\vspace*{-9pt}
\Caption{Отношение сигнал/интерференция: аппроксимация 
(\textit{1}~--- нормальное распределение; \textit{2}~--- гиперэкспоненциальное
распределение) и~имитационное\protect\linebreak моделирование~(\textit{3}).
Сплошные кривые и~черные значки~--- $\gamma\hm=2$; пунктирные и~штриховые кривые 
и~пустые значки~--- $\gamma\hm=3$}
\end{minipage}
%\end{figure*}
\hfill
%\begin{figure*} %fig3
\vspace*{1pt}
\begin{minipage}[t]{78mm}
\begin{center}
\mbox{%
\epsfxsize=70.342mm
\epsfbox{gai-3.eps}
}
\end{center}
\vspace*{-9pt}
\Caption{Отношение сигнал/интерференция: среда распространения сигнала:
\textit{1}~--- стекло 13~мм, $B\hm=2$~дБ;
\textit{2}~--- кирпич 178~мм, $B\hm=5$~дБ;
\textit{3}~--- бетон 102~мм, $B\hm=12$~дБ;
\textit{4}~--- каменная кладка 406~мм, $B\hm=17$~дБ;
\textit{5}~--- железобетон 203~мм, $B\hm=27$~дБ;
\textit{6}~--- бетон 305~мм, $B\hm=35$~дБ}
\end{minipage}
\vspace*{6pt}
\end{figure*}

\section{Численные примеры}

  Результаты численного анализа, которые позволяют оценить отношение 
сиг\-нал/ин\-тер\-фе\-рен\-ция на приемнике в~центральном помещении, 
пред\-став\-ле\-ны на рис.~2 и~3. Моделирование проводилось для квадратных 
помещений с~длиной стороны $c\hm= 10$ для двух значений коэффициента 
затухания $\gamma\hm=2$ и~3. Число экспериментов по 
моделированию положения мобильных устройств в~помещениях для 
построения эмпирической функции плотности с.в.~SIR отношения  
сиг\-нал/ин\-тер\-фе\-рен\-ция составило~10$^6$. 
  
  Форма графика для аппроксимации с~помощью нормального закона на 
рис.~2 повторяет форму кривой имитационного моделирования. Однако 
существенный сдвиг не позволяет рекомендовать эту аппроксимацию для 
оценки отношения сиг\-нал/ин\-тер\-фе\-рен\-ция, особенно при малых 
значениях этого показателя, которые важны для приложений. При 
$\mathrm{SIR}\hm<1$ мощность суммарного интерферирующего сигнала превосходит 
мощность полезного сигнала, что дает скорость передачи данных в~канале, 
неприемлемую для большинства услуг, предоставляемых в~сетях 
последующих поколений. Для приложений важно оценить квантили нижних 
уровней с.в.\ отношения сиг\-нал/ин\-тер\-фе\-рен\-ция, показывающие долю 
мобильных пользователей, которые не смогут получить запрошенную услугу 
из-за низкого качества передачи данных в~канале. 

В~ходе численного 
эксперимента для оценки функции плот\-ности отношения 
сиг\-нал/ин\-тер\-ферен\-ция была предложена аппроксимация мощ\-ности 
интерфериру\-юще\-го сигнала от одного\linebreak источника с~помощью усеченного 
трехпараметрического гиперэкспо\-нен\-ци\-ального распределения, па\-ра\-мет\-ры 
которого подобраны так, что относительная погрешность аппроксимации не 
превышает~4\%. Плот\-ность распределения~$W_\chi(x)$ отношения  
сиг\-нал/ин\-тер\-фе\-рен\-ция~$\chi$ от каждого источника имеет вид:
$$
W_\chi(x)=\begin{cases}
0\,, &\ x<0{,}5\,;\\[3pt]
q\alpha_1 e^{\alpha_1x} +(1-q)\alpha_2 e^{\alpha_2x}\,, &\ x\geq 0{,}5\,,
\end{cases}
$$
 где для $\gamma\hm=2$ $q \hm= 0{,}517$,  $\alpha_1\hm=8571{,}284$ 
 и~$\alpha_2\hm=1516{,}775$,
 а~для  $\gamma\hm=3$ $q\hm= 0{,}534$,  $\alpha\hm=1295{,}618$ 
и~$\alpha_2\hm= 280{,}362$.
    
  Эта аппроксимация, график которой также приведен на рис.~2 
и~с~высокой степенью точности повторяет кривую имитационного 
моделирования, может быть рекомендована для оценки отношения 
 сиг\-нал/ин\-тер\-фе\-рен\-ция снизу.
  
 


  Рисунок 3 иллюстрирует зависимость характеристик получаемого 
приемником ин\-тер\-фе\-ри\-ру\-юще\-го сигнала мобильных устройств, ра\-бо\-та\-ющих 
в~смежных помещениях, от материала, из которого изготовлены 
межкомнатные перегородки. 
  
 


  Показаны полученные с~помощью имитационного моделирования графики 
плотности распределения с.в.~SIR отношения сигнал/интерференция 
с~коэффициентом затухания $\gamma\hm=3$ для нескольких значений 
коэффициента потерь~$B$ при про\-хож\-де\-нии сигнала сквозь различные 
среды, взятых из стандарта Международного союза  
электросвязи~\cite{16-gai}. На практике показатель SIR часто измеряют 
в~децибелах, поэтому на рис.~3 по оси абсцисс отложено отношение  
сиг\-нал/ин\-тер\-фе\-рен\-ция~$S_{\mathrm{dB}}$ в~децибелах, которое получено 
из величины SIR дополнительным преобразованием $S_{\mathrm{dB}}\hm= 
10\lg(\mathrm{SIR})$. Здесь величина~0~дБ означает, что мощности полезного 
и~ин\-тер\-фе\-ри\-ру\-юще\-го сигналов равны. Заметим, что, например, для 
беспроводных сетей чет\-вер\-то\-го поколения LTE при $S_{\mathrm{dB}}\hm\geq 13$~дБ 
качество\linebreak канала считается хорошим, а~при $S_{\mathrm{dB}}\hm\geq 20$~дБ~--- 
отличным. Современные методы модуляции\linebreak и~кодирования в~сетях 
по\-сле\-ду\-ющих поколений, в~част\-ности мультиплексирование с~разделением 
по ортогональным час\-то\-там (Orthogonal Frequency Division Multiplexing, 
OFDM), позволяют передавать данные даже при отрицательном 
значении~$S_{\mathrm{dB}}$, т.\,е.\ в~случае, когда в~получаемом приемником сигнале 
помех больше, чем полезной час\-ти. Основываясь на результатах 
проведенного эксперимента, можно сделать вывод, что экранирование 
заметно влияет на значение отношения сиг\-нал/ин\-тер\-фе\-рен\-ция, при 
этом существенную роль играют как материал, из которого изготовлен экран, 
так и~толщина этого экрана. Если стеклянные перегородки между 
помещениями практически не гасят интерферирующий сигнал мобильных 
устройств, работающих в~смежных помещениях на той же или близких 
радиочастотах, то наличие перегородки в~виде бетонной стены толщиной~30~см 
обеспечивает отличное качество передачи данных ($S_{\mathrm{dB}}\hm\geq 
20$~дБ) для подавляющего большинства (свыше~97\%) мобильных 
устройств.

\vspace*{-6pt}
  
\section{Заключение}

  В статье представлен аналитический метод оценки плотности 
распределения с.в.\ отношение сиг\-нал/ин\-тер\-фе\-рен\-ция при 
взаимодействии устройств внутри смежных помещений, имеющих 
прямоугольную форму. Исследована интерференция в~восходящем 
радиоканале от мобильного устройства к~точке доступа, при этом 
учитывается потеря мощности при прохождении сигнала\linebreak сквозь различные 
среды (пространство внутри помеще\-ний, стены, межэтажные перекрытия). 
Заме\-тим, что интерференция в~нисходящем радиоканале для описанной 
задачи взаимодействия мобильного устройства с~точкой доступа исследована 
в~\cite{13-gai}. При этом, как и~в исследованном в~настоящей статье случае, 
распределение с.в.\ отношения сиг\-нал/ин\-тер\-фе\-рен\-ция имеет 
достаточно громоздкое представление.
  
  Предложен метод оценки отношения сиг\-нал/\linebreak ин\-тер\-фе\-рен\-ция  
с~по\-мощью ап\-прок\-си\-ми\-ру\-юще\-го распределения для интерферирующего 
сигнала от одного источника. Сравнение аппроксимации для случая 
нормального и~гиперэкспоненциально-\linebreak го распределений с~результатами 
имитационного\linebreak моделирования выявило, что более точную аппроксимацию 
дает полученное численно гиперэкспоненциальное распределение, которое 
можно использовать для оценки снизу отношения  
сиг\-нал/\linebreak ин\-тер\-фе\-рен\-ция. 

Приведена оценка отношения 
  сиг\-нал/ин\-тер\-фе\-рен\-ция для различных материалов стен, которая 
позволяет сделать вывод о~существенной за\-ви\-си\-мости производительности 
беспроводных соединений в~рассмотренном сценарии взаимодействия 
беспроводных устройств от потерь мощности за счет распространения 
сигнала в~пространстве и~материала, из которого изготовлены стены между 
помещениями. 
  
  \smallskip
  
  Авторы выражают благодарность магистру кафедры прикладной 
информатики и~теории вероятностей РУДН Р.~Ковальчукову за вычисления 
и~оформление графиков в~статье.

\vspace*{-6pt}

{\small\frenchspacing
 {%\baselineskip=10.8pt
 \addcontentsline{toc}{section}{References}
 \begin{thebibliography}{99}
\bibitem{1-gai}
Cisco Visual Networking Index: Global Mobile Data Traffic Forecast Update, 2015--2020 White 
Paper.~--- Cisco, February~01, 2016. {\sf  
http://www.cisco.com/c/en/us/ solutions/collateral/service-provider/visual-networking-index-vni/mobile-white-paper-c11-520862.html}.
\bibitem{2-gai}
Ericsson mobility report.~--- Ericsson, 2016. {\sf  
https://www. ericsson.com/res/docs/2016/ericsson-mobility-report-2016.pdf}.
\bibitem{3-gai}
Overview of Internet of Thing: Recommendation Y.2060 (06/2012).~--- ITU-T, 2012-06-15. {\sf 
https://www. itu.int/rec/T-REC-Y.2060-201206-I}.
\bibitem{4-gai}
Nokia demonstrates world's first 5G-ready network.~--- Nokia, June~28, 2016. {\sf  
http://company.nokia.com/en/ news/press-releases/2016/06/28/nokia-demonstrates-worlds-first-5g-ready-network}.
\bibitem{5-gai}
МегаФон показала мобильный 5G-интернет на скорости 5~Гбит/с. {\sf ferra.ru}, 
21.09.2016. {\sf http://www.ferra. ru/ru/mobile/news/2016/09/21/megafon-nn-5g}.
\bibitem{6-gai}
\Au{Andrews J., Claussen~H., Dohler~M., Rangan~S.} Femtocells: Past, present, and future~// 
IEEE J.~Sel. Areas Commun., 2012. Vol.~30. No.\,3. P.~497--508.
\bibitem{7-gai}
\Au{Lee J., Wang H., Andrews~J., Hong~D.} Outage probability of cognitive relay networks 
with interference constraints~// IEEE Trans. Wir. Comm., 2011. Vol.~10. No.\,2. P.~390--395.
\bibitem{8-gai}
\Au{Fodor G., Parkvall~S., Sorrentino~S., Wallentin~P., Lu~Q., Brahmi~N.} Device-to-device 
communications for national security and public safety~// IEEE Access, 2014. Vol.~2.  
P.~1510--1520.
\bibitem{9-gai}
\Au{Sathya V., Ramamurthy~A., Kumar~S., Tamma~B.} On improving SINR in LTE HetNets 
with D2D relays~// Comput. Commun., 2016. Vol.~83. P.~27--44.
\bibitem{10-gai}
\Au{Baccelli F., \mbox{B{\ptb{\l}}aszczyszyn}~B.} Stochastic geometry and wireless 
networks.~// Foundations and Trends$^\registered$ in Networking, 2010. Vol.~3. No.\,3-4. 
P.~249--449; Vol.~4. No.\,1-2. P.~1--312. doi: 10.1561/1300000006.
\bibitem{11-gai}
\Au{Abaev P., Samouylov~K., Gaidamaka~Yu., Shorgin~S.} Tractable distance distribution 
approximations for hardcore processes~// Distributed Computer and Communication Networks: 
Control, Computation, Communications Conference Proceedings.~--- 
Communications in Computer and Information Science. Vol.~678.~--- Springer, 2016 (in 
press).
\bibitem{12-gai}
\Au{Гайдамака~Ю.\,В., Самуйлов~А.\,К.} Метод расчета характеристик интерференции 
двух взаимодействующих устройств в~беспроводной гетерогенной сети~// Информатика и~
её применения, 2015. Т.~9. Вып.~1. С.~10--15.
\bibitem{13-gai}
\Au{Samuylov A., Moltchanov~D., Gaidamaka~Yu., Begishev~V., Kovalchukov~R., Abaev~P., 
Shorgin~S.} SIR analysis in square-shaped indoor premises~// 30th European Conference on 
Modelling and Simulation Proceedings.~--- Regensburg, Germany: Digitaldruck Pirrot GmbH. 
P.~692--697.
\bibitem{14-gai}
\Au{Samuylov A., Ometov~A., Begishev~V., Kovalchukov~R., Moltchanov~D., Gaidamaka~Yu., 
Samouylov~K., Andreev~S., Koucheryavy~Y.} Analytical performance estimation of  
network-assisted D2D communications in urban scenarios with rectangular cells~// Trans.  
Emerging Telecommunications Technol., 12~November, 2015. P.~1--14. doi: 
10.1002/ett.2999A.
\bibitem{15-gai}
\Au{Левин Б.\,Р.} Теоретические основы статистической радиотехники.~--- 3-е изд.~--- М.: 
Радио и~связь, 1989. 656~с.
\bibitem{16-gai}
Propagation data and prediction methods for the planning of indoor radiocommunication systems 
and radio local area networks in the frequency range 300~ MHz to~100~GHz: Recommendation 
P.1238-8 (07/2015).~--- ITU-R, 2015-07-29. {\sf http://www.itu.int/rec/R-REC-P.1238-8-201507-I}.

 \end{thebibliography}

 }
 }

\end{multicols}

\vspace*{-9pt}

\hfill{\small\textit{Поступила в~редакцию 14.10.16}}

\vspace*{8pt}

%\newpage

%\vspace*{-24pt}

\hrule

\vspace*{2pt}

\hrule

\vspace*{8pt}


\def\tit{INTERFERENCE ANALYSIS OF~THE~DEVICE-TO-DEVICE 
COMMUNICATIONS MODEL WITH~REGARD TO~A~SIGNAL 
PROPAGATION ENVIRONMENT}

\def\titkol{Interference analysis of~the~device-to-device 
communications model with~regard to~a~signal 
propagation environment}

\def\aut{Yu.\,V.~Gaidamaka$^{1,2}$, S.\,D.~Andreev$^{1,2}$, 
E.\,S.~Sopin$^{1,2}$, K.\,E.~Samouylov$^{1,2}$, and~S.\,Ya.~Shorgin$^2$}

\def\autkol{Yu.\,V.~Gaidamaka, S.\,D.~Andreev, 
E.\,S.~Sopin, et al.}
%K.\,E.~Samouylov, and~S.\,Ya.~Shorgin}

\titel{\tit}{\aut}{\autkol}{\titkol}

\vspace*{-9pt}


    
\noindent
$^1$Peoples' Friendship University of Russia, 6~Miklukho-Maklaya Str., Moscow 117198, 
       Russian Federation
       
       \noindent
     $^2$Institute of Informatics Problems, Federal Research Center ``Computer Science and 
Control'' of the Russian\linebreak
$\hphantom{^1}$Academy of Sciences, 44-2~Vavilov Str., Moscow 119333, Russian 
Federation

\def\leftfootline{\small{\textbf{\thepage}
\hfill INFORMATIKA I EE PRIMENENIYA~--- INFORMATICS AND
APPLICATIONS\ \ \ 2016\ \ \ volume~10\ \ \ issue\ 4}
}%
 \def\rightfootline{\small{INFORMATIKA I EE PRIMENENIYA~---
INFORMATICS AND APPLICATIONS\ \ \ 2016\ \ \ volume~10\ \ \ issue\ 4
\hfill \textbf{\thepage}}}

\vspace*{3pt}




\Abste{Modern dense 4G and 5G networks allow placing a wireless access point in every indoor 
location to optimize wireless coverage. The paper analyzes the signal-to-interference 
ratio (SIR) for 
wireless systems operating in adjacent premises, and takes into account the loss of signal power 
during penetration of a signal through different environments. As analytical formulas 
for  
SIR are cumbersome, two approximations of the probability density 
function of interference are developed: an analytical expression for the total interfering signal in 
the form of the normal distribution and a three-parameter hyperexponential distribution with 
numerically assorted parameters for simulation of the interfering signal from each device. As 
a~result, the last one gives the most accurate approximation, which can be used for lower-bound 
estimation of SIR. The authors analyzed SIR in the case of the uniform distribution of devices in the premises of the rectangular shape 
for five wall materials (glass, brick, concrete, masonry blocks, and reinforced concrete) and varying 
wall thickness, and found that the presence of walls between rooms fundamentally changes the 
structure of SIR density.}

\KWE{wireless networks; signal-to-interference ratio; indoor propagation; wall penetration}


\DOI{10.14357/19922264160401} 

\vspace*{-12pt}

\Ack
\noindent
This work was financially supported by the Russian Science Foundation (grant  
No.\,16-11-10227).



%\vspace*{3pt}

  \begin{multicols}{2}

\renewcommand{\bibname}{\protect\rmfamily References}
%\renewcommand{\bibname}{\large\protect\rm References}

{\small\frenchspacing
 {%\baselineskip=10.8pt
 \addcontentsline{toc}{section}{References}
 \begin{thebibliography}{99}

\bibitem{1-gai-1}
Cisco Visual Networking Index: Global Mobile Data Traffic Forecast Update, 2015--2020. 
Available  at: {\sf  
www.cisco.com/c/en/us/solutions/collateral/service-provider/visual-networking-index-vni/mobile-white-paper-c11-520862.html} (accessed November~1, 2016).
\bibitem{2-gai-1}
Ericsson mobility report. Available at: {\sf
www.ericsson. com/res/docs/2016/ericsson-mobility-report-2016.pdf} (accessed November~1, 
2016).
\bibitem{3-gai-1}
Recommendation ITU-T Y.2060. 2012. Overview of Internet of Thing. Available at: {\sf 
www.itu.int/rec/T-REC-Y.2060-201206-I} (accessed November~1, 2016). 
\bibitem{4-gai-1}
Nokia demonstrates world's first 5G-ready network. Available at: {\sf 
company.nokia.com/en/news/press-releases/\linebreak 2016/06/28/nokia-demonstrates-worlds-first-5g-ready-network} (accessed November~1, 2016). 
\bibitem{5-gai-1}
Megafon pokazala mobil'nyy 5G-internet na skorosti 5~Gbit/s [MegaFon showed 
5G mobile-Internet at the speed of 5~Gbps]. Available at: {\sf 
http://www.ferra.ru/ ru/mobile/news/2016/09/21/megafon-nn-5g} (accessed November~1, 2016).
\bibitem{6-gai-1}
\Aue{Andrews, J., H. Claussen, M.~Dohler, and S.~Rangan.} 2012. Femtocells: Past, present, 
and future. \textit{IEEE JSAC} 30:497--508.
\bibitem{7-gai-1}
\Aue{Lee, J., H. Wang, J.~Andrews, and D.~Hong.} 2011. Outage probability of cognitive relay 
networks with interference constraints. \textit{IEEE Trans. Wir. Comm.} 10(2):390--395.



\bibitem{8-gai-1}
\Aue{Fodor, G., S. Parkvall, S.~Sorrentino, P.~Wallentin, Q.~ Lu, and N.~Brahmi}. 2014. 
Device-to-device communications for national security and public safety. \textit{IEEE Access} 
2:1510--1520.
\bibitem{9-gai-1}
\Aue{Sathya, V., A. Ramamurthy, S.~Kumar, and B.~Tamma}. 2015. On improving SINR in 
LTE hetnets with D2D relays. \textit{Comput. Commun.} 83:27--44.
\bibitem{10-gai-1}
\Aue{Baccelli, F., and B.~\mbox{B{\ptb{\l}}aszczyszyn}.} 2010. Stochastic geometry and 
wireless networks: Vol.~I Theory. \textit{Foundations and Trends$^\registered$ in Networking} 
3(3-4):249--449; 4(1-2):1--312. doi: 10.1561/1300000006.
\bibitem{11-gai-1}
\Aue{Abaev, P., K. Samouylov, Yu.~Gaidamaka, and S.~Shorgin}. 2016 (in press). Tractable 
distance distribution approximations for hardcore processes. \textit{DCCN 2016, CCIS 678}. 
Switzerland: Springer International Publishing. 
\bibitem{12-gai-1}
\Aue{Gaidamaka, Yu., and A.~Samuylov}. 2015. Metod rascheta kharakteristik interferentsii 
dvukh vza\-imo\-dey\-st\-vu\-yushchikh ustroystv v~besprovodnoy geterogennoy seti [Method for 
calculating numerical characteristics of two devices interference for device to device 
communications in a wireless heterogeneous network]. \textit{Informatika i~ee Primeneniya~--- 
Inform. Appl.} 9(1):10--15. 
\bibitem{13-gai-1}
\Aue{Samuylov, A., D. Moltchanov, Yu.~Gaidamaka, V.~Begishev, R.~Kovalchukov, 
P.~Abaev, and  S.~Shorgin.} 2016. SIR analysis in square-shaped indoor premises. 
\textit{30th 
European Conference on Modelling and Simulation Proceedings}. Regensburg,
Germany: Digitaldruck Pirrot GmbH. 
692--697.
\bibitem{14-gai-1}
\Aue{Samuylov, A., A.~Ometov, V.~Begishev, R.~Kovalchukov, D.~Moltchanov, 
Yu.~Gaidamaka, K.~Samouylov, S.~Andreev, and Y.~Koucheryavy.} 2015. Analytical 
performance estimation of network-assisted D2D communications in urban scenarios with 
rectangular cells. \textit{Trans. Emerging Telecommunications Technol.}  
1--14. doi: 10.1002/ett.2999A. 
\bibitem{15-gai-1}
\Aue{Levin, B.\,R.} 1989. \textit{Teoreticheskie osnovy statisticheskoy radiotekhniki} 
[Theoretical basis of statistical radiotechnics]. 3rd ed. Moscow: Radio and Communications. 
656~p.
\bibitem{16-gai-1}
ITU-R. 2015. Propagation data and prediction methods for the planning of indoor 
radiocommunication systems and radio local area networks in the frequency range~300~MHz 
to~100~GHz. Recommendation ITU-R P.1238-8 (07-2015).
\end{thebibliography}

 }
 }

\end{multicols}

\vspace*{-3pt}

\hfill{\small\textit{Received October 14, 2016}}

\Contr

\noindent
\textbf{Gaidamaka Yuliya V.} (b.\ 1971)~--- Candidate of Science (PhD) in physics and mathematics, 
associate professor, Peoples' Friendship University of Russia, 6~Miklukho-Maklaya Str., Moscow 117198, 
Russian Federation; senior scientist, Institute of Informatics Problems, Federal Research Center 
``Computer Science and Control'' of the Russian Academy of Sciences, 44-2~Vavilov Str., Moscow 
119333, Russian Federation; \mbox{ygaidamaka@sci.pfu.edu.ru}

\vspace*{3pt}

\noindent
\textbf{Andreev Sergey D.} (b.\ 1984)~--- Candidate of Sciences (PhD) in technology; associate 
professor, Peoples' Friendship University of Russia, 6~Miklukho-Maklaya Str., Moscow 117198, 
Russian Federation; senior scientist, Institute of Informatics Problems, Federal Research Center 
``Computer Science and Control'' of the Russian Academy of Sciences, 44-2~Vavilov Str., Moscow 
119333, Russian Federation; \mbox{serge.аndeev@gmаil.com}

\vspace*{3pt}

\noindent
\textbf{Sopin Eduard S.} (b.\ 1986)~--- Candidate of Science (PhD) in physics and mathematics, 
associate professor,  Peoples' Friendship University of Russia, 6~Miklukho-Maklaya Str., Moscow 
117198, 
Russian Federation; senior scientist, Institute of Informatics Problems, Federal Research Center 
``Computer Science and Control'' of the Russian Academy of Sciences, 44-2~Vavilov Str., Moscow 
119333, Russian Federation; \mbox{esopin@sci.pfu.edu.ru}

\vspace*{3pt}

\noindent
\textbf{Samouylov Konstantin E.} (b.\ 1955)~--- Doctor of Science in technology, professor, Head of 
Department, Peoples' Friendship University of Russia, 6~Miklukho-Maklaya Str., Moscow 117198, 
Russian Federation; leading scientist, Institute of Informatics Problems, Federal Research Center 
``Computer Science and Control'' of the Russian Academy of Sciences, 44-2~Vavilov Str., Moscow 
119333, Russian Federation; \mbox{ksam@sci.pfu.edu.ru}

\vspace*{3pt}

\noindent
\textbf{Shorgin Sergey Ya.} (b.\ 1952)~--- Doctor of Science in physics and mathematics, professor, 
Deputy Director, Federal Research Center ``Computer Science and Control'' of the Russian Academy of 
Sciences, 44-2~Vavilov Str., Moscow 119333, Russian Federation; \mbox{sshorgin@ipiran.ru}
\label{end\stat}


\renewcommand{\bibname}{\protect\rm Литература} 