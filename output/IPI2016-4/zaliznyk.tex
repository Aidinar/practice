\def\stat{zalizniak}

\def\tit{БАЗА ДАННЫХ БЕЗЛИЧНЫХ ГЛАГОЛЬНЫХ КОНСТРУКЦИЙ РУССКОГО ЯЗЫКА$^*$}

\def\titkol{База данных безличных глагольных конструкций русского языка}

\def\aut{Анна А.~Зализняк$^1$, М.\,Г.~Кружков$^2$}

\def\autkol{Анна А.~Зализняк, М.\,Г.~Кружков}

\titel{\tit}{\aut}{\autkol}{\titkol}

\index{Зализняк Анна А.}
\index{Кружков М.\,Г.}
\index{Zalizniak Anna A.}
\index{Kruzhkov M.\,G.}


{\renewcommand{\thefootnote}{\fnsymbol{footnote}} \footnotetext[1]
{Работа выполнена при поддержке РГНФ (проект 15-04-00507).}}


\renewcommand{\thefootnote}{\arabic{footnote}}
\footnotetext[1]{Институт языкознания Российской академии наук; Институт проблем информатики Федерального 
исследовательского центра <<Информатика и~управ\-ле\-ние>> Российской академии наук, 
\mbox{anna.zalizniak@gmail.com}}
\footnotetext[2]{Институт проблем информатики Федерального исследовательского центра 
<<Информатика и~управ\-ле\-ние>> Российской академии наук, \mbox{magnit75@yandex.ru}}

\vspace*{-6pt}


  \Abst{Предлагается описание базы данных (БД) безличных глагольных 
конструкций (БГК), разработанной в~целях информационной поддержки 
лингвистического исследования БГК русского 
языка в~зеркале их перевода на иностранные языки. Показано, каким образом 
концепция построения надкорпусных БД (НБД) адаптирована под 
создание данного ин\-фор\-ма\-ци\-он\-но-линг\-ви\-сти\-че\-ско\-го ресурса. Переводные 
соответствия в~БД БГК представлены в~виде упорядоченных пар, элементами 
которых являются формальные описания соответствующих друг другу 
  лек\-си\-ко-грам\-ма\-ти\-че\-ских форм (ЛГФ) на языке оригинала и~языке 
перевода. Также описана методика построения переводных 
соответствий в~БД и~рассмотрены проблемы, связанные с~процедурой 
поиска безличных конструкций в~электронных лингвистических корпусах, 
а~также некоторые варианты их решения. Использование БД БГК и~других НБД 
значительно расширяет возможности лингвистов, использующих корпусные 
методы как для моноязычного, так и~для контрастивного анализа исследуемых 
языковых единиц, в~том числе благодаря поисковым и~статистическим 
функциям, интегрированным в~эти~БД.}
  
  \KW{компьютерная лингвистика; контрастивная лингвистика; 
информационные технологии; электронные корпуса текстов; надкорпусные 
базы данных; русский язык; безличные конструкции}

\DOI{10.14357/19922264160414}

\vspace*{-6pt} 


\vskip 8pt plus 9pt minus 6pt

\thispagestyle{headings}

\begin{multicols}{2}

\label{st\stat}
  
\section{Введение}

  База данных БГК русского языка и~их 
переводных эквивалентов была создана в~ходе реализации проекта 
<<Контрастивное корпусное исследование глагольных конструкций русского 
языка: семантика, грамматика, идиоматика>> с~целью решения следующих 
задач:
  \begin{itemize}
\item совершенствование метода унидирекционального контрастивного 
анализа, включающего построение моно- и~полиэквиваленций и~их 
последующий лингвистический и~статистический анализ; 
\item разработка метода идентификации безличных конструкций; 
\item построение типологии конструкций русского языка, выражающих 
значение безличности.
\end{itemize}

  Категория безличности является традиционным объектом русской 
грамматической теории; наиболее полно, последовательно и~на корпусном 
материале она описана в~статье~[1]; тем не менее здесь еще остается ряд 
нерешенных вопросов. В~рамках данного проекта рассматриваются только 
глагольные конструкции, со\-став\-ля\-ющие лишь незначительную\linebreak\vspace*{-12pt}

\columnbreak

\noindent
 часть всего 
многообразия русских безличных конструкций, пред\-став\-ля\-ющих собой, как 
известно, выдающуюся особенность русской грамматики~[2, с.~413--430]. 
В~данном исследовании принято расширенное понимание категории 
безличности: к~классу безличных конструкций относятся также конструкции, 
которые в~русской грамматике традиционно обозначаются как 
  <<не\-опре\-де\-лен\-но-лич\-ные>> (с~глаголом в~форме~3~л.\ мн.~ч.)\ 
и~<<обоб\-щен\-но-лич\-ные>> (с~глаголом в~форме~2~л.\ ед.~ч., реже~--- 
2~л.\linebreak мн.~ч.) (cм.\ об этих конструкциях, в~частности,~[3]). Такое 
понимание является концептуальной основой для проектирования как 
поисковой подсистемы, так и~в~целом для создания БД русских 
БГК. Ее формирование представляет собой 
важный этап в~развитии концепции НБД, а~также на пути 
создания грамматики конструкций русского языка, опирающейся на корпусные 
данные. 
  
  В данном исследовании применяется метод \textit{унидирек\-ционального 
корпусного анализа}, предложенный и~разработанный в~ходе выполнения ряда 
проектов, суть которого состоит в~том, что ана\-ли\-зи\-ру\-емые языковые единицы 
рассматриваются <<в~зеркале перевода>>: иноязычный текстовый эквивалент 
анализируемой единицы русского языка рас\-смат\-ри\-ва\-ет\-ся как источник 
сведений о~его семантике, в~том числе служит средством выявления 
имплицитных семантических компонентов. Таким образом, иноязычный текст 
в~обоих направлениях перевода служит не объектом, а лишь инструментом 
анализа~[4--6]. 
  
  База данных БГК русского языка  
основана на концепции построения \textit{надкорпусных баз данных}, 
описанной в~работах~[4--11]\footnote{Термин <<надкорпусная база данных>> 
введен в~работе~\cite{12-kr}.}. В~рамках этой концепции уже было создано три 
лингвистических БД: БД личных глагольных конструкций  
русского языка~[7--9], БД лингвоспецифичных единиц 
русского языка~\cite{6-kr, 10-kr} и~БД коннекторов русского 
языка~\cite{11-kr}. В~следующем разделе будет описана структура НБД и~то, 
как она применяется для поиска и~описания 
БГК русского языка и~их переводных соответствий в~БД~БГК.

\section{Структура надкорпусных баз данных и~ее приложение 
к~базе~данных безличных глагольных~конструкций}

  Надкорпусные базы данных используются для хранения информации 
о следующих объектах, выявляемых в~параллельных 
корпусах\footnote{В~настоящее время во всех НБД используются тексты 
параллельных подкорпусов Национального корпуса русского языка (НКРЯ, {\sf 
http://www.ruscorpora.ru/}).}:
  \begin{itemize}
\item языковые единицы некоторого языка, явля\-ющи\-еся предметом 
рассмотрения в~рамках конкретного лингвистического исследования;
\item соответствующие данным единицам фрагменты текста перевода на 
один или несколько языков; 
\item переводные соответствия, представляющие собой упорядоченные пары 
(=\;кор\-те\-жи), объединяющие два вышеупомянутых объекта (на первой 
позиции~--- единица языка оригинала, на второй позиции~--- единица языка 
перевода). 
\end{itemize}
  
  В БД БГК эти три объекта выглядят следующим образом:
  \begin{enumerate}[(1)]
\item БГК русского языка;
\item соответствующие им конструкции в~других языках (французском, 
немецком);
\item переводные соответствия, т.\,е.\ упорядоченные пары, объединяющие 
вышеупомянутые объ\-екты. 
\end{enumerate}

  Для этих объектов в~БД БГК используется следующая терминология. 
Исследуемые конструкции русского языка, а также соответствующие им 
конструкции в~других языках называются  
\textit{лек\-си\-ко-грам\-ма\-ти\-че\-ски\-ми формами}\footnote{Термин был введен в~\cite{12-kr}.}. 
Лек\-си\-ко-грам\-ма\-ти\-че\-ская форма перевода, 
соответствующая некоторой ЛГФ оригинала, называется ее 
\textit{функционально эквивалентным фрагментом} (сокращенно 
ФЭФ\footnote{Термин введен Д.\,О.~Добровольским~\cite{13-kr, 14-kr}.}). 
Переводное соответствие, представляющее собой упорядоченную пару 
$\langle$ЛГФ, ФЭФ$\rangle$, называется \textit{моноэквиваленцией} 
(сокращенно МЭ)\footnote{Понятие \textit{моноэквиваленции} было введено 
и~определено в~\cite{7-kr}.}. Такие переводные соответствия будем называть 
\textit{прямыми}. Прямые соответствия позволяют описывать так называемые 
\textit{модели} перевода. 

В~БД БГК также могут сохраняться \textit{обратные} 
переводные соответствия, сформированные на основе переводов иноязычных 
текстов на русский язык. В~этом случае глагольная конструкция русского 
языка, выявленная в~переводе, находится на второй позиции пары в~МЭ, а на 
первой~--- тот фрагмент иноязычного текста, который вызвал появление 
данной русской конструкции в~переводе, т.\,е.\ послужил \textit{стимулом 
перевода}\footnote{Подробнее о \textit{моделях} и~\textit{стимулах} перевода 
см.~\cite[с.~102]{8-kr}.}.
  
  Нужно отметить, что в~БД БГК и~других НБД между множеством 
конструкций русского языка и~множеством их ФЭФ в~языках перевода 
существует важное различие. Конструкции русского языка являются объектом 
нашего исследования, поэтому класс таких конструкций изначально является 
закрытым. 

Множество конструкций русского языка формируется на основе 
заранее заданных типов (например, в~БД БГК в~рассмотрение включаются лишь 
БГК). Что же касается их иноязычных 
соответствий, то они образуют открытый класс, поскольку заранее неизвестно, 
какие переводные эквиваленты ис\-сле\-ду\-емых конструкций русского языка 
будут выявлены в~процессе создания БД, и~в~этом и~состоит одна из 
задач настоящего исследования: работа с~параллельными текстами позволяет 
обнаружить соответствия, о~которых до его начала исследователи могли не 
подозревать~[15, с.~221; 16]. Соответственно, множество 
типов ЛГФ для языков перевода создается и~расширяется в~процессе 
построения МЭ по мере выявления новых видов переводных 
соответствий. При этом нередки случаи, когда в~переводе для анализируемой 
русской конструкции вообще не находится никакого соответ-\linebreak\vspace*{-12pt}

\pagebreak

\noindent
ствия или 
соответствие не может быть однозначно установлено\footnote[1]{В таких случаях 
ФЭФ перевода получает, соответственно, пометы ZERO и~UNKNOWN.}.
  
  Несмотря на вышеупомянутое различие, концепция НБД предполагает 
единую структуру описания ЛГФ на русском и~других языках, в~которой схемы 
описания конструкций на разных языках различаются только наборами 
используемых признаков. 
  
  Согласно концепции построения НБД исследователи не используют никакую 
заранее созданную классификацию рассматриваемых языковых единиц 
(в~случае БД БГК~--- безличных глагольных форм), а ограничиваются 
системой аннотирования ЛГФ, предусматривающей присвоение каждой ЛГФ 
набора признаков из некоторого списка. Такой способ формирования БД БГК 
позволяет максимально ускорить процесс обработки параллельных текстов, 
отложив решение более сложных классификационных вопросов до этапа 
анализа исследователями уже размеченного массива данных.
  
  Для разметки ЛГФ каждого языка (русский, французский, немецкий) 
используются свои наборы признаков, которые составляются исследователями 
перед началом работы и~могут пополняться и~видоизменяться в~ходе 
строительства МЭ. Это в~большей степени касается признаков ЛГФ 
иноязычных текстов, поскольку, как было сказано, в~ходе работы прямые МЭ 
выявляют новые модели, а~обратные~--- новые стимулы перевода. При этом 
НБД построена таким образом, что для добавления новых и~модификации 
старых признаков не требуется корректировать структуру БД~--- 
идентификаторы и~описания со\-от\-вет\-ст\-ву\-ющих признаков хранятся в~таблицах 
реляционной БД и~для внесения изменений достаточно добавить 
в~соответствующую таблицу новые строки или изменить содержание уже 
имеющихся строк.
  
  Для каждого языка признаки, использующиеся для аннотации ЛГФ, делятся 
на две группы: \textit{базовые виды} и~\textit{дополнительные признаки}. 
Предполагается, что все множество исследуемых конструкций русского языка 
принадлежит к~некоторому классу (в~случае с~БД БГК это безличные 
глагольные формы), который можно разбить на некоторое число 
непересекающихся подклассов, при этом принадлежность конструкции 
к~определенному подклассу фиксируется с~помощью признака, который 
называется \textit{базовым видом}. Таким образом, каждая ЛГФ всегда должна 
иметь один и~только один базовый вид.
{\looseness=-1

}
  
  Остальные признаки, использующиеся для аннотации ЛГФ, называются 
\textit{дополнительными признаками}. Каждой ЛГФ может быть присвоено 
ноль, один или более дополнительных признаков. Если в~НБД фиксируется 
большое число дополнительных признаков, то они для удобства группируются 
в~клас\-те\-ры. При этом по ходу наполнения БД БГК базовые виды 
и~дополнительные признаки могут добавляться, удаляться, объединяться, 
перемещаться из одного кластера в~другой, а также могут создаваться новые 
кластеры дополнительных признаков (при этом иногда может возникать 
необходимость коррекции разметки для уже построенных ЛГФ).
  
  Надо отметить, что выделение некоторой группы признаков в~качестве 
набора базовых видов для заданного класса конструкций является 
в~значительной степени условным. Этот выбор может определяться не 
концептуальными, а утилитарными соображениями, так как он ставит своей 
целью упрощение и~ускорение процесса построения МЭ. Например, в~БД БГК 
в~качестве набора базовых видов русского языка можно было бы выделить 
некоторый набор типов безличных конструкций в~соответствии  
с~ка\-кой-ли\-бо существующей классификацией (см., например,~[1]). Однако 
недостаток этого подхода состоит в~том, что определение типа безличной 
конструкции часто является нетривиальной задачей. Более того, построение 
типологии безличных конструкций~--- это, наоборот, одна из задач, которая 
решается при помощи БД БГК. Поэтому в~БД БГК в~качестве базовых видов 
берутся лексические единицы~--- чаще всего это глагол в~форме инфинитива, 
а~такие свойства, как время, вид и~лицо глагола, фигурируют в~качестве 
дополнительных признаков; иногда в~качестве базового вида используется 
глагол в~определенной форме (например, в~качестве самостоятельных ЛГФ 
рассматриваются единицы \textit{кажется}, \textit{может быть}); в~качестве 
самостоятельных ЛГФ рассматриваются также фразеологические единицы 
с~глагольной вершиной (например, \textit{ничего не поделаешь}, \textit{откуда 
ни возьмись} и~т.\,п.). В~других НБД базовые виды могут определяться 
грамматическими признаками\footnote[2]{Например, в~БД личных глагольных конструкций 
базовые виды формируются комбинацией признаков 
<<время>>, <<вид>> и~<<наклонение>>~[12--14].}.

\begin{figure*} %fig1
\vspace*{1pt}
\begin{center}
\mbox{%
\epsfxsize=133.224mm
\epsfbox{kru-1.eps}
}
\end{center}
\vspace*{-9pt}
\Caption{Построение <<скелета>> МЭ (пример)}
\vspace*{4pt}
\end{figure*}


\vspace*{-6pt}
  
\section{Методика построения моноэквиаленций в~базе~данных~безличных
глагольных~конструкций}

  Построение МЭ в~БД БГК осуществляют 
пользователи, свободно владеющие обоими языками соответствующей 
языковой пары. Пользователей этой категории будем называть 
\textit{строителями} МЭ. Система доступа к~БД БГК является 
многопользовательской, благодаря чему построением МЭ в~БД БГК могут 
одновременно заниматься несколько строителей. 
  
  Войдя в~систему, строитель МЭ в~соответствии с~поставленной перед ним 
задачей задает нужное направление перевода (например, переводы с~русского 
на французский), если нужно, выбирает для данного направления перевода 
конкретные произведения и~их переводы, для которых ему предстоит строить 
МЭ. Затем строитель задает \textit{первичный запрос} (см.\ разд.~4) на поиск 
безличных конструкций определенного типа в~выбранной части параллельного 
корпуса. В~результате выполнения первичного запроса система выдает 
строителю список выровненных пар фрагментов, в~которых могут находиться 
искомые конструкции (обычно запрос специфицирует признаки одного или 
нескольких слов, входящих в~текст русского фрагмента пары). Из-за 
имеющейся в~русском языке лексической и~грамматической омонимии, а также 
по ряду других причин (см.\ ниже) не все выданные по запросу пары 
в~действительности содержат конструкции искомого вида. Строитель 
последовательно просматривает каждую из найденных пар, и~если она 
действительно содержит искомую конструкцию, то для нее строится МЭ. 
  
  Сначала строитель создает <<скелет>> МЭ. Он выделяет в~русском 
фрагменте данной пары \textit{минимальный контекст} ЛГФ~---  совокупность 
элементов фразы, которая позволяет на базовом уровне интерпретировать 
употребление анализируемой русской ЛГФ в~данной фразе. Затем он находит 
и~выделяет в~переводном фрагменте той же пары минимальный контекст ФЭФ 
для данной русской ЛГФ. Из выделенных слов в~русском и~переводном 
фрагментах создаются заготовки для русской ЛГФ и~ее ФЭФ (переводной 
ЛГФ). Затем заготовка русской ЛГФ объединяется в~пару с~заготовкой ФЭФ 
перевода, в~результате чего создается неразмеченный <<скелет>> МЭ. Пример 
построения <<скелета>> МЭ схематично представлен на рис.~1. 


  Далее строитель размечает полученный <<скелет>> МЭ. В~минимальных 
контекстах русской ЛГФ и~переводного ФЭФ он выделяет так называемые 
<<главные слова>>, т.\,е.\ слова, входящие в~конструкцию: для русской ЛГФ 
это безличная глагольная форма с~наиболее тесно связанными с~ней словами 
(=\;эле\-мен\-та\-ми конструкции), а для ФЭФ перевода~--- функциональный 
эквивалент этой конструкции. Для примера из рис.~1 в~русской части это будут 
слова \textit{работается}, а также \textit{мне}, \textit{лучше}; во  
французской~--- \textit{pour moi}, \textit{je travaille}, \textit{mieux}.
  
  После этого русской ЛГФ и~переводному ФЭФ присваиваются значения 
параметров <<базовый вид>> и~<<дополнительные признаки>> (рис.~2). 
Если в~списке уже имеющихся базовых видов отсутствует нужная единица, она 
добавляется в~список; такая необходимость возникает, в~особенности для ФЭФ 
перевода, поскольку, как упоминалось выше, по мере обработки параллельного 
корпуса могут выявляться новые модели и~стимулы перевода. 
  
  Наконец, строитель может присвоить определенные признаки самой МЭ~--- 
из отдельного списка признаков, которые призваны фиксировать различные 
особенности переводного соответствия в~целом и~не могут быть отнесены по 
отдельности ни к~оригиналу, ни к~переводу. Одним из таких признаков является 
признак смены подлежащего при переводе (SubjCh). Пример размеченной МЭ 
схематично представлен на рис.~2.

\begin{figure*} %fig2
\vspace*{1pt}
\begin{center}
\mbox{%
\epsfxsize=128.783mm
\epsfbox{kru-2.eps}
}
\end{center}
\vspace*{-9pt}
\Caption{Размеченная МЭ (пример). Выделены главные слова, указаны базовый вид 
и~некоторые дополнительные признаки для ЛГФ и~ФЭФ, указан признак МЭ}
\vspace*{12pt}
\end{figure*}

  Построенные и~размеченные МЭ на следующем этапе проверяются 
экспертами. Эксперт может откорректировать состав слов, входящих в~контекст 
ЛГФ и~ФЭФ, и~их <<главные слова>>, изменить проставленные признаки, дать 
оценку качества построенной МЭ, а~также оставить текстовое замечание, что 
помогает поддерживать обратную связь между экспертами и~строителями.
  
\section{Построение первичных запросов}

  Чтобы облегчить поиск безличных конструкций, разработчики НБД создают 
\textit{первичные поисковые запросы}, предоставляющие в~распоряжение 
строителей множество пар предложений, в~которых потенциально могут 
присутствовать БГК. Далее строители вручную 
выбирают те пары, где (в~русской части) действительно имеется искомая 
конструкция, и~осуществляют построение и~аннотацию МЭ.
  
  Первичные запросы базируются на морфологической разметке слов 
в~параллельном подкорпусе НКРЯ ({\sf http://www.ruscorpora.ru/corpora-
morph.html}). В~силу омонимичности многих русских словоформ 
в~морфологической разметке НКРЯ допускается указание у~одной и~той же 
словоформы нескольких вариантов морфологического разбора, что приводит 
к~появлению шума в~результате выполнения первичных запросов.
  
  Как уже говорилось, в~данном исследовании принято расширенное 
понимание категории безличности~--- в~том отношении, что сюда включаются 
также неопределенно-личные и~обоб\-щен\-но-лич\-ные конструкции; 
свойством, объединяющим все типы безличных конструкций, считается 
отсутствие подлежащего, т.\,е.\ именной группы в~номинативе. Однако при 
анализе предложения при помощи формальных процедур достаточно трудно 
определить отсутствие в~нем подлежащего\footnote{Ср.\ понятие <<синтаксического 
нуля>> в~позиции подлежащего в~\cite{17-kr}. Однако, как известно, отсутствие именной группы 
в~номинативе~--- достаточно сложно идентифицируемый факт как с~практической, так и~с теоретической точки 
зрения (см.\ об этом, в~частности,~\cite{18-kr, 19-kr}). Особую проблему составляет трактовка случаев 
отсутствия личного местоимения. Иначе говоря, современное описание грамматики русского языка не 
содержит набора признаков и~критериев для формализованной идентификации отсутствия именной группы 
в~номинативе.}. С~другой стороны, как было сказано, в~рамках данного проекта 
рассмотрение ограничивается лишь \textit{глагольными} конструкциями; при 
этом из рассмотрения на данном этапе исключаются конструкции с~глаголами 
\textit{быть}, \textit{бывать}, \textit{стать}, \textit{становиться}, 
с~предикативом (\textit{сложно}, \textit{вероятно}, \textit{стыдно}, 
\textit{холодно}, \textit{надо}, \textit{пора} и~т.\,п.)\ и~с~инфинитивом 
(\textit{тебе ходить}, \textit{нечего сказать} и~т.\,п.).
  
  В~частности, в~БД БГК первичные запросы составлялись для поиска 
безличных форм, вклю\-ча\-ющих глагол в~форме:
  \begin{itemize}
\item наст./буд.\ вр.\ 2-го л.\ ед.~ч.\ ({\bfseries\textit{сидишь}} тут 
\textit{целый день как дурак}; \textit{тут за день так} 
{\bfseries\textit{накувыркаешься}});
\item наст./буд.\ вр.\ 3-го~л.\ ед.~ч. ({\bfseries\textit{приходится}} 
\textit{согласиться}; \textit{как вам не} {\bfseries\textit{надоест}}); 
\item наст.\ вр.\ 3-го~л. мн.~ч.\ (\textit{с~друзьями так не} 
{\bfseries\textit{поступают}});
\item прош.\ вр.\ ср.\ р.\ ед.~ч.\ ({\bfseries\textit{оказалось}}, \textit{что 
он не виноват});
\item прош.\ вр.\ ср.\ р.\ мн.~ч.\ (\textit{к~вам} 
{\bfseries\textit{пришли}}).
  \end{itemize}
  
  При этом в~искомых конструкциях не должно быть подлежащего 
в~именительном падеже, выраженного именной группой или местоимением, 
которое может согласоваться с~данным глаголом.
  
  Помимо уже упомянутой проблемы с~шумом (когда в~выдачу попадают 
слова, которые системой ошибочно принимаются за глаголы, например 
\textit{опускал ноги с}~{\bfseries\textit{постели}} \textit{на пол}), при поиске 
безличных конструкций в~БД БГК нередко могут возникать потери: безличные 
конструкции, присутствующие в~корпусе, не попадают в~выдачу. Это 
происходит, когда из результатов первичных запросов исключаются 
фрагменты, где фигурирует элемент, который может ошибочно 
интерпретироваться как подлежащее, согласующееся с~искомым глаголом.
  
  Вот несколько причин, которые могут обуслов\-ли\-вать такую ошибочную 
интерпретацию слов в~качестве подлежащего:
  \begin{itemize}
\item некоторые слова, омонимичные существительным в~именительном 
падеже, могут фигурировать в~тексте в~иной функции, например: \textit{раз}, 
\textit{уж}, \textit{том}, \textit{знать}, \textit{чай}, \textit{жила}, 
\textit{стать}, \textit{мол}, \textit{берет}, \textit{надел}, \textit{постой} 
и~т.\,д.:

\textit{Помилуй, чего тебе еще? от тебя и~так} {\bfseries\textit{уж несет}} 
\textit{розовой помадой}\ldots

\textit{В мыслях недостаточно последовательности, и, когда я излагаю их 
на бумаге, мне всякий} {\bfseries\textit{раз кажется}}, \textit{что я утерял 
чутье к~их органической связи}.

\textit{Главным образом, я потому не поехал за границу, что вестей туда 
из России доходит мало, а}~{\bfseries\textit{знать хочется}};
\item многие местоимения (\textit{то}, \textit{что}, 
\textit{что-то}, \textit{все}, \textit{всё} и~т. д.), могут 
выступать как в~роли подлежащего, так и~в других функциях (союз, 
дискурсивное слово\footnote[1]{В~некоторых случаях даже строителям МЭ 
бывает трудно без расширенного контекста определить, какую роль играет 
такое слово в~конкретном примере, например: \textit{Но теперь его вдруг} 
{\bfseries\textit{что-то}} \textit{потянуло к~людям}.}):

\textit{На это он заметил, что я еще слишком молода}, 
{\bfseries\textit{что}} \textit{у~меня еще в~голове} 
{\bfseries\textit{бродит.}} 

\textit{Его} {\bfseries\textit{всё тянет}} \textit{в~ту сторону, где только и~
знают, что гуляют}.

\textit{А} {\bfseries\textit{то выходит}} \textit{по твоему рассказу, что он 
действительно родился!..}

\textit{Ей} {\bfseries\textit{всё хочется}}, \textit{чтобы все считали, что 
она покровительствует};
\item нйденное в~непосредственном окружении глагола слово 
в~именительном падеже может на самом деле относиться к~другому глаголу:

\textit{Ну, будет другой} {\bfseries\textit{редактор}} \textit{и~даже}, 
{\bfseries\textit{может}} \textit{быть, еще красноречивее прежнего.}

\textit{Если как следует провентилировать этот} 
{\bfseries\textit{вопрос}}, {\bfseries\textit{выходит}}, \textit{что я, 
в~сущности, даже и~не знал-то как следует покойника}.

{\bfseries\textit{Фельдмаршал}} \textit{мой}, {\bfseries\textit{кажется}}, 
\textit{говорит дело}.

\textit{Здесь все} {\bfseries\textit{дело}}, {\bfseries\textit{кажется}}, 
\textit{совершенно очевидно}\ldots

\end{itemize}

  В совокупности эти и~некоторые другие причины (например, возможность 
различного порядка следования подлежащего и~сказуемого в~русском 
предложении) могут приводить к~заметным потерям, поэтому при построении 
первичных запросов разработчикам приходится искать способы их уменьшить. 
В~частности: 
  \begin{itemize}
\item составляются списки <<квазисуществительных>>, которые запрос не 
должен по умолчанию воспринимать как существительные (\textit{раз}, 
\textit{уж}, \textit{том}, \textit{знать} и~т.\,д.); 
\item составляются списки местоимений, которые, в~отличие от остальных 
местоимений, почти всегда должны интерпретироваться как потенциальные 
подлежащие (\textit{который}, \textit{которое}, \textit{он}, \textit{она}, 
\textit{это} и~т.\,д.);
\item область поиска потенциального подлежащего ограничивается 
ближайшими к~глаголу знаками препинания и~т.\,д.
\end{itemize}

  Поскольку, как было отмечено выше, не существует набора признаков 
и~критериев для формализованной идентификации отсутствия именной группы 
в~номинативе, при построении первичных запросов на поиск безличных форм 
разработчикам приходится искать приемлемый компромисс между уровнем 
шума и~потерь, так как, если речь не идет о сплошном аннотировании текстов, 
уменьшение потерь почти всегда ведет к~увеличению шума.

\section{Поиск и~статистика}

  После завершения аннотирования ЛГФ и~МЭ в~БД БГК становится 
достаточно легко находить в~массиве данных те МЭ, которые удовлетворяют 
заданным критериям. Поисковая система БД БГК позволяет указывать при 
поиске МЭ следующие признаки:
  \begin{itemize}
\item базовый вид ЛГФ оригинала;
\item базовый вид ФЭФ перевода;
\item дополнительные признаки ЛГФ;
\end{itemize}

\pagebreak

\end{multicols}

\begin{table*}\small
  \begin{center}
  \Caption{Фрагмент результатов выполнения поискового запроса. Показано 
4~из~22~найденных МЭ}
  \vspace*{2ex}
  
  \begin{tabular}{|p{35mm}|p{28mm}|p{35mm}|p{28mm}|}
  \hline
  \multicolumn{1}{|c|}{\tabcolsep=0pt\begin{tabular}{c}Контекст\\ ЛГФ 
оригинала\end{tabular}} & 
  \multicolumn{1}{c|}{\tabcolsep=0pt\begin{tabular}{c}Вид\\ и~дополнительные\\\ признаки 
ЛГФ\\ оригинала\end{tabular}} &
  \multicolumn{1}{c|}{\tabcolsep=0pt\begin{tabular}{c}Контекст ЛГФ\\ перевода 
(ФЭФ)\end{tabular}} & 
  \multicolumn{1}{c|}{\tabcolsep=0pt\begin{tabular}{c}Вид\\ и~дополнительные\\ признаки 
ЛГФ\\ перевода (ФЭФ)\end{tabular}}\\
  \hline
  Сестра теперь, %\newline 
  впрочем, \textbf{кажется}, %\newline 
  обеспечена\ldots & 
  \textbf{кажется}\newline
  $\langle$Impers$\rangle$\newline
  $\langle$V-IPF$\rangle$\newline
  $\langle$Pres$\rangle$\newline
  $\langle$3sg$\rangle$\newline
  $\langle$Refl$\rangle$\newline
  $\langle$Parenth$\rangle$   &
  \textbf{Ma soeur semble} d'ailleurs d$\acute{\mbox{e}}$somais 
{\!\!\ptb{\`{a}}}~l'abri du besoin\ldots  &
  \textbf{sembler}\newline 
  $\langle$Pr$\rangle$\newline
  $\langle$3sg$\rangle$\\
  \hline
  \textbf{кажется}, доктор теперь %\newline
  уже лишнее &
  \textbf{кажется}\newline
  $\langle$Impers$\rangle$\newline
  $\langle$V-IPF$\rangle$\newline
  $\langle$Pres$\rangle$\newline
  $\langle$3sg$\rangle$\newline
  $\langle$Refl$\rangle$\newline
  $\langle$Parenth$\rangle$   &
  \textbf{le m$\acute{\mbox{e}}$dicin semblait} 
\mbox{d$\acute{\mbox{e}}$j{\!\ptb{\`{a}}}} \textbf{inutile} &
  \textbf{sembler}\newline
  $\langle$Imparf$\rangle$\newline
  $\langle$3sg$\rangle$\newline
  $\langle$[V.]\;+\;\{Adj.\}$\rangle$\\
  \hline
  \textbf{Кажется}, и~печалями и~радостями он управ\-лял, как движением рук &
  \textbf{кажется}\newline
  $\langle$Impers$\rangle$\newline
  $\langle$V-IPF$\rangle$\newline
  $\langle$Pres$\rangle$\newline
  $\langle$3sg$\rangle$\newline
  $\langle$Refl$\rangle$\newline
  $\langle$Parenth$\rangle$   &
  \textbf{Il semble commander} {\ptb{\`{a}}}~ses joies et {\!\!\ptb{\`{a}}}~ses 
tristesses comme il dirige les mouvements de ses bras & 
  \textbf{sembler}\newline
  $\langle$Pr$\rangle$\newline
  $\langle$3sg$\rangle$\newline
  $\langle$[V.]\;+\;\{Inf.\}$\rangle$\\
  \hline
  все, \textbf{казалось}, лежит в~торжественном покое.
  &
  \textbf{кажется}\newline
  $\langle$Impers$\rangle$\newline
  $\langle$V-IPF$\rangle$\newline
  $\langle$Past$\rangle$\newline
  $\langle$Sg$\rangle$\newline
  $\langle$Refl$\rangle$\newline
  $\langle$Parenth$\rangle$   &
  \textbf{Elle semblait} silencieuse, solennelle &
  \textbf{sembler}\newline
  $\langle$Imparf$\rangle$\newline
  $\langle$3sg$\rangle$\newline
  $\langle$[V.]\;+\;\{Adj.\}$\rangle$\\
  \hline
  \end{tabular}
  \end{center}
%  \vspace*{-2pt}
\vspace*{8pt}
  \end{table*}
  
  \begin{multicols}{2}

\begin{itemize}
\item дополнительные признаки ФЭФ;
\item признаки МЭ;
\item названия текстов, авторы текстов и~переводов и~т.\,д.
\end{itemize}

  Все указанные признаки могут задаваться (или исключаться из результатов 
поиска) одновременно, при этом возвращаемый набор МЭ будет удовлетворять 
всем указанным требованиям. Например, в~БД БГК пользователь может задать 
для поиска следующий набор признаков: (1)~базовый вид ЛГФ оригинала~--- 
<<\textit{кажется}>>; (2)~дополнительный признак ЛГФ оригинала~--- 
$\langle$Parenth$\rangle$ (вводное слово); (3)~базовый вид ФЭФ перевода~--- 
<<\textit{sembler}>>; (4)~признак МЭ~--- $\langle${SubjCh}$\rangle$ (смена 
подлежащего). Такой запрос возвращает~22~результата (из общего массива 
в~2100~МЭ), 4~из которых показаны в~табл.~1.
  
  
  
    
  Данные БД БГК также можно использовать для получения статистики по 
моделям и~стимулам перевода для различных анализируемых глагольных 
конструкций на основе всего массива данных, содержащихся в~БД БГК (или же 
на основе определенного подмножества этого массива). Разумеется, делать 
выводы на основе такой статистики следует лишь после лингвистической 
экспертизы представительного массива данных, поскольку результаты могут 
зависеть от состава анализируемого корпуса, выбранной исследователями 
схемы аннотации и~т.\,д. Поэтому для пользователей БД предусмотрена 
возможность верифицировать полученную статистику: перейдя от 
количественных результатов непосредственно к~МЭ, на основе которых они 
были сгенерированы, пользователь может оценить на качественном уровне 
зависимость значения указанных параметров от контекста, а также от жанра 
текста и~стиля конкретного автора.
  
  В качестве примера приводится таблица частотностей переводных 
соответствий (базовых видов французских ФЭФ) для русской ЛГФ 
<<\textit{хотеться}>>
 (табл.~2). В~БД БГК численные результаты (столбец\linebreak\vspace*{-12pt}

\noindent
{{\tablename~2}\ \ \small{Количественная статистика по вариантам перевода безличных конструкций с~ЛГФ 
\textit{хотеться}}}

\vspace*{3pt}

{\small
 \begin{center}  %
 \tabcolsep=8.5pt
 \begin{tabular}{|l|c|r|}
\cline{1-1}
\textbf{хотеться} \ $\underline{73}$&\multicolumn{2}{c}{\ }\\
\hline
\multicolumn{1}{|c|}{ФЭФ французского языка} & 
\multicolumn{1}{c|}{\tabcolsep=0pt\begin{tabular}{c}Количество\\ МЭ\end{tabular}} &\multicolumn{1}{c|}{\%}\\
\hline
avoir envie & $\underline{32}$ & ${43{,}84}$\\
vouloir & $\underline{24}$ & ${32{,}88}$\\
selon ses desirs & $\underline{2}$ & ${2{,}74}$\\
UNKNOWN & $\underline{2}$ & ${2{,}74}$\\
pr$\acute{\mbox{e}}$f$\acute{\mbox{e}}$rer & $\underline{2}$ & ${2{,}74}$\\
br$\hat{\mbox{u}}$ler d'envie & $\underline{2}$ & ${2{,}74}$\\
avoir faim & $\underline{2}$ & ${2{,}74}$\\
aimer & $\underline{1}$ & ${1{,}37}$\\
tenir & $\underline{1}$ & ${1{,}37}$\\
chercher & $\underline{1}$ & ${1{,}37}$\\
avoir soif & $\underline{1}$ & ${1{,}37}$\\
avoir sommeil & $\underline{1}$ & ${1{,}37}$\\
d$\acute{\mbox{e}}$sirer & $\underline{1}$ & ${1{,}37}$\\
envie & $\underline{1}$& ${1{,}37}$\\
\hline
\end{tabular}
\end{center}
}

\vspace*{12pt}




\noindent
<<Количество МЭ>>) оформлены в~виде гиперссылок,
 по которым пользователи 
могут перейти на страницу, где находятся все МЭ, на основе которых были 
получены приведенные данные.
  



\section{Заключение}

  В процессе создания БД БГК была разработана поисковая подсистема, 
которая обеспечивает выполнение широкого спектра запросов. Проведенные 
эксперименты продемонстрировали ее высокую эффективность с~точки зрения 
решаемых лингвистических задач. Важно отметить, особую ценность для 
развития лингвистической теории полученного <<шума>> и~обнаруженных 
<<потерь>>. Их экспертный анализ позволит сформулировать более четкие 
критерии для их поиска в~корпусах и~БД и~уточнить понятие 
безличной конструкции, что будет способствовать развитию грамматики 
конструкций русского языка. 
  
  База данных БГК стала уже четвертой лингвистической БД, созданной 
в~рамках концепции\linebreak построения НБД, что свидетельствует 
о~жизнеспособности этой концепции в~сфере разработки инструментов 
лингвистического сопоставительного анализа и~формирования 
информационных ресурсов, не имеющих отечественных и~зарубежных 
аналогов. Разработка БД БГК подтвердила, что концепция НБД и~метод 
унидирекционального контрастивного анализа применимы к~исследованиям 
разноплановых языковых единиц и~явлений (личные глагольные формы, 
лингвоспецифичные единицы, коннекторы, безличные глагольные формы,  
ло\-ги\-ко-се\-ман\-ти\-че\-ские отношения в~тексте).
  
{\small\frenchspacing
 {%\baselineskip=10.8pt
 \addcontentsline{toc}{section}{References}
 \begin{thebibliography}{99}
\bibitem{1-kr}
\Au{Летучий А.\,Б.} Безличность. Материалы для проекта корпусного описания русской 
грамматики.~--- М., 2011. {\sf http://rusgram.ru/Безличность}.
\bibitem{2-kr}
\Au{Wierzbicka A.} Semantics, culture, and cognition. 
Universal human concepts in culture-specific configurations.~--- New York\,--\,Oxford: Oxford 
University Press, 1992. 496~p.
\bibitem{3-kr}
\Au{Булыгина Т.\,В., Шмелев А.\,Д.} Я,~ты и~другие в~русском синтаксисе~// Языковая 
концептуализация мира (на материале русской грамматики).~--- М.: Школа <<Языки 
русской культуры>>, 1997. С.~335--352.
\bibitem{4-kr}
\Au{Бунтман Н.\,В., Зализняк Анна~А., Зацман~И.\,М., Кружков~М.\,Г., 
Лощилова~Е.\,Ю., Сичинава~Д.\,В.} Инфор\-мационные технологии корпусных 
исследований: принципы построения кросслингвистических баз данных~// Информатика 
и~её применения, 2014. Т.~8. Вып.~2. С.~98--110.
\bibitem{5-kr}
\Au{Зализняк Анна А., Зацман~И.\,М., Инькова~О.\,Ю., Кружков~М.\,Г.} Надкорпусные 
базы данных как лингвистический ресурс~// Корпусная лингвистика-2015: Тр. 7-й 
Междунар. конф.~--- СПб.: СПбГУ, 2015. С.~211--218.
\bibitem{6-kr}
\Au{Зализняк Анна А.} База данных межъязыковых эквиваленций как инструмент 
лингвистического анализа~// Компьютерная лингвистика и~интеллектуальные 
технологии: по мат-лам ежегодной Междунар. конф. <<Диалог>>.~--- М.: РГГУ, 2016. 
Вып.~15(22). С.~763--775.
\bibitem{7-kr}
\Au{Loiseau S., Sitchinava D.\,V., Zalizniak~A.\,A., Zatsman~I.\,M.} Information technologies 
for creating the database of equivalent verbal forms in the Russian--French multivariant parallel 
corpus~// Информатика и~её применения, 2013. T.~7. Вып.~2. С.~100--109.
\bibitem{8-kr}
\Au{Zalizniak Anna A., Sitchinava~D.\,V., Loiseau~S., Kruzhkov~M., Zatsman~I.\,M.} Database 
of equivalent verbal forms in a~Russian--French multivariant parallel corpus~// 2013 Conference 
(International) on Artificial Intelligence.~--- Las Vegas, NV, USA: CSREA Press, 2013. 
Vol.~1. P.~101--107.
\bibitem{9-kr}
\Au{Kruzhkov M.\,G., Buntman~N.\,V., Loshchilova~E.\,Ju., Sitchinava~D.\,V., Zalizniak 
Anna~A., Zatsman~I.\,M.} A~database of Russian verbal forms and their French translation 
equivalents~// Компьютерная лингвистика и~интеллектуальные технологии: по мат-лам 
ежегодной Междунар. конф. <<Диалог>>.~--- М.: РГГУ, 2014. Вып.~13(20). C.~284--296.
\bibitem{10-kr}
\Au{Зализняк Анна А.} Лингвоспецифичные единицы русского языка в~свете 
контрастивного корпусного\linebreak анализа~// Компьютерная лингвистика и~интеллектуальные 
технологии: по мат-лам ежегодной Междунар. конф. <<Диалог>>.~--- М.: РГГУ, 2015. 
Вып.~14(21). С.~651--662.
\bibitem{11-kr}
\Au{Зацман И.\,М., Инькова~О.\,Ю., Кружков~М.\,Г., Попкова~Н.\,А.} Представление 
кроссязыковых знаний о~коннекторах в~надкорпусных базах данных~// Информатика и~
её применения, 2016. Т.~10. Вып.~1. С.~106--118.
\bibitem{12-kr}
\Au{Кружков М.\,Г.} Информационные ресурсы контрастивных лингвистических 
исследований: электронные корпуса текстов~// Системы и~средства информатики, 2015. 
Т.~25. №\,2. С.~140--159.
\bibitem{13-kr}
\Au{Добровольский Д.\,О., Кретов~А.\,А., Шаров~С.\,А.} Корпус параллельных текстов: 
архитектура и~возможности использования~// Национальный корпус русского языка: 
2003--2005.~--- М.: Индрик, 2005. С.~263--296.
\bibitem{14-kr}
\Au{Добровольский Д.\,О., Кретов~А.\,А., Шаров~С.\,А.} Корпус параллельных текстов~// 
Научная и~техническая информация. Сер.~2: Информационные процессы и~сис\-те\-мы, 
2005. №\,6. С.~16--27.
\bibitem{15-kr}
\Au{Stubbs M.} Words and phrases. Corpus studies of lexical semantics.~--- Oxford: Blackwell, 
2002. 287~p.
\bibitem{16-kr}
\Au{Zatsman I., Buntman~N., Kruzhkov~M., Nuriev~V., Zalizniak Anna~A.} Conceptual 
framework for development of computer technology supporting cross-linguistic knowledge 
discovery~// 15th European Conference on Knowledge Management Proceedings.~--- Reading, MA, USA: 
Academic Publishing International Ltd., 2014. P.~1063--1071.
\bibitem{17-kr}
\Au{Мельчук И.\,А.} О~синтаксическом нуле~// Типология пассивных конструкций. 
Диатезы и~залоги.~--- Л.: Наука, 1974. C.~343--361.
\bibitem{18-kr}
\Au{Guiraud-Weber M.} L'effacement du sujet au nominatif dans 
l'$\acute{\mbox{e}}$nonc$\acute{\mbox{e}}$ en russe moderne~// Revue des 
$\acute{\mbox{E}}$tudes slaves, 1983. Vol.~55. No.\,1. P.~79--86.
\bibitem{19-kr}
\Au{Гиро-Вебер М.} Субъектные черты и~проблема подлежащего в~русском языке~// 
Revue des $\acute{\mbox{E}}$tudes slaves, 2002. Vol.~74. No.\,2-3. P.~279--289.
 \end{thebibliography}

 }
 }

\end{multicols}

\vspace*{-6pt}

\hfill{\small\textit{Поступила в~редакцию 13.10.16}}

\vspace*{8pt}

%\newpage

%\vspace*{-24pt}

\hrule

\vspace*{2pt}

\hrule

%\vspace*{8pt}


\def\tit{DATABASE OF~RUSSIAN IMPERSONAL VERBAL CONSTRUCTIONS}

\def\titkol{Database of~Russian impersonal verbal constructions}

\def\aut{Anna A.~Zalizniak$^{1,2}$  and~M.\,G.~Kruzhkov$^2$}

\def\autkol{Anna A.~Zalizniak  and~M.\,G.~Kruzhkov}

\titel{\tit}{\aut}{\autkol}{\titkol}

\vspace*{-9pt}




\noindent
$^1$Institute of Linguistics, Russian Academy of Sciences, 
1-1~Bolshoy Kislovskiy pereulok, 
Moscow 125009, Russian\linebreak
$\hphantom{^1}$Federation

\noindent
$^2$Institute of Informatics Problems, Federal Research Center ``Computer Science and Control'' of 
the Russian\linebreak
$\hphantom{^1}$Academy of Sciences, 44-2~Vavilov Str., Moscow 119333, Russian Federation



\def\leftfootline{\small{\textbf{\thepage}
\hfill INFORMATIKA I EE PRIMENENIYA~--- INFORMATICS AND
APPLICATIONS\ \ \ 2016\ \ \ volume~10\ \ \ issue\ 4}
}%
 \def\rightfootline{\small{INFORMATIKA I EE PRIMENENIYA~---
INFORMATICS AND APPLICATIONS\ \ \ 2016\ \ \ volume~10\ \ \ issue\ 4
\hfill \textbf{\thepage}}}

\vspace*{3pt}





\Abste{This article presents the Database of Russian Impersonal Verbal Constructions that has been 
developed to support a~linguistic research of Russian impersonal verbal construction as mirrored in 
translations into other languages. This information resource was developed based on the concept of 
Supracorpora Databases (SCDBs). Translation correspondences in the Database of Russian 
Impersonal Verbal Constructions are presented as ordered pairs that combine formal descriptions of 
corresponding lexical-grammatical forms found in source and target texts of a~parallel 
corpus. The paper also provides description of the methodology for creation of translation 
correspondences in the database. Some of the problems related to the task of finding Russian 
impersonal verbal constructions in corpora are considered and approaches to solving those 
problems are proposed. Thanks to integrated search and statistical functions, the Database of 
Russian Impersonal Verbal Constructions and other SCDBs significantly extend capabilities of 
linguistic experts using corpus-based methods to analyze specific linguistic items, both 
independently and in contrast with other languages.} 

\KWE{computer linguistics; contrastive linguistics; information technologies; electronic corpora; 
supracorpora databases; Russian language; impersonal constructions}


\DOI{10.14357/19922264160414} 

\vspace*{-12pt}

\Ack
\noindent
This research was performed in the Institute of Informatics Problems, Federal Research Center 
``Computer Science and Control'' of the Russian Academy of Sciences, with financial support of the 
Russian Foundation for Humanities (grant No.\,15-04-00507).


\vspace*{6pt}

  \begin{multicols}{2}

\renewcommand{\bibname}{\protect\rmfamily References}
%\renewcommand{\bibname}{\large\protect\rm References}

{\small\frenchspacing
 {%\baselineskip=10.8pt
 \addcontentsline{toc}{section}{References}
 \begin{thebibliography}{99}

\bibitem{1-kr-1}
\Aue{Letuchiy, A.\,B.} 2011. \textit{Bezlichnost'. Materialy dlya proekta korpusnogo 
opisaniya russkoy grammatiki} [Impersonality. Materials for the Russian corpus-based grammar 
description project]. Moscow. Available at: {\sf http://\linebreak rusgram.ru/Безличность} (accessed 
October~12, 2016).
\bibitem{2-kr-1}
\Aue{Wierzbicka, A.} 1992. \textit{Semantics, culture, and cognition. Universal human 
concepts in culture-specific configurations}. New York\,--\,Oxford: Oxford University Press. 
496~p.
\bibitem{3-kr-1}
\Aue{Bulygina, T.\,V., and A.\,D.~Shmelev}. 1997. Ya, ty i~drugie v~russkom sintaksise 
[Me, 
you, and others in Russian syntax]. \textit{Yazykovaya kontseptualizatsiya mira (na materiale 
russkoy grammatiki)} [Language-based conceptualization of the world (based on Russian 
grammar)]. Moscow: Shkola ``Yazyki russkoy kul'tury.'' 335--352.
\bibitem{4-kr-1}
\Aue{Buntman, N.\,V., Anna A.~Zaliznyak,  I.\,M.~Zatsman, M.\,G.~Kruzhkov, 
E.\,Yu.~Loshchilova,  and D.\,V.~Sichinava}. 2014. 
Informatsionnye tekhnologii korpusnykh issledovaniy: Printsipy postroeniya  
kross-lingvisticheskikh baz dannykh [Information technologies for corpus studies: 
Underpinnings for cross-linguistic database creation]. \textit{Informatika i~ee Primeneniya~--- 
Inform. Appl.} 8(2):98--110.
\bibitem{5-kr-1}
\Aue{Zaliznyak, Anna A., I.\,M.~Zatsman, O.\,Yu.~In'kova, and M.\,G.~Kruzhkov}. 2015. 
Nadkorpusnye bazy dannykh kak lingvisticheskiy resurs [Supracorpora databases as linguistic 
resource]. \textit{7th Conference 
(International) on Corpus Linguistics Proceedings}. St.\ Petersburg: SPbGU. 211--218.
\bibitem{6-kr-1}
\Aue{Zaliznyak, Anna A.} 2016. Baza dannykh mezh"\-yazy\-ko\-vykh ekvivalentsiy kak 
instrument ling\-vi\-sti\-che\-sko\-go analiza [Database of cross-linguistic equivalences as a tool for 
linguistic analysis]. \textit{Computer Linguistics and Intellectual 
Technologies: Conference (International) ``Dialog'' Proceedings}. Moscow: RGGU.  
15(22):763--775.
\bibitem{7-kr-1}
\Aue{Loiseau, S., D.\,V.~Sitchinava, Anna A.~Zalizniak,  and I.\,M.~Zatsman}. 2013. 
Information technologies for creating the database of equivalent verbal forms in the  
Russian--French multivariant parallel corpus. \textit{Informatika i~ee Primeneniya~--- Inform. 
Appl.} 7(2):100--109.
\bibitem{8-kr-1}
\Aue{Zalizniak, Anna A., D.\,V.~Sitchinava, S.~Loiseau, M.~Kruzhkov, and I.\,M.~Zatsman}. 
2013. Database of equivalent verbal forms in a~Russian--French multivariant parallel corpus. 
\textit{2013  Conference (International) on Artificial Intelligence}. Las Vegas,
NV: CRSEA 
Press. 1:101--107.
\bibitem{9-kr-1}
\Aue{Kruzhkov, M.\,G., N.\,V.~Buntman, E.\,Ju.~Loshchilova, D.\,V.~Sitchinava, Anna 
A.~Zalizniak,  and I.\,M.~Zatsman}. 2014. A~database of Russian verbal forms and their 
French translation equivalents. \textit{Computer Linguistics and 
Intellectual Technologies: Conference (International) ``Dialog'' Proceedings}. Moscow: RGGU. 
13(20):284--296.
\bibitem{10-kr-1}
\Aue{Zalizniak, Anna A.} 2015. Lingvospetsifichnye edi\-ni\-tsy russkogo yazyka v~svete 
kontrastivnogo korpusnogo analiza
[Lingvospecific units of Russian in the light of the contrast corpus analysis].
\textit{Computer Linguistics and 
Intellectual Technologies: Conference (International) ``Dialog'' Proceedings}. 
Moscow: RGGU. 
14(21):651--662.
\bibitem{11-kr-1}
\Aue{Zatsman, I.\,M., O.\,Yu.~In'kova, M.\,G.~Kruzhkov, and N.\,A.~Popkova}. 2016. 
Predstavlenie krossyazykovykh znaniy o konnektorakh v nadkorpusnykh bazakh dannykh 
[Representation of cross-lingual knowledge about connectors in supracorpora databases]. 
\textit{Informatika i~ee Primeneniya~--- Inform. Appl.} 10(1):106--118.
\bibitem{12-kr-1}
\Aue{Kruzhkov, M.\,G.} 2015. Informatsionnye resursy kontrastivnykh lingvisticheskikh 
issledovaniy: Elektronnye korpusa tekstov [Information resources for contrastive studies: 
Electronic text corpora]. \textit{Sistemy i~Sredstva Informatiki~--- Systems and Means of 
Informatics} 25(2):140--159.
\bibitem{13-kr-1}
\Aue{Dobrovol'skiy, D.\,O., A.\,A.~Kretov, and S.\,A.~Sharov}. 2005. Korpus parallel'nykh 
tekstov: Arkhitektura i~voz\-mozh\-no\-sti ispol'zovaniya [Corpus of parallel texts: Architecture and 
applications]. \textit{Natsional'nyy korpus russkogo yazyka: 2003--2005} [Russian National 
Corpus: 2003--2005]. Moscow: Indrik. 263--296.
\bibitem{14-kr-1}
\Aue{Dobrovol'skiy, D.\,O., A.\,A.~Kretov, and S.\,A.~Sharov}. 2005. Korpus parallel'nykh 
tekstov [Corpus of parallel texts]. \textit{Nauchnaya i~tekhnicheskaya informatsiya. Ser.~2 
``Informatsionnye protsessy i~sistemy''} [Scientific and technical information. Ser.~2 
``Informational processes and systems''] 6:16--27.
\bibitem{15-kr-1}
\Au{Stubbs, M.} 2002. \textit{Words and phrases: Corpus studies of lexical semantics.}~--- 
Oxford: Blackwell. 287~p.
\bibitem{16-kr-1}
\Aue{Zatsman, I., N.~Buntman, M.~Kruzhkov, V.~Nuriev, and Anna A.~Zalizniak}. 2014. 
Conceptual framework for development of computer technology supporting cross-linguistic 
knowledge discovery. \textit{15th European Conference on Knowledge Management 
Proceedings}. Reading, MA: Academic Publishing International. 1063--1071.
\bibitem{17-kr-1}
\Aue{Mel'chuk, I.\,A.} 1974. O~sintaksicheskom nule [On syntactic zero]. \textit{Tipologiya 
passivnykh konstruktsiy. Diatezy i~zalogi} [Typology of passive constructions. Diatheses and 
voices]. Leningrad: Nauka. 343--361.
\bibitem{18-kr-1}
\Aue{Guiraud-Weber, M.} 1983. L'effacement du sujet au nominatif dans 
l'$\acute{\mbox{e}}$nonc$\acute{\mbox{e}}$ en russe moderne [Diffusion of the subject in 
nominative case in modern Russian utterances]. \textit{Revue des $\acute{\mbox{E}}$tudes 
slaves} [J.~Slavic Studies] 55(1):79--86.
\bibitem{19-kr-1}
\Aue{Guiraud-Weber, M.} 2002. Sub"ektnye cherty i~problema podlezhashchego v~russkom 
yazyke [Subjective features and problem of the subject in Russian language]. \textit{Revue des 
$\acute{\mbox{E}}$tudes slaves} [J.~Slavic Studies] 74(2-3):279--289.
\end{thebibliography}

 }
 }

\end{multicols}

\vspace*{-3pt}

\hfill{\small\textit{Received October 13, 2016}}

\Contr

\noindent
\textbf{Zalizniak Anna A.} (b.\ 1959)~---  Doctor of Science of philology, leading scientist, 
Institute of Linguistics, Russian Academy of Sciences, 1-1~Bolshoy Kislovskiy pereulok, Moscow, 
125009, Russian Federation; leading scientist, Institute of Informatics Problems, Federal Research 
Center ``Computer Science and Control'' of the Russian Academy of Sciences, 44-2~Vavilov Str., 
Moscow 119333, Russian Federation; \mbox{anna.zalizniak@gmail.com}

\vspace*{3pt}

\noindent
\textbf{Kruzhkov Mikhail G.} (b.\ 1975)~--- leading programmer, Institute of Informatics 
Problems, Federal Research Center ``Computer Science and Control'' of the Russian Academy of 
Sciences, 44-2~Vavilov Str., Moscow 119333, Russian Federation; \mbox{magnit75@yandex.ru}
\label{end\stat}


\renewcommand{\bibname}{\protect\rm Литература} 