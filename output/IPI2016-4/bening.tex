
\def\stat{bening}

\def\tit{ВЫЧИСЛЕНИЕ АСИМПТОТИЧЕСКОГО ДЕФЕКТА НЕКОТОРЫХ СТАТИСТИЧЕСКИХ
ПРОЦЕДУР, ОСНОВАННЫХ НА ВЫБОРКАХ СЛУЧАЙНОГО ОБЪЕМА$^*$}

\def\titkol{Вычисление асимптотического дефекта некоторых статистических
процедур, основанных на выборках} %случайного объема}

\def\aut{В.\,Е.~Бенинг$^1$}

\def\autkol{В.\,Е.~Бенинг}

\titel{\tit}{\aut}{\autkol}{\titkol}

\index{Бенинг В.\,Е.}
\index{Bening V.\,E.}


{\renewcommand{\thefootnote}{\fnsymbol{footnote}} \footnotetext[1]
{Работа
выполнена при поддержке Российского научного фонда (проект
14-11-00364).}}


\renewcommand{\thefootnote}{\arabic{footnote}}
\footnotetext[1]{Факультет вычислительной математики 
и~кибернетики Московского государственного университета имени М.\,В.~Ломоносова; 
Институт проблем информатики Федерального исследовательского центра <<Информатика 
и~управление>> Российской академии наук, \mbox{bening@yandex.ru}}


\Abst{Рассматривается случай, когда число наблюдений случайно. Это приводит 
к~возникновению распределений с~тяжелыми хвостами и~к~изменению эффективности 
обычно используемых статистических процедур. Проведено асимптотическое сравнение 
статистических процедур, основанных на выборках случайного и~неслучайного объема. 
Для этого используется понятие <<асимптотический дефект>>, 
которое имеет смысл добавочного числа наблюдений, необходимого данной процедуре 
для достижения того же качества, что и~оптимальной процедуре. С~по\-мощью этого 
понятия сравниваются оценки, доверительные множества и~статистические критерии 
в~случае, когда число наблюдений случайно.}


\KW{доверительное множество; статистическая гипотеза; асимптотический дефект; 
выборка случайного объема; распределение Пуассона; биномиальное распределение}

\DOI{10.14357/19922264160404}  


\vskip 10pt plus 9pt minus 6pt

\thispagestyle{headings}

\begin{multicols}{2}

\label{st\stat}

\section{Введение}

В~работе развивается подход, предложенный в~ра\-бо\-тах~[1, 2]. Напомним
кратко постановку задачи и~основные обозначения. Рассмотрим сначала
задачу статистического оценивания известной пара\-мет\-ри\-че\-ской функции
$g(\theta)$, зависящей от неизвестного параметра~$\theta$, 
и~обозначим через~$m(n)$ необходимое число наблюдений, которое
требуется оценке $\delta_{m(n)}(X_1,\ldots,X_{m(n)})$ для достижения
такого же качества (например, среднеквадратичного отклонения или
дисперсии), что и~оценке $\delta^*_{n}(X_1,\ldots,X_{n})$,
основанной на~$n$~наблюдениях $X_1,\ldots,X_n$. Рас\-смат\-ри\-ва\-ет\-ся
асимптотический подход, означающий, что $n \hm\to \infty$. Под
асимптотической относительной эффективностью (АОЭ) оценки
$\delta_n(X_1,\ldots,X_n)$ по отношению к~оценке
$\delta^*_{n}(X_1,\ldots,X_{n})$ понимается предел (в~случае его
существования и~не\-за\-ви\-си\-мости от по\-сле\-до\-ва\-тель\-ности~$m(n)$) вида
(см., например,~\cite[с.~305]{4-ben}):
$$
e \equiv \lim\limits_{n\to\infty} \fr{n}{m(n)}\,.
$$
Вместо отношения необходимого числа наблюдений, естественно, можно
было бы рассматривать разность вида $m(n) \hm- n$, которая тоже имеет
наглядный смысл необходимого дополнительного числа наблюдений,
требующихся оценке $\delta_n(X_1,\ldots,X_n)$. Однако исторически
сложилось так, что многие авторы сначала исследовали асимптотические
свойства отношения~$n/m(n)$ (возможно, в~силу относительной простоты
его поведения).

Впервые общее асимптотическое исследование поведения разности $m(n)
\hm- n$ было предпринято в~1970~г.\ Ходжесом и~Леманом~[4]. 
Они назвали разность $m(n) \hm- n$ дефектом (deficiency)
конкурирующей оценки~$\delta_n$ относительно оценки~$\delta^*_n$ 
и~предложили обозначение:
\begin{equation}
d_n = m(n) - n\,.
\label{e1.1}
\end{equation}
Если предел $\lim_{n\to\infty} d_n$ существует, то он называется
\textit{асимптотическим дефектом} оценки~$\delta_n$ относительно 
оценки~$\delta^*_n$ и~обозначается символом~$d$. Часто~$d$ называют просто
дефектом~$\delta_n$ относительно~$\delta_n^*$. Заметим, что если АОЭ
$e\hm \ne 1$, то $d\hm = \infty$, и~этот случай малоинтересен. В~работе~[4] 
также было отмечено, что существуют статистические задачи, 
в~которых типичным образом возникает случай $e = 1$ (см., например,
книгу~\cite{5-ben}), т.\,е.\ в~этом случае понятие АОЭ не дает ответа на
вопрос, какая оценка лучше, и~понятие дефекта проясняет эту ситуацию,
поскольку в~этом случае асимптотический дефект может, в~принципе,
быть любым.

Обозначим функции риска оценок~$\delta_n$ и~$\delta^*_n$ соответственно через

\noindent
\begin{align*}
R_n(\theta)& = {\sf E}_\theta \left(\delta_n(X_1,\ldots,X_n) 
- g(\theta)\right)^2\,;\\
R^*_n(\theta) & = {\sf E}_\theta \left(\delta^*_n(X_1,\ldots,X_n)
 - g(\theta)\right)^2\,,
\end{align*}
где $g(\theta)$~--- оцениваемая функция, а~$\theta$~--- 
неизвестный параметр (произвольной природы), тогда по определению 
величины $d_n(\theta)\hm \equiv d_n\hm = m(n)\hm - n$ 
для каждого~$n$ должно выполняться равенство:
\begin{equation}
R^*_n(\theta) = R_{m(n)}(\theta)\,. 
\label{e1.2}
\end{equation}
При решении уравнения~(\ref{e1.2}) целочисленную величину~$m(n)$ 
следует рассматривать как переменную, принимающую произвольные действительные значения. 
Для этого можно определить функцию риска $R_{m(n)}(\theta)$ для нецелых значений~$m(n)$ 
по формуле:
\begin{multline*}
R_{m(n)}(\theta) = \left(1 - m(n) + [m(n)]\right) R_{[m(n)]}(\theta) +{}\\
{}+
\left(m(n) - [m(n)]\right) R_{[m(n)]+1}(\theta)
\end{multline*}
(см.\ работу~[4]).

Типичным образом функции риска~$R^*_n(\theta)$ и~$R_n(\theta)$ не известны точно, 
и~используются их аппроксимации.
Предположим, что для функций риска~$R^*_n(\theta)$ и~$R_n(\theta)$ справедливы 
асимптотические разложения вида:
\begin{align}
R^*_n &= \fr{a(\theta)}{n^r} + \fr{b(\theta)}{n^{r+s}} + 
o\left(n^{-r-s}\right)\,;
\label{e1.3}
\\
R_n & = \fr{a(\theta)}{n^r} + \fr{c(\theta)}{n^{r+s}} + 
o\left(n^{-r-s}\right)\,,
\label{e1.4}
\end{align}
где $a(\theta)$, $b(\theta)$ и~$c(\theta)$~--- некоторые постоянные, не зависящие 
от~$n$, а~$r\hm > 0$, $s\hm > 0$~--- некоторые константы, определяющие 
порядок убывания по~$n$ этих функций риска. Первый член в~этих асимптотических 
разложениях одинаков, и~это отражает тот факт, что АОЭ этих оценок равна единице. 
Из соотношений~(\ref{e1.1})--(\ref{e1.4}) легко получить, что (см.\
работу~[4] или книгу~\cite[с.~310]{4-ben})
\begin{equation}
d_n(\theta) \equiv \fr{c(\theta) - b(\theta)}{r a(\theta)} n^{1-s} +
O\left(n^{1-s}\right)\,.
\label{e1.5}
\end{equation}
Таким образом, асимптотический дефект имеет вид:
\begin{equation}
d(\theta) \equiv d = 
\begin{cases}
\pm\infty\,,  &\  0 < s < 1\,;\\
\fr{c(\theta) - b(\theta)}{r a(\theta)}\,,&\  s = 1\,;\\
0\,,&\  s > 1\,.
\end{cases}
\label{e1.6}
\end{equation}
Случай, когда выполняется равенство $s\hm = 1$, представляется наиболее интересным, 
поскольку при этом асимптотический дефект конечен.
Ходжес и~Леман в~работе~[4] привели ряд простых примеров, показывающих естественность 
возникновения этого случая в~математической статистике.

Совершенно аналогично определяется дефект и~в~общем случае асимптотического 
сравнения двух статистических процедур, соответственно с~мерами качества~$\pi_{n}$ 
и~$\pi_{n}^*$. В~этом случае необходимое число наблюдений~$k_n$ для первой 
процедуры определяется из равенства (считая~$k_n$ непрерывной переменной)

\noindent
$$
\pi_{k_n} = \pi_n^*\,,
$$
а предел вида (в~случае его существования)

\noindent
$$
d = \lim\limits_{n\to\infty} \left(k_n - n\right)
$$
называется асимптотическим дефектом первой процедуры относительно второй. 
Если для~$\pi_{n}^*$ и~$\pi_{n}$ выполняются формулы типа~(\ref{e1.3}) 
и~(\ref{e1.4}), то для асимптотического дефекта~$d$ справедливы соотношения 
типа~(\ref{e1.5}) и~(\ref{e1.6}).

В~настоящей работе приведены примеры вычисления асимптотического дефекта в~задачах 
доверительного оценивания и~проверки статистических гипотез с~помощью статистик, 
основанных на выборках случайного объема. Кратко рассмотрена байесовская постановка 
задачи статистического оценивания. Рассмотрены также задачи проверки статистических 
гипотез в~случае выборок случайного объема. Используя асимптотический\linebreak дефект, 
проведено асимптотическое сравнение качества этих процедур, основанных на 
выборках случайного и~неслучайного объема.


\section{Статистическое оценивание параметров распределения}

Рассмотрим случайные величины (с.в.)\ $N_1, N_2, \ldots$ и~$X_1, X_2, \ldots$, 
заданные на одном и~том же вероятностном 
пространстве $(\Omega,\,{\cal A},\,{\sf p})$. При этом с.в.\ 
$X_1, X_2, \ldots X_n$ имеют смысл статистических наблюдений, 
а~с.в.~$N_n$ трактуется как случайный объем выборки, зависящий от 
натурального параметра $n\hm\in \mathbb{N}$.
Типичным образом будем предполагать, что

\noindent
$$
{\sf E} N_n = n\,,
$$
т.\,е.\ в~среднем объем случайной выборки равен размеру выборки неслучайного размера. 
При на\-хож\-де\-нии дефектов статистических процедур, основанных на выборках случайного 
объема~$N_{m(n)}$, и~соответст\-ву\-ющей процедуры, основанной на выборке неслучайного 
объема~$n$, фактически сравнивается средний объем случайной выборки~$m(n)$ и~$n$ 
с~помощью величины $d_n\hm = m(n)\hm - n$ и~ее предела.

Предположим, что для каждого $n\hm\geq1$ с.в.~$N_n$ принимает только 
натуральные значения (т.\,е.\ $N_n \hm\in \mathbb{N}$) и~не зависит от 
последовательности с.в.\ $X_1, X_2, \ldots$ Всюду далее предполагается, что с.в.\
$X_1, X_2, \ldots$ независимы, одинаково распределены и~имеют распределение, 
зависящее от неизвестного параметра $\theta\hm \in \Theta$, при этом множество~$\Theta$ 
может иметь произвольную природу.

Для каждого $n\hm\geq1$ обозначим через $T_n\hm=T_n(X_1,\ldots,X_n)$ некоторую 
статистику, т.\,е.\ действительную измеримую функцию, зависящую от 
наблюдений $X_1,\ldots,X_n$. Для каждого $n\hm\geq1$ определим статистику~$T_{N_n}$, 
зависящую от выборки случайного объема, как
$$
T_{N_n}(\omega) \equiv T_{N_n(\omega)}\left(X_1(\omega),\ldots,X_{N_n(\omega)}(\omega)\right)\,,\enskip
\omega \in \Omega\,.
$$
Далее под статистикой в~этом разделе будем понимать оценку 
(см., например, книгу~[6]) действительной известной 
функции~$g(\theta)$, зависящей от неизвестного параметра $\theta\hm \in \Theta$, 
и~будем обозначать ее символами типа $\delta_n(X_1,\ldots,X_n)$.

В этом разделе иллюстрируются следствия~4.3, 4.6, 4.7 и~теорема~3.2 
из работы~\cite{2-ben}
рассмотрением случая, когда исходная выборка $X_1,\ldots,X_n$ представляет 
собой независимые одинаково нормально распределенные 
с~параметрами $(\theta, \sigma^2)$ с.в. Для удобства ссылок 
приведем без доказательства необходимые результаты из работы~\cite{2-ben}.

\smallskip

\noindent
\textbf{Следствие~2.1.}\ {Пусть существуют числа $a(\theta)$,
$b(\theta)$ и~$r\hm > 0$, $s\hm > 0$ такие, что}
$$
R^*_n(\theta) = \fr{a(\theta)}{n^r} + \fr{b(\theta)}{n^{r+s}}\,,
$$
{тогда}
$$
R_n(\theta) = a(\theta) {\sf E} N_n^{-r} + b(\theta) {\sf E} N_n^{-r-s}\,.
$$


\noindent
\textbf{Теорема~3.1.}\ \textit{Пусть существуют числа~$a(\theta)$,
$b(\theta)$ и~$k_1$, $k_2$ такие, что справедливы соотношения}:
\begin{multline*}
R^*_n(\theta) = {\sf E}_\theta\left(\delta_{n}(X_1,\ldots,X_{n}) 
- g(\theta)\right)^2 ={}\\
{}= \fr{a(\theta)}{n} + \fr{b(\theta)}{n^2} + o \left(n^{-2}\right)
\end{multline*}
\textit{и}
$$
{\sf E} N_n^{-1} = \fr{1}{n} + \fr{k_1}{n^2} + o \left(n^{-2}\right)\,;
$$
$$
{\sf E} N_n^{-2} = \fr{k_2}{n^2} + o\left(n^{-2}\right)\,;\quad 
{\sf E} N_n^{-3} = o \left(n^{-2}\right)\,,
$$
\textit{тогда для асимптотического дефекта оценки 
$\delta_{N_n}(X_1,\ldots,X_{N_n})$ относительно 
оценки $\delta_n(X_1,\ldots$\linebreak $\ldots,X_n)$ справедливо равенство}:
$$
d(\theta) = \fr{k_1 a(\theta) + b(\theta) (k_2 - 1)}{a(\theta)}\,.
$$


\noindent
\textbf{Следствие~3.1.} Пусть в~условиях теоремы~3.1 $k_2\hm = 1$, тогда
асимптотический дефект оценки $\delta_{N_n}(X_1,\ldots,X_{N_n})$ 
относительно оценки $\delta_n(X_1,\ldots,X_n)$ равен~$k_1$, т.\,е.\ 
он не зависит от вида оценки и~имеет вид:
$$
d(\theta) = k_1\,.
$$


\noindent
\textbf{Теорема~3.2.}\ \textit{Пусть существуют числа~$a(\theta)$, 
$b(\theta)$ такие, что для функции риска оценки $\delta_{n}(X_1,\ldots,X_{n})$ 
справедливо соотношение}:
$$ %\begin{multline*}
R^*_n(\theta)= {\sf E}_\theta\left(\delta_{n}(X_1,\ldots,X_{n}) - g(\theta)\right)^2 
= \fr{a(\theta)}{n} + \fr{b(\theta)}{n^2}\,.
$$ %\end{multline*}
\textit{Пусть случайные величины $N_{ni}$, $i\hm = 1, 2$, 
принимают натуральные значения и~не зависят от наблюдений $X_1, X_2,\ldots$ 
Предположим, что для некоторых чисел~$k_{1i}$, $k_{2i}$, $i\hm = 1,2$, 
справедливы равенства}:
\begin{align*}
{\sf E} N_{ni}^{-1} &= \fr{1}{n} + \fr{k_{1i}}{n^2} + o\left(n^{-2}\right)\,,
\\
{\sf E} N_{ni}^{-2} &= \fr{k_{2i}}{n^2} + o \left(n^{-2}\right)\,,\enskip  i = 1,2\,.
\end{align*}
\textit{Тогда для асимптотического дефекта оценки $\delta_{n}^{(2)}\hm \equiv 
\delta_{N_{n2}}(X_1,\ldots,X_{N_{n2}})$ относительно оценки $\delta_{n}^{(1)}\hm 
\equiv \delta_{N_{n1}}(X_1,\ldots,X_{N_{n1}})$ справедливо равенство}:
$$
d_{21}(\theta) = \fr{a(\theta)(k_{12} - k_{11}) + b(\theta) (k_{22} - 
k_{21})}{a(\theta)}\,.
$$

\noindent
\textbf{Следствие~4.2.}\ 
Пусть с.в.~$M_n$ имеет биномиальное распределение 
с~параметрами $m(n\hm-1)$, $n\hm \geqslant 2$, и~$p \hm= 1/m$, 
где $m\hm \geqslant 2$~--- фиксированное натуральное число. Тогда для 
с.в.
$$
N_n = M_n + 1
$$
справедливы равенства:
\begin{align*}
{\sf E} N_n &= n\,;
\\
{\sf E} N_n^{-1} &= \fr{m\left(1 - \left(1 - {1}/{m}\right)^{m(n-1)+1}\right)}
{m(n - 1) + 1} = {}\\
&\hspace*{20mm}{}=\fr{1}{n}+
\fr{m - 1}{mn^2} + O(n^{-3})\,;
\\
{\sf E} N_n^{-2} &= \fr{1}{n^2} + O (n^{-3})\,,\enskip n \to \infty\,.
\end{align*}


\noindent
\textbf{Следствие~4.3.}\ Пусть с.в.~$M_n$ имеет 
биномиальное распределение с~параметрами $m(n-1)$, $n \hm\geqslant 2$, 
и~$p \hm= 1/m$, где $m\hm \geqslant 2$~--- фиксированное натуральное число и~$$
N_n = M_n + 1\,.
$$
Предположим также, что существуют числа $a(\theta)$ и~$b(\theta)$ такие, 
что справедливо соотношение:
$$ %\begin{multline*}
R^*_n(\theta) = {\sf E}_\theta\left(\delta_{n}(X_1,\ldots,X_{n}) - g(\theta)\right)^2 
= \fr{a(\theta)}{n} + \fr{b(\theta)}{n^2}\,.
$$ %\end{multline*}
Тогда для асимптотического дефекта оценки $\delta_{N_n}(X_1,\ldots,X_{N_n})$
 относительно оценки $\delta_n(X_1,\ldots,X_n)$ справедливо равенство:
$$
d(\theta) = \fr{m - 1}{m}\,.
$$


\noindent
\textbf{Следствие~4.4.}\ 
Пусть с.в.~$N_n$ имеет геометрическое распределение 
с~параметром $p\hm = 1/n$, $n\hm \geqslant 2$, $n\hm \in \mathbb{N}$. 
Тогда справедливы равенства:
\begin{align*}
{\sf E} N_n &= n\,;
\\
{\sf E} N_n^{-1} &= \fr{\log n}{n - 1} = 
\fr{\log n}{n} + \fr{\log n}{n^2} + O\left(\fr{\log n}{n^3}\right)\,;
\\
{\sf E} N_n^{-2} &= \fr{\pi^2}{6n} - \fr{\log n}{n^2} + 
\fr{1}{n^2} \left(\fr{\pi^2}{6} - 1\right) + O\left(\fr{\log n}{n^3}\right)\,,\\
&\hspace*{57mm}n \to \infty\,.
\end{align*}


\noindent
\textbf{Следствие~4.5.}\ 
Пусть с.в.~$M_n$ имеет распределение Пуассона с~параметром 
$\lambda\hm = n \hm- 1$, $n\hm \geqslant 2$, $n\hm \in \mathbb{N}$, и~$$
N_n = M_n + 1\,.
$$
Тогда справедливы равенства:
$$
{\sf E} N_n = n\,;\enskip
{\sf E} N_n^{-1} = \fr{1}{n} + \fr{1}{n^2} + O \left(\fr{1}{n^3}\right)\,;
$$
$$
{\sf E} N_n^{-2} = \fr{1}{n^2} + O\left(\fr{1}{n^3}\right)\,,\enskip
n \to \infty\,.
$$


\noindent
\textbf{Следствие~4.6.}\ 
Пусть с.в.~$M_n$ имеет распределение Пуассона 
с~параметром $\lambda\hm = n\hm-1$,  $n\hm \geqslant 2$, и~$$
N_n = M_n + 1\,.
$$
Предположим также, что существуют числа~$a(\theta)$ и~$b(\theta)$ 
такие, что справедливо соотношение:
$$ %\begin{multline*}
R^*_n(\theta) = {\sf E}_\theta\left(\delta_{n}(X_1,\ldots,X_{n}) - g(\theta)\right)^2 
= \fr{a(\theta)}{n} + \frac{b(\theta)}{n^2}\,.
$$ %\end{multline*}
Тогда для асимптотического дефекта оценки $\delta_{N_n}(X_1,\ldots,X_{N_n})$ 
относительно оценки $\delta_n(X_1,\ldots,X_n)$ справедливо равенство:
$$
d(\theta) = 1\,.
$$

\noindent
\textbf{Следствие~4.7.}\ Пусть с.в.~$N_n$ имеет геометрическое распределение 
с~параметром $p = 1/n$, $n \geqslant 2$.
Предположим также, что существуют числа $a(\theta)$  и~$b(\theta)$ такие, 
что для функции риска оценки $\delta_{n}(X_1,\ldots,X_{n})$ справедливо соотношение:
$$ %\begin{multline*}
R^*_n(\theta) = {\sf E}_\theta\left(\delta_{n}(X_1,\ldots,X_{n}) -
 g(\theta)\right)^2 
= \fr{a(\theta)}{n} + \fr{b(\theta)}{n^2}\,.
$$ %\end{multline*}
Тогда для функции риска оценки $\delta_{N_n}(X_1,\ldots,X_{N_n})$ справедливо 
равенство:
\begin{multline*}
R_n(\theta) = {\sf E}_\theta\left(\delta_{N_n}(X_1,\ldots,X_{N_n}) - 
g(\theta)\right)^2 ={}
\\
{}= a(\theta) {\sf E} N_n^{-1} + b(\theta) {\sf E} N_n^{-2} ={}
\\
{}= \fr{a(\theta) \log n}{n} + \fr{\pi^2 b(\theta)}{6n} +
\fr{(a(\theta) - b(\theta)) \log n}{n^2} +{}
\\
{}+ \fr{b(\theta)}{n^2} \left(\fr{\pi^2}{6} - 1\right) + O
\left(\fr{\log n}{n^3}\right)\,,\enskip n \to \infty\,.
\end{multline*}

В этом разделе рассмотрена также байесовская постановка задачи оценивания. 
Предположим сначала, что требуется оценить функцию
$
g(\theta) = \theta^2
$
в присутствии мешающего параметра~$\sigma^2$. Оценка максимального правдоподобия 
в~этом случае имеет вид:
$$
\delta_{0n}\left(X_1,\ldots,X_n\right) = \left(\bar X_n\right)^2\,,
$$
где
$$
\bar X_n = \fr{1}{n} \sum\limits_{i=1}^n X_i\,.
$$
Если $\sigma^2$ известно, то несмещенная оценка с~минимальной дисперсией 
для функции~$g(\theta)$, основанная на~$n$ наблюдениях, есть
$$
\delta_{1n}\left(X_1,\ldots,X_n\right) = \left(\bar X_n\right)^2 -\fr{\sigma^2}{n}\,.
$$
Обе эти оценки являются частными ($c\hm = 0$, $c\hm = 1$) случаями оценок вида:
\begin{equation}
\delta_{cn}\left(X_1,\ldots,X_n\right) = 
\left(\bar X_n\right)^2 - \fr{c \sigma^2}{n}\,,\enskip
c \in \mathbb{R}\,.
\label{e2.1}
\end{equation}
Функция риска этих оценок имеет вид (см.~\cite[с.~302]{4-ben}):
\begin{multline}
R^*_{cn}(\theta) = {\sf E}_\theta \left(\delta_{cn}\left(X_1,\ldots,X_n\right) - 
\theta^2\right)^2 = {}\\
{}=\fr{4 \sigma^2\theta^2}{n} + 
\fr{(c^2 - 2c + 3) \sigma^4}{n^2}\,.\label{e2.2}
\end{multline}
Если $\sigma^2$ неизвестно, то несмещенная оценка с~минимальной дисперсией 
для функции~$g(\theta)$, основанная на~$n$~наблюдениях, есть
\begin{equation*}
\delta_{n}\left(X_1,\ldots,X_n\right) = \left(\bar X_n\right)^2 - 
\fr{S^2_n}{n}\,, %\label{e2.3}
\end{equation*}
где
$$
S^2_n = \fr{1}{n-1} \sum\limits_{i=1}^n \left(X_i - \bar X_n\right)^2\,.
$$
Используя независимость~$\bar X_n$ и~$S_n^2$, нетрудно получить, 
что функция риска оценки~$\delta_n$ имеет вид:
\begin{multline*}
R^*_{n}(\theta) = {\sf E}_\theta \left(\delta_{n}\left(X_1,\ldots,X_n\right) - 
\theta^2\right)^2 = {}\\
{}={\sf D}_\theta \bar X_n^2 + \fr{1}{n^2}\, 
{\sf D}_\theta S_n^2 =
\fr{4\sigma^2\theta^2}{n} + \fr{4\sigma^4}{n^2}\,. %\label{e2.4}
\end{multline*}
Рассмотрим теперь оценки $\delta_{cn}(X_1,\ldots,X_n)$ 
и~$\delta_{n}(X_1,\ldots,X_n)$, основанные на выборке случайного объема~$N_n$. 
Согласно следствию~2.1 из работы~\cite{2-ben} имеем:
\begin{multline*}
R_{cn}(\theta) = {\sf E}_\theta \left(\delta_{cN_n}\left(X_1,\ldots,X_{N_n}\right) 
- \theta^2\right)^2 ={}\\
{}= 4\sigma^2\theta^2 {\sf E} N_n^{-1} + 
\left(c^2 - 2c + 3\right) \sigma^4 {\sf E} N_n^{-2}\,;
%\label{e2.5}
\end{multline*}

\vspace*{-12pt}

\noindent
\begin{multline*}
R_{n}(\theta) = {\sf E}_\theta \left(\delta_{N_n}\left(X_1,\ldots,X_{N_n}\right) - 
\theta^2\right)^2 ={}\\
{}=
4\sigma^2\theta^2 {\sf E} N_n^{-1} + 4\sigma^4 {\sf E} N_n^{-2}\,.
%\label{e2.6}
\end{multline*}

\noindent
\textbf{Лемма~2.1.}\
\textit{Пусть с.в.~$M_n$ имеет биномиальное распределение 
с~параметрами $m(n-1)$, $n\hm \geqslant 2$, и~$p\hm = 1/m$, где $m\hm \geqslant 2$~--- фиксированное натуральное число, и}
$$
N_n = M_n + 1\,.
$$
\textit{Тогда асимптотические дефекты оценок
$\delta_{cN_n}(X_1,\ldots,X_{N_n})$, $\delta_{N_n}(X_1,\ldots,X_{N_n})$ 
относительно оценок
$\delta_{cn}(X_1,\ldots,X_{n})$, $\delta_{n}(X_1,\ldots,X_{n})$ 
соответственно равны}:
$$
d_c(\theta) = \fr{m - 1}{m}\,;\quad d(\theta) = \fr{m - 1}{m}\,.
$$
\textit{Асимптотический дефект оценки
$\delta_{cn}(X_1,\ldots,X_{n})$ относительно оценки
$\delta_{n}(X_1,\ldots,X_{n})$ равен}:
$$
d(c, \theta) = \fr{(c^2 - 2c - 1)\sigma^2}{4\theta^2}\,.
$$
\textit{Асимптотический дефект оценки
$\delta_{cN_{n}}(X_1,\ldots,X_{N_n})$ относительно оценки
$\delta_{N_{n}}(X_1,\ldots,X_{N_n})$ также равен}:
$$
d(c, \theta) = \fr{(c^2 - 2c - 1)\sigma^2}{4\theta^2}\,.
$$


\noindent
Д\,о\,к\,а\,з\,а\,т\,е\,л\,ь\,с\,т\,в\,о~~непосредственно 
вытекает из следствия~4.3 и~формул~(1.6), 
(5.5), (5.6), теоремы~3.1 и~следствия~4.2 из работы~\cite{2-ben}.

\smallskip


\noindent
\textbf{Лемма~2.2.}\
\textit{Пусть с.в.~$M_n$ имеет распределение Пуассона 
с~параметром $\lambda\hm = n\hm - 1$, $n\hm \geqslant 2$, $n\hm \in \mathbb{N}$, и}
$$
N_n = M_n + 1\,.
$$
\textit{Тогда асимптотические дефекты оценок
$\delta_{cN_n}(X_1,\ldots,X_{N_n})$ 
 и~$\delta_{N_n}(X_1,\ldots,X_{N_n})$ относительно оценок
$\delta_{cn}(X_1,\ldots,X_{n})$ и~$\delta_{n}(X_1,\ldots,X_{n})$ 
соответственно равны}:
$$
d_c(\theta) = 1\,;\quad d(\theta) = 1\,.
$$
\textit{Асимптотический дефект оценки
$\delta_{cN_{n}}(X_1,\ldots,X_{N_{n}})$ относительно оценки
$\delta_{N_{n}}(X_1,\ldots,X_{N_{n}})$ равен}:
$$
d(c, \theta) = \fr{(c^2 - 2c - 1)\sigma^2}{4\theta^2}\,.
$$

\noindent
Д\,о\,к\,а\,з\,а\,т\,е\,л\,ь\,с\,т\,в\,о~~непосредственно 
вытекает из следствия~4.6 и~формул~(5.5), (5.6), 
теоремы~3.1, следствия~ 3.1 и~следствия~4.4 из работы~\cite{2-ben}.
Непосредственным следствием теоремы~3.2 и~следствий~4.2, 4.5~\cite{2-ben}
и~формулы~(\ref{e2.2}) является следующая теорема.

\smallskip

\noindent
\textbf{Теорема~2.1.}\ \textit{Пусть с.в.~$M_{n1}$ имеет 
биномиальное распределение с~параметрами $m(n-1)$, $n \geqslant 2$, и~$p = 1/m,$ 
где $m \geqslant 2$~--- фиксированное натуральное число, и}
$$
N_{n1} = M_{n1} + 1\,.
$$
\textit{Пусть также с.в.~$M_{n2}$ имеет распределение Пуассона 
с~параметром $\lambda\hm = n\hm - 1$,
$n\hm \geqslant 2$, $n\hm \in \mathbb{N}$, и}

\noindent
$$
N_{n2} = M_{n2} + 1\,.
$$
\textit{Предположим, что с.в.~$N_{ni}$,
$i\hm = 1,2$, не зависят от наблюдений $X_1, X_2,\ldots$
Тогда для асимптотического дефекта оценки} (\textit{см}.~(\ref{e2.1})) 
$\delta_{cn}^{(2)}\hm \equiv \delta_{cN_{n2}}(X_1,\ldots,X_{N_{n2}})$ 
\textit{относительно оценки $\delta_{cn}^{(1)}\hm \equiv 
\delta_{cN_{n1}}(X_1,\ldots,X_{N_{n1}})$ справедливо равенство}:

\noindent
$$
d_{21}(\theta) = - \fr{1}{m}\,.
$$

\noindent
\textbf{Теорема~2.2.}\
\textit{Пусть с.в.~$N_n$ имеет геометрическое распределение 
с~параметром $p\hm= 1/n$, $n\hm\geqslant 2$.
Тогда для функции риска оценки $\delta_{cN_n}(X_1,\ldots,X_{N_n})$ 
справедливо равенство}:

\noindent
\begin{multline*}
R_{cn}(\theta) = {\sf E}_\theta\left(\delta_{cN_n}
\left(X_1,\ldots,X_{N_n}\right) - \theta^2\right)^2 ={}
\\
{}= \fr{4\sigma^2\theta^2 \log n}{n} +\fr{\pi^2 (c^2 - 2c + 3)\sigma^4}{6n} +{}\\
{}+
 \fr{(4\sigma^2\theta^2 - (c^2 - 2c + 3)\sigma^4) \log n}{n^2} +{}
\\
{}+ \fr{(c^2 - 2c + 3)\sigma^4}{n^2} \left(\fr{\pi^2}{6} - 1\right) + 
O\left(\fr{\log n}{n^3}\right)\,,\enskip  n \to \infty\,.\hspace*{-5.3156pt}
\end{multline*}

\noindent
Д\,о\,к\,а\,з\,а\,т\,е\,л\,ь\,с\,т\,в\,о~~теоремы~2.2 вытекает из 
следствия~4.7~\cite{2-ben} и~формулы~(\ref{e2.2}).

\smallskip

\noindent
\textbf{Замечание~2.1.}\
Рассмотрим теперь байесовскую постановку. Пусть исходная выборка $X_1,\ldots,X_n$ 
представляет собой независимые одинаково нормально распределенные с.в.\
с~параметрами:
$$
{\sf E}_\theta X_1 = \theta\,;\quad {\sf D}_\theta X_1 = \sigma^2\,.
$$
Предположим, что дисперсия~$\sigma^2$ известна, и~рассмотрим оптимальную 
несмещенную оценку

\noindent
$$
\bar X_n = \fr{1}{n} \sum\limits_{i=1}^n X_i
$$
функции
$$
g(\theta) = \theta\,.
$$
Ее функция риска имеет вид:
$$
R^*_{n}(\theta) = {\sf E}_\theta \left(\bar X_n - \theta\right)^2 =
 {\sf D}_\theta \bar X_n = \fr{\sigma^2}{n}\,.
$$
Предположим теперь, что параметр~$\theta$ имеет априорное нормальное распределение 
с~параметрами $\mu\hm \in \mathbb{R}$ и~$b^2 \hm> 0$. Тогда смещенная 
байесовская оценка параметра~$\theta$ имеет вид (см.~\cite[с.~222]{4-ben}):
$$
\theta_n = \fr{nb^2\bar X_n + \sigma^2\mu}{nb^2 + \sigma^2}\,.
$$
Нетрудно видеть, что функция риска этой оценки может быть записана 
в~виде (см.~\cite[формулы~(6.2) и~(6.3)]{1-ben}): 
\begin{multline*}
R_{n}(\theta) = {\sf E}_\theta \left(\theta_n - \theta\right)^2 ={}\\
{}=
\fr{\sigma^2}{(nb^2 + \sigma^2)^2} \left(nb^4 +\sigma^2(\theta - \mu)^2\right) =
\\
{}= \fr{\sigma^2}{n} + \fr{\sigma^4}{n^2b^4}
 \left((\theta - \mu)^2 - 2b^2\right)
+  O\left(n^{-3}\right)\,.
\end{multline*}
Таким образом, при естественных условиях на случайный объем выборки~$N_n$ 
асимптотический \mbox{дефект} оценки~$\theta_{N_n}$ относительно наилучшей оценки~$\bar X_n$ 
при неслучайном объеме выборки, например, в~пуассоновском случае равен:
$$
d = \fr{\sigma^2}{b^4} \left((\theta - \mu)^2 - 2b^2\right) + 1\,.
$$
Для биномиального случая соответственно имеем:
$$
d = \fr{\sigma^2}{b^4} \left((\theta - \mu)^2 - 2b^2\right) + \fr{m - 1}{m}\,.
$$
В геометрическом случае для функции риска оценки~$\bar X_{N_n}$ 
справедливо представление:
\begin{multline*}
R^*_{N_n}(\theta) = {\sf E}_\theta \left(\bar X_{N_n} - \theta\right)^2 ={}\\
{}=
\fr{\sigma^2\log n}{n} + \fr{\sigma^2\log n}{n^2} +  O\left(\fr{\log n}{n^3}\right)\,.
\end{multline*}

\noindent
\textbf{Замечание~2.2.}
Рассмотрим теперь байесовскую постановку для биномиального распределения. 
Пусть исходная выборка $X_1,\ldots,X_n$ представляет собой независимые одинаково 
распределенные с.в., принимающие значения~0 и~1 
с~вероятностями $\theta\hm \in (0,1)$ и~$1\hm - \theta$ соответственно.
Оптимальная несмещенная оценка для параметра~$\theta$ есть
$$
\bar X_n = \fr{1}{n} \sum\limits_{i=1}^n X_i\,.
$$
Ее функция риска имеет вид:
$$
R^*_{n}(\theta) = {\sf E}_\theta \left(\bar X_n - \theta\right)^2 =
 {\sf D}_\theta \bar X_n =\fr{\theta(1 - \theta)}{n}\,.
$$
Предположим теперь, что параметр~$\theta$ имеет априорное бе\-та-рас\-пре\-де\-ле\-ние 
с~параметрами $(a,b)$, $a\hm > 0$, $b\hm > 0$. Тогда байесовская оценка 
параметра~$\theta$ имеет вид (см.~\cite[с.~221]{4-ben}):
$$
\hat\theta_n = \fr{a + n\bar X_n}{a + b + n}\,.
$$
Нетрудно видеть, что функция риска этой оценки может быть записана в~виде:
\begin{multline*}
R_{n}\left(\hat\theta\right) = {\sf E}_\theta \left(\hat\theta_n - \theta\right)^2 ={}\\
{}=
\fr{1}{(a + b + n)^2} \left(n\theta(1 - \theta) + \left(a(1 -
\theta) - \theta b\right)^2\right) ={}
\\
{}= \fr{\theta(1 - \theta)}{n} + \fr{\left(\! \left(a\left(1 - \theta\right) - 
\theta b\right)^2\! -\theta(1 - \theta)2(a + b)\!\right)}{n^2} + {}\\
{}+ O\left(n^{-3}\right).
\end{multline*}
Таким образом, при естественных условиях на случайный объем выборки~$N_n$ 
асимптотический\linebreak дефект оценки~$\theta_{N_n}$ относительно наилучшей оценки~$\bar X_n$ 
при неслучайном объеме выборки, например, в~пуассоновском случае равен:
$$
d(\theta) = \fr{\left( \left(a(1 - \theta) - \theta b\right)^2 -
\theta(1 - \theta)2(a + b)\right)}{\theta(1 - \theta)} + 1\,.
$$
Если $a = b$ и~$\theta\hm = 1/2$, то эта формула упрощается и~$$
d\left(\fr{1}{2}\right) = 1 - 4a\,.
$$
Если случайный объем выборки~$N_n$ имеет биномиальное распределение, то соответствующий 
асимп\-то\-ти\-че\-ский дефект равен:
$$
d(\theta) = \fr{\left( \!\left(a(1 - \theta) - \theta b\right)^2\! -
\theta(1 - \theta)2(a + b)\!\right)}{\theta(1 - \theta)} + \fr{m - 1}{m}.\hspace*{-0.83063pt}
$$
В случае $a = b$ и~$\theta\hm = 1/2$ имеем:
$$
d\left(\fr{1}{2}\right) = \fr{m - 1}{m} - 4a\,.
$$
В геометрическом случае для функции риска оценки~$\bar X_{N_n}$ справедливо 
представление:
\begin{multline*}
R^*_{N_n}(\theta) = {\sf E}_\theta \left(\bar X_{N_n} - \theta\right)^2 ={}\\
{}=
\fr{\theta(1 - \theta)\log n}{n} + \fr{\theta(1 - \theta)\log n}{n^2} + 
 O\left(\fr{\log n}{n^3}\right)\,.
\end{multline*}


\section{Доверительное оценивание}

В этом разделе находятся асимптотические дефекты некоторых доверительных 
интервалов, основанных на выборках случайного объема.

Пусть исходные наблюдения $X_1,\ldots,X_n$ представляют собой независимые 
одинаково распределенные с.в.\ с~плотностью вида:
\begin{equation}
p(x,\theta) = \begin{cases}
e^{-x + \theta}\,, &\  x \geqslant \theta\,;\\
0\,, &\  x < \theta,
\end{cases}\enskip
\theta \in \mathbb{R}\,.
\label{e3.1}
\end{equation}
Нетрудно видеть, что для каждого $\alpha\hm \in (0,1)$ и~каж\-до\-го 
$n\hm \in \mathbb{N}$ 
в~качестве доверительного интервала с~коэффициентом доверия $1\hm - \alpha$ для 
параметра~$\theta$ можно взять интервал вида:
\begin{equation}
\left(X_{n:1} + \fr{\log\alpha}{n}, X_{n:1}\right)\,,
\label{e3.2}
\end{equation}
где $X_{n:1} \equiv \min\{X_1,\ldots,X_n\}$~--- 
минимальная порядковая статистика, построенная по исходной выборке
$X_1,\ldots,X_n$.
Длина этого доверительного интервала неслучайна и~равна
\begin{equation}
-\fr{\log\alpha}{n}\,.
\label{e3.3}
\end{equation}
В качестве меры качества доверительного интервала~(\ref{e3.2}) 
выберем его длину~(\ref{e3.3}).

Рассмотрим теперь доверительный интервал для параметра~$\theta$, 
основанный на выборке случайного объема~$N_n$ вида:
\begin{equation}
\left(X_{N_n:1} + \fr{\log\alpha}{N_n},\, X_{N_n:1}\right)\,.
\label{e3.4}
\end{equation}
Нетрудно видеть, что он имеет также коэффициент доверия 
$1\hm - \alpha$ и~случайную длину
$-{\log\alpha}/{N_n}$.

Рассмотрим в~качестве меры качества такого рода доверительных интервалов 
их среднюю длину, т.\,е.\ величину
\begin{equation}
-\log\alpha {\sf E} N_n^{-1}
\label{e3.5}
\end{equation}
при условии ${\sf E} N_n \hm= n$.
Найдем дефект случайного доверительного интервала~(\ref{e3.4}) 
относительно неслучайного доверительного интервала~(\ref{e3.2}).

\bigskip

\noindent
\textbf{Теорема~3.1.}\
\textit{Пусть исходные наблюдения $X_1,\ldots,X_n$ 
представляют собой независимые одинаково распределенные 
с.в.\ с~плотностью вида}~(\ref{e3.1}). \textit{Тогда}
\begin{enumerate}[(1)]
\item \textit{если с.в.~$M_n$ имеет биномиальное распределение 
с~параметрами $m(n-1)$, $n\hm \geqslant 2$, 
и~$p\hm = 1/m$, где $m\hm \geqslant 2$~--- фиксированное натуральное чис\-ло,~и}
$$
N_n = M_n + 1\,,
$$
\textit{то асимптотический дефект случайного доверительного интервала}~(\ref{e3.4}) 
\textit{относительно неслучайного доверительного интервала}~(\ref{e3.2}) \textit{равен}
$$
d = \fr{m - 1}{m}\,;
$$

\item \textit{если с.в.~$M_n$ имеет распределение Пуассона 
с~параметром $\lambda\hm = n\hm - 1$, $n\hm \geqslant 2$, 
$n\hm \in \mathbb{N}$, и}
$$
N_n = M_n + 1\,,
$$
\textit{то асимптотический дефект случайного доверительного интервала}~(\ref{e3.4}) 
\textit{относительно неслучайного доверительного интервала}~(\ref{e3.2}) \textit{равен}
$$
d = 1\,;
$$

\item 
\textit{если с.в.~$N_n$ имеет геометрическое распределение 
с~параметром $p\hm = 1/n$, $n\hm \geqslant 2$, то средняя длина случайного 
доверительного интервала}~(\ref{e3.4}) \textit{равна}
\begin{multline*}
-\log\alpha {\sf E} N_n^{-1} = - \fr{\log\alpha\log n}{n - 1} = {}\\
{}=
- \log\alpha \left(\fr{\log n}{n} + \fr{\log n}{n^2} + O
\left(\fr{\log n}{n^3}\right)\right)\,.
\end{multline*}
\end{enumerate}

\noindent
Д\,о\,к\,а\,з\,а\,т\,е\,л\,ь\,с\,т\,в\,о~~теоремы~3.1 непосредственно следует 
из формул~(\ref{e3.3}), (\ref{e3.5}), (\ref{e1.6}) и~следствий~4.2, 4.5 
и~4.4 работы~\cite{2-ben}.

Аналогично можно получить следующую тео\-рему.

\bigskip

\noindent
\textbf{Теорема~3.2.}\ \textit{Пусть исходные наблюдения $X_1,\ldots,X_n$ 
имеют плотность вида}~(\ref{e3.1}). 
\textit{Предположим, что с.в.~$M_{n1}$ 
имеет биномиальное распределение с~па\-ра\-мет\-ра\-ми $m(n-1)$, $n\hm \geqslant 2$, и~$p\hm = 1/m$, где $m\hm \geqslant 2$~--- фиксированное натуральное число, и~$$
N_{n1} = M_{n1} + 1\,.
$$
Пусть также с.в.~$M_{n2}$ имеет распределение 
Пуассона с~параметром $\lambda\hm = n\hm - 1$, $n\hm \geqslant 2$,
$n\hm \in \mathbb{N}$, и}
$$
N_{n2} = M_{n2} + 1\,.
$$
\textit{Предположим, что с.в.~$N_{ni}$, $i\hm = 1,2$, не зависят 
от наблюдений $X_1, X_2,\ldots$
Тогда для асимптотического дефекта (с~критерием качества}~(\ref{e3.5})) 
\textit{доверительного интервала}
$\left(X_{N_ {n1}:1} + \log\alpha/{N_{n1}},\right.$\linebreak $\left.X_{N_{n1}:1}\right)$
\textit{относительно доверительного интервала}
$\left(X_{N_{n2}:1} \hm+ {\log\alpha}/{N_{n2}},\, X_{N_{n2}:1}\right)$
\textit{справедливо равенство}:
$$
d = - \fr{1}{m}\,.
$$

Пусть теперь исходные наблюдения $X_1,\ldots,X_n$ представляют собой независимые 
одинаково распределенные нормальные с~па\-ра\-мет\-ра\-ми $\theta\hm \in \mathbb{R}$,
$\sigma^2\hm > 0$ случайные величины. Предположим, что дисперсия~$\sigma^2$ известна.
Для параметра~$\theta$ рас\-смот\-рим доверительный интервал с~коэффициентом 
доверия $1\hm - \alpha$ вида
\begin{equation}
\left(\bar X_n - \fr{\sigma u_\alpha}{\sqrt n},\, 
\bar X_n + \fr{\sigma u_\alpha}{\sqrt n}\right)\,,
\label{e3.6}
\end{equation}
где
$$
\bar X_n = \fr{1}{n} \sum\limits_{i=1}^n X_i
$$
а стандартная нормальная функция распределения
$$
\Phi\left(u_\alpha\right) = 1 - \fr{\alpha}{2}\,.
$$
Длина этого доверительного интервала неслучайна и~равна
${2\sigma u_\alpha}/{\sqrt n}$.

В качестве меры качества доверительного интервала~(\ref{e3.6}) 
выберем квадрат его длины, т.\,е.\ величину
\begin{equation}
\fr{4\sigma^2 u_\alpha^2}{n}\,.
\label{e3.7}
\end{equation}
Рассмотрим теперь доверительный интервал для параметра~$\theta$, основанный 
на выборке случайного объема~$N_n$ вида:
\begin{equation}
\left(\bar X_{N_n} - \fr{\sigma u_\alpha}{\sqrt {N_n}},\, 
\bar X_{N_n} + \fr{\sigma u_\alpha}{\sqrt {N_n}}\right)\,.
\label{e3.8}
\end{equation}
Он имеет также коэффициент доверия $1\hm - \alpha$ и~случайную длину
${2\sigma u_\alpha}/{\sqrt {N_n}}$.

Аналогично в~качестве меры качества этого 
доверительного интервала рассмотрим средний квадрат его длины, т.\,е.\ величину
\begin{equation}
4\sigma^2 u_\alpha^2 {\sf E} N_n^{-1}\label{e3.9}
\end{equation}
при условии ${\sf E} N_n\hm = n$.
Найдем дефект случайного доверительного интервала~(\ref{e3.8}) 
относительно неслучайного доверительного интервала~(\ref{e3.6}).

\smallskip

\noindent
\textbf{Теорема~3.3.}\
\textit{Пусть исходные наблюдения $X_1,\ldots,X_n$ представляют собой независимые 
одинаково нормально распределенные с.в.\ с~параметрами 
$\theta\hm \in \mathbb{R}$ и~$\sigma^2\hm > 0$. Тогда}
\begin{enumerate}[(1)]
\item \textit{если с.в.~$M_n$ имеет биномиальное распределение 
с~параметрами $m(n-1)$, $n\hm \geqslant 2$, 
и~$p\hm = 1/m,$ где $m\hm \geqslant 2$~--- фиксированное натуральное число, и}
$$
N_n = M_n + 1\,,
$$
\textit{то асимптотический дефект случайного доверительного интервала}~(\ref{e3.8}) 
\textit{относительно неслучайного доверительного интервала}~(\ref{e3.6}) 
(\textit{с~критериями качества}~(\ref{e3.7}) \textit{и}~(\ref{e3.9})) \textit{равен}
$$
d = \fr{m - 1}{m}\,;
$$

\item \textit{если с.в.~$M_n$ имеет распределение Пуассона с~параметром 
$\lambda\hm = n\hm - 1$, $n\hm \geqslant 2$, $n\hm \in \mathbb{N}$, и}
$$
N_n = M_n + 1\,,
$$
\textit{то асимптотический дефект случайного доверительного интервала}~(\ref{e3.8}) 
\textit{относительно неслучайного доверительного интервала}~(\ref{e3.6}) \textit{равен}
$$
d = 1\,;
$$

\item \textit{если с.в.~$N_n$ имеет геометрическое распределение 
с~параметром $p\hm = 1/n$, $n\hm \geqslant 2$, то средний квадрат длины 
случайного доверительного интервала}~(\ref{e3.8}) \textit{равен}
\begin{multline*}
4\sigma^2 u_{\alpha}^2 {\sf E} N_n^{-1} = 
\fr{4\sigma^2 u_{\alpha}^2\log n}{n - 1} = {}\\
{}=
4\sigma^2 u_{\alpha}^2 \left(\fr{\log n}{n} + \fr{\log n}{n^2} +  O
\left(\fr{\log n}{n^3}\right)\right)\,.
\end{multline*}
\end{enumerate}

\noindent
Д\,о\,к\,а\,з\,а\,т\,е\,л\,ь\,с\,т\,в\,о~~теоремы~3.3 непосредственно 
следует из формул~(\ref{e3.7}), (\ref{e3.9}), (\ref{e1.6}) 
и~следствий~4.2, 4.5 и~4.4 из работы~\cite{2-ben}. 
Аналогично можно получить следующую теорему.

\smallskip

\noindent
\textbf{Теорема~3.4.}\ \textit{Пусть исходные наблюдения $X_1,\ldots,X_n$ имеют 
нормальное распределение с~параметрами $\theta\hm \in \mathbb{R}$ и~$\sigma^2\hm > 0$. 
Предположим, что с.в.~$M_{n1}$ имеет биномиальное распределение 
с~па\-ра\-мет\-ра\-ми $m(n-1)$, $n\hm \geqslant 2$, и~$p\hm = 1/m,$ где $m\hm \geqslant 2$~--- фиксированное натуральное число, и}
$$
N_{n1} = M_{n1} + 1\,.
$$
\textit{Пусть также с.в.~$M_{n2}$ имеет распределение Пуассона 
с~параметром $\lambda\hm = n\hm - 1$, $n\hm \geqslant 2$, $n\hm \in \mathbb{N}$, и}
$$
N_{n2} = M_{n2} + 1\,.
$$
\textit{Предположим, что с.в.~$N_{ni}$, $i\hm = 1,2$, не 
зависят от наблюдений $X_1, X_2,\ldots$
Тогда для асимптотического дефекта} (\textit{с~критерием качества}~(\ref{e3.9})) 
\textit{доверительного интервала}
$\left(\bar X_{N_{n1}} \hm- {\sigma u_\alpha}/{\sqrt {N_{n1}}},\, 
\bar X_{N_{n1}} \hm+ {\sigma u_\alpha}/{\sqrt {N_{n1}}}\right)$
\textit{относительно доверительного интервала}
$
\left(\bar X_{N_{n2}} \hm- {\sigma u_\alpha}/{\sqrt {N_{n2}}},\,
\bar X_{N_{n2}}\hm + {\sigma u_\alpha}/{\sqrt {N_{n2}}}\right)
$
\textit{справедливо равенство}:
$$
d = - \fr{1}{m}\,.
$$


\section{Проверка статистических гипотез}

В этом разделе находится асимптотический дефект критерия Стьюдента 
относительно наиболее мощного критерия в~случае выборки случайного объема 
из нормального распределения.

Пусть наблюдения $X_1,\ldots,X_n$ есть независимые одинаково распределенные 
нормальные с~па\-ра\-мет\-ра\-ми~$\theta\hm \in \mathbb{R}$, $\sigma^2\hm > 0$ случайные 
величины. Предположим, что дисперсия~$\sigma^2$ известна.
Рассмотрим задачу проверки простой гипотезы
\begin{equation*}
{\bf H}_0:\ \theta = 0
%\label{e4.1}
\end{equation*}
против сложной односторонней альтернативы
\begin{equation*}
{\bf H}_1:\ \theta > 0\,.
%\label{e4.2}
\end{equation*}
По лемме Неймана--Пир\-со\-на равномерно наиболее мощный критерий уровня 
значимости $\alpha\hm \in (0,1)$ существует и~отвергает гипотезу~${\bf H}_0$, если
\begin{equation}
\fr{\sqrt n \bar X_n}{\sigma} > u_{1-\alpha}\,,\label{e4.3}
\end{equation}
где
$$
\bar X_n = \fr{1}{n} \sum\limits_{i=1}^n X_i
$$
и
$$
\Phi\left(u_{1-\alpha}\right) = 1 - \alpha\,.
$$
Функция мощности этого критерия имеет вид:
\begin{equation*}
\beta^*_n(\theta) = \Phi\left(\sqrt {n}\,  \theta - u_{1-\alpha}\right)\,.
%\label{e4.4}
\end{equation*}
Предположим, что исходная выборка имеет случайный объем~$N_n$, и~рассмотрим критерий, 
отвергающий гипотезу~${\bf H}_0$, если (см.~(\ref{e4.3}))
\begin{equation}
\fr{\sqrt {N_n}\, \bar X_{N_n}}{\sigma} > u_{1-\alpha}\,.\label{e4.5}
\end{equation}
Тогда этот критерий также имеет размер~$\alpha$ и~мощность
\begin{multline*}
{\sf P}_\theta \left(\fr{\sqrt {N_n}\, \bar X_{N_n}}{\sigma} > 
u_{1-\alpha}\right) = {\sf E} \beta^*_{N_n}(\theta) ={}
\\
{}= {\sf E} \Phi\left(\sqrt {N_n}\, \theta - u_{1-\alpha}\right)\,.
%\label{e4.6}
\end{multline*}
Рассмотрим теперь для проверки гипотезы~${\bf H}_0$ против 
альтернативы~${\bf H}_1$ критерий Стьюдента, основанный на $k\hm \in \mathbb{N}$ 
наблюдениях и~имеющий критическую область вида
\begin{equation}
\fr{\sqrt {k}\,\bar X_k}{S_k} > c_{\alpha,k}\,,
\label{e4.7}
\end{equation}
где
$$
S_k^2 = \fr{1}{k-1} \sum\limits_{i=1}^k \left(X_i - \bar X_k\right)^2
$$
и~критическое значение~$c_{\alpha,k}$ выбрано так, чтобы этот критерий 
имел уровень значимости~$\alpha$.
В~работе~\cite[формула~(5.6)]{1-ben} для функции мощности этого критерия получено 
асимптотическое разложение вида

\noindent
\begin{multline*}
\hspace*{-2mm}\beta_k(\theta) = \Phi\!\left(\!\sqrt {k}\, \theta \left(1 - 
\fr{u_{1-\alpha}^2}{4k}\right) - u_{1-\alpha}\!\right) +
 O\left(k^{-2}\right) ={}
\\
\hspace*{-3mm}{}= \Phi\left(\sqrt {k}\, \theta \left(1 - \fr{u_{1-\alpha}^2}{4k}\right) 
- u_{1-\alpha} +  O\left(k^{-2}\right)\right)\,,\!\!
%\label{e4.8}
\end{multline*}
причем это равенство выполнено равномерно по~$\theta$. Приравнивая 
выражение~$\beta_{k_n}(\theta)$ к~$\beta^*_n(\theta)$, 
найдем асимптотический дефект критерия Стьюдента~(\ref{e4.7}) 
относительно равномерно наиболее мощного критерия~(\ref{e4.3}):
\begin{align*}
\fr{n}{k_n} &= 1 - \fr{u_{1-\alpha}^2}{2k_n} +  O\left(k_n^{-2}\right)\,;\\
d_n &= k_n - n \to d = \fr{u_{1-\alpha}^2}{2}\,,\enskip n \to \infty\,.
%\label{e4.9}
\end{align*}
Предположим, что исходная выборка имеет случайный объем~$N_n$, и~рассмотрим 
аналог критерия Стьюдента, отвергающий гипотезу~${\bf H}_0$, если (см.~(\ref{e4.7}))
\begin{equation}
\fr{\sqrt {N_n}\, \bar X_{N_n}}{S_{N_n}} > c_{\alpha,N_n}\,.
\label{e4.10}
\end{equation}
Тогда нетрудно видеть, что этот критерий также имеет размер~$\alpha$ и~мощность
\begin{multline*}
{\sf P}_\theta \left(\fr{\sqrt {N_n}\, \bar X_{N_n}}{S_{N_n}} > 
c_{\alpha,N_n}\right) = {\sf E} \beta_{N_n}(\theta) ={}
\\
{}= {\sf E} \Phi\left(\!\sqrt N_n \theta \left(1 - \fr{u_{1-\alpha}^2}{4N_n}\right) - 
u_{1-\alpha}\right) +
 O\left({\sf E} N_n^{-2}\!\right)={}
\\
{}= {\sf E} \Phi\left(\sqrt N_n\, \theta \left(\!1 - \fr{u_{1-\alpha}^2}{4N_n}\!\right) - 
u_{1-\alpha} + O\left({\sf E} N_n^{-2}\right)\right).\hspace*{-5.3pt}
%\label{e4.11}
\end{multline*}
Для нахождения дефектов критериев, основанных на выборках случайного 
объема, рассмотрим в~качестве меры качества критериев с~критическими 
областями~(\ref{e4.3}) и~(\ref{e4.7}) соответственно величины \mbox{вида}
\begin{align}
\pi_n^* &= \left(\Phi^{-1}\left(\beta_n^*(\theta)\right) + u_{1-\alpha}\right)^2 =
n \theta^2\,; \label{e4.12}
\\
\pi_n &= \left(\Phi^{-1}\left(\beta_n(\theta)\right) + u_{1-\alpha}\right)^2 ={}\notag\\
&\hspace*{8mm}{}=
\left(\sqrt n\, \theta \left(1 - \fr{u_{1-\alpha}^2}{4n}\right) + 
 O\left(n^{-2}\right)\right)^2\,.
 \label{e4.13}
\end{align}
В качестве меры качества критериев с~критическими областями~(\ref{e4.5}) и~(\ref{e4.10}), 
основанных на выборках случайного объема~$N_n$, рассмотрим соответственно 
усредненные величины~(\ref{e4.12}) и~(\ref{e4.13}):
\begin{align*}
{\sf E} \pi_{N_n}^* &= {\sf E} \left(\Phi^{-1}(\beta_{N_n}^*(\theta)) + 
u_{1-\alpha}\right)^2 =\theta^2 {\sf E} N_n\,;
\\
{\sf E} \pi_n &= {\sf E} \left(\Phi^{-1}\left(\beta_{N_n}(\theta)\right) + 
u_{1-\alpha}\right)^2 ={}\\
&\hspace*{7mm}{}=
{\sf E} \left(\sqrt N_{n}\, \theta \left(1 - \fr{u_{1-\alpha}^2}{4N_n}\right) +
O\left(N_{n}^{-2}\right)\right)^2\,.
\end{align*}
При условии
$$
N_n = n
$$
эти выражения соответственно приобретают вид:
\begin{align}
{\sf E} \pi_{N_n}^* &= \theta^2 n\,;
\label{e4.14}
\\
{\sf E} \pi_{N_n}& ={}\notag\\
&\hspace*{-11.5mm}{}= \theta^2 n - \fr{\theta^2 u_{1-\alpha}^2}{2} + 
\fr{\theta^2 u_{1-\alpha}^2}{16} {\sf E} N_n^{-1}\! +  O\!\left({\sf E} N_{n}^{-3/2}\!\right).\!\!\!
\label{e4.15}
\end{align}
Теперь для нахождения асимптотического дефекта из равенства
$$
{\sf E} \pi_{N_{k_n}} = {\sf E} \pi_{N_n}^*
$$
получаем соотношение:
\begin{equation*}
d_n = k_n - n = \fr{u_{1-\alpha}^2}{2} +  O\left({\sf E} N_{k_n}^{-1}\right)\,,\enskip
k_n \to \infty\,.
%\label{e4.16}
\end{equation*}
Из этой формулы получаем следующую теорему.

\smallskip

\noindent
\textbf{Теорема~4.1.}\ \textit{
Пусть исходные наблюдения $X_1,\ldots,X_n$ имеют нормальное распределение 
с~параметрами $\theta\hm \in \mathbb{R}$ и~$\sigma^2\hm > 0$. Предположим, что 
с.в.~$N_{n}$ не зависит от наблюдений $X_1,\ldots,X_n$ и~удовлетворяет 
сле\-ду\-ющим условиям}:
$$
N_n = n\,,\  {\sf E} N_{n}^{-1} \to 0\,,\  n \to \infty\,.
$$
\textit{Тогда асимптотический дефект критерия}~(\ref{e4.10}) 
\textit{относительно критерия}~(\ref{e4.5}) 
(\textit{с~мерами качества}~(\ref{e4.15}) \textit{и}~(\ref{e4.14})) \textit{равен}
$$
d = \fr{u_{1-\alpha}^2}{2}\,.
$$


\noindent
\textbf{Следствие~4.1.}\ Пусть с.в.~$N_n$ удовлетворяет условиям 
теоремы~3.3 (т.\,е.\ имеет соответственно пуассоновское, биномиальное или 
геометрическое распределение), тогда условия теоремы~4.1 выполнены и~асимптотический 
дефект равен
$$
d = \fr{u_{1-\alpha}^2}{2}\,.
$$

Найдем теперь дефекты критериев с~критическими областями~(\ref{e4.5}) и~(\ref{e4.10}) относительно 
критериев, основанных на выборках неслучайного объема и~имеющих соответственно 
критические области вида~(\ref{e4.3}) и~(\ref{e4.7}). При этом используем меры качества 
(см.~(\ref{e4.12}), (\ref{e4.13})):
\begin{equation}
\pi_n^* = n \theta^2\,,\quad 
{\sf E} \pi_{N_n}^* = \theta^2 {\sf E} N_{n}\,;
\label{e4.17}
\end{equation}

\vspace*{-12pt}

\noindent
\begin{multline}
\pi_n = n \theta^2 - \fr{\theta^2u_{1-\alpha}^2}{2} + 
\fr{\theta^2u_{1-\alpha}^4}{16n} +  O\left(n^{-3/2}\right)\,,
\\
{\sf E} \pi_n = \theta^2 {\sf E} N_n - \fr{\theta^2u_{1-\alpha}^2}{2} + 
\fr{\theta^2u_{1-\alpha}^4 {\sf E} N_n^{-1}}{16} +{}\\
{}+  O\left({\sf E} N_n^{-3/2}\right)\,.
\label{e4.18}
\end{multline}
Теперь, находя~$k_n$ из равенств
$$
{\sf E} \pi_{N_{k_n}}^* = \pi_n^*\,;\quad  {\sf E} \pi_{N_{k_n}} = \pi_n\,,
$$
получаем следующую теорему.

\smallskip

\noindent
\textbf{Теорема~4.2.}\ \textit{Пусть исходные наблюдения $X_1,\ldots,X_n$ 
имеют нормальное 
распределение с~параметрами $\theta\hm \in \mathbb{R}$ и~$\sigma^2\hm > 0$. 
Предположим, что с.в.~$N_{n}$ не зависит от 
наблюдений $X_1,\ldots,X_n$ и~удовлетворяет сле\-ду\-ющим условиям}:
$$
N_n = n\,,\ {\sf E} N_{n}^{-3/2} \to 0\,,\  n \to \infty\,.
$$
\textit{Тогда асимптотический дефект критерия}~(\ref{e4.5}) 
\textit{относительно критерия}~(\ref{e4.3}) 
(\textit{с~мерой качества}~(\ref{e4.17})) \textit{равен}
$$
d_n = k_n - n = 0\,, \  d = \lim\limits_{n\to\infty} d_n = 0\,.
$$
\textit{Асимптотический дефект критерия}~(\ref{e4.10}) 
\textit{относительно критерия}~(\ref{e4.7}) 
(\textit{с~мерой качества}~(\ref{e4.18})) \textit{равен}
\begin{multline*}
d_n = k_n - n = \fr{u^4_{1-\alpha}}{16}\left({\sf E} N^{-1}_{k_n} - \fr{1}{n}\right) +{}\\
{}+
 O\left({\sf E} N_{k_n}^{-3/2} + n^{-3/2}\right)\,, \ 
 d = \lim\limits_{n\to\infty} d_n = 0\,.
\end{multline*}


\noindent
\textbf{Следствие~4.2.}\ {Пусть с.в.~$N_n$ удовлетворяет 
условиям теоремы~$3.3$ (т.\,е.\ имеет соответственно пуассоновское, биномиальное 
или гео\-мет\-ри\-че\-ское распреде\-ле\-ние), тогда условия теоремы~$4.2$ выполнены и~соответствующие 
асимптотические дефекты равны нулю. Более того, в~этих случаях справедливы 
соответственно асимптотические представления для величины~$d_n$ 
в~случае критериев}~(\ref{e4.10}) {и}~(\ref{e4.7})
\begin{align*}
d_n& = k_n - n = \fr{u^4_{1-\alpha}}{16n^2} + o\left(n^{-2}\right)\,;
\\
d_n &= k_n - n = \fr{u^4_{1-\alpha}(m-1)}{16mn^2} +
o \left(n^{-2}\right)\,;
\\
d_n &= k_n - n = \fr{u^4_{1-\alpha}}{16}\left(\fr{\log n}{n - 1} - 
\fr{1}{n}\right) +  o\left(\fr{\log n}{n}\right)\,.
\end{align*}


\section{Заключение}

Таким образом, в~работе рассмотрены проблемы точечного оценивания, 
доверительного оценивания и~проверки гипотез, касающиеся неизвестных\linebreak 
параметров распределения. При этом предполага\-ется, что статистики, 
используемые в~статистиче\-ских процедурах, основаны на выборках случайного\linebreak объема. 

Рассмотрена также байесовская постановка задачи оценивания. 
С~по\-мощью понятия дефекта проведено асимптотическое сравнение качества таких 
процедур. 

Получены явные формулы для асимптотического дефекта, имеющего смысл
необходимого добавочного числа наблюдений. 

Подроб\-но рассмотрены случаи, когда 
случайный объем выборки имеет распределение Пуассона, биномиальное и~геометрическое 
распределение. Случай, когда исходные наблюдения распределены по нормальному 
закону с~неизвестными параметрами, приводится в~качестве примера, иллюстрирующего 
общие результаты. Находятся асимптотические дефекты доверительных интервалов, а также 
некоторых статистических критериев (например, критерия Стьюдента), построенных по 
выборкам случайного объема.

{\small\frenchspacing
 {%\baselineskip=10.8pt
 \addcontentsline{toc}{section}{References}
 \begin{thebibliography}{9}
\bibitem{2-ben} %1
\Au{Бенинг В.\,Е.} О~дефекте некоторых оценок, основанных на выборках случайного
объема~// Вестник Тверского гос. ун-та. Сер.:
Прикладная математика, 2015. Вып.~1. С.~5--14.

\bibitem{3-ben} %2
\Au{Бенинг В.\,Е., Королев~В.\,Ю., Савушкин~В.\,А.} О~сравнении качества оценок, 
основанных на выборках случайного объема, с~помощью понятия дефект~//
Статистические методы оценивания и~проверки гипотез, 2015. Вып.~26. С.~26--42.

 \bibitem{4-ben}  %3
\Au{Леман Э.} Теория точечного оценивания~/
Пер. с англ.~--- М.: Наука, 1994. 444~c.


\bibitem{1-ben} 
\Au{Hodges~J.\,L., Lehmann~E.\,L.} Deficiency~//
Ann. Math. Stat., 1970. Vol.~41. No.\,5. P.~783--801.

\bibitem{5-ben} 
\Au{Bening V.\,E.} Asymptotic theory of testing statistical hypotheses: 
Efficient statistics, optimality, power loss, and deficiency.~--- 
Utrecht: VSP, 2000. 277~p.

\bibitem {6-ben}
\Au{Крамер Г.} Математические методы статистики~/
Пер. с англ.~--- М.: Мир, 1976. 648~с.
(\Au{Cramer~H.}  
\textit{Mathematical methods of statistics}. Princeton, NJ,
USA: Princeton University Press, 1946. 656~p.)
 \end{thebibliography}

 }
 }

\end{multicols}

\vspace*{-3pt}

\hfill{\small\textit{Поступила в~редакцию 15.10.16}}

\vspace*{8pt}

%\newpage

%\vspace*{-24pt}

\hrule

\vspace*{2pt}

\hrule

%\vspace*{8pt}


\def\tit{CALCULATION OF~THE~ASYMPTOTIC DEFICIENCY\\ OF~SOME~STATISTICAL PROCEDURES\\
 BASED ON~SAMPLES WITH~RANDOM SIZES}

\def\titkol{Calculation of~the~asymptotic deficiency of~some 
statistical procedures based on samples with random sizes}

\def\aut{V.\,E.~Bening$^{1,2}$}

\def\autkol{V.\,E.~Bening}

\titel{\tit}{\aut}{\autkol}{\titkol}

\vspace*{-9pt}

\noindent
$^1$Department of Mathematical Statistics, Faculty of Computational Mathematics 
and Cybernetics,\linebreak
$\hphantom{^1}$M.\,V.~Lomonosov Moscow State University, 
1-52~Leninskiye Gory, GSP-1, Moscow 119991, Russian\linebreak
$\hphantom{^1}$Federation

\noindent
$^2$Institute of Informatics Problems, Federal Research Center 
``Computer Science and Control'' of the Russian\linebreak
$\hphantom{^1}$Academy of Sciences, 
44-2~Vavilov Str., Moscow 119333, Russian Federation



\def\leftfootline{\small{\textbf{\thepage}
\hfill INFORMATIKA I EE PRIMENENIYA~--- INFORMATICS AND
APPLICATIONS\ \ \ 2016\ \ \ volume~10\ \ \ issue\ 4}
}%
 \def\rightfootline{\small{INFORMATIKA I EE PRIMENENIYA~---
INFORMATICS AND APPLICATIONS\ \ \ 2016\ \ \ volume~10\ \ \ issue\ 4
\hfill \textbf{\thepage}}}

\vspace*{3pt}



\Abste{Statistical regularities of information flows in contemporary 
communication, computational and other information systems are characterized 
by the presence of the so-called ``heavy tails.'' The random character of the intensity 
of the flow of informative events results in that the available sample size 
(traditionally, this is the number of observations registered within a~certain time 
interval) is random. The randomness of the sample size cruciall
changes the 
asymptotic properties of the statistical procedures (e.\,g., estimators). The 
present paper consists of a~number of applications of the deficiency concept, i.\,e., 
the number of additional observations required by the less
effective procedure and, 
thereby, provides a~basis for deciding whether or not the price is too high. The 
deficiency was introduced by Hodges and Lehmann in~1970. In the paper, asymptotic 
deficiencies of statistical procedures based on samples with random sizes are 
considered. Three examples concerning testing statistical hypotheses, point, and 
confidence estimation are presented.}

\KWE{confidence set; statistical hypothesis; asymptotic deficiency; sample with 
random size; Poisson distribution; binomial distribution}


\DOI{10.14357/19922264160404}  

%\vspace*{-9pt}

\Ack
\noindent
This work was financially supported by the Russian Science Foundation 
(grant No.\,14-11-00364).


%\vspace*{3pt}

  \begin{multicols}{2}

\renewcommand{\bibname}{\protect\rmfamily References}
%\renewcommand{\bibname}{\large\protect\rm References}

{\small\frenchspacing
 {%\baselineskip=10.8pt
 \addcontentsline{toc}{section}{References}
 \begin{thebibliography}{9}
 
 \bibitem{5-ben-1} %1
\Aue{Bening, V.\,E.} 2015. O~defekte nekotorykh otsenok, osnovannykh na
vyborkakh sluchaynogo ob"ema [On deficiencies of some estimators based  
on samples with random sizes].
\textit{Vestnik Tverskogo gos. un-ta. Ser. 
Prikladnaya matematika} [Herald of Tver State University, Ser. Applied Mathematics] 
1:5--14.

\bibitem{6-ben-1} %2
\Aue{Bening, V.\,E., V.\,Yu.~Korolev, and V.\,A.~Savushkin.} 
2015. O~sravnenii kachestva otsenok, osnovannykh na vyborkakh sluchaynogo ob"ema, 
s~pomoshch'yu ponyatiya defekt [On the comparison of statistical estimators
based on samples with random sizes with the help of deficiency concept]. 
\textit{Statisticheskie metody otsenivaniya i~proverki gipotez} [Statistical Methods
of Estimation and Testing Hypotheses] 26:26--42.

\bibitem{3-ben-1} %3
\Aue{Lehmann, E.\,L., and G.~Casella}. 1998. 
\textit{Theory of point estimation}. 2nd ed. New York, NY: Springer Verlag. 470~p.

\bibitem{1-ben-1} %4
\Aue{Hodges, J.\,L., and E.\,L.~Lehmann}. 1970. 
Deficiency. \textit{Ann. Math. Stat.} 41(5):783--801.

\bibitem{4-ben-1} %5
\Aue{Bening, V.\,E.} 2000. \textit{Asymptotic theory of testing statistical hypotheses: 
Efficient statistics,
optimality, power loss, and deficiency}. Utrecht: VSP. 277~p.

\bibitem{2-ben-1}
\Aue{Cramer, H.} 1946. 
\textit{Mathematical methods of statistics}. Princeton, NJ: Princeton University Press.
656~p.





\end{thebibliography}

 }
 }

\end{multicols}

\vspace*{-3pt}

\hfill{\small\textit{Received October 15, 2016}}

\Contrl

\noindent
\textbf{Bening Vladimir E.} (b.\ 1954)~---
Doctor of Science in physics and mathematics, professor, Department of Mathematical 
Statistics, Faculty of Computational Mathematics and Cybernetics, 
Faculty of Computational Mathematics and Cybernetics, 
M.\,V.~Lomonosov Moscow State University, 1-52~Leninskiye Gory, GSP-1, 
Moscow 119991, Russian Federation; senior scientist,
Institute of Informatics Problems, Federal Research Center ``Computer Science 
and Control'' of the Russian Academy of Sciences, 44-2~Vavilov Str., Moscow 119333,  
Russian Federation; \mbox{bening@yandex.ru}
\label{end\stat}


\renewcommand{\bibname}{\protect\rm Литература} 