\documentclass[10pt]{book}
\usepackage[utf8]{inputenc}

\usepackage{latexsym,amssymb,amsfonts,amsmath,indentfirst,shapepar,%fleqn,%
picinpar,shadow,floatflt,enumerate,multicol,colortbl,moreverb,ipi}

\usepackage{rotating}
\usepackage{mathrsfs}
\usepackage[noend]{algorithmic}
\usepackage{ulem}
%\usepackage{graphicx}
%\usepackage{algorithm2e}

\input{epsf}

%\nofiles

%\includeonly{avtor} %+pdf
%\includeonly{obchak,avtor}
%\includeonly{pred}      %
%\includeonly{podgot-rus,podgot-eng}  %+pdf
%\includeonly{ocherk} %+
%\includeonly{nekrol} %+
%\includeonly{ipi-ind} 
%\includeonly{toc-rus, toc-en} 
%\includeonly{toc-en} 


%\includeonly{gaidamaka} %1pdf
%\includeonly{korr+kor+zeif2} %2+pdf
%\includeonly{kor-zeif-kor} %3pdf
%\includeonly{bening} %4pdf
%\includeonly{chertok} %5pdf
%\includeonly{konovalov} %6pdf
%\includeonly{kudr}  %7pdf
%\includeonly{shnurkov} %8pdf
%\includeonly{melnikov}  %9pdf
%\includeonly{grusho}  %10pdf
%\includeonly{zatsar} %11pdf
%\includeonly{kol-list} %12pdf
%\includeonly{karasikov}  %13pdf
%\includeonly{zaliznyk} %14pdf

%\includeonly{toc-rus, toc-en}
%\includeonly{obchak} %,toc-en}

%\includeonly{rekl}
%\includeonly{rekl-1}
%\includeonly{reshal}  %
%\includeonly{eng-index}
%\includeonly{cover3}

\usepackage{acad}
%\usepackage{courier}
\usepackage{decor}
\usepackage{newton}
\usepackage{pragmatica}
\usepackage{zapfchan}
\usepackage{petrotex}
\usepackage{bm}                     % полужирные греческие буквы
\usepackage{upgreek}                % прямые греческие буквы
\usepackage{eufrak}
\usepackage{verbatim}

\renewcommand{\bottomfraction}{0.99}
\renewcommand{\topfraction}{0.99}
\renewcommand{\textfraction}{0.01}

\setcounter{secnumdepth}{1} %здесь - 3 + chapter = 4

\arraycolsep=1.5pt

%\usepackage[pdftex]{graphicx}

%\usepackage{oz}

%NEW COMMANDS


\renewcommand*{\hm}[1]{#1\nobreak\discretionary{}%
            {\hbox{$\mathsurround=0pt #1$}}{}} %% Дублирует знаки операций
                               %при переносе в формуле (перед знаком, который
                               %надо продублировать ставится команда \hm)

%\newcommand{\endproof}{\hfill$\Box$}
%\renewcommand{\r}{\mathbb{R}}
\newcommand{\I}{{\rm I\hspace{-0.7mm}I}}
%\newcommand{\Ikl}{{\tt{1}}\hspace*{-1.44mm}\mathtt{1}}
\newcommand{\Ik}{\mbox{{\small \tt {1}}\hspace{-1.3mm}{\tt 1}}}
\newcommand{\argmin}{\mathop{\mathrm{arg}\,\mathrm{min}}}
\newcommand{\argmax}{\mathop{\mathrm{arg}\,\mathrm{max}}}
%\newcommand{\capr}{\mathop{\cap\,}}
%\newcommand{\cupr}{\mathop{\cup\,}}
%\def\argmin{\mathop{arg\,min}}

\def\vrp{\varphi}
\def\prt{\partial}
\def\mm{{\sf M}}
\def\modnop#1{\mathop{#1}\limits_{n}}
\def\eam{\mathbin{{\mathop{=}\limits^{\mathrm{def}}}}}
\def\dey#1#2{#1 (#2)}
\def\deyc#1#2{#1 \cdot  #2}
\def\ra#1{\;\mathop{\to}\limits^{#1}\;}
\def\raz#1{\;\mathop{\longrightarrow}\limits^{\!\!\!#1}\;}
\def\ral#1{\;\mathop{\longrightarrow}\limits^{#1}\;}

\newcommand{\Nor}{\mathcal{N}}
\newcommand{\T}{\mathbb{T}}
\newcommand{\Z}{\mathbb{Z}}



\newcommand{\il}[2]{\int\limits_{#1}^{#2}}%интеграл с пределами #1 и #2

\def\sm2{\mathop {\sum\limits^{n^\Theta}\sum\limits^{n^\Theta}}}
\def\sss{\sum\limits}
\def\tr{,\,\ldots\,,\,}
\def\rk{\right]}
\def\lk{\left[}
\def\rf{\right\}}
\def\lf{\left\{}
\def\lv{\,\left\vert}
\def\rv{\right\vert\,}
\def\iii{\int\limits}
\def\iin{\int\limits_{-\infty}^\infty}
\def\rrv{\right\vert}


\def\ee{{\cal E}}
\def\ww{{\cal W}}
\def\yy{{\cal Y}}
\def\vv{{\cal V}}

\newcommand{\R}{\mathbb R}
\newcommand{\E}{\mathbb E}
\newcommand{\N}{\mathbb N}

\renewcommand{\P}{\mathbb{P}}

\newcommand{\h}{{\bf H}}
\newcommand{\p}{{\sf P}}  % вероятность

\newcommand{\e}{{\sf E}}  % мат. ожидание
\newcommand{\D}{{\sf D}}  % дисперсия
\newcommand{\eps}{\varepsilon}
\newcommand{\vp}{{\mathbf p}}
\newcommand{\vz}{{\mathbf z}}
\newcommand{\vx}{{\mathbf x}}
\newcommand{\vf}{{\mathbf f}}
\newcommand{\F}{{\mathcal F}}
\def\ap{{\mathrm{ЭР}}}
\newcommand{\ud}{\Delta_n} %uniform ditance
\newcommand{\nud}{\Delta_n(x)}
\renewcommand{\Re}{\mathrm{Re}\,}

\newcommand{\abs}[1]{\left\vert#1\right\vert}

\newcommand{\norm}[1]{\left\Vert#1\right\Vert}
\def\da{(\Delta_t,A)}

\newcommand{\corr}{\mathrm{corr}}

\newcommand{\cov}{\mathrm{cov}}
\newcommand{\Expect}{\mathbb{E}}

\def\w{\omega}
\def\W{\Omega}

\def\inh{\int\limits_{nh}^{(n+1)h}}

\def\sumin{\sum_{i=1}^N}


\def\bxt{(Y,t)}
\def\xt{(y,t)}

\def\ovth{{\fr{\tau-nh}{h}}}
\def\ov{\overline}
\def\tm{\tilde m}
\def\tl{\tilde\lambda}
\def\tB{\widetilde B}
\def\tb{\tilde b}
\def\ld{\ldots}
\def\cd{\cdots}


\DeclareMathOperator{\sign}{sign}

%\newcommand{\gr}{{\geqslant}}


\newcommand{\g}{\mbox{\textit{g}}}

\renewcommand{\la}{\lambda}
\newcommand{\si}{\sigma}
\newcommand{\alp}{\alpha}

%\newcommand{\pto}{\stackrel{P}{\longrightarrow}} % сходимость по веpоятности

\newcommand{\eqd}{\stackrel{\mathrm{d}}{=}} % равенство по pаспpеделению
\newcommand{\eqdelta}{\stackrel{\Delta}{=}} % равенство по pаспpеделению

\def\be#1{\begin{equation}\label{#1}}
\def\ee{\end{equation}}
\def\re#1{(\ref{#1})}

\def\bn{\begin{enumerate}}
\def\en{\end{enumerate}}
\def\bi{\begin{itemize}}
\def\ei{\end{itemize}}
%\def\i{\item}

%\newcommand{\kp}{\kappa}
%\def\Q{{\cal Q}} \def\H{{\cal H}}
%\newcommand{\bet}{\beta_{2+\delta}}


%\newtheorem{definition}{Определение}
%\renewcommand{\thedefinition}{\arabic{definition}.}
%END NEW COMMANDS

%\renewcommand{\baselinestretch}{1.2}

%\pagestyle{myheadings}

\setlength{\textwidth}{167mm}      % 122mm
\setlength{\textheight}{658pt}
%\setlength{\textheight}{635.6pt}
\setlength{\columnsep}{4.5mm}

\setcounter{secnumdepth}{4}

%\addtolength{\headheight}{2pt}
%\addtolength{\headsep}{-2mm}

\addtolength{\topmargin}{-7mm}  % for printing


%\hoffset=-30mm  % From Yap
\hoffset=-23mm  % From Acrobat

%\voffset=0mm % From Yap
\voffset=-5mm   % From Acrobat

%\addtolength{\evensidemargin}{-2.5mm} % for printing
%\addtolength{\oddsidemargin}{2.5mm}  % for printing

\addtolength{\evensidemargin}{-12mm} % for printing
\addtolength{\oddsidemargin}{8mm}  % for printing

%\renewcommand{\thefootnote}{\fnsymbol{footnote}}
%\renewcommand{\thefootnote}{\arabic{footnote}}
\renewcommand{\figurename}{\protect\bf Рис.}
\renewcommand{\tablename}{\protect\bf Таблица}

\newcommand{\Caption}[1]{\caption{\protect\small %\baselineskip=2.5ex
#1}}

\renewcommand{\thefigure}{\arabic{figure}}
\renewcommand{\thetable}{\arabic{table}}
\renewcommand{\theequation}{\arabic{equation}}
\renewcommand{\thesection}{\arabic{section}}

\renewcommand{\contentsname}{СОДЕРЖАНИЕ}
\newcommand{\fr}[2]{\displaystyle\frac{\displaystyle #1\mathstrut}{\displaystyle #2\mathstrut}}

%\renewcommand{\thefootnote}{\fnsymbol{footnote}}
%\newcommand{\g}{\mbox{\textit{g}}}

%\newcommand{\Caption}[1]{\caption{\protect\small\baselineskip=2ex #1}}
\newcounter{razdel}
\setcounter{razdel}{0}


\newcommand{\titel}[4]{%
\

\vspace*{5pt}

\ifodd\therazdel {\raggedright\noindent\Large\textrm\textbf
 \lineskip .75em
  \baselineskip=3.2ex #1 \par}
\vskip 1em {\noindent\large\textrm\textbf #2 \par}
\addcontentsline{toc}{subsection}{{\textrm\textbf #3}\protect\newline #1}
\def\rightheadline{\underline{\noindent\hbox to \textwidth{\hfill\small\textrm{#4}
%\hfill \large\bf\thepage
}}}
\def\leftheadline{\underline{\noindent\parbox{\textwidth}{
%\raggedleft\large\bf\thepage \hfill
\small\textit{#3}\hfill}}}
\def\leftfootline{\small{\textbf{\thepage}
\hfill ИНФОРМАТИКА И ЕЁ ПРИМЕНЕНИЯ\ \ \ том~10\ \ \ выпуск 4\ \ \ 2016}
}%
 \def\rightfootline{\small{ИНФОРМАТИКА И ЕЁ ПРИМЕНЕНИЯ\ \ \ том~10\ \ \ выпуск~4\ \ \ 2016
\hfill \textbf{\thepage}}}
\vskip 2em \setcounter{figure}{0}
\setcounter{table}{0}
\setcounter{equation}{0}
\setcounter{section}{0}
\setcounter{subsection}{0}
\setcounter{subsubsection}{0}
\setcounter{footnote}{0}
\setcounter{razdel}{0}
%\end{flushleft}
\else {
 \raggedright\noindent\Large\textrm\textbf
 \lineskip .75em
\baselineskip=3.2ex #1 \par} \vskip 1em
%\begin{flushleft}
{\noindent\large\textrm\textbf #2 \par}
\addcontentsline{toc}{subsection}{{\textrm\textbf #3}\protect\newline #1}
\def\rightheadline{\underline{\noindent\hbox to \textwidth{\hfill\small\textrm{#4}
%\hfill \large\bf\thepage
}}}
\def\leftheadline{\underline{\noindent\parbox{\textwidth}{%\raggedleft\large\bf\thepage \hfill
\small\textit{#3}\hfill}}}
\def\leftfootline{\small{\textbf{\thepage}
\hfill ИНФОРМАТИКА И ЕЁ ПРИМЕНЕНИЯ\ \ \ том~10\ \ \ выпуск~4\ \ \ 2016}
}%
 \def\rightfootline{\small{ИНФОРМАТИКА И ЕЁ ПРИМЕНЕНИЯ\ \ \ том~10\ \ \ выпуск~4\ \ \ 2016
\hfill \textbf{\thepage}}} \vskip 2em \setcounter{figure}{0}
\setcounter{table}{0} \setcounter{equation}{0} \setcounter{section}{0}
\setcounter{subsection}{0} \setcounter{subsubsection}{0}
\setcounter{footnote}{0}
%\end{flushleft}
\fi}

\newcommand{\titelr}[2]{%
\

\vspace*{5pt}

\ifodd\therazdel {\raggedright\noindent%\Large\textrm\textbf
 \lineskip .75em
  \baselineskip=3.2ex #1 \par}
\vskip 1em {\noindent\normalsize\textrm\textbf #2 \par}
\else {
 \raggedright\noindent\Large\textrm\textbf
 \lineskip .75em
\baselineskip=3.2ex #1 \par} \vskip 1em
%\begin{flushleft}
{\noindent\large\textrm\textbf #2 \par
%\noindent\normalsize\textrm\textbf #2 \par
} \fi}

\newcommand{\titele}[5]{%
\

%\vspace*{5pt}

\ifodd\therazdel {\raggedright\noindent\large
\textrm\textbf
 \lineskip .75em
%  \baselineskip=3.2ex
#1 \par}
\vskip .5em {\noindent\large\textrm\textbf #2 \par}
\vskip .5em
 {\noindent\textrm #3 \par}
\addcontentsline{toc}{subsection}{{\textrm\textbf #1}\protect\newline #2}
\def\rightheadline{\underline{\noindent\hbox to \textwidth{\hfill\small\textrm{#4}
%\hfill \large\bf\thepage
}}}
\def\leftheadline{\underline{\noindent\parbox{\textwidth}{
%\raggedleft\large\bf\thepage \hfill
\small\textrm{#5}\hfill}}}
\def\leftfootline{\small{\textbf{\thepage}
\hfill ИНФОРМАТИКА И ЕЁ ПРИМЕНЕНИЯ\ \ \ том~10\ \ \ выпуск~4\ \ \ 2016}
}%
 \def\rightfootline{\small{ИНФОРМАТИКА И ЕЁ ПРИМЕНЕНИЯ\ \ \ том~10\ \ \ выпуск~4\ \ \ 2016
\hfill \textbf{\thepage}}} \vskip 1em \setcounter{figure}{0}
\setcounter{table}{0} \setcounter{equation}{0} \setcounter{section}{0}
\setcounter{subsection}{0} \setcounter{subsubsection}{0}
\setcounter{footnote}{0} \setcounter{razdel}{0}
%\end{flushleft}
\else {
 \raggedright\noindent\large
 \textrm\textbf
 \lineskip .75em
%\baselineskip=3.2ex
#1 \par} \vskip .5em
%\begin{flushleft}
{\noindent\large\textrm\textbf #2 \par} \vskip .5em
 {\noindent\textrm #3 \par}
\addcontentsline{toc}{subsection}{{\textrm\textbf #1}\protect\newline #2}
\def\rightheadline{\underline{\noindent\hbox to \textwidth{\hfill\small\textrm{#4}
%\hfill \large\bf\thepage
}}}
\def\leftheadline{\underline{\noindent\parbox{\textwidth}{%\raggedleft\large\bf\thepage \hfill
\small\textrm{#5}\hfill}}}
\def\leftfootline{\small{\textbf{\thepage}
\hfill ИНФОРМАТИКА И ЕЁ ПРИМЕНЕНИЯ\ \ \ том~10\ \ \ выпуск~4\ \ \ 2016}
}%
 \def\rightfootline{\small{ИНФОРМАТИКА И ЕЁ ПРИМЕНЕНИЯ\ \ \ том~10\ \ \ выпуск~4\ \ \ 2016
\hfill \textbf{\thepage}}} \vskip 1em \setcounter{figure}{0}
\setcounter{table}{0} \setcounter{equation}{0} \setcounter{section}{0}
\setcounter{subsection}{0} \setcounter{subsubsection}{0}
\setcounter{footnote}{0}
%\end{flushleft}
\fi}

\def\Abst#1{
\begin{center}\small\nwt
\parbox{150mm}{%\baselineskip=2.5ex
\textbf{Аннотация:}\ \
%\hspace*{\parindent}
#1}
\end{center}}
\def\Abste#1{
\begin{center}\small\nwt
\parbox{150mm}{%\baselineskip=2.5ex
\textbf{Abstract:}\ \
%\hspace*{\parindent}
#1}
\end{center}}

\def\DOI#1{
\begin{center}\small\nwt
\parbox{150mm}{%\baselineskip=2.5ex
\textbf{DOI:}\ \
%\hspace*{\parindent}
#1}
\end{center}}

\def\Abstend#1{
\begin{center}\small\nwt
\parbox{150mm}{%\baselineskip=2.5ex
%\hspace*{\parindent}
#1}
\end{center}}


\def\KW#1{
\begin{center}\small\nwt
\parbox{150mm}{%\baselineskip=2.5ex
\textbf{Ключевые слова:}\ \ #1}
\end{center}}

\def\KWE#1{
\begin{center}\small\nwt
\parbox{150mm}{%\baselineskip=2.5ex
\textbf{Keywords:}\ \ #1}
\end{center}}


\def\KWN#1{
%\begin{center}
%\small
%\parbox{150mm}\end{center}
}

\newcommand{\Avtors}[1]{%\smallskip
%\vspace*{.5pt}
\hangindent=23pt\noindent
%\nwt
{\bfseries#1}\
}


\renewcommand{\thesubsection}{\thesection.\arabic{subsection}\hspace*{-5pt}}
\renewcommand{\thesubsubsection}{\thesubsection\hspace*{5pt}.\arabic{subsubsection}\hspace*{-3pt}}

\newcommand{\Ack}{\section*{\protect\rmfamily Acknowledgments}\noindent}
\newcommand{\Contr}{\section*{\protect\rmfamily Contributors}\noindent}
\newcommand{\Contrl}{\section*{\protect\rmfamily Contributor}\noindent}

\makeindex


\begin{document}
\Rus

\nwt
%\ptb


%\renewcommand{\contentsname}{\protect\Large\bf Содержание}

\setcounter{tocdepth}{2}

%\tableofcontents

\renewcommand{\bibname}{\protect\rmfamily Литература}
  \def\Au#1{{\it #1}}
    \def\Aue#1{{#1}}

%\newcommand{\No}{№}
  \newcommand{\tg}{\,\mathrm{tg}\,}
    \newcommand{\ctg}{\,\mathrm{ctg}\,}
  \newcommand{\arctg}{\,\mathrm{arctg}\,}

\def\forallb{\mathop{\forall}}
\def\cupb{\mathop{\cup}}
\def\existsb{\mathop{\exists}}


\newpage
\addtocounter{razdel}{1}
%\def\razd{РЕГУЛИРУЕМЫЙ ЭЛЕКТРОПРИВОД ДЛЯ ЭЛЕКТРОЭНЕРГЕТИКИ}


\setcounter{page}{2}

%   { %\Large  
   { %\baselineskip=16.6pt
   
   \vspace*{-48pt}
   \begin{center}\LARGE
   \textit{Предисловие}
   \end{center}
   
   %\vspace*{2.5mm}
   
   \vspace*{25mm}
   
   \thispagestyle{empty}
   
   { %\small 

    
Вниманию читателей журнала <<Информатика и её применения>> предлагается 
очередной тематический выпуск <<Вероятностно-статистические методы и 
задачи информатики и информационных технологий>>. Предыдущие тематические 
выпуски журнала по данному направлению вышли в 2008~г.\ (т.~2, вып.~2), 
в 2009~г.\ (т.~3, вып.~3) и в 2010~г.\ (т.~4, вып.~2). 

Статьи, собранные в данном журнале, посвящены разработке новых вероятностно-статистических 
методов, ориентированных на применение к решению конкретных задач информатики и информационных 
технологий, а также~--- в ряде случаев~--- и других прикладных задач. Проблематика, охватываемая 
публикуемыми работами, развивается в рамках научного сотрудничества между Институтом проблем 
информатики Российской академии наук (ИПИ РАН) и Факультетом вычислительной математики и 
кибернетики Московского государственного университета им.\ М.\,В.~Ломоносова в ходе работ 
над совместными научными проектами (в том числе в рамках функционирования 
Научно-образовательного центра <<Вероятностно-статистические методы анализа рисков>>). 
Многие из авторов статей, включенных в данный номер журнала, являются активными участниками 
традиционного международного семинара по проблемам устойчивости стохастических моделей, 
руководимого В.\,М.~Золотаревым и В.\,Ю.~Королевым; регулярные сессии этого семинара 
проводятся под эгидой МГУ и ИПИ РАН (в 2011~г.\ указанный семинар проводится в октябре 
в Калининградской области РФ). 

Наряду с представителями ИПИ РАН и МГУ в число авторов данного выпуска журнала входят 
ученые из Научно-исследовательского института системных исследований РАН, Института 
проблем технологии микроэлектроники и особочистых материалов РАН, Института 
прикладных математических исследований Карельского НЦ РАН, Московского 
авиационного института, Вологодского государственного педагогического университета, 
НИИММ им.\ Н.\,Г.~Чеботарева, Казанского государственного университета, Дебреценского 
университета (Венгрия).

Несколько статей выпуска посвящено разработке и применению стохастических методов и 
информационных технологий для решения различных прикладных задач. В~работе В.\,Г.~Ушакова 
и О.\,В.~Шестакова рассмотрена задача определения вероятностных характеристик случайных 
функций по распределениям интегральных преобразований, возникающих в задачах эмиссионной 
томографии. В~статье Д.\,О.~Яковенко и М.\,А.~Целищева рассмотрены некоторые вопросы 
математической теории риска и предложен новый подход к диверсификации инвестиционных 
портфелей. Работа И.\,А.~Кудрявцевой и А.\,В.~Пантелеева посвящена построению и 
исследованию математической модели, описывающей динамику сильноионизованной плазмы. 
В~статье П.\,П.~Кольцова изучается качество работы ряда алгоритмов сегментации изображений. 
Статья А.\,Н.~Чупрунова и И.~Фазекаша посвящена вероятностному анализу числа без\-оши\-бочных 
блоков при помехоустойчивом кодировании; получены усиленные законы больших чисел для указанных 
величин.

В данном выпуске традиционно присутствует тематика, весьма активно разрабатываемая в течение 
многих лет специалистами ИПИ РАН и МГУ,~--- методы моделирования и управления для 
информационно-телекоммуникационных и вычислительных систем, в частности методы 
теории массового обслуживания. В~статье А.\,И.~Зейфмана с соавторами рассматриваются 
модели обслуживания, описываемые марковскими цепями с непрерывным временем в случае 
наличия катастроф. В~работе М.\,М.~Лери и И.\,А.~Чеплюковой рассматриваются случайные 
графы Интернет-типа, т.\,е.\ графы, степени вершин которых имеют степенные распределения; 
такие задачи находят применение при исследовании глобальных сетей передачи данных. 
Работа Р.\,В.~Разумчика посвящена исследованию систем массового обслуживания специального 
вида~--- с отрицательными заявками и хранением вытесненных заявок.

Ряд статей посвящен развитию перспективных теоретических 
вероятностно-статистических методов, которые находят широкое применение в различных 
задачах информатики и информационных технологий. В~работе В.\,Е.~Бенинга, А.\,К.~Горшенина 
и В.\,Ю.~Королева рассмотрена задача статистической проверки гипотез о числе компонент 
смеси вероятностных распределений, приводится конструкция асимптотически наиболее мощного 
критерия. Результаты этой работы найдут применение в ряде прикладных задач, использующих 
математическую модель смеси вероятностных распределений (в информатике, моделировании 
финансовых рынков, физике турбулентной плазмы и~т.\,д.). В~статье В.\,Ю.~Королева, 
И.\,Г.~Шевцовой и С.\,Я.~Шоргина строится новая, улучшенная оценка точности нормальной 
аппроксимации для пуассоновских случайных сумм; как известно, указанные случайные суммы 
широко используются в качестве моделей многих реальных объектов, в том числе в информатике, 
физике и других прикладных областях. Работа В.\,Г.~Ушакова и Н.\,Г.~Ушакова посвящена 
исследованию ядерной оценки плотности распределения; эти результаты могут применяться, 
в част\-ности, при анализе трафика в телекоммуникационных системах. Серьезные приложения 
в статистике могут получить результаты работы О.\,В.~Шестакова, в которой доказаны оценки 
скорости сходимости распределения выборочного абсолютного медианного отклонения к нормальному 
закону. 

\smallskip

Редакционная коллегия журнала выражает надежду, что данный тематический  выпуск 
будет интересен специалистам в области теории вероятностей и математической статистики 
и их применения к решению задач информатики и информационных технологий.
     
     %\vfill 
     \vspace*{20mm}
     \noindent
     Заместитель главного редактора журнала <<Информатика и её 
применения>>,\\
     директор ИПИ РАН, академик  \hfill
     \textit{И.\,А.~Соколов}\\
     
     \noindent
     Редактор-составитель тематического выпуска,\\
     профессор кафедры математической статистики факультета\\
      вычислительной математики и кибернетики МГУ им.\ М.\,В.~Ломоносова,\\
     ведущий научный сотрудник ИПИ РАН,\\ 
доктор физико-математических наук \hfill
      \textit{В.\,Ю.~Королев}
     
     } }
     }

\def\stat{gaidamaka}

\def\tit{МЕТОД РАСЧЕТА ХАРАКТЕРИСТИК ИНТЕРФЕРЕНЦИИ
ДВУХ ВЗАИМОДЕЙСТВУЮЩИХ УСТРОЙСТВ
В~БЕСПРОВОДНОЙ ГЕТЕРОГЕННОЙ СЕТИ$^*$}

\def\titkol{Метод расчета характеристик интерференции
двух взаимодействующих устройств
в~беспроводной гетерогенной сети}

\def\aut{Ю.\,В.~Гайдамака$^1$, А.\,К.~Самуйлов$^2$}

\def\autkol{Ю.\,В.~Гайдамака, А.\,К.~Самуйлов}

\titel{\tit}{\aut}{\autkol}{\titkol}

{\renewcommand{\thefootnote}{\fnsymbol{footnote}} \footnotetext[1]
{Работа выполнена при финансовой поддержке РФФИ (проекты 14-07-00090
и~15-07-03051).}}


\renewcommand{\thefootnote}{\arabic{footnote}}
\footnotetext[1]{Российский университет дружбы народов, ygaidamaka@sci.pfu.edu.ru}
\footnotetext[2]{Российский университет дружбы народов; Технологический университет г.\ Тампере, Финляндия,
aksamuylov@gmail.com}


\Abst{Одним из показателей качества функционирования современных беспроводных сетей
является отношение сигнала к~сумме интерференции и шума (SINR, Signal to Interference plus
Noise Ratio) в~беспроводных каналах связи. Поскольку значение этой характеристики
существенно зависит от расстояния между интерферирующими устройствами, задача оценки
значения SINR часто сводится к~вычислению длины одной из сторон треугольника,
в~вершинах которого находятся взаимодействующие устройства. В~данной статье решается
задача нахождения математического ожидания и~среднеквадратического отклонения
отношения сигнал/интерференция пары взаимодействующих устройств в достаточно общих
предположениях о~распределении случайных величин (с.в.)\ расстояний между
интерфери\-ру\-ющи\-ми устройствами. Предложенный метод может быть использован
в~качестве основы для анализа интерференции в~гетерогенной сети с~применением различных
беспроводных технологий, включая анализ беспроводных взаимодействий оконечных
устройств, на которые интерференция оказывает наиболее сильное воздействие.}

\KW{беспроводная сеть; LTE; интерференция; SINR; взаимодействие устройств; D2D}

\DOI{10.14357/19922264150102}


\vskip 14pt plus 9pt minus 6pt

\thispagestyle{headings}

\begin{multicols}{2}

\label{st\stat}

\section{Постановка задачи}

  В современных беспроводных сетях, построенных на базе технологии LTE
(Long Term Evolution), оценка интерференции между взаимодействующими
устройствами является одной из основных задач анализа показателей качества
функционирования~[1,~2]. Под интерференцией понимается взаимодействие
сигналов, передаваемых разными\linebreak источниками на одном и~том же канале.
Интерференция вызывает искажение сигнала рас\-смат\-ри\-ва\-емо\-го источника под
воздействием сигнала сторонне\-го источника. В~гетерогенных сетях
беспроводного взаимодействия оконечных устройств D2D
  (device-to-device)~[3], где плот\-ность интерферирующих объектов высока,
интерференция оказывает существенное влияние на принимаемый оконечным
устройством сигнал. При анализе беспроводных взаимодействий устройств
обычно рассматривается несколько источников сигнала (передатчиков),
распределенных на плоскости согласно некоторому закону~[4]. Упрощение
задачи состоит в~том, что, рассмотрев один передатчик и~оценив
характеристики интерференции на соответствующем ему приемном устройстве
(приемнике), можно предположить, что основные показатели будут идентичны
и~для остальных пар <<пе\-ре\-дат\-чик--при\-ем\-ник>>. В~данной статье
решается задача нахождения числовых характеристик отношения
сигнал/ин\-тер\-фе\-рен\-ция пары взаимодействующих устройств.

  Отношение сигнала к сумме интерференции и~шума, SINR,
  является одной из основных характеристик качества канала
  в~беспроводных сетях связи~[5--7]. Отношение сигнала к~сумме интерференции 
и~шума на стороне приемника определяется по следующей формуле:
  \begin{equation}
  \mathrm{SINR} = \fr{S}{\sigma^2 +I}\,,
  \label{e1-gai}
  \end{equation}
где $S$~--- мощность принимаемого сигнала от соответствующего
передатчика; $\sigma^2$~--- мощность шума; $I$~--- мощность принимаемого
сигнала от интерферирующих передатчиков. Согласно линейной модели~[4]
\begin{equation}
S=gl^{-\alpha}\,,
\label{e2-gai}
\end{equation}
где $g$~--- базовая мощность сигнала передатчика, соответствующего
рассматриваемому приемнику; $l$~--- расстояние между передатчиком
и~приемником; $\alpha$~--- коэффициент потерь (path loss exponent),
принимающий значение от~2 (при условии прямой видимости) до~6 (в~худшем
случае). Величина~$I$ в~знаменателе формулы~(1) соответствует суммарной
мощности сигнала от всех интерферирующих передатчиков, где каждое
слагаемое имеет вид~(2). Заметим, что принцип повторного использования
частот (frequency reuse) в~беспроводных сетях связи поколения 4G (4th
Generation) позволяет назначать одну и~ту же единицу ресурса сети (например,
один и~тот же ресурсный блок LTE) нескольким парам взаимодействующих
устройств, если интерференция не превосходит определенного стандартами
уровня.

  Рассмотрим случай, когда несколько принимающих устройств (приемников)
и~одно передающее устройство (передатчик), образующие кластер,
расположены на плоскости внутри круга радиуса~$r_0$, причем передатчик
расположен в центре круга. Такой кластер образуется, например, при
проведении интерактивного занятия преподавателя с учениками, когда можно
предположить, что передатчик располагается в центре круга, а приемники
равномерно распределены внутри круга. Для передачи данных на каждую пару
взаимодействующих устройств внутри кластера планировщиком распределения
радиоресурсов в беспроводной сети 4G назначается по одному ресурсному
блоку LTE, и тогда сигналы взаимодействующих пар не интерферируют друг
с~другом. Но если в соседнем помещении также проходит интерактивное
занятие и там использованы те же ресурсные блоки, то пары из соседних
кластеров, использующие один и тот же ресурсный блок, будут создавать
помехи друг другу. Сведем задачу к анализу взаимодействия двух пар
устройств в двух кластерах, как показано на рис.~1.



  Пару взаимодействующих устройств, для которой будем рассчитывать
показатели качества канала, назовем целевой, а соответствующую ей пару
устройств обозначим TR$_0\hm= \langle \mathrm{Tx}_0, \mathrm{Rx}_0\rangle$.
Остальные пары, которые создают помехи целевой паре 
$\mathrm{TR}_0$,\linebreak\vspace*{-12pt}
\begin{center}  %fig1
\vspace*{8pt}
\mbox{%
 \epsfxsize=77.569mm
 \epsfbox{gai-1.eps}
 }
\end{center}

\noindent
{{\figurename~1}\ \ \small{Схема взаимодействия интерферирующих устройств}}


%\vspace*{9pt}


\addtocounter{figure}{1}


\noindent
 обозначим $\mathrm{TR}_i\hm= \langle
\mathrm{Tx}_i, \mathrm{Rx}_i\rangle$ и~будем называть их интерферирующими. Расстояние
между Rx$_i$ и~Tx$_i$ обозначим $R_i$, а~расстояние между Tx$_0$ и~Tx$_i$
обозначим~$U_i$. Мощность интерферирующего сигнала от пары TR$_i$
является функцией расстояния между приемником Rx$_0$ из целевой пары и
интерферирующим передатчиком~Tx$_i$, которое обозначим~$D_i$. Угол
между прямой, соединяющей целевые передатчик~Tx$_0$ и~приемник~Rx$_0$,
и~прямой, соеди\-ня\-ющей передатчики~Tx$_0$ и~Tx$_i$,
обозначим~$\gamma_i$.

  Рассмотрим систему двух кластеров, показанную на рис.~1. В~условиях
отсутствия шума и~одинаковой базовой мощности~$g$ сигналов обоих
передатчиков искомой характеристикой является отношение
  сигнал/ин\-тер\-фе\-рен\-ция SIR для приемника~Rx$_0$, вычисляемое по
формуле:
  \begin{equation}
\mathrm{SIR}=\left( \fr{D_1}{R_0}\right)^{\alpha}\,.
  \label{e3-gai}
  \end{equation}

  Будем считать, что $R_0$, $U_i$ и~$\gamma_i$ являются
с.в.\ с~заданными функциями распределения. Задача состоит
в~нахождении числовых характеристик с.в.~SIR. Для
решения задачи в следующем разделе статьи предлагается метод нахождения
совместной плотности распределения с.в.~$R_0$ и~$D_i$, что позволяет
вычислять начальные моменты ${\sf E}[\mathrm{SIR}^n]$ с.в.~SIR.

\vspace*{-6pt}

\section{Метод расчета отношения сигнал/интерференция}

%\vspace*{-2pt}

  Как видно из формулы~(3), с.в.~SIR пропорциональна с.в.~$D_1$, которая,
в~свою очередь, зависит от с.в.~$R_0$. В~этом случае для нахождения
характеристик с.в.~SIR необходимо найти совместное распределение
с.в.~$R_0$ и~$D_1$.

  Введем обозначения $\xi_1{:=} R_0$, $\xi_2 {:=} U_1$, $\xi_3 {:=}
\gamma_1$, $\eta_1 {:=} D_1$. Тогда $w_{\xi_1,\xi_2,\xi_3}(x_1,x_2,x_3) {:=}$\linebreak
${=:}\;f_{R_0, U_1, \gamma_1}(x_1,x_2,x_3)$~--- совместная плот\-ность
распределения с.в.~$R_0$, $U_1$ и~$\gamma_1$, а~$W_{\xi_1,\eta_1}(x_1,y_1)
{:=} f_{R_0, D_1}(x_1,y_1)$~--- искомое совместное распределение с.в.~$R_0$
и~$D_1$. По теореме косинусов с.в.~$\eta_1$ является функцией с.в.~$\xi_1$,
$\xi_2$ и~$\xi_3$:
  \begin{equation}
  \eta_1=\sqrt{\xi_1^2+\xi_2^2-2\xi_1\xi_2\cos \xi_3}\,.
  \label{e4-gai}
  \end{equation}

  Следуя~\cite{8-gai, 9-gai}, введя вспомогательную
переменную $\eta_2\hm=\xi_3$, искомое распределение можно найти по
следующей формуле:
  \begin{multline}
W_{\xi_1, \eta_1} (y_1,y_2) ={}\\
{}=\sum\limits_{i=1}^2
\int\limits_{\mathrm{Y}_{3,j}}\!\!\! w_{\xi_1,\xi_2,\xi_3}\left(
y_1,\varphi_i(y_1,y_2,y_3),y_3\right) \times{}\\[-6pt]
{}\times
\left\vert \fr{\partial \varphi_j(y_1,y_2,y_3)}
{\partial y_2}\right\vert\,dy_3\,,
  \label{e5-gai}
  \end{multline}
где $\varphi_j$~--- обратное преобразование правой части формулы~(\ref{e4-gai})
относительно~$\xi_2$:
\begin{align*}
\varphi_1(y_1,y_2,y_3) &= y_1\cos y_3 +\sqrt{y_2^2-y_1^2+y_1^2\cos^2 y_3}\,;\\
\varphi_2(y_1,y_2,y_3) &= y_1\cos y_3 -\sqrt{y_2^2-y_1^2+y_1^2\cos^2 y_3}\,.
\end{align*}

  В формуле~(\ref{e5-gai}) области значений Y$_{3,j}$ переменной~$y_3$ для
$j$-й вет\-ви обратного преобразования определяются системой неравенств:
  \begin{equation}
  \left.
  \begin{array}{c}
  \varphi_j(y_1,y_2,y_3)\geq0\,;\\[6pt]
  y_1\geq 0\,;\\[6pt]
  y_2\geq 0\,;\\[6pt]
  0\leq y_3\leq 2\pi\,.
  \end{array}
  \right\}
  \label{e6-gai}
  \end{equation}

  Решая систему~(\ref{e6-gai}), нетрудно убедиться, что для первой ветви
обратного преобразования  $\mathrm{Y}_{3,1}\hm= \mathrm{Y}_{3,1}^1\cup
\mathrm{Y}_{3,1}^2\cup \mathrm{Y}_{3,1}^3$, где
  \begin{align}
  \hspace*{-2mm}\mathrm{Y}_{3,1}^1 &=\begin{cases}
  0\leq y_2\leq y_1;\\
  0\leq y_3\leq \fr{1}{2}\,\mathrm{arccos}\,\left( \fr{y_1^2-
2y_2^2}{y_1^2}\right);\end{cases}
  \label{e7-1-gai}
\\
\hspace*{-2mm}\mathrm{Y}_{3,1}^2 &= \begin{cases}
  0\leq y_2\leq y_1;\\
  2\pi -\fr{1}{2}\,\mathrm{arccos}\left( \fr{y_1^2-2y_2^2}{y_1^2}\right) \leq
y_3\leq 2\pi;
  \end{cases}\!\!\!\!\!
  \label{e7-2-gai}
  \\
\hspace*{-2mm}\mathrm{Y}_{3,1}^3 &= \begin{cases}
  y_2\geq y_1;\\
  0\leq y_3\leq 2\pi,
  \end{cases}\!\!\!\!\!\!\!\!\!
  \label{e7-3-gai}
  \end{align}
а для второй ветви  $\mathrm{Y}_{3.2}=\mathrm{Y}_{3,2}^1\cup
\mathrm{Y}_{3,2}^2$, где
\begin{align}
\label{e8-1-gai}
\hspace*{-2mm}\mathrm{Y}_{3,2}^1 &= \begin{cases}
0\leq y_2\leq y_1\,;\\
0\leq y_3\leq \fr{1}{2}\,\mathrm{arccos}\left( \fr{y_1^2-2y_2^2}{y_1^2}\right);
\end{cases}
\\
\hspace*{-2mm}\mathrm{Y}_{3,2}^2 &=\begin{cases}
0\leq y_2\leq y_1\,;\\
2\pi -\fr{1}{2}\,\mathrm{arccos} \left( \fr{y_1^2-2y_2^2}{y_1^2}\right) \leq y_3\leq
2\pi.\!\!\!\!\!\!\!\!
\end{cases}
\label{e8-2-gai}
\end{align}

  Таким образом, получена формула для вычисления совместной плотности
с.в.~$R_0$ и~$D_1$:
  \begin{multline}
  W_{\xi_1,\eta_1}(y_1,y_2) ={}\\
  {}=\sum\limits_{i=1}^2 \int\limits_{\mathrm{Y}_{3,i}}
\fr{w_{\xi_1,\xi_2,\xi_3} (y_1,\varphi_i(y_1,y_2,y_3),y_3) y_2} {\sqrt{y_2^2-
y_1^2+y_1^2\cos^2 y_3}}\,dy_3\,,
  \label{e9-gai}
  \end{multline}
где $\mathrm{Y}_{3,j}$ вычисляются по
формулам~(\ref{e7-1-gai})--(\ref{e8-2-gai}).

  В следующем разделе приведен пример численного анализа
с~использованием формул~(\ref{e7-1-gai})--(\ref{e9-gai}).

\section{Пример численного анализа}

  В рассматриваемом примере предложенный выше метод использован для
расчета начальных моментов ${\sf E}[\mathrm{SIR}^n]$ отношения сигнал/интерференция,
которые определяются следующей формулой:
  \begin{multline}
{\sf   E}[\mathrm{SIR}^n] ={}\\
{}=\int\limits_{0\leq y_1\leq r_0} \int\limits_{y_2\geq0} \left(
\fr{y_2}{y_2}\right)^{n\alpha} W_{\xi_1,\eta_1}(y_1,y_2)\,dy_2dy_1\,.
  \label{e10-gai}
  \end{multline}

  Рассматривается случай, когда целевой приемник Rx$_0$ находится внутри
круга единичного \mbox{радиуса} ($r_0\hm=1$), в центре которого расположен
передат\-чик~Tx$_0$, а~интерферирующий передатчик~Tx$_1$~--- в~кольце
вокруг передатчика~Tx$_0$ с~внутренним радиусом~$r_0$ и~внешним
радиусом~$h_0$ (рис.~2).

\begin{center}  %fig2
\vspace*{8pt}
\mbox{%
 \epsfxsize=77.111mm
 \epsfbox{gai-2.eps}
 }


\noindent
{{\figurename~2}\ \ \small{Пример взаимодействия двух устройств}}

\end{center}


\vspace*{9pt}


\addtocounter{figure}{1}


%\noindent


  Тогда с.в.~$R_0$ расстояния от целевого передатчика Tx$_0$ до
соответствующего ему приемника Rx$_0$ и~с.в.~$U_1$ расстояния от целевого
передатчика~Tx$_0$ до интерферирующего передатчика~Tx$_1$ имеют
распределения
  \begin{alignat*}{2}
  f_{R_0}(r) &= 2r\,,&\quad 0&\leq r\leq1\,;\\
  f_{U_1}(u) &= \fr{2u}{h_0^2-1}\,,&\quad 1&\leq u\leq h_0\,.
  \end{alignat*}
Будем считать, что с.в.\ угла~$\gamma_1$ равномерно распределена на отрезке
$[0,\,2\pi]$, а~коэффициент потерь в~формуле~(2) принимает значение
$\alpha\hm=2$. Приняты условные единицы измерения: например, расстояние
между взаимодействующими устройствами может измеряться в~метрах,
а~величина SIR~--- в~децибелах.

  По формулам~(\ref{e7-1-gai})--(\ref{e10-gai}) рассчитано математическое
ожидание отношения сигнал/ин\-тер\-фе\-рен\-ция ${\sf E}[\mathrm{SIR}]$, представленное в
таблице в зависимости от радиуса внешней границы кольца, внутри которого
распределены интерферирующие передатчики. В~таблице также показаны
значения математического ожидания расстояния  ${\sf E}[U_1]$ от целевого
передатчика~Tx$_0$ до интерферирующего передатчика~Tx$_1$.

%  \begin{table*}\small
  \begin{center}
  \begin{tabular}{|c|c|c|}
  \multicolumn{3}{p{48mm}}{Математическое ожидание величины~SIR}\\
  \multicolumn{3}{c}{\ }\\[-5pt]
  \hline
\ \ \ \ $h_0$\ \ \ \ &\ \ \ \ ${\sf E}[U_1]$\ \ \ \ &${\sf E}[\mathrm{SIR}]$\\
\hline
2&1,56&4,84985\\
3&2,17&7,41701\\
4&2,8\hphantom{9}&9,54562\\
5&3,44&11,30286\hphantom{9}\\
\hline
\end{tabular}
\end{center}
%\end{table*}

\begin{center}  %fig3
\vspace*{18pt}
\mbox{%
 \epsfxsize=77.754mm
 \epsfbox{gai-3.eps}
 }
 \end{center}


\noindent
{{\figurename~3}\ \ \small{Числовые характеристики отношения сигнал/ин\-тер\-фе\-рен\-ция: \textit{1}~---
${\sf E}[\mathrm{SIR}]$; \textit{2}~--- $\sigma_{\mathrm{SIR}}$}}

\vspace*{18pt}

  Также были рассчитаны математическое ожидание ${\sf E}[\mathrm{SIR}]$
  и~среднеквадратическое \mbox{отклонение} $\sigma_{\mathrm{SIR}}\hm= \sqrt{{\sf E}[\mathrm{SIR}^2]-
{\sf E}[\mathrm{SIR}]^2}$ отношения сигнал/ин\-тер\-фе\-рен\-ция, показанные на рис.~3 в
зави\-си\-мости от математического ожидания расстояния ${\sf E}[U_1]$ между
целевым передатчиком~Tx$_0$ и~интерферирующим передатчиком~Tx$_1$. Из
таблицы и~графиков видно, что с~ростом расстояния между целевым
и~интерферирующим передатчиком обе \mbox{числовые} характеристики отношения
сиг\-нал/ин\-тер\-фе\-рен\-ция растут, поскольку мощность интерферирующего сигнала
убывает. Вычисления проводились с~использованием встроенных средств
пакета программ Wolfram Mathematica~[10].



\section{Заключение}

  В настоящей статье метод преобразования с.в.\ применен для
анализа основной характеристики качества функционирования беспроводных
сетей, а~именно: отношения сигнал/ин\-тер\-фе\-рен\-ция при заданных
распределениях расстояний между интерферирующими устройствами.
Приведенный пример показывает, что чис\-лен\-ный анализ является достаточно
трудоемким даже в простейших предположениях о~распределении исходных
с.в., а~для оценки характеристик интерференции в~условиях
наличия в~беспроводной сети нескольких источников интерференции требуется
разработка приближенных методов и~имитационных моделей, как это сделано,
например, в~[11]. Задача с несколькими источниками интерференции
в~беспроводных гетерогенных сетях взаимодействующих устройств
представляется особенно актуальной ввиду быст\-ро\-го развития сетей 4G
и~принятия в~ближайшем будущем стандартов для беспроводных сетей 5G~[12].
{\looseness=1

}

  \bigskip

Авторы выражают благодарность проф.\ К.\,Е.~Самуйлову за
плодотворное обсуждение и ценные советы.


{\small\frenchspacing
 {%\baselineskip=10.8pt
 \addcontentsline{toc}{section}{References}
 \begin{thebibliography}{99}
\bibitem{1-gai}
\Au{Гайдамака~Ю.\,В., Ефимушкина~Т.\,В., Самуйлов~А.\,К., Самуйлов~К.\,Е.} Задачи
оптимального планирования межуровневого интерфейса в беспроводных сетях~//
Информатика и~её применения, 2012. Т.~6. Вып.~3. С.~75--81.
\bibitem{2-gai}
\Au{Basharin G.\,P., Gaidamaka Yu.\,V., Samouylov~K.\,E.} Mathematical theory of teletraffic and
its application to the analysis of multiservice communication of next generation networks~//
Autom. Control Comp. Sci., 2013. Vol.~47. No.\,2. P.~62--69.
\bibitem{3-gai}
\Au{Andreev S., Pyattaev A., Johnsson~K., Galinina~O., Koucheryavy~Y.} Cellular traffic
offloading onto network-assisted device-to-device connections~// IEEE Commun. Mag.,
2014. Vol.~52. No.\,4. {\sf http://ieeexplore.ieee.org/\linebreak xpl/tocresult.jsp?isnumber=6807935}.
\bibitem{4-gai}
\Au{Baccelli F., Blaszczyszyn B.} Stochastic geometry and wireless networks. Vol.~I: Theory.~---
Boston: NoW Publs. Inc., 2009. 164~p.


\bibitem{6-gai} %5
\Au{Erturk M.\,C., Mukherjee S., Ishii~H., Arslan~H.} Distributions of transmit power and SINR in
device-to-device networks~// IEEE Commun. Lett., 2013. Vol.~17. No.\,2. {\sf
http://ieeexplore.ieee.org/xpl/tocresult.jsp?isnumber=\linebreak 6472443}.

\bibitem{7-gai} %6
\Au{Kim M., Han Y., Yoon~Y., Chong~Y., Lee~H.} Modeling of adjacent channel interference in
heterogeneous wireless networks~// IEEE Commun. Lett., 2013. Vol.~17. No.\,9. {\sf
http://ieeexplore.ieee.org/xpl/tocresult.jsp?isnumber=\linebreak 6604524}.

\bibitem{5-gai} %7
\Au{Andrews J.\,G., Singh S., Ye~Q., Lin~X., Dhillon~H.\,S.} An overview of load balancing in
hetnets: Old myths and open problems~// IEEE Wirel. Commun., 2014. Vol.~21. No.\,2.
{\sf http://ieeexplore.ieee.org/xpl/tocresult.\linebreak jsp?isnumber=6812279}.


\bibitem{8-gai}
\Au{Левин Б.\,Р.} Теоретические основы статистической радиотехники.~--- 3-е изд.~--- М.:
Радио и связь, 1989. 656~с.
\bibitem{9-gai}
\Au{Mardia K., Jupp P.} Directional statistics.~--- Wiley Press, 1999. 441~p.
\bibitem{10-gai}
Wolfram Mathematica: Программное обеспечение для технических вычислений. {\sf
http://www.wolfram.\linebreak com/mathematica}.
\bibitem{11-gai}
\Au{Гайдамака Ю.\,В., Печинкин А.\,В., Разумчик~Р.\,В., Самуйлов~А.\,К., Самуйлов~К.\,Е.,
Соколов~И.\,А., Сопин~Э.\,С., Шоргин~С.\,Я.} Распределение времени выхода из множества
состояний перегрузки в системе $M\vert M\vert 1\vert \langle L,H\rangle \vert \langle
H,R\rangle$ с~гистерезисным управлением нагрузкой~// Информатика и~её применения,
2013. Т.~7. Вып.~4. С.~20--33.
\bibitem{12-gai}
\Au{Tehrani M., Uysal M., Yanikomeroglu~H.} Device-to-device communication in 5G cellular
networks: Challenges, solutions, and future directions~// IEEE Commun. Mag., 2014.
Vol.~52. No.\,5. {\sf http://ieeexplore. ieee.org/xpl/tocresult.jsp?isnumber=6815882}.
 \end{thebibliography}

 }
 }

\end{multicols}

\vspace*{-3pt}

\hfill{\small\textit{Поступила в редакцию 20.01.15}}

%\newpage

\vspace*{12pt}

\hrule

\vspace*{2pt}

\hrule

%\vspace*{12pt}

\def\tit{METHOD FOR CALCULATING NUMERICAL
CHARACTERISTICS OF~TWO DEVICES INTERFERENCE
FOR~DEVICE-TO-DEVICE COMMUNICATIONS
IN~A~WIRELESS HETEROGENEOUS NETWORK}

\def\titkol{Method for calculating numerical
characteristics of~two devices interference
for~D2D communications
in~a~wireless %heterogeneous
network}

\def\aut{Yu.~Gaidamaka$^1$ and A.~Samuylov$^{1,2}$}

\def\autkol{Yu.~Gaidamaka and A.~Samuylov}

\titel{\tit}{\aut}{\autkol}{\titkol}

\vspace*{-9pt}

 \noindent
$^1$Peoples' Friendship University of Russia,
Applied Probability and Informatics Department,
6~Miklukho-Maklaya\linebreak
$\hphantom{^1}$Str., Moscow 117198, Russian Federation

\noindent
$^2$Tampere University of Technology,
Department of Electronics and Communications Engineering,
10 Korkeak-\linebreak
$\hphantom{^1}$oulunkatu,  Tampere 33720, Finland


\def\leftfootline{\small{\textbf{\thepage}
\hfill INFORMATIKA I EE PRIMENENIYA~--- INFORMATICS AND
APPLICATIONS\ \ \ 2015\ \ \ volume~9\ \ \ issue\ 1}
}%
 \def\rightfootline{\small{INFORMATIKA I EE PRIMENENIYA~---
INFORMATICS AND APPLICATIONS\ \ \ 2015\ \ \ volume~9\ \ \ issue\ 1
\hfill \textbf{\thepage}}}

\vspace*{3pt}


\Abste{In wireless networks, one of the key performance metrics is the signal to noise ratio, SINR. As this metric
highly depends on the distance between the interfering devices, the problem of SINR estimation is often reduced to the
calculation of a triangle's side length, where the vertices represent the interacting devices. This paper addresses the
problem of calculating the numerical characteristics of the signal to interference ratio for a pair of interfering devices
determined by the known distributions of distances between the entities in question. The proposed method can be used
as a basis for analyzing heterogeneous networks, including the analysis of
device-to-device (D2D) communications as one of
the interference-limited cases.}

\KWE{wireless network; LTE; interference; SINR; D2D}




\DOI{10.14357/19922264150102}

\Ack
\noindent
The reported study was partially supported by the Russian Foundation for Basic
Research,  research projects Nos.\,14-07-00090 and
15-07-03051.



%\vspace*{3pt}

  \begin{multicols}{2}

\renewcommand{\bibname}{\protect\rmfamily References}
%\renewcommand{\bibname}{\large\protect\rm References}



{\small\frenchspacing
 {%\baselineskip=10.8pt
 \addcontentsline{toc}{section}{References}
 \begin{thebibliography}{99}
\bibitem{1-gai-1}
\Aue{Gaidamaka, Yu.\,V., T.\,V. Efimushkina, A.\,K.~Samuylov, and K.\,E.~Samouylov}. 2012.
Zadachi optimal'nogo planirovaniya mezhurovnevogo interfeysa v besprovodnykh setyakh
[Cross-layer optimization planning problems in wireless networks]. \textit{Informatika i~ee
Primeneniya}~--- \textit{Inform. Appl.} 6(3):75--81.
\bibitem{2-gai-1}
\Aue{Basharin, G.\,P., Yu.\,V. Gaidamaka, and K.\,E.~Samouylov}. 2013. Mathematical theory of
teletraffic and its application to the analysis of multiservice communication of next generation
networks. \textit{Autom. Control Comp. Sci.} 47 (2):62--69.
\bibitem{3-gai-1}
\Aue{Andreev, S., A. Pyattaev, K.~Johnsson, O.~Galinina, and Y.~Koucheryavy}. 2014. Cellular
traffic offloading onto network-assisted device-to-device connections. \textit{IEEE
Commun. Mag.} 52(4). Available at: {\sf
http://ieeexplore.ieee.\linebreak org/xpl/tocresult.jsp?isnumber=6807935} (accessed January~10, 2015).
\bibitem{4-gai-1}
\Aue{Baccelli, F., and B. Blaszczyszyn.} 2009. \textit{Stochastic geometry and wireless networks}.
Vol.~I: Theory. Boston: NoW Publs. Inc. 164~p.



January~10, 2015). %5
\bibitem{6-gai-1}
\Aue{Erturk, M.\,C., S. Mukherjee, H.~Ishii, and H.~Arslan}. 2013. Distributions of transmit
power and SINR in device-to-device networks. \textit{IEEE Commun. Lett.} 17(2).
Available at: {\sf http://ieeexplore.ieee.org/xpl/tocresult.jsp?isnumber=\linebreak 6472443} (accessed
January~10, 2015).
\bibitem{7-gai-1} %6
\Aue{Kim, M., Y. Han, Y.~Yoon, Y.~Chong, and H.~Lee}. 2013. Modeling of adjacent channel
interference in heterogeneous wireless networks. \textit{IEEE Commun. Lett.} 17(9).
Available at: {\sf http://ieeexplore.ieee.org/\linebreak xpl/tocresult.jsp?isnumber=6604524} (accessed
January~10, 2015).

\bibitem{5-gai-1} %7
\Aue{Andrews, J.\,G., S. Singh, Q.~Ye, X.~Lin, and H.\,S.~Dhillon}. 2014. An overview of load
balancing in hetnets: Old myths and open problems. \textit{IEEE Wirel. Commun.} 21(2).
Available at: {\sf http://ieeexplore.ieee.org/\linebreak xpl/tocresult.jsp?isnumber=6812279} (accessed

\bibitem{8-gai-1}
\Aue{Levin, B.\,R.} 1989. \textit{Teoreticheskie osnovy statisticheskoy radiotekhniki} [Theoretical
basis of statistical radiotechnics]. 3rd ed. Moscow: Radio and Communications. 656~p.
\bibitem{9-gai-1}
\Aue{Mardia, K., and P. Jupp}. 1999. \textit{Directional statistics}. 1st ed. Wiley Press. 441~p.
\bibitem{10-gai-1}
Wolfram mathematica: Software for technical computing. [Free access] Available at: {\sf
http://www.wolfram.\linebreak com/mathematica} (accessed December~1, 2014).
\bibitem{11-gai-1}
\Aue{Gaidamaka, Yu.\,V., A.\,V. Pechinkin, R.\,V.~Razumchik, A.\,K.~Samuylov,
K.\,E.~Samouylov, I.\,A.~Sokolov, E.\,S.~Sopin, and S.\,Ya.~Shorgin}. 2013. Raspredelenie
vremeni vykhoda iz mnozhestva sostoyaniy peregruzki v~sisteme $M\vert M\vert 1\vert \langle
L,H\rangle \vert \langle H,R\rangle$ s~gisterezisnym upravleniem nagruzkoy [The distribution of
the return time from the set of overload states to the set of normal load states in a system $M\vert
M\vert 1\vert \langle L,H\rangle \vert \langle H,R\rangle$ with hysteretic load control].
\textit{Informatika i~ee~Primeneniya}~--- \textit{Inform. Appl.} 7(4):20--33.
\bibitem{12-gai-1}
\Aue{Tehrani, M., M. Uysal, and H.~Yanikomeroglu.} 2014. Device-to-device communication in
5G cellular networks: Challenges, solutions, and future directions.
\textit{IEEE Commun. Mag.} 52(5). Available at: {\sf http://ieeexplore.\linebreak ieee.org/xpl/tocresult.jsp?isnumber=6815882}
(accessed January~10, 2015).
\end{thebibliography}

 }
 }

\end{multicols}

\vspace*{-3pt}

\hfill{\small\textit{Received January 20, 2015}}

%\vspace*{-18pt}


\Contr

\noindent
\textbf{Gaidamaka Yuliya V.} (b.\ 1971)~---
Candidate of Science (PhD) in physics and mathematics, associate
professor, Applied Probability and Informatics Department, Peoples' Friendship University of Russia,
6~Miklukho-Maklaya Str., Moscow 117198, Russian Federation;
ygaidamaka@sci.pfu.edu.ru

\vspace*{3pt}

\noindent
\textbf{Samuylov Andrey K.} (b.\ 1988)~---
PhD student, Peoples' Friendship University of Russia, Moscow 117198, Russian
Federation; researcher, Department of Electronics and Communications Engineering,  Tampere
University of Technology, 10 Korkeakoulunkatu, Tampere 33720, Finland;
aksamuylov@gmail.com

\label{end\stat}

\renewcommand{\bibname}{\protect\rm Литература} %1
%\newcommand{\tod}{\stackrel{d}{\longrightarrow}}

\def\stat{korr+kor}

\def\tit{ТЕОРЕМА ПУАССОНА ДЛЯ СХЕМЫ ИСПЫТАНИЙ БЕРНУЛЛИ\\ СО~СЛУЧАЙНОЙ
ВЕРОЯТНОСТЬЮ УСПЕХА\\ И~ДИСКРЕТНЫЙ АНАЛОГ РАСПРЕДЕЛЕНИЯ
ВЕЙБУЛЛА$^*$}

\def\titkol{Теорема Пуассона для схемы испытаний Бернулли со случайной
вероятностью успеха} % и~дискретный аналог распределения Вейбулла}

\def\aut{В.\,Ю.~Королев$^1$, А.\,Ю.~Корчагин$^2$, А.\,И.~Зейфман$^3$}

\def\autkol{В.\,Ю.~Королев, А.\,Ю.~Корчагин, А.\,И.~Зейфман}

\titel{\tit}{\aut}{\autkol}{\titkol}

\index{Королев В.\,Ю.}
\index{Зейфман А.\,И.}
\index{Корчагин А.\,Ю.}
\index{Korolev V.\,Yu.}
\index{Zeifman A.\,I.}
\index{Korchagin A.\,Yu.}


{\renewcommand{\thefootnote}{\fnsymbol{footnote}} \footnotetext[1]
{Работа выполнена при поддержке Российского научного
фонда (проект 14-11-00397).}}


\renewcommand{\thefootnote}{\arabic{footnote}}
\footnotetext[1]{Факультет вычислительной математики и~кибернетики 
Московского государственного 
университета им.\ М.\,В.~Ломоносова; Институт проб\-лем информатики Федерального 
исследовательского центра <<Информатика и~управ\-ле\-ние>> Российской академии наук, 
\mbox{vkorolev@cs.msu.ru}}
\footnotetext[2]{Факультет вычислительной математики и~кибернетики 
Московского государственного университета им.\ М.\,В.~Ломоносова; Институт проб\-лем информатики Федерального 
исследовательского центра <<Информатика и~управ\-ле\-ние>> Российской академии наук,
\mbox{sasha.korchagin@gmail.com}}
\footnotetext[3]{Вологодский государственный университет; Институт проб\-лем 
информатики Федерального исследовательского центра <<Информатика и~управ\-ле\-ние>>
 Российской академии наук; Институт со\-ци\-аль\-но-эко\-но\-ми\-че\-ско\-го 
 развития территорий 
 Российской академии наук, \mbox{a\_zeifman@mail.ru}}
 
 \vspace*{-6pt}

\Abst{Рассматривается задача, связанная с~испытаниями
Бернулли со случайной вероятностью успеха. Сначала в~результате
<<предварительного>> эксперимента определяется значение случайной
величины $\pi\hm\in(0,1)$, которое принимается в~качестве вероятности
успеха в~испытаниях Бернулли. Затем случайная величина $N$
определяется как число успехов в~$k\hm\in\mathbb{N}$ испытаниях
Бернулли с~так определенной ве\-ро\-ят\-ностью успеха~$\pi$. Распределение
случайной величины $N$ называется $\pi$-сме\-шан\-ным биномиальным. 
В~рамках такой схемы испытаний Бернулли со случайной вероятностью
успеха формулируется <<случайный>> аналог классической теоремы
Пуассона для $\pi$-сме\-шан\-ных биномиальных распределений, в~котором
предельным законом оказывается смешанное пуассоновское
распределение. Особое внимание уделено случаю, в~котором смешивающим
распределением является распределение Вейбулла. Соответствующее
смешанное пуассоновское распределение~--- пуас\-сон-вей\-бул\-лов\-ское
распределение~--- предложено в~качестве дискретного аналога
распределения Вейбулла. Обсуждаются некоторые свойства
пу\-ас\-сон-вей\-бул\-лов\-ско\-го распределения. В~частности, показано, что это
распределение является смешанным геометрическим. Предложен
двухэтапный сеточный алгоритм оценивания параметров смешанных
пуассоновских распределений и,~в~част\-ности, пуас\-сон-вей\-бул\-лов\-ско\-го
распределения. Построены статистические оценки верхней границы
сетки. Приведены примеры вычислений по предложенному алгоритму.}

\KW{испытания Бернулли со случайной вероятностью
успеха; смешанное биномиальное распределение; теорема Пуассона;
смешанное пуассоновское распределение; распределение Вейбулла;
пу\-ас\-сон-вей\-бул\-лов\-ское распределение; смешанное геометрическое
распределение; ЕМ-ал\-го\-ритм}

\DOI{10.14357/19922264160402} 

\vspace*{-6pt}


\vskip 8pt plus 9pt minus 6pt

\thispagestyle{headings}

\begin{multicols}{2}

\label{st\stat}



\section{Введение}

Исследование, некоторые результаты которого излагаются в~данной
статье, мотивировано несколькими обстоятельствами. Во-пер\-вых, 
в~последнее время получил серьезное развитие метод прогнозирования
временн$\acute{\mbox{ы}}$х характеристик катастроф в~неоднородных потоках
экстремальных событий (см., например,~\cite{GrigoryevaKorolevSokolov2013}).
 Этот метод можно считать
глубокой модернизацией метода превышений порога (POT-method, POT~---
Peaks Over Threshold). В~рамках этого метода исходный ряд
(маркированный точечный процесс) прореживается таким образом, что
все наблюдения (точки), <<марки>> которых меньше указанного порога,
выбрасываются. При этом, как правило, величина порога определяется
статистически, и~потому вероятность, с~которой очередное
наблюдение отбрасывается, является случайной.

Во-вторых, схема простого прореживания процессов восстановления
(см., например,~\cite{Renyi1956-k, Mogyorodi1971}) предос\-тав\-ля\-ет
естественный подход к~определению \textit{редкого} события, в~рамках
которого это понятие\linebreak связыва\-ется с~хорошо известной конструкцией
пуассоновского процесса, 
характеризующегося показательностью распределения интервалов времени
между событиями (<<восстановлениями>>) в~классе процессов восстановления. Было бы желательно иметь
столь же простой и~основанный на прореживании подход к~конструкции
смешанного пуассоновского процесса. В~рамках такого подхода,
построив смешанный пуассоновский процесс как асимптотическую
аппроксимацию для дважды стохастически прореженных процессов
восстановления, можно надеяться получить дополнительное понимание
структуры многих смежных математических моделей, в~частности
популярных ныне байесовских моделей и~методов, и~более осмысленно
подойти в~выбору соответствующего смешивающего (<<априорного>>)
распределения.

В дальнейшем будет удобнее вести изложение не в~терминах
распределений, а~в~терминах случайных величин (с.в.)\ (предполагая,
что все они заданы на одном вероятностном пространстве
$(\Omega,\mathfrak{A}, {\sf P})$).

Символы $\eqd$ и~$\Longrightarrow$ будут соответственно обозначать
совпадение распределений и~сходимость по распределению.

Функция распределения (ф.р.)\ и~плотность строго устойчивого
распределения с~характеристическим показателем~$\alpha$ и~параметром
формы~$\theta$, определяемого характеристической функцией
$$
\mathfrak{f}_{\alpha,\theta}(t)=
\exp\left\{-|t|^{\alpha}\exp
\left\{-\fr{1}{2}\,i\pi\theta\alpha\mathrm{sign}\,t\right\}\right\},\enskip
t\in\mathbb{R}\,,
$$
где $0<\alpha\le2$, $|\theta|\hm\le\min\{1,{2}/{\alpha}-1\}$, будут
соответственно обозначаться $G_{\alpha,\theta}(x)$ 
и~$g_{\alpha,\theta}(x)$ (см., например,~\cite{Zolotarev1983-k}). Любую
с.в.\ с~ф.р.~$G_{\alpha,\theta}(x)$ будем обозначать~$S_{\alpha,\theta}$.

Симметричным строго устойчивым распределениям соответствует значение
$\theta\hm=0$ и~х.ф.~$\mathfrak{f}_{\alpha,0}(t)\hm=e^{-|t|^{\alpha}}$,
$t\hm\in\mathbb{R}$. Односторонним строго устойчивым законам, сосредоточенным
на неотрицательной полуоси, соответствуют значения $\theta=1$ и
$0\hm<\alpha\hm\le1$. Пары $\alpha\hm=1$, $\theta\hm=\pm1$ отвечают
распределениям, вырожденным в~$\pm1$ соответственно. Остальные
устойчивые распределения абсолютно непрерывны. Явные выражения
устойчивых плотностей в~терминах элементарных функций отсутствуют за
четырьмя исключениями (нормальный закон ($\alpha\hm=2$, $\theta\hm=0$),
распределение Коши ($\alpha\hm=1$, $\theta\hm=0$), распределение Леви
($\alpha\hm=1/2$, $\theta\hm=1$) и~распределение, симметричное 
к~распределению Леви ($\alpha\hm=1/2$, $\theta\hm=-1$)). Выражения
устойчивых плотностей в~терминах функций Фокса (обобщенных
$G$-функ\-ций Мейера) можно найти в~\cite{Schneider1986-k, UchaikinZolotarev1999-k}.

\section{Смешанные биномиальные распределения и~их~асимп\-то\-ти\-чес\-кое поведение}

Рассмотрим задачу, связанную с~испытаниями Бернулли со случайной
вероятностью успеха. Сначала в~результате <<предварительного>>
эксперимента определяется значение с.в.\ $\pi\hm\in(0,1)$. Это
значение принимается в~качестве вероятности успеха в~испытаниях
Бернулли. Затем случайная величина $N$ определяется как число
успехов в~$k\hm\in\mathbb{N}$ испытаниях Бернулли с~так определенной
вероятностью успеха~$\pi$. Чтобы описать бесконечную малость
вероятности успеха~$\pi$, снабдим последнюю и~(для общности)
параметр~$k$, а также, соответственно, с.в.~$N$ <<бесконечно
большим>> индексом~$n$, позволяющим проследить сходимость
последовательности с.в.\ $\pi\hm=\pi_n$ к~нулю при $n\hm\to\infty$. В~свою
очередь, бесконечная малость~$\pi_n$ означает, что успехи являются
редкими событиями в~рамках рассматриваемой последовательности
испытаний Бернулли.

В рамках схемы испытаний Бернулли со случайной вероятностью успеха,
описанной выше, можно сформулировать и~доказать <<случайный>> аналог
классической теоремы Пуассона (так называемого <<закона малых
чисел>>) для \textit{$\pi_n$-сме\-шан\-ных биномиальных распределений} со
случайной ве\-ро\-ят\-ностью успеха и~неограниченно возрастающим
це\-ло\-чис\-лен\-ным параметром~$k_n$ (<<чис\-лом испытаний>>). В~известных
вариантах <<случайного>> аналога тео\-ре\-мы Пуассона (см., 
к~примеру,~\cite{KorolevBeningShorgin2011}), наоборот, случайным считалось число
испытаний, а~вероятность успеха оставалась неслучайной.

Пусть $k_n\hm\in\mathbb{N}$, $k=1,2,\ldots$ Будем говорить, что с.в.~$Q_n$ 
имеет \textit{$\pi_n$-сме\-шан\-ное биномиальное распределение} 
с~параметром~$k_n$, если

\vspace*{-2pt} 

\noindent
\begin{multline}
 {\sf P}\left(Q_n=j\right)=C_{k_n}^j\int\limits_{0}^{1}z^k(1-z)^{k_n-j}\,
d{\sf P}\left(\pi_n<z\right)\,,\\ j=0,1,\ldots,k_n\,.
\label{e1-kk}
\end{multline}

\vspace*{-2pt}

\noindent
Для $x\in\mathbb{R}$ обозначим $B_n(x)\hm={\sf P}(Q_n\hm<x)$. Пусть $N$~---
положительная с.в. Смешанная пуассоновская ф.р.\ со структурной с.в.~$N$ 
(по терминологии, принятой в~\cite{Grandell1997}) будет
обозначаться~$\Pi^{(N)}(x)$:
$$
\Pi^{(N)}(x+0)=\sum\limits_{j=0}^{[x]}\fr{1}{j!}\int\limits_{0}^{\infty}e^{-z}z^j\,
d{\sf P}(N<z)\,,\enskip x\in\mathbb{R}\,.
$$
В~\cite{KorolevKorchaginZeifman2017} доказана следующая теорема.

\smallskip

\noindent
\textbf{Теорема~1.}\ %\cite{KorolevKorchaginZeifman2017}. 
\textit{Пусть
$(k_n)_{n\ge1}$~--- неограниченно возрас-\linebreak тающая последовательность
натуральных чисел. Пусть $Q_n$~--- с.в.\ с~$\pi_n$-сме\-шан\-ным
биномиальным распределением $(1)$ с~целочисленным параметром~$k_n$ 
и~ф.р.~$B_n(x)$. Предположим, что в}~(\ref{e1-kk}) \textit{с.в.~$\pi_n$ бесконечно
малы в~том смысле, что существует с.в.~$N$ такая, что ${\sf P}
(0\hm< N\hm<\infty)\hm=1$ и~выполнено условие $k_n\pi_n\hm\Longrightarrow N$ при
$n\hm\to\infty$. Тогда} $B_n(x)\hm\Longrightarrow \Pi^{(N)}(x)$
$(n\hm\to\infty)$.

\smallskip

Необходимо отметить, что если прореживание процесса происходит по
<<независимой>> схе-\linebreak\vspace*{-12pt}

\pagebreak

\noindent
ме, в~которой имеется двойной массив
$\{\pi_{n,j}\}_{j\ge1}$,
 $n\hm=1,2,\ldots$, независимых в~каждой серии с.в., 
причем $\pi_{n,j}\eqd\pi_n$, $j\hm=1,2\ldots$, при каждом $n\hm\ge1$, так
что $j$-я точка исходного процесса удаляется с~ве\-ро\-ят\-ностью
$1\hm-\pi_{n,j}$, то, как несложно убедиться, процесс прореживания
сводится к~классическому варианту, в~котором предельный процесс
является <<чистым>> пуассоновским.

\section{Пуассон-вейбулловское распределение}

Приведем два хорошо известных примера смешанных пуассоновских
распределений. Во-пер\-вых, это \textit{геометрическое распределение} как
дискретный\linebreak аналог непрерывного показательного распределения, который
получается, если в~смешанной пу\-ассоновской модели <<случайный>>
параметр пу\-ас\-соновского распределения имеет показательное
распреде\-ление.

Во-вторых, это \textit{отрицательное биномиальное\linebreak распределение} как
дискретный аналог непрерывного гам\-ма-рас\-пре\-де\-ле\-ния, который
получается, если в~смешанной пуассоновской модели <<случайный>>
параметр пуассоновского распределения имеет гам\-ма-рас\-пре\-де\-ле\-ние.

Пусть $W_{\gamma, \mu}$~--- с.в., имеющая распределение Вейбулла 
с~параметром масштаба $\mu\hm>0$ и~параметром формы $\gamma\hm>0$: 
${\sf P}(W_{\gamma,\mu}\hm<x)\hm=1\hm-\exp\{-\mu x^{\gamma}\}$, $x\hm\ge0$, 
и~${\sf P}(W_{\gamma,\mu}\hm<0)\hm=0$. Распределение Вейбулла играет важную роль
во многих прикладных задачах в~качестве популярной и~адекватной
модели распределения времени жизни или безотказной работы. Известен
дискретный аналог распределения Вейбулла, который получается
формальным <<квантованием>> непрерывного распределения Вейбулла:
если классический дискретный аналог с.в.~$W_{\gamma,\mu}$ 
с~распределением Вейбулла обозначить~$\widetilde W_{\gamma,\mu}$, то
обычно полагают
$$
{\sf P}\left(\widetilde W_{\gamma,\mu}=k\right)=e^{-\mu k^{\gamma}}-
e^{-\mu(k+1)^{\gamma}}\,,\enskip k=0,1,\ldots
$$
(см., например,~\cite{NakagawaOsaki1975}). У~такого формального
подхода есть существенный недостаток: крайне сложно (если вообще
возможно) сформулировать предельную теорему в~бо\-лее-ме\-нее простой
предельной схеме (например, суммирования или взятия экстремумов с.в.), 
в~которой такое распределение было бы предельным. А~стало быть,
весьма проблематичным становится теоретическое обоснование
воз\-мож\-ности использования такого распределения в~качестве
асимптотической аппроксимации.

Используя аппарат смешанных пуассоновских распределений (являющихся
предельными, например, в~теореме~1 и~сходных с~ней), можно предложить
альтернативный аналог дискретного распределения Вейбулла как
пуас\-сон-вей\-бул\-лов\-ское распределение, т.\,е.\ смешанное пуассоновское
распределение, в~котором смешивание происходит по распределению
Вейбулла.

Рассмотрим с.в.~$V_{\gamma,\mu}$, имеющую смешанное пуассоновское
распределение вида

\vspace*{-2pt}

\noindent
\begin{multline*}
{\sf P}(V_{\gamma,\mu}=k)=\fr{1}{k!}\int\limits_{0}^{\infty}e^{-z}z^kd{\sf P}
\left(W_{\gamma,\mu}<z\right)={}\\
{}=
\fr{\mu\gamma}{k!}\!\int\limits_{0}^{\infty}\!z^{k+\gamma-1}\exp\left\{-\left(z+\mu 
z^{\gamma}\right)\right\}\,dz\,,\ k=0,1,2,\ldots\hspace*{-6.64555pt}
\end{multline*}

\vspace*{-2pt}

\noindent
Такое распределение будем называть \textit{пуас\-сон-вей\-бул\-лов\-ским}.
Возможность считать пуас\-сон-вей\-буллов\-ское распределение дискретным
аналогом распределения Вейбулла обусловлена не только формальным
сходством с~отрицательным биномиальным или геометрическим
распределениями, являющимися смешанными пуассоновскими
распределениями, в~которых смешивание происходит именно по тем
непрерывным распределениям (соответственно гамма- и~показательному),
дискретными аналогами которых они являются. Дополнительным
аргументом можно считать следующее асимптотическое свойство
пуас\-сон-вей\-бул\-лов\-ско\-го распределения.

\smallskip

\noindent
\textbf{Теорема~2}.\ \textit{Справедливо следующее асимптотическое
соотношение: при} $\mu\hm\to 0$
$$
{\sf P}\left(\mu^{1/\gamma}V_{\gamma,\mu}<x\right)\Longrightarrow 
{\sf P}\left(W_{\gamma,1}<x\right)\,.
$$

\smallskip

\noindent
Д\,о\,к\,а\,з\,а\,т\,е\,л\,ь\,с\,т\,в\,о\ этого утверждения основано на том, что,
во-пер\-вых, $V_{\gamma,\mu}\eqd P_1(W_{\gamma,\mu})$, где $P_1(t)$,
$t\hm\ge0$,~--- стандартный пуассоновский процесс, независимый от 
с.в.~$W_{\gamma,\mu}$, во-вто\-рых, $W_{\gamma,\mu}\eqd \mu^{-1/\gamma}W_{\gamma,1}$ 
и,~в-треть\-их, $z^{-1}P_1(z\Lambda)\hm\Longrightarrow \Lambda$ при $z\hm\to\infty$, где
$\Lambda$~--- независимая от $P_1(t)$ неотрицательная с.в.\ (см.,
например,~\cite{KorolevBeningShorgin2011}).

\smallskip

Особый интерес представляет случай $0\hm<\gamma\hm\le1$. Распределения
Вейбулла с~такими па\-ра\-мет\-ра\-ми формы называются \textit{растянутыми
пока\-зательными}\linebreak (stretched exponential)~[11--13]. 
Они занимают промежуточное
место между распределениями с~хвос\-та\-ми, убывающими степенн$\acute{\mbox{ы}}$м
образом, и~рас-\linebreak пределениями с~экспоненциально убывающими хвос\-та\-ми,
играющими важную роль при математическом моделировании явлений 
и~процессов на финансовых рынках. Оказывается, что
пу\-ас\-сон-вей\-бул\-лов\-ское распределение с.в.~$V_{\gamma,\mu}$ 
с~$0\hm<\gamma\hm\le1$ является смешанным геометрическим, т.\,е.\ описывает
распределение числа испытаний до первого успеха в~описанной выше
схеме испытаний Бернулли со случайной вероятностью успеха, име\-ющей
специальное распределение. Для простоты без ограничения общности 
в~следующем утверждении считаем, что $\mu\hm=1$.

\smallskip

\noindent
\textbf{Теорема~ 3.}\ \textit{Пуас\-сон-вей\-бул\-лов\-ское распределение 
с.в.~$V_{\gamma,1}$ с~$0\hm<\gamma\hm\le1$ является смешанным гео\-мет\-ри\-че\-ским}:
$$
{\sf P}(V_{\gamma,1}=k)= 
\int\limits_{0}^{1}z(1-z)^kp_{\gamma,1}(z)dz,\ \ \ \ k=0,1,\ldots,
$$
\textit{где плотность $p_{\gamma,1}(z)$ имеет вид}:
$$
p_{\gamma,1}(z)=\fr{1}{(1-z)^{2}}\,g_{\gamma,1}\left(\fr{z}{1-z}\right)\,,\enskip
0\le z\le1\,.
$$

%\smallskip

\noindent
Д\,о\,к\,а\,з\,а\,т\,е\,л\,ь\,с\,т\,в\,о\,.\ 
В~работе~\cite{Korolev2016Weibull} показано,
что если $0\hm<\gamma\hm\le1$, то $W_{\gamma,1}\eqd W_{1,1}S_{\gamma,1}^{-1}$, 
где с.в.~$W_{1,1}$ и~$S_{\gamma,1}$
независимы. Тогда для $k\hm\in\{0\}\cup\mathbb{N}$
\begin{multline*}
{\sf P}\left(V_{\gamma,1}=k\right)={\sf P}
\left(P_1(W_{\gamma,1})=k\right)={}\\[1pt]
{}=\int\limits_{0}^{\infty}e^{-\lambda}
\fr{\lambda^k}{k!}\left(\int\limits_{0}^{\infty} ze^{-\lambda z}
g_{\gamma,1}(z)\,dz\right)\,d\lambda={}
\\[1pt]
{}=\fr{1}{k!}\int\limits_{0}^{\infty}
zg_{\gamma,1}(z)\left(
\int\limits_{0}^{\infty}e^{-\lambda(z+1)}\lambda^k\,d\lambda\right)\,dz={}\\[1pt]
{}=
\fr{\Gamma(k+1)}{k!}\int\limits_{0}^{\infty}\fr{zg_{\gamma,1}(z)}{(z+1)^{k+1}}\,dz={}
\\[1pt]
{}=\int\limits_{0}^{\infty}\fr{z}{z+1}\left(1-\fr{z}{z+1}\right)^kg_{\gamma,1}(z)\,dz={}\\[1pt]
{}=
\int\limits_{0}^{1}z(1-z)^k g_{\gamma,1}\left(
\fr{z}{1-z}\right)\fr{dz}{(1-z)^{2}}={}\\[1pt]
{}=\int\limits_{0}^{1}z(1-z)^kp_{\gamma,1}(z)\,dz\,,
\end{multline*}
что и~требовалось доказать.

\smallskip

Как известно, производящая функция моментов с.в.~$W_{\gamma,\mu}$
имеет вид:
\begin{multline*}
\hspace*{-1.18353pt}\Psi_{\gamma,\mu}(t)={\sf E}
\exp\left\{tW_{\gamma,\mu}\right\}=
\sum\limits_{n=0}^{\infty}\fr{t^n}{\mu^{n/\gamma}n!}\,\Gamma
\left(1+\fr{n}{\gamma}\right)\,,\\[1pt]
t\in\mathbb{R}.
\vspace*{-12pt}
\end{multline*}


\noindent
В~общем случае производящая функция $\psi_{\gamma,\mu}(s)\hm \equiv 
{\sf E}s^{V_{\gamma,\mu}}$ пуас\-сон-вей\-бул\-лов\-ско\-го распределения имеет вид:

\noindent
\begin{multline*}
\psi_{\gamma,\mu}(s)=\sum\limits_{k=0}^{\infty}s^k{\sf P}
\left(V_{\gamma,\mu}=k\right)={}\\[1pt]
{}=\sum\limits_{k=0}^{\infty}
\fr{s^k}{k!}\int\limits_{0}^{\infty}e^{-z}z^k\,d{\sf P}
\left(W_{\gamma,\mu}<z\right)={}\\[1pt]
{}=\int\limits_{0}^{\infty}
e^{-z}\left[\sum\limits_{k=0}^{\infty}\fr{(sz)^k}{k!}\right]\,d{\sf P}
\left(W_{\gamma,\mu}<z\right)= {}\\[1pt]
{}=\int\limits_{0}^{\infty}e^{z(s-1)}\,d{\sf P}(W_{\gamma,\mu}<z)={\sf E}
\exp\left\{(s-1)W_{\gamma,\mu}\right\}={}\\[1pt]
{}=
\Psi_{\gamma,\mu}(s-1)=\sum\limits_{n=0}^{\infty}\fr{(s-1)^n}{\mu^{n/\gamma}n!}\,
\Gamma\left(1+\fr{n}{\gamma}\right)\,,\\[1pt]
 s\in[0,1]\,.
\end{multline*}
Отсюда имеем:
\begin{multline*}
{\sf P}\left(V_{\gamma,\mu}=0\right)=
\psi_{\gamma,\mu}(s)|_{s=0}=\Psi_{\gamma,\mu}(-1)={}\\[1pt]
{}=
\sum\limits_{n=0}^{\infty}\fr{\Gamma(1+{n}/{\gamma})(-1)^n}{\mu^{n/\gamma}n!}\,;
\end{multline*}

\vspace*{-12pt}

\begin{multline*}
{\sf P}\left(V_{\gamma,\mu}=1\right)=
\fr{d\psi_{\gamma,\mu}(s)}{ds}\Big|_{s=0}={}\\[1pt]
{}=
\sum\limits_{n=0}^{\infty}\fr{\Gamma(1+{n}/{\gamma})}{\mu^{n/\gamma}n!}\,
\fr{d}{ds}\left(s-1\right)^n\Big|_{s=0}={}\\[1pt]
{}=
\sum\limits_{n=1}^{\infty}\fr{\Gamma(1+{n}/{\gamma})(-1)^{n-1}}{\mu^{n/\gamma}(n-1)!}\,;
\end{multline*}

\vspace*{-12pt}

\begin{multline*}
{\sf P}\left(V_{\gamma,\mu}=2\right)=
\fr{d^2\psi_{\gamma,\mu}(s)}{2ds^2}\Big|_{s=0}={}\\[1pt]
{}=
\fr{1}{2}\sum\limits_{n=0}^{\infty}\fr{\Gamma(1+{n}/{\gamma})}
{\mu^{n/\gamma}n!}\,\fr{d^2}{ds^2}\left(s-1\right)^n\Big|_{s=0}={}\\[1pt]
{}=
\fr{1}{2}\sum\limits_{n=2}^{\infty}\fr{\Gamma(1+{n}/{\gamma})(-1)^{n-2}}
{\mu^{n/\gamma}(n-2)!};\ldots;
\end{multline*}

\vspace*{-12pt}

\begin{multline*}
{\sf P}(V_{\gamma,\mu}=k)=\fr{d^k\psi_{\gamma,\mu}(s)}{k!ds^k}\Big|_{s=0}={}\\[1pt]
{}=
\fr{1}{k!}\sum\limits_{n=k}^{\infty}\fr{\Gamma(1+{n}/{\gamma})(-1)^{n-k}}
{\mu^{n/\gamma}(n-k)!}\,,\enskip k=3,4,\ldots
\end{multline*}
Как видно, это распределение весьма громоздко и~<<прямое>>
оценивание его параметров представляет нетривиальную задачу.

\section{Двухэтапный сеточный ЕМ-алгоритм для~оценивания параметров 
смешанных пуассоновских
распределений и,~в~частности, параметров пуассон-вейбулловского
рас\-пре\-де\-ле\-ния}

По сути оценивание параметров смешанных пуассоновских моделей
сводится к~оцениванию смешивающего распределения. Традиционно с~этой
целью используется классический ЕМ (expectation-maximization) ал\-го\-ритм~\cite{Karlis2005}.
Однако иногда, в~част\-ности в~пуас\-сон-вей\-бул\-лов\-ском \mbox{случае},
классический ЕМ-ал\-го\-ритм оказывается менее эффективным, чем
альтернативный двухэтапный сеточный ЕМ-ал\-го\-ритм оценивания
параметров смешанных пуассоновских распределений.

Следует отметить, что сеточные методы разделения смесей довольно
эффективны не только при оценивании параметров смешанных
пуассоновских распределений, но и~при разделении конечных или
произвольных дис\-пер\-си\-он\-но-сдви\-го\-вых смесей нормальных 
законов~\cite{KorolevKorchagin2014}.

Рассмотрим следующий двухэтапный метод разделения смешанных
пуассоновских распределений на примере оценивания параметров~$\gamma$, 
$\mu$ пу\-ас\-сон-вей\-бул\-лов\-ско\-го распределения
$\Pi^{(N)}(x)\hm=\Pi^{(W_{\gamma,\mu})}(x)$.

На первом этапе на положительной полупрямой выделим основную часть
носителя смешивающего распределения, т.\,е.\ ограниченный интервал,
вероятность которого, вычисленная в~соответствии со смешивающим
распределением, практически равна единице. На этот интервал накинем
конечную сетку, содержащую (возможно, очень большое чис\-ло)
$K\hm\in\mathbb{N}$ \textit{известных} узлов $\lambda_1,\ldots,\lambda_K$.
Приблизим искомое смешанное пуассоновское распределение конечной
смесью пуассоновских законов:
\begin{equation}
\Pi^{(W_{\gamma,\mu})}(x+0)\approx\sum\limits_{j=0}^{[x]}
\fr{1}{j!}\sum\limits_{i=1}^K p_i e^{-\lambda_i}\lambda_i^j\,,\enskip
x\in\mathbb{R}\,.
\label{e2-kk}
\end{equation}
В смеси, стоящей в~правой части соотношения~(\ref{e2-kk}), неизвестными
являются только па\-ра\-мет\-ры $p_1,\ldots,p_{K}$. Пусть $x_1,\ldots,x_n$~---
анализируемая выборка значений случайной величины с~оце\-ни\-ва\-емым
смешанным пуассоновским распределением. Итерационный процесс,
определяющий сеточный ЕМ-ал\-го\-ритм для данной задачи, задается
следующим образом. Пусть $p_1^{(m)},\ldots,p_{K-1}^{(m)}$~--- оценки
па\-ра\-мет\-ров $p_1,\ldots,p_{K-1}$ на $m$-й итерации,
$p_K^{(m)}\hm=1\hm-p_1^{(m)}-\cdots -p_{K-1}^{(m)}$. Для $i\hm=1,\ldots,K$,
$j\hm=1,\ldots,n$ обозначим $\phi_{ij}\hm=e^{-\lambda_i+x_j\ln \lambda_i}$.
Тогда, используя стандартные рассуждения, определяющие
вычислительные формулы EM-ал\-го\-рит\-ма для параметров конечной смеси
вероятностных распределений (см, например,~[17,
разд.~5.3.7--5.3.8]), следует положить:

\vspace*{-2pt}

\noindent
\begin{multline}
p_i^{(m+1)}=\fr{p_i^{(m)}}{n}\sum\limits_{j=1}^n
\fr{\phi_{ij}}{\sum\nolimits_{r=1}^Kp_r^{(m)}\phi_{rj}}\,,
\\
 i=1,\ldots,K\,.
\label{e3-kk}
\end{multline}
Как видим, итерационный процесс, задаваемый соотношением~(\ref{e3-kk}), очень
прост. В~силу монотонности классического ЕМ-алгоритма справедливо
сле\-ду\-ющее утверждение.

\smallskip

\noindent
\textbf{Теорема~4.}\ \textit{Пусть узлы $\lambda_1,\ldots,\lambda_K$ сетки
различны, неотрицательны и~известны. Тогда итерационный процесс}~(\ref{e3-kk}) 
\textit{является монотонным, т.\,е.\ каждая его итерация не уменьшает
целевую сеточную функцию правдоподобия}


\noindent
$$
L\left(p_1,\ldots,p_K;x_1,\ldots,x_n\right)=\prod\limits_{j=1}^n\left[
\sum\limits_{i=1}^K
p_i\phi_{ij}\right]\,.
$$

%\smallskip

Заметим, что, как показано в~[17, разд.~5.7.4],
сеточная функция правдоподобия $L(p_1,\ldots,p_{K};\,x_1,\ldots,x_n)$
вогнута по аргументам $p_1,\ldots,p_{K}$. Поэтому на каждом шаге
итерационного процесса вместо соотношения~(\ref{e3-kk}) можно использовать
любой более быстрый алгоритм максимизации функции
$L(p_1,\ldots,p_{K};\,x_1,\ldots,x_n)$ по переменным $p_1,\ldots,p_{K}$.
Например, оценки весов $p_1,\ldots,p_K$ можно искать методом условного
градиента~\cite{Korolev2011, KorolevNazarov2010}.

\smallskip

Таким образом, на первом этапе получаются оценки весов~$p_i$ всех
узлов~$\lambda_i$, $i\hm=1,\ldots,K$, конечной сетки, накинутой на
носитель смешивающего распределения.

На втором этапе остается применить ка\-кой-ли\-бо стандартный метод
подгонки распределения Вейбулла ${\sf P}(W_{\gamma,\mu}\hm<x)$ 
к~эмпирическим данным типа гистограммы $(\lambda_1, p_1),\ldots,
(\lambda_K, p_K)$, полученным на первом этапе. Например, параметры~$\gamma$ 
и~$\mu$ можно оценить, минимизируя соответствующую
статистику хи-квад\-рат. Или же, например, можно решить задачу
наименьших квадратов:

\vspace*{-2pt}

\noindent
\begin{multline*}
(\gamma^*,\mu^*)={}\\
{}=\arg\min_{\gamma,\mu}\sum_{i=1}^K\left[p_i-
\exp\left\{-\fr{\mu}{2}\left(\lambda_{i-1}+\lambda_i\right)^{\gamma}\right\}+{}\right.\\
\left.{}+
\exp\left\{-\fr{\mu}{2}\left(\lambda_i+\lambda_{i+1}\right)^{\gamma}\right\}\right]^2\,,
\end{multline*}
где $\lambda_0\hm=0$, $\lambda_{K+1}\hm=\infty$.

\pagebreak

На практике хорошие результаты показал подход с~решением задачи
наименьших квадратов. Для поиска параметров использовался алгоритм
{\sf ns2sol}, описанный в~книге~\cite{DSch1983}. Указанный алгоритм
доступен во многих статистических пакетах, отличается высоким
быстродействием и~возможностью при желании задавать разумные
интервалы для поиска параметров.

Также хорошие результаты показал метод поиска наилучшего
распределения в~смысле минимизации расстояния Куль\-ба\-ка--Лейб\-ле\-ра,
который в~данном случае эквивалентен максимизации правдоподобия
полученной гистограммы в~классе распределений Вейбулла.

При фиксированной сетке двухэтапный метод дает лишь приближенные
оценки параметров смешанных пуассоновских распределений, причем
точность приближения зависит от успешного выбора сетки. Некоторые
аспекты этого выбора будут рассмотрены в~следующем разделе. Говорить
о~состоятельности получаемых оценок при фиксированной сетке нельзя.
Но если объем выборки неограниченно возрастает и~вместе с~ним
согласованно увеличивается число узлов, то вопрос о~состоятельности
получаемых оценок приобретает смысл и~будет рассмотрен в~одной из
следующих публикаций.

\section{О~практическом выборе сетки на~первом этапе двухэтапного
сеточного ЕМ-алгоритма для~разделения смесей пуас\-со\-нов\-ских
распределений}

Естественно, что при использовании указанного двухэтапного метода 
в~динамическом режиме крайне важным становится вопрос о~выборе
наиболее эффективных и~быстродействующих численных процедур и~их
параметров. В~част\-ности, исключительную важность приобретает
правильный выбор границ сетки на первом этапе. Рас\-смот\-рим этот
вопрос подробнее.

Формально рассматриваемая задача выглядит так: по наблюдаемым
значениям $x_1,\ldots,x_n$ требуется построить статистическую оценку
верхней границы квантилей заданного порядка сме\-ши\-ва\-юще\-го закона так,
чтобы как можно точнее оценить носитель смешивающего распределения.

В дальнейшем будем считать, что $x_1,\ldots,x_n$~--- независимые
реализации с.в.\  $X\hm\eqd P_1(\Lambda)$, где $P_1(t)$~--- стандартный
пуассоновский процесс, $t\hm\ge0$, $\Lambda$~--- независимая от~$P_1(t)$
неотрицательная с.в. В~случае пу\-ас\-сон-вей\-бул\-лов\-ско\-го распределения
$X\hm\eqd V_{\gamma,\mu}$, $\Lambda\hm\eqd W_{\gamma,\mu}$.

Сначала рассмотрим более общий случай и~предположим, что ${\sf E}
\Lambda^2\hm<\infty$. Тогда ${\sf E}X\hm={\sf E}\Lambda$, ${\sf D}X\hm=
{\sf E}\Lambda\hm+{\sf D}\Lambda\hm={\sf E}X\hm+{\sf D}\Lambda$. Следовательно,
${\sf D}\Lambda\hm={\sf D}X\hm-{\sf E}X$. Но ${\sf D}\Lambda\hm={\sf E}
\Lambda^2\hm-({\sf E}\Lambda)^2\hm={\sf E}\Lambda^2\hm-({\sf E}X)^2$.
Поэтому ${\sf E}\Lambda^2\hm={\sf D}\Lambda\hm+({\sf E}X)^2\hm={\sf D}X\hm-
{\sf E}X\hm+({\sf E}X)^2\hm={\sf E}X^2\hm-{\sf E}X$. Таким образом, для
$\lambda\hm>0$ по неравенству Маркова имеем:
\begin{equation}
{\sf P}(\Lambda\ge\lambda)\le\frac{{\sf
E}\Lambda^2}{\lambda^2}=\frac{{\sf E}X^2-{\sf
E}X}{\lambda^2}.\label{e4-kk}
\end{equation}
Отсюда, задав произвольно малое $\varepsilon\hm>0$, можно \mbox{найти}
приближенную верхнюю оценку 
$(1\hm-\varepsilon)$-кван\-ти\-ли~$\lambda^{(1-\varepsilon)}$ 
с.в.~$\Lambda$. С~этой целью положим:
$$
\lambda_{\varepsilon}=\sqrt{\fr{{\sf E}X^2-{\sf E}X}{\varepsilon}}\,.
$$

\begin{figure*}[b] %fig1
\vspace*{6pt}
\begin{center}
\mbox{%
\epsfxsize=163.367mm
\epsfbox{kor-1.eps}
}
\end{center}
\vspace*{-9pt}
  \Caption{Графики <<истинной>> смешивающей плотности с~$\mu\hm=1$ 
  и~$\gamma\hm=1/2$ (\textit{1}) и~ее оценки, полученной
двухэтапным сеточным методом~(\textit{2})~(\textit{а}) и~графики
соответствующего <<истинного>> пу\-ас\-сон-вей\-бул\-лов\-ско\-го 
распределения~(\textit{3}) и~его статистической оценки~(\textit{4})~(\textit{б}). Объем выборки $n\hm=500$}
%\end{figure*}
%\begin{figure*} %fig2
\vspace*{12pt}
\begin{center}
\mbox{%
\epsfxsize=162.372mm
\epsfbox{kor-2.eps}
}
\end{center}
\vspace*{-9pt}
   \Caption{Графики <<истинной>> смешивающей плотности с~$\mu\hm=2$ 
   и~$\gamma\hm=1/2$ (\textit{1}) и~ее оценки, полученной
двухэтапным сеточным методом~(\textit{2})~(\textit{а}) и~графики
соответствующего <<истинного>> пу\-ас\-сон-вей\-бул\-лов\-ско\-го 
распределения~(\textit{3}) и~его статистической оценки~(\textit{4})~(\textit{б}). Объем выборки
$n\hm=500$}
\end{figure*}


\noindent
Тогда из~(\ref{e4-kk}) вытекает, что ${\sf P}
(\Lambda\hm\ge\lambda_{\varepsilon})\hm\le\varepsilon$, т.\,е.\ можно
положить $\lambda_K\hm=\lambda^{(1-\varepsilon)}$, причем
$$
\lambda^{(1-\varepsilon)}\le\lambda_{\varepsilon}=\sqrt{\fr{{\sf E}
X^2-{\sf E}X}
{\varepsilon}}\approx\sqrt{\fr{1}{n\varepsilon}\sum\limits_{j=1}^nx_j(x_j-1)}\,.
$$

Теперь отдельно рассмотрим случай, когда ${\sf E}e^X\hm<\infty$. Этот
случай, в~част\-ности, имеет место для пу\-ас\-сон-вей\-бул\-лов\-ско\-го
распределения $X\hm\eqd V_{\gamma,\mu}$, $\Lambda\hm\eqd W_{\gamma,\mu}$,
если $\gamma\hm>1$ или $\gamma\hm=1$ и~$\mu\hm>e\hm-1$. Имеем
\begin{multline*}
{\sf E}e^X=\sum\limits_{k=0}^{\infty}\fr{e^k}{k!}
\int\limits_{0}^{\infty}e^{-\lambda}\lambda^kd{\sf P}
(\Lambda<\lambda)={}\\
{}=\int\limits_{0}^{\infty}e^{-\lambda}
\left(\sum\limits_{k=0}^{\infty}\fr{(e\lambda)^k}{k!}\right)\,d{\sf P}
(\Lambda<\lambda)={}
\\
{}=
\int\limits_{0}^{\infty}e^{\lambda(e-1)}\,d{\sf P}(\Lambda<\lambda)=
{\sf E}e^{\Lambda(e-1)}\,.
\end{multline*}
Тогда
\begin{multline}
{\sf P}(\Lambda\ge\lambda)={\sf P}
\left(\Lambda(e-1)\ge\lambda(e-1)\right)\le{}\\
{}\le \fr{{\sf E}
e^{\Lambda(e-1)}}{e^{\lambda(e-1)}}=\fr{{\sf E} e^X}{e^{\lambda(e-1)}}\,.
\label{e5-kk}
\end{multline}
Для произвольно малого положительного~$\varepsilon$ положим:
$$
\lambda_{\varepsilon}=\fr{\ln{\sf E}e^X-\ln\varepsilon}{e-1}\,.
$$
Тогда из~(\ref{e5-kk}) вытекает, что ${\sf P}
(\Lambda\hm\ge\lambda_{\varepsilon})\hm\le\varepsilon$, т.\,е.\ можно
положить $\lambda_K\hm=\lambda^{(1-\varepsilon)}$, причем
\begin{multline*}
\lambda_K=\lambda^{(1-\varepsilon)}\le\lambda_{\varepsilon}=\fr{\ln{\sf E}
e^X-\ln\varepsilon}{e-1}\approx{}\\
{}\approx \fr{1}{e-1}\ln\left( 
\fr{1}{n\varepsilon}\sum\limits_{j=1}^ne^{x_j}\right)\,.
\end{multline*}

\vspace*{-9pt}

\section{Примеры}

\vspace*{-2pt}

В этом разделе приведены результаты тестовых вычислений по
описанному выше алгоритму (рис.~1--4). Моделировались искусственные выборки из
пу\-ас\-сон-вей\-бул\-лов\-ско\-го распределения, к~которым применялся описанный
выше двухэтап-\linebreak\vspace*{-10pt}

\columnbreak

\noindent
ный метод. Необходимо отметить, что во всех случаях
размер сетки был относительно небольшим ($K\hm=15$), тем не менее
достигнута приемлемая точность. В~каждой паре рисунков слева~---
графики <<истинной>> смешивающей плотности и~ее оценки, полученной
сеточным методом, справа~--- графики соответствующего <<истинного>>
пуас\-сон-вей\-бул\-лов\-ско\-го распределения и~его статистической оценки.

%\bigskip

Авторы выражают благодарность И.\,Г.~Шевцовой и~А.\,В.~Дорофеевой 
за участие в~работе по тес\-ти\-ро\-ва\-нию алгоритма, проведению вычислений\linebreak
 и~построению 
графиков, проведенную в~рамках выполнения проекта №\,14-11-00397 
Российского научного фонда.

\pagebreak




\end{multicols}

\begin{figure*} %fig3
\vspace*{1pt}
\begin{center}
\mbox{%
\epsfxsize=163.372mm
\epsfbox{kor-3.eps}
}
\end{center}
\vspace*{-9pt}
    \Caption{Графики <<истинной>> смешивающей плотности 
    с~$\mu\hm=0{,}2$ и~$\gamma\hm=2$ (\textit{1}) и~ее оценки, полученной
двухэтапным сеточным методом~(\textit{2}) и~графики
соответствующего <<истинного>> пу\-ас\-сон-вей\-бул\-лов\-ско\-го распределения
(\textit{3}) и~его статистической оценки~(\textit{4})~(\textit{б}). Объем выборки
$n\hm=1000$}
%\end{figure*}
%\begin{figure*} %fig4
\vspace*{12pt}
\begin{center}
\mbox{%
\epsfxsize=163.472mm
\epsfbox{kor-4.eps}
}
\end{center}
\vspace*{-9pt}
    \Caption{Графики <<истинной>> смешивающей плотности 
    с~$\mu=2$ и~$\gamma=1/2$~(\textit{1}) и~ее оценки, полученной
двухэтапным сеточным методом~(\textit{2}) и~графики
соответствующего <<истинного>> пу\-ас\-сон-вей\-бул\-лов\-ско\-го 
распределения~(\textit{3}) и~его статистической оценки~(\textit{4})~(\textit{б}). Объем выборки
$n\hm=1000$}
\end{figure*}

\begin{multicols}{2}


{\small\frenchspacing
 {%\baselineskip=10.8pt
 \addcontentsline{toc}{section}{References}
 \begin{thebibliography}{99}
    
    \bibitem{GrigoryevaKorolevSokolov2013}
\Au{Григорьева М.\,Е., Королев В.\,Ю., Соколов~И.\,А.} Предель\-ная теорема
    для геометрических сумм неза\-ви\-симых неодинаково распределенных
    случайных ве\-личин и~ее применение к~прогнозированию вероятности
    катастроф в~неоднородных потоках экстремальных событий~//
    Информатика и~её применения, 2013. Т.~7. Вып.~4. С.~11--19.
    
    \bibitem{Renyi1956-k}
\Au{R$\acute{\mbox{e}}$nyi A.} A~Poisson-folyamat egy jellemzese~// Maguar Tud. Acad.
    Mat. Int. Kozl., 1956. Vol.~1. P.~519--527.
    
    \bibitem{Mogyorodi1971}
\Au{Mogyorodi J.} Some notes on thinning recurrent flows~// 
Litovsky Math. Sbornik, 1971. Vol.~11. P.~303--315.
    
    \bibitem{Zolotarev1983-k}
\Au{Золотарев В.\,М.} Одномерные устойчивые распределения.~--- М.: Наука, 1983. 304~с.
    
    \bibitem{Schneider1986-k}
\Au{Schneider W.\,R.} Stable distributions: Fox
    function representationand generalization~// 
    Stochastic processes in classical and quantum systems~/ 
    Eds. S.~Albeverio, G.~Casati, D.~Merlini.~--- Berlin: Springer, 1986. P.~497--511.
    
    \bibitem{UchaikinZolotarev1999-k} 
    \Au{Uchaikin V.\,V., Zolotarev~V.\,M.}
    Chance and stability.~--- Utrecht: VSP, 1999. 570~p.
    
    \bibitem{KorolevBeningShorgin2011} %%%% оставить
\Au{Королев В.\,Ю., Бенинг~В.\,Е., Шоргин~С.\,Я.} Математические
        основы теории риска.~--- 2-е изд.~--- М.: Физматлит, 2011. 591~с.
    
    \bibitem{Grandell1997} %%%% оставить
    \Au{Grandell J.} Mixed Poisson processes.~--- London: Chapman and Hall,
    1997. 268~p.
    
    \bibitem{KorolevKorchaginZeifman2017} %%%%% оставить
\Au{Korolev V.\,Yu., Korchagin~A.\,Yu., Zeifman~A.\,I.} On doubly
    stochastic rarefaction of renewal processes~// 14th 
     Conference (International) of Numerical Analysis and Applied
    Mathematics Proceedings.~---
    American Institute of Physics Proceedings, 2017 (in press).
    
    \bibitem{NakagawaOsaki1975} %%%% оставить
    \Au{Nakagawa T., Osaki~Sh.} The discrete Weibull distribution~// IEEE
    Trans. Reliab., 1975. Vol.~24. P.~300--301.
    
    \bibitem{LaherrereSornette1998}
    \Au{\mbox{Laherr{\!\!\ptb{\`{e}}}re}~J., Sornette D.}
    Stretched exponential distributions in nature and economy: ``Fat
    tails'' with characteristic scales~// Eur. Phys.~J.~B,
    1998. Vol.~2. P.~525--539.
    
    \bibitem{Sornette_et_al2005} 
    \Au{Malevergne Y., Pisarenko~V., Sornette~D.}
    Empirical distributions of stock returns: Between the
    stretched exponential and the power law?~// Quant. Financ.,
    2005. Vol.~5. P.~379--401.
    
    \bibitem{Sornette_et_al2006} 
    \Au{Malevergne Y., Pisarenko~V., Sornette~D.}
    On the power of generalized extreme value (GEV) and
    generalized Pareto distribution (GDP) estimators for empirical
    distributions of stock returns~// Appl. Financ. Econ., 2006.
    Vol.~16. P.~271--289.
    
    \bibitem{Korolev2016Weibull}
\Au{Korolev V.\,Yu.} Product representations for random variables with
    the Weibull distributions and their applications~// J.~Math. Sci., 
    2016. Vol.~218. No.\,3. P.~298--313.
    
    \bibitem{Karlis2005} %%%% оставить
\Au{Karlis D.} An EM algorithm for mixed Poisson distributions~// ASTIN
    Bull., 2005. Vol.~35. P.~3--24.
    
    \bibitem{KorolevKorchagin2014} %%%%% оставить
    \Au{Королев В.\,Ю., Корчагин~А.\,Ю.} Модифицированный сеточный метод
    разделения дис\-пер\-си\-он\-но-сдви\-го\-вых смесей нормальных законов~//
    Информатика и~её применения, 2014. Т.~8. Вып.~4. С.~11--19.
    
    \bibitem{Korolev2011} %%%% оставить
\Au{Королев В.\,Ю.} Вероятностно-статистические методы декомпозиции
    волатильности хаотических процессов.~--- М.: Изд-во Московского
    ун-та, 2011. 510~с.
    
    \bibitem{KorolevNazarov2010} %%%% оставить
\Au{Королев В.\,Ю., Назаров~А.\,Л.} Разделение смесей вероятностных
    распределений при помощи сеточных методов моментов и~максимального
    правдоподобия~// Автоматика и~телемеханика, 2010. Вып.~3. С.~98--116.
    
    \bibitem{DSch1983} %%%% оставить
\Au{Dennis J.\,E., Schnabel~R.\,B.} Numerical methods for unconstrained
    optimization and nonlinear equations.~--- Englewood Cliffs:
    Prentice-Hall, 1983. 375~p.
 \end{thebibliography}

 }
 }

\end{multicols}

\vspace*{-3pt}

\hfill{\small\textit{Поступила в~редакцию 15.10.16}}

\vspace*{8pt}

%\newpage

%\vspace*{-24pt}

\hrule

\vspace*{2pt}

\hrule

%\vspace*{8pt}


\def\tit{THE POISSON THEOREM FOR BERNOULLI TRIALS WITH~A~RANDOM
PROBABILITY OF~SUCCESS\\ AND~A~DISCRETE ANALOG OF~THE~WEIBULL
DISTRIBUTION}

\def\titkol{The Poisson theorem for Bernoulli trials with a random
probability of success and~a~discrete analog of the Weibull
distribution}

\def\aut{V.\,Yu.~Korolev$^{1,2}$, A.\,Yu.~Korchagin$^{1,2}$, and~A.\,I.~Zeifman$^{2,3,4}$}

\def\autkol{V.\,Yu.~Korolev, A.\,Yu.~Korchagin, and~A.\,I.~Zeifman}

\titel{\tit}{\aut}{\autkol}{\titkol}

\vspace*{-9pt}


    
\noindent
  
\noindent
$^1$Faculty of Computational Mathematics and Cybernetics, 
M.\,V.~Lomonosov Moscow State University, 
1-52~Lenin-\linebreak
$\hphantom{^1}$skiye Gory, GSP-1, Moscow 119991, Russian Federation

\noindent
$^2$Institute of Informatics Problems, Federal Research Center 
``Computer Science and Control'' of the Russian\linebreak
$\hphantom{^1}$Academy of Sciences, 44-2~Vavilov Str., 
Moscow 119333,  Russian Federation

\noindent
$^3$Vologda State University, 15~Lenin Str., Vologda 160000, Russian Federation

\noindent
$^4$ISEDT RAS, 56-A~Gorky Str., Vologda 16001, Russian Federation



\def\leftfootline{\small{\textbf{\thepage}
\hfill INFORMATIKA I EE PRIMENENIYA~--- INFORMATICS AND
APPLICATIONS\ \ \ 2016\ \ \ volume~10\ \ \ issue\ 4}
}%
 \def\rightfootline{\small{INFORMATIKA I EE PRIMENENIYA~---
INFORMATICS AND APPLICATIONS\ \ \ 2016\ \ \ volume~10\ \ \ issue\ 4
\hfill \textbf{\thepage}}}

\vspace*{3pt}    


\Abste{A problem related to the Bernoulli trials with 
a~random probability of success is considered. First, as a~result of
the preliminary experiment, the value of the random variable
$\pi\in(0,1)$ is determined that is taken as the probability of
success in the Bernoulli trials. Then, the random variable $N$ is
determined as the number of successes in $k\in\mathbb{N}$ Bernoulli
trials with the so determined success probability~$\pi$. The
distribution of the random variable~$N$ is called $\pi$-mixed
binomial. Within the framework of these Bernoulli trials with the
random probability of success, a~``random'' analog of the classical
Poisson theorem is formulated for the $\pi$-mixed binomial distributions, in which
the limit distribution turns out to be the mixed Poisson distribution. Special
attention is paid to the case where mixing is performed with
respect to the Weibull distribution. The corresponding mixed Poisson
distribution called Poisson--Weibull law is proposed as a~discrete
analog of the Weibull distribution. Some properties of the
Poisson--Weibull distribution are discussed. In particular, it is
shown that this distribution can be represented as the mixed geometric
distribution. A~two-stage grid algorithm is proposed for 
estimation of parameters of mixed Poisson distributions and, in
particular, of the Poisson--Weibull distribution. Statistical estimators
for the upper bound of the grid are constructed. The examples of
practical computations performed by the proposed algorithm are presented.}


\KWE{Bernoulli trials with a random probability of
success; mixed binomial distribution; Poisson theorem; mixed Poisson
distribution; Weibull distribution; Poisson--Weibull distribution;
mixed geometric distribution; EM-algorithm}

\DOI{10.14357/19922264160402} 

%\vspace*{-9pt}

\Ack
    \noindent
This work was financially supported by the Russian Science Foundation 
(grant No.\ 14-11-00397).



%\vspace*{3pt}

  \begin{multicols}{2}

\renewcommand{\bibname}{\protect\rmfamily References}
%\renewcommand{\bibname}{\large\protect\rm References}

{\small\frenchspacing
 {%\baselineskip=10.8pt
 \addcontentsline{toc}{section}{References}
 \begin{thebibliography}{99}

    \bibitem{GrigoryevaKorolevSokolov2013_eng} %%%% оставить
    \Aue{Grigoryeva, M.\,E., V.\,Yu.~Korolev, and I.\,A.~Sokolov}. 2013.
    Predel'naya teorema dlya geometricheskikh summ ne\-za\-vi\-si\-mykh neodinakovo
    raspredelennykh sluchaynykh velichin i~ee primenenie k~progrozirovaniyu
    veroyatnosti ka\-tast\-rof v~neodnorodnykh potokakh ekstremal'nykh sobytiy
    [A~limit theorem for geometric sums of independent nonidentically
    distributed random variables and its application to the prediction of the probabilities of
    catastrophes in
    nonhomogeneous flows of extremal events].
    \textit{Informatika i~ee Primeneniya~--- Inform. Appl.} 7(4):11--19.
    
    \bibitem{Renyi1956_eng-k} %%%%% оставить
    \Aue{R$\acute{\mbox{e}}$nyi,~A.} 1956.
    A~Poisson-folyamat egy jellemzese. \textit{Maguar Tud. Acad. Mat. Int. Kozl.} 1:519--527.
    
    \bibitem{Mogyorodi1971_eng} %%%%% оставить
    \Aue{Mogyorodi, J.} 1971.
    Some notes on thinning recurrent flows. \textit{Litovsky Math. Sbornik} 11:303--315.
    
    \bibitem{Zolotarev1983_eng-k} 
    \Aue{Zolotarev, V.\,M.} 1983.
    \textit{Odnomernye ustoychivye raspredeleniya} 
    [One-dimensional stable distributions]. Moscow: Nauka. 304~p.
    
    
    \bibitem{Schneider1986_eng-k} 
    \Aue{Schneider, W.\,R.} 1986.
    Stable distributions: Fox function representationand generalization. 
    \textit{Stochastic processes in classical and quantum systems}. 
    Eds.\ S.~Albeverio, G.~Casati, and D.~Merlini. Berlin: Springer. 497--511.
    
    \bibitem{UchaikinZolotarev1999_eng-k} 
    \Aue{Uchaikin, V.\,V., and V.\,M.~Zolotarev}. 1999.
    \textit{Chance and stability.} Utrecht: VSP. 570~p.
    
    \bibitem{KorolevBeningShorgin2011_eng} %%%% оставить
\Aue{Korolev, V.\,Yu., V.\,E.~Bening, and S.\,Ya.~Shorgin}. 2011.
    \textit{Matematicheskie osnovy teorii riska} 
    [Mathematical fundamentals of risk theory]. 2nd ed. Moscow: Fizmatlit. 591~p.
    
    \bibitem{Grandell1997_eng} %%%% оставить
\Aue{Grandell, J.} 1997.
    \textit{Mixed Poisson processes.} London: Chapman and Hall. 268~p.
    
    \bibitem{KorolevKorchaginZeifman2017_eng} %%%%% оставить
\Aue{Korolev, V.\,Yu., A.\,Yu.~Korchagin, and A.\,I.~Zeifman}. 2017 (in press).
    On doubly stochastic rarefaction of renewal processes. \textit{14th 
     Conference (International) of Numerical Analysis and Applied
        Mathematics Proceedings}.
    American Institute of Physics Proceedings. 
    
    \bibitem{NakagawaOsaki1975_eng} %%%% оставить
    \Aue{Nakagawa, T., and Sh.~Osaki}. 1975.
    The discrete Weibull distribution. \textit{IEEE Trans. Reliab.} 24:300--301.
    
    \bibitem{LaherrereSornette1998_eng}
    \Aue{\mbox{Laherr{\!\ptb{\`{e}}}re}, J., and D.~Sornette}. 1998.
    Stretched exponential distributions in nature and economy: ``Fat
    tails'' with characteristic scales. \textit{Eur. Phys. J.~B} 2:525--539.
    
    \bibitem{Sornette_et_al2005_eng}
    \Aue{Malevergne, Y., V.~Pisarenko, and D.~Sornette}. 2005.
    Empirical distributions of stock returns: Between the
    stretched exponential and the power law? \textit{Quant. Financ.} 5:379--401.
    
    \bibitem{Sornette_et_al2006_eng}
    \Aue{Malevergne, Y., V.~Pisarenko, and D.~Sornette}. 2006.
    On the power of generalized extreme value (GEV) and
    generalized Pareto distribution (GDP) estimators for empirical
    distributions of stock returns. \textit{Appl. Financ. Econ.} 16:271--289.
    
    \bibitem{Korolev2016Weibull_eng} 
    \Aue{Korolev, V.\,Yu.} 2016.
    Product representations for random variables with the Weibull distributions 
    and their applications. \textit{J.~Math. Sci.} 218(3):298--313.
    
    \bibitem{Karlis2005_eng} %%%% оставить
    \Aue{Karlis, D.} 2005.
    An EM algorithm for mixed Poisson distributions. 
    \textit{ASTIN Bull.} 35:3--24.
    
    \bibitem{KorolevKorchagin2014_eng} %%%%% оставить
    \Aue{Korolev, V.\,Yu., and A.\,Yu.~Korchagin}. 2014.
    Modi\-fi\-tsi\-ro\-van\-nyy setochnyy metod razdeleniya dispersionno-sdvigovykh smesey
    normal'nykh zakonov [Modified grid method for decomposition of 
    mean-variance normal mixtures].
    \textit{Informatika i~ee Primeneniya--- Inform. Appl.} 8(4):11--19.
    
    \bibitem{Korolev2011_eng} %%%% оставить
    \Aue{Korolev, V.\,Yu.} 2011.
    \textit{Veroyatnostno-statisticheskie metody dekompozitsii volatil'nosti
        khaoticheskikh protsessov} [Probablity-based method for volatility
        decomposition of chaotic processes]. Moscow: Moscow University Press. 510~p.
    
    \bibitem{KorolevNazarov2010_eng} %%%% оставить
   \Aue{Korolev, V.\,Yu., and A.\,L.~Nazarov}. 2010.
    %Razdeleniye smesei veroyatnostnih raspredeleniy pri pomoshi setochnyh
%    metodov momentov i~maksimalnogo pravdopodobiya [
Separating mixtures of probability distributions with the 
grid maximum likelihood method].
    \textit{Avtomat. Rem. Contr.} 71(3):455--472.
    
    \bibitem{DSch1983_eng} %%%% оставить
    \Aue{Dennis, J.\,E., and R.\,B.~Schnabel}. 1983.
    \textit{Numerical methods for unconstrained optimization and nonlinear equations.} 
    Englewood Cliffs:     Prentice-Hall. 375~p.
    \end{thebibliography}

 }
 }

\end{multicols}

\vspace*{-3pt}

\hfill{\small\textit{Received October 15, 2016}}

\Contr


\noindent
\textbf{Korolev Victor Yu.} (b.\ 1954)~--- Doctor of Science in physics and mathematics, professor, 
Head of the Department of Mathematical Statistics, 
Faculty of Computational Mathematics and Cybernetics, 
M.\,V.~Lomonosov Moscow State University, 
1-52~Leninskiye Gory, GSP-1, Moscow 119991, Russian Federation; 
leading scientist, Institute of Informatics Problems, Federal Research Center 
``Computer Science and Control'' of the Russian Academy of Sciences, 44-2~Vavilov Str., 
Moscow 119333,  Russian Federation; \mbox{vkorolev@cs.msu.su} 

 \vspace*{3pt}

\noindent
\textbf{Korchagin Alexander Yu.} (b.\ 1989)~---
junior scientist, Faculty of Computational Mathematics and Cybernetics, 
M.\,V.~Lomonosov Moscow State University, 1-52~Leninskiye Gory, GSP-1, Moscow 119991, 
Russian Federation; Institute of Informatics Problems, Federal Research Center 
``Computer Science and Control'' of the Russian Academy of Sciences, 44-2~Vavilov Str., 
Moscow 119333, Russian Federation; \mbox{sasha.korchagin@gmail.com}

\vspace*{3pt}

\noindent
\textbf{Zeifman Alexander I.} (b.\ 1954)~---
Doctor of Science in physics and mathematics, professor, Head of Department, 
Vologda State University, 15~Lenin Str., Vologda 160000, Russian Federation; 
senior scientist, Institute of Informatics Problems, Federal Research Center 
``Computer Science and Control'' of the Russian Academy of Sciences, 44-2~Vavilov Str., 
Moscow 119333, Russian Federation; principal scientist, 
ISEDT RAS, 56-A~Gorky Str., Vologda 16001, Russian Federation; a\_zeifman@mail.ru
\label{end\stat}


\renewcommand{\bibname}{\protect\rm Литература}  %2
\def\stat{kor-zeif-kor}

\def\tit{НЕСИММЕТРИЧНЫЕ РАСПРЕДЕЛЕНИЯ ЛИННИКА\\ КАК ПРЕДЕЛЬНЫЕ
ЗАКОНЫ ДЛЯ СЛУЧАЙНЫХ СУММ НЕЗАВИСИМЫХ СЛУЧАЙНЫХ ВЕЛИЧИН\\ С~КОНЕЧНЫМИ
ДИСПЕРСИЯМИ$^*$}

\def\titkol{Несимметричные распределения Линника как предельные
законы для случайных сумм %независимых 
случайных величин} % с~конечными дисперсиями}

\def\aut{В.\,Ю.~Королев$^1$, А.\,И.~Зейфман$^2$,  А.\,Ю.~Корчагин$^3$}

\def\autkol{В.\,Ю.~Королев, А.\,И.~Зейфман,  А.\,Ю.~Корчагин}

\titel{\tit}{\aut}{\autkol}{\titkol}

\index{Королев В.\,Ю.}
\index{Зейфман А.\,И.}
\index{Корчагин А.\,Ю.}
\index{Korolev V.\,Yu.}
\index{Zeifman A.\,I.}
\index{Korchagin A.\,Yu.}


{\renewcommand{\thefootnote}{\fnsymbol{footnote}} \footnotetext[1]
{Работа выполнена при финансовой поддержке
Российского научного фонда (проект 14-11-00364).}}


\renewcommand{\thefootnote}{\arabic{footnote}}
\footnotetext[1]{Факультет вычислительной математики и~кибернетики Московского государственного 
университета им.\ М.\,В.~Ломоносова; 
Институт проблем информатики Федерального исследовательского центра 
<<Информатика и~управление>> Российской академии наук, \mbox{vkorolev@cs.msu.ru}}
\footnotetext[2]{Вологодский государственный университет; Институт проблем информатики 
Федерального исследовательского центра <<Информатика и~управление>> 
Российской академии наук; Институт со\-ци\-аль\-но-эко\-но\-ми\-че\-ско\-го развития 
территории Российской академии наук, Факультет вычислительной математики и~кибернетики Московского государственного 
университета им.\ М.\,В.~Ломоносова, \mbox{a\_zeifman@mail.ru}}
\footnotetext[3]{Факультет вычислительной математики и~кибернетики Московского 
государственного университета им.\ М.\,В.~Ломоносова, \mbox{sasha.korchagin@gmail.com}}

\vspace*{6pt}

\Abst{Распределения Линника (симметричные гео\-мет\-ри\-чески
устойчивые распределения) находят широкое применение в~финансовой
математике, телекоммуникационных системах, астрофизике, генетике.
Такие распределения являются предельными для геометрических сумм
независимых одинаково распределенных случайных величин (с.в.),
распределения которых принадлежат области нормального притяжения
симметричного строго устойчивого распределения. В~статье
рассматриваются три несим\-мет\-рич\-ных обобщения распределения Линника.
Традиционный (и формальный) подход к~не\-сим\-мет\-рич\-но\-му обобщению
распределения Линника заключается в~рассмотрении геометрических сумм
случайных слагаемых, распределения которых притягиваются к~\textit{не\-сим\-мет\-рич\-но\-му} 
строго устойчивому распределению. Дис\-пер\-сии таких
слагаемых бесконечны. Поскольку при моделировании реальных явлений,
как правило, нет веских причин отвергать предположение о конечности
дисперсии элементарных слагаемых, в~качестве альтернатив
традиционному подходу в~статье предложены несимметричные обобщения,
основанные на представлении распределения Линника в~виде смеси
нормальных распределений и~смеси распределений Лапласа. Приведены
примеры предельных теорем для сумм случайного числа независимых
с.в.\ \textit{с~конечными дисперсиями}, в~которых
предложенные несимметричные распределения Линника выступают 
в~качестве предельных законов.}

\KW{распределение Линника; распределение Лапласа;
распределение Мит\-таг--Леф\-фле\-ра; нормальное распределение; масштабная
смесь; дис\-пер\-си\-он\-но-сдви\-го\-вая смесь нормальных законов; устойчивое
распределение; геометрически устойчивое распределение}

\DOI{10.14357/19922264160403}

\vspace*{6pt}  


\vskip 10pt plus 9pt minus 6pt

\thispagestyle{headings}

\begin{multicols}{2}

\label{st\stat}

\section{Введение}

Распределения Линника (симметричные гео\-мет\-ри\-чески устойчивые
распределения) имеют доволь\-но широкое применение в~финансовой
математике, телекоммуникационных системах, астрофизике, генетике.
Разнообразные приложения распределений Линника описаны 
в~работах~\cite{MittnikRachev1991, Kotz2001}. В~частности, распределения
Линника возникают при изучении механизма синтеза мелатонина 
в~человеческом организме, солнечных нейтринных потоков в~космосе,
явлений рос\-та-упад\-ка в~природе, эконометрических явлений и~т.\,п.

Геометрически устойчивые законы и~только они могут быть предельными
распределениями для геометрических случайных сумм независимых
одинаково распределенных с.в. Поэтому
традиционно\linebreak несимметричное обобщение распределения Линника
достигается за счет того, что в~\textit{геометрической} случайной сумме
рассматриваются скошенные\linebreak слагаемые. При этом распределения
слагаемых принадлежат области нормального притяжения несимметричного
устойчивого закона с~некоторым показателем $\alpha\hm\in(0,2]$ и,~значит, 
при $0\hm<\alpha\hm<2$ имеют бесконечные моменты порядков, больших
или равных~$\alpha$. Что касается случая $\alpha\hm=2$, когда конечна
дисперсия, то в~рамках схемы гео\-мет\-ри\-че\-ско\-го суммирования он
неизбежно приводит к~единственно возможному распределению~---
распределению Лап\-ласа.

Однако при использовании распределений Линника в~качестве моделей
реальных явлений нельзя не задуматься над вопросом о том, что если
используется аддитивная структурная модель реального процесса типа
случайно остановленного случайного блуждания, то какая комбинация
условий встречается чаще:
\begin{itemize}
\item распределение числа слагаемых (числа скачков) является геометрическим
(асимптотически экспоненциальным), но слагаемые \mbox{(скачки)} имеют столь
тяжелые хвосты, что как минимум у них бесконечна дисперсия, или

\item вторые моменты (дисперсии) слагаемых (скачков) конечны, но
число слагаемых отличается нерегулярным поведением, допускающим
иногда возможность очень больших значений?
\end{itemize}

Поскольку, как правило, при моделировании реальных явлений нет веских
причин отвергать предположение о конечности дисперсии скачков,
вторая комбинация как минимум заслуживает внимательного изучения.

Оказывается, что распределения Линника допускают представление 
в~виде масштабных смесей нормальных законов. Это означает, что они
могут быть предельными в~аналогах центральной предельной теоремы для
случайных сумм независимых с.в.\ с~\textit{конечными} 
дисперсиями~\cite{KorolevZeifman2016a, KorolevZeifman2016b}, что открывает пути
альтернативных несимметричных обобщений этих распределений, которым
и~посвящена данная статья.

\section{Вспомогательные сведения}

В~дальнейшем иногда будет удобнее вести изложение не в~терминах
распределений, а~в~терминах с.в., предполагая, что все они заданы
на одном вероятностном пространстве $(\Omega,\mathfrak{A}, {\sf P})$.

Случайная величина со стандартной показательной функцией распределения (ф.р.)\
будет обозначаться~$W_1$: ${\sf P}(W_1<x)\hm=\left[1-e^{-x}\right]{\bf 1}
(x\hm\geqslant0)$ (здесь и~далее символ~$\mathbf{1}(C)$ обозначает индикатор
множества~$C$). Случайная величина со стандартной нормальной ф.р.~$\Phi(x)$
будет обозначаться~$X$,
$$
{\sf P}(X<x)=\Phi(x)=\fr{1}{\sqrt{2\pi}}\int\limits_{-\infty}^{x}e^{-z^2/2}\,dz\,,\enskip
x\in\mathbb{R}\,.
$$
Функция распределения и~ плот\-ность строго устойчивого распределения 
с~характеристическим показателем~$\alpha$ и~па\-ра\-мет\-ром формы~$\theta$,
определяемого характеристической функцией (х.ф.)
$$
\mathfrak{f}_{\alpha,\theta}(t)=
\exp\left\{-|t|^{\alpha}\exp\left\{-\fr{1}{2}\,
i\pi\theta\alpha\,\mathrm{sign}\,t\right\}
\right\}\,,\enskip
t\in\mathbb{R}\,,
$$
где $0<\alpha\leqslant2$, $|\theta|\hm\leqslant\min\{1,({2}/{\alpha})-1\}$, будут
соответственно обозначаться $G_{\alpha,\theta}(x)$ 
и~$g_{\alpha,\theta}(x)$ (см., например,~\cite{Zolotarev1983}). Любую
с.в.\ с~ф.р.~$G_{\alpha,\theta}(x)$ будем обозначать~$S_{\alpha,\theta}$. 
Симметричным строго устойчивым распределениям
соответствует значение $\theta\hm=0$ и~х.ф.
\begin{equation}
\mathfrak{f}_{\alpha,0}(t)=e^{-|t|^{\alpha}}\,,\enskip t\in\mathbb{R}\,.
\label{e1-kz}
\end{equation}
Отсюда несложно видеть, что $S_{2,0}\eqd\sqrt{2}X$.

Односторонним строго устойчивым законам, сосредоточенным на
неотрицательной полуоси, соответствуют значения $\theta\hm=1$ 
и~$0\hm<\alpha\hm\leqslant1$. Пары $\alpha\hm=1$, $\theta\hm=\pm1$ отвечают
распределениям, вы\-рож\-ден\-ным в~$\pm1$ соответственно. Остальные
устойчивые распределения абсолютно непрерывны. Явные выражения
устойчивых плотностей в~терминах элементарных функций отсутствуют за
четырьмя исключениями (нормальный закон ($\alpha\hm=2$, $\theta\hm=0$),
распределение Коши ($\alpha\hm=1$, $\theta\hm=0$), распределение Леви
($\alpha\hm=1/2$, $\theta\hm=1$) и~распределение, симметричное 
к~распределению Леви ($\alpha\hm=1/2$, $\theta\hm=-1$)). Выражения
устойчивых плотностей в~терминах функций Фокса (обобщенных
$G$-функ\-ций Мейера) можно найти в~\cite{Schneider1986, UchaikinZolotarev1999}.

Хорошо известно, что если $0\hm<\alpha\hm<2$, 
то ${\sf E}|S_{\alpha,\theta}|^{\beta}\hm<\infty$ для любого
$\beta\hm\in(0,\alpha)$, при этом моменты с.в.~$S_{\alpha,\theta}$ порядков
$\beta\hm>\alpha$ не существуют (см., например,~\cite{Zolotarev1983}).
Несмотря на отсутствие явных выражений плотностей устойчивых
распределений в~терминах элементарных функций, можно 
показать~\cite{KorolevWeibull2016}, что для $0\hm<\beta\hm<\alpha\hm<2$
$$
{\sf E}|S_{\alpha,0}|^{\beta}=\fr{2^{\beta}}{\sqrt{\pi}}\,
\fr{\Gamma(({\beta+1})/{2})\Gamma(1-{\beta}/{\alpha})}
{\Gamma({2}/{\beta}-1)}
$$
и для $0<\beta<\alpha\hm\leqslant 1$
$$
{\sf
E}S_{\alpha,1}^{\beta}=\fr{\Gamma(1-{\beta}/{\alpha})}{\Gamma(1-\beta)}\,.
$$

Символы $\eqd$ и~$\Longrightarrow$ будут соответственно обозначать
совпадение распределений и~сходимость по распределению.

Говорят, что распределение с.в.~$Y$ принадлежит 
к~области нормального притяжения строго устойчивого закона
$G_{\alpha,\theta}$, $\mathcal{L}(Y)\hm\in \mathrm{DNA}\left(G_{\alpha,\theta}\right)$,
если существует конечная положительная постоянная~$c$ та\-кая,~что
$$
\fr{c}{n^{1/\alpha}}\sum\limits_{j=1}^nX_j\Longrightarrow
S_{\alpha,\theta}\enskip (n\to\infty)\,,
$$
где $X_1,X_2,\ldots$~--- независимые копии с.в.~$Y$. В~дальнейшем
будем рассматривать случай стандартного масштаба и~полагаем $c\hm=1$. 
В~работе~\cite{Tucker1975} было показано, что если $\mathcal{L}(Y)\hm\in
\mathrm{DNA}\left(G_{\alpha,\theta}\right)$, то ${\sf E}|Y|^{\beta}\hm=\infty$ для любого
$\beta\hm>\alpha$.

Распределение $H$ с.в.~$Q$ называется геометрически устойчивым,
если оно является слабым пределом геометрических случайных сумм
независимых одинаково распределенных с.в., а~именно: если существует
последовательность независимых одинаково распределенных с.в.\
$X_1,X_2,\ldots$ и~с.в.~$V_p$, имеющая геометрическое распределение
\begin{equation*}
{\sf P}(V_p=n)=p(1-p)^{n-1}\,,\enskip n=1,2,\ldots\,,\ \  p\in(0,1)\,,
%\label{e2-kz}
\end{equation*}
при каждом $p\hm\in(0,1)$ независимая от $X_1,X_2,\ldots,$ 
и~положительные константы~$a_p\hm>0$ такие, что
$$
a_p\left(X_1+\cdots+X_{V_p}\right)\Longrightarrow Q
$$
при $p\to 0$ (см., например,~[1, 10--12]). 
В~работе~\cite{KlebanovManiaMelamed1984} (также см., 
например,~\cite{Rachev1991, GnedenkoKorolev1996}) показано, что распределение~$H$ 
является геометрически устойчивым тогда и~только тогда, когда
соответствующая ему х.ф.~$\mathfrak{h}(t)$ допускает представление
\begin{equation}
\mathfrak{h}(t)=\left(1-\log\mathfrak{f}_{\alpha,\theta}(t)\right)^{-1}\,,\enskip
t\in\mathbb{R}\,,
\label{e3-kz}
\end{equation}
при некоторых $\alpha\hm\in(0,2]$ 
и~$\theta\hm\in[-\min\{1,{2}/{\alpha}\hm-1\},\,\min\{1,{2}/{\alpha}\hm-1\}]$.

Некоторые результаты данной работы будут существенно опираться на
следующее вспомогательное утверждение. Рассмотрим последовательность
с.в.~$Y_1, Y_2,\ldots$ Пусть $N_1,N_2,\ldots$~---
на\-ту\-раль\-но\-знач\-ные с.в.\ такие, что при каждом~$n$ с.в.~$N_n$ 
независима от последовательности $Y_1,Y_2,\ldots$ Всюду далее
сходимость подразумевается при $n\hm\to\infty$.

\smallskip

\noindent
\textbf{Лемма~1.}\ \textit{Предположим, что существуют неограниченно
возрастающая $($убывающая к~нулю$)$ последовательность положительных
чисел $\{b_n\}_{n\geqslant1}$ и~с.в.~$Y$ такие, что}
$$
\fr{Y_n}{b_n}\Longrightarrow Y\,.
$$
\textit{Если существуют неограниченно возрастающая $($убывающая к~нулю$)$
последовательность положительных чисел $\{d_n\}_{n\geqslant1}$ и~с.в.~$V$
такие, что}
\begin{equation}
\fr{b_{N_n}}{d_n}\Longrightarrow V\,,
\label{e4-kz}
\end{equation}
\textit{то}
\begin{equation}
\fr{Y_{N_n}}{d_n}\Longrightarrow Y V\,,\label{e5-kz}
\end{equation}
\textit{причем случайные сомножители в~правой части}~(\ref{e5-kz}) \textit{независимы. Если
дополнительно $N_n\hm\longrightarrow\infty$ по вероятности и~семейство
масштабных смесей ф.р.\ с.в.~$Y$ идентифицируемо, то условие}~(\ref{e4-kz})
 \textit{не только достаточно для}~(\ref{e5-kz}), \textit{но и~необходимо.}

\smallskip

\noindent
Д\,о\,к\,а\,з\,а\,т\,е\,л\,ь\,с\,т\,в\,о\ \  см.~в~\cite{Korolev1994} (случай
$b_n,d_n\hm\to\infty$), \cite{Korolev1995} (случай $b_n,d_n\hm\to 0$) 
или~\cite{BeningKorolev2002}, теорема~3.5.5.

\section{Распределения Линника}

Распределения с~х.ф.
$$
\mathfrak{f}^{L}_{\alpha}(t)=\left(1+|t|^{\alpha}\right)^{-1}\,,\enskip
t\in\mathbb{R}\,,
$$
где $0<\alpha\hm\leqslant2$, принято называть \textit{распределениями Линника}
(в~работе~\cite{Pillai1985} предложено альтернативное менее
употребительное название \textit{$\alpha$-Laplace distribution}). Они
были введены Ю.\,В.~Линником в~ 1953~г.~\cite{Linnik1953}. При
$\alpha\hm=2$ распределение Линника превращается в~распределение
Лапласа, соответствующее плотности
\begin{equation}
f^{\Lambda}(x)=\fr{1}{2}\,e^{-|x|}\,,\enskip
x\in\mathbb{R}\,.
\label{e6-kz}
\end{equation}
Лапласовская с.в.\ с~плот\-ностью~(\ref{e6-kz}) и~ее ф.р.\ будут соответственно
обозначаться~$\Lambda$ и~$F^{\Lambda}(x)$.

Случайная величина, имеющая распределение Линника с~параметром~$\alpha$, ее ф.р.\
и~плот\-ность будут соответственно обозначаться~$L_{\alpha}$,
$F_{\alpha}^{L}$ и~$f_{\alpha}^{L}$. При этом $F_2^{L}(x)\hm\equiv
F^{\Lambda}(x)$, $x\hm\in\mathbb{R}$.

Относительно недавно эти распределения и~их обобщения вновь
привлекли внимание исследователей как вполне адекватные модели
многих реальных явлений. Распределения Линника обладают многими
интересными свойствами. Лаха~\cite{Laha1961} (также см.~\cite{Lukacs1970}) 
доказал унимодальность распределений Линника. 
В~работах \cite{KotzOstrovskiiHayfavi1995a, KotzOstrovskiiHayfavi1995b} исследованы 
свойства плотности~$f_{\alpha}^{L}$. Показано, что для~$f_{\alpha}^{L}$ справедливо
интегральное представление
\begin{equation}
\hspace*{-2mm}f_{\alpha}^{L}=\fr{\sin(\pi\alpha/2)}{\pi}\!
\int\limits_{0}^{\infty}\!\!\fr{z^{\alpha}e^{-z|x|}\,dz}{1+z^{2\alpha}+2\cos(\pi\alpha/2)}\,,\enskip
x\in\mathbb{R}.\!\!\label{e7-kz}
\end{equation}
Сабу и~Пиллаи~\cite{SabuPillai1987} получили представление плотности~$f_{\alpha}^{L}$ 
в~терминах обобщенных $G$-функ\-ций Мейера. Лин~\cite{Lin1994} 
доказал саморазложимость~$F_{\alpha}^{L}$.
Существо\-вание моментов с.в.~$L_{\alpha}$ обсуждается в~работе~\cite{Anderson1992}. 
Абсолютные моменты порядков $\beta\hm<\alpha$ с.в.~$L_{\alpha}$ имеют вид:
$$
{\sf E}|L_{\alpha}|^{\beta}=\fr{2^{\beta}}{\sqrt{\pi}}\,
\fr{\Gamma(1+{\beta}/{\alpha})
\Gamma(({1+\beta})/{2})\Gamma(1-{\beta}/{\alpha})}
{\Gamma(1-{\beta}/{2})}\,.
$$
Распределения Линника безгранично делимы~\cite{Devroye1990}, имеют
бесконечный пик плотности в~нуле при $\alpha\hm\leqslant1$~\cite{Devroye1990}. 
В~работе~\cite{Jacquesetal1999} показано, что
при $0\hm<\alpha\hm<2$ хвосты распределения Лапласа убывают степенн$\acute{\mbox{ы}}$м
образом:
$$
\lim\limits_{x\to\infty}x^{\alpha}
\left[1-F^{L}_{\alpha}(x)\right]=\fr{\Gamma(\alpha)}{\pi}\sin\fr{\pi\alpha}{2}\,.
$$

Из представлений~(\ref{e1-kz}) и~(\ref{e4-kz}) вытекает, что \textit{распределения Линника
и только они являются симметричными геометрически устойчивыми
законами}.

В работах~\cite{KorolevZeifman2016a,
KorolevZeifman2016b, KotzOstrovskii1996, Pakes1998} получены разнообразные представления
распределений Линника в~виде смесей. Некоторые из этих представлений
будут приведены и~использованы ниже. Другие аналитические 
и~асимптотические свойства распределения Линника рассмотрены 
в~\cite{KotzOstrovskiiHayfavi1995a, KotzOstrovskiiHayfavi1995b}.

\section{Распределения Миттаг--Леффлера}

Пусть $\alpha\in(0,1)$ и~$M_{\alpha}$~--- неотрицательная с.в.\
с~преобразованием Лап\-ла\-са--Стиль\-тье\-са (п.~Л.--С.)
\begin{equation}
{\sf E}\exp\{-sM_{\alpha1}\}=\left(1+s^{\alpha}\right)^{-1}\,,\enskip
s\geqslant0\,.\label{e8-kz}
\end{equation}
Распределения с~п.~Л.--С.~(\ref{e8-kz}) принято называть \textit{распределениями
Мит\-таг--Леф\-фле\-ра}. Происхождение этого названия связано с~тем, что
плотность, соответствующая п.~Л.--С.~(\ref{e8-kz}), имеет вид:
\begin{multline}
f_{\alpha}^{M}(x)=\fr{1}{x^{1-\alpha}}\sum\limits_{n=0}^{\infty}
\fr{(-1)^nx^{\alpha n}}{\Gamma(\alpha n+1)}=-\fr{d}{dx}\,E_{\alpha}(-x^{\alpha})\,,\\
x\geqslant0\,,\label{e9-kz}
\end{multline}
где $E_{\alpha}(z)$~--- функция Мит\-таг--Леф\-фле\-ра индекса~$\alpha$,
определяемая как степенной ряд
$$
E_{\alpha}(z)=\sum\limits_{n=0}^{\infty}\fr{z^n}{\Gamma(\alpha
n+1)}\,,\enskip \alpha>0\,,\ z\in\mathbb{Z}\,.
$$
Функция распределения, соответствующая плотности~(\ref{e9-kz}), будет обозначаться
$F_{\alpha}^{M}(x)$.

Для ф.р.~$F_{\alpha}^{M}(x)$ при $x\hm>0$ справедливо интегральное
представление:
\begin{multline}
F_{\alpha}^{M}(x)=1-\int\limits_{0}^{\infty}e^{-xz}f_{\alpha,1}^{Q}(z)\,dz={}\\
{}=
1-\fr{\sin(\pi\alpha)}{\pi}\int\limits_{0}^{\infty}
\fr{z^{\alpha-1}e^{-zx}\,dz}{1+z^{2\alpha}+2z^{\alpha}\cos(\pi\alpha)}\,.
\label{e10-kz}
\end{multline}

При $\alpha=1$ распределение Мит\-таг--Леф\-фле\-ра превращается 
в~стандартное показательное распределение: $M_1\eqd W_1$. Но при
$\alpha\hm<1$ плотность~(\ref{e9-kz}) имеет хвост, убывающий степенн$\acute{\mbox{ы}}$м
образом: если $0\hm<\alpha\hm<1$, то
$$
\lim\limits_{x\to\infty}x^{\alpha+1}f_\alpha^{M}(x)=
\fr{\Gamma(\alpha+1)}{\pi}\sin\pi\alpha
$$
(см., например,~\cite{Kilbas2014}).

Моменты с.в.\ с~распределением Мит\-таг--Леф\-фле\-ра порядков
$\beta\hm\geqslant\alpha$ бесконечны, но при $0\hm<\beta\hm<\alpha\hm<1$
$$
{\sf E}M_{\alpha}^{\beta}=
\Gamma\left(1+\fr{\beta}{\alpha}\right)\Gamma\left(1-\fr{\beta}{\alpha}\right)\,.
$$

Распределение Мит\-таг--Леф\-фле\-ра геометрически устойчиво. Еще в~1965~г.\
И.\,Н.~Коваленко~\cite{Kovalenko1965} показал, что распределения с~п.~Л.--С.~(\ref{e8-kz}) 
и~только они являются возможными предельными
распределениями для надлежащим образом нормированных геометрических
сумм вида $a_p(X_1+\cdots+X_{V_p})$ независимых
неотрицательных с.в.\ при $p\hm\to0$. Доказательства этого результата
были воспроизведены в~книгах~\cite{GnedenkoKorolev1996, GnedenkoKovalenko1968,
GnedenkoKovalenko1989}, где вместо
термина <<распределения Мит\-таг-Леф\-фле\-ра>> класс распределений 
с~п.~Л.--С.~(\ref{e8-kz}) был назван \textit{классом}~$\mathcal{K}$ в~честь И.\,Н.~Коваленко.

Спустя 25~лет упомянутое предельное свойство
распределений с~п.~Л.--С.~(\ref{e8-kz}) было переоткрыто 
Р.~Пиллаи~\cite{Pillai1989, Pillai1990}, который предложил для них
использовать термин \textit{распределения Мит\-таг--Леф\-фле\-ра}, ставший
общепринятым.

Распределения Мит\-таг--Леф\-фле\-ра используются при описании аномальной
диффузии или эффектов релаксации (см.~\cite{WeronKotulski1996,
GorenfloMainardi2006} и~дальнейшие ссылки в~этих работах).

При каждом $\alpha\hm\in(0,1]$ распределение Мит\-таг--Леф\-фле\-ра является
смешанным показательным распределением:
$$
M_{\alpha}\eqd W_1 \fr{S_{\alpha,1}}{S'_{\alpha,1}}\,,
$$
где $S'_{\alpha,1}\eqd S_{\alpha,1}$ и~все с.в.\ в~правой части
независимы. Доказательство этого факта можно найти 
в~работах~\cite{KorolevZeifman2016a, KorolevZeifman2016b, KotzOstrovskii1996},
где, в~част\-ности, показано, что плотность $p_{\alpha}(x)$ отношения
$S_{\alpha,1}/S'_{\alpha,1}$ двух независимых односторонних строго
устойчивых с.в.\ с~характеристическим показателем $\alpha\hm\in(0,1)$
имеет вид:
\begin{equation}
p_{\alpha}(x)=\fr{\sin(\pi\alpha)x^{\alpha-1}}
{\pi[1+x^{2\alpha}+2x^{\alpha}\cos(\pi\alpha)]}\,,\enskip
x>0\,.\label{e11-kz}
\end{equation}

Для дальнейшего важно подчеркнуть, что \textit{распределения
Мит\-таг--Леф\-фле\-ра и~только они являются геометрически устойчивыми
законами, сосредоточенными на неотрицательной полуоси}.

\section{Традиционный подход к~определению несимметричных
распределений Линника}

Канонический вид~(\ref{e4-kz}) х.ф.\ геометрически устойчивого распределения
обусловлен определением последнего и~теоремой переноса
Гне\-ден\-ко--Фа\-хи\-ма: пусть $\{X_{n,j}\}_{j\hm\in\mathbb{N}}$,
$n\hm\in\mathbb{N}$,~--- последовательность серий независимых 
и~одинаково в~каждой серии распределенных с.в.,
$\{N_n\}_{n\in\mathbb{N}}$~--- последовательность неотрицательных
целочисленных с.в., при каждом $n\hm\geqslant1$ независимых от
$X_{n,1},X_{n,2},\ldots$ Для $k\hm\in\mathbb{N}\cup\{0\}$ обозначим
$S_{n,k}=X_{n,1}+\cdots+X_{n,k}$ ($S_{n,0}\hm=0$). Согласно теореме
переноса Гне\-ден\-ко--Фа\-хи\-ма~\cite{GnedenkoFahim1969} (также см.,
например,~\cite{GnedenkoKorolev1996}), если существует
последовательность $\{k_n\}_{n\in\mathbb{N}}$ натуральных чисел и~с.в.~$Y$ 
и~$N$ такие, что при $n\hm\to\infty$ $S_{n,k_n}\hm \Longrightarrow Y$ и
\begin{equation}
\fr{N_n}{k_n}\Longrightarrow N\,,\label{e12-kz}
\end{equation}
то $ S_{n,N_n}\hm\Longrightarrow Z$, где $Z$~--- с.в.\ с~х.ф.
$$
\mathfrak{f}(t)=\int\limits_{0}^{\infty}\left(\mathfrak{h}(t)\right)^u\,dA(u)\,,\enskip
t\in\mathbb{R}\,.
$$
Здесь $\mathfrak{h}(t)$~--- х.ф.\ с.в.~$Y$; $A(u)\hm={\sf P}(N<u)$.

Если п.~Л.--С. ${\sf E}e^{-sN}$ с.в.~$N$ обозначить~$\psi_N(s)$,
$s\hm\geqslant0$, то х.ф.~$\mathfrak{f}(t)$ можно записать в~виде:
$$
\mathfrak{f}(t)=\psi_N(-\log\mathfrak{h}(t))\,,\enskip t\in\mathbb{R}\,.
$$
При $k_n=n$, $N_n\eqd V_{1/n}$ в~силу теоремы Реньи~\cite{Renyi1956}
в~(\ref{e12-kz}) имеем $N\eqd W_1$. При этом $\psi_N(s)\hm=(1\hm+s)^{-1}$, так что 
в~таком случае
$$
\mathfrak{f}(t)=\psi(-\log\mathfrak{h}(t))=\left(1-\log\mathfrak{h}(t)\right)^{-1}\,,\enskip
t\in\mathbb{R}\,.
$$
Если, более того, $X_{n,j}\eqd n^{-1/\alpha}X_j$ для всех
$n,j\hm\in\mathbb{N}$, где $X_1,X_2,\ldots$~--- независимые одинаково
распределенные с.в.\ с~$\mathcal{L}(X_1)\hm\in \mathrm{DNA}(G_{\alpha,\theta})$
при некоторых допустимых значениях~ $\alpha$ и~$\theta$, то х.ф.~$\mathfrak{f}(t)$, 
предельная для геометрических случайных сумм,
имеет вид~(\ref{e4-kz}).
%
Поэтому базирующаяся на схеме \textit{геометрического} случайного
суммирования и~традиционном представлении о~распределениях Линника
как симметричных \textit{геометрически устойчивых} законах идея
несимметричного их обобщения заключается в~замене
$g_{\alpha,0}(t)\hm=e^{-|t|^{\alpha}}$ в~(\ref{e4-kz}) на $g_{\alpha,\theta}(t)$
с~произвольным допустимым~$\theta$, что приводит к~распределениям с~х.ф.\ вида:
$$
\widetilde{\mathfrak{f}}^{L}_{\alpha}(t)=
\left(1+|t|^{\alpha}\exp\left\{-
\fr{1}{2}\,i\pi\theta\alpha\,\mathrm{sign}\,t\right\}\right)^{-1}\,,\enskip
t\in\mathbb{R}
$$
(см., например,~\cite{KotzOstrovskiiHayfavi1995a,
KotzOstrovskiiHayfavi1995b}). Такие распределения будем называть 
\textit{несимметричными распределениями Линника первого рода}. Бо\-лее-ме\-нее
полная библиография по этой теме приведена в~ \cite{LimTeo2009}.

Другими словами, традиционно несимметричное обобщение распределений
Линника достигается за счет того, что в~\textit{геометрической}
случайной сумме рассматриваются скошенные слагаемые, распределения
которых принадлежат области нормального притяжения несимметричного
устойчивого закона.

\section{Представление распределений Линника в~виде масштабных
смесей нормальных или~ лапласовых законов}

Распределения Линника допускают представление в~виде масштабных
смесей нормальных законов. Это означает, что они могут быть
предельными в~аналогах центральной предельной теоремы для случайных
сумм независимых с.в.\ с~\textit{конечными} 
дисперсиями~\cite{KorolevZeifman2016a, KorolevZeifman2016b}, что открывает пути
альтернативных несимметричных обобщений этих распределений.

В работах~\cite{KorolevZeifman2016b, KotzOstrovskii1996} установлена
интересная связь между распределениями Линника, Лапласа 
и~Мит\-таг--Леф\-фле\-ра и~показано, что
\begin{equation}
L_{\alpha}\eqd X\sqrt{2M_{\alpha/2}}\eqd
\Lambda\sqrt{\fr{S_{\alpha/2,1}}{S'_{\alpha/2,1}}}\,,
\label{e13-kz}
\end{equation}
где все сомножители независимы и~$S'_{\alpha/2,1}\eqd
S_{\alpha/2,1}$.

Как уже отмечалось, все распределения Линника и~только они являются
симметричными геометрически устойчивыми законами. Все распределения
Мит\-таг--Леф\-фле\-ра и~только они являются односторонними геометрически
устойчивыми законами. С~учетом упоминавшегося выше соотношения
$S_{2,0}\eqd\sqrt{2}X$ левое равенство~(\ref{e13-kz}) можно переписать в~виде
$L_{\alpha}\eqd S_{2,0}\sqrt{M_{\alpha/2}}$, являющемся частным
случаем более общего утверждения, доказанного в~работе~\cite{KorolevZeifman2016b}: 
если $\alpha\hm\in(0,2]$ и~$\alpha'\hm\in(0,1]$, то справедливо соотношение
$$
L_{\alpha\alpha'}\eqd S_{\alpha,0}M_{\alpha'}^{1/\alpha}\,,
$$
представляющее собой аналог <<теоремы умножения>> устойчивых
с.в.\ (см., например, теорему~3.3.1.\
в~\cite{Zolotarev1983}), в~классе геометрически устойчивых
распределений. Это интересная иллюстрация изоморфизма класса
геометрически устойчивых распределений классу устойчивых
распределений, установленного в~работе~\cite{KlebanovManiaMelamed1984}.

Представление~(\ref{e13-kz}) означает, что схема гео\-мет\-ри\-че\-ско\-го случайного
суммирования отнюдь не исчерпывает все возможные предельные
постановки задач для случайных сумм и~других последовательностей 
с~независимыми случайными индексами, в~которых распределение Линника
может выступать в~качестве предельного. Соответствующие примеры
предельных теорем приведены в~\cite{KorolevZeifman2016b}.

\section{Несимметричные распределения Линника
как~масштабные смеси несимметричных распределений Лапласа}

В этом разделе будет реализован формальный подход к~несимметричному
обобщению распределения Линника, для чего будет использовано правое
равенство~(\ref{e13-kz}). В~результате будет получено несимметричное
распределение, каждая ветвь которого (положительная и~отрицательная)
будут копиями соответствующих ветвей \textit{разных} распределений
Лап\-ласа.

Пусть $a_1$ и~$a_2$~--- два положительных числа. Будем говорить, что
с.в.~$\Lambda_{a_1,a_2}$ имеет \textit{несимметричное распределение
Лапласа с~параметрами~$a_1$ и~$a_2$}, если ее ф.р.\ имеет вид:
\begin{multline*}
F^{\Lambda}_{a_1,a_2}(x)\equiv {\sf P}(\Lambda_{a_1,a_2}<x) ={}\\
{}=
\begin{cases}
\displaystyle\fr{a_1 }{a_1+a_2}\, e^{-a_2|x|}\,, & x\leq0\,; \\
\displaystyle 1-\fr{a_2}{a_1+a_2}\, e^{-a_1x}\,, & x> 0\,.
\end{cases}
\end{multline*}
Несложно видеть, что плотность $f^{\Lambda}_{a_1,a_2}(x)$,
соответствующая ф.р.~$F^{\Lambda}_{a_1,a_2}(x)$, имеет вид:
$$
f^{\Lambda}_{a_1,a_2}(x)=
\begin{cases}
\displaystyle\fr{a_1 a_2}{a_1+a_2}\, e^{a_2
x}\,,& x\leqslant 0\,;\\
\displaystyle \fr{a_1 a_2}{a_1+a_2}\, e^{-a_1 x}\,,&
x>0\,.
\end{cases}
$$
Несимметричное распределение Лапласа является популярной моделью,
широко используемой в~разных областях (см., например,~\cite{Kotz2001}). 
Следующая\linebreak лемма, доказательство которой можно
найти, например, в~\cite{KorolevKurmangazievaZeifman2016},
утверждает, что это распределение является специальной
дис\-пер\-си\-он\-но-сдви\-го\-вой смесью нормальных законов.

\smallskip

\noindent
\textbf{Лемма~2.}\ \textit{Пусть $\mu\hm\in\mathbb{R}$, $\sigma^2\hm\in(0,\infty)$,
$\lambda\hm\in(0,\infty)$. Предположим, что с.в.~$Y$ допускает
представление}
$$
Y\eqd \fr{\sigma}{\sqrt{\lambda}}\,
X\sqrt{W_1}+\mu \fr{W_1}{\lambda}\,,
$$
\textit{где с.в.~$X$ имеет стандартное нормальное распределение, 
с.в.~$W_1$ имеет стандартное показательное распределение $($т.\,е.\ 
с.в.~$W_1/\lambda$ имеет показательное распределение с~параметром~$\lambda)$, 
причем с.в.~$X$ и~$W_1$ независимы. Тогда
$Y\eqd\Lambda_{a_1,a_2}$, т.\,е.}
$$
{\sf P}(Y<x)={\sf E}\Phi\left(\fr{\lambda x-\mu
W_1}{\sigma\sqrt{\lambda W_1}}\right)=F^{\Lambda}_{a_1,a_2}(x)\,,\enskip
x\in\mathbb{R}\,,
$$
\textit{где}
$$
a_1=\fr{1}{\sqrt{\mu^2+2\lambda\sigma^2}+\mu}\,;\enskip
a_2=\fr{1}{\sqrt{\mu^2+2\lambda\sigma^2}-\mu}\,.
\label{e14-kz}
$$

\smallskip

Обозначим $Q_{\alpha/2}\eqd\sqrt{S_{\alpha/2,1}/S'_{\alpha/2,1}}$,
где с.в.~$S_{\alpha/2,1}$ и~$S'_{\alpha/2,1}$ независимы и~имеют
одинаковое одностороннее устойчивое распределение 
с~характеристическим показателем~$\alpha/2$. Легко видеть, что
$Q_{\alpha/2}\eqd Q^{-1}_{\alpha/2}$. Теперь из правого равенства~(\ref{e13-kz}), 
(\ref{e11-kz}) и~(\ref{e7-kz}) вытекает, что для любых $\alpha\hm\in(0,2]$ и~$y\hm\geqslant 0$
\begin{multline*}
{\sf P}\left(\Lambda_{a_1,a_2} Q_{\alpha/2}>y\right)= {\sf P}
\left(\Lambda_{a_1,a_2}>y Q_{\alpha/2}\right)={}
\\
{}=
\fr{a_2\sin(\pi\alpha/2)}{\pi(a_1+a_2)}\int\limits_{0}^{\infty}
\fr{z^{\alpha}e^{-a_1yz}\,dz}{1+z^{2\alpha}+2x^{\alpha}\cos(\pi\alpha/2)}={}\\
{}=
1-\fr{a_2}{a_1+a_2}\,F^{L}_{\alpha}\left(a_1y\right)\,,
\end{multline*}
а для $y<0$
\begin{multline*}
{\sf P}\left(\Lambda_{a_1,a_2} Q_{\alpha/2}<y\right)= {\sf P}
\left(\Lambda_{a_1,a_2}<y Q_{\alpha/2}\right)={}
\\{}=
\fr{a_1\sin(\pi\alpha/2)}{\pi(a_1+a_2)}\int\limits_{0}^{\infty}
\fr{z^{\alpha}e^{-a_2|y|z}\,dz}{1+z^{2\alpha}+2x^{\alpha}\cos(\pi\alpha/2)}={}\\
{}=
\fr{a_1}{a_1+a_2}F^{L}_{\alpha}\left(a_2y\right)\,.
\end{multline*}
Таким образом, естественно получено формальное несимметричное
обобщение распределения Линника.

\smallskip

\noindent
\textbf{Определение~1.}\ Пусть $a_1$ и~$a_2$~--- два положительных
числа, $\alpha\hm\in(0,2]$. Будем говорить, что с.в.~$\widehat
L_{\alpha;a_1,a_2}$ имеет \textit{несимметричное распределение Линника
второго рода с~параметрами~$\alpha$, $a_1$ и~$a_2$}, если ее ф.р.\
имеет вид:

\noindent
\begin{multline*}
\widehat F^{L}_{\alpha;a_1,a_2}(x)\equiv {\sf P}\left(
\widehat L_{\alpha;a_1,a_2}<x\right) ={}\\
{}=
\begin{cases}
\displaystyle\fr{a_1 }{a_1+a_2}\, F^{L}_{\alpha}\left(a_2x\right)\,, & x\leq0\,;\\
\displaystyle 1-\fr{a_2}{a_1+a_2}\, F^{L}_{\alpha}\left(a_1x\right)\,, &
x> 0\,.
\end{cases}
\end{multline*}

\smallskip

Необходимо особо отметить, что с~формальной точки зрения
несимметричное распределение Линника второго рода является
специальной сдвиг-мас\-шаб\-ной смесью нормальных законов. Более того,
справедливо следующее утверждение.

\smallskip

\noindent
\textbf{Теорема~1.}\ \textit{Пусть $\alpha\hm\in(0,1]$, $\mu\hm\in\mathbb{R}$,
$\sigma^2\hm\in(0,\infty)$, $\lambda\hm\in(0,\infty)$. Предположим, что с.в.~$Z$ 
допускает представление}
$$
Z\eqd \left(\fr{\sigma}{\sqrt{\lambda}}\, X\sqrt{W_1}+\fr{\mu
W_1}{\lambda}\right) Q_{\alpha/2}\,,
$$
\textit{где с.в.~$X$, $W_1$, $Q_{\alpha/2}$ независимы, с.в.~$X$ имеет
стандартное нормальное распределение, с.в.~$W_1$ имеет стандартное
показательное распределение $($т.\,е.\ с.в.~$W_1/\lambda$ имеет
показательное распределение с~параметром~$\lambda)$. Тогда $Z\eqd
\widehat L_{\alpha;a_1,a_2}$, т.\,е.}
\begin{multline*}
{\sf P}(Z<x)={\sf E}\Phi\left(\fr{\lambda x-\mu W_1
Q_{\alpha/2}}{\sigma\sqrt{\lambda W_1}Q_{\alpha/2}}\right)={}\\
{}={\sf P}
\left(\widehat L_{\alpha;a_1,a_2}<x\right)=\widehat F^{L}_{\alpha;a_1,a_2}(x)\,,\enskip
x\in\mathbb{R}\,,
\end{multline*}
\textit{где $a_1$ и~$a_2$ имеют вид}~(\ref{e14-kz}).

\section{Сходимость распределений дважды случайных сумм независимых одинаково
распределенных случайных величин к~несимметричному распределению
Линника второго рода}

Приведем пример предельной схемы типа <<рандомизированного>> закона
больших чисел для сумм независимых с.в.\ с~\textit{конечными
математическими ожиданиями}, в~которой в~качестве предельных
возникают несимметричные распределения Линника второго рода. Из-за
непростой связи смешива\-ющих с.в.\ в~случайном сдвиге и~случайном
изменении масштаба в~соответствующей смеси (см.\ теорему~1), обычной
схемы случайного суммирования для этой цели недостаточно 
и~приходится вводить\linebreak\vspace*{-12pt}

\columnbreak

\noindent
 в~модель дополнительный источник случайности 
и~рас\-смат\-ри\-вать так называемые \textit{дважды случайные} суммы.
{\looseness=1

}


Пусть $a_1$ и~$a_2$~--- два конечных положительных числа. Пусть
$X^{(1)}_1,X^{(1)}_2,\ldots$~--- независимые одинаково распределенные
с.в.\ такие, что ${\sf E}X^{(1)}_1\hm=a_2^{-1}$,
$X^{(2)}_1,X^{(2)}_2,\ldots$~--- независимые одинаково распределенные
с.в.\ такие, что ${\sf E}X^{(2)}_1\hm=a_1^{-1}$. Тогда по закону
больших чисел
\begin{equation}
\fr{1}{n}\sum\limits_{j=1}^n X_j^{(1)}\Longrightarrow\fr{1}{a_2}\,;\quad
\fr{1}{n}\sum\limits_{j=1}^n X_j^{(2)}\Longrightarrow\fr{1}{a_1}\label{e15-kz}
\end{equation}
при $n\to\infty$.

Пусть $V_{1/n}^{(1)}$ и~$V_{1/n}^{(2)}$~--- с.в.\ с~одинаковым
геометрическим распределением~(\ref{e3-kz}) с~параметром $p\hm=1/n$. Будем
считать, что с.в.~$V_{1/n}^{(1)},
V_{1/n}^{(2)},X^{(1)}_1,X^{(1)}_2,\ldots,X^{(2)}_1,X^{(2)}_2,\ldots$
независимы.\ \  Для каждого $n\hm\in\mathbb{N}$ введем с.в.
$$
Y_n=
\begin{cases}
\displaystyle\sum\limits_{j=1}^{V_{1/n}^{(1)}}X^{(1)}_j
& \mbox{с~вероятностью }
\displaystyle\fr{a_2}{a_1+a_2}\,;\\
\displaystyle -\sum\limits_{j=1}^{V_{1/n}^{(2)}}X^{(2)}_j &
\mbox{с~вероятностью }\displaystyle\fr{a_1}{a_1+a_2}\,.
\end{cases}
$$
Тогда по теореме Реньи из~(\ref{e15-kz}) вытекает, что при $n\hm\to\infty$
\begin{equation}
\fr{Y_n}{n}\Longrightarrow\Lambda_{a_1,a_2}\,.\label{e16-kz}
\end{equation}
Пусть теперь $\alpha\hm\in(0,2]$ и~$N_n$~--- целочисленная
не\-от\-ри\-ца\-тель\-ная с.в., независимая от последовательности
$Y_1,Y_2,\ldots$ и~такая, что
\begin{equation}
\fr{N_n}{n}\Longrightarrow Q_{\alpha/2}\label{e17-kz}
\end{equation}
при $n\to\infty$. Такая с.в.\ может быть построена, например,
следующим образом. Пусть~$P(t)$, $t\geqslant\linebreak \geqslant 0$,~--- стандартный
пуассоновский процесс (пуассоновский процесс с~единичной
интенсивностью),\linebreak независимый от с.в.~$Q_{\alpha/2}$. Положим $N_n\hm=
P(nQ_{\alpha/2})$. Несложно убедиться, что такие с.в.~$N_n$
удовлетворяют~(\ref{e17-kz}).

Тогда по лемме~1 из~(\ref{e16-kz}) и~(\ref{e17-kz}) вытекает, что
$$
\fr{Y_{N_n}}{n}\Longrightarrow\Lambda_{a_1,a_2}
Q_{\alpha/2}\eqd \widehat L_{\alpha;a_1,a_2}\,.
$$

\section{Несимметричные распределения Линника как~дисперсионно-сдвиговые
смеси нормальных законов}

Хотя несимметричное распределение Линника второго рода 
$\widehat F^{L}_{\alpha;a_1,a_2}(x)$, введенное в~предыдущем разделе, является
специальной сдвиг-мас\-штаб-\linebreak ной смесью нормальных законов, пример
предельной схемы, скажем, для сумм случайного числа независимых
с.в., в~которой такое распределение\linebreak возникает 
в~качестве предельного, не так прост, поскольку параметры, по которым
происходит смешивание, связаны нетривиальным образом. 
{\looseness=1

}

Однако
возможен еще один подход к~определению %\linebreak 
несимметричного распределения
Линника, для %\linebreak 
которого подобная предельная схема строится довольно
просто. 
%
Этот подход основан на левом равенстве~(\ref{e13-kz})~--- представлении
распределения Линника в~виде масштабной смеси нормальных %\linebreak
 законов, 
в~которой смешивающим является распределение Мит\-таг--Леф\-фле\-ра. 

В~рамках
описываемого подхода несимметричное обобщение достигается за счет
рассмотрения дис\-пер\-си\-он\-но-сдви\-го\-вых смесей нормальных законов вместо
чисто масштабных смесей.

Вероятностные модели типа дис\-пер\-си\-он\-но-сдви\-го\-вых смесей нормальных
законов рассматриваются в~качестве базовых во многих практических
задачах. 

Подобные модели уже хорошо себя зарекомендовали во многих
исследованиях, где они продемонстрировали очень высокую
адекватность. Последнее обстоятельство можно легко объяснить\linebreak
довольно большим числом настраиваемых па\-ра\-мет\-ров в~указанных
моделях. Однако на самом деле их адекватность имеет гораздо более
глубокие теоретические обоснования, а~именно: дис\-пер\-си\-он\-но-сдви\-го\-вые
смеси нормальных законов являются предельными законами в~довольно
прос\-тых предельных теоремах для случайно остановленных случайных
блуж\-да\-ний. 

Такие теоремы позволяют однозначно связать конкретный
смешивающий закон в~дис\-пер\-си\-он\-но-сдви\-го\-вых смесях с~поведением
интенсивности потока информативных событий, в~результате которых
накапливаются данные, характеризующие анализируемый случайный
процесс. Тем самым эти теоремы как бы позволяют разделить вклады
внешних и~внутренних факторов в~случайность поведения анализируемого
процесса.

\columnbreak

Понятие дисперсионно-сдвиговой смеси нормальных законов (normal
variance-mean mixture) введено в~1970--1980-х~гг.\
 в~работах О.-Е.~Барн\-дорфф-Ниль\-се\-на и~его 
коллег~[43--45] как довольно гибкое обобщение
нормального распреде\-ления.
{\looseness=1

}

Пусть $\beta\in\mathbb{R}$, $\alpha\hm\in\mathbb{R}$,
$0\hm<\sigma\hm<\infty$, $A(x)$~--- функция распределения, все точки роста
которой сосредоточены на~$\mathbb{R}_+$. Дис\-пер\-си\-он\-но-сдви\-го\-вой
смесью нормальных законов называется ф.р.
\begin{equation}
F(x)=\int\limits_{0}^{\infty}\Phi\left(\fr{x-\beta-\alpha
z}{\sigma\sqrt{z}}\right)\,dA(z)\,,\enskip x\in\mathbb{R}\,.
\label{e18-kz}
\end{equation}

Обратим внимание, что в~соотношении~(\ref{e18-kz}) смешивание происходит
одновременно и~по па\-ра\-мет\-ру сдвига, и~по параметру масштаба, но так
как эти параметры в~(\ref{e18-kz}) связаны жесткой за\-ви\-си\-мостью, при которой
параметры положения (\textit{сдвига}) смешиваемых нормальных законов
пропорциональны их \textit{дисперсиям}, то фактически смесь~(\ref{e18-kz})
является однопараметрической. Именно поэтому смеси вида~(\ref{e18-kz})
называются \textit{дис\-пер\-си\-он\-но-сдви\-го\-выми}. 
{\looseness=1

}

Класс
дис\-пер\-си\-он\-но-сдви\-го\-вых смесей нор\-маль\-ных законов обширен и~содержит,
в~част\-ности, обобщенные гиперболические законы \mbox{[43--45]}
и~обобщенные дисперсионные гам\-ма-рас\-пре\-де\-ле\-ния~\cite{KorolevSokolov2012, KorolevZaks2013}, 
демонстрирующие отличное
согласие со статистическими данными из самых разных областей~--- от
атмосферной турбулентности до финансовых рынков.

Без существенного ограничения общности для простоты далее будем
считать, что $\beta\hm=0$.

\bigskip

\noindent
\textbf{Определение~2.}\ Пусть $\mu\hm\in\mathbb{R}$, $\sigma\hm>0$,
$\alpha\hm\in(0,2]$. Будем говорить, что с.в.~$\widetilde L_{\alpha;\mu,\sigma}$ 
имеет несимметричное распределение Линника
третьего рода с~параметрами~$\alpha$, $\mu$ и~$\sigma$, если ее 
ф.р.~$\widetilde F^{L}_{\alpha;\mu,\sigma}(x)$ является
дис\-пер\-си\-он\-но-сдви\-го\-вой смесью нормальных законов вида:
$$
\widetilde F^{L}_{\alpha;\mu,\sigma}(x)=
\int\limits_{0}^{\infty}\Phi\left(\fr{x-\mu z}
{\sigma\sqrt{z}}\right)\,dF^{M}_{\alpha/2}(z)\,,\enskip x\in\mathbb{R}\,,
$$
где смешивающая ф.р.\ Мит\-таг--Леф\-фле\-ра~$F^{M}_{\alpha/2}$ имеет вид~(\ref{e10-kz}).

\smallskip

Из представления~(\ref{e13-kz}) вытекает, что при $\mu\hm=0$ несимметричное
распределение Линника третьего рода превращается в~обычное
симметричное распределение Линника.

\section{Сходимость распределений случайных сумм независимых одинаково
распределенных случайных величин к~несимметричному распределению
Линника третьего рода}

Пусть $\{X_{n,j}\}_{j\geqslant1}$, $n\hm=1,2,\ldots$,~--- последовательность
серий одинаково в~каждой серии распределенных с.в. Пусть~$\{N_n\}_{n\geqslant1}$~--- 
последовательность неотрицательных
целочисленных с.в.\ таких, что при каждом $n\hm\geqslant1$ 
с.в.~$N_n,X_{n,1},X_{n,2},\ldots$ независимы. Напомним, что используется
обозначение $S_{n,k}\hm=X_{n,1}+\cdots +X_{n,k}$, $n,k\in\mathbb{N}$. 
В~работе~\cite{Korolev2013} доказано следующее утверждение.

\smallskip

\noindent
\textbf{Лемма~3.}\ \textit{Предположим, что существуют последовательность
натуральных чисел~$\{k_n\}_{n\geqslant 1}$ и~числа $\mu\hm\in\mathbb{R}$ 
и~$\sigma\hm\in(0,\infty)$ такие, что}
\begin{equation}
{\sf P}\left(S_{n,k_n}<x\right)\Longrightarrow
\Phi\left(\fr{x-\mu}{\sigma}\right)\,.\label{e19-kz}
\end{equation}
\textit{Предположим, что $N_n\hm\to\infty$ по вероятности. Тогда распределения
случайных сумм~$S_{N_n}$ независимых одинаково распределенных с.в.\
слабо сходятся к~некоторой ф.р.~$F(x)$}:
$$
{\sf P}\left(S_{n,N_n}<x\right)\Longrightarrow F(x)\,,
$$
\textit{если и~только если существует ф.р.~$H(x)$ такая, что $H(0)\hm=0$},

\noindent
\begin{gather*}
F(x)=\int\limits_{0}^{\infty}\Phi\left(\fr{x-\mu z}{\sigma\sqrt{z}}\right)\,dH(z)\,;
\\
%\textit{и}
%$$
{\sf P}\left(N_n<xk_n\right)\Longrightarrow H(x)\,.
\end{gather*}


Из леммы~3 и~определения~2 непосредственно вытекает следующее
утверждение, устанав\-ли\-ва\-ющее необходимые и~достаточные условия
сходимости распределений случайных сумм независимых одинаково
распределенных с.в.\ c~\textit{конечными дис\-пер\-си\-ями} к~несимметричному
распределению Линника третьего рода.

\smallskip

\noindent
\textbf{Теорема~2.}\ \textit{Предположим, что существуют последовательность
натуральных чисел~$\{k_n\}_{n\geqslant1}$ и~числа $\mu\hm\in\mathbb{R}$ 
и~$\sigma\hm>0$ такие, что имеет место сходимость}~(\ref{e19-kz}). \textit{Предположим,
что $N_n\hm\to\infty$ по вероятности. Тогда распределения случайных
сумм~$S_{N_n}$ независимых одинаково распределенных с.в.\ слабо
сходятся к~несимметричному распределению Линника третьего рода
$\widetilde F^{L}_{\alpha;\mu,\sigma}$ при некотором}
$\alpha\hm\in(0,2]$:
$$
{\sf P}\left(S_{n,N_n}<x\right)\Longrightarrow \widetilde
F^{L}_{\alpha;\mu,\sigma}(x)\,,
$$

\noindent
\textit{если и~только если}
$$
{\sf P}\left(N_n<xk_n\right)\Longrightarrow F^{M}_{\alpha/2}(x)\,,
$$
\textit{где ф.р.\ Мит\-таг--Леф\-фле\-ра~$F^{M}_{\alpha/2}$ имеет вид}~(\ref{e10-kz}).

\vspace*{3pt}

Примеры индексов~$N_n$, удовлетворяющих условию теоремы~2, приведены
в~\cite{KorolevZeifman2016b}.

\vspace*{-12pt}


{\small\frenchspacing
 {%\baselineskip=10.8pt
 \addcontentsline{toc}{section}{References}
 \begin{thebibliography}{99}
\bibitem{MittnikRachev1991} 
\Au{Mittnik S., Rachev S.} Modeling asset returns with alternative stable models~// 
Economet. Rev., 1993. Vol.~12. P.~261--330.

\bibitem{Kotz2001} 
\Au{Kotz S., Kozubowski~T.\,J., Podgorski~K.} 
The Laplace distribution and generalizations: A~revisit with 
applications to communications, economics, engineering, and finance.~--- 
Boston: Birkhauser, 2001. 349~p.

\bibitem{KorolevZeifman2016a}
\Au{Korolev V.\,Yu., Zeifman~A.\,I.} 
A~note on mixture representations for the Linnik and Mittag--Leffler 
distributions and their applications~// J.~Math. Sci., 2016. Vol.~218. P.~314--327.

\bibitem{KorolevZeifman2016b} 
\Au{Korolev V.\,Yu., Zeifman~A.\,I.} 
Convergence of random sums and statistics constructed from samples with random sizes to the Linnik 
and Mittag--Leffler distributions and their generalizations~// 
J.~Korean Stat. Soc., 2016.  
arXiv:1602.02480v1.

\bibitem{Zolotarev1983} 
\Au{Золотарев В.\,М.} Одномерные устойчивые распределения.~--- М.: Наука, 1983.
304~с.

\bibitem{Schneider1986} 
\Au{Schneider W.\,R.} Stable distributions: Fox function representation and 
generalization~// Stochastic processes in classical and quantum systems~/ 
Eds. S.~Albeverio, G.~ Casati, D.~Merlini.~--- Berlin: Springer, 1986. P.~497--511.

\bibitem{UchaikinZolotarev1999} 
\Au{Uchaikin V.\,V., Zolotarev~V.\,M.} Chance and stability.~--- Utrecht: VSP, 1999.
570~p.

\bibitem{KorolevWeibull2016} 
\Au{Korolev V.\,Yu.} Product representations for random variables with the
 Weibull distributions and their applications~// J.~Math. Sci., 2016. Vol.~218. No.\,3. P.~298--313.

\bibitem{Tucker1975} 
\Au{Tucker H.} On moments of distribution functions attracted to stable laws~// 
Houston J.~Math., 1975. Vol.~1. No.\,1. P.~149--152.

\bibitem{KlebanovManiaMelamed1984} 
\Au{Клебанов Л.\,Б., Мания~Г.\,М., Меламед~И.\,А.} Одна задача В.\,М.~Золотарева 
и~аналоги безгранично делимых и~устойчивых распределений в~схеме суммирования 
случайного числа случайных величин~// Теория вероятностей и~ее применения, 1984. 
Т.~29. Вып.~4. С.~791--794.

\bibitem{KlebanovRachev1996} 
\Au{Klebanov L.\,B., Rachev~S.\,T.} Sums of a~random number of random variables 
and their approximations with $\varepsilon$-accompanying infinitely divisible laws~// 
Serdica, 1996. Vol.~22. P.~471--498.

\bibitem{Bunge1996} 
\Au{Bunge J.} Compositions semigroups and random stability~// 
Ann. Probab., 1996. Vol.~24. P.~1476--1489.

%\pagebreak

\bibitem{Rachev1991} 
\Au{Rachev S.\,T.} Probability metrics and the stability of stochastic models.~--- 
 Chichester--New York: Wiley, 1991. 494~p.
 
 \pagebreak

\bibitem{GnedenkoKorolev1996} 
\Au{Gnedenko B.\,V., Korolev~V.\,Yu.} Random summation: Limit theorems and 
applications.~--- Boca Raton: CRC Press, 1996. 267~p.

\bibitem{Korolev1994} 
\Au{Королев В.\,Ю.} Сходимость случайных последовательностей с~независимыми случайными 
индексами.~I~// Теория вероятностей и~ее применения, 1994. Т.~39. Вып.~2. С.~313--333.

\bibitem{Korolev1995} 
\Au{Королев В.\,Ю.} Сходимость случайных последовательностей с~независимыми 
случайными индексами.~II~// Теория вероятностей и~ее применения, 1995. Т.~40. Вып.~4. С.~907--910.

\bibitem{BeningKorolev2002} 
\Au{Bening V.\,E., Korolev~ V\, Yu.} 
Generalized Poisson models and their applications in insurance and finance.~--- 
Utrecht: VSP, 2002. 434~p.

\bibitem{Pillai1985} 
\Au{Pillai R.\,N.}  Semi-$\alpha$-Laplace distributions~//
Commun. Stat. Theor. Meth., 1985. Vol.~14. P.~991--1000.

\bibitem{Linnik1953} 
\Au{Линник Ю.\,В.} Линейные формы и~статистические критерии.~I,~II~// 
Украинский математический журнал, 1953. Т.~5. Вып.~2. С.~207--243; 
Вып.~3. С.~247--290.

\bibitem{Laha1961} 
\Au{Laha R.\,G. } On a class of unimodal distributions~// 
Proc. Am. Math. Soc., 1961. Vol.~12. P.~181--184.

\bibitem{Lukacs1970} 
\Au{Лукач Е.} Характеристические функции.~--- М.: Наука, 1979. 424~с.

\bibitem{KotzOstrovskiiHayfavi1995a} 
\Au{Kotz S., Ostrovskii~I.\,V., Hayfavi~A.} Analytic and asymptotic properties 
of Linnik's probability densities,~I~// J.~Math. Anal. Appl., 1995. Vol.~193. P.~353--371.

\bibitem{KotzOstrovskiiHayfavi1995b} 
\Au{Kotz S., Ostrovskii~I.\,V., Hayfavi~A.} 
Analytic and asymptotic properties of Linnik's probability densities,~II~// 
J.~Math. Anal. Appl., 1995. Vol.~193. P.~497--521.

\bibitem{SabuPillai1987} 
\Au{Sabu G., Pillai~R.\,N.} Multivariate $\alpha$-Laplace distributions~// 
J.~Nat. Acad. Math., 1987. Vol.~5. P.~13--18.

\bibitem{Lin1994} 
\Au{Lin G.\,D.} Characterizations of the Laplace and related distributions 
via geometric compound~// Sankhya, A1, 1994. Vol.~56. P.~1--9.

\bibitem{Anderson1992} 
\Au{Anderson D.\,N.} A~multivariate Linnik distribution~// 
Stat. Probabil. Lett., 1992. Vol.~14. P.~333--336.

\bibitem{Devroye1990} 
\Au{Devroye L.} A~note on Linnik's distribution~// 
Stat. Probabil. Lett., 1990. Vol.~9. P.~305--306.

\bibitem{Jacquesetal1999} 
\Au{Jacques C., \mbox{R{\!\ptb{{\`{e}}}}millard}~ B., Theodorescu~R.} 
Estimation of Linnik law parameters~// Stat. Decision, 1999. Vol.~17. No.\,3. P.~213--236.


\bibitem{KotzOstrovskii1996} 
\Au{Kotz S., Ostrovskii~I.\,V.} A~mixture representation of the Linnik distribution~// 
Stat. Probabil. Lett., 1996. Vol.~26. P.~61--64.

\bibitem{Pakes1998} 
\Au{Pakes A.\,G.} Mixture representations for symmetric generalized Linnik laws~// 
Stat. Probabil. Lett., 1998. Vol.~37. P.~213--221.

\bibitem{Kilbas2014} 
\Au{Gorenflo R., Kilbas~A.\,A., Mainardi~F., Rogosin~S.\,V.} 
Mittag--Leffler functions, related topics and applications.~--- Berlin\,--\,New York: Springer, 2014.
420~p.

\bibitem{Kovalenko1965} 
\Au{Kovalenko I.\,N.} On the class of limit distributions for rarefied flows 
of homogeneous events~// Lith. Math.~J., 1965. Vol.~5. No.\,4. P.~569--573.

\bibitem{GnedenkoKovalenko1968} 
\Au{Gnedenko B.\,V., Kovalenko~I.\,N.} Introduction to queueing theory.~--- 
Jerusalem: Israel Program for Scientific Translations, 1968. 281~p.

\bibitem{GnedenkoKovalenko1989} 
\Au{Gnedenko B.\,V., Kovalenko~I.\,N.} Introduction to queueing theory.~--- 2nd ed.~--- 
Boston: Birkhauser, 1989. 314~p.

\bibitem{Pillai1989} 
\Au{Pillai R.\,N.} Harmonic mixtures and geometric infinite divisibility~// 
J.~Indian Stat. Ass., 1990. Vol.~28. P.~87--98.

\bibitem{Pillai1990} 
\Au{Pillai R.\,N.} On Mittag--Leffler functions and related distributions~// 
Ann. Inst. Stat. Math., 1990. Vol.~42. P.~157--161.

\bibitem{WeronKotulski1996} 
\Au{Weron K., Kotulski~M.} On the Cole--Cole relaxation function and related 
Mittag--Leffler distributions~// Physica A, 1996. Vol.~232. P.~180--188.

\bibitem{GorenfloMainardi2006} 
\Au{Gorenflo R., Mainardi~F.} Continuous time random walk, Mittag--Leffler 
waiting time and fractional diffusion: Mathematical aspects~// 
 Anomalous transport: Foundations and applications~/ Eds. 
 R.~Klages, G.~Radons, I.\,M.~Sokolov.~--- Weinheim, Germany: Wiley-VCH, 2008. P.~93--127. http://arxiv.org/abs/0705.0797.

\bibitem{GnedenkoFahim1969} 
\Au{Гнеденко Б.\,В., Фахим~Х.} Об одной теореме переноса~// Докл. АН СССР, 1969. Т.~187. Вып.~1. С.~15--17.

\bibitem{Renyi1956} 
\Au{R$\acute{\mbox{e}}$nyi A. } A~Poisson-folyamat egy jellemzese~// 
Maguar Tud. Acad. Mat. Int. Kozl., 1956. Vol.~1. P.~519--527.

\bibitem{LimTeo2009} 
\Au{Lim S.\,C., Teo~L.\,P.} Analytic and asymptotic properties of multivariate
 generalized Linnik's probability densities~// 
 J.~Fourier Anal. Appl., 2010. Vol.~16. P.~715--747.

\bibitem{KorolevKurmangazievaZeifman2016} 
\Au{Korolev V.\,Yu., Kurmangazieva~L., Zeifman~A.\,I.} On asymmetric generalization 
of the Weibull distribution by scale-location mixing of normal laws~// J.~Korean 
Stat. Soc., 2016. Vol.~45. P.~238--249. arXiv:1506.06232.

\bibitem{BN1977} 
\Au{Barndorff-Nielsen O.-E.} Exponentially decreasing distributions for the 
logarithm of particle size~// Proc. Roy. Soc. Lond. A, 1977. Vol.~A(353). P.~401--419.


\bibitem{BN1982} 
\Au{Barndorff-Nielsen O.-E., Kent~J., \mbox{S\!{\!\ptb{\o}}\,rensen}~M.} 
Normal variance-mean mixtures and $z$-distributions~// 
Int. Stat. Rev., 1977. Vol.~50. No.\,2. P.~145--159.

\bibitem{BN1978} 
\Au{Barndorff-Nielsen O.-E.} Hyperbolic distributions and distributions of hyperbolae~// 
Scand. J.~Stat., 1978. Vol.~5. P.~151--157.


\bibitem{KorolevSokolov2012} 
\Au{Королев В.\,Ю., Соколов~И.\,А.} Скошенные распределения Стьюдента,
 дисперсионные гам\-ма-рас\-пре\-де\-ле\-ния и~их обобщения как асимптотические 
 аппроксимации~// Информатика и~её применения, 2012. Т.~6. Вып.~1. С.~2--10.

\bibitem{KorolevZaks2013} 
\Au{Закс Л.\,М., Королев~В.\,Ю.} Обобщенные дисперсионные гамма-распределения 
как предельные для случайных сумм~// Информатика и~её применения, 2013. Т.~7. Вып.~1. С.~105--115.

\bibitem{Korolev2013}
\Au{Королев В.\,Ю.} Обобщенные гиперболические распределения как предельные 
для случайных сумм~// Теория вероятностей и~ее применения, 2013. Т.~58. Вып.~1. С.~117--132.

\bibitem{ErdoganOstrovskii1997} 
\Au{\mbox{Erdo\!{\!\ptb{\v{g}}}an}~M.\,B., Ostrovskii~I.\,V.} 
Analytic and asymptotic properties of generalized Linnik probability densities~//
J.~Math. Anal. Appl., 1998. Vol.~217. P.~555--578.

\bibitem{ErdoganOstrovskii1998} 
\Au{\mbox{Erdo\!{\!\ptb{\v{g}}}an}~M.\,B., Ostrovskii~I.\,V.} On mixture representation
 of the Linnik density~// J.~Aust. Math. Soc.~A, 1998. Vol.~64. P.~317--326.
 
 \pagebreak

\bibitem{Kalashnikov1997} 
\Au{Kalashnikov V.\,V.} Geometric sums: Bounds for rare events with applications.~--- 
Dordrecht: Kluwer Academic Publs., 1997. 270~p.

\bibitem{Pakes1992} 
\Au{Pakes A.\,G.} A~ characterization of gamma mixtures of stable laws motivated by 
limit theorems~// Stat. Neerl., 1992. Vol.~2-3. P.~209--218.

\bibitem{Kozubowski1998} 
\Au{Kozubowski T.\,J.} Mixture representation of Linnik distribution revisited~// 
Stat. Probabil. Lett., 1998. Vol.~38. P.~157--160.

\bibitem{Kozubowski1999} 
\Au{Kozubowski T.\,J.} Exponential mixture representation 
of geometric stable distributions~// 
Ann. Inst. Stat. Math., 1999. Vol.~52. No.\,2. P.~231--238.
 \end{thebibliography}

 }
 }

\end{multicols}

\vspace*{-6pt}

\hfill{\small\textit{Поступила в~редакцию 14.10.16}}

\vspace*{8pt}

%\newpage

%\vspace*{-24pt}

\hrule

\vspace*{2pt}

\hrule

\vspace*{8pt}


\def\tit{ASYMMETRIC LINNIK DISTRIBUTIONS AS~LIMIT LAWS FOR~RANDOM SUMS 
OF~INDEPENDENT RANDOM VARIABLES WITH~FINITE VARIANCES}

\def\titkol{Asymmetric Linnik distributions as limit laws for random sums of 
independent random variables with finite variances}

\def\aut{V.\,Yu.~Korolev$^{1,2}$, A.\,I.~Zeifman$^{1,2,3,4}$, and~A.\,Yu.~Korchagin$^1$}

\def\autkol{V.\,Yu.~Korolev, A.\,I.~Zeifman, and~A.\,Yu.~Korchagin}

\titel{\tit}{\aut}{\autkol}{\titkol}

\vspace*{-9pt}


\noindent
$^1$Faculty of Computational Mathematics and Cybernetics, 
M.\,V.~Lomonosov Moscow State University, 1-52~Lenin-\linebreak
$\hphantom{^1}$skiye Gory, GSP-1, 
Moscow 119991, Russian Federation

\noindent
$^2$Institute of Informatics 
Problems, Federal Research Center ``Computer Science and Control'' 
of the Russian\linebreak
$\hphantom{^1}$Academy of Sciences, 44-2~Vavilov Str., Moscow 119333,  
Russian Federation

\noindent
$^3$Vologda State University, 15~Lenin Str., Vologda 160000, Russian Federation

\noindent
$^4$ISEDT RAS, 56-A~Gorky Str., Vologda 160001, 
Russian Federation

\def\leftfootline{\small{\textbf{\thepage}
\hfill INFORMATIKA I EE PRIMENENIYA~--- INFORMATICS AND
APPLICATIONS\ \ \ 2016\ \ \ volume~10\ \ \ issue\ 4}
}%
 \def\rightfootline{\small{INFORMATIKA I EE PRIMENENIYA~---
INFORMATICS AND APPLICATIONS\ \ \ 2016\ \ \ volume~10\ \ \ issue\ 4
\hfill \textbf{\thepage}}}

\vspace*{3pt}


\Abste{Linnik distributions (symmetric geometrically stable distributions) 
are widely applied in financial mathematics, telecommunication systems modeling, 
astrophysics, and genetics. These distributions are limiting for geometric 
sums of independent identically distributed random variables whose distribution 
belongs to the domain of normal attraction of a~symmetric strictly stable distribution. 
In the paper, three asymmetric generalizations of the Linnik distribution are 
considered. The traditional (and formal) approach to the asymmetric generalization 
of the Linnik distribution consists in the consideration of geometric 
sums of random summands whose distributions are attracted to an asymmetric 
strictly stable distribution. The variances of such summands are infinite. 
Since in modeling real phenomena, as a~rule, there are no solid reasons to reject 
the assumption of the finiteness of the variances of elementary summands, in 
the paper, two alternative asymmetric generalizations are proposed based 
on the representability of the Linnik distribution as a~scale mixture of normal 
laws or a~scale mixture of Laplace laws. Examples are presented of limit theorems 
for sums of a~random number of independent random variables with finite variances 
in which the proposed asymmetric Linnik distributions appear as limit laws.}

\KWE{Linnik distribution; Laplace distribution; Mittag--Leffler distribution; 
normal distribution; scale mixture; normal variance-mean mixture; 
stable distribution; geometrically stable distribution}


\DOI{10.14357/19922264160403}  

\vspace*{-9pt}

\Ack
\noindent
This work was financially supported by the Russian Science Foundation 
(grant No.\,14-11-00364).



%\vspace*{3pt}

  \begin{multicols}{2}

\renewcommand{\bibname}{\protect\rmfamily References}
%\renewcommand{\bibname}{\large\protect\rm References}

{\small\frenchspacing
 {%\baselineskip=10.8pt
 \addcontentsline{toc}{section}{References}
 \begin{thebibliography}{99}

\bibitem{1-kz}
\Aue{Mittnik, S., and~S.~Rachev}. 1993. Modeling asset returns with alternative 
stable models. \textit{Economet. Rev.} 12:261--330.
\bibitem{2-kz}
\Aue{Kotz, S., T.\,J.~Kozubowski, and K.~Podgorski}. 2001. \textit{The 
Laplace distribution and generalizations: A~revisit with applications to communications, 
economics, engineering, and finance}. Boston: Birkhauser. 349~p. 
\bibitem{3-kz}
\Aue{Korolev, V.\,Yu., and A.\,I.~Zeifman}. 2016. 
A~note on mixture representations for the Linnik and Mittag--Leffler 
distributions and their applications. \textit{J.~Math. Sci}. 218(3):314--327. 
\bibitem{4-kz}
\Aue{Korolev, V.\,Yu., and A.\,I.~Zeifman}. 2016. Convergence of random
sums and statistics constructed 
from samples with random sizes to the Linnik and Mittag--Leffler distributions and 
their generalizations. \textit{J.~Korean Stat. Soc}. 
Available at: arXiv:1602.02480v1 (accessed December~10, 2016). 
\bibitem{5-kz}
\Aue{Zolotarev, V.\,M.} 1986. \textit{One-dimensional stable distributions}. 
Providence: AMS. 284~p.
\bibitem{6-kz}
\Aue{Schneider, W.\,R.} 1986. Stable distributions: Fox function representation and 
generalization. \textit{Stochastic processes in classical and quantum systems}. 
Eds.\ S.~Albeverio, G.~Casati, and D.~Merlini. Berlin: Springer.  497--511. 
\bibitem{7-kz}
\Aue{Uchaikin, V.\,V., and V.\,M.~Zolotarev}. 1999. 
\textit{Chance and stability}. Utrecht: VSP. 570~p. 
\bibitem{8-kz}
\Aue{Korolev, V.\,Yu.} 2016. Product representations for random variables with the 
Weibull distributions and their applications. \textit{J.~Math. Sci}. 218(3):298--313. 
\bibitem{9-kz}
\Aue{Tucker, H.} 1975. On moments of distribution functions attracted to stable laws. 
\textit{Houston J.~Math.} 1(1):149--152. 
\bibitem{10-kz}
\Aue{Klebanov, L.\,B., G.\,M.~Maniya, and I.\,A.~Melamed}. 1985. 
A~problem of Zolotarev and analogs of infinitely 
divisible and stable distributions in a~scheme for summing a~random 
number of random variables. \textit{Theor. Probab. Appl.} 29(4):791--794.
\bibitem{11-kz}
\Aue{Klebanov, L.\,B., and S.\,T.~Rachev}. 1996. Sums of a~random number 
of random variables and their approximations with $\varepsilon$-accompanying 
infinitely divisible laws. \textit{Serdica} 22:471--498. 
\bibitem{12-kz}
\Aue{Bunge, J.} 1996. Compositions semigroups and random stability. 
\textit{Ann. Probab.} 24:1476--1489. 
\bibitem{13-kz}
\Aue{Rachev, S.\,T.} 1991. \textit{Probability metrics and the stability of stochastic 
models}. Chichester\,--\,New York: Wiley. 494~p.
\bibitem{14-kz}
\Aue{Gnedenko, B.\,V., and V.\,Yu.~Korolev}. 1886. \textit{Random 
summation: Limit theorems and applications}. Boca Raton: CRC Press. 267~p.
\bibitem{15-kz}
\Aue{Korolev, V.\,Yu.} 1995. 
Convergence of random sequences with the independent random 
indices~I. \textit{Theor. Probab. Appl.} 39(2):282--297. 
\bibitem{16-kz}
\Aue{Korolev, V.\,Yu.} 1996. Convergence of random sequences with the independent random 
indices~II. \textit{Theor. Probab. Appl.} 40(4):770--772. 

\bibitem{17-kz}
\Aue{Bening, V.\,E., and V.\,Yu.~Korolev}. 2002. 
\textit{Generalized Poisson models and their applications in insurance and finance}. 
Utrecht: VSP. 434~p.
\bibitem{18-kz}
\Aue{Pillai, R.\,N.} 1985. Semi-$\alpha$-Laplace distributions.
\textit{Commun. Stat. Theor. Meth.} 14:991--1000. 
\bibitem{19-kz}
\Aue{Linnik, Yu.\,V.} 1953. Lineynye formy i~statisticheskiye kriterii. I, II 
[Linear forms and statistical criteria. I, II]. 
\textit{Ukr. Math.~J.} 5(2):207--243; 3:247--290. 


\bibitem{20-kz}
\Aue{Laha, R.\,G.} 1961. On a~class of unimodal distributions. 
\textit{Proc. Am. Math. Soc.} 12:181--184. 
\bibitem{21-kz}
\Aue{Lukacs, E.} 1970. \textit{Characteristic functions}. 2nd ed. 
London: Griffin. 350~p. 

\bibitem{22-kz}
\Aue{Kotz, S., I.\,V.~Ostrovskii, and A.~Hayfavi}. 1995. 
Analytic and asymptotic properties of Linnik's probability densities, I. 
\textit{J.~Math. Anal. Appl.} 193:353--371. 
\bibitem{23-kz}
\Aue{Kotz, S., I.\,V.~Ostrovskii, and A.~Hayfavi}. 1995. 
Analytic and asymptotic properties of Linnik's probability densities, II. 
\textit{J.~Math. Anal. Appl.} 193:497--521. 
\bibitem{24-kz}
\Aue{Sabu, G., and R.\,N.~Pillai}. 1987. Multivariate $\alpha$-Laplace distributions. 
\textit{J.~Nat. Acad. Math.} 5:13--18. 
\bibitem{25-kz}
\Aue{Lin, G.\,D.} 1994. Characterizations of the Laplace and related distributions 
via geometric compound. \textit{Sankhya, A1} 56:1--9. 
\bibitem{26-kz}
\Aue{Anderson, D.\,N.} 1992. A~multivariate Linnik distribution. 
\textit{Stat. Probabil. Lett.} 14:333--336. 
\bibitem{27-kz}
\Aue{Devroye, L.} 1990. A~note on Linnik's distribution. 
\textit{Stat. Probabil. Lett}. 9:305--306. 
\bibitem{28-kz}
\Aue{Jacques, C., B.~\mbox{R{\!\ptb{\`{e}}}emillard}, and R.~Theodorescu}. 
1999. Estimation of Linnik law parameters. \textit{Stat. Decision} 17(3):213--236. 

\bibitem{30-kz}
\Aue{Kotz, S., and I.\,V.~Ostrovskii}. 1996. A~mixture representation of the 
Linnik distribution. \textit{Stat. Probabil. Lett.} 26:61--64. 
\bibitem{29-kz}
\Aue{Pakes, A.\,G.} 1998. Mixture representations for symmetric generalized Linnik 
laws. \textit{Stat. Probabil. Lett.} 37:213--221.  

\bibitem{31-kz}
\Aue{Gorenflo, R., A.\,A.~Kilbas, F.~Mainardi, and S.\,V.~Rogosin}. 
2014. \textit{Mittag--Leffler functions, related topics and applications}. 
Berlin\,--\,New York: Springer. 420~p.
\bibitem{32-kz}
\Aue{Kovalenko, I.\,N.} 1965. On the class of limit distributions for rarefied flows 
of homogeneous events. \textit{Lith. Math.~J.} 5(4):569--573. 



\bibitem{33-kz}
\Aue{Gnedenko, B.\,V., and I.\,N.~Kovalenko}. 1968. 
\textit{Introduction to queueing theory}. 
Jerusalem: Israel Program for Scientific Translations. 281~p. 
\bibitem{34-kz}
\Aue{Gnedenko, B.\,V., and I.\,N.~Kovalenko}. 
1989. \textit{Introduction to queueing theory}. 2nd ed. Boston: Birkhauser. 314~p.
\bibitem{35-kz}
\Aue{Pillai, R.\,N.} 1990. Harmonic mixtures and geometric infinite divisibility. 
\textit{J.~Indian Stat. Ass.} 28:87--98. 
\bibitem{36-kz}
\Aue{Pillai, R.\,N.} 1990. On Mittag--Leffler functions and related distributions. 
\textit{Ann. Inst. Stat. Math.} 42:157--161. 
\bibitem{37-kz}
\Aue{Weron, K., and M.~Kotulski}. 1996. On the Cole--Cole relaxation function 
and related Mittag--Leffler distributions. \textit{Physica A} 232:180--188. 
\bibitem{38-kz}
\Aue{Gorenflo, R., and F.~Mainardi}. 2008. Continuous time random walk, 
Mittag--Leffler waiting time and fractional diffusion: Mathematical aspects. 
\textit{Anomalous transport: Foundations and applications}. Eds.\
R.~Klages, G.~Radons, and I.\,M.~Sokolov. Weinheim, Germany: Wiley-VCH. 93--127. 
\bibitem{39-kz}
\Aue{Gnedenko, B.\,V., and H.~Fahim}. 1969. Ob odnoy teoreme perenosa 
[About one transfer theorem]. \textit{Dokl. USSR Akad. Nauk} 187(1):15--17. 
\bibitem{40-kz}
\Aue{R$\acute{\mbox{e}}$enyi, A.} 1956. A~Poisson-folyamat egy jellemzese. 
\textit{Maguar Tud. Acad. Mat. Int. Kozl.} 1:519--527. 
\bibitem{41-kz}
\Aue{Lim, S.\,C., and L.\,P.~Teo}. 2010. Analytic and asymptotic properties 
of multivariate generalized Linnik's probability densities. 
\textit{J.~Fourier Anal. Appl.} 16:715--747. 
\bibitem{42-kz}
\Aue{Korolev, V.\,Yu., L.~Kurmangazieva, and A.\,I.~Zeifman}. 2016.
 On asymmetric generalization of the Weibull distribution by scale-location mixing 
 of normal laws. \textit{J.~Korean Stat. Soc.} 45:238--249. arXiv:1506.06232. 
\bibitem{43-kz}
\Aue{Barndorff-Nielsen, O.\,E.} 1977. Exponentially decreasing distributions for the 
logarithm of particle size. \textit{Proc. Roy. Soc. Lond. A} 353:401--419. 

\bibitem{45-kz}
\Aue{Barndorff-Nielsen, O.\,E., J.~Kent, and M.~\mbox{S\!{\ptb{\o}}rensen}.} 
1982. Normal variance-mean mixtures and \mbox{$z$-distributions}. 
\textit{Int. Stat. Rev.} 50(2):145--159. 

\pagebreak

\bibitem{44-kz}
\Aue{Barndorff-Nielsen, O.\,E.} 1978. Hyperbolic distributions and distributions 
of hyperbolae. \textit{Scand. J.~Stat.} 5:151--157. 
\bibitem{46-kz}
\Aue{Korolev, V.\,Yu., and I.\,A.~Sokolov}. 2012. Skoshennye raspredeleniya 
St'yudenta, dispersionnye gamma-raspredeleniya i~ikh obobshcheniya kak asimptoticheskie 
approksimatsii [Skewed Student's distributions, variance gamma distributions, 
and their generalizations as asymptotic approximations]. 
\textit{Informatika i~ee Primeneniya~--- Inform. Appl.} 6(1):2--10. 
\bibitem{47-kz}
\Aue{Zaks, L.\,M., and V.\,Yu.~Korolev}. 2013. Obobshchennye dispersionnye 
gamma-raspredeleniya kak predel'nye dlya sluchaynykh summ 
[Generalized variance gamma distributions as limiting for random sums]. 
\textit{Informatika i~ee Primeneniya~--- Inform. Appl.} 7(1):105--115. 

\bibitem{48-kz}
\Aue{Korolev, V.\,Yu.} 2014. Generalized hyperbolic laws as limit
distributions for random sums. \textit{Theor. Probab. Appl.} 58(1):63--75.

%\columnbreak

\bibitem{49-kz}
\Aue{\mbox{Erdo\!\!{\ptb{\v{g}}}an}, M.\,B., and I.\,V.~Ostrovskii}. 
1998. Analytic and asymptotic properties of generalized Linnik probability densities. 
\textit{J.~Math. Anal. Appl}. 217:555--578.  
\bibitem{50-kz}
\Aue{\mbox{Erdo\!\!{\ptb{\v{g}}}an}, M.\,B., and I.\,V.~Ostrovskii}. 1998. 
On mixture representation of the Linnik density.
\textit{J.~Aust. Math. Soc.~A} 64:317--326. 
\bibitem{51-kz}
\Aue{Kalashnikov, V.\,V.} 1997. \textit{Geometric sums: Bounds for rare events with 
applications}. Dordrecht: Kluwer Academic Publs. 270~p.
\bibitem{52-kz}
\Aue{Pakes, A.\,G.} 1992. A~characterization of gamma mixtures of stable 
laws motivated by limit theorems. \textit{Stat. Neerl.} 2-3:209--218. 
\bibitem{53-kz}
\Aue{Kozubowski, T.\,J.} 1998. Mixture representation of Linnik distribution revisited.
\textit{Stat. Probabil. Lett.} 38:157--160. 
\bibitem{54-kz}
\Aue{Kozubowski, T.\,J.} 1999. Exponential mixture representation of geometric 
stable distributions. \textit{Ann. Inst. Stat. Math.}
52(2):231--238. 
\end{thebibliography}

 }
 }

\end{multicols}

\vspace*{-3pt}

\hfill{\small\textit{Received October 14, 2016}}

\Contr

\noindent
\textbf{Korolev Victor Yu.} (b.\ 1954)~---
Doctor of Science in physics and mathematics, professor, Head of the Department 
of Mathematical Statistics, 
Faculty of Computational Mathematics and Cybernetics, 
M.\,V.~Lomonosov Moscow State University, 1-52~Leninskiye Gory, GSP-1, 
Moscow 119991, Russian Federation; leading scientist, Institute of Informatics 
Problems, Federal Research Center ``Computer Science and Control'' 
of the Russian Academy of Sciences, 44-2~Vavilov Str., Moscow 119333,  
Russian Federation; \mbox{vkorolev@cs.msu.su} 

\vspace*{3pt} 

\noindent
\textbf{Zeifman Alexander I.} (b.\ 1954)~---
Doctor of Science in physics and mathematics, professor, Head of Department, 
Vologda State University, 15~Lenin Str., Vologda 160000, Russian Federation; 
senior scientist, Institute of Informatics Problems, Federal Research Center 
``Computer Science and Control'' of the Russian Academy of Sciences, 
44-2~Vavilov Str., Moscow 119333, Russian Federation; 
principal scientist, ISEDT RAS, 56-A~Gorky Str., Vologda 160001, 
Russian Federation; Faculty of Computational Mathematics and Cybernetics, 
M.\,V.~Lomonosov Moscow State University, 1-52~Leninskiye Gory, GSP-1, 
Moscow 119991, Russian Federation; \mbox{a\_zeifman@mail.ru}

 \vspace*{3pt}
 
 \noindent
 \textbf{Korchagin Alexander Yu.} (b.\ 1989)~---
 junior scientist, Faculty of Computational Mathematics and Cybernetics, 
 M.\,V.~Lomonosov Moscow State University, 1-52~Leninskiye Gory, GSP-1, 
 Moscow 119991, Russian Federation; \mbox{sasha.korchagin@gmail.com}
\label{end\stat}


\renewcommand{\bibname}{\protect\rm Литература}  %3
\def\stat{bening}


\def\tit{АСИМПТОТИЧЕСКОЕ
РАЗЛОЖЕНИЕ ДЛЯ МОЩНОСТИ КРИТЕРИЯ, ОСНОВАННОГО НА ВЫБОРОЧНОЙ
МЕДИАНЕ, В~СЛУЧАЕ РАСПРЕДЕЛЕНИЯ ЛАПЛАСА$^*$}
\def\titkol{Асимптотическое
разложение для мощности критерия, основанного на выборочной
медиане} %, в случае распределения Лапласа}

\def\autkol{В.\,Е.~Бенинг, А.\,В.~Сипина}
\def\aut{В.\,Е.~Бенинг$^1$, А.\,В.~Сипина$^2$}

\titel{\tit}{\aut}{\autkol}{\titkol}

{\renewcommand{\thefootnote}{\fnsymbol{footnote}}\footnotetext[1]
{Работа выполнена
при финансовой поддержке РФФИ, проекты 08-01-00567 и
08-07-00152.}}

\renewcommand{\thefootnote}{\arabic{footnote}}
\footnotetext[1]{Московский государственный университет им.\
М.\,В.~Ломоносова, факультет вычислительной математики и
кибернетики; Институт проблем информатики Российской академии наук, bening@yandex.ru}
\footnotetext[2]{Московский государственный университет им.\
М.\,В.~Ломоносова, факультет вычислительной математики и
кибернетики, anna@sipin.ru}



\Abst{В работе прямыми методами, использующими асимптотические разложения,
получена формула для предельного отклонения мощности критерия, 
основанного на выборочной медиане, от мощности наилучшего критерия в случае распределения Лапласа.}

\KW{выборочная медиана; асимптотичсекое разложение; функция мощности; распределение Лапласа}

      \vskip 18pt plus 9pt minus 6pt

      \thispagestyle{headings}

      \begin{multicols}{2}

      \label{st\stat}


\section{Введение}

Следуя работе~\cite{3ben}, рассмотрим задачу проверки гипотезы
\begin{equation*}
{\sf H}_0: \theta = 0     
%\label{e1.1b}
\end{equation*}
против последовательности сложных близких альтернатив вида
\begin{equation*}
{\sf H}_{n,1}: \theta = \fr{t}{\sqrt{n}}\,,\quad 0<t<C\,,\quad
 C > 0
% \label{e1.2b}
\end{equation*}
на основе выборки $(X_1, \ldots , X_n)$~--- независимых одинаково распределенных наблюдений, имеющих распределение Лапласа 
с плотностью
\begin{equation}
p(x, \theta) = \fr{1}{2}e^{-|x-\theta|}\,, \quad x,\:
\theta \in{\sf R}^1\,. 
\label{e1.3b}
\end{equation}
Распределение Лапласа широко применяется в прикладной статистике, например
в задачах вы\-де\-ле\-ния полезного сигнала на фоне помех~[2--4].
Естественность возникновения этого распределения обоснована в
работе~\cite{6ben}.

Для каждого фиксированного $t\in (0,C]$
обозначим через~$\beta_n^*(t)$ мощность наилучшего критерия размера
$\alpha\in (0,1)$. По лемме Неймана--Пирсона %\linebreak 
[6, с.~94]
такой критерий всегда существует и  основан на логарифме отношения правдоподобия
\begin{equation}
\Lambda_n(t) = 
\sum_{i=1}^{n}\left( \left|X_i\right|-\left|X_i-tn^{-1/2}\right|\right)\,.
 \label{e1.4b}
\end{equation}
В работах~\cite{3ben, 2ben} рассмотрен критерий, основанный на знаковой статистике,
и получена формула для предельного отклонения мощности данного
критерия от мощности наилучшего критерия, основанного на~$\Lambda_n(t)$.
Поскольку у плотности~$p(x,\theta)$ не существует производной по~$\theta$ в 
точке $\theta = 0$, то это семейство не является регулярным.
Это выражается в нарушении естественного порядка~$n^{-1}$ разности мощностей
этих критериев и приводит к порядку~$n^{-1/2}$.

В  работе рассматривается статистика
\begin{equation*}
T_n = \sqrt{2k}\, \zeta_n\,,\quad k=\left[\fr{n}{2}\right]\,, 
%\label{e1.5b}
\end{equation*}
где $\zeta_n$~--- выборочная медиана:
\begin{equation*}
\zeta_n= 
\begin{cases}
X_{(k+1)}\,, & n=2k+1\,; \\
\fr{X_{(k)}+X_{(k+1)}}{2}\,, &  n=2k\,.
\end{cases}
%\label{e1.6b}
\end{equation*}
Заметим, что в случае распределения Лапласа выборочная медиана
совпадает с оценкой максимального правдоподобия (см.~\cite{1ben}).

Обозначим через~$\beta_n(t)$ мощность критерия размера $\alpha\in (0,1)$,
основанного на статистике~$T_n$. В работе получено асимптотическое
разложение для~$\beta_n(t)$ и вычислен предел разности мощностей~$\beta_n^*(t)$ и~$\beta_n(t)$
$$
r(t)\equiv\lim_{n\to\infty}\sqrt n\left(\beta_n^*(t)-\beta_n(t)\right)
$$
критериев (см.~(\ref{e2.14b})),
основанных соответственно на статистиках~$\Lambda_n$ и~$T_n$.

В работе также приведено полное доказательство  (см.~\cite{5ben})
представления выборочной медианы в виде случайной суммы
независимых экспоненциально распределенных  случайных величин.


\section{Асимптотическое разложение для мощности критерия,
основанного на выборочной медиане}

В этом разделе будет построено  асимптотическое разложение  для мощности~$\beta_n(t)$.
Основой для его получения служит  работа~\cite{1ben} (см.\ теорему~2.1),
в которой получено разложение для функции распределения выборочной медианы.
Члены порядка~$n^{-1/2}$ в разложении для функции распределения выборочной медианы
без доказательства приведены  также в работе~\cite{9ben}.

\medskip
\noindent
\textbf{Теорема 1.} {\it Для мощности~$\beta_n(t)$ равномерно по
$t\in(0,C]$, $C>0$,
справедливо следующее асимптотическое разложение:
\begin{equation*}
\beta_n(t)=
\begin{cases}
\Phi(t-u_\alpha)-\fr{t(2u_\alpha-t)}{2\sqrt{n}}\,\varphi(u_\alpha-t)+{} \\
\hspace*{8mm}{}+o\left(n^{-1/2}\right)\,,  \quad t \le u_\alpha\,,\enskip  \alpha <\fr{1}{2}\,;\\
\Phi(t-u_\alpha)-\fr{2u_\alpha^2+t^2-2u_\alpha t}{2\sqrt{n}}\,\varphi(u_\alpha -t)+{}\\
\hspace*{8mm}{}+o\left(n^{-1/2}\right)\,, \quad t>u_\alpha\,, \enskip \alpha <\fr{1}{2}\,;\\
\Phi(t-u_\alpha)+\fr{t(2u_\alpha-t)}{2\sqrt{n}}\,\varphi(u_\alpha -t)+{}\\
\hspace*{22mm}{}+{} o\left(n^{-1/2}\right)\,, \quad 
\alpha \ge \fr{1}{2}\,,
\end{cases}\hspace*{-6pt}
%\label{e2.1b}
\end{equation*}
где  $\Phi(x)$  и~ $\varphi(x)$~---  функция распределения и
плотность стандартного нормального закона и $\Phi(u_\alpha)=1-\alpha$.}

\medskip

\noindent
Д\,о\,к\,а\,з\,а\,т\,е\,л\,ь\,с\,т\,в\,о\,.\
Для доказательства теоремы воспользуемся асимптотическим разложением
для функции распределения выборочной медианы в случае
распределения Лапласа из работы~\cite{1ben} (см.\ формулу~(1.3)):
\begin{multline}
\p_{n,\theta} \left( \sqrt{2k}(\zeta_n - \theta) < x \right) = 
\Phi(x)-\fr{x|x|}{2\sqrt{2k}}\,\varphi(x)+{}\\
{}+
\fr{x(18+10x^2-3x^4)}{48k}\,\varphi(x)+ o(n^{-1})\,.
\label{e2.2b}
\end{multline}
Подберем критическое значение~$d_n$, исходя из условия
\begin{equation*}
\p_{n,0}(T_n>d_n)=\alpha+ o(n^{-1})\,.
%\label{e2.3b}
\end{equation*}
Будем искать $d_n$ в виде
\begin{equation*}
d_n = u_\alpha +\fr{a}{\sqrt{2k}}+\fr{b}{2k}\,.
%\label{e2.4b}
\end{equation*}
Из формулы~(\ref{e2.2b}) следует, что

\noindent
\begin{multline}
\p_{n, 0} \left( T_n> d_n \right) = 1 -
\Phi(d_n)+\fr{d_n|d_n|}{2\sqrt{2k}}\varphi(d_n)-{}\\
{}-
\fr{d_n(18+10d_n^2-3d_n^4)}{48k}\,\varphi(d_n)+ o(n^{-1})\,.
\label{e2.5b}
\end{multline}
Чтобы раскрыть модуль в выражении~(\ref{e2.5b}),  рас\-смот\-рим два случая:
$\alpha<1/2$ и $\alpha \ge 1/2$.

Рассмотрим случай $\alpha < 1/2$. Это означает, что при достаточно
больших $n$ справедливо неравенство $d_n > 0$.
Подставляя выражение для~$d_n$ в формулу~(\ref{e2.5b}) и применяя следующие разложения:
\begin{multline*}
\Phi(d_n)=\Phi\left(u_\alpha+\fr{a}{\sqrt{2k}}+\fr{b}{2k}\right)=
\Phi(u_\alpha)+{}\\
{}+
\left(\fr{a}{\sqrt{2k}}+\fr{b}{2k}\right)\varphi(u_\alpha)-
\fr{u_\alpha a^2}{4k}\varphi(u_\alpha)+ o(n^{-1})\,;
\end{multline*}
\vspace*{-12pt}

\noindent
\begin{multline*}
\varphi(d_n)=\varphi\left(u_\alpha+\fr{a}{\sqrt{2k}}
+\fr{b}{2k}\right)= {}\\
{}=
\varphi(u_\alpha)-\left(\fr{a}{\sqrt{2k}}+\fr{b}{2k}\right)u_\alpha
\varphi(u_\alpha)+ o(n^{-1})\,,
\end{multline*}
получаем
\begin{multline*}
1-\Phi(u_\alpha)-\left(\fr{a}{\sqrt{2k}}+
\fr{b}{2k}\right)\varphi(u_\alpha)+\fr{u_\alpha a^2}{4k}\,\phi(u_\alpha)
+{}\\
{}+\fr{(u_\alpha+(a/\sqrt{2k})+b/(2k))^2}
{2\sqrt{2k}}\times{}\\
{}\times \left(\varphi(u_\alpha) - \fr{a}{\sqrt{2k}}\,u_\alpha
\varphi(u_\alpha)\right)-{}\\
{}-
\fr{u_\alpha(18+10u_\alpha^2-3u_\alpha^4)}{48k}\,\varphi(u_\alpha)=
\alpha + o(n^{-1})\,.
\end{multline*}
Приравнивая коэффициенты при~$1/\sqrt{2k}$ и~$1/(2k)$ к нулю,
находим выражения для~$a$ и~$b$:
\begin{gather*}
a=\fr{u_\alpha^2}{2}\,;
\\
b=-\fr{3}{4}\,u_\alpha+\fr{1}{12}\,u_\alpha^3\,;
\\
d_n = u_\alpha+\fr{u_\alpha^2}{2\sqrt{2k}}-\fr{3}{8k}\,
u_\alpha+\fr{1}{24k}\,u_\alpha^3\,.
\end{gather*}
Теперь для получения асимптотического разложения мощности критерия используем
разложение
\begin{multline*}
\p_{n,tn^{-1/2}}(T_n<x)= \Phi\left(x-t\sqrt{2k}n^{-1/2}\right) -{}\\
{}-
\fr{\left(x-t\sqrt{2k}n^{-1/2}\right)\left| x\:-\:t\sqrt{2k}\,n^{-1/2}\right|}{2\sqrt{2k}}\,
{}\times{}\\
{}\times\varphi(x-t\sqrt{2k}\,n^{-1/2})+ {}
\end{multline*}
\begin{multline*}
{}+
\fr{ x-t\sqrt{2k}\,n^{-1/2}}{48k}
\left(18+10(x-
t\sqrt{2k}\,n^{-1/2})^2-{}\right.\\
\left.{}-3(x-t\sqrt{2k}\,n^{-1/2})^4\right)\times{}
\\
{}\times\varphi\left(x-t\sqrt{2k}\,n^{-1/2}\right)+ o\left(n^{-1}\right)\,,
%\label{e2.6b}
\end{multline*}
которое  получается при подстановке $\theta=tn^{-1/2}$ в
формулу~(\ref{e2.2b}).

Имеем
\begin{multline*}
\beta_n(t)=\p_{n,tn^{-1/2}}\left(T_n>d_n\right) ={}\\
{}=
1-\Phi\left(d_n-t\right) +
\fr{\left(d_n-t\right)\left|d_n-t\right|}{2\sqrt{2k}}\,\varphi\left(d_n-t\right)-{}
\\\!
{}-\fr{d_n-t}{48k}\left(18+10\left(d_n-t\right)^2
-3(d_n-t)^4\right)\, \varphi\left(d_n-t\right)+{}\\
{}+ o\left(n^{-1}\right)\,.
%\label{e2.7b}
\end{multline*}
Аналогично предыдущему, рассмотрим  два случая: $t\le u_\alpha$ и
$t>u_\alpha$.

Пусть сначала $t \le u_\alpha$.
Используя разложения
\begin{multline*}
\Phi\left(d_n-t\right)={}\\
{}=\Phi\left(u_\alpha-t+
\fr{u_\alpha^2}{2\sqrt{2k}}-\fr{3}{8k}\,u_\alpha+
\fr{1}{24k}\,u_\alpha^3\right)={}\\
{}=\Phi\left(u_\alpha-t\right)+
\left(\fr{u_\alpha^2}{2\sqrt{2k}}-\fr{3}{8k}\,u_\alpha+
\fr{1}{24k}\,u_\alpha^3\right)\times{}\\
{}\times\varphi\left(u_\alpha-t\sqrt{2k}\,n^{-1/2}\right)-{}
\\
{}-
\fr{\left(u_\alpha-t\sqrt{2k}\,n^{-1/2}\right)\varphi\left(u_\alpha-
t\sqrt{2k}\,n^{-1/2}\right)u_\alpha^4}{16k}+{}\\
{}+ o\left(n^{-1}\right)\,; 
%\label{e2.8b}
\end{multline*}

\vspace*{-12pt}

\noindent
\begin{multline*}
\varphi\left(d_n-t\right)={}\\
{}= \varphi\left(u_\alpha-t+
\fr{u_\alpha^2}{2\sqrt{2k}}-\fr{3}{8k}\,u_\alpha+
\fr{1}{24k}\,u_\alpha^3\right)={}\\
{}=
\varphi\left(u_\alpha-t\right)-\left(u_\alpha-t\right)
\varphi\left(u_\alpha-t\right)\fr{u_\alpha^2}{2\sqrt{2k}}+{}\\
{}+
o\left(n^{-1/2}\right)\,,
%\label{e2.9}
\end{multline*}
получаем, что
\begin{multline*}
\beta_n(t)=1-\Phi\left(u_\alpha-t\right)-
\fr{u_\alpha^2}{2\sqrt{2k}}\,\varphi\left(u_\alpha-t\right)+{}\\
{}+\fr{u_\alpha^2}{2\sqrt{2k}}\,\varphi(u_\alpha-t)-
\fr{2u_\alpha t - t^2}{2\sqrt{2k}}\,\varphi(u_\alpha-t)+{}\\
{}+
o\left(n^{-1/2}\right)=
\Phi\left(t-u_\alpha\right)-\fr{t\left(2u_\alpha - t\right)}{2\sqrt{2k}}\,
\varphi\left(u_\alpha - t\right)+{}\\
{}+ o\left(n^{-1/2}\right)\,.
%\label{e2.10b}
\end{multline*}
Во втором случае при $t > u_\alpha$  выражение
для мощности приобретает вид:

\noindent
\begin{multline*}
\beta_n(t)=\Phi\left(t-u_\alpha\right)-{}\\
{}-
\fr{t\left(2u_\alpha^2+t^2 -2u_\alpha t\right)}{2\sqrt{n}}\,
\varphi\left(u_\alpha-t\right)+ o\left(n^{-1/2}\right)\,.
%\label{e2.11b}
\end{multline*}
При $\alpha \ge 1/2$  аналогичным образом имеем
\begin{multline*}
\beta_n(t)={}\\
{}=
 \Phi\left(t-u_\alpha\right)+
\fr{t\left(2u_\alpha - t\right)}{2\sqrt{n}}\,\varphi\left(u_\alpha - t\right)+
o\left(n^{-1/2}\right)\,.
%\label{e2.12b}
\end{multline*}
Из этих формул следует утверждение теоремы.~$\Box$

\medskip

В работе~\cite{2ben} было показано, что для мощ\-ности~$\beta_n^*(t)$ 
критерия размера $\alpha\in (0,1)$, осно\-ван\-но\-го на
логарифме отношения прав\-до\-подобия~$\Lambda_n(t)$~(\ref{e1.4b}),
справедливо  асимптотическое\linebreak разложение
\begin{equation*}
\beta_n^*(t)=\Phi(t-u_\alpha) - \fr{t^2}{6\sqrt{n}}\,
\varphi(t-u_\alpha)+ o(n^{-1/2})\,.
%\label{e2.13b}
\end{equation*}
Используя это разложение и теорему~1, получаем формулу
для предельного отклонения нормированной разности мощностей
рассматриваемых критериев:
\begin{multline}
r(t)= \lim_{n \to \infty}\sqrt{n}(\beta_n^*(t)-\beta_n(t))
={}\\
{}=
\begin{cases}
\left(t u_\alpha-\fr{2t^2}{3}\right)
\varphi(u_\alpha-t)\,,\\
\hspace*{30mm} t \le u_\alpha\,,\enskip \alpha < \fr{1}{2}\,; \\
\left(u_\alpha^2+\fr{t^2}{3}-u_\alpha t \right)
\varphi(u_\alpha - t)\,,\\
\hspace*{30mm}  t>u_\alpha\,,\enskip \alpha<\fr{1}{2}\,; \\
\left(\fr{t^2}{3}-t u_\alpha\right)\varphi(u_\alpha-t)\,, \quad\quad\ \  \alpha \ge \fr{1}{2}\,. 
\end{cases}
\label{e2.14b}
\end{multline}

\section{Представление выборочной медианы в~виде случайной суммы}

В этом разделе докажем лемму о представлении выборочной медианы
в случае распределения Лапласа в виде суммы случайного числа
независимых экспоненциально распределенных случайных величин.
Формулы для представления порядковых статистик в случае распределения
Лапласа в виде подобной суммы приведены в работе~[4, с.~63],
но без строгого доказательства.

\bigskip

\noindent
\textbf{Лемма 1.}
{\it В случае распределения Лапласа выборочную медиану
можно представить в следующем виде (здесь равенства по распределению):
\begin{align}
\zeta_{2k+1} &\stackrel{d}{=}\delta_{2k+1}
\sum\limits_{j=k+1}^{K_{2k+1}}{\fr{W_j}{j}}\,;
\label{e3.1b}\\
\zeta_{2k}&\stackrel{d}{=}\fr{W_1-W_2}{2k}\,\mathbf{1}(B_{2k+1}=k)+{}\notag\\[1pt]
&\!\!\!\!\!\!\!\!\!\!\!\!\!\!{}+
\left(\delta_{2k}\sum\limits_{j=k+1}^{K_{2k+1}}\fr{W_j}{j}+
\delta_{2k}\fr{W_k}{2k}\right)\mathbf{1}\left(B_{2k+1} \ne k\right)\,,
\label{e3.2b}
\end{align}
где
$$
\delta_n=\mathrm{sign}\left(B_n-k-\fr{1}{2}\right)\,,
$$
$W_j$~--- независимые экспоненциально (с параметром~1) распределенные
случайные величины; $B_n$~--- бернуллиевские случайные величины с параметрами
$p=1/2$ и~$n$, независимые от~$W_j$;
\begin{equation*}
K_n = \max\left(B_n, \bar{B_n}\right)\,,\quad
\bar{B_n}= n - B_n\,.
\end{equation*}
}

\smallskip

\noindent
Д\,о\,к\,а\,з\,а\,т\,е\,л\,ь\,с\,т\,в\,о\,.

Вначале докажем две вспомогательные формулы, справедливые для любого
действительного чис\-ла~$s$
и любых натуральных чисел~$a$ и~ $b$:
\begin{gather}
\prod\limits_{j=a}^{a+b}{\fr{1}{j+is}}=
\sum\limits_{j=0}^b \fr{(-1)^j}{(a+j+is)(b-j)!j!}\,;
\label{e3.3b}
\\[3pt]
\!\!\!\!\!\!\!\!\sum\limits_{l=0}^k\fr{k!}{l!} \prod\limits_{j=a}^{a+k-l}\fr{1}{j+is}=
\sum\limits_{l=0}^k \begin{pmatrix}
k\\ l\end{pmatrix}
\fr{(-1)^l 2^{k-l}}{a+l+is}\,.
\label{e3.4b}
\end{gather}
Формулу~(\ref{e3.3b}) докажем методом математической индукции.

При $b=1$ формула верна. Предполагая ее верной при $b\ge1$,
докажем что она  верна и  при~$b+1$:
\begin{multline*}
\prod\limits_{j=a}^{a+b+1}\fr{1}{j+is}=\fr{1}{a+b+1+is}\prod\limits_{j=a}^{a+b}
\fr{1}{j+is}={}\\[2pt]
{}=
\fr{1}{a+b+1+is}\left(\sum\limits_{l=0}^k 
\begin{pmatrix}
k\\ l
\end{pmatrix}
\fr{(-1)^l 2^{k-l}}
{a+l+is}\right)={}\\[2pt]
{}=
\sum\limits_{j=0}^{b}\fr{(-1)^j}{(b-j)!j!} \left(\fr{1}{(b+1-j)(a+j+is)}
- {}\right.\\[2pt]
\left.{}-\fr{1}{(b+1-j)(a+b+1+is)} \right)={}
\end{multline*}
\begin{multline*}
{}=
\sum\limits_{j=0}^{b}\fr{(-1)^j}{(a+j+is)(b+1-j)!j!}-{}\\
{}-
\fr{1}{a+b+1+is}\sum\limits_{j=0}^{b}\fr{(-1)^j}{(b-j+1)!j!}\,.
\end{multline*}
Заметим, что
\begin{multline*}
\!\!\sum\limits_{j=0}^b\fr{(-1)^j}{(b-j+1)!j!}=
\sum\limits_{j=0}^{b+1}\fr{(-1)^j}{(b-j+1)!j!}
-\fr{(-1)^{b+1}}{(b+1)!}={}\\
{}=
\fr{1}{(b+1)!}(1-1)^{b+1}-\fr{(-1)^{b+1}}{(b+1)!}=
-\fr{(-1)^{b+1}}{(b+1)!}\,.
\end{multline*}
И следовательно, формула~(\ref{e3.3b}) доказана.
Формула~(\ref{e3.4b}) следует  из доказанной формулы~(\ref{e3.3b}), по\-скольку
\begin{multline*}
\sum_{l=0}^k{\fr{k!}{l!}}\prod\limits_{j=a}^{a+k-l}\fr{1}{j+is}={}\\
{}=
\sum\limits_{l=0}^{k}\fr{k!}{l!}\sum\limits_{j=0}^{k-l}
\fr{(-1)^j}{(a+j+is)(k-l-j)! j!}={}\\
{}
=\sum\limits_{j=0}^{k}\fr{(-1)^j}{a+j+is}\sum\limits_{l=0}^{k-j}
\begin{pmatrix}
k\\ j
\end{pmatrix}
\begin{pmatrix}
k-j\\  l
\end{pmatrix}={}\\
{}=
\sum\limits_{j=0}^k\fr{(-1)^j}{a+j+is}
\begin{pmatrix}
k\\ j
\end{pmatrix}
2^{k-j}\,.
\end{multline*}
Теперь приступим к доказательству основного утверждения леммы.
Рассмотрим сначала случай $n=2k+1$.
Плотность $(k+1)$-й порядковой статистики, как известно,
выражается формулой (см.~\cite{4ben})
\begin{equation*}
p_{2k+1}(x) = (2k+1)
\begin{pmatrix}
2k\\  k\end{pmatrix}
f(x)(F(x)(1-F(x))^k\,,
%\label{3.5b}
\end{equation*}
где $f(x)$ и  $F(x)$~--- соответственно плотность и
функция распределения исходных случайных величин.

Найдем характеристическую функцию~$\phi_{2k+1}(s)$ выборочной
медианы~$\zeta_{2k+1}$:
\begin{multline*}
\phi_{2k+1}(s)=\e e^{is\zeta_{2k+1}}=
\int\limits_{-\infty}^{\infty}e^{isx}f(x)\,dx={}\\
{}=
(2k+1)
\begin{pmatrix}
2k\\  k\end{pmatrix}
2^{-(k+1)}\times{}\\
{}\times
\sum\limits_{j=0}^k (-1)^j 2^{-j}
\begin{pmatrix}
k\\ e j\end{pmatrix}
\fr{2(k+1+j)}{(k+1+j)^2+s^2}\,.
%\label{e3.6b}
\end{multline*}
Теперь найдем характеристическую функцию~$f_{2k+1}(s)$ случайной величины, определенной\linebreak\vspace*{-12pt}\pagebreak

\noindent
в правой части  формулы~(\ref{e3.1b}).
С учетом того, что
 характеристическая функция стандартной экспоненциальной
случайной величины равна $1/(1-is)$, имеем
\begin{multline*}
f_{2k+1}(s)={}\\
{}=
\sum\limits_{l=0}^{2k+1}\e \exp \left(is\delta_{2k+1}
\sum\limits_{j=k+1}^{K_{2k+1}}\fr{W_j}{j}\right)\mathbf{1}(B_{2k+1}=l)={}
\\
=2^{-(2k+1)}\left(\sum\limits_{l=0}^k \begin{pmatrix}
2k+1\\  l\end{pmatrix}
\prod\limits_{j=k+1}^{2k+1-l}\fr{j}{j+is}+{}\right.\\
\left.{}+
\sum\limits_{l=k+1}^{2k+1}
\begin{pmatrix}
2k+1\\ l\end{pmatrix}
\prod\limits_{j=k+1}^{l}\fr{j}{j-is}
\right)={}\\
{}
=2^{-(2k+1)}(2k+1)
\begin{pmatrix}
2k\\ k\end{pmatrix}
\sum\limits_{l=0}^k\fr{k!}{l!}
\left(\prod\limits_{j=k+1}^{2k+1-l}\fr{1}{j+is} +{}\right.\\
\left.{}+
\prod\limits_{j=k+1}^{2k+1-l}\fr{1}{j-is}\right)\,.
%\label{e3.7b}
\end{multline*}
Применяя формулу~(\ref{e3.4b}), получаем
\begin{multline*}
f_{2k+1}(s)=(2k+1)
\begin{pmatrix}
2k\\  k
\end{pmatrix}
2^{-(k+1)}\times{}\\
{}\times
\sum\limits_{j=0}^k(-1)^j 2^{-j}
\begin{pmatrix}
k\\  j\end{pmatrix}
\fr{2(k+1+j)}{(k+1+j)^2+s^2}\,.
%\label{e3.8b}
\end{multline*}
Значит, $f_{2k+1}(s)\equiv\phi_{2k+1}(s)$ и представление~(\ref{e3.1b}) доказано.
\medskip

Перейдем теперь к рассмотрению случая четного $n=2k$.
Совместная плотность двух порядковых статистик~$X_{(k)}$ и~$X_{(k+1)}$
определяется формулой (см.~\cite{4ben})
\begin{equation*}
p(x,y)=\fr{(2k)!}{((k-1)!)^2}\,(F(x)(1-F(y)))^{k-1}f(x)f(y)\,.
%\label{e3.9b}
\end{equation*}
Из этой формулы нетрудно получить, что плотность случайной величины
$$
\zeta_{2k}=\fr{X_{(k)}+X_{(k+1)}}{2}
$$
равна
\begin{multline*}
p_{2k}(x) = \fr{(2k)!}{2^k ((k-1)!)^2}\times{}\\
{}\times
\left(\sum_{j=0}^{k-2}\fr{(-1)^j
\begin{pmatrix}
k-1\\ j
\end{pmatrix}
2^{-j}}{k-1-j}
e^{-(k+1+j)|x|}\times{}\right.
\end{multline*}
\begin{multline}
\left.{}\times \left(1-e^{-(k-1-j)|x|}\right)- \right.{}\\
{}\left.
- \fr{(-1)^k}{2^{k-1}}|x|e^{-2k|x|} + \fr{1}{k2^k}e^{-2k|x|}
\vphantom{\fr{(-1)^j
\begin{pmatrix}
k-1\\ j
\end{pmatrix}
2^{-j}}{k-1-j}}\right)\,.
\label{e3.10b}
\end{multline}
Подробный вывод этой формулы приведен в работе~\cite{8ben}.
Исходя их формулы~(\ref{e3.10b}), найдем характеристическую функцию~$\phi_{2k}(s)$
выборочной медианы~$\zeta_{2k}$:
\begin{multline*}
\!\phi_{2k}(s)=
\fr{(2k)!}{2^k ((k-1)!)^2}
\left( \sum\limits_{j=0}^{k-2}
\fr{(-1)^j
\begin{pmatrix}
k-1\\ j
\end{pmatrix}
2^{-j}}{k-1-j}\times{}\right.
\\
\left.{}\times
\left(
\fr{2(k+1+j)}{(k+1+j)^2+s^2}  -
 \fr{4k}{4k^2+s^2} \right)-{}\right.\\
\left. {}- 
 \fr{(-1)^k}{2^{k-2}(4k^2+s^2)} + \fr{1}{2^{k-2}(4k^2+s^2)}
 \vphantom{\sum\limits_{j=0}^{k-2}
\fr{(-1)^j
\begin{pmatrix}
k-1\\ j
\end{pmatrix}
2^{-j}}{k-1-j}}
\right)\,.
%\label{e3.11b}
\end{multline*}
Найдем теперь характеристическую функ-\linebreak цию~$f_{2k}(s)$ случайной величины,
определенной
 в правой части формулы~(\ref{e3.2b}). Учитывая формулу~(\ref{e3.4b}), получим
\begin{multline*}
f_{2k}(s)=\sum\limits_{l=0}^{k-1}{\p(B_{2k}=l)
\fr{2k}{2k+is}\prod\limits_{j=k+1}^{2k-l}{\fr{j}{j+is}}}+{}\\
{}+
\sum\limits_{j=k+1}^{2k}{\p(B_{2k}=l)\fr{2k}{2k-is}\prod\limits_{j=k+1}^{2k-l}\fr{j}{j-is}}+{}\\
{}+
\p(B_{2k}=k)\fr{4k^2}{4k^2+s^2}={}\\
{}=
\fr{(2k)!}{2^k ((k-1)!)^2} \left( \fr{1}{2^{k-2}(4k^2+s^2)}
+{}\right.\\
\left.{}+2^{1-k}\sum\limits_{l=0}^{k-1}(-1)^l 2^{k-l-1}
\begin{pmatrix}
k-1\\ l\end{pmatrix}\times\right.{}\\
{}\times
\left( \fr{1}{(2k+is)(k+1+l-is)}+{}\right.\\
\left.\left.{}+ 
\fr{1}{(2k-is)(k+1+l-is)}\right) \right)\,.
\end{multline*}
Применяя при $l \ne k-1$ следующее соотношение:
\begin{multline*}
\fr{1}{(2k+is)(k+1+l+is)}={}\\
{}=
\fr{1}{k-1-l}\left( \fr{1}{k+1+l+is} - \fr{1}{2k+is}\right)\,,
\end{multline*}
получаем равенство

\noindent
\begin{multline*}
f_{2k}(s)=
\fr{(2k)!}{2^k ((k-1)!)^2}
\left( \sum\limits_{j=0}^{k-2}
\fr{(-1)^j 
\begin{pmatrix}
k-1\\ j
\end{pmatrix}
2^{-j}}{k-1-j}\times{}\right.\\
\left.{}\times
\left(
\fr{2\left(k+1+j\right)}{(k+1+j)^2+s^2} 
-  \fr{4k}{4k^2+s^2} \right)
-{}\right. \\
\left.{}- \fr{\left(-1\right)^k}{2^{k-2}\left(4k^2+s^2\right)} + \fr{1}{2^{k-2}(4k^2+s^2)}
\vphantom{\sum_{j=0}^{k-2}
\fr{(-1)^j 
\begin{pmatrix}
k-1\\ j
\end{pmatrix}
2^{-j}}{k-1-j}}
\right)\,.
%\label{e3.12b}
\end{multline*}
Таким образом,  $\phi_{2k}(s)\equiv f_{2k}(s)$ и утверждение~(\ref{e3.2b})
доказано.~$\Box$

{\small\frenchspacing
{%\baselineskip=10.8pt
\addcontentsline{toc}{section}{Литература}
\begin{thebibliography}{9}

\bibitem{3ben} %1
\Au{Королев Р.\,А., Тестова  А.\,В., Бенинг~В.\,Е.} 
О мощ\-ности асимптотически оптимального критерия в случае 
распределения Лапласа~// Вестник Тверского Государственного Университета, 
серия Прикладная математика, 2008. Вып.~8. №\,4(64). С.~5--23.

\bibitem{9ben} %2
\Au{Takeuchi K.} 
Asymptotic theory of statistical estimation.~---  Tokyo, 1974. (In Japanese.)

\bibitem{1ben} %3
\Au{Бурнашев М.\,В.} 
Асимптотические разложения для 
медианной оценки параметра~// Теор. вероятн. и ее
прим., 1996. Т.~41. Вып.~4. С.~738--753.

\bibitem{5ben}  %4
\Au{Kotz S., Kozubowski~T.\,J., Podgorski~K.}
The Laplace distribution and generalizations: 
A revisit with applications to communications, economics, engineering, 
and finance.~--- Birkhauser, 2001.  P.~349.

\bibitem{6ben}  %5
\Au{Бенинг В.\,Е., Королев В.\,Ю.}
Некоторые статистические  задачи, связанные с распределением Лапласа~// 
Информатика и её применения, 2008. Т.~2.  Вып.~2. С.~19--34.

\bibitem{7ben}  %6
\Au{Леман Э.} 
Проверка статистических гипотез.~--- М.: Наука, 1964. 498~с.

\bibitem{2ben} %7
\Au{Королев Р.\,А., Бенинг В.\,Е.}
Асимптотические 
разложения для мощностей критериев в случае распределения Лапласа~//
Вестник Тверского Государственного Университета, серия 
Прикладная математика, 2008. Вып.~3(10). №\,26(86). С.~97--107.

\bibitem{4ben} %8
\Au{David H.\,A., Nagaraja H.\,N.}
Order Statistics.  3rd ed.~--- New Jersey: Wiley, 2003.  P.~458.

\label{end\stat}

\bibitem{8ben} %9
\Au{Asrabadi B.\,R.} 
The exact confidence interval for 
the scale parameter and the MVUE of the Laplace distribution~// 
Communications in statistics. Theory and methods, 1985. Vol.~14. No.\,3. 
P.~713--733.

 \end{thebibliography}
}
}
\end{multicols} %4
\def\stat{chertok}

\def\tit{МЕТОД КУМУЛЯТИВНЫХ СУММ ДЛЯ~ПОИСКА СМЕНЫ РЕЖИМА В~ПРОЦЕССЕ 
ОРНШТЕЙНА--УЛЕНБЕКА\\ НА~ОСНОВЕ ПРОЦЕССА ЛЕВИ$^*$}

\def\titkol{Метод кумулятивных сумм для поиска смены режима в~процессе 
Орнштейна--Уленбека на основе процесса Леви}

\def\aut{А.\,В.~Черток$^1$, А.\,И.~Каданер$^2$, Г.\,Т.~Хазеева$^3$, И.\,А.~Соколов$^4$}

\def\autkol{А.\,В.~Черток, А.\,И.~Каданер, Г.\,Т.~Хазеева, И.\,А.~Соколов}

\titel{\tit}{\aut}{\autkol}{\titkol}

\index{Черток А.\,В.}
\index{Каданер А.\,И.}
\index{Хазеева Г.\,Т.}
\index{Соколов И.\,А.}
\index{Chertok A.\,V.}
\index{Kadaner A.\,I.}
\index{Khazeeva G.\,T.} 
\index{Sokolov I.\,A.}


{\renewcommand{\thefootnote}{\fnsymbol{footnote}} \footnotetext[1]
{Работа выполнена при частичной 
финансовой поддержке РФФИ (проект 14-07-00041).}}


\renewcommand{\thefootnote}{\arabic{footnote}}
\footnotetext[1]{Факультет вычислительной математики и~кибернетики 
Московского государственного университета им.\ М.\,В.~Ломоносова; Сбербанк России, 
\mbox{avchertok.sbt@sberbank.ru}}
\footnotetext[2]{Механико-математический 
факультет Московского государственного университета им.\ М.\,В.~Ломоносова; 
Сбербанк России, \mbox{aikadaner.sbt@sberbank.ru}}
\footnotetext[3]{Факультет вычислительной математики и~кибернетики 
Московского государственного университета им.~М.\,В.~Ломоносова, 
\mbox{gelana.khazeyeva@gmail.com}}
\footnotetext[4]{Институт проб\-лем информатики Федерального 
исследовательского центра <<Информатика и~управ\-ле\-ние>> Российской академии наук, 
\mbox{isokolov@ipiran.ru}}

\vspace*{-3pt}

\Abst{Рассматривается процесс Орн\-штей\-на--Улен\-бе\-ка (ОУ) с~трендом 
на основе процесса Леви для моделирования финансовых временных рядов. 
Продемонстрировано, что использование процесса Леви в~основе процесса 
ОУ дает больше гибкости для описания финансовых 
временных рядов по 
сравнению с~классической гауссовой моделью. В~частности, процесс Леви позволяет 
моделировать остатки с~тяжелыми хвостами, что является  распространенным 
свойством реальных временных рядов. Приводятся эффективные решения для 
оценивания параметров модели с~использованием таких методов, как OLS (ordinary least squares)
и~RLS (regularized least squares). 
Решается задача поиска моментов смены режима в~модели при условии поступления 
данных в~режиме реального времени. Приведен алгоритм, основанный на  
CUSUM (CUmulative SUM) ме\-то\-дах,  способный последовательно обрабатывать смены режима и~поддерживать 
параметры модели актуальными для каждого момента времени.  Решение задачи поиска 
разладки модели и~соответствующих смен режима имеет важное прикладное значение, 
поскольку в~большинстве случаев параметры моделей, описывающих динамику реальных 
систем, меняются во времени под действием внешних факторов.}

\KW{случайные процессы; процессы со свойством возвратности к~среднему; 
процесс Орн\-штей\-на--Улен\-бе\-ка, управляемый процессом Леви; процесс 
Орн\-штей\-на--Улен\-бе\-ка с~трендом; смена режима; CUSUM-ал\-го\-ритмы}

\DOI{10.14357/19922264160405}

\vspace*{-3pt} 


\vskip 10pt plus 9pt minus 6pt

\thispagestyle{headings}

\begin{multicols}{2}

\label{st\stat}

\section{Введение}

Процессы со свойством возвратности к~среднему играют важную роль в~моделировании 
динамики явлений из различных областей человеческой деятельности.  В~частности, 
эти процессы привлекательны для моделирования различных явлений в~эконометрике, 
таких как процентные ставки, курсы обмена валют и~цены на сырьевые товары, где 
свойство возвратности к~среднему имеет фундаментальную природу. 

В~работе~\cite{brigo2007} рассмотрено несколько видов случайных процессов со свойством 
возвратности к~среднему и~описаны их основные характеристики.
В~настоящей статье в~качестве такого процесса рассматривается процесс 
ОУ, управляемый процессом Леви. 

Классическая версия 
процесса была 
впервые представлена в~совместной работе голландских физиков Л.\,С.~Орнштейна 
и~Дж.\,Е.~Уленбека~\cite{ou1930} в~качестве модели, которая способна описать данные 
с~гауссовской и~диффузионной структурой. В~экономике же классический процесс 
ОУ известен как модель Васичека благодаря фундаментальной 
работе~\cite{vasicek1977}, где автор предлагает использовать ее для 
моделирования временн$\acute{\mbox{о}}$го ряда процентной ставки. Ее основной недостаток 
заключается в~том, что существует ненулевая вероятность появления отрицательных 
значений, нереалистичных для экономических процессов. Для решения данной 
проблемы позднее была разработана экспоненциальная модель Васичека, а~также 
модель процесса Кок\-са--Ин\-гер\-сол\-ла--Рос\-са, также называемая <<мо\-делью 
с~квад\-рат\-ным корнем>>, в~которой процентная ставка принимает только 
неотрицательные значения и~имеет гам\-ма-рас\-пре\-де\-ле\-ние~\cite{cox1985}.

        \begin{figure*} %fig1
        \vspace*{1pt}
\begin{center}
\mbox{%
\epsfxsize=109.749mm
\epsfbox{che-1.eps}
}
\end{center}
\vspace*{-9pt}
 \Caption{График соотношения цен для фьючерсов компаний 
<<Лукойл>> и~<<Роснефть>>}
                \label{rtsmixpic}
        \end{figure*}


В настоящей статье подтверждается тот факт, что предположение нормальности 
в~классической версии процесса ОУ не описывает реальную структуру 
данных, и~поэтому рассматривается обоб\-щение классического процесса~--- процесс 
ОУ,\linebreak управ\-ля\-емый процессом Леви. Некоторые его модификации 
изучены в~работе~\cite{GarOlk2000}. Предложено рас\-смат\-ри\-вать нормальный обратный 
гауссовский и~дис\-пер\-сионный гам\-ма-про\-цесс для описания динамики его остатков. 
Распределения прираще\-ний этих процессов имеют хвосты тяжелее, чем у~нормального 
распределения, что часто встречается в~реальных данных. 

Дополнительная мотивация 
в~использовании именно этих распределений исходит из приложений в~финансах. 
Например, дисперсионное гам\-ма-рас\-пре\-де\-ле\-ние используется для моделирования цен 
акций, как это делается в~работе~\cite{Fin2009}, а~нормальное обратное 
гауссовское распределение хорошо описывает логарифмические приращения цен 
активов, например в~работе А.\,В.~Кузьминой~\cite{Kuzmina2011} это подтверждается 
на примере данных о~цене фьючерса RTS.

Для более общего механизма построения моделей к~классической модели ОУ 
добавляется линейная составляющая, или тренд. Такой подход позволяет 
моделировать большее число явлений, не выходя за рамки одной модели.

Как известно, финансовые рынки являются динамическими и~нестационарными 
системами. Поэтому отношения, связывающие различные факторы рынка, склонны 
меняться во времени. Пример данного явления продемонстрирован на рис.~\ref{rtsmixpic}. 
По оси~$x$ отложены цены фьючерса на акции компании 
<<Роснефть>> (ROSN), а~по оси~$y$~--- цены фьючерса на акции компании <<Лукойл>> 
(LKOH) с~08.01.2013 по~28.10.2016. Видно, что параметры этой зависимости 
являются также изменяющимися во времени на протяжении дня, так как можно 
отчетливо выделить области, где точки группируются в~окрестностях прямых 
с~разными параметрами.

Все это ставит задачу определения моментов, в~которые предложенная для описания 
данных модель с~определенными параметрами перестает работать, после чего процесс 
начинает следовать той же самой модели, но уже с~другими параметрами. В~данной 
статье эта проблема решена для модели\linebreak
 ОУ. Более того, предлагается процедура 
оценивания параметров  и~обнаружения смен режима в~реаль\-ном времени 
с~использованием RLS или рекурсивного метода наименьших квадратов для оценивания 
параметров, а~также алгоритм, основанный на CUSUM-про\-це\-ду\-рах для обнаружения 
смен режима. В~конце статьи предложенная процедура применяется на различных 
данных.


\section{Моделирование временного ряда}

        \subsection {Одномерный процесс Орнштейна--Уленбека}

       Процесс ОУ с~трендом, управляемый процессом Леви, 
определяется как решение стохастического дифференциального уравнения  (СДУ):       
\begin{align*}
d\left(X_t -\mu -  \nu t\right)& = -\alpha\left(X_t - \mu - \nu 
t\right) dt +  dL_{\lambda t}\,,\\ 
&\hspace{45mm}\forall\  t>0\,;  \\
X(0) &= X_0\,,  
       \end{align*}
         где $\alpha, \mu \in \mathbb{R}$; $L_t$~--- процесс Леви; $X_0$~--- 
некоторая случайная величина, независимая от~$\{L_t\}$; $\nu$ определяет 
постоянный на всем промежутке времени линейный тренд. Параметр~$\mu$ здесь 
означает долгосрочное среднее, а~$\alpha$ определяет скорость стремления 
процесса возвращаться к~своему среднему~--- тренду.

Как показано в~\cite{Protter}, данное СДУ имеет сле\-ду\-ющее решение:
\begin{multline*}
X(t) =\nu t + \mu + \exp\left(-\alpha t\right) \times{}\\
{}\times
\left(
\left(X_0 - \mu \right)+ \int\limits_0^t\exp(\alpha  s)\,dL_{\lambda s}\right)\,, 
\quad X_0 = X(0)\,,
\end{multline*}
или
\begin{multline}
X(t + \tau) =\mu +\nu  (t + \tau) +   \exp
\left(-\alpha \tau\right)\times{}\\
\!\!{}\times\! \left(\!\!
(X(t) - \mu - \nu  t) + \exp(- \alpha t)\!\int\limits_t^{t + \tau}\!\!
\exp(\alpha s)\,dL_{\lambda s}\!\right).\!\!\!\!
\label{explicit_ou}
\end{multline}
Отсюда, в~частности, следует, что $X_t$~--- марковский процесс. Еще стоит 
заметить, что данное решение единственно с~точностью до неотличимости~\cite{sato}. 
Для более подробного рассмотрения процессов Леви см.~[8, 10].

Для удобства обозначим через $Y_t\hm = X_t \hm-\mu\hm- \nu t$ соответствующий приведенный 
процесс ОУ без тренда и~имеющий нулевое среднее.

Свойство возвратности  процесса~$Y_t$ к~нулевому уровню  при $\alpha \hm> 0$ может 
быть получено из~(\ref{explicit_ou}):  если~$Y_t$ стал больше~0 в~момент 
времени~$t$, то коэффициент при~$dt$ отрицательный и~$Y_t$ будет стремиться 
немедленно вернуться к~0; аналогично происходит, если случайный процесс 
становится меньше~0.

\subsubsection{Авторегрессия и~оценка параметров}

Пусть $ X ^ * = {\left(X^*_{t_i}\right)}_ {i = 1, \ldots, N} $~--- 
наблюдения с~интервалом 
$\Delta \hm=  1$ процесса, описываемого определенной выше моделью ОУ с~трендом. 
В~дискретном случае уравнение процесса~(\ref{explicit_ou}) выглядит следующим 
образом:
        \begin{multline}
         \label{OUtrend_d}
        X_{i+1} = \mu + \mu_0\left(1 - e^{-\lambda}\right) + 
        \mu \left(1 - e^{-\lambda}\right)i+ {}\\
        {}+e^{-\lambda } X_{i} + l_i\,,
        \end{multline}
        где $l_i $~--- некоторая случайная величина с~нулевым средним.

        Соотношение~(\ref{OUtrend_d}) описывается регрессионной моделью. Запишем 
его в~виде:
        \begin{equation*} 
%        \label{OUtrend_regr}
        X_{i+1} = c + b t_i + a X_{i} + l_i\,.
        \end{equation*}

        Чтобы оценить параметры $a$, $b$ и~$c$ регрессии, можно воспользоваться 
методом наименьших квад\-ра\-тов и~получить оценки~$\hat{a}$, $\hat{b}$ и~$\hat{c}.$ 
Тогда параметры исходного процесса ОУ с~трендом можно получить 
из соотношений:
        \begin{equation*}
                \hat{\lambda} = -\fr{\ln\hat{a}}{\tau}\,;\quad
                \hat{\mu} = \fr{\hat{b}}{1-\hat{a}}\,; \quad
                \hat{\mu}_0 = \fr{\hat{c} - \hat{\mu} \tau}{1 - \hat{a}}\,.
        \end{equation*}

        Из независимости приращений также можно явно посчитать логарифмическую 
функцию правдоподобия: 
\begin{multline*}
L\left(X^*, \theta\right) =  \sum\limits_{k = 2}^n \ln 
f_{Y_i |Y_{i -1}}(X_i , \theta) = {}\\
{}=  \sum\limits_{k = 2}^n \ln 
f\left(Y_i - a Y_{i - 1} - c,\theta\right)\,,
\end{multline*}
где $\theta$~--- параметры модели.

\subsubsection{Симуляция}

     Используя соотношения авторегрессионного вида процесса ОУ, можно 
смоделировать процесс ОУ итеративно, задав некоторую начальную 
точку~$X_0$. На рис.~2 проиллюстрирован построенный 
итеративно  процесс ОУ с~положительным трендом.

{ \begin{center}  %fig2
 \vspace*{6pt}
 \mbox{%
\epsfxsize=77.781mm
\epsfbox{che-2.eps}
}
\end{center}

%\vspace*{-3pt}


\noindent
{{\figurename~2}\ \ \small{Пример процесса ОУ с~трендом ($\alpha\hm=0{,}5$, 
$\nu\hm=0{,}1$, $\mu_0\hm=0$ и~$\sigma\hm=1$)}}
}

\addtocounter{figure}{1}

\begin{table*}[b]\small
\begin{center}
\Caption{Характеристики дисперсионного гам\-ма-про\-цес\-са $V \hm= (V_t)_{t \geqslant 0}$}
\label{table1}
\vspace*{2ex}

                \begin{tabular}{|c|c|c|c|}
                        \hline
                       \tabcolsep=0pt\begin{tabular}{c}
                        Математическое\\ ожидание\end{tabular} &                         Дисперсия &
                                               Асимметрия&                         Эксцесс\\
                                               \hline
                                               &&&\\[-9pt]
                        $\theta t$  & $\left(\sigma^2 + \nu \theta^2\right) t$ 
 & $\displaystyle
                        \fr{\theta \nu \left(3 \sigma^ 2 + 2 \nu 
\theta^2\right)}{t^{{1}/{2}}} \left(\sigma^2 + \nu \theta ^2\right)^{{3/}{2}}$ 
 & $\displaystyle 3 \left( 1 + \fr{2 \nu}{t} - \nu \theta^4 
t \left(\sigma^2 + \nu \theta^2\right)^{-2} \right) $ \\
                        \hline
                \end{tabular}
        \end{center}
%\end{table*}
%\begin{table*}\small
\begin{center}
\Caption{Характеристики нормального обратного гауссовского распределения}
\label{table2}
\vspace*{2ex}

                \begin{tabular}{|c|c|c|c|}
                        \hline
                        Математическое ожидание & 
                                                Дисперсия &
                                                                        Асимметрия &
                                    Эксцесс \\
 \hline
 &&&\\[-9pt]
 $\mu + \delta \tau$ &
 $\displaystyle\fr{\delta^2(1 + \tau^2)}{\xi}$ 
& $\displaystyle\fr{3}{\tau \sqrt{\xi (1 + \tau^2)}}$ 
& $\displaystyle\fr{3}{\xi} \left(  1 + 4 \fr{\tau^2}
                        {1 + \tau^2}  \right)$  \\[8pt]
                        \hline
                \end{tabular}
        \end{center}
\end{table*}
        

\subsection{Частные случаи моделирования остатков процесса Орнштейна--Уленбека}

В работе~\cite{taufer} авторы приводят быстрые и~эффективные методики для 
симуляции различных ОУ-про\-цес\-сов, управляемых процессом Леви, а~также  
описание множества различных частных его случаев. Будем рассматривать три типа 
процессов Леви: винеровский процесс, дисперсионный гам\-ма-про\-цесс (VG), а~также 
нормальный обратный гауссовский процесс (NIG).  Оба последних процесса 
моделируют тяжелые хвосты и~принадлежат классу обобщенных гиперболических 
распределений. Они часто применяются в~финансах и~эконометрике (для 
VG см.~[12,  13]), для NIG см.~[10, 14--16]).

\subsubsection {Дисперсионный гамма-процесс}

\noindent
\textbf{Определение 2.1.}\
Случайная величина~$\xi$ имеет дисперсионное гам\-ма-рас\-пре\-де\-ле\-ние, 
если ее плотность распределения имеет вид:
\begin{multline} \label{vgpdf}
 f_\xi(x) = \int\limits_{0}^{\infty} \fr{1}{\sigma  \sqrt{2 \pi g}} 
 \exp \left( - \fr{(x - \theta g)^2}{2 \sigma^2 g}  \right) \times{}\\
 {}\times
\fr{g ^{{1}/{\nu} - 1} \exp \left( - {g}/{\nu} \right)}
{\nu  ^{{1}/{\nu}} \Gamma \left({1}/{\nu}\right) }\, dg\,,\enskip x \in \mathbb{R},
                \end{multline}
                где $\Gamma (x)$, $x\hm>0$,~--- гам\-ма-функ\-ция, 
                а~$\sigma \hm> 0$, $\nu \hm> 0$,  
$\theta \hm\in \mathbb{R}$.


        Обозначение: $\xi \sim V(\sigma, \nu, \theta)$.

\smallskip

\noindent
\textbf{Определение~2.2.}\
        Случайный  процесс  $V \hm= (V_t)_{t \hm\geqslant 0} $  с~ параметрами        
$\sigma\hm >0$, $\nu \hm> 0$, $\theta \hm\in \mathbb{R} $, заданный на вероятностном 
пространстве $ (\Omega, F, \mathbb{P}) $ со
        значениями в~$ \mathbb{R}$, называется дисперсионным гам\-ма-про\-цес\-сом, 
если $V_0\overset{\mathrm{p.n.}}{=} 0$, $V$ имеет независимые приращения и~для любых 
$s \hm\geqslant 0$, $t \hm\geqslant 0$ 
$V$ имеет стационарные приращения с~дисперсионным 
гам\-ма-рас\-пре\-де\-ле\-ни\-ем~(\ref{vgpdf}) с~параметрами $\sigma \sqrt{t}\hm > 0$, 
$\nu/t \hm> 0$ и~$t \theta\hm > 0$,~т.\,е.\
$$
V_{t+s} - V_s \overset{\mathrm{d}}{=} V_t - V_0 \sim 
V\left(\sigma \sqrt{t}, \fr{\nu}{t}, t \theta\right) \,.
$$

\smallskip

        Характеристики дисперсионного гам\-ма-про\-цес\-са $V \hm= (V_t)_{t \geqslant 0}$ 
с параметрами~$\sigma$, $\nu$ и~$\theta$ представлены в~табл.~1.


       
        В работе~\cite{madancarr} показано, что плотность дисперсионного 
        гамма-процесса  $V \hm= (V_t)_{t \geqslant 0}$ выражается аналитически с~использованием 
модифицированной функции Бесселя второго рода с~индексом~$\nu$.

        \subsubsection {Нормальный обратный гауссовский процесс}

\noindent
\textbf{Определение 2.3.}\  Случайная величина~$\eta$ имеет 
нормальное обратное гауссовское распределение с~параметрами~$\alpha$, $\beta$, 
$\delta$ и~$\mu$ ($\eta \hm\sim \mathrm{NIG}\,(\alpha, \beta, \delta, \mu)$), если ее плотность 
распределения имеет вид:
                \begin{multline*} 
%                \label{nigpdf}
                   \hspace*{-3mm}f_\eta(x, \alpha, \beta, \delta, \mu) = \fr{\alpha 
\delta}{\pi} \exp \left(\delta \sqrt{\alpha^2 - \beta^2} + \beta (x - \mu)\right) 
\times{}\\
{}\times \fr{K_1\left(\alpha \sqrt{\delta^2 + (x - \mu)^2}\right)}{\sqrt{\delta ^2 + (x - 
\mu)^2}}\,,
                \end{multline*}
                где $K_1(z) = (1/2) \int\nolimits_{0}^{\infty} \exp 
                (-({1}/{2}) z (u \hm+ u^{-1}))\,du$, $z\hm>0$,~--- 
                модифицированная функция Бесселя 
второго рода с~индексом~1, $\alpha \hm> 0$, $-\alpha \hm< \beta \hm< \alpha$, 
$\delta \hm> 0$,  $\mu \hm\in  \mathbb{R}$, $x\hm>0$.

                Параметры $\alpha$, $\beta$, $\delta$ и~$\mu$ являются параметрами 
формы, асимметрии, масштаба и~расположения соответственно. 

\smallskip

\noindent
\textbf{Определение 2.4.}\
                Случайный процесс $N \hm= (N_t)_{t \geqslant 0}$ с~параметрами 
$\alpha$, $\beta$, $\delta$ и~$\mu$, заданный на вероятностном пространстве $(\Omega, 
F, P)$ со значениями  в~$\mathbb{R}$,\linebreak
 называется нормальным обратным гауссовским 
процессом, если $N_0 \overset{\mathrm{p.n.}}{=} 0$, $N$ имеет независимые приращения 
и~для любых $s \hm\geqslant 0$, $t \hm\geqslant 0$ $N$ имеет стационарные приращения 
с~нормальным обратным гауссовским распределением:
$$
    N_{t+s} - N_s  \overset{\mathrm{d}}{=} N_t - N_0 \sim 
\mathrm{NIG}\,( \alpha, \beta, \delta t, \mu t) 
  $$
        с~параметрами $\alpha \hm> 0$, 
        $-\alpha \hm< \beta \hm< \alpha$, $\delta t\hm > 0$ 
и~$\mu t \hm\in  \mathbb{R}$.

\smallskip

    Плотность нормального обратного гауссовского распределения может быть 
представлена в~аналитической форме.

        Характеристики нормального обратного гауссовского распределения 
представлены в~табл.~\ref{table2}.



\section {Оценивание параметров и~поиск смен режима в~реальном времени}

В~данной разделе рассмотрено моделирование и~описание данных при условии их 
поступления в~режиме реального времени, когда значения выборки данных поступают 
одно за другим. Специфика данной задачи заключается в~высокой скорости 
поступления данных в~ее приложениях и~их большом объеме, поэтому любые 
приводимые алгоритмы должны быть достаточно быстрыми и~эффективно использовать 
компьютерную память.

\subsection{Оценивание параметров}

Без каких-ли\-бо ограничений на компьютерные мощности самым очевидным решением для 
оценивания параметров было бы на каждом шаге использовать метод наименьших 
квадратов (OLS). Чтобы удовлетворить необходимость обрабатывать потоковые 
данные, воспользуемся рекурсивным методом наименьших квадратов (RLS). Данный 
алгоритм на каждом шаге обновляет рекурсивно оценку параметра~$\theta$ 
и~ковариационную матрицу~$X^{\mathrm{T}} X$  вмес\-то того, чтобы насчитываться с~нуля каждый 
раз. Данный алгоритм и~его реализация хорошо известны и~могут быть найдены 
в~\cite{haykin}.

\subsection{Постановка простейшей смены режима}

        В реальной жизни некоторые явления могут быть связаны отношениями, 
например линейными, параметры которых изменяются во времени. Самым простым 
подобным примером является процесс, описываемый следующим образом:      
 \begin{equation*}
        M_t= 
                        \begin{cases}
                M_t^1\,, &\ t \leqslant t^*\,; \\
                M_t^2\,, &\ t > t^*\,,
                \end{cases}
        \end{equation*}
        где $t \in [1,\ldots,T]$ обозначает время; $ t^*$~--- критическое 
значение внешней переменной~$t$, или момент смены режима (change point, regime 
switch);  $M^{1,2}$~--- это две различные модели, соответствующие разным 
временным промежуткам: до и~после.
        В общем случае нельзя точно определить значение~$t^*$. Задача состоит 
        в~том, чтобы наилучшим образом оценить ее значение, имея на входе выборку 
наблюдений, при условии что на данном временн$\acute{\mbox{о}}$м промежутке произошла ровно одна 
смена режима (рис.~3), а~также оценить параметры старой и~новой модели.   
В~данном случае 
рассматривается модель ОУ и~исследуемый процесс выглядит 
следующим образом:

\columnbreak

\noindent
 \begin{center}  %fig1
 \vspace*{-2pt}
 \mbox{%
\epsfxsize=77.781mm
\epsfbox{che-3.eps}
}

\vspace*{3pt}

\noindent
{{\figurename~3}\ \ \small{Пример смены режима}}
\end{center}



 \vspace*{6pt}



\addtocounter{figure}{1}


\noindent
\begin {equation*}
        X_t =                \begin{cases}
                \mathrm{OU}_t^1\,, &\ t \leqslant t^*\,; \\
                \mathrm{OU}_t^2\,, &\ t > t^*\,,
                \end{cases}
        \end {equation*}
        где $\mathrm{OU}^i$~--- процессы ОУ, описанные выше.

        

\subsection{Постановка задачи для~потоковых данных}

            В случае потока данных значения выборки $X_1,X_2,\ldots,X_n, \ldots$ 
поступают последовательно. В~этом случае нет предпосылок для того, чтобы 
в~ка\-кой-то момент произошла смена режима, а~сами смены могут происходить 
последовательно много раз. Задача состоит в~том, чтобы последовательно их 
обнаруживать и~предоставлять оценку для параметров новой модели. Эффективность 
методов определяется тем, как часто метод ошибочно определяет режимы и~как 
быстро он способен обнаруживать смену режима.

\subsection{Решение задачи}

В современной литературе можно найти множество методов  для определения смен 
режима. Основным применяемым подходом является по\-стро\-ение по наблюдаемой системе  
некоторого детектора (change-detector), который сигнализирует, когда параметры 
модели перестают соответствовать выборке наблюдений и~предположительно сменился 
режим. Одним из таких методов  является CUSUM-тест, или метод кумулятивных сумм, 
который рассматривается в~данной работе для обнаружения смен режима. Стоит 
отметить другие известные методы, такие как метод GLT (generalized likehood 
test)~\cite{appel} и~MLT (marginalized likelihood text)~\cite{gustafsson}.

\subsubsection{Краткое описание CUSUM-методов}

В данной статье будут рассмотрены два базовых CUSUM-ме\-то\-да: CUSUM-ме\-тод, 
основанный на максимизации правдоподобия выборки наблюдений, и~CUSUM-ме\-тод, 
определяющий смену среднего значения выборки наблюдений.

\subsubsection*{CUSUM-метод максимизации правдоподобия}

Пусть есть некоторый поток данных $X^* \hm= X^*_1,\ldots X^*_n,\ldots $, 
элементы которого 
являются выборкой независимых одинаково распределенных случайных величин. 
Обозначим плотность соответствующей случайной величины через $p_\theta(x).$ 
Предполагается, что в~ка\-кой-то момент времени~$r^*$ может произойти смена режима 
модели. Это означает, что до~$r^*$ в~модели действует параметр~$\theta_0$, 
а~после~---~$\theta_1$. Введем соответствующие гипотезы: гипотезу <<неизменности>> 
модели~$H_0$ (разладки не произошло) и~гипотезу~$H_1$ об <<одноразовой 
разладке>>. Одним из самым известных и~простых методов для нахождения разладки 
модели является тест отношения правдоподобий~\cite{kay}.

\noindent
\textbf{Алгоритм 3.1.}\ %\begin{algorithm}
 Определим логарифмическое отношение правдоподобий для моделей~$H_0$ и~$H_1$: 
\begin{equation*}
\mathrm{LLR} \left(X, r^*\right) = 
\ln\fr{p_{H_1}(X)}{p_{H_0}(X)}\,.
\end{equation*} 
Тогда  принимается гипотеза~$H_1$, если $\mathrm{LLR} \hm>h$, где параметр~$h$  
отвечает за чувствительность 
алгоритма: чем меньше~$h$, тем быстрее будет происходить обнаружение разладки, 
но при этом тем больше будет срабатывать ложных сигналов.

\smallskip

К~сожалению,  в~данном случае неизвестно значение~$r^*$ и~поэтому явно посчитать 
$p_{H_1}(X)$ не представляется возможным. Данную проблему можно решить, перебрав 
все значения~$r^*$ и~взяв то, которому соответствует максимальное значение~$\mathrm{LLR}$. 
Данный метод называется обобщенным методом максимального 
правдоподобия (GLT).


\smallskip

\noindent
\textbf{Алгоритм 3.2.} %\begin{algorithm}\label{algo2}
Определим обобщенное логарифмическое отношение 
правдоподобий для выборки размера~$N$: 
\begin{multline*}
\mathrm{GLLR}(X) =\max\limits_{1   \leqslant r^* \leqslant N } 
\mathrm{LLR}(X, r^*)  ={}\\
{}=\max\limits_{1  \leqslant r^* \leqslant N } 
\ln\fr{p_{H_1}(X)}{p_{H_0}(X)}  =\max\limits_{1  \leqslant r^* \leqslant N } 
\sum\limits_{i = r^*}^N \ln \fr{p_{\theta_1}(X_i)}{p_{\theta_0}(X_i)}\,. 
%\label{gllr} 
\end{multline*}
Тогда на каждом шаге~$n$ принимается гипотеза~$H_1$  против гипотезы~$H_0$, если 
$\mathrm{GLLR}(X)\hm>h.$

\smallskip

Введем  кумулятивную сумму точечных отношений правдоподобий:
\begin{equation*}
S(n) = \sum\limits_{i = 1}^n \ln  
\fr{p_{\theta_1}(X_i)}{p_{\theta_0}(X_i)}\,. 
\end{equation*}
Тогда 
\begin{align*}
\mathrm{LLR} (X, r^*) &= S(N) - S(r^*)\,; \\
\mathrm{GLLR}(X, N) &= S(N) - \min\limits_{1  \leqslant r^* \leqslant N  }S(r^*)\,,
\end{align*}
где
$$
\hat{r}^* = \mathop{\mathrm{argmin}}\limits_{1  \leqslant r^* \leqslant N}S(r^*)\,.
$$
Заметим, что для того чтобы использовать алгоритм~3.2, достаточно 
считать кумулятивную сумму~$S$. Более того, так как уровень~$h$ берется 
положительным, вместо того, чтобы явно насчитывать значение~$\mathrm{GLLR}$ на 
каждом шаге, достаточно рекурсивно считать функцию 
\begin{equation*}
 G(N)  =\max\left(G(N- 1) +\ln \fr{p_{\theta_1}(X_N)}{p_{\theta_0}(X_N)}\,, 
\;0\right),
\end{equation*}
которая совпадает с~$\mathrm{GLLR}$ там, где последняя положительна. В~этом 
и~заключается метод кумулятивных сумм. Из вышесказанного вытекает следующий 
алгоритм, эквивалентный алгоритму~3.2:

\smallskip

\noindent
\textbf{Алгоритм 3.3.} %\begin{algorithm}
На каждом шаге~$N$ принимается гипотеза~$H_1$ против гипотезы~$H_0$, если 
$G(N)\hm>h.$


\smallskip

На практике значение~$\theta_0$ можно оценить, а~значение~$\theta_1$ неизвестно. 
Поэтому  берут $\theta_1 \hm= \theta_0 \hm+ \delta$, где~$\delta$~--- минимальная 
величина, изменение которой хотят детектировать.

\subsubsection*{CUSUM-метод определения смены среднего выборки}

%\smallskip

\noindent
\textbf{Алгоритм 3.4.} %\begin{algorithm}
Определим рекурсивно 
\begin{equation*} 
S_n^+=\max\left (S_{n- 1}^+ + 
\fr{X_N - \mu_0}{\sigma} - k, \;0\right)\,, 
\end{equation*}
где $\mu_0$~--- среднее текущей модели; $\sigma$~--- среднее текущей выборки; $k$~--- 
уровень чувствительности метода к~разбросам. Тогда считается, что принимается 
гипотеза~$H_1$, если $S^+_n \hm> h.$

\smallskip

Подробнее с~алгоритмом можно 
ознакомиться, например, в~\cite{page61}.

\begin{figure*}[b] %fig4
 \vspace*{1pt}
 \begin{minipage}[t]{79mm}
\begin{center}
\mbox{%
\epsfxsize=77.835mm
\epsfbox{che-5.eps}
}
\end{center}
\vspace*{-9pt}
  \Caption{Частичная автокорреляционная функция}\label{pacfpic}
  \end{minipage}
%\end{figure*}
\hfill
%\begin{figure*} %fig5
        \vspace*{1pt}
         \begin{minipage}[t]{79mm}
\begin{center}
\mbox{%
\epsfxsize=78.035mm
\epsfbox{che-4.eps}
}
\end{center}
\vspace*{-9pt}
  \Caption{Автокорреляционная функция}\label{acfpic}
    \end{minipage}
\end{figure*}

\subsection{Общий алгоритм для процесса Орнштейна--Уленбека}

Как было замечено ранее, приведенный процесс ОУ с~трендом обладает 
авторегрессионным свойством AR(1). Поэтому можно применять CUSUM-ме\-тод 
максимального правдоподобия для приращений~$l_i$ данного процесса. Данный метод 
будем\linebreak
 применять для детектирования смены во\-ла\-тиль\-ности модели, в~то время как 
для обнаружения смены среднего, или тренда, будем применять CUSUM-ме\-тод поиска 
смены среднего. Применить первый алгоритм для детектирования среднего 
оказывается сложно из-за большого числа параметров, которые нельзя адекватно 
оценить, в~частности па\-ра\-мет\-ров~$\mu_0$ и~$\nu$.

Таким образом, общий алгоритм следующий:

\smallskip

\noindent
\textbf{Алгоритм 3.5.} %%rithm}

\noindent
\begin{enumerate}[1.]
\item На каждом шаге оцениваем наиболее вероятные параметры выборки~$\theta_0$ 
для выбранной модели с~помощью метода RLS.
\item На каждом шаге считаем значения детекторов CUSUM смены волатильности 
и~смены тренда.
\item В случае, когда детекторы сигнализируют\linebreak о~смене режима, проходимся общим 
методом обобщенного отношения правдоподобий (GLT) по выборке и~находим наиболее 
вероятную точку смены режима~$r^*$ с~оценкой параметров~$\theta_1$. Далее 
исключаем из выборки все ее элементы до~$r^*$ и~продолжаем процедуру алгоритма 
с~$\theta_0:=\theta_1$.
\end{enumerate}


\section{Анализ данных}

       В этом разделе описывается моделирование процессом ОУ 
реальных финансовых данных. В~качестве данных были выбраны секундные данные по 
ценам фьючерсов на индекс RTS ($x_t$) и~на акции компании <<Газпром>> ($y_t$) 
с~MOEX за~07.10.2014. Если наблюдения сделаны через равные промежутки времени, то 
можно рассматривать их как временной ряд.
Предполагается, что разность $z_t \hm= x_t \hm - 6  y_t$ обладает свойством 
стационарности и~может быть описана с~помощью процесса ОУ. Для 
того чтобы ряд имел свойство авторегрессии, вычитаем из ряда его скользящее 
среднее с~периодом 5~мин.

Тест Дики--Фуллера (с уровнем зна\-чи\-мости 0,05) подтверждает предположение 
о~стационар\-ности: значение статистики Ди\-ки--Фул\-ле\-ра: $-25{,}374$; $p$-зна\-че\-ние: 
0,001. Тест  отвергает нулевую гипотезу о существовании единичного корня 
с~уровнем значимости~0,05, что подтверждает стационарность данного ряда.  Для 
проверки наличия свойства AR(1) проанализируем вид автокорреляционной (ACF) 
и~частичной автокорреляционной (PACF) функций.

         Для модели AR(1) характерен следующий вид автокорреляционных 
         и~частичных автокорреляционных функций: график ACF экспоненциально убывает, 
         а~график PACF имеет пик при значении сдвига, равном~1, и~практически равен~0 при 
значениях сдвига более высокого порядка.

       
       На рис.~\ref{pacfpic} изображен график PACF для рас\-смат\-ри\-ва\-емых данных. 
Заметно, что он имеет пик при сдвиге~1 и~практически равен нулю для сдвигов 
более высокого порядка.
        На рис.~\ref{acfpic} ACF для исходных данных убывает экспоненциально.
        Такое поведение графиков ACF и~PACF соответствует модели AR(1).

        Теперь можно говорить о том, что данные представляют собой процесс 
ОУ, и~перейти к~оценке его параметров.

        Для начала оценим параметры~$\theta$ и~$\alpha$ с~по\-мощью метода 
наименьших квадратов, получаем оценки параметров $\hat \theta \hm= 0$ и~$\hat \alpha 
\hm= 0{,}7506$.


        Для того чтобы оценить качество полученных оценок для данного процесса, 
построим QQ-плот для оцененных параметров нормального распределения для остатков 
(рис.~\ref{qqplot}). Это график, где по оси~$x$~--- квантили теоретического 
распределения, а~по оси~$y$~--- эмпирические квантили данных. Если теоретическое 
распределение хорошо описывает\linebreak\vspace*{-12pt}

\pagebreak

\end{multicols}

\begin{figure*} %fig6
        \vspace*{1pt}
\begin{center}
\mbox{%
\epsfxsize=161.601mm
\epsfbox{che-6.eps}
}
\end{center}
\vspace*{-11pt}
\Caption{Графики QQ-plot для оценивания параметров}\label{qqplot}
\vspace*{-3pt}
\end{figure*}

\begin{multicols}{2}

\noindent
 реальные данные, то график <<кван\-тиль--кван\-тиль>> 
близок к~прямой $y \hm= x$.


        Видно, что нормальное распределение не очень\linebreak хорошо описывает 
распределение остатков. По\-пробуем вместо нормального распределение\linebreak использовать 
распределение с~более тяжелыми хвостами, например дисперсионное 
гам\-ма-рас\-пре\-де\-ле\-ние и~нормальное обратное гауссовское распределение. Для оценки 
параметров этих распределений используем метод максимального правдоподобия, 
описанный в~п.~3.2.3.

       

        Анализируя графики QQ-plot (см.\ рис.~\ref{qqplot}) для оцененных параметров 
дисперсионного гамма- и~нормального обратного гауссовского распределений для 
остатков, можно прийти к~выводу, что дисперсионное гамма- и~нормальное обратное 
гауссовское распределение лучше описывают структуру независимых приращений 
в~процессе~ОУ.

        Для того чтобы оценить качество полученных результатов, применим критерий 
согласия Колмогорова для новой выборки данных. Результат подсчета статистики 
представлен в~табл.~3.



        По результатам, представленным в~табл.~3, \mbox{можно} заключить, 
что гипотеза о нормальном рас\-пре\-делении остатков отвергнута при уровне 
зна\-чи\-мости~0,01, гипотеза о~дисперсионном гам\-ма-рас\-пре\-де\-ле\-нии остатков 
и~нормальном обратном гауссовском распределении остатков принята при уровне 
значимости~0,01.

\vspace*{12pt}

\noindent
{{\tablename~3}\ \ \small{Оценки параметров с~результатом критерия Колмогорова}}

\vspace*{1pt}

{\small
 \begin{center}  %
\tabcolsep=3pt
                        \begin{tabular}{|c|c|c|c|c|}
                                \hline
                                \multicolumn{3}{|c|} {Оценка параметра} &
\tabcolsep=0pt\begin{tabular}{c} Значение\\ статистики \end{tabular}&  
$p$-значение  \\
                                \hline
\multicolumn{1}{|c|}{\raisebox{-6pt}[0pt][0pt]{$N(\mu, \sigma^2)$}}
                                & $\hat \mu$ & 0 &  &   \\ 
                              %  \cline{2-3}
                                & $\hat \sigma$ & 
3,6885 &
\multicolumn{1}{c|}{\raisebox{6pt}[0pt][0pt]{ 0,087731}} & 
\multicolumn{1}{c|}{\raisebox{6pt}[0pt][0pt]{$1{,}6679\cdot 10^{-12}$}}\\
                                \hline
                                & $\hat \sigma$ & 3,6712 &  &   \\ 
                                %\cline{2-3}
$\mathrm{VG}\,(\sigma, \nu, \theta)$& $\hat \nu$ & 1,5226 & 0,026806 &   0,14799  \\ 
%\cline{2-3}
                                & $\hat \theta$ & 0,0379 &  & 
\\
                                \hline
\multicolumn{1}{|c|}{\raisebox{-18pt}[0pt][0pt]{$\mathrm{NIG}\,(\theta, \xi, \delta, \mu)$}}
                                & $\hat \theta$ & 0,0592 &  &   \\ 
                                %\cline{2-3}
                                & $\hat \xi$ & 0,3714 &  &   \\ 
                                %\cline{2-3}
                                & $\hat \delta$ & 2,2690 &  &   \\ 
                                %\cline{2-3}
                                & $\hat \mu$ & $-$0,0963 & 
 \multicolumn{1}{c|}{\raisebox{18pt}[0pt][0pt]{0,034654}} & 
 \multicolumn{1}{c|}{\raisebox{18pt}[0pt][0pt]{0,025953}}  
\\
                                \hline
                        \end{tabular}
                        \vspace*{3pt}
\end{center}
}

\pagebreak

\end{multicols}

 \begin{figure*} %fig7
         \vspace*{1pt}
         \begin{minipage}[t]{80mm}
\begin{center}
\mbox{%
\epsfxsize=78.067mm
\epsfbox{che-7.eps}
}
\end{center}
\vspace*{-9pt}
        \Caption{Пример применения CUSUM-алгоритма}
        \label{cusum_vola}
%        \end{figure*}
\end{minipage}
\hfill
%        \begin{figure*} %fig8[H]
                 \vspace*{1pt}
                          \begin{minipage}[t]{80mm}
\begin{center}
\mbox{%
\epsfxsize=78.067mm
\epsfbox{che-8.eps}
}
\end{center}
\vspace*{-9pt}
               \Caption{Пример применения CUSUM-алгоритма}
        \label{cusum_mean}
        \end{minipage}
        \end{figure*}

\begin{multicols}{2}

        Полученный результат говорит о том, что структура реальных данных 
сложнее, чем может описать классический процесс ОУ. 
Целесообразнее использовать обобщенный процесс ОУ, где процесс 
броуновского движения заменен на процесс Леви.

        \subsection{Применение алгоритма для~детектирования изменения 
волатильности}

       Будем рассматривать гауссовский процесс ОУ. Для этого 
были сгенерированы две выборки процесса размера~100 с~$\sigma_1\hm=1$ и~$\sigma_2\hm=3.$ 
Для CUSUM-тес\-та будем брать $\theta_1\hm=2$, т.\,е.\ $\delta\hm=1$. Уровень $h\hm=45.$ 
Результат применения алгоритма проиллюстрирован на рис.~7. На 
рис.~7,\,\textit{а} изображена выборка сгенерированного процесса ОУ со сменой 
режима, в~то время как на рис.~7,\,\textit{б} построено значение CUSUM-де\-тек\-то\-ра. Сплошная 
вертикальная линяя обозначает фактическую смену режима, а~пунктирная~--- время ее 
обнаружения. Заметим, что смена режима могла бы быть обнаружена быстрее при 
другом выборе уровня~$h$.

        \subsection{Применение алгоритма для~обнаружения тренда}

        Для обнаружения тренда также были сгенерированы две выборки гауссовского 
процесса ОУ, которые потом были склеены. Параметры процессов 
следующие: $\alpha_0\hm=0{,}5$, $\nu_0\hm=0$, $\mu_0^0\hm=0$, 
$\sigma_0\hm=0{,}6$, $\alpha_1\hm=0{,}5$, 
$\nu_1\hm=0{,}05$, $\mu_0^1\hm=0$ и~$\sigma_1\hm=0{,}6$. 
Аналогично построена выборка и~значения 
CUSUM-де\-тек\-то\-ра. Уровень $h\hm=13.$ Алгоритм успешно определил смену режима 
(рис.~\ref{cusum_mean}).
       



\section {Заключение}

В статье рассмотрен процесс ОУ с~трендом, управ\-ля\-емый процессом 
Леви, для описания финансовых временн$\acute{\mbox{ы}}$х рядов.
На реальных данных было показано, что дисперсионный гамма- и~нормальный обратный 
гауссовский процессы в~качестве процесса Леви способны гораздо точнее описывать 
реальные явления. Также были рас\-смот\-ре\-ны проб\-ле\-мы разладки модели и~поиска смены 
режима в~реальном времени. Была представлена процедура обнаружения разладки 
модели, а~также определения параметров новой модели. Данный алгоритм способен 
детектировать многократные смены режима последовательно, сохраняя текущую модель 
актуальной для текущего потока данных.

{\small\frenchspacing
 {%\baselineskip=10.8pt
 \addcontentsline{toc}{section}{References}
 \begin{thebibliography}{99}
\bibitem{brigo2007}
    \Au{Brigo D., Dalessandro~A., Neugebauer~M., Triki~F.} A~stochastic 
processes toolkit for risk management.~--- London: King's College 
London, November 2007.  Working paper. 48~p.


\bibitem{ou1930}
\Au{Ornstein L.\,S., Uhlenbeck~G.\,E.} On the theory of the Brownian motion~// 
Phys. Rev., 1930. Vol.~36. No.\,5. P.~823.
    
    \bibitem{vasicek1977}
    \Au{Vasicek O.} An equilibrium characterization of the term structure~// 
J.~Financ. Econ., 1977. Vol.~5. P.~177.

\bibitem{cox1985}
        \Au{Cox J.\,C., Ingersoll E., Jr., Ross~S.\,A.} A~theory of the term 
structure of interest rates~//  Econometrica, 1985. Vol.~53. No.\,2.  P.~385--407.



\bibitem{GarOlk2000}
    \Au{Garbaczewski P., Olkiewicz~R.} Ornstein--Uhlenbeck--Cauchy process~// 
J.~Math. Phys., 2000. Vol.~41. P.~6843.

\bibitem{Fin2009}
    \Au{Finlay R.} The variance gamma (VG) model with long range dependence: 
A~model for financial data incorporating long range dependence in squared 
returns.~--- Sydney, Australia: University of Sydney, School of 
Mathematics and Statistics, 2009. PhD Thesis. 144~p.

\bibitem{Kuzmina2011}
    \Au{Кузьмина А.\,В.} Моделирование нормального обратного гауссовского 
процесса и~оценивание его параметров~// Информатика, 2011. №\,2. С.~133--136.

    \bibitem{Protter}
    \Au{Protter P.} Stochastic integration and differential equations.~--- 
Heidelberg: Springer-Verlag, 1990. 415~p.
    
           \bibitem{sato}
    \Au{Sato K.\,I.} L$\acute{\mbox{e}}$vy processes and 
    infinitely divisible distributions.~--- Cambridge: Cambridge University Press, 1999.
    500~p.
    
        \bibitem{nielsen}
\Au{Barndorff-Nielsen O.\,E., Shephard~N.} Non-Gaussian Ornstein--Uhlenbeck-based 
models and some of their uses in financial economics~// 
J.~Roy. Stat. Soc. B, 2001. Vol.~63. P.~167--241.
    
    \bibitem{taufer}
\Au{Taufer E., Leonenko~N.} Simulation of L$\acute{\mbox{e}}$vy-driven 
Ornstein--Uhlenbeck processes with given marginal distribution~// 
Comput. Stat. Data An., 2008. Vol.~53.  P.~2427--2437.

\bibitem{madanseneta}
    \Au{Madan D.\,B., Seneta~E.} The VG model for share market returns~// 
    J.~Bus., 1990. Vol.~63. P.~511--524.

\bibitem{madancarr}
\Au{Madan D.\,B., Carr P.\,P., Chang~E.\,C.} The variance gamma 
process and option pricing~// Eur. Finance Rev., 1998. Vol.~2. P.~79--105.

\bibitem{nielsen2}
 \Au{Barndorff-Nielsen O.\,E.} Normal inverse Gaussian distributions and 
stochastic volatility modelling~// Scand. J.~Stat., 1997. Vol.~24. No.\,1. P.~1--13.
    
   
    
    \bibitem{rydberg}
\Au{Rydberg H.} The Normal inverse Gaussian L$\acute{\mbox{e}}$vy process: Simulation 
and approximation~// Commun. Stat. Stochastic Models, 1997. 
Vol.~13. No.\,4. P.~887--910.

 \bibitem{nielsen3}
   \Au{Barndorff-Nielsen O.\,E.} Processes of normal inverse Gaussian type~// 
Financ. Stoch., 1998. Vol.~2. P.~41--68.

\bibitem{haykin}
    \Au{Haykin S.} Adaptive filter theory.~--- 3rd ed.~--- Upper Saddle River, NJ, USA: 
Prentice Hall, 1996. 989~p.

\bibitem{appel}
\Au{Appel U., Brandt~A.\,V.} Adaptive sequential segmentation of 
piecewise stationary time series~// Inform. Sci., 1983. Vol.~29. P.~27--56.

\bibitem{gustafsson}
\Au{Gustafsson F.} The marginalized likelihood ratio test for detecting 
abrupt changes~// IEEE Trans. Automat. Contr., 1996. Vol.~41. P.~66--78.
    
    \bibitem{kay}
    \Au{Kay S.} Fundamentals of statistical signal processing. Vol.~I. 
Estimation theory.~--- Upper Saddle River, NJ, USA: Prentice Hall, 1993. 625~p.

\bibitem{page61}
   \Au{Page E.\,S.} Cumulative sum control chart~// Technometrics, 1961. 
Vol.~3. P.~1--9.
 \end{thebibliography}

 }
 }

\end{multicols}

\vspace*{-6pt}

\hfill{\small\textit{Поступила в~редакцию 20.10.16}}

\vspace*{8pt}

%\newpage

%\vspace*{-24pt}

\hrule

\vspace*{2pt}

\hrule

\vspace*{8pt}


\def\tit{REGIME SWITCHING DETECTION FOR~THE~LEVY DRIVEN ORNSTEIN--UHLENBECK PROCESS 
USING CUSUM METHODS}

\def\titkol{Regime switching detection for the Levy driven Ornstein--Uhlenbeck process 
using CUSUM methods}

\def\aut{A.\,V.~Chertok$^{1,2}$, A.\,I.~Kadaner$^{2,3}$, G.\,T.~Khazeeva$^1$, 
and~I.\,A.~Sokolov$^4$}

\def\autkol{A.\,V.~Chertok, A.\,I.~Kadaner, G.\,T.~Khazeyeva, 
and~I.\,A.~Sokolov}

\titel{\tit}{\aut}{\autkol}{\titkol}

\vspace*{-9pt}

\noindent
$^1$Faculty of Computational Mathematics and Cybernetics, 
M.\,V.~Lomonosov Moscow State University, 1-52~Lenin-\linebreak
$\hphantom{^1}$skiye Gory, GSP-1, 
Moscow 119991, Russian Federation

\noindent
$^2$Sberbank of Russia, 19~Vavilov Str., Moscow 117999, Russian Federation

\noindent
$^3$Faculty of Mechanics and Mathematics, 
M.\,V.~Lomonosov Moscow State University, Main Building, Leninskiye\linebreak
$\hphantom{^1}$Gory, 
GSP-1, Moscow 119991, Russian Federation

\noindent
$^4$Federal Research Center ``Computer Science and Control'' 
of the Russian Academy of Sciences, 44-2~Vavilov\linebreak 
$\hphantom{^1}$Str., Moscow 119333, 
Russian Federation


\def\leftfootline{\small{\textbf{\thepage}
\hfill INFORMATIKA I EE PRIMENENIYA~--- INFORMATICS AND
APPLICATIONS\ \ \ 2016\ \ \ volume~10\ \ \ issue\ 4}
}%
 \def\rightfootline{\small{INFORMATIKA I EE PRIMENENIYA~---
INFORMATICS AND APPLICATIONS\ \ \ 2016\ \ \ volume~10\ \ \ issue\ 4
\hfill \textbf{\thepage}}}

\vspace*{3pt}



\Abste{The article considers using a trending Ornstein--Uhlenbeck process, driven 
by a~Levy process, for modeling financial time series. The authors demonstrate 
that the Levy driven model gives more flexibility to describe financial time series 
than the simple classical model. In particular, the Levy driven model allows 
modeling distributions with heavy tails, which is a~common property of time series 
in real applications. The authors describe efficient methods for estimating model 
parameters using such methods as OLS (ordinary least squares)
and RLS (regularized least squares). The article also solves the regime 
switching problem in a~real time data stream. The authors built an algorithm based 
on CUSUM (CUmulative SUM) methods that is capable of determining regime switches consecutively as 
they happen online and keep model parameters up to date. Solution of the regime 
switching problem is important in real applications, since the dynamics of real 
systems tend to change over time under the influence of external factors.} 

\KWE{random process; mean-reverting process; Ornstein--Uhlenbeck process driven 
by Levy process; trending Ornstein--Uhlenbeck process; regime switch; 
change point detection; CUSUM algorithm}



\DOI{10.14357/19922264160405} 

\vspace*{-16pt}

\Ack
\noindent
The research was partially supported by the Russian Foundation for Basic Research 
(project 14-07-00041).



%\vspace*{3pt}

  \begin{multicols}{2}

\renewcommand{\bibname}{\protect\rmfamily References}
%\renewcommand{\bibname}{\large\protect\rm References}

{\small\frenchspacing
 {%\baselineskip=10.8pt
 \addcontentsline{toc}{section}{References}
 \begin{thebibliography}{99}

\bibitem{1-ch-1}
\Aue{Brigo, D., A.~Dalessandro, M.~Neugebauer, and F.~Triki}. 2007. 
{A~stochastic processes toolkit for risk management}. 
London: King's College London.  Working paper. 48~p.
\bibitem{2-ch-1}
\Aue{Ornstein, L.\,S., and G.\,E.~Uhlenbeck}. 1930. On the theory of the Brownian motion. 
\textit{Phys. Rev.} 36(5):823.
\bibitem{3-ch-1}
\Aue{Vasicek, O.} 1977. An equilibrium characterization of the term structure. 
\textit{J.~Financ. Econ.} 5(2):177--188.
\bibitem{4-ch-1}
\Aue{Cox, J.\,C., E.~Ingersoll, Jr., and S.\,A.~Ross}. 1985. 
A~theory of the term structure of interest rates. \textit{Econometrica} 53(2):385--407.
\bibitem{5-ch-1}
\Aue{Garbaczewski, P., and R.~Olkiewicz}. 2000. Ornstein--Uhlenbeck--Cauchy process. 
\textit{J.~Math. Phys.} 41:6843--6860.
\bibitem{6-ch-1}
\Aue{Finlay, R.} 2009. The variance gamma (VG) model with long range dependence: 
A~model for financial data incorporating long range dependence in squared returns.
Sydney, Australia: University of Sydney, School of Mathematics and Statistics. 
 PhD Thesis. 144~p.
\bibitem{7-ch-1}
\Aue{Kuzmina, A.\,V.} 2011. Modelirovanie normal'nogo obratnogo gaussovskogo 
protsessa i~otsenivanie ego papametrov [Normal inverse Gaussian distribution 
modeling and its parameters estimation]. 
\textit{Vestnik Belorusskogo gosudarstvennogo universiteta. Ser.~1: Fizika. Matematika. 
Informatika} [Herald of the Belarusian State University. Ser.~1: 
Physics. Mathematics. Informatics] 2:133--136. 
\bibitem{8-ch-1}
\Aue{Protter, P.} 1990. 
\textit{Stochastic integration and differential equations.}  
Heidelberg: Springer-Verlag. 415~p.
\bibitem{9-ch-1}
\Aue{Sato, K.\,I.} 1999. \textit{L$\acute{\mbox{e}}$vy processes and infinitely divisible 
distributions.}  Cambridge: Cambridge University Press. 500~p.
\bibitem{10-ch-1}
\Aue{Barndorff-Nielsen, O.\,E., and N.~Shephard}. 2001. Non-Gaussian 
Ornstein--Uhlenbeck-based models and some of their uses in financial economics. 
\textit{J.~Roy. Stat. Soc.~B} 63:167--241.
\bibitem{11-ch-1}
\Aue{Taufer, E., and N.~Leonenko.} 2008. Simulation of L$\acute{\mbox{e}}$vy-driven 
Ornstein--Uhlenbeck processes with given marginal distribution. 
\textit{Comput. Stat. Data An.} 53:2427--2437.
\bibitem{12-ch-1}
\Aue{Madan, D.\,B., and E.~Seneta}. 1990. 
The VG model for share market returns. \textit{J.~Bus.} 63:511--524.
\bibitem{13-ch-1}
\Aue{Madan, D.\,B, P.\,P.~Carr, and E.\,C.~Chang}. 1998. 
The variance gamma process and option pricing. \textit{Eur. Finance Rev.} 2:79--105.
\bibitem{14-ch-1}
\Aue{Barndorff-Nielsen, O.\,E.} 1997. 
Normal inverse Gaussian distributions and stochastic volatility modeling. 
\textit{Scand. J.~Stat.} 24(1):1--13.

\bibitem{16-ch-1}
\Aue{Rydberg, H.} 1997. The normal inverse Gaussian L$\acute{\mbox{e}}$vy process: 
Simulation and approximation. \textit{Commun. Stat. Stochastic Models} 
13(4):887--910.
\bibitem{15-ch-1}
\Aue{Barndorff-Nielsen, O.\,E.} 1998. Processes of normal inverse Gaussian type. 
\textit{Financ.  Stoch.} 2:41--68.
\bibitem{17-ch-1}
\Aue{Haykin, S.} 1996. \textit{Adaptive filter theory}. 3rd ed. 
Upper Saddle River, NJ: Prentice Hall. 989~p.
\bibitem{18-ch-1}
\Aue{Appel, U., and A.\,V.~Brandt}. 1983. 
Adaptive sequential segmentation of piecewise stationary time series.  
\textit{Inform. Sci.} 29:27--56.
\bibitem{19-ch-1}
\Aue{Gustafsson, F.} 1996. The marginalized likelihood ratio test for 
detecting abrupt changes. \textit{IEEE Trans. Automat. Contr.} 41:66--78.
\bibitem{20-ch-1}
\Aue{Kay, S.} 1993. \textit{Fundamentals of statistical signal processing. Vol.~I. 
Estimation theory}.  Upper Saddle River, NJ: Prentice Hall. 625~p.
\bibitem{21-ch-1}
\Aue{Page, E.\,S.} 1961. Cumulative sum control chart. 
\textit{Technometrics} 3:1--9.
\end{thebibliography}

 }
 }

\end{multicols}

\vspace*{-6pt}

\hfill{\small\textit{Received October 20, 2016}}

\vspace*{-18pt}

\Contr

\vspace*{-2pt}

\noindent
\textbf{Chertok Andrey V.} (b.\ 1987)~--- 
junior scientist, Faculty of Computational Mathematics and Cybernetics, 
M.\,V.~Lo\-monosov Moscow State University, 1-52~Leninskiye Gory, GSP-1, Moscow 119991, 
Russian Federation; Head of R\&D, Data Science, Sberbank of Russia, 19~Vavilov Str.,
Moscow 117999, Russian Federation; \mbox{avchertok.sbt@sberbank.ru}

 \vspace*{1pt}
 
\noindent
\textbf{Kadaner Arsenii I.} (b.\ 1995)~--- 
student, Faculty of Mechanics and Mathematics, 
M.\,V.~Lomonosov Moscow State University, 
Main Building, Leninskiye Gory, GSP-1, Moscow 119991, Russian Federation; 
data scientist, Sberbank of Russia, 19~Vavilov Str., Moscow 117999, 
Russian Federation; \mbox{aikadaner.sbt@sberbank.ru}

  \vspace*{1pt}
 
\noindent
\textbf{Khazeeva Gelana T.} (b.\ 1993)~---
 student,  Faculty of Computational Mathematics and Cybernetics, M.\,V.~Lo\-monosov 
 Moscow State University, 1-52~Leninskiye Gory, GSP-1, Moscow 119991, 
 Russian Federation; \mbox{gelana.khazeyeva@gmail.com} 

 
 \vspace*{1pt}
 
\noindent
\textbf{Sokolov Igor A.} (b.\ 1954)~---
Academician of the Russian Academy of Sciences, Doctor of Science in technology, 
Director, Federal Research Center ``Computer Science and Control'' of 
the Russian Academy of Sciences, 44-2~Vavilov Str., Moscow 119333, Russian Federation; 
\mbox{isokolov@ipiran.ru}
\label{end\stat}


\renewcommand{\bibname}{\protect\rm Литература}  %5

\def\stat{konovalov}

\def\tit{ОБ АДАПТИВНЫХ СТРАТЕГИЯХ И~УСЛОВИЯХ~ИХ~СУЩЕСТВОВАНИЯ$^*$}

\def\titkol{Об адаптивных стратегиях и~условиях их 
существования}

\def\autkol{М.\,Г.~Коновалов}

\def\aut{М.\,Г.~Коновалов$^1$}

\titel{\tit}{\aut}{\autkol}{\titkol}

{\renewcommand{\thefootnote}{\fnsymbol{footnote}}\footnotetext[1]
{Работа выполнена при поддержке РФФИ, грант № 11-07-00112.}}

\renewcommand{\thefootnote}{\arabic{footnote}}
\footnotetext[1]{Институт проблем информатики Российской академии наук, mkonovalov@ipiran.ru}



\Abst{Рассматривается задача оптимального управления в отсутствие априорной 
информации об управляемом объекте. Решением задачи является построение адаптивных 
стратегий на основе наблюдений, доступных в процессе управления. Изучаются 
некоторые условия адаптивной управляемости объекта. В~качестве математической 
модели используются управляемые случайные последовательности.}

\KW{управляемые случайные последовательности; адаптивные стратегии; условия 
существования}

\vskip 14pt plus 9pt minus 6pt

      \thispagestyle{headings}

      \begin{multicols}{2}

            \label{st\stat}


\section{Введение}

  Тема статьи относится к области адаптивных методов обработки информации с целью 
принятия оптимальных решений. Потребность в адаптивном\linebreak
подходе возникает в задачах 
с большой информационной неопределенностью, что наиболее характерно для 
телекоммуникационных систем, автоматизированных производственных процессов, 
робототех\-ни\-ки и других сфер, неразрывно связанных с компьютерной обработкой 
информации. Понятие неопределенности многозначно и связано с отсутствием априорных 
сведений, недетерминированностью, а также с неполнотой наблюдений. 
К~перечисленным факторам в нарастающей степени добавляется <<избыточность>> 
информации, которая порождается чрезмерно прогрессирующими объемами 
передаваемой и хранимой информации и обусловлена экспоненциальным ростом 
пропускной способности телекоммуникационных сетей, а также емкостей носителей 
информации.
  
  Идея адаптации (приспособления, самоорганизации), заимствованная из 
биологического мира, начала активно эксплуатироваться в науке примерно с середины 
прошлого века. Кратко, она заключается в том, чтобы, целенаправленно взаимодействуя с 
окружающей средой, отбирать и использовать поступающую информацию, необходимую 
для принятия оптимальных решений с точки зрения поставленной цели.
  
  Данная статья посвящена теоретическим аспектам адаптации. В~качестве исходного 
пред\-став\-ле\-ния использована схема, которая опирается на пред\-став\-ление о паре 
  <<объект--субъект>>, взаимодействующей в дискретном времени путем 
попеременного обмена сигналами. При этом субъект воздействует на объект с помощью 
управлений, получая в ответ сигналы, называемые наблюдениями. Действия субъекта 
преследуют цель, выраженную в наличии определенных свойств у траектории 
наблюдений.
  
  Основная отличительная особенность заключается в предположении, что действия 
субъекта происходят при недостаточной информации об объекте. В~качестве 
математической модели объекта взята конструкция управляемой случайной 
последовательности. В~терминах этого аппарата легко очерчиваются четыре аспекта 
информационной неопределенности:
  \begin{enumerate}[(1)]
\item недетерминированность понимается как стохастичность;
\item недостаток информации об объекте трактуется как неполное знание вероятностного 
распределения, задающего процесс;
  \item неполнота наблюдений означает, что состояния процесса наблюдаются лишь 
частично;
  \item недостаток знаний выражается в неумении \mbox{найти} или рассчитать ту или иную 
характеристику, связанную со случайной последовательностью, даже при наличии 
априорной информации о распределении процесса и полной его наблюдаемости.
  \end{enumerate}
  
  Субъект ассоциируется с алгоритмом, согласно которому выбираются управления, 
регулирующие траекторию случайной последовательности. Такой алгоритм принято 
называть стратегией управ\-ле\-ния. Задача заключается в том, чтобы выбрать стратегию, 
достигающую цели в ситуации, когда информация субъекта об объекте ограничена. 
По-дру\-го\-му можно сказать, что речь идет о построении стратегии, достигающей цели (в 
данном случае~--- максимизации предельного среднего дохода) для любого процесса из 
некоторого заданного класса объектов. Такие стратегии называют адаптивными по 
отношению к заданному классу объектов~[1].
  
  В разд.~2 даются формальные определения объекта, цели и адаптивной стратегии 
управления.
  
  В разд.~3 анализируются условия существования адаптивной стратегии. В~качестве 
необходимых условий обсуждаются два требования, которые, как представляется, должны 
выполняться из интуитивных соображений.
  
  Первое из необходимых условий связано с принципиальной особенностью адаптивных 
стратегий, которые, прежде чем выйти на <<оптимальный режим>>, должны затратить 
некоторое время на <<обуче\-ние>>. (На самом деле в рассматриваемой постановке процесс 
обучения для адаптивных стратегий длится даже неограниченно долго.) Естественно 
предположить, что подобные стратегии могут реализоваться, только если в процессе 
обучения не будут совершены <<непоправимые ошибки>>. Это соображение 
раскрывается на примерах и получает формальное описание.
  
  Второе необходимое условие является менее очевидным. Оно связано с гипотезой о 
том, что адаптивная стратегия управления классом случайных последовательностей 
существует лишь тогда, когда для данного класса возможно построение так называемой 
адаптивной стратегии перебора. Это выражается в том, что существует и заранее известно 
некоторое счетное множество вариантов поведения, среди которого для данного класса 
обязательно найдется оптимальный или близкий к нему вариант. Данное соображение 
также иллюстрировано примерами и приведена теорема о критерии существования 
адаптивной стратегии для определенного класса объектов.
  
  Подход, использованный в статье, а также полученные результаты являются 
продолжением направления, представленного в работе~[2].
  
\section{Постановка задачи адаптивного управления}
  
  Пусть  время $t$ пробегает значения 0, 1, \ldots\ и пусть заданы измеримые 
пространства $(X,\mathbf{X})$, $(Y,\mathbf{Y})$, $(Z,\mathbf{Z})$ (соответственно 
пространства \textit{состояний}, \textit{управлений} и \textit{наблюдений}).
  
  Общая траектория процесса упорядочена в виде последовательности $x_0, y_1, 
z_1,x_1,\ldots$\linebreak $\ldots , x_{t-1},y_t,z_t,x_t,\ldots$ Предыстория процесса до момента~$t$ 
включительно обозначается как

\noindent
  \begin{gather*}
 \! x^t=x_0^t=(x_0,\ldots , x_{t-1});\ \ \ y^t=y_1^t=(y_1, \ldots , y_{t-1});\\
  z^t=z_1^t=(z_1,  \ldots , z_{t-1})\,.
  \end{gather*}
  
  Траектории процесса определяются последовательностями условных вероятностных 
распределений~$\mu$, $\nu$ и~$\sigma$.
  
  Последовательность $\mu\hm=(\mu_0,\mu_1,\ldots ,\mu_t, \ldots)$ задает механизм 
смены состояний. В~этой последовательности $\mu_0$~--- вероятностное распределение 
на $(X,\mathbf{X})$; $\mu_t=\mu_t(A\vert x^{t-1},y^t)$, $t\hm>0$~---  условная 
(переходная) вероятность, которая при любых наборах $(x^{t-1},y^t)$ является 
вероятностной мерой на $(X,\mathbf{X})$ и при любом $A\hm\in X$ является измеримой 
функцией относительно $x^{t-1},y^t$.
  
  Последовательность $\nu\hm=(\nu_1, \ldots , \nu_t, \ldots)$ задает механизм появления 
наблюдений. В~этой последовательности каждый элемент $\nu_t\hm=\nu_t(C\vert x^{t-1}, 
y^t)$, $t\hm>0$, представляет собой условное распределение, которое при любом условии 
является вероятностной мерой на $(Z,\mathbf{Z})$ и для любого $C\hm\in Z$ является 
измеримой функцией относительно переменных, стоящих в условии. Пара $o\hm= 
(\mu,\nu)$ называется объектом.
  
  Последовательность $\sigma\hm= (\sigma_1, \ldots , \sigma_t. \ldots)$ называется 
(допустимой) \textit{стратегией} и определяет выбор управлений. В~этой 
последовательности:
%\smallskip
   $\sigma_1\hm=\sigma_1(\cdot)$~--- вероятностная мера на $(Y,\mathbf{Y})$; 
      $\sigma_{t+1}\hm=\sigma_{t+1}(B\vert y^t,z^t)$, $t\hm>0$,~--- условная вероятность, 
которая при любых $y^t,z^t$ является вероятностной мерой на $(Y,\mathbf{Y})$ и при 
любом $B\hm\in Y$ является измеримой функцией относительно $y^t,z^t$. Элементы 
последовательности~$\sigma$ называются (допустимыми) \textit{правилами}.

%\smallskip
  
  Введем обозначение для прямых произведений множеств:
  $$
  \Omega_0=X\,;\enskip \Omega_t=X^{t+1}\times Y^t\times Z^t\,,\enskip t>0\,,
  $$
а также для наименьших $\sigma$-ал\-гебр, порожденных соответствующими 
$\sigma$-ал\-геб\-рами:
$$
\mathbf{F}_0=\mathbf{X}\,;\enskip \mathbf{F}_t=\mathbf{X}\otimes \mathbf{Y}\otimes 
\mathbf{Z}\otimes \mathbf{X}\otimes \cdots \otimes \mathbf{Y}\otimes \mathbf{Z}\otimes 
\mathbf{X}
$$
($\mathbf{X}$ повторяется $t+1$ раз, $\mathbf{Y}$ и $\mathbf{Z}$~--- $t$ раз, $t\hm>0$).
  
  Положим
  
  \vspace*{3pt}
  
  \noindent
  $$
  \Omega =\prod\limits_{t\geq 0}\Omega_t\,;\enskip 
\mathbf{F}=\mathop{\otimes}\limits_{t\geq0}\mathbf{F}_t\,.
  $$ 
  
  Согласно общей теории~\cite{3-kon} последовательности $o\hm=(\mu,\nu)$ и~$\sigma$ 
порождают на пространстве $(\Omega, \mathbf{F})$ вероятностную меру $\mathbf{P}\hm= 
\mathbf{P}_{o,\sigma}\hm=\mathbf{P}_{\mu,\nu,\sigma}$, которая согласована с 
элементами этих последовательностей следующим образом. Случайные 
последова\-тель\-ности

\vspace*{-3pt}

\noindent
  \begin{gather*}
  x_t=x_t(\omega)\,;\enskip  
  y_{t+1}=y_{t+1}(\omega)\,;\\
  z_{t+1}= z_{t+1}(\omega)\,,\enskip  \omega\in \Omega\,,\  t\geq 0\,,
  \end{gather*}
удовлетворяют соотношениям:

\pagebreak

\noindent
$$
\mathbf{P}(x_0(\omega)\in A_0)=\int\limits_{A_0} \mu_0(dx_0)\,;
$$

\vspace*{-12pt}

\noindent
\begin{multline*}
\mathbf{P}\left(x_0(\omega)\in A_0\,,\  y_1(\omega)\in B_1\,,\ 
z_1(\omega)\in C_1, \ldots \right.\\[1pt]
\left.{}\ldots\,,
y_t(\omega)\in B_t\,,\  z_t(\omega)\in C_t\,,\  x_t(\omega)\in A_t\right)={}\\[1pt]
{}=\int\limits_{A_0}\mu_0(dx_0)\int\limits_{B_1}\sigma_1(dy_1)\int\limits_{C_1}\nu_1(dz_1
\vert x_0, y_1)\cdots{}\\[1pt]
{}\cdots
\int\limits_{B_t}\sigma_t\left(dy_t\vert y^{t-1},z^{t-1}\right) 
\int\limits_{C_t} \nu_t\left( dz_t\vert x^{t-
1},y^t\right) \times{}\\[1pt]
{}\times
\int\limits_{A_t} \mu_t\left( dx_t\vert x^{t-1},y^t\right)
\end{multline*}
для любых $A_t\in X$, $B_{t+1}\hm\in Y$, $C_{t+1}\hm\in Z$, $t\hm\geq 0$.
  
  По определению стратегии, ее правила зависят от предыдущих управлений и 
наблюдений, но не от предыдущих состояний. Это соответствует предположению о том, 
что состояния объекта не наблюдаемы в ходе процесса управления. В~частных случаях 
объект $o\hm=(\mu,\nu)$ может, конечно, описывать полностью наблюдаемый процесс. 
Например, если все множества $X_t$ содержат один и тот же единственный элемент. 
Другой простой пример~--- когда наблюдения тождественны состояниям. Однако на 
самом деле, как показывает лемма~1, с формальной точки зрения рассмотрение объекта с 
<<ненаблюдаемой>> компонентой всегда можно заменить изучением полностью 
наблюдаемого процесса.
  
  \medskip
  
  \noindent
  \textbf{Лемма 1.} \textit{Для любого объекта $o\hm=(\mu,\nu)$ условная вероятность 
$\mathbf{P}\left(dz_t\vert y^t,z^{t-1}\right)$ не зависит от стратегии~$\sigma$ при любых 
$t\hm>0$.}
  
  \medskip
  
  \noindent
  Д\,о\,к\,а\,з\,а\,т\,е\,л\,ь\,с\,т\,в\,о\,.\ Согласно отмеченной выше согласованности 
условных распределений $\mu,\nu,o$ и порождаемой ими меры~\textbf{P} имеем 
соотношения:
  \begin{multline*}
  I_1=\mathbf{P}\left(
  y_1(\omega)\in B_1,\ z_1(\omega)\in C_1\right) ={}\\[1pt]
  {}=
  \mathbf{P}\left( x_0(\omega)\in X_0\,,\ y_1(\omega)\in B_1\,,\ z_1(\omega)\in 
C_1\right)={}\\[1pt]
  {}=\int\limits_{X_0} \int\limits_{B_1} \int\limits_{C_1} \mu_0\left(dx_0\right) 
\sigma_1\left(dy_1\right) \nu_1\left(dz_1\vert x_0,y_1\right)={}\\[1pt]
  {}= \int\limits_{B_1}\int\limits_{C_1}\sigma_1\left(dy_1\right) \int\limits_{X_0}\mu_0\left( 
dx_0\right) \nu_1\left( dz_1\vert x_0,y_1\right)\,,
  \end{multline*}
справедливые при любых $B_1\hm\in Y$ и $C_1\in Z$. Кроме того, по определению 
условной вероятности
$$
I_1=\int\limits_{B_1}\int\limits_{C_1}\sigma_1\left(dy_1\right) \mathbf{P}\left(dz_1\vert 
y_1\right)\,.
$$
  
  Сравнивая оба выражения для~$I_1$, получаем, что
  $$
  \mathbf{P}\left( dz_1\vert 
y_1\right)=\int\limits_{X_0}\mu_0\left(dx_0\right)\nu_1\left(dz_1\vert x_0, y_1\right)\,,
  $$
т.\,е.\ утверждение леммы справедливо для $t\hm=1$. Пусть оно верно для $n\hm=1, 2, 
\ldots , t\hm-1$. Для любых $B_1\hm\in Y$, $C_1\hm\in Z$, \ldots , $B_{t-1}\hm\in Y$, 
$C_t\hm\in Z$ имеем:

\noindent
\begin{multline*}
I_t=\mathbf{P}\left( y_1(\omega)\in B_1\,,\ z_1(\omega)\in C_1, \ldots{}\right.\\[1pt]
\left.{}\ldots , y_t(\omega)\in B_t\,,\ 
z_t(\omega) \in C_t\right)={}\\[1pt]
{}=
\mathbf{P}\left( x_0(\omega)\in X\,,\ y_1(\omega)\in B_1\,,\ z_1(\omega)\in C_1\,, \ldots\right.\\[1pt]
\left.{}\ldots , x_{t-
1}(\omega)\in X\,,\ y_t(\omega)\in B_t\,,\ z_t(\omega)\in C_t\right)={}\\[1pt]
{}=
\int\limits_X \int\limits_{B_1} \int\limits_{C_1}\ldots \\[1pt]
\ldots\int\limits_X \int\limits_{B_t} 
\int\limits_{C_t} \mu_0\left( dx_0\right) \sigma_1\left( dy_1\right) \nu_1\left( dz_1\vert 
x_0,y_1\right)\cdots{}\\[1pt]
\cdots \mu_{t-1}\left( dx_{t-1}\vert x^{t-2} y^{t-1}\right) \sigma_t \left( dy_t\vert y^{t-
1},z^{t-1}\right)\times{}\\[1pt]
{}\times \nu_t\left( dz_t\vert x^{t-1},y^t\right)={}\\[1pt]
{}=\int\limits_{B_1} \sigma_1\left( dy_1\right) \int\limits_{C_1} \int\limits_{B_2} 
\sigma_2\left( dy_2\vert z_1\right)\cdots\\[1pt]
\cdots \int\limits_{C_{t-1}}\int\limits_{B_t} \sigma_t \left( 
dy_t\vert y^{t-1},z^{t-1}\right)\times{}\\[1pt]
{}\times \int\limits_X \mu_0\left(dx_0\right) \nu_1\left( dz_1\vert 
x_o,y_1\right)\cdots{}\\[1pt]
{}\cdots \int\limits_{X_{t-1}}\mu_{t-1}\left( dx_{t-1}\vert x^{t-2} y^{t-1}\right) \nu_t \left( 
dz_t\vert x^{t-1}, y^t\right)={}\\[1pt]
{}=\int\limits_{B_1} \sigma_1\left( dy_1\right) \int\limits_{C_1} 
\int\limits_{B_2}\sigma_2\left( dy_2\vert z_1\right)\cdots\\[1pt]
\cdots \int\limits_{C_{t-1}} 
\int\limits_{B_t} \sigma_t\left( dy_t\vert y^{t-1},z^{t-1}\right) \int\limits_{C_t} 
\mathbf{P}\left( dz_1\vert y_1\right)\ldots{}\\[1pt]
{}\cdots \mathbf{P}\left( dz_{t-1}\vert y^{t-1},z^{t-2}\right) \mathbf{P}\left( dz_t\vert y^t, 
z^{t-1}\right)\,.
\end{multline*}
Отсюда получаем, что

\noindent
  \begin{multline*}
\hspace*{-6.95218pt}\mathbf{P}\left( dz_1\vert y_1\right)\cdots \mathbf{P}\left( dz_{t-1}\vert y^{t-1},z^{t-
2}\right) \mathbf{P}\left( dz_t\vert y^t,z^{t-1}\right)={}\\[1pt]
  {}=\int\limits_X \mu_0\left( dx_0\right) \nu_1\left( dz_1\vert x_o,y_1\right)\cdots \\[1pt]
  \cdots
\int\limits_X \mu_{t-1}\left( dx_{t-1}\vert x^{t-2}y^{t-1}\right) \nu_t\left( dz_t\vert x^{t-
1},y^t\right)\,.
  \end{multline*}
  
  Следовательно, по предположению индукции $\mathbf{P}\left( dz_t\vert y^t,z^{t-
1}\right)$ не зависит от~$\sigma$.
  
  Таким образом, не уменьшая общности, можно ограничиться (что и будет сделано в 
оставшейся части текста) рассмотрением полностью наблюда-\linebreak\vspace*{-12pt}

\pagebreak

\noindent
емых объектов $o\hm=\mu$, 
управляемых (допустимыми) стратегиями~$\sigma$ c правилами вида
  $$
  \sigma_1=\sigma_1\left(\cdot\right)\,;\enskip \sigma_{t+1}=\sigma_{t+1}\left( \cdot \vert 
y^t,x^t\right)\,,\enskip t>0\,.
  $$
(Множество всех таких стратегий при заданных пространствах состояний и управлений 
далее обозначается через~$\Sigma$.) В~этом случае вероятностная мера 
$\mathbf{P}\hm=\mathbf{P}_{\mu,\sigma}$ определена на пространстве $(\Omega, 
\mathbf{F})$, в котором $\Omega\hm=\prod\limits_{t\geq0} X^{t+1}\times Y^t$, 
$\mathbf{F}\mathop{\otimes}\limits_{t\geq0} \mathbf{F}_t$, где $\mathbf{F}_0\hm=\mathbf{X}$; 
$\mathbf{F}_t=\mathbf{X}\otimes \mathbf{Y}\otimes \mathbf{X}\otimes \cdots \otimes 
\mathbf{Y}\otimes \mathbf{X}$ и согласована с последовательностями~$\mu$ и~$\sigma$. 
Через $\mathbf{F}_t$ обозначена $\sigma$-ал\-геб\-ра, порожденная предысторией 
$(x^t,y^t)$ до момента~$t$ включительно.
  
  В то же время необходимо заметить, что предположение о наличии 
<<двухступенчатой>> структуры у объектов (со\-сто\-яние--наблю\-де\-ние) может 
принести пользу при их изучении. Так происходит, например, в теории частично 
наблюдаемых управляемых марковских процессов.
  
  Предположим далее, что на наблюдаемой части траектории процесса задан 
одношаговый доход (в момент~$t$), и будем считать, что этот доход имеет вид 
$g_t\hm=g(x_t)$, где $g:\ X\rightarrow (0,\,1)\subset \mathbb{R}$~--- измеримая числовая 
функция со значениями из интервала (0,\,1).
  
  Обозначим через $v_{t,s}\hm=s^{-1}\sum\limits_{n=1}^s g_{t+n}$ среднее 
арифметическое доходов на промежутке от $t+1$ до $t\hm+s$ ($t\hm\geq0$, $s\hm\geq 1$).
  
  Если объект~$\mu$ управляется согласно стратегии~$\sigma$, то число
  $$
  w_t(\mu,\sigma) =\sup \left\{ c:\ \mathbf{P}_{\mu,\sigma} \left( 
\lim\limits_{\overline{s\rightarrow\infty}} v_{t,s}>c\right) =1\right\}
  $$
характеризует получаемый при этом гарантированный предельный средний доход 
начиная с момента $t=1$. Поскольку $\lim\limits_{\overline{s\rightarrow\infty}} v_{t,s}$ не 
зависит от~$t$, то $w_0(\mu,\sigma)\hm=w_1(\mu,\sigma)\hm=w_2(\mu,\sigma)\hm=\cdots$. 
Величина $w(\mu,\sigma)\hm=w_0(\mu,\sigma)$ играет в дальнейшем роль целевой 
функции и называется просто \textit{доходом} (при управлении объектом~$\mu$ с 
помощью стратегии~$\sigma$).
  
  Из определения дохода следует, что для любого $t>0$ выполняется условие
  $$
  \mathbf{P}_{\mu,\sigma}\left( \lim\limits_{\overline{s\rightarrow\infty}} v_{t,s}\geq 
w(\mu,\sigma)\vert \mathbf{F}_{t-1}\right)=1
  $$
почти наверное.
  Столь общее определение дохода, без предположений об эргодичности, оказывается 
полезным в теоретических рассмотрениях, однако на практике все же среднее 
арифметическое ведет себя более или менее регулярным образом. Поэтому введем 
следующее определение.
{ %\looseness=1

}
  
  Стратегия~$\sigma$ называется \textit{эргодической} по отношению к классу~$M$, 
если для любого объекта $\mu\hm\in M$ и любого $\varepsilon\hm>0$ выполняется 
условие $\sum\limits_{s=1}^\infty a_s\hm<\infty$, где $a_s\hm= 
a_s(\mu,\sigma,\varepsilon)\hm=\sup\limits_{t\geq0} \mathbf{P}_{\mu,\sigma}\left( \left\vert 
v_{t,s}-w(\mu,\sigma)\right\vert >\varepsilon\vert \mathbf{F}_t\right)$. Обозначим еще
  $$
  W=W(\mu) =\sup\limits_\sigma w(\mu,\sigma)\,,
  $$
где точная верхняя грань берется по всем допустимым стратегиям. Стратегия~$\sigma$ 
называется $\varepsilon$-\textit{оп\-ти\-маль\-ной}, если выполняется неравенство
$$
w(\mu,\sigma)\geq W-\varepsilon\,,\enskip \varepsilon\geq 0\,.
$$
  
  Далее объекты будут объединяться в множества объектов (классы объектов). При этом 
без дополнительных оговорок всюду предполагается, что
  \begin{itemize}
  \item все объекты из класса имеют одинаковые пространства состояний, управлений (и 
наблюдений);
  \item в качестве множества допустимых стратегий берется определенное выше 
множество~$\Sigma$;
  \item функция одношаговых доходов~$g$ одна и та же для всех объектов.
  \end{itemize}
  
  Пусть $M$~--- класс объектов. Стратегия~$\sigma$ является равномерно 
  $\varepsilon$-оп\-ти\-маль\-ной относительно этого класса, если последнее неравенство 
выполняется для всех $\mu\hm\in M$. Такую стратегию будем называть также 
  $\varepsilon$-\textit{адап\-тив\-ной} по отношению к классу~$M$. Класс объектов, для 
которого существует $\varepsilon$-адап\-тив\-ная стратегия, называется 
  $\varepsilon$-\textit{адап\-тив\-но управ\-ля\-емым}. (Если $\varepsilon\hm=0$, то 
приставка <<$\varepsilon$->> в этих определениях опускается.)
  
  Основная задача адаптивного управления заключается в построении адаптивных 
стратегий для различных классов объектов. 

К~настоящему вре\-ме\-ни получено много 
решений для многочисленных вариантов этой задачи. Подобные результаты являются 
фактически достаточными условиями адаптивной управ\-ля\-емости. Ниже, однако, будет 
уделено внимание также необходимым условиям существования адаптивных стратегий. 
Подчеркнем, что рассматриваемая постановка задачи предполагает, по сути, наличие 
лишь минимальной априорной информации об объекте управления~--- необходимо знать 
множество управлений~$Y$.

\section{Некоторые условия адаптивной управляемости}

  Пусть $\mu\in M$~--- фиксированный объект, а $\sigma\hm\in \Sigma$~--- 
фиксированная стратегия из некоторой среды. Набор, состоящий из первых $t$ правил 
стратегии~$\sigma$, будем обозначать через $\sigma^t\hm=(\sigma_1, \ldots , \sigma_t)$. 
Таким образом, $\sigma\hm=(\sigma^t, \sigma_{t+1},\sigma_{t+2}, \ldots)$. Положим
  $$
  w_t^*(\mu,\sigma) =w_t^*(\mu,\sigma^t)=\sup\limits_{\sigma_{t+1},\sigma_{t+2}, \ldots} 
w_t(\mu,\sigma)\,,
  $$
где верхняя грань берется по всем допустимым правилам начиная с момента $t\hm+1$. В 
этих обозначениях $w_0^*(\mu,\sigma) \hm=W(\mu)$. Ясно, что $W(\mu)\hm\geq 
w_1^*(\mu,\sigma)\hm\geq w_2^*(\mu,\sigma)\geq \cdots$
  
  Стратегию~$\sigma$ назовем $\varepsilon$-\textit{по\-вреж\-да\-ющей} для 
объекта~$\mu$, если
  $$
  \inf\left\{ t:\ w_t^*(\mu,\sigma)<W(\mu)-\varepsilon\right\} <\infty\,,\enskip \varepsilon>0\,.
  $$
  
  Пример~1 показывает, что существуют объекты, для которых каждая стратегия~--- 
$\varepsilon$-по\-вреж\-да\-ющая (с разными значениями~$\varepsilon$).
  
  \medskip
  
  \noindent
  \textbf{Пример~1.} Множество~$X$ состояний объекта~$\mu$ образовано точками с 
неотрицательными целочисленными координатами на плоскости, $X\hm=\{ (i,j), 
i\hm\geq0,\ j\hm\geq0\}$. Множество управлений $Y\hm=\{1;2\}$. Начальное состояние 
$x_0=(0,\,0)$. Детерминированные переходы между состояниями заданы следующим 
образом ($t\hm>0$, $i\hm\geq0$):
  \begin{align*}
  \mu_t\left( x_t=(i+1{,}0)\vert x_{t-1}=(i,0),y_t=1\right)&=1\,;\\
  \mu_t\left( x_t=(i,j+1)\vert x_{t-1}=(i,j),y_t=1\right)&=1\,,\ j>0\,;\\
  \mu_t\left( x_t=(i,j+1)\vert x_{t-1}=(i,j),y_t=2\right)&=1\,, j\geq 0\,.
  \end{align*}
  
  Одношаговые доходы определены как $g(i,0)\hm=0$, $g(i,j)\hm=1-2^{-i}$ для $i\geq 0$, 
$j\hm>0$.
  
  Стратегия, состоящая из бесконечного повторения управления~1, приносит доход~0. 
Стратегия, в которой управление~2 первый раз применяется (детерминировано) в 
момент~$t$, приносит доход $1\hm-2^{t-1}$, что меньше максимально возможного 
на~$2^{t-1}$. Рандомизация правил и их зависимость от предыстории не вносит 
принципиальных изменений~--- каждая стратегия остается 
  $\varepsilon$-по\-вреж\-да\-ющей относительно предельно наибольшего, но 
недостижимого значения~1.
  
  В примере~2 оптимальная стратегия для любого объекта из класса является 
повреждающей для остальных объектов.
  
  \medskip
  
  \noindent
  \textbf{Пример~2.} Пусть $X\hm= \{0, 1, 2, \ldots\}\cup \{a,b\}$; $Y\hm=\{0;\,1\}$; 
$g(a)\hm=1$; $g(b)\hm=g(i)\hm=0$, $i\hm\geq0$. Зададим счетное множество объектов 
$M\hm=\{\mu^{(k)},\ k\hm=0, 1, \ldots\}$. Пусть для всех~$k$:
  \begin{align*}
  \mu^{(k)}(x_0=0)&=1\,;\\
  \mu^{(k)}(x_{t+1}=i+1\vert x_t=i, y_t=0)&=1\,,\enskip i\geq0\,;\\
     \mu^{(k)}(x_{t+1}=a\vert x_t=k,y_t=1)&=1\,;\\
     \mu^{(k)}(x_{t+1}=b\vert x_t=i,y_t=1) &=1\,,\enskip i\not=k\,;\\
     \mu^{(k)}(x_{t+1}=a\vert x_t=a,y_t=j)&={}\\
&\hspace*{-45mm}{}=\mu^{(k)}(x_{t+1}=b\vert 
x_t=b,y_t=j)=1\,,\enskip j=0\vee 1\,.
     \end{align*}
  
  Таким образом, состояния $a$ и $b$~--- погло\-ща\-ющие, причем в состояние~$a$, 
приносящее максимальный доход, объект~$\mu^{(k)}$ может попасть, только если 
применить управление~1, находясь в со\-сто\-янии~$k$. Первые (существенные) правила 
оптимальной стратегии для объекта~$\mu^{(k)}$ требуют применения управления~0 до 
достижения состояния~$k$, а затем применения в этом состоянии управления~1. Однако 
такая стратегия является повреждающей для всех остальных объектов. Следовательно, для 
класса~$M$ не существует равномерно оптимальной стра\-тегии.
{\looseness=1

}
  
  Пусть $M$~--- класс объектов. Обозначим через $\Sigma_\varepsilon(\mu)$ множество 
$\varepsilon$-по\-вреж\-да\-ющих стратегий для объекта~$\mu$, $\mu\hm\in M$. Положим 
$\Sigma_\varepsilon(M)\bigcap\limits_{\mu\in M}\left( \Sigma\backslash 
\Sigma_\varepsilon(\mu)\right)$.
  
  \medskip
  
  \noindent
  \textbf{Лемма~2.} \textit{Для того чтобы существовала $\varepsilon$-адап\-тив\-ная 
стратегия, необходимо, чтобы $\Sigma_\varepsilon(M)\not=\emptyset$.}
  \medskip
  
  \noindent
  Д\,о\,к\,а\,з\,а\,т\,е\,л\,ь\,с\,т\,в\,о\,.\ Если $\Sigma_\varepsilon\not= \emptyset$, то любая 
допустимая стратегия хотя бы для одного из объектов является 
  $\varepsilon$-по\-вреж\-да\-ющей и, следовательно, не является 
  $\varepsilon$-оп\-ти\-маль\-ной, а потому не может быть равномерно 
  $\varepsilon$-оп\-ти\-маль\-ной по отношению к классу~$M$.
  
  В примере~3, несмотря на наличие по\-вреж\-да\-ющих стратегий, адаптивная стратегия 
существует.
  
  \medskip
  
  \noindent
  \textbf{Пример~3.} Пусть $X\hm=Y\hm=\{1, \ldots , K\}$ и пусть задана 
детерминированная функция~$f:\ X\hm\rightarrow X$, которая представляет собой 
циклическую подстановку на множестве~$X$,  т.\,е.\ $f(i)\not= f(j)$, если $i\not= j$; 
$i,j\hm=1, \ldots , K$. Рассмотрим следующий неоднородный во времени 
детерминированный объект. Положим
  \begin{align*}
  \mu_0(x_0=1)&=1\,;\\
  \mu_t(x_t=f(k)\vert x^{t-1},y^t) &= I_{\{y_t=k\}}\,,\ 0<k\,,\ t\leq K\,;\\
  \mu_t(x_t=f(k)\vert x^{t-1},y^t) &=I_{\{y_{K+1}=k}\,,\\
  & \hspace*{10mm}0<k\leq K\,,\enskip t>K
  \end{align*}
($I_A$~--- индикатор события~$A$).
  
  Одношаговые доходы определим как $g(i)\hm=i$, $i\hm\in X$.
  
  Так определенный объект обозначим через~$\mu^f$. Ясно, что для этого объекта 
траектория управ\-ля\-емо\-го процесса, начиная с момента $K+1$, и, следовательно, доход 
зависят исключительно от управ\-ле\-ния, примененного в момент $K+1$. Доход будет 
максимален (и равен~$K$) тогда и только тогда, когда $y_{K+1}\hm= k^\prime \hm= 
k^*(f)\hm=\argmax\limits_{1\leq k\leq K} f(k)$.
  
  Пусть $M=\{\mu^f\}$~--- совокупность всех объектов данного вида (которая содержит 
$K!$ элементов). Очевидно, для класса~$M$ существует равномерно оптимальная 
стратегия, доставляющая доход, равный~$K$. Например, достаточно вначале в моменты 
$t\hm=1, \ldots , K$ по одному разу применить каждое из управлений, а затем в момент 
$K+1$ применить управление~$k^*$, которое будет выявлено путем наблюдения за 
полученными одношаговыми доходами. Таким образом, на первых тактах необходимо совершить 
<<обучение>>~--- выявить управление, приносящее наибольший одношаговый доход. 
В~то же время существуют и повреждающие стратегии. Например, стратегия, в которой 
первые $K$ правил заключаются в применении управления~1. Правило~$\sigma_{K+1}$ 
такой стратегии может быть построено только в виде зависимости от управления~1 и от 
значения $f(1)$, поэтому при любом его определении найдется объект~$\mu^f$, для 
которого в момент $K+1$ будет с положительной вероятностью предписано применение 
неоптимального управления, и, следовательно, доход будет меньше~$K$.
  
  В примере 3 <<обучение>> оказалось возможным только благодаря знанию структуры 
процессов. Если бы заранее не было известно, что необходимо на первых тактах по разу 
<<испробовать>> все управ\-ле\-ния, то легко можно было пропустить период, когда 
возможно обучение, и совершить тем самым <<непоправимую ошибку>>. Следовательно, 
для того чтобы конструктивно построить равномерно оптимальную стратегию, 
необходима дополнительная информация. Это противоречит избранному принципу 
постановки задачи~--- минимальности априорной информации об объекте. 

Введем более 
жесткое определение адаптивной стратегии, которое, в част\-ности, устраняет указанное 
несоответствие.
  
  Пусть $M$~--- некоторый класс объектов. Эргодическая стратегия~$\sigma$ (ее 
определение дано в конце разд.~2) называется \textit{устойчивой} по отношению к 
классу~$M$, если для любого объекта $\mu\hm\in M$ стратегия~$\tilde{\sigma}$, 
полученная из стратегии~$\sigma$ путем произвольной (допустимой) замены конечного 
числа правил, (1)~имеет одинаковый со стратегией доход 
$w(\mu,\sigma)\hm=w(\mu,\tilde{\sigma})$ и (2)~является эргодической по отношению к 
классу~$M$.

%\columnbreak
  
  Адаптивная стратегия для класса~$M$ называется \textit{строго адаптивной}, если она 
устойчивая по отношению к этому классу.
  
  \medskip
  
  \noindent
  \textbf{Пример~4.} Легко показать, что строго адаптивными являются 
многочисленные адаптивные стратегии для класса управляемых конечных связных 
марковских цепей~[1, 2].
  
  Рассмотрим еще один мотив, выдвигаемый в качестве необходимого условия 
адаптивной управ\-ля\-емости.
  
  \medskip
  
  \noindent
  \textbf{Пример~5.} Пусть класс объектов состоит из функций вещественного 
аргумента~$u$ вида $\mu^y\hm=\mu^y(u)\hm=I_{\{u=y\}}$, $y\hm\in [0,\,1]$. (В~терминах 
управляемых случайных последовательностей: $X\hm= \{0;1]\}$, $Y\hm=[0,1]$; 
$\mu_t(x_t\vert x^{t-1},y^t)\hm=x_t I_{\{y_t=y\}}+ (1-x_t)I_{\{y_t=y\}}$; $g(x)\hm=x$, 
$x\hm\in X$.) Интуитивно представляется очевидным, что невозможно найти максимум 
такой функции за счетное число шагов, если не знать значение, в котором она обращается 
в единицу. В~то же время формально для каждого объекта~$\mu^y$ существует 
оптимальная стратегия. Например, можно постоянно повторять управление~$y$. Однако 
не существует стратегии, равномерно оптимальной по отношению к классу 
$M\hm=\{\mu^y\}$. В~такой стратегии для каждого $y\hm\in [0,\,1]$ необходимо должно 
было бы выполняться следующее условие: $\sigma_t(y_t=y\vert \cdot)>0$ хотя бы для 
одного значения~$t$. Но это невозможно, поскольку для фиксированного значения~$t$ 
данное неравенство может быть выполнено лишь для счетного множества значений~$y$, а 
$t$ также пробегает счетное множество значений. Счетное объединение счетных 
множеств само счетно, поэтому необходимое неравенство не может быть выполнено для 
всех точек на отрезке [0,\,1].
  
  Аналогичные рассуждения показывают, что в данном примере не существует счетного 
множества стратегий, обладающего тем свойством, что для любого объекта найдется 
$\varepsilon$-оп\-ти\-маль\-ная стратегия из этого множества.
  
  Конечное или счетное множество стратегий $\Sigma\hm=\{\sigma(1),\sigma(2), \ldots \}$ 
назовем \textit{базовым} по отношению к классу объектов $M\hm\in \mathcal{M}$, если:
  \begin{enumerate}[(1)]
  \item для любого объекта из $M$ и любого $\varepsilon\hm>0$ существует оптимальная 
стратегия из множества~$\Sigma$;
  \item любая стратегия $\sigma(i)$ является устойчивой по отношению к классу~$M$.
  \end{enumerate}
  
  \smallskip
  
  \noindent
  \textbf{Теорема.} \textit{Строго адаптивная стратегия для класса объектов~$M$ 
существует тогда и только тогда, когда для этого класса существует базовое 
множество стратегий~$\Sigma$.}


%\hfill {\large Приложение~1}

\bigskip

%\pagebreak

\noindent
Д\,о\,к\,а\,з\,а\,т\,е\,л\,ь\,с\,т\,в\,о\ \ теоремы.

Необходимость условий в данном случае является тривиальной, поскольку строго 
адаптивная стратегия, если она существует, образует базовое множество 
стратегий~$\Sigma$, состоящее из одного элемента.
  
  Докажем достаточность. Определим с по\-мощью стратегий из~$\Sigma$ новую 
стратегию $a$ следующим образом. Обозначим
  $$
  \theta_{t,n}=\mathrm{Int}\left(\left( 1-v_{t,n}\right)^{-n}\right)\,,
  $$
где $\mathrm{Int}\left(a\right)$ означает целую часть числа~$a$, и зададим 
последовательность марковских моментов $\tau\hm=\{\tau_n\}$ с помощью рекуррентных 
соотношений

\pagebreak

\noindent
$$
\tau_0=0\,,\enskip \tau_n=\tau_{n-1}+n+\theta_n\,,
$$
где $\theta_n\hm=\theta_{\tau_{n-1},n}$. Соответствующие $\sigma$-ал\-геб\-ры обозначим 
$\mathbf{F}_{(n)}\hm=\mathbf{F}_{\tau_{n-1}}$.
  
  Будем считать, что на пространстве $(\Omega,\mathbf{F})$ задана последовательность 
случайных величин $\beta\hm=\{\beta_n\}$, независимых 
относительно~$\mathbf{F}_{(n)}$. Каждая случайная величина имеет одно и то же 
невырожденное распределение $\{b_i\}$ на множестве номеров стратегий из~$\Sigma$.
  
  Определим правила стратегии $a\hm=a(\Sigma,\beta)$ формулой
  $$
  a_t=\sum\limits_{n=1}^\infty \sigma_t(\beta_n) I_{\{\tau_{n-1}<t\leq \tau_n\}}\,,
  $$
где $\sigma_t(\beta_n)$~--- правило стратегии $\sigma(i)\hm\in\Sigma$ в момент~$t$, если 
$\beta_n\hm=i$.
  
  Наглядно работа стратегии~$a$ выглядит следующим образом. Процесс управления 
разбивается на этапы. Этап с номером $n$ начинается в момент $\tau_{n-1}+1$ и 
оканчивается в момент~$\tau_n;\tau_0\hm=0$. В~момент, предшествующий началу 
очередного этапа, определяется номер стратегии в множестве~$\Sigma$, из которой будут 
взяты правила для применения на данном этапе. Этот номер равен значению случайной 
величины~$\beta_n$. Продолжительность $n$-го этапа равна $n\hm+\theta_n$ и зависит, 
следовательно, от номера этапа и от оценки качества применяемой стратегии, полученной 
в течение первых $n$ тактов этапа. Стратегия~$a$ называется стратегией перебора~[2]. 
Таким образом, последовательность~$\beta$ определяет на каждом этапе выбор стратегии 
из множества~$\Sigma$, правила из которой применяются на этом этапе.
  
  Пусть задан объект $\mu\hm\in M$ и пусть $W\hm=W(\mu)$~--- точная верхняя грань 
доходов для этого объекта, взятая по всем допустимым стратегиям, и пусть %также
  \begin{alignat*}{2}
  W_i&=w(\mu,\sigma(i))\,; &\enskip v_n^{(1)}&=v_{\tau_{n-1},n}\,;\\
  v_n^{(2)}&=v_{\tau_{n-1},n+\theta_n}\,; &\enskip \Delta_n&=\tau_n-\tau_{n-1}=n+\theta_n\,.
  \end{alignat*}
  
  Для произвольного $\varepsilon>0$ определим множества
  $$
  A_n^{(k)}(\varepsilon)=\left\{ v_n^{(k)}\geq W-\varepsilon\right\}\,,
  $$
обозначая их дополнения $\overline{A_n^{(k)}(\varepsilon)}$, $k=1, 2$.
  
  Обозначим
  \begin{align*}
  s_n^{(1)} &= \sum\limits_{l=1}^n I_{A_l^{(1)}(\varepsilon)\cap 
{A_l^{(2)}(2\varepsilon)}} \Delta_l\,;\\
  s_n^{(2)} &= \sum\limits_{l=1}^n I_{A_l^{(1)}\cap 
\overline{A_l^{(2)}(2\varepsilon)}}\Delta_l\,;\\
  s_n^{(3)} &= \sum\limits_{l=1}^n I_{\overline{A_l^{(1)}(\varepsilon)}}\Delta_l\,,
  \end{align*}
так что $\tau_n\hm=\sum\limits_{l=1}^n \Delta_l\hm= s_n^{(1)}\hm+ s_n^{(2)}\hm+ 
s_n^{(3)}$.

\columnbreak

  
  С~помощью введенных обозначений запишем оценку для усредненного дохода к 
моменту~$\tau_n$:
  \begin{multline}
  w_n=\fr{1}{\tau_n}\sum\limits_{t=1}^{\tau_n} g_t=\fr{\sum\limits_{l=1}^n 
v_l^{(2)}\Delta_l} {\sum\limits_{l=1}^n \Delta_l}\geq{}\\
{}\geq (W-2\varepsilon) \fr{s_n^{(1)}} 
{s_n^{(1)}+s_n^{(2)}+s_n^{(3)}}\,.
  \label{e1-kon}
  \end{multline}
  
  Для оценки суммы $s_n^{(1)}$ запишем неравенство
  $$
  s_n^{(1)}\geq \Delta_{v_n}\,,
  $$
в котором обозначено
$$
v_n=\max\left\{ l:\ l\leq n,\ A_l^{(1)}(\varepsilon)\cap A_l^{(2)}(2\varepsilon)\right\}\,.
$$
  
  Оценим вероятность события $B_n\hm=\{v_n\hm\leq n-\ln n\}$, для которого выполняется 
включение
  $$
  B_n\subset \bigcap\limits_{n-\ln n<l\leq n} 
  \overline{A_l^{(1)}(\varepsilon)}\cap \overline{A_l^{(2)}(2\varepsilon)}\,.
  $$
  
  Согласно определениям эргодической стратегии, базового множества стратегий и 
семейства случайных величин~$\beta$ имеем:
  \begin{multline*}
  \mathbf{P}_{a} \left( \overline{A_l^{(1)}(\varepsilon)}\cup\overline{A_l^{(2)} 
(2\varepsilon)}\,\Big\vert \mathbf{F}_{(l)}\right)\leq{}\\
  {}\leq
  \sum\limits_{\substack{{i\in \mathcal{I};}\\ {W_i\leq W-\varepsilon/2}}}\!\!\!\!
   \mathbf{P}_{a}\left(\beta_l=i\vert 
\mathbf{F}_{(l)}\right)+{}\\
{}+  %\substack{{i=\overline{1,n}}\\ {j=\overline{1,l}}}
\sum\limits_{\substack{{i\in \mathcal{I};}\\ {W_i\leq W-\varepsilon/2}}}\!\!\!\!
\mathbf{P}_{a}\left( \overline{A_l^{(1)}(\varepsilon)}, \ \beta_l=i
\vert \mathbf{F}_{(l)}\right)\leq{}\\
  {}\leq \sum\limits_{\substack{{i\in \mathcal{I};}\\ {W_i\leq W-\varepsilon/2}}}\!\!\!\!
  \mathrm{P}_{a}(\beta_l=i)+{}\\
{}+\sum\limits_{\substack{{i\in \mathcal{I};}\\ {W_i> W-
\varepsilon/2}}}
\!\!\!\!\mathbf{P}_{a}\left( v_l^{(1)}\leq W_i-\fr{\varepsilon}{2}, \beta_l=i\vert 
\mathbf{F}_{(l)}\right) \leq{}\\
  {}\leq \sum\limits_{\substack{{i\in \mathcal{I};}\\ {W_i\leq W-\varepsilon/2}}}\!\!\!\!
   b_i+a_l\left( 
\fr{\varepsilon}{2}\right) \leq q<1
  \end{multline*}
при всех достаточно больших~$l$. Отсюда следует, что для всех достаточно больших 
значений~$n$ выполняется неравенство
$$
\mathbf{P}_a(B_n)\leq q^{n-\ln n}\,.
$$
  
  Следовательно, согласно лемме Бо\-ре\-ля--Кан\-тел\-ли
  \begin{equation}
  \mathbf{P}_{a}\left( \overline{\lim\limits_{n\rightarrow\infty}} B_n\right)=0\,.
  \label{e2-kon}
  \end{equation}
  
  Это означает, что
  $$
  s_n^{(1)}\geq \Delta_{v_n}\geq (1-W-\varepsilon)^{-n+\ln n}\,.
  $$
  
  Оценим сумму $s_n^{(2)}$. Обозначив 
$C_n\hm=A_n^{(1)}(\varepsilon)\cap$\linebreak 
$\cap\overline{A_n^{(2)}(2\varepsilon)}$ и $W_{(n)}\hm=\sum\limits_{i\in 
I} W_i I_{\{\beta_n=i\}}$, получим:
  \begin{multline*}
  \mathrm{P}_{a}\left(C_n\vert \mathrm{ F}_{(n)}\right)=
  \mathrm{P}_{a|} \left( C_n, W_{(n)}<W-\fr{3\varepsilon}{2}\vert \mathrm{
  F}_{(n)}\right) +{}\\
  {}+ \mathrm{P}_{a}\left( 
  C_n, W_{(n)}\geq W-\fr{3\varepsilon}{2}\vert \mathrm{
  F}_{(n)}\right)\leq{}\\
  {}\leq \mathrm{P}_{a}\left( v_n^{(1)}>W-\varepsilon,\, W_{(n)}<W-\fr{3\varepsilon}{2}\vert \mathrm{
  F}_{(n)}\right)+{}\\
  {}+
  \mathrm{P}_{a} \left( v_n^{(2)}\leq W-2\varepsilon,\, W_{(n)}\geq W-
\fr{3\varepsilon}{2}\vert \mathrm{
  F}_{(n)}\right)\leq{}\\
  {}\leq \sum\limits_{i\in \mathcal{I}; W_i\leq W- \varepsilon/2} \mathrm{P}_{a}\left(
  v_{\tau_n,n}>W_i+\fr{\varepsilon}{2},\, \beta_l=i\vert\mathrm{F}_{(n)}\right)+{}\\
  {}+\sum\limits_{\substack{{i\in \mathcal{I};}\\ {W_i> W- 3\varepsilon/2}}}\!\!\!\!
   \mathbf{P}_{a} \left( 
v_{\tau_n,n+\theta_n}\leq W_i-\fr{\varepsilon}{2},\,\beta_l=i\vert\mathbf{F}_{(n)}\right)\leq {}\\
{}\leq
a_n\left( \fr{\varepsilon}{2}\right)\,.
  \end{multline*}
  
  Из определения базового множества стратегий следует, что
  $$
  \sum\limits_{n=1}^\infty \mathbf{P}_{a} (C_n)<\infty\,,
  $$
поэтому согласно лемме Бо\-ре\-ля--Кан\-тел\-ли полу\-чаем:
\begin{equation}
\mathbf{P}_{a}\left( \overline{\lim\limits_{n\rightarrow\infty}} C_n\right) =0\,.
\label{e3-kon}
\end{equation}
  
  Отсюда следует, что
  $$
  \sup\limits_n s_n^{(2)}\leq c<\infty\,.
  $$
  
  Для суммы $s_n^{(3)}$ имеем следующую оценку:
  $$
  s_n^{(3)}\geq \sum\limits_{l=1}^n \left(n+(1-W+\varepsilon)^{-l}\right)< n^2+n(1-
W+\varepsilon)^{-n}.
  $$
  
  Подставляя оценки, полученные для сумм $s_n^{(k)}$, в неравенство~(\ref{e1-kon}), 
получаем:
  \begin{multline*}
  w_n\geq (W-\varepsilon) \left( 1+\fr{s_n^{(2)}+s_n^{(3)}}{s_n^{(1)}}\right)^{-1}\geq 
{}\\
  {}\geq (W-\varepsilon)\left( 1+\fr{c+n^2+n(1-W+\varepsilon)^{-n}}{(1-W-\varepsilon/2)^{-
n+\ln n}}\right)^{-1}\geq{}\\
{}\geq W-3\varepsilon
  \end{multline*}
для всех достаточно больших значений~$n$. Отсюда
\begin{equation}
\lim\limits_{\overline{n\rightarrow\infty}} w_n\geq W\,.
\label{e4-kon}
\end{equation}
  
  Рассмотрим далее множество
  $$
  \Omega^\prime =\left\{ \lim\limits_{n\rightarrow\infty} w_n =W\right\}\cap 
\overline{B}\cap\overline{C}\,,
  $$
где $\overline{B}$ и $\overline{C}$ означают соответственно дополнения к множествам 
$B\hm= \overline{\lim\limits_{n\rightarrow\infty}} B_n$ и $C\hm= 
\overline{\lim\limits_{n\rightarrow\infty}} C_n$.
  
  Согласно формулам~(\ref{e2-kon})--(\ref{e4-kon})
  $$
  \mathbf{P}_{a}\left(\Omega^\prime\right) =1\,.
  $$
  
  Определим следующие события:
  
  \noindent
  \begin{align*}
  D_{n,t}^{(1)} &= \left\{ \tau_{n-1}<t\leq \tau_{n-1}+n\right\} \cap \Omega^\prime\,;\\
  D_{n,t}^{(2)} &= \left\{\tau_{n-1}+n<t\leq \tau_n\right\}\cap \Omega^\prime\,;\\
  D_{n,t}^{(3)} &= \left\{ \tau_{n-1}<t\leq \tau_n\right\} \cap \Omega^\prime\,.
  \end{align*}
  
  На множестве $D_{n,t}^{(1)}$ усредненный доход $v_t\hm=v_{0,t}\hm=
  t^{-1}\sum\limits_{s=1}^t g_s$ оценивается с помощью формулы~(\ref{e1-kon}) как
  
    \noindent
  $$
  v_t\geq \fr{\tau_{n-1} w_n}{\tau_{n-1}+n+\theta_n}\geq W-\varepsilon_n^{(1)}\,,
  $$
где $\varepsilon_n^{(1)}\hm\rightarrow0$ при $n\hm\rightarrow\infty$.
  
  Пусть событие $D_{n,t}^{(2)}$ имеет место. Тогда $\theta_n\geq (1\hm- 
W\hm+\varepsilon)^{-n}$. Кроме того, из определения событий $B_n$, $B$, 
$D_{n,t}^{(2)}$ следует, что для всех достаточно больших значений~$n$ выполняется 
неравенство $v_n\hm> n-\ln n$. Следовательно, на множестве~$D_n^{(2)}$ справедлива 
оценка

  \noindent
  $$
  v_t\geq \fr{\tau_{n-1} w_n}{\tau_{n-1}+n+\theta_n}\geq W-\varepsilon_n^{(2)}\,,
  $$
где $\varepsilon_n^{(2)}\hm\rightarrow0$ при $n\hm\rightarrow\infty$.
  
  Из определения событий $C_n$, $C$, $D_{n,t}^{(3)}$ вытекает, что
  
    \noindent
  $$
  D_{n,t}^{(3)} \subset \left\{ \min\limits_{n<m\leq n+\theta_n} v_{n,m}\geq W-
2\varepsilon\right\}\,,
  $$
поэтому на множестве $D_n^{(3)}$ справедливы неравенства:

  \noindent
\begin{multline*}
\!\!v_t\geq \fr{\tau_{n-1} w_n}{t}+\left(1- \fr{\tau_{n-1}}{t}\right) \left( 1-\tau_n\right)^{-1} 
\!\!\sum\limits_{s=\tau_{n-1}+1}^t \!\!\!\!g_s\geq{}\\
{}\geq \fr{\tau_{n-1} w_n}{t}+\left( 1-\fr{\tau_{n-1}}{t}\right)\left( W-2\varepsilon\right) \geq 
W-2\varepsilon -\varepsilon_n^{(3)},
\end{multline*}
где $\varepsilon_n^{(3)}\rightarrow0$ при $n\hm\rightarrow\infty$.

\pagebreak
  
  Таким образом, на множестве
  $$
  D_{n,t}=\bigcup\limits_{k=1}^3 D_{n,t}^{(k)} = \left\{ \tau_{n-1}<t\leq \tau_n\right\} \cap 
\Omega^\prime
  $$
имеет место оценка $v_n\hm\geq W-\varepsilon-\varepsilon_n$, где 
$\varepsilon_n\hm\rightarrow 0$ при $n\hm\rightarrow\infty$. Достаточность утверждения 
теоремы следует из соотношений $\Omega\hm= \bigcup\limits_{n=1}^\infty \left\{ \tau_{n-
1}\hm<t\hm\leq \tau_n\right\}$ и $\lim\limits_{t\rightarrow\infty} I_{D_{n,t}}\hm=0$.

\section{Заключение}

  Адаптивные стратегии, позволяющие достигать цели в условиях информационной 
неопреде\-лен\-ности, основываясь на <<обучении>> в процессе взаимодействия с объектом, 
находят все более широкое практическое применение. 

В~этой работе было уделено 
внимание теоретическим аспектам адаптивного подхода. Сформулированы определения 
адаптивных стратегий и приведена формальная постановка задачи адаптивного 
управления. Сформулированы и доказаны некоторые утверждения о необходимых 
условиях и достаточных условиях адап\-тив\-ной управляемости. 

Продолжение исследований 
в данном на\-прав\-ле\-нии позволит найти ответы на принципиальные вопросы, в каких 
ситуациях можно рассчитывать на <<приспособление к неизвестной среде>> и сколь 
универсальными могут быть <<обучающиеся>> алгоритмы.



{\small\frenchspacing
{%\baselineskip=10.8pt
\addcontentsline{toc}{section}{Литература}
\begin{thebibliography}{9}


  \bibitem{1-kon}
  \Au{Sragovich~V.\,G.}
  Mathematical theory of adaptive control.~--- Singapore: World Scientific, 2006.
  \bibitem{2-kon}
  \Au{Коновалов~М.\,Г.}
  Методы адаптивной обработки информации и их приложения.~--- М.: ИПИ РАН, 2007.
  
  \label{end\stat}
  
  \bibitem{3-kon}
  \Au{Неве~Ж.}
  Математические основы теории вероятностей.~--- М.: Мир, 1969.
\end{thebibliography}
}
}


\end{multicols} %6
\def\stat{kudr}

\def\tit{ПРИБЛИЖЕННЫЕ МЕТОДЫ РЕШЕНИЯ ЗАДАЧИ ДИАГНОСТИКИ ПЛОСКИМ 
ЗОНДОМ СИЛЬНОИОНИЗОВАННОЙ ПЛАЗМЫ С~УЧЕТОМ КУЛОНОВСКИХ 
СТОЛКНОВЕНИЙ}

\def\titkol{Приближенные методы решения задачи диагностики плоским 
зондом сильноионизованной плазмы} %с~учетом Кулоновских  столкновений}

\def\autkol{И.\,А.~Кудрявцева, А.\,В.~Пантелеев}
\def\aut{И.\,А.~Кудрявцева$^1$, А.\,В.~Пантелеев$^2$}

\titel{\tit}{\aut}{\autkol}{\titkol}

%{\renewcommand{\thefootnote}{\fnsymbol{footnote}}\footnotetext[1]
%{Работа поддержана Российским фондом фундаментальных исследований
%(проекты 11-01-00515а и 11-07-00112а), а также Министерством
%образования и науки РФ в рамках ФЦП <<Научные и
%научно-педагогические кадры инновационной России на 2009--2013~годы>>.}}


\renewcommand{\thefootnote}{\arabic{footnote}}
\footnotetext[1]{Московский авиационный институт, irina.home.mail@mail.ru}
\footnotetext[2]{Московский авиационный институт, avpanteleev@inbox.ru}

\vspace*{-2pt}

\Abst{Сформирована математическая модель, описывающая динамику сильноионизованной 
плазмы с учетом столкновений заряженных частиц вблизи плоского зонда. Модель включает уравнение 
Фоккера--Планка и уравнение Пуассона. Предложено два подхода к решению задачи: на основе метода 
статистических испытаний Мон\-те-Кар\-ло и на основе композиции метода крупных частиц и метода 
расщепления.} 

\vspace*{-2pt}

\KW{телекоммуникационные системы; метод Монте-Карло; метод крупных частиц; метод 
расщепления; зонд; уравнение Фоккера--Планка; уравнение Пуассона} 

\vspace*{-4pt}

 \vskip 8pt plus 9pt minus 6pt

      \thispagestyle{headings}

      \begin{multicols}{2}
      
            \label{st\stat}

\section{Введение}

В настоящее время в области телекоммуникаций все более востребованными становятся 
информационные технологии, основанные на использовании математических моделей и численных 
методов физики плазмы. Поэтому особенно актуальным является решение разнообразных задач анализа 
поведения плазмы, включающих в себя формирование новых моделей и методов их исследования. 
Помимо этого, в разработке телекоммуникационного оборудования эффективно используются 
собственно физические свойства плазмы. В~частности, изготовлена антенна, работа которой основана 
на газовом разряде низкотемпературной плазмы~[1], интенсивно ведутся разработки по созданию и 
усовершенствованию источников бесперебойного питания на основе плазменных элементов~[2, 3]. 
      
      Одним из наиболее перспективных направлений для построения систем оптической 
беспроводной связи является использование лазеров~\cite{4-k, 5-k}. В~этой связи большое внимание 
уделяется использованию плазмы при разработке импульсных сильноточных коммутаторов~\cite{6-k}, 
так как практическое применение подобных разработок требует повышения уровня надежности и 
быстродействия лазерных систем.
      
      Исследования низкотемпературной плазмы также связаны с разработками в области дальней 
космической связи, так как моделирование процессов взаимодействия заряженного тела с верхними 
слоями атмосферы позволяет предлагать способы улучшения существующих систем радиосвязи с 
космическими летательными аппаратами~\cite{7-k}. 
      
      Наряду с этим актуальными также являются задачи диагностики плазмы, поскольку перспективы 
ее использования в области телекоммуникаций после более полного изучения физических свойств 
могут значительно расшириться. 

Для диагностики плазмы применяют зондовые методы исследования~[8--11]. Эти методы относятся к 
классу контактных методов; как следствие, возникает сложность в исследовании пристеночной области 
вблизи зонда, которая характеризуется достаточно сложным распределением потенциала и функциями 
распределения, отличными от максвелловских. 

Данная работа посвящена исследованию переходного режима обтекания заряженного тела плазмой. Для 
переходного режима выполняется следующее условие: длина свободного пробега иона до столкновения 
с нейтральным атомом или другим ионом невелика по сравнению с характерными размерами тела. 
В~этом случае возникает необходимость учета столкновений заряженных частиц с нейтральными 
атомами и кулоновских столкновений. В~работах~\cite{10-k, 11-k} подробно рассмотрена модель с 
учетом столкновений заряженных частиц с нейтральными атомами. В~настоящей статье представлена 
теоретическая модель, описывающая влияния ион-ионных и ион-элек\-т\-рон\-ных столкновений на 
измеряемые характеристики плазмы, что ранее детально не исследовалось.
      
      В~рамках данной работы предлагается модель, описывающая динамику сильноионизованной 
плазмы с учетом кулоновских столкновений. Эта модель учитывает такие процессы взаимодействия, 
как перенос частиц и столкновения между заряженными частицами типа <<ион--ион>> и 
      <<ион--электрон>> под влиянием макроскопического электрического поля. Перечисленные 
процессы описываются самосогласованной системой уравнений, включающей уравнение 
      Фок\-ке\-ра--План\-ка и уравнение Пуассона~[12].
      
      Вычислительная модель задачи строится на основе двух методов: метода статистических 
испытаний Мон\-те-Кар\-ло и композиции метода крупных частиц и метода расщепления. Приведены 
результаты численного моделирования, полученные с использованием вышеперечисленных методов.

\vspace*{-4pt}

\section{Постановка задачи}

\vspace*{-2pt}

Рассматривается следующая физическая постановка зондовой задачи~[11]. В~невозмущенную 
бесконечно протяженную плазму, состоящую из электронов и однозарядных ионов, внесена большая\linebreak 
заряженная до потенциала $\varphi_p$ плоскость. Плоскость, расположенная поперек потока плазмы, 
является идеально поглощающей для электронов. Ионы при ударе о плоскость нейтрализуются. 
Предполагается, что частицы в плазме движутся под действием внешнего электрического поля, 
магнитное поле отсутствует. Концентрации ионов $n_{i\infty}$ и электронов $n_{e\infty}$, а также 
температуры данных час\-тиц~$T_{i\infty}$ 
и~$T_{e\infty}$ в невозмущенной плазме заданы. За начальные 
функции распределения обоих типов час\-тиц принимаются функции распределения Максвелла. 
      
      Требуется с учетом столкновений между заряженными частицами найти напряженность 
самосогласованного электрического поля $\vec{E}(\vec{r},t)$, функции распределения однозарядных 
ионов $f_i(\vec{r}, \vec{v}, t)$ и электронов $f_e(\vec{r}, \vec{v}, t)$, 
а также их моменты (плотности 
токов ионов и электронов  $j_i(\vec{r},t)\hm
=q\int f_i(\vec{r}, \vec{v}, t)\vec{v}\,d\vec{v}$, $j_e(\vec{r},t) 
\hm={\sf e}\int f_e(\vec{r},\vec{v},t)\vec{v}\,d\vec{v}$, где $q=Z_i{\sf e}$, $Z_i=1$~--- заряд иона, ${\sf 
e}$~--- заряд электрона; концентрации ионов и электронов $n_i(\vec{r},t)\hm=\int 
f_i(\vec{r},\vec{v},t)\,d\vec{v}$, $n_e(\vec{r},t)\hm=\int f_e(\vec{r},\vec{v}, t)\,d\vec{v}$). 
Поведение частиц во 
времени~$t$ характеризуется ра\-ди\-ус-век\-то\-ром~$\vec{r}$ и вектором скорости~$\vec{v}$.
      
      Математическая модель, соответствующая данной физической постановке задачи, имеет 
вид~\cite{11-k, 13-k}:

\noindent
      \begin{equation}
      \left.
      \begin{array}{c}
      \fr{\partial f_\alpha (\vec{r},\vec{v},t)}{\partial t}+
      \vec{v}\fr{\partial f_\alpha (\vec{r},\vec{v},t)}{ 
\partial \vec{r}}+
\fr{\vec{F}_\alpha(\vec{r},t)}{m_\alpha}\times{}\\[4pt]
{}\times\fr{\partial f_\alpha(\vec{r},\vec{v},t)}{ \partial 
\vec{v}}=
\left(\fr{\partial f_\alpha(\vec{r},\vec{v},t)}{ \partial t}\right)_{\mathrm{с}}+S_\alpha 
(\vec{r},\vec{v},t)\,;\\[6pt]
      \Delta\varphi(\vec{r},t)=-\fr{{\sf e}}{\varepsilon_0}\left( n_i(\vec{r},t)-n_e(\vec{r},t)\right)\,;\\[6pt]
      \vec{E}(\vec{r},t)=-\nabla \varphi(\vec{r},t)\,.
      \end{array}\!\!
      \right\}\!\!
      \label{e1-k}
      \end{equation}
Здесь первое уравнение~--- уравнение Фок\-ке\-ра--План\-ка для частиц сорта~$\alpha$ ($\alpha=i,e$), 
второе~--- уравнение Пуассона для самосогласованного электрического поля; 
$f_\alpha(\vec{r},\vec{v},t)$~--- функция\linebreak
распределения час\-тиц сорта~$\alpha$; $(\partial 
f_\alpha(\vec{r},\vec{v},t)/\partial t)_{\mathrm{с}}$~--- 
оператор столкновений Фок\-ке\-ра--План\-ка; 
функция~$S_\alpha(\vec{r},\vec{v},t)$ описывает источники или стоки\linebreak
 час\-тиц; 
$\vec{F}_\alpha(\vec{r},t)=q_\alpha\vec{E}(\vec{r},t)$, где $\vec{E}(\vec{r},t)$~--- напряженность 
самосогласованного электрического поля, 
$$
q_\alpha =
\begin{cases}
-{\sf e}\,, & \alpha=e\,,\\
{\sf e}\,, & \alpha=i\,;
\end{cases}
$$
$\varphi(\vec{r},t)$~--- потенциал самосогласованного электрического поля; $n_\alpha(\vec{r},t)$ ($\alpha 
\hm=i,e$)~--- концентрация частиц сорта~$\alpha$; $m_\alpha$~--- масса частицы сорта~$\alpha$; 
$\varepsilon_0$~--- электрическая постоянная. 

Оператор столкновений Фок\-ке\-ра--План\-ка имеет вид~\cite{13-k, 14-k}
\begin{multline*}
\fr{1}{\Gamma_\alpha}\left( \fr{\partial f_\alpha}{\partial t}\right)_{\mathrm{с}} 
=\fr{1}{2}\,\nabla_v\nabla_v:\left(f_\alpha\nabla_v\nabla_vg_\alpha(\vec{r},\vec{v},t)\right)-{}\\
{}-
\nabla_v\cdot\left(f_\alpha\nabla_v h_\alpha\right)\,,
\end{multline*}
где $\nabla_v\nabla_v g_\alpha(\vec{r},\vec{v},t)$~--- ковариантная тензорная производная второго ранга, 
знак двоеточия ($:$) обозначает операцию двойного суммирования:
\begin{gather*}
\Gamma_\alpha=\fr{Z_\alpha^4 {\sf e}^4}{4\pi \varepsilon_0^2 m^2_\alpha}\,\ln D_\alpha\,;
\\
D_\alpha =\fr{12\pi\varepsilon_0 kT_{\alpha\infty}}{Z_\alpha^2 {\sf e}^2}\left( \fr{\varepsilon_0 k 
T_{e\infty}}{n_{e\infty} {\sf e}^2}\right)^{1/2}\,;\\
g_\alpha (\vec{r},\vec{v},t)=\sum\limits_{b=i,e}\left( \fr{Z_b}{Z_\alpha}\right) \int f_b 
(\vec{r},{\vec{v}}^{\,\prime},t)\left\vert \vec{v}-{\vec{v}}^{\,\prime}\right\vert\,d\vec{v}^{\,\prime}\,;\\
h_\alpha (\vec{r},\vec{v},t)=\sum\limits_{b=i,e} \fr{m_\alpha+m_b}{m_b} 
\left(\fr{Z_b}{Z_\alpha}\right)
\int
\fr{f_b(\vec{r},{\vec{v}}^{\,\prime}, t)}{\vert \vec{v}-{\vec{v}}^{\,\prime}\vert}
\,d{\vec{v}}^{\,\prime}\,;\\
Z_\alpha =1\,, \quad \alpha=i,e\,.
\end{gather*}
 
К системе уравнений~(\ref{e1-k}) необходимо добавить начальные и краевые условия:
\begin{equation}
\!\left.
\begin{array}{rrl}
t=0:\ & f_\alpha(\vec{r},\vec{v},0)&=f_\alpha^{\mathrm{maksv}}\,,\enskip \alpha=i,e;\\[9pt]
\vec{r}\in \Omega_p:\ & f_\alpha(\vec{r},\vec{v},t)\big\vert_{\vec{r}\in\Omega_p}&=0\,,\enskip \alpha=i,e\,;\\[9pt]
&\varphi(\vec{r},t)\big\vert_{\vec{r}\in\Omega_p}&=\varphi_p\,;\\[9pt]
\vec{r}\in\Omega_\infty:\ & 
f_\alpha(\vec{r},\vec{v},t)\big\vert_{\vec{r}\in\Omega_\infty}&= %{}\\[9pt]
f_\alpha^{\mathrm{maksv}}\,,\enskip \alpha=i,e\,;\\[9pt]
&\varphi(\vec{r},t)\big\vert_{\vec{r}\in\Omega_\infty}&=0\,,
\end{array}\!\!
\right\}\!\!\!\!
\label{e2-k}
\end{equation}
    где 
    
    \noindent
    \begin{multline*}
    f_\alpha^{\mathrm{maksv}}=n_{\alpha\infty}\left(\fr{m_\alpha}{2k\pi T_{\alpha\infty}}\right)^{3/2}\times{}\\
    {}\times
    \exp\left( -
\fr{m_\alpha}{2kT_{\alpha\infty}}\left\vert\vec{v}-\vec{v}_\infty\right\vert^2\right)\,,
\enskip \alpha=i, e\,;
\end{multline*} 
$\Omega_p$ и $\Omega_\infty$~--- множество радиус-векторов час\-тиц, концы которых принадлежат плоскости зонда и 
границе возмущенной зоны соответственно.

Для решения поставленной задачи введем декартову систему координат таким образом, чтобы 
заряженная плоскость совпала с плоскостью~$0xz$. Тогда положение частицы в пространстве будет 
определяться координатами $x,y,z$, а скорость~--- координатами $v_x, v_y, v_z$. В~силу того что 
плоскость является бесконечно большой в сравнении с характерным размером задачи, функции 
распределения частиц будут зависеть только от переменных $y, v_y, t$.

Поставленную задачу предлагается решать независимо двумя методами. Первый метод основывается на 
методе статистических испытаний Мон\-те-Кар\-ло, второй метод является композицией метода 
расщепления и метода крупных частиц.

\section{Применение метода Монте-Карло}

Запишем самосогласованную систему уравнений~(\ref{e1-k}) и~(\ref{e2-k}) в декартовой системе 
координат с учетом сделанных предположений:
\begin{equation}
\left.
\begin{array}{l}
\fr{\partial f_\alpha}{\partial t}+
v_y\fr{\partial f_\alpha}{\partial y}+\fr{F_y^\alpha}{m_\alpha}\,\fr{\partial 
f_\alpha}{\partial v_y}=\fr{1}{2}\,\fr{\partial^2 }{\partial [v_y]^2}\times{}\\
{}\times \left( 
f_\alpha\fr{\partial^2 g_\alpha  }{\partial [v_y]^2}\right) -
\fr{\partial}{\partial v_y}\left( f_\alpha\fr{\partial h_\alpha}{\partial v_y}\right)\,,
\enskip \alpha=i,e\,;\\[6pt]
    \fr{\partial^2\varphi}{\partial y^2} =-\fr{{\sf e}}{\varepsilon_0}\left(n_i-n_e\right)\,;
    \enskip E_y=-
\fr{\partial\varphi}{\partial y}\,;\\[6pt]
\hspace*{3.1mm}    t=0:\  \hspace*{2.6mm}f_\alpha(y,v_y,0)=f_\alpha^{\mathrm{maksv}}\,,\ \alpha=i,e\,;\\[9pt]
\hspace*{2.9mm} y=0:\ \hspace*{2.8mm}f_\alpha(0,v_y,t)=0\,,\ \alpha=i,e\,;\\[9pt]
\hspace*{24.3mm}\varphi(0,t)=\varphi_p\,;\\[9pt]
y=y_\infty:\ f_\alpha(y_\infty, v_y, t)=f_\alpha^{\mathrm{maksv}}\,,\ \alpha=i,e\,;\\[9pt]
\hspace*{21.5mm}\varphi(y_\infty, t)=0\,.
\end{array}
\right \}
\label{e3-k}
\end{equation}

В полученной системе уравнений~(\ref{e3-k}) перейдем к безразмерным величинам, применив 
соотношение $X=M_X \hat{X}$, где $M_X$~--- масштаб размерной величины~$X$, $\hat{X}$~--- 
безразмерная величина~$X$. В~качестве используемых масштабов были взяты следующие: радиус 
Дебая, скорость теплового движения частиц, концентрация частиц в невозмущенной плазме, потенциал, 
возникающий при разделении зарядов в дебаевской сфере, и производные от них величины.

Система безразмерных уравнений имеет следующий вид:
%\noindent
\begin{equation}
\left.
\begin{array}{l}
\fr{\partial 
\hat{f}_\alpha}{\partial\hat{t}}+A_\alpha\fr{\partial\hat{f}_\alpha}{\partial\hat{y}}+
B_\alpha\hat{E}_y\fr{\partial\hat{f}_\alpha}{\partial \hat{v}_y}={}\\
\!{}=
\fr{\partial^2}{\partial[\hat{v}_y]^2}\left(D_\alpha 
\hat{f}_\alpha\right)-\fr{\partial}{\partial\hat{v}_y}\left(K_\alpha \hat{f}_\alpha\right),\enskip 
\alpha=i,e;\\[9pt]
\fr{\partial^2\hat{\varphi}}{\partial\hat{y}^2}=-\left(\hat{n}_i-\hat{n}_e\right)\,;\enskip \hat{e}_y=-
\fr{\partial\hat\varphi}{\partial\hat{y}}\,;\\[9pt]
\hspace*{3.1mm}\hat{t}=0:\ \hspace*{2.6mm}\hat{f}_\alpha(\hat{y},\hat{v}_y,0)=\hat{f}_\alpha^{\mathrm{maksv}}\,,\enskip \alpha-i,e\,;\\[9pt]
\hspace*{2.9mm}\hat{y}=0:\ \hspace*{2.8mm}\hat{f}_\alpha(0,\hat{v}_y,\hat{t})=0\,,\enskip \alpha=i,e\,;\\[9pt]
\hspace*{24.3mm}\hat\varphi(0,\hat{t})=\hat{\varphi}_p\,;\\[9pt]
\hat{y}=\hat{y}_\infty:\ \hat{f}_\alpha(\hat{y}_\infty, \hat{v}_y, \hat{t})=\hat{f}^{\mathrm{maksv}}_\alpha\,,\enskip 
\alpha=i,e\,;\\[9pt]
\hspace*{21.5mm}\hat\varphi(\hat{y}_\infty,\hat{t})=0\,.
\end{array}
\right\}
\label{e4-k}
\end{equation}
Здесь 

\vspace*{-2pt}

\noindent
\begin{gather*}
A_\alpha=\sqrt{\delta_\alpha }\,\hat{v}_y\,;\enskip 
B_\alpha=\sqrt{\delta_\alpha}\,\fr{z_\alpha}{2\varepsilon_\alpha}\,;\\
\delta_\alpha=\fr{\varepsilon_\alpha}{\mu_\alpha}\,;\enskip 
\varepsilon_\alpha=\fr{T_{\alpha\infty}}{T_{i\infty}}\,;\\
\mu_\alpha=\fr{m_\alpha}{m_i}\,;\enskip 
D_\alpha=A_g^\alpha\fr{\partial^2\hat{g}_\alpha}{\partial  [\hat{v}_y]^2}\,;\\
K_\alpha=A_h^\alpha \fr{\partial \hat{h}_\alpha}{\partial \hat{v}_y}\,,\enskip \alpha=i,e\,,
\end{gather*}
где $A_g^\alpha$ и $A_h^\alpha$~--- коэффициенты, определяемые характерными параметрами 
задачи~\cite{15-k}.

Поиск решения самосогласованной системы уравнений~(\ref{e4-k}) осуществляется по следующей 
схе-\linebreak ме. Вначале находятся значения напряженности\linebreak
 электрического поля по значениям потенциала, 
полученным из граничной задачи для уравнения Пуассона. Далее, используя найденные значения 
напряженности, решается уравнение Фок\-ке\-ра--План\-ка путем перехода к стохастическому 
дифференциальному уравнению (СДУ) Ито:

\noindent
\begin{multline*}
d\Theta_\alpha(\hat{t}) = a_\alpha \left(\hat{t},\Theta_\alpha(\hat{t})\right)+{}\\
{}+\sigma\left(
\hat{t},\Theta_\alpha(\hat{t})\right)\,dW(\hat{t})\,,\quad \alpha=i,e\,,
%\label{e5-k}
\end{multline*}
где 

\noindent
\begin{align*}
\Theta_\alpha(\hat{t})&=\begin{bmatrix}
\hat{y}(\hat{t})\\ \hat{v}_y(\hat{t})
\end{bmatrix}\,;\\
a_\alpha\left(\hat{t},\Theta_\alpha(\hat{t})\right)&=\begin{bmatrix}
-A_\alpha\\ -K_\alpha -B_\alpha \hat{E}_y
\end{bmatrix}\,;\\
\sigma_\alpha\left(\hat{t},\Theta_\alpha(\hat{t})\right)\sigma_\alpha^{\mathrm{T}}\left( 
\hat{t},\Theta_\alpha(\hat{t})\right)&=D_\alpha\,,\enskip \alpha=i,e\,;
\end{align*} 
$W(\hat{t})$~--- стандартный винеровский случайный процесс.
\pagebreak

Для нахождения значений вектора состояния~$\Theta_\alpha(\hat{t})$ применим явную разностную 
схему стохастического метода Эйлера~\cite{16-k}:
\begin{multline*}
\Theta_\alpha^{n+1}=\Theta_\alpha^n +h_\tau a_\alpha \left( \hat{t}_n, \Theta_\alpha^n\right)+\sigma_\alpha 
\left( \hat{t}_n, \Theta_\alpha^n\right)\Delta W_n\,,\\ 
n=0,\ldots , N\,,\ \alpha=i,e\,,
%\label{e6-k}
\end{multline*}
где $\Theta_\alpha^n$, $n=0,\ldots , N$,~--- приближенное значение вектора 
состояния~$\Theta_\alpha(\hat{t})$, $\alpha=i,e$, в момент времени $\hat{t}\hm=\hat{t}_n$, 
$\hat{t}_n\hm=n h_\tau$, $n=0,\ldots , N$; $h_\tau$~--- достаточно малый шаг интегрирования; $\Delta 
W_n$, $n=0,\ldots ,N$,~--- величина приращения винеровского процесса~$W(\hat{t})$ на отрезке $\left[ 
\hat{t}_n,\,\hat{t}_{n+1}\right]$, по определению независимая от~$\Theta_\alpha^0$, 
$\Delta W_0,\ldots , 
\Delta W_{n-1}$: $\Delta W_n\hm=W(\hat{t}_{n-1})\hm-W(\hat{t}_n)$; $\Delta W_n\hm\sim N(0,\,h_\tau)$, 
т.\,е.\ $\Delta W_n$ представляют собой гауссовские случайные величины с нулевыми математическими 
ожиданиями и дисперсиями, равными шагу интегрирования; $\Theta_\alpha^0$~--- значение вектора 
состояния $\Theta_\alpha(\hat{t})$, $\alpha\hm=i,e$, в момент времени $\hat{t}=0$, 
$\Theta_\alpha^0\hm\sim \hat{f}_\alpha^{\mathrm{maksv}}$. 

Частные производные $\partial^2\hat{g}_\alpha/\partial[\hat{v}_y]^2$ и $\partial \hat{h}_\alpha/\partial 
\hat{v}_y$, являющиеся составляющими матрицы $\sigma_\alpha (\hat{t}_n, 
\Theta_\alpha^n)\sigma_\alpha^{\mathrm{T}}(\hat{t}_n,\Theta_\alpha^n)$ и вектора $a_\alpha(\hat{t}_n, 
\Theta_\alpha^n)$ соответственно, аппроксимируются со вторым порядком точности на трехточечном 
шаблоне на основе значений~$\hat{g}_\alpha$ и~$\hat{h}_\alpha$~\cite{17-k}.
      
      В выражения для функций~$\hat{g}_\alpha$ и~$\hat{h}_\alpha$ входят интегралы, которые 
вычисляются методом Мон\-те-Кар\-ло с использованием набора значений скоростной компоненты 
вектора состояния~$\hat{v}_y$, полученных из решения СДУ Ито:
      \begin{equation*}
      \int \hat{f}_\alpha \left\vert \hat{v}_y-
\hat{v}_y^\prime\right\vert\,dv_y^\prime=M\left(\zeta\left(\hat{V}_y\right)\right)\,,
\end{equation*}
где
$$
      \zeta\left(\hat{V}_y\right)=\left\vert \hat{v}_y-\hat{V}_y\right\vert\,,\enskip \hat{V}_y\sim 
\hat{f}_\alpha\,.
  $$
      
      Для вычисления напряженности самосогласованного электрического поля $\hat{E}_y=-
\partial\hat{\varphi}/\partial\hat{y}$, входящей в вектор $a_\alpha(\hat{t}_n, \Theta_\alpha^n)$, необходимо 
аналогично аппроксимировать со вторым порядком точности производную 
$\partial\hat{\varphi}/\partial\hat{y}$ на трехточечном шаблоне с использованием значений 
потенциала~$\hat{\varphi}$~\cite{17-k}. Значения потенциала~$\hat\varphi$ находятся из решения 
уравнения Пуассона. 
      
      Граничную задачу для уравнения Пуассона 
      \begin{align*}
      \fr{\partial^2 \hat\varphi}{\partial \hat{y}^2} & = -\left(\hat{n}_i-\hat{n}_e\right)\,;\\
      \hat{\varphi}\big|_{\hat{y}=0} &=\hat{\varphi}_p\,;\\
      \hat{\varphi}\big|_{\hat{y}_\infty=0} &=0
      \end{align*}
    предлагается решать путем перехода к конечно-разностной системе с последующим ее решением 
методом прогонки~\cite{17-k}:

\noindent
\begin{gather*}
\hat{\varphi}^n_{l-1}+2\hat{\varphi}_l^n+\hat{\varphi}^n_{l+1}=
h_y\hat{\delta}_l^n\,,\enskip l=1,\ldots , 
N_y\,;\\
\hat{\delta}_l^n=-\left( \hat{n}^n_{i,l}-\hat{n}^n_{e,l}\right)\,;\enskip 
\hat{\varphi}_0=\hat{\varphi}_p\,;\enskip \hat{\varphi}_{N_y}=0\,,
\end{gather*}
где $N_y$~--- число шагов по переменной~$\hat{y}$, $h_y$~--- величина шагов разбиения по~$\hat{y}$. 
      
      Концентрации $\hat{n}_\alpha$, $\alpha=i,e$, и плотности токов частиц на зонд~$\hat{f}_\alpha$, 
$\alpha=i,e$, вычисляются согласно описанному выше методу Мон\-те-Карло.

\section{Применение метода расщепления и~метода крупных~частиц}

Решение задачи в данном случае предлагается начать с записи правой части уравнения 
Фок\-ке\-ра--План\-ка в декартовой системе координат в виде:
$$
\mathbf{Q} f_\alpha = \fr{1}{2}\,\fr{\partial^2 f_\alpha}{\partial [v_y]^2}\,\fr{\partial^2 g_\alpha}{\partial 
[v_y]^2}+\fr{\partial f_\alpha}{\partial v_y}\,\fr{\partial C_\alpha}{\partial v_y}+H_\alpha\,,\enskip 
\alpha=i,e\,,
$$  
где 
\begin{align*}
C_\alpha(\vec{r},\vec{v},t)&=
\begin{cases}
\fr{1-\gamma}{Z_i^2}\int\fr{f_e(\vec{r},{\vec{v}}^{\,\prime},t)}{|\vec{v}-{\vec{v}}^{\,\prime} |}\,d{\vec{v}}^{\,\prime}\,, 
&\alpha=i\,;\\[9pt]
\fr{Z_i^2(\gamma-1)}{\gamma}\int \fr{f_i(\vec{r},{\vec{v}}^{\,\prime}, t)}
{|\vec{v}-{\vec{v}}^{\,\prime} 
|}\,d{\vec{v}}^{\,\prime}\,, &\alpha=e\,;
\end{cases} 
\\
H_\alpha&=
\begin{cases}
4\pi \left( \fr{\gamma f_e}{Z_i^2}+f_i\right)f_i\,, & \alpha=i\,;\\[9pt]
4\pi\left(\fr{Z_i^2 f_i}{\gamma}+f_e\right)f_e\,, &\alpha=e\,.
\end{cases}
\end{align*}
Тогда при переходе к безразмерным величинам (см.\ разд.~3) система~(\ref{e1-k}) запишется 
следующим образом:
      \begin{equation}
      \left.
\!\!\begin{array}{l}
      \fr{\partial 
\hat{f}_\alpha}{\partial\hat{t}}+A_\alpha\fr{\partial\hat{f}_\alpha}{\partial\hat{y}}+
B_\alpha  \hat{E}_y
\fr{\partial\hat{f}_\alpha}{\partial\hat{v}_\alpha}=\tilde{\mathbf{Q}}\hat{f}_\alpha\,,\enskip 
\alpha=i,e;\\[9pt]
      \fr{\partial^2\hat{\varphi}}{\partial\hat{y}^2}=-\left( \hat{n}_i-\hat{n}_e\right)\,,\enskip \hat{E}_y=-
\fr{\partial\hat\varphi}{\partial\hat{y}}\,,\\[9pt]
\hspace*{3.1mm}\hat{t}=0:\ \hspace*{2.6mm}\hat{f}_\alpha(\hat{y},\hat{v}_y, 0)=\hat{f}_\alpha^{\mathrm{maksv}}\,,\enskip \alpha=i,e\,,\\[9pt]
\hspace*{2.9mm} \hat{y}=0:\ \hspace*{2.8mm}\hat{f}_\alpha(0,\hat{v}_y,\hat{t})=0\,,\enskip \alpha=i,e\,;\\[9pt]
\hspace*{24.3mm}\hat\varphi(0,\hat{t})=\hat{\varphi}_p\,;\\[9pt]
      \hat{y}=\hat{y}_\infty:\ \hat{f}_\alpha(\hat{y}_\infty, 
\hat{v}_y,\hat{t})=\hat{f}_\alpha^{\mathrm{maksv}}\,,\enskip \alpha=i,e\,;\\[9pt]
\hspace*{21.5mm}\hat{\varphi}(\hat{y}_\infty,\hat{t})=0\,,\\[9pt]
    \end{array}
\right\}\!\!
\label{e7-k}
\end{equation}
где 
\begin{gather*}
\tilde{\mathbf{Q}} \hat{f}_\alpha=D_\alpha\fr{\partial^2\hat{f}_\alpha}{\partial 
[\hat{v}_y]^2}+K_\alpha\fr{\partial\hat{f}_\alpha}{\partial\hat{v}_y}+H_\alpha\,;\\
D_\alpha=A_g^\alpha\fr{\partial^2\hat{g}_\alpha}{\partial [\hat{v}_y]^2}\,;\enskip 
K_\alpha=A_h^\alpha \fr{\partial \hat{h}_\alpha}{\partial\hat{v}_y}\,,\ \alpha=i,e\,.
\end{gather*}

Для решения системы уравнений~(\ref{e7-k}) применяется модификация метода 
расщепления~\cite{17-k}, согласно которой исходная задача разбивается на две вспомогательные. Такое 
разбиение можно осуществить, переписав уравнение Фок\-ке\-ра--План\-ка в следующем виде:
$$
\fr{\partial\hat{f}_\alpha}{\partial\hat{t}} =
\tilde{\mathbf{Q}}_1\hat{f}_\alpha+\tilde{\mathbf{Q}}_2\hat{f}_\alpha\,,
$$
где 
\begin{align*}
\tilde{\mathbf{Q}}_1\hat{f}_\alpha &=-
\left(A_\alpha\fr{\partial\hat{f}_\alpha}{\partial\hat{y}}+
B_\alpha\fr{\partial\hat{f}_\alpha}{\partial\hat{y}}
\right)\,;\\
\tilde{\mathbf{Q}}_2\hat{f}_\alpha 
&=\left(D_\alpha\fr{\partial^2\hat{f}_\alpha}{\partial[\hat{v}_y]^2}+K_\alpha\fr{\partial 
\hat{f}_\alpha}{\partial\hat{v}_y}+H_\alpha\right)\,.
\end{align*}

      Правая часть уравнения Фок\-ке\-ра--План\-ка представляет собой сумму двух операторов, 
первый из которых отвечает за перенос частиц, второй~--- за столкновения заряженных частиц. 
В~результате образуются следующие задачи, которые решаются последовательно:
      \begin{itemize}
\item первая задача:
\begin{align*}
&\fr{\partial w_\alpha(\hat{y},\hat{v}_y,\hat{t})}{\partial\hat{t}} =\mathbf{Q}_1 
w_\alpha(\hat{y},\hat{v}_y,\hat{t})\,,\enskip \alpha=i,e\,;\\[9pt]
&\fr{\partial^2\hat\varphi}{\partial\hat{y}^2}=-\left(\hat{n}_i-\hat{n}_e\right)\,;\enskip
\hat{E}_y=-
\fr{\partial\hat\varphi}{\partial\hat{y}}\,;\\[9pt]
&w_\alpha(\hat{y},\hat{v}_y,\hat{t}^n)=\hat{f}_\alpha(\hat{y},\hat{v}_y,\hat{t}^n)\,,\enskip n=0,\ldots ,N-
1\,;\\[9pt]
&\hspace{2.9mm}\hat{y}=0:\ \hspace*{2.9mm}w_\alpha(0,\hat{v}_y,\hat{t})=0\,,\enskip \alpha=i,e\,;\\[9pt]
&\hspace*{25.1mm}\hat\varphi(0,\hat{t})=\hat{\varphi}_p\,;\\[9pt]
&\hat{y}=\hat{y}_\infty:\ w_\alpha(\hat{y}_\infty, \hat{v}_y, \hat{t})=
\hat{f}_\alpha^{\mathrm{maksv}}\,,\enskip 
\alpha=i,e\,;\\[9pt]
&\hspace*{22.5mm}\hat\varphi(\hat{y}_\infty,\hat{t})=0\,;
\end{align*}
\item вторая задача:
\begin{align*}
\!\!\!\!\!\!\!\fr{\partial s_\alpha(\hat{y},\hat{v}_y,\hat{t})}{\partial \hat{t}} &=\mathbf{Q}_2 
s_\alpha(\hat{y},\hat{v}_y,\hat{t})\,, & \alpha&=i,e\,;\\
\!\!\!\!\!\!\!s_\alpha (\hat{y},\hat{v}_y,\hat{t}^n) &=w_\alpha (\hat{y},\hat{v}_y, \hat{t}^{n+1}),& n&=0,\ldots ,N-
1.
\end{align*}
\end{itemize}

Первая задача представляет собой систему безразмерных уравнений Вла\-со\-ва--Пуас\-со\-на. Для ее 
решения применяется метод крупных частиц~\cite{18-k}. Согласно этому методу решение задачи 
осуществляется путем расщепления на два этапа: на первом этапе не учитываются конвективные члены 
и решение получается обычным интегрированием на неподвижной эйлеровой сетке, а на втором этапе 
рассматривается система, которая описывает перенос частиц в лагранжевой системе координат. Кроме 
того, на первом этапе необходимо решить уравнение Пуассона для получения значений потенциала 
самосогласованного электрического поля. Для этого применяется метод, описанный в разд.~3. 

Вторая задача решается путем перехода к ко\-неч\-но-раз\-ност\-ной сис\-те\-ме. При этом частные 
производные $\partial^2\hat{g}_\alpha/\partial[\hat{v}_y]^2$ и $\partial\hat{h}_\alpha/\partial\hat{v}_y$ 
аппроксимируются со вторым порядком точности с использованием трехточечного шаблона, а 
производная $\partial s_\alpha/\partial\hat{t}$ аппроксимируется на двухточечном шаблоне с первым 
порядком точности~\cite{16-k}. К~полученной системе разностных уравнений предлагается применить 
один из классических методов решения систем линейных уравнений, например метод 
Гаусса~\cite{19-k}.
      
      Решением первой задачи является функция $w_\alpha(\hat{y}, \hat{v}_y, \hat{t}^n)$, 
$n\hm=0,\ldots ,N$, , которая дает начальное условие для второй задачи. Решая вторую задачу, находим 
функцию $s_\alpha(\hat{y},\hat{v}_y,\hat{t}^n)\hm=\hat{f}_\alpha(\hat{y},\hat{v}_y,\hat{t}^n)$, 
$n=1,\ldots ,N$, $\alpha=i,e$, которая определяет решение $\hat{f}_\alpha(\hat{y},\hat{v}_y,\hat{t}^n)$, 
$\alpha=i,e$, исходной системы~(\ref{e7-k}) для рассматриваемых моментов времени $n=1,\ldots ,N$.

Моменты функций распределения $\hat{f}_\alpha$, $\alpha=i,e$, находятся с помощью методов 
численного интегрирования, например метода трапеций~\cite{19-k}.

\section{Результаты численного моделирования}

Для двух описанных выше методов реализованы две отдельные программы в среде {Matlab~7.0}. 
Эти программы позволяют по заданным значениям концентраций и температур частиц $n_{i\infty}$, 
$n_{e\infty}$, $T_{i\infty}$ и~$T_{e\infty}$ в невозмущенной плазме, а также потенциала~$\varphi_p$, 
подаваемого на зонд, изучить эволюцию во времени плотностей тока частиц~$j_i$ и~$j_e$, концентраций 
частиц~$n_i$  и~$n_e$ в произвольной точке пространства в возмущенной зоне, а также динамику 
изменения напряженности~$E_y$ самосогласованного электрического поля во времени и пространстве.

С использованием разработанных программ проведены серии расчетных экспериментов, в которых 
значение концентраций варьировалось в пределах $n_{i\infty} \hm = n_{e\infty}\hm =10^{18}\div 
10^{22}$~м$^{-3}$. Значение температур было выбрано неизменным и равным $T_{i\infty}\hm = 
T_{e\infty}\hm=3000$~K, а значения потенциала, подаваемого на зонд, изменялись в пределах 
$\varphi_p\hm=0\div 2{,}6$~В.

На рис.~1  и~2 приведены графики изменения напряженности самосогласованного электрического
 поля (см.\ рис.~1) и плотности токов ионов (см.\linebreak\vspace*{-12pt}

\pagebreak

\end{multicols}

\begin{figure} %fig1
\vspace*{1pt}
\begin{center}
\mbox{%
\epsfxsize=162.594mm
\epsfbox{kud-1.eps}
}
\end{center}
\vspace*{-9pt}
\Caption{Динамика изменения плотности тока ионов во времени в фиксированной точке возмущенной 
зоны для значений потенциала: \textit{1}~--- $\varphi_p=-6$; 
\textit{2}~--- $\varphi_p=-16$; \textit{3}~--- $\varphi_p=- 30$ 
в случае применения методов Монте-Карло~(\textit{а}) 
и крупных частиц~(\textit{б})}
\end{figure}

\begin{figure} %fig2
\vspace*{1pt}
\begin{center}
\mbox{%
\epsfxsize=162.713mm
\epsfbox{kud-2.eps}
}
\end{center}
\vspace*{-9pt}
\Caption{Динамика изменения напряженности электрического поля во времени в фиксированной точке 
возмущенной зоны для значений потенциала: 
\textit{1}~--- $\varphi_p=-6$; \textit{2}~--- $\varphi_p=-16$; 
\textit{3}~--- $\varphi_p=-30$ в случае применения методов Монте-Карло~(\textit{а}) и
крупных частиц~(\textit{б})
}
\end{figure}

\begin{multicols}{2}

\noindent
 рис.~2) во времени в фиксированной точке пространства 
возмущенной зоны в случае применения обоих разработанных алгоритмов.


На основании полученных результатов можно отметить похожее поведение зависимостей 
напряженности электрического поля и плотности тока от времени в двух рассматриваемых случаях. 
Графики кривых сначала убывают, затем начинают возрастать, выходя в некоторый момент 
времени~$t^\prime$ (момент установления) на стационарные значения. 

Одинаковое поведение 
напряженности и плот\-ности тока можно объяснить из следующих соображений: плотность тока ионов в 
данной области пространства равна произведению концентрации ионов на их направленную скорость и 
на заряд иона. Скорость ионов, в свою очередь, зависит от заряда, массы и напряженности 
электрического поля. 
%\columnbreak

При внесении в плазму отрицательно заряженного зонда возникает электрическое поле, которое 
нарушает квазинейтральность плазмы. Для того чтобы компенсировать действие внешнего 
электрического поля, ионы устремляются к зонду, а электроны~--- от зонда. Это приводит к дисбалансу 
концентраций вблизи зонда и, как следствие, к увеличению разности потенциалов; график 
напряженности электрического поля убывает. Вскоре разделение зарядов компенсирует внешнее 
электрическое поле; график выходит на стационарное значение. 

Также можно отметить, что значения 
напряженности электрического поля и плотности тока частиц на зонд в момент установления для двух 
методов совпадают. 

Момент установления~$t^\prime$ зависит от при\-ме\-ня\-емо\-го метода решения. В~случае метода 
Мон\-те-Кар\-ло $t^\prime=3{,}5\div 4$~ед., а для метода крупных частиц совместно с методом 
расщепления $t^\prime\hm=5\div 5{,}5$~ед. Используя ко\-неч\-но-раз\-ност\-ный метод, можно 
получить динамику изменения функций распределения частиц~$f_\alpha$, $\alpha=i,e$, во времени и 
пространстве. Функции распределения позволяют наглядно представить влияние на картину 
распределения частиц вблизи зонда самой поверхности зонда и электрического поля.

\section{Заключение}
      
      В работе найдено решение задачи диагностики плоским зондом сильноионизованной плазмы с 
учетом столкновений заряженных частиц. Разработана математическая модель исследуемого явления, 
описываемая уравнениями Фок\-ке\-ра--План\-ка и Пуассона. Решение получено двумя методами:\linebreak 
статистическим и ко\-неч\-но-раз\-ност\-ным на основе\linebreak сформированных алгоритмов. Приведены 
резуль-\linebreak таты численного моделирования при различных\linebreak характерных параметрах задачи.
 Из  проведенных 
вычислительных экспериментов вытекает, что искомые величины: напряженность 
электрического поля, плотности токов частиц на зонд, концентрации частиц вблизи зонда~--- как по 
характеру зависимости, так и по числовым значениям совпадают. При применении метода 
      Мон\-те-Кар\-ло момент установления наступает быстрее по сравнению с конечно-разностным 
методом, однако конечно-разностный метод позволяет получить более наглядные результаты.

{\small\frenchspacing
{%\baselineskip=10.8pt
\addcontentsline{toc}{section}{Литература}
\begin{thebibliography}{99}

\bibitem{1-k}
\Au{Alexeff I., Anderson T.}
Experimental and theoretical results with plasma antenna~// IEEE Trans. Plasma Sci., 2006. Vol.~34. 
No.\,2. P.~166--172.

\bibitem{2-k}
\Au{Сысун В.\,И.}
Сильноионизованная низкотемпературная плазма в приборах электронной техники: Методы 
исследования, свойства, применение. Дисс. \ldots д-ра физ.-мат. наук в форме науч. докл.: 
01.04.08.~--- Пет\-ро\-за\-водск, 1996.

\bibitem{3-k}
\Au{Тухас В.\,А.}
Методология создания средств измерений и испытаний на устойчивость к кондуктивным помехам~// 
Мат-лы VI Междунар. симп. по электромагнитной совместимости и 
электромагнитной экологии.~--- СПб., 2005. С.~231--234.

\bibitem{4-k}
\Au{Гудзенко Л.\,И., Яковленко С.\,И.}
Плазменные лазеры.~--- М.: Атомиздат, 1978.  256~с.

\bibitem{5-k}
\Au{Звелто О.}
Принципы лазеров.~--- М.: Мир, 1990.  560~с.

\bibitem{6-k}
\Au{Сысун В.\,И., Хромой Ю.\,Д.}
Расширение канала мощного импульсного разряда в парах ртути~// Электронная техника, 1974. 
Сер.~4. Вып.~10. С.~80--85. 

\bibitem{7-k}
\Au{Винклер Дж.\,Р.}
Искусственные пучки частиц в космической плазме.~--- М.: Мир, 1985.  451~с.

\bibitem{8-k}
\Au{Bernstein I.\,B., Rabinowitz I.\,N.}
Theory of electrostatic probes in low-density plasma~// Phys. Fluids, 1959. Vol.~2. No.\,2. P.~112--121. 

\bibitem{9-k}
\Au{Альперт Я.\,Л., Гуревич А.\,В., Питаевский~Л.\,П.}
Искусственные спутники в разреженной плазме.~--- М.: Наука, 1964.  282~с.

\bibitem{10-k}
\Au{Чан П., Тэлбот Л., Турян~К.}
Электрические зонды в неподвижной и движущейся плазме.~--- М.: Мир, 1978.  202~с.

\bibitem{11-k}
\Au{Алексеев Б.\,В., Котельников В.\,А.}
Зондовый метод диагностики плазмы.~--- М.: Энергоатомиздат, 1989.  240~с.

\bibitem{12-k}
\Au{Пантелеев А.\,В., Кудрявцева И.\,А.}
Формирование математической модели двухкомпонентной плазмы с учетом столкновений 
заряженных частиц в случае плоского зонда~// Теоретические вопросы вычислительной техники и 
программного обеспечения: Межвузовский сб. научн. тр.~--- М.: МИРЭА, 2006. С.~11--21.

\bibitem{13-k}
\Au{Олдер Б.}
Вычислительные методы в физике плазмы.~--- М.: Мир, 1974.  111~с.

\bibitem{14-k}
\Au{Montgomery D.\,C., Tidman D.\,A.}
Plasma kinetic theory.~--- New York, 1964. 

\bibitem{15-k}
\Au{Кудрявцева И.\,А., Пантелеев А.\,В.}
Применение метода Мон\-те-Кар\-ло для анализа поведения двухкомпонентной плазмы с учетом 
столкновений между заряженными частицами~// Теоретические вопросы\linebreak
вычислительной техники и 
программного обеспечения: Межвузовский сб. научн. тр.~--- М.: МИРЭА, 2008. С.~122--128. 

\bibitem{16-k}
\Au{Семенов В.\,В., Пантелеев А.\,В., Руденко~Е.\,А., Бор\-та\-ков\-ский~А.\,С.}
Методы описания, анализа и синтеза нелинейных систем управления.~--- М.: МАИ, 1993.  312~с.

\bibitem{17-k}
\Au{Киреев В.\,И., Пантелеев А.\,В.}
Численные методы в примерах и задачах.~--- М.: Высшая школа, 2006.  480~с.

\bibitem{18-k}
\Au{Белоцерковский О.\,М., Давыдов~Ю.\,М.}
Метод крупных частиц в газовой динамике. Вычислительный эксперимент.~--- М.: Наука, 
Физматгиз, 1982.

\label{end\stat}

\bibitem{19-k}
\Au{Вержбицкий В.\,М.}
Основы численных методов.~--- М.: Высшая школа, 2002.  840~с.
 \end{thebibliography}
}
}


\end{multicols}         %7

\def\stat{shnurkov}

\def\tit{АНАЛИТИЧЕСКОЕ РЕШЕНИЕ ЗАДАЧИ ОПТИМАЛЬНОГО УПРАВЛЕНИЯ ПОЛУМАРКОВСКИМ ПРОЦЕССОМ\\ 
С~КОНЕЧНЫМ МНОЖЕСТВОМ СОСТОЯНИЙ$^*$}

\def\titkol{Аналитическое решение задачи оптимального управления полумарковским 
процессом} %с~конечным множеством состояний}

\def\aut{П.\,В.~Шнурков$^1$, А.\,К.~Горшенин$^2$, В.\,В.~Белоусов$^3$}

\def\autkol{П.\,В.~Шнурков, А.\,К.~Горшенин, В.\,В.~Белоусов}

\titel{\tit}{\aut}{\autkol}{\titkol}

\index{Шнурков П.\,В.}
\index{Горшенин А.\,К.}
\index{Белоусов В.\,В.}
\index{Shnurkov P.\,V.}
\index{Gorshenin A.\,K.}
\index{Belousov V.\,V.}


{\renewcommand{\thefootnote}{\fnsymbol{footnote}} \footnotetext[1]
{Работа выполнена при частичной поддержке РФФИ (проект 15-07-05316).}}


\renewcommand{\thefootnote}{\arabic{footnote}}
\footnotetext[1]{Национальный исследовательский университет <<Высшая школа экономики>>, 
\mbox{pshnurkov@hse.ru}}
\footnotetext[2]{Институт проблем информатики Федерального исследовательского центра <<Информатика 
и~управ\-ле\-ние>> Российской академии наук, \mbox{agorshenin@frccsc.ru}}
\footnotetext[3]{Институт проблем информатики Федерального исследовательского центра <<Информатика 
и~управление>> Российской академии наук, \mbox{vbelousov@ipiran.ru}}

%\vspace*{-6pt}

\Abst{Настоящее исследование посвящено теоретическому обоснованию нового метода 
нахождения оптимальной стратегии управления полумарковским процессом с~конечным 
множеством состояний. Рассматриваются марковские рандомизированные стратегии 
управления, определяемые конечным набором вероятностных мер, соответствующих 
каждому состоянию. Характеристикой качества управления служит стационарный 
стоимостной показатель. Данный показатель представляет собой дроб\-но-ли\-ней\-ный 
интегральный функционал от набора вероятностных мер, задающих стратегию управления. 
Для этого функционала известны явные аналитические представления подынтегральных 
функций числителя и~знаменателя. Дальнейшие результаты основываются на новой 
усиленной и~обобщенной форме теоремы об экстремуме дроб\-но-ли\-ней\-но\-го интегрального 
функционала. Доказывается, что проблемы существования оптимальной стратегии управления 
полумарковским процессом и~ее нахождения сводятся к~задаче численного исследования 
на глобальный экстремум заданной функции от конечного числа вещественных переменных.}

\KW{оптимальное управление полумарковским процессом; стационарный стоимостной 
показатель качества управления; дроб\-но-ли\-ней\-ный интегральный функционал}

\DOI{10.14357/19922264160408} 

\vspace*{9pt}


\vskip 10pt plus 9pt minus 6pt

\thispagestyle{headings}

\begin{multicols}{2}

\label{st\stat}

\section{Введение}

Теория оптимального управления марковскими и~полумарковскими случайными 
процессами интенсивно развивается с~начала 1960-х~гг. Еще в~первых 
основополагающих исследованиях рассматривались не только проблемы существования 
оптимальных стратегий управления, но и~способы нахождения этих стратегий. 

Для решения таких проблем, имеющих алгоритмическое содержание, использовались 
открытые незадолго до этого мощные методы прикладной математики: линейное 
программирование и~динамическое программирование. Отметим, прежде всего, 
классическую работу Р.~Ховарда~\cite{1}, в~которой метод динамического 
программирования был применен для решения проблемы оптимального управления 
марковским процессом с~непрерывным временем. В~дальнейшем В.\,В.~Рыков~\cite{2} 
доказал, что для аналогичной модели управления марковским процессом с~учетом 
переоценки оптимальной стратегией также является стационарная.

Важную роль в~развитии теории управления случайными процессами сыграла работа 
В.~Джевелла~\cite{3}, в~которой были впервые рассмотрены полумарковские модели 
управления для вариантов с~переоценкой и~без переоценки. Данная работа была 
переведена на русский язык и~послужила основой для многих последующих работ 
отечественных и~зарубежных специалистов. В~частности, Б.~Фокс показал~\cite{4}, 
что оптимальной стратегией управления полумарковским процессом в~варианте без 
переоценки является стационарная; аналогичные результаты были получены Э.~Денардо 
и~для варианта с~переоценкой~\cite{5}.

Среди последующих исследований алгоритмической направленности отметим работы 
Р.~Ховарда~\cite{6}, Б.~Фокса~\cite{4}, а также С.~Осаки и~Х.~Майна~\cite{7}. 
В~этих работах для нахождения оптимальных стратегий управления полумарковскими 
процессами использовался метод линейного программирования.

В 1970~г.\ была опубликована фундаментальная монография Х.~Майна и~С.~Осаки~\cite{8}, 
переведенная на русский язык в~1977~г., в~которой были систе\-ма\-ти\-зи\-ро\-ва\-ны и~изложены 
основные результаты по теории оптимального управления марковскими и~полумарковскими 
случайными процессами. Фактически данная книга стала итогом исследований по проблемам 
стохастического управления\linebreak
 за~10~лет. Отметим, что в~этой монографии рас\-смат\-ри\-ва\-лись 
марковские и~полумарковские модели управления с~конечными множествами состояний 
и~допустимых решений, принимаемых \mbox{в~каждом} состоянии. Были получены принципиальные 
тео\-ре\-ти\-че\-ские результаты, заключающиеся в~том, что оптимальные стратегии управ\-ле\-ния 
для основных видов рас\-смат\-ри\-ва\-емых моделей с~переоценкой и~без переоценки являются 
детерминированными и~стационарными. Были разработаны и~обоснованы процедуры нахождения 
оптимальных стратегий управления. В~частности, для модели управления полумарковским 
процессом без переоценки, когда множество со\-сто\-яний образует один эргодический класс, 
а~показатель качества управления пред\-став\-ля\-ет собой стационарный средний удельный 
доход (см.~[8, гл.~5, п.~5.5]), процедура поиска оптимальной рандомизированной 
стратегии осуществлялась методом линейного программирования. Обратим особое внимание 
на данный результат, поскольку аналогичная модель управления полумарковским 
процессом будет рассмотрена в~настоящей работе.

Принципиальную роль в~развитии теории стохастического управления сыграла 
монография И.\,И.~Гихмана и~А.\,В.~Скорохода~\cite{9}. В~этой книге были впервые 
систематически изложены основы теории оптимального управления случайными процессами 
с~дискретным и~непрерывным временем, включая теорию управления процессами, которые 
описываются стохастическими дифференциальными уравнениями. Отдельно были рас\-смот\-ре\-ны 
проблемы управления марковскими процессами с~дискретным временем и~скачкообразными 
марковскими процессами с~непрерывным временем. Роли множеств состояний и~допустимых 
управ\-ле\-ний играли пространства весьма общей структуры. Для широких классов функционалов 
качества управ\-ле\-ния (так называемых эволюционных функционалов в~марковских моделях 
с~дискретным временем и~интегральных функционалов накопления в~марковских моделях 
с~непрерывным временем) были доказаны теоремы о~существовании и~формах пред\-став\-ле\-ния 
оптимальных стратегий управ\-ле\-ния. Было установлено, что для однородных марковских 
моделей оптимальные стратегии управ\-ле\-ния существуют, являются стационарными 
и~детерминированными. Иначе говоря, такие стратегии задаются детерминированными 
функциями, аргументом которых является со\-сто\-яние сис\-те\-мы в~момент принятия решения, 
и~не зависящими от самого момента принятия решения. Что же касается важного вопроса 
о~формах представления этих функций, то их можно охарактеризовать следующим образом. 
Были найдены функциональные уравнения, осложненные условием экстремума, которым 
удовле\-тво\-ря\-ют упомянутые функции. По существу эти соотношения пред\-став\-ля\-ют собой 
уравнения Беллмана для соответствующих динамических стохастических моделей.

Особо отметим, что в~монографии~\cite{9} не рас\-смат\-ри\-ва\-лись проблемы управления 
полумарковскими процессами. Однако дальнейшее развитие общей теории управления 
такими процессами шло по пути, идейно намеченному в~указанной книге.

В последующие годы развитие теории управ\-ле\-ния полумарковскими процессами 
осуществля-\linebreak лось по направлению усложнения моделей и~обобщения исходных предположений. 
Например,\linebreak в~работах~\cite{10, 11} рассмотрены управляемые по\-лумарковские процессы при 
весьма общих предположениях относительно характера пространств состояний и~управлений. 
Проблемы управления исследовались по отношению к~различным видам целевых показателей, 
обобщающих упомянутый выше стационарный показатель средней удельной прибыли. В~этих 
работах доказывается, что оптимальная стратегия управления по отношению к~каж\-до\-му из 
показателей существует и~является одной и~той же стационарной детерминированной 
стратегией, определяемой некоторой функцией, заданной на множестве со\-сто\-яний процесса. 
Об этой функции известно лишь то, что она удовлетворяет некоторому интегральному 
уравнению, которое по содержанию пред\-став\-ля\-ет собой уравнение Бел\-лма\-на для 
соответствующей задачи управ\-ления.

Среди исследований, предшествовавших настоящему, отметим работу 
В.\,А.~Каштанова~[12, гл. 13]. В этом разделе коллективной монографии~\cite{12} 
автором была рассмотрена проблема оптимального управления полумарковским 
процессом с~конечным множеством состояний и~множеством возможных решений, 
которое представляет собой произвольный интервал множества вещественных чисел. 
Модель относится к~виду моделей без переоценки, показателем качества управления 
служит стационарное значение среднего удельного дохода, определяемое аналогично 
классическим работам~\cite{3, 8}. Рандомизированное управление в~каждом состоянии 
определяется в~соответствии с~вероятностным распределением, совокупность которых 
задает\linebreak
 стратегию управления. В.\,А.~Каш\-та\-но\-вым было\linebreak сформулировано утверждение о том, 
что стацио\-нарное значение среднего удельного дохода представляет собой 
дроб\-но-ли\-ней\-ный интегральный функционал от набора вероятностных распределений, 
образующих стратегию управления. При этом\linebreak ранее~[12, гл.~10; 13] было уста\-нов\-ле\-но, 
что дроб\-но-ли\-ней\-ный функционал достигает экстремума на вырожденных распределениях. 
Отсюда естест-\linebreak венно следует, что оптимальная стратегия управ\-ле-ния является 
детерминированной и~должна\linebreak определяться точкой экстремума функции, представляющей 
собой отношение подынтегральных функций чис\-ли\-те\-ля и~знаменателя данного 
дроб\-но-ли\-ней\-но\-го функционала. Однако в~\cite{12} не были получены явные 
представления для указан-\linebreak ных функций. Кроме того, приведенный в~гл.~10 
монографии~\cite{12} вариант теоремы об экстремуме дроб\-но-ли\-ней\-но\-го 
интегрального функционала требовал проверки выполнения условия существования 
этого экстремума. Такие условия указаны не были. В~связи с~этими обстоятельствами 
использовать полученные в~\cite{12} результаты для доказательства существования 
оптимальной детерминированной стратегии управ\-ле\-ния полумарковским процессом и~для 
строгого обоснования способа нахождения такой стратегии оказалось невозможным.

Настоящее исследование посвящено теоретическому обоснованию нового метода 
нахождения\linebreak оптимальной стратегии управления полумарковским процессом с~конечным 
множеством со\-сто\-яний. Рассматриваются марковские рандомизи\-рованные стратегии 
управления, определяемые конеч\-ным набором вероятностных мер, соответствующих 
каждому состоянию. Показателем качества управления служит уже упоминавшийся 
классический  показатель~--- стационарное значение средней удельной прибыли. 
Доказано, что этот показатель представляет собой дроб\-но-ли\-ней\-ный интегральный 
функционал от набора вероятностных мер, задающих стратегию управления. При этом, 
в~отличие от~\cite{12}, получены явные аналитические представления для подынтегральных 
функций числителя и~знаменателя этого дроб\-но-ли\-ней\-но\-го\linebreak
 функционала. Дальнейшие 
результаты основываются на новой усиленной и~обобщенной форме\linebreak
 теоремы об экстремуме 
дроб\-но-ли\-ней\-но\-го интегрального функционала, впервые опубликованной 
в~работе П.\,В.~Шнуркова~\cite{14}. Согласно\linebreak
 утверж\-де\-нию этой теоремы, если 
существует глобальный экстремум так называемой основной функции дроб\-но-ли\-ней\-но\-го 
функционала, которая пред\-став\-ля\-ет собой отношение подынтегральных функций чис\-ли\-те\-ля 
и~знаменателя, то существует безусловный экстремум самого дроб\-но-ли\-ней\-но\-го 
функционала, который достигается на наборе вырожденных вероятностных распределений, 
сосредоточенных в~точке глобального экстремума. В~этом случае оптимальная стратегия 
управ\-ле\-ния существует, является стационарной и~детерминированной и~определяется точкой, 
в~которой основная\linebreak функция достигает глобального экстремума. Таким\linebreak образом, проблемы 
существования оптимальной стратегии управ\-ле\-ния полумарковским процессом и~ее 
нахождения сводятся к~задаче чис\-лен\-но\-го исследования на глобальный экстремум 
заданной функции от конечного чис\-ла вещественных переменных.

\section{Общее описание модели управления полумарковским случайным процессом}

Построим модель управления полумарковским случайным процессом, следуя общему 
подходу, принятому в~классических работах~\cite{3, 8}. Пусть $\xi(t)$~--- 
случайный полумарковский процесс с~конечным множеством состояний
$X\hm=\{1,2,\ldots, N\}$, $N\hm< \infty$. Обозначим через~$t_n$, $n=0,1,2,\ldots$, 
$t_0\hm=0$, случайные моменты изменения состояний данного процесса, 
$\theta_n\hm=t_{n+1}-t_n$, $n\hm=0,1,2,\ldots$, $\xi_n\hm=\xi(t_n)\hm=\xi(t_n+0)$, 
$n\hm=0,1,2,\ldots$ (предполагается, что траектории процесса~$\xi(t)$ 
непрерывны справа). Случайная последовательность~$\{\xi_n\}$
образует цепь Маркова, вложенную в~полумарковский процесс~$\xi(t)$.
Зададим набор измеримых пространств\linebreak $(U_1, \mathscr{B}_1), 
(U_2, \mathscr{B}_2), \ldots, (U_N, \mathscr{B}_N)$, где $U_i$~--- 
множество возможных допустимых управ\-ле\-ний, $\mathscr{B}_i$~--- $\sigma$-ал\-геб\-ра 
подмножеств множества~$U_i$, вклю\-ча\-ющая в~себя все одноточечные подмножества\linebreak  
множества~$U_i$, т.\,е.\ если $u_i \hm\in U_i$, то $\{u_i\} \hm\in \mathscr{B}_i$, 
$i\hm=1,2,\ldots, N$.
Пусть $\Gamma_i$~--- некоторое множество всевозможных вероятностных мер $\Psi_i 
\hm \in \Gamma_i$, заданных на $\sigma$-ал\-геб\-ре~$\mathscr{B}_i$, $i\hm=1,2,\ldots,N$.

Поскольку идейное содержание и~свойства вероятностных мер существенно используются 
в~данной работе, укажем на некоторые фундаментальные издания, в~которых 
изложена соответствующая тео\-рия. Понятие и~основные свойства вероятностной 
меры определены и~подробно проанализированы в~книге А.\,Н.~Ширяева~\cite[гл.~II]{15}. 
Глубокое изложение основ теории вероятностных мер имеется также в~книге 
А.\,А.~Боровкова~\cite{16}. Заметим попутно, что в~книге~\cite{16} имеются разделы, 
посвященные изложению основ теории полумарковских и~регенерирующих случайных процессов. 
Из зарубежных изданий отметим фундаментальную работу П.~Хеннекена и~А.~Тортра~\cite{17}, 
основная часть которой посвящена изложению математических основ теории вероятностей.

Введем специальное понятие вырожденной вероятностной меры, которое будет часто 
использоваться в~дальнейшем. Пусть $(U_0, \mathscr{B}_0)$~--- некоторое измеримое 
пространство, $\mathscr{B}_0$~--- $\sigma$-ал\-геб\-ра подмножеств множества~$U_0$, 
включающая в~себя все одноточечные подмножества этого множества.

\medskip

\noindent
\textbf{Определение 1.}\ Вероятностная мера~$\Psi^*$, заданная 
на $\sigma$-ал\-геб\-рe~$\mathscr{B}_0$, называется вырожденной, если существует 
такой элемент $u^* \hm\in U_0$, для которого выполняются условия $\Psi^*(\{u^*\})\hm=
1$, $\Psi^*(U_0 \setminus \{u^*\})\hm=0$, где $\{u^*\}=u^*$~--- 
множество, состоящее из единственной точки $u^* \hm\in U_0$. Соответствующая 
точка $u^* \hm\in U_0$ будет называться точкой сосредоточения вырожденной 
вероятностной меры~$\Psi^*$.
Таким образом, всякая вырожденная вероятностная мера~$\Psi^*$ определяется 
своей точкой сосредоточения~$u^*$. В~дальнейшем будем использовать 
обозначение~$\Psi_{u^*}^{*}$, имея в~виду, что вырожденная вероятностная мера~$\Psi^*$ 
сосредоточена в~точке~$u^*$.
Отметим также, что вырожденная вероятностная мера~$\Psi_{u^*}^{*}$ соответствует 
детерминированной величине, которая принимает фиксированное значение $u\hm=u^*$ 
с~вероятностью, равной единице.

\medskip

Обозначим через $\Gamma_0$ множество всех  вероятностных мер, заданных 
на измеримом пространстве ($U_0, \mathscr{B}_0$), а через~$\Gamma_0^*$~--- 
множество всех вырожденных вероятностных мер, заданных на этом пространстве, 
$\Gamma_0^*\hm\in \Gamma_0$. Аналогичные обозначения будут использоваться 
и~в~дальнейшем. Заметим, что множество~$\Gamma_0^*$ находится во взаимно
 однозначном соответствии с~множеством точек сосредоточения вырожденных 
 вероятностных мер, т.\,е.\ с~множеством~$U_0$.

Пусть $\Gamma_i^{*}$~--- множество всех вырожденных мер, заданных на 
$\sigma$-ал\-геб\-ре~$\mathscr{B}_i$, $\Gamma_i^{*}\hm\subset \Gamma_i$.
Произвольная вероятностная мера~$\Psi_i$ описывает случайную величину, 
принимающую значения в~$U_i$, а вырожденная мера~$\Psi_i^*$, сосредоточенная 
в~точке~$u_i^*$, соответствует детерминированной величине $u_i^*\hm\in U_i$.
Предполагается, что соответствующие конструкции определены на всех измеримых 
пространствах управлений $(U_1, \mathscr{B}_1), (U_2, \mathscr{B}_2), \ldots, 
(U_N,\mathscr{B}_N)$.

Предположим, что управления случайным полумарковским процессом~$\xi(t)$ 
осуществляются в~моменты времени~$t_n,$ $n\hm=0,1,2,\ldots,$
непосредственно после изменения состояния процесса. Если\linebreak 
$\xi_n\hm=\xi(t_n)\hm=i \hm\in X$, то значение управления представляет 
собой случайную величину~$u_n$, принимающую значения в~множестве допустимых 
управ\-ле\-ний~$U_i$ и~описываемую вероятностной мерой (распределе\-ни\-ем 
вероятностей) $\Psi_i \hm\in \Gamma_i$.
Будем предполагать, что при фиксированном условии $\xi_n\hm=\xi(t_n)=i$ 
управ\-ле\-ние определяется независимо от прошлого поведения процесса~$\xi(t)$ 
и~вероятностная мера~$\Psi_i$,
описывающая стохастическое управление~$u_n$, зависит только от состояния $i\hm\in X$.
Тогда выбор управ\-ле\-ний в~моменты изменения состояний $\{t_n, n\hm=0,1,2,\ldots \}$ 
описывается набором вероятностных мер (распределений вероятностей) 
$(\Psi_1, \Psi_2,\ldots, \Psi_N)$, 
$\Psi_i \hm\in \Gamma_i$, $i\hm=1,2,\ldots,N$.
Назовем любой такой набор стратегией управ\-ле\-ния полумарковским процессом~$\xi(t)$. 
По своим свойствам такая стратегия является марковской, однородной 
и~рандомизированной.

Следуя классической монографии П.~Халмоша~\cite[гл.~VII]{18}, 
рассмотрим декартово произведение пространств $U\hm=U_1\times U_2\times \cdots\times U_N$ 
и~соответствующих $\sigma$-ал\-гебр $\mathscr{B}\hm=\mathscr{B}_1\times \mathscr{B}_2
\times \cdots \times\mathscr{B}_N$. Обозначим через $\Psi\hm=\Psi_1\times \Psi_2\times \cdots
\times \Psi_N$ вероятностную меру на~$(U,\mathscr{B})$, определяемую как 
произведение мер $\Psi_1,\Psi_2, \ldots , \Psi_N$, где $\Psi_i \hm\in \Gamma_i$, 
$i\hm=1,2,\ldots,N$. Обозначим также через~$\Gamma$ множество вероятностных мер~$\Psi$, 
заданных на~$(U,\mathscr{B})$, которые пред\-став\-ля\-ют собой произведение 
мер $\Psi_1,\Psi_2, \ldots , \Psi_N$, где $\Psi_i \hm\in \Gamma_i$, $i\hm=1,2,\ldots,N$.
Множество~$\Gamma$ можно отож\-де\-ст\-вить с~множеством всех стратегий управ\-ле\-ния 
полумарковским процессом~$\xi(t)$.

Проблема оптимального управления полумар\-ковским процессом~$\xi(t)$ будет в~дальнейшем 
сформулирована в~виде задачи безусловного экстремума некоторого функционала 
$I(\Psi)\hm=I(\Psi_1,\Psi_2, \ldots , \Psi_N)$, заданного на множестве 
допустимых стратегий управления. Содержание показателя качества управления~$I(\Psi)$, 
аналитическое представление для него, а~также описание множества допустимых 
стратегий управления будут приведены в~последующих разделах данной работы.

Для получения дальнейших результатов потребуются различные вероятностные 
характеристики управляемого полумарковского процесса~$\xi(t)$. Как известно из
 общей теории полумарковских процессов~\cite{19, 20}, 
 основной вероятностной характеристикой такого процесса является так называемая 
 полумарковская функция. Определим эту функцию для процесса с~управлением 
 (см.~\cite[гл.~5]{8}):
\begin{multline}
Q_{ij}(t,u)=
{\sf P}\left(\xi_{n+1}=j,\theta_n<t \mid \xi_n=i, u_n=u\right)\,,\\
t\in [0,\infty)\,,\ u\in U_i\,;\ i,j\in X=\{1,2,\ldots,N\}\,. \label{e1}
\end{multline}
Используя полумарковские функции, можно получить вероятности перехода 
управляемой цепи Маркова~$\{\xi_n\}$:
\begin{multline}
p_{ij}(u)={\sf P}\left(\xi_{n+1}=j \mid \xi_n=i, u_n=u\right)= {}\\
{}=
\lim\limits_{t\rightarrow\infty}Q_{ij}(t,u)\,,\enskip
u\in U_i\,;\enskip i,j\in X\,, 
\label{e2}
\end{multline}
а также функции распределения длительностей пребывания полумарковского 
процесса~$\xi(t)$ в~соответствующих состояниях:

\noindent
\begin{multline}
H_{i}(t,u)={\sf P}\left(\theta_n<t \mid \xi_n=i, u_n=u\right)={}\\
{}=
\sum\limits_{j\in X}Q_{ij}(t,u)\,,\enskip
t\in [0,\infty)\,,\  u\in U_i\,; \  i\in X\,. 
\label{e3}
\end{multline}

Обозначим через
\begin{multline}
T_{i}(u)=\mathbf{E}\left[\theta_n \mid \xi_n=i, u_n=u\right]={}\\
{}=
\int\limits_0^{\infty}\left[1-H_i(t,u)\right]\,dt\,,\enskip
u\in U_i\,,\ i\in X\,, 
\label{e4}
\end{multline}
математические ожидания длительностей пребывания полумарковского процесса~$\xi(t)$ 
в~каждом из состояний.

Введенные выше характеристики~(1)--(4) определены для случая, когда 
в~момент изменения состояния~$t_n$ процесс оказывается в~состоянии~$i$ 
и~принимается решение $u\hm\in U_i$. При заданной стратегии управления 
$\Psi\hm=\left(\Psi_1,\Psi_2, \ldots , \Psi_N\right)$ можно записать 
соответствующие вероятностные характеристики без условия на управление, а~именно:
\begin{multline*}
Q_{ij}(t)={\sf P}\left(\xi_{n+1}=j,\theta_n<t \mid \xi_n=i\right)={}\\
{}=
\int\limits_{U_i}Q_{ij}(t,u) \,d\Psi_i(u)\,,\enskip 
t\in [0,\infty)\,,\ i,j\in X\,; %\label{e5}
\end{multline*}

\vspace*{-12pt}

\noindent
\begin{multline}
p_{ij}={\sf P}\left(\xi_{n+1}=j \mid \xi_n=i\right)=
\int\limits_{U_i}p_{ij}(u)\, d\Psi_i(u)\,,\\  
i,j\in X\,; 
\label{e6}
\end{multline}

\vspace*{-9pt}

\noindent
\begin{equation}
T_{i}=\mathbf{E}\left[\theta_n \mid \xi_n=i\right]=
\int\limits_{U_i}T_{i}(u)\,d\Psi_i(u)\,,\enskip i\in X\,. 
\label{e7}
\end{equation}
В дальнейшем будем предполагать, что для рас\-смат\-ри\-ва\-емой 
полумарковской модели заданы вероятностные характеристики 
$p_{ij}(u)$, $u\hm\in U_i$, $i,j\hm\in X$, и~$T_i(u)$, $u\hm\in U_i$, $i\hm\in X$, 
определяемые соотношениями~(\ref{e2}) и~(\ref{e4}). 
Для фиксированной стратегии управления $\Psi\hm=(\Psi_1, \Psi_2,\ldots, \Psi_N)$ 
соответствующие вероятностные характеристики~$p_{ij}$ и~ $T_i$, $i,j\hm\in X,$ 
определены равенствами~(\ref{e6}) и~(\ref{e7}) без условий на управление.

\section{Стационарный стоимостной показатель качества управления}

Определим некоторый стоимостной аддитивный функционал, связанный 
с~рассматриваемым полумарковским процессом~$\xi(t)$. По содержанию этот функционал 
представляет собой случайный\linebreak доход или прибыль, накопленную за период времени $[0,t]$. 
Определения такого функционала приведены в~основополагающих работах~[3; 8, гл.~5].\linebreak 
Обозначим через $\widetilde{v}(t)$, $t\hm\geq 0$, значение этого аддитивного 
функционала в~момент времени~$t$; $\widetilde{v}_n\hm=\widetilde{v}(t_n\hm+0)$~--- 
соответствующее значение непосредственно после очередного момента изменения\linebreak 
состояния~$t_n$, $n\hm=0,1,2,\ldots$; $\widetilde{v}_0\hm=v_0$~--- 
заданное начальное значение в~момент $t\hm=0$. Рассмотрим величину
\begin{multline}
d_i(u)=\mathbf{E}\left[\widetilde{v}_{n+1}-\widetilde{v}_n \mid \xi_n=i\,, 
u_n=u\right]\,,\\
u\in U_i\,, \enskip i\in X\,, \label{e8}
\end{multline}
представляющую собой математическое ожидание приращения стоимостного 
аддитивного функционала за период времени между последовательными 
изменениями состояния полумарковского процесса~$\xi(t)$. Тогда соответствующее 
математическое ожидание, вычисляемое без условия на решение, 
принимаемое в~момент времени~$t_n$, представляется в~виде:
\begin{equation*}
d_i=\mathbf{E}\left[\widetilde{v}_{n+1}-\widetilde{v}_n \mid \xi_n=i\right]=
\!\int\limits_{U_i}\!d_i(u)\,d\Psi_i(u)\,,\ i\in X \,. %\label{e9}
\end{equation*}

Предположим, что для заданной стратегии управ\-ле\-ния 
$\Psi\hm=(\Psi_1,\Psi_2,\ldots,\Psi_N)$ вложенная цепь Маркова~$\{\xi_n\}$ 
имеет ровно один класс возвратных положительных состояний (по терминологии, 
принятой в~\cite{8}, такое множество состояний называется эргодическим классом). 
Как известно~\cite[гл.~VIII]{15}, данное условие является необходимым 
и~достаточным для существования единственного\linebreak стационарного распределения. 
Обозначим это стационарное распределение цепи Маркова~$\{\xi_n\}$ через 
$\pi\hm=(\pi_1, \pi_2,\ldots, \pi_N)$. Заметим, что данное\linebreak распределение зависит  
от стратегии управления $\Psi\hm=(\Psi_1,\Psi_2,\ldots,\Psi_N)$. При указанном 
условии имеет место следующий результат, называемый эргодической теоремой 
для аддитивного стоимостного функционала:
\begin{equation}
I=\lim\limits_{t\rightarrow\infty}\fr{\mathbf{E}\widetilde{v}(t)}{t}=
\fr{\sum\nolimits_{i=1}^N d_i\pi_i}{\sum\nolimits_{i=1}^N T_i\pi_i}\,. 
\label{e10}
\end{equation}

Соотношение~(\ref{e10}) доказано в~работе~\cite[гл.~5]{8}. Заметим, что аналогичные 
результаты имеют мес\-то для гораздо более общих полумарковских моделей~\cite{10, 11}.

По своему прикладному содержанию величина, определяемая соотношением~(\ref{e10}), 
представляет собой
среднюю удельную прибыль, связанную с~эволюцией системы в~стационарном
режиме. Кроме того, величина~$I$ представляет собой функционал от
набора вероятностных распределений~$\Psi_{i}$, $i\hm\in\lbrace 1,\ldots
,N\rbrace $, определяющих стратегию управле-\linebreak\vspace*{-12pt}

\pagebreak

\noindent
ния системой. 
В~дальнейшем будем рассматривать стационарный стоимостной функционал 
$I\hm=I(\Psi_{1},\Psi_{2},\ldots , \Psi_{N})$ как
показатель качества управ\-ле\-ния системой и~построенным полумарковским
процессом~$\xi (t)$.

\section{Представление стационарного показателя в~форме
дробно-линейного интегрального функционала}

В данном разделе будет приведено утверждение об аналитическом
представлении стационарного стоимостного функционала~(\ref{e10}), 
служащего критерием качества управления в~рассматриваемой задаче управления 
полумарковским процессом.

\smallskip

\noindent
\textbf{Теорема 1.} \textit{Стационарный стоимостной показатель, 
определяемый равенством}~(\ref{e10}), \textit{представляет собой дроб\-но-ли\-ней\-ный
функционал от вероятностных распределений~$\Psi_{i}(u_{i})$,
$i\hm\in\{1,\dots,N\}$. Данный функционал задается
аналитически следующей формулой:}
\begin{multline}
I=I(\Psi_{1},\ldots, \Psi_{N})={}\\
\hspace*{-2mm}{}=\!
\fr{\int\nolimits_{U_1}\!{\cdots\! 
\int\nolimits_{U_N}\!{A(u_{1},\ldots ,u_{N})d\Psi_{1}(u_{1})\cdots
\,d\Psi_{N}(u_{N})}}}{\int\nolimits_{U_1}{\!\cdots\! \int\nolimits_{U_N}\!{B(u_{1},\ldots ,u_{N})
\,d\Psi_{1}(u_{1})\ldots
d\Psi_{N}(u_{N})}}},\!\!\! \label{e11}
\end{multline}
\textit{где подынтегральные функции числителя и~знаменателя выражаются
соотношениями}:
\begin{align}
A(u_{1},\ldots
,u_{N})&={}\notag\\
&\hspace*{-20mm}{}=\sum\limits_{i=1}^{N}{d_{i}(u_{i})}{\widehat{D}}^{(i)}(u_{1}, \ldots
,u_{i-1},u_{i+1}, \ldots , u_{N})\,;  \label{e12}\\
 B(u_{1},\ldots
,u_{N})&={}\notag\\
&\hspace*{-20mm}{}=\sum\limits_{i=1}^{N}{T_{i}(u_{i})}{\widehat{D}}^{(i)}(u_{1}, \ldots
,u_{i-1},u_{i+1}, \ldots , u_{N})\,.  \label{e13}
\end{align}
\textit{В свою очередь, функции} ${\widehat{D}}^{(i)}(u_{1}, \ldots
,u_{i-1},u_{i+1}, \ldots$\linebreak $\ldots , u_{N})$, $i\hm\in\{1,\dots,N\}$, 
\textit{входящие в~правые части формул}~(\ref{e12}) и~(\ref{e13}), 
\textit{определяются следующим образом:}

\noindent
\begin{multline}
{\widehat{D}}^{(i)}(u_{1}, \ldots ,u_{i-1},u_{i+1}, \ldots , u_{N})={}
\\
{}=(-1)^{N+i+2}\sum\limits_{\alpha ^{(N),i}}{(-1)}^{\delta (\alpha
^{(N),i}) }{\widehat{D}}_{0}^{(i)}\left(\alpha ^{(N),i},u_{1}, \ldots\right.\\
\left.\ldots , u_{i-1},u_{i+1}, \ldots , u_{N}\right)\,. \label{e14}
\end{multline}
\textit{Здесь} $\alpha ^{(N),i}=(\alpha _{1}, \ldots , \alpha _{i-1},\alpha_{i+1}, \ldots , 
\alpha _{N})$~\textit{--- произвольная
перестановка чисел }$(1, \ldots , i-1, i+1, \ldots , N)$;
$\delta
(\alpha ^{(N),i})$~\textit{--- число инверсий в~перестановке} 
$\alpha ^{(N),i}$;

\noindent
\begin{multline}
{\widehat{D}}_{0}^{(i)}(\alpha ^{(N),i},u_{1}, \ldots ,u_{i-1},u_{i+1},
\ldots , u_{N})={}\\
{} ={\widetilde{p}}_{1,\alpha _{1}}\left(u_{1}\right)\cdots {\widetilde{p}}_{i-1,\alpha
_{i-1}}\left(u_{i-1}\right){\widetilde{p}}_{i+1,\alpha _{i+1}}\left(u_{i+1}\right)\cdots\\
\cdots
{\widetilde{p}}_{N,\alpha _{N}}\left(u_{N}\right)\,, 
\label{e15}
\end{multline}
где
\begin{multline}
 {\widetilde{p}}_{k,\alpha _{k}}(u_{k})=
\begin{cases}
p_{kk}(u_{k})-1,\  & \alpha _{k}=k\,; \\
p_{k,\alpha _{k}}(u_{k}),\  & \alpha _{k}\ne k, \\
\end{cases}\\
 k=1, \ldots , i-1, i+1, \ldots ,N\,. \label{e16}
 \end{multline}
\textit{Функции $p_{ij}(u_i)$, $T_{i}(u_{i})$ и~$d_{i}(u_{i})$,
$u_i\hm\in U_i$, $i,j\hm\in \{1,2,\ldots,N\}$, 
входящие в~соотношения}~(\ref{e12})--(\ref{e16}), 
\textit{определяются равенствами}~(\ref{e2}), (\ref{e4}) \textit{и}~(\ref{e8}) \textit{соответственно.}

\smallskip

\noindent
Д\,о\,к\,а\,з\,а\,т\,е\,л\,ь\,с\,т\,в\,о\ теоремы~1 
в~весьма сжатой форме приведено в~работе~\cite{21}. Читателю, интересующемуся 
более подробным обоснованием данного результата, порекомендуем обратиться к~тексту 
кандидатской диссертации А.\,В.~Иванова~\cite[гл.~3]{22}.

\smallskip

Итак, теорема~1 позволяет получить явное аналитическое представление 
для стационарного стоимостного показателя вида~(\ref{e10}) в~форме 
дроб\-но-ли\-ней\-но\-го интегрального функционала от набора\linebreak вероятностных мер 
$\Psi\hm=(\Psi_{1},\Psi_{2},\ldots , \Psi_{N})$, за\-да\-ющих стратегию управления 
полумарковским процессом~$\xi(t)$. При этом подынтегральные функции числителя 
и~знаменателя задаются формулами~(\ref{e12}), (\ref{e13}) 
и~вспомогательными равенствами~(\ref{e14})--(\ref{e16}). Таким образом, функция
\begin{equation}
C\left(u_1, u_2,\ldots, u_N\right)=\fr{A(u_1, u_2,\ldots, u_N)}{B(u_1, u_2,\ldots, u_N)}\,,
\label{e17}
\end{equation}
которая в~дальнейшем будет называться основной функцией дроб\-но-ли\-ней\-но\-го 
интегрального функционала~(\ref{e11}) и~которая будет играть важную роль 
в~дальнейшем исследовании, также явно определяется формулами~(\ref{e17}), 
(\ref{e12}), (\ref{e13}).

\section{Формальная постановка оптимизационной задачи 
и~условия существования оптимальной стратегии управления полумарковским процессом}

Будем рассматривать проблему управления полумарковским процессом~$\xi(t)$ в~форме 
экстремальной задачи
\begin{multline}
I(\Psi)=I\left(\Psi_1, \Psi_2,\ldots,\Psi_N\right)\rightarrow \mathrm{extr}\,,
\\
\Psi=\left(\Psi_1, \Psi_2,\ldots,\Psi_N\right)\in\Gamma\,. \label{e18}
\end{multline}
При этом показатель качества управления~$I(\Psi)$ представляет собой 
дроб\-но-ли\-ней\-ный интегральный функционал вида~(\ref{e11}).

Для решения экстремальной задачи~(\ref{e18}) воспользуемся некоторым утверждением 
об экстремуме дроб\-но-ли\-ней\-но\-го интегрального функционала. Прежде 
чем сформулировать данное утверждение, отметим, что в~теории оптимизации 
хорошо известны задачи, в~которых целевая функция представляет собой 
отношение двух линейных отображений, а имеющиеся ограничения также линейны. 
Такой раздел называется дроб\-но-ли\-ней\-ным программированием. Основные
 теоретические результаты данного направления изложены в~работе~\cite{23},
  там же приведена подробная библиография. В~дальнейшем потребуется некоторый 
  специальный результат о безусловном экстремуме дроб\-но-ли\-ней\-но\-го 
  интегрального функционала вида~(\ref{e11}), который был впервые сформулирован 
  в~работе~\cite{14}. Заметим, что для использования этого результата необходимо, 
  чтобы выполнялись некоторые предварительные условия, которые в~данном случае 
  можно сформулировать следующим образом:
\begin{enumerate}[1.]
\item Интегральные выражения
\begin{align*}
I_1(\Psi)&=I_1\left(\Psi_1,\Psi_2,\ldots,\Psi_N\right)={}&\\
&\hspace*{-13mm}{}=\int\limits_{U_1}\!\cdots\!
\int\limits_{U_N}\!\!A\left(u_1,\ldots ,u_N\right)\,
d\Psi_1\left(u_1\right) %d\Psi_2\left(u_2\right)
\cdots
 d\Psi_N\left(u_N\right)\,;
\\
I_2(\Psi)&=I_2\left(\Psi_1,\Psi_2,\ldots,\Psi_N\right)={}&\\
&\hspace*{-13mm}{}=\int\limits_{U_1}\!\cdots\!\int\limits_{U_N}\!\!
B\left(u_1,\ldots,u_N\right)\,
d\Psi_1\left(u_1\right)% d\Psi_2\left(u_2\right)\cdots\\
\cdots d\Psi_N\left(u_N\right)
\end{align*}
определены для всех стратегий управления $\Psi\hm=(\Psi_1, \ldots,\Psi_N)
\hm\in \Gamma$.

\item Функционал $I_2(\Psi)=I_2(\Psi_1, \ldots,\Psi_N)\hm\neq 0$ 
для всех $\Psi\hm=(\Psi_1, \ldots,\Psi_N)\hm\in \Gamma$.

\item Множество $\Gamma$ включает в~себя множество всех вырожденных 
вероятностных мер: $\Gamma^* \hm\subset \Gamma$.
\end{enumerate}

Сделаем несколько важных замечаний по поводу введенных предварительных условий.

\smallskip

\noindent
\textbf{Замечание~1.}\ Из условия~2 следует, что функция $B(u_1, u_2,\ldots, u_N)$ 
не может принимать значения разных знаков. С~учетом условия~3 
получаем, что указанная функция должна обладать \mbox{свойством} строгой 
знакопостоянности на всем множестве~$U$. С~другой стороны, если выполняется 
условие строгой знакопостоянности функции $B(u_1, u_2,\ldots, u_N), 
(u_1, u_2,\ldots, u_N)\hm\in U$, то условие~2 выполняется автоматически.

\smallskip

\noindent
\textbf{Замечание~2.}\ Если рассматривать в~качестве целевого функционала 
$I(\Psi_1, \Psi_2,\ldots,\Psi_N)$ экстремальной задачи~(\ref{e18}) 
стационарный стоимостной пока\-затель~(\ref{e10}), то функция $B(u_1,u_2,\ldots,u_N)$ 
имеет\linebreak следующее теоретическое содержание. Данная функция представляет собой условное 
математическое ожидание длительности периода времени между соседними моментами 
изменения со\-сто\-яния полумарковского процесса~$\xi(t)$ при условии, что стратегия 
его управ\-ле\-ния является детерминированной и~задается набором значений аргументов 
$(u_1,u_2,\ldots,u_N)$. Тогда условие строгой положительности функции 
$B(u_1,u_2,\ldots,u_N)$ при всех $(u_1,u_2,\ldots,u_N)\hm\in U$ является естественным 
и~фактически означает, что при любой заданной детерминированной стратегии 
управ\-ле\-ния процесс~$\xi(t)$ не имеет мгновенных со\-сто\-яний, длительность пребывания 
в~которых равна нулю.

\smallskip

\noindent
\textbf{Замечание~3.}\ Сделаем некоторые замечания, связан\-ные с~подынтегральной 
функцией числителя дроб\-но-ли\-ней\-но\-го интегрального функционала~(\ref{e11}). 
Как и~ранее, будем рассматривать в~качестве целевого функционала $I(\Psi_1, \Psi_2,\ldots,\Psi_N)$\linebreak 
экстремальной задачи~(\ref{e18}) стационарный стоимостной показатель~(\ref{e10}). 
Тогда для любого фиксированного набора значений аргументов $(u_1,u_2,\ldots,u_N)\hm\in U$ 
значение функции $A(u_1,u_2,\ldots\linebreak \ldots,u_N)$ представляет собой условное математическое
 ожидание приращения рассматриваемого стоимостного функционала, 
 происшедшее за время пребывания полумарковского процесса~$\xi(t)$ в~некотором 
 фиксированном  состоянии при условии, что стратегия управления является 
 детерминированной и~задается указанным набором $(u_1,u_2,\ldots,u_N)\hm\in U$. 
 Отметим, что в~теореме об экстремуме дроб\-но-ли\-ней\-но\-го интегрального 
 функционала, доказанной в~работе~\cite[гл.~10]{12}, 
 на подынтегральную функцию числителя накладываются условия ограниченности на 
 всем множестве значений аргумента. Для многих математических моделей и~связанных 
 с~ними задач оптимального управления такое условие является излишне ограничительным. 
 В~качестве примера можно привести модели оптимального управления запасом непрерывного 
 продукта, рассмотренные в~работах~\cite{27, 28}. 
 В~настоящем исследовании на функцию $A(u_1,u_2,\ldots,u_N)$ накладывается только 
 условие интегрируемости по любому заданному набору вероятностных мер 
 $\Psi\hm=(\Psi_1, \Psi_2,\ldots,\Psi_N)$, образующему стратегию управления 
 полумарковским процессом~$\xi(t)$ (условие~1 системы предварительных условий).

\smallskip

\noindent
\textbf{Замечание~4.} Условия~1--3 являются необходимыми для корректной 
постановки задачи безусловного экстремума дроб\-но-ли\-ней\-но\-го интегрального 
функционала. Если этот функционал служит показателем качества в~задаче оптимального 
управления случайным процессом, то необходимо добавить к~этим условиям дополнительное, 
связанное с~некоторой регулярностью самого управляемого процесса, а~именно: некоторый 
содержательный показатель, связанный с~поведением этого процесса, должен существовать 
и~быть представимым в~виде дроб\-но-ли\-ней\-но\-го интегрального функционала. 
Если потребовать, чтобы выполнялось эргодическое соотношение~(\ref{e10}), 
то можно использовать\linebreak теорему~1 и~сформулировать задачу оптимального управ\-ле\-ния 
в~виде~(\ref{e18}) для дроб\-но-ли\-ней\-но\-го\linebreak интегрального функционала~(\ref{e11}). 
Таким образом, необходимо ввести условие, обеспечивающее существование единственного 
стационарного распределения вложенной цепи Маркова и~выполнение\linebreak соотношения~(\ref{e10}). 
По аналогии с~[8, гл.~5] сформулируем это дополнительное условие в~следующем виде:
\begin{enumerate}
\setcounter{enumi}{3}
\item Для любой рассматриваемой стратегии управ\-ле\-ния $\Psi\hm=
(\Psi_1, \Psi_2,\ldots,\Psi_N)\hm\in \Gamma$ вложенная цепь Маркова 
полумарковского процесса $\xi(t)$ имеет ровно один класс возвратных 
положительных состояний.
\end{enumerate}

Теперь определим понятие допустимой стратегии управления полумарковским процессом 
с~конечным множеством состояний.

\smallskip

\noindent
\textbf{Определение~2.}\ Назовем стратегию управления 
$\Psi\hm=(\Psi_1, \Psi_2,\ldots,\Psi_N)$ 
допустимой в~данной задаче, если она удовлетворяет условиям~1--4.


\smallskip

\noindent
\textbf{Замечание~5.}\ Как следует из замечания~1, если потребовать, 
чтобы функция $B(u_1, u_2,\ldots,u_N)$ являлась строго знакопостоянной при 
всех $(u_1, u_2,\ldots,u_N)\hm\in U$, то можно считать допустимыми стратегии 
$(\Psi_1, \Psi_2,\ldots,\Psi_N)$, удовлетворяющие условиям~1, 3, 4. С~учетом замечания~2 
о~естественном характере условия строгой знакопостоянности функции $B(u_1,u_2,\ldots,u_N)$ 
при всех значениях аргументов $(u_1, u_2,\ldots,u_N)\hm\in U$ будем требовать 
выполнения этого условия в~формулировке приводимой в~дальнейшем основной 
теоремы об оптимальной стратегии управления полумарковским процессом.

\smallskip

\noindent
\textbf{Замечание~6.}\ Ниже будет сформулирована и~доказана основная 
теорема об оптимальной стра\-тегии управления полумарковским процессом с~конеч\-ным 
множеством состояний. Будем формулировать эту теорему по отношению к~экстремальной 
задаче~(\ref{e18}), в~которой целевой функционал $I(\Psi_1, \Psi_2,\ldots,\Psi_N)$ 
имеет вид дроб\-но-ли\-ней\-но\-го интегрального функционала. 
Это обстоятельство связано с~тем, что целевой функционал в~задаче 
оптимального управления необязательно должен иметь характер стационарного 
стоимостного показателя вида~(\ref{e10}). В~частности, еще в~1983~г.\ П.\,В.~Шнурковым 
было установлено~\cite{24}, что ряд показателей, связанных 
с~временем пребывания управляемого полумарковского процесса в~заданном конечном 
подмножестве состояний, имеет структуру дроб\-но-ли\-ней\-но\-го интегрального 
функционала от набора вероятностных мер, определяющих стратегию управления. 
Таким образом, рассматриваемая задача управления имеет более общий характер, 
чем задача, в~которой целевой функционал представляет собой стационарный 
стоимостной показатель вида~(\ref{e10}).






\smallskip

\noindent
\textbf{Замечание~7.}\ Если рассматривать задачу оптимального управления 
полумарковским процессом, в~кото\-рой целевой функционал не совпадает 
со стационарным стоимостным показателем~(\ref{e10}), то возможно, что могут 
потребоваться другие дополнительные условия, обеспечивающие существование этого 
показателя и~его представление в~форме~(\ref{e11}). В~связи с~этим в~формулировке 
основной теоремы будем использовать термин допустимые стратегии в~широком смысле, 
имея в~виду выполнение всех необходимых условий для каждого рассмат\-ри\-ва\-емо\-го 
показателя качества управления.

\smallskip


\noindent
\textbf{Замечание 8.} Множество допустимых стратегий может 
не совпадать с~множеством всех возможных стратегий управления. 
В~частности, допустимые стратегии могут состоять только из дискретных вероятностных 
мер $\Psi_1, \Psi_2,\ldots,\Psi_N$, т.\,е.\ таких, которые сосредоточены на дискретных 
множествах точек пространств $U_1, U_2,\ldots,U_N$.

\section{Теоретическое решение задачи оптимального управления}

Перейдем к~формулировке и~доказательству тео\-ре\-мы об 
оптимальной стратегии управ\-ле\-ния полумарковским процессом с~конечным 
множеством состояний.

\smallskip

\noindent
\textbf{Теорема~2.} \textit{Рассмотрим проблему оптимального управ\-ле\-ния 
полумарковским процессом~$\xi(t)$ в~виде экстремальной задачи}~(\ref{e18}), 
\textit{определенной на множестве допустимых стратегий $\Gamma$, 
для дроб\-но-ли\-ней\-но\-го 
функционала}~(\ref{e11}). \textit{Пусть функция $B(u_1,u_2,\ldots,u_N)$, 
входящая в~определение функционала}~(\ref{e11}),
\textit{является строго знакопостоянной (строго положительной или строго отрицательной) 
при всех значениях аргументов $(u_1,u_2,\ldots,u_N)\hm\in U$.
Тогда справедливы сле\-ду\-ющие утверждения}:
\begin{enumerate}[1.]
\item \textit{Если функция} $C(u_1,u_2,\ldots,u_N)\hm=A(u_1,u_2,\ldots$\linebreak
$\ldots,u_N)/{B(u_1,u_2,\ldots,u_N)}$ 
\textit{ограничена сверху или снизу и~достигает глобального экст\-ре\-му\-ма на множестве
$U\hm=U_1\times U_2\times \cdots \times U_N$ (максимума или минимума), 
то оптимальная стратегия управления полумарковским процессом~$\xi(t)$ существует, 
является детерминированной и~определяется
вырожденной вероятностной мерой $\Psi^*\hm\in \Gamma^*$, сосредоточенной в~точке, 
в~которой достига\-ет соответствующего экстремума функция $C(u_1,u_2,\ldots,u_N)$,
и~при этом выполняются соотношения}:
\begin{multline}  %{\substack{{i=\overline{1,n}}\\ {j=\overline{1,l}}}}
\max\limits_{\Psi \in \Gamma} I(\Psi)=
\max\limits_{\substack{{\Psi_i \in \Gamma_i\,,}\\ 
{i=\overline{1,N}}}}
I\left(\Psi_1,\Psi_2,\ldots,\Psi_N\right)={}\\
{}=
\max\limits_{\substack{{\Psi_i^* \in \Gamma_i^*,}\\ 
{i=\overline{1,N}}}}
 I\left(\Psi_1^*,\Psi_2^*,\ldots,\Psi_N^*\right)={}\\
{}=\max\limits_{(u_1,u_2,\ldots,u_N)\in U}\fr{A(u_1,u_2,\ldots,u_N)}
{B(u_1,u_2,\ldots,u_N)}\,; \label{e19}
\end{multline}

\vspace*{-12pt}

\noindent
\begin{multline*}
\min\limits_{\Psi \in \Gamma} I(\Psi)=
\min\limits_{\substack{{\Psi_i \in \Gamma_i\,,}\\ 
{i=\overline{1,N}}}} I\left(\Psi_1,\Psi_2,\ldots,\Psi_N\right)={}\\
{}=
\min\limits_{\substack{{\Psi_i^* \in \Gamma_i^*,}\\ 
{i=\overline{1,N}}}}
I\left(\Psi_1^*,\Psi_2^*,\ldots,\Psi_N^*\right)={}\\
{}=\min\limits_{(u_1,u_2,\ldots,u_N)\in U}\fr{A(u_1,u_2,\ldots,u_N)}
{B(u_1,u_2,\ldots,u_N)}\,. %\label{e20}
\end{multline*}
\item \textit{Если функция $C(u_1,u_2,\ldots,u_N)\hm=
{A(u_1,u_2,\ldots,u_N)}/{B(u_1,u_2,\ldots,u_N)}$ ограничена сверху или снизу, 
но не достигает глобального экстремума на множестве $U\hm=U_1\times U_2\times\cdots
\times U_N$,
то для любого $\varepsilon\hm > 0$ можно выбрать $\varepsilon$-оп\-ти\-маль\-ную 
детерминированную стратегию управления полумарковским процессом~$\xi(t)$, 
которая определяется вырожденной
вероятностной мерой $\Psi^{*(+)}(\varepsilon)\hm\in \Gamma^*$ или вырожденной
вероятностной мерой $\Psi^{*(-)}(\varepsilon)\hm\in \Gamma^*$, в~зависимости от 
вида экстремума (максимума или минимума) в~задаче}~(\ref{e18}). 
\textit{При этом вероятностная мера $\Psi^{*(+)}(\varepsilon)\hm\in \Gamma^*$ может быть 
сосредоточена в~любой точке $\left(u_1^{(+)}(\varepsilon),u_2^{(+)}(\varepsilon),\ldots,
u_N^{(+)}(\varepsilon)\right)$, удовлетворяющей соотношению}:
\begin{multline}
\sup\limits_{(u_1,u_2,\ldots,u_N) \in U}
\fr{A(u_1,u_2,\ldots,u_N)}{B(u_1,u_2,\ldots,u_N)}-\varepsilon <{}\\
{}<
\fr{A\left(u_1^{(+)}(\varepsilon),u_2^{(+)}(\varepsilon),\ldots,u_N^{(+)}
(\varepsilon)\right)}
{B\left(u_1^{(+)}(\varepsilon),u_2^{(+)}(\varepsilon),\ldots,u_N^{(+)}
(\varepsilon)\right)}<{}\\
{}<\sup\limits_{(u_1,u_2,\ldots,u_N) \in U}
\fr{A(u_1,u_2,\ldots,u_N)}{B(u_1,u_2,\ldots,u_N)}<\infty\,, 
\label{e21}
\end{multline}
\textit{если функция $C(u_1,u_2,\ldots,u_N)$ ограничена сверху 
и~экстремальная задача}~(\ref{e18}) 
\textit{представляет собой задачу на максимум. Аналогично вероятностная мера 
$\Psi^{*(-)}(\varepsilon)\hm\in \Gamma^*$ может быть сосредоточена в~любой точке 
$\left(u_1^{(-)}(\varepsilon),u_2^{(-)}(\varepsilon),\ldots,u_N^{(-)}(\varepsilon)
\right)$, удовлетворяющей соотношению}:

\noindent
\begin{multline*}
-\infty<\inf\limits_{(u_1,u_2,\ldots,u_N) \in U}\fr{A(u_1,u_2,\ldots,u_N)}
{B(u_1,u_2,\ldots,u_N)} <{}\\
{}<
\fr{A\left(u_1^{(-)}(\varepsilon),u_2^{(-)}
(\varepsilon),\ldots,u_N^{(-)}(\varepsilon)\right)}
{B\left(u_1^{(-)}(\varepsilon),u_2^{(-)}(\varepsilon),\ldots,
u_N^{(-)}(\varepsilon)\right)}<{}\\
{}<\inf\limits_{(u_1,u_2,\ldots,u_N) \in U}
\fr{A(u_1,u_2,\ldots,u_N)}{B(u_1,u_2,\ldots,u_N)}+\varepsilon\,, 
%\label{e22}
\end{multline*}
\textit{если функция $C(u_1,u_2,\ldots,u_N)$ ограничена снизу и~экстремальная 
задача}~(\ref{e18})  \textit{представляет собой задачу на минимум}.
\item \textit{Если функция $C(u_1,u_2,\ldots,u_N)\hm=
{A(u_1,u_2,\ldots,u_N)}/{B(u_1,u_2,\ldots,u_N)}$ не ограничена сверху 
или снизу, то оптимальной стратегии управления в~смысле
соответствующей экстремальной задачи не существует. 
При этом найдется такая последовательность вырожденных вероятностных
мер~$\Psi^{*(+)}(n)$, сосредоточенных в~точках 
$\left(u_1^{(+)}(n),u_2^{(+)}(n),\ldots,u_N^{(+)}(n)\right)$, $n\hm=1,2,\dots $, 
для которых выполняется соотношение}:
\begin{multline*}
I\left(\Psi^*(n)\right)={}\\
{}=
I\left(\Psi_1^{*(+)}(n),\Psi_2^{*(+)}(n),\ldots,\Psi_N^{*(+)}(n)\right)={}\\
{}=\fr{A\left(u_1^{(+)}(n),u_2^{(+)}(n),\ldots,u_N^{(+)}(n)\right)}
{B\left(u_1^{(+)}(n),u_2^{(+)}(n),\ldots,u_N^{(+)}(n)\right)}\to 
\infty\\
\mbox{при}\ n\to\infty\,, 
%\label{e23}
\end{multline*}
\textit{если функция $C(u_1,u_2,\ldots,u_N)$ не ограничена сверху. 
Аналогично найдется такая последовательность вырожденных вероятностных
мер~$\Psi^{*(-)}(n)$, сосредоточенных в~точках 
$\left(u_1^{(-)}(n),u_2^{(-)}(n),\ldots,u_N^{(-)}(n)\right)$, 
$n\hm=1,2,\dots $, для которых выполняется соотношение}:
\begin{multline*}
I\left(\Psi^{*(-)}(n)\right)={}\\
{}= I
\left(\Psi_1^{*(-)}(n),\Psi_2^{*(-)}(n),\ldots,\Psi_N^{*(-)}(n)\right)={}\\
{}=\fr{A\left(u_1^{(-)}(n),u_2^{(-)}(n),\ldots,u_N^{(-)}(n)\right)}
{B\left(u_1^{(-)}(n),u_2^{(-)}(n),\ldots,u_N^{(-)}(n)\right)}\to 
-\infty\\
\mbox{при}~~n\to\infty\,,  
%\label{e24}
\end{multline*}
\textit{если функция $C(u_1,u_2,\ldots,u_N)$ не ограничена \mbox{снизу}}.
\end{enumerate}
\textit{При этом сформулированные утверждения каждого пункта теоремы~$2$ 
могут выполняться как по отдельности, для одного из двух
видов экстремума, так и~совместно, для обоих видов экстремума.}

\smallskip

Прежде чем непосредственно доказывать теорему~2, докажем некоторые 
вспомогательные утверждения.

\smallskip

\noindent
\textbf{Лемма~1.}\ 
\textit{Рассмотрим дроб\-но-ли\-ней\-ный интегральный функционал 
$I(\Psi_1, \Psi_2,\ldots, \Psi_N)$ вида}~(\ref{e11}), 
\textit{заданный на некотором множестве наборов вероятностных мер 
$\Psi\hm=(\Psi_1, \Psi_2,\ldots, \Psi_N)\hm \in \Gamma$. Предположим, что на 
множестве~$\Gamma$ выполняется условие~$1$ из набора предварительных условий 
и~функция $B(u_1, u_2,\ldots, u_N)$  обладает свойством строгой знакопостоянности 
при всех $(u_1, u_2,\ldots, u_N) \hm\in U$. Тогда справедливы следующие утверждения}:
\begin{enumerate}[1.]
\item \textit{Если основная функция 
$C(u_1, u_2,\ldots, u_N)\hm={A(u_1, u_2,\ldots, u_N)}/{B(u_1, u_2,\ldots, u_N)}$ 
ограничена сверху, т.\,е.\ выполняется условие}
\begin{multline}
C\left(u_1, u_2,\ldots, u_N\right)=
\fr{A(u_1, u_2,\ldots, u_N)}{B(u_1, u_2,\ldots, u_N)}\leq {}\\
{}\leq
c_0^{(+)}<\infty \,, \enskip \left(u_1, u_2,\ldots, u_N\right) \in U\,, \label{e25}
\end{multline}
\textit{то имеет место неравенство}:
\begin{equation}
I\left(\Psi_1, \Psi_2,\ldots, \Psi_N\right)\leq c_0^{(+)} 
\label{e26}
\end{equation}
\textit{для всех} $(\Psi_1, \Psi_2,\ldots, \Psi_N) \in \Gamma$.
\item \textit{Если основная функция 
$C(u_1, u_2,\ldots, u_N)\hm={A(u_1, u_2,\ldots, u_N)}/{B(u_1, u_2,\ldots, u_N)}$ 
ограничена снизу, т.\,е.\ выполняется условие}
\begin{multline*}
C\left(u_1, u_2,\ldots, u_N\right)=\fr{A(u_1, u_2,\ldots, u_N)}{B(u_1, u_2,\ldots, 
u_N)}\geq{}\\
{}\geq c_0^{(-)}>-\infty \,, 
\left(u_1, u_2,\ldots, u_N\right) \in U\,, 
%\label{e27}
\end{multline*}
\textit{то имеет место неравенство}:
\begin{equation*}
I\left(\Psi_1, \Psi_2,\ldots, \Psi_N\right)\geq c_0^{(-)} 
%\label{e28}
\end{equation*}
\textit{для всех} $(\Psi_1, \Psi_2,\ldots, \Psi_N) \hm\in \Gamma$.
\end{enumerate}

\noindent
Д\,о\,к\,а\,з\,а\,т\,е\,л\,ь\,с\,т\,в\,о\ \ леммы~1.\ 
Докажем первое утверждение леммы. Предположим сначала, 
что функция $B(u_1, u_2,\ldots,  u_N)$ строго положительна:
\begin{equation}
B\left(u_1, u_2,\ldots, u_N\right)>0\,,\enskip
\left(u_1, u_2,\ldots, u_N\right)\in U\,. \label{e29}
\end{equation}
Заметим, что в~таком случае по свойству интеграла~\cite[гл.~V]{18}
\begin{multline}
\hspace*{-2mm}\int\limits_{U_1}\!\!\cdots\! \!\int\limits_{U_N}\!\!B(u_1, \ldots,u_N) \,
d\Psi_1(u_1)%d\Psi_2(u_2)\cdots\\
\cdots d\Psi_N(u_N)>0 \!\!\!\!\label{e30}
\end{multline}
для любого фиксированного набора $\Psi\hm=(\Psi_1, \ldots, \Psi_N)\hm\in \Gamma$.
Из неравенства~(\ref{e25}) с~уче\-том~(\ref{e29}) получаем:
\begin{multline}
\hspace*{-4mm}A\left(u_1,\ldots, u_N\right)\leq{}\\
\hspace*{-4mm}{}\leq c_0^{(+)} B\left(u_1, \ldots, u_N\right)\,, 
\left(u_1, \ldots, u_N\right)\in U\,. \label{e31}
\end{multline}
В свою очередь, из неравенства~(\ref{e31}) и~свойств интеграла следует:
\begin{multline}
\int\limits_{U_1}\!\!\cdots\! \!\int\limits_{U_N}\!\!A(u_1,\ldots, u_N) \,
d\Psi_1\left(u_1\right)%d\Psi_2\left(u_2\right)\cdots\\
\cdots d\Psi_N\left(u_N\right)\leq\\
\hspace*{-24pt}\leq 
c_0^{(+)}\!\!\int\limits_{U_1}\!\!\cdots\!\! \int\limits_{U_N}\!\!\!B\!\left(u_1,\ldots, u_N\right)
 d\Psi_1\!\left(u_1\right)\!%d\Psi_2\left(u_2\right)\cdots\\
 \cdots d\Psi_N\!\left(u_N\right)\!\! 
 \label{e32}
\end{multline}
для любого фиксированного набора $\Psi\hm=(\Psi_1, \ldots, \Psi_N)\hm\in \Gamma$. 
Но тогда из~(\ref{e32}) с~учетом~(\ref{e30}) получаем:
\begin{multline}
I(\Psi_1, \ldots, \Psi_N)={}\\
{}=
\fr{\int\nolimits_{U_1}\!\cdots\! \int\nolimits_{U_N}\!\!A\left(u_1, \ldots, u_N\right)\,
 d\Psi_1(u_1)\cdots d\Psi_N(u_N)}{
\int\nolimits_{U_1}\!\cdots\! \int\nolimits_{U_N}\!\!B\left(u_1, \ldots, u_N\right)\,
 d\Psi_1(u_1)
 \cdots d\Psi_N(u_N)}\leq{}\\
 {}\leq c_0^{(+)} 
 \label{e33}
\end{multline}
для любого фиксированного набора $(\Psi_1, \ldots\linebreak\ldots, \Psi_N)\hm\in \Gamma$.

Предположим теперь, что функция $B(u_1,\ldots, u_N)$ строго отрицательна:
\begin{equation}
B(u_1,\ldots, u_N)<0 \quad \left(u_1, \ldots, u_N\right)\in U\,. 
\label{e34}
\end{equation}
Тогда
\begin{multline}
\hspace*{-6pt}\int\limits_{U_1}\!\!\cdots\!\! \int\limits_{U_N}\!\!B\!\left(u_1,\ldots, u_N\right)\!
 d\Psi_1(u_1) \cdots d\Psi_N(u_N)<0 \!\!\!
 \label{e35}
\end{multline}
для любого фиксированного набора $(\Psi_1, \ldots\linebreak \ldots, \Psi_N)\hm\in \Gamma$.

Как и~ранее, будем исходить из неравенства~(\ref{e25}). 
При выполнении условий~(\ref{e34}) и~(\ref{e35}) характер неравенств~(\ref{e31}) 
и~(\ref{e32}) меняется на противоположный, но характер неравенства~(\ref{e33}) 
остается неизменным. Таким образом, для любой функции 
$B(u_1, u_2,\ldots, u_N)$, обладающей свойством строгой знакопостоянности, 
из условия~(\ref{e25}) следует выполнение неравенства~(\ref{e33}), 
которое совпадает с~(\ref{e26}). Первое утверждение леммы~1 доказано. 
Второе утверждение доказывается аналогично. Лемма~1 доказана.

\smallskip

\noindent
\textbf{Лемма 2.} \textit{Рассмотрим дроб\-но-ли\-ней\-ный интегральный функционал 
$I(\Psi_1, \Psi_2,\ldots, \Psi_N)$ вида}~(\ref{e11}), 
\textit{заданный на некотором множестве наборов вероятностных мер 
$\Psi\hm=(\Psi_1, \Psi_2,\ldots, \Psi_N)\hm\in \Gamma$. Предпо\-ложим, что на 
множестве~$\Gamma$ выполняется условие~$1$ из набора предварительных условий 
и~функция $B(u_1, u_2,\ldots, u_N)$ обладает свойством строгой знакопостоянности 
при всех $(u_1, u_2,\ldots, u_N)\hm\in U$. Тогда справедливы следующие утверждения}:
\begin{enumerate}[1.]
\item \textit{Если основная функция $C(u_1, u_2,\ldots, u_N)\hm=
{A(u_1, u_2,\ldots, u_N)}/{B(u_1, u_2,\ldots, u_N)}$ ограничена сверху, 
но не достигает своего максимального 
значения, то имеет место неравенство}:
\begin{multline}
I\left(\Psi_1, \Psi_2,\ldots, \Psi_N\right)<{}\\
{}< \sup\limits_{(u_1, u_2,\ldots, u_N)\in U}
 C\left(u_1, u_2,\ldots, u_N\right)<\infty \label{e36}
\end{multline}
\textit{для всех} $(\Psi_1, \Psi_2,\ldots, \Psi_N)\in \Gamma$.
\item \textit{Если основная функция $C(u_1, u_2,\ldots, u_N)\hm=
{A(u_1, u_2,\ldots, u_N)}/{B(u_1, u_2,\ldots, u_N)}$ ограничена снизу, 
но не достигает своего минимального значения, то имеет место неравенство}:
\begin{multline*}
I\left(\Psi_1, \Psi_2,\ldots, \Psi_N\right)>{}\\
{}> \inf\limits_{(u_1, u_2,\ldots, u_N)\in U} 
C\left(u_1, u_2,\ldots, u_N\right)>-\infty 
%\label{e37}
\end{multline*}
\textit{для всех} $(\Psi_1, \Psi_2,\ldots, \Psi_N)\hm\in \Gamma$.
\end{enumerate}

\noindent
Д\,о\,к\,а\,з\,а\,т\,е\,л\,ь\,с\,т\,в\,о\ \ леммы~2. 
Докажем первое утверждение леммы. Поскольку множество значений 
основной функции $C(u_1, u_2,\ldots, u_N)$ ограничено сверху, оно имеет конечную 
верхнюю грань:
$$
\exists \sup\limits_{(u_1, u_2,\ldots, u_N)\in U} 
C\left(u_1, u_2,\ldots, u_N\right)<\infty
$$
(см.~\cite[гл.~1, \S3, п.~3.4, теорема~1]{25}).

По условию функция $C(u_1, u_2,\ldots, u_N)$ не достигает своего максимального 
значения. Следовательно, выполняется неравенство:
\begin{multline}
C(u_1, u_2,\ldots, u_N)=\fr{A(u_1, u_2,\ldots, u_N)}{B(u_1, u_2,\ldots, u_N)}<{}\\
{}< 
\sup\limits_{(u_1, u_2,\ldots, u_N)\in U} C(u_1, u_2,\ldots, u_N)<\infty\,, 
\\
\left(u_1, u_2,\ldots, u_N\right)\in U\,.
\label{e38}
\end{multline}
Взяв за основу строгое неравенство~(\ref{e38}), проведем рассуждения, аналогичные тем, 
которые были проведены в~лемме~1 по отношению к~неравенству~(\ref{e25}). 
В~результате получим строгое неравенство~(\ref{e36}).

Второе утверждение леммы~2 доказывается аналогично. Лемма~2 доказана.

\noindent
Д\,о\,к\,а\,з\,а\,т\,е\,л\,ь\,с\,т\,в\,о\ 
\ теоремы~2.
Начнем с~доказательства утверждения~1. Предположим сначала, что основная 
функция $C(u_1, u_2,\ldots, u_N)={A(u_1, u_2,\ldots, u_N)}/{B(u_1, u_2,\ldots, u_N)}$ 
ограничена сверху и~достигает глобального максимума на множестве~$U$ 
в~некоторой точке $u^{(+)}\hm=\left(u^{(+)}_1,u^{(+)}_2,\ldots,u^{(+)}_N\right)\hm\in U$,
а~именно:
\begin{multline*}
\max\limits_{(u_1, u_2,\ldots, u_N)\in U} C\left(u_1, u_2,\ldots, u_N\right) = {}\\
{}=
C\left(u^{(+)}_1,u^{(+)}_2,\ldots,u^{(+)}_N\right)<\infty\,.
\end{multline*}
Тогда выполняется соотношение:
\begin{multline}
C(u_1, u_2,\ldots, u_N)=\fr{A(u_1, u_2,\ldots, u_N)}{B(u_1, u_2,\ldots, u_N)}
\leq{}\\
{}\leq C\left(u^{(+)}_1,u^{(+)}_2,\ldots,u^{(+)}_N\right)<\infty\,, 
\\
\left(u_1, u_2,\ldots, u_N\right)\in U\,.
\label{e39}
\end{multline}
Условия леммы~1 выполнены, и~можно воспользоваться ее утверждениями. 
Согласно первому из них, если выполняется неравенство~(\ref{e39}), 
то имеет место соотношение:
\begin{equation*}
I(\Psi_1, \Psi_2,\ldots, \Psi_N)\leq 
C\left(u^{(+)}_1,u^{(+)}_2,\ldots,u^{(+)}_N\right)<\infty 
%\label{e40}
\end{equation*}
для всех стратегий управления $\Psi\hm=(\Psi_1, \Psi_2,\ldots\linebreak
\ldots, \Psi_N)\hm\in \Gamma$.

Таким образом, множество значений дроб\-но-ли\-ней\-но\-го интегрального 
функционала $I(\Psi_1, \Psi_2,\ldots, \Psi_N)$ ограничено сверху при всех 
$\Psi\hm=(\Psi_1, \Psi_2,\ldots, \Psi_N)\hm\in \Gamma$. Тогда существует верхняя 
грань этого множества и~выполняется неравенство:
\begin{multline}
\sup\limits_{(\Psi_1, \Psi_2,\ldots, \Psi_N)\in \Gamma} 
I\left(\Psi_1, \Psi_2,\ldots, \Psi_N\right)\leq {}\\
{}\leq
C\left(u^{(+)}_1,u^{(+)}_2,\ldots,u^{(+)}_N\right). \label{e41}
\end{multline}
Рассмотрим детерминированную стратегию управ\-ле\-ния 
$\Psi^{*(+)}\hm=\left(\Psi_1^{*(+)}, \Psi_2^{*(+)},\ldots, \Psi_N^{*(+)}\right)$, 
в~которой каждая вероятностная мера~$\Psi_i^{*(+)}$ является вы\-рож\-ден\-ной 
и~сосредоточена в~точке $u_i^{(+)}$, $i\hm=\overline{1, N}$.
По свойству интеграла
\begin{multline}
I\left(\Psi_1^{*(+)}, \Psi_2^{*(+)},\ldots ,\Psi_N^{*(+)}\right)={}\\
{}=
C\left(u^{(+)}_1,u^{(+)}_2,\ldots,u^{(+)}_N\right). \label{e42}
\end{multline}
Из соотношений~(\ref{e41}) и~(\ref{e42}) получаем:
\begin{multline}
\sup\limits_{(\Psi_1, \Psi_2,\ldots, \Psi_N)\in \Gamma} 
I\left(\Psi_1, \Psi_2,\ldots, \Psi_N\right)\leq{}\\
{}\leq
 I\left(\Psi_1^{*(+)}, 
\Psi_2^{*(+)},\ldots, \Psi_N^{*(+)}\right). \label{e43}
\end{multline}
Заметим дополнительно, что выполняются отношения принадлежности:
\begin{equation}
\Psi^{*(+)}=\left(\Psi_1^{*(+)}, \Psi_2^{*(+)},\ldots, \Psi_N^{*(+)}\right) 
\in \Gamma^* \subset \Gamma\,. \label{e44}
\end{equation}
Из~(\ref{e44}) и~свойства верхней грани следует:
\begin{multline}
\sup\limits_{\left(\Psi_1^{*}, \Psi_2^{*},\ldots, \Psi_N^{*}\right) \in \Gamma^*} 
I\left(\Psi_1^{*}, \Psi_2^{*},\ldots, \Psi_N^{*}\right)\leq {}\\
{}\leq
\sup\limits_{\left(\Psi_1, \Psi_2,\ldots, \Psi_N\right) 
\in \Gamma} I\left(\Psi_1, \Psi_2,\ldots, \Psi_N\right)\,. 
\label{e45}
\end{multline}
Объединяя~(\ref{e42}), (\ref{e43}) и~(\ref{e45}), получаем соотношение:
\begin{multline}
\sup\limits_{\left(\Psi_1^{*}, \Psi_2^{*},\ldots, \Psi_N^{*}\right) 
\in \Gamma^*} I\left(\Psi_1^{*}, \Psi_2^{*},\ldots, 
\Psi_N^{*}\right)\leq{}\\
{}\leq \sup\limits_{\left(\Psi_1, \Psi_2,\ldots, \Psi_N\right) 
\in \Gamma} I\left(\Psi_1, \Psi_2,\ldots, \Psi_N\right)\leq{}\\
{}\leq I\left(\Psi_1^{*(+)}, \Psi_2^{*(+)},\ldots, \Psi_N^{*(+)}\right)={}\\
{}=
\fr{A\left(u^{(+)}_1,u^{(+)}_2,\ldots,u^{(+)}_N\right)}{B\left(u^{(+)}_1,u^{(+)}_2,
\ldots,u^{(+)}_N\right)}\,.
 \label{e46}
\end{multline}
Из соотношения~(\ref{e46}) с~учетом~(\ref{e44}) получаем, что максимум 
функционала $I(\Psi_1, \Psi_2,\ldots, \Psi_N)$ на множестве допустимых стратегий 
$\Psi\hm=(\Psi_1, \Psi_2,\ldots, \Psi_N)\hm\in \Gamma$ существует и~достигается 
на детерминированной стратегии $\left(\Psi_1^{*(+)}, \Psi_2^{*(+)},\ldots, 
\Psi_N^{*(+)}\right)$.

Кроме того, выполняются соотношения~(\ref{e19}). Таким образом, утверждение~1 
в~случае, когда основная функция $C(u_1, u_2,\ldots, u_N)$ достигает глобального 
максимума, доказано. Соответствующее утверждение в~случае, когда основная функция 
$C(u_1, u_2,\ldots, u_N)$ достигает глобального минимума, доказывается аналогично. 
При этом используется второе утверждение леммы~1.

\smallskip

Перейдем к~доказательству второго утверждения теоремы~2. Предположим, что основная 
функция $C(u_1, u_2,\ldots, u_N)\hm=A(u_1, u_2,\ldots$\linebreak
$\ldots, u_N)/{B(u_1, u_2,\ldots, u_N)}$ 
ограничена сверху, но не достигает глобального максимума на множестве 
$U \hm= U_1 \times U_2 \times \cdots \times U_N$. Тогда множество значений 
основной функции имеет конечную верхнюю грань:

\noindent
\begin{multline*}
C\left(u_1, u_2,\ldots, u_N\right)=\fr{A(u_1, u_2,\ldots, u_N)}
{B(u_1, u_2,\ldots, u_N)}<{}\\
{}<
\sup\limits_{(u_1, u_2,\ldots, u_N)\in U} \fr{A(u_1, u_2,\ldots, u_N)}
{B(u_1, u_2,\ldots, u_N)}<\infty\,, 
\\
\left(u_1, u_2,\ldots, u_N\right)\in U\,.
%\label{e47}
\end{multline*}
По определению верхней грани для любого фиксированного $\varepsilon \hm>0$ 
существует точка $(u_1^{(+)}(\varepsilon), u_2^{(+)}(\varepsilon),\ldots, 
u_N^{(+)}(\varepsilon))$ такая, что выполняется двойное неравенство~(\ref{e21}) 
(см.~\cite[гл.~1, \S\,3, п.~3.4]{25}). Иначе говоря, значение основной функции 
в~указанной точке лежит в~левой \mbox{$\varepsilon$-окрест}\-ности верхней грани. 
Рассмотрим детерминированную стратегию управления 
$\Psi^{*(+)}(\varepsilon)\hm=\!\left(\Psi_1^{*(+)}(\varepsilon), 
\Psi_2^{*(+)}(\varepsilon),\ldots, \Psi_N^{*(+)}(\varepsilon)\!\right)$, компонентами\linebreak 
которой являются вырожденные вероятностные меры $\Psi_1^{*(+)}(\varepsilon), 
\Psi_2^{*(+)}(\varepsilon),\ldots, \Psi_N^{*(+)}(\varepsilon)$, причем вырожденная 
мера~$\Psi_i^{*(+)}(\varepsilon)$ сосредоточена в~точке~$u_i^{(+)}(\varepsilon)$,
$i\hm=1,2,\ldots,N$.

По свойству интеграла
\begin{multline}
I\left(\Psi_1^{*(+)}(\varepsilon), \Psi_2^{*(+)}(\varepsilon),\ldots,
 \Psi_N^{*(+)}(\varepsilon)\right)={}\\
 {}=
 C\left(u_1^{(+)}(\varepsilon), u_2^{(+)}(\varepsilon),\ldots, 
 u_N^{(+)}(\varepsilon)\right)\,. 
 \label{e48}
\end{multline}
Из соотношения~(\ref{e48}) с~учетом указанного свойства основной функции получаем:
\begin{multline}
\sup\limits_{(u_1, u_2,\ldots, u_N)\in U} \fr{A(u_1, u_2,\ldots, u_N)}
{B(u_1, u_2,\ldots, u_N)}-\varepsilon<{}\\
{}< I\left(\Psi_1^{*(+)}(\varepsilon), 
\Psi_2^{*(+)}(\varepsilon),\ldots, \Psi_N^{*(+)}(\varepsilon)\right)<{}
\\
{}< \sup\limits_{(u_1, u_2,\ldots, u_N)\in U} \fr{A(u_1, u_2,\ldots, u_N)}
{B(u_1, u_2,\ldots, u_N)}<\infty\,. 
\label{e49}
\end{multline}
Заметим также, что в~рассматриваемом случае выполнены условия леммы~2. 
Воспользуемся первым утверждением этой леммы, а~именно соотношением~(\ref{e36}):
\begin{multline}
I(\Psi_1, \Psi_2,\ldots, \Psi_N)< {}\\
{}<\sup\limits_{(u_1, u_2,\ldots, u_N)
\in U} \fr{A(u_1, u_2,\ldots, u_N)}{B(u_1, u_2,\ldots, u_N)}<\infty 
\label{e50}
\end{multline}
для всех $(\Psi_1, \Psi_2,\ldots, \Psi_N)\in\Gamma$.

Из соотношений~(\ref{e49}) и~(\ref{e50}) следует, что детерминированная стратегия 
$\Psi^{*(+)}(\varepsilon)\hm=\left(\Psi_1^{*(+)}(\varepsilon), \Psi_2^{*(+)}(\varepsilon),
\ldots, \Psi_N^{*(+)}(\varepsilon)\right)$, опре\-де\-ля\-емая набором вырожденных 
вероятностных мер, сосредоточенных в~соответствующих точках 
$\left(u_1^{(+)}(\varepsilon), u_2^{(+)}(\varepsilon),\ldots, 
u_N^{(+)}(\varepsilon)\right)$, является $\varepsilon$-оп\-ти\-маль\-ной. 
Вторая часть утверждения~2 теоремы~2, связанная со свойствами нижней грани, 
доказывается аналогично.

Докажем третье утверждение теоремы~2. Предположим, что множество значений 
основной функции $C(u_1, u_2,\ldots, u_N)\hm=
A(u_1, u_2,\ldots$\linebreak $\ldots, u_N)/{B(u_1, u_2,\ldots, u_N)}$
не является ограниченным сверху на множестве $U\hm=U_1\times U_2 \times \cdots $\linebreak
$\cdots \times U_N$.
Тогда существует последовательность\linebreak точек $\left(u_1^{(+)}(n), u_2^{(+)}(n),
\ldots,u_N^{(+)}(n)\right)\hm\in U$, $n\hm=1,2,\ldots$, для которой
\begin{multline}
C\left(u_1^{(+)}(n), u_2^{(+)}(n),\ldots,u_N^{(+)}(n)\right)={}\\
{}=
\fr{A\left(u_1^{(+)}(n), u_2^{(+)}(n),\ldots,u_N^{(+)}(n)\right)}
{B\left(u_1^{(+)}(n), u_2^{(+)}(n),\ldots,u_N^{(+)}(n)\right)}
\longrightarrow \infty \,,\\
n\rightarrow \infty\,.
\label{e51}
\end{multline}
Зафиксируем некоторую последовательность точек $\left(u_1^{(+)}(n), u_2^{(+)}(n),
\ldots,u_N^{(+)}(n)\right)\hm\in U$, $n\hm=1,2,\ldots$, обладающих указанным свойством, 
и~рассмотрим последовательность детерминированных  стратегий управления 
$\Psi^{*(+)}(n)\hm=\left(\Psi_1^{*(+)}(n), \Psi_2^{*(+)}(n),\ldots, 
\Psi_N^{*(+)}(n)\right)$, $n\hm=1,2,\ldots$, определяемых набором вырожденных 
вероятностных мер, сосредоточенных в~соответствующих точках 
$\left(u_1^{(+)}(n), u_2^{(+)}(n),\ldots,u_N^{(+)}(n)\right)$, $n\hm=1,2,\ldots$ 
По свойству интеграла для любого фиксированного значения $n=1,2,\ldots$ 
выполняется равенство:
\begin{multline}
I \left(\Psi^{*(+)}(n)\right)={}\\
{}=I\left(\Psi_1^{*(+)}(n), \Psi_2^{*(+)}(n),\ldots,
 \Psi_N^{*(+)}(n)\right)={}\\
{}=\fr{A\left(u_1^{(+)}(n), u_2^{(+)}(n),\ldots,u_N^{(+)}(n)\right)}
{B\left(u_1^{(+)}(n), u_2^{(+)}(n),\ldots,u_N^{(+)}(n)\right)}\,. 
\label{e52}
\end{multline}
Из соотношений~(\ref{e51}) и~(\ref{e52}) следует, что
\begin{multline}
I\left(\Psi^{*(+)}(n)\right)={}\\
{}=I\left(\Psi_1^{*(+)}(n), \Psi_2^{*(+)}(n),\ldots, 
\Psi_N^{*(+)}(n)\right)\longrightarrow\infty\,,\\ 
n \rightarrow\infty\,.
 \label{e53}
\end{multline}
Соотношение~(\ref{e53}) означает, что множество значе\-ний дроб\-но-ли\-ней\-но\-го 
интегрального функциона\-ла $I(\Psi_1, \Psi_2,\ldots, \Psi_N)$ вида~(\ref{e11}) 
не ограничено сверху\linebreak на множестве наборов вырожденных вероятностных мер 
$\left(\Psi_1^{*(+)}(n), \Psi_2^{*(+)}(n),\ldots, \Psi_N^{*(+)}(n)\right)\hm\in\Gamma^*$, 
а~следовательно, и~на более широком\linebreak множестве наборов вероятностных 
мер $(\Psi_1, \Psi_2,\ldots$\linebreak $\ldots, \Psi_N)\hm\in\Gamma$. В~таком случае решения экстремальной 
задачи~(\ref{e18}) в~форме задачи на максимум не существует. Соответствующее утвержде\-ние 
для варианта, когда множество значений основной функции $C(u_1, u_2,\ldots,u_N)
\hm=A(u_1, u_2,\ldots$\linebreak $\ldots,u_N)/{B(u_1, u_2,\ldots,u_N)}$ 
не является ограниченным снизу, доказывается аналогично. Третье утверж\-де\-ние теоремы~2 
доказано. Тем самым тео\-ре\-ма~2 доказана полностью.

\smallskip

Применим теорему~2 для решения поставленной задачи оптимального управления. 
Из утверждения этой теоремы следует, что для доказательства су-\linebreak ществования 
оптимального управ\-ле\-ния и~его нахождения необходимо исследовать на 
глобальный экстремум основную функцию дроб\-но-ли\-ней\-но\-го интегрального 
функционала $C(u_1,u_2,\ldots,u_N)$, определяемую формулой~(\ref{e17}) с~учетом 
равенств~(\ref{e12})--(\ref{e16}). В~некоторых случаях, например когда основной 
процесс~$\xi(t)$ является регенерирующим, а~стоимостные характеристики 
модели задаются линейными функциями, такое исследование можно провести 
аналитически. Однако для подавляющего большинства полумарковских моделей 
для этого необходимо использовать численные методы.

\section{Заключение}

В заключительной части работы приведем \mbox{краткое} описание теоретической 
основы метода решения задачи оптимального управления полумарковским 
процессом с~конечным множеством состояний.

\begin{enumerate}[1.]
\item Исходная проблема оптимального управления формулируется в~виде 
экстремальной задачи~(\ref{e18}). Целевым показателем качества управ\-ле\-ния в~данной задаче 
служит величина~(\ref{e10}), которая имеет характер средней удельной прибыли.
\item Доказывается, что стационарный показатель~(\ref{e10}) представим в~виде 
дроб\-но-ли\-ней\-но\-го интегрального функционала~(\ref{e11}), для которого явно 
определяются подынтегральные функции числителя и~знаменателя, а~следовательно, 
и~основная функция данного функционала.
\item Используется теорема об экстремуме дроб\-но-ли\-ней\-но\-го интегрального 
функционала. На основании утверждений этой теоремы уста\-нав\-ли\-ва\-ет\-ся, что 
исходная задача оптимального управления сводится к~исследованию на глобальный 
экстремум основной функции этого функционала, для которой получено явное 
аналитическое представление.
\end{enumerate}

Заметим, что такое исследование задач оптимального управления 
стохастическими системами фактически уже было проведено в~ряде работ П.\,В.~Шнуркова 
и~его соавторов. В~частности, в~работе~\cite{26} была рассмотрена модель 
управления для обрывающегося процесса восстановления, описывающего функционирование 
некоторой технической системы. Задача управления решалась для различных показателей 
эффективности и~надежности этой системы, имеющих структуру дроб\-но-ли\-ней\-но\-го 
интегрального функционала.

В работах~\cite{27, 28} рассматривались модели регенерирующих процессов 
для исследования сис\-тем управления запасами. Различные показатели качества 
управления были представлены в~форме дроб\-но-ли\-ней\-ных интегральных функционалов. 
Основные функции этих функционалов были найде\-ны в~явной форме и~исследовались 
на глобальный экстремум. В~работах~\cite{21,29} рассматривалась достаточно 
сложная полумарковская модель с~конечным множеством состояний, описывающая 
сис\-те\-му управления запасом непрерывного продукта. Показатели качества управления в~этой 
модели также имели структуру дроб\-но-ли\-ней\-ных интегральных функционалов, 
для основных функций которых были найдены явные аналитические представления. 
Упомянем также работы~\cite{30, 31}, в~которых была исследована полумарковская 
модель с~дис\-крет\-но-не\-пре\-рыв\-ным фазовым пространством. Показатели 
качества управления в~этой  модели были найдены в~явной форме как функции от 
двух непрерывных параметров управления.

Фактически во всех упомянутых работах уже был использован метод решения задачи 
оптимального управления регенерирующим или полумарковским случайным процессом, 
основанный на исследовании экстремальных свойств основной функции соответствующего 
дроб\-но-ли\-ней\-но\-го интегрального функционала. Из соображений, изложенных 
во\linebreak введении, следует, что в~период написания и~пуб\-ли\-кации этих работ данный метод 
не имел стро\-гого обоснования. Однако после публикации\linebreak работы~\cite{14} и~настоящего 
исследования можно утверж\-дать, что полученные в~них результаты полностью теоретически 
обоснованы.

Таким образом, изложенный выше метод решения проблемы оптимального управления 
полумарковскими процессами с~конечными множествами состояний может быть успешно 
реализован для многих задач, рассматриваемых в~различных областях прикладной 
теории вероятностей.

Практическая реализация численной процедуры поиска оптимального решения на примере\linebreak 
полумарковской модели управления запасом непрерывного продукта (подробнее 
см.~\cite{21, 29}), ба\-зи\-ру\-юща\-яся на изложенных выше результатах (в~частности, 
теореме~1), была осуществлена А.\,К.~Горшениным и~соавторами 
в~статье~\cite{Gorshenin2015}. Коротко опишем наиболее важные аспекты этой работы.

Для решения поставленной задачи опти\-мального управления была создана 
специальная программа \verb"Inventory" на встроенном языке программирования 
пакета \verb"MATLAB", ее возможности\linebreak кратко представле\-ны в~упомянутой ранее 
\mbox{статье}~\cite{Gorshenin2015}. В~программе \verb"Inventory" реализованы функции 
для оценивания через заданные исходные параметры вероятностных и~стоимостных 
характеристик модели, которые в~дальнейшем используются для поиска значений 
основной функции дроб\-но-ли\-ней\-но\-го функционала~(\ref{e17}). Точка глобального 
экстремума этой функции и~определяет оптимальное управление.

В качестве начальных данных необходимо задание следующих параметров:
\begin{itemize}
\item спрос и~вместимость склада;
\item разбиение множества значений объема запаса;
\item вероятностные характеристики, описывающие модель пополнения запаса;
\item условные математические ожидания длительностей задержек пополнения запаса;
\item функции для характеризации затрат и~доходов.
\end{itemize}

По итогам работы программы \verb"Inventory" ряд вспомогательных функций 
представляется в~аналитической форме (в частности, с~использованием аппарата 
символьных вычислений  \verb"Symbolic Toolbox"\linebreak пакета \verb"MATLAB"), выводится 
точка глобального экстремума функции нескольких вещественных переменных~(\ref{e17}), 
найденная с~помощью применения численных и~при\-бли\-жен\-но-ана\-ли\-ти\-че\-ских\linebreak 
аппроксимаций. 
Также формируются графики оценок значений ве\-ро\-ят\-ност\-но-сто\-и\-мост\-ных 
характеристик 
и~основной функции дроб\-но-ли\-ней\-но\-го функционала~(\ref{e17}), либо трехмерных 
сечений в~случае наличия более трех параметров управления (переменных).

Функциональность пакета \verb"Inventory" может быть расширена для практической 
реализации метода решения задачи поиска оптимального управ\-ле\-ния полумарковскими 
процессами с~конечными множествами состояний, рассмотренного в~данной статье.


 {\small\frenchspacing
 {%\baselineskip=10.8pt
 \addcontentsline{toc}{section}{References}
 \begin{thebibliography}{99}
 \bibitem{1}
\Au{Ховард Р.} Динамическое программирование и~марковские процессы~/ 
Пер. с~англ.~--- М.: Сов. радио, 1964. 189~с.
(\Au{Howard~R.\,A.} Dynamic programming and Markov processes.~--- 
Cambridge, MA, USA: MIT Press, 1960. 136~p.)
\bibitem{2} 
\Au{Рыков В.\,В.} Управляемые марковские процессы с~конечными пространствами 
состояний и~управлений~// Теория вероятностей и~ее применения, 1966. Т.~11. 
Вып.~2. С.~343--351.
\bibitem{3} 
\Au{Джевелл В.} Управляемые полумарковские процессы~// Кибернетич. сборник.~--- 
М.: Мир, 1967. Вып.~4. С.~97--134.
%{\em Jewell W.\,S.} Markov-renewal programming~// Operations Research, 1963. Vol.~11. P.~938--971.
\bibitem{4} 
\Au{Fox B.} Markov renewal programming by linear fractional programming~// 
SIAM J.~Appl. Math., 1966. Vol.~14. P.~1418--1432.
\bibitem{5} 
\Au{Denardo E.\,V.} Contraction mappings in the theory underlying dinamic programming~// 
SIAM Rev., 1967. Vol.~9. P.~165--177.

\bibitem{6} 
\Au{Howard R.\,A.} Research in semi-Markovian decision structures~// 
J.~Oper. Res. Soc. Japan, 1963. Vol.~6. P.~163--199.
\bibitem{7} 
\Au{Osaki S., Mine H.} Linear programming algorithms for Markovian decision processes~//
 J.~Math. Anal. Appl., 1968. Vol.~22. P.~356--381.
\bibitem{8} 
\Au{Майн Х., Осаки С.} Марковские процессы принятия решений~/ Пер. с~англ.~--- 
М.: Наука, 1977. 176~с.
(\Au{Mine~H., Osaki~S.} 
Markovian decision processes.~--- New York, NY, USA: 
American Elsevier Publishing Co., 1970. 142~p.)
\bibitem{9} 
\Au{Гихман И.\,И., Скороход А.\,В.} Управляемые случайные процессы.~--- 
Киев: Наукова думка, 1977. 251~с.
\bibitem{10} 
\Au{Luque-Vasquez F., Herndndez-Lerma~О.} Semi-Markov control models with average costs~// 
Appl. Math., 1999. Vol.~26. No.\,3. P.~315--331.
\bibitem{11} 
\Au{Vega-Amaya O., Luque-Vasquez~F.} Sample-path average cost optimality for 
semi-Markov control processes on Borel spaces: Unbounded costs and mean holding times~// 
Appl. Math., 2000. Vol.~27. No.\,3. P.~343--367.
\bibitem{12} 
Вопросы математической теории надежности~/ Под ред. Б.\,В. Гнеденко.~--- 
М.: Радио и~связь, 1983. 376~с.
\bibitem{13} 
\Au{Барзилович Е.\,Ю., Каштанов~В.\,А.} Некоторые математические вопросы теории 
обслуживания сложных систем.~---  М.: Сов. радио, 1971. 272~с.
\bibitem{14} 
\Au{Шнурков П.\,В.} О~решении проблемы безусловного экстремума для 
дроб\-но-ли\-ней\-но\-го интегрального функционала на множестве вероятностных мер~// 
Докл. РАН. Сер. Математика, 2016. Т.~470. №\,4. C.~387--392.
\bibitem{15} 
\Au{Ширяев А.\,Н.}  Вероятность.~--- М.:~МЦНМО, 2011. Кн.~1. 552~с. Кн.~2. 968~с.
\bibitem{16} 
\Au{Боровков А.\,А.} Теория вероятностей.~--- М.: Либроком, 2009. 656~c.
\bibitem{17} 
\Au{Хеннекен П.\,Л., Тортра А.} Теория вероятностей 
и~некоторые ее приложения.~--- М.: Наука, 1974. 472~c.
\bibitem{18} 
\Au{Халмош П.} Теория меры~/ Пер. с~англ.~--- М.: ИЛ, 1953. 282~c.
(\Au{Halmos~P.} Measure theory.~--- Litton Educational Publishing, Inc. 1950. 304~p.)
\bibitem{19} 
\Au{Королюк В.\,С., Турбин~А.\,Ф.} Полумарковские процессы и~их приложения.~--- 
Киев:~Наукова думка, 1976. 184~с.
\bibitem{20} 
\Au{Janssen J., Manca R.} Applied semi-Markov processes.~--- New York,
NY, USA: Springer, 2006. 309~p.
\bibitem{21} 
\Au{Шнурков П.\,В., Иванов~А.\,В.} Анализ дискретной полумарковской модели
 управления запасом непрерывного продукта при периодическом прекращении потребления~// 
 Дискретная математика, 2014. Т.~26. Вып.~1. С.~143--154.
\bibitem{22} 
\Au{Иванов~А.\,В.} Анализ дискретной полумарковской модели
 управления запасом непрерывного продукта при периодическом прекращении 
 потребления.~--- М.: НИУ ВШЭ, 2014.  Дисс.\ \ldots\ канд. физ.-мат. наук. 120~с.
\bibitem{23}  %23
\Au{Bajalinov~E.\,B.} Linear-fractional programming. 
Theory, methods, applications and software.~--- 
Boston/\linebreak Dordrecht/London: Kluwer Academic Publs., 2003. 423~p.

\bibitem{27} %27
\Au{Шнурков П.\,В., Мельников~Р.\,В.} Оптимальное управление запасом 
непрерывного продукта в~модели регенерации~// Обозрение прикладной 
и~промышленной математики, 2006. Т.~13. Вып.~3. С.~434--452.
\bibitem{28} 
\Au{Шнурков П.\,В., Мельников~Р.\,В.} 
Исследование проб\-ле\-мы управления запасом непрерывного продукта при детерминированной 
задержке поставки~// Автоматика и~телемеханика, 2008. Т.~10. С.~93--113.


\bibitem{24}  %26
\Au{Шнурков П.\,В.} Методы исследования задач оптимального обслуживания 
в~математической теории надежности.~--- 
М.: МИЭМ, 1983.  Дисс.\ \ldots\ канд. физ.-мат. наук.

 \bibitem{25}  %25
\Au{Кудрявцев Л.\,Д.} Курс математического анализа. Т.~1.~--- 
М.: Дрофа, 2006. 704~с.

\bibitem{26} %24
\Au{Шнурков П.\,В.} Оптимальное обслуживание на периоде 
до первого отказа системы~// Применение аналитических методов в~вероятностных
 задачах.~--- Киев: Институт математики АН УССР, 1986. С.~121--129.

\bibitem{29} 
\Au{Шнурков П.\,В., Иванов~А.\,В.} Исследование задачи оптимизации в~дискретной 
полумарковской модели управления непрерывным запасом~// Вестник МГТУ им.\ 
Н.\,Э. Баумана. Сер.\ Естественные науки, 2013. Т.~3. Вып.~50. С.~62--87.
\bibitem{30} 
\Au{Shnourkoff P.\,V.} The two-element system with one 
restoring device optimum maintenance~// Stoch. Anal. Appl., 1997. 
Vol.~15. No.\,5. P.~823--837.
\bibitem{31} 
\Au{Shnourkoff P.\,V.} The two-element system optimum maintenance tills the first fail~// 
Stoch. Anal. Appl., 2001. Vol.~19. No.\,6. P.~1005--1024.
\bibitem{Gorshenin2015} 
\Au{Gorshenin~A.\,K., Belousov~V.\,V., Shnourkoff~P.\,V.,
Ivanov~A.\,V.} Numerical research of the optimal control problem in the semi-Markov 
inventory model~// AIP Conference Proceedings, 2015. Vol.~1648. {250007}. 4~p.
%\bibitem{33} {\em Горшенин А.\,К., Белоусов В.\,В., Шнурков П.\,В.} 2016. Система управления запасами на основе стохастических полумарковских моделей. Свидетельство о государственной регистрации программы для ЭВМ \textnumero 2016619021.
 \end{thebibliography}

 }
 }

\end{multicols}

\vspace*{-6pt}

\hfill{\small\textit{Поступила в~редакцию 15.07.16}}

%\vspace*{8pt}

\newpage

\vspace*{-24pt}

%\hrule

%\vspace*{2pt}

%\hrule

%\vspace*{8pt}


\def\tit{ANALYTICAL SOLUTION OF~THE~OPTIMAL CONTROL TASK OF~A~SEMI-MARKOV 
PROCESS WITH~FINITE SET OF~STATES}

\def\titkol{Analytical solution of~the~optimal control task of~a~semi-Markov 
process with~finite set of~states}

\def\aut{P.\,V.~Shnurkov$^{1}$, A.\,K.~Gorshenin$^{2}$, and~V.\,V.~Belousov$^{2}$}

\def\autkol{P.\,V.~Shnurkov, A.\,K.~Gorshenin, and~V.\,V.~Belousov}

\titel{\tit}{\aut}{\autkol}{\titkol}

\vspace*{-9pt}


    
\noindent
$^1$National Research University Higher School of Economics, 34~Tallinskaya Str., 
Moscow, 123458, Russian\linebreak
$\hphantom{^9}$Federation

\noindent
$^2$Institute of Informatics Problems, Federal Research Center 
``Computer Science and Control'' of the Russian\linebreak
$\hphantom{^9}$Academy of Sciences, 44-2~Vavilova Str., 
Moscow 119333, Russian Federation



\def\leftfootline{\small{\textbf{\thepage}
\hfill INFORMATIKA I EE PRIMENENIYA~--- INFORMATICS AND
APPLICATIONS\ \ \ 2016\ \ \ volume~10\ \ \ issue\ 4}
}%
 \def\rightfootline{\small{INFORMATIKA I EE PRIMENENIYA~---
INFORMATICS AND APPLICATIONS\ \ \ 2016\ \ \ volume~10\ \ \ issue\ 4
\hfill \textbf{\thepage}}}

\vspace*{3pt}


\Abste{The theoretical verification of the new method of finding 
the optimal strategy of control of a~semi-Markov process with finite set of states is 
presented. The paper considers Markov randomized strategies of control, determined by 
a~finite collection of probability measures, corresponding to each state. The quality 
characteristic is the stationary cost index. This index is a~linear-fractional integral 
functional, depending on collection of probability measures, giving the strategy of control. 
Explicit analytical forms of integrands of numerator and denominator of this 
linear-fractional integral functional are known. The basis of consequent results is 
the new generalized and strengthened form of the theorem about an extremum of 
a~linear-fractional integral functional. It is proved that problems of existence 
of an optimal control strategy of a~semi-Markov process and finding this strategy 
can be reduced to the task of numerical analysis of global extremum for 
the given function, depending on finite number of real arguments.}

\KWE{optimal control of a~semi-Markov process; stationary cost index of quality control; 
linear-fractional integral functional}




\DOI{10.14357/19922264160408} 

\vspace*{-16pt}

\Ack
\noindent
The research was partially supported by the Russian Foundation 
for Basic Research (project 15-07-05316).



%\vspace*{3pt}

  \begin{multicols}{2}

\renewcommand{\bibname}{\protect\rmfamily References}
%\renewcommand{\bibname}{\large\protect\rm References}

{\small\frenchspacing
 {%\baselineskip=10.8pt
 \addcontentsline{toc}{section}{References}
 \begin{thebibliography}{99}
\bibitem{1-1}
\Aue{Howard,~R.\,A.} 1960. \textit{Dynamic programming and Markov processes}. 
Cambridge, MA: MIT Press. 136~p.
\bibitem{2-1}
\Aue{Rykov,~V.\,V.} 1966. Upravlyaemye markovskie protsessy 
s~konechnymi prostranstvami sostoyaniy i~upravleniy 
[Controlled Markov processes with finite spaces of states and controls ]. 
\textit{Teoriya veroyatnostey i~ee primeneniya} 
[Theory of Probability and Its Applications] 11(2):343--351.
\bibitem{3-1}
\Aue{Jewell,~W.\,S.} 1963. Markov-renewal programming. 
\textit{Oper. Res.} 11:938--971.
\bibitem{4-1}
\Aue{Fox,~B.} 1966. Markov renewal programming by linear fractional programming. 
\textit{SIAM J.~Appl. Math.} 14:1418--1432.
\bibitem{5-1}
\Aue{Denardo, E.\,V.} 1967. Contraction mappings in the theory underlying dinamic 
programming. \textit{SIAM Rev.} 9:165--177.
\bibitem{6-1}
\Aue{Howard,~R.\,A.} 1963. Research in semi-Markovian decision structures. 
\textit{J.~Oper. Res. Soc. Japan} 6:163--199.
\bibitem{7-1}
\Aue{Osaki,~S., and H.~Mine.} 1968. Linear programming algorithms 
for Markovian decision processes. \textit{J.~Math. Anal. Appl.} 22:356--381.
\bibitem{8-1}
\Aue{Mine,~H., and S.~Osaki.} 1970. 
\textit{Markovian decision processes}. New York, NY: Elsevier. 142~p.
\bibitem{9-1}
\Aue{Gikhman,~I.\,I., and A.\,V.~Skorokhod.} 1977. 
\textit{Upravlyaemye sluchaynye protsessy} 
[Controlled random processes]. Kiev: Naukova Dumka. 251~p.
\bibitem{10-1}
\Aue{Luque-Vasquez,~F., and О.~Herndndez-Lerma.} 1999. 
Semi-Markov control models with average costs. \textit{Appl. Math.} 26(3):315--331.
\bibitem{11-1}
\Aue{Vega-Amaya,~O., and  F.~Luque-Vasquez.} 2000.  
Sample-path average cost optimality for semi-Markov control processes on Borel spaces: 
Unbounded costs and mean holding times. \textit{Appl. Math.} 27(3):343--367.
\bibitem{12-1}
Gnedenko,~B.~V., ed. 1983. 
\textit{Voprosy matematicheskoy teorii nadezhnosti} 
[Problems of the mathematical theory of reliability].  Moscow: Radio i~svyaz'. 376~p.
\bibitem{13-1}
\Aue{Barzilovich,~E.\,Yu., and V.\,A.~Kashtanov.} 1971. 
\textit{Nekotorye matematicheskie voprosy teorii obsluzhivaniya slozhnykh sistem}  
[Some mathematical questions in theory of complex systems maintenance]. 
Moscow: Sovetskoe radio. 272~p.
\bibitem{14-1}
\Aue{Shnurkov,~P.\,V.} 2016. Solution of the unconditional extremum problem for 
a~linear-fractional 
integral functional on a~set of probability measures. 
\textit{Dokl. Math.} 94(2):550--554.
\bibitem{15-1} %14
\Aue{Shiryaev,~A.\,N.} 2016. 
\textit{Probability-1}. Graduate texts in mathematics ser.
New York, NY: Springer. Vol.~95. 503~p.;
2017. \textit{Probability-2.} Vol.~900. 500~p.
\bibitem{16-1}
\Aue{Borovkov,~А.\,А.} 2009. 
\textit{Teoriya veroyatnostey} [Probability theory]. Moscow: Librokom. 656~p.
\bibitem{17-1}
\Aue{Khenneken,~P.\,L., and A.~Tortra.} 1974. 
\textit{Teoriya veroyatnostey i~nekotorye ee prilozheniya} 
[Probability theory and some of its applications]. Moscow: Nauka. 472~p.
\bibitem{18-1}
\Aue{Halmos,~P.} 1950. \textit{Measure theory}. Litton Educational Publishing. 304~p.
\bibitem{19-1}
\Aue{Korolyuk, V.\,S., and A.\,F.~Turbin.} 1976. 
\textit{Polumarkovskie protsessy i~ikh prilozheniya} 
[Semi-Markov processes and their applications]. Kiev: Naukova Dumka. 184~p.
\bibitem{20-1}
\Aue{Janssen,~J., and R.~Manca.} 2006. 
\textit{Applied semi-Markov processes}. New York, NY: Springer. 309~p.
\bibitem{21-1}
\Aue{Shnurkov,~P.\,V, and A.\,V~Ivanov.} 2015. Analysis of a~discrete semi-Markov model of continuous inventory 
control with periodic interruptions of consumption. 
\textit{Discrete Math. \mbox{Appl}.} 25(1):59--67.
\bibitem{22-1} %21
\Aue{Ivanov,~A.\,V.} 2014. Analiz diskretnoy polumarkovskoy modeli upravleniya 
zapasom nepreryvnogo produkta pri periodicheskom prekrashchenii potrebleniya 
[Analysis of a~discrete semi-Markov control model of continuous product inventory 
in a~periodic cessation of consumption].  
Moscow: Natsional'nyy Issledovatel'skiy Universitet ``Vysshaya Shkola Ekonomiki.'' 
PhD Thesis. 120~p.
\bibitem{23-1} %22
\Aue{Bajalinov,~E.\,B.} 2003. 
\textit{Linear-fractional programming. Theory, methods, applications and software}. 
Boston/\linebreak Dordrecht/London: Kluwer Academic Publs. 423~p.
\bibitem{26-1} %24
\Aue{Shnurkov,~P.\,V., and R.\,V.~Mel'nikov.} 2006. Optimal'noe upravlenie 
zapasom nepreryvnogo produkta v modeli regeneratsii [Optimal control of 
a~continuous product inventory in the regeneration model]. 
\textit{Obozrenie prikladnoy i~promyshlennoy matematiki} [Rev. Appl. Ind. Math.]
13(3):434--452.

\bibitem{25-1} %25
\Aue{Shnurkov,~P.\,V., and R.\,V.~Mel'nikov.} 2008. 
Analysis of the problem of continuous-product inventory control under deterministic 
lead time. \textit{Automat. Rem. Contr.} 69(10):1734--1751.

\columnbreak

\bibitem{24-1} %26
\Aue{Shnurkov,~P.\,V.} 1983. Metody issledovaniya zadach optimal'nogo obsluzhivaniya 
v~matematicheskoy teorii nadezhnosti [Research methods of optimal service problems 
in the mathematical theory of reliability].  
Moscow: Moskovskiy Institut Elektronnogo Mashinostroeniya.  PhD Thesis. 


\bibitem{27-1} %27
\Aue{Kudryavtsev,~L.\,D.} 2006. 
\textit{Kurs matematicheskogo analiza} 
[A~course of mathematical analysis]. Vol.~1. Moscow: Drofa. 704~p.

\bibitem{28-1}
\Aue{Shnurkov,~P.\,V.} 1986. Optimal'noe obsluzhivanie na periode do 
pervogo otkaza sistemy [The optimum service period until the first system failure]. 
\textit{Primenenie analiticheskikh metodov v~veroyatnostnykh zadachakh} 
[The application of analytical methods in probabilistic tasks]. Kiev:
Institute of Mathematics of the Academy of Sciences of the USSR. 121--129.

\bibitem{29-1}
\Aue{Shnurkov,~P.\,V., and A.\,V.~Ivanov.} 2013. Issledovanie zadachi optimizatsii 
v~diskretnoy polumarkovskoy modeli upravleniya nepreryvnym zapasom 
[Study of the optimization problem in discrete semi-Markov model of continuous 
inventory control]. \textit{Vestnik MGTU im.\ N.\,E.~Baumana. Ser. 
Estestvennye nauki} [Vestnik of MSTU named after N.\,E.~Bauman. Ser. Natural sciences] 
3(50):62--87.
\bibitem{30-1}
\Aue{Shnourkoff,~P.\,V.} 1997. The two-element system with one restoring device 
optimum maintenance.  \textit{Stoch. Anal. Appl.} 15(5):823--837.
\bibitem{31-1}
\Aue{Shnourkoff,~P.\,V.} 2001. The two-element system optimum maintenance tills 
the first fail. \textit{Stoch. Anal. Appl.} 19(6):1005--1024.
\bibitem{32-1}
\Aue{Gorshenin,~A.\,K., V.\,V.~Belousov, P.\,V.~Shnourkoff, and A.\,V.~Ivanov.}
2015. Numerical research of the optimal control problem in the semi-Markov 
inventory model. \textit{AIP Conference Proceedings} 1648:250007.
\end{thebibliography}

 }
 }

\end{multicols}

\vspace*{-3pt}

\hfill{\small\textit{Received July 15, 2016}}

\Contr

\noindent
\textbf{Shnurkov Peter V.} (b.\ 1953)~---
 Candidate of Science (PhD) in physics and mathematics, 
 associate professor, National Research University Higher School of Economics, 
 34~Tallinskaya Str., Moscow 123458, Russian Federation; \mbox{pshnurkov@hse.ru} 
 
 \vspace*{3pt}
 
 \noindent
\textbf{Gorshenin Andrey K.}  (b.\ 1986)~---
Candidate of Science (PhD) in physics and mathematics, leading scientist, 
Institute of Informatics Problems, Federal Research Center ``Computer Science 
and Control'' of the Russian Academy of Sciences, 44-2~Vavilov Str., Moscow 119333, 
Russian Federation; associate professor, Federal State Budget Educational 
Institution of Higher Education ``Moscow Technological University,'' 
78~Vernadskogo Ave., Moscow 119454, Russian Federation;
\mbox{agorshenin@frccsc.ru}

\vspace*{3pt}

\noindent
\textbf{Belousov Vasiliy V.} (b.\ 1977)~---
Candidate of Science (PhD) in technology, senior scientist, Institute of 
Informatics Problems, Federal Research Center ``Computer Science and Control'' 
of the Russian Academy of Sciences, 44-2~Vavilov Str., Moscow 119333, Russian 
Federation; \mbox{VBelousov@ipiran.ru}
\label{end\stat}


\renewcommand{\bibname}{\protect\rm Литература}  %8
%\include{shnurkov-old}
\def\stat{melnikov}

\def\tit{СТАТИСТИЧЕСКИЕ СВОЙСТВА ДВОИЧНЫХ 
НЕАВТОНОМНЫХ РЕГИСТРОВ СДВИГА С~ВНУТРЕННИМ 
СУММИРОВАНИЕМ$^*$}

\def\titkol{Статистические свойства двоичных 
неавтономных регистров сдвига с~внутренним 
суммированием}

\def\aut{С.\,Ю.~Мельников$^1$, К.\,Е.~Самуйлов$^2$}

\def\autkol{С.\,Ю.~Мельников, К.\,Е.~Самуйлов}

\titel{\tit}{\aut}{\autkol}{\titkol}

\index{Мельников С.\,Ю.}
\index{Самуйлов К.\,Е.}
\index{Melnikov S.\,Yu.}
\index{Samouylov K.\,E.}
 

{\renewcommand{\thefootnote}{\fnsymbol{footnote}} \footnotetext[1]
{Публикация подготовлена при поддержке Программы РУДН <<5-100>> 
(К.\,Е.~Самуйлов, постановка задачи) и~при финансовой поддержке РФФИ (проекты 
18-00-01555, 18-00-01685 и~19-07-00933).}}


\renewcommand{\thefootnote}{\arabic{footnote}}
\footnotetext[1]{Российский университет дружбы народов, melnikov@linfotech.ru}
\footnotetext[2]{Российский университет дружбы народов, ksam@sci.pfu.edu.ru}

%\vspace*{-6pt}

  \Abst{Проводится сравнение статистических и~алгебраических свойств двоичных 
неавтономных регистров сдвига и~регистров сдвига с~внутренним суммированием, при 
шаге которых вектор состояния суммируется со своим сдвигом на один шаг. Доказан 
изоморфизм графов переходов этих автоматов. Показано, что при бернуллиевском 
случайном входе стационарное распределение на состояниях регистра с~внутренним 
суммированием равномерное. Получен вид вероятностной функции этих регистров. 
Показано, что при определенных ограничениях на функцию выходов регистры 
с~внутренним суммированием не че\-за\-ро\-во-на\-след\-ст\-вен\-ные. Предъявлены входные 
последовательности, которые обладают свойством устойчивости относительных частот 
произвольных мультиграмм, в~то время как выходные последовательности таким свойством 
не обладают.} 
  \KW{автомат со случайным входом; регистр сдвига; граф де Брейна; вероятностная 
функция}
  
\DOI{10.14357/19922264200211}
 
 
%\vspace*{9pt}


\vskip 10pt plus 9pt minus 6pt

\thispagestyle{headings}

\begin{multicols}{2}

\label{st\stat}
  
   
\section{Введение}

  При конструировании генераторов случайных последовательностей широко 
используются как линейные, так и~нелинейные регистры сдвига с~теми или 
иными элементами усложнения или обратной связи~[1]. Это во многом 
обусловлено совокупностью <<хороших>> комбинаторных и~структурных 
свойств графов де Брейна, описывающих преобразования информации в~таких 
регистрах.
  
  Пусть $f(x_1,x_2,\ldots , x_n)$~--- булева функция от~$n$ двоичных 
переменных, $n\hm= 1,2,\ldots$ Будем рас\-смат\-ри\-вать два автомата Мура, 
множество состояний каждого из которых есть $V_n\hm= \{0,1\}^n$, входной 
и~выходной алфавиты~--- множества $\{0,1\}$, выходом служит значение 
функции~$f$ от текущего состояния. 
  
  Автомат (регистр сдвига), который обозначим $A_f$, под действием 
входного символа $a_0\hm\in \{0,1\}$ из состояния $(a_1, a_2, \ldots ,a_n)$ 
переходит в~состояние $(a_0, a_1, \ldots , a_{n-1})$.
   
  Регистром сдвига с~внутренним суммированием назовем автомат, который 
под действием входного символа $a_0\hm\in \{0,1\}$ из состояния 
$(a_1,a_2,\ldots , a_n)$ переходит в~состояние $(a_0\oplus a_1, a_1\oplus 
a_2,\ldots , a_{_n-1}\oplus a_n)$, где $\oplus$~--- суммирование по модулю~2. 
Такой автомат будем обозначать~$A_f^{\oplus}$.
  
  Алгебраические и~статистические свойства обычных регистров хорошо 
изучены. Тео\-ре\-ти\-ко-ав\-то\-мат\-ные свойства регистров, 
аналогичных~$A_f^{\oplus}$, в~автономном случае рассматривались в~[2], 
а~вопросы их аппаратной реализации~--- в~[3]. В~настоящей работе доказывается 
изоморфизм графов переходов и~проводится сравнение статистических свойств 
автоматов~$A_f$ и~$A_f^{\oplus}$.
  
\section{Связь графов переходов автоматов~$A_f$ 
и~$A_f^{\oplus}$}

  Графом переходов автомата~$A_f$, как нетрудно видеть, служит 
ориентированный граф с~множеством вершин~$V_n$, с~дугами, выходящими 
из вершин $(a_1, a_2, \ldots , a_n)$ и~заходящими в~вершины 
$(a_0,a_1,a_2,\ldots , a_{n-1})$, $a_i\hm\in \{0,1\}$. Это хорошо известный граф 
де Брейна, который будем обозначать~$G_n$.
  
  Граф переходов регистра сдвига с~внутренним суммированием 
с~накопителем размера $n\hm=1,2,\ldots$~--- это ориентированный граф 
с~множеством вершин~$V_n$, содержащий дуги, выходящие из вершин $(a_1, 
a_2, \ldots , a_n)$, и~заходящие в~вершины $(a_0\oplus a_1, a_1\oplus a_2, \ldots 
, a_{n-1}\oplus a_n)$, $a_i\hm\in \{0,1\}$. Такой граф будем 
обозначать~$\Gamma_n$. Графы~$G_3$ и~$\Gamma_3$ пред\-став\-ле\-ны на 
рисунке.
  
  \begin{figure*} %fig1
\vspace*{1pt}
 \begin{center}
 \mbox{%
 \epsfxsize=158.296mm 
\epsfbox{mel-1.eps}
 }
\vspace*{6pt}

  {\small Графы $G_3$~(\textit{а}) и~$\Gamma_3$~(\textit{б})}
 \end{center}
\vspace*{-12pt}
  \end{figure*}
  
  Изоморфизм графов~$G_n$ и~$\Gamma_n$ может быть задан с~помощью 
комбинаторных формул обращения~[4].
  
  \smallskip
  
  \noindent
  \textbf{Утверждение~1.}\ \textit{Пусть $(a_1, a_2, \ldots , a_n)$, $a_i\hm\in 
\{0,1\}$,~--- вершина графа~$G_n$, $(b_1, b_2, \ldots , b_n)$, $b_i\hm\in 
\{0,1\}$,--- вершина графа~$\Gamma_n$. Отображение $\varphi_n: V_n\hm\to 
V_n$, определяемое формулой
  $\varphi_n (a_1, a_2, \ldots , a_n)\hm= (b_1, b_2, \ldots , b_n)$, где}
  $$
  b_t=\sum\limits^n_{j=t}  
\begin{pmatrix}  
n-t\\ n-j
  \end{pmatrix} a_j\,\mathrm{mod}\,2\,,\enskip t=1,\ldots ,n\,,
  $$
\textit{является изоморфизмом графов~$G_n$ и~$\Gamma_n$}.

\smallskip

\noindent
  Д\,о\,к\,а\,з\,а\,т\,е\,л\,ь\,с\,т\,в\,о\,.\ \ Отображение~$\varphi_n$~--- это 
линейное преобразование пространства~$V_n$ с~треугольной матрицей, 
составленной из биномиальных коэффициентов:
  $$\
  A=\left(
  \begin{array}{cccc}
  \begin{pmatrix} n-1\\ n-1\end{pmatrix}&
  \begin{pmatrix} n-1\\ n-2\end{pmatrix} &\cdots& \begin{pmatrix} n-1\\ 
0\end{pmatrix}\\[9pt]
  0&\begin{pmatrix} n-2\\ n-2\end{pmatrix} & \cdots & 
  \begin{pmatrix} n-2\\ 0\end{pmatrix}\\[6pt]
  \cdots &\cdots &\cdots &\cdots\\[6pt]
 0& 0 &\cdots &\begin{pmatrix}  0 \\ 0\end{pmatrix}
  \end{array}
  \right) 
  \left( \mathrm{mod}\,2\right)\,.
    $$
  %
  Поскольку $\mathrm{Det}\,(A)\hm=1$, отображение~$\varphi_n$ биективно.
  
  Осталось показать, что данное отображение соседние вершины графа~$G_n$ 
переводит в~соседние вершины графа~$\Gamma_n$. Пусть вершины $(a_1, a_2, 
\ldots , a_n)$ и~$(a_0, a_1,\ldots , a_{n-1})$ соединены дугой в~графе~$G_n$. 
Покажем, что их образы $\varphi_n (a_1, a_2, \ldots , a_n)$ 
и~$\varphi_n(a_0,a_1,\ldots , a_{n-1})$ также соединены дугой 
в~графе~$\Gamma_n$.
  
  Рассмотрим $k$-ю и~($k+1$)-ю координаты первого вектора  
и~($k +1$)-ю координату второго, $k\hm= 1,2,\ldots , n-1$:

\noindent
  \begin{align*}
  \left[ \varphi_n\left(a_1,a_2,\ldots ,a_n\right)\right]_k&= \sum\limits_{i=k}^n 
\begin{pmatrix}
  n-k\\ n-i\end{pmatrix} a_i\,;\\
  \left[ \varphi_n\left(a_1,a_2,\ldots ,a_n\right)\right]_{k+1}&= 
\sum\limits_{i=k+1}^n \begin{pmatrix}
  n-k-1\\ n-i\end{pmatrix} a_i\,;\\
  \left[ \varphi_n\left(a_0,a_1,\ldots , 
a_{n-1}\right)\right]_{k+1}&= \displaystyle\sum\limits_{i=k+1}^n \begin{pmatrix}
  n-k-1\\ n-i\end{pmatrix} a_{i-1}\,,\\
  \end{align*}
  
  Покажем, что для $k\hm= 1,2,\ldots , n-1$ выполняется соотношение:
  
  \noindent
  \begin{multline*}
  \left[ \varphi_n\left( a_1,a_2,\ldots , a_n\right)\right]_k+
  \left[ \varphi_n\left( a_1,a_2,\ldots , a_n\right) \right]_{k+1}={}\\
  {}=\left[ \varphi_n\left( a_0,a_1,\ldots , a_{n-1}\right)\right]_{k+1} 
  (\mathrm{mod}\,2)\,.
  \end{multline*}
  
  В самом деле: 
  
  \noindent
  \begin{multline*}
  \left[ \varphi_n\left( a_1,a_2,\ldots , a_n\right)\right]_k+ 
  \left[ \varphi_n\left( a_1,a_2,\ldots , a_n\right) \right]_{k+1}={}\\
  {}=\sum\limits^n_{i=k} \begin{pmatrix} n-k\\ n-i\end{pmatrix} 
  a_i\left( \mathrm{mod}\,2\right)+{}\\
  {} +\sum\limits^n_{i=k+1} \begin{pmatrix} n-k-1\\ n-i \end{pmatrix} 
  a_i \left(\mathrm{mod}\,2\right)={}
  \end{multline*}
  
\noindent
  \begin{multline*}
    {}=\sum\limits^n_{i=k} \left( \begin{pmatrix} n-k\\ n-i\end{pmatrix}+ 
 \begin{pmatrix} n-k-1\\ n-i\end{pmatrix}\right) 
 a_i \left( \mathrm{mod}\,2\right) ={}\\
  {}= 
  \sum\limits^n_{i=k} \begin{pmatrix} n-k-1\\ n-i-1\end{pmatrix} 
  a_i\left( \mathrm{mod}\,2\right)={}\\
  {}=\sum\limits^n_{j=k+1} \begin{pmatrix} n-k-1\\ n-j\end{pmatrix} 
  a_{j-1} \left(\mathrm{mod}\,2\right)={}\\
  {}=
  \left[ \varphi_n\left( a_0, a_1, \ldots , a_{n-1}\right)\right]_{k+1}\,.
  \end{multline*}
  %
  Здесь использовалось соотношение: 
  $$
  \begin{pmatrix}
  a\\b\end{pmatrix} +\begin{pmatrix}
  a-1\\b \end{pmatrix}= \begin{pmatrix}
  a-1\\ b-1 \end{pmatrix} \left( \mathrm{mod}\,2\right)\,,
  $$
  вытекающее из известного комбинаторного тождества
  $$
  \begin{pmatrix}
  a\\b\end{pmatrix}=\begin{pmatrix}
  a-1\\ b \end{pmatrix}+\begin{pmatrix}
  a-1\\ b-1\end{pmatrix}\,.
  $$
  
  
  \noindent
  \textbf{Пример.}\ $n=4$. Изоморфизм $G_4\hm\cong \Gamma_4$ 
описывается следующим образом:
  $$
  A=\!\begin{pmatrix} 
  1&3&3&1\\
  0&1&2&1\\
  0&0&1&1\\
  0&0&0&1\end{pmatrix}\!\left(\mathrm{mod}\,2\right)\!;\enskip
  \varphi_4=\!\begin{pmatrix}
  a_1\oplus a_2\oplus a_3\oplus a_4\\
  a_2\oplus a_4\\
  a_3\oplus a_4\\
  a_4 \end{pmatrix}\!.
  $$
  
  \noindent
  \textbf{Утверждение~2.}\ \textit{Справедливо равенство} $\varphi_n^{-
1}\hm=\varphi_n$, $n\hm=1,2,\ldots$
  
  \smallskip
  
  \noindent
  Д\,о\,к\,а\,з\,а\,т\,е\,л\,ь\,с\,т\,в\,о\,.\ \  
В~\cite[п.~2.2]{5-mel} показано, что мат\-ри\-ца, обратная к
  $$
  A=\displaystyle \left( \begin{pmatrix} n-1-i\\  
n-1-j\end{pmatrix}\right)_{i,j=0}^{n-1},
$$
 имеет вид: 
 $$
 A^{-1}= 
\displaystyle\left( (-1)^{i+j}\begin{pmatrix} n-1-i\\ n-1-
j\end{pmatrix}\right)_{i,j=0}^{n-1}.
$$
 По модулю два эти матрицы равны.
  
  В~[6] доказано, что у~графа~$G_n$ существуют ровно два автоморфизма, 
тождественный и~соответствующий инверсии координат вершин.
  
  \smallskip
  
  \noindent
  \textbf{Утверждение~3.} \textit{Пусть~$\phi$ и~$\eta$~--- преобразования 
линейного пространства~$V_n$ вида}:
  \begin{align*}
  \phi\left(x_1,x_2,\ldots , x_n\right)&= 
  \left( \overline{x_1},\overline{x_2},\ldots , \overline{x_n}\right)\,;\\
  \eta\left(x_1,x_2,\ldots , x_n\right) &= \left( x_1, x_2,\ldots ,  
x_{n-1}, \overline{x_n}\right)\,.
  \end{align*}
\textit{Тогда} 
\begin{itemize}
\item[\,] $\phi: G_n\to G_n$~--- \textit{автоморфизм графа}~$G_n$;
\item[\,] $\eta: \Gamma_n\to\Gamma_n$--- \textit{автоморфизм 
графа}~$\Gamma_n$,
\end{itemize}
\textit{диаграмма} 
$$
\begin{array}{ccc}
G_n&\overset{\varphi_n}{\cong}&\Gamma_n\\[3pt]
\downarrow\ \ \phi & & \downarrow\ \ \eta\\[3pt]
G_n & \overset{\varphi_n}{\cong}&\Gamma_n
\end{array}
$$
\textit{коммутативна}.
  
  \smallskip
  
  \noindent
  Д\,о\,к\,а\,з\,а\,т\,е\,л\,ь\,с\,т\,в\,о\,.\ \ Очевидно, что инверсия координат 
вершин задает автоморфизм графа де Брейна. Для доказательства 
коммутативности диаграммы и~того, что отображение $\eta: \Gamma_n\to 
\Gamma_n$ есть\linebreak автоморфизм графа~$\Gamma_n$, равенство 
$\phi\varphi_n\hm=\varphi_n\eta$ проверяется непостредственно. 
  
\section{Вероятностная функция двоичных регистров сдвига 
с~внутренним суммированием}

  
  Вероятностная функция~[7] конечного сильносвязного автомата определяется 
как действительная функция, заданная на множестве стохастических векторов, 
соответствующих возможным \mbox{полиномиальным} распределениям на входном 
алфавите. Значение функции есть предел относительной частоты встречаемости 
знака в~выходной последовательности автомата в~предположении, что на его 
вход поступает последовательность независимых одинаково распределенных по 
заданной полиномиальной схеме случайных величин. В~двоичном случае 
входная последовательность бернуллиевская с~параметром~$p$, $0\hm< p 
\hm<1$. В~качестве знака выходной последовательности рассматривается 
единица. Для вывода формулы вероятностной функции 
автомата~$A_f^{\oplus}$ получим стационарное распределение на состояниях 
автомата. Определим случайное блуж\-да\-ние на графе~$\Gamma_n$ следующим 
образом. Начальная вершина выбирается в~соответствии с~некоторой 
полиномиальной схемой на множестве~$V_n$. Шаг блуждания проходит по 
одной из исходящих из нее дуг. Предположим, что переходы из вершины 
в~вершину независимы и~вероятность шага блуждания из вершины $(a_1, a_2, 
\ldots , a_n)$ в~вершину 
$(1\oplus a_1, a_1\oplus a_2,\ldots , a_{n-1}\oplus a_n)$ 
равна~$p$, а~в~вершину $(a_1,a_1\oplus a_2,\ldots ,  
a_{n-1}\oplus a_{n})$ равна $1-p$, $0\hm< p\hm<1$. 
  
  \smallskip
  
  \noindent
  \textbf{Утверждение~4.} \textit{Стационарное распределение описанного 
случайного блуждания имеет вид}:
  $$
  P\left( a_1, a_2,\ldots , a_n\right)=\fr{1}{2^n}\,.
  $$
  
  
  \noindent
  Д\,о\,к\,а\,з\,а\,т\,е\,л\,ь\,\,с\,т\,в\,о\,.\ \ Покажем, что матрица переходных 
вероятностей марковской цепи, опи\-сы\-ва\-ющей рассматриваемое случайное 
блуждание, дваж\-ды стохастическая. Ориентированный граф~$\Gamma_n$~--- 
регулярный степени~2, т.\,е.\ из каждой вершины выходят две дуги и~в~каждую 
вершину заходят две дуги, что вытекает из доказанного выше 
изоморфизма~$\Gamma_n$ и~$G_n$. Рассмотрим вершину $(a_1, a_2, \ldots , 
a_n)$. В~нее можно попасть из двух разных вершин, которые обозначим 
$(b_1,b_2,\ldots , b_n)$ (по дуге, которая помечена символом~$b_0$) 
и~$(c_1,c_2,\ldots ,c_n)$ (по дуге, которая помечена символом~$c_0$). Тогда 
справедливо соотношение:
\begin{multline*}
  (b_1,b_2,\ldots , b_n)\oplus (b_0,b_1,\ldots , b_{n-1})= {}\\
  {}=(c_1,c_2,\ldots , c_n)\oplus (c_0,c_1,\ldots , c_{n-1})\,.
\end{multline*}
  
  Поскольку $(b_1,b_2,\ldots ,b_n)\not= (c_1,c_2,\ldots , c_n)$, то из этого 
соотношения нетрудно вывести, что $b_0\not= c_0$. Это означает, что одна из 
дуг, заходящих в~вершину $(a_1, a_2, \ldots , a_n)$, помечена символом~1, 
а~другая~--- символом~0. Следовательно, в~матрице переходных 
вероятностей в~столбце, соответствующем вершине $(a_1, a_2, \ldots , a_n)$, 
расположено ровно два ненулевых элемента: $p$ и~$1\hm-p$. Поэтому данная 
матрица~--- дваж\-ды стохастическая, что означает равномерность стационарного 
распределения. 
  
  \smallskip
  
  \noindent
  \textbf{Утверждение~5.}\ \textit{Вероятностная функция регистра сдвига 
с~внутренним суммированием с~выходной функцией~$f$ имеет вид}:
  $$
  P_{A_f^{\oplus}}(p)=\fr{\| f\|}{2^n}\,,
  $$ 
  где $\|f\|$~--- вес функции~$f$:
  $$
  \|f\|= \sum\limits_{(x_1,x_2,\ldots ,x_n)\in V_n} f\left(x_1,x_2,\ldots , x_n\right)\,.
  $$

\section{Чезарово-наследственность регистра сдвига 
с~внутренним суммированием}

  В~[8] дано следующее определение: слово называется чезаровским для 
бесконечной последовательности, если определен предел относительной 
частоты его встречаемости в~растущих начальных отрезках этой 
последовательности. Последовательность над некоторым алфавитом 
называется чезаровской, если произвольное слово над этим алфавитом является 
для нее чезаровским. Конечный автомат называют че\-за\-ро\-во-на\-след\-ст\-вен\-ным, 
если из любого начального состояния чезаровские 
последовательности во входном алфавите он перерабатывает в~чезаровские 
последовательности в~выходном алфавите. В~[8] показано, что при любой 
функции~$f$ автомат~$A_f$ че\-за\-ро\-во-на\-след\-ст\-вен\-ный.
  
  \smallskip
  
  \noindent
  \textbf{Утверждение~6.}\ \textit{Если для булевой функции $f(x_1,x_2,\ldots , 
x_n)$ выполнено условие 
$$
f(0,0,\ldots ,0)\not= f(0,0,\ldots , 0,1),
$$ 
то~$A_f^{\oplus}$ не  
че\-за\-ро\-во-на\-след\-ст\-вен\-ный автомат}.
  
  \smallskip
  
  \noindent 
  Д\,о\,к\,а\,з\,а\,т\,е\,\,л\,ь\,с\,т\,в\,о\,.\ \ В~графе автомата~$A_f$, как нетрудно 
видеть, существует ровно один цикл, движение по которому происходит при 
подаче на вход последовательности, состоящей только из нулей, это петля 
в~нулевой вершине. Для автомата~$A_f^{\oplus}$ таких циклов существует 
несколько. В~самом деле, поскольку при нулевом символе на входе автомата 
состояние $(a_1, a_2, \ldots , a_n)$ переходит в~состояние $(a_1,a_1\oplus 
a_2,\ldots , a_{n-1}\oplus a_n)$, условие наличия цикла длины~  (петли) 
выглядит так: $a_2\hm= a_1\oplus a_2, a_3\hm=a_2\oplus a_3,\ldots , 
a_n\hm=a_{n-1}\oplus a_n$. Отсюда получаем равенства: $a_i\hm=0$, 
$i\hm=1,2,\ldots , n-1$, $a_n\hm=0,1$. Поэтому в~графе 
автомата~$A_f^{\oplus}$ имеются две петли, движение по которым происходит 
при подаче~  на вход: петля в~вершине $(0,0,\ldots ,0)$ и~петля в~вершине 
$(0,0,\ldots , 0,1)$ (см.\ рисунок). 
  
  Пусть теперь выполнено условие утверждения. Из него следует, что при 
движении по одной из указанных выше петель выходная последовательность 
автомата~$A_f^{\oplus}$ состоит из единиц, а~при движении по другой эта 
последовательность состоит из нулей. Поскольку диаметр графа~$\Gamma_n$ 
равен~$n$, из состояния $(0,0,\ldots ,0)$ в~состояние $(0,0,\ldots , 0,1)$ 
и~наоборот можно перейти за не более чем $n$ шагов. Через~$\xi_{01}$ 
и~$\xi_{10}$   обозначим входные последовательности, которые 
обеспечивают эти переходы. Рассмотрим теперь бесконечную входную 
последовательность
  $$
  \chi=\xi_0\, 0^{k_1}\, \xi_{01}\, 0^{k_2}\, \xi_{10}\,
   0^{k_3}\, \xi_{01}\, 0^{k_4} \,
\xi_{10}\ldots ,
  $$
где $\xi_0$~--- входная последовательность, переводящая 
автомат~$A_f^{\oplus}$ из заданного состояния в~состояние $(0,,0,\ldots ,0)$; 
$0^{k_i}$~--- последовательность, состоящая из~$k_i$ нулей, $k_i$~--- целые 
числа, $i\hm=1,2,\ldots$ Выходной последовательностью, очевидно, будет
$$
\gamma= \zeta_0 \,0^{k_1} \,\zeta_{01}\, 1^{k_2}\, \zeta_{10}\, 0^{k_3}\, \zeta_{01} \,
1^{k_4}\, \zeta_{10}\ldots\,,
$$
где $\zeta_0$, $\zeta_{01}$ и~$\zeta_{10}$~--- некоторые двоичные 
последовательности, длина каждой из которых не превосходит~$n$; 
$1^{k_j}$~--- последовательность, состоящая из $k_j$ единиц.
  
  Нетрудно убедиться в~том, что при $k_i\hm= 2^{2^i}$ 
последовательность~$\xi$ чезаровская, а~для последовательности~$\gamma$ не 
существует предела отностительной частоты встречаемости единицы 
в~растущих начальных отрезках. Поэтому она не чезаровская 
и~автомат~$A_f^{\oplus}$ не че\-за\-ро\-во-на\-след\-ст\-вен\-ный.
  
\section{Заключение}

  Регистры сдвига различных типов широко применяются в~качестве узлов 
генераторов случайных последовательностей. В~работе исследованы 
алгебраические и~статистические свойства семейства двоичных регистров 
сдвига с~внутренним суммированием. Доказан изоморфизм графов переходов 
регистров сдвига с~внутренним суммированием и~обычных регистров. 
Доказано, что, в~отличие от\linebreak обычного регистра сдвига, при случайном 
бернуллиевском входе вероятностное распределение на состояниях регистра 
с~внутренним суммированием равномерное. Показано, что рассматриваемые 
регистры не обладают свойством че\-за\-ро\-во-на\-след\-ст\-вен\-ности, которым 
обладают обычные регистры сдвига. 
  
{\small\frenchspacing
 {%\baselineskip=10.8pt
 \addcontentsline{toc}{section}{References}
 \begin{thebibliography}{9}
  
\bibitem{1-mel}
\Au{Грушо А.\,А., Применко Э.\,А., Тимонина~Е.\,Е.} Теоретические основы компьютерной 
безопасности.~--- М.: 
Академия, 2009. 267~с.
\bibitem{2-mel}
\Au{Golomb S.\,W.} Shift register sequences.~--- Laguna Hills, CA, USA: Aegean Park Press, 
1981. 247~p.
\bibitem{3-mel}
The VLSI handbook~/ Ed. W.-K.~Chen.~--- 2nd ed.~--- Chicago, IL, USA: CRC Press, 2006. 2320~p. 
\bibitem{4-mel}
\Au{Сачков В.\,Н.} Курс комбинаторного анализа.~--- Ижевск: НИЦ РХД, 2013. 336~с. 
\bibitem{5-mel}
\Au{Риордан Дж.} Комбинаторные тождества~/ Пер. с~англ.~--- М.: Наука, 1982. 256~с. 
(\Au{Riordan~J.} 1968. Combinatorial identities.~--- New York, NY, USA: Wiley. 256~p.)
\bibitem{6-mel}
\Au{Liu M.} Homomorphisms and automorphisms of 2-D de Bruijn--Good graphs~// Discrete 
Math., 1990. Vol.~85. Iss.~1. P.~105--109. 
\bibitem{7-mel}
\Au{Melnikov S.\,Yu., Samouylov~K.\,E.} The recognition of the output function of a~finite 
automaton with random input~// Distributed Computer and Communication Networks: 
21st Conference (International) Revised Selected Papers~/ Eds. 
V.\,M.~Vishnevskiy, D.\,V.~Kozyrev.~--- Communications in computer and information 
science ser.~--- Springer, 2018. Vol.~919. P.~525--531. doi: 10.1007/978-3-319-99447-5\_45.
\bibitem{8-mel}
\Au{Мельников С.\,Ю.} О переработке конечными автоматами чезаровских 
последовательностей~// Лесной вестник, 2004. №\,1(32). 
С.~169--174.
\end{thebibliography}

 }
 }

\end{multicols}

\vspace*{-9pt}

\hfill{\small\textit{Поступила в~редакцию 14.04.20}}

\vspace*{6pt}

%\pagebreak

%\newpage

%\vspace*{-28pt}

\hrule

\vspace*{2pt}

\hrule

%\vspace*{-2pt}

\def\tit{STATISTICAL PROPERTIES OF~BINARY NONAUTONOMOUS SHIFT REGISTERS 
WITH~INTERNAL XOR}


\def\titkol{Statistical properties of binary nonautonomous shift registers with 
internal XOR}

\def\aut{S.\,Yu.~Melnikov and K.\,E.~Samouylov}

\def\autkol{S.\,Yu.~Melnikov and K.\,E.~Samouylov}

\titel{\tit}{\aut}{\autkol}{\titkol}

\vspace*{-15pt}


\noindent
Peoples' Friendship University of Russia (RUDN University), 6~Miklukho-Maklaya Str., Moscow 117198, Russian 
Federation

\def\leftfootline{\small{\textbf{\thepage}
\hfill INFORMATIKA I EE PRIMENENIYA~--- INFORMATICS AND
APPLICATIONS\ \ \ 2020\ \ \ volume~14\ \ \ issue\ 2}
}%
 \def\rightfootline{\small{INFORMATIKA I EE PRIMENENIYA~---
INFORMATICS AND APPLICATIONS\ \ \ 2020\ \ \ volume~14\ \ \ issue\ 2
\hfill \textbf{\thepage}}}

\vspace*{3pt} 


\Abste{ The statistical and algebraic properties of binary nonautonomous shift registers and shift registers with internal 
XOR are compared, during which the state vector is summed with its one-step shift. The isomorphism of transition 
graphs of these automata is proved. It is shown that, with a~Bernoulli random input, the stationary distribution of the 
register states with internal XOR is uniform. The form of the probability function of these registers is obtained. It is 
shown that, under certain conditions on the output function, registers with internal XOR are not Cesaro-hereditary. The 
authors show input sequences that possess the property of stability of the relative frequencies of arbitrary multigrams, 
while output sequences do not have this property.}

\KWE{random input automata; shift register; de Bruijn graph}


\DOI{10.14357/19922264200211} 

\vspace*{-20pt}

\Ack
\noindent
 The publication has been prepared with the support of the ``RUDN University 
Program 5-100'' (recipient K.\,E.~Samouylov, problem statement). The reported 
study was partially funded by the Russian Foundation for Basic Research, projects  
18-00-01555, 18-00-01685, and 19-07-00933.


\vspace*{5pt}

 \begin{multicols}{2}

\renewcommand{\bibname}{\protect\rmfamily References}
%\renewcommand{\bibname}{\large\protect\rm References}

{\small\frenchspacing
 {%\baselineskip=10.8pt
 \addcontentsline{toc}{section}{References}
 \begin{thebibliography}{9}
 
\vspace*{-8pt}

\bibitem{1-mel-1}
\Aue{Grusho, A.\,A., E.\,A. Primenko, and E.\,E.~Timonina.} 2009. 
\textit{Teoreticheskie osnovy komp'yuternoy bezopasnosti} 
[Theoretical foundations of computer 
security]. Moscow: Akademiya. 267~p.
\bibitem{2-mel-1}
\Aue{Golomb, S.\,W.} 1981. \textit{Shift register sequences}. Laguna Hills, CA: 
Aegean Park Press. 247~p.
\bibitem{3-mel-1}
Chen, W.-K., ed. 2006. \textit{The VLSI handbook}. 2nd ed. Chicago, IL: CRC Press. 
2320~p.

\pagebreak 
 
\bibitem{4-mel-1}
\Aue{Sachkov, V.\,N.} 2013. \textit{Kurs kombinatornogo analiza} [Combinatorial 
analysis course]. Izhevsk: NITS RKHD. 336~p.
\bibitem{5-mel-1}
\Aue{Riordan, J.} 1968. \textit{Combinatorial identities}. New York, NY: Wiley. 
256~p.
\bibitem{6-mel-1}
\Aue{Liu, M.} 1990. Homomorphisms and automorphisms of \mbox{2-D} de Bruijn-good 
graphs. \textit{Discrete Math.} 85(1):105--109.
\bibitem{7-mel-1}
\Aue{Melnikov, S.\,Yu., and K.\,E.~Samouylov.} 2018. The recognition of the output 
function of a~finite automaton with random input. \textit{Distributed Computer and 
Communication Networks: 21st Conference (International) Revised Selected 
Papers}. Eds. V.\,M.~Vishnevskiy and D.\,V.~Kozyrev. Communications in 
computer and information science ser. Springer. 919:525--531.
\bibitem{8-mel-1}
\Aue{Melnikov, S.\,Yu.} 2004. O~pererabotke konechnymi avtomatami 
chezarovskikh posledovatel'nostey [On the
 finite automa with Cesaro sequences input]. 
\textit{Lesnoy Vestnik} [Forestry Bull.] 1(32):169--174.

\end{thebibliography}

 }
 }

\end{multicols}

\vspace*{-9pt}

\hfill{\small\textit{Received April 14, 2020}}

%\pagebreak



\Contr

\noindent
\textbf{Melnikov Sergey Yu.} (b.\ 1963)~--- Candidate of Science (PhD) in physics 
and mathematics, doctoral student, Peoples' Friendship University of Russia (RUDN 
University), 6~Miklukho-Maklaya Str., Moscow 117198, Russian Federation; 
\mbox{melnikov@linfotech.ru}

\vspace*{3pt}

\noindent
\textbf{Samouylov Konstantin E.} (b.\ 1955)~--- Doctor of Science in technology, 
professor, Head of Department, Peoples' Friendship University of Russia (RUDN 
University), 6~Miklukho-Maklaya Str., Moscow 117198, Russian Federation; 
\mbox{ksam@sci.pfu.edu.ru}


\label{end\stat}

\renewcommand{\bibname}{\protect\rm Литература}   %9
\def\stat{grusho}

\def\tit{АРХИТЕКТУРНЫЕ РЕШЕНИЯ В~ЗАДАЧЕ ВЫЯВЛЕНИЯ МОШЕННИЧЕСТВА ПРИ~АНАЛИЗЕ 
ИНФОРМАЦИОННЫХ ПОТОКОВ В~ЦИФРОВОЙ ЭКОНОМИКЕ$^*$}

\def\titkol{Архитектурные решения в~задаче выявления мошенничества при~анализе 
информационных потоков в
%~цифровой 
экономике}

\def\aut{А.\,А.~Грушо$^1$, М.\,И.~Забежайло$^2$, Н.\,А.~Грушо$^3$, 
Е.\,Е.~Тимонина$^4$}

\def\autkol{А.\,А.~Грушо, М.\,И.~Забежайло, Н.\,А.~Грушо, 
Е.\,Е.~Тимонина}

\titel{\tit}{\aut}{\autkol}{\titkol}

\index{Грушо А.\,А.}
\index{Забежайло М.\,И.}
\index{Грушо Н.\,А.}
\index{Тимонина Е.\,Е.}
\index{Grusho A.\,A.}
\index{Zabezhailo M.\,I.}
\index{Grusho N.\,A.}
\index{Timonina E.\,E.}


{\renewcommand{\thefootnote}{\fnsymbol{footnote}} \footnotetext[1]
{Работа частично поддержана РФФИ (проекты 18-29-03081 и~18-07-00274).}}


\renewcommand{\thefootnote}{\arabic{footnote}}
\footnotetext[1]{Институт проблем информатики Федерального исследовательского центра <<Информатика и~управление>> 
Российской академии наук, grusho@yandex.ru}
\footnotetext[2]{Институт проблем информатики Федерального исследовательского центра <<Информатика и~управление>> 
Российской академии наук, m.zabezhailo@yandex.ru}
\footnotetext[3]{Институт проблем информатики Федерального исследовательского центра <<Информатика и~управление>> 
Российской академии наук, info@itake.ru}
\footnotetext[4]{Институт проблем информатики Федерального исследовательского центра <<Информатика и~управление>> 
Российской академии наук, eltimon@yandex.ru}

\vspace*{-12pt}
   

 
  
  \Abst{Cформулирован подход к~исследованию некоторых видов мошенничества в~цифровой 
экономике с~использованием причинно-следственных связей. Во всех видах рассматриваемых 
мошенничеств должно наблюдаться несоответствие между целями финансовых транзакций 
и~реальной стоимостью достижения этих целей. Данные о транзакциях можно собирать, 
наблюдая информационные потоки, в~которых отражаются эти транзакции. Архитектура сбора 
данных и~их анализа может быть организована с~помощью распределенных реестров 
с~централизованным консенсусом, что позволяет создать аналог электронной бухгалтерской 
книги, фиксирующей финансово-экономическую деятельность субъектов цифровой экономики в~регионе. 
  Рассматриваемые методы выявления мошенничества основаны на противоречиях 
между действиями, описанными в~транзакциях, и~информацией, содержащейся в~планах, 
стандартах, прецедентах и~др. Рассмотрен метод, основанный на некоторой упрощенной схеме 
реализации абстрактного проекта. Для выявления противоречий необходимо проводить анализ 
от следствия к~причине, т.\,е.\ искать аномалии в~информации, описывающей порождение 
наблюдаемых следствий. 
  Показано, как в~реализации проекта можно выделять простые <<необходимые условия>> 
нарушения при\-чин\-но-след\-ст\-вен\-ных связей, т.\,е.\ множество <<необходимых условий>>, 
нарушение которых свидетельствует о наличии мошенничества. Это множество <<необходимых 
условий>> можно назвать метаданными для контроля проекта на выявление мошенничества.} 
 
 
  \KW{цифровая экономика; информационные потоки; при\-чин\-но-след\-ст\-вен\-ные связи; 
выявление мошеннических схем} 

\DOI{10.14357/19922264190204}
  
\vspace*{-4pt}


\vskip 10pt plus 9pt minus 6pt

\thispagestyle{headings}

\begin{multicols}{2}

\label{st\stat}

\section{Введение}

\vspace*{3pt}

  В работе сформулирован подход к~исследованию некоторых видов 
мошенничества в~цифровой экономике с~использованием  
при\-чин\-но-след\-ст\-вен\-ных связей. Рассматриваются три вида мошенничества, 
а именно:
  \begin{enumerate}[(1)]
\item отмыв денег; 
\item обман при выполнении договорных обязательств при реализации 
технических проектов (строительные проекты и~др.); 
\item незаконный вывод денег. 
\end{enumerate}

  Названные виды мошенничества могут быть сведены к~решению одного типа 
задач. Для отмывания денег источник должен заключать фиктивные контракты, 
в~соответствии с~которыми будут переводиться средства за заведомо ненужную 
работу и~материалы. 
  
  Мошенничество, связанное с~невыполнением договорных обязательств, связано 
со снижением качества услуг, качества и~количества закупаемых 
материалов, выполнением работ с~ненадлежащим качеством. 
  
  Вывод денег связан с~переводом средств фир\-мам-од\-но\-днев\-кам, которые 
заведомо не могут выполнить обязательства по контрактам, за которые им 
переводятся средства. 
  
  Таким образом, во всех трех видах рассматриваемых мошенничеств должно 
наблюдаться несоответствие между целями финансовых транзакций и~реальной 
стоимостью достижения этих целей. Данные о транзакциях можно собирать, 
наблюдая информационные потоки, в~которых отражаются эти транзакции. 
  
  Однако для наблюдения таких информационных потоков необходимо создавать 
архитектуру\linebreak телекоммуникационной системы, позволяющей перехватывать 
и~собирать данные о всех транзакциях. Например, такая архитектура может быть 
организована с~помощью распределенных реестров с~централизованным 
консенсусом, т.\,е.\ все информационные потоки, сформированные в~цифровой 
экономике и~несущие информацию о транзакциях, проходят через некоторый 
центральный узел, запоминающий их в~форме распределенного реестра. Такие 
реестры могут дублироваться в~аналогичных центрах различных регионов, что 
позволяет создать аналог электронной бухгалтерской книги, фиксирующей 
фи\-нан\-со\-во-эко\-но\-ми\-че\-скую деятельность субъектов цифровой экономики. Такой 
подход предложено реализовать на базе системы ситуационных центров, что 
отражено в~работах~[1, 2].
  
  Собранная из информационных потоков информация о~транзакциях, т.\,е.\ 
о~контрактах, договорах, платежах, отчетах, закупленных материалах, 
характеристиках исполнителей работ и~др., собирается в~базе данных в~указанном 
центре. Согласно теории интеллектуальных сис\-тем~[3], эту базу данных можно 
называть базой фактов (БФ). Базу фактов можно представить как бинарную мат\-ри\-цу, 
строки которой описывают характеристики, входящие в~транзакции, а столбцы 
нумеруются характеристиками. Строки матрицы будем называть 
\textit{объектами}~[4, 5]. 
  
  Рассматриваемые в~работе методы выявления мошенничества будут основаны 
на противоречиях между действиями, описанными в~транзакциях, и~информацией, 
содержащейся в~планах, стандартах, прецедентах и~др. Для нахождения 
противоречий в~архитектуре центра предусмотрена другая база данных~--- база 
знаний (БЗ)~\cite{3-gr, 6-gr}, которая устроена так же, как БФ. 
  
  Информация в~БЗ собирается на основе положительного опыта или расчетов. 
Используя БЗ, можно выводить факты нарушения при\-чин\-но-след\-ст\-вен\-ных 
связей. Нарушения при\-чин\-но-след\-ст\-вен\-ных связей будем называть 
\textit{аномалиями}. 
  
  Для упрощения дальнейшее изложение будет вестись в~рамках поиска 
противоречий при выполнении некоторого абстрактного проекта. Выявление 
аномалий будет происходить на основе фактов из БФ с~помощью знаний из БЗ 
методами искусственного интеллекта и~интеллектуального анализа 
данных~\cite{6-gr}. 

\vspace*{-10pt}
  
  \section{Модели}
  
  \vspace*{-3pt}
  
  Наиболее сложная из рассмотренных выше задач~--- выявление противоречий, 
т.\,е.\ использование БЗ для получения новых знаний и~выявление аномалий из 
полученных фактов. 
  
  Все способы выявления противоречий основаны на определении 
  причинно-следственных связей. При этом противоречия в~параметрах транзакций по 
отношению к~требуемым в~БЗ составляют сущность аномалий. 
  
   Далее будет рассмотрен метод, основанный на некоторой упрощенной схеме 
реализации абстрактного проекта. 
  
  Каждый проект имеет цель: например, цель представляет собой построение 
некоторой системы. Воспользуемся структурным подходом, который позволяет 
строить проект на основе разбиения системы на подсистемы и~определения 
взаимодействий подсистем~\cite{7-gr}. При этом каждая подсистема также 
представима структурной моделью. 
  
  Как сама система, так и~каждая ее подсистема имеют свой функционал 
и~спецификацию, па\-ра\-мет\-ры настройки и~домены параметров настройки. Кроме 
этих характеристик существует множество характеристик, связанных 
с~<<жизненным циклом>> создания системы. Сюда входят работы, ресурсы, 
сроки выполнения работ по созданию подсистем и~самой системы, стоимости 
компонентов и~материалов, стоимости работ, схемы поставок, договорные 
обязательства и~др. Все характеристики связаны между собой, поэтому можно 
говорить о стоимости и~времени изготовления структурных компонентов системы. 
  
  Одной из важнейших характеристик является смета (система смет для 
подсистем). Смета сопоставляет каждому компоненту системы стоимость его 
изготовления и~настройки. 
  
  Схема построения системы может быть пред\-став\-ле\-на диаграммой, 
изображенной на рис.~1. 

{ \begin{center}  %fig1
 \vspace*{9pt}
   \mbox{%
 \epsfxsize=79mm 
 \epsfbox{gru-1.eps}
 }


\vspace*{9pt}


\noindent
{{\figurename~1}\ \ \small{Диаграмма достижения цели}}
\end{center}
}

\vspace*{9pt}

\addtocounter{figure}{1}
  
  


  Представленная на рис.~1 диаграмма позволяет описать основные классы 
возможных противоречий при достижении цели. Противоречия возникают, когда 
данные БФ не соответствуют требуемым характеристикам. 
  
  
  \section{Потенциальные классы аномалий при~достижении цели}
  
  Выделим четыре потенциальных класса противоречий, которые показывают, 
каким образом нужно искать эти противоречия.
  
 
  Противоречие цели и~проекта (рис.~2) возникает при отсутствии обоснования 
или в~случае логического противоречия между возможностями проектируемого 
функционала и~целью системы. Отметим, что в~проект входят сроки, перечень 
работ, материалы, настройки, которые описываются соответствующими 
параметрами и~допустимыми значениями этих параметров. Проект формируется 
на основе БЗ и~расчетов, исходя из информации, полученной по аналогии 
с~другими проектами и~решениями, которые считаются апробированными. 
  
  Отметим, что цель порождает проект и~в этом смысле является причиной 
проекта. Однако для анализа противоречий необходимо двигаться по штриховой 
стрелке диаграммы (см.\ рис.~2) от проекта к~цели. В~самом деле, любой компонент 
проекта направлен на теоретическое достижение цели. Цель~--- сложный объект, 
поэтому в~проекте могут возникнуть характеристики, противоречащие хотя бы 
некоторым характеристикам цели. Это делает проект противоречивым, но вывод 
об этом может быть сделан только на уровне описания цели. 
  

  Противоречия между проектом и~его реализацией, исключая настройки 
(рис.~3), могут возникать, например, при закупке исполнителем материалов более 
низкого качества по более низким ценам, при попытках достижения требуемых 
сроков работы за счет снижения качества выполнения работ, за счет нахождения 
<<объективных>> причин для увеличения сроков работы и,~следовательно, 
увеличения цены реализации проекта. 


  Для выявления указанных противоречий необходимо двигаться по диаграмме 
(см.\ рис.~3) в~обратную сторону в~соответствии со~штриховыми стрелками. 
Действительно, выявить противоречия между характеристиками закупленных 
материалов и~требуемыми по проекту можно только при обращении к~проекту 
и~его спецификациям. Манипуляции со сроками работы также можно выявить 
только при обращении к~соответствующим расчетам в~проекте. Задержки в~сроках 
работы, связанные с~поставками материалов, можно определить только на 
предыдущем этапе диаграммы (см.\ рис.~3) в~описании проекта. 


  


  Противоречия между реализацией проекта и~его настройкой (рис.~4) возникает, 
когда не удается добиться требуемых значений параметров функционала, не 
удается обеспечить необходимый уровень\linebreak\vspace*{-12pt}

{ \begin{center}  %fig2
 \vspace*{-6pt}
   \mbox{%
 \epsfxsize=16mm 
 \epsfbox{gru-2.eps}
 }


\vspace*{6pt}


\noindent
{{\figurename~2}\ \ \small{Противоречия цели и~проекта}}
\end{center}
}

%\vspace*{9pt}

\addtocounter{figure}{1}

{ \begin{center}  %fig3
 \vspace*{6pt}
    \mbox{%
 \epsfxsize=79mm 
 \epsfbox{gru-3.eps}
 }


\end{center}

\vspace*{-2pt}


\noindent
{{\figurename~3}\ \ \small{Противоречия проекта и~его реализации (без настройки)}}
}

\vspace*{6pt}

\addtocounter{figure}{1}

{ \begin{center}  %fig4
 \vspace*{1pt}
   \mbox{%
 \epsfxsize=54.5mm 
 \epsfbox{gru-4.eps}
 }


\end{center}


\noindent
{{\figurename~4}\ \ \small{Противоречия реализации проекта и~его на\-стройки}}
}

%\vspace*{9pt}

\addtocounter{figure}{1}

{ \begin{center}  %fig5
 \vspace*{5pt}
    \mbox{%
 \epsfxsize=79mm 
 \epsfbox{gru-5.eps}
 }


\end{center}



\noindent
{{\figurename~5}\ \ \small{Противоречия цели и~достигнутой реализации проекта}}
}

\vspace*{6pt}

\addtocounter{figure}{1}

\noindent
 качества реализации проекта. Для 
определения противоречия в~настройках надо опять же двигаться по диаграмме 
(см.\ рис.~4) в~обратную сторону по штриховым стрелкам, так как для выявления 
характеристик результатов работы, которые не дают возможности реализации 
определенного функционала, необходимо иметь информацию о результатах этой 
работы. 


  



  Противоречие между целью и~достигнутой реализацией проекта (рис.~5) 
возникает, когда реализованная система не позволяет достичь цели. В~этом случае 
опять противоречие нужно искать, двигаясь от цели к~реальному достигнутому 
функционалу по штриховой стрелке (см.\ рис.~5).
  
  Суммируя положения, изложенные в~данном разделе, приходим к~выводу, что 
для выявления противоречий необходимо проводить анализ от следствия 
к~причине, т.\,е.\ искать аномалии в~информации, описывающей порождение 
наблюдаемых следствий. 
  
  
  \section{Связь противоречий и~причин}
  
  Прежде чем построить связь между причинами и~противоречиями, кратко 
опишем простейшую модель связи этих понятий. Причины и~противоречия будут 
сформулированы для представления компонентов системы как объектов, 
обладающих наборами известных характеристик~\cite{4-gr, 5-gr}. 
  
  Пусть $U\hm=\{\alpha, \beta, \ldots\}$~--- совокупность характеристик 
(пространство характеристик). Согласно~\cite{4-gr} \textit{объектом}~$O$ 
называется любое подмножество характеристик $O\hm\subseteq U$. Рассмотрим 
последовательность объектов, возможно в~различных пространствах 
характеристик. 
  
  \smallskip
  
  \noindent
  \textbf{Определение~1.}\ Объект~$P$ с~числом характеристик, большим или 
равным~2, является \textit{причиной} объекта (\textit{свойства})~$B$ в~цепочке 
наблюдаемых объектов тогда и~только тогда, когда выполнены следующие 
условия:
  \begin{enumerate}[(1)]
\item для каждого объекта~$C$, если $P\hm\subseteq C$, то $C\mapsto B$, где 
$C\mapsto B$ означает, что объект~$B$ присутствует в~объекте, следующем за 
объектом~$C$;
\item объект~$P$ является минимальным объектом, удовлетворяющим 
условию~1, а~именно: $\forall \alpha\hm\in P$ объект~$P\backslash \{\alpha\}$ 
не является причиной, т.\,е.\ $\exists C:\ \alpha\not\in C$, $P\backslash 
\{\alpha\}\hm\subseteq C$ и~$C\not\mapsto B$, где $C\not\mapsto B$ означает, 
что~$B$ не может содержаться в~объекте, следующем за объектом~$C$. 
\end{enumerate}

  Приведенное определение причины является упрощением причин, 
возникающих в~реальном мире. Например, реальные причины могут возникать\linebreak 
как совокупность характеристик из разных пространств. Одно следствие может 
порождаться разными причинами или возникать из внешних\linebreak и~ненаблюдаемых 
характеристик. Однако пред\-став\-лен\-ная далее формализация позволяет доступно 
изложить при\-чин\-но-след\-ст\-вен\-ные истоки противоречий, которые 
инициируют в~дальнейшем глубокое исследование рассматриваемых процессов.
  
  Будем считать, что для любого интересующего нас свойства~$B$ существует 
причина. Тогда справедлива следующая теорема.
  
  \smallskip
  
  \noindent
  \textbf{Теорема~1.}\ \textit{Для любого свойства~$B$ существует 
единственная причина}. 
  
  \smallskip
  
  \noindent
  Д\,о\,к\,а\,з\,а\,т\,е\,л\,ь\,с\,т\,в\,о\,.\ \ Доказательство будем вести от противного, 
т.\,е.\ предположим, что существуют две причины свойства~$B$: $P$ 
и~$P^\prime$, $P\hm\not= P^\prime$. Тогда существует $\alpha\hm\in U$, которое 
удовлетворяет одному из двух условий:
  \begin{itemize}
\item[(а)] $\alpha\in P$, $\alpha\notin P^\prime$;
\item[(б)] $\alpha\notin P$, $\alpha \in P^\prime$.
\end{itemize}

  Пусть выполняется условие~(б). Тогда $P^\prime\backslash \{\alpha\}$ не 
является причиной по условию~2 определения~1, т.\,е.\ $\exists C$ такое, что 
$\alpha\notin C$, $P^\prime\backslash \{\alpha\}\hm\subseteq C$ и~$C\not\mapsto B$. 
Но если~$B$ произошло и~$P$ его причина, то $C\mapsto B$, что противоречит 
предположению. Теорема~1 доказана.
  
  \smallskip
  
  \noindent
  \textbf{Лемма.} \textit{Если $P$~--- причина появления свойства~$B$, то 
объект~$B$ определяет существование свойства~$P$ в~объекте, 
предшествующем~$B$. }
  
  \smallskip
  
  \noindent
  Д\,о\,к\,а\,з\,а\,т\,е\,л\,ь\,с\,т\,в\,о\,.\ \ Из предположения, что у~каж\-до\-го 
свойства~$B$ есть причина, и~условия, что~$P$ является причиной~$B$, следует, 
что при появлении в~данных свойства~$B$ объект~$C$, предшествующий 
появлению~$B$, содержит как часть объект~$P$. Это следует из теоремы~1 
и~определения причины. 
  
  Докажем принцип <<необходимого условия>>, который, несмотря на простоту 
доказательства, будет играть в~дальнейшем существенную роль.
  
  \smallskip
  
  \noindent
  \textbf{Теорема~2.} \textit{Если~$P$~--- причина появления свойства~$B$ 
и~$A\hm\subseteq P$, то объект~$B$ определяет наличие свойства~$A$ 
в~объекте, предшествующем~$B$}. 
  
  \smallskip
  
  \noindent
  Д\,о\,к\,а\,з\,а\,т\,е\,л\,ь\,с\,т\,в\,о\,.\ \ Пусть в~данных имеется объект~$B$ 
и~$P\mapsto B$, тогда в~силу существования и~единственности причины~$B$ 
в~данных должен существовать объект~$C$, предшествующий~$B$ 
и~содержащий причину~$P$. Поскольку $A\hm\subseteq P$ и~$B$ содержит 
причину~$P$, то $B\mapsto A$. С~учетом леммы теорема~2 доказана.
  
  \smallskip
  
  Пусть даны пространства $U_1, U_2,\ldots$ и~имеется последовательность 
данных (процесс выполнения этапов проекта в~соответствии с~рис.~1) $A, B, 
\ldots$, где каждый объект является подмножеством некоторого 
пространства~$U_i$, $i\hm=1,\ldots$ Тогда в~объекте~$A$ присутствует 
причина~$P$ появления интересующего нас свойства~$C$ в~объекте~$B$. Пусть 
$P\hm\subseteq A$, тогда по теореме~2 $\forall \alpha\hm\in P$:  
$C\mapsto \{\alpha\}$, т.\,е.\ из появления~$C$ следует появление 
характеристики~$\alpha$ в~предшествующем объекте. Это необходимое условие 
того, что~$C$ удовлетворяет причинно-следственным связям развития процесса 
выполнения проекта. Если для~$C$ нет характеристики~$\alpha$, которую можно 
отнести к~причине~$C$, то можно считать, что нарушена  
при\-чин\-но-след\-ст\-вен\-ная связь и~$C$~--- аномальный объект. 
  
  \smallskip
  
  \noindent
  \textbf{Пример.} Если объект~$C$ состоит в~получении суммы~$a$ 
фирмой~$K$, то согласно теореме~2 в~пред\-шест\-ву\-ющем объекте должна 
существовать причина перевода суммы~$a$ на фирму~$K$. Если эта причина 
в~проекте отсутствует, то это можно считать признаком мошеннической схемы. 
Все проекты по предположению собираются из <<кубиков>>, содержащихся в~БЗ. 
Тогда можно сравнить цену объекта~$C$, породившего получение суммы~$a$, 
и~сумму, присутствующую в~смете проекта. Если разница велика, то это либо 
ошибка проекта, либо признак мошеннической схемы.
  
  \section{Поиск противоречий на~основе~принципа <<необходимых~условий>>}
   
  Как было показано в~разд.~3, нахождение противоречий соответствуют 
движению от следствия к~причине. Для каждого объекта в~наблюдаемых данных 
выявление причин его появления является трудоемкой задачей. Кроме того, при 
реализации контроля соблюдения при\-чин\-но-след\-ст\-вен\-ных связей на 
большом множестве участников экономической деятельности задача анализа 
причин становится трудоемкой. Поэтому процедуру контроля необходимо разбить 
на два этапа, где первый этап состоит в~анализе простых <<необходимых 
условий>> проявления мошенничества, когда используется хотя бы одна 
известная характеристика причины. Второй этап (в~режиме офлайн) состоит 
в~выявлении причин, позволяющих провести анализ источников мошеннических 
схем. 
  
  Один из подходов к~выбору <<необходимых условий>> состоит в~построении 
множества подцелей исходной цели проекта (структурный метод построения 
проекта~\cite{7-gr}). Каждая подцель описывается диаграммой на рис.~1, 
и~реализации подцелей должны образовывать полный функционал цели. Это 
является необходимым, но не достаточным условием достижения цели, так как 
при таком подходе отсутствует компонент согласования всех подцелей в~единую 
систему. Однако такой подход значительно упрощает анализ выполнения проекта 
на предмет поиска мошенничества. Если признаки мошенничества будут 
обнаружены в~реализации хотя бы одной из подцелей, то это значит, что 
мошенничество присутствует в~реализации всего проекта. 
  
  Аналогично в~реализации каждого этапа в~любой из подцелей можно выделять 
простые <<необходимые условия>> нарушения при\-чин\-но-след\-ст\-венн\-ых 
связей. 
  
  Таким образом, получается множество <<необходимых условий>>, нарушение 
которых свидетельствует о наличии мошенничества. Это множество 
<<необходимых условий>> можно назвать метаданными~[8, 9] для контроля 
проекта на выявление мошенничества. 
  
  
  \section{Заключение }
  
  В поиске противоречий необходимо от транзакций, соответствующих 
следствиям при\-чин\-но-след\-ст\-вен\-ных связей, переходить к~анализу причин 
наблюдаемых следствий. Это сложная задача, которая связана с~описанием причин 
определенных свойств. 
  
  В работе представлена модель, позволяющая строить множество необходимых 
условий соответствия наблюдаемого следствия вызвавшей его причине. Этот 
подход делает поиск противоречий вполне вычислимой задачей, но не гарантирует 
успех. 
  
  {\small\frenchspacing
 {%\baselineskip=10.8pt
 \addcontentsline{toc}{section}{References}
 \begin{thebibliography}{9}
\bibitem{1-gr}
\Au{Грушо А.\,А., Зацаринный~А.\,А., Тимонина~Е.\,Е.} Блокчейны цифровой экономики на базе 
системы ситуационных центров и~централизованного консенсуса~// Радиолокация, навигация, 
связь: Мат-лы XXV Междунар. научн.-технич. конф.~---
Воронеж: Издательский дом ВГУ, 2019. Т.~6. С.~183--191. 
\bibitem{2-gr}
\Au{Grusho A., Zatsarinny~A., Timonina~E.} A~system approach to information security in 
distributed ledgers on the situational centers platform.~---
Lecture notes in computer science ser.~--- Springer, 2019 
(in press).
\bibitem{3-gr}
\Au{Финн В.\,К.} Искусственный интеллект: Методология, применения, философия.~--- М.: 
Красанд, 2011. 448~с.

\bibitem{5-gr} %4
\Au{Аншаков~О.\,М., Фабрикантова~Е.\,Ф.} ДСМ-ме\-тод автоматического порождения 
гипотез: Логические и~эпистемологические основания.~--- М.: Либроком, 2009. 432~с.

\bibitem{4-gr} %5
\Au{Poelmans J., Elzinga~P., Viaene~S., Dedene~G.} Formal concept analysis in knowledge 
discovery: A~survey~// Conceptual structures: From information to intelligence~/ Eds.\ M.~Croitoru, 
S.~Ferr$\acute{\mbox{e}}$, and D.~Lukose.~--- Lecture notes in computer science 
ser.~--- Berlin--Heidelberg: Springer, 2010. Vol.~6208.  P.~139--153.

\bibitem{6-gr}
\Au{Панкратова~Е.\,С., Финн~В.\,К.} Автоматическое по\-рож\-де\-ние гипотез в~интеллектуальных 
системах.~--- М.: Либроком, 2009. 528~с. 
\bibitem{7-gr}
\Au{Денисов А.\,А., Колесников~Д.\,Н.} Теория больших систем управления.~--- Л.: Энергоиздат, 1982. 488~с.

\bibitem{9-gr}
\Au{Грушо А.\,А., Грушо Н.\,А., Забежайло~М.\,И., Смирнов~Д.\,В., Тимонина~Е.\,Е.} 
Параметризация в~прикладных задачах поиска эмпирических причин~// Информатика и~её 
применения, 2018. Т.~12. Вып.~3. С.~62--66.

\bibitem{8-gr}
\Au{Грушо А.\,А., Грушо Н.\,А., Левыкин~М.\,В., Тимонина~Е.\,Е.} Методы идентификации 
захвата хоста в~распределенной ин\-фор\-ма\-ци\-он\-но-вы\-чис\-ли\-тель\-ной сис\-те\-ме, 
защищенной с~помощью метаданных~// Информатика и~её применения, 2018. Т.~12. Вып.~4. 
С.~41--45.

 \end{thebibliography}

 }
 }

\end{multicols}

\vspace*{-3pt}

\hfill{\small\textit{Поступила в~редакцию 03.04.19}}

%\vspace*{8pt}

%\pagebreak

\newpage

\vspace*{-28pt}

%\hrule

%\vspace*{2pt}

%\hrule

%\vspace*{-2pt}

\def\tit{ARCHITECTURAL DECISIONS IN~THE~PROBLEM 
OF~IDENTIFICATION OF~FRAUD IN~THE~ANALYSIS 
OF~INFORMATION FLOWS IN~DIGITAL ECONOMY\\[-5pt]}


\def\titkol{Architectural decisions in~the~problem 
of~identification of~fraud in~the~analysis 
of~information flows in~digital economy}

\def\aut{A.\,A.~Grusho, M.\,I.~Zabezhailo, N.\,A.~Grusho, and~E.\,E.~Timonina}

\def\autkol{A.\,A.~Grusho, M.\,I.~Zabezhailo, N.\,A.~Grusho, and~E.\,E.~Timonina}

\titel{\tit}{\aut}{\autkol}{\titkol}

\vspace*{-13pt}


 \noindent
   Institute of Informatics Problems, Federal Research Center ``Computer Sciences and 
Control'' of the Russian Academy of Sciences; 44-2~Vavilov Str., Moscow 119133, 
Russian Federation

\def\leftfootline{\small{\textbf{\thepage}
\hfill INFORMATIKA I EE PRIMENENIYA~--- INFORMATICS AND
APPLICATIONS\ \ \ 2019\ \ \ volume~13\ \ \ issue\ 2}
}%
 \def\rightfootline{\small{INFORMATIKA I EE PRIMENENIYA~---
INFORMATICS AND APPLICATIONS\ \ \ 2019\ \ \ volume~13\ \ \ issue\ 2
\hfill \textbf{\thepage}}}

\vspace*{3pt}


   
     
   \Abste{An approach to a~research of some types of fraud in digital economy with the usage of relationships of 
cause and effect is formulated. In all types of the considered frauds, the discrepancy between the 
purposes of financial transactions and actual cost of achievement of these purposes
has to be observed. Data on 
transactions can be collected by observing information flows in which these transactions are reflected. 
The architecture of data collection and their analysis can be organized by means of the distributed 
ledgers with the centralized consensus that allows creating an analog of the electronic account book 
fixing financial and economic activity of subjects of digital economy in the region. 
   The methods of fraud identification considered are based on the contradictions 
between actions described in transactions and information, which is contained in plans, standards, 
precedents, etc. 
   The method based on a~simplified scheme of implementation of the abstract project is considered. 
For identification of contradictions, it is necessary to carry out the analysis from the effect to the cause, 
i.\,e., to look for anomalies in information describing the generation of the observed effects. 
   It is shown how in implementation of the project it is possible to allocate simple ``necessary 
conditions'' of violation of cause and effect relationships, i.\,e., a~set of ``necessary conditions'' 
violation of which demonstrates fraud existence. It is possible to call this set of "necessary conditions" 
by metadata for control of the project for fraud identification.} 
   
   \KWE{digital economy; information flows; relationships of reason and effect; detection of 
fraudulent schemes}
   
  

 \DOI{10.14357/19922264190204}

\vspace*{-20pt}

 \Ack
   \noindent
   The work was partially supported by the Russian Foundation for Basic Research (projects  
18-29-03081 and 18-07-00274).



%\vspace*{6pt}

  \begin{multicols}{2}

\renewcommand{\bibname}{\protect\rmfamily References}
%\renewcommand{\bibname}{\large\protect\rm References}

{\small\frenchspacing
 {\baselineskip=10.5pt
 \addcontentsline{toc}{section}{References}
 \begin{thebibliography}{9}
\bibitem{1-gr-1}
\Aue{Grusho, A.\,A., A.\,A.~Zatsarinny, and E.\,E.~Timonina.} 2019. Blokcheyny tsifrovoy ekonomiki 
na baze sistemy situatsionnykh tsentrov i~tsentralizovannogo konsensusa [Blockchains of digital 
economy on the basis of the system of the situational centres and the centralized consensus]. 
\textit{25th Scientific and Technical Conference (International) ``Radar-Location, Navigation, 
Communication'' Proceedings}. Voronezh: VSU Publs. 6:183--191.
\bibitem{2-gr-1}
\Aue{Grusho, A., A.~Zatsarinny, and E.~Timonina.} 2019 (in press). 
A~system approach to information security 
in distributed ledgers on the situational centers platform. 
Lecture notes in computer science ser. Springer.
\bibitem{3-gr-1}
\Aue{Finn, V.\,K.} 2011. \textit{Iskusstvennyy intellekt: Metodologiya, primeneniya, filosofiya} 
[Artificial intelligence: Methodology, applications, philosophy]. Moscow: KRASAND. 448~p.

\bibitem{5-gr-1}
\Aue{Anshakov, O.\,M., and E.\,F.~Fabrikantova}. 2009. \textit{DSM-metod avtomaticheskogo porozhdeniya gipotez: Logicheskie 
i~epistemologicheskie osnovaniya} [JSM-method of automatic hypothesis generation: Logical and 
epistemological]. Moscow: KD LIBROKOM. 432~p.
\bibitem{4-gr-1} %5
\Aue{Poelmans, J., P.~Elzinga, S.~Viaene, and G.~Dedene.} 2010. Formal concept analysis in 
knowledge discovery: A~survey. \textit{Conceptual structures: From information to intelligence}. 
Eds.\ M.~Croitoru, S.~Ferr$\acute{\mbox{e}}$, and D.~Lukose. Lecture notes in 
computer science ser. Berlin--Heidelberg: Springer. 6208:139--153.

\bibitem{6-gr-1}
\Aue{Pankratov, E.\,S., and V.\,K.~Finn}. 
2009. \textit{Avtomaticheskoe porozhdenie gipotez v~intellektual'nykh 
sistemakh} [Automatic hypotheses generation in intelligent systems]. Moscow: KD 
\mbox{LIBROKOM}.  528~p. 
\bibitem{7-gr-1}
\Aue{Denisov, A.\,A., and D.\,N.~Kolesnikov.} 1982. \textit{Teoriya bol'shikh 
sistem upravleniya} [Theory of big control systems]. Leningrad: Energoizdat. 488~p.

\bibitem{9-gr-1}
\Aue{Grusho, A.\,A., N.\,A.~Grusho, M.\,I.~Zabezhailo, D.\,V.~Smirnov, and 
E.\,E.~Timonina.} 2018. 
Parametrizatsiya v~prikladnykh zadachakh poiska empiricheskikh prichin 
[Parametrization in applied 
problems of search of the empirical reasons]. 
\textit{Informatika i~ee Primeneniya~--- 
Inform. Appl.} 12(3):62--66.

\bibitem{8-gr-1}
\Aue{Grusho, A.\,A., N.\,A.~Grusho, M.\,V.~Levykin, and E.\,E.~Timonina.} 2018. Metody 
identifikatsii zakhvata khosta v~raspredelennoy informatsionno-vychislitel'noy sisteme, 
zashchishchennoy s~pomoshch'yu metadannykh [Methods of identification of host capture 
in the  distributed information system which is protected on the base of meta data].
\textit{Informatika i~ee 
Primeneniya~--- Inform. Appl.} 12(4):41--45.
{ %\looseness=1

}

\end{thebibliography}

 }
 }

\end{multicols}

\vspace*{-12pt}

\hfill{\small\textit{Received April 3, 2019}}

%\pagebreak

%\vspace*{-18pt}

\Contr

\noindent
\textbf{Grusho Alexander A.} (b.\ 1946)~--- Doctor of Science in physics and 
mathematics, professor, principal scientist, Institute of Informatics Problems, 
Federal Research Center ``Computer Sciences and Control'' of the Russian 
Academy of Sciences; 44-2~Vavilov Str., Moscow 119133, Russian Federation; 
\mbox{grusho@yandex.ru} 

\vspace*{3pt}

\noindent
\textbf{Zabezhailo Michael I.} (b.\ 1956)~--- Doctor of Science in physics and 
mathematics, principal scientist, Institute of Informatics Problems, Federal Research 
Center ``Computer Sciences and Control'' of the Russian Academy of Sciences;  
44-2~Vavilov Str., Moscow 119133, Russian Federation; 
\mbox{m.zabezhailo@yandex.ru} 

\vspace*{3pt}


\noindent
\textbf{Grusho Nikolai A.} (b.\ 1982)~--- Candidate of Science (PhD) in physics 
and mathematics, senior scientist, Institute of Informatics Problems, Federal 
Research Center ``Computer Sciences and Control'' of the Russian Academy of 
Sciences; 44-2~Vavilov Str., Moscow 119133, Russian Federation; 
\mbox{info@itake.ru} 

\vspace*{3pt}


\noindent
\textbf{Timonina Elena E.} (b.\ 1952)~--- Doctor of Science in technology, 
professor, leading scientist, Institute of Informatics Problems, Federal Research 
Center ``Computer Sciences and Control'' of the Russian Academy of Sciences;  
44-2~Vavilov Str., Moscow 119133, Russian Federation; 
\mbox{eltimon@yandex.ru} 

\label{end\stat}

\renewcommand{\bibname}{\protect\rm Литература}    %10
\def\stat{zatsar}

\def\tit{СИСТЕМОТЕХНИЧЕСКИЕ ПОДХОДЫ К~СОЗДАНИЮ\\ 
СИСТЕМЫ ПОДДЕРЖКИ ПРИНЯТИЯ РЕШЕНИЙ\\ НА~ОСНОВЕ 
СИТУАЦИОННОГО АНАЛИЗА}

\def\titkol{Системотехнические подходы к~созданию 
системы поддержки принятия решений на~основе 
ситуационного анализа}

\def\aut{А.\,А.~Зацаринный$^1$, А.\,П.~Сучков$^2$}

\def\autkol{А.\,А.~Зацаринный, А.\,П.~Сучков}

\titel{\tit}{\aut}{\autkol}{\titkol}

\index{Зацаринный А.\,А.}
\index{Сучков А.\,П.}
\index{Zatsarinny A.\,A.}
\index{Suchkov A.\,P.}


%{\renewcommand{\thefootnote}{\fnsymbol{footnote}} \footnotetext[1]
%{Работа выполнена при финансовой поддержке РФФИ (проект 16-37-00485).}}


\renewcommand{\thefootnote}{\arabic{footnote}}
\footnotetext[1]{Институт проблем информатики Федерального исследовательского центра 
<<Информатика и~управ\-ле\-ние>> Российской академии наук, \mbox{AZatsarinny@ipiran.ru}}
\footnotetext[2]{Институт проблем информатики Федерального исследовательского центра 
<<Информатика и~управ\-ле\-ние>> Российской академии наук, \mbox{Asuchkov@ipiran.ru}}

      

\Abst{Обсуждаются вопросы создания сис\-тем поддержки принятия решений 
(СППР) на основе ситуационного анализа текущей и~прогнозируемой обстановки 
в~контролируемом пространстве органа управления. Как правило, такие сис\-те\-мы 
управления в~режиме реального времени опираются на ситуационные центры (СЦ)~--- 
совокупность информационных, программных и~аппаратных средств, а также 
обслуживающего персонала, реализующих информационные технологии по мониторингу 
обстановки, ее ситуационному анализу для выработки решений и~алгоритмов применения 
управляющих воздействий. Рассмотрены содержательные характеристики составляющих 
частей СППР, реализующих полный цикл управления от целеполагания до контроля 
исполнения принимаемых решений. Отмечается, что реализация СППР зависит от уровня 
сис\-те\-мы управ\-ле\-ния~--- стратегического, оперативного, тактического, базового, приводятся 
функциональные особенности и~способы анализа обстановки на различных уровнях 
сис\-те\-мы управ\-ления.}

\KW{ситуационный анализ; сис\-те\-ма поддержки принятия решений; сис\-те\-ма управ\-ле\-ния; 
ситуационный центр}

\DOI{10.14357/19922264160411} 


\vskip 10pt plus 9pt minus 6pt

\thispagestyle{headings}

\begin{multicols}{2}

\label{st\stat}

\section{Введение}

     В Стратегии национальной безопасности Российской Федерации 
(утверждена Указом Президента Российской Федерации от~31~декабря 
2015~г. №\,683)~[1] определено, что информационную основу реализации 
Стратегии составляет федеральная информационная сис\-те\-ма стратегического 
планирования, включающая в~себя информационные ресурсы органов 
государственной власти и~органов местного самоуправления, сис\-те\-мы 
распределенных СЦ и~государственных научных 
организаций. В~рамках такой сис\-те\-мы должна быть реализована поддержка 
управленческих решений в~интересах центральных органов исполнительной 
власти на основе организации взаимодействия региональных 
и~ведомственных СЦ, а~также других информационных 
сис\-тем. Для эффективного решения этой задачи необходимо создание СППР 
в~со\-ста\-ве СЦ и~придания им принципиально новых качеств. 
     
     В связи с~этим целью статьи является обоснование сис\-те\-мо\-тех\-ни\-че\-ских 
и~методических подхо\-дов к~структурному и~функциональному составу\linebreak 
СППР и~ее месту в~составе СЦ, обеспечивающих 
информационно-аналитическую поддержку принятия управленческих 
решений в~рамках государственного управления, стратегического 
планирования и~мониторинга реализации документов стратегического 
планирования в~Российской Феде-\linebreak рации. 

\vspace*{-6pt}
     
\section{Базовые понятия }

\vspace*{-2pt}

    При рассмотрении сис\-тем\-ных и~методических вопросов создания СППР, 
основанных на ситуационном анализе, в~статье используется ряд базовых 
понятий: событие, обстановка, ситуация, угроза, управление, цели 
управления и~др.~[2]. 
    
    \textit{Ситуация} определяется состоянием взаимосвязанных 
\textit{элементов обстановки} в~контролируемом пространстве; изменения 
обстановки определя-\linebreak ются \textit{событиями}, образующими некоторые 
разворачивающиеся во времени наблюдаемые и~ре\-гист\-ри\-ру\-емые потоки. При 
этом под \textit{управлением}\linebreak понимается \textbf{целенаправленное} 
воздействие органа управления на подчиненные ему или взаимодействующие 
элементы обстановки (ресурсы). 
    
    Совокупность ситуаций в~сис\-те\-ме управ\-ле\-ния распадается на текущие, 
прогнозируемые и~целевые ситуации. При этом текущие ситуации являются 
результатом наблюдения и~регистрации событий, прогнозируемые 
определяются методами ситуационного анализа, а целевые отражают 
краткосрочные, среднесрочные и~долгосрочные цели управления. Последнее 
немаловажно, так как зачастую ситуационный анализ понимается как 
обеспечение реакций сис\-те\-мы управ\-ле\-ния на чрезвычайные ситуации после 
того, как они сложились. Однако теория ситуационного подхода 
предполагает учет <<планируемой и~прогнозируемой обстановки>>, 
отражающей стратегические, тактические и~оперативные \textit{цели 
управления}, а~также учет факторов самоорганизации управляющего 
сегмента сис\-те\-мы, определяющих стимулы для достижения этих 
целей~[2,~3]. Под \textit{угрозой} в~процессах управления понимается 
ситуация или совокупность ситуаций, развитие которых противоречит целям 
управления и~отдаляет текущее состояние от целевого.
    
    В конце 1970-х~гг.\ была создана модель сис\-те\-мы управ\-ле\-ния  
<<наблю\-де\-ние--ори\-ен\-ти\-ро\-ва\-ние--ре\-ше\-ние--дей\-ст\-вие>> 
(НОРД) для принятия решений при ведении боевых действий~[4, 5]. 
В~настоящее время эта модель активно используется во многих сис\-те\-мах 
управ\-ле\-ния разных отраслей~[6]. В~рамках ситуационного подхода 
к~управлению предложена модифицированная модель, включающая 
дополнительную стадию управляющего цикла~--- целеполагание~[7].
    
    \textbf{Целеполагание} (стадия~Ц)~--- формализованное представление 
целевых показателей, установление количественных 
и~временн$\acute{\mbox{ы}}$х критериев их достижения.
    
    \textbf{Мониторинг} (стадия~М)~--- это процесс сбора информации об 
окружающей среде в~контролируемом пространстве, включая состояние 
целевых показателей. Стадия М также принимает внутренние инструкции от 
стадии анализа (А), так же как и~поддержку от процессов~Р и~Д. 
    
    \textbf{Анализ} (стадия~А)~--- оценка ситуации (типовая, нетиповая), 
анализ существующего опыта, пополнение опыта, обеспечивает внутреннюю 
поддержку~М (корректировка фильтров).
    
    \textbf{Решение} (стадия~Р)~--- это процесс осуществления выбора 
среди гипотез о состоянии окружающей среды и~возможной реакции на него. 
Процесс~Р руководствуется прямой внутренней связью с~процессом~А 
и~обеспечивает внутреннюю поддержку процесса~М, возможна 
корректировка целевых показателей (стадия~Ц).
    
    \textbf{Действие} (стадия~Д)~--- это процесс выполнения выбранной 
реакции путем взаимодействия с~окружающей средой. Действие принимает 
внутренние руководства от процесса~А, также оно напрямую связано с~Р. 
Оно обеспечивает внутреннюю поддержку~Ц и~М.
    
    Особенности реализации цикла управления в~сис\-те\-ме, реализующей 
процессы стратегического планирования и~управления, заключаются в~том, 
что содержательно стадии~Ц, А и~Р реализуются непосредственно высшими 
органами исполнительной власти. Это означает осуществление сле\-ду\-ющих 
основных функций:
    \begin{itemize}
\item  доведение до подчиненных органов данных целеполагания 
и~стратегического планирования на основе их формализации 
и~регламентации обмена (стадия~Ц);
\item регламентированный сбор данных о состоянии целевых показателей от 
органов испол\-нительной власти и~об обстановке в~конт\-ро\-ли\-ру\-емом 
пространстве по определенному\linebreak регламенту и~в~режиме реального времени 
(стадия~М);
\item обмен аналитическими данными участников\linebreak стратегического 
планирования по целеполаганию, прогнозированию, планированию 
и~программированию~--- федеральных органов исполнительной власти 
(ФОИВ), субъектов Россий\-ской Федерации и~муниципальных образований, 
отраслей экономики и~сфер государственного и~муниципального управления 
(стадия~А);
\item  доведение до подчиненных органов принимаемых решений по 
применению сил и~средств и,~возможно, по корректировке стратегических 
планов с~целью достижения поставленных стратегических целей (стадия~Р) 
и~контроль исполнения решений (стратегических планов) на основе 
докладов (стадия~Д).
    \end{itemize}
    
    На тактическом и~базовом уровнях управления осуществляются,  
во-пер\-вых, реализация функ-\linebreak ций мониторинга контролируемого 
пространства и~организа\-ции учета контролируемых объектов (стадия~М),  
во-вто\-рых, специальный анализ фактографических данных о конкретных 
элементах обстановки, формализованных в~виде семантической сети, 
позволяющий выявлять неочевидные связи между элементами обстановки, 
определять схожие про\-стран\-ст\-вен\-но-со\-бы\-тий\-ные ситуации, выявлять 
ассоциативные связи и~закономерности с~\mbox{целью} поддержки процессов 
принятия решений (стадия~А), в-треть\-их, процессы принятия решений 
по планированию применения сил и~средств на период времени и~по 
складывающейся обстановке в~соответствии с~указаниями вышестоящих 
органов (стадии~Р и~Д).

\begin{figure*}[b] %fig1
\vspace*{1pt}
\begin{center}
\mbox{%
\epsfxsize=160.901mm
\epsfbox{zac-1.eps}
}
\end{center}
\vspace*{-9pt}
\Caption{Обобщенная функциональная структура СЦ}
\end{figure*}
    

\section{Ситуационный центр как составляющая современной системы 
управления}
    
    Определим СЦ сис\-те\-мы управ\-ле\-ния как совокупность 
информационных, программных и~аппаратных средств, а~также 
обслуживающего персонала, реализующих информационные технологии\linebreak по 
мониторингу обстановки, ее ситуационному анализу для выработки решений 
и~алгоритмов применения управляющих воздействий с~\mbox{целью} эффективной 
реализации функций управления и~минимизации ущерба от угроз в~зоне 
ответствен\-ности\linebreak органа управ\-ле\-ния, доведения их до объектов управ\-ле\-ния 
и~контроля исполнения,
    
    По сути дела, СЦ является составной частью сис\-те\-мы 
управ\-ле\-ния, осуществляющей автоматизацию ряда функций всего органа 
управления и~отдельных должностных лиц.
    
    Исходя из накопленного в~Институте проблем информатики РАН опыта 
разработки крупных информационных сис\-тем в~интересах органов 
государственной власти, в~организационной структуре СЦ можно выделить 
четыре основных функциональных сегмента (рис.~1)~\cite{8-zat}:
    \begin{enumerate}[(1)]
\item сегмент руководства (лиц, принимающих решения, ЛПР); 
\item сегмент мониторинга состояния контролируемых объектов 
и~окружающей среды и~сбора информации; 
\item сегмент ситуационного анализа и~сис\-те\-ма\-ти\-за\-ции информации;
\item сегмент администрирования и~эксплуатации.
\end{enumerate}
    При этом СППР базируется на ресурсах всех четырех сегментов. Вместе 
с тем центральным звеном СЦ и~его СППР, обеспечивающим реализацию 
основной функции сис\-те\-мы управ\-ле\-ния по эффективному управлению 
силами и~средствами, является \textit{сегмент ситуационного анализа 
и~сис\-те\-ма\-ти\-за\-ции информации}. Он должен обеспечивать реализацию 
следующих функций:
    \begin{itemize}
\item возможность визуализации результатов анализа обстановки на 
индивидуальных и~коллективных средствах отображения;
\item во взаимодействии с~сегментом мониторинга получение данных 
о~состоянии обстановки от собственных (субъективных и~объективных 
средств наблюдения и~контроля) и~внешних по отношению к~сис\-те\-ме 
источников информации (ведомственных, межведомственных, 
международных, независимых и~др.);
\item извлечение фактов, структуризация и~формализация разнородных 
данных о~значимых событиях в~соответствии с~выбранной информационной 
моделью предметной области;
\item формирование хранилищ ситуационных данных;
\item формирование способов визуализации агрегированных данных 
о~складывающейся обстановке для ЛПР и~оперативного состава;
\item формирование отчетности и~служебной документации;
\item расчет первичных и~интегральных показателей обстановки, а~также 
статистическая оценка характеристик ненаблюдаемых элементов обстановки;
\item решение задач перспективного планирования, контроль исполнения 
решений по планированию;
\item выявление значимых ситуаций, их ранжирование по степени 
важности, видам и~типам, формирование текущего перечня 
аналитических задач по складывающейся обстановке и~по поручениям 
руководства;
\item  выработка вариантов решений по применению управляющих 
воздействий для достижения целевых ситуаций, формирование спо\-собов 
наглядного представления вариантов\linebreak реше\-ния для ЛПР (оперативное 
планирование);
\item прогнозирование развития обстановки и~процесса реализации целей 
сис\-те\-мы управ\-ле\-ния на основе сформированных ситуационных моделей 
и~моделей угроз, в~том числе и~с~учетом применения выработанных 
вариантов решений;
\item обеспечение процессов принятия решений комплексом  
ин\-фор\-ма\-ци\-он\-но-рас\-чет\-ных задач (ИРЗ).
    \end{itemize}
    
    Наряду с~перечисленными в~СППР СЦ реализуются важнейшие функции 
администрирования аналитической под\-сис\-те\-мы~СЦ:
    \begin{itemize}
\item формирование и~корректировка сис\-те\-мы целей управ\-ления;
\item формирование, настройка и~корректировка сис\-те\-мы моделей целей 
управления, обстановки, ситуаций и~угроз;
\item формирование, настройка и~корректировка сис\-те\-мы расчетных 
показателей, характеризующих обстановку и~ее элементы;
\item формирование, настройка и~корректировка сис\-те\-мы критериев, 
пороговых значений, эвристик, параметров расчетных алгоритмов.
\end{itemize}

\section{Целеполагание~--- определение целей системы управления}

    Под \textit{целью ситуационного анализа} предлагается понимать 
поддержку процессов принятия решений для достижения поставленных 
целей путем применения доступных в~сис\-те\-ме управ\-ле\-ния сил и~средств 
(ресурсов).
    
    Целесообразность деятельности сис\-те\-мы управ\-ле\-ния определяется 
иерархической сис\-те\-мой целей\linebreak (подцелей). Для ФОИВ она задается 
законодательно, а также при определении приоритетов в~орга\-низации 
деятельности сис\-те\-мы управ\-ле\-ния первым\linebreak лицом (руководителем). 
Формирование сис\-те\-мы целей сопровождается формированием сис\-те\-мы 
показателей реализации целей (подцелей) и~критериев достижения целей. 
Показатели являются вычисляемыми величинами как функции обстановки 
или экспертно оцениваемые параметры. Критерии достижения обычно 
формулируются как некие пороговые плановые значения на временн$\acute{\mbox{о}}$й 
шкале.
    
    Эффективность сис\-те\-мы управ\-ле\-ния в~каждый момент времени 
определяется, во-пер\-вых, степенью достижения пороговых значений 
планируемых целевых показателей, во-вто\-рых, объемом затрачиваемых 
ресурсов на единицу оптимизируемого целевого показателя.
    
    Цели управления формируются на основании сис\-тем\-но\-го анализа  
нор\-ма\-тив\-но-пра\-во\-вых основ функционирования сис\-те\-мы управ\-ле\-ния. 
Цели управления образуют дерево целей, детализация которого (число 
уровней) определяется воз\-мож\-ностью декомпозиции конкретной цели на 
значимые подцели. Цели и~подцели должны обладать индикаторами 
состояния (как правило, \%) и~весовыми коэффициентами доли подцели 
в~реализации всей цели. Цели могут включать ориентиры развития сис\-те\-мы 
управления, установленные первым лицом.

\begin{figure*} %fig2
\vspace*{1pt}
\begin{center}
\mbox{%
\epsfxsize=165.008mm
\epsfbox{zac-2.eps}
}
\end{center}
\vspace*{-9pt}
\Caption{Обобщенная структура сис\-те\-мы целей}
\end{figure*}
    
    Выбор структуры сис\-те\-мы целей предлагается осуществлять с~учетом 
следующих соображений.
    \begin{enumerate}[1.]
    \item Цели управления сложной управляющей сис\-те\-мой определяются 
нор\-ма\-тив\-но-пра\-во\-вы\-ми документа\-ми, регламентирующими ее 
функционирование, и, как правило, образуют \textbf{иерархическую 
структуру} в~соответствии со структурой направлений деятельности 
(рис.~2).
    \item Ситуационный подход к~управлению предполагает реагирование на 
складывающуюся обстановку в~режиме реального времени. В~силу этого, 
помимо фиксированных целей в~сис\-те\-ме управ\-ле\-ния необходим механизм 
формирования \textbf{динамических целей}, отражающих процесс 
нормализации складывающихся чрезвычайных ситуаций и~присутствующих 
в~сис\-те\-ме целеполагания на период существования ситуации.
    \item В~концепции <<управления по целям>> эффективность 
целеполагания проверяется по критериям SMART~\cite{9-zat}: цель должна 
быть конкретная, измеримая (подразумевает количественную измеримость 
результата), достижимая (должна быть выполнимой), реалистичная 
(достижение цели должно быть обеспечено ресурсами), привязанная  
к~точ\-ке/ин\-тер\-ва\-лу времени.
    \end{enumerate}
    
    Данный подход накладывает \textbf{требования на атрибуты целей} 
в~части формирования количественных характеристик их достижения, 
плановых характеристик, критериев достижения (см.\ рис.~2). 
    


    Основные атрибуты цели:\\[-14pt]
    \begin{itemize}
\item описание~--- дает определение и~конкретизацию цели;\\[-14pt]
\item весовой коэффициент~--- определяет вклад подцели 
в~вышестоящую цель;\\[-14pt]
\item индикатор~--- задает количественный показатель достижения 
результата;\\[-14pt]
\item критерий~--- задает способ определения достижения результата 
с~помощью индикатора;\\[-14pt]
\item план~--- определяет количественные значения критерия 
достижения цели и~требуемые вре\-мен\-н$\acute{\mbox{ы}}$е параметры.
\end{itemize}

\vspace*{-9pt}

\section{Анализ обстановки и~выработка вариантов решений}

\vspace*{-2pt}

\subsection{Мониторинг обстановки}

\vspace*{-1pt}

     В процессе мониторинга контролируемых элементов обстановки 
осуществляются (рис.~3):
     \begin{itemize}
\item сбор данных о состоянии контролируемых объектов, анализ 
неструктурированной информации с~целью извлечения фактов и~знаний; 
\item постановка объектов на контроль (оператор, автоматически); 
\item отображение контролируемых объектов по шкале состояний и~по 
критериям~--- соотношение текущего или прогнозируемого значения 
индикатора (интегрального показателя) и~сис\-те\-мы порогов, обеспечивающих 
градацию состояния (<<типовое>>, <<чрезвычайное>>, <<критическое>> 
или другие подобные).
\end{itemize}

    По данным мониторинга контролируемых элементов обстановки из 
различных источников формируется \textit{хранилище} СППР, которое 
пред\-став\-ляет собой совокупность взаимоувязанных на\linebreak основе единого 
информационного и~лингвистического обеспечения баз данных (БД): 
обстановки (события, ситуации, элементы окружающей\linebreak среды), сил и~средств 
(свои силы и~средства, противодействующие силы и~средства, так\-ти\-ко-тех\-ни\-че\-ские
характеристики), целевых 
показателей (первичные показатели, интегральные показатели,\linebreak индикаторы, 
критерии), типовых решений (типовые решения, конкретные решения), 
ретроспективная (нормализованные исторические данные, архив 
обстановки), нормативных документов, биб\-лио\-те\-ка математических моделей.

\vspace*{-6pt}

\subsection{Поддержка процесса принятия решений}

\vspace*{-2pt}

    На основе мониторинга текущей обстановки и~поступления событийной 
информации в~хранилище осуществляется расчет заданных в~сис\-те\-ме 
первичных и~интегральных показателей обстановки и~целевых показателей 
в~двух режимах: по регламенту (с~определенной периодичностью) и~по 
запросу пользователя с~использованием блоков расчетов, блока первичного, 
краткосрочного, среднесрочного и~долгосрочного анализа, блока 
визуализации  и~блока поддержки принятия решений (рис.~4).\linebreak\vspace*{-12pt}


\pagebreak

\end{multicols}
\begin{figure*} %fig3
\vspace*{1pt}
\begin{center}
\mbox{%
\epsfxsize=157.334mm
\epsfbox{zac-3.eps}
}
\end{center}
\vspace*{-6pt}
\Caption{Мониторинг обстановки}
\vspace*{6pt}
\end{figure*}

\begin{multicols}{2}




    
    При этом реализуются следующие функции.
    \begin{enumerate}[1.]
\item  Создание (привязка существующих) динамических моделей 
обстановки:
\begin{itemize}
\item моделей <<нормальной>> обстановки;
\item моделей для прогноза обстановки;
\item моделей для анализа трендов, циклов, аномалий обстановки.
\end{itemize}

    \item Проведение оперативного анализа текущей обстановки 
с~использованием математических методов (см.\ рис.~4):
\begin{itemize}
\item анализ отклонения от <<нормальной>> текущей обстановки;
\item прогноз развития обстановки;
\item анализ трендов, циклов, аномалий обстановки;
\item выявление и~идентификация значимых ситуаций 
на основе выявления типовых кон-\linebreak\vspace*{-12pt}

\columnbreak

\noindent
фигураций событий 
и~правил идентификации, идентификация типа ситуации, 
фор\-ми\-ро\-ва\-ние неотложных целей.\\[-7.5pt]
\end{itemize}
    \item Визуализация и~индикация состояний контролируемых объектов 
    с~использованием полученных результатов анализа (наглядное пред\-став\-ле\-ние 
текущей с~индикацией ситуаций,\linebreak требующих принятия решения или 
применения типовых решений).\\[-6pt]
    \item Выработка вариантов решений по складыва\-ющейся обстановке 
(решение содержит динамическую цель, перечень подцелей (с~весами~--- 
доли подцели в~реализации всей цели),\linebreak сроки достижения подцелей, 
ответственных, совокупность типовых уведомлений и~рапортов):
\begin{itemize}
\item применение типовых решений по типовым ситуациям (привязка их 
к~реальной обстановке);
\end{itemize}
\end{enumerate}



\pagebreak

\end{multicols}

\begin{figure*} %fig4
\vspace*{1pt}
\begin{center}
\mbox{%
\epsfxsize=164.07mm
\epsfbox{zac-4.eps}
}
\end{center}
\vspace*{-11pt}
\Caption{Структура блока принятия решений}
\vspace*{-3pt}
\end{figure*}

\begin{multicols}{2}

\noindent
\begin{enumerate}
\item[\ ]
\vspace*{-13pt}
\begin{itemize}
\item выработка вариантов решения экспертным путем в~случае критических 
и чрезвычайных ситуаций;\\[-15pt]
\item анализ развития обстановки с~учетом вариантов решений (прогноз 
благоприятного и~неблагоприятного развития обстановки, расчет 
вероятностей выполнения задач, оценка вариантов решений).
\end{itemize}
\end{enumerate}

\vspace*{-6pt}

    \subsection{Реализация решений }
    
    \vspace*{-2pt}
    
    На данной стадии осуществляется мониторинг процессов реализации 
решений по краткосрочным, среднесрочным и~долгосрочным планам 
(решение содержит цель, перечень подцелей (с~весами~--- доли подцели 
в~реализации всей цели), сроки достижения подцелей, ответственных, виды 
отчетности):
\begin{itemize}
\item сбор информации по ходу выполнения плана (отчетность), 
визуализация хода исполнения, контроль исполнения;\\[-15pt]
\item сравнительный анализ показателей плана по целям и~подцелям 
и~текущей обстановки, включая расчет степени реализации плана 
и~прогнозирование возможности реализации плана;\\[-15pt]
\item реализация обратной связи по уточнению решения по планированию 
с~целью обеспечения выполнения плана;\\[-15pt]
\item доведение уточненного решения (уведомления) и~контроль исполнения.
\end{itemize}

    Мониторинг реализации решений по ситуациям (решение содержит 
динамическую цель, перечень подцелей (с~весами~--- доли подцели 
в~реализации всей цели), сроки достижения подцелей, ответственных, 
совокупность типовых уведомлений и~рапортов): 
    \begin{itemize}
\item сбор информации по ходу выполнения решения (рапорты), 
визуализация хода исполнения, контроль исполнения;
\item сравнительный анализ показателей по целям и~подцелям и~текущей 
обстановки, включая расчет степени реализации решения и~прогнозирование 
возможности реализации решения;
\item реализация обратной связи по уточнению решения по ситуации с~целью 
обеспечения выполнения плана.
\item доведение уточненного решения (уведомления) и~контроль исполнения.
\end{itemize}

\vspace*{-6pt}

\section{Заключение}

\noindent
\begin{enumerate}[1.]
\item В современных условиях развития информационных сис\-тем особую 
значимость приобретает актуальность исследования сис\-те\-мо\-тех\-ни\-че\-ских 
и~технологических вопросов создания в~составе СЦ
СППР.
\item Важнейшей методологической и~концептуальной основой СППР 
является полнофункциональный цикл управления, включающий стадии 
целеполагания, мониторинга обстановки, анализа обстановки, выработки 
вариантов решений и~их реализации.
\item В СППР реализуются следующие функциональные задачи:
\begin{itemize}
\item мониторинг контролируемых элементов обстановки;
\item расчет характеристик событийной информации (первичные 
и~интегральные показатели текущей обстановки и~состояния 
целей);
\item визуализация текущего состояния обстановки;
\item визуализация текущего состояния индикаторов целей;
\item блок анализа и~принятия решений.
\item мониторинг контролируемых решений;
\item формирование документов и~отчетов.
\end{itemize}
\item Важнейшим сис\-те\-мо\-обра\-зу\-ющим компонентом СППР является 
хранилище, формируемое в~автоматизированном режиме из различных 
источников в~виде совокупности взаимоувязанных на основе единого 
информационного и~лингвистического обеспечения БД (о~событиях, 
силах и~средствах, целевых показателях и~критериях, типовых решений, 
ретроспективной информации, нормативных документов, математических 
моделей).
\item Предложенные в~статье сис\-те\-мо\-тех\-ни\-че\-ские подходы и~решения 
апробированы в~рамках нескольких проектов по созданию крупных 
территориально распределенных  
ин\-фор\-ма\-ци\-он\-но-ана\-ли\-ти\-че\-ских сис\-тем специального 
назначения.
\end{enumerate}

\vspace*{-6pt}

{\small\frenchspacing
 {%\baselineskip=10.8pt
 \addcontentsline{toc}{section}{References}
 \begin{thebibliography}{9}

\bibitem{1-zat}
Стратегия национальной безопасности Российской Федерации. Утверждена Указом 
Президента Российской Федерации от 31~декабря 2015~г. №\,683. 
\bibitem{2-zat}
\Au{Зацаринный А.\,А., Сучков А.\,П.} Некоторые подходы к~ситуационному анализу 
потоков событий~// Открытое образование, 2012. №\,1. С.~39--45.
\bibitem{3-zat}
\Au{Бир С.\,Э.} Мозг фирмы~/
Пер. с~англ.~--- М.: Радио и~связь, 1993. 416~с.
(\Au{Beer~S.}  {Brain of the firm}.~--- Allen Lane, The Penguin Press, London; Herder 
and Herder, USA, 1972. 416~p.)

\bibitem{5-zat}%4
\Au{Grant Т., Kooter В.} Comparing OODA \& other models as operational view~C2 
architecture~// 10th Command and Control Research Technology Symposium (International) 
Proceedings.~--- McLean, VA, USA, 2005.
\bibitem{4-zat} %5
\Au{Ивлев А.\,А.} Основы теории Бойда. Направления развития, применения 
и~реализации.~--- SlideShare, 2008. 64~с. {\sf  
http://www.slideshare.net/defensenetwork/ss-10380168}.
\bibitem{6-zat}
\Au{Босов А.\,В., Зацаринный А.\,А., Сучков~А.\,П.} Некоторые общие подходы 
к~формированию функциональных требований к~ситуационным центрам и~их 
реализации~// Системы и~средства информатики, 2010. Вып.~20. №\,3. С.~98--125.
\bibitem{7-zat}
\Au{Сучков А.\,П.} Формирование сис\-те\-мы целей для ситуационного управ\-ле\-ния~// 
Сис\-те\-мы и~средства информатики, 2013. Т.~23. №\,2. С.~171--182.
\bibitem{8-zat}
\Au{Зацаринный А.\,А., Сучков~А.\,П., Козлов~С.\,В.} Особенности проектирования 
и~функционирования сис\-те\-мы ситуационных центров~// Системы высокой доступности, 
2012. Т.~8. №\,1. С.~12--21.
\bibitem{9-zat}
\Au{Doran G.\,T.} There's a~S.M.A.R.T.\ way to write management's goals and objectives~// 
Manag. Rev., 1981. Vol.~70. Iss.~11. P.~35--36.
 \end{thebibliography}

 }
 }

\end{multicols}

\vspace*{-6pt}

\hfill{\small\textit{Поступила в~редакцию 23.08.16}}

%\vspace*{8pt}

\newpage

\vspace*{-24pt}

%\hrule

%\vspace*{2pt}

%\hrule

%\vspace*{8pt}


\def\tit{SYSTEMS ENGINEERING APPROACHES TO~THE~ESTABLISHMENT 
OF~A~SYSTEM FOR~DECISION SUPPORT BASED ON~SITUATIONAL ANALYSIS}

\def\titkol{Systems engineering approaches to~the~establishment 
of~a~system for~decision support based on~situational analysis}

\def\aut{A.\,A.~Zatsarinny and A.\,P.~Suchkov}

\def\autkol{A.\,A.~Zatsarinny and A.\,P.~Suchkov}

\titel{\tit}{\aut}{\autkol}{\titkol}

\vspace*{-9pt}


\noindent
Institute of Informatics Problems, 
Federal Research Center ``Computer Sciences and Control'' of the 
Russian Academy of Sciences, 44-2~Vavilov Str., Moscow 119333, 
Russian Federation



\def\leftfootline{\small{\textbf{\thepage}
\hfill INFORMATIKA I EE PRIMENENIYA~--- INFORMATICS AND
APPLICATIONS\ \ \ 2016\ \ \ volume~10\ \ \ issue\ 4}
}%
 \def\rightfootline{\small{INFORMATIKA I EE PRIMENENIYA~---
INFORMATICS AND APPLICATIONS\ \ \ 2016\ \ \ volume~10\ \ \ issue\ 4
\hfill \textbf{\thepage}}}

\vspace*{3pt}

 
\Abste{The article discusses the issues of decision-making support systems (DMSS) 
creation based on the situational analysis of the current and projected situation in the 
controlled space. Typically, such control systems in real time are based on situational 
centers, which are sets of information, software, hardware, and staff implementing 
information technology to monitor the situation and its situational analysis to develop 
solutions and algorithms application of control actions. The paper considers 
characteristics of the DMSS components, implementing the full management cycle from 
goal setting to execution control decisions. It is noted that the implementation of the 
decision support system depends on the level of management~--- strategic, operational, tactical, basic, and 
functional features and methods of analysis of the situation at different levels of the 
control system.}

\KWE{situational analysis; system of decision-making process support; management 
system; situational center}

\DOI{10.14357/19922264160411} 

%\vspace*{-9pt}

%\Ack
%\noindent


%\vspace*{3pt}

  \begin{multicols}{2}

\renewcommand{\bibname}{\protect\rmfamily References}
%\renewcommand{\bibname}{\large\protect\rm References}

{\small\frenchspacing
 {%\baselineskip=10.8pt
 \addcontentsline{toc}{section}{References}
 \begin{thebibliography}{9}

\bibitem{1-zat-1}
Strategiya natsional'noy bezopasnosti Rossiyskoy Fe\-de\-ra\-tsii [The National Security Strategy of 
the Russian Federation]. Approved by the Decree of the President of the Russian Federation 
No.\,683, 31.12.2015.
\bibitem{2-zat-1}
\Aue{Zatsarinny, A.\,A., and A.\,P.~Suchkov.} 2012. Nekotorye podkhody k~situatsionnomu 
analizu potokov sobytiy [Some approaches to the situational analysis of the flows of events]. 
\textit{Otkrytoe obrazovanie} [Open Education] 1:39--45.
\bibitem{3-zat-1}
\Aue{Beer, S.} 1972. \textit{Brain of the firm}. Allen Lane, The Penguin Press, London; Herder 
and Herder, USA. 416~p. 

\bibitem{5-zat-1}
\Aue{Grant, Т., and B. Кoote.} 2005. Comparing OODA \& other models as operational view C2 
architecture. \textit{10th Command and Control Research Technology Symposium 
(International) Proceedings}. McLean, VA. USA. 
\bibitem{4-zat-1}
\Aue{Ivlev, A.\,A.} 2008. \textit{Osnovy teorii Boyda. Napravleniya razvitiya, primeneniya 
i~realizatsii} [Fundamentals of the theory of Boyd. Areas of development, application, and 
implementation]. SlideShare. Available at: {\sf http://www.slideshare.net/defensenetwork/ss-10380168} (accessed  October~29, 2016).
\bibitem{6-zat-1}
\Aue{Bosov, A.\,V., A.\,A.~Zatsarinny, A.\,P.~Suchkov}. 2010. Nekotorye obshchie podkhody 
k~formirovaniyu funktsional'nykh trebovaniy k~situatsionnym tsentram i~ikh realizatsii [Some 
common approaches to the formation of functional requirements for situation centers and their 
implementation]. \textit{Sistemy i~Sredstva Informatiki~--- Systems and Means of Informatics} 
20(3):98--125.
\bibitem{7-zat-1}
\Aue{Suchkov, A.\,P.} 2013. Formirovanie sistemy tseley dlya si\-tu\-a\-tsi\-on\-no\-go upravleniya 
[The formation of the objective system to situational management]. \textit{Sistemy i~Sredstva 
Informatiki~--- Systems and Means of Informatics} 23(2):171--182.
\bibitem{8-zat-1}
\Aue{Zatsarinny, A.\,A., A.\,P.~Suchkov, and S.\,V.~Kozlov}. 2012. Osobennosti proektirovaniya 
i~funktsionirovaniya sistemy situatsionnykh tsentrov [Features of the design and functioning of 
the situational centers ]. \textit{Sistemy Vysokoy Dostupnosti} [High Availability Systems]  
8(1):12--21.
\bibitem{9-zat-1}
\Aue{Doran, G.\,T.} 1981. There's a~S.M.A.R.T. way to write management's goals and 
objectives. \textit{Manag. Rev.} 70(11):35--36.
\end{thebibliography}

 }
 }

\end{multicols}

\vspace*{-6pt}

\hfill{\small\textit{Received August 23, 2016}}

\vspace*{-12pt}

\Contr

\noindent
\textbf{Zatsarinny Alexander A.} (b.\ 1951)~--- Doctor of Science in technology, 
professor, 
Deputy Director, Federal Research Center ``Computer Sciences and Control'' of the 
Russian Academy of Sciences, 44-2~Vavilov Str., Moscow 119333, Russian Federation; 
\mbox{AZatsarinny@ipiran.ru}


\vspace*{3pt}


\noindent
\textbf{Suchkov Alexander P.} (b.\ 1954)~--- Doctor of Science in technology, 
leading scientist, Institute of Informatics Problems, Federal Research Center 
``Computer Science and Control'' of the 
Russian Academy of Sciences, 44-2~Vavilov Str., Moscow 119333, 
Russian Federation; \mbox{Asuchkov@ipiran.ru}

 


\label{end\stat}


\renewcommand{\bibname}{\protect\rm Литература}  %11
\def\stat{listopad}

\def\tit{НЕФОРМАЛЬНАЯ АКСИОМАТИЧЕСКАЯ ТЕОРИЯ РОЛЕВЫХ 
ВИЗУАЛЬНЫХ МОДЕЛЕЙ$^*$}

\def\titkol{Неформальная аксиоматическая теория ролевых 
визуальных моделей}

\def\aut{А.\,В. Колесников$^1$, С.\,В.~Листопад$^2$, С.\,Б.~Румовская$^3$, 
В.\,И.~Данишевский$^4$}

\def\autkol{А.\,В. Колесников, С.\,В.~Листопад, С.\,Б.~Румовская, 
В.\,И.~Данишевский}

\titel{\tit}{\aut}{\autkol}{\titkol}

\index{Колесников А.\,В.}
\index{Листопад С.\,В.}
\index{Румовская С.\,Б.} 
\index{Данишевский В.\,И.}
\index{Kolesnikov A.\,V.}
\index{Listopad S.\,V.}
\index{Rumovskaya S.\,B.} 
\index{Danishevsky V.\,I.}


{\renewcommand{\thefootnote}{\fnsymbol{footnote}} \footnotetext[1]
{Работа выполнена при поддержке РФФИ (проект 16-07-00271а).}}


\renewcommand{\thefootnote}{\arabic{footnote}}
\footnotetext[1]{ Балтийский федеральный университет им.\ И.~Канта, Калининградский филиал Федерального 
исследовательского центра <<Информатика и~управление>> Российской академии наук, 
\mbox{avkolesnikov@yandex.ru}}
\footnotetext[2]{Калининградский филиал Федерального исследовательского центра <<Информатика и~управление>> 
Российской академии наук, \mbox{ser-list-post@yandex.ru}}
\footnotetext[3]{Калининградский филиал Федерального исследовательского центра <<Информатика и~управление>> 
Российской академии наук, \mbox{sophiyabr@gmail.com}}
\footnotetext[4]{Балтийский федеральный университет им.\ И.~Канта, \mbox{danishevskii.v.i@mail.ru}}
   
   \Abst{Актуальность построения неформальной аксиоматической тео\-рии ролевых 
визуальных моделей обусловлена моделированием ви\-зу\-аль\-но-об\-раз\-ных рассуждений 
в~гибридных и~синергетических интеллектуальных сис\-те\-мах. Основные исследования  
ви\-зу\-аль\-но-об\-раз\-ных рассуждений сосредоточены на специальных визуальных языках 
представления некоторых видов данных, информации и~знаний. Отсутствие 
формализованных моделей визуальных языков~--- причина высокой наукоемкости 
разработки специальных сред манипулирования и~обработки визуальных моделей. 
Построение неформальной аксиоматической тео\-рии ролевых визуальных моделей~--- шаг 
к~новому классу интеллектуальных сис\-тем, релевантных реальным коллективам, 
принимающим решения,~--- гибридным интеллектуальным сис\-те\-мам (ГиИС) с~гетерогенным 
визуальным полем, имитирующим сотрудничество, относительность и~дополнительность 
коллективного интеллекта, рассуждающим на символьных и~визуальных языках.}
  
  \KW{гибридная интеллектуальная сис\-те\-ма; гетерогенное визуальное поле; визуальный 
язык; семиотическая сис\-тема}

\DOI{10.14357/19922264160412} 


\vskip 10pt plus 9pt minus 6pt

\thispagestyle{headings}

\begin{multicols}{2}

\label{st\stat}
  
\section{Введение}

  Принятие коллективных решений~--- сложное активное взаимодействие 
участников и~обеспечение взаимопонимания между ними. Для интенсификации 
этих процессов применяются методы\linebreak визуализации информации, своеобразные 
визуальные языки, наглядно описывающие структуру, свойства и~отношения 
понятий предметной об\-ласти. Правильно составленный график или диаграмму 
значительно проще анализировать, чем\linebreak мно\-го\-стра\-нич\-ную таблицу 
с~результатами измерений, а~тем более текстовое их описание. Визуализация 
информации позволяет имитировать рассужде\-ния на основе визуальных 
образов, обладающих большей конкретностью и~интегрированностью, чем 
символические представления. 
  
  Рассуждения на визуальных образах рас\-смат\-ри\-ва\-лись в~работах 
Д.\,А.~Поспелова, Г.\,П.~Щед\-ро\-виц\-ко\-го, Ю.\,Р.~Валькмана, 
Б.\,А.~Кобринского, О.\,П.~Кузнецова, Г.\,С.~Осипова, В.\,Б.~Тарасова, 
И.\,Б.~Фоминых, Т.\,А.~Гавриловой, А.\,Е.~Янковской. Визуальные языки 
разработаны для функционального программирования, программирования на 
примерах, для конечных автоматов, потоков данных и~других областей~[1]. 
Реализация этих языков требует значительных усилий, разработки для каждого 
случая специальных сред создания, манипулирования и~обработки визуальных 
моделей. Для их снижения предлагается неформальная аксиоматическая тео\-рия 
ролевых визуальных моделей на основе принципов тео\-рии сис\-тем и~сис\-тем\-но\-го 
анализа. 

\vspace*{-6pt}
  
\section{Понятие неформальной аксиоматической теории ролевых 
визуальных моделей}
  
  Рассмотрим существо разработки и~использования визуальных моделей на 
теоретическом уровне, построив неформальную аксиоматическую тео\-рию 
языков визуального моделирования сложных сис\-тем. В~качестве обобщения 
результатов работ по имитации рассуждений на визуальных образах~[2--8] 
предлагается неформальная аксиоматическая тео\-рия визуального языка как 
семиотической системы:
  \begin{equation}
  \mathrm{vl}=\langle \mathrm{VT}, \mathrm{VS}, \mathrm{VA}, 
  \mathrm{VP}, \upsilon\tau, \upsilon\sigma, \upsilon\alpha, 
\upsilon\pi\rangle\,,
  \label{e1-ls}
  \end{equation}
где VT, VS и~VA~--- множества основных символов, синтаксических 
правил и~ак\-си\-ом-зна\-ний о~предметной области (семантических правил) 
соответственно; VP~--- множество правил вывода решений (прагматических 
правил); $\upsilon\tau$, $\upsilon\sigma$, $\upsilon\alpha$ и~$\upsilon\pi$~--- 
правила изменения множеств VT, VS, VA и~VP соответственно. 
Множества~VT, VS, VA, VP, $\upsilon\tau$, $\upsilon\sigma$, 
$\upsilon\alpha$ и~$\upsilon\pi$  из~(1) определяются выражениями:
\begin{align}
\mathrm{VT}&=\langle P,D, \mathrm{VR}\rangle\,;\label{e2-ls}\\
\mathrm{VS} &= \langle \mathrm{VT}, \mathrm{VN}, \mathrm{PRU}\rangle\,;\label{e3-ls}\\
\mathrm{VA} &= \langle \mathrm{DO}, G^{\mathrm{RES}}, G^{\mathrm{PR}}, G^R\rangle\,,\label{e4-ls}
   \end{align}
где
   \begin{gather*}
   \mathrm{DO} = \langle \mathrm{RES}, \mathrm{PR}, R\rangle\,,\\
   G^{\mathrm{RES}}:\ \mathrm{RES}\to P\,,\\ 
   G^{\mathrm{PR}}:\ \mathrm{PR}\to D\,,\\ 
   G^R:\ R\to \mathrm{VR}\,;
%   \label{e5-ls}
   \end{gather*}
   \begin{align}
   \mathrm{VP}&=\left\{\langle \mathrm{AG}, \mathrm{act}, M, W\rangle\right\}\,;\label{e6-ls}\\
   \upsilon\tau &= \langle \Delta P, \Delta D, \Delta \mathrm{VR}\rangle\,;\label{e7-ls}\\
   \upsilon\sigma &= \langle \upsilon\tau, \Delta \mathrm{VN}, \Delta \mathrm{PRU}\rangle\,;\label{e8-ls}\\
   \upsilon\alpha &= \langle \Delta \mathrm{DO}, G^{\Delta \mathrm{RES}}, 
   G^{\Delta \mathrm{PR}}, G^{\Delta R}\rangle\,; 
\label{e9-ls}
\end{align}
где
\begin{gather*}
   \Delta \mathrm{DO} = \langle \Delta \mathrm{RES}, \Delta \mathrm{PR}, \Delta R\rangle\,, %\label{e10-ls}
\\
G^{\Delta \mathrm{RES}}:\  \Delta \mathrm{RES}\to \Delta P\,,\\ 
   G^{\Delta \mathrm{PR}}:\ \Delta \mathrm{PR}\to \Delta D\,,\\
   G^{\Delta R}:\ \Delta R\to \Delta \mathrm{VR}\,;
\end{gather*}
   \begin{equation}
   \upsilon\pi = \left\{ \langle \Delta \mathrm{AG}, \Delta \mathrm{act}, \Delta M, \Delta W\rangle\right\}\,.
   \label{e12-ls}
   \end{equation}
Здесь помимо ранее введенных обозначений~$P$~--- множество визуальных 
примитивов; $D$~--- множест\-во визуальных измерений, характеризующих 
визуальные примитивы; VR$^n$~--- множество визуальных отношений между 
одним и~более примитивами~\cite{4-ls};\linebreak VN~--- словарь нетерминальных 
символов; PRU~--- множество продукционных правил; RES, PR и~$R$~--- 
множества ресурсов, свойств и~отношений соответственно; AG~--- 
множество носителей языка (экспер\-тов, элементов, агентов), которым 
адресована норма поведения (различные социальные запреты и~ограничения, 
накладываемые сообществом на отдельного носителя); $\mathrm{act}\hm\in \mathrm{ACT}$~--- 
действие, определенное на множестве действий ACT и~являющееся объектом 
нормативной регуляции (содержание нормы); $M$~--- множество систем 
модальностей, связанных с~действием, например система норм, выраженных 
деонтическими модальностями: $M_N\hm= \{\mathrm{О}, \mathrm{Р}, 
\mathrm{Б}, \mathrm{З}\}$, где О~--- <<обязательно>>, Р~--- <<разрешено>>, Б~--- 
<<безразлично>>, З~--- <<запрещено>>; $W$~--- множество миров, в~которых 
применима норма (условия приложения, обстоятельства, в~которых должно или 
не должно выполняться действие)~\cite{9-ls}; $\Delta P$, $\Delta D$, $\Delta 
\mathrm{VR}$, $\Delta \mathrm{VN}$, $\Delta \mathrm{PRU}$, $\Delta \mathrm{RES}$, 
$\Delta \mathrm{PR}$, $\Delta R$, $\Delta  \mathrm{AG}$, $\Delta M$ и~$\Delta W$~--- 
множества допустимых изменений множеств 
$P$, $D$, VR, VN, PRU, RES, PR, $R$, AG, $M$ и~$W$ 
соответственно; $\Delta \mathrm{act}$~--- множество допустимых изменений содержания 
нормы~act.

  Как показано в~\cite{10-ls}, языки профессиональной деятельности, в~том 
числе и~визуальный язык,~--- полиязыки~\cite{11-ls}. С~одной стороны, это 
обусловлено присущей языку структурированностью внешнего мира. Это 
приводит к~необходимости пред\-став\-ле\-ния в~языке знаков, обозначающих 
ресурсы, свойства, действия, структуры, ситуации, со\-сто\-яния, поведение, 
а~с~учетом деятельности субъекта управ\-ле\-ния~--- целей, задач, планов, 
оценок. В~этом случае визуальный язык может рассматриваться как 
многослойная структура, описывающая решение сложной задачи комбинацией 
нескольких взаимоувязанных процессов рассуждений на разных\linebreak \mbox{языках}.
  
  С~другой стороны, полиязыковой характер языка профессиональной 
деятельности~--- следствие эволюции систем управления в~сторону 
многомодельных, гибридных и~гибридных адаптивных сис\-тем 
управ\-ле\-ния~\cite{11-ls}. Это приводит к~узкой специ\-ализации  
управ\-лен\-цев-экс\-пер\-тов по профессиональным нишам и~к~тому, что 
информация\linebreak в~коллективе, принимающем решения, пред\-став\-ляется на широком 
спектре языков професси\-о\-нальной деятельности со своими относительно 
незави\-симыми задачами, лексикой, данными, знаниями,  
принципами~\cite{11-ls}. Стратификация по\linebreak нишам при разработке 
функциональных ГиИС редуцирует 
сложность задач до прос\-тых элементов~--- подзадач в~локальных подобластях 
мира управ\-ле\-ния~--- профессиональных нишах. 
  
  Рассмотрим подходы к~представлению визуального языка как гетерогенной 
структуры.

\begin{figure*} %fig1
\vspace*{1pt}
\begin{center}
\mbox{%
\epsfxsize=138.171mm
\epsfbox{kol-1.eps}
}
\end{center}
\vspace*{6pt}

\noindent
{\small Гетерогенное визуально-образное ядро в~гетерогенной знаково-языковой оболочке. 
Уровни знаковых и~графических высказываний: {1}~--- концептуального и~визуального 
базиса; {2}~--- о~ресурсах, действиях и~своствах; 3~--- об иерархиях ресурсов, 
действий, свойств; {4}~--- о~пространственных и~производственных структурах; 
{5}~--- о~со\-сто\-яни\-ях, ситуациях и~событиях; {6}~--- о~задачах и~проблемах; 
{7}~--- о~ моделях рассуждений экспертов; {8}~--- об интегрированной модели 
рассуждений коллективного интеллекта}
\vspace*{6pt}
\end{figure*}

  
\section{Многослойная модель визуального языка}

  В работе~\cite{11-ls} информационный язык пред\-став\-лен семейством языков 
описания ресурсов, операций, структур, ситуаций, состояния, поведения
объекта управления, а~также целей, планов и~задач.\linebreak  В~\cite{12-ls} выделено~8~уровней 
визуальных языков для реализации автоматизированных 
рас\-суж\-де\-ний в~интеллектуальных системах: (1)~концептуального и~визуального 
базиса~vl$^1$; (2)~ресурсов, действий и~свойств~vl$^2$; (3)~иерархий ресурсов, 
действий, свойств~vl$^3$; (4)~пространственных и~производственных 
структур~vl$^4$; (5)~состояний, ситуаций и~событий~vl$^5$; (6)~задач  
и~проб\-лем~vl$^6$; (7)~моделей рассуждений экспертов~vl$^7$; 
(8)~интегрированных\linebreak моделей рассуждений коллективного интеллекта~vl$^8$. 
В~этом случае у разработчика есть набор средств-ком\-по\-нен\-тов для 
конструирования из них метаязыка, описывающего решение сложной задачи 
комбинацией нескольких взаимоувязанных процессов рассуждений на разных 
языках. При этом в~зависимости от требований поставленной задачи отдельные 
уровни могут отсутствовать. Таким образом, визуальный метаязык может быть 
представлен выражением
  \begin{equation}
\mathrm{mvl}=\langle \mathrm{vl}^1, \mathrm{vl}^2, \mathrm{vl}^3, \mathrm{vl}^4, 
\mathrm{vl}^5, \mathrm{vl}^6, \mathrm{vl}^7, \mathrm{vl}^8, \mathrm{VLR}\rangle\,,
  \label{e13-ls}
  \end{equation}
где VLR~--- множество отношений между элементами языков~vl$^k$, 
$k\hm\in \mathbb{N}$, $k\hm\in [1,\,8]$. 
  
  Метаязык визуализируется <<слоеным пирогом>> (см.\ рисунок). В~его 
основании~--- словари понятий и~отношений,  
кон\-цеп\-ту\-аль\-но-ви\-зу\-альн\-ый базис, над которым строится семейство 
упорядоченных по уровням языков описания. 
  


  Как показано на рисунке, на каждом языковом уровне выделяется 
гетерогенное об\-раз\-но-ви\-зу\-аль\-ное ядро базовых для данного уровня 
знаков~VT$^k$. Визуальное ядро языков высшего уровня включает знаки ядра 
более низкого уровня $\mathrm{VT}\subseteq \mathrm{VT}^{k+1}$ и~может содержать знаки, 
сформированные вне ядра на языке более низкого уровня VT$^{k+1}\cap 
\mathrm{VN}^k\not= \varnothing$, $k\hm\in \mathbb{N}$, $k\hm\in [1,\,7]$.
  
  Рассмотрим отношения между языками различных уровней на примере их 
множеств синтаксических правил. В~первом слое расположены словари 
понятий и~отношений~--- кон\-цеп\-ту\-аль\-но-ви\-зу\-аль\-ный базис языка. 
Язык первого уровня~vl$^1$ использует эвристические правила PRU$^{-1}$ 
для построения из~$P^1$, $D^1$ и~VR$^1$~знаков производных (составных) 
отношений vr$^{n1}\hm\in \mathrm{VR}{n1}\subseteq \mathrm{VT}^1$:
  $$
  \mathrm{vl}^1\left(P^1, D^1, \mathrm{VR}^1, \mathrm{PRU}^1\right) = 
  \left\{ \mathrm{vr}^{n1}\right\}\,.
  $$
  
  В языке второго уровня~Vl$^2$ эвристики~PRU$^2$ используются, чтобы 
сформировать графические образы ресурсов res$^2\hm\in \mathrm{RES}^2\subseteq 
\mathrm{VT}^2$, действий act$^2\hm\in \mathrm{ACT}^2\subseteq \mathrm{VT}^2$ 
и~свойств pr$^2\hm\in 
\mathrm{PR}^2\subseteq \mathrm{VT}^2$ без учета их иерархичности с~помощью отношений 
определения $\mathrm{VR}_1^{n1}\subseteq \mathrm{VR}^{n1}$:
  $$
  \mathrm{vl}^2\!\left( P^1, D^1, \mathrm{VR}_1^{n1}, 
  \mathrm{PRU}^2\right) =\mathrm{RES}^2\cup \mathrm{PR}^2\cup \mathrm{ACT}^2\,.
  $$
  
  На третьем уровне отношениями включения VR$_5^{n1}\subseteq \mathrm{VR}^{n1}$ 
и~эвристиками~PRU$^3$ формализованы иерархии ресурсов res$^{n3}\hm\in 
\mathrm{RES}^{n3}\subseteq \mathrm{VT}^3$, действий act$^{n3}\hm\in \mathrm{ACT}^{n3}
\hm\subseteq \mathrm{VT}^3$ 
и~свойств pr$^{n3}\hm\in \mathrm{PR}^{n3}\subseteq \mathrm{VT}^3$:
  \begin{multline*}
  \mathrm{vl}^3\!\left(P^1, D^1, \mathrm{RES}^2, 
  \mathrm{PR}^2, \mathrm{ACT}^2, \mathrm{VR}_5^{n1}, \mathrm{PRU}^3\right) = {}\\
  {}=
\mathrm{RES}^{n3}\cup \mathrm{PR}^{n3}\cup \mathrm{ACT}^{n3}\,.
\end{multline*}
    
  Четвертый уровень на основе знаков предыду\-щих уровней, 
временн$\acute{\mbox{ы}}$х VR$_3^{n1}\subseteq \mathrm{VR}^{n1}$, 
про\-странственных VR$_4^{n1}\subseteq \mathrm{VR}^{n1}$  
и~при\-чин\-но-след\-ст\-вен\-ных VR$_6^{n1}\subseteq \mathrm{VR}^{n1}$ отношений, 
а~также\linebreak эвристик~PRU$^4$ формализует пространственные str$_1^4\hm\in 
\mathrm{STR}_1^4\subseteq \mathrm{VT}^4$, опе\-ра\-ци\-о\-наль\-но-тех\-но\-ло\-ги\-че\-ские 
str$_3^4\hm\in \mathrm{STR}_3^4\subseteq \mathrm{VT}^4$ структуры:
  \begin{multline*}
  \mathrm{vl}^4\!\left( P^1, D^1, \mathrm{RES}^{n3}, \mathrm{PR}^{n3}, 
  \mathrm{ACT}^{n3}, \mathrm{VR}_3^{n1}, \mathrm{VR}_4^{n1},\right.\\
\left.  \mathrm{VR}_6^{n1}, \mathrm{PRU}^4\right) = \mathrm{STR}_1^4\cup \mathrm{STR}_3^4\,.
\end{multline*}
  
  На пятом уровне эвристиками~PRU$^5$ формализуют зна\-ки-си\-ту\-ации 
$\mathrm{sit}^5\hm\in \mathrm{SIT}^5\subseteq \mathrm{VT}^5$ 
и~зна\-ки-со\-сто\-яния st$^5\hm\in 
\mathrm{ST}^5\subseteq \mathrm{VT}^5$:
    $$
\mathrm{vl}^5\!\left( \mathrm{STR}_1^4, \mathrm{STR}_3^4, \mathrm{PRU}^5\right) =
\mathrm{SIT}^5\cup \mathrm{ST}^5\,.
   $$
  
  На шестом уровне на основе знаков предыдущих уровней 
и~эвристик~PRU$^6$ специфицируются знаки однородных prb$^{h6} \hm\in 
\mathrm{PRB}^{h6}\subseteq \mathrm{VT}^6$ и~неоднородных prb$^{u6}\hm\in \mathrm{PRB}^{u6}\subseteq 
\mathrm{VT}^6$ задач:
   \begin{multline*}
\mathrm{vl}^6\!\left( P^1, D^1, \mathrm{RES}^{n3}, \mathrm{PR}^{n3}, 
\mathrm{ACT}^{n3}, \mathrm{VR}^{n1}, \mathrm{ST}^5,\right.\\ 
\left.\mathrm{PRU}^6\right) = \mathrm{PRB}^{h6}\cup \mathrm{PRB}^{u6}\,.
  \end{multline*}
  
  На седьмом уровне эвристиками~PRU$^7$ формируются знаки автономных 
методов решения задач $\mathrm{met}^{a7}\hm\in \mathrm{MET}^{a7}\subseteq \mathrm{VT}^7$, 
имитирующих рассуждения отдельно взятого эксперта:
  \begin{multline*}
  \mathrm{vl}^7\!\left( P^1, D^1, \mathrm{RES}^{n3}, \mathrm{PR}^{n3}, 
  \mathrm{ACT}^{n3}, \mathrm{VR}^{n1}, \mathrm{PRU}^7\right) = {}\\
  {}=
\mathrm{MET}^{a7}\,.
  \end{multline*}
  
  На восьмом уровне на основе знаков предыдущих уровней и~
эвристик~PRU$^8$ специфицируются знаки интегрированных методов 
решения задач met$^{u8}\hm\in \mathrm{MET}^{u8}\subseteq \mathrm{VT}^8$, имитирующих 
рассуждения коллектива экспертов:
  \begin{multline*}
  \mathrm{vl}^8\!\left( P^1, D^1, \mathrm{RES}^{n3}, \mathrm{PR}^{n3},
  \mathrm{ACT}^{n3}, \mathrm{VR}^{n1}, \mathrm{ST}^5,\right.\\
\left.   \mathrm{SIT}^5, 
\mathrm{PRB}^{h6}, \mathrm{PRB}^{u6}, \mathrm{MET}^{a7}, \mathrm{PRU}^8\right)=
 \mathrm{MET}^{u8}\,.
  \end{multline*}
  
  Такая многослойная модель визуального языка~--- инструмент сложного 
описания предметной области на различных уровнях обобщенности, 
формирующий его из набора более простых моделей. 

Другой подход 
к~снижению сложности по\-стро\-ения моделей предметной области практических 
задач~--- стратификация языка по профессиональным нишам.
  
\section{Модель гетерогенного визуального поля}

  Гетерогенность визуального поля, проявляющаяся в~разнообразии 
информации, обусловлена отсутствием универсального описания любой 
предметной области. На практике используются более сотни методов 
визуального структурирования. Это обусловлено существенными различиями 
в~природе, особенностях и~свойствах знаний о~различных\linebreak предметных 
областях. В~работах~[13--15] проанализированы наиболее 
известные методы визуализации, определены критерии классификации 
и~классы методов. Определив релевантность метода\linebreak классам, а~классов~--- 
классам задачам, можно разработать стратегию выбора визуальных языков для 
решения подзадач элементами ГиИС. Практические задачи требуют их 
комбинирования и~установления соответствия элементов разных языков, 
насколько это возможно. В~результате при моделировании  
ви\-зу\-аль\-но-об\-раз\-ных рассуждений в~ГиИС формируется гетерогенное 
визуальное поле, обеспечивающее взаимодействие элементов ГиИС, 
рассуждающих на разных визуальных языках.
  
  Формально гетерогенное визуальное поле может быть представлено в~виде:
  \begin{equation}
  \mathrm{GVF}=\langle \mathrm{MVL}, \mathrm{COR}^{\mathrm{VL}}\rangle\,.
  \label{e14-ls}
  \end{equation}
Здесь $\mathrm{MVL}$~--- множество визуальных метаязыков гетерогенного визуального 
поля, построенных в~соответствии с~(\ref{e13-ls}):
  \begin{gather*}
  \mathrm{MVL}=\left\{ \mathrm{mvl}_1,\ldots , \mathrm{mvl}_{N_{\mathrm{MVL}}}\right\}\,;
  \end{gather*}
  $\mathrm{COR}^{\mathrm{VL}}$~--- множество 
соответствий элементов визуальных языков, входящих в~метаязыки 
mvl$_i\hm\in \mathrm{MVL}$:
\begin{multline*}
  \mathrm{COR}^{\mathrm{VL}} ={}\\
  {}=\langle G^{\mathrm{VT}}, G^{\mathrm{VS}}, G^{\mathrm{VA}}, 
  G^{\mathrm{VP}}, G^{\upsilon\tau}, 
G^{\upsilon\sigma}, G^{\upsilon\alpha}, G^{\upsilon\pi}\rangle\,,
\end{multline*}
  где
\begin{alignat*}{3}
  G^{\mathrm{VT}}_{ij}:\ & \quad&\mathrm{VT}_i&\to \mathrm{VT}_j\,,\ &
  G^{\mathrm{VT}}_{ij}&\subseteq G^{\mathrm{VT}}\,;\\
  G_{ij}^{\mathrm{VS}}:\ & &\mathrm{VS}_i&\to \mathrm{VS}_j\,,\ &G_{ij}^{\mathrm{VS}}
 & \subseteq G^{\mathrm{VS}}\,;\\
  G_{ij}^{\mathrm{VA}}:\ &&\mathrm{VA}_i&\to \mathrm{VA}_j\,,\ &\quad G_{ij}^{\mathrm{VA}}
  &\subseteq G^{\mathrm{VA}}\,;\\
  G_{ij}^{\mathrm{VP}}:\ &&\mathrm{VP}_i&\to \mathrm{VP}_j\,,\ &\quad G_{ij}^{\mathrm{VP}}
  &\subseteq G^{\mathrm{VP}}\,;\\
  G_{ij}^{\upsilon\tau}:\ &&\upsilon\tau_i&\to \upsilon\tau_j\,,\ 
&G^{\upsilon\tau}_{ij}&\subseteq G^{\upsilon\tau}\,;\\
  G_{ij}^{\upsilon\sigma}:\ &&\upsilon\sigma_i&\to \upsilon\sigma_j\,,\ 
&G_{ij}^{\upsilon\sigma}&\subseteq G^{\upsilon\sigma}\,;\\
  G_{ij}^{\upsilon\alpha}:\ &&\upsilon\alpha_i&\to \upsilon\alpha_j\,,\ 
&G_{ij}^{\upsilon\alpha} &\subseteq G^{\upsilon\alpha}\,;\\
  G_{ij}^{\upsilon\pi}:\ &&\upsilon\pi_i&\to \upsilon\pi_j\,,\ &G_{ij}^{\upsilon\pi} 
&\subseteq G^{\upsilon\pi}\,, 
\end{alignat*}

\vspace*{-6pt}

\noindent
$$
  \hspace*{40mm}i,j\in \left[1, N_{\mathrm{MVL}}\right]\,,\enskip i\not=j\,.
  $$
 
  
  Предлагаемая неформальная аксиоматическая тео\-рия ролевых визуальных 
моделей~(\ref{e1-ls})--(\ref{e12-ls}) с~учетом моделей многослойной модели 
визуального языка~(\ref{e13-ls}) и~гетерогенного визуального  
поля~(\ref{e14-ls})\linebreak будет положена в~основу нового класса функциональных 
ГиИС, имитирующих работу коллек\-тивного интеллекта по поиску решений над 
ге\-те\-ро\-генными модельным и~визуальным полями. \mbox{Сочетание} символьных 
и~ви\-зу\-аль\-но-об\-раз\-ных рассуждений в~таких системах обеспечит их 
релевантность реальным коллективам, принимающим решения в~условиях 
сложных задач.

\vspace*{-3pt}
  
\section{Заключение}

  Предложена неформальная аксиоматическая тео\-рия ролевых визуальных 
моделей~--- основа автоматизированного решения сложных задач на основе 
визуальных образов, визуального управления. Предложена многослойная 
модель визуального языка и~формализованная модель гетерогенного 
визуального поля.
  
  Использование указанных моделей дает возможность реализовать ГиИС, 
способные динамически синтезировать интегрированную модель и~метод над 
гетерогенными модельным и~визуальным полями и~имитировать 
сотрудничество, относительность и~дополнительность коллективного 
интеллекта для поиска решений на символьных и~визуальных языках. Гибридные интеллектуальные
сис\-те\-мы 
такого класса смогут управ\-лять имитационным процессом в~зависимости от 
неопределенности проблемной ситуации: когда область явлений 
формализована (частично формализована), подключать для поиска решений 
знания экспертов из гетерогенного модельного поля, а когда есть существенная 
неопределенность, не сни\-ма\-емая точным анализом 
  и~ло\-ги\-ко-ма\-те\-ма\-ти\-че\-ски\-ми рассуждениями, привести в~действие 
механизмы ви\-зу\-аль\-но-про\-стран\-ст\-вен\-но\-го, образного мышления, 
имитируя <<скачки>> в~гибридном пространстве состояний функциональной 
ГиИС, соответству\-ющие мгновенному 
интуитивному инсайту, озарению, прерывающему  
ло\-ги\-ко-ма\-те\-ма\-ти\-че\-ские рассуждения.
  
{\small\frenchspacing
 {%\baselineskip=10.8pt
 \addcontentsline{toc}{section}{References}
 \begin{thebibliography}{99}
  \bibitem{1-ls}
  \Au{Golin E.\,J., Reiss S.\,P.} The specification of visual language syntax~// 
J.~Visual Lang. Comput., 1990. Vol.~1. P.~141--157.
  \bibitem{2-ls}
  \Au{Bowman W.\,J.} Graphic communication.~--- New York, NY, USA: John 
Wiley, 1968. 210~p. 
\bibitem{3-ls}
\Au{Lakin F.} Visual grammars for visual languages~// 6th National Conference on 
Artificial Intelligence Proceedings.~--- Menlo Park, CA, USA: AAAI 
Press, 1987. P.~683--688. 
  \bibitem{4-ls}
  \Au{Narayanan N.\,H., Hubscher R.} Visual language theory: Towards 
  a~human--computer interaction perspective~// Visual language theory.~--- New 
York, NY, USA: Springer-Verlag, 1998. P.~81--128.
 \bibitem{8-ls} %5
  \Au{Kremer R.} Visual languages for knowledge representation~//  
11th Workshop on Knowledge Acquisition, Modeling and Management, 1998. {\sf 
http://ksi.cpsc.\linebreak ucalgary.ca/KAW/KAW98/kremer}.
  \bibitem{5-ls} %6
  \Au{Осипов Г.\,С.} От ситуационного управления к~прикладной семиотике~// 
Новости искусственного интеллекта, 2002. №\,6(54). С.~3--7. 

  \bibitem{7-ls} %7
  \Au{Fitrianie S., Rothkrantz~L.\,J.\,M.}  Two-dimensional visual language 
grammar.~--- Delft, The Netherlands: Delft University of Technology, 
2008. {\sf http://mmi.tudelft.nl/ pub/siska/TSD~2DVisLangGrammar.pdf}.
\bibitem{6-ls} %8
\Au{Sibbet D.} Visual leaders: New tools for visioning, management, and 
organization change.~--- Hoboken, NJ, USA: Wiley, 2013. 229~p.
 
  \bibitem{9-ls} %9
  \Au{Тарасов В.\,Б.} Проблема понимания: настоящее и~будущее 
искусственного интеллекта~// Открытые семантические технологии 
проектирования интеллектуальных систем: Мат-лы V~Междунар. 
науч.-технич. конф.~--- Минск: БГУИР, 2015. С.~25--42.
  \bibitem{10-ls}
  \Au{Колесников А.\,В., Кириков~И.\,А.} Методология и~технология решения 
сложных задач методами функциональных гибридных интеллектуальных  
сис\-тем.~--- М.: ИПИ РАН, 2007. 387~с.
  \bibitem{11-ls}
  \Au{Колесников А.\,В.} Гибридные интеллектуальные системы. Теория 
и~технология разработки.~--- СПб.: \mbox{СПбГТУ}, 2001. 711~с.
  \bibitem{12-ls}
  \Au{Колесников А.\,В., Листопад~С.\,В.}  
Кон\-цеп\-ту\-аль\-но-ви\-зу\-аль\-ные основы виртуальных гетерогенных 
коллективов, поддерживающих принятие решений~// Гиб\-рид\-ные 
и~синергетические интеллектуальные\linebreak системы: Мат-лы III~Всеросс. 
Поспеловской конф. с~междунар. участием.~--- Калининград: 
БФУ им.\ И.~Канта, 2016. С.~8--56.
  \bibitem{13-ls}
  \Au{Mazza R.} Introduction to information visualization.~--- London:  
Springer-Verlag, 2009. 139~p.


  \bibitem{14-ls}
  \Au{Lengler R., Eppler~M. }A~periodic table of visualization methods~// Visual 
literacy: An e-learning tutorial on visualization for communication, engineering 
and business. {\sf http://www.visual-literacy.org/periodic\_ table/periodic\_table.html}.
  \bibitem{15-ls}
  \Au{Li K., Tiwari A., Alcock~J., Bermell-Garcia~P.} Categorisation of 
visualisation methods to support the design of human--computer interaction 
systems~// Appl. Ergon., 2016. Vol.~55. P.~85--107.
 \end{thebibliography}

 }
 }

\end{multicols}

\vspace*{-3pt}

\hfill{\small\textit{Поступила в~редакцию 16.10.16}}

%\vspace*{8pt}

\newpage

\vspace*{-28pt}

%\hrule

%\vspace*{2pt}

%\hrule

%\vspace*{8pt}


\def\tit{INFORMAL AXIOMATIC THEORY OF~THE~ROLE VISUAL MODELS}

\def\titkol{Informal axiomatic theory of~the~role visual models}

\def\aut{A.\,V.~Kolesnikov$^{1,2}$, S.\,V.~Listopad$^2$, S.\,B.~Rumovskaya$^2$, 
and~V.\,I.~Danishevsky$^1$}

\def\autkol{A.\,V.~Kolesnikov, S.\,V.~Listopad, S.\,B.~Rumovskaya, 
and~V.\,I.~Danishevsky}

\titel{\tit}{\aut}{\autkol}{\titkol}

\vspace*{-9pt}

 \noindent
  $^1$Immanuel Kant Baltic Federal University, 14~A.~Nevskogo Str., Kaliningrad 
236041, Russian Federation

   \noindent
   $^2$Kaliningrad Branch of the Federal Research Center ``Computer Science and 
Control'' of the Russian Academy\linebreak
$\hphantom{^1}$of Sciences, 5~Gostinaya Str, Kaliningrad 236000, 
Russian Federation



\def\leftfootline{\small{\textbf{\thepage}
\hfill INFORMATIKA I EE PRIMENENIYA~--- INFORMATICS AND
APPLICATIONS\ \ \ 2016\ \ \ volume~10\ \ \ issue\ 4}
}%
 \def\rightfootline{\small{INFORMATIKA I EE PRIMENENIYA~---
INFORMATICS AND APPLICATIONS\ \ \ 2016\ \ \ volume~10\ \ \ issue\ 4
\hfill \textbf{\thepage}}}

\vspace*{14pt}
  
  
  
   \Abste{The relevance of creation of the informal axiomatic theory of the role visual models is caused by 
modeling visual-imaginative reasoning in hybrid and synergistic intelligent systems. Most of the research in 
the field of visual-imaginative reasoning is focused on developing special visual languages to represent 
certain kinds of data, information, and knowledge. The lack of formal models of the visual languages is the 
cause of high research and development intensity of special media for handling and processing of visual 
models. Creation of the informal axiomatic theory of the role visual models is a~step to a~new class of 
intelligent systems that are relevant to the real decision-making teams, i.\,e., hybrid intelligent systems with 
heterogeneous visual field, imitating cooperation, complementarity, and relativity of collective intelligence, 
reasoning using the symbolic and visual languages.}

\vspace*{1pt}
   
   \KWE{hybrid intelligent system; heterogeneous visual field; visual language; semiotic system}
   
   \vspace*{1pt}
   
\DOI{10.14357/19922264160412} 

%\vspace*{6pt}

\Ack
  \noindent
  The research was supported by the Russian Foundation for Basic Research 
(project No.\,16-07-00271a).


\vspace*{12pt}

  \begin{multicols}{2}

\renewcommand{\bibname}{\protect\rmfamily References}
%\renewcommand{\bibname}{\large\protect\rm References}

{\small\frenchspacing
 {%\baselineskip=10.8pt
 \addcontentsline{toc}{section}{References}
 \begin{thebibliography}{99}
  
    \bibitem{1-ls-1}
  \Aue{Golin, E.\,J., and S.\,P.~Reiss.} 1990. The specification of visual language 
syntax. \textit{J.~Visual Lang. Comput.} 1:141--157.
  \bibitem{2-ls-1}
  \Aue{Bowman, W.\,J.} 1968. \textit{Graphic communication}. New York, NY: 
John Wiley. 210~p.
  \bibitem{3-ls-1}
  \Aue{Lakin, F.} 1987. Visual grammars for visual languages. \textit{6th National 
Conference on Artificial Intelligence Proceedings}. Menlo Park, 
CA: AAAI Press. 683--688. 
  \bibitem{4-ls-1}
  \Au{Narayanan, N.\,H., and R. Hubscher}. 1998. Visual language theory: Towards 
a~human--computer interaction perspective. \textit{Visual language theory}. New 
York, NY: Springer-Verlag. 81--128.
\bibitem{8-ls-1} %5
  \Aue{Kremer, R.} 1998. Visual languages for knowledge representation. 
\textit{11th Workshop on Knowledge Acquisition, Modeling and Management}. Available at: 
{\sf http:// ksi.cpsc.ucalgary.ca/KAW/KAW98/kremer/} (accessed September~5, 2016).
  \bibitem{5-ls-1} %6
  \Aue{Osipov, G.\,S.} 2002. Ot situatsionnogo upravleniya k~prikladnoy semiotike 
[From situational management to applied semiotics]. \textit{Novosti iskusstvennogo 
intellekta} [Artificial Intelligence News] 6(54):3--7.

\bibitem{7-ls-1} %7
  \Aue{Fitrianie, S., and L.\,J.\,M.~Rothkrantz}. 2008.
  \textit{Two-dimensional visual language 
grammar}. Delft, The Netherlands: Delft University of Technology.
Available at: {\sf http:// mmi.tudelft.nl/pub/siska/TSD2DVisLangGrammar.pdf} 
(accessed September~5, 2016).

 
  \bibitem{6-ls-1} %8
  \Aue{Sibbet, D.} 2013. \textit{Visual leaders: New tools for visioning, 
management, and organization change}. Hoboken, NJ: Wiley. 229~p.
  
 
  \bibitem{9-ls-1}
  \Aue{Tarasov, V.\,B.} 2015. Problema ponimaniya: nastoyashchee i~budushchee 
iskusstvennogo intellekta [The problem of understanding: The present and the future 
of artificial intelligence]. \textit{5th Scientific and Technical 
Conference (International) ``Open Semantic 
Technologies for Intelligent Systems'' Proceedings}. Minsk: BSUIR. 
25--42.
  \bibitem{10-ls-1}
  \Aue{Kolesnikov, A.\,V., and I.\,A.~Kirikov}. 2007. \textit{Metodologiya 
i~tekhnologiya resheniya slozhnykh zadach metodami funktsional'nykh gibridnykh 
intellektual'nykh sistem} [Methodology and technology of solving complex problems 
by the methods of functional hybrid intelligent systems]. Moscow: IPI RAN. 387~p.
  \bibitem{11-ls-1}
  \Aue{Kolesnikov, A.\,V.} 2001. \textit{Gibridnye intellektual'nye sistemy. Teoriya 
i~tekhnologiya razrabotki} [Hybrid intelligent systems: Theory and technology of 
development]. St. Petersburg: SPbGTU Publ. 711~p.
  \bibitem{12-ls-1}
  \Aue{Kolesnikov, A.\,V., and S.\,V.~Listopad}. 2016. Kon\-tsep\-tu\-al'\-no-vi\-zu\-al'\-nye 
osnovy virtual'nykh geterogennykh kollektivov, podderzhivayushchikh prinyatie 
resheniy [Conceptual and visual basics of virtual heterogeneous teams\linebreak supporting 
decision-making]. \textit{Gibridnye i~sinergeticheskie intellektual'nye sistemy: 
mat-ly III~Vseross. Pospelovskoy konf. s~mezhdunar. uchastiem}  
[3rd All-Russia Pospelov Conference with International Participation ``Hybrid and 
synergistic intelligent systems'' Proceedings]. Kaliningrad: IKBFU Publ. 8--56.

\pagebreak

  \bibitem{13-ls-1}
  \Aue{Mazza, R.} 2009. \textit{Introduction to information visualization}. London: 
Springer-Verlag. 139~p.


  \bibitem{14-ls-1}
  \Aue{Lengler, R., and M.~Eppler}. A~periodic table of visualization methods. 
\textit{Visual 
literacy: An e-learning tutorial on visualization for communication, engineering 
and business}. 
Available at: {\sf http://www.visual-literacy.org/periodic\_\linebreak table/periodic\_table.html} 
(accessed September~5,\linebreak 2016).
  \bibitem{15-ls-1}
  \Aue{Li, K., A.~Tiwari, J.~Alcock, and P.~Bermell-Garcia}. 2016. Categorisation 
of visualisation methods to support the design of human--computer interaction 
systems. \textit{Appl. Ergon.} 55:85--107.
\end{thebibliography}

 }
 }

\end{multicols}

\vspace*{-3pt}

\hfill{\small\textit{Received October 16, 2016}}
  
  \Contr
  
  \noindent
  \textbf{Kolesnikov Alexander V.} (b.\ 1948)~--- Doctor of Science in 
technology; professor, Department of Telecommunications, Immanuel Kant Baltic 
Federal University, 14~A.~Nevskogo Str., Kaliningrad 236041, Russian Federation; 
senior scientist, Kaliningrad Branch of the Federal Research Center ``Computer 
Science and Control'' of the Russian Academy of Sciences, 5~Gostinaya Str, 
Kaliningrad 236000, Russian Federation, \mbox{avkolesnikov@yandex.ru} 
  
  \vspace*{4pt}
  
  \noindent
  \textbf{Listopad Sergey V.} (b.\ 1984)~--- Candidate of  Science (PhD) in 
technology, senior scientist, Kaliningrad Branch of the Federal Research Center 
``Computer Science and Control'' of the Russian Academy of Sciences, 5~Gostinaya 
Str, Kaliningrad 236000, Russian Federation, \mbox{ser-list-post@yandex.ru} 
  
  \vspace*{4pt}
  
  \noindent
  \textbf{Rumovskaya Sophiya B.} (b.\ 1985)~--- programmer~I, Kaliningrad 
Branch of the Federal Research Center ``Computer Science and Control'' of the 
Russian Academy of Sciences, 5~Gostinaya Str, Kaliningrad 236000, Russian 
Federation, \mbox{sophiyabr@gmail.com}  
  
  
  \vspace*{4pt}
  
  \noindent
  \textbf{Danishevskii Vladislav I.} (b.\ 1992)~--- PhD student, Immanuel Kant 
Baltic Federal University, 14~A.~Nevskogo Str., Kaliningrad 236041, Russian 
Federation; \mbox{danishevskii.v.i@mail.ru} 
\label{end\stat}


\renewcommand{\bibname}{\protect\rm Литература} 
     %12

\let\varvec\vec
%\renewcommand{\vec}[1]{\mathbf{#1}}
\renewcommand{\vec}[1]{\boldsymbol{#1}}

\def\stat{karasikov}

\def\tit{КЛАССИФИКАЦИЯ ВРЕМЕННЫХ РЯДОВ В~ПРОСТРАНСТВЕ ПАРАМЕТРОВ ПОРОЖДАЮЩИХ МОДЕЛЕЙ$^*$}

\def\titkol{Классификация временных рядов в~пространстве параметров порождающих моделей}

\def\aut{М.\,Е.~Карасиков$^1$, В.\,В.~Стрижов$^2$}

\def\autkol{М.\,Е.~Карасиков, В.\,В.~Стрижов}

\titel{\tit}{\aut}{\autkol}{\titkol}

\index{Карасиков М.\,Е.}
\index{Стрижов В.\,В.}
\index{Karasikov M.\,E.}
\index{Strijov V.\,V.}


{\renewcommand{\thefootnote}{\fnsymbol{footnote}} \footnotetext[1]
{Работа выполнена при финансовой поддержке РФФИ (проект 16-37-00485).}}


\renewcommand{\thefootnote}{\arabic{footnote}}
\footnotetext[1]{Московский физико-технический институт, Сколковский институт науки и~технологий, 
    \mbox{karasikov@phystech.edu}}
\footnotetext[2]{Вычислительный центр им.\ А.\,А.~Дородницына Федерального исследовательского 
    центра <<Информатика и~управ\-ле\-ние>> Российской академии наук, 
    \mbox{strijov@ccas.ru}}

      
    

\Abst{Работа посвящена задаче многоклассовой признаковой классификации временных рядов.
    Признаковая классификация временных рядов заключается в~сопоставлении каждому 
    временному ряду его краткого признакового описания и~позволяет решать задачу 
    классификации в~пространстве признаков.
Исследуются методы построения пространства признаков временн$\acute{\mbox{ы}}$х рядов,
    при этом временной ряд рассматривается как последовательность сегментов, 
    аппроксимируемых некоторой параметрической моделью, параметры которой используются 
    в~качестве их признаковых описаний.
    Построенное признаковое описание сегмента временного ряда наследует от 
    модели аппроксимации такое полезное свойство, как инвариантность относительно 
    сдвига.
    Для решения задачи классификации в~качестве признаковых описаний временн$\acute{\mbox{ы}}$х рядов 
    предлагается использовать распределения параметров аппроксимирующих сегменты 
    моделей, что обобщает базовые методы, использующие непосредственно сами параметры 
    аппроксимирующих моделей.
    Проведен ряд вычислительных экспериментов на реальных данных, показавших 
    высокое качество решения задачи многоклассовой классификации.
    Эксперименты показали превосходство предлагаемого метода над базовым 
    и~многими распространенными методами классификации временных рядов на всех 
    рассмотренных наборах данных.}

\KW{временные ряды; многоклассовая классификация; сегментация временных рядов; 
гиперпараметры аппроксимирующей модели; модель авторегрессии; дискретное 
преобразование Фурье}

\DOI{10.14357/19922264160413} 


\vskip 10pt plus 9pt minus 6pt

\thispagestyle{headings}

\begin{multicols}{2}

\label{st\stat}

\section{Введение}
%\label{sec:introduction}

Временн$\acute{\mbox{ы}}$м рядом~$x$ будем называть конечную упорядоченную 
последовательность чисел
$$
x = \left[x^{(1)}, \dots, x^{(t)}\right]\,.
$$
Временн$\acute{\mbox{ы}}$е ряды являются объектом исследования 
в~таких задачах анализа данных, как прогнозирование,
  обнаружение аномалий, сегментация~\cite{geurts2005segment},
  клас\-те\-ри\-за\-ция и~классификация~\cite{geurts2005segment}.
Обзор по задачам и~методам анализа временн$\acute{\mbox{ы}}$х рядов дается 
в~\cite{Esling:2012:TDM:2379776.2379788}.
Последние годы связаны с~ростом интереса к~данной области, проявляющемся 
в~непрекращающемся предложении новых методов анализа временных рядов~--- метрик, 
алгоритмов сегментации, кластеризации и~др.

В данной работе рассматривается задача классификации временн$\acute{\mbox{ы}}$х рядов, 
возникающая во многих приложениях
  (медицинская диагностика по электрокардиограммам~\cite{basil2014automatic} 
  и~электроэнцефалограммам~\cite{alomari2013automated},
  классификация типов физической активности по данным 
  акселерометра~\cite{Kwapisz:2011:ARU:1964897.1964918},
  верификация динамических подписей~\cite{gruber2006signature}~и~т.\,д.).

Формально задача классификации в~общем виде ставится следующим образом.
Пусть~$X$~--- множество описаний объектов произвольной природы,
$Y$~--- конечное множество меток классов.
Предполагается существование целевой функции~--- отоб\-ра\-же\-ния~$y:\;X
\hm\to Y$,
значения которого известны только на~объектах обучающей выборки
$$
    \mathfrak{D} = \left\{(x_1,y_1),\dots,(x_m,y_m)\right\} \subset X\times Y\,.
$$
Требуется построить классификатор~$a:\;X\to Y$~--- отображение,
приближающее целевую функцию~$y$ на~множестве~$X$.
При $|Y|\hm>2$ задачу классификации будем называть многоклассовой.
Задачей классификации временн$\acute{\mbox{ы}}$х рядов будем называть задачу классификации, 
в~которой объектами классификации являются временн$\acute{\mbox{ы}}$е ряды.

Задание метрики, или функции расстояния~\cite{Ding:2008:QMT:1454159.1454226}, 
на парах временн$\acute{\mbox{ы}}$х рядов позволяет применять метрические методы классификации.
При удачном выборе метрики классификация может производиться простейшими метрическими 
алгоритмами классификации, например методом ближайшего соседа~\cite{jeong2011weighted}.
Данный подход к~решению задачи классификации временн$\acute{\mbox{ы}}$х рядов чрезвычайно распространен 
в~силу того, что позволяет свести исходную задачу классификации временн$\acute{\mbox{ы}}$х рядов 
к~задаче выбора метрики.

Второй подход к~решению задачи классификации состоит в~построении для 
каждого временн$\acute{\mbox{о}}$го ряда его информативного признакового 
описания~$\mathbf{f}:\;X\hm\to\mathbb{R}^n$, позволяющего строить точные 
классификаторы с~хорошей обобщающей способностью.
Построение информативного пространства признаков исходных объектов
 множества~$X$,\linebreak
  позволяющего добиться заданной точности классификации и~значительно 
 упрощающего по\-сле\-ду\-ющий анализ, является важнейшим этапом решения задачи классификации.
Признаки могут задавать\-ся экспертом.
Так, в~работе~\cite{Nanopoulos01feature-basedclassification} предлагается использовать 
в~качестве признаков статистические функции (среднее, отклонения от среднего, 
коэффициенты эксцесса и~др.).
Стоит заметить, что при таком подходе к~построению пространства признаков 
час\-то удается добиться необходимого качества классификации путем выбора 
соответствующих конкретной задаче признаков (см., например,~\cite{wiens2012patient}), 
а~сам выбор признаков становится важной технической задачей.
Другой метод построения пространства признаков заключается в~задании 
параметрической регрессионной или аппроксимирующей модели временн$\acute{\mbox{о}}$го ряда.
Тогда в~качестве признаков временн$\acute{\mbox{ы}}$х рядов будут выступать параметры 
настроенной модели.
В~работе~\cite{morchen2003time} в~качестве признаков предлагается 
использовать коэффициенты дискретного преобразования Фурье (DFT) 
и~дискретного вейв\-лет-пре\-обра\-зо\-ва\-ния (DWT), 
а~в~\cite{kini2013large, kuznetsov2015description}~--- модели авторегрессии.
%В~\cite{kalliovirta2015gaussian} исследуются свойства смеси моделей авторегрессии.
Таким образом, при данном методе построения признаковых описаний 
возникает задача выбора аппроксимирующей модели временн$\acute{\mbox{о}}$го ряда.
%Об исчерпывающих исследованиях этой задачи авторам неизвестно.

В работе исследуются методы классификации временн$\acute{\mbox{ы}}$х рядов, использующие 
в~качестве их признаковых описаний параметры аппроксимирующих моделей.
Приводится сравнение моделей аппроксимации.
Из временн$\acute{\mbox{о}}$го ряда могут извлекаться сегменты~--- его 
подпоследовательности, для которых признаковые описания строятся так же, как и~для 
исходных временн$\acute{\mbox{ы}}$х рядов.
Использование подпоследовательностей позволяет обобщить алгоритмы классификации.
 Так, в~работе~\cite{geurts2005segment} предлагается алгоритм классификации 
 временн$\acute{\mbox{ы}}$х\linebreak
  рядов методом голосования их случайных сегментов 
 (непрерывных подпоследовательностей со\linebreak случайным начальным элементом).
В~данной\linebreak
 работе предлагается алгоритм классификации вре\-мен\-н$\acute{\mbox{ы}}$х рядов в~пространстве 
параметров распределений признаков их сегментов, который сравнивается с~родственным 
ему алгоритмом голосования сегментов~\cite{geurts2005segment}.
В~разд.~7 приводятся результаты экспериментов на реальных данных, показывающие 
высокое качество и~общность предлагаемого алгоритма в~сочетании с~методом 
признаковых описаний временн$\acute{\mbox{ы}}$х рядов параметрами аппроксимирующих их моделей.


\section{Постановка задачи}
%\label{sec:problem_statement}

Поставим задачу многоклассовой классификации временн$\acute{\mbox{ы}}$х рядов в~общем виде.
Пусть $(X,\rho)$~--- метрическое пространство временн$\acute{\mbox{ы}}$х рядов, 
$Y$~--- множество меток классов, $\mathfrak{D}\subset X\times Y$~--- 
конечная обучающая выборка.

Пусть~$S$~--- процедура сегментации:
  \begin{equation}
  \label{eq:segmentation}
  S(x)\subset 2^{\mathbf{S}(x)}\,,
  \end{equation}
  где $\mathbf{S}(x)$~--- множество всех сегментов временн$\acute{\mbox{о}}$го ряда~$x\hm\in X$;
  $\mathbf{f}(S(x))\hm\in\mathbb{R}^n$~--- процедура по\-стро\-ения 
  признакового описания набора сегментов;
  $b$~--- алгоритм многоклассовой классификации:
  \begin{equation}
  \label{eq:classification}
  b:\;\mathbb{R}^n\to Y\,.
  \end{equation}

Рассмотрим семейство~$A=\left\{a:\;X\hm\to Y\right\}$ алгоритмов классификации вида
\begin{equation}
\label{eq:classifiers}
a=b\circ \mathbf{f}\circ S\,.
\end{equation}

Пусть задана функция потерь
$
\mathscr{L}:\;X\times Y\times Y\hm\to \mathbb{R}
$
и функционал качества
\begin{equation}
\label{eq:empirical_risk}
Q(a,\mathfrak{D})=
\fr{1}{|\mathfrak{D}|}
\sum\limits_{(x,y)\in\mathfrak{D}}\mathscr{L}\left(x, a(x),y\right)\,.
\end{equation}

В качестве методов обучения~$\mu(\mathfrak{D})\in A$ будем использовать следующие:
$$
\mu_{\mathbf{f},S}(\mathfrak{D})=\hat{b}\circ \mathbf{f}\circ S\,,
$$
где~$\hat{b}$~--- минимизатор эмпирического риска:
$$
\hat{b}=\argmin_{b}Q(b\circ \mathbf{f}\circ S,\mathfrak{D})\,.
$$

Оптимальный метод обучения определяется по скользящему контролю:
$$
\mu^* = \argmin_{\mathbf{f},\,S}\widehat{\mathrm{CV}}(\mu_{\mathbf{f},S},\mathfrak{D})\,,
$$
где $\widehat{\mathrm{CV}}(\mu,\mathfrak{D})$~--- внешний критерий качества метода 
обучения~$\mu$;
при этом исходная обучающая выборка~$\mathfrak{D}$ случайно разбивается~$r$~раз 
на обучающую и~контрольную~($\mathfrak{D}\hm=\mathfrak{L}_1\sqcup\mathfrak{T}_1=
\dots=\mathfrak{L}_r\sqcup\mathfrak{T}_r$),
\begin{equation}
\label{eq:cross_validation}
\widehat{\mathrm{CV}}(\mu,\mathfrak{D})=
\fr{1}{r}\sum\limits_{v=1}^{r}Q(\mu(\mathfrak{L}_v),\mathfrak{T}_v)\,,
\end{equation}
где
\begin{equation}
\label{eq:total_quality}
Q(a,\mathfrak{T})=\fr{1}{|\mathfrak{T}|}
\sum\limits_{(x,y)\in\mathfrak{T}}\vec1\{a(x)=y\}\,.
\end{equation}
Средняя точность (precision) классификации объектов класса~$c\hm\in Y$ оценивается функционалом скользящего контроля~\eqref{eq:cross_validation} с~модифицированным функционалом качества~$Q$:
\begin{equation}
\label{eq:class_quality}
Q_c(a,\mathfrak{T})=
\fr{\left|\left\{(x,y)\in\mathfrak{T}\,|\,a(x)=y=
c\right\}\right|}{\left|\left\{(x,y)\in\mathfrak{T}\,|\,y=c\right\}\right|}\,.
\end{equation}

\section{Сегментация временных рядов}
%\label{sec:segmenting}

\noindent
\textbf{Определение 1.}\
Сегментом временн$\acute{\mbox{о}}$го ряда~$x\hm=[x^{(1)},\dots,x^{(t)}]$ будем 
называть любую его непрерывную подпоследовательность~$s\hm=[x^{(i)}]_{i=t_0}^{t_1}$, 
$1\hm\leqslant t_0\hm\leqslant t_1\hm\leqslant t.$

\smallskip

\noindent
\textbf{Определение 2.}\
Под сегментацией будем понимать отображение временн$\acute{\mbox{ы}}$х рядов 
во множество их сегментов~\eqref{eq:segmentation}.


\smallskip

\subsection*{Примеры}

\begin{enumerate}[1.]
\item
  Тривиальная сегментация
  \begin{equation}
  \label{eq:equal_fragmenting}
  S(x)=\{x\},\ \forall x\in X\,.
  \end{equation}

\item
  Случайное выделение сегментов некоторой длины~$\ell$~\cite{geurts2005segment}.

\item
  Важным является случай квазипериодичности временн$\acute{\mbox{о}}$го 
  ряда, когда сам ряд состоит из похожих в~определенном смысле сегментов, 
  называемых периодами:
  \begin{multline*}
%  \label{eq:periodic}
  x=\left[\underbrace{x^{(1)},\dots,x^{(t_1)}}_{s^{(1)}},\underbrace{x^{(t_1+1)},
\dots,x^{(t_2)}}_{s^{(2)}},\dots\right.\\
\left.\dots,\underbrace{x^{(t_{p-1}+1)},\dots,x^{(t)}}_{s^{(p)}}
  \right]\,.
  \end{multline*}
  Тогда в~качестве процедуры сегментации можно взять разбиение на периоды:
  \begin{equation*}
  %\label{eq:period_segmentation}
  S(x)= \left\{s^{(1)},\dots,s^{(p)}\right\}\,.
  \end{equation*}
\end{enumerate}


\section{Аппроксимирующая модель сегмента временного ряда}
%\label{sec:regression_model}

Поскольку сегмент временн$\acute{\mbox{о}}$го ряда сам является временн$\acute{\mbox{ы}}$м 
рядом, в~этом разделе слово сегмент будем опускать.

\smallskip

\noindent
\textbf{Определение 3.}\
Параметрической аппроксими\-ру\-ющей моделью временн$\acute{\mbox{о}}$го ряда~$x$ будем называть отображение
\begin{equation}
\label{eq:regression}
g:\;\mathbb{R}^n\times X\to X\,.
\end{equation}

\smallskip

В слово <<аппроксимирующая>> вкладывается тот смысл, что модель должна приближать 
временн$\acute{\mbox{о}}$й ряд в~пространстве $(X,\rho)$, т.\,е.\ 
для некоторого $\mathbf{w}\hm\in \mathbb{R}^n$
$$
g(\mathbf{w},x)=\hat{x}\,,
$$
где
$$
\rho(\hat{x},x)<\varepsilon\,.
$$
При этом естественно взять в~качестве признакового описания временн$\acute{\mbox{о}}$го ряда~$x$
 вектор оптимальных параметров его модели.

\smallskip
%\label{def:feature_description}

\noindent
\textbf{Определение 4.}\
Признаковым описанием вре\-мен\-н$\acute{\mbox{о}}$го ряда~$x$, порожденным 
параметрической моделью~$g(\mathbf{w},x)$, назовем вектор оптимальных 
па\-ра\-мет\-ров этой модели:
\begin{equation}
\label{eq:feature_solution}
\mathbf{w}_g(x)=
\argmin_{\mathbf{w}\in \mathbb{R}^n} \rho\left(g(\mathbf{w},x),x\right)\,.
\end{equation}


В качестве аппроксимирующих моделей предлагается использовать следующие.
\begin{enumerate}[1.]
\item \textbf{Модель линейной регрессии}.
Пусть задан $r$-ком\-по\-нент\-ный временной 
ряд (например, время и~три пространственные координаты):
$$
x = [\vec{x}^{(1)}, \dots, \vec{x}^{(t)}]\,,
$$
где
$$
\vec{x}^{(k)}=[x_1^{(k)},\dots,x_r^{(k)}]^{\mathrm{T}},\enskip k=1,\dots,t\,.
$$
Рассмотрим модель линейной регрессии одной из компонент 
временн$\acute{\mbox{о}}$го ряда на остальные компоненты как аппроксимирующую модель:
$$
g(\mathbf{w},x)=\left[\hat{\vec{x}}^{(1)},\dots,\hat{\vec{x}}^{(t)}\right]\,,
$$
 где 
 $$
\hat{\vec{x}}^{(k)}=\left[x_1^{(k)},\dots,x_{r-1}^{(k)},\hat{x}_r^{(k)}\right]^{\mathrm{T}},\enskip 
k\hm=1,\dots,t\,,
$$
$$
\underbrace{
\begin{bmatrix}
\hat{x}_r^{(1)} \\
\vdots  \\
\hat{x}_r^{(t)}
\end{bmatrix}
}_{\hat{\mathbf{x}}_r}
=
\underbrace{
\begin{bmatrix}
x_1^{(1)} & \cdots & x_{r-1}^{(1)} \\
\vdots    & \ddots & \vdots       \\
x_1^{(t)} & \cdots & x_{r-1}^{(t)}
\end{bmatrix}
}_{\mathbf{X}}
\underbrace{
\begin{bmatrix}
w_1 \\
\vdots  \\
w_{r-1}
\end{bmatrix}
}_{\mathbf{w}}.
$$
Тогда, выбрав в~качестве~$\rho$ евклидово расстояние, по 
определению~4 получим признаковое описание объекта~$x$:

\noindent
\begin{multline}
\label{eq:linear_regression}
\hspace*{-1mm}\mathbf{w}_g(x)=
\argmin\limits_{\mathbf{w}\in \mathbb{R}^n} \|\mathbf{x}_r-\hat{\mathbf{x}}_r\|^2_2={}\\
\hspace*{-1mm}{}=
\argmin\limits_{\mathbf{w}\in \mathbb{R}^n} \|\mathbf{x}_r-\mathbf{X}\mathbf{w}\|^2_2=
\left(\mathbf{X}^{\mathsf{T}}\mathbf{X}\right)^{-1}\mathbf{X}^{\mathsf{T}}\mathbf{x}_r\,.
\end{multline}

\item \textbf{Модель авторегрессии {\boldmath{$\mathbf{AR}(p)$}}}.

Задан временной ряд
$$
x = [x^{(1)},\dots,x^{(t)}],\ x^{(k)}\in\mathbb{R}\,,\enskip k=1,\dots,t\,.
$$
Выберем в~качестве модели аппроксимации авторегрессионную модель порядка~$p$:
\begin{equation*}
g(\mathbf{w},x)=\left[\hat{x}^{(1)},\dots,\hat{x}^{(t)}\right]\,,
\label{eq:autoregressive_model}
\end{equation*}
где
\begin{equation*}
\hat{x}^{(k)}=
\begin{cases}
x^{(k)}\,, & k=1,\dots,p\,;\\
w_0 + \sum\limits_{i=1}^{p} w_i x^{(k-i)}\,, & k=p+1,\dots,t\,.
\end{cases}
\end{equation*}
Далее признаковое описание определяется аналогично случаю линейной 
регрессии~\eqref{eq:linear_regression}.

\item \textbf{Дискретное преобразование Фурье}.
Задан временной ряд
$$
x = \left[x^{(0)},\dots,x^{(t-1)}\right],\ x^{(k)}\in\mathbb{C},\ k=0,\dots,t-1.
$$
Взяв в~качестве аппроксимирующей модели обратное преобразование Фурье
$$
g(\mathbf{w},x)=\left[\hat{x}^{(0)},\dots,\hat{x}^{(t-1)}\right]\,,
$$
где
\begin{multline}
\label{eq:fourier_approximation}
\hat{x}^{(k)}=\fr{1}{t}\sum\limits_{j=0}^{t-1} 
\left(w_{2j}+iw_{2j+1}\right) e^{({2\pi i}/t)kj}\,,\\ k=0,\dots,t-1\,,
\end{multline}
получим, что признаковым описанием вре\-мен\-н$\acute{\mbox{о}}$го ряда~$x$ является прямое преобразование:
\begin{equation}
\label{eq:fourier}
\mathbf{w}_g(x)=\left[w_0,\dots,w_{2t-1}\right]\,,
\end{equation}
где 
\begin{multline*}
w_{2k}+iw_{2k+1}=\sum\limits_{j=0}^{t-1} x^{(j)} 
e^{-({2\pi i}/{t})kj}\,,\\ k=0,\dots,t-1\,.
\end{multline*}
Переписывая~\eqref{eq:fourier_approximation} в~матричном виде, 
заметим, что, как и~в предыдущих случаях, параметры модели~$\mathbf{w}$ 
эквивалентно находятся при помощи линейной регрессии временн$\acute{\mbox{о}}$го 
ряда на столбцы матрицы Фурье.
Выбор лишь некоторых комплексных амплитуд соответствует регрессии временн$\acute{\mbox{о}}$го 
ряда на соответствующие столбцы матрицы Фурье.
Случай дискретного вейв\-лет-пре\-об\-ра\-зо\-ва\-ния аналогичен.
\end{enumerate}

Заметим, что в~первых двух случаях используются билинейные аппроксимирующие 
моде-\linebreak\vspace*{-12pt}

\columnbreak

\noindent
ли~$g(\mathbf{w},x)$, а~в~третьем~--- линейная.
Приведенные примеры демонстрируют большую общность построения пространства 
признаков при помощи моделей типа~\eqref{eq:regression} и~решения оптимизационной\linebreak 
задачи~\eqref{eq:feature_solution}.
Вообще говоря, при $|X|\hm\geqslant 2$ любая процедура построения признаковых 
описаний~$\mathbf{f}:\;X\hm\to \mathbb{R}^n$ задается эквивалентно решением 
оптимизационной задачи~\eqref{eq:feature_solution} при выборе соответствующей 
пары~$(g,\rho)$.

\section{Распределения признаков сегментов}
%\label{sec:distribution}

Объединим идеи, изложенные в~предыдущих разделах.
Согласно аппроксимирующей модели~\eqref{eq:regression}\linebreak получим для 
каж\-до\-го сегмента~$s^{(k)}\hm\in S(x)\hm=\left\{s^{(1)},\ldots,s^{(p)}\right\}$ 
временн$\acute{\mbox{о}}$го ряда~$x$ его признаковое 
описание~$\mathbf{w}^{(k)}:=\mathbf{w}_g(s^{(k)})$, решив оптимизационную 
задачу~\eqref{eq:feature_solution}.
Тогда всему набору \mbox{сегментов}~$S(x)$ будет соответствовать выборка:
\begin{equation}
\label{eq:segments_features}
\vec{F}=\left(\mathbf{w}^{(1)},\dots,\mathbf{w}^{(p)}\right)\,.
\end{equation}
Примем гипотезу простоты выборки~\eqref{eq:segments_features}.

\smallskip

\noindent
\textbf{Гипотеза~1.}\
\textit{Выборка~$\vec{F}\hm=\left(\vec{f}^{(1)},\dots,\vec{f}^{(p)}\right)$~--- 
прос\-тая, т.\,е.\ случайная, независимая и~однородная, где}
 $\mathbf{w}^{(k)}\sim\mathsf{P}_0$.


\smallskip

Пусть имеется параметрическое семейство 
распределений~$\left\{\mathsf{P}_{\vec\theta}\right\}_{\vec\theta\in 
\Theta}$.
Будем рассматривать вероятностную модель, в~которой объект~$x$ 
зависит от случайного параметра~$\vec\theta$.
\smallskip

\noindent
\textbf{Гипотеза~2.}\
$p(x|\vec\theta,y)=p(x|\vec\theta)$.

\smallskip
Тогда
\begin{multline*}
p(x,y)=
p(\vec{F},y)={}\\
{}=
\int\limits_{\Theta}p(\vec{F},\vec\theta,y)\,d\vec\theta=
\int\limits_{\Theta}p(\vec{F}|\vec\theta)p(\vec\theta,y)\,d\vec\theta\,.
\end{multline*}
При этом распределение~$p(\vec\theta,y)$ предлагается оценивать на этапе 
обучения, где признаковыми описаниями объектов~$x_i$ задачи классификации являются 
оценки параметров~$\vec\theta_i$:
$$
\hat{\vec\theta}_i=T(x_i)=T(\vec{F}_i)\,.
$$
Получив оценку~$\hat{p}(\vec\theta,y)$, находим оценку плот\-ности~$\hat{p}(x,y)$:
$$
\hat{p}(x,y)=
\int\limits_{\Theta}p(\vec{F}|\vec\theta)\hat{p}(\vec\theta,y)\,d\vec\theta\,,
$$
по которой строится байесовский классификатор.

\pagebreak

В алгоритмической постановке задачи классификации получим~$\hat{p}(y|\vec\theta)
\hm=\delta(a(\vec\theta),y)$ и
$$
\hat{p}(\vec\theta,y)=\delta(a(\vec\theta),y)p(\vec\theta)\,.
$$
Тогда
\begin{multline*}
\hat{p}(x,y)=
\int\limits_{\Theta}\!p\left(\vec{F}|\vec\theta\right)\hat{p}
(\vec\theta,y)d\vec\theta={}\\
{}=
\int\limits_{\Theta}\!p\left(\vec{F}|\vec\theta\right)\delta\left(a\left(\vec\theta\right),y\right)
p\left(\vec\theta\right)\,d\vec\theta={}\\
{}=\int\limits_{a^{-1}(y)}\!p\left(\vec{F}|\vec\theta\right)p\left(\vec\theta\right)\,d\vec\theta=
\int\limits_{a^{-1}(y)}\!p\left(\vec\theta|\vec{F}\right)p\left(\vec{F}\right)\,d\vec\theta\,.
\end{multline*}
Приближая распределение~$p(\vec\theta|\vec{F})$ вырожденным 
$\delta(\vec\theta-T(\vec{F}))$, получим
\begin{multline*}
\hat{p}(y|x)=
\!\int\limits_{a^{-1}(y)}\!\!\! \!p\left(\vec\theta|\vec{F}\right)\,d\vec\theta=
\!\int\limits_{a^{-1}(y)}\!\!\!\!\delta\left(\vec\theta-T\left(\vec{F}\right)\right)\,
d\vec\theta={}\\
{}=
\delta\left(a(T(\vec{F})),y\right).
\end{multline*}
Таким образом, задача классификации временн$\acute{\mbox{ы}}$х рядов свелась к~задаче 
классификации оценок параметров распределений 
семейства~$\left\{\mathsf{P}_{\vec\theta}\right\}_{\vec\theta\in \Theta}$.

В качестве оценок параметров~$\vec\theta$ предлагается брать оценки максимального 
правдоподобия:
\begin{multline*}
\hat{\vec\theta}=
T(x)=
\argmax_{\vec\theta\in\Theta}\mathcal{L}\left(\vec\theta\,|\,x\right)=
\argmax_{\vec\theta\in\Theta}p(\vec{F}|\vec\theta)={}\\
{}=
\argmax_{\vec\theta\in\Theta}\prod_{k}p(\mathbf{w}^{(k)}|\vec\theta).
\end{multline*}

Заметим, что в~частном случае тривиальной сегментации~\eqref{eq:equal_fragmenting} и~семейства вырожденных распределений оценка~$\hat{\vec\theta}$ является исходным признаковым описанием.
Таким образом, предложенный подход к~построению признакового описания временн$\acute{\mbox{о}}$го ряда
\begin{equation*}
%\label{eq:parameter_estimation}
\mathbf{f}:\;x\mapsto\hat{\vec\theta}
\end{equation*}
является достаточно общим и~при этом хорошо интерпретируется.


\section{Алгоритм классификации}
%\label{sec:classification}

Для завершения построения классификатора временн$\acute{\mbox{ы}}$х 
рядов~\eqref{eq:classifiers} построим многоклассовый 
классификатор~$b$~\eqref{eq:classification} по обучающей 
выборке~$\left\{(\mathbf{f}(x),y)\,|\,(x,y)\hm\in\mathfrak{D}\right\}$.

Сведем задачу многоклассовой классификации к~задачам бинарной классификации 
при помощи стратегий One-vs-All и~One-vs-One.

В~данной работе для решения задач бинарной классификации, где~$Y=\{-1,+1\}$, 
берутся различные модификации SVM (support vector machine).

\vspace*{-6pt}


\section{Вычислительный эксперимент}
%\label{sec:computational_experiment}

\vspace*{-2pt}

Вычислительный эксперимент проводился на данных для задачи классификации 
типов физической активности человека.

\vspace*{-6pt}

\subsection{Датасет WISDM}

\vspace*{-2pt}

Датасет (набор данных) WISDM~\cite{Kwapisz:2011:ARU:1964897.1964918} содержит 
показания акселерометра для~6~видов человеческой активности.
Необработанные данные, пред\-став\-ля\-ющие собой последовательность размеченных 
показаний акселерометра (по тройке чисел на каждый отсчет времени с~интервалом 
в~50~мс), были разбиты на временн$\acute{\mbox{ы}}$е ряды длиной 
по~200~отсчетов~(10~с).
Распределение полученных временн$\acute{\mbox{ы}}$х рядов по классам приведено 
в~табл.~1.

\vspace*{12pt}

\noindent
 %tabl1
%\vspace*{3pt}
{{\tablename~1}\ \ \small{Распределение временн$\acute{\mbox{ы}}$х рядов по классам. Набор данных
  WISDM}}

{\small
\begin{center}
   \tabcolsep=14pt
  \begin{tabular}{|l|c|}
    \hline
    \multicolumn{1}{|c|}{Классы} & Число объектов\\
    \hline
    1.\ Jogging (бежит) &  1624\hphantom{9}\\
    2.\ Walking (идет) &  2087\hphantom{9}\\
    3.\ Upstairs (поднимается)& 549\\
    4.\ Downstairs (спускается)& 438\\
    5.\ Sitting (сидит) & 276\\
    6.\ Standing (стоит)& 231\\
     \hline
  \end{tabular}
  \end{center}}
  
  \addtocounter{table}{1}
%\end{table*}

\subsubsection{Ручное выделение признаков}

\paragraph*{Выбор признаков.}
%\label{par:manual_feature_selection}
Каждая компонента вре\-мен\-н$\acute{\mbox{о}}$\-го ряда описывалась ее средним, 
стандартным отклонением, средним модулем отклонения от среднего, гистограммой 
с~10~областями равной \mbox{ширины}.
Полученные признаки для каждой компоненты объединялись, и~к~ним добавлялся признак 
средней величины ускорения.
Таким образом, каждый временной ряд описывался~40~признаками.


\vspace*{-12pt}

\paragraph*{Классификатор.}
Задача многоклассовой классификации сводилась к~задаче бинарной 
классификации при помощи подхода One-vs-One.
В качестве бинарного классификатора использовался SVM с~RBF (radial basis function)
яд\-ром 
и~па\-ра\-мет\-ра\-ми $C\hm=8{,}5$ и~$\gamma\hm=0{,}12$.

\vspace*{-12pt}

\paragraph*{Результаты.}
На диаграмме рис.~1 демонстрируется качество классификации при усреднении по 
$r\hm=50$  случайным разбиениям исходной выборки на тес\-то\-вую и~контрольную 
в~пропорции~7 к~3.





Как видно из~табл.~2, классы~2, 3 и~4 недостаточно хорошо отделяются друг от друга.

\pagebreak

\end{multicols}
\begin{figure*} %fig1-2
 \vspace*{1pt}
 \begin{minipage}[t]{80mm}
 \begin{center}  
 \mbox{%
\epsfxsize=78.057mm
\epsfbox{kar-1.eps}
}
\end{center}
\vspace*{-11pt}
\Caption{Набор данных WISDM.
Средняя точность~0,9726~--- вычисляется по формуле~\eqref{eq:total_quality}.
Средние точности классификации для каждого класса вычисляются по формуле~\eqref{eq:class_quality}}
%\end{figure}
%\begin{figure}
\end{minipage}
\hfill
 \vspace*{1pt}
  \begin{minipage}[t]{80mm}
 \begin{center}  
 \mbox{%
\epsfxsize=78.057mm
\epsfbox{kar-2.eps}
}
\end{center}
\vspace*{-11pt}
\Caption{Точность классификации для параметров модели авторегрессии в~качестве признаковых описаний}
\end{minipage}
\end{figure*}

\begin{table*}\small %tabl2
\begin{minipage}[t]{80mm}
\begin{center}
\Caption{Усредненная матрица неточностей. Ручное выделение признаков. 
Набор данных WISDM\newline
}
\vspace*{2ex}

\tabcolsep=6.4pt
\begin{tabular}{|c|c|c|c|c|c|c|}
\hline
Класс & \multicolumn{6}{c|}{Предсказанный класс} \\ 
  \cline{2-7}
объекта & $1$ & $2$ & $3$ & $4$ & $5$ & $6$\\ 
\hline
 1 & \textbf{1,00} & 0,00 & 0,00 & 0,00 & 0,00 & 0,00\\ 
 2 & 0,00 & \textbf{0,99} & 0,01 & 0,00 & 0,00 & 0,00\\ 
 3 & 0,03 & 0,04 & \textbf{0,89} & $0,04$ & 0,00 & 0,00\\ 
 4 & 0,02 & 0,05 & 0,05 & \textbf{0,88} & 0,00 & 0,00\\ 
 5 & 0,01 & 0,00 & 0,00 & 0,00 & \textbf{0,98} & 0,00\\ 
 6 & 0,00 & 0,00 & 0,00 & 0,00 & 0,00 & \textbf{1,00}\\ 
 \hline
\end{tabular}
\end{center}
%\end{table*}
\end{minipage}
\hfill
%\begin{table}\small %tabl3
\begin{minipage}[t]{80mm}
\begin{center}
\Caption{Усредненная матрица неточностей. Признаки, порожденные моделью 
авторегрессии. Набор данных ~WISDM}
\vspace*{2ex}


\tabcolsep=6.5pt
\begin{tabular}{|c|c|c|c|c|c|c|}
\hline
Класс & \multicolumn{6}{c|}{Предсказанный класс} \\ 
  \cline{2-7}
объекта & 1 & 2 & 3 & 4 & 5 & 6\\ 
  \hline
 1 & \textbf{1,00} & 0,00 & 0,00 & 0,00 & 0,00 & 0,00\\ 
 2 & 0,00 & \textbf{0,99} & 0,00 & 0,00 & 0,00 & 0,00\\ 
 3 & 0,01 & 0,02 & \textbf{0,95} & 0,02 & 0,00 & 0,00\\ 
 4 & 0,00 & 0,02 & 0,04 & \textbf{0,94} & 0,00 & 0,00\\ 
 5 & 0,01 & 0,00 & 0,00 & 0,00 & \textbf{0,97} & 0,01\\ 
 6 & 0,01 & 0,00 & 0,00 & 0,00 & 0,01 & \textbf{0,97}\\ 
\hline
\end{tabular}
\end{center}
\end{minipage}
\end{table*}

\begin{multicols}{2}


\subsubsection{Модель авторегрессии} %~(\ref{eq:autoregressive_model})}

\paragraph*{Признаковое описание.}
%\label{par:ar_feature_selection}
Во втором эксперименте в~качестве признаковых описаний временн$\acute{\mbox{ы}}$х 
рядов использовались все статистические функции, что брались в~первом эксперименте, 
за исключением гистограммы, вместо которой использовалось~7~коэффициентов 
модели авторегрессии AR($6$)~(см.~\eqref{eq:autoregressive_model}).
Таким образом, каждый временной ряд описывался~31~числом.
Также проводилась предварительная нормализация признаков.

\vspace*{-6pt}

\paragraph*{Классификатор.}
Задача многоклассовой классификации сводилась к~задаче 
бинарной классификации при помощи подхода One-vs-All.
В~качестве бинарного классификатора использовалась SVM с~RBF-яд\=ром 
и~параметрами $C\hm=8$ и~$\gamma\hm=0{,}8$.

\vspace*{-6pt}

\paragraph*{Результаты.}
На диаграмме рис.~2 и~в~табл.~3 показано качество классификации при усреднении по
$r\hm=50$ случайным разбиениям исходной выборки на тестовую и~контрольную 
в~отношении~7 к~3.




Несмотря на неравномерное распределение объектов по классам, 
использование признакового описания, порожденного моделью авторегрессии, 
позволяет значительно повысить качество классификации.
Точность построенного классификатора минимальна для~4-го класса~--- Downstairs~--- 
и~со\-став\-ля\-ет~94,3\%.

\subsection{Датасет USC-HAD}
%\label{seq:usc_had_dataset}

\begin{figure*}[b] %fig3
\vspace*{6pt}
\begin{center}
\mbox{%
\epsfxsize=128.626mm
\epsfbox{kar-3.eps}
}
\end{center}
\vspace*{-9pt}
\Caption{Точность классификации для параметров модели авторегрессии 
в~качестве признаковых описаний}
\label{fig:USCHAD_AR_FOURIER}
\end{figure*}


Датасет USC-HAD~\cite{mi12:ubicomp-sagaware} содержит показания акселерометра 
для~12~типов физической активности человека:
\begin{enumerate}[1)]
  \item walk forward (идет вперед);
  \item walk left (идет влево);
  \item walk right (идет вправо);
  \item go upstairs (подъем по лестнице);
  \item go downstairs (спуск по лестнице);
  \item run forward (бежит вперед);
  \item jump up and down (делает прыжок);
  \item sit and fidget (сидит);
  \item stand (стоит);
  
  \pagebreak
  
  
  \item sleep (спит);
  \item elevator up (поднимается в~лифте);
  \item elevator down (спускается в~лифте).
\end{enumerate}

Выборка содержит примерно по~70~шестикомпонентных временн$\acute{\mbox{ы}}$х 
рядов для каждого класса, а средняя длина временн$\acute{\mbox{о}}$го ряда~--- 3300. 
Частота записи измерений сенсора~100~Гц.

\vspace*{-6pt}


\subsubsection{Модель авторегрессии %~(\ref{eq:autoregressive_model}) 
и~Фурье} %~(\ref{eq:fourier})}

\vspace*{-2pt}

\paragraph*{Признаковое описание.}
%\label{par:ar_fourier_feature_selection_USCHAD}
Исходные временн$\acute{\mbox{ы}}$е ряды приводились к~частоте~10~Гц 
при помощи осреднения.

В качестве признаковых описаний преобразованных временн$\acute{\mbox{ы}}$х рядов брались 
статистические функции, описанные в~п.~7.1.1, 
за исключением гистограммы.
Также для каждой компоненты отдельно и~для модуля результирующего 
ускорения и~поворота добавлялось по~11~параметров авторегрессионной 
модели~$\text{AR}(10)$~(см.~\eqref{eq:autoregressive_model}).
Затем проводилась нормализация признаков и~добавлялись коэффициенты 
Фурье~\eqref{eq:fourier} с~индексами~3--12.
Таким образом, каждый~6-ком\-по\-нент\-ный временной ряд описывался~128~признаками.

\vspace*{-12pt}

\paragraph*{Классификатор.}
Задача многоклассовой классификации сводилась к~задаче бинарной 
классификации при помощи подхода One-vs-One.
В~качестве бинарного классификатора использовалась SVM с~RBF-яд\-ром и~параметрами 
$C\hm=10$ и~$\gamma\hm=0{,}13$.

%\vspace*{-12pt}

\paragraph*{Результаты.}
На диаграмме рис.~3 показано качество классификации 
при усреднении по $r\hm=500$ случайным разбиениям исходной выборки на тес\-то\-вую 
и~контрольную в~отношении~7 к~3.






Из~табл.~4 видно, что использование коэффициентов Фурье значительно повысило 
качество классификации.
Хуже всего класс~8 (sit and fidget) отделяется от класса~9 (stand).
Точность классификации для него составляет~92,2\%.

\vspace*{-6pt}


\subsubsection{Классификация голосованием и~классификация в~пространстве 
распределений параметров}

\vspace*{-2pt}

Рассмотрим алгоритм классификации в~сочетании с~процедурой сегментации временн$\acute{\mbox{ы}}$х 
рядов.
В качестве процедуры сегментации~$S(x)$ (см.~\eqref{eq:segmentation}) 
будем использовать выделение сегментов фиксированной длины.
Решим задачу классификации для первых~10~классов (за исключением <<elevator up>> 
и~<<elevator down>>, которые плохо отделяются друг от друга при малой длине сегментов) 
двумя алгоритмами.

В алгоритме голосования классификатор~$b:\;\mathbb{R}^n\hm\to Y$ обучается на 
новой обучающей выборке для сегментов исходных временн$\acute{\mbox{ы}}$х рядов

\noindent
$$
\mathfrak{D}_S=\left\{(\mathbf{w}_g(s),y):\;(x,y)\in\mathfrak{D},\,s\in S(x)\right\}.
$$
Далее производится голосование
$\hat{y}\hm=\argmax_{y}\sum\limits_{s\in S(x)}1\left[b(\mathbf{w}_g(s))=y\right].$

Алгоритм классификации в~пространстве гиперпараметров 
(распределений параметров аппрокси-\linebreak\vspace*{-12pt}

\pagebreak

\end{multicols}

\begin{table*}\small %tabl4
\begin{center}
\parbox{396pt}{\Caption{Усредненная матрица неточностей. Признаки, порожденные моделью авторегрессии. 
Набор данных USC-HAD}
\label{tbl:USCHAD_AR_FOURIER_confusion}
}

\vspace*{2ex}

\begin{tabular}{|c|c|c|c|c|c|c|c|c|c|c|c|c|}
  \hline
Класс & \multicolumn{12}{c|}{Предсказанный класс} \\ 
\cline{2-13}
объекта & 1 & 2 & 3 & 4 & 5 & 6 & 7 & 8 & 9 & 10 & 11 & 12\\ 
\hline
 1& \textbf{0{,}99} & 0,00 & 0,00 & 0,00 & 0,01 & 0,00 & 0,00 & 0,00 & 0,00 & 0,00 & 0,00 & 0,00\\ 
 2& 0,01 & \textbf{0{,}97} & 0,01 & 0,00 & 0,01 & 0,00 & 0,00 & 0,00 & 0,00 & 0,00 & 0,00 & 0,00\\ 
 3& 0,00 & 0,00 & \textbf{1{,}00} & 0,00 & 0,00 & 0,00 & 0,00 & 0,00 & 0,00 & 0,00 & 0,00 & 0,00\\ 
 4& 0,00 & 0,00 & 0,00 & \textbf{0{,}99} & 0,01 & 0,00 & 0,00 & 0,00 & 0,00 & 0,00 & 0,00 & 0,00\\ 
 5& 0,00 & 0,00 & 0,00 & 0,01 & \textbf{0{,}97} & 0,02 & 0,00 & 0,00 & 0,00 & 0,00 & 0,00 & 0,00\\ 
 6& 0,00 & 0,00 & 0,00 & 0,00 & 0,00 & \textbf{1{,}00} & 0,00 & 0,00 & 0,00 & 0,00 & 0,00 & 0,00\\ 
 7& 0,00 & 0,00 & 0,00 & 0,00 & 0,00 & 0,00 & \textbf{0{,}99} & 0,00 & 0,00 & 0,00 & 0,00 & 0,00\\ 
 8& 0,00 & 0,00 & 0,00 & 0,00 & 0,00 & 0,00 & 0,00 & \textbf{0{,}92} & 0,08 & 0,00 & 0,00 & 0,00\\ 
 9& 0,00 & 0,00 & 0,00 & 0,00 & 0,00 & 0,00 & 0,00 & 0,01 & \textbf{0{,}99} & 0,00 & 0,00 & 0,00\\ 
 10\hphantom{9} & 0,00 & 0,00 & 0,00 & 0,00 & 0,00 & 0,00 & 0,00 & 0,00 & 0,00 & \textbf{1{,}00} & 0,00 & 0,00\\ 
 11\hphantom{9} & 0,00 & 0,00 & 0,00 & 0,00 & 0,00& 0,00 & 0,00 & 0,00 & 0,00 & 0,00 & 
 \textbf{1{,}00} & 0,00\\ 
 12\hphantom{9} & 0,00 & 0,00 & 0,00 & 0,00 & 0,00 &0,00 & 0,00 & 0,00 & 0,02 & 0,00 & 0,01 & \textbf{0{,}97}\\ 
 \hline
\end{tabular}
\end{center}
\end{table*}
\begin{figure*} %fig4
\vspace*{1pt}
\begin{center}
\mbox{%
\epsfxsize=87.865mm
\epsfbox{kar-4.eps}
}
\end{center}
\vspace*{-9pt}
\Caption{Зависимость средней точности классификации от длины сегментов:
\textit{1}~--- голосование;
\textit{2}~--- гиперпараметры.
Набор данных USC-HAD, первые $10$~классов.
Точность классификации вычисляется по формуле~\eqref{eq:total_quality}}
\label{fig:USCHAD_AR_VOTING_VS_DISTR}
\end{figure*}

\begin{multicols}{2}

\noindent
мирующих моделей был описан 
в~разд.~5).
В~эксперименте использовалось семейство нормальных распреде\-лений с~диагональной 
ковариационной мат\-рицей.

Задача многоклассовой классификации решалась при помощи подхода One-vs-One 
бинарными классификаторами SVM с~RBF-яд\-ром и~параметрами $C\hm=100$
и~$\gamma\hm=0{,}017$.

На графике рис.~4 приведены результаты для средней 
точности решения задачи многоклассовой классификации обоими алгоритмами.

Из графика можно видеть, что оба алгоритма позволяют повысить качество 
классификации, причем алгоритм классификации в~пространстве гиперпараметров 
при длине сегмента~50 достигает качества~98,2\% и~показывает результат выше, 
чем алгоритм голосования.

Объединим результаты из последних двух экспериментов.
Будем обучать два классификатора.
Первый классификатор~$a_1$~--- One-vs-One SVM с~RBF-яд\-ром и~параметрами
$C\hm=10$ и~$\gamma\hm=0{,}13$~--- будет разделять классы~11, 12 и~первые 
десять классов для исходных временн$\acute{\mbox{ы}}$х рядов.
Второй классификатор~$a_2$~--- One-vs-One SVM с~RBF-яд\-ром 
и~параметрами $C\hm=100$ и~$\gamma\hm=0{,}017$~--- 
классификатор в~пространстве гиперпараметров, описанный в~предыду\-щем эксперименте.

Итоговый классификатор выглядит следующим образом:
\begin{equation}
\label{eq:final_classifier}
a(x)=\begin{cases}
a_1(x), &\ a_1(x)\in\{11, 12\}\,;\\
a_2(x)\ &\ \mbox{иначе}\,.
\end{cases}
\end{equation}


\paragraph*{Результаты.}
На диаграмме рис.~5 и~в~табл.~5 демонстрируется качество 
классификации построенного классификатора~\eqref{eq:final_classifier} 
при усреднении по $r\hm=500$ случайным разбиениям исходной выборки на тес\-то\-вую 
и~контрольную в~отношении~7 к~3.

\pagebreak

\end{multicols}

\begin{figure*} %fig5
\vspace*{1pt}
\begin{center}
\mbox{%
\epsfxsize=128.626mm
\epsfbox{kar-5.eps}
}
\end{center}
\vspace*{-9pt}
\Caption{Точность классификации для гиперпараметров в~качестве признаковых описаний.
Набор данных USC-HAD}
\label{fig:hyperparams}
\end{figure*}


\begin{table*}\small %tabl5
\begin{center}
\parbox{380pt}{\Caption{Усредненная матрица неточностей. Признаки~--- гиперпараметры. Набор данных
USC-HAD}

}
\label{tbl:hyperparams_confusion}
\vspace*{2ex}

\begin{tabular}{|c|c|c|c|c|c|c|c|c|c|c|c|c|}
  \hline
Класс & \multicolumn{12}{c|}{Предсказанный класс} \\ 
\cline{2-13}
объекта & 1 & 2 & 3 & 4 & 5 & 6 & 7 & 8& 9 & 10 & 11 & 12\\ 
\hline
 1 & \textbf{1{,}00} & 0,00 & 0,00 & 0,00 & 0,00 & 0,00 & 0,00 & 0,00 & 0,00 & 0,00 & 0,00 & 0,00\\ 
 2 & 0,01 & \textbf{0,98} & 0,01 & 0,00 & 0,00 & 0,00 & 0,00 & 0,00 & 0,00 & 0,00 & 0,00 & 0,00\\ 
 3 & 0,00 & 0,00 & \textbf{0{,}99} & 0,01 & 0,00 & 0,00 & 0,00 & 0,00 & 0,00 & 0,00 & 0,00 & 0,00\\ 
 4 & 0,00 & 0,00 & 0,00 & \textbf{0{,}99} & 0,01 & 0,00 & 0,00 & 0,00 & 0,00 & 0,00 & 0,00 & 0,00\\ 
 5 & 0,01 & 0,01 & 0,00 & 0,00 & \textbf{0{,}97} & 0,01 & 0,00 & 0,00 & 0,00 & 0,00 & 0,00 & 0,00\\ 
 6 & 0,00 & 0,00 & 0,00 & 0,00 & 0,00 & \textbf{1{,}00} & 0,00 & 0,00 & 0,00 & 0,00 & 0,00 & 0,00\\ 
 7 & 0,00 & 0,00 & 0,00 & 0,00 & 0,00 & 0,00 & \textbf{0{,}99} & 0,00 & 0,00 & 0,00 & 0,00 & 0,00\\ 
 8 & 0,00 & 0,00 & 0,00 & 0,00 & 0,00 & 0,00 & 0,00 & \textbf{0{,}93} & 0,06 & 0,00 & 0,00 & 0,00\\
 9 & 0,00 & 0,00 & 0,00 & 0,00 & 0,00 & 0,00 & 0,00 & 0,03 & \textbf{0{,}97} & 0,00 & 0,00 & 0,00\\
 10\hphantom{9} & 0,00 & 0,00 & 0,00 & 0,00 & 0,00 & 0,00 & 0,00 & 0,00 & 0,00 & \textbf{1{,}00} & 0,00 & 0,00\\ 
 11\hphantom{9} & 0,00 & 0,00 & 0,00 & 0,00 & 0,00 & 0,00 & 0,00 & 0,00 & 0,00 & 0,00 & \textbf{0{,}99} & 0,00\\
 12\hphantom{9} & 0,00 & 0,00 & 0,00 & 0,00 & 0,00 & 0,00 & 0,00 & 0,00 & 0,02 & 0,00 & 0,01 & 
 \textbf{0{,}97}\\
 \hline
\end{tabular}
\end{center}
\end{table*}

\begin{multicols}{2}

\section{Заключение}

В работе показано, что метод признакового описания временн$\acute{\mbox{о}}$го 
ряда оптимальными параметрами аппроксимирующих его моделей дает высокое 
качество решения задачи классификации.
Предложенный метод вычислительно эффективен и~не требователен к~памяти 
вычислительного устройства.

В~работе также предложен алгоритм классификации временн$\acute{\mbox{ы}}$х рядов в~пространстве 
распределений параметров моделей, порождающих их\linebreak сегменты. 
Он обобщает предыдущий метод классификации временн$\acute{\mbox{ы}}$х рядов и~позволяет 
производить более тонкую настройку алгоритма клас\-си\-фи\-кации.
{ %\looseness=1

}

{\small\frenchspacing
 {%\baselineskip=10.8pt
 \addcontentsline{toc}{section}{References}
 \begin{thebibliography}{99}
\bibitem{geurts2005segment}
\Au{Geurts~P., Wehenkel~L.}
 Segment and combine approach for non-parametric time-series classification~//
{Knowledge discovery in databases: PKDD 2005}.~--- Berlin--Heidelberg: Springer, 2005. 
P.~478--485.

\bibitem{Esling:2012:TDM:2379776.2379788}
\Au{Esling~P., Agon~C.}
Time-series data mining~//
\newblock {ACM Comput. Surv.}, 2012. Vol.~45. No.\,1. Article~12. P.~1--34.

\bibitem{basil2014automatic}
\Au{Basil~T., Lakshminarayan~C.}
Automatic classification of heartbeats~//
{22nd European Signal Processing Conference Proceedings}, 2014. P.~1542--1546.

\bibitem{alomari2013automated}
\Au{Alomari~M.\,H., Samaha~A., AlKamha~K.}
 Automated classification of l/r hand movement eeg signals using advanced 
 feature extraction and machine learning~//
{Int. J.~Adv. Comput. Sci. Appl.}, 2013. Vol.~4. No.\,6. P.~207--212.

\bibitem{Kwapisz:2011:ARU:1964897.1964918}
\Au{Kwapisz~J.\,R., Weiss~G.\,M., Moore~S.\,A.}
Activity recognition using cell phone accelerometers~//
{ACM SigKDD Explorations Newsletter}, 2011. Vol.~12. No.\,2. P.~74--82.

\bibitem{gruber2006signature}
\Au{Gruber~C., Coduro~M., Sick~B.}
 Signature verification with dynamic rbf networks and time series motifs~//
{10th  Workshop (International) on Frontiers in Handwriting Recognition}.
 La Baule, 2006. 
P.~455--460.

\bibitem{Ding:2008:QMT:1454159.1454226}
\Au{Ding~H., Trajcevski~G., Scheuermann~P., Wang~X., Keogh~E.}
Querying and mining of time series data: Experimental comparison of representations and distance measures~//
{Proc. VLDB Endow}, 2008. Vol.~1. No.\,2. P.~1542--1552.
 doi:10.14778/1454159.1454226.

\bibitem{jeong2011weighted}
\Au{Jeong~Y.\,S., Jeong~M.\,K., Omitaomu~O.\,A.}
Weighted dynamic time warping for time series classification~//
{Pattern Recogn.}, 2011. Vol.~44. No.\,9. P.~2231--2240.
doi:10.1016/j.patcog.2010.09.022.

\bibitem{Nanopoulos01feature-basedclassification}
\Au{Nanopoulos~A., Alcock~R., Manolopoulos~Y.}
Feature-based classification of time-series data~//
{Int. J.~Comput. Res.}, 2001. Vol.~10. P.~49--61.

\bibitem{wiens2012patient}
\Au{Wiens~J., Horvitz~E., Guttag~J.\,V.}
Patient risk stratification for hospital-associated c. diff as a time-series classification task~//
{Adv. Neur. Inform. Proc. Syst.}, 2012. Vol.~25. P.~467--475.

\bibitem{morchen2003time}
\Au{M$\ddot{\mbox{o}}$rchen~F.}
 Time series feature extraction for data mining using dwt and dft,
 2003. Unpubl.

\bibitem{kini2013large}
\Au{Kini~B.\,V., Sekhar~C.\,C.}
 Large margin mixture of ar models for time series classification~//
{Appl. Soft Comp.}, 2013. Vol.~13. No.\,1. P.~361--371.

\bibitem{kuznetsov2015description}
\Au{Кузнецов М.\,П., Ивкин Н.\,П.}
 Алгоритм классификации временных рядов акселерометра по комбинированному признаковому описанию~//
 {Машинное обучение и~анализ данных}, 2015. Т.~1. №\,11. С.~1471--1483.

\bibitem{mi12:ubicomp-sagaware}
\Au{Zhang~M., Sawchuk~A.\,A.}
 USC-HAD: A~daily activity dataset for ubiquitous activity recognition 
 using wearable sensors~//
 {ACM Conference (International) on Ubiquitous Computing Workshop on Situation, 
 Activity and Goal Awareness}.~--- Pittsburgh, PA, USA, 2012.
 \end{thebibliography}

 }
 }

\end{multicols}

%\vspace*{-6pt}

\hfill{\small\textit{Поступила в~редакцию 10.05.16}}

\vspace*{14pt}

%\newpage

%\vspace*{-24pt}

\hrule

\vspace*{2pt}

\hrule

\vspace*{8pt}


\def\tit{FEATURE-BASED TIME-SERIES CLASSIFICATION}

\def\titkol{Feature-based time-series classification}

\def\aut{M.\,E.~Karasikov$^{1,2}$ and V.\,V.~Strijov$^3$}

\def\autkol{M.\,E.~Karasikov and V.\,V.~Strijov}

\titel{\tit}{\aut}{\autkol}{\titkol}

\vspace*{-9pt}


    
\noindent
   $^1$Moscow Institute of Physics and Technology,
    9~Institutskiy Per., 
Dolgoprudny, Moscow Region 141700, Russian\linebreak
$\hphantom{^1}$Federation
    
\noindent
$^2$Skolkovo Institute of Science and Technology, Skolkovo Innovation Center,
Building~3, Moscow 143016, Russian\linebreak
$\hphantom{^1}$Federation

\noindent
$^3$A.\,A.~Dorodnicyn Computing Center, 
Federal Research Center ``Computer Science and Control'' 
of the Russian\linebreak
$\hphantom{^1}$Academy of Sciences, 44-2~Vavilov Str., Moscow 119333, 
Russian Federation



\def\leftfootline{\small{\textbf{\thepage}
\hfill INFORMATIKA I EE PRIMENENIYA~--- INFORMATICS AND
APPLICATIONS\ \ \ 2016\ \ \ volume~10\ \ \ issue\ 4}
}%
 \def\rightfootline{\small{INFORMATIKA I EE PRIMENENIYA~---
INFORMATICS AND APPLICATIONS\ \ \ 2016\ \ \ volume~10\ \ \ issue\ 4
\hfill \textbf{\thepage}}}

\vspace*{3pt}


     
\Abste{The paper is devoted to the multiclass time series classification problem.
The feature-based approach that uses meaningful and concise representations for feature space construction is applied.
    A~time series is considered as a sequence of segments 
    approximated by parametric models, and their parameters are used as time series 
    features.
        This feature construction method inherits from the
        approximation model such unique properties as shift invariance.
    The authors propose an approach to solve the time series classification problem 
    using distributions of parameters of the approximation model.
    The proposed approach is applied to the human activity classification problem.
    The computational experiments on real data demonstrate superiority of
    the proposed algorithm over baseline solutions.}

\KWE{time series; multiclass classification; time series segmentation; hyperparameters of approximation model; autoregressive model; discrete Fourier transform}

\DOI{10.14357/19922264160413} 


%\vspace*{-9pt}

\Ack
\noindent
The work was supported by the Russian Foundation for Basic Research 
(project 16-37-00485).

\pagebreak

%\vspace*{3pt}

  \begin{multicols}{2}

\renewcommand{\bibname}{\protect\rmfamily References}
%\renewcommand{\bibname}{\large\protect\rm References}

{\small\frenchspacing
 {%\baselineskip=10.8pt
 \addcontentsline{toc}{section}{References}
 \begin{thebibliography}{99}

\bibitem{geurts2005segment-1}
\Aue{Geurts,~P., and L.~Wehenkel}.
2005.
 Segment and combine approach for non-parametric time-series classification.
\textit{Knowledge Discovery in Databases: PKDD 2005}. 
Berlin--Heidelberg: Springer. 478--485.

\bibitem{Esling:2012:TDM:2379776.23797880-1}
\Aue{Esling,~P., and C.~Agon}.
2012.
Time-series data mining.
\textit{ACM Comput. Surv}. 45(1):12:1--12:34.

\bibitem{basil2014automatic-1}
\Aue{Basil,~T., and C.~Lakshminarayan}.
2014.
Automatic classification of heartbeats.
\textit{22nd European Signal Processing Conference Proceedings}. 1542--1546.

\bibitem{alomari2013automated-1}
\Aue{Alomari,~M.\,H., A.~Samaha, and K.~AlKamha}.
2013.
Automated classification of l/r hand movement eeg signals using advanced feature extraction and machine learning.
\textit{Int. J.~Adv. Comput. Sci. Appl.} 4(6):207--212.

\bibitem{Kwapisz:2011:ARU:1964897.1964918-1}
\Aue{Kwapisz,~J.\,R., G.\,M.~Weiss, and S.\,A.~Moore}.
2011.
Activity recognition using cell phone accelerometers.
\textit{ACM SigKDD Explorations Newsletter} 12(2):74--82.

\bibitem{gruber2006signature-1}
\Aue{Gruber,~C., M.~Coduro, and B.~Sick}.
2006.
 Signature verification with dynamic rbf networks and time series motifs.
\textit{10th  Workshop (International) on Frontiers in Handwriting Recognition}.
 La Baule. 455-460.

\bibitem{Ding:2008:QMT:1454159.1454226-1}
\Aue{Ding,~H., G.~Trajcevski, P.~Scheuermann, X.~Wang, and E.~Keogh}.
2008.
 Querying and mining of time series data: Experimental comparison of 
 representations and distance measures.
\textit{Proc. VLDB Endow} 1(2):1542--1552.
 doi: 10.14778/1454159.1454226.

\bibitem{jeong2011weighted-1}
\Aue{Jeong,~Y.\,S., M.\,K.~Jeong, and O.\,A.~Omitaomu}.
2011.
 Weighted dynamic time warping for time series classification.
\textit{Pattern Recogn.}. 44(9):2231--2240.
doi: 10.1016/j.patcog.2010.09.022.

\bibitem{Nanopoulos01feature-basedclassification-1}
\Aue{Nanopoulos,~A., R.~Alcock, and Y.~Manolopoulos}.
2001.
Feature-based classification of time-series data.
\textit{Int. J.~Comput. Res.} 10:49--61.

\bibitem{wiens2012patient-1}
\Aue{Wiens,~J., E.~Horvitz, and J.\,V.~Guttag.}
2012.
Patient risk stratification for hospital-associated c. diff as a time-series classification task.
\textit{Adv. Neur. Inform. Proc. Syst.} 25:467--475.

\bibitem{morchen2003time-1}
\Aue{M$\ddot{\mbox{o}}$rchen,~F.}
2003.
Time series feature extraction for data mining using dwt and dft.
Unpubl.

\bibitem{kini2013large-1}
\Aue{Kini,~B.\,V., and C.\,C.~Sekhar}.
2013.
 Large margin mixture of ar models for time series classification.
\textit{Appl. Soft Comp.} 13(1):361--371.

\bibitem{kuznetsov2015description-1}
\Au{Kuznetsov,~M.\,P., and N.\,P.~Ivkin.}
2015.
Algoritm klassifikatsii vremennykh ryadov akselerometra po kombinirovannomu 
priznakovomu opisaniyu [Time series classification algorithm using combined feature description].
\textit{Mashinnoe obuchenie i~analiz dannykh} [Machine Learning and Data Analysis]
 1(11):1471--1483.

\bibitem{mi12:ubicomp-sagaware-1}
\Aue{Zhang,~M., and A.\,A.~Sawchuk.}
2012.
USC-HAD: A~daily activity dataset for ubiquitous activity recognition
  using wearable sensors.
\textit{ACM Conference (International)
on Ubiquitous Computing Workshop on Situation, Activity and Goal Awareness}. 
Pittsburgh, PA.
\end{thebibliography}

 }
 }

\end{multicols}

\vspace*{-3pt}

\hfill{\small\textit{Received May 10, 2016}}


\Contr

\noindent
\textbf{Karasikov Mikhail E.} (b.\ 1992)~--- 
student, Moscow Institute of Physics and Technology, 9~Institutskiy Per., 
Dolgoprudny, Moscow Region 141700, Russian Federation;  
student, Skolkovo Institute of Science and Technology, Skolkovo Innovation Center,
Building~3, Moscow 143016, Russian Federation; \mbox{karasikov@phystech.edu}

\vspace*{3pt}

\noindent
\textbf{Strijov Vadim V.} (b.\ 1967)~--- 
Doctor of Science in physics and mathematics, leading scientist, 
A.\,A.~Dorodnicyn Computing Center, Federal Research Center ``Computer Science and Control'' 
of the Russian Academy of Sciences, 44-2~Vavilov Str., Moscow 119333, 
Russian Federation; \mbox{strijov@ccas.ru}
\label{end\stat}


\renewcommand{\bibname}{\protect\rm Литература}   %13
\def\stat{zalizniak}

\def\tit{БАЗА ДАННЫХ БЕЗЛИЧНЫХ ГЛАГОЛЬНЫХ КОНСТРУКЦИЙ РУССКОГО ЯЗЫКА$^*$}

\def\titkol{База данных безличных глагольных конструкций русского языка}

\def\aut{Анна А.~Зализняк$^1$, М.\,Г.~Кружков$^2$}

\def\autkol{Анна А.~Зализняк, М.\,Г.~Кружков}

\titel{\tit}{\aut}{\autkol}{\titkol}

\index{Зализняк Анна А.}
\index{Кружков М.\,Г.}
\index{Zalizniak Anna A.}
\index{Kruzhkov M.\,G.}


{\renewcommand{\thefootnote}{\fnsymbol{footnote}} \footnotetext[1]
{Работа выполнена при поддержке РГНФ (проект 15-04-00507).}}


\renewcommand{\thefootnote}{\arabic{footnote}}
\footnotetext[1]{Институт языкознания Российской академии наук; Институт проблем информатики Федерального 
исследовательского центра <<Информатика и~управ\-ле\-ние>> Российской академии наук, 
\mbox{anna.zalizniak@gmail.com}}
\footnotetext[2]{Институт проблем информатики Федерального исследовательского центра 
<<Информатика и~управ\-ле\-ние>> Российской академии наук, \mbox{magnit75@yandex.ru}}

\vspace*{-6pt}


  \Abst{Предлагается описание базы данных (БД) безличных глагольных 
конструкций (БГК), разработанной в~целях информационной поддержки 
лингвистического исследования БГК русского 
языка в~зеркале их перевода на иностранные языки. Показано, каким образом 
концепция построения надкорпусных БД (НБД) адаптирована под 
создание данного ин\-фор\-ма\-ци\-он\-но-линг\-ви\-сти\-че\-ско\-го ресурса. Переводные 
соответствия в~БД БГК представлены в~виде упорядоченных пар, элементами 
которых являются формальные описания соответствующих друг другу 
  лек\-си\-ко-грам\-ма\-ти\-че\-ских форм (ЛГФ) на языке оригинала и~языке 
перевода. Также описана методика построения переводных 
соответствий в~БД и~рассмотрены проблемы, связанные с~процедурой 
поиска безличных конструкций в~электронных лингвистических корпусах, 
а~также некоторые варианты их решения. Использование БД БГК и~других НБД 
значительно расширяет возможности лингвистов, использующих корпусные 
методы как для моноязычного, так и~для контрастивного анализа исследуемых 
языковых единиц, в~том числе благодаря поисковым и~статистическим 
функциям, интегрированным в~эти~БД.}
  
  \KW{компьютерная лингвистика; контрастивная лингвистика; 
информационные технологии; электронные корпуса текстов; надкорпусные 
базы данных; русский язык; безличные конструкции}

\DOI{10.14357/19922264160414}

\vspace*{-6pt} 


\vskip 8pt plus 9pt minus 6pt

\thispagestyle{headings}

\begin{multicols}{2}

\label{st\stat}
  
\section{Введение}

  База данных БГК русского языка и~их 
переводных эквивалентов была создана в~ходе реализации проекта 
<<Контрастивное корпусное исследование глагольных конструкций русского 
языка: семантика, грамматика, идиоматика>> с~целью решения следующих 
задач:
  \begin{itemize}
\item совершенствование метода унидирекционального контрастивного 
анализа, включающего построение моно- и~полиэквиваленций и~их 
последующий лингвистический и~статистический анализ; 
\item разработка метода идентификации безличных конструкций; 
\item построение типологии конструкций русского языка, выражающих 
значение безличности.
\end{itemize}

  Категория безличности является традиционным объектом русской 
грамматической теории; наиболее полно, последовательно и~на корпусном 
материале она описана в~статье~[1]; тем не менее здесь еще остается ряд 
нерешенных вопросов. В~рамках данного проекта рассматриваются только 
глагольные конструкции, со\-став\-ля\-ющие лишь незначительную\linebreak\vspace*{-12pt}

\columnbreak

\noindent
 часть всего 
многообразия русских безличных конструкций, пред\-став\-ля\-ющих собой, как 
известно, выдающуюся особенность русской грамматики~[2, с.~413--430]. 
В~данном исследовании принято расширенное понимание категории 
безличности: к~классу безличных конструкций относятся также конструкции, 
которые в~русской грамматике традиционно обозначаются как 
  <<не\-опре\-де\-лен\-но-лич\-ные>> (с~глаголом в~форме~3~л.\ мн.~ч.)\ 
и~<<обоб\-щен\-но-лич\-ные>> (с~глаголом в~форме~2~л.\ ед.~ч., реже~--- 
2~л.\linebreak мн.~ч.) (cм.\ об этих конструкциях, в~частности,~[3]). Такое 
понимание является концептуальной основой для проектирования как 
поисковой подсистемы, так и~в~целом для создания БД русских 
БГК. Ее формирование представляет собой 
важный этап в~развитии концепции НБД, а~также на пути 
создания грамматики конструкций русского языка, опирающейся на корпусные 
данные. 
  
  В данном исследовании применяется метод \textit{унидирек\-ционального 
корпусного анализа}, предложенный и~разработанный в~ходе выполнения ряда 
проектов, суть которого состоит в~том, что ана\-ли\-зи\-ру\-емые языковые единицы 
рассматриваются <<в~зеркале перевода>>: иноязычный текстовый эквивалент 
анализируемой единицы русского языка рас\-смат\-ри\-ва\-ет\-ся как источник 
сведений о~его семантике, в~том числе служит средством выявления 
имплицитных семантических компонентов. Таким образом, иноязычный текст 
в~обоих направлениях перевода служит не объектом, а лишь инструментом 
анализа~[4--6]. 
  
  База данных БГК русского языка  
основана на концепции построения \textit{надкорпусных баз данных}, 
описанной в~работах~[4--11]\footnote{Термин <<надкорпусная база данных>> 
введен в~работе~\cite{12-kr}.}. В~рамках этой концепции уже было создано три 
лингвистических БД: БД личных глагольных конструкций  
русского языка~[7--9], БД лингвоспецифичных единиц 
русского языка~\cite{6-kr, 10-kr} и~БД коннекторов русского 
языка~\cite{11-kr}. В~следующем разделе будет описана структура НБД и~то, 
как она применяется для поиска и~описания 
БГК русского языка и~их переводных соответствий в~БД~БГК.

\section{Структура надкорпусных баз данных и~ее приложение 
к~базе~данных безличных глагольных~конструкций}

  Надкорпусные базы данных используются для хранения информации 
о следующих объектах, выявляемых в~параллельных 
корпусах\footnote{В~настоящее время во всех НБД используются тексты 
параллельных подкорпусов Национального корпуса русского языка (НКРЯ, {\sf 
http://www.ruscorpora.ru/}).}:
  \begin{itemize}
\item языковые единицы некоторого языка, явля\-ющи\-еся предметом 
рассмотрения в~рамках конкретного лингвистического исследования;
\item соответствующие данным единицам фрагменты текста перевода на 
один или несколько языков; 
\item переводные соответствия, представляющие собой упорядоченные пары 
(=\;кор\-те\-жи), объединяющие два вышеупомянутых объекта (на первой 
позиции~--- единица языка оригинала, на второй позиции~--- единица языка 
перевода). 
\end{itemize}
  
  В БД БГК эти три объекта выглядят следующим образом:
  \begin{enumerate}[(1)]
\item БГК русского языка;
\item соответствующие им конструкции в~других языках (французском, 
немецком);
\item переводные соответствия, т.\,е.\ упорядоченные пары, объединяющие 
вышеупомянутые объ\-екты. 
\end{enumerate}

  Для этих объектов в~БД БГК используется следующая терминология. 
Исследуемые конструкции русского языка, а также соответствующие им 
конструкции в~других языках называются  
\textit{лек\-си\-ко-грам\-ма\-ти\-че\-ски\-ми формами}\footnote{Термин был введен в~\cite{12-kr}.}. 
Лек\-си\-ко-грам\-ма\-ти\-че\-ская форма перевода, 
соответствующая некоторой ЛГФ оригинала, называется ее 
\textit{функционально эквивалентным фрагментом} (сокращенно 
ФЭФ\footnote{Термин введен Д.\,О.~Добровольским~\cite{13-kr, 14-kr}.}). 
Переводное соответствие, представляющее собой упорядоченную пару 
$\langle$ЛГФ, ФЭФ$\rangle$, называется \textit{моноэквиваленцией} 
(сокращенно МЭ)\footnote{Понятие \textit{моноэквиваленции} было введено 
и~определено в~\cite{7-kr}.}. Такие переводные соответствия будем называть 
\textit{прямыми}. Прямые соответствия позволяют описывать так называемые 
\textit{модели} перевода. 

В~БД БГК также могут сохраняться \textit{обратные} 
переводные соответствия, сформированные на основе переводов иноязычных 
текстов на русский язык. В~этом случае глагольная конструкция русского 
языка, выявленная в~переводе, находится на второй позиции пары в~МЭ, а на 
первой~--- тот фрагмент иноязычного текста, который вызвал появление 
данной русской конструкции в~переводе, т.\,е.\ послужил \textit{стимулом 
перевода}\footnote{Подробнее о \textit{моделях} и~\textit{стимулах} перевода 
см.~\cite[с.~102]{8-kr}.}.
  
  Нужно отметить, что в~БД БГК и~других НБД между множеством 
конструкций русского языка и~множеством их ФЭФ в~языках перевода 
существует важное различие. Конструкции русского языка являются объектом 
нашего исследования, поэтому класс таких конструкций изначально является 
закрытым. 

Множество конструкций русского языка формируется на основе 
заранее заданных типов (например, в~БД БГК в~рассмотрение включаются лишь 
БГК). Что же касается их иноязычных 
соответствий, то они образуют открытый класс, поскольку заранее неизвестно, 
какие переводные эквиваленты ис\-сле\-ду\-емых конструкций русского языка 
будут выявлены в~процессе создания БД, и~в~этом и~состоит одна из 
задач настоящего исследования: работа с~параллельными текстами позволяет 
обнаружить соответствия, о~которых до его начала исследователи могли не 
подозревать~[15, с.~221; 16]. Соответственно, множество 
типов ЛГФ для языков перевода создается и~расширяется в~процессе 
построения МЭ по мере выявления новых видов переводных 
соответствий. При этом нередки случаи, когда в~переводе для анализируемой 
русской конструкции вообще не находится никакого соответ-\linebreak\vspace*{-12pt}

\pagebreak

\noindent
ствия или 
соответствие не может быть однозначно установлено\footnote[1]{В таких случаях 
ФЭФ перевода получает, соответственно, пометы ZERO и~UNKNOWN.}.
  
  Несмотря на вышеупомянутое различие, концепция НБД предполагает 
единую структуру описания ЛГФ на русском и~других языках, в~которой схемы 
описания конструкций на разных языках различаются только наборами 
используемых признаков. 
  
  Согласно концепции построения НБД исследователи не используют никакую 
заранее созданную классификацию рассматриваемых языковых единиц 
(в~случае БД БГК~--- безличных глагольных форм), а ограничиваются 
системой аннотирования ЛГФ, предусматривающей присвоение каждой ЛГФ 
набора признаков из некоторого списка. Такой способ формирования БД БГК 
позволяет максимально ускорить процесс обработки параллельных текстов, 
отложив решение более сложных классификационных вопросов до этапа 
анализа исследователями уже размеченного массива данных.
  
  Для разметки ЛГФ каждого языка (русский, французский, немецкий) 
используются свои наборы признаков, которые составляются исследователями 
перед началом работы и~могут пополняться и~видоизменяться в~ходе 
строительства МЭ. Это в~большей степени касается признаков ЛГФ 
иноязычных текстов, поскольку, как было сказано, в~ходе работы прямые МЭ 
выявляют новые модели, а~обратные~--- новые стимулы перевода. При этом 
НБД построена таким образом, что для добавления новых и~модификации 
старых признаков не требуется корректировать структуру БД~--- 
идентификаторы и~описания со\-от\-вет\-ст\-ву\-ющих признаков хранятся в~таблицах 
реляционной БД и~для внесения изменений достаточно добавить 
в~соответствующую таблицу новые строки или изменить содержание уже 
имеющихся строк.
  
  Для каждого языка признаки, использующиеся для аннотации ЛГФ, делятся 
на две группы: \textit{базовые виды} и~\textit{дополнительные признаки}. 
Предполагается, что все множество исследуемых конструкций русского языка 
принадлежит к~некоторому классу (в~случае с~БД БГК это безличные 
глагольные формы), который можно разбить на некоторое число 
непересекающихся подклассов, при этом принадлежность конструкции 
к~определенному подклассу фиксируется с~помощью признака, который 
называется \textit{базовым видом}. Таким образом, каждая ЛГФ всегда должна 
иметь один и~только один базовый вид.
{\looseness=-1

}
  
  Остальные признаки, использующиеся для аннотации ЛГФ, называются 
\textit{дополнительными признаками}. Каждой ЛГФ может быть присвоено 
ноль, один или более дополнительных признаков. Если в~НБД фиксируется 
большое число дополнительных признаков, то они для удобства группируются 
в~клас\-те\-ры. При этом по ходу наполнения БД БГК базовые виды 
и~дополнительные признаки могут добавляться, удаляться, объединяться, 
перемещаться из одного кластера в~другой, а также могут создаваться новые 
кластеры дополнительных признаков (при этом иногда может возникать 
необходимость коррекции разметки для уже построенных ЛГФ).
  
  Надо отметить, что выделение некоторой группы признаков в~качестве 
набора базовых видов для заданного класса конструкций является 
в~значительной степени условным. Этот выбор может определяться не 
концептуальными, а утилитарными соображениями, так как он ставит своей 
целью упрощение и~ускорение процесса построения МЭ. Например, в~БД БГК 
в~качестве набора базовых видов русского языка можно было бы выделить 
некоторый набор типов безличных конструкций в~соответствии  
с~ка\-кой-ли\-бо существующей классификацией (см., например,~[1]). Однако 
недостаток этого подхода состоит в~том, что определение типа безличной 
конструкции часто является нетривиальной задачей. Более того, построение 
типологии безличных конструкций~--- это, наоборот, одна из задач, которая 
решается при помощи БД БГК. Поэтому в~БД БГК в~качестве базовых видов 
берутся лексические единицы~--- чаще всего это глагол в~форме инфинитива, 
а~такие свойства, как время, вид и~лицо глагола, фигурируют в~качестве 
дополнительных признаков; иногда в~качестве базового вида используется 
глагол в~определенной форме (например, в~качестве самостоятельных ЛГФ 
рассматриваются единицы \textit{кажется}, \textit{может быть}); в~качестве 
самостоятельных ЛГФ рассматриваются также фразеологические единицы 
с~глагольной вершиной (например, \textit{ничего не поделаешь}, \textit{откуда 
ни возьмись} и~т.\,п.). В~других НБД базовые виды могут определяться 
грамматическими признаками\footnote[2]{Например, в~БД личных глагольных конструкций 
базовые виды формируются комбинацией признаков 
<<время>>, <<вид>> и~<<наклонение>>~[12--14].}.

\begin{figure*} %fig1
\vspace*{1pt}
\begin{center}
\mbox{%
\epsfxsize=133.224mm
\epsfbox{kru-1.eps}
}
\end{center}
\vspace*{-9pt}
\Caption{Построение <<скелета>> МЭ (пример)}
\vspace*{4pt}
\end{figure*}


\vspace*{-6pt}
  
\section{Методика построения моноэквиаленций в~базе~данных~безличных
глагольных~конструкций}

  Построение МЭ в~БД БГК осуществляют 
пользователи, свободно владеющие обоими языками соответствующей 
языковой пары. Пользователей этой категории будем называть 
\textit{строителями} МЭ. Система доступа к~БД БГК является 
многопользовательской, благодаря чему построением МЭ в~БД БГК могут 
одновременно заниматься несколько строителей. 
  
  Войдя в~систему, строитель МЭ в~соответствии с~поставленной перед ним 
задачей задает нужное направление перевода (например, переводы с~русского 
на французский), если нужно, выбирает для данного направления перевода 
конкретные произведения и~их переводы, для которых ему предстоит строить 
МЭ. Затем строитель задает \textit{первичный запрос} (см.\ разд.~4) на поиск 
безличных конструкций определенного типа в~выбранной части параллельного 
корпуса. В~результате выполнения первичного запроса система выдает 
строителю список выровненных пар фрагментов, в~которых могут находиться 
искомые конструкции (обычно запрос специфицирует признаки одного или 
нескольких слов, входящих в~текст русского фрагмента пары). Из-за 
имеющейся в~русском языке лексической и~грамматической омонимии, а также 
по ряду других причин (см.\ ниже) не все выданные по запросу пары 
в~действительности содержат конструкции искомого вида. Строитель 
последовательно просматривает каждую из найденных пар, и~если она 
действительно содержит искомую конструкцию, то для нее строится МЭ. 
  
  Сначала строитель создает <<скелет>> МЭ. Он выделяет в~русском 
фрагменте данной пары \textit{минимальный контекст} ЛГФ~---  совокупность 
элементов фразы, которая позволяет на базовом уровне интерпретировать 
употребление анализируемой русской ЛГФ в~данной фразе. Затем он находит 
и~выделяет в~переводном фрагменте той же пары минимальный контекст ФЭФ 
для данной русской ЛГФ. Из выделенных слов в~русском и~переводном 
фрагментах создаются заготовки для русской ЛГФ и~ее ФЭФ (переводной 
ЛГФ). Затем заготовка русской ЛГФ объединяется в~пару с~заготовкой ФЭФ 
перевода, в~результате чего создается неразмеченный <<скелет>> МЭ. Пример 
построения <<скелета>> МЭ схематично представлен на рис.~1. 


  Далее строитель размечает полученный <<скелет>> МЭ. В~минимальных 
контекстах русской ЛГФ и~переводного ФЭФ он выделяет так называемые 
<<главные слова>>, т.\,е.\ слова, входящие в~конструкцию: для русской ЛГФ 
это безличная глагольная форма с~наиболее тесно связанными с~ней словами 
(=\;эле\-мен\-та\-ми конструкции), а для ФЭФ перевода~--- функциональный 
эквивалент этой конструкции. Для примера из рис.~1 в~русской части это будут 
слова \textit{работается}, а также \textit{мне}, \textit{лучше}; во  
французской~--- \textit{pour moi}, \textit{je travaille}, \textit{mieux}.
  
  После этого русской ЛГФ и~переводному ФЭФ присваиваются значения 
параметров <<базовый вид>> и~<<дополнительные признаки>> (рис.~2). 
Если в~списке уже имеющихся базовых видов отсутствует нужная единица, она 
добавляется в~список; такая необходимость возникает, в~особенности для ФЭФ 
перевода, поскольку, как упоминалось выше, по мере обработки параллельного 
корпуса могут выявляться новые модели и~стимулы перевода. 
  
  Наконец, строитель может присвоить определенные признаки самой МЭ~--- 
из отдельного списка признаков, которые призваны фиксировать различные 
особенности переводного соответствия в~целом и~не могут быть отнесены по 
отдельности ни к~оригиналу, ни к~переводу. Одним из таких признаков является 
признак смены подлежащего при переводе (SubjCh). Пример размеченной МЭ 
схематично представлен на рис.~2.

\begin{figure*} %fig2
\vspace*{1pt}
\begin{center}
\mbox{%
\epsfxsize=128.783mm
\epsfbox{kru-2.eps}
}
\end{center}
\vspace*{-9pt}
\Caption{Размеченная МЭ (пример). Выделены главные слова, указаны базовый вид 
и~некоторые дополнительные признаки для ЛГФ и~ФЭФ, указан признак МЭ}
\vspace*{12pt}
\end{figure*}

  Построенные и~размеченные МЭ на следующем этапе проверяются 
экспертами. Эксперт может откорректировать состав слов, входящих в~контекст 
ЛГФ и~ФЭФ, и~их <<главные слова>>, изменить проставленные признаки, дать 
оценку качества построенной МЭ, а~также оставить текстовое замечание, что 
помогает поддерживать обратную связь между экспертами и~строителями.
  
\section{Построение первичных запросов}

  Чтобы облегчить поиск безличных конструкций, разработчики НБД создают 
\textit{первичные поисковые запросы}, предоставляющие в~распоряжение 
строителей множество пар предложений, в~которых потенциально могут 
присутствовать БГК. Далее строители вручную 
выбирают те пары, где (в~русской части) действительно имеется искомая 
конструкция, и~осуществляют построение и~аннотацию МЭ.
  
  Первичные запросы базируются на морфологической разметке слов 
в~параллельном подкорпусе НКРЯ ({\sf http://www.ruscorpora.ru/corpora-
morph.html}). В~силу омонимичности многих русских словоформ 
в~морфологической разметке НКРЯ допускается указание у~одной и~той же 
словоформы нескольких вариантов морфологического разбора, что приводит 
к~появлению шума в~результате выполнения первичных запросов.
  
  Как уже говорилось, в~данном исследовании принято расширенное 
понимание категории безличности~--- в~том отношении, что сюда включаются 
также неопределенно-личные и~обоб\-щен\-но-лич\-ные конструкции; 
свойством, объединяющим все типы безличных конструкций, считается 
отсутствие подлежащего, т.\,е.\ именной группы в~номинативе. Однако при 
анализе предложения при помощи формальных процедур достаточно трудно 
определить отсутствие в~нем подлежащего\footnote{Ср.\ понятие <<синтаксического 
нуля>> в~позиции подлежащего в~\cite{17-kr}. Однако, как известно, отсутствие именной группы 
в~номинативе~--- достаточно сложно идентифицируемый факт как с~практической, так и~с теоретической точки 
зрения (см.\ об этом, в~частности,~\cite{18-kr, 19-kr}). Особую проблему составляет трактовка случаев 
отсутствия личного местоимения. Иначе говоря, современное описание грамматики русского языка не 
содержит набора признаков и~критериев для формализованной идентификации отсутствия именной группы 
в~номинативе.}. С~другой стороны, как было сказано, в~рамках данного проекта 
рассмотрение ограничивается лишь \textit{глагольными} конструкциями; при 
этом из рассмотрения на данном этапе исключаются конструкции с~глаголами 
\textit{быть}, \textit{бывать}, \textit{стать}, \textit{становиться}, 
с~предикативом (\textit{сложно}, \textit{вероятно}, \textit{стыдно}, 
\textit{холодно}, \textit{надо}, \textit{пора} и~т.\,п.)\ и~с~инфинитивом 
(\textit{тебе ходить}, \textit{нечего сказать} и~т.\,п.).
  
  В~частности, в~БД БГК первичные запросы составлялись для поиска 
безличных форм, вклю\-ча\-ющих глагол в~форме:
  \begin{itemize}
\item наст./буд.\ вр.\ 2-го л.\ ед.~ч.\ ({\bfseries\textit{сидишь}} тут 
\textit{целый день как дурак}; \textit{тут за день так} 
{\bfseries\textit{накувыркаешься}});
\item наст./буд.\ вр.\ 3-го~л.\ ед.~ч. ({\bfseries\textit{приходится}} 
\textit{согласиться}; \textit{как вам не} {\bfseries\textit{надоест}}); 
\item наст.\ вр.\ 3-го~л. мн.~ч.\ (\textit{с~друзьями так не} 
{\bfseries\textit{поступают}});
\item прош.\ вр.\ ср.\ р.\ ед.~ч.\ ({\bfseries\textit{оказалось}}, \textit{что 
он не виноват});
\item прош.\ вр.\ ср.\ р.\ мн.~ч.\ (\textit{к~вам} 
{\bfseries\textit{пришли}}).
  \end{itemize}
  
  При этом в~искомых конструкциях не должно быть подлежащего 
в~именительном падеже, выраженного именной группой или местоимением, 
которое может согласоваться с~данным глаголом.
  
  Помимо уже упомянутой проблемы с~шумом (когда в~выдачу попадают 
слова, которые системой ошибочно принимаются за глаголы, например 
\textit{опускал ноги с}~{\bfseries\textit{постели}} \textit{на пол}), при поиске 
безличных конструкций в~БД БГК нередко могут возникать потери: безличные 
конструкции, присутствующие в~корпусе, не попадают в~выдачу. Это 
происходит, когда из результатов первичных запросов исключаются 
фрагменты, где фигурирует элемент, который может ошибочно 
интерпретироваться как подлежащее, согласующееся с~искомым глаголом.
  
  Вот несколько причин, которые могут обуслов\-ли\-вать такую ошибочную 
интерпретацию слов в~качестве подлежащего:
  \begin{itemize}
\item некоторые слова, омонимичные существительным в~именительном 
падеже, могут фигурировать в~тексте в~иной функции, например: \textit{раз}, 
\textit{уж}, \textit{том}, \textit{знать}, \textit{чай}, \textit{жила}, 
\textit{стать}, \textit{мол}, \textit{берет}, \textit{надел}, \textit{постой} 
и~т.\,д.:

\textit{Помилуй, чего тебе еще? от тебя и~так} {\bfseries\textit{уж несет}} 
\textit{розовой помадой}\ldots

\textit{В мыслях недостаточно последовательности, и, когда я излагаю их 
на бумаге, мне всякий} {\bfseries\textit{раз кажется}}, \textit{что я утерял 
чутье к~их органической связи}.

\textit{Главным образом, я потому не поехал за границу, что вестей туда 
из России доходит мало, а}~{\bfseries\textit{знать хочется}};
\item многие местоимения (\textit{то}, \textit{что}, 
\textit{что-то}, \textit{все}, \textit{всё} и~т. д.), могут 
выступать как в~роли подлежащего, так и~в других функциях (союз, 
дискурсивное слово\footnote[1]{В~некоторых случаях даже строителям МЭ 
бывает трудно без расширенного контекста определить, какую роль играет 
такое слово в~конкретном примере, например: \textit{Но теперь его вдруг} 
{\bfseries\textit{что-то}} \textit{потянуло к~людям}.}):

\textit{На это он заметил, что я еще слишком молода}, 
{\bfseries\textit{что}} \textit{у~меня еще в~голове} 
{\bfseries\textit{бродит.}} 

\textit{Его} {\bfseries\textit{всё тянет}} \textit{в~ту сторону, где только и~
знают, что гуляют}.

\textit{А} {\bfseries\textit{то выходит}} \textit{по твоему рассказу, что он 
действительно родился!..}

\textit{Ей} {\bfseries\textit{всё хочется}}, \textit{чтобы все считали, что 
она покровительствует};
\item нйденное в~непосредственном окружении глагола слово 
в~именительном падеже может на самом деле относиться к~другому глаголу:

\textit{Ну, будет другой} {\bfseries\textit{редактор}} \textit{и~даже}, 
{\bfseries\textit{может}} \textit{быть, еще красноречивее прежнего.}

\textit{Если как следует провентилировать этот} 
{\bfseries\textit{вопрос}}, {\bfseries\textit{выходит}}, \textit{что я, 
в~сущности, даже и~не знал-то как следует покойника}.

{\bfseries\textit{Фельдмаршал}} \textit{мой}, {\bfseries\textit{кажется}}, 
\textit{говорит дело}.

\textit{Здесь все} {\bfseries\textit{дело}}, {\bfseries\textit{кажется}}, 
\textit{совершенно очевидно}\ldots

\end{itemize}

  В совокупности эти и~некоторые другие причины (например, возможность 
различного порядка следования подлежащего и~сказуемого в~русском 
предложении) могут приводить к~заметным потерям, поэтому при построении 
первичных запросов разработчикам приходится искать способы их уменьшить. 
В~частности: 
  \begin{itemize}
\item составляются списки <<квазисуществительных>>, которые запрос не 
должен по умолчанию воспринимать как существительные (\textit{раз}, 
\textit{уж}, \textit{том}, \textit{знать} и~т.\,д.); 
\item составляются списки местоимений, которые, в~отличие от остальных 
местоимений, почти всегда должны интерпретироваться как потенциальные 
подлежащие (\textit{который}, \textit{которое}, \textit{он}, \textit{она}, 
\textit{это} и~т.\,д.);
\item область поиска потенциального подлежащего ограничивается 
ближайшими к~глаголу знаками препинания и~т.\,д.
\end{itemize}

  Поскольку, как было отмечено выше, не существует набора признаков 
и~критериев для формализованной идентификации отсутствия именной группы 
в~номинативе, при построении первичных запросов на поиск безличных форм 
разработчикам приходится искать приемлемый компромисс между уровнем 
шума и~потерь, так как, если речь не идет о сплошном аннотировании текстов, 
уменьшение потерь почти всегда ведет к~увеличению шума.

\section{Поиск и~статистика}

  После завершения аннотирования ЛГФ и~МЭ в~БД БГК становится 
достаточно легко находить в~массиве данных те МЭ, которые удовлетворяют 
заданным критериям. Поисковая система БД БГК позволяет указывать при 
поиске МЭ следующие признаки:
  \begin{itemize}
\item базовый вид ЛГФ оригинала;
\item базовый вид ФЭФ перевода;
\item дополнительные признаки ЛГФ;
\end{itemize}

\pagebreak

\end{multicols}

\begin{table*}\small
  \begin{center}
  \Caption{Фрагмент результатов выполнения поискового запроса. Показано 
4~из~22~найденных МЭ}
  \vspace*{2ex}
  
  \begin{tabular}{|p{35mm}|p{28mm}|p{35mm}|p{28mm}|}
  \hline
  \multicolumn{1}{|c|}{\tabcolsep=0pt\begin{tabular}{c}Контекст\\ ЛГФ 
оригинала\end{tabular}} & 
  \multicolumn{1}{c|}{\tabcolsep=0pt\begin{tabular}{c}Вид\\ и~дополнительные\\\ признаки 
ЛГФ\\ оригинала\end{tabular}} &
  \multicolumn{1}{c|}{\tabcolsep=0pt\begin{tabular}{c}Контекст ЛГФ\\ перевода 
(ФЭФ)\end{tabular}} & 
  \multicolumn{1}{c|}{\tabcolsep=0pt\begin{tabular}{c}Вид\\ и~дополнительные\\ признаки 
ЛГФ\\ перевода (ФЭФ)\end{tabular}}\\
  \hline
  Сестра теперь, %\newline 
  впрочем, \textbf{кажется}, %\newline 
  обеспечена\ldots & 
  \textbf{кажется}\newline
  $\langle$Impers$\rangle$\newline
  $\langle$V-IPF$\rangle$\newline
  $\langle$Pres$\rangle$\newline
  $\langle$3sg$\rangle$\newline
  $\langle$Refl$\rangle$\newline
  $\langle$Parenth$\rangle$   &
  \textbf{Ma soeur semble} d'ailleurs d$\acute{\mbox{e}}$somais 
{\!\!\ptb{\`{a}}}~l'abri du besoin\ldots  &
  \textbf{sembler}\newline 
  $\langle$Pr$\rangle$\newline
  $\langle$3sg$\rangle$\\
  \hline
  \textbf{кажется}, доктор теперь %\newline
  уже лишнее &
  \textbf{кажется}\newline
  $\langle$Impers$\rangle$\newline
  $\langle$V-IPF$\rangle$\newline
  $\langle$Pres$\rangle$\newline
  $\langle$3sg$\rangle$\newline
  $\langle$Refl$\rangle$\newline
  $\langle$Parenth$\rangle$   &
  \textbf{le m$\acute{\mbox{e}}$dicin semblait} 
\mbox{d$\acute{\mbox{e}}$j{\!\ptb{\`{a}}}} \textbf{inutile} &
  \textbf{sembler}\newline
  $\langle$Imparf$\rangle$\newline
  $\langle$3sg$\rangle$\newline
  $\langle$[V.]\;+\;\{Adj.\}$\rangle$\\
  \hline
  \textbf{Кажется}, и~печалями и~радостями он управ\-лял, как движением рук &
  \textbf{кажется}\newline
  $\langle$Impers$\rangle$\newline
  $\langle$V-IPF$\rangle$\newline
  $\langle$Pres$\rangle$\newline
  $\langle$3sg$\rangle$\newline
  $\langle$Refl$\rangle$\newline
  $\langle$Parenth$\rangle$   &
  \textbf{Il semble commander} {\ptb{\`{a}}}~ses joies et {\!\!\ptb{\`{a}}}~ses 
tristesses comme il dirige les mouvements de ses bras & 
  \textbf{sembler}\newline
  $\langle$Pr$\rangle$\newline
  $\langle$3sg$\rangle$\newline
  $\langle$[V.]\;+\;\{Inf.\}$\rangle$\\
  \hline
  все, \textbf{казалось}, лежит в~торжественном покое.
  &
  \textbf{кажется}\newline
  $\langle$Impers$\rangle$\newline
  $\langle$V-IPF$\rangle$\newline
  $\langle$Past$\rangle$\newline
  $\langle$Sg$\rangle$\newline
  $\langle$Refl$\rangle$\newline
  $\langle$Parenth$\rangle$   &
  \textbf{Elle semblait} silencieuse, solennelle &
  \textbf{sembler}\newline
  $\langle$Imparf$\rangle$\newline
  $\langle$3sg$\rangle$\newline
  $\langle$[V.]\;+\;\{Adj.\}$\rangle$\\
  \hline
  \end{tabular}
  \end{center}
%  \vspace*{-2pt}
\vspace*{8pt}
  \end{table*}
  
  \begin{multicols}{2}

\begin{itemize}
\item дополнительные признаки ФЭФ;
\item признаки МЭ;
\item названия текстов, авторы текстов и~переводов и~т.\,д.
\end{itemize}

  Все указанные признаки могут задаваться (или исключаться из результатов 
поиска) одновременно, при этом возвращаемый набор МЭ будет удовлетворять 
всем указанным требованиям. Например, в~БД БГК пользователь может задать 
для поиска следующий набор признаков: (1)~базовый вид ЛГФ оригинала~--- 
<<\textit{кажется}>>; (2)~дополнительный признак ЛГФ оригинала~--- 
$\langle$Parenth$\rangle$ (вводное слово); (3)~базовый вид ФЭФ перевода~--- 
<<\textit{sembler}>>; (4)~признак МЭ~--- $\langle${SubjCh}$\rangle$ (смена 
подлежащего). Такой запрос возвращает~22~результата (из общего массива 
в~2100~МЭ), 4~из которых показаны в~табл.~1.
  
  
  
    
  Данные БД БГК также можно использовать для получения статистики по 
моделям и~стимулам перевода для различных анализируемых глагольных 
конструкций на основе всего массива данных, содержащихся в~БД БГК (или же 
на основе определенного подмножества этого массива). Разумеется, делать 
выводы на основе такой статистики следует лишь после лингвистической 
экспертизы представительного массива данных, поскольку результаты могут 
зависеть от состава анализируемого корпуса, выбранной исследователями 
схемы аннотации и~т.\,д. Поэтому для пользователей БД предусмотрена 
возможность верифицировать полученную статистику: перейдя от 
количественных результатов непосредственно к~МЭ, на основе которых они 
были сгенерированы, пользователь может оценить на качественном уровне 
зависимость значения указанных параметров от контекста, а также от жанра 
текста и~стиля конкретного автора.
  
  В качестве примера приводится таблица частотностей переводных 
соответствий (базовых видов французских ФЭФ) для русской ЛГФ 
<<\textit{хотеться}>>
 (табл.~2). В~БД БГК численные результаты (столбец\linebreak\vspace*{-12pt}

\noindent
{{\tablename~2}\ \ \small{Количественная статистика по вариантам перевода безличных конструкций с~ЛГФ 
\textit{хотеться}}}

\vspace*{3pt}

{\small
 \begin{center}  %
 \tabcolsep=8.5pt
 \begin{tabular}{|l|c|r|}
\cline{1-1}
\textbf{хотеться} \ $\underline{73}$&\multicolumn{2}{c}{\ }\\
\hline
\multicolumn{1}{|c|}{ФЭФ французского языка} & 
\multicolumn{1}{c|}{\tabcolsep=0pt\begin{tabular}{c}Количество\\ МЭ\end{tabular}} &\multicolumn{1}{c|}{\%}\\
\hline
avoir envie & $\underline{32}$ & ${43{,}84}$\\
vouloir & $\underline{24}$ & ${32{,}88}$\\
selon ses desirs & $\underline{2}$ & ${2{,}74}$\\
UNKNOWN & $\underline{2}$ & ${2{,}74}$\\
pr$\acute{\mbox{e}}$f$\acute{\mbox{e}}$rer & $\underline{2}$ & ${2{,}74}$\\
br$\hat{\mbox{u}}$ler d'envie & $\underline{2}$ & ${2{,}74}$\\
avoir faim & $\underline{2}$ & ${2{,}74}$\\
aimer & $\underline{1}$ & ${1{,}37}$\\
tenir & $\underline{1}$ & ${1{,}37}$\\
chercher & $\underline{1}$ & ${1{,}37}$\\
avoir soif & $\underline{1}$ & ${1{,}37}$\\
avoir sommeil & $\underline{1}$ & ${1{,}37}$\\
d$\acute{\mbox{e}}$sirer & $\underline{1}$ & ${1{,}37}$\\
envie & $\underline{1}$& ${1{,}37}$\\
\hline
\end{tabular}
\end{center}
}

\vspace*{12pt}




\noindent
<<Количество МЭ>>) оформлены в~виде гиперссылок,
 по которым пользователи 
могут перейти на страницу, где находятся все МЭ, на основе которых были 
получены приведенные данные.
  



\section{Заключение}

  В процессе создания БД БГК была разработана поисковая подсистема, 
которая обеспечивает выполнение широкого спектра запросов. Проведенные 
эксперименты продемонстрировали ее высокую эффективность с~точки зрения 
решаемых лингвистических задач. Важно отметить, особую ценность для 
развития лингвистической теории полученного <<шума>> и~обнаруженных 
<<потерь>>. Их экспертный анализ позволит сформулировать более четкие 
критерии для их поиска в~корпусах и~БД и~уточнить понятие 
безличной конструкции, что будет способствовать развитию грамматики 
конструкций русского языка. 
  
  База данных БГК стала уже четвертой лингвистической БД, созданной 
в~рамках концепции\linebreak построения НБД, что свидетельствует 
о~жизнеспособности этой концепции в~сфере разработки инструментов 
лингвистического сопоставительного анализа и~формирования 
информационных ресурсов, не имеющих отечественных и~зарубежных 
аналогов. Разработка БД БГК подтвердила, что концепция НБД и~метод 
унидирекционального контрастивного анализа применимы к~исследованиям 
разноплановых языковых единиц и~явлений (личные глагольные формы, 
лингвоспецифичные единицы, коннекторы, безличные глагольные формы,  
ло\-ги\-ко-се\-ман\-ти\-че\-ские отношения в~тексте).
  
{\small\frenchspacing
 {%\baselineskip=10.8pt
 \addcontentsline{toc}{section}{References}
 \begin{thebibliography}{99}
\bibitem{1-kr}
\Au{Летучий А.\,Б.} Безличность. Материалы для проекта корпусного описания русской 
грамматики.~--- М., 2011. {\sf http://rusgram.ru/Безличность}.
\bibitem{2-kr}
\Au{Wierzbicka A.} Semantics, culture, and cognition. 
Universal human concepts in culture-specific configurations.~--- New York\,--\,Oxford: Oxford 
University Press, 1992. 496~p.
\bibitem{3-kr}
\Au{Булыгина Т.\,В., Шмелев А.\,Д.} Я,~ты и~другие в~русском синтаксисе~// Языковая 
концептуализация мира (на материале русской грамматики).~--- М.: Школа <<Языки 
русской культуры>>, 1997. С.~335--352.
\bibitem{4-kr}
\Au{Бунтман Н.\,В., Зализняк Анна~А., Зацман~И.\,М., Кружков~М.\,Г., 
Лощилова~Е.\,Ю., Сичинава~Д.\,В.} Инфор\-мационные технологии корпусных 
исследований: принципы построения кросслингвистических баз данных~// Информатика 
и~её применения, 2014. Т.~8. Вып.~2. С.~98--110.
\bibitem{5-kr}
\Au{Зализняк Анна А., Зацман~И.\,М., Инькова~О.\,Ю., Кружков~М.\,Г.} Надкорпусные 
базы данных как лингвистический ресурс~// Корпусная лингвистика-2015: Тр. 7-й 
Междунар. конф.~--- СПб.: СПбГУ, 2015. С.~211--218.
\bibitem{6-kr}
\Au{Зализняк Анна А.} База данных межъязыковых эквиваленций как инструмент 
лингвистического анализа~// Компьютерная лингвистика и~интеллектуальные 
технологии: по мат-лам ежегодной Междунар. конф. <<Диалог>>.~--- М.: РГГУ, 2016. 
Вып.~15(22). С.~763--775.
\bibitem{7-kr}
\Au{Loiseau S., Sitchinava D.\,V., Zalizniak~A.\,A., Zatsman~I.\,M.} Information technologies 
for creating the database of equivalent verbal forms in the Russian--French multivariant parallel 
corpus~// Информатика и~её применения, 2013. T.~7. Вып.~2. С.~100--109.
\bibitem{8-kr}
\Au{Zalizniak Anna A., Sitchinava~D.\,V., Loiseau~S., Kruzhkov~M., Zatsman~I.\,M.} Database 
of equivalent verbal forms in a~Russian--French multivariant parallel corpus~// 2013 Conference 
(International) on Artificial Intelligence.~--- Las Vegas, NV, USA: CSREA Press, 2013. 
Vol.~1. P.~101--107.
\bibitem{9-kr}
\Au{Kruzhkov M.\,G., Buntman~N.\,V., Loshchilova~E.\,Ju., Sitchinava~D.\,V., Zalizniak 
Anna~A., Zatsman~I.\,M.} A~database of Russian verbal forms and their French translation 
equivalents~// Компьютерная лингвистика и~интеллектуальные технологии: по мат-лам 
ежегодной Междунар. конф. <<Диалог>>.~--- М.: РГГУ, 2014. Вып.~13(20). C.~284--296.
\bibitem{10-kr}
\Au{Зализняк Анна А.} Лингвоспецифичные единицы русского языка в~свете 
контрастивного корпусного\linebreak анализа~// Компьютерная лингвистика и~интеллектуальные 
технологии: по мат-лам ежегодной Междунар. конф. <<Диалог>>.~--- М.: РГГУ, 2015. 
Вып.~14(21). С.~651--662.
\bibitem{11-kr}
\Au{Зацман И.\,М., Инькова~О.\,Ю., Кружков~М.\,Г., Попкова~Н.\,А.} Представление 
кроссязыковых знаний о~коннекторах в~надкорпусных базах данных~// Информатика и~
её применения, 2016. Т.~10. Вып.~1. С.~106--118.
\bibitem{12-kr}
\Au{Кружков М.\,Г.} Информационные ресурсы контрастивных лингвистических 
исследований: электронные корпуса текстов~// Системы и~средства информатики, 2015. 
Т.~25. №\,2. С.~140--159.
\bibitem{13-kr}
\Au{Добровольский Д.\,О., Кретов~А.\,А., Шаров~С.\,А.} Корпус параллельных текстов: 
архитектура и~возможности использования~// Национальный корпус русского языка: 
2003--2005.~--- М.: Индрик, 2005. С.~263--296.
\bibitem{14-kr}
\Au{Добровольский Д.\,О., Кретов~А.\,А., Шаров~С.\,А.} Корпус параллельных текстов~// 
Научная и~техническая информация. Сер.~2: Информационные процессы и~сис\-те\-мы, 
2005. №\,6. С.~16--27.
\bibitem{15-kr}
\Au{Stubbs M.} Words and phrases. Corpus studies of lexical semantics.~--- Oxford: Blackwell, 
2002. 287~p.
\bibitem{16-kr}
\Au{Zatsman I., Buntman~N., Kruzhkov~M., Nuriev~V., Zalizniak Anna~A.} Conceptual 
framework for development of computer technology supporting cross-linguistic knowledge 
discovery~// 15th European Conference on Knowledge Management Proceedings.~--- Reading, MA, USA: 
Academic Publishing International Ltd., 2014. P.~1063--1071.
\bibitem{17-kr}
\Au{Мельчук И.\,А.} О~синтаксическом нуле~// Типология пассивных конструкций. 
Диатезы и~залоги.~--- Л.: Наука, 1974. C.~343--361.
\bibitem{18-kr}
\Au{Guiraud-Weber M.} L'effacement du sujet au nominatif dans 
l'$\acute{\mbox{e}}$nonc$\acute{\mbox{e}}$ en russe moderne~// Revue des 
$\acute{\mbox{E}}$tudes slaves, 1983. Vol.~55. No.\,1. P.~79--86.
\bibitem{19-kr}
\Au{Гиро-Вебер М.} Субъектные черты и~проблема подлежащего в~русском языке~// 
Revue des $\acute{\mbox{E}}$tudes slaves, 2002. Vol.~74. No.\,2-3. P.~279--289.
 \end{thebibliography}

 }
 }

\end{multicols}

\vspace*{-6pt}

\hfill{\small\textit{Поступила в~редакцию 13.10.16}}

\vspace*{8pt}

%\newpage

%\vspace*{-24pt}

\hrule

\vspace*{2pt}

\hrule

%\vspace*{8pt}


\def\tit{DATABASE OF~RUSSIAN IMPERSONAL VERBAL CONSTRUCTIONS}

\def\titkol{Database of~Russian impersonal verbal constructions}

\def\aut{Anna A.~Zalizniak$^{1,2}$  and~M.\,G.~Kruzhkov$^2$}

\def\autkol{Anna A.~Zalizniak  and~M.\,G.~Kruzhkov}

\titel{\tit}{\aut}{\autkol}{\titkol}

\vspace*{-9pt}




\noindent
$^1$Institute of Linguistics, Russian Academy of Sciences, 
1-1~Bolshoy Kislovskiy pereulok, 
Moscow 125009, Russian\linebreak
$\hphantom{^1}$Federation

\noindent
$^2$Institute of Informatics Problems, Federal Research Center ``Computer Science and Control'' of 
the Russian\linebreak
$\hphantom{^1}$Academy of Sciences, 44-2~Vavilov Str., Moscow 119333, Russian Federation



\def\leftfootline{\small{\textbf{\thepage}
\hfill INFORMATIKA I EE PRIMENENIYA~--- INFORMATICS AND
APPLICATIONS\ \ \ 2016\ \ \ volume~10\ \ \ issue\ 4}
}%
 \def\rightfootline{\small{INFORMATIKA I EE PRIMENENIYA~---
INFORMATICS AND APPLICATIONS\ \ \ 2016\ \ \ volume~10\ \ \ issue\ 4
\hfill \textbf{\thepage}}}

\vspace*{3pt}





\Abste{This article presents the Database of Russian Impersonal Verbal Constructions that has been 
developed to support a~linguistic research of Russian impersonal verbal construction as mirrored in 
translations into other languages. This information resource was developed based on the concept of 
Supracorpora Databases (SCDBs). Translation correspondences in the Database of Russian 
Impersonal Verbal Constructions are presented as ordered pairs that combine formal descriptions of 
corresponding lexical-grammatical forms found in source and target texts of a~parallel 
corpus. The paper also provides description of the methodology for creation of translation 
correspondences in the database. Some of the problems related to the task of finding Russian 
impersonal verbal constructions in corpora are considered and approaches to solving those 
problems are proposed. Thanks to integrated search and statistical functions, the Database of 
Russian Impersonal Verbal Constructions and other SCDBs significantly extend capabilities of 
linguistic experts using corpus-based methods to analyze specific linguistic items, both 
independently and in contrast with other languages.} 

\KWE{computer linguistics; contrastive linguistics; information technologies; electronic corpora; 
supracorpora databases; Russian language; impersonal constructions}


\DOI{10.14357/19922264160414} 

\vspace*{-12pt}

\Ack
\noindent
This research was performed in the Institute of Informatics Problems, Federal Research Center 
``Computer Science and Control'' of the Russian Academy of Sciences, with financial support of the 
Russian Foundation for Humanities (grant No.\,15-04-00507).


\vspace*{6pt}

  \begin{multicols}{2}

\renewcommand{\bibname}{\protect\rmfamily References}
%\renewcommand{\bibname}{\large\protect\rm References}

{\small\frenchspacing
 {%\baselineskip=10.8pt
 \addcontentsline{toc}{section}{References}
 \begin{thebibliography}{99}

\bibitem{1-kr-1}
\Aue{Letuchiy, A.\,B.} 2011. \textit{Bezlichnost'. Materialy dlya proekta korpusnogo 
opisaniya russkoy grammatiki} [Impersonality. Materials for the Russian corpus-based grammar 
description project]. Moscow. Available at: {\sf http://\linebreak rusgram.ru/Безличность} (accessed 
October~12, 2016).
\bibitem{2-kr-1}
\Aue{Wierzbicka, A.} 1992. \textit{Semantics, culture, and cognition. Universal human 
concepts in culture-specific configurations}. New York\,--\,Oxford: Oxford University Press. 
496~p.
\bibitem{3-kr-1}
\Aue{Bulygina, T.\,V., and A.\,D.~Shmelev}. 1997. Ya, ty i~drugie v~russkom sintaksise 
[Me, 
you, and others in Russian syntax]. \textit{Yazykovaya kontseptualizatsiya mira (na materiale 
russkoy grammatiki)} [Language-based conceptualization of the world (based on Russian 
grammar)]. Moscow: Shkola ``Yazyki russkoy kul'tury.'' 335--352.
\bibitem{4-kr-1}
\Aue{Buntman, N.\,V., Anna A.~Zaliznyak,  I.\,M.~Zatsman, M.\,G.~Kruzhkov, 
E.\,Yu.~Loshchilova,  and D.\,V.~Sichinava}. 2014. 
Informatsionnye tekhnologii korpusnykh issledovaniy: Printsipy postroeniya  
kross-lingvisticheskikh baz dannykh [Information technologies for corpus studies: 
Underpinnings for cross-linguistic database creation]. \textit{Informatika i~ee Primeneniya~--- 
Inform. Appl.} 8(2):98--110.
\bibitem{5-kr-1}
\Aue{Zaliznyak, Anna A., I.\,M.~Zatsman, O.\,Yu.~In'kova, and M.\,G.~Kruzhkov}. 2015. 
Nadkorpusnye bazy dannykh kak lingvisticheskiy resurs [Supracorpora databases as linguistic 
resource]. \textit{7th Conference 
(International) on Corpus Linguistics Proceedings}. St.\ Petersburg: SPbGU. 211--218.
\bibitem{6-kr-1}
\Aue{Zaliznyak, Anna A.} 2016. Baza dannykh mezh"\-yazy\-ko\-vykh ekvivalentsiy kak 
instrument ling\-vi\-sti\-che\-sko\-go analiza [Database of cross-linguistic equivalences as a tool for 
linguistic analysis]. \textit{Computer Linguistics and Intellectual 
Technologies: Conference (International) ``Dialog'' Proceedings}. Moscow: RGGU.  
15(22):763--775.
\bibitem{7-kr-1}
\Aue{Loiseau, S., D.\,V.~Sitchinava, Anna A.~Zalizniak,  and I.\,M.~Zatsman}. 2013. 
Information technologies for creating the database of equivalent verbal forms in the  
Russian--French multivariant parallel corpus. \textit{Informatika i~ee Primeneniya~--- Inform. 
Appl.} 7(2):100--109.
\bibitem{8-kr-1}
\Aue{Zalizniak, Anna A., D.\,V.~Sitchinava, S.~Loiseau, M.~Kruzhkov, and I.\,M.~Zatsman}. 
2013. Database of equivalent verbal forms in a~Russian--French multivariant parallel corpus. 
\textit{2013  Conference (International) on Artificial Intelligence}. Las Vegas,
NV: CRSEA 
Press. 1:101--107.
\bibitem{9-kr-1}
\Aue{Kruzhkov, M.\,G., N.\,V.~Buntman, E.\,Ju.~Loshchilova, D.\,V.~Sitchinava, Anna 
A.~Zalizniak,  and I.\,M.~Zatsman}. 2014. A~database of Russian verbal forms and their 
French translation equivalents. \textit{Computer Linguistics and 
Intellectual Technologies: Conference (International) ``Dialog'' Proceedings}. Moscow: RGGU. 
13(20):284--296.
\bibitem{10-kr-1}
\Aue{Zalizniak, Anna A.} 2015. Lingvospetsifichnye edi\-ni\-tsy russkogo yazyka v~svete 
kontrastivnogo korpusnogo analiza
[Lingvospecific units of Russian in the light of the contrast corpus analysis].
\textit{Computer Linguistics and 
Intellectual Technologies: Conference (International) ``Dialog'' Proceedings}. 
Moscow: RGGU. 
14(21):651--662.
\bibitem{11-kr-1}
\Aue{Zatsman, I.\,M., O.\,Yu.~In'kova, M.\,G.~Kruzhkov, and N.\,A.~Popkova}. 2016. 
Predstavlenie krossyazykovykh znaniy o konnektorakh v nadkorpusnykh bazakh dannykh 
[Representation of cross-lingual knowledge about connectors in supracorpora databases]. 
\textit{Informatika i~ee Primeneniya~--- Inform. Appl.} 10(1):106--118.
\bibitem{12-kr-1}
\Aue{Kruzhkov, M.\,G.} 2015. Informatsionnye resursy kontrastivnykh lingvisticheskikh 
issledovaniy: Elektronnye korpusa tekstov [Information resources for contrastive studies: 
Electronic text corpora]. \textit{Sistemy i~Sredstva Informatiki~--- Systems and Means of 
Informatics} 25(2):140--159.
\bibitem{13-kr-1}
\Aue{Dobrovol'skiy, D.\,O., A.\,A.~Kretov, and S.\,A.~Sharov}. 2005. Korpus parallel'nykh 
tekstov: Arkhitektura i~voz\-mozh\-no\-sti ispol'zovaniya [Corpus of parallel texts: Architecture and 
applications]. \textit{Natsional'nyy korpus russkogo yazyka: 2003--2005} [Russian National 
Corpus: 2003--2005]. Moscow: Indrik. 263--296.
\bibitem{14-kr-1}
\Aue{Dobrovol'skiy, D.\,O., A.\,A.~Kretov, and S.\,A.~Sharov}. 2005. Korpus parallel'nykh 
tekstov [Corpus of parallel texts]. \textit{Nauchnaya i~tekhnicheskaya informatsiya. Ser.~2 
``Informatsionnye protsessy i~sistemy''} [Scientific and technical information. Ser.~2 
``Informational processes and systems''] 6:16--27.
\bibitem{15-kr-1}
\Au{Stubbs, M.} 2002. \textit{Words and phrases: Corpus studies of lexical semantics.}~--- 
Oxford: Blackwell. 287~p.
\bibitem{16-kr-1}
\Aue{Zatsman, I., N.~Buntman, M.~Kruzhkov, V.~Nuriev, and Anna A.~Zalizniak}. 2014. 
Conceptual framework for development of computer technology supporting cross-linguistic 
knowledge discovery. \textit{15th European Conference on Knowledge Management 
Proceedings}. Reading, MA: Academic Publishing International. 1063--1071.
\bibitem{17-kr-1}
\Aue{Mel'chuk, I.\,A.} 1974. O~sintaksicheskom nule [On syntactic zero]. \textit{Tipologiya 
passivnykh konstruktsiy. Diatezy i~zalogi} [Typology of passive constructions. Diatheses and 
voices]. Leningrad: Nauka. 343--361.
\bibitem{18-kr-1}
\Aue{Guiraud-Weber, M.} 1983. L'effacement du sujet au nominatif dans 
l'$\acute{\mbox{e}}$nonc$\acute{\mbox{e}}$ en russe moderne [Diffusion of the subject in 
nominative case in modern Russian utterances]. \textit{Revue des $\acute{\mbox{E}}$tudes 
slaves} [J.~Slavic Studies] 55(1):79--86.
\bibitem{19-kr-1}
\Aue{Guiraud-Weber, M.} 2002. Sub"ektnye cherty i~problema podlezhashchego v~russkom 
yazyke [Subjective features and problem of the subject in Russian language]. \textit{Revue des 
$\acute{\mbox{E}}$tudes slaves} [J.~Slavic Studies] 74(2-3):279--289.
\end{thebibliography}

 }
 }

\end{multicols}

\vspace*{-3pt}

\hfill{\small\textit{Received October 13, 2016}}

\Contr

\noindent
\textbf{Zalizniak Anna A.} (b.\ 1959)~---  Doctor of Science of philology, leading scientist, 
Institute of Linguistics, Russian Academy of Sciences, 1-1~Bolshoy Kislovskiy pereulok, Moscow, 
125009, Russian Federation; leading scientist, Institute of Informatics Problems, Federal Research 
Center ``Computer Science and Control'' of the Russian Academy of Sciences, 44-2~Vavilov Str., 
Moscow 119333, Russian Federation; \mbox{anna.zalizniak@gmail.com}

\vspace*{3pt}

\noindent
\textbf{Kruzhkov Mikhail G.} (b.\ 1975)~--- leading programmer, Institute of Informatics 
Problems, Federal Research Center ``Computer Science and Control'' of the Russian Academy of 
Sciences, 44-2~Vavilov Str., Moscow 119333, Russian Federation; \mbox{magnit75@yandex.ru}
\label{end\stat}


\renewcommand{\bibname}{\protect\rm Литература}  %14



%%%%%%%%%%%%%%%%%%%%%%%%%%%%%%%%%%%%%%%%%%%%%%%

%\def\stat{rez}
{%\hrule\par
%\vskip 7pt % 7pt
\raggedleft\Large \bf%\baselineskip=3.2ex
Р\,Е\,Ц\,Е\,Н\,З\,И\,И \vskip 17pt
    \hrule
    \par
\vskip 6pt plus 6pt minus 3pt }

%\thispagestyle{headings} %с верхним колонтитулом
%\thispagestyle{myheadings} %с нижним колонтитулом, но в верхнем РЕЦЕНЗИИ

\def\tit{НОВАЯ КНИГА И.\,Н.~СИНИЦЫНА, А.\,С.~ШАЛАМОВА <<ЛЕКЦИИ ПО ТЕОРИИ 
ИНТЕГРИРОВАННОЙ ЛОГИСТИЧЕСКОЙ ПОДДЕРЖКИ>> (М.: ТОРУС ПРЕСС, 2012. 624~с.)}

%1
\def\aut{Д.ф.-м.н., профессор С.\,Я.~Шоргин}

\def\auf{\ }

\def\leftkol{\ % РЕЦЕНЗИИ
}

\def\rightkol{ \ } 

%\def\leftkol{\ } % ENGLISH ABSTRACTS}

%\def\rightkol{\ } %ENGLISH ABSTRACTS}

%\def\leftkol{РЕЦЕНЗИИ}

%\def\rightkol{РЕЦЕНЗИИ}

\titele{\tit}{\aut}{\auf}{\leftkol}{\rightkol}
\vspace*{-18pt}


     \label{st\stat}

     \begin{multicols}{2}
     {\small
     {\baselineskip=10.1pt
     

      В книге представлено системное изложение теоретических основ одного из новейших 
направлений в \mbox{об\-ласти} экономики послепродажного обслуживания изделий наукоемкой 
продукции (ИНП) длительного пользования~--- интегрированной логистической поддержки
(ИЛП). 
{\looseness=1

}

Приведены также результаты новых работ, выполненных в Институте проблем информатики 
Российской академии наук в рамках научного направления <<Информационные технологии и 
анализ сложных сис\-тем>>.
 {%\looseness=1

}
     
      Излагаемые в книге научные подходы позво\-ляют карди\-наль\-но реформировать 
существующие системы производства и эксплуатации ИНП путем создания и внед\-ре\-ния 
методов рационального и оптимального управ\-ле\-ния процессами расходования 
вре\-мен\-н$\acute{\mbox{ы}}$х, 
мате\-ри\-аль\-ных, трудовых и других ресурсов на всех стадиях жизненного цикла изделий (ЖЦИ) по 
критериям экономической целесообразности и эф\-фек\-тив\-ности.
  {\looseness=1

}
    
      В книге приведен краткий обзор причин возник\-новения и
      развития CALS-методологии как основы 
современных международных стандартов по созданию и функционированию глобальных 
ин\-фор\-ма\-ци\-он\-но-ком\-му\-ни\-ка\-ци\-он\-ных систем, ее ключевых возможностей и эффективности 
результатов ее использования. 
Авторы %\linebreak 
предлагают ряд научных обоснований для разработки 
единой теории проектирования и управления систем ИЛП для полноценного использования 
преимуществ %\linebreak
 суще\-ст\-ву\-ющей методологии, определяют \mbox{общую} структурную схему 
комплексной системы <<ИНП-СППО>> и необходимость разработки для ее описания 
гибридных стохастических моделей.
{%\looseness=1

}

%\columnbreak
      
      Книга состоит из пяти частей, где последовательно излагается материал по каждой из 
следующих тем: <<Интегрированная логистическая поддержка>>, <<Теория гибридных 
стохастических систем и компьютерная поддержка исследований и разработок>>, <<Основы 
математического моделирования, анализа и синтеза систем послепродажного обслуживания>>, 
<<Определение и анализ показателей экспортного потенциала ИНП при проектировании>>, 
<<Задачи управления поддержкой послепродажного обслуживания>>, а также 
<<Моделирование инвестиционных процессов ИЛП в условиях неравновесных финансовых 
рынков>>. 
   
      В конце каждой главы приведены выводы и даны вопросы и задания для 
самоконтроля. В~приложениях содержатся основные определения по программам работ по 
анализу ИЛП, логистическим базам данных и компьютерным решениям, эквивалентной статистической 
линеаризации нелинейных преобразований ИЛП, справочный материал, а также развернутые 
уравнения для вероятностных характеристик.


      \def\leftkol{РЕЦЕНЗИИ}

\def\rightkol{РЕЦЕНЗИИ} 

      
      Книга заинтересует широкий круг специалистов и может быть использована научными 
проектными организациями в сфере промышленного производства ИНП. Большое количество 
иллюстраций, примеров и вопросов, обращенных к читателю, позволяет использовать книгу 
также в качестве учебного пособия для студентов и аспирантов машиностроительных, 
транспортных и~других специальностей, а также для самостоятельного изучения. 
{%\looseness=-1

}

Книга 
представляет несомненный интерес для специалистов и студентов в области прикладной 
математики и информатики.
    

}

}
\end{multicols}

%\newpage

\def\stat{authorsrus}
{%\hrule\par
%\vskip 7pt % 7pt
\raggedleft\Large \bf%\baselineskip=3.2ex
О\,Б\ \ А\,В\,Т\,О\,Р\,А\,Х \vskip 17pt
    \hrule
    \par
\vskip 21pt plus 8pt minus 4pt }


\def\tit{\ }

\def\aut{\ }

\def\auf{\ }

\def\leftkol{\ } % ENGLISH ABSTRACTS}

\def\rightkol{ОБ АВТОРАХ} %ENGLISH ABSTRACTS}

\titele{\tit}{\aut}{\auf}{\leftkol}{\rightkol}
      
            \label{st\stat}



\vspace*{24pt}

\begin{multicols}{2}




\noindent
\textbf{Архипов Олег Петрович} (р.\ 1948)~---
кандидат технических наук, директор Орловского филиала Института проб\-лем информатики
Российской академии наук
%302025, г.Орел, Московское шоссе, д.137

\vspace*{3pt}

\noindent
\textbf{Бирюкова Татьяна Константиновна} (р.\ 1968)~---
кандидат фи\-зи\-ко-ма\-те\-ма\-ти\-че\-ских наук, старший научный сотрудник Института проб\-лем информатики
Российской академии наук

\vspace*{3pt}

\noindent 
\textbf{Бобков  Сергей Геннадьевич} (р.\ 1955)~---
доктор технических наук,  заведующий отделением На\-уч\-но-ис\-сле\-до\-ва\-тель\-ско\-го 
института системных исследований Российской академии наук
%117218, Москва, Нахимовский просп., 36, к.1 

\vspace*{3pt}

\noindent \textbf{Васильев Николай Семенович} (р.\ 1952)~--- доктор 
фи\-зи\-ко-ма\-те\-ма\-ти\-че\-ских наук, профессор, 
МГТУ им.\ Н.\,Э.~Баумана 
%, Москва 105005, 2-я Бауманская ул., д.~5,

\vspace*{3pt}

\noindent
\textbf{Гершкович Максим Михайлович} (р.\ 1968)~---
старший научный сотрудник Института проб\-лем информатики
Российской академии наук

\vspace*{3pt}

\noindent 
\textbf{Дьяченко Юрий Георгиевич} (р.\ 1958)~--- кандидат технических наук, 
старший научный сотрудник Института проб\-лем информатики
Российской академии наук

\vspace*{3pt}

\noindent 
\textbf{Ерошенко Александр Андреевич} (р.\ 1989)~--- аспирант кафедры 
математической статистики факультета вычисли\-тельной математики и кибернетики 
Московского государственного университета им.\ М.\,В.~Ломоносова
%119991, Москва ГСП-1, Ленинские горы, д.\ 1, стр. 52

\vspace*{3pt}
 
\noindent 
\textbf{Захаров Виктор Николаевич} (р.\ 1948)~--- 
доктор технических наук, доцент, ученый секретарь Института проб\-лем информатики
Российской академии наук

\vspace*{3pt}

\noindent
\textbf{Зейфман Александр Израилевич} (р.\ 1954)~---
доктор фи\-зи\-ко-ма\-те\-ма\-ти\-че\-ских наук, профессор, 
заведующий кафедрой Вологодского государственного университета; 
старший научный сотрудник Института проб\-лем информатики
Российской академии наук; главный научный сотрудник ИСЭРТ Российской академии наук

\vspace*{3pt}

\noindent
\textbf{Зыкин Сергей Владимирович} (р.\ 1959)~--- 
доктор технических наук, профессор, заведующий лабораторией Института математики 
им.\ С.\,Л.~Соболева Сибирского отделения Российской академии наук, Новосибирск 
%630090, пр.\ ак.\ Коптюга, 4 

\vspace*{4pt}

\noindent
\textbf{Киреев Владимир Иванович} (р.\ 1938)~---
доктор фи\-зи\-ко-ма\-те\-ма\-ти\-че\-ских наук, профессор Московского 
государственного горного университета
%Адрес: Россия, 119991, г. Москва, Ленинский проспект, д. 6

%\columnbreak

\vspace*{4pt}

\noindent
\textbf{Козеренко Елена Борисовна} (р.\ 1959)~---
кандидат филологических наук, заведующая лабораторией Института проб\-лем информатики
Российской академии наук

\vspace*{4pt}

\noindent
\textbf{Королев Виктор Юрьевич} (р.\ 1954)~--- доктор
фи\-зи\-ко-ма\-те\-ма\-ти\-че\-ских наук, профессор кафедры математической 
статистики факультета вычисли\-тельной математики и кибернетики 
Московского государственного университета; 
ведущий научный сотрудник Института проб\-лем информатики
Российской академии наук

\vspace*{4pt}

\noindent
\textbf{Коротышева Анна Владимировна} (р.\ 1988)~---
старший преподаватель Вологодского государственного университета

\vspace*{4pt}

\noindent 
\textbf{Кун Де Турк} (р.\ 1981)~--- научный сотрудник 
исследовательской группы SMACS факультета телекоммуникаций и обработки информации
Университета Гента, Бельгия
%В-9000 Гент, Бельгия

\vspace*{4pt}

\noindent
\textbf{Лупенцов Олег Сергеевич} (р.\ 1986)~---
аспирант Омского государственного института сервиса
%Омск 644043, ул.\ Певцова 13

\vspace*{4pt}

\noindent
\textbf{Лучко Олег Николаевич} (р.\ 1961)~---
кандидат педагогических наук, профессор, заведующий кафедрой 
Омского государственного института сервиса
%Омск 644043, ул.\ Певцова 13

\vspace*{4pt}

\noindent
\textbf{Малашенко Юрий Евгеньевич} (р.\ 1946)~---
доктор фи\-зи\-ко-ма\-те\-ма\-ти\-че\-ских наук, заведующий сектором 
Вычислительного центра им.\ А.\,А.~Дородницына Российской академии наук
%Адрес: 119333, Москва, ул. Вавилова, 40,

\vspace*{4pt}

\noindent
\textbf{Маньяков Юрий Анатольевич} (р.\ 1984)~---
кандидат технических наук, научный сотрудник Орловского филиала Института проб\-лем информатики
Российской академии наук
%302025, г.Орел, Московское шоссе, д.137

\vspace*{4pt}

\noindent
\textbf{Маренко Валентина Афанасьевна} (р.\ 1951)~---
кандидат технических наук, доцент, старший научный сотрудник 
Института математики им.\ С.\,Л.~Соболева Сибирского отделения Российской академии наук
%Новосибирск 630090, пр. ак. Коптюга, 4 

\vspace*{3pt}

\noindent 
\textbf{Морозов Евсей Викторович} (р.\ 1947)~--- доктор 
фи\-зи\-ко-ма\-те\-ма\-ти\-че\-ских, профессор, ведущий научный сотрудник 
Института прикладных математических исследований Карельского научного центра Российской
академии наук; 
%%185910 Россия, Республика Карелия, г.\ Петрозаводск, ул.\ Пушкинская, 11
профессор Петрозаводского государственного университета, Петрозаводск
%185910 Россия, Республика Карелия, г.\ Петрозаводск, пр.\ Ленина, 33

%\pagebreak

\vspace*{3pt}

\noindent
\textbf{Назарова Ирина Александровна} (р.\ 1966)~---
кандидат фи\-зи\-ко-ма\-те\-ма\-ти\-че\-ских наук, 
научный сотрудник Вычислительного центра им.\ А.\,А.~Дородницына Российской академии наук 
%Адрес: 119333, Москва, ул. Вавилова, 40

\vspace*{3pt}

\noindent
\textbf{Павлов Игорь Валерианович} (р.\ 1945)~--- 
доктор фи\-зи\-ко-ма\-те\-ма\-ти\-че\-ских наук, профессор МГТУ им.\ Н.\,Э.~Баумана 
%Москва 105005, 2-я Бауманская ул., д.~5 

%\pagebreak

\vspace*{3pt}

\noindent 
\textbf{Потахина Любовь Викторовна} (р.\ 1989)~--- аспирантка
Института прикладных математических исследований Карельского научного центра
Российской академии наук; 
%%185910 Россия, Республика Карелия, г.\ Петрозаводск, ул.\ Пушкинская, 11
инженер Петрозаводского государственного университета, Петрозаводск
%185910 Россия, Республика Карелия, г.\ Петрозаводск, пр.\ Ленина, 33

\vspace*{3pt}

\noindent 
\textbf{Рождественский Юрий Владимирович} (р.\ 1952)~--- 
кандидат технических наук, заведующий сектором Института проб\-лем информатики
Российской академии наук

\vspace*{3pt}

\noindent 
\textbf{Синицын Игорь Николаевич} (р.\ 1940)~--- доктор технических наук,
профессор, заслуженный деятель\linebreak\vspace*{-12pt}

\columnbreak

\noindent
 науки РФ, заведующий отделом Института проб\-лем информатики
Российской академии наук

\vspace*{7pt}


\noindent
\textbf{Сиротинин Денис Олегович} (р.\ 1984)~---
кандидат технических наук, научный сотрудник Орловского филиала Института проб\-лем информатики
Российской академии наук
%302025, г.Орел, Московское шоссе, д.137

\vspace*{7pt}

%\columnbreak

\noindent 
\textbf{Соколов  Игорь Анатольевич} (р.\ 1954)~--- академик (действительный член) Российской 
академии наук, доктор технических наук, директор Института проб\-лем информатики
Российской академии наук

\vspace*{7pt}

\noindent
\textbf{Степченков Юрий Афанасьевич} (р.\ 1951)~---
кандидат технических наук, заведующий отделом Института проб\-лем информатики
Российской академии наук

\vspace*{7pt}

\noindent
\textbf{Сурков Алексей Викторович} (р.\ 1978)~--- 
старший научный сотрудник На\-уч\-но-ис\-сле\-до\-ва\-тель\-ско\-го 
института системных исследований Российской академии наук
%117218, Москва, Нахимовский просп., 36, к.1 

\vspace*{7pt}

\noindent 
\textbf{Шестаков Олег Владимирович} (р.\ 1976)~--- доктор 
фи\-зи\-ко-ма\-те\-ма\-ти\-че\-ских, доцент кафедры математической статистики 
факультета вычисли\-тельной математики и кибернетики Московского 
государственного университета им.\ М.\,В.~Ломоносова; 
%119991, Москва ГСП-1, Ленинские горы, д.\ 1, стр. 52
старший научный сотрудник Института проб\-лем информатики
Российской академии наук
%, Москва 119333, ул. Вавилова, д.~44, корп.~2

\vspace*{7pt}

\noindent 
\textbf{Шоргин Сергей Яковлевич} (р.\ 1952.)~--- доктор
фи\-зи\-ко-ма\-те\-ма\-ти\-че\-ских наук, профессор, заместитель директора Института 
проб\-лем информатики Российской академии наук





%%%%%%%%%%%%%%%%%%%%%%%%%%%%%%%%%%%%%%%%%%%%%%%%%%%%%%%%%%%%%%%%%%%%%%%%%%%%%%%




%\def\rightkol{ОБ АВТОРАХ}
%\def\leftkol{ОБ АВТОРАХ}

 \label{end\stat}





%\def\leftfootline{\small{\textbf{\thepage}
%\hfill ИНФОРМАТИКА И ЕЁ ПРИМЕНЕНИЯ\ \ \ том~7\ \ \ выпуск~1\ \ \ 2013}
%}%
% \def\rightfootline{\small{ИНФОРМАТИКА И ЕЁ ПРИМЕНЕНИЯ\ \ \ том~7\ \ \ выпуск~1\ \ \ 2013
%\hfill \textbf{\thepage}}}


%\thispagestyle{myheadings}



\end{multicols}

\newpage

%\end{document}

%
\def\stat{rekl}
%\label{preobr}

%\def\tit{АКАДЕМИК ПУГАЧЁВ  ВЛАДИМИР СЕМЁНОВИЧ\\
%25.03.1911--25.03.1998}


%   \vspace*{-48pt}
%   \begin{center}\LARGE
%Академик Пугачёв  Владимир Семёнович\\ (25.03.1911--25.03.1998)
%   \end{center}

   %\vspace*{2.5mm}

   \begin{center}

{\prgsh\LARGE
ЮБИЛЕИ}

\end{center}
%\hrule

\vspace*{6pt}


   \vspace*{8mm}

   \thispagestyle{empty}


%\def\stat{emel}


\section*{К 70-летию заместителя директора ИПИ РАН,\\ члена редколлегии журнала
<<Информатика и её применения>>\\ доктора технических наук В.\,И.~Будзко}

\vspace*{18pt}




          \begin{multicols}{2}

%            \label{st\stat}

\begin{center}
\vspace*{1pt}
\mbox{%
\epsfxsize=78mm
\epsfbox{bud-1.eps}
}
\end{center}

\vspace*{12pt}

      14 августа 2014~г.\ исполнилось 70~лет за\-мес\-ти\-те\-лю директора ИПИ РАН по
научной работе доктору технических наук Владимиру Игоревичу Будзко.

      Владимир Игоревич Будзко родился в г.~Москве. Высшее образование получил на факультете
элект\-рон\-но-вы\-чис\-ли\-тель\-ных устройств в Московском
ин\-же\-нер\-но-фи\-зи\-че\-ском институте
(МИФИ), который он окончил в 1968~г., после чего был на\-прав\-лен для прохождения
службы в одну из войс\-ко\-вых частей, где прошел путь от инженера до первого заместителя
командира войсковой части.

      С приходом В.\,И.~Будзко в ИПИ РАН (2001~г.)\ в институте
сформировалось новое научное на\-прав\-ле\-ние теоретических исследований~--- <<Постро\-ение
ин\-фор\-ма\-ци\-он\-но-те\-ле\-ком\-му\-ни\-ка\-ци\-он\-ных\linebreak сис\-тем
высокой до\-ступ\-ности>>. В~рамках этого
направления выполнен широкий круг фундаментальных исследований по поиску подходов и
определению принципов построения средств обеспечения доступности, конфиденциальности
и целостности современных крупномасштабных
ин\-фор\-ма\-ци\-он\-но-те\-ле\-ком\-му\-ни\-ка\-ци\-он\-ных
сис\-тем (ИТС). Разработаны основные сис\-тем\-но-тех\-ни\-че\-ские принципы и базовые
архитектурные решения построения перспективных для условий России ИТС с
централизованной обработкой и хранением информации, сочетающих в себе свойства
высокой доступности, отказо- и катастрофоустойчивости, информационной защищенности.
Определены принципы, методы и математические основы рационального построения и
оптимизации средств восстановления функционирования центров обработки данных (ЦОД)
после возникновения отказов и катастроф, передачи и хранения данных, обеспечения
информационной безопасности при достижении минимальной совокупной стоимости
владения такими системами. Результаты нашли практическое воплощение при реализации
проектов в интересах ряда отечественных государственных и негосударственных
организаций, таких как Банк России (БР), Внешторгбанк, ОАО <<ГМК <<Норильский Никель>>,
<<Газпром>>, Минэкономразвития России, Правительство Москвы, а также ряд силовых
ведомств.

      Под руководством В.\,И.~Будзко начиная с 2001~г.\ выполнен комплекс
      на\-уч\-но-ис\-сле\-до\-ва\-тель\-ских и
      опыт\-но-кон\-ст\-рук\-тор\-ских работ (свыше 100~проектов),
направленных на развитие электронной информационной технологии БР.
Разработаны концепции развития ИТС БР сначала до 2008~г., а затем до 2013~г., которые
были приняты в качестве основы проведения технической политики. За реализацию проекта
<<Катастрофоустойчивая тер\-ри\-то\-ри\-аль\-но-рас\-пре\-де\-лен\-ная
      ин\-фор\-ма\-ци\-он\-но-те\-ле\-ком\-му\-ни\-ка\-ци\-он\-ная сис\-те\-ма централизованной
обработки банковской информации>> В.\,И.~Будзко удостоен Премии Правительства РФ в
области науки и техники за 2010~г.

      В.\,И.~Будзко возглавлял и возглавляет работы по ряду других прикладных проектов,
связанных с созданием, совершенствованием и развитием крупномасштабных ИТС.

      В.\,И.~Будзко~--- генерал-майор, доктор технических наук, член-кор\-рес\-пон\-дент
Академии криптографии РФ, известный ученый в области информатики и применения
информационных технологий при построении территориально распределенных ИТС
различного назначения. Является автором свыше 250~научных работ, опубликованных в
на\-уч\-но-тех\-ни\-че\-ских и специальных изданиях.

    \thispagestyle{empty}

      В.\,И.~Будзко уделяет большое внимание подготовке научных кадров. Под его
руководством защищено 6~диссертаций на соискание ученой степени кандидата
технических наук. Свыше 30~лет он читает лекции в ИКСИ Академии ФСБ, профессор
кафедры НИЯУ МИФИ. Является членом двух диссертационных советов, главным
редактором журнала <<Системы высокой доступности>> и членом редколлегии журнала
<<Информатика и её применения>>.

      \bigskip

      Редакционный совет и Редакционная коллегия журнала <<Информатика и её
применения>> сердечно поздравляют Владимира Игоревича Будзко с 70-ле\-ти\-ем и желают
крепкого здоровья и новых научных достижений.

\end{multicols}

\def\stat{cont}
{%\hrule\par
%\vskip 7pt % 7pt
\raggedleft\Large \bf%\baselineskip=3.2ex
А\,В\,Т\,О\,Р\,С\,К\,И\,Й\ \ У\,К\,А\,З\,А\,Т\,Е\,Л\,Ь\ \ З\,А\ \ 2\,0\,1\,0 г. \vskip 17pt
    \hrule
    \par
\vskip 21pt plus 6pt minus 3pt }

\label{st\stat}

\def\tit{\ }

\def\aut{\ }
\def\auf{\ }

\def\leftkol{\ } % ENGLISH ABSTRACTS}

\def\rightkol{\ } %АВТОРСКИЙ УКАЗАТЕЛЬ ЗА 2010 г.} %ENGLISH ABSTRACTS}

\titele{\tit}{\aut}{\auf}{\leftkol}{\rightkol}

\vspace*{-12pt}

{\tabcolsep=3pt
\begin{tabular}{p{388pt}rr}
&\textbf{Выпуск} & \textbf{Стр.}\\[6pt]
\hangindent=23pt\noindent\textbf{Арутюнян~А.\,Р.} Моделирование влияния деформаций отпечатков пальцев на 
точность\linebreak
\vspace*{-12pt}\\
\hspace*{23pt}дактилоскопической идентификации$\dotfill$&1&51\\
\hangindent=23pt\noindent\textbf{Архипов~О.\,П., Зыкова~З.\,П.} Интеграция гетерогенной информации о цветных 
пикселях\linebreak
\vspace*{-12pt}\\
\hspace*{23pt}и их цветовосприятии$\dotfill$&4&15\\
\hangindent=23pt\noindent\textbf{Баранов~С.\,И., Френкель~С.\,Л., Захаров~В.\,Н.} Полуформальная верификация 
цифрового устройства с конвейером, основанная на использовании алгоритмических машин\linebreak
\vspace*{-12pt}\\
\hspace*{23pt}состояния$\dotfill$&4&49\\
\textbf{Бекетова~И.\,В.} см.~Каратеев~С.\,Л.&&\\
\textbf{Белоусов~В.\,В.} см.~Синицын~И.\,Н.&&\\
\hangindent=23pt\noindent\textbf{Бенинг~В.\,Е., Королев~Р.\,А.} О предельном поведении мощностей критериев в 
случае\linebreak
\vspace*{-12pt}\\
\hspace*{23pt}распределения Лапласа$\dotfill$&2&63\\
\hangindent=23pt\noindent\textbf{Бенинг~В.\,Е., Сипина~А.\,В.} Асимптотическое разложение для мощности 
критерия,\linebreak
\vspace*{-12pt}\\
\hspace*{23pt}основанного на выборочной медиане, в случае распределения Лапласа$\dotfill$&1&18\\
\textbf{Бондаренко~А.\,В.} см.~Каратеев~С.\,Л.&&\\
\hangindent=23pt\noindent\textbf{Бородина~А.\,В., Морозов~Е.\,В.} Об оценивании асимптотики вероятности 
большого\linebreak
\vspace*{-12pt}\\
\hspace*{23pt}уклонения стационарной регенеративной очереди с одним прибором$\dotfill$&3&29\\
\hangindent=23pt\noindent\textbf{Бунтман~Н.\,В., Минель~Ж.-Л., Ле~Пезан~Д., Зацман~И.\,М.} Типология и 
компьютерное\linebreak
\vspace*{-12pt}\\
\hspace*{23pt}моделирование трудностей перевода$\dotfill$&3&77\\
\textbf{Визильтер~Ю.\,В.} см.~Каратеев~С.\,Л.&&\\
\hangindent=23pt\noindent\textbf{Гавриленко~С.\,В.} Оценки скорости сходимости распределений случайных сумм с 
безгранично делимыми индексами к нормальному закону$\dotfill$&4&81\\
\hangindent=23pt\noindent\textbf{Григорьева~М.\,Е., Шевцова~И.\,Г.} Уточнение неравенства 
Каца--Берри--Эссеена$\dotfill$&2&75\\
\hangindent=23pt\noindent\textbf{Грушо~А.\,А., Грушо~Н.\,А., Тимонина~Е.\,Е.} Поиск конфликтов в политиках 
безопасности: модель случайных графов$\dotfill$&3&38\\
\textbf{Грушо~Н.\,А.} см.~Грушо~А.\,А.&&\\
\hangindent=23pt\noindent\textbf{Гудков~В.\,Ю.} Математические модели изображения отпечатка пальца на основе 
описания линий$\dotfill$&1&58\\
\textbf{Гуртов~А.\,В.} см.~Лукьяненко~А.\,С.&&\\
\textbf{Желтов~С.\,Ю.} см.~Каратеев~С.\,Л.&&\\
\hangindent=23pt\noindent\textbf{Захаров~А.\,А., Серебряков~В.\,А.} Система управления электронной библиотекой 
LibMeta$\dotfill$&4&2\\
\textbf{Захаров~В.\,Н.} см.~Баранов~С.\,И.&&\\
\textbf{Захарова~Т.\,В.} см.~Матвеева~С.\,С.&&\\
\hangindent=23pt\noindent\textbf{Зацаринный~А.\,А., Чупраков~К.\,Г.} Некоторые аспекты выбора технологии для 
постро-\linebreak
\vspace*{-12pt}\\
\hspace*{23pt}ения систем отображения информации ситуационного центра$\dotfill$&3&59\\
\textbf{Зацман~И.\,М.} см.~Бунтман~Н.\,В.&&\\
\hangindent=23pt\noindent\textbf{Зейфман~А.\,И., Коротышева~А.\,В., Сатин~Я.\,А., Шоргин~С.\,Я.} Об 
устойчивости нестаци-\linebreak
\vspace*{-12pt}\\
\hspace*{23pt}онарных систем обслуживания с катастрофами$\dotfill$&3&9\\
\textbf{Зыкова~З.\,П.} см.~Архипов~О.\,П.&&\\
\hangindent=23pt\noindent\textbf{Илюшин~Г.\,Я., Соколов~И.\,А.} Организация управляемого доступа пользователей 
к\linebreak
\vspace*{-12pt}\\
\hspace*{23pt}разнородным ведомственным информационным ресурсам$\dotfill$&1&24\\
\hangindent=23pt\noindent\textbf{Кавагучи~Ю., Ульянов~В.\,В., Фуджикоши~Я.} Приближения для статистик, 
описывающих\linebreak
\vspace*{-12pt}\\
\hspace*{23pt}геометрические свойства данных большой размерности, с оценками 
ошибок$\dotfill$&1&12\\
\hangindent=23pt\noindent\textbf{Каратеев~С.\,Л., Бекетова~И.\,В., Ососков~М.\,В., Князь~В.\,А., 
Визильтер~Ю.\,В., Бондаренко~А.\,В., Желтов~С.\,Ю.} Автоматизированный контроль 
качества цифровых\linebreak
\vspace*{-12pt}\\
\hspace*{23pt}изображений для персональных документов$\dotfill$&1&65\\
\end{tabular}
}

\pagebreak

\def\leftkol{АВТОРСКИЙ УКАЗАТЕЛЬ ЗА 2010 г.} % ENGLISH ABSTRACTS}

\def\rightkol{АВТОРСКИЙ УКАЗАТЕЛЬ ЗА 2010 г.} %ENGLISH ABSTRACTS}

{\tabcolsep=3pt
\begin{tabular}{p{388pt}rr}
&\textbf{Выпуск} & \textbf{Стр.}\\[3pt]
\hangindent=23pt\noindent\textbf{Козеренко~Е.\,Б.} Лингвистические фильтры в статистических моделях машинного\linebreak
\vspace*{-12pt}\\
\hspace*{23pt}перевода$\dotfill$&2&83\\
\hangindent=23pt\noindent\textbf{Козеренко~Е.\,Б., Кузнецов~И.\,П.} Когнитивно-лингвистические представления в 
систе-\linebreak
\vspace*{-12pt}\\
\hspace*{23pt}мах обработки текстов$\dotfill$&3&69\\
\textbf{Князь~В.\,А.} см.~Каратеев~С.\,Л.&&\\
\hangindent=23pt\noindent\textbf{Колесников~А.\,В., Солдатов~С.\,А.} Алгоритм координации для гибридной 
интеллектуальной системы решения сложной задачи оперативно-производственного\linebreak
\vspace*{-12pt}\\
\hspace*{23pt}планирования$\dotfill$&4&61\\
\hangindent=23pt\noindent\textbf{Коновалов~М.\,Г.} О планировании потоков в системах вычислительных 
ресурсов$\dotfill$&2&3\\
\textbf{Конушин~А.\,С.} см.~Конушин~В.\,С.&&\\
\hangindent=23pt\noindent\textbf{Конушин~В.\,С., Кривовязь~Г.\,Р., Конушин~А.\,С.} Алгоритм распознавания людей 
в видео-\linebreak
\vspace*{-12pt}\\
\hspace*{23pt}последовательности по одежде$\dotfill$&1&74\\
\textbf{Корепанов~Э.\, Р.} см.~Синицын~И.\,Н.&&\\
\textbf{Королев~В.\,Ю.} см.~Соколов~И.\,А.&&\\
\textbf{Королев~Р.\,А.} см.~Бенинг~В.\,Е.&&\\
\textbf{Коротышева~А.\,В.} см.~Зейфман~А.\,И.&&\\
\hangindent=23pt\noindent\textbf{Кривенко~М.\,П.} Непараметрическое оценивание элементов байесовского 
клас\-си-\linebreak
\vspace*{-12pt}\\
\hspace*{23pt}фикатора$\dotfill$&2&13\\
\textbf{Кривовязь~Г.\,Р.} см.~Конушин~В.\,С.&&\\
\textbf{Крылов~А.\,С.} см.~Павельева~Е.\,А.&&\\
\hangindent=23pt\noindent\textbf{Крылов~В.\,А.} Моделирование и классификация многоканальных дистанционных\linebreak
\vspace*{-12pt}\\
\hspace*{23pt}изображений с использованием копул$\dotfill$&4&34\\
\hangindent=23pt\noindent\textbf{Крючин~О.\,В.} Разработка параллельных эвристических алгоритмов подбора 
весовых\linebreak
\vspace*{-12pt}\\
\hspace*{23pt}коэффициентов искусственной нейтронной сети$\dotfill$&2&53\\
\hangindent=23pt\noindent\textbf{Кудрявцев~А.\,А., Шоргин~С.\,Я.} Байесовские модели массового обслуживания и 
надеж-\linebreak
\vspace*{-12pt}\\
\hspace*{23pt}ности: характеристики среднего числа заявок в системе $M\vert M \vert 1\vert 
\infty$$\dotfill$&3&16\\
\hangindent=23pt\noindent\textbf{Кузнецов~А.\,А.} Связь между временными и структурно-топологическими 
характери-\linebreak
\vspace*{-12pt}\\
\hspace*{23pt}стиками диаграмм ритма сердца здоровых людей$\dotfill$&4&39\\
\textbf{Кузнецов~И.\,П.} см.~Козеренко~Е.\,Б.&&\\
\textbf{Ле~Пезан~Д.} см.~Бунтман~Н.\,В.&&\\
\hangindent=23pt\noindent\textbf{Лукьяненко~А.\,С., Морозов~Е.\,В., Гуртов~А.\,В.} Анализ сетевого протокола с общей 
функ-\linebreak
\vspace*{-12pt}\\
\hspace*{23pt}цией расширения окна передачи сообщения при конфликтах$\dotfill$&2&46\\
\hangindent=23pt\noindent\textbf{Лямин~О.\,О.} О предельном поведении мощностей критериев в случае обобщенного\linebreak
\vspace*{-12pt}\\
\hspace*{23pt}распределения Лапласа$\dotfill$&3&47\\
\hangindent=23pt\noindent\textbf{Маркин~А.\,В., Шестаков~О.\,В.} Асимптотики оценки риска при пороговой 
обработке\linebreak
\vspace*{-12pt}\\
\hspace*{23pt}вейвлет-вейглет коэффициентов в задаче томографии$\dotfill$&2&36\\
\hangindent=23pt\noindent\textbf{Матвеева~С.\,С., Захарова~Т.\,В.} Сети массового обслуживания с наименьшей 
длиной\linebreak
\vspace*{-12pt}\\
\hspace*{23pt}очереди$\dotfill$&3&22\\
\hangindent=23pt\noindent\textbf{Матюшенко~С.\,И.} Стационарные характеристики двухканальной системы 
обслужива-\linebreak
\vspace*{-12pt}\\
\hspace*{23pt}ния с переупорядочиванием заявок и распределениями фазового типа$\dotfill$&4&68\\
\textbf{Минель~Ж.-Л.} см.~Бунтман~Н.\,В.&&\\
\textbf{Морозов~Е.\,В.} см.~Бородина~А.\,В.&&\\
\textbf{Морозов~Е.\,В.} см.~Лукьяненко~А.\,С.&&\\
\textbf{Ососков~М.\,В.} см.~Каратеев~С.\,Л.&&\\
\hangindent=23pt\noindent\textbf{Павельева~Е.\,А., Крылов~А.\,С.} Поиск и анализ ключевых точек радужной 
оболочки\linebreak
\vspace*{-12pt}\\
\hspace*{23pt}глаза методом преобразования Эрмита$\dotfill$&1&79\\
\textbf{Печинкин~А.\,В.} см.~Френкель~С.\,Л.,&&\\
\hangindent=23pt\noindent\textbf{Протасов~В.\,И.} Составление субъективного портрета с использованием 
эволюционно-\linebreak
\vspace*{-12pt}\\
\hspace*{23pt}го морфинга и квалиметрия метода$\dotfill$&1&83\\
\hangindent=23pt\noindent\textbf{Рудаков~К.\,В., Торшин~И.\,Ю.} Вопросы разрешимости задачи распознавания 
вторичной\linebreak
\vspace*{-12pt}\\
\hspace*{23pt}структуры белка$\dotfill$&2&25\\
\textbf{Сатин~Я.\,А.} см.~Зейфман~А.\,И.&&\\
\hangindent=23pt\noindent\textbf{Сейфуль-Мулюков~Р.\,Б.} Нефть как носитель информации о своем 
происхождении,\linebreak
\vspace*{-12pt}\\
\hspace*{23pt}структуре и эволюции$\dotfill$&1&41\\
\end{tabular}
}

{\tabcolsep=3pt
\begin{tabular}{p{388pt}rr}
&\textbf{Выпуск} & \textbf{Стр.}\\[6pt]
\textbf{Семендяев~Н.\,Н.} см.~Синицын~И.\,Н.&&\\
\textbf{Серебряков~В.\,А.} см.~Захаров~А.\,А.&&\\
\textbf{Синицын~В.\,И.} см.~Синицын~И.\,Н.&&\\
\hangindent=23pt\noindent\textbf{Синицын~И.\,Н., Синицын~В.\,И., Корепанов~Э.\, Р., Белоусов~В.\,В., 
Семендяев~Н.\,Н.} Оперативное построение информационных моделей движения полюса 
Земли\linebreak
\vspace*{-12pt}\\
\hspace*{23pt}методами линейных и линеаризованных фильтров$\dotfill$&1&2\\
\textbf{Сипина~А.\,В.} см.~Бенинг~В.\,Е.&&\\
\hangindent=23pt\noindent\textbf{Соколов~И.\,А.} О работах заслуженного деятеля науки Российской Федерации 
И.\,Н.~Синицына в области информационных технологий и автоматизации (к 70-летию\linebreak
\vspace*{-12pt}\\
\hspace*{23pt}со дня рождения)$\dotfill$&3&84\\
\textbf{Соколов~И.\,А.} см.~Илюшин~Г.\,Я.&&\\
\hangindent=23pt\noindent\textbf{Соколов~И.\,А., Королев~В.\,Ю.} Предисловие$\dotfill$&2&2\\
\textbf{Солдатов~С.\,А.} см.~Колесников~А.\,В.&&\\
\hangindent=23pt\noindent\textbf{Степанов~С.\,Ю.} Использование координатного метода фрагментации 
коммутаторной\linebreak
\vspace*{-12pt}\\
\hspace*{23pt}нейронной сети для сокращения трафика$\dotfill$&2&57\\
\textbf{Тимонина~Е.\,Е.} см.~Грушо~А.\,А.&&\\
\textbf{Торшин~И.\,Ю.} см.~Рудаков~К.\,В.&&\\
\textbf{Ульянов~В.\,В.} см.~Кавагучи~Ю.&&\\
\textbf{Фазекаш~И.} см.~Чупрунов~А.\,Н.&&\\
\textbf{Френкель~С.\,Л.} см.~Баранов~С.\,И.&&\\
\hangindent=23pt\noindent\textbf{Френкель~С.\,Л., Печинкин~А.\,В.} Оценка времени самовосстановления в 
цифровых\linebreak
\vspace*{-12pt}\\
\hspace*{23pt}системах после сбоев, вызываемых переходными помехами$\dotfill$&3&2\\
\textbf{Фуджикоши~Я.} см.~Кавагучи~Ю.&&\\
\hangindent=23pt\noindent\textbf{Цискаридзе~А.\,К.} Математическая модель и метод восстановления позы человека 
по\linebreak
\vspace*{-12pt}\\
\hspace*{23pt}стереопаре силуэтных изображений$\dotfill$&4&27\\
\hangindent=23pt\noindent\textbf{Чупраков~К.\,Г.} К вопросу о размещении коллективных средств отображения в 
ситуа-\linebreak
\vspace*{-12pt}\\
\hspace*{23pt}ционном зале с заданными параметрами$\dotfill$&4&89\\
\textbf{Чупраков~К.\,Г.} см.~Зацаринный~А.\,А.&&\\
\hangindent=23pt\noindent\textbf{Чупрунов~А.\,Н., Фазекаш~И.} Законы повторного логарифма для числа 
безошибочных\linebreak
\vspace*{-12pt}\\
\hspace*{23pt}блоков при помехоустойчивом кодировании$\dotfill$&3&42\\
\textbf{Шевцова~И.\,Г.} см.~Григорьева~М.\,Е.&&\\
\hangindent=23pt\noindent\textbf{Шестаков~О.\,В.} Аппроксимация распределения оценки риска пороговой 
обработки вейвлет-коэффициентов нормальным распределением при использовании 
выбо-\linebreak
\vspace*{-12pt}\\
\hspace*{23pt}рочной дисперсии$\dotfill$&4&73\\
\textbf{Шестаков~О.\,В.} см.~Маркин~А.\,В.&&\\
\textbf{Шоргин~С.\,Я.} см.~Зейфман~А.\,И.&&\\
\textbf{Шоргин~С.\,Я.} см.~Кудрявцев~А.\,А.&&\\
\end{tabular}
}

%\thispagestyle{myheadings}
\def\leftfootline{\small{\textbf{\thepage}
\hfill ИНФОРМАТИКА И ЕЁ ПРИМЕНЕНИЯ\ \ \ том~4\ \ \ выпуск~4\ \ \ 2010}
}%
 \def\rightfootline{\small{ИНФОРМАТИКА И ЕЁ ПРИМЕНЕНИЯ\ \ \ том~4\ \ \ выпуск~4\ \ \ 2010
 \hfill \textbf{\thepage}}}
 \label{end\stat}


%Том 10 Выпуск 1-4 Год 2016

\def\stat{cont-e}
{%\hrule\par
%\vskip 7pt % 7pt
\raggedleft\Large \bf%\baselineskip=3.2ex
2\,0\,1\,6\ \ A\,U\,T\,H\,O\,R\ \ I\,N\,D\,E\,X \vskip 17pt
 \hrule
 \par
\vskip 21pt plus 6pt minus 3pt }

\label{st\stat}

\def\tit{\ }

\def\aut{\ }
\def\auf{\ }

\def\leftkol{\ } %2016 AUTHOR INDEX} % ENGLISH ABSTRACTS}

\def\rightkol{\ } %2016 AUTHOR INDEX} %ENGLISH ABSTRACTS}

\titele{\tit}{\aut}{\auf}{\leftkol}{\rightkol}

\def\leftfootline{\small{\textbf{\thepage}
\hfill INFORMATIKA I EE PRIMENENIYA~--- INFORMATICS AND APPLICATIONS\ \ \ 2016\
\ \ volume~10\ \ \ issue\ 4}
}%
 \def\rightfootline{\small{INFORMATIKA I EE PRIMENENIYA~--- INFORMATICS AND APPLICATIONS\ \ \ 2016\ \ \ volume~10\ \ \ issue\ 4
\hfill \textbf{\thepage}}}

\vspace*{-12pt}
\vspace*{-18pt}

{\tabcolsep=2.8pt
\begin{tabular}{p{382pt}cc}
&\textbf{Issue} & \textbf{Page}\\[6pt]
\Avtors{Agalarov~M.\,Ya.} see~Agalarov~Ya.\,M.&&\\
\Avtors{Agalarov~Ya.\,M., Agalarov~M.\,Ya., and
Shorgin~V.\,S.} About the optimal threshold of queue\linebreak
\\[-12pt]
\hspace*{23pt}length in a~particular problem of profit maximization
in the $M/G/1$ queuing system&2&70--79\\
\Avtors{Alexeyevsky~D.\,A.} BioNLP ontology extraction from 
a~restricted language corpus with\linebreak
\\[-12pt]
\hspace*{23pt}context-free grammars&1&119--128\\
\Avtors{Andreev~S.\,D.} see~Gaidamaka~Yu.\,V.&&\\
\Avtors{Andreev~S.\,D.} see~Ometov~A.\,Ya.&&\\
\Avtors{Arkhipov~O.\,P., Arkhipov~P.\,O., and Sidorkin~I.\,I.} The
option to create a~local coordinate\linebreak
\\[-12pt]
\hspace*{23pt}system for synchronization of selected images&3&91--97\\
\Avtors{Arkhipov~P.\,O.} see~Arkhipov~O.\,P.&&\\
\Avtors{Belousov~V.\,V.} see~Shnurkov~P.\,V.&&\\
\Avtors{Belousov~V.\,V.} see~Shnurkov~P.\,V.&&\\
\Avtors{Bening~V.\,E.} Calculation of~the~asymptotic deficiency
of~some statistical procedures based\linebreak
\\[-12pt]
\hspace*{23pt}on~samples with~random sizes&4&34--45\\
\Avtors{Borisov~A.\,V., Bosov~A.\,V., and Miller~G.\,B.} Modeling and
monitoring of VoIP connection&2&\hphantom{1}2--13\\
\Avtors{Bosov~A.\,V.} see~Borisov~A.\,V.&&\\
\Avtors{Briukhov~D.\,O.} see~Stupnikov~S.\,A.&&\\
\Avtors{Callaos~N.\,K.\ and Seyful-Mulyukov~R.\,B.} Complexity and
its information content&1&129--139\\
\Avtors{Chertok~A.\,V., Kadaner~A.\,I., Khazeeva~G.\,T., and
Sokolov~I.\,A.} Regime switching detection\linebreak
\\[-12pt]
\hspace*{23pt}for~the~Levy driven
Ornstein--Uhlenbeck process using CUSUM methods&4&46--56\\
\Avtors{Chichagov~V.\,V.} Asymptotic expansions of mean absolute
error of uniformly minimum variance unbiased and maximum likelihood
estimators on the one-parameter exponential\linebreak
\\[-12pt]
\hspace*{23pt}family model of lattice distributions&3&66--76\\
\Avtors{Danishevsky~V.\,I.} see~Kolesnikov A.\,V.&&\\
\Avtors{Fazliev~A.\,Z.} see~Kalinichenko~L.\,A.&&\\
\Avtors{Fedoseev~A.\,A.} What is behind the concept of ``knowledge in
small packages''&3&105--110\\
\Avtors{Gaidamaka~Yu.\,V., Andreev~S.\,D., Sopin~E.\,S.,
Samouylov~K.\,E., and Shorgin~S.\,Ya.} Interference analysis
of~the~device-to-device communications model with~regard to~a~signal\linebreak
\\[-12pt]
\hspace*{23pt}propagation environment&4&\hphantom{1}2--10\\
\Avtors{Gasilov~A.\,V.} see~Yakovlev~O.\,A.&&\\
\Avtors{Goncharov~A.\,V.\ and Strijov~V.\,V.} Metric time series
classification using weighted dynamic\linebreak
\\[-12pt]
\hspace*{23pt}warping relative to centroids of classes&2&36--47\\
\Avtors{Gordov~E.\,P.} see~Kalinichenko~L.\,A.&&\\
\Avtors{Gorshenin~A.\,K.} Concept of online service for stochastic
modeling of real processes&1&72--81\\
\Avtors{Gorshenin~A.\,K.} see~Shnurkov~P.\,V.&&\\
\Avtors{Gorshenin~A.\,K.} see~Shnurkov~P.\,V.&&\\
\Avtors{Grusho~A.\,A., Grusho~N.\,A., Zabezhailo~M.\,I., and
Timonina~E.\,E.} Integration of statistical and\linebreak
\\[-12pt]
\hspace*{23pt}deterministic methods for
analysis of information security&3&2--8\\
\Avtors{Grusho~A.\,A., Zabezhailo~M.\,I., and Zatsarinny~A.\,A.} On
the advanced procedure to reduce\linebreak
\\[-12pt]
\hspace*{23pt}calculation of Galois closures&4&\hphantom{1}96--104\\
\Avtors{Grusho~N.\,A.} see~Grusho~A.\,A.&&\\
\Avtors{Havanskov~V.\,A.} see~Minin~V.\,A.&&\\
\Avtors{Inkova~O.\,Yu.} see~Zatsman~I.\,M.&&\\
\Avtors{Isachenko~R.\,V.\ and Strijov~V.\,V.} Metric learning in
multiclass time series classification\linebreak
\\[-12pt]
\hspace*{23pt}problem&2&48--57\\
\end{tabular}
}
\pagebreak

\def\leftfootline{\small{\textbf{\thepage}
\hfill INFORMATIKA I EE PRIMENENIYA~--- INFORMATICS AND APPLICATIONS\ \ \ 2016\
\ \ volume~10\ \ \ issue\ 4}
}%
 \def\rightfootline{\small{INFORMATIKA I EE PRIMENENIYA~---
INFORMATICS AND APPLICATIONS\ \ \ 2016\ \ \ volume~10\ \ \ issue\ 4
\hfill \textbf{\thepage}}}

\def\leftkol{2016 AUTHOR INDEX} % ENGLISH ABSTRACTS}

\def\rightkol{2016 AUTHOR INDEX} %ENGLISH ABSTRACTS}


{\tabcolsep=2.83pt
\begin{tabular}{p{382pt}cc}
&\textbf{Issue} & \textbf{Page}\\[6pt]
\Avtors{Kadaner~A.\,I.} see~Chertok~A.\,V.&&\\[.255pt]
\Avtors{Kalinichenko~L.\,A., Volnova~A.\,A., Gordov~E.\,P.,
Kiselyova~N.\,N., Kovaleva~D.\,A., Malkov~O.\,Yu., Okladnikov~I.\,G.,
Podkolodnyy~N.\,L., Pozanenko~A.\,S., Ponomareva~N.\,V.,
Stupnikov~S.\,A.,} \textbf{and Fazliev~A.\,Z.} Data access challenges for data
intensive\linebreak
\\[-12pt]
\hspace*{23pt}research in Russia&1& 2--22\\[.255pt]
\Avtors{Karasikov~M.\,E.\ and Strijov~V.\,V.} Feature-based
time-series classification&4&121--131\\[.255pt]
\Avtors{Khazeeva~G.\,T.} see~Chertok~A.\,V.&&\\[.255pt]
\Avtors{Khokhlov~Yu.\,S.} Multivariate fractional Levy motion and its
applications&2&\hphantom{1}98--106\\[.255pt]
\Avtors{Kirikov~I.\,A., Kolesnikov~A.\,V., Listopad~S.\,V., and
Rumovskaya~S.\,B.} Fine-grained hybrid\linebreak
\\[-12pt]
\hspace*{23pt}intelligent systems. Part 2:
Bidirectional hybridization&1&\hphantom{1}96--105\\[.255pt]
\Avtors{Kirikov~I.\,A., Kolesnikov~A.\,V., Listopad~S.\,V., and
Rumovskaya~S.\,B.} ``Virtual council''~---\linebreak
\\[-12pt]
\hspace*{23pt}source environment
supporting complex diagnostic decision making&3&81--90\\[.255pt]
\Avtors{Kiselyova~N.\,N.} see~Kalinichenko~L.\,A.&&\\[.255pt]
\Avtors{Kolesnikov A.\,V., Listopad~S.\,V., Rumovskaya~S.\,B., and
Danishevsky~V.\,I.} Informal axiomatic\linebreak
\\[-12pt]
\hspace*{23pt}theory of~the~role visual models&4&114--120\\[.255pt]
\Avtors{Kolesnikov~A.\,V.} see~Kirikov~I.\,A.&&\\[.255pt]
\Avtors{Kolesnikov~A.\,V.} see~Kirikov~I.\,A.&&\\[.255pt]
\Avtors{Kolin~K.\,K.} Humanitarian aspects of information
security&3&111--121\\[.255pt]
\Avtors{Konovalov~M.\,G.\ and Razumchik~R.\,V.} Dispatching
to~two parallel nonobservable queues using\linebreak
\\[-12pt]
\hspace*{23pt}only static
information&4&57--67\\[.255pt]
\Avtors{Korchagin~A.\,Yu.} see~Korolev~V.\,Yu.&&\\[.255pt]
\Avtors{Korchagin~A.\,Yu.} see~Korolev~V.\,Yu.&&\\[.255pt]
\Avtors{Korepanov~E.\,R.} see~Sinitsyn~I.\,N.&&\\[.255pt]
\Avtors{Korepanov~E.\,R.} see~Sinitsyn~I.\,N.&&\\[.255pt]
\Avtors{Korolev~V.\,Yu., Korchagin~A.\,Yu., and Zeifman~A.\,I.} The
Poisson theorem for Bernoulli trials\linebreak
\\[-12pt]
\hspace*{23pt}with~a~random probability
of~success and~a~discrete analog of~the~Weibull distribution&4&11--20\\[.255pt]
\Avtors{Korolev~V.\,Yu., Zeifman~A.\,I., and Korchagin~A.\,Yu.}
Asymmetric Linnik distributions as~limit\linebreak
\\[-12pt]
\hspace*{23pt}laws for~random sums
of~independent random variables with~finite variances&4&21--33\\[.255pt]
\Avtors{Koucheryavy~E.\,A.} see~Ometov~A.\,Ya.&&\\[.255pt]
\Avtors{Kovaleva~D.\,A.} see~Kalinichenko~L.\,A.&&\\[.255pt]
\Avtors{Kovalyov~S.\,P.} Metaprogramming to increase
manufacturability of large-scale software-\linebreak
\\[-12pt]
\hspace*{23pt}intensive systems&1&56--66\\[.255pt]
\Avtors{Krivenko~M.\,P.} Significance tests of feature selection for
classification&3&32--40\\[.255pt]
\Avtors{Kruzhkov~M.\,G.} see~Zalizniak~Anna~A.&&\\[.255pt]
\Avtors{Kruzhkov~M.\,G.} see~Zatsman~I.\,M.&&\\[.255pt]
\Avtors{Kudryavtsev~A.\,A.} Bayesian queueing and reliability models:
\textit{A~priori} distributions with\linebreak
\\[-12pt]
\hspace*{23pt}compact support&1&67--71\\[.255pt]
\Avtors{Kudryavtsev~A.\,A.} Characteristics dependent on the balance
coefficient in Bayesian models\linebreak
\\[-12pt]
\hspace*{23pt}with compact support of \textit{a priori}
distributions&3&77--80\\[.255pt]
\Avtors{Kudryavtsev~A.\,A.\ and Palionnaia~S.\,I.} Bayesian recurrent
model of reliability growth:\linebreak
\\[-12pt]
\hspace*{23pt}Parabolic distribution of parameters&2&80--83\\[.255pt]
\Avtors{Kudryavtsev~A.\,A.\ and Titova~A.\,I.} Bayesian queuing
and~reliability models: Degenerate-\linebreak
\\[-12pt]
\hspace*{23pt}Weibull case&4&68--71\\[.255pt]
\Avtors{Leontyev~N.\,D.\ and Ushakov~V.\,G.} Analysis of a queueing
system with autoregressive arrivals\linebreak
\\[-12pt]
\hspace*{23pt}and nonpreemptive priority&3&15--22\\[.255pt]
\Avtors{Listopad~S.\,V.} see~Kirikov~I.\,A.&&\\[.255pt]
\Avtors{Listopad~S.\,V.} see~Kirikov~I.\,A.&&\\[.255pt]
\Avtors{Listopad~S.\,V.} see~Kolesnikov A.\,V.&&\\[.255pt]
\Avtors{Malkov~O.\,Yu.} see~Kalinichenko~L.\,A.&&\\[.255pt]
\Avtors{Markov~A.\,S., Monakhov~M.\,M., and
Ulyanov~V.\,V.} Generalized Cornish--Fisher expansions\linebreak
\\[-12pt]
\hspace*{23pt}for distributions of statistics based on samples
of random size&2&84--91\\[.255pt]
\Avtors{Melnikov~A.\,K.\ and Ronzhin~A.\,F.} Generalized statistical
method of~text analysis based\linebreak
\\[-12pt]
\hspace*{23pt}on~calculation of~probability distributions
of~statistical values&4&89--95\\
\end{tabular}
}
\pagebreak

\def\leftfootline{\small{\textbf{\thepage}
\hfill INFORMATIKA I EE PRIMENENIYA~--- INFORMATICS AND APPLICATIONS\ \ \ 2016\
\ \ volume~10\ \ \ issue\ 4}
}%
 \def\rightfootline{\small{INFORMATIKA I EE PRIMENENIYA~---
INFORMATICS AND APPLICATIONS\ \ \ 2016\ \ \ volume~10\ \ \ issue\ 4
\hfill \textbf{\thepage}}}

\def\leftkol{2016 AUTHOR INDEX} % ENGLISH ABSTRACTS}

\def\rightkol{2016 AUTHOR INDEX} %ENGLISH ABSTRACTS}


{\tabcolsep=3pt
\begin{tabular}{p{381pt}cc}
&\textbf{Issue} & \textbf{Page}\\[6pt]
\Avtors{Meykhanadzhyan~L.\,A.} Stationary characteristics of the finite
capacity queueing system with\linebreak
\\[-12pt]
\hspace*{23pt}inverse service order and generalized
probabilistic priority&2&123--131\\[.23pt]
\Avtors{Miller~G.\,B.} see~Borisov~A.\,V.&&\\[.23pt]
\Avtors{Minin~V.\,A., Zatsman~I.\,M., Havanskov~V.\,A., and
Shubnikov~S.\,K.} Intensity of citation of scientific publications in
inventions on information and computer technologies patented\linebreak
\\[-12pt]
\hspace*{23pt}in Russia by domestic and foreign applicants&2&107--122\\[.23pt]
\Avtors{Monakhov~M.\,M.} see~Markov~A.\,S.&&\\[.23pt]
\Avtors{Naumov~V.\,A.\ and Samouylov~K.\,E.} On relationship
between queuing systems with resources\linebreak
\\[-12pt]
\hspace*{23pt}and Erlang networks&3&\hphantom{1}9--14\\[.23pt]
\Avtors{Okladnikov~I.\,G.} see~Kalinichenko~L.\,A.&&\\[.23pt]
\Avtors{Ometov~A.\,Ya., Andreev~S.\,D., Turlikov~A.\,M., and
Koucheryavy~E.\,A.} Performance analysis of\linebreak
\\[-12pt]
\hspace*{23pt}a wireless data
aggregation system with contention for contemporary sensor
networks&3&23--31\\[.23pt]
\Avtors{Palionnaia~S.\,I.} see~Kudryavtsev~A.\,A.&&\\[.23pt]
\Avtors{Podkolodnyy~N.\,L.} see~Kalinichenko~L.\,A.&&\\[.23pt]
\Avtors{Ponomareva~N.\,V.} see~Kalinichenko~L.\,A.&&\\[.23pt]
\Avtors{Popkova~N.\,A.} see~Zatsman~I.\,M.&&\\[.23pt]
\Avtors{Pozanenko~A.\,S.} see~Kalinichenko~L.\,A.&&\\[.23pt]
\Avtors{Razumchik~R.\,V.} see~Konovalov~M.\,G.&&\\[.23pt]
\Avtors{Ronzhin~A.\,F.} see~Melnikov~A.\,K.&&\\[.23pt]
\Avtors{Rumovskaya~S.\,B.} see~Kirikov~I.\,A.&&\\[.23pt]
\Avtors{Rumovskaya~S.\,B.} see~Kirikov~I.\,A.&&\\[.23pt]
\Avtors{Rumovskaya~S.\,B.} see~Kolesnikov A.\,V.&&\\[.23pt]
\Avtors{Samouylov~K.\,E.} see~Gaidamaka~Yu.\,V.&&\\[.23pt]
\Avtors{Samouylov~K.\,E.} see~Naumov~V.\,A.&&\\[.23pt]
\Avtors{Serebryanskii~S.\,M.} see~Tyrsin~A.\,N.&&\\[.23pt]
\Avtors{Seyful-Mulyukov~R.\,B.} see~Callaos~N.\,K.&&\\[.23pt]
\Avtors{Shestakov~O.\,V.} Statistical properties of the denoising method
based on the stabilized hard\linebreak
\\[-12pt]
\hspace*{23pt}thresholding&2&65--69\\[.23pt]
\Avtors{Shestakov~O.\,V.} The strong law of large numbers for the risk
estimate in the problem of\linebreak
\\[-12pt]
\hspace*{23pt}tomographic image reconstruction from
projections with a correlated noise&3&41--45\\[.23pt]
\Avtors{Shestakov~O.\,V.} see~Zakharova~T.\,V.&&\\[.23pt]
\Avtors{Shnurkov~P.\,V., Gorshenin~A.\,K., and Belousov~V.\,V.}
Analytical solution of~the~optimal control\linebreak
\\[-12pt]
\hspace*{23pt}task of~a~semi-Markov
process with~finite set of~states&4&72--88\\[.23pt]
\Avtors{Shnurkov~P.\,V., Zasypko~V.\,V., Belousov~V.\,V., and
Gorshenin~A.\,K.} Development of the algorithm of numerical solution
of the optimal investment control problem\linebreak
\\[-12pt]
\hspace*{23pt}in the closed dynamical model of three-sector economy&1&82--95\\[.23pt]
\Avtors{Shorgin~S.\,Ya.} see~Gaidamaka~Yu.\,V.&&\\[.23pt]
\Avtors{Shorgin~V.\,S.} see~Agalarov~Ya.\,M.&&\\[.23pt]
\Avtors{Shubnikov~S.\,K.} see~Minin~V.\,A.&&\\[.23pt]
\Avtors{Sidorkin~I.\,I.} see~Arkhipov~O.\,P.&&\\[.23pt]
\Avtors{Sinitsyn~I.\,N.} Analytical modeling of processes in stochastic
systems with complex fractional\linebreak
\\[-12pt]
\hspace*{23pt}order Bessel nonlinearities&3&55--65\\[.23pt]
\Avtors{Sinitsyn~I.\,N.} Orthogonal supoptimal filters for nonlinear
stochastic systems on manifolds&1&34--44\\[.23pt]
\Avtors{Sinitsyn~I.\,N.\ and Korepanov~E.\,R.} Normal Pugachev
conditionally-optimal filters and extra-\linebreak
\\[-12pt]
\hspace*{23pt}polators for state linear stochastic systems&2&14--23\\[.23pt]
\Avtors{Sinitsyn~I.\,N.\ and Sinitsyn~V.\,I.} Analytical modeling of
distributions in stochastic systems on\linebreak
\\[-12pt]
\hspace*{23pt}manifolds based on ellipsoidal approximation&1&45--55\\[.23pt]
\Avtors{Sinitsyn~I.\,N., Sinitsyn~V.\,I., and
Korepanov~E.\,R.} Ellipsoidal suboptimal filters for nonlinear\linebreak
\\[-12pt]
\hspace*{23pt}stochastic systems on manifolds&2&24--35\\[.23pt]
\Avtors{Sinitsyn~V.\,I.} see~Sinitsyn~I.\,N.&&\\[.23pt]
\Avtors{Sinitsyn~V.\,I.} see~Sinitsyn~I.\,N.&&\\[.23pt]
\Avtors{Skvortsov~N.\,A.} see~Stupnikov~S.\,A.&&\\[.23pt]
\Avtors{Sokolov~I.\,A.} see~Chertok~A.\,V.&&\\
\end{tabular}
}
\pagebreak

\def\leftfootline{\small{\textbf{\thepage}
\hfill INFORMATIKA I EE PRIMENENIYA~--- INFORMATICS AND APPLICATIONS\ \ \ 2016\
\ \ volume~10\ \ \ issue\ 4}
}%
 \def\rightfootline{\small{INFORMATIKA I EE PRIMENENIYA~---
INFORMATICS AND APPLICATIONS\ \ \ 2016\ \ \ volume~10\ \ \ issue\ 4
\hfill \textbf{\thepage}}}

\def\leftkol{2016 AUTHOR INDEX} % ENGLISH ABSTRACTS}

\def\rightkol{2016 AUTHOR INDEX} %ENGLISH ABSTRACTS}


{\tabcolsep=3pt
\begin{tabular}{p{382pt}cc}
&\textbf{Issue} & \textbf{Page}\\[6pt]
\Avtors{Sopin~E.\,S.} see~Gaidamaka~Yu.\,V.&&\\
\Avtors{Strijov~V.\,V.} see~Goncharov~A.\,V.&&\\
\Avtors{Strijov~V.\,V.} see~Isachenko~R.\,V.&&\\
\Avtors{Strijov~V.\,V.} see~Karasikov~M.\,E.&&\\
\Avtors{Stupnikov~S.\,A., Briukhov~D.\,O., and Skvortsov~N.\,A.}
Co-lending systemic risk analysis over\linebreak
\\[-12pt]
\hspace*{23pt}heterogeneous data collections&1&23--33\\
\Avtors{Stupnikov~S.\,A.} see~Kalinichenko~L.\,A.&&\\
\Avtors{Suchkov~A.\,P.} see~Zatsarinny~A.\,A.&&\\
\Avtors{Timonina~E.\,E.} see~Grusho~A.\,A.&&\\
\Avtors{Titova~A.\,I.} see~Kudryavtsev~A.\,A.&&\\
\Avtors{Turlikov~A.\,M.} see~Ometov~A.\,Ya.&&\\
\Avtors{Tyrsin~A.\,N.\ and Serebryanskii~S.\,M.} Recognition of
dependences on the basis of inverse\linebreak
\\[-12pt]
\hspace*{23pt}mapping&2&58--64\\
\Avtors{Ulyanov~V.\,V.} see~Markov~A.\,S.&&\\
\Avtors{Ushakov~V.\,G.} Queueing system with working vacations and
hyperexponential input stream&2&92--97\\
\Avtors{Ushakov~V.\,G.} see~Leontyev~N.\,D.&&\\
\Avtors{Volnova~A.\,A.} see~Kalinichenko~L.\,A.&&\\
\Avtors{Yakovlev~O.\,A.\ and Gasilov~A.\,V.} Speeded-up stereo
matching using geodesic support weights&3&\hphantom{1}98--104\\
\Avtors{Zabezhailo~M.\,I.} see~Grusho~A.\,A.&&\\
\Avtors{Zabezhailo~M.\,I.} see~Grusho~A.\,A.&&\\
\Avtors{Zakharova~T.\,V.\ and Shestakov~O.\,V.} Precision analysis of
wavelet processing of aerodynamic\linebreak
\\[-12pt]
\hspace*{23pt}flow patterns&3&46--54\\
\Avtors{Zalizniak~Anna~A.\ and Kruzhkov~M.\,G.} Database
of~Russian impersonal verbal constructions&4&132--141\\
\Avtors{Zasypko~V.\,V.} see~Shnurkov~P.\,V.&&\\
\Avtors{Zatsarinny~A.\,A.\ and Suchkov~A.\,P.} Systems engineering
approaches to~the~establishment of\linebreak
\\[-12pt]
\hspace*{23pt}a~system for~decision support based
on~situational analysis&4&105--113\\
\Avtors{Zatsarinny~A.\,A.} see~Grusho~A.\,A.&&\\
\Avtors{Zatsman~I.\,M., Inkova~O.\,Yu., Kruzhkov~M.\,G., and
Popkova~N.\,A.} Representation of cross-\linebreak
\\[-12pt]
\hspace*{23pt}lingual knowledge about
connectors in supracorpora databases&1&106--118\\
\Avtors{Zatsman~I.\,M.} see~Minin~V.\,A.&&\\
\Avtors{Zeifman~A.\,I.} see~Korolev~V.\,Yu.&&\\
\Avtors{Zeifman~A.\,I.} see~Korolev~V.\,Yu.&&\\
\end{tabular}
}

%\thispagestyle{myheadings}
\def\leftfootline{\small{\textbf{\thepage}
\hfill INFORMATIKA I EE PRIMENENIYA~--- INFORMATICS AND APPLICATIONS\ \ \ 2016\
\ \ volume~10\ \ \ issue\ 4}
}%
 \def\rightfootline{\small{INFORMATIKA I EE PRIMENENIYA~---
INFORMATICS AND APPLICATIONS\ \ \ 2016\ \ \ volume~10\ \ \ issue\ 4
\hfill \textbf{\thepage}}}

 \label{end\stat}

\newpage

%\def\stat{rekl}
%\label{preobr}

%\def\tit{АКАДЕМИК ПУГАЧЁВ  ВЛАДИМИР СЕМЁНОВИЧ\\
%25.03.1911--25.03.1998}


%   \vspace*{-48pt}
%   \begin{center}\LARGE
%Академик Пугачёв  Владимир Семёнович\\ (25.03.1911--25.03.1998)
%   \end{center}
   
   %\vspace*{2.5mm}
   
   \begin{center}

{\prgsh\LARGE
ОБЪЯВЛЕНИЯ О КОНФЕРЕНЦИЯХ}

\end{center}
%\hrule

\vspace*{6pt}

   
   \vspace*{10mm}
   
   \thispagestyle{empty}

\noindent
\begin{tabular}{cc}
%\begin{center}
\multicolumn{1}{c}{\raisebox{-40pt}[0pt][0pt]{\mbox{%
\epsfxsize=33mm
\epsfbox{vspu.eps}
}}}
%\end{center}
&
\tabcolsep=0pt\begin{tabular}{c}
{\prg{\Large\textbf{XII Всероссийское совещание}}}\\[6pt]
{\prg{\Large\textbf{по проблемам управления}}}\\[12pt]
{\prg{\large 16--19 июня 2014~г.}}\\[6pt] 
{\prg{\large Институт проблем управления имени В.\,А.~Трапезникова РАН}}\\[6pt]
{\prg{\large Москва, Россия}}
\end{tabular}
\end{tabular}

\vspace*{60pt}

     
 { %\large    
 XII Всероссийское совещание по проблемам управления (ВСПУ XII), посвященное 75-летию 
Института проблем управления (ИПУ) имени В.\,А.~Трапезникова РАН, проводится 16--19~июня 
2014~г.\ 
в ИПУ РАН (г.~Москва, Россия). ВСПУ XII организуется ИПУ РАН при поддержке РФФИ, Отделения 
энергетики, машиностроения, механики и процессов управления Российской академии наук, 
Российского 
национального комитета по автоматическому управлению, Академии навигации и управ\-ле\-ния 
движением, 
Научного совета РАН по комплексным проблемам управления и автоматизации, Совета по 
мехатронике и робототехнике РАН. Официальный язык Совещания~--- русский.

\vspace*{24pt}
     
     \textbf{Направления работы}
     \begin{enumerate}[1.]
\item Теория систем управления
\item Управление подвижными объектами и навигация
\item Интеллектуальные системы управления
\item Управление в промышленности, транспортом и логистикой
\item Управление системами междисциплинарной природы
\item Средства измерения, вычислений и контроля в управлении
\item Системный анализ и принятие решений в задачах управления
\item Информационные технологии в управлении
\item Проблемы образования в области управления: современное содержание и технологии обучения
\end{enumerate}

\vspace*{24pt}

     Подробная информация о Совещании находится на сайте {\sf http://vspu2014.ipu.ru}. Срок 
окончательной подачи докладов через систему подачи докладов на сайте~--- \textbf{30~ноября} 
2013~г.
}

%\include{rekl-1}

%\end{document}

%   \vspace*{-48pt}

\begin{center}
\vspace*{6pt}
\mbox{%
\epsfxsize=53.502mm
\epsfbox{foto-1.eps}
}
\end{center}

\vspace*{6pt} %Академик


   \begin{center}
\fbox{\Large\textbf{Профессор Игорь Алексеевич Ушаков}}\\[12pt]
\textbf{\large 22.01.1935--27.02.2015}
   \end{center}


   %\vspace*{2.5mm}

   \vspace*{5mm}

   \thispagestyle{empty}

%\

%\vspace*{-12pt}


Редакционный совет и редакционная коллегия журнала <<Информатика и~её применения>> с~глубоким прискорбием извещают, что 27~февраля 2015~г.\ после тяжелой
и~продолжительной болезни скончался Игорь Алексеевич Ушаков~--- доктор технических наук, профессор, член редколлегии журнала <<Информатика и ее применения>>.

Игорь Алексеевич Ушаков окончил Московский авиационный институт, в~1963~г.\ защитил кандидатскую, а~в~1968~г.~--- докторскую диссертацию. С~1958 по 1989~гг.\ работал в~ряде научно-исследовательских организаций СССР, в~том числе руководил отделами в~НИИ АА и~ВЦ АН СССР; с 1969 по 1989 гг. преподавал в~МФТИ (был профессором, а~затем заведующим кафедрой) и~в~МЭИ. С~1989~г.~---- в~США: являлся профессором университета Дж.\ Вашингтона, университета Дж.\ Мэйсона и~Калифорнийского университета, сотрудником компаний MCI, Qualcomm и Hughes.

И.\,А.~Ушаков с момента основания журнала <<Надежность и~контроль качества>> был заместителем ответственного редактора, а~затем на протяжении многих лет членом редколлегии. В~2006~г.\ основал электронный международный журнал ``Reliability: Theory \& Application'', главным редактором которого оставался до конца жизни.

Учебниками и справочниками по теории надежности, написанными И.\,А.~Ушаковым, пользовались и~пользуются несколько поколений ученых и~специалистов в~разных странах мира.

Игорь Алексеевич всегда уделял огромное внимание работе с~молодежью; более~50 его учеников защитили докторские и~кандидатские диссертации.

И.\,А.~Ушаков вел активную научно-про\-све\-ти\-тель\-скую деятельность. В~частности, он был одним из организаторов и~руководителей Московского кабинета качества и~надежности при Политехническом музее (целью этого Кабинета было оказание консультаций работникам промышленных предприятий и~чтение курсов лекций для инженеров, занимающихся проблемой надежности). Находясь в~США, И.\,А.~Ушаков создал международный ин\-тер\-нет-фо\-рум им.\ Б.\,В.~Гнеденко, объединивший около~400~видных специалистов по приложениям теории вероятностей и~математической статистики, преимущественно в~об\-ласти теории надежности и~анализа риска, из десятков стран мира; коллективным членов этого Форума является и~наш журнал. Цели Форума~--- содействие контактам между специалистами из разных стран, организация обмена профессиональными 
новостями и~информацией (новые публикации, предстоящие события и~др.). Также необходимо отметить большое число на\-уч\-но-по\-пу\-ляр\-ных работ, опубликованных И.\,А.~Ушаковым.

И.\,А.~Ушаков обладал большим личным обаянием, имел широкий круг интересов. Все знавшие И.\,А.~Ушакова всегда будут помнить его как замечательного ученого и~прекрасного человека.

\bigskip

Редакционный совет и редакционная коллегия журнала <<Информатика и~её применения>> 
выражают глубокие соболезнования родным и близким покойного, всем, кто его знал и~работал с~ним.



%\end{document}

%\include{IPPM-25}

\def\stat{cont-rus}
{%\hrule\par
%\vskip 7pt % 7pt
\vspace*{-24pt}
\raggedleft\Large \bf%\baselineskip=3.2ex
Правила подготовки рукописей  для публикации в журнале
<<Информатика~и~её~применения>> \vskip 8pt
    \hrule
    \par
\vskip 14pt plus 6pt minus 3pt }

\label{st\stat}

\def\tit{\ }

\def\aut{\ }
\def\auf{\ }

\def\leftkol{\ }
% Правила подготовки рукописей  для публикации в журнале
%<<Информатика и её применения>>

\def\rightkol{\ }
%Правила подготовки рукописей  для публикации в журнале
%<<Информатика и её применения>>}


\titele{\tit}{\aut}{\auf}{\leftkol}{\rightkol}


\vspace*{-60pt}
{ %\small

Журнал <<Информатика и её применения>>
публикует теоретические, обзорные и дискуссионные статьи,
посвященные научным исследованиям и разработкам в области
информатики и ее приложений.

Журнал издается на русском языке. По специальному решению
редколлегии отдельные статьи могут печататься на английском языке.

Тематика журнала охватывает следующие направления:
\begin{itemize}
\item теоретические основы информатики;\\[-15pt]
      \item
математические методы исследования сложных систем и процессов;\\[-15pt]
           \item
информационные системы и сети;\\[-15pt]
                \item
информационные технологии;\\[-15pt]
                     \item
архитектура и программное обеспечение вычислительных комплексов и сетей.\\[-15pt]
\end{itemize}


\noindent
\begin{enumerate}[1.]
\item В журнале печатаются статьи, содержащие результаты, ранее не опубликованные и
не предназначенные к одновременной публикации в других изданиях.

%Публикация не должна нарушать закон об авторских правах.
Публикация предоставленной автором(ами) рукописи не должна нарушать 
положений глав~69, 70 раздела~VII части~IV Гражданского кодекса, 
которые определяют права на результаты интеллектуальной деятельности 
и~средства индивидуализации, в~том числе авторские права, в~РФ.

Ответственность за нарушение авторских прав, в~случае предъявления претензий к~редакции журнала,  
несут авторы статей.



Направляя рукопись в редакцию, авторы сохраняют свои права на данную
рукопись и при этом передают учредителям и редколлегии журнала неисключительные права на
издание статьи на русском языке 
(или на языке статьи, если он отличен от рус\-ско\-го) и~на перевод ее на английский
язык, а~также на
ее распространение в России и за рубежом. 
Каждый автор должен представить в~редакцию подписанный 
с~его стороны <<Лицензионный договор о~передаче неисключительных прав 
на использование произведения>>, текст которого размещен по адресу 
{\sf http://www.ipiran.ru/publications/licence.doc}. 
Этот договор может быть пред\-став\-лен в~бумажном (в~2-х экз.)\ 
или в~электронном виде (отсканированная копия заполненного и~подписанного документа).




Редколлегия вправе запросить у авторов экспертное заключение о возможности
пуб\-ли\-ка\-ции пред\-став\-лен\-ной статьи в открытой печати.\\[-13.5pt]

\item К статье прилагаются данные автора (авторов) (см.\ п.~8). При наличии нескольких
авторов указывается фамилия автора, ответственного за переписку с редакцией.\\[-13.5pt]

\item Редакция журнала осуществляет экспертизу присланных статей в соответствии с
принятой в журнале процедурой рецензирования.

Возвращение рукописи на доработку не означает ее принятия к печати.

Доработанный вариант с ответом на замечания рецензента необходимо прислать в
редакцию.\\[-13.5pt]

\item Решение редколлегии о публикации статьи или ее отклонении сообщается авторам.

Редколлегия может также направить авторам текст рецензии на их статью. Дискуссия по
поводу отклоненных статей не ведется.\\[-13.5pt]

%\pagebreak

\item Редактура статей высылается авторам для просмотра. Замечания к редактуре должны
быть присланы авторами в кратчайшие сроки.\\[-13.5pt]

\item Рукопись предоставляется в электронном виде в форматах MS WORD (.doc или
.docx) или \LaTeX\  (.tex), дополнительно~--- в формате .pdf, на дискете, лазерном диске
или электронной почтой. Предоставление бумажной рукописи необязательно.\\[-13.5pt]

\item При подготовке рукописи в MS Word рекомендуется использовать следующие
настройки.

Параметры страницы:
формат~--- А4; ориентация~--- книжная; поля (см): внутри~--- 2,5, снаружи~--- 1,5,
сверху~--- 2, снизу~--- 2, от края до нижнего колонтитула~--- 1,3.

Основной текст: стиль~--- <<Обычный>>, шрифт~--- Times New Roman, размер~---
14~пунк\-тов, абзацный отступ~--- 0,5~см, 1,5~интервала, выравнивание~--- по ширине.

\pagebreak

\def\leftkol{Правила подготовки рукописей  для публикации в журнале
<<Информатика и её применения>>}

\def\rightkol{Правила подготовки рукописей  для публикации в журнале
<<Информатика и её применения>>}



Рекомендуемый объем рукописи~--- не свыше 10~страниц указанного формата.
При превышении указанного объема редколлегия вправе потребовать от 
автора сокращения объема рукописи.


Сокращения слов, помимо стандартных, не допускаются. Допускается минимальное
количество аббревиатур.


Все страницы рукописи нумеруются.

Шаблоны оформления представлены в интернете:

\noindent
 {\sf
http://www.ipiran.ru/journal/template\_iiep\_ssi\_2024.zip}\\[-14pt]

\item Статья должна содержать следующую информацию на {\bfseries\textit{русском и
английском языках}}:\\[-16pt]

\begin{itemize}
\item название статьи;\\[-15pt]
\item Ф.И.О.\ авторов, на английском можно только имя и фамилию;\\[-15pt]
\item место работы, с указанием почтового адреса организации и электронного адреса каждого
автора;\\[-15pt]
\item сведения об авторах, в соответствии с форматом, образцы которого
представлены на страницах:



\def\leftfootline{\small{\textbf{\thepage}
\hfill ИНФОРМАТИКА И ЕЁ ПРИМЕНЕНИЯ\ \ \ том\ 18\ \ \ выпуск\ 3\ \ \ 2024}
}%
 \def\rightfootline{\small{ИНФОРМАТИКА И ЕЁ ПРИМЕНЕНИЯ\ \ \ том\ 18\ \ \ выпуск\ 3\ \ \ 2024
\hfill \textbf{\thepage}}}



{\sf http://www.ipiran.ru/journal/issues/2013\_07\_01/authors.asp} и

{\sf http://www.ipiran.ru/journal/issues/2013\_07\_01\_eng/authors.asp};
\item аннотация (не менее 100~слов на каждом из языков). Аннотация~--- это краткое
резюме работы, которое может публиковаться отдельно. Она является основным
источником информации в~ин\-фор\-ма\-ци\-он\-ных системах и базах данных. Английская
аннотация должна быть оригинальной, может не быть дословным переводом русского
текста и должна быть написана хорошим английским языком. В~аннотации не должно
быть ссылок на литературу и, по возможности, формул;\\[-15pt]
\item ключевые слова~--- желательно из принятых в мировой
на\-уч\-но-тех\-ни\-че\-ской литературе тематических тезаурусов. Предложения не
могут быть ключевыми словами;\\[-15pt]
\item источники финансирования работы (ссылки на гранты, проекты,
поддерживающие организации и~т.\,п.).
\end{itemize}



%\pagebreak

\item  Требования к спискам литературы.\\[-14pt]

Ссылки на литературу в тексте статьи нумеруются (в квадратных скобках) и
располагаются в каждом из списков литературы в порядке  первых упоминаний. Если источник имеет DOI и/или EDN,
то их необходимо указывать.

Списки литературы представляются в двух вариантах:\\[-14pt]


\noindent
\begin{enumerate}[(1)]
\item \textbf{Список литературы к русскоязычной части}. Русские и английские
работы~---  на языке и в алфавите оригинала;\\[-14.5pt]
\item  \textbf{References}. Русские работы и работы на других языках~--- в латинской
транслитерации с переводом на английский язык; английские работы и работы на других
языках~--- на языке оригинала.
\end{enumerate}

Необходимо для составления списка ``References'' пользоваться размещенной на сайте
{\sf http://www. translit.net/ru/bgn/} бесплатной программой транслитерации русского
 текста в~латиницу. %, при этом в~за\-клад\-ке <<варианты\ldots>> следует выбратьопцию BGN.

Список литературы ``References'' приводится полностью отдельным блоком, повторяя все
позиции из списка литературы к русскоязычной части, независимо от того, имеются или
нет в нем иностранные источники. Если в списке литературы к русскоязычной части есть
ссылки на иностранные публикации, набранные латиницей, они полностью повторяются в
списке ``References''.

Ниже приведены примеры ссылок на различные виды публикаций в списке ``References''.

\def\leftfootline{\small{\textbf{\thepage}
\hfill ИНФОРМАТИКА И ЕЁ ПРИМЕНЕНИЯ\ \ \ том\ 18\ \ \ выпуск\ 3\ \ \ 2024}
}%
 \def\rightfootline{\small{ИНФОРМАТИКА И ЕЁ ПРИМЕНЕНИЯ\ \ \ том\ 18\ \ \ выпуск\ 3\ \ \ 2024
\hfill \textbf{\thepage}}}

{\small

\noindent
\textbf{Описание статьи из журнала:}

\Aue{Zagurenko, A.\,G., V.\,A.~Korotovskikh, A.\,A.~Kolesnikov, A.\,V.~Timonov, and D.\,V.~Kardymon}. 2008.
Tekhniko-ekonomicheskaya optimizatsiya dizayna gidrorazryva plasta [Technical and
economic optimization of the design
of hydraulic fracturing]. \textit{Neftyanoe hozyaystvo} [\textit{Oil Industry}] 11:54--57.

\Aue{Zhang, Z., and D.~Zhu}. 2008. Experimental research on the localized
electrochemical micromachining. \textit{Russ. J.~Electrochem.}  44(8):926--930.
{\sf doi:10.1134/S1023193508080077}.

\noindent
\textbf{Описание статьи из электронного журнала:}

\Aue{Swaminathan, V., E.~Lepkoswka-White, and B.\,P.~Rao}. 1999. Browsers or buyers in cyberspace? An
investigation of electronic factors influencing electronic exchange. \textit{JCMC}
5(2). Available at: {\sf http://www.ascusc.org/jcmc/vol5/issue2/} (accessed April~28, 2011).

\def\leftkol{Правила подготовки рукописей  для публикации в журнале
<<Информатика и её применения>>}

\def\rightkol{Правила подготовки рукописей  для публикации в журнале
<<Информатика и её применения>>}


\noindent
\textbf{Описание статьи из продолжающегося издания (сборника трудов):}

\Aue{Astakhov, M.\,V., and T.\,V.~Tagantsev}. 2006. Eksperimental'noe
issledovanie prochnosti soedineniy ``stal'--kompozit'' [Experimental study of
the strength of joints ``steel--composite'']. \textit{Trudy MGTU
``Matematicheskoe modelirovanie slozhnykh tekh\-ni\-che\-skikh sistem''}
[\textit{Bauman MSTU ``Mathematical Modeling of Complex Technical
Systems'' Proceedings}]. 593:125--130.


\pagebreak



\noindent
\textbf{Описание материалов конференций:}

\Aue{Usmanov, T.\,S., A.\,A.~Gusmanov, I.\,Z.~Mullagalin, R.\,Ju.~Muhametshina, A.\,N.~Chervyakova, and
A.\,V.~Sveshnikov}. 2007. Osobennosti proektirovaniya razrabotki mestorozhdeniy
s primeneniem gidrorazryva
plasta [Features of the design of field development with the use of hydraulic fracturing].
\textit{Trudy 6-go
Mezhdu\-na\-rod\-no\-go Simpoziuma ``Novye resursosberegayushchie tekhnologii nedropol'zovaniya i povysheniya
neftegazootdachi''} [\textit{6th  Symposium (International) ``New Energy Saving Subsoil Technologies and
the Increasing of the Oil and Gas Impact'' Proceedings}]. Moscow. 267--272.



\def\leftfootline{\small{\textbf{\thepage}
\hfill ИНФОРМАТИКА И ЕЁ ПРИМЕНЕНИЯ\ \ \ том\ 18\ \ \ выпуск\ 3\ \ \ 2024}
}%
 \def\rightfootline{\small{ИНФОРМАТИКА И ЕЁ ПРИМЕНЕНИЯ\ \ \ том\ 18\ \ \ выпуск\ 3\ \ \ 2024
\hfill \textbf{\thepage}}}



\noindent
\textbf{Описание книги (монографии, сборники):}



Lindorf, L.\,S., and L.\,G.~Mamikoniants, eds. 1972.
\textit{Ekspluatatsiya turbogeneratorov s neposredstvennym
okhlazhdeniem} [\textit{Operation of turbine generators with direct cooling}].
Moscow: Energy Publs. 352~p.


\Aue{Latyshev, V.\,N.} 2009. \textit{Tribologiya rezaniya. Kn.~1: Friktsionnye protsessy
pri rezanii metallov}
[\textit{Tribology of cutting. Vol.~1: Frictional processes in metal cutting}]. Ivanovo: Ivanovskii
State Univ. 108~p.

\def\leftkol{Правила подготовки рукописей  для публикации в журнале
<<Информатика и её применения>>}

\def\rightkol{Правила подготовки рукописей  для публикации в журнале
<<Информатика и её применения>>}

\noindent
\textbf{Описание переводной книги}
(в списке литературы к русскоязычной части необходимо указать:~/ Пер.\ с англ.~---
после названия книги, а в конце ссылки указать оригинал книги в круглых скобках):
\begin{enumerate}[1.]
\item  В русскоязычной части:

\def\leftfootline{\small{\textbf{\thepage}
\hfill ИНФОРМАТИКА И ЕЁ ПРИМЕНЕНИЯ\ \ \ том\ 18\ \ \ выпуск\ 3\ \ \ 2024}
}%
 \def\rightfootline{\small{ИНФОРМАТИКА И ЕЁ ПРИМЕНЕНИЯ\ \ \ том\ 18\ \ \ выпуск\ 3\ \ \ 2024
\hfill \textbf{\thepage}}}

\Au{Тимошенко С.\,П., Янг Д.\,Х., Уивер~У.}
Колебания в инженерном деле~/ Пер.\ с англ.~--- М.: Машиностроение, 1985. 472~с.
(\Au{Timoshenko~S.\,P., Young~D.\,H., Weaver~W.}
Vibration problems in engineering.~--- 4th ed.~--- New York, NY, USA: Wiley, 1974. 521~p.)\\[-13.5pt]
\item  В англоязычной части:

\Aue{Timoshenko, S.\,P., D.\,H.~Young, and W.~Weaver}.
1974. \textit{Vibration problems in engineering}. 4th ed. New York: 
Wiley. 521~p.
\end{enumerate}

\vspace*{-3pt}


\noindent
\textbf{Описание неопубликованного документа:}


\Aue{Latypov, A.\,R., M.\,M.~Khasanov, and V.\,A.~Baikov}.
2004 (unpubl.). Geologiya i~dobycha (NGT GiD) [Geology and production (NGT GiD)]. Certificate on official registration of the computer program
No.\,2004611198. 

\noindent
\textbf{Описание интернет-ресурса:}


Pravila tsitirovaniya istochnikov [Rules for the citing of sources]. Available at: {\sf
http://www.scribd.com/doc/1034528/} (accessed February~7, 2011).

%\pagebreak

\noindent
\textbf{Описание диссертации или автореферата диссертации:}

\Aue{Semenov, V.\,I.}
2003. Matematicheskoe modelirovanie plazmy v sisteme kompaktnyy tor [Mathematical
modeling of the plasma in the compact torus].  Moscow.  D.Sc.\ Diss. 272~p.

\Aue{Kozhunova, O.\,S.} 2009. Tekhnologiya razrabotki semanticheskogo
slovarya informatsionnogo monitoringa [Technology of development of
semantic dictionary of information monitoring system].  Moscow: IPI RAN. PhD Thesis. 23~p.


\noindent
\textbf{Описание ГОСТа:}

GOST 8.586.5-2005. 2007. Metodika vypolneniya izmereniy. Izmerenie raskhoda i~kolichestva zhidkostey i~gazov
s~pomoshch'yu standartnykh suzhayushchikh ustroystv [Method of measurement.
Measurement of flow rate and volume of liquids and gases by means of orifice devices]. Moscow:
Standardinform  Publs. 10~p.

\noindent
\textbf{Описание патента:}

\Aue{Bolshakov, M.\,V., A.\,V.~Kulakov, A.\,N.~Lavrenov, and M.\,V.~Palkin}.
2006. Sposob orientirovaniya po krenu letatel'nogo
apparata s opti\-che\-skoy golovkoy
samonavedeniya [The way to orient on the roll of aircraft with optical homing head].
Patent RF No.\,2280590.
}

\item Присланные в редакцию материалы авторам не возвращаются.\\[-13.5pt]

\item При отправке файлов по электронной почте просим придерживаться следующих
правил:
\begin{itemize}
\item указывать в поле subject (тема) название журнала и фамилию автора;\\[-13.5pt]
\item указывать в тексте письма название статьи, авторов и~журнал, в~который направляется статья;\\[-13.5pt]
\item использовать attach (присоединение);\\[-13.5pt]
\item в состав электронной версии статьи должны входить: файл, содержащий текст
статьи, и файл(ы), содержащий(е) иллюстрации.\\[-13.5pt]
\end{itemize}

\item Журнал <<Информатика и её применения>> является некоммерческим изданием.
Плата за публикацию не взимается, гонорар авторам не выплачивается.
\end{enumerate}



\def\leftfootline{\small{\textbf{\thepage}
\hfill ИНФОРМАТИКА И ЕЁ ПРИМЕНЕНИЯ\ \ \ том\ 18\ \ \ выпуск\ 3\ \ \ 2024}
}%
 \def\rightfootline{\small{ИНФОРМАТИКА И ЕЁ ПРИМЕНЕНИЯ\ \ \ том\ 18\ \ \ выпуск\ 3\ \ \ 2024
\hfill \textbf{\thepage}}}


\vspace*{-1mm}

\begin{center}

\textbf{Адрес редакции журнала <<Информатика и её применения>>:} \\




Москва 119333, ул.~Вавилова, д.~44, корп.~2, ФИЦ ИУ РАН\\[-10pt]

\

Тел.: +7\,(499)\,135-86-92\ \ Факс:  +7\,(495)\,930-45-05\\[-10pt]

 \

e-mail:   {\sf iiep@frccsc.ru} (Стригина Светлана Николаевна)\\[-10pt]

\

{\sf http://www.ipiran.ru/journal/issues/}
\end{center}
}


\def\leftkol{Правила подготовки рукописей  для публикации в журнале
<<Информатика и её применения>>}

\def\rightkol{Правила подготовки рукописей  для публикации в журнале
<<Информатика и её применения>>}


\def\leftfootline{\small{\textbf{\thepage}
\hfill ИНФОРМАТИКА И ЕЁ ПРИМЕНЕНИЯ\ \ \ том\ 18\ \ \ выпуск\ 3\ \ \ 2024}
}%
 \def\rightfootline{\small{ИНФОРМАТИКА И ЕЁ ПРИМЕНЕНИЯ\ \ \ том\ 18\ \ \ выпуск\ 3\ \ \ 2024
\hfill \textbf{\thepage}}} 
\def\stat{podg-e}
{%\hrule\par
%\vskip 7pt % 7pt
\vspace*{-24pt}
\raggedleft\Large \bf%\baselineskip=3.2ex
Requirements for manuscripts submitted to Journal
``Informatics~and~Applications'' \vskip 8pt
    \hrule
    \par
\vskip 21pt plus 6pt minus 3pt }

\label{st\stat}

\def\tit{\ }

\def\aut{\ }
\def\auf{\ }

\def\leftkol{\ }

\def\rightkol{\ }
%Requirements for manuscripts submitted to Journal
%``Informatics~and~Applications''}

\titele{\tit}{\aut}{\auf}{\leftkol}{\rightkol}

\def\leftfootline{\small{\textbf{\thepage}
\hfill INFORMATIKA I EE PRIMENENIYA~--- INFORMATICS AND APPLICATIONS\ \ \ 2019\
\ \ volume~13\ \ \ issue\ 4}
}%
 \def\rightfootline{\small{INFORMATIKA I EE PRIMENENIYA~--- INFORMATICS AND APPLICATIONS\ \ \ 2019\ \ \ volume~13\ \ \ issue\ 4
\hfill \textbf{\thepage}}}

\vspace*{-60pt}

{\small

\noindent
Journal ``Informatics and Applications'' (Inform.\ Appl.)
publishes theoretical, review, and discussion
articles on the research and development in the
field of informatics and its applications.

The journal is published in Russian.
By a special decision of the editorial
board, some articles can be published in English.


The topics covered include the following areas:
\begin{itemize}
               \item
     theoretical fundamentals of informatics; \\[-14pt]
\item
mathematical methods for studying complex systems and processes; \\[-14pt]
\item
information systems and networks;\\[-14pt]
\item
information technologies; and \\[-14pt]
\item
architecture and software of computational complexes and networks. \\[-14pt]
\end{itemize}

\noindent
\begin{enumerate}[1.]
\item The Journal publishes original articles which have not been published before and are not
intended for simultaneous publication in other editions. An article submitted to the Journal must not violate the
Copyright law. Sending the manuscript to the Editorial Board, the authors retain all rights of the
owners of the manuscript and transfer the nonexclusive rights to publish the article in Russian
(or the language of the article, if not Russian) and its distribution in Russia and abroad to the
Founders and the Editorial Board. Authors should submit a letter to the Editorial Board in the
following form:

{\bfseries\textit{Agreement on the transfer of rights to publish:}}

``\textit{We, the undersigned authors of the manuscript ``\ldots'', pass to the
Founder and the Editorial Board of the Journal ``Informatics and Applications''
the nonexclusive right to publish the manuscript of the article in Russian (or
in English) in both print and electronic versions of the Journal. We affirm
that this publication does not violate the Copyright of other persons or
organizations.}

\textit{Author(s) signature(s): (name(s), address(es), date).}

This agreement should be submitted in paper form or in the form of a scanned copy (signed by
the authors).


%The Editorial Board has the right to request from the authors an official expert conclusion that
%the submitted article has no secret data prohibited for publication. \\[-13.5pt]
\item
A submitted article should be attached with \textbf{the data on the author(s)} (see item~8). If
there are several authors, the contact person should be indicated who is responsible for
correspondence with the Editorial Board and other authors about revisions and final approval
of the proofs.\\[-13.5pt]

\item The Editorial Board of the Journal examines the article according to the established
reviewing procedure. If the authors receive their article for correction after reviewing, it does not
mean that the article is approved for publication. The corrected article should be sent to the
Editorial Board for the subsequent review and approval.\\[-13.5pt]

\item The decision on the article publication or its rejection is communicated to the authors. The
Editorial Board may also send the reviews on the submitted articles to the authors. Any
discussion upon the rejected articles is not possible.\\[-13.5pt]

\item The edited articles will be sent to the authors for proofread. The comments of the authors
to the edited text of the article should be sent to the Editorial Board as soon as possible.\\[-13.5pt]

\item The manuscript of the article should be presented electronically in the MS WORD (.doc or
.docx) or \LaTeX\ (.tex) formats, and additionally in the .pdf format. All documents
 may be sent
by e-mail or provided on a CD or diskette. A~hard copy submission is not necessary.\\[-13.5pt]

\item The recommended typesetting instructions for manuscript.

Pages parameters: format A4, portrait orientation, document margins (cm): left~--- 2.5, right~---
1.5, above~--- 2.0, below~--- 2.0, footer 1.3.

Text: font~---Times New Roman, font size~--- 14, paragraph indent~--- 0.5, line spacing~--- 1.5,
justified alignment.

The recommended manuscript size: not more than 15~pages of the specified format.
If the specified size exceeded, the editorial board is entitled to require the author
to reduce the manuscript.

Use only standard abbreviations. Avoid  abbreviations in the title and
abstract. The full term for which an abbreviation stands should precede
its first use in the text unless it is a standard unit of measurement.

All pages of the manuscript should be numbered.

The templates for the manuscript typesetting are presented on site: {\sf
http://www.ipiran.ru/journal/template.doc}.\\[-13.5pt]


%\def\leftkol{Requirements for manuscripts submitted to Journal
%``Informatics~and~Applications''}

\item The articles should enclose data both in \textbf{Russian and English}:
\begin{itemize}
\item title;\\[-13.5pt]
\item author's name and surname;\\[-13.5pt]
\item affiliation~--- organization, its address with ZIP code, city, country, and
official e-mail address;\\[-13.5pt]
\item data on authors according to the format: (see site)

{\sf http://www.ipiran.ru/journal/issues/2013\_07\_01/authors.asp}  and

{\sf  http://www.ipiran.ru/journal/issues/2013\_07\_01\_eng/authors.asp};\\[-13.5pt]

\pagebreak

\def\leftfootline{\small{\textbf{\thepage}
\hfill INFORMATIKA I EE PRIMENENIYA~--- INFORMATICS AND APPLICATIONS\ \ \ 2019\
\ \ volume~13\ \ \ issue\ 4}
}%
 \def\rightfootline{\small{INFORMATIKA I EE PRIMENENIYA~--- INFORMATICS AND APPLICATIONS\ \ \ 2019\ \ \ volume~13\ \ \ issue\ 4
\hfill \textbf{\thepage}}}


%\def\leftkol{Requirements for manuscripts submitted to Journal
%``Informatics~and~Applications''}

%\def\rightkol{Requirements for manuscripts submitted to Journal
%``Informatics~and~Applications''}



\item abstract (not less than 100 words) both in Russian and in English. Abstract is a short
summary of the article that can be published separately. The abstract is the
main source of information on the article and it could be included in leading information
systems and data bases. The abstract in English has to be an original text and should
not be an exact translation of the Russian one. Good English is required.
In abstracts, avoid references and formulae;\\[-13.5pt]
\item indexing is performed on the basis of keywords. The use of keywords from the
internationally accepted thematic Thesauri is recommended.

%\def\leftkol{Requirements for manuscripts submitted to Journal
%``Informatics~and~Applications''}

%\def\rightkol{Requirements for manuscripts submitted to Journal
%``Informatics~and~Applications''}

Important! Keywords must not be sentences;
\item Acknowledgments.
\end{itemize}

\item References. Russian references have to be presented both in English translation and Latin
transliteration (refer {\sf http://www.translit.net/ru/bgn/}).

Please take into account the following examples of Russian references appearance:

\noindent
\textbf{Article in journal:}

\Aue{Zhang, Z., and D.~Zhu}. 2008. Experimental research on the localized electrochemical
micromachining.
\textit{Rus. J.~Electrochem.}  44(8):926--930. {\sf doi:10.1134/S1023193508080077}.


\noindent
\textbf{Journal article in electronic format:}

\Aue{Swaminathan, V., E.~Lepkoswka-White, and B.\,P.~Rao}. 1999. Browsers or buyers in
cyberspace? An
investigation of electronic factors influencing electronic exchange. \textit{JCMC}
5(2). Available at: {\sf http://www.ascusc.org/jcmc/vol5/issue2/} (accessed April~28, 2011).




\noindent
\textbf{Article from the continuing publication (collection of works, proceedings):}

\Aue{Astakhov, M.\,V., and T.\,V.~Tagantsev}. 2006. Eksperimental'noe
issledovanie prochnosti soedineniy ``stal'--kompozit'' [Experimental study of
the strength of joints ``steel--composite'']. \textit{Trudy MGTU
``Matematicheskoe modelirovanie slozhnykh tekh\-ni\-che\-skikh sistem''}
[\textit{Bauman MSTU ``Mathematical Modeling of Complex Technical
Systems'' Proceedings}]. 593:125--130.

\def\leftfootline{\small{\textbf{\thepage}
\hfill INFORMATIKA I EE PRIMENENIYA~--- INFORMATICS AND APPLICATIONS\ \ \ 2019\
\ \ volume~13\ \ \ issue\ 4}
}%
 \def\rightfootline{\small{INFORMATIKA I EE PRIMENENIYA~--- INFORMATICS AND APPLICATIONS\ \ \ 2019\ \ \ volume~13\ \ \ issue\ 4
\hfill \textbf{\thepage}}}

\def\leftkol{Requirements for manuscripts submitted to Journal
``Informatics~and~Applications''}

\def\rightkol{Requirements for manuscripts submitted to Journal
``Informatics~and~Applications''}

\noindent
\textbf{Conference proceedings:}

\Aue{Usmanov, T.\,S., A.\,A.~Gusmanov, I.\,Z.~Mullagalin, R.\,Ju.~Muhametshina,
A.\,N.~Chervyakova, and
A.\,V.~Sveshnikov}. 2007. Osobennosti proektirovaniya razrabotki mestorozhdeniy
s primeneniem gidrorazryva
plasta [Features of the design of field development with the use of hydraulic fracturing].
\textit{Trudy 6-go
Mezhdu\-na\-rod\-no\-go Simpoziuma ``Novye resursosberegayushchie tekhnologii
nedropol'zovaniya i povysheniya
neftegazootdachi''} [\textit{6th  Symposium (International) ``New Energy Saving Subsoil
Technologies and
the Increasing of the Oil and Gas Impact'' Proceedings}]. Moscow. 267--272.


\noindent
\textbf{Books and other monographs:}




Lindorf, L.\,S., and L.\,G.~Mamikoniants, eds. 1972.
\textit{Ekspluatatsiya turbogeneratorov s neposredstvennym
okhlazhdeniem} [\textit{Operation of turbine generators with direct cooling}].
Moscow: Energy Publs. 352~p.


%\Aue{Latyshev, V.\,N.} 2009. \textit{Tribologiya rezaniya. Kn.~1: Frikcionnye prosessy
%pri rezanii metallov}
%[\textit{Tribology of cutting. Vol.~1: Frictional processes in metal cutting}]. Ivanovo: Ivanovskii
%State Univ. 108~p.


%\noindent
%\textbf{Unpublished material:}

%\Aue{Latypov, A.\,R., M.\,M.~Khasanov, and V.\,A.~Baikov}.
%2004. Geology and production (NGT GiD). Certificate on official registration of the computer
%program
%No.\,2004611198. (In Russian, unpubl.)

%\noindent
%\textbf{Internet-source:}

%APA Style. 2011. Available at: {\sf http://www.apastyle.org/apa-style-help.aspx} (accessed
%February~5, 2011).

%Pravila citirovaniya istochnikov [Rules for the citing of sources]. Available at: {\sf
%http://www.scribd.com/doc/1034528/} (accessed February~7, 2011).


\noindent
\textbf{Dissertation and Thesis:}

%\Aue{Semenov, V.\,I.}
%2003. Matematicheskoe modelirovanie plazmy v sisteme kompaktnyy tor. [Mathematical
%modeling of the plasma in the compact torus]. D.Sc.\ Diss. Moscow. 272~p.

\Aue{Kozhunova, O.\,S.} 2009. Tekhnologiya razrabotki semanticheskogo
slovarya informatsionnogo monitoringa [Technology of development of
semantic dictionary of information monitoring system]. PhD Thesis. Moscow: IPI RAN. 23~p.


\noindent
\textbf{State standards and patents:}

GOST 8.586.5-2005. 2007. Metodika vypolneniya izmereniy. Izmerenie raskhoda i~kolichestva
zhidkostey i gazov 
s~pomoshch'yu standartnykh suzhayushchikh ustroystv [Method of measurement.
Measurement of flow rate and volume of liquids and gases by means of orifice devices]. M.:
Standardinform
Publs. 10~p.

%\noindent
%\textbf{Patent:}

\Aue{Bolshakov, M.\,V., A.\,V.~Kulakov, A.\,N.~Lavrenov, and M.\,V.~Palkin}.
2006. Sposob orientirovaniya po krenu letatel'nogo
apparata s opti\-che\-skoy golovkoy
samonavedeniya [The way to orient on the roll of aircraft with optical homing head].
Patent RF No.\,2280590.

References in Latin transcription are presented in the original language.

References in the text are numbered according to the order of their
first appearance; the number is
placed in square brackets. All items from the reference list should be
cited.\\[-13.5pt]

\item Manuscripts and additional materials are not returned to Authors by the Editorial Board.\\[-13.5pt]

\item Submissions of files by e-mail must include:\\[-13.5pt]
\begin{itemize}
\item   the journal title and author's name in the ``Subject'' field; \\[-13.5pt]
\item   an article and additional materials have to be attached using the ``attach'' function;\\[-13.5pt]
\item   an electronic version of the article should contain the file with the text and a separate file
with figures.\\[-13.5pt]
\end{itemize}

\item ``Informatics and Applications'' journal is not a profit publication. There are no
charges for the authors as well as there are no royalties.\\[-13.5pt]
\end{enumerate}

\def\leftfootline{\small{\textbf{\thepage}
\hfill INFORMATIKA I EE PRIMENENIYA~--- INFORMATICS AND APPLICATIONS\ \ \ 2019\
\ \ volume~13\ \ \ issue\ 4}
}%
 \def\rightfootline{\small{INFORMATIKA I EE PRIMENENIYA~--- INFORMATICS AND APPLICATIONS\ \ \ 2019\ \ \ volume~13\ \ \ issue\ 4
\hfill \textbf{\thepage}}}

\def\leftkol{Requirements for manuscripts submitted to Journal
``Informatics~and~Applications''}

\def\rightkol{Requirements for manuscripts submitted to Journal
``Informatics~and~Applications''}


%\vspace*{5mm}


\begin{center}
\textbf{Editorial Board address:} \\

%ABOUT AUTHORS



FRC CSC RAS, 44, block~2, Vavilov Str., Moscow 119333, Russia\\[-10pt]

\

Ph.: +7\,(499)\,135\,86\,92,\ \ Fax: +7\,(495)\,930\,45\,05\\[-10pt]

\

 e-mail: {\sf rust@ipiran.ru} (to Prof.\ Rustem Seyful-Mulyukov)\\[-10pt]

\

 {\sf http://www.ipiran.ru/english/journal.asp}
\end{center}
 }
%\thispagestyle{myheadings}

\def\leftkol{Requirements for manuscripts submitted to Journal
``Informatics~and~Applications''}

\def\rightkol{Requirements for manuscripts submitted to Journal
``Informatics~and~Applications''}

\def\leftfootline{\small{\textbf{\thepage}
\hfill INFORMATIKA I EE PRIMENENIYA~--- INFORMATICS AND APPLICATIONS\ \ \ 2019\
\ \ volume~13\ \ \ issue\ 4}
}%
 \def\rightfootline{\small{INFORMATIKA I EE PRIMENENIYA~--- INFORMATICS AND APPLICATIONS\ \ \ 2019\ \ \ volume~13\ \ \ issue\ 4
\hfill \textbf{\thepage}}}

 \label{end\stat}

\newpage

%\vspace*{-60pt} {\small
{\baselineskip=9.1pt
\section*{Правила подготовки рукописей статей для публикации в журнале
<<Информатика и её применения>>}

\thispagestyle{empty}

 Журнал <<Информатика и её применения>> публикует
теоретические, обзорные и дискуссионные статьи, посвященные научным
исследованиям и разработкам в области информатики и ее приложений. Журнал
издается на русском языке. По специальному решению редколлегии отдельные статьи,
в виде исключения, могут печататься на английском языке.
Тематика журнала охватывает следующие направления:
\begin{itemize}
\item теоретические основы информатики; %\\[-13.5pt]
\item математические методы исследования сложных систем и процессов; %\\[-13.5pt]
\item информационные системы и сети; %\\[-13.5pt]
\item информационные технологии; %\\[-13.5pt]
\item архитектура и программное
обеспечение вычислительных комплексов и сетей.
\end{itemize}
\begin{enumerate}
\item В журнале печатаются результаты, ранее не
опубликованные и не предназначенные к одновременной публикации в других
изданиях. Публикация не должна нарушать закон об авторских правах. Направляя
свою рукопись в редакцию, авторы автоматически передают учредителям и
редколлегии неисключительные права на издание данной статьи на русском языке и
на ее распространение в России и за рубежом. При этом за авторами сохраняются
все права как собственников данной рукописи. В связи с этим авторами должно
быть представлено в редакцию письмо в следующей форме:
Соглашение о передаче права на публикацию:

\textit{<<Мы, нижеподписавшиеся, авторы рукописи <<$\qquad\qquad$>>, передаем
учредителям и редколлегии журнала <<Информатика и её применения>>
неисключительное право опубликовать данную рукопись статьи на русском языке как
в печатной, так и в электронной версиях журнала. Мы подтверждаем, что данная
публикация не нарушает авторского права других лиц или организаций. Подписи
авторов: (ф.\,и.\,о., дата, адрес)>>.}

Указанное соглашение может быть представлено 
как в бумажном виде, так и в виде отсканированной копии (с подписями авторов).


Редколлегия вправе запросить у авторов экспертное заключение о возможности
опубликования представленной статьи в открытой печати. %\\[-13.5pt]
\item Статья
подписывается всеми авторами. На отдельном листе представляются данные автора
(или всех авторов): фамилия, полные имя и отчество, телефон, факс, e-mail,
почтовый адрес. Если работа выполнена несколькими авторами, указывается фамилия
одного из них, ответственного за переписку с редакцией. %\\[-13.5pt]
\item Редакция журнала
осуществляет самостоятельную экспертизу присланных статей. Возвращение рукописи
на доработку не означает, что статья уже принята к печати. Доработанный вариант
с ответом на замечания рецензента необходимо прислать в редакцию. %\\[-13.5pt]
\item Решение
редакционной коллегии о принятии статьи к печати или ее отклонении сообщается
авторам. Редколлегия не обязуется направлять рецензию авторам отклоненной
статьи. %\\[-13.5pt]
\item Корректура статей высылается авторам для просмотра. Редакция
просит авторов присылать свои замечания в кратчайшие сроки. %\\[-13.5pt]
\item При
подготовке рукописи в MS Word рекомендуется использовать следующие настройки.
Параметры страницы: формат~--- А4; ориентация~--- книжная; поля (см): внутри~---
2,5, снаружи~--- 1,5, сверху~--- 2, снизу~--- 2, от края до нижнего
колонтитула~--- 1,3. Основной текст: стиль~--- <<Обычный>>: шрифт Times New
Roman, размер 14~пунктов, абзацный отступ~--- 0,5~см, 1,5 интервала,
выравнивание~--- по ширине. Рекомендуемый объем рукописи~--- не свыше
25~страниц указанного формата. Ознакомиться с шаблонами, содержащими примеры
оформления, можно по адресу в Интернете:
\textsf{http://www.ipiran.ru/journal/template.doc}.
\item К рукописи, предоставляемой в 2-х
экземплярах, обязательно прилагается электронная версия статьи (как правило, в
форматах MS WORD (.doc) или \LaTeX\ (.tex), а также~--- дополнительно~--- в
формате .pdf) на дискете, лазерном диске или по электронной почте. Сокращения
слов, кроме стандартных, не применяются. Все страницы рукописи должны быть
пронумерованы. %\\[-13.5pt]
\item Статья должна содержать следующую информацию на русском и
английском языках: название, Ф.И.О. авторов, места работы авторов и их
электронные адреса, подробные сведения об авторах, оформленные в соответствии с форматом, 
определяемым файлами {\sf http://www.ipiran.ru/journal/issues/2011\_05\_01/authors.asp} и 
{\sf http://www.ipiran.ru/journal/issues/2011\_01\_eng/authors.asp},
аннотация (не более 100~слов), ключевые слова. Ссылки на
литературу в тексте статьи нумеруются (в квадратных скобках) и располагаются в
порядке их первого упоминания. В~списке литературы не должно быть позиций, на которые нет ссылки в тексте статьи.
Все фамилии авторов, заглавия статей, названия
книг, конференций и~т.\,п.\ даются на языке оригинала, если этот язык
использует кириллический или латинский алфавит. %\\[-13.5pt]
\item Присланные в редакцию материалы авторам не возвращаются.
\item При отправке файлов по электронной
почте просим придерживаться следующих правил:
\begin{itemize}
\item указывать в поле subject (тема) название журнала и фамилию автора; %\\[-13.5pt]
\item использовать attach (присоединение); %\\[-13.5pt]
\item в случае больших объемов информации возможно
использование общеизвестных архиваторов (ZIP, RAR); %\\[-13.5pt]
\item в состав электронной версии статьи должны входить: файл, содержащий текст статьи, и файл(ы),
содержащий(е) иллюстрации. %\\[-13.5pt]
\end{itemize}
\item Журнал <<Информатика и её применения>> является некоммерческим изданием. 
Плата за публикацию с авторов не взимается, гонорар авторам не выплачивается.
\end{enumerate}
\thispagestyle{empty}
\textbf{Адрес редакции:} Москва 119333,
ул.~Вавилова, д.~44, корп.~2, ИПИ РАН\\
\hphantom{\textbf{Адрес редакции:} }Тел.: +7 (499) 135-86-92\ \
Факс:  +7 (495) 930-45-05\ \  E-mail:   rust@ipiran.ru }
}

%\include{ipi-ind}

%\tableofcontents

\end{document}


%\tableofcontents

%\end{document}





%\def\stat{cont}
{%\hrule\par
%\vskip 7pt % 7pt
\raggedleft\Large \bf%\baselineskip=3.2ex
А\,В\,Т\,О\,Р\,С\,К\,И\,Й\ \ У\,К\,А\,З\,А\,Т\,Е\,Л\,Ь\ \ З\,А\ \ 2\,0\,0\,7 г. \vskip 17pt
    \hrule
    \par
\vskip 21pt plus 6pt minus 3pt }

\label{st\stat}

\def\tit{\ }

\def\aut{\ }
\def\auf{\ }

\def\leftkol{\ } % ENGLISH ABSTRACTS}

\def\rightkol{\ } %ENGLISH ABSTRACTS}

\titele{\tit}{\aut}{\auf}{\leftkol}{\rightkol}


\contentsline {chapter}{\ }{Выпуск \quad Стр.} 
\contentsline {section}{\textbf{Батракова Д.\,А., Королев В.\,Ю., Шоргин С.\,Я.}\ \ Новый метод вероятностно-ста\-ти\-сти\-че\-ско\-го анализа информационных потоков в\nobreakspace {}телекоммуникационных сетях}{\qquad 1 \qquad 40} 
\contentsline {section}{\textbf{Борисов А.\,В.}\ \ Байесовское оценивание в системах наблюдения с\nobreakspace {}марковскими скачкообразными процессами: игровой подход}{\qquad 2 \qquad 65}
\contentsline {section}{\textbf{Босов А.\,В., Иванов А.\,В.}\ \ Программная инфраструктура информационного Web-пор\-тала}{\qquad 2 \qquad 50}
\contentsline {section}{\textbf{Захаров В.\,Н., Калиниченко Л.\,А., Соколов И.\,А., Ступников С.\,А.}\ \ Конструирование канонических информационных моделей для интегрированных информационных систем}{\qquad 2 \qquad 15}
\contentsline {section}{\textbf{Захаров В.\,Н., Козмидиади В.\,А.}\ \ Средства обеспечения отказоустойчивости при\-ло\-жений}{\qquad 1 \qquad 14} 
\contentsline {section}{\textbf{Иванов А.\,В.}\ \ см. Босов А.\,В.\hfill\hfill\hfill\hfill\hfill\hfill\hfill\hfill\hfill\hfill\hfill\hfill\hfill\hfill\hfill\hfill\hfill\hfill\hfill\hfill\hfill\hfill\hfill\hfill\hfill\hfill\hfill\hfill\hfill\hfill\hfill\hfill\hfill\hfill\hfill}{\ }
\contentsline {section}{\textbf{Ильин В.\,Д., Соколов И.\,А.}\ \ Символьная модель системы знаний информатики в\nobreakspace {}че\-ло\-ве\-ко-автоматной среде}{\qquad 1 \qquad 66} 
\contentsline {section}{\textbf{Калиниченко Л.\,А.}\ \ см. Захаров В.\,Н.\hfill\hfill\hfill\hfill\hfill\hfill\hfill\hfill\hfill\hfill\hfill\hfill\hfill\hfill\hfill\hfill\hfill\hfill\hfill\hfill\hfill\hfill\hfill\hfill\hfill\hfill\hfill\hfill\hfill\hfill\hfill\hfill\hfill\hfill\hfill}{\ }
\contentsline {section}{\textbf{Козеренко Е.\,Б.}\ \ Лингвистическое моделирование для систем машинного перевода и обработки знаний}{\qquad 1 \qquad 54} 
\contentsline {section}{\textbf{Козмидиади В.\,А.}\ \ см. Захаров В.\,Н.\hfill\hfill\hfill\hfill\hfill\hfill\hfill\hfill\hfill\hfill\hfill\hfill\hfill\hfill\hfill\hfill\hfill\hfill\hfill\hfill\hfill\hfill\hfill\hfill\hfill\hfill\hfill\hfill\hfill\hfill\hfill\hfill\hfill\hfill\hfill }{\ } 
\contentsline {section}{\textbf{Королев В.\,Ю.}\ \ см. Батракова Д.\,А.\hfill\hfill\hfill\hfill\hfill\hfill\hfill\hfill\hfill\hfill\hfill\hfill\hfill\hfill\hfill\hfill\hfill\hfill\hfill\hfill\hfill\hfill\hfill\hfill\hfill\hfill\hfill\hfill\hfill\hfill\hfill\hfill\hfill\hfill\hfill}{\ } 
\contentsline {section}{\textbf{Кудрявцев А.\,А., Шоргин С.\,Я.}\ \ Байесовский подход к\nobreakspace {}анализу систем массового обслуживания и\nobreakspace {}показателей надежности}{\qquad 2 \qquad 76}
\contentsline {section}{\textbf{Печинкин А.\,В., Соколов И.\,А., Чаплыгин В.\,В.}\ \ Многолинейная система массового обслуживания с конечным накопителем и ненадежными приборами}{\qquad 1 \qquad 27} 
\contentsline {section}{\textbf{Печинкин А.\,В., Соколов И.\,А., Чаплыгин В.\,В.}\ \ Стационарные характеристики многолинейной\nobreakspace {}системы массового обслуживания с\nobreakspace {}одновременными отказами приборов}{\qquad 2 \qquad 39}
\contentsline {section}{\textbf{Синицын И.\,Н.}\ \ Корреляционные методы построения аналитических информационных моделей флуктуаций полюса Земли по априорным данным}{\qquad 2 \qquad \hphantom{9}2}
\contentsline {section}{\textbf{Синицын И.\,Н.}\ \ Развитие теории фильтров Пугачева для оперативной обработки информации в стохастических системах}{{\qquad 1 \qquad \hphantom{9}3}} 
\contentsline {section}{\textbf{Соколов И.\,А.}\ \ см. Захаров В.\,Н.\hfill\hfill\hfill\hfill\hfill\hfill\hfill\hfill\hfill\hfill\hfill\hfill\hfill\hfill\hfill\hfill\hfill\hfill\hfill\hfill\hfill\hfill\hfill\hfill\hfill\hfill\hfill\hfill\hfill\hfill\hfill\hfill\hfill\hfill\hfill}{\ }
\contentsline {section}{\textbf{Соколов И.\,А.}\ \ см. Ильин В.\,Д.\hfill\hfill\hfill\hfill\hfill\hfill\hfill\hfill\hfill\hfill\hfill\hfill\hfill\hfill\hfill\hfill\hfill\hfill\hfill\hfill\hfill\hfill\hfill\hfill\hfill\hfill\hfill\hfill\hfill\hfill\hfill\hfill\hfill\hfill\hfill}{\ } 
\contentsline {section}{\textbf{Соколов И.\,А.}\ \ см. Печинкин А.\,В.\hfill\hfill\hfill\hfill\hfill\hfill\hfill\hfill\hfill\hfill\hfill\hfill\hfill\hfill\hfill\hfill\hfill\hfill\hfill\hfill\hfill\hfill\hfill\hfill\hfill\hfill\hfill\hfill\hfill\hfill\hfill\hfill\hfill\hfill\hfill}{\ } 
\contentsline {section}{\textbf{Соколов И.\,А.}\ \ см. Печинкин А.\,В.\hfill\hfill\hfill\hfill\hfill\hfill\hfill\hfill\hfill\hfill\hfill\hfill\hfill\hfill\hfill\hfill\hfill\hfill\hfill\hfill\hfill\hfill\hfill\hfill\hfill\hfill\hfill\hfill\hfill\hfill\hfill\hfill\hfill\hfill\hfill}{\ }
\contentsline {section}{\textbf{Ступников С.\,А.}\ \ см. Захаров В.\,Н.\hfill\hfill\hfill\hfill\hfill\hfill\hfill\hfill\hfill\hfill\hfill\hfill\hfill\hfill\hfill\hfill\hfill\hfill\hfill\hfill\hfill\hfill\hfill\hfill\hfill\hfill\hfill\hfill\hfill\hfill\hfill\hfill\hfill\hfill\hfill}{\ }
\contentsline {section}{\textbf{Чаплыгин В.\,В.}\ \ см. Печинкин А.\,В.\hfill\hfill\hfill\hfill\hfill\hfill\hfill\hfill\hfill\hfill\hfill\hfill\hfill\hfill\hfill\hfill\hfill\hfill\hfill\hfill\hfill\hfill\hfill\hfill\hfill\hfill\hfill\hfill\hfill\hfill\hfill\hfill\hfill\hfill\hfill}{\ } 
\contentsline {section}{\textbf{Чаплыгин В.\,В.}\ \ см. Печинкин А.\,В.\hfill\hfill\hfill\hfill\hfill\hfill\hfill\hfill\hfill\hfill\hfill\hfill\hfill\hfill\hfill\hfill\hfill\hfill\hfill\hfill\hfill\hfill\hfill\hfill\hfill\hfill\hfill\hfill\hfill\hfill\hfill\hfill\hfill\hfill\hfill}{\ }
\contentsline {section}{\textbf{Шоргин С.\,Я.}\ \ см. Батракова Д.\,А.\hfill\hfill\hfill\hfill\hfill\hfill\hfill\hfill\hfill\hfill\hfill\hfill\hfill\hfill\hfill\hfill\hfill\hfill\hfill\hfill\hfill\hfill\hfill\hfill\hfill\hfill\hfill\hfill\hfill\hfill\hfill\hfill\hfill\hfill\hfill}{\ } 
\contentsline {section}{\textbf{Шоргин С.\,Я.}\ \ см. Кудрявцев А.\,А.\hfill\hfill\hfill\hfill\hfill\hfill\hfill\hfill\hfill\hfill\hfill\hfill\hfill\hfill\hfill\hfill\hfill\hfill\hfill\hfill\hfill\hfill\hfill\hfill\hfill\hfill\hfill\hfill\hfill\hfill\hfill\hfill\hfill\hfill\hfill}{\ }
%\thispagestyle{myheadings}
\def\leftfootline{\small{\textbf{\thepage}
\hfill ИНФОРМАТИКА И ЕЁ ПРИМЕНЕНИЯ\ \ \ том~1\ \ \ выпуск~2\ \ \ 2007}
}%
 \def\rightfootline{\small{ИНФОРМАТИКА И ЕЁ ПРИМЕНЕНИЯ\ \ \ том~1\ \ \ выпуск~2\ \ \ 2007
 \hfill \textbf{\thepage}}}
 \label{end\stat}

%\def\stat{cont-e}
{%\hrule\par
%\vskip 7pt % 7pt
\raggedleft\Large \bf%\baselineskip=3.2ex
2\,0\,0\,7\ \ A\,U\,T\,H\,O\,R\ \ I\,N\,D\,E\,X \vskip 17pt
    \hrule
    \par
\vskip 21pt plus 6pt minus 3pt }

\label{st\stat}

\def\tit{\ }

\def\aut{\ }
\def\auf{\ }

\def\leftkol{\ } % ENGLISH ABSTRACTS}

\def\rightkol{\ } %ENGLISH ABSTRACTS}

\titele{\tit}{\aut}{\auf}{\leftkol}{\rightkol}


\contentsline {chapter}{\ }{Issue \quad Page} 
\contentsline {subsection}{\textbf{Batrakova D.\,A., Korolev V.\,Yu., Shorgin S.\,Ya.}\ \ A New Method for the Probabilistic and Statistical Analysis of Information Flows in Telecommunication Networks}{\qquad 1 \qquad 40} 
\contentsline {subsection}{\textbf{Borisov A.\,V.}\ \ Bayesian Estimation in\nobreakspace {}Observation Systems with\nobreakspace {}Markov Jump Processes: Game-Theoretic Approach}{\qquad 2 \qquad 65} 
\contentsline {subsection}{\textbf{Bosov A.\,V., Ivanov A.\,V.}\ \ Linguistic Simulation for Machine Translation and Knowledge Management Systems}{\qquad 2 \qquad 50} 
\contentsline {subsection}{\textbf{Chaplygin V.\,V.} see Pechinkin A.\,V.\hfill\hfill\hfill\hfill\hfill\hfill\hfill\hfill\hfill\hfill\hfill\hfill\hfill\hfill\hfill\hfill\hfill\hfill\hfill\hfill\hfill\hfill\hfill\hfill\hfill\hfill\hfill\hfill\hfill\hfill\hfill\hfill\hfill\hfill\hfill}{\ }
\contentsline {subsection}{\textbf{Chaplygin V.\,V.} see Pechinkin A.\,V.\hfill\hfill\hfill\hfill\hfill\hfill\hfill\hfill\hfill\hfill\hfill\hfill\hfill\hfill\hfill\hfill\hfill\hfill\hfill\hfill\hfill\hfill\hfill\hfill\hfill\hfill\hfill\hfill\hfill\hfill\hfill\hfill\hfill\hfill\hfill}{\ }
\contentsline {subsection}{\textbf{Ilyin V.\,D., Sokolov I.\,A.}\ \ The Symbol Model of Informatics Knowledge System in Human-Automaton Environment}{\qquad 1 \qquad 66} 
\contentsline {subsection}{\textbf{Ivanov A.\,V.} see Bosov A.\,V.\hfill\hfill\hfill\hfill\hfill\hfill\hfill\hfill\hfill\hfill\hfill\hfill\hfill\hfill\hfill\hfill\hfill\hfill\hfill\hfill\hfill\hfill\hfill\hfill\hfill\hfill\hfill\hfill\hfill\hfill\hfill\hfill\hfill\hfill\hfill}{\ }
\contentsline {subsection}{\textbf{Kalinichenko L.\,A.} see Zakharov V.\,N.\hfill\hfill\hfill\hfill\hfill\hfill\hfill\hfill\hfill\hfill\hfill\hfill\hfill\hfill\hfill\hfill\hfill\hfill\hfill\hfill\hfill\hfill\hfill\hfill\hfill\hfill\hfill\hfill\hfill\hfill\hfill\hfill\hfill\hfill\hfill}{\ }
\contentsline {subsection}{\textbf{Korolev V.\,Yu.} see Batrakova D.\,A.\hfill\hfill\hfill\hfill\hfill\hfill\hfill\hfill\hfill\hfill\hfill\hfill\hfill\hfill\hfill\hfill\hfill\hfill\hfill\hfill\hfill\hfill\hfill\hfill\hfill\hfill\hfill\hfill\hfill\hfill\hfill\hfill\hfill\hfill\hfill}{\ }
\contentsline {subsection}{\textbf{Kozerenko E.\,B.}\ \ Linguistic Simulation for Machine Translation and Knowledge Management Systems}{\qquad 1 \qquad 54} 
\contentsline {subsection}{\textbf{Kozmidiady V.\,A.} see Zakharov V.\,N.\hfill\hfill\hfill\hfill\hfill\hfill\hfill\hfill\hfill\hfill\hfill\hfill\hfill\hfill\hfill\hfill\hfill\hfill\hfill\hfill\hfill\hfill\hfill\hfill\hfill\hfill\hfill\hfill\hfill\hfill\hfill\hfill\hfill\hfill\hfill}{\ }
\contentsline {subsection}{\textbf{Kudryavtsev A.\,A., Shorgin S.\,Ya.}\ \ Bayesian Approach to Queueing Systems and Reliability Characteristics}{\qquad 2 \qquad 76} 
\contentsline {subsection}{\textbf{Pechinkin A.\,V., Sokolov I.\,A., Chaplygin V.\,V.}\ \ Multichannel Queuing System with Finite Buffer and Unreliable Servers}{\qquad 1 \qquad 27} 
\contentsline {subsection}{\textbf{Pechinkin A.\,V., Sokolov I.\,A., Chaplygin V.\,V.}\ \ Stationary Characteristics of a Multichannel Queueing System with\nobreakspace {}Simultaneous Refusals of Servers}{\qquad 2 \qquad 39} 
\contentsline {subsection}{\textbf{Shorgin S.\,Ya.} see Batrakova D.\,A.\hfill\hfill\hfill\hfill\hfill\hfill\hfill\hfill\hfill\hfill\hfill\hfill\hfill\hfill\hfill\hfill\hfill\hfill\hfill\hfill\hfill\hfill\hfill\hfill\hfill\hfill\hfill\hfill\hfill\hfill\hfill\hfill\hfill\hfill\hfill}{\ }
\contentsline {subsection}{\textbf{Shorgin S.\,Ya.} see Kudryavtsev A.\,A.\hfill\hfill\hfill\hfill\hfill\hfill\hfill\hfill\hfill\hfill\hfill\hfill\hfill\hfill\hfill\hfill\hfill\hfill\hfill\hfill\hfill\hfill\hfill\hfill\hfill\hfill\hfill\hfill\hfill\hfill\hfill\hfill\hfill\hfill\hfill}{\ }
\contentsline {subsection}{\textbf{Sinitsyn I.\,N.}\ \ Correlational Methods for Analytical Informational Models of the Earth Pole Fluctuations Design Based on a priori Data}{\qquad 2 \qquad \hphantom{9}2}
\contentsline {subsection}{\textbf{Sinitsyn I.\,N.}\ \ Development of Pugachev Filtering for Stochastic Systems}{\qquad 1 \qquad \hphantom{9}3}
\contentsline {subsection}{\textbf{Sokolov I.\,A.} see Ilyin V.\,D.\hfill\hfill\hfill\hfill\hfill\hfill\hfill\hfill\hfill\hfill\hfill\hfill\hfill\hfill\hfill\hfill\hfill\hfill\hfill\hfill\hfill\hfill\hfill\hfill\hfill\hfill\hfill\hfill\hfill\hfill\hfill\hfill\hfill\hfill\hfill}{\ }
\contentsline {subsection}{\textbf{Sokolov I.\,A.} see Pechinkin A.\,V.\hfill\hfill\hfill\hfill\hfill\hfill\hfill\hfill\hfill\hfill\hfill\hfill\hfill\hfill\hfill\hfill\hfill\hfill\hfill\hfill\hfill\hfill\hfill\hfill\hfill\hfill\hfill\hfill\hfill\hfill\hfill\hfill\hfill\hfill\hfill}{\ }
\contentsline {subsection}{\textbf{Sokolov I.\,A.} see Pechinkin A.\,V.\hfill\hfill\hfill\hfill\hfill\hfill\hfill\hfill\hfill\hfill\hfill\hfill\hfill\hfill\hfill\hfill\hfill\hfill\hfill\hfill\hfill\hfill\hfill\hfill\hfill\hfill\hfill\hfill\hfill\hfill\hfill\hfill\hfill\hfill\hfill}{\ }
\contentsline {subsection}{\textbf{Sokolov I.\,A.} see Zakharov V.\,N.\hfill\hfill\hfill\hfill\hfill\hfill\hfill\hfill\hfill\hfill\hfill\hfill\hfill\hfill\hfill\hfill\hfill\hfill\hfill\hfill\hfill\hfill\hfill\hfill\hfill\hfill\hfill\hfill\hfill\hfill\hfill\hfill\hfill\hfill\hfill}{\ }
\contentsline {subsection}{\textbf{Stupnikov S.\,A.} see Zakharov V.\,N.\hfill\hfill\hfill\hfill\hfill\hfill\hfill\hfill\hfill\hfill\hfill\hfill\hfill\hfill\hfill\hfill\hfill\hfill\hfill\hfill\hfill\hfill\hfill\hfill\hfill\hfill\hfill\hfill\hfill\hfill\hfill\hfill\hfill\hfill\hfill}{\ }
\contentsline {subsection}{\textbf{Zakharov V.\,N., Kalinichenko L.\,A., Sokolov I.\,A., Stupnikov S.\,A.}\ \ Development of Canonical Information Models for Integrated Information Systems}{\qquad 2 \qquad 15} 
\contentsline {subsection}{\textbf{Zakharov V.\,N., Kozmidiady V.\,A.}\ \ Means Providing Applications Fault Tolerance}{\qquad 1 \qquad 14} 
\def\leftfootline{\small{\textbf{\thepage}
\hfill ИНФОРМАТИКА И ЕЁ ПРИМЕНЕНИЯ\ \ \ том~1\ \ \ выпуск~2\ \ \ 2007}
}%
 \def\rightfootline{\small{ИНФОРМАТИКА И ЕЁ ПРИМЕНЕНИЯ\ \ \ том~1\ \ \ выпуск~2\ \ \ 2007
 \hfill \textbf{\thepage}}}
 \label{end\stat}


%\tableofcontents


\end{document}

\newcommand{\Ack}{\subsection*{\protect\large\bf Acknowledgments}}