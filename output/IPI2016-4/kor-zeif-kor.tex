\def\stat{kor-zeif-kor}

\def\tit{НЕСИММЕТРИЧНЫЕ РАСПРЕДЕЛЕНИЯ ЛИННИКА\\ КАК ПРЕДЕЛЬНЫЕ
ЗАКОНЫ ДЛЯ СЛУЧАЙНЫХ СУММ НЕЗАВИСИМЫХ СЛУЧАЙНЫХ ВЕЛИЧИН\\ С~КОНЕЧНЫМИ
ДИСПЕРСИЯМИ$^*$}

\def\titkol{Несимметричные распределения Линника как предельные
законы для случайных сумм %независимых 
случайных величин} % с~конечными дисперсиями}

\def\aut{В.\,Ю.~Королев$^1$, А.\,И.~Зейфман$^2$,  А.\,Ю.~Корчагин$^3$}

\def\autkol{В.\,Ю.~Королев, А.\,И.~Зейфман,  А.\,Ю.~Корчагин}

\titel{\tit}{\aut}{\autkol}{\titkol}

\index{Королев В.\,Ю.}
\index{Зейфман А.\,И.}
\index{Корчагин А.\,Ю.}
\index{Korolev V.\,Yu.}
\index{Zeifman A.\,I.}
\index{Korchagin A.\,Yu.}


{\renewcommand{\thefootnote}{\fnsymbol{footnote}} \footnotetext[1]
{Работа выполнена при финансовой поддержке
Российского научного фонда (проект 14-11-00364).}}


\renewcommand{\thefootnote}{\arabic{footnote}}
\footnotetext[1]{Факультет вычислительной математики и~кибернетики Московского государственного 
университета им.\ М.\,В.~Ломоносова; 
Институт проблем информатики Федерального исследовательского центра 
<<Информатика и~управление>> Российской академии наук, \mbox{vkorolev@cs.msu.ru}}
\footnotetext[2]{Вологодский государственный университет; Институт проблем информатики 
Федерального исследовательского центра <<Информатика и~управление>> 
Российской академии наук; Институт со\-ци\-аль\-но-эко\-но\-ми\-че\-ско\-го развития 
территории Российской академии наук, Факультет вычислительной математики и~кибернетики Московского государственного 
университета им.\ М.\,В.~Ломоносова, \mbox{a\_zeifman@mail.ru}}
\footnotetext[3]{Факультет вычислительной математики и~кибернетики Московского 
государственного университета им.\ М.\,В.~Ломоносова, \mbox{sasha.korchagin@gmail.com}}

\vspace*{6pt}

\Abst{Распределения Линника (симметричные гео\-мет\-ри\-чески
устойчивые распределения) находят широкое применение в~финансовой
математике, телекоммуникационных системах, астрофизике, генетике.
Такие распределения являются предельными для геометрических сумм
независимых одинаково распределенных случайных величин (с.в.),
распределения которых принадлежат области нормального притяжения
симметричного строго устойчивого распределения. В~статье
рассматриваются три несим\-мет\-рич\-ных обобщения распределения Линника.
Традиционный (и формальный) подход к~не\-сим\-мет\-рич\-но\-му обобщению
распределения Линника заключается в~рассмотрении геометрических сумм
случайных слагаемых, распределения которых притягиваются к~\textit{не\-сим\-мет\-рич\-но\-му} 
строго устойчивому распределению. Дис\-пер\-сии таких
слагаемых бесконечны. Поскольку при моделировании реальных явлений,
как правило, нет веских причин отвергать предположение о конечности
дисперсии элементарных слагаемых, в~качестве альтернатив
традиционному подходу в~статье предложены несимметричные обобщения,
основанные на представлении распределения Линника в~виде смеси
нормальных распределений и~смеси распределений Лапласа. Приведены
примеры предельных теорем для сумм случайного числа независимых
с.в.\ \textit{с~конечными дисперсиями}, в~которых
предложенные несимметричные распределения Линника выступают 
в~качестве предельных законов.}

\KW{распределение Линника; распределение Лапласа;
распределение Мит\-таг--Леф\-фле\-ра; нормальное распределение; масштабная
смесь; дис\-пер\-си\-он\-но-сдви\-го\-вая смесь нормальных законов; устойчивое
распределение; геометрически устойчивое распределение}

\DOI{10.14357/19922264160403}

\vspace*{6pt}  


\vskip 10pt plus 9pt minus 6pt

\thispagestyle{headings}

\begin{multicols}{2}

\label{st\stat}

\section{Введение}

Распределения Линника (симметричные гео\-мет\-ри\-чески устойчивые
распределения) имеют доволь\-но широкое применение в~финансовой
математике, телекоммуникационных системах, астрофизике, генетике.
Разнообразные приложения распределений Линника описаны 
в~работах~\cite{MittnikRachev1991, Kotz2001}. В~частности, распределения
Линника возникают при изучении механизма синтеза мелатонина 
в~человеческом организме, солнечных нейтринных потоков в~космосе,
явлений рос\-та-упад\-ка в~природе, эконометрических явлений и~т.\,п.

Геометрически устойчивые законы и~только они могут быть предельными
распределениями для геометрических случайных сумм независимых
одинаково распределенных с.в. Поэтому
традиционно\linebreak несимметричное обобщение распределения Линника
достигается за счет того, что в~\textit{геометрической} случайной сумме
рассматриваются скошенные\linebreak слагаемые. При этом распределения
слагаемых принадлежат области нормального притяжения несимметричного
устойчивого закона с~некоторым показателем $\alpha\hm\in(0,2]$ и,~значит, 
при $0\hm<\alpha\hm<2$ имеют бесконечные моменты порядков, больших
или равных~$\alpha$. Что касается случая $\alpha\hm=2$, когда конечна
дисперсия, то в~рамках схемы гео\-мет\-ри\-че\-ско\-го суммирования он
неизбежно приводит к~единственно возможному распределению~---
распределению Лап\-ласа.

Однако при использовании распределений Линника в~качестве моделей
реальных явлений нельзя не задуматься над вопросом о том, что если
используется аддитивная структурная модель реального процесса типа
случайно остановленного случайного блуждания, то какая комбинация
условий встречается чаще:
\begin{itemize}
\item распределение числа слагаемых (числа скачков) является геометрическим
(асимптотически экспоненциальным), но слагаемые \mbox{(скачки)} имеют столь
тяжелые хвосты, что как минимум у них бесконечна дисперсия, или

\item вторые моменты (дисперсии) слагаемых (скачков) конечны, но
число слагаемых отличается нерегулярным поведением, допускающим
иногда возможность очень больших значений?
\end{itemize}

Поскольку, как правило, при моделировании реальных явлений нет веских
причин отвергать предположение о конечности дисперсии скачков,
вторая комбинация как минимум заслуживает внимательного изучения.

Оказывается, что распределения Линника допускают представление 
в~виде масштабных смесей нормальных законов. Это означает, что они
могут быть предельными в~аналогах центральной предельной теоремы для
случайных сумм независимых с.в.\ с~\textit{конечными} 
дисперсиями~\cite{KorolevZeifman2016a, KorolevZeifman2016b}, что открывает пути
альтернативных несимметричных обобщений этих распределений, которым
и~посвящена данная статья.

\section{Вспомогательные сведения}

В~дальнейшем иногда будет удобнее вести изложение не в~терминах
распределений, а~в~терминах с.в., предполагая, что все они заданы
на одном вероятностном пространстве $(\Omega,\mathfrak{A}, {\sf P})$.

Случайная величина со стандартной показательной функцией распределения (ф.р.)\
будет обозначаться~$W_1$: ${\sf P}(W_1<x)\hm=\left[1-e^{-x}\right]{\bf 1}
(x\hm\geqslant0)$ (здесь и~далее символ~$\mathbf{1}(C)$ обозначает индикатор
множества~$C$). Случайная величина со стандартной нормальной ф.р.~$\Phi(x)$
будет обозначаться~$X$,
$$
{\sf P}(X<x)=\Phi(x)=\fr{1}{\sqrt{2\pi}}\int\limits_{-\infty}^{x}e^{-z^2/2}\,dz\,,\enskip
x\in\mathbb{R}\,.
$$
Функция распределения и~ плот\-ность строго устойчивого распределения 
с~характеристическим показателем~$\alpha$ и~па\-ра\-мет\-ром формы~$\theta$,
определяемого характеристической функцией (х.ф.)
$$
\mathfrak{f}_{\alpha,\theta}(t)=
\exp\left\{-|t|^{\alpha}\exp\left\{-\fr{1}{2}\,
i\pi\theta\alpha\,\mathrm{sign}\,t\right\}
\right\}\,,\enskip
t\in\mathbb{R}\,,
$$
где $0<\alpha\leqslant2$, $|\theta|\hm\leqslant\min\{1,({2}/{\alpha})-1\}$, будут
соответственно обозначаться $G_{\alpha,\theta}(x)$ 
и~$g_{\alpha,\theta}(x)$ (см., например,~\cite{Zolotarev1983}). Любую
с.в.\ с~ф.р.~$G_{\alpha,\theta}(x)$ будем обозначать~$S_{\alpha,\theta}$. 
Симметричным строго устойчивым распределениям
соответствует значение $\theta\hm=0$ и~х.ф.
\begin{equation}
\mathfrak{f}_{\alpha,0}(t)=e^{-|t|^{\alpha}}\,,\enskip t\in\mathbb{R}\,.
\label{e1-kz}
\end{equation}
Отсюда несложно видеть, что $S_{2,0}\eqd\sqrt{2}X$.

Односторонним строго устойчивым законам, сосредоточенным на
неотрицательной полуоси, соответствуют значения $\theta\hm=1$ 
и~$0\hm<\alpha\hm\leqslant1$. Пары $\alpha\hm=1$, $\theta\hm=\pm1$ отвечают
распределениям, вы\-рож\-ден\-ным в~$\pm1$ соответственно. Остальные
устойчивые распределения абсолютно непрерывны. Явные выражения
устойчивых плотностей в~терминах элементарных функций отсутствуют за
четырьмя исключениями (нормальный закон ($\alpha\hm=2$, $\theta\hm=0$),
распределение Коши ($\alpha\hm=1$, $\theta\hm=0$), распределение Леви
($\alpha\hm=1/2$, $\theta\hm=1$) и~распределение, симметричное 
к~распределению Леви ($\alpha\hm=1/2$, $\theta\hm=-1$)). Выражения
устойчивых плотностей в~терминах функций Фокса (обобщенных
$G$-функ\-ций Мейера) можно найти в~\cite{Schneider1986, UchaikinZolotarev1999}.

Хорошо известно, что если $0\hm<\alpha\hm<2$, 
то ${\sf E}|S_{\alpha,\theta}|^{\beta}\hm<\infty$ для любого
$\beta\hm\in(0,\alpha)$, при этом моменты с.в.~$S_{\alpha,\theta}$ порядков
$\beta\hm>\alpha$ не существуют (см., например,~\cite{Zolotarev1983}).
Несмотря на отсутствие явных выражений плотностей устойчивых
распределений в~терминах элементарных функций, можно 
показать~\cite{KorolevWeibull2016}, что для $0\hm<\beta\hm<\alpha\hm<2$
$$
{\sf E}|S_{\alpha,0}|^{\beta}=\fr{2^{\beta}}{\sqrt{\pi}}\,
\fr{\Gamma(({\beta+1})/{2})\Gamma(1-{\beta}/{\alpha})}
{\Gamma({2}/{\beta}-1)}
$$
и для $0<\beta<\alpha\hm\leqslant 1$
$$
{\sf
E}S_{\alpha,1}^{\beta}=\fr{\Gamma(1-{\beta}/{\alpha})}{\Gamma(1-\beta)}\,.
$$

Символы $\eqd$ и~$\Longrightarrow$ будут соответственно обозначать
совпадение распределений и~сходимость по распределению.

Говорят, что распределение с.в.~$Y$ принадлежит 
к~области нормального притяжения строго устойчивого закона
$G_{\alpha,\theta}$, $\mathcal{L}(Y)\hm\in \mathrm{DNA}\left(G_{\alpha,\theta}\right)$,
если существует конечная положительная постоянная~$c$ та\-кая,~что
$$
\fr{c}{n^{1/\alpha}}\sum\limits_{j=1}^nX_j\Longrightarrow
S_{\alpha,\theta}\enskip (n\to\infty)\,,
$$
где $X_1,X_2,\ldots$~--- независимые копии с.в.~$Y$. В~дальнейшем
будем рассматривать случай стандартного масштаба и~полагаем $c\hm=1$. 
В~работе~\cite{Tucker1975} было показано, что если $\mathcal{L}(Y)\hm\in
\mathrm{DNA}\left(G_{\alpha,\theta}\right)$, то ${\sf E}|Y|^{\beta}\hm=\infty$ для любого
$\beta\hm>\alpha$.

Распределение $H$ с.в.~$Q$ называется геометрически устойчивым,
если оно является слабым пределом геометрических случайных сумм
независимых одинаково распределенных с.в., а~именно: если существует
последовательность независимых одинаково распределенных с.в.\
$X_1,X_2,\ldots$ и~с.в.~$V_p$, имеющая геометрическое распределение
\begin{equation*}
{\sf P}(V_p=n)=p(1-p)^{n-1}\,,\enskip n=1,2,\ldots\,,\ \  p\in(0,1)\,,
%\label{e2-kz}
\end{equation*}
при каждом $p\hm\in(0,1)$ независимая от $X_1,X_2,\ldots,$ 
и~положительные константы~$a_p\hm>0$ такие, что
$$
a_p\left(X_1+\cdots+X_{V_p}\right)\Longrightarrow Q
$$
при $p\to 0$ (см., например,~[1, 10--12]). 
В~работе~\cite{KlebanovManiaMelamed1984} (также см., 
например,~\cite{Rachev1991, GnedenkoKorolev1996}) показано, что распределение~$H$ 
является геометрически устойчивым тогда и~только тогда, когда
соответствующая ему х.ф.~$\mathfrak{h}(t)$ допускает представление
\begin{equation}
\mathfrak{h}(t)=\left(1-\log\mathfrak{f}_{\alpha,\theta}(t)\right)^{-1}\,,\enskip
t\in\mathbb{R}\,,
\label{e3-kz}
\end{equation}
при некоторых $\alpha\hm\in(0,2]$ 
и~$\theta\hm\in[-\min\{1,{2}/{\alpha}\hm-1\},\,\min\{1,{2}/{\alpha}\hm-1\}]$.

Некоторые результаты данной работы будут существенно опираться на
следующее вспомогательное утверждение. Рассмотрим последовательность
с.в.~$Y_1, Y_2,\ldots$ Пусть $N_1,N_2,\ldots$~---
на\-ту\-раль\-но\-знач\-ные с.в.\ такие, что при каждом~$n$ с.в.~$N_n$ 
независима от последовательности $Y_1,Y_2,\ldots$ Всюду далее
сходимость подразумевается при $n\hm\to\infty$.

\smallskip

\noindent
\textbf{Лемма~1.}\ \textit{Предположим, что существуют неограниченно
возрастающая $($убывающая к~нулю$)$ последовательность положительных
чисел $\{b_n\}_{n\geqslant1}$ и~с.в.~$Y$ такие, что}
$$
\fr{Y_n}{b_n}\Longrightarrow Y\,.
$$
\textit{Если существуют неограниченно возрастающая $($убывающая к~нулю$)$
последовательность положительных чисел $\{d_n\}_{n\geqslant1}$ и~с.в.~$V$
такие, что}
\begin{equation}
\fr{b_{N_n}}{d_n}\Longrightarrow V\,,
\label{e4-kz}
\end{equation}
\textit{то}
\begin{equation}
\fr{Y_{N_n}}{d_n}\Longrightarrow Y V\,,\label{e5-kz}
\end{equation}
\textit{причем случайные сомножители в~правой части}~(\ref{e5-kz}) \textit{независимы. Если
дополнительно $N_n\hm\longrightarrow\infty$ по вероятности и~семейство
масштабных смесей ф.р.\ с.в.~$Y$ идентифицируемо, то условие}~(\ref{e4-kz})
 \textit{не только достаточно для}~(\ref{e5-kz}), \textit{но и~необходимо.}

\smallskip

\noindent
Д\,о\,к\,а\,з\,а\,т\,е\,л\,ь\,с\,т\,в\,о\ \  см.~в~\cite{Korolev1994} (случай
$b_n,d_n\hm\to\infty$), \cite{Korolev1995} (случай $b_n,d_n\hm\to 0$) 
или~\cite{BeningKorolev2002}, теорема~3.5.5.

\section{Распределения Линника}

Распределения с~х.ф.
$$
\mathfrak{f}^{L}_{\alpha}(t)=\left(1+|t|^{\alpha}\right)^{-1}\,,\enskip
t\in\mathbb{R}\,,
$$
где $0<\alpha\hm\leqslant2$, принято называть \textit{распределениями Линника}
(в~работе~\cite{Pillai1985} предложено альтернативное менее
употребительное название \textit{$\alpha$-Laplace distribution}). Они
были введены Ю.\,В.~Линником в~ 1953~г.~\cite{Linnik1953}. При
$\alpha\hm=2$ распределение Линника превращается в~распределение
Лапласа, соответствующее плотности
\begin{equation}
f^{\Lambda}(x)=\fr{1}{2}\,e^{-|x|}\,,\enskip
x\in\mathbb{R}\,.
\label{e6-kz}
\end{equation}
Лапласовская с.в.\ с~плот\-ностью~(\ref{e6-kz}) и~ее ф.р.\ будут соответственно
обозначаться~$\Lambda$ и~$F^{\Lambda}(x)$.

Случайная величина, имеющая распределение Линника с~параметром~$\alpha$, ее ф.р.\
и~плот\-ность будут соответственно обозначаться~$L_{\alpha}$,
$F_{\alpha}^{L}$ и~$f_{\alpha}^{L}$. При этом $F_2^{L}(x)\hm\equiv
F^{\Lambda}(x)$, $x\hm\in\mathbb{R}$.

Относительно недавно эти распределения и~их обобщения вновь
привлекли внимание исследователей как вполне адекватные модели
многих реальных явлений. Распределения Линника обладают многими
интересными свойствами. Лаха~\cite{Laha1961} (также см.~\cite{Lukacs1970}) 
доказал унимодальность распределений Линника. 
В~работах \cite{KotzOstrovskiiHayfavi1995a, KotzOstrovskiiHayfavi1995b} исследованы 
свойства плотности~$f_{\alpha}^{L}$. Показано, что для~$f_{\alpha}^{L}$ справедливо
интегральное представление
\begin{equation}
\hspace*{-2mm}f_{\alpha}^{L}=\fr{\sin(\pi\alpha/2)}{\pi}\!
\int\limits_{0}^{\infty}\!\!\fr{z^{\alpha}e^{-z|x|}\,dz}{1+z^{2\alpha}+2\cos(\pi\alpha/2)}\,,\enskip
x\in\mathbb{R}.\!\!\label{e7-kz}
\end{equation}
Сабу и~Пиллаи~\cite{SabuPillai1987} получили представление плотности~$f_{\alpha}^{L}$ 
в~терминах обобщенных $G$-функ\-ций Мейера. Лин~\cite{Lin1994} 
доказал саморазложимость~$F_{\alpha}^{L}$.
Существо\-вание моментов с.в.~$L_{\alpha}$ обсуждается в~работе~\cite{Anderson1992}. 
Абсолютные моменты порядков $\beta\hm<\alpha$ с.в.~$L_{\alpha}$ имеют вид:
$$
{\sf E}|L_{\alpha}|^{\beta}=\fr{2^{\beta}}{\sqrt{\pi}}\,
\fr{\Gamma(1+{\beta}/{\alpha})
\Gamma(({1+\beta})/{2})\Gamma(1-{\beta}/{\alpha})}
{\Gamma(1-{\beta}/{2})}\,.
$$
Распределения Линника безгранично делимы~\cite{Devroye1990}, имеют
бесконечный пик плотности в~нуле при $\alpha\hm\leqslant1$~\cite{Devroye1990}. 
В~работе~\cite{Jacquesetal1999} показано, что
при $0\hm<\alpha\hm<2$ хвосты распределения Лапласа убывают степенн$\acute{\mbox{ы}}$м
образом:
$$
\lim\limits_{x\to\infty}x^{\alpha}
\left[1-F^{L}_{\alpha}(x)\right]=\fr{\Gamma(\alpha)}{\pi}\sin\fr{\pi\alpha}{2}\,.
$$

Из представлений~(\ref{e1-kz}) и~(\ref{e4-kz}) вытекает, что \textit{распределения Линника
и только они являются симметричными геометрически устойчивыми
законами}.

В работах~\cite{KorolevZeifman2016a,
KorolevZeifman2016b, KotzOstrovskii1996, Pakes1998} получены разнообразные представления
распределений Линника в~виде смесей. Некоторые из этих представлений
будут приведены и~использованы ниже. Другие аналитические 
и~асимптотические свойства распределения Линника рассмотрены 
в~\cite{KotzOstrovskiiHayfavi1995a, KotzOstrovskiiHayfavi1995b}.

\section{Распределения Миттаг--Леффлера}

Пусть $\alpha\in(0,1)$ и~$M_{\alpha}$~--- неотрицательная с.в.\
с~преобразованием Лап\-ла\-са--Стиль\-тье\-са (п.~Л.--С.)
\begin{equation}
{\sf E}\exp\{-sM_{\alpha1}\}=\left(1+s^{\alpha}\right)^{-1}\,,\enskip
s\geqslant0\,.\label{e8-kz}
\end{equation}
Распределения с~п.~Л.--С.~(\ref{e8-kz}) принято называть \textit{распределениями
Мит\-таг--Леф\-фле\-ра}. Происхождение этого названия связано с~тем, что
плотность, соответствующая п.~Л.--С.~(\ref{e8-kz}), имеет вид:
\begin{multline}
f_{\alpha}^{M}(x)=\fr{1}{x^{1-\alpha}}\sum\limits_{n=0}^{\infty}
\fr{(-1)^nx^{\alpha n}}{\Gamma(\alpha n+1)}=-\fr{d}{dx}\,E_{\alpha}(-x^{\alpha})\,,\\
x\geqslant0\,,\label{e9-kz}
\end{multline}
где $E_{\alpha}(z)$~--- функция Мит\-таг--Леф\-фле\-ра индекса~$\alpha$,
определяемая как степенной ряд
$$
E_{\alpha}(z)=\sum\limits_{n=0}^{\infty}\fr{z^n}{\Gamma(\alpha
n+1)}\,,\enskip \alpha>0\,,\ z\in\mathbb{Z}\,.
$$
Функция распределения, соответствующая плотности~(\ref{e9-kz}), будет обозначаться
$F_{\alpha}^{M}(x)$.

Для ф.р.~$F_{\alpha}^{M}(x)$ при $x\hm>0$ справедливо интегральное
представление:
\begin{multline}
F_{\alpha}^{M}(x)=1-\int\limits_{0}^{\infty}e^{-xz}f_{\alpha,1}^{Q}(z)\,dz={}\\
{}=
1-\fr{\sin(\pi\alpha)}{\pi}\int\limits_{0}^{\infty}
\fr{z^{\alpha-1}e^{-zx}\,dz}{1+z^{2\alpha}+2z^{\alpha}\cos(\pi\alpha)}\,.
\label{e10-kz}
\end{multline}

При $\alpha=1$ распределение Мит\-таг--Леф\-фле\-ра превращается 
в~стандартное показательное распределение: $M_1\eqd W_1$. Но при
$\alpha\hm<1$ плотность~(\ref{e9-kz}) имеет хвост, убывающий степенн$\acute{\mbox{ы}}$м
образом: если $0\hm<\alpha\hm<1$, то
$$
\lim\limits_{x\to\infty}x^{\alpha+1}f_\alpha^{M}(x)=
\fr{\Gamma(\alpha+1)}{\pi}\sin\pi\alpha
$$
(см., например,~\cite{Kilbas2014}).

Моменты с.в.\ с~распределением Мит\-таг--Леф\-фле\-ра порядков
$\beta\hm\geqslant\alpha$ бесконечны, но при $0\hm<\beta\hm<\alpha\hm<1$
$$
{\sf E}M_{\alpha}^{\beta}=
\Gamma\left(1+\fr{\beta}{\alpha}\right)\Gamma\left(1-\fr{\beta}{\alpha}\right)\,.
$$

Распределение Мит\-таг--Леф\-фле\-ра геометрически устойчиво. Еще в~1965~г.\
И.\,Н.~Коваленко~\cite{Kovalenko1965} показал, что распределения с~п.~Л.--С.~(\ref{e8-kz}) 
и~только они являются возможными предельными
распределениями для надлежащим образом нормированных геометрических
сумм вида $a_p(X_1+\cdots+X_{V_p})$ независимых
неотрицательных с.в.\ при $p\hm\to0$. Доказательства этого результата
были воспроизведены в~книгах~\cite{GnedenkoKorolev1996, GnedenkoKovalenko1968,
GnedenkoKovalenko1989}, где вместо
термина <<распределения Мит\-таг-Леф\-фле\-ра>> класс распределений 
с~п.~Л.--С.~(\ref{e8-kz}) был назван \textit{классом}~$\mathcal{K}$ в~честь И.\,Н.~Коваленко.

Спустя 25~лет упомянутое предельное свойство
распределений с~п.~Л.--С.~(\ref{e8-kz}) было переоткрыто 
Р.~Пиллаи~\cite{Pillai1989, Pillai1990}, который предложил для них
использовать термин \textit{распределения Мит\-таг--Леф\-фле\-ра}, ставший
общепринятым.

Распределения Мит\-таг--Леф\-фле\-ра используются при описании аномальной
диффузии или эффектов релаксации (см.~\cite{WeronKotulski1996,
GorenfloMainardi2006} и~дальнейшие ссылки в~этих работах).

При каждом $\alpha\hm\in(0,1]$ распределение Мит\-таг--Леф\-фле\-ра является
смешанным показательным распределением:
$$
M_{\alpha}\eqd W_1 \fr{S_{\alpha,1}}{S'_{\alpha,1}}\,,
$$
где $S'_{\alpha,1}\eqd S_{\alpha,1}$ и~все с.в.\ в~правой части
независимы. Доказательство этого факта можно найти 
в~работах~\cite{KorolevZeifman2016a, KorolevZeifman2016b, KotzOstrovskii1996},
где, в~част\-ности, показано, что плотность $p_{\alpha}(x)$ отношения
$S_{\alpha,1}/S'_{\alpha,1}$ двух независимых односторонних строго
устойчивых с.в.\ с~характеристическим показателем $\alpha\hm\in(0,1)$
имеет вид:
\begin{equation}
p_{\alpha}(x)=\fr{\sin(\pi\alpha)x^{\alpha-1}}
{\pi[1+x^{2\alpha}+2x^{\alpha}\cos(\pi\alpha)]}\,,\enskip
x>0\,.\label{e11-kz}
\end{equation}

Для дальнейшего важно подчеркнуть, что \textit{распределения
Мит\-таг--Леф\-фле\-ра и~только они являются геометрически устойчивыми
законами, сосредоточенными на неотрицательной полуоси}.

\section{Традиционный подход к~определению несимметричных
распределений Линника}

Канонический вид~(\ref{e4-kz}) х.ф.\ геометрически устойчивого распределения
обусловлен определением последнего и~теоремой переноса
Гне\-ден\-ко--Фа\-хи\-ма: пусть $\{X_{n,j}\}_{j\hm\in\mathbb{N}}$,
$n\hm\in\mathbb{N}$,~--- последовательность серий независимых 
и~одинаково в~каждой серии распределенных с.в.,
$\{N_n\}_{n\in\mathbb{N}}$~--- последовательность неотрицательных
целочисленных с.в., при каждом $n\hm\geqslant1$ независимых от
$X_{n,1},X_{n,2},\ldots$ Для $k\hm\in\mathbb{N}\cup\{0\}$ обозначим
$S_{n,k}=X_{n,1}+\cdots+X_{n,k}$ ($S_{n,0}\hm=0$). Согласно теореме
переноса Гне\-ден\-ко--Фа\-хи\-ма~\cite{GnedenkoFahim1969} (также см.,
например,~\cite{GnedenkoKorolev1996}), если существует
последовательность $\{k_n\}_{n\in\mathbb{N}}$ натуральных чисел и~с.в.~$Y$ 
и~$N$ такие, что при $n\hm\to\infty$ $S_{n,k_n}\hm \Longrightarrow Y$ и
\begin{equation}
\fr{N_n}{k_n}\Longrightarrow N\,,\label{e12-kz}
\end{equation}
то $ S_{n,N_n}\hm\Longrightarrow Z$, где $Z$~--- с.в.\ с~х.ф.
$$
\mathfrak{f}(t)=\int\limits_{0}^{\infty}\left(\mathfrak{h}(t)\right)^u\,dA(u)\,,\enskip
t\in\mathbb{R}\,.
$$
Здесь $\mathfrak{h}(t)$~--- х.ф.\ с.в.~$Y$; $A(u)\hm={\sf P}(N<u)$.

Если п.~Л.--С. ${\sf E}e^{-sN}$ с.в.~$N$ обозначить~$\psi_N(s)$,
$s\hm\geqslant0$, то х.ф.~$\mathfrak{f}(t)$ можно записать в~виде:
$$
\mathfrak{f}(t)=\psi_N(-\log\mathfrak{h}(t))\,,\enskip t\in\mathbb{R}\,.
$$
При $k_n=n$, $N_n\eqd V_{1/n}$ в~силу теоремы Реньи~\cite{Renyi1956}
в~(\ref{e12-kz}) имеем $N\eqd W_1$. При этом $\psi_N(s)\hm=(1\hm+s)^{-1}$, так что 
в~таком случае
$$
\mathfrak{f}(t)=\psi(-\log\mathfrak{h}(t))=\left(1-\log\mathfrak{h}(t)\right)^{-1}\,,\enskip
t\in\mathbb{R}\,.
$$
Если, более того, $X_{n,j}\eqd n^{-1/\alpha}X_j$ для всех
$n,j\hm\in\mathbb{N}$, где $X_1,X_2,\ldots$~--- независимые одинаково
распределенные с.в.\ с~$\mathcal{L}(X_1)\hm\in \mathrm{DNA}(G_{\alpha,\theta})$
при некоторых допустимых значениях~ $\alpha$ и~$\theta$, то х.ф.~$\mathfrak{f}(t)$, 
предельная для геометрических случайных сумм,
имеет вид~(\ref{e4-kz}).
%
Поэтому базирующаяся на схеме \textit{геометрического} случайного
суммирования и~традиционном представлении о~распределениях Линника
как симметричных \textit{геометрически устойчивых} законах идея
несимметричного их обобщения заключается в~замене
$g_{\alpha,0}(t)\hm=e^{-|t|^{\alpha}}$ в~(\ref{e4-kz}) на $g_{\alpha,\theta}(t)$
с~произвольным допустимым~$\theta$, что приводит к~распределениям с~х.ф.\ вида:
$$
\widetilde{\mathfrak{f}}^{L}_{\alpha}(t)=
\left(1+|t|^{\alpha}\exp\left\{-
\fr{1}{2}\,i\pi\theta\alpha\,\mathrm{sign}\,t\right\}\right)^{-1}\,,\enskip
t\in\mathbb{R}
$$
(см., например,~\cite{KotzOstrovskiiHayfavi1995a,
KotzOstrovskiiHayfavi1995b}). Такие распределения будем называть 
\textit{несимметричными распределениями Линника первого рода}. Бо\-лее-ме\-нее
полная библиография по этой теме приведена в~ \cite{LimTeo2009}.

Другими словами, традиционно несимметричное обобщение распределений
Линника достигается за счет того, что в~\textit{геометрической}
случайной сумме рассматриваются скошенные слагаемые, распределения
которых принадлежат области нормального притяжения несимметричного
устойчивого закона.

\section{Представление распределений Линника в~виде масштабных
смесей нормальных или~ лапласовых законов}

Распределения Линника допускают представление в~виде масштабных
смесей нормальных законов. Это означает, что они могут быть
предельными в~аналогах центральной предельной теоремы для случайных
сумм независимых с.в.\ с~\textit{конечными} 
дисперсиями~\cite{KorolevZeifman2016a, KorolevZeifman2016b}, что открывает пути
альтернативных несимметричных обобщений этих распределений.

В работах~\cite{KorolevZeifman2016b, KotzOstrovskii1996} установлена
интересная связь между распределениями Линника, Лапласа 
и~Мит\-таг--Леф\-фле\-ра и~показано, что
\begin{equation}
L_{\alpha}\eqd X\sqrt{2M_{\alpha/2}}\eqd
\Lambda\sqrt{\fr{S_{\alpha/2,1}}{S'_{\alpha/2,1}}}\,,
\label{e13-kz}
\end{equation}
где все сомножители независимы и~$S'_{\alpha/2,1}\eqd
S_{\alpha/2,1}$.

Как уже отмечалось, все распределения Линника и~только они являются
симметричными геометрически устойчивыми законами. Все распределения
Мит\-таг--Леф\-фле\-ра и~только они являются односторонними геометрически
устойчивыми законами. С~учетом упоминавшегося выше соотношения
$S_{2,0}\eqd\sqrt{2}X$ левое равенство~(\ref{e13-kz}) можно переписать в~виде
$L_{\alpha}\eqd S_{2,0}\sqrt{M_{\alpha/2}}$, являющемся частным
случаем более общего утверждения, доказанного в~работе~\cite{KorolevZeifman2016b}: 
если $\alpha\hm\in(0,2]$ и~$\alpha'\hm\in(0,1]$, то справедливо соотношение
$$
L_{\alpha\alpha'}\eqd S_{\alpha,0}M_{\alpha'}^{1/\alpha}\,,
$$
представляющее собой аналог <<теоремы умножения>> устойчивых
с.в.\ (см., например, теорему~3.3.1.\
в~\cite{Zolotarev1983}), в~классе геометрически устойчивых
распределений. Это интересная иллюстрация изоморфизма класса
геометрически устойчивых распределений классу устойчивых
распределений, установленного в~работе~\cite{KlebanovManiaMelamed1984}.

Представление~(\ref{e13-kz}) означает, что схема гео\-мет\-ри\-че\-ско\-го случайного
суммирования отнюдь не исчерпывает все возможные предельные
постановки задач для случайных сумм и~других последовательностей 
с~независимыми случайными индексами, в~которых распределение Линника
может выступать в~качестве предельного. Соответствующие примеры
предельных теорем приведены в~\cite{KorolevZeifman2016b}.

\section{Несимметричные распределения Линника
как~масштабные смеси несимметричных распределений Лапласа}

В этом разделе будет реализован формальный подход к~несимметричному
обобщению распределения Линника, для чего будет использовано правое
равенство~(\ref{e13-kz}). В~результате будет получено несимметричное
распределение, каждая ветвь которого (положительная и~отрицательная)
будут копиями соответствующих ветвей \textit{разных} распределений
Лап\-ласа.

Пусть $a_1$ и~$a_2$~--- два положительных числа. Будем говорить, что
с.в.~$\Lambda_{a_1,a_2}$ имеет \textit{несимметричное распределение
Лапласа с~параметрами~$a_1$ и~$a_2$}, если ее ф.р.\ имеет вид:
\begin{multline*}
F^{\Lambda}_{a_1,a_2}(x)\equiv {\sf P}(\Lambda_{a_1,a_2}<x) ={}\\
{}=
\begin{cases}
\displaystyle\fr{a_1 }{a_1+a_2}\, e^{-a_2|x|}\,, & x\leq0\,; \\
\displaystyle 1-\fr{a_2}{a_1+a_2}\, e^{-a_1x}\,, & x> 0\,.
\end{cases}
\end{multline*}
Несложно видеть, что плотность $f^{\Lambda}_{a_1,a_2}(x)$,
соответствующая ф.р.~$F^{\Lambda}_{a_1,a_2}(x)$, имеет вид:
$$
f^{\Lambda}_{a_1,a_2}(x)=
\begin{cases}
\displaystyle\fr{a_1 a_2}{a_1+a_2}\, e^{a_2
x}\,,& x\leqslant 0\,;\\
\displaystyle \fr{a_1 a_2}{a_1+a_2}\, e^{-a_1 x}\,,&
x>0\,.
\end{cases}
$$
Несимметричное распределение Лапласа является популярной моделью,
широко используемой в~разных областях (см., например,~\cite{Kotz2001}). 
Следующая\linebreak лемма, доказательство которой можно
найти, например, в~\cite{KorolevKurmangazievaZeifman2016},
утверждает, что это распределение является специальной
дис\-пер\-си\-он\-но-сдви\-го\-вой смесью нормальных законов.

\smallskip

\noindent
\textbf{Лемма~2.}\ \textit{Пусть $\mu\hm\in\mathbb{R}$, $\sigma^2\hm\in(0,\infty)$,
$\lambda\hm\in(0,\infty)$. Предположим, что с.в.~$Y$ допускает
представление}
$$
Y\eqd \fr{\sigma}{\sqrt{\lambda}}\,
X\sqrt{W_1}+\mu \fr{W_1}{\lambda}\,,
$$
\textit{где с.в.~$X$ имеет стандартное нормальное распределение, 
с.в.~$W_1$ имеет стандартное показательное распределение $($т.\,е.\ 
с.в.~$W_1/\lambda$ имеет показательное распределение с~параметром~$\lambda)$, 
причем с.в.~$X$ и~$W_1$ независимы. Тогда
$Y\eqd\Lambda_{a_1,a_2}$, т.\,е.}
$$
{\sf P}(Y<x)={\sf E}\Phi\left(\fr{\lambda x-\mu
W_1}{\sigma\sqrt{\lambda W_1}}\right)=F^{\Lambda}_{a_1,a_2}(x)\,,\enskip
x\in\mathbb{R}\,,
$$
\textit{где}
$$
a_1=\fr{1}{\sqrt{\mu^2+2\lambda\sigma^2}+\mu}\,;\enskip
a_2=\fr{1}{\sqrt{\mu^2+2\lambda\sigma^2}-\mu}\,.
\label{e14-kz}
$$

\smallskip

Обозначим $Q_{\alpha/2}\eqd\sqrt{S_{\alpha/2,1}/S'_{\alpha/2,1}}$,
где с.в.~$S_{\alpha/2,1}$ и~$S'_{\alpha/2,1}$ независимы и~имеют
одинаковое одностороннее устойчивое распределение 
с~характеристическим показателем~$\alpha/2$. Легко видеть, что
$Q_{\alpha/2}\eqd Q^{-1}_{\alpha/2}$. Теперь из правого равенства~(\ref{e13-kz}), 
(\ref{e11-kz}) и~(\ref{e7-kz}) вытекает, что для любых $\alpha\hm\in(0,2]$ и~$y\hm\geqslant 0$
\begin{multline*}
{\sf P}\left(\Lambda_{a_1,a_2} Q_{\alpha/2}>y\right)= {\sf P}
\left(\Lambda_{a_1,a_2}>y Q_{\alpha/2}\right)={}
\\
{}=
\fr{a_2\sin(\pi\alpha/2)}{\pi(a_1+a_2)}\int\limits_{0}^{\infty}
\fr{z^{\alpha}e^{-a_1yz}\,dz}{1+z^{2\alpha}+2x^{\alpha}\cos(\pi\alpha/2)}={}\\
{}=
1-\fr{a_2}{a_1+a_2}\,F^{L}_{\alpha}\left(a_1y\right)\,,
\end{multline*}
а для $y<0$
\begin{multline*}
{\sf P}\left(\Lambda_{a_1,a_2} Q_{\alpha/2}<y\right)= {\sf P}
\left(\Lambda_{a_1,a_2}<y Q_{\alpha/2}\right)={}
\\{}=
\fr{a_1\sin(\pi\alpha/2)}{\pi(a_1+a_2)}\int\limits_{0}^{\infty}
\fr{z^{\alpha}e^{-a_2|y|z}\,dz}{1+z^{2\alpha}+2x^{\alpha}\cos(\pi\alpha/2)}={}\\
{}=
\fr{a_1}{a_1+a_2}F^{L}_{\alpha}\left(a_2y\right)\,.
\end{multline*}
Таким образом, естественно получено формальное несимметричное
обобщение распределения Линника.

\smallskip

\noindent
\textbf{Определение~1.}\ Пусть $a_1$ и~$a_2$~--- два положительных
числа, $\alpha\hm\in(0,2]$. Будем говорить, что с.в.~$\widehat
L_{\alpha;a_1,a_2}$ имеет \textit{несимметричное распределение Линника
второго рода с~параметрами~$\alpha$, $a_1$ и~$a_2$}, если ее ф.р.\
имеет вид:

\noindent
\begin{multline*}
\widehat F^{L}_{\alpha;a_1,a_2}(x)\equiv {\sf P}\left(
\widehat L_{\alpha;a_1,a_2}<x\right) ={}\\
{}=
\begin{cases}
\displaystyle\fr{a_1 }{a_1+a_2}\, F^{L}_{\alpha}\left(a_2x\right)\,, & x\leq0\,;\\
\displaystyle 1-\fr{a_2}{a_1+a_2}\, F^{L}_{\alpha}\left(a_1x\right)\,, &
x> 0\,.
\end{cases}
\end{multline*}

\smallskip

Необходимо особо отметить, что с~формальной точки зрения
несимметричное распределение Линника второго рода является
специальной сдвиг-мас\-шаб\-ной смесью нормальных законов. Более того,
справедливо следующее утверждение.

\smallskip

\noindent
\textbf{Теорема~1.}\ \textit{Пусть $\alpha\hm\in(0,1]$, $\mu\hm\in\mathbb{R}$,
$\sigma^2\hm\in(0,\infty)$, $\lambda\hm\in(0,\infty)$. Предположим, что с.в.~$Z$ 
допускает представление}
$$
Z\eqd \left(\fr{\sigma}{\sqrt{\lambda}}\, X\sqrt{W_1}+\fr{\mu
W_1}{\lambda}\right) Q_{\alpha/2}\,,
$$
\textit{где с.в.~$X$, $W_1$, $Q_{\alpha/2}$ независимы, с.в.~$X$ имеет
стандартное нормальное распределение, с.в.~$W_1$ имеет стандартное
показательное распределение $($т.\,е.\ с.в.~$W_1/\lambda$ имеет
показательное распределение с~параметром~$\lambda)$. Тогда $Z\eqd
\widehat L_{\alpha;a_1,a_2}$, т.\,е.}
\begin{multline*}
{\sf P}(Z<x)={\sf E}\Phi\left(\fr{\lambda x-\mu W_1
Q_{\alpha/2}}{\sigma\sqrt{\lambda W_1}Q_{\alpha/2}}\right)={}\\
{}={\sf P}
\left(\widehat L_{\alpha;a_1,a_2}<x\right)=\widehat F^{L}_{\alpha;a_1,a_2}(x)\,,\enskip
x\in\mathbb{R}\,,
\end{multline*}
\textit{где $a_1$ и~$a_2$ имеют вид}~(\ref{e14-kz}).

\section{Сходимость распределений дважды случайных сумм независимых одинаково
распределенных случайных величин к~несимметричному распределению
Линника второго рода}

Приведем пример предельной схемы типа <<рандомизированного>> закона
больших чисел для сумм независимых с.в.\ с~\textit{конечными
математическими ожиданиями}, в~которой в~качестве предельных
возникают несимметричные распределения Линника второго рода. Из-за
непростой связи смешива\-ющих с.в.\ в~случайном сдвиге и~случайном
изменении масштаба в~соответствующей смеси (см.\ теорему~1), обычной
схемы случайного суммирования для этой цели недостаточно 
и~приходится вводить\linebreak\vspace*{-12pt}

\columnbreak

\noindent
 в~модель дополнительный источник случайности 
и~рас\-смат\-ри\-вать так называемые \textit{дважды случайные} суммы.
{\looseness=1

}


Пусть $a_1$ и~$a_2$~--- два конечных положительных числа. Пусть
$X^{(1)}_1,X^{(1)}_2,\ldots$~--- независимые одинаково распределенные
с.в.\ такие, что ${\sf E}X^{(1)}_1\hm=a_2^{-1}$,
$X^{(2)}_1,X^{(2)}_2,\ldots$~--- независимые одинаково распределенные
с.в.\ такие, что ${\sf E}X^{(2)}_1\hm=a_1^{-1}$. Тогда по закону
больших чисел
\begin{equation}
\fr{1}{n}\sum\limits_{j=1}^n X_j^{(1)}\Longrightarrow\fr{1}{a_2}\,;\quad
\fr{1}{n}\sum\limits_{j=1}^n X_j^{(2)}\Longrightarrow\fr{1}{a_1}\label{e15-kz}
\end{equation}
при $n\to\infty$.

Пусть $V_{1/n}^{(1)}$ и~$V_{1/n}^{(2)}$~--- с.в.\ с~одинаковым
геометрическим распределением~(\ref{e3-kz}) с~параметром $p\hm=1/n$. Будем
считать, что с.в.~$V_{1/n}^{(1)},
V_{1/n}^{(2)},X^{(1)}_1,X^{(1)}_2,\ldots,X^{(2)}_1,X^{(2)}_2,\ldots$
независимы.\ \  Для каждого $n\hm\in\mathbb{N}$ введем с.в.
$$
Y_n=
\begin{cases}
\displaystyle\sum\limits_{j=1}^{V_{1/n}^{(1)}}X^{(1)}_j
& \mbox{с~вероятностью }
\displaystyle\fr{a_2}{a_1+a_2}\,;\\
\displaystyle -\sum\limits_{j=1}^{V_{1/n}^{(2)}}X^{(2)}_j &
\mbox{с~вероятностью }\displaystyle\fr{a_1}{a_1+a_2}\,.
\end{cases}
$$
Тогда по теореме Реньи из~(\ref{e15-kz}) вытекает, что при $n\hm\to\infty$
\begin{equation}
\fr{Y_n}{n}\Longrightarrow\Lambda_{a_1,a_2}\,.\label{e16-kz}
\end{equation}
Пусть теперь $\alpha\hm\in(0,2]$ и~$N_n$~--- целочисленная
не\-от\-ри\-ца\-тель\-ная с.в., независимая от последовательности
$Y_1,Y_2,\ldots$ и~такая, что
\begin{equation}
\fr{N_n}{n}\Longrightarrow Q_{\alpha/2}\label{e17-kz}
\end{equation}
при $n\to\infty$. Такая с.в.\ может быть построена, например,
следующим образом. Пусть~$P(t)$, $t\geqslant\linebreak \geqslant 0$,~--- стандартный
пуассоновский процесс (пуассоновский процесс с~единичной
интенсивностью),\linebreak независимый от с.в.~$Q_{\alpha/2}$. Положим $N_n\hm=
P(nQ_{\alpha/2})$. Несложно убедиться, что такие с.в.~$N_n$
удовлетворяют~(\ref{e17-kz}).

Тогда по лемме~1 из~(\ref{e16-kz}) и~(\ref{e17-kz}) вытекает, что
$$
\fr{Y_{N_n}}{n}\Longrightarrow\Lambda_{a_1,a_2}
Q_{\alpha/2}\eqd \widehat L_{\alpha;a_1,a_2}\,.
$$

\section{Несимметричные распределения Линника как~дисперсионно-сдвиговые
смеси нормальных законов}

Хотя несимметричное распределение Линника второго рода 
$\widehat F^{L}_{\alpha;a_1,a_2}(x)$, введенное в~предыдущем разделе, является
специальной сдвиг-мас\-штаб-\linebreak ной смесью нормальных законов, пример
предельной схемы, скажем, для сумм случайного числа независимых
с.в., в~которой такое распределение\linebreak возникает 
в~качестве предельного, не так прост, поскольку параметры, по которым
происходит смешивание, связаны нетривиальным образом. 
{\looseness=1

}

Однако
возможен еще один подход к~определению %\linebreak 
несимметричного распределения
Линника, для %\linebreak 
которого подобная предельная схема строится довольно
просто. 
%
Этот подход основан на левом равенстве~(\ref{e13-kz})~--- представлении
распределения Линника в~виде масштабной смеси нормальных %\linebreak
 законов, 
в~которой смешивающим является распределение Мит\-таг--Леф\-фле\-ра. 

В~рамках
описываемого подхода несимметричное обобщение достигается за счет
рассмотрения дис\-пер\-си\-он\-но-сдви\-го\-вых смесей нормальных законов вместо
чисто масштабных смесей.

Вероятностные модели типа дис\-пер\-си\-он\-но-сдви\-го\-вых смесей нормальных
законов рассматриваются в~качестве базовых во многих практических
задачах. 

Подобные модели уже хорошо себя зарекомендовали во многих
исследованиях, где они продемонстрировали очень высокую
адекватность. Последнее обстоятельство можно легко объяснить\linebreak
довольно большим числом настраиваемых па\-ра\-мет\-ров в~указанных
моделях. Однако на самом деле их адекватность имеет гораздо более
глубокие теоретические обоснования, а~именно: дис\-пер\-си\-он\-но-сдви\-го\-вые
смеси нормальных законов являются предельными законами в~довольно
прос\-тых предельных теоремах для случайно остановленных случайных
блуж\-да\-ний. 

Такие теоремы позволяют однозначно связать конкретный
смешивающий закон в~дис\-пер\-си\-он\-но-сдви\-го\-вых смесях с~поведением
интенсивности потока информативных событий, в~результате которых
накапливаются данные, характеризующие анализируемый случайный
процесс. Тем самым эти теоремы как бы позволяют разделить вклады
внешних и~внутренних факторов в~случайность поведения анализируемого
процесса.

\columnbreak

Понятие дисперсионно-сдвиговой смеси нормальных законов (normal
variance-mean mixture) введено в~1970--1980-х~гг.\
 в~работах О.-Е.~Барн\-дорфф-Ниль\-се\-на и~его 
коллег~[43--45] как довольно гибкое обобщение
нормального распреде\-ления.
{\looseness=1

}

Пусть $\beta\in\mathbb{R}$, $\alpha\hm\in\mathbb{R}$,
$0\hm<\sigma\hm<\infty$, $A(x)$~--- функция распределения, все точки роста
которой сосредоточены на~$\mathbb{R}_+$. Дис\-пер\-си\-он\-но-сдви\-го\-вой
смесью нормальных законов называется ф.р.
\begin{equation}
F(x)=\int\limits_{0}^{\infty}\Phi\left(\fr{x-\beta-\alpha
z}{\sigma\sqrt{z}}\right)\,dA(z)\,,\enskip x\in\mathbb{R}\,.
\label{e18-kz}
\end{equation}

Обратим внимание, что в~соотношении~(\ref{e18-kz}) смешивание происходит
одновременно и~по па\-ра\-мет\-ру сдвига, и~по параметру масштаба, но так
как эти параметры в~(\ref{e18-kz}) связаны жесткой за\-ви\-си\-мостью, при которой
параметры положения (\textit{сдвига}) смешиваемых нормальных законов
пропорциональны их \textit{дисперсиям}, то фактически смесь~(\ref{e18-kz})
является однопараметрической. Именно поэтому смеси вида~(\ref{e18-kz})
называются \textit{дис\-пер\-си\-он\-но-сдви\-го\-выми}. 
{\looseness=1

}

Класс
дис\-пер\-си\-он\-но-сдви\-го\-вых смесей нор\-маль\-ных законов обширен и~содержит,
в~част\-ности, обобщенные гиперболические законы \mbox{[43--45]}
и~обобщенные дисперсионные гам\-ма-рас\-пре\-де\-ле\-ния~\cite{KorolevSokolov2012, KorolevZaks2013}, 
демонстрирующие отличное
согласие со статистическими данными из самых разных областей~--- от
атмосферной турбулентности до финансовых рынков.

Без существенного ограничения общности для простоты далее будем
считать, что $\beta\hm=0$.

\bigskip

\noindent
\textbf{Определение~2.}\ Пусть $\mu\hm\in\mathbb{R}$, $\sigma\hm>0$,
$\alpha\hm\in(0,2]$. Будем говорить, что с.в.~$\widetilde L_{\alpha;\mu,\sigma}$ 
имеет несимметричное распределение Линника
третьего рода с~параметрами~$\alpha$, $\mu$ и~$\sigma$, если ее 
ф.р.~$\widetilde F^{L}_{\alpha;\mu,\sigma}(x)$ является
дис\-пер\-си\-он\-но-сдви\-го\-вой смесью нормальных законов вида:
$$
\widetilde F^{L}_{\alpha;\mu,\sigma}(x)=
\int\limits_{0}^{\infty}\Phi\left(\fr{x-\mu z}
{\sigma\sqrt{z}}\right)\,dF^{M}_{\alpha/2}(z)\,,\enskip x\in\mathbb{R}\,,
$$
где смешивающая ф.р.\ Мит\-таг--Леф\-фле\-ра~$F^{M}_{\alpha/2}$ имеет вид~(\ref{e10-kz}).

\smallskip

Из представления~(\ref{e13-kz}) вытекает, что при $\mu\hm=0$ несимметричное
распределение Линника третьего рода превращается в~обычное
симметричное распределение Линника.

\section{Сходимость распределений случайных сумм независимых одинаково
распределенных случайных величин к~несимметричному распределению
Линника третьего рода}

Пусть $\{X_{n,j}\}_{j\geqslant1}$, $n\hm=1,2,\ldots$,~--- последовательность
серий одинаково в~каждой серии распределенных с.в. Пусть~$\{N_n\}_{n\geqslant1}$~--- 
последовательность неотрицательных
целочисленных с.в.\ таких, что при каждом $n\hm\geqslant1$ 
с.в.~$N_n,X_{n,1},X_{n,2},\ldots$ независимы. Напомним, что используется
обозначение $S_{n,k}\hm=X_{n,1}+\cdots +X_{n,k}$, $n,k\in\mathbb{N}$. 
В~работе~\cite{Korolev2013} доказано следующее утверждение.

\smallskip

\noindent
\textbf{Лемма~3.}\ \textit{Предположим, что существуют последовательность
натуральных чисел~$\{k_n\}_{n\geqslant 1}$ и~числа $\mu\hm\in\mathbb{R}$ 
и~$\sigma\hm\in(0,\infty)$ такие, что}
\begin{equation}
{\sf P}\left(S_{n,k_n}<x\right)\Longrightarrow
\Phi\left(\fr{x-\mu}{\sigma}\right)\,.\label{e19-kz}
\end{equation}
\textit{Предположим, что $N_n\hm\to\infty$ по вероятности. Тогда распределения
случайных сумм~$S_{N_n}$ независимых одинаково распределенных с.в.\
слабо сходятся к~некоторой ф.р.~$F(x)$}:
$$
{\sf P}\left(S_{n,N_n}<x\right)\Longrightarrow F(x)\,,
$$
\textit{если и~только если существует ф.р.~$H(x)$ такая, что $H(0)\hm=0$},

\noindent
\begin{gather*}
F(x)=\int\limits_{0}^{\infty}\Phi\left(\fr{x-\mu z}{\sigma\sqrt{z}}\right)\,dH(z)\,;
\\
%\textit{и}
%$$
{\sf P}\left(N_n<xk_n\right)\Longrightarrow H(x)\,.
\end{gather*}


Из леммы~3 и~определения~2 непосредственно вытекает следующее
утверждение, устанав\-ли\-ва\-ющее необходимые и~достаточные условия
сходимости распределений случайных сумм независимых одинаково
распределенных с.в.\ c~\textit{конечными дис\-пер\-си\-ями} к~несимметричному
распределению Линника третьего рода.

\smallskip

\noindent
\textbf{Теорема~2.}\ \textit{Предположим, что существуют последовательность
натуральных чисел~$\{k_n\}_{n\geqslant1}$ и~числа $\mu\hm\in\mathbb{R}$ 
и~$\sigma\hm>0$ такие, что имеет место сходимость}~(\ref{e19-kz}). \textit{Предположим,
что $N_n\hm\to\infty$ по вероятности. Тогда распределения случайных
сумм~$S_{N_n}$ независимых одинаково распределенных с.в.\ слабо
сходятся к~несимметричному распределению Линника третьего рода
$\widetilde F^{L}_{\alpha;\mu,\sigma}$ при некотором}
$\alpha\hm\in(0,2]$:
$$
{\sf P}\left(S_{n,N_n}<x\right)\Longrightarrow \widetilde
F^{L}_{\alpha;\mu,\sigma}(x)\,,
$$

\noindent
\textit{если и~только если}
$$
{\sf P}\left(N_n<xk_n\right)\Longrightarrow F^{M}_{\alpha/2}(x)\,,
$$
\textit{где ф.р.\ Мит\-таг--Леф\-фле\-ра~$F^{M}_{\alpha/2}$ имеет вид}~(\ref{e10-kz}).

\vspace*{3pt}

Примеры индексов~$N_n$, удовлетворяющих условию теоремы~2, приведены
в~\cite{KorolevZeifman2016b}.

\vspace*{-12pt}


{\small\frenchspacing
 {%\baselineskip=10.8pt
 \addcontentsline{toc}{section}{References}
 \begin{thebibliography}{99}
\bibitem{MittnikRachev1991} 
\Au{Mittnik S., Rachev S.} Modeling asset returns with alternative stable models~// 
Economet. Rev., 1993. Vol.~12. P.~261--330.

\bibitem{Kotz2001} 
\Au{Kotz S., Kozubowski~T.\,J., Podgorski~K.} 
The Laplace distribution and generalizations: A~revisit with 
applications to communications, economics, engineering, and finance.~--- 
Boston: Birkhauser, 2001. 349~p.

\bibitem{KorolevZeifman2016a}
\Au{Korolev V.\,Yu., Zeifman~A.\,I.} 
A~note on mixture representations for the Linnik and Mittag--Leffler 
distributions and their applications~// J.~Math. Sci., 2016. Vol.~218. P.~314--327.

\bibitem{KorolevZeifman2016b} 
\Au{Korolev V.\,Yu., Zeifman~A.\,I.} 
Convergence of random sums and statistics constructed from samples with random sizes to the Linnik 
and Mittag--Leffler distributions and their generalizations~// 
J.~Korean Stat. Soc., 2016.  
arXiv:1602.02480v1.

\bibitem{Zolotarev1983} 
\Au{Золотарев В.\,М.} Одномерные устойчивые распределения.~--- М.: Наука, 1983.
304~с.

\bibitem{Schneider1986} 
\Au{Schneider W.\,R.} Stable distributions: Fox function representation and 
generalization~// Stochastic processes in classical and quantum systems~/ 
Eds. S.~Albeverio, G.~ Casati, D.~Merlini.~--- Berlin: Springer, 1986. P.~497--511.

\bibitem{UchaikinZolotarev1999} 
\Au{Uchaikin V.\,V., Zolotarev~V.\,M.} Chance and stability.~--- Utrecht: VSP, 1999.
570~p.

\bibitem{KorolevWeibull2016} 
\Au{Korolev V.\,Yu.} Product representations for random variables with the
 Weibull distributions and their applications~// J.~Math. Sci., 2016. Vol.~218. No.\,3. P.~298--313.

\bibitem{Tucker1975} 
\Au{Tucker H.} On moments of distribution functions attracted to stable laws~// 
Houston J.~Math., 1975. Vol.~1. No.\,1. P.~149--152.

\bibitem{KlebanovManiaMelamed1984} 
\Au{Клебанов Л.\,Б., Мания~Г.\,М., Меламед~И.\,А.} Одна задача В.\,М.~Золотарева 
и~аналоги безгранично делимых и~устойчивых распределений в~схеме суммирования 
случайного числа случайных величин~// Теория вероятностей и~ее применения, 1984. 
Т.~29. Вып.~4. С.~791--794.

\bibitem{KlebanovRachev1996} 
\Au{Klebanov L.\,B., Rachev~S.\,T.} Sums of a~random number of random variables 
and their approximations with $\varepsilon$-accompanying infinitely divisible laws~// 
Serdica, 1996. Vol.~22. P.~471--498.

\bibitem{Bunge1996} 
\Au{Bunge J.} Compositions semigroups and random stability~// 
Ann. Probab., 1996. Vol.~24. P.~1476--1489.

%\pagebreak

\bibitem{Rachev1991} 
\Au{Rachev S.\,T.} Probability metrics and the stability of stochastic models.~--- 
 Chichester--New York: Wiley, 1991. 494~p.
 
 \pagebreak

\bibitem{GnedenkoKorolev1996} 
\Au{Gnedenko B.\,V., Korolev~V.\,Yu.} Random summation: Limit theorems and 
applications.~--- Boca Raton: CRC Press, 1996. 267~p.

\bibitem{Korolev1994} 
\Au{Королев В.\,Ю.} Сходимость случайных последовательностей с~независимыми случайными 
индексами.~I~// Теория вероятностей и~ее применения, 1994. Т.~39. Вып.~2. С.~313--333.

\bibitem{Korolev1995} 
\Au{Королев В.\,Ю.} Сходимость случайных последовательностей с~независимыми 
случайными индексами.~II~// Теория вероятностей и~ее применения, 1995. Т.~40. Вып.~4. С.~907--910.

\bibitem{BeningKorolev2002} 
\Au{Bening V.\,E., Korolev~ V\, Yu.} 
Generalized Poisson models and their applications in insurance and finance.~--- 
Utrecht: VSP, 2002. 434~p.

\bibitem{Pillai1985} 
\Au{Pillai R.\,N.}  Semi-$\alpha$-Laplace distributions~//
Commun. Stat. Theor. Meth., 1985. Vol.~14. P.~991--1000.

\bibitem{Linnik1953} 
\Au{Линник Ю.\,В.} Линейные формы и~статистические критерии.~I,~II~// 
Украинский математический журнал, 1953. Т.~5. Вып.~2. С.~207--243; 
Вып.~3. С.~247--290.

\bibitem{Laha1961} 
\Au{Laha R.\,G. } On a class of unimodal distributions~// 
Proc. Am. Math. Soc., 1961. Vol.~12. P.~181--184.

\bibitem{Lukacs1970} 
\Au{Лукач Е.} Характеристические функции.~--- М.: Наука, 1979. 424~с.

\bibitem{KotzOstrovskiiHayfavi1995a} 
\Au{Kotz S., Ostrovskii~I.\,V., Hayfavi~A.} Analytic and asymptotic properties 
of Linnik's probability densities,~I~// J.~Math. Anal. Appl., 1995. Vol.~193. P.~353--371.

\bibitem{KotzOstrovskiiHayfavi1995b} 
\Au{Kotz S., Ostrovskii~I.\,V., Hayfavi~A.} 
Analytic and asymptotic properties of Linnik's probability densities,~II~// 
J.~Math. Anal. Appl., 1995. Vol.~193. P.~497--521.

\bibitem{SabuPillai1987} 
\Au{Sabu G., Pillai~R.\,N.} Multivariate $\alpha$-Laplace distributions~// 
J.~Nat. Acad. Math., 1987. Vol.~5. P.~13--18.

\bibitem{Lin1994} 
\Au{Lin G.\,D.} Characterizations of the Laplace and related distributions 
via geometric compound~// Sankhya, A1, 1994. Vol.~56. P.~1--9.

\bibitem{Anderson1992} 
\Au{Anderson D.\,N.} A~multivariate Linnik distribution~// 
Stat. Probabil. Lett., 1992. Vol.~14. P.~333--336.

\bibitem{Devroye1990} 
\Au{Devroye L.} A~note on Linnik's distribution~// 
Stat. Probabil. Lett., 1990. Vol.~9. P.~305--306.

\bibitem{Jacquesetal1999} 
\Au{Jacques C., \mbox{R{\!\ptb{{\`{e}}}}millard}~ B., Theodorescu~R.} 
Estimation of Linnik law parameters~// Stat. Decision, 1999. Vol.~17. No.\,3. P.~213--236.


\bibitem{KotzOstrovskii1996} 
\Au{Kotz S., Ostrovskii~I.\,V.} A~mixture representation of the Linnik distribution~// 
Stat. Probabil. Lett., 1996. Vol.~26. P.~61--64.

\bibitem{Pakes1998} 
\Au{Pakes A.\,G.} Mixture representations for symmetric generalized Linnik laws~// 
Stat. Probabil. Lett., 1998. Vol.~37. P.~213--221.

\bibitem{Kilbas2014} 
\Au{Gorenflo R., Kilbas~A.\,A., Mainardi~F., Rogosin~S.\,V.} 
Mittag--Leffler functions, related topics and applications.~--- Berlin\,--\,New York: Springer, 2014.
420~p.

\bibitem{Kovalenko1965} 
\Au{Kovalenko I.\,N.} On the class of limit distributions for rarefied flows 
of homogeneous events~// Lith. Math.~J., 1965. Vol.~5. No.\,4. P.~569--573.

\bibitem{GnedenkoKovalenko1968} 
\Au{Gnedenko B.\,V., Kovalenko~I.\,N.} Introduction to queueing theory.~--- 
Jerusalem: Israel Program for Scientific Translations, 1968. 281~p.

\bibitem{GnedenkoKovalenko1989} 
\Au{Gnedenko B.\,V., Kovalenko~I.\,N.} Introduction to queueing theory.~--- 2nd ed.~--- 
Boston: Birkhauser, 1989. 314~p.

\bibitem{Pillai1989} 
\Au{Pillai R.\,N.} Harmonic mixtures and geometric infinite divisibility~// 
J.~Indian Stat. Ass., 1990. Vol.~28. P.~87--98.

\bibitem{Pillai1990} 
\Au{Pillai R.\,N.} On Mittag--Leffler functions and related distributions~// 
Ann. Inst. Stat. Math., 1990. Vol.~42. P.~157--161.

\bibitem{WeronKotulski1996} 
\Au{Weron K., Kotulski~M.} On the Cole--Cole relaxation function and related 
Mittag--Leffler distributions~// Physica A, 1996. Vol.~232. P.~180--188.

\bibitem{GorenfloMainardi2006} 
\Au{Gorenflo R., Mainardi~F.} Continuous time random walk, Mittag--Leffler 
waiting time and fractional diffusion: Mathematical aspects~// 
 Anomalous transport: Foundations and applications~/ Eds. 
 R.~Klages, G.~Radons, I.\,M.~Sokolov.~--- Weinheim, Germany: Wiley-VCH, 2008. P.~93--127. http://arxiv.org/abs/0705.0797.

\bibitem{GnedenkoFahim1969} 
\Au{Гнеденко Б.\,В., Фахим~Х.} Об одной теореме переноса~// Докл. АН СССР, 1969. Т.~187. Вып.~1. С.~15--17.

\bibitem{Renyi1956} 
\Au{R$\acute{\mbox{e}}$nyi A. } A~Poisson-folyamat egy jellemzese~// 
Maguar Tud. Acad. Mat. Int. Kozl., 1956. Vol.~1. P.~519--527.

\bibitem{LimTeo2009} 
\Au{Lim S.\,C., Teo~L.\,P.} Analytic and asymptotic properties of multivariate
 generalized Linnik's probability densities~// 
 J.~Fourier Anal. Appl., 2010. Vol.~16. P.~715--747.

\bibitem{KorolevKurmangazievaZeifman2016} 
\Au{Korolev V.\,Yu., Kurmangazieva~L., Zeifman~A.\,I.} On asymmetric generalization 
of the Weibull distribution by scale-location mixing of normal laws~// J.~Korean 
Stat. Soc., 2016. Vol.~45. P.~238--249. arXiv:1506.06232.

\bibitem{BN1977} 
\Au{Barndorff-Nielsen O.-E.} Exponentially decreasing distributions for the 
logarithm of particle size~// Proc. Roy. Soc. Lond. A, 1977. Vol.~A(353). P.~401--419.


\bibitem{BN1982} 
\Au{Barndorff-Nielsen O.-E., Kent~J., \mbox{S\!{\!\ptb{\o}}\,rensen}~M.} 
Normal variance-mean mixtures and $z$-distributions~// 
Int. Stat. Rev., 1977. Vol.~50. No.\,2. P.~145--159.

\bibitem{BN1978} 
\Au{Barndorff-Nielsen O.-E.} Hyperbolic distributions and distributions of hyperbolae~// 
Scand. J.~Stat., 1978. Vol.~5. P.~151--157.


\bibitem{KorolevSokolov2012} 
\Au{Королев В.\,Ю., Соколов~И.\,А.} Скошенные распределения Стьюдента,
 дисперсионные гам\-ма-рас\-пре\-де\-ле\-ния и~их обобщения как асимптотические 
 аппроксимации~// Информатика и~её применения, 2012. Т.~6. Вып.~1. С.~2--10.

\bibitem{KorolevZaks2013} 
\Au{Закс Л.\,М., Королев~В.\,Ю.} Обобщенные дисперсионные гамма-распределения 
как предельные для случайных сумм~// Информатика и~её применения, 2013. Т.~7. Вып.~1. С.~105--115.

\bibitem{Korolev2013}
\Au{Королев В.\,Ю.} Обобщенные гиперболические распределения как предельные 
для случайных сумм~// Теория вероятностей и~ее применения, 2013. Т.~58. Вып.~1. С.~117--132.

\bibitem{ErdoganOstrovskii1997} 
\Au{\mbox{Erdo\!{\!\ptb{\v{g}}}an}~M.\,B., Ostrovskii~I.\,V.} 
Analytic and asymptotic properties of generalized Linnik probability densities~//
J.~Math. Anal. Appl., 1998. Vol.~217. P.~555--578.

\bibitem{ErdoganOstrovskii1998} 
\Au{\mbox{Erdo\!{\!\ptb{\v{g}}}an}~M.\,B., Ostrovskii~I.\,V.} On mixture representation
 of the Linnik density~// J.~Aust. Math. Soc.~A, 1998. Vol.~64. P.~317--326.
 
 \pagebreak

\bibitem{Kalashnikov1997} 
\Au{Kalashnikov V.\,V.} Geometric sums: Bounds for rare events with applications.~--- 
Dordrecht: Kluwer Academic Publs., 1997. 270~p.

\bibitem{Pakes1992} 
\Au{Pakes A.\,G.} A~ characterization of gamma mixtures of stable laws motivated by 
limit theorems~// Stat. Neerl., 1992. Vol.~2-3. P.~209--218.

\bibitem{Kozubowski1998} 
\Au{Kozubowski T.\,J.} Mixture representation of Linnik distribution revisited~// 
Stat. Probabil. Lett., 1998. Vol.~38. P.~157--160.

\bibitem{Kozubowski1999} 
\Au{Kozubowski T.\,J.} Exponential mixture representation 
of geometric stable distributions~// 
Ann. Inst. Stat. Math., 1999. Vol.~52. No.\,2. P.~231--238.
 \end{thebibliography}

 }
 }

\end{multicols}

\vspace*{-6pt}

\hfill{\small\textit{Поступила в~редакцию 14.10.16}}

\vspace*{8pt}

%\newpage

%\vspace*{-24pt}

\hrule

\vspace*{2pt}

\hrule

\vspace*{8pt}


\def\tit{ASYMMETRIC LINNIK DISTRIBUTIONS AS~LIMIT LAWS FOR~RANDOM SUMS 
OF~INDEPENDENT RANDOM VARIABLES WITH~FINITE VARIANCES}

\def\titkol{Asymmetric Linnik distributions as limit laws for random sums of 
independent random variables with finite variances}

\def\aut{V.\,Yu.~Korolev$^{1,2}$, A.\,I.~Zeifman$^{1,2,3,4}$, and~A.\,Yu.~Korchagin$^1$}

\def\autkol{V.\,Yu.~Korolev, A.\,I.~Zeifman, and~A.\,Yu.~Korchagin}

\titel{\tit}{\aut}{\autkol}{\titkol}

\vspace*{-9pt}


\noindent
$^1$Faculty of Computational Mathematics and Cybernetics, 
M.\,V.~Lomonosov Moscow State University, 1-52~Lenin-\linebreak
$\hphantom{^1}$skiye Gory, GSP-1, 
Moscow 119991, Russian Federation

\noindent
$^2$Institute of Informatics 
Problems, Federal Research Center ``Computer Science and Control'' 
of the Russian\linebreak
$\hphantom{^1}$Academy of Sciences, 44-2~Vavilov Str., Moscow 119333,  
Russian Federation

\noindent
$^3$Vologda State University, 15~Lenin Str., Vologda 160000, Russian Federation

\noindent
$^4$ISEDT RAS, 56-A~Gorky Str., Vologda 160001, 
Russian Federation

\def\leftfootline{\small{\textbf{\thepage}
\hfill INFORMATIKA I EE PRIMENENIYA~--- INFORMATICS AND
APPLICATIONS\ \ \ 2016\ \ \ volume~10\ \ \ issue\ 4}
}%
 \def\rightfootline{\small{INFORMATIKA I EE PRIMENENIYA~---
INFORMATICS AND APPLICATIONS\ \ \ 2016\ \ \ volume~10\ \ \ issue\ 4
\hfill \textbf{\thepage}}}

\vspace*{3pt}


\Abste{Linnik distributions (symmetric geometrically stable distributions) 
are widely applied in financial mathematics, telecommunication systems modeling, 
astrophysics, and genetics. These distributions are limiting for geometric 
sums of independent identically distributed random variables whose distribution 
belongs to the domain of normal attraction of a~symmetric strictly stable distribution. 
In the paper, three asymmetric generalizations of the Linnik distribution are 
considered. The traditional (and formal) approach to the asymmetric generalization 
of the Linnik distribution consists in the consideration of geometric 
sums of random summands whose distributions are attracted to an asymmetric 
strictly stable distribution. The variances of such summands are infinite. 
Since in modeling real phenomena, as a~rule, there are no solid reasons to reject 
the assumption of the finiteness of the variances of elementary summands, in 
the paper, two alternative asymmetric generalizations are proposed based 
on the representability of the Linnik distribution as a~scale mixture of normal 
laws or a~scale mixture of Laplace laws. Examples are presented of limit theorems 
for sums of a~random number of independent random variables with finite variances 
in which the proposed asymmetric Linnik distributions appear as limit laws.}

\KWE{Linnik distribution; Laplace distribution; Mittag--Leffler distribution; 
normal distribution; scale mixture; normal variance-mean mixture; 
stable distribution; geometrically stable distribution}


\DOI{10.14357/19922264160403}  

\vspace*{-9pt}

\Ack
\noindent
This work was financially supported by the Russian Science Foundation 
(grant No.\,14-11-00364).



%\vspace*{3pt}

  \begin{multicols}{2}

\renewcommand{\bibname}{\protect\rmfamily References}
%\renewcommand{\bibname}{\large\protect\rm References}

{\small\frenchspacing
 {%\baselineskip=10.8pt
 \addcontentsline{toc}{section}{References}
 \begin{thebibliography}{99}

\bibitem{1-kz}
\Aue{Mittnik, S., and~S.~Rachev}. 1993. Modeling asset returns with alternative 
stable models. \textit{Economet. Rev.} 12:261--330.
\bibitem{2-kz}
\Aue{Kotz, S., T.\,J.~Kozubowski, and K.~Podgorski}. 2001. \textit{The 
Laplace distribution and generalizations: A~revisit with applications to communications, 
economics, engineering, and finance}. Boston: Birkhauser. 349~p. 
\bibitem{3-kz}
\Aue{Korolev, V.\,Yu., and A.\,I.~Zeifman}. 2016. 
A~note on mixture representations for the Linnik and Mittag--Leffler 
distributions and their applications. \textit{J.~Math. Sci}. 218(3):314--327. 
\bibitem{4-kz}
\Aue{Korolev, V.\,Yu., and A.\,I.~Zeifman}. 2016. Convergence of random
sums and statistics constructed 
from samples with random sizes to the Linnik and Mittag--Leffler distributions and 
their generalizations. \textit{J.~Korean Stat. Soc}. 
Available at: arXiv:1602.02480v1 (accessed December~10, 2016). 
\bibitem{5-kz}
\Aue{Zolotarev, V.\,M.} 1986. \textit{One-dimensional stable distributions}. 
Providence: AMS. 284~p.
\bibitem{6-kz}
\Aue{Schneider, W.\,R.} 1986. Stable distributions: Fox function representation and 
generalization. \textit{Stochastic processes in classical and quantum systems}. 
Eds.\ S.~Albeverio, G.~Casati, and D.~Merlini. Berlin: Springer.  497--511. 
\bibitem{7-kz}
\Aue{Uchaikin, V.\,V., and V.\,M.~Zolotarev}. 1999. 
\textit{Chance and stability}. Utrecht: VSP. 570~p. 
\bibitem{8-kz}
\Aue{Korolev, V.\,Yu.} 2016. Product representations for random variables with the 
Weibull distributions and their applications. \textit{J.~Math. Sci}. 218(3):298--313. 
\bibitem{9-kz}
\Aue{Tucker, H.} 1975. On moments of distribution functions attracted to stable laws. 
\textit{Houston J.~Math.} 1(1):149--152. 
\bibitem{10-kz}
\Aue{Klebanov, L.\,B., G.\,M.~Maniya, and I.\,A.~Melamed}. 1985. 
A~problem of Zolotarev and analogs of infinitely 
divisible and stable distributions in a~scheme for summing a~random 
number of random variables. \textit{Theor. Probab. Appl.} 29(4):791--794.
\bibitem{11-kz}
\Aue{Klebanov, L.\,B., and S.\,T.~Rachev}. 1996. Sums of a~random number 
of random variables and their approximations with $\varepsilon$-accompanying 
infinitely divisible laws. \textit{Serdica} 22:471--498. 
\bibitem{12-kz}
\Aue{Bunge, J.} 1996. Compositions semigroups and random stability. 
\textit{Ann. Probab.} 24:1476--1489. 
\bibitem{13-kz}
\Aue{Rachev, S.\,T.} 1991. \textit{Probability metrics and the stability of stochastic 
models}. Chichester\,--\,New York: Wiley. 494~p.
\bibitem{14-kz}
\Aue{Gnedenko, B.\,V., and V.\,Yu.~Korolev}. 1886. \textit{Random 
summation: Limit theorems and applications}. Boca Raton: CRC Press. 267~p.
\bibitem{15-kz}
\Aue{Korolev, V.\,Yu.} 1995. 
Convergence of random sequences with the independent random 
indices~I. \textit{Theor. Probab. Appl.} 39(2):282--297. 
\bibitem{16-kz}
\Aue{Korolev, V.\,Yu.} 1996. Convergence of random sequences with the independent random 
indices~II. \textit{Theor. Probab. Appl.} 40(4):770--772. 

\bibitem{17-kz}
\Aue{Bening, V.\,E., and V.\,Yu.~Korolev}. 2002. 
\textit{Generalized Poisson models and their applications in insurance and finance}. 
Utrecht: VSP. 434~p.
\bibitem{18-kz}
\Aue{Pillai, R.\,N.} 1985. Semi-$\alpha$-Laplace distributions.
\textit{Commun. Stat. Theor. Meth.} 14:991--1000. 
\bibitem{19-kz}
\Aue{Linnik, Yu.\,V.} 1953. Lineynye formy i~statisticheskiye kriterii. I, II 
[Linear forms and statistical criteria. I, II]. 
\textit{Ukr. Math.~J.} 5(2):207--243; 3:247--290. 


\bibitem{20-kz}
\Aue{Laha, R.\,G.} 1961. On a~class of unimodal distributions. 
\textit{Proc. Am. Math. Soc.} 12:181--184. 
\bibitem{21-kz}
\Aue{Lukacs, E.} 1970. \textit{Characteristic functions}. 2nd ed. 
London: Griffin. 350~p. 

\bibitem{22-kz}
\Aue{Kotz, S., I.\,V.~Ostrovskii, and A.~Hayfavi}. 1995. 
Analytic and asymptotic properties of Linnik's probability densities, I. 
\textit{J.~Math. Anal. Appl.} 193:353--371. 
\bibitem{23-kz}
\Aue{Kotz, S., I.\,V.~Ostrovskii, and A.~Hayfavi}. 1995. 
Analytic and asymptotic properties of Linnik's probability densities, II. 
\textit{J.~Math. Anal. Appl.} 193:497--521. 
\bibitem{24-kz}
\Aue{Sabu, G., and R.\,N.~Pillai}. 1987. Multivariate $\alpha$-Laplace distributions. 
\textit{J.~Nat. Acad. Math.} 5:13--18. 
\bibitem{25-kz}
\Aue{Lin, G.\,D.} 1994. Characterizations of the Laplace and related distributions 
via geometric compound. \textit{Sankhya, A1} 56:1--9. 
\bibitem{26-kz}
\Aue{Anderson, D.\,N.} 1992. A~multivariate Linnik distribution. 
\textit{Stat. Probabil. Lett.} 14:333--336. 
\bibitem{27-kz}
\Aue{Devroye, L.} 1990. A~note on Linnik's distribution. 
\textit{Stat. Probabil. Lett}. 9:305--306. 
\bibitem{28-kz}
\Aue{Jacques, C., B.~\mbox{R{\!\ptb{\`{e}}}emillard}, and R.~Theodorescu}. 
1999. Estimation of Linnik law parameters. \textit{Stat. Decision} 17(3):213--236. 

\bibitem{30-kz}
\Aue{Kotz, S., and I.\,V.~Ostrovskii}. 1996. A~mixture representation of the 
Linnik distribution. \textit{Stat. Probabil. Lett.} 26:61--64. 
\bibitem{29-kz}
\Aue{Pakes, A.\,G.} 1998. Mixture representations for symmetric generalized Linnik 
laws. \textit{Stat. Probabil. Lett.} 37:213--221.  

\bibitem{31-kz}
\Aue{Gorenflo, R., A.\,A.~Kilbas, F.~Mainardi, and S.\,V.~Rogosin}. 
2014. \textit{Mittag--Leffler functions, related topics and applications}. 
Berlin\,--\,New York: Springer. 420~p.
\bibitem{32-kz}
\Aue{Kovalenko, I.\,N.} 1965. On the class of limit distributions for rarefied flows 
of homogeneous events. \textit{Lith. Math.~J.} 5(4):569--573. 



\bibitem{33-kz}
\Aue{Gnedenko, B.\,V., and I.\,N.~Kovalenko}. 1968. 
\textit{Introduction to queueing theory}. 
Jerusalem: Israel Program for Scientific Translations. 281~p. 
\bibitem{34-kz}
\Aue{Gnedenko, B.\,V., and I.\,N.~Kovalenko}. 
1989. \textit{Introduction to queueing theory}. 2nd ed. Boston: Birkhauser. 314~p.
\bibitem{35-kz}
\Aue{Pillai, R.\,N.} 1990. Harmonic mixtures and geometric infinite divisibility. 
\textit{J.~Indian Stat. Ass.} 28:87--98. 
\bibitem{36-kz}
\Aue{Pillai, R.\,N.} 1990. On Mittag--Leffler functions and related distributions. 
\textit{Ann. Inst. Stat. Math.} 42:157--161. 
\bibitem{37-kz}
\Aue{Weron, K., and M.~Kotulski}. 1996. On the Cole--Cole relaxation function 
and related Mittag--Leffler distributions. \textit{Physica A} 232:180--188. 
\bibitem{38-kz}
\Aue{Gorenflo, R., and F.~Mainardi}. 2008. Continuous time random walk, 
Mittag--Leffler waiting time and fractional diffusion: Mathematical aspects. 
\textit{Anomalous transport: Foundations and applications}. Eds.\
R.~Klages, G.~Radons, and I.\,M.~Sokolov. Weinheim, Germany: Wiley-VCH. 93--127. 
\bibitem{39-kz}
\Aue{Gnedenko, B.\,V., and H.~Fahim}. 1969. Ob odnoy teoreme perenosa 
[About one transfer theorem]. \textit{Dokl. USSR Akad. Nauk} 187(1):15--17. 
\bibitem{40-kz}
\Aue{R$\acute{\mbox{e}}$enyi, A.} 1956. A~Poisson-folyamat egy jellemzese. 
\textit{Maguar Tud. Acad. Mat. Int. Kozl.} 1:519--527. 
\bibitem{41-kz}
\Aue{Lim, S.\,C., and L.\,P.~Teo}. 2010. Analytic and asymptotic properties 
of multivariate generalized Linnik's probability densities. 
\textit{J.~Fourier Anal. Appl.} 16:715--747. 
\bibitem{42-kz}
\Aue{Korolev, V.\,Yu., L.~Kurmangazieva, and A.\,I.~Zeifman}. 2016.
 On asymmetric generalization of the Weibull distribution by scale-location mixing 
 of normal laws. \textit{J.~Korean Stat. Soc.} 45:238--249. arXiv:1506.06232. 
\bibitem{43-kz}
\Aue{Barndorff-Nielsen, O.\,E.} 1977. Exponentially decreasing distributions for the 
logarithm of particle size. \textit{Proc. Roy. Soc. Lond. A} 353:401--419. 

\bibitem{45-kz}
\Aue{Barndorff-Nielsen, O.\,E., J.~Kent, and M.~\mbox{S\!{\ptb{\o}}rensen}.} 
1982. Normal variance-mean mixtures and \mbox{$z$-distributions}. 
\textit{Int. Stat. Rev.} 50(2):145--159. 

\pagebreak

\bibitem{44-kz}
\Aue{Barndorff-Nielsen, O.\,E.} 1978. Hyperbolic distributions and distributions 
of hyperbolae. \textit{Scand. J.~Stat.} 5:151--157. 
\bibitem{46-kz}
\Aue{Korolev, V.\,Yu., and I.\,A.~Sokolov}. 2012. Skoshennye raspredeleniya 
St'yudenta, dispersionnye gamma-raspredeleniya i~ikh obobshcheniya kak asimptoticheskie 
approksimatsii [Skewed Student's distributions, variance gamma distributions, 
and their generalizations as asymptotic approximations]. 
\textit{Informatika i~ee Primeneniya~--- Inform. Appl.} 6(1):2--10. 
\bibitem{47-kz}
\Aue{Zaks, L.\,M., and V.\,Yu.~Korolev}. 2013. Obobshchennye dispersionnye 
gamma-raspredeleniya kak predel'nye dlya sluchaynykh summ 
[Generalized variance gamma distributions as limiting for random sums]. 
\textit{Informatika i~ee Primeneniya~--- Inform. Appl.} 7(1):105--115. 

\bibitem{48-kz}
\Aue{Korolev, V.\,Yu.} 2014. Generalized hyperbolic laws as limit
distributions for random sums. \textit{Theor. Probab. Appl.} 58(1):63--75.

%\columnbreak

\bibitem{49-kz}
\Aue{\mbox{Erdo\!\!{\ptb{\v{g}}}an}, M.\,B., and I.\,V.~Ostrovskii}. 
1998. Analytic and asymptotic properties of generalized Linnik probability densities. 
\textit{J.~Math. Anal. Appl}. 217:555--578.  
\bibitem{50-kz}
\Aue{\mbox{Erdo\!\!{\ptb{\v{g}}}an}, M.\,B., and I.\,V.~Ostrovskii}. 1998. 
On mixture representation of the Linnik density.
\textit{J.~Aust. Math. Soc.~A} 64:317--326. 
\bibitem{51-kz}
\Aue{Kalashnikov, V.\,V.} 1997. \textit{Geometric sums: Bounds for rare events with 
applications}. Dordrecht: Kluwer Academic Publs. 270~p.
\bibitem{52-kz}
\Aue{Pakes, A.\,G.} 1992. A~characterization of gamma mixtures of stable 
laws motivated by limit theorems. \textit{Stat. Neerl.} 2-3:209--218. 
\bibitem{53-kz}
\Aue{Kozubowski, T.\,J.} 1998. Mixture representation of Linnik distribution revisited.
\textit{Stat. Probabil. Lett.} 38:157--160. 
\bibitem{54-kz}
\Aue{Kozubowski, T.\,J.} 1999. Exponential mixture representation of geometric 
stable distributions. \textit{Ann. Inst. Stat. Math.}
52(2):231--238. 
\end{thebibliography}

 }
 }

\end{multicols}

\vspace*{-3pt}

\hfill{\small\textit{Received October 14, 2016}}

\Contr

\noindent
\textbf{Korolev Victor Yu.} (b.\ 1954)~---
Doctor of Science in physics and mathematics, professor, Head of the Department 
of Mathematical Statistics, 
Faculty of Computational Mathematics and Cybernetics, 
M.\,V.~Lomonosov Moscow State University, 1-52~Leninskiye Gory, GSP-1, 
Moscow 119991, Russian Federation; leading scientist, Institute of Informatics 
Problems, Federal Research Center ``Computer Science and Control'' 
of the Russian Academy of Sciences, 44-2~Vavilov Str., Moscow 119333,  
Russian Federation; \mbox{vkorolev@cs.msu.su} 

\vspace*{3pt} 

\noindent
\textbf{Zeifman Alexander I.} (b.\ 1954)~---
Doctor of Science in physics and mathematics, professor, Head of Department, 
Vologda State University, 15~Lenin Str., Vologda 160000, Russian Federation; 
senior scientist, Institute of Informatics Problems, Federal Research Center 
``Computer Science and Control'' of the Russian Academy of Sciences, 
44-2~Vavilov Str., Moscow 119333, Russian Federation; 
principal scientist, ISEDT RAS, 56-A~Gorky Str., Vologda 160001, 
Russian Federation; Faculty of Computational Mathematics and Cybernetics, 
M.\,V.~Lomonosov Moscow State University, 1-52~Leninskiye Gory, GSP-1, 
Moscow 119991, Russian Federation; \mbox{a\_zeifman@mail.ru}

 \vspace*{3pt}
 
 \noindent
 \textbf{Korchagin Alexander Yu.} (b.\ 1989)~---
 junior scientist, Faculty of Computational Mathematics and Cybernetics, 
 M.\,V.~Lomonosov Moscow State University, 1-52~Leninskiye Gory, GSP-1, 
 Moscow 119991, Russian Federation; \mbox{sasha.korchagin@gmail.com}
\label{end\stat}


\renewcommand{\bibname}{\protect\rm Литература} 