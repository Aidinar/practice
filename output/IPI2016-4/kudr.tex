\def\stat{kudr}

\def\tit{БАЙЕСОВСКИЕ МОДЕЛИ МАССОВОГО ОБСЛУЖИВАНИЯ И~НАДЕЖНОСТИ:
ВЫРОЖДЕННО-ВЕЙБУЛЛОВСКИЙ СЛУЧАЙ}

\def\titkol{Байесовские модели массового обслуживания и~надежности:
вырожденно-вейбулловский случай}

\def\aut{А.\,А.~Кудрявцев$^1$, А.\,И.~Титова$^2$}

\def\autkol{А.\,А.~Кудрявцев, А.\,И.~Титова}

\titel{\tit}{\aut}{\autkol}{\titkol}

\index{Кудрявцев А.\,А.}
\index{Kudryavtsev A.\,A.}
\index{Titova A.\,I.}
\index{Титова А.\,И.}


%{\renewcommand{\thefootnote}{\fnsymbol{footnote}} \footnotetext[1]
%{Исследование выполнено при поддержке Российского научного фонда (проект 14-11-00397).}}


\renewcommand{\thefootnote}{\arabic{footnote}}
\footnotetext[1]{Московский государственный университет им.~М.\,В.~Ломоносова, 
факультет вычислительной математики и~кибернетики, 
%Институт проблем информатики Федерального исследовательского центра <<Информатика и~управ\-ле\-ние>> Российской академии наук, 
\mbox{nubigena@mail.ru}}
\footnotetext[2]{Московский государственный университет им.~М.\,В.~Ломоносова, 
факультет вычислительной математики и~кибернетики, \mbox{onkelskroot@gmail.com}}

\vspace*{-3pt}

\Abst{Данная работа посвящена изучению байесовских моделей массового обслуживания 
и~надежности. В~рамках байесовского подхода для классических постановок задач 
предполагается, что основные параметры системы не являются заданными, а~известны 
только их априорные распределения. За счет рандомизации таких параметров системы, 
как интенсивность входящего потока и~интенсивность обслуживания, происходит 
рандомизация различных характеристик системы, например коэффициента загрузки. 
Байесовский подход является целесообразным при изучении больших совокупностей 
однотипных систем или одной системы, характеристики которой меняются в~моменты времени, 
неизвестные исследователю. В~работе представлены конкретные результаты для 
вероятностных характеристик коэффициента загрузки и~вероятности <<непотери>> вызова 
в~случае, когда в~качестве пары априорных распределений параметров системы $M|M|1|0$ 
рассматриваются вырожденное распределение и~распределение Вейбулла}.

\KW{байесовский подход; системы массового обслуживания; надежность; смешанные
распределения; распределение Вейбулла; вырожденное распределение}

\DOI{10.14357/19922264160407} 


\vskip 10pt plus 9pt minus 6pt

\thispagestyle{headings}

\begin{multicols}{2}

\label{st\stat}


\section{Введение}

Нередко в~математических моделях жизненный цикл функционирования различных объектов 
зависит от параметров, <<способствующих>> и~<<препятствующих>> функционированию. 
В~моделях структур и~систем массового обслуживания к~параметрам, <<способствующим>> 
функционированию, можно отнести интенсивность обслуживания запросов, а~к~параметрам, 
<<препятствующим>> функционированию,~--- интенсивность входящего потока требований. 
Нетрудно заметить, что итоговые результаты работы системы зависят не столько от 
значений параметров, сколько от их соотношения.

Ниже будет рассмотрена система массового обслуживания $M|M|1|0$. Одним из основных 
показателей такой системы является ее коэффициент загрузки~$\rho$. Значение~$\rho$ 
определяется отношением параметра входящего потока~$\lambda$ к~параметру 
обслуживания~$\mu$. Многие характеристики разнообразных систем массового обслуживания 
зависят от величины~$\rho$, в~том числе вероятность <<непотери>> вызова 
$\pi \hm= {\mu/(\lambda \hm+ \mu)} \hm= {1/(1\hm+\rho)}$.

В рамках байесовских моделей массового обслуживания и~надежности в~классических 
задачах предполагается, что значения параметров~$\lambda$ и~$\mu$ неизвестны, однако 
имеется информация об их априорных распределениях.
Подробное описание предпосылок для исследования, особенностей 
и~библиографии байесовских моделей 
в~теории массового обслуживания и~надежности можно найти в~книге~\cite{KuSh2015}.

Далее для модели $M|M|1|0$ приводятся вероятностные характеристики коэффициента 
загрузки~$\rho$  и~вероятности <<непотери>> вызова~$\pi$ в~случае, когда в~качестве 
пары априорных распределений па\-ра\-мет\-ров системы~$\lambda$ и~$\mu$ рассматриваются 
вырожденное распределение и~распределение Вейбулла.

\vspace*{-9pt}

\section{Основные результаты}

Введем следующие обозначения: через $D(\lambda)$ обозначим вырожденное в~точке~$\lambda$ 
распределение, а через $W(p,\alpha)$~--- распределение Вейбулла 
с~плот\-ностью $w_{p,\alpha}(x)$, имеющей вид:

\vspace*{2pt}

\noindent
$$
w_{p,\alpha}(x) = \fr{px^{p-1}e^{-({x/\alpha})^{p}}}{{\alpha^{p}}}\,, \enskip x>0\,,\enskip
p>0\,,\enskip \alpha>0\,.
$$

\vspace*{-2pt}

Обозначим через $\mathcal{L}_s\left\lbrace f(x)\right\rbrace$
 преобразование Лап\-ла\-са соответсвующей функции:
 
 \vspace*{2pt}
 
 \noindent
\begin{equation}
\label{Laplas}
\mathcal{L}_s\left\lbrace f(x)\right\rbrace = \int\limits_0^{\infty}f(x)e^{-sx}\,dx\,.
\end{equation}

%\vspace*{-2pt}

\noindent
\textbf{Теорема~1.}\ \
\textit{Пусть параметр входящего потока~$\lambda$ имеет вырожденное распределение, 
а~параметр обслуживания~$\mu$ имеет распределение Вейбулла $W(p,\alpha)$. 
Тогда функция распределения, плотность и~моменты коэффициента загрузки~$\rho$ 
имеют вид}:
\begin{align}
F_\rho(x) &= e^{-\left(\lambda/({\alpha x})\right)^{p}}\,, \enskip x>0\,;\notag
\\
f_\rho(x) &= \fr{p\lambda^{p}e^{-({\lambda/(\alpha x)})^{p}}}{{\alpha^{p}x^{p+1}}}\,, \enskip
 x>0\,; \label{DWf_rho}\\
{\sf E}\rho^k &= \fr{\lambda^k}{{\alpha^k}}\,
\Gamma\left(\fr{p-k}{p}\right)\,, \enskip k<p\,,\notag
\end{align}
\textit{а функция распределения, плотность и~моменты вероятности <<непотери>> 
вызова~$\pi$ 
определяются соотношениями}:
\begin{align}
F_\pi(x) &= 1 - \exp\left\lbrace -\left(\fr{\lambda x}{{\alpha(1-x)}}\right)^p\right\rbrace \,, \enskip x\in(0,1)\,;
\notag\\
f_\pi(x) &= \fr{p\lambda^p x^{p-1}}{\alpha^p (1-x)^{p+1}} 
\exp\left\lbrace -\left(\fr{\lambda x}{{\alpha(1-x)}}\right)^p\right\rbrace \,,\notag \\
& \hspace*{48mm}x\in(0,1)\,;
\label{DWf_pi}
\\
{\sf E}\pi^k &=  \mathcal{L}_1\left\lbrace\left(\fr{\alpha x^{1/p}}
{\lambda + \alpha x^{1/p}}\right)^k\right\rbrace\,.\label{DWEpik}
\end{align}


\noindent
Д\,о\,к\,а\,з\,а\,т\,е\,л\,ь\,с\,т\,в\,о\,.\ \ Заметим, что

\noindent
\begin{multline*}
F_\rho(x) = 1 - \fr{p}{\alpha^p}\int\limits_0^{\lambda/x}t^{p-1}e^{-(t/\alpha)^{p}}\, dt 
= e^{-\left(\lambda/(\alpha x)\right)^{p}}\,, \\
 x>0\,.
\end{multline*}
Продифференцировав полученное выражение, получаем~(\ref{DWf_rho}).


Для $k$-го момента случайной величины~$\rho$ имеем:

\noindent
\begin{multline*}
{\sf E}\rho^k=\fr{p\lambda^{p}}{{\alpha^{p}}}\int\limits_0^{\infty}
\fr{e^{-(\lambda/(\alpha x))^{p}}}{{x^{p-k+1}}}\,dx={}\\
{}=
\fr{p\lambda^k}{\alpha^k}\int\limits_0^{\infty}z^{p-k-1}e^{-z^{p}}\,dz={}\\
{} = \fr{\lambda^k}{\alpha^k}\int\limits_0^{\infty}t^{(p-k)/p-1}e^{-t}\,dt = 
\fr{\lambda^k}{\alpha^k}\,\Gamma\left(\fr{p-k}{p}\right)\,, \enskip k<p.\hspace*{-3.8pt}
\end{multline*}

Теперь рассмотрим характеристики вероятности <<непотери>> вызова~$\pi$. 
Для функции распределения имеем:

\noindent
\begin{multline*}
F_\pi(x) = 1 - {\sf P}\left(\rho<\fr{1-x}{x}\right) = {}\\
{}=
1 - \exp\left\lbrace -\left(\fr{\lambda x}{\alpha(1-x)}\right)^p\right\rbrace \,, \enskip
x\in(0,1)\,.
\end{multline*}
Продифференцировав последнее выражение, получаем~(\ref{DWf_pi}).



Найдем ${\sf E}\pi^k$. Имеем:
\begin{multline*}
{\sf E}\pi^k = \int\limits_0^{1}\fr{px^{k-1}}{(1-x)}
\left(\fr{\lambda x}{\alpha(1-x)}\right)^p \times{}\\
{}\times\exp\left\lbrace -\left(\fr{\lambda x}
{\alpha(1-x)}\right)^p\right\rbrace \, dx=
\! \int\limits_0^{\infty}\!\fr{p\alpha^k z^{p+k-1} 
e^{-z^p}}{(\lambda + \alpha z)^k}\,dz.\hspace*{-6.55pt}
\end{multline*}
Воспользовавшись заменой $t\hm=z^p$  и~соотношением~(\ref{Laplas}), 
получим~(\ref{DWEpik}).
Теорема доказана.

\smallskip

\noindent
\textbf{Теорема 2.}\ \
\textit{Пусть параметр входящего потока~$\lambda$ имеет распределение 
Вейбулла $W(q,\theta)$, а параметр обслуживания~$\mu$ имеет вырожденное распределение. 
Тогда функция распределения, плотность и~моменты коэффициента загрузки~$\rho$ 
имеют вид}:
\begin{align}
F_\rho(x)&=1 - e^{-(\mu x/\theta)^{q}}\,, \enskip x>0\,;\notag
\\
f_\rho(x)&=q\left(\fr{\mu}{\theta}\right)^q x^{q-1}e^{-(\mu x/\theta)^q}\,, \enskip x>0\,;
\label{WDf_rho}
\\
{\sf E}\rho^k &= \left(\fr{\theta}{\mu}\right)^{k}\Gamma\left(\fr{k+q}{q}\right)\,,\notag
\end{align}
\textit{a функция распределения, плотность и~моменты вероятности <<непотери>> 
вызова~$\pi$ определяются соотношениями}:
\begin{align}
F_\pi(x) &=\exp\left\lbrace -\left(\fr{\mu(1-x)}{\theta x}\right)^q\right\rbrace \,,\enskip
x\in(0,1)\,;\notag
\\
\hspace*{-3pt}f_\pi(x) &= \fr{q\mu}{\theta x^2}\left(\fr{\mu(1-x)}{\theta x}\right)^{q-1}\hspace*{-6pt}
\exp{\left\lbrace-\left(\fr{\mu(1-x)}{\theta x}\right)^q\right\rbrace},\notag \\ 
&\hspace*{48mm}x\in(0,1)\,;\label{WDf_pi}
\\
{\sf E}\pi^k &= \mathcal{L}_1\left\lbrace\left(
\fr{\mu}{\mu+\theta x^{1/q}}\right)^k\right\rbrace\,.\notag
\end{align}


\noindent
Д\,о\,к\,а\,з\,а\,т\,е\,л\,ь\,с\,т\,в\,о\,.\ \
Аналогично доказательству теоремы~1 рассмотрим соотношение для 
функции распределения случайной величины~$\rho$:
$$
F_\rho(x) =\fr{q}{\theta^q}\int\limits_0^{\mu x}t^{q-1}e^{-(t/\theta)^{q}}\,dt 
= 1 - e^{-(\mu x/\theta)^{q}}\,, \enskip x>0\,,
$$
из которого вытекает~(\ref{WDf_rho}) и~цепочка равенств:

\vspace*{-2pt}

\noindent
\begin{multline*}
{\sf E}\rho^{k}=\fr{q\mu^q}{\theta^q}\int\limits_0^{\infty} 
x^{q + k -1}e^{-(\mu x/\theta)^q}\,dx={}\\
{}=
\fr{q\theta^k}{\mu^k}\int\limits_0^{\infty}z^{q+k-1}e^{-z^{q}}\,dz={}\\
{}= \int\limits_0^{\infty}\left(\fr{\theta}{\mu}\right)^{k}t^{{k/q}}e^{-t}\,dt = 
\left(\fr{\theta}{\mu}\right)^{k}\Gamma\left(\fr{k+q}{q}\right)\,.
\end{multline*}

\end{multicols}

\begin{table}\small
\begin{center}
\Caption{Частные значения ${\sf E}\rho$ ($\lambda \sim D(\lambda)$, $\mu \sim W(p, \alpha)$)}
\vspace*{2ex}

%\tabcolsep=2pt
\begin{tabular}{|c|c|c|c|c|c|c|c|c|c|c|}
\hline
&\multicolumn{10}{c|}{$p$; $\alpha$}\\
\cline{2-11}
\multicolumn{1}{|c|}{\raisebox{6pt}[0pt][0pt]{$\lambda$}}
& 3;  3& 3;  4& 3;  5& 3;  6& 3;  7& 3;  8& 3;  9& 3;  10& 3;  11& 3;  12\\
\hline
 0,5& 0,23& 0,17& 0,14& 0,11& 0,10& 0,08& 0,08& 0,07& 0,06& 0,06\\
 1,0& 0,45& 0,34& 0,27& 0,23& 0,19& 0,17& 0,15& 0,14& 0,12& 0,11\\
 1,5& 0,68& 0,51& 0,41& 0,34& 0,29& 0,25& 0,23& 0,20& 0,18& 0,17\\
 2,0& 0,90& 0,68& 0,54& 0,45& 0,39& 0,34& 0,30& 0,27& 0,25& 0,23\\
 2,5& 1,13& 0,85& 0,68& 0,56& 0,48& 0,42& 0,38& 0,34& 0,31& 0,28\\
 3,0& 1,35& 1,02& 0,81& 0,68& 0,58& 0,51& 0,45& 0,41& 0,37& 0,34\\
 3,5& 1,58& 1,18& 0,95& 0,79& 0,68& 0,59& 0,53& 0,47& 0,43& 0,39\\
 4,0& 1,81& 1,35& 1,08& 0,90& 0,77& 0,68& 0,60& 0,54& 0,49& 0,45\\
 4,5& 2,03& 1,52& 1,22& 1,02& 0,87& 0,76& 0,68& 0,61& 0,55& 0,51\\
 5,0& 2,26& 1,69& 1,35& 1,13& 0,97& 0,85& 0,75& 0,68& 0,62& 0,56\\
\hline
\end{tabular}
\end{center}
\end{table}
\begin{table}\small
\begin{center}
\Caption{Частные значения ${\sf E}\pi$ ($\lambda \sim W(q, \theta)$, $\mu \sim D(\mu)$)}
\vspace*{2ex}

%\tabcolsep=2pt
\begin{tabular}{|c|c|c|c|c|c|c|c|c|c|c|}
\hline
&\multicolumn{10}{c|}{$\mu$}\\
\cline{2-11}
\multicolumn{1}{|c|}{\raisebox{6pt}[0pt][0pt]{$q$; $\theta$}}
& 2,5& 3,0& 3,5& 4,0& 4,5& 5,0& 5,5& 6,0& 6,5& 7,0\\
\hline
 3;  1& 0,74& 0,60& 0,50& 0,43& 0,38& 0,34& 0,31& 0,28& 0,26& 0,24\\
 3;  2& 0,78& 0,64& 0,54& 0,48& 0,42& 0,38& 0,35& 0,32& 0,29& 0,27\\
 3;  3& 0,80& 0,67& 0,58& 0,51& 0,46& 0,42& 0,38& 0,35& 0,33& 0,30\\
 3;  4& 0,82& 0,70& 0,61& 0,54& 0,49& 0,45& 0,41& 0,38& 0,35& 0,33\\
 3;  5& 0,84& 0,72& 0,64& 0,57& 0,52& 0,48& 0,44& 0,41& 0,38& 0,36\\
 3;  6& 0,85& 0,74& 0,66& 0,60& 0,54& 0,50& 0,46& 0,43& 0,40& 0,38\\
 3;  7& 0,86& 0,76& 0,68& 0,62& 0,57& 0,52& 0,49& 0,45& 0,43& 0,40\\
 3;  8& 0,87& 0,78& 0,70& 0,64& 0,59& 0,54& 0,51& 0,48& 0,45& 0,42\\
 3;  9& 0,88& 0,79& 0,72& 0,66& 0,61& 0,56& 0,53& 0,49& 0,47& 0,44\\
 \hphantom{9}3; 10& 0,89& 0,80& 0,73& 0,67& 0,62& 0,58& 0,54& 0,51& 0,48& 0,46\\
\hline
\end{tabular}
\end{center}
\end{table}

\begin{multicols}{2}



Аналогично для вероятности <<непотери>> вызова~$\pi$ имеем:
\begin{multline*}
F_\pi(x) = 1 - {\sf P}\left(\rho<\fr{1-x}{x}\right)= {}\\
{}=
\exp\left\lbrace -\left(\fr{\mu(1-x)}{\theta x}\right)^q\right\rbrace \,,\enskip 
x\in(0,1)\,.
\end{multline*}
Продифференцировав последнее выражение, получаем~(\ref{WDf_pi}).
Для $k$-го момента случайной величины~$\pi$ справедливо:
\begin{multline*}
{\sf E}\pi^k = \int\limits_0^{1}\fr{qx^{k-1}}{1-x}\left(
\fr{\mu(1-x)}{\theta x}\right)^q \times{}\\
{}\times\exp\left\lbrace -\left(
\fr{\mu(1-x)}{\theta x}\right)^q\right\rbrace \,dx = \int\limits_0^{\infty}
\fr{q\mu^k z^{q-1} e^{-z^q}}{(\mu + \theta z)^k}\, dz={}\hspace*{-9pt}
\end{multline*}

\noindent
\begin{multline*}
{}=
 \int\limits_0^{\infty}
 \fr{\mu^k e^{-t}}{(\mu+\theta t^{1/q} )^k}\,dt = 
 \mathcal{L}_1\left\lbrace\left(
 \fr{\mu}{\mu+\theta t^{1/q}}\right)^k\right\rbrace\,.
 \end{multline*}
Теорема доказана.

\section{Численные результаты}

 В качестве иллюстрации приведем табл.~1 и~2, содержащие частные значения характеристик, 
 полученных в~теоремах~1 и~2, для некоторых наборов параметров. Значения в~таблицах 
 приведены с~точностью до сотых.

{\small\frenchspacing
 {%\baselineskip=10.8pt
 \addcontentsline{toc}{section}{References}
 \begin{thebibliography}{9}
 

\bibitem{KuSh2015}
\Au{Кудрявцев А.\,А., Шоргин~С.\,Я.}
Байесовские модели в~тео\-рии массового обслуживания и~надежности.~--- 
М.: ФИЦ ИУ РАН, 2015. 76~с.
\end{thebibliography}

 }
 }

\end{multicols}

\vspace*{-6pt}

\hfill{\small\textit{Поступила в~редакцию 29.06.16}}

%\vspace*{8pt}

\newpage

\vspace*{-24pt}

%\hrule

%\vspace*{2pt}

%\hrule

%\vspace*{8pt}


\def\tit{BAYESIAN QUEUING AND~RELIABILITY MODELS: DEGENERATE-WEIBULL CASE}

\def\titkol{Bayesian queuing and reliability models: Degenerate-Weibull case}

\def\aut{A.\,A.~Kudryavtsev and A.\,I.~Titova}

\def\autkol{A.\,A.~Kudryavtsev and A.\,I.~Titova}

\titel{\tit}{\aut}{\autkol}{\titkol}

\vspace*{-9pt}

\noindent
Department of Mathematical Statistics, Faculty of 
Computational Mathematics and Cybernetics,
M.\,V.~Lomonosov Moscow State University, 
1-52 Leninskiye Gory, GSP-1, Moscow 119991, Russian Federation



\def\leftfootline{\small{\textbf{\thepage}
\hfill INFORMATIKA I EE PRIMENENIYA~--- INFORMATICS AND
APPLICATIONS\ \ \ 2016\ \ \ volume~10\ \ \ issue\ 4}
}%
 \def\rightfootline{\small{INFORMATIKA I EE PRIMENENIYA~---
INFORMATICS AND APPLICATIONS\ \ \ 2016\ \ \ volume~10\ \ \ issue\ 4
\hfill \textbf{\thepage}}}

\vspace*{3pt}

\Abste{This paper is devoted to Bayesian queuing and reliability models. 
In the framework of the Bayesian approach, it is assumed that the key parameters 
of classical systems are random and only their \textit{a~priori} distributions are 
known. 
By randomizing system's parameters such as the input flow intensity and the service 
intensity, one may randomize system's characteristics, for example, system's loading factor.
 The Bayesian approach can be used in the case of studying large groups of systems 
 and devices or one system with variable characteristics. The results 
 for probability characteristics of the system's loading factor and the probability 
 that the claim received by the system will not be lost in the case of the system 
 of the $M|M|1|0$ type where one of the system's parameters has 
a~degenerate distribution and the other has the Weibull distribution are presented.}

\KWE{Bayesian approach; queuing theory; reliability theory; mixed distributions; Weibull distribution; degenerate distribution}

\DOI{10.14357/19922264160407} 

%\vspace*{-9pt}

%\Ack
%\noindent



%\vspace*{-3pt}

  \begin{multicols}{2}

\renewcommand{\bibname}{\protect\rmfamily References}
%\renewcommand{\bibname}{\large\protect\rm References}

{\small\frenchspacing
 {%\baselineskip=10.8pt
 \addcontentsline{toc}{section}{References}
 \begin{thebibliography}{9}
 
 \vspace*{-2pt}

\bibitem{1-kudr-1}
\Aue{Kudryavtsev, A.\,A., and S.\,Ya.~Shorgin}. 2015. 
\textit{Bayesovskie modeli v~teorii massovogo obsluzhivaniya i~nadezhnosti} 
[Bayesian models in mass service and reliability theories]. Moscow: Federal
Research Center ``Computer Sciences and Control'' of the Russian Academy of Sciences.\linebreak
 76~p.
\end{thebibliography}

 }
 }

\end{multicols}

\vspace*{-3pt}

\hfill{\small\textit{Received June 29, 2016}}

\Contr

\noindent
\textbf{Kudryavtsev Alexey A.} (b.\ 1978)~--- Candidate of Science (PhD) 
in physics and mathematics, associate professor, Department of Mathematical 
Statistics, Faculty of Computational Mathematics and Cybernetics, M.\,V.~Lomonosov 
Moscow State University, 1-52 Leninskiye Gory, GSP-1, Moscow 119991, 
Russian Federation; %Institute of Informatics Problems, Federal Research Center 
%``Computer Science and Control'' of the Russian Academy of Sciences, 
%44-2~Vavilov Str., Moscow 119333, Russian Federation; 
\mbox{nubigena@mail.ru}

\vspace*{3pt}

\noindent
\textbf{Titova Anastasiia I.} (b.\ 1995)~--- student,
Department of Mathematical 
Statistics, Faculty of Computational Mathematics and Cybernetics, M.\,V.~Lomonosov 
Moscow State University, 1-52 Leninskiye Gory, GSP-1, Moscow 119991, 
Russian Federation; \mbox{onkelskroot@gmail.com}


\label{end\stat}


\renewcommand{\bibname}{\protect\rm Литература} 