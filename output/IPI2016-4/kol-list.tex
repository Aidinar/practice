\def\stat{listopad}

\def\tit{НЕФОРМАЛЬНАЯ АКСИОМАТИЧЕСКАЯ ТЕОРИЯ РОЛЕВЫХ 
ВИЗУАЛЬНЫХ МОДЕЛЕЙ$^*$}

\def\titkol{Неформальная аксиоматическая теория ролевых 
визуальных моделей}

\def\aut{А.\,В. Колесников$^1$, С.\,В.~Листопад$^2$, С.\,Б.~Румовская$^3$, 
В.\,И.~Данишевский$^4$}

\def\autkol{А.\,В. Колесников, С.\,В.~Листопад, С.\,Б.~Румовская, 
В.\,И.~Данишевский}

\titel{\tit}{\aut}{\autkol}{\titkol}

\index{Колесников А.\,В.}
\index{Листопад С.\,В.}
\index{Румовская С.\,Б.} 
\index{Данишевский В.\,И.}
\index{Kolesnikov A.\,V.}
\index{Listopad S.\,V.}
\index{Rumovskaya S.\,B.} 
\index{Danishevsky V.\,I.}


{\renewcommand{\thefootnote}{\fnsymbol{footnote}} \footnotetext[1]
{Работа выполнена при поддержке РФФИ (проект 16-07-00271а).}}


\renewcommand{\thefootnote}{\arabic{footnote}}
\footnotetext[1]{ Балтийский федеральный университет им.\ И.~Канта, Калининградский филиал Федерального 
исследовательского центра <<Информатика и~управление>> Российской академии наук, 
\mbox{avkolesnikov@yandex.ru}}
\footnotetext[2]{Калининградский филиал Федерального исследовательского центра <<Информатика и~управление>> 
Российской академии наук, \mbox{ser-list-post@yandex.ru}}
\footnotetext[3]{Калининградский филиал Федерального исследовательского центра <<Информатика и~управление>> 
Российской академии наук, \mbox{sophiyabr@gmail.com}}
\footnotetext[4]{Балтийский федеральный университет им.\ И.~Канта, \mbox{danishevskii.v.i@mail.ru}}
   
   \Abst{Актуальность построения неформальной аксиоматической тео\-рии ролевых 
визуальных моделей обусловлена моделированием ви\-зу\-аль\-но-об\-раз\-ных рассуждений 
в~гибридных и~синергетических интеллектуальных сис\-те\-мах. Основные исследования  
ви\-зу\-аль\-но-об\-раз\-ных рассуждений сосредоточены на специальных визуальных языках 
представления некоторых видов данных, информации и~знаний. Отсутствие 
формализованных моделей визуальных языков~--- причина высокой наукоемкости 
разработки специальных сред манипулирования и~обработки визуальных моделей. 
Построение неформальной аксиоматической тео\-рии ролевых визуальных моделей~--- шаг 
к~новому классу интеллектуальных сис\-тем, релевантных реальным коллективам, 
принимающим решения,~--- гибридным интеллектуальным сис\-те\-мам (ГиИС) с~гетерогенным 
визуальным полем, имитирующим сотрудничество, относительность и~дополнительность 
коллективного интеллекта, рассуждающим на символьных и~визуальных языках.}
  
  \KW{гибридная интеллектуальная сис\-те\-ма; гетерогенное визуальное поле; визуальный 
язык; семиотическая сис\-тема}

\DOI{10.14357/19922264160412} 


\vskip 10pt plus 9pt minus 6pt

\thispagestyle{headings}

\begin{multicols}{2}

\label{st\stat}
  
\section{Введение}

  Принятие коллективных решений~--- сложное активное взаимодействие 
участников и~обеспечение взаимопонимания между ними. Для интенсификации 
этих процессов применяются методы\linebreak визуализации информации, своеобразные 
визуальные языки, наглядно описывающие структуру, свойства и~отношения 
понятий предметной об\-ласти. Правильно составленный график или диаграмму 
значительно проще анализировать, чем\linebreak мно\-го\-стра\-нич\-ную таблицу 
с~результатами измерений, а~тем более текстовое их описание. Визуализация 
информации позволяет имитировать рассужде\-ния на основе визуальных 
образов, обладающих большей конкретностью и~интегрированностью, чем 
символические представления. 
  
  Рассуждения на визуальных образах рас\-смат\-ри\-ва\-лись в~работах 
Д.\,А.~Поспелова, Г.\,П.~Щед\-ро\-виц\-ко\-го, Ю.\,Р.~Валькмана, 
Б.\,А.~Кобринского, О.\,П.~Кузнецова, Г.\,С.~Осипова, В.\,Б.~Тарасова, 
И.\,Б.~Фоминых, Т.\,А.~Гавриловой, А.\,Е.~Янковской. Визуальные языки 
разработаны для функционального программирования, программирования на 
примерах, для конечных автоматов, потоков данных и~других областей~[1]. 
Реализация этих языков требует значительных усилий, разработки для каждого 
случая специальных сред создания, манипулирования и~обработки визуальных 
моделей. Для их снижения предлагается неформальная аксиоматическая тео\-рия 
ролевых визуальных моделей на основе принципов тео\-рии сис\-тем и~сис\-тем\-но\-го 
анализа. 

\vspace*{-6pt}
  
\section{Понятие неформальной аксиоматической теории ролевых 
визуальных моделей}
  
  Рассмотрим существо разработки и~использования визуальных моделей на 
теоретическом уровне, построив неформальную аксиоматическую тео\-рию 
языков визуального моделирования сложных сис\-тем. В~качестве обобщения 
результатов работ по имитации рассуждений на визуальных образах~[2--8] 
предлагается неформальная аксиоматическая тео\-рия визуального языка как 
семиотической системы:
  \begin{equation}
  \mathrm{vl}=\langle \mathrm{VT}, \mathrm{VS}, \mathrm{VA}, 
  \mathrm{VP}, \upsilon\tau, \upsilon\sigma, \upsilon\alpha, 
\upsilon\pi\rangle\,,
  \label{e1-ls}
  \end{equation}
где VT, VS и~VA~--- множества основных символов, синтаксических 
правил и~ак\-си\-ом-зна\-ний о~предметной области (семантических правил) 
соответственно; VP~--- множество правил вывода решений (прагматических 
правил); $\upsilon\tau$, $\upsilon\sigma$, $\upsilon\alpha$ и~$\upsilon\pi$~--- 
правила изменения множеств VT, VS, VA и~VP соответственно. 
Множества~VT, VS, VA, VP, $\upsilon\tau$, $\upsilon\sigma$, 
$\upsilon\alpha$ и~$\upsilon\pi$  из~(1) определяются выражениями:
\begin{align}
\mathrm{VT}&=\langle P,D, \mathrm{VR}\rangle\,;\label{e2-ls}\\
\mathrm{VS} &= \langle \mathrm{VT}, \mathrm{VN}, \mathrm{PRU}\rangle\,;\label{e3-ls}\\
\mathrm{VA} &= \langle \mathrm{DO}, G^{\mathrm{RES}}, G^{\mathrm{PR}}, G^R\rangle\,,\label{e4-ls}
   \end{align}
где
   \begin{gather*}
   \mathrm{DO} = \langle \mathrm{RES}, \mathrm{PR}, R\rangle\,,\\
   G^{\mathrm{RES}}:\ \mathrm{RES}\to P\,,\\ 
   G^{\mathrm{PR}}:\ \mathrm{PR}\to D\,,\\ 
   G^R:\ R\to \mathrm{VR}\,;
%   \label{e5-ls}
   \end{gather*}
   \begin{align}
   \mathrm{VP}&=\left\{\langle \mathrm{AG}, \mathrm{act}, M, W\rangle\right\}\,;\label{e6-ls}\\
   \upsilon\tau &= \langle \Delta P, \Delta D, \Delta \mathrm{VR}\rangle\,;\label{e7-ls}\\
   \upsilon\sigma &= \langle \upsilon\tau, \Delta \mathrm{VN}, \Delta \mathrm{PRU}\rangle\,;\label{e8-ls}\\
   \upsilon\alpha &= \langle \Delta \mathrm{DO}, G^{\Delta \mathrm{RES}}, 
   G^{\Delta \mathrm{PR}}, G^{\Delta R}\rangle\,; 
\label{e9-ls}
\end{align}
где
\begin{gather*}
   \Delta \mathrm{DO} = \langle \Delta \mathrm{RES}, \Delta \mathrm{PR}, \Delta R\rangle\,, %\label{e10-ls}
\\
G^{\Delta \mathrm{RES}}:\  \Delta \mathrm{RES}\to \Delta P\,,\\ 
   G^{\Delta \mathrm{PR}}:\ \Delta \mathrm{PR}\to \Delta D\,,\\
   G^{\Delta R}:\ \Delta R\to \Delta \mathrm{VR}\,;
\end{gather*}
   \begin{equation}
   \upsilon\pi = \left\{ \langle \Delta \mathrm{AG}, \Delta \mathrm{act}, \Delta M, \Delta W\rangle\right\}\,.
   \label{e12-ls}
   \end{equation}
Здесь помимо ранее введенных обозначений~$P$~--- множество визуальных 
примитивов; $D$~--- множест\-во визуальных измерений, характеризующих 
визуальные примитивы; VR$^n$~--- множество визуальных отношений между 
одним и~более примитивами~\cite{4-ls};\linebreak VN~--- словарь нетерминальных 
символов; PRU~--- множество продукционных правил; RES, PR и~$R$~--- 
множества ресурсов, свойств и~отношений соответственно; AG~--- 
множество носителей языка (экспер\-тов, элементов, агентов), которым 
адресована норма поведения (различные социальные запреты и~ограничения, 
накладываемые сообществом на отдельного носителя); $\mathrm{act}\hm\in \mathrm{ACT}$~--- 
действие, определенное на множестве действий ACT и~являющееся объектом 
нормативной регуляции (содержание нормы); $M$~--- множество систем 
модальностей, связанных с~действием, например система норм, выраженных 
деонтическими модальностями: $M_N\hm= \{\mathrm{О}, \mathrm{Р}, 
\mathrm{Б}, \mathrm{З}\}$, где О~--- <<обязательно>>, Р~--- <<разрешено>>, Б~--- 
<<безразлично>>, З~--- <<запрещено>>; $W$~--- множество миров, в~которых 
применима норма (условия приложения, обстоятельства, в~которых должно или 
не должно выполняться действие)~\cite{9-ls}; $\Delta P$, $\Delta D$, $\Delta 
\mathrm{VR}$, $\Delta \mathrm{VN}$, $\Delta \mathrm{PRU}$, $\Delta \mathrm{RES}$, 
$\Delta \mathrm{PR}$, $\Delta R$, $\Delta  \mathrm{AG}$, $\Delta M$ и~$\Delta W$~--- 
множества допустимых изменений множеств 
$P$, $D$, VR, VN, PRU, RES, PR, $R$, AG, $M$ и~$W$ 
соответственно; $\Delta \mathrm{act}$~--- множество допустимых изменений содержания 
нормы~act.

  Как показано в~\cite{10-ls}, языки профессиональной деятельности, в~том 
числе и~визуальный язык,~--- полиязыки~\cite{11-ls}. С~одной стороны, это 
обусловлено присущей языку структурированностью внешнего мира. Это 
приводит к~необходимости пред\-став\-ле\-ния в~языке знаков, обозначающих 
ресурсы, свойства, действия, структуры, ситуации, со\-сто\-яния, поведение, 
а~с~учетом деятельности субъекта управ\-ле\-ния~--- целей, задач, планов, 
оценок. В~этом случае визуальный язык может рассматриваться как 
многослойная структура, описывающая решение сложной задачи комбинацией 
нескольких взаимоувязанных процессов рассуждений на разных\linebreak \mbox{языках}.
  
  С~другой стороны, полиязыковой характер языка профессиональной 
деятельности~--- следствие эволюции систем управления в~сторону 
многомодельных, гибридных и~гибридных адаптивных сис\-тем 
управ\-ле\-ния~\cite{11-ls}. Это приводит к~узкой специ\-ализации  
управ\-лен\-цев-экс\-пер\-тов по профессиональным нишам и~к~тому, что 
информация\linebreak в~коллективе, принимающем решения, пред\-став\-ляется на широком 
спектре языков професси\-о\-нальной деятельности со своими относительно 
незави\-симыми задачами, лексикой, данными, знаниями,  
принципами~\cite{11-ls}. Стратификация по\linebreak нишам при разработке 
функциональных ГиИС редуцирует 
сложность задач до прос\-тых элементов~--- подзадач в~локальных подобластях 
мира управ\-ле\-ния~--- профессиональных нишах. 
  
  Рассмотрим подходы к~представлению визуального языка как гетерогенной 
структуры.

\begin{figure*} %fig1
\vspace*{1pt}
\begin{center}
\mbox{%
\epsfxsize=138.171mm
\epsfbox{kol-1.eps}
}
\end{center}
\vspace*{6pt}

\noindent
{\small Гетерогенное визуально-образное ядро в~гетерогенной знаково-языковой оболочке. 
Уровни знаковых и~графических высказываний: {1}~--- концептуального и~визуального 
базиса; {2}~--- о~ресурсах, действиях и~своствах; 3~--- об иерархиях ресурсов, 
действий, свойств; {4}~--- о~пространственных и~производственных структурах; 
{5}~--- о~со\-сто\-яни\-ях, ситуациях и~событиях; {6}~--- о~задачах и~проблемах; 
{7}~--- о~ моделях рассуждений экспертов; {8}~--- об интегрированной модели 
рассуждений коллективного интеллекта}
\vspace*{6pt}
\end{figure*}

  
\section{Многослойная модель визуального языка}

  В работе~\cite{11-ls} информационный язык пред\-став\-лен семейством языков 
описания ресурсов, операций, структур, ситуаций, состояния, поведения
объекта управления, а~также целей, планов и~задач.\linebreak  В~\cite{12-ls} выделено~8~уровней 
визуальных языков для реализации автоматизированных 
рас\-суж\-де\-ний в~интеллектуальных системах: (1)~концептуального и~визуального 
базиса~vl$^1$; (2)~ресурсов, действий и~свойств~vl$^2$; (3)~иерархий ресурсов, 
действий, свойств~vl$^3$; (4)~пространственных и~производственных 
структур~vl$^4$; (5)~состояний, ситуаций и~событий~vl$^5$; (6)~задач  
и~проб\-лем~vl$^6$; (7)~моделей рассуждений экспертов~vl$^7$; 
(8)~интегрированных\linebreak моделей рассуждений коллективного интеллекта~vl$^8$. 
В~этом случае у разработчика есть набор средств-ком\-по\-нен\-тов для 
конструирования из них метаязыка, описывающего решение сложной задачи 
комбинацией нескольких взаимоувязанных процессов рассуждений на разных 
языках. При этом в~зависимости от требований поставленной задачи отдельные 
уровни могут отсутствовать. Таким образом, визуальный метаязык может быть 
представлен выражением
  \begin{equation}
\mathrm{mvl}=\langle \mathrm{vl}^1, \mathrm{vl}^2, \mathrm{vl}^3, \mathrm{vl}^4, 
\mathrm{vl}^5, \mathrm{vl}^6, \mathrm{vl}^7, \mathrm{vl}^8, \mathrm{VLR}\rangle\,,
  \label{e13-ls}
  \end{equation}
где VLR~--- множество отношений между элементами языков~vl$^k$, 
$k\hm\in \mathbb{N}$, $k\hm\in [1,\,8]$. 
  
  Метаязык визуализируется <<слоеным пирогом>> (см.\ рисунок). В~его 
основании~--- словари понятий и~отношений,  
кон\-цеп\-ту\-аль\-но-ви\-зу\-альн\-ый базис, над которым строится семейство 
упорядоченных по уровням языков описания. 
  


  Как показано на рисунке, на каждом языковом уровне выделяется 
гетерогенное об\-раз\-но-ви\-зу\-аль\-ное ядро базовых для данного уровня 
знаков~VT$^k$. Визуальное ядро языков высшего уровня включает знаки ядра 
более низкого уровня $\mathrm{VT}\subseteq \mathrm{VT}^{k+1}$ и~может содержать знаки, 
сформированные вне ядра на языке более низкого уровня VT$^{k+1}\cap 
\mathrm{VN}^k\not= \varnothing$, $k\hm\in \mathbb{N}$, $k\hm\in [1,\,7]$.
  
  Рассмотрим отношения между языками различных уровней на примере их 
множеств синтаксических правил. В~первом слое расположены словари 
понятий и~отношений~--- кон\-цеп\-ту\-аль\-но-ви\-зу\-аль\-ный базис языка. 
Язык первого уровня~vl$^1$ использует эвристические правила PRU$^{-1}$ 
для построения из~$P^1$, $D^1$ и~VR$^1$~знаков производных (составных) 
отношений vr$^{n1}\hm\in \mathrm{VR}{n1}\subseteq \mathrm{VT}^1$:
  $$
  \mathrm{vl}^1\left(P^1, D^1, \mathrm{VR}^1, \mathrm{PRU}^1\right) = 
  \left\{ \mathrm{vr}^{n1}\right\}\,.
  $$
  
  В языке второго уровня~Vl$^2$ эвристики~PRU$^2$ используются, чтобы 
сформировать графические образы ресурсов res$^2\hm\in \mathrm{RES}^2\subseteq 
\mathrm{VT}^2$, действий act$^2\hm\in \mathrm{ACT}^2\subseteq \mathrm{VT}^2$ 
и~свойств pr$^2\hm\in 
\mathrm{PR}^2\subseteq \mathrm{VT}^2$ без учета их иерархичности с~помощью отношений 
определения $\mathrm{VR}_1^{n1}\subseteq \mathrm{VR}^{n1}$:
  $$
  \mathrm{vl}^2\!\left( P^1, D^1, \mathrm{VR}_1^{n1}, 
  \mathrm{PRU}^2\right) =\mathrm{RES}^2\cup \mathrm{PR}^2\cup \mathrm{ACT}^2\,.
  $$
  
  На третьем уровне отношениями включения VR$_5^{n1}\subseteq \mathrm{VR}^{n1}$ 
и~эвристиками~PRU$^3$ формализованы иерархии ресурсов res$^{n3}\hm\in 
\mathrm{RES}^{n3}\subseteq \mathrm{VT}^3$, действий act$^{n3}\hm\in \mathrm{ACT}^{n3}
\hm\subseteq \mathrm{VT}^3$ 
и~свойств pr$^{n3}\hm\in \mathrm{PR}^{n3}\subseteq \mathrm{VT}^3$:
  \begin{multline*}
  \mathrm{vl}^3\!\left(P^1, D^1, \mathrm{RES}^2, 
  \mathrm{PR}^2, \mathrm{ACT}^2, \mathrm{VR}_5^{n1}, \mathrm{PRU}^3\right) = {}\\
  {}=
\mathrm{RES}^{n3}\cup \mathrm{PR}^{n3}\cup \mathrm{ACT}^{n3}\,.
\end{multline*}
    
  Четвертый уровень на основе знаков предыду\-щих уровней, 
временн$\acute{\mbox{ы}}$х VR$_3^{n1}\subseteq \mathrm{VR}^{n1}$, 
про\-странственных VR$_4^{n1}\subseteq \mathrm{VR}^{n1}$  
и~при\-чин\-но-след\-ст\-вен\-ных VR$_6^{n1}\subseteq \mathrm{VR}^{n1}$ отношений, 
а~также\linebreak эвристик~PRU$^4$ формализует пространственные str$_1^4\hm\in 
\mathrm{STR}_1^4\subseteq \mathrm{VT}^4$, опе\-ра\-ци\-о\-наль\-но-тех\-но\-ло\-ги\-че\-ские 
str$_3^4\hm\in \mathrm{STR}_3^4\subseteq \mathrm{VT}^4$ структуры:
  \begin{multline*}
  \mathrm{vl}^4\!\left( P^1, D^1, \mathrm{RES}^{n3}, \mathrm{PR}^{n3}, 
  \mathrm{ACT}^{n3}, \mathrm{VR}_3^{n1}, \mathrm{VR}_4^{n1},\right.\\
\left.  \mathrm{VR}_6^{n1}, \mathrm{PRU}^4\right) = \mathrm{STR}_1^4\cup \mathrm{STR}_3^4\,.
\end{multline*}
  
  На пятом уровне эвристиками~PRU$^5$ формализуют зна\-ки-си\-ту\-ации 
$\mathrm{sit}^5\hm\in \mathrm{SIT}^5\subseteq \mathrm{VT}^5$ 
и~зна\-ки-со\-сто\-яния st$^5\hm\in 
\mathrm{ST}^5\subseteq \mathrm{VT}^5$:
    $$
\mathrm{vl}^5\!\left( \mathrm{STR}_1^4, \mathrm{STR}_3^4, \mathrm{PRU}^5\right) =
\mathrm{SIT}^5\cup \mathrm{ST}^5\,.
   $$
  
  На шестом уровне на основе знаков предыдущих уровней 
и~эвристик~PRU$^6$ специфицируются знаки однородных prb$^{h6} \hm\in 
\mathrm{PRB}^{h6}\subseteq \mathrm{VT}^6$ и~неоднородных prb$^{u6}\hm\in \mathrm{PRB}^{u6}\subseteq 
\mathrm{VT}^6$ задач:
   \begin{multline*}
\mathrm{vl}^6\!\left( P^1, D^1, \mathrm{RES}^{n3}, \mathrm{PR}^{n3}, 
\mathrm{ACT}^{n3}, \mathrm{VR}^{n1}, \mathrm{ST}^5,\right.\\ 
\left.\mathrm{PRU}^6\right) = \mathrm{PRB}^{h6}\cup \mathrm{PRB}^{u6}\,.
  \end{multline*}
  
  На седьмом уровне эвристиками~PRU$^7$ формируются знаки автономных 
методов решения задач $\mathrm{met}^{a7}\hm\in \mathrm{MET}^{a7}\subseteq \mathrm{VT}^7$, 
имитирующих рассуждения отдельно взятого эксперта:
  \begin{multline*}
  \mathrm{vl}^7\!\left( P^1, D^1, \mathrm{RES}^{n3}, \mathrm{PR}^{n3}, 
  \mathrm{ACT}^{n3}, \mathrm{VR}^{n1}, \mathrm{PRU}^7\right) = {}\\
  {}=
\mathrm{MET}^{a7}\,.
  \end{multline*}
  
  На восьмом уровне на основе знаков предыдущих уровней и~
эвристик~PRU$^8$ специфицируются знаки интегрированных методов 
решения задач met$^{u8}\hm\in \mathrm{MET}^{u8}\subseteq \mathrm{VT}^8$, имитирующих 
рассуждения коллектива экспертов:
  \begin{multline*}
  \mathrm{vl}^8\!\left( P^1, D^1, \mathrm{RES}^{n3}, \mathrm{PR}^{n3},
  \mathrm{ACT}^{n3}, \mathrm{VR}^{n1}, \mathrm{ST}^5,\right.\\
\left.   \mathrm{SIT}^5, 
\mathrm{PRB}^{h6}, \mathrm{PRB}^{u6}, \mathrm{MET}^{a7}, \mathrm{PRU}^8\right)=
 \mathrm{MET}^{u8}\,.
  \end{multline*}
  
  Такая многослойная модель визуального языка~--- инструмент сложного 
описания предметной области на различных уровнях обобщенности, 
формирующий его из набора более простых моделей. 

Другой подход 
к~снижению сложности по\-стро\-ения моделей предметной области практических 
задач~--- стратификация языка по профессиональным нишам.
  
\section{Модель гетерогенного визуального поля}

  Гетерогенность визуального поля, проявляющаяся в~разнообразии 
информации, обусловлена отсутствием универсального описания любой 
предметной области. На практике используются более сотни методов 
визуального структурирования. Это обусловлено существенными различиями 
в~природе, особенностях и~свойствах знаний о~различных\linebreak предметных 
областях. В~работах~[13--15] проанализированы наиболее 
известные методы визуализации, определены критерии классификации 
и~классы методов. Определив релевантность метода\linebreak классам, а~классов~--- 
классам задачам, можно разработать стратегию выбора визуальных языков для 
решения подзадач элементами ГиИС. Практические задачи требуют их 
комбинирования и~установления соответствия элементов разных языков, 
насколько это возможно. В~результате при моделировании  
ви\-зу\-аль\-но-об\-раз\-ных рассуждений в~ГиИС формируется гетерогенное 
визуальное поле, обеспечивающее взаимодействие элементов ГиИС, 
рассуждающих на разных визуальных языках.
  
  Формально гетерогенное визуальное поле может быть представлено в~виде:
  \begin{equation}
  \mathrm{GVF}=\langle \mathrm{MVL}, \mathrm{COR}^{\mathrm{VL}}\rangle\,.
  \label{e14-ls}
  \end{equation}
Здесь $\mathrm{MVL}$~--- множество визуальных метаязыков гетерогенного визуального 
поля, построенных в~соответствии с~(\ref{e13-ls}):
  \begin{gather*}
  \mathrm{MVL}=\left\{ \mathrm{mvl}_1,\ldots , \mathrm{mvl}_{N_{\mathrm{MVL}}}\right\}\,;
  \end{gather*}
  $\mathrm{COR}^{\mathrm{VL}}$~--- множество 
соответствий элементов визуальных языков, входящих в~метаязыки 
mvl$_i\hm\in \mathrm{MVL}$:
\begin{multline*}
  \mathrm{COR}^{\mathrm{VL}} ={}\\
  {}=\langle G^{\mathrm{VT}}, G^{\mathrm{VS}}, G^{\mathrm{VA}}, 
  G^{\mathrm{VP}}, G^{\upsilon\tau}, 
G^{\upsilon\sigma}, G^{\upsilon\alpha}, G^{\upsilon\pi}\rangle\,,
\end{multline*}
  где
\begin{alignat*}{3}
  G^{\mathrm{VT}}_{ij}:\ & \quad&\mathrm{VT}_i&\to \mathrm{VT}_j\,,\ &
  G^{\mathrm{VT}}_{ij}&\subseteq G^{\mathrm{VT}}\,;\\
  G_{ij}^{\mathrm{VS}}:\ & &\mathrm{VS}_i&\to \mathrm{VS}_j\,,\ &G_{ij}^{\mathrm{VS}}
 & \subseteq G^{\mathrm{VS}}\,;\\
  G_{ij}^{\mathrm{VA}}:\ &&\mathrm{VA}_i&\to \mathrm{VA}_j\,,\ &\quad G_{ij}^{\mathrm{VA}}
  &\subseteq G^{\mathrm{VA}}\,;\\
  G_{ij}^{\mathrm{VP}}:\ &&\mathrm{VP}_i&\to \mathrm{VP}_j\,,\ &\quad G_{ij}^{\mathrm{VP}}
  &\subseteq G^{\mathrm{VP}}\,;\\
  G_{ij}^{\upsilon\tau}:\ &&\upsilon\tau_i&\to \upsilon\tau_j\,,\ 
&G^{\upsilon\tau}_{ij}&\subseteq G^{\upsilon\tau}\,;\\
  G_{ij}^{\upsilon\sigma}:\ &&\upsilon\sigma_i&\to \upsilon\sigma_j\,,\ 
&G_{ij}^{\upsilon\sigma}&\subseteq G^{\upsilon\sigma}\,;\\
  G_{ij}^{\upsilon\alpha}:\ &&\upsilon\alpha_i&\to \upsilon\alpha_j\,,\ 
&G_{ij}^{\upsilon\alpha} &\subseteq G^{\upsilon\alpha}\,;\\
  G_{ij}^{\upsilon\pi}:\ &&\upsilon\pi_i&\to \upsilon\pi_j\,,\ &G_{ij}^{\upsilon\pi} 
&\subseteq G^{\upsilon\pi}\,, 
\end{alignat*}

\vspace*{-6pt}

\noindent
$$
  \hspace*{40mm}i,j\in \left[1, N_{\mathrm{MVL}}\right]\,,\enskip i\not=j\,.
  $$
 
  
  Предлагаемая неформальная аксиоматическая тео\-рия ролевых визуальных 
моделей~(\ref{e1-ls})--(\ref{e12-ls}) с~учетом моделей многослойной модели 
визуального языка~(\ref{e13-ls}) и~гетерогенного визуального  
поля~(\ref{e14-ls})\linebreak будет положена в~основу нового класса функциональных 
ГиИС, имитирующих работу коллек\-тивного интеллекта по поиску решений над 
ге\-те\-ро\-генными модельным и~визуальным полями. \mbox{Сочетание} символьных 
и~ви\-зу\-аль\-но-об\-раз\-ных рассуждений в~таких системах обеспечит их 
релевантность реальным коллективам, принимающим решения в~условиях 
сложных задач.

\vspace*{-3pt}
  
\section{Заключение}

  Предложена неформальная аксиоматическая тео\-рия ролевых визуальных 
моделей~--- основа автоматизированного решения сложных задач на основе 
визуальных образов, визуального управления. Предложена многослойная 
модель визуального языка и~формализованная модель гетерогенного 
визуального поля.
  
  Использование указанных моделей дает возможность реализовать ГиИС, 
способные динамически синтезировать интегрированную модель и~метод над 
гетерогенными модельным и~визуальным полями и~имитировать 
сотрудничество, относительность и~дополнительность коллективного 
интеллекта для поиска решений на символьных и~визуальных языках. Гибридные интеллектуальные
сис\-те\-мы 
такого класса смогут управ\-лять имитационным процессом в~зависимости от 
неопределенности проблемной ситуации: когда область явлений 
формализована (частично формализована), подключать для поиска решений 
знания экспертов из гетерогенного модельного поля, а когда есть существенная 
неопределенность, не сни\-ма\-емая точным анализом 
  и~ло\-ги\-ко-ма\-те\-ма\-ти\-че\-ски\-ми рассуждениями, привести в~действие 
механизмы ви\-зу\-аль\-но-про\-стран\-ст\-вен\-но\-го, образного мышления, 
имитируя <<скачки>> в~гибридном пространстве состояний функциональной 
ГиИС, соответству\-ющие мгновенному 
интуитивному инсайту, озарению, прерывающему  
ло\-ги\-ко-ма\-те\-ма\-ти\-че\-ские рассуждения.
  
{\small\frenchspacing
 {%\baselineskip=10.8pt
 \addcontentsline{toc}{section}{References}
 \begin{thebibliography}{99}
  \bibitem{1-ls}
  \Au{Golin E.\,J., Reiss S.\,P.} The specification of visual language syntax~// 
J.~Visual Lang. Comput., 1990. Vol.~1. P.~141--157.
  \bibitem{2-ls}
  \Au{Bowman W.\,J.} Graphic communication.~--- New York, NY, USA: John 
Wiley, 1968. 210~p. 
\bibitem{3-ls}
\Au{Lakin F.} Visual grammars for visual languages~// 6th National Conference on 
Artificial Intelligence Proceedings.~--- Menlo Park, CA, USA: AAAI 
Press, 1987. P.~683--688. 
  \bibitem{4-ls}
  \Au{Narayanan N.\,H., Hubscher R.} Visual language theory: Towards 
  a~human--computer interaction perspective~// Visual language theory.~--- New 
York, NY, USA: Springer-Verlag, 1998. P.~81--128.
 \bibitem{8-ls} %5
  \Au{Kremer R.} Visual languages for knowledge representation~//  
11th Workshop on Knowledge Acquisition, Modeling and Management, 1998. {\sf 
http://ksi.cpsc.\linebreak ucalgary.ca/KAW/KAW98/kremer}.
  \bibitem{5-ls} %6
  \Au{Осипов Г.\,С.} От ситуационного управления к~прикладной семиотике~// 
Новости искусственного интеллекта, 2002. №\,6(54). С.~3--7. 

  \bibitem{7-ls} %7
  \Au{Fitrianie S., Rothkrantz~L.\,J.\,M.}  Two-dimensional visual language 
grammar.~--- Delft, The Netherlands: Delft University of Technology, 
2008. {\sf http://mmi.tudelft.nl/ pub/siska/TSD~2DVisLangGrammar.pdf}.
\bibitem{6-ls} %8
\Au{Sibbet D.} Visual leaders: New tools for visioning, management, and 
organization change.~--- Hoboken, NJ, USA: Wiley, 2013. 229~p.
 
  \bibitem{9-ls} %9
  \Au{Тарасов В.\,Б.} Проблема понимания: настоящее и~будущее 
искусственного интеллекта~// Открытые семантические технологии 
проектирования интеллектуальных систем: Мат-лы V~Междунар. 
науч.-технич. конф.~--- Минск: БГУИР, 2015. С.~25--42.
  \bibitem{10-ls}
  \Au{Колесников А.\,В., Кириков~И.\,А.} Методология и~технология решения 
сложных задач методами функциональных гибридных интеллектуальных  
сис\-тем.~--- М.: ИПИ РАН, 2007. 387~с.
  \bibitem{11-ls}
  \Au{Колесников А.\,В.} Гибридные интеллектуальные системы. Теория 
и~технология разработки.~--- СПб.: \mbox{СПбГТУ}, 2001. 711~с.
  \bibitem{12-ls}
  \Au{Колесников А.\,В., Листопад~С.\,В.}  
Кон\-цеп\-ту\-аль\-но-ви\-зу\-аль\-ные основы виртуальных гетерогенных 
коллективов, поддерживающих принятие решений~// Гиб\-рид\-ные 
и~синергетические интеллектуальные\linebreak системы: Мат-лы III~Всеросс. 
Поспеловской конф. с~междунар. участием.~--- Калининград: 
БФУ им.\ И.~Канта, 2016. С.~8--56.
  \bibitem{13-ls}
  \Au{Mazza R.} Introduction to information visualization.~--- London:  
Springer-Verlag, 2009. 139~p.


  \bibitem{14-ls}
  \Au{Lengler R., Eppler~M. }A~periodic table of visualization methods~// Visual 
literacy: An e-learning tutorial on visualization for communication, engineering 
and business. {\sf http://www.visual-literacy.org/periodic\_ table/periodic\_table.html}.
  \bibitem{15-ls}
  \Au{Li K., Tiwari A., Alcock~J., Bermell-Garcia~P.} Categorisation of 
visualisation methods to support the design of human--computer interaction 
systems~// Appl. Ergon., 2016. Vol.~55. P.~85--107.
 \end{thebibliography}

 }
 }

\end{multicols}

\vspace*{-3pt}

\hfill{\small\textit{Поступила в~редакцию 16.10.16}}

%\vspace*{8pt}

\newpage

\vspace*{-28pt}

%\hrule

%\vspace*{2pt}

%\hrule

%\vspace*{8pt}


\def\tit{INFORMAL AXIOMATIC THEORY OF~THE~ROLE VISUAL MODELS}

\def\titkol{Informal axiomatic theory of~the~role visual models}

\def\aut{A.\,V.~Kolesnikov$^{1,2}$, S.\,V.~Listopad$^2$, S.\,B.~Rumovskaya$^2$, 
and~V.\,I.~Danishevsky$^1$}

\def\autkol{A.\,V.~Kolesnikov, S.\,V.~Listopad, S.\,B.~Rumovskaya, 
and~V.\,I.~Danishevsky}

\titel{\tit}{\aut}{\autkol}{\titkol}

\vspace*{-9pt}

 \noindent
  $^1$Immanuel Kant Baltic Federal University, 14~A.~Nevskogo Str., Kaliningrad 
236041, Russian Federation

   \noindent
   $^2$Kaliningrad Branch of the Federal Research Center ``Computer Science and 
Control'' of the Russian Academy\linebreak
$\hphantom{^1}$of Sciences, 5~Gostinaya Str, Kaliningrad 236000, 
Russian Federation



\def\leftfootline{\small{\textbf{\thepage}
\hfill INFORMATIKA I EE PRIMENENIYA~--- INFORMATICS AND
APPLICATIONS\ \ \ 2016\ \ \ volume~10\ \ \ issue\ 4}
}%
 \def\rightfootline{\small{INFORMATIKA I EE PRIMENENIYA~---
INFORMATICS AND APPLICATIONS\ \ \ 2016\ \ \ volume~10\ \ \ issue\ 4
\hfill \textbf{\thepage}}}

\vspace*{14pt}
  
  
  
   \Abste{The relevance of creation of the informal axiomatic theory of the role visual models is caused by 
modeling visual-imaginative reasoning in hybrid and synergistic intelligent systems. Most of the research in 
the field of visual-imaginative reasoning is focused on developing special visual languages to represent 
certain kinds of data, information, and knowledge. The lack of formal models of the visual languages is the 
cause of high research and development intensity of special media for handling and processing of visual 
models. Creation of the informal axiomatic theory of the role visual models is a~step to a~new class of 
intelligent systems that are relevant to the real decision-making teams, i.\,e., hybrid intelligent systems with 
heterogeneous visual field, imitating cooperation, complementarity, and relativity of collective intelligence, 
reasoning using the symbolic and visual languages.}

\vspace*{1pt}
   
   \KWE{hybrid intelligent system; heterogeneous visual field; visual language; semiotic system}
   
   \vspace*{1pt}
   
\DOI{10.14357/19922264160412} 

%\vspace*{6pt}

\Ack
  \noindent
  The research was supported by the Russian Foundation for Basic Research 
(project No.\,16-07-00271a).


\vspace*{12pt}

  \begin{multicols}{2}

\renewcommand{\bibname}{\protect\rmfamily References}
%\renewcommand{\bibname}{\large\protect\rm References}

{\small\frenchspacing
 {%\baselineskip=10.8pt
 \addcontentsline{toc}{section}{References}
 \begin{thebibliography}{99}
  
    \bibitem{1-ls-1}
  \Aue{Golin, E.\,J., and S.\,P.~Reiss.} 1990. The specification of visual language 
syntax. \textit{J.~Visual Lang. Comput.} 1:141--157.
  \bibitem{2-ls-1}
  \Aue{Bowman, W.\,J.} 1968. \textit{Graphic communication}. New York, NY: 
John Wiley. 210~p.
  \bibitem{3-ls-1}
  \Aue{Lakin, F.} 1987. Visual grammars for visual languages. \textit{6th National 
Conference on Artificial Intelligence Proceedings}. Menlo Park, 
CA: AAAI Press. 683--688. 
  \bibitem{4-ls-1}
  \Au{Narayanan, N.\,H., and R. Hubscher}. 1998. Visual language theory: Towards 
a~human--computer interaction perspective. \textit{Visual language theory}. New 
York, NY: Springer-Verlag. 81--128.
\bibitem{8-ls-1} %5
  \Aue{Kremer, R.} 1998. Visual languages for knowledge representation. 
\textit{11th Workshop on Knowledge Acquisition, Modeling and Management}. Available at: 
{\sf http:// ksi.cpsc.ucalgary.ca/KAW/KAW98/kremer/} (accessed September~5, 2016).
  \bibitem{5-ls-1} %6
  \Aue{Osipov, G.\,S.} 2002. Ot situatsionnogo upravleniya k~prikladnoy semiotike 
[From situational management to applied semiotics]. \textit{Novosti iskusstvennogo 
intellekta} [Artificial Intelligence News] 6(54):3--7.

\bibitem{7-ls-1} %7
  \Aue{Fitrianie, S., and L.\,J.\,M.~Rothkrantz}. 2008.
  \textit{Two-dimensional visual language 
grammar}. Delft, The Netherlands: Delft University of Technology.
Available at: {\sf http:// mmi.tudelft.nl/pub/siska/TSD2DVisLangGrammar.pdf} 
(accessed September~5, 2016).

 
  \bibitem{6-ls-1} %8
  \Aue{Sibbet, D.} 2013. \textit{Visual leaders: New tools for visioning, 
management, and organization change}. Hoboken, NJ: Wiley. 229~p.
  
 
  \bibitem{9-ls-1}
  \Aue{Tarasov, V.\,B.} 2015. Problema ponimaniya: nastoyashchee i~budushchee 
iskusstvennogo intellekta [The problem of understanding: The present and the future 
of artificial intelligence]. \textit{5th Scientific and Technical 
Conference (International) ``Open Semantic 
Technologies for Intelligent Systems'' Proceedings}. Minsk: BSUIR. 
25--42.
  \bibitem{10-ls-1}
  \Aue{Kolesnikov, A.\,V., and I.\,A.~Kirikov}. 2007. \textit{Metodologiya 
i~tekhnologiya resheniya slozhnykh zadach metodami funktsional'nykh gibridnykh 
intellektual'nykh sistem} [Methodology and technology of solving complex problems 
by the methods of functional hybrid intelligent systems]. Moscow: IPI RAN. 387~p.
  \bibitem{11-ls-1}
  \Aue{Kolesnikov, A.\,V.} 2001. \textit{Gibridnye intellektual'nye sistemy. Teoriya 
i~tekhnologiya razrabotki} [Hybrid intelligent systems: Theory and technology of 
development]. St. Petersburg: SPbGTU Publ. 711~p.
  \bibitem{12-ls-1}
  \Aue{Kolesnikov, A.\,V., and S.\,V.~Listopad}. 2016. Kon\-tsep\-tu\-al'\-no-vi\-zu\-al'\-nye 
osnovy virtual'nykh geterogennykh kollektivov, podderzhivayushchikh prinyatie 
resheniy [Conceptual and visual basics of virtual heterogeneous teams\linebreak supporting 
decision-making]. \textit{Gibridnye i~sinergeticheskie intellektual'nye sistemy: 
mat-ly III~Vseross. Pospelovskoy konf. s~mezhdunar. uchastiem}  
[3rd All-Russia Pospelov Conference with International Participation ``Hybrid and 
synergistic intelligent systems'' Proceedings]. Kaliningrad: IKBFU Publ. 8--56.

\pagebreak

  \bibitem{13-ls-1}
  \Aue{Mazza, R.} 2009. \textit{Introduction to information visualization}. London: 
Springer-Verlag. 139~p.


  \bibitem{14-ls-1}
  \Aue{Lengler, R., and M.~Eppler}. A~periodic table of visualization methods. 
\textit{Visual 
literacy: An e-learning tutorial on visualization for communication, engineering 
and business}. 
Available at: {\sf http://www.visual-literacy.org/periodic\_\linebreak table/periodic\_table.html} 
(accessed September~5,\linebreak 2016).
  \bibitem{15-ls-1}
  \Aue{Li, K., A.~Tiwari, J.~Alcock, and P.~Bermell-Garcia}. 2016. Categorisation 
of visualisation methods to support the design of human--computer interaction 
systems. \textit{Appl. Ergon.} 55:85--107.
\end{thebibliography}

 }
 }

\end{multicols}

\vspace*{-3pt}

\hfill{\small\textit{Received October 16, 2016}}
  
  \Contr
  
  \noindent
  \textbf{Kolesnikov Alexander V.} (b.\ 1948)~--- Doctor of Science in 
technology; professor, Department of Telecommunications, Immanuel Kant Baltic 
Federal University, 14~A.~Nevskogo Str., Kaliningrad 236041, Russian Federation; 
senior scientist, Kaliningrad Branch of the Federal Research Center ``Computer 
Science and Control'' of the Russian Academy of Sciences, 5~Gostinaya Str, 
Kaliningrad 236000, Russian Federation, \mbox{avkolesnikov@yandex.ru} 
  
  \vspace*{4pt}
  
  \noindent
  \textbf{Listopad Sergey V.} (b.\ 1984)~--- Candidate of  Science (PhD) in 
technology, senior scientist, Kaliningrad Branch of the Federal Research Center 
``Computer Science and Control'' of the Russian Academy of Sciences, 5~Gostinaya 
Str, Kaliningrad 236000, Russian Federation, \mbox{ser-list-post@yandex.ru} 
  
  \vspace*{4pt}
  
  \noindent
  \textbf{Rumovskaya Sophiya B.} (b.\ 1985)~--- programmer~I, Kaliningrad 
Branch of the Federal Research Center ``Computer Science and Control'' of the 
Russian Academy of Sciences, 5~Gostinaya Str, Kaliningrad 236000, Russian 
Federation, \mbox{sophiyabr@gmail.com}  
  
  
  \vspace*{4pt}
  
  \noindent
  \textbf{Danishevskii Vladislav I.} (b.\ 1992)~--- PhD student, Immanuel Kant 
Baltic Federal University, 14~A.~Nevskogo Str., Kaliningrad 236041, Russian 
Federation; \mbox{danishevskii.v.i@mail.ru} 
\label{end\stat}


\renewcommand{\bibname}{\protect\rm Литература} 
    