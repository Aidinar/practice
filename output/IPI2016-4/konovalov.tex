\def\stat{konovalov}

\def\tit{О РАЗМЕЩЕНИИ ЗАДАНИЙ НА ДВУХ СЕРВЕРАХ ПРИ~НЕПОЛНОМ 
НАБЛЮДЕНИИ$^*$}

\def\titkol{О размещении заданий на двух серверах при неполном 
наблюдении}

\def\aut{М.\,Г.~Коновалов$^1$, Р.\,В.~Разумчик$^2$}

\def\autkol{М.\,Г.~Коновалов, Р.\,В.~Разумчик}

\titel{\tit}{\aut}{\autkol}{\titkol}

\index{Коновалов М.\,Г.}
\index{Разумчик Р.\,В.}


{\renewcommand{\thefootnote}{\fnsymbol{footnote}} \footnotetext[1]
{Работа выполнена при частичной поддержке РФФИ (проект 15-07-03406).}}


\renewcommand{\thefootnote}{\arabic{footnote}}
\footnotetext[1]{Институт проблем информатики Федерального 
исследовательского центра <<Информатика и~управление>> 
Российской академии наук, \mbox{mkonovalov@ipiran.ru}}
\footnotetext[2]{Институт проблем информатики Федерального исследовательского центра 
<<Информатика и~управление>> Российской академии наук,  Российский университет 
дружбы народов, \mbox{rrazumchik@ipiran.ru}}

  
    \Abst{Рассматривается задача эффективного распределения заданий в~системах обслуживания, 
состоящих из двух параллельно работающих серверов с~очередями неограниченной емкости и~одним 
диспетчером, немедленно распределяющим поступающие задания по серверам. При поступлении 
очередного задания диспетчер принимает решение, основываясь только на информации 
о~производительности серверов, распределении времени между поступлениями заданий и~распределении их 
размеров. Другая информация о состоянии системы (как, например, длина очередей) диспетчеру 
недоступна. Для таких ненаблюдаемых систем известно, что минимум среднего времени пребывания 
задания в~системе обеспечивает детерминированная (периодическая) стратегия. Поскольку при принятии 
решения диспетчер неявно наблюдает и~момент поступления задания в~систему, возникает следующий 
вопрос: возможно ли уменьшить среднее время пребывания в~системе, если учитывать предысторию 
моментов поступления заданий? В~работе с~помощью численных экспериментов дается положительный 
ответ. Диспетчеризация, которая позволяет это сделать, основывается помимо априорной информации 
о~системе на оценках состояния очереди по результатам имеющихся наблюдений. Достигаемый выигрыш 
составляет от 1,5\% до 10\% по сравнению с~известной оптимальной стратегией и~увеличивается при 
уменьшении дисперсии длины заданий. При малых значениях дисперсии предложенная стратегия является 
более эффективной, чем диспетчеризация по самой короткой очереди.}

\KW{системы с~параллельным обслуживанием; стратегии размещения заданий; диспетчеризация; 
управление при неполном наблюдении}

\DOI{10.14357/19922264160406} 


\vskip 10pt plus 9pt minus 6pt

\thispagestyle{headings}

\begin{multicols}{2}

\label{st\stat}

\section{Введение}

    Во многих технических, информационных и~других системах имеются параллельно 
работающие обслуживающие ресурсы (серверы), на которые поступают заявки на 
выполнение заданий. При этом возникает проблема построения оптимальной процедуры 
выбора сервера для выполнения очередного задания (проблема диспетчеризации). 
Оптимальность может пониматься в~разных смыслах, однако для распределенных 
компьютерных систем передачи, обработки и~хранения данных наиболее популярным 
критерием оптимальности является среднее время пребывания задания 
в~сис\-те\-ме (активно используется также среднее значение для отношения времени 
выполнения задания к~его размеру (в~зарубежной литературе~--- slowdown)). Проблема 
диспетчеризации является сложной, и~относящиеся к~ней задачи чаще всего не удается 
решить исключительно математическими методами. Поэтому нередким явлением стало 
применение методов статистического моделирования, эвристических идей и~инженерного 
подхода для получения удовлетворительных решений. К~настоящему времени по данной 
проблематике известно очень большое число работ теоретического и~прикладного 
характера (см., например, обзоры~\cite{3-kon, 2-kon}). Несмотря на это, круг нерешенных 
или не вполне решенных вопросов остается широким и, как свидетельствуют регулярно 
появляющиеся публикации в~ведущих периодических научных изданиях, исследования 
в~данной области актуальны~[3--5].
    
    Следуя~\cite{1-kon}, напомним, что оптимальная процедура в~рассматриваемых 
системах существенным образом зависит от:
    \begin{itemize}
\item  типа и~количества информации, доступной при принятии решения;
\item правил обработки заданий внутри серверов;
\item критерия оптимальности.
\end{itemize}
    
    В данной работе рассматривается проблема диспетчеризации в~следующих 
предположениях:
    \begin{enumerate}[(1)]
\item при принятии решения доступна априорная информация о системе 
(распределение времени между поступлениями заданий, распределение размера 
заданий, производительности серверов);
\item наблюдаются моменты поступления заданий и~известны принятые в~эти 
моменты решения;
\item серверы работают параллельно и~независимо друг от друга, переход заявок 
между серверами невозможен;
\item задания внутри сервера обслуживаются в~порядке поступления;
\item критерием оптимальности является минимум среднего времени пребывания 
задания в~сис\-теме.
\end{enumerate}
    
    
    Таким образом, в~данной статье речь пойдет об оптимальных с~точки зрения 
минимального среднего времени пребывания стратегиях в~<<почти>> ненаблюдаемых 
системах параллельной обработки заданий. Основная отличительная особенность 
постановки задачи в~том, что недоступна динамическая информация о~со\-сто\-янии 
системы (например, о~чис\-ле заданий на серверах, об остаточных временах 
обслуживания, о~размерах заданий и~др.). С~практической точки зрения данные системы 
находят применение при моделировании систем облачных вычислений, 
grid-систем\footnote{Строящихся на их основе систем добровольных вычислений (volunteer 
computing).} и~ряда других систем (см., например,~[6--8]).

\section{Постановка задачи}

    Проблема диспетчеризации порождает большое число задач, многие из которых 
формулируются в~рамках следующей схемы.
    
    В систему из двух серверов поступает рекуррентный входной поток заданий, так что 
интервалы времени между поступлениями заданий образуют последовательность 
независимых случайных величин с~одинаковым распределением. Задания имеют 
случайную длину (объем), которая определяется одним и~тем же распределением. Каждое 
поступившее задание должно быть немедленно направлено на один из серверов. Каждый 
сервер имеет неограниченный бункер (очередь) для хранения заданий. Выполнение 
заданий происходит в~порядке поступления, без перерывов между окончанием 
выполнения одного задания и~началом следующего (при наличии непустой очереди). 
Серверы работают независимо, без обмена заданиями. Описанная схема приводит 
к~следующей математической модели.
    
    Обозначим через $F$ функцию распределения длины интервала между 
последовательными поступлениями заданий, а~через~$G$~--- функцию распределения 
длины задания. Производительность (интенсивность обслуживания) одного из серверов 
полагаем равной~1. Производительность другого сервера (более быстрого) обозначим 
через~$v$ $(>1)$. Серверы и~связанные с~ними величины будем индексировать цифрами~0 
(более медленный сервер) и~1.
    
    Система функционирует в~непрерывном времени $t\hm\geq 0$. Пусть $0\hm\leq 
t_1<\cdots < t_n\hm<\cdots$~--- последовательность моментов поступления заданий в~систему, 
а~$\Delta_n \hm= t_{n+1}\hm- t_n$~--- промежутки между этими моментами. Обозначим 
через~$x_n$ длину задания, поступившего в~момент~$t_n$. Обозначим также 
через~$u_n^{(i)}$ время, необходимое для выполнения всех заданий, имеющихся на 
сервере~$i$ $(=0\vee1)$ к~моменту $t \hm= t_n\hm-0$ (т.\,е.\ в~момент, <<непосредственно 
пред\-шест\-ву\-ющий>> поступлению в~систему очередного задания). Уточним, что 
величины~$u_n^{(i)}$ включают, в~том числе, время, необходимое для завершения 
заданий, уже находящихся в~стадии непосредственного выполнения. Назовем эти 
величины <<остаточными временами>>. Решение (действие), при\-ни\-ма\-емое 
в~момент~$t_n$ относительно вновь поступившего задания, обозначим через~$y_n$. 
Полагаем, что $y_n\hm=0$, если задание направлено на сервер~0, и~$y_n\hm=1$, если 
решение противоположное. Очевидно, справедливы соотношения:
    \begin{align*}
    u_{n+1}^{(0)} &= \left( u_n^{(0)} +(1-y_n) x_n-\Delta_n\right)^+\,;\\
    u_{n+1}^{(1)} &= \left( u_n^{(1)} +y_n \fr{x_n}{v}-\Delta_n\right)^+\,,
    \end{align*}
где $a^+=\max (a,0)$.
    
    Задание, поступившее в~момент~$t_n$ и~обслуженное согласно правилу~$y_n$, 
проведет в~системе время, равное
    $$
    w_n= u_n^{(y_n)} +\fr{x_n}{1-y_n+y_n v}\,,
    $$
где первое слагаемое~--- это время, проведенное в~очереди, а~второе~--- время 
непосредственного выполнения. Таким образом, возникает общая задача отыскания такой 
стратегии $y\hm= \{y_n,\ n\hm=1,2,\ldots\}$, которая минимизировала бы предельное 
среднее время пребывания заданий в~системе, опре\-де\-ля\-емое как
$$
W= W(y;F,G) =\lim\limits_{N\to\infty} \fr{1}{N}\sum\limits_{n=1}^N {\sf E}_y w_n\,,
$$
где ${\sf E}_y$~--- интегрирование по мере, порождаемой последовательностью~$y$, 
а~существование предела обусловлено структурой этой последовательности, которая 
обычно выбирается из <<разумных соображений>> достаточно регулярной. 
Стратегия~$y$ называется диспетчеризацией, а~ее элементы $y_n$~--- правилами 
диспетчеризации.
    
    Сформулированная общая постановка задачи фактически распадается на ряд задач 
в~зависимости от того, на каком множестве допустимых диспетчеризаций осуществляется 
минимизация. В~свою очередь, задание множества допустимых стратегий можно 
интерпретировать с~точки зрения возможностей наблюдения за траекторией процесса. 
Если обозначить через~$h_n$ совокупность наблюдаемых параметров системы до 
момента принятия решения~$t_n$, то допустимое правило диспетчеризации имеет вид:
    $$
    y_n=f\left(h_n\right)\,,
    $$
где $f$~--- рандомизированная или детерми\-ни\-ро\-ванная функция со значениями~0 или~1, 
а~$h_n$ при\-нимает значения из некоторого множества наблюдений~$H_n$. Наиболее 
употребительными и~исследован\-ными являются следующие четыре варианта.
\begin{enumerate}[1.]
    \item  Крайний случай, приводящий к~наиболее бедному множеству допустимых 
стратегий~--- это отсутствие вообще каких-либо наблюдений, $H_n\hm= \emptyset$. 
В~этом случае допустимая стратегия\footnote{В зарубежной литературе~--- PAP (Probabilistic 
Allocation Policy), или RND (Random), или BS (Bernoulli splitting).} и~описывается 
одним-единственным па\-ра\-мет\-ром~--- вероятностью~$p$ выбора для очередного задания сервера 0. 
Известен ряд результатов, касающихся ее оптимальности, используя которые можно 
численно находить значения вероятностей~$p$ (см.,  
например,~[9--11]). Наиболее полно задача решается для пол\-ностью 
марковских систем и~систем с~входящим пуассоновским потоком заданий и~серверов типа 
$M/G/1$. В общих случаях (например, когда система состоит из серверов типа $G/G/s$ 
и~MAP$/G/s$, $s\hm\geq 1$), известны различные приближенные и~эвристические решения, 
которые получены с~использованием математического программирования. Важным 
обстоятельством, которое\linebreak позволяет упростить решение оптимизационной задачи в~случае 
вероятностной стратегии, является то, что если поток по\-сту\-па\-ющих в~сис\-те\-му заявок 
является рекуррентным, то и~<<прореженный>> поток на каж\-дый сервер так\-же является 
рекуррентным. Тогда система из нескольких серверов декомпозируется на несколько сис\-тем из 
  одного сервера, для которых можно использовать известные 
точные или приближенные формулы.
    \item Большее разнообразие в~выборе диспетчеризации получается, если допустить 
возможность наблюдения за траекторией принятых решений, что приводит к~допустимым 
правилам диспетчеризации вида:
    $$
    y_n=f\left( y_{n-k_n},\ldots, y_{n-1}\right)\,,
    $$
где $1\leq k_n\leq n-1$. В~этом случае предыс\-то\-рия к~моменту~$t_n$ определяется 
значением из множества $H_n\hm= Y_k\hm= \{0;1\}^{k_n}$, где чис\-ло~$k_n$ 
характеризует глубину предыстории, используемую в~момент~$t_n$. Большой интерес\linebreak 
представляют программные (детерминированные) стратегии, т.\,е.\ стратегии, 
параметризуемые бесконечными последовательностями $\{ a_1, a_2, \ldots, a_{n-1}, 
a_n,\ldots\}$, в~которых~$a_i$ означает, что  
\mbox{$a$-я} заявка направляется на сервер с~номером~$i$. Внимание к~ним связано 
с~интуитивным представлением о~том, что входящий поток на каждый сервер при 
детерминированной стратегии является более регулярным (т.\,е.\ <<менее>> случайным), 
чем при вероятностной, что может приводить к~уменьшению значения среднего времени 
пребывания в~системе (см.,\linebreak
 например,~\cite{14-kon, 13-kon}). В~общем случае (для 
произвольного числа серверов) нахождение оп\-тимальной детерминированной стратегии\linebreak 
является сложной задачей. Если детерминированная стратегия предписывает на\-прав\-лять 
на каждый из $N\hm\geq 2$ серверов заявки в~соот\-вет\-ст\-вии с~распределением $(p_1, \ldots, 
p_N)$, то из~\cite{15-kon} известно, что оптимальными\linebreak являются так называемые 
сбалансированные последовательности\footnote{Если последовательность состоит только 
из нулей и~единиц, то она называется сбалансированной, если число единиц в~любых двух 
произвольно выделенных подпоследовательностях фиксированной длины отличается не 
более чем на~1.}~\cite{6-kon}. Однако сбалансированные последовательности для 
произвольного распределения $(p_1, \ldots, p_N)$ существуют лишь в~редких случаях. 
Такими случаями являются система из произвольного чис\-ла одинаковых 
серверов\footnote{Например, серверов типа $\bullet/G/1$.} (здесь оптимальной является 
цик\-ли\-че\-ская стратегия\footnote{$n$-е поступающее задание направляется на сервер 
с~номером $(n\,\mathrm{mod}\,R)\hm+1$.}) и~случай двух серверов (здесь при 
рациональном значении~$p$ оптимальной является так на\-зы\-ва\-емая последовательность 
Битти\footnote{В~литературе встречаются и~другие названия~--- последовательность 
Штурма, бильярдная последовательность (см.\ подробнее в~\cite{24-kon, 16-kon}. Заметим, 
что для реализации такой диспетчеризации вообще не требуются наблюдения за 
траекторией принятых решений, а~для определения нужного сервера необходимо знать 
лишь порядковый номер поступающей заявки.}). Заметим, что в~последнем случае 
оптимальность стратегии зависит от значения~$p$ и~способа нахождения точного 
значения до сих пор не предложено (см.,\linebreak например,~\cite{18-kon, 17-kon}). Тем не менее 
простой эвристический подход к~нахождению значения~$p$ приводит к~значениям 
целевой функции, которые не удается уменьшить, не привлекая при диспетчеризации 
дополнительную информацию о~системе. При $N\hm\geq 3$ сбалансированную 
последовательность удается построить лишь в~частных случаях (см.\ 
подробнее в~\cite{1-kon, 6-kon}). Поэтому действуют по-дру\-го\-му: для заданного 
распределения $(p_1,\ldots, p_N)$ ищут детерминированную последовательность, 
расстояние\footnote[1]{Подробнее о том, как определяется расстояние между последовательностями, 
см.~\cite{16-kon}.} которой от сбалансированной последовательности было бы 
минимальным. Эта задача является комбинаторной, и~для нее известно несколько 
численных алгоритмов решения~\cite{16-kon, 19-kon, 20-kon}. В~наиболее важных случаях 
(в~случаях рациональных значений~$p_i$) результаты работы этих алгоритмов приводят к~периодическим последовательностям и~последовательностям специального вида~--- так 
называемым бильярдным последовательностям (см., например,~\cite{16-kon}).
    \item Частичное наблюдение за состоянием очереди означает, что постоянно известен 
ее размер. В~этом варианте, как правило, рассматриваются правила диспетчеризации вида
    $$
    y_n=f(l_n)\,,
    $$
где $l_n$~--- размер очереди к~моменту принятия решения в~момент~$t_n$. Таким 
образом, $H_n\hm= N\hm= \{0,1,2,\ldots\}$. Основным примером такого рода является 
диспетчеризация JSQ (Join the Shortest Queue), направляющая задание на сервер с~минимальной очередью. Она 
является оптимальной в~случае пуассоновского входящего потока, экспоненциальных 
времен обслуживания с~одинаковыми параметрами и~с~дисциплинами FCFS (first come~--- first served)
на всех 
серверах (подробнее см., например, обзоры~\cite{3-kon, 2-kon}).
    \item Наиболее полный вариант наблюдений предполагает возможность использовать 
при выборе сервера значения остаточных времен~$u_n^{(0)}$ и~$u_n^{(1)}$, а~также 
длину~$x_n$ нового задания. В~этом случае оптимальная диспетчеризация находится 
в~множестве стратегий вида 
$$y_n\hm= f\left(u_n^{(0)}, u_n^{(1)}, x_n\right)\,. $$

Среди 
диспетчеризаций такого вида простым и~в~то же время достаточно эффективным 
решением является стратегия myopic, когда вновь поступающее задание посылается на тот 
сервер, который минимизирует время, необходимое для освобождения от заданий всей 
системы целиком, в~предположении, что в~дальнейшем задания в~систему не поступают. 
Более сложная и~менее универсальная стратегия deep получена с~помощью использования 
приближенного алгоритма из теории марковского процесса принятия  
решений~\cite{21-kon}. Стратегия deep, возможно, является <<квазиоптимальной>> в~том 
смысле, что она дает значение целевой функции, близкое к~(неизвестному) оптимальному 
значению. Однако, несмотря на формальное сведение задачи диспетчеризации 
к~марковскому процессу принятия решений в~варианте с~полным наблюдением, до сих 
пор не известно, как находить оптимальное значение функционала.
    \end{enumerate}
    
    Следует подчеркнуть, что рассмотренные выше четыре основных примера не 
исчерпывают все возможности постановки задачи. Например, для вариантов с~неполным 
наблюдением можно ставить вопрос о том, улучшится ли показатель~$W$, если 
учитывать предысторию наблюдаемых компонент. Одна из таких постановок 
рассматривается далее.

\vspace*{-9pt}

\section{Диспетчеризация по~предыстории действий}

    Постановка задачи~2 из предыдущего раздела ограничивает доступные наблюдения 
совершенными действиями. Это весьма жесткие ограничения на допустимые стратегии 
диспетчеризации, которые влекут существенный проигрыш в~целевой функции по 
сравнению со стратегиями, использу\-ющи\-ми максимальную текущую информацию. Далее 
эти ограничения сохраняются, и~по-преж\-не\-му предполагается, что недоступны 
наблюдения, отражающие <<состояние бункера>>~--- количество заданий, объем 
нагрузки в~очереди, моменты начала и~окончания непосредственного выполнения за\-дания. 
Неизвестной также считается длина поступающего задания. Однако информация 
о~совершенных действиях понимается несколько более широко. Побудительный мотив 
можно выразить такими словами: зная, <<что было сделано>>, естественно допустить, что 
известно также, <<когда было сделано>>. Точнее говоря, помимо самих решений~$y_n$ 
предполагается использовать информацию о моментах времени~$t_n$, в~которые эти 
решения принимались. Таким образом, далее предполагается, что допустимыми являются 
диспетчеризации, правила которых основываются на предыстории принятых\linebreak\vspace*{-12pt}

\pagebreak

\noindent
 решений 
и~моментов поступления заданий. Это означает, что допустимым считается правило 
диспетчеризации, которое представимо в~виде (детерминированной или 
рандомизированной) функции вида:
$$
y_n= f\left(y_1, \ldots, y_{n-1}, t_1,\ldots, t_n\right)\,,
$$
или
$$
y_n= f\left( y_1,\ldots, y_{n-1}, \Delta_1,\ldots , \Delta_{n-1}\right)\,.
$$
    
    Множество доступных наблюдений к~моменту поступления $n$-го задания выражается 
как
    $$
    H_n=Y^n\times R^n\,,
    $$
где $R=[0,\infty)$.
    
    Обратим внимание, что сделанные предположения означают, что диспетчеризация 
происходит в~ситуации, когда не наблюдаемы важные для решения задачи оптимизации 
характеристики. Более того, не наблюдается даже показатель, подлежащий минимизации, 
в~данном случае это величина~$w_n$~--- время, проведенное в~системе заданием, 
поступившим в~момент~$t_n$. При этом остаются неизвестными обе компоненты этого 
показателя: и~время ожидания в~очереди, и~время непосредственного выполнения задания.
    
    Вопрос, который нас интересует: дает ли незначительная дополнительная информация 
(учет\linebreak
 моментов поступления заданий) возможность улучшить показатель качества~$W$? 
Интуитивная пред\-посылка для возможного положительного ответа\linebreak заключает\-ся 
в~следующем простом замечании: большие промежутки времени между поступлениями 
заданий повышают вероятность того, что серверы находятся в~состоянии с~меньшим 
остаточным временем, и~наоборот. Чтобы воспользоваться этим довольно расплывчатым 
соображением, можно попытаться на основании доступной информации получить хотя бы 
приближенную оценку остаточных времен. Будем обозначать такую <<оценку>> 
к~моменту~$t_n$ принятия очередного решения через $\hat{u}_n^{(i)}$ и~положим 
$\hat{u}_1^{(i)}\hm=0$, $i\hm=0\vee 1$.
    
    Для $n=2$ наблюдаемая предыстория~--- это пара $h_1\hm= (y_1,\Delta_1=t_2-t_1)$. 
Оценку~$\hat{u}_2^{(i)}$ определим как ${\sf E}_{h_1} u_2^{(i)}$, 
    где ${\sf E}_{h_1}$~--- условное математическое ожидание при условии, что 
предыстория к~моменту~$t_2$ была~$h_1$. Продолжая аналогичным образом, для 
$n\hm=2,\ldots, k$ определим оценки $\hat{u}_n^{(i)}\hm= {\sf E}_{h_n} u_n^{(i)}$, где 
$h_n\hm= (y_1,\Delta_1,\ldots , y_{n-1}, \Delta_{n-1})$.
    
    Фиксированное натуральное число~$k$ будем называть глубиной памяти. Начиная 
с~номера $n\hm= k\hm+1$, будем строить оценки, исходя из <<усеченной>> предыстории:
$$
h_{n,k}=\left( y_{n-k},\Delta_{n-k},\ldots ,y_{n-1}, \Delta_{n-1}\right)\,.
$$
    
    Полагаем $\hat{u}_n^{(i)} \hm= {\sf E}_{h_{n,k}}u_n^{(i)}$,  
где~${\sf E}_{h_{n,k}}$~--- это условное математическое ожидание при условии, что 
(а)~наблюдаемая часть предыстории на предшествующих~$k$ тактах была~$h_{n,k}$ 
и~(б)~остаточные времена к~моменту~$t_{n-k}$ равнялись $u^{(i)}_{n-k}\hm= 
\hat{u}^{(i)}_{n-k}$.
    
    Воспользуемся теперь оценками остаточных времен для построения диспетчеризации, 
которую обозначим MEM. За эвристическую основу возьмем алгоритм порогового 
типа~\cite{22-kon}, который является эффективной диспетчеризацией, а~в~случае, когда 
все задания имеют одинаковую длину, является даже оптимальным. Правило 
диспетчеризации определяем следующим образом: задание, поступившее в~момент~$t_n$,
\begin{itemize}
    \item
    направляется на сервер~0, если $\hat{u}_n^{(0)}\hm+T \hm < \hat{u}_n^{(1)}$;
    \item
    направляется на сервер~1, если $\hat{u}_n^{(0)}\hm+T \hm \geq \hat{u}_n^{(1)}$.
    \end{itemize}
    
    Пороговое значение~$T$ является фиксированной неотрицательной величиной.
    
    Для реализации диспетчеризации MEM необходимо уметь вычислять оценки 
остаточных времен и~определять пороговое значение. Оценки~$\hat{u}_n^{(i)}$ можно 
получать, усредняя результаты многократной имитации отрезка траектории процесса. Для 
данного~$n$   длина имитируемого отрезка составляет $l\hm= \min (k, n-k)$, начальное 
значение остаточных времен принимается равным~$\hat{u}^{(i)}_{n-l}$, а~значения 
действий и~промежутков между ними фиксированы и~совпадают с~наблюденной 
предысторией~$h_{n,k}$. Таким образом, одна имитационная итерация сводится  
к~$l$-крат\-ной реализации случайной величины с~распределением длины задания~$G$ 
и~соответствующему пересчету остаточных времен. Отметим, что, несмотря на конечную 
глубину предыстории, используемой при расчете оценок на каждом шаге, фактически 
диспетчеризация MEM учитывает, хотя и~косвенно, всю предысторию.

\begin{table*}\small %tabl1
\begin{center}
\Caption{Оптимальные значения параметра для стратегий RND и~PROG}
\vspace*{2ex}

\begin{tabular}{|c|c|c|c|c|c|c|c|c|c|c|}
\hline
Интенсивность &\multicolumn{10}{c}{Распределение длины заданий}\\
\cline{2-11}
входного &\multicolumn{2}{c|}{Экспоненциальное}&
\multicolumn{2}{c|}{Равномерное}&
\multicolumn{2}{c|}{Парето}&\multicolumn{2}{c|}{Нормальное}&
\multicolumn{2}{c|}{Вырожденное}\\
\cline{2-11}
потока&RND&PROG&RND&PROG&RND&PROG&RND&PROG&RND&PROG\\
\hline
0,6&0,01&0,85&0&1&0,01&0,99&0&1&0&1\\
0,9&0,14&0,75&0,07&0,78&0,08&0,78&0,04&0,83&0,03&0,86\\
1,2&0,21&0,73&0,18&0,72&0,20&0,75&0,17&0,73&0,16&0,75\\
1,5&0,25&0,71&0,24&0,70&0,25&0,73&0,23&0,69&0,23&0,68\\
1,8&0,28&0,70&0,27&0,68&0,28&0,70&0,27&0,67&0,27&0,67\\
2,1&0,30&0,69&0,30&0,68&0,30&0,68&0,29&0,67&0,30&0,67\\
2,4&0,31&0,68&0,31&0,68&0,32&0,67&0,31&0,67&0,31&0,67\\
    \hline
    \end{tabular}
    \end{center}
    \end{table*}
    
    
    Способ вычисления оптимального порога при детерминированных и~одинаковых 
длинах заданий описан в~\cite{23-kon}. Поскольку в~рассматриваемом случае нет 
 ка\-ко\-го-ли\-бо теоретически обоснованного способа определения наилучшего 
порогового значения, так же как нет теоретического обоснования для возможного 
преимущества стратегии MEM, то в~качестве параметра~$T$ предлагается брать значение, 
которое получается с~помощью алгоритма из~\cite{23-kon}.

\vspace*{-6pt}

\section{Результаты численных расчетов}

\vspace*{-3pt}

\subsection{Условия эксперимента}

    Рассматривается система из двух серверов: 0 и~1. Предполагается, что 
производительность одного сервера равна~1, а~другого~---~2. Входящий\linebreak\vspace*{-12pt}

\pagebreak

\noindent
 в~систему поток 
является пуассоновским с~па\-ра\-мет\-ром~$\lambda$.
    
    Распределение длин поступающих заданий выбирается из следующих вариантов:
    \begin{itemize}
     \item[(а)] экспоненциальное распределение с~па\-ра\-мет\-ром~1 (среднее и~дисперсия 
равны~1);
     \item[(б)] равномерное распределение на отрезке $[0{,}1; 1{,}9]$ (среднее равно~1, 
дисперсия~--- 0,27);
     \item[(в)] распределение Парето с~функцией распределения $F(t)\hm= 1\hm- (b/t)^a$, 
$t\hm\geq b$, где $b\hm= 0{,}6$, $a\hm= 2{,}5$ (среднее~---~1, дисперсия~--- 0,8);
     \item[(г)] $\max(10^{-7},\xi)$, где~$\xi$ имеет нормальное распределение со 
средним~1 и~дисперсией~0,1;
     \item[(д)] вырожденное распределение~--- длина всех заданий равна~1.
     \end{itemize}
    
    Далее проводится сравнение значений целевой функции для четырех диспетчеризаций 
(RND, PROG, JSQ и~MEM) при разных распределениях длины заданий и~при разных 
интенсивностях входного потока.
    
    Под стратегией RND будем понимать описанный в~разд.~2 алгоритм, в~котором 
серверы выбираются с~фиксированной вероятностью. В~нашем\linebreak случае стратегия RND 
задается единственным параметром~$p$~--- вероятностью выбора первого сервера. 
Предполагается, что этот параметр выбран оптимальным образом.
    
    Стратегия PROG использует упомянутую в~разд.~2 детерминированную 
последовательность Битти, состоящую из нулей и~единиц. Правило этой стратегии 
заключается в~том, чтобы направлять~\mbox{$k$-е} по счету задание на сервер~$i$, если $b_k\hm= 
\lfloor (k\hm+1)p \hm+ \varphi\rfloor \hm- \lfloor kp\hm+\varphi\rfloor \hm= i$, $i\hm= 0\vee 1$. 
Здесь $-\infty \hm< \varphi\hm < \infty$ и~$0\hm< p\hm<1$~--- произвольные рациональные 
числа, а~скобки означают целую часть числа. Параметр~$\varphi$ (<<фаза>>) 
обусловливает только сдвиг детерминированной последовательности и~не влияет на 
значение целевой функции, поэтому полагаем $\varphi\hm=0$. А~вот от\linebreak\vspace*{-12pt}

\columnbreak

%\vspace*{1pt}

\noindent
{\small
 \begin{center} 
\parbox{62mm}{{{\tablename~2}\ \ \small{Пороговые значения для алгоритма MEM}}
}

\vspace*{6pt}

 %
% \tabcolsep=7pt
 \begin{tabular}{|c|c|}
\hline
\tabcolsep=0pt\begin{tabular}{c}Интенсивность\\ входного потока\end{tabular}& Пороговое значение\\
\hline
0,6&0,38\\
0,9&0,33\\
1,2&0,28\\
1,5&0,25\\
1,8&0,22\\
2,1&0,19\\
2,4&0,17\\
\hline
\end{tabular}
\end{center}
\vspace*{6pt}
}



\addtocounter{table}{1}

\noindent
 параметра~$p$ 
значение показателя~$W$ зависит существенно, и~снова предполагается, что это значение 
выбрано наилучшим способом.
    
    Отыскание оптимальных значений параметра~$p$  для стратегий RND и~PROG, по 
существу, представляет собой отдельные задачи, которые в~данном случае решаются 
с~помощью специального адаптивного алгоритма на имитационной модели. Найденные 
значения представлены для различных распределений длины задания и~различных 
интенсивностей входного потока в~табл.~1.
    

    Диспетчеризация JSQ направляет задания каж\-дый раз в~наиболее короткую очередь.
    
    Пороговые значения, необходимые для реализации алгоритма MEM, полученные 
в~результате численного решения задачи о наилучшем пороге (см.\ разд.~3), указаны 
в~табл.~2.
    


    Глубина памяти $k$ в~алгоритме MEM динамически корректировалась в~процессе 
имитации в~диапазоне от~2 до~6.

\vspace*{-6pt}

\subsection{Результаты эксперимента}

    Численные результаты сравнения диспетчеризаций получены методом  
Мон\-те Кар\-ло и~изображены на рисунке. Значения целевой 
функции для различных стратегий обозначаются соответственно $W^{\mathrm{MEM}}$, 
$W^{\mathrm{RND}}$, $W^{\mathrm{PROG}}$ и~$W^{\mathrm{JSQ}}$. 
Представлены
относительные значения функций $W^{\mathrm{MEM}}$, $W^{\mathrm{PROG}}$ 
и~$W^{\mathrm{JSQ}}$  к~значениям функции~$W^{\mathrm{RND}}$.

\pagebreak

\end{multicols}

\begin{figure*} %fig1
    \vspace*{1pt}
\begin{center}
\mbox{%
\epsfxsize=164.486mm
\epsfbox{kon-1.eps}
}
\end{center}
%\vspace*{-9pt}
    \noindent
    {\small Сравнение эффективности диспетчеризаций в~системе из двух серверов:
    \textit{1}~--- $W^{\mathrm{RND}}/W^{\mathrm{RND}}$;
    \textit{2}~--- $W^{\mathrm{PROG}}/W^{\mathrm{RND}}$;
    \textit{3}~--- $W^{\mathrm{MEM}}/W^{\mathrm{RND}}$;
    \textit{4}~--- $W^{\mathrm{JSQ}}/W^{\mathrm{RND}}$. Длина заданий: 
(\textit{а})~экспоненциально распределена с~параметром~1; (\textit{б})~равномерно распределена на отрезке 
$[0{,}1;1{,}9]$; (\textit{в})~имеет распределение Парето с~параметрами~2,5 и~0,6; (\textit{г})~распределена 
как $\max(10^{-7}, \xi)$, где $\xi$~--- нормально распределенная случайная величина со средним~1 
и~дисперсией~0,1; (\textit{д})~равна~1}
    \end{figure*}

\begin{multicols}{2}




    
    
    В рассмотренных примерах стратегия RND, которая не использует никаких 
наблюдений, играет роль точки отсчета и~заведомо менее эффективна по сравнению 
с~остальными. Стратегия PROG формально также не использует наблюдений, но 
эквивалентна периодической детерминированной стратегии, т.\,е.\ эквивалентна 
стратегии, использующей предысторию совершенных действий. Эта стратегия является 
основным объектом сравнения для стратегии MEM.

\pagebreak

\vspace*{-12pt}

\noindent
{\small
 \begin{center} 
\parbox{74mm}{{{\tablename~3}\ \ \small{Средний выигрыш в~эффективности при стратегии MEM по сравнению со стратегией DET}}
}

\vspace*{8pt}

 %
% \tabcolsep=7pt
 \begin{tabular}{|l|c|c|}
\hline
\multicolumn{1}{|c|}{Распределение}&
Дисперсия &
Выигрыш, \%\\
\hline
Экспоненциальное&1\hphantom{,9}&2,8\\
Равномерное&0,8&4,0\\
Парето&\hphantom{9}0,27&5,5\\
Нормальное&0,1&5,3\\
Вырожденное&0\hphantom{,9} &6,1\\
\hline
\end{tabular}
\end{center}
\vspace*{6pt}
}



\addtocounter{table}{1}



    
    Во всех случаях диспетчеризация MEM с~предыс\-то\-ри\-ей моментов поступления заявок 
позволяет улучшить значение показателя~$W$ по сравнению с~детерминированной 
стратегией PROG. Достигаемый выигрыш лежит для отдельных значений в~диапазоне 
от~1,5\% до~10,0\%. При этом наблюдается уменьшение выигрыша с~ростом нагрузки.
    
    Средний выигрыш при разных интенсивностях зависит от типа распределения длины 
задания и~от дисперсии (табл.~3). Наблюдается рост относительного среднего выигрыша 
с~уменьшением дисперсии длины заданий.
    


    
    С влиянием дисперсии можно также связать результат сравнения стратегий MEM 
и~JSQ. Диспетчеризация JSQ в~контексте данного эксперимента интересна тем, что 
использует наблюдения, хотя и~неполные, но связанные с~состояниями очередей. 
Интуитивно можно предположить, что такие наблюдения должны давать существенные 
преимущества по сравнению с~полностью ненаблюдаемы\-ми характеристиками нагрузки. 
Это предположение, однако, подтверждается на приведенных рисунках только для 
больших значений дисперсии длины заданий (см.\
\textit{а}--\textit{в} на рисунке). Для малых 
дис\-пер\-сий (т.\,е.\ менее~0,2, см.\ \textit{г} и~\textit{д} на рисунке), 
стратегия JSQ дает 
относительно меньший выигрыш по сравнению с~оптимальной детерминированной 
диспетчеризацией PROG и, более того, проигрывает по\linebreak сравнению со стратегией MEM. 
Можно предположить, что последнее обстоятельство связано с~усилением влияния 
предыстории в~ситуации, когда мала дисперсия и,~соответственно, в~меньшей степени 
присутствует эффект перемешивания.

\section{Заключение}

    Неполнота наблюдений является частым ограничением во многих технических 
системах и~связана как с~недоступностью некоторых параметров для отслеживания, так 
и~с~невозможностью обработки слишком больших массивов наблюдений. Это порождает 
необходимость разрабатывать стратегии, основывающиеся на неполных или косвенных 
данных о состоянии процесса управления.
    
    Данная работа посвящена проблеме управления распределением заданий в~системе 
с~параллельным обслуживанием при минимальном наблюдении, когда неизвестны длины 
заданий и~состояния очередей. Известно, что если наблюдаема только последовательность 
номеров серверов, на которые направляются задания, то детерминированная стратегия 
(в~частных случаях~--- периодическая) позволяет получать оптимальные (или близкие 
к~оптимальным) значения среднего времени пребывания задания в~системе. Такая 
стратегия оказывается зависящей от предыстории действий. Возникает вопрос, улучшится 
ли эффективность, если дополнительно учитывать предысторию моментов поступления 
заданий. В~работе с~помощью численных экспериментов дается положительный ответ на 
этот вопрос на примере системы из двух серверов и~специальной эвристической 
диспетчеризации MEM, основанной на оценках состояния очереди по результатам 
имеющихся наблюдений и~априорной информации о распределении длины заданий. Во 
всех экспериментах с~различными интенсивностями пуассоновского входного потока 
и~разными распределениями длины заданий стратегия MEM показала выигрыш (от~1,5\% 
до~10\%) по сравнению с~оптимальной программной стратегией. Выигрыш оказался более 
существенным для малых значений дисперсии длины заданий. Для таких примеров 
стратегия MEM оказалась даже эффективнее, чем диспетчеризация по самой короткой 
очереди.
    
    Результаты проведенного анализа указывают на принципиальную возможность 
повышения качества частично наблюдаемых систем обслуживания путем построения 
стратегий диспетчеризации, учитывающих предысторию. Дальнейшие усилия будут 
направлены на разработку таких стратегий и~на повышение их практической ценности за 
счет уменьшения объема необходимой априорной информации о системе.
    
{\small\frenchspacing
 {%\baselineskip=10.8pt
 \addcontentsline{toc}{section}{References}
 \begin{thebibliography}{99}
    
 \bibitem{3-kon} %1
    \Au{Semchedine F., Bouallouche-Medjkoune~L., \mbox{A\!{\!\ptb{\"{\i}}}ssani}~D.} 
    Review: Task 
assignment policies in distributed server systems: A~survey~// J.~Netw. Comput. 
Appl., 2011. Vol.~34. No.\,4. P.~1123--1130.
    \bibitem{2-kon} %2
    \Au{Коновалов М.\,Г., Разумчик~Р.\,В.} Обзор моделей и~алгоритмов размещения 
заданий в~системах с~параллельным обслуживанием~// Информатика и~её применения, 
2015. Т.~9. Вып.~4. С.~56--67.
\bibitem{1-kon} %3
    \Au{Anselmi J., Gaujal B., Nesti~T.} Control of parallel non-observable queues: Asymptotic 
equivalence and optimality of periodic policies~// Stochastic Syst., 2015. Vol.~5.  
P.~120--145.
   
    
    \bibitem{5-kon} %4
    \Au{Gardner K., Borst S., Harchol-Balter~M.} Optimal scheduling for jobs with 
progressive deadlines~// IEEE Conference on Computer  
Communications (INFOCOM).~--- IEEE, 2015. P.~ 1113--1121.
\bibitem{4-kon} %5
    \Au{Hyyti$\ddot{a}$ E., Righter~R.} Routing jobs with deadlines to heterogeneous parallel 
servers~// Oper. Res. Lett., 2016. Vol.~44. No.\,4. P.~507--513.
    
 \bibitem{9-kon} %6
    \Au{Harchol-Balter M., Scheller-Wolf~A., Young~A.\,R.} Surprising results on task 
assignment in server farms with high-variability workloads~// ACM SIGMETRICS/Performance 
Proceedings.~--- Seattle: ACM, 2009. P.~287--298.
    \bibitem{7-kon} %7
    \Au{Javadi B., Kondo D., Vincent~J.-M., Anderson~D.\,P.} Discovering statistical models 
of availability in large distributed systems: An empirical study of seti@home~// IEEE Trans. 
Parall. Distr. Syst., 2011. Vol.~22. No.\,11. P.~1896--1903.
    \bibitem{8-kon} %8
    \Au{Javadi B., Thulasiraman~P., Buyya~R.} Cloud resource provisioning to extend the 
capacity of local resources in the presence of failures~// IEEE 14th  Conference 
(International)  on High Performance Computing and Communication  \&  IEEE 
9th  Conference (International) on Embedded Software and Systems.~--- IEEE 
Computer Society, 2012. P.~311--319.
   
    \bibitem{10-kon} %9
    \Au{Combe M.\,B., Boxma~O.\,J.} Optimization of static traffic allocation policies~// 
Theor. Comput. Sci., 1994. Vol.~125. No.\,1. P.~17--43.
 \bibitem{12-kon} %10
    \Au{Tang Ch.\,S., van Vliet~M.} Traffic allocation for manufacturing systems~// Eur.
J.~Oper. Res., 1994. Vol.~75. No.\,1. P.~171--185.
    \bibitem{11-kon} %11
    \Au{Sethuraman J., Squillante~M.\,S.} Optimal stochastic scheduling in multiclass parallel 
queues~// ACM SIGMETRICS'99 Proceedings.~--- New York, NY, USA: ACM, 1999.  
P.~93--102.
   
    
    \bibitem{14-kon} %12
    \Au{Humblet P.} Determinism minimizes waiting time in queues. The Laboratory for 
Information and Decision Systems Technical Report ser., 1982. LIDS-P/1207.
\bibitem{13-kon} %13
    \Au{Hajek B.} The proof of a folk theorem on queuing delay with applications to routing in 
networks~// J.~ACM, 1983. Vol.~ 30. No.\,4. P.~834--851. 
    \bibitem{15-kon} %14
    \Au{Altman E., Gaujal~B., Hordijk~A.} Balanced sequences and optimal routing~// 
J.~ACM, 2000. Vol.~47. No.\,4. P.~752--775. 
\bibitem{6-kon} %15
    \Au{Altman E., Gaujal B., Hordijk~A.} Balanced sequences and optimal routing~// 
J.~ACM, 2000. Vol.~47. No.\,4. P.~752--775. 
\bibitem{24-kon} %16
    \Au{Van der Laan~D.} The structure and performance of optimal routing sequences.~--- 
Universiteit Leiden, 2003. PhD Thesis.
    \bibitem{16-kon} %17
    \Au{Hordijk A., van der Laan~D.\,A.} Periodic routing to parallel queues and billiard 
sequences~// Math. Method. Oper. Res., 2004. Vol.~59. No.\,2. P.~173--192.
  \bibitem{18-kon} %18
    \Au{Gaujal B., Hyon~E.} Optimal routing policy in two deterministic queues~// 
Calculateurs \mbox{Parall{\!\ptb{\`{e}}}les}, 2001. Vol.~ 13. P.~601--634.
    \bibitem{17-kon} %19
    \Au{Van der Laan~D.\,A.} Routing jobs to servers with deterministic service times~// Math. 
Oper. Res., 2005. Vol.~30. No.\,1. P.~195--224.
  
    \bibitem{19-kon} %20
    \Au{Arian Y., Levy~Y.} Algorithms for generalized round robin routing~// Oper. Res. 
Lett., 1992. Vol.~ 12. P.~313--319.
    \bibitem{20-kon} %21
    \Au{Sano S., Miyoshi~N.} Applications of m-balanced sequences to some network 
scheduling problems~// Discrete Event System: Analysis and Control: 5th Workshop on Discrete 
Event Systems  Proceedings.~--- Kluwer Academic Publisher, 2000. 
 P.~317--325.
    \bibitem{21-kon} %22
    \Au{Hyyti$\ddot{\mbox{a}}$ E.} Lookahead actions in dispatching to parallel queues~// 
Perform. Evaluation, 2013. Vol.~ 70. No.\,10. P.~859--872.
    \bibitem{22-kon} %23
    \Au{Hyyti$\ddot{\mbox{a}}$ E.} Optimal routing of fixed size jobs to two parallel 
servers~// INFOR: Inform. Syst. Oper. Res., 2013. Vol.~51. No.\,4. 
P.~215--224.
    \bibitem{23-kon} %24
    \Au{Konovalov M., Razumchik~R.} Iterative algorithm for threshold calculation in the 
problem of routing fixed size jobs to two parallel servers~// J.~Telecommun. Inform. 
Technol., 2015. No.\,3. P.~32--38.
    
 \end{thebibliography}

 }
 }

\end{multicols}

\vspace*{-6pt}

\hfill{\small\textit{Поступила в~редакцию 19.09.16}}

\vspace*{8pt}

%\newpage

%\vspace*{-24pt}

\hrule

\vspace*{2pt}

\hrule

\vspace*{8pt}


\def\tit{DISPATCHING TO~TWO PARALLEL NONOBSERVABLE QUEUES 
USING ONLY STATIC INFORMATION}

\def\titkol{Dispatching to~two parallel nonobservable queues 
using only static information}

\def\aut{M.\,G.~Konovalov$^1$ and~R.\,V.~Razumchik$^{1,2}$}

\def\autkol{M.\,G.~Konovalov and~R.\,V.~Razumchik}

\titel{\tit}{\aut}{\autkol}{\titkol}

\vspace*{-9pt}


\noindent
$^1$Institute of Informatics Problems, Federal Research Center ``Computer Science and 
Control''' of the Russian\linebreak
$\hphantom{^1}$Academy of Sciences, 44-2~Vavilov Str., Moscow 119333, Russian 
Federation

\noindent
$^2$Peoples' Friendship University of Russia, 6~Miklukho-Maklaya Str., Moscow 117198, 
Russian Federation


\def\leftfootline{\small{\textbf{\thepage}
\hfill INFORMATIKA I EE PRIMENENIYA~--- INFORMATICS AND
APPLICATIONS\ \ \ 2016\ \ \ volume~10\ \ \ issue\ 4}
}%
 \def\rightfootline{\small{INFORMATIKA I EE PRIMENENIYA~---
INFORMATICS AND APPLICATIONS\ \ \ 2016\ \ \ volume~10\ \ \ issue\ 4
\hfill \textbf{\thepage}}}

\vspace*{3pt}


    \Abste{Consideration is given to the problem of dispatching independent jobs from one flow to two parallel 
single server queueing systems each with an infinite capacity queue. There is one dispatcher, which 
immediately\linebreak\vspace*{-12pt}}

\Abstend{makes decisions where to route newly incoming jobs. In order to make the decision, the dispatcher uses only static 
information about the system, i.\,e., servers' speeds, job interarrival time distribution, and job size distribution. The 
dynamic information (for example, current queue sizes) is unavailable for the dispatcher. For such 
nonobservable 
queues,  it is known that minimum mean sojourn time is achieved when the dispatcher uses the deterministic 
(periodic) policy. Even when using this optimal policy, the dispatcher also observes jobs' arrival instants but this 
information is left unused. The question posed in this paper is the following: Is it possible to reduce the mean 
sojourn time if one, in addition to the static information, also uses the historical information about jobs' arrival 
instants? Using simulation techniques, the authors show that the answer to the question is positive. Such policy uses 
static information and the estimates of the queue sizes based on multiple replications of the system's trajectory. 
Compared to the optimal policy, the achievable gain varies from~1,5\% to~10\%, and increases with the decrease of 
job size variance. When the job size variance is low, the proposed policy is even better than the dynamic  
join-the-shortest queue policy.}
\KWE{dispatching; static information; parallel service; queueing system; nonobservable queues}

\DOI{10.14357/19922264160406} 

%\vspace*{-9pt}

\Ack
\noindent
The research was supported by the Russian Foundation for Basic Research (project 15-07-03406).



\vspace*{12pt}

  \begin{multicols}{2}

\renewcommand{\bibname}{\protect\rmfamily References}
%\renewcommand{\bibname}{\large\protect\rm References}

{\small\frenchspacing
 {%\baselineskip=10.8pt
 \addcontentsline{toc}{section}{References}
 \begin{thebibliography}{99}
  \bibitem{3-kon-1} %1
    \Aue{Semchedine, F., L.~Bouallouche-Medjkoune, and D.~\mbox{A{\!\ptb{\"{\i}}}\,ssani}.} 2011. 
Review: Task assignment policies in distributed server systems: A~survey. \textit{J.~Netw.
Comput. Appl.} 34(4):1123--1130.
   
    \bibitem{2-kon-1} %2
    \Aue{Konovalov, M.\,G., and R.\,V.~Razumchik}. 2015. Obzor modeley i~algoritmov 
razmeshcheniya zadaniy v~sistemakh s~parallel'nym obsluzhivaniem [Methods and algorithms 
for job scheduling in systems with parallel service: A~survey]. \textit{Informatika i~ee 
primeneniya~--- Inform. Appl}.  9(4):56--67.
    \bibitem{1-kon-1} %3
    \Aue{Anselmi, J., B.~Gaujal, and T.~Nesti}. 2015. Control of parallel non-observable 
queues: Asymptotic equivalence and optimality of periodic policies. \textit{Stochastic 
Syst.}  5:120--145.
    
    \bibitem{5-kon-1} %4
    \Aue{Gardner, K., S. Borst, and M.~Harchol-Balter}. 2015. Optimal scheduling for jobs 
with progressive deadlines. \textit{IEEE Conference on Computer Communications 
(INFOCOM)}. IEEE. 1113--1121.
\bibitem{4-kon-1} %5
    \Aue{Hyyti$\ddot{\mbox{a}}$, E., and R.~Righter}. 2016. Routing jobs with deadlines to 
heterogeneous parallel servers. \textit{Oper. Res. Lett.} 44(4):507--513.
   
   
\bibitem{9-kon-1} %6
    \Aue{Harchol-Balter, M., A.~Scheller-Wolf, and A.\,R.~Young}. 2009. Surprising results on 
task assignment in server farms with high-variability workloads. \textit{ACM 
SIGMETRICS/Performance Proceedings}. Seattle: ACM. 287--298.
 \bibitem{7-kon-1}
    \Aue{Javadi, B., D. Kondo, J.-M.~Vincent, and D.\,P.~Anderson}. 2011. Discovering 
statistical models of availability in large distributed systems: An empirical study of seti@home. 
\textit{IEEE Trans. Parall. Distr. Syst.} 22(11):1896--1903.
    \bibitem{8-kon-1} %8
    \Aue{Javadi, B., P.~ Thulasiraman, and R.~Buyya}. 2012. Cloud resource provisioning to 
extend the capacity of local resources in the presence of failures. \textit{IEEE 14th  
Conference (International) on High Performance Computing and}\linebreak\vspace*{-12pt}

\columnbreak

\noindent
 \textit{Communication  \& 
 IEEE 9th  Conference (International) on Embedded Software and Systems 
Proceedings}. 311--319.
    
    \bibitem{10-kon-1} %9
    \Aue{Combe, M. B., and O.\,J.~Boxma}. 1994. Optimization of static traffic allocation 
policies. \textit{Theor. Comput. Sci.} 125(1):17--43.
   
    \bibitem{12-kon-1} %10
    \Aue{Tang, Ch.\,S., and M.~van Vliet}. 1994. Traffic allocation for manufacturing systems. 
\textit{Eur. J.~Oper. Res.} 75(1):171--85.
 \bibitem{11-kon-1} %11
    \Aue{Sethuraman, J., and M.\,S.~Squillante}. 1999. Optimal stochastic scheduling in 
multiclass parallel queues. \textit{ACM SIGMETRICS'99 Proceedings}. New York, NY: ACM. 
P.~93--102.
\bibitem{14-kon-1} %12
    \Aue{Humblet, P.} 1982. Determinism minimizes waiting time in queues. The Laboratory 
for Information and Decision Systems Technical Report ser. LIDS-P/1207.
    \bibitem{13-kon-1} %13
    \Aue{Hajek, B.} 1983. The proof of a folk theorem on queuing delay with applications to 
routing in networks. \textit{J.~ACM} 30(4):834--851. 
    
    \bibitem{15-kon-1} %14
    \Aue{Altman, E., B. Gaujal, and A.~ Hordijk.} 2000. Balanced sequences and optimal 
routing. \textit{J.~ACM}. 47(4):752--775. 
 \bibitem{6-kon-1} %15
    \Aue{Altman, E., B. Gaujal, and A.~Hordijk.} 2000. Balanced sequences and optimal 
routing. \textit{J.~ACM} 47(4):752--775. 
\bibitem{24-kon-1} %16
    \Aue{Van der Laan, D.} 2003. The structure and performance of optimal routing sequences.   
Universiteit Leiden. PhD Thesis.
    \bibitem{16-kon-1} %17
    \Aue{Hordijk, A., and D.\,A. van der~Laan.} 2004. Periodic routing to parallel queues and 
billiard sequences. \textit{Math. Method. Oper. Res.} 59(2):173--192.
\bibitem{18-kon-1} %18
    \Aue{Gaujal, B., and E.~Hyon.} 2001. 
    Optimal routing policy in two deterministic queues. 
\textit{Calculateurs \mbox{Parall{\!\!\ptb{\`{e}}}les}} 13:601--634.
    \bibitem{17-kon-1} %19
    \Aue{Van der Laan, D.\,A.} 2005. Routing jobs to servers with deterministic service times. 
\textit{Math. Oper. Res.} 30(1):195--224.
    
    \bibitem{19-kon-1} %20
    \Aue{Arian, Y., and Y.~Levy.} 1992. Algorithms for generalized round robin routing. 
\textit{Oper. Res. Lett.} 12:313--319.

\pagebreak

    \bibitem{20-kon-1} %21
    \Aue{Sano, S., and N.~Miyoshi.} 2000. Applications of \mbox{m-balanced} sequences to some 
network scheduling problems.
 \textit{Discrete Event System: Analysis and Control: 5th 
Workshop on Discrete Event Systems Proceedings.} Kluwer Academic 
Publisher. 317--325.
    \bibitem{21-kon-1} %22
    \Aue{Hyyti$\ddot{\mbox{a}}$, E.} 2013. Lookahead actions in dispatching to parallel 
queues. \textit{Perform. Evaluation} 70(10):859--872.
    \bibitem{22-kon-1} %23
    \Aue{Hyyti$\ddot{\mbox{a}}$, E.} 2013. Optimal routing of fixed size jobs to two parallel 
servers. \textit{INFOR: Inform. Syst. Oper. Res.} 51(4):215--224.
    \bibitem{23-kon-1} %24
    \Aue{Konovalov, M., and R.~Razumchik.} 2015. Iterative algorithm for threshold 
calculation in the problem of routing fixed size jobs to two parallel servers. 
\textit{J.~Telecommun. Inform. Technol.} 3:32--38.
    
\end{thebibliography}

 }
 }

\end{multicols}

\vspace*{-3pt}

\hfill{\small\textit{Received September 19, 2016}}    
    
    \Contr
    
    \noindent
    \textbf{Konovalov Mikhail G.} (b.\ 1950)~---
    Doctor of Science in technology, principal scientist, Institute of Informatics Problems, Federal 
Research Center ``Computer Science and Control'' of Russian Academy of Sciences, 44-2~Vavilov Str., 
Moscow 119333, Russian Federation; \mbox{mkonovalov@ipiran.ru}
    
    \vspace*{3pt}
    
    \noindent
    \textbf{Razumchik Rostislav V.} (b.\ 1984)~--- Candidate of Science (PhD) in physics and 
mathematics, leading scientist, Institute of Informatics Problems, Federal Research Center ``Computer 
Science and Control''' of the Russian Academy of Sciences, 44-2~Vavilov Str., Moscow 119333, Russian 
Federation; assistant professor, Peoples' Friendship University of Russia, 6~Miklukho-Maklaya Str., 
Moscow 117198, Russian Federation; \mbox{rrazumchik@ipiran.ru} 
    
\label{end\stat}


\renewcommand{\bibname}{\protect\rm Литература} 