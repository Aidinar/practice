\def\stat{sinits}

\def\tit{СТОХАСТИЧЕСКИЕ ИНФОРМАЦИОННЫЕ  ТЕХНОЛОГИИ ДЛЯ~ИССЛЕДОВАНИЯ
НЕЛИНЕЙНЫХ КРУГОВЫХ СТОХАСТИЧЕСКИХ СИСТЕМ$^*$}

\def\titkol{Стохастические информационные  технологии для~исследования
нелинейных круговых стохастических систем}

\def\autkol{И.\,Н.~Синицын}
\def\aut{И.\,Н.~Синицын$^1$}

\titel{\tit}{\aut}{\autkol}{\titkol}

{\renewcommand{\thefootnote}{\fnsymbol{footnote}}\footnotetext[1]
{Работа выполнена при финансовой поддержке РФФИ
(проект №\,10-07-00021) и программы ОНИТ РАН <<Информационные
технологии и анализ сложных систем>> (проект 1.5).}}


\renewcommand{\thefootnote}{\arabic{footnote}}
\footnotetext[1]{Институт проблем информатики Российской академии наук, sinitsin@dol.ru}


\Abst{Статья посвящена стохастическим (корреляционным и спект\-раль\-но-кор\-ре\-ля\-ци\-он\-ным) 
информационным технологиям аналитического и статистического анализа и моделирования 
процессов в нелинейных круговых стохастических системах на базе методов круговой 
статистической линеаризации <<намотанным>> нормальным распределением. В~основу технологий 
положены методы, алгоритмы и инструментальное программное обеспечение (ИПО)
CStS-ANALYSIS в среде  MATLAB.}

\KW{аналитическое моделирование; круговой стохастический процесс;
круговая стохастическая система; круговая статистическая линеаризация;
компьютерная поддержка статистических научных исследований; MATLAB;
корреляционные уравнения; спект\-раль\-но-кор\-ре\-ля\-ци\-он\-ные уравнения;
стохастические информационные технологии; статистическое моделирование}

 \vskip 14pt plus 9pt minus 6pt

      \thispagestyle{headings}

      \begin{multicols}{2}
      
            \label{st\stat}


\section{Введение}
Компьютерная поддержка научных исследований (КПНИ) как
 неотъемлемая часть автоматизации научных исследований
 становится все более характерным признаком современных научных
 исследований (НИ) и оказывает сильное влияние на их интенсивность и
 эффективность, превращается в важнейший фактор дальнейшего
 прогресса науки~[1, 2]. Современный этап развития КПНИ характеризуется  интенсивным
проникновением ее в новые сферы исследований и разработок,
расширением контингента пользователей, охватом всех
этапов исследований от сбора и первичной обработки данных,
управления экспериментами до анализа и перспективного планирования
основных на\-прав\-ле\-ний НИ и их информационных
технологий.

Под информационной технологией обычно понимают совокупность
систематических и массовых способов создания, накопления, обработки,
хранения, передачи и распределения информации (данных, знаний) с
помощью средств вычислительной техники и связи.

 На
практике обычно создается ИТ, рассчитанная на выполнение с ее
по\-мощью некоторой основной функции, что связано с необходимостью
решения нескольких типовых задач исследований.
Перечень основных функций довольно ограничен, а с другой стороны,
выполнение этих функций может потребоваться во многих применениях.
Это делает целесообразным выделение функ\-ци\-о\-наль\-но-ориен\-ти\-ро\-ван\-ных,
предметно-ориентированных и проб\-лем\-но-ориен\-ти\-ро\-ван\-ных ИТ~\cite{1-sin}.

 На примере статистических НИ в~\cite{1-sin} 
 рас\-смот\-ре\-ны современные принципы подходы и задачи КПНИ, сформулированы 
 требования к стохастическим ИТ (СтИТ) анализа, моделирования и синтеза 
 оптимальных, субоптимальных и услов\-но-оп\-ти\-маль\-ных фильтров для обработки 
 информации, описано ИПО, а 
 также некоторые приложения. В~качестве основных математических моделей в 
 СтИТ принимались стохастические дифференциальные, интегральные и смешанные 
 уравнения в евклидовом пространстве, а также их разностные аналоги. Для 
 круговых, сферических, кватернионных и других гипергеометрических 
 стохастических уравнений, относящихся к системам на многообразиях~\cite{3-sin}, 
 известные методы анализа, моделирования и синтеза требуют развития. Однако при 
 этом основные принципы, подходы и задачи статистических НИ сохраняются.

Обзор зарубежного универсального методического и программного обеспечения 
для математической статистики  круговых случайных величин и функций дан в~\cite{4-sin}. 
Отдельные прикладные задачи решены, например, в~[1--8].
В~ИПИ РАН начиная с 2010~г.\ в рамках тем, поддерживаемых РФФИ, 
ведутся работы по созданию методического обеспечения для анализа, 
моделирования и синтеза фильтров для обработки информации в круговых 
стохастических системах (КСтС)~\cite{9-sin, 10-sin}.

Рассмотрим полезные для практики простые квазилинейные, 
основанные на эквивалентной круговой статистической линеаризации (КСЛ), 
корреляционные и спектрально-корреляционные методы, алгоритмы и ИПО для 
оф\-лайн-ана\-ли\-за и моделирования круговых стохастических процессов 
(КСтП) в нелинейных КСтС.

\section{Статистическая линеаризация нелинейных преобразований круговых случайных величин}

Пусть сначала $X$ и $Y$~--- скалярные круговые случайные величины (КСВ), 
связанные между собой детерминированной нелинейной зависимостью
    \begin{equation}
    Y=\vrp (X)\,.\label{e2.1s}
    \end{equation}
Согласно принципу эквивалентной статистической линеаризации заменим нелинейную 
зависимость~(\ref{e2.1s}) приближенной линейной зависимостью:
\begin{equation}
\vrp (X) \approx U =\vrp_0 + k_1 (X-\mu)\,,\label{e2.2s}
\end{equation}
где $\mu=\mu_x$~--- круговое среднее направление КСВ~$X$. 
Параметры $\vrp_0$ и~$k_1$ находят из критерия минимума 
безусловного риска для выбранной функции потерь~$\ell (X,U)$:
\begin{equation}
R= \mm \lk\ell (X,U) \rk =\min \,,\label{e2.3s}
\end{equation}
где $\mm$~--- символ математического ожидания.

Если выбрать эквивалентное одномерное распределение (ЭР) КСВ~$X$ и функцию потерь в виде
\begin{equation}
\ell (X,U) =\left( e^{iX} - e^{iU}\right)^2\,,\label{e2.4s}
\end{equation}
то после подстановки~(\ref{e2.2s}) в~(\ref{e2.3s}) и~(\ref{e2.4s}) 
и приравнивания нулю частных производных $\prt R/\prt \vrp_0$ и $\prt R/\prt k_1$ 
получим одно комплексное уравнение для неизвестных параметров $\vrp_0$ и~$k_1$:
\begin{multline}    
\mm_\ap \exp \lf i \vrp (X)\rf ={}\\
{}=\mm_\ap \exp \lf i\lk \vrp_0 + k_1 (X-\mu)\rk\rf\,,\label{e2.5s}
\end{multline}
где $\mm_\ap$~--- символ математического ожидания по ЭР; 
коэффициенты КСЛ $\vrp_0$ и $k_1$ зависят от вероятностных характеристик КСВ~$X$.

Принимая в качестве ЭР для КСВ $X$ намотанное нормальное распределение с параметрами  
$\mu$ и~$\si$, т.\,е.\ $WN(\mu,\si)$~\cite{4-sin, 7-sin}, 
перепишем комплексное уравнение~(\ref{e2.5s}) в виде двух действительных уравнений:
\begin{equation*}
\vrp_0 (\mu,\si) =\psi (\mu,\si)\,; %\label{e2.6s}
\end{equation*}
\begin{equation*}
k_1 (\mu,\si) =\fr{\sqrt{-2\ln r(\mu,\si)}}{\si}\,, %\label{e2.7s}
\end{equation*}
где введено следующее обозначение: 
$$
re^{i\psi} =\mm_{WN} \exp \lf -i \vrp (X)\rf\,.
$$

Для скалярного нелинейного преобразования векторного аргумента
\begin{equation}
Y=\vrp (X_1\tr X_n)\label{e2.8s}
\end{equation}
при условии, что ЭР вектора КСВ  $X= [ X_1, \ldots$\linebreak 
$\ldots , X_n]^{\mathrm{T}}$ является известным 
намотанным нормальным распределением~\cite{4-sin, 7-sin}, 
уравнения принципа статистической линеаризации по критерию~(\ref{e2.4s}) имеют следующий вид:
\begin{equation*}
\vrp (X) \approx U = \vrp_0 +\sss_{h=1}^n k_{1h} X_h^0\,. %\label{e2.9s}
\end{equation*}
Здесь $\vrp_0$~--- первый векторный коэффициент КСЛ, равный
\begin{equation*}
\vrp_0 = \mm_{WN} \vrp(X)\,, %\label{e2.10s}
\end{equation*}
$k_{1h}$ $(h=1\tr n)$~--- второй векторный коэффициент КСЛ, который 
определяется путем решения алгебраической системы уравнений
\begin{equation*}
\sss_{j=1}^n k_{1h} K_{jh} = \mm_{WN} X_j^0 \vrp (X)\,, %\label{e2.11s}
\end{equation*}
где $K_{1h} =\mm_{WN} X_j^0 X_h^0$\ \,$(j,h\hm=1\tr n)$.

Аналогично выписываются формулы для коэффициентов 
КСЛ для векторных и матричных нелинейных преобразований, 
а также посредством канонических представлений~\cite{1-sin}.

Для типовых нелинейных преобразований~(\ref{e2.1s}) и~(\ref{e2.8s}) 
составлены таблицы и разработано ИПО 
CStS-ANALYSIS~\cite{11-sin}.

\section{Основные результаты}

\noindent
\textbf{Теорема 3.1.} \textit{Пусть нестационарная дифференциальная система}
\begin{equation*}
\dot Y =\vrp (Y,t) +V\,,\quad Y(t_0) = Y_0 %\label{e3.1s}
\end{equation*}
\textit{удовлетворяет следующим допущениям:}
\begin{enumerate}[(1)]
\item \textit{$n$-мер\-ный круговой $($на $[0, 2\pi])$  СтП $Y\hm=Y(t)$ 
обладает конечными вероятностными моментами второго порядка;}
\item
\textit{$n$-мерный круговой белый шум, понимаемый в строгом смысле, $(V=\dot W$, 
$W$~--- КСтП с независимыми приращениями на  $[0, 2\pi]$  и матрицей интенсивности $G(t))$;}
\item
\textit{детерминированное нелинейное преобразование  $\vrp (Y,t)$ не обладает  памятью и допускает 
КСЛ согласно алгоритмам разд.~2, причем статистически линеаризованная система для 
$Y^0 \hm= Y-m_y$:
\begin{equation}
{\dot Y}^0 = k_1 Y^0 + V\quad (m_y = \mm Y)\label{e3.2s}
\end{equation}
асимптотически устойчива.
Тогда корреляционное уравнение квазилинейного анализа и аналитического 
моделирования имеют следующий вид:}
\begin{align*}
\dot m_y &=\vrp_0(m_y, K_y,t)\,;\quad m_y (t_0) = m_{y0}\,;\\ %\label{e3.3s}
\dot K_y &= k_1 (m_y, K_y, t) K_y + {}\\
&\hspace*{8mm}{}+  K_y k_1^{\mathrm{T}} (m_y, K_y, t)+ G(t)\,;\\
K_y (t_0) &= K_{y0}\,;\\
\fr{\prt K_y(t_1, t_2)}{\prt t_2} &= 
K_y (t_1, t_2) k_1^{\mathrm{T}} (m_y, K_y, t_2)\,;\\
K_y (t_1, t_2)&= \begin{cases}
K_y(t_1,t_2) &\ \mbox{при\ \ } t_2>t_1\,;\\
K_y (t_2, t_1)^{\mathrm{T}} &\ \mbox{при\ \ } t_2<t_1\,,
\end{cases} 
\end{align*}
\textit{где  $m_y$, $K_y(t)$ и $K_y(t_1, t_2)$~--- соответственно вектор математического ожидания, 
ковариационная матрица  и матрица ковариационных функций КСтП~$Y(t)$}.
\end{enumerate}


\smallskip

\noindent
\textbf{Теорема 3.2.} \textit{В условиях теоремы}~3.1 
\textit{при стационарных функциях $\vrp (Y,t) \hm=\vrp(Y)$, $G(t) \hm=G$ 
корреляционные уравнения анализа и аналитического моделирования для КСтП $\tilde Y(t)$ 
имеют вид:}
\begin{gather}
\vrp_0 (\tilde m_y ,\tilde K_y)=0\,;\notag %label{e3.6s}
\\[3pt]
k_1 (\tilde m_y, \tilde K_y)\tilde K_y + \tilde K_y k_1 (\tilde m_y, \tilde K_y) + G =0\,\label{e3.7s}
\end{gather}

\vspace*{-3pt}

\noindent
\begin{equation}
\left.
\begin{array}{rl}
\fr{d\tilde k_y(\tau)}{d\tau}  &= k_1 (\tilde m_y , \tilde K_y) \tilde k_y(\tau)\,;\\[9pt]
k_y(\tau)&=\tilde{K}_y(t_1,t_1+\tau)\,,
\end{array}
\right\}
\label{e3.8s}
\end{equation}
\textit{где $\tilde m_y$, $\tilde K_y$ и $\tilde k_y(\tau)$~--- соответственно 
математическое ожидание, ковариационная матрица и матрица 
ковариационных функций $(\tau \hm= t_1 \hm- t_2)$ стационарного КСтП $\tilde Y(t)$}.

\smallskip

\noindent
\textbf{Теорема 3.3.} \textit{В~условиях теоремы}~3.2 \textit{уравнения}~(\ref{e3.7s}) 
\textit{и}~(\ref{e3.8s}), \textit{если вместо ковариационной матрицы $k_y(\tau)$ 
использовать спектральную плотность  $s_y(\w)$, допускают следующее спектральное представление:
\begin{align*}
\tilde K_y &=\iin s_y(\w; \tilde m_y, \tilde K_y)\, d\w\,; %\label{e3.9s}
\\
k_y(\tau) &=\iin e^{i\w\tau} s_y (\w; \tilde m_y, \tilde K_y)\, d\w\,, %\label{e3.10s}
\end{align*}
где $s_y (\w; \tilde m_y, \tilde K_y)$~--- матрица спектральных плот\-ностей:
\begin{equation*}
s_y (\w; \tilde m_y, \tilde K_y) = \Phi (i\w; \tilde m_y, \tilde K_y) 
\fr{G}{2\pi} I_n \Phi (i\w; \tilde m_y, \tilde K_y)^*\,; %\label{e3.11s}
\end{equation*}
$\Phi (i\w; \tilde m_y, \tilde K_y)$~--- передаточная функция 
статистически линеаризованной системы}~(\ref{e3.2s}):
 \begin{equation*}
 \Phi (i\w; \tilde m_y, \tilde K_y) = \left[k_1 (\tilde m_y,\tilde  K_y) - Ii\w\right]^{-1}\!;
 \ \ I=I_n\,;
% \label{e3.12s}
 \end{equation*}
\textit{$^*$~--- символ эрмитова сопряжения; $I_n$~--- единичная $(n\times n)$-мат\-рица}.


\medskip

\noindent
\textbf{Замечание.} Рассмотренные в~\cite{1-sin} другие схемы статистической линеаризации 
очевидным образом обобщаются на круговой случай. При этом могут быть использованы различные 
модели КСтС~\cite{9-sin}.

Алгоритмы теорем~3.1--3.3 и их дискретных версий лежат в основе СтИТ
анализа аналитического моделирования. Они реализованы  в ИПО\linebreak
CStS-ANALYSIS в среде  MATLAB~[9--11]. Инструментальное программное
обеспечение имеет возможность
реализовать также и статистическое моделирование КСтС для
следующих <<намотанных>> круговых распределений КСВ: решетчатого,
нормального,  Мизеса, равномерного, пуассонова, кардиоидного,
треугольного, Коши и других устойчивых распределений~\cite{4-sin, 7-sin}.
Точность алгоритмов анализа и аналитического моделирования
проверялась на радиотехнических примерах~\cite{6-sin}, а также методом
статистического моделирования.

\section{Заключение}

Принципы, подходы и задачи статистических научных исследований,
развитые в~\cite{1-sin} для стохастических систем в евклидовом пространстве,
сохраняются и для круговых систем. Методическое и алгоритмическое
обеспечение, основанное на статистической линеаризации для
эквивалентного <<намотанного>> нормального распределения, даются
теоремами~3.1--3.3. Разработано и испытано на ряде тестовых примеров
универсальное ИПО CStS-ANALYSIS
в среде  MATLAB для анализа, аналитического и статистического
моделирования.

{\small\frenchspacing
{%\baselineskip=10.8pt
\addcontentsline{toc}{section}{Литература}
\begin{thebibliography}{99}

\bibitem{1-sin}
\Au{Синицын И.\,Н.}
Канонические представления случайных функций и их применения в 
задачах компьютерной поддержки научных исследований.~--- М.: ТОРУС ПРЕСС, 2009.

\bibitem{2-sin}
\Au{Босов А.\,В., Будзко В.\,И., Захаров~В.\,Н., Козмидиади~В.\,А., 
Корепанов~Э.\,Р., Синицын~И.\,Н., Шоргин~С.\,Я., Ушмаев~О.\,С.}  
Информатика: состояние, проблемы, перспективы~/ Под ред.  И.\,А.~Соколова.~--- М.: ИПИ РАН, 2009.

\bibitem{3-sin}
\Au{Ватанабэ С., Икэда Н.}
 Стохастические дифференциальные уравнения и диффузионные процессы~/ Пер. с англ. 
 под ред.  А.\,Н.~Ширяева.~--- М.: Наука, 1986.

\bibitem{4-sin}
\Au{Rao Jammalamadaka S., Sen Gupta~A.}
  Topics in circular statistics.~--- Singapore: World Scientific, 2001.

\bibitem{5-sin}
\Au{Леви П.}
 Стохастические процессы и броуновское движение~/ Пер. с фр.  под ред. Н.\,Н.~Ченцова.~--- М.: Наука, 1972.

\bibitem{6-sin}
\Au{Тихонов В.\,И., Миронов М.\,А.}
 Марковские процессы.~--- М.: Сов. радио, 1977.

\bibitem{7-sin}
\Au{Мардиа К.} 
Статистический анализ угловых наблюдений~/ Пер. с англ. под ред. Л.\,Н.~Большева.~--- М.: Наука, 1978.
%\columnbreak

\bibitem{8-sin}
\Au{Морозов А.\,Н., Назолин А.\,Л.}
 Динамические системы с флуктуирующим временем.~--- М.: МГТУ им. Н.\,Э.~Баумана, 2001.

 \columnbreak


\bibitem{9-sin}
\Au{Синицын И.\,Н. }
 Канонические разложения случайных функций и их применение в стохастических 
 ин-\linebreak формационных технологиях научных исследований: Курс лекций~// 
 Распознавание образов и анализ изоб\-ра\-же\-ний: новые информационные технологии~--- 
 РОАИ-10-2010: Мат-лы Междунар. конф.~--- СПб., 2010.
 
 \vspace*{6pt}

\bibitem{10-sin}
\Au{Синицын И.\,Н., Корепанов Э.\,Р., Белоусов~В.\,В. и~др.}
Развитие компьютерной поддержки статистических научных исследований сис\-тем 
высокой точности и доступности~// Системы и средства информатики, 2011. Вып.~21. №\,1. С.~3--33.

 \vspace*{6pt}

\label{end\stat}

\bibitem{11-sin}
\Au{Sinitsyn I.\,N., Belousov V.\,V., Konashenkova~T.\,D.}
Software tools for circular stochastic systems analysis~/ 
29th  Seminar (International) on Stability Problems for  Stochastic Models and
5th Workshop ``Applied Problems in Theory of Probabilities and
Mathematical Statistics Related to Modeling of Information Systems'' (APTP\;+\;MS'2011) Book of
Abstracts.~---  M.: IPI RAS, 2011. P.~86--87.
 \end{thebibliography}
}
}


\end{multicols}       