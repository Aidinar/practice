\def\stat{bosov}

\def\tit{ЗАДАЧИ АНАЛИЗА И ОПТИМИЗАЦИИ ДЛЯ~МОДЕЛИ ПОЛЬЗОВАТЕЛЬСКОЙ 
АКТИВНОСТИ.\\ ЧАСТЬ~1. АНАЛИЗ И~ПРОГНОЗИРОВАНИЕ}

\def\titkol{Задачи анализа и оптимизации для~модели пользовательской 
активности. Часть~1. Анализ и~прогнозирование}

\def\autkol{А.\,В.~Босов}
\def\aut{А.\,В.~Босов$^1$}

\titel{\tit}{\aut}{\autkol}{\titkol}

%{\renewcommand{\thefootnote}{\fnsymbol{footnote}}\footnotetext[1]
%{Работа поддержана Российским фондом фундаментальных исследований
%(проекты 11-01-00515а и 11-07-00112а), а также Министерством
%образования и науки РФ в рамках ФЦП <<Научные и
%научно-педагогические кадры инновационной России на 2009--2013~годы>>.}}


\renewcommand{\thefootnote}{\arabic{footnote}}
\footnotetext[1]{Институт проблем информатики Российской академии наук, AVBosov@ipiran.ru}

  
\Abst{Предложена математическая модель описания активности пользователей, 
формирующих запросы в некоторой информационной сис\-те\-ме. Исследованы свойства 
модели, предложены процедуры прогнозирования и идентификации параметров 
уравнений. Представлены результаты чис\-лен\-ных экспериментов, в том числе с 
использованием реальных данных.}

\KW{информационная система; моделирование; эргодический случайный процесс; 
па\-ра\-мет\-ри\-че\-ская идентификация; прогнозирование}

 \vskip 14pt plus 9pt minus 6pt

      \thispagestyle{headings}

      \begin{multicols}{2}
      
            \label{st\stat}

\section{Введение}
  
  Уровень развития телекоммуникационных средств, совершенство информационных 
технологий, доступность и востребованность программного обеспечения в последние 
десятилетия не только привели к повсеместному распространению различных 
информационных сис\-тем, но и оформили их в качестве перспективного объекта применения 
самых разных математических методов. Современные информационные сис\-те\-мы 
объединяют разнотипные аппаратные ресурсы, коммуникационные средства, программные 
сис\-те\-мы, а также пользователей разного уровня. Масштабные характеристики их постоянно 
растут, что является стимулом к совершенствованию используемого для их 
анализа математического аппарата, созданию новых моделей, постановкам новых задач 
оптимизации.
  
  Наиболее популярное сложившееся направление исследований в этой области составляют 
задачи оптимизации процессов функционирования крупных 
  ин\-фор\-ма\-ци\-он\-но-те\-ле\-ком\-му\-ни\-ка\-ци\-он\-ных сис\-тем. Так, весьма развиты 
и широко используются различные методы моделирования и анализа потоков в 
телекоммуникационных сетях, основанные на теории массового обслуживания (см., 
например,~[1, 2]). Другой существенный класс задач, рассматриваемых в рамках данного 
направления, связан с управлением потоками информации, в том числе в Интернете, 
осуществляемым с помощью различных версий протокола сетевого взаимодействия TCP 
(Transfer Control Protocol)~[3, 4]. Постановки последних чаще всего основаны на моделях 
управляемых конечных цепей Маркова (см., например,~[5--7]).
  
  Эти и другие методы позволяют получать существенные результаты, но имеют и 
определенные ограничения применимости. Так, в~[8] отмечается: (1)~неадекватность 
моделей массового обслуживания, возникающая в связи с возрастающей слож\-ностью 
структур и дисциплин обслуживания; (2)~необходимость наличия детального описания 
структуры сети; (3)~недостаточность моделей, использующих в качестве входной 
информацию о наблюдаемых (статистических) показателях функционирования сети. Сюда 
же следует добавить (4)~отсутствие моделей и постановок, учитывающих специфику работы 
программного обеспечения, в частности в связи с влиянием на сис\-те\-му действий 
пользователей.
  
  Последнее обстоятельство представляется не вполне объяснимым, ведь хорошо известны 
многие практические примеры оптимизации функционирования программного обеспечения. 
К ним можно отнести и управление страничным файлом операционной сис\-те\-мы, и 
оптимизацию запросов реляционной сис\-те\-мы управ\-ле\-ни\-ями базами данных~[9], и 
алгоритмы диспетчеризации задач в многопроцессорных средах~[10], и многие другие. 
Однако в целом существенных примеров успешного применения математического аппарата 
к таким задачам немного.
  
  В данной работе объектом исследования является именно процесс функционирования 
про\-грам\-мной сис\-те\-мы, причем в качестве основного\linebreak факто\-ра влияния рассматривается 
пользовательская актив\-ность. Несмотря на достаточно общий характер постановок задач, 
рассматриваются они в связи с функционированием вполне конкретного программного 
обеспечения~--- Информационного веб-пор\-та\-ла~[11], представляющего собой достаточно 
сложное программное средство, реализующее детально описанные алгоритмы обработки 
информации. Это позволяет не просто предложить новые модели и сформулировать для них 
задачи оптимизации, но и проанализировать работоспособность предлагаемых процедур на 
реальных данных.
  
  Основная задача, решаемая далее, связана с моделированием процесса активности 
пользователей, использующих портал: предложена модель процесса в форме динамической 
стохастической сис\-те\-мы специального вида, выяснены ее свойства, разработаны процедуры 
прогнозирования и идентификации параметров модели. В~последующих ра-\linebreak ботах будут 
представлены постановки и решения задач оптимизации расходования вычислительных 
ресурсов, используемых порталом, на основе модели пользовательской активности.

\section{Используемые обозначения}
  
  Далее в работе будут использованы следующие обозначения:
  \begin{itemize}
  \item[$\bullet$] $\overset{\Delta}{=}$~--- равенство по определению;
\item[$\bullet$] \textbf{M}[$x$] и \textbf{M}[$x\vert\Im$]~--- соответственно безусловное 
математическое ожидание случайной величины~$x$ и условное математическое 
ожидание~$x$ относительно $\sigma$-ал\-геб\-ры~$\Im$;
\item[$\bullet$] $\mathbf{P}(\cdot)$ и $\mathbf{P}(\cdot\vert\Im)$~--- соответственно 
безусловная и условная относительно $\sigma$-ал\-геб\-ры~$\Im$ вероятностные меры, 
заданные на подходящем измеримом пространстве;
\item[$\bullet$] $\mathbf{B}(\mathbf{R}^1)$~--- $\sigma$-ал\-геб\-ра борелевских 
множеств на числовой прямой~$\mathbf{R}^1$;
\item[$\bullet$] $x^{\mathrm{T}}$~--- операция транспонирования вектора (мат\-ри\-цы)~$x$;
\item[$\bullet$]  $\mathrm{col}\left(x_1, \ldots , x_n\right)\overset{\Delta}{=}(x_1, \ldots , x_n)^{\mathrm{T}}$~--- 
век\-тор-стол\-бец с элементами $x_1, \ldots , x_n$;
\item[$\bullet$] $\mathrm{row} (x_1, \ldots , x_n)\overset{\Delta}{=}(x_1, \ldots , x_n)$~--- 
век\-тор-стро\-ка с элементами $x_1, \ldots , x_n$;
\item[$\bullet$] $e_k=(0, \ldots , 0,1,0, \ldots 0)^{\mathrm{T}}$~--- единичный вектор в пространстве 
$\mathbf{R}^n$, все координаты которого равны~0, а $k$-я координата рав\-на~1, $1\leq k\leq n$;
\item[$\bullet$]  $\Im_t^y\overset{\Delta}{=}\sigma \{y_\tau, \, \tau\leq t\}$~--- 
$\sigma$-ал\-геб\-ра, порожденная наблюдениями~$y_\tau$, $\tau\leq t$;
\item[$\bullet$] $\psi_x(x,t)$~--- безусловная плотность вероятности случайного 
процесса~$x_t$;
\item[$\bullet$]  $\overline{\psi}_x(x,t)$~--- условная плотность вероятности~$x_t$ 
относительно~$\Im^y_{t-1}$;
\item[$\bullet$] $\overline{\psi}_x(x,t,j)$, $j=0,1,\ldots$~--- условная плот\-ность 
вероятности~$x_{t+j}$ относительно $\sigma$-ал\-геб\-ры~$\Im^y_{t-1}$ (таким образом, 
$\overline{\psi}_x(x,t,0)\overset{\Delta}{=} \overline{\psi}_x(x,t)$);
\item[$\bullet$] $\hat{\psi}_x(x,t)$~--- условная плотность вероятности~$x_t$ 
относительно~$\Im_t^y$;
\item[$\bullet$]  $\mathbf{R}[\mathbf{a};\mathbf{b}]$~--- равномерное распределение на 
отрезке~[$a;b$].
\end{itemize}

\section{Модель активности пользователей портала}
  
  Сделаем вначале два замечания. Во-пер\-вых, об алгоритмах, реализуемых 
Информационным веб-пор\-та\-лом. Во-вто\-рых, о базовых предположениях, используемых 
при формировании модели.
  
  Основная функциональность портала обес\-пе\-чивает выполнение пользовательских 
запросов, формируемых средствами веб-сай\-та, множеством информационных 
  источников~--- <<внешних>> по отношению к порталу информационных сис\-тем, 
подключенных к нему с целью предоставления пользователю унифицированного доступа к 
разнородным данным, формируемым федеративной средой некоторой распределенной 
информационной сис\-те\-мы~[12]. Портал в этой сис\-те\-ме выполняет функции центрального 
веб-узла, единой точки доступа к данным и сервисам. Для целей данной работы достаточно 
учитывать следующую упрощенную цепочку действий, выполняемых программным 
обеспечением портала при обслуживании пользовательского запроса (подробнее 
реализуемые порталом алгоритмы описаны в~[13, 14]).
  
  Для выполнения запроса на поиск данных в информационных источниках, подключенных 
к порталу, пользователь заполняет поисковую форму на сайте портала, формируя тем самым 
пользовательский HTTP-за\-прос. Полученный запрос портальными под\-сис\-те\-ма\-ми вначале 
ассоциируется с набором команд~--- вызовов функций отбора данных одного из 
поддерживаемых порталом типов данных. Затем из каждой команды формируется набор\linebreak 
запросов к нескольким источникам, под\-дер\-жи\-ва\-ющим выбранный пользователем тип 
данных. Запро\-сы к источникам выполняются непосредственно их адаптерами. Чис\-ло 
пользовательских запросов зависит, очевидно, от текущего числа пользователей портала 
(пользовательской активности). Каждому пользовательскому запросу может быть поставлено 
в соответствие любое число команд (это обеспечивается шаблонами сайта портала, 
позволяющими объединять в одном URL запросы к произвольному набору типов данных 
портала). Число источников, участвующих в выполнении команды, также заранее 
неизвестно, так как зависит, с одной стороны, от выбранного типа данных, с другой~--- от 
заданного пользователем состава опрашиваемых источников.
  
  При формировании модели пользовательской активности необходимо учесть, во-пер\-вых, 
что, как и подавляющее большинство веб-сис\-тем, портал использует протокол HTTP, 
взаимодействие по которому не является непрерывным (disconnected) и не имеет состояния 
сеанса (stateless). Это означает, что текущее число пользователей неизвестно и судить о нем 
можно лишь по косвенным данным, например по поступающему числу пользовательских 
запросов, формируемых команд и/или вы\-пол\-ня\-емых запросов к источникам.
  
  Кроме того, в модели нужно учесть и целый набор неконтролируемых факторов, 
формируемых окружением портала, например производительность и доступность 
информационных сис\-тем источников, объемные характеристики под\-дер\-жи\-ва\-емых 
хранилищ, семантику поступающих запросов и проч. Наибольший вклад в неопределенность 
портального окружения, естественно, вносят его пользователи. Их численность 
представляется наиболее важным фактором, влияющим на функционирование портала, так 
как обслуживание пользовательских запросов является его основной задачей. В~конечном 
итоге именно сведения о числе обслуживаемых пользователей должны стать основным 
фактором, используемым при оптимизации работы программного обеспечения портала.
  
  Описывая далее пользовательскую активность, прежде всего постараемся учесть хорошо 
известный факт наличия спонтанных всплесков пользовательской активности (так 
называемые <<slashdot-эф\-фект>> или <<хабраэффект>>, получившие название от 
соответствующих сайтов~[15]). Такие всплески могут быть вызваны, например, появлением 
контента, вызывающего повышенный интерес, повышением деловой активности в связи с 
особыми событиями, наконец, распорядком рабочих/свободных часов. Терминологически 
такие обстоятельства предлагается характеризовать предположением о наличии нескольких 
режимов функционирования портала. В~каждом режиме число пользователей будем 
описывать простой линейной моделью. Скачкообразный характер рассматриваемого 
показателя будем учитывать предложением о смене моделей при изменениях режимов. 
Причем основанием для смены режима будем полагать достижение текущим показателем 
некоторого заданного порогового значения.
  
  Хорошо известны модели числовых рядов, обладающие таким характером,~--- линейные 
модели с порогом TAR, TARMA и другие их раз\-но\-вид\-ности~[16]. Кроме того, этот подход 
уже применялся в задаче оптимизации функционирования портала при взаимодействии с 
подключенными информационными источниками~[17]. Этим же принципом воспользуемся 
и для формирования модели пользовательской активности.
  
  Будем считать, что интервал работы портала $[t_0;+\infty)$ разбит на отрезки (можно 
считать равные) $t_0<t_1<t_2<\ldots < t_{n-1}$, на текущем отрезке $(t_{n-1};t_n]$ 
формируется значение показателя пользовательской активности~$x_{t_n}$. Практически 
значение данного показателя может задавать либо среднее число пользователей, 
обратившихся с запросами к порталу в течение заданного промежутка времени, либо их 
суммарное число. Дискретизация показателя, а также выбор достаточно большой длины 
интервала $(t_{n-1};t_n]$ (2, 5, 10 или даже 30~мин) объясняются необходимостью 
ограничения затрат вычислительных ресурсов, выделенных портальным приложениям, на 
оптимизацию собственной работы. Представляется очевидным, что использование 
непрерывной модели приведет к чрезмерным накладным расходам и будет оказывать 
слишком большое влияние на выполнение прямой функциональности портала.
  
  Для упрощения обозначений далее будем считать, что $t_0=0$, $t_1=1, \ldots$, 
$t_n=t,\ldots$, показатель пользовательской активности описывается процессом~$x_t$, 
$t=0,1,\ldots$
  
  В качестве наблюдений за состоянием~$x_t$ можно воспользоваться представляющимся 
достаточно обоснованным предположением о линейной зависимости между числом 
активных пользователей~$x_t$ и числом команд~$y_t$, выполненных порталом за интервал 
наблюдения:
  \begin{equation}
  y_t=cx_t+\sigma w_t\,,
  \label{e1-b}
  \end{equation}
где параметр $c$  определяет среднее число команд, формируемых одним пользователем за 
интервал наблюдения $(t-1;t]$; $w_t$~--- возмущение (шум), моделирующее отклонения 
числа команд от заданного среднего уровня (далее предполагается, что $\{w_t\}$~--- 
стандартный дискретный белый шум в узком смыс\-ле, сечения которого имеют плот\-ность 
ве\-ро\-ят\-ности $\varphi_w(\cdot)$); $\sigma$~--- характеристика величины разброса этого 
возмущения.

  Для моделирования спонтанных всплесков будем предполагать, что заданы $n>1$ 
характерных режимов пользовательской активности, каждому из которых соответствует 
интервал значений процесса~$x_t$. Иными словами, будем предполагать, что область 
значений~$x_t$ разбита на непересекающиеся интервалы~$\Delta_k$:

\noindent
  \begin{gather*}
  -\infty = a_1<a_2<\ldots < a_n<a_{n+1}=+\infty\,;\\
  \Delta_k =(a_k,a_{k+1}]\,,\enskip k=1,\ldots ,n-1\,, \ \Delta_n=(a_n,+\infty)\,.
  \end{gather*}
  
  Для описания текущего режима пользовательской активности определим индикаторную 
функцию~$\Theta(x)$:
  \begin{equation}
  \left.
  \begin{array}{rl}
  \Theta(x) &= \mathrm{col} \left( I_{\Delta_1}(x), \ldots , I_{\Delta_n}(x)\right)\,;\\[9pt]
  I_{\Delta_k} (x) &=  \begin{cases}
  1\,, & \mbox{\ если \ } x\in \Delta_k\,,\\
  0 \,, & \mbox{\ если \ } x\not\in \Delta_k\,.
  \end{cases}
  \end{array}
  \right\}
  \label{e2-b}
  \end{equation}
  Например, можно предположить, что есть три режима пользовательской активности:
  \begin{enumerate}[(1)]
\item режим слабой активности;
\item режим повседневной активности (описывается средним характерным числом 
пользователей, регулярно использующих портал в рабочее \mbox{время});
\item режим повышенной активности.
\end{enumerate}

  Для каждого режима зададим набор па\-ра\-мет\-ров 
  $a =\mathrm{row} \left( a_1, \ldots ,a_n\right)$, 
$q\hm=\mathrm{row}\left( q_1, \ldots , q_n\right)$, 
$b\hm=\mathrm{row}\left( b_1, \ldots , b_n\right)$, предполагая, что 
эволюция~$x_t$ в $k$-м режиме описывается простейшей линейной сис\-те\-мой~--- 
авторегрессией первого порядка:
  $$
  x_t=a_k x_{t-1} +q_k+b_k v_t\,.
  $$
  
  Если каждая из авторегрессий устойчива, то параметры $a_k$, $q_k$ для $k$-го режима 
определяют некоторое характерное среднее число пользователей, параметр~$b_k$ 
характеризует величину разброса, моделируемого возмущением~$v_t$.
  
  Таким образом, с учетом введенных обозначений параметров и соотношений~(\ref{e2-b}) 
получена следующая модель для описания показателя пользовательской активности:
\begin{multline}
  \!\!x_t=a\Theta(x_{t-1}) x_{t-1} +q\Theta(x_{t-1})+{}\\
  {}+b\Theta (x_{t-1})v_t\,, \enskip
  t=1, 2, \ldots
  \label{e3-b}
  \end{multline}
  
  Отметим, что согласно~(\ref{e3-b}) изменения режимов активности продуцируются 
непосредственно текущим значением показателя~$x_t$: при достижении им границы 
интервала, определенной для текущего режима, характер динамики (номер авторегрессии)\linebreak 
изменяется, и если при этом стационарные средние значения соответствующих устойчивых 
авторегрессий существенно отличаются, то за несколько последующих шагов (переходный 
процесс)\linebreak про\-изойдет и смена характера динамики~$x_t$, и резкая смена текущего числа 
пользователей.
  
\section{Свойства показателя пользовательской активности}
  
  В данном разделе исследованы свойства случайного процесса~$x_t$, описываемого 
уравнениями~(\ref{e2-b}) и~(\ref{e3-b}). Уточняя модель~(\ref{e3-b}), будем предполагать, 
что $\{v_t\}$~--- стандартный дискретный белый шум в узком смысле, сечения которого 
имеют плотность вероятности $\varphi_v(\cdot)$, $x_0$~--- случайная величина, не зависящая от 
$\{v_t\}$ и имеющая плотность вероятности~$\psi_0(\cdot)$.
  
  Обозначим переходную вероятность для процесса~$x_t$: 
$\mathbf{P}(x,t,B)\overset{\Delta}{=}\mathbf{P}(x_t\in B\vert x_0=x)$, $x\in \mathbf{R}^1$, 
$B\in \mathbf{B}(\mathbf{R}^1)$.
  
  \smallskip
  
  \noindent
  \textbf{Теорема 1.} \textit{Пусть выполнены следующие условия:}
  \begin{enumerate}[(1)]
  \item $\varphi_v(x) >0$, $\psi_0(x)>0$\ \ $\forall\  x \in \mathbf{R}^1$;
\item $\varphi_v(\cdot)$, $\psi_0(\cdot)$ \textit{непрерывны};
\item $\vert a_1\vert <1$, $\vert a_n\vert <1$;
\item $b_k>0$, $k=1, \ldots , n$.
\end{enumerate}
  \textit{Тогда процесс $x_t$ является эргодическим, т.\,е.\ существует единственное 
вероятностное распределение $\pi^*(\cdot)$ на $(\mathbf{R}^1, \mathbf{B}(\mathbf{R}^1))$}:
  \begin{equation} 
  \sup\limits_{\mathbf{B}(\mathbf{R}^1)}\left\vert \mathbf{P}(x,t,B)-\pi^*(B)\right\vert 
\rightarrow 0\ \mbox{при}\ t\rightarrow\infty\,,
  \label{e4-b}
  \end{equation}
\textit{где сходимость имеет место $\forall\ x\in \mathbf{R}^1$ и $\forall\ \psi_0(\cdot)$. При 
этом предельная мера $\pi^*(\cdot)$ абсолютно непрерывна относительно меры Лебега}.

\smallskip

\noindent
  Д\,о\,к\,а\,з\,а\,т\,е\,л\,ь\,с\,т\,в\,о\,.\ Заметим вначале, что в условиях теоремы 
маргинальная плотность вероятности $\psi_x(x,t)$ существует и описывается следующими 
рекуррентными соотношениями:
  \begin{equation}
  \left.
  \begin{array}{l}
  \psi_x(x,t)=\displaystyle\int\limits_{\mathbf{R}^1}\psi_x(y,t-1)\rho(x,y,t)\,dy={}\\
\!\!\!\!\!\!  {}=\displaystyle\sum\limits_{k=1}^n \int\limits_{\Delta_k} \!\psi_x(y,t-1)\fr{1}{b_k}\,\varphi_v\left( \fr{x-a_ky-
q_k}{b_k}\right)\,dy\,;\\[12pt]
  \psi_x(x,0)=\psi_0(x)\,,
  \end{array}\!
  \right\}\!\!\!
  \label{e5-b}
  \end{equation}
где переходная плотность $x_t$ за один шаг
$$
\rho(x,y,t) =\sum\limits_{k=1}^n I_{\Delta_k}(y)\fr{1}{b_k} \varphi_v \left( \fr{x-a_k y-
q_k}{b_k}\right)\,.
$$ 

\begin{figure*}[b] %fig1
\vspace*{1pt}
\begin{minipage}[t]{80mm}
\begin{center}
\mbox{%
\epsfxsize=78.407mm
\epsfbox{bos-1.eps}
}
\end{center}
\vspace*{-9pt}
\Caption{Выборочные функции
\label{f1-b}}
%\end{figure*}
\end{minipage}
\hfill
%\begin{figure*} %fig2
\vspace*{1pt}
\begin{minipage}[t]{80mm}
\begin{center}
\mbox{%
\epsfxsize=78.848mm
\epsfbox{bos-2.eps}
}
\end{center}
\vspace*{-9pt}
\Caption{Предельная плотность
\label{f2-b}}
\end{minipage}
\end{figure*}
  
  Обозначим $p_y(B)\overset{\Delta}{=} \mathbf{P}(y,1,B)$. Тогда
  \begin{multline*}
  p_y(B) =\int\limits_B \rho(x,y,1)\,dx={}\\
  {}=\int\limits_B \fr{1}{b_k}\,\varphi_v \left( \fr{x-a_ky-
q_k}{b_k}\right)\,dx
  \end{multline*} 
  для ${k:}\, y\in \Delta_k$. Отсюда и из неотри\-ца\-тель\-ности плотностей $\varphi_v(\cdot)$, 
$\psi_0(\cdot)$ следует, что $p_y(B)>0$ $\forall\ y\in \mathbf{R}^1$ и для любого 
борелевского~$B$, име\-юще\-го ненулевую меру Лебега. Отсюда следует, что у цепи~$x_t$ нет 
циклов и она неразложима.
  
  Сходимость~(\ref{e4-b}) обеспечивается условиями~(3) и~(4) теоремы~1 (см.\ замечание~3.1.2 к 
теореме~3.1 в~\cite{18-b}).
  
  Заметим теперь, что из~(\ref{e5-b}) с учетом не\-пре\-рыв\-ности $\varphi_v(\cdot)$, 
$\psi_0(\cdot)$ вытекает непрерывность $\psi_x(x,t)\ \forall\ t$, т.\,е.\ меры $\mathbf{P}(x,t,B)$ 
абсолютно непрерывны и имеют непрерывные плотности $\forall\ x\in \mathbf{R}^1$. Таким 
образом, сходимость~(\ref{e4-b}) гарантирует абсолютную непрерывность предельной меры 
$\pi^*(\cdot)$ и непрерывность предельной плотности~$\psi^*(x)$. Тео\-ре\-ма 
до\-ка\-зана.

  \smallskip
  
  \noindent
  \textbf{Следствие}. \textit{Если предельная плотность вероятности определяет 
распределение начального условия~$x_0$, т.\,е.\ $\psi_0(x)=\psi^*(x)$, то процесс~$x_t$ 
является стационарным в узком смысле. Если при этом $\mathbf{M}[v_t^2]<\infty$, то 
процесс~$x_t$ является стационарным в широком смысле}.
  
  \smallskip
  
  \noindent
  \textbf{Пример~1.} Вычислительные эксперименты для предложенной модели 
пользовательской актив\-ности проводились как на модельных данных, позволяющих в 
полном объеме протестировать предлагаемые далее оптимизационные процедуры, так и на 
реальных данных, позволяющих сделать некоторые выводы о возможностях практического 
применения. В~статье приведены два варианта расчетов. Вначале рассмотрим модельный 
вариант определения параметров. Воспользуемся предположением разд.~3 о трех уровнях 
пользовательской активности и зададим следующие три интервала: $\Delta_1=(-\infty;3]$; 
$\Delta_2=(3;7]$; $\Delta_3=(7;+\infty)$. Параметры уравнения~(\ref{e3-b}) и некоторые 
числовые характеристики процесса~$x_t$ приведены в табл.~1.
  
  Будем предполагать, что распределения, определяемые плотностями вероятности 
возмущений\linebreak\vspace*{-12pt}
\columnbreak

\noindent
\begin{center}
\noindent
{\tablename~1}\ \ \small{Параметры модели}
\end{center}
%\vspace*{2ex}

\begin{center}
%\tabcolsep=9pt
   
 {\small  \begin{tabular}{|c|c|c|c|p{1pt}p{1pt}|c|c|c|c|}
   \hline
$a_1$&$a_2$&$a_3$&$q_1$&\multicolumn{2}{c|}{$q_2$}&$q_3$&$b_1$&$b_2$&$b_3$\\
&&&&\hspace*{1pt}&\hspace*{1pt}
&&&&\\[-12pt]
\hline
0,3&0,4&0,7&1,4&\multicolumn{2}{c|}{3{,}0}&3,0&0,9&1,5&2,5\\
\hline
\multicolumn{5}{|c|}{$M[x_t]$ }&\multicolumn{5}{c|}{$D[x_t]$ }\\
\hline
\multicolumn{5}{|c|}{5,61}&\multicolumn{5}{c|}{19,77}\\
\hline
\multicolumn{3}{|c|}{\ } & \multicolumn{4}{c|}{\ } & \multicolumn{3}{c|}{\ }\\[-9pt]
\multicolumn{3}{|c|}{$M[\Theta^{\mathrm{T}}(x_t)e_1]$} &\multicolumn{4}{c|}{$M[\Theta^{\mathrm{T}}(x_t) e_2]$} 
&\multicolumn{3}{c|}{$M[\Theta^{\mathrm{T}}(x_t)e_3]$}\\
\hline
\multicolumn{3}{|c|}{0{,}3328}&\multicolumn{4}{c|}{0{,}3746}&\multicolumn{3}{c|}{0{,}2926}\\
\hline
\end{tabular}
}
\end{center}
\vspace*{12pt}

%\bigskip
\addtocounter{table}{1}
  

\noindent
 $\varphi_v(\cdot)$ и $\varphi_w(\cdot)$, являются стандартными гауссовскими, 
начальное условие~$x_0$ также предполагалось гауссовским со средним $M[x_t]$ и 
дисперсией $D[x_t]$, указанными в табл.~1, т.\,е.\ равными со\-от\-вет\-ст\-ву\-ющим моментам 
предельного распределения.


  Примеры траекторий $x_t$ с указанными па\-ра\-мет\-ра\-ми приведены на рис.~\ref{f1-b}, 
оценка (гистограмма) предельной плотности $\psi^*(x)$~--- на рис.~\ref{f2-b}. В~целом 
можно констатировать, что траектории иллюстрируют ожидаемое поведение, а вид 
предельной плотности вполне согласуется с переключающимся характером процесса.


  Второй вариант выполненных расчетов опирается на реальные данные~--- наблюдения за 
активностью пользователей портала РАН {\sf www.ras.ru}. Накоплена значительная 
статистика, включающая данные о числе команд, выполненных порталом на 10-ми\-нут\-ных
интервалах. На рис.~\ref{f3-b} представлены произвольно выбранные 5000~наблюдений. 
Отметим, что, вообще говоря, эти данные характеризуют пользовательскую активность лишь 
косвенно. Высказанные в разд.~3 предположения о модели наблюдений означают, что на 
рис.~\ref{f3-b} представлены реализации процесса~$y_t$ из~(\ref{e3-b}). Однако линейная 
зависимость в~(\ref{e3-b}) позволяет предполагать (по крайней мере, для целей первичного 
анализа), что динамический характер обоих процессов сходен.

\begin{figure*} %fig3
\vspace*{1pt}
\begin{minipage}[t]{80mm}
\begin{center}
\mbox{%
\epsfxsize=79mm
\epsfbox{bos-3.eps}
}
\end{center}
\vspace*{-11pt}
\Caption{Данные сайта {\sf www.ras.ru}
\label{f3-b}}
%\end{figure*}
\end{minipage}
\hfill
%\begin{figure*} %fig4
\vspace*{1pt}
\begin{minipage}[t]{80mm}
\begin{center}
\mbox{%
\epsfxsize=78.808mm
\epsfbox{bos-4.eps}
}
\end{center}
\vspace*{-11pt}
\Caption{Траектория с исключенным трендом
\label{f4-b}}
\end{minipage}
\vspace*{-6pt}
\end{figure*}

\begin{table*}[b]\small
\vspace*{-18pt}
\begin{center}
\Caption{Параметры модели}
\vspace*{2ex}


%\tabcolsep=9pt
\begin{tabular}{|c|c|c|c|p{1pt}p{1pt}|c|c|c|c|}
\hline
$a_1$&$a_2$&$a_3$&$q_1$&\multicolumn{2}{c|}{$q_2$}&$q_3$&$b_1$&$b_2$&$b_3$\\
&&&&\hspace*{1pt}&\hspace*{1pt}
&&&&\\[-12pt]
  \hline
0,361&0,353&0,725&$-67{,}78$&\multicolumn{2}{c|}{$-26{,}26$}&39,52&36,43&31,41&87,04\\
\hline
\multicolumn{5}{|c|}{$M[x_t]$ }&\multicolumn{5}{c|}{$D[x_t]$ }\\
\hline
\multicolumn{5}{|c|}{25{,}49}&\multicolumn{5}{c|}{10672{,}87}\\
\hline
\multicolumn{3}{|c|}{\ } & \multicolumn{4}{c|}{\ } & \multicolumn{3}{c|}{\ }\\[-9pt]
\multicolumn{3}{|c|}{$M[\Theta^{\mathrm{T}}(x_t)e_1]$} &\multicolumn{4}{c|}{$M[\Theta^{\mathrm{T}}(x_t) e_2]$} 
&\multicolumn{3}{c|}{$M[\Theta^{\mathrm{T}}(x_t)e_3]$}\\
\hline
\multicolumn{3}{|c|}{0{,}0534}&\multicolumn{4}{c|}{0{,}5668}&\multicolumn{3}{c|}
{0{,}3798}\\
\hline
\end{tabular}
\end{center}
\renewcommand{\figurename}{\protect\bf Таблица}
\renewcommand{\tablename}{\protect\bf Рис.}
\setcounter{table}{4}
%\begin{figure*} %fig5
\vspace*{9pt}
\begin{minipage}[t]{80mm}
\begin{center}
\mbox{%
\epsfxsize=78.884mm
\epsfbox{bos-5.eps}
}
\end{center}
\vspace*{-11pt}
\Caption{Выборочные функции
\label{f5-b}}
%\end{figure*}
\end{minipage}
\hfill
%\begin{figure*} %fig6
\vspace*{1pt}
\begin{minipage}[t]{80mm}
\begin{center}
\mbox{%
\epsfxsize=78.792mm
\epsfbox{bos-6.eps}
}
\end{center}
\vspace*{-11pt}
\Caption{Предельная плотность
\label{f6-b}}
\end{minipage}
%\end{figure*}
\end{table*}

\renewcommand{\figurename}{\protect\bf Рис.}
\renewcommand{\tablename}{\protect\bf Таблица}
\setcounter{table}{2}
\setcounter{figure}{6}

  
  Нетрудно видеть, что в имеющихся данных присутствует регулярный периодический 
тренд (линия тренда нанесена на рисунке темным цветом). Периодичность, как выяснилось, 
связана с естественными недельными колебаниями активности. Этот детерминированный 
компонент нетрудно было выделить. С~этой целью имеющаяся выборка была разделена на 
недельные интервалы, тренд сформирован из медианных оценок, вычисленных в каж\-дый 
момент времени (использование медианы в качестве оценки объясняется наличием выбросов 
в анализируемых данных). После этого из имеющихся наблюдений тренд был исключен (что 
объясняет, в частности, наличие отрицательных значений у процесса~$x_t$ далее). 
В~результате сформировался новый набор данных из приращений, фрагмент которых 
приведен на рис.~\ref{f4-b}. Эти данные позволили сформировать экспертные 
предположения о значениях параметров модели и некоторых ее числовых характеристиках 
(подробно использованные экспертом предложения описаны в~\cite{19-b}). Как и в 
предыдущем примере, определены три интервала пользовательской активности: $\Delta_1\hm=(-
\infty;-100]$; $\Delta_2\hm=(-100;10]$; $\Delta_3\hm=(10;+\infty)$; параметры и характеристики $x_t$ 
приведены в табл.~2.


  Примеры траекторий $x_t$ приведены на рис.~\ref{f5-b}, гистограмма предельной 
плотности $\psi^*(x)$ ~--- на рис.~\ref{f6-b}. Заметим, что по сравнению с предыдущим 
примером, в котором вклад всех трех режимов в предельной плотности ярко выражен, в 
данном случае первый режим (слабая активность) на форме распределения~$\psi^*(x)$ 
сказывается незначительно, что связано, очевидно, с незначительным (около 5\%) временем 
пребывания процесса в этом режиме.

\section{Прогнозирование}
  
  Стохастическую систему наблюдения~(\ref{e1-b})--(\ref{e3-b}) в дальнейшем 
предлагается использовать в постановках различных задач оптимизации распределением 
ресурсов, используемых программной сис\-те\-мой (порталом) для обслуживания запросов 
пользователей. Для целей данной работы достаточно заметить, что для широкого класса 
возможных оптимизационных постановок выработка управляющего воздействия в конечном 
итоге будет основываться на наилучшем в среднем квадратическом прогнозе чис\-ла активных 
пользователей. Поэтому в качестве первого шага в решении задач оптимизации 
распределения ресурсов следует рассматривать задачу прогнозирования. Поскольку в 
дальнейшем и критерии оптимизации будут квадратичными, то для существования 
необходимых моментных характеристик всюду далее будем предполагать, что выполнены 
условия $\mathbf{M}[v_t^2]\hm<\infty$, $\mathbf{M}[w_t^2]\hm<\infty$, 
$\mathbf{M}[x_0^2]\hm<\infty$.
  
  В случае портала цель прогнозирования можно уточнить: требуется иметь возможность 
прогнозировать число поступающих для выполнения команд. Таким образом, требуется по 
наблюдениям~$y_\tau$, $\tau\hm\leq t-1$, сформировать оценки $\overline{y}_{t+j,t-1}$ 
будущих наблюдений~$y_{t+j}$, $j=0,1,\ldots$
  
  \smallskip
  
  \noindent
  \textbf{Теорема~2.} \textit{Пусть для сис\-те\-мы наблюдения}~(\ref{e1-b})--(\ref{e3-b}) 
$\min\limits_{1\leq k\leq n} b_k>0$, $\sigma>0$. \textit{Тогда оптимальный в среднем 
квадратическом прогноз $\overline{y}_{t+j,t-1}$ выхода~$y_{t+j}$ имеет вид:}

\noindent
  \begin{equation}
\hspace*{-3mm}  \left.
  \begin{array}{rl}
  \overline{y}_{t+j,t-1}&=\displaystyle \sum\limits_{k=1}^n \int\limits_{\Delta_k} c(a_k \xi 
+q_k)\overline{\psi}_x(\xi, t-1, j)\,d\xi\,,\\[9pt]
&\hspace*{25mm}j=1,2, \ldots\,;\\[9pt]
  \overline{y}_{t,t-1} &= \displaystyle \sum\limits_{k=1}^n \int\limits_{\Delta_k} c(a_k\xi 
+q_k)\hat{\psi}_x (\xi,t-1)\,d\xi\,,
  \end{array}\!
  \right\}
  \label{e6-b}
  \end{equation}
\textit{где прогнозирующие плотности вероятности определяются соотношениями:}

\noindent
\begin{equation}
\left.
\begin{array}{rl}
\overline{\psi}_x(x,t,j)  &= \displaystyle\sum\limits_{k=1}^n \fr{1}{b_k}\int\limits_{\Delta_k} 
\overline{\psi}_x (\xi,t,j-1)\times{}\\[9pt]
&\hspace*{-8mm}{}\times \varphi_v \left( \fr{x-a_k\xi-q_k}{b_k}\right)\,d\xi\,;\ 
j=1,2, \ldots \,;\\[9pt]
\overline{\psi}_x(x,t,0) &\overset{\Delta}{=} \overline{\psi}_x(x,t)={}\\[9pt]
&\hspace*{-20mm}{}= \displaystyle \sum\limits_{k=1}^n \fr{1}{b_k} 
\int\limits_{\Delta_k} \hat{\psi}_x(\xi,t-1)\varphi_v \left( \fr{x-a_k\xi-q_k}{b_k}\right) \,d\xi\,.
\end{array}\!\!
\right\}
\label{e7-b}
\end{equation}
  
  \noindent
  \textbf{Замечание.} Соотношения~(\ref{e7-b}) требуется дополнить выражением для 
$\hat{\psi}_x(x,t)$~--- условной плотности вероятности~$x_t$ относительно $\Im_t^y$ 
(решение соответствующей задачи фильтрации подробно рассмотрено в~\cite{13-b, 14-b}):

\end{multicols}

\hrule

\vspace*{3pt}

  \begin{equation*}
  \hat{\psi}_x(x,t) = 
  \fr{\varphi_w \left( (y_t-cx)/\sigma\right) \sum\limits_{k=1}^n b_k^{-1} \int\limits_{\Delta_k}
\hat{\psi}_x (\xi,t-1)\varphi_v\left( (x-a_k\xi-q_k)/b_k\right)\,d\xi}
{\sum\limits_{k=1}^n b_k^{-1} 
\int\limits_{\mathbf{R}^1} \varphi_w \left( (y_t-cx)/\sigma\right) 
  \int\limits_{\Delta_k} \hat{\psi}_x 
  (\xi,t-1)\varphi_v\left( (x-a_k\xi-q_k)/b_k\right)\,d\xi dx}\,.
  \end{equation*}
  
  \smallskip
  
  \noindent
  Д\,о\,к\,а\,з\,а\,т\,е\,л\,ь\,с\,т\,в\,о\,.\ Для условной функции распределения вектора $z_t\hm 
=\mathrm{col}\left( x_t,y_t\right)$ относительно~$\Im_{t-1}^y$ имеем:

\noindent
  \begin{multline*}
  \mathbf{P}\left(x_t\leq x,\,y_t\leq y\vert \Im_{t-1}^y\right)= \mathbf{P}\left(
  \begin{array}{c}
  a\Theta(x_{t-1})x_{t-1}+q\Theta(x_{t-1})+b\Theta(x_{t-1})v_t\leq x\,,\\[9pt]
\hspace*{-10mm}  cx_t+\sigma w_t\leq y
  \end{array}
\Bigg  \vert
   \Im_{t-1}^y
  \right)={}\\
    {}=\mathbf{P}\left( v_t\leq \fr{x-a\Theta(x_{t-1})-q\Theta(x_{t-1})}{b\Theta(x_{t-1})}\,,\, 
w_t\leq \fr{y-cx_t}{\sigma}\left\vert\vphantom{\fr{y-cx_t}{\sigma}} \right.
\Im_{t-1}^y\right)={}\\
{}= \mathbf{P}\left( 
\begin{array}{c}
v_t\leq \displaystyle\sum\limits_{k=1}^n \fr{x-a_kx_{t-1}-q_k}{b_k}\,I_{\Delta_k}(x_{t-
1})\,,\\[9pt]
w_t\leq \fr{y-c(a\Theta(x_{t-1})x_{t-1}+q\Theta(x_{t-1})+b\Theta(x_{t-1}) v_t)}{\sigma}
\end{array}
\left\vert\vphantom{\begin{array}{c}
v_t\leq \displaystyle\sum\limits_{k=1}^n \fr{x-a_kx_{t-1}-q_k}{b_k}\,I_{\Delta_k}(x_{t-
1})\,,\\[9pt]
w_t\leq \fr{y-c(a\Theta(x_{t-1})x_{t-1}+q\Theta(x_{t-1})+b\Theta(x_{t-1}) v_t)}{\sigma}
\end{array}}\right.
 \Im^y_{t-1}\right)={}
  \end{multline*}
  
  \noindent
  \begin{multline*}
{}=
\sum\limits_{k=1}^n \mathbf{P}\left( 
\begin{array}{c}
v_t\leq \displaystyle \sum\limits_{k=1}^n \fr{x-a_kx_{t-1}-q_k}{b_k}\,,\ x_{t-1}\in 
\Delta_k\,,\\[9pt]
w_t\leq \fr{y-c(a_kx_{t-1}+q_k+b_k v_t)}{\sigma}
\end{array} 
\left\vert \vphantom{\begin{array}{c}
v_t\leq \displaystyle \sum\limits_{k=1}^n \fr{x-a_kx_{t-1}-q_k}{b_k}\,,\ x_{t-1}\in 
\Delta_k\,,\\[9pt]
w_t\leq \fr{y-c(a_kx_{t-1}+q_k+b_k v_t)}{\sigma}
\end{array} }\right.
 \Im^y_{t-1}\right)={}\\[12pt]
{}=\sum\limits_{k=1}^n \int\limits_{\Delta_k} \hat{\psi}_x(\xi, t-1) \int\limits_{-\infty}^{(x-
a_k\xi-q_k)/b_k} \varphi_v(v) \int\limits_{-\infty}^{(y-
c(a_k\xi+q_k+b_kv))/\sigma}\varphi_w(w)\,dw\, dv\, d\xi\,,
\end{multline*}

  \vspace*{9pt}

\hrule

  \vspace*{3pt}

\begin{multicols}{2}

%\vspace*{-3pt}

\noindent
где учтены предположение о существовании условной плотности вероятности 
$\hat{\psi}_x(x,t-1)$ и независимость величин~$v_t$ и $w_t$ друг от друга и от 
\mbox{$\sigma$-алгеб}\-ры~$\Im^y_{t-1}$. Последнее соотношение следует непосредст\-вен\-но из 
существования и свойств регулярных условных распределений для случайных 
последовательностей~\cite{20-b}. Дифференцируя полученное выражение с переменными 
верхними пределами по~$x$ и по~$y$, получаем условную плотность ве\-ро\-ят\-ности 
$\overline{\psi}_x(z,t)$, $z=\mathrm{col}\left( x,y\right)$:

\noindent
\begin{multline*}
\overline{\psi}_z(z,t) ={}\\
{}=\sum\limits_{k=1}^n \fr{1}{\sigma b_k} \int\limits_{\Delta_k} 
\hat{\psi}_x(\xi,t-1)\varphi_v\left( \fr{x-a_k\xi-q_k}{b_k}\right)\times{}\\
\times \varphi_w\left(\! \fr{1}{\sigma}\left(\! y-c\left(\! a_k\xi+q_k+b_k\fr{x-a_k\xi-
q_k}{b_k}\right)\!\right)\!\right)\!d\xi.
\end{multline*}

  Далее очевидные упрощения дают:
  
  \noindent
  \begin{multline*}
  \overline{\psi}_z(z,t) =\fr{1}{\sigma}\,\varphi_w \left( \fr{y-cx}{\sigma}\right) \times{}\\
  {}\times
\sum\limits_{k=1}^n \fr{1}{b_k} \int\limits_{\Delta_k} \hat{\psi}_x(\xi,t-1)\varphi_v \left( \fr{x-
a_k\xi-q_k}{b_k}\right)\,d\xi\,,
  \end{multline*}
откуда путем интегрирования по~$x$ получается маргинальная условная 
плотность~$\overline{\psi}_y(y,t)$:

\noindent
\begin{multline*}
\overline{\psi}_y (y,t) =\fr{1}{\sigma}\sum\limits_{k=1}^n \fr{1}{b_k} 
\int\limits_{\mathbf{R}^1} \varphi_w\left( \fr{y-cx}{\sigma}\right) \times{}\\
{}\times \int\limits_{\Delta_k} 
\hat{\psi}_x(\xi,t-1)\varphi_v\left( \fr{x-a_k\xi-q_k}{b_k}\right) \,d\xi\, dx\,.
\end{multline*}

 
  Отсюда получаем выражение для прогноза $\overline{y}_{t,t-1}$ выхода~$y_t$ по 
наблюдениям~$y_\tau$, $\tau\leq t-1$ (условное математическое ожидание~$y_t$ 
относительно~$\Im^y_{t-1}$~\cite{21-b}):

\noindent
  \begin{multline*}
  \overline{y}_{t,t-1}=\int\limits_{\mathbf{R}^1} y\overline{\psi}_y(y,t)\,dy= {}\\
  {}=
\sum\limits_{k=1}^n \int\limits_{\Delta_k} \hat{\psi}_x(\xi,t-1)\left( \fr{1}{b_k} 
\int\limits_{\mathbf{R}^1}\varphi_v \left( \fr{x-a_k\xi-q_k}{b_k} \right)\times{}\right.\\
{}\left.\times
 \left( \fr{1}{\sigma} 
\int\limits_{\mathbf{R}^1} y \varphi_w\left( \fr{y-cx}{\sigma} \right) dy \right) dx \right) d\xi\,,
  \end{multline*}
откуда после непосредственного вычисления внут\-рен\-них интегралов получается требуемое 
выражение~(\ref{e6-b}) для $\overline{y}_{t,t-1}$. 
Рекуррентные соотношения для $\hat{\psi}_x(x,t)$ получаем по формуле Байеса для 
плотностей вероятности~\cite{21-b} как отношение $\overline{\psi}_z(z,t)$ и 
$\overline{\psi}_y(y,t)$ при $y=y_t$.
  
  Далее заметим, что уравнение~(\ref{e7-b}) для плот\-ности вероятности 
$\overline{\psi}_x(x,t)\hm = \overline{\psi}_x(x,t,0)$ одношагового прогноза~$x_t$ 
относительно $\Im^y_{t-1}$ получается из выражения для $\overline{\psi}_z(z,t)$ путем 
интегрирования по~$y$:

\noindent
  \begin{multline*}
  \overline{\psi}_x(x,t) =\sum\limits_{k=1}^n \fr{1}{b_k} \int\limits_{\Delta_k} 
\hat{\psi}_x(\xi,t-1)\times{}\\
{}\times \varphi_v\left ( \fr{x-a_k\xi-q_k}{b_k}\right) \left( \fr{1}{\sigma} 
\int\limits_{\mathbf{R}^1} \varphi_w \left( \fr{y-cx}{\sigma}\right)dy\right)d\xi,
  \end{multline*}
где внутренний интеграл по~$y$, очевидно, равен~1.

  Далее для вычисления прогноза~$\overline{y}_{t+j,t-1}$ запишем:
  
  \end{multicols}
  
  \vspace*{3pt}
  
  \hrule
  
    \vspace*{6pt}
  
  \begin{equation*}
\mathbf{P}\left(x_{t+j}\leq x\,,\, y_{t+j}\leq y\vert \Im^y_{t-1}\right)
= \mathbf{P}\left( 
  \begin{array}{c}
  a\Theta(x_{t+j-1})x_{t+j-1}+q\Theta(x_{t+j-1})+b\Theta(x_{t+j-1})v_{t+j}\leq x\,,\\[9pt]
  cx_{t+j} +\sigma w_{t+j} \leq y
  \end{array}
  \Bigg\vert 
   \Im^y_{t-1}
  \right)
  \end{equation*}
  
  
  \begin{figure}[b] %fig7
\vspace*{1pt}
\begin{center}
\mbox{%
\epsfxsize=163.656mm
\epsfbox{bos-7.eps}
}
\end{center}
\vspace*{-6pt}
\Caption{Точность прогноза: \textit{1}~--- $\hat{D}[\overline{y}_{t+j,t-1}-y_{t+j}]$;
\textit{2}~--- $\hat{D}[\tilde{y}_{t+j,t-1}-y_{t+j}]$;
\textit{3}~--- $\hat{D}[y(t)]$, где $j = 1$~(\textit{а}); 2~(\textit{б});
3~(\textit{в}); 4~(\textit{г}); 5~(\textit{д}); $j=6$~(\textit{е})
\label{f7-b}}
\vspace*{12pt}
\end{figure} 
    \begin{multicols}{2}
    
    \noindent
и выполним с последним выражением преоб\-ра\-зо\-вания, полностью аналогичные 
проделанным выше
 при выводе уравнения для плотности вероятности 
$\overline{\psi}_z(z,t)$. 



В~результате после дифференцирования получим следующее 
выражение для про\-гно\-зи\-ру\-ющей плотности вероятности $\overline{\psi}_z(z,t,j)$ 
вектора~$z_{t+j}$, $j\hm=1, 2, \ldots$, относительно~$\Im^y_{t-1}$:

\noindent
\begin{multline*}
\overline{\psi}_z(z,t,j)=\fr{1}{\sigma}\varphi_w
\left( \fr{y-cx}{\sigma}\right)\times{}\\
{}\times \sum\limits_{k=1}^n \fr{1}{b_k} 
\int\limits_{\Delta_k} \overline{\psi}_k(\xi,t,j-1)\varphi_v \left( 
\fr{x-a_k\xi-q_k}{b_k}\right)\,d\xi
\end{multline*}
в предположении, что уравнение~(\ref{e7-b}) имеет место для $\overline{\psi}_x(x,t,j-1)$. 
Уравнение для плотности вероятности~$\overline{\psi}_x(x,t,j)$ состояния~$x_{t+j}$ 
относительно~$\Im^y_{t-1}$ получается отсюда путем интегрирования по~$y$, выражение 
для прогнозирующей плотности вероятности $\overline{\psi}_y(y,t,j)$ выхода~$y_{t+j}$ 
относительно~$\Im^y_{t-1}$~--- путем интегрирования по~$x$, а сам 
прогноз~$\overline{y}_{t+j,t-1}$ представляется в виде:
\begin{multline*}
\overline{y}_{t+j,t-1}= \int\limits_{\mathbf{R}^1} y\overline{\psi}_y(y,t,j)\,dy={}\\
{}=\sum\limits_{k=1}^n \int\limits_{\Delta_k} \overline{\psi}_x(\xi,t,j)\left( \fr{1}{b_k} 
\int\limits_{\mathbf{R}^1} \varphi_v\left( \fr{x-a_k\xi-q_k}{b_k}\right)\times{}\right.\\
{}\left.\times \left( \fr{1}{\sigma} 
\int\limits_{\mathbf{R}^1} y \varphi_w \left( \fr{y-cx}{\sigma} \right) \,dy \right) \,dx \right) \,d\xi\,.
\end{multline*}

  
  Непосредственное вычисление внутренних интегралов в полученном равенстве дает 
выражение~(\ref{e6-b}) для $\overline{y}_{t+j,t-1}$. 
Теорема доказана.

  
  \smallskip
  
  \noindent
  \textbf{Пример~2.} Анализ точности предложенной процедуры прогнозирования 
проведен на модельных данных из примера~1, параметры наблюдений~(\ref{e1-b}): 
$c\hm=3{,}0$, $\sigma\hm=9{,}1$.
  
  Расчеты по формулам~(\ref{e6-b}) и~(\ref{e7-b}) проводились для:
  \begin{itemize}
  \item 10~шагов траектории: $t=1, 2, \ldots , 10$;
  \item точность прогнозирования анализировалась по пучку из 10\,000 траекторий;
\item для интегрирования в~(\ref{e6-b}) и~(\ref{e7-b}) использовался метод 
Мон\-те-Кар\-ло, для чего моделировались 1000 случайных величин, равномерно 
распределенных на интервале $[-5;\,25]$.
\end{itemize}

  Рассчитывались прогнозы $\overline{y}_{\tau+j,\tau-1}$ для $y_{\tau+j}$, $\tau\hm= 2, 3, 
\ldots , 11$, $j\hm=0, 1, \ldots , 11-\tau$. Анализировались величины ошибок оценок 
$\overline{y}_{\tau+j,\tau-1}$, для чего по пучку траекторий оценивались дисперсии ошибок 
$\overline{y}_{\tau+j,\tau-1}\hm-y_{\tau+j}$. Для анализа качества фильтрации оценивалась 
также дисперсионная функция процесса~$y_t$ и дисперсия ошибок $\tilde{y}_{\tau+j,\tau-1} 
-y_{\tau+j}$ для тривиальной оценки $\tilde{y}_{\tau+j,\tau-1}$, вычисляемой в соответствии 
с уравнениями~(\ref{e1-b})--(\ref{e3-b}):
\begin{align*}
  \tilde{y}_{\tau+j,\tau-1} &=c\tilde{x}_{\tau+j,\tau-1}\,;\\
  \tilde{x}_{\tau+j,\tau-1}&=\\
  &\hspace*{-15mm}{}=a\Theta(\tilde{x}_{\tau+j-1,\tau-1})\tilde{x}_{\tau+j-1,\tau-
1}+q\Theta(\tilde{x}_{\tau+j-1,\tau-1})\,.
  \end{align*}
  
  Некоторые результаты расчетов приведены на рис.~\ref{f7-b}.
  
  Из приведенных результатов видно, что точность оценивания, обеспечиваемая 
оптимальным прогнозом $\overline{y}_{\tau+j,\tau-1}$, в сравнении с точностью прогноза 
$\tilde{y}_{\tau+j,\tau-1}$ на первых трех--четырех шагах ($j \hm= 0, 1, 2, 3$) хотя и выше, но 
отличия сравнительно небольшие (не более 10\%). Разница в качестве оценивания возрастает 
с увеличением шага прогнозирования, но вместе с тем дисперсия ошибки оценки 
оптимального прогноза приближается к дисперсии процесса, как и следовало ожидать. 
Кроме того, надо отметить недостаток тривиального прогноза $\tilde{y}_{\tau+j,\tau-1}$~--- с 
увеличением шага прогнозирования (см.\ рис.~\ref{f7-b},\,\textit{а}--\textit{в}) 
обеспечиваемая этим  прогнозом точность становится неудовлетворительной, так как 
дисперсия ошибки  превосходит дисперсию самого процесса.

\section{Идентификация параметров}

  Заметим, что задача оценивания решена в тео\-ре\-ме~2 в предположении, что параметры 
модели известны точно. На практике же можно предложить лишь их экспертную 
оценку~\cite{19-b}, судить о точности которой без достоверной процедуры идентификации 
невозможно. В~данном разделе предлагается процедура оценивания параметров $a$, $q$, 
$b$, $c$ и $\sigma$ модели~(\ref{e1-b})--(\ref{e3-b}) на основе байесовского подхода.
  
  Будем предполагать, что для некоторого неизвестного случайного параметра $\gamma 
\hm=\mathrm{col}\left( \gamma^1, \ldots ,\gamma^m\right)\in \mathbf{R}^m$ параметры модели могут быть 
записаны в виде $a\hm =a(\gamma)$, $q\hm =q(\gamma)$, $b\hm =b(\gamma)$, $c\hm 
=c(\gamma)$ и $\sigma\hm =\sigma(\gamma)$. Предположив, что существует и известна 
плотность вероятности параметра~$\gamma$, можно распространить результат теоремы~2 на 
новую модель наблюдения, решив параллельно и задачу идентификации, формулируемую 
теперь как задача оценивания векторного случайного параметра~$\gamma$ по наблюдениям 
$y_\tau$, $\tau\hm\leq t$. Отметим, что такая модель идентификации позволяет предложить 
оценку для любого набора параметров модели за счет изменения размерности 
вектора~$\gamma$ и комбинирования в выборе функций $ a(\gamma)$, $ q(\gamma)$, $ 
b(\gamma)$, $c(\gamma)$ и~$\sigma(\gamma)$. 
  
  Дополним уравнения состояния~(\ref{e3-b}) соотношением:
  \begin{equation}
  \gamma_t=\gamma_{t-1}\,,\quad t=1,2, \ldots; \qquad \gamma_0=\gamma\,,
  \label{e8-b}
  \end{equation}
где $\gamma =\mathrm{col}\left( \gamma^1, \ldots , \gamma^m\right)$~--- случайный вектор, не 
зависящий от $\{w_t\}$, $\{v_t\}$ и~$x_0$ и имеющий плотность 
вероятности~$\psi_\gamma(\cdot)$ 
  
  \smallskip
  
  \noindent
  \textbf{Теорема~3.} \textit{Пусть для сис\-те\-мы наблюдения}~(\ref{e1-b})--(\ref{e3-b}), 
(\ref{e8-b}) \textit{выполнено:}
  \begin{enumerate}[(1)]
  \item $\min\limits_{1\leq k\leq n} b_k(g)>0 \ \forall\ \ g\in \mathbf{R}^m:\ \psi_\gamma(g)>0$;
  \item $\sigma(g)>0\ \forall\ g\in \mathbf{R}^m:\ \psi_\gamma(g)>0$.
  \end{enumerate}
  
  \textit{Тогда условная плотность вероятности $\hat{\psi}_{x,\gamma}(x,g,t)$ существует и 
описывается урав\-не\-ниями:}



\noindent
  \begin{multline}
  \hat{\psi}_{x,\gamma}(x,g,t) ={}\\
  {}=\fr{\overline{\psi}_{z,\gamma}(z,g,t)}
{\int\limits_{\mathbf{R}^{m+1}}\overline{\psi}_{z,\gamma} (z,g,t)\,dx dg}\,, \quad
z=\mathrm{col}\left( x,y_t\right)\,;   \label{e9-b}
  \end{multline}
  
\vspace*{-12pt}
\noindent
\begin{multline}
  \overline{\psi}_{z,\gamma} (z,g,t) =\sum\limits_{k=1}^n \fr{1}{\sigma (g) b_k(g)}\,\varphi_w 
\left( \fr{y-c(g)x}{\sigma(g)}\right) \times{}\\
  {}\times \int\limits_{\Delta_k} \hat{\psi}_{x,\gamma} (\xi, g, t-1) \times{}\\
  {}\times \varphi_v\left( \fr{x-a_k(g) 
\xi -q_k(g)}{b_k(g)}\right)\,d\xi\,.
  \label{e10-b}
  \end{multline}
  
  \smallskip
  
  \noindent
  Д\,о\,к\,а\,з\,а\,т\,е\,л\,ь\,с\,т\,в\,о\,.\ Для условной функции распределения вектора $\mathrm{col} 
\left( x_t, y_t, \gamma_t\right)$ относительно $\Im^y_{t-1}$, проделав преобразования, 
полностью аналогичные выполненным при доказательстве теоремы~2, получаем:

\noindent
  \begin{multline*}
  P\left(x_t\leq x,\, y_t\leq y,\, \gamma_t\leq g \vert \Im^y_{t-1}\right)=
 {}\\
{}= \sum\limits_{k=1}^n \int\limits_{\Delta_k} \int\limits_{-\infty}^{g_1}\cdots \int\limits_{-
\infty}^{g_m} \hat{\psi}_x(\xi,\mu,t-1)\times{}\\
{}\times \int\limits_{-\infty}^{(x-a_k(\mu)\xi-q_k(\mu))/b^k(\mu)} 
\varphi_v(v) \times{}\\
{}\times\!\!
\int\limits_{-\infty}^{(y-
c(\mu)(a_k(\mu)\xi+q_k(\mu)+b_k(\mu)v))/\sigma(\mu)}\!\!\!\!\!\!\!\!\!\!\!\!\!\!
\varphi_w(w)\,dw d\mu_1 \cdots d\mu_m 
d\xi
\end{multline*}
в предположении существования условной плотности вероятности 
$\hat{\psi}_{x,\gamma}(x,g,t-1)$. Отсюда, дифференцируя интегралы с переменными 
верхними пределами по~$x$, по~$y$ и по~$g$, получаем, что существует условная 
плотность вероятности $\overline{\psi}_{z,\gamma}(z,g,t)$ следующего вида:

\noindent
\begin{multline*}
\overline{\psi}_{z,\gamma}(z,g,t) = \sum\limits_{k=1}^n \fr{1}{\sigma(g) b_k(g)} \times{}\\
{}\times
\int\limits_{\Delta_k} \hat{\psi}_{x,\gamma} (\xi,g,t-1)\varphi_v \left( \fr{x-a_k(g)\xi-
q_k(g)}{b_k(g)}\right)\times{}\\
{}\times \varphi_w\left( \fr{1}{\sigma(g)}\left( y-c(g)\left( 
\vphantom{\fr{x-a_k(g)\xi-q_k(g)}{b_k(g)}}
a_k(g)\xi+q_k(g)+{}\right.\right.\right.\\
\left.\left.\left.{}+b_k(g) \fr{x-a_k(g)\xi-q_k(g)}{b_k(g)}\right)\right)\right)\,d\xi\,.
\end{multline*}
  
  Отсюда уравнение~(\ref{e10-b}) получается после очевидных преобразований, а также 
путем интегрирования по~$x$ и по~$g$ получается маргинальная условная плотность 
$\overline{\psi}_y(y,t)$. Окончательно соотношение~(\ref{e9-b}) получаем по формуле Байеса 
для плотностей веро-\linebreak\vspace*{-12pt}
\columnbreak

\noindent
ятности~\cite{21-b} как отношение $\overline{\psi}_{z,\gamma}(z,g,t)$ и 
$\overline{\psi}_y(y,t)$, где $y=y_t$. Теорема доказана.
  
  \smallskip
  
  \noindent
  \textbf{Замечание~1.} В~соотношениях~(\ref{e9-b}) и~(\ref{e10-b}) используется 
начальное условие $\hat{\psi}_{x,\gamma}(x,g,0)\hm = \psi_0(x)\psi_\gamma(g)$. Эти 
соотношения полностью описывают рекуррентную процедуру пересчета искомой условной 
плотности вероятности вектора $\mathrm{col}\left( x_t,\gamma\right)$ относительно~$\Im^y_{t-1}$~---
решения задачи оптимальной фильтрации.
  
  \smallskip
  
  \noindent
  \textbf{Замечание~2.} Уравнения~(\ref{e9-b}) и~(\ref{e10-b}) могут быть преобразованы к 
виду формулы Байеса~\cite{21-b}, если использовать в них ненормированные условные 
плотности вероятности~\cite{22-b}. Пусть $t=1$. Тогда с учетом начального условия 
из~(\ref{e10-b}) получаем:

\noindent
  \begin{multline*}
  \overline{\psi}_{z,\gamma}(z,g,1) =\sum\limits_{k=1}^n \fr{1}{\sigma(g) b_k(g)}\,\varphi_w 
\left( \fr{y_1-c(g)x}{\sigma(g)}\right)\times{}\\
{}\times \psi_\gamma(g) \int\limits_{\Delta_k}\psi_0(\xi)\varphi_v 
\left(  \fr{x-a_k(g)\xi-q_k(g)}{b_k(g)}\right) \,d\xi\,.
  \end{multline*}
  
  Обозначим через $\tilde{\psi}_x (x,1\vert g)$ ненормированную условную плотность 
вероятности~$x_t$, $t=1$, относительно~$\Im^y_1$ и условия $\gamma_1=g$:

\noindent
  \begin{multline*}
  \tilde{\psi}_x(x,1\vert g) =\sum\limits_{k=1}^n \fr{1}{\sigma(g) b_k(g)}\,\varphi_w \left( 
\fr{y_1-c(g)x}{\sigma(g)}\right)\times{}\\
{}\times \int\limits_{\Delta_k} \psi_0(\xi) \varphi_v \left( \fr{x-a_k(g)\xi -
q_k(g)}{b_k(g)}\right) \,d\xi\,.
  \end{multline*}
  
  Используя $\tilde{\psi}_x(x,1\vert g)$ в качестве начального условия, запишем следующие 
рекуррентные уравнения:

\noindent
  \begin{multline}
  \tilde{\psi}_x(x,t\vert g)=\sum\limits_{k=1}^n \fr{1}{\sigma(g) b_k(g)}\,\varphi_w \left( \fr{y-
c(g)x}{\sigma(g)}\right)\times{}\\
\!\!\!{}\times  \int\limits_{\Delta_k}\! \tilde{\psi}_x(\xi,t-1\vert g) \varphi_v \left( 
  \fr{x-a_k(g)\xi-q_k(g)}{b_k(g)}\right)\,d\xi\,.
  \label{e11-b}\!\!\!
  \end{multline}
  Тогда~(\ref{e9-b}) приобретает вид:
  \begin{equation}
  \hat{\psi}_{x,\gamma}(x,g,t) =\fr{\tilde{\psi}_x(x,t\vert 
g)\psi_\gamma(g)}{\int\limits_{\mathbf{R}^{m+1}}\tilde{\psi}_x(x,t\vert g)\psi_\gamma(g)\,dx 
dg}\,.
  \label{e12-b}
  \end{equation}
  
  \begin{figure*}[b] %fig8
\vspace*{1pt}
\begin{minipage}[t]{80mm}
\begin{center}
\mbox{%
\epsfxsize=76.888mm
\epsfbox{bos-13.eps}
}
\end{center}
\vspace*{-9pt}
\Caption{Результаты идентификации (вариант~1):
\textit{1}~--- $a_1$; \textit{2}~---  $\hat{a}_1(t)$;
\textit{3}~--- $q_1$; \textit{4}~--- $\hat{q}_1(t)$;
\textit{5}~--- $b_1$; \textit{6}~---  $\hat{b}_1(t)$
\label{f13-b}}
%\end{figure*}
\end{minipage}
\hfill    
%\begin{figure*} %fig9
\vspace*{1pt}
\begin{minipage}[t]{80mm}
\begin{center}
\mbox{%
\epsfxsize=76.625mm
\epsfbox{bos-14.eps}
}
\end{center}
\vspace*{-9pt}
\Caption{Результаты идентификации (вариант~4):
\textit{1}~--- $c$; \textit{2}~---  $\hat{c}(t)$;
\textit{3}~--- $\sigma$; \textit{4}~--- $\hat{\sigma}(t)$
\label{f14-b}}
\end{minipage}
\end{figure*}
  
  При практическом применении алгоритма идентификации использование 
уравнений~(\ref{e11-b}) и~(\ref{e12-b}) вместо~(\ref{e9-b}) и~(\ref{e10-b}) облегчает численную 
реализацию за счет сокращения вычислительных затрат.
  
  \smallskip
  
  \noindent
  \textbf{Пример~3.} Предложенную в теореме~3 процедуру идентификации применим к 
реальным данным, а~именно: воспользуемся экспертной моделью из примера~1, 
предложенной для наблюдений за активностью пользователей портала 
РАН {\sf www.ras.ru}.\linebreak\vspace*{-12pt}
\pagebreak

\noindent
 
Экспертные оценки параметров уравнения~(\ref{e3-b}) из табл.~1 дополним 
предположением о параметрах уравнения~(\ref{e1-b}): $c=1$, $\sigma=1$.
  
  Рассматривались четыре варианта моделей идентификации параметров:
  \begin{enumerate}[(1)]
\item в первом случае неизвестными считались три параметра $a_1$, $q_1$ и $b_1$, 
определяющих динамику для первого уровня пользовательской ак\-тив\-ности;
\item во втором~--- выполнялось оценивание па\-ра\-мет\-ров $a_2$, $q_2$ и~$b_2$, 
определяющих динамику для второго уровня пользовательской активности;
\item в третьем~--- выполнялось оценивание па\-ра\-мет\-ров $a_3$, $q_3$ и~$b_3$, 
определяющих динамику для третьего уровня пользовательской активности;
\item в четвертом~--- выполнялось оценивание параметров~$c$ и~$\sigma$, определяющих 
уравнения наблюдений~(\ref{e1-b}).
  \end{enumerate}
  
  Полученные результаты в первых трех случаях содержательно идентичны, поэтому далее 
рас\-смат\-ри\-ва\-ют\-ся только первый и четвертый варианты. В~каж\-дом из них априорные 
распределения подлежащих идентификации параметров предполагались равномерными, 
параметры приведены в табл.~3 и~4. Расчеты выполнялись с использованием 
ненормированных распределений по формулам~(\ref{e11-b}) и~(\ref{e12-b}) для имеющихся 
5000~наблюдений. Для интегрирования методом Мон\-те-Кар\-ло моделировались 
1000~случайных величин для интегрирования 
по $d\xi(dx)$ (пределы интегрирования заданы 
интер-\linebreak\vspace*{-12pt}
\columnbreak

%\vspace*{6pt}

\noindent
\begin{center}
\noindent
{\tablename~3}\ \ \small{Параметры идентификации~--- вариант~1}
\end{center}
%\vspace*{2ex}

\begin{center}
%\tabcolsep=9pt
   

{\small   
\tabcolsep=5.2pt
\begin{tabular}{|c|c|l|}
   \hline
\tabcolsep=0pt\begin{tabular}{c}Идентифицируемые\\ параметры\end{tabular}& 
\tabcolsep=0pt\begin{tabular}{c}Оценка\\ эксперта\end{tabular} & 
\tabcolsep=0pt\begin{tabular}{c}Априорное\\ распределение\end{tabular}\\
\hline
$a_1$&\hphantom{9}0,361 & $\mathbf{R[0{,}1;0{,}9]}$\\
$q_1$& $-67{,}784$\hphantom{$-$} & $\mathbf{R[-100{,}0;-50{,}0]}$\\
$b_1$ & 36,432 & $\mathbf{R[0{,}0;50{,}0]}$\\
\hline
\end{tabular}
}
\end{center}
\vspace*{3pt}

  
\begin{center}
\noindent
{\tablename~4}\ \ \small{Параметры идентификации~--- вариант~4}
\end{center}
%\vspace*{2ex}

\begin{center}
\tabcolsep=8pt
{\small \begin{tabular}{|c|c|c|}
\hline
\tabcolsep=0pt\begin{tabular}{c}Идентифицируемые\\ параметры\end{tabular}& 
\tabcolsep=0pt\begin{tabular}{c}Оценка\\ эксперта\end{tabular} & 
\tabcolsep=0pt\begin{tabular}{c}Априорное\\ распределение\end{tabular}\\
\hline
$c$& 1,0 & $\mathbf{R[0{,}1; 5{,}1]}$\\
$\sigma$ & 1,0 & $\mathbf{R[0{,}1; 5{,}1]}$\\
\hline
\end{tabular}
}
\end{center}
\vspace*{12pt}

%\bigskip
\addtocounter{table}{2}


\noindent
валом $[-1000{,}0; 1000{,}0]$) и 10\,000 случайных величин для интегрирования по~$dg$.

На рис.~\ref{f13-b} и~\ref{f14-b} приведены результаты идентификации параметров.

  Из полученных результатов видно, что хотя в целом оценки, данные экспертом, 
достаточно обос\-но\-ва\-ны, но определенную коррекцию процедура идентификации все-таки 
вносит. Кроме того, проведенные расчеты свидетельствуют и о практической 
работоспособности предложенных методов оценивания, и об адекватности используемой 
математической модели реально наблюдаемым данным.

\vspace*{-6pt}

\section{Заключение}

\vspace*{-2pt}
  
  Представлена модель стохастической динамической сис\-те\-мы наблюдения, описывающей 
процес\-сы изменения активности пользователей, выполняющих запросы к некоторой 
информационной сис\-те\-ме. К~ключевым свойствам описываемого данной моделью процесса 
предложено отнести сле\-ду\-ющие предположения:
  \begin{itemize}
\item наличие нескольких характерных режимов, в рамках которых динамика изучаемого 
показателя описывается простой линейной моделью;
\item скачкообразное изменение характера показателя активности при достижении им 
заданных границ;
\item наличие косвенных линейных наблюдений за показателем.
\end{itemize}

  Случайный процесс, описывающий показатель активности, при практически выполнимых 
условиях обладает свойством эргодичности, т.\,е.\ существует предельное распределение, 
которое, с одной стороны, отражает свойственный процессу кусочный характер, с другой~--- 
гарантирует наличие стационарного режима распределения вероятностей, что очень 
существенно в практическом смысле.
  
  Применительно к предложенной модели решены две базовые задачи анализа~--- 
прогнозирования состояния и идентификации параметров. Работоспособность алгоритмов 
оценивания проиллюстрирована на численных экспериментах, в том числе с использованием 
реальных данных, полученных с сайта РАН {\sf www.ras.ru}.
  
  Сами по себе процедуры анализа могут быть использованы лишь в простейших задачах 
оптимизации работы самой информационной сис\-те\-мы, ее программного обеспечения. 
Однако их основная ценность заключается в возможности применения\linebreak
в полноценных 
постановках оптимизационных задач. Такие постановки возникают, например, в задачах 
управления вычислительными ресурсами, исполь\-зу\-емы\-ми информационной сис\-те\-мой при 
реализации собственной функциональности. Некоторые из решенных задач будут описаны в 
следующих работах.

\vspace*{-6pt}

{\small\frenchspacing
{%\baselineskip=10.8pt
\addcontentsline{toc}{section}{Литература}
\begin{thebibliography}{99}

\bibitem{1-b}
\Au{Микадзе И.\,С., Хочолава В.\,В.}
Исследование длины очереди в однолинейной СМО с ненадежным прибором~// Автоматика 
и телемеханика, 2005. №\,1. С.~72--81.

\bibitem{2-b}
\Au{Печинкин А.\,В., Соколов И.\,А., Чаплыгин~В.\,В.}
Многолинейная сис\-те\-ма массового обслуживания с конечным накопителем и ненадежными 
приборами~// Информатика и её применения, 2007. Т.~1. Вып.~1. С.~27--39.

\bibitem{3-b}
\Au{Jacobson V.}
Congestion avoidance and control~// \mbox{ACMSIGCOMM'88}, 1988.

\bibitem{4-b}
\Au{Allman M., Paxson V., Stevens~W.}
TCP congestion control~// RFC 2581, 1999. {\sf http://www.ietf.org/rfc/ rfc2581.txt}.

\bibitem{5-b}
\Au{Kelly F.\,P., Maulloo A., Tan~D.}
Rate control in communication networks: Shadow prices, proportional fairness and stability~// 
J.~Operational Res. Soc., 1998. Vol.~49. P.~237--252.

\bibitem{6-b}
\Au{Low S.\,H., Paganini F., Doyle~J.\,C.}
Internet congestion control~// IEEE Control Systems Magazine, 2002. Vol.~22. No.\,1. P.~28--43.

\bibitem{7-b}
\Au{Миллер Б.\,М., Миллер Г.\,Б., Семенихин~К.\,В.}
Методы синтеза оптимального управления марковским процессом с конечным множеством 
состояний при наличии ограничений~// Автоматика и телемеханика, 2011. №\,2. С.~111--130.

\bibitem{8-b}
\Au{Батракова Д.\,А., Королев В.\,Ю., Шоргин~С.\,Я.}
Новый метод вероятностно-статистического анализа информационных потоков в 
телекоммуникационных сетях~// Информатика и её применения, 2007. Т.~1. Вып.~1. 
С.~40--53.

\bibitem{9-b}
\Au{Дейт~К.\,Дж.}
Введение в сис\-те\-мы баз данных.~--- М.: Вильямс, 2000.

\bibitem{10-b}
\Au{Гордеев А.\,В., Молчанов~А.\,Ю.}
Системное программное обеспечение.~--- СПб.: Питер, 2001.

\bibitem{11-b}
Информационный веб-пор\-тал. Свидетельство об официальной регистрации программы для 
ЭВМ №\,2005612992. Зарегистрировано в Реестре программ для ЭВМ 18.11.2005~г.

\bibitem{12-b}
\Au{Босов А.\,В.}
Порталы в сис\-те\-мах органов государственной власти~// Информатика и её применения, 2008. 
Т.~2. Вып.~1. С.~24--34.

\bibitem{13-b}
\Au{Босов А.\,В., Борисов А.\,В.}
Модели оптимизации функционирования Информационного web-пор\-та\-ла~// Технологии 
программирования и хранения данных: Труды ИСА РАН, 2009. Т.~45. С.~107--133.

\bibitem{14-b}
\Au{Босов А.\,В.}
Моделирование и оптимизация процессов функционирования Информационного 
web-пор\-та\-ла~// Программирование, 2009. №\,6. С.~53--66.

\bibitem{15-b}
Slashdot effect. {\sf http://en.wikipedia.org/wiki/Slashdot\_ effect}.

\bibitem{16-b}
\Au{Tong H.}
Non-linear time series: A dynamical approach.~--- Oxford: Clarendon Press, 1990.

\bibitem{17-b}
\Au{Борисов А.\,В., Босов А.\,В., Стефанович~А.\,И.}
Оптимальное оценивание показателей функционирования информационного web-портала // 
Автоматика и телемеханика, 2010. №\,3. С.~16--33.

\bibitem{18-b}
\Au{Bhattacharya~R.\,N., Lee~C.}
Ergodicity of nonlinear first order autoregressive models~// J.~Theor. Prob., 1995. 
Vol.~8. No.\,1. P.~207--219.

\bibitem{19-b}
\Au{Иванов А.\,В.}
Математические модели базовых процессов функционирования информационного 
Web-пор\-та\-ла~// Системы и средства информатики, 2010. Вып.~20. №\,1. С.~106--132.

\bibitem{20-b}
\Au{Гихман И.\,И., Скороход А.\,В.}
Теория случайных процессов. Т.~1.~--- М.: Наука, 1971.

\bibitem{21-b}
\Au{Ширяев А.\,Н.}
Вероятность.~--- М.: Наука, 1989.

\label{end\stat}

\bibitem{22-b}
\Au{Aggoun L., Elliott~R.}
Measure theory and filtering. Introduction and applications.~--- Cambridge: Cambridge University 
Press, 2004.
 \end{thebibliography}
}
}


\end{multicols}       
  