\def\stat{zatsar}

\def\tit{УПРАВЛЕНИЕ ИНФОКОММУНИКАЦИОННЫМИ ПРОЕКТАМИ: 
<<СВОЕВРЕМЕННОСТЬ--ПРОИЗВОДИТЕЛЬНОСТЬ--ИНФОРМАЦИЯ>>}

\def\titkol{Управление инфокоммуникационными проектами: 
<<своевременность--производительность--информация>>}

\def\autkol{А.\,А.~Зацаринный, А.\,П.~Шабанов}
\def\aut{А.\,А.~Зацаринный$^1$, А.\,П.~Шабанов$^2$}

\titel{\tit}{\aut}{\autkol}{\titkol}

%{\renewcommand{\thefootnote}{\fnsymbol{footnote}}\footnotetext[1]
%{Работа поддержана Российским фондом фундаментальных исследований
%(проекты 11-01-00515а и 11-07-00112а), а также Министерством
%образования и науки РФ в рамках ФЦП <<Научные и
%научно-педагогические кадры инновационной России на 2009--2013~годы>>.}}


\renewcommand{\thefootnote}{\arabic{footnote}}
\footnotetext[1]{Институт проблем информатики Российской академии наук, AZatsarinny@ipiran.ru}
\footnotetext[2]{ООО <<ИБС Экспертиза>>, AShabanov@ibs.ru}

%\vspace*{6pt}      
      
      \Abst{Рассматривается методологический подход к управлению инфокоммуникационными 
проектами с обоснованием требований к производительности трактов конт\-роль\-но-тех\-но\-ло\-ги\-че\-ских 
сис\-тем (КТС) и 
к числу субъектов функциональных организационных структур, выполняющих работы в соответствии с 
принимаемыми из этих трактов сообщениями.}
      
      \KW{управление проектом; конт\-роль\-но-тех\-но\-ло\-ги\-че\-ская сис\-те\-ма; организационная 
структура; своевременность; производительность; информация}

%\vspace*{6pt}

 \vskip 14pt plus 9pt minus 6pt

      \thispagestyle{headings}

      \begin{multicols}{2}
      
            \label{st\stat}

\section{Постановка задач}

     В комплексе мер по обеспечению модернизации и технологического обновления 
производственной сферы страны 
     ин\-фор\-ма\-ци\-он\-но-те\-ле\-ком\-му\-ни\-ка\-ци\-он\-ные сис\-те\-мы (ИТКС) 
являются материальной основой для обеспечения высокой степени качества управ\-ле\-ния в 
организационных структурах любого масштаба. Поэтому одна из важнейших задач 
заключается в повышении эффективности управ\-ле\-ния проектами по созданию и 
сопровождению ИТКС. На этом пути возможны как учет фактического состояния или 
изменения структуры и параметров сис\-те\-мы в реальном времени, так и адаптация к 
различным факторам неопределенности на основе накопления и использования 
информации о предметных сущностях деятельности. Разработка материала настоящей 
статьи осуществлялась на принципах сис\-тем\-но\-го подхода~[1] с учетом принятой в 
рассматриваемой области терминологии, результатов авторских исследований, на основе 
положений и сис\-тем классификации, приведенных в документах:
     \begin{itemize}
\item ГОСТ Р ИСО/МЭК 15288 <<ИТ. Системная инженерия. Процессы жизненного цикла 
сис\-тем>>;
\item практическое руководство по инновационному управлению качеством и рисками в 
жизненном цикле сис\-тем~[2];
\item практическое руководство по проектированию ИТКС~[3].
\end{itemize}

     Ниже приведены основные термины и определения, используемые в статье:
     \begin{itemize}
\item инфокоммуникационные проекты~--- проекты по созданию, эксплуатации, 
модернизации, развитию ИТКС. Сложность проектов подразумевает наличие 
технических, организационных и ресурсных задач, решение которых предполагает 
нетривиальные подходы к их решению;
\item управление инфокоммуникационным проектом~--- управление процессами 
жизненного цик\-ла ИТКС. Жизненный цикл ИТКС~--- промежуток времени между 
моментом появления, зарождения проекта и моментом его ликвидации, завершения. 
Жизненный цикл ИТКС, как правило, включает стадии замысла, разработки, 
производства, применения, поддержки применения, прекращения применения и списания. 
Процессами жизненного цикла ИТКС являются: процессы соглашения; процессы 
предприятия; процессы проекта; технические про\-цессы;
{\looseness=1

}
\item контрольно-технологические сис\-те\-мы~--- информационные сис\-те\-мы, которые 
образуются на базе ресурсов ИТКС для передачи информации, относящейся к решению 
задач оперативного управления техническими объектами и их контроля;
\item своевременность представления информации~--- свойство сис\-те\-мы обеспечивать 
представление информации в задаваемые сроки, гарантирующие выполнение 
соответствующей функции согласно целевому назначению сис\-те\-мы;
\item функциональная организационная структура~--- объединение субъектов в штатной 
организаци-\linebreak\vspace*{-12pt}

\setcounter{figure}{1}
\begin{figure*}[b] %fig2
\vspace*{1pt}
\begin{center}
\mbox{%
\epsfxsize=122.607mm
\epsfbox{zac-2.eps}
}
\end{center}
\vspace*{-9pt}
\Caption{Обоснование требований к производительности тракта}
\end{figure*}

\pagebreak

\noindent
онной структуре предприятия, каждый из которых выполняет свои функции 
для решения общей для них и четко определенной, конкретной задачи;
\item технологическая информация~--- храня\-щи\-еся в базе данных конфигурационных 
единиц (\mbox{БДКЕ}) ИТКС взаимоувязанные модули, в своей совокупности описывающие 
предметные сущности, необходимые для выполнения работ.
{\looseness=1

}
     \end{itemize}
     
     В статье исследуются вопросы, относящиеся к \textit{проблеме обоснования 
требований к производительности} КТС. В~рамках этой проблемы решаются следующие 
задачи:
     \begin{itemize}
\item определение требований к произво\-ди\-тель\-ности трактов КТС в зависимости от 
заданных требований к своевременности представления информации (первая задача);
\item определение требований к числу субъектов функциональных организационных 
структур, выполняющих работы в соответствии с принимаемыми из трактов КТС 
сообщениями (вторая задача).
     \end{itemize}
     
     При управлении проектом обе задачи решаются на \textit{стадии замысла} в 
жизненном цикле ИТКС в ходе реализации \textit{технического процесса по анализу 
требований}. 

Управление процессом анализа требований осуществляется на основе 
обще\-сис\-тем\-но\-го подхода к созданию ИТКС, отображенного на рис.~1, который 
иллюстрирует приоритетность требований бизнеса перед другими компо\-нен\-тами. 
{ %\looseness=1

}


\begin{center} %fig1
\vspace*{9pt}
\mbox{%
\epsfxsize=62.196mm
\epsfbox{zac-1.eps}
}
\end{center}
\begin{center}
%\vspace*{6pt}
{{\figurename~1}\ \ \small{Подход к созданию ИТКС}}
\end{center}
%\vspace*{9pt}

%\smallskip
%\addtocounter{figure}{1}




\section{Управление техническим процессом анализа требований}
     
     Применяя данный подход к решению по\-став\-лен\-ных выше задач, проведем анализ: 
     \begin{itemize}
\item зависимости требований бизнеса к своевременности доставки сообщений и 
требований к производительности используемых для этого трактов ИТКС (рис.~2);
\item зависимости требований бизнеса к своевременности выполнения работ в 
соответствии с доставленным сообщением и требований к числу субъектов 
функциональной организационной структуры, созданной для выполнения этих работ 
(рис.~3).
\end{itemize}



     Технический процесс управления анализом требований в рассматриваемой здесь 
области включает в себя следующие основные функции:
     \begin{itemize}
\item разработка модели КТС как сис\-те\-мы биз\-нес-клас\-са по отношению к ИТКС;
\item разработка методического подхода к обоснованию требований к производительности 
трактов КТС и к числу субъектов функциональной организационной структуры.
     \end{itemize}
     
     Модель для описания КТС приведена на рис.~4.


     В состав КТС входят следующие, основные для рассматриваемых задач, части:
     \begin{itemize}
\item технические источники информации~--- датчики $\mathrm{Д}_{1-1}, \ldots , 
\mathrm{Д}_{1-N_1}$, осуществляющие автоматический контроль параметров объектов 
бизнеса, оснащенные преобразователями сигналов контроля в информационные 
сообщения; тип групповых трактов ИТКС, используемых для передачи сообщений, 
поступивших из технических источников информации, назовем техническим трактом 
КТС;
\item субъективные источники информации~--- субъекты бизнеса, оснащенные 
персональными компьютерами $\mathrm{Д}_{1-1}, \ldots , \mathrm{Д}_{1-N_n}$ для 
формирования сообщений об объектах; тип групповых трактов ИТКС, используемых для 
передачи сообщений из субъективных источников информации, назовем 
информационным трактом КТС;
\item функциональная организационная структура~--- субъекты бизнеса, оснащенные 
персональными компьютерами и выполняющие работы в соответствии с поступающими 
из трактов КТС сообщениями. 
\end{itemize}

\section{Методика обоснования требований к~производительности тракта}

     Методика обоснования требований к производительности $W_{\mathrm{доп}}$ 
группового тракта КТС состоит из следующих шагов:
     
\textit{Шаг 1.} Определение вероятностей $P_j$ того, что поступившее в тракт сообщение 
застанет в нем $j$ требований, где $j = 0, 1, 2$, \ldots

\textit{Шаг 2.} Выполнение последовательного суммирования вероятностей $P_j$ для 
значений $j \hm= 0, 1$,\linebreak 2, \ldots до тех пор, пока выполняется условие

\noindent
\begin{equation}
\sum\limits_{j=0}^{J_{\mathrm{доп}}} P_j\leq P_{\mathrm{доп}}\,,
\label{e2-z}
\end{equation}
где $J_{\mathrm{доп}}$~--- максимально допустимое число сообщений в тракте, при котором 
еще выполняется требование к вероятности $P_{\mathrm{доп}}$ непревышения 
максимально допустимого времени $T_{\mathrm{доп}}$ доставки сообщения (см.\ рис.~2). 
Условие~(\ref{e2-z}) эквивалентно условию:
\begin{equation*}
J_{\mathrm{доп}} T_{\mathrm{обсл}}\leq T_{\mathrm{доп}}\,,
%\label{e3-z}
\end{equation*}
где $T_{\mathrm{обсл}}$~--- временной интервал обслуживания (ВИО) одного сообщения.

     \begin{figure*} %fig3
\vspace*{1pt}
\begin{center}
\mbox{%
\epsfxsize=122.607mm
\epsfbox{zac-3.eps}
}
\end{center}
\vspace*{-9pt}
\Caption{Обоснование требований к числу субъектов организационной 
структуры}
%\end{figure*}
%\begin{figure*} %fig4
\vspace*{12pt}
\begin{center}
\mbox{%
\epsfxsize=135.753mm
\epsfbox{zac-4.eps}
}
\end{center}
\vspace*{-9pt}
\Caption{Модель КТС}
\end{figure*}


\textit{Шаг 3.} Определение минимально допустимой производительности тракта КТС:
\begin{equation*}
W_{\mathrm{доп}} =\fr{J_{\mathrm{доп}} K_{\mathrm{обсл}}}{ T_{\mathrm{доп}}}\,,
%\label{e4-z}
\end{equation*}
где $K_{\mathrm{обсл}}$~--- объем передаваемого в тракт сообщения, для технических трактов 
это постоянная величина, для информационных трактов используется среднее значение.
     
     При определении вероятностей $P_j$ для информационных трактов КТС 
применяется аппарат\linebreak тео\-рии массового обслуживания, например при расчетах можно 
использовать представленные в работах~\cite{2-z, 5-z} программно-математические и 
имитационные средства моделирования различных сис\-тем массового обслуживания.
     
     Для обоснования требований к производительности $W_{\mathrm{доп}}$ технического 
тракта КТС разработана модель тракта, приведенная на рис.~5.

\begin{figure*} %fig5
\vspace*{1pt}
\begin{center}
\mbox{%
\epsfxsize=128.599mm
\epsfbox{zac-5.eps}
}
\end{center}
\vspace*{-9pt}
\Caption{Модель технического тракта КТС: $T_{\mathrm{пост}}$~--- период 
поступления сообщения из каждого датчика; $N$~--- число датчиков, 
подключенных к тракту (число детерминированных потоков)}
\end{figure*}

    В случаях, когда загрузка $\rho$ тракта меньше~1, учитывая известный из теории 
массового обслуживания подход к аппроксимации $N$ детерминированных потоков 
суммирующим пуассоновским потоком, вероятности~$P_j$ можно определить с помощью 
формул~\cite{5-z}:
    \begin{equation}
    P_j=\begin{cases}
    1-\rho & \hspace*{-8mm}\mbox{\ для\ } j=0\,;\\
    (1-\rho)(e^\rho-1) & \hspace*{-8mm}\mbox{\ для\ } j=1\,;\\
    \displaystyle(1-\rho) \sum\limits_{k=1}^j \left \{ 
    \vphantom{\fr{(k\rho)^{j-k}}{(j-k)!}}
(-1)^{j-k} e^{k\rho}\times{}\right.&\\
    \left.{}\times  \left[ \fr{(k\rho)^{j-k}}{(j-
k)!}+\fr{(k\rho)^{j-k-1}}{(j-k-1)!}\right] \right\}  &\\
& \hspace*{-8mm}\mbox{\ для\ } j\geq 2\,.
    \end{cases}
    \label{e5-z}
    \end{equation}
     
     В то же время очевидно, что технический тракт КТС может быть загружен 
полностью ($\rho\hm=1$). Этому состоянию тракта соответствует условие:
     \begin{equation*}
     T_{\mathrm{пост}} = N  T_{\mathrm{обсл}}  \,.
%     \label{e6-z}
     \end{equation*}
     

     
Выделим следующие особенности функционирования технических трактов КТС:
\begin{itemize}
\item поступление сообщения в тракт от каждого из $N$ датчиков в каждом периоде 
$T_{\mathrm{пост}}$ может произойти равновероятно в любом $k$-м ВИО, где $1 \leq k\leq 
N$, независимо от моментов поступления сообщений от других датчиков;
\item число сообщений~$j$, которое застает в тракте вновь поступившее $k$-е сообщение, 
определяется не только его местом в периоде~$T_{\mathrm{пост}}$, но и тем, какое число 
сообщений застало в тракте предыдущее ($k- 1$)-е сообщение.
\end{itemize}

Данные особенности отражает граф вероятностей состояний тракта (рис.~6). 


На этом графе $P^k_N (j)$~--- вероятность того, что к моменту поступления $k$-го ($1 \leq 
k \leq N$) по порядку в периоде $T_{\mathrm{пост}}$ сообщения в тракте находится~$j$ 
($0 \hm\leq j\hm\leq k- 1$) сообщений при известном числе $N$ источников информации; 
при этом соблюдается условие:
\begin{equation*}
\sum\limits_{j=0}^{k-1} P_N^k(j)=1\,.
%\label{e7-z}
\end{equation*}
Тогда вероятности $P_j$ того, что поступившее в тракт сообщение застанет в нем $j$ 
сообщений ($0 \leq j \hm\leq N- 1$) при известном чис\-ле $N$ источников информации 
будут равны:
\begin{equation}
P_j=P_N(j) =\fr{1}{N}\sum\limits_{k=j+1}^N P_N^k(j)\,,
\label{e8-z}
\end{equation}

\begin{center} %fig6
\vspace*{2pt}
\mbox{%
\epsfxsize=78.171mm
\epsfbox{zac-6.eps}
}
\end{center}
%\begin{center}
\vspace*{3pt}
{{\figurename~6}\ \ \small{Граф вероятностей состояний технического тракта КТС}}
%\end{center}
\vspace*{11pt}

%\smallskip
\addtocounter{figure}{1}

\noindent
где 
\begin{equation}
\sum\limits_{j=0}^{N-1} P_N(j)=1\,. 
\label{e9-z}
\end{equation}

С помощью аппарата теории вероятностей и комбинаторного раздела математики, путем 
выяв-\linebreak\vspace*{-12pt}
\pagebreak


\begin{center} %fig7
\vspace*{2pt}
\mbox{%
\epsfxsize=79.197mm
\epsfbox{zac-7.eps}
}
\end{center}
%\begin{center}
\vspace*{3pt}
{{\figurename~7}\ \ \small{Пример расчета минимально допустимой производительности тракта: 
$P_j$~--- вероятность нахождения в тракте $j$~сообшений;
$J_{\mathrm{доп}}(M_{\mathrm{мин}})$~--- допустимое число сообщений в тракте 
(число субъектов функциональной организационной структуры); $T_{\mathrm{доп}} 
\hm= 5\,\mathrm{ВИО}$~--- максимально допустимое время доставки сообщения;
$N$~--- чис\-ло датчиков}}
%\end{center}
\vspace*{11pt}

%\smallskip
\addtocounter{figure}{1}

\noindent
 ления и анализа всевозможных комбинаций событий, в своей совокупности 
определяющих вероятности состояний тракта в различных ВИО периода 
$T_{\mathrm{пост}}$ для различных значений $N$ источников информации, в 
работе~\cite{6-z} получены следующие формулы для вероятностей~$P^k_N(j)$:
\begin{align}
P^1_N(0)&=1\,;\label{e10-z}\\
P^2_N(1) &=1\,;\label{e11-z}\\
P_N^k(j) &=\fr{(N-1)}{N^{N-2}}\left \{ 
\vphantom{\sum\limits_{m=1}^{j-1}}
\fr{(k-j)^{k-j-2}}{(k-j-1)!}\times{}\right.\notag\\
& {}\times\left( \vphantom{\sum\limits_{m=1}^{j-1}}
\fr{(N-k+j+1)^{N-k+j-1}}{(N-k+j)!}-{}\right.\notag\\
&\left.{}-
\sum\limits_{m=1}^{j-1}\fr{(N-k+j)^{N-k+j-m-1}}{(m-1)! (N-k+j-m)!}\right)+{}\notag\\
&{}+\sum\limits_{x=1}^{j-1} \left( 
\sum\limits_{y=0}^x (-1)^y(x-y+1)^y\times{}\right.\notag\\
&{}\times \fr{(k-j+x-y)^{k-j+x-
y-2}}{y!(k-j+x-y-1)!}\times{}\notag\\
&{}\times \left( \vphantom{\sum\limits_{z=0}^{j-x-1}}
\fr{(x+1)(N-k+j+1)^{N-k+j-x-1}}{(N-k+j-x)!}-{}\right.\notag\\
&\hspace*{-12mm}\left.\left.\left.{}-\sum\limits_{z=0}^{j-x-1}\fr{(N-k+j)^{l-k+j-x-z-1}}{z!(N-k+j-x-
z)!}\,(x+z)\right)\right)\right\}\!\label{e12-z}
\end{align}
для $N = 3, 4, \ldots$; $k = 3, 4, \ldots , N$; $j = 1, 2, \ldots , k- 1$;
\begin{equation}
P^k_N(0)=0 \ \mbox{для\ }k=2, 3, \ldots , N\,.
\label{e13-z}
\end{equation}

Подставляя (\ref{e10-z})--(\ref{e13-z}) в~(\ref{e8-z}) и выполняя действия в соответствии с 
шагами~2 и~3 вышеприведенной методики, получим значения $W_{\mathrm{доп}}$ 
минимально допустимой производительности технического тракта КТС для различного 
числа $N$ источников информации. На рис.~7 приведен пример расчета допустимого 
числа $J_{\mathrm{доп}}$ сообщений, при котором соблюдается вероятность 
$P_{\mathrm{доп}}$ непревышения максимально допустимого времени $T_{\mathrm{доп}}$ 
доставки сообщения (выбирается большее целое, $J_{\mathrm{доп}} = 5$).


\section{Функциональная организационная структура. Информация}
     
     Методика обоснования требований к числу $M_{\mathrm{мин}}$ субъектов 
функциональных организационных структур, выполняющих работы в соответствии с 
принимаемыми из трактов КТС сообщениями, состоит из следующих шагов: 
     
\textit{Шаг 1.} Определение вероятностей $P_j$ того, что при поступлении в 
функциональную организационную структуру нового сообщения в этой структуре уже 
выполняется $j$~работ ($j = 0, 1, 2, \ldots$). Эти действия выполняются так же, как и при 
обосновании производительности трактов КТС: с по\-мощью теории массового 
обслуживания, например с по\-мощью формул~(\ref{e5-z}) либо, если заданы значения 
максимально допустимого, как правило, нормированного в регламенте организационной 
структуры, времени $T_{\mathrm{макс}}$ выполнения работ, с помощью аппарата теории 
вероятностей, например с помощью формул~(\ref{e8-z})--(\ref{e13-z}).

\textit{Шаг 2.} Выполнение последовательного суммирования вероятностей $P_j$ для 
значений $j \hm= 0, 1, 2, \ldots$\ \ до тех пор, пока выполняется условие
\begin{equation*}
\sum\limits_{j=0}^{M_{\mathrm{мин}}} P_j\leq P_{\mathrm{мин}}\,,
%\label{e13-1-z}
\end{equation*}
где $M_{\mathrm{мин}}$~--- минимально допустимое число субъектов, при котором 
выполняется требование к ве\-ро\-ят\-ности $P_{\mathrm{мин}}$ непревышения времени 
$T_{\mathrm{макс}}$, параметр $M_{\mathrm{мин}}$ определяется аналогично параметру 
$J_{\mathrm{доп}}$, например, как показано на рис.~7, при этом
\begin{equation*}
M_{\mathrm{мин}} T^1_{\mathrm{обсл}}\leq T_{\mathrm{макс}}\,,
%\label{e14-z}
\end{equation*}
где $T^1_{\mathrm{обсл}}$~--- время выполнения одной работы, приведенное к одному 
субъекту.

%\end{multicols}

\begin{figure*} %fig8
\vspace*{1pt}
\begin{center}
\mbox{%
\epsfxsize=121.081mm
\epsfbox{zac-8.eps}
}
\end{center}
\vspace*{-9pt}
\Caption{Пример модели накопления и использования технологической 
информации}
\end{figure*}

\begin{figure*}[b] %fig9
\vspace*{1pt}
\begin{center}
\mbox{%
\epsfxsize=162.26mm
\epsfbox{zac-9.eps}
}
\end{center}
\vspace*{-9pt}
\Caption{Примеры оценки влияния объема технологической информации: 
(\textit{а})~объем известной информации, (\textit{б})~длительность интервала 
обслуживания (переменная составляющая); $n$~--- порядковые номера обновлений}
\end{figure*}



%\begin{multicols}{2}
     
     Современным функциональным организационным структурам КТС присущи 
следующие свойства: формализуемость управленческой де\-я\-тель\-ности; автоматизация 
процессов накопления и\linebreak
 использования технологической информации о предметных 
сущностях для выполнения работ; их исчисляемость. На рис.~8 приведен пример модели 
накопления и использования информации.


Указанные свойства позволяют принять следующие гипотезы:
\begin{itemize}
\item существует прямая зависимость объема технологической информации от количества 
предметных сущностей;
\item существует прямая зависимость уровня знаний субъектов от объема накопленной и 
освоенной ими технологической информации;
\item существует обратная зависимость времени выполнения работы от уровня знаний, 
доступ к которым имеют субъекты; косвенно данное утверждение подтверждается 
результатами исследований уровня знаний у работников в об\-ласти информационных 
технологий и его влиянием на эффективность их деятельности~\cite{7-z}.
\end{itemize}

Данные гипотезы приводят к утверждению, что нормированные значения 
времени~$T_{\mathrm{макс}}$ выполнения работ могут со временем пересматриваться в 
сторону уменьшения в соответствии с определенными законами $A(t)$, свойственными 
различным КТС:
\begin{equation*}
T_{\mathrm{макс}}=T_0-\int\limits_0^\infty A(t)\,dt\,.
\label{e15-z}
\end{equation*}
Выбор варианта подготовки технологических данных для КТС~--- это самостоятельная 
задача, во многом определяемая условиями функционирования конкретного предприятия. 
Приведенные на рис.~9 примеры~\cite{8-z} позволяют приблизительно оценить эффект от 
накопления и использования технологических данных. 

\vspace*{-6pt}

\section{Заключение}

     Рассмотренный в статье методологический подход к управлению сложными 
инфокоммуникационными проектами относится к стадии замысла в жизненном цикле 
ИТКС и к реализации технического процесса по анализу требований. Использование на 
практике данного подхода к определению производительности трактов КТС и к 
определению состава их организационных структур для предприятий государственного 
сектора и коммерческих предприятий позволит исключить или значительно снизить 
инвестиционные риски при строительстве сис\-тем и последовательно повышать 
эффективность управления при их эксплуатации. 

\vspace*{-6pt}
    
{\small\frenchspacing
{%\baselineskip=10.8pt
\addcontentsline{toc}{section}{Литература}
\begin{thebibliography}{9}

\bibitem{1-z}
\Au{Зацаринный А.\,А.}
Основные принципы сис\-тем\-но\-го подхода при проектировании, внедрении и 
развитии современных корпоративных сетей~// Системы и средства
 информатики.~--- М.: Наука, 2002. Вып.~12. С.~58--66.

\bibitem{2-z}
\Au{Костогрызов А.\,И., Степанов П.\,В.}
Инновационное управление качеством и рисками в жизненном цикле сис\-тем.~--- М.: 
ВПК, 2008. 404~с.

\bibitem{3-z}
\Au{Зацаринный А.\,А., Ионенков Ю.\,С., Козлов~С.\,В.}
Некоторые вопросы проектирования информационно-те\-ле\-ком\-му\-ни\-ка\-ционных 
сис\-тем.~--- М.: ИПИ РАН, 2010. 218~с.

%\bibitem{4-z}
%\Au{Соколов И.\,А., Зацаринный~А.\,А., Печинкин~А.\,В., Антонов~С.\,В., 
%Лызлова~И.\,В.}
%Комплекс прог\-рам\-мно-ма\-те\-ма\-тических средств моделирования 
%ин\-фор\-ма\-ци\-он\-но-те\-ле\-ком\-му\-ни\-ка\-ци\-он\-ных сис\-тем. Работа выполнена по результатам 
%фундаментальных исследований (грант №\,05-07-90103).

\bibitem{5-z}
\Au{Саати Т.\,Л.}
Элементы теории массового обслуживания и ее приложения.~--- М.: Сов.\ 
радио, 1971.

\bibitem{6-z}
\Au{Шабанов А.\,П.}
О~распределении времени ожидания внутри интервала занятости входного 
группового тракта устройства коммутации цифровых каналов~// МРС ТТЭ. Сер.~О, 
1981. Вып.~11. Деп. в ВИМИ.

\bibitem{7-z}
\Au{Гусев А.\,В., Сурков С.\,А.}
Влияние информационных технологий на эффективность использования результатов 
биз\-нес-обра\-зо\-ва\-ния~// Применение новых технологий в образовании: XV 
Междунар. конф.~--- Троицк: Байтик, 2004. С.~53.

\label{end\stat}

\bibitem{8-z}
\Au{Аракелян М.\,А., Шабанов А.\,П.}
Технологические данные в ИТ-под\-держ\-ке бизнеса~// Директор информационной 
службы, 2007. №\,1.
 \end{thebibliography}
}
}


\end{multicols}       