\def\stat{abstr}
{%\hrule\par
%\vskip 7pt % 7pt
\raggedleft\Large \bf%\baselineskip=3.2ex
A\,B\,S\,T\,R\,A\,C\,T\,S \vskip 17pt
    \hrule
    \par
\vskip 21pt plus 6pt minus 3pt }

\label{st\stat}

%\def\rightmark{\ }

%1
\def\tit{STOCHASTIC INFORMATIONAL TECHNOLOGIES FOR~NONLINEAR
CIRCULAR STOCHASTIC SYSTEMS ANALYSIS}

\def\aut{I.\,N.~Sinitsyn}

\def\auf{IPI RAN, sinitsin@dol.ru}

\def\leftkol{\ } % ENGLISH ABSTRACTS}
\def\rightkol{\ } %ENGLISH ABSTRACTS}

\titele{\tit}{\aut}{\auf}{\leftkol}{\rightkol}

%\vspace*{-2pt}

\noindent
The paper is devoted to stochastic informational technologies (StIT) for analysis, 
analytical and statistical modeling of circular processes in circular nonlinear 
stochastic systems (StS) based on statistical linearization by wrapped normal distribution. 
For StIT, software tools ``CStS-ANALYSIS'' in MATLAB are developed and tested.

\vspace*{-5pt}

\KWN{analytical modeling; circular stochastic process; circular stochastic system; 
circular statistical linearization; computer aided support of statistical scientific research; 
MATLAB; correlation equations; spectral and correlation equations; 
stochastic informational technologies; statistical modeling}

%\thispagestyle{myheadings}


\vskip 12pt plus 6pt minus 3pt

%2
\def\tit{DISCRETE TIME QUEUEING SYSTEM WITH~UNRELIABLE  SERVER}

\def\aut{A.\,V.~Pechinkin$^1$ and I.\,A.~Sokolov$^2$}

\def\auf{$^1$IPI RAN, apechinkin@ipiran.ru\\[1pt]
$^2$IPI RAN, isokolov@ipiran.ru}


\def\leftkol{\ } % ENGLISH ABSTRACTS}

\def\rightkol{\ } %ENGLISH ABSTRACTS}

\titele{\tit}{\aut}{\auf}{\leftkol}{\rightkol}

\vspace*{-2pt}

\noindent
Consideration is given to the discrete time queueing system
$\mbox{Geo}/\mbox{G}/1/\infty$ with server subject to two types of breakdowns.
The server can break down either
when it is busy or when it is idle,
which happens with different probabilities.
Repair time distribution depends on the type of the breakdown.
Expressions for the stationary probability distribution and
other main stationary characteristics are given.
It is shown how the obtained results
can be used to find some stationary characteristics of
$\vec{\mbox{Geo}}_2/\vec{\mbox{G}}_2/1/\infty$
with two types of customers and preemptive priority.

\vspace*{-5pt}

\KWN{queueing system; discrete time; unreliable server; breakdowns}

\vskip 12pt plus 6pt minus 3pt


%3


\def\tit{ON A CLASS OF MARKOVIAN QUEUES}

\def\aut{Ya.\,A.~Satin$^1$, A.\,I.~Zeifman$^2$, A.\,V.~Korotysheva$^3$, and S.\,Ya.~Shorgin$^4$}

\def\auf{$^1$Vologda State Pedagogical University, yacovi@mail.ru\\[1pt]
$^2$Vologda State Pedagogical University;
IPI RAN; VSCC CEMI RAS, a\_zeifman@mail.ru\\[1pt]
$^3$Vologda State Pedagogical University, a\_korotysheva@mail.ru\\[1pt]
$^4$IPI RAN, SShorgin@ipiran.ru}


\def\leftkol{\ } % ENGLISH ABSTRACTS}

\def\rightkol{\ } %ENGLISH ABSTRACTS}

\titele{\tit}{\aut}{\auf}{\leftkol}{\rightkol}

\vspace*{-2pt}

\noindent
The nonstationary continuous-time Markovian queueing models are considered.  
Arrival and service rates are supposed to be independent on the length of the queue.  The 
bounds of the rate of convergence and stability for some characteristics of such 
systems are obtained.

\vspace*{-5pt}

\KWN{nonstationary Markovian queues; rate of convergence; stability;  bounds}


\pagebreak

 \vskip 12pt plus 6pt minus 3pt

\def\leftkol{\ } % ENGLISH ABSTRACTS}
\def\rightkol{\ } %ENGLISH ABSTRACTS}

 %4
\def\tit{ACTIVITY MAXIMA IN FREE-SCALE RANDOM NETWORKS WITH~HEAVY TAILS}

\def\aut{A.\,V.~Lebedev}

\def\auf{M.\,V.~Lomonosov Moscow State University, avlebed@yandex.ru}



\titele{\tit}{\aut}{\auf}{\leftkol}{\rightkol}


\noindent
The oriented power-law random graphs are considered as the models of information networks, 
where each node has a random information activity whose distribution has heavy 
(regularly varying) tail. The model of a random graph, in which incoming degrees of vertices 
are independent and have distribution with power tail, is used. 
Sufficient conditions have been got under which the maximum total activity (of the node and 
its incoming neighbors) increases asymptotically as well as the maximum of individual 
activities, and therefore, for them, Frechet limit law is hold.

\vspace*{-2pt}

\KWN{maxima; random sums; free-scale networks; power law; random graph; heavy tail; 
regular variation; Frechet distribution}

 \vskip 12pt plus 6pt minus 3pt

%5
\def\tit{ANALYTICAL MODEL FOR~CALCULATING THE~PERFORMANCE PLAN 
OF~DISTRIBUTING MULTIPROCESSOR SYSTEM RESOURCES IN~SOLVING THE~PROBLEMS OF~SPECIAL CLASS}

\def\aut{M.\,Ya.~Agalarov}

\def\auf{IPI RAN, murad-agalarov@yandex.ru}

%\def\leftkol{ENGLISH ABSTRACTS}
%\def\rightkol{ENGLISH ABSTRACTS}

\titele{\tit}{\aut}{\auf}{\leftkol}{\rightkol}

\vspace*{-2pt}

\def\leftkol{ENGLISH ABSTRACTS}

\def\rightkol{ENGLISH ABSTRACTS}

\noindent
A model of multiprocessor system designed to solve the tasks that are parallelized 
on weakly dependent calculations is considered. 
As a model, multiservice queuing system with apparent losses, 
Poisson incoming flow, and general service time distribution functions of tasks
is used. Recursion formulas for calculating the stationary probability distribution 
of states and the explicit expressions for the probability of system failure for 
different types of tasks are obtained. In this model, 
a method for assessing the capacity of a multiprocessor system for a given resource allocation 
static plan is suggested.

\vspace*{-2pt}

\KWN{multiprocessor system; queuing system; multiservice system; 
distribution of computing resources}

 \vskip 12pt plus 6pt minus 3pt
 


%6
\def\tit{DECONVOLUTION UNDER PARTIALLY KNOWN ERROR DISTRIBUTION}

\def\aut{V.\,G.~Ushakov$^1$ and N.\,G.~Ushakov$^2$}

\def\auf{$^1$Department of Mathematical Statistics, 
Faculty of Computational Mathematics and Cybernetics,\\ 
$\hphantom{^1}$M.\,V.~Lomonosov Moscow State University; 
IPI RAN, vgushakov@mail.ru\\[1pt]
$^2$Institute of Microelectronics Technology and High Purity Materials,
Russian Academy of Sciences,\\
$\hphantom{^1}$ushakov@math.ntnu.no}


\def\leftkol{ENGLISH ABSTRACTS}

\def\rightkol{ENGLISH ABSTRACTS}

\titele{\tit}{\aut}{\auf}{\leftkol}{\rightkol}

\vspace*{-2pt}

\noindent
The problem
of nonparametric estimation of a probability distribution is considered for a case when
the sample is contaminated by a random noise. It is supposed
that the distribution of the error is known only partially.
Identifiability and consistent estimation are investigated.

%\vspace*{-3pt}

\KWN{nonparametric estimation; deconvolution}
%\pagebreak

\vskip 12pt plus 6pt minus 3pt

%7
\def\tit{ANALYSIS AND OPTIMIZATION PROBLEMS FOR~SOME USERS ACTIVITY MODEL. 
PART~1.~ANALYSIS AND~PREDICTION}

\def\aut{A.\,V.~Bosov}

\def\auf{IPI RAN, AVBosov@ipiran.ru}


\def\leftkol{ENGLISH ABSTRACTS}

\def\rightkol{ENGLISH ABSTRACTS}

\titele{\tit}{\aut}{\auf}{\leftkol}{\rightkol}

%\vspace*{-2pt}

\noindent
A mathematical model describing the activity
of users, forming a query to some information system, is suggested. 
The properties of the
model are investigated, the procedures of prediction and identification of
equations parameters are described. The results of numerical
experiments, including real data processing, are presented.
 
%\vspace*{-2pt}

\KWN{information systems; modeling; ergodic random process; parametric
identification; prediction}

\pagebreak

\vskip 12pt plus 6pt minus 3pt

% \vskip 12pt plus 6pt minus 3pt

%8
\def\tit{SOLUTION OF DBMS INTERACTION PROBLEM IN~THE~CROSS-PLATFORM LIBRARY EFFIDB}


\def\aut{A.\,V.~Yanushko$^1$, A.\,V.~Babanin$^2$, O.\,A.~Kuznetsova$^3$,
and~S.\,V.~Petrushenko$^4$}

\def\auf{$^1$ASoft, yan@asoft.ru\\[1pt]
$^2$Russian Research Institute of Computer Science and Information, ababanin@pvti.ru\\[1pt]
$^3$ASoft, ok@asoft.ru\\[1pt]
$^4$ASoft, op@asoft.ru}


\def\leftkol{ENGLISH ABSTRACTS}

\def\rightkol{ENGLISH ABSTRACTS}

\titele{\tit}{\aut}{\auf}{\leftkol}{\rightkol}

%\vspace*{-2pt}

\noindent 
The article covers the problems of unified interaction 
with different database management systems in various 
software environments. It examines present-day solutions in this 
field and analyzes their advantages and drawbacks. The 
requirements to cross platform instrument for interaction 
between C++ application code and DBMS are listed and solution implemented 
as a dynamic library is suggested. The library provides specialized
classes for each of the concepts of relational databases: the actual database
connection, table, tools for data manipulation, etc.
The limits of applicability 
of the proposed solutions and the practice of using the library in real projects are also analyzed. 
Code case studies are provided as well.

%\vspace*{-2pt}

\KWN{DBMS, C++, connectivity library, cross-platform}

  \vskip 12pt plus 6pt minus 3pt
  
  %9
\def\tit{PROBABILISTIC STATISTICAL EVALUATION OF~THE~INFORMATION
OBJECTS ADEQUACY}

\def\aut{L.\,A. Kuznetsov}
\def\auf{Lipetsk State Technical University, kuznetsov@stu.lipetsk.ru}

\def\leftkol{ENGLISH ABSTRACTS}

\def\rightkol{ENGLISH ABSTRACTS}

\titele{\tit}{\aut}{\auf}{\leftkol}{\rightkol}

%\vspace*{-2pt}

\noindent
Mathematical basis and original methodology for developing evaluation systems of semantic
proximity of information objects (IO) in natural language are presented. 
A probabilistic statistical representation of the compared IO is
introduced. The information 
theory is used to estimate the semantic proximity of IO. The methodology can be 
used for synthesis of computer-based systems. The results of the practical testing of the 
methodology effectiveness are presented.


%\vspace*{-2pt}

\KWN{information objects; natural language; semantic adequacy; probabilistic model; information theory}


\vskip 12pt plus 6pt minus 3pt

%10
\def\tit{INFORMATION AND TELECOMMUNICATION PROJECTS MANAGEMENT: 
``TIMELINESS--PERFORMANCE--INFORMATION''}

\def\aut{A.\,A.~Zatsarinny$^1$ and A.\,P.~Shabanov$^2$}

\def\auf{$^1$IPI RAN, AZatsarinny@ipiran.ru\\[1pt]
$^2$IBS Expertiza Company, AShabanov@ibs.ru}

\titele{\tit}{\aut}{\auf}{\leftkol}{\rightkol}

%\vspace*{-2pt}

\noindent
Methodological approach to the management of information and telecommunication projects
is discussed.  Justification of performance requirements to the
paths of the technological systems and  to the number of subjects of the functional 
organizational structures, performing work in accordance with 
the messages taken from these tracts is examined.

%\vspace*{-2pt}

\KWN{project management; technology system; organizational structure; timeliness; 
performance; information}

\pagebreak

\vskip 12pt plus 6pt minus 3pt

%11
\def\tit{MODELING OF PROCESSES FOR~CREATION OF~EXPERT KNOWLEDGE 
FOR~MONITORING OF~GOAL-ORIENTED PROGRAMME ACTIVITIES}

\def\aut{I.\,M.~Zatsman$^1$ and A.\,A.~Durnovo$^2$}

\def\auf{$^1$IPI RAN, iz\_ipi@a170.ipi.ac.ru\\[1pt]
$^2$IPI RAN, duralex49@mail.ru}


\def\leftkol{ENGLISH ABSTRACTS}
\def\rightkol{ENGLISH ABSTRACTS}

\titele{\tit}{\aut}{\auf}{\leftkol}{\rightkol}

%\vspace*{-2pt}
\noindent
Statement of a problem of goal-oriented knowledge representation about 
indicators of monitoring is considered and its decision consisting of 
four components is offered: ($i$)~stationary model of computer representation 
of goal-oriented knowledge about indicators; ($ii$)~Frege space for computer representation; 
($iii$)~nonstationary model of computer representation; and ($i\nu$)~the 
proactive dictionary of a lingware of an evaluation system. The first three of  the
four components represent a theoretical part of the decision of this problem, 
and the fourth component represents an applied part of its decision.

%\vspace*{-2pt}

\KWN{problem of goal-oriented knowledge representation about indicators; 
semiotic models of computer representation of knowledge about indicators; 
indicator concepts; indicator denotata}


\vskip 12pt plus 6pt minus 3pt


%12
\def\tit{TRANSFORMATIONAL MODELS OF LANGUAGE STRUCTURES 
FOR~MACHINE TRANSLATION FROM~FRENCH INTO~RUSSIAN}

\def\aut{Yu.\,I.~Morozova}

\def\auf{IPI RAN, yulia-ipi@yandex.ru}


\def\leftkol{ENGLISH ABSTRACTS}

\def\rightkol{ENGLISH ABSTRACTS}

\titele{\tit}{\aut}{\auf}{\leftkol}{\rightkol}

%\vspace*{-2pt}
\noindent
The paper focuses on the problems of studying transformational properties of language 
objects in the process of translation of predicative structures from French into Russian. 
The paper studies the cases when  in the process of translation,
predicative words change their syntactic category or their grammatical features.
The texts of patents in French and their translations into Russian performed by professional translators were used as the material for the research. 


%\vspace*{-5pt}

\KWN{machine translation from French into Russian; 
functional semantics; language transformations; head-driven grammars}

\vskip 12pt plus 6pt minus 3pt


%13
\def\tit{STRATEGIES OF~SYNTACTIC ANALYSIS BASED~ON~HEAD-DRIVEN GRAMMARS AND~METHODS 
OF~THEIR IMPLEMENTATION IN~INFORMATION 
SYSTEMS}

\def\aut{E.\,B.~Kozerenko$^1$ and P.\,V.~Ermakov$^2$}

\def\auf{$^1$IPI RAN, kozerenko@mail.ru\\[1pt]
$^2$IPI RAN, petcazay@mail.ru}


\def\leftkol{ENGLISH ABSTRACTS}

\def\rightkol{ENGLISH ABSTRACTS}

\titele{\tit}{\aut}{\auf}{\leftkol}{\rightkol}

%\vspace*{-2pt}
\noindent
The problems of design and development of syntactic parsers 
in multilingual natural language processing systems, machine translation, and knowledge 
extraction from texts are considered.  The grammar formalisms and approaches to parsers 
construction 
are considered that take into account such challenges of translation as language 
transformations. An approach based on the hybrid grammar catching the functional 
parameters of language structures is proposed.

 \label{end\stat}

%\vspace*{-5pt}

\KWN{formal grammars; machine translation; syntactic analysis; statistical models; 
functional approach}




%\pagebreak

 