\documentclass[10pt]{book}
\usepackage[utf8]{inputenc}

\usepackage{latexsym,amssymb,amsfonts,amsmath,indentfirst,shapepar,%fleqn,%
picinpar,shadow,floatflt,enumerate,multicol,colortbl,ipi}

\usepackage{rotating}
\usepackage{mathrsfs}

\input{epsf}

%\nofiles

%\includeonly{avtor,avtor-eng}
%\includeonly{avtor-eng}
%\includeonly{pred}      %+pdf
%\includeonly{podgot-1str}  %+
%\includeonly{ocherk} %+

%\includeonly{sinit}   %1pdf
%\includeonly{pechinkin}   %2pdf
%\includeonly{zeifman}   %3pdf
%\includeonly{lebedev}   %4pdf
%\includeonly{agalarov}    %5pdf
%\includeonly{ushakov}    %6pdf
%\includeonly{bosov}    %7+pdf
%\includeonly{yanushko}    %8pdf
%\includeonly{kuzn}    %9pdf
%\includeonly{zatsar}   %10
%\includeonly{zatsman}   %11pdf
%\includeonly{morozova}   %12pdf
%\includeonly{kozerenko}   %13pdf

%\includeonly{toc-rus, toc-en}
%\includeonly{obchak,toc-en}

%\includeonly{obchak}
%\includeonly{reshal}
%\includeonly{eng-index}
%\includeonly{cover3}

\usepackage{acad}
%\usepackage{courier}
\usepackage{decor}
\usepackage{newton}
\usepackage{pragmatica}
\usepackage{zapfchan}
\usepackage{petrotex}
\usepackage{bm}                     % полужирные греческие буквы
\usepackage{upgreek}                % прямые греческие буквы
\usepackage{eufrak}
%\usepackage{verbatim}

\renewcommand{\bottomfraction}{0.99}
\renewcommand{\topfraction}{0.99}
\renewcommand{\textfraction}{0.01}

\setcounter{secnumdepth}{1} %здесь - 3 + chapter = 4

\arraycolsep=1.5pt

%\usepackage[pdftex]{graphicx}

%\usepackage{oz}

%NEW COMMANDS


\renewcommand*{\hm}[1]{#1\nobreak\discretionary{}%
            {\hbox{$\mathsurround=0pt #1$}}{}} %% Дублирует знаки операций
                               %при переносе в формуле (перед знаком, который 
                               %надо продублировать ставится команда \hm)


\renewcommand{\r}{\mathbb{R}}
\newcommand{\I}{{\rm I\hspace{-0.7mm}I}}
\newcommand{\Ik}{\mbox{{\small \tt {1}}\hspace{-1.5mm}{\tt 1}}}

\def\vrp{\varphi}
\def\prt{\partial}
\def\mm{{\rm M}}

\newcommand{\il}[2]{\int\limits_{#1}^{#2}}%интеграл с пределами #1 и #2

\def\sss{\sum\limits}
\def\tr{\,,\,\ldots\,,\,}
\def\rk{\,\right]}
\def\lk{\left[\,}
\def\rf{\right\}}
\def\lf{\left\{}

\def\ee{{\cal E}}
\def\ww{{\cal W}}
\def\yy{{\cal Y}}
\def\vv{{\cal V}}


\newcommand{\h}{{\bf H}}
\newcommand{\p}{{\sf P}}  % вероятность
\newcommand{\e}{{\sf E}}  % мат. ожидание
\newcommand{\D}{{\sf D}}  % дисперсия
\newcommand{\eps}{\varepsilon}
\newcommand{\vp}{{\mathbf p}}
\newcommand{\vz}{{\mathbf z}}
\newcommand{\vx}{{\mathbf x}}
\newcommand{\vf}{{\mathbf f}}
%\newcommand{\vp}{\mathrm{v.p.}}
\newcommand{\F}{{\mathcal F}}
\def\ap{{\mathrm{ЭР}}}

\newcommand{\abs}[1]{\left\vert#1\right\vert}
\def\w{\omega}
\def\W{\Omega}
\def\iii{\int\limits}
\def\iin{\int\limits_{-\infty}^\infty}

\DeclareMathOperator{\sign}{sign}

%\newcommand{\gr}{{\geqslant}}

\newcommand{\g}{\mbox{\textit{g}}}

\renewcommand{\la}{\lambda}
\newcommand{\si}{\sigma}
\newcommand{\alp}{\alpha}

%\newcommand{\pto}{\stackrel{P}{\longrightarrow}} % сходимость по веpоятности

%\newcommand{\eqd}{\stackrel{d}{=}} % равенство по pаспpеделению

%\newcommand{\kp}{\kappa}
%\def\Q{{\cal Q}} \def\H{{\cal H}}
%\newcommand{\bet}{\beta_{2+\delta}}


%\newtheorem{definition}{Определение}
%\renewcommand{\thedefinition}{\arabic{definition}.}
%END NEW COMMANDS

%\renewcommand{\baselinestretch}{1.2}

%\pagestyle{myheadings}

\setlength{\textwidth}{167mm}      % 122mm
\setlength{\textheight}{658pt}
%\setlength{\textheight}{635.6pt}
\setlength{\columnsep}{4.5mm}

\setcounter{secnumdepth}{4}

%\addtolength{\headheight}{2pt}
%\addtolength{\headsep}{-2mm}

%\addtolength{\topmargin}{-20mm}  % for printing


\hoffset=-30mm  % From Yap
%\hoffset=-20mm  % From Acrobat

%\voffset=0mm % From Yap
%\voffset=-15mm   % From Acrobat

\addtolength{\evensidemargin}{-9.5mm} % for printing
\addtolength{\oddsidemargin}{9.5mm}  % for printing

%\renewcommand{\thefootnote}{\fnsymbol{footnote}}
%\renewcommand{\thefootnote}{\arabic{footnote}}
\renewcommand{\figurename}{\protect\bf Рис.}
\renewcommand{\tablename}{\protect\bf Таблица}

\newcommand{\Caption}[1]{\caption{\protect\small %\baselineskip=2.5ex
#1}}

\renewcommand{\thefigure}{\arabic{figure}}
\renewcommand{\thetable}{\arabic{table}}
\renewcommand{\theequation}{\arabic{equation}}
\renewcommand{\thesection}{\arabic{section}}

\renewcommand{\contentsname}{СОДЕРЖАНИЕ}
\newcommand{\fr}[2]{\displaystyle\frac{\displaystyle #1\mathstrut}{\displaystyle #2\mathstrut}}

%\renewcommand{\thefootnote}{\fnsymbol{footnote}}
%\newcommand{\g}{\mbox{\textit{g}}}

%\newcommand{\Caption}[1]{\caption{\protect\small\baselineskip=2ex #1}}
\newcounter{razdel}
\setcounter{razdel}{0}


\newcommand{\titel}[4]{%
\

\vspace*{5pt}

\ifodd\therazdel {\raggedright\noindent\Large\textrm\textbf
 \lineskip .75em
  \baselineskip=3.2ex #1 \par}
\vskip 1em {\noindent\large\textrm\textbf #2 \par}
\addcontentsline{toc}{subsection}{{\textrm\textbf #3}\protect\newline #1}
\def\rightheadline{\underline{\noindent\hbox to \textwidth{\hfill\small\textrm{#4}
%\hfill \large\bf\thepage
}}}
\def\leftheadline{\underline{\noindent\parbox{\textwidth}{
%\raggedleft\large\bf\thepage \hfill
\small\textit{#3}\hfill}}}
\def\leftfootline{\small{\textbf{\thepage}
\hfill ИНФОРМАТИКА И ЕЁ ПРИМЕНЕНИЯ\ \ \ том~5\ \ \ выпуск 4\ \ \ 2011}
}%
 \def\rightfootline{\small{ИНФОРМАТИКА И ЕЁ ПРИМЕНЕНИЯ\ \ \ том~5\ \ \ выпуск~4\ \ \ 2011
\hfill \textbf{\thepage}}} 
\vskip 2em \setcounter{figure}{0}
\setcounter{table}{0} 
\setcounter{equation}{0} 
\setcounter{section}{0}
\setcounter{subsection}{0} 
\setcounter{subsubsection}{0}
\setcounter{footnote}{0} 
\setcounter{razdel}{0}
%\end{flushleft}
\else {
 \raggedright\noindent\Large\textrm\textbf
 \lineskip .75em
\baselineskip=3.2ex #1 \par} \vskip 1em
%\begin{flushleft}
{\noindent\large\textrm\textbf #2 \par}
\addcontentsline{toc}{subsection}{{\textrm\textbf #3}\protect\newline #1}
\def\rightheadline{\underline{\noindent\hbox to \textwidth{\hfill\small\textrm{#4}
%\hfill \large\bf\thepage
}}}
\def\leftheadline{\underline{\noindent\parbox{\textwidth}{%\raggedleft\large\bf\thepage \hfill
\small\textit{#3}\hfill}}}
\def\leftfootline{\small{\textbf{\thepage}
\hfill ИНФОРМАТИКА И ЕЁ ПРИМЕНЕНИЯ\ \ \ том~5\ \ \ выпуск~4\ \ \ 2011}
}%
 \def\rightfootline{\small{ИНФОРМАТИКА И ЕЁ ПРИМЕНЕНИЯ\ \ \ том~5\ \ \ выпуск~4\ \ \ 2011
\hfill \textbf{\thepage}}} \vskip 2em \setcounter{figure}{0}
\setcounter{table}{0} \setcounter{equation}{0} \setcounter{section}{0}
\setcounter{subsection}{0} \setcounter{subsubsection}{0}
\setcounter{footnote}{0}
%\end{flushleft}
\fi}

\newcommand{\titelr}[2]{%
\

\vspace*{5pt}

\ifodd\therazdel {\raggedright\noindent\large\textrm\textbf
 \lineskip .75em
  \baselineskip=3.2ex #1 \par}
\vskip 1em {\noindent\normalsize\textrm\textbf #2 \par}
\else {
 \raggedright\noindent\large\textrm\textbf
 \lineskip .75em
\baselineskip=3.2ex #1 \par} \vskip 1em
%\begin{flushleft}
{\noindent\normalsize\textrm\textbf #2 \par}
\fi}

\newcommand{\titele}[5]{%
\

%\vspace*{5pt}

\ifodd\therazdel {\raggedright\noindent%\large
\textrm\textbf
 \lineskip .75em
%  \baselineskip=3.2ex
#1 \par}
\vskip .5em {\noindent\large\textrm\textbf #2 \par}
\vskip .5em
 {\noindent\textrm #3 \par}
\addcontentsline{toc}{subsection}{{\textrm\textbf #1}\protect\newline #2}
\def\rightheadline{\underline{\noindent\hbox to \textwidth{\hfill\small\textrm{#4}
%\hfill \large\bf\thepage
}}}
\def\leftheadline{\underline{\noindent\parbox{\textwidth}{
%\raggedleft\large\bf\thepage \hfill
\small\textrm{#5}\hfill}}}
\def\leftfootline{\small{\textbf{\thepage}
\hfill ИНФОРМАТИКА И ЕЁ ПРИМЕНЕНИЯ\ \ \ том~5\ \ \ выпуск~4\ \ \ 2011}
}%
 \def\rightfootline{\small{ИНФОРМАТИКА И ЕЁ ПРИМЕНЕНИЯ\ \ \ том~5\ \ \ выпуск~4\ \ \ 2011
\hfill \textbf{\thepage}}} \vskip 1em \setcounter{figure}{0}
\setcounter{table}{0} \setcounter{equation}{0} \setcounter{section}{0}
\setcounter{subsection}{0} \setcounter{subsubsection}{0}
\setcounter{footnote}{0} \setcounter{razdel}{0}
%\end{flushleft}
\else {
 \raggedright\noindent%\large
 \textrm\textbf
 \lineskip .75em
%\baselineskip=3.2ex
#1 \par} \vskip .5em
%\begin{flushleft}
{\noindent\large\textrm\textbf #2 \par} \vskip .5em
 {\noindent\textrm #3 \par}
\addcontentsline{toc}{subsection}{{\textrm\textbf #1}\protect\newline #2}
\def\rightheadline{\underline{\noindent\hbox to \textwidth{\hfill\small\textrm{#4}
%\hfill \large\bf\thepage
}}}
\def\leftheadline{\underline{\noindent\parbox{\textwidth}{%\raggedleft\large\bf\thepage \hfill
\small\textrm{#5}\hfill}}}
\def\leftfootline{\small{\textbf{\thepage}
\hfill ИНФОРМАТИКА И ЕЁ ПРИМЕНЕНИЯ\ \ \ том~5\ \ \ выпуск~4\ \ \ 2011}
}%
 \def\rightfootline{\small{ИНФОРМАТИКА И ЕЁ ПРИМЕНЕНИЯ\ \ \ том~5\ \ \ выпуск~4\ \ \ 2011
\hfill \textbf{\thepage}}} \vskip 1em \setcounter{figure}{0}
\setcounter{table}{0} \setcounter{equation}{0} \setcounter{section}{0}
\setcounter{subsection}{0} \setcounter{subsubsection}{0}
\setcounter{footnote}{0}
%\end{flushleft}
\fi}

\def\Abst#1{
\begin{center}\small\nwt
\parbox{150mm}{%\baselineskip=2.5ex
\textbf{Аннотация:}\ \
%\hspace*{\parindent}
#1}
\end{center}}
\def\Abste#1{
\begin{center}\small\nwt
\parbox{150mm}{%\baselineskip=2.5ex
\textbf{Abstract:}\ \
%\hspace*{\parindent}
#1}
\end{center}}

\def\KW#1{
\begin{center}\small\nwt
\parbox{150mm}{%\baselineskip=2.5ex
\textbf{Ключевые слова:}\ \ #1}
\end{center}}

\def\KWE#1{
\begin{center}\small\nwt
\parbox{150mm}{%\baselineskip=2.5ex
\textbf{Keywords:}\ \ #1}
\end{center}}


\def\KWN#1{
%\begin{center}
%\small
%\parbox{150mm}\end{center}
}

\renewcommand{\thesubsection}{\thesection.\arabic{subsection}\hspace*{-5pt}}
\renewcommand{\thesubsubsection}{\thesubsection\hspace*{5pt}.\arabic{subsubsection}\hspace*{-3pt}}

\begin{document}
\Rus

\nwt
%\ptb

%\renewcommand{\contentsname}{\protect\Large\bf Содержание}

\setcounter{tocdepth}{2}

%\tableofcontents

\renewcommand{\bibname}{\protect\rmfamily Литература}
  \def\Au#1{{\it #1}}

%\newcommand{\No}{№}
  \newcommand{\tg}{\,\mathrm{tg}\,}
    \newcommand{\ctg}{\,\mathrm{ctg}\,}
  \newcommand{\arctg}{\,\mathrm{arctg}\,}
  
\def\forallb{\mathop{\forall}}
\def\existsb{\mathop{\exists}}

\setcounter{page}{1}

\newpage
\addtocounter{razdel}{1}
%\def\razd{РЕГУЛИРУЕМЫЙ ЭЛЕКТРОПРИВОД ДЛЯ ЭЛЕКТРОЭНЕРГЕТИКИ}
%\newpage
%\def\stat{zakh}
\def\tit{СРЕДСТВА ОБЕСПЕЧЕНИЯ ОТКАЗОУСТОЙЧИВОСТИ ПРИЛОЖЕНИЙ}
\def\titkol{Средства обеспечения отказоустойчивости приложений}

\def\aut{В.\,Н.~Захаров$^1$, В.\,А.~Козмидиади$^2$}
\titel{\razd}{\tit}{\aut}{\titkol}


\Abst{Рассмотрены проблемы построения отказоустойчивых серверов, возникающие в связи с недетерминированностью поведения приложений. Предложена формальная модель, описывающая поведение приложения, основными объектами которой являются ресурсы и события. Предложены алгоритмы протоколирования работы приложения на резервном узле кластера, а также восстановления и продолжения его работы при отказе основного узла. При этом для клиентов сбой остается незаметным, за исключением некоторого увеличения времени обслуживания.}

\KW{сервер приложений $\bullet$ прозрачная отказоустойчивость $\diamond$
 процесс $\diamond$ ресурс $\diamond$ событие $\diamond$ контрольная точка
$\bullet$ детерминированность}

\vskip 12pt plus 6pt minus 3pt

\begin{multicols}{2}

\section*{ВВЕДЕНИЕ}

Средства вычислительной техники стали использоваться в областях,
требующих безотказной работы систем в течение многих лет (24 часа
в сутки, 365 дней в году).

\label{st\stat}

\footnotetext{$^1$ФГУП Центральный институт авиационного моторостроения
им. П.И. Баранова, Москва, Россия}
\footnotetext{$^2$ФГУП Центральный институт авиационного моторостроения
им. П.И. Баранова, Москва, Россия}

К таким областям относятся, например, центры хранения и обработки данных  в сетях (системы резервирования билетов, биллинговые,  банковские и т.д.), массированные распределенные вычисления (GRID-вычисления) и другие.

\thispagestyle{headings}

Обычно в подобных системах применяются частные решения, ориентированные в основном на обеспечение надежного хранения данных (например, файловые серверы, использующие для хранения RAID-контроллеры) и корректного их состояния при отказах (серверы баз данных с транзакционным выполнением запросов). Однако большинство приложений не гарантируют, что не произойдет потери части данных при отказе системы. Обычно предполагается, что клиентские средства должны повторять запросы после восстановления серверов, для того, чтобы данные не были потеряны, или что можно сделать возврат по времени на некоторое время назад и повторить работу с этого места. Однако далеко не все клиентские средства и условия применения приложений допускают это.

Отказоустойчивые системы для критически важных приложений, корректно решающие проблемы восстановления после сбоев,   предлагаемые ведущими производителями, как правило, дороги. Кроме того, они включают специфические серверные и клиентские приложения, не совместимые со стандартными приложениями, не обеспечивающими отказоустойчивость. Примером такого подхода к решению проблемы отказоустойчивости  хранения данных являются системы NetApp FAS компании Network Appliance, работающие на базе специализированной операционной системы Data ONTAP [1].

Построение отказоустойчивых систем, использующих серверы со стандартными приложениями, в свете вышесказанного, является актуальной проблемой, вызывающей значительный интерес. Рассмотрение методов достижения прозрачной отказоустойчивости таких систем и является предметом статьи.
\begin{figure*} %fig1
\vspace*{1pt}
\begin{center}
\mbox{%
\epsfxsize=1.6in
\epsfxsize=100mm
\epsfbox{BbR-1.eps}
}
\end{center}
\vspace*{-9pt}
\Caption{Базовый вариант трубы с разными выходными устройствами
(цилиндрическое, расширяющееся и сужающееся сопло)
\label{f1bab}}
\vspace*{-3pt}
\end{figure*}


\section{ОСНОВНЫЕ ПОНЯТИЯ И ПОДХОДЫ}

Под сервером в данной работе понимается вычислительный центр
(отдельный компьютер или кластер) в сети, предоставляющий клиентам
(пользователям, клиентским компьютерам) определенные услуги, разделяя
между ними свои ресурсы. Подобные серверы названы серверами приложений.
Широко распространенным примером сервера такого типа является файловый сервер, обеспечивающий удаленный коллективный доступ к файловой системе. Часто используются вычислительные серверы, предоставляющие клиентам возможность выполнять на них свои программы (например, в центрах коллективного пользования).


Обычно приложение представляет собой программу или группу программ, работающих в операционной среде, создаваемой операционной системой (в другой терминологии - один или несколько взаимодействующих процессов или потоков (threads)), которые реализуют функциональность сервера. Для построения отказоустойчивых серверов приложений широко используется кластерная технология. Следуя [2], кластером, названа разновидность параллельной или распределенной системы, которая:
\begin{itemize}
\item состоит из нескольких компьютеров (узлов кластера), связанных как минимум необходимыми коммуникационными каналами;
\item используется как единый, унифицированный компьютерный ресурс.
\end{itemize}

Прозрачная отказоустойчивость (Transparent Fault Tolerance, TFT) сервера приложений - это такое его поведение при возникновении аппаратных или программных отказов либо отказов в сети, при котором:
\begin{itemize}
\item отказ не вызывает потери или искажения данных, находящихся в базе данных сервера;
\item сервер продолжает нормально функционировать, несмотря на имевшие место отказы.
\end{itemize}

Клиенты сервера "не замечают" произошедших отказов. Единственным\footnote{допустимым
отклонением сервера от нормального поведения с точки зрения клиента является
некоторое увеличение времени обслуживания} (на несколько секунд или десятков секунд).

Обычно приложения, работающие на серверах приложений, не ориентированы на прозрачную отказоустойчивость. Они "заботятся" лишь о собственной целостности (например, состояния файловой системы или базы данных). Восстановление работоспособности сервера приводит к разрыву соединений с клиентами и потере их запросов. Это замечают клиенты - запросы следует повторять, на что клиентские приложения далеко не всегда рассчитаны. В данной работе предполагается, что приложения (прикладные программные средства), выполняемые на сервере, являются стандартными, то есть не имеют специальных средств, обеспечивающих отказоустойчивость.
\begin{figure*}[b] %fig1
\vspace*{1pt}
\begin{center}
\mbox{%
\epsfxsize=1.6in
\epsfxsize=100mm
\epsfbox{BbR-1.eps}
}
\end{center}
\vspace*{-9pt}
\Caption{Базовый вариант трубы с разными выходными устройствами
(цилиндрическое, расширяющееся и сужающееся сопло)
\label{f1bab}}
\vspace*{-3pt}
\end{figure*}

Серьезные исследования в области обеспечения отказоустойчивости серверов были развернуты после создания вычислительных серверов, предназначенных для решения задач, требующих больших вычислительных ресурсов. Решение этих задач выполняется на суперкомпьютерах, обеспечивающих массово-параллельные вычисления и представляющих собой кластеры из сотен и тысяч узлов (процессоров). Однако даже на этих "монстрах" решение может требовать десятков или сотен часов, и одиночный сбой, если не предприняты специальные меры, может привести к необходимости начинать работу сначала. Обычно решение вычислительной задачи в таких случаях осуществляется в модели относительно редко взаимодействующих между собой процессов, выполняемых на разных узлах кластера. Эти взаимодействия нужны для координации работы процессов, в частности, для обмена данными и промежуточными результатами. Взаимодействия опираются на специальный протокол, называемый MPI (Message-Passing Interface) и представляющий собой стандарт "de facto" [3].

Для преодоления последствий сбоя достаточно давно была разработана и широко применяется технология, опирающаяся на механизм контрольных точек (checkpoints) [4-6]. По этой технологии система должна иметь стабильную память, которая не меняется при отказах. Соответствующие программные средства периодически сохраняют информацию о состоянии процессов приложения в стабильной памяти. Все процессы также имеют доступ к устройству стабильной памяти.  В случае отказа или сбоя, записанная в стабильную память информация используется для повторения вычисления с момента, когда была записана эта информация, то есть выполняется откат назад по времени. Данные, сохранение которых позволяет выполнить откат, называются контрольной точкой. В качестве устройства стабильной памяти может использоваться дисковый том, энергонезависимая оперативная память, память другого узла или узлов кластера. В последнем случае узел, которому требуется сохранить информацию, пересылает ее через быстрый канал связи на другой узел. Стабильная память после отказа одного из узлов должна быть доступной узлу, на котором делается повтор.

Однако решение, опирающееся только на контрольные точки, не является прозрачным, поскольку не скрывает от клиентов факт отказа системы и требует от них выполнения определенных действий. Так как при работе процессы обмениваются сообщениями, возможны два варианта решения проблемы. Первый - все процессы выполняют записи контрольных точек одновременно, что затруднительно. Второй вариант, при несоблюдении синхронности, - возврат в каждом процессе к такому скоординированному набору контрольных точек, при котором невозможна противоречивая ситуация. Такая ситуация возникает, когда один процесс вернулся к контрольной точке, после которой он должен получить сообщение от другого процесса, а этот другой процесс вернулся к точке, которая следует за выдачей этого сообщения. Однако при повторе ожидаемое первым процессом сообщение не поступит. В этом случае  возможен эффект домино, в результате процессы оказываются отброшены как угодно далеко назад.

В этом состоит первая проблема, которую необходимо преодолеть.

Если нужно, чтобы последствия отказа узла не были видны клиенту,  это означает:
\begin{itemize}
\item клиент не должен терять и потом восстанавливать соединения с сервером;
\item клиент не должен повторять свои запросы;
\item клиент не должен повторно получать сообщения, которые он уже получил.
\end{itemize}

Вторая проблема, которую надо решать, связана с недетерминированностью поведения сервера приложений. Приведем пример.  Пусть имеется система продажи билетов на самолеты. Два клиента одновременно обратились к системе с запросом билета на один и тот же рейс. Клиентам безразлично, какие места им зарезервирует система. Система выполняет запросы клиентов параллельно, поэтому в какой-то момент между процессами, обрабатывающими эти запросы, может возникнуть конкуренция за ресурс - в данном случае, скажем, рейс. Один из процессов захватывает ресурс первым, резервирует место и освобождает ресурс. Потом второй процесс проделывает то же самое.

Порядок, в котором в этом примере процессы захватили ресурс, зависит от многих факторов и, в конечном счете, случаен. Однако  это не мешает правильному функционированию системы, поскольку клиентам важно одно - получить билеты, причем на разные места. Однако отсутствие детерминизма в поведении приложения приводит к тому, что при повторном выполнении могут быть получены другие результаты: например, клиенту уже сообщено, что ему зарезервировано место №5, а при повторе может получиться, что зарезервировано место №6. Система должна устранить это несоответствие и сделать его невидимым для клиента.
\begin{figure*} %fig1
\vspace*{1pt}
\begin{center}
\mbox{%
\epsfxsize=1.6in
\epsfxsize=100mm
\epsfbox{BbR-1.eps}
}
\end{center}
\vspace*{-9pt}
\Caption{Базовый вариант трубы с разными выходными устройствами
(цилиндрическое, расширяющееся и сужающееся сопло)
\label{f1bab}}
\vspace*{-3pt}
\end{figure*}

Недетерминированность поведения системы это следствие, по крайней мере, двух обстоятельств. Во-первых, это присущая системам с разделением времени неопределенность в порядке выполнения процессов. Во-вторых, это конкуренция процессов за общие ресурсы. Перечислим некоторые причины недетерминированного поведения приложений:
\begin{itemize}
\item синхронизация процессов с помощью семафоров или атомарных операций над операндами в общей памяти процессов;
\item зависимость от порядка получения клиентских запросов;
\item время, затраченное процессом на обработку полученного запроса;
\item генераторы случайных чисел;
\item системное управление процессами и потоками;
\item локальные таймеры;
\item доступ к реальному времени.
\end{itemize}

По различным  причинам время, которое тратится на выполнение вычислительной задачи с одними и теми же исходными данными, не является константой, то есть повторное выполнение может дать другое время. Процессы используют общие ресурсы, обращение к которым требует организации очередности выполнения (сериализации) - первым пришел, первым захватил. И, наконец,  результат работы процесса может зависеть от состояния ресурса, а это состояние может изменить другой процесс, ранее захвативший ресурс. Все это создает значительные трудности при попытках воспроизведения поведения процессов с сохраненной контрольной точки.

Прозрачная отказоустойчивость серверов приложений обычно осуществляется переносом приложения на другой узел кластера, идентичный первому по конфигурации аппаратных средств и операционной среды. Это делается методом, называемым snapshot/restore. На основном узле (оригинале)  периодически фиксируется состояние приложения на этом узле кластера (так называемый снимок или snapshot). После отказа оригинала на резервном узле (копии) делается восстановление (restore), то есть восстанавливается последнее зафиксированное состояние приложения. Операционная среда при этом приводится в состояние, которое соответствует моменту изготовления снимка. После этого узел-копия продолжает работу с зафиксированного места. Сравнение метода  snapshot/restore с другими подходами приведено в [7].

Ниже рассматриваются информационные  технологии, позволяющие решить ряд принципиальных вопросов, связанных с реализацией прозрачной отказоустойчивости серверов приложений. Ими являются:
\begin{itemize}
\item виртуализация операционной среды, в которой работает серверное приложение;
\item отказоустойчивая реализация протокола TCP;
\item создание контрольных точек состояния приложения и файловой системы, которые делаются внешним по отношению к приложению образом;
\item восстановление серверного приложения на основании контрольной точки.
\end{itemize}
\begin{figure*} %fig1
\vspace*{1pt}
\begin{center}
\mbox{%
\epsfxsize=1.6in
\epsfxsize=100mm
\epsfbox{BbR-1.eps}
}
\end{center}
\vspace*{-9pt}
\Caption{Базовый вариант трубы с разными выходными устройствами
(цилиндрическое, расширяющееся и сужающееся сопло)
\label{f1bab}}
\vspace*{-3pt}
\end{figure*}

\section{МОДЕЛЬ ОПИСАНИЯ ПОВЕДЕНИЯ ПРИЛОЖЕНИЯ}

Предлагаемый подход опирается на построение модели вычислений, связанной с использованием понятия времени в многопроцессных приложениях. Впервые подобные проблемы были изучены в классической работе Л. Лампорта [8].

Многопроцессными приложения называются потому, что в них параллельно работают несколько процессов. Процесс ведет себя детерминированно, пока в предписанном кодом порядке выполняет процессорные инструкции. Конечно, его работа может быть прервана практически в любой момент и процессор передан другому процессу или ядру. Поэтому абсолютное время, которое затрачивает процесс на выполнение определенной работы, не  константа, а случайная  величина. То же  относится к относительному времени, то есть времени, которое процесс занимал процессор,  поскольку одни и те же обращения к операционной среде могут вызвать работы разной длительности, а значит потребовать разное время на свое выполнение.

Кэшированность инструкций и данных, а также длина хэш-списков влияют на действительное время пребывания в операционной среде. Утрачивает смысл понятие одновременность действий, поскольку  нельзя установить, выполнили ли два разных процесса какие-либо действия одновременно или одно из них предшествовало другому. Таким образом, с процессом можно связать только его локальное время, которое линейно упорядочивает события,  происходившие в этом процессе.  Глобальное время, линейно упорядочивающее действия во всех процессах, отсутствует. Расстояние (в этом качестве используется время) между действиями оказывается случайной величиной.

Эти соображения важны, поскольку процессы в интересующих нас приложениях взаимодействуют и используют общие ресурсы. Для взаимодействия они используют средства синхронизации, предоставляемые операционной средой - например, наборы семафоров SVR4 (System V Release 4), POSIX-семафоры, бинарные семафоры и другие примитивы взаимного исключения (POSIX- mutual exclusion locks) и т.д. Подобные средства операционной среды, которые позволяют процессам синхронизировать свою деятельность друг с другом или сериализовать обращения к совместно используемым объектам,  будут ниже  называться ресурсами.

С каждым ресурсом связано свое локальное время, линейно упорядочивающее события в жизни ресурса. Например, в случае двоичных семафоров это создание семафора, а также его захват и освобождение процессом. Заметим, что событие - это не намерение процесса (например, захватить бинарный семафор), а сам факт захвата семафора процессом (т.е. успешное выполнение намерения). От изъявления намерения до его осуществления может многое произойти. Например, семафор, который хочет захватить рассматриваемый процесс, принадлежал другому процессу, потом тот процесс его освободил, но семафор был сначала передан операционной средой третьему процессу, который также на него претендовал, и т.д. Поведение рассматриваемого процесса в это время нас не интересует - он ресурсом еще не овладел, а только его захват определяет его дальнейшее поведение. По причинам,  изложенным выше, расстояние между двумя событиями - случайная величина. Однако, события замечательны тем, что они одновременно присутствуют и в локальном времени процесса, и в локальном времени ресурса. Поэтому все, что произошло в истории процесса или/и ресурса до этого события, предшествует ему. Далее  будет считаться, что истории и ресурсов и процессов состоят только из событий, причем между двумя последовательными событиями в жизни процесса последний ведет себя детерминированно.

Это означает, что на  поведении процесса сказывается только его предыдущая история, то есть состояние ресурсов, с которыми он взаимодействовал. Это свойство процессов ниже будет называться локальной детерминированностью. Этим свойством не обладают ресурсы, поскольку - следующее событие в истории ресурса не определяется однозначно по его предыдущей истории. Утверждение, касающееся детерминированного поведения процессов, неявно опирается на предположение,  что учтены все ресурсы, которые могут привести к  недетерминированности процессов.

Таким образом, описанное нами очень неформально время в многопроцессном комплексе представляет собой отношение частичного порядка, введенное на множестве событий. Зная полное состояние комплекса в некоторый момент времени,  нельзя однозначно определить, какое событие в истории ресурса наступит следующим. Можно говорить только о вероятности наступления того или иного события. Недетерминированность поведения есть следствие двух обстоятельств. Во-первых, это неопределенность времени, которое тратит процесс на переход от одного события к другому. Во-вторых, конкуренция процессов за общие ресурсы.

Выполнение приложения, на множестве событий которого введена частичная упорядоченность, можно описать направленным ациклическим графом выполнения. Вершинами этого графа являются события, с каждым  из которых связаны две входящие в него дуги. Одна дуга начинается в событии, которое непосредственно предшествует данному событию в истории процесса, другая - в истории ресурса.

Построение средств обеспечения прозрачной отказоустойчивости приложений опирается на следующее утверждение: для восстановления работы приложения после отказа достаточно располагать:
\begin{itemize}
\item контрольной точкой, которая отражает на некоторый момент времени состояния процессов и других ресурсов, образующих приложение;
\item графом выполнения приложения, который описывает работу приложения, начинающуюся с контрольной точки и заканчивающуюся отказом. Данные, которые нужны для построения графа выполнения, далее называются протоколом.
\end{itemize}
\begin{figure*} %fig1
\vspace*{1pt}
\begin{center}
\mbox{%
\epsfxsize=1.6in
\epsfxsize=100mm
\epsfbox{BbR-1.eps}
}
\end{center}
\vspace*{-9pt}
\Caption{Базовый вариант трубы с разными выходными устройствами
(цилиндрическое, расширяющееся и сужающееся сопло)
\label{f1bab}}
\vspace*{-3pt}
\end{figure*}

Вся эта информация должна находиться в стабильной памяти, не разрушающейся при отказе.

Ниже неформально описан алгоритм восстановления работы приложения после отказа, который опирается на наличие контрольной точки и графа выполнения. Будем считать, что в распоряжении имеются средства, позволяющие остановить процесс в тот момент, когда он намерен совершить некоторую операцию над ресурсом. Заметим, что событие в графе выполнения соответствует не изъявлению намерения, а его удовлетворению, то есть завершению выполнения операции.

Предварительно сделаем следующее:
\begin{itemize}
\item используя контрольную точку, приведем приложение в состояние, соответствующее этой контрольной точке;
\item в графе выполнения пометим все вершины (события) как "не наступившие". У некоторых вершин графа отсутствуют им непосредственно предшествующие; соответствующие события наступили сразу же после создания контрольной точки. Для каждой такой вершины включим в граф дополнительную вершину, ей предшествующую в истории процесса, и отметим эту дополнительную вершину как "наступившую";
\item разрешим процессам приложения выполняться.
\end{itemize}

Пусть некоторый процесс проявляет намерение выполнить операцию над каким-либо ресурсом. Отыщем для этого процесса в его истории последнее наступившее событие. Следующее в его истории событие - это то, которое соответствует требуемой операции. Посмотрим, наступило ли событие в истории ресурса, которое ему предшествует. Если нет, переведем процесс в состояния ожидания, отметив в предшествующем событии, что данный процесс ожидает его наступления. Если да, разрешим процессу выполняться, то есть выполнить операцию над ресурсом.

Пусть некоторый процесс объявляет, что он выполнил операцию над каким-либо ресурсом (это соответствует моменту протоколирования при оригинальном выполнении). Отыщем для этого процесса в его истории последнее наступившее событие и перейдем к следующему событию в его истории. Это опять то событие, которое мы рассматриваем. Отметим его как "наступившее". Если наступления этого события ожидал какой-нибудь процесс, выведем этот процесс из состояния ожидания. Наконец, разрешим процессу, выполнившему операцию, продолжаться дальше.

Когда выясняется, что наступили все события графа выполнения, повторное выполнение считается законченным.

Естественным следствием из сказанного является следующее утверждение: для того, чтобы размер протокола не рос неограниченно, нужно периодически создавать контрольные точки, очищая при этом протокол.

\section{ФОРМАЛЬНОЕ ОПИСАНИЕ МОДЕЛИ ПОВЕДЕНИЯ МНОГОПРОЦЕССНОГО ПРИЛОЖЕНИЯ}
\begin{figure*} %fig1
\vspace*{1pt}
\begin{center}
\mbox{%
\epsfxsize=1.6in
\epsfxsize=100mm
\epsfbox{BbR-1.eps}
}
\end{center}
\vspace*{-9pt}
\Caption{Базовый вариант трубы с разными выходными устройствами
(цилиндрическое, расширяющееся и сужающееся сопло)
\label{f1bab}}
\vspace*{-3pt}
\end{figure*}

Опишем формально поведение приложения, неформальное описание которого было приведено выше. Рассматриваются два типа объектов:
\begin{itemize}
\item ресурсы (r), например, наборы семафоров (POSIX- или SVR4-семафоры), бинарные семафоры (POSIX-mutex's), таймер реального времени, сокеты (sockets), то есть двусторонние виртуальные соединения с внешним миром;
\item процессы (p), например, процессы или потоки (threads) пользователя.
\end{itemize}

\end{multicols}

\label{end\stat}

%\def\stat{batr}

\def\tit{НОВЫЙ МЕТОД ВЕРОЯТНОСТНО-СТАТИСТИЧЕСКОГО\newline
АНАЛИЗА ИНФОРМАЦИОННЫХ ПОТОКОВ
В~ТЕЛЕКОММУНИКАЦИОННЫХ СЕТЯХ$^*$}
\def\titkol{Новый метод вероятностно-статистического
анализа информационных потоков
в~телекоммуникационных сетях}
\def\autkol{Д.\,А.~Батракова, В.\,Ю.~Королев, С.\,Я.~Шоргин}
\def\aut{Д.\,А.~Батракова$^1$, В.\,Ю.~Королев$^2$, С.\,Я.~Шоргин$^3$}

\titel{\tit}{\aut}{\autkol}{\titkol}

{\renewcommand{\thefootnote}{\fnsymbol{footnote}}\footnotetext[1]{Работа 
выполнена при поддержке РФФИ, проекты №№\,04-01-00671, 05-07-90103.} 
\renewcommand{\thefootnote}{\arabic{footnote}}}
 \footnotetext[1]{ИПИ РАН, 
daria.batrakova@gmail.com} \footnotetext[2]{Факультет вычислительной математики 
и кибернетики МГУ им.~М.\,В.~Ломоносова, ИПИ РАН, vkorolev@comtv.ru} 
\footnotetext[3]{ИПИ РАН, sshorgin@ipiran.ru}



\Abst{В данной работе предлагается метод исследования стохастической структуры
хаотических информационных потоков в сложных телекоммуникационных
сетях. Предлагаемый метод основан на стохастической модели
телекоммуникационной сети, в рамках которой она представляется в виде
суперпозиции некоторых простых последовательно-параллельных структур.
Эта модель естественно порождает смеси гамма-распределений для времени
выполнения (обработки) запроса сетью. Параметры получаемой смеси
гамма-распределений характеризуют стохастическую структуру
информационных потоков в сети. Для решения задачи статистического
оценивания параметров смесей экспоненциальных и гамма-распределений
(задачи разделения смесей) используется ЕМ-алгоритм. Чтобы проследить
изменение стохастической структуры информационных потоков во времени,
ЕМ-алгоритм применяется в режиме скользящего окна. Описывается
программный инструментарий для применения полученного решения к
реальным статистическим данным. Приводится интерпретация результатов.}

\KW{телекоммуникационные сети; информационные потоки;
разделение смесей  распределений;
метод скользящего окна;  программа для разделения смесей}

\vskip 24pt plus 9pt minus 6pt

\thispagestyle{headings}

\begin{multicols}{2}


\label{st\stat}

\section{Введение}

Развитие телекоммуникационных сетей, их усложнение поставило перед
инженерами важную прикладную задачу исследования характеристик
информационных потоков, возникающих в этих сетях. Здесь под
информационным потоком мы будем понимать упорядоченное движение
любого вида информации по сети.

Если на заре эры телекоммуникаций, в эпоху первых телефонных линий и
телеграфа эта проблема не была столь насущной, то со временем, при
постепенном охвате мирового пространства сетями возникла необходимость в
построении и исследовании адекватных моделей сетей и процессов,
происходящих в них.

\thispagestyle{headings}


Современные сети~--- это \textit{конвергентные} сети, т.е.\ совокупность крайне
разнородных как по топологии, так и по физической архитектуре сетей, которые
предлагают конечному пользователю самые разнообразные сервисы. Это~--- огромное
виртуальное и физическое пространство, состоящее из миллионов процессоров,
операционных платформ, линий передачи данных и стыковочных узлов.
%
Существует множество классификаций телекоммуникационных сетей по различным
признакам:
\begin{itemize}
\item масштабу (локальные сети~--- LAN, масштаба города~---
MAN, широкого масштаба~--- WAN);
\item топологии, или логической организации (<<звезда>>,
<<кольцо>>, <<шина>>);
\item физической организации (оптические сети, радио);
\item предлагаемым услугам (сотовые сети, для доступа в
Интернет);
\item назначению (военные, гражданские) и~др.
\end{itemize}


Конвергентная сеть входит во все эти классы, причем, как правило,
обладает всеми этими признаками. Поэтому построение модели для ее анализа
является и очень важной, и очень сложной задачей.

Существуют достаточно многочисленные математические методы, ориентированные на
моделирование и анализ телекоммуникационных сетей. В~большинстве своем они
основываются на теории массового обслуживания, то есть разделе теории
вероятностей, посвященном описанию функционирования сложных систем обслуживания
(в том чис\-ле телекоммуникационных сетей и систем) с помощью стохастических
процессов особого вида и анализу таких процессов. Указанная теория является
весьма развитой и широко применяется на практике. Тем не менее, ее применимость
ограничена~--- во-первых, все возрастающей сложностью структур и дисциплин
обслуживания в рас\-смат\-ри\-ва\-емых реальных сетях. Эта сложность во многих
случаях принципиально не может найти адекватного отображения в моделях
массового обслуживания, даже несмотря на постоянно растущую сложность самих
этих моделей. В результате даже модели, допускающие точный математический
анализ, дают возможность расчета всего лишь приближенных значений характеристик
реальных сетей, ибо предположения, принимаемые при построении моделей, во
многих случаях не соответствуют практике. Во-вторых, для описания
телекоммуникационной сети в виде сети массового обслуживания исследователь
должен располагать детальным описанием структуры сети, что далеко не всегда
имеет мес\-то на практике. В-третьих, разработано крайне мало моделей массового
обслуживания, в которых используется в качестве входной информация о
наблюдаемых (статистических) показателях функционирования сети; в то же время,
такая информация очень часто доступна исследователю, и ее использование при
анализе сети весьма целесообразно.

В данной работе предлагается в определенной степени альтернативный подход к
моделированию сложных телекоммуникационных сетей. Строится и исследуется
вероятностная модель сложной телекоммуникационной сети как суперпозиции
достаточно простых структур. При этом практически никакая априорная информация
о структуре исследуемой сети не используется~--- наоборот, в результате
исследования модели исследователь получает приближенное представление об этой
структуре. Характеристики типовых простых структур, составляющих в совокупности
модель сети, и сети в целом при этом подходе описываются
гам\-ма-рас\-пре\-де\-ле\-ни\-я\-ми. Ставится задача оценки параметров модели
на основе статистических данных о функционировании сети, а также предлагается
математическое решение этой задачи. В статье описан также созданный на основе
разработанных математических методов программный инструментарий и приведены
результаты расчетов для реального трафика. {\looseness=-1

}

\section{Математическая модель и~постановка задачи}

\subsection{Логическая модель сети}
 %1.1

Рассмотрим абстрактную \textit{конвергентную телекоммуникационную
сеть}. Это может быть как крупномасштабная транспортная сеть (WAN), сеть
отдельного оператора масштаба города (MAN) с различными сервисами, так и
локальная сеть (LAN).

Любой из этих случаев можно описать как ($p,\,q$)-\textit{сеть}.

\medskip
\textbf{Определение 1.} В теории графов и сетей под ($p,\,q)$-сетью понимается
набор вида $S =$\linebreak $=(G,\,V^\prime ,\,V^{\prime\prime})$, где $G$~---
граф, а $V^\prime$ и $V^{\prime\prime}$~--- выборки из множества $V(G)$ (вершин
графа) длины~$p$ и $q$ соответственно. При этом выборка $V^\prime$
($V^{\prime\prime}$) считается \textit{входной} (\textit{выходной}) выборкой, а
ее $i$-я вершина называется $i$-\textit{м} \textit{входным} (\textit{выходным})
\textit{полюсом} или, иначе, $i$-\textit{м} \textit{входом} (\textit{выходом})
сети~$S$. Вершины, не участвующие во входной и выходной выборках сети,
считаются ее внутренними вершинами~\cite{1bat}.

Сеть $S$ (рис.~\ref{f1bat}) имеет $p$ точек входа~--- точек соединения
с внешней средой (это могут быть точки стыковки разнородных сетей, сетей
различных операторов, физические подключения к интерфейсам
маршрутизаторов и~т.п.). Под \textit{внешней средой} будем понимать другие
сети, которые передают данные в сеть~$S$. Отдельные <<единицы>> данных
(кадры, сообщения, датаграммы, пакеты) поступают на входы сети~$S$,
обрабатываются и подаются на каждый из $q$ выходов, которые могут быть
соединены как с конечными пользователями, так и с другими сетями.
\begin{figure*} %fig1
\vspace*{1pt}
\begin{center}
\mbox{%
\epsfxsize=139.7mm \epsfbox{bat-1.eps}
%\epsfxsize=139.698mm
%\epsfbox{bek-3.eps}
}
\end{center}
\vspace*{-9pt} \Caption{Абстрактная телекоммуникационная сеть \label{f1bat}}
\end{figure*}

Следует отметить, что структура сложных телекоммуникационных сетей обладает
свойством некоторого самоподобия, т.е.\ на каком бы уровне сетевой архитектуры
мы ни рассматривали поведение информационных потоков, мы можем выделить
некоторые базовые структуры, субпотоки, суперпозицией которых мы можем получить
модель конкретной сети, какой бы уровень <<детализации>> сегментов сети мы ни
взяли. Так, например, физические подключения к интерфейсам оптического
кросс-коннекта в этом смысле подобны <<виртуальным>> подключениям к портам TCP
на сервере приложений.

Итак, независимо от уровня сетевой архитектуры мы можем
рассматривать некоторую величину, характеризующую количество каких-либо
ресурсов сети~$S$, занимаемых в процессе передачи и обработки данных.  Это
могут быть ресурсы, относящиеся как к <<объему>> (памяти сетевого
устройства, количеству занятых линий, размеру пакета), так и ко <<времени>>
(времени обслуживания заявки, времени простоя). Эта величина случайна, т.к.\
мы не можем абсолютно точно сказать для сложной телекоммуникационной
сети, какое сообщение на какой из входов поступит и какого размера оно будет.
Таким образом, случайный характер данной величины определяется
случайностью поведения внешней среды.

Пусть $R$~--- это описанная выше случайная величина, $R>0$. Далее, не
ограничивая общности, будем подразумевать под ней время, необходимое для
какой-либо операции сети (обработки <<единицы>> данных), предполагая, что
время обработки прямо зависит от объема сообщения.

\subsection{Вероятностная модель сети} %1.2.

Даже не зная реальной топологии сети, мы можем описать
функционирование некоторых ее участков как процесс выполнения операций
(задач сети) в последовательном  порядке (например, если доступен только
один канал данных) или как процесс одновременного выполнения субопераций
(когда доступно более одного пути выполнения). Это значит, что мы можем
представить функционирование сложной телекоммуникационной сети как
\textit{суперпозицию} таких <<последовательных>> и <<параллельных>>
блоков.

Для построения вероятностной модели распределения~$R$ используется
комбинация асимптотического подхода, основанного на предельных теоремах
теории вероятностей, и принципа максимальной неопределенности (энтропии).

Рассмотрим следующую модель. Предположим, что мы можем разделить
сеть~$S$ на несколько сегментов $S_i$. Пусть $T$~--- случайная величина,
время выполнения операции в отдельно взятом блоке $S_i$ (сегменте сети).

Если операции выполняются \textit{параллельно}, то время, необходимое
для их выполнения~--- это максимальное время, затрачиваемое на какую-либо
субоперацию:
$$
T = \underset{i}{\max}\, T_i\,,
$$
где $T_i$~--- случайные величины для со\-от\-вет\-ст\-ву\-ющих субопераций.
Модель такого выполнения пред\-став\-ле\-на на рис.~\ref{f2bat}.

\begin{figure*} %fig2
\vspace*{1pt}
\begin{center}
\mbox{%
\epsfxsize=117.271mm
\epsfbox{bat-2.eps}
}
\end{center}
\vspace*{-9pt}
\Caption{Параллельное выполнение
\label{f2bat}}
\end{figure*}

Известно, что предельное распределение экстремальных значений для
выборок ~--- это экспоненциальное распределение с плотностью~\cite{2bat}
$$
f(x) =
\begin{cases}
\lambda e^{-\lambda x}\,, & x>0\,,\\
0\,, & x\leq 0\,,
\end{cases}
$$
где $\lambda >0$~--- параметр масштаба.

 Учитывая также энтропийный подход, естественно будет считать
распределение $T$ экспоненциальным, т.к.\ экспоненциальное распределение
обладает наибольшей энтропией среди всех распределений с $x>0$.

Если же операции сети выполняются \textit{последовательно}, то величина
$T$~--- это сумма времен $T_i$, необходимых для выполнения каждой
субоперации:
$$
T = \sum\limits_i T_i\,,
$$
где $T_i$~--- случайные величины для со\-от\-вет\-ст\-ву\-ющих субопераций.
%
Такая модель представлена на рис.~\ref{f3bat}.

\begin{figure*} %fig3
\vspace*{1pt}
\begin{center}
\mbox{%
\epsfxsize=139.592mm
\epsfbox{bat-3.eps}
}
\end{center}
\vspace*{-9pt}
\Caption{Последовательное  выполнение
\label{f3bat}}
\end{figure*}

Это значит, что распределение общей длительности $T$ выполнения
блока~--- это свертка распределений <<элементарных>> времен $T_i$
(экспоненциально распределенных).

Известно, что результатом свертки экспоненциальных распределений
является гамма-распределение, определяемое плотностью
$$
\g_{\lambda , \alpha} (x) =
\begin{cases}
\fr{\lambda_0^{\alpha_0}}{\Gamma (\alpha_0 )}\,x^{\alpha_0-1}
e^{\lambda_0 x}\,, & x>0\,,\\
0,\, & x\leq 0\,,
\end{cases}
$$
где $\alpha >0$~--- параметр формы,  $\lambda >0$  параметр масштаба, а
$\Gamma (z)$~--- гамма-функция Эйлера:
$$
\Gamma (z) = \int\limits_0^\infty x^{z-1} e^{-x}\,dx\,.
$$

\begin{figure*} %fig4
\vspace*{1pt}
\begin{center}
\mbox{%
\epsfxsize=120.831mm
\epsfbox{bat-4.eps}
}
\end{center}
\vspace*{-9pt}
\Caption{Модель пути  обработки сообщения сетью~$S$
\label{f4bat}}
\end{figure*}

Известно~\cite{2bat}, что класс гамма-распределений замкнут над операцией
свертки, поэтому ре\-зуль\-ти\-ру\-ющее распределение случайной величины~$R$
будет также гамма-распределением
$$
\g_{\lambda , \alpha} (x) =
\begin{cases}
\fr{\lambda^{\alpha}}{\Gamma (\alpha )}\,x^{\alpha -1} e^{-\lambda x}\,, &
x>0\,,\\
0,\, & x\leq 0\,.
\end{cases}
$$

В силу случайного характера ввода данных в сеть~$S$ из внешней среды маршрут
передачи данных становится случайным, что представлено на рис.~\ref{f4bat}. Это
означает, что параметры ре\-зуль\-ти\-ру\-юще\-го распределения~$R$ тоже
случайны. Отсюда имеем следующую модель \textit{смеси
гам\-ма-рас\-пре\-де\-ле\-ний}, определяемой плотностью

\begin{equation} %1
p(x) = \iint \g_{\lambda , \alpha}(x)\,dH (\lambda ,\,\alpha )\,,
\end{equation}
где $H(\lambda , \alpha )$~--- смешивающая функция, функция распределения
входных параметров.

Поясним понятие \textit{смеси распределений}.

\medskip
\textbf{Определение~2.} Пусть имеется двух\-па\-ра\-мет\-ри\-че\-ское
семейство $n$-мерных плотностей  распределения
\begin{equation}
F = \{ f_\omega (x;\, \theta (\omega ))\}\,,
\end{equation}
где одномерный (целочисленный или непрерывный) параметр $\omega$ в
качестве нижнего индекса функции $f$ определяет специфику общего вида
каж\-до\-го компонента~--- распределения смеси, а в качестве аргумента при
многомерном, вообще говоря, параметре $\theta$ определяет зависимость
значений хотя бы части компонентов этого параметра от того, в каком именно
составляющем распределении $f_\omega$ он присутствует. Кроме того, пусть
$P = \{P(\omega )\}$~--- \textit{семейство смешивающих функций}
распределения.

Функция плотности распределения
\begin{equation}
f(x) = \int f_\omega (x;\,\theta(\omega ))\,dP (\omega )
\end{equation}
называется $P$-\textit{смесью} (или просто \textit{смесью})
\textit{распределений} семейства~$F$,  интеграл в~(3) понимается в смысле
Лебега--Стильтьеса~\cite{3bat}.

\medskip
\textbf{Определение 3.} Семейство смесей~(3) называется
\textit{идентифицируемым} (\textit{различимым}), если из равенства
$$
\int f_\omega (x;\,\theta(\omega ))\,dP (\omega ) =\int f_\omega
(x,\,\theta(\omega )) dP^*(\omega )
$$
следует, что $P(\omega ) \equiv P^*(\omega )$ для всех $P \in P(\omega
)$~\cite{3bat}.

\subsection{Постановка задачи} %1.3.

Перед нами встает задача \textit{разделения} такой смеси. Вообще говоря,
задача разделения смесей распределений со смешивающими функциями
общего вида является \textit{некорректно поставленной}, т.к.\ она допускает
существование нескольких решений. Поэтому будем искать решение в классе
\textit{конечных идентифицируемых смесей распределений}, где смешивающая
функция дискретна.

Для этого сузим данное выше определение и будем рассматривать в дальнейшем лишь 
случай конечного числа $k$ возможных значений па\-ра\-мет\-ра~$\omega$, что 
соответствует конечному числу скачков смешивающих функций $P(\omega )$.  
Величины этих скачков как раз и будут играть роль \textit{удельных весов} 
(\textit{априорных вероятностей}) $p_j$ компонентов смеси. Более того, в нашем 
случае мы постулируем также однотипность компонентов распределений $f_j$, т.е.\ 
принадлежность всех $f_j$ к одному общему па\-ра\-мет\-ри\-че\-ско\-му 
семейству $\{ f(X;\,\theta )\}$, где $\theta$~--- многомерный, вообще говоря, 
параметр. Так что~(3) в этом случае может быть записано в виде
\begin{equation} %4
p(x) = \sum\limits^k_{j=1} p_j f_j (x;\,\theta_j )\,.
\end{equation}

Переформулируем понятие идентифицируемости (различимости) смесей
специально применительно к такому виду смесей.

\medskip
\textbf{Определение 4.} \textit{Конечная смесь}~(3) называется
\textit{идентифицируемой} (\textit{различимой}), если из равенства
$$
\sum\limits_{j=1}^k p_j f_j (x;\,\theta_j ) = \sum\limits_{l=1}^{k^*} p_l^* f_l
(x;\,\theta_l^* )
$$
следует, что $k=k^*$ и для любого $j$ ($1\leq j \leq k$) найдется такое $l$ 
($1\leq l \leq k^*$), что $p_j = p_l^*$ и $f_j (x;\,\theta_j ) = f_l 
(x;\,\theta_l^* )$~\cite{3bat}.

Решить эту задачу в выборочном варианте~--- значит по выборке
классифицируемых наблюдений
$X_1,\ldots , X_n, $ извлеченной из генеральной совокупности, яв\-ля\-ющей\-ся смесью~(3)
генеральных совокупностей типа~(2) (при заданном общем виде составляющих
смесь функций $f_j (x;\,\theta_j )$), построить статистические оценки для числа
компонентов смеси~$k$, их удельных весов $p_j$ и, главное, для каждого из
компонентов %f_j (x;\,\theta_j )$ анализируемой смеси. Далее будет считать, что
функции $f_j$ однозначно определяются своими параметрами $\theta_j$: $f_j
=f(x;\,\theta_j)$.

Однако не следует ставить знак тождества между задачей разделения смеси
и задачей статистического оценивания параметров в модели~(4) по выборке $
X_1,\ldots , X_n$, поскольку задача разделения сохраняет смысл и
применительно к генеральным совокупностям, т.е.\ в теоретическом
варианте~\cite{3bat}.

Итак, для статистического анализа на основе реальных данных мы
аппроксимируем нашу общую модель~(1) следующей:
$$
p(x) \approx \hat{p}(x) = \sum\limits_{j=1}^k p_j \g_{\lambda_j , \alpha_j}
(x)\,,
$$
где $p_j$~--- дискретные смешивающие параметры, $\g_{\lambda_j , \alpha_j}
(x)$~--- плотности гамма-распределений.

Такая аппроксимация не только позволяет решить поставленную статистическую
задачу, но и полу\-чить наглядную визуализацию результатов. Существуют
достаточно эффективные методики разделения смесей распределений, среди них~---
семейство так называемых \textit{ЕМ-алгоритмов}
(\textit{Expectation-Maximization Algorithms}).

Полученные результаты могут быть достаточно просто интерпретированы. Число
компонентов смеси символизирует число типичных параллельных или
последовательных структур. Значения параметров составляющих смесь
гам\-ма-рас\-пре\-де\-ле\-ний показывают <<степень параллелизма>>
соответствующей структуры: чем ближе параметр формы к~1, тем выше эта
<<степень>>. И наоборот, чем дальше значение параметра формы от~1, тем больше
последовательных операций выполняется в соответствующем блоке.

Веса компонентов характеризуют примерную долю использования
ресурсов для сообщений, соответствующих каждому распределению входных
данных.

Итак, предложенный подход позволяет получить представление о
стохастической структуре телекоммуникационной сети.

\section{ЕМ-алгоритм разделения смесей распределений}

\subsection{Описание алгоритма} %2.1.

Определяемый ниже итерационный алгоритм решения поставленной в
предыдущем разделе задачи относится к процедурам, базирующимся на
\textit{методе максимального правдоподобия}.

Этот алгоритм позволяет находить максимум логарифмической функции
правдоподобия по параметрам $p_1,\,p_2,\ldots ,\,p_k$, $\theta_1 ,\,\theta_2,\ldots ,\,
\theta_k$ при фиксированном $k$ по выборке $X_1, \ldots , X_n$, т.е.\ решение
оптимизационной задачи вида

\begin{equation} \sum\limits_{i=1}^n \ln \left ( \sum\limits_{j=1}^k p_j f_j
(X_i;\,\theta_j )\right ) \rightarrow \underset{p_j,\,\theta_j}{\max}\,.
\end{equation}

Конкретные алгоритмы, построенные по этой схеме, часто называют
\textit{алгоритмами типа ЕМ}, поскольку в каждом из них можно выделить два
этапа, находящихся по отношению друг к другу в последовательности
итерационного взаимодействия: \textit{оценивание} (\textit{Estimation}) и
\textit{максимизация} (\textit{Maximization})~\cite{4bat}.

Введем в рассмотрение так называемые апостериорные вероятности
$\g_{ij}$ принадлежности наблюдения $X_i$ к $j$-му классу:
\begin{equation} %6
\g_{ij} = \fr{p_j f(X_i;\,\theta_j )}{\sum\limits_{l=1}^k p_l f(X_i;\,\theta_l 
)} \ (i=1,\ldots , n;\ j=1,\ldots ,k)\,.\!\!\end{equation} 
Очевидно, что для 
всех $i=1,\ldots ,n$ и $j=1,\ldots ,k$
$$
\g_{ij} \geq 0,\quad \sum_{j=1}^k \g_{ij} =1\,.
$$


Далее обозначим $\Theta = (p_1,\ldots p_k,\,\theta_1,\ldots ,\theta_k )$ и
представим анализируемую логарифмическую функцию правдоподобия
$$
\ln L(\Theta ) = \sum\limits_{i=1}^n \ln \left (\sum\limits_{j=1}^k p_j f_j
(X_i;\,\theta_j )\right )
$$
в виде
\begin{multline}
\ln L (\Theta ) = \sum\limits_{j=1}^k\sum\limits_{i=1}^n \g_{ij} \ln p_j+{}\\
{}+\sum\limits_{j=1}^k\sum\limits_{i=1}^n \g_{ij} f(X_i;\,\theta_j)-
\sum\limits_{j=1}^k\sum\limits_{i=1}^n \g_{ij} \ln \g_{ij}\,.
\end{multline}

Справедливость этого тождества легко проверяется с учетом
$$
\sum\limits_{j=1}^k \g_{ij} =1\,.
$$

Далее идея построения итерационного алгоритма вычисления оценок
$\hat{\Theta} = (\hat{p}_1,\ldots , \hat{p}_k,\
\hat{\theta}_1,\ldots , \hat{\theta}_k)$
для параметров $\Theta = (p_1,\ldots , p_k,\ \theta_1,\ldots , \theta_k)$ состоит в
следующем:
\begin{enumerate}[1.]
\item Выбирается некоторое \textit{начальное приближение}~$\hat{\Theta}^0$.
\item \textbf{E-step:} вычисляются по формулам~(6) начальные приближения
$\g_{ij}^0$ для апостериорных вероятностей $\g_{ij}$~--- \textit{этап
оценивания}.
\item \textbf{M-step:} затем, возвращаясь к~(7), при вычисленных значениях
$\g^0_{ij}$ следует определить значения $\hat{\Theta}^1$ из условия
максимизации отдельно каждого из первых двух слагаемых правой
части~(7), поскольку первое слагаемое
$$
\sum_{j=1}^k \sum_{i=1}^n \g_{ij} \ln p_j
$$
зависит только от параметров $p_j$, а второе слагаемое
$$
\sum_{j=1}^k \sum_{i=1}^n \g_{ij} f(X_i;\,\theta_j )
$$
зависит только от параметров $\theta_j$~--- \textit{этап максимизации}.
\item Проверяется \textit{условие останова}:
$$
\parallel \Theta^{(t)} - \Theta^{t-1}\parallel <\varepsilon\,,
$$
где $t$~--- номер итерации, а
$\parallel\bullet\parallel$~--- евклидова норма, для некоторого $\varepsilon
>0$.
\end{enumerate}

Очевидно, решение оптимизационной задачи
$$
\sum\limits_{j=1}^k\sum\limits_{i=1}^n \g_{ij}^{(t)}\ln p_j \rightarrow
\underset{p_j}{\max}
$$
дается выражением (с учетом $\sum_{j=1}^k p_j =1$):
$$
p_{ij}^{(t+1)} =\fr{1}{n}\,\sum\limits_{i=1}^n \g_{ij}^{(t)}\,,
$$
где $t$~--- номер итерации, $t = 0$, 1, 2,\,\ldots

Решение оптимизационной задачи
$$
\sum\limits_{j=1}^k \sum\limits_{i=1}^n \g_{ij}^{(t)} f(X_i;\,\theta_j )
\rightarrow \underset{\theta_j}{\max}
$$
получить намного проще решения задачи~(5): выражение для $\theta_j$
записывается с учетом знания конкретного вида функций
$f(X,\,\theta)$~\cite{3bat}.

\subsection{О сходимости алгоритма} %2.2.

В работе М.\,И.~Шлезингера~\cite{5bat}, где эта схема (позднее названная
ЕМ-схемой) впервые предложена, установлены и основные свойства
реа\-ли\-зу\-ющих ее алгоритмов. В частности, было доказано, что при достаточно
широких предположениях \textit{предельные точки} всякой последовательности,
порожденной итерациями ЕМ-алгоритма, являются стационарными точками
оптимизируемой логарифмической функции правдоподобия $\ln L(\Theta )$ и что
найдется неподвижная точка алгоритма, к которой будет сходиться каждая из таких
последовательностей. Если дополнительно потребовать положительной
определенности информационной мат\-ри\-цы Фишера для $\ln L(\Theta )$ при
истинных зна\-че\-ни\-ях па\-ра\-мет\-ра $\Theta$, то можно показать, что
асимптотически по $n\rightarrow\infty$ (т.е.\ при больших выборках) существует
единственное сходящееся (по веро\-ят\-но\-сти) решение $\hat{\Theta}(n)$
уравнений метода максимального правдоподобия и, кроме того, существует в
пространстве параметров $\Theta$ норма, в которой последовательность
$\Theta^{(t)}(n)$, порожденная ЕМ-ал\-го\-рит\-мом, сходится к $\hat{\Theta}
(n)$, если только начальная аппроксимация $\hat{\Theta}^0$ не была слишком
далека от~$\hat{\Theta} (n)$. {%\looseness=1

}

Таким образом, результаты исследования свойств ЕМ-алгоритмов метода
максимального правдоподобия разделения смеси и их практическое
использование показали, что они являются достаточно работоспособными (при
известном чис\-ле компонентов смеси) даже при большом чис\-ле $k$ компонентов и
при высоких размерностях анализируемого признака~$X$~\cite{3bat}.

\subsection{Уравнения для смеси экспоненциальных распределений}
%2.3.

Применим описанный выше алгоритм к разделению смеси
экспоненциальных распределений:
$$
p(x) = \sum\limits_{j=1}^k p_j \lambda_j e^{-\lambda_j x}\,.
$$
Получаем следующие итерационные уравнения:
\begin{align*}
\g_{ij}^{(t+1)} & = \fr{p_j^{(t)}\lambda_j^{(t)}e^{-
\lambda_j^{(t)}X_i}}{\sum\limits_{l=1}^k p_l^{(t)}\lambda_l^{(t)}
e^{-\lambda_l^{(t)}X_i}}\,,\\
p_j^{(t+1)} & = \fr{1}{n}\,\sum\limits_{i=1}^n \g_{ij}^{(t)}\,.
\end{align*}

Чтобы найти  оценки $\lambda_j$, подсчитаем первую производную функции
$$\sum_{j=1}^k\sum_{i=1}^n \g_{ij}^{(t)} \ln (\lambda_j e^{-\lambda_j X_i}):$$
\vspace*{-8pt}
\begin{multline*}
\left ( \sum\limits_{j=1}^k \sum\limits_{i=1}^n
\g_{ij}^{(t)}\ln \left ( \lambda_j
e^{-\lambda_j X_i} \right ) \right )^\prime \lambda_j =\\[-3pt]
{}= \left (
\sum\limits_{j=1}^k\sum\limits_{i=1}^n \g_{ij}^{(t)}\ln (\lambda_j -\lambda_j X_i )
\right )^\prime \lambda_j =\\[-3pt]
{}= \sum\limits_{i=1}^n \g_{ij}^{(t)}\left (
\fr{1}{\lambda_j} - X_i \right )\,.
\end{multline*}

Разрешая уравнение
$$
\sum\limits_{i=1}^n \g_{ij}^{(t)}\left ( \fr{1}{\lambda_j} -X_i\right ) =0
$$
относительно $\lambda_j$, получаем следующее итерационное уравнение:
$$
\lambda_j^{(t+1)} = \fr{\sum\limits_{i=1}^n
\g_{ij}^{(t)}}{\sum\limits_{i=1}^n \g_{ij}^{(t)} X_i}\,.
$$

\subsection{Уравнения для смеси гамма-распределений } %2.4.

Применим теперь ЕМ-алгоритм к смеси гам\-ма-рас\-пре\-де\-ле\-ний вида
$$
p(x) = \sum\limits_{j=1}^k p_j \fr{\alpha_j^{\alpha_j} x^{\alpha_j -
1}}{\lambda_j^{\alpha_j} \Gamma (\alpha_j )}\,e^{-(\alpha_j / \lambda_j)x}\,.
$$

Такая параметризация удобна для нахождения
оценок~$\alpha_j$~\cite{6bat}.

Аналогичным способом выписываются итерационные уравнения:
\begin{align*}
\g_{ij}^{(t+1)} & =   \fr{p_j^{(t)}\fr{(\alpha_j^{\alpha_j} )^{(t)}
x^{\alpha_j - 1}}{(\lambda_j^{\alpha_j} )^{(t)}\Gamma (\alpha_j)}\,
e^{-(\alpha_j /\gamma_j)^{(t)}x}}{\sum\limits_{l=1}^k
p_l^{(t)}\fr{(\alpha_l^{\alpha_l})^{(t)} x^{\alpha_l -
1}}{(\lambda_l^{\alpha_l})^{(t)}\Gamma (\alpha_l )}\,
e^{-(\alpha_l /\lambda_l)^{(t)} x}}\,,\\
p_j^{(t+1)} & = \fr{1}{n}\,\sum\limits_{i=1}^n \g_{ij}^{(t)}\,.
\end{align*}

Далее найдем оценки $\lambda_j$ для данного случая, приравнивая
производную
\begin{equation} %8
\sum\limits_{j=1}^k \sum\limits_{i=1}^n \g_{ij}^{(t)} \ln \left (
\fr{\alpha_j^{\alpha_j} x^{\alpha_j -1}}{\lambda_j^{\alpha_j}\Gamma
(\alpha_j)}\,e^{-(\alpha_j /\lambda_j) x}\right )
\end{equation}
по $\lambda_j$ к нулю и разрешая относительно~$\lambda_j$ уравнение:
$$
\sum\limits_{i=1}^n \g_{ij}^{(t+1)}\left ( \fr{\alpha_j^{(t)}}{\lambda_j^{(t)}}
- \fr{\alpha_j^{(t)}X_i}{\left ( \lambda_j^{(t)}\right )^2}\right ) =0 \,.
$$
Получаем
$$
\lambda_j^{(t+1)} = \fr{\sum\limits_{i=1}^n \g_{ij}^{(t)}
X_i}{\sum\limits_{i=1}^n \g_{ij}^{(t)}}\,.
$$

Для того чтобы получить итерационные уравнения для $\alpha_j$, найдем
первую производную~(8):
\begin{multline*}
\left ( \sum\limits_{j=1}^k\sum\limits_{i=1}^n \g_{ij}^{(t)}\ln \left (
\fr{\alpha_j^{\alpha_j} x^{\alpha_j -1}}{\lambda_j^{\alpha_j}\Gamma (\alpha_j
)}\,e^{-(\alpha_j /\lambda_j ) x} \right ) \right )^\prime \alpha_j ={}\\[-3pt]
{}=\left ( \sum\limits_{j=1}^k\sum\limits_{i=1}^n \g_{ij}^{(t)}\ln \left (
\fr{\alpha_j^{\alpha_j}}{\lambda_j^{\alpha_j}}\right ) - \ln \Gamma (\alpha_j )+{} \right.\\[-3pt]
{}+\left.
(\alpha_j -1 )\ln X_i - \fr{\alpha_j}{\lambda_j}\,X_i \right )^\prime \alpha_j =\\[-3pt]
{}=\sum\limits_{i=1}^n \g_{ij}^{(t)} \left ( \ln \alpha_j +1-\ln \lambda_j -
\fr{\Gamma^\prime (\alpha_j )}{\Gamma (\alpha_j)}\right.+\\[-3pt]
{}+\left. \ln X_i - \fr{X_i}{\lambda_j}\right )\,;
\end{multline*}
\begin{multline*}
\sum\limits_{i=1}^n \g_{ij}^{(t)} \left(  \ln \alpha_j +1 -\ln \lambda_j -{}\right. \\[-3pt]
\left. {}-\fr{\Gamma^\prime (\alpha_j )}{\Gamma (\alpha_j )}+\ln X_i 
-\fr{X_i}{\lambda_j} \right) =0\,;
\end{multline*}
\begin{multline}
\fr{\Gamma^\prime (\alpha_j )}{\Gamma (\alpha_j )} ={}\\[-3pt]
{}= \fr{\sum\limits_{i=1}^n \g_{ij}^{(t)} \left ( \ln \alpha_j +1-\ln\lambda_j 
+\ln X_i -\fr{X_i}{\lambda_j} \right )}{\sum\limits_{i=1}^n \g_{ij}^{(t)}}\,.
\end{multline}
%
Здесь $\Gamma^\prime (\alpha_j ) / \Gamma (\alpha_j )$~--- это
\textit{логарифмическая производная гамма-функции}. Для нее существует так
называемое \textit{разложение Абрамовитца}--\textit{Стигана}~\cite{4bat}:
$$
\fr{\Gamma^\prime (\alpha ) }{ \Gamma (\alpha )} = \mathrm{log}\,\alpha -
\fr{1}{2\alpha }-\fr{1}{12\alpha^2 }+\ldots
$$

Подставим первые три члена разложения в~(9) и разрешим это уравнение
относительно~$\alpha_j$:
$$
\alpha_{ij}^{(t+1)} = \fr{\sum\limits_{i=1}^n
\g_{ij}^{(t+1)}}{2\sum\limits_{i=1}^n \g_{ij}^{(t +1)}\left ( \fr{X_i}{\lambda_j^{(t)}} -
\ln \fr{X_i}{\lambda_j^{(t)}} -1\right )}\,.
$$
В итоге получаем итерационные уравнения для ~$\alpha_j$.

\section{Описание программного обеспечения (программа~ЕМ)}

\subsection{Назначение программы} %3.1.

Разработанная авторами статьи программа ЕМ предназначена для решения задачи
разделения смесей экспоненциальных и гамма-распределений, поставленной в
разд.~2, с использованием ЕМ-ал\-го\-рит\-ма и формул, описанных в разд.~3.

\subsection{Инструменты разработки} %3.2.

Для создания программы была использована среда разработки Microsoft
Visual Studio .NET 2005 и объектно-ориентированный язык C\#. Для
визуализации результатов была использована свободно распространяемая
графическая библиотека ZedGraph~\cite{7bat}.


\subsection{Возможности  программы} %3.3.

\noindent
\begin{itemize}
\item Загрузка выборочных данных из текстового файла
\item Оценивание по выборке параметров смеси экспоненциальных
распределений
\item Оценивание по выборке параметров смеси гамма-распределений
\item Отслеживание изменений параметров смесей распределений во
времени в режиме <<скользящего окна>>
\item Построение гистограммы по выборке
\end{itemize}

\subsection{Входные и выходные данные. Функционирование
программы} %3.4.

В качестве \textit{входных данных} программа ЕМ получает:
\begin{itemize}
\item выборочные данные из текстового файла;
\item число компонентов смеси;
\item размер <<скользящего окна>>;
\item размер класса гистограммы.
\end{itemize}

На \textit{выходе} мы получаем:
\begin{itemize}
\item точечные оценки параметров смеси экспоненциальных
распределений;
\item точечные оценки параметров смеси гамма-распределений;
\item графическое изображение результирующей смеси распределения;
\item графическое изображение компонентов каж\-дой смеси;
\item графическое изображение того, как меняются параметры смесей
распределений с течением времени в режиме <<скользящего окна>>;
\item гистограмма, построенная по выборке;
\item значение статистического теста.
\end{itemize}

Выборочные данные загружаются из текстового файла в память программы и подаются
на вход двум независимо работающим реализациям ЕМ-алгоритма~--- для
идентификации смеси экспоненциальных распределений и для идентификации смеси
гамма-распределений. Результатом их работы являются наборы значений оцениваемых
параметров модели, предложенной в разд.~2. Кроме того, результирующие
распределения визуализируются в виде графиков. В программе можно запустить
режим <<скользящего окна>>, который для всех подвыборок заданного
размера с помощью ЕМ-алгоритма оценивает параметры смесей распределений этих
подвыборок. Все действия программы документируются в окне информации.

\section{Описание тестовых расчетов}

С использованием разработанной программы были проведены тестовые
расчеты на выборочных данных реального сетевого трафика.

На вход программы EM были поданы выборки трафика:
\begin{enumerate}[I]
\item Между лабораторией Lawrence Berkeley (Berkeley, California) и
внешним миром размера примерно 7000~\cite{8bat}~--- \textit{выборка~1}.
\item
Сети радиодоступа ЗАО <<Синтерра>> размера примерно 1000~\cite{9bat}~---
 \textit{выборка~2}.
\end{enumerate}

\subsection{Выборка 1 ``Berkeley''} %5.1.

При числе компонентов смеси~5 и случайном начальном приближении
были получены результаты, представленные в табл.~\ref{t1bat}.


Данные результаты иллюстрирует рис.~\ref{f5bat}.

Гистограмма  на рис.~\ref{f6bat} показывает статистическую значимость
полученных результатов.

Данная выборка обладает той особенностью, что она собиралась в течение
достаточно длительного времени и в ней агрегирован самый разнородный
трафик. Поэтому в ней присутствует не только большое количество
<<коротких>> сообщений (что обычно для выборок из телетрафика), но и
некоторый массив сообщений средней длины, а также определенный
<<выброс>> больших сообщений. Это свидетельствует о \textit{пиковости}
телетрафика на довольно больших промежутках времени.

Как мы видим, ЕМ-алгоритм удачно справился с задачей идентификации
смеси.

\subsection{Выборка~2 ``Synterra''} %5.2.

Результаты применения ЕМ-алгоритма к выборке ``Synterra''
представлены в табл.~\ref{t2bat}.
\begin{table*}\small
\begin{minipage}[t]{76mm}
\begin{center}
\Caption{Результаты применения ЕМ-алго\-рит\-ма к выборке~1 ``Berkeley'' 
\label{t1bat}} \vspace*{2ex}

\tabcolsep=8.7pt
\begin{tabular}{|c|c|c|}
\hline
№&Начальное приближение&Результат\\
\hline
\multicolumn{3}{|c|}{$P$}\\
\hline
0&0,2&0,1896\\
1&0,2&0,1858\\
2&0,2&0,1830\\
3&0,2&0,2259\\
4&0,2&0,2154\\
\hline
\multicolumn{3}{|c|}{$\alpha$}\\
\hline
0&2,7028&10,9783\hphantom{9}\\
1&3,6273&5,8621 \\
2&5,7598&2,7092\\
3&0,2315&1,0235\\
4&0,9110&0,4772\\
\hline
\multicolumn{3}{|c|}{$\lambda$}\\
\hline
0&85,2066&137,1714  \\
1&23,9592&136,7349\\
2&63,8425&132,6482\\
3&\hphantom{9}1,8026&116,7317\\
4&98,3882&102,5278\\
\hline
\end{tabular}
\end{center}
\end{minipage}\hfill
\begin{minipage}[t]{76mm}
%\end{table*}
%\begin{table*}\small
\begin{center}
\Caption{Результаты применения ЕМ-алго\-рит\-ма к выборке~2 ``Synterra'' 
\label{t2bat}} \vspace*{2ex}

\tabcolsep=8.7pt
\begin{tabular}{|c|c|c|}
\hline
№&Начальное приближение&Результат\\
\hline
\multicolumn{3}{|c|}{$P$}\\
\hline
0&0,2&$0{,}3815\hphantom{{}\cdot 10^{-9}}$\\
1&0,2&$0{,}3594\hphantom{{}\cdot 10^{-9}}$\\
2&0,2&$0{,}2589\hphantom{{}\cdot 10^{-9}}$\\
3&0,2&$0{,}4401\cdot 10^{-9}$\\
4&0,2&$0{,}0\hphantom{{}\cdot 10^{-9}999}$\\
\hline
\multicolumn{3}{|c|}{$\alpha$}\\
\hline
0&6,0804&1,5833\\
1&3,1838&0,8554\\
2&1,4886&0,4557\\
3&4,6407&0,2278\\
4&3,7843&0,1139\\
\hline
\multicolumn{3}{|c|}{$\lambda$}\\
\hline
0&17,3387&15,8682\\
1&47,8294&16,9150\\
2&54,1984&19,2866\\
3&\hphantom{1}8,6254&19,2866\\
4&\hphantom{1}5,7252&19,2866\\
\hline
\end{tabular}
\end{center}
\end{minipage}
\end{table*}


Данные результаты иллюстрируют рис.~\ref{f7bat}.


Эти результаты также отражают действительную картину, как показано на
рис.~\ref{f8bat}.


Этот трафик был снят с базовой станции <<Лукойл-Юго-Запад>> сети
широкополосного радиодоступа ЗАО <<Синтерра>>. Сеть радиодоступа
является реализацией так называемой <<последней мили>>, переносящей два
разных вида трафика: данные (Ethernet пакеты) и голос (IP-телефония, VoIP).
Поэтому здесь присутствуют в качестве основной массы короткие, но
интенсивные сообщения (пакеты SIP и голосовые фреймы), а также длинные
сообщения, содержащие данные.

Как мы видим, программная реализация ЕМ-ал\-го\-рит\-ма успешно справилась с
задачей разделения смесей распределений для этих двух выборок, что делает
данную программу удобным инструментом построения стохастической картины
конкретной сети. По полученным данным, используя метод интерпретации,
предложенный в разд.~2, можно получить представление о количестве
последовательных и параллельных структур вероятностной модели сети.

\subsection{Режим <<скользящего окна>>} %5.3.

Результаты для выборки
``Berkeley'' в режиме <<скользящего окна>>  представлены
на рис.~\ref{f9bat}.


Данные графики показывают изменение параметров распределений подвыборок выборки 
``Berkeley''. Видно, что параметры распределений подвыборок не остаются 
неизменными во времени, наоборот, они имеют внешне случайный характер. На 
рис.~\ref{f9bat},\,\textit{в} видна даже своеобразная пульсация первой 
компоненты.
%
На основании расчетов можно сделать вывод о том, что пиковость трафика
обусловливается как формой, так и интенсивностью сообщений.

\section{Заключение}

В данной работе исследована вероятностная модель  информационных потоков,
возникающих в сложных телекоммуникационных конвергентных сетях, построенная с
помощью асимптотического и энтропийного подходов. Эта модель предполагает, что
функционирование сложной телекоммуникационной сети можно представить в виде
суперпозиции довольно простых стохастических структур~--- последовательных и
параллельных, которые по\-рож\-да\-ют смеси гамма-распределений для случайной
величины времени обработки и передачи сообщений в сети. Предложена простая
интерпретация параметров данной модели.
\begin{figure*} %fig5
\vspace*{1pt}
\begin{center}
\mbox{%
\epsfxsize=130mm %145.109mm 
\epsfbox{bat-5.eps} }
\end{center}
\vspace*{-13pt} \Caption{Компоненты смеси начального приближения~(\textit{а}) и 
результата~(\textit{б}) для выборки~1 ``Berkeley'' \label{f5bat}}
%\end{figure*}
%\begin{figure*} %fig6
\vspace*{12pt}
\begin{center}
\mbox{%
\epsfxsize=130mm %148.256mm 
\epsfbox{bat-7.eps} }
\end{center}
\vspace*{-13pt} \Caption{График смеси распределений~(\textit{1}) и гистограмма 
для выборки~1 ``Berkeley''~(\textit{2}) \label{f6bat}}
\end{figure*}



\begin{figure*} %fig7
\vspace*{1pt}
\begin{center}
\mbox{%
\epsfxsize=130mm %144.283mm 
\epsfbox{bat-8.eps} }
\end{center}
\vspace*{-16pt} \Caption{Компоненты смеси начального приближения~(\textit{а}) и 
результата~(\textit{б}) для выборки~2 ``Synterra'' \label{f7bat}}
%\end{figure*}
%\begin{figure*} %fig8
\vspace*{12pt}
\begin{center}
\mbox{%
\epsfxsize=130mm %148.256mm 
\epsfbox{bat-10.eps} }
\end{center}
\vspace*{-11pt} \Caption{График смеси распределений~(\textit{1}) и гистограмма
для выборки~2 ``Synterra''~(\textit{2}) \label{f8bat}}
\end{figure*}

\begin{figure*} %fig9
\vspace*{1pt}
\begin{center}
\mbox{%
\epsfxsize=119.041mm
\epsfbox{bat-11.eps} }
\end{center}
\vspace*{-9pt} \Caption{Изменение  смешивающих параметров~(\textit{а}), 
параметров формы~(\textit{б}) и параметров масштаба~(\textit{в}) во времени для 
выборки~1 ``Berkeley'' \label{f9bat}}
\end{figure*}

Для решения вытекающей из модели задачи предложен итерационный алгоритм,
базирующийся на методе максимального правдоподобия~--- ЕМ-ал\-го\-ритм, для
которого получены формулы для конкретного вида смесей~--- экспоненциальных и
гамма-распределений.
%
Кроме того, разработан программный инструментарий для оценки параметров 
предложенной модели на выборках из реальных трафиковых данных. Проведены 
исследования, которые подтвердили предположения вероятностной модели. 


Получение информации о стохастической структуре
телекоммуникационных сетей и наличие программных инструментов для
выявления более или менее стабильных структур позволит понять причины
возникновения неожиданных больших нагрузок, предотвратить такие нагрузки,
а также поможет в будущем в проектировании надежных, оптимальных по
стоимости и уровню сервиса телекоммуникационных сетей нового поколения.

%\vspace*{-15pt} 
{\small\frenchspacing
{%\baselineskip=10.8pt
\addcontentsline{toc}{section}{Литература}
\begin{thebibliography}{9}
\bibitem{1bat}
Teletraffic Engeneering Handbook. International Telecommunication Union, 
Geneva, 2005 {\sf http://www.itu.int}. \vspace*{5pt} 
\bibitem{2bat}
\Au{Севастьянов~Б.\,А.} Курс теории вероятностей и математической статистики. 
М., 2004. \vspace*{5pt} 
\bibitem{3bat}
\Au{Айвазян~C.\,А., Бухштабер~В.\,М., Енюков~И.\,С, Мешалкин~Л.\,Д.} Прикладная 
статистика. Классификация и снижение размерности~// Финансы и статистика. М., 
1989. \vspace*{5pt} 
\bibitem{4bat}
\Au{Bilmes~J.\,A.} A gentle tutorial of the EM algorithm and its application to 
parameter estimation for Gaussian mixture and hidden Markov models. Berkeley, 
CA, USA: International Computer Science Institute,  1998. \vspace*{5pt} 
\bibitem{5bat}
\Au{Шлезингер~М.\,И.} О самопроизвольном различении образов~// Шлезингер~М.\,И. 
Читающие. автоматы. Киев: Наукова думка, 1965. С.~38--45. \vspace*{5pt} 
\bibitem{6bat}
\Au{Hsiao~I.-T., Rangarajan~A., Gindi~G.}. Joint-MAP 
reconstruction/segmentation for transmission tomography using mixture-models as 
priors. Yale University, 1998. \vspace*{5pt} 
\bibitem{7bat}
{\sf http://zedgraph.org}. \vspace*{4pt} 
\bibitem{8bat}
{\sf http://ita.ee.lbl.gov/html/contrib/LBL-PKT.html}. \vspace*{5pt} 
\bibitem{9bat}
{\sf http://www.synterra.ru}.
\end{thebibliography}

} } \label{end\stat}
\end{multicols}


%\addtocounter{razdel}{1}
%\def\razd{НЕРЕГУЛИРУЕМЫЙ ЭЛЕКТРОПРИВОД ДЛЯ ЭЛЕКТРОЭНЕРГЕТИКИ}

\setcounter{page}{2}

%   { %\Large  
   { %\baselineskip=16.6pt
   
   \vspace*{-48pt}
   \begin{center}\LARGE
   \textit{Предисловие}
   \end{center}
   
   %\vspace*{2.5mm}
   
   \vspace*{25mm}
   
   \thispagestyle{empty}
   
   { %\small 

    
Вниманию читателей журнала <<Информатика и её применения>> предлагается 
очередной тематический выпуск <<Вероятностно-статистические методы и 
задачи информатики и информационных технологий>>. Предыдущие тематические 
выпуски журнала по данному направлению вышли в 2008~г.\ (т.~2, вып.~2), 
в 2009~г.\ (т.~3, вып.~3) и в 2010~г.\ (т.~4, вып.~2). 

Статьи, собранные в данном журнале, посвящены разработке новых вероятностно-статистических 
методов, ориентированных на применение к решению конкретных задач информатики и информационных 
технологий, а также~--- в ряде случаев~--- и других прикладных задач. Проблематика, охватываемая 
публикуемыми работами, развивается в рамках научного сотрудничества между Институтом проблем 
информатики Российской академии наук (ИПИ РАН) и Факультетом вычислительной математики и 
кибернетики Московского государственного университета им.\ М.\,В.~Ломоносова в ходе работ 
над совместными научными проектами (в том числе в рамках функционирования 
Научно-образовательного центра <<Вероятностно-статистические методы анализа рисков>>). 
Многие из авторов статей, включенных в данный номер журнала, являются активными участниками 
традиционного международного семинара по проблемам устойчивости стохастических моделей, 
руководимого В.\,М.~Золотаревым и В.\,Ю.~Королевым; регулярные сессии этого семинара 
проводятся под эгидой МГУ и ИПИ РАН (в 2011~г.\ указанный семинар проводится в октябре 
в Калининградской области РФ). 

Наряду с представителями ИПИ РАН и МГУ в число авторов данного выпуска журнала входят 
ученые из Научно-исследовательского института системных исследований РАН, Института 
проблем технологии микроэлектроники и особочистых материалов РАН, Института 
прикладных математических исследований Карельского НЦ РАН, Московского 
авиационного института, Вологодского государственного педагогического университета, 
НИИММ им.\ Н.\,Г.~Чеботарева, Казанского государственного университета, Дебреценского 
университета (Венгрия).

Несколько статей выпуска посвящено разработке и применению стохастических методов и 
информационных технологий для решения различных прикладных задач. В~работе В.\,Г.~Ушакова 
и О.\,В.~Шестакова рассмотрена задача определения вероятностных характеристик случайных 
функций по распределениям интегральных преобразований, возникающих в задачах эмиссионной 
томографии. В~статье Д.\,О.~Яковенко и М.\,А.~Целищева рассмотрены некоторые вопросы 
математической теории риска и предложен новый подход к диверсификации инвестиционных 
портфелей. Работа И.\,А.~Кудрявцевой и А.\,В.~Пантелеева посвящена построению и 
исследованию математической модели, описывающей динамику сильноионизованной плазмы. 
В~статье П.\,П.~Кольцова изучается качество работы ряда алгоритмов сегментации изображений. 
Статья А.\,Н.~Чупрунова и И.~Фазекаша посвящена вероятностному анализу числа без\-оши\-бочных 
блоков при помехоустойчивом кодировании; получены усиленные законы больших чисел для указанных 
величин.

В данном выпуске традиционно присутствует тематика, весьма активно разрабатываемая в течение 
многих лет специалистами ИПИ РАН и МГУ,~--- методы моделирования и управления для 
информационно-телекоммуникационных и вычислительных систем, в частности методы 
теории массового обслуживания. В~статье А.\,И.~Зейфмана с соавторами рассматриваются 
модели обслуживания, описываемые марковскими цепями с непрерывным временем в случае 
наличия катастроф. В~работе М.\,М.~Лери и И.\,А.~Чеплюковой рассматриваются случайные 
графы Интернет-типа, т.\,е.\ графы, степени вершин которых имеют степенные распределения; 
такие задачи находят применение при исследовании глобальных сетей передачи данных. 
Работа Р.\,В.~Разумчика посвящена исследованию систем массового обслуживания специального 
вида~--- с отрицательными заявками и хранением вытесненных заявок.

Ряд статей посвящен развитию перспективных теоретических 
вероятностно-статистических методов, которые находят широкое применение в различных 
задачах информатики и информационных технологий. В~работе В.\,Е.~Бенинга, А.\,К.~Горшенина 
и В.\,Ю.~Королева рассмотрена задача статистической проверки гипотез о числе компонент 
смеси вероятностных распределений, приводится конструкция асимптотически наиболее мощного 
критерия. Результаты этой работы найдут применение в ряде прикладных задач, использующих 
математическую модель смеси вероятностных распределений (в информатике, моделировании 
финансовых рынков, физике турбулентной плазмы и~т.\,д.). В~статье В.\,Ю.~Королева, 
И.\,Г.~Шевцовой и С.\,Я.~Шоргина строится новая, улучшенная оценка точности нормальной 
аппроксимации для пуассоновских случайных сумм; как известно, указанные случайные суммы 
широко используются в качестве моделей многих реальных объектов, в том числе в информатике, 
физике и других прикладных областях. Работа В.\,Г.~Ушакова и Н.\,Г.~Ушакова посвящена 
исследованию ядерной оценки плотности распределения; эти результаты могут применяться, 
в част\-ности, при анализе трафика в телекоммуникационных системах. Серьезные приложения 
в статистике могут получить результаты работы О.\,В.~Шестакова, в которой доказаны оценки 
скорости сходимости распределения выборочного абсолютного медианного отклонения к нормальному 
закону. 

\smallskip

Редакционная коллегия журнала выражает надежду, что данный тематический  выпуск 
будет интересен специалистам в области теории вероятностей и математической статистики 
и их применения к решению задач информатики и информационных технологий.
     
     %\vfill 
     \vspace*{20mm}
     \noindent
     Заместитель главного редактора журнала <<Информатика и её 
применения>>,\\
     директор ИПИ РАН, академик  \hfill
     \textit{И.\,А.~Соколов}\\
     
     \noindent
     Редактор-составитель тематического выпуска,\\
     профессор кафедры математической статистики факультета\\
      вычислительной математики и кибернетики МГУ им.\ М.\,В.~Ломоносова,\\
     ведущий научный сотрудник ИПИ РАН,\\ 
доктор физико-математических наук \hfill
      \textit{В.\,Ю.~Королев}
     
     } }
     }

\def\stat{sinits}

\def\tit{СТОХАСТИЧЕСКИЕ ИНФОРМАЦИОННЫЕ  ТЕХНОЛОГИИ ДЛЯ~ИССЛЕДОВАНИЯ
НЕЛИНЕЙНЫХ КРУГОВЫХ СТОХАСТИЧЕСКИХ СИСТЕМ$^*$}

\def\titkol{Стохастические информационные  технологии для~исследования
нелинейных круговых стохастических систем}

\def\autkol{И.\,Н.~Синицын}
\def\aut{И.\,Н.~Синицын$^1$}

\titel{\tit}{\aut}{\autkol}{\titkol}

{\renewcommand{\thefootnote}{\fnsymbol{footnote}}\footnotetext[1]
{Работа выполнена при финансовой поддержке РФФИ
(проект №\,10-07-00021) и программы ОНИТ РАН <<Информационные
технологии и анализ сложных систем>> (проект 1.5).}}


\renewcommand{\thefootnote}{\arabic{footnote}}
\footnotetext[1]{Институт проблем информатики Российской академии наук, sinitsin@dol.ru}


\Abst{Статья посвящена стохастическим (корреляционным и спект\-раль\-но-кор\-ре\-ля\-ци\-он\-ным) 
информационным технологиям аналитического и статистического анализа и моделирования 
процессов в нелинейных круговых стохастических системах на базе методов круговой 
статистической линеаризации <<намотанным>> нормальным распределением. В~основу технологий 
положены методы, алгоритмы и инструментальное программное обеспечение (ИПО)
CStS-ANALYSIS в среде  MATLAB.}

\KW{аналитическое моделирование; круговой стохастический процесс;
круговая стохастическая система; круговая статистическая линеаризация;
компьютерная поддержка статистических научных исследований; MATLAB;
корреляционные уравнения; спект\-раль\-но-кор\-ре\-ля\-ци\-он\-ные уравнения;
стохастические информационные технологии; статистическое моделирование}

 \vskip 14pt plus 9pt minus 6pt

      \thispagestyle{headings}

      \begin{multicols}{2}
      
            \label{st\stat}


\section{Введение}
Компьютерная поддержка научных исследований (КПНИ) как
 неотъемлемая часть автоматизации научных исследований
 становится все более характерным признаком современных научных
 исследований (НИ) и оказывает сильное влияние на их интенсивность и
 эффективность, превращается в важнейший фактор дальнейшего
 прогресса науки~[1, 2]. Современный этап развития КПНИ характеризуется  интенсивным
проникновением ее в новые сферы исследований и разработок,
расширением контингента пользователей, охватом всех
этапов исследований от сбора и первичной обработки данных,
управления экспериментами до анализа и перспективного планирования
основных на\-прав\-ле\-ний НИ и их информационных
технологий.

Под информационной технологией обычно понимают совокупность
систематических и массовых способов создания, накопления, обработки,
хранения, передачи и распределения информации (данных, знаний) с
помощью средств вычислительной техники и связи.

 На
практике обычно создается ИТ, рассчитанная на выполнение с ее
по\-мощью некоторой основной функции, что связано с необходимостью
решения нескольких типовых задач исследований.
Перечень основных функций довольно ограничен, а с другой стороны,
выполнение этих функций может потребоваться во многих применениях.
Это делает целесообразным выделение функ\-ци\-о\-наль\-но-ориен\-ти\-ро\-ван\-ных,
предметно-ориентированных и проб\-лем\-но-ориен\-ти\-ро\-ван\-ных ИТ~\cite{1-sin}.

 На примере статистических НИ в~\cite{1-sin} 
 рас\-смот\-ре\-ны современные принципы подходы и задачи КПНИ, сформулированы 
 требования к стохастическим ИТ (СтИТ) анализа, моделирования и синтеза 
 оптимальных, субоптимальных и услов\-но-оп\-ти\-маль\-ных фильтров для обработки 
 информации, описано ИПО, а 
 также некоторые приложения. В~качестве основных математических моделей в 
 СтИТ принимались стохастические дифференциальные, интегральные и смешанные 
 уравнения в евклидовом пространстве, а также их разностные аналоги. Для 
 круговых, сферических, кватернионных и других гипергеометрических 
 стохастических уравнений, относящихся к системам на многообразиях~\cite{3-sin}, 
 известные методы анализа, моделирования и синтеза требуют развития. Однако при 
 этом основные принципы, подходы и задачи статистических НИ сохраняются.

Обзор зарубежного универсального методического и программного обеспечения 
для математической статистики  круговых случайных величин и функций дан в~\cite{4-sin}. 
Отдельные прикладные задачи решены, например, в~[1--8].
В~ИПИ РАН начиная с 2010~г.\ в рамках тем, поддерживаемых РФФИ, 
ведутся работы по созданию методического обеспечения для анализа, 
моделирования и синтеза фильтров для обработки информации в круговых 
стохастических системах (КСтС)~\cite{9-sin, 10-sin}.

Рассмотрим полезные для практики простые квазилинейные, 
основанные на эквивалентной круговой статистической линеаризации (КСЛ), 
корреляционные и спектрально-корреляционные методы, алгоритмы и ИПО для 
оф\-лайн-ана\-ли\-за и моделирования круговых стохастических процессов 
(КСтП) в нелинейных КСтС.

\section{Статистическая линеаризация нелинейных преобразований круговых случайных величин}

Пусть сначала $X$ и $Y$~--- скалярные круговые случайные величины (КСВ), 
связанные между собой детерминированной нелинейной зависимостью
    \begin{equation}
    Y=\vrp (X)\,.\label{e2.1s}
    \end{equation}
Согласно принципу эквивалентной статистической линеаризации заменим нелинейную 
зависимость~(\ref{e2.1s}) приближенной линейной зависимостью:
\begin{equation}
\vrp (X) \approx U =\vrp_0 + k_1 (X-\mu)\,,\label{e2.2s}
\end{equation}
где $\mu=\mu_x$~--- круговое среднее направление КСВ~$X$. 
Параметры $\vrp_0$ и~$k_1$ находят из критерия минимума 
безусловного риска для выбранной функции потерь~$\ell (X,U)$:
\begin{equation}
R= \mm \lk\ell (X,U) \rk =\min \,,\label{e2.3s}
\end{equation}
где $\mm$~--- символ математического ожидания.

Если выбрать эквивалентное одномерное распределение (ЭР) КСВ~$X$ и функцию потерь в виде
\begin{equation}
\ell (X,U) =\left( e^{iX} - e^{iU}\right)^2\,,\label{e2.4s}
\end{equation}
то после подстановки~(\ref{e2.2s}) в~(\ref{e2.3s}) и~(\ref{e2.4s}) 
и приравнивания нулю частных производных $\prt R/\prt \vrp_0$ и $\prt R/\prt k_1$ 
получим одно комплексное уравнение для неизвестных параметров $\vrp_0$ и~$k_1$:
\begin{multline}    
\mm_\ap \exp \lf i \vrp (X)\rf ={}\\
{}=\mm_\ap \exp \lf i\lk \vrp_0 + k_1 (X-\mu)\rk\rf\,,\label{e2.5s}
\end{multline}
где $\mm_\ap$~--- символ математического ожидания по ЭР; 
коэффициенты КСЛ $\vrp_0$ и $k_1$ зависят от вероятностных характеристик КСВ~$X$.

Принимая в качестве ЭР для КСВ $X$ намотанное нормальное распределение с параметрами  
$\mu$ и~$\si$, т.\,е.\ $WN(\mu,\si)$~\cite{4-sin, 7-sin}, 
перепишем комплексное уравнение~(\ref{e2.5s}) в виде двух действительных уравнений:
\begin{equation*}
\vrp_0 (\mu,\si) =\psi (\mu,\si)\,; %\label{e2.6s}
\end{equation*}
\begin{equation*}
k_1 (\mu,\si) =\fr{\sqrt{-2\ln r(\mu,\si)}}{\si}\,, %\label{e2.7s}
\end{equation*}
где введено следующее обозначение: 
$$
re^{i\psi} =\mm_{WN} \exp \lf -i \vrp (X)\rf\,.
$$

Для скалярного нелинейного преобразования векторного аргумента
\begin{equation}
Y=\vrp (X_1\tr X_n)\label{e2.8s}
\end{equation}
при условии, что ЭР вектора КСВ  $X= [ X_1, \ldots$\linebreak 
$\ldots , X_n]^{\mathrm{T}}$ является известным 
намотанным нормальным распределением~\cite{4-sin, 7-sin}, 
уравнения принципа статистической линеаризации по критерию~(\ref{e2.4s}) имеют следующий вид:
\begin{equation*}
\vrp (X) \approx U = \vrp_0 +\sss_{h=1}^n k_{1h} X_h^0\,. %\label{e2.9s}
\end{equation*}
Здесь $\vrp_0$~--- первый векторный коэффициент КСЛ, равный
\begin{equation*}
\vrp_0 = \mm_{WN} \vrp(X)\,, %\label{e2.10s}
\end{equation*}
$k_{1h}$ $(h=1\tr n)$~--- второй векторный коэффициент КСЛ, который 
определяется путем решения алгебраической системы уравнений
\begin{equation*}
\sss_{j=1}^n k_{1h} K_{jh} = \mm_{WN} X_j^0 \vrp (X)\,, %\label{e2.11s}
\end{equation*}
где $K_{1h} =\mm_{WN} X_j^0 X_h^0$\ \,$(j,h\hm=1\tr n)$.

Аналогично выписываются формулы для коэффициентов 
КСЛ для векторных и матричных нелинейных преобразований, 
а также посредством канонических представлений~\cite{1-sin}.

Для типовых нелинейных преобразований~(\ref{e2.1s}) и~(\ref{e2.8s}) 
составлены таблицы и разработано ИПО 
CStS-ANALYSIS~\cite{11-sin}.

\section{Основные результаты}

\noindent
\textbf{Теорема 3.1.} \textit{Пусть нестационарная дифференциальная система}
\begin{equation*}
\dot Y =\vrp (Y,t) +V\,,\quad Y(t_0) = Y_0 %\label{e3.1s}
\end{equation*}
\textit{удовлетворяет следующим допущениям:}
\begin{enumerate}[(1)]
\item \textit{$n$-мер\-ный круговой $($на $[0, 2\pi])$  СтП $Y\hm=Y(t)$ 
обладает конечными вероятностными моментами второго порядка;}
\item
\textit{$n$-мерный круговой белый шум, понимаемый в строгом смысле, $(V=\dot W$, 
$W$~--- КСтП с независимыми приращениями на  $[0, 2\pi]$  и матрицей интенсивности $G(t))$;}
\item
\textit{детерминированное нелинейное преобразование  $\vrp (Y,t)$ не обладает  памятью и допускает 
КСЛ согласно алгоритмам разд.~2, причем статистически линеаризованная система для 
$Y^0 \hm= Y-m_y$:
\begin{equation}
{\dot Y}^0 = k_1 Y^0 + V\quad (m_y = \mm Y)\label{e3.2s}
\end{equation}
асимптотически устойчива.
Тогда корреляционное уравнение квазилинейного анализа и аналитического 
моделирования имеют следующий вид:}
\begin{align*}
\dot m_y &=\vrp_0(m_y, K_y,t)\,;\quad m_y (t_0) = m_{y0}\,;\\ %\label{e3.3s}
\dot K_y &= k_1 (m_y, K_y, t) K_y + {}\\
&\hspace*{8mm}{}+  K_y k_1^{\mathrm{T}} (m_y, K_y, t)+ G(t)\,;\\
K_y (t_0) &= K_{y0}\,;\\
\fr{\prt K_y(t_1, t_2)}{\prt t_2} &= 
K_y (t_1, t_2) k_1^{\mathrm{T}} (m_y, K_y, t_2)\,;\\
K_y (t_1, t_2)&= \begin{cases}
K_y(t_1,t_2) &\ \mbox{при\ \ } t_2>t_1\,;\\
K_y (t_2, t_1)^{\mathrm{T}} &\ \mbox{при\ \ } t_2<t_1\,,
\end{cases} 
\end{align*}
\textit{где  $m_y$, $K_y(t)$ и $K_y(t_1, t_2)$~--- соответственно вектор математического ожидания, 
ковариационная матрица  и матрица ковариационных функций КСтП~$Y(t)$}.
\end{enumerate}


\smallskip

\noindent
\textbf{Теорема 3.2.} \textit{В условиях теоремы}~3.1 
\textit{при стационарных функциях $\vrp (Y,t) \hm=\vrp(Y)$, $G(t) \hm=G$ 
корреляционные уравнения анализа и аналитического моделирования для КСтП $\tilde Y(t)$ 
имеют вид:}
\begin{gather}
\vrp_0 (\tilde m_y ,\tilde K_y)=0\,;\notag %label{e3.6s}
\\[3pt]
k_1 (\tilde m_y, \tilde K_y)\tilde K_y + \tilde K_y k_1 (\tilde m_y, \tilde K_y) + G =0\,\label{e3.7s}
\end{gather}

\vspace*{-3pt}

\noindent
\begin{equation}
\left.
\begin{array}{rl}
\fr{d\tilde k_y(\tau)}{d\tau}  &= k_1 (\tilde m_y , \tilde K_y) \tilde k_y(\tau)\,;\\[9pt]
k_y(\tau)&=\tilde{K}_y(t_1,t_1+\tau)\,,
\end{array}
\right\}
\label{e3.8s}
\end{equation}
\textit{где $\tilde m_y$, $\tilde K_y$ и $\tilde k_y(\tau)$~--- соответственно 
математическое ожидание, ковариационная матрица и матрица 
ковариационных функций $(\tau \hm= t_1 \hm- t_2)$ стационарного КСтП $\tilde Y(t)$}.

\smallskip

\noindent
\textbf{Теорема 3.3.} \textit{В~условиях теоремы}~3.2 \textit{уравнения}~(\ref{e3.7s}) 
\textit{и}~(\ref{e3.8s}), \textit{если вместо ковариационной матрицы $k_y(\tau)$ 
использовать спектральную плотность  $s_y(\w)$, допускают следующее спектральное представление:
\begin{align*}
\tilde K_y &=\iin s_y(\w; \tilde m_y, \tilde K_y)\, d\w\,; %\label{e3.9s}
\\
k_y(\tau) &=\iin e^{i\w\tau} s_y (\w; \tilde m_y, \tilde K_y)\, d\w\,, %\label{e3.10s}
\end{align*}
где $s_y (\w; \tilde m_y, \tilde K_y)$~--- матрица спектральных плот\-ностей:
\begin{equation*}
s_y (\w; \tilde m_y, \tilde K_y) = \Phi (i\w; \tilde m_y, \tilde K_y) 
\fr{G}{2\pi} I_n \Phi (i\w; \tilde m_y, \tilde K_y)^*\,; %\label{e3.11s}
\end{equation*}
$\Phi (i\w; \tilde m_y, \tilde K_y)$~--- передаточная функция 
статистически линеаризованной системы}~(\ref{e3.2s}):
 \begin{equation*}
 \Phi (i\w; \tilde m_y, \tilde K_y) = \left[k_1 (\tilde m_y,\tilde  K_y) - Ii\w\right]^{-1}\!;
 \ \ I=I_n\,;
% \label{e3.12s}
 \end{equation*}
\textit{$^*$~--- символ эрмитова сопряжения; $I_n$~--- единичная $(n\times n)$-мат\-рица}.


\medskip

\noindent
\textbf{Замечание.} Рассмотренные в~\cite{1-sin} другие схемы статистической линеаризации 
очевидным образом обобщаются на круговой случай. При этом могут быть использованы различные 
модели КСтС~\cite{9-sin}.

Алгоритмы теорем~3.1--3.3 и их дискретных версий лежат в основе СтИТ
анализа аналитического моделирования. Они реализованы  в ИПО\linebreak
CStS-ANALYSIS в среде  MATLAB~[9--11]. Инструментальное программное
обеспечение имеет возможность
реализовать также и статистическое моделирование КСтС для
следующих <<намотанных>> круговых распределений КСВ: решетчатого,
нормального,  Мизеса, равномерного, пуассонова, кардиоидного,
треугольного, Коши и других устойчивых распределений~\cite{4-sin, 7-sin}.
Точность алгоритмов анализа и аналитического моделирования
проверялась на радиотехнических примерах~\cite{6-sin}, а также методом
статистического моделирования.

\section{Заключение}

Принципы, подходы и задачи статистических научных исследований,
развитые в~\cite{1-sin} для стохастических систем в евклидовом пространстве,
сохраняются и для круговых систем. Методическое и алгоритмическое
обеспечение, основанное на статистической линеаризации для
эквивалентного <<намотанного>> нормального распределения, даются
теоремами~3.1--3.3. Разработано и испытано на ряде тестовых примеров
универсальное ИПО CStS-ANALYSIS
в среде  MATLAB для анализа, аналитического и статистического
моделирования.

{\small\frenchspacing
{%\baselineskip=10.8pt
\addcontentsline{toc}{section}{Литература}
\begin{thebibliography}{99}

\bibitem{1-sin}
\Au{Синицын И.\,Н.}
Канонические представления случайных функций и их применения в 
задачах компьютерной поддержки научных исследований.~--- М.: ТОРУС ПРЕСС, 2009.

\bibitem{2-sin}
\Au{Босов А.\,В., Будзко В.\,И., Захаров~В.\,Н., Козмидиади~В.\,А., 
Корепанов~Э.\,Р., Синицын~И.\,Н., Шоргин~С.\,Я., Ушмаев~О.\,С.}  
Информатика: состояние, проблемы, перспективы~/ Под ред.  И.\,А.~Соколова.~--- М.: ИПИ РАН, 2009.

\bibitem{3-sin}
\Au{Ватанабэ С., Икэда Н.}
 Стохастические дифференциальные уравнения и диффузионные процессы~/ Пер. с англ. 
 под ред.  А.\,Н.~Ширяева.~--- М.: Наука, 1986.

\bibitem{4-sin}
\Au{Rao Jammalamadaka S., Sen Gupta~A.}
  Topics in circular statistics.~--- Singapore: World Scientific, 2001.

\bibitem{5-sin}
\Au{Леви П.}
 Стохастические процессы и броуновское движение~/ Пер. с фр.  под ред. Н.\,Н.~Ченцова.~--- М.: Наука, 1972.

\bibitem{6-sin}
\Au{Тихонов В.\,И., Миронов М.\,А.}
 Марковские процессы.~--- М.: Сов. радио, 1977.

\bibitem{7-sin}
\Au{Мардиа К.} 
Статистический анализ угловых наблюдений~/ Пер. с англ. под ред. Л.\,Н.~Большева.~--- М.: Наука, 1978.
%\columnbreak

\bibitem{8-sin}
\Au{Морозов А.\,Н., Назолин А.\,Л.}
 Динамические системы с флуктуирующим временем.~--- М.: МГТУ им. Н.\,Э.~Баумана, 2001.

 \columnbreak


\bibitem{9-sin}
\Au{Синицын И.\,Н. }
 Канонические разложения случайных функций и их применение в стохастических 
 ин-\linebreak формационных технологиях научных исследований: Курс лекций~// 
 Распознавание образов и анализ изоб\-ра\-же\-ний: новые информационные технологии~--- 
 РОАИ-10-2010: Мат-лы Междунар. конф.~--- СПб., 2010.
 
 \vspace*{6pt}

\bibitem{10-sin}
\Au{Синицын И.\,Н., Корепанов Э.\,Р., Белоусов~В.\,В. и~др.}
Развитие компьютерной поддержки статистических научных исследований сис\-тем 
высокой точности и доступности~// Системы и средства информатики, 2011. Вып.~21. №\,1. С.~3--33.

 \vspace*{6pt}

\label{end\stat}

\bibitem{11-sin}
\Au{Sinitsyn I.\,N., Belousov V.\,V., Konashenkova~T.\,D.}
Software tools for circular stochastic systems analysis~/ 
29th  Seminar (International) on Stability Problems for  Stochastic Models and
5th Workshop ``Applied Problems in Theory of Probabilities and
Mathematical Statistics Related to Modeling of Information Systems'' (APTP\;+\;MS'2011) Book of
Abstracts.~---  M.: IPI RAS, 2011. P.~86--87.
 \end{thebibliography}
}
}


\end{multicols}          %1
\def\stat{pechinkin}


\def\tit{СОВМЕСТНОЕ СТАЦИОНАРНОЕ РАСПРЕДЕЛЕНИЕ
ЧИСЛА ЗАЯВОК В~НАКОПИТЕЛЕ И~В~БУНКЕРЕ
ПЕРЕУПОРЯДОЧЕНИЯ В~МНОГОКАНАЛЬНОЙ СИСТЕМЕ
ОБСЛУЖИВАНИЯ С~ПЕРЕУПОРЯДОЧЕНИЕМ
ЗАЯВОК$^*$}


\def\titkol{Совместное стационарное распределение
числа заявок в~накопителе и~в~бункере
переупорядочения} % в~многоканальной системе обслуживания с~переупорядочением заявок}

\def\aut{\fbox{А.\,В.\~Печинкин}$^1$, Р.\,В.~Разумчик$^2$}

\def\autkol{А.\,В.\~Печинкин, Р.\,В.~Разумчик}

\titel{\tit}{\aut}{\autkol}{\titkol}

{\renewcommand{\thefootnote}{\fnsymbol{footnote}} \footnotetext[1]
{Работа выполнена при частичной поддержке РФФИ (проект 13-07-00223).}}


\renewcommand{\thefootnote}{\arabic{footnote}}
\footnotetext[1]{Институт проблем информатики Российской академии наук}
\footnotetext[2]{Институт проблем информатики Российской академии наук; Российский
университет дружбы народов, rrazumchik@ieee.org}

%\vspace*{3pt}

\Abst{Рассматривается функционирующая в~непрерывном времени
многоканальная система обслуживания с~накопителем
бесконечной емкости и переупорядочением заявок.
В~систему поступает пуассоновский поток заявок, время
обслуживания каждым прибором распределено по
экспоненциальному закону с~одним и~тем же параметром.
При поступлении в~систему всем заявкам  присваивается
порядковый номер. На выходе из системы сохраняется
порядок между заявками, установленный при входе в~нее.
Заявки, завершившие обслуживание и~нарушившие установленный порядок,
накапливаются на выходе системы
в~бункере переупорядочения (БП), который также имеет неограниченную емкость.
Найдено совместное стационарное распределение
числа заявок в~накопителе и~суммарного числа
заявок в~БП в~терминах
вычислительных алгоритмов и~производящих функций (ПФ).
Приведены примеры расчетов по полученным
соотношениям.}

\KW{многолинейная система массового обслуживания;
переупорядочение; стационарное распределение
числа заявок}

\DOI{10.14357/19922264140401}


%\vspace*{3pt}

\vskip 12pt plus 9pt minus 6pt

\thispagestyle{headings}

\begin{multicols}{2}

\label{st\stat}


\section{Введение}

Для функционирования ряда
ин\-фор\-ма\-ци\-он\-но-те\-ле\-ком\-му\-ни\-ка\-ци\-он\-ных сис\-тем
и для предоставления на их основе услуг
необходимо соблюдение\linebreak требования сохранения порядка в~потоке передаваемых сообщений.
Различные действия, необходимые для этого, можно объединить
в~одно понятие~--- переупорядочение.
Для изучения влияния\linebreak \mbox{переупорядочения} на качество
функционирования ин\-фор\-ма\-ци\-он\-но-те\-ле\-ком\-му\-ни\-ка\-ци\-он\-ных
сис\-тем к~настоящему времени предложено множество
моделей, которые в~своей основе используют методы
и~модели теории массового обслуживания.
Исследуемая сис\-те\-ма обычно представляется в~виде
системы или сети массового обслуживания с одним\linebreak или
несколькими входящими потоками сообщений.
Эффект переупорядочения часто моделируется с~помощью
дополнительной очереди (БП),
в~которую попадают сообщения, обработанные\linebreak в~системе,
и~ожидают там до тех пор, пока порядок следования сообщений
нельзя будет восстановить.
Некоторый обзор работ в~этом направлении можно найти
в~\cite{a1, a2},
а~некоторые последние результаты~--- в~[3--8].

Настоящая работа является развитием \cite{a8}, в~которой
рассматривается система массового обслуживания (СМО)
с~переупорядочением в~виде марковской многоканальной
системы обслуживания неограниченной емкости и~бункером
переупорядочения, также имеющим неограниченную
емкость.
В~\cite{a8} была получена система уравнений равновесия для
совместного стационарного распределения чис\-ла заявок в~системе
и~бункере переупорядочения и~приведены некоторые результаты
численных расчетов.
Однако несомненный интерес представляют
две задачи, не освещенные в~\cite{a8}, которые и~являются
предметом данной статьи, а~именно:
разработка рекуррентного алгоритма расчета вышеупомянутого
совместного стационарного распределения и~нахождение
этого распределения в~терминах ПФ.

Статья организована таким образом.
В~разд.~2 приводится подробное описание
системы.
В~разд.~3 дается рекуррентный алгоритм расчета
совместного стационарного распределения, а~в~разд.~4
показано, как совместное стационарное распределение
можно найти в~терминах ПФ.
Примеры расчетов, проведенных по формулам разд.~4,
представлены в~разд.~5.
В~заключении сформулированы основные результаты работы.

\section{Описание системы}

Рассмотрим функционирующую в~непрерывном времени
$N$-ли\-ней\-ную ($N\hm\ge 2$) СМО с накопителем
неограниченной емкости, входящим пуассоновским
потоком заявок интенсивности~$\lambda$ \mbox{и~экспоненциальным}
распределением времени
обслуживания заявки каждым прибором с~па\-ра\-мет\-ром~$\mu$.


При поступлении в~систему всем заявкам  присваивается
порядковый номер.
На выходе из СМО сохраняется порядок между заявками,
установленный при входе в~нее.
Заявки, завершившие обслуживание и~нарушившие
установленный порядок, накапливаются на выходе
системы в~БП и~покидают СМО только
после того, как закончится обслуживание всех заявок с~меньшими номерами.
Такая СМО носит название системы с переупорядочением
заявок.

Предполагается также выполнение необходимого и~достаточного условия
существования стационарного режима функционирования СМО
$$\tilde {\rho}\hm=\fr{\rho}{N}<1\,,
$$
 где $\rho\hm=\lambda/\mu$.

\vspace*{-9pt}

\section{Алгоритм нахождения совместного стационарного распределения}

Предположим, что на приборах находится $n$, $n\hm=\overline{1,N}$, заявок.
Тогда заявкой первого уровня будем называть ту из них,
которая в~систему поступила последней, второго уровня~--- предпоследней,
$\ldots,$ $n$-го уровня~--- первой. При этом если $n\hm=N$ (все приборы
заняты), то находящиеся в~БП заявки, поступившие между заявками
второго и~первого уровней, будем называть заявками первой очереди,
заявки, поступившие между заявками третьего и~второго уровней,~---
заявками второй очереди, $\ldots,$ заявки, поступившие между
заявками $N$-го и~$(N-1)$-го уровней,~--- заявками $(N-1)$-й
очереди. Если же $n<N$, то  заявками первой очереди будем называть
заявки из БП, поступившие после заявки первого уровня, заявками
второй очереди~--- заявки, поступившие между заявками второго и~первого уровней,
и~т.\,д.

При $n\ge N$ обозначим через
$p^{(m)}_{n;i}$, ${m\hm=\overline{1,N-1}}$, ${i\hm\ge 0}$,
стационарную вероятность того, что в~системе на
приборах и~в накопителе находится~$n$~заявок,
а~в~БП имеется в~сумме~$i$~заявок первой,
второй, $\ldots,$ $m$-й очереди.
Через
$p^{(m)}_{n;i}$, ${m\hm=\overline{1,n}}$, ${i\hm\ge 0}$,
обозначим аналогичную стационарную вероятность
при $n\hm=\overline{1,N-1}$.
Через~$p_n$, $n\hm\ge 0$, обозначим
стационарную вероятность того, что в~системе на
приборах и~в накопителе (без учета числа заявок в~БП) находится~$n$~заявок.
Очевидно, что стационарные вероятности~$p_n$
определяются теми же самыми формулами, что и~в~обычной
марковской СМО $M/M/N/\infty$
(см., например,~\cite{boch}):
\begin{align}
p_{0} &= \left( \sum\limits_{i=0}^{N-1} \fr{\rho^i}{i!} +
\fr{\rho^N}{(N-1)! (N-\rho)}
\right)^{-1} \,;\label{3-1}
\\
p_{i} &= \begin{cases}
\fr{\rho^i }{i!} p_{0}\,, &\ i=\overline{1,N}\,,
\\
%\label{3-3}
\fr{\rho^i}{N!\, N^{i-N}} p_{0}
= \tilde \rho^{i-N} p_{N}\,, &\ i\ge N+1\,.
\end{cases}
\label{3-2}
\end{align}

Наконец, через $p_{n;i}$, ${n\hm\ge 1}$, ${i\hm\ge 0}$, обозначим
стационарную вероятность того, что в~системе на
приборах и~в накопителе находится~$n$~заявок,
а~в~БП~--- $i$~заявок.

Используя принцип глобального баланса, можно выписать систему уравнений для
вероятностей~$p^{(m)}_{n;i}$.
Для вероятностей $p^{(1)}_{n;i}$, $n\hm\ge N$,
$i \hm\ge 0$, справедливы уравнения:
\begin{align}
\hspace*{-2.8mm}p^{(1)}_{n;0} (\lambda+N\mu) &= p^{(1)}_{n-1;0} \lambda +
p_{n+1} (N-1) \mu \,,\ n\ge N;
\!\!\label{eq-1-1}
\\
\hspace*{-2.8mm}p^{(1)}_{n;i} (\lambda+N\mu) &= p^{(1)}_{n-1;i} \lambda +
p^{(1)}_{n+1;i-1} \mu \,,\notag\\
&\hspace*{25mm} n\ge N\,,\enskip i \ge 1\,.
\label{eq-1-2}\!\!
\end{align}
%%%%%%%%%%%%%%%%%%%%%%%
%%%%%%%%%%%%%%%%%%%%%%%
Для вероятностей $p^{(1)}_{N-1;i}$,\ \ $i \ge 0$,
справедливы уравнения:
%%%%%%%%%%%%%%%%%%%
\begin{align}
\label{eq-1-3}
p^{(1)}_{N-1;0} [\lambda+(N-1)\mu] &=
p_{N-2} \lambda + p_{N} (N-1)\mu\,;
\\
\label{eq-1-4}
p^{(1)}_{N-1;i} [\lambda+(N-1)\mu] &= p^{(1)}_{N;i-1} \mu\,,\enskip i \ge 1\,.
\end{align}
Для вероятностей
$p^{(1)}_{n;i}$, $n\hm=\overline{1,N-2}$, $i \hm\ge 0$,
справедливы уравнения
\begin{align}
\label{eq-1-5}
\hspace*{-2mm}p^{(1)}_{n;0} (\lambda+n\mu) &= p_{n-1} \lambda +
p^{(1)}_{n+1;0} n\mu ,\  n=\overline{1,N-2};
\\
\label{eq-1-6}
\hspace*{-2mm}p^{(1)}_{n;i} (\lambda+n\mu) &= p^{(1)}_{n+1;i} n\mu
+ p^{(2)}_{n+1;i-1} \mu \,,\notag\\
&\hspace*{15mm}n=\overline{1,N-2},\ \ i \ge 1.
\end{align}


Для остальных вероятностей
$p^{(m)}_{n;i}$, $m\hm=\overline{2,N-1}$, справедливы формулы:
\begin{align}
p^{(m)}_{n;0} (\lambda+N\mu) &= p^{(m)}_{n-1;0} \lambda +
p^{(m-1)}_{n+1;0} (N-m) \mu\,,\notag\\
& \hspace*{30mm}n\ge N\,; \label{bat-1}
\\
p^{(m)}_{n;i} (\lambda+N\mu) &= p^{(m)}_{n-1;i} \lambda +
p^{(m-1)}_{n+1;i} (N-m) \mu +{}\notag\\
&\hspace*{-10mm}{}+p^{(m)}_{n+1;i-1} m \mu \,,\enskip
n\ge N\,,\ \ i\ge 1\,;
\label{bat-2}
\end{align}

\noindent
\begin{align}
p^{(m)}_{N-1;0} [\lambda+(N-1)\mu] &={}\notag\\
{}=p^{(m-1)}_{N-2;0} \lambda
&{}=+ p^{(m-1)}_{N;0} (N-m) \mu \,;
\label{bat-3}
\end{align}

\noindent
\begin{multline}
p^{(m)}_{N-1;i} [\lambda+(N-1)\mu] =p^{(m-1)}_{N-2;i} \lambda+{}\\
{}+
p^{(m-1)}_{N;i} (N-m) \mu +p^{(m)}_{N;i-1} m \mu\,,\enskip i\ge 1\,;
\label{bat-4}
\end{multline}

\vspace*{-12pt}



\noindent
\begin{multline}
\label{bat-5}
p^{(m)}_{n;0} (\lambda+n\mu) = p^{(m-1)}_{n-1;0} \lambda+
p^{(m)}_{n+1;0} (n-m+1) \mu \,,\\
 n=\overline{m,N-2}\,;
\end{multline}

\noindent
\begin{multline}
\label{bat-6}
p^{(m)}_{n;i} (\lambda+n\mu) = p^{(m-1)}_{n-1;i} \lambda
+
p^{(m)}_{n+1;i} (n-m+1) \mu +{}\\
{}+ p^{(m+1)}_{n+1;i-1} m \mu\,,\enskip
 n=\overline{m,N-2}\,,\ \ i\ge 1\,.
\end{multline}

Решение данной системы уравнений позволяет
найти совместное стационарное распределение
$p_{n;i}$ числа заявок на приборах и~в
накопителе и~суммарного числа заявок в~БП в~виде следующих ра\-венств:
\begin{alignat*}{2}
%\label{bat-7}
p_{n;i} &= p^{(N-1)}_{n;i}\,, &\quad  n&\ge N\,,\ \ i\ge 0\,,
\\
%\label{bat-8}
p_{n;i} &= p^{(n)}_{n;i} \,, &\quad n&=\overline{1,N-1}\,,\ \ i\ge 0\,.
\end{alignat*}

Анализ системы~\eqref{eq-1-1}--\eqref{bat-6}
показал, что вычисление стационарных
вероятностей $p^{(m)}_{n;i}$ можно проводить
рекуррентным образом по следующему алгоритму.

\bigskip

\noindent
А\,л\,г\,о\,р\,и\,т\,м~1\ (\textbf{Алгоритм решения системы уравнений равновесия}).

\noindent
\textit{Задать} $\lambda$, $\mu$ и $n$.

\noindent
\textit{Для $n\ge 0$ рассчитать $p_{n}$ по
формулам}~\eqref{3-1} и~\eqref{3-2}.

\noindent
\textit{Рассчитать $p^{(1)}_{N-1;0}$ по формуле}~\eqref{eq-1-3}.

\noindent
\textit{Для $n\ge N$ рассчитать $p^{(1)}_{n;0}$ по
формуле}~\eqref{eq-1-1}.

\noindent
\textit{Для $i\ge1$}


\textit{рассчитать $p^{(1)}_{N-1;i}$ по формуле}~\eqref{eq-1-4}.

\textit{для $n\ge N$ рассчитать $p^{(1)}_{n;i}$ по формуле}~\eqref{eq-1-2}.

\noindent
\textit{Для $n=\overline{N-2,1}$ рассчитать $p^{(1)}_{n;0}$
по формуле}~\eqref{eq-1-5}.

\noindent
\textit{Для $m=\overline{2,N-1}$}

\textit{рассчитать $p^{(m)}_{N-1;0}$ по формуле}~\eqref{bat-3}.


\textit{для $n\ge N$ рассчитать $p^{(m)}_{n;0}$
   по формуле}~\eqref{bat-1};

\textit{для} $i\hm\ge1$

    \hspace*{9pt}\textit{рассчитать $p^{(1)}_{N-m;i}$ по
    формуле}~\eqref{eq-1-6};


    \hspace*{9pt}\textit{если $m \ne 2$, для}  $j\hm=\overline{2,m-1}$ \textit{рассчитать}\linebreak\vspace*{-12pt}

 \hspace*{9pt}\textit{$p^{(j)}_{N-m+j-1;i}$ по формуле}~\eqref{bat-6};

\hspace*{9pt}\textit{рассчитать $p^{(m)}_{N-1;i}$ по формуле}~\eqref{bat-4};

\hspace*{9pt}\textit{для $n\ge N$ рассчитать $p^{(m)}_{n;i}$
    по формуле}~\eqref{bat-2};

\textit{если {$m \ne N-1$}, для $m\hm=\overline{N-2,m}$
   рассчитать}\linebreak

   \textit{$p^{(m)}_{n;0}$ по формуле}~\eqref{bat-5}.

\bigskip

В~связи с~тем, что вычисление моментов после расчета
вероятностей по представленному алгоритму
может давать погрешности, в~следующем разделе
находятся формулы для совместного стационарного
распределения в~терминах ПФ.


\section{Использование производящих функций}

Система уравнений~\eqref{eq-1-1}--\eqref{bat-6}
допускает также решение с~помощью ПФ.
Для нахождения этого решения положим
\begin{equation*}
\label{f-m}
f_m(u,z) = \lambda u^2 - (\lambda + N\mu) u + m \mu z\,,\
 m=\overline{1,N-1}\,.
\end{equation*}

Обозначим через $u_m\hm=u_m(z)$, $m\hm=\overline{1,N-1}$,
минимальное решение уравнения
$$
f_m(u,z) = 0\,,
$$
т.\,е.
\begin{equation*}
%\label{sqrt}
u_m = \fr{\lambda + N\mu - \sqrt{(\lambda + N\mu)^2 - 4 m \lambda \mu z}}
{2 \lambda }\,.
\end{equation*}


Введем ПФ
\begin{multline*}
P^{(m)}_{n}(z) = \sum\limits_{i=0}^{\infty}
z^{i} p^{(m)}_{n;i}\,, \\
0<z<1\,, \ \ n\ge1\,,\ \
m=\overline{1,\min(n,N-1)} \,;
\end{multline*}

\vspace*{-12pt}


\noindent
\begin{multline*}
P^{(m)}(u,z) = \sum\limits_{n=N}^{\infty} u^{n-N} P^{(m)}_{n}(z)\,, \\
0<u,z<1\,, \ \ m=\overline{1,N-1}\,,
\end{multline*}
и, кроме того, положим
$$
P(u) = \sum\limits_{n=N}^{\infty} u^{n-N} p_{n}
= \fr{1}{1 - \tilde{\rho} u}\, p_N \,.
$$

Тогда, умножая~\eqref{eq-1-1} и~\eqref{eq-1-2}
на~$z^i$ и~суммируя по всем~$i$ от нуля до
бесконечности, получаем:
\begin{multline*}
%\label{eq-z-1}
(\lambda+N\mu) P^{(1)}_{n}(z) =
\lambda P^{(1)}_{n-1}(z) +
(N-1) \mu p_{n+1}
+ {}\\
{}+\mu z P^{(1)}_{n+1}(z)\,,\enskip n\ge N\,.
\end{multline*}
Умножая последнее выражение на $u^{n-N}$ и~суммируя по всем значениям $n\hm\ge N$,
после приведения подобных слагаемых имеем:
\begin{multline}
\label{eq-z-2}
f_1(u,z) P^{(1)}(u,z) =
\mu z P^{(1)}_{N}(z) -{}\\
{}- \lambda u P^{(1)}_{N-1}(z) -
(N-1) \mu [P(u) - p_{N}] \,.
\end{multline}


Теперь умножим \eqref{bat-1} и~\eqref{bat-2}
на~$z^i$ и~просуммируем по всем значениям $i\hm\ge0$.
В~результате приходим к~выражению:
\begin{multline*}
%\label{bat-2*}
(\lambda+N\mu) P^{(m)}_{n}(z) = \lambda P^{(m)}_{n-1}(z)
+{}\\
{}+(N-m) \mu P^{(m-1)}_{n+1}(z) +
m \mu z P^{(m)}_{n+1}(z) \,,\enskip n\ge N\,.
\end{multline*}
Умножая последнее выражение на $u^{n-N}$, после
суммирования по всем $n\hm\ge N$ получаем:

\pagebreak

\noindent
\begin{multline}
\label{bat-2*}
f_m(u,z) P^{(m)}(u,z) = m \mu z P^{(m)}_{N}(z)
- \lambda u P^{(m)}_{N-1}(z) -{}\\
{}-
(N-m) \mu [P^{(m-1)}(u,z) - P^{(m-1)}_{N}(z)]\,,\\ m=\overline{2,N-1}\,.
\end{multline}

Из уравнений~\eqref{eq-1-3} и~\eqref{eq-1-4}
после умножения на~$z^i$ и~суммирования по
всем значениям $i \hm\ge 0$ находим:
\begin{multline}
\label{eq-z-3}
P^{(1)}_{N-1}(z)=\fr{\lambda p_{N-2} + (N-1)\mu p_{N}}
{\lambda+(N-1)\mu }+{}\\
{}+ \fr{\mu z}{\lambda+(N-1)\mu} \,P^{(1)}_N(z)\,.
\end{multline}

Действуя аналогичным образом
с~уравнениями~\eqref{bat-3} и~\eqref{bat-4}, как и~с~уравнениями~\eqref{eq-1-3}
и~\eqref{eq-1-4}, приходим к выражению:
\begin{multline}
\label{bat-4*}
P^{(m)}_{N-1}(z) = \fr{ \lambda P^{(m-1)}_{N-2}(z) + (N-m) \mu P^{(m-1)}_{N}(z)
}{\lambda+(N-1)\mu }+{}\\
{}+\fr{m \mu z}{\lambda+(N-1)\mu}\,P^{(m)}_{N}(z) \,,\enskip m=\overline{2,N-1}\,.
\end{multline}


Домножая уравнения~\eqref{eq-1-5} и~\eqref{eq-1-6}
на~$z^i$, после суммирования по всем
значениям $i \hm\ge 0$ имеем:
\begin{multline}
\label{eq-z-4}
P^{(1)}_{n}(z)= \fr{ \lambda p_{n-1} + n \mu P^{(1)}_{n+1}(z) }{
\lambda+n\mu }+ \fr{\mu z}{\lambda+n\mu}\,P^{(2)}_{n+1}(z) \,,\\
n=\overline{1,N-2}\,.
\end{multline}

Наконец, производя аналогичные преобразования
с~уравнениями~\eqref{bat-5} и~\eqref{bat-6}, получаем:
\begin{multline}
\label{bat-6*}
P^{(m)}_n(z)= \fr {\lambda P^{(m-1)}_{n-1}(z) +
(n-m+1) \mu P^{(m)}_{n+1}(z)} {\lambda+n\mu}
+{}
\\
{}+
\fr{m \mu z}{\lambda+n\mu} P^{(m+1)}_{n+1}(z)\,,\enskip
m=\overline{2,N-2}\,,\\
n=\overline{m,N-2}\,.
\end{multline}

Уравнения~\eqref{eq-z-2}--\eqref{bat-6*} позволяют
находить выражения для всех
ПФ $P^{(m)}_{n}(z)$, $m\hm=\overline{1,N-1}$,
$n\hm=\overline{1,N-1}$, а~так\-же совместное
стационарное распределение рекуррентным образом.
Подставляя выражение для $P^{(1)}_{N-1}(z)$ из
формулы~\eqref{eq-z-3} в~формулу~\eqref{eq-z-2}, получаем:
\begin{multline}
P^{(1)}(u,z) = \left(
\left[
\mu z - \fr{\lambda \mu z u}{\lambda+(N-1)\mu}
\right] P^{(1)}_N(z) -{}\right.\\
{}-
\left[
\lambda u \fr{\lambda p_{N-2} + (N-1)\mu p_{N}}{\lambda+(N-1)\mu}+{}\right.\\
\left.\left.{}+
 (N-1) \mu [P(u) - p_{N}]
\vphantom{\fr{\lambda p_{N-2} + (N-1)\mu p_{N}}{\lambda+(N-1)\mu}}\right]
\right)
\Bigg /
f_1(u,z)\,,
\label{m25}
\end{multline}
откуда из равенства нулю в~точке $u_1(z)$ числителя и~знаменателя
правой части формулы~\eqref{m25} следует:
\columnbreak


%%%%%%%%%%%%%%%%%%%%%%%%%%%
\noindent
\begin{multline*}
%\label{r1}
P^{(1)}_N(z)= \left(
\lambda u_1(z) [\lambda p_{N-2} + (N-1)\mu p_{N}]
+{}\right.\\
{}+
\left.(\lambda+(N-1)\mu)(N-1) \mu \left[P(u_1(z)) - p_{N}\right]\right)\!\!\Big/\!\!
\left(\mu z \left[\lambda+{}\right.\right.\\
\left.\left.{}+(N-1)\mu  - \lambda u_1(z)\right]\right)\,.
\end{multline*}
%%%%%%%%%%%%%%%%%%%%%%%%%%%%%%%%%%%%%%%%%%
%%%%%%%%%%%%%%%%%%%%%%%%%%%%%%%%%%%%%%%%
Теперь, возвращаясь к~формуле~\eqref{eq-z-3},
получаем выражение для $P^{(1)}_{N-1}(z)$:
\begin{multline*}
%\label{r2}
P^{(1)}_{N-1}(z)=
\left([\lambda p_{N-2} + (N-1)\mu p_{N}]+{}\right.\\
\left.{}
+ (N-1) \mu [P(u_1(z)) - p_{N}]\right)\Big /
\left(\lambda+(N-1)\mu  - {}\right.\\
\left.{}-\lambda u_1(z)\right)\,.
\end{multline*}

Далее из равенства~\eqref{eq-z-4} выражаем $P^{(1)}_{N-2}(z)$ через
$P^{(2)}_{N-1}(z)$. Из равенства~\eqref{bat-4*} выражаем
$P^{(2)}_{N-1}(z)$ через $P^{(2)}_{N}(z)$. Подставляя полученное
выражение для $P^{(2)}_{N-1}(z)$ в~формулу~\eqref{bat-2*}, из
равенства нулю в~точке~$u_2$ левой и~правой части получившегося
равенства находим $P^{(2)}(u,z)$. Затем из равенства~\eqref{eq-z-4}
выражаем $P^{(1)}_{N-3}(z)$ через $P^{(2)}_{N-2}(z)$ и~т.\,д.

Продолжая эту процедуру, можно найти
соотношения для вычисления всех
ПФ $P^{(m)}_{n}(z)$, $m\hm=\overline{1,N-1}$, $n\hm=\overline{1,N-1}$.

С каждым шагом выражение для очередной ПФ становится все сложнее,
и~в итоге при большом числе приборов выписать явный вид всех ПФ не
удается. Тем не менее нахождение значений ПФ в~каждой точке $z \hm\ne
0$ можно свести к последовательному решению систем линейных
уравнений. Для этого обозначим через $A_n(z)$, $n\hm =\overline{2,N-1}$,
мат\-ри\-цы размера $(n+1)\times (n+1)$, име\-ющие
следующую структуру:
\begin{gather*}
\setcounter{MaxMatrixCols}{3}
A_2(z)=
\begin{pmatrix}
 2 \mu z   & 0  & -2 \mu z         \\
 - \lambda u_2(z) & - \mu z   &  \lambda +(N-1) \mu       \\
0  & \lambda +(N-2)\mu &    - \lambda
\end{pmatrix}\,;
\end{gather*}

\vspace*{-12pt}

\noindent
{ %\scriptsize
\begin{multline*}
\setcounter{MaxMatrixCols}{7}
A_n(z)=\left(
\begin{matrix}
 n \mu z   & 0  & - n \mu z &      \!\cdots\!          \\
 - \lambda u_n(z) \! & 0  & \! \lambda +(N-1) \mu \! & \!\cdots\!  \\
  \vdots   & \vdots & \vdots &  \!\cdots\! \\
 0   & 0 & 0&  \!\cdots\!  \\
 0      & 0 & 0    &     \!\cdots\! \\
 0   & - \mu z  &0   &  \cdots\! \\
0 & \!\lambda +(N-n)\mu \!&0  &  \!\cdots\!
\end{matrix}\right.\\
\left.\begin{matrix}
    \cdots\!     & 0    & 0       \\
    \cdots\!  & 0 & 0 \\
    \cdots\! & \vdots     & \vdots  \\
    \cdots\!  & - 3 \mu z     & 0   \\
    \cdots\! & \! \lambda +(N-n+2) &-2\mu z\\
    \cdots\! & - \lambda  & \! \lambda+(N-n+1)\mu\\
    \cdots\!  & 0  & - \lambda
\end{matrix}\right)\,,
\\ n =\overline{3,N-1}\,.
\end{multline*}
}

\noindent
Определим вектор-стр$\acute{\mbox{о}}$\-ки $\vec{a}_n(z)$ и~$\vec{b}_n(z)$
длины $(n+1)$ следующим образом:
\begin{multline*}
\vec{a}_n(z) = \left (
P^{(n)}_{N}(z), P^{(n)}_{N-1}(z), \dots\right.\\
\left.\dots,  P^{(2)}_{N-n+1}(z), P^{(1)}_{N-n}(z)
\right )\,,\enskip
n =\overline{2,N-1}\,;
\end{multline*}

\vspace*{-12pt}
\noindent
\begin{multline*}
\vec{b}_2(z) = \left (
(N-2) \mu [P^{(1)}(u_2,z) - P^{(1)}_{N}(z)] ,
\lambda p_{N-3}+{}\right.\\
\left.{}+ (N-2) \mu P^{(1)}_{N-1}(z),
(N-2) \mu P^{(1)}_{N}(z) \right)\,;
\end{multline*}

\vspace*{-12pt}

\noindent
\begin{multline*}
\vec{b}_n(z) = \left (
(N-n) \mu
[P^{(n-1)}(u_n,z) - P^{(n-1)}_{N}(z)],\right.
\\
\lambda p_{N-1-n}+ (N-n) \mu P^{(1)}_{N-1-(n-2)}(z),\\
(N-n) \mu P^{(n-1)}_{N}(z), (N-n)\mu  P^{(n-1)}_{N-1}(z),
\dots ,
\\
\left.
(N-n)\mu  P^{(3)}_{N-n+3}(z), (N-n)\mu  P^{(2)}_{N-n+2}(z)
\right )\,,\\
n =\overline{3,N-1}\,.
\end{multline*}
Тогда алгоритм нахождения ПФ состоит в~последовательном начиная с~$n\hm=2$ решении
системы линейных уравнений
$$
\vec{a}_n(z) A_n(z) = \vec{b}_n(z) \,.
$$
Из структуры матрицы $A_n(z)$, $n \hm=\overline{3,N-1}$, видно, что
она неприводима и~обладает свойством диагонального преобладания
т.\,е.\ перестановкой строк и~столбцов можно добиться того,
что в~каждой строке модуль диагонального элемента будет либо строго
больше, либо не меньше суммы модулей всех остальных элементов в~строке.
Покажем это. Если определить матрицы перестановки~$P^L_n$ и~$P^R_n$
размера $(n+1)\times (n+1)$ при $n \hm=\overline{3,N-1}$
следующим образом:
\begin{gather*}
\setcounter{MaxMatrixCols}{5}
P^L_n=
\begin{pmatrix}
 0   & 0  & \cdots & 0& 1 \\
 1   & 0  & \cdots & 0& 0 \\
  \vdots   &  \vdots  & \cdots &  \vdots &  \vdots \\
  0   & 0  & \cdots & 0& 0 \\
   0   & 0  & \cdots & 1& 0
\end{pmatrix}\,;
\enskip
\setcounter{MaxMatrixCols}{5}
P^R_n=
\begin{pmatrix}
 0   & 1  & \cdots & 0& 0         \\
 1   & 0  & \cdots & 0& 0 \\
   \vdots   &  \vdots  & \cdots &  \vdots &  \vdots \\
  0   & 0  & \cdots & 1& 0 \\
   0   & 0  & \cdots & 0& 1
\end{pmatrix}\,,
\end{gather*}
то матрица $P^L_n A_n(z)P^R_n$, $n \hm=\overline{3,N-1}$,
примет вид:
\begin{multline*}
\setcounter{MaxMatrixCols}{7}
P^L_n A_n(z)P^R_n={}\\
{}=\left(
\begin{matrix}
 \lambda +(N-n)\mu &0 & 0  & \cdots\\
 0  &  n \mu z   & - n \mu z &  \cdots       \\
  0  & - \lambda u_n(z) & \lambda +(N-1) \mu  & \cdots  \\
  \vdots   & \vdots & \vdots & \cdots   \\
 0   & 0 & 0& \cdots \\
 0      & 0 & 0    & \cdots  \\
 - \mu z  & 0   &0   & \cdots
\end{matrix}\right.
\end{multline*}

\noindent
\begin{equation*}
\hspace*{15mm}\left.\begin{matrix}
\cdots  & 0  & - \lambda\\
\cdots        & 0    & 0       \\
\cdots    & 0    & 0      \\
\cdots   & \vdots     & \vdots       \\
\cdots  & - 3 \mu z     & 0       \\
\cdots   & \lambda +(N-n+2) \mu & - 2 \mu z       \\
\cdots      & - \lambda  & \lambda +(N-n+1)\mu
\end{matrix}\right).
\end{equation*}
Легко видеть, что в~каждой строке модуль диагонального
элемента либо строго больше, либо не меньше суммы
модулей всех остальных элементов в~строке.
Тогда, как вытекает из следствия~6.2.27 в~\cite{horn},
у~матрицы $A_n(z)$ существует обратная
и,~значит, система $\vec{a}_n(z) A_n(z) \hm= \vec{b}_n(z)$
при $z\hm\neq 0$ имеет единственное решение.

\vspace*{-4pt}

\section{Примеры расчетов}

На основе полученных в~разд.~4 результатов {были} проведены расчеты
среднего и~дисперсии чис\-ла заявок в~БП,
а~также коэффициента корреляции числа заявок в~накопителе и~числа
заявок в~БП для различного чис\-ла
приборов~$N$~и~значений загрузки системы $\rho/N$. \mbox{Напомним}, что аналогичные
показатели были рассчитаны в~\cite{a8} по определению, на основе
стационарных вероятностей, рассчитанных по приведенному выше
алгоритму. Далее можно видеть, что результаты, полученные с~по\-мощью
ПФ, как и~ожидалось, полностью совпадают с~результатами,
представленными в~\cite{a8}.

На рис.~1 отражено поведение значения среднего числа заявок
в~БП в~зависимости от загрузки системы $\rho/N$.
Отметим, что полученные в~предыдущих  разделах результаты позволяют
рассчитывать такие
 характеристики, как среднее число заявок только
в~первой очереди в~БП, в~сумме в~первой и~во второй очередях в~БП
(когда обе очереди существуют), в~сумме в~первой, второй,\ldots ,
$(N-1)$-й очере-\linebreak\vspace*{-12pt}

\vspace*{6pt}

\begin{center}  %fig1
\vspace*{2pt}
\mbox{%
 \epsfxsize=75.145mm
 \epsfbox{pec-1.eps}
 }
\end{center}

\noindent
{{\figurename~1}\ \ \small{Поведение
 среднего числа заявок в~БП в~зависимости от загрузки
системы  $\rho/N$: \textit{1}~--- $N\hm=4$; \textit{2}~--- 7;
\textit{3}~--- $N=9$}}

%\vspace*{9pt}


\addtocounter{figure}{1}


\begin{center}  %fig2
\vspace*{2pt}
 \mbox{%
 \epsfxsize=75.027mm
 \epsfbox{pec-2.eps}
 }
 \end{center}

\noindent
{{\figurename~2}\ \ \small{Поведение среднего числа заявок в~первой
очереди в~БП~(\textit{1}), в~сумме в~первой и~во второй очередях в~БП~(\textit{2}),
в~сумме в~первой, второй и~третьей очередях в~БП~(\textit{3})
в~зависимости от загрузки системы $\rho/N$. Число
приборов $N\hm=4$}}

\vspace*{18pt}


\begin{center}  %fig3
\vspace*{2pt}
 \mbox{%
 \epsfxsize=74.929mm
 \epsfbox{pec-3.eps}
 }
 \end{center}

\noindent
{{\figurename~3}\ \ \small{Поведение
 дисперсии числа заявок в~БП в~зависимости от загрузки
системы  $\rho/N$: \textit{1}~--- $N\hm=4$; \textit{2}~--- 7; \textit{3}~--- $N=9$}}

\vspace*{18pt}

\begin{center}  %fig4
\vspace*{2pt}
 \mbox{%
 \epsfxsize=75.192mm
 \epsfbox{pec-4.eps}
 }
 \end{center}

\noindent
{{\figurename~4}\ \ \small{Поведение
 коэффициента корреляции числа заявок в~накопителе и~числа
заявок в~БП в~зависимости от загрузки системы  $\rho/N$:
\textit{1}~--- $N\hm=4$; \textit{2}~--- 7; \textit{3}~--- $N=9$}}


%\vspace*{9pt}


\noindent
дях в~БП (когда каждая из очередей существует).
Поведение данных характеристик в~зависимости от загрузки системы
$\rho/N$ для случая $N\hm=4$ пред\-став\-ле\-но на рис.~2.

На рис.~3 и~4 изображено поведение дисперсии числа
заявок в~БП и~поведение
коэффициента корреляции числа заявок в~накопителе и~числа
заявок в~БП соответственно.

Во всех расчетах интенсивность обслуживания заявок~$\mu$ принималась
равной~1.

%\addtocounter{figure}{1}
%%%%%%%%%%%%%%%%%%%%%%%%%%%%%%%%%%%%%%%%%%%%%%%%%%%%%

Анализируя графики на рис.~1--4, стоит отметить два момента. Среднее
число заявок в~БП не уходит в~бесконечность с ростом загрузки
(и~даже при загрузке больше единицы), что следует из формулы Литтла.
Число заявок в~накопителе и~число заявок в~БП весьма слабо
коррелированы, и~с~рос\-том числа приборов коэффициент корреляции
уменьшается.

\section{Заключение}

В настоящей работе рассмотрена функционирующая в~непрерывном времени
многоканальная система обслуживания с~накопителем бесконечной емкости
и~переупорядочением заявок.
В~систему поступает пуассоновский поток заявок, время
обслуживания каждым прибором распределено по
экспоненциальному закону с~одним и~тем же параметром.
Для нахождения совместного стационарного распределения
числа заявок в~накопителе и~суммарного числа
заявок в~БП получен рекуррентный алгоритм.
Также показано, как можно находить совместное распределение
в~терминах ПФ, которые облегчают расчет его моментов.

{\small\frenchspacing
 {%\baselineskip=10.8pt
 \addcontentsline{toc}{section}{References}
 \begin{thebibliography}{99}
 \bibitem{a1} %1
\Au{Boxma O., Koole G., Liu~Z.}
Queueing-theoretic solution methods for
models of parallel and distributed systems~//
Performance Evaluation of Parallel and Distributed Systems Solution
Methods, 1994. CWI Tract~105 and~106. P.~1--24.

\bibitem{a2} %2
\Au{Dimitrov B.}
Queues with resequencing. A~survey and recent results~//
{2nd World Congress on Nonlinear Analysis,
Theory, Methods, Applications Proceedings}, 1997. Vol.~30. No.\,8. P.~5447--5456.

\bibitem{a3} %3
\Au{Huisman T., Boucherie R.\,J.}
The sojourn time distribution in an infinite server
resequencing queue with dependent interarrival and
service times~// J.~Appl. Probab., 2002.
Vol.~39. No.\,3. P.~590--603.

\bibitem{a5} %4
\Au{Xia Y., Tse D.\,N.\,C.}
On the large deviations of resequencing
queue size: 2-$M$/$M$/1 сase~// IEEE Trans. Inform. Theory, 2008.
Vol.~54. No.\,9. P.~4107--4118.

\bibitem{a4} %5
\Au{Leung K., Li V.\,O.\,K.}
A~resequencing model for high-speed packet-switching networks~//
J.~Comput. Commun., 2010.
Vol.~33. No.\,4. P.~443--453.

\bibitem{a7} %6
\Au{Матюшенко С.\,И.} Стационарные характеристики двухканальной
системы обслуживания с~переупорядочением заявок и~распределениями
фазового типа~// Информатика и~её применения, 2010. Т.~4. Вып.~4.
С.~67--71.

\bibitem{a6} %7
\Au{De Nicola C., Pechinkin A.\,V., Razumchik~R.\,V.}
Stationary characteristics of homogenous Geo/Geo/2
queue with resequencing in discrete time~//
27th European Conference on Modelling and
Simulation Proceedings.~---- Aalesund, 2013. P.~594--600.

\bibitem{a7+} %8
\Au{Pechinkin A.\,V., Caraccio~I., Razumchik~R.\,V.}
Joint stationary distribution of queues in
homogenous $M\vert M\vert$3 queue with resequencing~//
28th European Conference on
Modelling and Simulation Proceedings.~--- Brescia, 2014. P.~558--564.

\bibitem{a8}
\Au{Pechinkin A.\,V., Caraccio~I., Razumchik~R.\,V.}
On joint stationary distribution in exponential
multiserver reordering queue~// 12th  Conference (International) on
Numerical Analysis and Applied Mathematics Proceedings, 2014 (in press).

\bibitem{boch}
\Au{Bocharov P.\,P., D'Apice C., Pechinkin~A.\,V., Salerno~S.}
Queueing theory.~--- Urecht, Boston: VSP, 2004. 446~p.

\bibitem{horn}
\Au{Horn R.\,A., Johnson C.\,R.}
Matrix analysis.~--- 2nd ed.~--- Cambridge: Cambridge University Press, 2013.
662~p.
 \end{thebibliography}

 }
 }

\end{multicols}

\vspace*{-9pt}

\hfill{\small\textit{Поступила в редакцию 28.10.14}}

%\newpage

\vspace*{12pt}

\hrule

\vspace*{2pt}

\hrule

%\vspace*{12pt}

\def\tit{JOINT STATIONARY DISTRIBUTION OF~THE~NUMBER OF~CUSTOMERS IN~THE~SYSTEM
AND REORDERING BUFFER IN~THE~MULTISERVER REORDERING QUEUE}

\def\titkol{Joint stationary distribution of~the~number of~customers in~the~system
and reordering buffer in~the~multiserver reordering queue}



\def\aut{\fbox{A.\,V.~Pechinkin}$^1$ and R.\,V.~Razumchik$^{1,2}$}

\def\autkol{A.\,V.~Pechinkin and R.\,V.~Razumchik}

\titel{\tit}{\aut}{\autkol}{\titkol}

\vspace*{-9pt}

\noindent
$^1$Institute of Informatics Problems, Russian Academy of Sciences,
44-2 Vavilov Str., Moscow 119333, Russian\\
$\hphantom{^1}$Federation


\noindent
$^2$Peoples' Friendship University of Russia,
6~Miklukho-Maklaya Str., Moscow 117198, Russian Federation



\def\leftfootline{\small{\textbf{\thepage}
\hfill INFORMATIKA I EE PRIMENENIYA~--- INFORMATICS AND
APPLICATIONS\ \ \ 2014\ \ \ volume~8\ \ \ issue\ 4}
}%
 \def\rightfootline{\small{INFORMATIKA I EE PRIMENENIYA~---
INFORMATICS AND APPLICATIONS\ \ \ 2014\ \ \ volume~8\ \ \ issue\ 4
\hfill \textbf{\thepage}}}

\vspace*{3pt}



\Abste{The paper considers a continuous-time multiserver queueing
system with buffer on infinite capacity and reordering. The Poisson
flow of customers arrives at the system. Service times of customers at
each server are exponentially distributed with the same parameter.
Each customer obtains a~sequential number upon arrival. The order of
customers upon arrival should be preserved upon departure from the system.
Customers whose service finished but which violated the order are kept in
the reordering buffer of infinite capacity. A~joint stationary distribution
of the number of customers in the buffer, servers, and
reordering buffer is obtained in terms of a~computational algorithm and
a~generating function. A~numerical example is provided.}


\KWE{queueing system; reordering; infinite capacity; joint distribution}

\DOI{10.14357/19922264140401}

%\vspace*{3pt}

\Ack
\noindent
The research was partially financially supported by the Russian Foundation for
Basic Research (project 13-07-00223).


  \begin{multicols}{2}

\renewcommand{\bibname}{\protect\rmfamily References}
%\renewcommand{\bibname}{\large\protect\rm References}



{\small\frenchspacing
 {%\baselineskip=10.8pt
 \addcontentsline{toc}{section}{References}
 \begin{thebibliography}{99}


 \bibitem{a1-1}
\Aue{Boxma O., G. Koole, and Z.~Liu}. 1994.
Queueing-theoretic solution methods for
models of parallel and distributed systems.
\textit{Performance Evaluation of Parallel and
Distributed Systems Solution Methods}.  CWI Tract 105
and 106:1--24.

\bibitem{a2-1}
\Aue{Dimitrov, B.} 1997.
Queues with resequencing. A~survey and recent results.
\textit{2nd World Congress on Nonlinear
Analysis, Theory, Methods, Applications Proceedings}. 30(8):5447--5456.

\bibitem{a3-1}
\Aue{Huisman, T., and R.\,J.~Boucherie}. 2002.
The sojourn time distribution in an infinite server
resequencing queue with dependent interarrival and service times.
\textit{J.~Appl. Probab}. 39(3):590--603.

\bibitem{a5-1}
\Aue{Xia, Y., and D.\,N.\,C.~Tse}. 2008.
On the large deviations of resequencing
queue size: 2-$M$/$M$/1 case.
\textit{IEEE Trans. Inform. Theory} 54(9):4107--4118.

\bibitem{a4-1} %5
\Aue{Leung, K., and V.\,O.\,K.~Li}. 2010.
A~resequencing model for high-speed
packet-switching networks.
\textit{J.~ Comput. Commun.} 33(4):443--453.

\bibitem{a7-1} %6
\Aue{Matyushenko, S.\,I.} 2010.
 Statsionarnye kharakteristiki
dvukh\-ka\-nal'\-noy sistemy obsluzhivaniya s~pe\-re\-upo\-rya\-do\-chi\-va\-ni\-em zayavok
i~raspredeleniyami
fazovogo tipa [Stationary characteristics of the two-channel
queueing system with reordering customers and distributions of phase type].
\textit{Informatika i ee Primemeniya}~--- \textit{Inform. Appl.}
4(4):67--71.

\bibitem{a6-1} %7
\Aue{De Nicola, C., A.\,V.~Pechinkin, and R.\,V.~Razumchik}. 2013.
Stationary characteristics of homogenous Geo/Geo/2
queue with resequencing in discrete time.
\textit{27th European Conference
on Modelling and Simulation Proceedings}. Aalesund. 594--600.

\bibitem{a7+-1}
\Aue{Pechinkin, A.\,V., I.~Caraccio, and R.\,V.~Razumchik}. 2014.
joint stationary distribution of queues
in homogenous $M \vert M \vert3$ queue with resequencing.
\textit{28th European Conference
on Modelling and Simulation Proceedings}. Brescia. 558--564.

\bibitem{a8-1}
\Aue{Pechinkin, A.\,V., I.~Caraccio, and R.\,V.~Razumchik}. 2014 (in press).
On joint stationary distribution in exponential
multiserver reordering queue.
\textit{12th  Conference (International) on
Numerical Analysis and Applied Mathematics Proceedings}.

\bibitem{boch-1}
\Aue{Bocharov,  P.\,P., C.~D'Apice, A.\,V.~Pechinkin, and S.~Salerno}. 2004.
\textit{Queueing theory}. Urecht, Boston: VSP. 446~p.

\bibitem{horn-1}
\Aue{Horn, R.\,A., and C.\,R.~Johnson}. 2013.
\textit{Matrix analysis}. Cambridge: Cambridge University Press. 662~p.
\end{thebibliography}

 }
 }

\end{multicols}

\vspace*{-6pt}

\hfill{\small\textit{Received October 28, 2014}}

\vspace*{-18pt}

\Contr

\noindent
\textbf{Pechinkin Alexander V.} (1946--2014)~--- Doctor
of Science in physics and mathematics; principal
scientist, Institute of Informatics Problems of
the Russian Academy of Sciences, 44-2 Vavilov Str.,
Moscow 119333, Russian Federation


\vspace*{3pt}

\noindent
\textbf{Razumchik Rostislav V.} (b.\ 1984)~--- Candidate
of Science (PhD) in physics and mathematics,
senior scientist, Institute of Informatics
Problems of the Russian Academy of Sciences, 44-2 Vavilov Str.,
Moscow 119333, Russian Federation;
associate professor,
Peoples' Friendship University of Russia,
6~Miklukho-Maklaya Str., Moscow 117198, Russian Federation;
rrazumchik@ieee.org


\label{end\stat}

\renewcommand{\bibname}{\protect\rm Литература}   %2
%\newcommand{\A}{{\mathbf A}}
%\newcommand{\B}{{\mathbf B}}
%\newcommand{\la}{{\lambda}}
%\newcommand{\be}{\begin{equation}}
%\newcommand{\ee}{\end{equation}}
%\newcommand{\ber}{\begin{eqnarray}}
%\newcommand{\eer}{\end{eqnarray}}

%\newcommand{\nin}{\noindent}
%\newcommand{\non}{\nonumber}
%\newcommand{\half}{\frac{1}{2}}
%\newcommand{\quarter}{\frac{1}{4}}

\def\stat{zeifman}

\def\tit{ОБ ОДНОМ КЛАССЕ МАРКОВСКИХ СИСТЕМ ОБСЛУЖИВАНИЯ$^*$}

\def\titkol{Об одном классе марковских систем обслуживания}

\def\autkol{Я.\,А.~Сатин, А.\,И.~Зейфман, А.\,В.~Коротышева, С.\,Я.~Шоргин}
\def\aut{Я.\,А.~Сатин$^1$, А.\,И.~Зейфман$^2$, А.\,В.~Коротышева$^3$, С.\,Я.~Шоргин$^4$}

\titel{\tit}{\aut}{\autkol}{\titkol}

{\renewcommand{\thefootnote}{\fnsymbol{footnote}}\footnotetext[1]
{Исследование поддержано РФФИ, гранты 11-07-00112-а и 11-01-12026-офи-м.}}


\renewcommand{\thefootnote}{\arabic{footnote}}
\footnotetext[1]{Вологодский государственный педагогический
университет, yacovi@mail.ru}
\footnotetext[2]{Вологодский государственный педагогический университет;  
Институт проблем информатики Российской академии наук; 
Институт социально-экономического развития территорий Российской академии наук,  a\_zeifman@mail.ru}
\footnotetext[3]{Вологодский государственный педагогический
университет,  a\_korotysheva@mail.ru}
\footnotetext[4]{Институт проблем информатики Российской академии наук, SShorgin@ipiran.ru}


\Abst{Рассматриваются модели обслуживания, описываемые конечными марковскими 
цепями с непрерывным временем. При этом предполагается,  что интенсивности 
поступления и обслуживания требований не зависят от числа требований в сис\-те\-ме. 
Получены оценки скорости сходимости и устойчивости различных характеристик таких сис\-тем.}

\KW{нестационарные марковские системы
обслуживания; скорость сходимости; устойчивость; оценки}

 \vskip 14pt plus 9pt minus 6pt

      \thispagestyle{headings}

      \begin{multicols}{2}
      
            \label{st\stat}

\section{Введение}

Классы систем массового обслуживания, описываемых процессами
рождения и гибели (стационарными и нестационарными, с катастрофами)
изучались начиная с 1970-х~гг.\ многими авторами
(см., например,~[1--6]). С~помощью методов,
разработанных одним из авторов настоящей \mbox{статьи}\linebreak (подробное изложение
этих методов приведено в~[7--9]), для таких сис\-тем
удалось получить точные оценки скорости сходимости и устойчивости.

Оказывается, этот же подход можно применить и к существенно более 
общему классу систем обслуживания.

Рассмотрим систему массового обслуживания, число требований в которой 
описывается нестационарной марковской цепью с непрерывным временем и 
конечным пространством состояний, причем требования могут поступать и 
обслуживаться группами.

Пусть $X=X(t)$, $t\geq 0$,~--- число требований в системе обслуживания ($0 \hm\le X(t) \hm\le r$).

Обозначим через 
\begin{gather*}
p_{ij}(s,t)=\mathrm{Pr}\left\{ X(t)=j\left| X(s)=i\right.
\right\}\,,\\
i,j \ge 0\,,\ 0\leq s\leq t\,,
\end{gather*}
переходные вероятности
процесса $X\hm=X(t)$, а через  $p_i(t)\hm=\mathrm{Pr}\left\{ X(t) \hm=i \right\}$~---
его вероятности состояний.

Будем предполагать, что интенсивности поступления и обслуживания $k$ требований в 
момент~$t$ в сис\-те\-ме об\-слу\-жи\-ва\-ния ($\lambda_{k}(t)$ и  $\mu_{k}(t)$ соответственно)  
не зависят от числа требований, находящихся в системе в момент~$t$, являются локально 
интегрируемыми на $[0,\infty)$ функциями времени~$t$ и, кроме того, 
$\lambda_{k+1}(t) \hm\le \lambda_{k}(t)$ и  $\mu_{k+1}(t) \hm\le \mu_{k}(t)$ при всех~$k$ 
и почти при всех $t \hm\ge 0$.

Тогда для описания вероятностной динамики процесса получаем прямую систему Колмогорова в виде
\begin{equation} 
\fr{d\vp}{dt}=A(t)\vp(t)\,,
\label{ur_1}
\end{equation}
 где
 {\footnotesize
\begin{multline*}
A(t)={}\\
{}=
\begin{pmatrix}
a_{00}(t) & \mu_1(t)  & \mu_2(t)   & \mu_3(t)  & \mu_4(t) & \cdots & \mu_r(t) \\
\la_1(t)   & a_{11}(t)  & \mu_1(t)  & \mu_2(t)   & \mu_3(t)  & \cdots & \mu_{r-1}(t) \\
\la_2(t)  & \la_1(t)    & a_{22}(t)& \mu_1(t)  & \mu2(t)    &  \cdots & \mu_{r-2}(t) \\
\cdots&\cdots&\cdots&\cdots&\cdots&\cdots&\cdots \\
\la_r(t) & \la_{r-1}(t) & \la_{r-2}(t) & \cdots & \la_2(t)  & \la_1 (t)   &  a_{rr}(t)
\end{pmatrix}\,,
\end{multline*}}
причем  
$$
a_{ii}(t)=-\sum\limits_{k=1}^{i}\mu_k(t) - \sum\limits_{k=1}^{r-i} \la_{r-k}(t)\,.
$$

Далее будем обозначать через $\|\bullet\|$  $l_1$-нор\-му, т.\,е.\ 
$\|{\vx}\|\hm=\sum|x_i|$, а $\|B\| \hm= \max\limits_j \sum\limits_i |b_{ij}|$, 
если $B \hm= (b_{ij})_{i,j=0}^{r}$.
%
Тогда, в частности, имеем 
$$
\|A(t)\| \le 2\sum\limits_{k=1}^{r}(\la_{k}(t)+ \mu_k(t))
$$ 
при  всех $t \hm\ge 0$.

Через 
$$
E(t,k) = E\left\{X(t)\left|X(0)\hm=k\right.\right\}
$$ 
будем далее обозначать математическое ожидание процесса (среднее число требований) в момент~$t$ 
при условии, что в нулевой момент времени он находится в состоянии~$k$, 
а через $E_{\bf p}(t)$ обозначим математическое ожидание процесса в момент~$t$ 
при начальном распределении вероятностей состояний $\mathbf{p}(0) \hm= \mathbf{p}$.

\section{Оценки скорости сходимости}

Рассмотрим вспомогательную последовательность положительных чисел $\{d_i\}$, $i\hm=1, \dots,r$.

Положим
\begin{equation*}
d=\min\limits_{1 \le i \le r} d_i\,; \enskip 
G=\sum\limits_{i=1}^r d_i\,; \enskip W=\min\limits_k \fr{d_k}{k}\,.
%\label{2.01}
\end{equation*}

Рассмотрим величины
\begin{multline*}
\alpha_i(t)= -a_{ii}(t)+\la_{r-i+1}(t)-\sum\limits_{k=1}^{i-1}(\mu_{i-k}(t)-{}\\
{}-
\mu_i(t))\fr{d_k}{d_i}-\sum\limits_{k=1}^{r-i}(\la_k(t)-\la_{i+r-1}(t))\fr{d_{k+i}}{d_i}\,,
%\label{2.02}
\end{multline*}

\noindent
\begin{equation*}
\alpha(t)=\min\limits_{1 \le i \le r}\alpha_i(t)\,.
%\label{2.03}
\end{equation*}

\smallskip

\noindent
\textbf{Теорема~1.} \textit{Пусть существует последовательность положительных 
чисел  $\{d_j\}$ такая, что}
\begin{equation}
\int\limits_0^{\infty} \alpha(t)\, dt = + \infty\,.
\label{2.031}
\end{equation}
\textit{Тогда $X(t)$ слабо эргодичен, при
любых начальных условиях} $\mathbf{p}^*(s)$, $\mathbf{p}^{**}(s)$ 
\textit{и любых $s$, $t$, $0\le s\le t$, справедлива оценка
\begin{equation} 
\label{2.04}
\|\vp^*(t)-\vp^{**}(t)\| \le \fr{8G}{d}\,e^{-\int\limits_s^t {\alpha(u)\,du}}\,.
\end{equation}
Кроме того,  $X(t)$ имеет предельное среднее $\phi(t)$ и при любых~$k$ и $t \hm\ge 0$ справедливо неравенство}:
\begin{equation}
\label{2.05}
|E(t,k)-\phi(t)|\le \fr{4G}{W}\,e^{-\int\limits_0^t {\alpha(u)\,du}}\,.
\end{equation}


\smallskip


\noindent
Д\,о\,к\,а\,з\,а\,т\,е\,л\,ь\,с\,т\,в\,о\,.\

Пользуясь предложенным в предыдущих работах способом, 
выразим 
$$
p_0=1-\sum\limits_{1\le i \le r}{p_i}\,.
$$

Тогда получим неоднородное уравнение:
\begin{equation} 
\label{ur_per}
\fr{d\vz}{dt}= B(t)\vz(t)+\vf(t)\,, 
%\label{2.06}
\end{equation}
\noindent
где $\vf(t)=\left(\la_1, \la_2,\cdots,\la_r \right)^{\mathrm{T}}$;

\end{multicols}


\hrule

\vspace*{6pt}

\begin{equation*}
B = \left(
\begin{array}{cccccccc}
a_{11}- \la_1   & \mu_1 - \la_1   & \mu_2 - \la_1   & \mu_3 -\la_1   & \cdots& \cdots & \mu_{r-1}- \la_1  \\
\la_1 -\la_2    & a_{22} -\la_2  & \mu_1-\la_2   & \mu_2 -\la_2     & \cdots&  \cdots & \mu_{r-2} -\la_2 \\
\la_2 -\la_3    & \la_1 -\la_3   & a_{33} -\la_3  & \mu_1-\la_2   & \cdots&  \cdots & \mu_{r-3} -\la_3 \\
\cdots&\cdots&\cdots&\cdots&\cdots&\cdots&\cdots \\
\la_{r-1} -\la_r  &\la_{r-2} -\la_r & \cdots & \cdots & \la_2 -\la_r   & \la_1 -\la_r     &  a_{rr} -\la_r
\end{array}
\right)\,.
%\label{2.07}
\end{equation*}

Рассмотрим треугольную матрицу
\begin{equation*}
D=\begin{pmatrix}
d_1   & d_1 & d_1 & \cdots & d_1 \\
0   & d_2  & d_2  &   \cdots & d_2 \\
\cdots&\cdots&\cdots&\cdots&\cdots \\
0  & 0 & \cdots & 0 &  d_r
\end{pmatrix}
%\label{2.08}
\end{equation*}
и соответствующую норму $\|{\bf z}\|_{D}\hm=\|D {\bf z}\|_1$.

Тогда имеем:
\begin{equation*}
 D BD^{-1}=\left(
\begin{array}{ccccccc}
a_{11}-\la_r  &  (\mu_1-\mu_2) \fr{d_1}{d_2}  & (\mu_2-\mu_3)\fr{d_1}{d_3}  & \cdots &  (\mu_{r-1}-\mu_r)\fr{d_1}{d_r} \\
(\la_1-\la_r) \fr{d_2}{d_1} &  a_{22}-\la_{r-1}  &(\mu_1-\mu_3)\fr{d_2}{d_3}  & \cdots &  (\mu_{r-2}-\mu_r)\fr{d_2}{d_r} \\
(\la_2-\la_r) \fr{d_3}{d_1} &  (\la_1-\la_{r-1})\fr{d_3}{d_2}   &a_{33}-\la_{r-2}   & \cdots &  (\mu_{r-3}-\mu_r)
\fr{d_3}{d_r}  \\
\cdots&\cdots&\cdots&\cdots&\cdots \\
(\la_{r-1} -\la_r) \fr{d_r}{d_1} & (\la_{r-2} -\la_{r-1}) \fr{d_r}{d_2}  & (\la_{r-3} -\la_{r-2}) \fr{d_r}{d_3}  & \cdots & a_{rr}-\la_1 \\
\end{array}
\right)\,.
%\label{2.09}
\end{equation*}


\begin{multicols}{2}


Далее, оценивая логарифмическую норму оператора~$B(t)$ (см., например, 
подробное рассмотрение в~[8--10]), получаем
\begin{multline*}
\gamma \left(B(t)\right)_{1D} = \gamma \left(DB(t)D^{-1}\right)_{1}={}\\
{}=
\max \left(\vphantom{\sum\limits_{k=1}^{i-1}}
a_{ii}(t) - \la_{r-i+1}(t) + \sum\limits_{k=1}^{i-1}\left(\mu_{i-k}(t)-{}\right.\right.\\
\left.\left.{}-\mu_i(t)\right)
\fr{d_k}{d_i} +
\sum\limits_{k=1}^{r-i}(\la_k(t)-\la_{i+r-1}(t))\fr{d_{k+i}}{d_i}\right) ={}\\
{}=
 - \min \alpha_i(t) = - \alpha(t)\,.
% \label{2.10}
\end{multline*}
Тогда\\[-7.9pt]
\begin{equation*}
\|\vz^*(t)-\vz^{**}(t)\|_{1D}\le  e^{-\int\limits_s^t {\alpha(u)du}}\|\vz^*(s)-\vz^{**}(s)\|_{1D}
%\label{2.11}
\end{equation*}
для всех $0 \le s \le t$ и любых начальных условий $\vz^*(s)$, $\vz^{**}(s)$.

Теперь, учитывая оценки для сравнения норм (см., например,~\cite{z08b}), получаем:
\begin{multline*}
\|\vp^*(t)-\vp^{**}(t)\| \le 2\|\vz^*(t)-\vz^{**}(t)\| \le{}\\
{}\le  \fr{4}{d}\|\vz^*(t)-\vz^{**}(t)\|_{1D}\le{} \\
{} \le \fr{4}{d}\,e^{-\int\limits_s^t {\alpha(u)\,du}}\|\vz^*(s)-\vz^{**}(s)\|_{1D} 
\le{}\\
{}\le
 \fr{4G}{d}\,e^{-\int\limits_s^t {\alpha(u)\,du}}\|\vz^*(s)-\vz^{**}(s)\| \le{} \\
{} \le  \fr{4G}{d}\,e^{-\int\limits_s^t {\alpha(u)\,du}}\|\vp^*(s)-\vp^{**}(s)\| \le 
\fr{8G}{d}\,e^{-\int\limits_s^t {\alpha(u)\,du}} 
%\label{2.11-a}
\end{multline*}
для любых начальных условий ${\bf p^*}(s)$, ${\bf p^{**}}(s)$ и любых $s,t$, $0\hm\le s\hm\le t$.

Из слабой эргодичности процесса с конечным пространством состояний 
вытекает существование предельного среднего, начальные условия для которого можно 
в общем случае выбрать произвольно.
Для оценки средних воспользуемся неравенством, приведенным в параграфе~2.3 из~\cite{z08b}:
\begin{multline*}
\|{\bf z}\|_{1D} = d_0 \left|\sum\limits_{i=1}^{\infty} p_i \right|
+ d_1 \left|\sum\limits_{i=2}^{\infty} p_i \right| + \dots \ge{}\\
{}\ge 
 W \sum\limits_{k \ge 1} k \left|\sum\limits_{i \ge k} p_i\right| \ge \fr{W}{2}
\sum\limits_{k \ge 1} k \left|p_k\right|\,.  
%\label{2.12}
\end{multline*}
Получаем теперь
\begin{multline*}
|E(t,k)-\phi(t)|\le \fr{2}{W}\,\|\vp^*(t)-\vp^{**}(t)\|_{1D}\le {} \\
{}\le\fr{2}{W}\,e^{-\int\limits_0^t {\alpha(u)\,du}}\|{\bf e}_k -
\vp^{**}(0)\|_{1D} \le \frac{4G}{W}e^{-\int\limits_0^t
{\alpha(u)\,du}}\,,
%\label{2.13}
\end{multline*}
что и требовалось доказать.
\columnbreak

%\smallskip

\noindent
\textbf{Замечание~1.} {Положим в условиях теоремы~1 
$$
\beta(t)=\max\limits_{1 \le i \le r}\alpha_i(t)\,.
$$ 
Тогда, пользуясь внедиагональной неотрицательностью матрицы $DB(t)D^{-1}$ 
с помощью методики, описанной в~\cite{z08b, z95b}, получаем справедливость неравенства

\noindent
\begin{equation*} 
%\label{2.14}
\|\vp^*(t)-\vp^{**}(t)\| \ge \fr{d}{8G}\,e^{-\int\limits_s^t {\beta(u)\,du}}
\end{equation*}
при любых $s$, $t$, $0\le s\le t$ и уже не при любых начальных условиях~${\bf p^*}(s)$, 
${\bf p^{**}}(s)$, а таких, что  $D\left({\bf p^*}(s) \hm-{\bf p^{**}}(s)\right) \hm\ge 0.$ 
Следовательно, оценки тео\-ре\-мы~1 будут заведомо иметь точный по времени порядок, если удастся 
выбрать вспомогательную последовательность $\{d_i\}$ так, что $\alpha(t)\hm=\beta(t)$, т.\,е.\ 
все $\alpha_i(t)$ одинаковы (не зависят от индекса~$i$)}.



\smallskip

Введем теперь в рассмотрение величины

\vspace*{-1pt}

\noindent
\begin{multline*}
\zeta_i(t)= -a_{ii}(t)+\la_{r-i+1}(t)+{}\\
{}+\sum\limits_{k=1}^{i-1}\left(\mu_{i-k}(t)-
\mu_i(t)\right) \fr{d_k}{d_i}+{}\\
{}+\sum\limits_{k=1}^{r-i}\left(\la_k(t)-\la_{i+r-1}(t)\right)\fr{d_{k+i}}{d_i}\,;
%\label{2.0211}
\end{multline*}
\begin{equation*}
\chi(t)=\max\limits_{1 \le i \le r}\zeta_i(t)\,.
%\label{2.0311}
\end{equation*}

\noindent
\textbf{Замечание 2.} {В условиях теоремы~1 при любых начальных условиях 
${\bf p^*}(s)$, ${\bf p^{**}}(s)$ и любых $s,t$,  $0\le s\le t$, 
справедлива следующая двухсторонняя оценка скорости сходимости:

\vspace*{-1pt}

\noindent
\begin{multline*} 
%\label{2.041}
\!\!\!\fr{d}{4G}\,e^{-\int\limits_s^t {\chi(u)\,du}}\|\vp^*(s)-\vp^{**}(s)\| \le
 \|\vp^*(t)-\vp^{**}(t)\| \le {}\\
 {}\le\fr{4G}{d}\,e^{-\int\limits_s^t {\alpha(u)\,du}}\|\vp^*(s)-\vp^{**}(s)\|.
\end{multline*}
Таким образом, можно оценить и сверху и снизу время  вхождения 
сис\-те\-мы обслуживания в предельный режим. Более подробно о получении 
нижних оценок см., например, в~\cite{z95b, gz05}.}

\smallskip

Рассмотрим два частных случая теоремы.

\smallskip

\noindent
\textbf{Следствие 1}. \textit{Пусть при выполнении остальных условий теоремы~1 
вместо}~(\ref{2.031}) \textit{выполняется условие $\alpha(t) \hm\ge \alpha \hm> 0$ 
почти при всех $t \hm\ge 0$. Тогда вместо}~(\ref{2.04}) \textit{и}~(\ref{2.05}) 
\textit{справедливы оценки}:

\vspace*{-1pt}

\noindent
\begin{align*} 
%\label{2.15}
\|\vp^*(t)-\vp^{**}(t)\| &\le \fr{8G}{d}\,e^{-\alpha \left(t-s\right)}\,;
\\
%\label{2.16}
|E(t,k)-\phi(t)|&\le \fr{4G}{W}\,e^{- \alpha t}\,.
\end{align*}

\pagebreak

%\smallskip

Положим 
\begin{gather*}
M_0=\max\limits_{|t-s|\le 1}\int\limits_s^t \alpha(u)\,du;\\
\alpha^* = \int\limits_0^1 \alpha(t)\, dt\,; \quad
M=e^{M_0+\alpha^*}\,.
\end{gather*}
С учетом неравенства 
$$
e^{-\int\limits_s^t {\alpha(u)\,du}} \hm\le M e^{-\alpha^* (t-s)}
$$ 
получаем следующее утверждение.

\smallskip

\noindent
\textbf{Следствие~2.} \textit{Пусть все $\lambda_k(t)$ и $\mu_k(t)$ 1-пе\-ри\-одич\-ны,  
а при выполнении остальных условий теоремы~1 вместо}~(\ref{2.031}) 
\textit{выполняется условие  $\alpha^* \hm> 0$.  Тогда предельный режим (скажем, $\vp^*(t)$) 
и соответствующее ему предельное среднее $\phi^*(t)$ можно выбрать 
1-пе\-ри\-оди\-че\-ски\-ми, а вместо}~(\ref{2.04}) \textit{и}~(\ref{2.05}) 
\textit{справедливы оценки}:
\begin{equation*} 
%\label{2.17}
\|\vp(t) - \vp^*(t)\| \le \fr{8GM}{d}\,e^{-\alpha^*t}
\end{equation*}
\textit{и, кроме того,}
\begin{equation*}
|E(t,k)-\phi^*(t)|\le \fr{4GM}{W}\,e^{-\alpha^*t}
%\label{2.18}
\end{equation*}
\textit{при любом $k$ и $t \ge 0$}.



\section{Устойчивость}

Рассмотрим также <<возмущенный>> процесс обслуживания $\bar{X}\hm=\bar{X}(t)$, $t\hm\geq 0$, 
в котором интенсивности поступления и обслуживания требований также не зависят от чис\-ла 
требований в системе, обозначая его соответствующие характеристики теми же буквами с 
чертой сверху. Для прос\-то\-ты записи оценок будем предполагать, что возмущения 
<<равномерно малы>>, т.\,е.\ выполняется неравенство $\| A(t)-\bar{A}(t)\| \hm\le \varepsilon$. 
Первые результаты для нестационарных цепей с непрерывным временем получены в~\cite{z85}, 
а детальное рассмотрение для более общего случая неравномерных оценок можно без труда 
провести так же, как это сделано в~\cite{z98, ae}. Для получения требуемых равномерных 
оценок устойчивости необходима экспоненциальная эргодичность соответствующего процесса, 
т.\,е.\ существование положительных констант $N$, $a$ таких, что  для правой части~(\ref{2.04}) 
справедливо неравенство:
\begin{equation}
e^{-\int\limits_s^t {\alpha(u)\,du}} \le Ne^{-a\left(t-s\right)}\,.
\label{3.01}
\end{equation}
Оценка~(\ref{3.01}) заведомо имеет место, в частности, если выполнены условия одного из следствий 
предыду\-ще\-го параграфа.

\smallskip

\noindent
\textbf{Теорема~2.}
\textit{Пусть выполнены условия теоремы~1 и}~(\ref{3.01}). \textit{Тогда при
 любых начальных условиях ${\bf p}(s)$ и ${\bar{\bf p}}(s)$ для процессов~$X(t)$ 
 и $\bar{X}(t)$ соответственно справедливы следующие оценки устойчивости:}
\begin{align*} 
%\label{3.02}
\limsup_{t \to \infty}  \|{\bf p}(t)- \bar{\bf p}(t)\| &\le
\fr{\varepsilon(1+\ln(4GN/d))}{a}\,;
\\
% \label{3.03}
\limsup\limits_{t \to \infty}   |E_{\bf p}(t)- \bar{E}_{\bar{\bf p}(t)}|&\le 
\fr{r \varepsilon(1+\ln(4GN/d))}{a}\,.
\end{align*}


\smallskip

\noindent
Д\,о\,к\,а\,з\,а\,т\,е\,л\,ь\,с\,т\,в\,о\ основано на подходе, 
введенном для стационарных процессов в~\cite{mit03} и описанном для нестационарной 
ситуации в~\cite{z11}.
Если  при любых начальных условиях для исходного процесса справедлива оценка
\begin{equation*} 
%\label{3.04}
\|\vp(t) - \vp^*(t)\| \le ce^{-b\left(t-s\right)}\,,
\end{equation*}
то, полагая
\begin{multline*}
\beta (t, s)=\sup\limits_{ \| {\bf v} \| =1, \sum {v_i}=0}
{\|V(t,s){\bf v}(t,s)\|} ={}\\
{}= \fr{1}{2} \max_{i,j} \sum\limits_k {|p_{ik}(t,
s)-p_{jk}(t, s)|}\,, 
\end{multline*}
где $V(t, s)$~--- матрица Коши
уравнения~(\ref{ur_1}), получаем в итоге следующее неравенство:
\begin{equation*}
\|{\bf p}(t)-\bar{\bf p}(t)\| \le{}
\begin{cases}
\|{\bf p}(s)-{\bf \bar{p}}(s)\|+ (t-s)\varepsilon \,, &\\
&\hspace*{-35mm} 0<t< b^{-1} \ln \left(\fr{c}{2}\right)\,; \\
b^{-1}\left(\ln \fr{c}{2} +1-\fr{c}{2}\,e^{-b(t-s)}\right)\varepsilon +{}&\\
{}+
\fr{c}{2}\,e^{-b(t-s)} \|{\bf p}(s)-{\bf \bar{p}}(s)\|\,, &\\
&\hspace*{-30mm}t\ge b^{-1}\ln \left(\fr{c}{2}\right)
\end{cases}
%\label{3.05}
\end{equation*}
для любых начальных условий ${\bf p}(s)$ и $\bar{\bf p}(s)$.
Из неравенств~(\ref{2.04}) и~(\ref{3.01}) вытекает, что $b=a$, $c={8GN}/{d}$.  
Устремив $t \hm\to \infty$ и взяв $s\hm=0$, получаем требуемые оценки.


\smallskip

\noindent
\textbf{Замечание~3.} 
В полученную оценку устойчивости для математического ожидания процесса 
в качестве множителя входит размерность~$r$, поэтому иногда лучший результат 
удается получить при помощи другого подхода, описанного в работе~\cite{z11}.

\smallskip

Положим 
$$
S=\max\limits_{{1 \le i, j \le r}} \fr{d_i}{d_j}\,,
$$ 
и пусть числа $K, L$ таковы, что 

\noindent
$$
d_1\la_1(t) + (d_1+d_2)\la_2(t) + \dots + 
\left(\sum\limits_{1 \le i \le r}d_i\right) \la_r(t) \le K\,,
$$ 
а 

\noindent
\begin{multline*}
d_1(\la_1(t)-\bar{\la}_1(t)) + (d_1+d_2)(\la_2(t)-\bar{\la}_2(t)) + \dots\\
\dots + 
\left(\sum\limits_{1 \le i \le r}d_i\right) (\la_r(t)-\bar{\la}_r(t)) \le 
L\varepsilon
\end{multline*} 
почти при всех $t \ge 0.$

\smallskip

\noindent
\textbf{Теорема~3.}
\textit{Пусть  выполнены условия теоремы~2 и, кроме того, при всех~$k$ 
и почти всех $t \hm\ge 0$ $\la_k(t) \hm< \infty$. Тогда при любых начальных условиях 
${\bf p}(s)$ и ${\bar{\bf p}}(s)$ для процессов $X(t)$ и $\bar{X}(t)$ 
соответственно справедливо неравенство}

\noindent
\begin{equation*}
\limsup\limits_{t \to \infty}   |E_{\bf p}(t)- \bar{E}_{\bar{\bf p}(t)}|\le 
\fr{ N\varepsilon\left(L a+ 2KNS\right)}{W a \left(a-2\varepsilon S\right)}\,.
\end{equation*}


\smallskip

\noindent
Д\,о\,к\,а\,з\,а\,т\,е\,л\,ь\,с\,т\,в\,о.\
 Перепишем исходную систему~(\ref{ur_per}) для невозмущенного процесса в следующем виде:
 \noindent
 
\begin{equation*}
\fr{d\vp}{dt}=\bar{B}(t)\vp(t) + {\bf f}(t)+\left(B(t)-\bar{B}(t)\right)\vp(t)\,.
%\label{eq112-n}
\end{equation*}
Тогда

\noindent
\begin{multline*}
\vp(t)=\bar{U}(t,0)\vp(0)+\int\limits_0^t \bar{U}(t,\tau){\bf{f}}(\tau) \, d\tau+{}\\
{}+\int\limits_0^t \bar{U}(t,\tau) \left(B(\tau)-\bar{B}(\tau)\right)\vp(\tau)\, d\tau\,;
\end{multline*}

\vspace*{-9pt}

\begin{equation*}
\hspace*{-15mm}\bar{\vp}(t)=\bar{U}(t,0)\bar{\vp}(0)+\int\limits_0^t \bar{U}(t,\tau){\bf{f}}(\tau) \, d\tau,
\end{equation*}
где $U(t,s)$~--- матрица Коши для уравнения~(\ref{ur_per}).
В любой норме при одинаковых начальных условиях получаем следующую оценку:
%\noindent
\begin{multline}
 \label{3000}
\!\!\!\!\!\!\left\|\vp(t)-\bar{\vp}(t)\right\|\le \!\!\int\limits_0^t \!\!\|\bar{U}(t,\tau)\|
\left(\| B(\tau)-\bar{B}(\tau)\| \|\vp(\tau)\| +\right.\\
\left.{}+ \| \vf(\tau)-\bar{\vf}(\tau)\|\right)\,d\tau\,.\!
\end{multline}
Имеем почти при всех $t \ge 0$:
\begin{equation*}
\|B(t)-\bar{B}(t)\|_{1D}=\|D(B(t)-\bar{B}(t))D^{-1}\| \le 2S\varepsilon\,;
%\label{3002}
\end{equation*}
%
%\vspace*{-14pt}
%
%\noindent
\begin{multline*}
\|{\bf f}(t)\|_{1D} \le d_1\la_1(t) + (d_1+d_2)\la_2(t) + \dots + {}\\
{}+
\left(\sum\limits_{1 \le i \le r}d_i\right) \la_r(t) \le K\,, 
\quad \|\vf(\tau)-\bar{\vf}(\tau)\|_{1D} \le L\varepsilon\,.
%\label{3002-a}
\end{multline*}
А тогда
\begin{multline*}
\gamma(\bar{B}(t))_{1D} \le \gamma(DB(t)D^{-1})+\|B(t)-\bar{B}(t)\|_{1D} \le  {}\\
{}\le -
\alpha(t)+2S \varepsilon \,.
% \label{3003}
\end{multline*}

Оценим теперь
\begin{multline*} 
%\label{8402}
\!\|{\bf p}(t)\|_{1D} \le
\|U(t){\bf p}(0) \|_{1D} +
 \int\limits_0^t \!\!\| U(t,\tau){\bf f}(\tau)\, d\tau \|_{1D} \le {}\\
 {}\le
 N e^{-a t} \| \vp(0)\|_{1D}  + \fr{K N}{a}.
\end{multline*}

 Тогда с учетом~(\ref{3000}) получаем:
\begin{multline*} 
%\label{3004}
\left\|\vp(t)-\bar{\vp}(t)\right\|_{1D}\le N\int\limits_0^t e^{-(a - 2\varepsilon S)(t-\tau)}\times{}\\
{}\times
\left(2S\varepsilon (N e^{-a \tau} \| \vp(0)\|_{1D}  + \fr{K N}{a}) +  L\varepsilon \right)\, d\tau  \le {} \\
{}\le  o(1)+\fr{ N\varepsilon(L+{2KNS}/{a})}{a-2\varepsilon S}\,. 
\end{multline*}

\vspace*{-9pt}

\section{Примеры}

\noindent
\textbf{Пример 1.}

Рассмотрим исходный процесс обслуживания с интенсивностями 
$\la_1(t)\hm=\la_2(t)\hm=\la_3(t)\hm=\la(t) \hm= 3\hm+\sin{2\pi t}$, 
$\mu_1(t)\hm=\mu_2(t)\hm=  \mu(t) \hm= 2\hm+\cos{2\pi t}$, 
$\la_4(t)=\ldots=\la_r(t)\hm=\mu_3(t)=\ldots=\mu_r(t)\hm=0$. Выберем последовательность  
$d_k\hm=h^k$, где $0{,}82 \hm< h \hm<1$. Тогда имеем
$$
d=h^r\,; \quad G \le \fr{h}{1-h}\,; \quad W=\fr{h^r}{r}\,.
$$

Будем предполагать, что возмущенный процесс имеет такую же структуру 
мат\-ри\-цы интенсивностей, причем $|\la(t)\hm-\bar{\la}(t)| \hm\le \varepsilon$ 
и  $|\mu(t)\hm-\bar{\mu}(t)| \hm\le \varepsilon$ почти при всех $t \hm\ge 0$. 
Отметим кстати, что при этом $\| A(t)\hm-\bar{A}(t)\| \hm\le 10 \varepsilon$ почти при 
всех $t \hm\ge 0$. Рассмотрим дальнейшие оценки:
$$
S=\fr{1}{h^2}\,; \ K=4 \left(3h+2h^2+h^3\right)\,; \ L=3h+2h^2+h^3\,;
$$
$$
\alpha(t) \ge \la(t)\left(3 - h - h^2 -h^3\right)-\mu(t)\left(\fr{1}{h^2}+\fr{1}{h}-2\right)\,;
$$
$$
\alpha^*= 3\left(3 - h - h^2 -h^3\right)-2\left(\fr{1}{h^2}+\fr{1}{h}-2\right)\,;
$$


\noindent
\begin{multline*}
M_0 \le \int\limits_0^1 |\alpha(t)|\, dt \le 4\left(3 - h - h^2 -h^3\right)+{}\\
{}+
3\left(\fr{1}{h^2}+\fr{1}{h}-2\right)\,;
\end{multline*}

\vspace*{-9pt}

\noindent
$$
M=e^{\alpha^*+M_0}\,.
$$

Если, например, взять 
$h\hm=0{,}9$, то $\alpha^*\hm=0{,}992$, $M_0\hm=3{,}281$, $M\hm=71{,}737$.

Тогда получаем следующие оценки.

По следствию~2
\begin{align*}
 \|{\bf p}(t)- {\bf p^{*}}(t)\| &\le \fr{8Me^{-\alpha^*t}}{h^{r-1}(1-h)}\,;\\
|E_{\bf p}(t)-\phi^*(t)| &\le  \fr{4Mre^{-\alpha^*t}}{h^{r-1}(1-h)}\,.
\end{align*}

По теореме~2 ($N=M$, $a=\alpha^*$) с использованием оценок следствия~2
\begin{align*}
\limsup\limits_{t \to \infty} \|{\bf p}(t)- \bar{\bf p}(t)\| &\le{} \notag\\
&\hspace*{-15mm}{}\le \fr{\varepsilon(1+\ln({4M}/({h^{r-1}(1-h)})))}{\alpha^*}\,;\\
\limsup\limits_{t \to \infty}   |E_{\bf p}(t)- \bar{E}_{\bar{\bf p}(t)}| &\le \notag\\
&\hspace*{-15mm}{}\le\fr{r\varepsilon(1+\ln(4M/(h^{r-1}(1-h))))}{\alpha^*}\,.
\end{align*}

По теореме~3 с использованием оценок следствия~2
\begin{multline*}
\limsup\limits_{t \to \infty}   |E_{\bf p}(t)- \bar{E}_{\bar{\bf p}(t)}| \le {}\\
{}\le
\fr{rM\varepsilon(3h+2h^2+h^3)(\alpha^* h^2+8M)}{h^r\alpha^*(\alpha^* h^2-2\varepsilon)}\,.
\end{multline*}

\noindent

\textbf{Пример 2.}

Рассмотрим процесс с интенсивностями 
$\la_1(t)\hm=\la_2(t)\hm=\ldots=\la_r(t) \hm= \la(t) \hm= 3\hm+\sin{2\pi t}$; 
$\mu_1(t)\hm=\mu_2(t)\hm= \mu(t) \hm= 2+\cos{2\pi t}$;
$\mu_3(t)=\ldots=\mu_r(t)=0$.

Будем предполагать, что возмущенный процесс имеет такую же структуру 
мат\-ри\-цы интен\-сив\-ностей, причем $|\la(t)-\bar{\la}(t)| \hm\le \varepsilon$ и  
$|\mu(t)-\bar{\mu}(t)| \hm\le \varepsilon$ почти при всех $t \hm\ge 0$. 
При этом будем иметь $\| A(t)\hm-\bar{A}(t)\| \hm\le 2r \varepsilon$ почти при всех $t \hm\ge 0$.

Выберем последовательность $d_k\hm=1$. Тогда  
\begin{gather*}
d=1\,; \enskip G=r\,; \enskip W=\fr{1}{r}\,; \enskip S=1\,; \\
K=\fr{4r(1+r)}{2}\,; \quad L=\fr{r(1+r)}{2}\,;
\\
\alpha(t)=\la(t)\,; \ \alpha=2\,; \ \alpha^*=3\,; M_0 \le 4\,; \ M \le  e^{7}\,.
\end{gather*}

И получаем следующие оценки.

\columnbreak

По следствию~1
\begin{align*}
 \|{\bf p^*}(t)- {\bf p^{**}}(t)\| &\le 8re^{-2t}\,;\\
|E_{\bf p}(t)- \phi(t)|&\le  4r^2 e^{-2t}\,.
\end{align*}

По следствию~2
\begin{align*}
\|{\bf p}(t)- {\bf p^{*}}(t)\| &\le 8re^{7-3t}\,;
\\[6pt]
|E_{\bf p}(t)- \phi^*(t)| &\le 4r^2 e^{7-3t}\,.
\end{align*}

По теореме~2 ($N=1$, $a=\alpha$) с учетом оценок следствия~1
\begin{align*}
\limsup\limits_{t \to \infty} \|{\bf p}(t)- \bar{\bf p}(t)\| &\le 
\fr{\varepsilon(1+\ln{4r})}{2}\,;
\\[6pt]
\limsup\limits_{t \to \infty}   |E_{\bf p}(t)- \bar{E}_{\bar{\bf p}(t)}|
&\le \fr{r\varepsilon(1+\ln{4r})}{2}\,.
\end{align*}

По теореме~2 ($N=M$, $a=\alpha^*$) с учетом оценок следствия~2
\begin{align*}
\limsup\limits_{t \to \infty} \|{\bf p}(t)- \bar{\bf p}(t)\| &\le 
\fr{\varepsilon(8+\ln{4r})}{3}\,;
\\
\limsup\limits_{t \to \infty}   \left|E_{\bf p}(t)- \bar{E}_{\bar{\bf p}(t)}\right| &\le 
\fr{r\varepsilon(8 + \ln{4r})}{3}\,.
\end{align*}

По теореме~3 с учетом оценок следствия~1
\begin{equation*}
\limsup\limits_{t \to \infty}   \left|E_{\bf p}(t)- \bar{E}_{\bar{\bf p}(t)}\right| \le 
\fr{5 \varepsilon r^2 (1+r)}{4(1- \varepsilon)}\,.
\end{equation*}

По теореме~3 с учетом оценок следствия~2
\begin{equation*}
\limsup\limits_{t \to \infty}   \left|E_{\bf p}(t)- \bar{E}_{\bar{\bf p}(t)}\right| \le 
\fr{\varepsilon e^{7} r^2 (1+r) (3+8e^{7})}{6(3-2\varepsilon)}\,.
\end{equation*}

{\small\frenchspacing
{%\baselineskip=10.8pt
\addcontentsline{toc}{section}{Литература}
\begin{thebibliography}{99}

 \bibitem{b} %1
\Au{Баруча-Рид~А.\,Т.} Элементы теории марковских процессов и их
приложения.~--- М.: Наука, 1969.

\bibitem{gm}  %2
\Au{Гнеденко~Б.\,В., Макаров~И.\,П.} Свойства решений задачи с потерями
в случае периодических интенсивностей~// Дифф. уравнения, 1971.
Вып.~7. С.~1696--1698.

\bibitem{g1}   %3
\Au{Gnedenko~D.\,B.} On a generalization of Erlang formulae~// 
Zastosow. Mat., 1971. Vol.~12. P.~239--242.

\bibitem{S}  %4
\Au{Саати~Т.\,Л.} Элементы теории массового обслуживания
 и ее приложения.~--- М.: Сов. радио, 1971.

\bibitem{g}  %5
\Au{Gnedenko~B., Soloviev~A.} On the conditions of the
existence of final probabilities for a Markov process~// Math.
Operations. Stat., 1973. P.~379--390.

\bibitem{gk} %6
\Au{Гнеденко~Б.\,В., Коваленко~И.\,Н.} Введение в теорию массового
обслуживания.~--- М.: Наука, 1987.
\pagebreak

\bibitem{gz00}   %7
\Au{Granovsky~B.\,L., Zeifman~A.\,I.}  The N-limit of spectral gap of 
a class of birth-death Markov chains~//
 Appl. Stoch. Models Business Ind., 2000. Vol.~16. P.~235--248.

\bibitem{z08b}  %8
\Au{Зейфман~А.\,И., Бенинг~В.\,Е., Соколов~И.\,А.} 
Марковские цепи и модели с непрерывным временем.~--- М.: Элекс-КМ, 2008.

\bibitem{dzp} %9
\Au{Van Doorn~E.\,A., Zeifman~A.\,I., Panfilova~T.\,L.}  
Bounds and asymptotics for the rate of convergence of birth-death processes~//  
Th. Prob. Appl., 2010. Vol.~54. P.~97--113.

\bibitem{z95b}   %10
\Au{Zeifman~A.\,I.} Upper and lower bounds on the rate of
convergence for nonhomogeneous birth and death processes~//  Stoch.
Proc. Appl., 1995. Vol.~59. P.~157--173.

\bibitem{gz05}  %11
\Au{Granovsky~B.\,L., Zeifman~A.\,I.} On the lower bound of the spectrum
 of some mean-field models~// Theory Prob. Appl., 2005. Vol.~49. P.~148--155.
 
\bibitem{z85}  %12
\Au{Zeifman~A.\,I.} Stability for contionuous-time
nonhomogeneous Markov chains~// Lect. Notes Math.,  1985. Vol.~1155.
P.~401--414.

\bibitem{z98} %13
\Au{Zeifman~A.} Stability of birth and death processes~// 
J.~Math. Sci., 1998. Vol.~91. P.~3023--3031.

\bibitem{ae} %14
\Au{Андреев~Д., Елесин~М., Кузнецов~А., Крылов~Е., Зейфман~А.}
Эргодичность и устойчивость нестационарных систем обслуживания~//
Теория вероятностей и математическая статистика, 2003. Т.~68.
С.~1--11.

\bibitem{mit03} %15
\Au{Mitrophanov~A.\,Yu.} Stability and exponential convergence of continuous-time 
Markov chains~//  J. Appl. Prob., 2003. Vol.~40. P.~970--979.

\label{end\stat} 

\bibitem{z11} %16
\Au{Зейфман~А.\,И., Коротышева~А.\,В., Панфилова~Т.\,Л., Шоргин~С.\,Я.} 
Оценки устойчивости  для некоторых систем обслуживания с катастрофами~//  
Информатика и её применения, 2011. Т.~5. Вып.~3. С.~27--33.
 \end{thebibliography}
}
}


\end{multicols}          %3
\def\stat{lebedev}

\def\tit{МАКСИМУМЫ АКТИВНОСТИ В~БЕЗМАСШТАБНЫХ СЛУЧАЙНЫХ СЕТЯХ С~ТЯЖЕЛЫМИ ХВОСТАМИ$^*$}

\def\titkol{Максимумы активности в~безмасштабных случайных сетях с~тяжелыми хвостами}

\def\autkol{А.\,В.~Лебедев}
\def\aut{А.\,В.~Лебедев$^1$}

\titel{\tit}{\aut}{\autkol}{\titkol}

{\renewcommand{\thefootnote}{\fnsymbol{footnote}}\footnotetext[1]
{Работа выполнена при поддержке РФФИ, грант 11-01-00050.}}


\renewcommand{\thefootnote}{\arabic{footnote}}
\footnotetext[1]{Механико-математический факультет
    Московского государственного университета им.\ М.~В.~Ломоносова,
    avlebed@yandex.ru}
    
  %  \vspace*{-6pt}

\Abst{Рассматриваются ориентированные степенные случайные графы как
    модели информационных сетей, где каждый узел обладает случайной информационной
    активностью, распределение которой имеет тяжелый (правильно меняющийся)
    хвост. Используется модель случайного графа, в которой входящие
    степени вершин независимы и имеют распределение со степенным хвостом.
    Выведены достаточные условия, при которых максимум
    суммарных активностей (по узлу и его входящим соседям) растет асимптотически
    так же, как и максимум индивидуальных активностей, и в силу этого для них
    имеет место предельный закон Фреше.}

 %   \vspace*{-2pt}
    
    \KW{максимумы; случайные суммы; безмасштабные сети;
    степенной закон; случайный граф; тяжелый хвост; правильное изменение;
    распределение Фреше}
    
%                    \vspace*{-9pt}
    
     \vskip 14pt plus 9pt minus 6pt

      \thispagestyle{headings}

      \begin{multicols}{2}
      
            \label{st\stat}
            


\section{Введение}

    Степенными (\textit{power-law}) или безмасштабными (\textit{free-scale}) называют
    случайные графы, у которых степени вершин подчиняются асимптотически
    степенному закону (с вероятностями $p_k\hm\sim ck^{-\beta}$, $k\hm\to\infty$,
    $\beta\hm>1$).
    Активные исследования данного класса графов и их приложений
    в последнее десятилетие были инициированы работой~\cite{Bar},
    где приведен ряд интересных примеров (Интернет, электрическая сеть,
    социальная сеть киноактеров).
    С~тех пор были предложены и изучены различные модели степенных графов.
    В~одних степенной закон возникает благодаря некоторому случайному
    процессу~\cite{Bar, Komp}, в других он постулируется изначально~\cite{Pow, Reit}.
    Следует отметить, что некоторые асимптотические свойства графов при
    одинаковом распределении степеней вершин могут оказаться общими,
    а другие зависят от выбора модели.

    Степенной граф может служить моделью некоторой информационной сети.
    Например, исследования кириллического сегмента <<Живого журнала>>
    ({\sf livejournal.com})~\cite{Komp}
    показывают, что он хорошо описывается степенным графом с $\beta\hm\approx 3$.
    Пусть каждый узел этой сети обладает случайной информационной активностью
    (интенсивностью производства информации). Имеется в виду среднее количество
    информации, производимой узлом в единицу времени. Активность будем
    полагать индивидуальной характеристикой, присущей узлу. Например, речь может идти
    о пользователях, которые пишут сообщения в Интернет.
    Предположим, что активности узлов
    независимы и одинаково распределены, причем их распределение~$F$ имеет тяжелый
    (правильно меняющийся) хвост, т.\,е.\ ${\bar F}(x)\hm\sim x^{-a}L(x)$,
    $x\hm\to\infty$, $a>0$, где $L(x)$~--- медленно меняющаяся функция~\cite[\S\ 8.8]{Fel}. 
    Такое предположение находится в русле современных
    представлений о распространенности степенных законов в природе, технике и
    человеческой деятельности. Активности и степени вершин (узлов) для
    простоты будем полагать независимыми.

    Рассмотрим суммарную активность в узле (т.\,е.\ сумму его собственной и
    ближайших соседей). Например, в <<Живом журнале>>
    каждый пользователь может оставлять свои записи и читать записи
    своих друзей, объединяемые для удобства в общую <<ленту друзей>> (френдленту).
    Далее будем интересоваться вопросом: когда максимум суммарных активностей
    растет асимптотически так же, как и максимум индивидуальных
    активностей узлов? В~этом случае для максимумов легко выводится
    предельный закон Фреше
    $\Phi_a(x)\hm=\exp\{-x^{-a}\}$, $x\hm>0$~[5, \S\ 8.8; 6, \S~3.3.1].

    Для модели степенного графа, введенной в~\cite{Pow},
    этот вопрос был изучен автором в~\cite{Leb2}.
    Там число вершин степени~$k$ полагалось
    детерминированным и равным $\lfloor e^\alpha/k^\beta\rfloor$,
    $\alpha$, $\beta>0$, $1\hm\le k\hm\le e^{\alpha/\beta}$,
    а распределение на множестве графов,
    удовле\-тво\-ря\-ющих этому условию, равномерным.
    Были получены достаточные условия того,
    что максимум сумм с ростом числа узлов (при $\alpha\hm\to\infty$)
    растет асимптотически так же, как и макcимум
    индивидуальных активностей: $a\hm<\beta\hm-3/2$, если $3/2\hm<\beta\hm<3$, и
    $a\hm<\beta/2$, если $\beta\hm\ge 3$. При этом применялись
    ранее полученные автором результаты для
    общей схемы максимумов сумм независимых случайных величин~\cite{Leb1}.

    Рассмотрим теперь модель ориентированного
    случайного графа, где направления ребер соответствуют направлениям
    передачи информации. Пусть имеется $n$ вершин и заданы независимые
    неотрицательные целочисленные случайные величины $K_1,\dots, K_n$,
    имеющие одинаковое
    распределение, заданное вероятностями $p_k\hm\sim ck^{-\beta}$, $k\hm\to\infty$,
    $\beta>1$. Положим $D_i\hm=\min\{K_i,n-1\}$. Для $i$-й вершины выберем
    случайным образом (равновероятно и независимо от выбора для других
    вершин) $D_i$ различных вершин из числа остальных (кроме $i$-й) и
    выпустим из них ребра, направленные в \mbox{$i$-ю} вершину. Полученный в
    результате граф можно отнести к степенным в том смысле, что входящие
    степени вершин распределены асимптотически по степенному закону.
    Суммарной активностью в узле в данном случае будем считать сумму
    собственной активности узла и всех узлов, из которых в него
    поступает информация (его входящих соседей).

    К сожалению, метод, использованный в~\cite{Leb2}, здесь
    не работает при $\beta\hm<3$, так как второй момент
    входящей степени вершины тогда растет слишком быстро при $n\hm\to\infty$.
    Эта проблема решается с по\-мощью урезания.
    При этом получаются более сильные ограничения на
    параметры, что связано с более быст\-рым ростом максимальной (входящей)
    степени вершины в графе. Однако поскольку используются лишь
    достаточные условия, не исключено, что эти ограничения в
    будущем могут быть ослаблены.

    %Результаты \cite{Leb2} и настоящей работы были кратко изложены в \cite{Leb3}.

    Отметим, что асимптотическая эквивалентность хвостов распределений
    суммы и максимума конечного числа независимых одинаково распределенных
    случайных величин в случае тяжелых хвостов
    представляет собой давно известный факт~\cite[\S\ 8.8]{Fel},
    обусловленный тем, что основной вклад в сумму дает самое большое
    слагаемое (максимум), а сумма остальных слагаемых по сравнению с
    ним оказывается мала. Теперь обобщим это утверждение
    на модель, где имеется
    некоторый набор случайных сумм со случайными числами слагаемых
    и от сумм берется максимум. По-преж\-не\-му оказывается, что основной
    вклад (в одну или несколько сумм, а значит, и в их максимум) дает только
    одно, максимальное слагаемое. Однако для этого хвост распределения
    слагаемых должен быть достаточно тяжелым.

    Проверка наличия подобного эффекта в реальных сетях, разумеется,
    требует экспериментального исследования, выходящего
    за рамки данной работы, которая имеет теоретический характер.
    
        \vspace*{-9pt}

    
    \section{Основной результат}
    
    \vspace*{-2pt}

    Будем рассматривать сети из $n$ узлов, затем устремляя $n$ к бесконечности.
    Обозначим через $M(n)$ максимум суммарных активностей (самого узла и его
    входящих соседей), а через $M_0(n)$~--- максимум индивидуальных активностей
    узлов. Требуется определить условия, при которых
    \begin{equation}
    \label{MP}
\fr{M(n)}{M_0(n)}\stackrel{P}{\to} 1\,,\quad n\to\infty\,.
    \end{equation}

    Введем неотрицательную функцию
    $u(s)$ такую, что $s{\bar F}(u(s))\hm\to 1$, $s\hm\to\infty$.
    Заметим, что $u(s)$ заведомо существует и правильно
    меняется с показателем $1/a$, т.\,е.\ $u(s)\hm\sim s^{1/a}L_2(s)$,
    $s\to\infty$, где $L_2(s)$~--- медленно меняющаяся функция~\cite[\S\ 1.5]{Sen}.

    Тогда имеет место предельный закон для максимумов независимых случайных
    величин в случае правильно меняющихся хвостов~[5, \S~8.8; 6, \S~3.3.1]:
    $$
    \lim\limits_{n\to\infty}{\bf P}\left(\fr{M_0(n)}{u(n)}\le x\right)=\Phi_a(x)\,,\quad x>0\,,
    $$
    что в сочетании с~(\ref{MP}) дает
    \begin{equation*}
%    \label{PZ}
    \lim\limits_{n\to\infty}{\bf P}\left(\fr{M(n)}{u(n)}\le x\right)=\Phi_a(x)\,,\quad x>0\,.
    \end{equation*}
    %В этом и заключается польза соотношения (\ref{MP}).

\smallskip

\noindent
\textbf{Теорема 1.} \textit{Соотношение}~(\ref{MP}) \textit{выполняется при
    $a\hm<\beta-2$, если $2\hm<\beta\hm<3$, и при $a\hm<(\beta-1)/2$,
    если $\beta\hm\ge 3$.}

    
    \smallskip
    
    \noindent
    Д\,о\,к\,а\,з\,а\,т\,е\,л\,ь\,с\,т\,в\,о\ теоремы будет приведено в разд.~4.

    
    \section{Общая схема максимумов сумм}

    Напомним введенную в~\cite{Leb1} схему (немного изменив обозначения).
    Пусть заданы случайный процесс $\Upsilon(t)$, $t\hm\in T$,
    значениями которого являются конечные
    классы конечных подмножеств ${\bf N}$,
    и семейство $\Xi\hm=\{\xi_{i,t},i\in {\bf N},t\in T\}$
    неотрицательных случайных величин, независимых и одинаково распределенных
    при любом фиксированном значении параметра $t\hm\in T$.
    Полагаем, что $\Upsilon$ и~$\Xi$ независимы.

    Для любых $A\subset {\bf N}$, $t\hm\in T$
    обозначим максимум набора случайных величин $\{\xi_{i,t}$, $i\hm\in A\}$
    через $M_t(A)$, $r$-й максимум (т.\,е.\ чис\-ло, стоящее $r$-м с конца в
    вариационном ряду)~--- через $M_t^{(r)}$, сумму~--- через $S_t(A)$.
    Пусть 
    $$
    U(t)\hm=\bigcup\limits_{A\in\Upsilon(t)}A\,.
    $$

    Введем случайные процессы, порожденные $\Upsilon$ и~$\Xi$:
    \begin{gather*}
    Z(t)=\!\!\sup\limits_{A\in\Upsilon(t)}\!\!S_t(A)\,;\enskip
    \kappa(t)=\!\!\sup\limits_{A\in\Upsilon(t)}\!\!|A|\,;
    \enskip\nu(t)=\left|U(t)\right|\,;\\
    \mu_1(t)=M_t(U(t))\,;\quad \mu_r(t)=M_t^{(r)}(U(t))\,,
 \end{gather*}
    где через $|A|$ обозначен размер (число элементов) множества~$A$.
    Через $|\Upsilon(t)|$ обозначим число различных множеств $A\hm\in\Upsilon(t)$.

    Предполагается, что $\nu(t)<\infty$ почти наверное (п.\,н.)\
    при всех $t\in T$, откуда следует конечность п.\,н.\
    всех процессов, введенных выше.

    Рассмотрим предельное поведение $Z(t)$ при $t\hm\to\infty$.

    Пусть существует случайный процесс $\rho(t)$ со
    значениями в ${\bf Z}_+$, измеримый относительно~$\Upsilon$ и такой, что
    $\rho(t)\hm\ge 1$ при $\nu(t)\hm\ge 1$, $\rho(t)\hm\le\nu(t)$ п.\,н. при всех $t\hm\in T$.

    Обозначим через $\pi(t)$ вероятность того, что для
    множества~$B$, равновероятно выбранного среди всех подмножеств $U(t)$,
    состоящих из $\rho(t)$ элементов,
    имеет место $\sup\limits_{A\in\Upsilon(t)}|A\cap B|\hm>1$.

\smallskip

\noindent
\textbf{Теорема I.} \textit{Пусть выполнены условия
    \begin{align}
    \label{us1}
    (\kappa(t)-1)\fr{\mu_{\rho(t)}(t)}{\mu_1(t)}\stackrel{P}{\to} 0,\quad t\to\infty\,;
\\
\label{us2}
    \pi(t)\to 0\,,\quad t\to\infty\,,
    \end{align}
    тогда
    \begin{equation}
    \label{res1}
    \fr{Z(t)}{\mu_1(t)}\stackrel{P}{\to} 1\,,\quad t\to\infty\,.
    \end{equation}
    }

    Доказано также следующее свойство порядковых статистик
    в случае правильно меняющихся хвостов.
    Пусть $X_n$, $n\ge 1$, независимы и имеют распределение~$F$
    с правильно меняющимся хвос\-том ${\bar F}(x)\hm\sim x^{-a}L(x)$,
    $x\hm\to\infty$, $a\hm>0$.
    Обозначим максимум
    $X_1,\dots, X_n$ через ${\tilde X}_n$ и $r$-й максимум через
    ${\tilde X}^{(r)}_n$.

\smallskip

\noindent
\textbf{Следствие II.}  \textit{Пусть $r_n\hm\sim n^\gamma$, $n\hm\to\infty$, $0\hm<\gamma\hm<1$ и
    $0\hm<\delta\hm<\gamma/a$, тогда
    $$
    \fr{n^\delta {\tilde X}^{(r_n)}_n}{{\tilde X}_n}\stackrel{P}{\to} 0\,,\quad
    n\to\infty\,.
    $$}
    
\vspace*{-12pt}

    
\section{Приложение к случайным графам}

    Адаптируем общую схему к изучению случайных графов.
    Рассмотрим процесс~$\Upsilon$ с дискретным временем, соответствующим
    числу узлов~$n$.
    Случайные величины $\xi_{i,n}$, $1\hm\le i\hm\le n$, описывают
    информационные активности узлов. Обозначим через
    $A_i$ множество из индекса~$i$ и индексов входящих соседей
    $i$-го узла, тогда набор множеств~$A$ получается из набора
    $A_1,\dots, A_n$ удалением повторов (если они есть). Имеем
    $|A_i|\hm=D_i\hm+1$ и $\kappa(n)\hm=\max\limits_{1\le i\le n}D_i\hm+1$.
    Очевидно, $|\Upsilon(n)|\hm\le n$ и $\nu(n)\hm=n$.
    Последовательность $\rho(n)$ далее будем полагать детерминированной.
    Кроме того, в используемых обозначениях $M(n)\hm=Z(n)$, $M_0(n)\hm=\mu_1(n)$ 
    и~(\ref{res1}) эквивалентно~(\ref{MP}).

    Обозначим
    $$
    Q(n,m)=n{\bf M}\left((D+1)D{\bf I}\{D\le m-1\}\right)\,,
    $$
    где $D\stackrel{d}{=}D_1$.

\medskip

\noindent
\textbf{Лемма 1.} \textit{При $n>2$ и $\rho(n)<n$ верно неравенство}
    $$
    \pi(n)\le\fr{\rho(n)(\rho(n)-1)}{2(n-1)(n-2)}\,Q(n,m)+
    {\bf P}(\kappa(n)>m)\,.
    $$

    \smallskip
    
    \noindent
    Д\,о\,к\,а\,з\,а\,т\,е\,л\,ь\,с\,т\,в\,о\,.\
    Событие $\{\sup\limits_{A\in\Upsilon(t)}|A\cap B|\hm>1\}$ представляет собой
    объединение событий $\{|A_i\cap B|\hm>1\}$, $1\hm\le i\hm\le n$.
    Пусть для простоты $B\hm=\{1,2,\dots, \rho(n)\}$ (в противном
    случае можно перенумеровать~$A_i$). Зафиксируем входящие степени
    вершин $d_1,\dots, d_n$. Тогда для $1\hm\le i\hm\le \rho(n)$ один элемент
    множества~$B$ заведомо принадлежит~$A_i$ (а именно, индекс~$i$),
    а любой другой принадлежит с вероятностью $d_i/(n-1)$. Для
    $\rho(n)+1\hm\le i\hm\le n$ любая пара индексов из~$B$ принадлежит~$A_i$
    с ве\-ро\-ят\-ностью $d_i(d_i-1)/((n-1)(n-2))$, а всего таких пар
    $\rho(n)(\rho(n)-1)/2$. Суммируя вероятности, получаем оценку сверху:
    $$
    \fr{\rho(n)-1}{n-1}\sum\limits_{i=1}^{\rho(n)}d_i+
    \fr{\rho(n)(\rho(n)-1)}{2(n-1)(n-2)}\sum\limits_{i=\rho(n)+1}^nd_i(d_i-1)\,.
    $$
    Обозначим 
    \begin{align*}
    q_1&={\bf M}(D{\bf I}\{D\le m-1\})\,;\\
q_2&={\bf M}(D(D-1){\bf I}\{D\le m-1\})\,.
\end{align*}

Усредняя по наборам входящих
    степеней вершин в области $\kappa(n)\hm\le m$, получаем оценку сверху:
\begin{multline*}
    \fr{\rho(n)-1}{n-1}\,\rho(n)q_1+
    \fr{\rho(n)(\rho(n)-1)}{2(n-1)(n-2)}\left(n-\rho(n)\right)q_2\le{}\\
{}\le\fr{\rho(n)(\rho(n)-1)}{2(n-1)(n-2)}\,n\left(2q_1+q_2\right)={}\\
{}=
    \fr{\rho(n)(\rho(n)-1)}{2(n-1)(n-2)}\,Q(n,m)\,.
\end{multline*}
    Учитывая также вероятность события $\{\kappa(n)>m\}$, получаем
    утверждение леммы.

\medskip

\noindent
\textbf{Лемма 2.} \textit{Пусть выполнены следующие условия:}
\begin{enumerate}[(1)]
\item \textit{все $\xi_{i,n}$ имеют одинаковое распределение $F$ на~${\bf R}_+$
    с хвостом ${\bar F}(x)\sim x^{-a}L(x)$, $x\to\infty$, $a>0$};
\item
    $m\sim n^\delta$, $n\to\infty$, $\delta>0$;
\item
    $Q(n,m)=O(n^b)$, $n\to\infty$,   $0<b<2$;
\item
    $\kappa(n)=o_p(m)$, $n\to\infty$;
\item
    $a<(2-b)/(2\delta)$.
    \end{enumerate}
    \textit{Тогда верно}~(\ref{MP}).

    
    \medskip
    
    \noindent
    Д\,о\,к\,а\,з\,а\,т\,е\,л\,ь\,с\,т\,в\,о\,.\
    Можно выбрать $\gamma\hm\in (0,1)$
    так, чтобы выполнялось неравенство $a\delta\hm<\gamma\hm<(2-b)/2$.
    Положим $\rho(n)\hm=[n^\gamma]$, тогда
    по следствию~II получаем~(\ref{us1}), а по лемме~1~---~(\ref{us2}),
    так что условия теоремы~I выполняются и верно соотношение~(\ref{res1}), 
    эквивалентное~(\ref{MP}).

\smallskip

\noindent
Д\,о\,к\,а\,з\,а\,т\,е\,л\,ь\,с\,т\,в\,о\ теоремы~1.\
    Поскольку $p_k\hm\sim ck^{-\beta}$,
    $k\hm\to\infty$, то хвост распределения имеет
    асимптотику ${\bar F}_K(k)\hm\sim c_1k^{-(\beta-1)}$.
    Отсюда получаем
    $\kappa(n)\hm=O_p(n^{1/(\beta-1)})$, $n\hm\to\infty$,
    и, следовательно, $\kappa(n)\hm\sim o_p(m)$ при любом
    $\delta\hm=(1+\varepsilon)/(\beta-1)$, $\varepsilon\hm>0$. Имеем
    \begin{multline*}
    Q(n,m)=n\sum\limits_{k=1}^{m-1}(k+1)kp_k={}\\
    {}=
\begin{cases}
    O(n^{1+\delta(3-\beta)})\,,& 1<\beta<3\,;\\
    O(n\ln n)\,,& \beta=3\,;\\
    O(n)\,, & \beta>3\,.
    \end{cases}
    \end{multline*}
    При $2<\beta<3$ применяем лемму 2 с $b\hm=2\delta\hm>1\hm+\delta(3\hm-\beta)$ и,
    устремляя~$\varepsilon$ к нулю,
    получаем достаточное условие $a\hm<\beta\hm-2$ для выполнения~(\ref{MP}).
    При $\beta\hm\ge 3$ применяем лемму 2 с $b\hm=1\hm+\varepsilon$ и
    аналогично получаем достаточное условие $a\hm<(\beta-1)/2$.
    
    {\small\frenchspacing
{%\baselineskip=10.8pt
\addcontentsline{toc}{section}{Литература}
\begin{thebibliography}{9}


    \bibitem{Bar}
    \Au{Barab\'asi A., Albert R.} Emergence of scaling in random networks~//
    Science, 1999. Vol.~286. P.~509--512.

    \bibitem{Komp}
    \Au{Захаров П.} Народ-бло\-го\-но\-сец~// Компьютерра,
    2007. №\,27-28. C.~36--39. {\sf http://offline.computerra.ru/2007/ 695/327726}.

    \bibitem{Pow}
    \Au{Aiello W., Chung F., Lu~L.} A random graph model for power law
    graphs~// Experimental Math., 2001. Vol.~10. No.~1. P.~53--66.

    \bibitem{Reit}
    \Au{Reittu H., Norros~I.} On the power-law random graph model
    of massive data network~// Performance Evaluation, 2004. Vol.~55. P.~3--23.

    \bibitem{Fel}
    \Au{Феллер В.} Введение в теорию вероятностей и ее приложения. Т.~2.~---
    М.: Мир, 1984.

    \bibitem{EKM}
    \Au{Embrechts P., Kl$\ddot{\mbox{u}}$ppelberg C., Mikosh~T.} Modelling
    extremal events for insurance and finance.~--- Springer-Verlag, 2003.

    \bibitem{Leb2}
    \Au{Лебедев А.\,В.} Максимумы активности в случайных сетях в
    случае тяжелых хвостов~// Проблемы передачи информации, 2008. Т.~44.
    №\,2. С.~96--100.

    \bibitem{Leb1}
\Au{Лебедев А.\,В.} Общая схема максимумов сумм независимых
    случайных величин и ее приложения~// Математические заметки, 2005.
    Т.~77. №\,4. С.~544--550.

\label{end\stat}

    \bibitem{Sen}
\Au{Сенета Е.} Правильно меняющиеся функции.~--- М.: Наука, 1985.
 \end{thebibliography}
}
}


\end{multicols}          %4
\def\stat{agalarov}


\def\tit{ПРИБЛИЖЕННЫЙ МЕТОД ВЫЧИСЛЕНИЯ ХАРАКТЕРИСТИК УЗЛА 
ТЕЛЕКОММУНИКАЦИОННОЙ СЕТИ С~ПОВТОРНЫМИ ПЕРЕДАЧАМИ}
\def\titkol{Приближенный метод вычисления характеристик узла 
телекоммуникационной сети с~повторными передачами} 

\def\autkol{Я.\,М.~Агаларов}
\def\aut{Я.\,М.~Агаларов$^1$}

\titel{\tit}{\aut}{\autkol}{\titkol}

%{\renewcommand{\thefootnote}{\fnsymbol{footnote}}\footnotetext[1]
%{Работа выполнена при поддержке РФФИ, проекты 08--07--00152 и 08--01--00567.}}

\renewcommand{\thefootnote}{\arabic{footnote}}
\footnotetext[1]{Институт проблем
информатики Российской академии наук, agglar@yandex.ru}

%\vspace*{-6pt}


\Abst{Рассмотрена модель узла коммутации пакетов c повторными передачами для двух 
схем распределения буферной памяти: полнодоступной и полного разделения. Предложен 
приближенный метод вычисления интенсивностей потоков и вероятностей блокировок узла. 
Получены необходимые и достаточные условия существования и единственности решения 
уравнения для потоков в узле при установившемся режиме работы и доказана сходимость 
итерационного метода решения указанного уравнения.}

\KW{узел коммутации пакетов; буферная память; повторные передачи; вероятности 
блокировок; итерационный метод}

      \vskip 18pt plus 9pt minus 6pt

      \thispagestyle{headings}

      \begin{multicols}{2}

      \label{st\stat}


\section{Введение}

    Одной из основных задач предварительного анализа 
телекоммуникационных сетей коммутации пакетов с ограниченной буферной 
памятью является расчет характеристик потоков и вероятностей блокировок в 
узлах связи. Важность указанных характеристик определяется тем, что от их 
значений существенным образом зависят другие основные показатели сети 
(пропускная способность, задержки пакетов и~др.). 

    Существует множество различных моделей узлов коммутации пакетов и 
методов их расчета (см., например,~[1--6]). Для моделей, рассматривающих 
узел с ограниченной буферной памятью как систему массового обслуживания 
(CMO) типа 
$
\begin{matrix}
M \\ \lambda
\end{matrix}
\left |
\begin{matrix}
M \\ \lambda
\end{matrix}
\right |
\overline{m} \vert N
$ или  $\vert PH\vert PH\vert 1\vert r$, в предположении отсутствия повторных 
передач пакетов получены точные методы вычисления характеристик 
узлов~[1, 3, 4, 6]. Приближенные методы расчета узлов, учитывающие повторные 
попытки передачи, используют модели типа $\vert PH\vert PH\vert 1\vert r$ или 
$
\begin{matrix}
M \\ \lambda
\end{matrix}
\left |
\begin{matrix}
M \\ \lambda
\end{matrix}
\right |
1 \vert N
$ и являются 
итерационными~[2, 3, 5, 7]. Для моделей типа 
$BM\!AP\vert PH\vert 1$, $M\vert G\vert 1\vert r$ и $M\!AP\vert 
(PH,PH)\vert 1$ с повторными заявками получены точные методы вычисления 
характеристик (например, в работах~[8--10]), которые также могут быть 
использованы при расчете узлов.

    Ниже будут рассмотрены модели узла коммутации пакетов с повторными 
передачами для двух схем распределения буферной памяти: с 
полнодоступными буферами и с полным разделением буферной памяти. 
Предлагается приближенный метод расчета характеристик, который в качестве 
модели узла использует СМО типа $
\begin{matrix}
M \\ \lambda
\end{matrix}
\left |
\begin{matrix}
M \\ \lambda
\end{matrix}
\right |
\overline{m} \vert N
$ с повторными заявками. Доказаны утверждения о 
достаточных и необходимых условиях существования и единственности 
решения уравнения для вероятности блокировки в установившемся режиме 
работы и сходимости предлагаемого итерационного метода. 

\section{Модель узла}

    Математическая модель узла представляется в виде СМО с ограниченной 
буферной памятью и различными потоками заявок, каждая из которых требует 
обслуживания только на одной из многоканальных линий связи. 

    Пусть $0<N<\infty$~--- число мест хранения в буферной памяти, $u$~--- 
узел связи, $v$~--- линия связи, $\Omega_u^+$~--- множество исходящих из 
узла~$u$ линий, $c_v$~--- канальная емкость линии~$v$. Поток заявок, 
тре\-бу\-ющих обслуживания на линии~$v$, назовем $v$-по\-то\-ком, заявки этого 
потока~--- $v$-за\-яв\-ка\-ми.


    Пусть выполняются следующие предположения: 
\begin{enumerate}[1.]
\item Места в буферной памяти распределяются согласно одной из двух 
схем:
\begin{enumerate}[($i$)]
\item полнодоступная схема~--- каждое свободное место хранения доступно 
любой заявке;
\item схема полного разделения памяти~--- $v$-за\-яв\-кам доступны всего 
$N_v$ мест, где $\sum\limits_{v\in\Omega_u^+} N_v=N$.
\end{enumerate}
\item Если в момент поступления $v$-заявки в буферной памяти есть 
доступное свободное место, то она сразу занимает это место. Если в момент 
поступления $v$-заявки в системе нет свободного доступного места 
хранения, то поступившая заявка через некоторое время повторно поступает 
на систему, оставаясь $v$-заявкой. 
\item Интенсивности первичных потоков $v$-заявок~--- заданные величины 
$0<\Lambda_v<\infty$, $v\in \Omega_u^+$. Суммарные потоки первичных и 
повторных $v$-заявок являются независимыми в совокупности 
пуассоновскими потоками. Для обслуживания $v$-заявки требуется 
одновременно одно место хранения и один канал типа~$v$, $v\in 
\Omega_u^+$.
\item Первичные нагрузки~--- реализуемые, т.\,е.\ в данном случае 
интенсивности входных первичных потоков равны интенсивностям 
выходных потоков выполненных заявок. 
\item Принятые в СМО $v$-заявки обслуживаются линией~$v$ в порядке 
поступления. 
\item Время занятия канала $v$-заявкой~--- экспоненциально 
распределенная случайная величина с параметром $0<\mu_v<\infty$, 
$v\in\Omega_u^+$, независимая от других случайных событий в узле.
\item Выполненная $v$-заявка с вероятностью~$B_v$ повторяется через 
заданное время~$\tau_v$ (тайм-аут) и с вероятностью $1-B_v$ покидает 
систему через время~$t_v$ навсегда, сразу освободив занятый канал и место 
буферной памяти.
\end{enumerate}

   Будем говорить, что узел блокирован для $v$-за\-яв\-ки, если в буферной 
памяти отсутствует доступное место хранения. Ставится задача вычисления 
вероятностей блокировок и интенсивностей потоков в узле.

\section{Вычисление вероятности блокировки и~интенсивностей~потоков} 

   Пусть $\Lambda_v^*$~--- интенсивность суммарного потока внешних 
заявок, требующих передачи по линии~$v$, $\pi_v$~--- вероятность блокировки 
узла для заявок, требующих передачи по исходящей из узла линии~$v$. 

    Пусть в узле используется полнодоступная схема распределения 
буферной памяти. Тогда, как следует из описания модели, $\pi_v 
=\pi_{v^\prime},\,v,\,v^\prime\in \Omega_u^+$, и для 
интенсивностей~$\Lambda_v^*$, $v\in\Omega_u^+$, справедливы соотношения:
\begin{equation*}
\Lambda_v^* = \fr{\Lambda_v}{1-\pi}\,,
%\label{e1aga}
\end{equation*}
    где
    $\pi =\pi_v$, $v\in\Omega_u^+$.

    Пусть 
    $\overline{k} = \{\overline{k}_v$, $v\in\Omega_u^+\}$~--- состояние 
буферной памяти узла, $\overline{k}_v =\left ( k_v,\,k_v^\prime,\,k_v^{\prime\prime}\right )$; 
$k_v$~--- число $v$-заявок в буферной 
памяти, ожидающих выполнения линией~$v$; $k^\prime_v$~--- число 
$v$-заявок в буферной памяти, ожидающих тайм-аут и неуспешно переданных 
в последующий узел; $k_v^{\prime\prime}$~--- число $v$-за\-явок в буферной 
памяти, успешно переданных в последующий узел и ожидающих 
потверждения; 
$A_m = \left \{ \overline{k}:\ \sum\limits_{v\in\Omega_u^+} \left ( 
k_v+k_v^\prime + k_v^{\prime\prime}\right ) =m \right \}$~--- множество различных 
состояний, при которых в памяти узла занято ровно $m$~буферов. Тогда с 
учетом введенных выше обозначений и предположений для ве\-ро\-ят\-ности 
блокировки узла можно написать формулу~\cite{1aga, 2aga}:
\begin{equation}
\pi = \fr{1}{G_N}\sum\limits_{\overline{k}\in A_N} 
p\left (\overline{k},\overline{\rho}^*\right )\,,
\label{e2aga}
\end{equation}
где  
\begin{gather}
p(\overline{k},\overline{\rho}^*) = \prod\limits_{v\in\Omega_u^+} z_v (\pi, 
\rho_v , k_v , k_v^\prime , k_v^{\prime\prime})\,;\\
z_v (\pi, \rho_v , k_v , k_v^\prime , k_v^{\prime\prime}) ={}\notag\\
\!\!{}=
\begin{cases}
 \fr{\rho_v^{\prime *k_v^\prime}}{k_v^{\prime}!}\,
\fr{\rho_v^{\prime\prime * k_v^{\prime\prime}}}{ k_v^{\prime\prime}!}  \,
\fr{\rho_v^{*k_v}}{ k_{v}!} 
&\mbox{при}\ k_v<c_v\,,\\
 \fr{\rho_v^{\prime * k_v^\prime}}{k_v^{\prime}!} \,
\fr{\rho_v^{\prime\prime * k_v^{\prime\prime}}} { k_v^{\prime\prime}!} 
\fr{\rho_v^{*k_v}}{ c_{v}!c_v^{k_v- c_v}} 
& \mbox{при}\ k_v\geq c_v\,;
\end{cases}\\
G_N = \sum\limits_{m=0}^N\sum\limits_{\overline{k}\in A_m}
p(\overline{k},\overline{\rho}^*)\,;\\ 
\overline{\rho}^*=\{\rho_v^*,\,v\in\Omega_u^+\}\,;\\
\rho_v^* = \fr{\rho_v}{1-\pi}\,;\quad \rho_v =\fr{\Lambda_v}{\mu_v(1- B_v)}\,;\\
\rho_v^{\prime *} =\rho_v^*\mu_v\tau_vB_v\,;\quad \rho_v^{\prime\prime *}=
p_v^* \mu_vt_v,\,\quad  v\in \Omega_u^+\,.\label{e3aga}
\end{gather}

Переобозначив $1-\pi$ через $y$, выражение в правой части равенства~(2)~--- через 
$p_{\overline{k}}(\overline{\rho},y)$, выражение в правой части равенства~(4)~--- 
через $g_N(\overline{\rho},y)$, а выражение в правой 
части равенства~(1)~--- через $1-q_N (\overline{\rho},y)$, 
где $\overline{\rho} = (\rho_v,\,v\in \Omega_u^+)$, $\rho_v = \rho_v^*y\;=$\linebreak 
$=\;\Lambda_v/(\mu_v(1-B_v))$, $v\in\Omega_u^+$, получим нелинейное уравнение 
относительно неизвестной переменной~$y$:
\begin{equation}
y=q_N(\overline{\rho},y)\,.
\label{e4aga}
\end{equation}

    Решим уравнение~(8). Как следует из~(2)--(7), верно 
равенство
\begin{equation}
q_N(\overline{\rho},y) = \fr{g_{N-1}(\overline{\rho},y )}{g_N(\overline{\rho},y)}\,.
\label{e5aga}
\end{equation}
Введем функцию  $d_n(\overline{\rho} ,y)$ среднего числа заявок в узле с 
буферной памятью емкости $n\geq 0$:
$$
d_n(\overline{\rho} ,y) = 
\fr{1}{g_n(\overline{\rho},y)}\,\sum\limits_{m=0}^n m\sum\limits_{\overline{k}\in 
A_m} p_{\overline{k}}(\overline{\rho},y)\,.
$$
Заметим, что $g_n$, $d_n$ и $q_n$, 
$n\geq 0$,~--- непрерывно-дифференцируемые функции по $y\in (0,\,1]$. Взяв 
производную функции~$g_n$ по~$y$, из~(2)--(7) получим
\begin{multline}
\fr{\partial g_n(\overline{\rho},y)}{\partial y} ={}\\
{}= -\sum\limits_{m=0}^n m 
\sum\limits_{\overline{k}\in A_m}\fr{\prod\limits_{v\in\Omega_u^+} z_n 
(0,\rho_v, k_v, k_v^\prime , k_v^{\prime\prime})}{y^{m+1}}={}\\
{}= -\fr{1}{y}\,g_n (\overline{\rho},y)d_n(\overline{\rho},y)\,.
\label{e6aga}
\end{multline}
Взяв производную функции $q_N$ по $y$, из~(\ref{e5aga}) и~(\ref{e6aga}) 
получим
\begin{equation}
\fr{\partial q_N(\overline{\rho},y)}{\partial y} = \fr{q_N(\overline{\rho},y)}{y}\left 
[ d_N (\overline{\rho},y)-d_{N-1}(\overline{\rho},y)\right ]\,.
\label{e7aga}
\end{equation}
    Докажем несколько утверждений о свойствах 
функции~$q_N(\overline{\rho},y)$.
\medskip

\noindent
\textbf{Утверждение 1.} \textit{Справедливы неравенства}
\begin{multline}
0<d_{n+1}(\overline{\rho},y)-d_n(\overline{\rho},y) <1\,,\\
\ \ \ \ \ \ \ \ \ \ \ \ \ \ \ \ \ \ \ \ y\in (0,\,1]\,, \ n\geq 0\,.
\label{e8aga}
\end{multline}


\noindent

Д\,о\,к\,а\,з\,а\,т\,е\,л\,ь\,с\,т\,в\,о\,.\ Подставив выражение для функции 
$d_n(\overline{\rho},y)$ и проведя преобразования, получим
\begin{multline*}
d_{n+1}(\overline{\rho},y) -d_n(\overline{\rho},y) = 
\fr{\sum\limits_{m=0}^{n+1}m\sum\limits_{\overline{k}\in A_m} 
p_{\overline{k}}(\overline{\rho},y)}
{\sum\limits_{m=0}^{n+1}
\sum\limits_{\overline{k}\in A_m} p_{\overline{k}}(\overline{\rho},y)} - {}\\
{}-
\fr{\sum\limits_{m=0}^n m \sum\limits_{\overline{k}\in A_m} p_{\overline{k}} 
(\overline{\rho},y)}{\sum\limits_{m=0}^n
\sum\limits_{\overline{k}\in A_m}p_{\overline{k}}(\overline{\rho},y)}={}\\
{}=\fr{\sum\limits_{m=1}^n m \sum\limits_{\overline{k}\in 
A_m}p_{\overline{k}}(\overline{\rho},y)+(n+1)\sum\limits_{\overline{k}\in 
A_{n+1}}  p_{\overline{k}}(\overline{\rho},y)}{\sum\limits_{m=0}^n\sum\limits_{\overline{k
}\in A_m}p_{\overline{k}}(\overline{\rho},y)+\sum\limits_{\overline{k}\in 
A_{n+1}}p_{\overline{k}}(\overline{\rho},y)} -{}
\end{multline*}
\begin{multline}
{}-
\fr{\sum\limits_{m=0}^n m 
\sum\limits_{\overline{k}\in A_m}p_{\overline{k}}(\overline{\rho},y)}
{\sum\limits_{m=0}^n\sum\limits_{\overline{k}\in A_m} 
p_{\overline{k}}(\overline{\rho},y)}={}\\
{}=\fr{(n+1)\sum\limits_{\overline{k}\in 
A_{n+1}}p_{\overline{k}}(\overline{\rho},y)g_n(\overline{\rho},y)}{g_{n+1}(\overline{\rho},y) g_n(\overline{\rho},y)} -{}\\
{}-
\fr{\sum\limits_{\overline{k}\in 
A_{n+1}}p_{\overline{k}}(\overline{\rho},y)\sum\limits_{m=0}^n  m 
\sum\limits_{\overline{k}\in A_m} p_{\overline{k}}(\overline{\rho},y) }
{g_{n+1}(\overline{\rho},y) g_n(\overline{\rho},y)}
={}\\
{}=\left [ 1-q_{n+1}(\overline{\rho},y)\right ] \left [n+1-d_n(\overline{\rho},y)\right ]\,.
\label{e9aga}
\end{multline}


    Докажем утверждение~1 методом индукции. При $n = 0$, как следует 
из~(\ref{e9aga}), имеем
$$
d_2(\overline{\rho},y) - d_1 (\overline{\rho},y) =1-q_1(\overline{\rho},y)\,,
$$
    т.\,е.\ утверждение~1 при $n = 0$ справедливо. 

    Пусть неравенства~(\ref{e8aga}) справедливы для некоторого $n > 0$. 
Докажем, что они справедливы и для $n + 1$. Из~(\ref{e9aga}) получаем
\begin{multline*}
d_{n+1}(\overline{\rho},y)- d_n(\overline{\rho},y)={}\\
{}=\left [ 1-
q_{n+1}(\overline{\rho},y)\right ] \left [n+1-d_n(\overline{\rho},y)\right ] ={}\\
{}= \left [ 1-
1-q_{n+1}(\overline{\rho},y)\right ] \left [ n-{}\right.\\
{}-\left. d_{n-1}(\overline{\rho},y)+d_{n-1}(\overline{\rho},y)-
d_n(\overline{\rho},y)+1\right ] ={}\\
{}=\left [ 1-q_{n+1}(\overline{\rho},y)\right ] 
\left [ n-d_{n-1}(\overline{\rho},y)-{}\right.\\
{}-\left. \left ( d_n(\overline{\rho},y)-d_{n-1}(\overline{\rho},y)\right )+1\right] = {}\\
{}=
\left [ 1-q_{n+1}(\overline{\rho},y)\right ]
\left [ 
\fr{d_n(\overline{\rho},y) -d_{n-1}(\overline{\rho},y)}{1-
q_n(\overline{\rho},y)}\right.-{}\\
{}-\left.
\left ( d_n(\overline{\rho},y)-d_{n-1}(\overline{\rho},y)\right )+1
\vphantom{\fr{d_n(\overline{\rho})}{(q_n)}}
\right ]={}\\
{}=
\left [ 1-q_{n+1}(\overline{\rho},y)\right ]
\left [ 
\vphantom{\fr{d_n(\overline{\rho})}{(q_n)}}
\left ( d_n(\overline{\rho},y\right)\right. -{}\\
 {}-\left.
d_{n-1}\left(\overline{\rho},y)\right )\fr{q_n(\overline{\rho},y)}{1-
q_n(\overline{\rho},y)}+1\right ]\,.
\end{multline*}
Так как по предположению $d_n (\overline{\rho},y) -d_{n-1}(\overline{\rho},y) 
>0$, то правая часть последнего равенства больше нуля; следовательно, 
$d_{n+1}(\overline{\rho},y)-d_n(\overline{\rho},y)>0$. 

    Продолжив преобразование правой части последнего равенства и 
учитывая предположение $d_n(\overline{\rho},y) -d_{n-1}(\overline{\rho},y)<1$, 
получим
\begin{multline*}
d_{n+1}((\overline{\rho},y) -d_n(\overline{\rho},y)<{}\\
{}< \left [ 1-
q_{n+1}(\overline{\rho},y)\right ]
\left ( \fr{q_n(\overline{\rho},y)}{1-q_n(\overline{\rho},y)}+1\right )={}\\
{}=
\fr{1-q_{n+1}(\overline{\rho},y)}{1-q_n(\overline{\rho},y)}<1\,,
\end{multline*}
так как $0< q_n(\overline{\rho},y)<q_{n+1}(\overline{\rho},y)<1$, $n>0$, $y\in 
(0,\,1]$.

Следовательно, утверждение~1 доказано.

\medskip

\noindent
\textbf{Утверждение 2.} $q_N(\overline{\rho},y)$~--- \textit{монотонно-воз\-рас\-та\-ющая 
функция по $y\in (0,\,1]$. При этом $0< q_N(\overline{\rho},y)\;\leq $\linebreak 
$\leq\;q_N(\overline{\rho},1) <1$, $y\in (0,\,1]$,  и $\underset{y\rightarrow 
0}{\mathrm{lim}}\,q_N(\overline{\rho},y) =0$}.

\medskip

\noindent
Д\,о\,к\,а\,з\,а\,т\,е\,л\,ь\,с\,т\,в\,о\,.\  Возрастание функции 
$q_N(\overline{\rho},y)$ следует непосредственно из~(\ref{e7aga}) и 
утверж\-де\-ния~1. Доказательство неравенств в условии утверждения очевидно 
следует из~(\ref{e5aga}) и вида функции $g_n (\overline{\rho},y)$, $n\geq 0$. 
Для предела имеем:
\begin{multline*}
\underset{y\rightarrow 0}{\mathrm{lim}}\,q_N(\overline{\rho},y) 
=\underset{y\rightarrow 0}{\mathrm{lim}}\,\fr{g_{N- 1}(\overline{\rho},y)}{g_N(\overline{\rho},y)} = {}\\
{}= \underset{y\rightarrow 0}{\mathrm{lim}}\,\left (
g_{N-1}(\overline{\rho},y)\Bigg / \left ( 
\vphantom{\prod\limits_{v\in\Omega_u^+}}
g_{N-1}(\overline{\rho},y)\right.\right.+{}\\
{}+\left.\left.\sum\limits_{\overline{k}\in A_N}\prod\limits_{v\in\Omega_u^+} 
\fr{z_v(0,\rho_v,k_v,k^\prime_v,k^{\prime\prime}_v)}{y^N}\right )\right ) = {}\\
{}= \underset{y\rightarrow 0}{\mathrm{lim}}\,\left (
y^N g_{N-1}(\overline{\rho},y)\Bigg / 
\left ( 
\vphantom{\prod\limits_{v\in\Omega_u^+}}
y^N g_{N-1}(\overline{\rho},y)+{}\right.\right.\\
{}+\left.\left.\sum\limits_{\overline{k}\in A_N}
\prod\limits_{v\in\Omega_u^+} z_v(0,\rho_v,k_v,k_v^\prime , k_v^{\prime\prime}) 
\right ) \right )=0\,.
\end{multline*}
    
\medskip

\noindent
\textbf{Утверждение 3.} \textit{Производная функции~$q_N (\overline{\rho},y)$ по 
$y\in (0,\,1]$ удовлетворяет следующим соотношениям}:
\begin{align}
\underset{y\rightarrow 0}{\mathrm{lim}}\fr{\partial q_N(\overline{p},y)}
{\partial  y} &= \fr{\sum\limits_{\overline{k}\in A_{N-1}} 
p_{\overline{k}}(\overline{\rho},1)}{\sum\limits_{\overline{k}\in 
A_N}p_{\overline{k}}(\overline{\rho},1)}\,;\label{e10aga}\\
\fr{\partial q_N(\overline{\rho},y)}{\partial y}\Big |_{y=1}&<1\,.\label{e11aga}
\end{align}

\medskip

\noindent
Д\,о\,к\,а\,з\,а\,т\,е\,л\,ь\,с\,т\,в\,о\,.\ Проведя преобразования 
функции~$q_N(\overline{\rho},y)$, получим:
\begin{multline*}
\underset{y\rightarrow 0}{\mathrm{lim}}\fr{q_N(\overline{\rho},y)}{y} = {}\\
\!\!{}=
\underset{y\rightarrow 0}{\mathrm{lim}}
\fr{\sum\limits_{m=0}^{N-1}\sum\limits_{\overline{k}\in A_m}
\!\!\left (\prod\limits_{v\in\Omega_u^+}\!\! 
z_v(0,\rho_v,k_v,k_v^\prime , k_v^{\prime\prime})\right )\!\!\Bigg /\!\! y^m}
{y\sum\limits_{m=0}^{N}\sum\limits_{\overline{k}\in A_m}
\!\!\left(\prod\limits_{v\in\Omega_u^+}\!\! z_v\left (0,\rho_v,k_v,k_v^\prime , 
k_v^{\prime\prime}\right )\right )\!\!\Bigg /\!\!y^m} = \!\!\!
\end{multline*}
\begin{multline*}
\!\!\!\!\!\!{}=\underset{y\rightarrow 0}{\mathrm{lim}}\,
\fr{\sum\limits_{m=0}^{N-1}\sum\limits_{\overline{k}\in A_m}
y^{N-1-m}\prod\limits_{v\in\Omega_u^+} z_v(0,\rho_v,k_v,k_v^\prime , 
k_v^{\prime\prime})}{\sum\limits_{m=0}^{N}\sum\limits_{\overline{k}
\in A_m} y^{N-m}
\prod\limits_{v\in\Omega_u^+} z_v(0,\rho_v,k_v,k_v^\prime , 
k_v^{\prime\prime})}={}\!\\
{}=\fr{\sum\limits_{\overline{k}\in A_{N-1}} p_{\overline{k}}(\overline{\rho},1)}{ 
\sum\limits_{\overline{k}\in A_{N}} p_{\overline{k}}(\overline{\rho},1)}\,.
\end{multline*}
Очевидно, $\underset{y\rightarrow 0}{\mathrm{lim}} \,[d_N (\overline{\rho},y) -
d_{N-1} (\overline{\rho},y)]=1$, так как $\underset{y\rightarrow 
0}{\mathrm{lim}}\,d_n (\overline{\rho},y)=n$, $n>0$.

Следовательно, учитывая~(\ref{e7aga}), получаем~(\ref{e10aga}). 
Справедливость~(\ref{e11aga}) непосредственно следует из~(\ref{e7aga}) и 
утверждения~1.

\medskip

\noindent
\textbf{Утверждение 4.} \textit{Пусть $y^*\in (0,\,1]$~--- решение 
уравнения}~(\ref{e4aga}). \textit{Тогда}
\begin{equation*}
\fr{\partial q_N(\overline{\rho},y)}{\partial y}\Big |_{y=y^*}<1\,.
%\label{e12aga}
\end{equation*}

\medskip

\noindent
Д\,о\,к\,а\,з\,а\,т\,е\,л\,ь\,с\,т\,в\,о\,.\ \ Доказательство следует из~(\ref{e7aga}), 
так как $q_N(\overline{\rho},y^*)/y^* =1$.
\medskip

\noindent
\textbf{Утверждение 5.} \textit{Уравнение}~(\ref{e4aga}) \textit{имеет решение $y^*\in 
(0,\,1)$ тогда и только тогда, когда} 
\begin{equation}
\fr{\sum\limits_{\overline{k}\in A_{N-1}} p_{\overline{k}}(\overline{\rho},1)}{ 
\sum\limits_{\overline{k}\in A_{N}} p_{\overline{k}}(\overline{\rho},1)} >1\,.
\label{e13aga}
\end{equation}
\textit{Если уравнение}~(\ref{e4aga}) \textit{имеет решение $y^*\in (0,\,1)$, то оно 
единственное положительное решение}.
\medskip

\noindent
Д\,о\,к\,а\,з\,а\,т\,е\,л\,ь\,с\,т\,в\,о\,.\ Пусть выполняется 
неравенство~(\ref{e13aga}). Тогда, как следует из утверждения~3, 
$\underset{y\rightarrow 0}{\mathrm{lim}} (\partial q_N(\overline{\rho},y)/\partial y) 
>1$. Кроме того, как следует из утверждения~2, 
$\underset{y\rightarrow 0}{\mathrm{lim}} q_N(\overline{\rho},y)=0$. Тогда, так 
как $q_N(\overline{\rho},y)$~--- непрерывно-дифференцируемая функция по 
$y\in (0,\,1]$, существует значение $y^\prime \in (0,\,1)$ такое, что 
$q_N(\overline{\rho},y)>y$ для всех $y\in (0,\,y^\prime]$ (следует из теоремы о 
конечном приращении~\cite{11aga}). В то же время, согласно утверждению~2, 
$q_N(\overline{\rho},y)<y$ в окрестности точки $y=1$ (рис.~\ref{f1aga},\,\textit{а}). 
Следовательно, кривая $x=q_N(\overline{\rho},y)$ пересекает прямую $x=y$ 
хотя бы в одной точке $y=y^*\in (0,\,1)$, т.\,е.\ уравнение~(\ref{e4aga}) имеет 
хотя бы одно решение $y^*\in (0,\,1)$.

\begin{figure*}
\vspace*{1pt}
\begin{center}
\vspace*{1pt}
\mbox{%
\epsfxsize=158mm
\epsfbox{aga-1.eps}
}
\end{center}
\vspace*{-9pt}
\Caption{Примеры кривых $x=q_N(\overline{\rho},y)$ и $x=y$ (\textit{а})~при существовании решения 
уравнения~(\ref{e5aga}) и (\textit{б})~при выполнении условий~(17)
\label{f1aga}}
\vspace*{6pt}
\end{figure*}

Пусть уравнение~(\ref{e4aga}) имеет решение $y^*\in (0,\,1)$ и 
\begin{equation}
\fr{\sum\limits_{\overline{k}\in A_{N-1}}p_{\overline{k}}(\overline{\rho},1)}{ 
\sum\limits_{\overline{k}\in A_{N}}p_{\overline{k}}(\overline{\rho},1)}\leq 
1\,.\label{e14aga}
\end{equation}
Тогда из условий утверждений~2 и~3 следует, что 
уравнение~(\ref{e4aga}) в интервале $(0,\,1)$ имеет более одного решения, что 
может быть только при существовании решения $y^\prime \in (0,\,1)$ такого, 
что в окрестности точки $y=y^\prime$ выполняются неравенства: 
$q_N(\overline{\rho},y)<y$ при $y<y^\prime$ и $q_N(\overline{\rho},y)>y$ при 
$y>y^\prime$, где $y$ принадлежит указанной окрест\-ности точки~$y^\prime$ 
(рис.~\ref{f1aga},\,\textit{б}). Тогда в точке $y=y^\prime$ производная функции 
$q_N(\overline{\rho},y)$ по $y$ больше~1, что противоречит утверждению~4. 
Следовательно, неравенство~(\ref{e13aga}) справедливо.


Пусть уравнение~(\ref{e4aga}) имеет более одного положительного 
решения. Рассуждая точно так же, как и выше (в случае выполнения 
условий~(\ref{e14aga})), получим противоречие утверждению~4. 
Следовательно, утверждение~5 справедливо.
\medskip

\noindent
\textbf{Следствие.} \textit{Неравенства}
\begin{gather*}
\fr{\mu_v c_v (1-B_v)}{\Lambda_v}>1\,,\quad \fr{1-B_v}{\Lambda_v \tau_v B_v}>1\,,\\ 
\fr{1-B_v}{\Lambda_v t_v}>1\,,\ v\in\Omega_u^+\,,
\end{gather*}
\textit{являются необходимым условием существования решения 
уравнения}~(\ref{e4aga}).

\medskip
\noindent
Д\,о\,к\,а\,з\,а\,т\,е\,л\,ь\,с\,т\,в\,о\,.\ Пусть $\overline{k}_v$~--- это 
набор~$\overline{k}$, у которого $k_v=0$. Преобразовав левую 
часть~(\ref{e13aga}), получим

\noindent
\begin{multline*}
\fr{\sum\limits_{\overline{k}\in A_{N-1}} p_{\overline{k}} (\overline{\rho},1)}
{ \sum\limits_{\overline{k}\in A_{N}} 
 p_{\overline{k}}(\overline{\rho},1)} 
={}
\\
{}=
\fr{\sum\limits_{\overline{k}\in A_{N-1}}\prod\limits_{v\in \Omega_u^+} 
z_v\left(0,\rho_v,k_v,k_v^\prime , k_v^{\prime\prime}\right)}
{\sum\limits_{\overline{k}\in A_{N}}
\prod\limits_{v\in \Omega_u^+} z_v\left (0,\rho_v,k_v,k_v^\prime , k_v^{\prime\prime}\right )} \leq{}
\\
{}\leq
\left ( 
\vphantom{\prod\limits_{v^\prime\in\Omega_u^+\backslash v}}
\fr{\mu_v c_v(1-B_v)}{\Lambda_v}\right. \times{}\\
{}\times \sum\limits_{k_v=0}^{N-1}\sum\limits_{\overline{k}_v\in A_{N-1-k_v}} z_v\left(0,\rho_v,k_v+1,k_v^\prime , 
k_v^{\prime\prime}\right )\times{}\\
{}\times \left.\prod\limits_{v^\prime\in\Omega_u^+\backslash v} z_v^\prime 
\left(0,\rho_v,k_v,k_v^\prime , k_v^{\prime\prime}\right) \right)
\Bigg /{}\\
\Bigg / \left ( 
\vphantom{\prod\limits_{v^\prime\in\Omega_u^+\backslash v}}
\sum\limits_{k_v=0}^{N-1} \sum\limits_{\overline{k}_v\in A_{N-1-k_v}} z_v 
\left (0,\rho_v,k_v+1,k_v^\prime , 
k_v^{\prime\prime}\right )\right. \times{}\\
{}\times \prod\limits_{v^\prime\in\Omega_u^+\backslash v} 
z_{v^\prime}\left(0,\rho_v,k_v,k^\prime , k_v^{\prime\prime}\right)+{}\\
{}+
\sum\limits_{\overline{k}_v\in A_N} z_v\left (0,\rho_v, 0,k_v^\prime , 
k_v^{\prime\prime}\right)\times{}\\
\left.{}\times \prod\limits_{v^\prime\in\Omega_u^+\backslash v}z_{v^\prime} 
\left(0,\rho_v,k_v,k_v^\prime , k_v^{\prime\prime}\right )\right )\,.
\end{multline*}
Как следует из правой части последнего неравенства, если 
$\mu_v c_v (1-B_v)/\Lambda_v \leq 1$, то она меньше~1. Поэтому, чтобы 
выполнилось условие~(\ref{e13aga}), необходимо выполнение первого 
неравенства в условии следствия для каждого $v\in\Omega_u^+$. Точно так же 
доказывается необходимость выполнения второго и третьего неравенств в 
условии следствия.

    Пусть $y[n]$, $n\geq 0$, последовательность, полученная по формуле 
$y[n+1]=q_N(\overline{\rho},y[n])$, $y[0]=1$.

\medskip

\noindent
\textbf{Утверждение 6.} \textit{Пусть $y^*\in (0,\,1)$~--- решение 
уравнения}~(8). \textit{Тогда последовательность $y[n]$, $n\geq 0$, сходится 
к решению~$y^*$}.

\medskip

\noindent
Д\,о\,к\,а\,з\,а\,т\,е\,л\,ь\,с\,т\,в\,о\,.\ Отметим, что $y[1]<y[0]$ (это следует из 
утверждения~2, так как $y[0]=1$). Пусть для некоторого $n>1$ выполняется 
условие $y[n]<y[n-1]$. Тогда, как следует из утверждения~2, указанное условие 
выполняется и для $n+1$, т.\,е.\ по индукции следует, что последовательность 
$y[n]$, $n\geq 0$, монотонно убывает. 

    Пусть для некоторого $n>0$ $y[n]>y^*$ (существование такого $n$ 
следует из равенства $y[0]=1$). Тогда, как следует из утверждения~2, 
$y[n+1]\;=$\linebreak $=\;q_N(\overline{\rho},y[n])>q_N(\overline{\rho},y^*) =y^*$, т.\,е.\ 
последовательность ограничена снизу величиной~$y^*$. Значит, существует 
$\underset{n\rightarrow \infty}{\mathrm{lim}}\,y[n]=y^0\geq y^*$. Так как 
$q_n(\overline{\rho},y)$~--- непрерывная по~$y$ функция, то можно написать 
$\underset{n\rightarrow 
\infty}{\mathrm{lim}}\,q_N(\overline{\rho},y[n])=q_N(\overline{\rho},y^0)=y^0$, 
т.\,е.\ $y^0$~--- решение уравнения~(\ref{e4aga}). Из единственности 
положительного решения уравнения~(\ref{e4aga}) получаем $y^0=y^*$.

    Пусть в узле используется схема полного разделения буферной памяти. 
Тогда для интенсив\-ностей~$\Lambda_v^*$, $v\in\Omega_u^+$, справедливы 
соотношения:
$$
\Lambda_v^* = \fr{\Lambda_v}{1-\pi_v}\,,
$$
где $v\in\Omega_u^+$.


Фиксируем произвольную линию сети~$v$. Пусть $\overline{k}_v = (k_v, 
k_v^\prime, k_v^{\prime\prime})$~--- состояние буферной памяти линии~$v$; 
$k_v$, $k_v^\prime$, $k_v^{\prime\prime}$ определены выше. Тогда с 
учетом введенных ранее предположений и обозначений для вероятности 
блокировки линии справедлива формула~\cite{4aga}:
\begin{equation}
\pi_v = \fr{1}{G_{N_v}}\sum\limits_{k_v=N_v} 
z_v(\pi_v,\rho_v,\overline{k}_v)\,,
\label{e15aga}
\end{equation}
где 
\begin{multline*}
z_v(\pi_v, \rho_v, \overline{k}_v)={}\\
{}=
\begin{cases}
\fr{\rho_v^{\prime * k_v^\prime}}{k_v^\prime !}\,
 \fr{\rho_v^{\prime\prime * k_v^{\prime\prime}}}{k_v^{\prime\prime}!}\,
 \fr{\rho_v^{*k_v}}{k_v !} & \mbox{при}\ k_v<c_v\,,\\
 \fr{\rho_v^{\prime *k_v^\prime}}{k_v^{\prime }! }
 \fr{\rho_v^{\prime\prime * k_v^{\prime\prime}}}{k_v^{\prime\prime}!}
\fr{\rho_v^{*k_v}}{c_v !c_v^{k_v-c_v}} & \mbox{при}\ k_v\geq c_v\,;
\end{cases}
\end{multline*}
\begin{align*}
G_{N_v} &= \sum\limits_{m=0}^{N_v} z_v (\pi_v ,\rho_v , \overline{k}_v)\,;\\ 
\rho_v^*&=\fr{\rho_v}{1-\pi_v}\,;
\end{align*}
$\rho_v$, $\rho_v^{\prime *}$, 
$\rho_v^{\prime\prime *}$, $v\in\Omega_u^+$ определены выше.
    
Пусть $y_v=1-\pi_v$, а $q_{N_v} (\rho_v, y_v)$~--- выражение в правой 
части~(\ref{e15aga}). Тогда из равенств~(\ref{e15aga}), взяв~$y_v$ в качестве 
неизвестной переменной, получим систему независимых уравнений:
\begin{equation}
y_v = q_{N_v}(\rho_v, y_v)\,, \quad v\in \Omega_u^+\,.
\label{e16aga}
\end{equation}
    
    Заметим, что для фиксированной $v$ и заданных параметров $\Lambda_v$, 
$\mu_v$, $\tau_v$, $t_v$, $N_v$, $v\in\Omega_u^+$, уравнение в~(\ref{e16aga}) 
является частным случаем уравнения~(\ref{e4aga}) и совпадает с последним, 
когда число исходящих линий из узла равно~1. Следовательно, для схемы 
полного разделения памяти также справедливы все приведенные выше 
утверждения~1--6 и следствие. Заметим, что неравенство~(\ref{e13aga}) в 
условии утверждения~5 при $B_v=0$ и $t_v=0$ преобразуется в неравенство 
$\Lambda_v / (\mu_v c_v) >1$, $v\in\Omega_u^+$. Последовательность 
$\overline{y}[n]$, $n\geq 0$, в утверждении~6 определяется по формуле:
    \begin{gather*}
    \overline{y}[n] =\{y_v[n],\ v\in\Omega_u^+\}\,,\
    y_v[n+1]=q_{N_v} (\rho_v,\,y_v[n])\,,\\
    y_v[0] =1\,,\quad n\geq 0\,,\quad v\in \Omega_u^+\,.
    \end{gather*}


\section{Алгоритм расчета} %4

    Для вычисления интенсивностей потоков и вероятностей блокировок в 
узле предлагается следующий алгоритм, описывающий изложенную выше 
итерационную процедуру. Введем обозначения:
$y_u^v$~--- вероятность блокировки узла для заявок, направляемых на 
линию~$v$,
\begin{gather*}
y_u^v  = 
\begin{cases}
y_u & \mbox{для}\ v\in\Omega_u^+\ \mbox{при}\\
&\mbox{полнодоступной схеме},\\
y_v & \mbox{при схеме полного распределения}\\
&\mbox{памяти};
\end{cases}
\\
q_N^v(\overline{\rho}_u^{-v}, y_u^v)  = 
\begin{cases}
q_N(\overline{\rho},y) & \mbox{для}\ v\in\Omega_u^+\ \mbox{при пол-}\\ 
&\mbox{нодоступной схеме},\\
q_{N_v}(\rho_v, y_v) & \mbox{при схеме полного}\\
&\mbox{распределения}\\ 
&\mbox{памяти},  v\in\Omega_u^+\,.
\end{cases}
\end{gather*}



Тогда уравнения~(\ref{e4aga}) и~(\ref{e16aga}) записываются в виде:
$$
y_u^v = q_N^v (\overline{\rho}^v_u, y^v_u)\,,\quad v\in \Omega_u^+\,.
$$
Для значений, вычисляемых на $k$-м шаге алгоритма, к 
обозначениям соответствующих параметров приписывается знак~$[k]$.
\pagebreak

\textbf{Шаг 0.} 
\begin{enumerate}[1.]
\item  \textit{Инициализация}. Вычисление начальных значений 
параметров~$\rho_v$, $v\in\Omega_u^+$: $\Lambda_v[0]=\Lambda_v$, 
$\rho_v[0]=\Lambda_v[0]/(\mu_v(1-B_v))$, $y_u^v[0]=1$.
\item \textit{Проверка условий существования решения}. Если для некоторой 
линии $v\in\Omega_u^+$ выполняется хотя бы одно неравенство $(c_v\mu_v(1-
B_v))/\Lambda_v[0]\;\leq$\linebreak $\leq\;1$, или $(1-B_v)/(\Lambda_v\tau_v B_v) \leq 1$, или 
$(t_v(1\;-$\linebreak $-\;B_v))/\Lambda_v[0] \leq 1$, то алгоритм заканчивает работу с 
результатом <<нагрузка не реализуема>>. Если в узле используется 
полнодоступная схема и $(c_v\mu_v(1-B_v))/\Lambda_v[0] > 1$, $(1-
B_v)/(\Lambda_v\tau_v B_v)\;>$\linebreak $>\;1$, $(t_v(1-B_v))/\Lambda_v[0] > 1$ для всех 
$v\in\Omega_u^+$, то проверяется условие~(\ref{e13aga}) для $\Lambda_v =
\Lambda_v[0]$, $v\in\Omega_u^+$, и при невыполнении этого условия алгоритм 
заканчивает работу с результатом <<нагрузка не реализуема>>.
\end{enumerate}

    При вычислении левой части неравенства~(\ref{e13aga}) рекомендуется 
использовать метод свертки Базена (см.~\cite{12aga}), позволяющий 
производить рекуррентные вычисления (подробно этот метод описан также 
в~[1, 3--6]).



\medskip
\textbf{Шаг~$k$} ($k > 0$):
\begin{enumerate}[1.]
\item \textit{Вычисление вероятностей блокировок}. Используя значения 
параметров $\overline{\rho}_u^v[k-1]$, $y_u^v[k-1]$, $v\in\Omega_u^+$, 
вычисление с помощью формул~(1)--(7) значений 
вероятностей $y[k]=1- \pi [k]$~--- в случае полнодоступной памяти, или 
$y_v[k]=1- \pi_v[k]$, $v\in\Omega_u^+$, с помощью формул~(\ref{e15aga})~--- в 
случае полного разделения памяти. При вычислении этих значений 
рекомендуется использовать метод свертки Базена.
    \item \textit{Проверка условий останова алгоритма}. Если хотя бы для 
одной $v\in\Omega_u^+$ для заданного значения точности   выполняется 
условие
$$
\fr{\vert \Lambda_v^*[k]-\Lambda_v^*[k-1]\vert}{\Lambda_v^*[k]}> \varepsilon\,,
$$
то вычисление параметров $\overline{\rho}_u^v[k]$, $v\in\Omega_u^+$, и 
переход к шагу~$k$, положив $k$ равным $k+1$, иначе алгоритм завершает 
работу. 
\end{enumerate}

    По завершении алгоритма либо выявится, что нагрузка в системе не 
реализуема, либо будут вычислены интенсивности потоков, поступающих на 
линии узла, и стационарные вероятности блокировок для заявок каждого типа. 
    
\section{Примеры расчета}

    Для проверки точности вычисления результатов с помощью 
предложенного выше алгоритма и приемлемости введенных предположений 
были проведены вычислительные эксперименты с использованием 
аналитических и имитационных моделей. Во всех рассмотренных ниже 
примерах потоки внешних заявок считаются пуассоновскими. 
В~табл.~1 приведены значения вероятности блокировок вновь 
поступивших извне заявок, полученные на основании точной формулы, 
приведенной в~\cite{4aga} для СМО типа $M\vert M\vert 1\vert 0$ с повторными 
заявками при экспоненциальном распределении интервала времени между 
повторными попытками (первая строка таблицы), алгоритма из подраздела~5 
настоящей статьи (вторая строка) и имитационной модели при постоянном 
интервале времени между повторными попытками, равном~10 (третья строка). 
Расчет табл.~1 проведен для узла с одной исходящей одноканальной 
линией при интенсивности первичного потока $\Lambda =1$ и емкости 
накопителя $N_v=1$. Таблицы~2 и~3 вычислены с помощью 
алгоритма из подраздела~5 и имитационной модели соответственно при одной 
исходящей линии с числом каналов~10.


    В табл.~\ref{t4aga} и~\ref{t5aga} приведены значения вероятности 
блокировки узла с тремя исходящими линиями канальной емкости~10 каждая 
при $\mu_v =0{,}2$, $v\in\Omega_u^+$,  вычисленные с помощью алгоритма из 
подраздела~5 и имитационной модели с интервалом повторной попытки, 
равным~10, соответственно. В табл.~\ref{t4aga} и~\ref{t5aga} знак <<--->> в 
ячейках означает, что предложенная нагрузка $\Lambda_v$, $v\in\Omega_u^+$, 
не реализуема.



В табл.~\ref{t6aga} отражены вероятности блокировки такого же узла с 
накопителем $N = 35$ при экспоненциальном распределении интервала 
времени между повторными попытками со средним значением~$\tau$. 


Результаты вычислительного эксперимента показывают, что с  увеличением 
длины интервала между повторными попытками  вероятность блокировки 
увеличивается и приближается к значению,\linebreak
вычисленному с помощью 
алгоритма из подраздела~5 (см.\ табл.~\ref{t4aga} и~\ref{t6aga}), т.\,е.\ при 
пуассоновском внешнем потоке заявок предположение, что суммарный 
входной поток заявок  является пуассоновским, вполне приемлемо для 
предварительного анализа характеристик узла (например, при  $\tau c_v\mu_v > 
10$). Как показывают табл.~1--3, вероятность блокировки 
узла существенно зависит от\linebreak 

\vspace*{6pt}
\noindent
%\begin{table*}\small %tabl1
{\small
{{\tablename~1}\ \ \small{Вероятности блокировок при одной исходящей одноканальной линии}}
%\label{t1aga}}
\vspace*{-3pt}

\begin{center}
{\tabcolsep=7.3pt
\begin{tabular}{|c|c|c|c|c|c|}
\hline
&\multicolumn{5}{c|}{$\mu$}\\
\cline{2-6}
\multicolumn{1}{|c|}{\raisebox{4pt}[0pt][0pt]{№}}&1{,}1&1{,}2&2&3&4\\
\hline
1&0,9091&0,8333&0,5000&0,3333&0,2500\\
2&0,9091&0,8333&0,5000&0,3333&0,2500\\
3&0,8867&0,8452&0,4944&0,3167&0,2396\\
\hline
\end{tabular}}
\end{center}
%\vspace*{-6pt}
%\end{table*}
}
%\bigskip
%\medskip
\addtocounter{table}{1}
\pagebreak

\end{multicols}

\renewcommand{\figurename}{\protect\bf Таблица}
%\renewcommand{\tablename}{\protect\bf Рис.}
\begin{figure*}
{\small
\begin{minipage}[t]{76mm}
%\begin{table*}\small %tabl2
\begin{center}
\Caption{Вероятности блокировок при одной исходящей многоканальной линии ($\varepsilon 
=0{,}0001$)
\label{t2aga}}
\vspace*{2ex}

\tabcolsep=6.5pt
\begin{tabular}{|c|c|c|c|c|c|}
\hline
&\multicolumn{5}{c|}{$\mu$}\\
\cline{2-6}
\multicolumn{1}{|c|}{\raisebox{4pt}[0pt][0pt]{$N$}}&0{,}11&0{,}12&0{,}2&0{,}3&0{,}4\\
\hline
10&0,4845&0,2935&0,0204&0,0017&0,0002\\
15&0,1181&0,0545&0,0006&0,0000&0,0000\\
20&0,0489&0,0167&0,0000&0,0000&0,0000\\
\hline
\end{tabular}
\end{center}
%\end{table*}
\end{minipage}
\hfill
\begin{minipage}[t]{76mm}
%\begin{table*}\small %tabl3
\begin{center}
\Caption{Вероятности блокировок при одной исходящей линии
\label{t3aga}}
\vspace*{2ex}

\tabcolsep=6.5pt
\begin{tabular}{|c|c|c|c|c|c|}
\hline
&\multicolumn{5}{c|}{$\mu_v$}\\
\cline{2-6}
\multicolumn{1}{|c|}{\raisebox{4pt}[0pt][0pt]{$N$}}&0{,}11&0{,}12&0{,}2&0{,}3&0{,}4\\
\hline
10&0,5247&0,3238&0,0219&0,0019&0,0001\\
15&0,1726&0,0912&0,0004&0,0001&0,0000\\
20&0,1180&0,0371&0,0000&0,0000&0,0000\\
\hline
\end{tabular}
\end{center}
%\end{table*}
\end{minipage}
}
\vspace*{6pt}
\end{figure*}

\renewcommand{\figurename}{\protect\bf Рис.}
\renewcommand{\tablename}{\protect\bf Таблица}
\addtocounter{table}{2}

\begin{table}\small %tabl4
\begin{center}
\parbox{400pt}{\Caption{Вероятности блокировок при трех исходящих линиях, вычисленные алгоритмом из 
подраздела~5 ($\varepsilon =0{,}0001$)
\label{t4aga}}
}

\vspace*{2ex}

\tabcolsep=8pt
\begin{tabular}{|c|c|c|c|c|c|c|c|c|c|}
\hline
&\multicolumn{9}{c|}{$\Lambda_v$}\\
\cline{2-10}
\multicolumn{1}{|c|}{\raisebox{4pt}[0pt][0pt]{$N$}}&1&1{,}1&1{,}2&1{,}3&1{,}4&1{,}5&1{,}6&1{,}7&1{,}8\\
\hline
20&0,0677&0,1423&0,2975&0,7653&---&---&---&---&---\\
25&0,0065&0,0173&0,0394&0,0827&0.1690&0.3827&---&---&---\\
30&0,0005&0,0019&0,0059&0,0155&0.0361&0.0790&0.1792&0,7259&---\\
35&0,0000&0,0002&0,0008&0,0030&0,0089&0,0234&0,0574&0,1505&---\\
40&0,0000&0,0000&0,0001&0,0005&0,0022&0,0075&0,0220&0,0617&0,2449\\
\hline
\end{tabular}
\end{center}
%\end{table}
\vspace*{6pt}
%\begin{table}\small %tabl5
\begin{center}
\parbox{400pt}{\Caption{Вероятности блокировок при трех исходящих линиях, вычисленные с помощью 
имитационной модели
\label{t5aga}}
}

\vspace*{2ex}

\tabcolsep=8pt
\begin{tabular}{|c|c|c|c|c|c|c|c|c|c|}
\hline
&\multicolumn{9}{c|}{$\Lambda_v$}\\
\cline{2-10}
\multicolumn{1}{|c|}{\raisebox{4pt}[0pt][0pt]{$N$}}&1&1{,}1&1{,}2&1{,}3&1{,}4&1{,}5&1{,}6&1{,}7&1{,}8\\
\hline
20&0,0786&0,1695&0,3549&0,7056&---&---&---&---&---\\
25&0,0069&0,0190&0,0441&0,0998&0,2266&0,4583&---&---&---\\
30&0,0007&0,0024&0,0075&0,0184&0,0462&0,1025&0,2380&0,6931&---\\
35&0,0000&0,0003&0,0007&0,0040&0,0129&0,0307&0,0890&0,2981&---\\
40&0,0000&0,0000&0,0000&0,0011&0,0041&0,0095&0,0346&0,0790&0,3179\\
\hline
\end{tabular}
\end{center}
%\end{table}
\vspace*{6pt}
%\begin{table}\small %tabl6
\begin{center}
\parbox{356pt}{\Caption{Зависимость вероятности блокировки при трех исходящих линиях, вы\-чис\-лен\-ные с 
помощью имитационной модели со случайным интервалом между повторными попытками
\label{t6aga}}
}

\vspace*{2ex}

\tabcolsep=8pt
\begin{tabular}{|c|c|c|c|c|c|c|c|c|}
\hline
&\multicolumn{8}{c|}{$\Lambda_v$}\\
\cline{2-9}
\multicolumn{1}{|c|}{\raisebox{6pt}[0pt][0pt]{$\tau$}}&1&1{,}1&1{,}2&1{,}3&1{,}4&1{,}5&1{,}6&1{,}7\\
\hline
\hphantom{9}1&0.0001&0,0001&0,0017&0,0063&0,0210&0,0733&0,1996&0,4222\\
\hphantom{9}5&0.0000&0,0002&0,0016&0,0036&0,0446&0,0159&0,1360&0,3273\\
10&0.0000&0,0002&0,0011&0,0036&0,0101&0,0430&0,0818&0,2774\\
20&0.0000&0,0003&0,0007&0,0029&0,0089&0,0257&0,0863&0,2045\\
     \hline
\end{tabular}
\end{center}
\end{table}


\begin{multicols}{2}


\noindent
числа каналов в линии при равной суммарной 
производительности. Кроме того, как видно из табл.~\ref{t5aga} и~\ref{t6aga}, 
вероятность блокировки в большей степени зависит от среднего значения 
длины интервала между повторными попытками передачи, чем от закона 
распределения длины интервала. Таким образом, предложенный в работе 
алгоритм позволяет вы\-чис\-лить с достаточной точностью вероятность 
блокировки узла, интенсивности повторных передач и предельную величину 
реализуемой нагрузки. Отметим, что полученные в данной статье результаты 
могут быть использованы для расчета нагрузок в телекоммуникационной сети с 
повторами заявок в предыдущем узле или из источника. 


{\small\frenchspacing
{%\baselineskip=10.8pt
\addcontentsline{toc}{section}{Литература}
\begin{thebibliography}{99}    
\bibitem{1aga}
\Au{Kamoun~F., Kleinrock~L.}
Analysis of shared finite storage in a computer networks node environment under 
general traffic conditions~// IEEE Trans. on Commun., 1980. Vol.~28. No.\,7. 
P.~992--1003.

\bibitem{6aga} %2
\Au{Агаларов~Я.\,М., Шоргин~С.\,Я.}
Рекуррентный метод вычисления параметров сетей связи~// Техника средств 
связи, 1986. Сер. <<Системы связи>>. Вып.~6. С.~42--46.

\bibitem{3aga}
\Au{Башарин Г.\,П., Бочаров~П.\,П., Коган~Я.\,А.}
Анализ очередей в вычислительных сетях.~--- М.: Наука, 1989. 

\bibitem{4aga}
\Au{Бочаров~П.\,П., Печинкин~А.\,В.}
Теория массового обслуживания.~--- М.: Изд-во РУДН, 1995. 

\bibitem{5aga}
\Au{Вишневский~В.\,М.} 
Теоретические основы проектирования компьютерных сетей.~--- М.: 
Техносфера, 2003. 

\bibitem{2aga} %6
\Au{Башарин Г.\,П.}
Лекции по математической теории телетрафика.~--- М.: Изд-во РУДН, 2007. 

\bibitem{7aga}
\Au{Таранцев~А.\,А.}
Инженерные методы теории массового обслуживания.~--- М.: Наука, 2007.

\bibitem{9aga} %8
\Au{D'Apice~C., De~Simone~T., Manzo~R., Rizelian~G.}
$M\vert G\vert 1\vert r$ retrial queueing system with priority service of primary 
customers and a customers-searching server~// Distributed Computer and 
Communication Networks. Stochastic Modelling and Optimization.~--- М.: 
Техносфера, 2003. P.~106--117.

\bibitem{8aga} %9
\Au{Klimenok~V.\,I., Kim~C.\,S.}
$BM\!AP$/$PH$/1 retrial system operating in random environment~// Proceedings of 
the 5th St.-Petersburg Workshop on Simulation, St.-Petersburg, June~26\,--\,July~2, 
2005.~--- St.-Petersburg: NII Chemistry St.-Petersburg University Publs., 
2005. P.~367--372.   

\bibitem{10aga}
\Au{Krishnamoorthy~A., Babu~S.}
$M\!AP\vert (PH,PH)/c$ retrial queue with selegeneration of priorities 
and non-preemptive service~// Proceedings of the 14th International Conference on 
Analytical and Stochastic Modeling Techniques and Applications, June~4--6, 
2007. Prague, Czech Republic.~--- Sbr.-Dudweiler: Digitaldruck Pirrot GmbH, 
2007. P.~70--74.

\bibitem{11aga}
\Au{Корн~Г., Корн~Т.}
Справочник по математике.~--- М.: Наука, 1974.

\label{end\stat}


\bibitem{12aga}
\Au{Buzen~J.\,P.}
Computational algorithm for closed queuing networks with exponential servers~// 
Communications ACM, 1973. Vol.~16. No.\,9. P.~527--531.
 \end{thebibliography}
}
}
\end{multicols}
 
 
     %5
\def\stat{kondranin+ushakov}

\def\tit{СИСТЕМА ОБСЛУЖИВАНИЯ С~ОТНОСИТЕЛЬНЫМ ПРИОРИТЕТОМ  И~ПРОФИЛАКТИКАМИ ПРИБОРА$^*$}

\def\titkol{Система обслуживания с~относительным приоритетом  и~профилактиками прибора}

\def\aut{Е.\,С.~Кондранин$^1$,  В.\,Г.~Ушаков$^2$}

\def\autkol{Е.\,С.~Кондранин,  В.\,Г.~Ушаков}

\titel{\tit}{\aut}{\autkol}{\titkol}

\index{Кондранин Е.\,С.}
\index{Ушаков В.\,Г.}
\index{Kondranin E.\,S.}
\index{Ushakov V.\,G.}




{\renewcommand{\thefootnote}{\fnsymbol{footnote}} \footnotetext[1]
{Работа выполнена при финансовой поддержке РФФИ (проект 18-07-00678).}}


\renewcommand{\thefootnote}{\arabic{footnote}}
\footnotetext[1]{Факультет вычислительной математики и~кибернетики Московского государственного 
университета им.\ М.\,В.~Ломоносова, \mbox{ekondranin@yandex.ru}}
\footnotetext[2]{Факультет вычислительной математики и~кибернетики
Московского государственного университета им.\ М.\,В.~Ломоносова;
Институт проб\-лем информатики Федерального исследовательского
центра <<Информатика и~управ\-ле\-ние>> Российской академии наук,
\mbox{vgushakov@mail.ru}}

\vspace*{-10pt}




\Abst{Изучена одноканальная система
массового обслуживания с~двумя типами требований, бесконечным
числом мест для ожидания, гиперэкспоненциальным входящим потоком 
и~профилактиками обслуживающего прибора при освобождении системы.
Тип  требования определяется случайно с~заданными вероятностями 
в~момент его поступления в~систему обслуживания. Требования первого
типа имеют относительный приоритет перед требованиями второго
типа. Найдено нестационарное совместное распределение числа
требований каждого типа в~системе. Профилактики прибора
заключаются в~том, что в~момент освобождения системы от требований
прибор на случайное время с~заданным распределением становится
недоступным для обслуживания. Если за время профилактики поступает
хотя бы одно требование, то начинается нормальное функционирование
системы. Если требования не поступают, то прибор отправляется на
новую профилактику. Такие системы хорошо описывают
функционирование большого числа реальных вычислительных и~информационных систем.}

\KW{гиперэкспоненциальный поток; профилактики
обслуживающего прибора; одноканальная система; относительный
приоритет; длина очереди}

\DOI{10.14357/19922264180405}
  
%\vspace*{4pt}


\vskip 10pt plus 9pt minus 6pt

\thispagestyle{headings}

\begin{multicols}{2}

\label{st\stat}

\section{Введение}

В классической системе массового обслуживания ожидание требований
в очереди связано только с~занятостью обслуживающего прибора. В~то
же время в~реальных системах сам  прибор может пребывать как 
в~активном, так и~в~неактивном состоянии. Такое неактивное
состояние прибора (в~литературе на английском языке используется
термин vacation, а~на русском~--- профилактика или прогулка) может
быть связано со многими причинами. В~част\-ности, сис\-те\-мы
обслуживания с~профилактиками прибора хорошо описывают
функционирование  реальных вычислительных и~информационных систем,
в которых наряду с~основными требованиями имеются второстепенные.
Второстепенные требования всегда присутствуют в~сис\-те\-ме, а~их
обслуживание может проводиться только тогда, когда нет основных,
т.\,е.\ в~фоновом режиме.

С точки зрения самого процесса профилактики прибора существует
несколько ее разновидностей. Во-пер\-вых, могут быть разными
правила, задающие условия начала профилактики: прибор может брать
перерыв только при  полном исчерпании требований в~очереди
(exhaustive service) либо при наличии определенного их числа
(nonexhaustive service). Во-вто\-рых, могут быть разными правила
возвращения прибора в~работу. С~этой точки зрения различают случаи
однократного (single vacation) и~многократного (multiple vacation)
перерыва в~работе. В~первом случае ушедший на профилактику прибор
после ее окончания находится в~рабочем состоянии независимо от
наличия требований в~системе. Во втором случае прибор, не
обнаружив новых требований в~очереди, уходит на новую
профилактику.


В работах~[1--4] можно найти обзор известных результатов, большое
число постановок задач, описание различных приложений и~обширную
библиографию по анализу систем с~профилактиками обслуживающего
прибора.


В настоящей работе исследуется совместное распределение длин
очередей в~нестационарном режиме в~однолинейной системе 
с~ожиданием, гиперэкспоненциальным входящим потоком, двумя типами
требований и~относительным приоритетом. Аналогичная неприоритетная
система обслуживания исследована в~[5].

\vspace*{-6pt}

\section{Описание модели}

Рассматривается однолинейная система массового обслуживания 
с~двумя приоритетными классами требований. Входящий поток~---
гиперэкспоненциальный с~функцией распределения интервалов между
поступлениями требований вида:
\begin{multline*}
A(t)=\sum\limits_{i=1}^kc_i\left(1-e^{-a_it}\right),\enskip t>0,\enskip
a_i>0,\enskip c_i>0,\\
a_i\ne a_j\,,\enskip i\ne j\,,\enskip  \sum\limits_{i=1}^k c_i=1\,.
\end{multline*}

Каждое поступившее требование направляется в~первый класс 
с~вероятностью~$p$ и~во второй класс с~вероятностью $1\hm-p$
независимо от остальных требований. Требования первого класса
обладают относительным приоритетом перед требованиями второго
класса. Длительности обслуживания требований $i$-го приоритетного
класса~--- независимые в~совокупности и~не зависящие от входящего
потока случайные величины с~функцией распределения~$B_i(x)$,
$i\hm=1,2.$
 Если в~некоторый момент времени система освободилась от требований, 
 то обслуживающий прибор
 отправляется на профилактику, которая длится случайное время с~функцией 
 распределения~$C(x).$
 Не ограничивая общности, будем считать, что $B_i(x)\hm<1$
 и~$C(x)\hm<1$  для любого~$x$ 
 и~существуют плотности
 распределения~$b_i(x)$ и~$c(x).$
  Обозначим:
$$
 \beta_i(s)=\int\limits_0^{\infty}e^{-sx}b_i(x)\,dx\,;\enskip 
  \gamma(s)=\int\limits_0^{\infty}e^{-sx}c(x)\,dx\,.
$$
Пока прибор находится на профилактике, он не доступен для
обслуживания. Если за время профилактики поступают требования,
после ее завершения начинается их обслуживание. Если ни одно
требование не поступает, то прибор отправляется на новую
профилактику. Длительности различных профилактик являются
независимыми случайными величинами 
и~не зависят от входящего потока и~времен обслуживания.

\section{Вспомогательные результаты}

  Рассмотрим многочлен по $\mu$ степени $k$ вида:
\begin{multline}
\label{1}
\prod\limits_{i=1}^k\left(\mu+a_i\right)-{}\\
{}-
\left(pz_1+(1-p)z_2\right)\sum\limits_{j=1}^kc_ja_j\prod\limits_{i\ne
j}\left(\mu+a_i\right)\,.
\end{multline}
Занумеруем его корни $\mu_1(z_1,z_2),\ldots,\mu_k(z_1,z_2)$ таким образом,
чтобы они были непрерывными функциями и~$\mu_1(1,1)\hm=0.$ Тогда
$\mathrm{Re}\, \mu_j\left(z_1,z_2\right)\hm<0$, $|z_1|\hm<1$, 
$|z_2|\hm<1,$ $\mu_i(z_1,z_2)\hm\ne \mu_j(z_1,z_2),$ $ i\hm\ne j$,
$j\hm=1,\ldots,k.$ Обозначим:
$$
\alpha_m(z_1,z_2)=\prod\limits_{j\ne m}\left(\mu_m\left(z_1,z_2\right)-
\mu_j\left(z_1,z_2\right)\right)\,.
$$
Справедливы следующие леммы.

\smallskip

\noindent
\textbf{Лемма~1.}\
\textit{Для любого $l=1,\ldots,\:k$ система уравнений}
$$
z_j=\beta_j(s-\mu_l(z_1,z_2)),\ \ j=1,2,
$$
\textit{имеет единственное решение $z_i=z_{il}(s)$ такое, 
что $|z_{il}(s)|\hm<1$ при $l\hm=2,\ldots, k,$ $\mathrm{Re}\, s\hm\geqslant 0,$ 
а~$z_{i1}(0)\hm=1$, $|z_{i1}(s)|\hm<1$ при} $\mathrm{Re}\, s\hm> 0$, $i\hm=1,2.$

\smallskip

\noindent
\textbf{Лемма~2.}\
\textit{При каждом $l\hm=1,\ldots,k$ уравнение}
$$
z_1=\beta_1\left(s-\mu_l(z_1,z_2)\right)
$$
\textit{имеет единственное решение $z_1\hm=z_{1l}(z_2,s),$ 
аналитическое в~области $\mathrm{Re}\, s\hm>0$, $|z_2|\hm<1.$
}

\smallskip

Положим
$$
\lambda_l(s)=\mu_l\left(z_{1l}(s),z_{2l}(s)\right)\,.
$$




\section{Распределение длины очереди}

  Гиперэкспоненциальный поток можно рас\-смат\-ри\-вать как
пуассоновский поток со случайной интен\-сив\-ностью~$a,$ которая
принимает $k$ различных значений $a_1,\ldots,a_k$  с~вероятностями
$c_1,\ldots,c_k.$ Текущее значение~$a$ разыгрывается в~момент
поступления требования и~не меняется между двумя соседними
поступлениями. Введем случайный процесс~$j(t)$ такой, что если
$a\hm=a_j$ в~момент времени $t,$ то $j(t)\hm=j.$

Целью работы является нахождение распределения случайного процесса
$\left(L_1(t),L_2(t)\right),$ где $L_i(t)$~--- число требований из
$i$-го приоритетного класса, находящихся в~системе в~момент
времени~$t.$

При сделанных предположениях относительно параметров изучаемой
системы обслуживания\linebreak процесс $\left(L_1(t),L_2(t)\right)$ не
является, вообще говоря, марковским. Пусть $i(t)=i$, $i\hm=1,2,$ если
в~момент времени~$t$ обслуживается требование из $i$-го
приоритетного класса, и~$i(t)\hm=0,$ если в~момент времени~$t$ прибор
находится на профилактике. Случайный процесс~$x(t)$ определим
следующим образом. Если $i(t)\hm\ne 0,$ то $x(t)$ есть
время, прошедшее с~начала обслуживания требования, находящегося на
приборе, до момента~$t.$ Если $i(t)\hm=0,$ то $x(t)$ есть время,
прошедшее с~начала профилактики прибора до момента~$t.$ Случайный
процесс $\left(L_1(t),L_2(t),i(t),j(t),x(t)\right)$ является
однородным марковским процессом. Положим
\begin{multline*}
P_{ij}(n_1,n_2,x,t)=\fr{\partial}{\partial x}
\mathbf{P}\left(L_1(t)=n_1,L_2(t)=n_2,\right.\\
\left. i(t)=i,j(t)=j,x(t)<x
\vphantom{L_1}\right)\,,\enskip 
 x\geqslant 0,\\ 
 j=1,\ldots,k,\enskip i=0,1,2;
\end{multline*}
\begin{gather*}
\eta_i(x)=\fr{b_i(x)}{1-B_i(x)},\ i=1,2;\enskip 
\eta_0(x)=\fr{c(x)}{1-C(x)}\,;\\
\delta_{i,j}=\begin{cases}
1,&\ i=j;\\ 
0,&\ i\ne j\,.
\end{cases}
\end{gather*}
Функции $P_{ij}(n_1,n_2,x,t)$  удовлетворяют при $x\hm>0$
системам дифференциальных уравнений:
\begin{multline}
\label{3}
\fr{\partial P_{ij}(n_1,n_2,x,t)}{\partial t}+\fr{\partial
P_{ij}(n_1,n_2,x,t)}{\partial
x}={}\\
{}=-(a_j+\eta_i(x))P_{ij}(n_1,n_2,x,t)+ {}\\
{}+
c_j\sum\limits_{l=1}^ka_l\left(p\:P_{il}(n_1-1,n_2,x,t)+{}\right.\\
\left.{}+
(1-p)P_{il}(n_1,n_2-1,x,t)\right)
\end{multline}
и краевым условиям при $x\hm=0$:
\begin{multline}
\label{5}
P_{0j}(n_1,n_2,0,t)=0,\ n_1+n_2>0;\\
P_{0j}(0,0,0,t)=\int\limits_0^{\infty}P_{0j}(0,0,x,t)\eta_0(x)\,dx+{}\\
 {}+\int\limits_0^{\infty}P_{1j}(1,0,x,t)\eta_1(x)dx+{}\\
 {}+
\int\limits_0^{\infty}P_{2j}(0,1,x,t)\eta_2(x)\,dx\,;
\end{multline}

\vspace*{-12pt}

\noindent
\begin{multline}
\label{6}
P_{1j}(n_1,n_2,0,t)+P_{2j}(n_1,n_2,0,t)={}\\
{}=\int\limits_0^{\infty}P_{1j}(n_1+1,n_2,x,t)\eta_1(x)\,dx+{}\\
{}+
\int\limits_0^{\infty}P_{2j}(n_1,n_2+1,x,t)\eta_2(x)\,dx+{}\\
{}+\int\limits_0^{\infty}P_{0j}(n_1,n_2,0,t)\eta_0(x)\,dx\,.
\end{multline}

Будем предполагать, что в~начальный момент времени $t\hm=0$ система
свободна от требований, а~с~начала профилактики прибора прошло
случайное время с~заданным распределением с~плотностью $d(x).$
Таким образом,
\begin{align*}
P_{ij}\left(n_1,n_2,x,0\right)&=0,\ i=1,2;
\\
P_{0j}\left(n_1,n_2,x,0\right)&=c_jd(x)\delta_{n_1+n_2,0},\ \
j=1,\ldots,k\,.
\end{align*}
Положим
\begin{multline*}
p_{ij}\left(z_1,z_2,x,s\right)={}\\
{}=\sum\limits_{n_1=0}^{\infty}
\sum\limits_{n_2=0}^{\infty}z_1^{n_1}z_2^{n_2}\!
\int\limits_0^{\infty}e^{-st}P_{ij}(n_1,n_2,x,t)\,dt\,;
\end{multline*}
$$
  \psi(s)=\int\limits_0^{\infty}e^{-sx}\,dx
  \int\limits_0^{\infty}\fr{c(u+x)d(u)}{1-C(u)}\,du\,.
$$
Тогда, учитывая начальные условия,  из \eqref{3}
получаем:
\begin{multline}
\label{7} 
\fr{\partial p_{ij}(z_1,z_2,x,s)}{\partial x}={}\\
{}=-\left(s+a_j+\eta_i(x)\right)p_{ij}
\left(z_1,z_2,x,s\right)+{}\\
{}+c_j\left(pz_1+(1-p)z_2\right)
\sum\limits_{l=1}^ka_lp_{il}\left(z_1,z_2,x,s\right),\\ 
i=1,2;
\end{multline}

\vspace*{-12pt}

\noindent
\begin{multline}
\label{8} 
\fr{\partial p_{0j}(z_1,z_2,x,s)}{\partial x}={}\\
{}=-\left(s+a_j+\eta_0(x)\right)p_{0j}\left(z_1,z_2,x,s\right)+{}\\
{}+c_j\left(pz_1+(1-p)z_2\right)\sum\limits_{l=1}^ka_lp_{0l}\left(z_1,z_2,x,s\right)+{}\\
{}+ c_jd(x).
\end{multline}
Решения \eqref{7} и~\eqref{8} имеют вид:
\begin{multline}
\label{9}
p_{ij}\left(z_1,z_2,x,s\right)=\left(1-B_i(x)\right)c_j\times{}\\
{}\times \sum\limits_{m=1}^k\fr{\gamma_i^{(m)}(z_1,z_2,s)}{\mu_m(z_1,z_2)+a_j}\,
e^{-(s-\mu_m(z_1,z_2))x}\,,\\
 i=1,2\,,
\end{multline}
\vspace*{-12pt}

\noindent
\begin{multline}
\label{10}
p_{0j}\left(z_1,z_2,x,s\right)={}\\
{}=\left(1-C(x)\right)
c_j\!\!\sum\limits_{m=1}^k\!\! e^{-(s-\mu_m(z_1,z_2))x}\!
\!\left(\!
\vphantom{\int\limits_{l=1}^k}
\delta^{(m)}\left(z_1,z_2,s\right)+{}\right.\\
%\left.
{}+\alpha_m^{-1}\left(z_1,z_2\right)
\prod\limits_{l=1}^k
\left(\mu_m\left(z_1,z_2\right)+a_l\right)\times{}\\
\left.{}\times \int\limits_0^x\!
e^{(s-\mu_m(z_1,z_2))u}
\fr{d(u)}{1-C(u)}\,du
\right)
\!\Bigg/ \!\left(\mu_m\left(z_1,z_2\right)+{}\right.\\
\left.{}+a_j\right)\,,
\end{multline}
где функции $\gamma_i^{(m)}(z_1,z_2,s)$  и~$\delta^{(m)}(z_1,z_2,s)$ являются
произвольными функциями указанных переменных и~определяются из
краевых условий. Из~\eqref{5} и~\eqref{6} получаем:
\begin{multline}
\label{11}
p_{1j}\left(z_1,z_2,0,s\right)+p_{2j}\left(z_1,z_2,0,s\right)={}\\
{}=z_1^{-1}\int\limits_0^{\infty}p_{1j}\left(z_1,z_2,x,s\right)\eta_1(x)\,dx+{}
\\
+z_2^{-1}\int\limits_0^{\infty}p_{2j}\left(z_1,z_2,x,s\right)\eta_2(x)\,dx+{}\\
{}+
\int\limits_0^{\infty}p_{0j}\left(z_1,z_2,x,s\right)\eta_0(x)\,dx
-p_{0j}\left(z_1,z_2,0,s\right)\,.
\end{multline}
Заметим, что $p_{0j}(z_1,z_2,0,s)$ не зависит от $z_1$ и~$z_2,$ т.\,е.\
$p_{0j}(z_1,z_2,0,s)\hm=q_j(s).$ 
Подставляя~\eqref{9} и~\eqref{10} в~\eqref{11}, получаем:
\begin{multline}
\label{12}
\gamma_1^{(m)}\left(z_1,z_2,s\right)\left(1-z_1^{-1}\beta_1(s-\mu_m(z_1,z_2))\right)+{}\\
{}+
\gamma_2^{(m)}(z_1,z_2,s)\left(1-z_2^{-1}\beta_2(s-\mu_m(z_1,z_2))\right)={}\\
{} =
\delta^{(m)}\left(z_1,z_2,s\right)\left(\gamma\left(s-\mu_m\left(z_1,z_2\right)\right)-1\right)+{}\\
{}+
\alpha_m^{-1}\left(z_1,z_2\right)\prod\limits_{l=1}^k
\left(\mu_m\left(z_1,z_2\right)+a_l\right)\psi\left(s-\mu_m(z_1,z_2)\right),\\
j=1,\ldots,k.
\end{multline}
В силу леммы~1 левая часть~\eqref{12} обращается в~0 при
$z_1\hm=z_{1m}(s)$ и~$z_2\hm=z_{2m}(s)$, $m\hm=1,\ldots,k.$ Следовательно,
\begin{multline}
\label{13}
\delta^{(m)}\left(z_{1m}(s),z_{2m}(s),s\right)={}\\
{}=\fr{\psi(s-\lambda_m(s))}{\alpha_m(z_{1m}(s),z_{2m}(s))
(1-\gamma(s-\lambda_m(s)))}\times{}\\
{}\times \prod\limits_{l=1}^k\left(\lambda_m(s)+a_l\right).
\end{multline}
Из \eqref{10} следует, что
$$
q_j(s)=c_j\sum\limits_{m=1}^k\fr{\delta^{(m)}(z_1,z_2,s)}{\mu_m(z_1,z_2)+a_j},\
j=1,\ldots,k .
$$
Решая эту систему уравнений относительно
$\delta^{(m)}(z_1,z_2,s),$ получаем:
\begin{multline}
\label{n1}
\delta^{(m)}(z_1,z_2,s)=\left(pz_1+(1-p)z_2\right)\times{}\\
{}\times
\fr{\prod\nolimits_{j=1}^k(\mu_m(z_1,z_2)+a_j)}
{\alpha_m(z_1,z_2)}\sum\limits_{l=1}^k\frac{a_lq_l(s)}{\mu_m(z_1,z_2)+a_l}.
\end{multline}
Подставляя в~\eqref{n1} $z_1\hm=z_{1m}(s)$ и~$z_2\hm=z_{2m}(s),$ имеем:
\begin{multline}
\label{14}
\delta^{(m)}\left(z_{1m}(s),z_{1m}(s),s\right)={}\\
{}=
\left(pz_{1m}(s)+(1-p)z_{2m}(s)\right)\times{}\\
{}\times
\fr{\prod\nolimits_{j=1}^k
(\lambda_m(s)+a_j)}{\alpha_m(z_{1m}(s),z_{1m}(s))}
\sum\limits_{l=1}^k\fr{a_lq_l(s)}{\lambda_m(s)+a_l}\,.
\end{multline}
Сравнивая два представления~\eqref{13} в~\eqref{14} для
$\delta^{(m)}(z_m(s),s),$ получаем систему уравнений для~$q_l(s)$:
\begin{multline*}
\sum\limits_{l=1}^k\fr{a_lq_l(s)}{\lambda_m(s)+a_l}={}\\
{}=\fr{\psi(s-\lambda_m(s))}{(pz_{1m}(s)+(1-p)z_{2m}(s))
(1-\gamma(s-\lambda_m(s)))},\\
m=1,\ldots,k\,,
\end{multline*}
из которой находим
\begin{multline}
\hspace*{-3pt}q_l(s)=c_l\prod\limits_{j=1}^k
\left(\lambda_l(s)+a_j\right) 
\sum\limits_{m=1}^k
%\fr
\psi(s-\lambda_m(s))\!\Bigg/ \!
\Bigg(\left(1-{}\right.\\
\left.
{}-\gamma\left(s-\lambda_m(s)\right)\right)(\lambda_m(s)+a_l)\times{}\\
{}\times \prod\limits_{n\ne m}(\lambda_m(s)-\lambda_n(s))\!\Bigg).
\label{15}
\end{multline}
Подставляя \eqref{15} в~\eqref{n1} и~учитывая~\eqref{1}, получаем:
\begin{multline*}
\delta^{(m)}(z_1,z_2,s)=\fr{(pz_1+(1-p)z_2)}{\alpha_m(z_1,z_2)}\times
\\
\times\sum\limits_{j=1}^k
\fr{\psi(s-\lambda_j(s))\prod\nolimits_{l=1}^k(\lambda_j(s)+a_l)}
{(pz_{1j}(s)+(1-p)z_{2j}(s))(1-\gamma(s-\lambda_j(s)))}\times{}\\
{}\times\prod\limits_{\nu\ne j}
\fr{\mu_m(z_1,z_2)-\lambda_{\nu}(s)}{\lambda_j(s)-\lambda_{\nu}(s)}\,.
\end{multline*}
Положим
$$
\lambda_m(z_2,s)=\mu_m\left(z_{1m}(z_2,s),z_2\right),\enskip m=1,\ldots,k\,.
$$
Подставляя в~\eqref{12} $z_1\hm=z_{1m}(z_2,s)$, имеем:
\begin{multline}
\label{1q}
\gamma_2^{(m)}\left(z_{1m}(z_2,s),z_2,s\right)={}\\
{}=\fr{\delta^{(m)}(z_{1m}(z_2,s),z_2,s)(\gamma_m(s-\lambda_m(z_2,s))-1)}
{1-z_2^{-1}\beta_2(s-\lambda_m(z_2,s))}+{}
\\
{}+\alpha_m^{-1}(z_{1m}(z_2,s),z_2)\psi(s-\lambda_m(z_2,s))
\prod\limits_{l=1}^k\left(\lambda_m(z_2,s)+{}\right.\\
\left.{}+a_l\right)\!\Bigg/\!
\left(
1-z_2^{-1}\beta_2(s-\lambda_m(z_2,s))\right).
\end{multline}
Далее, из~\eqref{9} следует:
$$
p_{2j}(z_1,z_2,0,s)=c_j\sum\limits_{m=1}^k
\fr{\gamma_2^{(m)}(z_1,z_2,s)}{\mu_m(z_1,z_2)+a_j}\,.
$$
Отсюда
\begin{multline}
\label{2q}
\gamma_2^{(m)}(z_1,z_2,s)=\fr{pz_1+(1-p)z_2}{\alpha_m(z_1,z_2)}\times{}\\
{}\times
\prod\limits_{j=1}^k(\mu_m(z_1,z_2)+a_j)
\sum\limits_{l=1}^k\fr{a_lp_{2l}(z_1,z_2,0,s)}{\mu_m(z_1,z_2)+a_l}\,.
\end{multline}
Так как $p_{2j}(z_1,z_2,0,s)$ не зависит от $z_1$, то
\begin{multline}
\label{3q}
p_{2j}\left(z_1,z_2,0,s\right)={}\\
{}=c_j
\sum\limits_{m=1}^k\fr{\gamma_2^{(m)}\left(z_{1m}(z_2,s),z_2,s\right)}{\lambda_m(z_2,s)+a_j}\,.
\end{multline}
Таким образом, соотношения~\eqref{1q}--\eqref{3q} полностью
определяют $\gamma_2^{(m)}(z_1,z_2,s)$ при любых $z_1$ и~$z_2$.
Теперь из~\eqref{12} можно найти $\gamma_2^{(m)}(z_1,z_2,s)$.

Все функции, необходимые для вычисления $p_{ij}(z_1,z_2,x,s)$,
$i\hm=0,1,2$, $j\hm=1,\ldots,k,$ найде-\linebreak\vspace*{-12pt}

\columnbreak

\noindent
ны. Искомая производящая функция
процесса $(L_1(t),L_2(t))$ равна:

\noindent
\begin{multline*}
\int\limits_0^{\infty}e^{-st}\mathbf{E}
z_1^{L_1(t)} z_2^{L_2(t)}\,dt={}\\
{}=
\sum\limits_{i=0}^2\sum\limits_{j=1}^k\int\limits_0^{\infty}p_{ij}
\left(z_1,z_2,x,s\right)\,dx\,.
\end{multline*}

\vspace*{-18pt}

{\small\frenchspacing
 {%\baselineskip=10.8pt
 \addcontentsline{toc}{section}{References}
 \begin{thebibliography}{9}
\bibitem{1-u}
\Au{Doshi B.\,T.} Queueing systems with vacations~--- a~survey~// 
Queueing Syst., 1986. Vol.~1.  P.~29--66.
\bibitem{2-u}
\Au{Takagi H.} Time-dependent analysis of $M\vert G\vert 1$ vacation models 
with exhaustive service~// Queueing Syst.,
1990. Vol.~6.  P.~369--390.
\bibitem{3-u}
\Au{Li J., Tian N., Zhang~Z.\,G. , Luh~H.\,P.} 
Analysis of the $M\vert G\vert 1$ queue with exponentially working vacations~--- 
a~matrix analytic approach~// Queueing Syst., 2009. Vol.~61.
P.~139--166.
\bibitem{4-u}
\Au{Bouman N., Borst S.\,C., Boxma~O.\,J., Leeuwaarden~J.\,S.\,H.} 
Queues with random back-offs~// Queueing Syst.,
2014. Vol.~77. P.~33--74.
\bibitem{5-u}
\Au{Ушаков~В.\,Г.} Система обслуживания с~гиперэкспоненциальным входящим потоком 
и~профилактиками прибора~// Информатика и~её применения, 2016. Т.~10. 
Вып.~2. С.~93--98.
 \end{thebibliography}

 }
 }

\end{multicols}

\vspace*{-9pt}

\hfill{\small\textit{Поступила в~редакцию 11.05.18}}

\vspace*{6pt}

%\pagebreak

%\newpage

%\vspace*{-28pt}

\hrule

\vspace*{2pt}

\hrule

%\vspace*{-2pt}

\def\tit{A~HEAD OF~THE~LINE PRIORITY QUEUE\\ WITH~WORKING VACATIONS}

\def\titkol{A head of the line priority queue with working vacations}

\def\aut{E.\,S.~Kondranin$^1$ and~V.\,G.~Ushakov$^{1,2}$}

\def\autkol{E.\,S.~Kondranin and~V.\,G.~Ushakov}

\titel{\tit}{\aut}{\autkol}{\titkol}

\vspace*{-11pt}


\noindent
$^1$Department of 
Mathematical Statistics, Faculty of Computational Mathematics and Cybernetics, 
M.\,V.~Lo\-mo-\linebreak
$\hphantom{^1}$no\-sov Moscow State University, 1-52~Leninskiye Gory, 
Moscow 119991, GSP-1, Russian Federation

\noindent
$^2$Institute of Informatics Problems, Federal Research Center 
``Computer Science and Control'' of the Russian\linebreak
$\hphantom{^1}$Academy of Sciences,  44-2~Vavilov Str., Moscow 119333, Russian Federation

\def\leftfootline{\small{\textbf{\thepage}
\hfill INFORMATIKA I EE PRIMENENIYA~--- INFORMATICS AND
APPLICATIONS\ \ \ 2018\ \ \ volume~12\ \ \ issue\ 4}
}%
 \def\rightfootline{\small{INFORMATIKA I EE PRIMENENIYA~---
INFORMATICS AND APPLICATIONS\ \ \ 2018\ \ \ volume~12\ \ \ issue\ 4
\hfill \textbf{\thepage}}}

\vspace*{3pt}



\Abste{The authors analyze the single-server queueing system with 
two types of customers, head of the line priority, hyperexponential 
input stream, and working vacations. The authors obtain the Laplace 
transform (with respect to an arbitrary point in time) of the joint 
distribution of server state, queue size, and elapsed time in that state. 
The authors restrict themselves to a~system with exhaustive service (the 
queue must be empty when the server starts a vacation) and multiple vacations. 
The queueing systems with vacations have been well studied because of their 
applications in modeling computer networks, communication, and manufacturing 
systems. For example, in many digital systems, the processor is multiplexed 
among a~number of jobs and, hence, is not available all the time to handle one job type. 
Besides such an application, theoretical interest in vacation models 
has been aroused with respect to their relationship with polling models.}

\KWE{hyperexponential input stream; working vacations; single server; 
head of the line priority; queue length}



\DOI{10.14357/19922264180405}

\vspace*{-14pt}

\Ack
\noindent
This work was supported by the Russian Foundation for Basic Research 
(project 18-07-00678).


%\vspace*{6pt}

  \begin{multicols}{2}

\renewcommand{\bibname}{\protect\rmfamily References}
%\renewcommand{\bibname}{\large\protect\rm References}

{\small\frenchspacing
 {%\baselineskip=10.8pt
 \addcontentsline{toc}{section}{References}
 \begin{thebibliography}{9}
\bibitem{1-u-1}
\Aue{Doshi, B.\,T.} 1986. Queueing systems with vacations~--- a~survey. 
\textit{Queueing Syst.} 1:29--66.
\bibitem{2-u-1}
\Aue{Takagi, H.} 1990. Time-dependent analysis of $M\vert G\vert M\vert 1$ 
vacation models with exhaustive service. \textit{Queueing Syst.} 6:369--390.
\bibitem{3-u-1}
\Aue{Li, J., N. Tian, Z.\,G.~Zhang,  and H.\,P.~Luh.} 2009. Analysis of the 
$M\vert G\vert 1$ queue with exponentially working vacations~--- 
a~matrix analytic approach. \textit{Queueing Syst.} 61:139--166.
{\looseness=1

}
\bibitem{4-u-1}
\Aue{Bouman, N., S.\,C.~Borst, O.\,J.~Boxma, and J.\,S.\,H.~Leeuwaarden.} 
2014. Queues with random back-offs. \textit{Queueing Syst.} 77:33--74.
\bibitem{5-u-1}
\Aue{Ushakov, V.\,G.} 2016. Sistema obsluzhivaniya s~gipereksponentsialnym 
vkhodyashchim potokom i~profilaktikami\linebreak pribora [Queueing system with working 
vacations and hyperexponential input stream]. 
\textit{Informatika i~ee Primeneniya~--- Inform. Appl.} 10(2):93--98.
\end{thebibliography}

 }
 }

\end{multicols}

\vspace*{-6pt}

\hfill{\small\textit{Received May 11, 2018}}

%\pagebreak

%\vspace*{-18pt}

\Contr

\noindent
\textbf{Kondranin Egor S.} (b.\ 1995)~---  MSc student, Department of 
Mathematical Statistics, Faculty of Computational Mathematics and Cybernetics, 
M.\,V.~Lomonosov Moscow State University, 1-52~Leninskiye Gory, 
Moscow 119991, GSP-1, Russian Federation; \mbox{ekondranin@yandex.ru}

\vspace*{6pt}

\noindent
\textbf{Ushakov Vladimir G.} (b.\ 1952)~--- 
Doctor of Science in physics and mathematics, professor, Department of Mathematical 
Statistics, Faculty of Computational Mathematics and Cybernetics, 
M.\,V.~Lomonosov Moscow State University, 1-52~Leninskiye Gory, Moscow 119991, 
GSP-1, Russian Federation; 
senior scientist, Institute of Informatics Problems, Federal Research Center 
``Computer Science and Control'' of the Russian Academy of Sciences, 
44-2~Vavilov Str., Moscow 119333, Russian Federation; \mbox{vgushakov@mail.ru}
\label{end\stat}

\renewcommand{\bibname}{\protect\rm Литература}           %6
\def\stat{bosov+stef}

\def\tit{УПРАВЛЕНИЕ ВЫХОДОМ СТОХАСТИЧЕСКОЙ ДИФФЕРЕНЦИАЛЬНОЙ СИСТЕМЫ 
ПО~КВАДРАТИЧНОМУ КРИТЕРИЮ. I.~ОПТИМАЛЬНОЕ РЕШЕНИЕ МЕТОДОМ 
ДИНАМИЧЕСКОГО ПРОГРАММИРОВАНИЯ$^*$}

\def\titkol{Управление выходом стохастической дифференциальной системы 
по~квадратичному критерию. I}
%.~Оптимальное решение методом 
%динамического программирования}

\def\aut{А.\,В.~Босов$^1$, А.\,И.~Стефанович$^2$}

\def\autkol{А.\,В.~Босов, А.\,И.~Стефанович}

\titel{\tit}{\aut}{\autkol}{\titkol}

\index{Босов А.\,В.}
\index{Стефанович А.\,И.}
\index{Bosov A.\,V.}
\index{Stefanovich A.\,I.}




{\renewcommand{\thefootnote}{\fnsymbol{footnote}} \footnotetext[1]
{Работа выполнена при частичной поддержке РФФИ (проект 16-07-00677).}}


\renewcommand{\thefootnote}{\arabic{footnote}}
\footnotetext[1]{Институт проблем информатики Федерального исследовательского центра <<Информатика 
и~управление>> Российской академии наук, \mbox{AVBosov@ipiran.ru}}
\footnotetext[2]{Институт проблем информатики Федерального исследовательского центра <<Информатика 
и~управление>> Российской академии наук, \mbox{AStefanovich@frccsc.ru}}

%\vspace*{8pt}



  
  \Abst{Решается задача оптимального управления для диффузионного процесса 
Ито и~линейного управ\-ля\-емо\-го выхода. Рассматриваемая постановка близка 
к~классической ли\-ней\-но-квад\-ра\-тич\-ной гауссовской задаче управления 
(linear-quadratic Gaussian (LQG) control). Отличия состоят в~том, что состояние описывается нелинейным 
дифференциальным уравнение Ито $dy_t\hm= A_t(y_t) \,dt\hm+ \Sigma_t(y_t)\,dv_t$ 
и~не зависит от управ\-ле\-ния~$u_t$, оптимизации подлежит управ\-ля\-емый 
линейный выход $dz_t\hm= a_t y_t\,dt\hm+ b_t z_t \,dt\hm+ c_t u_t \,dt\hm+ \sigma_t\, 
dw_t$. Дополнительные обобщения внесены в~квад\-ра\-тич\-ный критерий качества 
с~целью воз\-мож\-ности постановки таких задач, как отслеживание выходом 
состояния или управ\-ле\-ни\-ем~--- линейной комбинации состояния и~выхода. Для 
решения используется метод динамического программирования. Функцию 
Беллмана позволяет найти предположение о~ее структуре вида $V_t(y,z)\hm= 
\alpha_t z^2\hm+ \beta_t(y)z \hm+\gamma_t(y)$. Решение дают три 
дифференциальных уравнения для коэффициентов~$\alpha_t$, $\beta_t(y)$ 
и~$\gamma_t(y)$. Эти уравнения со\-став\-ля\-ют оптимальное решение 
рас\-смат\-ри\-ва\-емой задачи.}
  
  \KW{стохастическое дифференциальное уравнение; оптимальное управ\-ле\-ние; 
динамическое программирование; функция Беллмана; уравнение Риккати; 
линейные уравнения параболического типа}

\DOI{10.14357/19922264180314}
  
%\vspace*{4pt}


\vskip 10pt plus 9pt minus 6pt

\thispagestyle{headings}

\begin{multicols}{2}

\label{st\stat}

\section{Введение}

     Ключевые результаты в~области оптимизации стохастических 
динамических систем, со\-став\-ля\-ющие классическую теорию управления, 
получены более~40~лет назад (такова работа~[1] в~отношении задачи 
управ\-ле\-ния ли\-ней\-но-гаус\-сов\-ски\-ми стохастическими сис\-те\-ма\-ми по 
квад\-ра\-тич\-но\-му критерию). К~классической тео\-рии следует относить 
линейные модели стохастических сис\-тем и~квадратичный критерий качества. 
Это исходный базис, на котором основано множество успешно 
исследованных и~решенных задач стохастического управ\-ле\-ния 
и~оптимизации. 

Дальнейшее развитие~--- это новые модели и~критерии, но 
прежде всего это новые методы: от тео\-рии линейных регуляторов, метода 
динамического программирования и~принципа максимума к~адаптивному 
и~минимаксному подходу, импульсному управ\-ле\-нию и~т.\,д. Множество 
инноваций как в~час\-ти моделей, так и~в~час\-ти математического аппарата, 
имевших мес\-то в~по\-сле\-ду\-ющие годы, существенно обогатили тео\-рию 
управ\-ле\-ния. Но и~до настоящего времени линейные модели и~квадратичный 
критерий, несмотря на всю справедливую критику в~отношении их 
аде\-кват\-ности и~гиб\-кости, сохраняют исследовательский интерес и~находят 
современные области приложения.
     
     Не претендуя на сколь\-ко-ни\-будь полное обосно\-ва\-ние последнего 
тезиса, приведем несколько примеров, показавшихся наиболее ин\-те\-рес\-ными. 

Так, в~[2] решается ли\-ней\-но-квад\-ра\-тич\-ная за\-да\-ча в~игровой 
постановке с~запаздыванием. В~близ\-кой по модели работе~[3] задача 
управ\-ле\-ния ставится в~терминах $H_\infty$-ро\-баст\-ности. Точнее \mbox{называть} 
эту тематику $H_2/H_\infty$-управ\-ле\-ни\-ем, и~работ по этой теме очень 
много. Аккуратности ради следует уточнить, что под линейными 
понимаются модели с~мультипликативными по состоянию воз\-му\-ще\-ниями. 

Совсем другой класс моделей, особо популярных в~по\-след\-ние годы, 
составляют скачкообразные процессы. Например, линейные уравнения 
в~сочетании с~пуассоновскими скачками в~[4] используются в~моделях, 
описывающих различные показатели функционирования сетевых протоколов 
передачи данных транспортного уровня. Телекоммуникации представляют 
в~последние годы самый популярный прикладной материал для 
исследований, работ по этой проб\-ле\-ма\-ти\-ке множество, математические 
техники привлекаются самые разные и~самые современные, но и~линейным 
моделям место находится. Еще один любопытный пример исследования 
скачкообразного процесса и~оптимизации на основе квад\-ра\-тич\-но\-го критерия 
можно найти в~[5] применительно к~задаче инвестирования на финансовом 
рынке. Наконец, упомянем еще работу~[6], подводящую итог исследований 
в~отношении классической детерминированной  
ли\-ней\-но-квад\-ра\-тич\-ной задачи с~использованием техники матричных 
неравенств.
     
     В данной работе также эксплуатируются привлекательные свойства 
линейных моделей и~квад\-ра\-тич\-но\-го критерия, причем в~стохастической 
постановке. На\-прав\-ле\-ни\-ем для обобщения \mbox{выбрана} модель динамики 
сис\-те\-мы: основные усилия на\-прав\-ле\-ны на то, чтобы сделать ее нелинейной. 
Кроме того, пред\-став\-лен\-ная постановка может рас\-смат\-ри\-вать\-ся и~как 
обобщение ранее решенной задачи в~дискретном времени~[7, 8] на время 
непрерывное. В~упомянутых работах помимо собственно модельной 
постановки важна еще и~привлекаемая прикладная об\-ласть~--- 
функционирование сложных программных сис\-тем. Результатов, 
ориентированных непосредственно на такие приложения, к~настоящему 
времени пренебрежимо мало, поэтому~[7, 8]~--- это еще и~прикладное 
обоснование рас\-смат\-ри\-ва\-емой далее задачи.
     
     Оптимизируемая динамическая сис\-те\-ма описывается двумя 
уравнениями. Состояние задается нелинейным стохастическим 
дифференциальным уравнением Ито, не содержащим управ\-ля\-емой 
переменной. Возмущение здесь описывается стандартным винеровским 
процессом, накладываются простые условия существования 
и~един\-ст\-вен\-ности решения. Поскольку состояние не управ\-ля\-ет\-ся, то уместно 
его интерпретировать как слож\-ное внешнее возмущение. Вторая 
переменная~--- управ\-ля\-емый выход~--- задается линейным стохастическим 
дифференциальным уравнением. Цель оптимизации выхода формируется 
квадратичным критерием общего вида. Формальная постановка задачи 
приведена в~сле\-ду\-ющем разделе.
     
     Для решения задачи используется метод динамического 
программирования, решается уравнение Беллмана~[9]. Соответственно, 
в~результате получаются аналитические выражения и~для оптимального 
управ\-ле\-ния, и~для значения функционала качества. Технически 
традиционный, стандартный подход к~задаче обременен, пожалуй, 
единственной проблемой~--- поиском верного пред\-став\-ле\-ния структуры 
функции Беллмана. Справиться с~этой проблемой в~большей степени удается 
за счет результата, полученного при решении дискретного по времени 
аналога рассматриваемой постановки~\cite{8-bos}. Конечные соотношения 
для оптимального решения, как и~во всех подобных задачах, включая 
классическую ли\-ней\-но-квад\-ра\-тич\-ную, содержат решения 
определенных дифференциальных уравнений (обыкновенных и~в~частных 
производных). Вывод этих уравнений и~со\-став\-ля\-ет содержание первой час\-ти 
данной работы. Во второй части будет обсуждаться их приближенное 
чис\-лен\-ное решение и~компьютерные эксперименты.
     
     Кратко обозначим основные положения, при\-вле\-ка\-емые далее 
к~решению задачи, следуя в~основном обозначениям 
и~терминологии~\cite{9-bos}, а~именно: будем рассматривать задачу 
оптимального управления в~стохастической динамической сис\-те\-ме по полной 
информации, применяя метод динамического программирования. В~качестве 
целевого функционала, опре\-де\-ля\-юще\-го качество управ\-ле\-ния $U_0^T\hm= \{ 
u_t,\ 0\leq t\leq T\}$, выступает
     \begin{equation}
     J\left(U_0^T\right)={\sf E}\left\{ \int\limits_0^T L_t \left(x_t, u_t\right)\,dt+ 
l\left(x_T\right)\right\}\,.
     \label{e1-bos}
     \end{equation}
Здесь ${\sf E}\{\cdot\}$~--- оператор математического ожидания; $x_t$~--- 
случайный процесс, описываемый стохастическим дифференциальным 
уравнением Ито
     \begin{equation}
     dx_t=m_t\left( x_t, u_t\right) dt+ \sigma_t\left( x_t\right)dW_t\,,\enskip 
x_0=X\,,
     \label{e2-bos}
     \end{equation}
где $W_t$~--- стандартный винеровский процесс подходящей раз\-мер\-ности; 
$X$~--- случайный вектор.

     $U_0^T$ будем выбирать из класса допустимых неупреждающих (по 
отношению к~$W_t$) управлений~\cite{9-bos}. Соответственно, 
относительно функций сноса и~диффузии~$m_t$ и~$\sigma_t$  
в~(\ref{e2-bos}) будем предполагать выполненными ка\-кие-ли\-бо условия 
существования сильного решения для заданного до\-пус\-ти\-мо\-го управ\-ле\-ния. 
Например, для управ\-ле\-ния с~обратной связью $u_t\hm= u_t(x_t)$ будем 
считать, что $m_t(x,u_t(x))$ и~$\sigma_t(x)$ удовлетворяют условию 
линейного рос\-та и~локальному условию Липшица по~$x$ равномерно 
по~$t$ (т.\,е.\ условиям Ито).
     
     Для поиска оптимального управления, минимизирующего $J(U_0^T)$, 
рас\-смат\-ри\-ва\-ет\-ся функция Беллмана
     \begin{equation}
     V_t(x)=\left.\mathop{\mathrm{inf}}\limits_{U_t^T} {\sf E} \left\{ \int\limits_t^T 
L_t \left( x_t, u_t\right)\,dt+l\left( x_T\right) \right\vert \mathcal{F}_t^x\right\}\,,
     \label{e3-bos}
     \end{equation}
где $\mathcal{F}_t^x$~--- $\sigma$-ал\-геб\-ра, по\-рож\-ден\-ная~$x_\tau$, 
$0\hm\leq \tau\hm\leq t$, ${\sf E}\{\cdot\vert \mathcal{F}\}$~--- оператор условного 
математического ожидания относительно~$\mathcal{F}$. Соответственно, 
в~качестве достаточного условия оп\-ти\-маль\-ности воспользуемся уравнением 
динамического программирования
\begin{multline}
\fr{\partial V_t(x)}{\partial t} +\fr{1}{2}\sum\limits^n_{i,j=1} \sigma^2_{t_{ij}}
\fr{\partial^2 V_t(x)}{\partial x_i \partial x_j}+{}\\
{}+\min\limits_u\left[  
\sum\limits^n_{i=1} m_{t_i} \fr{\partial V_t(x)}{\partial x_i} + L_t(x,u)\right] 
=0\,,\\
V_T(x)=l(x)\,,
\label{e4-bos}
\end{multline}
где $m_{t_i}$~--- $i$-й элемент век\-тор-функ\-ции~$m_t(x,u)$; 
$\sigma^2_{t_{ij}} \hm= \sum\nolimits^m_{k=1} 
\sigma_{t_{ik}}\sigma_{t_{ki}}$, $\sigma_{t_{ij}}$~--- $i$-й по строке, $j$-й 
по столб\-цу элемент мат\-рич\-ной функции~$\sigma_t(x)$; $n$ и~$m$~--- 
размерности~$x_t$ и~$W_t$ соответственно.

     Традиционно в~рамках применения метода динамического 
программирования будем предполагать, что функции~$L_t$, $l$, $m_t$ 
и~$\sigma_t$ обеспечивают существование хотя бы одного решения 
уравнения~(\ref{e4-bos}), а~следовательно, и~оптимального 
управления~$u_t^*$, $0\hm\leq t\hm\leq T$, до\-став\-ля\-юще\-го минимум 
целевому функционалу~(\ref{e1-bos}). Задача оптимизации далее получается 
путем указания конкретных выражений для~$L_t$, $l$, $m_t$ и~$\sigma_t$.

\section{Постановка задачи управления выходом}

     Рассматриваемые далее случайные функции будут предполагаться 
скалярными. Такое упрощение позволит разгрузить выкладки и~итоговые 
выражения от не самых существенных деталей.
     
     Рассмотрим стохастическую дифференциальную сис\-те\-му, со\-сто\-яние 
которой представляет диффузи\-он\-ный процесс~$y_t$, описываемый 
нелинейным стохастическим дифференциальным уравнением Ито
     \begin{equation}
     dy_t=A_t\left( y_t\right) dt +\Sigma_t \left( y_t\right) dv_t\,,\enskip 
y_0=Y\,,
     \label{e5-bos}
     \end{equation}
где $v_t$~--- стандартный (одномерный) винеровский процесс; $Y$~--- 
случайная величина с~конечным вторым моментом; функции~$A_t$ 
и~$\Sigma_t$ удовлетворяют условиям Ито:
\begin{equation*}
\left\vert A_t(y)\right\vert +\left\vert \Sigma_t(y)\right\vert \leq C(1+\vert y\vert )\ 
\mbox{для\ всех } 0\leq t\leq T\,;
\end{equation*}

\vspace*{-12pt}

\noindent
\begin{multline*}
\hspace*{-2.10051pt}\left\vert A_t\left(y_1\right) -A_t \left( y_2\right) \right\vert +\left\vert 
\Sigma_t\left( y_1\right) -\Sigma_t \left(y_2\right)\right\vert \leq
C\left\vert y_1-y_2\right\vert\\
 \mbox{для\ всех\ } 0\leq t\leq T\ \mbox{и } 
y_1,y_2\in \mathbb{R}^1\,,
\end{multline*}
обеспечивающим существование единственного сильного (потраекторного) 
решения уравнения.
     
     Будем считать, что~$y_t$ описывает состояние некоторой 
динамической системы. Соответственно, поведение этой сис\-те\-мы опишем 
выходом, линейно связанным с~со\-сто\-янием:
     \begin{equation}
     dz_t=a_t y_t \,dt+ b_t z_t \,dt+ c_t u_t \,dt+\sigma_t \,dw_t\,,\enskip
     z_0=Z\,.
     \label{e6-bos}
     \end{equation}
Здесь $w_t$~--- не зависящий от~$v_t$, $Y$ и~$Z$ стандартный (одномерный) 
винеровский процесс; $Z$~--- случайная величина с~конечным вторым 
моментом; $u_t$~--- допустимое неупреждающее управ\-ле\-ние, качество 
которого определяется целевым функционалом следующего вида:
\begin{multline}
\!\hspace*{-3.98538pt}J\left( U_0^T\right) ={\sf E}\left\{ \int\limits_0^T \!\left( S_t\left( s_ty_t-g_t z_t -h_t 
u_t\right)^2 +G_t z_t^2+{}\right.\right.\\
\left.\left.{}+ H_t u_t^2
\vphantom{S_t\left( s_ty_t-g_t z_t -h_t 
u_t\right)^2}
\right) dt+S_T\left( s_T y_T -g_T 
z_T\right)^2+G_T z_T^2
\vphantom{\int\limits_0^T}\right\}\,,
\label{e7-bos}
\end{multline}
где $S_t$, $G_t$ и~$H_t$~--- неотрицательные функции\linebreak
$0\hm\leq t\hm\leq T$. 
Такой критерий отражает физический смысл задачи распределения ресурсов 
со\-глас\-но аналогичной~(\ref{e5-bos})--(\ref{e7-bos}) задаче для дис\-крет\-но\-го 
времени, рас\-смот\-рен\-ной в~\cite{7-bos}. В~част\-ности,  
функци\-онал~(\ref{e7-bos}) поз\-во\-ля\-ет ставить задачи отслеживания
 выходом 
со\-сто\-яния сис\-те\-мы, используя сла\-га\-емое $(y_t\hm- z_t)^2$, или 
управлением~--- линейной комбинации со\-сто\-яния и~выхода, сла\-га\-емое типа\linebreak 
$(y_t\hm+ z_t\hm- u_t)^2$. Поскольку задача формулируется 
в~предположении наличия пол\-ной информации о~со\-сто\-янии~$y_t$ 
и~выходе~$z_t$ (соответствующую $\sigma$-ал\-геб\-ру 
обозначим~$\mathcal{F}_t^{y,z}$), то допустимое управ\-ле\-ние ищется 
в~классе~$\mathcal{F}_t^{y,z}$-из\-ме\-ри\-мых неупреждающих функций 
(и,~как будет показано далее, оказывается управ\-ле\-ни\-ем с~обратной связью).

     Функции~$a_t$, $b_t$, $c_t$ и~$\sigma_t$ будем предполагать 
ограниченными: $\vert a_t\vert \hm+ \vert b_t\vert \hm+\vert c_t\vert \hm+ \vert 
\sigma_t \vert \hm\leq C$ для всех $0\hm\leq t\hm\leq T$, процесс  
управления~--- допустимым не\-упреж\-да\-ющим~\cite{9-bos}, обеспечивая, 
таким образом, существование сильного решения урав\-не\-ния~(\ref{e6-bos}) 
для любого допустимого управ\-ления.
     
     Задачу составляет поиск~$u_t^*$~--- допустимого управ\-ле\-ния, 
доставляющего минимум квад\-ра\-тич\-но\-му функционалу~$J(U_0^T)$.
      
     Поставленная задача очевидным образом формулируется в~терминах 
введенных выше в~(\ref{e1-bos})--(\ref{e3-bos}) обозначений, а~именно: 
     требуется обозначить
     \begin{gather*}
      x_t=\begin{pmatrix}
     y_t\\ z_t\end{pmatrix};\quad  m_t(x_t, u_t)=\begin{pmatrix}
     A_t(y_t)\\ a_t y_t +b_t z_t +c_t u_t\end{pmatrix};\\
     \sigma_t(x_t)= \begin{pmatrix}
     \Sigma_t(y_t)& 0\\
     0& \sigma_t\end{pmatrix};\quad W_t=\begin{pmatrix}
     v_t \\ w_t\end{pmatrix}
     %     \label{e8-bos}
     \end{gather*}
для записи уравнения со\-сто\-яния типа~(\ref{e2-bos}) и
\begin{align*}
L_t(x,u)&= L_t(y,z,u) ={}\\
&\hspace*{3mm}{}=S_t\left( s_t y-g_t z -h_t u\right)^2 +G_t z^2 +H_t  u^2\,;\\
l(x)&= l(y,z) =S_T \left( S_T y-g_T z\right)^2 +G_T z^2
%\label{e9-bos}
\end{align*}
для записи целевого функционала в~виде~(\ref{e1-bos}).

     Функция Беллмана~(\ref{e3-bos}) принимает вид 
     $V_t(x)\hm= V_t(y,z)$. Для записи со\-от\-вет\-ст\-ву\-юще\-го~(\ref{e4-bos}) 
уравнения Беллмана для~$V_t(y,z)$ заметим, что
     $$
     \left( \sigma^2_{t_{ij}}\right)_{i,j=1,2}= \begin{pmatrix}
     \Sigma_t^2(y) & 0\\
     0 & \sigma_t^2\end{pmatrix}\,.
     $$
     
     С~учетом перечисленных обозначений урав\-не\-ние динамического 
программирования~(\ref{e4-bos}) принимает вид:
     \begin{multline}
     \fr{\partial V_t(y,z)}{\partial t} +\fr{1}{2}\left( \Sigma_t^2(y) \fr{\partial^2 
V_t(y,z)} {\partial y^2}+\sigma_t^2\fr{\partial^2 V_t(y,z)} {\partial 
z^2}\right)+{}\\
    {}+\min\limits_u\! \left[ A_t(y) \fr{\partial V_t(y,z)}{\partial y}+\left( a_t 
y+b_t z+c_t u\right) \fr{\partial V_t(y,z)}{\partial z} +{}\right.\hspace*{-3pt}\\
\left.{}+ S_t\left( s_t y-g_t z-h_t 
u\right)^2+G_t z^2+H_t u^2
     \vphantom{\fr{\partial V_t(y,z)}{\partial y}}\right] =0\,,\\
     V_T(y,z)=S_T\left( s_T y-g_T z\right)^2+G_T z^2\,.
     \label{e10-bos}
     \end{multline}
     Это и~есть то самое уравнение, которое требуется решить: 
существование решения данного урав\-не\-ния суть достаточное условие 
оптимальности; оптимальное управ\-ле\-ние при этом~--- точ\-ка минимума 
со\-от\-вет\-ст\-ву\-юще\-го сла\-га\-емого.
     
\section{Динамическое программирование и~оптимальное 
управление}

     В рассматриваемой постановке линейность\linebreak выхода и~квадратичность 
критерия дают те же преимущества, что и~в~классической  
ли\-ней\-но-квад\-ра\-тич\-ной задаче управ\-ле\-ния~\cite{1-bos}, а~именно: 
позволяют сразу определить вид оптимального управ\-ле\-ния и~фактические 
условия его существования. Действительно, со\-хра\-няя в~(\ref{e10-bos}) под 
знаком $\min\nolimits_u$ только члены, зависящие от~$u$, получаем
     \begin{multline*}
     \fr{\partial V_t(y,z)}{\partial t} +\fr{1}{2}\left( \Sigma_t^2(y) \fr{\partial^2 
V_t(y,z)} {\partial y^2}+\sigma_t^2\fr{\partial^2 V_t(y,z)} {\partial 
z^2}\right)+{}\\
     {}+A_t(y)\fr{\partial V_t(y,z)}{\partial y}+\left( a_t y+b_t z\right) 
\fr{\partial V_t(y,z)}{\partial z}+{}\\
{}+S_t\left( s_t y-g_t z\right)^2 +G_t z^2+{}
\end{multline*}

\noindent
\begin{multline*}
     {}+\min\limits_u \left[ \left( c_t \fr{\partial V_t(y,z)}{\partial z}-2S_t \left( 
s_t y-g_t z\right) h_t\right)u +{}\right.\\
\left.{}+\left( S_t h_t^2+H_t\right) u^2
\vphantom{\fr{\partial V_t(y,z)}{\partial z}}
\right]=0\,,
     %\label{e11-bos}
     \end{multline*}
откуда в~предположении $S_t h_t^2\hm+ H_t\hm>0$ следует, что существует 
оптимальное управ\-ле\-ние, которое определяется равенством
\begin{multline}
u_t^* = u_t^*(y,z)=-\fr{1}{2}\left( S_t h_t^2 +H_t\right)^{-1} \left( c_t 
\fr{\partial V_t(y,z)}{\partial z}-{}\right.\\
\left.{}-2S_t\left( s_t y-g_t z\right) h_t
\vphantom{\fr{\partial V_t(y,z)}{\partial z}}
\right)
\label{e12-bos}
\end{multline}
и доставляет минимум соответствующему сла\-га\-емо\-му в~урав\-не\-нии Беллмана, 
равный
$-\left( S_t h_t^2\hm+\right.$\linebreak
$\left.{}+H_t\right)^{-1} \left( c_t 
{\partial V_t(y,z)}/{\partial 
z}\hm-2S_t\left( s_t y \hm-g_t z\right) h_t \right)^2/4.
$ 
     
     Отметим, что, как и~в~классической ли\-ней\-но-квад\-ра\-тич\-ной 
задаче, управ\-ле\-ние из класса до\-пус\-ти\-мых не\-упреж\-да\-ющих получилось 
управ\-ле\-ни\-ем с~обратной связью.
     
     Таким образом, функция Беллмана описывается сле\-ду\-ющим 
дифференциальным уравнением:
     \begin{multline}
     \fr{\partial V_t(y,z)}{\partial t} +\fr{1}{2}\left( \Sigma_t^2(y) \fr{\partial^2 
V_t(y,z)} {\partial y^2}+\sigma_t^2\fr{\partial^2 V_t(y,z)} {\partial 
z^2}\right)+{}\\
     {}+ A_t(y) \fr{\partial V_t(y,z)}{\partial y}+\left( a_t y+b_t z\right) 
\fr{\partial V_t(y,z)}{\partial z}+{}\\
{}+ S_t \left( s_t y- g_t z\right)^2 +G_t z^2-
 \fr{1}{4}\left( S_t h_t^2+H_t\right)^{-1}\times{}\\
 {}\times \left( c_t \fr{\partial V_t(y,z)} 
{\partial z}-2S_t\left( s_t y -g_t z\right) h_t \right)^2=0\,.
     \label{e13-bos}
     \end{multline}
     
     Возводя в~квадрат по\-след\-нее сла\-га\-емое в~(\ref{e13-bos}), перепишем 
его в~виде:
     \begin{multline}
     \fr{\partial V_t(y,z)}{\partial t} +\fr{1}{2}\left( \Sigma_t^2(y) \fr{\partial^2 
V_t(y,z)} {\partial y^2}+\sigma_t^2\fr{\partial^2 V_t(y,z)} {\partial 
z^2}\!\right)+{}\\
{}+A_t(y) \fr{\partial V_t(y,z)}{\partial y}
+ \left( 
\vphantom{\left( S_t h_t^2 +H_t\right)^{-1}}
a_t y+b_t z+{}\right.\\
\left.{}+\left( S_t h_t^2 +H_t\right)^{-1}
 c_t S_t \left( s_t y-g_t z\right) h_t
\right) 
     \fr{\partial V_t(y,z)}{\partial z}+{}\\
     {}+\left( S_t-\left( S_t h_t^2 +H_t\right)^{-1} S_t^2 h_t^2\right)\left( s_t y -
g_t z\right)^2+{}\\
     \!\!{}+
     G_t z^2 -\fr{1}{4}\left( S_t h_t^2+H_t\right)^{-1}\! c_t^2
     \left(\fr{\partial V_t(y,z)}{\partial z}\right)^{\!2}=0\,.\!\!
     \label{e14-bos}
     \end{multline}
     
     Рассматривая полученное уравнение, заметим, что его решение может 
быть пред\-став\-ле\-но в~виде:
   \begin{equation}
     V_t(y,z)= \alpha_t z^2+\beta_t(y) z +\gamma_t(y)\,,
     \label{e15-bos}
     \end{equation}
т.\,е.\ будем искать решение при дополнительном предположении 
о~квад\-ра\-тич\-ности функции Белл\-ма\-на по переменной~$z$, и~сведем, таким 
образом, поиск оптимального решения к~уравнениям относительно функций 
$\alpha_t$, $\beta_t(y)$ и~$\gamma_t(y)$. Отметим сразу, что явный вид 
функции~$\gamma_t(y)$ для реализации оптимального управ\-ле\-ния не 
требуется, однако далее будет предложен вариант вы\-чис\-ле\-ния и~этой 
функции, что пред\-став\-ля\-ет\-ся небесполезным, поскольку позволит выполнять 
расчет минимума целевого функционала. Источником для 
предложения~(\ref{e15-bos}) является уже упоминавшаяся аналогичная 
задача для случая дис\-крет\-но\-го времени~\cite{7-bos, 8-bos}. В~той задаче 
выражение для функции Беллмана получается формально без 
дополнительных усилий. При этом форма~(\ref{e15-bos}) обнаруживается 
как свойство оптимального решения. В~рассматриваемом случае 
непрерывного времени~(\ref{e15-bos}) постулируется, а~пра\-виль\-ность 
постулата под\-тверж\-да\-ет\-ся далее ре\-зуль\-ти\-ру\-ющи\-ми уравнениями 
для~$\alpha_t$, $\beta_t(y)$ и~$\gamma_t(y)$ Кроме того, данное 
предположение пред\-став\-ля\-ет\-ся вы\-те\-ка\-ющим из линейной структуры задачи 
в~отношении переменной~$z$, в~част\-ности, тем фактом, что такой вид 
функции Беллмана обеспечивает линейность оптимального 
управ\-ле\-ния~(\ref{e12-bos}) по~$z$.

     Граничное условие при выбранном предположении~(\ref{e15-bos}) 
принимает вид:

\noindent
     \begin{multline*}
     V_T(y,z)= S_T\left( s_T y- g_T z\right)^2+G_T z^2 ={}\\[-0.5pt]
     {}=\alpha_T z^2 
+\beta_T(y) z +\gamma_T(y)\,,
    \end{multline*}
т.\,е.

\noindent
\begin{align*}
\alpha_T&= S_T g_T^2 +G_T\,;\\[-0.5pt]
\beta_T(y)&=-2S_T s_T g_T y\,;\\[-0.5pt]
\gamma_T(y)&=S_T s_T^2 y^2\,.
%\label{e16-bos}
\end{align*}
          При этом само оптимальное управ\-ле\-ние, определенное 
выражением~(\ref{e12-bos}), оказывается управ\-ле\-ни\-ем с~обратной связью 
по~$y_t$ и~$z_t$:

\noindent
     \begin{multline}
     u_t^*=u_t^*(y,z) ={}\\[-0.5pt]
     {}=
     -\fr{1}{2}\left( S_t h_t^2 +H_t\right)^{-1}
     \left( c_t \left( 2\alpha_t z +\beta_t(y)\right) +{}\right.\\[-0.5pt]
    \left. {}+2S_t\left( s_t y-g_t z\right) 
h_t\right)\,.
     \label{e17-bos}
     \end{multline}
          Подставляем $V_t(y,z)\hm= \alpha_t z^2 \hm+ \beta_t(y) 
z\hm+\gamma_t(y)$ в~(\ref{e14-bos}):

\noindent
     \begin{multline*}
     \fr{\partial \alpha_t}{\partial t}\, z^2 +
     \fr{\partial \beta_t(y)}{\partial t}\,z +
     \fr{\partial \gamma_t(y)}{\partial t}+{}\\[-0.5pt]
     {}+\fr{1}{2}\left( \Sigma_t^2(y) \left( 
\fr{\partial^2\beta_t(y)}{\partial y^2}\,z +\fr{\partial^2 \gamma_t(y)}{\partial 
y^2}\right) +2\sigma_t^2\alpha_t\right)+{}\\[-0.5pt]
 {}+A_t(y)\left(\fr{\partial \beta_t(y)}{\partial y}\,z + \fr{\partial 
\gamma_t(y)}{\partial y}\right) +{}\\[-0.5pt]
\hspace*{-0.22987pt}{}+\left( a_t y+b_t z+\left( S_t h_t^2 +H_t\right)^{-1} c_t S_t \left( s_t y-
g_t z\right) h_t\right)\times{}
\end{multline*}

\noindent
\begin{multline*}
         {}\times \left( 2\alpha_t z+\beta_t(y)\right)+{}\\
     {}+\left( S_t-\left( S_t h_t^2 +H_t\right)^{-1} S_t^2 h_t^2\right)\left( s_t y-
g_t z\right)^2+{}\\
     {}+ G_t z^2 -\fr{1}{4}\left( S_t h_t^2 +H_t\right)^{-1} c_t^2 \left( 
2\alpha_t z+\beta_t(y)\right)^2=0\,.
     \end{multline*}
          Далее выделяем слагаемые при~$z^2$, $z$ и~$z^0$
          
          \noindent
     \begin{multline*}
     \fr{\partial \alpha_t}{\partial t}\, z^2 +\fr{\partial \beta_t(y)}{\partial t}\,z +
     \fr{\partial \gamma_t(y)}{\partial 
t}+\fr{1}{2}\,\Sigma_t^2(y)\fr{\partial^2\beta_t(y)}{\partial y^2}\,z+ {}\\
{}+
\fr{1}{2}\,\Sigma_t^2(y)\fr{\partial^2\gamma_t(y)}{\partial 
y^2}+\sigma_t^2\alpha_t+A_t(y)\fr{\partial \beta_t(y)}{\partial y}\,z +{}\\
{}+A_t(y) \fr{\partial 
\gamma_t(y)}{\partial y}+{}\\
{}+ 2\alpha_t \left( b_t -\left( S_t h_t^2+H_t\right)^{-1} c_t 
S_t h_t g_t \right)z^2+{}\\
     {}+
     \left( 2\alpha_t\left( \alpha_t+\left( S_t h_t^2+H_t\right)^{-1} c_t S_t h_t 
s_t\right)y +{}\right.\\
\left.{}+\beta_t(y) \left( b_t-\left( S_t h_t^2+H_t\right)^{-1} c_t S_t h_t 
g_t\right) \right) z+{}\\
     {}+\beta_t(y)\left( a_t +\left( S_t h_t^2+H_t\right)^{-1} c_t S_t h_t s_t\right) 
y+{}\\
{}+ \left( S_t -\left( S_t h_t^2+H_t\right)^{-1} S_t^2 h_t^2\right) g_t^2 z^2-{}\\
     {}- 2\left( S_t -\left( S_t h_t^2+H_t\right)^{-1} S_t^2 h_t^2\right) s_t g_t yz 
+{}\\
{}+
     \left( S_t-\left( S_t h_t^2+H_t\right)^{-1} S_t^2 h_t^2\right) s_t^2 y^2+{}\\
     {}+G_t z^2 -\left( S_t h_t^2 +H_t\right)^{-1} c_t^2 \alpha_t^2 z^2 -{}\\
     {}-\left( 
S_t h_t^2+H_t\right)^{-1} c_t^2 \alpha_t \beta_t(y) z-{}\\
{}-
\fr{1}{4}\left( S_t h_t^2+H_t\right)^{-1}  c_t^2 \beta_t^2(y)=0\,,
     \end{multline*}
группируем их и~получаем сле\-ду\-ющие уравнения:
\begin{itemize}
\item  для~$\alpha_t$:

\noindent
\begin{multline}
\fr{\partial\alpha_t}{\partial t}+2\alpha_t\left( b_t-\left( S_t h_t^2+H_t\right)^{-1} c_t 
S_t h_t g_t\right)+{}\\
{}+ \left( S_t- \left( S_t h_t^2+H_t\right)^{-1} S_t^2 h_t^2\right) 
g_t^2+G_t-{}\\
\hspace*{-8mm}{}-\left( S_t h_t^2+H_t\right)^{-1} c_t^2 \alpha_t^2 =0\,,\enskip \alpha_T=S_T 
g_t^2+G_T\,;\!\!
\label{e18-bos}
\end{multline}
\item для $\beta_t$:

\noindent
\begin{multline}
\fr{\partial\beta_t(y)}{\partial 
t}+\fr{1}{2}\,\Sigma_t^2(y)\fr{\partial^2\beta_t(y)}{\partial y^2} 
+A_t(y)\fr{\partial \beta_t(y)}{\partial y}+{}\\
{}+ 2\alpha_t\left( a_t +\left( S_t h_t^2+H_t\right)^{-1} c_t S_t h_t s_t\right) y+{}\\
{}+
\beta_t(y)\left( b_t -\left( S_t h_t^2 +H_t\right)^{-1} c_t S_t h_t g_t\right)-{}\\
{}-2\left( S_t-\left( S_t h_t^2+H_t\right)^{-1} S_t^2 h_t^2\right) s_t g_t y-{}
\\
{}-
\left( S_t h_t^2+H_t\right)^{-1} c_t^2 \alpha_t \beta_t(y)=0\,,\\
\beta_T(y)=-2S_T s_T g_T y\,;
\label{e19-bos}
\end{multline}
\item  для $\gamma_t$:
\begin{multline}
\hspace*{-0.8pt}\fr{\partial \gamma_t(y)}{\partial t}+\fr{1}{2}\,\Sigma_t^2(y)
\fr{\partial^2 \gamma_t(y)}{\partial y^2} +\sigma_t^2 \alpha_t +A_t(y)
\fr{\partial \gamma_t(y)}{\partial y}+{}\\
{}+ \beta_t(y)\left( a_t +\left( S_t h_t^2+H_t\right)^{-1} c_t S_t h_t s_t\right) y+{}\\
{}+
\left( S_t-\left( S_t h_t^2+H_t\right)^{-1} S_t^2 h_t^2\right)  s_t^2 y^2-{}\\
{}-\fr{1}{4}\left( S_t h_t^2+H_t\right)^{-1} c_t^2 \beta_t^2(y) =0\,,\\
\gamma_T(y)=S_T s_T^2 y^2\,.
\label{e20-bos}
\end{multline}
\end{itemize}
     
     Уравнение~(\ref{e18-bos}), легко заметить, является уравнением 
Риккати, которое в~силу сформулированного выше условия   
имеет единственное неотрицательное решение для всех $0\hm\leq t\hm\leq T$. 
Этот факт требует дополнительного комментария. Для получения 
уравнения~(\ref{e18-bos}) рас\-смот\-рим исходную задачу при дополнительных 
условиях $a_t\hm=0$ и~$s_t\hm=0$ для всех $0\hm\leq t\hm\leq T$. Нетрудно 
видеть, что эти условия рассматриваемую по\-ста\-нов\-ку сводят фактически 
к~классической ли\-ней\-но-квад\-ра\-тич\-ной задаче. Имеющуюся 
в~рассматриваемой формулировке чуть более общую форму целевой 
функции (принципиального значения это обобщение, конечно, не имеет) 
сведем к~классической еще одним предположением: $S_t\hm=0$ для всех 
$0\hm\leq t\hm\leq T$. Теперь уравнение для~$\alpha_t$ принимает хорошо 
известный вид:
     \begin{equation}
     \fr{\partial \alpha_t}{\partial t}+2\alpha_t b_t +G_t- H_t^{-1} c_t^2 
\alpha_t^2=0\,,\enskip \alpha_T=G_T\,.
     \label{e21-bos}
     \end{equation}

     В таком случае, как известно~\cite{10-bos}, существует единственное 
оптимальное управление~--- линейное с~обратной связью по выходу~$z_t$, 
с~коэффициентом усиления, опи\-сы\-ва\-емым уравнением  
Риккати~(\ref{e21-bos}). Именно этот результат дают  
уравнения~(\ref{e18-bos})--(\ref{e20-bos}) и~описываемая ими функция 
Беллмана~(\ref{e15-bos}), так как из $a_t\hm=0$ и~$s_t\hm=0$ немедленно 
следует, что $\beta_t(y)\hm=0$, откуда, в~свою очередь, с~учетом 
не\-за\-ви\-си\-мости решения от~$y_t$ следует, что $\gamma_t(y)\hm=\gamma_t$, 
т.\,е.\ не зависит от~$y$ и~задается уравнением: 
     $$
     \fr{\partial \gamma_t(y)}{\partial t} +\sigma^2_t \alpha_t=0\,,\enskip 
\gamma_T=0\,.
     $$ 
     Оптимальное управ\-ле\-ние при этом 
     $$
     u_t^*= -H_t^{-1} c_t \alpha_t z_t\,,
     $$
      т.\,е.\ все полностью совпадает с~известным классическим решением.
     
     С уравнениями~(\ref{e19-bos}) и~(\ref{e20-bos}) ситуация, естественно, 
обстоит сложнее. Это линейные уравнения второго порядка параболического 
типа, поскольку\linebreak
 $\Sigma_t^2(y)\hm>0$. Фактически отсутствуют 
конструктивные условия, гарантирующие существование их\linebreak
 решений 
(требовать, чтобы все фигурирующие в~уравнениях коэффициенты были 
представлены аналитическими функциями на всем пространстве значений, 
вряд ли целесообразно), поэтому далее будем предполагать, что данные 
уравнения имеют на рас\-смат\-ри\-ва\-емом интервале $0\hm\leq t\hm\leq T$ хотя 
бы одно ограниченное решение и~именно эти условия будем рас\-смат\-ри\-вать 
как достаточные условия существования оптимального решения 
рассматриваемой задачи.
     
     Таким образом, доказана следующая тео\-рема.
     
     \smallskip
     
     \noindent
     \textbf{Теорема.}\ \textit{Пусть для диффузионного 
процесса}~(\ref{e5-bos}) \textit{выполнены условия Ито, для 
     процесса}~(\ref{e6-bos})~--- \textit{ограничены коэффициенты, 
уравнения}~(\ref{e18-bos})--(\ref{e20-bos}) \textit{имеют ограниченные 
решения для $0\hm\leq t\hm\leq T$. Тогда минимум  
функционалу}~(\ref{e7-bos}) \textit{доставляет оптимальное 
управ\-ле\-ние}~(\ref{e17-bos}), \textit{где} $y\hm= y_t$; $z\hm=z_t$.
     
\section{Заключение}

     Рассмотренная задача оптимизации в~целом близка и~по модели, и~по 
критерию к~классической ли\-ней\-но-квад\-ра\-тич\-ной постановке. 
Принципиальным отличием является нелинейная модель для описания 
со\-сто\-яния динамической сис\-те\-мы, в~которой отсутствует управ\-ля\-ющее 
воздействие.\linebreak
 Такую модель наряду с~традиционной интер\-пре\-тацией  
<<со\-сто\-яние--вы\-ход>> мож\-но понимать как\linebreak модель неконтролируемого 
слож\-но\-го внешнего воздействия. Небольшое дополнительное отличие дает 
предложенная форма квад\-ра\-тич\-но\-го критерия, поз\-во\-ля\-ющая, в~част\-ности, 
ставить такие задачи, как отслеживание выходом или управ\-ле\-ни\-ем со\-сто\-яния 
сис\-те\-мы или ее выхода.
     
     Поскольку обсуждать возможности точного решения уравнений, 
определяющих оптимальное управ\-ле\-ние, не имеет смыс\-ла, наиболее 
актуальной далее является задача их приближенного чис\-лен\-но\-го решения 
и~анализа воз\-мож\-ности практической реализации. Этому посвящена вторая 
часть данной работы, пла\-ни\-ру\-емая к~выходу в~ближайшее время.

{\small\frenchspacing
 {%\baselineskip=10.8pt
 \addcontentsline{toc}{section}{References}
 \begin{thebibliography}{99}
\bibitem{1-bos}
\Au{Athans M.} Editorial on the LQG problem~// IEEE~T. Automat. Contr., 1971. Vol.~16. 
No.\,6. P.~528--552. doi: 10.1109/TAC.1971.1099845.
\bibitem{2-bos}
\Au{Wu Z.} Forward-backward stochastic differential equations, linear quadratic stochastic 
optimal control and nonzero sum differential games~// J.~Syst. Sci. Complex., 2005. Vol.~18. 
No.\,2. P.~179--192.
\bibitem{3-bos}
\Au{Chen B.\,S., Zhang~W.} Stochastic H2/H1 control with state-dependent noise~// IEEE 
T.~Automat. Contr., 2004. Vol.~49. No.\,1. P.~45--56. doi: 10.1109/TAC.2003.821400.
\bibitem{4-bos}
\Au{Bohacek S.} A~stochastic model of TCP and fair video transmission~// IEEE 
INFOCOM, 2003. Vol.~2. P.~1134--1144. doi: 10.1109/INFCOM.2003.1208950.
\bibitem{5-bos}
\Au{Домбровский В.\,В., Объедко~Т.\,Ю.} Управление с~прогнозированием системами 
с~марковскими скачками при ограничениях и~применение к~оптимизации 
инвестиционного портфеля~// Автомат. телемех., 2011. №\,5. С.~96--112. doi: 
10.1134/S0005117911050079.
\bibitem{6-bos}
\Au{Баландин Д.\,В., Коган~М.\,М.} Оптимальное линейно-квад\-ра\-тич\-ное управление: от 
матричных уравнений к~линейным матричным неравенствам~// Автомат. телемех., 2011. 
№\,11. С.~60--69. doi: 10.1134/ S0005117911110038.
\bibitem{7-bos}
\Au{Босов А.\,В.} Обобщенная задача распределения ресурсов программной системы~// 
Информатика и~её применения, 2014. Т.~8. Вып.~2. С.~39--47. doi: 
10.14357/19922264140204.
\bibitem{8-bos}
\Au{Босов А.\,В.} Управление линейным выходом дискретной стохастической системы по 
квадратичному критерию~// Изв. РАН. Теория и~системы управления, 2016. №\,3.  
С.~19--35. doi: 10.1134/S1064230716030060.
\bibitem{9-bos}
\Au{Флеминг У., Ришел~Р.} Оптимальное управление детерминированными 
и~стохастическими системами~/ Пер. с~англ.~--- М.: Мир, 1978. 316~с. 
(\Au{Fleming~W.\,H., Rishel~R.\,W.} Deterministic and stochastic optimal control.~--- New 
York, NY, USA: Springer-Verlag, 1975. 222~p.)
\bibitem{10-bos}
\Au{Девис М.\,Х.\,А.} Линейное оценивание и~стохастическое управление~/ Пер. с~англ.~--- 
М.: Наука, 1984. 206~с. (\Au{Davis~M.\,H.\,A.} Linear estimation and stochastic control.~--- 
London: Chapman and Hall, 1977. 224~p.)

 \end{thebibliography}

 }
 }

\end{multicols}

\vspace*{-6pt}

\hfill{\small\textit{Поступила в~редакцию 30.03.18}}

\vspace*{4pt}

%\newpage

%\vspace*{-24pt}

\hrule

\vspace*{2pt}

\hrule

\vspace*{-2pt}


\def\tit{STOCHASTIC DIFFERENTIAL SYSTEM OUTPUT CONTROL 
BY~THE~QUADRATIC CRITERION.~I.~DYNAMIC\\ PROGRAMMING 
OPTIMAL SOLUTION}


\def\titkol{Stochastic differential system output control 
by~the~quadratic criterion. I.~Dynamic programming 
optimal solution}

\def\aut{A.\,V.~Bosov and~A.\,I.~Stefanovich}

\def\autkol{A.\,V.~Bosov and~A.\,I.~Stefanovich}

\titel{\tit}{\aut}{\autkol}{\titkol}

\vspace*{-11pt}


\noindent
Institute of Informatics Problems, Federal Research Center ``Computer Science 
and Control'' of the Russian Academy of Sciences, 44-2~Vavilov Str., Moscow 
119333, Russian Federation


\def\leftfootline{\small{\textbf{\thepage}
\hfill INFORMATIKA I EE PRIMENENIYA~--- INFORMATICS AND
APPLICATIONS\ \ \ 2018\ \ \ volume~12\ \ \ issue\ 3}
}%
 \def\rightfootline{\small{INFORMATIKA I EE PRIMENENIYA~---
INFORMATICS AND APPLICATIONS\ \ \ 2018\ \ \ volume~12\ \ \ issue\ 3
\hfill \textbf{\thepage}}}

\vspace*{3pt}



\Abste{The problem of optimal control for the Ito diffusion 
process and a~controlled linear output is solved. The considered 
statement is close to the classical linear-quadratic Gaussian 
control  (LQG control) problem. Differences consist in the fact 
that the state is described by the nonlinear differential Ito equation  $dy_y = A_t(y_t) 
\,dt+\Sigma_t(y_t)\,dv_t$ and does not depend on the control~$u_t$, 
optimization subject is controlled linear output 
 $dz_t=a_ty_t\,dt +b_tz_t\,dt +c_t u_t\,dt +\sigma_t \,dw_t$. 
Additional generalizations are included in the quadratic 
quality criterion for the purpose of statement such problems 
as state tracking by output or a linear combination of state 
and output tracking by control. The method of dynamic programming 
is used for the solution. 
The assumption about Bellman function in the form  $V_t(y,z)= \alpha_t 
z^2+\beta_t(y) z+\gamma_t(y)$ allows one to find it. 
Three differential equations for the coefficients $\alpha_t$,  $\beta_t(y)$,
and $\gamma_t(y)$ give the solution. 
These equations constitute the optimal solution of the problem under consideration.}

\KWE{stochastic differential equation; optimal control; dynamic programming; 
Bellman function; Riccati equation; linear differential equations of parabolic type}


\DOI{10.14357/19922264180314}

\vspace*{-12pt}

\Ack
\noindent
This work was partially supported by the Russian Science Foundation (grant  
16-07-00677).



%\vspace*{6pt}

  \begin{multicols}{2}

\renewcommand{\bibname}{\protect\rmfamily References}
%\renewcommand{\bibname}{\large\protect\rm References}

{\small\frenchspacing
 {%\baselineskip=10.8pt
 \addcontentsline{toc}{section}{References}
 \begin{thebibliography}{99}
\bibitem{1-bos-1}
\Aue{Athans, M.} 1971. Editorial on the LQG problem. \textit{IEEE~T. 
Automat. Contr.} 16(6):528--552. doi: 10.1109/ TAC.1971.1099845.
\bibitem{2-bos-1}
\Aue{Wu, Z.} 2005. Forward-backward stochastic differential equations, linear 
quadratic stochastic optimal control and\linebreak\vspace*{-12pt}

\columnbreak

\noindent
 nonzero sum differential games. 
\textit{J.~Syst. Sci. Complex.} 18(2):179--192.
\bibitem{3-bos-1}
\Aue{Chen, B.\,S. and W.~Zhang.} 2004. Stochastic H2/H1 control with  
state-dependent noise. \textit{IEEE~T. Automat. Contr.} 49(1):45--56.
doi: 10.1109/TAC.2003.821400.
\bibitem{4-bos-1}
\Aue{Bohacek, S.} 2003. A~stochastic model of TCP and fair video 
transmission. \textit{IEEE INFOCOM}. 2:1134--1144.
doi: 10.1109/INFCOM.2003.1208950.
\bibitem{5-bos-1}
\Aue{Dombrovskii, V.\,V., and T.\,Yu.~Ob''edko.} 2011. Predictive control of 
systems with Markovian jumps under constraints and its application to the 
investment portfolio optimization. \textit{Automat. Rem. Contr.}  
72(5):989--1003.
\bibitem{6-bos-1}
\Aue{Balandin, D.\,V., and M.\,M.~Kogan.} 2011. Optimal linear-quadratic 
control: From matrix equations to linear matrix inequalities. \textit{Automat. 
Rem. Contr.} 72(11):2276--2284.
\bibitem{7-bos-1}
\Aue{Bosov, A.\,V.} 2014. Obobshchennaya zadacha raspredeleniya resursov 
programmnoy sistemy [The generalized problem of software system resources 
distribution]. \textit{Informatika i~ee Primeneniya~--- Inform. Appl.}  
8(2):39--47. doi: 
10.14357/19922264140204.
\bibitem{8-bos-1}
\Aue{Bosov, A.\,V.} 2016. Discrete stochastic system linear output control 
with respect to a quadratic criterion. \textit{J.~Comput. Syst. Sc. 
Int.} 55(3):349--364.
\bibitem{9-bos-1}
\Aue{Fleming, W.\,H., and R.\,W.~Rishel.} 1975. \textit{Deterministic and 
stochastic optimal control.} New York, NY: Springer-Verlag. 222~p.
\bibitem{10-bos-1}
\Aue{Davis, M.\,H.\,A.} 1977. \textit{Linear estimation and stochastic 
control.} London: Chapman and Hall. 224~p.
\end{thebibliography}

 }
 }

\end{multicols}

\vspace*{-6pt}

\hfill{\small\textit{Received March 30, 2018}}

%\pagebreak

%\vspace*{-18pt}
     
     \Contr
     
       \noindent
       \textbf{Bosov Alexey V.} (b.\ 1969)~--- Doctor of Science in technology, 
principal scientist, Institute of Informatics Problems, Federal Research 
Center ``Computer Science and Control'' of the Russian Academy of Sciences, 
44-2~Vavilov Str., Moscow 119333, Russian Federation; 
\mbox{AVBosov@ipiran.ru}
       
       \vspace*{3pt}
       
       \noindent
       \textbf{Stefanovich Alexey I.} (b.\ 1983)~--- principal specialist, 
Institute of Informatics Problems, Federal Research Center ``Computer Science 
and Control'' of the Russian Academy of Sciences, 44-2~Vavilov Str., Moscow 
119333, Russian Federation; \mbox{AStefanovich@frccsc.ru}
\label{end\stat}

\renewcommand{\bibname}{\protect\rm Литература}       

          %7
\def\stat{yanushko}

\def\tit{РЕШЕНИЕ ПРОБЛЕМ ВЗАИМОДЕЙСТВИЯ С~СУБД 
В~КРОССПЛАТФОРМЕННОЙ БИБЛИОТЕКЕ EFFIDB}

\def\titkol{Решение проблем взаимодействия с~СУБД 
в~кроссплатформенной библиотеке EFFIDB}

\def\autkol{А.\,В.~Янушко, А.\,В.~Бабанин, О.\,А.~Кузнецова, 
С.\,В.~Петрушенко}
\def\aut{А.\,В.~Янушко$^1$, А.\,В.~Бабанин$^2$, О.\,А.~Кузнецова$^3$, 
С.\,В.~Петрушенко$^4$}

\titel{\tit}{\aut}{\autkol}{\titkol}

%{\renewcommand{\thefootnote}{\fnsymbol{footnote}}\footnotetext[1]
%{Работа поддержана Российским фондом фундаментальных исследований
%(проекты 11-01-00515а и 11-07-00112а), а также Министерством
%образования и науки РФ в рамках ФЦП <<Научные и
%научно-педагогические кадры инновационной России на 2009--2013~годы>>.}}


\renewcommand{\thefootnote}{\arabic{footnote}}
\footnotetext[1]{АСофт, yan@asoft.ru}
\footnotetext[2]{Всероссийский научно-исследовательский институт проблем вычислительной техники и информатизации, 
ababanin@pvti.ru}
\footnotetext[3]{АСофт, ok@asoft.ru}
\footnotetext[4]{АСофт, op@asoft.ru}

\vspace*{6pt}

\Abst{Статья затрагивает проблемы унифицированного взаимодействия с разнотипными СУБД 
в различных программных средах. Рассмотрены существующие разработки в данной области, 
проанализированы их достоинства  и недостатки. Перечислены требования к кроссплатформенному 
инструменту взаимодействия прикладного кода С++ с СУБД. Предложено решение, 
реализованное в виде динамической библиотеки. Библиотека предоставляет специализированные 
классы для каждого из понятий реляционных баз данных: собственно база данных, соединение, 
таблица, средства манипулирования данными и~т.\,д. Проанализированы границы применимости 
предложенных решений и практика использования библиотеки в реальных проектах. 
Приведены примеры кода.}

\vspace*{2pt}

\KW{СУБД; C++; библиотека взаимодействия; кроссплатформенность}

\vspace*{6pt}

 \vskip 14pt plus 9pt minus 6pt

      \thispagestyle{headings}

      \begin{multicols}{2}
      
            \label{st\stat}

\section{Введение}

Современные информационные системы автоматизации деятельности предприятий 
пред\-став\-ляют собой многокомпонентные программные комплексы. С~прикладной точки 
зрения в любой информаци\-онной системе можно выделить три логических компоненты: 
слой отображения данных, слой обработки данных и слой хранения данных. 
{\looseness=1

}

Данная 
работа посвящена фундаментальной со\-став\-ля\-ющей информационной системы, а именно 
сис\-те\-мам управления базами данных и средствам взаимодействия серверов приложений с 
различными СУБД.

Несмотря на то что базы данных и СУБД со\-ставля\-ют обширную тему исследований и 
разработок, в этой области остается много нерешенных проб\-лем.

В статье рассматривается проблема организации работы приложений, написанных на C++, 
с множеством разнотипных СУБД. Предлагаемое решение заключается в создании 
промежуточного звена доступа к различным СУБД, при этом разработчик не должен 
задумываться о наличии такого промежуточного слоя и желательно, чтобы 
взаимодействие со средством доступа происходило стандартным образом, с 
использованием распространенных технологий.

\section{Универсальная библиотека взаимодействия с~СУБД~--- EffiDB}

\vspace*{-6pt}

C++ библиотека EffiDB представляет собой набор интерфейсов, позволяющий 
взаимодействовать единым образом с системами управления базами данных. Целью 
создания библиотеки ставилось предоставление унифицированного доступа к различным 
СУБД и изоляция прикладного программиста от какого бы то ни было непосредственного 
низкоуровневого взаимодействия с базой данных. 

Библиотека EffiDB доступна для использования в среде Linux (компилятор gcc версий~4.2--4.5), 
Windows (компилятор MS VC++ 2010), MacOS (компилятор gcc~4.2.1 и выше). Авторам 
неизвестно никаких препятствий для сборки библиотеки иными компиляторами, 
поддерживающими стандарт C++ ISO/IEC~14882 1998~г., а также препятствий для 
использования библиотеки под управлением иных POSIX-со\-вмес\-ти\-мых ОС. Однако 
никаких работ в этом направлении авторами не проводилось.

Специализация работы с каждой конкретной СУБД заключена в отдельных библиотеках, 
которые далее называются драйверами. Библиотека\linebreak 
\mbox{EffiDB} управ\-ля\-ет соединениями с базой данных,\linebreak 
формирует и исполняет SQL-за\-про\-сы, интерпретирует полученные результаты. Она 
также пред\-остав-\linebreak\vspace*{-12pt}
\pagebreak

\noindent
ля\-ет механизмы кэширования результатов запросов и автоматического 
менеджмента памяти, что повышает надежность и производительность кода.

В настоящий момент доступны драйверы для работы со следующими СУБД: MySQL, 
MSSQL, Oracle и SQLite.

В случае когда отдельные модули системы выполняют кооперированные действия с 
одним и тем же объектом базы данных, требуется, чтобы все операции совершались в 
рамках одной транзакции. Специализированный драйвер Invariant инициирует отдельный 
процесс исключительно для коммуникации с базой данных, что позволяет объединить все 
обращения к базе данных в одном соединении. 
Драйвер Invariant не имеет специфики СУБД и 
может быть использован с любой из поддерживаемых СУБД.

В зависимости от поставленных задач и степени подготовки разработчика можно 
выделить 4 уровня использования EffiDB:
\begin{enumerate}
\item \textbf{Создание программных приложений со статической структурой базы 
данных.} Простейший вариант использования библиотеки в приложениях с заранее 
определенной схемой базы\linebreak данных, которая не изменяется во время работы 
приложения.
\item \textbf{Создание программных приложений с динамической структурой 
базы данных.} Предусматривается возможность изменения или расширения схемы 
базы данных в процессе эксплуатации приложения.
\item \textbf{Разработка адаптеров для типов данных.} Биб\-ли\-о\-тека EffiDB не использует 
никаких <<внут\-рен\-них>> типов данных для хранения парамет-\linebreak ров запросов и 
результатов выборок из базы данных. Вмес\-то этого используется подход 
<<адаптеров>>, которые позволяют применять в \mbox{EffiDB} любые пользовательские типы 
данных. Поставка \mbox{EffiDB} включает в себя адаптеры для популярных (в том числе 
встроенных в C++) типов данных, таких как int, string, double и~т.\,п. Механизм 
адаптации любого другого типа данных в точности такой же. Работа с 
пользовательским типом данных, для которого разработан адаптер, не отличается от 
работы с типами, адаптеры для которых включены в дистрибутив библиотеки.
\item \textbf{Разработка драйверов для поддержки дополнительных СУБД.} 
Библиотека EffiDB не ограничивает пользователя перечисленными ранее СУБД. 
Открытый API (application programming interface)
библиотеки предоставляет техническую возможность разработать 
собственный драйвер для произвольной СУБД. Как и в случае с пользовательскими 
типами, вновь разработанный дополнительный драйвер СУБД предостав\-ляет все те же 
возможности, что и драйверы, входящие в поставку.
\end{enumerate}

Хотя это и может показаться парадоксальным на первый взгляд, дублирование 
информации приводит к ее потере. При наличии нескольких реп\-лик данных возникают 
проблемы синхронизации, определения степени актуальности каждого отдельного 
экземпляра и, как следствие, появляется риск потери информации. Достаточно 
распространена ситуация, когда необходимо обрабатывать агрегированную информацию 
из нескольких баз данных (в общем случае с использованием разных СУБД). Копирование 
данных в единое хранилище с целью их консолидации и дальнейшей обработки 
пред\-став\-ля\-ет\-ся неэффективным в силу вышеупомянутых причин.

Библиотека EffiDB позволяет одновременно обращаться к нескольким базам данных под управ\-ле\-ни\-ем 
разных СУБД, и необходимость в копировании информации не возникает. 

Возможность одновременной работы с разными СУБД оказывается полезной и в другой 
хорошо известной практической задаче~--- задаче миграции данных из одной 
информационной системы в другую. Благодаря EffiDB процедура миграции может 
выполняться напрямую без промежуточных шагов.

Библиотека EffiDB предназначена для по\-стро\-ения сложных многокомпонентных информа\-ци\-онных 
сис\-тем. В~таких сис\-те\-мах биз\-нес-ло\-ги\-ка \mbox{обычно} распределяется между несколькими 
приложениями (модулями), которые совместно работают с общей базой данных. 
Хрестоматийным примером, иллюстрирующим такую ситуацию, можно считать перевод 
денег с одного счета на другой: один модуль снимает деньги со сче\-та-до\-но\-ра и 
вызывает второй модуль, который, в свою очередь, зачисляет средства на счет-ак\-цеп\-тор.

Таким образом, одно действие (перевод средств) может быть представлено несколькими 
отдельными, но тем не менее зависимыми друг от друга операциями приложений над 
объектами одной базы данных. Реализация такого действия в информационной системе 
требует, чтобы каждый участвующий в процессе модуль последовательно совершал 
необходимую операцию и передавал управление следующему модулю до тех пор, пока 
все операции не будут выполнены. Обеспечение транзакционной целостности базы 
данных означает, что все операции (снятие и зачисление средств в вышеприведенном 
примере) должны производиться в одной транзакции. Это было бы невозможно, если бы
каждый модуль действовал изолированно и уста\-нав\-ли\-вал собственное соединение с базой 
дан-\linebreak ных.

Для устранения этого противоречия в EffiDB используется специализированный драйвер 
Invariant. Если в индивидуальных процессах приложений требуется взаимодействие с 
базой данных, драйвер создает специальный процесс для коммуникации с базой данных и 
устанавливает единое соединение с ней. В~результате логически связанные действия 
разных приложений выполняются в рамках одной транзакции.

\section{Аналогичные разработки}

В настоящий момент разработчики могут выбирать среди достаточно большого 
количества инструментов доступа к данным, хранящимся в реляционных СУБД. Можно 
выделить низкоуровневые средства доступа, наиболее популярные из которых ODBC (Open
DataBase Connectivity) (для 
языков С/С++) и JDBC (для Java), а также высокоуровневые~--- так называемые средства 
объект\-но-ре\-ля\-ци\-он\-но\-го отображения, весьма представительный список которых 
представлен в Wikipedia~\cite{6-y}.

Некоторые популярные инструменты сведены в табл.~1, где указаны поддерживаемые 
операционные системы, СУБД, языки программирования. Также в таблице содержатся 
наиболее важные с точки зрения разработчика прикладного программного
обеспечения (ПО) объективные параметры 
этих инструментов. Такие важнейшие параметры, как лаконичность и простота 
использования языка запросов, сложно сравнивать объективно, поэтому они исключены 
из рассмотрения.


\end{multicols}

\begin{table}[b]\small
\vspace*{-24pt}
\begin{center}
\Caption{Инструменты доступа к данным, хранящимся в РСУБД}
\vspace*{2ex}

\tabcolsep=3pt
\begin{tabular}{|c|c|c|c|c|c|c|}
\hline
Продукт &\tabcolsep=0pt\begin{tabular}{c}Операционная\\ система\end{tabular} &СУБД &Язык &
\tabcolsep=0pt\begin{tabular}{c}Уровень\\ изоляции\end{tabular} &
\tabcolsep=0pt\begin{tabular}{c}Контроль\\ синтаксиса\\ запросов\end{tabular} &Типы данных\\
\hline 
ODBC&\tabcolsep=0pt\begin{tabular}{c}Windows,\\ Unix\end{tabular}&
\tabcolsep=0pt\begin{tabular}{c}Все\\ промышленные\\ РСУБД\end{tabular}&С&Низкий&
\tabcolsep=0pt\begin{tabular}{c}В момент\\ исполнения\end{tabular}&
\tabcolsep=0pt\begin{tabular}{c}Простые ODBC \\ специфичные\end{tabular}\\
\hline
JDBC&\tabcolsep=0pt\begin{tabular}{c}Windows,\\ Unix\end{tabular}&
\tabcolsep=0pt\begin{tabular}{c}Все\\ промышленные\\ РСУБД\end{tabular}&Java&Низкий&
\tabcolsep=0pt\begin{tabular}{c}В момент\\ исполнения\end{tabular}&
\tabcolsep=0pt\begin{tabular}{c}Простые Java\\ специфичные\end{tabular}\\
\hline
LINQ&Windows&MS SQL$^{(*)}$&
\tabcolsep=0pt\begin{tabular}{c}C\#, VB.NET\\ и другие\\ MSIL-совместимые\\ .NET языки\end{tabular}&
\tabcolsep=0pt\begin{tabular}{c}Высокий. \\ Требует\\ перегенерации\\ модели данных\\
прикладного ПО\end{tabular}&
\tabcolsep=0pt\begin{tabular}{c}На этапе\\ разработки\end{tabular}&
\tabcolsep=0pt\begin{tabular}{c}Простые .NET \\ специфичные\end{tabular}\\
\hline
Hibernate&\tabcolsep=0pt\begin{tabular}{c}Windows,\\ Unix\end{tabular}&
\tabcolsep=0pt\begin{tabular}{c}Все\\ промышленные\\ РСУБД\end{tabular}&Java&
\tabcolsep=0pt\begin{tabular}{c}Высокий.\\ Допускает\\ переключение\\ между 
СУБД\\ <<на лету>>\end{tabular}&
\tabcolsep=0pt\begin{tabular}{c}В момент\\ исполнения\end{tabular}&
\tabcolsep=0pt\begin{tabular}{c}Простые Java\\ специфичные\end{tabular}\\
\hline
SOCI&Unix&\tabcolsep=0pt\begin{tabular}{c}Oracle,\\ PostgreSQL,\\ MySQL\end{tabular}&C++&
Низкий&\tabcolsep=0pt\begin{tabular}{c}В момент\\ исполнения\end{tabular}&
\tabcolsep=0pt\begin{tabular}{c}Стандартные\\ С++ типы\end{tabular}\\
\hline
ODB&\tabcolsep=0pt\begin{tabular}{c}Windows,\\ Unix\end{tabular}&
MySQL&C++&\tabcolsep=0pt\begin{tabular}{c}Высокий$^{(**)}$.\\ Требует\\ перекомпиляции\end{tabular}&
\tabcolsep=0pt\begin{tabular}{c}На этапе \\ разработки\end{tabular}&
\tabcolsep=0pt\begin{tabular}{c}Любые,\\ определяемые\\ пользователем\end{tabular}\\
\hline
EffiDB&\tabcolsep=0pt\begin{tabular}{c}Windows,\\ Unix\end{tabular}&
\tabcolsep=0pt\begin{tabular}{c}MSSQL,\\ Oracle,\\ MySQL,\\ 
SQLite$^{(***)}$\end{tabular}&
С++&\tabcolsep=0pt\begin{tabular}{c}Высокий.\\ Допускает \\ переключение\\ между СУБД\\ <<на лету>>\end{tabular}&
\tabcolsep=0pt\begin{tabular}{c}На этапе\\ разработки\end{tabular}&
\tabcolsep=0pt\begin{tabular}{c}Любые,\\ определяемые \\
пользователем\end{tabular}\\
\hline
\multicolumn{7}{p{164mm}}{$\hphantom{^{**}}^{(*)}$Поддержка Oracle, MySQL, PostgreSQL, SQLite возможна при покупке 
дополнительных инструментов сторонних разработчиков.\newline
$\hphantom{^*}^{(**)}$Поддерживается очень ограниченный набор возможностей SQL. В~частности, недоступно 
объединение таблиц (JOIN).\newline
$^{(***)}$Поддержка других РСУБД возможна в режиме ODBC-со\-вмес\-ти\-мости.}
\end{tabular}
\end{center}
\end{table}

\begin{multicols}{2}

Под термином <<уровень изоляции>> авторы понимают степень изоляции прикладного 
кода от специфики целевой СУБД, в том числе в разрезе поддержки того или иного 
диалекта SQL (Structured Query Language). Низкий уровень изоляции подразумевает необходимость внесения 
изменений в код прикладного ПО при смене СУБД, высокий уровень изоляции 
гарантирует переносимость кода прикладного ПО между СУБД без каких-либо 
изменений. Однако даже с высоким уровнем изоляции адаптация под новую СУБД может 
требовать перегенерации, перекомпиляции или статической перелинковки кода, а может 
осуществляться динамически во время работы программы. Контроль синтаксиса запросов 
в разных системах может осуществляться либо на этапе разработки прикладного ПО (во 
время компиляции/генерации кода) средствами самого инструмента, либо в момент 
исполнения SQL-за\-про\-са средствами СУБД. Первый подход существенно удешевляет 
разработку ПО за счет снижения затрат на тестирование (или повышает качество ПО при 
равных затратах). Набор поддерживаемых типов данных практически не влияет на выбор 
инструментов доступа при разработке прикладного ПО с нуля, однако в реальной жизни 
при наличии унаследованного кода или при желании разработчика использовать 
ка\-кие-ни\-будь дополнительные биб\-лио\-те\-ки/мо\-ду\-ли и~т.\,п.\ вопрос об используемых 
типах данных может оказаться критически важным. 

Анализ присутствующих на рынке инструментов доступа к 
реляционным СУБД (РСУБД) показывает, что 
наиболее продвинутым продуктом является LINQ (Language Integrated Query) компании 
Microsoft~--- встроенное в .NET средство доступа к данным. Полная изоляция от SQL, 
проверка синтаксиса на этапе компиляции, лаконичность и удобочитаемость синтаксиса, 
наличие средств генерации языковых объектов на основании существующей базы данных 
делают LINQ лучшим на данный момент средством доступа к реляционным данным с 
точки зрения удобства использования. Однако LINQ доступен только как часть 
платформы .NET для проектов, работающих под управлением ОС Windows, и нацелен на 
работу с СУБД MS~SQL, что существенно ограничивает область применимости этого 
инструмента и требует достаточно больших материальных затрат~--- хотя сам LINQ и 
является бесплатным, для его использования требуется как минимум покупка лицензий 
Windows и MS~SQL, а также зачастую средств разработки (Microsoft Visual Studio и~т.\,п.).

Другие инструменты, хотя и лишены различных недостатков LINQ, но предоставляют в 
целом меньшее, иногда существенно меньшее удобство при разработке прикладного ПО. 
Кроме того, большинство существующих инструментов ориентированы на 
интерпретируемые языки (PHP, Python, Ruby и~т.\,п.) или на языки, компилируемые в 
промежуточный код (Java, .NET), в то время как разработчики, использующие наиболее 
популярные компилируемые языки С++ и С, вынуждены пользоваться либо очень 
неудобными низкоуровневыми библиотеками ODBC (существует несколько реализаций), 
либо такими функционально ограниченными инструментами, как SOCI, ODB и~т.\,п. При 
этом, по данным крупнейшей Интернет-площадки свободных разработок sourceforge.net, 
проекты на C/C++ занимают лидирующие позиции~--- на их долю приходится более 
четверти всех проектов.

Такое положение вещей и побудило авторов к разработке собственного инструмента~--- 
многоплатформенного, поддерживающего разные СУБД, изолирующего прикладного 
программиста от специфики СУБД и SQL диалектов, способного работать с 
унаследованными приложениями/библиотеками, разработанными на C++ и С, 
обладающего компактным и легкочитаемым синтаксисом. 

\vspace*{-9pt}

\section{Цели создания EffiDB}

\vspace*{-2pt}

\noindent
\begin{itemize}
\item[$\bullet$] \textbf{Переносимость кода между СУБД.} В~IT-ин\-ду\-ст\-рии 
часто приходится разрабатывать большое количество приложений для разных 
клиентов. Как следствие, необходимо поддерживать разнообразные конфигурации 
клиентского оборудования и иногда переносить унаследованное 
ПО на новые платформы~\cite{1-y, 2-y}. Вообще говоря, это 
ПО должно быть в состоянии работать на\linebreak
 разных СУБД в зависимости от 
требований клиента~--- представляется нецелесообразным всякий раз, когда новому 
клиенту требуется очередная СУБД, заниматься ее поддержкой и\linebreak \mbox{обучать} собственных 
разработчиков работе с этой СУБД. Библиотека EffiDB обес\-пе\-чивает независимость 
кода программного приложения от СУБД. Библиотека покрывает\linebreak
 большую часть 
функциональности, пред\-остав\-ля\-емой различными СУБД, и гарантирует, что 
про\-грам\-мное приложение будет работать одинаковым образом на всех 
поддерживаемых СУБД.
\item[$\bullet$] \textbf{Проверка запросов на момент компиляции.} Приложения, 
взаимодействующие с базой данных, генерируют SQL-за\-про\-сы, которые затем 
интерпретируются и исполняются на СУБД. Так как SQL и код на C++, вообще говоря, 
никак не связаны между собой, построение SQL-за-\linebreak\vspace*{-12pt}
\pagebreak

\noindent
про\-сов подвержено ошибкам, 
причем обычно синтаксические ошибки, допущенные при создании запроса, 
обнаруживаются довольно поздно~--- только на этапе тестирования приложения. 
Библиотека EffiDB строит запросы автоматически и производит проверку в процессе 
компиляции~--- это гарантирует отсутствие синтаксических ошибок в SQL-за\-просах.
\item[$\bullet$] \textbf{Динамическое подключение базы данных.} В~некоторых 
случаях приложение должно работать с несколькими СУБД~--- одновременно либо с 
той или иной СУБД, определяемой конкретной инсталляцией. Так, например, такое 
требование актуально при интеграции унаследованных информационных систем, 
разработка которых велась независимо разными подразделениями и 
производителями~\cite{2-y}. Динамическое подключение базы данных, реализованное 
в \mbox{EffiDB}, позволяет <<на лету>> переключаться между СУБД: если изначально 
предполагалось использовать приложение с MySQL, то нет необходимости 
перекомпилировать его, например, под Oracle~--- для того чтобы запустить 
приложение на конкретной СУБД, достаточно подгрузить соответствующую 
динамическую библиотеку. Можно также менять СУБД в процессе работы 
приложения и использовать несколько СУБД одновременно.
\item[$\bullet$] \textbf{Лаконичный синтаксис.} Чрезвычайно важным\linebreak
 элементом 
разработки является лаконичность, понятность и эргономичность кода, \mbox{который} 
приходится писать прикладным раз\-ра\-ботчи\-кам. Любая недоработка в этой области 
приводит к снижению поддерживаемости и увеличивает количество ошибок. 
Благодаря\linebreak удобному синтаксису EffiDB объем текста, который требуется написать на 
C++ для создания SQL-за\-про\-са, практически не превосходит размер самого 
SQL-за\-про\-са.
\item[$\bullet$] \textbf{Простота построения запроса из одной таблицы.} На 
практике большинство SQL-за\-про\-сов представляют собой запросы по первичному 
ключу в одной таблице. Этот простейший и часто встречающийся частный случай 
имеет смысл рассматривать отдельно. Если не выделять этот важный сценарий и 
обрабатывать его обычным образом, то такой подход порождает определенные 
неудобства. Во-пер\-вых, таким образом упускается возможность повысить 
производительность. Во-вто\-рых, часто разработчики поддаются соблазну объединить 
несколько запросов в один и сконструировать сложный запрос <<на все случаи 
жизни>>~--- такие запросы могут снижать производительность очень сильно и, что 
хуже, непредсказуемым образом. Биб\-лио\-те\-ка EffiDB предоставляет специальный 
синтаксис для запросов из одной таблицы~--- это очень простой способ составления 
запроса, позволяющий обходиться минимумом кода. Фактически код C++ в этом 
случае даже короче, чем конечный SQL-за\-прос. Программисту не надо составлять 
сложные универсальные запросы~--- часто проще писать именно тот запрос, который 
нужен в данном конкретном случае.
\item[$\bullet$] \textbf{Автоматическое управление транзакциями.} Управление 
транзакциями в СУБД представляет собой нетривиальную задачу и, следовательно, 
порождает серьезные проблемы в проекте разработки программного 
продукта~\cite{4-y, 3-y}. Если фиксировать изменения (Commit changes) слишком 
часто, данные в БД могут потерять целостность. Если же фиксировать изменения 
слишком редко, то снижается производительность системы и становятся возможными 
клинчи (deadlocks). При создании библиотеки EffiDB авторы исходили из того 
соображения, что прикладной программист не должен заниматься управлением 
транзакциями, т.\,е.\ такими операциями, как управление соединениями, явная 
фиксация и откат транзакции. Библиотека обеспечивает автоматическое управление 
транзакциями, и прикладной программист не обязан думать о транзакциях. Тем не 
менее, если возникает необходимость явным образом управлять транзакциями, 
разработчик имеет возможность это делать.
\item[$\bullet$] \textbf{Автоматическая блокировка.} Довольно часто несколько 
транзакций пользуются одними и теми же данными, что приводит к хорошо известным 
проблемам управления параллельным выполнением операций. Необходимо соблюдать 
разумный баланс между произво\-ди\-тель\-ностью и изолированностью транзакций. Один 
из двух крайних подходов, который повышает производительность и пренебрегает 
изолированностью, приводит к потере данных. Противоположная стратегия, 
концентрирующаяся на изолированности транзакций, вызывает\linebreak
 взаимные блокировки 
и снижает производительность~\cite{4-y}. Авторы полагают, что целостность данных 
является жизненно важным\linebreak
 свойством информационной системы, и по умолчанию 
предпочитают изолированность производительности. Для того же, чтобы обеспечить 
приемлемую производительность и уменьшить вероятность клинчей, в EffiDB 
реализована поддержка автоматической блокировки. Различаются две категории 
транзакций: <<безопасные>> и быстрые транзакции чтения и <<потенциально 
опасные>> транзакции записи. Транзакции записи блокируют все данные, с которыми 
работают (путем использования\linebreak
 SELECT FOR UPDATE), что гарантирует 
изолированность. В~тех случаях, когда такая блокировка является избыточной, ее 
можно отключить.
\item[$\bullet$] \textbf{Поддержка произвольных типов данных.} Как правило, 
унаследованное ПО использует свои собственные типы данных, 
которые в общем случае не совпадают с типами данных СУБД. Если проект 
разработки ПО предполагает интеграцию такого рода 
приложений с одной или несколькими СУБД, необходимо преобразование типов 
данных. Библиотека \mbox{EffiDB} 
обеспечивает поддержку произвольных типов данных и пред\-остав\-ля\-ет 
возможность разработки <<адап\-те\-ра>> для собственных типов данных. Ниже 
приведен список типов данных, <<адап\-те\-ры>> для которых входят в дистрибутив 
библиотеки:
\begin{itemize}
\item[--] {\sf std::string};
\item[--] {\sf double};
\item[--] {\sf int32\_t};
\item[--] {\sf int64\_t};
\item[--] {\sf Int}~--- синоним для типа {\sf boost:: optional$\langle$int32\_t$\rangle$};
\item[--] {\sf Int64}~--- синоним для типа {\sf boost:: optional$\langle$int64\_t$\rangle$};
\item[--] {\sf Str}~--- синоним для типа {\sf boost:: optional$\langle$string$\rangle$};
\item[--] {\sf Time}~--- синоним для типа {\sf boost:: posix\_time::ptime};
\item[--] {\sf Effi::Blob} для больших блоков бинарных данных;
\item[--] {\sf Effi::GeoData} для больших блоков бинарных данных, используемых в 
геоинформационных приложениях;
\item[--] {\sf Effi::Decimal}.
\end{itemize}
EffiDB использует типы данных библиотеки Boost Optional~[6].
\end{itemize}

\vspace*{-9pt}

\section{Концепции EffiDB}

\vspace*{-2pt}

Перейдем к изложению идей, положенных в основу библиотеки EffiDB. Для изоляции 
программиста от специфики взаимодействия с конкретной СУБД EffiDB использует 
специализированные классы C++ для каждого из понятий реляционных баз данных.

В тех случаях, когда это представляется полезным, текст сопровождается примерами 
использования библиотеки.

\vspace*{-6pt}

\subsection{База данных} %5.1

\vspace*{-1pt}

Понятию <<база данных>> соответствует класс\linebreak {\sf Effi::DataBase}. 
Этот класс представляет 
собой фабрику объектов Connection (соединение)~--- соединение является единственным 
способом общения с сервером базы данных. 
{\small
\begin{verbatim}
DataBase database=
                new DataBase(libName, dbParams);
\end{verbatim}}

\noindent
Здесь {\sf libName}~--- имя драйвера СУБД, а {\sf dbParams}~--- параметры базы данных. Набор 
параметров зависит от СУБД. Так, для работы с MySQL инициализация может иметь вид:
{\small
\begin{verbatim}
string libName="libardbms_mysql.so";
std::map<string, string> dbParams;
dbParams["DSN"]="<Имя источника данных>";
dbParams["AutoCommit"]="off";
dbParams["UserName"]="<Имя\ пользователя>";
dbParams["Password"]="<Пароль>";
dbParams["Port"]="<Порт>"};
dbParams["Host"]="<Адрес сервера базы данных>";
\end{verbatim}}

\vspace*{-6pt}

\subsection{Соединение} %5.2

Соединение с базой данных позволяет прикладному ПО
общаться с ПО сервера базы данных. Соединение является 
необходимым элементом процесса, позволяющим посылать команды СУБД и получать 
ответы.

Процедура установления соединения является время- и ресурсоемкой операцией, и было 
бы\linebreak
неэффективно инициировать новое соединение всякий раз, когда приложению нужно 
связаться\linebreak
с базой данных. Поэтому в EffiDB реализован\linebreak
алгоритм организации связного 
пула соединений. Когда приложению требуется связь с базой\linebreak
 данных, сначала происходит 
поиск свободного\linebreak соединения в пуле уже существующих соединений. Только в том 
случае, если свободных\linebreak соединений нет, создается новое соединение. Эта возможность 
реализована в функции {\sf DataBase::\linebreak GetConnection()}. Когда необходимость в соединении 
пропадает, приложение должно освободить его функцией {\sf DataBase::ReleaseConnection()}. 
Таким образом \mbox{EffiDB} формирует пул свободных переиспользуемых соединений.

Соединение представлено классом {\sf Effi::\linebreak Connection}. 
\begin{verbatim}
Connection* conn;
conn=database->GetConnection();
...
database->ReleaseConnection(conn);
\end{verbatim}

\subsection{Средства манипулирования данными} %5.3

Как уже было сказано, код приложения должен быть унифицированным и работать с 
любой из поддерживаемых СУБД. С другой стороны, каждое SQL-вы\-ра\-же\-ние 
исполняется на вполне конкретной СУБД и, следовательно, должно подчиняться строгим 
правилам синтаксиса данной СУБД.

Чтобы разрешить это противоречие, EffiDB строит SQL-запрос в несколько этапов. 
Па\-ра\-мет\-ры SQL-за\-про\-са аккумулируются одним из четырех классов (каждый класс 
соответствует одному из видов SQL-вы\-ра\-же\-ний): 
{\sf Effi::Selector}, {\sf Effi::Inserter}, 
{\sf Effi::Deleter}, {\sf Effi::Updater}. 
К~параметрам относится\linebreak
 все необходимое для формирования 
запроса: множество выбираемых данных, список таблиц, из\linebreak которых выбираются данные, 
условия выборки, условия группировки, ограничения, условия сортировки и~т.\,п. Далее 
будем называть эти четыре класса <<агентами>>.

Классы-агенты не зависят от специфики СУБД. Их назначение~--- сбор полной 
информации, достаточной для построения выражения, при этом они не формируют 
собственно SQL-вы\-ра\-же\-ния. Биб\-лиотека \mbox{EffiDB} 
позволяет строить разнообразные выражения, в 
том числе с использованием псевдонимов, инструкций join, сортировки и~т.\,п.

Каждый класс-агент имеет функцию {\sf Execute()}, которая, в свою очередь, вызывает 
соот\-вет\-ст\-ву\-ющую функцию драйвера СУБД. Эта функция формирует строку 
SQL-за\-про\-са в соответствии с синтаксисом SQL конкретной СУБД.

\section{Формирование SQL-запросов}

Приведем несколько примеров использования библиотеки (предположим, что работаем с 
MySQL).

Рассмотрим базу данных книжного магазина. Книга имеет название, автора и цену. 
Известно количество экземпляров каждой книги.

Цель~--- сформировать типичные SQL-запросы из приложения, написанного на C++.

База данных содержит, в частности, таблицу Bookstore~--- она представляет собой список 
книг: 

\end{multicols}

\hrule

\begin{verbatim}
ID Title                                    Author      Price   Qty

101 Algorithms + Data Structures = Programs Wirth       120.35   14
187 1001 Nights                             NULL        999.99    2
318 The Decameron                           Boccaccio  1285.99    1
325 C++                                     Stroustrup   59.85   10
400 The Art of Computer Programming         Knuth       299.99    0
\end{verbatim}

\vspace*{3pt}

\hrule

\begin{multicols}{2}


\subsection{Установление соединения} %6.1

В начале работы необходимо определить, драйвер какой СУБД будет использован, задать 
параметры базы данных, создать объект DataBase и установить соединение: 
\begin{verbatim}
DataBase* database= 
       new DataBase(libName, dbParams);
Connection* conn;
try {
  conn=database->GetConnection();
}
catch(Exception e) {
  cerr << "ERROR: " << e.What() << endl;
  database->ReleaseConnection(conn);
  throw;
}
\end{verbatim}


\subsection{Выборка из таблицы} %6.2

Рассмотрим теперь задачу построения запроса, выбирающего дорогие книги и 
возвращающего идентификатор книги, ее название и цену с учетом НДС.

Таблица базы данных представлена объектом класса {\sf Effi::SimpleTable}, колонки 
таблицы~--- объектами класса {\sf Effi::Column}:
\begin{verbatim}
// Определение таблицы базы данных
SimpleTable bookStore("Bookstore");

// Определение колонок таблицы, необходимых
// для запроса
Column ID(bookStore, "ID");
Column Title(bookStore, "Title");
Column Price(bookStore, "Price");
\end{verbatim}

Для доступа к результатам запроса используются вспомогательные переменные. 
Соответствие между полями таблиц базы данных и вспомогательными переменными 
устанавливается посредством технологии связывания (\textit{binding}), как будет показано 
ниже. 
\begin{verbatim}
// Определение вспомогательных переменных
// для доступа к результатам запроса
int id;
string title;
double totalPrice;

// Пороговое значение цены
double threshold = 300;
// Ставка НДС, \%
double vat = 0.2;
\end{verbatim}

Собственно формирование запроса состоит из указания колонок и связанных с ними 
вспомогательных переменных, а также установки критериев выбора: 
{\small \begin{verbatim}
Selector s;
// Определение набора колонок таблицы
// и связывание их
// со вспомогательными переменными
s << ID.Bind(id) << Title.Bind(title) << 
(Price*(1+vat)).As("TotalPrice"). 
Bind(totalPrice);
// Определение критериев выборки
s.Where(Price > threshold);
\end{verbatim}}

Выполнение запроса осуществляется посредством вызова метода {\sf Effi::Selector::Execute()}: 
\begin{verbatim}
// Выполнение SQL-запроса
DataSet data=s.Execute(conn);
\end{verbatim}
Результатом выполнения SQL-за\-про\-са типа SELECT является набор записей. Доступ к 
результатам запроса осуществляется посредством класса\linebreak {\sf Effi::DataSet}. Функция 
{\sf DataSet::Fetch()} считывает очередную запись в соответствующие вспомогательные 
переменные, связанные с колонками табли\-цы результатов функцией {\sf Bind()}. Так, в 
рассмат\-ри\-ваемом примере колонка {\sf ID} связана с переменной~id, колонка {\sf Title}~--- с 
переменной\linebreak title, а выражение ({\sf Price*(1+vat)})~--- с переменной totalPrice. После 
считывания внутренний указатель функции {\sf Fetch()} смещается на следующую запись. Если 
запись последняя, то функция {\sf Fetch()} возвращает false. Если поле таблицы результатов не 
связано ни с одной вспомогательной переменной, то данные этого поля недоступны из 
С++ кода.

Следующий код выводит результат исполнения запроса SELECT: 
\begin{verbatim}
while(data.Fetch()) {
   cerr << id << "; " << title << "; " 
   << totalPrice << endl;
}
\end{verbatim}

Сформированный SQL-запрос имеет вид:
\begin{verbatim}
SELECT ID, Title, Price*1.2 AS TotalPrice
FROM Bookstore
WHERE Price > 300;
\end{verbatim}
и возвращает следующие данные: 
\begin{verbatim}
ID   Title          TotalPrice

187; 1001 Nights;   1199.99
318; The Decameron; 1543.19
\end{verbatim}

\subsection{Классы-обертки реляционных таблиц} %6.3

Описанный выше метод универсален в том смысле, что позволяет работать с таблицами, 
структура которых неизвестна на момент компиляции.

Представляется, однако, что в большинстве реальных ситуаций структура базы данных 
известна заранее, и тогда такой подход оказывается избыточным и неудобным~--- он 
требует создавать пред\-став\-ле\-ния реляционных сущностей (таблиц и колонок базы 
данных) в С++ всякий раз, когда с ними работают. Так, если некая таблица участвует в 
одном SQL-за\-про\-се дважды, то требуется создавать два объекта класса {\sf SimpleTable}, 
описывать для каждого из них одинаковый набор колонок и~т.\,п. 

Библиотека EffiDB предоставляет возможность создать универсальный класс-оберт\-ку для работы с 
известной на момент компиляции таблицей базы данных. Такой класс должен быть 
производным от класса {\sf Effi::Domain}: 
\end{multicols}

\hrule

\vspace*{6pt}

\begin{verbatim}
class Bookstore : public Domain {
public :
  struct BookstoreDomain {
  Column ID;
  Column Title;
  Column Author;
  Column Price;
    BookstoreDomain(const SimpleTable& table)
    : ID(table, "ID");
    , Title(table, "Title");
    , Author(table, "Author");
    , Price(table, "Price"){}
  } Domain_;
  boost::optional<int> ID;
  boost::optional<string> Title;
  boost::optional<string> Author;
  Decimal Price;
  Bookstore() : Domain("Bookstore"), Domain_(*this) {
    // COLUMN_PK означает, что колонка входит в первичный ключ
    columns_.push_back(ColSpec(Domain_.ID.Bind(ID), COLUMN_PK));
    // COLUMN_ ORDINARY означает, что колонка не является ключевой
    columns_.push_back(ColSpec(Domain_.Title.Bind(Title), COLUMN_ORDINARY));
    columns_.push_back(ColSpec(Domain_.Author.Bind(Author), 
COLUMN_ORDINARY));
    columns_.push_back(ColSpec(Domain_.Price.Bind(Price), COLUMN_ORDINARY));
  }
  Bookstore(const Bookstore& copy) : Domain("Bookstore"), Domain_(*this)
  , ID(copy.ID);
  , Name(copy.Title);
  , Author(copy.Author);
  , Phone(copy.Price) {
    columns_.push_back(ColSpec(Domain_.ID.Bind(ID), COLUMN_SK));
    columns_.push_back(ColSpec(Domain_.Title.Bind(Title), COLUMN_ORDINARY));
    columns_.push_back(ColSpec(Domain_.Author.Bind(Author), 
COLUMN_ORDINARY));
    columns_.push_back(ColSpec(Domain_.Price.Bind(Price), COLUMN_ORDINARY));
  }
  Bookstore& operator= (const Bookstore& copy) {
    ID = copy.ID;
    Title =copy.Title;
    Author=copy.Author;
    Price =copy.Price;
    return *this;
  }
  virtual ~Bookstore() {}
  BookstoreDomain* operator->() { return &Domain_; }
};
\end{verbatim}

\vspace*{6pt}

\hrule

\begin{multicols}{2}

В классе-обертке определяются колонки таблицы и набор вспомогательных переменных, 
соответствующий набору колонок, а также устанавливается соответствие между ними. 

Имена объек\-тов-ко\-ло\-нок и имена вспомогательных переменных задаются 
совпадающими с именами полей таблиц. Имена объектов-колонок доступны посредством 
оператора <<$\rightarrow$>>, например Bookstore$\rightarrow$Price, а значения полей~--- 
посредством оператора <<.>>, например \mbox{Bookstore.Price}. 

Колонки, входящие в первичный 
ключ таблицы, помечаются как ключевые.

Таким образом, необходимость всякий раз при использовании экземпляров класса 
Bookstore связывать колонки таблицы со вспомогательными переменными отпадает. 

Только в тех случаях, когда вспомогательные переменные не имеют соответствующей 
колонки, но требуются в конкретном запросе (totalPrice в рассматриваемом примере), 
следует по-прежнему явным образом создавать сами переменные и устанавливать их связь 
с колонками таблицы результатов запроса.

Работа с классами-обертками обеспечивает ряд немаловажных преимуществ.

Во-первых, описание таких сущностей базы данных, как таблица и ее колонки, 
локализовано в одном месте кода. Это снижает до минимума риск появления 
синтаксических ошибок и облегчает процесс поддержки и развития приложения.

Во-вторых, код, формирующий рассмотренный выше SQL-за\-прос, оказывается гораздо 
короче: 
\begin{verbatim}
Bookstore tbl;
double threshold = 300;
double vat = 0.2;
double totalPrice;
Selector s;
s << tbl->ID << tbl->Title
  << (tbl->Price*(1+vat))
  .As("TotalPrice").Bind(totalPrice);
s.Where(tbl->Price>threshold);
\end{verbatim}

Легко видеть, что создание объекта, пред\-став\-ля\-юще\-го таблицу, сводится теперь к 
объявлению переменной, а объявление вспомогательных переменных и связывание их с 
колонками базы данных не требуется.

И, наконец, рассмотрим сценарий, в котором выгода от использования клас\-са-оберт\-ки 
наиболее существенна и очевидна. На практике чаще всего используются SQL-за\-про\-сы 
по первичному ключу таблицы. Синтаксис EffiDB в этом случае крайне прост, а размер 
текста кода оказывается даже меньшим, чем размер итогового SQL-за\-проса.

Попробуем, к примеру, выполнить все 4~возможных вида SQL-выражений на 
рассматриваемой базе данных: 
\begin{verbatim}
// Определение уникального кода
// для новой книги
// и проверка его доступности tbl.ID=100;
if (tbl.Select(conn)) throw Exception 
("Книга с таким кодом уже существует.");

// Добавление новой книги
tbl.Title = "Hobbit";
tbl.Author = "Tolkien";
tbl.Insert(conn);

// Изменение названия книги
tbl.Title = "Hobbit or There and
            Back Again";
tbl.Update(conn);

// Удаление книги
tbl.Delete(conn);
\end{verbatim}

Соответствующие SQL-выражения: 
\begin{verbatim}
SELECT *
FROM Bookstore
WHERE ID = 100;

INSERT INTO Bookstore (ID, Title, 
       Author, Price, Qty)
VALUES (100, "Hobbit", "Tolkien", 
       NULL, NULL);

UPDATE Bookstore
SET Title = "Hobbit or There and Back 
 Again", Author = "Tolkien", 
 Price = NULL, Qty = NULL
WHERE ID = 100;

DELETE FROM Bookstore
WHERE ID = 100;
\end{verbatim}

Аналогичным образом составляются запросы по составному первичному ключу.

\section{Интегрированная среда разработки приложений Effi}

Библиотека EffiDB является лишь одним из компонентов интегрированной среды 
разработки приложений Effi.

Разработчик, использующий среду Effi, описывает приложение на декларативном языке в 
формате XML (eXtensible Markup Language), 
оперируя понятиями классов, методов и элементов графического 
интерфейса. Все объекты C++ кода, представляющие понятия реляционной базы данных, 
генерируются автоматически. В~част\-ности, получается полностью готовый к 
использованию класс-оберт\-ка. Разработчик имеет возможность сосредоточиться на 
программировании логики приложения и не тратить усилия на описание вспомогательных 
объектов.

\section{Заключение}

Для решения проблемы унифицированного\linebreak взаимодействия с разнотипными СУБД 
создана оригинальная С++ библиотека EffiDB; библиоте-\linebreak ка доступна в сети Интернет по 
адресу {\sf http://\linebreak sourceforge.net/projects/effidb}~\cite{7-y}.

Библиотека EffiDB гарантирует полную переносимость программных приложений за счет изоляции 
прикладного кода от специфики конкретной СУБД. В~библиотеке реализован механизм 
динамического подключения драйверов разных СУБД в процессе работы приложения, а 
также возможность одновременного подключения нескольких СУБД. Библиотека 
обеспечивает защиту от синтаксических ошибок в SQL-выражениях за счет разбиения 
построения SQL-выражений на несколько этапов и проверки на момент компиляции.

Библиотека EffiDB используется в ряде проектов разработки ПО 
компании ASoft {\sf http://asoft.ru}, среди которых:
\begin{itemize}
\item аппаратно-программный комплекс центра хранения электронных копий 
материалов уголовных дел для Министерства внутренних дел РФ <<Невод-Р>>;
\item объединенная поисковая федеральная система генетической идентификации 
<<Ксенон-2>>;
\item единая автоматизированная информационная система дежурных частей органов 
внутренних дел;\\[-5pt]
\item система автоматизации деятельности по линии борьбы с экстремизмом для 
Департамента по противодействию экстремизму МВД Рос-\linebreak сии;\\[-5pt]
\item автоматизированная система управления взаимоотношениями с контрагентами 
ASoft CRM;\\[-5pt]
\item геоинформационная система хранения, обработки и визуального представления 
картографической информации;\\[-5pt]
\item конструктор отчетов, позволяющий строить отчеты любой степени сложности и 
вложенности, а также задавать макет представления дан-\linebreak ных;\\[-5pt]
\item информационная система для совместной работы над документами ASoft 
Collaboration.
\end{itemize}

{\small\frenchspacing
{%\baselineskip=10.8pt
\addcontentsline{toc}{section}{Литература}
\begin{thebibliography}{9}

 \bibitem{6-y} %1
Список современых средств объектно-ре\-ля\-ци\-он\-но\-го отображения доступа к данным.  {\sf 
http://en.wikipedia. org/wiki/List\_of\_object-relational\_mapping\_software}.

\bibitem{1-y} %2
\Au{Янушко А.\,В.} Современные реляционные СУБД~// Банки и технологии, 1998. №\,2.

\bibitem{2-y} %3
\Au{Ambler S.\,W., Sadalage P.\,J.}. Refactoring databases: Evolutionary 
database design.~--- Addison-Wesley Professional, 2006. 384~p.

\bibitem{4-y} %4
\Au{Gray J., Reuter A.} Transaction processing: Concepts and techniques.~--- Morgan 
Kaufmann Publs., 1993.

\bibitem{3-y} %5
\Au{Бегг К., Коннолли Т., Страчан~А.} Базы данных: Проектирование, реализация и 
сопровождение: Теория и практика~/ Пер. с англ.~--- М.: Вильямс, 2001.

\bibitem{5-y} %6
Библиотека Boost Optional в сети Интернет. {\sf 
http:// www.boost.org/doc/libs/1\_41\_0/libs/optional/doc/ html/index.html}.

\label{end\stat}

\bibitem{7-y} %7
Библиотека EffiDB в сети Интернет. {\sf http://sourceforge. net/projects/effidb}.
 \end{thebibliography}
}
}


\end{multicols}           %8
\def\stat{kuzn}

\def\tit{ВЕРОЯТНОСТНО-СТАТИСТИЧЕСКАЯ ОЦЕНКА 
АДЕКВАТНОСТИ ИНФОРМАЦИОННЫХ ОБЪЕКТОВ}

\def\titkol{Вероятностно-статистическая оценка 
адекватности информационных объектов}

\def\autkol{Л.\,А.~Кузнецов}
\def\aut{Л.\,А.~Кузнецов$^1$}

\titel{\tit}{\aut}{\autkol}{\titkol}

%{\renewcommand{\thefootnote}{\fnsymbol{footnote}}\footnotetext[1]
%{Работа поддержана Российским фондом фундаментальных исследований
%(проекты 11-01-00515а и 11-07-00112а), а также Министерством
%образования и науки РФ в рамках ФЦП <<Научные и
%научно-педагогические кадры инновационной России на 2009--2013~годы>>.}}


\renewcommand{\thefootnote}{\arabic{footnote}}
\footnotetext[1]{Липецкий государственный технический университет, kuznetsov@stu.lipetsk.ru}


\Abst{Приведены математические основы и оригинальная методология разработки 
систем оценки семантической близости информационных объектов (ИО), представленных на 
естественном языке. Вводится ве\-ро\-ят\-но\-ст\-но-ста\-ти\-сти\-че\-ское представление 
сопоставляемых ИО. Используется теория информации для 
оценки уровня семантической близости ИО. Методология 
доведена до алгоритмов ее реализации в виде соответствующей автоматизированной 
системы. Представлены результаты практической проверки эффективности методологии. 
}

\vspace*{2pt}

\KW{информационные объекты; естественный язык; семантическая адекватность; 
вероятностная модель; теория информации}

\vspace*{6pt}

 \vskip 14pt plus 9pt minus 6pt

      \thispagestyle{headings}

      \begin{multicols}{2}
      
            \label{st\stat}

\section{Введение. Информация, знания, семантический анализ}
   
   Основным мотивом перехода от индустриальной к постиндустриальной 
модели развития в промышленно развитых странах, начавшегося в конце 
прошлого века, является стремительное увеличение скорости развития науки 
и знания. В~постиндустриальной модели развития, по мнению ведущих 
западных ученых в области социального развития и управления, 
принципиальным является изменение статуса и значения информации, науки 
и знания, которые становятся важнейшими факторами, определяющими 
эволюцию общества. 
   
   Питер Ф.~Друкер, признанный специалист в области организационного 
управления, пишет: <<Изменение значения знания, начавшееся 250~лет тому 
назад, преобразовало общество и экономику. Знание стало сегодня основным 
условием производства. Традиционные <<факторы производства>>~--- земля 
(природные ресурсы), рабочая сила и капитал~--- не исчезли, но приобрели 
второстепенное значение. Эти ресурсы можно получить, причем без особого 
труда, если есть необходимые знания>>~\cite{1-k}. 
   
   Информация и знания становятся главной движущей силой 
экономического развития и перехо-\linebreak дят из категории бесплатного 
общественного бла-\linebreak га в категорию товара. В~промышленно развитых\linebreak \mbox{странах} 
разработка и внедрение технологических инноваций~--- решающий фактор 
социального и экономического развития, залог экономической безопасности. 
В~США, по оценкам американских специалистов, прирост душевого 
национального дохода благодаря этому фактору составляет 90\%. 
   
   Беспрецедентный рост потока информации и знаний, скорости их 
передачи и возможностей доступа на первый план научных проблем 
выдвигает разработку технологий их автоматической обработки. Б$\acute{\mbox{о}}$льшая 
часть существующих и вновь формируемых знаний и информации 
представлена на естественном языке. 

Одной из актуальных, 
фундаментальных проблем в области обработки информации становится 
обеспечение возможности формального семантического сравнения, оценки 
семантической \mbox{бли\-зости} ИО, представленных 
на естественном языке. Разработка формализованных технологий оценки 
семантической близости ИО позволила бы перейти к практической 
реализации важных задач в сфере обработки информации, распространения 
знаний и образования. 
   
   В настоящее время в литературе задача семантического сравнения двух 
текстов, в основном, рас\-смат\-ри\-ва\-ется в контексте дубликатов в 
   веб-докумен\-тах и в системах автоматизированного перевода. При 
поиске дубликатов опираются на число слов, совпавших в двух текстах. 
Алгоритм сравнения на основе шинглов является наиболее простым и 
распространенным. Такой подход используется для нахождения копий 
текстов, полученных копированием и перестановкой слов, но он не позволяет 
оценить семантическую близость текстов.
   
   При более сложном анализе текстов учитывается структура входящих в 
них предложений. В~предложениях выделяются элементы (слова или группы 
слов) и сопоставляются определенные шаблоны для этих элементов. Данный 
подход описан в книге~\cite{2-k} и используется в ряде кандидатских 
диссертаций. 
   
   Однако задача оценки информационной бли\-зости двух текстов в 
обнаруженных автором работах не затрагивается. Используемые там 
концепции не ориентированы на ее решение и не могут быть использованы в 
качестве основы для ее решения.
   
   В данной статье предлагается оригинальная методология оценки 
информационной близости текстов на основании вероятностно-ста\-ти\-сти\-че\-ско\-го 
подхода и теории информации. Реализация методологии 
позволит перейти к практическому решению разнообразных задач, в которых 
требуется определять меру информационной адекватности\linebreak документов, 
представленных на естественном языке. В~част\-ности, разработанная 
концепция будет использована для синтеза автоматизированных сис\-тем 
оценки уровня знаний. 

\section{Формализация анализа текстов}

   Развитие информационных технологий, предо\-став\-ля\-ющих широкие 
возможности ав\-то\-ма\-ти\-зи\-рован\-но\-го анализа и обработки вербально 
пред\-став\-лен\-ной информации, существенно повысило интерес к разработке 
формальных методов исследования и сопоставления текстов. Современные 
компьютерные системы позволяют хранить и обрабатывать практически 
неограниченные объемы текстовой информации. Это стимулирует 
разработку формальных методов для поддержки выполнения постоянно 
расширяющихся и углубляющихся исследований информации, 
представленной на естественном языке. В~настоящее время интенсивно 
разрабатываются формальные методы, позволяющие автоматизировать 
решение задач в области морфологического, синтаксического и 
семантического анализа текстовой информации. 
   
   Из имеющихся публикаций следует, что методологии различных видов 
анализа базируются на сходных концепциях и достаточно близки по своему 
содержанию. Формализация морфологического анализа направлена на 
алгоритмическое пред\-став\-ле\-ние грамматики русского языка. Час\-ти речи 
русского языка определены, однозначно определены формы, в которых они 
могут быть, определены правила, следуя которым должно осуществляться 
изменение слов, принадлежащих к различным час\-тям речи при образовании 
соответствующих форм. Значительная часть правил изменения частей речи 
уже отражена в словарях. Все правила могут быть представлены в виде 
соответствующих процедур, функций, подпрограмм и~т.\,п. В~соответствии 
с имеющимися правилами может быть идентифицировано, какой частью 
речи является конкретное слово, в какой форме оно находится. Поэтому 
может быть написана программа, обеспечивающая автоматизированное 
выполнение морфологического анализа, так что на ее вход будет поступать 
слово предложения, а на выходе она выдаст результат анализа: какой частью 
речи является данное слово и в какой форме оно находится. 
   
   Формализация синтаксического анализа~--- задача также понятная: 
существует синтаксис русского языка, представляющий свод правил, следуя 
которым могут быть достаточно четко определены члены предложения. Раз 
правила существуют, то их можно представить в виде набора процедур, 
обеспечивающих выявление состояния (роли) каждого слова в предложении. 
Правила и процедуры могут быть более или менее сложными, но 
принципиально то, что правила имеются, а следовательно, и процедуры 
могут быть синтезированы по ним. На основании этих процедур может быть 
разработана система синтаксического анализа, которая, получая на свой вход 
предложение, выполнит его разбор и анализ и на выходе, как примерный 
ученик выдаст о каждом слове, входящем в предложение, информацию: 
каким его членом оно является. 
   
   Проблема семантического анализа текстов интенсивно исследуется в 
различных аспектах: разрабатываются правила и алгоритмы анализа 
предложений, выявления их структуры, установления соответствия между 
разноязычными текстами, поиска информации и~т.\,д. При этом, однако, 
формализация семантического анализа ка\-ко\-го-ли\-бо одноязычного текста 
или даже одного предложения представляется задачей весьма малопонятной. 
   
   Под формализацией обычно понимается однозначное математическое 
представление существующих правил, которые, возможно, в текстовом, 
вербальном виде содержат определение способа извлечения нужных 
сведений из начальных, исходно заданных данных. В~контексте 
формализации семантического анализа математическому оформлению 
должны подлежать правила, позволяющие извлечь из слова его смысл. Но, в 
отличие от морфологии и синтаксиса, не существует ка\-ких-ли\-бо 
формальных семантических правил, следуя которым можно было бы 
установить смысл, вложенный в предложение или в каждое отдельное его 
слово. Поэтому невозможно представить систему семантического анализа в 
виде упорядоченного набора правил или предписаний, которая (по аналогии 
с системами морфологического или синтаксического анализа) получала бы 
на входе предложение или слово, а на выходе выдавала бы его смысл. Ибо 
слово и есть его смысл. В~толковом словаре, конечно, разъясняется смысл 
отдельных слов, но, в конечном итоге, это разъяснение представляет 
сопоставление одному слову других слов, близких по смыслу, и следует из 
словаря, а не из каких-либо правил, которые можно было бы формализовать.
   
Следовательно, если морфологический и синтаксический анализ действительно 
представляют анализ в соответствии со смыслом этого слова, т.\,е.\ разбор 
предложения на составляющие его элементы и выяснение их роли и 
состояния, то семантический анализ может пониматься только в смысле 
сравнения и выяснения смысловой близости разных слов и текстов. Только 
при наличии эталона анализируемого предложения, смысл которого 
известен, опираясь на словари, в которых отражена семантическая близость 
отдельных слов и словосочетаний, может быть получен ответ, что 
анализируемое предложение находится в некотором соответствии с эталоном 
и, следовательно, имеет определенный смысл. Например, при переводе 
смысл на язы\-ке-ори\-ги\-на\-ле принимается за известный эталон. С~помощью 
словаря, в котором имеется соответствие между словами и 
словосочетаниями, находится соответствующее выражение на другом языке. 
Важно понимать, что соответствие при этом следует не из слов, а из словаря. 
   
Таким образом, представляется, что при сопоставлении одноязычных текстов 
более правильно говорить не об их семантическом анализе, а об 
уста\-нов\-ле\-нии уровня их информационной адекватности, об определении 
взаимного количества информации, общего для сравниваемых текстов, из 
общего объема информации, содержащегося в одном из них, принимаемом за 
эталонный текст. 
   
   Автоматизированная технология должна обеспечивать реализацию 
функций определения пересечения, общей части дубля и эталона. Для этого в 
автоматизированной технологии должны быть разработаны формально-математические 
инструменты для представления текстов дубля и эталона в 
виде, позволяющем оценить количество информации, содержащейся в них, 
определить долю информации в дубле, отражающую содержание эталона, и 
на этой основе сформировать оценку их семантической близости. 
   
\section{Ограниченность возможностей детерминированного 
подхода}
   
   Имеется принципиальное базовое отличие семантического анализа от 
морфологического и синтак\-си\-че\-ско\-го. Отмеченное кратко выше показывает, 
что объектом морфологического и синтаксического анализа является 
фиксированный, \mbox{полностью} однозначно определенный текстовый фрагмент. 
Определение роли и состояния оборотов или отдельных слов, составляющих 
предложения анализируемого фрагмента, производится по четко 
определенным, детерминированным правилам, которые могут быть 
представлены в виде более или менее сложных алгоритмов, формирующих 
морфологические или синтаксические характеристики предложений, 
оборотов и слов. Процесс и правила формирования морфологических и 
синтаксических характеристик и сами характеристики определенны и 
закономерны. Поэтому эти виды анализа закономерны или 
детерминированы.
   
   На первый взгляд, кажется заманчивой идея представить текст эталона и 
дубля в виде предложений, предложения в виде деревьев или иных 
детерминированных структур и затем сравнить структурированное таким 
образом представление эталона и дубля. Однако русский язык, а здесь 
подразумевается, что именно он используется для вербального 
представления информации, совершенно игнорирует какие-либо 
структурные ограничения по расположению членов предложения, по виду 
предложений, формированию фрагментов предложений из групп слов, 
изобилует бесконечным многообразием форм управления отдельными 
словами и группами слов. По этой причине можно ожидать, что в эталоне и 
дубле не окажется тождественно равных структурных единиц, а чис\-ло 
альтернатив, подлежащих сравнению, может быть бесконечным. В~такой 
ситуации становится принципиальной проблема определения альтернатив и 
логических правил их разрешения. Поэтому представляется, что 
детерминированный семантический анализ не вполне соответствует 
содержательному существу проб\-лемы.
   
   По мнению автора, семантический анализ как термин не вполне удачен. 
Речь может идти об установлении уровня информационного соответствия 
содержания одного, анализируемого текста, который здесь именуется 
дублем, содержанию другого, именуемого эталоном, текста. Представляется, 
что реализация сравнения, выявления уровня соответствия нескольких 
русскоязычных текстов использованием детерминированного 
структурирования и детерминированных правил оценки близости не 
представляется практически возможной. С~учетом интеллектуальной 
специфики одна и та же информация может случайным образом облекаться в 
различную текстовую оболочку. Основная проб\-ле\-ма оценки степени 
близости информационных объектов, представленных на естественном 
языке, следует из семантической многозначности слов и наличия синонимов. 
Эти обстоятельства приводят к неоднозначности лексического представления 
семантического содержания текстов. Проблемы неоднозначности текстовой 
информации известны и активно исследуются специалистами в области 
русского языка. Детерминированный подход сравнения текстов оказывается 
нацеленным фактически на формальное описание тонкостей образования 
синтаксических форм русского языка, многообразие которых представляется 
бесконечным. 
   
   Очевидно, что случай полного совпадения\linebreak текстов, используемый при 
формировании рег\-ла\-мента доступа к информации, здесь не рас\-смат\-ри\-ва\-ет\-ся. 
Должна присутствовать возможность выделения общности 
информационного содержания со\-по\-став\-ля\-емых информационных объектов 
из их случайным образом выбранной формы представления на естественном 
языке. Решение такой задачи может быть получено только при описании 
взаимосвязи семантического (информационного) содержания и лексического 
оформления обоих срав\-ни\-ва\-емых текстов с вероятностно-статистических 
позиций.

\section{Элементы теории информации}

   Более полувека существует в виде научной дисциплины теория 
информации. Ее основоположником является американский специалист в 
об\-ласти передачи информации в технических линиях связи Клод 
   Шен\-нон~\cite{3-k}. Значительный вклад в теорию информации, 
особенно в строгое доказательство ее основных принципов, внесли советские\linebreak 
ученые школы А.\,Н.~Колмогорова~\cite{4-k}. В~теории информа\-ции 
исследуются проблемы передачи и преобразования информации, при этом 
вводится количественная оценка информации. Применительно к проблеме 
сравнения близости ИО, которой посвящена 
данная статья, является важным, что в теории информации разработаны 
теоретические основы исследования бли\-зости сообщений, переданного 
передатчиком и принятого приемником на другом конце линии связи. При 
этом вводится количественная мера информации, которая позволяет 
осуществить сопоставление информационной емкости переданного и 
принятого сообщений и на этой основе оценить искажение (потери) 
информации в линии связи при ее пе\-ре\-даче. 
   
   Теория информации может быть использована для решения проблемы 
оценки близости ИО, представленных на естественном языке. Ничто не 
мешает вместо сообщений, принятого и переданного, рассматривать 
ИО, трактуя один из них~--- аналог переданного 
сообщения~--- как эталонный информационный объект (ЭИО), а другой~--- 
аналог принятого сообщения~--- как дубль ЭИО (ДИО). 
Как будет видно дальше, мера количества информации в 
одном объекте о другом симметрична, поэтому при количественной оценке 
их близости не важно, какой объект считать эталоном, а какой~--- дублем. 
Понятно, что сравнение ЭИО и ДИО на абсолютное их совпадение 
неприемлемо. В~теории информации исследуется количественная, а не 
содержательная сторона информации. В~связи с наличием случайных помех 
в системах формирования и линиях передачи информации сообщения, 
переданное и принятое, интерпретируются случайными величинами~$\xi$. 
Шенноном было предложено использовать энтропию\footnote{Энтропия (от 
греческого \textit{entropia}~--- превращение) введена в 1865~г.\ немецким физиком Р.~Клаузиусом как 
функция состояния термодинамической системы, изменение которой $dS$ в равновесном процессе равно 
отношению количества теплоты $dQ$, подведенного к системе или отведенного от нее, к 
термодинамической температуре системы~$T$: $dS=dQ/T$. Л.~Больцман, один из основателей 
статистической термодинамики, предложил использовать энтропию как меру вероятности пребывания 
системы в данном состоянии. Шеннон ввел энтропию в теорию информации в качестве меры количества 
информации, которое выражается через распределение вероятностей.} как вероятностную меру 
количества информации~\cite{3-k}.
   
   Энтропия исхода определяется в виде логарифма вероятности этого 
исхода:
   \begin{equation}
   H(\xi_i)=- \log p(\xi_i)\,,
   \label{e1-k}
   \end{equation}
а усредненная энтропия случайной величины~$\xi$ выражается через 
функцию распределения ее вероятностей в виде:
\begin{equation}
H_\xi =- \sum\limits_\xi p(\xi)\log p(\xi)\,,
\label{e2-k}
\end{equation}
где $\xi$~--- случайная величина; $p(\xi)\leq 1$~--- распределение ее 
вероятностей. 
   
   В рассматриваемом здесь случае анализа текстов случайными 
величинами могут быть слова или другие конструкции. Под количеством 
информации в теории информации понимается неопределенность, 
устраняемая в результате выяснения исхода, т.\,е.\ значения, принимаемого 
случайной величиной. 
   
   Простейший содержательный пример в контексте статьи может быть 
следующим. Слова: \textit{пример, образец, экспонат} в некотором контексте 
являются синонимами и могут использоваться для обозначения 
объекта~$\xi$ с вероятностями: $p(\xi=\;\mbox{\textit{пример}})\hm=0{,}2$; 
$p(\xi=\;\mbox{\textit{образец}})\hm=0{,}4$; 
$р(\xi=\;\mbox{\textit{экспонат}})\hm=0{,}6$. В~этом случае количество 
информации, получаемое при реализации конкретного исхода, допустим, при 
использовании слова экспонат, т.\,е.\ $\xi\hm=\;\mbox{\textit{экспонат}}$, будет 
в соответствии с~(\ref{e1-k}) равно $\log p(0{,}6)$, а усредненная 
неопределенность объекта~$\xi$ будет по~(\ref{e2-k}) равна: $H_\xi\hm=- 
p(0{,}2)\log p(0{,}2)\hm- p(0{,}4)\log p(0{,}4)\hm- p(0{,}6)\log p(0{,}6)$. 
В~теории информации наиболее часто используются логарифмы по 
основанию~2, в этом случае количество информации определяется в битах.
   
   В реальности чаще интерес представляет сравнение информационной 
емкости сообщений, оценка имеющейся в них совместной информации. Для 
этого по распределениям вероятностей сообщений определяется энтропия 
каждого из них (количество информации в каждом из них), а по совместному 
распределению вероятностей~--- совместная энтропия. По энтропиям 
оценивается количество взаимной информации. 
   
   При использовании теории информации для описания закономерностей 
передачи информации энтропия переданного сообщения определяет 
количество переданной информации, а энтропия принятого сообщения~--- 
количество принятой информации. Общая часть в переданном и принятом 
сообщениях определяет количество взаимной информации. Обозначив через 
$\xi$ переданное сообщение, а через~$\eta$~--- принятое, количество 
взаимной информации $I (\xi\eta)$, или количество информации, 
содержащееся в принятом сообщении~$\eta$ из переданного 
сообщения~$\xi$, можно определить, следуя теории информации~\cite{3-k}, 
в виде: 
   \begin{equation}
   I_{\xi\eta} =\int\limits_X\! \int\limits_Y p_{\xi\eta} (x,y)\log \fr{p_{\xi\eta} 
(x,y)}{p_\xi(x)p_\eta(y)}\,dxdy\,,
   \label{e3-k}
   \end{equation}
где $p_\xi(x)$~--- плотность распределения переданного сообщения;
   $p_\eta(y)$~--- плотность распределения принятого сообщения;
   $p_{\xi\eta}(x,y)$~--- плотность совместного распределения; 
   $X$ и $Y$~--- области определения~$x$ и~$y$ соответственно.
   
   В~(\ref{e3-k}) имеет место равноправное симметричное вхождение~$\xi$ 
и~$\eta$, поэтому взаимная информация симметрична относительно~$\xi$ 
и~$\eta$. Отсюда следует, что при количественной оценке взаимной 
информации не важно, какое сообщение выступает в роли переданного, а 
какое~--- в роли принятого. 
   
   Именно взаимная информация может использоваться в качестве меры 
подобия ИО. Нетрудно видеть, что содержательное существо теории 
информации, направленное на оценку потерь информации при ее передаче, 
адекватно содержательной сущности многих задач в области исследования 
семантической близости ИО на естественном языке. Применение теории 
информации для решения проблемы оценки близости ИО может 
основываться на замене сообщений исследуемыми ИО. 
Один объект (эталон) может интерпретироваться переданным 
сообщением, а второй (дубль)~--- принятым сообщением. Вследствие 
симметрии от перемены ролей количество взаимной информации не 
изменится. Формула~(\ref{e3-k}) отражает количество взаимной информации 
непрерывных сообщений, точнее сообщений, случайный характер которых 
определяется непрерывными функциями распределения вероятностей вида 
$p_\xi(x)$, $x\in X \hm= [x^\prime, x^{\prime\prime}]$, где $x^\prime$ и 
$x^{\prime\prime}$~--- предельные значения~$x$. 
   
   При анализе текстов в качестве случайных величин будут выступать 
синтаксические или морфологические компоненты, которые являются 
дискретными величинами. Их случайными лексическими значениями будут 
выступать слова или словосочетания, характеризуемые дискретными 
ве\-ро\-ят\-но\-стя\-ми, как это показано в приведенном выше кратком примере. 
В~примере под случайным объектом или компонентом~$\xi$ может 
пониматься подлежащее при использовании синтаксической структуризации 
или существительное при использовании морфологической структуризации. 
Компонент~$\xi$ в обоих случаях может принимать случайные значения 
\textit{пример}, \textit{образец}, \textit{экспонат} с вероятностями 
$p(\xi=\;\mbox{\textit{пример}})\hm=0{,}2$; 
$p(\xi=\;\mbox{\textit{образец}})\hm=0{,}4$; 
$p(\xi=\;\mbox{\textit{экспонат}})\hm=0{,}6$. Дискретные значения 
вероятностей могут суммироваться, и поэтому вмес\-то интегралов, 
присутствующих в~(\ref{e3-k}), будут исполь-\linebreak зоваться суммы по всем 
возможным значениям\linebreak случайных величин. Количество взаимной 
информации в дискретном случае будет определяться следующим образом:
   \begin{equation}
   I_{\xi\eta} =\sum\limits_{\xi\in X} \sum\limits_{\eta\in Y} p(\xi,\eta) \log 
\fr{p(\xi,\eta)}{p(\xi)p(\eta)}\,,
   \label{e4-k}
   \end{equation}
где $X = \{x_1, x_2, \ldots , x_n\}$, $Y \hm= \{y_1, y_2, \ldots , y_m\}$~--- 
множества значений случайных величин~$\xi$ и~$\eta$, 
   $p(\xi)$, $p(\eta)$ и $p(\xi,\eta)$~--- распределения их вероятностей. 
   
   Данная статья посвящена изложению общей концепции 
автоматизированной технологии оценки степени близости ИО, 
представленных на естественном языке. Поэтому здесь ограничимся этим 
кратким представлением основного существа теории информации и ее 
дальнейшее использование объясним <<на словах>>. Достаточно полные и 
строгие сведения по энтропии и взаимной информации интересующийся 
читатель сможет найти в оригинальной литературе, например~[3--5], а их 
применение в далекой от передачи информации области управления 
качеством и технологиями~--- в работах автора~\cite{6-k, 7-k} и~др.

\section{Понятие вероятностной модели}

   Структуризация текста с целью извлечения заключенного в нем смысла 
представляется не вполне определенной задачей. Неопределенность следует 
из сложности представления ее содержательного существа, откуда вытекают 
и проблемы с определением методов решения. При оценке степени подобия 
содержания двух ИО смысл каждого из них, вообще 
говоря, интереса не представляет, так как целью является не выяснение 
семантики, а оценка степени их содержательного подобия. Для этого 
необходимо оценить меру совпадения в текстах того, о чем (ком) идет речь, 
что, как, где, когда и~т.\,п.\ с ними происходит или они делают. 
   
   Синтез формального подхода к оценке бли\-зости ИО, представленных на 
естественном языке, тре\-бу\-ет формального пред\-став\-ления самих 
срав\-ни\-ва\-емых объектов, т.\,е.\ разработки модели представления ИО. 
Математические модели объектов\linebreak являются основой для разработки систем 
управления этими объектами, а также решения задач анализа объектов, 
исследования взаимосвязей между компонентами, образующими объект, 
синтеза суж\-дений о состоянии и эволюции объекта. Поэтому структура и 
содержание модели должны разрабатываться с учетом четкого представления 
целей, для достижения которых она будет использоваться. Модель должна 
адекватно отражать все наиболее важные для правильного решения 
поставленной задачи содержательные аспекты объекта и игнорировать те, 
которые, усложняя модель, не способствуют повышению качества решения. 
Модели одного и того же объекта, предназначенные для решения различных 
задач, могут значительно различаться глубиной учета отдельных деталей. 
   
   Применительно к проблеме формального представления 
ИО в задачах оценки их семантической близости 
модель должна обеспечивать возможность сопоставления эквивалентных 
компонентов объектов, отражающих содержание сопоставляемых ИО, и 
игнорировать стилистические тонкости, влия\-ющие на форму представления 
содержания, но не на его смысл. 
   
   Обычно первым шагом при построении модели является структуризация 
объекта, выделение его компонентов, которые в совокупности определяют 
рассматриваемый объект. 

При разработке структуры модели ИО можно было 
бы, следуя имеющимся в литературе примерам, исходить из структуры 
простых предложений, в виде совокупности которых тем или иным способом 
может быть пред\-став\-лен ИО. Простое предложение русская грамматика 
определяет центральной грамматической единицей. <<Это определяется тем, 
что простое предложение представляет собой элементарную 
предназначенную для передачи относительно законченной информации 
единицу\ldots>>~\cite[с.~405]{8-k}. Но далее следуют 154~параграфа, в 
которых излагаются типы и формы простых предложений. Их многообразие 
и присутствие неполной четкости деления по типам и формам делает 
нереальной задачу формального описания даже простых предложений, не 
говоря о более сложных типах предложений. Именно по этой причине 
детерминированный подход, опирающийся на представления русской 
грамматики, как отмечалось выше, представляется малопригодным для 
анализа семантической бли\-зости ИО. 
   
   Вследствие того, что целью разрабатываемой методологии является не 
анализ текстов с позиций грамматики русского языка, а сопоставление их 
семантического содержания, которое может\linebreak случайным образом облекаться в 
лексическую оболочку, к определению структуры модели пред\-став\-ля\-ет\-ся 
целесообразным подойти с вероятностно-ста\-ти\-сти\-че\-ских позиций. 
   
   В теории вероятностей~\cite{9-k} существует вероятностная модель, 
которая позволяет дать формальное, максимально полное описание 
   ве\-ро\-ят\-но\-ст\-но-ста\-ти\-сти\-че\-ско\-го объекта. Она определяется 
на множестве элементарных событий $\{\omega_1, \omega_2, \ldots , 
\omega_n\}$, которое образует пространство элементарных событий, или 
исходов $\Omega =\{\omega_1, \omega_2, \ldots , \omega_n\}$. Известны 
вероятности элементарных событий $p(\omega_i)$, $i=1, 2, \ldots , n$. На 
множестве элементарных событий задается алгебра $\aleph=(A_j \vert 
A_j\subseteq\Omega$) или, иначе, система случайных событий, составленных 
каким-ли\-бо определенным образом из элементарных событий $\omega_i 
\in\Omega$. Для каждого из случайных событий $A_j=\{\omega_i\in \Omega\}$, 
образующих алгебру, по вероятностям элементарных исходов $p(\omega_i)$, 
$\omega_i\in A_j$, определяется его вероятность $P(A_j)$. 
   
   Набор: множество элементарных событий $\Omega \hm=\{\omega_1, \ldots , 
\omega_n\}$, система случайных событий (ал\-геб\-ра) $\aleph=(A_j\vert  
A_j\subseteq\Omega$) и вероятности случайных событий $P(A_j)$~--- образует 
вероятностную модель случайного объекта. Она содержит всю информацию, 
которой может быть охарактеризован случайный объект. Формально 
вероятностная модель (или вероятностное пространство эксперимента с 
конечным пространством исходов~$\Omega$ и алгеброй событий~$\aleph$) 
может быть представлена в виде:
   \begin{equation}
   M_\Omega =\{ \Omega, \aleph, P(A)\}\,,
   \label{e5-k}
   \end{equation}
где $\Omega= \{\omega_1, \omega_2, \ldots , \omega_n\}$, $\aleph= (A_j\vert A_j 
\subseteq \Omega)$, $P(A) \hm= (P(A_j)\vert A_j\in \aleph)$.

\begin{table*}[b]\small
\begin{center}
\Caption{Представление ВСММ ИО (фрагмент)}
\vspace*{2ex}

\tabcolsep=4.5pt
\begin{tabular}{|c|c|c|c|c|c|c|c|c|}
\hline
\multicolumn{2}{|c|}{Существительные}&\multicolumn{2}{c|}{Прилагательные}&
\multicolumn{2}{c|}{Числительные}&\ldots&\multicolumn{2}{c|}{Глаголы}\\
\hline
Слова&Характеристики&Слова&Характеристики&Слова&Характеристики&\ldots&Слова&Характеристики\\
\hline
1 Дом&$p$(дом)&Серый&$p$(сер.)&Три&$p$ (три)&\ldots&Стоит&$P$(стоит)\\
2 Стол&$p$(стол)&Белый&$p$ (бел.)&Два&$p$ (два)&\ldots&Идет&$P$ (идет)\\
\ldots&\ldots&\ldots&\ldots&\ldots&\ldots&\ldots&\ldots&\ldots\\
\hline
\end{tabular}
\end{center}
\end{table*}
   
   Определить вероятностную модель конкретного случайного объекта 
значит определить все ее элементы~--- множество элементарных исходов, 
сис\-те\-му случайных событий и их вероятности~--- для этого конкретного 
объекта.

\section{Вероятностно-статистическая морфологическая модель 
информационного объекта}
   
   Информационный объект может быть пред\-став\-лен в виде вероятностной 
модели. В нем множество элементарных исходов $\Omega = \{\w_1, \w_2, 
\ldots , \w_n\}$ представляют слова $\w_i$, $i=1, 2, \ldots , n$, со\-став\-ля\-ющие 
текст ИО. Существуют системы структуризации лексического материала. 
Достаточно общими и пригодными для использования при разработке 
ве\-ро\-ят\-но\-ст\-но-ста\-ти\-сти\-че\-ской модели ИО являются синтаксическая и 
морфологическая структуризации русского языка. Морфологическая 
структуризация задается определением частей речи русского языка, которые 
разделяют язык на самые крупные грамматические классы слов~\cite{8-k}. 
Различают десять частей речи, среди которых шесть знаменательных: 
существительные, прилагательные, чис\-ли\-тель\-ные, 
   мес\-то\-име\-ния-су\-ще\-ст\-ви\-тель\-ные, наречия, глаголы и три 
служебные: предлоги, союзы, частицы. Десятой частью являются 
междометия. Части речи, к которым относятся отдельные слова, могут 
трактоваться случайными событиями~$A_j$, $j = 1, 2, \ldots , 10$. Каждое 
отдельное слово (реализация, элементарный исход) $\omega_i$, $i=1, 2, \ldots 
, n$, входит в текст с определенной вероятностью~$p(\omega_i)$. В~тексте 
роль вероятности играет относительная частота $p(\omega_ii)=n_i/n$, где 
$n_i$~--- число употреблений в ИО слова~$i$, $n$~--- общее количество слов 
в ИО. Относительная частота получается экспериментально и называется в 
теории вероятностей эмпирической вероятностью. По вероятностям 
$p(\omega_i)$ отдельных слов вычисляются вероятности событий~$A_j$~--- 
час\-тей речи. Вероятностно-статистическая модель ИО~(\ref{e5-k}), в которой 
алгебра (способ структуризации) слов определяется морфологией, может 
быть названа вероятностно-статистической морфологической моделью 
(ВСММ) ИО, которая может быть по аналогии с~(\ref{e5-k}) записана в виде: 
   \begin{equation}
   M_M =\{\Omega, \aleph_M, P(A)\}\,,
   \label{e6-k}
   \end{equation}
где индекс $M$ подчеркивает морфологический характер модели, который 
отражается через определение алгебры~$\aleph_M$.
   
   Конкретный ИО представляется в виде соответствующего 
   ве\-ро\-ят\-но\-ст\-но-ста\-ти\-сти\-че\-ско\-го морфологического образа 
ИО. Он синтезируется на основа\-нии модели~(\ref{e6-k}) введением 
конкретного множества элементарных исходов $\W_O= (\w_1, \w_2, \ldots$\linebreak $\ldots , 
\w_n)$ -- слов. Обозначение $\W_O$ подчеркивает, что это множество слов 
конкретного ИО. Множество структурируется в соответствии с введенной 
ал\-геб\-рой $\aleph_M=(A_1, A_2, \ldots , A_J)$, где $J$~--- число случайных 
событий (частей речи), используемых в образе, $J\leq  10$, т.\,е.\ некоторые 
части речи, например междометия, предлоги, могут не использоваться при 
формировании образа. В~результате пол\-ностью определяется 
ве\-ро\-ят\-но\-ст\-но-ста\-ти\-сти\-че\-ский морфологический образ (ВСМО) ИО ВСММ~(\ref{e5-k}) 
конкретного ИО, который может быть представлен в виде:
   \begin{equation}
   O_M=\{ \W_O,\aleph_M,P_O(A)\}\,,
   \label{e7-k}
   \end{equation}
где обозначения следуют из~(\ref{e5-k}), (\ref{e6-k}) и текста. 
   
   По множеству элементарных исходов $\W_O$ вычисляются 
количественные характеристики образа: вероятности $p(\w_i)$ и вероятности 
случайных событий $P(A_j)$. По вероятностям может быть в соответствии 
с~(\ref{e2-k}) определена энтропия $H_O$ образа ИО, характеризующая 
количество информации в об\-разе.
   
   Для пояснения, возможно, непривычного для исследований в области 
русского языка подхода и терминологии воспользуемся наглядной 
иллюстрацией. Вероятностно-ста\-ти\-сти\-че\-ская морфологическая
модель ИО может быть представлена в виде табл.~1.
   
   В шапке таблицы для сокращения размеров примера указаны в явном 
виде только 4~части речи из десяти, имеющихся в языке. В~модели будет 
использоваться таблица с полным набором частей речи. Шапка таблицы 
является атрибутом модели. Она отражает структуру ВСММ и является 
общей для представления образов всех ИО. Части речи, указанные в шапке, 
трактуются случайными величинами. В~вероятностном смысле шапка 
таблицы содержит все рассматриваемые при морфологическом подходе 
случайные события или полное поле событий. Смысл полного поля событий 
в данном контексте в том, что любое встреченное в тексте (в ИО) слово 
относится к одному из них (является ка\-кой-либо частью речи). 
   
   Все то, что находится в таблице под шапкой, отражает конкретный ИО, 
т.\,е.\ определяет ВСМО ИО~(\ref{e7-k}). Слова \textit{дом}, \textit{стол} 
являются в примере случайными значениями, которые приняла часть речи 
<<существительное>>, аналогично \textit{серый}, \textit{белый}~--- 
случайные значения <<прилагательного>>, \textit{стоит}, \textit{идет}~--- 
<<глагола>>. Кроме собственно значений, которые принимают части речи в 
ИО, в модели отражаются их случайные характеристики, например 
относительные частоты.
   
   Вероятностно-ста\-ти\-сти\-че\-ский морфологический
образ~(\ref{e7-k}) может быть сформирован для любого произвольного 
ИО, представленного на русском языке, да и не на 
русском тоже. При его формировании могут быть использованы имеющиеся 
достаточно эффективные инструменты морфологического анализа текстов, 
которые позволяют автоматизировать процедуры отнесения слов к частям 
речи. 
   
   При решении задачи оценки семантической близости 
ИО ВСМО синтезируется для обоих сравниваемых объектов. Для 
определенности один из них называется эталоном (ИОЭ), его 
морфологический образ обозначается $O_{\mathrm{МЭ}}$, а второй~--- дублем 
(ИОД), его образ~--- $O_{\mathrm{МД}}$. Ве\-ро\-ят\-но\-ст\-но-ста\-ти\-сти\-че\-ский 
морфологический образ содержит все слова ИО, 
структурированные по частям речи, ве\-ро\-ят\-ности отдельных слов и частей 
речи в ИО. Ве\-ро\-ят\-ности количественно характеризуют ВСМО ИОЭ и ВСМО 
ИОД. На их основе может быть осуществлена оценка количества 
информации в каждом из объектов и оценено количество взаимной 
информации~(\ref{e4-k}) в объектах. Выражение для оценки количества 
взаимной информации~(\ref{e4-k}) может быть приведено к виду:
   \begin{equation}
   I_{\mathrm{МЭД}} = H(O_{\mathrm{МЭ}}) + H(O_{\mathrm{МД}})- 
H(O_{\mathrm{МЭ}}, O_{\mathrm{МД}})\,,
   \label{e8-k}
   \end{equation}
где обозначения совпадают с введенными раньше. 
   
   Можно утверждать, что полное совпадение ВСМО объектов будет иметь 
место при полной идентичности ИОД и ИОЭ. Наличие отклонений ВСМО 
дубля от ВСМО эталона будет указывать на несовпадение содержания ИО, 
количественной оценкой которого является количество совместной 
информации. 
   
   Формирование ВСМО объектов может опираться на имеющиеся 
фундаментальные исследования в области морфологии русского языка и 
разработки\linebreak мощных инструментов морфологической структуризации. 
В~частности, выполнение морфологической структуризации в данном 
исследовании опирает\-ся на электронную версию словаря 
А.\,А.~Зализняка~\cite{10-k}, для практического использования которой 
разработаны оригинальные программные продукты. 

\section{Вероятностно-статистическая синтаксическая модель 
информационного объекта}
   
   С другой стороны, в русском языке классифицированы члены 
предложения. Определение членов предложения задает синтаксическую 
структуру русского языка. Важно подчеркнуть, что части речи обладают 
общностью синтаксических функций, так что эти два способа 
структуризации лексического состава русского языка взаимосвязаны и 
дополняют друг друга. Вероятностно-статистическая модель, синтезируемая 
на синтаксической основе, будет отличаться только системой случайных 
событий, образующих полное поле событий, т.\,е.\ шапкой таблицы, 
отражающей модель. 
   
   Между морфологической и синтаксической структуризацией имеется 
значительное отличие, сле\-ду\-ющее из того, что морфологическая 
структури\-за\-ция является фиксированной, так как имеется всего 10~час\-тей 
речи. Синтаксическая структуризация допускает введение более детальной 
структуры членов предложения. Она определяется разработчиком системы 
сравнения объектов и допускает определенный волюнтаризм в выборе 
системы случайных событий. Для возможности отражения семантических 
оттенков в систему случайных событий могут быть введены разнообразные 
синтаксические конструкции, связанные со спецификой содержания 
сравниваемых ИО. Понятно, что увеличение отражаемого в модели 
разнообразия конструкций, с одной стороны, будет способствовать 
повышению качества сравнения текстов, а с другой~--- усложнению модели. 
Однако если учесть табличное представление модели, стандартное для 
реляционных баз данных, то увеличение таблиц не приведет к 
принципиальным затруднениям при реализации систем оценки близости ИО. 
   
   Вероятностно-статистическая синтаксическая модель (ВССМ) ИО 
аналогична ВСММ ИО и может быть представлена в виде таблицы, подобной 
табл.~1. Шапка таблицы будет отражать принцип 
структурирования по случайным событиям, которые в ней связываются уже с 
членами предложения, т.\,е.\ с синтаксическими конструкциями языка. 
Алгебра вероятностной модели в этом случае будет определяться типами 
членов предложения, которые используются для представления ИО: 
$\aleph_C \hm= \{B_1, B_2, \ldots , B_L)$, где $B_1$~--- подлежащее; $B_2$~--- 
сказуемое; $B_3$~--- определение и~т.\,д.;
   $L$~--- общее число типов членов предложения, используемых в 
синтаксической модели и образующих в ней полное поле событий; 
   $B_l$, $l \hm= 1, 2, \ldots , L$, как и $A_j$, трактуются как случайные 
синтаксические события. Таким образом, ВССМ будет иметь вид: 
\begin{equation}
M_C = \{\Omega, \aleph_C,P(B)\}\,,
\label{e9-k}
\end{equation}
отличающийся от~(\ref{e6-k}) только алгеброй~$\aleph_C$.

Исследуемый ИО посредством какого-либо синтаксического анализатора 
разделяется на члены предложения. На основе этого разделения 
формируются случайные события и синтезируется вероятностно-статистический 
синтаксический образ (ВССО) ИО:
\begin{equation}
O_C=\{\W_O,\aleph_C,P_O(B)\}\,.
\label{e10-k}
\end{equation}
   
   Для оценки семантической близости ИО ВССО 
синтезируется для обоих сравниваемых объектов. На их основе может быть 
оценено количество взаимной информации в сравниваемых 
объектах~(\ref{e4-k}). Выражение для оценки количества взаимной 
информации получается из~(\ref{e8-k}) заменой морфологических образов 
син\-так\-си\-че\-скими:
   \begin{equation}
I_{\mathrm{СЭД}} = H(O_{\mathrm{СЭ}}) + H(O_{\mathrm{СД}})- 
H(O_{\mathrm{СЭ}}, O_{\mathrm{СД}})\,,
\label{e11-k}
\end{equation}
где обозначения совпадают с введенными раньше. 
   
   Синтаксический анализ текста представляет отдельную проблему, 
отличающуюся от рассматриваемой здесь. Поэтому для определения 
инструментов формирования синтаксических образов были 
проанализированы имеющиеся в литературе наработки в этом направлении и 
практически использовался <<Синтаксический анализатор Cognitive 
Dwarf 2.0>>~[11--13].

\vspace*{-0.9pt}

\section{Методология оценки семантической близости информационных объектов}
   
   В морфологической и в синтаксической структуре структурные 
компоненты несут достаточно определенную и близкую семантическую 
нагрузку. Поэтому могут быть установлены отношения эквивалентности 
между компонентами двух структур. Вследствие того, что эти структуры 
охватывают весь лексический состав и грамматический строй русского 
языка, они обеспечивают отражение семантического содержания текстов и, 
следовательно, являются достаточными для сопоставления этого 
семантического содержания. 
   
   Таким образом, ИО может быть представлен в виде 
морфологического\addtolength{\footnotesep}{1.351pt}\footnote{Минимальное количество грамматических терминов, 
используемых в статье, заимствовано из~\cite{8-k} с единственной целью: приблизить 
изложение терминологически к области русского языка, хотя содержание статьи, как 
представляется, достаточно далеко от вопросов собственно языка.}\addtolength{\footnotesep}{-1.351pt} и/или 
синтаксического образа. Формальное представление ИО в виде 
математических объектов~--- ве\-ро\-ят\-но\-ст\-но-ста\-ти\-сти\-че\-ских 
образов~--- позволяет использовать математический аппарат для получения 
количественных оценок их близости. Можно утверждать, что полное 
совпадение как ВСМО, так и ВССО со\-по\-став\-ля\-емых объектов будет 
соответствовать равенству представляемых ими ИО. 

Оценка степени близости ИО, представленных на естественном языке, может 
быть реализована на основе применения вероятностно-статистической 
морфологической и/или синтаксической модели. 
   
   Мерой степени близости служит энтропия и взаимная информация, 
количественные значения которых вычисляются по~(\ref{e2-k}), (\ref{e8-k}) 
и~(\ref{e11-k}). Выше отмечалось, что под количеством информации в теории
информации понимается количество неопределенности случайного объекта, 
которое исчезает при выяснении этой неопределенности. Неопределенность 
объекта характеризуется распределением его вероятностей. 
В~использованном выше примере объекта~$\xi$ с синонимами было задано 
распределение вероятностей: $p(\xi=\;\mbox{\textit{пример}}) \hm= 0{,}2$; 
$p(\xi\hm=\;\mbox{\textit{образец}}) \hm= 0{,}4$; $p(\xi\hm=\;\mbox{\textit{экспонат}}) 
\hm= 0{,}6$. Так что в этом случае энтропия отдельных исходов будет: 
$H(\xi=\;\mbox{\textit{пример}}) \hm=- \log 0{,}2$, 
$H(\xi=\;\mbox{\textit{образец}}) \hm=- \log 0{,}4$, а 
$H(\xi=\;\mbox{\textit{экспонат}}) \hm=- \log 0{,}6$, а\linebreak усредненная энтропия 
$H_\xi \hm=- 0{,}2 \log 0{,}2\hm- 0{,}4 \log 0{,}4\hm - 0{,}6 \log 0{,}6$. Таким образом, 
энтропия будет некоторым числом, зависящим от распределения 
вероятностей случайных событий, но не от их содержания. 
   
   Из выражений~(\ref{e8-k}) и (\ref{e11-k}) для взаимной информации 
можно видеть, что, во-пер\-вых, она тоже является числом, так как 
выражается через числа~--- значения соответствующих энтропий. 
   Во-вто\-рых, выражения~(\ref{e8-k}) и~(\ref{e11-k}) отражают смысл 
взаимной информации как меры неопределенности. 

Пусть используются 
синтаксические образы со\-по\-став\-ля\-емых объектов, а их взаимная информация 
оценивается выражением~(\ref{e11-k}). Рассмотрим два предельных случая. 
   
   В первом пусть ВССО эталона $O_{\mathrm{СЭ}}$ не имеет ничего общего 
с ИССО дубля~$O_{\mathrm{СД}}$. Отсутствие общего означает, что 
вероятность совместного распределения $P(O_{\mathrm{СЭ}},O_{\mathrm{СД}}) 
\hm= 0$. В~теории информации принято считать $0 \log 0=0$, поэтому 
$H(O_{\mathrm{СЭ}}, O_{\mathrm{СД}}) \hm= 0$
 и из~(\ref{e11-k}) следует, что 
количество совместной информации, содержащееся в двух со\-по\-став\-ля\-емых 
объектах, равно их общей неопределенности: $I_{\mathrm{СЭД}} \hm= 
H(O_{\mathrm{СЭ}}) \hm+ H(O_{\mathrm{СД}})$.
   
   Во втором случае пусть дубль полностью совпадает с эталоном, т.\,е.\ 
$O_{\mathrm{СЭ}}\hm=O_{\mathrm{СД}}$. Тогда совпадут энтропии 
$H(O_{\mathrm{СЭ}}) \hm= H(O_{\mathrm{СД}})$, более того, и совместная 
энтропия $H(O_{\mathrm{СЭ}}, O_{\mathrm{СД}})$ будет равна энтропии 
эталона или дубля. Так что количество совместной информации будет равно 
$I_{\mathrm{СЭД}} = H(O_{\mathrm{СЭ}})$, т.\,е.\ неопределенность дубля 
отсутствует, вся неопределенность связана только с неопределенностью 
эталона, только с количеством заключенной в нем информации. Отсюда 
можно заключить, что количество информации~(\ref{e11-k}) изменяется от 
значения\linebreak
$I_{\mathrm{СЭД}} \hm= H(O_{\mathrm{СЭ}})$ до значения 
$I_{\mathrm{СЭД}}\hm = H(O_{\mathrm{СЭ}}) \hm+ H(O_{\mathrm{СД}})$. При 
этом взаимная информация~---\linebreak
 величина положительная. Это следует из 
положительности энтропий: 
$
p(\xi) \leq 1$, $\log p(\xi)\hm\leq  0
$ и, следовательно, 
$$
H(\xi) =- \log p(\xi) \geq 0
$$ 
и факта 
$$
H(O+{\mathrm{СЭ}})  + H(O_{\mathrm{СД}})  \geq H(O_{\mathrm{СЭ}}, O_{\mathrm{СД}})\,.
$$ 

Разумеется, такой же результат может быть получен и для~(\ref{e8-k}). 
Вследствие свойств логарифмической функции количество информации 
изменяется монотонно в определенных выше пределах, что и позволяет 
использовать его в качестве меры семантической близости ИО. 
   
   Для практического применения абстрактные значения энтропии и 
взаимной информации необходимо проградуировать в некоторых понятных и 
связанных с содержательным существом задачи мерах оценки семантической 
близости ИО. 
%
Такая градуировка (тарирование) их значений может 
осуществляться разными способами и, в частности, обеспечивать реализацию 
функций адаптации сис\-те\-мы к различным задачам и типам ИО. Например, в 
простейшем случае может быть взят реальный эталонный ИО такого типа, 
для работы с которым предполагается использовать систему. 
%
Искажением 
эталонного ИО случайным образом и в заданных объемах может быть 
получена серия дублей с известной степенью семантического несовпадения. 
Для каждой пары <<эта\-лон--дубль>> находится значение взаимной 
информации, которое сопоставляется с известной степенью семантического 
несовпадения. На основании сопоставления определяется линейное 
преобразование перевода количества информации в удобную меру оценки 
степени семантического соответствия ИО.
   
   Методология реализуется в виде последовательности следующих этапов: 
   \begin{itemize}
\item выбор вида и формирование вероятностно-ста\-ти\-сти\-че\-ской модели 
конкретизацией ал\-геб\-ры (системы случайных событий) $\aleph_M$ или 
$\aleph_C$;
\item введение содержания (текстов) образов ИО;
\item формирование образа эталонного ИО в виде $O_{\mathrm{МЭ}}$ или 
$O_{\mathrm{СЭ}}$;
\item формирование образа второго ИО (дубля) в виде $O_{\mathrm{МД}}$ или 
$O_{\mathrm{СД}}$; 
\item определение энтропии эталонного ИО в виде $H(O_{\mathrm{МЭ}})$ или 
$H(O_{\mathrm{СЭ}})$;
\item определение энтропии второго ИО (дубля) в виде $H(O_{\mathrm{МД}})$ 
или $H(O_{\mathrm{СД}})$;
\item определение совместной энтропии эталонного ИО и дубля в виде 
$H(O_{\mathrm{МЭ}}, O_{\mathrm{МД}})$ или $H(O_{\mathrm{СЭ}}, 
O_{\mathrm{СД}})$; 
\item определение совместной информации в соответствии с~(\ref{e8-k}) 
или/и в соответствии с~(\ref{e11-k});
\item перевод совместной информации в выбранную меру информационной 
близости ИО.
\end{itemize}

   Разработанная методология оценки семантической близости ИО, 
кажущаяся, на первый взгляд, совершенно от семантики оторванной, имеет 
глубокую содержательную основу. Если следовать более общему 
представлению языка, чем детальное грамматическое, то множества 
элементарных исходов (слов), образующих в вероятностно-ста\-ти\-сти\-че\-ских 
образах случайные события (час\-ти речи в ВСМО и члены предложения в 
ВССО)\linebreak могут трактоваться как соответствующие обобщенные час\-ти речи и 
члены предложения, образующие эти образы. Такое представление позволяет 
выделить главное содержание в сравниваемых ИО.\linebreak Содержательным 
примером, подтверждающим реальность разработанного подхода, является 
достаточно час\-то встречающееся в реальности продуктивное общение людей 
на плохо знакомом им \mbox{языке}. Они не владеют склонениями, спряжениями, 
формами времени и другими элементами грамматики, но, зная две--три сотни 
слов, достаточно успешно общаются, вполне понимая друг друга. 
   
   Здесь излагается ядро методологии и не рассматриваются возможности 
привлечения дополнительных инструментов, повышающих адекватность 
оценки, таких как использование синонимов, введение весовых 
коэффициентов, детализация и комбинация событий и~т.\,п. 
   
   Заметим еще, что такой подход может быть использован и для оценки 
близости ИО, реализованных на других языках и с использованием иных 
алфавитов. Другие языки, например английский или немецкий, отличаются 
от русского в сторону уменьшения свободы в порядке слов и разнообразия 
способов управления, что упрощает задачи их структурирования и 
построения вероятностно-ста\-ти\-сти\-че\-ских моделей ИО, не требуя изменения 
методологии. 
   
   В ряде случаев ИО могут быть представлены с использованием не 
естественного языка, а, например, формального математического языка 
формул. В~этом случае изменяется входной алфавит и, возможно, принцип 
синтеза алгебры случайных событий. Но эти изменения не касаются 
представленной здесь собственно методологии оценки подобия ИО. 

\vspace*{-6pt}
   
\section{Оценка знаний}

\vspace*{-2pt}
   
   Одной из проблем, для решения которой предпринята данная разработка, 
является автоматизированный контроль знаний. Использование для этой 
цели системы тестов представляется автору неприемлемым по множеству 
причин. На кафедре АСУ Липецкого государственного технического 
университета разрабатывается <<Автоматизированная система поддержки 
образовательной программы обучения>> (АСПОП)~\cite{14-k}, одним из 
важнейших компонентов которой является подсистема автоматизированного 
контроля знаний. Концепция подсистемы базируется на изложенной 
методологии. 
   
   Практическая проверка в минимально возможном объеме 
работоспособности принципиальных положений концепции осуществлена 
проверкой знаний студентов. При реализации проверки студентам на экране 
демонстрировался эталонный ответ из АСПОП, который они воспроизводили 
на память и заносили в компьютер. По эталонным ответам из АСПОП 
формировались ВСМО $O_{\mathrm{МЭ}}$ и ВССО $O_{\mathrm{СЭ}}$. По 
ответам студентов формировались соответствующие ВСМО 
$O_{\mathrm{МД}}$ и ВССО $O_{\mathrm{СД}}$. По морфологическим образам 
$O_{\mathrm{МЭ}}$ и $O_{\mathrm{МД}}$ определялись энтропии 
$H(O_{\mathrm{МЭ}})$, $H(O_{\mathrm{МД}})$ и $H(O_{\mathrm{МЭ}}, 
O_{\mathrm{МД}})$, а по ним оценивалось количество взаимной информации. 
Также обрабатывались синтаксические образы эталонных ответов и их 
дублей~--- ответов студентов. 
   
   Распечатанные эталонные ответы и ответы студентов анонимно 
сопоставлялись группой преподавателей, которые выставляли оценки 
студентам по существующей методике по 100-балль\-ной шкале. Оценки 
преподавателей надлежащим образом усреднялись. По оценкам 
преподавателей и количествам взаимной информации, определенным 
автоматизированной системой, определялись параметры масштабного 
преобразования количества информации в принятые в университете 
   100-балль\-ные оценки. После введения коэффициентов масштабного 
преобразования система, как и преподаватели, выдавала 100-балль\-ные 
оценки. 
   
   Оценки, автоматически сформированные сис\-те\-мой, были сопоставлены с 
оценками, вы\-став\-лен\-ны\-ми преподавателями. В~итоге было получено, что 
среднее квадратическое отклонение оценок, вычисленных системой на 
основании сопоставлений вероятностно-статистических образов эталона и 
ответа по изложенной методологии, от оценок, выставленных 
преподавателями, по 100-балль\-ной шкале составило 10\%--15\%. 
Результаты со\-по\-став\-ле\-ния ВСМО и ВССО эталона и ответа оказались 
достаточно близкими. Отметим, что это была пробная проверка, 
предпринятая исключительно для обретения уверенности в практической 
эффективности оригинальной концепции.
   
   Углубление и детализация вероятностно-ста\-ти\-сти\-че\-ских моделей ИО на 
естественном языке, их исследование и применение представляют 
неограниченное, научно новое и практически полезное поле деятельности, в 
освоении которого автор может оказать посильную помощь. 

\vspace*{-6pt}
   
\section{Заключение}

\vspace*{-2pt}

   Разработана оригинальная методология оценки степени семантической 
близости инфор\-ма\-ционных объектов. Методология может служить 
   фор\-маль\-но-ма\-те\-ма\-ти\-че\-ской основой в сфере современных 
информационных технологий для решения разнообразных задач сравнения и 
оценки подобия информационных объектов, представленных на 
естественном языке. Методология вводит ве\-ро\-ят\-но\-ст\-но-ста\-ти\-сти\-че\-скую 
модель представления русскоязычного текста и определяет способы 
представления текстов в виде ве\-ро\-ят\-но\-ст\-но-ста\-ти\-сти\-че\-ских 
морфологических и синтаксических образов, которые позволяют оценить 
количественно и объем информации в информационных объектах, и степень 
их семантического совпадения. Экспериментальная прикидочная проверка 
показала эффективность применения методологии для разработки 
автоматизированных систем оценки знаний. Практическое применение 
методологии только в этой сфере может привести к принципиальным 
изменениям в сфере образования. 

\vspace*{-6pt}

{\small\frenchspacing
{%\baselineskip=10.8pt
\addcontentsline{toc}{section}{Литература}
\begin{thebibliography}{99}

\bibitem{1-k}
\Au{Друкер П.}
Посткапиталистическое общество. Новая постиндустриальная волна на Западе: 
Антология~/ Под ред. В.\,Л.~Иноземцева.~--- М.: Academia, 1990. 

\bibitem{2-k}
\Au{Мельчук И.\,А.}
Опыт теории лингвистических моделей <<Смысл\;$\leftrightarrow$\;Текст>>.~--- 
2-е изд.~--- М.: Школа <<Языки русской культуры>>, 1999.

\bibitem{3-k}
\Au{Шеннон К.}
Математическая теория связи. 1948~// Работы по теории информации и 
кибернетике~/ Пер. с англ. под ред. Р.\,Л.~Добрушина и О.\,Б.~Лупанова.~--- 
М.: ИЛ, 1963.

\bibitem{4-k}
\Au{Колмогоров А.\,Н.}
Теория информации и теория алгоритмов.~--- М.: Наука, 1987.

\bibitem{5-k}
\Au{Стратонович Р.\,Л.} Теория информации.~--- М.: Сов. радио, 1975.

\bibitem{6-k}
\Au{Кузнецов Л.\,А.}
Введение в САПР производства проката.~--- М.: Металлургия, 1991.

\bibitem{7-k}
\Au{Kuznetsov L.\,A.}
The entropy and information application to identify fuzzy sets~//  ICSC Symposium 
(International) on Fuzzy Logic Proceedings.~---  
Academic Press, 1995. P.~A109--A111.

\bibitem{8-k}
\Au{Белоусов В.\,Н., Ковтунова И.\,И., Кручинина~И.\,Н. и~др.}
Краткая русская грамматика~/ Под ред. Н.\,Ю.~Шведовой и 
В.\,В.~Лопатина~--- М.: Рус. яз., 1989.

\bibitem{9-k}
\Au{Гнеденко Б.\,В.}
Курс теории вероятностей: Учебник.~--- 9-е изд., испр.~--- М.: ЛКИ, 2007.

\bibitem{10-k}
\Au{Зализняк А.\,А.}
Грамматический словарь русского языка: Словоизменение.~--- 3-е изд., 
стер.~--- М.: Рус. яз.,1987. 880~с.
\bibitem{11-k}
Синтаксический анализатор Cognitive Dwarf 2.0. {\sf 
http://cs.isa.ru:10000/dwarf/d2/dw2.html}.

\bibitem{12-k}
\Au{Антонова А.\,А., Мисюрев~А.\,В.}
Реализация синтаксического разбора для русского и английского языков~// 
Системный анализ и информационные технологии (САИТ 2005): Мат-лы 
I~Междунар.\ конф.~--- Пе\-ре\-славль-За\-лес\-ский, 2005.~--- 
Переславль-Залесский, 2005. С.~245--249.

\bibitem{13-k}
\Au{Антонова А.\,А., Мисюрев А.\,В.}
Синтаксический анализатор для русского и английского языков~// Сб. трудов 
ИСА РАН~/ Под ред. В.\,Л.~Арлазарова и Н.\,Е.~Емельянова.~--- М.: УРСС, 
2007.

\label{end\stat}

\bibitem{14-k}
\Au{Кузнецов Л.\,А., Фарафонов А.\,С., Тищенко~А.\,Д., Капнин~А.\,В.}
Автоматизированная система поддержки образовательной программы 
обучения~// Качество. Инновации. Образование, 2010. №\,9. С.~12--20. 
 \end{thebibliography}
}
}


\end{multicols}           %9
\def\stat{zatsar}

\def\tit{СИСТЕМОТЕХНИЧЕСКИЕ ПОДХОДЫ К~СОЗДАНИЮ\\ 
СИСТЕМЫ ПОДДЕРЖКИ ПРИНЯТИЯ РЕШЕНИЙ\\ НА~ОСНОВЕ 
СИТУАЦИОННОГО АНАЛИЗА}

\def\titkol{Системотехнические подходы к~созданию 
системы поддержки принятия решений на~основе 
ситуационного анализа}

\def\aut{А.\,А.~Зацаринный$^1$, А.\,П.~Сучков$^2$}

\def\autkol{А.\,А.~Зацаринный, А.\,П.~Сучков}

\titel{\tit}{\aut}{\autkol}{\titkol}

\index{Зацаринный А.\,А.}
\index{Сучков А.\,П.}
\index{Zatsarinny A.\,A.}
\index{Suchkov A.\,P.}


%{\renewcommand{\thefootnote}{\fnsymbol{footnote}} \footnotetext[1]
%{Работа выполнена при финансовой поддержке РФФИ (проект 16-37-00485).}}


\renewcommand{\thefootnote}{\arabic{footnote}}
\footnotetext[1]{Институт проблем информатики Федерального исследовательского центра 
<<Информатика и~управ\-ле\-ние>> Российской академии наук, \mbox{AZatsarinny@ipiran.ru}}
\footnotetext[2]{Институт проблем информатики Федерального исследовательского центра 
<<Информатика и~управ\-ле\-ние>> Российской академии наук, \mbox{Asuchkov@ipiran.ru}}

      

\Abst{Обсуждаются вопросы создания сис\-тем поддержки принятия решений 
(СППР) на основе ситуационного анализа текущей и~прогнозируемой обстановки 
в~контролируемом пространстве органа управления. Как правило, такие сис\-те\-мы 
управления в~режиме реального времени опираются на ситуационные центры (СЦ)~--- 
совокупность информационных, программных и~аппаратных средств, а также 
обслуживающего персонала, реализующих информационные технологии по мониторингу 
обстановки, ее ситуационному анализу для выработки решений и~алгоритмов применения 
управляющих воздействий. Рассмотрены содержательные характеристики составляющих 
частей СППР, реализующих полный цикл управления от целеполагания до контроля 
исполнения принимаемых решений. Отмечается, что реализация СППР зависит от уровня 
сис\-те\-мы управ\-ле\-ния~--- стратегического, оперативного, тактического, базового, приводятся 
функциональные особенности и~способы анализа обстановки на различных уровнях 
сис\-те\-мы управ\-ления.}

\KW{ситуационный анализ; сис\-те\-ма поддержки принятия решений; сис\-те\-ма управ\-ле\-ния; 
ситуационный центр}

\DOI{10.14357/19922264160411} 


\vskip 10pt plus 9pt minus 6pt

\thispagestyle{headings}

\begin{multicols}{2}

\label{st\stat}

\section{Введение}

     В Стратегии национальной безопасности Российской Федерации 
(утверждена Указом Президента Российской Федерации от~31~декабря 
2015~г. №\,683)~[1] определено, что информационную основу реализации 
Стратегии составляет федеральная информационная сис\-те\-ма стратегического 
планирования, включающая в~себя информационные ресурсы органов 
государственной власти и~органов местного самоуправления, сис\-те\-мы 
распределенных СЦ и~государственных научных 
организаций. В~рамках такой сис\-те\-мы должна быть реализована поддержка 
управленческих решений в~интересах центральных органов исполнительной 
власти на основе организации взаимодействия региональных 
и~ведомственных СЦ, а~также других информационных 
сис\-тем. Для эффективного решения этой задачи необходимо создание СППР 
в~со\-ста\-ве СЦ и~придания им принципиально новых качеств. 
     
     В связи с~этим целью статьи является обоснование сис\-те\-мо\-тех\-ни\-че\-ских 
и~методических подхо\-дов к~структурному и~функциональному составу\linebreak 
СППР и~ее месту в~составе СЦ, обеспечивающих 
информационно-аналитическую поддержку принятия управленческих 
решений в~рамках государственного управления, стратегического 
планирования и~мониторинга реализации документов стратегического 
планирования в~Российской Феде-\linebreak рации. 

\vspace*{-6pt}
     
\section{Базовые понятия }

\vspace*{-2pt}

    При рассмотрении сис\-тем\-ных и~методических вопросов создания СППР, 
основанных на ситуационном анализе, в~статье используется ряд базовых 
понятий: событие, обстановка, ситуация, угроза, управление, цели 
управления и~др.~[2]. 
    
    \textit{Ситуация} определяется состоянием взаимосвязанных 
\textit{элементов обстановки} в~контролируемом пространстве; изменения 
обстановки определя-\linebreak ются \textit{событиями}, образующими некоторые 
разворачивающиеся во времени наблюдаемые и~ре\-гист\-ри\-ру\-емые потоки. При 
этом под \textit{управлением}\linebreak понимается \textbf{целенаправленное} 
воздействие органа управления на подчиненные ему или взаимодействующие 
элементы обстановки (ресурсы). 
    
    Совокупность ситуаций в~сис\-те\-ме управ\-ле\-ния распадается на текущие, 
прогнозируемые и~целевые ситуации. При этом текущие ситуации являются 
результатом наблюдения и~регистрации событий, прогнозируемые 
определяются методами ситуационного анализа, а целевые отражают 
краткосрочные, среднесрочные и~долгосрочные цели управления. Последнее 
немаловажно, так как зачастую ситуационный анализ понимается как 
обеспечение реакций сис\-те\-мы управ\-ле\-ния на чрезвычайные ситуации после 
того, как они сложились. Однако теория ситуационного подхода 
предполагает учет <<планируемой и~прогнозируемой обстановки>>, 
отражающей стратегические, тактические и~оперативные \textit{цели 
управления}, а~также учет факторов самоорганизации управляющего 
сегмента сис\-те\-мы, определяющих стимулы для достижения этих 
целей~[2,~3]. Под \textit{угрозой} в~процессах управления понимается 
ситуация или совокупность ситуаций, развитие которых противоречит целям 
управления и~отдаляет текущее состояние от целевого.
    
    В конце 1970-х~гг.\ была создана модель сис\-те\-мы управ\-ле\-ния  
<<наблю\-де\-ние--ори\-ен\-ти\-ро\-ва\-ние--ре\-ше\-ние--дей\-ст\-вие>> 
(НОРД) для принятия решений при ведении боевых действий~[4, 5]. 
В~настоящее время эта модель активно используется во многих сис\-те\-мах 
управ\-ле\-ния разных отраслей~[6]. В~рамках ситуационного подхода 
к~управлению предложена модифицированная модель, включающая 
дополнительную стадию управляющего цикла~--- целеполагание~[7].
    
    \textbf{Целеполагание} (стадия~Ц)~--- формализованное представление 
целевых показателей, установление количественных 
и~временн$\acute{\mbox{ы}}$х критериев их достижения.
    
    \textbf{Мониторинг} (стадия~М)~--- это процесс сбора информации об 
окружающей среде в~контролируемом пространстве, включая состояние 
целевых показателей. Стадия М также принимает внутренние инструкции от 
стадии анализа (А), так же как и~поддержку от процессов~Р и~Д. 
    
    \textbf{Анализ} (стадия~А)~--- оценка ситуации (типовая, нетиповая), 
анализ существующего опыта, пополнение опыта, обеспечивает внутреннюю 
поддержку~М (корректировка фильтров).
    
    \textbf{Решение} (стадия~Р)~--- это процесс осуществления выбора 
среди гипотез о состоянии окружающей среды и~возможной реакции на него. 
Процесс~Р руководствуется прямой внутренней связью с~процессом~А 
и~обеспечивает внутреннюю поддержку процесса~М, возможна 
корректировка целевых показателей (стадия~Ц).
    
    \textbf{Действие} (стадия~Д)~--- это процесс выполнения выбранной 
реакции путем взаимодействия с~окружающей средой. Действие принимает 
внутренние руководства от процесса~А, также оно напрямую связано с~Р. 
Оно обеспечивает внутреннюю поддержку~Ц и~М.
    
    Особенности реализации цикла управления в~сис\-те\-ме, реализующей 
процессы стратегического планирования и~управления, заключаются в~том, 
что содержательно стадии~Ц, А и~Р реализуются непосредственно высшими 
органами исполнительной власти. Это означает осуществление сле\-ду\-ющих 
основных функций:
    \begin{itemize}
\item  доведение до подчиненных органов данных целеполагания 
и~стратегического планирования на основе их формализации 
и~регламентации обмена (стадия~Ц);
\item регламентированный сбор данных о состоянии целевых показателей от 
органов испол\-нительной власти и~об обстановке в~конт\-ро\-ли\-ру\-емом 
пространстве по определенному\linebreak регламенту и~в~режиме реального времени 
(стадия~М);
\item обмен аналитическими данными участников\linebreak стратегического 
планирования по целеполаганию, прогнозированию, планированию 
и~программированию~--- федеральных органов исполнительной власти 
(ФОИВ), субъектов Россий\-ской Федерации и~муниципальных образований, 
отраслей экономики и~сфер государственного и~муниципального управления 
(стадия~А);
\item  доведение до подчиненных органов принимаемых решений по 
применению сил и~средств и,~возможно, по корректировке стратегических 
планов с~целью достижения поставленных стратегических целей (стадия~Р) 
и~контроль исполнения решений (стратегических планов) на основе 
докладов (стадия~Д).
    \end{itemize}
    
    На тактическом и~базовом уровнях управления осуществляются,  
во-пер\-вых, реализация функ-\linebreak ций мониторинга контролируемого 
пространства и~организа\-ции учета контролируемых объектов (стадия~М),  
во-вто\-рых, специальный анализ фактографических данных о конкретных 
элементах обстановки, формализованных в~виде семантической сети, 
позволяющий выявлять неочевидные связи между элементами обстановки, 
определять схожие про\-стран\-ст\-вен\-но-со\-бы\-тий\-ные ситуации, выявлять 
ассоциативные связи и~закономерности с~\mbox{целью} поддержки процессов 
принятия решений (стадия~А), в-треть\-их, процессы принятия решений 
по планированию применения сил и~средств на период времени и~по 
складывающейся обстановке в~соответствии с~указаниями вышестоящих 
органов (стадии~Р и~Д).

\begin{figure*}[b] %fig1
\vspace*{1pt}
\begin{center}
\mbox{%
\epsfxsize=160.901mm
\epsfbox{zac-1.eps}
}
\end{center}
\vspace*{-9pt}
\Caption{Обобщенная функциональная структура СЦ}
\end{figure*}
    

\section{Ситуационный центр как составляющая современной системы 
управления}
    
    Определим СЦ сис\-те\-мы управ\-ле\-ния как совокупность 
информационных, программных и~аппаратных средств, а~также 
обслуживающего персонала, реализующих информационные технологии\linebreak по 
мониторингу обстановки, ее ситуационному анализу для выработки решений 
и~алгоритмов применения управляющих воздействий с~\mbox{целью} эффективной 
реализации функций управления и~минимизации ущерба от угроз в~зоне 
ответствен\-ности\linebreak органа управ\-ле\-ния, доведения их до объектов управ\-ле\-ния 
и~контроля исполнения,
    
    По сути дела, СЦ является составной частью сис\-те\-мы 
управ\-ле\-ния, осуществляющей автоматизацию ряда функций всего органа 
управления и~отдельных должностных лиц.
    
    Исходя из накопленного в~Институте проблем информатики РАН опыта 
разработки крупных информационных сис\-тем в~интересах органов 
государственной власти, в~организационной структуре СЦ можно выделить 
четыре основных функциональных сегмента (рис.~1)~\cite{8-zat}:
    \begin{enumerate}[(1)]
\item сегмент руководства (лиц, принимающих решения, ЛПР); 
\item сегмент мониторинга состояния контролируемых объектов 
и~окружающей среды и~сбора информации; 
\item сегмент ситуационного анализа и~сис\-те\-ма\-ти\-за\-ции информации;
\item сегмент администрирования и~эксплуатации.
\end{enumerate}
    При этом СППР базируется на ресурсах всех четырех сегментов. Вместе 
с тем центральным звеном СЦ и~его СППР, обеспечивающим реализацию 
основной функции сис\-те\-мы управ\-ле\-ния по эффективному управлению 
силами и~средствами, является \textit{сегмент ситуационного анализа 
и~сис\-те\-ма\-ти\-за\-ции информации}. Он должен обеспечивать реализацию 
следующих функций:
    \begin{itemize}
\item возможность визуализации результатов анализа обстановки на 
индивидуальных и~коллективных средствах отображения;
\item во взаимодействии с~сегментом мониторинга получение данных 
о~состоянии обстановки от собственных (субъективных и~объективных 
средств наблюдения и~контроля) и~внешних по отношению к~сис\-те\-ме 
источников информации (ведомственных, межведомственных, 
международных, независимых и~др.);
\item извлечение фактов, структуризация и~формализация разнородных 
данных о~значимых событиях в~соответствии с~выбранной информационной 
моделью предметной области;
\item формирование хранилищ ситуационных данных;
\item формирование способов визуализации агрегированных данных 
о~складывающейся обстановке для ЛПР и~оперативного состава;
\item формирование отчетности и~служебной документации;
\item расчет первичных и~интегральных показателей обстановки, а~также 
статистическая оценка характеристик ненаблюдаемых элементов обстановки;
\item решение задач перспективного планирования, контроль исполнения 
решений по планированию;
\item выявление значимых ситуаций, их ранжирование по степени 
важности, видам и~типам, формирование текущего перечня 
аналитических задач по складывающейся обстановке и~по поручениям 
руководства;
\item  выработка вариантов решений по применению управляющих 
воздействий для достижения целевых ситуаций, формирование спо\-собов 
наглядного представления вариантов\linebreak реше\-ния для ЛПР (оперативное 
планирование);
\item прогнозирование развития обстановки и~процесса реализации целей 
сис\-те\-мы управ\-ле\-ния на основе сформированных ситуационных моделей 
и~моделей угроз, в~том числе и~с~учетом применения выработанных 
вариантов решений;
\item обеспечение процессов принятия решений комплексом  
ин\-фор\-ма\-ци\-он\-но-рас\-чет\-ных задач (ИРЗ).
    \end{itemize}
    
    Наряду с~перечисленными в~СППР СЦ реализуются важнейшие функции 
администрирования аналитической под\-сис\-те\-мы~СЦ:
    \begin{itemize}
\item формирование и~корректировка сис\-те\-мы целей управ\-ления;
\item формирование, настройка и~корректировка сис\-те\-мы моделей целей 
управления, обстановки, ситуаций и~угроз;
\item формирование, настройка и~корректировка сис\-те\-мы расчетных 
показателей, характеризующих обстановку и~ее элементы;
\item формирование, настройка и~корректировка сис\-те\-мы критериев, 
пороговых значений, эвристик, параметров расчетных алгоритмов.
\end{itemize}

\section{Целеполагание~--- определение целей системы управления}

    Под \textit{целью ситуационного анализа} предлагается понимать 
поддержку процессов принятия решений для достижения поставленных 
целей путем применения доступных в~сис\-те\-ме управ\-ле\-ния сил и~средств 
(ресурсов).
    
    Целесообразность деятельности сис\-те\-мы управ\-ле\-ния определяется 
иерархической сис\-те\-мой целей\linebreak (подцелей). Для ФОИВ она задается 
законодательно, а также при определении приоритетов в~орга\-низации 
деятельности сис\-те\-мы управ\-ле\-ния первым\linebreak лицом (руководителем). 
Формирование сис\-те\-мы целей сопровождается формированием сис\-те\-мы 
показателей реализации целей (подцелей) и~критериев достижения целей. 
Показатели являются вычисляемыми величинами как функции обстановки 
или экспертно оцениваемые параметры. Критерии достижения обычно 
формулируются как некие пороговые плановые значения на временн$\acute{\mbox{о}}$й 
шкале.
    
    Эффективность сис\-те\-мы управ\-ле\-ния в~каждый момент времени 
определяется, во-пер\-вых, степенью достижения пороговых значений 
планируемых целевых показателей, во-вто\-рых, объемом затрачиваемых 
ресурсов на единицу оптимизируемого целевого показателя.
    
    Цели управления формируются на основании сис\-тем\-но\-го анализа  
нор\-ма\-тив\-но-пра\-во\-вых основ функционирования сис\-те\-мы управ\-ле\-ния. 
Цели управления образуют дерево целей, детализация которого (число 
уровней) определяется воз\-мож\-ностью декомпозиции конкретной цели на 
значимые подцели. Цели и~подцели должны обладать индикаторами 
состояния (как правило, \%) и~весовыми коэффициентами доли подцели 
в~реализации всей цели. Цели могут включать ориентиры развития сис\-те\-мы 
управления, установленные первым лицом.

\begin{figure*} %fig2
\vspace*{1pt}
\begin{center}
\mbox{%
\epsfxsize=165.008mm
\epsfbox{zac-2.eps}
}
\end{center}
\vspace*{-9pt}
\Caption{Обобщенная структура сис\-те\-мы целей}
\end{figure*}
    
    Выбор структуры сис\-те\-мы целей предлагается осуществлять с~учетом 
следующих соображений.
    \begin{enumerate}[1.]
    \item Цели управления сложной управляющей сис\-те\-мой определяются 
нор\-ма\-тив\-но-пра\-во\-вы\-ми документа\-ми, регламентирующими ее 
функционирование, и, как правило, образуют \textbf{иерархическую 
структуру} в~соответствии со структурой направлений деятельности 
(рис.~2).
    \item Ситуационный подход к~управлению предполагает реагирование на 
складывающуюся обстановку в~режиме реального времени. В~силу этого, 
помимо фиксированных целей в~сис\-те\-ме управ\-ле\-ния необходим механизм 
формирования \textbf{динамических целей}, отражающих процесс 
нормализации складывающихся чрезвычайных ситуаций и~присутствующих 
в~сис\-те\-ме целеполагания на период существования ситуации.
    \item В~концепции <<управления по целям>> эффективность 
целеполагания проверяется по критериям SMART~\cite{9-zat}: цель должна 
быть конкретная, измеримая (подразумевает количественную измеримость 
результата), достижимая (должна быть выполнимой), реалистичная 
(достижение цели должно быть обеспечено ресурсами), привязанная  
к~точ\-ке/ин\-тер\-ва\-лу времени.
    \end{enumerate}
    
    Данный подход накладывает \textbf{требования на атрибуты целей} 
в~части формирования количественных характеристик их достижения, 
плановых характеристик, критериев достижения (см.\ рис.~2). 
    


    Основные атрибуты цели:\\[-14pt]
    \begin{itemize}
\item описание~--- дает определение и~конкретизацию цели;\\[-14pt]
\item весовой коэффициент~--- определяет вклад подцели 
в~вышестоящую цель;\\[-14pt]
\item индикатор~--- задает количественный показатель достижения 
результата;\\[-14pt]
\item критерий~--- задает способ определения достижения результата 
с~помощью индикатора;\\[-14pt]
\item план~--- определяет количественные значения критерия 
достижения цели и~требуемые вре\-мен\-н$\acute{\mbox{ы}}$е параметры.
\end{itemize}

\vspace*{-9pt}

\section{Анализ обстановки и~выработка вариантов решений}

\vspace*{-2pt}

\subsection{Мониторинг обстановки}

\vspace*{-1pt}

     В процессе мониторинга контролируемых элементов обстановки 
осуществляются (рис.~3):
     \begin{itemize}
\item сбор данных о состоянии контролируемых объектов, анализ 
неструктурированной информации с~целью извлечения фактов и~знаний; 
\item постановка объектов на контроль (оператор, автоматически); 
\item отображение контролируемых объектов по шкале состояний и~по 
критериям~--- соотношение текущего или прогнозируемого значения 
индикатора (интегрального показателя) и~сис\-те\-мы порогов, обеспечивающих 
градацию состояния (<<типовое>>, <<чрезвычайное>>, <<критическое>> 
или другие подобные).
\end{itemize}

    По данным мониторинга контролируемых элементов обстановки из 
различных источников формируется \textit{хранилище} СППР, которое 
пред\-став\-ляет собой совокупность взаимоувязанных на\linebreak основе единого 
информационного и~лингвистического обеспечения баз данных (БД): 
обстановки (события, ситуации, элементы окружающей\linebreak среды), сил и~средств 
(свои силы и~средства, противодействующие силы и~средства, так\-ти\-ко-тех\-ни\-че\-ские
характеристики), целевых 
показателей (первичные показатели, интегральные показатели,\linebreak индикаторы, 
критерии), типовых решений (типовые решения, конкретные решения), 
ретроспективная (нормализованные исторические данные, архив 
обстановки), нормативных документов, биб\-лио\-те\-ка математических моделей.

\vspace*{-6pt}

\subsection{Поддержка процесса принятия решений}

\vspace*{-2pt}

    На основе мониторинга текущей обстановки и~поступления событийной 
информации в~хранилище осуществляется расчет заданных в~сис\-те\-ме 
первичных и~интегральных показателей обстановки и~целевых показателей 
в~двух режимах: по регламенту (с~определенной периодичностью) и~по 
запросу пользователя с~использованием блоков расчетов, блока первичного, 
краткосрочного, среднесрочного и~долгосрочного анализа, блока 
визуализации  и~блока поддержки принятия решений (рис.~4).\linebreak\vspace*{-12pt}


\pagebreak

\end{multicols}
\begin{figure*} %fig3
\vspace*{1pt}
\begin{center}
\mbox{%
\epsfxsize=157.334mm
\epsfbox{zac-3.eps}
}
\end{center}
\vspace*{-6pt}
\Caption{Мониторинг обстановки}
\vspace*{6pt}
\end{figure*}

\begin{multicols}{2}




    
    При этом реализуются следующие функции.
    \begin{enumerate}[1.]
\item  Создание (привязка существующих) динамических моделей 
обстановки:
\begin{itemize}
\item моделей <<нормальной>> обстановки;
\item моделей для прогноза обстановки;
\item моделей для анализа трендов, циклов, аномалий обстановки.
\end{itemize}

    \item Проведение оперативного анализа текущей обстановки 
с~использованием математических методов (см.\ рис.~4):
\begin{itemize}
\item анализ отклонения от <<нормальной>> текущей обстановки;
\item прогноз развития обстановки;
\item анализ трендов, циклов, аномалий обстановки;
\item выявление и~идентификация значимых ситуаций 
на основе выявления типовых кон-\linebreak\vspace*{-12pt}

\columnbreak

\noindent
фигураций событий 
и~правил идентификации, идентификация типа ситуации, 
фор\-ми\-ро\-ва\-ние неотложных целей.\\[-7.5pt]
\end{itemize}
    \item Визуализация и~индикация состояний контролируемых объектов 
    с~использованием полученных результатов анализа (наглядное пред\-став\-ле\-ние 
текущей с~индикацией ситуаций,\linebreak требующих принятия решения или 
применения типовых решений).\\[-6pt]
    \item Выработка вариантов решений по складыва\-ющейся обстановке 
(решение содержит динамическую цель, перечень подцелей (с~весами~--- 
доли подцели в~реализации всей цели),\linebreak сроки достижения подцелей, 
ответственных, совокупность типовых уведомлений и~рапортов):
\begin{itemize}
\item применение типовых решений по типовым ситуациям (привязка их 
к~реальной обстановке);
\end{itemize}
\end{enumerate}



\pagebreak

\end{multicols}

\begin{figure*} %fig4
\vspace*{1pt}
\begin{center}
\mbox{%
\epsfxsize=164.07mm
\epsfbox{zac-4.eps}
}
\end{center}
\vspace*{-11pt}
\Caption{Структура блока принятия решений}
\vspace*{-3pt}
\end{figure*}

\begin{multicols}{2}

\noindent
\begin{enumerate}
\item[\ ]
\vspace*{-13pt}
\begin{itemize}
\item выработка вариантов решения экспертным путем в~случае критических 
и чрезвычайных ситуаций;\\[-15pt]
\item анализ развития обстановки с~учетом вариантов решений (прогноз 
благоприятного и~неблагоприятного развития обстановки, расчет 
вероятностей выполнения задач, оценка вариантов решений).
\end{itemize}
\end{enumerate}

\vspace*{-6pt}

    \subsection{Реализация решений }
    
    \vspace*{-2pt}
    
    На данной стадии осуществляется мониторинг процессов реализации 
решений по краткосрочным, среднесрочным и~долгосрочным планам 
(решение содержит цель, перечень подцелей (с~весами~--- доли подцели 
в~реализации всей цели), сроки достижения подцелей, ответственных, виды 
отчетности):
\begin{itemize}
\item сбор информации по ходу выполнения плана (отчетность), 
визуализация хода исполнения, контроль исполнения;\\[-15pt]
\item сравнительный анализ показателей плана по целям и~подцелям 
и~текущей обстановки, включая расчет степени реализации плана 
и~прогнозирование возможности реализации плана;\\[-15pt]
\item реализация обратной связи по уточнению решения по планированию 
с~целью обеспечения выполнения плана;\\[-15pt]
\item доведение уточненного решения (уведомления) и~контроль исполнения.
\end{itemize}

    Мониторинг реализации решений по ситуациям (решение содержит 
динамическую цель, перечень подцелей (с~весами~--- доли подцели 
в~реализации всей цели), сроки достижения подцелей, ответственных, 
совокупность типовых уведомлений и~рапортов): 
    \begin{itemize}
\item сбор информации по ходу выполнения решения (рапорты), 
визуализация хода исполнения, контроль исполнения;
\item сравнительный анализ показателей по целям и~подцелям и~текущей 
обстановки, включая расчет степени реализации решения и~прогнозирование 
возможности реализации решения;
\item реализация обратной связи по уточнению решения по ситуации с~целью 
обеспечения выполнения плана.
\item доведение уточненного решения (уведомления) и~контроль исполнения.
\end{itemize}

\vspace*{-6pt}

\section{Заключение}

\noindent
\begin{enumerate}[1.]
\item В современных условиях развития информационных сис\-тем особую 
значимость приобретает актуальность исследования сис\-те\-мо\-тех\-ни\-че\-ских 
и~технологических вопросов создания в~составе СЦ
СППР.
\item Важнейшей методологической и~концептуальной основой СППР 
является полнофункциональный цикл управления, включающий стадии 
целеполагания, мониторинга обстановки, анализа обстановки, выработки 
вариантов решений и~их реализации.
\item В СППР реализуются следующие функциональные задачи:
\begin{itemize}
\item мониторинг контролируемых элементов обстановки;
\item расчет характеристик событийной информации (первичные 
и~интегральные показатели текущей обстановки и~состояния 
целей);
\item визуализация текущего состояния обстановки;
\item визуализация текущего состояния индикаторов целей;
\item блок анализа и~принятия решений.
\item мониторинг контролируемых решений;
\item формирование документов и~отчетов.
\end{itemize}
\item Важнейшим сис\-те\-мо\-обра\-зу\-ющим компонентом СППР является 
хранилище, формируемое в~автоматизированном режиме из различных 
источников в~виде совокупности взаимоувязанных на основе единого 
информационного и~лингвистического обеспечения БД (о~событиях, 
силах и~средствах, целевых показателях и~критериях, типовых решений, 
ретроспективной информации, нормативных документов, математических 
моделей).
\item Предложенные в~статье сис\-те\-мо\-тех\-ни\-че\-ские подходы и~решения 
апробированы в~рамках нескольких проектов по созданию крупных 
территориально распределенных  
ин\-фор\-ма\-ци\-он\-но-ана\-ли\-ти\-че\-ских сис\-тем специального 
назначения.
\end{enumerate}

\vspace*{-6pt}

{\small\frenchspacing
 {%\baselineskip=10.8pt
 \addcontentsline{toc}{section}{References}
 \begin{thebibliography}{9}

\bibitem{1-zat}
Стратегия национальной безопасности Российской Федерации. Утверждена Указом 
Президента Российской Федерации от 31~декабря 2015~г. №\,683. 
\bibitem{2-zat}
\Au{Зацаринный А.\,А., Сучков А.\,П.} Некоторые подходы к~ситуационному анализу 
потоков событий~// Открытое образование, 2012. №\,1. С.~39--45.
\bibitem{3-zat}
\Au{Бир С.\,Э.} Мозг фирмы~/
Пер. с~англ.~--- М.: Радио и~связь, 1993. 416~с.
(\Au{Beer~S.}  {Brain of the firm}.~--- Allen Lane, The Penguin Press, London; Herder 
and Herder, USA, 1972. 416~p.)

\bibitem{5-zat}%4
\Au{Grant Т., Kooter В.} Comparing OODA \& other models as operational view~C2 
architecture~// 10th Command and Control Research Technology Symposium (International) 
Proceedings.~--- McLean, VA, USA, 2005.
\bibitem{4-zat} %5
\Au{Ивлев А.\,А.} Основы теории Бойда. Направления развития, применения 
и~реализации.~--- SlideShare, 2008. 64~с. {\sf  
http://www.slideshare.net/defensenetwork/ss-10380168}.
\bibitem{6-zat}
\Au{Босов А.\,В., Зацаринный А.\,А., Сучков~А.\,П.} Некоторые общие подходы 
к~формированию функциональных требований к~ситуационным центрам и~их 
реализации~// Системы и~средства информатики, 2010. Вып.~20. №\,3. С.~98--125.
\bibitem{7-zat}
\Au{Сучков А.\,П.} Формирование сис\-те\-мы целей для ситуационного управ\-ле\-ния~// 
Сис\-те\-мы и~средства информатики, 2013. Т.~23. №\,2. С.~171--182.
\bibitem{8-zat}
\Au{Зацаринный А.\,А., Сучков~А.\,П., Козлов~С.\,В.} Особенности проектирования 
и~функционирования сис\-те\-мы ситуационных центров~// Системы высокой доступности, 
2012. Т.~8. №\,1. С.~12--21.
\bibitem{9-zat}
\Au{Doran G.\,T.} There's a~S.M.A.R.T.\ way to write management's goals and objectives~// 
Manag. Rev., 1981. Vol.~70. Iss.~11. P.~35--36.
 \end{thebibliography}

 }
 }

\end{multicols}

\vspace*{-6pt}

\hfill{\small\textit{Поступила в~редакцию 23.08.16}}

%\vspace*{8pt}

\newpage

\vspace*{-24pt}

%\hrule

%\vspace*{2pt}

%\hrule

%\vspace*{8pt}


\def\tit{SYSTEMS ENGINEERING APPROACHES TO~THE~ESTABLISHMENT 
OF~A~SYSTEM FOR~DECISION SUPPORT BASED ON~SITUATIONAL ANALYSIS}

\def\titkol{Systems engineering approaches to~the~establishment 
of~a~system for~decision support based on~situational analysis}

\def\aut{A.\,A.~Zatsarinny and A.\,P.~Suchkov}

\def\autkol{A.\,A.~Zatsarinny and A.\,P.~Suchkov}

\titel{\tit}{\aut}{\autkol}{\titkol}

\vspace*{-9pt}


\noindent
Institute of Informatics Problems, 
Federal Research Center ``Computer Sciences and Control'' of the 
Russian Academy of Sciences, 44-2~Vavilov Str., Moscow 119333, 
Russian Federation



\def\leftfootline{\small{\textbf{\thepage}
\hfill INFORMATIKA I EE PRIMENENIYA~--- INFORMATICS AND
APPLICATIONS\ \ \ 2016\ \ \ volume~10\ \ \ issue\ 4}
}%
 \def\rightfootline{\small{INFORMATIKA I EE PRIMENENIYA~---
INFORMATICS AND APPLICATIONS\ \ \ 2016\ \ \ volume~10\ \ \ issue\ 4
\hfill \textbf{\thepage}}}

\vspace*{3pt}

 
\Abste{The article discusses the issues of decision-making support systems (DMSS) 
creation based on the situational analysis of the current and projected situation in the 
controlled space. Typically, such control systems in real time are based on situational 
centers, which are sets of information, software, hardware, and staff implementing 
information technology to monitor the situation and its situational analysis to develop 
solutions and algorithms application of control actions. The paper considers 
characteristics of the DMSS components, implementing the full management cycle from 
goal setting to execution control decisions. It is noted that the implementation of the 
decision support system depends on the level of management~--- strategic, operational, tactical, basic, and 
functional features and methods of analysis of the situation at different levels of the 
control system.}

\KWE{situational analysis; system of decision-making process support; management 
system; situational center}

\DOI{10.14357/19922264160411} 

%\vspace*{-9pt}

%\Ack
%\noindent


%\vspace*{3pt}

  \begin{multicols}{2}

\renewcommand{\bibname}{\protect\rmfamily References}
%\renewcommand{\bibname}{\large\protect\rm References}

{\small\frenchspacing
 {%\baselineskip=10.8pt
 \addcontentsline{toc}{section}{References}
 \begin{thebibliography}{9}

\bibitem{1-zat-1}
Strategiya natsional'noy bezopasnosti Rossiyskoy Fe\-de\-ra\-tsii [The National Security Strategy of 
the Russian Federation]. Approved by the Decree of the President of the Russian Federation 
No.\,683, 31.12.2015.
\bibitem{2-zat-1}
\Aue{Zatsarinny, A.\,A., and A.\,P.~Suchkov.} 2012. Nekotorye podkhody k~situatsionnomu 
analizu potokov sobytiy [Some approaches to the situational analysis of the flows of events]. 
\textit{Otkrytoe obrazovanie} [Open Education] 1:39--45.
\bibitem{3-zat-1}
\Aue{Beer, S.} 1972. \textit{Brain of the firm}. Allen Lane, The Penguin Press, London; Herder 
and Herder, USA. 416~p. 

\bibitem{5-zat-1}
\Aue{Grant, Т., and B. Кoote.} 2005. Comparing OODA \& other models as operational view C2 
architecture. \textit{10th Command and Control Research Technology Symposium 
(International) Proceedings}. McLean, VA. USA. 
\bibitem{4-zat-1}
\Aue{Ivlev, A.\,A.} 2008. \textit{Osnovy teorii Boyda. Napravleniya razvitiya, primeneniya 
i~realizatsii} [Fundamentals of the theory of Boyd. Areas of development, application, and 
implementation]. SlideShare. Available at: {\sf http://www.slideshare.net/defensenetwork/ss-10380168} (accessed  October~29, 2016).
\bibitem{6-zat-1}
\Aue{Bosov, A.\,V., A.\,A.~Zatsarinny, A.\,P.~Suchkov}. 2010. Nekotorye obshchie podkhody 
k~formirovaniyu funktsional'nykh trebovaniy k~situatsionnym tsentram i~ikh realizatsii [Some 
common approaches to the formation of functional requirements for situation centers and their 
implementation]. \textit{Sistemy i~Sredstva Informatiki~--- Systems and Means of Informatics} 
20(3):98--125.
\bibitem{7-zat-1}
\Aue{Suchkov, A.\,P.} 2013. Formirovanie sistemy tseley dlya si\-tu\-a\-tsi\-on\-no\-go upravleniya 
[The formation of the objective system to situational management]. \textit{Sistemy i~Sredstva 
Informatiki~--- Systems and Means of Informatics} 23(2):171--182.
\bibitem{8-zat-1}
\Aue{Zatsarinny, A.\,A., A.\,P.~Suchkov, and S.\,V.~Kozlov}. 2012. Osobennosti proektirovaniya 
i~funktsionirovaniya sistemy situatsionnykh tsentrov [Features of the design and functioning of 
the situational centers ]. \textit{Sistemy Vysokoy Dostupnosti} [High Availability Systems]  
8(1):12--21.
\bibitem{9-zat-1}
\Aue{Doran, G.\,T.} 1981. There's a~S.M.A.R.T. way to write management's goals and 
objectives. \textit{Manag. Rev.} 70(11):35--36.
\end{thebibliography}

 }
 }

\end{multicols}

\vspace*{-6pt}

\hfill{\small\textit{Received August 23, 2016}}

\vspace*{-12pt}

\Contr

\noindent
\textbf{Zatsarinny Alexander A.} (b.\ 1951)~--- Doctor of Science in technology, 
professor, 
Deputy Director, Federal Research Center ``Computer Sciences and Control'' of the 
Russian Academy of Sciences, 44-2~Vavilov Str., Moscow 119333, Russian Federation; 
\mbox{AZatsarinny@ipiran.ru}


\vspace*{3pt}


\noindent
\textbf{Suchkov Alexander P.} (b.\ 1954)~--- Doctor of Science in technology, 
leading scientist, Institute of Informatics Problems, Federal Research Center 
``Computer Science and Control'' of the 
Russian Academy of Sciences, 44-2~Vavilov Str., Moscow 119333, 
Russian Federation; \mbox{Asuchkov@ipiran.ru}

 


\label{end\stat}


\renewcommand{\bibname}{\protect\rm Литература}    %10
\def\stat{zatsman}

\def\tit{ТРАНСФОРМАЦИИ ОБЪЕКТОВ ПЕРВОГО И~ВТОРОГО ПОРЯДКА 
В~ЛЕКСИКОГРАФИЧЕСКОЙ ИНФОРМАЦИОННОЙ СИСТЕМЕ$^*$}

\def\titkol{Трансформации объектов первого и~второго порядка 
в~лексикографической информационной системе}

\def\aut{И.\,М.~Зацман$^1$}

\def\autkol{И.\,М.~Зацман}

\titel{\tit}{\aut}{\autkol}{\titkol}

\index{Зацман И.\,М.}
\index{Zatsman I.\,M.}


{\renewcommand{\thefootnote}{\fnsymbol{footnote}} \footnotetext[1]
{Исследование выполнено в~ФИЦ ИУ РАН за счет гранта Российского научного фонда №\,24-18-00155, {\sf 
https://rscf.ru/project/24-18-00155}. Работа выполнялась с~использованием инфраструктуры Центра 
коллективного пользования <<Высокопроизводительные вычисления и~большие данные>> (ЦКП 
<<Информатика>>) ФИЦ ИУ РАН (г.\ Москва).}}


\renewcommand{\thefootnote}{\arabic{footnote}}
\footnotetext[1]{ Федеральный исследовательский центр <<Информатика и~управление>> Российской академии наук, 
\mbox{izatsman@yandex.ru}}

\vspace*{-12pt}


  
  \Abst{Рассматриваются теоретические основания проектирования информационных 
технологий (ИТ) интеграции двуязычных словарей и~параллельных корпусов. Дано описание 
первых результатов создания третьего уровня классификации трансформаций объектов 
предметной области информатики, которую предполагается использовать при создании 
концепции лексикографической информационной системы, обеспечивающей интеграцию. 
Все сущности информатики в~статье разделены на два глобальных класса: объекты и~их 
трансформации. Для каждого такого класса конструируется своя классификация. Ранее были 
описаны два верхних уровня классификации трансформаций объектов предметной области. 
В~данной статье рассматривается третий уровень этой классификации. Основанием для 
построения самого верхнего ее уровня служило деление предметной области информатики 
на среды (ментальная, сенсорно воспринимаемая, цифровая и~ряд других сред), каждая из 
которых по определению включает объекты одной природы. Основанием для построения 
второго уровня классификации трансформаций объектов служила типология знаковых  
сис\-тем А.~Соломоника. Цель статьи состоит в~систематизации трансформаций первого 
и~второго порядка объектов предметной области на третьем уровне этой классификации. 
Основанием для систематизации служит средовая версия иерархии Акоффа.}
  
  \KW{объекты предметной области; трансформации объектов; классификация; данные; 
информация; знание; лексикографическая информационная сис\-тема}

\DOI{10.14357/19922264240211}{VZTGVV}
  
\vspace*{3pt}


\vskip 10pt plus 9pt minus 6pt

\thispagestyle{headings}

\begin{multicols}{2}

\label{st\stat}
  
\section{Введение}

\vspace*{-9pt}

  Возникновение параллельных корпусов, в~которых предложениям 
оригинального текста со\-по\-став\-ле\-ны предложения его перевода, обеспечило 
возможность контрастивного лингвистического\linebreak \mbox{анализа} на принципиально 
новом уровне полноты и~точности, недостижимом в~докорпусную эпоху. 
Пионерскими в~этой области стали работы \mbox{1990-х~гг}. Стига Йоханссона  
с~анг\-ло-нор\-веж\-ским корпусом~[1]. В России параллельные корпусы стали 
формироваться в~начале XXI~века в~рамках Национального корпуса русского 
языка~[2].
  
  Создатели двуязычных словарей используют параллельные корпусы для 
сбора материала и~эмпирической проверки своих гипотез, касающихся 
межъязы\-ко\-вой эквивалентности. Ценность параллельных корпусов 
определяется тем, что в~лингвистике этап сбора исходного материала считается 
наиболее трудоемким и~наименее творческим, а~параллельные корпусы 
позволяют значительно сэкономить время и~силы для творческого этапа 
создания словарей~[3].
 % 
  При этом двуязычные словари, создаваемые на основе исходного материала, 
извлеченного из параллельных корпусов, сейчас формируются без связей с~их 
текстами. Другими словами, онлайновые связи созданных словарей 
с~параллельными корпусами, которые служили источниками исходного 
материала, отсутствуют. 

Параллельные корпусы постоянно пополняются 
новыми текстами, в~предложениях которых можно обнаружить новые значения 
слов и~устойчивых словосочетаний. Однако при этом отсутствуют методы 
и~средства оперативного обновления словарей по корпусным данным. 
В~настоящее время проблема установления связей между двуязычными 
словарями и~параллельными корпусами (далее~--- проблема интеграции) 
находится на стадии поиска концептуальных подходов к~их интеграции на 
уровне значений.
  
  Подход к~решению проблемы интеграции, предлагаемый в~статье, учитывает 
  и~появление новых значений слов и~устойчивых словосочетаний, и~динамику 
смысловых значений, которая обусловлена развитием и~пополнением знания 
лингвистов, фиксирующих эти значения в~результате семантического анализа 
пополняемых корпусных данных. Проведенные эксперименты показали, что 
обнаружение нового лингвистического знания обусловливает и~формирование 
дефиниций новых значений, и~пересмотр уже существующих дефиниций~[4, 5].
  
  Например, в~проведенных экспериментах с~использованием ЦКП 
<<Информатика>> ФИЦ ИУ РАН фиксировалась эволюция значений немецких 
модальных глаголов, исходное состояние значений которых было описано 
в~не\-мец\-ко-рус\-ском словаре. В~экспериментальном массиве текстов как 
потенциальных источниках нового знания 16\,268 предложений содержали 
немецкие модальные глаголы и~в~2041 из них встречался глагол sollen. 
В~начале эксперимента в~словаре были описаны~12~значений этого модального 
глагола. По окончании эксперимента лингвисты обнаружили два новых его 
значения, согласовали их дефиниции и~описали эволюцию дефиниций~[6, 7].
  
  Таким образом, для решения проблемы интеграции требуется фиксировать 
новое знание, обнаруженное лингвистами в~текстовых данных параллельных 
корпусов, отслеживать эволюцию знания, представленного в~виде дефиниций 
значений слов и~устойчивых словосочетаний, и,~соответственно, 
актуализировать электронные двуязычные словари. Предлагаемый 
концептуальный подход к~интеграции, который планируется реализовать 
в~процессе проектирования лексикографической информационной сис\-те\-мы, 
фиксирующей эволюцию лингвистического знания, основан на решении 
следующих задач:\\[-14pt]
  \begin{itemize}
  \item категоризация трех базовых понятий информатики, включенных 
  в~иерархию Акоффа~[8] (данные, информация, знание), на объекты 
проектируемой сис\-те\-мы, которая необходима, чтобы фиксировать 
<<кванты>> нового знания и~отслеживать его эволюцию в~этой сис\-теме;\\[-15pt]
  \item  систематизация трансформаций объектов этой сис\-темы.\\[-14pt]
  \end{itemize}
  
  Цель статьи и~состоит в~решении двух задач: категоризации трех базовых 
понятий информатики на объекты лексикографической информационной  
сис\-те\-мы и~сис\-те\-ма\-ти\-за\-ции трансформаций первого и~второго порядка 
ее объектов.
  
  Трансформациями первого порядка, о которых сказано в~формулировке цели 
статьи, называются взаимные преобразования между двумя объектами  
сис\-те\-мы одной природы. Например, перевод в~сис\-те\-ме текста с~русского 
языка на английский относится к~ним. Трансформациями второго порядка 
и~выше называются взаимные преобразования между двумя и~более объектами 
разной природы. Например, кодирование символов текс\-та компьютерными 
кодами и~их декодирование относятся по определению к~трансформациям 
второго порядка.

%\vspace*{-9pt}
  
\section{Процессы трансформаций в~информатике}

%\vspace*{-3pt}

Процессы трансформаций, рассматриваемые в~статье, относятся к~теоретическому ядру информатики, а~не 
только к~проектированию лексикографической информационной сис\-те\-мы. Например, из трех основных 
подходов к~описанию предметной об\-ласти информатики\footnote{В статье предметная область информатики 
трактуется согласно концепции полиадического компьютинга Пола Розенблума~\cite{9-zac}.} (объектный, 
трансформационный и~синтетический) сис\-те\-ма\-ти\-за\-ция трансформаций ближе всего ко второму 
подходу. Примерами первого подхода, в~рамках которого основное внимание уделяется объектам предметной 
области информатики и~в~меньшей степени отношениям\linebreak между ними, могут служить  
работы~\cite{8-zac, 10-zac, 11-zac}; \mbox{примерами} второго подхода, в~рамках которого основное внимание 
уделяется трансформациям и~в~меньшей степени трансформируемым объектам,~---  
работы~\cite{12-zac, 13-zac}; примерами третьего, синтетического подхода, в~котором уделяется внимание 
и~объектам предметной об\-ласти информатики, и~отношениям между ними, могут служить работы~\cite{14-zac, 
15-zac, 16-zac, 17-zac, 18-zac}.

  Таким образом, для описания трансформаций объектов лексикографической 
информационной\linebreak системы предпочтительнее всего трансформационный 
подход, который упоминается и~в определениях информатики. Например, 
в~2009~г.\ П.~Деннинг и~П.~Розенблум сформулировали суть \mbox{информатики} как 
компьютинга следующим образом: <<$\ldots$информатика~--- это не просто 
алгоритмы и~структуры данных; это преобразования [трансформации] 
представлений>>~\cite{12-zac}. Чуть позже, в~контексте краткого описания 
парадигмы информатики как компьютинга, П.~Деннинг и~П.~Фриман изменили 
эту формулировку на такую: <<Центральный объект внимания в~информатике 
можно определить как информационные процессы~--- \textit{естественные или 
искусственные процессы, преобразующие информацию} (курсив мой~--- 
И.\,З.)>>~\cite{13-zac}. Согласно парадигме, предлагаемой авторами этой 
статьи, на начальном этапе проектирования автоматизированных систем 
базовыми элементами моделей их функционирования служат 
\textit{информационные про\-цессы}.
  
  Однако если 15~лет назад в~формулировке из работы~\cite{13-zac} шла речь 
о~процессах, преобразующих информацию, то в~последние~10~лет в~спектр 
процессов трансформаций все чаще стали включать процессы, преобразующие 
не только информацию, но также и~другие объекты автоматизированных 
систем, в~первую очередь данные и~знания~[19--21]. Например, Виктория 
Стодден, позиционируя науку о~данных как одну из дисциплин информатики, 
говорит, что центральный объект исследований в~науке о~данных~--- это 
<<изучение обобщаемого извлечения знания из данных>>~\cite{21-zac}. 
Увеличение и~чис\-ла объектов, и~спект\-ра процессов их трансформаций 
в~автоматизированных сис\-те\-мах обуслов\-ли\-ва\-ет не\-об\-хо\-ди\-мость 
систематизации и~объектов, и~процессов их трансформаций на начальном этапе 
проектирования сис\-тем.
  
  Для создания концепции лексикографической информационной сис\-те\-мы 
и~проектирования ИТ, обеспечивающих интеграцию 
двуязычных словарей и~параллельных корпусов, сначала выполним 
категоризацию на объекты этой сис\-те\-мы трех базовых понятий информатики 
(данные, информация, знание) в~контексте построения классификаций 
сущностей ее предметной об\-ласти.
  
  Необходимость использования классификаций информатики в~процессе 
создания концепции проиллюстрируем, используя иерархию  
Акоффа~\cite{8-zac}. Он использовал принцип их вертикального размещения 
в~иерархии снизу вверх: данные, информация и~знание. Еще в~ней есть термин 
<<мудрость>>, который в~статье не рассматривается. Такое размещение Акофф 
прокомментировал так: <<Каждое из пе\-ре\-чис\-лен\-ных понятий [кроме данных] 
содержит в~себе нижестоящие$\ldots$>>~\cite{8-zac}.
  
  Этому принципу размещения и~комментарию Акоффа свойственны 
недостатки, проанализированные, в~частности, в~работе~\cite{10-zac}. Главный 
вывод, к~которому пришла Роули после изучения иерархии Акоффа, 
заключается в~следующем: <<$\ldots$информация определяется в~терминах 
данных, знание~--- в~терминах информации$\ldots$ но существует меньше 
консенсуса в~описании трансформаций, которые преобразуют сущности, 
расположенные ниже в~иерархии, в~те, которые находятся над ними, что 
приводит к~их терминологической неопределенности>>~\cite{10-zac}. Причина 
этой неопределенности, скорее всего, в~том, что базовые понятия информатики 
включены в~иерархию Акоффа изолированно от общего контекста 
классификаций сущностей ее предметной об\-ласти.

%\vspace*{-9pt}
  
\section{Классификации сущностей информатики}


%\vspace*{-2pt}

  Все сущности предметной области информатики в~работах~[22, 23] 
разделены на два глобальных класса: ее объекты и~их трансформации. Для 
каждого такого класса была предложена своя классификация. 
В~работе~\cite{22-zac} дано описание классификации объектов предметной 
области информатики, первый уровень которой содержит базовые понятия ее 
предметной области (данные, информация, знания и~др.).  
В~работе~\cite{23-zac} дано описание двух верхних уровней классификации 
трансформаций объектов предметной об\-ласти (см.\ рисунок 
в~работе~\cite{23-zac}). Основанием для построения самого верхнего ее уровня послужило деление 
предметной области информатики на среды\footnote{В~работе~\cite{24-zac} дано описание пяти сред 
предметной области информатики (ментальная; сенсорно воспринимаемая, или информационная; 
цифровая; нейро- и~ДНК-среда), каждая из которых по определению включает объекты одной и~той же 
природы.} и~степень разнообразия природы объектов, вовлеченных в~трансформации:
\begin{itemize}
\item  первый класс верхнего уровня классификации включает 
трансформации объектов в~пределах среды только одной природы 
(трансформации первого порядка);
\item  второй класс включает трансформации объектов, относящихся 
к~двум средам разной природы (трансформации второго порядка);
\item третий и~последующие классы включают трансформации объектов, 
относящихся к~трем и~более средам разной природы (трансформации 
третьего и~более высоких порядков).
\end{itemize}

  В работе~\cite{23-zac} были приведены примеры для трех первых классов 
трансформаций, включая пример трансформаций объектов, относящихся 
к~двум средам разной природы (компьютерное кодирование символов текстов 
с~по\-мощью таб\-лиц Unicode).
  
Основанием для построения второго уровня классификации трансформаций объектов послужила типология 
знаковых сис\-тем А.~Соломоника~\cite[c.~131]{25-zac}: естественные знаковые сис\-те\-мы, образные,  
ес\-тест\-вен\-но-язы\-ко\-в$\acute{\mbox{ы}}$е,  
вер\-баль\-но-не\-сло\-вес\-ные сис\-те\-мы записи\footnote{Под системой записи понимается знаковая 
система, сочетающая вербальные знаки с~несловесными (языки нотной записи, карт, таблиц и~др.).} 
и~формализованные знаковые сис\-те\-мы, включая математические. Введем понятие обобщенного текста~--- 
это текст, который может быть создан в~любой из перечисленных знаковых систем. Тогда обобщенные тексты 
могут быть естественными, образными, ес\-тест\-вен\-но-язы\-ко\-в$\acute{\mbox{ы}}$\-ми,  
вер\-баль\-но-не\-сло\-вес\-ны\-ми и~формализованными. Второй уровень классификации трансформаций 
охватывает не все виды объектов предметной  
об\-ласти информатики, а~только перечисленные~5~видов текс\-тов и~их представления, вовлеченные 
в~процессы трансформаций в~одной или более средах вместе с~данными, знанием и~его концептами.

\begin{figure*}[b] %fig1
\vspace*{6pt}
      \begin{center}
     \mbox{%
\epsfxsize=121.191mm 
\epsfbox{zac-1.eps}
}
\end{center}
\vspace*{-6pt}
\Caption{Средовая версия иерархии Акоффа}
\end{figure*}

\section{Классификация трансформаций: построение~третьего 
уровня}

  Основанием для систематизации трансформаций первого и~второго порядка 
на третьем уровне этой классификации служит иерархия Акоффа~\cite{8-zac}, 
на основе которой и~была создана ее средов$\acute{\mbox{а}}$я версия~[26, 
27]. Для создания средов$\acute{\mbox{о}}$й версии была выполнена 
категоризация трех базовых понятий информатики (данные, информация, 
знания) на объекты лексикографической информационной сис\-те\-мы 
в~процессе создания ее концепции\linebreak (рис.~1).
  


  В отличие от классической иерархии Акоффа, в~ее 
средов$\acute{\mbox{о}}$й версии различаются три вида данных: сенсорно 
воспринимаемые, цифровые и~те данные, которые генерируются 
искусственными нейронными сетями (ИНС) в~системах искусственного интеллекта 
(далее~--- ИИ-дан\-ные). Последний вид данных необходим, например, для 
различения входа и~выхода процесса применения обученной 
ИНС в~цифровой модели генерации знания, описанию которой 
посвящена работа~\cite{27-zac}.
  
  Также предлагается различать два вида информации: сенсорно 
воспринимаемая и~цифровая. Кроме знания в~средов$\acute{\mbox{у}}$ю 
версию добавлены концепты и~ментальные образы сенсорно воспринимаемых 
данных. Последние служат промежуточной сущностью между сенсорно 
воспринимаемыми данными и~генерируемым знанием при описании процессов 
извлечения знания из текстовых данных лексикографической информационной 
системы. Описание объектов средов$\acute{\mbox{о}}$й версии иерархии 
Акоффа (см.\ рис.~1) и~отношений между ними дано в~работах~\cite{26-zac, 28-zac}.
  
  В средов$\acute{\mbox{о}}$й версии число объектов равно восьми. Если 
учитывать направления трансформаций, то между восемью объектами на 
рис.~1 она включает~16 их видов (трансформации на границе между сенсорно 
воспринимаемыми данными и~информацией, обозначенные символом~<<?>>, 
в~статье не рас\-смат\-ри\-ва\-ют\-ся). В~будущем число объектов 
в~средов$\acute{\mbox{о}}$й версии, которая выбрана как основание для 
сис\-те\-ма\-ти\-за\-ции трансформаций первого и~второго порядка, может быть 
увеличено. Для построения классификации трансформаций 
важ\-но не возможное увеличение числа объектов 
и~трансформаций между ними, а то, что их виды в~средов$\acute{\mbox{о}}$й 
версии распределены между трансформациями первого и~второго порядка. Из 
16~видов на рис.~1 шесть относятся к~трансформациям первого порядка, это\linebreak 
виды с~номерами~7, 8, 13--16 (далее~--- типология трансформаций первого 
порядка), а~десять~--- к~трансформациям второго порядка, это виды 
с~\mbox{номерами}~1--6 и~9--12 (далее~--- типология трансформаций второго 
порядка). Разместим обе типологии на третьем уровне классификации (см.\ ее 
схему на рис.~2). Перечислим виды трансформаций первой типологии, вводя 
в~скобках их краткие названия, используемые ниже на рис.~3:
  \begin{description}
  \item[\,] 7~--- членение знания на концепты с~помощью одной или нескольких 
знаковых систем (далее~--- членение знания);
  \item[\,] 8~--- формирование знания на основе концептов (формирование 
знания);
  \item[\,] 13~--- обучение ИНС;
  \end{description}
  
  \vspace*{-6pt}
  
  \pagebreak
  
  \end{multicols}
  
  \begin{figure*} %fig2
\vspace*{1pt}
      \begin{center}
     \mbox{%
\epsfxsize=127.513mm 
\epsfbox{zac-2.eps}
}
\end{center}
\vspace*{-9pt}
\Caption{Схема трех верхних уровней классификации трансформаций объектов (объединены 
по три слоя и~для второго, и~для третьего уровней этой классификации)}
\end{figure*}
  
  \begin{multicols}{2}
  
  \noindent
  \begin{description}
  \item[\,] 14~--- восстановление обучающей информации на основе 
содержания обученной ИНС (обращение ИНС);
  \item[\,] 15~--- использование обученной ИНС (использование ИНС);



  \item[\,] 16~--- восстановление исходных данных, соответствующих 
полученным результатам работы обучен\-ной ИНС (восстановление исходных данных 
по результатам ИНС).
  \end{description}
  
  
  Не все виды трансформаций 13--16 поддерживаются в~конкретных системах 
искусственного интеллекта, но с~теоретической точки зрения все их 
предлагается включить в~первую типологию для полноты спектра видов 
трансформаций.
  
  Перечислим виды трансформаций второй типологии:
  \begin{description}
  \item[\,] 1~--- декодирование цифровых данных в~компьютерных системах 
(декодирование данных);
  \item[\,]  2~--- кодирование сенсорно воспринимаемых данных (кодирование 
данных);
  \item[\,] 3~--- ментальное копирование сенсорно воспринимаемых данных 
(ментальное копирование);
  \item[\,] 4~--- восстановление сенсорно воспринимаемых данных по 
ментальным образам (восстановление по образам);
  \item[\,] 5~--- смысловая интерпретация без деления на концепты ментальных 
образов сенсорно воспринимаемых данных (смысловая интерпретация);
  \item[\,] 6~--- восстановление ментальных образов (восстановление образов);
  \item[\,] 9~--- представление концептов в~виде сенсорно воспринимаемой 
информации, например текс\-та\-ми, формулами, таблицами, рисунками и~т.\,д.\ 
(представление концептов);
  \item[\,] 10~--- понимание смысла сенсорно воспринимаемой информации 
(понимание смысла);
  \item[\,] 11~--- кодирование сенсорно воспринимаемой информации 
(кодирование информации);
\end{description}

\vspace*{-6pt}

\pagebreak

\end{multicols}

\begin{figure*} %fig3
\vspace*{1pt}
      \begin{center}
     \mbox{%
\epsfxsize=163mm 
\epsfbox{zac-3.eps}
}
\end{center}
\vspace*{-9pt}
\Caption{Схема частного случая классификации трансформаций объектов (трансформации 
пронумерованы согласно рис.~1)}
\end{figure*}

\begin{multicols}{2}

\noindent
\begin{description}

  \item[\,] 12~--- декодирование цифровой информации (декодирование 
информации).
  \end{description}
  
  Отметим, что в~существующих ИТ
  и~компьютерных системах наиболее часто используются виды 
трансформаций~13 и~15 типологии первого порядка и~1, 2, 11 и~12 типологии 
второго порядка. На рис.~2 в~первом слое третьего уровня классификации 
показаны типологии первого порядка без указания числа трансформаций в~них 
и~без детализации трансформируемых объектов.
  
  Во втором слое третьего уровня классификации условно (без названий) 
показаны типологии второго порядка. Также на рис.~2 в~третьем слое третьего 
уровня классификации условно (также без названий) показаны типологии 
третьего порядка, которые планируется рассмотреть в~отдельной статье. По 
определению они должны включать трансформации между тремя объектами 
разной природы, но средов$\acute{\mbox{а}}$я версия иерархии Акоффа 
включает трансформации только между двумя объектами разной природы. 
Поэтому потребуется другое основание для их систематизации (ранее были 
рассмотрены отдельные примеры трансформаций третьего 
порядка\footnote{Далеко не всегда трансформации третьего и~более высоких порядков можно 
рассматривать как последовательность трансформаций второго порядка. Примером этого могут 
служить трансформации в~процессе обучения пациента пользованию роботизированной рукой, 
охватывающие личностные концепты пациента, релевантные его намерениям, сигналы активности 
мозга как объекты нейросреды и~компьютерные коды~\cite{29-zac}.}~\cite{29-zac}).

\section{Классификация трансформаций: частный~случай}

  Выше было отмечено, что в~будущем число объектов 
в~средов$\acute{\mbox{о}}$й версии иерархии Акоффа может быть увеличено. 
Это означает, что увеличатся и~чис\-ло объектов, и~чис\-ло трансформаций между 
ними в~классификации трансформаций, так как эта средов$\acute{\mbox{а}}$я 
версия служит по определению основанием для систематизации 
трансформаций первого и~второго порядка. Поэтому на третьем уровне рис.~2 
указаны типологии без детализации объектов и~без указания числа 
трансформаций в~каждой из них. С~одной стороны, при таком подходе 
получаем достаточно общий вид этой классификации, так как она не зависит от 
числа объектов в~том или ином варианте средов$\acute{\mbox{о}}$й версии 
(и~это существенно упрощает рис.~2). С~другой стороны, на третьем уровне 
такой общей классификации подразумевается, но не эксплицируется природа 
трансформируемых объектов и~их возможные сочетания в~трансформациях. 

При проектировании лексикографической информационной системы важно 
эксплицировать природу трансформируемых объектов и~их возможные 
сочетания.
  %
  Поэтому в~парадигму информатики~\cite{30-zac} кроме общей 
классификации трансформаций предлагается включать и~ее частные случаи, 
эксплицирующие природу трансформируемых объектов. 

В~этом разделе 
рассмотрим один частный случай, когда используются только естественные 
знаковые сис\-те\-мы из типологии А.~Соломоника~\cite{25-zac} вместе 
с~данными, знанием и~его концептами. Чис\-ло естественных языков при этом не 
ограничено. И~этот частный случай классификации включает только три 
класса природных трансформаций (первого, второго и~третьего порядка, см.\ 
схему классификации на рис.~3).
  
  Первый и~второй уровни схемы общей классификации (см.\ рис.~2) можно 
объединить в~один уровень в~этом частном случае. Ниже этого уровня 
приведено содержание типологий первого и~второго порядка без содержания 
типологий третьего по\-рядка.




  Наполнение типологий первого и~второго порядка соответствует 
средов$\acute{\mbox{о}}$й версии иерархии Акоффа на рис.~1, содержащей 
6~видов трансформаций типологии первого порядка и~10~видов 
трансформаций типологии второго порядка (на рис.~3 стрелки указывают 
направления трансформаций согласно средов$\acute{\mbox{о}}$й версии на рис.~1).
  
  Таким образом, частный случай классификации содержит для этих двух 
типологий 16~теоретически возможных трансформаций, 6 из которых 
в~настоящее время в~существующих ИТ применяются наиболее часто: виды 
трансформаций~1, 2, 11 и~12 типологии второго порядка реализуются 
с~помощью тех или иных методов ко\-ди\-ро\-ва\-ния/де\-ко\-ди\-ро\-ва\-ния 
(например, с~использованием таблиц Unicode), а~виды трансформаций~13 и~15
 в~типологии первого порядка реализуются полностью с~по\-мощью процессов 
цифровой обработки компьютерами.
  
  Остальные виды трансформаций или применяются намного реже (это 
виды~3, 5, 7, 9 и~10), или находятся в~стадии поиска и~разработки (14 и~16) или 
в~настоящее время носят только теоретический характер, обеспечивая полноту 
первой и~второй типологий (4, 6 и~8). Знаком~<<?>> обозначены те виды 
трансформаций, которые по определению не существуют в~используемой 
парадигме информатики~\cite{30-zac}. Однако возможно, что в~других 
будущих подходах к~построению ее парадигмы эти виды трансформаций будут 
существовать.
  
\section{Заключение}

  На сегодняшний день процесс построения классификаций объектов 
предметной области информатики~\cite{22-zac} и~их  
трансформаций~\cite{23-zac} еще не завершен. Однако первые результаты их 
построения уже используются для создания концепции лексикографической 
информационной сис\-те\-мы, обеспечивающей интеграцию двуязычных 
словарей и~параллельных корпусов.
  
  \bigskip
  
  
  Автор признателен рецензентам за помощь в~улучшении статьи.
  
{\small\frenchspacing
 { %\baselineskip=10.6pt
 %\addcontentsline{toc}{section}{References}
 \begin{thebibliography}{99}
\bibitem{1-zac}
\Au{Aijmer K., Altenberg~B.} Advances in corpus-based contrastive linguistics. Studies in honour 
of Stig Johansson.~--- Amsterdam: John Benjamins, 2013. 295~p.  doi: 10.1075/scl.54.
\bibitem{2-zac}
\Au{Добровольский Д.\,О., Кретов~А.\, А., Шаров~С.\,А.} Корпус параллельных текстов~// 
Научная и~техническая информация. Сер.~2: Информационные процессы и~сис\-те\-мы, 2005. 
№\,6. С.~16--27.
\bibitem{3-zac}
\Au{Добровольский Д.\,О.} Корпус параллельных текстов и~сопоставительная 
лексикология~// Труды Института русского языка им.\ В.\,В.~Виноградова, 2015. №\,6. 
С.~413--449. EDN: VJQBHP.
\bibitem{4-zac}
\Au{Гончаров А.\,А., Зацман~И.\,М., Кружков~М.\,Г.} Эволюция классификаций 
в~надкорпусных базах данных~// Информатика и~её применения, 2020. Т.~14. Вып.~4. 
С.~108--116. doi: 10.14357/19922264200415.  
EDN: \mbox{GKWBZT}.
\bibitem{5-zac}
\Au{Гончаров А.\, А., Зацман И. \,М., Кружков~М.\, Г}. Представление новых 
лексикографических знаний в~динамических классификационных сис\-те\-мах~// 
Информатика и~её применения, 2021. Т.~15. Вып.~1. С.~86--93.  doi: 10.14357/19922264210112. EDN: OPEFXW.
\bibitem{6-zac}
\Au{Zatsman I.} Finding and filling lacunas in linguistic typologies~// 15th Forum (International) 
on Knowledge Asset Dynamics Proceedings.~--- Matera, Italy: Institute of Knowledge Asset 
Management, 2020. P.~780--793.
\bibitem{7-zac}
\Au{Zatsman I.} Three-dimensional encoding of emerging meanings in AI-systems~// 21st 
European Conference on Knowledge Management Proceedings.~--- Reading, U.K.: Academic 
Publishing International Ltd., 2020. P.~878--887.
\bibitem{8-zac}
\Au{Ackoff R.} From data to wisdom~// J.~Applied Systems Analysis, 1989. Vol.~16. No.\,1. P.~3--9.
\bibitem{9-zac}
\Au{Rosenbloom P.\,S.} On computing: The fourth great scientific domain.~--- Cambridge, MA, 
USA: MIT Press, 2013. 307~p.
\bibitem{10-zac}
\Au{Rowley J.} The wisdom hierarchy: Representations of the DIKW hierarchy~// J.~Inf. 
Sci., 2007. Vol.~33. Iss.~2. P.~163--180. doi: 10.1177/0165551506070706.
\bibitem{11-zac} 
\Au{Frick$\acute{\mbox{e}}$~M.\,H.} Data--Information--Knowledge--Wisdom (DIKW) pyramid, 
framework, continuum~// Encyclopedia of big data~/ Eds. L.~Schintler, C.~McNeely.~--- Cham: 
Springer, 2018. 4~p. doi: 10.1007/978-3-319-32001-4\_331-1.
\bibitem{12-zac}
\Au{Denning P., Rosenbloom~P.} Computing: The fourth great domain of science~// Commun. 
ACM, 2009. Vol.~52. Iss.~9. P.~27--29.
\bibitem{13-zac}
\Au{Denning P., Freeman~P.} Computing's paradigm~// Commun.  ACM, 2009. Vol.~52. 
Iss.~12. P.~28--30. doi: 10.1145/ 1610252.1610265.
\bibitem{17-zac} %14
\Au{Farradane J.} Knowledge, information, and information science~// J.~Inf. Sci., 
1980. Vol.~2. Iss.~2. P.~75--80. doi: 10.1177/01655515800020020.

\bibitem{15-zac}
\Au{Шрейдер Ю.\,А.} Информация и~знание~// Сис\-тем\-ная концепция информационных 
процессов.~--- М.: ВНИИСИ, 1988. С.~47--52.
\bibitem{16-zac}
\Au{Ingwersen P.} Information and information science~// Enclyclopaedie of library and 
information science~/ Eds. J.\,D.~McDonald, 
M.~Levine-Clark.~--- New York, NY, USA: Marcel Dekker Inc., 1992. Vol.~56. Sup.~19. 
P.~137--174.

\bibitem{14-zac} %17
Информатика как наука об информации: Информационный, документальный, 
технологический, экономический, социальный и~организационный аспекты~/ Под ред. 
Р.\,С.~Гиляревского.~--- М.: Фаир-Пресс, 2006. 592~с.

\bibitem{18-zac}
\Au{Hjorland B.} Library and information science: practice, theory, and philosophical basis~// 
Inform. Process. Manag., 2000. Vol.~36. Iss.~3. P.~501--531. doi:  
10.1016/S0306-\mbox{4573(99)00038-2}.
\bibitem{19-zac}
Deep shift~--- technology tipping points and societal impact.~--- Geneva: WE Forum, 2015. 44~p. 
{\sf http://www3.weforum.org/docs/WEF\_GAC15\_ Technological\_Tipping\_Points\_report\_2015.pdf}.
\bibitem{20-zac}
\Au{Berman F., Rutenbar~R., Hailpern~B., Christensen~H., Davidson~S., Estrin~D., 
Franklin~M., Martonosi~M., Raghavan~P., Stodden~V., Szalay~A.\,S.} Realizing the potential of 
data science~// Commun.  ACM, 2018. Vol.~61. Iss.~4. P.~67--72. doi: 10.1145/3188721.

\bibitem{21-zac}
\Au{Stodden V.} The data science life cycle: A~disciplined approach to advancing data science as 
a~science~// Commun.  ACM, 2020. Vol.~63. Iss.~7. P.~58--66. doi: 10.1145/ 3360646.


\bibitem{23-zac} %22
\Au{Зацман И.\,М.} Научная парадигма информатики: классификация трансформаций 
объектов предметной об\-ласти~// Системы и~средства информатики, 2023. Т.~33. №\,4. 
С.~126--138. doi: 10.14357/08696527230412. EDN: ZIKUWO.

\bibitem{22-zac} %23
\Au{Зацман И.\,М.} Научная парадигма информатики: классификация объектов предметной  
об\-ласти~// Информатика и~её применения, 2023. Т.~17. Вып.~4. С.~96--103. doi: 
10.14357/19922264230413. EDN: FIUQAT.

\bibitem{24-zac}
\Au{Зацман И.\,М.} О~научной парадигме информатики: верхний уровень классификации 
объектов ее предметной об\-ласти~// Информатика и~её применения, 2022. Т.~16. Вып.~4. 
С.~73--79. doi: 10.14357/ 19922264220411. EDN: XZNKVI.

\bibitem{25-zac}
\Au{Соломоник А.\,Б.} Философия знаковых систем и~язык.~--- М.: ЛКИ, 2011. 408~с.
\bibitem{26-zac}
\Au{Зацман И.\,М.} Трансформация иерархии Акоффа в~научной парадигме информатики~// 
Информатика и~её применения, 2023. Т.~17. Вып.~3. С.~107--113. doi: 
10.14357/19922264230315. EDN: UMVRRV.

\bibitem{27-zac}
\Au{Zatsman I.} Building digital spiral models of knowledge generation~// 19th Forum 
(International) on Knowledge Asset Dynamics Proceedings.~--- Matera, Italy: Arts for Business 
Institute, 2024. P.~2185--2196.
\bibitem{28-zac}
\Au{Zatsman I.} Digital spiral model of knowledge creation and encoding its dynamics~// 18th 
Forum (International) on Knowledge Asset Dynamics Proceedings.~--- Matera, Italy: Arts for 
Business Institute, 2023. P.~581--596.
\bibitem{29-zac}
\Au{Зацман И.\,М.} Интерфейсы третьего порядка в~информатике~// Информатика и~её 
применения, 2019. Т.~13. Вып.~3. С.~82--89. doi: 10.14357/19922264190312. EDN: 
EHRQLF.

\bibitem{30-zac}
\Au{Зацман И.\,М.} Научная парадигма информатики как третьей культуры~//  
На\-уч\-но-тех\-ни\-че\-ская информация. Сер.~1: Организация и~методика информационной 
работы, 2023. №\,11. С.~1--14.

\end{thebibliography}

 }
 }

\end{multicols}

\vspace*{-9pt}

\hfill{\small\textit{Поступила в~редакцию 14.04.24}}

\vspace*{4pt}

%\pagebreak

%\newpage

%\vspace*{-28pt}

\hrule

\vspace*{2pt}

\hrule



\def\tit{OBJECT TRANSFORMATIONS OF~THE~FIRST AND~SECOND ORDER
IN~A~LEXICOGRAPHIC INFORMATION SYSTEM\\[-5pt]}


\def\titkol{Object transformations of~the~first and~second order
in~a~lexicographic information system}


\def\aut{I.\,M.~Zatsman}

\def\autkol{I.\,M.~Zatsman}

\titel{\tit}{\aut}{\autkol}{\titkol}

\vspace*{-13pt}


\noindent
Federal Research Center ``Computer Science and Control'' of the Russian Academy of Sciences, 
44-2~Vavilov Str., Moscow 119133, Russian Federation


\def\leftfootline{\small{\textbf{\thepage}
\hfill INFORMATIKA I EE PRIMENENIYA~--- INFORMATICS AND
APPLICATIONS\ \ \ 2024\ \ \ volume~18\ \ \ issue\ 2}
}%
 \def\rightfootline{\small{INFORMATIKA I EE PRIMENENIYA~---
INFORMATICS AND APPLICATIONS\ \ \ 2024\ \ \ volume~18\ \ \ issue\ 2
\hfill \textbf{\thepage}}}

\vspace*{2pt}



\Abste{The theoretical foundations of the design of information technologies used for 
the integration of bilingual dictionaries and parallel corpora are considered. The 
description of the first outcomes of the creation of the third\linebreak\vspace*{-12pt}}

\Abstend{ level of object 
transformations classification in the subject domain of informatics, which is supposed 
to be used
in creating the lexicographic information system providing integration, is 
given. All the entities of informatics are divided into two global classes: objects and 
their transformations. For each such class, its own classification is constructed. 
Previously, the two upper levels of the object transformation classification in the subject 
domain have been described. The present paper discusses the third level of this classification. The 
basis for the construction of its highest level was the division of the subject domain of 
informatics into media (mental, sensory, digital, and a~number of other media), each 
of which by definition includes objects of the same nature. The Solomonick's 
typology of sign systems served as the basis for constructing the second level of the 
object transformation classification. The aim of the paper is to systematize object 
transformations of the first and second orders at the third level of this classification. 
The basis for systematization is the medium version of the Ackoff's hierarchy.}

\KWE{subject domain objects; object transformations; classification; data; 
information; knowledge; lexicographic information system}


\DOI{10.14357/19922264240211}{VZTGVV}

\vspace*{-12pt}

\Ack

\vspace*{-3pt}


\noindent
The reported study was funded by the Russian Science Foundation, project  
No.\,24-18-00155, {\sf 
https://rscf.ru/project/24-18-00155}. The research was carried out using the infrastructure of the Shared 
Research Facilities ``High Performance Computing and Big Data'' (CKP 
``Informatics'') of FRC CSC RAS (Moscow) .
   


  \begin{multicols}{2}

\renewcommand{\bibname}{\protect\rmfamily References}
%\renewcommand{\bibname}{\large\protect\rm References}

{\small\frenchspacing
 {%\baselineskip=10.8pt
 \addcontentsline{toc}{section}{References}
 \begin{thebibliography}{99} 
\bibitem{1-zac-1}
\Aue{Aijmer, K., and B.~Altenberg.} 2013. \textit{Advances in corpus-based 
contrastive linguistics. Studies in honour of Stig Johansson}. Amsterdam: John 
Benjamins. 295~p. doi: 10.1075/scl.54.
\bibitem{2-zac-1}
\Aue{Dobrovolskiy, D.\,O., A.\,A.~Kretov, and S.\,A.~Sharov.} 2005. Korpus 
parallel'nykh tekstov [Corpus of parallel texts]. \textit{Nauchnaya i~tekhnicheskaya 
informatsiya. Ser. 2. Informatsionnye protsessy i~sistemy} [Scientific and Technical 
Information. Ser.~2: Information Processes and Systems] 6:16--27.
\bibitem{3-zac-1}
\Aue{Dobrovolskiy, D.\,O.} 2015. Korpus parallel'nykh tekstov i~sopostavitel'naya 
leksikologiya [The corpus of parallel texts and contrastive lexicology]. \textit{Trudy 
Instituta russkogo yazyka im. V.\,V.~Vinogradova} [Proceedings of the 
V.\,V.~Vinogradov Russian Language Institute] 6:413--449. EDN: VJQBHP.
\bibitem{4-zac-1}
\Aue{Goncharov, A.\,A., I.\,M.~Zatsman, and M.\,G.~Kruzhkov.} 2020. Evolyutsiya 
klassifikatsiy v~nadkorpusnykh ba\-zakh dannykh [Evolution of classifications in 
supracorpora databases]. \textit{Informatika i~ee Primeneniya~--- Inform. \mbox{Appl.}}  
14(4):108--116. doi: 10.14357/19922264200415.  
EDN: GKWBZT.
\bibitem{5-zac-1}
\Aue{Goncharov, A.\,A., I.\,M.~Zatsman, and M.\,G.~Kruzhkov.} 2021. 
Predstavlenie novykh leksikograficheskikh znaniy v~dinamicheskikh 
klassifikatsionnykh sistemakh [Representation of new lexicographical knowledge in 
dynamic classification systems]. \textit{Informatika i~ee Primeneniya~--- Inform. 
Appl.} 15(1):86--93. doi: 10.14357/19922264210112. EDN: OPEFXW.
\bibitem{6-zac-1}
\Aue{Zatsman, I.} 2020. Finding and filling lacunas in linguistic typologies. 
\textit{15th Forum (International) on Knowledge Asset Dynamics Proceedings}. 
Matera, Italy: Institute of Knowledge Asset Management. 780--793.
\bibitem{7-zac-1}
\Aue{Zatsman, I.} 2020. Three-dimensional encoding of emerging meanings in  
AI-systems. \textit{21st European Conference on Knowledge Management 
Proceedings}. Reading, U.K.: Academic Publishing International Ltd. 878--887.
\bibitem{8-zac-1}
\Aue{Ackoff, R.} 1989. From data to wisdom. \textit{J.~Applied Systems Analysis} 
16(1):3--9.
\bibitem{9-zac-1}
\Aue{Rosenbloom, P.\,S.} 2013. \textit{On computing: The fourth great scientific 
domain}. Cambridge, MA: MIT Press. 307~p.
\bibitem{10-zac-1}
\Aue{Rowley, J.} 2007. The wisdom hierarchy: Representations of the DIKW 
hierarchy. \textit{J.~Inf. Sci.} 33(2):163--180. doi: 10.1177/0165551506070706.
\bibitem{11-zac-1}
\Aue{Frick$\acute{\mbox{e}}$, M.\,H.} 2018.  
Data-Information-Knowledge-Wisdom (DIKW) pyramid, framework, continuum. 
\textit{Encyclopedia of big data}. Eds. L.~Schintler and C.~McNeely. Cham: 
Springer. 4~p. doi: 10.1007/978-3-319-32001- 4\_331-1.
\bibitem{12-zac-1}
\Aue{Denning, P., and P.~Rosenbloom.} 2009. Computing: The fourth great domain 
of science. \textit{Commun. ACM} 52(9):27--29.
\bibitem{13-zac-1}
\Aue{Denning, P., and P.~Freeman.} 2009. Computing's paradigm. \textit{Commun. 
ACM} 52(12):28--30. doi: 10.1145/ 1610252.1610265.

\bibitem{17-zac-1} %14
\Aue{Farradane, J.} 1980. Knowledge, information, and information science. 
\textit{J.~Inf. Sci.} 2(2):75--80. doi: 10.1177/ 01655515800020020.

\bibitem{15-zac-1}
\Aue{Shreyder, Yu.\,A.} 1988. Informatsiya i~znanie [Information and knowledge]. 
\textit{Sistemnaya kontseptsiya in\-for\-ma\-tsi\-on\-nykh protsessov} [System concept of 
information processes]. Moscow: VNIISI. 47--52.
\bibitem{16-zac-1}
\Aue{Ingwersen, P.} 1995. Information and information science. 
\textit{Encyclopedia of library and information science}. Eds. J.\,D.~McDonald and 
M.~Levine-Clark. New York, NY: Marcel Dekker Inc. 56(19):137--174.

\bibitem{14-zac-1} %17
Gilyarevskiy, R.\,S., ed. 2006. \textit{Informatika kak nauka ob informatsii: 
informatsionnyy, dokumental'nyy, tekh\-no\-lo\-gi\-che\-skiy, ekonomicheskiy, sotsial'nyy 
i~organizatsionnyy aspekty} [Informatics as information science: Informational, 
documentary, technological, economic, social, and organizational dimensions]. 
Moscow: FAIR-PRESS. 592~p.

\bibitem{18-zac-1}
\Aue{Hjorland, B.} 2000. Library and information science: Practice, theory, and 
philosophical basis. \textit{Inform. Process. Manag.} 36(3):501--531. doi:  
10.1016/S0306-\mbox{4573(99)00038-2}.
\bibitem{19-zac-1}
Deep shift~--- technology tipping points and societal impact. 2015. \textit{World Economic 
Forum}. Geneva. 44~p. Available at: {\sf 
http://www3.weforum.org/docs/WEF\_ GAC15\_Technological\_Tipping\_Points\_report\_2015.pdf} (accessed May~20, 
2024).
\bibitem{20-zac-1}
\Aue{Berman, F., R.~Rutenbar, B.~Hailpern, H.~Christensen, S.~Davidson, 
D.~Estrin, M.~Franklin, M.~Martonosi, P.~Raghavan, V.~Stodden, and 
A.\,S.~Szalay.} 2018. Realizing the potential of data science. \textit{Commun. ACM} 
61(4):67--72. doi: 10.1145/3188721.
\bibitem{21-zac-1}
\Aue{Stodden, V.} 2020. The data science life cycle: A~disciplined approach to 
advancing data science as a~science. \textit{Commun. ACM} 
 63(7):58--66. doi: 10.1145/3360646.

\bibitem{23-zac-1} %22
\Aue{Zatsman, I.\,M.} 2023. Nauchnaya paradigma informatiki: klassifikatsiya 
transformatsiy ob''ektov predmetnoy oblasti [Scientific paradigm of informatics: 
Transformation classification of domain objects]. \textit{Sistemy i~Sredstva 
Informatiki~--- Systems and Means of Informatics} 33(4):126--138. doi: 
10.14357/08696527230412. EDN: ZIKUWO.

\bibitem{22-zac-1} %23
\Aue{Zatsman, I.\,M.} 2023. Nauchnaya paradigma informatiki: klassifikatsiya 
ob''ektov predmetnoy oblasti [Scientific paradigm of informatics: Classification of 
domain objects]. \textit{Informatika i~ee Primeneniya~--- Inform. Appl.} 
 17(4):96--103. doi: 10.14357/19922264230413. EDN: FIUQAT.
 
\bibitem{24-zac-1}
\Aue{   Zatsman, I.\,M.} 2022. O nauchnoy paradigme informatiki: verkhniy uroven' 
klassifikatsii ob''ektov ee predmetnoy oblasti [On the scientific paradigm of 
informatics: The classification high level of its objects]. \textit{Informatika i~ee 
Primeneniya~--- Inform. Appl.} 16(4):73--79. doi: 10.14357/19922264220411. EDN: 
XZNKVI.
\bibitem{25-zac-1}
\Aue{Solomonick, A.\,B.} 2011. \textit{Filosofiya znakovykh system i~yazyk} 
[Philosophy of sign systems and language]. Moscow: LKI. 408~p.
\bibitem{26-zac-1}
\Aue{Zatsman, I.\,M.} 2023. Transformatsiya ierarkhii Akoffa v~nauchnoy 
paradigme informatiki [Transformation of the Ackoff's hierarchy in the scientific 
paradigm of informatics]. \textit{Informatika i~ee Primeneniya~--- Inform. \mbox{Appl.}} 
17(3):107--113. doi: 10.14357/19922264230315. EDN: UMVRRV.
\bibitem{27-zac-1}
\Aue{Zatsman, I.} 2024. Building digital spiral models of knowledge 
generation. \textit{19th Forum (International) on Knowledge Asset Dynamics 
Proceedings}. Matera, Italy: Arts for Business Institute. 2185--2196.
\bibitem{28-zac-1}
\Aue{Zatsman, I.} 2023. Digital spiral model of knowledge creation and encoding its 
dynamics. \textit{18th Forum (International) on Knowledge Asset Dynamics 
Proceedings}. Matera, Italy: Arts for Business Institute. 581--596.
\bibitem{29-zac-1}
\Aue{Zatsman, I.\,M.} 2019. Interfeysy tret'ego poryadka v~informatike 
 [Third-order interfaces in informatics]. \textit{Informatika i~ee Primeneniya~--- 
Inform. Appl.} 13(3):82--89. doi: 10.14357/19922264190312. EDN: EHRQLF.
\bibitem{30-zac-1}
\Aue{Zatsman, I.} 2023. Scientific paradigm of informatics as a~third culture. 
\textit{Scientific Technical Information Processing} 50(4):246--258. doi: 
10.3103/S0147688223040111. EDN: CKHMYS.

\end{thebibliography}

 }
 }

\end{multicols}

\vspace*{-6pt}

\hfill{\small\textit{Received April 14, 2024}} 


\vspace*{-12pt}


\Contrl

\vspace*{-3pt}

\noindent
\textbf{Zatsman Igor M.} (b.\ 1952)~--- Doctor of Science in technology, head of 
department, Federal Research Center ``Computer Science and Control'' of the 
Russian Academy of Sciences, 44-2~Vavilov Str., Moscow 119333, Russian 
Federation; \mbox{izatsman@yandex.ru}





\label{end\stat}

\renewcommand{\bibname}{\protect\rm Литература}    %11
\def\stat{morozova}

\def\tit{ТРАНСФОРМАЦИОННЫЕ МОДЕЛИ ЯЗЫКОВЫХ СТРУКТУР 
ДЛЯ~ФРАНЦУЗСКО-РУССКОГО МАШИННОГО ПЕРЕВОДА}

\def\titkol{Трансформационные модели языковых структур 
для~французско-русского машинного перевода}

\def\autkol{Ю.\,И.~Морозова}
\def\aut{Ю.\,И.~Морозова$^1$}

\titel{\tit}{\aut}{\autkol}{\titkol}

%{\renewcommand{\thefootnote}{\fnsymbol{footnote}}\footnotetext[1]
%{Работа поддержана Российским фондом фундаментальных исследований
%(проекты 11-01-00515а и 11-07-00112а), а также Министерством
%образования и науки РФ в рамках ФЦП <<Научные и
%научно-педагогические кадры инновационной России на 2009--2013~годы>>.}}


\renewcommand{\thefootnote}{\arabic{footnote}}
\footnotetext[1]{Институт проблем информатики Российской академии наук, yulia-ipi@yandex.ru}


\Abst{Данная работа посвящена актуальным проблемам исследования 
трансформационных свойств языковых объектов при переводе предикативных 
структур с французского языка на русский. Основное внимание уделено 
изменению категориальной принадлежности и изменению грамматических 
характеристик предикатных слов при переводе. Материалом исследования 
послужили патентные тексты на французском языке и их переводы на русский 
язык, выполненные спе\-ци\-а\-ли\-ста\-ми-пе\-ре\-вод\-чи\-ками. }

\KW{французско-русский автоматический перевод; функциональная семантика; 
языковые трансформации; вершинные грамматики}

 \vskip 14pt plus 9pt minus 6pt

      \thispagestyle{headings}

      \begin{multicols}{2}
      
            \label{st\stat}

\section{Введение}

Данное исследование направлено на исследование предикатных фразовых 
структур на основе вершинных грамматик применительно к задачам 
моделирования машинного перевода и извлечения знаний из текста для 
французско-русского на\-прав\-ле\-ния. Основной задачей ставилось создание 
унифицированной модели функциональных значений синтаксем, в которой 
бы учитывались сдвиги значений, производимые переводческими 
трансформациями. Предикатные слова являются вершинами синтаксической 
структуры предложения, а также вершинами внутреннего представления 
знаний в структурах баз знаний, поэтому описание их дистрибутивных и 
трансформационных свойств имеет первостепенное значение.

Исследования ведутся в рамках проекта по созданию многоязычного 
лингвистического процессора для задач машинного перевода и извлечения 
знаний из текстов, разрабатываемого на основе 
функ\-ци\-о\-наль\-но-се\-ман\-ти\-че\-ско\-го подхода~[1]. В~качестве 
материала исследования были использованы фрагменты параллельных 
текстов патентов, содержащие предикатные выражения. Модель перевода с 
учетом трансформаций для рус\-ско-фран\-цуз\-ской\linebreak языковой пары основана 
на многовариантной когнитивной трансферной грамматике (МКТГ), 
разработанной Е.\,Б.~Козеренко~[1--4]. Данный формализм имеет 
определенные черты грамматики\linebreak составляющих и вершинной грамматики 
HPSG (Head-driven phrase structure grammar)~[5]. 
Преимущество данного формализма заключается в том, что он 
позволяет описывать как отношения линейного порядка, так и отношения 
зависимости в рамках одной и той же фразовой структуры. Формализмы, 
основанные на порождающей грамматике Хомского и на вершинной 
грамматике HPSG, широко применяются для создания сис\-тем 
автоматической обработки текстов на английском языке и других 
европейских языках (французском, испанском, немецком, чешском). Однако 
возможности применения данных формализмов для автоматической 
обработки русского языка изучены недостаточно. 
     
\section{Грамматические формализмы, используемые для~создания 
лингвистических процессоров}
      
     Для формального описания синтаксиса естест\-вен\-ных языков 
применительно к задачам автоматической обработки языка наиболее часто 
использу\-ются следующие виды формализмов: регулярные грамматики, 
     кон\-текст\-но-сво\-бод\-ные грамматики,\linebreak мягко 
     кон\-текст\-но-за\-ви\-си\-мые грамматики. Регулярные грамматики не 
могут быть использованы для полноценного описания синтаксиса. Данный\linebreak 
формализм используется только для частичного синтаксического анализа 
предложений (shallow parsing). С~помощью кон\-текст\-но-сво\-бод\-ных 
грамматик можно описать большинство предложений естественного языка, 
однако грамматики данного класса не позволяют описывать предложения с 
разрывными структурами. Наконец, мягко кон\-текст\-но-за\-ви\-си\-мые 
грамматики являются наиболее мощным формализмом и позволяют 
описывать любые виды предложений естественных языков, однако их 
применение в сис\-те\-мах автоматической обработки естественного языка 
связано с большими\linebreak
 вычислительными затратами. Для описания явлений 
естественных языков применяются, в основном, кон\-текст\-но-сво\-бод\-ные 
грамматики с некоторыми расширениями. Однако вопрос о выборе наиболее 
адекватного формализма для описания синтаксиса естественных языков и 
создания линг\-ви\-сти\-че\-ских процессоров остается открытым. Во многих 
сис\-те\-мах автоматической обработки текс\-тов на английском языке 
используются модернизированные грамматики Н.~Хомского~\cite{6-mor}. 
     
     Для создания сис\-тем автоматической обработки текстов на языках с 
богатой морфологией и относительно свободным порядком слов часто 
используется вершинная грамматика HPSG, разработанная Карлом Поллардом и Иваном Сагом~\cite{5-mor}. 
Согласно данной теории описание грамматики языка должно состоять из 
очень подробного словаря и очень небольшого количества грамматических 
правил, носящих универсальный характер. Словарь имеет хорошо 
проработанную иерархическую структуру, которая характеризуется 
наследованием свойств по умолчанию. В~словарном описании слов, которые 
могут являться вершинами синтаксических групп (существительных, 
глаголов, предлогов, прилагательных) есть поле HEAD, в котором 
описываются такие важные с точки зрения синтаксического поведения слова 
свойства, как часть речи, признаки согласования, форма, предикативность и~др. 
Данные свойства передаются группам, порождаемым данными 
вершинами, в соответствии с правилами грамматики. Таким образом, 
процесс порождения правильно построенных предложений определяется 
свойствами вершин (отсюда название Head-driven phrase structure grammar). 
Одним из основных понятий HPSG является структура свойств (feature 
structure). Это набор атрибутов с их значениями, например словарное 
описание лексемы задается следующей структурой свойств: [PHON$\ldots$ 
SYN$\ldots$ SEM$\ldots$ ARG-ST$\ldots$].
     
     Значение каждого из свойств может пред\-став\-лять собой как единый 
элемент, так и структуру свойств, например свойство AGR (согласование) 
имеет следующую структуру: AGR [PER$\ldots$, NUM$\ldots$, 
GEND$\ldots$].
     
     Унификационный механизм, использующийся в грамматике HPSG, 
позволяет объединить два описания структур свойств. Результатом данной 
операции является структура свойств, содержащая информацию из обеих 
структур. Механизм унификации используется при проверке согласования 
морфологических характеристик слов, необходимой для включения их в одну 
синтаксическую группу. Вершинная грамматика HPSG была с успехом 
применена при создании сис\-тем автоматической обработки текстов на 
разных языках (английском, французском, чешском). Один из примеров такой 
сис\-те\-мы~--- грамматика английского языка \mbox{LinGO} English Resource Grammar 
(ERG) и синтаксически аннотированный корпус Redwoods, размеченный 
автоматически с использованием грамматики ERG~\cite{7-mor}.
     
     В работе~\cite{8-mor} предлагается формализм, име\-ющий некоторые 
черты универсальной грамматики Хомского, вершинной грамматики HPSG и 
лек\-сико-функ\-ци\-о\-наль\-ной грамматики LFG
(Lexical functional grammar). С~использованием 
данного\linebreak формализма была реализована программа синтаксического анализа 
русского языка, которая строит структуру предложения в двух аспектах: как 
структуру составляющих и как функциональную структуру. Для описания 
согласования морфологических характеристик применяется аппарат 
унификации, используемый в вершинной грамматике HPSG. В~данной 
работе обосновывается возможность применения грамматик составляющих с 
различными модификациями для автоматической обработки текстов на 
русском языке (в частности, для перевода с русского языка на другой язык).

\section{Современные подходы к~проблеме машинного перевода}
     
     В области машинного перевода существуют два основных направления 
исследований~--- подход на основе правил и статистический подход. 
Системы, созданные в рамках подхода на основе правил, включают в себя 
компоненты, отвечающие за последовательный морфологический, 
синтаксический и семантический анализ предложений исходного языка и 
синтез предложений целевого языка (с прохождением тех же уровней). 
Создание таких сис\-тем требует многолетней кропотливой работы 
лингвистов, так как для функционирования сис\-те\-мы необходим словарь с 
подробными син\-так\-ти\-ко-се\-ман\-ти\-че\-ски\-ми описаниями словарных 
единиц и правила анализа и синтеза предложений (морфологического, 
синтаксического и семантического уров\-ней). Достоинствами данного 
подхода являются высокое качество перевода, соответствие теоретическим 
концепциям и возможность удобного внесения изменений. Недостатками 
являются большие трудозатраты для создания словарей и сис\-тем правил 
перевода. 
     
     Статистический подход заключается в выявлении закономерностей 
перевода путем автоматического анализа параллельных текстов с 
использованием методов математической статистики и без использования 
лингвистических знаний. Достоинством статистического подхода является 
быстрота создания подобных сис\-тем. Для того чтобы сис\-те\-ма начала 
работать, необходим лишь текстовый корпус, переводческий словарь 
(возможно, неполный) и словарь основ. По данным из~\cite{9-mor}, 
требуется всего несколько часов для того, чтобы сис\-те\-ма начала работать, и 
1--2~недели, чтобы настроить ее и получать приемлемые результаты. 
Недостатком данного подхода является необходимость использования 
больших параллельных корпусов (от~1~млн слов) для получения 
удовлетворительных результатов перевода. Не для всех языковых пар 
существуют такие большие текстовые коллекции. Если же использовать не 
только статистические методы, но добавить и частичную лингвистическую 
разметку, размер корпуса можно существенно уменьшить (с~1~млн до 
300~тыс.\ слов)~\cite{10-mor}.
     
     Современный период развития исследований и разработок в области 
машинного перевода и сис\-тем извлечения знаний из текстов характеризует-\linebreak ся 
интенсивным процессом <<гибридизации>> подходов и моделей. Создатели 
сис\-тем, основанных на правилах, вводят в правила различные стохастические 
модели, которые позволяют отобразить\linebreak
 динамику и разнообразие языковых 
форм и значений, порождаемых в процессе речевой дея\-тель\-ности, а 
сторонники статистических методов построения лингвистических моделей 
все чаще\linebreak обращаются к подходам, основанным на лингвистических знаниях, 
рассматривая их как средства <<интеллектуализации>> сис\-тем. В~настоящее 
время появляется все больше исследований в рамках синергетического 
подхода, использующего лингвистические знания, статистические методы и 
механизмы машинного обучения~\cite{3-mor}. Как пишут авторы~\cite{11-mor}, 
<<наше убеждение состоит в том, что в долгосрочной перспективе 
самые эффективные технологии машинного перевода объединят в себе 
преимущества обоих подходов>>. По мнению авто-\linebreak ров~\cite{11-mor}, подход на основе 
правил следует применять для анализа тех уровней языка, для которых\linebreak 
существуют детальные лингвистические теории, описывающие по\-дав\-ля\-ющее 
большинство случаев, в то время как статистический подход следует 
применять для извлечения лексической и предметно-ори\-ен\-ти\-ро\-ван\-ной 
лингвистической информации, для которой пока что не существует 
разработанной теории. 
     
     Наиболее перспективными направлениями в области статистического 
машинного перевода являются перевод цепочек слов (phrase-based translation) 
и синтаксический перевод (syntax-based translation). При использовании 
метода перевода цепочек слов сопоставлению и переводу подвергаются 
цепочки слов (обычно не длиннее трех слов), выделенные путем применения 
статистических методик. Они не всегда совпадают со словосочетаниями в 
традиционном лингвистическом понимании (группа слов, взаимосвязанных 
синтаксически и семантически). Например, группа слов <<\textit{in 
accordance with the}>> является цепочкой слов, подлежащей переводу, в 
рамках статистического машинного перевода, но не является 
словосочетанием в лингвистическом смысле. При синтаксическом переводе 
сопоставлению и переводу подвергаются синтаксические поддеревья, а не 
конкретные слова или словосочетания.
     
\section{Создание системы правил для~русско-французского 
машинного перевода}

     В работе~\cite{12-mor} описывается сис\-те\-ма перевода с английского 
на французский язык, сочетающая в себе традиционный правиловый 
подход и статистический подход~--- перевод цепочек слов с использованием 
соответствий, извлеченных из параллельного текстового корпуса. В~качестве 
цепочек слов авторы предлагают использовать не любые последовательности 
слов, а только синтаксически мотивированные, другими словами, из текста 
извлекаются именные, глагольные группы, группы прилагательных и 
наречий. При выборе наилучшего варианта перевода цепочки слов 
предпочтение отдается цепочке слов, имеющей ту же самую синтаксическую 
категорию, т.\,е.\ в качестве перевода для именных групп используются 
именные группы и~т.\,д. Французско-русское направление машинного 
перевода в нашей стране развивается с самого начала исследований по 
машинному переводу. Первыми появились экспериментальные сис\-те\-мы 
фран\-цуз\-ско-рус\-ско\-го автоматического перевода ФРАП 
     (1976--1986~гг.)~\cite{13-mor} и \mbox{ЭТАП-1} (1985~г.)~\cite{14-mor}. Эти 
сис\-те\-мы были основаны на последовательном морфологическом, 
синтаксическом и семантическом анализе предложений исходного языка с 
последующим синтезом предложений целевого языка (с прохождением тех 
же уровней). В~сис\-те\-мах использовались словари с подробными 
     син\-так\-ти\-ко-се\-ман\-ти\-че\-ски\-ми описаниями слов и сис\-те\-мы 
правил анализа и синтеза предложений естественного языка.
     
     В 1990-е~гг.\ появилась первая коммерческая сис\-те\-ма автоматического 
перевода фран\-цуз\-ско-рус\-ско\-го направления \mbox{ПРОМТ}. В~основу 
архитектуры сис\-тем было положено представление процесса перевода как 
процесса с объект\-но-ориен\-ти\-ро\-ван\-ной организацией, основанной на 
иерархии обрабатываемых компонентов предложения. В~сис\-те\-мах работают 
сетевые грамматики, близкие по типу к расширенным сетям переходов, а 
также процедурные алгоритмы заполнения и трансформаций фреймовых 
структур для анализа сложных предикатов~\cite{15-mor}.
     
     Качество перевода в современных сис\-те\-мах машинного перевода 
фран\-цуз\-ско-рус\-ско\-го направления достигло высокого уровня, однако 
многие особенности синтаксиса русского языка, а также \mbox{многие} типы 
регулярных трансформаций, происходящих при переводе с русского языка на 
французский, остаются неучтенными в этих сис\-темах. 
     
     В рамках описываемых проектов разрабатывается сис\-те\-ма правил 
трансфера синтаксических структур, учитывающая возможность 
синтаксических трансформаций при переводе и многовариантность перевода. 
Козеренко была разработана многоязычная семантическая грамматика 
русского и английского языков для задач автоматической обработки 
текстов~--- МКТГ~[1--4]. Данная грамматика является разновидностью 
уни\-фи\-ка\-ци\-он\-но-по\-рож\-да\-ющей грамматики. Многовариантные правила 
функ\-ци\-о\-наль\-но-се\-ман\-ти\-че\-ско\-го переноса фразовых структур 
задают алгоритм перевода с одного языка на другой, причем учитывается 
вероятность каждого из вариантов перевода. Функциональные\linebreak значения 
языковых единиц отражены в рас\-ши\-ренной сис\-те\-ме ка\-те\-го\-ри\-аль\-но-функ\-ци\-о\-наль\-ных 
ат\-рибутов. Структуры атрибутов и значений и правила
их преобразования задаются в виде кон\-текст\-но-сво\-бод\-ных и мягко 
     кон\-текст\-но-за\-ви\-си\-мых продукционных правил. Отношения 
зависимости реализуются через механизм головных вершин фразовых 
структур, а сами фразовые структуры задают линейные последовательности 
языковых объектов. Лингвистический процессор сегментирует входные 
предложения на фразовые структуры и осуществляет трансфер этих структур 
в соответствующие им структуры целевого языка. Сегментация фразовых 
структур входного предложения проводится с учетом смысла структур, 
который при переводе должен быть передан средствами целевого языка. 
Задачей проведенных исследований ставилось создание сис\-те\-мы правил 
многовариантного трансфера для перевода с русского языка на французский.
     
     С точки зрения синтаксической структуры предложения русский и 
французский языки очень сильно отличаются друг от друга. Во французском 
языке большинство предложений двусоставны, т.\,е.\ и подлежащие, и 
сказуемые выражены на поверхностном уров\-не, причем сказуемое всегда 
выражается личной формой глагола. В~рус\-ском языке кроме канонической 
структуры <<Подлежащее (выраженное существительным в именительном 
падеже)\;+\;сказуемое (выраженное личным глаголом)>> возможны также 
другие синтаксические струк-\linebreak туры:
       
       $\bullet$~В предложении отсутствует сказуемое, выраженное глаголом в личной 
форме. Сказуемое выражено кратким причастием, кратким или полным прилагательным, 
существительным, предложной группой, инфинитивом и~др.
     
     \smallskip
     
     \noindent
     \textbf{Примеры}:
       
     \textit{Дом красив} (краткое прилагательное). \textit{Дом построен} 
(краткое причастие). \textit{Пьер~--- учащийся} (существительное).
       
       \smallskip
     
     Также к данному классу относятся случаи назывных предложений, 
состоящих из одного подлежащего, выраженного существительным в 
именительном падеже (например, заголовки) и случаи эллипсиса, когда 
глагол в личной форме <<подразумевается>>, но не выражен в 
поверхностной структуре предложения. Приведем пример эллипсиса из 
текста научного патента:
       
     \textit{Рисунки~1--4 ИЗОБРАЖАЮТ продольный разрез половины 
детали различных вариантов, соответствующих выполнению зубного 
штифта согласно первому варианту осуществления изобретения} (полная 
структура).
       
     \textit{Рисунки 5 и 6~--- продольный разрез половины детали 
моноблочного компонента протеза} (структура с эллипсисом).
     
     Подобные структуры являются трудными для синтаксического анализа 
и перевода на французский язык, так как при переводе требуется вставить 
пропущенный глагол в личной форме (глагол \textit{быть} или другой 
глагол). Многие из структур данного вида в существующих сис\-те\-мах 
автоматического перевода с русского языка на французский язык 
обрабатываются некорректно.

$\bullet$~В предложении отсутствует подлежащее, выраженное 
существительным в именительном падеже. К~данному типу предложений 
относятся безличные предложения, не\-опре\-де\-лен\-но-лич\-ные 
предложения, опре\-де\-лен\-но-лич\-ные предложения, инфинитивные 
предложения. 

\smallskip

\noindent
\textbf{Примеры:}
       
     \textit{Мне нравится работать} (безличное предложение).
       
     \textit{Маше подарили книгу} (не\-опре\-де\-лен\-но-лич\-ное 
предложение).
       
     \textit{Еду в кино} (опре\-де\-лен\-но-лич\-ное предложение).
       
     \textit{Нам бы сессию сдать} (инфинитивное предложение).
       
       \smallskip
       
     Также к данному классу относятся предложения, в которых 
подлежащее выражено инфинитивом. 
     
     \smallskip
     
     \noindent
     \textbf{Пример:}
       
     \textit{Курить~--- здоровью вредить.}
       
       \smallskip
       
     Предложения с подобной синтаксической структурой также являются 
источником значительных трудностей при автоматическом анализе и 
переводе, так как французский язык требует, чтобы в каждом предложении 
было подлежащее, выраженное существительным или местоимением без 
предлогов (функционально соответствует существительному в именительном 
падеже в русском языке). Следовательно, при переводе предложения 
русского языка, в котором нет подлежащего, выраженного существительным 
в именительном падеже, требуется восстановить подлежащее, используя 
<<формальное>> подлежащее (безличное местоимение \textit{il}) или личное 
местоимение. 
     
     Чтобы обосновать описание всех типов предложений в создаваемой 
сис\-те\-ме правил многовариантного перевода, было проведено исследование 
частоты встречаемости предложений различных типов в текстах научных 
патентов. Предложения были разделены на 3~класса.
       \begin{description}
     \item[Класс 1.] Подлежащее (выраженное существительным в 
именительном падеже)\;+\;сказуемое (выраженное глаголом в личной форме).
     \item[Класс 2.] Сказуемое, выраженное глаголом в личной форме, 
отсутствует. 
     \item[Класс 3.] Подлежащее, выраженное существительным в 
именительном падеже, отсутствует.
     \end{description}
     
     Некоторые предложения относятся одновременно и к классу~2, и к 
классу~3. Такие предложения были отнесены к классу~2.
     
     В результате распределения предложений по груп\-пам и подсчета 
относительной частоты\linebreak встречаемости предложений каждого вида в текс\-тах 
на\-уч\-ных патентов были получены следующие\linebreak результаты:
     класс~1~--- 48\%; класс~2~--- 38\%;\linebreak класс~3~--- 14\%.
     
     Как видно, предложения, относящиеся к каж\-до\-му из трех классов, 
встречаются в текстах с достаточно большой частотой, и существует 
необходимость включения правил для всех трех классов в сис\-те\-му 
многовариантного трансфера.
       
     Кроме различия структурных типов предложений французский и 
русский языки также очень существенно различаются между собой в аспекте 
порядка слов в предложении. Во французском языке большинство 
предложений имеют канонический порядок слов:
       
\begin{center}
 \textit{Подлежащее\,--\,сказуемое\,--\,прямое дополнение\,--\,косвенные 
дополнения}.
\end{center}
     
     В русском языке данный порядок регулярно нарушается как в устной, 
так и в письменной речи. Несоответствие порядка слов в предложениях на 
русском и французском языке создает необходимость изменения порядка 
слов при переводе. 
     
     При создании сис\-те\-мы правил перевода будем использовать 
классификации предложений русского языка, изложенные в учебниках по 
русскому синтаксису~\cite{16-mor, 17-mor}, а также типичные переводческие 
трансформации, описанные в учебниках по переводу с французского языка 
на русский~\cite{18-mor}.
     
     Рассмотрим наиболее частотные типы предложений русского языка, 
которые создают трудности при переводе, на материале патентных текстов.

$\bullet$~Безличные предложения. Будем понимать под безличным 
предложением такое предложение, которое содержит глагол в личной форме, 
но не содержит существительного в именительном падеже, которое 
выполняло бы роль подлежащего. Безличным предложениям русского языка 
соответствуют предложения с безличным местоимением \textit{il} 
французского языка. Пример перевода:

     \textit{Однако} {\bfseries\textit{оказалось}}, \textit{что эта прочность 
может в конечном счете привести к нарушению надежности 
соединения}.\;$\rightarrow$\;\textit{Toutefois}, {\bfseries\textit{il est apparu}} 
\textit{que cette robustesse pouvait finalement porter atteinte}  
$\grave{\mbox{\textit{a}}}$ \textit{la fiabilit}$\acute{\mbox{\textit{e}}}$ \textit{de 
la liaison}.
{\looseness=1

}
     
     Правило переноса выглядит следующим образом:
     \begin{multline*}
     \mathrm{V[Person~3, Number SG, Gender NEUTR]} \&{}\\
     {}\&  \mathrm{NO  NP[Case NOM]} \rightarrow \mathrm{Il} +{}\\
     {}+\mathrm{ V[Person 3, Number SG, Gender  MASC]}\,.
     \end{multline*}

$\bullet$~Неопределенно-личные предложения.

\medskip

\noindent
\textbf{Пример:}
     
\begin{center}
     \textit{Указанное раструбное соединение осуществляют}~[$\ldots$]. 
     \end{center}
     
     При переводе на французский язык чаще всего используется пассивная 
конструкция:

\begin{center}
          \textit{Cet embo}$\hat{\iota}$\textit{tement est 
effectu}$\acute{\mbox{\textit{e}}}$ [$\ldots$].
\end{center}
     
     Возможен и другой вариант перевода, с использованием безличного 
местоимения \textit{on}:
     
\begin{center}
     \textit{On effectue cet embo}$\hat{\iota}$\textit{tement} [$\ldots$].
     \end{center}
     
     Правило переноса выглядит следующим образом:
     \begin{multline*}
     \mathrm{NP[Case Acc] + V[Person 3, Number PL]}\rightarrow {}\\
     {}\rightarrow  \left \{\mathrm{NP^* + V(be)^*} +{}\right.\\
\left.{}+ \mathrm{V[FORM PART, TENSE PAST]^*}\right\} \\
\mathrm{OR} \left\{\mathrm{On + V[Person 3, Number SG] + NP}\right\}\,.
\end{multline*}
     
     Знак $^*$ означает согласование морфологических признаков.
     
\begin{figure*}[b] %fig1
\vspace*{6pt}
\begin{center}
\mbox{%
\epsfxsize=159.941mm
\epsfbox{mor-1.eps}
}
\end{center}
\vspace*{-9pt}
\Caption{Распределение по частям речи в русских и французских научных текстах 
патентных рефератов: (\textit{а})~реферат патента WO2004009333; 
(\textit{б})~реферат патента WO2004017987;
\textit{1}~--- предложения; \textit{2}~--- строки; \textit{3}~--- 
существительные; \textit{4}~--- глаголы; \textit{5}~--- причастия;
\textit{6}~--- деепричастия (герундии)}
%\end{figure*}
%\begin{figure*} %fig2
\vspace*{12pt}
     \begin{center}
     {\tabcolsep=3pt
     \begin{tabular}{lcl}
     \textbf{Цель, назначение} &&\\
     \textit{Русский язык} &&\textit{Французский язык}\\
     \textbf{Существительное (98)} &&\textbf{Инфинитив (72)}\\
     Инфинитив (2)  &$\leftarrow\,\rightarrow$ &Существительное (26)\\
     Придаточное предложение (0)&&Придаточное предложение (2)
     \end{tabular}
     }
          \end{center}
          \vspace*{-6pt}
\Caption{Правила когнитивного переноса для функциональных значений цели и 
назначения}
\end{figure*}


     
\section{Трансфер пропозиционального ядра в~русско-французской 
языковой паре}
       
     Основу семантико-синтаксической структуры предложения составляет 
пропозициональное ядро, прежде всего языковые средства предикации. Были 
изучены структуры когнитивного переноса в \mbox{рамках} поля функционального 
переноса (ПФП) первичной и вторичной предикации для 
     рус\-ско-фран\-цуз\-ской языковой пары по аналогии с 
     рус\-ско-анг\-лий\-ской языковой парой. Были выделены базовые 
правила когнитивного переноса для различных функциональных значений 
(частотные характеристики были выделены на основании анализа патентных 
текстов). Материалом анализа послужили параллельные тексты патентов 
и/или рефератов патентов на русском и французском языках, взятые из базы 
данных Роспатента.
     
     Сравнение русских и французских текстов рефератов научных 
патентов показало, что доля дей\-ст\-ви\-тель\-но параллельных текстов в них 
составляет примерно 30\%. Остальные тексты можно \mbox{назвать}\linebreak 
     ког\-ни\-тив\-но-со\-по\-ста\-ви\-мы\-ми, причем объем русского текста 
может превышать объем французского на две трети. Однако распределение 
по частям речи в русских и французских научных текстах патентных 
рефератов (и самих патентов) очень близко по составу и объему, что 
отражено на рис.~1. Русский текст в целом на 30\%--35\% более 
номинативен, чем французский, в котором в поле вторичной предикации 
предпочтение отдается инфинитиву (в русском~--- отглагольным 
существительным).
     
     В первом примере тексты рефератов не параллельные, а 
     когни\-тив\-но-со\-по\-ста\-ви\-мые, во втором тексты русского и 
французского реферата параллельны: перевод выполнен точно, почти 
дословно. В~любом случае, как видно из примеров, и в русских, и во 
французских патентных текстах очень высока доля именных групп, что 
вообще всегда характерно для на\-уч\-но-тех\-ни\-че\-ских текстов.
     
     Правила когнитивного переноса для функциональных значений цели и 
назначения представлены на рис.~2.
     

     Таким образом, набор структур, используемых для выражения цели 
действия, одинаков для русского и французского языка, однако французский 
тяготеет к инфинитивной структуре, а русский~--- к именной (\textit{Для 
увеличения способности сети к обобщению}$\ldots$~/ \textit{Afin d'augmenter 
la capacit$\acute{\mbox{\textit{e}}}$ du r$\acute{\mbox{\textit{e}}}$seau de 
g$\acute{\mbox{\textit{e}}}$n$\acute{\mbox{\textit{e}}}$raliser}\ldots).
     
     \smallskip
     
     \noindent
     \textbf{Примеры.}
     \begin{enumerate}
     \item $[$Cat~: VerbNoun$]$ \{для распознавания\} \{pour la reconnaissance\}~--- 
предложная группа: предлог\;+\;существительное.
     \item  $[$Cat~: VerbInf$]$ \{\textit{чтобы распознать}\} \{\textit{afin de 
reconn$\hat{\mbox{\textit{a}}}$itre}\}~--- союз\;+\;инфинитив.
     \item $[$Cat : Sentence$]$ \{\textit{чтобы распознавание было 
эффективным}\} \{\textit{pour que la reconnaissance soit\linebreak
efficace}\}~--- 
придаточное предложение, присоединяемое подчинительной связью (союзом\linebreak 
цели). При трансформации русского отглагольного существительного во 
французский инфинитив необходимо сделать выбор между его активной и 
пассивной формой. Видимо, в рамках сис\-те\-мы автоматического перевода 
данный выбор лучше всего осуществляется с применением статистических 
данных (активный инфинитив встречается в текстах намного чаще 
пассивного; в анализируемых текстах французский пассивный инфинитив в 
качестве перевода русского отглагольного существительного встретился в 
13\% случаев).
     \end{enumerate}
     
\section{Заключение}
     
     Были описаны предикативные синтаксические структуры русского 
языка, принадлежащие к функционально-семантическому полю первичной и 
вторичной предикации и соответствующие\linebreak им синтаксические структуры 
французского языка. Была составлена подробная классификация ти-\linebreak пов 
предложений русского языка с точки зрения синтаксиса и соответствующих 
им синтаксических типов во французском языке, которая может быть 
использована при написании правил переноса синтаксических структур, 
происходящего при переводе с русского языка на французский. 
В~классификации учтено синтаксическое многообразие русского языка: 
назывные предложения, безглагольная предикация (в случае невыраженного 
глагола <<быть>> в настоящем времени), безличные предложения, 
     опре\-де\-лен\-но-лич\-ные предложения, не\-опре\-де\-лен\-но-лич\-ные 
предложения, двусоставные предложения с различным типом ска\-зу\-емых 
(глагольные, именные и~пр.). 
     
     Были изучены категориальные трансформации предикативных 
структур, происходящие при переводе с русского языка на французский и в 
обратном\linebreak
направлении. Моделирование трансформаций\linebreak
предикативных 
структур для задачи машинного перевода является актуальной задачей, так 
как это явление мало исследовано с точки зрения компьютерной реализации 
и недостаточно учтено в дей\-ст\-ву\-ющих сис\-те\-мах машинного перевода. Кроме того, правила, 
задающие функциональную синонимию языковых конструкций, могут 
использоваться также при машинном обучении на корпусе параллельных 
текстов, позволяя избежать формирования избыточных правил и <<шумов>>. 
     
     Дальнейшие исследования будут направлены на уточнение сис\-те\-мы 
синтаксических соответствий с помощью параллельного корпуса текстов 
научных патентов, а также на расширение числа типов трансформаций при 
рус\-ско-фран\-цуз\-ском машинном переводе, дальнейшее изучение 
     дис\-три\-бу\-тив\-но-транс\-фор\-ма\-ци\-он\-ных характеристик 
языковых структур и сбор статистической информации.

{\small\frenchspacing
{%\baselineskip=10.8pt
\addcontentsline{toc}{section}{Литература}
\begin{thebibliography}{99}

      
     \bibitem{1-mor}
     \Au{Козеренко Е.\,Б.}
     Моделирование переноса функциональных значений для 
     анг\-ло-рус\-ско\-го машинного перевода~// Компьютерная лингвистика 
и интеллектуальные технологии: Труды Междунар. конф. Диалог'2004.~--- 
М.: Наука, 2004.
     
     \bibitem{2-mor}
     \Au{Kozerenko E.\,B.}
     Cognitive approach to language structure segmentation for machine 
translation algorithms~// Conference (International ) on Machine Learning, 
Models, Technologies and Applications Proceedings.~--- Las Vegas, USA, 2003. 
     P.~49--55.
     
     \bibitem{4-mor} %3
     \Au{Козеренко Е.\,Б.}
     Функционально-семантические инварианты для алгоритмов 
синтаксического анализа и разметки полнотекстового научного документа~// 
Системы и средства информатики.~--- М.:\ Наука, 2003. Вып.~13. 
     С.~298--312.
     
     \bibitem{3-mor} %4
     \Au{Козеренко Е.\,Б.}
     Лингвистическое моделирование для сис\-тем машинного перевода и 
обработки знаний~// Информатика и её применения, 2007. Т.~1. Вып.~1. 
     С.~54--66.
          
          \bibitem{5-mor}
     \Au{Sag I., Wasow Th., Bender E.\,M.}
     Syntactic theory: A formal introduction.~--- Stanford: CSLI Publications, 
2003.
     
     \bibitem{6-mor}
     \Au{Chomsky N., Lasnik H.}
     The theory of principles and parameters~// The minimalist program.~--- 
Cambridge: MIT Press, 1995.
     
     \bibitem{7-mor}
     \Au{Oepen S., Toutanova K., Shieber~S., Manning~C., Flickinger~D., 
Brants~T.}
     The LinGO Redwoods Treebank: Motivation and preliminary 
applications~// 19th Conference (International) on Computational Linguistics 
Proceedings.~--- Taipei, Taiwan, 2002. P.~1253--1257.
     
     \bibitem{8-mor}
     \Au{Перекрестенко А.}
     Разработка и программная реализация сис\-те\-мы автоматического 
выделения синтаксических групп для естественных языков~// Системы и 
средства информатики.~--- М.: Наука, 2007. Вып.~17. С.~273--291. 
      
      \bibitem{9-mor}
      \Au{Brown R.\,D.}
      Example-based machine translation in the Pangloss system~// 16th 
Conference (International) on Computational Linguistics (COLING-96) 
Proceedings.~--- Copenhagen, Denmark, 1996. P.~169--174.
      
      \bibitem{10-mor}
      \Au{Brown R.\,D.}
      Adding linguistic knowledge to a lexical example-based translation 
system~// 8th Conference (International) on Theoretical and Methodological Issues 
in Machine Translation Proceedings.~--- Chester, UK, 1999. P.~22--32.
     
     \bibitem{11-mor}
     \Au{Grishman R., Kosaka~M.}
     Combining rationalist and empiricist approaches to machine translation~// 
4th Conference (International) on Theoretical and Methodological Issues in 
Machine Translation Proceedings.~--- Montreal, Canada, 1992. P.~263--274.
     
     \bibitem{12-mor}
     \Au{Dugast L., Senellart J., Koehn~P.}
     Selective addition of corpus-extracted phrasal lexical rules to a rule-based 
machine translation system~// 12th Machine Translation Summit Proceedings.~--- 
Ottawa, ON, Canada, 2009. P.~222--229.
     
     \bibitem{13-mor}
     \Au{Леонтьева~Н.\,Н., Никогосов~С.\,Л.}
     Система ФРАП как информационная сис\-те\-ма~// Актуальные вопросы 
практической реализации сис\-тем автоматического перевода.~--- М.: МГУ, 
1982. С.~134--166.
     
     \bibitem{14-mor}
     \Au{Апресян~Ю.\,Д., Богуславский~И.\,М., Иомдин~Л.\,Л.\ и~др.}
     Лингвистическое обеспечение сис\-те\-мы фран\-цуз\-ско-рус\-ско\-го 
автоматического перевода ЭТАП-1. 1.~Общая характеристика сис\-те\-мы~// 
Теория и модели знаний (Теория и практика создания сис\-тем 
искусственного интеллекта): Труды по искусственному интеллекту. Ученые 
записки Тартуского гос. ун-та.~--- Тарту, 1985. 
Вып.~714. С.~20--39.
     
     \bibitem{15-mor}
     \Au{Соколова~С.}
     Как переводит компьютер. {\sf 
http:// www.translationmemory.ru/technology/articles/article\_\linebreak Sokolova.php}.
     
     \bibitem{16-mor}
     \Au{Валгина Н.\,С.}
     Синтаксис современного русского языка.~--- М.: Агар, 2000.  416~с.
     
     \bibitem{17-mor}
     \Au{Шелякин М.\,А.}
     Справочник по русской грамматике.~--- М.: Дрофа, 2006.  355~с.
     
     \label{end\stat}
     
     \bibitem{18-mor}
     \Au{Гак В.\,Г., Григорьев Б.\,Б.}
     Теория и практика перевода: Французский язык.~--- СПб.: 
Интердиалект+, 2000. 456~с.
 \end{thebibliography}
}
}


\end{multicols}          %12
\def\stat{kozerenko}

\def\tit{КОГНИТИВНО-ЛИНГВИСТИЧЕСКИЕ ПРЕДСТАВЛЕНИЯ 
В~СИСТЕМАХ ОБРАБОТКИ ТЕКСТОВ}

\def\titkol{Когнитивно-лингвистические представления 
в~системах обработки текстов}

\def\autkol{Е.\,Б.~Козеренко, И.\,П.~Кузнецов}
\def\aut{Е.\,Б.~Козеренко$^1$, И.\,П.~Кузнецов$^2$}

\titel{\tit}{\aut}{\autkol}{\titkol}

%{\renewcommand{\thefootnote}{\fnsymbol{footnote}}\footnotetext[1]
%{Работа выполнена при поддержке Российского фонда фундаментальных
%исследований, проект~10-01-00480. Статья написана на основе материалов доклада, 
%представленного на IV Международном семинаре <<Прикладные задачи теории вероятностей 
%и математической статистики, связанные с моделированием информационных систем>> 
%(зимняя сессия, Аоста, Италия, январь--февраль 2010 г.).}}

\renewcommand{\thefootnote}{\arabic{footnote}}
\footnotetext[1]{Институт проблем информатики Российской академии наук, kozerenko@mail.ru}
\footnotetext[2]{Институт проблем информатики Российской академии наук, igor-kuz@mtu-net.ru}


\Abst{Рассмотрены вопросы проектирования и развития 
семантико-синтаксических и лексико-семантических представлений в 
лингвистических процессорах ряда систем, основанных на аппарате расширенных 
семантических сетей (РСС). Системы этого класса создаются для извлечения знаний из 
текстов на естественных языках, отображения извлеченных сущностей и связей в 
структуры базы знаний (БЗ) и использования знаний для поддержки экспертных 
аналитических решений в различных сферах приложения. В~фокусе внимания 
находятся ин\-же\-нер\-но-линг\-ви\-сти\-че\-ские представления, позволяющие 
построить целостную работающую лингвистическую модель, которая 
модифицируется в зависимости от конкретной задачи: от <<тяжелой>> формы на 
основе детальных глубинных представлений до фокусных редуцированных 
оболочек, настроенных на узкую предметную область (ПО) и ограниченный язык 
общения. Особое внимание уделяется способам описания 
дис\-три\-бу\-тив\-но-транс\-фор\-ма\-ци\-он\-ных признаков языковых объектов.}

\KW{интеллектуальные системы; семантические представления; лингвистические 
процессоры; обработка естественного языка; извлечение знаний}

       \vskip 14pt plus 9pt minus 6pt

      \thispagestyle{headings}

      \begin{multicols}{2}

      \label{st\stat}

\section{Введение}

     Данная работа посвящена проблемам создания\linebreak 
     когни\-тив\-но-линг\-ви\-сти\-че\-ских моделей естественного языка для 
различных классов информационных систем и описанию опыта создания 
линг\-ви\-сти\-че\-ских представлений для интеллектуальных\linebreak технологий 
обработки текстов. Вопросы извлечения знаний из текстов и создания модели 
естественного языка рассматриваются в единстве. В центре внимания будут 
находиться лингвистические процессоры интеллектуальных систем, 
разработанных на основе аппарата \textit{расширенных семантических 
сетей}~[1--5]. %\cite{1koz}--\cite{3koz}, \cite{18koz}--\cite{19koz}. 
Будем 
называть их \textit{РСС-сис\-те\-мы}. Эти системы создавались коллективом 
разработчиков, включая авторов данной статьи в Институте проб\-лем 
информатики РАН на протяжении целого ряда лет в рамках 
исследовательских проектов и прикладных систем, ориентированных на 
конкретные ПО заказчиков. Можно выделить четыре 
поколения РСС-систем. Ко\-гни\-тив\-но-линг\-ви\-сти\-че\-ские 
представления, заложенные в основу систем этого класса, прошли 
определенный эволюционный путь. 
     
     Интеллектуальные РСС-сис\-те\-мы содержат развитые \textit{базы 
знаний}, при этом знания представлены в виде записей на языке 
РСС, называемых 
     \textit{РСС-струк\-ту\-ра\-ми}. Лингвистические знания, таким 
образом, являются частным случаем <<знаний>> и также представлены в 
виде записей на языке РСС. Основным 
конструктивным элементом РСС\linebreak является именованный $N$-мест\-ный 
предикат, на\-зы\-ва\-емый <<\textit{фрагментом}>>. Все множество языковых 
объектов задается в виде системы пре\-ди\-кат\-но-ак\-тант\-ных структур, при этом 
поддерживаются механизмы представления вложенных структур, что дает 
очень мощные изобразительные возможности для описания объектов 
различных языковых уровней. Очень важными факторами являются 
однородность и единообразие лингвистических представлений. 
     
     В процессе анализа и синтеза предложений естественного языка 
используется фор\-маль\-но-грам\-ма\-ти\-че\-ский аппарат, сходный с 
грамматиками зависимостей. При этом подходе опорными элементами 
служат слова и конструкции, выполняющие роль предикатов в предложении, 
и результатом анализа предложения должен стать один предикат, 
соответствующий сказуемому рассматриваемого предложения (т.\,е.\ 
основному глаголу в личной форме или другому основному предикатному 
выражению). Таким образом, в процессе анализа происходит выявление 
\textit{когнитивных опор} предложения: <<слов-дейст\-вий>> и 
     <<слов-от\-но\-ше\-ний>>, т.\,е.\ глаголов и других слов, имеющих 
синтактико-семантические валентности. Примером <<слов-от\-но\-ше\-ний>> 
могут служить, например, слова <<отец>>, <<друг>> и~т.\,п., т.\,е.\ в данном 
случае <<отношения>> (или \textit{функции}~--- в терминах языка логики 
предикатов 1-го порядка)~--- это слова, которые задают сильные, четко 
выраженные син\-так\-ти\-ко-се\-ман\-ти\-че\-ские ожидания. 
     
     Семантический анализ в ин\-же\-нер\-но-линг\-ви\-сти\-че\-ском 
понимании~--- это процесс перевода ес\-тест\-вен\-но-язы\-ко\-вых 
выражений во <<внутренние>> структуры БЗ, в 
рассматриваемой ситуации этими <<внутренними>> структурами являются 
записи на языке РСС. Таким образом, структуры БЗ~--- это код смысла в 
интеллектуальных информационных системах подобного рода. 
     
     В работе рассматриваются ин\-же\-нер\-но-линг\-ви\-сти\-че\-ские 
решения в системах с <<пол\-ным>> линг\-ви\-сти\-че\-ским анализом~--- это 
     сис\-те\-мы 1-го и 2-го поколения: ДИЕС1, ДИЕС2, 
     Логос-Д~\cite{2koz, 3koz}~--- и сис\-те\-мах с <<фактографическим>> 
подходом: интеллектуальных системах поддержки аналитических решений 
(ИСПАР)~\cite{18koz, 19koz}, где целью анализа является выделение 
сущностей и связей из текстов,~--- это системы 3-го и 4-го поколения. 

\section{Процесс концептуально-лингвистического моделирования 
в системах, основанных на аппарате расширенных семантических сетей}
     
\subsection{Центральные вопросы семантического моделирования} %2.1
     
     Концептуально-лингвистическое моделирование (КЛМ)~--- это 
процесс построения ес\-тест\-вен\-но-язы\-ко\-вой модели ПО (рис.~1), синтезирующий в себе подходы 
концептуального и лингвистического моделирования~[1--3]. 
По\-стро\-ение концептуально-лингвистической модели некоторой 
ПО подразделяется на следующие этапы:
     \begin{itemize}
     \item построение собственно концептуальной модели, т.\,е.\ вычленение 
базовых понятий, организация их в ро\-до-ви\-до\-вые деревья и определение 
связей между ними;
     \item разработка идеографического словаря ПО, т.\,е.\ 
лексическое наполнение концептуальной модели;
     \item ввод базовых правил, описывающих на естественном языке 
<<модель мира>>, релевантную данной ПО.
     \end{itemize}
     
     
     Методика КЛМ на 
основе аппарата РСС базируется на следующих принципах:
     \begin{itemize}
\item модель должна быть <<открытой>>, т.\,е.\ поддерживать эффективный 
механизм расширения и обновления информации;
\begin{center} %fig1
%\vspace*{3pt}
\hspace*{-10.7158pt}\mbox{%
\epsfxsize=77.871mm
\epsfbox{koz-1.eps}
}\hspace{10.7158pt}
%\end{center}
\vspace*{4pt}
%\begin{center}
{{\figurename~1}\ \ \small{Процесс КЛМ}}
\end{center}
\vspace*{3pt}

%\bigskip
\addtocounter{figure}{1}
\item модель представления <<смысла>> должна учитывать факты 
экстралингвистической реаль\-ности, которые в виде правил и отношений 
составляют некоторую базовую <<модель мира>>, достраиваемую 
конкретными моделями ПО;
\item модель должна быть практичной, т.\,е.\ не перегруженной детальными 
описаниями связей и отношений между понятиями, чтобы обеспечить 
возможность ее реализации, но в то же время отражать всю релевантную 
конкретной задаче информацию.
\end{itemize}

     \begin{figure*} %fig2
%     \begin{center}
\hspace*{23mm}\{(ВЫРАБАТЫВА895\_\_)(DICSEM)\\
\hspace*{23mm}COORD(PROGNOZ1,RUS,ВЫРАБАТЫВА895\_\_,S50\_31\_51\_20,\%)\\
\hspace*{23mm}SUB(UNIV,0+)~SUB(UNIV,1+)~SUB(UNIV,2+)\\
\hspace*{23mm}ВЫРАБАТЫВ(0-,1-,2-/3+)~INFI(3-)~ПРИДЕТСЯ(3-)~ПРИДЕТСЯ(3$-$/4+) \\
\hspace*{23mm}FUT1(4$-$)~SUB(СРЕД,5+)
%\end{center}
%\vspace*{2pt}
\Caption{Пример записи представления глагола <<вырабатывать>> в семантическом 
словаре
\label{f2koz}}
%\vspace*{6pt}
\end{figure*}

     Реалистичный подход к постановке задачи диктует необходимость 
ограничения моделируемого подмножества естественного языка. Суть 
ограничений сводится к следующему:
     \begin{enumerate}[(1)]
     \item анализируемые текстовые материалы содержат 
экспертные знания из конкретных ПО (в разработанных 
авторами системах это были такие ПО, как диагностика 
брака при изготовлении микросхем, социальное прогнозирование, 
криминалистика и другие);
     \item в целях максимально возможного устранения 
неоднозначности словарь строится по модульному принципу: есть некоторая 
наиболее общая часть (1--2~уровня), которая достраивается специальными 
словарями для каж\-дой отдельной~ПО.
     \end{enumerate}
     
     Предлагаемая модель лексической семантики основана на принципе 
<<ядерного>> значения, реализуемого в контексте данной 
ПО, с последующим индуктивным наращиванием других значений (если 
они актуализируются в рас\-смат\-ри\-ва\-емых контекстах). Также используется 
таксономия, которая реализуется в виде иерархических деревьев классов 
слов. 
     
     Общая <<модель мира>> системы является основой для моделей ПО. 
Элементами этой модели служат классы слов, которые подразделяются на 
понятия/имена, отношения, действия, свойства, характеристики действий, 
временные и пространственные характеристики.
     
     Самым общим понятием является \textit{концепт}, или 
\textit{универсальный класс}, который подразделяется на объект, ситуацию, 
процесс и~др. 
     
     Слова, относящиеся к классам действий и отношений, представлены 
как се\-ман\-ти\-ко-син\-так\-си\-че\-ские фреймы, задающие 
     пре\-ди\-кат\-но-ак\-тант\-ные структуры (модель управления). Однако 
в описываемом подходе (назовем его РСС-под\-хо\-дом) существенно 
расширена область значений актантов. Суть расширения состоит, во-первых, 
в том, что в роли актантов могут выступать не только простые объекты, 
соответствующие отдельным словам, но и структурные объекты, 
представляющие словосочетания и фразы, а во-вторых, в том, что понятие 
падежа включает в себя не только семантические, но и синтаксические 
признаки.
     
     Подход, основанный на РСС, позволяет отражать произвольный 
уровень вложенности структур за счет пропозициональных вершин 
семантической сети. Это обеспечивает представление\linebreak сложных 
синтаксических конструкций фраз\linebreak естественного языка, а также позволяет 
отразить\linebreak структурный характер лексической семантики,\linebreak которая в 
предлагаемой модели имеет иерар\-хи\-че\-ски-се\-те\-вую структуру. 
Линг\-ви\-сти\-че\-ские зна-\linebreak ния пред\-став\-ле\-ны в системном словаре и 
декла\-ра\-тивных модулях линг\-ви\-сти\-че\-ско\-го процессора.\linebreak В РСС-сис\-те\-мах 
так\-же реализована функция динамически форми\-ру\-емо\-го семантического 
словаря, который на основе исходной лингвистической информации 
достраивается системой автоматически в процессе об\-ра\-бот\-ки конкретных 
текстов. На рис.~\ref{f2koz} пред\-став\-ле\-но \mbox{такое} <<внутреннее>> описание 
глагола в семантическом словаре. Этот словарь автоматически генерируется 
РСС-системами ДИЕС2, ЛОГОС-Д, ИКС в процессе обработки 
     естест\-вен\-но-язы\-ко\-вых \mbox{текстов}. 
     {\looseness=1
     
     }
     
     
\subsection{Особенности применения аппарата расширенных семантических сетей 
в~когнитивно-лингвистическом моделировании} %2.2
     
     Дадим краткое описание аппарата РСС и  
обос\-ну\-ем выбор именно этого метода представления для моделирования 
естественного языка. Классическое понятие семантической сети сводится к 
следующему: задаются некоторые вершины, соответствующие объектам,  
вершины связываются дугами, которые помечаются именами отношений. 
Однако с помощью подобных сетей оказывается трудно представлять 
сложные виды информации, например, когда объекты, связанные 
отношениями, образуют агрегаты и когда отношения связываются между 
собой отношениями и~др. Поэтому в сети вводятся вершины, 
соответствующие именам отношений, а также специальный композиционный 
элемент, называемый вершиной связи. Вершина связи как бы <<разрывает>> 
дугу и подсоединяется одним ребром к вершине-отношению, а другими 
ребрами~--- к вершинам-объектам. Расширенная семантическая сеть является развитием такого сорта 
сетей в направлении повышения изобразительных возможностей при 
сохранении свойства однородности.
     
     Основой РСС является множество вершин ($V$), из которых 
составляются элементарные фрагменты (ЭФ) вида
     $
     V_0(V_1,V_2,\ldots ,V_k/V_{k+1})
     $, 
     где
$V_0, V_1, V_2,\ldots , V_k, V_{k+1}>0$.
     
     
     Такой фрагмент представляет $k$-местное отношение. Позиции 
вершин в ЭФ определяют их роли. 
Вершина~$V_0$ ставится в соответствие имени отношения, 
вершины~$V_1$, $V_2$, \ldots , $V_k$~--- объектам, участ\-ву\-ющим в 
отношении, а вершина~$V_{k+1}$, отделенная косой линией,~--- всей 
совокупности упомянутых объектов с учетом их отношения. В~дальнейшем 
будем $V_{k+1}$ называть $C$-вершиной ЭФ.\linebreak 
Множество ЭФ образует РСС. 
С~помощью РСС представляются наборы отношений, различные ситуации, 
сце\-нарии. Сильной стороной РСС-под\-хо\-да является возможность 
однородного пред\-став\-ле\-ния как предметной (концептуальной), так и 
лингвистической информации, что обеспечивает эффективную обработку 
знаний и поддержание непротиворечи\-вости~БЗ.
          \begin{figure*} %fig3
     \vspace*{1pt}
\begin{center}
\mbox{%
\epsfxsize=125.039mm
\epsfbox{koz-3.eps}
}
\end{center}
\vspace*{-9pt}
     \Caption{Семантико-синтаксический анализ без выявления глагольных 
словоформ
      \label{f3koz}}
\vspace*{12pt}
 %     \end{figure*}
%            \begin{figure*} %fig4
           \vspace*{1pt}
\begin{center}
\mbox{%
\epsfxsize=103.129mm
\epsfbox{koz-4.eps}
}
\end{center}
\vspace*{-9pt}
      \Caption{Целостная семантическая структура предложения
      \label{f4koz}}
      \end{figure*}

     
     Посредством РСС в БЗ представлены лингвистические  и 
предметные знания. Обработка этих знаний осуществляется 
продукциями языка ДЕКЛ, на котором реализованы сле\-ду\-ющие шесть 
блоков: морфологического анализа, семанти\-ческого анализа слов, 
син\-так\-ти\-ко-се\-ман\-ти\-че\-ско\-го анализа форм, 
прагматических функций, организации системной активности и 
обратный лингвистический процессор. С~помощью продукций 
осущест\-вля\-ет\-ся последовательное преобразование сети~--- РСС. При этом 
проходятся фазы, соответствующие уровню понимания входного текста. 
Рас\-смот\-рим~их.
     \begin{enumerate}[1.]
     \item На первом шаге анализа строится 
пространственная структура предложения с морфологической информацией 
для каждого слова.\linebreak Каж\-дый член предложения представляется вершиной 
семантической сети. Вместо слова генерируется код (если слово 
многозначно, т.\,е.\ принадлежит к нескольким классам,~--- то более одного 
кода). Основой кода служит корень слова. На этом этапе предложение 
представляется в виде набора фрагментов типа LRR (специальных меток 
результатов 1-го этапа анализа), объединяемых в целостную структуру 
посредством вершины связи. Результат 1-го этапа постоянно обращается к 
словарю: <<Что значит данное слово?>>
     \item На втором этапе каждой вершине сопоставляется семантический 
класс и присваивается новый код. За словами (т.\,е.\ конкретными вершинами 
РСС) система видит объекты, действия, свойства, т.\,е.\ строит 
классификации. Производится се\-ман\-ти\-ко-син\-так\-си\-че\-ский анализ 
без выявления глагольных словоформ, при этом предложение представляется 
в виде совокупности фрагментов типа SEM и SEMD~--- специальных меток 
результатов 2-го этапа анализа (рис.~\ref{f3koz}).
     \item На третьем этапе происходит частичное <<сворачивание>> 
синтаксических структур в более компактные (например, свойство объекта и 
сам объект) с присваиванием нового кода и строится фрагмент для объекта, 
обладающего этим свойством.
     \begin{figure*}[b] %fig5
          \vspace*{12pt}
\begin{center}
\mbox{%
\epsfxsize=147.485mm
\epsfbox{koz-5.eps}
}
\end{center}
\vspace*{-9pt}
     \Caption{Глубинная структура предложений
      \label{f5koz}}
      \end{figure*}      
     \item На четвертом этапе выявляются отношения и действия и 
производится анализ непосредственного контекста на соответствие заданным 
семантическим падежам. Система проверяет, подходят ли объекты 
(концепты, понятия) на аргументные места данного действия или отношения. 
При этом отглагольные существительные (<<делатель>>, т.\,е.\ агент 
действия, или <<делание>>~--- процесс~--- анализируются как слова с 
двойной природой: вначале как действия, а затем как объекты). Результатом 
этого этапа является целостная семантическая структура предложения, 
которая представляется фрагментом типа SEMSTR~--- метки результата 4-го 
этапа анализа (рис.~\ref{f4koz}).
     \item На пятом этапе происходит анализ прагматики: установление 
кореференциальных отношений, частичное восстановление эллиптических 
конструкций, система производит дальнейшие действия с построенными 
фрагментами.
     \end{enumerate}

     
Система ДИЕС допускает ввод полисемичных форм глаголов. Для этого следует 
воспользоваться формальной записью лингвистических знаний. 
     В~сис\-те\-мах, основанных на РСС, все функции реализованы на 
единой основе~--- в рамках языков РСС и ДЕКЛ, которые были разработаны 
с ориентацией на задачи обработки естественного языка.

%\vspace*{-6pt}

\section{Представление семантики глаголов, глубинные 
и~поверхностные структуры}
     
     В процессе анализа выявляются семантические вершины предложения: 
происходит выявление <<слов-дей\-ст\-вий>>, т.\,е.\ глаголов, и 
     <<слов-от\-но\-ше\-ний>>. Что же является конструктивной основой\linebreak 
задания семантических представлений предикатных слов и выражений? Как 
убедительно показано в работе~\cite{4koz}, семантика глагола 
определяется его дис\-три\-бу\-тив\-но-транс\-фор\-ма\-ци\-он\-ны\-ми\linebreak 
свойствами. Поэтому смысл предикатных выражений должен кодироваться с 
учетом их дистрибутивных и трансформационных признаков. 
     
     Выдвинутая рядом лингвистов (Хомский, Филлмор) гипотеза о том, что 
все предложения имеют глубинные и поверхностные 
     структуры~[7--10], явилась очень продуктивным 
источником проектных решений при создании первых РСС-сис\-тем и 
развивалась в дальнейшем. 

В~тео\-ре\-ти\-ко-линг\-ви\-сти\-че\-ском 
понимании глубинная структура~--- это абстракция, содержащая все 
элементы, необходимые для образования поверхностных структур 
предложений со сходной семантикой. 

     В~ин\-же\-нер\-но-линг\-ви\-сти\-че\-ском понимании\linebreak глубинная 
структура~--- это запись на языке БЗ, например на языке РСС, 
которая может быть представлена в <<поверхностном>> виде на одном из 
естественных языков в результате конечного числа определенных 
преобразований. Например, предложения

\noindent
\begin{align*}    
(1)\ &\mbox{\textit{The programmer writes the code}}\\
(2)\ &\mbox{\textit{The code is written by the programmer}}
\end{align*}
имеют истоком одну глубинную структуру:

\medskip

\noindent
     \begin{verbatim}
  Programmer <---- write ----> Code
      agent                   object,
\end{verbatim}

\medskip

\noindent
хотя и отличаются своими поверхностными структурами. В~каждом из них 
имеется агент (the programmer), объект (the code) и действие (write).\linebreak Согласно 
концепции \textit{падежной грамматики} Филлмора~\cite{5koz} глубинная 
структура для обоих предложений инвариантна. Эту структуру можно 
представить в виде скобочной записи $V(\mathrm{AGENT}, \mathrm{OBJECT})$. В~графическом 
виде глубинная структура предложения также может быть представлена 
диаграммой в виде дерева, где отражены инвариантные отношения 
зависимости между предикатной вершиной и актантами (рис.~\ref{f5koz}), 
причем в таком представлении явным образом разграничиваются 
\textit{модальность} (MOD) и \textit{пропозиция} (PROP).
     

     В исходном варианте~\cite{5koz} теория признавала шесть падежей: 
агентив, инструменталис, датив, объектив, локатив и фактитив. По мере 
развития теории~\cite{8koz} происходило увеличение числа падежей, однако 
<<умножение>> количества падежей утяжеляет первоначальную 
конфигурацию, поэтому при построении инженерных семантических 
представлений требуется некоторый <<компромиссный>> вариант, 
сочетающий в себе необходимую полноту, с одной стороны, и простоту и 
гибкость, с другой.

\begin{figure*}[b] %fig6
\vspace*{24pt}
\begin{center}
\mbox{%
\epsfxsize=156.873mm
\epsfbox{koz-6.eps}
}
\end{center}
%\vspace*{-9pt}
\Caption{Обобщенное функциональное представление систем ИСПАР
\label{f6koz}}
\end{figure*}
     
%\vspace*{-6pt}

\section{Некоторые базовые аспекты построения многоязычных 
систем}
     
     Одним из приоритетных направлений развития РСС-сис\-тем является 
обеспечение обработки текстов на нескольких языках, прежде всего для 
рус\-ско-анг\-лий\-ской языковой пары. В системах 2-го поколения~--- ДИЕС2, 
ИКС, ЛОГОС-Д были реализованы лингвистические процессоры и словари 
для русского и английского языка, позволявшие обрабатывать тексты для 
ряда ПО. При этом поддерживался как режим ввода 
лингвистических знаний линг\-вис\-том-ана\-ли\-ти\-ком, так и 
автоматический режим самообучения системы по вводимым \mbox{текстам}. 
{\looseness=1

}

Проводились также эксперименты с итальянским и французским языком. 
При создании многоязычных систем авторы обращались к европейским 
языкам. Очевидно, что европейские языки обладают большим числом общих 
правил, чем любой из них с языками других групп. Но при этом все 
естественные языки обладают общей структурой на самом глубинном 
уровне. На этом уровне располагаются главные элементы естественного 
языка: \textit{предложение}, \textit{модальность}, \textit{пропозиция}.
     
     Моделирование смысловых представлений~--- это процесс, 
развивающийся в направлении от поверхностных семантических структур к 
глубинным. Поиск такого внутреннего представления смысла в условиях 
многоязычной ситуации является на\-прав\-ле\-ни\-ем развития методов 
     КЛМ на базе  РСС. 
     
%     \vspace*{-48pt}
     
\section{Интеллектуальные системы поддержки аналитических 
решений}
     
Системы РСС 3-го и 4-го поколения на\-прав\-ле\-ны на извлечение знаний 
в виде \textit{объектов}, или \textit{сущностей}, и связей между ними из 
пред\-мет\-но-ориен\-ти\-ро\-ван\-ных текстов на русском и английском 
языке~\cite{18koz, 19koz}.

    
В настоящее время во всем мире активно ведутся работы по созданию 
систем извлечения фактов из текстов на естественных языках~[11--14], создаются развитые тезаурусы и 
онтологии~\cite{17koz}. Сис\-те\-мы РСС функционально шире, поскольку 
имеют возможность не только извлекать факты, но и поддерживать 
механизмы логического анализа и экспертного вывода на основе 
извлеченных знаний. Сис\-те\-ма\-ми такого рода являются ИСПАР. В~целом это 
направление исследований требует дальнейшей проработки 
     лек\-си\-ко-се\-ман\-ти\-че\-ских представлений, создания 
     пред\-мет\-но-ориен\-ти\-ро\-ван\-ных семантических словарей. 

Обобщенное функциональное представление систем ИСПАР дано на 
рис.~\ref{f6koz}. 
     
     В рамках ИСПАР на основе РСС 
(\mbox{ИСПАР}--РСС) были реализованы полномасштабные и\linebreak пилотные 
проекты для ряда ПО: криминалистики, управления 
кадрами, мониторинга финансово-экономического кризиса и 
др.~\cite{18koz, 19koz}.

\section{Применение аппарата расширенных семантических сетей в~лингвистических 
исследованиях}
     
     В настоящее время в рамках проектов, на\-прав\-лен\-ных на создание 
открытых лингвистических ресурсов~\cite{20koz} для 
     на\-уч\-но-прак\-ти\-че\-ских целей, ведутся работы по выравниванию 
параллельных текстов научных статей, патентов и 
     фи\-нан\-со\-во-эко\-но\-ми\-че\-ских текстов. В~качестве одного из 
методов выравнивания используется РСС-под\-ход, поскольку он позволяет 
отразить глу\-бин\-но-се\-ман\-ти\-че\-ский уровень языковых структур. 

На  рис.~7 представлен фрагмент первого этапа лингвистического 
анализа в многоязычных системах. Для <<идеальной>> ситуации, когда 
структуры исходного текста и текста перевода практически совпадают, такая 
ситуация имеет место в меньшинстве случаев. Основные трудности 
возникают при наличии переводческих трансформаций в параллельных 
текстах. Особое внимание следует уделять гла\-голь\-но-имен\-ным 
трансформациям, например явлению \textit{номинализации}, поскольку она 
очень продуктивна для всех исследовавшихся языков.

     
     Ключевой задачей при разработке методов сопоставления 
параллельных текстов является выявление и детальное описание тех 
языковых трансформаций, которые имеют место при переводе 
     естест\-вен\-но-язы\-ко\-вых конструкций с одного языка на 
другой~\cite{9koz}, потому что далеко не всегда некое содержание 
передается струк\-тур\-но-по\-доб\-ны\-ми средствами в текстах на разных 
языках. Сравнительное исследование употребления различных частей речи в 
параллельных текстах на разных языках создает основу для выявления и 
описания языковых транс-\linebreak

\begin{center} %fig7
\vspace*{3pt}
\mbox{%
\epsfxsize=79.726mm
\epsfbox{koz-7.eps}
}
\end{center}
\vspace*{4pt}
%\begin{center}
{{\figurename~7}\ \ \small{Первый этап анализа параллельных текстов ($W_n$
обозначает словоформу с номером~$n$, $1\leq n\geq 5$)}}
%\end{center}
%\vspace*{9pt}

%\bigskip
\addtocounter{figure}{1}
      

\noindent 
формаций, при этом центральной трансформацией
является \textit{номинализация}. Явление номинализации
было исследовано в 
ряде работ отечественных и зарубежных лингвистов~[17--20]. 
Ближе всего к правильному, по мнению авторов данной статьи, 
пониманию этого явления следующие определения номинализации: 
<<конструкции\ldots называются номинализованными~--- в том смысле, что 
их естественно рассматривать как результат номинализации конструкций с 
предикативным употреблением глаголов и прилагательных>>; 
<<номинализация~--- это синтаксический процесс, который соотносит 
предложения с именными группами>>~\cite{9koz, 10koz}. Выявление 
номинализованных конструкций в параллельных научных и патентных 
текстах на русском, английском, французском и немецком языках в научных 
и патентных текстах и сопоставительное описание гла\-голь\-но-имен\-ных 
межъязыковых трансформаций~--- одна из центральных задач 
     ин\-же\-нер\-но-линг\-ви\-сти\-че\-ских исследований. 
     
     Следующей базовой трансформацией в исследуемых текстах на 
нескольких европейских языках является адъек\-тив\-но-ад\-вер\-би\-аль\-ное 
преобразование. Это означает, что при переводе с одного языка на другой 
происходит синтаксическое преобразование имен прилагательных в наречия 
и обратное преобразование~--- наречий в прилагательные. Установление 
семантических соответствий между этими языковыми объектами также 
возможно осуществить посредством аппарата~РСС. 
     
     При семантическом выравнивании непараллельных текстов, имеющих 
одну и ту же денотативную составляющую, аппарат РСС позволяет выявить в 
текстах когнитивные опоры (слова с сильной валентностью~--- 
     <<сло\-ва-дейст\-вия>> и <<сло\-ва-от\-но\-ше\-ния>>) и установить 
между ними семантические соответствия.

\section{Заключение}

     В данной работе представлен опыт создания и развития 
     когни\-тив\-но-линг\-ви\-сти\-че\-ских пред\-став\-ле\-ний в 
интеллектуальных информационных сис\-те\-мах, разработанных на основе 
аппарата РСС. Аппарат РСС 
обеспечивает мощные изобразительные возможности для описания всех 
уровней естественного языка, включая уровень 
     глу\-бин\-но-се\-ман\-ти\-че\-ских представлений и межъязыковых 
соответствий. Конкретные лингвистические процессоры, которые были 
созданы на основе этого подхода, прошли определенный путь развития и 
позволили выработать проектные решения для основных задач текущего 
этапа~--- извлечения и обработки содержательных знаний из текстов на 
естественных языках и сопоставления языковых структур в текстах на 
различных языках с учетом базовых трансформаций.
     
     Проблема извлечения и обработки знаний открывает перспективы 
развития интеллектуальных направлений компьютерной лингвистики, 
поскольку ее основной акцент смещен в сторону\linebreak глубинных представлений 
языка, в которых используются как грамматические (морфологические и 
синтаксические), так и семантические атрибуты для описания языковых 
объектов. Проводи-\linebreak мые авторами исследования параллельных текстов 
направлены также на рассмотрение этой проблемы~\cite{20koz}. 
Центральное место в проводящихся линг\-ви\-сти\-че\-ских исследованиях 
занимает изучение и формализация процессов трансформации языковых 
структур, особенно все варианты глагольно-но\-ми\-на\-тив\-ных трансформаций, 
создание развитых дис\-три\-бу\-тив\-но-транс\-фор\-ма\-ци\-он\-ных 
описаний предикатых структур для рассматриваемых языков. 
     
     Для задач извлечения знаний и создания \mbox{ИСПАР} 
     дис\-три\-бу\-тив\-но-транс\-фор\-ма\-ци\-он\-ные описания имеют 
особое значение, поскольку таким образом задаются все возможные способы 
перевода языковых структур в пре\-ди\-кат\-но-ар\-гу\-мент\-ные 
представления, которые затем используются в процедурах обработки знаний.

{\small\frenchspacing
{%\baselineskip=10.8pt
%\addcontentsline{toc}{section}{Литература}
\begin{thebibliography}{99}

     \bibitem{1koz}
     \Au{Кузнецов~И.\,П.}
     Семантические представления.~--- М.: Наука, 1986. 290~с.
     
     \bibitem{2koz}
     \Au{Козеренко~Е.\,Б.}
     Кон\-цеп\-ту\-аль\-но-линг\-вис\-ти\-че\-ское моделирование в среде 
интеллектуального редактора знаний ИКС~// Проблемы проектирования и 
использования баз знаний.~--- Киев: Ин-т кибернетики им.\ В.\,М.~Глушкова, 
1992. C.~73--79.
     
     \bibitem{3koz}
     \Au{Kozerenko~E.\,B.}
     Multilingual processors: A unified approach to semantic and syntactic 
knowledge presentation~// Conference (International ) on Artificial Intelligence 
IC-AI'2001 Proceedings. Las Vegas, Nevada, USA. June 25--28, 2001.~--- Las 
Vegas: CSREA Press, 2001. P.~1277--1282.

     \bibitem{18koz} %4
     \Au{Kuznetsov~I.\,P., Efimov~D.\,A., Kozerenko~E.\,B.}
     Tools for tuning the semantic processor to application areas~// ICAI'09 
Proceedings, WORLDCOMP'09. July 13--16, 2009. Las Vegas, Nevada, USA. 
Vol.~I.~--- Las Vegas: CRSEA Press, 2009. P.~467--472.
     
     \bibitem{19koz} %5
     \Au{Kuznetsov~I.\,P., Kozerenko~E.\,B., Kuznetsov~K.\,I., 
Timonina~N.\,O.}
     Intelligent system for entities extraction (ISEE) from natural language 
texts~// Workshop (International) on Conceptual Structures for Extracting Natural 
Language Semantics (Sense'09) at the 17th Conference 
(International ) on Conceptual Structures (ICCS'09) Proceedings. University Higher School of 
Economics. Moscow, Russia, 2009. P.~17--25.
     
     \bibitem{4koz} %6
     \Au{Апресян~Ю.\,Д.}
     Экспериментальное исследование семантики русского глагола.~--- М.: 
Наука, 1967.  252~с.
     
     \bibitem{5koz} %7
     \Au{Филлмор~Ч.}
     Дело о падеже~// Новое в зарубежной линг\-вистике, 1968. Вып.~X. С.~369--495.
     
     \bibitem{6koz} %8
     \Au{Хомский~Н.}
     Аспекты теории синтаксиса.~--- М.: МГУ, 1972.
     
     \bibitem{7koz} %9
     \Au{Хомский Н.}
     Язык и мышление.~--- М.: МГУ, 1972.
     
     
     \bibitem{8koz} %10
     \Au{Fillmore~C.}
     The case for case reopened~// Syntax and Semantics. Vol.~8.~--- N.Y.: 
Academic Press, 1977. 
     

          \bibitem{15koz} %11
     FASTUS: A cascaded finite-state trasducer for extracting information from 
natural-language text~// AIC, SRI International, Menlo Park, California, 1996. 
     
     \bibitem{16koz} %12
     \Au{Han~J., Pei~Y., Mao~R.}
     Mining frequent patterns without candidate generation: A frequent-pattern 
tree approach~// Data Mining and Knowledge Discovery, 2004. Vol.~8. No.\,1. 
P.~53--87.
     
     
     \bibitem{13koz} %13
     \Au{Cunningham~H.}
     Automatic information extraction~// Encyclopedia of Language and 
Linguistics. 2nd ed.~--- Elsevier, 2005.
     
     \bibitem{14koz} %14
     \Au{Han~J., Kamber~M.}
     Data mining: Concepts and techniques.~--- Morgan Kaufmann, 2006.
     
     
     \bibitem{17koz} %15
     \Au{Добров~Б.\,В., Лукашевич~Н.\,В.}
     Онтологии для автоматической обработки текстов: Описание понятий 
и лексических значений~// Компьютерная лингвистика и интеллектуальные 
технологии: Тр. межд. конф. <<Диалог'06>>. Бекасово, 31~мая\,--\,4~июня 
2006. С.~138--142.

     \bibitem{20koz} %16
     \Au{Kozerenko~E.\,B.}
     INTERTEXT: A multilingual knowledge base for machine translation~// 
Conference (International) on Machine Learning, Models, Technologies and 
Applications Proceedings. June 25--28, 2007. Las Vegas, USA.~--- Las Vegas: 
CSREA Press, 2007. P.~238--243.

     \bibitem{9koz} %17
     \Au{Жолковский~А.\,К., Мельчук~И.\,А.}
     О семантическом синтезе~// Проблемы кибернетики, 1967. Вып.~19.
     
         
     \bibitem{11koz} %18
     \Au{Jacobs~R.\,A., Rosenbaum P.\,S.}
     English transformational grammar.~--- Blaisdell, 1968.
     

\label{end\stat}
     
          \bibitem{12koz} %19
     \Au{Балли~Ш.}
     Общая лингвистика и вопросы французского языка. 2-е изд.~--- М.: 
УРСС, 2001.

\bibitem{10koz} %20
     \Au{Падучева~Е.\,В.}
     О~семантике синтаксиса: Мат-лы к трансформационной 
грамматике русского языка. 2-е изд.~--- М: КомКнига, 2007.  296~с. 
     
 \end{thebibliography}
}
}


\end{multicols}    %13

%   { %\Large  
   { %\baselineskip=16.6pt
   
   \vspace*{-48pt}
   \begin{center}\LARGE
   \textit{Предисловие}
   \end{center}
   
   %\vspace*{2.5mm}
   
   \vspace*{25mm}
   
   \thispagestyle{empty}
   
   { %\small 

    
Вниманию читателей журнала <<Информатика и её применения>> предлагается 
очередной тематический выпуск <<Вероятностно-статистические методы и 
задачи информатики и информационных технологий>>. Предыдущие тематические 
выпуски журнала по данному направлению вышли в 2008~г.\ (т.~2, вып.~2), 
в 2009~г.\ (т.~3, вып.~3) и в 2010~г.\ (т.~4, вып.~2). 

Статьи, собранные в данном журнале, посвящены разработке новых вероятностно-статистических 
методов, ориентированных на применение к решению конкретных задач информатики и информационных 
технологий, а также~--- в ряде случаев~--- и других прикладных задач. Проблематика, охватываемая 
публикуемыми работами, развивается в рамках научного сотрудничества между Институтом проблем 
информатики Российской академии наук (ИПИ РАН) и Факультетом вычислительной математики и 
кибернетики Московского государственного университета им.\ М.\,В.~Ломоносова в ходе работ 
над совместными научными проектами (в том числе в рамках функционирования 
Научно-образовательного центра <<Вероятностно-статистические методы анализа рисков>>). 
Многие из авторов статей, включенных в данный номер журнала, являются активными участниками 
традиционного международного семинара по проблемам устойчивости стохастических моделей, 
руководимого В.\,М.~Золотаревым и В.\,Ю.~Королевым; регулярные сессии этого семинара 
проводятся под эгидой МГУ и ИПИ РАН (в 2011~г.\ указанный семинар проводится в октябре 
в Калининградской области РФ). 

Наряду с представителями ИПИ РАН и МГУ в число авторов данного выпуска журнала входят 
ученые из Научно-исследовательского института системных исследований РАН, Института 
проблем технологии микроэлектроники и особочистых материалов РАН, Института 
прикладных математических исследований Карельского НЦ РАН, Московского 
авиационного института, Вологодского государственного педагогического университета, 
НИИММ им.\ Н.\,Г.~Чеботарева, Казанского государственного университета, Дебреценского 
университета (Венгрия).

Несколько статей выпуска посвящено разработке и применению стохастических методов и 
информационных технологий для решения различных прикладных задач. В~работе В.\,Г.~Ушакова 
и О.\,В.~Шестакова рассмотрена задача определения вероятностных характеристик случайных 
функций по распределениям интегральных преобразований, возникающих в задачах эмиссионной 
томографии. В~статье Д.\,О.~Яковенко и М.\,А.~Целищева рассмотрены некоторые вопросы 
математической теории риска и предложен новый подход к диверсификации инвестиционных 
портфелей. Работа И.\,А.~Кудрявцевой и А.\,В.~Пантелеева посвящена построению и 
исследованию математической модели, описывающей динамику сильноионизованной плазмы. 
В~статье П.\,П.~Кольцова изучается качество работы ряда алгоритмов сегментации изображений. 
Статья А.\,Н.~Чупрунова и И.~Фазекаша посвящена вероятностному анализу числа без\-оши\-бочных 
блоков при помехоустойчивом кодировании; получены усиленные законы больших чисел для указанных 
величин.

В данном выпуске традиционно присутствует тематика, весьма активно разрабатываемая в течение 
многих лет специалистами ИПИ РАН и МГУ,~--- методы моделирования и управления для 
информационно-телекоммуникационных и вычислительных систем, в частности методы 
теории массового обслуживания. В~статье А.\,И.~Зейфмана с соавторами рассматриваются 
модели обслуживания, описываемые марковскими цепями с непрерывным временем в случае 
наличия катастроф. В~работе М.\,М.~Лери и И.\,А.~Чеплюковой рассматриваются случайные 
графы Интернет-типа, т.\,е.\ графы, степени вершин которых имеют степенные распределения; 
такие задачи находят применение при исследовании глобальных сетей передачи данных. 
Работа Р.\,В.~Разумчика посвящена исследованию систем массового обслуживания специального 
вида~--- с отрицательными заявками и хранением вытесненных заявок.

Ряд статей посвящен развитию перспективных теоретических 
вероятностно-статистических методов, которые находят широкое применение в различных 
задачах информатики и информационных технологий. В~работе В.\,Е.~Бенинга, А.\,К.~Горшенина 
и В.\,Ю.~Королева рассмотрена задача статистической проверки гипотез о числе компонент 
смеси вероятностных распределений, приводится конструкция асимптотически наиболее мощного 
критерия. Результаты этой работы найдут применение в ряде прикладных задач, использующих 
математическую модель смеси вероятностных распределений (в информатике, моделировании 
финансовых рынков, физике турбулентной плазмы и~т.\,д.). В~статье В.\,Ю.~Королева, 
И.\,Г.~Шевцовой и С.\,Я.~Шоргина строится новая, улучшенная оценка точности нормальной 
аппроксимации для пуассоновских случайных сумм; как известно, указанные случайные суммы 
широко используются в качестве моделей многих реальных объектов, в том числе в информатике, 
физике и других прикладных областях. Работа В.\,Г.~Ушакова и Н.\,Г.~Ушакова посвящена 
исследованию ядерной оценки плотности распределения; эти результаты могут применяться, 
в част\-ности, при анализе трафика в телекоммуникационных системах. Серьезные приложения 
в статистике могут получить результаты работы О.\,В.~Шестакова, в которой доказаны оценки 
скорости сходимости распределения выборочного абсолютного медианного отклонения к нормальному 
закону. 

\smallskip

Редакционная коллегия журнала выражает надежду, что данный тематический  выпуск 
будет интересен специалистам в области теории вероятностей и математической статистики 
и их применения к решению задач информатики и информационных технологий.
     
     %\vfill 
     \vspace*{20mm}
     \noindent
     Заместитель главного редактора журнала <<Информатика и её 
применения>>,\\
     директор ИПИ РАН, академик  \hfill
     \textit{И.\,А.~Соколов}\\
     
     \noindent
     Редактор-составитель тематического выпуска,\\
     профессор кафедры математической статистики факультета\\
      вычислительной математики и кибернетики МГУ им.\ М.\,В.~Ломоносова,\\
     ведущий научный сотрудник ИПИ РАН,\\ 
доктор физико-математических наук \hfill
      \textit{В.\,Ю.~Королев}
     
     } }
     }

%%%%%%%%%%%%%%%%%%%%%%%%%%%%%%%%%%%%%%%%%%%%%%%


                       
%\end{document}

%\def\stat{rez}
{%\hrule\par
%\vskip 7pt % 7pt
\raggedleft\Large \bf%\baselineskip=3.2ex
Р\,Е\,Ц\,Е\,Н\,З\,И\,И \vskip 17pt
    \hrule
    \par
\vskip 6pt plus 6pt minus 3pt }

%\thispagestyle{headings} %с верхним колонтитулом
%\thispagestyle{myheadings} %с нижним колонтитулом, но в верхнем РЕЦЕНЗИИ

\def\tit{НОВАЯ КНИГА И.\,Н.~СИНИЦЫНА, А.\,С.~ШАЛАМОВА <<ЛЕКЦИИ ПО ТЕОРИИ 
ИНТЕГРИРОВАННОЙ ЛОГИСТИЧЕСКОЙ ПОДДЕРЖКИ>> (М.: ТОРУС ПРЕСС, 2012. 624~с.)}

%1
\def\aut{Д.ф.-м.н., профессор С.\,Я.~Шоргин}

\def\auf{\ }

\def\leftkol{\ % РЕЦЕНЗИИ
}

\def\rightkol{ \ } 

%\def\leftkol{\ } % ENGLISH ABSTRACTS}

%\def\rightkol{\ } %ENGLISH ABSTRACTS}

%\def\leftkol{РЕЦЕНЗИИ}

%\def\rightkol{РЕЦЕНЗИИ}

\titele{\tit}{\aut}{\auf}{\leftkol}{\rightkol}
\vspace*{-18pt}


     \label{st\stat}

     \begin{multicols}{2}
     {\small
     {\baselineskip=10.1pt
     

      В книге представлено системное изложение теоретических основ одного из новейших 
направлений в \mbox{об\-ласти} экономики послепродажного обслуживания изделий наукоемкой 
продукции (ИНП) длительного пользования~--- интегрированной логистической поддержки
(ИЛП). 
{\looseness=1

}

Приведены также результаты новых работ, выполненных в Институте проблем информатики 
Российской академии наук в рамках научного направления <<Информационные технологии и 
анализ сложных сис\-тем>>.
 {%\looseness=1

}
     
      Излагаемые в книге научные подходы позво\-ляют карди\-наль\-но реформировать 
существующие системы производства и эксплуатации ИНП путем создания и внед\-ре\-ния 
методов рационального и оптимального управ\-ле\-ния процессами расходования 
вре\-мен\-н$\acute{\mbox{ы}}$х, 
мате\-ри\-аль\-ных, трудовых и других ресурсов на всех стадиях жизненного цикла изделий (ЖЦИ) по 
критериям экономической целесообразности и эф\-фек\-тив\-ности.
  {\looseness=1

}
    
      В книге приведен краткий обзор причин возник\-новения и
      развития CALS-методологии как основы 
современных международных стандартов по созданию и функционированию глобальных 
ин\-фор\-ма\-ци\-он\-но-ком\-му\-ни\-ка\-ци\-он\-ных систем, ее ключевых возможностей и эффективности 
результатов ее использования. 
Авторы %\linebreak 
предлагают ряд научных обоснований для разработки 
единой теории проектирования и управления систем ИЛП для полноценного использования 
преимуществ %\linebreak
 суще\-ст\-ву\-ющей методологии, определяют \mbox{общую} структурную схему 
комплексной системы <<ИНП-СППО>> и необходимость разработки для ее описания 
гибридных стохастических моделей.
{%\looseness=1

}

%\columnbreak
      
      Книга состоит из пяти частей, где последовательно излагается материал по каждой из 
следующих тем: <<Интегрированная логистическая поддержка>>, <<Теория гибридных 
стохастических систем и компьютерная поддержка исследований и разработок>>, <<Основы 
математического моделирования, анализа и синтеза систем послепродажного обслуживания>>, 
<<Определение и анализ показателей экспортного потенциала ИНП при проектировании>>, 
<<Задачи управления поддержкой послепродажного обслуживания>>, а также 
<<Моделирование инвестиционных процессов ИЛП в условиях неравновесных финансовых 
рынков>>. 
   
      В конце каждой главы приведены выводы и даны вопросы и задания для 
самоконтроля. В~приложениях содержатся основные определения по программам работ по 
анализу ИЛП, логистическим базам данных и компьютерным решениям, эквивалентной статистической 
линеаризации нелинейных преобразований ИЛП, справочный материал, а также развернутые 
уравнения для вероятностных характеристик.


      \def\leftkol{РЕЦЕНЗИИ}

\def\rightkol{РЕЦЕНЗИИ} 

      
      Книга заинтересует широкий круг специалистов и может быть использована научными 
проектными организациями в сфере промышленного производства ИНП. Большое количество 
иллюстраций, примеров и вопросов, обращенных к читателю, позволяет использовать книгу 
также в качестве учебного пособия для студентов и аспирантов машиностроительных, 
транспортных и~других специальностей, а также для самостоятельного изучения. 
{%\looseness=-1

}

Книга 
представляет несомненный интерес для специалистов и студентов в области прикладной 
математики и информатики.
    

}

}
\end{multicols}

%\newpage

\include{obchak}



\def\stat{authorsrus}
{%\hrule\par
%\vskip 7pt % 7pt
\raggedleft\Large \bf%\baselineskip=3.2ex
О\,Б\ \ А\,В\,Т\,О\,Р\,А\,Х \vskip 17pt
    \hrule
    \par
\vskip 21pt plus 8pt minus 4pt }


\def\tit{\ }

\def\aut{\ }

\def\auf{\ }

\def\leftkol{\ } % ENGLISH ABSTRACTS}

\def\rightkol{ОБ АВТОРАХ} %ENGLISH ABSTRACTS}

\titele{\tit}{\aut}{\auf}{\leftkol}{\rightkol}
      
            \label{st\stat}



\vspace*{24pt}

\begin{multicols}{2}




\noindent
\textbf{Архипов Олег Петрович} (р.\ 1948)~---
кандидат технических наук, директор Орловского филиала Института проб\-лем информатики
Российской академии наук
%302025, г.Орел, Московское шоссе, д.137

\vspace*{3pt}

\noindent
\textbf{Бирюкова Татьяна Константиновна} (р.\ 1968)~---
кандидат фи\-зи\-ко-ма\-те\-ма\-ти\-че\-ских наук, старший научный сотрудник Института проб\-лем информатики
Российской академии наук

\vspace*{3pt}

\noindent 
\textbf{Бобков  Сергей Геннадьевич} (р.\ 1955)~---
доктор технических наук,  заведующий отделением На\-уч\-но-ис\-сле\-до\-ва\-тель\-ско\-го 
института системных исследований Российской академии наук
%117218, Москва, Нахимовский просп., 36, к.1 

\vspace*{3pt}

\noindent \textbf{Васильев Николай Семенович} (р.\ 1952)~--- доктор 
фи\-зи\-ко-ма\-те\-ма\-ти\-че\-ских наук, профессор, 
МГТУ им.\ Н.\,Э.~Баумана 
%, Москва 105005, 2-я Бауманская ул., д.~5,

\vspace*{3pt}

\noindent
\textbf{Гершкович Максим Михайлович} (р.\ 1968)~---
старший научный сотрудник Института проб\-лем информатики
Российской академии наук

\vspace*{3pt}

\noindent 
\textbf{Дьяченко Юрий Георгиевич} (р.\ 1958)~--- кандидат технических наук, 
старший научный сотрудник Института проб\-лем информатики
Российской академии наук

\vspace*{3pt}

\noindent 
\textbf{Ерошенко Александр Андреевич} (р.\ 1989)~--- аспирант кафедры 
математической статистики факультета вычисли\-тельной математики и кибернетики 
Московского государственного университета им.\ М.\,В.~Ломоносова
%119991, Москва ГСП-1, Ленинские горы, д.\ 1, стр. 52

\vspace*{3pt}
 
\noindent 
\textbf{Захаров Виктор Николаевич} (р.\ 1948)~--- 
доктор технических наук, доцент, ученый секретарь Института проб\-лем информатики
Российской академии наук

\vspace*{3pt}

\noindent
\textbf{Зейфман Александр Израилевич} (р.\ 1954)~---
доктор фи\-зи\-ко-ма\-те\-ма\-ти\-че\-ских наук, профессор, 
заведующий кафедрой Вологодского государственного университета; 
старший научный сотрудник Института проб\-лем информатики
Российской академии наук; главный научный сотрудник ИСЭРТ Российской академии наук

\vspace*{3pt}

\noindent
\textbf{Зыкин Сергей Владимирович} (р.\ 1959)~--- 
доктор технических наук, профессор, заведующий лабораторией Института математики 
им.\ С.\,Л.~Соболева Сибирского отделения Российской академии наук, Новосибирск 
%630090, пр.\ ак.\ Коптюга, 4 

\vspace*{4pt}

\noindent
\textbf{Киреев Владимир Иванович} (р.\ 1938)~---
доктор фи\-зи\-ко-ма\-те\-ма\-ти\-че\-ских наук, профессор Московского 
государственного горного университета
%Адрес: Россия, 119991, г. Москва, Ленинский проспект, д. 6

%\columnbreak

\vspace*{4pt}

\noindent
\textbf{Козеренко Елена Борисовна} (р.\ 1959)~---
кандидат филологических наук, заведующая лабораторией Института проб\-лем информатики
Российской академии наук

\vspace*{4pt}

\noindent
\textbf{Королев Виктор Юрьевич} (р.\ 1954)~--- доктор
фи\-зи\-ко-ма\-те\-ма\-ти\-че\-ских наук, профессор кафедры математической 
статистики факультета вычисли\-тельной математики и кибернетики 
Московского государственного университета; 
ведущий научный сотрудник Института проб\-лем информатики
Российской академии наук

\vspace*{4pt}

\noindent
\textbf{Коротышева Анна Владимировна} (р.\ 1988)~---
старший преподаватель Вологодского государственного университета

\vspace*{4pt}

\noindent 
\textbf{Кун Де Турк} (р.\ 1981)~--- научный сотрудник 
исследовательской группы SMACS факультета телекоммуникаций и обработки информации
Университета Гента, Бельгия
%В-9000 Гент, Бельгия

\vspace*{4pt}

\noindent
\textbf{Лупенцов Олег Сергеевич} (р.\ 1986)~---
аспирант Омского государственного института сервиса
%Омск 644043, ул.\ Певцова 13

\vspace*{4pt}

\noindent
\textbf{Лучко Олег Николаевич} (р.\ 1961)~---
кандидат педагогических наук, профессор, заведующий кафедрой 
Омского государственного института сервиса
%Омск 644043, ул.\ Певцова 13

\vspace*{4pt}

\noindent
\textbf{Малашенко Юрий Евгеньевич} (р.\ 1946)~---
доктор фи\-зи\-ко-ма\-те\-ма\-ти\-че\-ских наук, заведующий сектором 
Вычислительного центра им.\ А.\,А.~Дородницына Российской академии наук
%Адрес: 119333, Москва, ул. Вавилова, 40,

\vspace*{4pt}

\noindent
\textbf{Маньяков Юрий Анатольевич} (р.\ 1984)~---
кандидат технических наук, научный сотрудник Орловского филиала Института проб\-лем информатики
Российской академии наук
%302025, г.Орел, Московское шоссе, д.137

\vspace*{4pt}

\noindent
\textbf{Маренко Валентина Афанасьевна} (р.\ 1951)~---
кандидат технических наук, доцент, старший научный сотрудник 
Института математики им.\ С.\,Л.~Соболева Сибирского отделения Российской академии наук
%Новосибирск 630090, пр. ак. Коптюга, 4 

\vspace*{3pt}

\noindent 
\textbf{Морозов Евсей Викторович} (р.\ 1947)~--- доктор 
фи\-зи\-ко-ма\-те\-ма\-ти\-че\-ских, профессор, ведущий научный сотрудник 
Института прикладных математических исследований Карельского научного центра Российской
академии наук; 
%%185910 Россия, Республика Карелия, г.\ Петрозаводск, ул.\ Пушкинская, 11
профессор Петрозаводского государственного университета, Петрозаводск
%185910 Россия, Республика Карелия, г.\ Петрозаводск, пр.\ Ленина, 33

%\pagebreak

\vspace*{3pt}

\noindent
\textbf{Назарова Ирина Александровна} (р.\ 1966)~---
кандидат фи\-зи\-ко-ма\-те\-ма\-ти\-че\-ских наук, 
научный сотрудник Вычислительного центра им.\ А.\,А.~Дородницына Российской академии наук 
%Адрес: 119333, Москва, ул. Вавилова, 40

\vspace*{3pt}

\noindent
\textbf{Павлов Игорь Валерианович} (р.\ 1945)~--- 
доктор фи\-зи\-ко-ма\-те\-ма\-ти\-че\-ских наук, профессор МГТУ им.\ Н.\,Э.~Баумана 
%Москва 105005, 2-я Бауманская ул., д.~5 

%\pagebreak

\vspace*{3pt}

\noindent 
\textbf{Потахина Любовь Викторовна} (р.\ 1989)~--- аспирантка
Института прикладных математических исследований Карельского научного центра
Российской академии наук; 
%%185910 Россия, Республика Карелия, г.\ Петрозаводск, ул.\ Пушкинская, 11
инженер Петрозаводского государственного университета, Петрозаводск
%185910 Россия, Республика Карелия, г.\ Петрозаводск, пр.\ Ленина, 33

\vspace*{3pt}

\noindent 
\textbf{Рождественский Юрий Владимирович} (р.\ 1952)~--- 
кандидат технических наук, заведующий сектором Института проб\-лем информатики
Российской академии наук

\vspace*{3pt}

\noindent 
\textbf{Синицын Игорь Николаевич} (р.\ 1940)~--- доктор технических наук,
профессор, заслуженный деятель\linebreak\vspace*{-12pt}

\columnbreak

\noindent
 науки РФ, заведующий отделом Института проб\-лем информатики
Российской академии наук

\vspace*{7pt}


\noindent
\textbf{Сиротинин Денис Олегович} (р.\ 1984)~---
кандидат технических наук, научный сотрудник Орловского филиала Института проб\-лем информатики
Российской академии наук
%302025, г.Орел, Московское шоссе, д.137

\vspace*{7pt}

%\columnbreak

\noindent 
\textbf{Соколов  Игорь Анатольевич} (р.\ 1954)~--- академик (действительный член) Российской 
академии наук, доктор технических наук, директор Института проб\-лем информатики
Российской академии наук

\vspace*{7pt}

\noindent
\textbf{Степченков Юрий Афанасьевич} (р.\ 1951)~---
кандидат технических наук, заведующий отделом Института проб\-лем информатики
Российской академии наук

\vspace*{7pt}

\noindent
\textbf{Сурков Алексей Викторович} (р.\ 1978)~--- 
старший научный сотрудник На\-уч\-но-ис\-сле\-до\-ва\-тель\-ско\-го 
института системных исследований Российской академии наук
%117218, Москва, Нахимовский просп., 36, к.1 

\vspace*{7pt}

\noindent 
\textbf{Шестаков Олег Владимирович} (р.\ 1976)~--- доктор 
фи\-зи\-ко-ма\-те\-ма\-ти\-че\-ских, доцент кафедры математической статистики 
факультета вычисли\-тельной математики и кибернетики Московского 
государственного университета им.\ М.\,В.~Ломоносова; 
%119991, Москва ГСП-1, Ленинские горы, д.\ 1, стр. 52
старший научный сотрудник Института проб\-лем информатики
Российской академии наук
%, Москва 119333, ул. Вавилова, д.~44, корп.~2

\vspace*{7pt}

\noindent 
\textbf{Шоргин Сергей Яковлевич} (р.\ 1952.)~--- доктор
фи\-зи\-ко-ма\-те\-ма\-ти\-че\-ских наук, профессор, заместитель директора Института 
проб\-лем информатики Российской академии наук





%%%%%%%%%%%%%%%%%%%%%%%%%%%%%%%%%%%%%%%%%%%%%%%%%%%%%%%%%%%%%%%%%%%%%%%%%%%%%%%




%\def\rightkol{ОБ АВТОРАХ}
%\def\leftkol{ОБ АВТОРАХ}

 \label{end\stat}





%\def\leftfootline{\small{\textbf{\thepage}
%\hfill ИНФОРМАТИКА И ЕЁ ПРИМЕНЕНИЯ\ \ \ том~7\ \ \ выпуск~1\ \ \ 2013}
%}%
% \def\rightfootline{\small{ИНФОРМАТИКА И ЕЁ ПРИМЕНЕНИЯ\ \ \ том~7\ \ \ выпуск~1\ \ \ 2013
%\hfill \textbf{\thepage}}}


%\thispagestyle{myheadings}



\end{multicols}

\newpage


%\vspace*{-48pt}
\begin{center}\LARGE
\textit{About Authors}
\end{center}

\thispagestyle{empty}
\def\tit{\ }

\def\aut{\ }

\def\auf{\ }


\def\leftkol{ABOUT AUTHORS}

\def\rightkol{ABOUT AUTHORS}

\vspace*{-18pt}

\titele{\tit}{\aut}{\auf}{\leftkol}{\rightkol}

%\vspace*{36pt}

\def\rightmark{{\noindent\hbox to \textwidth{\hfill\small ABOUT AUTHORS
%\hfill \large\bf\thepage
}}}
\def\leftmark{{\noindent\parbox{\textwidth}{
%\raggedleft\large\bf\thepage \hfill
\small\textrm{ABOUT AUTHORS}\hfill}}}


\def\leftfootline{\small{\textbf{\thepage}
\hfill ИНФОРМАТИКА И ЕЁ ПРИМЕНЕНИЯ\ \ \ том~6\ \ \ выпуск~2\ \ \ 2012}
}%
 \def\rightfootline{\small{ИНФОРМАТИКА И ЕЁ ПРИМЕНЕНИЯ\ \ \ том~6\ \ \ выпуск~2\ \ \ 2012
\hfill \textbf{\thepage}}}


\begin{multicols}{2}

\noindent
\textbf{Agalarov Yaver M.} (b.\ 1952)~--- Candidate of Science (PhD)
in technology, 
leading scientist, Institute of Informatics Problems, Russian Academy of Sciences

\vspace*{5pt}


  \noindent
\textbf{Bosov Alexey V.} (b.\ 1969)~--- Doctor of Science in technology, Head of
Laboratory, Institute of Informatics Problems, Russian Academy of Sciences

\vspace*{5pt}


\noindent
\textbf{Dulin Sergey K.} (b.\ 1950)~--- Doctor of Science in technology, 
professor, senior scientist, Institute of Informatics Problems, Russian Academy of Sciences

\vspace*{5pt}

\noindent
\textbf{Gorshenin Andrey K.}~--- (b.\ 1986)~--- Candidate of Science (PhD)
in physics and mathematics,
senior scientist, Institute of Informatics Problems, Russian Academy of Sciences

\vspace*{5pt}

\noindent
\textbf{Kalenov Nikolay E.}  (b.\ 1945)~--- Doctor of Science in technology,
professor, Director, Library for Natural Sciences,  Russian Academy of Sciences 

\vspace*{5pt}

\noindent
\textbf{Kalinichenko Leonid A.} (b.\ 1937)~--- Doctor of Science in physics and mathematics, 
professor, Honored scientist of RF, 
Head of Laboratory, Institute of Informatics Problems, Russian Academy of Sciences 

\vspace*{5pt}

\noindent
\textbf{Karpov Alexey A.} (b.\ 1978)~--- Candidate of Science (PhD) in technology, 
senior scientist, St.\ Petersburg Institute for
Informatics and Automation,  Russian Academy of Sciences

\vspace*{5pt}

\noindent
\textbf{Kuznetsov Igor P.} (b.\ 1938)~--- Doctor of Science in technology, 
professor, principal scientist, Institute of Informatics Problems, Russian Academy of Sciences

\vspace*{5pt}


\noindent
\textbf{Markova Natalia A.} (b.\ 1950)~--- Candidate of Science (PhD) in
physics and mathematics, leading scientist,  
Institute of Informatics Problems, Russian Academy of Sciences

\vspace*{5pt}

\noindent
\textbf{Nikolaev Andrey V.} (b.\ 1985)~--- Candidate of Science (PhD) in technology, 
senior lecturer, Tchaikovsky Technological Institute, Branch of the Izhevsk State Technical 
University

\vspace*{6pt}

\noindent
\textbf{Pavlov Igor V.} (b.\ 1945)~---  Doctor of Science in physics and mathematics,
professor, Bauman Moscow State Technical University

\vspace*{6pt}

%\columnbreak

\noindent
\textbf{Rozenberg Igor N.} (b.\ 1965)~--- Doctor of Science in technology, 
First Deputy Director General, Research \& Design Institute for Information 
Technology, Signalling and Telecommunications on Railway Transport (JSC NIIAS)

\vspace*{6pt}


\noindent
\textbf{Semenov Konstantin K.} (b.\ 1986)~--- MPhil, 
associate professor, St.\ Petersburg State Polytechnical University

\vspace*{6pt}

\noindent
\textbf{Sharnin Mikhail M.} (b.\ 1959)~--- Candidate of Science (PhD) 
in technology, senior scientist, Institute of Informatics Problems, Russian Academy of Sciences

\vspace*{6pt}

\noindent 
\textbf{Shestakov Oleg V.} (b.\ 1976)~--- Candidate of Science (PhD) in physics and mathematics,
associate professor, Department of Mathematical Statistics, Faculty of Computational Mathematics and Cybernetics,
M.\,V.~Lomonosov Moscow State University; senior scientist, Institute of Informatics Problems, 
Russian Academy of Sciences

\vspace*{6pt}

\noindent
\textbf{Stupnikov Sergey A.} (b.\ 1978)~--- Candidate of Science (PhD) in technology, 
senior scientist, Institute of Informatics Problems, Russian Academy of Sciences 

\vspace*{6pt}

\noindent
\textbf{Umansky Vladimir I.} (b.\ 1954)~--- Candidate of Science (PhD) in technology, 
Director General, ``IntechGeoTrans'' Closed Joint Stock Company

\vspace*{6pt}

\noindent
\textbf{Zhevnerchuk Dmitry V.} (b.\ 1978)~--- Candidate of Science (PhD) in technology, 
associate professor, Tchaikovsky Technological Institute, Branch of the Izhevsk State 
Technical University

%\vspace*{6pt}

\def\leftfootline{\small{\textbf{\thepage}
\hfill ИНФОРМАТИКА И ЕЁ ПРИМЕНЕНИЯ\ \ \ том~6\ \ \ выпуск~2\ \ \ 2012}
}%
 \def\rightfootline{\small{ИНФОРМАТИКА И ЕЁ ПРИМЕНЕНИЯ\ \ \ том~6\ \ \ выпуск~2\ \ \ 2012
\hfill \textbf{\thepage}}}



%\thispagestyle{myheadings}

\end{multicols}
\newpage



%\include{IPPM-25}

\def\stat{cont}
{%\hrule\par
%\vskip 7pt % 7pt
\raggedleft\Large \bf%\baselineskip=3.2ex
А\,В\,Т\,О\,Р\,С\,К\,И\,Й\ \ У\,К\,А\,З\,А\,Т\,Е\,Л\,Ь\ \ З\,А\ \ 2\,0\,1\,0 г. \vskip 17pt
    \hrule
    \par
\vskip 21pt plus 6pt minus 3pt }

\label{st\stat}

\def\tit{\ }

\def\aut{\ }
\def\auf{\ }

\def\leftkol{\ } % ENGLISH ABSTRACTS}

\def\rightkol{\ } %АВТОРСКИЙ УКАЗАТЕЛЬ ЗА 2010 г.} %ENGLISH ABSTRACTS}

\titele{\tit}{\aut}{\auf}{\leftkol}{\rightkol}

\vspace*{-12pt}

{\tabcolsep=3pt
\begin{tabular}{p{388pt}rr}
&\textbf{Выпуск} & \textbf{Стр.}\\[6pt]
\hangindent=23pt\noindent\textbf{Арутюнян~А.\,Р.} Моделирование влияния деформаций отпечатков пальцев на 
точность\linebreak
\vspace*{-12pt}\\
\hspace*{23pt}дактилоскопической идентификации$\dotfill$&1&51\\
\hangindent=23pt\noindent\textbf{Архипов~О.\,П., Зыкова~З.\,П.} Интеграция гетерогенной информации о цветных 
пикселях\linebreak
\vspace*{-12pt}\\
\hspace*{23pt}и их цветовосприятии$\dotfill$&4&15\\
\hangindent=23pt\noindent\textbf{Баранов~С.\,И., Френкель~С.\,Л., Захаров~В.\,Н.} Полуформальная верификация 
цифрового устройства с конвейером, основанная на использовании алгоритмических машин\linebreak
\vspace*{-12pt}\\
\hspace*{23pt}состояния$\dotfill$&4&49\\
\textbf{Бекетова~И.\,В.} см.~Каратеев~С.\,Л.&&\\
\textbf{Белоусов~В.\,В.} см.~Синицын~И.\,Н.&&\\
\hangindent=23pt\noindent\textbf{Бенинг~В.\,Е., Королев~Р.\,А.} О предельном поведении мощностей критериев в 
случае\linebreak
\vspace*{-12pt}\\
\hspace*{23pt}распределения Лапласа$\dotfill$&2&63\\
\hangindent=23pt\noindent\textbf{Бенинг~В.\,Е., Сипина~А.\,В.} Асимптотическое разложение для мощности 
критерия,\linebreak
\vspace*{-12pt}\\
\hspace*{23pt}основанного на выборочной медиане, в случае распределения Лапласа$\dotfill$&1&18\\
\textbf{Бондаренко~А.\,В.} см.~Каратеев~С.\,Л.&&\\
\hangindent=23pt\noindent\textbf{Бородина~А.\,В., Морозов~Е.\,В.} Об оценивании асимптотики вероятности 
большого\linebreak
\vspace*{-12pt}\\
\hspace*{23pt}уклонения стационарной регенеративной очереди с одним прибором$\dotfill$&3&29\\
\hangindent=23pt\noindent\textbf{Бунтман~Н.\,В., Минель~Ж.-Л., Ле~Пезан~Д., Зацман~И.\,М.} Типология и 
компьютерное\linebreak
\vspace*{-12pt}\\
\hspace*{23pt}моделирование трудностей перевода$\dotfill$&3&77\\
\textbf{Визильтер~Ю.\,В.} см.~Каратеев~С.\,Л.&&\\
\hangindent=23pt\noindent\textbf{Гавриленко~С.\,В.} Оценки скорости сходимости распределений случайных сумм с 
безгранично делимыми индексами к нормальному закону$\dotfill$&4&81\\
\hangindent=23pt\noindent\textbf{Григорьева~М.\,Е., Шевцова~И.\,Г.} Уточнение неравенства 
Каца--Берри--Эссеена$\dotfill$&2&75\\
\hangindent=23pt\noindent\textbf{Грушо~А.\,А., Грушо~Н.\,А., Тимонина~Е.\,Е.} Поиск конфликтов в политиках 
безопасности: модель случайных графов$\dotfill$&3&38\\
\textbf{Грушо~Н.\,А.} см.~Грушо~А.\,А.&&\\
\hangindent=23pt\noindent\textbf{Гудков~В.\,Ю.} Математические модели изображения отпечатка пальца на основе 
описания линий$\dotfill$&1&58\\
\textbf{Гуртов~А.\,В.} см.~Лукьяненко~А.\,С.&&\\
\textbf{Желтов~С.\,Ю.} см.~Каратеев~С.\,Л.&&\\
\hangindent=23pt\noindent\textbf{Захаров~А.\,А., Серебряков~В.\,А.} Система управления электронной библиотекой 
LibMeta$\dotfill$&4&2\\
\textbf{Захаров~В.\,Н.} см.~Баранов~С.\,И.&&\\
\textbf{Захарова~Т.\,В.} см.~Матвеева~С.\,С.&&\\
\hangindent=23pt\noindent\textbf{Зацаринный~А.\,А., Чупраков~К.\,Г.} Некоторые аспекты выбора технологии для 
постро-\linebreak
\vspace*{-12pt}\\
\hspace*{23pt}ения систем отображения информации ситуационного центра$\dotfill$&3&59\\
\textbf{Зацман~И.\,М.} см.~Бунтман~Н.\,В.&&\\
\hangindent=23pt\noindent\textbf{Зейфман~А.\,И., Коротышева~А.\,В., Сатин~Я.\,А., Шоргин~С.\,Я.} Об 
устойчивости нестаци-\linebreak
\vspace*{-12pt}\\
\hspace*{23pt}онарных систем обслуживания с катастрофами$\dotfill$&3&9\\
\textbf{Зыкова~З.\,П.} см.~Архипов~О.\,П.&&\\
\hangindent=23pt\noindent\textbf{Илюшин~Г.\,Я., Соколов~И.\,А.} Организация управляемого доступа пользователей 
к\linebreak
\vspace*{-12pt}\\
\hspace*{23pt}разнородным ведомственным информационным ресурсам$\dotfill$&1&24\\
\hangindent=23pt\noindent\textbf{Кавагучи~Ю., Ульянов~В.\,В., Фуджикоши~Я.} Приближения для статистик, 
описывающих\linebreak
\vspace*{-12pt}\\
\hspace*{23pt}геометрические свойства данных большой размерности, с оценками 
ошибок$\dotfill$&1&12\\
\hangindent=23pt\noindent\textbf{Каратеев~С.\,Л., Бекетова~И.\,В., Ососков~М.\,В., Князь~В.\,А., 
Визильтер~Ю.\,В., Бондаренко~А.\,В., Желтов~С.\,Ю.} Автоматизированный контроль 
качества цифровых\linebreak
\vspace*{-12pt}\\
\hspace*{23pt}изображений для персональных документов$\dotfill$&1&65\\
\end{tabular}
}

\pagebreak

\def\leftkol{АВТОРСКИЙ УКАЗАТЕЛЬ ЗА 2010 г.} % ENGLISH ABSTRACTS}

\def\rightkol{АВТОРСКИЙ УКАЗАТЕЛЬ ЗА 2010 г.} %ENGLISH ABSTRACTS}

{\tabcolsep=3pt
\begin{tabular}{p{388pt}rr}
&\textbf{Выпуск} & \textbf{Стр.}\\[3pt]
\hangindent=23pt\noindent\textbf{Козеренко~Е.\,Б.} Лингвистические фильтры в статистических моделях машинного\linebreak
\vspace*{-12pt}\\
\hspace*{23pt}перевода$\dotfill$&2&83\\
\hangindent=23pt\noindent\textbf{Козеренко~Е.\,Б., Кузнецов~И.\,П.} Когнитивно-лингвистические представления в 
систе-\linebreak
\vspace*{-12pt}\\
\hspace*{23pt}мах обработки текстов$\dotfill$&3&69\\
\textbf{Князь~В.\,А.} см.~Каратеев~С.\,Л.&&\\
\hangindent=23pt\noindent\textbf{Колесников~А.\,В., Солдатов~С.\,А.} Алгоритм координации для гибридной 
интеллектуальной системы решения сложной задачи оперативно-производственного\linebreak
\vspace*{-12pt}\\
\hspace*{23pt}планирования$\dotfill$&4&61\\
\hangindent=23pt\noindent\textbf{Коновалов~М.\,Г.} О планировании потоков в системах вычислительных 
ресурсов$\dotfill$&2&3\\
\textbf{Конушин~А.\,С.} см.~Конушин~В.\,С.&&\\
\hangindent=23pt\noindent\textbf{Конушин~В.\,С., Кривовязь~Г.\,Р., Конушин~А.\,С.} Алгоритм распознавания людей 
в видео-\linebreak
\vspace*{-12pt}\\
\hspace*{23pt}последовательности по одежде$\dotfill$&1&74\\
\textbf{Корепанов~Э.\, Р.} см.~Синицын~И.\,Н.&&\\
\textbf{Королев~В.\,Ю.} см.~Соколов~И.\,А.&&\\
\textbf{Королев~Р.\,А.} см.~Бенинг~В.\,Е.&&\\
\textbf{Коротышева~А.\,В.} см.~Зейфман~А.\,И.&&\\
\hangindent=23pt\noindent\textbf{Кривенко~М.\,П.} Непараметрическое оценивание элементов байесовского 
клас\-си-\linebreak
\vspace*{-12pt}\\
\hspace*{23pt}фикатора$\dotfill$&2&13\\
\textbf{Кривовязь~Г.\,Р.} см.~Конушин~В.\,С.&&\\
\textbf{Крылов~А.\,С.} см.~Павельева~Е.\,А.&&\\
\hangindent=23pt\noindent\textbf{Крылов~В.\,А.} Моделирование и классификация многоканальных дистанционных\linebreak
\vspace*{-12pt}\\
\hspace*{23pt}изображений с использованием копул$\dotfill$&4&34\\
\hangindent=23pt\noindent\textbf{Крючин~О.\,В.} Разработка параллельных эвристических алгоритмов подбора 
весовых\linebreak
\vspace*{-12pt}\\
\hspace*{23pt}коэффициентов искусственной нейтронной сети$\dotfill$&2&53\\
\hangindent=23pt\noindent\textbf{Кудрявцев~А.\,А., Шоргин~С.\,Я.} Байесовские модели массового обслуживания и 
надеж-\linebreak
\vspace*{-12pt}\\
\hspace*{23pt}ности: характеристики среднего числа заявок в системе $M\vert M \vert 1\vert 
\infty$$\dotfill$&3&16\\
\hangindent=23pt\noindent\textbf{Кузнецов~А.\,А.} Связь между временными и структурно-топологическими 
характери-\linebreak
\vspace*{-12pt}\\
\hspace*{23pt}стиками диаграмм ритма сердца здоровых людей$\dotfill$&4&39\\
\textbf{Кузнецов~И.\,П.} см.~Козеренко~Е.\,Б.&&\\
\textbf{Ле~Пезан~Д.} см.~Бунтман~Н.\,В.&&\\
\hangindent=23pt\noindent\textbf{Лукьяненко~А.\,С., Морозов~Е.\,В., Гуртов~А.\,В.} Анализ сетевого протокола с общей 
функ-\linebreak
\vspace*{-12pt}\\
\hspace*{23pt}цией расширения окна передачи сообщения при конфликтах$\dotfill$&2&46\\
\hangindent=23pt\noindent\textbf{Лямин~О.\,О.} О предельном поведении мощностей критериев в случае обобщенного\linebreak
\vspace*{-12pt}\\
\hspace*{23pt}распределения Лапласа$\dotfill$&3&47\\
\hangindent=23pt\noindent\textbf{Маркин~А.\,В., Шестаков~О.\,В.} Асимптотики оценки риска при пороговой 
обработке\linebreak
\vspace*{-12pt}\\
\hspace*{23pt}вейвлет-вейглет коэффициентов в задаче томографии$\dotfill$&2&36\\
\hangindent=23pt\noindent\textbf{Матвеева~С.\,С., Захарова~Т.\,В.} Сети массового обслуживания с наименьшей 
длиной\linebreak
\vspace*{-12pt}\\
\hspace*{23pt}очереди$\dotfill$&3&22\\
\hangindent=23pt\noindent\textbf{Матюшенко~С.\,И.} Стационарные характеристики двухканальной системы 
обслужива-\linebreak
\vspace*{-12pt}\\
\hspace*{23pt}ния с переупорядочиванием заявок и распределениями фазового типа$\dotfill$&4&68\\
\textbf{Минель~Ж.-Л.} см.~Бунтман~Н.\,В.&&\\
\textbf{Морозов~Е.\,В.} см.~Бородина~А.\,В.&&\\
\textbf{Морозов~Е.\,В.} см.~Лукьяненко~А.\,С.&&\\
\textbf{Ососков~М.\,В.} см.~Каратеев~С.\,Л.&&\\
\hangindent=23pt\noindent\textbf{Павельева~Е.\,А., Крылов~А.\,С.} Поиск и анализ ключевых точек радужной 
оболочки\linebreak
\vspace*{-12pt}\\
\hspace*{23pt}глаза методом преобразования Эрмита$\dotfill$&1&79\\
\textbf{Печинкин~А.\,В.} см.~Френкель~С.\,Л.,&&\\
\hangindent=23pt\noindent\textbf{Протасов~В.\,И.} Составление субъективного портрета с использованием 
эволюционно-\linebreak
\vspace*{-12pt}\\
\hspace*{23pt}го морфинга и квалиметрия метода$\dotfill$&1&83\\
\hangindent=23pt\noindent\textbf{Рудаков~К.\,В., Торшин~И.\,Ю.} Вопросы разрешимости задачи распознавания 
вторичной\linebreak
\vspace*{-12pt}\\
\hspace*{23pt}структуры белка$\dotfill$&2&25\\
\textbf{Сатин~Я.\,А.} см.~Зейфман~А.\,И.&&\\
\hangindent=23pt\noindent\textbf{Сейфуль-Мулюков~Р.\,Б.} Нефть как носитель информации о своем 
происхождении,\linebreak
\vspace*{-12pt}\\
\hspace*{23pt}структуре и эволюции$\dotfill$&1&41\\
\end{tabular}
}

{\tabcolsep=3pt
\begin{tabular}{p{388pt}rr}
&\textbf{Выпуск} & \textbf{Стр.}\\[6pt]
\textbf{Семендяев~Н.\,Н.} см.~Синицын~И.\,Н.&&\\
\textbf{Серебряков~В.\,А.} см.~Захаров~А.\,А.&&\\
\textbf{Синицын~В.\,И.} см.~Синицын~И.\,Н.&&\\
\hangindent=23pt\noindent\textbf{Синицын~И.\,Н., Синицын~В.\,И., Корепанов~Э.\, Р., Белоусов~В.\,В., 
Семендяев~Н.\,Н.} Оперативное построение информационных моделей движения полюса 
Земли\linebreak
\vspace*{-12pt}\\
\hspace*{23pt}методами линейных и линеаризованных фильтров$\dotfill$&1&2\\
\textbf{Сипина~А.\,В.} см.~Бенинг~В.\,Е.&&\\
\hangindent=23pt\noindent\textbf{Соколов~И.\,А.} О работах заслуженного деятеля науки Российской Федерации 
И.\,Н.~Синицына в области информационных технологий и автоматизации (к 70-летию\linebreak
\vspace*{-12pt}\\
\hspace*{23pt}со дня рождения)$\dotfill$&3&84\\
\textbf{Соколов~И.\,А.} см.~Илюшин~Г.\,Я.&&\\
\hangindent=23pt\noindent\textbf{Соколов~И.\,А., Королев~В.\,Ю.} Предисловие$\dotfill$&2&2\\
\textbf{Солдатов~С.\,А.} см.~Колесников~А.\,В.&&\\
\hangindent=23pt\noindent\textbf{Степанов~С.\,Ю.} Использование координатного метода фрагментации 
коммутаторной\linebreak
\vspace*{-12pt}\\
\hspace*{23pt}нейронной сети для сокращения трафика$\dotfill$&2&57\\
\textbf{Тимонина~Е.\,Е.} см.~Грушо~А.\,А.&&\\
\textbf{Торшин~И.\,Ю.} см.~Рудаков~К.\,В.&&\\
\textbf{Ульянов~В.\,В.} см.~Кавагучи~Ю.&&\\
\textbf{Фазекаш~И.} см.~Чупрунов~А.\,Н.&&\\
\textbf{Френкель~С.\,Л.} см.~Баранов~С.\,И.&&\\
\hangindent=23pt\noindent\textbf{Френкель~С.\,Л., Печинкин~А.\,В.} Оценка времени самовосстановления в 
цифровых\linebreak
\vspace*{-12pt}\\
\hspace*{23pt}системах после сбоев, вызываемых переходными помехами$\dotfill$&3&2\\
\textbf{Фуджикоши~Я.} см.~Кавагучи~Ю.&&\\
\hangindent=23pt\noindent\textbf{Цискаридзе~А.\,К.} Математическая модель и метод восстановления позы человека 
по\linebreak
\vspace*{-12pt}\\
\hspace*{23pt}стереопаре силуэтных изображений$\dotfill$&4&27\\
\hangindent=23pt\noindent\textbf{Чупраков~К.\,Г.} К вопросу о размещении коллективных средств отображения в 
ситуа-\linebreak
\vspace*{-12pt}\\
\hspace*{23pt}ционном зале с заданными параметрами$\dotfill$&4&89\\
\textbf{Чупраков~К.\,Г.} см.~Зацаринный~А.\,А.&&\\
\hangindent=23pt\noindent\textbf{Чупрунов~А.\,Н., Фазекаш~И.} Законы повторного логарифма для числа 
безошибочных\linebreak
\vspace*{-12pt}\\
\hspace*{23pt}блоков при помехоустойчивом кодировании$\dotfill$&3&42\\
\textbf{Шевцова~И.\,Г.} см.~Григорьева~М.\,Е.&&\\
\hangindent=23pt\noindent\textbf{Шестаков~О.\,В.} Аппроксимация распределения оценки риска пороговой 
обработки вейвлет-коэффициентов нормальным распределением при использовании 
выбо-\linebreak
\vspace*{-12pt}\\
\hspace*{23pt}рочной дисперсии$\dotfill$&4&73\\
\textbf{Шестаков~О.\,В.} см.~Маркин~А.\,В.&&\\
\textbf{Шоргин~С.\,Я.} см.~Зейфман~А.\,И.&&\\
\textbf{Шоргин~С.\,Я.} см.~Кудрявцев~А.\,А.&&\\
\end{tabular}
}

%\thispagestyle{myheadings}
\def\leftfootline{\small{\textbf{\thepage}
\hfill ИНФОРМАТИКА И ЕЁ ПРИМЕНЕНИЯ\ \ \ том~4\ \ \ выпуск~4\ \ \ 2010}
}%
 \def\rightfootline{\small{ИНФОРМАТИКА И ЕЁ ПРИМЕНЕНИЯ\ \ \ том~4\ \ \ выпуск~4\ \ \ 2010
 \hfill \textbf{\thepage}}}
 \label{end\stat}


%Том 10 Выпуск 1-4 Год 2016

\def\stat{cont-e}
{%\hrule\par
%\vskip 7pt % 7pt
\raggedleft\Large \bf%\baselineskip=3.2ex
2\,0\,1\,6\ \ A\,U\,T\,H\,O\,R\ \ I\,N\,D\,E\,X \vskip 17pt
 \hrule
 \par
\vskip 21pt plus 6pt minus 3pt }

\label{st\stat}

\def\tit{\ }

\def\aut{\ }
\def\auf{\ }

\def\leftkol{\ } %2016 AUTHOR INDEX} % ENGLISH ABSTRACTS}

\def\rightkol{\ } %2016 AUTHOR INDEX} %ENGLISH ABSTRACTS}

\titele{\tit}{\aut}{\auf}{\leftkol}{\rightkol}

\def\leftfootline{\small{\textbf{\thepage}
\hfill INFORMATIKA I EE PRIMENENIYA~--- INFORMATICS AND APPLICATIONS\ \ \ 2016\
\ \ volume~10\ \ \ issue\ 4}
}%
 \def\rightfootline{\small{INFORMATIKA I EE PRIMENENIYA~--- INFORMATICS AND APPLICATIONS\ \ \ 2016\ \ \ volume~10\ \ \ issue\ 4
\hfill \textbf{\thepage}}}

\vspace*{-12pt}
\vspace*{-18pt}

{\tabcolsep=2.8pt
\begin{tabular}{p{382pt}cc}
&\textbf{Issue} & \textbf{Page}\\[6pt]
\Avtors{Agalarov~M.\,Ya.} see~Agalarov~Ya.\,M.&&\\
\Avtors{Agalarov~Ya.\,M., Agalarov~M.\,Ya., and
Shorgin~V.\,S.} About the optimal threshold of queue\linebreak
\\[-12pt]
\hspace*{23pt}length in a~particular problem of profit maximization
in the $M/G/1$ queuing system&2&70--79\\
\Avtors{Alexeyevsky~D.\,A.} BioNLP ontology extraction from 
a~restricted language corpus with\linebreak
\\[-12pt]
\hspace*{23pt}context-free grammars&1&119--128\\
\Avtors{Andreev~S.\,D.} see~Gaidamaka~Yu.\,V.&&\\
\Avtors{Andreev~S.\,D.} see~Ometov~A.\,Ya.&&\\
\Avtors{Arkhipov~O.\,P., Arkhipov~P.\,O., and Sidorkin~I.\,I.} The
option to create a~local coordinate\linebreak
\\[-12pt]
\hspace*{23pt}system for synchronization of selected images&3&91--97\\
\Avtors{Arkhipov~P.\,O.} see~Arkhipov~O.\,P.&&\\
\Avtors{Belousov~V.\,V.} see~Shnurkov~P.\,V.&&\\
\Avtors{Belousov~V.\,V.} see~Shnurkov~P.\,V.&&\\
\Avtors{Bening~V.\,E.} Calculation of~the~asymptotic deficiency
of~some statistical procedures based\linebreak
\\[-12pt]
\hspace*{23pt}on~samples with~random sizes&4&34--45\\
\Avtors{Borisov~A.\,V., Bosov~A.\,V., and Miller~G.\,B.} Modeling and
monitoring of VoIP connection&2&\hphantom{1}2--13\\
\Avtors{Bosov~A.\,V.} see~Borisov~A.\,V.&&\\
\Avtors{Briukhov~D.\,O.} see~Stupnikov~S.\,A.&&\\
\Avtors{Callaos~N.\,K.\ and Seyful-Mulyukov~R.\,B.} Complexity and
its information content&1&129--139\\
\Avtors{Chertok~A.\,V., Kadaner~A.\,I., Khazeeva~G.\,T., and
Sokolov~I.\,A.} Regime switching detection\linebreak
\\[-12pt]
\hspace*{23pt}for~the~Levy driven
Ornstein--Uhlenbeck process using CUSUM methods&4&46--56\\
\Avtors{Chichagov~V.\,V.} Asymptotic expansions of mean absolute
error of uniformly minimum variance unbiased and maximum likelihood
estimators on the one-parameter exponential\linebreak
\\[-12pt]
\hspace*{23pt}family model of lattice distributions&3&66--76\\
\Avtors{Danishevsky~V.\,I.} see~Kolesnikov A.\,V.&&\\
\Avtors{Fazliev~A.\,Z.} see~Kalinichenko~L.\,A.&&\\
\Avtors{Fedoseev~A.\,A.} What is behind the concept of ``knowledge in
small packages''&3&105--110\\
\Avtors{Gaidamaka~Yu.\,V., Andreev~S.\,D., Sopin~E.\,S.,
Samouylov~K.\,E., and Shorgin~S.\,Ya.} Interference analysis
of~the~device-to-device communications model with~regard to~a~signal\linebreak
\\[-12pt]
\hspace*{23pt}propagation environment&4&\hphantom{1}2--10\\
\Avtors{Gasilov~A.\,V.} see~Yakovlev~O.\,A.&&\\
\Avtors{Goncharov~A.\,V.\ and Strijov~V.\,V.} Metric time series
classification using weighted dynamic\linebreak
\\[-12pt]
\hspace*{23pt}warping relative to centroids of classes&2&36--47\\
\Avtors{Gordov~E.\,P.} see~Kalinichenko~L.\,A.&&\\
\Avtors{Gorshenin~A.\,K.} Concept of online service for stochastic
modeling of real processes&1&72--81\\
\Avtors{Gorshenin~A.\,K.} see~Shnurkov~P.\,V.&&\\
\Avtors{Gorshenin~A.\,K.} see~Shnurkov~P.\,V.&&\\
\Avtors{Grusho~A.\,A., Grusho~N.\,A., Zabezhailo~M.\,I., and
Timonina~E.\,E.} Integration of statistical and\linebreak
\\[-12pt]
\hspace*{23pt}deterministic methods for
analysis of information security&3&2--8\\
\Avtors{Grusho~A.\,A., Zabezhailo~M.\,I., and Zatsarinny~A.\,A.} On
the advanced procedure to reduce\linebreak
\\[-12pt]
\hspace*{23pt}calculation of Galois closures&4&\hphantom{1}96--104\\
\Avtors{Grusho~N.\,A.} see~Grusho~A.\,A.&&\\
\Avtors{Havanskov~V.\,A.} see~Minin~V.\,A.&&\\
\Avtors{Inkova~O.\,Yu.} see~Zatsman~I.\,M.&&\\
\Avtors{Isachenko~R.\,V.\ and Strijov~V.\,V.} Metric learning in
multiclass time series classification\linebreak
\\[-12pt]
\hspace*{23pt}problem&2&48--57\\
\end{tabular}
}
\pagebreak

\def\leftfootline{\small{\textbf{\thepage}
\hfill INFORMATIKA I EE PRIMENENIYA~--- INFORMATICS AND APPLICATIONS\ \ \ 2016\
\ \ volume~10\ \ \ issue\ 4}
}%
 \def\rightfootline{\small{INFORMATIKA I EE PRIMENENIYA~---
INFORMATICS AND APPLICATIONS\ \ \ 2016\ \ \ volume~10\ \ \ issue\ 4
\hfill \textbf{\thepage}}}

\def\leftkol{2016 AUTHOR INDEX} % ENGLISH ABSTRACTS}

\def\rightkol{2016 AUTHOR INDEX} %ENGLISH ABSTRACTS}


{\tabcolsep=2.83pt
\begin{tabular}{p{382pt}cc}
&\textbf{Issue} & \textbf{Page}\\[6pt]
\Avtors{Kadaner~A.\,I.} see~Chertok~A.\,V.&&\\[.255pt]
\Avtors{Kalinichenko~L.\,A., Volnova~A.\,A., Gordov~E.\,P.,
Kiselyova~N.\,N., Kovaleva~D.\,A., Malkov~O.\,Yu., Okladnikov~I.\,G.,
Podkolodnyy~N.\,L., Pozanenko~A.\,S., Ponomareva~N.\,V.,
Stupnikov~S.\,A.,} \textbf{and Fazliev~A.\,Z.} Data access challenges for data
intensive\linebreak
\\[-12pt]
\hspace*{23pt}research in Russia&1& 2--22\\[.255pt]
\Avtors{Karasikov~M.\,E.\ and Strijov~V.\,V.} Feature-based
time-series classification&4&121--131\\[.255pt]
\Avtors{Khazeeva~G.\,T.} see~Chertok~A.\,V.&&\\[.255pt]
\Avtors{Khokhlov~Yu.\,S.} Multivariate fractional Levy motion and its
applications&2&\hphantom{1}98--106\\[.255pt]
\Avtors{Kirikov~I.\,A., Kolesnikov~A.\,V., Listopad~S.\,V., and
Rumovskaya~S.\,B.} Fine-grained hybrid\linebreak
\\[-12pt]
\hspace*{23pt}intelligent systems. Part 2:
Bidirectional hybridization&1&\hphantom{1}96--105\\[.255pt]
\Avtors{Kirikov~I.\,A., Kolesnikov~A.\,V., Listopad~S.\,V., and
Rumovskaya~S.\,B.} ``Virtual council''~---\linebreak
\\[-12pt]
\hspace*{23pt}source environment
supporting complex diagnostic decision making&3&81--90\\[.255pt]
\Avtors{Kiselyova~N.\,N.} see~Kalinichenko~L.\,A.&&\\[.255pt]
\Avtors{Kolesnikov A.\,V., Listopad~S.\,V., Rumovskaya~S.\,B., and
Danishevsky~V.\,I.} Informal axiomatic\linebreak
\\[-12pt]
\hspace*{23pt}theory of~the~role visual models&4&114--120\\[.255pt]
\Avtors{Kolesnikov~A.\,V.} see~Kirikov~I.\,A.&&\\[.255pt]
\Avtors{Kolesnikov~A.\,V.} see~Kirikov~I.\,A.&&\\[.255pt]
\Avtors{Kolin~K.\,K.} Humanitarian aspects of information
security&3&111--121\\[.255pt]
\Avtors{Konovalov~M.\,G.\ and Razumchik~R.\,V.} Dispatching
to~two parallel nonobservable queues using\linebreak
\\[-12pt]
\hspace*{23pt}only static
information&4&57--67\\[.255pt]
\Avtors{Korchagin~A.\,Yu.} see~Korolev~V.\,Yu.&&\\[.255pt]
\Avtors{Korchagin~A.\,Yu.} see~Korolev~V.\,Yu.&&\\[.255pt]
\Avtors{Korepanov~E.\,R.} see~Sinitsyn~I.\,N.&&\\[.255pt]
\Avtors{Korepanov~E.\,R.} see~Sinitsyn~I.\,N.&&\\[.255pt]
\Avtors{Korolev~V.\,Yu., Korchagin~A.\,Yu., and Zeifman~A.\,I.} The
Poisson theorem for Bernoulli trials\linebreak
\\[-12pt]
\hspace*{23pt}with~a~random probability
of~success and~a~discrete analog of~the~Weibull distribution&4&11--20\\[.255pt]
\Avtors{Korolev~V.\,Yu., Zeifman~A.\,I., and Korchagin~A.\,Yu.}
Asymmetric Linnik distributions as~limit\linebreak
\\[-12pt]
\hspace*{23pt}laws for~random sums
of~independent random variables with~finite variances&4&21--33\\[.255pt]
\Avtors{Koucheryavy~E.\,A.} see~Ometov~A.\,Ya.&&\\[.255pt]
\Avtors{Kovaleva~D.\,A.} see~Kalinichenko~L.\,A.&&\\[.255pt]
\Avtors{Kovalyov~S.\,P.} Metaprogramming to increase
manufacturability of large-scale software-\linebreak
\\[-12pt]
\hspace*{23pt}intensive systems&1&56--66\\[.255pt]
\Avtors{Krivenko~M.\,P.} Significance tests of feature selection for
classification&3&32--40\\[.255pt]
\Avtors{Kruzhkov~M.\,G.} see~Zalizniak~Anna~A.&&\\[.255pt]
\Avtors{Kruzhkov~M.\,G.} see~Zatsman~I.\,M.&&\\[.255pt]
\Avtors{Kudryavtsev~A.\,A.} Bayesian queueing and reliability models:
\textit{A~priori} distributions with\linebreak
\\[-12pt]
\hspace*{23pt}compact support&1&67--71\\[.255pt]
\Avtors{Kudryavtsev~A.\,A.} Characteristics dependent on the balance
coefficient in Bayesian models\linebreak
\\[-12pt]
\hspace*{23pt}with compact support of \textit{a priori}
distributions&3&77--80\\[.255pt]
\Avtors{Kudryavtsev~A.\,A.\ and Palionnaia~S.\,I.} Bayesian recurrent
model of reliability growth:\linebreak
\\[-12pt]
\hspace*{23pt}Parabolic distribution of parameters&2&80--83\\[.255pt]
\Avtors{Kudryavtsev~A.\,A.\ and Titova~A.\,I.} Bayesian queuing
and~reliability models: Degenerate-\linebreak
\\[-12pt]
\hspace*{23pt}Weibull case&4&68--71\\[.255pt]
\Avtors{Leontyev~N.\,D.\ and Ushakov~V.\,G.} Analysis of a queueing
system with autoregressive arrivals\linebreak
\\[-12pt]
\hspace*{23pt}and nonpreemptive priority&3&15--22\\[.255pt]
\Avtors{Listopad~S.\,V.} see~Kirikov~I.\,A.&&\\[.255pt]
\Avtors{Listopad~S.\,V.} see~Kirikov~I.\,A.&&\\[.255pt]
\Avtors{Listopad~S.\,V.} see~Kolesnikov A.\,V.&&\\[.255pt]
\Avtors{Malkov~O.\,Yu.} see~Kalinichenko~L.\,A.&&\\[.255pt]
\Avtors{Markov~A.\,S., Monakhov~M.\,M., and
Ulyanov~V.\,V.} Generalized Cornish--Fisher expansions\linebreak
\\[-12pt]
\hspace*{23pt}for distributions of statistics based on samples
of random size&2&84--91\\[.255pt]
\Avtors{Melnikov~A.\,K.\ and Ronzhin~A.\,F.} Generalized statistical
method of~text analysis based\linebreak
\\[-12pt]
\hspace*{23pt}on~calculation of~probability distributions
of~statistical values&4&89--95\\
\end{tabular}
}
\pagebreak

\def\leftfootline{\small{\textbf{\thepage}
\hfill INFORMATIKA I EE PRIMENENIYA~--- INFORMATICS AND APPLICATIONS\ \ \ 2016\
\ \ volume~10\ \ \ issue\ 4}
}%
 \def\rightfootline{\small{INFORMATIKA I EE PRIMENENIYA~---
INFORMATICS AND APPLICATIONS\ \ \ 2016\ \ \ volume~10\ \ \ issue\ 4
\hfill \textbf{\thepage}}}

\def\leftkol{2016 AUTHOR INDEX} % ENGLISH ABSTRACTS}

\def\rightkol{2016 AUTHOR INDEX} %ENGLISH ABSTRACTS}


{\tabcolsep=3pt
\begin{tabular}{p{381pt}cc}
&\textbf{Issue} & \textbf{Page}\\[6pt]
\Avtors{Meykhanadzhyan~L.\,A.} Stationary characteristics of the finite
capacity queueing system with\linebreak
\\[-12pt]
\hspace*{23pt}inverse service order and generalized
probabilistic priority&2&123--131\\[.23pt]
\Avtors{Miller~G.\,B.} see~Borisov~A.\,V.&&\\[.23pt]
\Avtors{Minin~V.\,A., Zatsman~I.\,M., Havanskov~V.\,A., and
Shubnikov~S.\,K.} Intensity of citation of scientific publications in
inventions on information and computer technologies patented\linebreak
\\[-12pt]
\hspace*{23pt}in Russia by domestic and foreign applicants&2&107--122\\[.23pt]
\Avtors{Monakhov~M.\,M.} see~Markov~A.\,S.&&\\[.23pt]
\Avtors{Naumov~V.\,A.\ and Samouylov~K.\,E.} On relationship
between queuing systems with resources\linebreak
\\[-12pt]
\hspace*{23pt}and Erlang networks&3&\hphantom{1}9--14\\[.23pt]
\Avtors{Okladnikov~I.\,G.} see~Kalinichenko~L.\,A.&&\\[.23pt]
\Avtors{Ometov~A.\,Ya., Andreev~S.\,D., Turlikov~A.\,M., and
Koucheryavy~E.\,A.} Performance analysis of\linebreak
\\[-12pt]
\hspace*{23pt}a wireless data
aggregation system with contention for contemporary sensor
networks&3&23--31\\[.23pt]
\Avtors{Palionnaia~S.\,I.} see~Kudryavtsev~A.\,A.&&\\[.23pt]
\Avtors{Podkolodnyy~N.\,L.} see~Kalinichenko~L.\,A.&&\\[.23pt]
\Avtors{Ponomareva~N.\,V.} see~Kalinichenko~L.\,A.&&\\[.23pt]
\Avtors{Popkova~N.\,A.} see~Zatsman~I.\,M.&&\\[.23pt]
\Avtors{Pozanenko~A.\,S.} see~Kalinichenko~L.\,A.&&\\[.23pt]
\Avtors{Razumchik~R.\,V.} see~Konovalov~M.\,G.&&\\[.23pt]
\Avtors{Ronzhin~A.\,F.} see~Melnikov~A.\,K.&&\\[.23pt]
\Avtors{Rumovskaya~S.\,B.} see~Kirikov~I.\,A.&&\\[.23pt]
\Avtors{Rumovskaya~S.\,B.} see~Kirikov~I.\,A.&&\\[.23pt]
\Avtors{Rumovskaya~S.\,B.} see~Kolesnikov A.\,V.&&\\[.23pt]
\Avtors{Samouylov~K.\,E.} see~Gaidamaka~Yu.\,V.&&\\[.23pt]
\Avtors{Samouylov~K.\,E.} see~Naumov~V.\,A.&&\\[.23pt]
\Avtors{Serebryanskii~S.\,M.} see~Tyrsin~A.\,N.&&\\[.23pt]
\Avtors{Seyful-Mulyukov~R.\,B.} see~Callaos~N.\,K.&&\\[.23pt]
\Avtors{Shestakov~O.\,V.} Statistical properties of the denoising method
based on the stabilized hard\linebreak
\\[-12pt]
\hspace*{23pt}thresholding&2&65--69\\[.23pt]
\Avtors{Shestakov~O.\,V.} The strong law of large numbers for the risk
estimate in the problem of\linebreak
\\[-12pt]
\hspace*{23pt}tomographic image reconstruction from
projections with a correlated noise&3&41--45\\[.23pt]
\Avtors{Shestakov~O.\,V.} see~Zakharova~T.\,V.&&\\[.23pt]
\Avtors{Shnurkov~P.\,V., Gorshenin~A.\,K., and Belousov~V.\,V.}
Analytical solution of~the~optimal control\linebreak
\\[-12pt]
\hspace*{23pt}task of~a~semi-Markov
process with~finite set of~states&4&72--88\\[.23pt]
\Avtors{Shnurkov~P.\,V., Zasypko~V.\,V., Belousov~V.\,V., and
Gorshenin~A.\,K.} Development of the algorithm of numerical solution
of the optimal investment control problem\linebreak
\\[-12pt]
\hspace*{23pt}in the closed dynamical model of three-sector economy&1&82--95\\[.23pt]
\Avtors{Shorgin~S.\,Ya.} see~Gaidamaka~Yu.\,V.&&\\[.23pt]
\Avtors{Shorgin~V.\,S.} see~Agalarov~Ya.\,M.&&\\[.23pt]
\Avtors{Shubnikov~S.\,K.} see~Minin~V.\,A.&&\\[.23pt]
\Avtors{Sidorkin~I.\,I.} see~Arkhipov~O.\,P.&&\\[.23pt]
\Avtors{Sinitsyn~I.\,N.} Analytical modeling of processes in stochastic
systems with complex fractional\linebreak
\\[-12pt]
\hspace*{23pt}order Bessel nonlinearities&3&55--65\\[.23pt]
\Avtors{Sinitsyn~I.\,N.} Orthogonal supoptimal filters for nonlinear
stochastic systems on manifolds&1&34--44\\[.23pt]
\Avtors{Sinitsyn~I.\,N.\ and Korepanov~E.\,R.} Normal Pugachev
conditionally-optimal filters and extra-\linebreak
\\[-12pt]
\hspace*{23pt}polators for state linear stochastic systems&2&14--23\\[.23pt]
\Avtors{Sinitsyn~I.\,N.\ and Sinitsyn~V.\,I.} Analytical modeling of
distributions in stochastic systems on\linebreak
\\[-12pt]
\hspace*{23pt}manifolds based on ellipsoidal approximation&1&45--55\\[.23pt]
\Avtors{Sinitsyn~I.\,N., Sinitsyn~V.\,I., and
Korepanov~E.\,R.} Ellipsoidal suboptimal filters for nonlinear\linebreak
\\[-12pt]
\hspace*{23pt}stochastic systems on manifolds&2&24--35\\[.23pt]
\Avtors{Sinitsyn~V.\,I.} see~Sinitsyn~I.\,N.&&\\[.23pt]
\Avtors{Sinitsyn~V.\,I.} see~Sinitsyn~I.\,N.&&\\[.23pt]
\Avtors{Skvortsov~N.\,A.} see~Stupnikov~S.\,A.&&\\[.23pt]
\Avtors{Sokolov~I.\,A.} see~Chertok~A.\,V.&&\\
\end{tabular}
}
\pagebreak

\def\leftfootline{\small{\textbf{\thepage}
\hfill INFORMATIKA I EE PRIMENENIYA~--- INFORMATICS AND APPLICATIONS\ \ \ 2016\
\ \ volume~10\ \ \ issue\ 4}
}%
 \def\rightfootline{\small{INFORMATIKA I EE PRIMENENIYA~---
INFORMATICS AND APPLICATIONS\ \ \ 2016\ \ \ volume~10\ \ \ issue\ 4
\hfill \textbf{\thepage}}}

\def\leftkol{2016 AUTHOR INDEX} % ENGLISH ABSTRACTS}

\def\rightkol{2016 AUTHOR INDEX} %ENGLISH ABSTRACTS}


{\tabcolsep=3pt
\begin{tabular}{p{382pt}cc}
&\textbf{Issue} & \textbf{Page}\\[6pt]
\Avtors{Sopin~E.\,S.} see~Gaidamaka~Yu.\,V.&&\\
\Avtors{Strijov~V.\,V.} see~Goncharov~A.\,V.&&\\
\Avtors{Strijov~V.\,V.} see~Isachenko~R.\,V.&&\\
\Avtors{Strijov~V.\,V.} see~Karasikov~M.\,E.&&\\
\Avtors{Stupnikov~S.\,A., Briukhov~D.\,O., and Skvortsov~N.\,A.}
Co-lending systemic risk analysis over\linebreak
\\[-12pt]
\hspace*{23pt}heterogeneous data collections&1&23--33\\
\Avtors{Stupnikov~S.\,A.} see~Kalinichenko~L.\,A.&&\\
\Avtors{Suchkov~A.\,P.} see~Zatsarinny~A.\,A.&&\\
\Avtors{Timonina~E.\,E.} see~Grusho~A.\,A.&&\\
\Avtors{Titova~A.\,I.} see~Kudryavtsev~A.\,A.&&\\
\Avtors{Turlikov~A.\,M.} see~Ometov~A.\,Ya.&&\\
\Avtors{Tyrsin~A.\,N.\ and Serebryanskii~S.\,M.} Recognition of
dependences on the basis of inverse\linebreak
\\[-12pt]
\hspace*{23pt}mapping&2&58--64\\
\Avtors{Ulyanov~V.\,V.} see~Markov~A.\,S.&&\\
\Avtors{Ushakov~V.\,G.} Queueing system with working vacations and
hyperexponential input stream&2&92--97\\
\Avtors{Ushakov~V.\,G.} see~Leontyev~N.\,D.&&\\
\Avtors{Volnova~A.\,A.} see~Kalinichenko~L.\,A.&&\\
\Avtors{Yakovlev~O.\,A.\ and Gasilov~A.\,V.} Speeded-up stereo
matching using geodesic support weights&3&\hphantom{1}98--104\\
\Avtors{Zabezhailo~M.\,I.} see~Grusho~A.\,A.&&\\
\Avtors{Zabezhailo~M.\,I.} see~Grusho~A.\,A.&&\\
\Avtors{Zakharova~T.\,V.\ and Shestakov~O.\,V.} Precision analysis of
wavelet processing of aerodynamic\linebreak
\\[-12pt]
\hspace*{23pt}flow patterns&3&46--54\\
\Avtors{Zalizniak~Anna~A.\ and Kruzhkov~M.\,G.} Database
of~Russian impersonal verbal constructions&4&132--141\\
\Avtors{Zasypko~V.\,V.} see~Shnurkov~P.\,V.&&\\
\Avtors{Zatsarinny~A.\,A.\ and Suchkov~A.\,P.} Systems engineering
approaches to~the~establishment of\linebreak
\\[-12pt]
\hspace*{23pt}a~system for~decision support based
on~situational analysis&4&105--113\\
\Avtors{Zatsarinny~A.\,A.} see~Grusho~A.\,A.&&\\
\Avtors{Zatsman~I.\,M., Inkova~O.\,Yu., Kruzhkov~M.\,G., and
Popkova~N.\,A.} Representation of cross-\linebreak
\\[-12pt]
\hspace*{23pt}lingual knowledge about
connectors in supracorpora databases&1&106--118\\
\Avtors{Zatsman~I.\,M.} see~Minin~V.\,A.&&\\
\Avtors{Zeifman~A.\,I.} see~Korolev~V.\,Yu.&&\\
\Avtors{Zeifman~A.\,I.} see~Korolev~V.\,Yu.&&\\
\end{tabular}
}

%\thispagestyle{myheadings}
\def\leftfootline{\small{\textbf{\thepage}
\hfill INFORMATIKA I EE PRIMENENIYA~--- INFORMATICS AND APPLICATIONS\ \ \ 2016\
\ \ volume~10\ \ \ issue\ 4}
}%
 \def\rightfootline{\small{INFORMATIKA I EE PRIMENENIYA~---
INFORMATICS AND APPLICATIONS\ \ \ 2016\ \ \ volume~10\ \ \ issue\ 4
\hfill \textbf{\thepage}}}

 \label{end\stat}

\newpage


\vspace*{-60pt} {\small
{\baselineskip=9.1pt
\section*{Правила подготовки рукописей статей для публикации в журнале
<<Информатика и её применения>>}

\thispagestyle{empty}

 Журнал <<Информатика и её применения>> публикует
теоретические, обзорные и дискуссионные статьи, посвященные научным
исследованиям и разработкам в области информатики и ее приложений. Журнал
издается на русском языке. По специальному решению редколлегии отдельные статьи,
в виде исключения, могут печататься на английском языке.
Тематика журнала охватывает следующие направления:
\begin{itemize}
\item теоретические основы информатики; %\\[-13.5pt]
\item математические методы исследования сложных систем и процессов; %\\[-13.5pt]
\item информационные системы и сети; %\\[-13.5pt]
\item информационные технологии; %\\[-13.5pt]
\item архитектура и программное
обеспечение вычислительных комплексов и сетей.
\end{itemize}
\begin{enumerate}
\item В журнале печатаются результаты, ранее не
опубликованные и не предназначенные к одновременной публикации в других
изданиях. Публикация не должна нарушать закон об авторских правах. Направляя
свою рукопись в редакцию, авторы автоматически передают учредителям и
редколлегии неисключительные права на издание данной статьи на русском языке и
на ее распространение в России и за рубежом. При этом за авторами сохраняются
все права как собственников данной рукописи. В связи с этим авторами должно
быть представлено в редакцию письмо в следующей форме:
Соглашение о передаче права на публикацию:

\textit{<<Мы, нижеподписавшиеся, авторы рукописи <<$\qquad\qquad$>>, передаем
учредителям и редколлегии журнала <<Информатика и её применения>>
неисключительное право опубликовать данную рукопись статьи на русском языке как
в печатной, так и в электронной версиях журнала. Мы подтверждаем, что данная
публикация не нарушает авторского права других лиц или организаций. Подписи
авторов: (ф.\,и.\,о., дата, адрес)>>.}

Указанное соглашение может быть представлено 
как в бумажном виде, так и в виде отсканированной копии (с подписями авторов).


Редколлегия вправе запросить у авторов экспертное заключение о возможности
опубликования представленной статьи в открытой печати. %\\[-13.5pt]
\item Статья
подписывается всеми авторами. На отдельном листе представляются данные автора
(или всех авторов): фамилия, полные имя и отчество, телефон, факс, e-mail,
почтовый адрес. Если работа выполнена несколькими авторами, указывается фамилия
одного из них, ответственного за переписку с редакцией. %\\[-13.5pt]
\item Редакция журнала
осуществляет самостоятельную экспертизу присланных статей. Возвращение рукописи
на доработку не означает, что статья уже принята к печати. Доработанный вариант
с ответом на замечания рецензента необходимо прислать в редакцию. %\\[-13.5pt]
\item Решение
редакционной коллегии о принятии статьи к печати или ее отклонении сообщается
авторам. Редколлегия не обязуется направлять рецензию авторам отклоненной
статьи. %\\[-13.5pt]
\item Корректура статей высылается авторам для просмотра. Редакция
просит авторов присылать свои замечания в кратчайшие сроки. %\\[-13.5pt]
\item При
подготовке рукописи в MS Word рекомендуется использовать следующие настройки.
Параметры страницы: формат~--- А4; ориентация~--- книжная; поля (см): внутри~---
2,5, снаружи~--- 1,5, сверху~--- 2, снизу~--- 2, от края до нижнего
колонтитула~--- 1,3. Основной текст: стиль~--- <<Обычный>>: шрифт Times New
Roman, размер 14~пунктов, абзацный отступ~--- 0,5~см, 1,5 интервала,
выравнивание~--- по ширине. Рекомендуемый объем рукописи~--- не свыше
25~страниц указанного формата. Ознакомиться с шаблонами, содержащими примеры
оформления, можно по адресу в Интернете:
\textsf{http://www.ipiran.ru/journal/template.doc}.
\item К рукописи, предоставляемой в 2-х
экземплярах, обязательно прилагается электронная версия статьи (как правило, в
форматах MS WORD (.doc) или \LaTeX\ (.tex), а также~--- дополнительно~--- в
формате .pdf) на дискете, лазерном диске или по электронной почте. Сокращения
слов, кроме стандартных, не применяются. Все страницы рукописи должны быть
пронумерованы. %\\[-13.5pt]
\item Статья должна содержать следующую информацию на русском и
английском языках: название, Ф.И.О. авторов, места работы авторов и их
электронные адреса, подробные сведения об авторах, оформленные в соответствии с форматом, 
определяемым файлами {\sf http://www.ipiran.ru/journal/issues/2011\_05\_01/authors.asp} и 
{\sf http://www.ipiran.ru/journal/issues/2011\_01\_eng/authors.asp},
аннотация (не более 100~слов), ключевые слова. Ссылки на
литературу в тексте статьи нумеруются (в квадратных скобках) и располагаются в
порядке их первого упоминания. В~списке литературы не должно быть позиций, на которые нет ссылки в тексте статьи.
Все фамилии авторов, заглавия статей, названия
книг, конференций и~т.\,п.\ даются на языке оригинала, если этот язык
использует кириллический или латинский алфавит. %\\[-13.5pt]
\item Присланные в редакцию материалы авторам не возвращаются.
\item При отправке файлов по электронной
почте просим придерживаться следующих правил:
\begin{itemize}
\item указывать в поле subject (тема) название журнала и фамилию автора; %\\[-13.5pt]
\item использовать attach (присоединение); %\\[-13.5pt]
\item в случае больших объемов информации возможно
использование общеизвестных архиваторов (ZIP, RAR); %\\[-13.5pt]
\item в состав электронной версии статьи должны входить: файл, содержащий текст статьи, и файл(ы),
содержащий(е) иллюстрации. %\\[-13.5pt]
\end{itemize}
\item Журнал <<Информатика и её применения>> является некоммерческим изданием. 
Плата за публикацию с авторов не взимается, гонорар авторам не выплачивается.
\end{enumerate}
\thispagestyle{empty}
\textbf{Адрес редакции:} Москва 119333,
ул.~Вавилова, д.~44, корп.~2, ИПИ РАН\\
\hphantom{\textbf{Адрес редакции:} }Тел.: +7 (499) 135-86-92\ \
Факс:  +7 (495) 930-45-05\ \  E-mail:   rust@ipiran.ru }
}

\end{document}


%\tableofcontents

%\end{document}





%\def\stat{cont}
{%\hrule\par
%\vskip 7pt % 7pt
\raggedleft\Large \bf%\baselineskip=3.2ex
А\,В\,Т\,О\,Р\,С\,К\,И\,Й\ \ У\,К\,А\,З\,А\,Т\,Е\,Л\,Ь\ \ З\,А\ \ 2\,0\,0\,7 г. \vskip 17pt
    \hrule
    \par
\vskip 21pt plus 6pt minus 3pt }

\label{st\stat}

\def\tit{\ }

\def\aut{\ }
\def\auf{\ }

\def\leftkol{\ } % ENGLISH ABSTRACTS}

\def\rightkol{\ } %ENGLISH ABSTRACTS}

\titele{\tit}{\aut}{\auf}{\leftkol}{\rightkol}


\contentsline {chapter}{\ }{Выпуск \quad Стр.} 
\contentsline {section}{\textbf{Батракова Д.\,А., Королев В.\,Ю., Шоргин С.\,Я.}\ \ Новый метод вероятностно-ста\-ти\-сти\-че\-ско\-го анализа информационных потоков в\nobreakspace {}телекоммуникационных сетях}{\qquad 1 \qquad 40} 
\contentsline {section}{\textbf{Борисов А.\,В.}\ \ Байесовское оценивание в системах наблюдения с\nobreakspace {}марковскими скачкообразными процессами: игровой подход}{\qquad 2 \qquad 65}
\contentsline {section}{\textbf{Босов А.\,В., Иванов А.\,В.}\ \ Программная инфраструктура информационного Web-пор\-тала}{\qquad 2 \qquad 50}
\contentsline {section}{\textbf{Захаров В.\,Н., Калиниченко Л.\,А., Соколов И.\,А., Ступников С.\,А.}\ \ Конструирование канонических информационных моделей для интегрированных информационных систем}{\qquad 2 \qquad 15}
\contentsline {section}{\textbf{Захаров В.\,Н., Козмидиади В.\,А.}\ \ Средства обеспечения отказоустойчивости при\-ло\-жений}{\qquad 1 \qquad 14} 
\contentsline {section}{\textbf{Иванов А.\,В.}\ \ см. Босов А.\,В.\hfill\hfill\hfill\hfill\hfill\hfill\hfill\hfill\hfill\hfill\hfill\hfill\hfill\hfill\hfill\hfill\hfill\hfill\hfill\hfill\hfill\hfill\hfill\hfill\hfill\hfill\hfill\hfill\hfill\hfill\hfill\hfill\hfill\hfill\hfill}{\ }
\contentsline {section}{\textbf{Ильин В.\,Д., Соколов И.\,А.}\ \ Символьная модель системы знаний информатики в\nobreakspace {}че\-ло\-ве\-ко-автоматной среде}{\qquad 1 \qquad 66} 
\contentsline {section}{\textbf{Калиниченко Л.\,А.}\ \ см. Захаров В.\,Н.\hfill\hfill\hfill\hfill\hfill\hfill\hfill\hfill\hfill\hfill\hfill\hfill\hfill\hfill\hfill\hfill\hfill\hfill\hfill\hfill\hfill\hfill\hfill\hfill\hfill\hfill\hfill\hfill\hfill\hfill\hfill\hfill\hfill\hfill\hfill}{\ }
\contentsline {section}{\textbf{Козеренко Е.\,Б.}\ \ Лингвистическое моделирование для систем машинного перевода и обработки знаний}{\qquad 1 \qquad 54} 
\contentsline {section}{\textbf{Козмидиади В.\,А.}\ \ см. Захаров В.\,Н.\hfill\hfill\hfill\hfill\hfill\hfill\hfill\hfill\hfill\hfill\hfill\hfill\hfill\hfill\hfill\hfill\hfill\hfill\hfill\hfill\hfill\hfill\hfill\hfill\hfill\hfill\hfill\hfill\hfill\hfill\hfill\hfill\hfill\hfill\hfill }{\ } 
\contentsline {section}{\textbf{Королев В.\,Ю.}\ \ см. Батракова Д.\,А.\hfill\hfill\hfill\hfill\hfill\hfill\hfill\hfill\hfill\hfill\hfill\hfill\hfill\hfill\hfill\hfill\hfill\hfill\hfill\hfill\hfill\hfill\hfill\hfill\hfill\hfill\hfill\hfill\hfill\hfill\hfill\hfill\hfill\hfill\hfill}{\ } 
\contentsline {section}{\textbf{Кудрявцев А.\,А., Шоргин С.\,Я.}\ \ Байесовский подход к\nobreakspace {}анализу систем массового обслуживания и\nobreakspace {}показателей надежности}{\qquad 2 \qquad 76}
\contentsline {section}{\textbf{Печинкин А.\,В., Соколов И.\,А., Чаплыгин В.\,В.}\ \ Многолинейная система массового обслуживания с конечным накопителем и ненадежными приборами}{\qquad 1 \qquad 27} 
\contentsline {section}{\textbf{Печинкин А.\,В., Соколов И.\,А., Чаплыгин В.\,В.}\ \ Стационарные характеристики многолинейной\nobreakspace {}системы массового обслуживания с\nobreakspace {}одновременными отказами приборов}{\qquad 2 \qquad 39}
\contentsline {section}{\textbf{Синицын И.\,Н.}\ \ Корреляционные методы построения аналитических информационных моделей флуктуаций полюса Земли по априорным данным}{\qquad 2 \qquad \hphantom{9}2}
\contentsline {section}{\textbf{Синицын И.\,Н.}\ \ Развитие теории фильтров Пугачева для оперативной обработки информации в стохастических системах}{{\qquad 1 \qquad \hphantom{9}3}} 
\contentsline {section}{\textbf{Соколов И.\,А.}\ \ см. Захаров В.\,Н.\hfill\hfill\hfill\hfill\hfill\hfill\hfill\hfill\hfill\hfill\hfill\hfill\hfill\hfill\hfill\hfill\hfill\hfill\hfill\hfill\hfill\hfill\hfill\hfill\hfill\hfill\hfill\hfill\hfill\hfill\hfill\hfill\hfill\hfill\hfill}{\ }
\contentsline {section}{\textbf{Соколов И.\,А.}\ \ см. Ильин В.\,Д.\hfill\hfill\hfill\hfill\hfill\hfill\hfill\hfill\hfill\hfill\hfill\hfill\hfill\hfill\hfill\hfill\hfill\hfill\hfill\hfill\hfill\hfill\hfill\hfill\hfill\hfill\hfill\hfill\hfill\hfill\hfill\hfill\hfill\hfill\hfill}{\ } 
\contentsline {section}{\textbf{Соколов И.\,А.}\ \ см. Печинкин А.\,В.\hfill\hfill\hfill\hfill\hfill\hfill\hfill\hfill\hfill\hfill\hfill\hfill\hfill\hfill\hfill\hfill\hfill\hfill\hfill\hfill\hfill\hfill\hfill\hfill\hfill\hfill\hfill\hfill\hfill\hfill\hfill\hfill\hfill\hfill\hfill}{\ } 
\contentsline {section}{\textbf{Соколов И.\,А.}\ \ см. Печинкин А.\,В.\hfill\hfill\hfill\hfill\hfill\hfill\hfill\hfill\hfill\hfill\hfill\hfill\hfill\hfill\hfill\hfill\hfill\hfill\hfill\hfill\hfill\hfill\hfill\hfill\hfill\hfill\hfill\hfill\hfill\hfill\hfill\hfill\hfill\hfill\hfill}{\ }
\contentsline {section}{\textbf{Ступников С.\,А.}\ \ см. Захаров В.\,Н.\hfill\hfill\hfill\hfill\hfill\hfill\hfill\hfill\hfill\hfill\hfill\hfill\hfill\hfill\hfill\hfill\hfill\hfill\hfill\hfill\hfill\hfill\hfill\hfill\hfill\hfill\hfill\hfill\hfill\hfill\hfill\hfill\hfill\hfill\hfill}{\ }
\contentsline {section}{\textbf{Чаплыгин В.\,В.}\ \ см. Печинкин А.\,В.\hfill\hfill\hfill\hfill\hfill\hfill\hfill\hfill\hfill\hfill\hfill\hfill\hfill\hfill\hfill\hfill\hfill\hfill\hfill\hfill\hfill\hfill\hfill\hfill\hfill\hfill\hfill\hfill\hfill\hfill\hfill\hfill\hfill\hfill\hfill}{\ } 
\contentsline {section}{\textbf{Чаплыгин В.\,В.}\ \ см. Печинкин А.\,В.\hfill\hfill\hfill\hfill\hfill\hfill\hfill\hfill\hfill\hfill\hfill\hfill\hfill\hfill\hfill\hfill\hfill\hfill\hfill\hfill\hfill\hfill\hfill\hfill\hfill\hfill\hfill\hfill\hfill\hfill\hfill\hfill\hfill\hfill\hfill}{\ }
\contentsline {section}{\textbf{Шоргин С.\,Я.}\ \ см. Батракова Д.\,А.\hfill\hfill\hfill\hfill\hfill\hfill\hfill\hfill\hfill\hfill\hfill\hfill\hfill\hfill\hfill\hfill\hfill\hfill\hfill\hfill\hfill\hfill\hfill\hfill\hfill\hfill\hfill\hfill\hfill\hfill\hfill\hfill\hfill\hfill\hfill}{\ } 
\contentsline {section}{\textbf{Шоргин С.\,Я.}\ \ см. Кудрявцев А.\,А.\hfill\hfill\hfill\hfill\hfill\hfill\hfill\hfill\hfill\hfill\hfill\hfill\hfill\hfill\hfill\hfill\hfill\hfill\hfill\hfill\hfill\hfill\hfill\hfill\hfill\hfill\hfill\hfill\hfill\hfill\hfill\hfill\hfill\hfill\hfill}{\ }
%\thispagestyle{myheadings}
\def\leftfootline{\small{\textbf{\thepage}
\hfill ИНФОРМАТИКА И ЕЁ ПРИМЕНЕНИЯ\ \ \ том~1\ \ \ выпуск~2\ \ \ 2007}
}%
 \def\rightfootline{\small{ИНФОРМАТИКА И ЕЁ ПРИМЕНЕНИЯ\ \ \ том~1\ \ \ выпуск~2\ \ \ 2007
 \hfill \textbf{\thepage}}}
 \label{end\stat}

%\def\stat{cont-e}
{%\hrule\par
%\vskip 7pt % 7pt
\raggedleft\Large \bf%\baselineskip=3.2ex
2\,0\,0\,7\ \ A\,U\,T\,H\,O\,R\ \ I\,N\,D\,E\,X \vskip 17pt
    \hrule
    \par
\vskip 21pt plus 6pt minus 3pt }

\label{st\stat}

\def\tit{\ }

\def\aut{\ }
\def\auf{\ }

\def\leftkol{\ } % ENGLISH ABSTRACTS}

\def\rightkol{\ } %ENGLISH ABSTRACTS}

\titele{\tit}{\aut}{\auf}{\leftkol}{\rightkol}


\contentsline {chapter}{\ }{Issue \quad Page} 
\contentsline {subsection}{\textbf{Batrakova D.\,A., Korolev V.\,Yu., Shorgin S.\,Ya.}\ \ A New Method for the Probabilistic and Statistical Analysis of Information Flows in Telecommunication Networks}{\qquad 1 \qquad 40} 
\contentsline {subsection}{\textbf{Borisov A.\,V.}\ \ Bayesian Estimation in\nobreakspace {}Observation Systems with\nobreakspace {}Markov Jump Processes: Game-Theoretic Approach}{\qquad 2 \qquad 65} 
\contentsline {subsection}{\textbf{Bosov A.\,V., Ivanov A.\,V.}\ \ Linguistic Simulation for Machine Translation and Knowledge Management Systems}{\qquad 2 \qquad 50} 
\contentsline {subsection}{\textbf{Chaplygin V.\,V.} see Pechinkin A.\,V.\hfill\hfill\hfill\hfill\hfill\hfill\hfill\hfill\hfill\hfill\hfill\hfill\hfill\hfill\hfill\hfill\hfill\hfill\hfill\hfill\hfill\hfill\hfill\hfill\hfill\hfill\hfill\hfill\hfill\hfill\hfill\hfill\hfill\hfill\hfill}{\ }
\contentsline {subsection}{\textbf{Chaplygin V.\,V.} see Pechinkin A.\,V.\hfill\hfill\hfill\hfill\hfill\hfill\hfill\hfill\hfill\hfill\hfill\hfill\hfill\hfill\hfill\hfill\hfill\hfill\hfill\hfill\hfill\hfill\hfill\hfill\hfill\hfill\hfill\hfill\hfill\hfill\hfill\hfill\hfill\hfill\hfill}{\ }
\contentsline {subsection}{\textbf{Ilyin V.\,D., Sokolov I.\,A.}\ \ The Symbol Model of Informatics Knowledge System in Human-Automaton Environment}{\qquad 1 \qquad 66} 
\contentsline {subsection}{\textbf{Ivanov A.\,V.} see Bosov A.\,V.\hfill\hfill\hfill\hfill\hfill\hfill\hfill\hfill\hfill\hfill\hfill\hfill\hfill\hfill\hfill\hfill\hfill\hfill\hfill\hfill\hfill\hfill\hfill\hfill\hfill\hfill\hfill\hfill\hfill\hfill\hfill\hfill\hfill\hfill\hfill}{\ }
\contentsline {subsection}{\textbf{Kalinichenko L.\,A.} see Zakharov V.\,N.\hfill\hfill\hfill\hfill\hfill\hfill\hfill\hfill\hfill\hfill\hfill\hfill\hfill\hfill\hfill\hfill\hfill\hfill\hfill\hfill\hfill\hfill\hfill\hfill\hfill\hfill\hfill\hfill\hfill\hfill\hfill\hfill\hfill\hfill\hfill}{\ }
\contentsline {subsection}{\textbf{Korolev V.\,Yu.} see Batrakova D.\,A.\hfill\hfill\hfill\hfill\hfill\hfill\hfill\hfill\hfill\hfill\hfill\hfill\hfill\hfill\hfill\hfill\hfill\hfill\hfill\hfill\hfill\hfill\hfill\hfill\hfill\hfill\hfill\hfill\hfill\hfill\hfill\hfill\hfill\hfill\hfill}{\ }
\contentsline {subsection}{\textbf{Kozerenko E.\,B.}\ \ Linguistic Simulation for Machine Translation and Knowledge Management Systems}{\qquad 1 \qquad 54} 
\contentsline {subsection}{\textbf{Kozmidiady V.\,A.} see Zakharov V.\,N.\hfill\hfill\hfill\hfill\hfill\hfill\hfill\hfill\hfill\hfill\hfill\hfill\hfill\hfill\hfill\hfill\hfill\hfill\hfill\hfill\hfill\hfill\hfill\hfill\hfill\hfill\hfill\hfill\hfill\hfill\hfill\hfill\hfill\hfill\hfill}{\ }
\contentsline {subsection}{\textbf{Kudryavtsev A.\,A., Shorgin S.\,Ya.}\ \ Bayesian Approach to Queueing Systems and Reliability Characteristics}{\qquad 2 \qquad 76} 
\contentsline {subsection}{\textbf{Pechinkin A.\,V., Sokolov I.\,A., Chaplygin V.\,V.}\ \ Multichannel Queuing System with Finite Buffer and Unreliable Servers}{\qquad 1 \qquad 27} 
\contentsline {subsection}{\textbf{Pechinkin A.\,V., Sokolov I.\,A., Chaplygin V.\,V.}\ \ Stationary Characteristics of a Multichannel Queueing System with\nobreakspace {}Simultaneous Refusals of Servers}{\qquad 2 \qquad 39} 
\contentsline {subsection}{\textbf{Shorgin S.\,Ya.} see Batrakova D.\,A.\hfill\hfill\hfill\hfill\hfill\hfill\hfill\hfill\hfill\hfill\hfill\hfill\hfill\hfill\hfill\hfill\hfill\hfill\hfill\hfill\hfill\hfill\hfill\hfill\hfill\hfill\hfill\hfill\hfill\hfill\hfill\hfill\hfill\hfill\hfill}{\ }
\contentsline {subsection}{\textbf{Shorgin S.\,Ya.} see Kudryavtsev A.\,A.\hfill\hfill\hfill\hfill\hfill\hfill\hfill\hfill\hfill\hfill\hfill\hfill\hfill\hfill\hfill\hfill\hfill\hfill\hfill\hfill\hfill\hfill\hfill\hfill\hfill\hfill\hfill\hfill\hfill\hfill\hfill\hfill\hfill\hfill\hfill}{\ }
\contentsline {subsection}{\textbf{Sinitsyn I.\,N.}\ \ Correlational Methods for Analytical Informational Models of the Earth Pole Fluctuations Design Based on a priori Data}{\qquad 2 \qquad \hphantom{9}2}
\contentsline {subsection}{\textbf{Sinitsyn I.\,N.}\ \ Development of Pugachev Filtering for Stochastic Systems}{\qquad 1 \qquad \hphantom{9}3}
\contentsline {subsection}{\textbf{Sokolov I.\,A.} see Ilyin V.\,D.\hfill\hfill\hfill\hfill\hfill\hfill\hfill\hfill\hfill\hfill\hfill\hfill\hfill\hfill\hfill\hfill\hfill\hfill\hfill\hfill\hfill\hfill\hfill\hfill\hfill\hfill\hfill\hfill\hfill\hfill\hfill\hfill\hfill\hfill\hfill}{\ }
\contentsline {subsection}{\textbf{Sokolov I.\,A.} see Pechinkin A.\,V.\hfill\hfill\hfill\hfill\hfill\hfill\hfill\hfill\hfill\hfill\hfill\hfill\hfill\hfill\hfill\hfill\hfill\hfill\hfill\hfill\hfill\hfill\hfill\hfill\hfill\hfill\hfill\hfill\hfill\hfill\hfill\hfill\hfill\hfill\hfill}{\ }
\contentsline {subsection}{\textbf{Sokolov I.\,A.} see Pechinkin A.\,V.\hfill\hfill\hfill\hfill\hfill\hfill\hfill\hfill\hfill\hfill\hfill\hfill\hfill\hfill\hfill\hfill\hfill\hfill\hfill\hfill\hfill\hfill\hfill\hfill\hfill\hfill\hfill\hfill\hfill\hfill\hfill\hfill\hfill\hfill\hfill}{\ }
\contentsline {subsection}{\textbf{Sokolov I.\,A.} see Zakharov V.\,N.\hfill\hfill\hfill\hfill\hfill\hfill\hfill\hfill\hfill\hfill\hfill\hfill\hfill\hfill\hfill\hfill\hfill\hfill\hfill\hfill\hfill\hfill\hfill\hfill\hfill\hfill\hfill\hfill\hfill\hfill\hfill\hfill\hfill\hfill\hfill}{\ }
\contentsline {subsection}{\textbf{Stupnikov S.\,A.} see Zakharov V.\,N.\hfill\hfill\hfill\hfill\hfill\hfill\hfill\hfill\hfill\hfill\hfill\hfill\hfill\hfill\hfill\hfill\hfill\hfill\hfill\hfill\hfill\hfill\hfill\hfill\hfill\hfill\hfill\hfill\hfill\hfill\hfill\hfill\hfill\hfill\hfill}{\ }
\contentsline {subsection}{\textbf{Zakharov V.\,N., Kalinichenko L.\,A., Sokolov I.\,A., Stupnikov S.\,A.}\ \ Development of Canonical Information Models for Integrated Information Systems}{\qquad 2 \qquad 15} 
\contentsline {subsection}{\textbf{Zakharov V.\,N., Kozmidiady V.\,A.}\ \ Means Providing Applications Fault Tolerance}{\qquad 1 \qquad 14} 
\def\leftfootline{\small{\textbf{\thepage}
\hfill ИНФОРМАТИКА И ЕЁ ПРИМЕНЕНИЯ\ \ \ том~1\ \ \ выпуск~2\ \ \ 2007}
}%
 \def\rightfootline{\small{ИНФОРМАТИКА И ЕЁ ПРИМЕНЕНИЯ\ \ \ том~1\ \ \ выпуск~2\ \ \ 2007
 \hfill \textbf{\thepage}}}
 \label{end\stat}


%\tableofcontents


\end{document}