\def\stat{morozova}

\def\tit{ТРАНСФОРМАЦИОННЫЕ МОДЕЛИ ЯЗЫКОВЫХ СТРУКТУР 
ДЛЯ~ФРАНЦУЗСКО-РУССКОГО МАШИННОГО ПЕРЕВОДА}

\def\titkol{Трансформационные модели языковых структур 
для~французско-русского машинного перевода}

\def\autkol{Ю.\,И.~Морозова}
\def\aut{Ю.\,И.~Морозова$^1$}

\titel{\tit}{\aut}{\autkol}{\titkol}

%{\renewcommand{\thefootnote}{\fnsymbol{footnote}}\footnotetext[1]
%{Работа поддержана Российским фондом фундаментальных исследований
%(проекты 11-01-00515а и 11-07-00112а), а также Министерством
%образования и науки РФ в рамках ФЦП <<Научные и
%научно-педагогические кадры инновационной России на 2009--2013~годы>>.}}


\renewcommand{\thefootnote}{\arabic{footnote}}
\footnotetext[1]{Институт проблем информатики Российской академии наук, yulia-ipi@yandex.ru}


\Abst{Данная работа посвящена актуальным проблемам исследования 
трансформационных свойств языковых объектов при переводе предикативных 
структур с французского языка на русский. Основное внимание уделено 
изменению категориальной принадлежности и изменению грамматических 
характеристик предикатных слов при переводе. Материалом исследования 
послужили патентные тексты на французском языке и их переводы на русский 
язык, выполненные спе\-ци\-а\-ли\-ста\-ми-пе\-ре\-вод\-чи\-ками. }

\KW{французско-русский автоматический перевод; функциональная семантика; 
языковые трансформации; вершинные грамматики}

 \vskip 14pt plus 9pt minus 6pt

      \thispagestyle{headings}

      \begin{multicols}{2}
      
            \label{st\stat}

\section{Введение}

Данное исследование направлено на исследование предикатных фразовых 
структур на основе вершинных грамматик применительно к задачам 
моделирования машинного перевода и извлечения знаний из текста для 
французско-русского на\-прав\-ле\-ния. Основной задачей ставилось создание 
унифицированной модели функциональных значений синтаксем, в которой 
бы учитывались сдвиги значений, производимые переводческими 
трансформациями. Предикатные слова являются вершинами синтаксической 
структуры предложения, а также вершинами внутреннего представления 
знаний в структурах баз знаний, поэтому описание их дистрибутивных и 
трансформационных свойств имеет первостепенное значение.

Исследования ведутся в рамках проекта по созданию многоязычного 
лингвистического процессора для задач машинного перевода и извлечения 
знаний из текстов, разрабатываемого на основе 
функ\-ци\-о\-наль\-но-се\-ман\-ти\-че\-ско\-го подхода~[1]. В~качестве 
материала исследования были использованы фрагменты параллельных 
текстов патентов, содержащие предикатные выражения. Модель перевода с 
учетом трансформаций для рус\-ско-фран\-цуз\-ской\linebreak языковой пары основана 
на многовариантной когнитивной трансферной грамматике (МКТГ), 
разработанной Е.\,Б.~Козеренко~[1--4]. Данный формализм имеет 
определенные черты грамматики\linebreak составляющих и вершинной грамматики 
HPSG (Head-driven phrase structure grammar)~[5]. 
Преимущество данного формализма заключается в том, что он 
позволяет описывать как отношения линейного порядка, так и отношения 
зависимости в рамках одной и той же фразовой структуры. Формализмы, 
основанные на порождающей грамматике Хомского и на вершинной 
грамматике HPSG, широко применяются для создания сис\-тем 
автоматической обработки текстов на английском языке и других 
европейских языках (французском, испанском, немецком, чешском). Однако 
возможности применения данных формализмов для автоматической 
обработки русского языка изучены недостаточно. 
     
\section{Грамматические формализмы, используемые для~создания 
лингвистических процессоров}
      
     Для формального описания синтаксиса естест\-вен\-ных языков 
применительно к задачам автоматической обработки языка наиболее часто 
использу\-ются следующие виды формализмов: регулярные грамматики, 
     кон\-текст\-но-сво\-бод\-ные грамматики,\linebreak мягко 
     кон\-текст\-но-за\-ви\-си\-мые грамматики. Регулярные грамматики не 
могут быть использованы для полноценного описания синтаксиса. Данный\linebreak 
формализм используется только для частичного синтаксического анализа 
предложений (shallow parsing). С~помощью кон\-текст\-но-сво\-бод\-ных 
грамматик можно описать большинство предложений естественного языка, 
однако грамматики данного класса не позволяют описывать предложения с 
разрывными структурами. Наконец, мягко кон\-текст\-но-за\-ви\-си\-мые 
грамматики являются наиболее мощным формализмом и позволяют 
описывать любые виды предложений естественных языков, однако их 
применение в сис\-те\-мах автоматической обработки естественного языка 
связано с большими\linebreak
 вычислительными затратами. Для описания явлений 
естественных языков применяются, в основном, кон\-текст\-но-сво\-бод\-ные 
грамматики с некоторыми расширениями. Однако вопрос о выборе наиболее 
адекватного формализма для описания синтаксиса естественных языков и 
создания линг\-ви\-сти\-че\-ских процессоров остается открытым. Во многих 
сис\-те\-мах автоматической обработки текс\-тов на английском языке 
используются модернизированные грамматики Н.~Хомского~\cite{6-mor}. 
     
     Для создания сис\-тем автоматической обработки текстов на языках с 
богатой морфологией и относительно свободным порядком слов часто 
используется вершинная грамматика HPSG, разработанная Карлом Поллардом и Иваном Сагом~\cite{5-mor}. 
Согласно данной теории описание грамматики языка должно состоять из 
очень подробного словаря и очень небольшого количества грамматических 
правил, носящих универсальный характер. Словарь имеет хорошо 
проработанную иерархическую структуру, которая характеризуется 
наследованием свойств по умолчанию. В~словарном описании слов, которые 
могут являться вершинами синтаксических групп (существительных, 
глаголов, предлогов, прилагательных) есть поле HEAD, в котором 
описываются такие важные с точки зрения синтаксического поведения слова 
свойства, как часть речи, признаки согласования, форма, предикативность и~др. 
Данные свойства передаются группам, порождаемым данными 
вершинами, в соответствии с правилами грамматики. Таким образом, 
процесс порождения правильно построенных предложений определяется 
свойствами вершин (отсюда название Head-driven phrase structure grammar). 
Одним из основных понятий HPSG является структура свойств (feature 
structure). Это набор атрибутов с их значениями, например словарное 
описание лексемы задается следующей структурой свойств: [PHON$\ldots$ 
SYN$\ldots$ SEM$\ldots$ ARG-ST$\ldots$].
     
     Значение каждого из свойств может пред\-став\-лять собой как единый 
элемент, так и структуру свойств, например свойство AGR (согласование) 
имеет следующую структуру: AGR [PER$\ldots$, NUM$\ldots$, 
GEND$\ldots$].
     
     Унификационный механизм, использующийся в грамматике HPSG, 
позволяет объединить два описания структур свойств. Результатом данной 
операции является структура свойств, содержащая информацию из обеих 
структур. Механизм унификации используется при проверке согласования 
морфологических характеристик слов, необходимой для включения их в одну 
синтаксическую группу. Вершинная грамматика HPSG была с успехом 
применена при создании сис\-тем автоматической обработки текстов на 
разных языках (английском, французском, чешском). Один из примеров такой 
сис\-те\-мы~--- грамматика английского языка \mbox{LinGO} English Resource Grammar 
(ERG) и синтаксически аннотированный корпус Redwoods, размеченный 
автоматически с использованием грамматики ERG~\cite{7-mor}.
     
     В работе~\cite{8-mor} предлагается формализм, име\-ющий некоторые 
черты универсальной грамматики Хомского, вершинной грамматики HPSG и 
лек\-сико-функ\-ци\-о\-наль\-ной грамматики LFG
(Lexical functional grammar). С~использованием 
данного\linebreak формализма была реализована программа синтаксического анализа 
русского языка, которая строит структуру предложения в двух аспектах: как 
структуру составляющих и как функциональную структуру. Для описания 
согласования морфологических характеристик применяется аппарат 
унификации, используемый в вершинной грамматике HPSG. В~данной 
работе обосновывается возможность применения грамматик составляющих с 
различными модификациями для автоматической обработки текстов на 
русском языке (в частности, для перевода с русского языка на другой язык).

\section{Современные подходы к~проблеме машинного перевода}
     
     В области машинного перевода существуют два основных направления 
исследований~--- подход на основе правил и статистический подход. 
Системы, созданные в рамках подхода на основе правил, включают в себя 
компоненты, отвечающие за последовательный морфологический, 
синтаксический и семантический анализ предложений исходного языка и 
синтез предложений целевого языка (с прохождением тех же уровней). 
Создание таких сис\-тем требует многолетней кропотливой работы 
лингвистов, так как для функционирования сис\-те\-мы необходим словарь с 
подробными син\-так\-ти\-ко-се\-ман\-ти\-че\-ски\-ми описаниями словарных 
единиц и правила анализа и синтеза предложений (морфологического, 
синтаксического и семантического уров\-ней). Достоинствами данного 
подхода являются высокое качество перевода, соответствие теоретическим 
концепциям и возможность удобного внесения изменений. Недостатками 
являются большие трудозатраты для создания словарей и сис\-тем правил 
перевода. 
     
     Статистический подход заключается в выявлении закономерностей 
перевода путем автоматического анализа параллельных текстов с 
использованием методов математической статистики и без использования 
лингвистических знаний. Достоинством статистического подхода является 
быстрота создания подобных сис\-тем. Для того чтобы сис\-те\-ма начала 
работать, необходим лишь текстовый корпус, переводческий словарь 
(возможно, неполный) и словарь основ. По данным из~\cite{9-mor}, 
требуется всего несколько часов для того, чтобы сис\-те\-ма начала работать, и 
1--2~недели, чтобы настроить ее и получать приемлемые результаты. 
Недостатком данного подхода является необходимость использования 
больших параллельных корпусов (от~1~млн слов) для получения 
удовлетворительных результатов перевода. Не для всех языковых пар 
существуют такие большие текстовые коллекции. Если же использовать не 
только статистические методы, но добавить и частичную лингвистическую 
разметку, размер корпуса можно существенно уменьшить (с~1~млн до 
300~тыс.\ слов)~\cite{10-mor}.
     
     Современный период развития исследований и разработок в области 
машинного перевода и сис\-тем извлечения знаний из текстов характеризует-\linebreak ся 
интенсивным процессом <<гибридизации>> подходов и моделей. Создатели 
сис\-тем, основанных на правилах, вводят в правила различные стохастические 
модели, которые позволяют отобразить\linebreak
 динамику и разнообразие языковых 
форм и значений, порождаемых в процессе речевой дея\-тель\-ности, а 
сторонники статистических методов построения лингвистических моделей 
все чаще\linebreak обращаются к подходам, основанным на лингвистических знаниях, 
рассматривая их как средства <<интеллектуализации>> сис\-тем. В~настоящее 
время появляется все больше исследований в рамках синергетического 
подхода, использующего лингвистические знания, статистические методы и 
механизмы машинного обучения~\cite{3-mor}. Как пишут авторы~\cite{11-mor}, 
<<наше убеждение состоит в том, что в долгосрочной перспективе 
самые эффективные технологии машинного перевода объединят в себе 
преимущества обоих подходов>>. По мнению авто-\linebreak ров~\cite{11-mor}, подход на основе 
правил следует применять для анализа тех уровней языка, для которых\linebreak 
существуют детальные лингвистические теории, описывающие по\-дав\-ля\-ющее 
большинство случаев, в то время как статистический подход следует 
применять для извлечения лексической и предметно-ори\-ен\-ти\-ро\-ван\-ной 
лингвистической информации, для которой пока что не существует 
разработанной теории. 
     
     Наиболее перспективными направлениями в области статистического 
машинного перевода являются перевод цепочек слов (phrase-based translation) 
и синтаксический перевод (syntax-based translation). При использовании 
метода перевода цепочек слов сопоставлению и переводу подвергаются 
цепочки слов (обычно не длиннее трех слов), выделенные путем применения 
статистических методик. Они не всегда совпадают со словосочетаниями в 
традиционном лингвистическом понимании (группа слов, взаимосвязанных 
синтаксически и семантически). Например, группа слов <<\textit{in 
accordance with the}>> является цепочкой слов, подлежащей переводу, в 
рамках статистического машинного перевода, но не является 
словосочетанием в лингвистическом смысле. При синтаксическом переводе 
сопоставлению и переводу подвергаются синтаксические поддеревья, а не 
конкретные слова или словосочетания.
     
\section{Создание системы правил для~русско-французского 
машинного перевода}

     В работе~\cite{12-mor} описывается сис\-те\-ма перевода с английского 
на французский язык, сочетающая в себе традиционный правиловый 
подход и статистический подход~--- перевод цепочек слов с использованием 
соответствий, извлеченных из параллельного текстового корпуса. В~качестве 
цепочек слов авторы предлагают использовать не любые последовательности 
слов, а только синтаксически мотивированные, другими словами, из текста 
извлекаются именные, глагольные группы, группы прилагательных и 
наречий. При выборе наилучшего варианта перевода цепочки слов 
предпочтение отдается цепочке слов, имеющей ту же самую синтаксическую 
категорию, т.\,е.\ в качестве перевода для именных групп используются 
именные группы и~т.\,д. Французско-русское направление машинного 
перевода в нашей стране развивается с самого начала исследований по 
машинному переводу. Первыми появились экспериментальные сис\-те\-мы 
фран\-цуз\-ско-рус\-ско\-го автоматического перевода ФРАП 
     (1976--1986~гг.)~\cite{13-mor} и \mbox{ЭТАП-1} (1985~г.)~\cite{14-mor}. Эти 
сис\-те\-мы были основаны на последовательном морфологическом, 
синтаксическом и семантическом анализе предложений исходного языка с 
последующим синтезом предложений целевого языка (с прохождением тех 
же уровней). В~сис\-те\-мах использовались словари с подробными 
     син\-так\-ти\-ко-се\-ман\-ти\-че\-ски\-ми описаниями слов и сис\-те\-мы 
правил анализа и синтеза предложений естественного языка.
     
     В 1990-е~гг.\ появилась первая коммерческая сис\-те\-ма автоматического 
перевода фран\-цуз\-ско-рус\-ско\-го направления \mbox{ПРОМТ}. В~основу 
архитектуры сис\-тем было положено представление процесса перевода как 
процесса с объект\-но-ориен\-ти\-ро\-ван\-ной организацией, основанной на 
иерархии обрабатываемых компонентов предложения. В~сис\-те\-мах работают 
сетевые грамматики, близкие по типу к расширенным сетям переходов, а 
также процедурные алгоритмы заполнения и трансформаций фреймовых 
структур для анализа сложных предикатов~\cite{15-mor}.
     
     Качество перевода в современных сис\-те\-мах машинного перевода 
фран\-цуз\-ско-рус\-ско\-го направления достигло высокого уровня, однако 
многие особенности синтаксиса русского языка, а также \mbox{многие} типы 
регулярных трансформаций, происходящих при переводе с русского языка на 
французский, остаются неучтенными в этих сис\-темах. 
     
     В рамках описываемых проектов разрабатывается сис\-те\-ма правил 
трансфера синтаксических структур, учитывающая возможность 
синтаксических трансформаций при переводе и многовариантность перевода. 
Козеренко была разработана многоязычная семантическая грамматика 
русского и английского языков для задач автоматической обработки 
текстов~--- МКТГ~[1--4]. Данная грамматика является разновидностью 
уни\-фи\-ка\-ци\-он\-но-по\-рож\-да\-ющей грамматики. Многовариантные правила 
функ\-ци\-о\-наль\-но-се\-ман\-ти\-че\-ско\-го переноса фразовых структур 
задают алгоритм перевода с одного языка на другой, причем учитывается 
вероятность каждого из вариантов перевода. Функциональные\linebreak значения 
языковых единиц отражены в рас\-ши\-ренной сис\-те\-ме ка\-те\-го\-ри\-аль\-но-функ\-ци\-о\-наль\-ных 
ат\-рибутов. Структуры атрибутов и значений и правила
их преобразования задаются в виде кон\-текст\-но-сво\-бод\-ных и мягко 
     кон\-текст\-но-за\-ви\-си\-мых продукционных правил. Отношения 
зависимости реализуются через механизм головных вершин фразовых 
структур, а сами фразовые структуры задают линейные последовательности 
языковых объектов. Лингвистический процессор сегментирует входные 
предложения на фразовые структуры и осуществляет трансфер этих структур 
в соответствующие им структуры целевого языка. Сегментация фразовых 
структур входного предложения проводится с учетом смысла структур, 
который при переводе должен быть передан средствами целевого языка. 
Задачей проведенных исследований ставилось создание сис\-те\-мы правил 
многовариантного трансфера для перевода с русского языка на французский.
     
     С точки зрения синтаксической структуры предложения русский и 
французский языки очень сильно отличаются друг от друга. Во французском 
языке большинство предложений двусоставны, т.\,е.\ и подлежащие, и 
сказуемые выражены на поверхностном уров\-не, причем сказуемое всегда 
выражается личной формой глагола. В~рус\-ском языке кроме канонической 
структуры <<Подлежащее (выраженное существительным в именительном 
падеже)\;+\;сказуемое (выраженное личным глаголом)>> возможны также 
другие синтаксические струк-\linebreak туры:
       
       $\bullet$~В предложении отсутствует сказуемое, выраженное глаголом в личной 
форме. Сказуемое выражено кратким причастием, кратким или полным прилагательным, 
существительным, предложной группой, инфинитивом и~др.
     
     \smallskip
     
     \noindent
     \textbf{Примеры}:
       
     \textit{Дом красив} (краткое прилагательное). \textit{Дом построен} 
(краткое причастие). \textit{Пьер~--- учащийся} (существительное).
       
       \smallskip
     
     Также к данному классу относятся случаи назывных предложений, 
состоящих из одного подлежащего, выраженного существительным в 
именительном падеже (например, заголовки) и случаи эллипсиса, когда 
глагол в личной форме <<подразумевается>>, но не выражен в 
поверхностной структуре предложения. Приведем пример эллипсиса из 
текста научного патента:
       
     \textit{Рисунки~1--4 ИЗОБРАЖАЮТ продольный разрез половины 
детали различных вариантов, соответствующих выполнению зубного 
штифта согласно первому варианту осуществления изобретения} (полная 
структура).
       
     \textit{Рисунки 5 и 6~--- продольный разрез половины детали 
моноблочного компонента протеза} (структура с эллипсисом).
     
     Подобные структуры являются трудными для синтаксического анализа 
и перевода на французский язык, так как при переводе требуется вставить 
пропущенный глагол в личной форме (глагол \textit{быть} или другой 
глагол). Многие из структур данного вида в существующих сис\-те\-мах 
автоматического перевода с русского языка на французский язык 
обрабатываются некорректно.

$\bullet$~В предложении отсутствует подлежащее, выраженное 
существительным в именительном падеже. К~данному типу предложений 
относятся безличные предложения, не\-опре\-де\-лен\-но-лич\-ные 
предложения, опре\-де\-лен\-но-лич\-ные предложения, инфинитивные 
предложения. 

\smallskip

\noindent
\textbf{Примеры:}
       
     \textit{Мне нравится работать} (безличное предложение).
       
     \textit{Маше подарили книгу} (не\-опре\-де\-лен\-но-лич\-ное 
предложение).
       
     \textit{Еду в кино} (опре\-де\-лен\-но-лич\-ное предложение).
       
     \textit{Нам бы сессию сдать} (инфинитивное предложение).
       
       \smallskip
       
     Также к данному классу относятся предложения, в которых 
подлежащее выражено инфинитивом. 
     
     \smallskip
     
     \noindent
     \textbf{Пример:}
       
     \textit{Курить~--- здоровью вредить.}
       
       \smallskip
       
     Предложения с подобной синтаксической структурой также являются 
источником значительных трудностей при автоматическом анализе и 
переводе, так как французский язык требует, чтобы в каждом предложении 
было подлежащее, выраженное существительным или местоимением без 
предлогов (функционально соответствует существительному в именительном 
падеже в русском языке). Следовательно, при переводе предложения 
русского языка, в котором нет подлежащего, выраженного существительным 
в именительном падеже, требуется восстановить подлежащее, используя 
<<формальное>> подлежащее (безличное местоимение \textit{il}) или личное 
местоимение. 
     
     Чтобы обосновать описание всех типов предложений в создаваемой 
сис\-те\-ме правил многовариантного перевода, было проведено исследование 
частоты встречаемости предложений различных типов в текстах научных 
патентов. Предложения были разделены на 3~класса.
       \begin{description}
     \item[Класс 1.] Подлежащее (выраженное существительным в 
именительном падеже)\;+\;сказуемое (выраженное глаголом в личной форме).
     \item[Класс 2.] Сказуемое, выраженное глаголом в личной форме, 
отсутствует. 
     \item[Класс 3.] Подлежащее, выраженное существительным в 
именительном падеже, отсутствует.
     \end{description}
     
     Некоторые предложения относятся одновременно и к классу~2, и к 
классу~3. Такие предложения были отнесены к классу~2.
     
     В результате распределения предложений по груп\-пам и подсчета 
относительной частоты\linebreak встречаемости предложений каждого вида в текс\-тах 
на\-уч\-ных патентов были получены следующие\linebreak результаты:
     класс~1~--- 48\%; класс~2~--- 38\%;\linebreak класс~3~--- 14\%.
     
     Как видно, предложения, относящиеся к каж\-до\-му из трех классов, 
встречаются в текстах с достаточно большой частотой, и существует 
необходимость включения правил для всех трех классов в сис\-те\-му 
многовариантного трансфера.
       
     Кроме различия структурных типов предложений французский и 
русский языки также очень существенно различаются между собой в аспекте 
порядка слов в предложении. Во французском языке большинство 
предложений имеют канонический порядок слов:
       
\begin{center}
 \textit{Подлежащее\,--\,сказуемое\,--\,прямое дополнение\,--\,косвенные 
дополнения}.
\end{center}
     
     В русском языке данный порядок регулярно нарушается как в устной, 
так и в письменной речи. Несоответствие порядка слов в предложениях на 
русском и французском языке создает необходимость изменения порядка 
слов при переводе. 
     
     При создании сис\-те\-мы правил перевода будем использовать 
классификации предложений русского языка, изложенные в учебниках по 
русскому синтаксису~\cite{16-mor, 17-mor}, а также типичные переводческие 
трансформации, описанные в учебниках по переводу с французского языка 
на русский~\cite{18-mor}.
     
     Рассмотрим наиболее частотные типы предложений русского языка, 
которые создают трудности при переводе, на материале патентных текстов.

$\bullet$~Безличные предложения. Будем понимать под безличным 
предложением такое предложение, которое содержит глагол в личной форме, 
но не содержит существительного в именительном падеже, которое 
выполняло бы роль подлежащего. Безличным предложениям русского языка 
соответствуют предложения с безличным местоимением \textit{il} 
французского языка. Пример перевода:

     \textit{Однако} {\bfseries\textit{оказалось}}, \textit{что эта прочность 
может в конечном счете привести к нарушению надежности 
соединения}.\;$\rightarrow$\;\textit{Toutefois}, {\bfseries\textit{il est apparu}} 
\textit{que cette robustesse pouvait finalement porter atteinte}  
$\grave{\mbox{\textit{a}}}$ \textit{la fiabilit}$\acute{\mbox{\textit{e}}}$ \textit{de 
la liaison}.
{\looseness=1

}
     
     Правило переноса выглядит следующим образом:
     \begin{multline*}
     \mathrm{V[Person~3, Number SG, Gender NEUTR]} \&{}\\
     {}\&  \mathrm{NO  NP[Case NOM]} \rightarrow \mathrm{Il} +{}\\
     {}+\mathrm{ V[Person 3, Number SG, Gender  MASC]}\,.
     \end{multline*}

$\bullet$~Неопределенно-личные предложения.

\medskip

\noindent
\textbf{Пример:}
     
\begin{center}
     \textit{Указанное раструбное соединение осуществляют}~[$\ldots$]. 
     \end{center}
     
     При переводе на французский язык чаще всего используется пассивная 
конструкция:

\begin{center}
          \textit{Cet embo}$\hat{\iota}$\textit{tement est 
effectu}$\acute{\mbox{\textit{e}}}$ [$\ldots$].
\end{center}
     
     Возможен и другой вариант перевода, с использованием безличного 
местоимения \textit{on}:
     
\begin{center}
     \textit{On effectue cet embo}$\hat{\iota}$\textit{tement} [$\ldots$].
     \end{center}
     
     Правило переноса выглядит следующим образом:
     \begin{multline*}
     \mathrm{NP[Case Acc] + V[Person 3, Number PL]}\rightarrow {}\\
     {}\rightarrow  \left \{\mathrm{NP^* + V(be)^*} +{}\right.\\
\left.{}+ \mathrm{V[FORM PART, TENSE PAST]^*}\right\} \\
\mathrm{OR} \left\{\mathrm{On + V[Person 3, Number SG] + NP}\right\}\,.
\end{multline*}
     
     Знак $^*$ означает согласование морфологических признаков.
     
\begin{figure*}[b] %fig1
\vspace*{6pt}
\begin{center}
\mbox{%
\epsfxsize=159.941mm
\epsfbox{mor-1.eps}
}
\end{center}
\vspace*{-9pt}
\Caption{Распределение по частям речи в русских и французских научных текстах 
патентных рефератов: (\textit{а})~реферат патента WO2004009333; 
(\textit{б})~реферат патента WO2004017987;
\textit{1}~--- предложения; \textit{2}~--- строки; \textit{3}~--- 
существительные; \textit{4}~--- глаголы; \textit{5}~--- причастия;
\textit{6}~--- деепричастия (герундии)}
%\end{figure*}
%\begin{figure*} %fig2
\vspace*{12pt}
     \begin{center}
     {\tabcolsep=3pt
     \begin{tabular}{lcl}
     \textbf{Цель, назначение} &&\\
     \textit{Русский язык} &&\textit{Французский язык}\\
     \textbf{Существительное (98)} &&\textbf{Инфинитив (72)}\\
     Инфинитив (2)  &$\leftarrow\,\rightarrow$ &Существительное (26)\\
     Придаточное предложение (0)&&Придаточное предложение (2)
     \end{tabular}
     }
          \end{center}
          \vspace*{-6pt}
\Caption{Правила когнитивного переноса для функциональных значений цели и 
назначения}
\end{figure*}


     
\section{Трансфер пропозиционального ядра в~русско-французской 
языковой паре}
       
     Основу семантико-синтаксической структуры предложения составляет 
пропозициональное ядро, прежде всего языковые средства предикации. Были 
изучены структуры когнитивного переноса в \mbox{рамках} поля функционального 
переноса (ПФП) первичной и вторичной предикации для 
     рус\-ско-фран\-цуз\-ской языковой пары по аналогии с 
     рус\-ско-анг\-лий\-ской языковой парой. Были выделены базовые 
правила когнитивного переноса для различных функциональных значений 
(частотные характеристики были выделены на основании анализа патентных 
текстов). Материалом анализа послужили параллельные тексты патентов 
и/или рефератов патентов на русском и французском языках, взятые из базы 
данных Роспатента.
     
     Сравнение русских и французских текстов рефератов научных 
патентов показало, что доля дей\-ст\-ви\-тель\-но параллельных текстов в них 
составляет примерно 30\%. Остальные тексты можно \mbox{назвать}\linebreak 
     ког\-ни\-тив\-но-со\-по\-ста\-ви\-мы\-ми, причем объем русского текста 
может превышать объем французского на две трети. Однако распределение 
по частям речи в русских и французских научных текстах патентных 
рефератов (и самих патентов) очень близко по составу и объему, что 
отражено на рис.~1. Русский текст в целом на 30\%--35\% более 
номинативен, чем французский, в котором в поле вторичной предикации 
предпочтение отдается инфинитиву (в русском~--- отглагольным 
существительным).
     
     В первом примере тексты рефератов не параллельные, а 
     когни\-тив\-но-со\-по\-ста\-ви\-мые, во втором тексты русского и 
французского реферата параллельны: перевод выполнен точно, почти 
дословно. В~любом случае, как видно из примеров, и в русских, и во 
французских патентных текстах очень высока доля именных групп, что 
вообще всегда характерно для на\-уч\-но-тех\-ни\-че\-ских текстов.
     
     Правила когнитивного переноса для функциональных значений цели и 
назначения представлены на рис.~2.
     

     Таким образом, набор структур, используемых для выражения цели 
действия, одинаков для русского и французского языка, однако французский 
тяготеет к инфинитивной структуре, а русский~--- к именной (\textit{Для 
увеличения способности сети к обобщению}$\ldots$~/ \textit{Afin d'augmenter 
la capacit$\acute{\mbox{\textit{e}}}$ du r$\acute{\mbox{\textit{e}}}$seau de 
g$\acute{\mbox{\textit{e}}}$n$\acute{\mbox{\textit{e}}}$raliser}\ldots).
     
     \smallskip
     
     \noindent
     \textbf{Примеры.}
     \begin{enumerate}
     \item $[$Cat~: VerbNoun$]$ \{для распознавания\} \{pour la reconnaissance\}~--- 
предложная группа: предлог\;+\;существительное.
     \item  $[$Cat~: VerbInf$]$ \{\textit{чтобы распознать}\} \{\textit{afin de 
reconn$\hat{\mbox{\textit{a}}}$itre}\}~--- союз\;+\;инфинитив.
     \item $[$Cat : Sentence$]$ \{\textit{чтобы распознавание было 
эффективным}\} \{\textit{pour que la reconnaissance soit\linebreak
efficace}\}~--- 
придаточное предложение, присоединяемое подчинительной связью (союзом\linebreak 
цели). При трансформации русского отглагольного существительного во 
французский инфинитив необходимо сделать выбор между его активной и 
пассивной формой. Видимо, в рамках сис\-те\-мы автоматического перевода 
данный выбор лучше всего осуществляется с применением статистических 
данных (активный инфинитив встречается в текстах намного чаще 
пассивного; в анализируемых текстах французский пассивный инфинитив в 
качестве перевода русского отглагольного существительного встретился в 
13\% случаев).
     \end{enumerate}
     
\section{Заключение}
     
     Были описаны предикативные синтаксические структуры русского 
языка, принадлежащие к функционально-семантическому полю первичной и 
вторичной предикации и соответствующие\linebreak им синтаксические структуры 
французского языка. Была составлена подробная классификация ти-\linebreak пов 
предложений русского языка с точки зрения синтаксиса и соответствующих 
им синтаксических типов во французском языке, которая может быть 
использована при написании правил переноса синтаксических структур, 
происходящего при переводе с русского языка на французский. 
В~классификации учтено синтаксическое многообразие русского языка: 
назывные предложения, безглагольная предикация (в случае невыраженного 
глагола <<быть>> в настоящем времени), безличные предложения, 
     опре\-де\-лен\-но-лич\-ные предложения, не\-опре\-де\-лен\-но-лич\-ные 
предложения, двусоставные предложения с различным типом ска\-зу\-емых 
(глагольные, именные и~пр.). 
     
     Были изучены категориальные трансформации предикативных 
структур, происходящие при переводе с русского языка на французский и в 
обратном\linebreak
направлении. Моделирование трансформаций\linebreak
предикативных 
структур для задачи машинного перевода является актуальной задачей, так 
как это явление мало исследовано с точки зрения компьютерной реализации 
и недостаточно учтено в дей\-ст\-ву\-ющих сис\-те\-мах машинного перевода. Кроме того, правила, 
задающие функциональную синонимию языковых конструкций, могут 
использоваться также при машинном обучении на корпусе параллельных 
текстов, позволяя избежать формирования избыточных правил и <<шумов>>. 
     
     Дальнейшие исследования будут направлены на уточнение сис\-те\-мы 
синтаксических соответствий с помощью параллельного корпуса текстов 
научных патентов, а также на расширение числа типов трансформаций при 
рус\-ско-фран\-цуз\-ском машинном переводе, дальнейшее изучение 
     дис\-три\-бу\-тив\-но-транс\-фор\-ма\-ци\-он\-ных характеристик 
языковых структур и сбор статистической информации.

{\small\frenchspacing
{%\baselineskip=10.8pt
\addcontentsline{toc}{section}{Литература}
\begin{thebibliography}{99}

      
     \bibitem{1-mor}
     \Au{Козеренко Е.\,Б.}
     Моделирование переноса функциональных значений для 
     анг\-ло-рус\-ско\-го машинного перевода~// Компьютерная лингвистика 
и интеллектуальные технологии: Труды Междунар. конф. Диалог'2004.~--- 
М.: Наука, 2004.
     
     \bibitem{2-mor}
     \Au{Kozerenko E.\,B.}
     Cognitive approach to language structure segmentation for machine 
translation algorithms~// Conference (International ) on Machine Learning, 
Models, Technologies and Applications Proceedings.~--- Las Vegas, USA, 2003. 
     P.~49--55.
     
     \bibitem{4-mor} %3
     \Au{Козеренко Е.\,Б.}
     Функционально-семантические инварианты для алгоритмов 
синтаксического анализа и разметки полнотекстового научного документа~// 
Системы и средства информатики.~--- М.:\ Наука, 2003. Вып.~13. 
     С.~298--312.
     
     \bibitem{3-mor} %4
     \Au{Козеренко Е.\,Б.}
     Лингвистическое моделирование для сис\-тем машинного перевода и 
обработки знаний~// Информатика и её применения, 2007. Т.~1. Вып.~1. 
     С.~54--66.
          
          \bibitem{5-mor}
     \Au{Sag I., Wasow Th., Bender E.\,M.}
     Syntactic theory: A formal introduction.~--- Stanford: CSLI Publications, 
2003.
     
     \bibitem{6-mor}
     \Au{Chomsky N., Lasnik H.}
     The theory of principles and parameters~// The minimalist program.~--- 
Cambridge: MIT Press, 1995.
     
     \bibitem{7-mor}
     \Au{Oepen S., Toutanova K., Shieber~S., Manning~C., Flickinger~D., 
Brants~T.}
     The LinGO Redwoods Treebank: Motivation and preliminary 
applications~// 19th Conference (International) on Computational Linguistics 
Proceedings.~--- Taipei, Taiwan, 2002. P.~1253--1257.
     
     \bibitem{8-mor}
     \Au{Перекрестенко А.}
     Разработка и программная реализация сис\-те\-мы автоматического 
выделения синтаксических групп для естественных языков~// Системы и 
средства информатики.~--- М.: Наука, 2007. Вып.~17. С.~273--291. 
      
      \bibitem{9-mor}
      \Au{Brown R.\,D.}
      Example-based machine translation in the Pangloss system~// 16th 
Conference (International) on Computational Linguistics (COLING-96) 
Proceedings.~--- Copenhagen, Denmark, 1996. P.~169--174.
      
      \bibitem{10-mor}
      \Au{Brown R.\,D.}
      Adding linguistic knowledge to a lexical example-based translation 
system~// 8th Conference (International) on Theoretical and Methodological Issues 
in Machine Translation Proceedings.~--- Chester, UK, 1999. P.~22--32.
     
     \bibitem{11-mor}
     \Au{Grishman R., Kosaka~M.}
     Combining rationalist and empiricist approaches to machine translation~// 
4th Conference (International) on Theoretical and Methodological Issues in 
Machine Translation Proceedings.~--- Montreal, Canada, 1992. P.~263--274.
     
     \bibitem{12-mor}
     \Au{Dugast L., Senellart J., Koehn~P.}
     Selective addition of corpus-extracted phrasal lexical rules to a rule-based 
machine translation system~// 12th Machine Translation Summit Proceedings.~--- 
Ottawa, ON, Canada, 2009. P.~222--229.
     
     \bibitem{13-mor}
     \Au{Леонтьева~Н.\,Н., Никогосов~С.\,Л.}
     Система ФРАП как информационная сис\-те\-ма~// Актуальные вопросы 
практической реализации сис\-тем автоматического перевода.~--- М.: МГУ, 
1982. С.~134--166.
     
     \bibitem{14-mor}
     \Au{Апресян~Ю.\,Д., Богуславский~И.\,М., Иомдин~Л.\,Л.\ и~др.}
     Лингвистическое обеспечение сис\-те\-мы фран\-цуз\-ско-рус\-ско\-го 
автоматического перевода ЭТАП-1. 1.~Общая характеристика сис\-те\-мы~// 
Теория и модели знаний (Теория и практика создания сис\-тем 
искусственного интеллекта): Труды по искусственному интеллекту. Ученые 
записки Тартуского гос. ун-та.~--- Тарту, 1985. 
Вып.~714. С.~20--39.
     
     \bibitem{15-mor}
     \Au{Соколова~С.}
     Как переводит компьютер. {\sf 
http:// www.translationmemory.ru/technology/articles/article\_\linebreak Sokolova.php}.
     
     \bibitem{16-mor}
     \Au{Валгина Н.\,С.}
     Синтаксис современного русского языка.~--- М.: Агар, 2000.  416~с.
     
     \bibitem{17-mor}
     \Au{Шелякин М.\,А.}
     Справочник по русской грамматике.~--- М.: Дрофа, 2006.  355~с.
     
     \label{end\stat}
     
     \bibitem{18-mor}
     \Au{Гак В.\,Г., Григорьев Б.\,Б.}
     Теория и практика перевода: Французский язык.~--- СПб.: 
Интердиалект+, 2000. 456~с.
 \end{thebibliography}
}
}


\end{multicols}       