\def\a{\overline a}
\def\d{\delta}
\def\hb{\hat b}
\def\hc{\hat c}
\def\hd{\hat \delta}
\def\hv{\hat v}
\def\oa{\overline a}
\def\oc{\overline c}

%\def\slae{\hbox{\sl\ae}}
%\def\slae{\ae}

\def\slae{\mbox{{\textsf{\ae}}}}

%\def\slae{\mbox{{\ptb{\ae}}}}

\def\tb{\tilde b}
\def\tc{\tilde c}
\def\td{\tilde \delta}
\def\tp{\tilde p}
\def\tP{\tilde P}

\def\stat{pech}

\def\tit{СИСТЕМА МАССОВОГО ОБСЛУЖИВАНИЯ С~НЕНАДЕЖНЫМ ПРИБОРОМ В~ДИСКРЕТНОМ ВРЕМЕНИ$^*$}

\def\titkol{Система массового обслуживания с ненадежным прибором в дискретном времени}

\def\autkol{А.\,В.~Печинкин, И.\,А.~Соколов}
\def\aut{А.\,В.~Печинкин$^1$, И.\,А.~Соколов$^2$}

\titel{\tit}{\aut}{\autkol}{\titkol}

{\renewcommand{\thefootnote}{\fnsymbol{footnote}}\footnotetext[1]
{Работа выполнена при поддержке РФФИ (грант 11-07-00112).}}


\renewcommand{\thefootnote}{\arabic{footnote}}
\footnotetext[1]{Институт проблем информатики Российской академии наук, apechinkin@ipiran.ru}
\footnotetext[2]{Институт проблем информатики Российской академии наук, isokolov@ipiran.ru}

\vspace*{-12pt}

\Abst{Для функционирующей в дискретном времени однолинейной
сис\-те\-мы массового обслуживания (СМО)  $\mbox{Geo}/\mbox{G}/1/\infty$ с ненадежным прибором, который
может отказывать как в рабочем, так и в свободном
состоянии, причем с разными вероятностями отказов
и распределениями времен ремонта, найдены
основные стационарные характеристики обслуживания.
В~качестве примера в последнем разделе рассмотрен
один из вариантов СМО $\vec{\mbox{Geo}}_2/\vec{\mbox{G}}_2/1/\infty$
с двумя типами заявок и абсолютным приоритетом.}

\vspace*{-3pt}

\KW{система массового обслуживания; дискретное время;
ненадежный прибор}

 \vskip 14pt plus 9pt minus 6pt

      \thispagestyle{headings}

      \begin{multicols}{2}

            \label{st\stat}


\section{Введение}

В системах обслуживания, в том числе в современных
инфотелекоммуникационных сис\-те\-мах, часто возникает ситуация,
когда прибор подвержен различного рода поломкам и неисправностям, что
приводит к ухудшению обслуживания и даже полным остановкам устройств.
Поэтому сис\-те\-мам с ненадежными приборами уделялось и уделяется
большое внимание в различных областях науки и техники.
В~част\-ности, этой тематике посвящено значительное число работ в
теории массового обслуживания.

Следует отметить, что в подавляющем числе работ по изучению
СМО с ненадежными приборами
рассматриваются сис\-те\-мы с непрерывным временем.
Некоторое представление о таких работах можно найти в~\cite{P.S.Ch}.
Однако в последнее время в связи с применением новых технологий,
особенно цифровых методов передачи данных, существенно возрос
интерес к СМО, функционирующим в дискретном времени.
Здесь можно упомянуть работы~[2--11].

В настоящей статье для определенной в сле\-ду\-ющем разделе
функционирующей в дискретном времени однолинейной СМО с
ненадежным прибором найдены основные стационарные
характеристики обслуживания.
В~отличие от других работ, предполагается, что отказы прибора
могут возникать не только при обслуживании заявок, но и
в свободной сис\-теме.
Это предположение в дискретном времени приводит к дополнительным
трудностям из-за возможности одновременного появления сразу
нескольких событий.

В разд.~3 получены соотношения для вычисления в терминах
преобразования Лап\-ла\-са--Стилть\-е\-са (ПЛС) распределения
общего времени пребывания заявки на приборе (с учетом
возможных прерываний на ремонт прибора).

В разд.~4 вводится вложенная цепь Маркова и определяются
стационарные вероятности этой \mbox{цепи}.

В разд.~5 находятся стационарные распределения чис\-ла
заявок по времени и по моментам поступления в сис\-тему.

В разд. 6 вычисляется стационарное распределение
времени пребывания заявки в сис\-теме.

В разд. 7 рассмотрен один из вариантов СМО
$\vec{\mbox{Geo}}_2/\vec {\mbox{G}}_2/1/\infty$ с двумя типами заявок и
абсолютным приоритетом.
Показано, как с по\-мощью по\-лу\-чен\-ных в предыдущих разделах
результатов \mbox{можно} вычислить некоторые стационарные
ха\-рак\-те\-ри\-стики обслуживания неприоритетных заявок в сис\-те\-ме,
в которую поступает два независимых гео\-мет\-ри\-че\-ских потока.
Отметим, что наличие нескольких вариантов системы с
приоритетом обуслов\-ле\-но следующим обстоятельством.
В~отличие от непрерывного времени, в дискретном времени
входящий геометрический поток заявок двух типов можно
определять по-раз\-но\-му.
Действительно, суперпозиция двух независимых геометрических
потоков не является геометрическим потоком (точнее говоря,
является геометрическим неординарным потоком, в котором
одновременно могут прийти либо одна, либо две заявки), и
наоборот: после разреживания геометрического потока на два,
хотя каждый из полученных потоков и будет гео\-мет\-ри\-че\-ским,
но в совокупности потоки будут зависимыми (поскольку
появление заявки одного потока исключает появление заявки
второго потока).
{\looseness=-1

}

Наконец, в разд.~8 приведен пример расчета стационарных
средних характеристик рас\-смат\-ри\-ва\-емой СМО, проведенного с
помощью полученных формул.


%%%%%%%%%%%%%%%%%%%%%%%%%

Отметим, что примененная в данной работе методика была использована
в~[12, 13] для анализа некоторых СМО с отрицательными заявками
и в~\cite{A-M} для исследования СМО $\mbox{MAP}_K/\mbox{G}_K/1$ с обобщенной
дисциплиной преимущественного разделения прибора.

%%%%%%%%%%%%%%%%%%%%%%%%%%


Далее в этой статье для сокращения записи ве\-ро\-ят\-ность
дополнительного события будем снабжать чертой сверху.
Например, $\a = 1-a$.


% 2

\section{Описание системы и~обозначения}

Рассмотрим функционирующую в дискретном времени СМО
$\mbox{Geo}/\mbox{G}/1/\infty$.

Входящий в систему поток является геометрическим:
в конце каждого такта с ве\-ро\-ят\-ностью~$a$, не зависящей от
всей предыстории функционирования сис\-те\-мы до данного такта,
поступает новая заявка, которая тут же переходит на прибор,
если прибор свободен, и становится в очередь в противном
случае.
Среднее число $\hat a$ тактов между поступлениями заявок
определяется формулой:
$$
\hat a = \sum\limits_{k=1}^{\infty}k \oa^{k-1} a
= \fr{1}{a}\,.
$$

В системе имеется один прибор, который может находиться как в
исправном, так и в неисправном состоянии.

Число тактов обслуживания заявки исправным прибором (которое далее
будем называть длиной заявки) распределено по закону $\{b_k$, $k\hm \ge 0
\}$ (предполагается, что длина заявки не может быть нулевой, т.\,е.\
$b_0\hm=0$). Производящую функцию (ПФ) и среднее значение длины заявки будем
обозначать через
$$
\beta (z) =\sum\limits_{k=0}^{\infty} b_k z^k\,;\quad
\hb=\beta'(1)=\sum\limits_{k=0}^{\infty} k b_k\,.
$$

Опишем подробно функционирование прибора,
который может отказывать (в конце такта) как в свободном со\-сто\-янии,
так и при обслуживании заявки.
Пусть $c^*$~--- ве\-ро\-ят\-ность отказа на такте, если прибор
свободен, и $c$~--- если на приборе находится заявка.
Введем ПФ
\begin{align*}
\sigma (z) &= \sum\limits_{k=1}^{\infty} \oc^{k-1} c z^k
= \fr{cz}{1-\oc z}\,;
\\
\sigma^* (z) &= \sum\limits_{k=1}^{\infty}  (\oc^*)^{k-1} c^* z^k
= \fr{c^*z}{1-\oc^* z}
\end{align*}
соответственно для
геометрических распределений времени
до момента отказа занятого и свободного прибора.

Если прибор отказывает в свободном состоянии, то он ремонтируется
случайное время, распределенное по закону~$\{c^*_j$, $j\ge 0\}$
(предполагается, что $c^*_0=0$).
При этом первая заявка, поступившая в свободную сис\-те\-му в момент
ремонта прибора, становится на прибор, но ее обслуживание
начинается только после окончания ремонта.
Остальные заявки скапливаются в очереди.

Если же в момент отказа прибора на нем находится заявка, то
ее обслуживание прекращается, а прибор ремонтируется случайное
время, распределенное по закону $\{c_j$, $j\ge 0\}$
(предполагается, что $c_0\hm=0$).
При этом заявка продолжает находиться на приборе, но ее
обслуживание возобновляется только после окончания ремонта,
причем суммарная обслуженная длина засчитывается при продолжении
обслуживания (дисциплина с дообслуживанием).
Поступающие заявки, как и прежде, становятся в очередь.

Обозначим ПФ и среднее значение распределения $\{c_j$, $j\ge 0\}$ через

\noindent
\begin{align*}
\chi(z)&=\sum\limits_{j=0}^{\infty} z^j c_j\,;\\
\hc=\chi'(1)&=\sum\limits_{k=0}^{\infty} k c_k\,,
\end{align*}
а ПФ и среднее значение распределения $\{c^*_j$, $j\ge 0\}$~--- через

\noindent
\begin{align*}
\chi^*(z)&=\sum\limits_{j=0}^{\infty} z^j c^*_j\,;\\
\hc^*={\chi^*}'(1)&=\sum\limits_{k=0}^{\infty} k c^*_k\,.
\end{align*}


Примем дополнительно следующие правила:
\begin{itemize} %[1)]
\item очередной отказ прибора не может наступить сразу же в момент
окончания ремонта (прибор обязательно должен находиться в
исправном состоянии хотя бы один такт);
\item
если перед некоторым моментом в системе нет заявок, прибор
исправен и в этот момент поступает новая заявка, то она сразу же
переходит на прибор, причем прибор с вероятностью~$c$ отказывает
и ремонтируется случайное время, распределенное по закону
$\{c_j$, $j\ge 0\}$;
\item
если перед некоторым моментом в системе нет заявок, (свободный)
прибор неисправен, а в\linebreak\vspace*{-12pt}
\pagebreak

\noindent
 этот момент заканчивается ремонт прибора
и поступает новая заявка, то она сразу же переходит на прибор и
начинает обслуживаться;
\item
если оканчивается последний такт обслуживания заявки на приборе,
в очереди есть еще заявки или в систему поступает новая заявка,
то заявка с прибора покидает систему, на прибор переходит
следующая заявка, а прибор с вероятностью~$c$ отказывает и
ремонтируется случайное время, распределенное по закону
$\{c_j$, $j\ge 0\}$;
\item
если оканчивается обслуживание заявки на приборе, в очереди
больше нет заявок и в сис\-те\-му не поступает новая заявка, то
прибор отказывает с вероятностью~$c^*$, а ремонтируется случайное
время, распределенное по закону $\{c^*_j$, $j\ge 0\}$.
\end{itemize}

Положим:
\begin{align*}
C_i&=\sum_{j=i}^\infty c_j\,,
\quad i\ge 0\,;
\\
C^*_i&=\sum_{j=i}^\infty c^*_j\,,
\quad  i\ge 0\,.
\end{align*}


Выбор заявок из очереди на обслуживание осуществляется в
порядке поступления в систему.


Наряду с предположением о конечности средних времен
обслуживания и ремонта, которое, как будет видно из дальнейшего,
является необходимым для существования стационарного режима
функционирования данной сис\-те\-мы, при вычислении стационарных
средних значений числа заявок в сис\-те\-ме и времени пребывания
заявки в сис\-те\-ме будем считать, что конечными являются также
вторые моменты времен обслуживания и ремонта.

% 3

\section{Время пребывания заявки на~приборе}

Исследование СМО $\mbox{Geo}/\mbox{G}/1/\infty$ с ненадежным
прибором начнем с вычисления распределения общего времени
пребывания заявки на приборе (с учетом возможных
прерываний на ремонт прибора, в том числе перед
началом обслуживания).
Здесь возможны два случая:
либо заявка поступает в свободную систему,
либо в систему, в которой уже имеются (по крайней мере,
непосредственно перед моментом поступления) другие заявки.

Начнем со второго случая.
Здесь необходимым условием поступления заявки
на прибор является окончание обслуживания предыдущей
заявки, и поэтому в момент поступления заявки на прибор
он обязательно должен находиться в исправном состоянии
(но сразу же после поступления заявки прибор
с вероятностью $c$ может отказать и тогда ремонтируется
случайное время, распределенное по закону
$\{c_j$, $j\hm\ge 0\}$).
Обозначим через $\{d_j$, $j\hm\ge 0\}$ распределение
времени пребывания на приборе такой заявки.

Пусть длина заявки равна $i$, $i\ge 0$, тактов.
За это время с вероятностью $\begin{pmatrix}i\\ k\end{pmatrix} c^k \oc^{i-k}$,\ \
$k=\overline{0,i}$, произойдет
ровно~$k$ прерываний обслуживания (на\-пом\-ним, что число~$k$
учитывает возможное прерывание обслуживания в момент постановки
данной заявки на прибор).
Поскольку времена прерываний~--- независимые одинаково
распределенные случайные величины с распределением
$\{c_j$, $j\ge 0\}$, то общее время прерываний при условии,
что произошло ровно $k$ прерываний, имеет ПФ $\chi^{k}(z)$.
С~учетом собственной длины $i$-й заявки получаем для ПФ
$\delta (z)=\sum\limits_{j=0}^{\infty} z^j d_j$
времени пребывания на приборе заявки, поступившей из очереди,
выражение:
\begin{multline*}
%\label{delta}
\delta (z)
= \sum_{i=0}^{\infty} z^i b_i \sum_{k=0}^{i}
\chi^k(z) \begin{pmatrix}i\\ k\end{pmatrix} c^k \oc^{i-k}
= {}\\
{}=\sum_{i=0}^{\infty} [c z \chi(z) + \oc z]^i b_i
= \beta (c z \chi(z) + \oc z).
\end{multline*}
Среднее время пребывания на приборе такой заявки задается
формулой:
\begin{equation*}
\label{hd}
\hat\delta = \delta'(1) = \beta'(1) [1 + c \chi'(1)] = \hb (1 + c \hc)\,.
\end{equation*}

Перейдем теперь к первому случаю.

Предположим сначала, что в сис\-те\-му вообще не поступают
заявки, и рассмотрим два процесса восстановления.

Интервал между соседними восстановлениями у обоих процессов
восстановления представляет собой число тактов между соседними
моментами окончания ремонта прибора, или, что с точки зрения
распределения то же самое, между соседними моментами отказа
прибора, и состоит из двух независимых частей.
Первая часть~--- время до момента первого отказа после
окончания ремонта, имеющее геометрическое распределение с
па\-ра\-мет\-ром~$c^*$.
Вторая часть~--- время ремонта неисправного прибора, имеющее
распределение $\{c^*_j$, $j\hm \ge 0\}$.
Поэтому ПФ
$\varphi(z)=\sum\limits_{j=0}^{\infty} z^j f_j$
распределения
$\{f_j$, $j\hm \ge 0\}$ интервала между соседними восстановлениями
представима в виде:
\begin{equation*}
\label{phi}
\varphi(z) = \sigma^*(z) \chi^*(z) =\fr{c^* z}{1 - \oc^* z} \chi^*(z)\,.
\end{equation*}

Первый момент восстановления первого процесса восстановления
с вероятностью $\oc^*$ равен нулю
и с вероятностью $c^* c^*_j$ равен $j$, $j\ge 1$.
Первый момент восстановления второго процесса восстановления
имеет (начиная с нуля!) геометрическое распределение с
параметром~$c^*$.

Введенные таким образом процессы вос\-ста\-нов\-ле\-ния имеют очень
простой смысл.
Предположим, что в начальный момент~0 система освобождается
от заявок (уходит с прибора последняя заявка) и полностью
прекращается поступление новых заявок.
Тогда моменты восстановления первого процесса восстановления
представляют собой моменты окончания ремонта прибора (сюда
же относится и момент~0, если в этот момент прибор не
отказывает), а моменты восстановления второго процесса
восстановления совпадают с моментами отказа прибора.

Обозначим через $\{h_{1,k},\ \ k\ge0\}$ и
$\{h_{2,k},\ \ k\ge0\}$ ряды восстановления этих процессов.
Эти ряды будут иметь ПФ
\begin{align*}
%\label{ae1}
{\slae}_{1}(z) &= \sum_{k=0}^{\infty} z^k h_{1,k} =
\fr{\oc^* + c^* \chi^*(z) }{1-\varphi(z)}\,;
\\
%\label{ae2}
{\slae}_{2}(z) &= \sum_{k=0}^{\infty} z^k h_{2,k}
= \fr{c^*}{(1-\oc^* z) [1-\varphi(z)]}\,.
\end{align*}
Очевидно также, что
\begin{equation*}
\label{ae0}
h_{1,0} = {\slae}_{1}(0) = \oc^*\,.
\end{equation*}

Вернемся к исходной СМО и предположим, что в свободную
систему поступает заявка.
В~этом случае она сразу же попадает на прибор, но прибор
может находиться на ремонте, и поэтому еще до фактического
начала обслуживания заявка может провести на приборе
некоторое время.
Распределение времени пребывания на приборе заявки,
поступившей в свободную систему, обозначим через
$\{d^{*}_k$, $k\ge 0\}$.

Для вычисления $\{d^{*}_k$, $k\ge 0\}$ предположим, что
с момента освобождения прибора прошло $i\hm \ge 1$ тактов и в
этот момент в систему поступила заявка.

Теперь возможны следующие варианты:
\begin{itemize} %[1)]
\item на $m$-м, $m=\overline{0,i-1}$, такте закончился
ремонт прибора (с вероятностью $h_{1,m}$) и больше за
${i-1-m}$ тактов прибор не отказывал (с вероятностью
$(\oc^*)^{i-1-m}$).
Тогда заявка сразу же переходит на прибор, но прибор с
ве\-ро\-ят\-ностью~$c$ начинает ремонтироваться, и время
пребывания заявки на приборе имеет распределение
$\{d_k$, $k\ge 0\}$;
\item
окончание ремонта прибора происходит в момент
$i$ (с вероятностью $h_{1,i}$) $i\ge 1$, сов\-па\-да\-ющий
с моментом поступления заявки.
В~этом варианте она также переходит на прибор, но, в
отличие от предыдущего варианта, сразу же начинает
обслуживаться, и распределение $\{d^*_{0,k}$, $k\ge 0\}$
времени ее пребывания на приборе имеет ПФ

\noindent
\begin{multline*}
\label{delta1*}
\delta^*_0(z) = \sum_{k=0}^{\infty} z^k d^*_{0,k} = {}\\
{}=\sum_{k=1}^{\infty} z^k b_k \sum_{j=0}^{k-1}
\chi^j(z) \begin{pmatrix}k-1\\ j\end{pmatrix} c^j \oc^{k-1-j}
={}\\
{}=
z \sum_{k=1}^{\infty} [c z \chi(z) + \oc z)]^{k-1} b_k
= \fr{z \beta (c z \chi(z) + \oc z)}{c z \chi(z) + \oc z}
={}\\
{}=
\fr{\delta(z) }{c \chi(z) + \oc}
\end{multline*}
со средним значением
\begin{multline*}
%\label{delta1*prime}
\hat\delta^*_0 = {\delta^*_0}'(1) = {}\\
{}=\fr{\delta'(1) }{c \chi(1) + \oc} -
\delta(1)\fr{c \chi'(1)}{[c \chi(1) + \oc]^2}
= \hd - c\hc\,;
\end{multline*}
\item
ремонт прибора начался за $m$, $m=\overline{1,i}$,
тактов до поступления заявки в систему (с вероятностью
$h_{2,i-m}$) и не закончился в момент ее поступления
(с вероятностью $C^*_{m+1}$).
Тогда заявка пребывает на приборе время $k$, $k\ge 1$,
если сначала она ожидает (на приборе) $j$, $j=\overline{1,k}$,
тактов окончания ремонта прибора (с вероятностью
$c^*_{m+j}/C^*_{m+1}$) и после окончания ремонта находится
на приборе еще $k-j$ тактов (с вероятностью $d^*_{0,k-j}$).
Значит, в этом варианте вероятность $d^*_{1,k}(m)$ того,
что заявка будет находиться на приборе $k$ тактов,
определяется формулой:
$$
d^*_{1,k}(m) = \sum_{j=1}^{k} \fr{c^*_{m+j}}{C^*_{m+1}} d^*_{0,k-j}\,,
\enskip k\ge 1\,,
\ \ m=\overline{1,i}\,.
$$
\end{itemize}

Таким образом, распределение $\{d^*_{k}(i)$, $k\ge 0\}$,
$i\ge 1$, времени нахождения заявки на приборе при условии,
что заявка поступила на такте~$i$, $i\ge 1$, после
освобождения прибора, определяется формулой:

\noindent
\begin{multline*}
d^*_{k}(i)= \sum_{m=0}^{i-1} h_{1,m} (\oc^*)^{i-1-m} d_k +{}\\
{}+
h_{1,i} d^*_{0,k}+ \sum_{m=1}^{i} h_{2,i-m} \sum_{j=1}^{k} c^*_{m+j} d^*_{0,k-j}\,,\\
k\ge 0\,, \enskip  i\ge 1\,,
\end{multline*}
или в терминах ПФ
$\d^*(z;i)\hm=\sum\limits_{k=0}^{\infty} z^k d^*_{k}(i)$:
\pagebreak

\noindent
\begin{multline*}
\d^*(z;i)= \sum_{m=0}^{i-1} h_{1,m} (\oc^*)^{i-1-m} \d(z)
+ h_{1,i} \d^*_{0}(z) +{}\\
{}+ \sum_{k=0}^{\infty} z^k \sum_{m=1}^{i} h_{2,i-m}
\sum_{j=1}^{k} c^*_{m+j} d^*_{0,k-j}\,,
\ \ i\ge 1\,.
\end{multline*}

Поскольку число тактов от момента освобождения сис\-те\-мы до момента
поступления следующей заявки в свободную систему имеет геометрическое
распределение с параметром $a$, то, применяя формулу полной
вероятности, получаем для ПФ
$\delta^{*}(z)\hm=\sum\limits_{k=0}^{\infty} z^{k} d^{*}_k$
безусловного распределения $\{d^{*}_k$, $k\ge 0\}$ после
элементарных преобразований следующее выражение:
\begin{multline*}
\!\!\delta^{*}(z)= \sum_{i=1}^{\infty} a \a^{i-1} \d^{*}(z;i) =
a \bigg( \d(z)
{\slae}_{1}(\a) \fr{1}{1 - \a \oc^*}
+ {}\\
{}+\;\d^*_{0}(z)
\fr{{\slae}_{1}(\a) - h_{1,0} }{\a}
+ \d^*_0(z) {\slae}_{2}(\a)
\fr{\a \chi^*(z) - z \chi^*(\a)}{\a (z-\a)}
\bigg).\!\!\!\!
\end{multline*}

Среднее время $\hd^*$ пребывания на приборе заявки,
поступившей в свободную систему, вычисляется по формуле:
\begin{multline*}
\hd^{*}= {\delta^*}'(1) = a \bigg(
\hd {\slae}_{1}(\a) \fr{1}{1 - \a \oc^*}
+ {\hd^*_{0}}
\fr{{\slae}_{1}(\a) - h_{1,0} }{\a}
+ {}\\
{}+
{\hd^*_0} {\slae}_{2}(\a) \fr{\a - \chi^*(\a) }{a \a}+ {}\\
{}+
{\slae}_{2}(\a) \bigg[ \fr{\a \hc^{*} - \chi^*(\a) }{a \a}-
\fr{\a - \chi^*(\a) }{a^2 \a}\bigg]
\bigg)\,,
\end{multline*}
или после несложных, но утомительных преобразований:
$$
\hd^{*} = \hb (1 + c \hc) - \fr{c^* [ (1 + a c\hc) (1 - \chi^*(\a)) - a \hc^{*}
]}{a[1 - \oc^* \a - c^* \a \chi^*(\a)]}\,.
$$

% 4
\section{Стационарные вероятности состояний по вложенной цепи~Маркова}

Пусть $\tau_{n}$, $n\ge 1$,~--- последовательные моменты
уходов заявок из системы.
Обозначим через $\nu_{n}$ число заявок в системе сразу же после
момента $\tau_{n}$
(отметим, что $\nu_{n}$ определяется после ухода обслуженной
заявки, а если в момент $\tau_{n}$ в систему поступает новая
заявка~--- то после ее прихода).
Очевидно, что $\{\nu_{n},\ \ n\ge 1\}$~--- (однородная)
цепь Маркова.

Положим
\begin{align*}
\tp_{j} &= \sum_{k=j}^\infty d_k \begin{pmatrix}k \\ j\end{pmatrix} a^{j} \a^{k-j}=
\fr{a^{j} }{j!} \d^{(j)}(\a)\,,
\  j\ge 0\,;
\\
\tp^*_{j} &= \sum_{k=j}^\infty d^*_k \begin{pmatrix}k \\ j\end{pmatrix}
a^{j} \a^{k-j} =
\fr{a^{j} }{j!} \d^{*(j)}(\a)\,,
 \ j\ge 0\,.
\end{align*}
Как нетрудно видеть, $\tp_{j}$ и  $\tp^*_{j}$ представляют
собой вероятности того, что за время пребывания на приборе
заявки, поступившей в систему при наличии других заявок,
или заявки, поступившей в свободную систему,
в систему придет еще $j$ заявок.

Выпишем переходные вероятности цепи $\nu _{n}$.

Если $i\ge 1$, то переходные вероятности $p_{i,j}$ определяются
точно так же, как и для системы $\mbox{Geo}/\mbox{G}/1/\infty$ с надежным
прибором, за исключением того, что
вместо распределения $\{b_k$, $k\hm\ge 0\}$ нужно взять
распределение $\{d_k$, $k\hm\ge 0\}$.
Поэтому
\begin{multline*}
p_{i,j} = \sum_{k=j-i+1}^\infty d_k \begin{pmatrix}k \\ j-i+1\end{pmatrix} a^{j-i+1}
\a^{k-j+i-1}
={}\\
{}= \tp_{j-i+1}\,, \ \ i\ge 1\,,
 \ j\ge i-1.
\end{multline*}
Однако, если $i=0$, то, в отличие от СМО $\mbox{Geo}/\mbox{G}/1/\infty$,
поступающая в свободную систему заявка пребывает на приборе
время, распределенное по закону $\{d^{*}_k$, $k\hm\ge 0\}$.
Значит,
$$
p_{0,j} = \sum\limits_{k=j}^\infty d^*_k \begin{pmatrix}k \\ j\end{pmatrix} a^{j} \a^{k-j}
= \tp^*_{j}\,,  \ j\ge 0\,.
$$
Окончательно получаем, что матрица переходных
вероятностей вложенной цепи Маркова
$\{\nu _{n}$, $n\ge 0\}$ имеет вид:
$$
P= (p_{ij}) =
\begin{pmatrix}
\tp^*_0 &   \tp^*_1 &  \tp^*_2 &  \cdots \\
\tp_0   &   \tp_1   &  \tp_2   &  \cdots \\
0       &   \tp_0   &  \tp_1   &  \cdots \\
0       &   0       &  \tp_0   &  \cdots \\
\vdots  &   \vdots  &  \vdots  &  \ddots
\end{pmatrix}\,.
$$

Будем предполагать, что средние времена $\hc$ и $\hc^{*}$
ремонта прибора и средняя длина $\hb$ заявки конечны и,
кроме того, выполнено условие $\rho = a \hd<1$.
Тогда существуют предельные (стационарные) вероятности
состояний
$p^{+}_{i}\hm={\lim\limits_{n\to\infty} {\bf P}\{\nu_{n}\hm=i}\}$,
$i\hm\ge0$, по вложенной цепи Маркова $\nu _{n}$.
Необходимость вышеперечисленных условий для эргодичности
вложенной цепи будет показана ниже, достаточность их нетрудно
получить, воспользовавшись критерием Мустафы.

Для определения $p^{+}_{i}$, $i\hm\ge0$, выпишем систему
уравнений равновесия (СУР):
\begin{equation}
\label{SUR}
p^{+}_{i} = p^{+}_{0} \tp^{*}_{i}+ \sum\limits_{k=1}^{i+1} p^{+}_{k} \tp_{i-k+1}\,,
\ \ i\ge0\,.
\end{equation}

Система~(\ref{SUR}) легко решается рекуррентно:
\begin{equation*}
p^{+}_{i}= p^{+}_{0}r_{i}\,, \enskip i\ge1\,,
\end{equation*}
где $r_{i}$ определяются рекуррентной формулой:
\begin{align*}
r_{1} &= \fr{1}{\tp_{0}}\,(1-\tp^{*}_{0})\,;
\\
r_{i} &= \fr{1}{\tp_0} \left[ r_{i-1} - \tp^{*}_{i-1} -
\sum_{k=1}^{i-1} r_{k} \tp_{i-k} \right]\,,
\ \ i\ge 2\,.
\end{align*}

Более рациональный алгоритм вычисления $p^{+}_{i}$ можно
получить, если при составлении СУР воспользоваться
уравнениями частичного баланса.
Действительно, рассматривая вероятности перехода за один
шаг цепи Маркова из множества состояний с номерами, не
меньшими~$i$, $i\hm\ge 1$, в множество состояний с номерами,
меньшими~$i$, и обратно, получаем систему уравнений:
$$
p^{+}_{i} \tp_{0} = p^{+}_{0} \tP^{*}_{i} + \sum\limits_{j=1}^{i-1} p^{+}_{j} \tP_{i-j}\,,
\ \ i\ge 1\,,
$$
где
$$
\tP^{*}_{i}= \sum_{j=i}^{\infty} \tp^{*}_{j}\,, \enskip
\tP_{i} = \sum_{j=i}^{\infty} \tp_{j}\,,
\ \ i\ge 1\,,
$$
которая также решается рекуррентно, но, в отличие от
описанного ранее алгоритма, при решении не содержит вычитаний,
что позволяет во многих случаях существенно повысить точность
вычислений.

Воспользуемся теперь ПФ
$P^{+}(z)=\sum\limits_{i=0}^{\infty} p^{+}_{i}z^{i}$.
Умножая $i$-е уравнение сис\-те\-мы~(\ref{SUR}) на $z^{i}$ и
суммируя, получаем:
\begin{equation*}
P^{+}(z)= p^{+}_{0}\delta ^{*}(\a + a z) + \fr{1}{z}\, [P^{+}(z) - p^{+}_{0}]\,
\delta (\a + a z)\,,
\end{equation*}
откуда находим:
\begin{equation}
\label{P+}
P^{+}(z) = \fr{\delta(\a + a z) - z\,\delta^*(\a + a z) }{
\delta(\a + a z) - z}\,
p^{+}_{0}\,.
\end{equation}

Для определения $p^{+}_{0}$, как обычно, воспользуемся условием
нормировки:
$$
\sum\limits_{i=0}^{\infty} p^{+}_{i}=P^{+}(1)=1\,.
$$
Применяя правило Лопиталя, имеем:
\begin{equation*}
\fr{a\hd - 1 - a\hd^{*} }{ a\hd - 1}\, p^{+}_{0} = 1\,,
\end{equation*}
что приводит к следующему выражению для $p^{+}_{0}$:
\begin{equation*}
\label{p0+}
p^{+}_{0} = \fr{1-a\hd }{1-a\hd+a \hd^*} = \fr{1-\rho }{1-\rho+a\hd^*}\,.
\end{equation*}
Из полученной формулы, в частности, следует, что условия
$\rho \hm<1$ и $\hc^*\hm<\infty$ являются необходимыми
для эргодичности вложенной цепи Маркова.

Дифференцируя формулу~(\ref{P+}) в точке $z\hm=1$
соответствующее число раз, можно получить моменты любого
порядка стационарного распределения чис\-ла заявок в системе
по вложенной цепи Маркова.
Так, среднее число заявок задается формулой:
\begin{equation*}
N^+ = {P^{+}}'(1) = \fr{a^2\td }{2(1 - \rho)}
- \fr{ a^2\td  - 2a\hd^*-a^2\td^*}{2(1 - \rho + a \hd^*)}\,,
\end{equation*}
где
$$
\td = \tb (1 + c \hc)^2 + \hb c(2\hc+\tc)\,;
$$
%%%%%%%%%%%%%%%%%%%%%%%%

\vspace*{-12pt}

\noindent
\begin{multline*}
\td^* = {\delta^*}''(1) = a \bigg(
\td {\slae}_{1}(\a) \fr{1}{1 - \a \oc^*}
+ {}\\
{}+\td_0^* \fr{{\slae}_{1}(\a) - \oc^* }{\a}
+ \td_0^* {\slae}_{2}(\a) \fr{\a - \chi^*(\a) }{a \a}
+ {}\\
{}+
2{\hd^*_0} {\slae}_{2}(\a) \bigg[
\fr{a \hc^* - 1 + \chi^*(\a) }{a^2} \bigg]
+ {}\\
{}+{\slae}_{2}(\a) \bigg[
\fr{\tc^* }{a}
- 2\fr{a \hc^* - 1 + \chi^*(\a) }{a^3}
\bigg] \bigg)
\,;
\end{multline*}

\vspace*{-6pt}

\noindent
\begin{align*}
\tb &= \beta''(1) = \sum\limits_{k=1}^{\infty} k (k-1) b_k\,;
\\
\tc &= \chi''(1)  = \sum_{k=1}^{\infty} k (k-1) c_k\,;
\\
\td^*_0 &= {\d_0^*}''(1) = \td - 2 c\hc \hd - c\tc + 2 c \hc^2 \,;
\\
\tc^* &= {\chi^*}''(1) = \sum_{k=1}^{\infty} k (k-1) c^*_k\,.
\end{align*}

Далее при вычислении стационарных вероятностей числа заявок
по моментам поступления в систему удобно пользоваться другой
цепью Маркова $\{\nu^-_{n}$, $n\ge 1\}$, отличающейся от
введенной выше только тем, что если в момент $\tau_{n}$ в
систему поступает новая заявка, то число $\nu^-_{n}$
определяется до ее прихода.

Обозначая через $p^{-}_{i}$,  $i\ge 0$, стационарные
вероятности цепи $\nu^-_{n}$, получаем следующие соотношения,
связывающие $p^{-}_{i}$ и $p^{+}_{i}$:
\begin{align}
\label{SUR0-}
p^{+}_{0}&= \a p^{-}_{0} \,;
\\
\label{SUR-}
p^{+}_{i} &= \a p^{-}_{i} + a p^{-}_{i-1}\,,
\ \ i\ge1\,.
\end{align}

Вводя ПФ $P^{-}(z)=\sum\limits_{i=0}^\infty z^i p^{-}_{i}$,
получаем из равенств~(\ref{SUR0-}) и~(\ref{SUR-})
\begin{equation*}
\label{SUR*}
P^{-}(z)= \fr{1}{az+\a} P^+(z)\,.
\end{equation*}

Приведем также формулу для среднего числа заявок $N^-$
по цепи Маркова $\nu^-_{n}$:
$$
N^-= {P^{-}}'(1) = N^+ - a \,.
$$

%5
\section{Стационарные вероятности состояний по~времени
и~по~моментам поступления заявок}

Пусть в некоторый момент в свободную систему (на свободный
прибор, который, возможно, в этот момент уже ремонтируется
или только начинает ремонтироваться) поступает заявка,
обслуживание которой (с учетом возможных прерываний
обслуживания на ремонт прибора) закончится через $k$ тактов.

Обозначим через $t_i(k)$, $k\ge 1$, $i=\overline{0,k-1}$,
среднее число таких тактов до момента $k$ окончания обслуживания
выделенной заявки, когда в системе в очереди будет находиться $i$
заявок. Через $t_i$, $i\ge 0$, обозначим аналогичное среднее, но
для заявки произвольной длины, поступающей в тот момент, когда
заканчивается обслуживание заявки на приборе (и в очереди
отсутствуют другие заявки), а через $t^*_i$, $i\ge 0$,~--- для
заявки произвольной длины, поступающей в свободную систему (т.\,е.\
систему, в которой, по крайней мере, уже на предыдущем такте
отсутствовали заявки). Наконец, через $s^*_0$ обозначим среднюю
длину свободного периода, т.\,е.\ среднее время от момента
освобождения сис\-те\-мы до момента поступления следующей заявки. Тогда

\noindent
\begin{align*}
t_i(k) &= \sum_{j=i}^{k-1} \begin{pmatrix}j \\ i\end{pmatrix} a^i \a^{j-i}\,,
 \ k\ge 1\,,
 \ i=\overline{0,k-1}\,;
\\
t_i &= \sum_{k=i+1}^\infty t_i(k) d_k = \sum_{k=i+1}^\infty \sum_{j=i}^{k-1}
\begin{pmatrix}j \\ i\end{pmatrix} a^i \a^{j-i} d_k\,,\\
&\hspace*{50mm} i\ge 0\,;
\\
t^*_i &= \sum_{k=i+1}^\infty t_i(k) d^*_k = \sum\limits_{k=i+1}^\infty
\sum_{j=i}^{k-1} \begin{pmatrix}j \\ i\end{pmatrix} a^i \a^{j-i} d^*_k\,,\\
&\hspace*{50mm}  i\ge 0\,;
\\
s^*_0 &= \sum_{k=1}^\infty k a \a^{k-1} = \fr{1}{a}\,.
\end{align*}

Заметим, что ПФ $T(z)\hm=\sum\limits_{i=0}^{\infty} t_{i}z^{i}$ и
$T^*(z)\hm=\sum\limits_{i=0}^{\infty} t_{i}^*z^{i}$
последовательностей $Х\{t_i$, $i\ge 0\}$ и
$\{t^*_i$, $i\ge 0\}$ имеют вид:
\begin{multline*}
T(z) =
\sum_{i=0}^\infty z^i \sum_{k=i+1}^\infty \sum_{j=i}^{k-1} \begin{pmatrix}j \\ i\end{pmatrix} a^i \a^{j-i} d_k
={}\\
{}= \fr{1-\d(az + \a)}{a(1 - z)}\,;
\end{multline*}

\noindent
\begin{multline*}
T^*(z) = \sum_{i=0}^\infty z^i \sum_{k=i+1}^\infty \sum_{j=i}^{k-1}
\begin{pmatrix}j \\ i\end{pmatrix} a^i \a^{j-i} d^*_k
={}\\
{}=\fr{1 - \d^*(az + \a)}{a(1 - z)}\,.
\end{multline*}

\vspace*{-6pt}

Введем обозначения:
\begin{description}
\item[\,]  $t$ --- среднее число тактов между соседними моментами
изменения состояний вложенной цепи Маркова в стационарном
режиме функционирования системы;
\item[\,]
$s_i$,\ $i \ge 0$, --- среднее число таких тактов между
соседними моментами изменения состояний вложенной цепи
Маркова в стационарном режиме функционирования системы,
когда в системе находится $i$ заявок.
\end{description}

Тогда

\noindent
\begin{align*}
t&= p_0^+ (s^*_0 + \hd^*) + (1-p_0^+) \hd = \fr{1}{a}\,;\\
s_0 &= p^+_0 s^*_{0}\,;\\
s_i &= p^+_0 t^*_{i-1} + \sum_{j=1}^{i} p_j^+ t_{i-j}\,,
 \ i \ge 1\,.
\end{align*}

Стационарное распределение $\{p_{i}$, $i\ge 0\}$ числа заявок в
системе по времени (т.\,е.\ по интервалам времени между соседними
моментами окончания тактов) находится по формуле:

\noindent
\begin{equation*}
p_i= \fr{s_i}{t} = a s_i\,,
\enskip  i \ge 0\,.
\end{equation*}
Переходя к ПФ $P(z)\hm=\sum\limits_{i=0}^{\infty} p_{i}z^{i}$
и производя элементарные преобразования, получаем:

\noindent
\begin{multline}
\label{on_time}
P(z)= \fr{1}{t} \Bigg(
p^{+}_{0} \Bigg[ s^*_0 + \sum\limits_{i=1}^{\infty} z^i t^*_{i-1} \Bigg]
+ {}\\
{}+\sum\limits_{i=1}^{\infty} z^i \sum_{j=1}^{i} p_j^+ t_{i-j}
\Bigg)
= P^+(z)\,.
\end{multline}
Таким образом, стационарное распределение числа заявок
в системе по времени совпадает со стационарным
распределением числа заявок по цепи Маркова $\nu_{n}$,
порожденной моментами уходов заявок из системы.
Иными словами, для рассматриваемой системы выполнен
закон стационарной очереди Хинчина.
В~частности, совпадают моменты любых порядков
стационарных распределений числа заявок в системе по
времени и по цепи Маркова~$\nu_{n}$.
Например, среднее число заявок в сис\-те\-ме~$N$ по времени
имеет вид:
\begin{equation*}
N = P'(1) = N^+ \,.
\end{equation*}

%\pagebreak

Стационарные вероятности $p_k^*$, $k\ge 0$, того,
что в момент поступления заявки в систему в ней
будет $k$ заявок (напомним, что в соответствии с
принятыми правилами в число $k$ не входит заявка,
обслуживание которой заканчивается в этот момент)
можно найти, например, следующим образом.

Введем обозначения:
\begin{gather*}
D_i= \sum_{j=i+1}^\infty d_j\,, \quad i\ge 0\,;\\
D^*_i = \sum_{j=i+1}^\infty d^*_j\,, \quad i\ge 0\,.
\end{gather*}

В момент~0 (прихода выделенной заявки) в сис\-те\-ме будут
отсутствовать заявки, если в момент $-i$, $i\ge 0$,
закончилось обслуживание единственной заявки (с
вероятностью $p^-_0/t$) и за время $i$ новые заявки
не поступят (с вероятностью $\a^i$).
Поэтому
$$
p_0^* = \fr{1}{t} p_{0}^- \sum_{i=0}^{\infty} \a^{i}
= \fr{1}{at} p_{0}^- = p_{0}^-\,.
$$

В системе в момент 0 (прихода выделенной заявки) будет~$k$, $k\ge
1$, других заявок в одном из трех случаев:
\begin{enumerate}[(1)]
\item в момент $-i$, $i\ge k$, закончилось обслуживание
единственной заявки (с вероятностью $p^-_0/t$), и в этот
же момент в систему (на прибор) поступила новая заявка
(с ве\-ро\-ят\-ностью~$a$), которая будет обслуживаться более
$i$~тактов (с вероятностью $D_i$) и за $i-1$ тактов,
оставшихся до момента~0, поступит еще $k-1$ заявок
(с вероятностью $\begin{pmatrix}
i-1 \\ k-1\end{pmatrix} a^{k-1} \a^{i-k}$);
\item
в некоторый предыдущий момент произошло последнее
до момента~0 окончание обслуживания единственной заявки
(с вероятностью $p^-_0/t$), но, в отличие от предыдущего
случая, в этот момент новая заявка в систему не
поступила (с вероятностью~$\a$).
Затем в момент~$-i$, $i\ge k$, в свободную систему (на прибор)
поступила новая заявка, которая будет обслуживаться более
$i$~тактов (с вероятностью $D^*_i$) и за $i-1$ тактов,
оставшихся до момента~0, поступит еще $k-1$ заявок
(с вероятностью $\begin{pmatrix}i-1 \\ k-1\end{pmatrix} a^{k-1} \a^{i-k}$);
\item
в момент $-i$, $i\ge k-j$, закончилось обслуживание
заявки на приборе и в системе осталось
$j$, $j=\overline{1,k}$, других заявок (с вероятностью
$p^-_j/t$).
Новая заявка, перешедшая из очереди на прибор, будет
обслуживаться более
$i$ тактов (с вероятностью $D_i$), и за $i$ тактов,
оставшихся до момента~0, поступит еще $k-j$ заявок
(с вероятностью $\begin{pmatrix}i \\ k-j\end{pmatrix} a^{k-j} \a^{i-k+j}$).
Таким образом,

\noindent
\begin{multline*}
p_k^* = \fr{1}{t} \Bigg(
p_{0}^- a \sum_{i=k}^{\infty} D_i \begin{pmatrix}i-1 \\ k-1\end{pmatrix} a^{k-1} \a^{i-k}
+{}\\
{}+ p_{0}^- \a \sum_{i=k}^{\infty} D^*_i \begin{pmatrix}i-1 \\ k-1\end{pmatrix} a^{k-1} \a^{i-k}
+{}\\
\hspace*{-5pt}{}+
\sum_{j=1}^{k} p_{j}^- \sum_{i=k-j}^{\infty} D_i \begin{pmatrix}i \\ k-j\end{pmatrix} a^{k-j} \a^{i-k+j}
\Bigg),
\ \ k\ge 1\,.\!\!\!
\end{multline*}
\end{enumerate}

В терминах ПФ $P^*(z)=\sum\limits_{k=0}^{\infty} z^k p_{k}^*$
стационарное распределение числа заявок в системе по
моментам поступления заявок определяется выражением:

\noindent
\begin{multline*}
P^*(z) = \fr{1}{1-z} \Big[
\fr{\d(az+\a) - z \d^*(az+\a) }{az+\a}\, p_{0}^+ +{}\\
{}+ [1 - \d(az+\a)] P^-(z)
\Big]
\,,
\end{multline*}
или после элементарных преобразований
$$
P^*(z) = P^-(z) \,,
$$
т.\,е.\ совпадает со стационарным распределением числа заявок по цепи
Маркова $\nu^-_{n}$.

Соответственно, среднее число заявок $N^*$ по моментам
поступления совпадает со средним чис\-лом заявок~$N^-$
по цепи Маркова~$\nu^-_{n}$.

%6
\section{Стационарное распределение времени пребывания заявки в~системе}

Рассмотрим сначала моменты после окончания очередного такта
функционирующей в стационарном режиме системы и введем
следующие стационарные вероятности:
\begin{description}
\item[\,]
$w_0$ --- вероятность того, что прибор свободен и
исправен;
\item[\,]
$w_{0,i}$, $i\ge1$, --- вероятность того, что прибор
свободен, но неисправен, причем до окончания ремонта
осталось $i$ тактов;
\item[\,]
$w_{i}$, $i\ge1$, --- вероятность того, что в системе
имеются заявки и суммарная работа по обслуживанию всех
этих заявок (с учетом возможных прерываний) равна~$i$.
\end{description}

Тогда
\begin{align}
\label{w_0}
w_0 &= \a (\oc^* w_0 + w_{0,1} + \oc^* w_1)\,;
\\[6pt]
\label{w_0i}
w_{0,i} &= \a w_{0,i+1} + \a c^* c^*_i (w_0 + w_1)\,,
\ \ i\ge 1\,;
\\
\label{w_i}
w_{i} &= \a w_{i+1} + a d_i w_0 + \sum_{j=1}^i a w_{0,j} d^{*}_{0,i-j+1} +{}\notag\\
&\hspace*{20mm}{}+ \sum_{j=1}^i
a w_{j} d_{i-j+1}\,,
\ \ i\ge 1\,.
\end{align}

Равенства (\ref{w_0i}) и~(\ref{w_i}) в терминах ПФ
$w_{0}(z)\hm=\sum\limits_{i=1}^\infty w_{0,i} z^i$
и $w(z)\hm=\sum\limits_{i=1}^\infty w_{i} z^i$
могут быть записаны в виде:
\begin{multline}
\label{fw_z}
w_{0}(z)=\a \fr{1 }{z} \left(w_0(z) - z w_{0,1}\right)+{}\\
{}+\a c^* \chi^*(z) (w_0 + w_1)\,;
\end{multline}

\vspace*{-12pt}

\noindent
\begin{multline}
\label{fw_0z}
w(z) = \a \fr{1}{z} \left(w(z) - zw_{1}\right)
+ a \d(z) w_0 +{}\\
{}+ a \fr{1}{z} \left[w_{0}(z) \d^{*}_0(z) + w(z) \d(z)\right]\,.
\end{multline}

Из формулы~(\ref{fw_z}) имеем:
\begin{equation}
\label{w_0zz}
w_{0}(z) = \fr{\a z [c^* \chi^*(z) (w_0 + w_1) - w_{0,1}] }{z - \a}\,.
\end{equation}
Поскольку в точке $z=\a$ знаменатель правой части равенства~(\ref{w_0zz})
обращается в нуль, то в силу непрерывности
функции $w_{0}(z)$ на отрезке $[0,1]$ в этой точке в нуль
должен обращаться и числитель, что дает
\begin{equation}
\label{w_0zzz}
w_{0,1}= c^* \chi^*(\a) (w_0 + w_1)\,.
\end{equation}

Формулы (\ref{w_0}), (\ref{w_0zz}) и~(\ref{w_0zzz})
позволяют выразить $w_{1}$ и $w_0(z)$ через $w_0$:
\begin{align}
w_1 &= \Big[ \fr{1}{\a [\oc^* + c^* \chi^*(\a)]} - 1
\Big] w_0\,;\notag\\[6pt]
\label{qw_0zz}
w_{0}(z) &= \fr{\a c^* z [\chi^*(z) - \chi^*(\a)](w_0 + w_1) }{z - \a}
={}\notag\\
&\hspace*{12mm}{}= \fr{c^* z [\chi^*(z) - \chi^*(\a)] }{[\oc^* + c^* \chi^*(\a)] (z - \a)} w_0\,.
\end{align}

Подставляя найденные значения $w_{1}$ и $w_0(z)$ в~(\ref{fw_0z}), имеем:
\begin{multline}
\label{w(z)}
w(z)= \fr{z}{ z - \a - a \d(z)}
\bigg(
a \d(z) - \fr{1}{\oc^* + c^* \chi^*(\a)} + {}\\[6pt]
{}+\a + a c^* \d^{*}_0(z) \fr{\chi^*(z) - \chi^*(\a)
}{[\oc^* + c^* \chi^*(\a)] (z - \a) }
\bigg) w_0\,.
\end{multline}

Для определения $w_0$ воспользуемся условием нормировки,
которое в данном случае записывается так:
$$
w_0 + w_0(1) + w(1) = 1\,.
$$
Подставляя в это равенство вместо $w_{0}(1)$ и $w(1)$
их значения, вычисленные по формулам~(\ref{qw_0zz}) и~(\ref{w(z)}),
применяя правило Лопиталя и производя элементарные преобразования, получаем:
$$
w_0 = \fr{1 - \rho }{1 - \rho + a \hd^*}
\fr{ a [\oc^* + c^* \chi^*(\a)] }{1 - \a \oc^* - \a c^* \chi^*(\a)}\,.
$$

Естественно,

\noindent
$$
w_0 + w_{0}(1) = \fr{1 - \rho }{1 - \rho + a \hd^*} = p_0 = p_0^+ \,.
$$

Введем теперь следующие стационарные вероятности, связанные
с временем ожидания начала обслуживания заявки:
\begin{description}
\item[\,] $w_0^-$ --- вероятность того, что по\-сту\-па\-ющая заявка сразу
же попадает на прибор (который, возможно, ремонтируется);
\item[\,]
$w_{i}^-$, $i\ge1$, --- вероятность того, что по\-сту\-па\-ющая
заявка будет ожидать попадания на прибор (который, возможно,
ремонтируется) $i$ тактов;
\item[\,]
$w_0^+$ --- вероятность того, что по\-сту\-па\-ющая заявка сразу
же начнет обслуживаться (прибор свободен и исправен);
\item[\,]
$w_{i}^+$, $i\ge1$, --- вероятность того, что по\-сту\-па\-ющая
заявка начнет обслуживаться (на исправном приборе) через $i$
тактов.
\end{description}

Тогда

\noindent
\begin{align*}
w_0^- &= w_0 + \sum_{j=1}^\infty w_{0,j} + w_1 = w_0 + w_{0}(1) + w_1 = \fr{1 }{\a} p_0\,;
\\
w_i^-&= w_{i+1}\,, \quad i\ge 1\,;
\\
w_0^+&= w_0 \oc + w_{0,1} + w_1 \oc = {}\\
&\hspace*{8mm}{}=\fr{1 - \rho}{1 - \rho + a \hd^*} \cdot
\fr{a [\oc + c^* \chi^*(\a)] }{\a[1 - \a \oc^* - \a c^* \chi^*(\a)]}\,;
\\
w_i^+ &= w_0 c c_i + w_{0,i+1} + w_{i+1} \oc + {}\\
&\hspace*{20mm}{}+\sum_{j=0}^{i-1} w_{j+1} c c_{i-j},
\enskip i\ge 1.
\end{align*}

В терминах ПФ
$w^-(z) = \sum\limits_{i=0}^{\infty} z^i w^-_i$
стационарного распределения времени ожидания момента
поступления заявки на прибор и ПФ
$w^+(z) \hm= \sum\limits_{i=0}^{\infty} z^i w^+_i$
стационарного распределения времени ожидания начала
обслуживания заявки на приборе последние формулы
приобретают вид:
\begin{align}
%\label{time-}
w^-(z)&= w_0 + w_{0}(1) + \fr{1}{z} w(z)\,;\notag\\
\label{time+}
w^+(z) &= \oc w_0 + c \chi(z) w_0 + \fr{1}{z}\, w_0(z) + {}\\
&\hspace*{14mm}{}+\fr{\oc }{z}\, w(z) + \fr{c}{z}\,\chi(z) w(z)\,.
\end{align}

Наконец, стационарная вероятность $v_i$, $i\ge0$,
того, что заявка будет находиться в системе $i$ тактов,
определяется соотношением:
$$
v_i = \sum_{j=0}^{i} w^+_j d^{*}_{0,i-j}\,,
\ \ i\ge0\,,
$$
или в терминах ПФ
$v(z) = \sum\limits_{i=0}^{\infty} z^i v_i$
стационарного распределения общего
времени пребывания заявки в системе:
\begin{equation}
\label{full}
v(z)= w^+(z) \d^{*}_0(z)\,.
\end{equation}

Как обычно, дифференцируя полученные в этом разделе формулы,
можно найти моменты соответствующих характеристик.
В~частности, стационарные средние значения времени ожидания
момента поступления заявки на прибор, времени ожидания
начала обслуживания заявки на приборе и общего времени
пребывания заявки в системе определяются выражениями:
\begin{align*}
\hat w^- &= {w^-}'(1) = w'(1) - w(1) \,;\\
\hat w^+&= {w^+}'(1) = c \hc w_0 + w'_0(1) - w_0(1) +{}\\
&\hspace*{30mm}{}+ w'(1) - (1 - c \hc) w(1) \,;
\\
\hat v &= v'(1) = \hat w^+ + \hd^*_0 \,,
\end{align*}
где
%%%%%%%%%%%%%%%%%%%%%%%%
\begin{align*}
w_{0}(1) &= \fr{c^* [1 - \chi^*(\a)] }{a [\oc^* + c^* \chi^*(\a)]}w_0\,;
\\
w'_{0}(1) &= w_{0}(1) +{}\\
&{}+ \fr{1}{\oc^* + c^* \chi^*(\a)} \Big(
\fr{c^* {\hc^*} }{a} - \fr{c^* [1 - \chi^*(\a)] }{a^2} \Big) w_0\,;
\\
w(1) &= \fr{a \hd^* }{1 - \rho + a \hd^*}\,;
\\
w'(1) &= w(1) + \fr{1}{2 (1 - \rho)}\, a \td w(1) + {}\\
&\hspace*{-7.73407pt}{}+ \fr{1}{2 (1 - \rho)} \bigg(
a \td + \fr{c^* }{\oc^* + c^* \chi^*(\a)}
\Big[\td^{*}_0 [1 - \chi^*(\a)]
+{}\\
&\hspace*{10mm}{}+2 \hd^{*}_0 \hc^* +\tc^* - 2\fr{1}{a} \Big( \hd^{*}_0 [1 - \chi^*(\a)]
+ {}\\
&\hspace{28mm}{}+\hc^* - \fr{1 - \chi^*(\a) }{a } \Big] \Big) \bigg) w_0\,.
\end{align*}

Производя арифметические преобразования, приходим к равенству:
$$
N^+ = a \hv\,,
$$
которое представляет собой формулу Литтла для данной системы.

%7
\section{Система $\mbox{Geo}_2/\mbox{G}_2/1/\infty$ с двумя типами заявок
и абсолютным приоритетом}

Рассмотрим теперь СМО $\vec{\mbox{Geo}}_2/\vec {\mbox{G}}_2/1/\infty$, в
которую поступают два независимых геометрических потока
заявок.

Заявки второго потока, или неприоритетные заявки, поступают на такте
с вероятностью $a$ и обслуживаются в соответствии с распределением
$\{b_k,\ \ k\ge 0\}$. Заявки первого потока, или приоритетные
заявки, поступают на такте с вероятностью $c$ и обслуживаются в
соответствии с распределением $\{g_k,\ \ k\ge 0\}$. Заявки первого
потока имеют абсолютный приоритет перед заявками второго потока,
т.\,е.\ если поступающая заявка первого потока застает на приборе
заявку второго потока, то она прерывает ее обслуживание и сама
становится на прибор. Заявка второго потока с прерванным
обслуживанием становится в очередь неприоритетных заявок и, как
только система освобождается от всех приоритетных заявок,
возвращается на прибор и продолжает обслуживаться. Заявки одного
потока обслуживаются в порядке поступления.

Покажем, что те формулы, которые были получены
выше для СМО $\mbox{Geo}/\mbox{G}/1/\infty$ с ненадежным прибором,
могут быть применены и к сис\-те\-ме
$\vec{\mbox{Geo}}_{2}/\vec{\mbox{G}}_{2}/1/\infty$
с абсолютным приоритетом для нахождения стационарных
распределений чис\-ла заявок в сис\-те\-ме и времени пребывания
заявки каж\-до\-го потока в сис\-теме.

Заметим, прежде всего, что неприоритетные заявки не
оказывают никакого влияния на обслуживание приоритетных заявок.
Поэтому распределения числа приоритетных заявок в системе
и времени обслуживания приоритетной заявки определяются
как и для обычной системы $\mbox{Geo}/\mbox{G}/1/\infty$, в которую
поступают только приоритетные заявки.

При определении стационарных характеристик для неприоритетных заявок
предположим, что неприоритетные заявки обслуживаются <<вслепую>>,
т.\,е.\ они не знают, из-за чего происходит прерывание обслуживания.
Тогда с их точки зрения система представляет собой просто систему
$\mbox{Geo}/\mbox{G}/1/\infty$
 с ненадежным прибором, в которой вероятность
 поступления заявки на такте равна вероятности $a$\linebreak\vspace*{-12pt}

\pagebreak

\noindent
 поступления
неприоритетной заявки, вероятности отказов прибора в свободном и
занятом состояниях совпадают и равны $c$, а время обслуживания
заявки имеет распределение $\{b_k$, $k\ge 0\}$. Кроме того,
совпадают распределения $\{c_k$, $k\ge 0\}$ и $\{c^*_k$, $k\ge
0\}$ времен ремонта прибора. Однако прерывания происходят не на
время обслуживания приоритетной заявки, а на то время, пока система
полностью не освободится от всех приоритетных заявок. Поэтому
распределения $\{c_k$, $k\ge 0\}$ и $\{c^*_k$, $k\ge 0\}$ равны не
$\{g_k$, $k\ge 0\}$, а распределению периода занятости системы
приоритетными заявками. Напомним, что ПФ $\gamma (z)$ периода
занятости системы $\vec{\mbox{Geo}}_{2}/\vec{\mbox{G}}_{2}/1/\infty$ приоритетными
заявками, совпадающая с ПФ периода занятости сис\-те\-мы
$\mbox{Geo}/\mbox{G}/1/\infty$, в которую поступают только приоритетные заявки,
определяется из уравнения:
$$
\gamma (z) = g(c z + \oc z \gamma(z))\,,
$$
где $g(z)$--- ПФ распределения $\{g_k$, $k\ge 0\}$:

\noindent
$$
g(z) = \sum_{k=0}^\infty z^k g_k\,.
$$


Подставляя эти значения в формулы~(\ref{on_time}), (\ref{time+})
и~(\ref{full}),
получаем ПФ стационарных распределений числа неприоритетных
заявок в системе, времени ожидания неприоритетной заявкой
начала обслуживания и полного времени пребывания
неприоритетной заявки в системе.

Используя результаты предыдущих разделов, нетрудно показать,
что необходимым и достаточным условием существования
стационарного режима функционирования является
$$
\rho = a \hat b + c \hat g < 1\,,
$$
где $\hat b = \sum\limits_{k=0}^{\infty} k b_k$ и
$\hat g \hm= \sum\limits_{k=0}^{\infty} k g_k$~--- средние
длины приоритетных и неприоритетных заявок соответственно, а
$\rho$~--- суммарная загрузка системы,
а также найти стационарные средние значения~$N$, $w$ и~$v$ для
неприоритетных заявок.

\vspace*{-9pt}

%8
\section{Заключение}

Таким образом, в настоящей статье получены аналитические
соотношения, позволяющие вы\-чис\-лять основные стационарные
характеристики обслуживания функционирующей в дискретном
време\-ни однолинейной СМО с ненадежным прибором, который
может отказывать как в рабочем, так и в свободном
состояниях.
Отметим, что отказ свободного прибора и последующий ремонт
можно трактовать как проведение случайной профилактики,
что и сделаем в следующем примере.

%\smallskip


\noindent
\textbf{Пример.}
Пусть рассматриваемая СМО имеет следующие параметры:
\begin{description}
\item[\,]
средняя длина заявки $\hat b=4{,}6$;\\[-14pt]
\item[\,]
факториальный момент второго порядка длины заявки $\tilde b=17{,}0$;\\[-14pt]
\item[\,]
вероятность отказа занятого прибора $c=0{,}3$;\\[-14pt]
\item[\,]
среднее время ремонта прибора, отказавшего в занятом
состоянии, $\hat c=3{,}9$;\\[-14pt]
\item[\,]
факториальный момент второго порядка времени ремонта
прибора, отказавшего в занятом состоянии, $\tilde c=16{,}1$;\\[-14pt]
\item[\,]
вероятность отказа свободного прибора (начала
профилактики) $c^*=0{,}11$;\\[-14pt]
\item[\,]
параметр распределения времени ремонта (профилактики)
прибора, отказавшего в свободном состоянии, $\lambda=0{,}4$.
\end{description}

Предполагается, что время профилактики имеет сдвинутое
на единицу распределение Паскаля с показателем~2 и
параметром~$\lambda$ (это распределение представляет
собой распределение суммы двух независимых геометрических
случайных величин с параметром~$\lambda$).

В~табл.~1 и~2 для некоторых значений вероятности~$a$
поступления заявки на такте приведены результаты расчетов
загрузки~$\rho$, среднего числа заявок~$N$ и среднего
времени~$\hat v$ пребывания заявки в системе для 4-x вариантов:
\begin{description}
\item[\,]
(AA) --- прибор в рабочем состоянии не отказывает, профилактика
не проводится;\\[-14pt]
\item[\,]
(AB) --- прибор в рабочем состоянии не отказывает, профилактика
проводится;\\[-14pt]
\item[\,]
(BA) --- прибор в рабочем состоянии отказывает, профилактика
не проводится;\\[-14pt]
%\begin{description}
\item[\,]
(BB) --- прибор в рабочем состоянии отказывает, профилактика
проводится.
\end{description}

Как видно из таблиц, для данного примера наличие или
отсутствие профилактики при прочих\linebreak\vspace*{-12pt}
%\end{description}

\vspace*{4pt}
\noindent
\begin{center}
\noindent
{\tablename~1}\ \ \small{Результаты расчетов вариантов AA и AB}
\end{center}
%\vspace*{2ex}

\begin{center}
{\small
\tabcolsep=5.5pt
\begin{tabular}{|c|c|r|r|r|r|}
\hline
\raisebox{-6pt}[0pt][0pt]{ $a$} &  \raisebox{-6pt}[0pt][0pt]{$\rho$}
& \multicolumn{2}{c|}{ $N$} &\multicolumn{2}{c|}{ $\hv$} \\
\cline{3-6}
& & \multicolumn{1}{c|}{AA} & \multicolumn{1}{c|}{AB} & \multicolumn{1}{c|}{AA} & \multicolumn{1}{c|}{AB} \\
\hline
 0,03 & 0,138 & 0,1469 & 0,1694 & 4,8958 & 5,6470 \\
 0,05 & 0,230 & 0,2576 & 0,2954 & 5,1520 & 5,9070 \\
 0,07 & 0,322 & 0,3834 & 0,4365 & 5,4776 & 6,2362 \\
 0,09 & 0,414 & 0,5315 & 0,6001 & 5,9055 & 6,6673 \\
 0,11 & 0,506 & 0,7142 & 0,7983 & 6,4927 & 7,2576 \\
 0,13 & 0,598 & 0,9553 & 1,0551 & 7,3486 & 8,1165 \\
 0,15 & 0,690 & 1,3069 & 1,4225 & 8,7129 & 9,4832 \\
 0,17 & 0,782 & 1,9088 & 2,0402 & 11,228\hphantom{9} & 12,001\hphantom{9} \\
 0,19 & 0,874 & 3,3093 & 3,4566 & 17,418\hphantom{9} & 18,192\hphantom{9} \\
 0,21 & 0,966 & 11,991\hphantom{9} & 12,154\hphantom{9} & 57,100\hphantom{9} & 57,877\hphantom{9} \\
\hline
\end{tabular}
}
\end{center}
%\vspace*{12pt}

%\bigskip
\addtocounter{table}{1}


\begin{table*}\small
\begin{center}
\Caption{Результаты расчетов вариантов BA и BB}
\vspace*{2ex}

\begin{tabular}{|c|c|r|r|r|r|}
\hline
\raisebox{-6pt}[0pt][0pt]{ $a$} &  \raisebox{-6pt}[0pt][0pt]{$\rho$}
& \multicolumn{2}{c|}{ $N$} &\multicolumn{2}{c|}{ $\hv$} \\
\cline{3-6}
& & \multicolumn{1}{c|}{BA} & \multicolumn{1}{c|}{BB} & \multicolumn{1}{c|}{BA} & \multicolumn{1}{c|}{BB} \\
\hline
 0,01 & 0,0998 & 0,1061 & 0,1101 & 10,610 & 11,006 \\
 0,02 & 0,1996 & 0,2279 & 0,2360 & 11,394 & 11,799 \\
 0,03 & 0,2995 & 0,3721 & 0,3845 & 12,402 & 12,816 \\
 0,04 & 0,3993 & 0,5498 & 0,5667 & 13,745 & 14,167 \\
 0,05 & 0,4991 & 0,7812 & 0,8026 & 15,624 & 16,053 \\
 0,06 & 0,5989 & 1,1062 & 1,1324 & 18,437 & 18,873 \\
 0,07 & 0,6987 & 1,6180 & 1,6491 & 23,114 & 23,558 \\
 0,08 & 0,7986 & 2,5942 & 2,6302 & 32,427 & 32,878 \\
 0,09 & 0,8984 & 5,4033 & 5,4445 & 60,036 & 60,494 \\
 0,10 & 0,9982 & 314,98\hphantom{99} & 315,03\hphantom{99} & 3149,8\hphantom{99} & 3150,3\hphantom{99} \\
\hline
\end{tabular}
\end{center}
\end{table*}


\noindent
 равных параметрах
практически не влияет на характеристики функционирования.
В~то же время замена ненадежного прибора безотказным
значительно улучшает качество функционирования СМО
(в частности, пропускаемая нагрузка~--- максимальное
значение вероятности~$a$ поступления заявки на такте~---
увеличивается более чем в два раза).

\vspace*{-6pt}

{\small\frenchspacing
{%\baselineskip=10.8pt
\addcontentsline{toc}{section}{Литература}
\begin{thebibliography}{99}

\vspace*{-3pt}

\bibitem{P.S.Ch}
\Au{Печинкин А.\,В., Соколов И.\,А., Чаплыгин В.\,В.} Многолинейная
система массового обслуживания с групповым отказом приборов~//
Информатика и её применения, 2009. Т.~3. Вып.~3. С.~4--15.


\bibitem{1}
\Au{Gaver D.\,P.} A waiting line with interrupted service,
including priorities~// J.\ Roy.\ Stat.\ Soc.~B, 1962. Vol.~24.
P.~73--90.

\bibitem{2}
\Au{Dimitrov B., Dokev Ch.} The single server queue system with
non-reliable server in discrete time. Non-stationary characteristics~//
Ann.\ of Univ.\ of Sofia. Ser.\ Math., 1981. Vol.~70. P.~175--190.

\bibitem{3}
\Au{Altiok. T.} Queueing modeling of a single processor with
failures~// Performance Evaluation, 1989. Vol.~9. No.\,2. P.~93--102.

\bibitem{4}
\Au{Takagi H.}
Queueing analysis. Vol.~3: Discrete-time systems.~---
Amsterdam: North-Holland, 1993.

\bibitem{5}
\Au{Duan-Shin Lee.} Analysis of a single server queue with
semi-Markovian service interruption~// Queueing Syst.: Theory Appl., 1997. Vol.~27. No.\,1--2. P.~153--178.

\bibitem{6}
\Au{Fiems D., Bruneel H.} Analysis of a discrete-time queueing
system with timed vacations~// Queueing Syst., 2002. Vol.~42.
P.~243--254.

\bibitem{7}
\Au{Fiems D., Steyaert B., Bruneel~H.} Analysis of a discrete-time
GI--G--1 queueing model subjected to bursty interruptions~//
Computers Operations Res., 2003. Vol.~30. P.~139--153.

\bibitem{8}
\Au{Moreno P.} A discrete-time retrial queue with unreliable server
and general server lifetime~// J. Math. Sci.,
2006. Vol.~132. P.~643--655.

\bibitem{9}
\Au{Atencia I., Moreno P.} A discrete-time $\mbox{Geo}/\mbox{G}/1$ retrial queue
with the server subject to starting failures~// Annals Operations
Res., 2006. Vol.~141. No.\,1. P.~85--107.

\bibitem{10}
\Au{Demoor T., Fiems D., Walraevens~J.,  Bruneel~H.}
The preemptive repeat hybrid server interruption model.~---
Berlin: Springer-Verlag, 2010.

%%%%%%%%%%%%%%%%%%%%%%%%%%%%%%%%%%

\bibitem{B-A}
\Au{Bocharov P.\,P., D'Apice~C., Manzo~R., Pechinkin~A.\,V.}
Analysis of the multi-server Markov queuing system with unlimited
buffer and negative customers~// Automation Remote Control,
2007. Vol.~68. No.\,1. P.~85--94.

\bibitem{M-C}
\Au{Manzo R., Cascone N., Razumchik~R.\,V.} Exponential queuing
system with negative customers and bunker for ousted customers~//
Automation Remote Control, 2008. Vol.~69. No.\,9. P.~1542--1551.

\label{end\stat}

\bibitem{A-M}
\Au{D'Apice C., Manzo~R., Pechinkin~A.\,V.} A finite $\mbox{MAP}_K/\mbox{G}_K/1$
queueing system with generalized foreground-background
processor-sharing discipline~// Automation Remote Control, 2004.
Vol.~65. No.\,11. P.~114--121.
 \end{thebibliography}
}
}


\end{multicols}