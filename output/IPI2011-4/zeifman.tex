%\newcommand{\A}{{\mathbf A}}
%\newcommand{\B}{{\mathbf B}}
%\newcommand{\la}{{\lambda}}
%\newcommand{\be}{\begin{equation}}
%\newcommand{\ee}{\end{equation}}
%\newcommand{\ber}{\begin{eqnarray}}
%\newcommand{\eer}{\end{eqnarray}}

%\newcommand{\nin}{\noindent}
%\newcommand{\non}{\nonumber}
%\newcommand{\half}{\frac{1}{2}}
%\newcommand{\quarter}{\frac{1}{4}}

\def\stat{zeifman}

\def\tit{ОБ ОДНОМ КЛАССЕ МАРКОВСКИХ СИСТЕМ ОБСЛУЖИВАНИЯ$^*$}

\def\titkol{Об одном классе марковских систем обслуживания}

\def\autkol{Я.\,А.~Сатин, А.\,И.~Зейфман, А.\,В.~Коротышева, С.\,Я.~Шоргин}
\def\aut{Я.\,А.~Сатин$^1$, А.\,И.~Зейфман$^2$, А.\,В.~Коротышева$^3$, С.\,Я.~Шоргин$^4$}

\titel{\tit}{\aut}{\autkol}{\titkol}

{\renewcommand{\thefootnote}{\fnsymbol{footnote}}\footnotetext[1]
{Исследование поддержано РФФИ, гранты 11-07-00112-а и 11-01-12026-офи-м.}}


\renewcommand{\thefootnote}{\arabic{footnote}}
\footnotetext[1]{Вологодский государственный педагогический
университет, yacovi@mail.ru}
\footnotetext[2]{Вологодский государственный педагогический университет;  
Институт проблем информатики Российской академии наук; 
Институт социально-экономического развития территорий Российской академии наук,  a\_zeifman@mail.ru}
\footnotetext[3]{Вологодский государственный педагогический
университет,  a\_korotysheva@mail.ru}
\footnotetext[4]{Институт проблем информатики Российской академии наук, SShorgin@ipiran.ru}


\Abst{Рассматриваются модели обслуживания, описываемые конечными марковскими 
цепями с непрерывным временем. При этом предполагается,  что интенсивности 
поступления и обслуживания требований не зависят от числа требований в сис\-те\-ме. 
Получены оценки скорости сходимости и устойчивости различных характеристик таких сис\-тем.}

\KW{нестационарные марковские системы
обслуживания; скорость сходимости; устойчивость; оценки}

 \vskip 14pt plus 9pt minus 6pt

      \thispagestyle{headings}

      \begin{multicols}{2}
      
            \label{st\stat}

\section{Введение}

Классы систем массового обслуживания, описываемых процессами
рождения и гибели (стационарными и нестационарными, с катастрофами)
изучались начиная с 1970-х~гг.\ многими авторами
(см., например,~[1--6]). С~помощью методов,
разработанных одним из авторов настоящей \mbox{статьи}\linebreak (подробное изложение
этих методов приведено в~[7--9]), для таких сис\-тем
удалось получить точные оценки скорости сходимости и устойчивости.

Оказывается, этот же подход можно применить и к существенно более 
общему классу систем обслуживания.

Рассмотрим систему массового обслуживания, число требований в которой 
описывается нестационарной марковской цепью с непрерывным временем и 
конечным пространством состояний, причем требования могут поступать и 
обслуживаться группами.

Пусть $X=X(t)$, $t\geq 0$,~--- число требований в системе обслуживания ($0 \hm\le X(t) \hm\le r$).

Обозначим через 
\begin{gather*}
p_{ij}(s,t)=\mathrm{Pr}\left\{ X(t)=j\left| X(s)=i\right.
\right\}\,,\\
i,j \ge 0\,,\ 0\leq s\leq t\,,
\end{gather*}
переходные вероятности
процесса $X\hm=X(t)$, а через  $p_i(t)\hm=\mathrm{Pr}\left\{ X(t) \hm=i \right\}$~---
его вероятности состояний.

Будем предполагать, что интенсивности поступления и обслуживания $k$ требований в 
момент~$t$ в сис\-те\-ме об\-слу\-жи\-ва\-ния ($\lambda_{k}(t)$ и  $\mu_{k}(t)$ соответственно)  
не зависят от числа требований, находящихся в системе в момент~$t$, являются локально 
интегрируемыми на $[0,\infty)$ функциями времени~$t$ и, кроме того, 
$\lambda_{k+1}(t) \hm\le \lambda_{k}(t)$ и  $\mu_{k+1}(t) \hm\le \mu_{k}(t)$ при всех~$k$ 
и почти при всех $t \hm\ge 0$.

Тогда для описания вероятностной динамики процесса получаем прямую систему Колмогорова в виде
\begin{equation} 
\fr{d\vp}{dt}=A(t)\vp(t)\,,
\label{ur_1}
\end{equation}
 где
 {\footnotesize
\begin{multline*}
A(t)={}\\
{}=
\begin{pmatrix}
a_{00}(t) & \mu_1(t)  & \mu_2(t)   & \mu_3(t)  & \mu_4(t) & \cdots & \mu_r(t) \\
\la_1(t)   & a_{11}(t)  & \mu_1(t)  & \mu_2(t)   & \mu_3(t)  & \cdots & \mu_{r-1}(t) \\
\la_2(t)  & \la_1(t)    & a_{22}(t)& \mu_1(t)  & \mu2(t)    &  \cdots & \mu_{r-2}(t) \\
\cdots&\cdots&\cdots&\cdots&\cdots&\cdots&\cdots \\
\la_r(t) & \la_{r-1}(t) & \la_{r-2}(t) & \cdots & \la_2(t)  & \la_1 (t)   &  a_{rr}(t)
\end{pmatrix}\,,
\end{multline*}}
причем  
$$
a_{ii}(t)=-\sum\limits_{k=1}^{i}\mu_k(t) - \sum\limits_{k=1}^{r-i} \la_{r-k}(t)\,.
$$

Далее будем обозначать через $\|\bullet\|$  $l_1$-нор\-му, т.\,е.\ 
$\|{\vx}\|\hm=\sum|x_i|$, а $\|B\| \hm= \max\limits_j \sum\limits_i |b_{ij}|$, 
если $B \hm= (b_{ij})_{i,j=0}^{r}$.
%
Тогда, в частности, имеем 
$$
\|A(t)\| \le 2\sum\limits_{k=1}^{r}(\la_{k}(t)+ \mu_k(t))
$$ 
при  всех $t \hm\ge 0$.

Через 
$$
E(t,k) = E\left\{X(t)\left|X(0)\hm=k\right.\right\}
$$ 
будем далее обозначать математическое ожидание процесса (среднее число требований) в момент~$t$ 
при условии, что в нулевой момент времени он находится в состоянии~$k$, 
а через $E_{\bf p}(t)$ обозначим математическое ожидание процесса в момент~$t$ 
при начальном распределении вероятностей состояний $\mathbf{p}(0) \hm= \mathbf{p}$.

\section{Оценки скорости сходимости}

Рассмотрим вспомогательную последовательность положительных чисел $\{d_i\}$, $i\hm=1, \dots,r$.

Положим
\begin{equation*}
d=\min\limits_{1 \le i \le r} d_i\,; \enskip 
G=\sum\limits_{i=1}^r d_i\,; \enskip W=\min\limits_k \fr{d_k}{k}\,.
%\label{2.01}
\end{equation*}

Рассмотрим величины
\begin{multline*}
\alpha_i(t)= -a_{ii}(t)+\la_{r-i+1}(t)-\sum\limits_{k=1}^{i-1}(\mu_{i-k}(t)-{}\\
{}-
\mu_i(t))\fr{d_k}{d_i}-\sum\limits_{k=1}^{r-i}(\la_k(t)-\la_{i+r-1}(t))\fr{d_{k+i}}{d_i}\,,
%\label{2.02}
\end{multline*}

\noindent
\begin{equation*}
\alpha(t)=\min\limits_{1 \le i \le r}\alpha_i(t)\,.
%\label{2.03}
\end{equation*}

\smallskip

\noindent
\textbf{Теорема~1.} \textit{Пусть существует последовательность положительных 
чисел  $\{d_j\}$ такая, что}
\begin{equation}
\int\limits_0^{\infty} \alpha(t)\, dt = + \infty\,.
\label{2.031}
\end{equation}
\textit{Тогда $X(t)$ слабо эргодичен, при
любых начальных условиях} $\mathbf{p}^*(s)$, $\mathbf{p}^{**}(s)$ 
\textit{и любых $s$, $t$, $0\le s\le t$, справедлива оценка
\begin{equation} 
\label{2.04}
\|\vp^*(t)-\vp^{**}(t)\| \le \fr{8G}{d}\,e^{-\int\limits_s^t {\alpha(u)\,du}}\,.
\end{equation}
Кроме того,  $X(t)$ имеет предельное среднее $\phi(t)$ и при любых~$k$ и $t \hm\ge 0$ справедливо неравенство}:
\begin{equation}
\label{2.05}
|E(t,k)-\phi(t)|\le \fr{4G}{W}\,e^{-\int\limits_0^t {\alpha(u)\,du}}\,.
\end{equation}


\smallskip


\noindent
Д\,о\,к\,а\,з\,а\,т\,е\,л\,ь\,с\,т\,в\,о\,.\

Пользуясь предложенным в предыдущих работах способом, 
выразим 
$$
p_0=1-\sum\limits_{1\le i \le r}{p_i}\,.
$$

Тогда получим неоднородное уравнение:
\begin{equation} 
\label{ur_per}
\fr{d\vz}{dt}= B(t)\vz(t)+\vf(t)\,, 
%\label{2.06}
\end{equation}
\noindent
где $\vf(t)=\left(\la_1, \la_2,\cdots,\la_r \right)^{\mathrm{T}}$;

\end{multicols}


\hrule

\vspace*{6pt}

\begin{equation*}
B = \left(
\begin{array}{cccccccc}
a_{11}- \la_1   & \mu_1 - \la_1   & \mu_2 - \la_1   & \mu_3 -\la_1   & \cdots& \cdots & \mu_{r-1}- \la_1  \\
\la_1 -\la_2    & a_{22} -\la_2  & \mu_1-\la_2   & \mu_2 -\la_2     & \cdots&  \cdots & \mu_{r-2} -\la_2 \\
\la_2 -\la_3    & \la_1 -\la_3   & a_{33} -\la_3  & \mu_1-\la_2   & \cdots&  \cdots & \mu_{r-3} -\la_3 \\
\cdots&\cdots&\cdots&\cdots&\cdots&\cdots&\cdots \\
\la_{r-1} -\la_r  &\la_{r-2} -\la_r & \cdots & \cdots & \la_2 -\la_r   & \la_1 -\la_r     &  a_{rr} -\la_r
\end{array}
\right)\,.
%\label{2.07}
\end{equation*}

Рассмотрим треугольную матрицу
\begin{equation*}
D=\begin{pmatrix}
d_1   & d_1 & d_1 & \cdots & d_1 \\
0   & d_2  & d_2  &   \cdots & d_2 \\
\cdots&\cdots&\cdots&\cdots&\cdots \\
0  & 0 & \cdots & 0 &  d_r
\end{pmatrix}
%\label{2.08}
\end{equation*}
и соответствующую норму $\|{\bf z}\|_{D}\hm=\|D {\bf z}\|_1$.

Тогда имеем:
\begin{equation*}
 D BD^{-1}=\left(
\begin{array}{ccccccc}
a_{11}-\la_r  &  (\mu_1-\mu_2) \fr{d_1}{d_2}  & (\mu_2-\mu_3)\fr{d_1}{d_3}  & \cdots &  (\mu_{r-1}-\mu_r)\fr{d_1}{d_r} \\
(\la_1-\la_r) \fr{d_2}{d_1} &  a_{22}-\la_{r-1}  &(\mu_1-\mu_3)\fr{d_2}{d_3}  & \cdots &  (\mu_{r-2}-\mu_r)\fr{d_2}{d_r} \\
(\la_2-\la_r) \fr{d_3}{d_1} &  (\la_1-\la_{r-1})\fr{d_3}{d_2}   &a_{33}-\la_{r-2}   & \cdots &  (\mu_{r-3}-\mu_r)
\fr{d_3}{d_r}  \\
\cdots&\cdots&\cdots&\cdots&\cdots \\
(\la_{r-1} -\la_r) \fr{d_r}{d_1} & (\la_{r-2} -\la_{r-1}) \fr{d_r}{d_2}  & (\la_{r-3} -\la_{r-2}) \fr{d_r}{d_3}  & \cdots & a_{rr}-\la_1 \\
\end{array}
\right)\,.
%\label{2.09}
\end{equation*}


\begin{multicols}{2}


Далее, оценивая логарифмическую норму оператора~$B(t)$ (см., например, 
подробное рассмотрение в~[8--10]), получаем
\begin{multline*}
\gamma \left(B(t)\right)_{1D} = \gamma \left(DB(t)D^{-1}\right)_{1}={}\\
{}=
\max \left(\vphantom{\sum\limits_{k=1}^{i-1}}
a_{ii}(t) - \la_{r-i+1}(t) + \sum\limits_{k=1}^{i-1}\left(\mu_{i-k}(t)-{}\right.\right.\\
\left.\left.{}-\mu_i(t)\right)
\fr{d_k}{d_i} +
\sum\limits_{k=1}^{r-i}(\la_k(t)-\la_{i+r-1}(t))\fr{d_{k+i}}{d_i}\right) ={}\\
{}=
 - \min \alpha_i(t) = - \alpha(t)\,.
% \label{2.10}
\end{multline*}
Тогда\\[-7.9pt]
\begin{equation*}
\|\vz^*(t)-\vz^{**}(t)\|_{1D}\le  e^{-\int\limits_s^t {\alpha(u)du}}\|\vz^*(s)-\vz^{**}(s)\|_{1D}
%\label{2.11}
\end{equation*}
для всех $0 \le s \le t$ и любых начальных условий $\vz^*(s)$, $\vz^{**}(s)$.

Теперь, учитывая оценки для сравнения норм (см., например,~\cite{z08b}), получаем:
\begin{multline*}
\|\vp^*(t)-\vp^{**}(t)\| \le 2\|\vz^*(t)-\vz^{**}(t)\| \le{}\\
{}\le  \fr{4}{d}\|\vz^*(t)-\vz^{**}(t)\|_{1D}\le{} \\
{} \le \fr{4}{d}\,e^{-\int\limits_s^t {\alpha(u)\,du}}\|\vz^*(s)-\vz^{**}(s)\|_{1D} 
\le{}\\
{}\le
 \fr{4G}{d}\,e^{-\int\limits_s^t {\alpha(u)\,du}}\|\vz^*(s)-\vz^{**}(s)\| \le{} \\
{} \le  \fr{4G}{d}\,e^{-\int\limits_s^t {\alpha(u)\,du}}\|\vp^*(s)-\vp^{**}(s)\| \le 
\fr{8G}{d}\,e^{-\int\limits_s^t {\alpha(u)\,du}} 
%\label{2.11-a}
\end{multline*}
для любых начальных условий ${\bf p^*}(s)$, ${\bf p^{**}}(s)$ и любых $s,t$, $0\hm\le s\hm\le t$.

Из слабой эргодичности процесса с конечным пространством состояний 
вытекает существование предельного среднего, начальные условия для которого можно 
в общем случае выбрать произвольно.
Для оценки средних воспользуемся неравенством, приведенным в параграфе~2.3 из~\cite{z08b}:
\begin{multline*}
\|{\bf z}\|_{1D} = d_0 \left|\sum\limits_{i=1}^{\infty} p_i \right|
+ d_1 \left|\sum\limits_{i=2}^{\infty} p_i \right| + \dots \ge{}\\
{}\ge 
 W \sum\limits_{k \ge 1} k \left|\sum\limits_{i \ge k} p_i\right| \ge \fr{W}{2}
\sum\limits_{k \ge 1} k \left|p_k\right|\,.  
%\label{2.12}
\end{multline*}
Получаем теперь
\begin{multline*}
|E(t,k)-\phi(t)|\le \fr{2}{W}\,\|\vp^*(t)-\vp^{**}(t)\|_{1D}\le {} \\
{}\le\fr{2}{W}\,e^{-\int\limits_0^t {\alpha(u)\,du}}\|{\bf e}_k -
\vp^{**}(0)\|_{1D} \le \frac{4G}{W}e^{-\int\limits_0^t
{\alpha(u)\,du}}\,,
%\label{2.13}
\end{multline*}
что и требовалось доказать.
\columnbreak

%\smallskip

\noindent
\textbf{Замечание~1.} {Положим в условиях теоремы~1 
$$
\beta(t)=\max\limits_{1 \le i \le r}\alpha_i(t)\,.
$$ 
Тогда, пользуясь внедиагональной неотрицательностью матрицы $DB(t)D^{-1}$ 
с помощью методики, описанной в~\cite{z08b, z95b}, получаем справедливость неравенства

\noindent
\begin{equation*} 
%\label{2.14}
\|\vp^*(t)-\vp^{**}(t)\| \ge \fr{d}{8G}\,e^{-\int\limits_s^t {\beta(u)\,du}}
\end{equation*}
при любых $s$, $t$, $0\le s\le t$ и уже не при любых начальных условиях~${\bf p^*}(s)$, 
${\bf p^{**}}(s)$, а таких, что  $D\left({\bf p^*}(s) \hm-{\bf p^{**}}(s)\right) \hm\ge 0.$ 
Следовательно, оценки тео\-ре\-мы~1 будут заведомо иметь точный по времени порядок, если удастся 
выбрать вспомогательную последовательность $\{d_i\}$ так, что $\alpha(t)\hm=\beta(t)$, т.\,е.\ 
все $\alpha_i(t)$ одинаковы (не зависят от индекса~$i$)}.



\smallskip

Введем теперь в рассмотрение величины

\vspace*{-1pt}

\noindent
\begin{multline*}
\zeta_i(t)= -a_{ii}(t)+\la_{r-i+1}(t)+{}\\
{}+\sum\limits_{k=1}^{i-1}\left(\mu_{i-k}(t)-
\mu_i(t)\right) \fr{d_k}{d_i}+{}\\
{}+\sum\limits_{k=1}^{r-i}\left(\la_k(t)-\la_{i+r-1}(t)\right)\fr{d_{k+i}}{d_i}\,;
%\label{2.0211}
\end{multline*}
\begin{equation*}
\chi(t)=\max\limits_{1 \le i \le r}\zeta_i(t)\,.
%\label{2.0311}
\end{equation*}

\noindent
\textbf{Замечание 2.} {В условиях теоремы~1 при любых начальных условиях 
${\bf p^*}(s)$, ${\bf p^{**}}(s)$ и любых $s,t$,  $0\le s\le t$, 
справедлива следующая двухсторонняя оценка скорости сходимости:

\vspace*{-1pt}

\noindent
\begin{multline*} 
%\label{2.041}
\!\!\!\fr{d}{4G}\,e^{-\int\limits_s^t {\chi(u)\,du}}\|\vp^*(s)-\vp^{**}(s)\| \le
 \|\vp^*(t)-\vp^{**}(t)\| \le {}\\
 {}\le\fr{4G}{d}\,e^{-\int\limits_s^t {\alpha(u)\,du}}\|\vp^*(s)-\vp^{**}(s)\|.
\end{multline*}
Таким образом, можно оценить и сверху и снизу время  вхождения 
сис\-те\-мы обслуживания в предельный режим. Более подробно о получении 
нижних оценок см., например, в~\cite{z95b, gz05}.}

\smallskip

Рассмотрим два частных случая теоремы.

\smallskip

\noindent
\textbf{Следствие 1}. \textit{Пусть при выполнении остальных условий теоремы~1 
вместо}~(\ref{2.031}) \textit{выполняется условие $\alpha(t) \hm\ge \alpha \hm> 0$ 
почти при всех $t \hm\ge 0$. Тогда вместо}~(\ref{2.04}) \textit{и}~(\ref{2.05}) 
\textit{справедливы оценки}:

\vspace*{-1pt}

\noindent
\begin{align*} 
%\label{2.15}
\|\vp^*(t)-\vp^{**}(t)\| &\le \fr{8G}{d}\,e^{-\alpha \left(t-s\right)}\,;
\\
%\label{2.16}
|E(t,k)-\phi(t)|&\le \fr{4G}{W}\,e^{- \alpha t}\,.
\end{align*}

\pagebreak

%\smallskip

Положим 
\begin{gather*}
M_0=\max\limits_{|t-s|\le 1}\int\limits_s^t \alpha(u)\,du;\\
\alpha^* = \int\limits_0^1 \alpha(t)\, dt\,; \quad
M=e^{M_0+\alpha^*}\,.
\end{gather*}
С учетом неравенства 
$$
e^{-\int\limits_s^t {\alpha(u)\,du}} \hm\le M e^{-\alpha^* (t-s)}
$$ 
получаем следующее утверждение.

\smallskip

\noindent
\textbf{Следствие~2.} \textit{Пусть все $\lambda_k(t)$ и $\mu_k(t)$ 1-пе\-ри\-одич\-ны,  
а при выполнении остальных условий теоремы~1 вместо}~(\ref{2.031}) 
\textit{выполняется условие  $\alpha^* \hm> 0$.  Тогда предельный режим (скажем, $\vp^*(t)$) 
и соответствующее ему предельное среднее $\phi^*(t)$ можно выбрать 
1-пе\-ри\-оди\-че\-ски\-ми, а вместо}~(\ref{2.04}) \textit{и}~(\ref{2.05}) 
\textit{справедливы оценки}:
\begin{equation*} 
%\label{2.17}
\|\vp(t) - \vp^*(t)\| \le \fr{8GM}{d}\,e^{-\alpha^*t}
\end{equation*}
\textit{и, кроме того,}
\begin{equation*}
|E(t,k)-\phi^*(t)|\le \fr{4GM}{W}\,e^{-\alpha^*t}
%\label{2.18}
\end{equation*}
\textit{при любом $k$ и $t \ge 0$}.



\section{Устойчивость}

Рассмотрим также <<возмущенный>> процесс обслуживания $\bar{X}\hm=\bar{X}(t)$, $t\hm\geq 0$, 
в котором интенсивности поступления и обслуживания требований также не зависят от чис\-ла 
требований в системе, обозначая его соответствующие характеристики теми же буквами с 
чертой сверху. Для прос\-то\-ты записи оценок будем предполагать, что возмущения 
<<равномерно малы>>, т.\,е.\ выполняется неравенство $\| A(t)-\bar{A}(t)\| \hm\le \varepsilon$. 
Первые результаты для нестационарных цепей с непрерывным временем получены в~\cite{z85}, 
а детальное рассмотрение для более общего случая неравномерных оценок можно без труда 
провести так же, как это сделано в~\cite{z98, ae}. Для получения требуемых равномерных 
оценок устойчивости необходима экспоненциальная эргодичность соответствующего процесса, 
т.\,е.\ существование положительных констант $N$, $a$ таких, что  для правой части~(\ref{2.04}) 
справедливо неравенство:
\begin{equation}
e^{-\int\limits_s^t {\alpha(u)\,du}} \le Ne^{-a\left(t-s\right)}\,.
\label{3.01}
\end{equation}
Оценка~(\ref{3.01}) заведомо имеет место, в частности, если выполнены условия одного из следствий 
предыду\-ще\-го параграфа.

\smallskip

\noindent
\textbf{Теорема~2.}
\textit{Пусть выполнены условия теоремы~1 и}~(\ref{3.01}). \textit{Тогда при
 любых начальных условиях ${\bf p}(s)$ и ${\bar{\bf p}}(s)$ для процессов~$X(t)$ 
 и $\bar{X}(t)$ соответственно справедливы следующие оценки устойчивости:}
\begin{align*} 
%\label{3.02}
\limsup_{t \to \infty}  \|{\bf p}(t)- \bar{\bf p}(t)\| &\le
\fr{\varepsilon(1+\ln(4GN/d))}{a}\,;
\\
% \label{3.03}
\limsup\limits_{t \to \infty}   |E_{\bf p}(t)- \bar{E}_{\bar{\bf p}(t)}|&\le 
\fr{r \varepsilon(1+\ln(4GN/d))}{a}\,.
\end{align*}


\smallskip

\noindent
Д\,о\,к\,а\,з\,а\,т\,е\,л\,ь\,с\,т\,в\,о\ основано на подходе, 
введенном для стационарных процессов в~\cite{mit03} и описанном для нестационарной 
ситуации в~\cite{z11}.
Если  при любых начальных условиях для исходного процесса справедлива оценка
\begin{equation*} 
%\label{3.04}
\|\vp(t) - \vp^*(t)\| \le ce^{-b\left(t-s\right)}\,,
\end{equation*}
то, полагая
\begin{multline*}
\beta (t, s)=\sup\limits_{ \| {\bf v} \| =1, \sum {v_i}=0}
{\|V(t,s){\bf v}(t,s)\|} ={}\\
{}= \fr{1}{2} \max_{i,j} \sum\limits_k {|p_{ik}(t,
s)-p_{jk}(t, s)|}\,, 
\end{multline*}
где $V(t, s)$~--- матрица Коши
уравнения~(\ref{ur_1}), получаем в итоге следующее неравенство:
\begin{equation*}
\|{\bf p}(t)-\bar{\bf p}(t)\| \le{}
\begin{cases}
\|{\bf p}(s)-{\bf \bar{p}}(s)\|+ (t-s)\varepsilon \,, &\\
&\hspace*{-35mm} 0<t< b^{-1} \ln \left(\fr{c}{2}\right)\,; \\
b^{-1}\left(\ln \fr{c}{2} +1-\fr{c}{2}\,e^{-b(t-s)}\right)\varepsilon +{}&\\
{}+
\fr{c}{2}\,e^{-b(t-s)} \|{\bf p}(s)-{\bf \bar{p}}(s)\|\,, &\\
&\hspace*{-30mm}t\ge b^{-1}\ln \left(\fr{c}{2}\right)
\end{cases}
%\label{3.05}
\end{equation*}
для любых начальных условий ${\bf p}(s)$ и $\bar{\bf p}(s)$.
Из неравенств~(\ref{2.04}) и~(\ref{3.01}) вытекает, что $b=a$, $c={8GN}/{d}$.  
Устремив $t \hm\to \infty$ и взяв $s\hm=0$, получаем требуемые оценки.


\smallskip

\noindent
\textbf{Замечание~3.} 
В полученную оценку устойчивости для математического ожидания процесса 
в качестве множителя входит размерность~$r$, поэтому иногда лучший результат 
удается получить при помощи другого подхода, описанного в работе~\cite{z11}.

\smallskip

Положим 
$$
S=\max\limits_{{1 \le i, j \le r}} \fr{d_i}{d_j}\,,
$$ 
и пусть числа $K, L$ таковы, что 

\noindent
$$
d_1\la_1(t) + (d_1+d_2)\la_2(t) + \dots + 
\left(\sum\limits_{1 \le i \le r}d_i\right) \la_r(t) \le K\,,
$$ 
а 

\noindent
\begin{multline*}
d_1(\la_1(t)-\bar{\la}_1(t)) + (d_1+d_2)(\la_2(t)-\bar{\la}_2(t)) + \dots\\
\dots + 
\left(\sum\limits_{1 \le i \le r}d_i\right) (\la_r(t)-\bar{\la}_r(t)) \le 
L\varepsilon
\end{multline*} 
почти при всех $t \ge 0.$

\smallskip

\noindent
\textbf{Теорема~3.}
\textit{Пусть  выполнены условия теоремы~2 и, кроме того, при всех~$k$ 
и почти всех $t \hm\ge 0$ $\la_k(t) \hm< \infty$. Тогда при любых начальных условиях 
${\bf p}(s)$ и ${\bar{\bf p}}(s)$ для процессов $X(t)$ и $\bar{X}(t)$ 
соответственно справедливо неравенство}

\noindent
\begin{equation*}
\limsup\limits_{t \to \infty}   |E_{\bf p}(t)- \bar{E}_{\bar{\bf p}(t)}|\le 
\fr{ N\varepsilon\left(L a+ 2KNS\right)}{W a \left(a-2\varepsilon S\right)}\,.
\end{equation*}


\smallskip

\noindent
Д\,о\,к\,а\,з\,а\,т\,е\,л\,ь\,с\,т\,в\,о.\
 Перепишем исходную систему~(\ref{ur_per}) для невозмущенного процесса в следующем виде:
 \noindent
 
\begin{equation*}
\fr{d\vp}{dt}=\bar{B}(t)\vp(t) + {\bf f}(t)+\left(B(t)-\bar{B}(t)\right)\vp(t)\,.
%\label{eq112-n}
\end{equation*}
Тогда

\noindent
\begin{multline*}
\vp(t)=\bar{U}(t,0)\vp(0)+\int\limits_0^t \bar{U}(t,\tau){\bf{f}}(\tau) \, d\tau+{}\\
{}+\int\limits_0^t \bar{U}(t,\tau) \left(B(\tau)-\bar{B}(\tau)\right)\vp(\tau)\, d\tau\,;
\end{multline*}

\vspace*{-9pt}

\begin{equation*}
\hspace*{-15mm}\bar{\vp}(t)=\bar{U}(t,0)\bar{\vp}(0)+\int\limits_0^t \bar{U}(t,\tau){\bf{f}}(\tau) \, d\tau,
\end{equation*}
где $U(t,s)$~--- матрица Коши для уравнения~(\ref{ur_per}).
В любой норме при одинаковых начальных условиях получаем следующую оценку:
%\noindent
\begin{multline}
 \label{3000}
\!\!\!\!\!\!\left\|\vp(t)-\bar{\vp}(t)\right\|\le \!\!\int\limits_0^t \!\!\|\bar{U}(t,\tau)\|
\left(\| B(\tau)-\bar{B}(\tau)\| \|\vp(\tau)\| +\right.\\
\left.{}+ \| \vf(\tau)-\bar{\vf}(\tau)\|\right)\,d\tau\,.\!
\end{multline}
Имеем почти при всех $t \ge 0$:
\begin{equation*}
\|B(t)-\bar{B}(t)\|_{1D}=\|D(B(t)-\bar{B}(t))D^{-1}\| \le 2S\varepsilon\,;
%\label{3002}
\end{equation*}
%
%\vspace*{-14pt}
%
%\noindent
\begin{multline*}
\|{\bf f}(t)\|_{1D} \le d_1\la_1(t) + (d_1+d_2)\la_2(t) + \dots + {}\\
{}+
\left(\sum\limits_{1 \le i \le r}d_i\right) \la_r(t) \le K\,, 
\quad \|\vf(\tau)-\bar{\vf}(\tau)\|_{1D} \le L\varepsilon\,.
%\label{3002-a}
\end{multline*}
А тогда
\begin{multline*}
\gamma(\bar{B}(t))_{1D} \le \gamma(DB(t)D^{-1})+\|B(t)-\bar{B}(t)\|_{1D} \le  {}\\
{}\le -
\alpha(t)+2S \varepsilon \,.
% \label{3003}
\end{multline*}

Оценим теперь
\begin{multline*} 
%\label{8402}
\!\|{\bf p}(t)\|_{1D} \le
\|U(t){\bf p}(0) \|_{1D} +
 \int\limits_0^t \!\!\| U(t,\tau){\bf f}(\tau)\, d\tau \|_{1D} \le {}\\
 {}\le
 N e^{-a t} \| \vp(0)\|_{1D}  + \fr{K N}{a}.
\end{multline*}

 Тогда с учетом~(\ref{3000}) получаем:
\begin{multline*} 
%\label{3004}
\left\|\vp(t)-\bar{\vp}(t)\right\|_{1D}\le N\int\limits_0^t e^{-(a - 2\varepsilon S)(t-\tau)}\times{}\\
{}\times
\left(2S\varepsilon (N e^{-a \tau} \| \vp(0)\|_{1D}  + \fr{K N}{a}) +  L\varepsilon \right)\, d\tau  \le {} \\
{}\le  o(1)+\fr{ N\varepsilon(L+{2KNS}/{a})}{a-2\varepsilon S}\,. 
\end{multline*}

\vspace*{-9pt}

\section{Примеры}

\noindent
\textbf{Пример 1.}

Рассмотрим исходный процесс обслуживания с интенсивностями 
$\la_1(t)\hm=\la_2(t)\hm=\la_3(t)\hm=\la(t) \hm= 3\hm+\sin{2\pi t}$, 
$\mu_1(t)\hm=\mu_2(t)\hm=  \mu(t) \hm= 2\hm+\cos{2\pi t}$, 
$\la_4(t)=\ldots=\la_r(t)\hm=\mu_3(t)=\ldots=\mu_r(t)\hm=0$. Выберем последовательность  
$d_k\hm=h^k$, где $0{,}82 \hm< h \hm<1$. Тогда имеем
$$
d=h^r\,; \quad G \le \fr{h}{1-h}\,; \quad W=\fr{h^r}{r}\,.
$$

Будем предполагать, что возмущенный процесс имеет такую же структуру 
мат\-ри\-цы интенсивностей, причем $|\la(t)\hm-\bar{\la}(t)| \hm\le \varepsilon$ 
и  $|\mu(t)\hm-\bar{\mu}(t)| \hm\le \varepsilon$ почти при всех $t \hm\ge 0$. 
Отметим кстати, что при этом $\| A(t)\hm-\bar{A}(t)\| \hm\le 10 \varepsilon$ почти при 
всех $t \hm\ge 0$. Рассмотрим дальнейшие оценки:
$$
S=\fr{1}{h^2}\,; \ K=4 \left(3h+2h^2+h^3\right)\,; \ L=3h+2h^2+h^3\,;
$$
$$
\alpha(t) \ge \la(t)\left(3 - h - h^2 -h^3\right)-\mu(t)\left(\fr{1}{h^2}+\fr{1}{h}-2\right)\,;
$$
$$
\alpha^*= 3\left(3 - h - h^2 -h^3\right)-2\left(\fr{1}{h^2}+\fr{1}{h}-2\right)\,;
$$


\noindent
\begin{multline*}
M_0 \le \int\limits_0^1 |\alpha(t)|\, dt \le 4\left(3 - h - h^2 -h^3\right)+{}\\
{}+
3\left(\fr{1}{h^2}+\fr{1}{h}-2\right)\,;
\end{multline*}

\vspace*{-9pt}

\noindent
$$
M=e^{\alpha^*+M_0}\,.
$$

Если, например, взять 
$h\hm=0{,}9$, то $\alpha^*\hm=0{,}992$, $M_0\hm=3{,}281$, $M\hm=71{,}737$.

Тогда получаем следующие оценки.

По следствию~2
\begin{align*}
 \|{\bf p}(t)- {\bf p^{*}}(t)\| &\le \fr{8Me^{-\alpha^*t}}{h^{r-1}(1-h)}\,;\\
|E_{\bf p}(t)-\phi^*(t)| &\le  \fr{4Mre^{-\alpha^*t}}{h^{r-1}(1-h)}\,.
\end{align*}

По теореме~2 ($N=M$, $a=\alpha^*$) с использованием оценок следствия~2
\begin{align*}
\limsup\limits_{t \to \infty} \|{\bf p}(t)- \bar{\bf p}(t)\| &\le{} \notag\\
&\hspace*{-15mm}{}\le \fr{\varepsilon(1+\ln({4M}/({h^{r-1}(1-h)})))}{\alpha^*}\,;\\
\limsup\limits_{t \to \infty}   |E_{\bf p}(t)- \bar{E}_{\bar{\bf p}(t)}| &\le \notag\\
&\hspace*{-15mm}{}\le\fr{r\varepsilon(1+\ln(4M/(h^{r-1}(1-h))))}{\alpha^*}\,.
\end{align*}

По теореме~3 с использованием оценок следствия~2
\begin{multline*}
\limsup\limits_{t \to \infty}   |E_{\bf p}(t)- \bar{E}_{\bar{\bf p}(t)}| \le {}\\
{}\le
\fr{rM\varepsilon(3h+2h^2+h^3)(\alpha^* h^2+8M)}{h^r\alpha^*(\alpha^* h^2-2\varepsilon)}\,.
\end{multline*}

\noindent

\textbf{Пример 2.}

Рассмотрим процесс с интенсивностями 
$\la_1(t)\hm=\la_2(t)\hm=\ldots=\la_r(t) \hm= \la(t) \hm= 3\hm+\sin{2\pi t}$; 
$\mu_1(t)\hm=\mu_2(t)\hm= \mu(t) \hm= 2+\cos{2\pi t}$;
$\mu_3(t)=\ldots=\mu_r(t)=0$.

Будем предполагать, что возмущенный процесс имеет такую же структуру 
мат\-ри\-цы интен\-сив\-ностей, причем $|\la(t)-\bar{\la}(t)| \hm\le \varepsilon$ и  
$|\mu(t)-\bar{\mu}(t)| \hm\le \varepsilon$ почти при всех $t \hm\ge 0$. 
При этом будем иметь $\| A(t)\hm-\bar{A}(t)\| \hm\le 2r \varepsilon$ почти при всех $t \hm\ge 0$.

Выберем последовательность $d_k\hm=1$. Тогда  
\begin{gather*}
d=1\,; \enskip G=r\,; \enskip W=\fr{1}{r}\,; \enskip S=1\,; \\
K=\fr{4r(1+r)}{2}\,; \quad L=\fr{r(1+r)}{2}\,;
\\
\alpha(t)=\la(t)\,; \ \alpha=2\,; \ \alpha^*=3\,; M_0 \le 4\,; \ M \le  e^{7}\,.
\end{gather*}

И получаем следующие оценки.

\columnbreak

По следствию~1
\begin{align*}
 \|{\bf p^*}(t)- {\bf p^{**}}(t)\| &\le 8re^{-2t}\,;\\
|E_{\bf p}(t)- \phi(t)|&\le  4r^2 e^{-2t}\,.
\end{align*}

По следствию~2
\begin{align*}
\|{\bf p}(t)- {\bf p^{*}}(t)\| &\le 8re^{7-3t}\,;
\\[6pt]
|E_{\bf p}(t)- \phi^*(t)| &\le 4r^2 e^{7-3t}\,.
\end{align*}

По теореме~2 ($N=1$, $a=\alpha$) с учетом оценок следствия~1
\begin{align*}
\limsup\limits_{t \to \infty} \|{\bf p}(t)- \bar{\bf p}(t)\| &\le 
\fr{\varepsilon(1+\ln{4r})}{2}\,;
\\[6pt]
\limsup\limits_{t \to \infty}   |E_{\bf p}(t)- \bar{E}_{\bar{\bf p}(t)}|
&\le \fr{r\varepsilon(1+\ln{4r})}{2}\,.
\end{align*}

По теореме~2 ($N=M$, $a=\alpha^*$) с учетом оценок следствия~2
\begin{align*}
\limsup\limits_{t \to \infty} \|{\bf p}(t)- \bar{\bf p}(t)\| &\le 
\fr{\varepsilon(8+\ln{4r})}{3}\,;
\\
\limsup\limits_{t \to \infty}   \left|E_{\bf p}(t)- \bar{E}_{\bar{\bf p}(t)}\right| &\le 
\fr{r\varepsilon(8 + \ln{4r})}{3}\,.
\end{align*}

По теореме~3 с учетом оценок следствия~1
\begin{equation*}
\limsup\limits_{t \to \infty}   \left|E_{\bf p}(t)- \bar{E}_{\bar{\bf p}(t)}\right| \le 
\fr{5 \varepsilon r^2 (1+r)}{4(1- \varepsilon)}\,.
\end{equation*}

По теореме~3 с учетом оценок следствия~2
\begin{equation*}
\limsup\limits_{t \to \infty}   \left|E_{\bf p}(t)- \bar{E}_{\bar{\bf p}(t)}\right| \le 
\fr{\varepsilon e^{7} r^2 (1+r) (3+8e^{7})}{6(3-2\varepsilon)}\,.
\end{equation*}

{\small\frenchspacing
{%\baselineskip=10.8pt
\addcontentsline{toc}{section}{Литература}
\begin{thebibliography}{99}

 \bibitem{b} %1
\Au{Баруча-Рид~А.\,Т.} Элементы теории марковских процессов и их
приложения.~--- М.: Наука, 1969.

\bibitem{gm}  %2
\Au{Гнеденко~Б.\,В., Макаров~И.\,П.} Свойства решений задачи с потерями
в случае периодических интенсивностей~// Дифф. уравнения, 1971.
Вып.~7. С.~1696--1698.

\bibitem{g1}   %3
\Au{Gnedenko~D.\,B.} On a generalization of Erlang formulae~// 
Zastosow. Mat., 1971. Vol.~12. P.~239--242.

\bibitem{S}  %4
\Au{Саати~Т.\,Л.} Элементы теории массового обслуживания
 и ее приложения.~--- М.: Сов. радио, 1971.

\bibitem{g}  %5
\Au{Gnedenko~B., Soloviev~A.} On the conditions of the
existence of final probabilities for a Markov process~// Math.
Operations. Stat., 1973. P.~379--390.

\bibitem{gk} %6
\Au{Гнеденко~Б.\,В., Коваленко~И.\,Н.} Введение в теорию массового
обслуживания.~--- М.: Наука, 1987.
\pagebreak

\bibitem{gz00}   %7
\Au{Granovsky~B.\,L., Zeifman~A.\,I.}  The N-limit of spectral gap of 
a class of birth-death Markov chains~//
 Appl. Stoch. Models Business Ind., 2000. Vol.~16. P.~235--248.

\bibitem{z08b}  %8
\Au{Зейфман~А.\,И., Бенинг~В.\,Е., Соколов~И.\,А.} 
Марковские цепи и модели с непрерывным временем.~--- М.: Элекс-КМ, 2008.

\bibitem{dzp} %9
\Au{Van Doorn~E.\,A., Zeifman~A.\,I., Panfilova~T.\,L.}  
Bounds and asymptotics for the rate of convergence of birth-death processes~//  
Th. Prob. Appl., 2010. Vol.~54. P.~97--113.

\bibitem{z95b}   %10
\Au{Zeifman~A.\,I.} Upper and lower bounds on the rate of
convergence for nonhomogeneous birth and death processes~//  Stoch.
Proc. Appl., 1995. Vol.~59. P.~157--173.

\bibitem{gz05}  %11
\Au{Granovsky~B.\,L., Zeifman~A.\,I.} On the lower bound of the spectrum
 of some mean-field models~// Theory Prob. Appl., 2005. Vol.~49. P.~148--155.
 
\bibitem{z85}  %12
\Au{Zeifman~A.\,I.} Stability for contionuous-time
nonhomogeneous Markov chains~// Lect. Notes Math.,  1985. Vol.~1155.
P.~401--414.

\bibitem{z98} %13
\Au{Zeifman~A.} Stability of birth and death processes~// 
J.~Math. Sci., 1998. Vol.~91. P.~3023--3031.

\bibitem{ae} %14
\Au{Андреев~Д., Елесин~М., Кузнецов~А., Крылов~Е., Зейфман~А.}
Эргодичность и устойчивость нестационарных систем обслуживания~//
Теория вероятностей и математическая статистика, 2003. Т.~68.
С.~1--11.

\bibitem{mit03} %15
\Au{Mitrophanov~A.\,Yu.} Stability and exponential convergence of continuous-time 
Markov chains~//  J. Appl. Prob., 2003. Vol.~40. P.~970--979.

\label{end\stat} 

\bibitem{z11} %16
\Au{Зейфман~А.\,И., Коротышева~А.\,В., Панфилова~Т.\,Л., Шоргин~С.\,Я.} 
Оценки устойчивости  для некоторых систем обслуживания с катастрофами~//  
Информатика и её применения, 2011. Т.~5. Вып.~3. С.~27--33.
 \end{thebibliography}
}
}


\end{multicols}       