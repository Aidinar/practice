\def\stat{lebedev}

\def\tit{МАКСИМУМЫ АКТИВНОСТИ В~БЕЗМАСШТАБНЫХ СЛУЧАЙНЫХ СЕТЯХ С~ТЯЖЕЛЫМИ ХВОСТАМИ$^*$}

\def\titkol{Максимумы активности в~безмасштабных случайных сетях с~тяжелыми хвостами}

\def\autkol{А.\,В.~Лебедев}
\def\aut{А.\,В.~Лебедев$^1$}

\titel{\tit}{\aut}{\autkol}{\titkol}

{\renewcommand{\thefootnote}{\fnsymbol{footnote}}\footnotetext[1]
{Работа выполнена при поддержке РФФИ, грант 11-01-00050.}}


\renewcommand{\thefootnote}{\arabic{footnote}}
\footnotetext[1]{Механико-математический факультет
    Московского государственного университета им.\ М.~В.~Ломоносова,
    avlebed@yandex.ru}
    
  %  \vspace*{-6pt}

\Abst{Рассматриваются ориентированные степенные случайные графы как
    модели информационных сетей, где каждый узел обладает случайной информационной
    активностью, распределение которой имеет тяжелый (правильно меняющийся)
    хвост. Используется модель случайного графа, в которой входящие
    степени вершин независимы и имеют распределение со степенным хвостом.
    Выведены достаточные условия, при которых максимум
    суммарных активностей (по узлу и его входящим соседям) растет асимптотически
    так же, как и максимум индивидуальных активностей, и в силу этого для них
    имеет место предельный закон Фреше.}

 %   \vspace*{-2pt}
    
    \KW{максимумы; случайные суммы; безмасштабные сети;
    степенной закон; случайный граф; тяжелый хвост; правильное изменение;
    распределение Фреше}
    
%                    \vspace*{-9pt}
    
     \vskip 14pt plus 9pt minus 6pt

      \thispagestyle{headings}

      \begin{multicols}{2}
      
            \label{st\stat}
            


\section{Введение}

    Степенными (\textit{power-law}) или безмасштабными (\textit{free-scale}) называют
    случайные графы, у которых степени вершин подчиняются асимптотически
    степенному закону (с вероятностями $p_k\hm\sim ck^{-\beta}$, $k\hm\to\infty$,
    $\beta\hm>1$).
    Активные исследования данного класса графов и их приложений
    в последнее десятилетие были инициированы работой~\cite{Bar},
    где приведен ряд интересных примеров (Интернет, электрическая сеть,
    социальная сеть киноактеров).
    С~тех пор были предложены и изучены различные модели степенных графов.
    В~одних степенной закон возникает благодаря некоторому случайному
    процессу~\cite{Bar, Komp}, в других он постулируется изначально~\cite{Pow, Reit}.
    Следует отметить, что некоторые асимптотические свойства графов при
    одинаковом распределении степеней вершин могут оказаться общими,
    а другие зависят от выбора модели.

    Степенной граф может служить моделью некоторой информационной сети.
    Например, исследования кириллического сегмента <<Живого журнала>>
    ({\sf livejournal.com})~\cite{Komp}
    показывают, что он хорошо описывается степенным графом с $\beta\hm\approx 3$.
    Пусть каждый узел этой сети обладает случайной информационной активностью
    (интенсивностью производства информации). Имеется в виду среднее количество
    информации, производимой узлом в единицу времени. Активность будем
    полагать индивидуальной характеристикой, присущей узлу. Например, речь может идти
    о пользователях, которые пишут сообщения в Интернет.
    Предположим, что активности узлов
    независимы и одинаково распределены, причем их распределение~$F$ имеет тяжелый
    (правильно меняющийся) хвост, т.\,е.\ ${\bar F}(x)\hm\sim x^{-a}L(x)$,
    $x\hm\to\infty$, $a>0$, где $L(x)$~--- медленно меняющаяся функция~\cite[\S\ 8.8]{Fel}. 
    Такое предположение находится в русле современных
    представлений о распространенности степенных законов в природе, технике и
    человеческой деятельности. Активности и степени вершин (узлов) для
    простоты будем полагать независимыми.

    Рассмотрим суммарную активность в узле (т.\,е.\ сумму его собственной и
    ближайших соседей). Например, в <<Живом журнале>>
    каждый пользователь может оставлять свои записи и читать записи
    своих друзей, объединяемые для удобства в общую <<ленту друзей>> (френдленту).
    Далее будем интересоваться вопросом: когда максимум суммарных активностей
    растет асимптотически так же, как и максимум индивидуальных
    активностей узлов? В~этом случае для максимумов легко выводится
    предельный закон Фреше
    $\Phi_a(x)\hm=\exp\{-x^{-a}\}$, $x\hm>0$~[5, \S\ 8.8; 6, \S~3.3.1].

    Для модели степенного графа, введенной в~\cite{Pow},
    этот вопрос был изучен автором в~\cite{Leb2}.
    Там число вершин степени~$k$ полагалось
    детерминированным и равным $\lfloor e^\alpha/k^\beta\rfloor$,
    $\alpha$, $\beta>0$, $1\hm\le k\hm\le e^{\alpha/\beta}$,
    а распределение на множестве графов,
    удовле\-тво\-ря\-ющих этому условию, равномерным.
    Были получены достаточные условия того,
    что максимум сумм с ростом числа узлов (при $\alpha\hm\to\infty$)
    растет асимптотически так же, как и макcимум
    индивидуальных активностей: $a\hm<\beta\hm-3/2$, если $3/2\hm<\beta\hm<3$, и
    $a\hm<\beta/2$, если $\beta\hm\ge 3$. При этом применялись
    ранее полученные автором результаты для
    общей схемы максимумов сумм независимых случайных величин~\cite{Leb1}.

    Рассмотрим теперь модель ориентированного
    случайного графа, где направления ребер соответствуют направлениям
    передачи информации. Пусть имеется $n$ вершин и заданы независимые
    неотрицательные целочисленные случайные величины $K_1,\dots, K_n$,
    имеющие одинаковое
    распределение, заданное вероятностями $p_k\hm\sim ck^{-\beta}$, $k\hm\to\infty$,
    $\beta>1$. Положим $D_i\hm=\min\{K_i,n-1\}$. Для $i$-й вершины выберем
    случайным образом (равновероятно и независимо от выбора для других
    вершин) $D_i$ различных вершин из числа остальных (кроме $i$-й) и
    выпустим из них ребра, направленные в \mbox{$i$-ю} вершину. Полученный в
    результате граф можно отнести к степенным в том смысле, что входящие
    степени вершин распределены асимптотически по степенному закону.
    Суммарной активностью в узле в данном случае будем считать сумму
    собственной активности узла и всех узлов, из которых в него
    поступает информация (его входящих соседей).

    К сожалению, метод, использованный в~\cite{Leb2}, здесь
    не работает при $\beta\hm<3$, так как второй момент
    входящей степени вершины тогда растет слишком быстро при $n\hm\to\infty$.
    Эта проблема решается с по\-мощью урезания.
    При этом получаются более сильные ограничения на
    параметры, что связано с более быст\-рым ростом максимальной (входящей)
    степени вершины в графе. Однако поскольку используются лишь
    достаточные условия, не исключено, что эти ограничения в
    будущем могут быть ослаблены.

    %Результаты \cite{Leb2} и настоящей работы были кратко изложены в \cite{Leb3}.

    Отметим, что асимптотическая эквивалентность хвостов распределений
    суммы и максимума конечного числа независимых одинаково распределенных
    случайных величин в случае тяжелых хвостов
    представляет собой давно известный факт~\cite[\S\ 8.8]{Fel},
    обусловленный тем, что основной вклад в сумму дает самое большое
    слагаемое (максимум), а сумма остальных слагаемых по сравнению с
    ним оказывается мала. Теперь обобщим это утверждение
    на модель, где имеется
    некоторый набор случайных сумм со случайными числами слагаемых
    и от сумм берется максимум. По-преж\-не\-му оказывается, что основной
    вклад (в одну или несколько сумм, а значит, и в их максимум) дает только
    одно, максимальное слагаемое. Однако для этого хвост распределения
    слагаемых должен быть достаточно тяжелым.

    Проверка наличия подобного эффекта в реальных сетях, разумеется,
    требует экспериментального исследования, выходящего
    за рамки данной работы, которая имеет теоретический характер.
    
        \vspace*{-9pt}

    
    \section{Основной результат}
    
    \vspace*{-2pt}

    Будем рассматривать сети из $n$ узлов, затем устремляя $n$ к бесконечности.
    Обозначим через $M(n)$ максимум суммарных активностей (самого узла и его
    входящих соседей), а через $M_0(n)$~--- максимум индивидуальных активностей
    узлов. Требуется определить условия, при которых
    \begin{equation}
    \label{MP}
\fr{M(n)}{M_0(n)}\stackrel{P}{\to} 1\,,\quad n\to\infty\,.
    \end{equation}

    Введем неотрицательную функцию
    $u(s)$ такую, что $s{\bar F}(u(s))\hm\to 1$, $s\hm\to\infty$.
    Заметим, что $u(s)$ заведомо существует и правильно
    меняется с показателем $1/a$, т.\,е.\ $u(s)\hm\sim s^{1/a}L_2(s)$,
    $s\to\infty$, где $L_2(s)$~--- медленно меняющаяся функция~\cite[\S\ 1.5]{Sen}.

    Тогда имеет место предельный закон для максимумов независимых случайных
    величин в случае правильно меняющихся хвостов~[5, \S~8.8; 6, \S~3.3.1]:
    $$
    \lim\limits_{n\to\infty}{\bf P}\left(\fr{M_0(n)}{u(n)}\le x\right)=\Phi_a(x)\,,\quad x>0\,,
    $$
    что в сочетании с~(\ref{MP}) дает
    \begin{equation*}
%    \label{PZ}
    \lim\limits_{n\to\infty}{\bf P}\left(\fr{M(n)}{u(n)}\le x\right)=\Phi_a(x)\,,\quad x>0\,.
    \end{equation*}
    %В этом и заключается польза соотношения (\ref{MP}).

\smallskip

\noindent
\textbf{Теорема 1.} \textit{Соотношение}~(\ref{MP}) \textit{выполняется при
    $a\hm<\beta-2$, если $2\hm<\beta\hm<3$, и при $a\hm<(\beta-1)/2$,
    если $\beta\hm\ge 3$.}

    
    \smallskip
    
    \noindent
    Д\,о\,к\,а\,з\,а\,т\,е\,л\,ь\,с\,т\,в\,о\ теоремы будет приведено в разд.~4.

    
    \section{Общая схема максимумов сумм}

    Напомним введенную в~\cite{Leb1} схему (немного изменив обозначения).
    Пусть заданы случайный процесс $\Upsilon(t)$, $t\hm\in T$,
    значениями которого являются конечные
    классы конечных подмножеств ${\bf N}$,
    и семейство $\Xi\hm=\{\xi_{i,t},i\in {\bf N},t\in T\}$
    неотрицательных случайных величин, независимых и одинаково распределенных
    при любом фиксированном значении параметра $t\hm\in T$.
    Полагаем, что $\Upsilon$ и~$\Xi$ независимы.

    Для любых $A\subset {\bf N}$, $t\hm\in T$
    обозначим максимум набора случайных величин $\{\xi_{i,t}$, $i\hm\in A\}$
    через $M_t(A)$, $r$-й максимум (т.\,е.\ чис\-ло, стоящее $r$-м с конца в
    вариационном ряду)~--- через $M_t^{(r)}$, сумму~--- через $S_t(A)$.
    Пусть 
    $$
    U(t)\hm=\bigcup\limits_{A\in\Upsilon(t)}A\,.
    $$

    Введем случайные процессы, порожденные $\Upsilon$ и~$\Xi$:
    \begin{gather*}
    Z(t)=\!\!\sup\limits_{A\in\Upsilon(t)}\!\!S_t(A)\,;\enskip
    \kappa(t)=\!\!\sup\limits_{A\in\Upsilon(t)}\!\!|A|\,;
    \enskip\nu(t)=\left|U(t)\right|\,;\\
    \mu_1(t)=M_t(U(t))\,;\quad \mu_r(t)=M_t^{(r)}(U(t))\,,
 \end{gather*}
    где через $|A|$ обозначен размер (число элементов) множества~$A$.
    Через $|\Upsilon(t)|$ обозначим число различных множеств $A\hm\in\Upsilon(t)$.

    Предполагается, что $\nu(t)<\infty$ почти наверное (п.\,н.)\
    при всех $t\in T$, откуда следует конечность п.\,н.\
    всех процессов, введенных выше.

    Рассмотрим предельное поведение $Z(t)$ при $t\hm\to\infty$.

    Пусть существует случайный процесс $\rho(t)$ со
    значениями в ${\bf Z}_+$, измеримый относительно~$\Upsilon$ и такой, что
    $\rho(t)\hm\ge 1$ при $\nu(t)\hm\ge 1$, $\rho(t)\hm\le\nu(t)$ п.\,н. при всех $t\hm\in T$.

    Обозначим через $\pi(t)$ вероятность того, что для
    множества~$B$, равновероятно выбранного среди всех подмножеств $U(t)$,
    состоящих из $\rho(t)$ элементов,
    имеет место $\sup\limits_{A\in\Upsilon(t)}|A\cap B|\hm>1$.

\smallskip

\noindent
\textbf{Теорема I.} \textit{Пусть выполнены условия
    \begin{align}
    \label{us1}
    (\kappa(t)-1)\fr{\mu_{\rho(t)}(t)}{\mu_1(t)}\stackrel{P}{\to} 0,\quad t\to\infty\,;
\\
\label{us2}
    \pi(t)\to 0\,,\quad t\to\infty\,,
    \end{align}
    тогда
    \begin{equation}
    \label{res1}
    \fr{Z(t)}{\mu_1(t)}\stackrel{P}{\to} 1\,,\quad t\to\infty\,.
    \end{equation}
    }

    Доказано также следующее свойство порядковых статистик
    в случае правильно меняющихся хвостов.
    Пусть $X_n$, $n\ge 1$, независимы и имеют распределение~$F$
    с правильно меняющимся хвос\-том ${\bar F}(x)\hm\sim x^{-a}L(x)$,
    $x\hm\to\infty$, $a\hm>0$.
    Обозначим максимум
    $X_1,\dots, X_n$ через ${\tilde X}_n$ и $r$-й максимум через
    ${\tilde X}^{(r)}_n$.

\smallskip

\noindent
\textbf{Следствие II.}  \textit{Пусть $r_n\hm\sim n^\gamma$, $n\hm\to\infty$, $0\hm<\gamma\hm<1$ и
    $0\hm<\delta\hm<\gamma/a$, тогда
    $$
    \fr{n^\delta {\tilde X}^{(r_n)}_n}{{\tilde X}_n}\stackrel{P}{\to} 0\,,\quad
    n\to\infty\,.
    $$}
    
\vspace*{-12pt}

    
\section{Приложение к случайным графам}

    Адаптируем общую схему к изучению случайных графов.
    Рассмотрим процесс~$\Upsilon$ с дискретным временем, соответствующим
    числу узлов~$n$.
    Случайные величины $\xi_{i,n}$, $1\hm\le i\hm\le n$, описывают
    информационные активности узлов. Обозначим через
    $A_i$ множество из индекса~$i$ и индексов входящих соседей
    $i$-го узла, тогда набор множеств~$A$ получается из набора
    $A_1,\dots, A_n$ удалением повторов (если они есть). Имеем
    $|A_i|\hm=D_i\hm+1$ и $\kappa(n)\hm=\max\limits_{1\le i\le n}D_i\hm+1$.
    Очевидно, $|\Upsilon(n)|\hm\le n$ и $\nu(n)\hm=n$.
    Последовательность $\rho(n)$ далее будем полагать детерминированной.
    Кроме того, в используемых обозначениях $M(n)\hm=Z(n)$, $M_0(n)\hm=\mu_1(n)$ 
    и~(\ref{res1}) эквивалентно~(\ref{MP}).

    Обозначим
    $$
    Q(n,m)=n{\bf M}\left((D+1)D{\bf I}\{D\le m-1\}\right)\,,
    $$
    где $D\stackrel{d}{=}D_1$.

\medskip

\noindent
\textbf{Лемма 1.} \textit{При $n>2$ и $\rho(n)<n$ верно неравенство}
    $$
    \pi(n)\le\fr{\rho(n)(\rho(n)-1)}{2(n-1)(n-2)}\,Q(n,m)+
    {\bf P}(\kappa(n)>m)\,.
    $$

    \smallskip
    
    \noindent
    Д\,о\,к\,а\,з\,а\,т\,е\,л\,ь\,с\,т\,в\,о\,.\
    Событие $\{\sup\limits_{A\in\Upsilon(t)}|A\cap B|\hm>1\}$ представляет собой
    объединение событий $\{|A_i\cap B|\hm>1\}$, $1\hm\le i\hm\le n$.
    Пусть для простоты $B\hm=\{1,2,\dots, \rho(n)\}$ (в противном
    случае можно перенумеровать~$A_i$). Зафиксируем входящие степени
    вершин $d_1,\dots, d_n$. Тогда для $1\hm\le i\hm\le \rho(n)$ один элемент
    множества~$B$ заведомо принадлежит~$A_i$ (а именно, индекс~$i$),
    а любой другой принадлежит с вероятностью $d_i/(n-1)$. Для
    $\rho(n)+1\hm\le i\hm\le n$ любая пара индексов из~$B$ принадлежит~$A_i$
    с ве\-ро\-ят\-ностью $d_i(d_i-1)/((n-1)(n-2))$, а всего таких пар
    $\rho(n)(\rho(n)-1)/2$. Суммируя вероятности, получаем оценку сверху:
    $$
    \fr{\rho(n)-1}{n-1}\sum\limits_{i=1}^{\rho(n)}d_i+
    \fr{\rho(n)(\rho(n)-1)}{2(n-1)(n-2)}\sum\limits_{i=\rho(n)+1}^nd_i(d_i-1)\,.
    $$
    Обозначим 
    \begin{align*}
    q_1&={\bf M}(D{\bf I}\{D\le m-1\})\,;\\
q_2&={\bf M}(D(D-1){\bf I}\{D\le m-1\})\,.
\end{align*}

Усредняя по наборам входящих
    степеней вершин в области $\kappa(n)\hm\le m$, получаем оценку сверху:
\begin{multline*}
    \fr{\rho(n)-1}{n-1}\,\rho(n)q_1+
    \fr{\rho(n)(\rho(n)-1)}{2(n-1)(n-2)}\left(n-\rho(n)\right)q_2\le{}\\
{}\le\fr{\rho(n)(\rho(n)-1)}{2(n-1)(n-2)}\,n\left(2q_1+q_2\right)={}\\
{}=
    \fr{\rho(n)(\rho(n)-1)}{2(n-1)(n-2)}\,Q(n,m)\,.
\end{multline*}
    Учитывая также вероятность события $\{\kappa(n)>m\}$, получаем
    утверждение леммы.

\medskip

\noindent
\textbf{Лемма 2.} \textit{Пусть выполнены следующие условия:}
\begin{enumerate}[(1)]
\item \textit{все $\xi_{i,n}$ имеют одинаковое распределение $F$ на~${\bf R}_+$
    с хвостом ${\bar F}(x)\sim x^{-a}L(x)$, $x\to\infty$, $a>0$};
\item
    $m\sim n^\delta$, $n\to\infty$, $\delta>0$;
\item
    $Q(n,m)=O(n^b)$, $n\to\infty$,   $0<b<2$;
\item
    $\kappa(n)=o_p(m)$, $n\to\infty$;
\item
    $a<(2-b)/(2\delta)$.
    \end{enumerate}
    \textit{Тогда верно}~(\ref{MP}).

    
    \medskip
    
    \noindent
    Д\,о\,к\,а\,з\,а\,т\,е\,л\,ь\,с\,т\,в\,о\,.\
    Можно выбрать $\gamma\hm\in (0,1)$
    так, чтобы выполнялось неравенство $a\delta\hm<\gamma\hm<(2-b)/2$.
    Положим $\rho(n)\hm=[n^\gamma]$, тогда
    по следствию~II получаем~(\ref{us1}), а по лемме~1~---~(\ref{us2}),
    так что условия теоремы~I выполняются и верно соотношение~(\ref{res1}), 
    эквивалентное~(\ref{MP}).

\smallskip

\noindent
Д\,о\,к\,а\,з\,а\,т\,е\,л\,ь\,с\,т\,в\,о\ теоремы~1.\
    Поскольку $p_k\hm\sim ck^{-\beta}$,
    $k\hm\to\infty$, то хвост распределения имеет
    асимптотику ${\bar F}_K(k)\hm\sim c_1k^{-(\beta-1)}$.
    Отсюда получаем
    $\kappa(n)\hm=O_p(n^{1/(\beta-1)})$, $n\hm\to\infty$,
    и, следовательно, $\kappa(n)\hm\sim o_p(m)$ при любом
    $\delta\hm=(1+\varepsilon)/(\beta-1)$, $\varepsilon\hm>0$. Имеем
    \begin{multline*}
    Q(n,m)=n\sum\limits_{k=1}^{m-1}(k+1)kp_k={}\\
    {}=
\begin{cases}
    O(n^{1+\delta(3-\beta)})\,,& 1<\beta<3\,;\\
    O(n\ln n)\,,& \beta=3\,;\\
    O(n)\,, & \beta>3\,.
    \end{cases}
    \end{multline*}
    При $2<\beta<3$ применяем лемму 2 с $b\hm=2\delta\hm>1\hm+\delta(3\hm-\beta)$ и,
    устремляя~$\varepsilon$ к нулю,
    получаем достаточное условие $a\hm<\beta\hm-2$ для выполнения~(\ref{MP}).
    При $\beta\hm\ge 3$ применяем лемму 2 с $b\hm=1\hm+\varepsilon$ и
    аналогично получаем достаточное условие $a\hm<(\beta-1)/2$.
    
    {\small\frenchspacing
{%\baselineskip=10.8pt
\addcontentsline{toc}{section}{Литература}
\begin{thebibliography}{9}


    \bibitem{Bar}
    \Au{Barab\'asi A., Albert R.} Emergence of scaling in random networks~//
    Science, 1999. Vol.~286. P.~509--512.

    \bibitem{Komp}
    \Au{Захаров П.} Народ-бло\-го\-но\-сец~// Компьютерра,
    2007. №\,27-28. C.~36--39. {\sf http://offline.computerra.ru/2007/ 695/327726}.

    \bibitem{Pow}
    \Au{Aiello W., Chung F., Lu~L.} A random graph model for power law
    graphs~// Experimental Math., 2001. Vol.~10. No.~1. P.~53--66.

    \bibitem{Reit}
    \Au{Reittu H., Norros~I.} On the power-law random graph model
    of massive data network~// Performance Evaluation, 2004. Vol.~55. P.~3--23.

    \bibitem{Fel}
    \Au{Феллер В.} Введение в теорию вероятностей и ее приложения. Т.~2.~---
    М.: Мир, 1984.

    \bibitem{EKM}
    \Au{Embrechts P., Kl$\ddot{\mbox{u}}$ppelberg C., Mikosh~T.} Modelling
    extremal events for insurance and finance.~--- Springer-Verlag, 2003.

    \bibitem{Leb2}
    \Au{Лебедев А.\,В.} Максимумы активности в случайных сетях в
    случае тяжелых хвостов~// Проблемы передачи информации, 2008. Т.~44.
    №\,2. С.~96--100.

    \bibitem{Leb1}
\Au{Лебедев А.\,В.} Общая схема максимумов сумм независимых
    случайных величин и ее приложения~// Математические заметки, 2005.
    Т.~77. №\,4. С.~544--550.

\label{end\stat}

    \bibitem{Sen}
\Au{Сенета Е.} Правильно меняющиеся функции.~--- М.: Наука, 1985.
 \end{thebibliography}
}
}


\end{multicols}       