\def\stat{agalarov}

\def\tit{АНАЛИТИЧЕСКАЯ МОДЕЛЬ РАСЧЕТА ЭФФЕКТИВНОСТИ 
ПЛАНА РАСПРЕДЕЛЕНИЯ ВЫЧИСЛИТЕЛЬНЫХ РЕСУРСОВ 
МНОГОПРОЦЕССОРНОЙ СИСТЕМЫ ПРИ~РЕШЕНИИ 
СПЕЦИАЛЬНОГО КЛАССА ЗАДАЧ}

\def\titkol{Аналитическая модель расчета эффективности 
плана распределения вычислительных ресурсов} %многопроцессорной системы при решении специального класса задач}

\def\autkol{М.\,Я.~Агаларов}
\def\aut{М.\,Я.~Агаларов$^1$}

\titel{\tit}{\aut}{\autkol}{\titkol}

%{\renewcommand{\thefootnote}{\fnsymbol{footnote}}\footnotetext[1]
%{Работа поддержана Российским фондом фундаментальных исследований
%(проекты 11-01-00515а и 11-07-00112а), а также Министерством
%образования и науки РФ в рамках ФЦП <<Научные и
%научно-педагогические кадры инновационной России на 2009--2013~годы>>.}}


\renewcommand{\thefootnote}{\arabic{footnote}}
\footnotetext[1]{Институт проблем информатики Российской академии наук, murad-agalarov@yandex.ru}

     
      \Abst{Рассмотрена модель многопроцессорной системы, предназначенная для 
решения задач, распараллеливаемых на слабо зависимые вычисления. В~качестве модели 
использована мультисервисная система массового обслуживания (СМО) с явными потерями, 
пуассоновскими входящими потоками и общими функциями распределения времени 
обслуживания заявок. Получены рекуррентные формулы для вычисления стационарного 
распределения вероятностей состояний и явные выражения для вероятностей отказов 
системы различным типам заявок. В~рамках данной модели предложен метод оценки 
пропускной способности многопроцессорной системы при заданном статическом плане 
распределения вычислительных ресурсов. }
      
      \KW{многопроцессорная система; система массового обслуживания; 
мультисервисная сис\-те\-ма; распределение вычислительных ресурсов}

 \vskip 14pt plus 9pt minus 6pt

      \thispagestyle{headings}

      \begin{multicols}{2}
      
            \label{st\stat}

     
\section{Введение}
     
     Одной из основных проблем, возникающих при проектировании и 
эксплуатации многопроцессорных сис\-тем, является выработка рекомендаций 
по рациональному использованию ресурсов вычислительной среды. При оценке 
эффективности проектных решений по распределению ресурсов и потоков 
задач в многопроцессорных сис\-те\-мах используются детерминированные и 
вероятностные модели. Использование той или иной модели зависит от 
архитектуры вычислительной сис\-те\-мы, которая, в свою очередь, зависит от 
типа решаемых задач. Детерминированные модели основаны на теории 
расписаний, в рамках которой разработано немало методов определения 
очередности выполнения задач и расчета планов распределения 
вычислительных ресурсов~[1--3]. Недостатком этих моделей является 
отсутствие учета случайных факторов, что вызвало широкое распространение 
вероятностных моделей, базирующихся на теории массового обслуживания 
(см., например, [4--10]). 
     
     Ниже рассматривается многопроцессорная сис\-те\-ма, используемая для 
решения задач, распараллеливаемых на независимые (слабо зависимые)\linebreak 
вы\-чис\-ле\-ния (аналогично таким задачам, как восстановление 
криптографированного текста с по\-мощью всех возможных ключей 
шифрования, \mbox{поиск} в больших объемах данных по ключевым признакам, поиск 
вариантов в пошаговых игровых программах, множественное вы\-чис\-ле\-ние 
целевой функции в процедурах многомерной нелинейной оптимизации 
и~т.\,д.)~\cite{3ag, 11ag, 12ag}. 
     
     Для исследования рассматриваемой многопроцессорной сис\-те\-мы 
предлагается модель, пред\-став\-лен\-ная в виде мультисервисной СМО 
с явными потерями и общими функциями 
распределения времени обслуживания заявок. Для предлагаемой \mbox{модели} 
получены формулы и алгоритмы расчета показателей производительности многопроцессорной системы. 

\begin{figure*} %fig1
\vspace*{1pt}
\begin{center}
\mbox{%
\epsfxsize=103.258mm
\epsfbox{aga-1.eps}
}
\end{center}
\vspace*{-9pt}
\Caption{Структурная схема ВК}
\end{figure*}

\section{Описание задачи}

   Рассматривается специализированный вычислительный комплекс (ВК), 
функциональная структура которого условно представлена четырьмя 
основными компонентами (рис.~1): поток задач, центр управления (ЦУ), 
линии передачи данных и вычислительные ресурсы. Вычислительные 
ресурсы~--- множество единиц ресурсов с заданными параметрами 
производительности и надежности. На ВК поступает поток задач, требующих 
выполнения на вычислительных ресурсах за указанный заранее для каждой 
задачи интервал времени (контрольное время). Задача представляет собой 
совокупность элементарных заданий, каждое из которых может независимо 
выполняться на любом ресурсе. Каж\-дая задача может содержать одно 
элементарное задание, имеющее некоторый отличительный признак 
(называемый в дальнейшем ключом). Целью выполнения задачи является поиск 
ключа. Центр управ\-ле\-ния принимает решение о допуске и выполняет подготовку 
поступившей задачи к вычислениям, управляет коммуникационными 
ресурсами и процессом выполнения задачи на вычислительных ресурсах.


   
   Обработка и выполнение задачи в ВК происходит следующим образом. Центр управ\-ле\-ния
для каждой задачи с учетом контрольного для нее времени вычисляет по 
заданному алгоритму необходимое число ресурсов, которые до завершения 
выполнения задачи доступны только ей. Если в ВК отсутствует требуемое 
число свободных ресурсов, задача не принимается и теряется (или повторяется 
позже). В~ЦУ принятая задача разбивается на блоки из элементарных заданий, 
называемые в дальнейшем пакетами. Пакеты направляются на ресурсы 
группами, называемыми посылками, в каждой из которых (за исключением, 
возможно, последней) количество пакетов совпадает с количеством 
выделенных для задачи ресурсов.

Посылка занимает и освобождает все выделенные ресурсы одновременно, при 
этом один пакет занимает один ресурс. Выполнение пакета считается 
успешным, если во время его выполнения не произошел сбой ресурса, иначе 
выполнение пакета считается неуспешным, и в составе какой-нибудь новой 
посылки задача заново направляется из ЦУ на выполнение. Выполнение 
посылки считается завершенным, если выполнены (успешно или неуспешно) 
все составляющие ее пакеты. Очередную посылку ЦУ направляет на ресурсы 
только после завершения выполнения предыдущей посылки. Считается, что 
ключ с равной вероятностью находится в одном из пакетов задачи. Выполнение 
задачи завершается либо в момент достижения контрольного времени, либо в 
момент завершения выполнения очередной посылки, если обнаружен пакет с 
ключом, либо после успешного выполнения всех пакетов задачи. Задача 
покидает сис\-те\-му в момент завершения выполнения, освобождая одновременно 
все занятые ею ресурсы.
     
     Требуется вычислить пропускную способность (интенсивность 
выходного потока задач) ВК и загруженность ресурсов при заданном плане 
распределения ресурсов. 

\section{Модель вычислительного комплекса}
     
     В качестве модели ВК рассматривается мультисервисная СМО с явными 
потерями и произвольным временем обслуживания. В~дальнейшем, \mbox{чтобы} 
описать работу модели в терминах теории массового обслуживания, вместо 
термина <<задача>> будем использовать термин <<заявка>>. На вход сис\-те\-мы, 
состоящей из $R$ одинаковых общедоступных независимых каналов, поступает 
пуассоновский поток заявок интенсивности~$\lambda$, имеющих случай\-ную 
длину (случайное число элементарных заданий), распределенную равномерно 
на мно\-жест\-ве~$L$. Без потери общности будем считать, что $L$~--- конечное 
множество положительных чисел, т.\,е.\ $\vert L \vert <\infty$. Пусть $L_j$~--- 
полуинтервалы такие, что $L_j\bigcap L_i=\varnothing$ при $j\not=i$, $j,i=1, \ldots 
,M$, $\bigcap\limits_{j=1}^M L_j=L$, $M\leq \vert L\vert$. Поступающие заявки 
образуют $M$ потоков по следующему правилу: заявка принадлежит $j$-му 
потоку (является $j$-заявкой, заявкой $j$-го типа), если ее длина попадает в 
полуинтервал~$L_j$, $j=1, \ldots ,M$. 
     
     Процесс обслуживания заявки состоит из двух этапов и происходит 
следующим образом. На первом этапе при наличии в сис\-те\-ме необходимого 
чис\-ла свободных каналов поступившая $j$-заявка разбивается в сис\-те\-ме на 
пакеты длиной $lp_j$. Время разбиения $j$-заявки на пакеты считается 
независимой случайной величиной с произвольной функцией распределения 
$GZ_j(t)$ со средним значением $\overline{tz}_j<t_{\mathrm{contr}z,j}$. Считается, что 
обслуживание заявки на первом этапе происходит как в СМО типа $M/G/\infty$.
     
     Необходимое заявкам число каналов определяется по следующему 
правилу. Для каждого $j$-потока вычисляется величина 
     $$
     m_j=\max \left\{\! \mbox{целая часть} \left[ \! \fr{lz_j}{v(t_{\mathrm{contr}z,j}-
\overline{tz}_j)}\!\right] +d_j,1\!\right\},
     $$ 
     где $m_j\leq R$; $lz_j$~--- максимальная длина $j$-за\-яв\-ки; $v$~--- 
пропускная способность канала (максимальная скорость передачи каналом 
элементарных заданий); $t_{\mathrm{contr}z,j}>0$~--- контрольное время обслуживания 
$j$-за\-яв\-ки; $d_j\geq 0$~--- заданное целое чис\-ло, $j=1, \ldots , M$. 
Поступившая $j$-за\-яв\-ка, если в сис\-те\-ме есть не меньше $m_j$ свободных 
каналов, занимает $m_j$ каналов одновременно, иначе она теряется. Набор 
$\overline{m}=\{ m_j,\ j=1, \ldots , M\}$ будем называть планом распределения 
ресурсов. 

На втором этапе обслуживания формируются наборы из $m_j$ пакетов 
(посылка), которые последовательно передаются на выделенные $m_j$ каналов. 
Каждый из этих каналов принимает на обслуживание по одному поступающему 
пакету. Один из пакетов заявки имеет некоторый признак, называемый ключом, 
при обнаружении которого обслуживание заявки заканчивается. Считается, что 
ключ может обнаружиться в любом из пакетов заявки с равной вероятностью. 
Каналы сис\-те\-мы могут подвергаться самовосстанавливающимся коротким 
отказам (сбоям), которые вызывают в обслуживаемых пакетах ошибки. 
Вероятность появления ошибки при обслуживании пакета $j$-за\-яв\-ки~--- 
заданная величина, равная~$\alpha_j$. Если пакет обслужен без ошибок 
(успешное обслуживание), то он удаляется из сис\-те\-мы, иначе обслуживание 
пакета производится заново в составе какой-нибудь последующей посылки 
этой же заявки. Пустой пакет всегда обслуживается успешно.

Очередная посылка $j$-заявки направляется на каналы только после окончания 
обслуживания предыду\-щей посылки и при условии, что не завершилось 
обслуживание заявки. Посылка считается обслуженной, если обслужены все 
составляющие ее пакеты. Отметим, что в последней посылке заявки часть 
пакетов могут быть частично или полностью пустыми. Обслуживание 
$j$-за\-яв\-ки завершается либо в момент превышения контрольного времени ее 
выполнения $t_{\mathrm{contr}z,j}$, либо в момент окончания обслуживания очередной 
посылки, если обнаружен ключ в одном из успешно обслуженных пакетов 
(пакеты с ошибками не проверяются на наличие ключа), либо успешно 
обслужены все пакеты задачи. Будем считать, что время обслуживания одной 
посылки $j$-заявки является независимой случайной величиной~$tp_j$ с 
произвольной функцией распределения~$G_j(t)$ с конечным средним 
значением~$\overline{tp}_j$. 

Обслуживание $j$-заявки называется успешным, если длительность 
обслуживания не превысила $t_{\mathrm{contr}z,j}$. Заявка после обслуживания 
покидает сис\-те\-му, освобождая одновременно все занятые ею каналы.

Как видно из описания модели, она относится к классу мультисервисных СМО 
с явными потерями, где для обслуживания различных потоков заявок 
выделяется различное число каналов.
   
   Ставится задача расчета стационарных характеристик описанной модели: 
распределение вероятностей числа заявок в сис\-те\-ме, загруженность каналов, 
интенсивность выходного потока успешно обслуженных заявок (обслуженных 
за контрольное время).
   
\section{Решение задачи}

   Найдем вид функции распределения времени обслуживания (пребывания в 
сис\-те\-ме) заявки \mbox{$j$-го} типа ($j\hm=1, \ldots , M$) в мультисервисной СМО, 
описанной в предыдущем разделе в качестве модели~ВК.

Рассмотрим сначала случай $t_{\mathrm{contr}z,j}=\infty$, $j\hm=1, \ldots , M$. Фиксируем 
некоторый тип заявок~$j$. Пусть заявка $j$-го типа обслуживается на 
$m_j=m\hm\geq 1$ каналах, состоит из $n$~пакетов и $\alpha_j\hm=\alpha$. Назовем 
шагом процесса обслуживания заявки отрезок времени обслуживания одной 
посылки заявки, состоянием заявки~--- число пакетов, оставшихся после 
очередного шага. Обозначим через $X_k$ состояние заявки после $k$-го шага, 
$Y$~--- число шагов, выполненных за время обслуживания заявки. Так как 
процесс обслуживания отдельной посылки не зависит от процессов 
обслуживания других посылок заявки, то значение $X_k$ зависит только от 
$X_{k-1}$, и поэтому процесс обслуживания заявки $X=\{ X_k,\ k\hm=1,2,\ldots 
\}$ является однородной (см. ниже матрицу вероятностей перехода) 
марковской цепью с поглощающим состоянием~<<0>>. Заметим, что для 
рассматриваемого случая ($t_{\mathrm{contr}z,j}\hm=\infty$) цепь находится в 
состоянии~<<0>>, если успешно обслужены пакеты в последней посылке 
заявки или при обслуживании очередной посылки обнаружен ключ.
    
    Переход процесса из состояния~$i$ в состояние $j>0$ за один шаг 
равносилен тому, что из $m$~пакетов успешно завершили обслуживание $(i-j)$ 
пакетов, $0\leq (i-j)\leq m$, и в них не обнаружен ключ. Переход процесса~$X$ 
из состояния~$i$ в состояние~<<0>> за один шаг равносилен тому, что из $m$ 
пакетов успешно завершили обслуживание $(i-j)$ пакетов, $0\leq (i-j)\leq m$, и 
в одном из них обнаружен ключ. Событие, состоящее в том, что из 
$m$~пакетов успешно завершили обслуживание $(i-j)$ пакетов, $0\leq (i-j)\leq 
m$, в данном случае то же самое, что в схеме Бернулли с вероятностью успеха 
$\alpha$ в $m$~испытаниях произошло $(i-j)$ успехов. Заметим, что переход 
процесса~$i$ в состояние~$j$ при $i<j$~--- невозможное событие. Тогда 
вероятность того, что на одном шаге из $m$~пакетов успешно завершили 
обслуживание $(i-j)$ пакетов, $0\leq i-j\leq m$, равна
    \begin{equation*}
    b_{ij} =\begin{cases}
    \begin{pmatrix}
    m\\ i-j
    \end{pmatrix} (1-\alpha)^{i-j} \alpha^{m-(i-j)}\,, &\\
    & \hspace*{-35mm}\mbox{\ если\ }
    m \leq i\leq n\,,\ 0\leq i-j \leq m\,;\\
    0 & \hspace*{-35mm}\mbox{\ в остальных случаях\,.}
    \end{cases}
    \end{equation*}
Матрица вероятностей $B=(b_{ij})$ имеет вид:
\end{multicols}

\hrule

\vspace*{9pt}

\noindent
$$
B=\left\{
\begin{array}{ccccccccccccccc} %\begin{pmatrix}
1        & 0        & \cdots &\cdots &\cdots &\cdots &\cdots &\cdots &\cdots    &\cdots &\cdots &\cdots &\cdots &\cdots &0\\
b_{10}   & b_{11}   & 0      &\cdots &\cdots &\cdots &\cdots &\cdots &\cdots    &\cdots &\cdots &\cdots &\cdots &\cdots &0\\
b_{20}   & b_{21}   & b_{22} & 0     &\cdots &\cdots &\cdots &\cdots &\cdots    &\cdots &\cdots &\cdots &\cdots &\cdots &0\\
\vdots   &\vdots    &\vdots  &\ddots &\ddots &\ddots &\ddots &\ddots &\ddots    &\ddots &\ddots &\ddots &\ddots &\ddots &\vdots \\
b_{m-10} & b_{m-11} & \cdots & \cdots&\cdots &\cdots &\cdots &\cdots &b_{m-1m-1}&   0   &\cdots &\cdots &\cdots &\cdots &0\\
a_0      & a_1      & \cdots & \cdots&\cdots &\cdots &\cdots &\cdots &a_m       & 0     &\cdots &\cdots &\cdots &\cdots &0\\
0        & a_0      & a_1    & \cdots&\cdots &\cdots &\cdots &\cdots &\cdots    &a_m    & 0     &\cdots &\cdots &\cdots &0\\
0        & 0        & a_0    & a_1   &\cdots &\cdots &\cdots &\cdots &\cdots    &\cdots &   a_m & 0     &\cdots &\cdots &0\\
\vdots   & \vdots   & \ddots &\ddots &\ddots &\ddots &\ddots &\ddots &\ddots    &\ddots &\ddots &\ddots &\ddots &\ddots & \vdots\\
0        & \cdots   & \cdots & \cdots&  a_0  & a_1   &\cdots &\cdots & \cdots   &\cdots &\cdots & \cdots&\cdots & a_m   & 0\\
0        & \cdots   & \cdots & \cdots&\cdots & a_0   & a_1   &\cdots & \cdots   &\cdots &\cdots & \cdots&\cdots & \cdots& a_m
\end{array}
\right \}
%\end{pmatrix}
\,,
$$

\vspace*{3pt}

\hrule

\begin{multicols}{2}

\noindent
где 

\vspace*{-5pt}

\noindent
\begin{align*}
a_j&=b_{mj} =\begin{pmatrix}
m \\ m-j \end{pmatrix} (1-\alpha)^{m-j} \alpha^j\,,\ 0\leq j\leq m\,;\\
b_{ij} &= a_{m-(i-j)}\,,\ m\leq i\leq n\,,\ 0\leq i-j\leq m\,.
\end{align*}

   Так как ключ может находиться с равной вероятностью в любом успешно 
обслуженном пакете, то матрица переходов $P=(p_{ji})$ марковской цепи~$X$ 
имеет вид:
   \begin{equation*}
   p_{ji}=\begin{cases}
   \displaystyle\sum\limits^j_{i=\max\{0,\,j-m\}} b_{ji}\fr{j-i}{j} & \mbox{\ при \ } i=0\,;\\
   b_{ji}\fr{i}{j} &  \hspace*{-20mm}\mbox{\ при \ } i= \max\{1,\,j-m\}, \ldots , j
   \end{cases}
   \end{equation*}
при $j=1, \ldots , n$ и $p_{00}=1$, где $p_{ji}$~--- вероятность перехода цепи из 
состояния~$j$ в состояние~$i$. Следовательно, закон распределения случайной 
величины~$Y$ имеет вид:
$$
P\{Y=k\} = \begin{cases}
f(n;m;k) & \mbox{\ при \ } k=1,2, \ldots ; \\
0 &\mbox{\ при \ } k=0\,;\\
0 & \mbox{\ при \ } k=\infty\,,
\end{cases}
$$
где $f(n;m;k)=\sum\limits_{j=1}^n P(j;n;m;k-1)p_{j0}$~--- вероятность того, что 
впервые цепь попадет в по\-гло\-ща\-ющее состояние (либо все пакеты 
обслужены, либо обнаружен ключ) на $k$-м шаге, 
$P(i;n;m;k)\hm=\sum\limits_{j=i}^{i+m} P(j;n;m;k-1)p_{ji}$~--- вероятность 
того, что через $k$~шагов цепь окажется в состоянии $i\geq 1$, 
$P(i;n;m;k)=0$, если $i<n-km$ и $n-km>0$;
$$
P(i;n;m;0) =\begin{cases} 
1\,, & \mbox{\ если\ } i=n\,;\\
0\,, & \mbox{\ если\ } i<n\,.
\end{cases}
$$
   
   Обозначим через $N_j$ число различных значений $n_i$ величины~$n$ для 
$j$-заявки при различных $lz_j\in L_j$, $q_{ji}$~--- вероятность того, что 
$n=n_i$, $i=1, \ldots , N_j$. Заметим, что для равномерно распределенной 
на~$L_j$ длины $j$-заявки при фиксированных~$L_j$ и~$lp_j$ 
величины~$q_{ji}$, $i=1, \ldots , N_j$, легко вычисляются. Тогда функция 
распределения времени обслуживания \mbox{$j$-за}\-яв\-ки при $t_{\mathrm{contr}z,j}=\infty$ для 
случая непрерывных случайных величин~$tz_j$ и~$tp_j$ имеет вид:
   \begin{multline}
   B_j(t) ={}\\
   {}= GZ_j * \sum\limits_{i=1}^{N_j} q_{ji} \sum\limits_{k=1}^\infty 
f(n_i;m_j;k) \overbrace{G_j*\cdots*G_j}^k(t)\,,\\
 0\leq t\,,
   \label{e1ag}
   \end{multline}
где *~--- знак свертки функций; $\overbrace{ G_j*\ldots*G_j }^k$~--- 
$k$-крат\-ная свертка функции~$G_j$;
$$
\alpha =\alpha_j\,;\ n_i =
\begin{cases}
\fr{lz_i}{lp_j}\,, & \hspace*{-26mm}\mbox{\ если\ } \fr{lz_i}{lp_j}\mbox{~--- целое число}\,;\\
\mbox{целая часть\ } \left[\fr{lz_i}{lp_j}\right]+1 & \mbox{\ иначе}\,.
\end{cases}
$$
Среднее время обслуживания $j$-заявки в этом случае согласно 
формуле~(\ref{e1ag}) будет равно
\begin{equation}
\overline{to}_j =\overline{tz}_j+\overline{tp}_j \sum\limits_{i=1}^{N_j} q_{ji} 
\sum\limits_{k=1}^\infty kf(n_i;m_j;k)\,.
\label{e2ag}
\end{equation}
   
Вернемся к случаю $t_{\mathrm{contr}z,j}<\infty$. Тогда функция распределения времени 
обслуживания $j$-заявки имеет вид:
$$
\tilde{B}_j(t) =\begin{cases}
B_j(t)\,, & \mbox{\ если\ } t<t_{\mathrm{contr}z,j}\,;\\
1\,, & \mbox{\ если\ } t\geq t_{\mathrm{contr}z,j}\,.
\end{cases}
$$

Для среднего значения времени обслуживания имеем:
$$
\overline{to}_j = \int\limits_0^{t_{\mathrm{contr}z,j}} t\,dB_j(t) +\left[ 1-
B_j(t_{\mathrm{contr}z,j})\right] t_{\mathrm{contr}z,j}\,.
$$

Для случая детерминированных величин $tz_j$ и $tp_j$ имеем:
$$
\tilde{B}_j(t) =
\begin{cases}
\sum\limits_{i=1}^{N_j} q_{ji} \sum\limits_{k=1}^{K_j} f(n_i;m_j;k) \chi(t-tz_j-
ktp_j)\,, & \\
& \hspace*{-45mm}\mbox{если}\ t<t_{\mathrm{contr}z,j}\,;\\ 
1\,, &\ \hspace*{-34mm}\mbox{если}\  t\geq t_{\mathrm{contr}z,j}\,;
\end{cases}
$$

\vspace*{-6pt}

\noindent
\begin{multline*}
\overline{to}_j = tz_j+tp_j\sum\limits_{i=1}^{N_j} q_{ji} \sum\limits_{k=1}^{K_j} 
kf(n_i;m_j;k)+{}\\
{}+t_{\mathrm{contr}z,j} \left[ 1-\sum\limits_{i=1}^{N_j} q_{ji}\sum\limits_{k=1}^{K_j} 
f(n_i;m_j;k)\right]\,,
\end{multline*}
где $K_j$~--- целое число такое, что выполняются неравенства:
$$
K_j tp_j \hm<  t_{\mathrm{contr}z,j}-tz_j\leq (K_j+1)tp_j\,,
$$
где $\chi(t)$~--- функция Хевисайда:
$$
\chi(t) = \begin{cases}
0\,, & \ t<0\,;\\
1\,, & \ t\geq 0\,.
\end{cases}
$$
   
   Для мультисервисной СМО в случае экспоненциальных времен 
обслуживания заявок известна мультипликативная формула расчета 
стационарных вероятностей числа заявок в сис\-те\-ме (вывод формул см., 
например, в~[13--16]), которая согласно утверж\-де\-ни\-ям, 
приведенным в некоторых работах (см., например,~[13, 15--17]), 
инвариантна к виду функций распределения времени обслуживания. 
Обозначим через $\overline{k} =\{ k_1, \ldots , k_M\}$ вектор числа заявок в 
сис\-те\-ме, где $k_j$~--- число \mbox{$j$-за}\-явок в сис\-те\-ме; $P(\overline{k})$~---
стационарная вероятность того, что вектор состояния сис\-те\-мы 
равен~$\overline{k}$; $\overline{\Omega}_j$~--- множество всевозможных 
значений вектора~$\overline{k}$, при которых $j$-заявка получает отказ (т.\,е.\ 
$\overline{\Omega}_j = \{ \overline{k}:\ R-m_j< \sum\limits_{i=1}^M k_i m_i\leq 
R\}$).
   
   В обозначениях, приведенных выше, для стационарных вероятностей 
состояний сис\-те\-мы справедлива формула:
   $$ 
   p(\overline{k}) =p(\overline{0}) \prod\limits_{j=1}^M \fr{\rho_j^{k_j}}{k_j!}\,,
   $$
где $\rho_j=\lambda_j (\overline{to}_j-tz_j)$, вероятность $p(\overline{0})$ 
вычисляется из условия нормировки и равна
$$
p(\overline{0}) =\left[ \sum\limits_{\overline{k}\in\Omega} \prod\limits_{j=1}^M 
\fr{\rho_j^{k_j}}{k_j!}\right]^{-1}\,.
$$
Согласно последней формуле вероятность того, что $j$-за\-яв\-ка получит 
отказ, равна:
$$
\pi_j=\sum\limits_{\overline{k} \in \overline{\Omega}_j} p(\overline{k})\,.
$$
Получаем, что интенсивность выходного потока своевременно обслуженных 
заявок равна:
\begin{equation}
\Lambda_{\mathrm{вых}} =\sum\limits_{j=1}^M \lambda_j\left(1-\pi_j\right) 
B_j(t_{\mathrm{contr}z,j})\,.
\label{e3ag}
\end{equation}
    
Коэффициент использования сис\-те\-мы (загруженность сис\-те\-мы) равен:
\begin{equation}
U=\sum\limits_{d=1}^R \fr{dq(d)}{R}\,,
\label{e4ag}
\end{equation}
где $q(d) =\sum\limits_{\overline{k}\in D_d} p(\overline{k})$, $D_d=\left\{ 
\overline{k}\in \Omega:\ \sum\limits_{j=1}^M k_j m_j =d\right\}$.
Вероятности $p(\overline{k})$ и~$q(d)$ вычисляются по рекуррентным 
формулам (см., например,~\cite{13ag, 14ag, 16ag}).

\begin{table*}[b]\small
\begin{center}
\Caption{Интенсивность выходного потока своевременно выполненных задач
}
\vspace*{2ex}

\begin{tabular}{|c|c|c|c|c|c|c|c|}
\hline
$k$ & \multicolumn{7}{c|}{$lp_j$}\\
\cline{2-8}
 &1&2&4&6&8&10&15\\
 \hline
1&0,376290&0,365933&0,338158&0,337962&0,303518&0,297001&0,246217\\
2&0,498492&0,497666&0,494870&0,490414&0,484638&0,472359&0,451572\\
3&0,482897&0,476613&0,457543&0,437682&0,412980&0,408341&0,326782\\
4&0,454405&0,440319&0,410806&0,377289&0,358364&0,318597&0,247202\\
5&0,421171&0,402171&0,355661&0,336658&0,289217&0,254539&0,196318\\
\hline
\end{tabular}
\end{center}
%\vspace*{-36pt}
%\end{table}
\renewcommand{\figurename}{\protect\bf Таблица}
\renewcommand{\tablename}{\protect\bf Рис.}
%\begin{figure}[b] %fig2
\vspace*{12pt}
\begin{center}
\mbox{%
\epsfxsize=80.012mm
\epsfbox{aga-2.eps}
}
\end{center}
\vspace*{-9pt}
\Caption{Зависимости интенсивности суммарного потока своевременно 
выполненных задач от плана распределения ресурсов: цифрами обозначены ряды;
$\Lambda$~--- интенсивность 
выходного потока; $m_1$, $m_2$, $m_3$, $m_4$, $m_5$~--- число ресурсов, 
выделенных соответственно потокам 1-, 2-, 3-, 4-, 5-го типа, $m_1=k$, $m_2=2k$, 
$m_3=3k$, $m_4=4k$, $m_5=5k$; $k$~--- коэффициент увеличения числа 
ресурсов}
\end{table*}


\renewcommand{\figurename}{\protect\bf Рис.}
\renewcommand{\tablename}{\protect\bf Таблица}



\section{Вычислительный эксперимент}

   В качестве примера использования модели были проведены расчеты 
значений выходных потоков для нескольких планов распределения ресурсов 
между потоками задач различной длины. Суммарный поток задач является 
пуассоновским с интенсивностью~0,5, распределение длины задачи считается 
равномерным на множестве значений $\{30, 60, 90, 120, 150\}$. Задачи разбиты 
на 5~типов в\linebreak соответствии с длинами 30, 60, 90, 120 и~150, тре\-буемое время 
выполнения (контрольное время)\linebreak задачи любого типа равно~30. Согласно 
<<справедливому>> плану минимальное необходимое чис\-ло ресурсов $r_j\hm=j$ 
для $j$-задачи, $j=1, \ldots , 5$. Другие исходные данные: $v=1$; $R=100$; 
$\alpha_j =1-(1-0{,}001)^{lp_j}$; $tz_j=0{,}1$; $tp_j =0{,}1 * j * (lp_j+1)$, $j=1, 
\ldots , 5$. В~данном примере задача не снимается с обслуживания из-за 
превышения контрольного времени и среднее время обслуживания задачи 
считается по формуле~(\ref{e2ag}). 



В табл.~1 и на рис.~2 приведены зависимости интенсивности суммарного 
потока своевременно выполненных задач от плана распределения ресурсов 
$m_j=kr_j$, $j=1, \ldots ,5$ при различных длинах пакетов. Задача считается 
своевременно выполненной, если время обслуживания не превысило 
контрольное время. Ряды~1--7 соответствуют длинам пакетов $lp_j=1$, 2, 4, 6, 
8, 10, 15, $j=1, \ldots ,5$.

\section{Заключение}
   
   Как следует из описания данной модели, распределение каналов происходит 
в ЦУ после поступления задачи с учетом ее длины, заданного контрольного 
времени и производительности каналов, и при этом производительность 
ресурсов считается одинаковой. В~действительности ВК может состоять из 
ресурсов, которые имеют различные производительность и вероятность сбоя. 
Очевидно, для более эффективного использования ВК распределение ресурсов 
следует производить с учетом их производительности и надежности. Отметим, 
что предложенная модель ВК может быть использована для оценки и 
оптимизации эффективности следующего плана распределения ресурсов, 
учитывающего производительности ресурсов.
Согласно указанному плану 
множество ресурсов разделено на подмножества так, что ресурсы в одном 
подмножестве имеют одинаковую производительность. Задано также правило 
распределения поступающих задач по указанным подмножествам и известны 
интенсивности потоков задач, поступающих на подмножества ресурсов. Для 
каждого подмножества распределение ресурсов происходит по такому же 
плану, что и в разработанной выше модели. Тогда рассчитав интенсивность 
выходного потока задач для каждого подмножества ресурсов по 
формуле~(\ref{e3ag}) и взяв их сумму, получим интенсивность выходного 
потока для всего множества ресурсов. По формуле~(\ref{e4ag}) вычисляется 
загруженность ресурсов в каждом подмножестве.
   
   Модель легко обобщается на случай, когда задача содержит несколько 
ключевых элементарных заданий, которые равновероятно обнаруживаются в 
любом пакете и задача снимается с выполнения при обнаружении любого из 
этих ключей. Модель также легко обобщается на случай, когда задача содержит 
ключи с заданной вероятностью.
   
   Разработанная модель ВК может быть использована для оценки 
эффективности и оптимизации плана распределения вычислительных ресурсов 
на определенном выше множестве статических планов типа~$\overline{m}$ в 
смысле максимизации интенсивности выходного 
потока~$\Lambda_{\mathrm{вых}}$ по параметрам $L_j$, $lp_j$, $t_{\mathrm{contr}z,j}$, 
$m_j$, $j=1, \ldots ,M$.

{\small\frenchspacing
{%\baselineskip=10.8pt
\addcontentsline{toc}{section}{Литература}
\begin{thebibliography}{99}


\bibitem{1ag}
\Au{Головкин Б.\,А.}
Расчет характеристик и планирование параллельных вычислительных 
процессов.~--- М.: Радио и связь, 1983.

\bibitem{2ag}
\Au{Гуз Д.\,С., Красовский Д.\,В., Фуругян~М.\,Г.}
Эффективные алгоритмы планирования вычислений в многопроцессорных 
сис\-те\-мах реального времени.~--- М.: ВЦ РАН, 2004. 

\bibitem{3ag}
\Au{Голосов П.\,Е., Козлов М.\,В., Малашенко~Ю.\,Е., На\-за\-ро\-ва~И.\,А., 
Ронжин~А.\,Ф.}
Модель сис\-те\-мы управ\-ле\-ния специализированным вычислительным 
комплек\-сом~// Сообщения по прикладной математике.~--- М.: ВЦ РАН, 
2010.

\bibitem{5ag} %4
\Au{Weiss G., Pinedo M.}
Scheduling tasks with exponential service times on non-identical processors to 
minimize various cost functions~// J. Appl. Prob., 1980. No.\,17. P.~187--202. 


\bibitem{8ag} %5
\Au{Лобанов Л.\,П., Кударенко А.\,А., Пивоваров~Н.\,В., Терсков~В.\,А.}
Метод анализа одного класса сис\-тем массового обслуживания для оценки 
производительности многопроцессорных вычислительных сис\-тем~// 
Программирование, 1988. №\,5. С.~6--12 . 

\bibitem{6ag} %6
\Au{Borst S., Boxma O., Groote~J.\,F., Mauw~S.}
Task allocation in a multi-server system~// J. Scheduling, 2003. No.\,6. 
P.~423--436.

\bibitem{4ag} %7
\Au{Фельдман Л.\,П., Михайлова Т.\,В.}
Использование аналитических методов для оценки эффективности 
многопроцессорных вычислительных сис\-тем~// Электронное 
моделирование, 2007. Т.~29. №\,2. С.~17--27. 

\bibitem{9ag} %8
\Au{Куланов С.\,А.}
Принципы оценки финансовых показателей ГРИД-сис\-тем~// Jet Info, 2007. 
№\,12. С.~16--27. 

\bibitem{7ag} %9
\Au{Koole G., Righter R.}
Resource allocation in grid computing~// J. Scheduling , 2008. No.\,11. 
P.~163--173.


\bibitem{10ag}
\Au{Подкопаев И.\,В.}
Исследование и разработка методов повышения эффективности управления 
вы\-чис\-ли\-тель\-ны\-ми мощностями в кластерах рабочих станций. Автореф. 
дисс. \ldots канд. техн. наук.~--- М.:\linebreak МИЭТ(ТУ), 2010.

\bibitem{11ag}
Сайт distributed.net.~--- {\sf http://www.distributed.net}. 

\bibitem{12ag}
\Au{Цирюлик О.}
QNX: кластерные вычисления~// Программное обеспечение. Сис\-те\-мы 
реального времени. СТА, 2004. №\,3. С.~54--62. 

\bibitem{13ag}
\Au{Ross K.\,W.}
Multiservice loss models for broadband telecommunication networks.~--- 
Springer Verlag, 1995. 

\bibitem{14ag}
\Au{Башарин~Г.\,П.}
Лекции по математической теории телетрафика.~--- М.: РУДН, 2007. 

\bibitem{17ag}
\Au{Степанов С.\,Н.}
Основы телетрафика мультисервисных сетей.~--- М.: Эко-Трендз, 2010. 



\bibitem{16ag}
\Au{Iversen V.\,B.}
Teletraffic engineering and network plan-\linebreak ning. DTU Course 34340.~--- Denmark:
Technical University of Denmark, Revised January~24, 2011. {\sf 
http://oldwww.com.dtu.dk/education/34340/material/ telenook2011pdf.pdf}.

\label{end\stat}

\bibitem{15ag}
\Au{Васькин Ю.\,А., Пшеничников А.\,П., Степанов~М.\,С.}
Распределение канального ресурса при обслуживании мультисервисного 
трафика~// T-COMМ~--- Телекоммуникации и транспорт, 2009. №\,4. 
С.~46--48. 


 \end{thebibliography}
}
}


\end{multicols}       