\def\stat{zatsman}

\def\tit{МОДЕЛИРОВАНИЕ ПРОЦЕССОВ ФОРМИРОВАНИЯ ЭКСПЕРТНЫХ 
ЗНАНИЙ ДЛЯ~МОНИТОРИНГА ПРОГРАММНО-ЦЕЛЕВОЙ 
ДЕЯТЕЛЬНОСТИ$^*$}

\def\titkol{Моделирование процессов формирования экспертных 
знаний для~мониторинга программно-целевой 
деятельности}

\def\autkol{И.\,М.~Зацман, А.\,А.~Дурново}
\def\aut{И.\,М.~Зацман$^1$, А.\,А.~Дурново$^2$}

\titel{\tit}{\aut}{\autkol}{\titkol}

{\renewcommand{\thefootnote}{\fnsymbol{footnote}}\footnotetext[1]
{Работа выполнена при поддержке РФФИ, грант №\,09-07-00156.}}


\renewcommand{\thefootnote}{\arabic{footnote}}
\footnotetext[1]{Институт проблем информатики Российской академии наук, iz\_ipi@a170.ipi.ac.ru}
\footnotetext[2]{Институт проблем информатики Российской академии наук, duralex49@mail.ru}

%\vspace*{6pt}   


   \Abst{Рассматривается постановка проблемы компьютерного представления целевых систем знаний (ЦСЗ) 
об индикаторах мониторинга и предлагается ее решение, состоящее из четырех компонентов: (1)~стационарной 
модели компьютерного представления ЦСЗ об индикаторах; (2)~множества точек, на основе которого 
определяются значения количественных характеристик процессов формирования ЦСЗ; (3)~нестационарной 
модели компьютерного представления ЦСЗ; (4)~проективного словаря системы мониторинга для 
компьютерного представления ЦСЗ об индикаторах в динамике их формирования. Первые три компонента 
представляют собой теоретическую часть решения этой проблемы, а четвертый~--- прикладную часть ее 
решения.}
    
    \KW{проблема представления целевых сис\-тем знаний об индикаторах; лакунарность сис\-тем 
знаний об индикаторах; семиотические модели представления знаний об индикаторах; концепты 
индикаторов; денотаты индикаторов}

 \vskip 14pt plus 9pt minus 6pt

      \thispagestyle{headings}

      \begin{multicols}{2}
      
            \label{st\stat}


\section{Введение}

   В процессе реформирования бюджетного процесса программно-целевая деятельность 
стала доминирующей во всех сферах, где расходуются бюджет\-ные средства. Согласно 
Концепции реформирования бюджетного процесса с 2004~г.\ в РФ осуществляется переход 
на реализацию долгосрочных программ с явным описанием их целей, задач, ресурсов и 
ожидаемых результатов, а также с использованием индикаторов для мониторинга и 
оценивания полученных результатов, эффективности и результативности программно-це\-ле\-вой 
деятель\-ности~[1].
   
   Применение индикаторов, традиционно используемых при статистическом наблюдении 
бюджетного процесса до его реформирования, выяви-\linebreak ло неполноту имеющихся наборов 
индикаторов. Так,
 они не позволяют оценивать согласованность поставленных целей и 
имеющихся ресурсов. Не\-об\-ходимость разработки новых индикаторов стимулировала 
проведение исследований с целью формирования новых концептуальных подходов и 
методов проектирования индикаторов, так как в бюджетном процессе принятие решений 
стало существенно зависеть от результатов мониторинга и индикаторного оценивания 
программно-целевой деятельности.
   
   Целенаправленное формирование знаний, концептуальных подходов и создание новых 
методов для разработки программно-ориентированных индикаторов стало актуальной 
проблемой для многих сфер деятельности, что нашло отражение в официальных документах. 
Так, задачи разработки новых методов оценки эффективности научной деятельности, 
создания систем мониторинга, анализа и оценки результатов деятельности юридических и 
физических субъектов сферы науки включены в тематическое направление <<29. Системы 
автоматизации, CALS-технологии, математические модели и методы исследования сложных 
управляющих сис\-тем и процессов>> Программы фундаментальных научных исследований 
РАН на 2008--2012~годы~[2].
   
   Кроме прикладного значения исследования процессов целенаправленного формирования 
знаний имеют фундаментальный научный аспект. В~настоящее время формируется новое 
научное направление (новая проблематика), объеди\-ня\-ющее исследования процессов 
целенаправленного формирования новых знаний и их компьютерного представления. 
С~прикладной точки зрения это направление связано с широким спектром сфер 
деятельности.
   
   Становление этого направления во многом было обусловлено 7-й Рамочной программой 
Европейского Союза (ЕС). В документах этой программы, принятой на период с 2007 по 
2013~гг., для сферы информационно-коммуникационных технологий (ИКТ) определен ряд 
актуальных направлений исследований и разработок, включая исследования возможностей 
существующих и разработку новых видов ИКТ, обеспечивающих представление в 
информационных системах формируемых знаний и целенаправленное влияние на процесс их 
формирования~[3--7].
   
   Результаты анализа этих направлений, приведенные в работе~[8], позволяют утверждать, 
что ИКТ играют ключевую роль в исследованиях процессов целенаправленного 
формирования знаний, в том числе в решении следующих проблем:
   \begin{itemize}
\item формирование и компьютерное представление в циф\-ро\-вой электронной среде 
(далее~--- циф\-ровая среда)\footnote{Согласно ГОСТ Р 52292-2004, электронная среда~--- 
это среда технических устройств (аппаратных средств), функционирующих на основе 
физических законов и используемых в информационной технологии при обработке, 
хранении и передаче данных. Циф\-ро\-вая электронная среда~--- это цифровые технические 
устройства (аппаратные средства) электронной среды, функционирующие на основе физических законов и 
используемые в информационной технологии при обработке, хранении и передаче данных.} личностных 
и согласованных концептов как структурных элементов ЦСЗ, 
формируемых экспертами,~--- \textit{проблема представления ЦСЗ};
\item анализ и оценивание степени релевантности разных вариантов ЦСЗ социальным, 
экономическим, технологическим и другим общественно значимым потребностям, в 
интересах которых они были сформированы,~--- \textit{проблема релевантности ЦСЗ};
\item целенаправленное влияние средствами ИКТ на формирование и эволюцию ЦСЗ, 
необходимое для получения запланированных результатов,~--- \textit{проблема 
направляемого развития ЦСЗ}.
   \end{itemize}
   
   Данная статья относится к первой из трех перечисленных проблем, т.\,е.\ к проблеме 
представления ЦСЗ. Эта проблема рассматривается с точки зрения создания систем 
информационного мониторинга и индикаторного оценивания программно-целевой 
деятельности (ПЦД). Предполагается, что в интересах мониторинга и 
индикаторного оценивания группа экспертов должна сформировать или пополнить ЦСЗ о 
программно-ориентированных индикаторах.
   
   Модели формирования новых или развития существующих систем знаний по степени их 
общ\-ности предлагается разделить на две основные категории: (1)~концептуальные модели, 
которые не зависят от той предметной области, где происходят процессы 
формирования или пополнения систем знаний; (2)~пред\-мет\-но-ори\-ен\-ти\-ро\-ван\-ные 
модели, которые зависят от этой предметной области.
   
   Широко известная в настоящее время спиральная модель формирования знаний, которая 
описана в работах Нонака и Такеучи~[9--11]\footnote{По данным Google Scholar, на 17.06.2011 
общее число цитирований работ~\cite{9-zat, 10-zat} равно 22\,344.}, и обобщенный вариант этой модели, 
предложенный в работах Вежбицки и Накамори~[12, 13], не зависят от предметной области 
и с точки зрения предложенного деления на две категории являются концептуальными. 
   
   Опыт применения ИКТ, разработанных в исследовательском институте JAIST (Japan 
Advanced Institute of Science and Technology) для реализации спиральной модели в интересах 
поддержки процесса формирования научных знаний~[14], позволяет утверждать, что, с 
одной стороны, спиральная модель и ее обобщение уже используются на практике, с другой 
стороны, остался ряд нерешенных вопросов, ограничивающих сферу их использования.
   
   Перечислим основные вопросы, которые не были решены в рамках спиральной модели и 
ее обобщенного варианта:
   \begin{itemize} %[1)]
\item не фиксируются изменения состояния формируемых личностных знаний человека в 
зависимости от времени;
   \item не определены объекты интерпретации, являющиеся источниками новых знаний 
человека;
   \item не выделены структурные элементы фор\-ми\-ру\-емых знаний, соответствующие 
объектам интерпретации;
   \item не фиксируются моменты времени начала и завершения генерации каждого нового 
структурного элемента знаний.
   \end{itemize}
   
   В интересах решения вышеперечисленных вопросов на основе спиральной модели были 
разработаны концептуальные основы~[15] и семиотические модели целенаправленного 
формирования и компьютерного представления формируемых знаний, которые относятся к 
категории концептуальных моделей. Разработанные стационарная и нестационарная 
семиотические модели не зависят от предметной области, в которой происходят процессы 
формирования или пополнения систем знаний, что иллюстрируется примерами из разных 
предметных областей~[16--18].
   
   Основной целью данной работы является описание \textit{предметно-ориентированных} 
семиотических моделей процессов целенаправленного формирования знаний об 
индикаторах, формируемых группой экспертов для мониторинга и индикаторного 
оценивания ПЦД в сфере науки. Еще одной целью является описание концептуальных и 
технических решений по созданию проективного словаря~\cite{18-zat}, с помощью которого 
был проведен эксперимент по реализации компьютерного представления экспертных знаний 
об индикаторах с отражением динамики их формирования.

\section{Процессы целенаправленного формирования знаний 
об~индикаторах}

   В предлагаемой постановке проблемы пред\-став\-ле\-ния ЦСЗ ключевыми являются 
процессы концептуализации и интерпретации, т.\,е.\ формирование смыслового содержания 
индикаторов при их разработке экспертами. Понятия <<концептуализация>> и 
<<интерпретация>> являются ключевыми и для решения проблемы представления ЦСЗ.
   
   \textit{Концептуализацию} определим как итерационный процесс формирования 
экспертом в течение некоторого периода времени концепта (смыслового содержания) нового 
индикатора. Концепт индикатора как структурный элемент знаний эксперта является 
результатом анализа алгоритма из\-ме\-ня\-емой экспертом компьютерной программы и данных, 
используемых для определения значений этого индикатора, а также его значений, 
вычисленных этой программой.
   
   Для обозначения объекта содержательного анализа, который состоит из компьютерной 
программы, данных и вычисленных значений индикатора, в статье используется термин 
<<денотат индикатора>>, или просто <<денотат>>, когда речь идет о стационарной модели, 
или термин <<состояние денотата>>, когда речь идет о нестационарной модели.
   
   Предполагается, что на каждой итерации группа экспертов может формировать несколько 
новых индикаторов одновременно и что каждая итерация концептуализации состоит из двух 
стадий: стадии изменения денотатов формируемых индикаторов (результат изменения будем 
называть новым состоянием денотата) и стадии интерпретации. Последовательные моменты 
времени начала стадий интерпретации обозначим как $\{t_i -\Delta_i$, $\Delta_i <t_i - 
   t_{i-1}$, $t_0\hm=0$, $t_i \hm> t_{i-1}$, $i\hm=1, 2, \ldots\}$, а завершения стадий интерпретации 
обозначим как $\{t_i$, $i=1, 2, \ldots\}$.
   
   Предполагается, что на стадии интерпретации компьютерная программа и данные, 
используемые этой программой для определения значений индикатора, остаются 
неизменными. Изменения могут происходить только на стадиях изменения денотатов 
формируемых индикаторов, т.\,е.\ на интервалах времени $[t_{i-1}, t_i - \Delta_i]$.
   
   \textit{Интерпретацию} определим как процесс содержательного анализа денотата, 
выполняемый экспертом на одном из интервалов времени $[t_i -\Delta_i, t_i]$, или состояния 
денотата индикатора, если программа и/или данные изменяются в течение 
концептуализации.
   
   Предполагается, что группа экспертов, разрабатывающая новые индикаторы, 
руководствуется явно заданными целями генерации ЦСЗ о про\-грам\-мно-ориентированных 
индикаторах. Между любыми двумя последовательными стадиями интерпретации эксперты 
могут вносить изменения в алгоритмы компьютерных программ и в данные, используемые 
для определения значений индикаторов.
   
   Если на некоторой стадии интерпретации денотата было сгенерировано и описано 
смысловое содержание (концепт) нового индикатора, а затем между двумя 
последовательными стадиями интерпретации был изменен алгоритм и/или данные (было 
получено новое состояние денотата), то на следующей стадии может быть сгенерирован 
другой концепт разрабатываемого индикатора как результат интерпретации нового 
состояния денотата.
   
   Используя вышеопределенные понятия <<концептуализация>> и <<интерпретация>>, 
\textbf{сформулируем проблему представления ЦСЗ} о новых 
   программно-ориентированных индикаторах следующим образом. Предположим, что 
перед группой экспертов поставлена задача формирования или пополнения ЦСЗ об 
индикаторах, предназначенных для мониторинга и оценивания ПЦД, используя некоторую 
существующую систему классификации индикаторов.
   
   Процесс формирования ЦСЗ об индикаторах представляет собой итерационный процесс 
концептуализации денотатов, целью которого является формирование новых 
   программно-ори\-ен\-ти\-ро\-ван\-ных индикаторов. Каждое состояние денотата индикатора (как 
зафиксированный в циф\-ро\-вой среде объект интерпретации) представляет собой 
совокупность трех следующих компонентов, которые не изменяются на стадии 
интерпретации:
   \begin{enumerate}[(1)]
\item программы определения значений нового индикатора;
\item данных, которые обрабатываются программой определения его значений;
\item значений нового индикатора, которые получены в результате выполнения программы.
\end{enumerate}

   Между стадиями интерпретации эксперты имеют возможность изменять программы и 
данные, которые обрабатываются программой. Эксперт может написать новую (изменить 
существующую) программу определения значений индикатора и/или подготовить (изменить) 
данные, которые обрабатываются этой программой, затем выполнить этот вариант 
программы и получить результаты вы\-чис\-ле\-ний. На каждой стадии интерпретации эксперт 
анализирует алгоритм программы, использованные данные и полученные значения с целью 
формирования на этой стадии его личностного концепта (смыслового содержания) варианта 
нового индикатора.
   
   Каждая стадия интерпретации денотата индикатора необязательно завершается 
формированием концепта индикатора (например, эксперт изменил алгоритм и/или данные, 
но не смог интерпретировать полученное состояние денотата). В~случае формирования 
концепта эксперт выражает свое личностное понимание индикатора сначала в виде позиции 
(рубрики) в заданной системе классификации индикаторов, а затем дефиницией. Также он 
может присвоить индикатору имя.
   
   Кроме того, он начинает процесс согласования полученного им личностного концепта, 
выраженного в виде рубрики и дефиниции, а также имени индикатора с другими экспертами 
группы. Цель процесса согласования~--- убедить других экспертов в корректности 
построения и правильности своей интерпретации индикатора, представленной в виде 
рубрики в заданной системе классификации и дефиниции, а также обосновать выбор имени 
индикатора. Если он получит согласование еще хотя бы одного эксперта, то концепт этого 
индикатора получает статус согласованного концепта.
   
   Каждая стадия интерпретации индикатора может завершиться одним из четырех 
вариантов следующих результатов, которые регистрируются в лингвистическом и других 
видах обеспечения системы информационного мониторинга:
   \begin{enumerate}[(1)]
\item написана новая (изменена существующая) программа вычисления значений варианта 
нового индикатора и/или подготовлены (изменены) данные, которые обрабатываются этой 
программой, и получены результаты вычислений, но личностный концепт не 
сформирован (получено новое неинтерпретированное состояние денотата, иначе говоря, 
понимание этого состояния денотата отсутствует);
\item сформирован личностный концепт и определена его позиция (рубрика) в заданной 
системе классификации, но отсутствуют дефиниция как вербальное описание концепта 
нового индикатора и имя для полученного варианта нового индикатора;
\item дополнительно экспертом сформулирована дефиниция как вербальное описание 
концепта нового индикатора и выбрано для него имя;
\item позиция (рубрика) в заданной системе классификации, дефиниция и имя нового 
индикатора согласованы с одним или несколькими экспертами группы.
\end{enumerate}

   На следующей стадии интерпретации эксперт может сформировать новый денотат 
(изменить существующий денотат, получив новое его состояние)\linebreak
и повторить перечисленные 
операции или продолжить процесс согласования ранее созданных индикаторов с другими 
экспертами группы. Если в\linebreak
последнем случае согласование состоялось, то результатом будет 
изменение степени согласован\-ности индикаторов между экспертами. Предполагается, что 
задан некоторый пороговый уровень\linebreak степени согласованности, при достижении которого 
индикатор получает статус \textit{сформированного}.
   
   При перечисленных исходных условиях и вышеописанных вариантах формирования 
индикаторов требуется:
   \begin{itemize}
\item определить пространство компьютерных кодов состояний денотатов индикаторов, 
их концептов и имен, включающее ось времени, для отражения в этом пространстве 
результатов концептуализации денотатов индикаторов и при этом зафиксировать моменты 
времени начала и завершения каждой стадии интерпретации;
\item разработать модели компьютерного пред\-став\-ле\-ния экспертных знаний для описания 
процес\-сов формирования индикаторов в пространстве компьютерных кодов состояний 
денотатов, концептов и имен индикаторов.
\end{itemize}

   Существенной особенностью сформулированной проблемы является 
представление результатов\linebreak концептуализации денотатов индикаторов в пространстве 
компьютерных кодов состояний денотатов, концептов и имен индикаторов. Сама постановка 
этой проблемы говорит о том, что в ней\linebreak
охватываются сущности различной природы из трех 
разных сред: ментальной (концепты индикаторов), со\-ци\-аль\-но-коммуникационной (имена 
индикаторов, присвоенные экспертами) и цифровой (три компонента денотатов, а также 
компьютерные коды концептов, имен и состояний денотатов, присвоенные системой 
информационного мо\-ни\-то\-ринга).

\begin{figure*}[b] %fig1
\vspace*{9pt}
\begin{center}
\mbox{%
\epsfxsize=123.91mm
\epsfbox{zatc-1.eps}
}
\end{center}
\vspace*{-9pt}
\Caption{Стационарная семиотическая модель индикатора, его семокод и формокод}
\end{figure*}
   
   В приведенной постановке решением сформулированной проблемы представления ЦСЗ 
будем считать создание совокупности таких компонентов, как:
   \begin{enumerate}[(1)]
\item стационарная модель компьютерного пред\-став\-ле\-ния знаний о \textit{неизменяемых} 
про\-грам\-мно-ори\-ен\-ти\-ро\-ван\-ных индикаторах, которые\linebreak\vspace*{-12pt}

\pagebreak

\noindent
 являются составной частью ЦСЗ, и 
модель компьютерного представления в некоторый момент времени формируемых знаний об 
\textit{изменяемых} индикаторах;
   \item множество точек пространства действительных чисел $\mathbf{R}_2$, на основе 
которого определяются значения количественных характеристик процесса формирования 
ЦСЗ (это множество точек будем называть пространством Фреге);
   \item нестационарная модель компьютерного представления ЦСЗ об \textit{изменяемых} 
программно-ори\-ен\-ти\-ро\-ван\-ных индикаторах, формируемых группой экспертов;
   \item компонент лингвистического обеспечения сис\-те\-мы информационного мониторинга 
для компьютерного представления ЦСЗ в динамике формирования индикаторов, который 
будем называть \textit{проективным словарем}. Этот словарь представляет собой один из 
возможных вариантов технической реализации моделей компьютерного представления ЦСЗ.
   \end{enumerate}
   
   Первые три из четырех компонентов решения сформулированной проблемы 
представления ЦСЗ, которым посвящен следующий раздел статьи, представляют собой 
теоретическую часть решения этой проблемы. Четвертый компонент, которому посвящен 
раздел~4, представляет собой один из вариантов прикладного ее решения.

\section{Предметно-ориентированные семиотические модели}

   \subsection{Первый этап решения проблемы} %3.1
   
   На первом этапе решения этой проблемы определим стационарную семиотическую 
модель, которая предназначена для компьютерного кодирования в системе 
информационного мониторинга ПЦД экспертных знаний о неизменяемых индикаторах, уже 
сформированных группой экспертов. Затем определим нестационарную семиотическую 
модель, которая предназначена для компьютерного кодирования экспертных знаний об 
изменяемых индикаторах, формируемых группой экспертов.
   
   Сформулируем три исходных положения для построения обеих моделей согласно 
вышеприведенной постановке проблемы представления ЦСЗ, используя рис.~1.

   \textbf{Положение 1.} В процессе целенаправленного формирования знаний об 
индикаторах для каждого индикатора должны кодироваться три его компонента: концепт 
(смысловое содержание), имя и денотат индикатора, если кодируемый индикатор является 
стабильным, или состояние денотата изменяемого индикатора в некоторый момент времени, 
если кодируемый индикатор является формируемым, т.\,е.\ нестабильным.
   
   Концепт, имя и денотат (состояние денотата) являются вершинами \textit{треугольника 
Фреге} соответствующего индикатора\footnote{Чтобы построить треугольник Фреге для 
изменяемого индикатора в некоторый момент времени, необходимо 
использовать понятие авторского знака 
индикатора, две стороны которого: форма знака (имя индикатора) и значение знака (концепт 
индикатора)~--- могут находиться в отношении временной связи. Эта связь является опосредованной 
сознанием автора этого знака и представляет собой нестабильное единство, которое посредством сенсорно 
воспринимаемой автором формы знака репрезентирует персонально приданное этому знаку значение. 
Значение авторского знака индикатора является результатом личностного анализа автором объекта 
интерпретации как состояния рассматриваемого денотата в момент времени, указанный автором этого 
знака при его регистрации в системе информационного мониторинга~\cite{8-zat, 15-zat}.}~[19--21].
   
   \textbf{Положение 2.} Концепт, имя и денотат индикатора имеют различную природу. 
Поэтому для построения стационарной семиотической модели будем использовать 
следующие три среды: ментальную, социально-коммуникационную и цифровую 
(см.\ рис.~1).
   
   \textbf{Положение 3.} Каждый дескриптор, который строится экспертом в целях 
описания фор\-ми\-ру\-емо\-го им индикатора, включает три уникальных (в рамках проективного 
словаря) идентификатора:
   \begin{enumerate}[(1)]
\item \textit{семантический идентификатор}, предназначенный для компьютерного 
представления в системе информационного мониторинга концепта индикатора;
\item \textit{информационный идентификатор}, предназначенный для компьютерного 
кодирования имени индикатора;
\item \textit{объектный идентификатор}, предназначенный для кодирования денотата 
индикатора, если кодируемый индикатор является стабильным, или состояния денотата 
изменяемого индикатора, если кодируемый индикатор является формируемым, т.\,е.\ 
нестабильным.
\end{enumerate}

   Совокупность числовых представлений семантического, информационного и объектного 
идентификаторов дескриптора будем называть цифровым семиотическим треугольником по 
аналогии с треугольником Фреге.
   
   \smallskip
   
   \noindent
   \textbf{Определение 1.} Пусть для некоторого стабильного индикатора известны 
его концепт, имя и денотат, которые не изменяются во времени, относятся к ментальной, 
социально-коммуникационной и цифровой средам соответственно и образуют треугольник 
Фреге. Тогда \textbf{стационарной семиотической моделью}
компьютерного кодирования этого индикатора называется его треугольник Фреге, для 
которого построен дескриптор и трем вершинам которого назначены семантический 
идентификатор для концепта индикатора, информационный идентификатор для имени и 
объектный идентификатор для денотата как совокупности трех компонентов: программы, 
данных и значений этого индикатора.
   \smallskip
   
   Рисунок~1 включает ментальную, социально-ком\-му\-ни\-ка\-ци\-он\-ную и циф\-ро\-вую среды, 
содержащие соответственно концепт индикатора, его имя и денотат, которые и образуют 
треугольник Фре-\linebreak ге (составлен тремя полужирными отрезками).\linebreak
 Построенный дескриптор с 
тремя вершинами,\linebreak
которым согласно определению~1 назначены семантический 
идентификатор для концепта, информационный идентификатор для имени и объектный 
идентификатор для денотата этого индикатора, обозначен равнобедренным треугольником с 
двойными точечными сторонами. На этом же рисунке двумя кругами условно обозначены 
два понятия: <<\textit{семокод}>> и <<\textit{формокод}>>, которые введем как следствие 
определения~1.
   
   \smallskip
   
   \noindent
   \textbf{Следствие 1.} \textit{Пусть для некоторого стабильного индикатора заданы 
семантический идентификатор для концепта индикатора и информационный идентификатор 
для его имени. Тогда} {\bfseries\textit{семокодом}} 
\textit{этого индикатора называется 
совокупность концепта индикатора и уникального семантического идентификатора 
дескриптора индикатора, а} {\bfseries\textit{формокодом}} 
\textit{этого индикатора называется совокупность имени 
индикатора и уникального информационного идентификатора дескриптора 
индикатора}~\cite{15-zat}.
   
   \smallskip
   
   \noindent
   \textbf{Определение~2.} {Пусть для некоторого нестабильного индикатора 
известны в некоторый момент времени состояние его денотата, концепт как структурный 
элемент экспертных знаний, полученный в результате интерпретации этого состояния, и имя 
этого индикатора. Тогда семиотической моделью компьютерного кодирования индикатора в 
этот момент времени называется соответствующий ему треугольник Фреге, для которого 
построен дескриптор и трем вершинам которого (концепту, имени и состоянию денотата) 
назначены семантический идентификатор для концепта, информационный идентификатор 
для имени и объектный идентификатор для состояния денотата как совокупности трех 
компонентов: программы, данных и значений индикатора в этот момент времени.}
   
   \smallskip
   
   \noindent
   \textbf{Следствие 2.} 
\textit{Пусть для некоторого нестабильного индикатора заданы в 
некоторый момент времени семантический идентификатор для концепта индикатора и 
информационный идентификатор для его имени. Тогда} 
{\bfseries\textit{семокодом}} 
\textit{этого индикатора называется совокупность концепта и уникального семантического 
идентификатора дескриптора индикатора, а формокодом этого индикатора называется 
совокупность имени индикатора и уникального информационного идентификатора 
дескриптора индикатора}~\cite{15-zat}.
   
   В отличие от концепта индикатора, его имени и денотата, понятия <<семокод>> и 
<<формокод>> являются двуедиными сущностями, которые относятся одновременно к двум 
средам. Поэтому семокод находится на границе ментальной и циф\-ро\-вой 
сред, а формокод~--- на границе социально-коммуникационной и цифровой сред (см.\ 
рис.~1).
   
   Предложенные определения двух моделей и их следствия являются первым из трех 
компонентов теоретической части решения проблемы пред\-став\-ле\-ния ЦСЗ об индикаторах. 
Этот компонент, получен\-ный на первом этапе, представляет собой модели компьютерного 
кодирования вершин треугольника Фреге, принадлежащих разным средам, а также понятия 
<<семокод>> и <<формокод>>, используемые при описании технических решений 
компьютерного кодирования концептов и имен индикаторов.
   
   Модель компьютерного представления \textit{в некоторый момент времени} 
формируемых знаний об\linebreak
\textit{изменяемых} индикаторах понадобится далее для 
определения пространства Фреге и построения нестационарной модели. Отметим, что в 
определении~1 говорится о денотатах, так как стационарная модель 
используется для случая не\-из\-ме\-ня\-емо\-го индикатора, а в определении~2~--- о 
состояниях денотатов. При этом предполагается, что фиксируется момент времени 
завершения стадии интерпретации изменяемого индикатора, рассматривается 
соответствующее состояние денотата этого индикатора и семиотическая модель применяется 
для кодирования вершин треугольника Фреге только в этот момент времени.
   
   \subsection{Второй этап решения проблемы} %3.2
   
   На втором этапе решения проблемы представления ЦСЗ об индикаторах определим 
пространство Фреге для компьютерного представления процессов формирования ЦСЗ об 
индикаторах, формируемых экспертами. Кроме вышеперечисленных положений~1--3 
дополнительно понадобятся следующие исходные положения, которые будем нумеровать 
начиная с номера~4.
   
   \textbf{Положение 4.} На каждой итерации концептуализации группа экспертов может 
формировать несколько новых индикаторов одновременно, и\linebreak
 каж\-дая итерация состоит из 
двух стадий: стадии изменения денотатов формируемых индикаторов (результат изменения 
денотата будем называть новым его состоянием) и стадии интерпретации. Каждая стадия 
интерпретации может завершиться одним из \textit{четырех возможных вариантов} 
окончания, которые были описаны в предыдущем разделе.
   
   \textbf{Положение 5.} Результаты каждого из четырех вариантов окончания стадии 
интерпретации должны регистрироваться в лингвистическом и других видах обеспечения 
системы информационного мониторинга по завершении каждой итерации концептулизации 
индикатора, в том числе регистрируются \textit{моменты времени начала и завершения} 
каждой стадии интерпретации. Последовательные моменты времени начала стадий 
интерпретации были ранее обозначены как $\{t_i -\Delta_i$, $\Delta_i < t_i - t_{i-1}$, 
$t_0\hm=0$, $t_i 
> t_{i-1}$, $i=1, 2, \ldots\}$, а моменты времени завершения стадий интерпретации~--- как 
$\{t_i$, $i = 1, 2, \ldots\}$.
   
   \textbf{Положение 6.} Для первого варианта окончания стадии интерпретации должны 
регистрироваться момент времени окончания стадии, используемый вариант программы 
вычисления значений варианта нового индикатора, данные, которые обрабатываются этим 
вариантом программы, и полученные результаты вычисления значений индикатора, т.\,е.\ 
фиксируется каждый момент времени формирования состояния денотата индикатора и само 
это состояние как совокупность трех перечисленных компонентов. В~проективном 
словаре экс\-пер\-том-разработчиком программы строится дескриптор, содержащий ссылки на 
каждый из трех компонентов, а также в нем запоминается персональный идентификатор 
этого эксперта.
   
   \textbf{Положение 7.} Для второго варианта окончания стадии интерпретации 
дополнительно к тре-\linebreak бованиям положения~6 должна быть выбрана\linebreak
позиция (\textit{рубрика 
для концепта индикатора}) в сущест\-ву\-ющей системе классификации и указан персональный 
идентификатор эксперта, который выбрал эту рубрику.
   
   \textbf{Положение 8.} Для третьего варианта дополнительно к требованиям положений~6 
и~7 должны регистрироваться \textit{дефиниция} и выбранное \textit{имя} для нового 
индикатора, а также персональный идентификатор эксперта, который разработал эту 
дефиницию и выбрал имя.
   
   \textbf{Положение 9.} Для четвертого варианта дополнительно к требованиям 
положений~6--8 должны регистрироваться результаты согласования в группе экспертов 
рубрики заданной системы классификации, дефиниции и имени нового индикатора, а также 
персональные идентификаторы соответствующих экспертов. 
   
   \textbf{Положение 10.} Ссылки на три компонента каж\-до\-го нового состояния денотата 
индикатора, его рубрика в заданной системе классификации, де\-финиция и имя, а также 
моменты времени начала и\linebreak
 завершения каждой стадии интерпретации этого индикатора 
должны храниться в системе информационного мониторинга в виде \textit{атрибутов 
дескриптора} проективного словаря лингвистического обеспечения этой системы, 
построенного на данной стадии интерпретации индикатора.
   
   \textbf{Положение 11.} Кроме ссылок на три компонента каждого нового состояния 
денотата индикатора дескриптор также должен включать атрибуты, фиксирующие его 
положение внутри проективного словаря, которые являются ссылками на другие 
дескрипторы (эти ссылки строятся на основе заданной классификации индикаторов~\cite{22-zat}).
   
   Используя вышеперечисленные положения, опре\-де\-лим вариант пространства Фреге, 
пред\-на\-зна\-чен\-ный для количественного описания процессов целенаправленного 
формирования ЦСЗ об индика\-торах, используя семантические, информационные и 
объектные идентификаторы дескрипторов состояний этих индикаторов, зарегистрированных 
в системе информационного мониторинга. Определенный ниже вариант пространства Фреге 
представляет собой предметно-ориентированную версию пространства Фреге, построенного 
в работе~\cite{17-zat}.
   
   Состояния формируемых индикаторов будем\linebreak фиксировать в последовательные моменты 
вре-\linebreak мени, которые обозначены как $\{t_i$, $i=1, 2, \ldots\}$,\linebreak где $t_i$~--- момент 
времени \textit{завершения} $i$-й стадии интерпретации индикаторов, в том чис\-ле 
по\-стро\-ения дескрипторов новых состояний индикаторов. В~каж\-дый из этих моментов 
времени будем фиксировать описание концептов формируемых индикаторов (их рубрики 
и дефиниции); соответствующие им информационные объекты (имена индикаторов); 
три компонента состояния каждого из тех денотатов, которые были сгенерированы или 
изменены на интервале времени $[t_{i-1}, t_i - \Delta_i]$, т.\,е.\ к \textit{моменту времени 
начала} $i$-й стадии интерпретации.
{\looseness=1

}
   
   За $t_1$ принимается тот момент времени, когда эксперты начали фиксировать в системе 
информационного мониторинга процесс формирования ЦСЗ об индикаторах. Сам процесс 
формирования ЦСЗ мог начаться и до момента времени $t_1$, но момент времени 
завершения первой стадии интерпретации и построения дескрипторов обозначим именно 
как~$t_1$.
   
   Предполагается, что в каждый из моментов времени $\{t_i$, $i=1, 2, \ldots\}$, когда 
завершаются \mbox{стадии} интерпретации индикаторов и построения дескрипторов новых 
состояний индикаторов, сис\-те\-ма информационного мониторинга генерирует три уникальных 
(в рамках проективного словаря) идентификатора и ими кодируются вершины 
семиотических треугольников Фреге согласно семиотической модели. При этом в каждый 
момент времени~$t_i$ может быть закодировано одновременно несколько 
формируемых индикаторов с использованием уникальных идентификаторов дескрипторов, 
если эксперты их построят.
   
   \smallskip
   
   \noindent
   \textbf{Определение 3.} Пусть заданы моменты времени начала стадий 
интерпретации $\{t_i -\Delta_i$, $\Delta_i < t_i - t_{i-1}$, 
$t_0\hm=0$, $t_i > t_{i-1}$, $i=1, 2, 
\ldots\}$\footnote{В общем случае в процессе формирования или пополнения ЦСЗ об индикаторах задаются 
не моменты времени начала и завершения стадий интерпретации, а алгоритм, с помощью которого в момент 
времени $t_{i-1}$ можно определить моменты времени $(t_i - \Delta_i)$ и~$t_i$.}, моменты времени 
завершения этих стадий $\{t_i,\ i=1, 2, \ldots\}$ и пусть для каждого момента времени~$t_i$ 
известно число тех индикаторов, денотаты которых были сгенерированы или изменены на 
интервале времени $[t_{i-1}, t_i -\Delta_i]$, которое обозначим как~$S_i$. В~общем случае 
чис\-ло~$S_i$ может зависеть от длины интервала времени $[t_{i-1}, t_i - \Delta_i]$. 
Предположим, что для каждого из $S_i$~денотатов экспертам известны их состояния, эти 
состояния стали объектами интерпретации и экспертами было построено $S_i$ дескрипторов 
на интервале времени $[t_i - \Delta_i, t_i]$.
   
Тогда {\bfseries\textit{пространство Фреге}} в целях регистрации характеристик 
процесса формирования ЦСЗ об индикаторах в системах информационного мониторинга 
определим как четырехмерное пространство действительных чисел~\textbf{R}$_2$, в 
котором задано множество точек $\{(t_i, n_{ij}, m_{ij}, k_{ij})$, 
$j = 1, \ldots , S_i$, $i \hm= 1, 2,  \ldots\}$, где
\begin{enumerate}[(1)]
\item $t_i$~--- момент времени завершения $i$-й стадии интерпретации, в течение 
которой было построено $S_i$ дескрипторов на интервале времени $[t_{i-1}, t_i -
\Delta_i]$, и генерации для каждого дескриптора трех уникальных идентификаторов в 
момент времени~$t_i$;
\item $n_{ij}$~--- числовое представление семантического идентификатора $j$-го 
дескриптора, сгенерированного в момент времени~$t_i$ ($n_{ij}=0$, если кон\-цеп\-та 
нет, т.\,е.\ у экспертов отсутствует понимание совокупности программы, данных и 
значений, зафиксированных с помощью этого дескриптора; эту совокупность 
эксперты пытались интерпретировать на интервале времени $[t_i - \Delta_i, t_i]$;
\item $m_{ij}$~--- числовое представление информационного идентификатора имени 
$j$-го дескриптора, сгенерированного в момент времени~$t_i$ ($m_{ij}= 0$, если 
имени нет);
\item $k_{ij}$~--- числовое представление объектного идентификатора состояния 
денотата $j$-го дескриптора, сгенерированного в момент времени~$t_i$; каж\-дое 
состояние денотата представляет собой совокупность программы вычисления, 
обрабатываемых данных и значений этого индикатора, вычисленных на интервале 
времени $[t_{i-1}, t_i -\Delta_i]$.
\end{enumerate}


   По определению пространство Фреге, которое является вторым компонентом решения 
проблемы, включает ось времени и три оси с числовыми значениями идентификаторов 
концептов, имен и состояний денотатов индикаторов. Для дескрипторов, построенных на 
интервалах времени $\{[t_i - \Delta_i, t_i]$, $i\hm= 1, 2, \ldots\}$, пространство Фреге не 
показывает связей идентификаторов дескрипторов с концептами, именами и состояниями 
денотатов соответствующих индикаторов. Однако в любой момент времени $t_i$ эти связи 
фиксируются в описании тех дескрипторов, которые были построены на интервале времени 
$[t_i - \Delta_i, t_i]$.
   
   \subsection{Третий этап решения проблемы} %3.3
   
   На третьем этапе теоретической части решения проблемы представления ЦСЗ об 
индикаторах определим нестационарную семиотическую \mbox{модель}, которая является третьим 
компонентом решения этой проблемы.
   
\smallskip
   
   \noindent
   \textbf{Определение 4.} {Пусть заданы моменты времени начала стадий 
интерпретации $\{t_i -\Delta_i$, $\Delta_i < t_i - t_{i-1}$, $t_0 \hm= 0$, 
$t_i > t_{i-1}$, $i = 1, 2, \ldots \}$, 
моменты времени завершения этих стадий $\{t_i, i \hm =1, 2, \ldots\}$ и пусть для каждого 
момента времени $t_i$ известно число тех индикаторов, денотаты которых были 
сгенерированы или изменены на интервале времени $[t_{i-1}, t_i -\Delta_i]$, которое 
обозначим как~$S_i$. 

Предположим, что для каждого из $S_i$~денотатов экспертам известны 
их состояния, эти состояния стали объектами интерпретации и экспертами было построено $S_i$ 
дескрипторов на интервале времени $[t_i - \Delta_i, t_i]$}.
   
Тогда {\bfseries\textit{нестационарной семиотической моделью}} компьютерного представления 
концептов ЦСЗ о формируемых индикаторах, кодирования состояний их денотатов и имен в 
дискретные моменты времени $\{t_i, i \hm= 1, 2,  \ldots\}$ называется совокупность 
следующих трех составляющих:
   \begin{enumerate}[(1)]
\item {множество семиотических треугольников Фреге для индикаторов, описание которых 
в виде дескрипторов получено экспертами на интервалах времени $\{[t_i -\Delta_i, t_i], i 
\hm =1, 2, \ldots\}$ (это множество обозначим как $\{T_{ij}, j\hm=1, \ldots , S_i$, $i \hm= 1, 2, 
\ldots\})$};
\item {множество цифровых семиотических треугольников, соответствующих $\{T_{ij}\}$, 
которые состоят из числовых представлений семантического, информационного и 
объектного идентификаторов, генерируемых системой информационного мониторинга на 
интервалах времени $\{[t_i -\Delta_i, t_i], i = 1, 2, \ldots\}$ согласно некоторому заданному 
правилу назначения идентификаторов (множество цифровых семиотических 
треугольников обозначим как $\{D_{ij}, j = 1, \ldots, S_i$, $i = 1, 2, \ldots\}$)};
\item пространство Фреге, определенное как множество точек 
$\{(t_i, n_{ij}, m_{ij}, k_{ij})$, 
$j=1, \ldots , S_i$, $i \hm= 1, 2, \ldots\}$, полученных в результате отображения в 
$\mathbf{R}_2$ цифровых семиотических треугольников $\{D_{i,j}\}$.
\end{enumerate}


   Первая из трех составляющих модели описывает содержательные характеристики 
процесса формирования индикаторов в виде дескрипторов, в том числе состояния их 
денотатов, концепты и имена, вторая составляющая представляет собой числовые 
представления семантического, информационного и объектного идентификаторов, а третья 
составляющая представляет собой множество точек, которое планируется использовать для 
построения различных функциональных характеристик процесса формирования индикаторов 
(пример одной из характеристик будет рассмотрен далее в этом раз\-деле).
{\looseness=1

}
   
   Отметим взаимную связанность трех перечисленных составляющих нестационарной 
семиотической модели. Зная некоторую точку $(t_i, n_{ij}, m_{ij}, k_{ij})$ в пространстве 
Фреге, можно определить момент времени $t_i$, когда был сформирован соответствующий 
этой точке дескриптор, и его уникальный объектный идентификатор $k_{ij}$, который 
является одной из вершин цифрового семиотического треугольника~$D_{ij}$. Именно по 
такому коду можно найти этот дескриптор (построенный для семиотического треугольника 
Фреге~$T_{ij}$), который содержит:
   \begin{itemize}
   \item дефиницию концепта $j$-го индикатора, сформированного на интервале времени 
$[t_i -\Delta_i, t_i]$ (если $n_{ij}=0$, то рубрика и дефиниция отсутствуют);
   \item имя (название) $j$-го индикатора, выбранного на интервале времени $[t_i -\Delta_i, 
t_i]$ (если $m_{ij}=0$, то имя отсутствует);
   \item ссылки на описание программы вычисления значений $j$-го индикатора и 
обрабатываемых этой программой данных, использованных для вычисления значений $j$-го 
индикатора на интервале времени $[t_i -\Delta_i, t_i]$.
   \end{itemize}
   
   И наоборот, зная момент времени~$t_i$, можно найти все дескрипторы, построенные для 
треуголь-\linebreak\vspace*{-12pt}
\pagebreak

\noindent
ников $T_{ij}$, число которых равно~$S_i$. Для каждого из этих дескрипторов 
можно определить его идентификаторы, числовые значения которых~$n_{ij}$, $m_{ij}$, 
$k_{ij}$ вместе с~$t_i$ дадут точку ($t_i, n_{ij}, m_{ij}, k_{ij}$) в пространстве Фреге.
   
   Определенная нестационарная семиотическая модель является третьим компонентом 
решения проблемы представления ЦСЗ об индикаторах. В~этом определении множество 
точек $\{(t_i, n_{ij}, m_{ij}, k_{ij})\}$ является зависимым от используемого правила 
назначения информационного, семантического и объектного идентификаторов для 
дескрипторов проективного словаря.
   
   В качестве примера одной их характеристик процесса формирования индикаторов 
построим функцию степени согласованности индикаторов между экспертами, отражающую 
динамику их согласования. 

В~процессе построения будут использоваться данные 
проведенного эксперимента по разработке нескольких вариантов индикатора, 
характеризующих распределение публикационной активности научного коллектива по 
возрастным группам (далее~--- ЦСЗ об индикаторах возрастного распределения публикаций 
(ВРП))~\cite{8-zat}.
   
   Сначала определим множество целочисленных параметров $\{L^p_{ij}\geq0$, $j=1, \ldots , 
S_i$, $i = 1, 2,  \ldots$, $p= 0, 1, 2,  \ldots\}$, значение каждого из которых равно числу 
экспертов, согласованно интерпретирующих в момент времени $t_{i+p}$ концепт $j$-го 
индикатора, сгенерированного на интервале времени $[t_i -\Delta_i, t_i]$, согласных с 
выбором его имени и состоянием его денотата, созданным на интервале времени $[t_{i-1}, t_i -
\Delta_i]$.
   
   Предположим также, что задано число $N_C$ как\linebreak
   граница между категориями 
формируемых и сформированных вариантов индикатора ВРП. В~проведенном эксперименте 
вариант индикатора считался сформированным, если число экспертов, согласо\-ванно 
интерпретирующих его концепт и согласных с выбором его имени, было больше или равно 
$N_C$.\linebreak
 Было также определено понятие неактуального дескриптора варианта индикатора 
ВРП: если в момент времени~$t_i$ все эксперты, участ\-ву\-ющие в\linebreak
 формировании вариантов, 
приняли решение о неперспективности дальнейшего использования некоторого уже 
построенного дескриптора, то в момент времени~$t_i$ он помечается как неактуальный. 

В~процессе проведения эксперимента \textit{функция степени согласованности}, которую 
обозначим как $F_{\mathrm{con}}$, была определена следующим образом: для любого~$i$ при 
условии, что $S_i\not=0$, 

\noindent
\begin{multline*}
   F_{\mathrm{con}} (i, p, n_{ij}, m_{ij}, k_{ij}) ={}\\
   {}=
   \begin{cases} 
   0,\, & \mbox{если дескриптор с кодом $n_{ij}$ является}\\
   & \mbox{\textit{неактуальным} в момент~$t_{i+p}$};\\
   1\,, & \mbox{если дескриптор с кодом $n_{ij}$ является}\\
   &\mbox{\textit{авторским} в момент $t_{i+p}$};\\
   L^p_{ij}\,, & \mbox{если дескриптор с кодом $n_{ij}$ является}\\
   &\mbox{\textit{коллективным}, а число экспертов,}\\
   &\mbox{его  согласовавших в момент~$t_{i+p}$,}\\
   &\mbox{меньше~$N_C$};\\
   N_C\,, & \mbox{если дескриптор с кодом $n_{ij}$ является}\\
   &\mbox{\textit{сформированным} в момент~$t_{i+p}$,}\\
   &\mbox{т.\,е.\ 
число\ экспертов,\ его\ согласовав-}\\
&\mbox{ших\ в\ момент~$t_{i+p}$,\ равно}\\ 
&\mbox{или\ больше~$N_C$},
   \end{cases}
   \end{multline*}
   
   \vspace*{-2pt}
   
   \noindent
где $p = 0, 1, 2, \ldots$ является пятым измерением области определения~$F_{\mathrm{con}}$ 
кроме  четырех измерений~$i$, $n$, $m$ и~$k$ (пятое измерение введено для отражения изменения 
степени согласованности дескрипторов индикаторов, начиная с момента создания каждого из 
этих дескрипторов).
   
   Согласно этому определению, $F_{\mathrm{con}}=1$ для всех тех семиотических 
треугольников~$T_{ij}$, которым в момент времени~$t_{i+p}$ соответствует авторский 
дескриптор с личностным концептом, который не согласован с другими экспертами. Если в 
некоторый момент времени эксперт-автор этого дескриптора и все остальные эксперты, 
участвующие в формировании ЦСЗ, отмечали его как неактуальный для процесса 
построения ЦСЗ, то тогда $F_{\mathrm{con}}\hm=0$.
   
   Значения определенной функции $F_{\mathrm{con}}$ не зависят от числовых представлений 
семантических, информационных и объектных идентификаторов $n_{ij}$, $m_{ij}$ и~$k_{ij}$, 
а зависят только от хода процесса согласования экспертами смысла вариантов индикатора, 
т.\,е.\ их рубрик и дефиниций, а также имен вариантов индикатора ВРП.

\begin{figure*} %fig2
\vspace*{1pt}
\begin{center}
\mbox{%
\epsfxsize=122.775mm
\epsfbox{zatc-2.eps}
}
\end{center}
\vspace*{-12pt}
\Caption{Десять из 20 значений $F_{\mathrm{con}}$ на пяти итерациях в следующих точках~$t_{i+p}$: 
($i = 1$, $p = 0$); ($i = 1$, $p =1$); ($i = 3$, $p = 0$); ($i = 4$, $p = 0$); 
($i = 4$, $p = 1$). 
Сплошные стрелки соединяют те пары значений функции, которым соответствуют идентичные 
концепты; точечными стрелками обозначены отношения наследования
}
\end{figure*}

   
   Рассмотрим первые пять итераций проведенного эксперимента. В~процессе разработки 
вариантов индикатора ВРП участвовало пять экспертов (обозначим их как {A}, 
{Б}, {В}, {Г} и~{Д}).
   
%   \smallskip
   
   \textit{Первая итерация}. Эксперт~{А} создал первый вариант при следующих 
условиях:\\[-14pt]
   \begin{itemize}
\item для вычисления значений первого варианта индикатора ВРП использовались статьи 
сотрудников одного из подразделений ИПИ РАН, напечатанные в журналах и сборниках 
в 2009~г.\ и введенные в базы данных системы информационного мониторинга;
\item возрастная группа\footnote{В эксперименте учитывались 14~возрастных групп: 20--24, 
25--29 и далее до группы 85--89~лет.} каждого соавтора статьи получала за одну статью 
1~балл;
\pagebreak
\item отсутствовала нормализация относительно чис\-лен\-ности возрастных групп.
\end{itemize}

   В это же время (на этой же итерации) эксперты~{Б} и {В} создали совместно и 
согласованно второй вариант индикатора ВРП при тех же первом и третьем условиях, но 
второе из трех вышеперечисленных условий отличалось: возрастная группа каждого 
соавтора получала $1/N$ балла за статью, у которой $N$ соавторов.
   
   На первой итерации порождается два треугольника Фреге (следовательно, $S_i=2$ для 
$i = 1$), для них строятся два дескриптора и вычисляются два значения $F_{\mathrm{con}}$ в точке 
($i = 1$, $p = 0$):
   \begin{enumerate}[(1)]
\item первое значение $F_{con}$ равно 1, так как дескриптор эксперта~{А} на 
первой итерации является авторским; это значение обозначено треугольником с 
вершиной ниже его основания (см.\ рис.~2, на котором все значения~$F_{\mathrm{con}}$ для 
дескрипторов первого варианта индикатора, порожденных на последующих итерациях, 
обозначены таким же треугольником);
\item второе значение равно 2, так как дескриптор экспертов~{Б} и~{В} на 
первой итерации является коллективным; это значение обозначено треугольником с 
вершиной выше его основания (см.\ рис.~2, на котором все значения $F_{\mathrm{con}}$ для 
дескрипторов второго варианта индикатора, порожденных на последующих итерациях, 
обозначены таким же треугольником).
\end{enumerate}

%   \smallskip
   
   \textit{Вторая итерация}. На второй итерации эксперт~{В} решил изменить свою 
точку зрения и принять точку зрения эксперта~{А}. Иначе говоря, эксперт~{В} на 
второй итерации отказывает в согласовании второму варианту индикатора ВРП, так как 
считает правильным добавлять всем соавторам по одному\linebreak
 баллу, и согласовывает первый 
вариант индика\-тора ВРП. Первый вариант, сгенерированный\linebreak экспертом~{А} 
становится коллективным, а второй~--- ав\-тор\-ским. Эти варианты индикатора ВРП 
идентичны вариантам первой итерации, что обозначено двумя серыми треугольниками и двумя 
пересе\-ка\-ющи\-ми\-ся сплошными стрелками (см.\ рис.~2).
   
   Так как имеющиеся концепты, имена и состояния денотатов на второй итерации не 
изменялись, а новые не формировались, то новые дескрипторы не строились и число 
порожденных треугольников Фреге на второй итерации равно нулю ($S_i = 0$ для $i = 2$), 
значения функции $F_{\mathrm{con}}$ не определены в точке ($i = 2$, $p = 0$) при $S_2 = 0$, но два 
значения этой функции определены в точке ($i = 1$, $p = 1$), так как $S_1 = 2$. Эти значения 
равны~2 и~1 (см.\ рис.~2).
   
%   \smallskip
   
   \textit{Третья итерация}. Эксперты~{А}, {Б} и~{В} одновременно 
принимают решение учесть численности возрастных групп при вычислении значений своих 
вариантов индикатора ВРП, что находит отражение в изменении соответствующих 
алгоритмов программ вычисления их значений с целью нормализации. При этом на третьей 
итерации эксперты связывают новые порожденные концепты с концептами, созданными на 
первой итерации и изменившими степень согласованности на второй итерации, 
отношениями наследования, что обозначено двумя точечными стрелками. Так как 
формируются два новых концепта, то на третьей итерации для них строятся два дескриптора 
проективного словаря ($S_i = 2$ для $i = 3$). При этом вычисляются четыре значения 
функции $F_{\mathrm{con}}$: по два значения в точках ($i = 3$, $p = 0$) и ($i = 1$, $p = 2$).


   \smallskip
   
   \textit{Четвертая итерация}. Эксперты~{А}, {Б} и~{В} одновременно 
приняли решение учитывать только те статьи, которые опубликованы в журналах из перечня 
ВАК. На этой итерации эксперты связывают новые порожденные концепты с концептами, 
созданными на третьей итерации, отношениями наследования, что обозначено еще двумя 
точечными стрелками. Так как формируются два новых концепта, то на четвертой итерации 
для них строятся два дескриптора проективного словаря ($S_i = 2$ для $i = 4$). При этом 
вычисляются шесть значений функции $F_{\mathrm{con}}$: по два значения в точках ($i = 4, p = 0$), 
($i = 3, p = 1$) и ($i = 1, p = 3$).
   
   \smallskip
   
   \textit{Пятая итерация}. На этой итерации разработки эксперты оставляют неизменными 
состояния денотатов, концепты и имена двух вариантов индикатора ВРП, учитывающие 
численность групп и перечень журналов ВАК, но точки зрения экспертов изменяются 
следующим образом. Эксперт~{В} отказывается от варианта эксперта~{А}. 
Следовательно, этот концепт становится личностным концептом эксперта~{А}. Другой 
вариант приобретает двух новых сторонников~--- экспертов~{Г} и~{Д}, что 
приводит к изменению степени его согласованности. У~этих двух концептов изменилась 
только степень согласованности, но они остались идентичными концептам, 
сгенерированным на четвертой итерации, что условно обозначено двумя серыми 
треугольниками на пятой итерации и двумя пересекающимися сплошными стрелками между 
четвертой и пятой итерациями (см.\ рис.~2).
   
   Так как новые концепты не формировались, а существующие не изменялись, то на пятой 
итерации новые дескрипторы не строились и число по\-рож\-ден\-ных треугольников Фреге 
равно нулю ($S_i = 0$ для $i = 5$). Значение функции $F_{\mathrm{con}}$ не определено для пары 
($i = 5, p = 0$). При этом вычисляются шесть значений функции $F_{\mathrm{con}}$ в других точках: 
по два значения в точках ($i = 4, p = 1$), ($i = 3, p = 2$) и ($i = 1, p = 4$) (см.\ рис.~2).
   
   Все значения функции $F_{\mathrm{con}}$ для новых или измененных концептов обозначены 
черными треугольниками, а им идентичные~--- серыми треугольниками (у этих концептов 
изменилась только степень их согласованности экспертами, формирующими ЦСЗ).
   
   Рисунок~2 содержит не все 20~значений функции степени согласованности~$F_{\mathrm{con}}$, 
вычисленных на первых пяти итерациях, а только те ее 10~значений, соответствующие 
дескрипторы которых либо связаны отношениями наследования, либо изменили степень 
согласованности.
   
   Рассмотренный эксперимент по компьютерному представлению экспертных знаний и 
вы\-чис-\linebreak ле\-нию значений функции степени со\-гла\-со\-ван\-ности~$F_{\mathrm{con}}$ иллюстрирует 
потенциал использования пространства Фреге для построения функциональных 
характеристик процесса формирования индикаторов и областей определения 
соответствующих функций.
   
   Таким образом, получены три из четырех компонентов решения сформулированной 
проблемы представления ЦСЗ, которые представляют собой теоретическую часть решения 
этой проблемы. 
   
\section{Проективный словарь лингвистического обеспечения}
   
   Данный раздел посвящен описанию выбранного подхода к технической реализации 
положений~6--9, которые были перечислены в разд.~3.\linebreak Рассматривается реализация именно 
этих четы-\linebreak рех
положений, поскольку они формулируют непосредственные требования к 
созданию проективного словаря лингвистического обеспечения\linebreak
 сис\-те\-мы информационного 
мониторинга (как чет\-вер\-то\-го компонента решения сформулированной проб\-лемы).
   
   В процессе создания проективного словаря с помощью реляционной СУБД Microsoft SQL 
\mbox{Server} разработано шесть таблиц реляционной базы данных (РБД), названия которых можно 
увидеть в первой строке табл.~1. Столбцы, кроме первого, соответствуют 
таблицам РБД. Первый столбец содержит номера тех четырех 
положений (6, 7, 8 и~9), на основе которых разрабатывались шесть таблиц РБД.
   
   В ячейке этой таблицы может стоять символ <<+>>, либо ячейка может быть не 
заполнена. Незаполненность ячейки говорит о том, что для реализации положения 
соответствующая таблица РБД не используется. Символ <<+>> в ячейке строки с 
номером~$N$ говорит о том, что для технической реализации положения с номером~$N$ 
всегда используется таблица РБД из соответствующего этой ячейке столбца.

\begin{table*}\small
\begin{center}
\Caption{Взаимосвязь положений из разд.~3 с таблицами проективного словаря}
\vspace*{2ex}
\begin{tabular}{|c|c|c|c|c|c|c|}
\hline
Номер
положения&Денотат&Рубрика&Дефиниция&Коллектив&Дескриптор&Эксперт\\
\hline
6&+&&&&+&+\\
7&+&+&&&+&+\\
8&+&+&+&&+&+\\
9&+&+&+&+&+&+\\
\hline
\end{tabular}
\end{center}
\end{table*}

   Таблица \textit{Денотат} предназначена для описания состояний денотатов индикаторов. 
В столбце \textit{Денотат} проставлен символ <<+>> для всех четырех положений. Это 
означает, что для любого из четырех возможных вариантов окончания стадии 
интерпретации в таблице \textit{Денотат} регистрируются: идентификатор денотата 
индикатора, идентификатор состояния денотата, ссылки на предыдущее и последующее 
состояния денотата, момент времени окончания стадии интерпретации, ссылки на описание 
используемого варианта программы вычисления значений нового индикатора, данные, 
которые обрабатываются этим вариантом программы, и полученные результаты вычисления 
значений индикатора, а также персональный идентификатор того эксперта, который является 
автором созданного состояния денотата.
   
   Таблица \textit{Рубрика} предназначена для указания одной или нескольких рубрик 
используемых сис\-тем классификации индикаторов с помощью ссылок на эти системы (а 
также используемых версий систем классификации), отобранных в результате 
интерпретации состояний денотатов индикаторов. В~столбце \textit{Рубрика} проставлен 
символ <<+>> для положений~7, 8 и~9. Это означает, что для любого из этих трех 
возможных вариантов окончания стадии интерпретации в таблице \textit{Рубрика} 
регистрируются: идентификатор денотата индикатора, идентификатор состояния денотата, 
идентификаторы используемых систем классификации индикаторов, версий систем 
классификации, рубрики как минимум из одной системы, ссылки на ключевые слова из 
терминологических портретов этих рубрик, отобранных экспертом, а также персональный 
идентификатор этого эксперта.
   
   Таблица \textit{Дефиниция} предназначена для описания структурированных и 
параметризованных дефиниций, полученных в результате интерпретации состояний 
денотатов индикаторов. Подобная дефиниция представляет собой совокупность текстовых 
фрагментов, в которую включены параметры используемых вариантов программ и 
параметры отбора тех данных, которые включены в это состояние денотата~\cite{23-zat}. 
В~столбце \textit{Дефиниция} проставлен символ <<+>> для положений~8 и~9. Это 
означает, что для любого из этих двух возможных вариантов окончания стадии 
интерпретации в таблице \textit{Дефиниция} регистрируются: идентификатор денотата 
индикатора, идентификатор состояния денотата, ссылка на структурированную и 
параметризованную дефиницию, значения параметров ис\-поль\-зу\-емых вариантов программ и 
параметров отбора данных, выбранные экспертом, а также персональный идентификатор 
этого эксперта.
   
   Таблица \textit{Коллектив} предназначена для хранения списка тех экспертов, с 
которыми были согласованы построенный денотат индикатора, описание его 
интерпретируемого состояния, в том числе рубрики систем классификации, дефиниция и 
имя. В~столбце \textit{Коллектив} поставлен символ <<+>> для положения~9. Это означает, 
что для четвертого варианта\linebreak окончания стадии интерпретации в таблице \textit{Коллектив} 
регистрируются: идентификатор денотата индикатора, идентификатор состояния денотата,\linebreak 
персональный идентификатор эксперта-автора описания этого состояния, а также 
идентификаторы тех экспертов, с которыми были согласованы построенный денотат и 
описание его интерпретируемого состояния.
   
   Для последних двух столбцов символ <<+>> проставлен для всех четырех положений. 
Это означает, что в таблицах \textit{Дескриптор} и \textit{Эксперт} для любого из 
четырех возможных вариантов окончания стадии интерпретации в этих таблицах 
регистрируются описание каждого дескриптора и данные об экспертах.
   
   Необходимо заметить, что строки в таблицах РБД никогда не редактируются и не 
удаляются. Любое изменение формируемого индикатора по\-рож\-да\-ет одну или несколько 
новых строк, соответствующих новому состоянию денотата, концепту или имени этого 
индикатора. Потерявшие актуальность строки остаются в таблице и помечаются как 
<<закрытые>>. Это позволяет в любой момент времени восстановить любое предыдущее 
состояние формируемого индикатора, а также проследить его эволюцию во времени.
   
   Первый вариант технической реализации выбранного подхода к построению таблиц РБД 
использовался для проведения эксперимента по реализа\-ции компьютерного представления 
экспертных знаний об индикаторах в динамике их формирования. Этот вариант технической 
реализации охватывал не все предусмотренные возможности таблиц РБД. В~частности, 
использовались только три из четырех возможных вариантов окончания стадии 
интерпретации, кроме второго, т.\,е.\ в описании каждого состояния формируемого 
индикатора отсутствовали рубрики систем классификации индикаторов и ссылки на 
ключевые слова из терминологических портретов этих рубрик.

\section{Заключение}

   Основные полученные результаты, рассмотренные в статье, заключаются в следующем. 
Во-пер\-вых, определены две основные категории моделей\linebreak
формирования новых или 
пополнения существу\-ющих систем знаний (концептуальные и пред\-мет\-но-ори\-ен\-ти\-ро\-ван\-ные), 
которые отличаются по\linebreak
степени общности. Во-вторых, на основе концептуальных 
семиотических моделей представления процессов формирования знаний из 
   работ~\cite{16-zat, 17-zat} дано описание предметно-ориентированных моделей для 
случая формирования или пополнения системы экспертных знаний об индикаторах 
мониторинга. В-третьих, предложено решение проблемы представления ЦСЗ об индикаторах 
мониторинга, включающее теоретическую и прикладную части ее решения.
   
   В теоретической части решения определена функция степени согласованности 
формируемых индикаторов между экспертами в динамике их формирования. В~прикладной 
части решения предложен подход к технической реализации разработанных моделей.
   
   Проведенный эксперимент позволяет сделать следующий вывод: создание семиотических 
моделей компьютерного представления экспертных знаний об индикаторах, разработка на их 
основе проективного словаря и предлагаемый подход к его технической реализации дают 
возможность группе экспертов совместно разрабатывать новые алгоритмы для вычисления 
значений прог\-рам\-мно-ори\-ен\-ти\-ро\-ван\-ных индикаторов и развивать систему экспертных 
знаний об индикаторах мониторинга. Проективный словарь предназначен для уменьшения 
лакунарности этой системы, и процесс его формирования отражает целенаправленную 
деятельность экспертов по заполнению лакун в системе экспертных знаний об индикаторах 
теми структурными элементами знаний, которые оказались необходимы для мониторинга и 
индикаторного оценивания ПЦД. При этом у экспертов имеется возможность фиксировать в 
проективном словаре этапы формирования смыслового содержания индикаторов, а также 
различия в рубриках, дефинициях, именах формируемых индикаторов и степень их 
согласованности между экспертами.

{\small\frenchspacing
{ %\baselineskip=12pt
\addcontentsline{toc}{section}{Литература}
\begin{thebibliography}{99}

\bibitem{1-zat}
Концепция реформирования бюджетного процесса в Российской Федерации в 2004--2006~годах. 
Одобрена постановлением Правительства РФ от 22~мая 2004~года №\,249 <<О мерах по повышению 
результативности бюджетных расходов>>. {\sf http://government.consultant.ru/page.aspx?787610} (дата 
обращения: 09.07.2011).

\bibitem{2-zat}
Программа фундаментальных научных исследований государственных академий наук на 
2008--2012~годы.~--- М.: Наука, 2008.

\bibitem{3-zat}
CORDIS ICT Programme Home. {\sf http://cordis.\linebreak europa.eu/fp7/ict/programme/ home\_en.html} (дата 
обращения: 23.05.2011).

\bibitem{4-zat}
ICT FP7 Work Programme 2007-08. {\sf ftp.cordis. europa.eu/pub/fp7/ict/docs/ict-wp-2007-08\_en.pdf} (дата 
обращения: 23.05.2011).

\bibitem{5-zat}
ICT FP7 Work Programme 2009-10. {\sf ftp.cordis. europa.eu/pub/fp7/ict/docs/ict-wp-2009-10\_en.pdf} (дата 
обращения: 23.05.2011).

\bibitem{6-zat}
ICT FP7 Work Programme 2011-12. {\sf ftp.cordis. europa.eu/pub/fp7/ict/docs/ict-wp-2011-12\_en.pdf} 
(дата обращения: 23.05.2011).

\bibitem{7-zat}
FP7 Exploratory Workshop 4 <<Knowledge Anywhere Anytime>>. 
{\sf http://cordis.europa.eu/ist/directorate\_f/ f\_ws4.htm} (дата обращения: 23.05.2011).

\bibitem{8-zat}
\Au{Зацман И.\,М., Косарик В.\,В., Курчавова~О.\,А.}
Задачи представления личностных и коллективных концептов в цифровой среде~// Информатика и её 
применения, 2008. Т.~2. Вып.~3. С.~54--69.
\bibitem{9-zat}
\Au{Nonaka I.}
The knowledge-creating company~// Harvard Business Rev., 1991. Vol.~69. No.\,6. P.~96--104.

\bibitem{10-zat}
\Au{Nonaka I., Takeuchi H.}
The knowledge-creating company.~--- Oxford; N.Y.: Oxford University Press, 1995. 
(Пер. на русск. яз.: \Au{Нонака И., Такеучи Х.} Компания~--- создатель знания.~--- М.: Олимп-бизнес, 2003.)

\bibitem{11-zat}
Knowledge emergence~/ Eds. I.~Nonaka, T.~Nishiguchi.~--- Oxford; N.Y.: Oxford University Press, 2001.

\bibitem{12-zat}
\Au{Wierzbicki A.\,P., Nakamori Y.}
Basic dimensions of creative space~// Creative space: Models of creative processes for knowledge 
civilization age~/ Eds. A.\,P.~Wierzbicki,  Y.~Nakamori.~--- Berlin--Heidelberg: Springer Verlag, 2006. 
P.~59--90.

\bibitem{13-zat}
\Au{Wierzbicki A.\,P., Nakamori Y.}
Knowledge sciences: Some new developments~// Zeitschrift f$\ddot{\mbox{u}}$r Betriebswirtschaft, 2007. 
Vol.~77. No.\,3. P.~271--295.

\bibitem{14-zat}
\Au{Ren H., Tian J., Nakamori~Y., Wierzbicki~A.\,P.}
Electronic support for knowledge creation in a research institute~// J.~Syst. Sci. Syst. Eng. 2007. 
Vol.~16. No.\,2. P.~235--253.

\bibitem{15-zat}
\Au{Зацман И.\,М.}
Концептуальный поиск и качество информации.~--- М.: Наука, 2003.
\pagebreak

\bibitem{16-zat}
\Au{Зацман И.\,М.}
Семиотическая модель взаимосвязей концептов, информационных объектов и компьютерных 
кодов~// Информатика и её применения, 2009. Т.~3. Вып.~2. С.~65--81.

\bibitem{17-zat}
\Au{Зацман И.\,М.}
Нестационарная семиотическая модель компьютерного кодирования концептов, информационных 
объектов и денотатов~// Информатика и её применения, 2009. Т.~3. Вып.~4. С.~87--101.

\bibitem{18-zat}
\Au{Zatsman I., Durnovo A.}
Incompleteness problem for indicators system of research programme~// 11th Conference (International) on 
Science and Technology Indicators (STI'2010): Book of Abstracts.~--- Leiden: Universiteit Leiden, 2010. 
P.~309--311.

\bibitem{19-zat}
\Au{Успенский В.\,А.}
К публикации статьи Г.~Фреге <<Смысл и денотат>>~// Семиотика и информатика.~--- М.: 
Языки русской культуры, 1997. Вып.~35. С.~351--352.

\bibitem{20-zat}
\Au{Фреге Г.}
Смысл и денотат~// Семиотика и информатика.~--- М.: Языки русской культуры, 1997. Вып.~35. 
С.~352--379.

\bibitem{21-zat}
\Au{Фреге Г.}
Понятие и вещь~// Семиотика и информатика.~--- М.: Языки русской культуры, 1997. 
Вып.~35. С.~380--396.

\bibitem{22-zat}
\Au{Зацман И.\,М.}
Категоризация результатов и индикаторов программ научных исследований в информационных 
системах мониторинга~// Системы и средства информатики.~--- М.: ИПИ РАН, 2009. Доп. вып. 
С.~200--219.

\label{end\stat}

\bibitem{23-zat}
\Au{Кожунова О.\,С.}
Технология разработки семантического словаря системы информационного мониторинга: 
Автореф. дисс. \ldots канд. техн. наук.~--- М.: ИПИ РАН, 2009. 23~с.
 \end{thebibliography}
}
}


\end{multicols}       