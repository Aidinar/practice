 \def\stat{stef+sushko}

\def\tit{ОБРАТИМОЕ СЖАТИЕ ДАННЫХ ПОСРЕДСТВОМ УНИВЕРСАЛЬНОГО АРИФМЕТИЧЕСКОГО КОДИРОВАНИЯ}

\def\titkol{Обратимое сжатие данных посредством универсального арифметического кодирования}

\def\aut{А.\,И.~Стефанович$^1$, Д.\,В.~Сушко$^2$}

\def\autkol{А.\,И.~Стефанович, Д.\,В.~Сушко}

\titel{\tit}{\aut}{\autkol}{\titkol}

\index{Стефанович А.\,И.}
\index{Сушко Д.\,В.}
\index{Stefanovich A.\,I.}
\index{Sushko D.\,V.}


%{\renewcommand{\thefootnote}{\fnsymbol{footnote}} \footnotetext[1]
%{Работа выполнена при финансовой поддержке РФФИ (проекты 16-07-00677 
%и~15-37-20611-мол\_а\_вед).}}


\renewcommand{\thefootnote}{\arabic{footnote}}
\footnotetext[1]{Институт проблем информатики Федерального исследовательского центра
 <<Информатика и~управление>> Российской академии наук, \mbox{astefanovich@ipiran.ru}}
\footnotetext[2]{Институт проблем информатики Федерального исследовательского центра 
<<Информатика и~управление>> Российской академии наук, \mbox{dsushko@ipiran.ru}}


\Abst{Рассмотрен общий подход к~задаче обратимого сжатия, т.\,е.\
 сжатия без потерь, цифровых данных, основанный на универсальном 
 арифметическом кодировании данных с~неизвестной статистикой. 
 Для описания данных используется модель источника с~вычислимой 
 последовательностью состояний. В~рамках этого подхода сформулированы задачи, 
 решение которых для данных конкретного типа позволяет получить конкретные методы 
 и~алгоритмы сжатия. В~качестве объекта исследования рассмотрены данные 
 компьютерной томографии. Предложены два метода обратимого сжатия томограмм. 
 Первый предполагает кодирование ошибок предсказания, второй~--- 
 кодирование компонент двумерного дискретного вей\-в\-лет-пре\-об\-ра\-зо\-ва\-ния. 
 Проведено подробное исследование этих методов, построены эффективные 
 алгоритмы их реализации и~получены индивидуальные оценки скорости 
 кодирования алгоритмов. Представлены результаты сравнения скоростей 
 кодирования томограмм построенными алгоритмами и~алгоритмами стандарта JPEG~2000. 
 Результаты демонстрируют высокое качество построенных алгоритмов, а~также 
 свидетельствуют о~больших потенциальных возможностях рассмотренного подхода в~целом.}

\KW{обратимое сжатие данных; сжатие без потерь; универсальное кодирование; арифметическое кодирование; компьютерная томограмма}

\DOI{10.14357/19922264170103} 


\vskip 10pt plus 9pt minus 6pt

\thispagestyle{headings}

\begin{multicols}{2}

\label{st\stat}

\section{Введение}
%\label{sec0}

На протяжении последних нескольких десятилетий наблюдается бурный рост 
объема цифровых данных, накапливаемых в~результате проведения различных 
научных экспериментов, медицинских исследований и~т.\,д. Необходимость долгосрочного 
хранения (архивации) и~обмена такими данными делают задачу их сжатия (кодирования 
в~целях уменьшения объема данных) актуальной современной задачей. Во 
многих случаях важным дополнительным требованием к~процедуре сжатия является 
ее обратимость (т.\,е.\ отсутствие искажений, или потерь при кодировании). 
Это требование в~случае данных медицинских исследований часто продиктовано 
соображениями законодательного характера, а~в~случае данных научных исследований~--- 
высокой стоимостью и~трудоемкостью эксперимента и/или уникальностью данных.

В настоящей работе рассмотрен общий метод (общий подход), 
предназначенный для решения задачи обратимого сжатия цифровых данных. 
Данный метод, основанный на использовании универсального арифметического 
кодирования, впервые был предложен в~работе~\cite{b01}. Построение в~рамках 
общего подхода некоторого конкретного метода, предназначенного для сжатия 
данных определенного типа, связано с~необходимостью решить ряд задач по 
адаптации общего метода к~таким данным. Постановки соответствующих задач 
приведены в~работе.

В качестве объекта исследования в~работе используются данные компьютерной 
томографии (томограммы). Для данных указанного типа в~рамках общего подхода 
построены два различных метода\linebreak сжатия. В~работе проведено подробное 
исследова\-ние этих методов, предложены эффективные алго\-ритмы их реализации и~получены 
индивидуальные оценки скорости кодирования (степени сжатия) этих алгоритмов.

Поскольку томограммы представляют собой изображения, их обратимое сжатие может 
быть осуществлено альтернативным способом в~рамках группы методов стандарта JPEG~2000. 
В~ходе проведенных исследований было осуществлено сжатие томограмм посредством 
эталонной реализации (Jasper) стандарта JPEG~2000 и~произведено сравнение 
скоростей кодирования алгоритмов JPEG~2000 и~эффективных алгоритмов, предложенных 
в~работе. Данное сравнение продемонстрировало, во-пер\-вых, 
качество разработанных алгоритмов и,~во-вто\-рых, высокий потенциал общего 
метода в~целом.

Работа имеет следующую структуру. В разд.~2 приведены 
необходимые сведения об арифметическом кодировании, описана общая 
схема универсального кодирования и~в~общем виде сформулированы задачи, решение 
которых необходимо при адаптации общего метода универсального кодирования 
к~конкретному типу данных. Раздел~3 содержит краткую информацию 
о~том, что пред\-став\-ля\-ют собой компьютерные томограммы. В~разд.~4 
общая схема универсального кодирования адаптирована таким образом, чтобы 
ее можно было применить для сжатия значений ошибок предсказания компьютерной 
томограммы. Основным результатом раздела является построение метода сжатия, 
основанного на кодировании ошибок предсказания, и~построение эффективных оценок 
ско\-рости кодирования метода. В~разд.~5 аналогичные результаты получены 
для метода сжатия, ориентированного на кодирование значений компонент дискретного 
вей\-в\-лет-пре\-об\-ра\-зо\-ва\-ния томограммы. 
В~заключение приведено сравнение эф\-фек\-тив\-ности построенных методов и~методов JPG~2000.

\vspace*{-9pt}

\section{Универсальное арифметическое кодирование}
%\label{sec1}

В настоящем разделе приведено описание общей схемы предлагаемого метода 
универсального арифметического кодирования. Обозначены постановки основных задач, 
решение которых обеспечивает эффективное применение этой схемы для сжатия данных 
конкретного типа.

\vspace*{-9pt}

\subsection{Арифметическое кодирование}
%\label{sec11}

Приведем сведения об арифметическом кодировании в~необходимом для дальнейшего 
изложения объеме. Подробному рассмотрению арифметического кодирования посвящена, 
например, работа~\cite{b02}.

Пусть ${\cal A}= \{a\}$~--- некоторое конечное множество (алфавит), 
состоящее из $A\doteq|{\cal A}|$ элементов, и~пусть  $\mathbf{x}\hm=\{x_n\}$,  
$n\hm=0,\ldots,N-1$,~--- подлежащая кодированию последовательность элементов 
множества~${\cal A}$. Процесс кодирования заключается в~последовательном 
просмотре всех значений~$x_n$, вычислении кодовой вероятности~$Q(\mathbf{x})$ 
(положительного вещественного числа, не превышающего единицы) и~формировании по 
этой кодовой вероятности двоичного кодового слова (результата сжатия) 
длины~$L(\mathbf{x})$ битов:
\begin{equation}
\label{eq1}
L(\mathbf{x}) = \left[-\fr{\log_{2}Q(\mathbf{x})}{2}\right]_{-}+1 \leq 
-\log_{2}Q(\mathbf{x})+2\,.
\end{equation}

\columnbreak

\noindent
Здесь и~далее $[\cdot]_{-}$~--- целая часть числа. Вычисление кодовой 
вероятности осуществляется рекуррентно. Начальная кодовая вероятность 
выбирается равной единице  ($Q_{-1}\hm=1$). 
В~момент поступления на вход кодера очередного значения~$x_n$ 
кодеру долж\-но быть известно (задано заранее и/или сфор\-мировано в~процессе 
кодирования предыдущих элементов) условное кодовое распределение 
вероятностей  $\{q_n(a|x_{n-1},\ldots,x_0)$, 
$a\hm\in{\cal A}\}$. {\it Условное кодовое распределение}~--- 
это набор неотрицательных вещественных чисел, таких что
\begin{equation}
\label{eq2}
\sum_{a\in{\cal A}}q_n(a|x_{n-1},\dots,x_0) = 1\,;
\end{equation}
кроме того, равенство $q_n(a|x_{n-1},\dots,x_0)\hm=0$ для некоторого
 конкретного значения~$a$ допустимо только в~том случае, если 
 выполнение равенства $x_n\hm=a$ невозможно априори. Шаг рекурсии заключается 
 в~умножении текущей кодовой вероятности на условную кодовую вероятность 
 значения~$x_n$:
\begin{multline*}
Q_n\left(x_0,\dots,x_n\right)={}\\
{}=Q_{n-1}\left(x_0,\dots,x_{n-1}\right)
q\left(x_n|x_{n-1},\dots,x_0\right)\,.
\end{multline*}
Результатом выполнения~$N$~шагов рекурсии является вычисление кодовой вероятности 
всей последовательности:
\begin{equation}
\label{eq3}
Q(\mathbf{x}) =\prod\limits_{n=0}^{N-1} q_n\left(x_n|x_{n-1},\dots,x_0\right)\,.
\end{equation}
Детали процедуры формирования кодового слова по кодовой вероятности не 
принципиальны для рассмотрения и~опускаются.

Восстановление исходных данных по кодовому слову осуществляется 
декодером последовательно и~без задержки. В~момент восстановления очередного 
значения~$x_n$ декодеру уже известны все предыдущие значения $\{x_0,\dots,x_{n-1}\}$ 
и,~кроме того, должно быть известно условное кодовое распределение 
$\{q_n(a|x_{n-1},\dots,x_0)$, $a\hm\in{\cal A}\}$, использованное ранее 
в~процессе кодирования. Это позволяет декодеру восстановить значение~$x_n$.

Таким образом, ключевую роль в~процессе арифметического кодирования играют 
условные кодовые распределения вероятностей $\{q_n(a|x_{n-1},\dots,x_0)$,
$a\hm\in{\cal A}$, $n\hm=0,\ldots,N-1\}$, выбор которых определяет длину 
кодового слова, т.\,е.\ степень сжатия исходных данных. При этом 
построение условных кодовых распределений, обеспечивающих получение 
возможно более коротких кодовых слов для входных данных с~неизвестной 
(не полностью известной) статистикой,~--- задача универсального кодирования.

\subsection{Статистическая модель}

В качестве статистической модели исходных данных используем так 
называемую модель \textit{источника с~вычислимой последовательностью состояний}. 
В~основе модели лежит следующее предположение: вероятность того, что значение~$x_n$ 
очередного элемента последовательности равна заданному значению
$a\hm\in{\cal A}$, зависит только от значений~$\tau$ предшествующих элементов 
последовательности, т.\,е.\ 
$p(x_n=a)\hm=p(x_n|x_{n-1},\ldots,x_{n-\tau})$. Пусть~${\cal S}$~--- 
некоторое подмножество множества  ${\cal A}^\tau \hm= 
\underbrace{{\cal A}\times\dots\times{\cal A}}_{\tau}$. 
Назовем ${\cal S}\hm\subset{\cal A}^\tau $ состоянием (источника), если
\begin{multline*}
p\left(x_n|x'_{n-1},\dots,x'_{n-\tau}\right) = 
p\left(x_n|x''_{n-1},\dots,x''_{n-\tau}\right)\\
\forall \, \left(x'_{n-1},\dots,x'_{n-\tau}\right),\,
\left(x''_{n-1},\dots,x''_{n-\tau}\right)\in{\cal S}\,.
\end{multline*}
Для условного распределения вероятностей, соответствующего состоянию~${\cal S}$, 
используем обозначение~$p(a|{\cal S})$. Множество состояний ${\frak S}\hm=\{\cal S\}$ 
назовем полным множеством независимых состояний, если
$$
\bigcup\limits_{{\cal S}\in{\frak S}}{\cal S} = {\cal A}^\tau\,;
\quad  \quad
{\cal S}'\bigcap{\cal S}'' = \varnothing\quad
\forall\,{\cal S}',{\cal S}''\in{\frak S}\,.
$$
Все рассматриваемые далее множества состояний являются полными и~независимыми.

Название модели~--- модель источника с~вы\-чис\-ли\-мой последовательностью состояний~--- 
связано со следующей возможной ее интерпретацией. Элементы 
последовательности~$\mathbf{x}\hm=\{x_n\}$ один за другим <<порождаются>> 
источником данных, который в~каж\-дый <<момент времени~$n$>>  
находится в~некотором состоянии~${\cal S}$ из множества состояний 
источника~${\frak S}$; при этом $p(x_n\hm=a)\hm= p(x_n|{\cal S})$. 
Для краткости можно говорить об элементе~$x$ последовательности, <<порожденном>> 
источником в~состоянии~${\cal S}$, как об элементе состояния~${\cal S}$ и~записывать 
это в~виде $x\hm\in{\cal S}$.

В соответствии с~принятой моделью данных естественно использовать общее условное 
кодовое распределение вероятностей $\{q(a|{\cal S}),\,a\hm\in{\cal A}\}$ 
при ко\-ди\-ро\-ва\-нии-де\-ко\-ди\-ро\-ва\-нии всех значений, <<порождаемых>> 
источником в~каждом отдельном состоянии. Всего в~процессе кодирования 
используется $S\doteq|{\frak S}|$ различных условных кодовых 
распределений вероятностей (по числу состояний источника). Поскольку 
арифметическое кодирование осуществляется последовательно, а~декодирование~--- 
последовательно и~без задержки, в~момент ко\-ди\-ро\-ва\-ния-де\-ко\-ди\-ро\-ва\-ния 
очередного значения все предыдущие значения уже известны как кодеру, так и~декодеру. 
Поэтому как кодер, так и~декодер в~состоянии вычислить текущее состояние источника 
и~использовать соответствующее условное кодовое распределение.

\textit{Скоростью кодирования} (средней скоростью кодирования)~$V$ называется отношение 
длины кодового слова~$L(\mathbf{x})$ к~числу элементов~$N$ кодируемой 
последовательности; единица измерения скорости кодирования~--- 
бит/пик\-сель (б/п). С~учетом принятых предположений 
из формул~(\ref{eq1}) и~(\ref{eq3}) с~точностью до малого члена порядка~$\sim 2/N$ 
имеем:
\begin{equation}
\label{eq4}
V(\mathbf{x}) = \sum\limits_{{\cal S}\in{\frak S}}\fr{N({\cal S})}{N}
\sum\limits_{x\in{\cal A}}\fr{N(x|{\cal S})}{N({\cal S})} 
\left[-\log_{2}q(x|{\cal S})\right]\,,
\end{equation}
где $N({\cal S})$~--- число элементов в~состоянии~${\cal S}$; $N(x|{\cal S})$~--- 
число элементов в~состоянии~${\cal S}$, принимающих значение~$x$; внешняя 
сумма берется по всем состояниям источника; внутренняя сумма~--- 
по всем встречающимся в~данном состоянии значениям. Если использовать соглашение о~том, 
что $0\cdot\log0\hm=0$, то внутреннюю сумму можно распространить на 
все множество значений~$\cal A$. Действительно, в~силу предъявляемых 
к~условным кодовым вероятностям требований равенство $q(x|{\cal S})\hm=0$ 
влечет $N(x|{\cal S})\hm=0$.

Величины $N({\cal S})/N$  (${\cal S}\hm\in{\frak S} $) и~$N(x|{\cal S})/N({\cal S})$  
($x\hm\in{\cal S}\,,{\cal S}\hm\in{\frak S}$) образуют соответственно 
частотное распределение для состояний и~условные частотные распределения значений 
в~со\-сто\-яни\-ях. Используя для этих величин обозначения~$f(\cal S)$ 
и~$f(x|\cal S)$, перепишем формулу~(\ref{eq4}) для скорости кодирования в~виде:
\begin{equation}
\label{eq5}
V(\mathbf{x}) = \sum\limits_{{\cal S}\in{\frak S}} f({\cal S}) V(\mathbf{x}|{\cal S})\,,
\end{equation}
где $V(\mathbf{x}|{\cal S})$--- скорость кодирования подпоследовательности элементов состояния~${\cal S}$, 
или скорость кодирования состояния~${\cal S}$:
\begin{equation}
\label{eq6}
V(\mathbf{x}|{\cal S}) = \sum\limits_{x\in{\cal A}} 
f(x|{\cal S}) \left[-\log_{2}q(x|{\cal S})\right]\,.
\end{equation}
 Формулу для скорости кодирования 
состояния можно тождественно переписать в~виде суммы двух слагаемых:
\begin{equation}
\label{eq7}
V(\mathbf{x}|{\cal S}) = H(\mathbf{x}|{\cal S})+R(\mathbf{x}|{\cal S})\,,
\end{equation}
где
\begin{align}
\label{eq8}
H(\mathbf{x}|{\cal S}) &= \sum\limits_{x\in{\cal A}} f(x|{\cal S}) 
\left[-\log_{2}f(x|{\cal S})\right]\,;
\\
\label{eq9}
R(\mathbf{x}|{\cal S}) &= \sum\limits_{x\in{\cal A}} f(x|{\cal S}) 
\left[-\log_{2}\fr{q(x|{\cal S})}{f(x|{\cal S})}\right]\,.
\end{align}
Первое слагаемое $H(\mathbf{x}|{\cal S})$~--- 
это \textit{квазиэнтропия} (или эмпирическая энтропия) состояния. 
Квазиэнтропия не зависит от условного кодового распределения и,~очевидно, 
является неотрицательной  ($H(\mathbf{x}|{\cal S})\hm\ge 0$). Рассмотрим 
второе слагаемое $R(\mathbf{x}|{\cal S})$~--- 
избыточность кодирования состояния. Учитывая справедливое для всех $\alpha\hm>0$ 
элементарное неравенство
\begin{equation}
\label{eq10}
-\log_{2}(\alpha) \ge 1-\alpha\,,
\end{equation}
обращающееся в~равенство только в~случае $\alpha\hm=1$, имеем:
\begin{multline*}
R(\mathbf{x}|{\cal S}) = \sum\limits_{x\in{\cal A}} f(x|{\cal S}) 
\left[-\log_{2}\fr{q(x|{\cal S})}{f(x|{\cal S})}\right] \ge{}\\
{}\ge
\sum\limits_{x\in{\cal A}} f(x|{\cal S}) \left[1 - 
\fr{q(x|{\cal S})}{f(x|{\cal S})}\right] ={}\\
{}=
\sum\limits_{x\in{\cal A}} \left[f(x|{\cal S}) - q(x|{\cal S})\right] = 0\,,
\end{multline*}
т.\,е. $R(\mathbf{x}|{\cal S})\hm\ge0$, причем $R(\mathbf{x}|{\cal S})\hm=0$ тогда 
и~только тогда, когда $f(x|{\cal S})= q(x|{\cal S})$. Таким образом, 
показано, что квазиэнтропия состояния~--- это минималь\-ная скорость кодирования 
со\-сто\-яния, которая достигается при обращении в~нуль из\-бы\-точ\-ности, т.\,е.\ 
при использовании частотных вероятностей в~качестве кодовых вероятностей. 
Подставляя теперь~(\ref{eq7}) в~(\ref{eq5}), получаем скорость кодирования 
(всей последовательности) в~виде суммы двух не\-от\-ри\-ца\-тель\-ных слагаемых:
\begin{equation}
\label{eq11}
V(\mathbf{x}) = H(\mathbf{x}) + R(\mathbf{x})\,,
\end{equation}
равных
\begin{multline}
\label{eq12}
H(\mathbf{x}) = \sum\limits_{{\cal S}\in{\frak S}} f({\cal S}) 
H(\mathbf{x}|{\cal S}) \equiv{}\\
{}\equiv
\sum\limits_{{\cal S}\in{\frak S}} f({\cal S}) 
\sum\limits_{x\in{\cal A}} f(x|{\cal S}) \left[-\log_{2}f(x|{\cal S})\right]\,;
\end{multline}

\vspace*{-12pt}

\noindent
\begin{multline}
\label{eq13}
R(\mathbf{x}) =\sum\limits_{{\cal S}\in{\frak S}} f({\cal S}) 
R(\mathbf{x}|{\cal S}) \equiv{}\\
{}\equiv
\sum\limits_{{\cal S}\in{\frak S}} f({\cal S})
\sum\limits_{x\in{\cal A}} f(x|{\cal S})\left[
-\log_{2}\fr{q(x|{\cal S})}{f(x|{\cal S})}\right]\,,
\end{multline}
которые суть квазиэнтропия и~избыточность кодирования 
(всей последовательности) соответственно. При этом квазиэнтропия 
не зависит от условных кодовых распределений и~представляет собой 
минимальную скорость кодирования, которая достигается при обращении 
в~нуль избыточности, т.\,е.\ при использовании частотных вероятностей 
в~качестве кодовых вероятностей.

Обычно при решении практических задач сжатия данных множество состояний неизвестно. 
Более того, как правило, невозможно даже установить, насколько описанная выше 
модель источника адекватна реальным данным. При таком положении вещей данное 
выше определение состояний становится совершенно неконструктивным и~бесполезным 
с~практической точки зрения. Поэтому\linebreak определим состояния по-но\-во\-му, 
взяв за основу условные кодовые вероятности, а~именно: будем по определению 
считать состоянием под\-мно\-жество ${\cal S}\hm\subset{\cal A}^\tau$ такое, 
что значения~$x_n$ всех тех элементов последовательности, которым 
пред\-шест\-ву\-ют элементы последовательности со значениями 
$\{x_{n-1},\ldots,x_{n-\tau}\}\in{\cal S}$, кодируются одним общим 
кодовым условным распределением~$q(a|{\cal S})$. Таким образом, состояние 
характеризуется тем, что все значения его элементов кодируются одним 
распределением. При этом формулы~(\ref{eq5})--(\ref{eq13}), разумеется, 
остаются в~силе, а~задача универсального кодирования может быть сформулирована 
как задача выбора множества состояний~${\frak S}$ и~задача построения совокупности 
условных кодовых вероятностей $\{q(a|{\cal S})\}$ 
($a\hm\in{\cal A}$, ${\cal S}\hm\in{\frak S}$) для выбранного множества со\-сто\-яний.

\vspace*{-9pt}

\subsection{Выбор множества состояний}

Рассмотрим первую задачу универсального кодирования~--- выбор множества состояний. 
Вообще говоря, эта задача должна решаться отдельно для каж\-до\-го типа исходных 
данных на основе имеющейся априорной информации и/или принятой модели данных. 
Существуют, однако, некоторые общие соображения. Квазиэнтропия состояния не 
зависит от кодовых вероятностей и~определяется только частотными вероятностями 
значений в~данном состоянии. Соответственно, квазиэнтропия зависит только от 
множества состояний~${\frak S}$ в~целом. Естественно попытаться выбрать состояния 
так, чтобы минимизировать квазиэнтропию. При этом следует с~самого начала иметь в~виду, 
что возможна ситуация, когда квазиэнтропия мала, но избыточность (также зависящая 
от множества состояний) не\-до\-пус\-ти\-мо велика и,~как следствие, недопустимо 
велика и~скорость кодирования. Описанная ситуация, например, заведомо имеет 
место в~том случае, когда число состояний велико (сравнимо по величине с~количеством 
отсчетов исходных данных).

Рассмотрим множества ${\cal S}\,,{\cal S}'\,,{\cal S}''\subset{\cal A}^\tau$ такие, 
что ${\cal S}'\cap{\cal S}''\hm =\varnothing$, ${\cal S}'\cup{\cal S}'' \hm={\cal S}$. 
Снова используя неравенство~(\ref{eq10}), имеем:
\begin{multline*}
f({\cal S}) H(\mathbf{x}|{\cal S}) =
f({\cal S})\sum\limits_{x\in{\cal A}} f(x|{\cal S}) 
\left[-\log_{2}f(x|{\cal S})\right] ={} \\
{}= f\left( {\cal S}'\right)\sum\limits_{x\in{\cal A}} f\left(x|{\cal S}'\right) 
\left[-\log_{2}f(x|{\cal S})\right] +{}\\
{}+
f\left({\cal S}''\right)\sum\limits_{x\in{\cal A}} f\left(x|{\cal S}''\right) 
\left[-\log_{2}f(x|{\cal S})\right] ={} \\
{}= f\left({\cal S}'\right) H\left(\mathbf{x}|{\cal S}'\right) +{}
\end{multline*}

\noindent
\begin{multline*}
{}+
f\left({\cal S}'\right)\sum\limits_{x\in{\cal A}}
    f\left(x|{\cal S}'\right) \left[-\log_{2}\fr{f(x|{\cal S})}
    {f(x|{\cal S}')}\right] +{}\\
{}+ f\left({\cal S}''\right) H\left(\mathbf{x}|{\cal S}''\right) +{}\\
{}+
f\left({\cal S}''\right)\sum\limits_{x\in{\cal A}}
    f\left(x|{\cal S}''\right) \left[-\log_{2}\fr{f(x|{\cal S})}{f\left(x|{\cal S}''\right)}\right] \ge{} \\
{}\ge f\left({\cal S}'\right)\left\{H\left(\mathbf{x}|{\cal S}'\right) +
\sum\limits_{x\in{\cal A}} \left[f\left(x|{\cal S}'\right)-
f\left(x|{\cal S}\right)\right] \right\} +{}\\
{}+
f\left({\cal S}''\right)\left\{H\left(\mathbf{x}|{\cal S}''\right) +
\sum\limits_{x\in{\cal A}} \left[f\left(x|{\cal S}''\right)-
f(x|{\cal S})\right] \right\} ={} \\
{}= f\left({\cal S}'\right) H\left(\mathbf{x}|{\cal S}'\right) + 
f\left({\cal S}''\right) H\left(\mathbf{x}|{\cal S}''\right)\,.
\end{multline*}
Итак,
\begin{multline}
\label{eq14}
H(\mathbf{x}|{\cal S}) \ge{}\\
{}\ge
f\left({\cal S}'|{\cal S}\right) H\left(\mathbf{x}|{\cal S}'\right) + 
f\left({\cal S}''|{\cal S}\right) H\left(\mathbf{x}|{\cal S}''\right)\,,
\end{multline}
где $f({\cal S}'|{\cal S})\hm=f({\cal S}')/f({\cal S})$ 
и~$f({\cal S}''|{\cal S})\hm=f({\cal S}'')/f({\cal S})$~--- 
условные частотные вероятности состояний~${\cal S}'$ и~${\cal S}''$ 
соответственно, а~равенство имеет место только в~том случае, если $f(x|{\cal S})
\equiv f(x|{\cal S}')\equiv f(x|{\cal S}'')$. Таким образом, установлено, 
что квазиэнтропия является выпуклой функцией: при разбиении любого состояния 
квазиэнтропия не увеличивается. Поэтому критерием выбора множества~${\cal S}$ 
в~качестве состояния может служить условие
\begin{multline*}
H(\mathbf{x}|{\cal S}) -
\min\limits_{{\cal S}'\subset{\cal S}}
\left\{ f({\cal S}'|{\cal S}) H(\mathbf{x}|{\cal S}') +{}\right.\\
\left.{}+
f({\cal S}\!\setminus\!{\cal S}'|{\cal S}) H(\mathbf{x}|{\cal S}\!\setminus\!{\cal S}') 
\right\} < \varepsilon\,,
\end{multline*}
в котором значение $\varepsilon\hm>0$ должно быть выбрано исходя из 
практических требований.

Квазиэнтропия всей последовательности зависит от множества состояний: 
$H(\mathbf{x})\equiv H(\mathbf{x},{\frak S})$. Если зафиксировать общее 
число состояний~$S$, то величина
\begin{equation}
\label{eq15}
\hat{H}^{S}(\mathbf{x}) = \min\limits_{{\frak S}:\:|{\frak S}|=S} 
H(\mathbf{x},{\frak S})
\end{equation}
представляет собой оптимальную (при заданном числе состояний) 
квазиэнтропию, а~соответ\-ст\-ву\-ющее мно\-жество~$\hat{\frak S}$~--- 
оптимальное мно\-жество, которое естественно использовать в~качестве 
мно\-же\-ст\-ва состояний при кодировании.

Из неравенства~(\ref{eq14}) сразу следует, что оптимальная 
квазиэнтропия~$\hat{H}^{S}(\mathbf{x})$ при увеличении числа состояний~$S$ 
не возрастает. Это позволяет использовать условие
$$
\hat{H}^{S}(\mathbf{x}) - \hat{H}^{S+1}(\mathbf{x}) < \varepsilon
$$
как критерий для определения числа состояний. При этом конкретное 
значение $\varepsilon\hm>0$ должно выбираться исходя из практических требований.

Проверка любого из приведенных выше условий связана с~перебором 
всех подмножеств множества~${\cal A}^\tau$, что, как правило, не может 
быть реализовано на практике уже в~случае $\tau\hm=2$. Поэтому сказанное может 
рассматриваться лишь как <<общее направление движения>>: указанные критерии 
должны быть адаптированы к~конкретному типу данных с~привлечением априорной 
информации и~дополнительных гипотез.

\subsection{Построение кодовых распределений}

Выше было показано, что минимальная скорость кодирования для 
заданного множества состояний достигается тогда и~только тогда, 
когда $q(x|{\cal S})\hm\equiv f(x|{\cal S})$ для всех состояний, т.\,е.\
 в~качестве условных кодовых распределений используются условные частотные 
 распределения. Условные частотные распределения~$f(x|{\cal S})$ 
 априори не известны, но могут быть вычислены кодером по исходным 
 данным~$\mathbf{x}$, что позволяет использовать упрощенное комбинаторное 
 кодирование.

Кодовое слово комбинаторного кода состоит из двух частей. Первая часть 
(преамбула) содержит значения $N(x|{\cal S})$ для всех $x\hm\in{\cal A}$,  
${\cal S}\hm\in{\frak S}$, которые вычисляются в~процессе кодирования. 
Длина преамбулы равна $S(A-1)(\log_{2}N+1)$ бит. Вторая часть~--- 
результат арифметического кодирования последовательности $\mathbf{x}\hm=\{x_n\}$ 
с~по\-мощью частотных распределений $f(x|{\cal S})$. Получив кодовое слово, 
декодер выделяет преамбулу, <<считывает>> значения $N(x|{\cal S})$ 
и~вычисляет~$N({\cal S})$~--- суммы $N(x|{\cal S})$ по всем $x\hm\in{\cal A}$. 
В~результате становятся известными частотные распределения $f(x|{\cal S})$, 
использовавшиеся при кодировании, что позволяет однозначно декодировать 
вторую часть кодового слова и~восстановить исходную последовательность.

Избыточность описанной процедуры комбинаторного кодирования определяется 
длиной преамбулы, т.\,е.\ длиной данных, которые должны быть дополнительно 
переданы декодеру, и~равна
\begin{equation}
\label{eq16}
R = R_{\mathrm{T}} = \fr{S(A-1)(\log_{2}N+1)}{N}\,.
\end{equation}
При больших значениях~$A$ величина~(\ref{eq16}) не\-до\-пус\-ти\-мо велика. 
Действительно, при $S\hm=5$, $A\hm=2^{12}$ и~$N\hm=512^2$ имеем 
$R_{\mathrm{T}}\hm\sim1{,}4$~б/п.

Отметим, что в~общем случае (без ка\-ких-ли\-бо 
дополнительных предположений относительно исходных данных) использование 
более совершенных по сравнению с~комбинаторным кодированием методов 
позволяет уменьшить избыточность приблизительно в~два раза; это, 
однако, не решает проблему.

Приступим к~рассмотрению впервые предложенного в~работе~\cite{b01} метода 
построения кодовых распределений, который представляет собой, по существу, 
некоторую модификацию метода комбинаторного кодирования.

 Без ограничения 
общности можно считать, что множество~${\cal A}$ состоит из идущих подряд 
целых чисел. В~основе метода лежит следующее предположение: для любых исходных 
данных частотные распределения $f(x|{\cal S})$ представляют собой достаточно 
<<гладкие>> функции (точнее, отсчеты достаточно <<гладких>> функций). 


Основная идея метода заключается в~том, чтобы аппроксимировать соответствующие 
распределения простыми аналитическими функциями из зара\-нее выбранного класса, 
такого что каждая функция класса однозначно определяется значениями 
некоторого небольшого числа параметров. 
Кодер использует значения построенных аппроксимирующих функций в~качестве 
кодовых распределений, а~для передачи необходимой информации декодеру 
достаточно передать значения параметров, определяющих эти функции. При этом 
избыточность~$R_{\mathrm{T}}$, связанная с~передачей декодеру дополнительной 
информации, кардинально уменьшается, но появляется избыточность 
арифметического кодирования~$R(\mathbf{x})$ (см.\ формулы~(\ref{eq11}) и~(\ref{eq7})), 
поскольку теперь кодовые вероятности не равны частотным вероятностям. 
Величина~$R(\mathbf{x})$ определяется качеством аппроксимации 
и~в~конечном счете адекватностью используемого основного предполо-\linebreak жения.

Перейдем теперь к~более детальному рассмотрению предлагаемого метода. 
Обычно одна простая аналитическая функция не обеспечивает приемлемой 
точ\-ности аппроксимации частотных распределений $f(x|{\cal S})$  
на всем множестве~${\cal A}$. Поэтому разобьем все множество~${\cal A}$ 
на диапазоны, состоящие из идущих подряд целых чисел, с~тем чтобы использовать в~каждом 
диапазоне свою аппроксимирующую функцию. Такое разбиение задается, разумеется, 
границами диапазонов. Пусть ${\cal I}\hm=[a_B,\,a_E]$, $a_B\hm\le a_E$,~--- 
отдельный диапазон; $a_E-a_B+1$~--- чис\-ло значений в~данном диапазоне; 
${\frak I}$~--- все множество диапазонов; $I\doteq |{\frak I}|$~--- 
общее число диапазонов (для разных состояний используются разные разбиения 
на диапазоны).

Пусть $N({\cal I}|{\cal S})$~--- число элементов состояния~${\cal S}$, 
значения которых попадают в~диапазон~${\cal I}$, ${\cal I}\hm\in{\frak I}({\cal S})$. 
Тогда $f({\cal I}|{\cal S})\hm=N({\cal I}|{\cal S})/N({\cal S})$ 
и~$f(x|{\cal I},{\cal S})\hm=N({x|\cal S})/N({\cal I}|{\cal S})$,  $x\hm\in{\cal I}$,~--- 
условные частотные распределения вероятностей диапазонов и~значений 
в~диапазоне~${\cal I}$ в~данном состоянии~${\cal S}$. Для каждого диапазона 
будем использовать свою собственную нормированную функцию 
распределения $q(x|{\cal I},{\cal S})$, которая аппроксимирует 
функцию $f(x|{\cal I},{\cal S})$. Функция $q(x|{\cal I},{\cal S})$~--- 
условное кодовое распределение значений в~данном диапазоне. Общее условное 
кодовое распределение для со\-сто\-яния~${\cal S}$   имеет вид:
\begin{equation}
\label{eq17}
q(x|{\cal S}) = f({\cal I}|{\cal S}) q(x|{\cal I},{\cal S})\,, 
\quad x\in{\cal I}\in{\frak I({\cal S})}\,.
\end{equation}
Из формул~(\ref{eq17}) и~(\ref{eq9}) следует, что избыточность 
арифметического кодирования состояния~${\cal S}$ может быть представлена в~виде:
\begin{equation}
\label{eq18}
R(\mathbf{x}|{\cal S}) =
\sum\limits_{{\cal I}\in{\frak I}({\cal S})} f({\cal I}|{\cal S}) 
R(\mathbf{x}|{\cal I},{\cal S})\,,
\end{equation}
где $R(\mathbf{x}|{\cal I},{\cal S})$~--- избыточность арифметического 
кодирования отдельного диапазона~--- имеет вид
\begin{equation}
\label{eq19}
R(\mathbf{x}|{\cal I},{\cal S}) = \sum\limits_{x\in{\cal I}} f(x|{\cal I},{\cal S})
\left[ -\log_{2}\fr{q(x|{\cal I},{\cal S})}{f(x|{\cal I},{\cal S})}\right]\,.
\end{equation}

Остановимся подробнее на способе выбора аппроксимирующей функции 
$q(x|{\cal I},{\cal S})$. Сразу отметим, что случай, когда диапазон 
состоит из единственной точки ($a_B\hm=a_E$), является тривиальным,\linebreak
 а~избыточность 
кодирования такого диапазона\linebreak равна нулю. Поэтому далее будем считать, 
что $a_B\hm<a_E$. 

Пусть ${\cal P}\hm=\{p(\xi;\boldsymbol{\gamma}):\xi\in[0,\,1], 
\boldsymbol{\gamma}\hm\in\boldsymbol{\Gamma}\}$~--- 
некоторое $\boldsymbol{\gamma}$-па\-ра\-мет\-ри\-че\-ское 
семейство положительных функций вещественного аргумента~$\xi$,  
$\boldsymbol{\Gamma}$~--- область допустимых значений параметров. 
Выбор отрезка~$[0,\,1]$  в~качестве области определения функций 
не ограничивает общности, поскольку сдвиг и~масштабирование при 
необходимости могут быть отнесены к~числу параметров~$\boldsymbol{\gamma}$. 
Предполагается, что класс функций~$\cal{P}$ известен как кодеру, так и~декодеру. 
При любых значениях параметров вели-\linebreak чины
\begin{equation}
q(x;\boldsymbol{\gamma}|{\cal I},{\cal S}) = 
c^{-1}p\left( \fr{x-a_B}{a_E-a_B}\,;\boldsymbol{\gamma}\right)\,,
\label{eq20}
\end{equation}
где
$$
c=\sum\limits_{k=0}^{a_E-a_B} p\left( \fr{k}{a_E-a_B}\,;\boldsymbol{\gamma}\right)\,,
\quad x\in{\cal I}\in{\frak I}({\cal S})\,,
$$
удовлетворяют необходимым условиям и~могут использоваться в~качестве 
условных кодовых распределений. Наличие в~формуле~(\ref{eq20}) нормировочной 
константы~$c$ обеспечивает выполнение условия~(\ref{eq2}) и~позволяет, не 
ограничивая общности, наложить на все функции семейства~${\cal P}$ одно из 
условий вида $p(0)\hm=1$ или $p(1)\hm=1$.

В процессе кодирования величины $f(x|{\cal I},{\cal S})$ могут быть вычислены. 
Это позволяет выбрать пара-\linebreak\vspace*{-12pt}

\pagebreak

\noindent
метры $\boldsymbol{\gamma}\hm=
\hat{\boldsymbol{\gamma}}\hm\in\boldsymbol{\Gamma}$ так, чтобы избыточность 
кодирования диапазона~(\ref{eq19}) была минимальной при\linebreak использовании 
функции $q(x|{\cal I},{\cal S}) \hm= q(x;\hat{\boldsymbol{\gamma}}|{\cal I},{\cal S})$ 
в~качестве условного кодового распределения. Значения 
параметров~$\hat{\boldsymbol{\gamma}}$ должны быть переданы декодеру, 
что позволит реконструировать функцию  $q(x|{\cal I},{\cal S}) \hm= 
q(x;\hat{\boldsymbol{\gamma}}|{\cal I},{\cal S})$ и~использовать 
ее в~процессе восстановления.

Для заданного разбиения на диапазоны (множества~${\frak I}({\cal S})$) 
избыточность арифметического кодирования состояния~(\ref{eq18}) при 
использовании функций $q(x|{\cal I},{\cal S})$ минимальна. Однако эта величина 
сильно зависит от выбора разбиения и~является неустойчивой по отношению к~этому выбору: 
малое изменение границ диапазонов может приводить к~заметному изменению избыточности. 
Поэтому разбиение на диапазоны целесообразно проводить на этапе кодирования, 
выбирая ${\frak I}({\cal S})$ так, чтобы по возможности уменьшить величину~(\ref{eq18}). 
При этом соответствующая информация (границы диапазонов) должна быть передана 
декодеру.

Рассмотрим преамбулу комбинаторного кода, соответствующего описанному выше методу. 
Помимо значений $N({\cal I}|{\cal S})$ преамбула должна включать\linebreak значения 
границ диапазонов выбранных раз\-би\-ений~${\frak I}({\cal S})$ (по одной границе на 
диапазон) и~значения параметров 
$\hat{\boldsymbol{\gamma}}\hm=\hat{\boldsymbol{\gamma}}({\cal I},{\cal S})$. 
Оценим длину преамбулы. Для описания одного значения $N({\cal I}|{\cal S})$\linebreak 
требуется $\sim\log_{2}N$\,бит, одной границы диапазона~---   $\sim\log_{2}A$~бит. 
Пусть~$G$~--- длина описания одного набора параметров  $\hat{\boldsymbol{\gamma}}\hm=
\hat{\boldsymbol{\gamma}}({\cal I},{\cal S})$. Общая длина преамбулы оценивается 
как произведение общего числа использованных диапазонов и~суммы трех указанных 
величин, а~для избыточности~$R_{\mathrm{T}}$, 
связанной с~передачей дополнительной информации в~преамбуле, имеем следующую оценку:
$$
R_{\mathrm{T}} \simeq \fr{1}{N} \left( 
\log_{2}N+\log_{2}A +G\right) \sum\limits_{{\cal S}\in{\frak S}}I({\cal S)}\,.
$$
Пусть, как и~ранее, $S\hm=5$, $A\hm=2^{12}$ и~$N\hm=512^{2}$. 
Предположим, что для каждого состояния используется по~10~диапазонов и~$G\hm=50$, 
т.\,е. для описания одного набора параметров требуется~50~бит. 
В~таком случае имеем $R_{\mathrm{T}}\hm\sim 0{,}015$~б/п, 
что представляется достаточно малой величиной.

Таким образом, есть все основания полагать, что\linebreak полная избыточность 
описанного метода по\-стро\-ения кодовых распределений определяется главным образом 
избыточностью арифметического кодирова\-ния. 

Применение метода для сжатия конкретного 
типа данных требует его дополнительной адап\-та\-ции, а~оценка его эффективности~--- 
вопрос, который должен решаться экспериментально.

\section{Компьютерные томограммы}

\vspace*{-9pt}

Компьютерная (рентгеновская) томограмма представляет собой квадратное 
полутоновое изоб\-ра\-же\-ние размера~$512\times512$ и~глубины яр\-кости~16~бит,\linebreak 
которое получено в~результате применения алгоритма томографического восстановления 
(реконструкции) к~данным сканирования и~содержит информацию о~рентгеновской 
плотности тканей пациен\-та в~плоскости, перпендикулярной аксиальной оси сканирующей 
системы (томографа).

Значения рентгеновской плотности принято выражать в~единицах шкалы Хаунсфилда (HU). 
Шкала состоит из целых значений в~диапазоне $[-1024, 3071]$, ширина диапазона~--- 
12~бит. Рентгеновская плотность воды при нормальных условиях принята за нуль, 
рентгеновская плотность воздуха при нормальных условиях по определению считается 
равной~$-1000$. Для некоторого материала c~линейным коэффициентом поглощения~$\mu$ 
значение рентгеновской плотности по шкале HU равно $1000(\mu\hm-\mu_0)/\mu_0$, 
где~$\mu_0$~--- линейный коэффициент поглощения воды при той же (эффективной) 
энергии, а~значения округляются до ближайшего целого. Приведем для справки 
некоторые значения рентгеновской плотности в~единицах HU: легочные ткани~---  
$\sim-850\ldots-700$, жировые ткани~--- $\sim-120\ldots-30$, мышечные ткани~--- 
$\sim+20\ldots+40$, костные ткани~--- $\sim+300\ldots+800$.

При восстановлении значений томограммы~$X^{\mathrm{T}}$ 
для записи выраженных в~единицах~HU значений рентгеновской плотности используются 
два байта (16~бит). Способ размещения~12~значащих битов шкалы Хаунсфилда в~16~битах 
яр\-кости точки томограммы варьируется в~зависимости от используемого компьютерного 
томографа. Кроме того, поскольку в~обычных режимах работы томографа область 
восстановления представляет собой круг, а~томограмма представляет собой 
квадрат, в~который этот круг вписан, то отдельное (фоновое) значение приписывается 
тем точкам изображения, в~которых реконструкция не производилась. Таким образом, 
яркость томограммы принимает не более $2^{12}+1$ значений и~существует взаимно 
однозначное амплитудное преобразование  $X^{\mathrm{T}}\hm\to X$, отоб\-ра\-жа\-ющее 
исходные значения в~диапазон $[0,\,4096]$ так, что фоновое значение отоб\-ра\-жа\-ет\-ся 
в~нуль, а~для остальных значений после преобразования справедливо равенство
 $X\hm=X^{\mathrm{HU}}\hm+1024\hm+1$, где $X^{\mathrm{HU}}$~--- 
 значения в~единицах HU. В~силу сказанного везде далее будем считать, если не 
 оговорено противное, что амплитудное преобразование выполнено, диапазон 
 значений яркости томограммы равен $[0,\,4096]$, значение нуль является 
 фоновым значением.
 
 \pagebreak
 
 \begin{figure*} %fig1
\vspace*{1pt}
\begin{center}
\mbox{%
\epsfxsize=148.291mm
\epsfbox{ste-1.eps}
}
\end{center}
\vspace*{-9pt}
\Caption{Томограммы брюшной полости Т1~(\textit{а}) и Т2~(\textit{б}),
        легких~T3~(\textit{в}) и~Т4~(\textit{г}) и~головного мозга~Т5~(\textit{д}) 
        и~T6~(\textit{е}).
        Для первой, второй и~третьей пары томограмм использованы окна визуализации,
        равные соответственно $[850,\,1250]$, $[0,\,1100]$ и~$[1020,\,1120]$}
        \label{fig1}
        \end{figure*}
        
        


Итак, подлежащие сжатию данные представляют собой квадратную 
матрицу $\mathbf{X}\hm=[X_{l,m}]$, $0\hm\le l\hm\le L\hm-1$, 
$0\hm\le m\hm\le M\hm-1$, $L\hm=M\hm=512$. Пусть $N\hm=L\times M$~--- 
общее число элементов мат\-ри\-цы. Значения элементов лежат в~диапазоне 
${\cal A}\hm=[0,\,4096]$. Для применения описанной в~разд.~2
схемы необходимо выбрать некоторый способ упорядочения элементов 
мат\-ри\-цы. Примем естественный способ упорядочения, соответствующий построчному 
сканированию слева направо и~сверху вниз. При этом данные можно рассматривать 
как последовательность отсчетов $\mathbf{X}\hm=[X_{n}]$,  $0\hm\le n\hm\le N\hm-1$, 
причем $X_{l,m}\hm=X_{lM+m}$. Везде, где это не вызывает недоразумений, 
будем использовать сокращенную запись и~обозначать через~$X$ текущий элемент 
мат\-ри\-цы~$X_n$, а~через~$U$ и~$L$~--- соседние к~нему сверху и~слева 
элементы~$X_{n-M}$ и~$X_{n-1}$. Если~$X$~--- элемент первой строки и/или первого 
столбца, т.\,е.\ верхний и/или левый соседний элемент отсутствует, то будем полагать 
$U\hm=0$ и/или $L\hm=0$. Такое соглашение отвечает специфике томографических 
изоб\-ра\-же\-ний (на\-пом\-ним, что~0~--- фоновое значение). В~соответствии с~принятым 
способом упорядочения элементы~$U$ и~$L$ являются предшествующими по 
отношению к~текущему элементу~$X$  и~могут использоваться для определения 
текущего состояния ис\-точ\-ника.
{\looseness=1

}

\begin{figure*} %fig2
\vspace*{1pt}
\begin{center}
\mbox{%
\epsfxsize=148.291mm
\epsfbox{ste-2.eps}
}
\end{center}
\vspace*{-9pt}
\Caption{Ошибки предсказания для томограмм~T3~(\textit{а}) и~T6~(\textit{б}).
        Для ошибок  предсказания томограммы~T3~(\textit{а}) использовано окно
        визуализации~$[-550,\,+550]$,  для ошибок предсказания
         томограммы~T6~(\textit{б})~---
        окно визуализации~$[-50,\,+50]$
}
\label{fig2}
\end{figure*}


В качестве экспериментального материала в~работе используются шесть 
томограмм~Т1--Т6 трех видов тканей: брюшной полости~--- Т1 и~Т2, легких~---
Т3 и~Т4 и~головного мозга~--- Т5 и~Т6, которые предоставлены Отделением лучевой 
диагностики Клиники пропедевтики внутренних болезней им.\
 В.\,Х.~Василенко (томограф HiSpeed CT/i компании General Electric). 
 В~данном случае значение $X^{\mathrm{T}}\hm=0\mathrm{x}7830$ является 
 фоновым. Остальные значения   связаны со значениями рентгеновской плот\-ности, 
 выраженной в~единицах HU, следующим образом: 
 $X^{\mathrm{T}}\hm=0\mathrm{x}8400\hm+ X^{\mathrm{HU}}$. 
 Поэтому амплитудное преобразование, о~котором шла речь выше, имеет сле\-ду\-ющий вид:
\begin{equation}
\left.
\begin{array}{rl}
\hspace*{-2mm}X^{\mathrm{T}} &\to X=0 \; \left(X^{\mathrm{T}}=0\mathrm{x}7830\right)\,; \\
\hspace*{-2mm}X^{\mathrm{T}} &\to X=X^{\mathrm{T}}-0\mathrm{x}7\mathrm{FFF}\  
\left(X^{\mathrm{T}}\neq 0\mathrm{x}7830\right)\,.
\end{array}\!
\right\}\!
\label{eq21}
\end{equation}
Префикс <<$0{\mathrm{x}}$>> использован выше для обозначения шестнадцатеричной 
записи целых чисел.

Томограммы Т1--Т6 (после амплитудного преобразова\-ния) представлены на 
рис.~\ref{fig1}. Для визу\-а\-лизации томограмм брюшной полости использовано 
окно визуализации $[850,\, 1250]$, для томограмм легких~--- 
значительно более широкое окно $[0,\,1100]$, для томограмм головного мозга~--- 
узкое окно $[1020,\,1120]$. Напомним, что визуализация\linebreak изображения 
в~окне $[x_{\min},\,x_{\max}]$ предполагает\linebreak преобразование значений яркости, 
при котором диапазон $[x_{\min},\,x_{\max}]$ линейно отображается на 
стандартный диапазон $[0,\,255]$, значения $x \hm< x_{\min}$ 
отоб\-ра\-жа\-ют\-ся в~значение~0, значения $x\hm> x_{\max}$~--- 
в~значение~255, и~вывод полученного таким образом изображения на экран монитора 
или устройства печати. Окно визуализации для томограмм обычно выбирается исходя 
из диагностических задач.

\vspace*{-6pt}


\section{Кодирование ошибок предсказания}

Адаптацию представленной в~разд.~2 общей схемы для сжатия томографических 
данных начнем с~рассмотрения метода, основанного на универсальном кодировании 
ошибок предсказания. Метод был впервые предложен в~работе~\cite{b01}.

\vspace*{-6pt}

\subsection{Ошибки предсказания}

Простейшее предсказание для текущего элемента~$X$ имеет вид $[(U\!+\!L)/2]_{-}$, 
где~$U$
и~$L$ суть ближайшие соседние сверху и~слева элементы, а~$[\cdot]_{-}$~--- 
целая часть числа, т.\,е.\ деление предполагается целочис\-лен\-ным. 
В~соответствии с~принятым способом упорядочения к~моменту рассмотрения очередного 
элемента~$X$ элементы~$U$ и~$L$ уже известны как кодеру, так и~декодеру 
и,~следовательно, известно предсказание. Поэтому описание исходных значений 
эквивалентно описанию значений ошибок предсказания $\Delta\hm=X\hm-[(U\hm+L)/2]_{-}$. 
Заметим, что в~целом ошибки предсказания образуют изображение~$\boldsymbol{\Delta}$ 
того же размера, что и~исходное изображение. Диапазон возможных значений ошибок 
предсказания вдвое шире диапазона исходных значений: ${\cal A}_{\Delta}\hm=
[-4096,\,+4096]$. Несмотря на это, распределение значений ошибок внутри 
диапазона является значительно менее равномерным: функция распределения имеет 
ярко выраженный максимум вблизи нуля. Поэтому кодирование значений ошибок предсказания 
оказывается выгоднее, чем кодирование исходных значений.

На рис.~2,\,\textit{а} представлена ошибка предсказания для томограммы~T3, 
на рис.~2,\,\textit{б}~--- 
ошибка предсказания для томограммы~T6. Окна визуализации симметричны относительно 
значения нуль и~имеют ту же ширину, что и~окна, использованные при визуализации 
соответствующих томограмм на рис.~\ref{fig1}.

\vspace*{-6pt}


\subsection{Множества состояний и~оценки минимальной скорости кодирования}

Примем гипотезу, согласно которой распределение значений очередной ошибки 
предсказания~$\Delta$ зависит только от значений элементов~$U$ и~$L$ исходного 
изображения. Функция

\noindent
\begin{equation}
\label{eq22}
\sigma_1(U,L)=|U-L|
\end{equation}
характеризует изменение значений элементов в~окрестности рассматриваемой точки. 
Примем ги-\linebreak потезу, согласно которой чем ближе значения функции~$\sigma_1$ 
для разных рассматриваемых точек, тем\linebreak меньше различие соответствующих 
распределений, измеряемое избыточностью их совместного кодирования. Данная 
гипотеза приводит к~сле\-ду\-юще\-му способу построения множества состояний~$\frak S$ 
источника. Прежде всего, учитывая специфику томограмм, определим фоновое 
состояние~${\cal S}_0$\linebreak   следующим образом: источник находится в~фоновом 
состоянии тогда и~только тогда, когда $U\hm=L\hm=0$. Далее выберем множество 
порогов ${\frak T}\hm=\{{\cal T}_1,{\cal T}_2,\dots,{\cal T}_T \}$, состоящее 
из~$T$~различных упорядочен\-ных по возрастанию натуральных чисел. Если источник 
не находится в~фоновом состоянии~${\cal S}_0$ и~выполнено условие
\begin{equation}
\label{eq23}
{\cal T}_{k-1} \leq \sigma_1(U,L) < {\cal T}_{k}\,, \qquad k=1,\dots,T\,,
\end{equation}
то источник находится в~состоянии~${\cal S}_k$. Если, наконец,
\begin{equation}
\label{eq24}
{\cal T}_{T} \leq \sigma_1(U,L)\,,
\end{equation}
то источник находится в~состоянии~${\cal S}_{T+1}$. Таким образом, 
функция~$\sigma_1$ и~значения порогов~$\frak T$ определяют множество 
состояний источника~${\frak S}({\frak T})$, которое состоит из~$T\hm+2$~состояний 
(считая фоновое).

Множеству состояний источника~${\frak S}(\frak T)$ соответствует квазиэнтропия 
$H\hm=H({\frak S}({\frak T}))\hm\equiv H({\frak T})$. 
Отметим, что в~силу выпуклости квазиэнтропии (см.\ формулу~(\ref{eq14})) 
добавление дополнительного порога может лишь уменьшать ее значение.

В подразд.~2.3 была сформулирована оптимизационная задача~(\ref{eq15}), 
которая предполагает нахож\-де\-ние оптимального множества со\-сто\-яний источника 
при фиксированном общем числе со\-сто\-яний и~вычисление соответствующей 
квазиэнтропии. При выбранном способе построения со\-сто\-яний данная задача 
превращается в~задачу нахождения оптимальных порогов и~принимает следующий вид:
\begin{equation}
\label{eq25}
\hat{H}^{T}(\boldsymbol{\Delta}) =
\min\limits_{0<{\cal T}_1<\dots<{\cal T}_T}  H(\boldsymbol{\Delta},{\frak T})\,,
\end{equation}
где общее число используемых порогов~$T$ фиксировано. Величина~$\hat{H}^{T}$~--- 
оптимальная квазиэнтропия; для обозначения множества порогов, реализующих минимум 
в~(\ref{eq25}), будем использовать обозначение $\hat{\frak T}^T\hm=
\{{\hat{\cal T}}^{T}_{1}, {\hat{\cal T}}^{T}_{2},\dots, {\hat{\cal T}}^{T}_{T}\}$.

В отличие от общей задачи~(\ref{eq15}), задача~(\ref{eq25}) 
допускает численное решение, в~результате которого можно получить 
оценки оптимальной квазиэнтропии, т.\,е.\ оценки для минимальной скорости 
кодирования используемого метода.

% Table 1
\begin{table*}\small
\begin{center}
\Caption{Оптимальные пороги и~квазиэнтропия}
\label{tab1}
\vspace{2ex}
\begin{tabular}{|c|c|rc|cc|cc|}
\hline
&&&&&&&\\[-9pt]
 T & $\hat{H}^0$ $(T=0)$ & $\hat{\frak T}^1;$ & $\hat{H}^1$ $(T=1)$ &
$\hat{\frak T}^2;$ & $\hat{H}^2$ $(T=2)$ & $\hat{\frak T}^3;$ & 
$\hat{H}^3$ $(T=3)$\\
\hline
Т1 & 4,768833 & \{37\}; & 4,564031 & \{23,101\}; & 4,507584 & \{15,37,114\}; & 
4,486921 \\
%\hline
Т2 & 5,032069 & \{41\}; & 4,835185 & \{29,113\}; & 4,790141 & \{16,37,117\}; & 4,771703\\
%\hline
Т3 & 6,451010 & \{149\}; & 6,335030 & \{101,257\}; & 6,309464 & \{87,175,468\}; & 6,294528 \\
%\hline
Т4 & 6,374494 & \{161\}; & 6,250491 & \{87,270\}; & 6,218691 & \{1,87,270\}; & 6,199943 \\
%\hline
Т5 & 4,674353 & \{19\}; & 4,381157 & \{1,21\}; & 4,276768 & \{1,14,61\}; & 4,215797 \\
%\hline
Т6 & 4,378846 & \{16\}; & 4,023612 & \{12,44\}; & 3,976094 & \{12,39,274\}; & 3,957160 \\
\hline
\end{tabular}
\end{center}
\end{table*}


Значение квазиэнтропии при фиксированных значениях порогов может быть 
вычислено по формуле~(\ref{eq12}) (в качестве значений отсчетов нужно 
использовать значения ошибок предсказания). Квази\-энтропия зависит от значений 
порогов сложным нерегулярным образом, поэтому единственным способом точного 
решения оптимизационной\linebreak задачи~(\ref{eq25}) является прямой перебор 
в~пространст-\linebreak ве 
параметров (допустимых значений порогов).\linebreak Результаты проведенных для томограмм~Т1--Т6 
чис\-лен\-ных экспериментов представлены в~табл.~\ref{tab1}. 
Вычисления были проведены для общего числа порогов~$T$, принимающего значения~0, 1, 
2 и~3, при этом общее число состояний было равно соответственно~2, 3, 4 и~5. 
Заметим, что случай $T\hm=0$ отвечает использованию двух состояний: <<фоновое>> 
и~<<нефоновое>>.


Приведенные данные показывают, что величина~$\hat{H}^T$ монотонно убывает 
с~ростом~$T$, т.\,е.\ увеличение общего числа порогов уменьшает нижнюю оценку 
скорости кодирования. Однако уже при малых значениях~$T$ наступает насыщение, 
и~дальнейшее увеличение числа порогов может дать лишь незначительный выигрыш в~скорости 
кодирования. Действительно, для любой томограммы разность $\hat{H}^2\hm-\hat{H}^3$ 
уже находится в~пределах нескольких сотых долей, а~при добавлении еще одного порога 
квазиэнтропия уменьшается не более чем на тысячные доли. Поэтому использование 
большого числа порогов не имеет смысла и~целесообразно ограничиться не более чем 
тремя порогами (использовать не более пяти состояний источника).

Как уже было указано выше, единственным способом решения экстремальной 
задачи~(\ref{eq25}) является прямой перебор. При этом вычисление квазиэнтропии 
для каждого набора порогов связано с~просмотром значений всего изображения. 
Поэтому нахождение оптимальных значений порогов представляет собой трудную 
вычислительную задачу, которая не может быть решена за приемлемое на этапе 
сжатия время уже в~двумерном пространстве параметров, т.\,е.\ для двух порогов. 
Решение же задачи~(\ref{eq25}) в~трехмерном пространстве параметров ($T\hm=3$) 
при использовании современной вычислительной техники занимает десятки часов. 
Поэтому необходим эффективный алгоритм построения множества состояний (порогов), 
реализация которого в~процессе кодирования томограммы не приводила бы к~большим 
временн$\acute{\mbox{ы}}$м затратам. Такой алгоритм был предложен в~работах~\cite{b03,b04}.

Алгоритм предполагает вместо трех оптимальных порогов 
 $\hat{\frak T}^3\hm=\{{\hat{\cal T}}^{3}_{1}, {\hat{\cal T}}^{3}_{2}, 
 {\hat{\cal T}}^{3}_{3}\}$ использовать при построении состояний три 
 квазиоптимальных порога $\tilde{\frak T}^3\hm=\{{\tilde{\cal T}}^{3}_{1}, 
 {\tilde{\cal T}}^{3}_{2}, {\tilde{\cal T}}^{3}_{3}\}$, которые находятся 
 следующим образом. Сначала находится порог~${\tilde{\cal T}}^{3}_{2}$ как 
 решение задачи~(\ref{eq25}) при общем числе порогов $T\hm=1$:  
 ${\tilde{\cal T}}^{3}_{2}\hm={\hat{\cal T}}^{1}_{1}$. Далее находятся 
 пороги ${\tilde{\cal T}}^{3}_{1}$, ${\tilde{\cal T}}^{3}_{3}$  
 как решения экстремальных задач
 \begin{equation}
\left.
\begin{array}{rl}
H^{2}\left({\tilde{\cal T}}^{3}_{1},{\tilde{\cal T}}^{3}_{2}\right) &=
\min\limits_{{\cal T}:\;{\cal T} < \tilde{\cal T}^{3}_{2}}  
H^2\left({\cal T},\tilde{\cal T}^{3}_{2}\right)\,;
\\
H^{2}\left({\tilde{\cal T}}^{3}_{2},{\tilde{\cal T}}^{3}_{3}\right) &=
\min\limits_{{\cal T}:\;\tilde{\cal T}^{3}_{2} < 
{\cal T}}  H^2\left(\tilde{\cal T}^{3}_{2},{\cal T}\right)
\end{array}
\right\}
\label{eq26}
\end{equation}
соответственно. Таким образом, нахождение трех квазиоптимальных 
порогов предполагает последовательное решение трех одномерных оптимизационных 
задач, что может быть сделано за приемлемое время в~процессе кодирования.

Пусть $\tilde{H}^3$~--- квазиэнтропия, соответствующая трем квазиоптимальным порогам. 
Поскольку до\-бав\-ле\-ние порогов может приводить только к~уменьшению квазиэнтропии, 
величина~$\tilde{H}^3$ не превышает величины~$\hat{H}^1$ (оптимальной квазиэнтропии 
при использовании одного порога):
\begin{multline*}
\tilde{H}^3 =
H^3\left({\tilde{\cal T}}^{3}_{1}, {\tilde{\cal T}}^{3}_{2}, 
{\tilde{\cal T}}^{3}_{3}\right) =
H^3\left({\tilde{\cal T}}^{3}_{1}, {\hat{\cal T}}^{1}_{1},
 {\tilde{\cal T}}^{3}_{3}\right) \leq{}\\
 {}\leq
H^1\left({\hat{\cal T}}^{1}_{1}\right) =
\hat{H}^1\,.
\end{multline*}
Более того, пусть выполнено условие ${\hat{\cal T}}^{2}_{1}\hm\leq 
{\hat{\cal T}}^{1}_{1}\hm\leq {\hat{\cal T}}^{2}_{2}$, т.\,е.\ значение 
одного оптимального порога (при $T\hm=1$) расположено между значениями двух 
оптимальных порогов (при $T\hm=2$). Тогда квазиэнтропия~$\tilde{H}^3$ не 
превышает квазиэнтропии $\hat{H}^2$ двух оптимальных порогов:
\begin{multline*}
\tilde{H}^3 =
H^3\left({\tilde{\cal T}}^{3}_{1}, {\tilde{\cal T}}^{3}_{2}, {\tilde{\cal T}}^{3}_{3}\right)\leq
H^3\left({\tilde{\cal T}}^{2}_{1}, {\hat{\cal T}}^{1}_{1}, 
{\tilde{\cal T}}^{2}_{2}\right) \leq{}\\
{}\leq
H^2\left({\hat{\cal T}}^{2}_{1},{\hat{\cal T}}^{2}_{2}\right) =
\hat{H}^2\,,
\end{multline*}
причем первое неравенство в~цепочке справедливо, поскольку 
пороги ${\tilde{\cal T}}^{3}_{1}$ и~${\tilde{\cal T}}^{3}_{3}$ суть 
решения экстремальных задач~(\ref{eq26}), а~второе~--- 
поскольку при до\-бав\-ле\-нии порогов квазиэнтропия не возрастает. Заметим, 
что указанное условие выполнено для всех томограмм~Т1--Т6, как показывают 
приведенные в~табл.~\ref{tab1} данные.

В табл.~2 приведены значения квазиоптимальных порогов и~соответствующие 
значения квазиэнтропии, посчитанные для томограмм~Т1--Т6. Там\linebreak\vspace*{-12pt} 



\begin{figure*}[b] %fig3
\vspace*{1pt}
\begin{center}
\mbox{%
\epsfxsize=161.036mm
\epsfbox{ste-3.eps}
}
\end{center}
\vspace*{-9pt}
\Caption{Частотные распределения значений ошибок предсказания томограммы~Т1 для
        фонового состояния~${\cal S}_0$~(\textit{а}) и~для состояний
        ${\cal S}_1$, ${\cal S}_2$, ${\cal S}_3$ и~${\cal S}_4$~(\textit{б}),
        соответствующих трем квазиоптимальным порогам:
        $f(0|{\cal S}_1)\hm< f(0|{\cal S}_2)\hm< f(0|{\cal S}_3)\hm< f(0|{\cal S}_4)$}
\label{fig3}
\end{figure*}


\pagebreak



{\small \begin{center}  %tabl2
 \noindent
{{\tablename~2}\ \ \small{Квазиоптимальные пороги и~квазиэнтропия}}
\vspace*{2ex}


\begin{tabular}{|c|c|c|c|c|}
\hline
&&&&\\[-9pt]
 T & ${\tilde{\frak T}}^3$ & $\tilde{H}^3$ & 
$\hat{H}^2-\tilde{H}^3$ & $\tilde{H}^3-\hat{H}^3$ \\
\hline
Т1 & \{15,37,114\} & 4,486921 & 0,020663 & 0,000000 \\
%\hline
Т2 & \{17,41,117\} & 4,771758 & 0,018383 & 0,000055 \\
%\hline
Т3 & \{70,149,419\} & 6,295014 & 0,014450 & 0,000486 \\
%\hline
Т4 & \{69,161,429\} & 6,204381 & 0,014310 & 0,004438 \\
%\hline
Т5 & \{1,19,79\} & 4,219947 & 0,056821 & 0,004150 \\
%\hline
Т6 & \{9,16,57\} & 3,959734 & 0,016360 & 0,002574 \\
\hline
\end{tabular}
\end{center}
\vspace*{12pt}
}

\addtocounter{table}{1}

\noindent
же для удобства 
приведены значения величин $\hat{H}^2\hm-\tilde{H}^3$ и~$\tilde{H}^3\hm-\hat{H}^3$.

Анализ приведенных данных показывает, что, во-пер\-вых, 
использование трех квазиоптимальных 
 порогов всегда обеспечивает некоторый выигрыш 
по сравнению с~использованием двух оптимальных. Во-вто\-рых, 
величина $\tilde{H}^3\hm-\hat{H}^3$  не превышает~0,0045~б/п, 
а~величина отношения $(\tilde{H}^3\hm-\hat{H}^3)/\hat{H}^3$ не превышает~0,001. 
Следовательно, использование квазиоптимальных порогов не приводит к~заметным 
издержкам по сравнению с~использованием трех оптимальных порогов. 

Таким образом, описанный способ нахождения квазиоптимальных порогов 
полностью решает основную поставленную задачу и~дает эффективный алгоритм 
построения множества состояний.

Построенное по квазиоптимальному множеству порогов~$\tilde{\frak T}^3$ множество 
состояний будем далее называть квазиоптимальным множеством 
состояний и~обозначать $\tilde{\frak S}^5\hm=\tilde{\frak S}(\tilde{\frak T}^3)$.

В ходе проведенных работ был исследован вопрос о~возможности уменьшения 
оценок для минимальной скорости кодирования за счет использования предсказаний 
другого типа и/или других способов построения состояний источника. Ответ оказался 
отрицательным: приведенные выше оценки не удалось улучшить сколь\-ко-ни\-будь заметно.
{\looseness=1

}

\subsection{Оценки избыточности кодирования}

Используем описанную в~подразд.~2.4 общую схему для построения кодовых 
распределений состо\-яний источника. В~качестве множества состояний будем использовать 
квазиоптимальное множество со\-сто\-яний~${\tilde{\frak S}^5}$, которое состоит из пяти 
со\-сто\-яний, построенных по трем квазиоптимальным порогам~${\tilde{\frak T}^3}$ (см.\
  табл.~2). Заметим, что, поскольку значения порогов вычисляются для
   каждой томограммы в~процессе кодирования, их придется передавать декодеру 
   в~преамбуле.

На рис.~\ref{fig3} в~качестве характерного примера представлены частотные 
распределения значений ошибок предсказания для томограммы брюшной полости~Т1. 
График на рис.~\ref{fig3},\,\textit{а} отвечает фоновому состоянию~${\cal S}_0$, 
графики на рис.~\ref{fig3},\,\textit{б}~--- 
остальным состояниям ${\cal S}_1$--${\cal S}_4$. Распределения на 
рис.~\ref{fig3},\,\textit{б} легко различаются: чем больше номер состояния, 
тем больше значение соответствующего распределения в~нуле и~тем <<шире>> 
соответствующая кривая. Масштабы по оси ординат на двух рисунках различаются 
на два порядка.



Рассмотрим частотное распределение значений ошибок предсказания в~фоновом 
состоянии~${\cal S}_0$. Определяющее состояние условие имеет вид $U\hm=L\hm=0$. 
Отсюда сразу следует, что значения ошибок предсказания~$\Delta$  в~фоновом 
состоянии не могут быть отрицательными. Далее напомним, что нуль~--- 
это уникальное значение, приписываемое точкам фона, т.\,е.\ 
тем и~только тем точкам томограммы, где вос\-ста\-нов\-ле\-ние не производится. 
Такие точки (точки фона) образуют дополнение круга восстановления до квадрата, 
в~который этот круг вписан. Поэтому состояние~${\cal S}_0$, во-пер\-вых, 
состоит практически из одних точек фона и,~во-вто\-рых, включает практически 
все точки фона. Исключения того или иного рода сводятся лишь к~небольшому 
количеству точек, расположенных вблизи границы круга восстановления. 
При этом не попадающие в~фоновое состояние~${\cal S}_0$ точки фона автоматически 
попадают в~первое состояние~${\cal S}_1$, а~значения томограммы~$X$ в~<<лишних>> 
точках, совпадающие в~данном случае со значениями ошибок предсказания~$\Delta$, 
определяются плот\-ностью воздуха и~не могут быть велики. Таким образом, значения 
ошибок предсказания неотрицательны, а~их частотное распределение имеет более 
чем выраженный максимум в~нуле и~относительно неширокий диапазон. Поскольку 
значение максимума распределения определяется геометрией, то для всех используемых 
томограмм оно одинаково и~равно $f(0|{\cal S}_0)\hm= 0{,}997279$. 
Максимальное ненулевое значение для представленного на рис.~\ref{fig3},\,\textit{а} 
распределения равно~141. Полная частотная вероятность фонового состояния также 
определяется только геометрией и~для всех томограмм равна $f({\cal S}_0)\hm= 
0{,}210285$.
{\looseness=1

}



Отмеченные выше специфические особенности фонового состояния 
позволяют предположить, что построение кодовых вероятностей может быть 
осуществлено без особого труда и~не сопряжено с~преодолением каких бы то 
ни было трудностей. Действительно, в~случае томограммы~Т1 использование 
даже простейшего равномерного условного кодового распределения 
$q(\Delta|[1,141],{\cal S}_0)\hm=1/141$ обеспечивает приемлемое значение 
избыточности арифметического кодирования $R(\boldsymbol{\Delta}|{\cal S}_0)$ 
фонового состояния (см.~(\ref{eq18}) и~(\ref{eq19})) на уровне~0,008~б/п. 
При этом вклад в~общую избыточность арифметического 
кодирования~$R(\boldsymbol{\Delta})$ (см.~(\ref{eq13})) заведомо не 
превысит~0,002~б/п.

Рассмотрим частотные распределения значений ошибок предсказания в~других состояниях. 
Диа\-пазон всех возможных значений ошибок пред\-сказания составляет ${\cal A}_{\Delta}\hm= 
[-4096,+4096]$.\linebreak В~действительности диапазоны, в~которых частот\-ные распределения 
отличны от нуля, заметно менее широкие. Для состояний~1--4 томограммы~Т1, например, 
соответствующие диапазоны равны $[-1183,+348]$, $[-464,+384]$, $[-1271,+418]$ 
и~$[-910,+746]$, а~на рис.~\ref{fig3},\,\textit{б} 
представлены лишь <<центральные>> час\-ти соответствующих распределений. 
Условные частотные вероятности~$f({\cal I}|{\cal S})$ изображенного на 
рисунке диапазона $[-150,+150]$ для состояний~1--4 составляют $0{,}999502$, 
$0{,}996579$, $0{,}963086$ и~$0{,}629420$. Это означает, что нетривиальные 
части распределений для состояний~1--3 представлены на рис.~\ref{fig3},\,\textit{б} 
практически полностью. Отметим, что в~рас\-смат\-ри\-ва\-емом примере полные 
частотные вероятности $f({\cal S})$ состояний~1--4 со\-став\-ля\-ют $0{,}536415$, 
$0{,}146095$, $0{,}062107$ и~$0{,}045097$ соответственно.

Рисунок~\ref{fig3},\,\textit{б} показывает, что частотные распределения 
представляют собой достаточно <<гладкие>> функции. Это дает основания 
полагать, что их аппроксимация в~рамках описанной 
в~подразд.~2.4 общей схемы приведет в~конечном счете к~малой избыточности 
арифметического кодирования.  Кроме того, нетрудно заметить, что частотные 
распределения в~целом имеют симметричный относительно нуля вид. 
Поэтому в~работе~\cite{b01} было предложено ограничиться использованием 
кодовых распределений, также симметричных относительно нуля. При этом общая 
схема аппроксимации несколько изменяется; рассмотрим соответствующие изменения.

Зафиксируем некоторое состояние ${\cal S}\hm\neq{\cal S}_0$, $f(\Delta|{\cal S})$~--- 
соответствующее условное частотное распределение. Пусть $q(\Delta|{\cal S})$~---  
симметричное относительно значения $\Delta\hm= 0$  кодовое распределение 
$q(\Delta|{\cal S})\hm= q(-\Delta|{\cal S})$ и,~кроме того, 
$q(0|{\cal S})\hm= f(0|{\cal S})$. В~таком случае формулу~(\ref{eq9})
 для избыточности арифметического кодирования состояния можно тождественно 
 переписать в~виде двух слагаемых:
\begin{equation}
\label{eq27}
R(\boldsymbol{\Delta}|{\cal S}) =
R_G(\boldsymbol{\Delta}|{\cal S)} + R_Q(\boldsymbol{\Delta}|{\cal S)} \,,
\end{equation}
где
\begin{equation}
\label{eq28}
R_G(\boldsymbol{\Delta}|{\cal S}) = \sum\limits_{\Delta\in{\cal A}_{\Delta}}
f(\Delta|{\cal S}) \left[-\log_{2}\fr{g(\Delta|{\cal S})}{f(\Delta|{\cal S})}\right]\,;
\end{equation}
\begin{equation}
\hspace*{-2mm}R_Q(\boldsymbol{\Delta}|{\cal S}) = 2\hspace*{-2mm}
 \sum\limits_{\Delta\in{\cal A}_{\Delta},\,\Delta>0}\hspace*{-2mm}
g(\Delta|{\cal S}) \left[-\log_{2}\fr{q(\Delta|{\cal S})}{g(\Delta|{\cal S})}\right]\,.
\!\!\label{eq29}
\end{equation}
Входящая в~формулы~(\ref{eq28}) и~(\ref{eq29}) функция $g(\Delta|{\cal S})$ равна
\begin{equation}
\label{eq30}
g(\Delta|{\cal S}) = \fr{1}{2} \left[ f(\Delta|{\cal S}) + f(-\Delta|{\cal S}) \right]
\end{equation}
и~представляет собой результат симметризации условного частотного 
распределения~$ f(\Delta|{\cal S})$.

Первое слагаемое в~формуле~(\ref{eq27}) представ\-ляет собой избыточность 
арифметического ко\-ди\-ро\-вания исходного распределения~$f$ посредством 
симметризованного распределения~$g$, не зависит от кодового распределения и~называется 
далее \textit{избыточностью симметризации}. 
Второе слагаемое~--- это избыточность кодирования симметричного распределения~$g$ 
симметричным кодовым распределением~$q$ при условии $q(0)\hm= g(0) \hm= f(0)$.

Заметим, что использование при арифметическом кодировании симметричного 
относительно значения нуль кодового распределения эквивалентна использованию 
вероятностей $2q(|\Delta|)$ для описания абсолютных значений ошибки 
предсказания~$|\Delta|$ ($|\Delta|\hm\neq 0)$ и~одного бита для описания знака ошибки. 
При этом избыточность симметризации~--- <<плата>> за использование отдельного бита 
для описания знака.

Для симметричных распределений избыточность симметризации обращается в~нуль, 
а~для распределений, близких к~симметричным, невелика. Для рассматриваемого примера 
(томограммы~Т1) избыточность симметризации состояний~1--4 
составляет~$0{,}001949$, $0{,}005553$, $0{,}025371$ и~$0{,}046791$~б/п 
соответственно, а~полная (суммарная) избыточность симметризации, равная
\begin{equation}
\label{eq31}
R_G(\boldsymbol{\Delta}) = \sum\limits_{{\cal S}\neq{\cal S}_0}
f({\cal S}) R_{G}(\boldsymbol{\Delta}|{\cal S})\,,
\end{equation}
составляет $0{,}005543$~б/п.

Приведенные данные показывают, что издержки, обусловленные использованием 
только сим\-мет\-рич\-ных кодовых распределений, весьма незначительны. Поэтому 
для всех состояний (кроме\linebreak фоново\-го) можно ограничиться построением именно 
таких распределений. Кодовые вероятности должны строиться как результат 
аппроксимации симметризованных частотных распределений~$g$, 
чтобы минимизировать величину~(\ref{eq29}). Такая задача проще исходной 
общей задачи, поскольку, во-пер\-вых, ее нужно решить только для диапазона 
значений $\Delta\hm> 0$ и, во-вто\-рых, усреднение~(\ref{eq30}) несколько 
увеличивает гладкость функции, подлежащей аппроксимации. Кроме того, 
использование симметричных кодовых распределений уменьшает количество 
параметров, которые необходимо передавать в~преамбуле, т.\,е.\ 
уменьшает избыточность передачи~$R_{\mathrm{T}}$.

Введем ряд необходимых обозначений. Обозначим через $a_{\max}^+({\cal S})$ 
максимальное значение, которое принимает модуль ошибки предсказания 
в~состоянии~${\cal S}$. В~рассматриваемом примере величины $a_{\max}^+({\cal S})$ 
для состояний~0--4 равны~141, 1183, 464, 1271 и~910. 
Очевидно, что $f(\Delta|{\cal S}_0)\hm= 0$ при $\Delta\hm< 0$ 
и~$\Delta\hm> a_{\max}^+({\cal S}_0)$ для фонового состояния,  $g(\Delta|{\cal S})\hm= 
0$  при $|\Delta|\hm> a_{\max}^+({\cal S})$ для остальных состояний. 
Поэтому задачу аппроксимации достаточно решить для множества значений 
$[0,a_{\max}^+({\cal S})]$. Выделим значение нуль в~отдельный диапазон~$[0]$, 
состоящий из одной точки. Пусть ${\frak I}^+({\cal S})$~--- 
разбиение на диапазоны оставшегося множества значений $[1,a_{\max}^+({\cal S})]$ 
(для разных состояний используются разные разбиения). Для состояний 
${\cal S}\hm\neq{\cal S}_0$ введем в~рассмотрение величины:

\noindent
\begin{align*}
g({\cal I}|{\cal S}) &= \sum\limits_{\Delta\in{\cal I}\in{\frak I}^+(\cal S)} 
\hspace*{-2mm}g(\Delta|{\cal S})\,;
\\
g(\Delta|{\cal I},{\cal S})& =\fr{g(\Delta|{\cal S})}{g({\cal I},{\cal S})}, \enskip
\Delta\in{\cal I}\in{\frak I}^+(\cal S)\,.
\end{align*}
Для симметризованных частотных распределений данные величины являются аналогами 
величин $f({\cal I}|{\cal S})$ и~$f(\Delta|{\cal I},{\cal S})$ и~связаны с~ними 
сле\-ду\-ющим образом:

\noindent
\begin{align*}
g({\cal I}|{\cal S}) &= \fr{1}{2} \left[ 
f({\cal I}|{\cal S}) + f(-{\cal I}|{\cal S}) \right]\,; \\
g(\Delta|{\cal I},{\cal S}) &=
\fr{1}{2} \left[ f(\Delta|{\cal I},{\cal S}) + f(-\Delta|{\cal I},{\cal S}) \right]\,.
\end{align*}
Как и~ранее, для искомых условных кодовых распределений вероятности значений ошибки 
предсказания в~диапазоне ${\cal I}\hm\in{\frak I}^+({\cal S})$ состояния~${\cal S}$ 
использу\-ем обозначение $q(\Delta|{\cal I},{\cal S})$. С~учетом принятых обозначений 
формулу~(\ref{eq29}) можно переписать в~следующем виде:
\begin{equation}
\label{eq32}
R_Q(\boldsymbol{\Delta}|{\cal S}) = 2 \sum\limits_{{\cal I}\in{\frak I}^+({\cal S})}
g({\cal I}|{\cal S}) R_Q(\boldsymbol{\Delta}|{\cal I},{\cal S})\,,
\end{equation}
где

\vspace*{-4pt}

\noindent
\begin{multline}
\label{eq33}
R_Q(\boldsymbol{\Delta}|{\cal I},{\cal S}) = {}\\
{}=\hspace*{-2mm}
\sum\limits_{\Delta\in{\cal I}\in{\frak I}^+({\cal S)}}\hspace*{-4mm}
g(\Delta|{\cal I},{\cal S})
\left[-\log_{2}\fr{q(\Delta|{\cal I},{\cal S})}{g(\Delta|{\cal I},{\cal S})}\right]\,.
\end{multline}
Формулы~(\ref{eq32}) и~(\ref{eq33}) являются аналогами формул~(\ref{eq18}) 
и~(\ref{eq19}) для симметричного случая.

% Table 3
\begin{table*}\small
\begin{center}
\Caption{Оптимальные кодовые распределения для состояний томограммы~Т1}
\label{tab3}
\vspace*{2ex}

\begin{tabular}{|c|c|c|c|c|c|c|}
\hline
&&&&&&\\[-11pt]
${\cal S}$ & $({\cal I},\,{\cal I}^+)\,({\cal S})$ & 
$(f,g)({\cal I},{\cal S})$ & $\alpha$ & $\nu$ & $(R,R_Q)({\cal I},{\cal S})$ &
$(R,R_Q)({\cal S})$ \\
\hline
&&&&&&\\[-9pt]
         0 & $[1,~141]$ & $0{,}002721$ & $3,{3}9\cdot10^2$ & 3,5 & 2,335142 & 0,006354 \\
\hline
&&&&&&\\[-9pt]
 & $[1,~13]$ & 0,422606 & $6{,}51\cdot10^0$ & 1,5 & 0,000082 &  \\
1           & $[14,~32]$ & 0,041218 & $2{,}39\cdot10^1$ & 0,8 & 0,002555 &{0,002186}\\
           & $[33,~1183]$ & 0,005839 & $4{,}21\cdot10^6$ & 0,5 & 0,163215 &\\
\hline
&&&&&&\\[-9pt]
 & $[1,~21]$ & 0,420897 & $5{,}41\cdot10^0$ & 1,7 & 0,000354 & \\
2           &$[22,~60]$ & 0,053332 & $2{,}87\cdot10^1$ & 0,7 & 0,005296 &{0,007429}\\
           & $[61,~464]$ & 0,008277 & $4{,}15\cdot10^2$ & 0,5 & 0,396677 &\\
\hline
&&&&&&\\[-9pt]
 & $[1,~47]$ & 0,406425 & $4{,}84\cdot10^0$ & 3,4 & 0,004503 & \\
3           & $[48,~301]$ & 0,082673 & $9{,}42\cdot10^1$ & 0,5 & 0,078880 &{0,026150}\\
           & $[302,~1271]$ & 0,004699 & $2{,}24\cdot10^6$ & 0,8 & 1,005211 &\\
\hline
&&&&&&\\[-9pt]
 & $[1,~55]$ & 0,092243 & $1{,}00\cdot10^0$ & 0\hphantom{,0}   & 0,025029 & \\
4           & $[56,~249]$ & 0,335138 & $3{,}60\cdot10^0$ & 1,4 & 0,022354 & {0,047695}\\
           & $[250,~910]$ & 0,071519 & $2{,}08\cdot10^5$ & 1,0 & 0,196409 &\\
\hline
\end{tabular}
\end{center}
\end{table*}


Конкретизируем класс функций, используемых при аппроксимации. 
Частотное распределение для фонового состояния является вырожденным (см.~выше), 
и~выбор класса диктуется не\-об\-хо\-ди\-мостью аппроксимировать симметризованные час\-тот\-ные 
распределения остальных состояний. В~об\-ласти $\Delta\hm> 0$ эти распределения 
в~целом имеют тенденцию убывать с~ростом аргумента~$\Delta$, поэтому можно 
ограничиться рассмотрением не\-воз\-рас\-та\-ющих функций. Невозрастающие, 
положительные на отрезке~$[0,1]$ линейные функции, принимающие значение~$1$ 
в~точке нуль, имеют вид:
$$
p_{\mathrm{lin}}\left(\xi;\alpha\right) = 1 - 
\fr{\alpha-1}{\alpha}\xi\,, \quad \alpha>1\,.
$$
В соответствии с~формулой~(\ref{eq20}) параметр~$\alpha$ определяет отношение 
условных кодовых вероятностей в~начале~$a_B$  и~в~кон\-це~$a_E$ рассматриваемого 
диапазона:

\noindent
$$
\alpha = \fr{q\left(a_B;\alpha|{\cal I},{\cal S}\right)}{q\left(a_E;\alpha|{\cal I},
{\cal S}\right)}\,.
$$
Убывающие экспоненциальные функции, принимающие значение~1 в~точке нуль, имеют вид:

\noindent
$$
p_{\exp}(\xi;\alpha,\nu) = \exp\left(-\ln\alpha\cdot\xi^\nu\right), 
\enskip \alpha>1,\ \nu>0\,,
$$

\pagebreak

\noindent
причем смысл параметра~$\alpha$ тот же, что и~выше. Далее будем использовать 
соглашение, в~соответствии с~которым $p(\xi;\alpha,0)\hm= p_{\mathrm{lin}}
(\xi;\alpha)$  и~$p(\xi;\alpha,\nu)\hm= p_{\exp}(\xi;\alpha,\nu)$ при условии 
$\nu\hm> 0$.



Рассмотрим класс~${\cal P}$, состоящий из функций $p(\xi ;\alpha,\nu)$. 
Определим множество возможных значений параметров. Множество значений параметра~$\nu$ 
состоит из значения~0 и~значений~0,5, 0,6, \ldots,~3,5, всего~32~возможных значения. 
Множество значений параметра~$\alpha$ включает те числа, которые в~нормализованном 
десятичном представлении имеют мантиссу, состоящую из трех цифр, и~порядок от~0 
до~7. Указанный класс оказывается достаточным для успешного решения задачи 
аппроксимации.

При заданных значениях параметров~$\nu$ и~$\alpha$  условное кодовое 
распределение для диапазона $q(\Delta|{\cal I},{\cal S}) \hm= 
q(\Delta;\nu,\alpha|{\cal I},{\cal S})$, $\Delta\hm\in{\cal I}\hm\in{\frak I}^+({\cal S})$, 
определяется по соответствующей функции класса~${\cal P}$  в~соответствии 
с~формулой~(\ref{eq20}). Общее кодовое распределение $q(\Delta|{\cal S}_0)$ 
в~случае фонового со\-сто\-яния определяется формулой~(\ref{eq17}), 
в~остальных случаях используется аналогичная формула $q(\Delta|{\cal S})\hm= 
g({\cal I}|{\cal S})q(\Delta|{\cal I},{\cal S})$.

При построении кодового распределения для фонового состояния 
достаточно единственного диапазона $[1,a_{\max}^+({\cal S}_0)]$, т.\,е.\
 разбиение ${\frak I}^+({\cal S}_0)$ тривиально, при этом параметры~$\nu$ 
 и~$\alpha$  выбираются так, чтобы минимизировать избыточность~(\ref{eq19}) 
 (или~(\ref{eq18}), что в~данном случае одно и~то же). Кодовые распределения 
 для остальных состояний построим следующим образом: зафиксируем общее 
 число диапазонов  $I^+({\cal S})$ в~разбиении интервала $[1,a_{\max}^+({\cal S})]$ 
 и~выберем границы диапазонов и~параметры~$\nu$ и~$\alpha$ так, чтобы 
 минимизировать избыточность~(\ref{eq29}). Построенные таким образом кодовые 
 распределения будем называть \textit{оптимальными}, а~соответствующую процедуру~--- 
 \textit{оптимальной}.
 {\looseness=1
 
 }

В табл.~\ref{tab3} представлены результаты по\-стро\-ения оптимальных условных 
кодовых распределений ($I^+({\cal S})\hm=3$, ${\cal S}\hm\neq {\cal S}_0$) для 
состояний томограммы~Т1. В~первом столбце приведены номера\linebreak состояний, во втором~--- 
диапазоны разбиения. В~третьем столбце для фонового состояния приведено значениe
 величины $f({\cal I},{\cal S})$, а~для остальных состояний~--- 
 значения величины $g({\cal I},{\cal S})$ для каж\-до\-го из соответствующих диапазонов. 
 В~четвертом и~пятом столбцах приведены значения па\-ра\-мет\-ров~$\alpha$ и~$\nu$, 
 которые определяют условное кодовое распределение $q(\Delta|{\cal I},{\cal S})$. 
 В~шестом столбце для единственного использованного диапазона фонового состояния 
 приведено значение избыточности $R(\boldsymbol{\Delta}|{\cal I},{\cal S}) \hm= 
 R({\cal I},{\cal S})$, а~для остальных состояний приведены значения 
 избыточности $R_Q(\boldsymbol{\Delta}|{\cal I},{\cal S}) \hm= 
 R_Q({\cal I},{\cal S})$ для каждого из использованных диапазонов. 
 В~последнем столбце приведено значение полной избыточности 
 $R(\boldsymbol{\Delta}|{\cal S}) \hm= R({\cal S})$ для фонового состояния и~значения 
 полной избыточности $R_Q(\boldsymbol{\Delta}|{\cal S}) \hm= 
 R_Q({\cal S})$ для остальных состояний.


Приведенные в~табл.~\ref{tab3} данные позволяют получить точную оценку 
полной избыточности арифметического кодирования всей томограммы~Т1 по формуле:
$$
R(\boldsymbol{\Delta}) = f({\cal S}_0) R(\boldsymbol{\Delta}|{\cal S}_0) +
\sum\limits_{{\cal S}\neq{\cal S}_0} f({\cal S})R_Q({\cal S}) + 
R_G(\boldsymbol{\Delta})\,,
$$
где третье слагаемое (полная избыточность симметризации), 
которое задается формулами~(\ref{eq28}) и~(\ref{eq31}), уже было 
вычислено ранее и~составляет~$0{,}005543$~б/п. Используя приведенные ранее 
значения $f(\cal S)$, ${\cal S}\hm\in\tilde{\frak S}^5$, находим, что искомая 
избыточность равна~$0{,}012912$~б/п.

% Table 4
\begin{table*}[b]\small
\begin{center}
\Caption{Избыточность кодирования (оптимальная процедура)}
\label{tab4}
\vspace{2ex}

\begin{tabular}{|c|c|c|c|c|c|}
\hline
&&&&&\\[-9pt]
 T & $R_G$ &  $R$ & $R_{\mathrm{T}}$ & $R+R_{\mathrm{T}}$ & 
 $(R+R_{\mathrm{T}})/\tilde{H}^3$ \\
\hline
T1 & 0,005543 & 0,012912 & \multicolumn{1}{c|}
{\raisebox{-28pt}[0pt][0pt]{0,002518}} & 0,015429 & 0,003439\\
%\cline{1-3} \cline{5-6}
T2 & 0,005890 & 0,012415 &                      & 0,014933 & 0,003129\\
%\cline{1-3} \cline{5-6}
T3 & 0,009416 & 0,019322 &                      & 0,021840 & 0,003469\\
%\cline{1-3} \cline{5-6}
T4 & 0,009942 & 0,020528 &                      & 0,023046 & 0,003714\\
%\cline{1-3} \cline{5-6}
T5 & 0,006532 & 0,012582 &                      & 0,015099 & 0,003578\\
%\cline{1-3} \cline{5-6}
T6 & 0,008045 & 0,013486 &                      & 0,016004 & 0,004042\\
\hline
\end{tabular}
\end{center}
\end{table*}


Оценим избыточность передачи~$R_{\rm T}$, т.\,е.\ избыточность, связанную 
с~необходимостью передавать значения вычисляемых на этапе кодирования параметров
 декодеру. К~числу таких параметров относятся значения трех квазиоптимальных порогов, 
 границы диапазонов разбиения (по одной верхней границе на каждый диапазон), 
 условные частотные вероятности диапазонов и~значения параметров~$\alpha$  
 и~$\nu$ в~каждом диапазоне. Функция $\sigma_1(U,L)\hm= |U-L|$, участвующая 
 в~определении состояний, принимает значения от~0 до~4096, 
 поэтому область возможных значений порогов~--- $[1,\,4096]$; следовательно, 
 для описания трех значений порогов достаточно $3\cdot 12\hm=36$~бит. 
 Аналогично для описания одной границы диапазона достаточно~12~бит. 
 Поскольку томограмма содержит~$2^{18}$~пикселей, то для описания условной 
 частотной вероятности одного диапазона заведомо достаточно~18~бит. 
 Для описания значений пары параметров~$\alpha$  и~$\nu$  требуется, как  легко видеть, $(10\hm+5)\hm+3\hm= 18$~бит. Итак, для описания одного диапазона 
 нужно~48~бит, а~общее число диапазонов равно $1\hm+4\cdot 3\hm=13$. 
 Таким образом, полная длина преамбулы составляет~660~бит, 
 а~избыточность передачи, равная отношению длины преамбулы к~общему числу пикселей, 
 составляет~0,002518~б/п.

Складывая индивидуальную избыточность арифметического кодирования томограммы~T1 
и~избыточность передачи, одинаковую для любой томограммы, получаем общую избыточность 
кодирования метода для томограммы~T1: $0{,}015429$~б/п.

В табл.~\ref{tab4} приведены оценки минимальной избыточности кодирования 
томограмм~T1--T6 с~использованием описанной выше оптимальной процеду\-ры. 
Результаты для томограмм~T2--T6 получены в~точности так же, как и~соответствующие 
результаты для томограммы~T1 ранее; детали, относящиеся к~разбиениям на диапазоны 
и~условным кодовым распределениям (аналогичные приведенным в~табл.~\ref{tab3} 
данным для томограммы~T1), опущены в~целях экономии места. 
В~первом столбце таблицы указана томограмма, в~остальных столбцах~--- 
соответствующие значения избыточности сим\-мет\-ри\-за\-ции, индивидуальной избыточности 
арифметического кодирования (с~учетом сим\-мет\-ри\-за\-ции), избыточности передачи, общей 
избыточности кодирования метода и~относительной общей избыточности кодирования 
($H\hm=\tilde{H}^3$).


Представленные в~табл.~\ref{tab4} данные свидетельствуют об 
исключительном качестве предложенного метода построения оптимальных 
кодовых вероятностей. Действительно, даже в~худшем случае (томограмма~T6) 
относительная общая избыточность едва превышает~0,4\%. 
Таким образом, метод позволяет практически достигнуть нижней границы ско\-рости 
кодирования (квазиэнтропии), что, в~свою очередь, подтверждает справедливость 
гипотез, использованных при разработке метода.

Рассмотренная оптимальная процедура построения кодовых распределений 
предполагает решение многомерной оптимизационной задачи по минимизации 
избыточности~(\ref{eq29}) в~пространстве значе\-ний параметров аппроксимации и~границ 
диапазонов при фиксированном общем числе диапазонов. Для всех состояний, за 
исключением фонового (где определение оптимальных границ не требуется), данная 
задача является вычислительно сложной и~не может быть решена за приемлемое время 
на этапе кодирования. Поэтому оптимальная процедура может быть использована 
в~алгоритмах сжатия, ориентированных на практическое применение, только для 
фонового состояния.

% Table 5
\begin{table*}[b]\small 
\vspace*{-12pt}
\begin{center}
\Caption{Кодовые распределения для состояний томограммы~Т1 (процедура уравнивания)}
\label{tab5}
\vspace{2ex}

\tabcolsep=10pt
\begin{tabular}{|c|c|c|c|c|c|c|}
\hline
${\cal S}$ & $({\cal I},\,{\frak I}^+)({\cal S})$ & 
$(f,g)({\cal I},{\cal S})$ & $\alpha$ & $\nu$ & $(R,R_Q)({\cal I},{\cal S})$ &
$(R,R_Q)({\cal S})$\\
\hline
&&&&&&\\[-9pt]
         0 & $[1,~141]$ & 0,002721 & $3{,}39\cdot10^2$ & 3,5 & 2,335142 & 0,006354 \\
\hline
&&&&&&\\[-9pt]
\multicolumn{1}{|c|}
{\raisebox{-6pt}[0pt][0pt]{1}} & $[1,~28]$ & 0,461939 & $1{,}87\cdot10^2$ & 1,3 & 0,002176 & 
\multicolumn{1}{c|}
{\raisebox{-6pt}[0pt][0pt]{0,004097}}\\
           & $[29,~1183]$ & 0,007723 & $1{,}15\cdot10^7$ & 0,5 & 0,135063 &\\
\hline
&&&&&&\\[-9pt]
\multicolumn{1}{|c|}
{\raisebox{-6pt}[0pt][0pt]{2}} & $[1,~51]$ & 0,471905 & $2{,}17\cdot10^2$ & 1,3 & 0,008063 & 
\multicolumn{1}{c|}
{\raisebox{-6pt}[0pt][0pt]{0,014564}}\\
           & $[52,~464]$ & 0,010601 & $7{,}67\cdot10^2$ & 0,6 & 0,327971 &\\
\hline
&&&&&&\\[-9pt]
 & $[1,~92]$ & 0,457159 & $7{,}13\cdot10^1$ & 1,8 & 0,019413 &\\
3           & $[93,~424]$ & 0,036454 & $2{,}66\cdot10^1$ & 0,6 & 0,247027 &0,037744\\
           & $[425,~1271]$ & 0,000184 & $8{,}89\cdot10^2$ & 0,5 & 5,390520 &\\
\hline
&&&&&&\\[-9pt]
 & $[1,~119]$ & 0,254779 & $1{,}00\cdot10^0$ & 0\hphantom{,0}   & 0,048731 & \\
4           & $[120,~393]$ & 0,240611 & $5{,}58\cdot10^1$ & 0\hphantom{,0}   & 0,051161 &0,062751\\
           & $[394,~910]$ & 0,003510 & $4{,}01\cdot10^2$ & 0,5 & 1,894615 &\\
\hline
\end{tabular}
\end{center}
\end{table*}

% Table 6
\begin{table*}\small
\begin{center}
\Caption{Избыточность кодирования (процедура уравнивания)}
\label{tab6}
\vspace{2ex}

\begin{tabular}{|c|c|c|c|c|c|}
\hline
&&&&&\\[-9pt]
T  & $R_G$ &  $R$ & $R_{\mathrm{T}}$ & $R+R_{\mathrm{T}}$ & 
$(R+R_{\mathrm{T}})/\tilde{H}^3$\\
\hline
T1 & 0,005543 & 0,016378 & 0,002151 & 0,018529 & 0,004130\\
%\hline
T2 & 0,005890 & 0,015653 & 0,002151 & 0,017804 & 0,003731\\
%\hline
T3 & 0,009416 & 0,022028 & 0,001968 & 0,023996 & 0,003812\\
%\hline
T4 & 0,009942 & 0,023203 & 0,002151 & 0,025354 & 0,004087\\
%\hline
T5 & 0,006532 & 0,015675 & 0,002334 & 0,018009 & 0,004268\\
%\hline
T6 & 0,008045 & 0,016231 & 0,002151 & 0,018382 & 0,004642\\
\hline
\end{tabular}
\end{center}
\end{table*}

Рассмотрим альтернативную процедуру построения кодовых распределений для состояний, 
отличных от фонового (${\cal S}\hm\neq {\cal S}_0$), которая, в~отличие от 
оптимальной процедуры, допускает <<быструю>> реализацию. Зафиксируем общее 
для всех со\-сто\-яний число~$I_{\max}^+$~--- 
максимально возможное число диапазонов разбиения интервалов 
$[1,\,a_{\max}^+({\cal S})]$. Зафиксируем некоторое положительное число~$\rho\hm> 0$ 
и~определим для каждого состояния величину
$$
r({\cal S}) = \fr{\rho H}{2( S-1) I_{\max}^+ f({\cal S})}\,.
$$
Напомним, что $H\hm=H(\boldsymbol{\Delta})$~--- квазиэнтропия томограммы; $f(\cal S)$~--- 
частотная вероятность состояния; $S\hm=5$~--- 
общее число состояний. Для каждого состояния разбиение интервала 
$[1,\,a_{\max}^+({\cal S})]$ на диапазоны осуществляется рекурсивно. 
В~начале очередного шага рекурсии известно начало очередного диапазона~$a_B$,
 требуется определить его конец~$a_E$ и~параметры аппроксимации. Конец 
 текущего диапазона выбирается следующим образом:
$$
a_E = \hspace*{-8pt}\max\limits_{a_B\le a\le a_{\max}^+({\cal S})}\hspace*{-5pt}
\left\{ a:\, g({\cal I}_a,{\cal S})
R_Q(\boldsymbol{\Delta}|{\cal I}_a,{\cal S}) \le r({\cal S})\right\},
$$
где ${\cal I}_a\hm= [a_B,\,a]$, а~$R_Q(\boldsymbol{\Delta}|{\cal I}_a,{\cal S})$~--- 
решение оптимизационной задачи по минимизации выражения~(\ref{eq33}). 
Одновременно с~определением искомого конца диапазона становятся 
известны и~параметры аппроксимации, поскольку решение оптимизационной 
задачи предполагает их нахождение. Рекурсия завершается в~любом из следующих случаев:
\begin{enumerate}[(1)]
\item исчерпан весь интервал $[1,\,a_{\max}^+({\cal S})]$ 
(полученное очередное  значение~$a_E$ равно $a_{\max}^+({\cal S})$);
\item исчерпано максимально допустимое общее чис\-ло диапазонов разбиения (в~этом 
случае просто полагается $a_E\hm= a_{\max}^+({\cal S})$).
\end{enumerate}

Нетрудно видеть, что правая граница очередного диапазона разбиения 
(кроме последнего) для любого состояния выбирается так, чтобы вклад 
избыточности кодирования этого диапазона в~суммарную избыточность кодирования 
был порядка $\rho H/[2( S\hm-1) I_{\max}^+]$, но не превышал указанной величины. 
В~результате все построенные диапазоны всех состояний (за исключением последних 
диапазонов каждого из состояний) вносят приблизительно равные вклады в~суммарную 
избыточность кодирования. Поэтому далее будем называть рассматриваемую процедуру 
\textit{процедурой уравнивания}. 
Кроме того, ясно, что если завершение рекурсии для всех состояний было связано 
с~выполнением первого из указанных выше условий, то суммарная избыточность 
кодирования не превышает значения~$\rho H$. Таким образом, используемая в~процедуре 
уравнивания величина~$\rho$ играет роль общего относительного <<целевого>> 
уровня кодовой избыточ\-ности.

Основное достоинство процедуры уравнивания заключается в~том, что она 
допускает быструю реализацию и, следовательно, может использоваться на 
практике в~процессе кодирования.

Анализ приведенных выше данных, относящихся к~избыточности кодирования с~использованием 
оптимальной процедуры, показывает, что разумным, например, является выбор значения\linebreak 
$\rho\hm=0{,}004$ и~использование $I_{\max}^+\hm=4$, т.\,е.\ не более 
четырех диапазонов разбиения для каждого состояния. Указанные значения используются 
всюду далее. 

В~табл.~\ref{tab5} представлены результаты построения условных 
кодовых распределений для состояний томограммы~T1 с~по\-мощью процедуры уравнивания. 
Таблица имеет ту же структуру, что и~табл.~\ref{tab3}.



Избыточность кодирования $R_Q({\cal S})$ отдельных состояний, как и~следовало 
ожидать, оказалась несколько больше, чем при использовании оптимальных 
кодовых распределений. Суммарный вклад величин $R_Q({\cal S})$ в~общую избыточность 
кодирования метода составляет менее $0{,}01$~б/п, т.\,е.\ 
менее 0{,}23\% от величины квазиэнтропии томограммы, что заметно меньше 
использованного в~процедуре уравнивания значения целевого уровня кодовой избыточности. 
Полная избыточность арифметического кодирования всей томограммы~Т1, 
включающая вклад фонового состояния и~избыточность симметризации, 
равна~0,016378~б/п. Избыточность передачи~$R_{\mathrm{T}}$ 
при использовании процедуры уравнивания зависит от общего числа реально использованных 
диапазонов и~в~данном случае составляет~0,002151~б/п, что несколько меньше 
избыточности передачи при оптимальной процедуре. Наконец, общая избыточность 
кодирования метода, использующего процедуру уравнивания при построении 
кодовых вероятностей, для томограммы~T1 со\-став\-ля\-ет~0,018529~б/п.

В табл.~\ref{tab6} приведены оценки избыточности кодирования всех томограмм~T1--T6 
с~использованием описанной выше процедуры уравнивания при построении кодовых 
вероятностей. Таблица имеет ту же структуру, что и~табл.~\ref{tab4}. Общее число 
используемых диапазонов не является фиксированным, равно

\noindent
$$
 I(\frak S) =1 +  \sum\limits_{{\cal S}\in{\frak S},\, {\cal S}\neq{\cal S}_0} 
 I^+({\cal S})\,,
$$
где $I^+({\cal S})$~--- число диапазонов в~построенном разбиении 
${\frak I}^+({\cal S})$, и~может варьироваться от томограммы к~томограмме. 
Поэтому и~значения избыточности передачи~$R_{\mathrm{T}}$ в~табл.~\ref{tab6} несколько 
различаются.



Представленные в~табл.~\ref{tab6} данные свидетельствуют об отличном 
качестве предложенного метода построения кодовых вероятностей на основе 
процедуры уравнивания. В~худшем случае (томограмма~T6) относительная общая 
избыточность не превышает~0,5\%. В~действительности результаты применения 
процедуры уравнивания лишь немного уступают аналогичным результатам, 
полученным с~использованием оптимальной процедуры. При этом время 
вычислений оказывается несопоставимо меньше.

\vspace*{-6pt}


\section{Кодирование компонент дискретного вейвлет-преобразования}

Адаптируем теперь представленную в~разд.~2 общую схему так, чтобы получить 
метод сжатия, основанный на универсальном кодировании значений компонент 
дискретного вейв\-лет-пре\-об\-ра\-зо\-ва\-ния (ДВП) томограмм. Метод был впервые 
предложен в~работе~\cite{b05}.

\vspace*{-6pt}

\subsection{Дискретное вейвлет-преобразование}

Напомним, прежде всего, что представляет собой ДВП последовательности. 
Пусть $\mathbf{x}\hm=\{x_n\}$~--- суммируемая с~квадратом вещественная 
последовательность целого аргумента. Прямое ДВП \mbox{такой} последовательности~--- 
это разложение данной последовательности на две составляющие (компоненты), 
которое осуществляется следующим образом.\linebreak Сначала вычисляются свертки 
последователь\-ности~$\mathbf {x}$  с~двумя заданными фильтрами разложения 
(анализа), низкочастотным фильтром~${\boldsymbol\mu}^0$ и~высокочастотным 
фильтром~${\boldsymbol\mu}^1$  (оба фильтра суть сум\-мируемые с~квадратом 
вещественные последовательности). Затем две полученные в~результате 
фильт\-ра\-ции последовательности прореживаются, т.\,е.\ в~них удерживаются лишь 
четные члены. В~итоге имеем суммируемые с~квадратом вещественные последовательности 
целого аргумента~$\mathbf{x}^{0}$ и~$\mathbf{x}^{1}$:

\vspace*{2pt}

\noindent
\begin{equation}
\label{eq34}
x_n^i=\sum\limits_{k=-\infty}^{+\infty} \mu_{2n-k}^{i}\,x_k\,, \quad i=0,\,1\,,
\end{equation}

\vspace*{-2pt}

\noindent
которые представляют собой приближение (низкочастотную составляющую) и~детальную 
(высокочастотную) составляющую исходной последовательности, при этом каждая 
имеет вдвое меньшее разрешение.

Обратное ДВП --- это восстановление исход-\linebreak ной последовательности по ее 
приближению\linebreak и~детальной составляющей. Разбавим обе 
по\-сле\-до\-ва\-тель\-ности~$\mathbf{x}^{0}$ и~$\mathbf{x}^{1}$ нулями (т.\,е.\ 
построим последовательности с~нулевыми нечетными членами и~четными членами, 
заданными последовательностями~$\mathbf{x}^{0}$ и~$\mathbf{x}^{1}$), 
затем вычислим свертки полученных последовательностей с~некоторыми фильтрами 
синтеза~$\boldsymbol\nu^0$ и~$\boldsymbol\nu^1$ и~сложим результаты:

\vspace*{2pt}

\noindent
\begin{equation}
\label{eq35}
\tilde{x}_n =\sum\limits_{k=-\infty}^{+\infty}
\left(\nu_{n-2k}^{0}\,x_k^0 + \nu_{n-2k}^{1}\,x_k^1 \right) \,.
\end{equation}

\vspace*{-2pt}

\noindent
Чтобы преобразование~(\ref{eq35}) действительно было обратным по отношению 
к~преобразованию~(\ref{eq34}),
т.\,е.\ чтобы выполнялось равенство $\mathbf{x}\hm=\tilde{\mathbf{x}}$,
система фильтров~$\boldsymbol\mu^0$, $\boldsymbol\mu^1$, 
$\boldsymbol\nu^0$ и~$\boldsymbol\nu^1$ должна удовлетворять условию 
восстановления (см.,~например,~\cite{b06}). Этому условию, в~част\-ности, 
удовлетворяет система фильтров конечной длины, впервые предложенная в~\cite{b07}:

\pagebreak

\end{multicols}

\noindent
{ %\tiny %scriptsize %\footnotesize
\begin{equation}
\left.
{\begin{array}{llllll}
\mu_{-2}^0=-\fr{1}{8},&\ \mu_{-1}^0=\fr{1}{4},&\ \mu_{0}^0=\fr{3}{4},&\ \mu_{1}^0=\fr{1}{4},
&\ \mu_{2}^0=-\fr{1}{8}; &\ \\[12pt]
\mu_{-2}^1=-\fr{1}{2},&\ \mu_{-1}^1=1,&\ \mu_{0}^1=-\fr{1}{2}; &\ &\ &\ \\[12pt]
&\ \nu_{-1}^0=\fr{1}{2},&\ \nu_{0}^0=1,&\ \nu_{1}^0=\fr{1}{2}; &\ &\ \\[12pt]
&\ \nu_{-1}^1=-\fr{1}{8},&\ \nu_{0}^1=-\fr{1}{4},&\ \nu_{1}^1=\fr{3}{4},&\ \nu_{2}^1=-\fr{1}{4},&\
\nu_{3}^1=-\fr{1}{8};
\end{array}}
\right\}\!\!\!
\label{eq36}
\end{equation}

}

\begin{multicols}{2}

\noindent
остальные коэффициенты фильтров равны нулю. Эта система, получившая широкое 
распространение и~вошедшая в~стандарт сжатия JPEG 2000, используется в~настоящей 
работе.

Эффективность применения ДВП для сжатия данных обусловлена тем, что 
при удачном выборе фильтров разложения значения детальных со\-став\-ля\-ющих 
распределены значительно более неравномерно, чем значения исходного сигнала. 
Однако требование обратимости (отсутствия искажений) при сжатии сигналов 
конечной длины с~целыми значениями приводит к~серьезному дополнительному 
требованию: ДВП должно быть свободно от ошибок округления. Если 
ДВП с~фильтрами~(\ref{eq36}) реализуется непосредственно по 
формулам~(\ref{eq34}) и~(\ref{eq35}),\linebreak то для того чтобы избежать 
ошибок округления, необходи\-мо на этапе вычисления прямого преобразования 
добавить дополнительные биты для хранения остатков от деления: три бита 
для приближения и~один бит для детальной составляющей. По\-па\-да\-ющие в~эти 
дополнительные <<младшие>>\linebreak
 биты остатки представляют собой практически 
<<белый шум>>, что неприемлемо с~точки зрения последующего сжатия. 
Решение проб\-ле\-мы заключается в~том, чтобы реализовывать ДВП посредством 
так называемой лиф\-тинг-схе\-мы~\cite{b08}. При этом появляется возможность 
взаимосогласованным образом округлять результаты при вычислении прямого и~обратного 
преобразований, что, во-пер\-вых, обеспечивает точное восстановление 
и,~во-вто\-рых, избавляет от необходимости хранить (и~сжимать) остатки. 
Кроме того, аккуратный учет краевых эффектов в~лиф\-тинг-схе\-ме позволяет 
добиться того, что для конечной последовательности~$\mathbf{ x}$ длины~$N$ 
приближение~$\mathbf{ x}^0$ и~детальная составляющая~$\mathbf{x}^1$ имеют 
длины, в~точ\-ности равные $[N/2+1]_-$ и~$[N/2]_-$ соответственно.

В двумерном случае ДВП представляет собой суперпозицию одномерных преобразований, 
применяемых раздельно к~строкам и~столбцам. Если $\mathbf{X}\hm=\{X_{l,m}\}$~--- 
двумерная вещественная последовательность, то прямое ДВП имеет 
аналогичный~(\ref{eq34}) вид:

\noindent
$$
X_{l,m}^{i,j}=\sum\limits_{k,\,k'=-\infty}^{+\infty}
\mu_{2l-k}^{i}\,\mu_{2m-k'}^{j}\,X_{k,k'}\,,
\enskip i,j=0,\,1\,,
$$
а его результатом является разложение исходной последовательности 
на четыре компоненты (со\-став\-ля\-ющие) вдвое меньшего разрешения.

Последовательность $\mathbf{X}^{0,0}\doteq\mathbf{X}^{\mathrm{A}}$, 
полученная с~применением низкочастотной фильтрации по строкам и~столбцам, 
представляет собой приближение, а~остальные три последовательности
$\mathbf{X}^{0,1}\doteq\mathbf{X}^{\mathrm{V}}$, 
$\mathbf{X}^{1,0}\doteq\mathbf{X}^{\mathrm{H}}$ 
и~$\mathbf{X}^{1,1}\doteq\mathbf{X}^{\mathrm{D}}$, в~построении\linebreak которых 
участвует высокочастотный фильтр,~--- детальные со\-став\-ля\-ющие (вертикальные, 
горизонтальные и~диагональные соответственно). Такое разложение называется 
\textit{одномасштабным} ДВП. Обратное преобразование производится очевидным 
образом при помощи одномерных обратных ДВП, применяемых раздельно к~строкам и~столбцам, 
и~описывается двумерным аналогом формулы~(\ref{eq35}). Как и~в~одномерном случае, 
возможна лиф\-тинг-ре\-а\-ли\-за\-ция двумерного ДВП. Это дает возможность 
использовать целую арифметику и~добиться того, что для изображения 
(конечной двумерной последовательности, т.\,е.\ мат\-ри\-цы) размера 
$L\times M$ приближение и~вертикальные, горизонтальные и~диагональные составляющие 
имеют следующие размеры: $[L/2+1]_-\times[M/2+1]_-$, 
$[L/2+1]_-\times[M/2]_-$, $[L/2]_-\times[M/2+1]_-$ и~$[L/2]_-\times[M/2]_-$.

\begin{figure*} %fig4
\vspace*{1pt}
\begin{center}
\mbox{%
\epsfxsize=148.291mm
\epsfbox{ste-4.eps}
}
\end{center}
\vspace*{-9pt}
\Caption{Одномасштабное разложение томограмм~T3~(\textit{а}) и~T6~(\textit{б})}
\label{fig4}
\vspace*{4pt}
\end{figure*}


Результатом применения прямого ДВП к~томограмме~$\mathbf{X}$ размера $512\times512$ 
является ее одномасштабное разложение на четыре компоненты (изоб\-ра\-же\-ния) 
размерами $256\times256$: приближение~$\mathbf {X}^{\mathrm{A}}$ 
и~вертикальные~$\mathbf{X}^{\mathrm{V}}$, горизонтальные~$\mathbf{X}^{\mathrm{H}}$ 
и~диагональные~$\mathbf{X}^{\mathrm{D}}$ детальные составляющие. Первая ком-\linebreak понента~--- 
результат низкочастотной фильтрации (сглаживания) элементов исходного изображения 
по строкам и~столбцам, следующие две получены сглаживанием по одной координате 
и~вы\-со\-ко\-час\-тот\-ной фильтрацией по другой, а~последняя~--- 
результат высокочастотной фильтрации по обеим координатам (в~каждом случае 
фильт\-ра\-ция сопровождается прореживанием, уменьшающим размер вдвое 
по каждой координате). Как уже было указано выше, в~данной работе используется 
ДВП, основанное на системе фильтров~(\ref{eq36}), поэтому диапазоны значений 
яркости компонент равны соответственно  $[-2560,\,6656]$, 
$[-6144,\,6144]$, $[-6144,\,6144]$ и~$[-8192,\,8192]$. На рис.~\ref{fig4} 
представлены одномасштабные разложения томограмм~T3 и~T6. 
В~левом верхнем квадрате каждого рисунка располагается приближение для 
соответствующей томограммы (использованы те же окна визуализации, что и~на 
рис.~\ref{fig1}). В~правом верхнем, левом нижнем и~правом нижнем квадратах 
располагаются соответственно вертикальные, горизонтальные и~диагональные составляющие. 
При этом использованы существенно более узкие, чем для приближений, окна 
визуализации с~центром в~нуле: окно $[-150,\,+150]$ для детальных составляющих~T3 
и~окно $[-20,\,+20]$ для детальных составляющих~T6.

\vspace*{-4pt}


\subsection{Особенности метода кодирования}


В разд.~4 метод универсального кодирования ошибок предсказания для 
томограмм был рас\-смот\-рен во всех деталях. После внесения некоторых изменений 
непринципиального характера по существу тот же метод может быть с~успехом 
применен для кодирования компонент ДВП томограмм. Данный раздел посвящен 
рассмотрению тех особенностей, которые отличают метод сжатия (кодирования) 
компонент ДВП от аналогичного метода, используемого при сжатии ошибок предсказания.

Компоненты томограммы, полученные в~результате ДВП, сжимаются по отдельности 
(независимо друг от друга). Поскольку все компоненты имеют одинаковые размеры, 
значения квазиэнтропии и~избыточности для всей томограммы равны средним арифметическим 
значениям соответствующих величин для всех компонент ДВП. Как 
показывает рис.~\ref{fig4}, статистические свойства приближения и~детальных 
составляющих значительно различаются, поэтому и~методы их сжатия, вообще говоря, 
должны различаться.

Начнем с~рассмотрения метода сжатия компоненты~$\mathbf{X}^{\mathrm{A}}$ (приближения). 
Как показывает сравнение рис.~\ref{fig1} и~\ref{fig4}, приближение визуально 
почти не отличается от исходного изоб\-ра\-же\-ния. Стати\-стические свойства приближения 
и~исходного изоб\-ра\-же\-ния также близки, несмотря на расширение диапазона значений 
в~2,25~раза и~появление отрицательных значений. Поэтому для сжатия 
приближения можно прямо использовать описанный в~разд.~4 
метод и~кодировать значения~$\boldsymbol{\Delta}\mathbf{X}^{\mathrm{A}}$, т.\,е.\
значения ошибок предсказания приближения. В~час\-ти построения множества состояний 
метод вообще не претерпевает изменений. В~час\-ти по\-стро\-ения кодовых вероятностей в~метод 
требуется внести единственное изменение, которое затрагивает только процедуру 
по\-стро\-ения кодовых вероятностей для фонового со\-сто\-яния и~подробно рассмотрено далее.
{ %\looseness=-1

}

Рассмотрим теперь детальные составляющие. Отсчеты детальных составляющих 
принимают значения в~симметричных относительно нуля диапазонах, которые 
для компонент  $\mathbf{X}^{\mathrm{V}}$ и~$\mathbf{X}^{\mathrm{H}}$ в~3~раза, 
а~для компоненты $\mathbf{X}^{\mathrm{D}}$ в~4~раза шире, чем 
диапазон исходных значений. Однако функции распределения значений концентрируются 
в~небольших окрестностях нуля: на рис.~\ref{fig4} детальные составляющие менее 
контрастны, чем приближение, несмотря на то что для их визуализации использовано 
значительно более узкое окно. Поэтому следует кодировать непосредственно значения 
этих составляющих.

Детальные составляющие описывают отличие элементов исходного изображения 
от сглаженных значений. Как и~раньше, элементы, расположенные слева и~сверху 
от рассматриваемого, будем обозначать через~$U$ и~$L$. Однако теперь значения 
самих этих элементов характеризуют скорость изменения значений исходного 
изображения по крайней мере по одной координате. Поэтому при построении 
состояний в~формулах~(\ref{eq23}) и~(\ref{eq24}) вместо функции~$\sigma_1$ будем 
использовать функцию:

\vspace*{3pt}

\noindent
\begin{equation}
\label{eq37}
\sigma_2(U,L) = |U|+|L|\,.
\end{equation}
Такой выбор предпочтительнее, чем $|U\hm+L|$, поскольку~$U$ и~$L$ 
могут иметь разные знаки. Действительно, если  $U\hm\sim -L \hm\gg 0$, то скорость 
изменения значений элементов велика, но при этом $|U\hm+L|\hm\sim 0$. Замена функции, 
используемой при по\-стро\-ении состояний,~--- 
это главное изменение, которое необходимо внести в~метод при сжатии детальных 
составляющих.

В ходе выполнения работ был подробно исследован вопрос о~целесообразности 
использования при построении множества со\-сто\-яний функции~$\sigma_2$ другого вида. 
Действительно, вертикальные детальные со\-став\-ля\-ющие~$\mathbf{X}^{\mathrm{V}}$ 
получены низкочастотной фильтрацией по столбцам и~высокочастотной по строкам, 
а~горизонтальные детальные со\-став\-ля\-ющие~$\mathbf{X}^{\mathrm{H}}$~--- 
низкочастотной фильтрацией по строкам и~высокочастотной по столбцам. Эта асим\-мет\-рия 
строк и~столбцов никак не учтена в~способе построения состояний при использовании 
функции~(\ref{eq37}): состояния инвариантны относительно транспонирования. 
Поэтому априори представляется обоснованным рассмотреть обобщение~(\ref{eq37}) \mbox{вида}

\vspace*{-3pt}

\noindent
\begin{multline}
\label{eq38}
\sigma_2(U,L;\alpha) = (1+\alpha)|U| + (1-\alpha)|L|\,,\\
-1\leq\alpha\leq +1\,,
\end{multline}

\vspace*{-1pt}

\noindent
и использовать при построении состояний для детальных составляющих конкретного 
типа функцию~$\sigma_2$  вида~(\ref{eq38}) с~конкретным (отличным от нуля) 
значением~$\alpha$.

Для проверки гипотезы был проведен эксперимент, в~ходе которого 
значения оптимальной квазиэнтропии детальных составляющих, полученные с~использованием
 функций вида~(\ref{eq38}) и~значениями~$\alpha$, равными~$\pm1/3$, 
 сравнивались с~полученными с~использованием функции~(\ref{eq37}) аналогичными 
 значениями. Вычисления проводились для детальных составляющих всех~6~томограмм 
 при общем числе выбираемых порогов  $T\hm=1,2$ и~3, всего~108~вариантов. 
 Полученные результаты опровергают выска-\linebreak занное предположение. 
 В~подавляющем большинстве случаев (92~из~108) 
 наименьшее значение\linebreak\vspace*{-12pt}
 
 \columnbreak
 
 \noindent
 квазиэнтропии достигалось при использовании функции~(\ref{eq37}). 
 В~остальных~16~случаях выигрыш, обусловленный использованием функции вида~(\ref{eq38}) 
 со значением $\alpha\hm\neq 0$, оказался ничтожным и~не превысил~0,008~б/п 
 ни в~одном из случаев. При этом уменьшение квазиэнтропии для томограммы в~целом 
 как результат использования для его детальных составляющих оптимальных функций 
 вида~(\ref{eq38}) оказалось еще меньше: оно не превысило~0,002~б/п. 
 Поэтому использование функций вида~(\ref{eq38}) в~процессе построения 
 со\-сто\-яний для детальных со\-став\-ля\-ющих было признано нецелесообразным.

Рассмотрим теперь уже упомянутое выше изменение процедуры построения кодовых
 вероятностей, используемое как при сжатии приближений, так и~при сжатии 
 детальных составляющих, но затрагивающее лишь фоновое состояние. 
 Необходимость данного изменения обусловлена тем, что отсчеты любой из компонент 
 ДВП могут иметь отрицательные значения. Это не сказывается на построении кодовых 
 вероятностей для нефоновых состояний, поскольку в~этом случае предусмотрено 
 использование процедуры симметризации. Для фонового состояния ситуация отличается. 
 С~одной стороны, для любой компоненты частотное распределение значений в~фоновом 
 со\-сто\-янии по-преж\-не\-му несимметрично и~использование сим\-мет\-ри\-за\-ции привело бы 
 к~неоправданным издержкам. С~другой стороны, вероятности отрицательных значений 
 в~фоновом состоянии, вообще говоря, не равны нулю и~для таких значений должны быть 
 построены ненулевые кодовые вероятности. Поэтому при построении аппроксимации для 
 фонового состояния нельзя ограничиться диапазоном ${\cal I}_0^+\hm=
 [1,\,a_{\max}^+({\cal S}_0)]$, а~необходимо добавить диапазон ${\cal I}_0^-\hm=
 [a_{\min}^-({\cal S}_0),\,-1]$, где $a_{\min}^-({\cal S}_0)$~--- 
 минимальное\linebreak значение, которое принимают отсчеты рассмат\-риваемой компоненты 
 в~фоновом состоянии,\linebreak и~построить соответствующие условные кодовые вероятности. Вклад 
 избыточности кодирования нового диапазона $R(\mathbf{X}|{\cal I}_0^-,{\cal S}_0)$ 
 в~общую избыточность кодирования составляет 
 $f({\cal I}_0^-|{\cal S}_0)R(\mathbf{X}|{\cal I}_0^-,{\cal S}_0)$, где $\mathbf{X}$~--- 
 любая из компонент ДВП.

Задача аппроксимации для отрицательного диапазона~${\cal I}^-$ сводится к~уже 
рассмотренной ранее задаче для положительного диапазона ${\cal I}^+\hm=
-{\cal I}^-$. Пусть $f(x)$, $x\hm\in{\cal I}^-$,~--- 
заданное в~отрицательном диапазоне условное частотное распределение. 
Функция~$\varphi(x)\hm= f(-x)$, $x\hm\in{\cal I}^+$,~--- 
условное распределение в~положительном диапазоне, поэтому рассмотренная 
в~подразд.~4.3 процедура позволяет построить для него 
аппроксимацию $q_\varphi(x)$, $x\hm\in{\cal I}^+$. В~качестве искомого условного 
кодового распределения будем использовать функцию $q(x)\hm= 
q_\varphi(-x)$, $x\hm\in{\cal I}^-$.

% Table 7
\begin{table*}[b]\small
\vspace*{-6pt}
\begin{center}
\Caption{Оптимальные/квазиоптимальные пороги и~квазиэнтропия компонент ДВП}
\label{tab7}
\vspace{2ex}

\begin{tabular}{|c|cc|cc|cc|}
\hline
&&&&&&\\[-9pt]
 X & $\hat{\frak T}^2;$ & $\hat{H}^2$ $(T=2)$ & $\hat{\frak T}^3;$ &
  $\hat{H}^3$ $(T=3)$ & 
$\tilde{\frak T}^3;$ & $\tilde{H}^3$ $(T=3)$ \\
%Data
\hline
A1 & \{23,182\}; &  4,880032 & ~\{20,73,271\}; &  4,847027 & ~\{14,29,182\}; &  4,853077 \\
%\hline
V1 & \{12,34\}; &  4,010317 & ~~~\{9,21,42\}; &  3,997519 & ~~\{12,28,87\}; &  3,998197\\
%\hline
H1 & \{20,81\}; &  4,523917 & \{11,30,115\}; &  4,489731 & \{12,41,116\}; &  4,490816\\
%\hline
D1 & \{13,27\}; &  4,018969 & \{13,26,52\}; &  4,008997 & \{11,19,49\}; &  4,010081\\
%\hline
A2 & \{28,175\}; &  5,124079 &\{23,85,312\}; &  5,087575 &\{13,35,175\}; &  5,096258\\
%\hline
V2 & \{17,42\}; &  4,242733 & \{13,30,69\}; &  4,227881 & \{13,26,69\}; &  4,228058\\
%\hline
H2 & \{17,63\}; &  4,725712 & \{17,51,150\}; &  4,696555 &\{17,51,150\}; &  4,696555\\
%\hline
D2 & \{15,36\}; &  4,292950 & \{11,21,43\}; &  4,281943 & \{11,21,43\}; &  4,281943\\
%\hline
A3 & \{56,241\}; &  6,491513 & \{51,193,584\}; &  6,449757 & \{42,115,428\}; &  6,454966\\
%\hline
V3 & \{68,191\}; &  6,064261 & \{58,148,335\}; &  6,045233 & \{51,104,205\}; &  6,046978\\
%\hline
H3 & \{78,229\}; &  6,223272 & \{61,155,318\}; &  6,199585 & \{61,155,318\}; &  6,199585\\
%\hline
D3 & \{64,145\}; &  6,119812 & \{61,117,233\}; &  6,105214 & ~\{61,99,217\}; &  6,106998\\
%\hline
A4 & \{62,276\}; &  6,448976 & \{53,166,577\}; &  6,406838 & \{48,150,562\}; &  6,408289\\
%\hline
V4 & \{68,203\}; &  5,985891 & \{48,107,283\}; &  5,967160 & \{48,107,283\}; &  5,967160\\
%\hline
H4 & \{78,230\}; &  6,130190 & \{57,140,347\}; &  6,103864 & \{57,140,347\}; &  6,103864\\
%\hline
D4 & \{69,147\}; &  6,024354 & \{63,133,321\}; &  6,011689 & \{42,83,175\}; &  6,012719\\
%\hline
A5 & \{12,92\}; &  4,942235 & \{1,16,100\}; &  4,856658 & \{1,22,116\}; &  4,863834\\
%\hline
V5 & \{13,39\}; &  3,891974 & \{10,24,87\}; &  3,866175 & \{10,21,71\}; &  3,867450\\
%\hline
H5 & \{11,32\}; &  3,636817 & \{6,17,60\}; &  3,614506 & \{6,17,60\}; &  3,614506\\
%\hline
D5 & \{8,17\}; &  3,092293 & \{8,15,35\}; &  3,082019 & \{6,12,35\}; &  3,082257\\
%\hline
A6 & \{16,87\}; &  4,661787 &\{15,47,163\}; &  4,627884 & \{12,22,89\}; &  4,640066\\
%\hline
V6 & \{11,39\}; &  3,686671 &\{9,20,79\}; &  3,664356 & \{8,15,63\}; &  3,666472\\
%\hline
H6 & \{10,26\}; &  3,474091 &\{7,13,41\}; &  3,458590 & \{7,12,41\}; &  3,458997\\
%\hline
D6 & \{5,10\}; &  2,865786 & \{5,8,20\}; &  2,855021 & \{5,8,20\}; &  2,855021\\
\hline
\end{tabular}
\end{center}
\end{table*}

\subsection{Оценки минимальной скорости и~избыточности кодирования}


Оценки минимальной скорости кодирования (квазиэнтропии) компонент 
ДВП томограмм T1--T6, отвечающие множествам состояний, построенных 
с~использованием двух и~трех оптимальных и~трех квазиоптимальных порогов, 
приведены в~табл.~\ref{tab7}. Там же приведены и~соответствующие значения 
порогов. Первый столбец таблицы определяет рассматриваемое изображение 
(указаны тип компоненты ДВП и~номер томограммы). Во втором и~третьем столбцах 
приведены значения двух и~трех оптимальных порогов и~соответствующие значения 
квазиэнтропии. Последний столбец содержит те же данные для квазиоптимальных 
порогов. В~соответствии со сказанным в~подразд.~5.2, в~случае 
приближений (компоненты A1--A6) множества состояний строились для ошибок 
предсказания с~использованием функции~(\ref{eq22}); оценки квазиэнтропии 
получены также для ошибок предсказания. В~случае детальных составляющих 
(остальные компоненты) множества состояний строились для значений компонент 
с~использованием функции~(\ref{eq37}); оценки квазиэнтропии получены также 
для значений компонент.

Анализ приведенных данных кроме всего прочего еще раз демонстрирует 
эффективность использования трех квазиоптимальных порогов.
%
Поскольку минимальная скорость кодирования (квазиэнтропия) томограммы 
в~целом равна среднему арифметическому значений квазиэнтропии всех компонент ДВП, 
она может легко быть вычислена на основе данных табл.~\ref{tab7}. Соответст\-ву\-ющие 
результаты для томограмм~T1--T6 приведены в~табл.~8.

Сравнение приведенных результатов с~аналогичными результатами 
из подразд.~4.2 показывает, что квазиэнтропия кодирования компонент 
ДВП томограммы всегда оказывается меньше соответствующей квазиэнтропии 
кодирования ошибок предсказания той же томограммы. Достигаемый выигрыш 
варьируется от нескольких сотых долей для томограмм легких до нескольких 
десятых долей для томограмм мозга.

Построение кодовых распределений и~оценка избыточности арифметического 
кодирования для любой из компонент ДВП осуществляется так, как описано 
в~подразд.~4.3. Единственное отличие~--- использование дополнительного 
отрицательного диапазона для фонового состояния~--- подробно рассмотрено 
в~подразд.~5.2.



% Table 9
\setcounter{table}{8}

\begin{table*}[b]\small
\vspace*{-9pt}
\begin{center}
\Caption{Избыточность кодирования компонент ДВП (процедура уравнивания)}
\label{tab9}
\vspace{2ex}

\tabcolsep=11pt
\begin{tabular}{|c|c|c|c|c|c|}
\hline
&&&&&\\[-9pt]
X & $R_G$ &  $R$ & $R_{\mathrm{T}}$ & $R+R_{\mathrm{T}}$ & 
$(R+R_{\mathrm{T}})/\tilde{H}^3$ \\
\hline
A1 & 0,021421 & 0,052898 & 0,013824 & 0,066723 & 0,013749\\
%\hline
V1 & 0,007909 & 0,019402 & 0,010681 & 0,030083 & 0,007524\\
%\hline
H1 & 0,013685 & 0,027911 & 0,011398 & 0,039309 & 0,008753\\
%\hline
D1 & 0,005286 & 0,014373 & 0,009247 & 0,023619 & 0,005890\\
%\hline
A2 & 0,023268 & 0,054725 & 0,013824 & 0,068549 & 0,013451\\
%\hline
V2 & 0,007788 & 0,019579 & 0,009964 & 0,029543 & 0,006987\\
%\hline
H2 & 0,012490 & 0,026999 & 0,010681 & 0,037680 & 0,008023\\
%\hline
D2 & 0,005202 & 0,014967 & 0,009964 & 0,024931 & 0,005822\\
%\hline
A3 & 0,036772 & 0,084136 & 0,013824 & 0,097960 & 0,015176\\
%\hline
V3 & 0,016965 & 0,042210 & 0,012833 & 0,055043 & 0,009102\\
%\hline
H3 & 0,022136 & 0,048502 & 0,012833 & 0,061335 & 0,009893\\
%\hline
D3 & 0,014539 & 0,033993 & 0,011398 & 0,045392 & 0,007433\\
%\hline
A4 & 0,041553 & 0,093684 & 0,013824 & 0,107509 & 0,016776\\
%\hline
V4 & 0,018914 & 0,045172 & 0,012833 & 0,058004 & 0,009721\\
%\hline
H4 & 0,022429 & 0,051096 & 0,012833 & 0,063929 & 0,010473\\
%\hline
D4 & 0,014004 & 0,033910 & 0,012115 & 0,046025 & 0,007655\\
%\hline
A5 & 0,022703 & 0,052829 & 0,012360 & 0,065189 & 0,013403\\
%\hline
V5 & 0,007669 & 0,018427 & 0,011398 & 0,029825 & 0,007712\\
%\hline
H5 & 0,012084 & 0,022112 & 0,011398 & 0,033511 & 0,009271\\
%\hline
D5 & 0,006518 & 0,014612 & 0,010681 & 0,025293 & 0,008206\\
%\hline
A6 & 0,023082 & 0,053538 & 0,013824 & 0,067363 & 0,014518\\
%\hline
V6 & 0,012474 & 0,022548 & 0,011398 & 0,033947 & 0,009259\\
%\hline
H6 & 0,016700 & 0,025930 & 0,011398 & 0,037328 & 0,010792\\
%\hline
D6 & 0,010453 & 0,017056 & 0,009247 & 0,026303 & 0,009213\\
\hline
\end{tabular}
\end{center}
%\end{table*}
% Table 10
%\begin{table*}\small
\begin{center}
\Caption{Избыточность кодирования в~целом (процедура уравнивания)}
\label{tab10}
\vspace{2ex}

\tabcolsep=10pt
\begin{tabular}{|c|c|c|c|c|c|}
\hline
&&&&&\\[-9pt]
T & $R_G$ &  $R$ & $R_{\rm T}$ & $R+R_{\rm T}$ & $(R+R_{\rm T})/\tilde{H}^3$\\
\hline
T1 & 0,012075 & 0,028646 & 0,011288 & 0,039934 & 0,009205\\
%\hline
T2 & 0,012187 & 0,029068 & 0,011108 & 0,040176 & 0,008780\\
%\hline
T3 & 0,022603 & 0,052210 & 0,012722 & 0,064933 & 0,010469\\
%\hline
T4 & 0,024225 & 0,055966 & 0,012901 & 0,068867 & 0,011247\\
%\hline
T5 & 0,012244 & 0,026995 & 0,011459 & 0,038455 & 0,009970\\
%\hline
T6 & 0,015677 & 0,029768 & 0,011467 & 0,041235 & 0,011281\\
\hline
\end{tabular}
\end{center}
\end{table*}

В табл.~\ref{tab9} представлены оценки избыточности кодирования 
компонент ДВП томограмм~T1--T6, полученные с~использованием процедуры 
урав-\linebreak\vspace*{-12pt}

\pagebreak

{\small   %tabl8
 \noindent
{{\tablename~8}\ \ \small{Квазиэнтропия в~целом (оптимальные/квазиоптимальные пороги)}}
%\vspace*{2ex}

\begin{center}
\tabcolsep=10pt
\begin{tabular}{|c|c|c|c|}
\hline
&&&\\[-9pt]
T  & $\hat{H}^2$ $(T=2)$ & $\hat{H}^3$ $(T=3)$ &  $\tilde{H}^3$ $(T=3)$\\
\hline
T1 & 4,358309 & 4,335819 & 4,338043 \\
%\hline
T2 & 4,596369 & 4,573489 & 4,575704 \\
%\hline
T3 & 6,224715 & 6,199947 & 6,202132 \\
%\hline
T4 & 6,147353 & 6,122388 & 6,123008 \\
%\hline
T5 & 3,890830 & 3,854840 & 3,857012 \\
%\hline
T6 & 3,672084 & 3,651463 & 3,655139 \\
\hline
\end{tabular}
\end{center}
\vspace*{16pt}
}



\noindent
нивания при построении кодовых распределений. Параметры процедуры 
выбирались так же, как и~ранее ($\rho\hm= 0{,}004$ и~$I_{\max}^+\hm=4$). 
Как и~для ошибок предсказания, использовано квазиоптимальное множество 
состояний (свое для каждой компоненты ДВП). Структура таблицы аналогична 
структу\-ре табл.~\ref{tab4} и~\ref{tab6}. В~первом столбце таблицы указаны 
тип компоненты ДВП и~номер томограммы. В~остальных столбцах~--- 
соответствующие значения из\-бы\-точ\-ности симметризации, индивидуальной из\-бы\-точ\-ности 
арифметического кодирования (с~учетом симметризации), избыточности передачи, общей 
избыточности кодирования метода и~относительной общей избыточности кодирования. 
Аккуратный подсчет числа битов, необходимых для передачи параметров декодеру, 
может быть произведен так же, как 
в~подразд.~4.3. Подсчет должен производиться отдельно для компоненты каждого типа.




Поскольку значение избыточности кодирования томограммы в~целом равно 
среднему арифметическому значений избыточности всех компонент ДВП, 
она может легко быть вычислена на основе данных табл.~\ref{tab9}. 
Соответствующие результаты для томограмм~T1--T6 приведены в~табл.~\ref{tab10}.

% Table 11
\begin{table*}[b]\small
\vspace*{-9pt}
\begin{center}
\Caption{Скорости кодирования томограмм~T1--T6}
\label{tab11}
\vspace{2ex}

\begin{tabular}{|c|c|c|c|c|}
\hline
&Ошибки&&\multicolumn{2}{c|}{JP2}\\
\cline{4-5}
\multicolumn{1}{|c|}
{\raisebox{6pt}[0pt][0pt]{ T}} &  предсказания & 
\multicolumn{1}{c|}
{\raisebox{6pt}[0pt][0pt]{\tabcolsep=0pt\begin{tabular}{c}Компоненты\\ ДВП\end{tabular}}} 
& Вариант 1 & Вариант~2\\
\hline
T1 & 4,505450 & 4,377977 & 4,696686 & 4,481018\\
%\hline
T2 & 4,789562 & 4,615880 & 4,925232 & 4,706512\\
%\hline
T3 & 6,319010 & 6,267065 & 6,672668 & 6,481934\\
%\hline
T4 & 6,229735 & 6,191875 & 6,603790 & 6,408722\\
%\hline
T5 & 4,237956 & 3,895467 & 4,252380 & 3,941132\\
%\hline
T6 & 3,978116 & 3,696374 & 3,959625 & 3,703583\\
\hline
\end{tabular}
\end{center}
\end{table*}


Сравнение приведенных данных с~аналогичными данными табл.~\ref{tab6}, 
относящимися к~кодированию ошибок предсказания, обнаруживает более чем 
двукратное увеличение общей избыточности кодирования метода. Это вызвано и~увеличением 
избыточности арифметического кодирования, и~увеличением избыточности передачи. 

Причинами увеличения избыточности арифметического кодирования являются увеличение 
избыточности симметризации и~снижение качества\linebreak аппроксимации. И~то, и~другое 
обусловлено четырехкратным %\linebreak 
уменьшением числа отсчетов компонент 
ДВП в~сравнении с~исходным числом отсчетов. В~ре\-зуль\-та\-те частотные распределения\linebreak 
стано\-вят\-ся менее <<представительными>>, что увеличивает асим\-мет\-рию и~снижает 
качество аппрокси\-мации. 
{ %\looseness=1

}

Увеличение избыточности передачи связано, главным 
образом, с~необходимостью вместо одного набора параметров передать четыре 
набора, по одному для каждой компоненты. Кроме того, снижение качества 
аппроксимации приводит к~увеличению числа диапазонов, которые строятся в~процессе 
применения процедуры уравни\-вания.
{\looseness=1

}

Несмотря на все сказанное, общая относительная избыточность кодирования метода 
находится на уровне~1\%, что с~практической точки зрения может быть признано 
более чем хорошим результатом.

\vspace*{-6pt}

\section{Заключение}

В работе построены и~исследованы два метода обратимого сжатия компьютерных 
томограмм. Первый метод предполагает кодирование ошибок предсказания томограммы, 
второй~--- кодирование компонентов ДВП томограммы. Оба метода получены как результат 
адаптации к~конкретному типу данных некоторого общего метода (подхода), основанного 
на применении универсального арифметического кодирования. Для каждого из методов 
получены эффективные индивидуальные оценки скорости кодирования.

Томограммы представляют собой полутоновые изображения, поэтому их обратимое 
сжатие может быть осуществлено методом, входящим в~стандарт JPEG~2000. Сравнение 
скоростей кодирования рассмотрен\-ных в~работе методов и~алгоритма обратимого 
сжатия JPEG~2000 позволяет вынести суждения как об эффективности конкретных 
предложенных методов, так и~о~потенциальных возможностях предложенного общего 
подхода в~целом.

В сводной табл.~\ref{tab11} представлены экспериментально полученные оценки 
скорости кодирования (в битах на пиксель) для томограмм~T1--T6. Во второй 
колонке приведены результаты, отвеча\-ющие кодированию значений ошибок предсказания\linebreak 
(см.\ разд.~4), в~третьей~--- кодированию значений компонент ДВП (см.\ разд.~5). 
В~обоих случаях приведенные скорости кодирования соответствуют алгоритмам, 
использующим квазиоптимальное множество состояний и~процедуру уравнивания 
при построении кодовых вероятностей, т.\,е.\ алгоритмам, допускающим <<быструю>> 
реализацию. В~чет\-вер\-той колонке приведена скорость кодирования в~случае применения 
алгоритма JPEG~2000 к~исходным данным томограммы (JP2, вариант~1), в~пятой --- 
скорость кодирования в~случае применения алгоритма JPEG~2000 к~данным, 
полученным после амплитудного преобразования~(\ref{eq21}) (JP2, вариант~2). 
В~последних двух случаях для кодирования использована эталонная реализация (Jasper) 
стандарта JPEG~2000 (библиотека доступна по адресу 
{\sf http://www.ece.uvic.ca/$\sim$mdadams/jasper}).


Приведенные в~табл.~\ref{tab11} результаты показывают, что 
наименьшая скорость кодирования (наибольшая степень сжатия) 
достигается при использовании метода кодирования компонент ДВП. Отметим, 
что применение к~данным преобразования~(\ref{eq21}) заметно уменьшает 
скорость кодирования данных алгоритмом JPEG~2000. 

Можно констатировать, 
что разработанные для сжатия томограмм методы кодирования весьма эффективны, 
хотя выигрыш, который достигается по сравнению с~алгоритмом JPEG~2000, 
по-ви\-ди\-мо\-му, недостаточен, чтобы рекомендовать их практическое внедрение. 
Более важным является то, что полученные результаты демонстрируют большие потенциальные 
возможности общего метода (подхода), основанного на применении универсального 
арифметического кодирования, который можно с~успехом адаптировать для построения 
методов сжатия таких данных, которые не являются изображениями и~где алгоритм JPEG~2000 
неприменим. Это может стать целью дальнейших исследований. В~част\-ности, большой 
интерес представляет задача построения метода обратимого сжатия карт 
силовых кривых атом\-но-си\-ло\-вой микроскопии.

%\vspace*{-9pt}


{\small\frenchspacing
 {%\baselineskip=10.8pt
 \addcontentsline{toc}{section}{References}
 \begin{thebibliography}{9}
\bibitem{b01}
\Au{Сушко Д.\,В., Штарьков~Ю.\,М.} О~сжатии томографических данных~// 
Информационные процессы, 2008. Т.~8. №\,4. С.~240--255.
\bibitem{b02}
\Au{Witten I.\,H., Neal R.\,M., Cleary~J.\,G.} Arithmetic coding for data compression~// 
Commun. ACM, 1987. Vol.~30. No.\,6. P.~520--540.
\bibitem{b03}
\Au{Сушко Д.\,В.} Выбор состояний источника при сжатии томограмм~// 
Информационные процессы, 2010. Т.~10. №\,3. С.~237--244.
\bibitem{b04}
\Au{Sushko D.\,V.}  Choice of source states for compression of tomograms~// 
J.~Commun. Technol. El., 2011. Vol.~56. No.\,6. P.~716--721.
\bibitem{b05}
\Au{Сушко Д.\,В., Штарьков~Ю.\,М.}  Вей\-в\-лет-пре\-об\-ра\-зо\-ва\-ния 
и~сжатие компьютерных томограмм~// Информационные процессы, 2009. Т.~9. №\,2. С.~105--115.
\bibitem{b06}
\Au{Добеши И.} Десять лекций по вейвлетам~/ Пер. с~англ.~--- 
Ижевск: Регулярная и~хаотическая динамика, 2001. 464~с. 
(\Au{Daubechies~I.} Ten lectures on wavelets.~--- CBMS-NSF regional conference 
ser. in applied mathematics.~--- SIAM, 1992. Vol.~61. 377~p.)
\bibitem{b07}
\Au{Le Gall D., Tabatabai~A.}  Sub-band coding of digital images using symmetric short 
kernel filters and arithmetic coding techniques~// IEEE  Conference  (International) on
Acoustics,  Speech, and Signal Processing Proceedings, 1988. P.~761--764.
\bibitem{b08}
\Au{Sweldens~W.} The lifting scheme: A~custom-design construction 
of biorthogonal wavelets~//  Appl. Comput. Harmon. Anal., 1996.  
Vol.~3. No.\,2. P.~186--200.
 \end{thebibliography}

 }
 }

\end{multicols}

\vspace*{-9pt}

\hfill{\small\textit{Поступила в~редакцию 30.11.16}}

\vspace*{6pt}

%\newpage

%\vspace*{-24pt}

\hrule

\vspace*{2pt}

\hrule

%\vspace*{8pt}


\def\tit{REVERSIBLE DATA COMPRESSION BY~UNIVERSAL~ARITHMETIC~CODING}

\def\titkol{Reversible data compression by~universal arithmetic coding}

\def\aut{A.\,I.~Stefanovich and D.\,V.~Sushko}

\def\autkol{A.\,I.~Stefanovich and D.\,V.~Sushko}

\titel{\tit}{\aut}{\autkol}{\titkol}

\vspace*{-9pt}


 \noindent
Institute of Informatics Problems, 
Federal Research Center ``Computer Science and Control'' 
of the Russian Academy of Sciences, 44-2~Vavilov Str., Moscow 119333, 
Russian Federation 



\def\leftfootline{\small{\textbf{\thepage}
\hfill INFORMATIKA I EE PRIMENENIYA~--- INFORMATICS AND
APPLICATIONS\ \ \ 2017\ \ \ volume~11\ \ \ issue\ 1}
}%
 \def\rightfootline{\small{INFORMATIKA I EE PRIMENENIYA~---
INFORMATICS AND APPLICATIONS\ \ \ 2017\ \ \ volume~11\ \ \ issue\ 1
\hfill \textbf{\thepage}}}

\vspace*{3pt}



\Abste{The paper considers the general approach to the reversible (lossless) 
digital data compression problem, which is based on universal arithmetic coding 
of data with unknown statistics. A~model of a source with calculable sequence 
of states is used for data description. Within the approach, the tasks of obtaining 
specific compression methods and algorithms for particular data types are set up. 
The authors use computed tomography data (tomograms) as the object of the study 
and present two methods of lossless compression of tomograms. The first method 
encodes prediction errors of tomograms; the second method encodes components 
of discrete wavelet transform of tomograms. These methods are examined in 
details, effective compression algorithms are constructed, and individual 
estimates of bit rate are obtained for the algorithms. The bit rates of the 
constructed algorithms and the lossless compression algorithms of the JPEG~2000~standard 
are compared. The results demonstrate high quality of the 
constructed algorithms and indicate great potential of the approach in general.}

\KWE{reversible data compression; lossless compression; universal coding; 
arithmetic coding; computed tomography}

\DOI{10.14357/19922264170103}

%\vspace*{-9pt}

%\Ack



%\vspace*{3pt}

  \begin{multicols}{2}

\renewcommand{\bibname}{\protect\rmfamily References}
%\renewcommand{\bibname}{\large\protect\rm References}

{\small\frenchspacing
 {%\baselineskip=10.8pt
 \addcontentsline{toc}{section}{References}
 \begin{thebibliography}{9}
\bibitem{1-sh-1}
\Aue{Sushko, D.\,V., and Yu.\,M.~Shtar'kov}. 2008. 
O~szhatii tomograficheskikh dannykh [On tomography data compression]. 
\textit{Informatsionnye protsessy} [Information Processes] 8(4):240--255.
\bibitem{2-sh-1}
\Aue{Witten, I.\,H., R.\,M.~Neal, and J.\,G.~Cleary}. 1987. 
Arithmetic coding for data compression. \textit{Commun. ACM} 30(6):520--540.
\bibitem{3-sh-1}
\Aue{Sushko, D.\,V.} 2010. Vybor sostoyaniy istochnika pri szhatii tomogramm 
[Choice of source states for compression of tomograms]. 
\textit{Informatsionnye protsessy} [Information Processes] 10(3):237--244.
\bibitem{4-sh-1}
\Aue{Sushko, D.\,V.} 2011. Choice of source states for compression of tomograms. 
\textit{J.~Commun. Technol. El.} 56(6):716--721.
\bibitem{5-sh-1}
\Aue{Sushko, D.\,V., and Yu.\,M.~Shtar'kov}. 2009. Veyvlet-preobrazovaniya 
i~szhatie komp'yuternykh tomogramm [Wavelet transforms and computed tomogram 
compression]. \textit{Informatsionnye protsessy} [Information Processes] 9(2):105--115.
\bibitem{6-sh-1}
\Aue{Daubechies, I.} 1992.
\textit{Ten lectures on wavelets}. 
CBMS-NSF regional conference ser. in applied mathematics. SIAM. Vol.~61. 377~p.
\bibitem{7-sh-1}
\Aue{Le Gall,~D., and A.~Tabatabai}. 1988. 
Sub-band coding of digital images using symmetric short kernel filters 
and arithmetic coding techniques. 
\textit{IEEE Conference (International) on Acoustics,  Speech,
and Signal Processing Proceedings}. 761--764.
\bibitem{8-sh-1}
\Aue{Sweldens, W.} 1996. The lifting scheme: 
A~custom-design construction of biorthogonal wavelets. 
\textit{Appl. Comput. Harmon. Anal.} 3(2):186--200.
\end{thebibliography}

 }
 }

\end{multicols}

\vspace*{-3pt}

\hfill{\small\textit{Received November 30, 2016}}


\Contr

\noindent
\textbf{Stefanovich Alexei I.} (b.\ 1983)~---
scientist, Institute of Informatics Problems, Federal Research Center 
``Computer Science and Control'' of the Russian Academy of Sciences, 44-2~Vavilov
Str., Moscow 119333, Russian Federation; \mbox{astefanovich@ipiran.ru} 

\vspace*{3pt}

\noindent
\textbf{Sushko Dmitry V.} (b.\ 1962)~---
Candidate of Science (PhD) in physics and mathematics, senior scientist, Institute 
of Informatics Problems, Federal Research Center ``Computer Science
and Control'' of the Russian Academy of Sciences, 44-2~Vavilov Str., Moscow 119333, 
Russian Federation; \mbox{dsushko@ipiran.ru} 
\label{end\stat}


\renewcommand{\bibname}{\protect\rm Литература} 