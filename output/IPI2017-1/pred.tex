{ %\Large  
{ %\baselineskip=16.6pt

\vspace*{-48pt}
\begin{center}\LARGE
\textit{Уважаемый читатель!}
\end{center}

%\vspace*{2.5mm}

\vspace*{4mm}

\thispagestyle{empty}

{\small

 
В~2017~г.\ исполняется 10~лет со времени выхода в~свет первого 
номера журнала <<Информатика и~её применения>>~--- 
научного журнала Российской академии наук, издающегося под 
на\-уч\-но-ме\-то\-ди\-че\-ским руководством Отделения нанотехнологий 
и~информационных технологий Российской академии наук. Учредителем журнала 
является Федеральный исследовательский центр <<Информатика и~управ\-ле\-ние>> 
Российской академии наук (ФИЦ ИУ РАН) (до~2015~г.~--- 
Институт проб\-лем информатики РАН).

Необходимость издания такого журнала была вызвана активным развитием 
информатики и~информационных технологий, большой важностью этого научного 
направления для развития страны, проникновением информационных технологий 
во все сферы жизни современного общества.

Тематику журнала определяет тот факт, что информатика~--- это комплексная 
фундаментальная научная дисциплина, опирающаяся на достижения 
ряда других наук, в~том числе математики, физики, лингвистики и~др. 
Одновременно журнал уделяет большое внимание современным информационным технологиям, 
являющимся приложениями результатов информатики как фундаментальной науки.

За прошедшие 10~лет (2007--2016~гг.)\ издано~38~выпусков журнала. В~них 
размещено~452~публикации, в~том числе~430~научных статей и~22~информационных 
публикации (обзоры, рецензии и~др.). Среди авторов журнала представители ведущих 
научных организаций и~университетов страны, в~том числе Московского государственного 
университета им.\ М.\,В.~Ломоносова, ФИЦ ИУ РАН (в~том числе ИПИ РАН, ВЦ 
им.\ А.\,А.~Дородницына РАН, ИСА РАН), Института точной механики и~вычислительной 
техники им.\ С.\,А.~Лебедева РАН, Института космических исследований РАН, 
Института астрономии РАН, ряда институтов Сибирского отделения РАН, МФТИ, МИФИ, 
Высшей школы экономики, Санкт-Пе\-тер\-бург\-ско\-го государственного университета, 
Санкт-Пе\-тер\-бург\-ско\-го государственного политехнического университета 
Петра Великого, Санкт-Пе\-тер\-бург\-ско\-го государственного университета 
телекоммуникаций им.\ проф.\ М.\,А.~Бонч-Бруе\-ви\-ча, 
Российского университета дружбы народов, Балтийского федерального университета 
имени Иммануила Канта, Вологодского государственного университета и~др. 
Публиковались статьи зарубежных авторов, в~том числе ученых из Израиля, 
США, Финляндии, Франции, Швейцарии, Швеции и~других стран. 

В конце настоящего выпуска журнала помещен указатель статей, 
опуб\-ли\-ко\-ван\-ных в~томах~1--10 (2007--2016~гг.).

Журнал включен в~Российский индекс научного цитирования и~в~базу 
данных RSCI Web of Science, перечень ВАК, базу данных CrossRef 
и~информационную систему <<Общероссийский математический портал MathNet>>. 
С~2015~г.\ журнал индексируется в~библиографической и~реферативной базе 
данных SCOPUS.

Мы всегда будем помнить ушедших из жизни членов редакционного совета 
и~редакционной коллегии журнала: академика С.\,К.~Коровина, профессоров 
А.\,В.~Печинкина и~И.\,А.~Ушакова, которые внесли неоценимый вклад в~становление 
и~развитие журнала.

После объединения в~2015~г.\ трех учреждений Российской академии наук~--- 
Института проблем информатики, Вычислительного центра им.\ А.\,А.~Дородницына 
и~Института системного анализа~--- в~Федеральное государственное учреждение 
<<Федеральный исследовательский центр <<Информатика и~управ\-ле\-ние>> 
Российской академии наук>> (ФИЦ ИУ РАН) именно этот Центр стал базовой организацией 
для издания журнала, что существенно расширило как тематику журнала, 
так и~его возможности по привлечению новых авторов, в~том числе и~зарубежных.

В настоящее время тематику журнала в~первую очередь составляют:
\begin{itemize}
\item    теоретические основы информатики;\\[-14.5pt] 
\item    математические методы исследования сложных систем и~процессов;\\[-14.5pt]
\item    информационные системы и~сети;\\[-14.5pt]
\item    информационные технологии;\\[-14.5pt]
\item    архитектура и~программное обеспечение вычислительных комплексов и~сетей. 
\end{itemize}

Эти направления особенно важны в~связи с необходимостью решения задач 
формирования технологической базы инновационного развития, обеспечения 
на\-уч\-но-тех\-но\-ло\-ги\-че\-ско\-го прорыва в~области создания и~развития 
отечественных информационных и~коммуникационных технологий в~интересах 
достижения высокого качества и~стабильности систем управления и~предоставления 
услуг в~экономической и~социальной сферах. 

Мы, как и~ранее, приглашаем авторов представлять для публикации в~журнале 
статьи как с достижениями в~области теоретических проблем информатики, так 
и~с~изложением результатов ее практического приложения, а~также 
рецензии на наиболее интересные книжные новинки в~области информатики 
и~информационных технологий, объявления о~крупнейших международных 
и~всероссийских конференциях, различных научных мероприятиях 
по этой тематике и~другие информационные материалы.

Надеемся, что и~в~дальнейшем содержание статей, помещаемых в~журнале, 
будет вызывать интерес научной общественности. Редакционный совет, редколлегия 
и~редакция журнала, со своей стороны, сделают все для того, 
чтобы журнал и~впредь своевременно и~подробно информировал читателей 
о~новейших достижениях информатики и~ее актуальных практических приложениях.

                

      
\vfill
\noindent
Главный редактор журнала <<Информатика и~её применения>>,\\
академик  РАН\hfill
\textit{И.\,А.~Соколов}\\[-6pt]

%\noindent
%Редактор-составитель тематического выпуска, профессор кафедры математической статистики\\
%факультета вычислительной математики и~кибернетики МГУ им.~М.\,В.~Ломоносова,\\
%ведущий научный сотрудник ИПИ РАН, доктор физико-математических наук\hfill
% \textit{В.\,Ю.~Королев}


} }
}
      