\renewcommand{\figurename}{\protect\bf Figure}
\renewcommand{\tablename}{\protect\bf Table}

\def\stat{self-mul}

\def\tit{INFORMATICS AND~ITS~ROLE FOR~THE~STUDY OF~GENESIS 
AND~PROPERTIES OF~COMPLEX NATURAL SYSTEMS$^*$}

\def\titkol{Informatics and its role for the study of genesis 
and properties of complex natural systems}

\def\autkol{R.\,B.~Seyful-Mulyukov}

\def\aut{R.\,B.~Seyful-Mulyukov$^1$}

\titel{\tit}{\aut}{\autkol}{\titkol}


\index{Seyful-Mulyukov R.\,B.}
\index{Сейфуль-Мулюков Р.\,Б.}

{\renewcommand{\thefootnote}{\fnsymbol{footnote}}
\footnotetext[1] { The investigation was carried out according to the Program 
``Informatics 
methods in development of the petroleum origin theory and elaboration of new 
technologies for exploring petroleum and gas accumulations and providing energy 
security of the Russian Federation'' under the general theme ``Society 
informatization and information security.''}}

\renewcommand{\thefootnote}{\arabic{footnote}}
\footnotetext[1]{Institute of Informatics Problems, Federal Research Center 
``Computer Science and Control'' of the Russian Academy of 
Sciences, 44-2~Vavilov Str.,  Moscow 119333, Russian Federation, \mbox{rust@ipiran.ru}}




\def\leftfootline{\small{\textbf{\thepage}
\hfill INFORMATIKA I EE PRIMENENIYA~--- INFORMATICS AND APPLICATIONS\ \ \ 2017\ \ \ volume~11\ \ \ issue\ 1}
}%
 \def\rightfootline{\small{INFORMATIKA I EE PRIMENENIYA~--- INFORMATICS AND APPLICATIONS\ \ \ 2017\ \ \ volume~11\ \ \ issue\ 1
\hfill \textbf{\thepage}}}



\Abste{The paper considers the history of cognition of information as a~phenomenon and 
informatics as its quantitative and qualitative development. The logical connection between such 
notions as information, informatics, complexity, and complex natural self-organizing systems is 
investigated. It is considered that information, besides its usual traditional meaning, is one of the 
main properties of matter. Informatics is considered as an instrument for cognition of 
development and structure of complex natural systems. Petroleum is chosen as an example of 
such system. It is proved that petroleum, as well as each its hydrocarbon molecule, possesses 
corpuscular properties, and petroleum as a~whole has information volume calculated in bits. 
A~new approach is proposed for petroleum accumulations exploration. It is based on the fact 
that petroleum generation is a~discrete process. Consequently, the process of discovering 
petroleum accumulations has two stages. The first stage is characterized by static 
uncertainty
and the second stage is characterized by dynamic uncertainty. Both types of uncertainty need to 
be removed. The paper presents technologies and methods of solving these problems.}

\KWE{informatics; informatization; complex natural system; petroleum origin; petroleum 
exploration; static uncertainty; dynamic uncertainty}

\DOI{10.14357/19922264170111} 

%\vspace*{9pt}


\vskip 10pt plus 9pt minus 6pt

      \thispagestyle{myheadings}

      \begin{multicols}{2}

                  \label{st\stat}

\noindent
   Historically, ``information'' was understood as data about an event, a~state, or 
other characteristics of a~phenomenon that living species transmitted to each other. 
In particular, information is a~light, heat, or sound signal from a~natural 
phenomenon, which plants receive or to which they react. Information is a~signal 
of different types that terrestrial animals, pests, birds, as well as marine creatures 
receive or exchange with each other.
{\looseness=1

}
   
   In economics, main data are expressed with digits. In industry and agriculture, 
``information'' is the name of specifications of goods and services or unit 
measurements, such as weight, mass, volume, size, distance, and others. In 
education, ``information'' is all totality of the basic knowledge about nature, 
history, and laws. This knowledge has vital importance for development of 
mankind. In each sphere of activities, ``information" is understood in a~specific 
way.
   
   The notion of ``informatization'' appeared in the information theory and 
different spheres of information applications. Its appearance was caused by a~rapid 
increase of flow of information, which was used in everyday life, industry, science, 
culture, education, social, and other spheres.
   
   Information has to be transmitted, received, processed, interpreted, stored, and 
undergo many other manipulations. It is necessary for the right positioning of an 
individual or a~community in a~society and has vital importance for economic 
independency and national security. All of these activities are informatization.
   
   Informatization is not a~one-time campaign. Everyday activities introduced the 
public consciousness to the necessity of informatization of all spheres of social 
activities. K.~Kolin proved that informatization has to be perceived in the public 
consciousness as a~powerful instrument for qualitative modification of education, 
science development, new technologies application, improving management, and 
other activities. All of these activities have vital importance for development and 
national security~[1].
   
   This paper presents application of informatization to solving one of 
fundamental problems of natural sciences~--- the problem of genesis of petroleum 
and natural hydrocarbon gas. For this reason, the paper contains a~short introduction 
to the history of cognition of the notion of ``information.'' It does not mean only 
social phenomena or informatics as an instrument, which provides information 
storage, usage, transmission, and processing. ``Information'' also means the 
properties of matter, which are associated with the notion of a~``complex natural 
system.'' These systems can be cognized using the laws of informatics.
   
   Understanding of the notion of ``information'' depends on means of its 
transmission, usage, application, and many other factors and is always subjective. 
There is no exact definition of ``information,'' which would be universally 
recognized, and such definition is not possible in principle. The development of 
civilization and growth of our knowledge about matter, movement, time, and space 
resulted in a~new deeper and more comprehensive understanding of ``information.'' 
The most important achievement was the identification of ``information'' and 
``uncertainty'' that were measured by the C.~Shannon's mathematical theory of 
information~[2]. It was a~qualitative definition of ``information.'' For the first time 
in the history of information science, A.~Ursul presented an integrated 
philosophical definition of ``information'' that shows the relation between its 
quantitative and qualitative content~[3].
   
   The revolution of information cognition was the result of detection of its new 
meaning. Physicists M.~Planck and L.~de~Broglie~[4] investigated matter on 
atomic and subatomic levels and proved that besides its usual meaning as 
something existing in our consciousness, information is also one of the main 
properties of matter that exists beyond men's consciousness or wish. A.~Zeilinger 
proved that each elementary particle of an atom contains one bit of 
information~[5]. 
   
   Further investigation of qualitative and quantitative characteristics of 
information caused the appearance of new disciplines. First among them was 
cybernetics that N.~Wiener defined as a~``scientific study of control and 
communication in animals and machines''~[6]. The practical application of this 
idea was the computer. Later, the term ``informatics'' was introduced by 
K.~Steinbuch~[7]. Since~1966, informatics was positioned as a~science about 
collection, storage, distribution, retrieval, and use of scientific and technical 
information~[8].
    
    Russian and American scientists continued to investigate informatics and 
information science theory and applied problems, which are based on 
achievements of mathematics, physics, cybernetics, and philosophy. The approach 
of American information specialists and their understanding of ``information'' and 
``information science'' is best described in two monographs. The first one considers 
information science as a~metascience~[9]. Its integral parts are mathematics, 
linguistics, psychology, library science, engineering science, and computer 
science~[10]. Physics and cybernetics were predecessors of information science; 
therefore, they are not present in this list. However, these disciplines proved 
physical nature of information as one of its main properties.
    
    Another monograph is an official publication of the American Society for 
Information Science and Technology (ASIS\&T). The monograph follows the 
ideas of Otten and Debone. Information science is considered mainly as 
investigation of mental perception and interpretation of information as 
a~phenomenon existing in our consciousness. A~considerable part of the 
monograph is devoted to information in economics. In the case of market 
economy, ``information'' (in other words, ``uncertainty'' or ``information entropy'') 
is related to the notion of ``value.'' It means that the Shannon's theory, which deals 
with quantity of information, represents the quantitative side of information with 
its value. In this context, information has value since it can be bought or sold in 
order to decrease uncertainty and create a~product with a~larger value.
   
   The development of the information theory relates to I.~Gurevich~[11]. Basing 
on the postulate that information is one of the main fundamental properties of 
matter, he calculated information content of each chemical element of the 
Mendeleev table in bits. Besides, he proved that laws of fundamental sciences 
including informatics make it possible to understand development of complex 
natural and social systems.
{\looseness=1

}
   
   An indicative example, which demonstrates effectiveness of applying 
informatics laws to the study of complex natural systems, is the problem of origin 
of petroleum and natural hydrocarbon gas. According to the currently dominating 
model, petroleum origin is mainly the problem of petroleum geology and 
geochemistry. This model has become insufficient, which results in a~considerable 
decrease of effectiveness of exploration works. Surprisingly, application of 
informatics laws solves this problem. In informatics, the nature and properties of 
petroleum are represented as direct or indirect consequences of the fact that 
petroleum is a~complex self-developing natural system~[12, 13]. Its nature, main 
properties, and genesis become understandable, if one considers them in the 
context of the phenomenon of complexity and informatics laws.
   
The new understanding of deep inorganic nature of petroleum as a~phenomenon 
that is not related to the biosphere in any way leads to the necessity of changing the 
ideology of exploration of its commercial accumulations. The current methodology 
is based on the assumption that remainders of the biosphere's plants and animals 
served as raw material for petroleum and gas generation. However, nobody has 
ever proved the existence of mother rocks generating petroleum and their ability to 
transform organic matter into hydrocarbon molecules for sure. Nobody has ever 
proved that petroleum is able to migrate through geological rocks, so media keeps 
its initial hydrocarbon composition. The deep inorganic nature of petroleum proves 
its place in the hierarchy of organization of matter on the Earth. There are three 
main forms of organization of matter. The higher form is represented by billions of 
\textit{cells} of living species. The intermediary form is represented by thousands 
of petroleum hydrocarbon \textit{molecules}. The lower form is represented by 
several \textit{atoms} of chemical elements composing crystals of rocks. 

The most simple petroleum hydrocarbon molecules transform into the most 
complex ones, which is a~self-organizing discrete process. As the result, a~complex 
natural system appears, which possesses several properties. This paper considers 
the properties, which are the most significant for petroleum generation, namely, 
\textit{existence}, \textit{development}, and \textit{cognoscibility}~[14]. These 
properties are typical for any complex natural system.

System uniqueness is one of existence features. Petroleum is a~unique phenomenon 
that is created by a~unique set of geological, geochemical, fluid-dynamic, and other 
natural conditions. They are always different for each period of the Earth's history. 
Petroleum composition and structure correspond completely to the current period 
of the Earth's development. 
     
     System constant movement is one of \textit{development} features. 
Petroleum is permanently migrating and therefore changing. Petroleum, which is 
conserved in one place by geological media and keeps its original hydrocarbon 
composition for hundreds of millions of years, does not exist anymore. 
Consequently, inside the Earth, there is no Devonian (360~M years old), 
Carboniferous (300~M years old), Permian (251~M years old), and Jurassic 
(145~M years old) petroleum. Its generation does not take millions of years, but 
hundreds or thousands of years. Petroleum generation has the following stages: 
hydrocarbon radicals and methane molecules appear in the Earth's upper mantle, 
migrate through the Earth's crust, interact with rocks of geological media, 
transform into hydrocarbon molecules of different kinds, and finally, accumulate 
within a~reservoir as petroleum.
{\looseness=1

}
     
     \textit{Cognoscibility} of a~complex system pertains to the gnoseological 
aspect of petroleum nature. There is a~principal question: Is our perception of 
petroleum adequate to its real nature and age or nonadequate? There are two 
possible answers to this question. One answer corresponds to the organic paradigm 
of petroleum genesis, which considers that composition, properties, and other 
features of petroleum just pumped from a~reservoir are related to the phenomena, 
which were generated hundreds of millions years ago. It means that animals and 
plants of the previous epoch, which served as a~source for kerogen generation, had 
the same composition as they have now. Another opposite idea is that petroleum 
composition and genesis correspond completely to the modern epoch of the Earth's 
development. 
     
     It is absolutely clear that we cognize composition and features of a~complex 
system as a~natural phenomenon, which is generated by all totality of modern 
geological, geochemical, thermodynamic, and other conditions. Otherwise, one has 
to accept that~300, 200, and~100~M years ago geology, geochemistry, 
thermodynamics, and many other natural conditions of the Earth's interior were the 
same as they are now. It is impossible in principle. So, according to the 
\textit{cognoscibility} feature, petroleum is a~modern complex abiogenic natural 
system, which consists of several thousands of hydrocarbon molecules, which are 
not related to the biosphere.
{\looseness=1

}
     
     The petroleum example allows demonstrating one more important feature. 
The necessity of informatization of scientific research follows from the fact that 
information is one of the main properties of matter. Petroleum characteristics have 
never been considered before in this context. Petroleum possesses physical and 
chemical properties, as well as information content, which corresponds to the 
quantity of carbon and hydrogen atoms, which compose hydrocarbon 
molecules~\cite{13-seif}. In this case, one has to consider that any atom has 
information volume that is calculated in bits. Petroleum consists of hydrocarbon 
molecules on~95\%. They are divided into three main groups: paraffin  
(30\%--35\%), naphthenic (25\%--75\%), and aromatic (10\%--15\%). There are 
different kinds of petroleum with different quantities of hydrocarbon molecules of 
different kinds: light oil (C$_{32}$H$_{66}$SN), low-gravity oil 
(C$_{32}$H$_{66}$OSN), and bitumen (C$_{45}$H$_{51}$O$_2$SN). 
Correspondingly, their information volume is~16\,224, 16\,729, and~17\,789~bits.
     
     Application of the complex natural system ideology and informatics laws 
opens new possibilities for petroleum, such as exploration in different geological 
media. Therefore, the procedure of exploration of commercial petroleum 
accumulations requires reconsideration. Static and dynamic uncertainty are typical 
for a~complex natural self-organizing system. Both types of uncertainty have to be 
removed in order to find the place where petroleum is accumulated. This 
exploration strategy was proposed for the first time in~\cite{12-seif}.
     
     Static uncertainty applies to geological elements, whose location in the 
Earth's interior has not changed during all the time of petroleum generation and 
accumulation. These geological elements are trap, reservoir, impermeable 
covering, and canal for petroleum migration. Petroleum accumulation could not 
have happened without these geological elements. They must be present together, 
and their static uncertainty has to be removed in the Aristotelian logic terms~--- 
``yes'' or ``no'' only. The units used to express exact characteristics of a~given 
geological element are its size, density, permeability, fracturing, degree, as well as 
petrographic and chemical composition of reservoir, trap, and others. All of these 
characteristics can be successfully determined by seismic survey technologies. 
However, problems arise when all geological elements, provided by three-dimensional (3D) 
mathematical models of petroleum bearing basins, are present, but exploration 
wells turn out to be dry. Currently, the average statistical percent of successful 
exploration does not exceed~30\%. 
     
     Therefore, besides static uncertainty, there exists another type of uncertainty 
that does not deal with stable geological elements, but with results of a~dynamic 
discrete process. The result is a~petroleum accumulation, which is located 
sometimes under~10~km of sedimentary rocks. Fixing the stages of petroleum 
generation and the trajectory of its migration in time and space is impossible. It 
means that one cannot present a~discrete process as a~Cantorian set and the result 
of a~discrete process cannot be determined unambiguously by the Boolean classic 
logic, since mathematical calculation of this logic needs alternating values~``1'' 
or~``0'' only. 
     
     Hence, the problem of removing dynamic uncertainty and deciding whether 
a~petroleum accumulation exists or not does not have a~unique unambiguous 
solution. If a~problem does not have a~unique solution in principle, one can pass to 
its multiple-valued solution. In this case, intermediate values are introduced. The 
probability of accumulation presence in deep strata can be expressed with values 
0.1, 0.2, 0.25,\ $\ldots$\,,\ 0.7,\ $\ldots$\ until~0.95. It means that if static uncertainty was 
removed for sure, then the probability of removing its dynamic uncertainty could 
be calculated by applying the fuzzy set theory equations~\cite{15-seif}. Petroleum 
geologists and geophysicists possess all totality of information about the Earth's 
interior, including its 3D mathematical model, as well as experience and intuition. 
However, in many cases, this information is insufficient for planning well drilling 
in a~certain point. One can get additional data on that point calculated 
mathematically to be sure that the probability of discovering a~petroleum 
accumulation is equal to~25\%, 75\%, or even~95\%.
     
     The fuzzy set can consist of multiple direct and indirect geochemical indices 
of hydrocarbon molecules detected in different elements of environment. Light 
gasiform hydrocarbon molecules belong only to petroleum, which migrated from 
the pool to the surface and accumulated in the soil, snow, plants, air located close 
to the surface, and other elements of environment. Hydrocarbon molecules can be 
detected by modern geochemical methods.
     
     The new approach described in the 
paper made it possible to propose a~solution of the fundamental problems of 
petroleum and gas genesis and of exploration of their commercial accumulations in 
a~nontraditional, nontrivial way. 
    
\renewcommand{\bibname}{\protect\rmfamily References}


{\small\frenchspacing
{%\baselineskip=10.8pt
\begin{thebibliography}{99}

\bibitem{1-seif}
\Aue{Kolin, K.\,K.} 2010. \textit{Filosofskie problemy informatiki} 
[Philosophic problems of informatics]. Moscow: BINOM. 259~p.
\bibitem{2-seif}
\Aue{Shannon, C.\,E.} 1948. A~mathematical theory of communication. 
\textit{Bell Syst. Tech.~J.} 27:379--423, 623--656.
\bibitem{3-seif}
\Aue{Ursul, A.\,D.} 2010. \textit{Priroda informatsii: Filosofskiy ocherk} 
[The nature of information: A~philosophic essay]. 2nd ed. Chelyabinsk: 
CHGAKI. 231~p.
\bibitem{4-seif}
\Aue{De Broiglie, L.} 1927. Wave mechanics and the atomic structure of 
matter and radiation. \textit{J.~Phys. Radium} 8(5):225--241.
\bibitem{5-seif}
\Aue{Zeilinger, A.} 1999. A~foundation principal for quantum mechanics. 
\textit{Found. Phys.} 29(4):631--643.
\bibitem{6-seif}
\Aue{Wiener, N.} 1961. \textit{Cybernetics, or control and communication 
in the animal and the machine}. 2nd rev. ed. Cambridge: MIT Press. 
232~p.
\bibitem{7-seif}
\Aue{Steinbuch, K.} 1957. Informatik. \textit{Automatische 
Informationsverarbeitung, SEG-Nachrichten} (Technische Mitteilunger der 
Standard Elektrik Gruppe). No.~4. 171~p.
\bibitem{8-seif}
\Aue{Mikhaylov, A.\,I., A.\,I.~Chernyy, and R.\,S.~Gilyarevskiy}. 1968. 
\textit{Osnovy informatiki} [The foundations of informatics]. Moscow: 
Nauka. 425~p.
\bibitem{9-seif}
\Aue{Otten, K., and A.~Debons}. 1970. Towards a~metascience of 
information: Informatology. \textit{J.~Am. Soc. Inform. 
Sci.} 21:89--94.
\bibitem{10-seif}
\Aue{Norton, M.\,J.} 2010. \textit{Introductory concepts in information 
science}. 2nd ed. ASIST monograph ser. Information Today. 210~p.
\bibitem{11-seif}
\Aue{Gurevich, I.\,M.} 2007. \textit{Zakony informatiki~--- osnova 
stroeniya i~poznaniya slozhnykh system} [The laws of informatics as a~basis 
of cognizing complex systems]. Moscow: TORUS PRESS. 399~p.
\bibitem{12-seif}
\Aue{Heylighen, F.} 2008. Сomplexity and self-organization. 
\textit{Encyclopedia of library and information sciences}. Eds. M.\,J.~Bates 
and M.\,N.~Maack. Taylor \& Frances. 9~p. Available at: {\sf 
http://pespmc1.vub.ac.be/Papers/ELIS-complexity.pdf} (accessed 
February~8, 2017).
\bibitem{13-seif}
\Aue{Seyful-Mulyukov, R.\,B.} 2012. \textit{Neft' i~gaz. Glubinnaya priroda 
i~ee prikladnoe znachenie} [Petroleum and gas. Deep nature and its applied 
meaning].  Moscow: TORUS PRESS. 214~p. 
\bibitem{14-seif}
\Aue{Seyful-Mulyukov, R.\,B.} 2010. \textit{Neft'~--- uglevodorodnye 
posledovatel'nosti: Analiz modeley genezisa i~evolyutsii} [Petroleum~--- 
hydrocarbon sequences: An analysis of models of genesis and evolution]. 
Moscow: 11~format. 173~p.
\bibitem{15-seif}
\Aue{Zadeh, L.\,A.} 1975. The concept of linguistic variable and its 
application to approximate reasoning. \textit{Inform. Sci.} 8:199--249,  
310--357; 9:43--80.
\end{thebibliography} }
 }

\end{multicols}

\vspace*{-6pt}

\hfill{\small\textit{Received October 11, 2016}}

\vspace*{-24pt}

\Contrl


\vspace*{3pt}

\noindent
\textbf{Seyful-Mulyukov Rustem B.} (b.\ 1928)~--- Doctor of Science in 
geology, professor, Head of Laboratory, Institute of Informatics Problems, 
Federal Research Center ``Computer Science and Control'' of the Russian 
Academy of Sciences, 44-2~Vavilov Str,  Moscow 119333, Russian Federation; 
\mbox{rust@ipiran.ru}


%\vspace*{8pt}

%\hrule

%\vspace*{2pt}

%\hrule

\newpage

\vspace*{-24pt}



\def\tit{ИНФОРМАТИКА И~ЕЕ РОЛЬ В~ПОЗНАНИИ 
ОБРАЗОВАНИЯ И~СВОЙСТВ СЛОЖНОЙ ПРИРОДНОЙ 
СИСТЕМЫ}

\def\aut{Р.\,Б.~Сейфуль-Мулюков}


\def\titkol{Информатика и~ее роль в~познании 
образования и~свойств сложной природной 
системы}

\def\autkol{Р.\,Б.~Сейфуль-Мулюков}

%{\renewcommand{\thefootnote}{\fnsymbol{footnote}}
%\footnotetext[1]{Работа проводится при финансовой поддержке Программы
%стратегического развития Петрозаводского государственного университета в~рамках
%на\-уч\-но-ис\-сле\-до\-ва\-тель\-ской деятельности.}}


\titel{\tit}{\aut}{\autkol}{\titkol}

\vspace*{-12pt}

\noindent
Институт проблем информатики Федерального исследовательского центра <<Информатика 
и~управление>>
Российской академии наук

\vspace*{6pt}

\def\leftfootline{\small{\textbf{\thepage}
\hfill ИНФОРМАТИКА И ЕЁ ПРИМЕНЕНИЯ\ \ \ том\ 11\ \ \ выпуск\ 1\ \ \ 2017}
}%
 \def\rightfootline{\small{ИНФОРМАТИКА И ЕЁ ПРИМЕНЕНИЯ\ \ \ том\ 11\ \ \ выпуск\ 1\ \ \ 2017
\hfill \textbf{\thepage}}}



\Abst{Рассматривается история познания феномена <<информации>> 
и~информатики как междисциплинарной науки, изучающей качественные 
и~количественные особенности ее практических приложений. 
Представляется логическая связь таких широко распространенных понятий, 
как информация, информатика, сложность, сложные природные 
самоорганизующиеся системы. Принимается во внимание, что информация 
кроме традиционного, общепринятого значения является одним из свойств 
материи. Информатика наряду с~другими особенностями является 
инструментом познания развития и~строения сложных природных 
самоорганизующихся систем. В~качестве примера такой системы выбрана 
нефть. Доказывается, что нефть обладает корпускулярными свойствами 
и~каждая молекула углеводорода имеет объем информации в~битах. 
Предлагается новый подход к~поиску месторождений нефти, основанный на 
том факте, что ее образование~--- это дискретный процесс. Соответственно, 
обнаружение его результата является раскрытием статической 
и~динамической неопределенности. Рассматриваются методы и~технологии 
их раскрытия.}

\KW{информатика; информатизация; природные сложные системы; 
образование нефти; поиски нефти; статическая неопределенность; 
динамическая неопределенность} 
   
\DOI{10.14357/19922264160111} 

%\vspace*{18pt}


 \begin{multicols}{2}

\renewcommand{\bibname}{\protect\rmfamily Литература}
%\renewcommand{\bibname}{\large\protect\rm References}

{\small\frenchspacing
{%\baselineskip=10.8pt
\begin{thebibliography}{99}
\bibitem{1-seif-1}
\Au{Колин К.\,К.} Философские проблемы информатики.~--- М.: БИНОМ, 
2010. 259~с.
\bibitem{2-seif-1}
\Au{Shannon C.\,E.} A~mathematical theory of communication~// Bell Syst. 
Tech.~J., 1948. Vol.~27. P.~379--423, 623--656.
\bibitem{3-seif-1}
\Au{Урсул А.\,Д.} Природа информации: Философский очерк.~--- 2-е изд.~--- 
Челябинск: ЧГАКИ, 2010. 231~с.
\bibitem{4-seif-1}
\Au{De Broiglie L.} Wave mechanics and the atomic structure of matter and 
radiation~// J.~Phys. Radium, 1927. Vol.~8. No.\,5. P.~225--241.
\bibitem{5-seif-1}
\Au{Zeilinger A.} A~foundation principal for quantum mechanics~// Found. 
Phys., 1999. Vol.~29. No.\,4. P.~631--643.
\bibitem{6-seif-1}
\Au{Wiener N.} Cybernetics, or control and communication in the animal and the 
machine.~--- 2nd rev. ed.~--- Cambridge: MIT Press, 1961. 232~p.
\bibitem{7-seif-1}
\Au{Steinbuch K.} Informatik~// Automatische Informationsverarbeitung,  
SEG-Nachrichten (Technische Mitteilunger der Standard Elektrik Gruppe), 1957. 
No.\,4. 171~p.
\bibitem{8-seif-1}
\Au{Михайлов А.\,И., Черный~А.\,И., Гиляревский~Р.\,С.} Основы 
информатики.~--- М.: Наука, 1968. 425~с.
\bibitem{9-seif-1}
\Au{Otten K., Debons~A.} Towards a~metascience of information: 
Informatology~// J.~Am. Soc. Inform. Sci., 1970. Vol.~21.  
P.~89--94.
\bibitem{10-seif-1}
\Au{Norton M.\,J.} Introductory concepts in information science.~--- 2nd ed.~--- 
ASIST monograph ser.~--- Information Today, 2010. 210~p.
\bibitem{11-seif-1}
\Au{Гуревич И.\,М.} Законы информатики~--- основа строения и познания 
сложных систем.~--- М.: ТОРУС ПРЕСС, 2007. 399~с.
\bibitem{12-seif-1}
\Au{Heylighen F.} Сomplexity and self-organization~// Encyclopedia of library 
and information sciences~/ Eds. M.\,J.~Bates, M.\,N.~Maack.~--- Taylor \& 
Frances, 2008. 20~p. {\sf http://pespmc1.vub.ac.be/Papers/ELIS-complexity.pdf}.
\bibitem{13-seif-1}
\Au{Сейфуль-Мулюков Р.\,Б.} Нефть и газ. Глубинная природа и~ее 
прикладное значение.~--- М: ТОРУС ПРЕСС, 2012. 214~с.
\bibitem{14-seif-1}
\Au{Сейфуль-Мулюков Р.\,Б.} Нефть~--- углеводородные последовательности: 
анализ моделей генезиса и эволюции.~--- М.: 11~формат, 2010. 173~с.
\bibitem{15-seif-1}
\Au{Заде Л.} Понятие лингвистической переменной и~его применение 
к~принятию приближенных решений.~--- М.: Мир, 1976. 176~с.

\end{thebibliography}
} }

\end{multicols}

 \label{end\stat}

 \vspace*{-3pt}

\hfill{\small\textit{Поступила в~редакцию  11.10.2016}}
%\renewcommand{\bibname}{\protect\rm Литература}
\renewcommand{\figurename}{\protect\bf Рис.}
\renewcommand{\tablename}{\protect\bf Таблица} 