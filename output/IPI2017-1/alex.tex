 
 \def\stat{alex}

\def\tit{ИНДИВИДУАЛИЗАЦИЯ ПРОЦЕССА ОБУЧЕНИЯ В~РЕЖИМЕ ВЕБ-КОНФЕРЕНЦИИ 
НА~ОСНОВЕ ИЕРАРХИЧЕСКОЙ НЕЧЕТКОЙ~ЭКСПЕРТНОЙ~СИСТЕМЫ}

\def\titkol{Индивидуализация процесса обучения в~режиме веб-конференции 
на~основе иерархической %нечеткой 
экспертной системы}

\def\aut{А.\,С.~Алексейчук$^1$, А.\,В.~Пантелеев$^2$}

\def\autkol{А.\,С.~Алексейчук, А.\,В.~Пантелеев}

\titel{\tit}{\aut}{\autkol}{\titkol}

\index{Алексейчук А.\,С.}
\index{Пантелеев А.\,В.}
\index{Alekseychuk A.\,S.}
\index{Panteleev A.\,V.}


%{\renewcommand{\thefootnote}{\fnsymbol{footnote}} \footnotetext[1]
%{Работа выполнена при финансовой поддержке РФФИ (проекты 16-07-00677 
%и~15-37-20611-мол\_а\_вед).}}


\renewcommand{\thefootnote}{\arabic{footnote}}
\footnotetext[1]{Московский авиационный институт (национальный исследовательский университет), 
\mbox{alexejchuk@gmail.com}}
\footnotetext[2]{Московский авиационный институт (национальный исследовательский 
университет), \mbox{avpanteleev@inbox.ru}}
  
  \Abst{Рассматривается модель учебного процесса, реализуемого системой 
дистанционного обучения (СДО) в~формате веб-кон\-фе\-рен\-ции. Приводится постановка задачи 
индивидуализации учебного процесса (построения индивидуальной траектории обучения) 
для каждого студента путем выбора подходящего уровня сложности заданий. Предложен 
способ индивидуализации процесса обучения при помощи методов искусственного 
интеллекта и~приведено описание программного комплекса, реализующего как 
дистанционное обучение в~формате веб-кон\-фе\-рен\-ции, так и~управление процессом 
обучения при помощи входящей в~его состав иерархической нечеткой экспертной системы. 
Данная система назначает каждому студенту наиболее рекомендуемый уровень сложности 
предстоящего занятия, исходя из имеющихся исходных данных о студенте и~данных об 
оценках за предыдущие занятия, и~таким образом автоматически формирует расписание 
занятий, группируя студентов с~близким уровнем подготовки. Приведен пример расчетов, 
производимых экспертной системой при построении индивидуальной траектории обучения 
студента.}
  
  \KW{дистанционное обучение; веб-конференция; экспертная система; иерархический 
нечеткий вывод}

\DOI{10.14357/19922264170108}  


\vskip 10pt plus 9pt minus 6pt

\thispagestyle{headings}

\begin{multicols}{2}

\label{st\stat}
  
\section{Введение}
  
  В настоящее время перспективным направлением информатизации 
образования в~вузах является применение СДО 
со встроенными системами веб-кон\-фе\-рен\-ций, позволяющими всем 
участникам учебного процесса видеть и~слышать друг друга в~режиме 
реального времени. Особенностью подоб\-ных СДО является то, что 
взаимодействие студентов и~преподавателя происходит в~заранее определенное 
время, а~трудоемкость проведения занятий быст\-ро возрастает с~ростом числа 
участников конференции. В~данной статье предложен подход к~планированию 
занятий, заключающийся в~разделении студентов на группы и~проведении 
занятий раздельно с~использованием задач различного уровня слож\-ности. Это 
позволит обеспечить присутствие на каждом практическом занятии студентов 
с~сопоставимым уровнем подготовки и~одновременно индивидуализировать 
процесс обуче\-ния каждого студента. Для реализации индивидуального подхода 
предлагается использовать иерархическую нечеткую экспертную сис\-те\-му, 
основанную на модели нечеткого логического вывода~[1]. Ее задача 
в~СДО, предназначенной для выработки навыков решения типовых задач 
предмета,~--- выбирать для каждого студента очередной уровень сложности 
решаемых задач в~зависимости от текущего уровня его подготовки. 
  
  Нечеткие системы уже нашли применение в~сфере высшего образования. 
Так, в~[2] предлагается использовать иерархическую нечеткую базу знаний для 
вычисления общего рейтинга студента, учитывающего его академическую 
успе\-ва\-емость, общественную и~на\-уч\-но-ис\-сле\-до\-ва\-тель\-скую 
деятельность и~т.\,д. В~[3] рассмотрено проектирование иерархической 
нечеткой экспертной сис\-те\-мы, предназначенной для планирования набора 
студентов в~вуз. В~[4] предлагается использовать систему иерархического 
нечеткого вывода для построения\linebreak рейтинга студентов на основе информации об 
оценке за каждое задание, затраченном времени и~экспертных оценках 
сложности данного задания. Имеют\-ся примеры применения методов 
искусственного интеллекта для индивидуализации процесса обучения в~СДО, 
не относящихся к~сис\-те\-мам реального времени~[5]. Описаний систем 
управления учебным процессом в~СДО реального времени, а также примеров 
применения нечетких систем в~сфере обучения в~режиме реального времени 
в~отечественной и~зарубежной научной литературе не приводилось. Поэтому 
предметом исследовательского интереса стало применение нечеткой 
экспертной системы при практическом использовании СДО для управления 
учебным процессом и~индивидуализации процесса обучения каждого студента.

\vspace*{-10pt}
  
\section{Постановка задачи формирования индивидуальной 
траектории обучения}
  
  Пусть учебным планом по дисциплине предусмотрено некоторое число 
занятий~$N$. Сценарий обучения включает встречу участников в~назначенное 
время в~виртуальной аудитории, объяснение материала преподавателем 
с~использованием доступных мультимедийных средств, а также решение всеми 
участниками одного или нескольких типовых заданий (упражнений) в~реальном 
времени под контролем преподавателя. После каждого занятия по его 
результатам с~учетом предыстории и~данных о начальном уровне подготовки 
студенты делятся на три нечетких класса: <<сильные>>, <<средние>> 
и~<<слабые>>. Преподаватель для одного и~того же занятия составляет несколько 
вариантов (уровней) учебных заданий, различающихся по слож\-ности, и~дает 
оценки сложности каждого созданного уровня для каждого класса студентов. 
Перед следующим занятием происходит процедура составления расписания, 
при котором каждому студенту из числа изучающих данную дисциплину 
назначается один определенный уровень сложности заданий предстоящего 
занятия. Соответственно, множество студентов~$S$ разбивается на 
подмножества, соответствующие уровням сложности этого занятия. 
Количество и~состав этих подмножеств зависят от чис\-ла уровней слож\-ности 
занятий и~меняется от занятия к~занятию. Последовательность уровней 
слож\-ности, проходимых каждым студентом при посещении всех занятий 
изучаемой дисциплины, образует его индивидуальную траекторию обучения.
  
  \smallskip
  
  \noindent
  \textbf{Определение.} Индивидуальная траектория обучения  
студента~$T$~--- это последовательность пар вида $\langle$\textit{номер 
занятия; номер уровня сложности ре\-ша\-емых студентом на занятии 
задач}$\rangle$.
  
  \smallskip
  
  Обозначим через $i\hm\in [1,N]$ номер текущего занятия, т.\,е.\ ближайшего 
предстоящего занятия, для которого составляется расписание, если  
($i\hm-1$)-е занятие уже проведено. Модель занятия в~рас\-смат\-ри\-ва\-емой СДО 
представляет собой кортеж 
  \begin{multline*}
  \left( M_i, \left( X_{i1},\overline{C}_i(1)\right), \left( 
X_{i2},\overline{C}_i(2)\right)\,,\ldots\right.\\
\left.\ldots , \left( X_{iM_i}, 
\overline{C}_i(M_i)\right)\right)\,,\ i\in [1,N]\,,
 \end{multline*}
 
 
 { \begin{center}  %fig1
 \vspace*{1pt}
 \mbox{%
\epsfxsize=78.425mm
\epsfbox{ale-1.eps}
}
\end{center}

%\vspace*{-3pt}


\noindent
{{\figurename~1}\ \ \small{Пример оценок сложности задания: \textit{1}~--- сильные студенты;
\textit{2}~--- средние; \textit{3}~--- слабые студенты}}
}

\vspace*{12pt}

\addtocounter{figure}{1}

 
 
 \noindent
  где $M_i$~--- число уровней сложности $i$-го занятия; $X_{i1},\ldots, 
X_{iM_i}$~--- текстовые, графические и~формульные материалы для каждого 
уровня слож\-ности, хранящиеся в~базе данных; $\overline{C}_i(1),\ldots 
\overline{C}_i(M_i)$~--- оценки слож\-ности каждого уровня. Обозначим через  
$j\hm\in [1;M_i]$ номер рассматриваемого уровня сложности текущего занятия. 
Оценка сложности для $j$-го уровня представляется вектором 
$\overline{C}_i(j)\hm= (C_{i1}(j), C_{i2}(j), C_{i3}(j))$, где $C_{i1}(j)\hm\in 
[0,1]$~--- оценка сложности $j$-го уровня $i$-го занятия для класса 
<<сильные>>; $C_{i2}(j)$~--- оценка сложности для класса <<средние>>; 
$C_{i3}(j)$~--- оценка для класса <<слабые>>. На рис.~1 проиллюстрирован 
пример оценок сложности для занятия, включающего четыре уровня 
сложности.

\begin{figure*}[b] %fig2
 \vspace*{1pt}
\begin{center}
\mbox{%
\epsfxsize=165.499mm
\epsfbox{ale-2.eps}
}
\end{center}
\vspace*{-9pt}
  \Caption{Структурная схема иерархической нечеткой экспертной системы}
  \end{figure*}
  

  В качестве модели студента используются тройки величин $\langle\langle 
w\rangle_1^{i-1}, w_{i-1}, \overline{D}\rangle$. Здесь
  \begin{itemize}
\item $\langle w\rangle_1^{i-1}\in [0;1]$~--- средневзвешенная сумма оценок, 
полученных студентом по данной дис\-циплине с~первого до ($i-1$)-го занятия 
включитель\-но (при их наличии), вычисляемая по формуле:
\begin{equation}
\langle w\rangle_1^{i-1}= \fr{\sum\nolimits_{k=1}^{i-1}  
e^{-\beta(i-k-1)}w_k} {\sum\nolimits_{k=1}^{i-1} e^{-\beta(i-k-1)}}\,, 
\enskip i\geq 2\,,
\label{e1-al}
\end{equation}
  где $w_k\in [0;1]$~--- оценка за $k$-е занятие, $k\hm=1,\ldots, N$; 
$\beta\hm\in [0,\beta_{\max}]$~--- задаваемый преподавателем параметр, 
характеризующий скорость уменьшения влияния предыдущих оценок на 
средневзвешенную сумму и~позволяющий учитывать изменение 
подготовленности студента по данной дисциплине со временем; 
\item $w_{i-1}\in [0;1]$~--- оценка, полученная студентом за предыдущее 
занятие (при наличии); 
\item $\overline{D}$~--- вектор, содержащий общие данные о~студенте, 
имеющиеся на момент его поступления в~учебное заведение: возраст, пол, 
отделение (платное или бюджетное), средний балл\linebreak в~аттестате, оценка за единый
государственный экзамен (ЕГЭ) 
по дисциплине и~оценка за вступительный тест. Пол студента и~отделение 
представлены числовыми значениями: 0~--- мужской пол, 1~--- женский пол; 
0~--- бюджетное отделение, 1~--- платное отделение.
\end{itemize}
  
  Требуется создать модель работы информационной системы, автоматически 
составляющей расписание проведения текущего занятия для различных 
уровней, т.\,е.\ формирующей разбиение множества студентов~$S$ на 
подмножества  $S_1,S_2,\ldots, S_{M_i}$, соответствующие уровням 
сложности занятия.
  
  Для разбиения множества~$S$ требуется вы\-чис\-лить рекомендуемый номер 
уровня~$J_i$ текущего\linebreak занятия для каждого студента с~использованием всей 
доступной информации, содержащейся в~модели студента и~модели занятия, 
т.\,е.\ $J_i\hm= J_i\left( \langle w\rangle_1^{i-1}, w_{i-1}, \overline{D}, 
\overline{C}_{i-1}(J_{i-1}), \left\{ 
\overline{C}_i(j)\right\}\left\vert_{j=1}^{M_i}\right.\right)$, $i\hm\geq 2$. Для $i\hm=1$ 
(первого занятия) номер рекомендуемого уровня должен определяться без 
учета оценки за предыдущее занятие и~истории оценок, т.\,е. $J_1\hm= J_1\left( 
\overline{D}, \left\{ \overline{C}_1(j)\right\} \left\vert_{j=1}^{M_1}\right.\right)$.
  
  Для построения математической модели задачи выбора рекомендуемого 
уровня сложности введем понятие степени рекомендуемости.
  
  \smallskip
  
  \noindent
  \textbf{Определение.} \textit{Степень рекомендуемости}~--- это чис\-ловой 
параметр~$Q_i(j)\hm\in [0;1]$, соответствующий данному студенту и~$j$-му 
уровню сложности $i$-го занятия и~отражающий степень соответствия этого 
уровня сложности текущему состоянию модели студента. 
  
  Вычислив $Q_i(j)$ для всех $j\hm= 1,\ldots, M_i$, \mbox{можно} определить 
рекомендуемый уровень, для кото\-ро\-го достигается максимум степени 
ре\-ко\-мен\-ду\-емости:\\[-8pt] 
$$
J_i\hm=\mathrm{arg}\,\max\limits_{j=1,\ldots, M_i} Q_i(j)\,. 
$$
Индивидуальная траектория обучения, сформированная в~соответствии с~этой 
моделью, будет иметь вид:\\[-12pt]
$$
T= \left(\langle 1,J_1\rangle, \langle 2,J_2\rangle, 
\langle 3,J_3\rangle, \ldots, \langle N, J_N\rangle\right)\,.
$$

\vspace*{-10pt}
  
\section{Структура иерархической нечеткой экспертной системы}
  
  Для построения траекторий обучения студентов предлагается экспертная 
система, интегрированная в~единую среду с~рассматриваемой СДО. Она 
представляет собой информационную систему, основанную на методах 
искусственного интеллекта и~использующую механизм нечеткого вывода. 
Структура экспертной системы приведена на рис.~2. В~ее состав входят 
следующие компоненты: пять блоков фаззификации, два блока композиции, 
три блока нечеткого вывода и~блок дефаззификации. 


  Для моделирования переменных используются лингвистические переменные с~конечным набором термов, выбранных таким образом, чтобы экспертам было 
удобно составлять правила нечеткого вывода на естественном языке, например: 
<<Если сбалансированная оценка за предыдущее занятие отличная, текущая 
задача сложная и~текущая успеваемость студента средняя, то значение степени 
рекомендуемости текущей задачи~--- высокое>>. 

\begin{figure*} %fig3
   \vspace*{1pt}
\begin{center}
\mbox{%
\epsfxsize=161.261mm
\epsfbox{ale-3.eps}
}
\end{center}
\vspace*{-9pt}
  \Caption{Функции принадлежности термов переменных~$C$ <<сложность 
занятия>>~(\textit{а}) и~$W$ <<оценка за 
занятие>>~(\textit{б}) }
  \end{figure*}
  
  \textit{Блок фаззификации оценок сложности $i$-го занятия} осуществляет 
фаззификацию оценок сложности $j$-го уровня текущего занятия, 
пред\-став\-лен\-ных компонентами вектора $\overline{C}_i(j)\hm= \left( C_{i1}(j),\right.$\linebreak
$\left. 
C_{i2}(j), C_{i3}(j)\right)$. Графики функций принадлежности термов 
$\mu^1_C(x)$, $\mu^2_C(x)$ и~$\mu^3_C(x)$ лингвистической переменной~$C$ 
<<сложность занятия>> приведены на рис.~3,\,\textit{а}. Выход  
блока~--- матрица $A_i(j)\hm= \| a_{kp,i}(j)\|$, где $a_{kp,i}(j)\hm= 
\mu^k_C(C_{ip}(j))$, $k\hm= \overline{1,3}$, $p\hm= \overline{1,3}$, $j\hm= 
1,\ldots, M_i$.
  
  Следующий за ним блок композиции осуществ\-ляет вычисление вектора 
 $\tilde{C}_i(j)$, соответст\-ву\-юще\-го\linebreak
  степеням принадлежности сложности текущего 
зада\-ния для данного студента термам лингвистической переменной~$C$ 
<<сложность задания>> с~терм-мно\-же\-ст\-вом $T_C$\;=\;\{<<высокая>>, 
<<средняя>>, <<низкая>>\}. Для расчета используется модель текущей\linebreak 
успе\-ва\-емости студента, которой соответствует лингвистическая 
переменная~$L$ <<текущая ус\-пе\-ва\-емость>> с~терм-мно\-же\-ст\-вом 
$T_L$\;=\;\{<<высокая>>, <<средняя>>, <<низкая>>\}. Вычисление 
производится при помощи оператора максиминной композиции по формуле 
$\tilde{C}_i(j)\hm= A_i(j)\circ \overline{L}$, где $A_i(j)$~--- матрица, 
представляющая результат фаззификации всех компонент 
вектора~$\overline{C}_i(j)$; $\overline{L}$~--- вектор, представляющий 
значение лингвистической переменной~$L$ <<текущая успеваемость>>. 
Компоненты вектора~$\tilde{C}_i(j)$ вы\-чис\-ля\-ют\-ся по формуле:
$$
\tilde{C}_{k,i}(j)= \max\limits_{p=\overline{1,3}}\min \left\{ a_{kp,i}(j), 
L_p\right\},\enskip  k=\overline{1,3}\,.
$$

 
  \textit{Блок фаззификации оценок сложности $(i\hm-1)$-го занятия} 
осуществляет фаззификацию оценок сложности предыдущего пройденного 
занятия $\overline{C}_{i-1}(J_{i-1})$. Номер предыдущего уровня 
сложности~$J_{i-1}$ известен и~хранится в~базе данных. Выходом блока 
является матрица $A_{i-1}(J_{i-1})\hm= \| a_{kp}(J_{i-1})\|_{i-1}$, где 
$a_{kp}(J_{i-1})\hm= \mu_C^k(C_{i-1,p}(J_{i-1}))$. Принцип работы этого 
блока и~следующего за ним блока композиции аналогичны принципам работы 
блока фаззификации оценок сложности $i$-го занятия и~следующего за ним 
блока композиции.
  
  \textit{Блок фаззификации оценки за $(i\hm-1)$-е занятие} осуществляет 
вычисление вектора, соответст\-ву\-юще\-го степеням принадлежности оценки за 
предыду\-щее пройденное занятие термам лингвистической переменной~$W$ 
<<оценка за занятие>>. Входной переменной блока является значение 
оценки~$w_{i-1}$, принимающее непрерывные значения из отрезка $[0,1]$, 
а~выход~--- вектор\\[-9pt] 
$$
\tilde{w}_{i-1}= \left( \mu^1_w(w_{i-1})\ 
  \mu^2_w(w_{i-1})\ \mu^3_w (w_{i-1})\ \mu_w^4(w_{i-1})\right)^{\mathrm{T}}.
  $$
Поскольку для со\-став\-ле\-ния правил вывода пре\-подавателю удобно оперировать 
привычными оценка\-ми из четырехбалльной сис\-те\-мы, то в~качестве %\linebreak  
терм-мно\-жест\-ва лингвистической переменной~$W$ выбрано мно\-же\-ст\-во 
$T_W$\;=\;\{<<отлично>>, <<хоро\-шо>>, <<удовле\-тво\-ри\-тель\-но>>, 
<<неудовлетворительно>>\}. Функции принадлежности термов приведены на 
рис.~3,\,\textit{б}. Если система оценки знаний имеет вид <<за\-чет--не\-за\-чет>>, то 
используются два терма и~соответствующие функции принадлеж\-ности.
  
  
  
  
  \textit{Блок фаззификации средневзвешенной суммы оценок} осуществляет 
вычисление вектора, соответствующего степеням принадлежности 
средневзвешенной суммы оценок термам лингвистической\linebreak переменной~$W$ 
<<оценка за занятие>>. В~случае отсутствия истории оценок у~данного 
студента блок исключается из работы системы. Входной переменной является 
средневзвешенная сумма оценок~$\langle w\rangle_1^{i-1}$, вычисленная по 
формуле~(\ref{e1-al}). Работа блока аналогична работе блока фаззификации 
оценки за ($i\hm-1$)-е занятие. 

\begin{table*}[b]\small %tabl1
  \begin{center}
  \Caption{Значения переменной <<текущая успеваемость студента>>}
  \vspace*{2ex}
  
  \begin{tabular}{|l|c|c|c|c|c|}
  \hline
\multicolumn{1}{|c|}{Начальный уровень}&\multicolumn{5}{c|}{Взвешенная 
средняя оценка}\\
\cline{2-6}
\multicolumn{1}{|c|}{студента}&Отлично&Хорошо&Удовлетворительно&Неудовлетворительно&Отсутствует\\
\hline
Сильный&Высокая&Средняя&Средняя&Низкая&Высокая\\
Средний&Высокая&Средняя&Низкая&Низкая&Средняя\\
Слабый&Средняя&Средняя&Низкая&Низкая&Низкая\\
\hline
\end{tabular}
\end{center}
\end{table*}
  
  \textit{Блок фаззификации начального уровня подготовки студента} 
предназначен для моделирования начального уровня успеваемости каждого 
студента на основе общей информации о~студентах, хранящейся в~базе данных. 
Модель начального уровня\linebreak успеваемости студента представлена 
лингвистической переменной   <<начальный уровень успе\-ва\-емости>> 
  с~терм-мно\-жест\-вом $T_B$\;=\;\{<<сильный>>, <<средний>>, 
<<слабый>>\}.
  
  Задачу вычисления степеней принадлежности уровня данного студента 
термам лингвистической переменной~$B$ можно рассматривать как задачу 
нечеткой классификации при наличии обучающей выборки. Нечеткая 
классификация представляет собой разбиение множества студентов на три 
класса~--- <<сильные>>, <<средние>> и~<<слабые>>, при которой\linebreak каждый 
студент может принадлежать одновре\-менно нескольким классам с~различной 
степенью принадлеж\-ности. Каждый элемент обучающей выборки содержит 
вектор признаков~$\overline{D}$, а~также экспертную информацию о~том, 
к~каким классам и~с~какой степенью принадлежности относится данный 
студент. 
  
  Блок фаззификации реализован на основе нейронной сети. Вход блока~--- 
вектор признаков рассматриваемого студента~$\overline{D}$ с~координатами, 
приведенными линейным преобразованием к~отрезку $[-1,1]$. Выход блока~--- 
век\-тор-стол\-бец $\tilde{B}\hm= \| b_i\|$, $i\hm=\overline{1,3}$, где $b_i$~--- 
степень принадлежности студента $i$-му классу. Для экспериментального 
определения конфигурации сети и~ее обучения была использована выборка 
из~220~записей о студентах. Проведено многократное обучение сети 
с~различными вариантами архитектуры и~параметрами алгоритмов обучения. 
В~результате получена следующая конфигурация. Архитектура сети 
представляет собой многослойный перцептрон с~6~входными нейронами 
(в~соответствии с~размерностью вектора~$\overline{D}$), двумя скрытыми 
слоями (4~нейрона в~первом скрытом слое, 5~нейронов во втором) 
и~3~нейронами выходного слоя. Функция активации нейронов скрытых 
слоев~--- симметричная сигмоидная, описываемая формулой 
$$
f_z(x)=\fr{2}{1+e^{-zx}}-1
$$ 
с~крутизной функции активации $z\hm=0{,}7$. Алгоритм 
обучения~--- <<быст\-рый алгоритм обратного распространения ошибки>> 
(resilient propagation algorithm)~\cite{6-al}, число эпох обучения~---~70.
  
  Следующие друг за другом три блока нечеткого вывода образуют 
иерархическую систему, где результаты работы первого и~второго блоков не 
подвергаются дефаззификации, а непосредственно передаются на вход третьего 
блока. Каждый из блоков нечеткого вывода использует схему нечеткого вывода 
Мамдани~\cite{1-al, 7-al}. Структура базы знаний каждого из блоков 
представляется в~виде набора лингвистических правил вида:

\noindent
\begin{multline*}
  R_l: \mbox{Если\ } x_1\ \mbox{есть } A_{l1}\ \mbox{и } x_2\  \mbox{есть } 
A_{l2}\ \mbox{и}\ \ldots\ \\
\ldots \mbox{ и } x_p\ \mbox{есть } A_{lp},\ \mbox{то } y\  
\mbox{есть } B_l^m, \enskip l=1,\ldots, L\,,
  \end{multline*}
где $L$~--- количество правил вывода блока; $x_1,\ldots ,x_p$~--- входные 
лингвистические переменные; $A_{l1},\ldots,A_{lp}$~--- термы входных 
переменных; $y$~--- выходная лингвистическая переменная; $B_l^m$ 
$(m\hm=1,\ldots K_y)$~--- $m$-й терм выходной переменной, находящийся 
в~заключении $l$-го правила; $K_y$~--- чис\-ло термов выходной переменной; 
$p$~--- количество входных переменных блока. 
  
  \textit{Блок нечеткого вывода~I~уровня иерархии} (моделирования 
текущей успеваемости студента) принимает на вход средневзвешенную сумму 
оценок $\langle \tilde{w}\rangle_1^{i-1}$ и~начальный уровень подготовки 
студента~$\tilde{B}$. Выход блока~--- вектор~$\tilde{L}$, представляющий 
значение лингвистической переменной~$L$ <<текущая успева\-емость 
студента>>. Правила вывода блока приведены в~табл.~1. 
  
  


  
  \textit{Блок нечеткого вывода II~уровня иерархии} (моделирования 
сбалансированной оценки за предыду\-щее занятие) принимает на вход 
переменные: $\tilde{C}_{i-1}$~--- нечеткая оценка сложности пройденного 
уровня ($i-1$)-го занятия для данного студента;\linebreak $\tilde{w}_{i-1}$~--- оценка, 
полученная студентом на предыду\-щем занятии. Выход блока~--- 
вектор~$\tilde{w}^C_{i-1}$, представля\-ющий степени принадлежности 
сбалансированной оценки термам лингвистической переменной~$W$ <<оценка 
за занятие>>. Сбалансированная оценка~--- это оценка, скорректированная 
с~учетом соответствия сложности данной задачи уровню подготовки студента. 
Правила вывода блока~II~уровня приведены в~табл.~2.
  
  \begin{table*}\small %tabl2
\vspace*{-12pt}
  \begin{center}
  \Caption{Значения переменной <<сбалансированная оценка>>}
  \vspace*{2ex}
  
  \begin{tabular}{|l|c|c|c|}
  \hline
\multicolumn{1}{|c|}{Оценка}&\multicolumn{3}{c|}{Сложность занятия}\\
\cline{2-4}
\multicolumn{1}{|c|}{за занятие}&Высокая&Средняя&Низкая\\
\hline
Отлично&Отлично&Отлично&Хорошо\\
Хорошо&Хорошо&Хорошо&Удовлетворительно\\
Удовлетворительно&Хорошо&Удовлетворительно&Удовлетворительно\\
Неудовлетворительно&Удовлетворительно&Неудовлетворительно& 
Неудовлетворительно\\
\hline
\end{tabular}
\end{center}
%\vspace*{-3pt}
%\end{table*}
%\begin{table*}\small %tabl3
\begin{center}
\Caption{Значения переменной <<cтепень рекомендуемости>>}
\vspace*{2ex}

\begin{tabular}{|l|c|c|c|} %
\hline
\multicolumn{1}{|c|}{\raisebox{-6pt}[0pt][0pt]{Успеваемость}}&\multicolumn{3}{c|}{Сложность}\\
\cline{2-4}
&Высокая&Средняя&Низкая\\
\hline
Сильный&Очень высокая&Высокая&Очень низкая\\
Средний&Высокая&Высокая&Низкая\\
Слабый&Средняя&Очень высокая&Высокая\\
\hline
\end{tabular}
\end{center}
\vspace*{-3pt}
\end{table*}



 
  В случае отсутствия оценки за предыдущее занятие блок исключается из 
работы экспертной сис\-темы.
  
  \textit{Блок нечеткого вывода III~уровня иерархии} (вычисления 
степени рекомендуемости) принимает\linebreak на вход переменные: $\tilde{C}_i$~--- 
оценка сложности $j$-го уровня $i$-го (текущего) занятия для данного 
студента; $\tilde{L}$~--- текущий уровень подготовки студента; 
$\tilde{w}^C_{i-1}$~--- сбалансированная оценка за предыду\-щее занятие (при 
наличии). Выход блока представляет собой агрегированную функцию 
принадлежности $\mu_{Q,i,j}(x)$ лингвистической переменной~$Q$ <<степень 
рекомендуемости>> с~терм-мно\-же\-ст\-вом $T_Q$\;=\;$\{$<<очень 
высокая>>, <<высокая>>, <<средняя>>, <<ниже среднего>>, <<низкая>>$\}$.
  
  Правила вывода блока при оценке <<отлично>> за предыдущее занятие 
представлены в~табл.~3. Подобным образом составлены правила вывода и~для 
других значений оценки за предыдущее занятие~--- <<хорошо>>, 
<<удовлетворительно>>, <<неудовлетворительно>> и~для случая отсутствия 
оценки.
  

  

  
  Нечеткий вывод по схеме Мамдани включает следующие этапы.
  \begin{description}
\item[\,]  Первый этап~--- нахождение степени истинности антецедента каждого 
правила с~использованием операции вычисления минимума: 
$$
\alpha_l= 
\min\limits_{k=1,\ldots, p} A_{lk},\enskip l\hm = 1,\ldots, L\,.
$$
\item[\,]  
  Второй этап~--- это процедура нахождения степени принадлежности 
выходной переменной~$\tilde{y}$ термам выходной переменной блока. 
Обозначим через $\{l_n\}$ множество номеров правил вывода, содержащих 
в~заключении $n$-й терм выходной переменной~$B_l^n$. Тогда в~блоках~I 
и~II уровня иерархии $n$-я компонента вектора выходной 
переменной~$\tilde{y}$ будет определяться по формуле:
  $$
  y_n= \max\limits_{l\in \{l_n\}} \left\{ \alpha_l\right\} =\max\limits_{l\in \{l_n\}} 
\left\{ \min\limits_{k=1,\ldots, p} A_{lk}\right\}\,.
  $$
  \end{description}
  
  Для I и~II блоков на этом этапе алгоритм работы заканчивается, и~вектор 
выходной переменной~$\tilde{y}$ передается на следующий уровень иерархии 
вывода. В~III~блоке предусмотрены еще два этапа~--- активизация 
подзаключений и~агрегация. Пусть $\mu_{Q_l,i,j}(x)$~--- функция 
принадлежности $l$-го терма выходной переменной~$Q$ для $j$-го уровня 
$i$-го занятия. Активизация подзаключений представляет собой построение 
усеченных функций принадлежности $\mu^*_{Q_l,i,j}(x)$ с~уровнем 
отсечения~$\alpha_l$:
  $$
  \mu^*_{Q_l,i,j}(x)=\min\left\{ \mu_{Q_l,i,j}(x),\,\alpha_l\right\}\,,\ l=1,\ldots, L\,.
  $$
  
  Затем производится агрегация путем объединения полученных усеченных 
функций принадлежности: 
$$
\mu_{Q,i,j}(x)= \max\limits_{l=1,\ldots, L} 
\mu^*_{Q_l,i,j}(x)\,.
$$
  
  \textit{Блок дефаззификации} принимает на вход агрегированную функцию 
принадлежности $\mu_{Q,i,j}(x)$ из~III~блока нечеткого вывода, а выходом 
блока является окончательное числовое значение степе-\linebreak\vspace*{-12pt}

{ \begin{center}  %fig4
 \vspace*{14pt}
 \mbox{%
\epsfxsize=77.087mm
\epsfbox{ale-5.eps}
}


\vspace*{7pt}


\noindent
{{\figurename~4}\ \ \small{Дефаззификация выходной переменной}}
\end{center}
}

%\vspace*{12pt}

\addtocounter{figure}{1}

 \begin{figure*}[b] %fig5
   \vspace*{12pt}
\begin{center}
\mbox{%
\epsfxsize=159.725mm
\epsfbox{ale-6.eps}
}
\end{center}
\vspace*{-9pt}
  \Caption{Структурная схема СДО}
  \end{figure*}
  

\pagebreak

\noindent
ни  рекомендуемости~$Q_i(j)$. Дефаззификация в~этом блоке производится 
методом поиска <<центра тя\-жести<< (рис.~4) по формуле:
  \begin{equation}
  Q_i(j)= \fr{\int\nolimits_0^1 x\mu_{Q,i,j}(x)\,dx}{\int\nolimits_0^1 
\mu_{Q,i,j}(x)\,dx}\,.
  \label{e2-al}
  \end{equation}
    
  Интегралы в~формуле~(\ref{e2-al}) находятся численно с~использованием 
метода трапеций.
  
  
  
  После нахождения искомой степени рекомендуемости~$Q_i(j)$ для всех 
$j\hm=1,\ldots, M_i$ студенту назначается уровень~$J_i$ с~наибольшим 
значением~$Q_i$. Расчет по приведенной выше методике производится для 
каждого студента, и~в~соответствии с~результатами расчетов составляется 
расписание, в~котором каждому студенту назначен наиболее ре\-ко\-мен\-ду\-емый 
ему уровень.

\vspace*{-6pt}
  
\section{Структура системы дистанционного обучения}
  
  Экспертная система, рассмотренная в~статье, входит в~состав системы 
дистанционного обучения, обеспечивающей полноценный учебный процесс 
и~включающей в~себя функции администрирования, управления 
пользователями (студентами и~преподавателями), составления учебных планов и~заданий, подготовки и~проведения занятий, анализа результатов студентов 
в~различных разрезах, проведения открытых веб-конференций и~консультаций. 
Набор функций системы и~порядок работы с~ней подробно рассмотрены 
в~\cite{8-al}. 
  
  Структура компонентов СДО показана на рис.~5 в~виде UML
  (unified modeling language) диа\-грам\-мы. 
В~структуре можно выделить пять основных компонентов: клиентское 
приложение, серверное приложение, медиасервер, HTTP (hypertext transfer protocol)
сер\-вер и~система 
управления базой данных (СУБД). В~качестве HTTP-сер\-ве\-ра и~СУБД 
использованы готовые программные решения~--- \mbox{Nginx} и~\mbox{PostgreSQL}, 
остальные компоненты являются оригинальными. 
  
 
  Серверное приложение обрабатывает HTTP-за\-про\-сы от клиентов, создает 
и~обрабатывает запросы к~базе данных, формирует страницы веб-ин\-тер\-фей\-са 
(front-end) программного комплекса, отвечает за авторизацию пользователей. 
Приложение создано с~использованием фреймворка Ruby on Rails. Экспертная 
система входит в~состав серверного приложения в~виде библиотеки алгоритмов, 
написанных на языке Ruby.

\begin{table*}[b]\small %tabl4
%\vspace*{-2pt}
  \begin{center}
  \Caption{Результаты моделирования траектории обучения}
  \vspace*{2ex}
  
  \begin{tabular}{|c|c|c|c|c|c|c|c|}
  \hline
\multicolumn{1}{|c|}{\raisebox{-6pt}[0pt][0pt]
{\tabcolsep=0pt\begin{tabular}{c}Номер\\ попытки\end{tabular}}}&
\multicolumn{1}{c|}{\raisebox{-6pt}[0pt][0pt]
{\tabcolsep=0pt\begin{tabular}{c}Номер\\ занятия\end{tabular}}}&\multicolumn{4}{c|}{\tabcolsep=0pt\begin{tabular}{c}Степень рекомендуемости\\ 
уровней сложности\end{tabular}}&
\multicolumn{1}{c|}{\raisebox{-6pt}[0pt][0pt]
{\tabcolsep=0pt\begin{tabular}{c}Рекомендуемый\\ уровень\end{tabular}}}&
\tabcolsep=0pt\begin{tabular}{c}Оценка\\ за\end{tabular}\\
\cline{3-6}
&&1&2&3&4&& занятие\\
\hline
1&1&0,525&\textbf{0,560}&0,493&0,475&2&100\hphantom{9}\\
2&2&\textbf{0,793}&0,773&0,547&0,528&1&100\hphantom{9}\\
3&3&\textbf{0,853}&0,755&0,425&0,147&1&45\\
4&3 (повтор)&0,580&\textbf{0,620}&0,519&0,419&2&90\\
5&4 &0,595&\textbf{0,634}&0,567&0,438&2&95\\
6&5 &\textbf{0,832}&0,683&0,549&0,167&1&40\\
7&5 (повтор)&0,474&\textbf{0,698}&0,640&0,512&2&30\\
8&5 (повтор)&0,405&0,577&0,577&\textbf{0,601}&4&70\\
9&6&0,470&0,552&\textbf{0,567}&0,557&3&80\\
10\hphantom{9}&7&0,487&\textbf{0,556}&0,506&0,457&2&100\hphantom{9}\\
\hline
\end{tabular}
\end{center}
\end{table*}
  
  
  Клиентское приложение для проведения веб-кон\-фе\-рен\-ций представляет 
собой графическое SWF (Small Web Format) приложение, встроенное в~веб-ин\-тер\-фейс 
программного комплекса. Приложение служит для отображения 
пользовательского интерфейса веб-кон\-фе\-рен\-ции, обработки 
и~синхронизации действий пользователей, кодирования и~декодирования 
аудио- и~видеопотоков. Приложение реализовано на основе  
Flash-тех\-но\-ло\-гии.
  
  HTTP-сервер~--- это промежуточный компонент, который перенаправляет 
HTTP-за\-п\-ро\-сы пользователей серверному приложению и~передает ответы 
приложения обратно пользователям, а~также передает статические ресурсы~--- 
файлы изображений, CSS (cascading style sheet) и~JS
(Java Script) фай\-лы, SWF-файл клиентского приложения. 
  
  Медиасервер~--- это компонент, осуществля\-ющий обмен аудио- 
и~видеопотоками между всеми участниками веб-кон\-фе\-рен\-ции 
и~обеспечивающий работу мультимедийных функций режима реального 
времени: показ презентаций, рисование на виртуальной доске, обмен файлами, 
показ рабочего стола, общение в~чате, решение математических задач 
в~графическом интерфейсе с~WYSIWYG (what you see is what you get)
ре\-дак\-то\-ром. Медиасервер 
реализован при помощи технологии Java.
  
  %\vspace*{-10pt}

\section{Моделирование траектории обучения для одного студента}
  
  Рассмотрим результат моделирования индивидуальной траектории для 
одного студента при помощи указанной СДО. 

Моделируемый курс состоит из 
$N\hm=7$ занятий, для каждого из которых задано $M_i\hm=4$ уровня 
сложности заданий. Уровни пронумерованы в~порядке уменьшения сложности. 
Пример оценок сложности для одного из занятий приведен на рис.~1, для 
остальных занятий оценки сложности выглядят аналогично. В~случае 
получения оценки ниже порогового значения в~60\% по ка\-ко\-му-ли\-бо 
занятию студенту назначается повторное прохождение того же занятия, 
а~требуемый уровень сложности вновь определяется экспертной системой 
с~учетом последней полученной оценки. Обучение считается законченным, 
когда выполняется условие $w_N\hm\geq 60\%$, т.\,е.\ студент получил 
положительные оценки за все занятия курса.
  
  В табл.~4 приведены результаты расчета траек\-тории. Показаны степени 
рекомендуемости для каждо\-го уровня сложности и~выделен уровень 
с~наибольшей степенью рекомендуемости. В~последнем столбце приведена 
последовательность оценок~$w_i$, использованных для моделирования 
процесса обуче\-ния. Выбраны следующие исходные данные о~студенте: 
возраст~--- 17~лет; пол~--- мужской; отделение~--- бюджетное; средняя оценка 
в~аттестате~--- 4,0; оценка за ЕГЭ по изучаемой дис\-цип\-ли\-не~--- 52~балла; 
оценка за вступительный тест~--- 57~баллов.
  
  
  На рис.~6 приведена визуализация полученной траектории. 
Прямоугольниками выделено повторное прохождение одного и~того же занятия 
после получения оценки ниже~60\%.
  
  \begin{figure*} %fig6
   \vspace*{1pt}
\begin{center}
\mbox{%
\epsfxsize=160.48mm
\epsfbox{ale-7.eps}
}
\end{center}
\vspace*{-9pt}
  \Caption{Пример траектории обучения студента}
  \end{figure*}
  
  Как видно из графика, экспертная система рекомендовала начать обучение со 
второго уровня сложности ввиду хороших, но не самых высоких входных 
результатов студента. После получения высокой оценки (100\%) система 
предложила решать задачи первого, самого сложного уровня. После получения 
низкой оценки (45\%) за третье занятие студенту предложено пройти второй, 
более простой уровень того же занятия. Далее после получения высоких оценок 
траектория возвращается к~первому уровню. После получения очень низких 
оценок траектория спускается до самого простого, четвертого уровня. Далее 
траектория вновь поднимается до второго уровня, но не до первого, так как 
сказывается недавняя история низких оценок за~5-е~занятие.
  
\section{Выводы}
  
  В статье приведено описание системы дистанционного обучения на основе 
веб-кон\-фе\-рен\-ций, позволяющей планировать и~проводить занятия в~режиме 
реального времени. 

Предложена иерархическая нечеткая экспертная система, 
основанная на методах искусственного интеллекта, поз\-во\-ля\-ющая 
автоматически обрабатывать информацию о~студентах и~экспертные данные 
для помощи преподавателю в~формировании стратегии обучения каждого 
студента. Как следует из результатов моделирования, разработанная система 
эффективно справляется с~построением индивидуальной траектории студента, 
учитывая начальный уровень подготовки студента и~изменение его уровня 
подготовки в~процессе накопления истории полученных результатов 
обучения и~оперативно реагируя на получение различных текущих оценок. 

На кафедре 
<<Математическая кибернетика>> МАИ было проведено тестирование 
созданной сис\-те\-мы для дистанционного обучения студентов факультета, 
расположенного в~г.~Луховицы. Получены положительные отзывы студентов 
и~администрации об удобстве и~функциональности разработанной СДО, 
а~также значительно упрощена работа преподавателя благодаря раздельному 
проведению занятий разного уровня сложности со студентами сопоставимого 
уровня подготовки.
  
{\small\frenchspacing
 {%\baselineskip=10.8pt
 \addcontentsline{toc}{section}{References}
 \begin{thebibliography}{9}
\bibitem{1-al}
\Au{Passino K.\,M., Yurkovich~S.} Fuzzy control.~--- Addison Wesley 
Longman, 1998. 502~p.
\bibitem{2-al}
\Au{Гоппов С.\,И.} Разработка нечеткой модели общего рейтинга студента 
на основе иерархической базы~// Студенческий научный форум: Мат-лы 
III~конф.~--- СПб: СПбГЭУ, 2011. С.~1--6.
\bibitem{3-al}
\Au{Берёза А.\,Н., Ершова~Е.\,А.} Поддержка принятия решения при 
планировании набора абитуриентов в~вузе на основе нечетких моделей~// 
Известия Южного федерального университета. Технические науки, 2011. 
№\,7. С.~131--136.
\bibitem{4-al}
\Au{Saleh I., Kim~S.} A fuzzy system for evaluating students' learning 
achievement~// Expert Syst. Appl., 2009. Vol.~36. P.~6236--6243.
\bibitem{5-al}
\Au{Stathacopoulou~R., Magoulas~G.\,D., Grigoriadou~M., Samarakou~M.} 
Neuro-fuzzy knowledge processing in intelligent learning environments for 
improved student diagnosis~// Inform. Sciences, 2005. Vol.~170. P.~273--307. 



\bibitem{6-al} 
\Au{Igel Ch., Husken~M.} Empirical evaluation of the improved Rprop learning 
algorithms~// Neurocomputing, 2003. Vol.~50. P.~105--123.

\bibitem{7-al} 
\Au{Mamdani E.\,H.} Application of fuzzy logic to approximate reasoning using 
linguistic synthesis~// Fuzzy Set. Syst., 1977. Vol.~26. P.~1182--1191.
\bibitem{8-al}
\Au{Алексейчук А.\,С.} Программная система для обучения студентов 
экономических специальностей в~режиме веб-кон\-фе\-рен\-ции~// Научный 
альманах. Вып.~21: Инновации в~экономике и~менеджменте 
в~аэрокосмической промышленности: Мат-лы XII~конф. молодых ученых 
и~студентов.~--- М.: Доброе слово, 2016. С.~190--198.

 \end{thebibliography}

 }
 }

\end{multicols}

\vspace*{-3pt}

\hfill{\small\textit{Поступила в~редакцию 7.10.16}}

%\vspace*{8pt}

\newpage

\vspace*{-24pt}

%\hrule

%\vspace*{2pt}

%\hrule

%\vspace*{8pt}


\def\tit{MODELING INDIVIDUALIZATION OF~THE~LEARNING PROCESS 
IN~THE~FORM OF~WEB CONFERENCE USING A~HIERARCHICAL FUZZY 
EXPERT SYSTEM}

\def\titkol{Modeling individualization of~the~learning process 
in~the~form of~web conference using a~hierarchical fuzzy 
expert system}

\def\aut{A.\,S.~Alekseychuk and A.\,V.~Panteleev}

\def\autkol{A.\,S.~Alekseychuk and A.\,V.~Panteleev}

\titel{\tit}{\aut}{\autkol}{\titkol}

\vspace*{-9pt}


 \noindent
Faculty of Applied Mathematics and Physics, 
Moscow Aviation Institute (National Research 
University), 4~Volokolamskoye Highway, A-80, GSP-3, 
Moscow 125993, Russian Federation 



\def\leftfootline{\small{\textbf{\thepage}
\hfill INFORMATIKA I EE PRIMENENIYA~--- INFORMATICS AND
APPLICATIONS\ \ \ 2017\ \ \ volume~11\ \ \ issue\ 1}
}%
 \def\rightfootline{\small{INFORMATIKA I EE PRIMENENIYA~---
INFORMATICS AND APPLICATIONS\ \ \ 2017\ \ \ volume~11\ \ \ issue\ 1
\hfill \textbf{\thepage}}}

\vspace*{3pt}
  
  
    
  \Abste{The paper describes a model of educational process in the form of web 
conference. The problem of individualization of learning process for each student is 
solved by selecting the appropriate level of difficulty for each lesson. The proposed 
method of individualization is implemented using artificial intelligence methods. The 
paper describes a software system that implements distance learning in the form of 
web conference as well as management of the learning process with a~built-in 
hierarchical fuzzy expert system. This system assigns the most recommended level of 
difficulty of the upcoming lessons to each student using available initial data about 
the student and his or her previous grades. The system automatically generates 
a~schedule where students with similar levels of performance are grouped together. 
An example of calculations made by the expert system is provided.}

  \KWE{distance learning; web conference; expert system; hierarchical fuzzy 
inference}
  
\DOI{10.14357/19922264170108}  

%\vspace*{-9pt}

%\Ack



%\vspace*{3pt}

  \begin{multicols}{2}

\renewcommand{\bibname}{\protect\rmfamily References}
%\renewcommand{\bibname}{\large\protect\rm References}

{\small\frenchspacing
 {%\baselineskip=10.8pt
 \addcontentsline{toc}{section}{References}
 \begin{thebibliography}{9}
\bibitem{1-al-1}
\Aue{Passino, K.\,M., and S. Yurkovich}.  1998. \textit{Fuzzy control}. Addison 
Wesley Longman, Inc. 502~p.
\bibitem{2-al-1}
\Aue{Goppov, S.\,I.} 2011. Razrabotka nechetkoy modeli obshchego reytinga 
studenta na osnove ierarkhicheskoy bazy [The fuzzy model of general student 
rating based on hierarchical database]. \textit{Mat-ly III~konf. ``Studencheskiy 
nauchnyy forum''}  [3rd~``Students Scientific Forum'' Conference Proceedings]. 
Saint Petersburg: Saint Petersburg State University of Economics. 1--6.
\bibitem{3-al-1}
\Aue{Bereza, A.\,N., and E.\,A.~Ershova}. 2011. Podderzhka pri\-nya\-tiya 
resheniya pri planirovanii nabora abiturientov v~vuze na osnove nechetkikh 
modeley [Decision support in planning a set of entrants into university based on 
fuzzy models]. \textit{Izvestiya Yuzhnogo federal'nogo universiteta. 
Tekhnicheskie nauki} [Southern Federal University Izvestiya, Engineering
Sciences] 7:131--136.
\bibitem{4-al-1}
\Aue{Saleh, I., and S.~Kim}. 2009. A~fuzzy system for evaluating students' 
learning achievement. \textit{Expert Syst. Appl.} 36:6236--6243. 
\bibitem{5-al-1}
\Au{Stathacopoulou, R., G.\,D. Magoulas, M.~Grigoriadou, and 
M.~Samarakou}. 2005. Neuro-fuzzy knowledge processing in intelligent learning 
environments for improved student diagnosis. \textit{Inform. Sciences}  
170:273--307. 



\bibitem{6-al-1}
\Aue{Igel, Ch., and M.~Husken}. 2003. Empirical evaluation of the improved 
Rprop learning algorithms. \textit{Neurocomputing} 50:105--123.

\bibitem{7-al-1}
\Aue{Mamdani, E.\,H.} 1977. Application of fuzzy logic to approximate 
reasoning using linguistic synthesis. \textit{Fuzzy Set. Syst.} 26:1182--1191.

\bibitem{8-al-1}
\Aue{Alekseychuk, A.\,S.} 2016. Programmnaya sistema dlya obucheniya 
studentov ekonomicheskikh spetsial'nostey v~rezhime veb-konferentsii [Software 
system for teaching students of economic specialties in web conference form]. 
\textit{Nauchnyy al'manakh. Vypusk~21: Mat-ly XII~konf. molodykh uchenykh 
i~studentov ``Innovatsii v~ekonomike i~menedzhmente v~aerokosmicheskoy 
promyshlennosti''} [Scientific almanac. Vol.~21: 12th Conference of Young 
Scientists and Students ``Innovations in Economics and Management in the 
Aerospace Industry'' Proceedings]. Moscow. 190--198.
\end{thebibliography}

 }
 }

\end{multicols}

\vspace*{-3pt}

\hfill{\small\textit{Received October 7, 2016}}
  
  \Contr
  
  \noindent
  \textbf{Alekseychuk Andrei S.} (b.\ 1985)~--- PhD student, Faculty of Applied Mathematics 
and Physics, Moscow Aviation Institute (National Research University), 4~Volokolamskoye 
Highway, A-80, GSP-3, Moscow 125993, Russian Federation; \mbox{alexejchuk@gmail.com}
  
  \vspace*{3pt}
  
  \noindent
  \textbf{Panteleev Andrei V.} (b.\ 1955)~--- Doctor of Science in physics and mathematics; 
professor, Head of Department, Faculty of Applied Mathematics and Physics, Moscow Aviation 
Institute (National Research University), 4~Volokolamskoye Highway, A-80, GSP-3, Moscow 
125993, Russian Federation; \mbox{avpanteleev@inbox.ru}
  
\label{end\stat}


\renewcommand{\bibname}{\protect\rm Литература} 
  