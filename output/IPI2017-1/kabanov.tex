\renewcommand{\figurename}{\protect\bf Figure}
\renewcommand{\tablename}{\protect\bf Table}

\def\stat{kabanov}


\def\tit{ON UNIQUENESS OF CLEARING VECTORS REDUCING~THE~SYSTEMIC RISK}

\def\titkol{On uniqueness of clearing vectors reducing the systemic risk}

\def\autkol{Kh.\ El Bitar,  Yu.~Kabanov, and~R.~Mokbel}

\def\aut{Kh.\ El Bitar$^1$,  Yu.~Kabanov$^2$, and~R.~Mokbel$^3$}

\titel{\tit}{\aut}{\autkol}{\titkol}

%{\renewcommand{\thefootnote}{\fnsymbol{footnote}}
%\footnotetext[1] {The 
%research of Yuri Kabanov was done under partial financial support   of the grant 
%of  RSF No.\,14-49-00079.}}

\renewcommand{\thefootnote}{\arabic{footnote}}
\footnotetext[1]{Laboratoire de Math$\acute{\mbox{e}}$matiques, Universit$\acute{\mbox{e}}$ de 
Franche-Comt$\acute{\mbox{e}}$, 16~Route de Gray, 25030 \mbox{Besan{\!\ptb{\c{c}}}on}, CEDEX, France, 
\mbox{khalilbitar\_aw@hotmail.com}}
\footnotetext[2]{Laboratoire de 
Math$\acute{\mbox{e}}$matiques, Universit$\acute{\mbox{e}}$ de
 Franche-Comt$\acute{\mbox{e}}$, 16~Route de Gray, 25030 
\mbox{Besan{\!\ptb{\c{c}}}on}, CEDEX, France; 
Institute of Informatics Problems, Federal Research 
Center ``Computer Science and Control'' of the Russian Academy of Sciences, 
44-2~Vavilov Str., Moscow 119333, Russian Federation; 
National Research University 
``MPEI,'' 14~Krasnokazarmennaya Str., Moscow, 111250, Russian Federation, 
\mbox{Youri.Kabanov@univ-fcomte.fr}}
\footnotetext[3]{Laboratoire de 
Math$\acute{\mbox{e}}$matiques, Universit$\acute{\mbox{e}}$ de 
Franche-Comt$\acute{\mbox{e}}$, 
16~Route de Gray, 25030  \mbox{Besan{\!\ptb{\c{c}}}on}, CEDEX, France,
\mbox{ritamokbel@hotmail.com}}

\index{El Bitar Kh.}
\index{Kabanov Yu.}
\index{Mokbel R.}
\index{Эль Битар Х.}
\index{Кабанов Ю.}
\index{Мокбель Р.}


\vspace*{-12pt}

\def\leftfootline{\small{\textbf{\thepage}
\hfill INFORMATIKA I EE PRIMENENIYA~--- INFORMATICS AND APPLICATIONS\ \ \ 2017\ \ \ volume~11\ \ \ issue\ 1}
}%
 \def\rightfootline{\small{INFORMATIKA I EE PRIMENENIYA~--- INFORMATICS AND APPLICATIONS\ \ \ 2017\ \ \ volume~11\ \ \ issue\ 1
\hfill \textbf{\thepage}}}




\Abste{Clearing of financial system, i.\,e., of a~network of interconnecting banks, is 
a~procedure of simultaneous repaying debts to reduce their total volume. The 
vector whose components are  repayments of each bank
is called clearing vector.  In  simple models  considered  by Eisenberg and Noe 
(2001) and, independently,  by Suzuki (2002), it was shown that
the  clearing  to the minimal value of debts  accordingly to natural rules  can 
be formulated as fixpoint problems.
The existence
of their solutions, i.\,e., of clearing vectors,  is rather straightforward and can 
be obtained by a~direct reference to the Knaster--Tarski or Brouwer theorems.  
The uniqueness of clearing vectors is a~more delicate problem which was solved 
by Eisenberg and Noe  using a~graph structure of the financial network.  
The uniqueness  results have been proved in two generalizations of the  Eisenberg--Noe model:  
in the Elsinger model with seniority of liabilities and in the Amini--Filipovic--Minca 
type model with several
types of illiquid assets whose firing sale has a~market impact.}

\KWE{systemic risk;  financial networks; clearing; Knaster--Tarski 
theorem; Eisenberg--Noe model; debt seniority; price impact}

\DOI{10.14357/19922264170110} 

\vspace*{7pt}


\vskip 12pt plus 9pt minus 6pt

      \thispagestyle{myheadings}

      \begin{multicols}{2}

                  \label{st\stat}


\section{Introduction}

\noindent
To explain the clearing problem, let us start with the simplest example of 
a~financial
system with two agents each having in a~cash 10 dollars. The first agent gets 
from the second a~credit of~1M  dollars, the second gets from the first  a~credit 
of~1~M and 1~dollars. Apparently, as a~result, both agents have a~huge liabilities with 
respect to each other. Of course, the agents can be asked to reduce their 
liabilities by reimbursing credits partially (e.\,g., to the levels~0.5~M and 
0.5~M\;+\;1 in liabilities and~10~dollars both in cash) or completely, with zero 
liabilities and cash reserves~11 and~9~dollars, respectively. Intuitively, the 
situation where the liability is reduced (i.\,e., the system is cleared) seems to 
be less risky: if one of the agents became bankrupt and only the percentage of the 
huge debt value  can be reimbursed, the creditor's losses will be also huge. For 
complex financial systems involving large numbers of agents
with chains of borrowing,  the clearing problem, that is, the reduction of 
absolute values by reimbursement, looks much more complicated.
{ %\looseness=1

}

In the influential paper~\cite{Eisenberg-Noe} published in 2001, Eisenberg and 
Noe suggested a~clearing procedure in the model describing a~financial system 
composed by~$N$~banks (under ``banks''  can be understood  various financial 
institutions); a~more general model was introduced independently at the same 
time by Suzuki~\cite{Suzuki}.   The assets of the bank are cash and interbank 
exposures which are, in turn, liabilities for its debtors.  The clearing 
consists in simultaneous paying all debts. Each bank pays to its counterparties 
the debts \textit{pro rata} of their relative volume using its cash reserve and 
money collected from the credited banks. The rule is: either all debts are payed 
in full or the zero level of the equity is attained and the bank defaults. The 
totals reimbursed by banks form an $N$-dimensional clearing vector. A~remarkable 
feature is that this vector is a~fixed point of a~monotone mapping of a~complete 
lattice into itself and its existence follows immediately from the 
Knaster--Tarski  theorem, a~beautiful and fairy simple result which proof needs only 
a~few lines of arguments~\cite{Tarski}. The uniqueness of the clearing vector is a~more 
delicate 
result involving the graph structure of the system.

The ideas of the  Eisenberg--Noe paper happened to be very fruitful and their 
model
was generalized in many directions having not only financial importance but 
posing  interesting mathematical questions. One of them is the question on 
uniqueness of clearing vector or   equilibrium  on financial market.

The first theorem provides a~new sufficient condition for the Elsinger model of 
clearing with debts priority structure. This model is given by a~set of 
liability matrices corresponding to each seniority. The idea of the present approach is 
to use the largest clearing vector which always exists to construct a~new 
liability matrix generating a~graph structure with which one can work in 
a~similar way as in the Eisenberg--Noe model.
The second theorem deals with the uniqueness of  equilibrium in a~clearing  
model with several illiquid assets and a~market impact.  In the presence of 
several illiquid assets,  the banks are faced the choice of  asset selling 
strategies. The proportional scheme of selling similar to that in the 
paper by Cont--Wagalath~\cite{Cont-Wag} has been used 
leaving game-theoretical versions for 
future studies.  In the case of one illiquid asset, the
obtained result is close to that  
of the study by Amini--Filipovic--Minca~\cite{AFM}, but the present definition of the 
equilibrium is different (but equivalent).

The structure of the note is as follows. In the introductory section~2, 
 the general principle and results are discussed briefly in the framework of the 
Eisenberg--Noe model. To facilitate the comparison with further development, 
also, short proofs are provided.
In section~3, a~uniqueness of the clearing vector for the Elsinger model 
where senior  liabilities should be reimbursed before the juniors ones. Section~4 
contains
the sufficient condition  for the uniqueness of the equilibrium in the model
where clearing requires selling of the illiquid assets with price impact.  
Economically speaking, it is  oriented to the recovering of the market  after 
fire sales.  For the reader convenience,  in Appendix, 
a~short information about the Knaster--Tarski theorem adapted to 
the present authors' needs is provided.



\noindent
\textbf{Notations.}\ The partial ordering in~$\mathbb{R}^n$ and its 
subsets  induced by the cone~$\mathbb{R}^n_+$ is denoted by $\ge$. In other words, the inequality $y\ge x$ 
is understood componentwise. Also, the symbols $x\wedge y$ and $x\vee y$ mean, 
respectively, the componentwise minimum and maximum, $x^+:=x\vee 0$ and\linebreak 
$x^-:=(- x)^+$.
The notation $[x,z]$ is used for the order interval, i.\,e.,
$[x,z]=\{y\in \mathbb{R}^n:\ x\le y\le z \}$.
If $A\subseteq [x,z]$, then $\inf A$ is the unique element $\underline y\in 
[x,z]$ such
that $\underline y\le y$ for all $y\in A$ and for any $\tilde y$  such that 
$\tilde y\le y$ for all $y\in A$, one has $\tilde y\le \underline y$, that 
is, the component $\underline y^i=\inf \{y^i:\ y\in A\}$ for  $i=1,\dots,n$.

The matrix notations are used where the vectors are columns, $'$ is the symbol of 
transpose, and\linebreak  ${\bf 1}':=(1,\dots,1)$ (the dimension of the vector is supposed 
to be clear from the context).

\vspace*{-9pt}

\section{The Eisenberg--Noe Model}

\noindent
In~\cite{Eisenberg-Noe}, Eisenberg and Noe investigated the model 
describing a~financial system composed of $N$ banks (under ``banks"  can be 
understood  various financial institutions). In the aggregate oversimplified  
form, the balance sheet of the bank $i$ can be split into two parts: assets and 
liabilities. The assets are of two types:  interbank assets (exposures)~$\tilde X^i$ 
and cash~$e^i$.  The liabilities are: interbank debts (liabilities)~$\tilde L^i$ 
and the equity~$C ^i$ (or proper capital reserve) equalizing the two sides 
of the balance sheet:
\vspace*{2pt}

\noindent
$$
e^i+\tilde X^i= \tilde L^i + C^i\,.
$$

\vspace*{-2pt}

\noindent
All these values are assumed to be greater or equal to zero. The condition that 
$C^i\ge 0$ means that the bank is solvent.

More detailed balance sheet provides the information on the values  of 
liabilities of the bank  $i$ to the bank $j$, namely,  vectors 
$(L^{i1},\ldots,L^{iN})'$ of liabilities and  $(X^{i1},\ldots,X^{iN})$ of exposures.
 With this, one 
has  $\tilde X^i=X^{i1}+\cdots+X^{iN}$ and $\tilde L^i =L^{i1}+\cdots+L^{iN}$.

The matrix $L=(L^{ij})$ with positive entries and zero diagonal defines the
total interbank exposures. Since the value of the exposure of~$i$ to~$j$ is the 
value of the liability of~$j$ to~$i$, one has that $L'=X$.  So, 
the matrix $L$ and the vector~$e$ give a~description of a~financial system in 
this model.

Put

\vspace*{-3pt}

\noindent
$$
\Pi^{ij}:=
\begin{cases}
\fr {L^{ij}}{\tilde L^i}=\displaystyle \fr {L^{ij}}{\sum\nolimits_j L^{ij}} 
&\ \mbox{if } \tilde L^i\neq 0\,; \\
\delta^{ij} &\  \mbox{otherwise}
\end{cases}
$$
where the Kronecker symbol $\delta^{ij}=0$ for $i\neq j$ and $\delta^{ii}=1$.
Then,~$\Pi^{ij}$  describes the proportion of the value debtor $i$ due to the 
creditor~$j$ of the total interbank debt of~$i$; $\Pi=(\Pi^{ij})$  is called 
relative liabilities matrix. Note that in this definition, to get a~stochastic 
matrix $\Pi$, we deviate from~\cite{Eisenberg-Noe} where $\Pi^{ii}=0$ when 
$L^i= 0$.

%As an example consider the simplest system with two banks where 
%$L^{12}=L^{21}+\varepsilon$ where $\e<0$ can be thought small with respect to~$L^{21}$.  
%After paying debts in the cleared system the matrix of liabilities will have the 
%entries $L^{12}_{c}=\e$, $L^{21}_c=0$. That is the values of debts are reduced 
%and so are eventual values of losses in the case of defaults of a~partner.

In general, financial system   may have a~complicated structure with cyclical 
interdependences and  banks may have large exposures within cycles. To reduce 
them, one can impose a~clearing mechanism satisfying several natural 
requirements: limited liability and proportionality. Formally,  this  leads to 
the concept of a~\textit{clearing payment vector} $p^*\in \prod_i[0,\tilde L^i]$ 
satisfying the following properties:
\begin{itemize}
\item[$a.$] \textit{Limiting liability}. For every $i$,
$$
p_i^*\le e^i+\sum\limits_j\Pi^{ji}p_j^*\,.
$$

\item[$b.$] \textit{Absolute priority.} For every $i$, either $p^*_i=\tilde L^i$, or
$$
p_i^*= e^i+ \sum\limits_j\Pi^{ji}p_j^*.
$$
\end{itemize}
One may think that the  central clearing authority forces  each bank to make 
a~``fair'' payment of debts in such\linebreak\vspace*{-12pt}

\pagebreak

\noindent
 a~way that, having  the total payment~$p_i^*$, 
the bank~$i$ remains solvent and  pays to~$j$ the fraction $p_i^*\Pi^{ij}$ in 
such a~way that either its total debts are paid,  or all the resources are 
exhausted.

Alternatively, the conditions~$a$ and~$b$ can be written in the following way:
\begin{equation}
\label{p^*} p^*=\min \left\{ e+\Pi' p^*, \tilde L\right\}
\end{equation}
where the minimum is understood in the componentwise sense, i.\,e., accordingly to 
the partial ordering defined by the cone~${\mathbb{R}}^N_+$.

The main result of Eisenberg and Noe asserts that the set of clearing vectors is 
nonempty. Moreover, there are the minimal and the maximal clearing vectors,  
denoted here~$\underline p$ and~$\bar p$, respectively.  This assertion follows
immediately from the Knaster--Tarski fixed point theorem: the monotone mapping 
$f:p\mapsto (e+\Pi'p)\wedge \tilde L$ of a~complete lattice $[0,\tilde L]$ into 
itself has the largest and the smallest fixed points (for 
details, see section~5). The set $[0,\tilde L]$ is convex and compact and~$f$ is a~continuous 
mapping. So, the existence of its fixed point follows also from the classical  
Brouwer theorem.

Using the obvious identity $(x-y)^+=x -x\wedge y$, one can rewrite 
Eq.~(\ref{p^*}) in the following equivalent form:
\begin{equation}
\label{alt1}
\left(e+\Pi'p^*-\tilde L\right)^+=e+\Pi'p^*-p^*
\end{equation}
where the left-hand side is the equity vector of the system after clearing.

%After clearing by an arbitrary clearing (outflow) vector $p^*$ the equity 
%vector of the system  is
%$$(e+\Pi'p^*-\tilde L)^+=e+\Pi'p^*-p^*; $$
%this equality is nothing but the equation (\ref{p^*}) written in an equivalent 
%form.

An important but simple observation: {\it the equity (after clearing) does not 
depend on the clearing vector}.
Indeed,~$\Pi$~being  a~stochastic matrix, ${\bf 1}'\Pi'={\bf 1}'$.  Therefore, 
multiplying  the above representation~(\ref{alt1}) from the left by~${\bf 1}'$, 
one gets that  the sum of equities
$$
{\bf 1}'\left(e+\Pi'p^*-\tilde L\right)^+={\bf 1}'e
$$
is equal to the sum of the initial cash reserves, that is, invariant with respect 
to the choice of the clearing vector.
On the other hand, by monotonicity, one has that
$$
\left(e+\Pi'p^*-\tilde L\right)^+\le \left(e+\Pi'\bar p-\tilde L\right)^+.
$$
If the both sides here are not equal, then
$$
{\bf 1}'\left(e+\Pi'p^* - \tilde L\right)^+< {\bf 1}'\left(e+\Pi'\bar p-\tilde L\right)^+$$
in contradiction with the invariance of the  total of equities.

\smallskip

\noindent
\textbf{Sufficient condition for the uniqueness of the clearing vector.}
As in~\cite{Eisenberg-Noe}, let us assume for simplicity that $\tilde L^i>0$ 
for all~$i$.

For a~stochastic matrix~$\Pi$,  we say that
$I\subseteq  \{1,\ldots,N\}$ is
a~($\Pi$-)\textit{surplus set} if $\Pi^{ij}=0$ for all $i\in I$, $j\in I^c$, 
and~$\sum_{j\in I}e^j>0$.

\columnbreak

Recall that~$j$ is the creditor of $i$ if $\Pi^{ij}>0$ (i.\,e., $\Pi^{ij}>0$); in 
this case, let us use, as in the  theory of Markov chains or in the graph 
theory,  the notation $i\to j$.

Let us denote by $o(i)$ {\it the orbit of $i$} that is the set of all~$j$ for which 
there is a~directed path 
$$
i\to i_1\to i_2\to\cdots\to j\,,$$ 
i.\,e.,  $o(i)$ is the set 
of all direct or indirect creditors of~$i$.

Note that the orbit $o(i)$ with $\sum_{j\in I}e^j>0$ is a~surplus set. Indeed,
if\ $\Pi^{jj'}>0$ for some $j\in o(i)$, $j'\notin o(i)$, i.\,e.,  $j\to j'$, then 
there is
a~path 
$$i\to i_1\to i_2\to\cdots\to j \to  j'\,.
$$


\noindent
\textbf{Lemma~1.}\
%\label{equity>0}
 \textit{Suppose that the market is cleared by a~vector $p^*\in [0,\tilde L]$. Let~$I$ 
be a~surplus set.  Then, at least one node of~$I$ has a~strictly positive equity 
value}.

\textit{In particular,
any orbit~$o(i)$ with $\sum_{j\in o(i)}e^j>0$ has an element with strictly  
positive equity value}.

\smallskip

\noindent
P\,r\,o\,o\,f\,.\ \  Multiplying the identity~(\ref{alt1}) by~${\bf 1}'_I$ and noticing 
that
$({\bf 1}'_I\Pi')^i=1$ for $i\in I$,
one obtains that
$$
{\bf 1}'_I \left(e+\Pi'p^*-\tilde L\right)^+\ge {\bf 1}'_I e>0
$$
implying the claim.~$\square$

\smallskip

A financial system is called \textit{regular} if for  every~$i$, the orbit~$o(i)$ is 
a~surplus set.

\smallskip

\noindent
\textbf{Theorem~1.}\
%\label{uni1}
\textit{Suppose that the financial system is regular.
Then}, $\underline p=\bar p$.

\smallskip

\noindent
P\,r\,o\,o\,f\,.\ \  Suppose that~$\underline p$ and~$\bar p$ are not equal, i.\,e., 
$\underline p\le \bar p$ but for some~$i$, one 
has the strict inequality  $\underline p^i<\bar p^i$.
Denote by~$C$ the vector of equities (it is common for all clearing vectors).
By assumption, the orbit~$o(i)$ is a~surplus set and by Lemma~1, it 
contains an element~$m$ with the equity value $C^m>0$. By definition of the 
orbit, there is a~path $i\to i_1\to \cdots \to m$ and one may assume without loss of 
generality that in this path,~$m$ is  the first node with strictly positive 
equity value.

First, let us prove that  one may consider only the case where the path
consists of one step,  i.\,e., $i\to m$.  To this end, let us check that
$\underline p^{i_1}<\bar p^{i_1}$ if $i_1\neq m$. In other words, the property 
that $\underline p^i\neq \bar p^i$ propagates along the path.

Suppose that $\bar p^{i_1}< \tilde L^{i_1}$. Then, also, $\underline p^{i_1}< 
\tilde L^{i_1}$.  In such a~case,

\vspace*{3pt}

\noindent
$$
 \underline p^{i_1}=e^{i_1}+ \sum\limits_j\Pi^{ji_1}\underline p^j\,, \enskip \bar 
p^{i_1}=e^{i_1}+\sum\limits_j\Pi^{ji_1}\bar p^j
$$
and one has  that

\vspace*{3pt}

\noindent
$$
\bar p^{i_1}-\underline p^{i_1}=\sum\limits_j\Pi^{ji_1}\left(\bar p^j-\underline p^j\right)>0
$$

\vspace*{-6pt}

\noindent
because   $\Pi^{ii_1}>0$, that is, $\underline p^{i_1}<  \bar p^{i_1}$. This 
inequality also holds trivially, if
$\bar p^{i_1}= \tilde L^{i_1}$ but $\underline p^{i_1}< \tilde L^{i_1}$.
 The remaining\linebreak\vspace*{-12pt}
 
 \pagebreak
 
 \noindent
  case where
$\underline p^{i_1}=\bar p^{i_1}=\tilde L^{i_1}$ is excluded as it is supposed that 
$C^{i_1}=0$.  Indeed, according to~(\ref{alt1}),  this leads to the equalities:
$$
e^{i_1}+ \sum\limits_j\Pi^{ji_1}\bar p^j - \tilde L^{i_1}=0\,;\enskip
e^{i_1}+  \sum\limits_j\Pi^{ji_1}\underline p^j - \tilde L^{i_1}=0\,,
$$
implying the identity
$$
\sum\limits_j\Pi^{ji_1}\left(\bar p^j-\underline p^j\right)=0
$$
which cannot be true since in the above sum, the term corresponding to $j=i$ is 
strictly positive.

So, it is sufficient to consider only one-step case. Since $C^m>0$, one has the 
representations:
\begin{align*}
C^m&=e^{m}+ \sum\limits_j\Pi^{jm}\underline p^j - \tilde L^{m}\,; \\
C^m&=e^{m}+ \sum\limits_j\Pi^{jm}\bar p^j- \tilde L^{m}\,.
\end{align*}
As above, one again obtains the impossible equality:
$$
\sum\limits_j\Pi^{jm}\left(\bar p^j-\underline p^j\right)=0\,.
$$
Therefore, the  assumption $\underline p^i<\bar p^i$ leads to a~contradiction. 
The
uniqueness of clearing vector is proven.~$\square$


\smallskip

\noindent
\textbf{Remark~1.}\
The above  theorem reveals that the problem to find a~clearing 
vector is ill-posed. Indeed, adding an infinitesimally small amount $\varepsilon>0$ 
(say,  one cent) to the initial endowments leads to a~unique clearing vector. Similar 
effect will have small increase in liabilities. One can think that the ``true'' 
liability matrix has all elements strictly positive and that in the model matrix, zero 
elements appeared because liabilities are neglected.
These phenomena are related to the ill-posedness of the spectral problem for 
stochastic matrices. Another question is which clearing vector is natural.


\smallskip


The above proof  is rather straightforward and uses graph-theoretical language.  
One can get another one  appealing to the contraction property of the mapping 
$f:p\mapsto (e+\Pi'p)\wedge \tilde L$ defined on the set $[0,\tilde L]$ equipped 
with $l_1$-distance $|p-\tilde p|_1$.

\smallskip

\noindent
\textbf{Proposition.}\
For every $p,\tilde p\in [0,\tilde L]$
\begin{equation*}
%\label{non-exp}
\left\vert f(p)-f(\tilde p)\right\vert_1\le \left\vert\Pi' (p-\tilde p)
\right\vert_1\le \left\vert p-\tilde p\right\vert_1\,.
\end{equation*}
Moreover, the first relation above is the equality if and only if the
union of subsets $A:=\{i:\ (\Pi'p)^i=(\Pi'\tilde p)^i\}$ and $B:=\{i:\ 
(\Pi'p)^i,(\Pi'\tilde p)^i\le \tilde L^i-e^i\}$ is the set of indices 
$\{1,\dots, N\}$.

\smallskip
%Moreover, if for each $i$ the sum $\sum_{j\in o(i)} e^i>0$, then the mapping 
%$f$ is a~contraction %on the set ${\rm Fix}_f$, i.e. the above inequality is 
%strict when the fixed points $p\neq \tilde p$.

\noindent
P\,r\,o\,o\,f\,.\ \ Using the elementary inequality $|a\wedge c-b\wedge c|$\linebreak $\le |a-b|$ 
which holds as  the
equality if and only if when\linebreak $a=b$ or $a,b\le c$, one obtains that
$|f(p)-f(\tilde p)|_1$\linebreak $\le |\Pi'p-\Pi'\tilde p|_1$
where the equality holds if and only if for every~$i$, one has 
$(\Pi'p)^i=(\Pi'\tilde p)^i$ or
$(\Pi'p)^i,(\Pi'\tilde p)^i$\linebreak $\le \tilde L^i-e^i$. Since $|\Pi'y|_1\le 
|\Pi'|_1|y|_1$ and $|\Pi'|_1=1$, one has the claim.~$\square$

\smallskip

Let us consider  the case where the matrix~$\Pi$ is irreducible. Suppose that 
${\bf 1}'e>0$ and~$p$ and~$\tilde p$ are two different fixed points of the 
mapping~$f$. According to above proposition,
$$
\sum\limits_{j\in B}\Pi^{ji}\left(p^j-\tilde p^j\right)=p^ i-\tilde p^i\,, \enskip i\in B\,.
$$
This means that  the nonzero vector with the coordinates $p^ i-\tilde p^i$, 
$i\in B$, is a~left eigenvector of the matrix
$(\Pi^{ij})_{i,j\in B}$ corresponding to unit eigenvalue. This is possible only 
if the latter matrix coincides with~$\Pi$. Thus, $p=f(p)=e+\Pi'p$. Since  
${\bf 1}'\Pi'p={\bf 1}'p$, one gets that ${\bf 1}'e=0$
which is a~contradiction.  Using the decomposition of the matrix~$\Pi$ on the 
irreducible component, one gets that  the clearing vector  is unique if for any 
irreducible component, there is a~node with strictly positive initial endowment.



\section{The Elsinger Model}

\noindent
In the present paper,  a~simplified version of the Elsinger model
introduced in~\cite{Elsinger2011}, where the interbank debts may be 
senior and junior, is considered. In this model, the system of~$N$ banks is described by the 
vector
of cash reserves and by~$M$~matrices $L_1=(L^{ij}_1), \ldots, L_M=(L^{ij}_M)$ 
representing the hierarchy of liabilities with decreasing seniority,  that is, 
the element~$L^{ij}_1$ represents the debt of the bank~$i$ to the bank~$j$ of the 
highest seniority, etc.,  $\sum_jL^{ij}_S$ is the total of  debts of the bank~$i$ 
of the seniority~$S$.

The relative liabilities are defined by  the matrix~$\Pi_S$ with
$$
\Pi_S^{ij}=\fr {L_S^{ij}}{\tilde L_S^i}=\fr {L_S^{ij}}{\sum\nolimits_j L_S^{ij}}\,.
$$
The clearing procedure requires the complete reimbursement of the debts starting 
from the highest priority and for each seniority level, the distribution is 
proportional
to the volume of debts of this seniority. For the bank~$i$, let us denote  by $p^i_S$ 
the value distributed to cover the debts of the seniority~$S$. So, the clearing 
can be described by the set of vectors~$p_S$, $S=1,\ldots, M$, which can be 
considered as a~``long'' vector from~$(\mathbb{R}^N)^M$  satisfying the system of 
equations:
\begin{equation*}
p_{1}^{i}=\min \left\{e^i+\sum\limits_S \sum\limits_j\Pi_S^{ji}p_{S}^{j}, \tilde L_1^i 
\right\}\,;
\end{equation*}
\begin{align*}
p_{S}^{i}&=\min\left\{\left(e^i+\sum\limits_S \sum\limits_j\Pi_S^{ji}p_{S}^{j}-
\sum\limits_{r<S}\tilde 
L_r^i\right)^+, \tilde L_S^i \right\}\,,  \\
&\hspace*{57mm}1<S\le M\,.
\end{align*}
In a~vector form, these equations can be written as follows:

\vspace*{-4pt}

\noindent
\begin{multline}
\label{SM}
p_{S}^{}=\left(e+\sum\limits_S \hspace*{-1.2pt}
\Pi_S'p_{S}-\sum\limits_{r<S}\hspace*{-1.2pt}\tilde L_r\right)^+\wedge  
\tilde  L_S\,,  \\ S=1,\ldots,M\,.
\end{multline}
It is clear that for the partial ordering in~$(\mathbb{R}^N)^M$ induced by the 
cone~$(\mathbb{R}^N_+)^M$, the function

\vspace*{-4pt}

\noindent
\begin{multline*}
\left(p_1,\ldots,p_M\right)\mapsto \left(
\left(e+\sum\limits_S \Pi_S'p_{S}^* \right)^+\wedge \tilde L_1 
,\ldots\right.\\
\left.\ldots,\left(e+\sum\limits_S \Pi_S'p_{S}^*-\sum\limits_{r<M}\tilde L_r\right)^+ 
\wedge L_M 
\right)
\end{multline*}
is a~monotone mapping of the order interval 
$$
[0,\tilde L_1] \times\cdots\times 
[0,\tilde L_M]\subset (\mathbb{R}^N)^M
$$ 
into itself.
 Thus, according to the Knaster--Tarski theorem, the set of fixed points of this 
mapping, i.\,e., the solutions of Eq.~(\ref{SM}), is nonempty and has the 
maximal and the minimal elements.

In the case of liabilities of different seniority after clearing by the vector 
$p\in (\mathbb{R}^N)^M$,  the equity vector $C\in \mathbb{R}^N$ has the form:
$$
C=\left(e+\sum\limits_S \Pi_S'p_{S}-\sum\limits_S \tilde L_S\right)^+\,.
$$

%\smallskip

\noindent
\textbf{Lemma~2.}\
\textit{The equity vector does not depend on the clearing vector}.

\vspace*{2pt}

\noindent
P\,r\,o\,o\,f\,.\ \  Note that
$$
\left(e+\sum\limits_S\Pi'_Sp_S\right)\wedge \sum\limits_S \tilde L^i_S=\sum\limits_S p_S\,.
$$
Therefore,
$$
\left(e+\sum\limits_S \Pi_S'p_{S}-\sum\limits_S \tilde L_S\right)^+=
e+\sum\limits_S \Pi_S'p_{S}-\sum\limits_S  p_{S}\,.
$$
With this identity, the reasoning is analogous to that with a~single seniority 
class.~$\square$

\vspace*{2pt}

The aim of this section is to provide a~sufficient condition for the uniqueness 
of clearing vector using a~specific graph structure induced by the matrices~$\Pi_S$.

For a~given clearing vector~$p$, let us define the \textit{default index}~$d^i$ of the 
node~$i$ as the smallest~$r$  such that
$$
\bar p_r^i=e^i+ \sum\limits_S \sum\limits_j\Pi_S^{ji}\bar p_{S}^j-\sum\limits_{r'< r}\tilde 
L_{r'}^{i}\,.
$$
In another words,~$d^i$ is the lowest seniority for which the bank equity after 
clearing is equal to zero. Define the matrix $\Delta=\Delta(p)$ by putting 
$$
\Delta^{ij}=
\begin{cases}
1 &\ \mbox{if\ \ } \Pi_{d(i)}^{ij}>0\,;\\
0 &\ \mbox{otherwise}.
\end{cases}
$$

%\columnbreak

\noindent
Let us use 
the notation $i\leadsto j$ if $\Delta^{ij}=1$ and  denote by $O(i)$ \textit{the 
$\Delta $-orbit of $i$} that is the set of all~$j$ for which there is 
a~directed path $i\leadsto i_1\leadsto i_2\leadsto\cdots\leadsto j$.

\vspace*{2pt}

\noindent
\textbf{Theorem~2.}\
\textit{Suppose that for the clearing vector $\bar p$, any $\Delta $-orbit is a~surplus 
set.
Then, the clearing vector is unique}.

\vspace*{2pt}

\noindent
P\,r\,o\,o\,f\,.\ \  By definition, the default index
$$
d^i:=\min\left\{r:\ \bar p_r^i=e^i+ \sum\limits_S \sum\limits_j\Pi_S^{ji}
\bar p_{S}^j-\sum\limits_{r'<  r}\tilde L_{r'}^{i}\right \}\,.
$$
It follows that $\bar p_r^i=0$; hence,  $\underline p_r^i=0$ for every $r>d^i$.
Suppose that
$\underline p_{d^i}^i<\bar p_{d^i}^i$ and consider a~path 
$$
i\leadsto  i_1\leadsto i_2\leadsto\cdots \leadsto m
$$ 
ending up at the node with strictly  positive equity value.

First, let us show that at least for one seniority $\underline p^{i_1}_S<\bar 
p^{i_1}_S$.

Let $r':=d^{i_1}$.  By definition, one has: 
$$
\bar p^{i_1}_r=\begin{cases}
\tilde L^{i_1}_r\,, & r\le r'\,;\\
0\,,  & r>r'\,.
\end{cases}
$$
 The claim 
holds, if  $\underline p^{i_1}_r<\tilde L^{i_1}_r$
for some $r<r'$. Thus, it remains to consider only the case where $\underline 
p^{i_1}_r=\bar p^{i_1}_r = \tilde L^{i_1}_r$
for all $r<r'$ and prove that  $\underline p^{i_1}_{r'}<\bar p^{i_1}_{r'}$.
One has the alternative: either $\underline p^{i_1}_{r'}<\bar p^{i_1}_{r'}\le  
\tilde L^{i_1}_r$ (what is needed), or
$\underline p^{i_1}_{r'}=\bar p^{i_1}_{r'}\le  \tilde L^{i_1}_r$. The second 
case is impossible, since the equalities

\noindent
\begin{align*}
\bar p^{i_1}_{r'}&=e^{i_1}+ \sum\limits_S \sum\limits_j\Pi_S^{ji_1}\bar p_{S}^j-
\sum\limits_{r<  r'}\tilde L_{r}^{i_1}\,;\\
\underline p^{i_1}_{r'}&=e^{i_1}+ \sum\limits_S 
\sum\limits_j\Pi_S^{ji_1}\underline p_{S}^j-
\sum\limits_{r< r'}\tilde L_{r}^{i_1}
\end{align*}
imply that

\noindent
\begin{multline*}
\bar p^{i_1}_{r'}-\underline p^{i_1}_{r'}=\sum\limits_S \sum\limits_j\Pi_S^{ji_1}
\left(\bar  p_{S}^j-\underline  p_{S}^j\right)\\
{}\ge \Pi_{d^i}^{ii_1}
\left(\bar p_{d^i}^i-\underline   p_{d^i}^i\right)>0\,.
\end{multline*}
This is a~contradiction.

\pagebreak

The above argument reduces the problem to the case $i\leadsto m$ and the node~$m$ 
has a~strictly positive equity.  The equity~$C^m$ does not depend on the 
clearing vector.  Therefore,

\noindent
\begin{align*}
C^m&=e^{m}+ \sum\limits_S \sum\limits_j\Pi_S^{jm}\bar p_{S}^j-
\sum\limits_{S}\tilde L_{S}^{m}\,;\\
C^m&=e^{m}+ \sum\limits_S \sum\limits_j\Pi_S^{jm}\underline p_{S}^j-
\sum\limits_{S}\tilde L_{S}^{m}\,.
\end{align*}


\noindent
It follows that
$$
0=\sum\limits_S \sum\limits_j\Pi_S^{jm}\left(\bar p_{S}^j-\underline p_{S}^j\right)\ge 
\Pi_{d^i}^{im}\left(\bar p_{d^i}^i-\underline  p_{d^i}^i\right)>0\,.
$$
This contradiction shows that $\underline p=\bar p$.

\subsection{Example~1}

\noindent
Let us consider the system consisting of~3~nodes with the initial cash 
endowments
given by the vector $e=(0.1,0,0)$ and the liability and the "distribution"  
matrices corresponding to senior and junior debts:
\begin{alignat*}{2}
L_S&=
\begin{pmatrix}
0 & 1 & 0\\
1 & 0 & 1\\
0 & 2 & 0
\end{pmatrix}\,; &\enskip
L_J&=\begin{pmatrix}
0 & 0 & 0\\
0& 0 & 2\\
0 & 0 & 0
\end{pmatrix}\,;
\\[9pt]
\Pi_S&=
\begin{pmatrix}
0 & 1 & 0\\
0.5 & 0 & 0.5\\
0 & 1 & 0
\end{pmatrix}\,; &\enskip
\Pi_J&=\begin{pmatrix}
0 & 0 & 0\\
0& 0 & 1\\
0 & 0 & 0
\end{pmatrix}.
\end{alignat*}
For this model, the vectors of total liabilities corresponding to the senior and 
junior debts are, respectively, $\tilde L_S=(1,2,2)$ and   $\tilde L_J=(0,2,0)$.

The equations for clearing vectors are:
\begin{align*}
p_S^1 & =  \left(0.1+0.5\, p_S^2 \right)\wedge 1\,;\\
p_S^2 & =  \left(p_S^1+p_S^3 \right)\wedge 2\,;\\
p_S^3 & =  \left(0.5\, p_S^2+p_J^2\right)\wedge 2\,;\\
p_J^1 & = 0\,;\\
p_J^2 & = \left(p_S^1+p_S^3-2\right)^+\wedge 2\,;\\
p_J^3 & = 0.
\end{align*}
It is not difficult to check that there are infinite set of clearing vectors.
Namely, one has that $p_S=(1,2,1+t)$ and $p_J=(0,t,0)$ where $t\in [0,1]$.
The minimal clearing vector corresponds to $t=0$ and the maximal corresponds to 
$t=1$.

\subsection{Example~2}

\noindent
The vector of cash endowments and the matrix of the senior debts  is the same as 
in Example~1. The junior debts matrix $L_J$ and the corresponding 
distribution matrix~$\Pi_J$ are now:
$$
L_J=\begin{pmatrix}
0 & 0 & 0\\
0.4& 0 & 1.6\\
0 & 0 & 0
\end{pmatrix}\,;
 \enskip
\Pi_J=\begin{pmatrix}
0 & 0 & 0\\
0.2& 0 & 0.8\\
0 & 0 & 0
\end{pmatrix}\,.
$$
We are looking for positive solutions of the following  equations:
\begin{align*}
p_S^1 & =  \left(0.1+0.5\, p_S^2 + 0.2\, p_J^2\right)\wedge 1\,;\\
p_S^2 & =  \left(p_S^1+p_S^3 \right)\wedge 2\,;\\
p_S^3 & =  \left(0.5\, p_S^2+0.8\, p_J^2\right)\wedge 2\,;\\
p_J^1 & = 0\,;\\
p_J^2 & = \left(p_S^1+p_S^3-2\right)^+\wedge 2\,;\\
p_J^3 & = 0\,.
\end{align*}
Note that $p_S^1\le 1$ and $p_S^2\le 2$; hence, $p_J^2\le 1$ and the 3rd equation 
is linear:
\begin{equation}
\label{pS3}
p_S^3  =  0.5\, p_S^2+0.8\, p_J^2.
\end{equation}
Substituting into the 2nd equation this expression for~$p_S^3$ and the 
expression for~$p_S^1$ from the 1st equation, one gets that
\begin{equation*}
p_S^2 \!=\!\left(\!\left(0.1+0.5\, p_S^2 + 0.2\, p_J^2\right)\wedge 1+
0.5\, p_S^2+0.8\, p_J^2 \right)\wedge 2.
\end{equation*}
The inequality $p_S^1< 1$ is impossible since in this case, $0.1+0.5\, p_S^2 + 
0.2\, p_J^2<1$, implying that
$$
p_S^2 =\left(0.1+p_S^2 + p_J^2\right)\wedge 2\,.
$$
For positive values of unknown variables, the last equality may hold only if  
$p_S^2=2$ but then, the 1st equation tells one that  $p_S^1=1$.

Thus, it was determined that $p_S^1=1$.

Combining the 2nd equation with~(\ref{pS3}), one obtains the equality
$$
p_S^2  =  \left(1+0.5\, p_S^2+0.8\, p_J^2\right)\wedge 2
$$
implying that $p_S^2=2$.

Available information allows one to reduce
the 5th equation to the simple one of the  form
$p_J^2  = 0.8\left(p_J^2\right)^+\wedge 2$ having the unique solution  $p_J^2=0$.

Summarizing, one gets that  $p_S=(1,2,1)$ and $p_J=(0,0,0)$.

\smallskip

\noindent
\textbf{Comment.} In the first example, the bank 1 has met all liabilities and 
finished with a~positive equity,  the bank~2 has payed the senior liabilities 
but defaulted on the junior debts, the bank~3 has defaulted already at the 
senior debts; and the 
bank~2 has no junior liabilities with the bank~1.  So, the $\Delta$-orbit of the 
banks~2 and~3 are not surplus sets and there are infinite many clearing vectors. 
In the second example, the bank~2 has a~junior debt to bank~1, 
all  $\Delta$-orbits are surplus sets, and the clearing vector is unique.


\section{Models with Illiquid Assets and~a~Price Impact}

\noindent
Let us consider the clearing problem without seniority structure where the bank~$i$ 
owns not only cash~$e^i$ but also~$K$~illiquid assets, in quantities 
$y^{i1},\dots y^{iK}$ represented in  the model by the row~$i$ of the matrix 
$Y=(y^{im})$, $i\le N$, $m\le K$. The nominal prices per unit  of illiquid 
assets are strictly positive  numbers $Q^1,\ldots,Q^K$.  The clearing might  
require their partial   sale  influencing   the market price. If the bank sells  
$u^{im}\in [0,y^{im}]$ units of the $m$th assets for the price~$q_m$, its 
total increase in cash is
$$
(Uq)^i=\sum\limits_{m=1}^K u^{im}q^m\,.
$$

\textbf{The price formation}  is modeled by the inverse demand function 
$F_0:\mathbb{R}^K\to \mathbb{R}^K$ assumed to be continuous and monotone 
decreasing ($F_0(z)\le F_0(x)$ when $z\ge x$ in the sense of partial ordering 
defined by~$\mathbb{R}^K_+$) and 
such that $F_0(0)=Q$ and $F^m_0(Y'{\bf 1})>0$ for $m=1,\ldots , K$.  The first 
condition means that in the absence
of supply, the prices are just the nominal prices while  the second one shows 
that even in the case of total sale, the prices of illiquid assets remain strictly 
positive.


\textbf{The clearing rules:} each bank pays  debts in accordance to the matrix of 
relative liabilities
and sells illiquid assets if it has insufficient amount of cash. The result of 
clearing should be: all
debts of the bank are covered or its equity falls down  to zero.



In the case of several illiquid assets,  there is a~problem how the banks chose 
their strategies of selling. In principle, one can imagine the situation that 
they have  full freedom and, acting in the noncooperative way, drop down the 
market of  illiquid assets because of an excessive supply. It seems reasonable 
that the central authority may  impose extra rules on selling illiquid assets. 
Let us suppose that this is done by prescribing that the bank~$i$ must sell all 
assets in the same proportion~$\alpha^{i}$:
\begin{equation*}
\alpha^i(q)=\fr{\left(\tilde L^i -e^i - \sum\nolimits_j\Pi^{ji}p^j \right)^+}
{ \sum\nolimits_k  y^{ik} q^k}\,\wedge 1\,,\enskip i=1,\dots, N\,.
\end{equation*}
This formula means that for a~fixed market price, the bank does not sell illiquid 
assets
if its  cash reserve together with collected debts covers the liabilities.
In the another extreme case where
$$
\tilde L^i -e^i - \sum\limits_j\Pi^{ji}p^j \ge \sum\limits_k y^{ik} q^k=(Yq)^i\,,
$$
all illiquid assets have to be sold and the bank defaults. In the intermediate
case, the bank sells a~share $\alpha^i\in ]0,1[$ of the $m$th asset adding to its 
cash an extra amount
$(({\tilde L^i -e^i - \sum\nolimits_j\Pi^{ji}p^j})/{\sum\nolimits_k y^{ik} 
q^k})\,y^{im}q_m$.
The total increase in cash allows to cover the liabilities.

Under such a~rule, the  $i$th bank sells~$u^{im}:=u^{im}(p,q)$ units of the $m$th asset where
\begin{equation*}
u^{im}
{}:=\fr{y^{im}\left(\tilde L^i -e^i - \sum\nolimits_j\Pi^{ji}p^j 
\right)^+}{ \sum\nolimits_k y^{ik} q^k}\,\wedge y^{im}.
\end{equation*}
The total supply of the illiquid assets is given by the vector ${\bf 1}'U(p,q)$ 
where
$U(p,q)$ is the matrix with entries given by the above formula.

Define the equilibrium vector 
$$
\left(p^*,q^*\right)\in \left[0,\tilde L\right] \times \left[ F_0(1Y),Q\right]
$$ 
as 
the solution of the system of $N+K$ equations written in the matrix form as
\begin{align}
\label{firstM}
p&=(e+U(p,q)q+\Pi'p)\wedge \tilde L\,;\\
\label{secondM}
q&=F_0(U'(p,q){\bf 1})\,.
\end{align}
The existence of the equilibrium is easy. Indeed,
check that 
\begin{gather*}
U'(p,q){\bf 1}\ge U'\left(\tilde p,\tilde q\right){\bf 1}\,;\\
U(p,q)q+\Pi'p\le  U\left(\tilde p,\tilde q\right)\tilde q+\Pi'\tilde p
\end{gather*}
when $(\tilde p,\tilde q)\ge (p,q)$. Denoting  $F(p,q)$ the right-hand side 
of the first equation, one obtains that  
$$
(p,q)\mapsto \left(F(p,q),F_0\left(U'(p,q)\right){\bf  1}\right)
$$ 
is a~monotone  mapping of the order interval $[0,\tilde L]\linebreak\times [ F_0(1Y),Q]$ into 
itself.  According to Knaster--Tarski theorem, the set of its fixed points is nonempty 
and contains the minimal and maximal elements $(\underline p^*, \underline q^*)$ 
and $(\bar p^*,\bar q^*)$.

For a~fixed $q$, the function $p\to F(p,q)$ is monotone. Thus, by the 
Knaster--Tarski theorem, the set of solutions of Eq.~(\ref{firstM}) is nonempty 
and contains, in particular, the maximal element~$\bar p(q)$.

For any fixed $q\in [F_0(Y),Q]$, the largest solution $\bar p=\bar p(q)$ 
of~(\ref{firstM}) is given by formula:
$$
\bar p=\sup\left\{p\in [0,\tilde L]:\ p\le \left(e+U(p,q)q+\Pi'p\right)\wedge \tilde L\right\}
$$
implying that $q\mapsto \bar p(q)$ is an increasing (and continuous) function on 
$[F_0(Y),Q]$.  It follows that the supply function
$$
q\mapsto \zeta(q):=U'(\bar p(q),q){\bf 1}
$$
is decreasing and, therefore, $q\mapsto F_0(\zeta(q))$ is an increasing 
(and continuous) mapping of the interval  $[F_0(Y),Q]$ into itself and, 
therefore, it has  the minimal and maximal fixed points that will be denoted by~$q_1$ 
and~$q_2$.

\smallskip

\noindent
\textbf{Lemma~3.}\
\textit{Suppose that the scalar function $x\to x'F_0(x)$ is strictly increasing on 
$[F_0(Y),Q]$. Then, the
solution of the equation  $q=F_0(\zeta(q))$ is unique, i.\,e.}, $q_1=q_2$.

\smallskip

\noindent
P\,r\,o\,o\,f\,.\ \
Arguing by contradiction, suppose that  $q_1\neq q_2$.     Since $q_1\le q_2$ 
and $\zeta(\cdot)$ is decreasing,   $\zeta(q_1)\ge \zeta(q_2)$. Moreover, 
$\zeta(q_1)\neq \zeta(q_2)$ as the values of~$F_0$ at these points are~$q_1$ 
and~$q_2$.
 The assumed strict monotonicity  implies that
 $$
 \zeta'(q_1)F_0( \zeta(q_1))> \zeta'(q_2)F_0( \zeta(q_2)).
 $$
It follows that
$$
\zeta'\left(q_1\right) q_1> \zeta'\left(q_2\right)q_2\,.
 $$
To get a~contradiction, it is sufficient to show that
$$
\Delta:= \zeta'\left(q_2\right)q_2-\zeta'\left(q_1\right)q_1\ge 0\,.
$$
Let $\bar p_k:=\bar p(q_k)$ and let
$$
D_k:=\left\{i:\ \left(\tilde L-e-\Pi'\bar p\left(q_k\right)\right)^i\ge 
\left(Yq_k\right)^i\right\}\,,
$$
i.\,e., $D_k$ is the set of banks that are forced to sell all their illiquid assets 
for the price~$q_k$, $k=1,2$. Since~$\bar p(\cdot)$ is increasing, $D_2\subseteq D_1$.  
With the 
notation~${\bf 1}'_{A}$ for the row-vector representing the indicator function
of the subset $A\subseteq \{1,\dots, N\}$, one has, taking into account that 
$a^+=a+a^-$:
\begin{multline*}
\zeta'\left(q_k\right)q_k={\bf 1}'_{D_k}Yq_k\\
{}+{\bf 1}'_{D_k^c}\left(\tilde L-e-\Pi'\bar 
p_k\right)+{\bf 1}'_{D_k^c}\left(\tilde L-e-\Pi'\bar p_k\right)^-.
\end{multline*}
This formula leads to the representation:
\begin{multline*}
\Delta={\bf 1}'_{D_2}Y(q_2-q_1)-{\bf 1}'_{D_1\setminus D_2}Yq_1\\
{}- {\bf 1}'_{D_1^c}
\Pi'\left(\bar p_2-  \bar p_1\right)+{\bf 1}'_{D_2^c\setminus D_1^c}
\left(\tilde L-e -\Pi'\bar p_2\right)\\
{}+ {\bf 1}'_{D_1^c}\left(\left(\tilde L-e -\Pi'\bar p_2\right)^- -
\left(\tilde L-e -\Pi'\bar p_1\right)^-\right)\\
+
{\bf 1}'_{D_2^c\setminus D_1^c}\left(\tilde L-e -\Pi'\bar p_2\right)^-.
\end{multline*}
Since the function $x\to x^-$ (on ${\mathbb{R}}^N$) is positive and decreasing, the 
last two terms in the right-hand side are positive. Regrouping  the third and 
the forth  terms, one gets that
\begin{multline}
\label{ineq1}
\Delta\ge{\bf 1}'_{D_2}Y\left(q_2-q_1\right)-{\bf 1}'_{D_1\setminus D_2}q_1Y
- {\bf 1}'_{D_2^c}\Pi'(\bar p_2-
 \bar p_1)\\
 {}+{\bf 1}'_{D_1\setminus D_2}\left(\tilde L-e -\Pi'\bar p_1\right)\,.
\end{multline}
From Eq.~(\ref{firstM}), it follows that
\begin{multline*}
{\bf 1}'\Pi'\left(\bar p_2-  \bar p_1\right)=
{\bf 1}'\left(\bar p_2-  \bar p_1\right)={\bf 1}'_{D_1}\left(\bar p_2-  \bar p_1\right)\\
{}={\bf 1}'_{D_2}\left(q_2u\left(\bar p_2,q_2\right)-q_1u
\left(\bar p_1,q_1\right)+\Pi'\left(\bar p_2-  \bar p_1\right)\right)\\
{}+{\bf 1}'_{D_1\setminus D_2}\left(\tilde L -\left(e+q_1u\left(\bar p_1,q_1
\right) +\Pi'\bar p_1\right)\right)
\end{multline*}
implying that

\columnbreak

\noindent
\begin{multline*}
 {\bf 1}'_{D_2^c}\Pi'\left(\bar p_2-
 \bar p_1\right)={\bf 1}'_{D_2}\left(U\left(\bar p_2,q_2\right)q_2\right.\\
 \left.{}-
 U\left(\bar p_1,q_1\right)q_1\right)-{\bf  1}'_{D_1\setminus D_2}
 U\left(\bar p_1,q_1\right)q_1\\
{}+{\bf 1}'_{D_1\setminus D_2}\left(\tilde L-e -\Pi'\bar p_1\right)\,.
\end{multline*}
Substituting this expression in~(\ref{ineq1}), one has:
\begin{multline*}
\Delta\ge{\bf 1}'_{D_2}Y\left(q_2-q_1\right)-{\bf 1}'_{D_1\setminus D_2}Yq_1\\
{}-{\bf 1}'_{D_2}\left(U\left(
\bar p_2,q_2\right)q_2-U\left(\bar p_1,q_1\right)q_1\right)\\
{}+
{\bf 1}'_{D_1\setminus D_2}q_1u\left(\bar p_1,q_1\right)=0
\end{multline*}
since the cash increment $(U(\bar p_2,q_2)q_2)^i=(Yq)^i$ for the bank $i\in D_2$ 
and $(U(\bar p_1,q_1)q_1)^i=(Yq_1)^i$ for $i\in D_1\supseteq D_2$.~$\square$


\smallskip

\noindent
\textbf{Theorem~3.}\
\textit{Suppose that the scalar function $x\to x'F_0(x)$ is strictly increasing on 
$[F_0(Y),Q]$. Then, there is $q^*$ such that  the set of solutions of the 
system}~(\ref{firstM}),  (\ref{secondM})
\textit{is contained in the interval  with the extremities $(\underline p(q^*),q^*)$ and 
$(\bar p(q^*),q^*)$.
In particular, if for each~$q$ the solution of}~(\ref{firstM}) \textit{is unique, then 
the solution of the system is also unique}.

\smallskip

\noindent
P\,r\,o\,o\,f\,.\ \ 
Let~$\Gamma$ be the set of~$q$ for which $(p,q)$ is a~solution  of 
the system~(\ref{firstM}),  (\ref{secondM}). If $q^*\in \Gamma$, then $(\bar 
p(q^*),q^*)$
is the solution of~(\ref{firstM}),  (\ref{secondM}). According to  
lemma~3, the point~$q^*$ is uniquely defined. This implies the result.~$\square$

\smallskip

Note that the uniqueness of the solution of~(\ref{firstM}) is guarantied if  for 
each~$i$,
the orbit of~$i$ contains an element with positive cash reserve.

\smallskip

\noindent
\textbf{Remark~2.}
In~\cite{AFM},  it was considered  a~model coinciding with studied 
above
in the case of a~single illiquid asset. The difference is that in the cited 
paper, the equilibrium is defined  as a~vector $(p,q)$ satisfying the
system of equations:
\begin{align}
\label{firstAFM}
p&=\left(e+qy+ \Pi'p\right)^+\wedge \tilde L\,; \\
%\label{secondAFM}
q&=F_0\left({\bf 1}'\left(\left(q^{-1}
\left(\tilde L-e-\Pi'p\right)^+\right)\wedge y\right)\right).\notag
\end{align}
To our opinion, the definition of the equilibrium given 
by the system~(\ref{firstM}), 
(\ref{secondM}), which is in the one liquid asset case has the  form:
\begin{align}
p&=\left(e+\left(\tilde L-e-\Pi'p\right)^+\wedge (qy)+ 
\Pi'p\right)\wedge \tilde L\,; \label{firstAFM1}
\\
%\label{secondAFM1}
q&=F_0\left({\bf 1}'\left(\left(q^{-1}\left(\tilde L-e-\Pi'p\right)^+
\right)\wedge y\right)\right), \notag
\end{align}
 is more natural.  In fact, the   right-hand sides of~(\ref{firstAFM}) 
 and~(\ref{firstAFM1}) as functions $R_1(p,q)$ and $R_2(p,q)$ defined
 on $[0,\tilde L]\times [ F_0(1Y),Q]$ coincide.  To see this, fix~$i$ and  
consider the three possible cases.
\begin{enumerate}[1.]
\item  Let  $e^i+qy+ (\Pi'p)^i\le \tilde L^i$. Then, the expressions for 
$R^i_1(p,q)$ and $R^i_2(p,q)$ have the same form  $e^i+qy+ (\Pi'p)^i$.

\item Let $e^i+qy+ (\Pi'p)^i> \tilde L^i$ and $\tilde L^i-e^i - (\Pi'p)^i\ge 0$. 
Then, the values $R^i_1(p,q)$ and $R^i_2(p,q)$ are equal to~$\tilde L^i$.

\item Let $e^i+qy+ (\Pi'p)^i> \tilde L^i$ and $\tilde L^i-e^i - (\Pi'p)^i<0$. 
Then, the value of $R^i_1(p,q)$ is $\tilde L^i$ and the value of $R^2_1(p,q)$ is 
$(e^i + (\Pi'p)^i)\wedge \tilde L^i=\tilde L^i$.
\end{enumerate}

\vspace*{-18pt}


{\small
\section*{\raggedleft Appendix}

%\vspace*{-6pt}

\subsection*{Knaster--Tarski Fixpoint Theorem}
%\label{app}

\noindent
Let $X$ be a~set with a~partial ordering~$\ge$ and let~$A$ be its nonempty 
subset.
By definition,~$\sup A$ is an element~$\bar x$ such that $\bar x\ge x$ for all 
$x\in A$ and if~$\bar x'$ is such that  $\bar x'\ge x$ for all $x\in A$, then 
$\bar x'\ge \bar x$. The definition of~$\inf A$ follows the same pattern but 
with the dual ordering~$\le$.  A~partially ordered set~$X$ is  {\it complete 
lattice} if for any its nonempty subset~$A$,
there exist~$\inf A$ and~$\sup A$.

\smallskip

\noindent
\textbf{Theorem~4.}\
\textit{Let $X$ be a~complete lattice and let $f : X \mapsto X$ be an order-preserving 
mapping, $L:=\{x:\  f(x)\le x\}$, $U:=\{x:\ f(x)\ge x\}$.   The set
$L\cap U$ of fixed points of~$f$
is nonempty and has the smallest and the largest fixed points  which are, 
respectively, $\underline x:=\inf L$ and}   $\bar x:=\sup U$.

\smallskip

 \noindent
P\,r\,o\,o\,f\,.\ \  
Note that $L\neq \emptyset$ since it contains the element~$\sup X$.
Take arbitrary $x\in L$. Then, $\underline x\le x$
implying that $f(\underline x)\le f(x) \le x$. Thus, $f(\underline x)\le 
\underline x$ as~$\underline x$ is~$\inf L$. So, $\underline x\in L$. 
Since $f(L)\subseteq L$, 
also $f(\underline x)\in L$; hence,  $\underline x\le f(\underline x)$, i.\,e., 
$\underline x= f(\underline x)$. All fixed points belong to~$L$ and, 
therefore,~$\underline x$ is the smallest one.

The proof of the statement for the largest fixed point is analogous.~$\square$

\smallskip

 \noindent
 \textbf{Corollary.}\
\textit{Let $f(\cdot;y)$ be an order-preserving mapping of a~complete lattice $(X,\ge)$ into 
itself, depending on the parameter~$y$ from a~partially ordered set 
$(Y,\succeq)$.
Suppose that $f(\cdot,y)$ is increasing in~$y$, that is, $f(x,y')\ge f(x,y)$ for all 
$x\in X$ when $y'\succeq y$.  Then, the smallest and the largest fixed points are 
also increasing in}~$y$.

\smallskip

\noindent
P\,r\,o\,o\,f\,.\ \ The claim is obvious because the sets   
$$
L(y):=\{x:\  f(x,y)\le x\}
$$ 
are decreasing and the sets 
$$
U(y):=\{x:\ f(x,y)\ge x\}$$ 
are increasing in~$y$
(see~\cite{Milgrom-Roberts}).

These general results are applied to the order intervals $[a,b]\subset \mathbb{R}^d$
with the ordering induced by~$\mathbb{R}^d_+$.

}

\vspace*{-6pt}

\Ack
\noindent
The 
research of Yuri Kabanov was done under partial financial support   of the grant 
of  the Russian Science Foundation No.\,14-49-00079.


\renewcommand{\bibname}{\protect\rmfamily References}

\vspace*{-6pt}

{\small\frenchspacing
{%\baselineskip=10.8pt
\begin{thebibliography}{9}

\bibitem{Eisenberg-Noe} %1
\Aue{Eisenberg, L., and T.\,H.~Noe}. 2001. Systemic risk in financial systems. 
\textit{Manag. Sci.} 47(2):236--249.

\bibitem{Suzuki} %2
\Aue{Suzuki, T.} 2002. Valuing corporate debt: The effect of cross-holdings of stock 
and debt. \textit{J.~Oper. Res. Soc. Japan} 45(2):123--144.

\bibitem{Tarski} %3
\Aue{Tarski, A.} 1955. A~lattice-theoretical fixpoint theorem and its applications. 
\textit{Pacific J.~Math.} 5(2):285--309.


\bibitem{Cont-Wag} %4
\Aue{Cont, R., and L.~Wagalath}. 2015. Fire sale forensics: Measuring endogenous risk. 
\textit{Math. Finance} 26:835--866. %doi: 10.1111/mafi.12071.

\bibitem{AFM} %5
\Aue{Amini, H., D.~Filipovi$\acute{\mbox{c}}$, and A.~Minca.} 2015. To fully net or not to net: Adverse 
effects of partial multilateral netting. %Swiss Finance Institute Research Paper  series. No.~14-63. Forthcoming in ``
\textit{Oper. Res.} 64(5):1135--1142.

\bibitem{Elsinger2011} %6
\Aue{Elsinger, H.} 2009. Financial networks, cross holdings, and limited liability. 
Working paper from Oesterreichische Nationalbank.

\bibitem{Milgrom-Roberts} %7
\Aue{Milgrom, J., and J.~Roberts.} 1994. Comparing equilibria. 
\textit{Am. Econ. Rev.}  84:441--454.




\end{thebibliography} } }

\end{multicols}

\vspace*{-6pt}

\hfill{\small\textit{Received September 25, 2016}}

\vspace*{-18pt}

\Contr

%\vspace*{-3pt}

\noindent
\textbf{El Bitar  Khalil} (b.\ 1981)~--- 
PhD student, Laboratoire de Mathematiques, Universite de Franche-Comte, 
16~Route de Gray, 25030, \mbox{Besan{\!\ptb{\c{c}}}on}, CEDEX, France; 
\mbox{khalilbitar\_aw@hotmail.com}  

 \vspace*{1pt}
 
 \noindent
 \textbf{Kabanov Yuri M.} (b.\ 1948)~---
  professor, Laboratoire de Mathematiques, Universite de Franche-Comte, 
  16~Route de Gray, 25030, Besancon, CEDEX, France; leading scientist, 
  Institute of Informatics Problems, Federal Research Center 
  ``Computer Science and Control'' of the Russian Academy of Sciences,  
  44-2~Vavilov Str., Moscow 119333, Russian Federation; 
  National Research University ``MPEI,'' 14~Krasnokazarmennaya Str., 
  Moscow 111250, Russian Federation; \mbox{Youri.Kabanov@univ-fcomte.fr} 

\vspace*{1pt}
 
 \noindent
 \textbf{Mokbel Rita} (b.\ 1981)~--- 
 PhD student, Laboratoire de Mathematiques, Universite de Franche-Comte, 
 16~Route de Gray, 25030, Besancon, CEDEX, France; \mbox{ritamokbel@hotmail.com}




%\vspace*{8pt}

%\hrule

%\vspace*{2pt}

%\hrule

\newpage

\vspace*{-24pt}



\def\tit{О~ЕДИНСТВЕННОСТИ КЛИРИНГОВЫХ ВЕКТОРОВ, РЕДУЦИРУЮЩИХ 
СИСТЕМНЫЙ РИСК$^*$}

\def\aut{Х.~Эль Битар$^1$, Ю.~Кабанов$^{1,2,3}$, Р.~Мокбель$^1$}


\def\titkol{О~единственности клиринговых векторов, редуцирующих 
системный риск}

\def\autkol{Х.~Эль Битар, Ю.~Кабанов, Р.~Мокбель}

{\renewcommand{\thefootnote}{\fnsymbol{footnote}}
\footnotetext[1]{Представленные в настоящей статье результаты исследований, проведенных 
Ю.\,М.~Кабановым, были получены при частичной финансовой поддержке 
Российского научного фонда (проект №\,14-49-00079).}}


\titel{\tit}{\aut}{\autkol}{\titkol}

\vspace*{-12pt}

\noindent
$^1$Лаборатория математики Университета Франш-Кон\-те, г.~Безансон, Франция

\noindent
$^2$Институт проблем информатики Федерального исследовательского
центра <<Информатика и~управление>>\linebreak
$\hphantom{^1}$Российской академии наук, Российский
университет дружбы народов

\noindent
$^3$Национальный исследовательский университет <<МЭИ>>

\vspace*{6pt}

\def\leftfootline{\small{\textbf{\thepage}
\hfill ИНФОРМАТИКА И ЕЁ ПРИМЕНЕНИЯ\ \ \ том\ 11\ \ \ выпуск\ 1\ \ \ 2017}
}%
 \def\rightfootline{\small{ИНФОРМАТИКА И ЕЁ ПРИМЕНЕНИЯ\ \ \ том\ 11\ \ \ выпуск\ 1\ \ \ 2017
\hfill \textbf{\thepage}}}


\Abst{В~финансовых системах, т.\,е.\ в сети взаимосвязанных банков, 
процедура взаимозачета, или клиринга, состоит в~одновременной выплате 
задолженностей с~целью уменьшения общей их суммы в~системе. Вектор, компоненты 
которого есть суммарные выплаты каждого банка системы, называется клиринговым 
вектором. В~простых моделях, предложенных Айзенбергом и Ноэ (2001) и~независимо 
Судзуки (2002) было показано, что полный клиринг описывается вектором, который 
является неподвижной точкой некоторого отображения. Существование клирингового 
вектора может быть получено прямой ссылкой на теоремы о~неподвижной точке 
Кнас\-те\-ра--Тар\-скo\-го или Брауэра. Вопрос о~его единственности является более 
деликатным. Айзенберг и Ноэ получили достаточное условие единственности 
в~терминах графа связей финансовой системы. В~настоящей работе доказывается 
единственность для двух более общих моделей: модели Эльсингера с~приоритетами 
долгов и~модели типа Ами\-ни--Фи\-ли\-по\-ви\-ча--Мин\-ки, 
в~которой банки имеют неликвидные 
активы, продажа которых влияет на их рыночную цену.}

\KW{системный риск; финансовые сети; клиринг; теорема 
Кнас\-те\-ра--Тар\-ско\-го; модель Ай\-зен\-бер\-га--Ноэ; приоритет финансовых обязательств; 
влияние на ценообразование}



\DOI{10.14357/19922264170110}

%\vspace*{6pt}


 \begin{multicols}{2}

\renewcommand{\bibname}{\protect\rmfamily Литература}
%\renewcommand{\bibname}{\large\protect\rm References}

{\small\frenchspacing
{%\baselineskip=10.8pt
\begin{thebibliography}{9}
\bibitem{3-kab} %1
\Au{Eisenberg L., Noe~T.\,H.} Systemic risk in financial systems~// 
Manag. Sci., 2001. Vol.~47. No.\,2. P.~236--249.
\bibitem{6-kab} %2
\Au{Suzuki T.} Valuing corporate debt: The effect of cross-holdings of stock and debt~// 
J.~Oper. Res. Soc. Japan, 2002. Vol.~45. No.\,2. P.~123--144.
\bibitem{7-kab} %3
\Au{Tarski A.} A~lattice-theoretical fixpoint theorem and its applications~// 
Pacific J.~Math., 1955. Vol.~5. No.\,2. P.~285--309.

\bibitem{2-kab} %4
\Au{Cont R., Wagalath~L.} Fire sale forensics: Measuring endogenous risk~// 
Math.  Finance, 2015. Vol.~26. P.~835--866. %doi: 10.1111/mafi.12071.
\bibitem{1-kab} %5
\Au{Amini H., Filipovi$\acute{\mbox{c}}$~D., Minca~A.} To fully net or not to net: 
Adverse effects of partial multilateral netting~// Oper. Res., 2015. Vol.~62.
No.\,5. P.~1135--1142.

\bibitem{4-kab} %6
\Au{Elsinger H.} Financial networks, cross holdings, and limited liability. 
Working paper from Oesterreichische Nationalbank, 2009.
\bibitem{5-kab} %7
\Au{Milgrom J., Roberts~J.} Comparing equilibria~// Am. Econ. Rev., 1994. 
Vol.~84. P.~441--454.


\end{thebibliography}
} }

\end{multicols}

 \label{end\stat}

 \vspace*{-3pt}

\hfill{\small\textit{Поступила в редакцию  25.09.2016}}
%\renewcommand{\bibname}{\protect\rm Литература}
\renewcommand{\figurename}{\protect\bf Рис.}
\renewcommand{\tablename}{\protect\bf Таблица}