\def\byy{{\bar Y'}}
\def\aa{{\cal A}}
\def\mm{{\sf M}}

 \def\stat{sinits}

\def\tit{АНАЛИТИЧЕСКОЕ МОДЕЛИРОВАНИЕ 
ШИРОКОПОЛОСНЫХ ПРОЦЕССОВ В~СТОХАСТИЧЕСКИХ СИСТЕМАХ, НЕ~РАЗРЕШЕННЫХ
ОТНОСИТЕЛЬНО ПРОИЗВОДНЫХ$^*$}

\def\titkol{Аналитическое моделирование 
широкополосных процессов в~СтС, %стохастических системах, 
не~разрешенных
относительно производных}

\def\aut{И.\,Н.~Синицын$^1$}

\def\autkol{И.\,Н.~Синицын}

\titel{\tit}{\aut}{\autkol}{\titkol}

\index{Синицын И.\,Н.}
\index{Sinitsyn I.\,N.}



{\renewcommand{\thefootnote}{\fnsymbol{footnote}} \footnotetext[1]
{Работа выполнена при финансовой поддержке  РФФИ (проект 15-07-02244).}}


\renewcommand{\thefootnote}{\arabic{footnote}}
\footnotetext[1]{Институт проблем
информатики Федерального исследовательского центра
<<Информатика и~управ\-ле\-ние>> Российской академии наук, \mbox{sinitsin@dol.ru}}

\vspace*{12pt}


\Abst{Рассматриваются конечномерные нелинейные стационарные и~нестационарные 
стохастические системы (СтС), не разрешенные относительно производных в~условиях 
широкополосных гауссовских и~негауссовских возмущений. Такие СтС описывают поведение 
многих технических систем информатики и~управления. Для аналитического 
моделирования и~оценки чувствительности к~параметрам нормальных (гауссовских) 
нестационарных и~стационарных стохастических процессов (СтП) в~таких  СтС разработаны два 
метода аналитического моделирования (МАМ) на основе метода статистической линеаризации (МСЛ) 
и~метода нормальной аппроксимации (МНА). Приведены типовые нелинейности, 
не разрешенные относительно производных, а~также 
два тестовых примера. Алгоритмы положены в~основу разрабатываемого инструментального 
программного обеспечения для решения задач надежности и~безопасности 
технических систем.}

\KW{аналитическое моделирование;
метод нормальной аппроксимации (МНА);
метод статистической линеаризации (МСЛ);
нормальный (гауссовский) стохастических процесс (СтП);
стохастическая система, не разрешенная относительно производных;
чувствительность к~параметрам}

\DOI{10.14357/19922264170101}  

\vspace*{12pt}


\vskip 12pt plus 9pt minus 6pt

\thispagestyle{headings}

\begin{multicols}{2}

\label{st\stat}

\section{Введение}

Методы аналитического моделирования  СтП в~нелинейных 
СтС, разрешенных относительно производных, на основе 
МСЛ и~МНА подробно описаны  в~[1--7].

Рассмотрим обобщение~\cite{4-sin, 5-sin} 
на случай нелинейных дифференциальных СтС, не разрешенных относительно производных. 
Такими уравнениями описываются элементы многих технических средств информатики 
и~управления.

В разд.~2 и~приложении~П1 приведены типовые нелинейности, зависящие от производных, и~их 
статистическая линеаризация. 

Раздел~3 содержит изложение МАМ на основе МСЛ для 
негладких нелинейностей. Особое внимание уделено уравнениям чувствительности МАМ 
на основе МСЛ. 

В~разд.~4  сформулированы некоторые замечания и~МАМ на основе МНА 
для гладких нелинейностей. 

В~приложениях~П2  и~П3 приведены тестовые примеры, 
иллюстрирующие применение МСЛ и~МНА.

%\vspace*{-pt}

\section{Типовые нелинейности, зависящие от~производных, и~их статистическая 
линеаризация}

Рассмотрим сначала скалярную действительную нелинейность следующего вида:
    \begin{equation}
    Z_t = \varphi \left(t, \Theta, Y_t, \dot Y_t \tr Y_t^{(k)}, U_t \tr U_t^{(r)}\right)\,.
    \label{e2.1-sin}
    \end{equation}
Здесь $\Theta$~--- определяющие параметры нелиней\-ности; $Y_t \hm= Y(t)$ 
и~$Y_t^{(k)}$~--- скалярный СтП и~его $k$-я производная 
$(k\hm=1,2,\ldots)$; $U_t\hm = U(t)$ и~$U_t^{(r)}$~--- 
скалярное стохастическое возмущение и~ее $r$-я
 производная $(r\hm=1,2,\ldots)$.

 Важный класс нелинейностей~(\ref{e2.1-sin}) составляют нелинейности, в~которые 
 возмущение~$U_t$ и~его производные входят аддитивно:
   \begin{multline}
    Z_t = \varphi^+ \left(t, \Theta, Y_t, \dot Y_t\tr Y_t^{(k)}\right) + {}\\
    {}+
    \varphi^- \left(t, \Theta, U_t, \dot U_t \tr U_t^{(r)}\right).
    \label{e2.2-sin}
    \end{multline}
Обозначим через 
$$
X_t = \lk Y_t, \dot Y_t \tr Y_t^{(k)}, U_t, \dot 
U_t\tr U_t^{(r)}\rk^{\mathrm{T}}
$$ 
составной вектор.
Тогда при условии конечности ковариационных моментов СтП~$X_t$ согласно 
МСЛ нелинейность~(\ref{e2.1-sin}) допускает следующее приближенное представление:
    \begin{equation}
    Z_t= \varphi = \varphi\left(t, \Theta, X_t\right) 
    \approx \varphi_0 + k_{1x}^\varphi X_t^0\,.
    \label{e2.3-sin}
    \end{equation}
Здесь $\varphi_0 = \varphi_0 (t, \Theta, m_t^x, K_t^x)$ 
и~$k_{1x}^\varphi\hm = k_{1x}^\varphi (t, \Theta, m_t^x, K_t^x)$~--- 
векторные и~матричные коэффициенты статистической линеаризации, определяемые 
уравнениями:
\begin{align}
\varphi_0 &= \mm_N [\varphi]\,;\label{e2.4-sin}\\
    k_{1x}^\varphi &= \lk \left(\fr{\prt}{\prt m_t^x}\right)
    \varphi_0\rk^{\mathrm{T}}\,,\label{e2.5-sin}
    \end{align}
где $X_t^0 = X_t \hm- m_t^x$, $m_t^x \hm= \mm_N \lk X_t\rk = \lk m_{1t} 
\cdots m_{nt}\rk^{\mathrm{T}}$, $K_t^x \hm= \lk k_{pq}\rk \hm= \mm_N \lk X_t^0 X_t^{0\mathrm{T}}
\rk^{\mathrm{T}}$ $(p,q \hm= 1\tr n)$; $\mm_N$~--- символ математического 
ожидания для \mbox{$n$-мер}\-но\-го нормального распределения, $n\hm= k+1$.

Для векторных нелинейностей векторного аргумента~(\ref{e2.1-sin}) 
аналогичные представления имеют место для компонент~$\varphi_l$.  
В~этом случае составной вектор~$\bar X_t$ будет иметь вид:  
$$
\bar X_t \hm= \lk Y_t^{\mathrm{T}} \dot Y_t^{\mathrm{T}} \cdots Y_t^{(k)\mathrm{T}}\, 
U_t^{\mathrm{T}} \cdots U_t^{\mathrm{T}} \cdots U_t^{(r)\mathrm{T}}\rk^{\mathrm{T}}\,,
$$ 
где $ \mathrm{dim}\, 
\bar X_t \hm= n_Y (k+1) \hm+ n^U (r+1)$.

\section{Основные результаты}

Пусть задана векторная нестационарная дифференциальная СтС, не разрешенная 
относительно производной СтП~$Y_t$, вида
    \begin{multline}
\varphi =\varphi \left( t, \Theta, Y_t, \cdots Y_t \tr Y_k^{(k)}, U_t\right) =
     0 \\
     (k=0,1,2,\ldots)\,; \label{e3.1-sin}\end{multline}
     
     \vspace*{-3pt}
     
     \noindent
\begin{equation}
    \dot U_t = a_0^U + a_1^U U_t + b_0^U V_t\,.\label{e3.2-sin}
    \end{equation}
Здесь $\varphi$~---  в~общем случае негладкая функция относительно  $Y_t^{(k)}$; 
$a_0$ и~$a_1^U$~--- век\-тор\-но-мат\-рич\-ные коэффициенты формирующего фильтра (ФФ); 
$V_t$~--- белый (в~общем случае негауссовский СтП) шум с~нулевым математическим 
ожиданием и~матрицей интенсивностей $\nu_t \hm= \nu^V(t)$. Выделим переменные
    \begin{multline*}
    \bar Y_t' =\lk Y_t^{\mathrm{T}}\dot{Y}_t^{\mathrm{T}}\cdots 
    Y_t^{(k-1)T}\rk^{\mathrm{T}}= 
    \lk Y_t^{\mathrm{T}} \bar Y_{1t}^{\mathrm{T}} \cdots 
    \bar Y_{k-1,t}^{\mathrm{T}} \rk^{\mathrm{T}},\\ 
    Y_t^{(k)} = \bar Y_{kt}\,,
   \end{multline*}
и применим к~уравнению~(\ref{e3.1-sin}) МСЛ. В~результате получим векторное уравнение:
    \begin{equation}
    \varphi_0 + k_{1\bar Y_k}^\varphi \bar Y_{kt}^0 + 
    k_{1U}^\varphi U_t^0=0\,.\label{e3.3-sin}
    \end{equation}
Коэффициенты статистической линеаризации в~(\ref{e3.3-sin}) 
будут параметрически зависеть от переменных $t$ и~$\Theta$, математических 
ожиданий $m_t^{\bar Y'}$, $m_t^{\bar Y_k}$ и~$m_t^U$, ковариационных и~взаимно 
ковариационных матриц $K_t^{\bar Y'}$, $K_t^{\bar Y_k}$, $K_t^U$, $K_t^{\bar Y'\bar Y_k}$, 
$K_t^{\bar Y'U}$ и~$K_t^{\bar Y_kU}$.
Разделяя в~(\ref{e3.3-sin}) детерминированные и~центрированные части, придем 
с~учетом~(\ref{e3.2-sin}) к~векторным уравнениям для математических ожиданий
    \begin{align}
    \varphi_0 &=0\,;\label{e3.4-sin}\\
    \dot m_t^U &= a_0^U+ a_1^U m_t^U\label{e3.5-sin}
    \end{align}
и центрированных переменных
    \begin{gather}
    k_{1\bar Y'}^\varphi \bar Y_t^{'0} + k_{1\bar Y_k}^\varphi 
    \bar Y_{kt}^0 + k_{1U}^\varphi U_t^0 =0\,;\label{e3.6-sin}\\
\dot U_t^0 = a_1^U U_t + b_0^U V_t\,.\notag %\label{e3.7-sin}
\end{gather}
Полагая
    \begin{equation}
    \mathrm{det}\, \lk k_{1\bar Y_k}^\varphi\rk\ne 0\,,\label{e3.8-sin}
    \end{equation}
разрешим~(\ref{e3.6-sin}) относительно старшей производной:
  \begin{equation}
    \hspace*{-2mm}\bar Y_{kt}^0 = Y_t^{(k)0} =- \left( k_{1\bar Y_k}^\varphi \right)^{-1}\! 
    \lk k_{1\byy}^\varphi {\bar Y}_t^{'0} + k_{1U}^\varphi U_t^0\rk.\!\!
    \label{e3.9-sin}
    \end{equation}
Введем центрированный составной вектор
$ X_t^0 \hm= \lk {X'}_t^{0\mathrm{T}} U_t^{0\mathrm{T}}\rk^{\mathrm{T}}$.
Здесь ${X'}_t^{0}\hm = I_k {Y'}_t^{0}$, где $I_k$~--- единичная мат\-ри\-ца. 
Тогда совокупная система уравнений для центрированных переменных примет следу\-ющий вид:
    \begin{equation}
    \dot X_t^0 = a^X X_t^0 + b^X V_t\,.\label{e3.10-sin}
    \end{equation}
Здесь введены обозначения для блочных матриц:
    $$
    a^X =\begin{bmatrix}
    a^{\bar X'}& a^{\bar X' U}\\
    a^{U\bar X'}& a^U \end{bmatrix}\,; \qquad
    b^X=\begin{bmatrix}
    0\\ b_0^U\end{bmatrix}\,; 
    $$
    $$
    a^{\bar X'} =\begin{bmatrix}
    I_k& 0\\
    -(k_{1\bar Y_k}^\varphi )^{-1} k_{1\bar X'}^\varphi & 0\end{bmatrix}\,;
$$
$$
      a^{\byy U} = \begin{bmatrix}
    0& 0\\
    0 &-(k_{1 \bar Y_k}^\varphi )^{-1} k_{1U}^\varphi\end{bmatrix}\,;
    $$
    $$
    a^{U\bar X'} = \begin{bmatrix}
    0&0\\
    0&0\end{bmatrix}\,;\qquad 
        a^U =\begin{bmatrix}
    0&0\\
    0&a_1^U\end{bmatrix}\,.
    $$

Применяя к~линейным уравнениям~(\ref{e3.10-sin}) 
корреляционную теорию линейных СтС~\cite{1-sin, 2-sin}, 
придем к~следующим обыкновенным дифференциальным уравнениям для ковариационной 
матрицы $K_t^X \hm= \mm_N \lk X_t^0 X_t^{0\mathrm{T}}\rk $ и~матрицы ковариационных 
функций $K^X  (t_1, t_2) \hm= \mm_N \lk X_{t_1}^0 X_{t_2}^{0\mathrm{T}}\rk$:
   \begin{multline}
   \dot K_t^X = a^X K_t^X + K_t^X \left(a^X\right)^{\mathrm{T}} + 
   b^X \nu \left(b^X\right)^{\mathrm{T}}\,,\\
 K_{t_0}^X = K_0^X\,;\label{e3.11-sin}\end{multline}
 
 \vspace*{-8pt}
 
 \noindent
 \begin{equation}
\fr{\prt K^X (t_1, t_2)}{\prt t_2} = K^X \left(t_1,t_2\right) 
\left(a_{t_2}^X\right)^{\mathrm{T}}\label{e3.12-sin}
 \end{equation}
при начальном условии $K^X (t_1, t_1) \hm= K_{t_1}^X$.

Коэффициенты статистической линеаризации~$\varphi_0$ и~$k_{1\bar Y_k}^\varphi$  
в~(\ref{e3.3-sin}) зависят от математического ожидания и~ковариационной 
матрицы вектора~$\bar Y_k$. Для этого используется уравнение~(\ref{e3.4-sin}) 
и~соотношение
    \begin{equation}
    \dot m_t^{\bar Y_{k-1}} = m_t^{\bar Y_k}\,.\label{e3.13-sin}
    \end{equation}
Для определения $K_t^{\bar Y_k}$ применяется уравнение~(\ref{e3.9-sin}). 
В~результате получим для него искомое соотношение:
    \begin{multline}
    K_t^{\bar Y_k}  = \lk \left(k_{1\bar Y_k}^\varphi \right)^{-1} 
    k_{1\byy}^\varphi \rk 
    K_t^\byy \lk \left(k_{1\bar Y_k}^\varphi \right)^{-1} k_{1\byy}^\varphi 
    \rk^{\mathrm{T}} +{}\\
{}+ \lk (k_{1\bar Y_k}^\varphi )^{-1} k_{1U}^\varphi\rk K_t^U \lk 
\left(k_{1\bar Y_k}^\varphi \right)^{-1} k_{1U}^\varphi\rk^{\mathrm{T}} +{}\\
{}+\lk \left(k_{1\bar Y_k}^\varphi \right)^{-1} k_{1\byy}^\varphi\rk 
K_t^{\byy U} \lk \left(k_{1\bar Y_k}^\varphi \right)^{-1}k_{1U}^\varphi 
\rk^{\mathrm{T}} +{}\\
{}+\lk \left(k_{1\bar Y_k}^\varphi \right)^{-1}k_{1U}^\varphi\rk K_t^{U\bar Y'} 
\lk \left(k_{1\bar Y_k}^\varphi \right)^{-1} k_{1\byy}^\varphi\rk^{\mathrm{T}}\,.
\label{e3.14-sin}
\end{multline}

Таким образом, имеем следующий результат.

\smallskip

\noindent
\textbf{Теорема~1.}\ \textit{Пусть в~нестационарной нелинейной СтС, не разрешенной 
относительно производных,}~(\ref{e3.1-sin}), (\ref{e3.2-sin})
\textit{выполнены следующие условия}:
\begin{itemize}
\item[$1^0$] \textit{дисперсии и~ковариационные моменты СтП $Y_t$ и~их производных 
до $k$-го порядка и~СтП~$U_t$ ко-\linebreak нечны};

\item[$2^0$] \textit{векторная функция $\varphi$ в}~(\ref{e3.1-sin}) 
\textit{действительна и~допускает статистическую линеаризацию по 
Казакову}~(\ref{e3.3-sin});

\item[$3^0$] \textit{имеет место условие}~(\ref{e3.8-sin}).
\end{itemize}

\textit{Тогда в~основе МАМ согласно МСЛ лежат уравнения}~(\ref{e3.4-sin}), 
(\ref{e3.5-sin}), (\ref{e3.11-sin}) \textit{и}~(\ref{e3.12-sin}) 
\textit{при условиях}~(\ref{e3.13-sin}) \textit{и}~(\ref{e3.14-sin}).

\smallskip

В стационарном случае, когда $\nu \hm= \nu_*$, $m_t^X\hm= m_*^X$, $K_t^X\hm=K_*^X$ 
и~$K^X(t_1, t_2) \hm= k^X (\tau)$ $(\tau \hm= t_1\hm- t_2)$, уравнения МАМ 
на основе МСЛ имеют следующий вид:
  \begin{align}
  \varphi_{0*} &=0\,;\label{e3.15-sin}\\
    a_*^X K_*^X + K_*^X \left(a_*^X\right)^{\mathrm{T}} + b_*^X \nu_* 
    \left(b_*^X\right)^{\mathrm{T}} &=0\,;
    \label{e3.16-sin}
    \end{align}
    \begin{equation}
    \left.
    \begin{array}{rl}
    \dot k^X_\tau (\tau) &= a_*^X k^X(\tau)\,;\\[6pt] 
    k^X_\tau (0) &= K^X_* \enskip(\forall\ \tau >0)\,;\\[6pt] 
    k_\tau^X(\tau) &= k_\tau^X (-\tau)^{\mathrm{T}} \enskip 
    (\forall \tau <0)\,.
    \end{array}
    \right\}
    \label{e3.17-sin}
\end{equation}
Таким образом, получена следующая теорема.

\smallskip

\noindent
\textbf{Теорема~2.} \textit{Пусть в~условиях теоремы~$1$ 
уравнения}~(\ref{e3.1-sin}), (\ref{e3.2-sin}) 
\textit{стационарны и~существует ковариационно стационарный СтП, а матрица~$a_*^X$ 
асимптотически устойчива. Тогда в~основе МАМ согласно МСЛ лежат 
уравнения}~(\ref{e3.15-sin})--(\ref{e3.17-sin}).


\smallskip

Если для оценки чувствительности МАМ к~параметрам~$\Theta$, явно входящим в~(\ref{e3.1-sin}), 
применить метод функций чувствительности первого порядка для $\nabla^\Theta m_t^X$, 
$\nabla^\Theta K_t^X$ и~$\nabla^\Theta K^X (t_1, t_2)$, то в~условиях теоремы~1 
будем иметь следующие уравнения~\cite{8-sin}:
   \begin{gather}
   \dot \nabla^\Theta m_t^X = A_t^m (t,\Theta)\,,\enskip 
    \nabla^\Theta m_{t_0}^X =0\,;\label{e3.18-sin}\\[2pt]
    \dot \nabla^\Theta K_t^X = A_t^X (t,\Theta)\,,\enskip 
    \nabla^\Theta K_{t_0}^X =0\,;\label{e3.19-sin}
    \end{gather}
    
    \vspace*{-14pt}
    
    \noindent
    \begin{multline}
     \fr{\prt \nabla^\Theta K^X (t_1,t_2)}{\prt t_2} = 
    A_{t_1t_2}^K \left(t_1, t_2,\Theta\right)\,,\\
    \nabla^\Theta K^X \left(t_0, t_0\right)=0\,.
        \label{e3.20-sin}
    \end{multline}
Здесь функции $A$ получаются путем дифференцирования правых частей уравнений теоремы~1 
по параметру~$\Theta$.

Таким образом, получен следующий результат.

\medskip

\noindent
\textbf{Теорема~3.}\ \textit{Пусть выполнены все условия теоремы~$1$. 
Тогда уравнения для первых функций чувствитель\-ности будут иметь вид}~(\ref{e3.18-sin})--(\ref{e3.20-sin}) 
\textit{при условии ко\-неч\-ности правых частей}.

\smallskip

Аналогичная теорема имеет место в~стационарном случае, 
когда выполнены условия теоремы~2.

Для алгоритмизации МАМ~(\ref{e2.1-sin})--(\ref{e2.3-sin}) на основе теорем~1--3 
необходимо уметь вычислять интегралы~(\ref{e2.4-sin}), (\ref{e2.5-sin}). 
Численные методы описаны в~\cite{8-sin, 9-sin}. 
При этом численное интегрирование уравнений для математических ожиданий 
и~ковариационных моментов должно быть согласовано с~вычислением коэффициентов 
статистической линеаризации.


\section{Замечания и~обобщения}


\noindent
\textbf{4.1.}\ Примеры некоторых типовых нелинейностей~(\ref{e2.1-sin}), 
а~также выражения для их 
коэффициентов статистической линеаризации даны в~приложении~П1.

\noindent
\textbf{4.2.}\ Путем расширения вектора~$X_t$ аналогичные тео\-ре\-мы устанавливаются для 
нелинейностей в~(\ref{e3.1-sin}), в~которые входят производные $U_t^{(r)}$ 
$(r\hm=1,2,\ldots)$.

\noindent
\textbf{4.3.}\  Для достаточно гладких нелинейностей в~(\ref{e3.1-sin}) 
можно применить аналог известного метода <<интегрирования посредством 
дифференцирования>>, если воспользоваться формулами Ито~\cite{1-sin, 2-sin} 
для стохастического дифференцирования сложных функций.

В самом деле, результаты разд.~3 допускают обобщение на случай гладких функций
\begin{equation}
\varphi =\varphi\left(t, \Theta, Y_t, \dot Y_t \tr Y_t^{(k)}, U_t\right) =0
    \label{e4.1-sin}
\end{equation}
и нелинейного ФФ вида
\begin{equation}
\dot U_t = a^U \left(t, \Theta,U_t\right) dt + b^U \left(t, \Theta,U_t\right) V_t^U\,.
\label{e4.2-sin}
\end{equation}
Здесь $a^U \hm= a^U(t, \Theta,U_t)$ и~$b^U(t, \Theta,U_t)$~--- 
$(n^Y\times 1)$- и~$(n^Y\times n_U^V)$-мер\-ные функции; $V_t^U$~--- 
белый шум в~строгом смысле~\cite{1-sin, 2-sin}, допускающий представление
    \begin{align*}
       V_t^U &= \dot W_t^U\,; \\
   W_t^U &= W_0^U (t,\Theta) + 
    \iii_{R_0^q} c^U (\Theta,\rho) P^0 (t,\Theta, d\rho)\,,
    %    \label{e4.3-sin}
    \end{align*}
где $\nu_t$~--- его интенсивность:
   \begin{equation*}
   \nu_t =\nu_t^{W_0} +  \iii_{R_0^q} c^U (\Theta,\rho) 
    \left[c^U (\Theta,\rho)\right]^{\mathrm{T}} \nu_P (t,\Theta,\rho)\,d\rho\,;
    %\label{e4.4-sin}
    \end{equation*}
$c^U\hm = c^U(\Theta,\rho)$~--- 
известная векторная функция той же размерности, что и~$W_t^0$, 
а~интеграл при любом $t\hm\ge t_0$ представляет собой стохастический интеграл 
по центрированной пуассоновской мере $P^0(t,\Theta,\aa)$, независимой от~$W_0$ 
и~имеющей независимые значения на попарно непересекающихся множествах; $\aa$~--- 
борелевское множество пространства~$R_0^q$ с~выколотым началом; $\nu^W$, 
$\nu^{W_0}$ и~$\nu^P$~--- интенсивности $W_t^U$, $W_{0t}$ и~$P^0$. 
Уравнение~(\ref{e4.2-sin}) понимается в~смысле Ито и~имеет единственное решение 
в~среднем квадратическом~\cite{1-sin, 2-sin}.


Для гладких функций в~(\ref{e4.1-sin}) выполним с~уравнениями~(\ref{e4.1-sin}) 
и~(\ref{e4.2-sin}) следующие преобразования:
\begin{enumerate}[(1)]
\item будем дифференцировать сполна по~$t$ левые части уравнений~(\ref{e4.1-sin}) 
по обобщенной формуле Ито~\cite{1-sin, 2-sin} 
до тех пор, пока не появятся производные белого шума. 
В~результате получим следующую систему уравнений:
  \begin{equation}
  \varphi=0\,,\enskip \dot \varphi =0\tr \varphi^{(h)}=0\,;\label{e4.5-sin}
  \end{equation}

\item введем вектор  $Z_t = \lk Z_{T}^{'\mathrm{T}} Z_t^{''\mathrm{T}}\rk^{\mathrm{T}}$, 
составленный из $Z_t' = \lk Y_t^{\mathrm{T}} \dot{Y}_t^{\mathrm{T}}\cdots 
Y_t^{(k-1)\mathrm{T}}\rk^{\mathrm{T}}$ и~вспомогательного вектора $Z_t''$, 
определяемого уравнениями~(\ref{e4.5-sin});

\item в~результате придем к~уравнениям, разрешенным относительно дифференциалов 
следу\-юще\-го вида:
\begin{equation*}
dZ_t = a^Z  dt + b^Z  d W_0 + \iii_{R_0^q} c^Z P^0(\Theta,t,du)\,,
%\label{e4.6-sin}
\end{equation*}
для которых уравнения МНА имеют известный вид~\cite{1-sin, 2-sin}:
  \begin{gather}
  \dot m_t^Z = \Phi_t^m \left(t,\Theta,m_t^Z, K_t^z\right)\,; \enskip 
  m_0^Z = m_{t_0}^Z\,,\label{e4.7-sin}\\
  \dot K_t^Z = \Phi_t^K \left(t,\Theta,m_t^Z, K_t^z\right)\,; \enskip 
  K_0^Z = K_{t_0}^Z\,,\label{e4.8-sin}
  \end{gather}
  
  \vspace*{-14pt}
  
  \noindent
  \begin{multline}
   \fr{\prt K^Z (t_1, t_2)}{\prt t_2} = \Phi_{t_1, t_2}^K 
    \left(t_1, t_2,\Theta,m_{t_2}^Z, K_{t_2}^z\right)\,, \\
        K^Z\left(t_1, t_2\right) = K_{t_1}^Z\,.
               \label{e4.9-sin}
    \end{multline}
Здесь введены следующие обозначения:
    \begin{equation}
    \Phi_t^m\left(t,\Theta,m_t^Z, K_t^z\right)= \mm_N\lk a^Z\rk\,;\label{e4.10-sin}
    \end{equation}
    
            \vspace*{-15pt}
        
        \noindent
        \begin{multline}
    \Phi_t^K \left(t,\Theta,m_t^Z, K_t^z\right) =
        \Phi_{1t} \left(t,\Theta,m_t^Z, K_t^z\right)+{}\\
\hspace*{-5mm}{}+
    \Phi_{1t}^{\mathrm{T}} \left(t,\Theta,m_t^Z, K_t^z\right)+
    \Phi_{2t}\left(t,\Theta,m_t^Z, K_t^z\right)\,;\label{e4.11-sin}
    \end{multline}
    
    \vspace*{-3pt}
    
    \noindent
        \begin{equation}
\hspace*{-8mm}\Phi_{1t} \left(t,\Theta,m_t^Z, K_t^z\right)=
   \mm_N \lk a^Z \left(Z_t - m_t^Z\right)^{\mathrm{T}}\rk\,;\label{e4.12-sin}
   \end{equation}
   \begin{equation}
   \Phi_{2t} \left(t,\Theta,m_t^Z, K_t^z\right)=
   \mm_N \lk \si\left(t,\Theta,Z_t\right)\rk\,;\label{e4.13-sin}
\end{equation}
   
           \vspace*{-12pt}
        
        \noindent
   \begin{multline}
\si\left(t,\Theta,Z_t\right)= 
      \si_0\left(t,\Theta,Z_t\right) +{}\\
  \hspace*{-8pt}{}+\iii_{R_0^q}  
     c^Z\left(t,\Theta,Z_t,u\right) c^Z \left(t,\Theta,Z_t,u\right)^{\mathrm{T}} 
     \nu_P (t,\Theta,du)\,,\\
\si_0\left(t,\Theta,Z_t\right) ={}\\
{}=  b^Z \left(t,\Theta,Z_t\right) 
    \nu_0 (t,\Theta) b^Z\left(t,\Theta,Z_t\right)^{\mathrm{T}}\,;\label{e4.14-sin}
        \end{multline}
        
        \vspace*{-12pt}
        
        \noindent
        \begin{multline}
\Phi^K_{t_1,t_2} \left(t_1, t_2,\Theta,m_{t_2}^Z, K_{t_2}^z\right)={}\\
{}=
    K^Z \left(t_1, t_2\right) \left(K_{t_2}^Z\right)^{-1}\! \Phi_{1t}^{\mathrm{T}}\,.
    \label{e4.15-sin}
\end{multline}
    \end{enumerate}

Таким образом, имеем следующие результаты.

\smallskip

\noindent
\textbf{Теорема~4.} 
\textit{Пусть для СтС}~(\ref{e4.1-sin}), (\ref{e4.2-sin}) \textit{выполнены условия}:
\begin{itemize}
\item[$1^0$] \textit{правые части уравнений}~(\ref{e4.1-sin}) 
\textit{имеют производные по явно входящим переменным $Y_t\tr Y_t^{(k)}$ до 
порядка}~$n_Y^\varphi$;

\item[$2^0$] \textit{уравнения}~(\ref{e4.1-sin}) \textit{совместно с}~(\ref{e4.5-sin}) 
\textit{допускают приведение к~виду}~(\ref{e4.15-sin}) 
\textit{и~имеют единственное решение, понимаемое в~среднем квадратическом (с.к.)}.
\end{itemize}
\textit{Тогда уравнения МАМ на основе МНА имеют вид}~(\ref{e4.7-sin})--(\ref{e4.9-sin}) 
\textit{при условии конечности интегралов}~(\ref{e4.10-sin})--(\ref{e4.14-sin}).


Аналогичная теорема устанавливается в~стационарном случае.

\begin{table*}[b]\small
\begin{center}
%\vspace*{2ex}

\begin{tabular}{|c|c|c|}
\multicolumn{3}{c}{Коэффициенты МСЛ для типовых нелинейностей}\\
\multicolumn{2}{c}{\ }\\[-6pt]
\hline
№ &$\varphi$ & $\varphi_0$\\
\hline
&\\[-9pt]
1.&$\dot Y^2$&$D_{\dot Y} (1+m_{\dot Y}^2 D_{\dot Y}^{-1})$\\[3pt]
\hline
&\\[-9pt]
2.&$\dot Y^3$ &$D_{\dot Y} (3+m_{\dot Y}^2 D_{\dot Y}^{-1})$\\[3pt]
\hline
&\\[-9pt]
3.&$\dot Y^{2n+1}$&$\displaystyle 
\sss_{l=0}^{2n+1} C_{2n+1}^l m_{\dot Y}^l \mu_{2n+1 -l}$\\[10pt]
\hline
&\\[-9pt]
4.&$\mathrm{sgn}\, \dot Y$&
$\displaystyle
2\Phi \left(\fr{m_{\dot Y}}{\sqrt{D_{\dot Y}}}\right)$\\[10pt]
\hline
&\\[-9pt]
5. &$\dot Y^2  \,\mathrm{sgn}\, \dot Y$&
$\displaystyle
2D_{\dot Y}\lk (1+m_{\dot Y}^2 D_{\dot Y}^{-1}) \Phi\left(
\fr{m_{\dot Y}}{\sqrt{D_{\dot Y}} }\right) + 
\fr{1}{\sqrt{2\pi D_{\dot Y}}} \exp \left(-
\fr{m_{\dot Y}^2 }{2D_{\dot Y}} \right)\rk$\\[10pt]
\hline
&\\[-9pt]
6. &$1( \dot Y)$ &
$\displaystyle\fr{1}{2}+\Phi \left(\fr{m_{\dot Y}}{\sqrt{D_{\dot Y}}}\right)$\\[10pt]
\hline
&\\[-9pt]
7. &\tabcolsep=0pt\begin{tabular}{c}
$Y_1^{h_1} \ldots Y_n^{h_n}$\ $(h_1 \tr h_n$ ---\\
целые неотрицательные числа: \\ $h=h_1+\cdots +h_n)$\end{tabular}
&
\tabcolsep=0pt\begin{tabular}{c}
$\displaystyle \alp_h = m_p \alp_{h-e_p} + \sss_{r=1}^n h_r k_{pr} 
\alp_{h-e_p-e_r}-k_{pp} \alp_{h-2e_p}$,\\ 
$\alp_0 =1\,,\enskip $  $\alp_{e_p} =m_p$\end{tabular}\\
\hline
\end{tabular}
\end{center}
\vspace*{9pt}
\end{table*}


%\pagebreak

\noindent
\textbf{4.4.}\ Для повышения точности МАМ можно воспользоваться известными методами 
параметризации распределений (моментов, семиинвариантов, ортогональных разложений 
плотностей, эллипсоидальной аппроксимации и~линеаризации и~др.), изложенными 
в~[1--3].

\noindent
\textbf{4.5.}\ В~том случае когда вектор возмущений~$U_t$ или нелинейность~$\varphi$ 
заданы каноническими разложениями, полученные результаты допускают обобщение, 
если воспользоваться результатами~\cite{2-sin, 3-sin}.

\noindent
\textbf{4.6.}\ Особый интерес представляет развитие МАМ для узкополосных СтП на 
основе метода гармонического баланса или метода медленно меняющихся амплитуд.

\noindent
\textbf{4.7.}\ Полученные результаты допускают непосредственное развитие и~для 
разностных СтС.

\section{Заключение}

На основе приближенного метода статистической линеаризации разработаны 
алгоритмы аналитического моделирования векторных широкополосных нормальных 
процессов в~динамических СтС, не разрешенных относительно 
производных. Получены уравнения МНА для гладких функций~(\ref{e3.1-sin}), 
нелинейных ФФ и~ви\-не\-ров\-ско-пуас\-со\-нов\-ских шумов. Особое внимание 
уделено уравнениям чувствительности к~па\-ра\-метрам.

В приложении~П1 приведены выражения для типовых нелинейностей и~коэффициентов 
статистической линеаризации, а~в~приложениях~П2 и~П3~--- тестовые примеры.

Алгоритмы положены в~основу разрабатываемого инструментального программного 
обеспечения StS-ANALYSIS'2017 для решения задач надежности и~безопасности 
технических систем.



%\setcounter{equation}{0}
{\small \section*{\raggedleft Приложения}

%\renewcommand{\theequation}{П.\arabic{equation}}



\noindent
\textbf{П.1.}\ Приведенные в~\cite{4-sin} типовые одно- и~многомерные нелинейности, если 
в~качестве  одной из переменных выбрать~$\dot Y_t$, могут служить моделями 
нелинейностей, не разрешенных относительно производных.

В таблице на основе~\cite{4-sin} составлены аналитические выражения для некоторых 
типовых одно- и~многомерных нелинейностей. Здесь введены следующие обозначения:
 $$
 h = \lk h_1 \cdots h_n\rk^{\mathrm{T}};\enskip  | h| = h_1 +\cdots + h_n;$$
 $$
e_p = \lk 0\cdots \fr{1}{p}\cdots 0\rk;
$$ 
 $$
 k_p = \lk k_{p_1} \cdots k_{p_n}\rk\enskip (p\hm= 1\tr n)^p;\enskip 
K=\lk k_{pq}\rk\,,
$$ 
а~через~$\alp_h$ и~$\mu_h$ обозначены вероятностные 
 начальные и~центральные моменты порядка~$h$, $\alp_0 \hm=1$, $\alp_{e_p}\hm= m_p$. 
 Функцию Лапласа и~интеграл от нее возьмем в~виде~\cite{1-sin, 2-sin, 4-sin}:
    $$\Phi(z) = \fr{1}{\sqrt{2\pi}} \iii_0^z e^{-t^2/2}\, dt\,;\enskip 
    \Phi'(z) = \fr{1}{\sqrt{2\pi}} \iii_0^z e^{-z^2/2}\, dt\,.
    $$



\smallskip

\noindent
\textbf{П.2.}\ Рассмотрим МАМ для одномерной СтС вида:
    \begin{equation}
    \left.
    \begin{array}{rl}
    \psi \left(\dot Y_t\right) &= a_0^Y + a_1^Y Y_t + a_1^U U_t\,;\\[6pt] 
    \dot U_t &= a_0^U + a_1^U U_t + b_0^U V_t\,,
    \end{array}
    \right\}
    \label{p1}
    \end{equation}
где $\psi (\dot Y_t)$~--- нелинейная недифференцируемая функ-\linebreak ция.

Уравнения МАМ на основе МСЛ согласно теореме~1 имеют следующий вид:
    \begin{equation*}
    0^Y + a_1^Y m_t^Y + a_1^U m_t^U\,,\enskip 
    \dot m_t^Y = m_t^{\dot Y}\,; %\label{p2}
    \end{equation*}
    \begin{equation*}
    k_{1\dot Y}^\psi (m^{\dot Y}, D^{\dot Y}) \dot Y_t^0 \approx  
    a_1^Y Y_t^0 + a_1^U U_t^0\,,\enskip 
    \dot U_t^0 = a_1^U U_t^0 + b_0^U V_t\,; %\label{p3}
    \end{equation*}
    
    \vspace*{-12pt}
    
    \noindent
    \begin{multline*}
     \dot D_t^U = 2a_1^U D_t^U + \left(b_0^U\right)^2\nu\,, \\ 
     \dot D_t^Y=2\lk   a_1^Y\left(k_{1\dot Y}^\psi\right)^{-1} D_t^Y + 
     a_1^U \left(k_{1\dot Y}^\psi\right)^{-1} K_t^{UY}\rk\,;
     \end{multline*}
     
     \vspace*{-6pt}
     
     \noindent
    \begin{equation*}
    \dot K_t^{UY} = \lk a_1^U + a_1^Y \left(k_{1\dot Y}^\psi\right)^{-1}\rk K_t^{UY}+
    a_1^U \left(k_{1\dot Y}^\psi\right)^{-1}D_t^U\,; %\label{p4}
    \end{equation*}
    
    \vspace*{-12pt}

\noindent
    \begin{multline*}
          D_t^{\dot Y} = \lk a_1^Y \left(k_{1\dot Y}^\psi\right)^{-1}\rk^2 D_t^Y + 
          \lk a_1^U\left(k_{1\dot Y}^\psi\right)^{-1}\rk^2 D_t^U + {}\\
          {}+
          2a_1^Y a_1^U \left(k_{1\dot Y}^\psi\right)^{-2} K_t^{UY}\,.
          %\label{p5}
          \end{multline*}
Здесь через  $\psi_0 \hm=\psi_0 (m^{\dot Y}, D^{\dot Y})$ 
и~$k_{1\dot Y}^\psi (m^{\dot Y}, D^{\dot Y})$ обозначены коэффициенты 
статистической линеаризации нелинейности $\psi(\dot Y_t)$.

Согласно теореме~2 для стационарного случая, когда  $a_0^U \hm=0$, $\nu\hm=\nu_*$ 
при условии асимптотической устойчи\-вости матрицы
    \begin{equation*}
    a=\begin{bmatrix}
    a_1^U\left(k_{1\dot Y}^\psi\right)^{-1}_*& a_1^Y\left(k_{1\dot Y}^\psi\right)^{-1}_*
    \end{bmatrix}\,,
     %\label{p6}
     \end{equation*}
уравнения МАМ имеют вид:
    \begin{equation}
    \psi_0 =0\,; \label{p7}
    \end{equation}
    \begin{equation}
    D_*^U =\left(b_0^U\right)^2 \fr{\nu^*}{2 a_1^U}\,;
    \label{p8}
    \end{equation}
    \begin{equation}
    \left(a_1^Y D_*^Y + a_1^U K_*^{UY}\right)=0\,; 
    \label{p9}
    \end{equation}
    \begin{equation}
    \lk a_1^U+ a_1^Y \left(k_{1\dot Y}^\psi\right)^{-1}_*\rk K_*^{UY} +  
    a_1^U\left(k_{1\dot Y}^\psi\right)^{-1}_* D_*^U =0\,; 
    \label{p10}
    \end{equation}
    \begin{equation*}
    \left(k_{1\dot Y}^\psi\right)^{2}_* D_*^{\dot Y} -\left(a_1^Y\right)^2 D_*^Y - 
    \left(a_1^U\right)^2 D_*^U - 2 a_1^Y a_1^U K_*^{UY}=0\,. 
    %\label{p11}
    \end{equation*}
Отсюда, в~частности, при  $\dot \psi \hm= \mathrm{sgn}\, \dot Y$ и~п.~4 таблицы 
имеем:
        \begin{equation}
        \left.
        \begin{array}{rl}
          \hspace*{-2mm} \psi_0 \left(m^{\dot Y}, D^{\dot Y}\right)& =2 \Phi \left(\fr{m_{\dot Y}}{\sqrt{D_{\dot Y}}}
    \right)\,; \\[6pt]
             \hspace*{-2mm} k_{1\dot Y}^\psi \left(m^{\dot Y}, D^{\dot Y}\right) &= 
    \fr{1}{\sqrt{D_*^{\dot Y}  }}\,\fr{2}{\sqrt{2\pi}} \exp\! \lk - 
    \fr{1}{2} \! \left(\!\fr{m^{\dot Y 2}}{D^{\dot Y}}\right)\! \rk.
    \end{array}\!\!
    \right\}\!\!
    \label{p12}
        \end{equation}

Следовательно, в~силу~(\ref{p7}) и~(\ref{p12}) находим:
    $$
    m^{\dot Y}_* =0\,;\enskip 
    k_{1\dot Y}^\psi =k_{1\dot Y}^\psi \left(0, D^{\dot Y}_*\right) = 
    \fr{2}{\sqrt{\pi D_*^{\dot Y}  }}\,;
    $$
    \begin{equation}
    a=\begin{bmatrix}
     a_{11}&a_{12}\\
    a_{21}&a_{22}\end{bmatrix}\,,
    \label{p13}
    \end{equation}
    где
    \begin{alignat*}{2}
    a_{11} &= a_1^U\,; &\enskip 
    a_{12}&= 0\,; \\[6pt] 
    a_{21}&= \sqrt{\fr{\pi D_*^{\dot Y}}{2}}\,; &\enskip 
    a_{22}&= a_1^Y \sqrt{\fr{\pi D_*^{\dot Y}}{2}}\,.
   \end{alignat*}

Условие асимптотической устойчивости матрицы~(\ref{p13}) требует соблюдения 
условий $a_1^U \hm<0$ и~$a_1^Y\hm<0$. Стационарные значения 
$D_*^{\dot U}$, $D_*^Y$, $K^{UY}_*$ находятся из~(\ref{p8})--(\ref{p10}).

\smallskip

\noindent
\textbf{П.3.}\
Рассмотрим СтС вида
    \begin{equation*}
    \dot Y_t^2 =- \alp^2 \left(Y_t^2 - U_t^2\right)\,, \enskip 
    \dot U_t = a_0^U + a_1^U U_t + b_0^U V_t\,.
    %\label{p14}
    \end{equation*}

Найдем сначала уравнения МАМ согласно МСМ. Учитывая соотношения
    \begin{equation*}
    X_t^2 \approx \lk \left(m_t^X\right)^2 + D_t^X \rk + 2 m_t^X X_t^0
    %\label{p15}
    \end{equation*}
для  $X_t = \lf Y_t,\dot Y_t, U_t\rf$,  придем после применения МСЛ 
к~следующим уравнениям для математических ожиданий и~центрированных составляющих:
    \begin{equation*}
    \left( m^{\dot Y}_t\right)^2 + D^{\dot Y}_t= -\alp^2 
    \lk \left(m^{Y}_t\right)^2 - \left(m_t^U\right)^2 + 
    D_t^Y - D_t^U\rk\,; %\label{p16}
    \end{equation*}
    \begin{equation*}
\dot m^{U}_t= a_0^U + a_1^U m_t^U\,; 
%\label{p17}
\end{equation*}
    \begin{equation*}
    m^{\dot Y}_t\dot Y_t^0 =-2\alp^2 \left(m_t^Y Y_t^0 - m_t^U U_t^0\right)\,;
    %\label{p18}
    \end{equation*}
    \begin{equation*}
    \dot U_t^0 =a_1^U U_t^0 + b_0 V_t\,.
    %\label{p19}
    \end{equation*}

Применяя уравнения теоремы~1, положив $X_t\hm = \lk U_t^0\, Y_t^0\rk^{\mathrm{T}}$ 
и~обозначив
    $$
    \bar a=\begin{bmatrix}
    -a_1^U&0\\
    \alp^2 m_t^U \left(m_t^{\dot Y}\right)^{-1}&- \alp^2 m_t^Y 
    \left(m_t^{\dot Y}\right)^{-1}\end{bmatrix}\,;\enskip
    b=\begin{bmatrix}
    b_0\\ 0\end{bmatrix}\,, 
    $$
придем к~следующей взаимосвязанной системе уравнений для~$m_t^Y$, $m_t^{\dot Y}$, 
$D_t^Y$, $D_t^U$, $K_t^{UY}$ и~$D_t^{\dot Y}$:
    \begin{equation*}
    m_t^{\dot Y}= \left(m_t^{Y}\right)^\bullet\,;
    %\label{p20}
    \end{equation*}
    
    \vspace*{-12pt}
    
    \noindent
    \begin{multline}
D_t^{\dot Y}= D\lk -\alp^2 m_t^Y \left(m_t^{\dot Y}\right)^{-1} Y_t^0 + 
\alp^2 m_t^U U_t^0\rk ={}\\[2pt]
{}= \alp^4 \left\{ \lk m_t^Y \left(m_t^{\dot Y}\right)^{-1}\rk^2 
D_t^Y+ \lk m_t^U \left(m_t^{\dot Y}\right)^{-1}\rk^2 D_t^U - {}\right.\\[2pt]
\left.{}-2 m_t^Y m_t^U \left(m_t^{\dot Y}\right)^{-2} K^{UY}
\vphantom{\lk m_t^Y \left(m_t^{\dot Y}\right)^{-1}\rk^2 }
\right\}\,;
\label{p21}
\end{multline}
    \begin{equation}
D_t^{U}=-2a_1^U D_t^U + b_0^2 \nu\,,\enskip 
D_{t_0}^U = D_0^U\,;
\label{p22}
\end{equation}

\vspace*{-10pt}

\noindent
    \begin{multline}
\dot D_t^{Y}=  -2\alp^2\lk m_t^Y \left(m_t^{Y}\right)^{-1} 
 D_t^Y -  m_t^U \left(m_t^{\dot Y}\right)^{-1}K_t^{UY}\rk\,,\\[3pt]  
 D_{t_0}^Y=D_0^Y\,;
 \label{p23}
 \end{multline}
 
   \vspace*{-12pt}

\noindent
    \begin{multline}
    \dot K_t^{UY}=  \alp^2\ m_t^U \left(m_t^{\dot Y}\right)^{-1} D_t - {}\\
    {}-
    \lk a_1^U + \alp^2 m_t^Y \left(m_t^{\dot Y}\right)^{-1}\rk K_t^{UY}\,,\enskip  
    K_{t_0}^{UY}=K_0^{0Y}\,.
    \label{p24}
    \end{multline}

В стационарном случае, используя теорему~2 при $a_0^U\hm=0$ и~постоянных 
параметрах~$\alp$, $a_1^U\hm<0$, $ b_0^U$ и~$\nu^*$, находим
    \begin{equation*}
    m_*^U =0 \,,\enskip 
    D_*^U = \fr{b_0^2 \nu^*}{2 a_1^U}\,. 
    %\label{p25}
    \end{equation*}
При этом моменты $m_t^Y$, $D_t^Y$, $K_t^{UY}$ и~$D_t^{\dot Y}$ 
нестационарны и~определяются из~(\ref{p21})--(\ref{p24}). 
Последние допускают следующую запись:
     \begin{equation*}
     m_t^{\dot Y} = \left(m_t^Y\right)^\bullet\,;
     %\label{p26}
     \end{equation*}
    \begin{equation}
    \dot D_t^{Y} =\gamma_1 D_t^Y\,,\enskip 
     \gamma_1 = 2 \alp^2 m_t^Y \left(m_t^{\dot Y}\right)^{-1}\,;
     \label{p27}
     \end{equation}
    \begin{equation}
    \dot K_t^{UY} =\gamma_2 K_t^{UY}\,,\  
     \gamma_2 =-\lk a_1^U+\alp^2 m_t^Y \left(m_t^{\dot Y}\right)^{-1}\rk;\!
     \label{p28}
     \end{equation}
    \begin{equation*}
    \left(m_t^{\dot Y}\right)^{2}+D_t^{\dot Y} =
     -\alp^2 \lk \left(m_t^{Y}\right)^{2}+D_t^Y - D_t^U\rk\,;
    % \label{p29}
     \end{equation*}
    \begin{equation*}
    D_t^{\dot Y}=  \alp^4  \left( m_t^{Y} \right)^{2}\left( m_t^{\dot Y}\right)^{-2}
     D_t^Y\,.
     %\label{p30}
     \end{equation*}

В зависимости от знаков  $\gamma_1$ и~$\gamma_2$ в~(\ref{p27}) и~(\ref{p28}) 
наблюдается эффект убывания или возрастания~$D_t^Y$ и~$K_t^{UY}$. При этом 
дифференциальное уравнения для~$m_t^Y$ в~силу~(\ref{p22}) и~(\ref{p24}) 
имеет следующий вид:
    \begin{equation*}
    \left(\dot m_t^{Y}\right)^2 =-\alp^2 \!\left(m_t^{Y}\right)^{2}\! + 
    \alp^2 D_*^U - \alp^2\! \left[ 1+ 
    \fr{\alp^2 \left(m_t^{Y}\right)^{2}}{\left(\dot m_t^{Y}\right)^{2}}\right]\!\! D_t^Y.
%    \label{p31}
    \end{equation*}

Таким образом, уравнения для математического ожидания~$m_t^Y$ и~дисперсии~$D_t^Y$ 
взаимосвязаны и~требуют совместного интегрирования. За счет стохастичности 
параметра~$U$ уравнение для~$m_t^Y$ будет дифференциальным уравнением 4-го порядка, 
не разрешенным относительно производной. При отсутствии стохастичности па\-ра\-мет\-ра~$U_t$ 
решение для~$m_t^Y$ будет гармонической функцией периода~$2\pi/\alp$.

Теперь применим МНА, учитывая, что левая часть первого уравнения~(\ref{p1}) является 
гладкой функцией. Во-пер\-вых, продифференцируем сполна по времени левую и~правую 
части первого уравнения~(\ref{p1}). 
Во-вто\-рых, применим формулу Ито для квадратической 
функции
    \begin{multline*}
    dU_t^2 = \lk 2 U_t \left( a_0^U + a_1^U U_t \right) + 
    \left(b_0^U\right)^2 \nu \rk dt + 2 U_t b_0^U dW_t\,,\\
    \dot W_t = V_t\,. %\label{p32}
    \end{multline*}
В результате для переменных $Y_t \hm= \bar Y_1$, 
$\dot Y_t\hm =\dot{\bar Y}_1 \hm= \bar Y_2$ и~$ \bar Y_3 \hm= U_t$ получим 
следующую точную нелинейную систему стохастических уравнений с~параметрическим шумом:

\noindent
    \begin{align*}
    \dot{\bar Y}_1&=\bar Y_2\,;\\ 
    \dot{\bar Y}_2& =-\alp^2 \bar Y_1 + 
    \alp^2 \lk \fr{\nu}{2} \left(b_0^U\right)^2 + a_0^U \bar Y_3 + 
    a_1^U \bar Y_3^2\rk Y_2^{-1}+ {}\\
    &\hspace*{50mm}{}+\alp^2 b_0^U \bar Y_3 Y_2^{-1} V_t\,;
   \\
    \dot{\bar Y}_3 &= a_0^U + a_1 \bar Y_3 + b_0^U V_t\,. 
%    \label{p33}
    \end{align*}
Это точная система, она разрешена относительно производных 
и~решается стандартным МНА на основе теоремы~4. 
В~отличие от МАМ по МСЛ, МАМ на основе МНА позволяет рассматривать случай, 
когда $m_t^Y\hm=0$.

}

{\small\frenchspacing
 {%\baselineskip=10.8pt
 \addcontentsline{toc}{section}{References}
 \begin{thebibliography}{9}

\bibitem{1-sin}
 \Au{Пугачёв В.\,С., Синицын~И.\,Н.}
Стохастические дифференциальные системы. Анализ и~фильтрация.~--- М.:
Наука,  1990.  632~с. 
 (\Au{Pugachev~V.\,S., Sinitsyn~I.\,N.} Stochastic differential systems.
Analysis and filtering.~--- Chichester\,--\,New York, NY, USA: Jonh Wiley, 1987.
549~p.)

\bibitem{2-sin} 
\Au{Пугачёв В.\,С., Синицын И.\,Н.}
Теория стохастических систем.~--- М.: Логос, 2000; 2004. 1000~с.
%[Англ. пер. Stochastic Systems. Theory and  Applications. --
%Singapore: World Scientific, 2001. 908~p.].

\bibitem{3-sin}
\Au{Синицын И.\,Н.}
Канонические представления случайных функций и~их применение 
в~задачах компьютерной поддержки научных исследований.~--- М.: ТОРУС
ПРЕСС, 2009. 768~с.


\bibitem{4-sin}
\Au{Синицын~И.\,Н.,  Синицын~В.\,И. }
Лекции по нормальной и~эллипсоидальной аппроксимации распределений 
в~стохастических системах.~--- М.: ТОРУС ПРЕСС, 2013. 488~с.

\bibitem{5-sin}
\Au{Синицын И.\,Н. }
Параметрическое статистическое и~аналитическое моделирование распределений в~нелинейных 
стохастических системах на многообразиях~// Информатика и~её применения, 2013. 
Т.~7. Вып.~2. С.~4--16.

\bibitem{6-sin}
\Au{Синицын И.\,Н., Синицын~В.\,И.}
Аналитическое моделирование нормальных процессов в~стохастических системах со 
сложными нелинейностями~// Информатика и~её применения, 2014. Т.~8. Вып.~3. С.~2--4.

\bibitem{7-sin}
\Au{Синицын И.\,Н., Синицын~В.\,И., Сергеев~И.\,В., Корепанов~Э.\,Р.,  
Белоусов~В.\,В., Шоргин~В.\,С.}
Математическое обеспечение аналитического моделирования стохастических систем со 
сложными нелинейностями~// Системы и~средства информатики, 2014. Т.~24. №\,3. С.~4--29.


\bibitem{8-sin}
Справочник по специальным функциям~/ Под ред. М.~Абрамовича, И.~Стигана.~--- 
М.: Наука, 1979. 832~с.

\bibitem{9-sin}
\Au{Попов Б.\,А., Теслер~Г.\,С. }
Вычисление функций на ЭВМ: Справочник.~--- Киев: Наукова думка, 1984. 599~с.
 \end{thebibliography}

 }
 }

\end{multicols}

\vspace*{-3pt}

\hfill{\small\textit{Поступила в~редакцию 25.10.16}}

%\vspace*{8pt}

\newpage

\vspace*{-24pt}

%\hrule

%\vspace*{2pt}

%\hrule

%\vspace*{8pt}


\def\tit{ANALYTICAL MODELING OF~WIDE BAND~PROCESSES 
IN~STOCHASTIC~SYSTEMS WITH~UNSOLVED DERIVATIVES}

\def\titkol{Analytical modeling of~processes in~stochastic
systems with~unsolved derivatives}

\def\aut{I.\,N. Sinitsyn}

\def\autkol{I.\,N. Sinitsyn}

\titel{\tit}{\aut}{\autkol}{\titkol}

\vspace*{-9pt}


 \noindent
Institute of Informatics Problems, Federal Research Center 
``Computer Science and Control'' of the Russian Academy of Sciences,
44-2~Vavilov Str., Moscow 119333, Russian Federation



\def\leftfootline{\small{\textbf{\thepage}
\hfill INFORMATIKA I EE PRIMENENIYA~--- INFORMATICS AND
APPLICATIONS\ \ \ 2017\ \ \ volume~11\ \ \ issue\ 1}
}%
 \def\rightfootline{\small{INFORMATIKA I EE PRIMENENIYA~---
INFORMATICS AND APPLICATIONS\ \ \ 2017\ \ \ volume~11\ \ \ issue\ 1
\hfill \textbf{\thepage}}}

\vspace*{3pt}

\Abste{Finite dimensional nonlinear nonstationary and stationary stochastic 
systems with unsolved derivatives with broadband Gaussian and non-Gaussian 
noises are considered. Such equations describe technical devices in informatics 
and control. Two methods of analytical modeling and sensitivity analysis 
of nonstationary and stationary processes based on the methods of statistical 
linearization (MSL) and normal approximation (MNA)  are developed. 
Typical nonlinearities with unsolved derivatives and MSL coefficients 
and test examples are given. The developed analytical modeling 
algorithms based on MSL are the basis of software tools for modeling reliability 
and security of technical devices.}

\KWE{analytical modeling;
method of normal approximation (MNA);
method of statistical linearization (MSL);
normal (Gaussian) stochastic process;
sensitivity to parameters;
stochastic system with unsolved derivatives}



\DOI{10.14357/19922264170101}  

%\vspace*{-9pt}

\Ack
The work was supported  by the Russian Foundation for Basic
Research (project 15-07-02244).



%\vspace*{3pt}

  \begin{multicols}{2}

\renewcommand{\bibname}{\protect\rmfamily References}
%\renewcommand{\bibname}{\large\protect\rm References}

{\small\frenchspacing
 {%\baselineskip=10.8pt
 \addcontentsline{toc}{section}{References}
 \begin{thebibliography}{9}


\bibitem{1-sin-1}
 \Aue{Pugachev, V.\,S., and I.\,N.~Sinitsyn.} 
1987. \textit{Stochastic differential systems.
Analysis and filtering}. Chichester\,--\,New York, NY: Jonh Wiley.
549~p.


\bibitem{2-sin-1}
\Aue{Pugachev, V.\,S., and I.\,N.~Sinitsyn.} 2000, 2004.
Teoriya stokhasticheskikh sistem [Stochastic systems. Theory and  applications]. 
Moscow: Logos. 1000~p.  %[Angl. per. . -- Singapore: World Scientific, 2001].

\bibitem{3-sin-1}
\Aue{Sinitsyn,    I.\,N.}   2009.    
Kanonicheskie   predstavleniya sluchaynykh  funktsiy i~ikh primenenie 
v~zadachakh komp'yutcrnoy podderzhki nauchnykh issledovaniy 
[Canonical expansions of random functions and their application to scientific 
computer-aided support]. Moscow: TORUS PRESS. 768~p.

\bibitem{4-sin-1}
\Aue{Sinitsyn, I.\,N., and V.\,I.~Sinitsyn}. 2013. 
Lektsii po normal'noy i~ellipsoidal'noy approksimatsii raspredeleniy 
v~stokhasticheskikh sistemakh [Lectures on normal and ellipsoidal approximation 
of distributions in stochastic systems). Moscow: TORUS PRESS. 488~p.

\bibitem{5-sin-1}
\Aue{Sinitsyn,   I.\,N.}  2013.  Parametricheskoe statisticheskoe
i~analiticheskoe   modelirovanie   raspredeleniy   
v~nelineynykh stokhasticheskikh sistemakh na mnogoobraziyakh 
[Parametric statistical and analytical modeling of
distributions in stochastic systems on manifolds]. 
\textit{Informatika i~ee Primeneniya~--- Inform. Appl.} 7(2):4--16.

\bibitem{6-sin-1}
\Aue{Sinitsyn, I.\,N., and V.\,I.~Sinitsyn}. 2014. Analiticheskoe
modelirovanie normal'nykh protsessov v~stokhasticheskikh 
sistemakh so slozhnymi nelineynostyami [Analytical
modeling of normal processes in stochastic systems with
complex nonlinearities]. \textit{Informatika i~ee Primeneniya~--- 
Inform. Appl.} 8(3):2--4.

\bibitem{7-sin-1}
\Aue{Sinitsyn, I.\,N., V.\,I.~Sinitsyn, I.\,V.~Sergeev, E.\,R.~Korepanov, 
V.\,V.~Belousov, and V.\,S.~Shorgin}. 2014. Matematicheskoe obespechenie
analiticheskogo modelirovaniya sto\-kha\-sti\-che\-skikh 
sistem so slozhnymi nelineynostyami 
[Mathematical software for analytical modeling of stochastic systems with complex 
nonlinearities]. 
\textit{Sistemy i~Sredstva Informatiki~--- Systems and Means of Informatics} 24(3):4--29.


\bibitem{8-sin-1}
Abramovich, M., and I.~Stigan, eds. 1979. \textit{Spravochnik po
spetsial'nym funktsiyam} [Handbook on special functions].
Moscow: Nauka. 832~p.

\bibitem{9-sin-1}
\Aue{Popov, B.\,A., and G.\,S.~Tesler}. 1984. \textit{Vychislenie funk\-tsiy
na EVM: Spravochnik} [Computing of functions: Handbook]. Kiev:
Naukova dumka. 599~p. 
\end{thebibliography}

 }
 }

\end{multicols}

\vspace*{-3pt}

\hfill{\small\textit{Received October 25, 2016}}

\Contrl

\noindent
\textbf{Sinitsyn Igor N.} (b.\ 1940)~---
Doctor of Science in technology, professor, Honored scientist of RF, 
principal scientist, Institute of Informatics Problems, Federal Research Center 
``Computer Science and Control'' of the Russian Academy of Sciences, 44-2~Vavilov 
Str., Moscow 119333, Russian Federation; \mbox{sinitsin@dol.ru}




\label{end\stat}


\renewcommand{\bibname}{\protect\rm Литература} 