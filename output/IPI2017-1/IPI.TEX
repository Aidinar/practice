\documentclass[10pt]{book}
\usepackage[utf8]{inputenc}

\usepackage{latexsym,amssymb,amsfonts,amsmath,indentfirst,shapepar,%fleqn,%
picinpar,shadow,floatflt,enumerate,multicol,colortbl,moreverb,ipi}

\usepackage{rotating}
\usepackage{mathrsfs}
\usepackage[noend]{algorithmic}
\usepackage{ulem}
%\usepackage{graphicx}
%\usepackage{algorithm2e}

\input{epsf}

%\nofiles

%\includeonly{avtor} %+pdf
%\includeonly{obchak,avtor}
%\includeonly{pred}                 %+
%\includeonly{podgot-rus,podgot-eng}  
%\includeonly{ocherk} 
%\includeonly{nekrol} 
%\includeonly{ipi-ind} 
%\includeonly{toc-rus}
%\includeonly{toc-en} 


%\includeonly{sinit}                %1+pdfотпр
%\includeonly{borisov}              %2+pdfотпр
%\includeonly{stefan+sushko}        %3+pdfотрп
%\includeonly{frenkel}              %4+pdfотпр
%\includeonly{gorshenin}            %5+pdfотпр
%\includeonly{dokukin}              %6+pdfотпр
%\includeonly{gorb-lub}             %7+pdfотпр
%\includeonly{alex}                 %8+pdfотпр
%\includeonly{zatsman}              %9+pdfотпр
%\includeonly{kabanov}              %10+pdfотпр
%\includeonly{seif-mul}             %11+pdfотпр



%\includeonly{toc-rus, toc-en}
%\includeonly{obchak} %,toc-en}

%\includeonly{rekl}
%\includeonly{rekl-1}
%\includeonly{reshal}  %
%\includeonly{eng-index}
%\includeonly{cover3}

\usepackage{acad}
%\usepackage{courier}
\usepackage{decor}
\usepackage{newton}
\usepackage{pragmatica}
\usepackage{zapfchan}
\usepackage{petrotex}
\usepackage{bm}                     % полужирные греческие буквы
\usepackage{upgreek}                % прямые греческие буквы
\usepackage{eufrak}
\usepackage{verbatim}

\renewcommand{\bottomfraction}{0.99}
\renewcommand{\topfraction}{0.99}
\renewcommand{\textfraction}{0.01}

\setcounter{secnumdepth}{1} %здесь - 3 + chapter = 4

\arraycolsep=1.5pt

%\usepackage[pdftex]{graphicx}

%\usepackage{oz}

%NEW COMMANDS


\renewcommand*{\hm}[1]{#1\nobreak\discretionary{}%
            {\hbox{$\mathsurround=0pt #1$}}{}} %% Дублирует знаки операций
                               %при переносе в формуле (перед знаком, который
                               %надо продублировать ставится команда \hm)

%\newcommand{\endproof}{\hfill$\Box$}
%\renewcommand{\r}{\mathbb{R}}
\newcommand{\I}{{\rm I\hspace{-0.7mm}I}}
%\newcommand{\Ikl}{{\tt{1}}\hspace*{-1.44mm}\mathtt{1}}
\newcommand{\Ik}{\mbox{{\small \tt {1}}\hspace{-1.3mm}{\tt 1}}}
\newcommand{\argmin}{\mathop{\mathrm{arg}\,\mathrm{min}}}
\newcommand{\argmax}{\mathop{\mathrm{arg}\,\mathrm{max}}}
%\newcommand{\capr}{\mathop{\cap\,}}
%\newcommand{\cupr}{\mathop{\cup\,}}
%\def\argmin{\mathop{arg\,min}}

\def\vrp{\varphi}
\def\prt{\partial}
\def\mm{{\sf M}}
\def\modnop#1{\mathop{#1}\limits_{n}}
\def\eam{\mathbin{{\mathop{=}\limits^{\mathrm{def}}}}}
\def\dey#1#2{#1 (#2)}
\def\deyc#1#2{#1 \cdot  #2}
\def\ra#1{\;\mathop{\to}\limits^{#1}\;}
\def\raz#1{\;\mathop{\longrightarrow}\limits^{\!\!\!#1}\;}
\def\ral#1{\;\mathop{\longrightarrow}\limits^{#1}\;}

\newcommand{\Nor}{\mathcal{N}}
\newcommand{\T}{\mathbb{T}}
\newcommand{\Z}{\mathbb{Z}}



\newcommand{\il}[2]{\int\limits_{#1}^{#2}}%интеграл с пределами #1 и #2

\def\sm2{\mathop {\sum\limits^{n^\Theta}\sum\limits^{n^\Theta}}}
\def\sss{\sum\limits}
\def\tr{,\,\ldots\,,\,}
\def\rk{\right]}
\def\lk{\left[}
\def\rf{\right\}}
\def\lf{\left\{}
\def\lv{\,\left\vert}
\def\rv{\right\vert\,}
\def\iii{\int\limits}
\def\iin{\int\limits_{-\infty}^\infty}
\def\rrv{\right\vert}


\def\ee{{\cal E}}
\def\ww{{\cal W}}
\def\yy{{\cal Y}}
\def\vv{{\cal V}}

\newcommand{\R}{\mathbb R}
\newcommand{\E}{\mathbb E}
\newcommand{\N}{\mathbb N}

\renewcommand{\P}{\mathbb{P}}

\newcommand{\h}{{\bf H}}
\newcommand{\p}{{\sf P}}  % вероятность

\newcommand{\e}{{\sf E}}  % мат. ожидание
\newcommand{\D}{{\sf D}}  % дисперсия
\newcommand{\eps}{\varepsilon}
\newcommand{\vp}{{\mathbf p}}
\newcommand{\vz}{{\mathbf z}}
\newcommand{\vx}{{\mathbf x}}
\newcommand{\vf}{{\mathbf f}}
\newcommand{\F}{{\mathcal F}}
\def\ap{{\mathrm{ЭР}}}
\newcommand{\ud}{\Delta_n} %uniform ditance
\newcommand{\nud}{\Delta_n(x)}
\renewcommand{\Re}{\mathrm{Re}\,}

\newcommand{\abs}[1]{\left\vert#1\right\vert}

\newcommand{\norm}[1]{\left\Vert#1\right\Vert}
\def\da{(\Delta_t,A)}

\newcommand{\corr}{\mathrm{corr}}

\newcommand{\cov}{\mathrm{cov}}
\newcommand{\Expect}{\mathbb{E}}

\def\w{\omega}
\def\W{\Omega}

\def\inh{\int\limits_{nh}^{(n+1)h}}

\def\sumin{\sum_{i=1}^N}


\def\bxt{(Y,t)}
\def\xt{(y,t)}

\def\ovth{{\fr{\tau-nh}{h}}}
\def\ov{\overline}
\def\tm{\tilde m}
\def\tl{\tilde\lambda}
\def\tB{\widetilde B}
\def\tb{\tilde b}
\def\ld{\ldots}
\def\cd{\cdots}


\DeclareMathOperator{\sign}{sign}

%\newcommand{\gr}{{\geqslant}}


\newcommand{\g}{\mbox{\textit{g}}}

\renewcommand{\la}{\lambda}
\newcommand{\si}{\sigma}
\newcommand{\alp}{\alpha}

%\newcommand{\pto}{\stackrel{P}{\longrightarrow}} % сходимость по веpоятности

\newcommand{\eqd}{\stackrel{\mathrm{d}}{=}} % равенство по pаспpеделению
\newcommand{\eqdelta}{\stackrel{\Delta}{=}} % равенство по pаспpеделению

\def\be#1{\begin{equation}\label{#1}}
\def\ee{\end{equation}}
\def\re#1{(\ref{#1})}

\def\bn{\begin{enumerate}}
\def\en{\end{enumerate}}
\def\bi{\begin{itemize}}
\def\ei{\end{itemize}}
%\def\i{\item}

%\newcommand{\kp}{\kappa}
%\def\Q{{\cal Q}} \def\H{{\cal H}}
%\newcommand{\bet}{\beta_{2+\delta}}


%\newtheorem{definition}{Определение}
%\renewcommand{\thedefinition}{\arabic{definition}.}
%END NEW COMMANDS

%\renewcommand{\baselinestretch}{1.2}

%\pagestyle{myheadings}

\setlength{\textwidth}{167mm}      % 122mm
\setlength{\textheight}{658pt}
%\setlength{\textheight}{635.6pt}
\setlength{\columnsep}{4.5mm}

\setcounter{secnumdepth}{4}

%\addtolength{\headheight}{2pt}
%\addtolength{\headsep}{-2mm}

\addtolength{\topmargin}{-7mm}  % for printing


%\hoffset=-30mm  % From Yap
\hoffset=-23mm  % From Acrobat

%\voffset=0mm % From Yap
\voffset=-5mm   % From Acrobat

%\addtolength{\evensidemargin}{-2.5mm} % for printing
%\addtolength{\oddsidemargin}{2.5mm}  % for printing

\addtolength{\evensidemargin}{-12mm} % for printing
\addtolength{\oddsidemargin}{8mm}  % for printing

%\renewcommand{\thefootnote}{\fnsymbol{footnote}}
%\renewcommand{\thefootnote}{\arabic{footnote}}
\renewcommand{\figurename}{\protect\bf Рис.}
\renewcommand{\tablename}{\protect\bf Таблица}

\newcommand{\Caption}[1]{\caption{\protect\small %\baselineskip=2.5ex
#1}}

\renewcommand{\thefigure}{\arabic{figure}}
\renewcommand{\thetable}{\arabic{table}}
\renewcommand{\theequation}{\arabic{equation}}
\renewcommand{\thesection}{\arabic{section}}

\renewcommand{\contentsname}{СОДЕРЖАНИЕ}
\newcommand{\fr}[2]{\displaystyle\frac{\displaystyle #1\mathstrut}{\displaystyle #2\mathstrut}}

%\renewcommand{\thefootnote}{\fnsymbol{footnote}}
%\newcommand{\g}{\mbox{\textit{g}}}

%\newcommand{\Caption}[1]{\caption{\protect\small\baselineskip=2ex #1}}
\newcounter{razdel}
\setcounter{razdel}{0}


\newcommand{\titel}[4]{%
\

\vspace*{5pt}

\ifodd\therazdel {\raggedright\noindent\Large\textrm\textbf
 \lineskip .75em
  \baselineskip=3.2ex #1 \par}
\vskip 1em {\noindent\large\textrm\textbf #2 \par}
\addcontentsline{toc}{subsection}{{\textrm\textbf #1}\protect\newline #2}
\def\rightheadline{\underline{\noindent\hbox to \textwidth{\hfill\small\textrm{#4}
%\hfill \large\bf\thepage
}}}
\def\leftheadline{\underline{\noindent\parbox{\textwidth}{
%\raggedleft\large\bf\thepage \hfill
\small\textit{#3}\hfill}}}
\def\leftfootline{\small{\textbf{\thepage}
\hfill ИНФОРМАТИКА И ЕЁ ПРИМЕНЕНИЯ\ \ \ том~11\ \ \ выпуск 1\ \ \ 2017}
}%
 \def\rightfootline{\small{ИНФОРМАТИКА И ЕЁ ПРИМЕНЕНИЯ\ \ \ том~11\ \ \ выпуск~1\ \ \ 2017
\hfill \textbf{\thepage}}}
\vskip 2em \setcounter{figure}{0}
\setcounter{table}{0}
\setcounter{equation}{0}
\setcounter{section}{0}
\setcounter{subsection}{0}
\setcounter{subsubsection}{0}
\setcounter{footnote}{0}
\setcounter{razdel}{0}
%\end{flushleft}
\else {
 \raggedright\noindent\Large\textrm\textbf
 \lineskip .75em
\baselineskip=3.2ex #1 \par} \vskip 1em
%\begin{flushleft}
{\noindent\large\textrm\textbf #2 \par}
\addcontentsline{toc}{subsection}{{\textrm\textbf #1}\protect\newline #2}
\def\rightheadline{\underline{\noindent\hbox to \textwidth{\hfill\small\textrm{#4}
%\hfill \large\bf\thepage
}}}
\def\leftheadline{\underline{\noindent\parbox{\textwidth}{%\raggedleft\large\bf\thepage \hfill
\small\textit{#3}\hfill}}}
\def\leftfootline{\small{\textbf{\thepage}
\hfill ИНФОРМАТИКА И ЕЁ ПРИМЕНЕНИЯ\ \ \ том~11\ \ \ выпуск~1\ \ \ 2017}
}%
 \def\rightfootline{\small{ИНФОРМАТИКА И ЕЁ ПРИМЕНЕНИЯ\ \ \ том~11\ \ \ выпуск~1\ \ \ 2017
\hfill \textbf{\thepage}}} \vskip 2em \setcounter{figure}{0}
\setcounter{table}{0} \setcounter{equation}{0} \setcounter{section}{0}
\setcounter{subsection}{0} \setcounter{subsubsection}{0}
\setcounter{footnote}{0}
%\end{flushleft}
\fi}

\newcommand{\titelr}[2]{%
\

\vspace*{5pt}

\ifodd\therazdel {\raggedright\noindent%\Large\textrm\textbf
 \lineskip .75em
  \baselineskip=3.2ex #1 \par}
\vskip 1em {\noindent\normalsize\textrm\textbf #2 \par}
\else {
 \raggedright\noindent\Large\textrm\textbf
 \lineskip .75em
\baselineskip=3.2ex #1 \par} \vskip 1em
%\begin{flushleft}
{\noindent\large\textrm\textbf #2 \par
%\noindent\normalsize\textrm\textbf #2 \par
} \fi}

\newcommand{\titele}[5]{%
\

%\vspace*{5pt}

\ifodd\therazdel {\raggedright\noindent\large
\textrm\textbf
 \lineskip .75em
%  \baselineskip=3.2ex
#1 \par}
\vskip .5em {\noindent\large\textrm\textbf #2 \par}
\vskip .5em
 {\noindent\textrm #3 \par}
\addcontentsline{toc}{subsection}{{\textrm\textbf #1}\protect\newline #2}
\def\rightheadline{\underline{\noindent\hbox to \textwidth{\hfill\small\textrm{#4}
%\hfill \large\bf\thepage
}}}
\def\leftheadline{\underline{\noindent\parbox{\textwidth}{
%\raggedleft\large\bf\thepage \hfill
\small\textrm{#5}\hfill}}}
\def\leftfootline{\small{\textbf{\thepage}
\hfill ИНФОРМАТИКА И ЕЁ ПРИМЕНЕНИЯ\ \ \ том~11\ \ \ выпуск~1\ \ \ 2017}
}%
 \def\rightfootline{\small{ИНФОРМАТИКА И ЕЁ ПРИМЕНЕНИЯ\ \ \ том~11\ \ \ выпуск~1\ \ \ 2017
\hfill \textbf{\thepage}}} \vskip 1em \setcounter{figure}{0}
\setcounter{table}{0} \setcounter{equation}{0} \setcounter{section}{0}
\setcounter{subsection}{0} \setcounter{subsubsection}{0}
\setcounter{footnote}{0} \setcounter{razdel}{0}
%\end{flushleft}
\else {
 \raggedright\noindent\large
 \textrm\textbf
 \lineskip .75em
%\baselineskip=3.2ex
#1 \par} \vskip .5em
%\begin{flushleft}
{\noindent\large\textrm\textbf #2 \par} \vskip .5em
 {\noindent\textrm #3 \par}
\addcontentsline{toc}{subsection}{{\textrm\textbf #1}\protect\newline #2}
\def\rightheadline{\underline{\noindent\hbox to \textwidth{\hfill\small\textrm{#4}
%\hfill \large\bf\thepage
}}}
\def\leftheadline{\underline{\noindent\parbox{\textwidth}{%\raggedleft\large\bf\thepage \hfill
\small\textrm{#5}\hfill}}}
\def\leftfootline{\small{\textbf{\thepage}
\hfill ИНФОРМАТИКА И ЕЁ ПРИМЕНЕНИЯ\ \ \ том~11\ \ \ выпуск~1\ \ \ 2017}
}%
 \def\rightfootline{\small{ИНФОРМАТИКА И ЕЁ ПРИМЕНЕНИЯ\ \ \ том~11\ \ \ выпуск~1\ \ \ 2017
\hfill \textbf{\thepage}}} \vskip 1em \setcounter{figure}{0}
\setcounter{table}{0} \setcounter{equation}{0} \setcounter{section}{0}
\setcounter{subsection}{0} \setcounter{subsubsection}{0}
\setcounter{footnote}{0}
%\end{flushleft}
\fi}

\def\Abst#1{
\begin{center}\small\nwt
\parbox{150mm}{%\baselineskip=2.5ex
\textbf{Аннотация:}\ \
%\hspace*{\parindent}
#1}
\end{center}}
\def\Abste#1{
\begin{center}\small\nwt
\parbox{150mm}{%\baselineskip=2.5ex
\textbf{Abstract:}\ \
%\hspace*{\parindent}
#1}
\end{center}}

\def\DOI#1{
\begin{center}\small\nwt
\parbox{150mm}{%\baselineskip=2.5ex
\textbf{DOI:}\ \
%\hspace*{\parindent}
#1}
\end{center}}

\def\Abstend#1{
\begin{center}\small\nwt
\parbox{150mm}{%\baselineskip=2.5ex
%\hspace*{\parindent}
#1}
\end{center}}


\def\KW#1{
\begin{center}\small\nwt
\parbox{150mm}{%\baselineskip=2.5ex
\textbf{Ключевые слова:}\ \ #1}
\end{center}}

\def\KWE#1{
\begin{center}\small\nwt
\parbox{150mm}{%\baselineskip=2.5ex
\textbf{Keywords:}\ \ #1}
\end{center}}


\def\KWN#1{
%\begin{center}
%\small
%\parbox{150mm}\end{center}
}

\newcommand{\Avtors}[1]{%\smallskip
%\vspace*{.5pt}
\hangindent=23pt\noindent
%\nwt
{\bfseries#1}\
}


\renewcommand{\thesubsection}{\thesection.\arabic{subsection}\hspace*{-5pt}}
\renewcommand{\thesubsubsection}{\thesubsection\hspace*{5pt}.\arabic{subsubsection}\hspace*{-3pt}}

\newcommand{\Ack}{\section*{\protect\rmfamily Acknowledgments}\noindent}
\newcommand{\Contr}{\section*{\protect\rmfamily Contributors}\noindent}
\newcommand{\Contrl}{\section*{\protect\rmfamily Contributor}\noindent}

\makeindex


\begin{document}
\Rus

\nwt
%\ptb


%\renewcommand{\contentsname}{\protect\Large\bf Содержание}

\setcounter{tocdepth}{2}

%\tableofcontents

\renewcommand{\bibname}{\protect\rmfamily Литература}
  \def\Au#1{{\it #1}}
    \def\Aue#1{{#1}}

%\newcommand{\No}{№}
  \newcommand{\tg}{\,\mathrm{tg}\,}
    \newcommand{\ctg}{\,\mathrm{ctg}\,}
  \newcommand{\arctg}{\,\mathrm{arctg}\,}

\def\forallb{\mathop{\forall}}
\def\cupb{\mathop{\cup}}
\def\existsb{\mathop{\exists}}


\newpage
\addtocounter{razdel}{1}
%\def\razd{РЕГУЛИРУЕМЫЙ ЭЛЕКТРОПРИВОД ДЛЯ ЭЛЕКТРОЭНЕРГЕТИКИ}


\setcounter{page}{2}

   { %\Large  
   { %\baselineskip=16.6pt
   
   \vspace*{-48pt}
   \begin{center}\LARGE
   \textit{Предисловие}
   \end{center}
   
   %\vspace*{2.5mm}
   
   \vspace*{25mm}
   
   \thispagestyle{empty}
   
   { %\small 

    
Вниманию читателей журнала <<Информатика и её применения>> предлагается 
очередной тематический выпуск <<Вероятностно-статистические методы и 
задачи информатики и информационных технологий>>. Предыдущие тематические 
выпуски журнала по данному направлению вышли в 2008~г.\ (т.~2, вып.~2), 
в 2009~г.\ (т.~3, вып.~3) и в 2010~г.\ (т.~4, вып.~2). 

Статьи, собранные в данном журнале, посвящены разработке новых вероятностно-статистических 
методов, ориентированных на применение к решению конкретных задач информатики и информационных 
технологий, а также~--- в ряде случаев~--- и других прикладных задач. Проблематика, охватываемая 
публикуемыми работами, развивается в рамках научного сотрудничества между Институтом проблем 
информатики Российской академии наук (ИПИ РАН) и Факультетом вычислительной математики и 
кибернетики Московского государственного университета им.\ М.\,В.~Ломоносова в ходе работ 
над совместными научными проектами (в том числе в рамках функционирования 
Научно-образовательного центра <<Вероятностно-статистические методы анализа рисков>>). 
Многие из авторов статей, включенных в данный номер журнала, являются активными участниками 
традиционного международного семинара по проблемам устойчивости стохастических моделей, 
руководимого В.\,М.~Золотаревым и В.\,Ю.~Королевым; регулярные сессии этого семинара 
проводятся под эгидой МГУ и ИПИ РАН (в 2011~г.\ указанный семинар проводится в октябре 
в Калининградской области РФ). 

Наряду с представителями ИПИ РАН и МГУ в число авторов данного выпуска журнала входят 
ученые из Научно-исследовательского института системных исследований РАН, Института 
проблем технологии микроэлектроники и особочистых материалов РАН, Института 
прикладных математических исследований Карельского НЦ РАН, Московского 
авиационного института, Вологодского государственного педагогического университета, 
НИИММ им.\ Н.\,Г.~Чеботарева, Казанского государственного университета, Дебреценского 
университета (Венгрия).

Несколько статей выпуска посвящено разработке и применению стохастических методов и 
информационных технологий для решения различных прикладных задач. В~работе В.\,Г.~Ушакова 
и О.\,В.~Шестакова рассмотрена задача определения вероятностных характеристик случайных 
функций по распределениям интегральных преобразований, возникающих в задачах эмиссионной 
томографии. В~статье Д.\,О.~Яковенко и М.\,А.~Целищева рассмотрены некоторые вопросы 
математической теории риска и предложен новый подход к диверсификации инвестиционных 
портфелей. Работа И.\,А.~Кудрявцевой и А.\,В.~Пантелеева посвящена построению и 
исследованию математической модели, описывающей динамику сильноионизованной плазмы. 
В~статье П.\,П.~Кольцова изучается качество работы ряда алгоритмов сегментации изображений. 
Статья А.\,Н.~Чупрунова и И.~Фазекаша посвящена вероятностному анализу числа без\-оши\-бочных 
блоков при помехоустойчивом кодировании; получены усиленные законы больших чисел для указанных 
величин.

В данном выпуске традиционно присутствует тематика, весьма активно разрабатываемая в течение 
многих лет специалистами ИПИ РАН и МГУ,~--- методы моделирования и управления для 
информационно-телекоммуникационных и вычислительных систем, в частности методы 
теории массового обслуживания. В~статье А.\,И.~Зейфмана с соавторами рассматриваются 
модели обслуживания, описываемые марковскими цепями с непрерывным временем в случае 
наличия катастроф. В~работе М.\,М.~Лери и И.\,А.~Чеплюковой рассматриваются случайные 
графы Интернет-типа, т.\,е.\ графы, степени вершин которых имеют степенные распределения; 
такие задачи находят применение при исследовании глобальных сетей передачи данных. 
Работа Р.\,В.~Разумчика посвящена исследованию систем массового обслуживания специального 
вида~--- с отрицательными заявками и хранением вытесненных заявок.

Ряд статей посвящен развитию перспективных теоретических 
вероятностно-статистических методов, которые находят широкое применение в различных 
задачах информатики и информационных технологий. В~работе В.\,Е.~Бенинга, А.\,К.~Горшенина 
и В.\,Ю.~Королева рассмотрена задача статистической проверки гипотез о числе компонент 
смеси вероятностных распределений, приводится конструкция асимптотически наиболее мощного 
критерия. Результаты этой работы найдут применение в ряде прикладных задач, использующих 
математическую модель смеси вероятностных распределений (в информатике, моделировании 
финансовых рынков, физике турбулентной плазмы и~т.\,д.). В~статье В.\,Ю.~Королева, 
И.\,Г.~Шевцовой и С.\,Я.~Шоргина строится новая, улучшенная оценка точности нормальной 
аппроксимации для пуассоновских случайных сумм; как известно, указанные случайные суммы 
широко используются в качестве моделей многих реальных объектов, в том числе в информатике, 
физике и других прикладных областях. Работа В.\,Г.~Ушакова и Н.\,Г.~Ушакова посвящена 
исследованию ядерной оценки плотности распределения; эти результаты могут применяться, 
в част\-ности, при анализе трафика в телекоммуникационных системах. Серьезные приложения 
в статистике могут получить результаты работы О.\,В.~Шестакова, в которой доказаны оценки 
скорости сходимости распределения выборочного абсолютного медианного отклонения к нормальному 
закону. 

\smallskip

Редакционная коллегия журнала выражает надежду, что данный тематический  выпуск 
будет интересен специалистам в области теории вероятностей и математической статистики 
и их применения к решению задач информатики и информационных технологий.
     
     %\vfill 
     \vspace*{20mm}
     \noindent
     Заместитель главного редактора журнала <<Информатика и её 
применения>>,\\
     директор ИПИ РАН, академик  \hfill
     \textit{И.\,А.~Соколов}\\
     
     \noindent
     Редактор-составитель тематического выпуска,\\
     профессор кафедры математической статистики факультета\\
      вычислительной математики и кибернетики МГУ им.\ М.\,В.~Ломоносова,\\
     ведущий научный сотрудник ИПИ РАН,\\ 
доктор физико-математических наук \hfill
      \textit{В.\,Ю.~Королев}
     
     } }
     }

\def\stat{sinits}

\def\tit{СТОХАСТИЧЕСКИЕ ИНФОРМАЦИОННЫЕ  ТЕХНОЛОГИИ ДЛЯ~ИССЛЕДОВАНИЯ
НЕЛИНЕЙНЫХ КРУГОВЫХ СТОХАСТИЧЕСКИХ СИСТЕМ$^*$}

\def\titkol{Стохастические информационные  технологии для~исследования
нелинейных круговых стохастических систем}

\def\autkol{И.\,Н.~Синицын}
\def\aut{И.\,Н.~Синицын$^1$}

\titel{\tit}{\aut}{\autkol}{\titkol}

{\renewcommand{\thefootnote}{\fnsymbol{footnote}}\footnotetext[1]
{Работа выполнена при финансовой поддержке РФФИ
(проект №\,10-07-00021) и программы ОНИТ РАН <<Информационные
технологии и анализ сложных систем>> (проект 1.5).}}


\renewcommand{\thefootnote}{\arabic{footnote}}
\footnotetext[1]{Институт проблем информатики Российской академии наук, sinitsin@dol.ru}


\Abst{Статья посвящена стохастическим (корреляционным и спект\-раль\-но-кор\-ре\-ля\-ци\-он\-ным) 
информационным технологиям аналитического и статистического анализа и моделирования 
процессов в нелинейных круговых стохастических системах на базе методов круговой 
статистической линеаризации <<намотанным>> нормальным распределением. В~основу технологий 
положены методы, алгоритмы и инструментальное программное обеспечение (ИПО)
CStS-ANALYSIS в среде  MATLAB.}

\KW{аналитическое моделирование; круговой стохастический процесс;
круговая стохастическая система; круговая статистическая линеаризация;
компьютерная поддержка статистических научных исследований; MATLAB;
корреляционные уравнения; спект\-раль\-но-кор\-ре\-ля\-ци\-он\-ные уравнения;
стохастические информационные технологии; статистическое моделирование}

 \vskip 14pt plus 9pt minus 6pt

      \thispagestyle{headings}

      \begin{multicols}{2}
      
            \label{st\stat}


\section{Введение}
Компьютерная поддержка научных исследований (КПНИ) как
 неотъемлемая часть автоматизации научных исследований
 становится все более характерным признаком современных научных
 исследований (НИ) и оказывает сильное влияние на их интенсивность и
 эффективность, превращается в важнейший фактор дальнейшего
 прогресса науки~[1, 2]. Современный этап развития КПНИ характеризуется  интенсивным
проникновением ее в новые сферы исследований и разработок,
расширением контингента пользователей, охватом всех
этапов исследований от сбора и первичной обработки данных,
управления экспериментами до анализа и перспективного планирования
основных на\-прав\-ле\-ний НИ и их информационных
технологий.

Под информационной технологией обычно понимают совокупность
систематических и массовых способов создания, накопления, обработки,
хранения, передачи и распределения информации (данных, знаний) с
помощью средств вычислительной техники и связи.

 На
практике обычно создается ИТ, рассчитанная на выполнение с ее
по\-мощью некоторой основной функции, что связано с необходимостью
решения нескольких типовых задач исследований.
Перечень основных функций довольно ограничен, а с другой стороны,
выполнение этих функций может потребоваться во многих применениях.
Это делает целесообразным выделение функ\-ци\-о\-наль\-но-ориен\-ти\-ро\-ван\-ных,
предметно-ориентированных и проб\-лем\-но-ориен\-ти\-ро\-ван\-ных ИТ~\cite{1-sin}.

 На примере статистических НИ в~\cite{1-sin} 
 рас\-смот\-ре\-ны современные принципы подходы и задачи КПНИ, сформулированы 
 требования к стохастическим ИТ (СтИТ) анализа, моделирования и синтеза 
 оптимальных, субоптимальных и услов\-но-оп\-ти\-маль\-ных фильтров для обработки 
 информации, описано ИПО, а 
 также некоторые приложения. В~качестве основных математических моделей в 
 СтИТ принимались стохастические дифференциальные, интегральные и смешанные 
 уравнения в евклидовом пространстве, а также их разностные аналоги. Для 
 круговых, сферических, кватернионных и других гипергеометрических 
 стохастических уравнений, относящихся к системам на многообразиях~\cite{3-sin}, 
 известные методы анализа, моделирования и синтеза требуют развития. Однако при 
 этом основные принципы, подходы и задачи статистических НИ сохраняются.

Обзор зарубежного универсального методического и программного обеспечения 
для математической статистики  круговых случайных величин и функций дан в~\cite{4-sin}. 
Отдельные прикладные задачи решены, например, в~[1--8].
В~ИПИ РАН начиная с 2010~г.\ в рамках тем, поддерживаемых РФФИ, 
ведутся работы по созданию методического обеспечения для анализа, 
моделирования и синтеза фильтров для обработки информации в круговых 
стохастических системах (КСтС)~\cite{9-sin, 10-sin}.

Рассмотрим полезные для практики простые квазилинейные, 
основанные на эквивалентной круговой статистической линеаризации (КСЛ), 
корреляционные и спектрально-корреляционные методы, алгоритмы и ИПО для 
оф\-лайн-ана\-ли\-за и моделирования круговых стохастических процессов 
(КСтП) в нелинейных КСтС.

\section{Статистическая линеаризация нелинейных преобразований круговых случайных величин}

Пусть сначала $X$ и $Y$~--- скалярные круговые случайные величины (КСВ), 
связанные между собой детерминированной нелинейной зависимостью
    \begin{equation}
    Y=\vrp (X)\,.\label{e2.1s}
    \end{equation}
Согласно принципу эквивалентной статистической линеаризации заменим нелинейную 
зависимость~(\ref{e2.1s}) приближенной линейной зависимостью:
\begin{equation}
\vrp (X) \approx U =\vrp_0 + k_1 (X-\mu)\,,\label{e2.2s}
\end{equation}
где $\mu=\mu_x$~--- круговое среднее направление КСВ~$X$. 
Параметры $\vrp_0$ и~$k_1$ находят из критерия минимума 
безусловного риска для выбранной функции потерь~$\ell (X,U)$:
\begin{equation}
R= \mm \lk\ell (X,U) \rk =\min \,,\label{e2.3s}
\end{equation}
где $\mm$~--- символ математического ожидания.

Если выбрать эквивалентное одномерное распределение (ЭР) КСВ~$X$ и функцию потерь в виде
\begin{equation}
\ell (X,U) =\left( e^{iX} - e^{iU}\right)^2\,,\label{e2.4s}
\end{equation}
то после подстановки~(\ref{e2.2s}) в~(\ref{e2.3s}) и~(\ref{e2.4s}) 
и приравнивания нулю частных производных $\prt R/\prt \vrp_0$ и $\prt R/\prt k_1$ 
получим одно комплексное уравнение для неизвестных параметров $\vrp_0$ и~$k_1$:
\begin{multline}    
\mm_\ap \exp \lf i \vrp (X)\rf ={}\\
{}=\mm_\ap \exp \lf i\lk \vrp_0 + k_1 (X-\mu)\rk\rf\,,\label{e2.5s}
\end{multline}
где $\mm_\ap$~--- символ математического ожидания по ЭР; 
коэффициенты КСЛ $\vrp_0$ и $k_1$ зависят от вероятностных характеристик КСВ~$X$.

Принимая в качестве ЭР для КСВ $X$ намотанное нормальное распределение с параметрами  
$\mu$ и~$\si$, т.\,е.\ $WN(\mu,\si)$~\cite{4-sin, 7-sin}, 
перепишем комплексное уравнение~(\ref{e2.5s}) в виде двух действительных уравнений:
\begin{equation*}
\vrp_0 (\mu,\si) =\psi (\mu,\si)\,; %\label{e2.6s}
\end{equation*}
\begin{equation*}
k_1 (\mu,\si) =\fr{\sqrt{-2\ln r(\mu,\si)}}{\si}\,, %\label{e2.7s}
\end{equation*}
где введено следующее обозначение: 
$$
re^{i\psi} =\mm_{WN} \exp \lf -i \vrp (X)\rf\,.
$$

Для скалярного нелинейного преобразования векторного аргумента
\begin{equation}
Y=\vrp (X_1\tr X_n)\label{e2.8s}
\end{equation}
при условии, что ЭР вектора КСВ  $X= [ X_1, \ldots$\linebreak 
$\ldots , X_n]^{\mathrm{T}}$ является известным 
намотанным нормальным распределением~\cite{4-sin, 7-sin}, 
уравнения принципа статистической линеаризации по критерию~(\ref{e2.4s}) имеют следующий вид:
\begin{equation*}
\vrp (X) \approx U = \vrp_0 +\sss_{h=1}^n k_{1h} X_h^0\,. %\label{e2.9s}
\end{equation*}
Здесь $\vrp_0$~--- первый векторный коэффициент КСЛ, равный
\begin{equation*}
\vrp_0 = \mm_{WN} \vrp(X)\,, %\label{e2.10s}
\end{equation*}
$k_{1h}$ $(h=1\tr n)$~--- второй векторный коэффициент КСЛ, который 
определяется путем решения алгебраической системы уравнений
\begin{equation*}
\sss_{j=1}^n k_{1h} K_{jh} = \mm_{WN} X_j^0 \vrp (X)\,, %\label{e2.11s}
\end{equation*}
где $K_{1h} =\mm_{WN} X_j^0 X_h^0$\ \,$(j,h\hm=1\tr n)$.

Аналогично выписываются формулы для коэффициентов 
КСЛ для векторных и матричных нелинейных преобразований, 
а также посредством канонических представлений~\cite{1-sin}.

Для типовых нелинейных преобразований~(\ref{e2.1s}) и~(\ref{e2.8s}) 
составлены таблицы и разработано ИПО 
CStS-ANALYSIS~\cite{11-sin}.

\section{Основные результаты}

\noindent
\textbf{Теорема 3.1.} \textit{Пусть нестационарная дифференциальная система}
\begin{equation*}
\dot Y =\vrp (Y,t) +V\,,\quad Y(t_0) = Y_0 %\label{e3.1s}
\end{equation*}
\textit{удовлетворяет следующим допущениям:}
\begin{enumerate}[(1)]
\item \textit{$n$-мер\-ный круговой $($на $[0, 2\pi])$  СтП $Y\hm=Y(t)$ 
обладает конечными вероятностными моментами второго порядка;}
\item
\textit{$n$-мерный круговой белый шум, понимаемый в строгом смысле, $(V=\dot W$, 
$W$~--- КСтП с независимыми приращениями на  $[0, 2\pi]$  и матрицей интенсивности $G(t))$;}
\item
\textit{детерминированное нелинейное преобразование  $\vrp (Y,t)$ не обладает  памятью и допускает 
КСЛ согласно алгоритмам разд.~2, причем статистически линеаризованная система для 
$Y^0 \hm= Y-m_y$:
\begin{equation}
{\dot Y}^0 = k_1 Y^0 + V\quad (m_y = \mm Y)\label{e3.2s}
\end{equation}
асимптотически устойчива.
Тогда корреляционное уравнение квазилинейного анализа и аналитического 
моделирования имеют следующий вид:}
\begin{align*}
\dot m_y &=\vrp_0(m_y, K_y,t)\,;\quad m_y (t_0) = m_{y0}\,;\\ %\label{e3.3s}
\dot K_y &= k_1 (m_y, K_y, t) K_y + {}\\
&\hspace*{8mm}{}+  K_y k_1^{\mathrm{T}} (m_y, K_y, t)+ G(t)\,;\\
K_y (t_0) &= K_{y0}\,;\\
\fr{\prt K_y(t_1, t_2)}{\prt t_2} &= 
K_y (t_1, t_2) k_1^{\mathrm{T}} (m_y, K_y, t_2)\,;\\
K_y (t_1, t_2)&= \begin{cases}
K_y(t_1,t_2) &\ \mbox{при\ \ } t_2>t_1\,;\\
K_y (t_2, t_1)^{\mathrm{T}} &\ \mbox{при\ \ } t_2<t_1\,,
\end{cases} 
\end{align*}
\textit{где  $m_y$, $K_y(t)$ и $K_y(t_1, t_2)$~--- соответственно вектор математического ожидания, 
ковариационная матрица  и матрица ковариационных функций КСтП~$Y(t)$}.
\end{enumerate}


\smallskip

\noindent
\textbf{Теорема 3.2.} \textit{В условиях теоремы}~3.1 
\textit{при стационарных функциях $\vrp (Y,t) \hm=\vrp(Y)$, $G(t) \hm=G$ 
корреляционные уравнения анализа и аналитического моделирования для КСтП $\tilde Y(t)$ 
имеют вид:}
\begin{gather}
\vrp_0 (\tilde m_y ,\tilde K_y)=0\,;\notag %label{e3.6s}
\\[3pt]
k_1 (\tilde m_y, \tilde K_y)\tilde K_y + \tilde K_y k_1 (\tilde m_y, \tilde K_y) + G =0\,\label{e3.7s}
\end{gather}

\vspace*{-3pt}

\noindent
\begin{equation}
\left.
\begin{array}{rl}
\fr{d\tilde k_y(\tau)}{d\tau}  &= k_1 (\tilde m_y , \tilde K_y) \tilde k_y(\tau)\,;\\[9pt]
k_y(\tau)&=\tilde{K}_y(t_1,t_1+\tau)\,,
\end{array}
\right\}
\label{e3.8s}
\end{equation}
\textit{где $\tilde m_y$, $\tilde K_y$ и $\tilde k_y(\tau)$~--- соответственно 
математическое ожидание, ковариационная матрица и матрица 
ковариационных функций $(\tau \hm= t_1 \hm- t_2)$ стационарного КСтП $\tilde Y(t)$}.

\smallskip

\noindent
\textbf{Теорема 3.3.} \textit{В~условиях теоремы}~3.2 \textit{уравнения}~(\ref{e3.7s}) 
\textit{и}~(\ref{e3.8s}), \textit{если вместо ковариационной матрицы $k_y(\tau)$ 
использовать спектральную плотность  $s_y(\w)$, допускают следующее спектральное представление:
\begin{align*}
\tilde K_y &=\iin s_y(\w; \tilde m_y, \tilde K_y)\, d\w\,; %\label{e3.9s}
\\
k_y(\tau) &=\iin e^{i\w\tau} s_y (\w; \tilde m_y, \tilde K_y)\, d\w\,, %\label{e3.10s}
\end{align*}
где $s_y (\w; \tilde m_y, \tilde K_y)$~--- матрица спектральных плот\-ностей:
\begin{equation*}
s_y (\w; \tilde m_y, \tilde K_y) = \Phi (i\w; \tilde m_y, \tilde K_y) 
\fr{G}{2\pi} I_n \Phi (i\w; \tilde m_y, \tilde K_y)^*\,; %\label{e3.11s}
\end{equation*}
$\Phi (i\w; \tilde m_y, \tilde K_y)$~--- передаточная функция 
статистически линеаризованной системы}~(\ref{e3.2s}):
 \begin{equation*}
 \Phi (i\w; \tilde m_y, \tilde K_y) = \left[k_1 (\tilde m_y,\tilde  K_y) - Ii\w\right]^{-1}\!;
 \ \ I=I_n\,;
% \label{e3.12s}
 \end{equation*}
\textit{$^*$~--- символ эрмитова сопряжения; $I_n$~--- единичная $(n\times n)$-мат\-рица}.


\medskip

\noindent
\textbf{Замечание.} Рассмотренные в~\cite{1-sin} другие схемы статистической линеаризации 
очевидным образом обобщаются на круговой случай. При этом могут быть использованы различные 
модели КСтС~\cite{9-sin}.

Алгоритмы теорем~3.1--3.3 и их дискретных версий лежат в основе СтИТ
анализа аналитического моделирования. Они реализованы  в ИПО\linebreak
CStS-ANALYSIS в среде  MATLAB~[9--11]. Инструментальное программное
обеспечение имеет возможность
реализовать также и статистическое моделирование КСтС для
следующих <<намотанных>> круговых распределений КСВ: решетчатого,
нормального,  Мизеса, равномерного, пуассонова, кардиоидного,
треугольного, Коши и других устойчивых распределений~\cite{4-sin, 7-sin}.
Точность алгоритмов анализа и аналитического моделирования
проверялась на радиотехнических примерах~\cite{6-sin}, а также методом
статистического моделирования.

\section{Заключение}

Принципы, подходы и задачи статистических научных исследований,
развитые в~\cite{1-sin} для стохастических систем в евклидовом пространстве,
сохраняются и для круговых систем. Методическое и алгоритмическое
обеспечение, основанное на статистической линеаризации для
эквивалентного <<намотанного>> нормального распределения, даются
теоремами~3.1--3.3. Разработано и испытано на ряде тестовых примеров
универсальное ИПО CStS-ANALYSIS
в среде  MATLAB для анализа, аналитического и статистического
моделирования.

{\small\frenchspacing
{%\baselineskip=10.8pt
\addcontentsline{toc}{section}{Литература}
\begin{thebibliography}{99}

\bibitem{1-sin}
\Au{Синицын И.\,Н.}
Канонические представления случайных функций и их применения в 
задачах компьютерной поддержки научных исследований.~--- М.: ТОРУС ПРЕСС, 2009.

\bibitem{2-sin}
\Au{Босов А.\,В., Будзко В.\,И., Захаров~В.\,Н., Козмидиади~В.\,А., 
Корепанов~Э.\,Р., Синицын~И.\,Н., Шоргин~С.\,Я., Ушмаев~О.\,С.}  
Информатика: состояние, проблемы, перспективы~/ Под ред.  И.\,А.~Соколова.~--- М.: ИПИ РАН, 2009.

\bibitem{3-sin}
\Au{Ватанабэ С., Икэда Н.}
 Стохастические дифференциальные уравнения и диффузионные процессы~/ Пер. с англ. 
 под ред.  А.\,Н.~Ширяева.~--- М.: Наука, 1986.

\bibitem{4-sin}
\Au{Rao Jammalamadaka S., Sen Gupta~A.}
  Topics in circular statistics.~--- Singapore: World Scientific, 2001.

\bibitem{5-sin}
\Au{Леви П.}
 Стохастические процессы и броуновское движение~/ Пер. с фр.  под ред. Н.\,Н.~Ченцова.~--- М.: Наука, 1972.

\bibitem{6-sin}
\Au{Тихонов В.\,И., Миронов М.\,А.}
 Марковские процессы.~--- М.: Сов. радио, 1977.

\bibitem{7-sin}
\Au{Мардиа К.} 
Статистический анализ угловых наблюдений~/ Пер. с англ. под ред. Л.\,Н.~Большева.~--- М.: Наука, 1978.
%\columnbreak

\bibitem{8-sin}
\Au{Морозов А.\,Н., Назолин А.\,Л.}
 Динамические системы с флуктуирующим временем.~--- М.: МГТУ им. Н.\,Э.~Баумана, 2001.

 \columnbreak


\bibitem{9-sin}
\Au{Синицын И.\,Н. }
 Канонические разложения случайных функций и их применение в стохастических 
 ин-\linebreak формационных технологиях научных исследований: Курс лекций~// 
 Распознавание образов и анализ изоб\-ра\-же\-ний: новые информационные технологии~--- 
 РОАИ-10-2010: Мат-лы Междунар. конф.~--- СПб., 2010.
 
 \vspace*{6pt}

\bibitem{10-sin}
\Au{Синицын И.\,Н., Корепанов Э.\,Р., Белоусов~В.\,В. и~др.}
Развитие компьютерной поддержки статистических научных исследований сис\-тем 
высокой точности и доступности~// Системы и средства информатики, 2011. Вып.~21. №\,1. С.~3--33.

 \vspace*{6pt}

\label{end\stat}

\bibitem{11-sin}
\Au{Sinitsyn I.\,N., Belousov V.\,V., Konashenkova~T.\,D.}
Software tools for circular stochastic systems analysis~/ 
29th  Seminar (International) on Stability Problems for  Stochastic Models and
5th Workshop ``Applied Problems in Theory of Probabilities and
Mathematical Statistics Related to Modeling of Information Systems'' (APTP\;+\;MS'2011) Book of
Abstracts.~---  M.: IPI RAS, 2011. P.~86--87.
 \end{thebibliography}
}
}


\end{multicols}         %1
%\newcommand {\ff}{{\mathcal F}}
\newcommand {\ebd}{\triangleq}
\newcommand{\me}[2]{\mathbf{E}_{ #1 }\left\{ \mathop{#2} \right\} }



\def\stat{borisov}

\def\tit{ФИЛЬТРАЦИЯ СОСТОЯНИЙ МАРКОВСКИХ СКАЧКООБРАЗНЫХ ПРОЦЕССОВ 
ПО~ДИСКРЕТИЗОВАННЫМ НАБЛЮДЕНИЯМ$^*$}

\def\titkol{Фильтрация состояний марковских скачкообразных процессов 
по~дискретизованным наблюдениям}

\def\aut{А.\,В.~Борисов$^1$}

\def\autkol{А.\,В.~Борисов}

\titel{\tit}{\aut}{\autkol}{\titkol}

\index{Борисов А.\,В.}
\index{Borisov A.\,A.}




{\renewcommand{\thefootnote}{\fnsymbol{footnote}} \footnotetext[1]
{Работа выполнена при частичной поддержке РФФИ (проект 16-07-00677).}}


\renewcommand{\thefootnote}{\arabic{footnote}}
\footnotetext[1]{Институт проблем информатики Федерального исследовательского центра <<Информатика 
и~управление>> Российской академии наук,
\mbox{aborisov@frccsc.ru}}

%\vspace*{8pt}



\Abst{Статья посвящена решению задачи оптимальной 
фильтрации состояний однородного марковского скачкообразного процесса (МСП). 
Наблюдения представляют собой приращения случайных процессов~--- интегральных 
преобразований состояний, зашумленные винеровскими процессами, интенсивность 
которых также зависит от оцениваемого состояния. Оптимальная оценка в~моменты 
получения нового наблюдения вычисляется как функция предыдущей оценки и~новых 
наблюдений, а~между моментами наблюдений~--- простейшим прогнозом в~силу системы 
уравнений Колмогорова. Рекуррентная формула пересчета ресурсозатратна, так как 
содержит  интегралы~--- мас\-штаб\-но-сдви\-го\-вые смеси многомерных гауссиан, 
где в~качестве смешивающих выступают распределения времени пребывания 
состояния в~каждом из возможных значений. Предложены более простые аппроксимации, 
основанные на предположении об ограниченности числа скачков состояния за время между 
наблюдениями. Получены универсальные локальная и~глобальная характеристики точности 
аппроксимаций, зависящие от па\-ра\-мет\-ров оцениваемого процесса, величины 
временн$\acute{\mbox{о}}$го шага  между наблюдениями и~максимального числа учитываемых скачков.}

\KW{марковский скачкообразный процесс; оптимальная фильтрация; мультипликативные 
шумы в~наблюдениях; стохастическое дифференциальное уравнение; численная аппроксимация}

\DOI{10.14357/19922264180316}
  
%\vspace*{4pt}


\vskip 10pt plus 9pt minus 6pt

\thispagestyle{headings}

\begin{multicols}{2}

\label{st\stat}



 \section{Введение}
 
 Фильтр Вонэма~\cite{Won_65}~--- один из редких удачных случаев, когда 
 оценка оптимальной фильтрации состо\-яния стохастической системы наблюдения 
 выражается в~виде решения некоторой замк\-ну\-той\linebreak конечномерной сис\-те\-мы 
 стохастических дифференциальных уравнений. 
 
 Алгоритм данного фильт\-ра 
 позволяет вычислить оценку фильт\-ра\-ции со\-сто\-яния \textit{марковского скачкообразного 
 процесса} с~\mbox{конечным} множеством состояний по наблюдениям в~присутствии 
 аддитивных винеровских шумов. Теоретически оптимальная оценка со\-сто\-яния~--- 
 его условное распределение в~текущий момент времени~--- 
 обладает очевидными свойствами неотрицательности и~нормировки. 
 При чис\-лен\-ной реализации данного фильтра классическим методом 
 Эй\-ле\-ра--Ма\-ру\-ямы~\cite{KP_92} данные свойства могут не сохраняться и~процедура 
 вы\-чис\-ле\-ния становится неустойчивой.  В~связи с~этим обстоятельством разрабатывались 
 другие алгоритмы чис\-лен\-но\-го решения уравнения фильтра Вонэма, обладающие 
 требуемыми свойствами устойчивости (см.~\cite{YZL_04, PR_10} и~библиографию в~них). 
 В~час\-ти этих работ доказана лишь слабая сходимость пред\-ла\-га\-емых аппроксимационных 
 схем к~оценке фильт\-ра Вонэма, в~то время как ка\-кая-ли\-бо 
 характеризация точ\-ности этих приближений отсутствует.
 
 В~\cite{B_18} было представлено распространение фильт\-ра Вонэма на случай 
 наблюдений с~мультипликативными шумами. При этом уравнение обобщенного 
 фильт\-ра содержит в~правой части квадратическую характеристику шумов в~наблюдениях. 
 Данный процесс на практике никогда не наблюдается непосредственно, а~является лишь 
 некоторым нелинейным интегральным преобразованием наблюдений. Очевидно, что 
 имеющиеся в~настоящий момент времени алгоритмы приближенного вычисления оценки 
 фильтрации Вонэма для данной системы не подходят. 
 
 Целью предлагаемой работы является ис\-поль\-зование результатов оптимальной 
 фильтрации со\-стояний сис\-тем с~дискретным временем для аппроксимации решения 
 аналогичной задачи для\linebreak стохастических дифференциальных сис\-тем. 
 
 Статья организована следующим образом. Раздел~2 содержит формальную постановку 
 задачи фильт\-ра\-ции со\-сто\-яний однородного МСП с~конечным множеством со\-сто\-яний 
 по наблюдениям, полученным путем временн$\acute{\mbox{о}}$й дискретизации процессов с~непрерывным 
 временем~--- интегральных преобразований со\-сто\-яния сис\-те\-мы в~присутствии 
 мультипликативных винеровских шумов.\linebreak
  В~разд.~3 пред\-став\-ле\-но решение поставленной 
 задачи фильт\-ра\-ции: пересчет оценок со\-сто\-яний в~момент получения новых 
 дискретизованных наблюдений выполняется в~соответствии с~некоторыми\linebreak 
 рекуррентными интегральными соотношениями, в~то время как между 
 моментами наблюдений оценка корректируется в~соответствии с~прогнозом в~силу 
 сис\-те\-мы уравнений Колмогорова. Вы\-чис\-ли\-тель\-ная слож\-ность 
 упомянутых выше интегральных\linebreak 
 соотношений связана с~тем, что в~расчет принимается воз\-мож\-ность того, что между 
 моментами наблюдений оцениваемое со\-сто\-яние может совершить произвольное чис\-ло 
 скачков. В~разд.~4 пред\-став\-лен более простой алгоритм приближенного вы\-чис\-ле\-ния 
 оценки фильт\-ра\-ции, основанный на ограничении возможного числа учитываемых скачков 
 МСП. Доказана тео\-ре\-ма, опре\-де\-ля\-ющая как\linebreak
  локальную (одношаговую), так и~глобальную 
 (многошаговую) характеристики точ\-ности предложенного при\-бли\-же\-ния~--- 
 $\ell_1$-нор\-мы ошибки аппроксимации. Полученные характеристики являются\linebreak 
 универсальными, т.\,е.\ не асимптотическими по шагу дискретизации, и~зависят от характеристик 
 самого МСП, %\linebreak
  шага временн$\acute{\mbox{о}}$й дискретизации и~чис\-ла
  скачков со\-сто\-яния, учи\-ты\-ва\-емых 
 на шаге. Об\-суж\-де\-ние результатов и~заключительные комментарии пред\-став\-ле\-ны 
 в~разд.~5.
 
 \section{Постановка задачи фильтрации}
 
 На полном вероятностном пространстве с~фильт\-ра\-цией 
 $(\Omega,\mathcal{F},\mathcal{P},\{\mathcal{F}_{t}\}_{t \geqslant 0})$ рассматривается система наблюдений
\begin{equation}
 \left.
 \begin{array}{rl}
 \displaystyle X_t &=X_0 +  \displaystyle
 \int\limits_0^t \Lambda^{\top}X_{s}\,ds + \mu_s\,;  \\[6pt]
 \displaystyle Y_k &=  \displaystyle\int\limits_{t_{k-1}}^{t_k}fX_s\,ds+
 \int\limits_{t_{k-1}}^{t_k} 
 \sum\limits_{n=1}^NX_s^ng_n \,dW_s, \\[6pt]
 &\hspace*{10mm}\{t_k\}_{k \geqslant 0}: \; 0 = t_0 < t_1 < t_2\cdots,
 \end{array}
 \right\}
 \label{eq:obsys_1}
 \end{equation}
 где
  \begin{itemize}
  \item
  $X_t \ebd \mathrm{col}\left(X_t^1,\ldots,X_t^N\right) \hm\in \mathbb{S}^N$~--- 
  ненаблюда\-емое состояние системы, являющееся однородным МСП с~конечным 
  множеством состояний $ \mathbb{S}^N \ebd$\linebreak $\ebd \{e_1,\ldots,e_N\}$ ($\mathbb{S}^N$~--- 
  множество единичных векторов евклидова пространства~$\mathbb{R}^N$), 
  матрицей интенсивностей переходов~$\Lambda$ и~начальным распределением~$\pi$;
  \item
  $\mu_t \ebd \mathrm{col}\left(
  \mu_t^1,\ldots,\mu_t^N\right)\hm\in \mathbb{R}^N$~--- 
  ${\mathcal{F}}_t$-со\-гла\-со\-ван\-ный мартингал;
  \item
  $\{Y_k\}_{k \in \mathbb{N}}:\;  Y_k \ebd \mathrm{col}\left(Y_k^1,\ldots,Y_k^M\right) 
  \hm\in \mathbb{R}^M$~--- последовательность дискретизованных наблюдений, 
  доступных в~известные неслучайные  моменты времени~$\{t_k\}_{k \in \mathbb{N}}$,
в~которых $W_t \ebd$\linebreak $\ebd \mathrm{col}\left(W_t^1,\ldots,W_t^M\right) \hm\in \mathbb{R}^M$
 является ${\mathcal{F}}_t$-со\-гла\-со\-ван\-ным стандартным винеровским процессом, 
 определяющим шумы в~наблюдениях,\linebreak  $f$~--- $(M \times N)$-мер\-ная 
 мат\-ри\-ца плана наблюдений, а~набор мат\-риц~$\{g_n\}_{n=\overline{1,N}}$ 
 характеризует интенсивности шумов в~зависимости от текущего состояния~$X_t$.
  \end{itemize}
  
  Введем также в~рассмотрение неубывающие семейства $\sigma$-ал\-гебр 
  $\mathcal{O}_k \ebd \sigma\{ Y_{\ell}: \; 1 \hm\leqslant \ell \hm\leqslant k\}$ 
  и~$\mathcal{O}_t \ebd  \mathcal{O}_{k(t)}$, где 
  $k(t) \ebd \sum\nolimits_{j \in \mathbb{N}}\mathbf{I}(t-t_{j})$; 
  $\mathcal{O}_0 \ebd \{\varnothing,\; \Omega\}$.
  
   \textit{Задача оптимальной фильтрации состояния~$X$ по наблюдениям~$Y$} 
   заключается в~нахождении \textit{условного математического ожидания} (УМО)
  \begin{equation*}
  \widehat{X}_t \ebd {\sf E}\left\{X_t|\mathcal{O}_{t} \right\}\,.
 % \label{eq:fest_1}
  \end{equation*}
  
  Относительно системы~(\ref{eq:obsys_1})  сделаны следующие предположения:
   \begin{itemize}
 \item[(а)]
 ${\mathcal{F}}_t \equiv {\mathcal{F}}_{t}^X \bigvee 
 {\mathcal{F}}_{t}^W $ для любого $t \hm\geqslant 0$;
 \item[(б)]
 шумы в~наблюдениях равномерно невырожденные, т.\,е.\
  $g_ng_n^{\top} \hm\geqslant \alpha I \hm> 0$ для всех $n\hm=\overline{1,N}$ 
  и~некоторого $\alpha\hm>0$.
% \item
 % Верно неравенство
  %\begin{equation}
  %\min_{1\leqslant k \leqslant N}|\lambda_{kk}| > 0.
  %\label{eq:ineq_0}
  % \end{equation}
 %\item
 %Для любого $t \geqslant 0$ все компоненты вектора $p_t \ebd \me{}{X_t}$ строго %положительны. 
 \end{itemize} 

 \section{Уравнения оптимального фильтра} 
 
 Для получения уравнений оптимального фильт\-ра воспользуемся подходом, 
 применяемым для решения аналогичной задачи в~стохастических сис\-те\-мах 
 наблюдения с~дискретным временем~\cite{BSh_85}. 
 Воспользу\-ем\-ся методом математической индукции. 
 
 При $r=0$ 
 \begin{equation}
 \widehat{X}_{t_0}={\sf E}\{X_0|\mathcal{O}_0\}={\sf E}\{X_0\}=\pi\,.
 \label{eq:in_cond}
 \end{equation} 
 
 Пусть для некоторого $ r \hm\geqslant 0$ известна оценка оптимальной 
 фильтрации~$\widehat{X}_{t_r} \hm= {\sf E}{X_{t_r} |\mathcal{O}_r}$. 
 Определим оценку оптимальной фильтрации~$\widehat{X}_{t} $ для $t\hm \in (t_r,t_{r+1}]$. 
 
 Для произвольного момента $t \hm\in (t_r,t_{r+1})$ в~силу мартингального 
 разложения МСП~$X_t$ и~свойств УМО верна следующая цепочка равенств:
 \begin{multline*}
 \widehat{X}_{t} = {\sf E}\left\{X_t | \mathcal{O}_r\right\}={}\\
 {}=
 {\sf E}\left\{{\cal P}^{\top}(t_r,t)X_{t_r}+
 \int\limits_{t_r}^t{\cal P}^{\top}(t_r,s)\,dM_s\big\vert \mathcal{O}_r\right\} = {}
\end{multline*}

\noindent
   \begin{multline}
 \hspace*{-11.66pt}{}=\mathcal{P}^{\top}(t_r,t)\widehat{X}_{t_r} + {\sf E}\hspace*{-2pt}
 \left\{{\sf E}\hspace*{-2pt}\left\{\int\limits_{t_r}^t\hspace*{-2pt}\mathcal{P}^{\top}(t_r,s)\,dM_s |
 {\mathcal{F}}_{t_r}\right\}\!\big\vert 
 \mathcal{O}_r\!\right\} ={}\hspace*{-4.24124pt}\\
 {}=
  \mathcal{P}^{\top}(t_r,t)\widehat{X}_{t_r}\,,
 \label{eq:bw_obs}
 \end{multline}
 где $\mathcal{P}(s,t)$ $(s \hm\leqslant t)$~--- матрица переходной ве\-ро\-ят\-ности МСП 
 на промежутке $[s,t]$, являющаяся решением сис\-те\-мы дифференциальных 
 уравнений Колмогорова
 \begin{equation*}
 \mathcal{P}'_t(s,t) = \mathcal{P}(s,t) \Lambda, \enskip t > s, \enskip \mathcal{P}(s,s) = I.
 \end{equation*}
 В случае однородного МСП $\mathcal{P}(s,t) \hm= e^{(t-s)\Lambda}$.
 
 Далее необходимо определить совместное распределение $(X_{t_{r+1}},Y_{r+1})$ 
 относительно~$ \mathcal{O}_r$. Из модели наблюдений следует, что 
 распределение~$Y_{r+1}$ относительно 
 $\sigma$-ал\-геб\-ры~$\mathcal{F}^X_{t_{r+1}} \vee \mathcal{O}_r$~---
 гауссовское с~параметрами 
 \begin{align*}
{\sf E}\left\{Y_{r+1}|{\mathcal{F}}^X_{t_{r+1}}\right\}& = f \tau_{r+1}\,; \\[6pt]
 \mathrm{cov} \left(Y_{r+1},Y_{r+1}|{\mathcal{F}}^X_{t_{r+1}}\right) &= 
 \displaystyle\sum\limits_{n=1}^N \tau_{r+1}^n g_ng_n^{\top}\,,
% \label{eq:occup_1}
 \end{align*}
 где $\tau_{r+1} \hm= \tau_{r+1}(X(\omega))=
 \mathrm{col}\left(\tau_{r+1}^1,\ldots,\tau_{r+1}^N\right) \ebd$\linebreak
 $\ebd 
 \int\nolimits_{t_r}^{t_{r+1}}X_s\,ds$~--- случайный вектор, $n$-я 
 компонента которого равна времени пребывания процесса~$X$ в~со\-сто\-янии~$e_n$ 
 на  интервале времени $[t_r, t_{r+1}]$. 
 Обозначим через $\mathcal{D}_{r+1} \ebd \{u=\mathrm{col}\,(u^1,\ldots,u^N):\; 
 u_m \hm\geqslant 0,\; \sum\nolimits_{m=1}^Mu_m\hm= t_{r+1}-t_r\}$ $(M-1)$-мер\-ный 
 симплекс в~пространстве~$\mathbb{R}^M$, являющийся носителем распределения 
 вектора~$\tau_{r+1}$. Пусть $\rho^{k,\ell}_{r+1}(du)$~--- 
 распределение вектора $\tau_{r+1} X_{t_{r+1}}^{\ell}$ при условии $X_{t_r}\hm=e_k$, 
 т.\,е.\ 
 для любого $\mathcal{A} \hm\in \mathcal{B}(\mathbb{R}^M)$ верно тождество:
\begin{multline*}
 \mathbf{P}\left\{\omega: \; X_{t_{r+1}}(\omega)=e_{\ell},\right.\\
 \left. 
 \tau_{r+1}(X(\omega)) \in \mathcal{A}\;|\;X_{t_r}=e_k\right\} \equiv
   \rho^{k,\ell}_{r+1}(\mathcal{A})\,.
\end{multline*}
 
Обозначим через
\begin{multline*}
 \mathcal{N}(y,m,K) \ebd (2\pi)^{-M/2} \mathrm{ det}^{-1/2} K\times{}\\
 {}\times\exp
 \left\{ -\fr{1}{2}\left(y-m\right)^{\top}K^{-1}(y-m)\right\}
\end{multline*}
 $M$-мер\-ную плот\-ность гауссовского распределения с~математическим 
 ожиданием~$m$ и~ковариационной матрицей~$K$.
 
 Из марковского свойства  $\{X_{t_{r}},Y_{r})\}_{r \geqslant 0}$ 
 относительно~${\mathcal{F}}_{t_{r}}$~\cite{ZhSh_95} и~теоремы Фубини следует, что 
 для любого  множества $\mathcal{A} \hm\in \mathcal{B}(\mathbb{R}^M)$ 
 верна следующая цепочка равенств:
 \begin{multline*}
 {\sf E}\left\{X_{t_{r+1}}\mathbf{I}_{\mathcal{A}}
 \left(Y_{r+1}\right)\big|\mathcal{O}_r\right\}={}\\
 {}=
{\sf E}\left\{{\sf E}\left\{X_{t_{r+1}}\mathbf{I}_{\mathcal{A}}
\left(Y_{r+1}\right)\big|
\mathcal{F}^X_{t_{r+1}} \vee \mathcal{O}_r\right\}
 \big|\mathcal{O}_r\right\} = {}
\end{multline*}

\noindent
\begin{multline*}
 %{}=
% {\sf E}\left\{{\sf E}\left\{X_{t_{r+1}}\mathbf{I}_{\mathcal{A}}
% \left(Y_{r+1}\right)\vert X_{t_r}\right\}
% \vert\mathcal{O}_r\right\} = {}\\
% {}=
%{\sf E}\left\{\sum\limits_{k=1}^N {\sf E}\left\{X_{t_{r+1}}\mathbf{I}_{\mathcal{A}}
%\left(Y_{r+1}\right)  \big| X_{t_r}=e_k\right\}X_{t_r}^k
% \big|\mathcal{O}_r\right\} = {}\\ 
% {}=
% \sum\limits_{k=1}^N{\sf E}
% \left\{X_{t_{r+1}}\mathbf{I}_{\mathcal{A}}\left(Y_{r+1}\right)\bigl| X_{t_r}=e_k\right\} 
% \widehat{X}_{t_r}^k ={}\\
% {}=\!
% \sum\limits_{k=1}^N{\sf E}
% \left\{{\sf E}\left\{X_{t_{r+1}}\mathbf{I}_{\mathcal{A}}
% \left(Y_{r+1}\right)\!\bigl| {\mathcal{F}}_{t_{r+1}}\right\}\!\bigl| 
% X_{t_r}\!=e_k\right\} \widehat{X}_{t_r}^k ={}\\
% {}=
% \sum\limits_{k=1}^N {\sf E}\left\{
% \vphantom{\int\limits_A\left(\sum\limits_{p=1}^N\right)}
% X_{t_{r+1}} \times{}\right.\\
% {}\times\int\limits_{\mathcal{A}}  
% \mathcal{N}\left(y,f \tau_{r+1}(X),\sum\limits_{p=1}^N \tau_{r+1}^p(X) g_pg_p^{\top}\right)dy
% \Biggl| X_{t_r}={}\\
%\left. {}=e_k
% \vphantom{\int\limits_A\left(\sum\limits_{p=1}^N\right)}
%\right\} \widehat{X}_{t_r}^k = 
% \sum\limits_{k=1}^N \int\limits_{\mathcal{A}}{\sf E}\left\{ 
% \vphantom{\sum\limits_{p=1}^N}
% X_{t_{r+1}} \times{}\right.\\
% {}\times\mathcal{N}\left(y,f \tau_{r+1}(X),\sum\limits_{p=1}^N \tau_{r+1}^p(X) 
% g_p g_p^{\top}\right)
% \Biggl| X_{t_r}={}\\
%\left. {}=e_k
%\vphantom{\sum\limits^N_{p=1}}
%\right\} \widehat{X}_{t_r}^k\, dy
 %={}\\
 {}=
 \sum\limits_{\ell=1}^N e_{\ell} \int\limits_{\mathcal{A}} 
 \left[ \sum\limits_{k=1}^N 
 \int\limits_{\mathcal{D}_{r+1}} 
 \mathcal{N}\left(y,f u,\sum_{p=1}^N u^p g_pg_p^{\top}\right)\times{}\right.\\
\left. {}\times
 \rho^{k,\ell}_{r+1}(du)\widehat{X}_{t_r}^k
 \vphantom{\int\limits_A\sum\limits_{p=1}^N}
 \right] 
 dy,
 \end{multline*}
 из чего следует, что интегранд в~квадратных скобках в~последнем выражении 
 определяет искомое совместное распределение $(X_{t_{r+1}},Y_{r+1})$ 
 относительно~$ \mathcal{O}_r$. Оценка~$\widehat{X}_{t_{r+1}}$ покомпонентно 
 определяется~\cite{BSh_85} с~помощью обобщенного варианта формулы Байеса:
 \begin{multline}
 \widehat{X}_{t_{r+1}}^j = {}\\
 \hspace*{-1mm}{}=
 \fr{\int\nolimits_{\mathcal{D}_{r+1}}\hspace*{-6mm} 
 \mathcal{N}\left(Y_{r+1},f u,\sum\nolimits_{p=1}^N \hspace*{-2mm}
 u^p g_pg_p^{\top}\!\right)\hspace*{-1mm}
 \sum\nolimits_{k=1}^N \hspace*{-2mm}
 \widehat{X}_{t_r}^k
 \rho^{k,j}_{r+1}(du)
 }
 { \int\nolimits_{\mathcal{D}_{r+1}} \hspace*{-6mm}
 \mathcal{N}\left(Y_{r+1},f v,\sum\nolimits_{q=1}^N \hspace*{-2mm}
 v^q g_qg_q^{\top}\!\right)\hspace*{-1mm}
 \sum\nolimits_{i,\ell=1}^N \hspace*{-2mm}
 \widehat{X}_{t_r}^i
 \rho^{i,\ell}_{r+1}(dv)
  },  \\ 
  j = \overline{1,N}\,.
 \label{eq:filt_1}
 \end{multline}
 Таким образом, доказана следующая
 
 %\smallskip
 
 \noindent
 \textbf{Лемма~1.}
\textit{Если для системы наблюдения}~(\ref{eq:obsys_1}) 
\textit{верны условия~(а) и~(б), то оценка~$\widehat{X}_t$ оптимальной фильтрации 
определяется формулой}~(\ref{eq:in_cond}) 
\textit{при $t\hm=0$, рекуррентным соотношением}~(\ref{eq:filt_1})~---
\textit{в~моменты~$t_{r+1}$ получения наблюдений~$Y_{r+1}$ 
и~формулой}~(\ref{eq:bw_obs})~--- 
\textit{в~промежутках времени между моментами получения наблюдений}.


\smallskip
 

 
 Несмотря на компактную запись~(\ref{eq:filt_1}), их прямая численная реализация 
 ресурсозатратна. Во-пер\-вых, в~(\ref{eq:filt_1}) требуется вычислять 
 распределения мас\-штаб\-но-сдви\-го\-вых смесей многомерных нормальных 
 распределений, что является трудоемкой\linebreak процедурой. Во-вто\-рых, 
 распределения~$\rho^{k,j}_{r+1}$ вре-\linebreak мени пребывания представляют собой 
 сумму\linebreak бесконечного ряда, слагаемые которого вычис\-ляются с~помощью 
 некоторой рекуррентной про\-це\-дуры~\cite{S_00}. В-третьих, 
 распределения~$\rho^{k,j}_{r+1}$ не являются абсолютно непрерывными 
 относительно меры Ле\-бега.
 { %\looseness=1
 
 }
 
 Следующий раздел посвящен численной аппроксимации~(\ref{eq:filt_1}) и~исследованию 
 ее точностных характеристик.
 
 \section{Приближенное вычисление оценки фильтрации}
 
 Без ограничения общности будем считать, что сетка~$\{t_r\}_{r \geqslant 0}$ 
 является равномерной с~шагом~$\Delta$, т.\,е.\ $t_r \hm= r \Delta$ 
 и~$\mathcal{D}_r \hm\equiv \mathcal{D}$.
 Обозначим через~$N_{r+1}$ об-\linebreak\vspace*{-12pt}
 
 \pagebreak
 
 \noindent
 щее число скачков процесса~$X_t$, имевших место 
 на промежутке $(t_r,t_{r+1}]$. Тогда из формулы полной вероятности следует, 
 что~(\ref{eq:filt_1}) представима в~виде:
 \begin{multline}
 \widehat{X}_{t_{r+1}}^j =  \left(
 \int\limits_{\mathcal{D}} 
 \mathcal{N}\left(Y_{r+1},f u,\sum\limits_{p=1}^N u^p g_pg_p^{\top}\right)\times{}\right.\\
\left. {}\times
 \sum\limits_{h=0}^{\infty}\sum\limits_{k=1}^N \widehat{X}_{t_r}^k
 \rho^{k,j,h}_{r+1}(du)
 \right)\Bigg/ \\
 \left(
 \vphantom{\sum\limits_{m=0}^{\infty}
 \sum\limits_{i,\ell=1}^N \widehat{X}_{t_r}^i
 \rho^{i,\ell,m}_{r+1}(dv)}
 \int\limits_{\mathcal{D}} 
 \mathcal{N}\left(Y_{r+1},f v,\sum\limits_{q=1}^N v^q g_qg_q^{\top}\right)\times{}\right.\\
\left.{}\times \sum\limits_{m=0}^{\infty}
 \sum\limits_{i,\ell=1}^N \widehat{X}_{t_r}^i
 \rho^{i,\ell,m}_{r+1}(dv)
 \right)
  \,, \enskip j = \overline{1,N}\,,
  \label{eq:filt_1_1}
 \end{multline}
 где 
 $ \rho^{k,j,h}_{r+1}(du)$~--- распределение вектора 
 $\tau_{r+1}X_{t_{r+1}}^{j}\mathbf{I}_{\{h\}}(N_{r+1})$ при 
 условии $X_{t_r}\hm=e_k$, т.\,е.\ 
 для любого $\mathcal{A} \hm\in \mathcal{B}(\mathbb{R}^M)$ верно тождество
\begin{multline*}
 \mathbf{P}\left\{\omega: \; X_{t_{r+1}}(\omega)=e_{j}, \; N_{r+1} = h,\right.\\ 
\left. \tau_{r+1}(X(\omega)) \in \mathcal{A}\;|\;X_{t_r}=e_k\right\} \equiv
  \rho^{k,j,h}_{r+1}(\mathcal{A}).
\end{multline*}
В качестве аппроксимации оценок можно использовать  
 $\overline{X}_{t_{r+1}}^n \ebd 
 \mathrm{col}\,(\overline{X}_{t_{r+1}}^{n,1},\ldots,\overline{X}_{t_{r+1}}^{n,N})$, 
 полученные из~(\ref{eq:filt_1_1}) путем урезания сумм ряда в~числителе и~знаменателе:
 
 \noindent
 \begin{multline}
 \overline{X}_{t_{r+1}}^{n,j} = 
 \left(
 \int\limits_{\mathcal{D}} 
 \mathcal{N}\left(Y_{r+1},f u,\sum\limits_{p=1}^N u^p g_pg_p^{\top}\right)\times{}\right.\\[-1pt]
\left.{}\times \sum\limits_{h=0}^{n}\sum\limits_{k=1}^N \overline{X}_{t_r}^k
 \rho^{k,j,h}_{r+1}(du)
 \right)\Bigg/ \\[-1pt]
 \left(
 \int\limits_{\mathcal{D}} 
 \mathcal{N}\left(Y_{r+1},f v,\sum\limits_{q=1}^N v^q g_qg_q^{\top}\right)\times{}\right.\\[-1pt]
\left. {}\times
 \sum\limits_{m=0}^{n}
 \sum\limits_{i,\ell=1}^N \overline{X}_{t_r}^i
 \rho^{i,\ell,m}_{r+1}(dv)
  \right)\,, \enskip
   j = \overline{1,N}.
  \label{eq:filt_2}
 \end{multline}
 Ниже по формуле полной вероятности получены интегралы из~(\ref{eq:filt_2}) для 
 $h\hm=0,1,2$:
 
\vspace*{-3pt}

 \noindent
  \begin{multline*}
 \int\limits_{\mathcal{D}}  \mathcal{N}
 \left(Y_{r+1},f u,\sum\limits_{p=1}^N u^p g_pg_p^{\top}\right) 
 \rho^{k,j,0}_{r+1}(du) = {}\\[-1pt]
 {}=
 \delta_{kj}\mathcal{N}\left(Y_{r+1},\Delta f^j,\Delta g_jg_j^{\top}\right)
 e^{\lambda_{jj}\Delta};
 %\label{eq:h0}
\\[-1pt]
 \int\limits_{\mathcal{D}}  \mathcal{N}\left(
 Y_{r+1},f u,\sum\limits_{p=1}^N u^p g_pg_p^{\top}\right) 
 \rho^{k,j,1}_{r+1}(du) ={} 
 \end{multline*}
 
 \noindent
 \begin{multline}
 \hspace*{-6.7pt}{}=\left(1-\delta_{kj}\right)\lambda_{kj}e^{\lambda_{jj}\Delta}
\! \int\limits_0^{\Delta}\!
 e^{(\lambda_{kk}-\lambda_{jj})u^k}
 \mathcal{N}\left(Y_{r+1},u^kf^k +{}\right.\hspace*{-0.28818pt}\\[-1pt]
\hspace*{-3mm}\left. {}+ \left(\Delta - u^k\right)f^j, u^k g_kg_k^{\top}+
 \left(\Delta-u^k\right)g_jg_j^{\top}\right)\,du^k;
 \label{eq:h1}
 \end{multline}
 
 \vspace*{-12pt}
 
 \noindent
 \begin{multline}
 \int\limits_D \mathcal{N}\left( 
Y_{r+1},f u,\sum\limits_{p=1}^N u^p g_pg_p^{\top}\right)du ={}\\[-1pt]
{}=
\sum\limits_{\substack{{\ell:\ell \neq k,}\\ {\ell \neq j}}}
 \lambda_{k\ell}\lambda_{\ell j} e^{\lambda_{jj}\Delta}\times {}\\[-1pt] 
 {}\times
 \int\limits_0^{\Delta} \int\limits_0^{\Delta-u^k} \!
e^{(\lambda_{kk}-\lambda_{\ell\ell})u^k+(\lambda_{\ell\ell}-
 \lambda_{jj})u^{\ell}}\times{} \\[-1pt] 
{}  \times
 \mathcal{N}\left(Y_{r+1},u^k f^k+u^{\ell}f^{\ell}+\left(
 \Delta-u^k-u^{\ell} \right)f^j,\right.\\[-1pt]
 \hspace*{-1mm}\left.
 u^k g_kg_k^{\top}+u^{\ell}g_{\ell}g_{\ell}^{\top}+\left(
 \Delta-u^k-u^{\ell} \right)
 g_jg_j^{\top}
 \right) du^{\ell}du^{k}, \!\!
  \label{eq:h2}
 \end{multline} 
 
\vspace*{-2pt}
 
 \noindent
  где  $\delta_{ij}$~--- символ Кронекера. Интегралы для $h\hm>2$ также могут 
  быть получены в~явном виде, однако их сложность резко возрастает.
 

   Так как система~(\ref{eq:obsys_1}) является автономной, то в~качестве локальной 
   характеристики бли\-зости~$\{\overline{X}_{t_r}\}$ 
   к~$\{\widehat{X}_{t_r}\}$ может быть выбрана величина
   
\noindent
 \begin{multline*}
 \overline{\sigma}(\pi) \ebd {\sf E}\left\{
 \|\widehat{X}_{t_{1}}(\pi, Y_{1}) - \overline{X}_{t_{1}}
 \left(\pi,Y_{1}\right)\|_{1}\right\} = {}\\
 {}=
 \sum\limits_{j=1}^N{\sf E}
 \left\{\left\vert \widehat{X}^j_{t_{1}}\left(\pi, Y_{1}\right) - \overline{X}^{n,j}_{t_{1}}
 \left(\pi,Y_{1}\right)\right\vert\right\}.
 %\label{eq:prec_1}
 \end{multline*}
 При этом начальное распределение $\pi \hm\in \mathcal{D}_1 \ebd $\linebreak $\ebd
 \{\mathrm{col}\,(\pi^1,\ldots,\pi^N):\;\pi^j > 0$, 
 $\sum\nolimits_{j=1}^N\pi^j\hm=1\}$ является начальным условием применения 
 одного шага рекурсии~(\ref{eq:filt_1}) или~(\ref{eq:filt_2}) для вычисления 
 оценки~$\widehat{X}_{t_{1}}$
   или~$\overline{X}_{t_{1}}$ соответственно. Фактически, 
 характеристика~$\overline{\sigma}(\pi)$ определяет, насколько сильно 
 рекурсивные схемы~(\ref{eq:filt_1}) и~(\ref{eq:filt_2}) разойдутся за 
 один шаг, стартуя из общей точки~$\pi$.
 
 Рекуррентные схемы~(\ref{eq:filt_1}) и~(\ref{eq:filt_2}), примененные~$r$~раз, 
 позволяют вычислить оценки~$\widehat{X}_{t_r}$ и~$\overline{X}_{t_r}$ 
 в~точке~$t_r$. В~качестве характеристики точности глобальной аппроксимации в~этом 
 случае естественно рассмотреть величину
 
 \vspace*{-2pt}
 
 \noindent
 \begin{equation*}
 \overline{\Sigma}_{t_r}(\pi) \ebd {\sf E}
 \left\{\|\widehat{X}_{t_{r}} - \overline{X}_{t_{r}}\|_{1}\right\} = 
 \!\sum\limits_{j=1}^N\!{\sf E}
 \left\{\left\vert \widehat{X}^j_{t_{r}} - 
 \overline{X}^{n,j}_{t_{r}}\right\vert \right\}.
% \label{eq:prec_2}
 \end{equation*}
 
 Следующее утверждение определяет оценки локальной и~глобальной 
 точности схемы аппроксимации~(\ref{eq:filt_2}).
 
 %\smallskip
 
 \noindent
 \textbf{Теорема~1.}\
\textit{Выполняются неравенства} 

%\vspace*{-2pt}

\noindent
 \begin{equation}
 \sup_{\pi \in \mathcal{D}_1} \overline{\sigma}(\pi) 
 \leqslant 2 \fr{(\overline{\lambda}\Delta)^{n+1}}{(n+1)!}\,;
 \label{eq:prec_loc}
\end{equation}

\noindent
\begin{align}
  \sup\limits_{\pi \in \mathcal{D}_1} \overline{\Sigma}_{t_r}(\pi)
   &\leqslant 2r \fr{(\overline{\lambda}\Delta)^{n+1}}{(n+1)!} +{}\notag\\[-0.5pt]
   &\hspace*{-20mm}{}+
  r(r-1)\left(
  \fr{(\overline{\lambda}\Delta)^{n+1}}{(n+1)!}
  \right)^2
  \left(
  1-\fr{(\overline{\lambda}\Delta)^{n+1}}{(n+1)!}
  \right)^{r-2},
 \label{eq:prec_glob}
 \end{align}
 
 \vspace*{-2pt}
 
 \noindent
 \textit{где} $\overline{\lambda} \ebd \max_{1 \leqslant j \leqslant N}|\lambda_{jj}|$.


%\smallskip

 Доказательство теоремы~1 приведено в~приложении.
 
 Данное утверждение представляет полезные оценки точности. Во-пер\-вых, 
 они являются равномерными по начальному распределению $\pi \hm\in \mathcal{D}_1$. 
 Во-вто\-рых, оценки носят универсальный, а~не асимптотический характер. Это 
 существенно в~практических задачах оценивания по дискретизованным 
 наблюдениям с~физическими или алгоритмическими ограничениями на шаг 
 по времени. Например, в~случае наблюдаемого процесса восстановления в~силу 
 центральной предельной теоремы для процессов восстановления~\cite{B_80} его
  приращения можно рассматривать как гауссовские случайные величины. 
  Однако данная аппроксимация обладает удовлетворительной точностью 
  только в~случае, когда шаг дискретизации по времени достаточно большой. 
 %
 В-третьих, неравенство~(\ref{eq:prec_glob}) позволяет получить порядок 
 аппроксимации при $\Delta \hm\to 0$. Зафиксируем момент времени $t\hm=T$ и~рассмотрим 
 характеристику $\sup\nolimits_{\pi \in \mathcal{D}_1} 
 \overline{\Sigma}_{T}(\pi)$ при $r\hm={T}/{\Delta}$ и~$\Delta \hm\to 0$. 
 Как только~$\Delta$ становится настолько мало, что 
 $\max\left({(\overline{\lambda}\Delta)^{n+1}}/{(n+1)!}, 
 \Delta ({T\lambda^{n+1}}/{(n+1)!})\right)\hm< 1$, из~(\ref{eq:prec_glob}) 
 следует неравенство
  %\begin{equation}
  $\sup\nolimits_{\pi \in \mathcal{D}_1} \overline{\Sigma}_{T}(\pi) 
  \hm\leqslant  ({3\overline{\lambda}^{n+1}}/{(n+1)!}) T\Delta^n.$
 %\label{eq:prec_asympt}
 %\end{equation}
 Это значит, что с~ростом времени~$T$ 
 ошибка аппроксимации копится пропорционально~$T$ и~при этом порядок точности 
 по~$\Delta$ равен~$n$.
 
 %\vspace*{-7pt}
 
  \section{Заключение}
  
  \vspace*{-4pt}
 
  В работе решена задача оценивания состояния однородного МСП по 
  дискретизованным наблюдениям. Получено аналитическое решение и~его 
  чис\-лен\-ные аппроксимации. Локальные и~глобальные показатели точ\-ности этих 
  приближений в~статье так\-же пред\-став\-ле\-ны. Примечательно, что  част\-ный случай 
  аппроксимаций~(\ref{eq:filt_2}) при $n\hm=0$ и~$\Lambda\hm=0$ был ранее 
  пред\-став\-лен в~\cite{B_17_1,B_17_2} для решения задачи байесовской классификации 
  случайного вектора по непрерывным наблюдениям с~мультипликативными шумами. 
 % 
Алгоритм оптимальной фильт\-ра\-ции и~его субоптимальные версии могут 
рас\-смат\-ри\-вать\-ся в~качестве основы чис\-лен\-ной реализации обобщения фильт\-ра 
Вонэма для сис\-тем с~мультипликативными шумами в~наблюдениях. 
Однако для их непосредственного использования необходимо решить 
следующие проб\-ле\-мы. Во-пер\-вых, в~(\ref{eq:h1}) и~(\ref{eq:h2}) присутствуют
 многомерные интегралы. Следует выяснить, какую результирующую погрешность 
 будут вносить ошибки их вы\-чис\-ле\-ния. Во-вто\-рых, представляется интересным 
 определить характеристики точ\-ности оптимальной фильт\-ра\-ции по дискретизованным 
 наблюдениям по отношению к~оптимальной фильт\-ра\-ции по непрерывным наблюдениям: 
 каков порядок точ\-ности по шагу временной дискретизации~$\Delta$? Для случая 
 вы\-чис\-ле\-ния классического фильт\-ра Вонэма с~по\-мощью алгоритма Эй\-ле\-ра--Ма\-ру\-ямы 
 подобный результат известен: порядок глобальной ошибки равен~${1}/{2}$. 
 Перечисленные задачи являются предметом дальнейших исследований.
 
 
  \vspace*{-10pt}
 
{\small
\subsection*{\raggedleft Приложение} 

\vspace*{-2pt}


\noindent
Д\,о\,к\,а\,з\,а\,т\,е\,л\,ь\,с\,т\,в\,о\ \ теоремы~1.\ \ Введем следующие 
обозначения для случайных величин и~мат\-риц, составленных из них:
\begin{align*}
\xi^{ji}(\ell)&\ebd 
\sum\limits_{h=0}^n \int\limits_{\mathcal{D}} 
 \mathcal{N}\left(Y_{\ell},f u,\sum\limits_{p=1}^N u^p g_pg_p^{\top}\right)
 \rho^{j,i,h}_{1}(du)\,; \\
  \theta^{ji}(\ell)&\ebd 
\sum\limits_{h=n+1}^{\infty} \int\limits_{\mathcal{D}} 
 \mathcal{N}\left(Y_{\ell},f u,\sum\limits_{p=1}^N u^p g_pg_p^{\top}\right)
 \rho^{j,i,h}_{1}(du)\,;
\\
 \xi(\ell)&\ebd \|\xi^{ji}(\ell)\|_{j,i=\overline{1,N}}\,,\quad 
 \Xi(r) \ebd \xi(r) \xi(r-1)\cdots \xi(1)\,;
 \\
 \theta(\ell)&\ebd \|\theta^{ji}(\ell)\|_{j,i=\overline{1,N}}\,, \quad 
 \Theta(r) \ebd \theta(r) \theta(r-1)\cdots \theta(1)\,.
%\label{eq:not_1}
\end{align*}
 
 Рекуррентные формулы~(\ref{eq:filt_1}) и~(\ref{eq:filt_2}) можно записать в~явной 
 форме
 
 
\noindent
\begin{align*}
 \widehat{X}_{t_r}& = \left( \mathbf{1}\left(\Xi(r) + 
 \Theta(r)\right)\pi\right)^{-1} \left(\Xi(r) + \Theta(r)\right)\pi\,;
\\
 \overline{X}_{t_r} &= \left( \mathbf{1}\Xi(r)\pi\right)^{-1} \Xi(r) \pi,
\end{align*}

\vspace*{-2pt}

\noindent
где $\mathbf{1} \ebd (1,\ldots,1)$~--- век\-тор-стро\-ка 
подходящей раз\-мер\-ности, составленная из единиц.

%Далее для краткости записи зависимость от~$r$ в~обозначениях~$\Xi(r)$ 
%и~$\Theta(r)$ будет опущена. 
Верна следующая цепочка неравенств:

 \vspace*{-3pt}

\noindent
\begin{multline}
\overline{\Sigma}_{t_r}(\pi)=%
%\me{}{\left\| 
%\widehat{X}_{t_r}(\pi, Y_1,\ldots,Y_r) - \overline{X}_{t_r}(\pi, Y_1,\ldots,Y_r)
%\right\|_1} =\\=
{\sf E}\left\{\left\| 
\fr{1}{\mathbf{1}\left(\Xi(r) + \Theta(r)\right)\pi} \left(\Xi(r) +{}\right.\right.\right.\\[-1pt]
\left.\left.\left.{}+ \Theta(r)\right)\pi
- \fr{1}{\mathbf{1}\Xi(r)\pi}\,\Xi(r) \pi
\right\|_1\right\} ={} \\[-1pt]
{}=
{\sf E}\left\{\fr{1}{\mathbf{1}\left(\Xi(r) + \Theta(r)\right)\pi \mathbf{1}\Xi(r)\pi}
\left\|
 \mathbf{1}\Xi(r) \pi \Theta(r)\pi -{}\right.\right.\\[-1pt]
\left.\left. {}- \mathbf{1}\Theta(r)\pi \Xi(r) \pi
 \right\|_1
 \vphantom{\fr{1}{\mathbf{1}\left(\Xi(r) + \Theta(r)\right)\pi \mathbf{1}\Xi(r)\pi}}
\right\} \leqslant {}\\[-1pt]
{}\leqslant 
{\sf E}\left\{\fr{1}{\mathbf{1}\left(\Xi(r) + \Theta(r)\right)\pi \mathbf{1}\Xi(r)\pi}
\left(
\mathbf{1}\Xi(r)\pi \| \Theta(r)\pi \|_1 +{}\right.\right.\\[-1pt]
\left.\left.{}+ \mathbf{1}\Theta(r)\pi 
\|
\Xi(r) \pi
\|_1
\right)
 \vphantom{\fr{1}{\mathbf{1}\left(\Xi(r) + \Theta(r)\right)\pi \mathbf{1}\Xi(r)\pi}}
\right\} ={}\\[-1pt]
{}=
2\,{\sf E}\left\{\fr{1}{\mathbf{1}\left(\Xi(r) + \Theta(r)\right)\pi}\mathbf{1}\Theta(r)\pi 
\right\}.
\label{eq:ineq_1}
\end{multline}

 
 \noindent
 Рассмотрим случайные события $a_{\ell} \ebd \{\omega \in \Omega: 
 N_{\ell}(\omega) \hm\leqslant n\}$, $\ell \hm= \overline{1,r}$, и~$A_r \ebd \{
 \omega\hm \in \Omega: \max_{1 \leqslant {\ell} \leqslant r}N_{\ell}(\omega) 
 \hm\leqslant n
 \}\hm=\prod\nolimits_{\ell=1}^r a_{\ell}$ и~оценку 
 $
 \widetilde{X}_{t_r}(\pi, Y_1,\ldots,Y_r)\ebd$\linebreak $\ebd
 {\sf E}\left\{X_{t_r}(\omega)\mathbf{I}_{A_r}(\omega)|\mathcal{O}_r\right\}.
 $
 Используя введенные выше обозначе\-ния и~абстрактный вариант формулы Байеса, 
 получаем, что
 
 \noindent
\begin{align}
\widetilde{X}_{t_r}& = \fr{1}{{\mathbf{1}\left(\Xi(r) + 
 \Theta(r)\right)\pi}}\,\Xi(r)\pi\,;\notag
 \\
\widehat{X}_{t_r} - \widetilde{X}_{t_r} &=
{\sf E}\left\{X_{t_r}(\omega)\mathbf{I}_{\overline{A}_r}(\omega)|\mathcal{O}_r\right\} ={}\notag\\[-1pt]
&\hspace*{17mm}{}= 
\fr{1}{\mathbf{1}\left(\Xi(r) + \Theta(r)\right)\pi}\Theta(r)\pi\,. 
\label{eq:eq_2}
 \end{align}
 Из (\ref{eq:ineq_1}) и~(\ref{eq:eq_2}) для $r\hm=1$ следует, что
 
 \vspace*{-4pt}
 
 \noindent
 \begin{multline}
 \overline{\sigma}(\pi) \leqslant 2\,{\sf E}
 \left\{\|{\sf E}\left\{X_{t_1}(\omega)\mathbf{I}_{\overline{a}_1}(\omega)|\mathcal{O}_1
 \right\}\|_1
 \right\} ={}\\[-1.5pt]
 {}=
 2\,{\sf E}\left\{\sum\limits_{n=1}^N {\sf E}
 \left\{X^n_{t_1}(\omega)\mathbf{I}_{\overline{a}_1}
 (\omega)|\mathcal{O}_1\right\}\right\} ={} \\[-2pt] 
 {}=
  2\,{\sf E}\left\{{\sf E}\left\{\mathbf{I}_{\overline{a}_1}(\omega)|\mathcal{O}_1
  \right\}\right\} =
   2 \mathbf{P}\left\{\overline{a}_1(\omega)\right\}.
\label{eq:ineq_3}
\end{multline}

 \vspace*{-2pt}
 
 \noindent
 Процесс $N^X_t$ общего числа скачков состояния~$X_t$ является считающим, и~его
  квадратическая характеристика равна 
  
\vspace*{-2pt}
  
  \noindent
 $$
 \langle N^X, N^X\rangle_t = - \int\limits_0^t \sum\limits_{n=1}^N \lambda_{nn} X_s^n\,ds\,,
 $$
 поэтому искомая вероятность ограничена сверху:
 $$ 
 \mathbf{P}\left\{\overline{a}_1(\omega)\right\} \leqslant 
 e^{-\overline{\lambda}\Delta}\sum\limits_{k=n+1}^{\infty} 
 \fr{(\overline{\lambda}\Delta)^{k}}{k!} <
 \fr{(\overline{\lambda}\Delta)^{n+1}}{(n+1)!}.
 $$
 
  \vspace*{-2pt}
  
  \noindent
 Из последнего неравенства и~(\ref{eq:ineq_3}) следует, что  для любого 
 начального распределения~$\pi$ выполняется неравенство $\overline{\sigma}(\pi)  
 \hm< 2({(\overline{\lambda}\Delta)^{n+1}}/{(n+1)!})$, т.\,е.\ 
 локальная оценка~(\ref{eq:prec_loc}) верна.
 
 С помощью марковского свойства пары $(X_t, N^X_t)$ и~последнего 
 неравенства можно оценить сверху вероятность 
 $\mathbf{P}\left\{\overline{A}_r(\omega)\right\}$:
 
  \vspace*{-2pt}
 
 \noindent
 \begin{multline*}
 \mathbf{P}\left\{\overline{A}_r(\omega)\right\} \leqslant 1 - \left(
 1- \fr{(\overline{\lambda}\Delta)^{n+1}}{(n+1)!}
 \right)^r \leqslant r \fr{(\overline{\lambda}\Delta)^{n+1}}{(n+1)!} + {}\\[-1pt]
 {}+\left|
 \sum\limits_{k=2}^r C_r^k \left(-\fr{(\overline{\lambda}\Delta)^{n+1}}{(n+1)!}
 \right)^k
 \right| \leqslant
 r \fr{(\overline{\lambda}\Delta)^{n+1}}{(n+1)!} +{}\\[-1pt]
 {}+\fr{r(r-1)}{2}
 \left(
 \fr{(\overline{\lambda}\Delta)^{n+1}}{(n+1)!}
 \right)^2
 \left(
 1-\fr{(\overline{\lambda}\Delta)^{n+1}}{(n+1)!}
 \right)^{r-2},
 \end{multline*} 
 из чего следует истинность глобальной оценки~(\ref{eq:prec_glob}).
Теорема~1 доказана.

}

%\vspace*{-12pt}

{\small\frenchspacing
 {%\baselineskip=10.8pt
 \addcontentsline{toc}{section}{References}
 \begin{thebibliography}{99}

\bibitem{Won_65}
\Au{Wonham W.} 
Some applications of stochastic differential equations to optimal
  nonlinear filtering~//
SIAM~J.~Control, 1965. Vol.~2. P.~347--369. 

\bibitem{KP_92}
\Au{Kloeden P., Platen E.} Numerical solution of stochastic
differential equations.~--- Berlin: Springer, 1992.~636~p.

\bibitem{YZL_04}
\Au{Yin G., Zhang Q., Liu Y.} 
Discrete-time approximation of Wonham filters~//
J.~Control Theory Applications, 2004. Iss.~2. P.~1--10.

\bibitem{PR_10}
\Au{Platen E., Rendek R.}
Quasi-exact approximation of hidden Markov chain filters~//
Communicat.~Stoch.~Analys., 2010. Vol.~4. Iss.~1. P.~129--142.

\bibitem{B_18}
\Au{Борисов А.} Фильтрация Вонэма по наблюдениям с~мультипликативными шумами~// 
Автоматика и~телемеханика, 2018.
№~1. C.~52--65. 
 
  \bibitem{BSh_85} %6
\Au{Бертсекас Д., Шрив С.} Стохастическое оптимальное управление. 
Случай дискретного времени~/ Пер. с~англ.~--- М.: Наука, 1985.~280~c.
(\Au{Betsekas~D.\,P., Shreve~S.\,E.} Stochastic optimal control:
The discrete-time case.~--- Orlando, FL, USA:
Academic Press Inc., 1978. 323~p.)

  \bibitem{ZhSh_95} %7
\Au{Жакод Ж., Ширяев А.} Предельные теоремы для случайных процессов,~I.~/
Пер. с~англ.~--- 
М.: Физматлит, 1995.~544~c.
(\Au{Jacod~J., Shiryaev~A.} Limit theorems for stochastic processes.~---
Berlin: Springer, 2003. 664~p.)

\bibitem{S_00}
\Au{Sericola B.} Occupation times in Markov processes~//
Commun. Stat. Stochastic Models, 2000. Vol.~16. Iss.~5. P.~479--510. 

  \bibitem{B_80}
\Au{Боровков А.} Асимптотические методы в~тео\-рии массового обслуживания.~--- 
М.: Физматлит, 1995.~384~c.

  \bibitem{B_17_1}
\Au{Борисов А.} Классификация по непрерывным наблюдениям с~мультипликативными шумами.~I. 
Формулы байесовской оценки~// Информатика и~её применения, 2017. Т.~11. Вып.~1. C.~11--19.
doi: 10.14357/19922264170102.

  \bibitem{B_17_2}
\Au{Борисов А.} Классификация по непрерывным наблюдениям с~мультипликативными 
шумами.~II. Алгоритм численной реализации оценки~// Информатика и~её 
применения, 2017. Т.~11. Вып.~2. C.~33--41.
doi: 10.14357/19922264170204.

 \end{thebibliography}

 }
 }

\end{multicols}

\vspace*{-4pt}

\hfill{\small\textit{Поступила в~редакцию 10.07.18}}

\vspace*{6pt}

%\pagebreak

%\newpage

%\vspace*{-28pt}

\hrule

\vspace*{2pt}

\hrule

%\vspace*{-2pt}

\def\tit{FILTERING OF~MARKOV JUMP PROCESSES\\ BY~DISCRETIZED OBSERVATIONS}

\def\titkol{Filtering of Markov jump processes by discretized observations}

\def\aut{A.\,V.~Borisov}

\def\autkol{A.\,V.~Borisov}

\titel{\tit}{\aut}{\autkol}{\titkol}

\vspace*{-11pt}


\noindent
Institute of Informatics Problems, Federal Research Center ``Computer Science 
and Control'' of the Russian Academy of Sciences, 44-2~Vavilov Str., Moscow 
119333, Russian Federation


\def\leftfootline{\small{\textbf{\thepage}
\hfill INFORMATIKA I EE PRIMENENIYA~--- INFORMATICS AND
APPLICATIONS\ \ \ 2018\ \ \ volume~12\ \ \ issue\ 3}
}%
 \def\rightfootline{\small{INFORMATIKA I EE PRIMENENIYA~---
INFORMATICS AND APPLICATIONS\ \ \ 2018\ \ \ volume~12\ \ \ issue\ 3
\hfill \textbf{\thepage}}}

\vspace*{6pt}



\Abste{The article is devoted to a~solution of the optimal filtering problem 
of a~homogenous Markov
jump process state. The available observations represent 
time increments of the integral transformations of the Markov\linebreak\vspace*{-12pt}}

\Abstend{state corrupted by 
Wiener processes. The noise intensity is also state-dependent. At the instant of 
the consecutive
observation obtaining, the optimal estimate is calculated recursively 
as a~function of previous estimate and the new observation, meanwhile between 
observations the filtering estimate is a simple forecast by virtue of the Kolmogorov 
differential system. The recursion is rather expensive because of  need to calculate 
the integrals, which are the location-scale mixtures of Gaussians. The mixing 
distributions represent the occupation of the state in each of possible values 
during the mid-observation intervals. The paper contains numerically cheaper 
approximations, based on the restriction of the state transitions number between 
the observations. Both the local and global characteristics of approximation 
accuracy are obtained as functions of the dynamics parameters, mid-observation 
interval length, and upper bound of transitions number.}

\KWE{Markov jump process; optimal filtering; multiplicative observation noises; 
stochastic differential equation; numerical approximation}




\DOI{10.14357/19922264180316}

%\vspace*{-14pt}

\Ack
\noindent
The work was supported in part by the Russian Foundation
for Basic Research (Project No.\,16-07-00677).



%\vspace*{6pt}

  \begin{multicols}{2}

\renewcommand{\bibname}{\protect\rmfamily References}
%\renewcommand{\bibname}{\large\protect\rm References}

{\small\frenchspacing
 {%\baselineskip=10.8pt
 \addcontentsline{toc}{section}{References}
 \begin{thebibliography}{99}
\bibitem{Won_65-1}
\Aue{Wonham, W.} 1965.
Some applications of stochastic differential equations to optimal
  nonlinear filtering.
\textit{SIAM~J.~Control} 2:347--369. 

\bibitem{KP_92-1}
\Aue{Kloeden,~P., and E.~Platen.} 1992. \textit{Numerical solution of stochastic
differential equations.} Berlin: Springer. 636~p.

\bibitem{YZL_04-1}
\Aue{Yin,~G., Q.~Zhang, and Y.~Liu.} 2004.
Discrete-time approximation of Wonham filters.
\textit{J.~Control Theory Applications} 2:1--10.

\bibitem{PR_10-1}
\Aue{Platen, E., and R.~Rendek.} 2010.
Quasi-exact approximation of hidden Markov chain filters.
\textit{Communicat. Stoch. Analys.} 4(1):129--142.

\bibitem{B_18-1}
\Aue{Borisov, A.} 2018. Wonham filtering by observations
with multiplicative noises. \textit{Automat.~Rem.~Contr.} 79(1):39--50.  
doi: 10.1134/ S0005117918010046.
 
  \bibitem{BSh_85-1}
\Aue{Bertsekas, D., and S.~Shreve.} 1996.
\textit{Stochastic optimal control: The discrete-time case}.
Nashua, NH: Athena Scientific. 330~p.
  
  \bibitem{ZhSh_95-1}
  \Aue{Jacod,~J., and A.~Shiryaev.} 2003.
\textit{Limit theorems for stochastic processes.}
Berlin: Springer. 664~p.

\bibitem{S_00-1}
\Aue{Sericola, B.}
2000. Occupation times in Markov processes.
\textit{Commun. Stat.} 16(5):479--510. 

  \bibitem{B_80-1}
\Aue{Borovkov, A.} 1984.
 \textit{Asymptotic methods in queueing theory}. 
 Hoboken, NJ: Wiley-Blackwell.~304~p.

  \bibitem{B_17_1-1}
  \Aue{Borisov, A.} 2017. 
  Klassifikatsiya po ne\-pre\-ryv\-nym nablyu\-de\-miyam s~mul'tiplikativnymi shumami. I. 
  Formuly bayesov\-skoy otsenki [Classification by continuous-time observations
in multiplicative noise. I.~Formulae for Bayesian 
estimate]. \textit{Informatika i~ee Primeneniya~--- Inform.~Appl.}
11(1):11--19. doi: 10.14357/19922264170102.

  \bibitem{B_17_2-1}
\Aue{Borisov, A.} 2017. Klassifikatsiya po nepreryvnym nablyudemiyam 
s~mul'tiplikativnymi summami. II.~Formuly bayesovskoy otsenki 
[Classification by continuous-time observations
in multiplicative noise. II.~Numerical algorithm].
\textit{Informatika i~ee Primeneniya~--- Inform.~Appl.}
11(2):33--41. doi: 10.14357/19922264170204.

\end{thebibliography}

 }
 }

\end{multicols}

\vspace*{-6pt}

\hfill{\small\textit{Received July 10, 2018}}

%\pagebreak

%\vspace*{-18pt}

\Contrl

\noindent
\textbf{Borisov Andrey V.} (b.\ 1965)~--- 
Doctor of Science in physics and mathematics, principal scientist, Institute of
Informatics Problems, Federal Research Center ``Computer Science and Control''
 of the Russian Academy of
Sciences, 44-2 Vavilov Str., Moscow 119333, Russian Federation; 
\mbox{aborisov@frccsc.ru}
\label{end\stat}

\renewcommand{\bibname}{\protect\rm Литература}        %2
 \def\stat{stef+sushko}

\def\tit{ОБРАТИМОЕ СЖАТИЕ ДАННЫХ ПОСРЕДСТВОМ УНИВЕРСАЛЬНОГО АРИФМЕТИЧЕСКОГО КОДИРОВАНИЯ}

\def\titkol{Обратимое сжатие данных посредством универсального арифметического кодирования}

\def\aut{А.\,И.~Стефанович$^1$, Д.\,В.~Сушко$^2$}

\def\autkol{А.\,И.~Стефанович, Д.\,В.~Сушко}

\titel{\tit}{\aut}{\autkol}{\titkol}

\index{Стефанович А.\,И.}
\index{Сушко Д.\,В.}
\index{Stefanovich A.\,I.}
\index{Sushko D.\,V.}


%{\renewcommand{\thefootnote}{\fnsymbol{footnote}} \footnotetext[1]
%{Работа выполнена при финансовой поддержке РФФИ (проекты 16-07-00677 
%и~15-37-20611-мол\_а\_вед).}}


\renewcommand{\thefootnote}{\arabic{footnote}}
\footnotetext[1]{Институт проблем информатики Федерального исследовательского центра
 <<Информатика и~управление>> Российской академии наук, \mbox{astefanovich@ipiran.ru}}
\footnotetext[2]{Институт проблем информатики Федерального исследовательского центра 
<<Информатика и~управление>> Российской академии наук, \mbox{dsushko@ipiran.ru}}


\Abst{Рассмотрен общий подход к~задаче обратимого сжатия, т.\,е.\
 сжатия без потерь, цифровых данных, основанный на универсальном 
 арифметическом кодировании данных с~неизвестной статистикой. 
 Для описания данных используется модель источника с~вычислимой 
 последовательностью состояний. В~рамках этого подхода сформулированы задачи, 
 решение которых для данных конкретного типа позволяет получить конкретные методы 
 и~алгоритмы сжатия. В~качестве объекта исследования рассмотрены данные 
 компьютерной томографии. Предложены два метода обратимого сжатия томограмм. 
 Первый предполагает кодирование ошибок предсказания, второй~--- 
 кодирование компонент двумерного дискретного вей\-в\-лет-пре\-об\-ра\-зо\-ва\-ния. 
 Проведено подробное исследование этих методов, построены эффективные 
 алгоритмы их реализации и~получены индивидуальные оценки скорости 
 кодирования алгоритмов. Представлены результаты сравнения скоростей 
 кодирования томограмм построенными алгоритмами и~алгоритмами стандарта JPEG~2000. 
 Результаты демонстрируют высокое качество построенных алгоритмов, а~также 
 свидетельствуют о~больших потенциальных возможностях рассмотренного подхода в~целом.}

\KW{обратимое сжатие данных; сжатие без потерь; универсальное кодирование; арифметическое кодирование; компьютерная томограмма}

\DOI{10.14357/19922264170103} 


\vskip 10pt plus 9pt minus 6pt

\thispagestyle{headings}

\begin{multicols}{2}

\label{st\stat}

\section{Введение}
%\label{sec0}

На протяжении последних нескольких десятилетий наблюдается бурный рост 
объема цифровых данных, накапливаемых в~результате проведения различных 
научных экспериментов, медицинских исследований и~т.\,д. Необходимость долгосрочного 
хранения (архивации) и~обмена такими данными делают задачу их сжатия (кодирования 
в~целях уменьшения объема данных) актуальной современной задачей. Во 
многих случаях важным дополнительным требованием к~процедуре сжатия является 
ее обратимость (т.\,е.\ отсутствие искажений, или потерь при кодировании). 
Это требование в~случае данных медицинских исследований часто продиктовано 
соображениями законодательного характера, а~в~случае данных научных исследований~--- 
высокой стоимостью и~трудоемкостью эксперимента и/или уникальностью данных.

В настоящей работе рассмотрен общий метод (общий подход), 
предназначенный для решения задачи обратимого сжатия цифровых данных. 
Данный метод, основанный на использовании универсального арифметического 
кодирования, впервые был предложен в~работе~\cite{b01}. Построение в~рамках 
общего подхода некоторого конкретного метода, предназначенного для сжатия 
данных определенного типа, связано с~необходимостью решить ряд задач по 
адаптации общего метода к~таким данным. Постановки соответствующих задач 
приведены в~работе.

В качестве объекта исследования в~работе используются данные компьютерной 
томографии (томограммы). Для данных указанного типа в~рамках общего подхода 
построены два различных метода\linebreak сжатия. В~работе проведено подробное 
исследова\-ние этих методов, предложены эффективные алго\-ритмы их реализации и~получены 
индивидуальные оценки скорости кодирования (степени сжатия) этих алгоритмов.

Поскольку томограммы представляют собой изображения, их обратимое сжатие может 
быть осуществлено альтернативным способом в~рамках группы методов стандарта JPEG~2000. 
В~ходе проведенных исследований было осуществлено сжатие томограмм посредством 
эталонной реализации (Jasper) стандарта JPEG~2000 и~произведено сравнение 
скоростей кодирования алгоритмов JPEG~2000 и~эффективных алгоритмов, предложенных 
в~работе. Данное сравнение продемонстрировало, во-пер\-вых, 
качество разработанных алгоритмов и,~во-вто\-рых, высокий потенциал общего 
метода в~целом.

Работа имеет следующую структуру. В разд.~2 приведены 
необходимые сведения об арифметическом кодировании, описана общая 
схема универсального кодирования и~в~общем виде сформулированы задачи, решение 
которых необходимо при адаптации общего метода универсального кодирования 
к~конкретному типу данных. Раздел~3 содержит краткую информацию 
о~том, что пред\-став\-ля\-ют собой компьютерные томограммы. В~разд.~4 
общая схема универсального кодирования адаптирована таким образом, чтобы 
ее можно было применить для сжатия значений ошибок предсказания компьютерной 
томограммы. Основным результатом раздела является построение метода сжатия, 
основанного на кодировании ошибок предсказания, и~построение эффективных оценок 
ско\-рости кодирования метода. В~разд.~5 аналогичные результаты получены 
для метода сжатия, ориентированного на кодирование значений компонент дискретного 
вей\-в\-лет-пре\-об\-ра\-зо\-ва\-ния томограммы. 
В~заключение приведено сравнение эф\-фек\-тив\-ности построенных методов и~методов JPG~2000.

\vspace*{-9pt}

\section{Универсальное арифметическое кодирование}
%\label{sec1}

В настоящем разделе приведено описание общей схемы предлагаемого метода 
универсального арифметического кодирования. Обозначены постановки основных задач, 
решение которых обеспечивает эффективное применение этой схемы для сжатия данных 
конкретного типа.

\vspace*{-9pt}

\subsection{Арифметическое кодирование}
%\label{sec11}

Приведем сведения об арифметическом кодировании в~необходимом для дальнейшего 
изложения объеме. Подробному рассмотрению арифметического кодирования посвящена, 
например, работа~\cite{b02}.

Пусть ${\cal A}= \{a\}$~--- некоторое конечное множество (алфавит), 
состоящее из $A\doteq|{\cal A}|$ элементов, и~пусть  $\mathbf{x}\hm=\{x_n\}$,  
$n\hm=0,\ldots,N-1$,~--- подлежащая кодированию последовательность элементов 
множества~${\cal A}$. Процесс кодирования заключается в~последовательном 
просмотре всех значений~$x_n$, вычислении кодовой вероятности~$Q(\mathbf{x})$ 
(положительного вещественного числа, не превышающего единицы) и~формировании по 
этой кодовой вероятности двоичного кодового слова (результата сжатия) 
длины~$L(\mathbf{x})$ битов:
\begin{equation}
\label{eq1}
L(\mathbf{x}) = \left[-\fr{\log_{2}Q(\mathbf{x})}{2}\right]_{-}+1 \leq 
-\log_{2}Q(\mathbf{x})+2\,.
\end{equation}

\columnbreak

\noindent
Здесь и~далее $[\cdot]_{-}$~--- целая часть числа. Вычисление кодовой 
вероятности осуществляется рекуррентно. Начальная кодовая вероятность 
выбирается равной единице  ($Q_{-1}\hm=1$). 
В~момент поступления на вход кодера очередного значения~$x_n$ 
кодеру долж\-но быть известно (задано заранее и/или сфор\-мировано в~процессе 
кодирования предыдущих элементов) условное кодовое распределение 
вероятностей  $\{q_n(a|x_{n-1},\ldots,x_0)$, 
$a\hm\in{\cal A}\}$. {\it Условное кодовое распределение}~--- 
это набор неотрицательных вещественных чисел, таких что
\begin{equation}
\label{eq2}
\sum_{a\in{\cal A}}q_n(a|x_{n-1},\dots,x_0) = 1\,;
\end{equation}
кроме того, равенство $q_n(a|x_{n-1},\dots,x_0)\hm=0$ для некоторого
 конкретного значения~$a$ допустимо только в~том случае, если 
 выполнение равенства $x_n\hm=a$ невозможно априори. Шаг рекурсии заключается 
 в~умножении текущей кодовой вероятности на условную кодовую вероятность 
 значения~$x_n$:
\begin{multline*}
Q_n\left(x_0,\dots,x_n\right)={}\\
{}=Q_{n-1}\left(x_0,\dots,x_{n-1}\right)
q\left(x_n|x_{n-1},\dots,x_0\right)\,.
\end{multline*}
Результатом выполнения~$N$~шагов рекурсии является вычисление кодовой вероятности 
всей последовательности:
\begin{equation}
\label{eq3}
Q(\mathbf{x}) =\prod\limits_{n=0}^{N-1} q_n\left(x_n|x_{n-1},\dots,x_0\right)\,.
\end{equation}
Детали процедуры формирования кодового слова по кодовой вероятности не 
принципиальны для рассмотрения и~опускаются.

Восстановление исходных данных по кодовому слову осуществляется 
декодером последовательно и~без задержки. В~момент восстановления очередного 
значения~$x_n$ декодеру уже известны все предыдущие значения $\{x_0,\dots,x_{n-1}\}$ 
и,~кроме того, должно быть известно условное кодовое распределение 
$\{q_n(a|x_{n-1},\dots,x_0)$, $a\hm\in{\cal A}\}$, использованное ранее 
в~процессе кодирования. Это позволяет декодеру восстановить значение~$x_n$.

Таким образом, ключевую роль в~процессе арифметического кодирования играют 
условные кодовые распределения вероятностей $\{q_n(a|x_{n-1},\dots,x_0)$,
$a\hm\in{\cal A}$, $n\hm=0,\ldots,N-1\}$, выбор которых определяет длину 
кодового слова, т.\,е.\ степень сжатия исходных данных. При этом 
построение условных кодовых распределений, обеспечивающих получение 
возможно более коротких кодовых слов для входных данных с~неизвестной 
(не полностью известной) статистикой,~--- задача универсального кодирования.

\subsection{Статистическая модель}

В качестве статистической модели исходных данных используем так 
называемую модель \textit{источника с~вычислимой последовательностью состояний}. 
В~основе модели лежит следующее предположение: вероятность того, что значение~$x_n$ 
очередного элемента последовательности равна заданному значению
$a\hm\in{\cal A}$, зависит только от значений~$\tau$ предшествующих элементов 
последовательности, т.\,е.\ 
$p(x_n=a)\hm=p(x_n|x_{n-1},\ldots,x_{n-\tau})$. Пусть~${\cal S}$~--- 
некоторое подмножество множества  ${\cal A}^\tau \hm= 
\underbrace{{\cal A}\times\dots\times{\cal A}}_{\tau}$. 
Назовем ${\cal S}\hm\subset{\cal A}^\tau $ состоянием (источника), если
\begin{multline*}
p\left(x_n|x'_{n-1},\dots,x'_{n-\tau}\right) = 
p\left(x_n|x''_{n-1},\dots,x''_{n-\tau}\right)\\
\forall \, \left(x'_{n-1},\dots,x'_{n-\tau}\right),\,
\left(x''_{n-1},\dots,x''_{n-\tau}\right)\in{\cal S}\,.
\end{multline*}
Для условного распределения вероятностей, соответствующего состоянию~${\cal S}$, 
используем обозначение~$p(a|{\cal S})$. Множество состояний ${\frak S}\hm=\{\cal S\}$ 
назовем полным множеством независимых состояний, если
$$
\bigcup\limits_{{\cal S}\in{\frak S}}{\cal S} = {\cal A}^\tau\,;
\quad  \quad
{\cal S}'\bigcap{\cal S}'' = \varnothing\quad
\forall\,{\cal S}',{\cal S}''\in{\frak S}\,.
$$
Все рассматриваемые далее множества состояний являются полными и~независимыми.

Название модели~--- модель источника с~вы\-чис\-ли\-мой последовательностью состояний~--- 
связано со следующей возможной ее интерпретацией. Элементы 
последовательности~$\mathbf{x}\hm=\{x_n\}$ один за другим <<порождаются>> 
источником данных, который в~каж\-дый <<момент времени~$n$>>  
находится в~некотором состоянии~${\cal S}$ из множества состояний 
источника~${\frak S}$; при этом $p(x_n\hm=a)\hm= p(x_n|{\cal S})$. 
Для краткости можно говорить об элементе~$x$ последовательности, <<порожденном>> 
источником в~состоянии~${\cal S}$, как об элементе состояния~${\cal S}$ и~записывать 
это в~виде $x\hm\in{\cal S}$.

В соответствии с~принятой моделью данных естественно использовать общее условное 
кодовое распределение вероятностей $\{q(a|{\cal S}),\,a\hm\in{\cal A}\}$ 
при ко\-ди\-ро\-ва\-нии-де\-ко\-ди\-ро\-ва\-нии всех значений, <<порождаемых>> 
источником в~каждом отдельном состоянии. Всего в~процессе кодирования 
используется $S\doteq|{\frak S}|$ различных условных кодовых 
распределений вероятностей (по числу состояний источника). Поскольку 
арифметическое кодирование осуществляется последовательно, а~декодирование~--- 
последовательно и~без задержки, в~момент ко\-ди\-ро\-ва\-ния-де\-ко\-ди\-ро\-ва\-ния 
очередного значения все предыдущие значения уже известны как кодеру, так и~декодеру. 
Поэтому как кодер, так и~декодер в~состоянии вычислить текущее состояние источника 
и~использовать соответствующее условное кодовое распределение.

\textit{Скоростью кодирования} (средней скоростью кодирования)~$V$ называется отношение 
длины кодового слова~$L(\mathbf{x})$ к~числу элементов~$N$ кодируемой 
последовательности; единица измерения скорости кодирования~--- 
бит/пик\-сель (б/п). С~учетом принятых предположений 
из формул~(\ref{eq1}) и~(\ref{eq3}) с~точностью до малого члена порядка~$\sim 2/N$ 
имеем:
\begin{equation}
\label{eq4}
V(\mathbf{x}) = \sum\limits_{{\cal S}\in{\frak S}}\fr{N({\cal S})}{N}
\sum\limits_{x\in{\cal A}}\fr{N(x|{\cal S})}{N({\cal S})} 
\left[-\log_{2}q(x|{\cal S})\right]\,,
\end{equation}
где $N({\cal S})$~--- число элементов в~состоянии~${\cal S}$; $N(x|{\cal S})$~--- 
число элементов в~состоянии~${\cal S}$, принимающих значение~$x$; внешняя 
сумма берется по всем состояниям источника; внутренняя сумма~--- 
по всем встречающимся в~данном состоянии значениям. Если использовать соглашение о~том, 
что $0\cdot\log0\hm=0$, то внутреннюю сумму можно распространить на 
все множество значений~$\cal A$. Действительно, в~силу предъявляемых 
к~условным кодовым вероятностям требований равенство $q(x|{\cal S})\hm=0$ 
влечет $N(x|{\cal S})\hm=0$.

Величины $N({\cal S})/N$  (${\cal S}\hm\in{\frak S} $) и~$N(x|{\cal S})/N({\cal S})$  
($x\hm\in{\cal S}\,,{\cal S}\hm\in{\frak S}$) образуют соответственно 
частотное распределение для состояний и~условные частотные распределения значений 
в~со\-сто\-яни\-ях. Используя для этих величин обозначения~$f(\cal S)$ 
и~$f(x|\cal S)$, перепишем формулу~(\ref{eq4}) для скорости кодирования в~виде:
\begin{equation}
\label{eq5}
V(\mathbf{x}) = \sum\limits_{{\cal S}\in{\frak S}} f({\cal S}) V(\mathbf{x}|{\cal S})\,,
\end{equation}
где $V(\mathbf{x}|{\cal S})$--- скорость кодирования подпоследовательности элементов состояния~${\cal S}$, 
или скорость кодирования состояния~${\cal S}$:
\begin{equation}
\label{eq6}
V(\mathbf{x}|{\cal S}) = \sum\limits_{x\in{\cal A}} 
f(x|{\cal S}) \left[-\log_{2}q(x|{\cal S})\right]\,.
\end{equation}
 Формулу для скорости кодирования 
состояния можно тождественно переписать в~виде суммы двух слагаемых:
\begin{equation}
\label{eq7}
V(\mathbf{x}|{\cal S}) = H(\mathbf{x}|{\cal S})+R(\mathbf{x}|{\cal S})\,,
\end{equation}
где
\begin{align}
\label{eq8}
H(\mathbf{x}|{\cal S}) &= \sum\limits_{x\in{\cal A}} f(x|{\cal S}) 
\left[-\log_{2}f(x|{\cal S})\right]\,;
\\
\label{eq9}
R(\mathbf{x}|{\cal S}) &= \sum\limits_{x\in{\cal A}} f(x|{\cal S}) 
\left[-\log_{2}\fr{q(x|{\cal S})}{f(x|{\cal S})}\right]\,.
\end{align}
Первое слагаемое $H(\mathbf{x}|{\cal S})$~--- 
это \textit{квазиэнтропия} (или эмпирическая энтропия) состояния. 
Квазиэнтропия не зависит от условного кодового распределения и,~очевидно, 
является неотрицательной  ($H(\mathbf{x}|{\cal S})\hm\ge 0$). Рассмотрим 
второе слагаемое $R(\mathbf{x}|{\cal S})$~--- 
избыточность кодирования состояния. Учитывая справедливое для всех $\alpha\hm>0$ 
элементарное неравенство
\begin{equation}
\label{eq10}
-\log_{2}(\alpha) \ge 1-\alpha\,,
\end{equation}
обращающееся в~равенство только в~случае $\alpha\hm=1$, имеем:
\begin{multline*}
R(\mathbf{x}|{\cal S}) = \sum\limits_{x\in{\cal A}} f(x|{\cal S}) 
\left[-\log_{2}\fr{q(x|{\cal S})}{f(x|{\cal S})}\right] \ge{}\\
{}\ge
\sum\limits_{x\in{\cal A}} f(x|{\cal S}) \left[1 - 
\fr{q(x|{\cal S})}{f(x|{\cal S})}\right] ={}\\
{}=
\sum\limits_{x\in{\cal A}} \left[f(x|{\cal S}) - q(x|{\cal S})\right] = 0\,,
\end{multline*}
т.\,е. $R(\mathbf{x}|{\cal S})\hm\ge0$, причем $R(\mathbf{x}|{\cal S})\hm=0$ тогда 
и~только тогда, когда $f(x|{\cal S})= q(x|{\cal S})$. Таким образом, 
показано, что квазиэнтропия состояния~--- это минималь\-ная скорость кодирования 
со\-сто\-яния, которая достигается при обращении в~нуль из\-бы\-точ\-ности, т.\,е.\ 
при использовании частотных вероятностей в~качестве кодовых вероятностей. 
Подставляя теперь~(\ref{eq7}) в~(\ref{eq5}), получаем скорость кодирования 
(всей последовательности) в~виде суммы двух не\-от\-ри\-ца\-тель\-ных слагаемых:
\begin{equation}
\label{eq11}
V(\mathbf{x}) = H(\mathbf{x}) + R(\mathbf{x})\,,
\end{equation}
равных
\begin{multline}
\label{eq12}
H(\mathbf{x}) = \sum\limits_{{\cal S}\in{\frak S}} f({\cal S}) 
H(\mathbf{x}|{\cal S}) \equiv{}\\
{}\equiv
\sum\limits_{{\cal S}\in{\frak S}} f({\cal S}) 
\sum\limits_{x\in{\cal A}} f(x|{\cal S}) \left[-\log_{2}f(x|{\cal S})\right]\,;
\end{multline}

\vspace*{-12pt}

\noindent
\begin{multline}
\label{eq13}
R(\mathbf{x}) =\sum\limits_{{\cal S}\in{\frak S}} f({\cal S}) 
R(\mathbf{x}|{\cal S}) \equiv{}\\
{}\equiv
\sum\limits_{{\cal S}\in{\frak S}} f({\cal S})
\sum\limits_{x\in{\cal A}} f(x|{\cal S})\left[
-\log_{2}\fr{q(x|{\cal S})}{f(x|{\cal S})}\right]\,,
\end{multline}
которые суть квазиэнтропия и~избыточность кодирования 
(всей последовательности) соответственно. При этом квазиэнтропия 
не зависит от условных кодовых распределений и~представляет собой 
минимальную скорость кодирования, которая достигается при обращении 
в~нуль избыточности, т.\,е.\ при использовании частотных вероятностей 
в~качестве кодовых вероятностей.

Обычно при решении практических задач сжатия данных множество состояний неизвестно. 
Более того, как правило, невозможно даже установить, насколько описанная выше 
модель источника адекватна реальным данным. При таком положении вещей данное 
выше определение состояний становится совершенно неконструктивным и~бесполезным 
с~практической точки зрения. Поэтому\linebreak определим состояния по-но\-во\-му, 
взяв за основу условные кодовые вероятности, а~именно: будем по определению 
считать состоянием под\-мно\-жество ${\cal S}\hm\subset{\cal A}^\tau$ такое, 
что значения~$x_n$ всех тех элементов последовательности, которым 
пред\-шест\-ву\-ют элементы последовательности со значениями 
$\{x_{n-1},\ldots,x_{n-\tau}\}\in{\cal S}$, кодируются одним общим 
кодовым условным распределением~$q(a|{\cal S})$. Таким образом, состояние 
характеризуется тем, что все значения его элементов кодируются одним 
распределением. При этом формулы~(\ref{eq5})--(\ref{eq13}), разумеется, 
остаются в~силе, а~задача универсального кодирования может быть сформулирована 
как задача выбора множества состояний~${\frak S}$ и~задача построения совокупности 
условных кодовых вероятностей $\{q(a|{\cal S})\}$ 
($a\hm\in{\cal A}$, ${\cal S}\hm\in{\frak S}$) для выбранного множества со\-сто\-яний.

\vspace*{-9pt}

\subsection{Выбор множества состояний}

Рассмотрим первую задачу универсального кодирования~--- выбор множества состояний. 
Вообще говоря, эта задача должна решаться отдельно для каж\-до\-го типа исходных 
данных на основе имеющейся априорной информации и/или принятой модели данных. 
Существуют, однако, некоторые общие соображения. Квазиэнтропия состояния не 
зависит от кодовых вероятностей и~определяется только частотными вероятностями 
значений в~данном состоянии. Соответственно, квазиэнтропия зависит только от 
множества состояний~${\frak S}$ в~целом. Естественно попытаться выбрать состояния 
так, чтобы минимизировать квазиэнтропию. При этом следует с~самого начала иметь в~виду, 
что возможна ситуация, когда квазиэнтропия мала, но избыточность (также зависящая 
от множества состояний) не\-до\-пус\-ти\-мо велика и,~как следствие, недопустимо 
велика и~скорость кодирования. Описанная ситуация, например, заведомо имеет 
место в~том случае, когда число состояний велико (сравнимо по величине с~количеством 
отсчетов исходных данных).

Рассмотрим множества ${\cal S}\,,{\cal S}'\,,{\cal S}''\subset{\cal A}^\tau$ такие, 
что ${\cal S}'\cap{\cal S}''\hm =\varnothing$, ${\cal S}'\cup{\cal S}'' \hm={\cal S}$. 
Снова используя неравенство~(\ref{eq10}), имеем:
\begin{multline*}
f({\cal S}) H(\mathbf{x}|{\cal S}) =
f({\cal S})\sum\limits_{x\in{\cal A}} f(x|{\cal S}) 
\left[-\log_{2}f(x|{\cal S})\right] ={} \\
{}= f\left( {\cal S}'\right)\sum\limits_{x\in{\cal A}} f\left(x|{\cal S}'\right) 
\left[-\log_{2}f(x|{\cal S})\right] +{}\\
{}+
f\left({\cal S}''\right)\sum\limits_{x\in{\cal A}} f\left(x|{\cal S}''\right) 
\left[-\log_{2}f(x|{\cal S})\right] ={} \\
{}= f\left({\cal S}'\right) H\left(\mathbf{x}|{\cal S}'\right) +{}
\end{multline*}

\noindent
\begin{multline*}
{}+
f\left({\cal S}'\right)\sum\limits_{x\in{\cal A}}
    f\left(x|{\cal S}'\right) \left[-\log_{2}\fr{f(x|{\cal S})}
    {f(x|{\cal S}')}\right] +{}\\
{}+ f\left({\cal S}''\right) H\left(\mathbf{x}|{\cal S}''\right) +{}\\
{}+
f\left({\cal S}''\right)\sum\limits_{x\in{\cal A}}
    f\left(x|{\cal S}''\right) \left[-\log_{2}\fr{f(x|{\cal S})}{f\left(x|{\cal S}''\right)}\right] \ge{} \\
{}\ge f\left({\cal S}'\right)\left\{H\left(\mathbf{x}|{\cal S}'\right) +
\sum\limits_{x\in{\cal A}} \left[f\left(x|{\cal S}'\right)-
f\left(x|{\cal S}\right)\right] \right\} +{}\\
{}+
f\left({\cal S}''\right)\left\{H\left(\mathbf{x}|{\cal S}''\right) +
\sum\limits_{x\in{\cal A}} \left[f\left(x|{\cal S}''\right)-
f(x|{\cal S})\right] \right\} ={} \\
{}= f\left({\cal S}'\right) H\left(\mathbf{x}|{\cal S}'\right) + 
f\left({\cal S}''\right) H\left(\mathbf{x}|{\cal S}''\right)\,.
\end{multline*}
Итак,
\begin{multline}
\label{eq14}
H(\mathbf{x}|{\cal S}) \ge{}\\
{}\ge
f\left({\cal S}'|{\cal S}\right) H\left(\mathbf{x}|{\cal S}'\right) + 
f\left({\cal S}''|{\cal S}\right) H\left(\mathbf{x}|{\cal S}''\right)\,,
\end{multline}
где $f({\cal S}'|{\cal S})\hm=f({\cal S}')/f({\cal S})$ 
и~$f({\cal S}''|{\cal S})\hm=f({\cal S}'')/f({\cal S})$~--- 
условные частотные вероятности состояний~${\cal S}'$ и~${\cal S}''$ 
соответственно, а~равенство имеет место только в~том случае, если $f(x|{\cal S})
\equiv f(x|{\cal S}')\equiv f(x|{\cal S}'')$. Таким образом, установлено, 
что квазиэнтропия является выпуклой функцией: при разбиении любого состояния 
квазиэнтропия не увеличивается. Поэтому критерием выбора множества~${\cal S}$ 
в~качестве состояния может служить условие
\begin{multline*}
H(\mathbf{x}|{\cal S}) -
\min\limits_{{\cal S}'\subset{\cal S}}
\left\{ f({\cal S}'|{\cal S}) H(\mathbf{x}|{\cal S}') +{}\right.\\
\left.{}+
f({\cal S}\!\setminus\!{\cal S}'|{\cal S}) H(\mathbf{x}|{\cal S}\!\setminus\!{\cal S}') 
\right\} < \varepsilon\,,
\end{multline*}
в котором значение $\varepsilon\hm>0$ должно быть выбрано исходя из 
практических требований.

Квазиэнтропия всей последовательности зависит от множества состояний: 
$H(\mathbf{x})\equiv H(\mathbf{x},{\frak S})$. Если зафиксировать общее 
число состояний~$S$, то величина
\begin{equation}
\label{eq15}
\hat{H}^{S}(\mathbf{x}) = \min\limits_{{\frak S}:\:|{\frak S}|=S} 
H(\mathbf{x},{\frak S})
\end{equation}
представляет собой оптимальную (при заданном числе состояний) 
квазиэнтропию, а~соответ\-ст\-ву\-ющее мно\-жество~$\hat{\frak S}$~--- 
оптимальное мно\-жество, которое естественно использовать в~качестве 
мно\-же\-ст\-ва состояний при кодировании.

Из неравенства~(\ref{eq14}) сразу следует, что оптимальная 
квазиэнтропия~$\hat{H}^{S}(\mathbf{x})$ при увеличении числа состояний~$S$ 
не возрастает. Это позволяет использовать условие
$$
\hat{H}^{S}(\mathbf{x}) - \hat{H}^{S+1}(\mathbf{x}) < \varepsilon
$$
как критерий для определения числа состояний. При этом конкретное 
значение $\varepsilon\hm>0$ должно выбираться исходя из практических требований.

Проверка любого из приведенных выше условий связана с~перебором 
всех подмножеств множества~${\cal A}^\tau$, что, как правило, не может 
быть реализовано на практике уже в~случае $\tau\hm=2$. Поэтому сказанное может 
рассматриваться лишь как <<общее направление движения>>: указанные критерии 
должны быть адаптированы к~конкретному типу данных с~привлечением априорной 
информации и~дополнительных гипотез.

\subsection{Построение кодовых распределений}

Выше было показано, что минимальная скорость кодирования для 
заданного множества состояний достигается тогда и~только тогда, 
когда $q(x|{\cal S})\hm\equiv f(x|{\cal S})$ для всех состояний, т.\,е.\
 в~качестве условных кодовых распределений используются условные частотные 
 распределения. Условные частотные распределения~$f(x|{\cal S})$ 
 априори не известны, но могут быть вычислены кодером по исходным 
 данным~$\mathbf{x}$, что позволяет использовать упрощенное комбинаторное 
 кодирование.

Кодовое слово комбинаторного кода состоит из двух частей. Первая часть 
(преамбула) содержит значения $N(x|{\cal S})$ для всех $x\hm\in{\cal A}$,  
${\cal S}\hm\in{\frak S}$, которые вычисляются в~процессе кодирования. 
Длина преамбулы равна $S(A-1)(\log_{2}N+1)$ бит. Вторая часть~--- 
результат арифметического кодирования последовательности $\mathbf{x}\hm=\{x_n\}$ 
с~по\-мощью частотных распределений $f(x|{\cal S})$. Получив кодовое слово, 
декодер выделяет преамбулу, <<считывает>> значения $N(x|{\cal S})$ 
и~вычисляет~$N({\cal S})$~--- суммы $N(x|{\cal S})$ по всем $x\hm\in{\cal A}$. 
В~результате становятся известными частотные распределения $f(x|{\cal S})$, 
использовавшиеся при кодировании, что позволяет однозначно декодировать 
вторую часть кодового слова и~восстановить исходную последовательность.

Избыточность описанной процедуры комбинаторного кодирования определяется 
длиной преамбулы, т.\,е.\ длиной данных, которые должны быть дополнительно 
переданы декодеру, и~равна
\begin{equation}
\label{eq16}
R = R_{\mathrm{T}} = \fr{S(A-1)(\log_{2}N+1)}{N}\,.
\end{equation}
При больших значениях~$A$ величина~(\ref{eq16}) не\-до\-пус\-ти\-мо велика. 
Действительно, при $S\hm=5$, $A\hm=2^{12}$ и~$N\hm=512^2$ имеем 
$R_{\mathrm{T}}\hm\sim1{,}4$~б/п.

Отметим, что в~общем случае (без ка\-ких-ли\-бо 
дополнительных предположений относительно исходных данных) использование 
более совершенных по сравнению с~комбинаторным кодированием методов 
позволяет уменьшить избыточность приблизительно в~два раза; это, 
однако, не решает проблему.

Приступим к~рассмотрению впервые предложенного в~работе~\cite{b01} метода 
построения кодовых распределений, который представляет собой, по существу, 
некоторую модификацию метода комбинаторного кодирования.

 Без ограничения 
общности можно считать, что множество~${\cal A}$ состоит из идущих подряд 
целых чисел. В~основе метода лежит следующее предположение: для любых исходных 
данных частотные распределения $f(x|{\cal S})$ представляют собой достаточно 
<<гладкие>> функции (точнее, отсчеты достаточно <<гладких>> функций). 


Основная идея метода заключается в~том, чтобы аппроксимировать соответствующие 
распределения простыми аналитическими функциями из зара\-нее выбранного класса, 
такого что каждая функция класса однозначно определяется значениями 
некоторого небольшого числа параметров. 
Кодер использует значения построенных аппроксимирующих функций в~качестве 
кодовых распределений, а~для передачи необходимой информации декодеру 
достаточно передать значения параметров, определяющих эти функции. При этом 
избыточность~$R_{\mathrm{T}}$, связанная с~передачей декодеру дополнительной 
информации, кардинально уменьшается, но появляется избыточность 
арифметического кодирования~$R(\mathbf{x})$ (см.\ формулы~(\ref{eq11}) и~(\ref{eq7})), 
поскольку теперь кодовые вероятности не равны частотным вероятностям. 
Величина~$R(\mathbf{x})$ определяется качеством аппроксимации 
и~в~конечном счете адекватностью используемого основного предполо-\linebreak жения.

Перейдем теперь к~более детальному рассмотрению предлагаемого метода. 
Обычно одна простая аналитическая функция не обеспечивает приемлемой 
точ\-ности аппроксимации частотных распределений $f(x|{\cal S})$  
на всем множестве~${\cal A}$. Поэтому разобьем все множество~${\cal A}$ 
на диапазоны, состоящие из идущих подряд целых чисел, с~тем чтобы использовать в~каждом 
диапазоне свою аппроксимирующую функцию. Такое разбиение задается, разумеется, 
границами диапазонов. Пусть ${\cal I}\hm=[a_B,\,a_E]$, $a_B\hm\le a_E$,~--- 
отдельный диапазон; $a_E-a_B+1$~--- чис\-ло значений в~данном диапазоне; 
${\frak I}$~--- все множество диапазонов; $I\doteq |{\frak I}|$~--- 
общее число диапазонов (для разных состояний используются разные разбиения 
на диапазоны).

Пусть $N({\cal I}|{\cal S})$~--- число элементов состояния~${\cal S}$, 
значения которых попадают в~диапазон~${\cal I}$, ${\cal I}\hm\in{\frak I}({\cal S})$. 
Тогда $f({\cal I}|{\cal S})\hm=N({\cal I}|{\cal S})/N({\cal S})$ 
и~$f(x|{\cal I},{\cal S})\hm=N({x|\cal S})/N({\cal I}|{\cal S})$,  $x\hm\in{\cal I}$,~--- 
условные частотные распределения вероятностей диапазонов и~значений 
в~диапазоне~${\cal I}$ в~данном состоянии~${\cal S}$. Для каждого диапазона 
будем использовать свою собственную нормированную функцию 
распределения $q(x|{\cal I},{\cal S})$, которая аппроксимирует 
функцию $f(x|{\cal I},{\cal S})$. Функция $q(x|{\cal I},{\cal S})$~--- 
условное кодовое распределение значений в~данном диапазоне. Общее условное 
кодовое распределение для со\-сто\-яния~${\cal S}$   имеет вид:
\begin{equation}
\label{eq17}
q(x|{\cal S}) = f({\cal I}|{\cal S}) q(x|{\cal I},{\cal S})\,, 
\quad x\in{\cal I}\in{\frak I({\cal S})}\,.
\end{equation}
Из формул~(\ref{eq17}) и~(\ref{eq9}) следует, что избыточность 
арифметического кодирования состояния~${\cal S}$ может быть представлена в~виде:
\begin{equation}
\label{eq18}
R(\mathbf{x}|{\cal S}) =
\sum\limits_{{\cal I}\in{\frak I}({\cal S})} f({\cal I}|{\cal S}) 
R(\mathbf{x}|{\cal I},{\cal S})\,,
\end{equation}
где $R(\mathbf{x}|{\cal I},{\cal S})$~--- избыточность арифметического 
кодирования отдельного диапазона~--- имеет вид
\begin{equation}
\label{eq19}
R(\mathbf{x}|{\cal I},{\cal S}) = \sum\limits_{x\in{\cal I}} f(x|{\cal I},{\cal S})
\left[ -\log_{2}\fr{q(x|{\cal I},{\cal S})}{f(x|{\cal I},{\cal S})}\right]\,.
\end{equation}

Остановимся подробнее на способе выбора аппроксимирующей функции 
$q(x|{\cal I},{\cal S})$. Сразу отметим, что случай, когда диапазон 
состоит из единственной точки ($a_B\hm=a_E$), является тривиальным,\linebreak
 а~избыточность 
кодирования такого диапазона\linebreak равна нулю. Поэтому далее будем считать, 
что $a_B\hm<a_E$. 

Пусть ${\cal P}\hm=\{p(\xi;\boldsymbol{\gamma}):\xi\in[0,\,1], 
\boldsymbol{\gamma}\hm\in\boldsymbol{\Gamma}\}$~--- 
некоторое $\boldsymbol{\gamma}$-па\-ра\-мет\-ри\-че\-ское 
семейство положительных функций вещественного аргумента~$\xi$,  
$\boldsymbol{\Gamma}$~--- область допустимых значений параметров. 
Выбор отрезка~$[0,\,1]$  в~качестве области определения функций 
не ограничивает общности, поскольку сдвиг и~масштабирование при 
необходимости могут быть отнесены к~числу параметров~$\boldsymbol{\gamma}$. 
Предполагается, что класс функций~$\cal{P}$ известен как кодеру, так и~декодеру. 
При любых значениях параметров вели-\linebreak чины
\begin{equation}
q(x;\boldsymbol{\gamma}|{\cal I},{\cal S}) = 
c^{-1}p\left( \fr{x-a_B}{a_E-a_B}\,;\boldsymbol{\gamma}\right)\,,
\label{eq20}
\end{equation}
где
$$
c=\sum\limits_{k=0}^{a_E-a_B} p\left( \fr{k}{a_E-a_B}\,;\boldsymbol{\gamma}\right)\,,
\quad x\in{\cal I}\in{\frak I}({\cal S})\,,
$$
удовлетворяют необходимым условиям и~могут использоваться в~качестве 
условных кодовых распределений. Наличие в~формуле~(\ref{eq20}) нормировочной 
константы~$c$ обеспечивает выполнение условия~(\ref{eq2}) и~позволяет, не 
ограничивая общности, наложить на все функции семейства~${\cal P}$ одно из 
условий вида $p(0)\hm=1$ или $p(1)\hm=1$.

В процессе кодирования величины $f(x|{\cal I},{\cal S})$ могут быть вычислены. 
Это позволяет выбрать пара-\linebreak\vspace*{-12pt}

\pagebreak

\noindent
метры $\boldsymbol{\gamma}\hm=
\hat{\boldsymbol{\gamma}}\hm\in\boldsymbol{\Gamma}$ так, чтобы избыточность 
кодирования диапазона~(\ref{eq19}) была минимальной при\linebreak использовании 
функции $q(x|{\cal I},{\cal S}) \hm= q(x;\hat{\boldsymbol{\gamma}}|{\cal I},{\cal S})$ 
в~качестве условного кодового распределения. Значения 
параметров~$\hat{\boldsymbol{\gamma}}$ должны быть переданы декодеру, 
что позволит реконструировать функцию  $q(x|{\cal I},{\cal S}) \hm= 
q(x;\hat{\boldsymbol{\gamma}}|{\cal I},{\cal S})$ и~использовать 
ее в~процессе восстановления.

Для заданного разбиения на диапазоны (множества~${\frak I}({\cal S})$) 
избыточность арифметического кодирования состояния~(\ref{eq18}) при 
использовании функций $q(x|{\cal I},{\cal S})$ минимальна. Однако эта величина 
сильно зависит от выбора разбиения и~является неустойчивой по отношению к~этому выбору: 
малое изменение границ диапазонов может приводить к~заметному изменению избыточности. 
Поэтому разбиение на диапазоны целесообразно проводить на этапе кодирования, 
выбирая ${\frak I}({\cal S})$ так, чтобы по возможности уменьшить величину~(\ref{eq18}). 
При этом соответствующая информация (границы диапазонов) должна быть передана 
декодеру.

Рассмотрим преамбулу комбинаторного кода, соответствующего описанному выше методу. 
Помимо значений $N({\cal I}|{\cal S})$ преамбула должна включать\linebreak значения 
границ диапазонов выбранных раз\-би\-ений~${\frak I}({\cal S})$ (по одной границе на 
диапазон) и~значения параметров 
$\hat{\boldsymbol{\gamma}}\hm=\hat{\boldsymbol{\gamma}}({\cal I},{\cal S})$. 
Оценим длину преамбулы. Для описания одного значения $N({\cal I}|{\cal S})$\linebreak 
требуется $\sim\log_{2}N$\,бит, одной границы диапазона~---   $\sim\log_{2}A$~бит. 
Пусть~$G$~--- длина описания одного набора параметров  $\hat{\boldsymbol{\gamma}}\hm=
\hat{\boldsymbol{\gamma}}({\cal I},{\cal S})$. Общая длина преамбулы оценивается 
как произведение общего числа использованных диапазонов и~суммы трех указанных 
величин, а~для избыточности~$R_{\mathrm{T}}$, 
связанной с~передачей дополнительной информации в~преамбуле, имеем следующую оценку:
$$
R_{\mathrm{T}} \simeq \fr{1}{N} \left( 
\log_{2}N+\log_{2}A +G\right) \sum\limits_{{\cal S}\in{\frak S}}I({\cal S)}\,.
$$
Пусть, как и~ранее, $S\hm=5$, $A\hm=2^{12}$ и~$N\hm=512^{2}$. 
Предположим, что для каждого состояния используется по~10~диапазонов и~$G\hm=50$, 
т.\,е. для описания одного набора параметров требуется~50~бит. 
В~таком случае имеем $R_{\mathrm{T}}\hm\sim 0{,}015$~б/п, 
что представляется достаточно малой величиной.

Таким образом, есть все основания полагать, что\linebreak полная избыточность 
описанного метода по\-стро\-ения кодовых распределений определяется главным образом 
избыточностью арифметического кодирова\-ния. 

Применение метода для сжатия конкретного 
типа данных требует его дополнительной адап\-та\-ции, а~оценка его эффективности~--- 
вопрос, который должен решаться экспериментально.

\section{Компьютерные томограммы}

\vspace*{-9pt}

Компьютерная (рентгеновская) томограмма представляет собой квадратное 
полутоновое изоб\-ра\-же\-ние размера~$512\times512$ и~глубины яр\-кости~16~бит,\linebreak 
которое получено в~результате применения алгоритма томографического восстановления 
(реконструкции) к~данным сканирования и~содержит информацию о~рентгеновской 
плотности тканей пациен\-та в~плоскости, перпендикулярной аксиальной оси сканирующей 
системы (томографа).

Значения рентгеновской плотности принято выражать в~единицах шкалы Хаунсфилда (HU). 
Шкала состоит из целых значений в~диапазоне $[-1024, 3071]$, ширина диапазона~--- 
12~бит. Рентгеновская плотность воды при нормальных условиях принята за нуль, 
рентгеновская плотность воздуха при нормальных условиях по определению считается 
равной~$-1000$. Для некоторого материала c~линейным коэффициентом поглощения~$\mu$ 
значение рентгеновской плотности по шкале HU равно $1000(\mu\hm-\mu_0)/\mu_0$, 
где~$\mu_0$~--- линейный коэффициент поглощения воды при той же (эффективной) 
энергии, а~значения округляются до ближайшего целого. Приведем для справки 
некоторые значения рентгеновской плотности в~единицах HU: легочные ткани~---  
$\sim-850\ldots-700$, жировые ткани~--- $\sim-120\ldots-30$, мышечные ткани~--- 
$\sim+20\ldots+40$, костные ткани~--- $\sim+300\ldots+800$.

При восстановлении значений томограммы~$X^{\mathrm{T}}$ 
для записи выраженных в~единицах~HU значений рентгеновской плотности используются 
два байта (16~бит). Способ размещения~12~значащих битов шкалы Хаунсфилда в~16~битах 
яр\-кости точки томограммы варьируется в~зависимости от используемого компьютерного 
томографа. Кроме того, поскольку в~обычных режимах работы томографа область 
восстановления представляет собой круг, а~томограмма представляет собой 
квадрат, в~который этот круг вписан, то отдельное (фоновое) значение приписывается 
тем точкам изображения, в~которых реконструкция не производилась. Таким образом, 
яркость томограммы принимает не более $2^{12}+1$ значений и~существует взаимно 
однозначное амплитудное преобразование  $X^{\mathrm{T}}\hm\to X$, отоб\-ра\-жа\-ющее 
исходные значения в~диапазон $[0,\,4096]$ так, что фоновое значение отоб\-ра\-жа\-ет\-ся 
в~нуль, а~для остальных значений после преобразования справедливо равенство
 $X\hm=X^{\mathrm{HU}}\hm+1024\hm+1$, где $X^{\mathrm{HU}}$~--- 
 значения в~единицах HU. В~силу сказанного везде далее будем считать, если не 
 оговорено противное, что амплитудное преобразование выполнено, диапазон 
 значений яркости томограммы равен $[0,\,4096]$, значение нуль является 
 фоновым значением.
 
 \pagebreak
 
 \begin{figure*} %fig1
\vspace*{1pt}
\begin{center}
\mbox{%
\epsfxsize=148.291mm
\epsfbox{ste-1.eps}
}
\end{center}
\vspace*{-9pt}
\Caption{Томограммы брюшной полости Т1~(\textit{а}) и Т2~(\textit{б}),
        легких~T3~(\textit{в}) и~Т4~(\textit{г}) и~головного мозга~Т5~(\textit{д}) 
        и~T6~(\textit{е}).
        Для первой, второй и~третьей пары томограмм использованы окна визуализации,
        равные соответственно $[850,\,1250]$, $[0,\,1100]$ и~$[1020,\,1120]$}
        \label{fig1}
        \end{figure*}
        
        


Итак, подлежащие сжатию данные представляют собой квадратную 
матрицу $\mathbf{X}\hm=[X_{l,m}]$, $0\hm\le l\hm\le L\hm-1$, 
$0\hm\le m\hm\le M\hm-1$, $L\hm=M\hm=512$. Пусть $N\hm=L\times M$~--- 
общее число элементов мат\-ри\-цы. Значения элементов лежат в~диапазоне 
${\cal A}\hm=[0,\,4096]$. Для применения описанной в~разд.~2
схемы необходимо выбрать некоторый способ упорядочения элементов 
мат\-ри\-цы. Примем естественный способ упорядочения, соответствующий построчному 
сканированию слева направо и~сверху вниз. При этом данные можно рассматривать 
как последовательность отсчетов $\mathbf{X}\hm=[X_{n}]$,  $0\hm\le n\hm\le N\hm-1$, 
причем $X_{l,m}\hm=X_{lM+m}$. Везде, где это не вызывает недоразумений, 
будем использовать сокращенную запись и~обозначать через~$X$ текущий элемент 
мат\-ри\-цы~$X_n$, а~через~$U$ и~$L$~--- соседние к~нему сверху и~слева 
элементы~$X_{n-M}$ и~$X_{n-1}$. Если~$X$~--- элемент первой строки и/или первого 
столбца, т.\,е.\ верхний и/или левый соседний элемент отсутствует, то будем полагать 
$U\hm=0$ и/или $L\hm=0$. Такое соглашение отвечает специфике томографических 
изоб\-ра\-же\-ний (на\-пом\-ним, что~0~--- фоновое значение). В~соответствии с~принятым 
способом упорядочения элементы~$U$ и~$L$ являются предшествующими по 
отношению к~текущему элементу~$X$  и~могут использоваться для определения 
текущего состояния ис\-точ\-ника.
{\looseness=1

}

\begin{figure*} %fig2
\vspace*{1pt}
\begin{center}
\mbox{%
\epsfxsize=148.291mm
\epsfbox{ste-2.eps}
}
\end{center}
\vspace*{-9pt}
\Caption{Ошибки предсказания для томограмм~T3~(\textit{а}) и~T6~(\textit{б}).
        Для ошибок  предсказания томограммы~T3~(\textit{а}) использовано окно
        визуализации~$[-550,\,+550]$,  для ошибок предсказания
         томограммы~T6~(\textit{б})~---
        окно визуализации~$[-50,\,+50]$
}
\label{fig2}
\end{figure*}


В качестве экспериментального материала в~работе используются шесть 
томограмм~Т1--Т6 трех видов тканей: брюшной полости~--- Т1 и~Т2, легких~---
Т3 и~Т4 и~головного мозга~--- Т5 и~Т6, которые предоставлены Отделением лучевой 
диагностики Клиники пропедевтики внутренних болезней им.\
 В.\,Х.~Василенко (томограф HiSpeed CT/i компании General Electric). 
 В~данном случае значение $X^{\mathrm{T}}\hm=0\mathrm{x}7830$ является 
 фоновым. Остальные значения   связаны со значениями рентгеновской плот\-ности, 
 выраженной в~единицах HU, следующим образом: 
 $X^{\mathrm{T}}\hm=0\mathrm{x}8400\hm+ X^{\mathrm{HU}}$. 
 Поэтому амплитудное преобразование, о~котором шла речь выше, имеет сле\-ду\-ющий вид:
\begin{equation}
\left.
\begin{array}{rl}
\hspace*{-2mm}X^{\mathrm{T}} &\to X=0 \; \left(X^{\mathrm{T}}=0\mathrm{x}7830\right)\,; \\
\hspace*{-2mm}X^{\mathrm{T}} &\to X=X^{\mathrm{T}}-0\mathrm{x}7\mathrm{FFF}\  
\left(X^{\mathrm{T}}\neq 0\mathrm{x}7830\right)\,.
\end{array}\!
\right\}\!
\label{eq21}
\end{equation}
Префикс <<$0{\mathrm{x}}$>> использован выше для обозначения шестнадцатеричной 
записи целых чисел.

Томограммы Т1--Т6 (после амплитудного преобразова\-ния) представлены на 
рис.~\ref{fig1}. Для визу\-а\-лизации томограмм брюшной полости использовано 
окно визуализации $[850,\, 1250]$, для томограмм легких~--- 
значительно более широкое окно $[0,\,1100]$, для томограмм головного мозга~--- 
узкое окно $[1020,\,1120]$. Напомним, что визуализация\linebreak изображения 
в~окне $[x_{\min},\,x_{\max}]$ предполагает\linebreak преобразование значений яркости, 
при котором диапазон $[x_{\min},\,x_{\max}]$ линейно отображается на 
стандартный диапазон $[0,\,255]$, значения $x \hm< x_{\min}$ 
отоб\-ра\-жа\-ют\-ся в~значение~0, значения $x\hm> x_{\max}$~--- 
в~значение~255, и~вывод полученного таким образом изображения на экран монитора 
или устройства печати. Окно визуализации для томограмм обычно выбирается исходя 
из диагностических задач.

\vspace*{-6pt}


\section{Кодирование ошибок предсказания}

Адаптацию представленной в~разд.~2 общей схемы для сжатия томографических 
данных начнем с~рассмотрения метода, основанного на универсальном кодировании 
ошибок предсказания. Метод был впервые предложен в~работе~\cite{b01}.

\vspace*{-6pt}

\subsection{Ошибки предсказания}

Простейшее предсказание для текущего элемента~$X$ имеет вид $[(U\!+\!L)/2]_{-}$, 
где~$U$
и~$L$ суть ближайшие соседние сверху и~слева элементы, а~$[\cdot]_{-}$~--- 
целая часть числа, т.\,е.\ деление предполагается целочис\-лен\-ным. 
В~соответствии с~принятым способом упорядочения к~моменту рассмотрения очередного 
элемента~$X$ элементы~$U$ и~$L$ уже известны как кодеру, так и~декодеру 
и,~следовательно, известно предсказание. Поэтому описание исходных значений 
эквивалентно описанию значений ошибок предсказания $\Delta\hm=X\hm-[(U\hm+L)/2]_{-}$. 
Заметим, что в~целом ошибки предсказания образуют изображение~$\boldsymbol{\Delta}$ 
того же размера, что и~исходное изображение. Диапазон возможных значений ошибок 
предсказания вдвое шире диапазона исходных значений: ${\cal A}_{\Delta}\hm=
[-4096,\,+4096]$. Несмотря на это, распределение значений ошибок внутри 
диапазона является значительно менее равномерным: функция распределения имеет 
ярко выраженный максимум вблизи нуля. Поэтому кодирование значений ошибок предсказания 
оказывается выгоднее, чем кодирование исходных значений.

На рис.~2,\,\textit{а} представлена ошибка предсказания для томограммы~T3, 
на рис.~2,\,\textit{б}~--- 
ошибка предсказания для томограммы~T6. Окна визуализации симметричны относительно 
значения нуль и~имеют ту же ширину, что и~окна, использованные при визуализации 
соответствующих томограмм на рис.~\ref{fig1}.

\vspace*{-6pt}


\subsection{Множества состояний и~оценки минимальной скорости кодирования}

Примем гипотезу, согласно которой распределение значений очередной ошибки 
предсказания~$\Delta$ зависит только от значений элементов~$U$ и~$L$ исходного 
изображения. Функция

\noindent
\begin{equation}
\label{eq22}
\sigma_1(U,L)=|U-L|
\end{equation}
характеризует изменение значений элементов в~окрестности рассматриваемой точки. 
Примем ги-\linebreak потезу, согласно которой чем ближе значения функции~$\sigma_1$ 
для разных рассматриваемых точек, тем\linebreak меньше различие соответствующих 
распределений, измеряемое избыточностью их совместного кодирования. Данная 
гипотеза приводит к~сле\-ду\-юще\-му способу построения множества состояний~$\frak S$ 
источника. Прежде всего, учитывая специфику томограмм, определим фоновое 
состояние~${\cal S}_0$\linebreak   следующим образом: источник находится в~фоновом 
состоянии тогда и~только тогда, когда $U\hm=L\hm=0$. Далее выберем множество 
порогов ${\frak T}\hm=\{{\cal T}_1,{\cal T}_2,\dots,{\cal T}_T \}$, состоящее 
из~$T$~различных упорядочен\-ных по возрастанию натуральных чисел. Если источник 
не находится в~фоновом состоянии~${\cal S}_0$ и~выполнено условие
\begin{equation}
\label{eq23}
{\cal T}_{k-1} \leq \sigma_1(U,L) < {\cal T}_{k}\,, \qquad k=1,\dots,T\,,
\end{equation}
то источник находится в~состоянии~${\cal S}_k$. Если, наконец,
\begin{equation}
\label{eq24}
{\cal T}_{T} \leq \sigma_1(U,L)\,,
\end{equation}
то источник находится в~состоянии~${\cal S}_{T+1}$. Таким образом, 
функция~$\sigma_1$ и~значения порогов~$\frak T$ определяют множество 
состояний источника~${\frak S}({\frak T})$, которое состоит из~$T\hm+2$~состояний 
(считая фоновое).

Множеству состояний источника~${\frak S}(\frak T)$ соответствует квазиэнтропия 
$H\hm=H({\frak S}({\frak T}))\hm\equiv H({\frak T})$. 
Отметим, что в~силу выпуклости квазиэнтропии (см.\ формулу~(\ref{eq14})) 
добавление дополнительного порога может лишь уменьшать ее значение.

В подразд.~2.3 была сформулирована оптимизационная задача~(\ref{eq15}), 
которая предполагает нахож\-де\-ние оптимального множества со\-сто\-яний источника 
при фиксированном общем числе со\-сто\-яний и~вычисление соответствующей 
квазиэнтропии. При выбранном способе построения со\-сто\-яний данная задача 
превращается в~задачу нахождения оптимальных порогов и~принимает следующий вид:
\begin{equation}
\label{eq25}
\hat{H}^{T}(\boldsymbol{\Delta}) =
\min\limits_{0<{\cal T}_1<\dots<{\cal T}_T}  H(\boldsymbol{\Delta},{\frak T})\,,
\end{equation}
где общее число используемых порогов~$T$ фиксировано. Величина~$\hat{H}^{T}$~--- 
оптимальная квазиэнтропия; для обозначения множества порогов, реализующих минимум 
в~(\ref{eq25}), будем использовать обозначение $\hat{\frak T}^T\hm=
\{{\hat{\cal T}}^{T}_{1}, {\hat{\cal T}}^{T}_{2},\dots, {\hat{\cal T}}^{T}_{T}\}$.

В отличие от общей задачи~(\ref{eq15}), задача~(\ref{eq25}) 
допускает численное решение, в~результате которого можно получить 
оценки оптимальной квазиэнтропии, т.\,е.\ оценки для минимальной скорости 
кодирования используемого метода.

% Table 1
\begin{table*}\small
\begin{center}
\Caption{Оптимальные пороги и~квазиэнтропия}
\label{tab1}
\vspace{2ex}
\begin{tabular}{|c|c|rc|cc|cc|}
\hline
&&&&&&&\\[-9pt]
 T & $\hat{H}^0$ $(T=0)$ & $\hat{\frak T}^1;$ & $\hat{H}^1$ $(T=1)$ &
$\hat{\frak T}^2;$ & $\hat{H}^2$ $(T=2)$ & $\hat{\frak T}^3;$ & 
$\hat{H}^3$ $(T=3)$\\
\hline
Т1 & 4,768833 & \{37\}; & 4,564031 & \{23,101\}; & 4,507584 & \{15,37,114\}; & 
4,486921 \\
%\hline
Т2 & 5,032069 & \{41\}; & 4,835185 & \{29,113\}; & 4,790141 & \{16,37,117\}; & 4,771703\\
%\hline
Т3 & 6,451010 & \{149\}; & 6,335030 & \{101,257\}; & 6,309464 & \{87,175,468\}; & 6,294528 \\
%\hline
Т4 & 6,374494 & \{161\}; & 6,250491 & \{87,270\}; & 6,218691 & \{1,87,270\}; & 6,199943 \\
%\hline
Т5 & 4,674353 & \{19\}; & 4,381157 & \{1,21\}; & 4,276768 & \{1,14,61\}; & 4,215797 \\
%\hline
Т6 & 4,378846 & \{16\}; & 4,023612 & \{12,44\}; & 3,976094 & \{12,39,274\}; & 3,957160 \\
\hline
\end{tabular}
\end{center}
\end{table*}


Значение квазиэнтропии при фиксированных значениях порогов может быть 
вычислено по формуле~(\ref{eq12}) (в качестве значений отсчетов нужно 
использовать значения ошибок предсказания). Квази\-энтропия зависит от значений 
порогов сложным нерегулярным образом, поэтому единственным способом точного 
решения оптимизационной\linebreak задачи~(\ref{eq25}) является прямой перебор 
в~пространст-\linebreak ве 
параметров (допустимых значений порогов).\linebreak Результаты проведенных для томограмм~Т1--Т6 
чис\-лен\-ных экспериментов представлены в~табл.~\ref{tab1}. 
Вычисления были проведены для общего числа порогов~$T$, принимающего значения~0, 1, 
2 и~3, при этом общее число состояний было равно соответственно~2, 3, 4 и~5. 
Заметим, что случай $T\hm=0$ отвечает использованию двух состояний: <<фоновое>> 
и~<<нефоновое>>.


Приведенные данные показывают, что величина~$\hat{H}^T$ монотонно убывает 
с~ростом~$T$, т.\,е.\ увеличение общего числа порогов уменьшает нижнюю оценку 
скорости кодирования. Однако уже при малых значениях~$T$ наступает насыщение, 
и~дальнейшее увеличение числа порогов может дать лишь незначительный выигрыш в~скорости 
кодирования. Действительно, для любой томограммы разность $\hat{H}^2\hm-\hat{H}^3$ 
уже находится в~пределах нескольких сотых долей, а~при добавлении еще одного порога 
квазиэнтропия уменьшается не более чем на тысячные доли. Поэтому использование 
большого числа порогов не имеет смысла и~целесообразно ограничиться не более чем 
тремя порогами (использовать не более пяти состояний источника).

Как уже было указано выше, единственным способом решения экстремальной 
задачи~(\ref{eq25}) является прямой перебор. При этом вычисление квазиэнтропии 
для каждого набора порогов связано с~просмотром значений всего изображения. 
Поэтому нахождение оптимальных значений порогов представляет собой трудную 
вычислительную задачу, которая не может быть решена за приемлемое на этапе 
сжатия время уже в~двумерном пространстве параметров, т.\,е.\ для двух порогов. 
Решение же задачи~(\ref{eq25}) в~трехмерном пространстве параметров ($T\hm=3$) 
при использовании современной вычислительной техники занимает десятки часов. 
Поэтому необходим эффективный алгоритм построения множества состояний (порогов), 
реализация которого в~процессе кодирования томограммы не приводила бы к~большим 
временн$\acute{\mbox{ы}}$м затратам. Такой алгоритм был предложен в~работах~\cite{b03,b04}.

Алгоритм предполагает вместо трех оптимальных порогов 
 $\hat{\frak T}^3\hm=\{{\hat{\cal T}}^{3}_{1}, {\hat{\cal T}}^{3}_{2}, 
 {\hat{\cal T}}^{3}_{3}\}$ использовать при построении состояний три 
 квазиоптимальных порога $\tilde{\frak T}^3\hm=\{{\tilde{\cal T}}^{3}_{1}, 
 {\tilde{\cal T}}^{3}_{2}, {\tilde{\cal T}}^{3}_{3}\}$, которые находятся 
 следующим образом. Сначала находится порог~${\tilde{\cal T}}^{3}_{2}$ как 
 решение задачи~(\ref{eq25}) при общем числе порогов $T\hm=1$:  
 ${\tilde{\cal T}}^{3}_{2}\hm={\hat{\cal T}}^{1}_{1}$. Далее находятся 
 пороги ${\tilde{\cal T}}^{3}_{1}$, ${\tilde{\cal T}}^{3}_{3}$  
 как решения экстремальных задач
 \begin{equation}
\left.
\begin{array}{rl}
H^{2}\left({\tilde{\cal T}}^{3}_{1},{\tilde{\cal T}}^{3}_{2}\right) &=
\min\limits_{{\cal T}:\;{\cal T} < \tilde{\cal T}^{3}_{2}}  
H^2\left({\cal T},\tilde{\cal T}^{3}_{2}\right)\,;
\\
H^{2}\left({\tilde{\cal T}}^{3}_{2},{\tilde{\cal T}}^{3}_{3}\right) &=
\min\limits_{{\cal T}:\;\tilde{\cal T}^{3}_{2} < 
{\cal T}}  H^2\left(\tilde{\cal T}^{3}_{2},{\cal T}\right)
\end{array}
\right\}
\label{eq26}
\end{equation}
соответственно. Таким образом, нахождение трех квазиоптимальных 
порогов предполагает последовательное решение трех одномерных оптимизационных 
задач, что может быть сделано за приемлемое время в~процессе кодирования.

Пусть $\tilde{H}^3$~--- квазиэнтропия, соответствующая трем квазиоптимальным порогам. 
Поскольку до\-бав\-ле\-ние порогов может приводить только к~уменьшению квазиэнтропии, 
величина~$\tilde{H}^3$ не превышает величины~$\hat{H}^1$ (оптимальной квазиэнтропии 
при использовании одного порога):
\begin{multline*}
\tilde{H}^3 =
H^3\left({\tilde{\cal T}}^{3}_{1}, {\tilde{\cal T}}^{3}_{2}, 
{\tilde{\cal T}}^{3}_{3}\right) =
H^3\left({\tilde{\cal T}}^{3}_{1}, {\hat{\cal T}}^{1}_{1},
 {\tilde{\cal T}}^{3}_{3}\right) \leq{}\\
 {}\leq
H^1\left({\hat{\cal T}}^{1}_{1}\right) =
\hat{H}^1\,.
\end{multline*}
Более того, пусть выполнено условие ${\hat{\cal T}}^{2}_{1}\hm\leq 
{\hat{\cal T}}^{1}_{1}\hm\leq {\hat{\cal T}}^{2}_{2}$, т.\,е.\ значение 
одного оптимального порога (при $T\hm=1$) расположено между значениями двух 
оптимальных порогов (при $T\hm=2$). Тогда квазиэнтропия~$\tilde{H}^3$ не 
превышает квазиэнтропии $\hat{H}^2$ двух оптимальных порогов:
\begin{multline*}
\tilde{H}^3 =
H^3\left({\tilde{\cal T}}^{3}_{1}, {\tilde{\cal T}}^{3}_{2}, {\tilde{\cal T}}^{3}_{3}\right)\leq
H^3\left({\tilde{\cal T}}^{2}_{1}, {\hat{\cal T}}^{1}_{1}, 
{\tilde{\cal T}}^{2}_{2}\right) \leq{}\\
{}\leq
H^2\left({\hat{\cal T}}^{2}_{1},{\hat{\cal T}}^{2}_{2}\right) =
\hat{H}^2\,,
\end{multline*}
причем первое неравенство в~цепочке справедливо, поскольку 
пороги ${\tilde{\cal T}}^{3}_{1}$ и~${\tilde{\cal T}}^{3}_{3}$ суть 
решения экстремальных задач~(\ref{eq26}), а~второе~--- 
поскольку при до\-бав\-ле\-нии порогов квазиэнтропия не возрастает. Заметим, 
что указанное условие выполнено для всех томограмм~Т1--Т6, как показывают 
приведенные в~табл.~\ref{tab1} данные.

В табл.~2 приведены значения квазиоптимальных порогов и~соответствующие 
значения квазиэнтропии, посчитанные для томограмм~Т1--Т6. Там\linebreak\vspace*{-12pt} 



\begin{figure*}[b] %fig3
\vspace*{1pt}
\begin{center}
\mbox{%
\epsfxsize=161.036mm
\epsfbox{ste-3.eps}
}
\end{center}
\vspace*{-9pt}
\Caption{Частотные распределения значений ошибок предсказания томограммы~Т1 для
        фонового состояния~${\cal S}_0$~(\textit{а}) и~для состояний
        ${\cal S}_1$, ${\cal S}_2$, ${\cal S}_3$ и~${\cal S}_4$~(\textit{б}),
        соответствующих трем квазиоптимальным порогам:
        $f(0|{\cal S}_1)\hm< f(0|{\cal S}_2)\hm< f(0|{\cal S}_3)\hm< f(0|{\cal S}_4)$}
\label{fig3}
\end{figure*}


\pagebreak



{\small \begin{center}  %tabl2
 \noindent
{{\tablename~2}\ \ \small{Квазиоптимальные пороги и~квазиэнтропия}}
\vspace*{2ex}


\begin{tabular}{|c|c|c|c|c|}
\hline
&&&&\\[-9pt]
 T & ${\tilde{\frak T}}^3$ & $\tilde{H}^3$ & 
$\hat{H}^2-\tilde{H}^3$ & $\tilde{H}^3-\hat{H}^3$ \\
\hline
Т1 & \{15,37,114\} & 4,486921 & 0,020663 & 0,000000 \\
%\hline
Т2 & \{17,41,117\} & 4,771758 & 0,018383 & 0,000055 \\
%\hline
Т3 & \{70,149,419\} & 6,295014 & 0,014450 & 0,000486 \\
%\hline
Т4 & \{69,161,429\} & 6,204381 & 0,014310 & 0,004438 \\
%\hline
Т5 & \{1,19,79\} & 4,219947 & 0,056821 & 0,004150 \\
%\hline
Т6 & \{9,16,57\} & 3,959734 & 0,016360 & 0,002574 \\
\hline
\end{tabular}
\end{center}
\vspace*{12pt}
}

\addtocounter{table}{1}

\noindent
же для удобства 
приведены значения величин $\hat{H}^2\hm-\tilde{H}^3$ и~$\tilde{H}^3\hm-\hat{H}^3$.

Анализ приведенных данных показывает, что, во-пер\-вых, 
использование трех квазиоптимальных 
 порогов всегда обеспечивает некоторый выигрыш 
по сравнению с~использованием двух оптимальных. Во-вто\-рых, 
величина $\tilde{H}^3\hm-\hat{H}^3$  не превышает~0,0045~б/п, 
а~величина отношения $(\tilde{H}^3\hm-\hat{H}^3)/\hat{H}^3$ не превышает~0,001. 
Следовательно, использование квазиоптимальных порогов не приводит к~заметным 
издержкам по сравнению с~использованием трех оптимальных порогов. 

Таким образом, описанный способ нахождения квазиоптимальных порогов 
полностью решает основную поставленную задачу и~дает эффективный алгоритм 
построения множества состояний.

Построенное по квазиоптимальному множеству порогов~$\tilde{\frak T}^3$ множество 
состояний будем далее называть квазиоптимальным множеством 
состояний и~обозначать $\tilde{\frak S}^5\hm=\tilde{\frak S}(\tilde{\frak T}^3)$.

В ходе проведенных работ был исследован вопрос о~возможности уменьшения 
оценок для минимальной скорости кодирования за счет использования предсказаний 
другого типа и/или других способов построения состояний источника. Ответ оказался 
отрицательным: приведенные выше оценки не удалось улучшить сколь\-ко-ни\-будь заметно.
{\looseness=1

}

\subsection{Оценки избыточности кодирования}

Используем описанную в~подразд.~2.4 общую схему для построения кодовых 
распределений состо\-яний источника. В~качестве множества состояний будем использовать 
квазиоптимальное множество со\-сто\-яний~${\tilde{\frak S}^5}$, которое состоит из пяти 
со\-сто\-яний, построенных по трем квазиоптимальным порогам~${\tilde{\frak T}^3}$ (см.\
  табл.~2). Заметим, что, поскольку значения порогов вычисляются для
   каждой томограммы в~процессе кодирования, их придется передавать декодеру 
   в~преамбуле.

На рис.~\ref{fig3} в~качестве характерного примера представлены частотные 
распределения значений ошибок предсказания для томограммы брюшной полости~Т1. 
График на рис.~\ref{fig3},\,\textit{а} отвечает фоновому состоянию~${\cal S}_0$, 
графики на рис.~\ref{fig3},\,\textit{б}~--- 
остальным состояниям ${\cal S}_1$--${\cal S}_4$. Распределения на 
рис.~\ref{fig3},\,\textit{б} легко различаются: чем больше номер состояния, 
тем больше значение соответствующего распределения в~нуле и~тем <<шире>> 
соответствующая кривая. Масштабы по оси ординат на двух рисунках различаются 
на два порядка.



Рассмотрим частотное распределение значений ошибок предсказания в~фоновом 
состоянии~${\cal S}_0$. Определяющее состояние условие имеет вид $U\hm=L\hm=0$. 
Отсюда сразу следует, что значения ошибок предсказания~$\Delta$  в~фоновом 
состоянии не могут быть отрицательными. Далее напомним, что нуль~--- 
это уникальное значение, приписываемое точкам фона, т.\,е.\ 
тем и~только тем точкам томограммы, где вос\-ста\-нов\-ле\-ние не производится. 
Такие точки (точки фона) образуют дополнение круга восстановления до квадрата, 
в~который этот круг вписан. Поэтому состояние~${\cal S}_0$, во-пер\-вых, 
состоит практически из одних точек фона и,~во-вто\-рых, включает практически 
все точки фона. Исключения того или иного рода сводятся лишь к~небольшому 
количеству точек, расположенных вблизи границы круга восстановления. 
При этом не попадающие в~фоновое состояние~${\cal S}_0$ точки фона автоматически 
попадают в~первое состояние~${\cal S}_1$, а~значения томограммы~$X$ в~<<лишних>> 
точках, совпадающие в~данном случае со значениями ошибок предсказания~$\Delta$, 
определяются плот\-ностью воздуха и~не могут быть велики. Таким образом, значения 
ошибок предсказания неотрицательны, а~их частотное распределение имеет более 
чем выраженный максимум в~нуле и~относительно неширокий диапазон. Поскольку 
значение максимума распределения определяется геометрией, то для всех используемых 
томограмм оно одинаково и~равно $f(0|{\cal S}_0)\hm= 0{,}997279$. 
Максимальное ненулевое значение для представленного на рис.~\ref{fig3},\,\textit{а} 
распределения равно~141. Полная частотная вероятность фонового состояния также 
определяется только геометрией и~для всех томограмм равна $f({\cal S}_0)\hm= 
0{,}210285$.
{\looseness=1

}



Отмеченные выше специфические особенности фонового состояния 
позволяют предположить, что построение кодовых вероятностей может быть 
осуществлено без особого труда и~не сопряжено с~преодолением каких бы то 
ни было трудностей. Действительно, в~случае томограммы~Т1 использование 
даже простейшего равномерного условного кодового распределения 
$q(\Delta|[1,141],{\cal S}_0)\hm=1/141$ обеспечивает приемлемое значение 
избыточности арифметического кодирования $R(\boldsymbol{\Delta}|{\cal S}_0)$ 
фонового состояния (см.~(\ref{eq18}) и~(\ref{eq19})) на уровне~0,008~б/п. 
При этом вклад в~общую избыточность арифметического 
кодирования~$R(\boldsymbol{\Delta})$ (см.~(\ref{eq13})) заведомо не 
превысит~0,002~б/п.

Рассмотрим частотные распределения значений ошибок предсказания в~других состояниях. 
Диа\-пазон всех возможных значений ошибок пред\-сказания составляет ${\cal A}_{\Delta}\hm= 
[-4096,+4096]$.\linebreak В~действительности диапазоны, в~которых частот\-ные распределения 
отличны от нуля, заметно менее широкие. Для состояний~1--4 томограммы~Т1, например, 
соответствующие диапазоны равны $[-1183,+348]$, $[-464,+384]$, $[-1271,+418]$ 
и~$[-910,+746]$, а~на рис.~\ref{fig3},\,\textit{б} 
представлены лишь <<центральные>> час\-ти соответствующих распределений. 
Условные частотные вероятности~$f({\cal I}|{\cal S})$ изображенного на 
рисунке диапазона $[-150,+150]$ для состояний~1--4 составляют $0{,}999502$, 
$0{,}996579$, $0{,}963086$ и~$0{,}629420$. Это означает, что нетривиальные 
части распределений для состояний~1--3 представлены на рис.~\ref{fig3},\,\textit{б} 
практически полностью. Отметим, что в~рас\-смат\-ри\-ва\-емом примере полные 
частотные вероятности $f({\cal S})$ состояний~1--4 со\-став\-ля\-ют $0{,}536415$, 
$0{,}146095$, $0{,}062107$ и~$0{,}045097$ соответственно.

Рисунок~\ref{fig3},\,\textit{б} показывает, что частотные распределения 
представляют собой достаточно <<гладкие>> функции. Это дает основания 
полагать, что их аппроксимация в~рамках описанной 
в~подразд.~2.4 общей схемы приведет в~конечном счете к~малой избыточности 
арифметического кодирования.  Кроме того, нетрудно заметить, что частотные 
распределения в~целом имеют симметричный относительно нуля вид. 
Поэтому в~работе~\cite{b01} было предложено ограничиться использованием 
кодовых распределений, также симметричных относительно нуля. При этом общая 
схема аппроксимации несколько изменяется; рассмотрим соответствующие изменения.

Зафиксируем некоторое состояние ${\cal S}\hm\neq{\cal S}_0$, $f(\Delta|{\cal S})$~--- 
соответствующее условное частотное распределение. Пусть $q(\Delta|{\cal S})$~---  
симметричное относительно значения $\Delta\hm= 0$  кодовое распределение 
$q(\Delta|{\cal S})\hm= q(-\Delta|{\cal S})$ и,~кроме того, 
$q(0|{\cal S})\hm= f(0|{\cal S})$. В~таком случае формулу~(\ref{eq9})
 для избыточности арифметического кодирования состояния можно тождественно 
 переписать в~виде двух слагаемых:
\begin{equation}
\label{eq27}
R(\boldsymbol{\Delta}|{\cal S}) =
R_G(\boldsymbol{\Delta}|{\cal S)} + R_Q(\boldsymbol{\Delta}|{\cal S)} \,,
\end{equation}
где
\begin{equation}
\label{eq28}
R_G(\boldsymbol{\Delta}|{\cal S}) = \sum\limits_{\Delta\in{\cal A}_{\Delta}}
f(\Delta|{\cal S}) \left[-\log_{2}\fr{g(\Delta|{\cal S})}{f(\Delta|{\cal S})}\right]\,;
\end{equation}
\begin{equation}
\hspace*{-2mm}R_Q(\boldsymbol{\Delta}|{\cal S}) = 2\hspace*{-2mm}
 \sum\limits_{\Delta\in{\cal A}_{\Delta},\,\Delta>0}\hspace*{-2mm}
g(\Delta|{\cal S}) \left[-\log_{2}\fr{q(\Delta|{\cal S})}{g(\Delta|{\cal S})}\right]\,.
\!\!\label{eq29}
\end{equation}
Входящая в~формулы~(\ref{eq28}) и~(\ref{eq29}) функция $g(\Delta|{\cal S})$ равна
\begin{equation}
\label{eq30}
g(\Delta|{\cal S}) = \fr{1}{2} \left[ f(\Delta|{\cal S}) + f(-\Delta|{\cal S}) \right]
\end{equation}
и~представляет собой результат симметризации условного частотного 
распределения~$ f(\Delta|{\cal S})$.

Первое слагаемое в~формуле~(\ref{eq27}) представ\-ляет собой избыточность 
арифметического ко\-ди\-ро\-вания исходного распределения~$f$ посредством 
симметризованного распределения~$g$, не зависит от кодового распределения и~называется 
далее \textit{избыточностью симметризации}. 
Второе слагаемое~--- это избыточность кодирования симметричного распределения~$g$ 
симметричным кодовым распределением~$q$ при условии $q(0)\hm= g(0) \hm= f(0)$.

Заметим, что использование при арифметическом кодировании симметричного 
относительно значения нуль кодового распределения эквивалентна использованию 
вероятностей $2q(|\Delta|)$ для описания абсолютных значений ошибки 
предсказания~$|\Delta|$ ($|\Delta|\hm\neq 0)$ и~одного бита для описания знака ошибки. 
При этом избыточность симметризации~--- <<плата>> за использование отдельного бита 
для описания знака.

Для симметричных распределений избыточность симметризации обращается в~нуль, 
а~для распределений, близких к~симметричным, невелика. Для рассматриваемого примера 
(томограммы~Т1) избыточность симметризации состояний~1--4 
составляет~$0{,}001949$, $0{,}005553$, $0{,}025371$ и~$0{,}046791$~б/п 
соответственно, а~полная (суммарная) избыточность симметризации, равная
\begin{equation}
\label{eq31}
R_G(\boldsymbol{\Delta}) = \sum\limits_{{\cal S}\neq{\cal S}_0}
f({\cal S}) R_{G}(\boldsymbol{\Delta}|{\cal S})\,,
\end{equation}
составляет $0{,}005543$~б/п.

Приведенные данные показывают, что издержки, обусловленные использованием 
только сим\-мет\-рич\-ных кодовых распределений, весьма незначительны. Поэтому 
для всех состояний (кроме\linebreak фоново\-го) можно ограничиться построением именно 
таких распределений. Кодовые вероятности должны строиться как результат 
аппроксимации симметризованных частотных распределений~$g$, 
чтобы минимизировать величину~(\ref{eq29}). Такая задача проще исходной 
общей задачи, поскольку, во-пер\-вых, ее нужно решить только для диапазона 
значений $\Delta\hm> 0$ и, во-вто\-рых, усреднение~(\ref{eq30}) несколько 
увеличивает гладкость функции, подлежащей аппроксимации. Кроме того, 
использование симметричных кодовых распределений уменьшает количество 
параметров, которые необходимо передавать в~преамбуле, т.\,е.\ 
уменьшает избыточность передачи~$R_{\mathrm{T}}$.

Введем ряд необходимых обозначений. Обозначим через $a_{\max}^+({\cal S})$ 
максимальное значение, которое принимает модуль ошибки предсказания 
в~состоянии~${\cal S}$. В~рассматриваемом примере величины $a_{\max}^+({\cal S})$ 
для состояний~0--4 равны~141, 1183, 464, 1271 и~910. 
Очевидно, что $f(\Delta|{\cal S}_0)\hm= 0$ при $\Delta\hm< 0$ 
и~$\Delta\hm> a_{\max}^+({\cal S}_0)$ для фонового состояния,  $g(\Delta|{\cal S})\hm= 
0$  при $|\Delta|\hm> a_{\max}^+({\cal S})$ для остальных состояний. 
Поэтому задачу аппроксимации достаточно решить для множества значений 
$[0,a_{\max}^+({\cal S})]$. Выделим значение нуль в~отдельный диапазон~$[0]$, 
состоящий из одной точки. Пусть ${\frak I}^+({\cal S})$~--- 
разбиение на диапазоны оставшегося множества значений $[1,a_{\max}^+({\cal S})]$ 
(для разных состояний используются разные разбиения). Для состояний 
${\cal S}\hm\neq{\cal S}_0$ введем в~рассмотрение величины:

\noindent
\begin{align*}
g({\cal I}|{\cal S}) &= \sum\limits_{\Delta\in{\cal I}\in{\frak I}^+(\cal S)} 
\hspace*{-2mm}g(\Delta|{\cal S})\,;
\\
g(\Delta|{\cal I},{\cal S})& =\fr{g(\Delta|{\cal S})}{g({\cal I},{\cal S})}, \enskip
\Delta\in{\cal I}\in{\frak I}^+(\cal S)\,.
\end{align*}
Для симметризованных частотных распределений данные величины являются аналогами 
величин $f({\cal I}|{\cal S})$ и~$f(\Delta|{\cal I},{\cal S})$ и~связаны с~ними 
сле\-ду\-ющим образом:

\noindent
\begin{align*}
g({\cal I}|{\cal S}) &= \fr{1}{2} \left[ 
f({\cal I}|{\cal S}) + f(-{\cal I}|{\cal S}) \right]\,; \\
g(\Delta|{\cal I},{\cal S}) &=
\fr{1}{2} \left[ f(\Delta|{\cal I},{\cal S}) + f(-\Delta|{\cal I},{\cal S}) \right]\,.
\end{align*}
Как и~ранее, для искомых условных кодовых распределений вероятности значений ошибки 
предсказания в~диапазоне ${\cal I}\hm\in{\frak I}^+({\cal S})$ состояния~${\cal S}$ 
использу\-ем обозначение $q(\Delta|{\cal I},{\cal S})$. С~учетом принятых обозначений 
формулу~(\ref{eq29}) можно переписать в~следующем виде:
\begin{equation}
\label{eq32}
R_Q(\boldsymbol{\Delta}|{\cal S}) = 2 \sum\limits_{{\cal I}\in{\frak I}^+({\cal S})}
g({\cal I}|{\cal S}) R_Q(\boldsymbol{\Delta}|{\cal I},{\cal S})\,,
\end{equation}
где

\vspace*{-4pt}

\noindent
\begin{multline}
\label{eq33}
R_Q(\boldsymbol{\Delta}|{\cal I},{\cal S}) = {}\\
{}=\hspace*{-2mm}
\sum\limits_{\Delta\in{\cal I}\in{\frak I}^+({\cal S)}}\hspace*{-4mm}
g(\Delta|{\cal I},{\cal S})
\left[-\log_{2}\fr{q(\Delta|{\cal I},{\cal S})}{g(\Delta|{\cal I},{\cal S})}\right]\,.
\end{multline}
Формулы~(\ref{eq32}) и~(\ref{eq33}) являются аналогами формул~(\ref{eq18}) 
и~(\ref{eq19}) для симметричного случая.

% Table 3
\begin{table*}\small
\begin{center}
\Caption{Оптимальные кодовые распределения для состояний томограммы~Т1}
\label{tab3}
\vspace*{2ex}

\begin{tabular}{|c|c|c|c|c|c|c|}
\hline
&&&&&&\\[-11pt]
${\cal S}$ & $({\cal I},\,{\cal I}^+)\,({\cal S})$ & 
$(f,g)({\cal I},{\cal S})$ & $\alpha$ & $\nu$ & $(R,R_Q)({\cal I},{\cal S})$ &
$(R,R_Q)({\cal S})$ \\
\hline
&&&&&&\\[-9pt]
         0 & $[1,~141]$ & $0{,}002721$ & $3,{3}9\cdot10^2$ & 3,5 & 2,335142 & 0,006354 \\
\hline
&&&&&&\\[-9pt]
 & $[1,~13]$ & 0,422606 & $6{,}51\cdot10^0$ & 1,5 & 0,000082 &  \\
1           & $[14,~32]$ & 0,041218 & $2{,}39\cdot10^1$ & 0,8 & 0,002555 &{0,002186}\\
           & $[33,~1183]$ & 0,005839 & $4{,}21\cdot10^6$ & 0,5 & 0,163215 &\\
\hline
&&&&&&\\[-9pt]
 & $[1,~21]$ & 0,420897 & $5{,}41\cdot10^0$ & 1,7 & 0,000354 & \\
2           &$[22,~60]$ & 0,053332 & $2{,}87\cdot10^1$ & 0,7 & 0,005296 &{0,007429}\\
           & $[61,~464]$ & 0,008277 & $4{,}15\cdot10^2$ & 0,5 & 0,396677 &\\
\hline
&&&&&&\\[-9pt]
 & $[1,~47]$ & 0,406425 & $4{,}84\cdot10^0$ & 3,4 & 0,004503 & \\
3           & $[48,~301]$ & 0,082673 & $9{,}42\cdot10^1$ & 0,5 & 0,078880 &{0,026150}\\
           & $[302,~1271]$ & 0,004699 & $2{,}24\cdot10^6$ & 0,8 & 1,005211 &\\
\hline
&&&&&&\\[-9pt]
 & $[1,~55]$ & 0,092243 & $1{,}00\cdot10^0$ & 0\hphantom{,0}   & 0,025029 & \\
4           & $[56,~249]$ & 0,335138 & $3{,}60\cdot10^0$ & 1,4 & 0,022354 & {0,047695}\\
           & $[250,~910]$ & 0,071519 & $2{,}08\cdot10^5$ & 1,0 & 0,196409 &\\
\hline
\end{tabular}
\end{center}
\end{table*}


Конкретизируем класс функций, используемых при аппроксимации. 
Частотное распределение для фонового состояния является вырожденным (см.~выше), 
и~выбор класса диктуется не\-об\-хо\-ди\-мостью аппроксимировать симметризованные час\-тот\-ные 
распределения остальных состояний. В~об\-ласти $\Delta\hm> 0$ эти распределения 
в~целом имеют тенденцию убывать с~ростом аргумента~$\Delta$, поэтому можно 
ограничиться рассмотрением не\-воз\-рас\-та\-ющих функций. Невозрастающие, 
положительные на отрезке~$[0,1]$ линейные функции, принимающие значение~$1$ 
в~точке нуль, имеют вид:
$$
p_{\mathrm{lin}}\left(\xi;\alpha\right) = 1 - 
\fr{\alpha-1}{\alpha}\xi\,, \quad \alpha>1\,.
$$
В соответствии с~формулой~(\ref{eq20}) параметр~$\alpha$ определяет отношение 
условных кодовых вероятностей в~начале~$a_B$  и~в~кон\-це~$a_E$ рассматриваемого 
диапазона:

\noindent
$$
\alpha = \fr{q\left(a_B;\alpha|{\cal I},{\cal S}\right)}{q\left(a_E;\alpha|{\cal I},
{\cal S}\right)}\,.
$$
Убывающие экспоненциальные функции, принимающие значение~1 в~точке нуль, имеют вид:

\noindent
$$
p_{\exp}(\xi;\alpha,\nu) = \exp\left(-\ln\alpha\cdot\xi^\nu\right), 
\enskip \alpha>1,\ \nu>0\,,
$$

\pagebreak

\noindent
причем смысл параметра~$\alpha$ тот же, что и~выше. Далее будем использовать 
соглашение, в~соответствии с~которым $p(\xi;\alpha,0)\hm= p_{\mathrm{lin}}
(\xi;\alpha)$  и~$p(\xi;\alpha,\nu)\hm= p_{\exp}(\xi;\alpha,\nu)$ при условии 
$\nu\hm> 0$.



Рассмотрим класс~${\cal P}$, состоящий из функций $p(\xi ;\alpha,\nu)$. 
Определим множество возможных значений параметров. Множество значений параметра~$\nu$ 
состоит из значения~0 и~значений~0,5, 0,6, \ldots,~3,5, всего~32~возможных значения. 
Множество значений параметра~$\alpha$ включает те числа, которые в~нормализованном 
десятичном представлении имеют мантиссу, состоящую из трех цифр, и~порядок от~0 
до~7. Указанный класс оказывается достаточным для успешного решения задачи 
аппроксимации.

При заданных значениях параметров~$\nu$ и~$\alpha$  условное кодовое 
распределение для диапазона $q(\Delta|{\cal I},{\cal S}) \hm= 
q(\Delta;\nu,\alpha|{\cal I},{\cal S})$, $\Delta\hm\in{\cal I}\hm\in{\frak I}^+({\cal S})$, 
определяется по соответствующей функции класса~${\cal P}$  в~соответствии 
с~формулой~(\ref{eq20}). Общее кодовое распределение $q(\Delta|{\cal S}_0)$ 
в~случае фонового со\-сто\-яния определяется формулой~(\ref{eq17}), 
в~остальных случаях используется аналогичная формула $q(\Delta|{\cal S})\hm= 
g({\cal I}|{\cal S})q(\Delta|{\cal I},{\cal S})$.

При построении кодового распределения для фонового состояния 
достаточно единственного диапазона $[1,a_{\max}^+({\cal S}_0)]$, т.\,е.\
 разбиение ${\frak I}^+({\cal S}_0)$ тривиально, при этом параметры~$\nu$ 
 и~$\alpha$  выбираются так, чтобы минимизировать избыточность~(\ref{eq19}) 
 (или~(\ref{eq18}), что в~данном случае одно и~то же). Кодовые распределения 
 для остальных состояний построим следующим образом: зафиксируем общее 
 число диапазонов  $I^+({\cal S})$ в~разбиении интервала $[1,a_{\max}^+({\cal S})]$ 
 и~выберем границы диапазонов и~параметры~$\nu$ и~$\alpha$ так, чтобы 
 минимизировать избыточность~(\ref{eq29}). Построенные таким образом кодовые 
 распределения будем называть \textit{оптимальными}, а~соответствующую процедуру~--- 
 \textit{оптимальной}.
 {\looseness=1
 
 }

В табл.~\ref{tab3} представлены результаты по\-стро\-ения оптимальных условных 
кодовых распределений ($I^+({\cal S})\hm=3$, ${\cal S}\hm\neq {\cal S}_0$) для 
состояний томограммы~Т1. В~первом столбце приведены номера\linebreak состояний, во втором~--- 
диапазоны разбиения. В~третьем столбце для фонового состояния приведено значениe
 величины $f({\cal I},{\cal S})$, а~для остальных состояний~--- 
 значения величины $g({\cal I},{\cal S})$ для каж\-до\-го из соответствующих диапазонов. 
 В~четвертом и~пятом столбцах приведены значения па\-ра\-мет\-ров~$\alpha$ и~$\nu$, 
 которые определяют условное кодовое распределение $q(\Delta|{\cal I},{\cal S})$. 
 В~шестом столбце для единственного использованного диапазона фонового состояния 
 приведено значение избыточности $R(\boldsymbol{\Delta}|{\cal I},{\cal S}) \hm= 
 R({\cal I},{\cal S})$, а~для остальных состояний приведены значения 
 избыточности $R_Q(\boldsymbol{\Delta}|{\cal I},{\cal S}) \hm= 
 R_Q({\cal I},{\cal S})$ для каждого из использованных диапазонов. 
 В~последнем столбце приведено значение полной избыточности 
 $R(\boldsymbol{\Delta}|{\cal S}) \hm= R({\cal S})$ для фонового состояния и~значения 
 полной избыточности $R_Q(\boldsymbol{\Delta}|{\cal S}) \hm= 
 R_Q({\cal S})$ для остальных состояний.


Приведенные в~табл.~\ref{tab3} данные позволяют получить точную оценку 
полной избыточности арифметического кодирования всей томограммы~Т1 по формуле:
$$
R(\boldsymbol{\Delta}) = f({\cal S}_0) R(\boldsymbol{\Delta}|{\cal S}_0) +
\sum\limits_{{\cal S}\neq{\cal S}_0} f({\cal S})R_Q({\cal S}) + 
R_G(\boldsymbol{\Delta})\,,
$$
где третье слагаемое (полная избыточность симметризации), 
которое задается формулами~(\ref{eq28}) и~(\ref{eq31}), уже было 
вычислено ранее и~составляет~$0{,}005543$~б/п. Используя приведенные ранее 
значения $f(\cal S)$, ${\cal S}\hm\in\tilde{\frak S}^5$, находим, что искомая 
избыточность равна~$0{,}012912$~б/п.

% Table 4
\begin{table*}[b]\small
\begin{center}
\Caption{Избыточность кодирования (оптимальная процедура)}
\label{tab4}
\vspace{2ex}

\begin{tabular}{|c|c|c|c|c|c|}
\hline
&&&&&\\[-9pt]
 T & $R_G$ &  $R$ & $R_{\mathrm{T}}$ & $R+R_{\mathrm{T}}$ & 
 $(R+R_{\mathrm{T}})/\tilde{H}^3$ \\
\hline
T1 & 0,005543 & 0,012912 & \multicolumn{1}{c|}
{\raisebox{-28pt}[0pt][0pt]{0,002518}} & 0,015429 & 0,003439\\
%\cline{1-3} \cline{5-6}
T2 & 0,005890 & 0,012415 &                      & 0,014933 & 0,003129\\
%\cline{1-3} \cline{5-6}
T3 & 0,009416 & 0,019322 &                      & 0,021840 & 0,003469\\
%\cline{1-3} \cline{5-6}
T4 & 0,009942 & 0,020528 &                      & 0,023046 & 0,003714\\
%\cline{1-3} \cline{5-6}
T5 & 0,006532 & 0,012582 &                      & 0,015099 & 0,003578\\
%\cline{1-3} \cline{5-6}
T6 & 0,008045 & 0,013486 &                      & 0,016004 & 0,004042\\
\hline
\end{tabular}
\end{center}
\end{table*}


Оценим избыточность передачи~$R_{\rm T}$, т.\,е.\ избыточность, связанную 
с~необходимостью передавать значения вычисляемых на этапе кодирования параметров
 декодеру. К~числу таких параметров относятся значения трех квазиоптимальных порогов, 
 границы диапазонов разбиения (по одной верхней границе на каждый диапазон), 
 условные частотные вероятности диапазонов и~значения параметров~$\alpha$  
 и~$\nu$ в~каждом диапазоне. Функция $\sigma_1(U,L)\hm= |U-L|$, участвующая 
 в~определении состояний, принимает значения от~0 до~4096, 
 поэтому область возможных значений порогов~--- $[1,\,4096]$; следовательно, 
 для описания трех значений порогов достаточно $3\cdot 12\hm=36$~бит. 
 Аналогично для описания одной границы диапазона достаточно~12~бит. 
 Поскольку томограмма содержит~$2^{18}$~пикселей, то для описания условной 
 частотной вероятности одного диапазона заведомо достаточно~18~бит. 
 Для описания значений пары параметров~$\alpha$  и~$\nu$  требуется, как  легко видеть, $(10\hm+5)\hm+3\hm= 18$~бит. Итак, для описания одного диапазона 
 нужно~48~бит, а~общее число диапазонов равно $1\hm+4\cdot 3\hm=13$. 
 Таким образом, полная длина преамбулы составляет~660~бит, 
 а~избыточность передачи, равная отношению длины преамбулы к~общему числу пикселей, 
 составляет~0,002518~б/п.

Складывая индивидуальную избыточность арифметического кодирования томограммы~T1 
и~избыточность передачи, одинаковую для любой томограммы, получаем общую избыточность 
кодирования метода для томограммы~T1: $0{,}015429$~б/п.

В табл.~\ref{tab4} приведены оценки минимальной избыточности кодирования 
томограмм~T1--T6 с~использованием описанной выше оптимальной процеду\-ры. 
Результаты для томограмм~T2--T6 получены в~точности так же, как и~соответствующие 
результаты для томограммы~T1 ранее; детали, относящиеся к~разбиениям на диапазоны 
и~условным кодовым распределениям (аналогичные приведенным в~табл.~\ref{tab3} 
данным для томограммы~T1), опущены в~целях экономии места. 
В~первом столбце таблицы указана томограмма, в~остальных столбцах~--- 
соответствующие значения избыточности сим\-мет\-ри\-за\-ции, индивидуальной избыточности 
арифметического кодирования (с~учетом сим\-мет\-ри\-за\-ции), избыточности передачи, общей 
избыточности кодирования метода и~относительной общей избыточности кодирования 
($H\hm=\tilde{H}^3$).


Представленные в~табл.~\ref{tab4} данные свидетельствуют об 
исключительном качестве предложенного метода построения оптимальных 
кодовых вероятностей. Действительно, даже в~худшем случае (томограмма~T6) 
относительная общая избыточность едва превышает~0,4\%. 
Таким образом, метод позволяет практически достигнуть нижней границы ско\-рости 
кодирования (квазиэнтропии), что, в~свою очередь, подтверждает справедливость 
гипотез, использованных при разработке метода.

Рассмотренная оптимальная процедура построения кодовых распределений 
предполагает решение многомерной оптимизационной задачи по минимизации 
избыточности~(\ref{eq29}) в~пространстве значе\-ний параметров аппроксимации и~границ 
диапазонов при фиксированном общем числе диапазонов. Для всех состояний, за 
исключением фонового (где определение оптимальных границ не требуется), данная 
задача является вычислительно сложной и~не может быть решена за приемлемое время 
на этапе кодирования. Поэтому оптимальная процедура может быть использована 
в~алгоритмах сжатия, ориентированных на практическое применение, только для 
фонового состояния.

% Table 5
\begin{table*}[b]\small 
\vspace*{-12pt}
\begin{center}
\Caption{Кодовые распределения для состояний томограммы~Т1 (процедура уравнивания)}
\label{tab5}
\vspace{2ex}

\tabcolsep=10pt
\begin{tabular}{|c|c|c|c|c|c|c|}
\hline
${\cal S}$ & $({\cal I},\,{\frak I}^+)({\cal S})$ & 
$(f,g)({\cal I},{\cal S})$ & $\alpha$ & $\nu$ & $(R,R_Q)({\cal I},{\cal S})$ &
$(R,R_Q)({\cal S})$\\
\hline
&&&&&&\\[-9pt]
         0 & $[1,~141]$ & 0,002721 & $3{,}39\cdot10^2$ & 3,5 & 2,335142 & 0,006354 \\
\hline
&&&&&&\\[-9pt]
\multicolumn{1}{|c|}
{\raisebox{-6pt}[0pt][0pt]{1}} & $[1,~28]$ & 0,461939 & $1{,}87\cdot10^2$ & 1,3 & 0,002176 & 
\multicolumn{1}{c|}
{\raisebox{-6pt}[0pt][0pt]{0,004097}}\\
           & $[29,~1183]$ & 0,007723 & $1{,}15\cdot10^7$ & 0,5 & 0,135063 &\\
\hline
&&&&&&\\[-9pt]
\multicolumn{1}{|c|}
{\raisebox{-6pt}[0pt][0pt]{2}} & $[1,~51]$ & 0,471905 & $2{,}17\cdot10^2$ & 1,3 & 0,008063 & 
\multicolumn{1}{c|}
{\raisebox{-6pt}[0pt][0pt]{0,014564}}\\
           & $[52,~464]$ & 0,010601 & $7{,}67\cdot10^2$ & 0,6 & 0,327971 &\\
\hline
&&&&&&\\[-9pt]
 & $[1,~92]$ & 0,457159 & $7{,}13\cdot10^1$ & 1,8 & 0,019413 &\\
3           & $[93,~424]$ & 0,036454 & $2{,}66\cdot10^1$ & 0,6 & 0,247027 &0,037744\\
           & $[425,~1271]$ & 0,000184 & $8{,}89\cdot10^2$ & 0,5 & 5,390520 &\\
\hline
&&&&&&\\[-9pt]
 & $[1,~119]$ & 0,254779 & $1{,}00\cdot10^0$ & 0\hphantom{,0}   & 0,048731 & \\
4           & $[120,~393]$ & 0,240611 & $5{,}58\cdot10^1$ & 0\hphantom{,0}   & 0,051161 &0,062751\\
           & $[394,~910]$ & 0,003510 & $4{,}01\cdot10^2$ & 0,5 & 1,894615 &\\
\hline
\end{tabular}
\end{center}
\end{table*}

% Table 6
\begin{table*}\small
\begin{center}
\Caption{Избыточность кодирования (процедура уравнивания)}
\label{tab6}
\vspace{2ex}

\begin{tabular}{|c|c|c|c|c|c|}
\hline
&&&&&\\[-9pt]
T  & $R_G$ &  $R$ & $R_{\mathrm{T}}$ & $R+R_{\mathrm{T}}$ & 
$(R+R_{\mathrm{T}})/\tilde{H}^3$\\
\hline
T1 & 0,005543 & 0,016378 & 0,002151 & 0,018529 & 0,004130\\
%\hline
T2 & 0,005890 & 0,015653 & 0,002151 & 0,017804 & 0,003731\\
%\hline
T3 & 0,009416 & 0,022028 & 0,001968 & 0,023996 & 0,003812\\
%\hline
T4 & 0,009942 & 0,023203 & 0,002151 & 0,025354 & 0,004087\\
%\hline
T5 & 0,006532 & 0,015675 & 0,002334 & 0,018009 & 0,004268\\
%\hline
T6 & 0,008045 & 0,016231 & 0,002151 & 0,018382 & 0,004642\\
\hline
\end{tabular}
\end{center}
\end{table*}

Рассмотрим альтернативную процедуру построения кодовых распределений для состояний, 
отличных от фонового (${\cal S}\hm\neq {\cal S}_0$), которая, в~отличие от 
оптимальной процедуры, допускает <<быструю>> реализацию. Зафиксируем общее 
для всех со\-сто\-яний число~$I_{\max}^+$~--- 
максимально возможное число диапазонов разбиения интервалов 
$[1,\,a_{\max}^+({\cal S})]$. Зафиксируем некоторое положительное число~$\rho\hm> 0$ 
и~определим для каждого состояния величину
$$
r({\cal S}) = \fr{\rho H}{2( S-1) I_{\max}^+ f({\cal S})}\,.
$$
Напомним, что $H\hm=H(\boldsymbol{\Delta})$~--- квазиэнтропия томограммы; $f(\cal S)$~--- 
частотная вероятность состояния; $S\hm=5$~--- 
общее число состояний. Для каждого состояния разбиение интервала 
$[1,\,a_{\max}^+({\cal S})]$ на диапазоны осуществляется рекурсивно. 
В~начале очередного шага рекурсии известно начало очередного диапазона~$a_B$,
 требуется определить его конец~$a_E$ и~параметры аппроксимации. Конец 
 текущего диапазона выбирается следующим образом:
$$
a_E = \hspace*{-8pt}\max\limits_{a_B\le a\le a_{\max}^+({\cal S})}\hspace*{-5pt}
\left\{ a:\, g({\cal I}_a,{\cal S})
R_Q(\boldsymbol{\Delta}|{\cal I}_a,{\cal S}) \le r({\cal S})\right\},
$$
где ${\cal I}_a\hm= [a_B,\,a]$, а~$R_Q(\boldsymbol{\Delta}|{\cal I}_a,{\cal S})$~--- 
решение оптимизационной задачи по минимизации выражения~(\ref{eq33}). 
Одновременно с~определением искомого конца диапазона становятся 
известны и~параметры аппроксимации, поскольку решение оптимизационной 
задачи предполагает их нахождение. Рекурсия завершается в~любом из следующих случаев:
\begin{enumerate}[(1)]
\item исчерпан весь интервал $[1,\,a_{\max}^+({\cal S})]$ 
(полученное очередное  значение~$a_E$ равно $a_{\max}^+({\cal S})$);
\item исчерпано максимально допустимое общее чис\-ло диапазонов разбиения (в~этом 
случае просто полагается $a_E\hm= a_{\max}^+({\cal S})$).
\end{enumerate}

Нетрудно видеть, что правая граница очередного диапазона разбиения 
(кроме последнего) для любого состояния выбирается так, чтобы вклад 
избыточности кодирования этого диапазона в~суммарную избыточность кодирования 
был порядка $\rho H/[2( S\hm-1) I_{\max}^+]$, но не превышал указанной величины. 
В~результате все построенные диапазоны всех состояний (за исключением последних 
диапазонов каждого из состояний) вносят приблизительно равные вклады в~суммарную 
избыточность кодирования. Поэтому далее будем называть рассматриваемую процедуру 
\textit{процедурой уравнивания}. 
Кроме того, ясно, что если завершение рекурсии для всех состояний было связано 
с~выполнением первого из указанных выше условий, то суммарная избыточность 
кодирования не превышает значения~$\rho H$. Таким образом, используемая в~процедуре 
уравнивания величина~$\rho$ играет роль общего относительного <<целевого>> 
уровня кодовой избыточ\-ности.

Основное достоинство процедуры уравнивания заключается в~том, что она 
допускает быструю реализацию и, следовательно, может использоваться на 
практике в~процессе кодирования.

Анализ приведенных выше данных, относящихся к~избыточности кодирования с~использованием 
оптимальной процедуры, показывает, что разумным, например, является выбор значения\linebreak 
$\rho\hm=0{,}004$ и~использование $I_{\max}^+\hm=4$, т.\,е.\ не более 
четырех диапазонов разбиения для каждого состояния. Указанные значения используются 
всюду далее. 

В~табл.~\ref{tab5} представлены результаты построения условных 
кодовых распределений для состояний томограммы~T1 с~по\-мощью процедуры уравнивания. 
Таблица имеет ту же структуру, что и~табл.~\ref{tab3}.



Избыточность кодирования $R_Q({\cal S})$ отдельных состояний, как и~следовало 
ожидать, оказалась несколько больше, чем при использовании оптимальных 
кодовых распределений. Суммарный вклад величин $R_Q({\cal S})$ в~общую избыточность 
кодирования метода составляет менее $0{,}01$~б/п, т.\,е.\ 
менее 0{,}23\% от величины квазиэнтропии томограммы, что заметно меньше 
использованного в~процедуре уравнивания значения целевого уровня кодовой избыточности. 
Полная избыточность арифметического кодирования всей томограммы~Т1, 
включающая вклад фонового состояния и~избыточность симметризации, 
равна~0,016378~б/п. Избыточность передачи~$R_{\mathrm{T}}$ 
при использовании процедуры уравнивания зависит от общего числа реально использованных 
диапазонов и~в~данном случае составляет~0,002151~б/п, что несколько меньше 
избыточности передачи при оптимальной процедуре. Наконец, общая избыточность 
кодирования метода, использующего процедуру уравнивания при построении 
кодовых вероятностей, для томограммы~T1 со\-став\-ля\-ет~0,018529~б/п.

В табл.~\ref{tab6} приведены оценки избыточности кодирования всех томограмм~T1--T6 
с~использованием описанной выше процедуры уравнивания при построении кодовых 
вероятностей. Таблица имеет ту же структуру, что и~табл.~\ref{tab4}. Общее число 
используемых диапазонов не является фиксированным, равно

\noindent
$$
 I(\frak S) =1 +  \sum\limits_{{\cal S}\in{\frak S},\, {\cal S}\neq{\cal S}_0} 
 I^+({\cal S})\,,
$$
где $I^+({\cal S})$~--- число диапазонов в~построенном разбиении 
${\frak I}^+({\cal S})$, и~может варьироваться от томограммы к~томограмме. 
Поэтому и~значения избыточности передачи~$R_{\mathrm{T}}$ в~табл.~\ref{tab6} несколько 
различаются.



Представленные в~табл.~\ref{tab6} данные свидетельствуют об отличном 
качестве предложенного метода построения кодовых вероятностей на основе 
процедуры уравнивания. В~худшем случае (томограмма~T6) относительная общая 
избыточность не превышает~0,5\%. В~действительности результаты применения 
процедуры уравнивания лишь немного уступают аналогичным результатам, 
полученным с~использованием оптимальной процедуры. При этом время 
вычислений оказывается несопоставимо меньше.

\vspace*{-6pt}


\section{Кодирование компонент дискретного вейвлет-преобразования}

Адаптируем теперь представленную в~разд.~2 общую схему так, чтобы получить 
метод сжатия, основанный на универсальном кодировании значений компонент 
дискретного вейв\-лет-пре\-об\-ра\-зо\-ва\-ния (ДВП) томограмм. Метод был впервые 
предложен в~работе~\cite{b05}.

\vspace*{-6pt}

\subsection{Дискретное вейвлет-преобразование}

Напомним, прежде всего, что представляет собой ДВП последовательности. 
Пусть $\mathbf{x}\hm=\{x_n\}$~--- суммируемая с~квадратом вещественная 
последовательность целого аргумента. Прямое ДВП \mbox{такой} последовательности~--- 
это разложение данной последовательности на две составляющие (компоненты), 
которое осуществляется следующим образом.\linebreak Сначала вычисляются свертки 
последователь\-ности~$\mathbf {x}$  с~двумя заданными фильтрами разложения 
(анализа), низкочастотным фильтром~${\boldsymbol\mu}^0$ и~высокочастотным 
фильтром~${\boldsymbol\mu}^1$  (оба фильтра суть сум\-мируемые с~квадратом 
вещественные последовательности). Затем две полученные в~результате 
фильт\-ра\-ции последовательности прореживаются, т.\,е.\ в~них удерживаются лишь 
четные члены. В~итоге имеем суммируемые с~квадратом вещественные последовательности 
целого аргумента~$\mathbf{x}^{0}$ и~$\mathbf{x}^{1}$:

\vspace*{2pt}

\noindent
\begin{equation}
\label{eq34}
x_n^i=\sum\limits_{k=-\infty}^{+\infty} \mu_{2n-k}^{i}\,x_k\,, \quad i=0,\,1\,,
\end{equation}

\vspace*{-2pt}

\noindent
которые представляют собой приближение (низкочастотную составляющую) и~детальную 
(высокочастотную) составляющую исходной последовательности, при этом каждая 
имеет вдвое меньшее разрешение.

Обратное ДВП --- это восстановление исход-\linebreak ной последовательности по ее 
приближению\linebreak и~детальной составляющей. Разбавим обе 
по\-сле\-до\-ва\-тель\-ности~$\mathbf{x}^{0}$ и~$\mathbf{x}^{1}$ нулями (т.\,е.\ 
построим последовательности с~нулевыми нечетными членами и~четными членами, 
заданными последовательностями~$\mathbf{x}^{0}$ и~$\mathbf{x}^{1}$), 
затем вычислим свертки полученных последовательностей с~некоторыми фильтрами 
синтеза~$\boldsymbol\nu^0$ и~$\boldsymbol\nu^1$ и~сложим результаты:

\vspace*{2pt}

\noindent
\begin{equation}
\label{eq35}
\tilde{x}_n =\sum\limits_{k=-\infty}^{+\infty}
\left(\nu_{n-2k}^{0}\,x_k^0 + \nu_{n-2k}^{1}\,x_k^1 \right) \,.
\end{equation}

\vspace*{-2pt}

\noindent
Чтобы преобразование~(\ref{eq35}) действительно было обратным по отношению 
к~преобразованию~(\ref{eq34}),
т.\,е.\ чтобы выполнялось равенство $\mathbf{x}\hm=\tilde{\mathbf{x}}$,
система фильтров~$\boldsymbol\mu^0$, $\boldsymbol\mu^1$, 
$\boldsymbol\nu^0$ и~$\boldsymbol\nu^1$ должна удовлетворять условию 
восстановления (см.,~например,~\cite{b06}). Этому условию, в~част\-ности, 
удовлетворяет система фильтров конечной длины, впервые предложенная в~\cite{b07}:

\pagebreak

\end{multicols}

\noindent
{ %\tiny %scriptsize %\footnotesize
\begin{equation}
\left.
{\begin{array}{llllll}
\mu_{-2}^0=-\fr{1}{8},&\ \mu_{-1}^0=\fr{1}{4},&\ \mu_{0}^0=\fr{3}{4},&\ \mu_{1}^0=\fr{1}{4},
&\ \mu_{2}^0=-\fr{1}{8}; &\ \\[12pt]
\mu_{-2}^1=-\fr{1}{2},&\ \mu_{-1}^1=1,&\ \mu_{0}^1=-\fr{1}{2}; &\ &\ &\ \\[12pt]
&\ \nu_{-1}^0=\fr{1}{2},&\ \nu_{0}^0=1,&\ \nu_{1}^0=\fr{1}{2}; &\ &\ \\[12pt]
&\ \nu_{-1}^1=-\fr{1}{8},&\ \nu_{0}^1=-\fr{1}{4},&\ \nu_{1}^1=\fr{3}{4},&\ \nu_{2}^1=-\fr{1}{4},&\
\nu_{3}^1=-\fr{1}{8};
\end{array}}
\right\}\!\!\!
\label{eq36}
\end{equation}

}

\begin{multicols}{2}

\noindent
остальные коэффициенты фильтров равны нулю. Эта система, получившая широкое 
распространение и~вошедшая в~стандарт сжатия JPEG 2000, используется в~настоящей 
работе.

Эффективность применения ДВП для сжатия данных обусловлена тем, что 
при удачном выборе фильтров разложения значения детальных со\-став\-ля\-ющих 
распределены значительно более неравномерно, чем значения исходного сигнала. 
Однако требование обратимости (отсутствия искажений) при сжатии сигналов 
конечной длины с~целыми значениями приводит к~серьезному дополнительному 
требованию: ДВП должно быть свободно от ошибок округления. Если 
ДВП с~фильтрами~(\ref{eq36}) реализуется непосредственно по 
формулам~(\ref{eq34}) и~(\ref{eq35}),\linebreak то для того чтобы избежать 
ошибок округления, необходи\-мо на этапе вычисления прямого преобразования 
добавить дополнительные биты для хранения остатков от деления: три бита 
для приближения и~один бит для детальной составляющей. По\-па\-да\-ющие в~эти 
дополнительные <<младшие>>\linebreak
 биты остатки представляют собой практически 
<<белый шум>>, что неприемлемо с~точки зрения последующего сжатия. 
Решение проб\-ле\-мы заключается в~том, чтобы реализовывать ДВП посредством 
так называемой лиф\-тинг-схе\-мы~\cite{b08}. При этом появляется возможность 
взаимосогласованным образом округлять результаты при вычислении прямого и~обратного 
преобразований, что, во-пер\-вых, обеспечивает точное восстановление 
и,~во-вто\-рых, избавляет от необходимости хранить (и~сжимать) остатки. 
Кроме того, аккуратный учет краевых эффектов в~лиф\-тинг-схе\-ме позволяет 
добиться того, что для конечной последовательности~$\mathbf{ x}$ длины~$N$ 
приближение~$\mathbf{ x}^0$ и~детальная составляющая~$\mathbf{x}^1$ имеют 
длины, в~точ\-ности равные $[N/2+1]_-$ и~$[N/2]_-$ соответственно.

В двумерном случае ДВП представляет собой суперпозицию одномерных преобразований, 
применяемых раздельно к~строкам и~столбцам. Если $\mathbf{X}\hm=\{X_{l,m}\}$~--- 
двумерная вещественная последовательность, то прямое ДВП имеет 
аналогичный~(\ref{eq34}) вид:

\noindent
$$
X_{l,m}^{i,j}=\sum\limits_{k,\,k'=-\infty}^{+\infty}
\mu_{2l-k}^{i}\,\mu_{2m-k'}^{j}\,X_{k,k'}\,,
\enskip i,j=0,\,1\,,
$$
а его результатом является разложение исходной последовательности 
на четыре компоненты (со\-став\-ля\-ющие) вдвое меньшего разрешения.

Последовательность $\mathbf{X}^{0,0}\doteq\mathbf{X}^{\mathrm{A}}$, 
полученная с~применением низкочастотной фильтрации по строкам и~столбцам, 
представляет собой приближение, а~остальные три последовательности
$\mathbf{X}^{0,1}\doteq\mathbf{X}^{\mathrm{V}}$, 
$\mathbf{X}^{1,0}\doteq\mathbf{X}^{\mathrm{H}}$ 
и~$\mathbf{X}^{1,1}\doteq\mathbf{X}^{\mathrm{D}}$, в~построении\linebreak которых 
участвует высокочастотный фильтр,~--- детальные со\-став\-ля\-ющие (вертикальные, 
горизонтальные и~диагональные соответственно). Такое разложение называется 
\textit{одномасштабным} ДВП. Обратное преобразование производится очевидным 
образом при помощи одномерных обратных ДВП, применяемых раздельно к~строкам и~столбцам, 
и~описывается двумерным аналогом формулы~(\ref{eq35}). Как и~в~одномерном случае, 
возможна лиф\-тинг-ре\-а\-ли\-за\-ция двумерного ДВП. Это дает возможность 
использовать целую арифметику и~добиться того, что для изображения 
(конечной двумерной последовательности, т.\,е.\ мат\-ри\-цы) размера 
$L\times M$ приближение и~вертикальные, горизонтальные и~диагональные составляющие 
имеют следующие размеры: $[L/2+1]_-\times[M/2+1]_-$, 
$[L/2+1]_-\times[M/2]_-$, $[L/2]_-\times[M/2+1]_-$ и~$[L/2]_-\times[M/2]_-$.

\begin{figure*} %fig4
\vspace*{1pt}
\begin{center}
\mbox{%
\epsfxsize=148.291mm
\epsfbox{ste-4.eps}
}
\end{center}
\vspace*{-9pt}
\Caption{Одномасштабное разложение томограмм~T3~(\textit{а}) и~T6~(\textit{б})}
\label{fig4}
\vspace*{4pt}
\end{figure*}


Результатом применения прямого ДВП к~томограмме~$\mathbf{X}$ размера $512\times512$ 
является ее одномасштабное разложение на четыре компоненты (изоб\-ра\-же\-ния) 
размерами $256\times256$: приближение~$\mathbf {X}^{\mathrm{A}}$ 
и~вертикальные~$\mathbf{X}^{\mathrm{V}}$, горизонтальные~$\mathbf{X}^{\mathrm{H}}$ 
и~диагональные~$\mathbf{X}^{\mathrm{D}}$ детальные составляющие. Первая ком-\linebreak понента~--- 
результат низкочастотной фильтрации (сглаживания) элементов исходного изображения 
по строкам и~столбцам, следующие две получены сглаживанием по одной координате 
и~вы\-со\-ко\-час\-тот\-ной фильтрацией по другой, а~последняя~--- 
результат высокочастотной фильтрации по обеим координатам (в~каждом случае 
фильт\-ра\-ция сопровождается прореживанием, уменьшающим размер вдвое 
по каждой координате). Как уже было указано выше, в~данной работе используется 
ДВП, основанное на системе фильтров~(\ref{eq36}), поэтому диапазоны значений 
яркости компонент равны соответственно  $[-2560,\,6656]$, 
$[-6144,\,6144]$, $[-6144,\,6144]$ и~$[-8192,\,8192]$. На рис.~\ref{fig4} 
представлены одномасштабные разложения томограмм~T3 и~T6. 
В~левом верхнем квадрате каждого рисунка располагается приближение для 
соответствующей томограммы (использованы те же окна визуализации, что и~на 
рис.~\ref{fig1}). В~правом верхнем, левом нижнем и~правом нижнем квадратах 
располагаются соответственно вертикальные, горизонтальные и~диагональные составляющие. 
При этом использованы существенно более узкие, чем для приближений, окна 
визуализации с~центром в~нуле: окно $[-150,\,+150]$ для детальных составляющих~T3 
и~окно $[-20,\,+20]$ для детальных составляющих~T6.

\vspace*{-4pt}


\subsection{Особенности метода кодирования}


В разд.~4 метод универсального кодирования ошибок предсказания для 
томограмм был рас\-смот\-рен во всех деталях. После внесения некоторых изменений 
непринципиального характера по существу тот же метод может быть с~успехом 
применен для кодирования компонент ДВП томограмм. Данный раздел посвящен 
рассмотрению тех особенностей, которые отличают метод сжатия (кодирования) 
компонент ДВП от аналогичного метода, используемого при сжатии ошибок предсказания.

Компоненты томограммы, полученные в~результате ДВП, сжимаются по отдельности 
(независимо друг от друга). Поскольку все компоненты имеют одинаковые размеры, 
значения квазиэнтропии и~избыточности для всей томограммы равны средним арифметическим 
значениям соответствующих величин для всех компонент ДВП. Как 
показывает рис.~\ref{fig4}, статистические свойства приближения и~детальных 
составляющих значительно различаются, поэтому и~методы их сжатия, вообще говоря, 
должны различаться.

Начнем с~рассмотрения метода сжатия компоненты~$\mathbf{X}^{\mathrm{A}}$ (приближения). 
Как показывает сравнение рис.~\ref{fig1} и~\ref{fig4}, приближение визуально 
почти не отличается от исходного изоб\-ра\-же\-ния. Стати\-стические свойства приближения 
и~исходного изоб\-ра\-же\-ния также близки, несмотря на расширение диапазона значений 
в~2,25~раза и~появление отрицательных значений. Поэтому для сжатия 
приближения можно прямо использовать описанный в~разд.~4 
метод и~кодировать значения~$\boldsymbol{\Delta}\mathbf{X}^{\mathrm{A}}$, т.\,е.\
значения ошибок предсказания приближения. В~час\-ти построения множества состояний 
метод вообще не претерпевает изменений. В~час\-ти по\-стро\-ения кодовых вероятностей в~метод 
требуется внести единственное изменение, которое затрагивает только процедуру 
по\-стро\-ения кодовых вероятностей для фонового со\-сто\-яния и~подробно рассмотрено далее.
{ %\looseness=-1

}

Рассмотрим теперь детальные составляющие. Отсчеты детальных составляющих 
принимают значения в~симметричных относительно нуля диапазонах, которые 
для компонент  $\mathbf{X}^{\mathrm{V}}$ и~$\mathbf{X}^{\mathrm{H}}$ в~3~раза, 
а~для компоненты $\mathbf{X}^{\mathrm{D}}$ в~4~раза шире, чем 
диапазон исходных значений. Однако функции распределения значений концентрируются 
в~небольших окрестностях нуля: на рис.~\ref{fig4} детальные составляющие менее 
контрастны, чем приближение, несмотря на то что для их визуализации использовано 
значительно более узкое окно. Поэтому следует кодировать непосредственно значения 
этих составляющих.

Детальные составляющие описывают отличие элементов исходного изображения 
от сглаженных значений. Как и~раньше, элементы, расположенные слева и~сверху 
от рассматриваемого, будем обозначать через~$U$ и~$L$. Однако теперь значения 
самих этих элементов характеризуют скорость изменения значений исходного 
изображения по крайней мере по одной координате. Поэтому при построении 
состояний в~формулах~(\ref{eq23}) и~(\ref{eq24}) вместо функции~$\sigma_1$ будем 
использовать функцию:

\vspace*{3pt}

\noindent
\begin{equation}
\label{eq37}
\sigma_2(U,L) = |U|+|L|\,.
\end{equation}
Такой выбор предпочтительнее, чем $|U\hm+L|$, поскольку~$U$ и~$L$ 
могут иметь разные знаки. Действительно, если  $U\hm\sim -L \hm\gg 0$, то скорость 
изменения значений элементов велика, но при этом $|U\hm+L|\hm\sim 0$. Замена функции, 
используемой при по\-стро\-ении состояний,~--- 
это главное изменение, которое необходимо внести в~метод при сжатии детальных 
составляющих.

В ходе выполнения работ был подробно исследован вопрос о~целесообразности 
использования при построении множества со\-сто\-яний функции~$\sigma_2$ другого вида. 
Действительно, вертикальные детальные со\-став\-ля\-ющие~$\mathbf{X}^{\mathrm{V}}$ 
получены низкочастотной фильтрацией по столбцам и~высокочастотной по строкам, 
а~горизонтальные детальные со\-став\-ля\-ющие~$\mathbf{X}^{\mathrm{H}}$~--- 
низкочастотной фильтрацией по строкам и~высокочастотной по столбцам. Эта асим\-мет\-рия 
строк и~столбцов никак не учтена в~способе построения состояний при использовании 
функции~(\ref{eq37}): состояния инвариантны относительно транспонирования. 
Поэтому априори представляется обоснованным рассмотреть обобщение~(\ref{eq37}) \mbox{вида}

\vspace*{-3pt}

\noindent
\begin{multline}
\label{eq38}
\sigma_2(U,L;\alpha) = (1+\alpha)|U| + (1-\alpha)|L|\,,\\
-1\leq\alpha\leq +1\,,
\end{multline}

\vspace*{-1pt}

\noindent
и использовать при построении состояний для детальных составляющих конкретного 
типа функцию~$\sigma_2$  вида~(\ref{eq38}) с~конкретным (отличным от нуля) 
значением~$\alpha$.

Для проверки гипотезы был проведен эксперимент, в~ходе которого 
значения оптимальной квазиэнтропии детальных составляющих, полученные с~использованием
 функций вида~(\ref{eq38}) и~значениями~$\alpha$, равными~$\pm1/3$, 
 сравнивались с~полученными с~использованием функции~(\ref{eq37}) аналогичными 
 значениями. Вычисления проводились для детальных составляющих всех~6~томограмм 
 при общем числе выбираемых порогов  $T\hm=1,2$ и~3, всего~108~вариантов. 
 Полученные результаты опровергают выска-\linebreak занное предположение. 
 В~подавляющем большинстве случаев (92~из~108) 
 наименьшее значение\linebreak\vspace*{-12pt}
 
 \columnbreak
 
 \noindent
 квазиэнтропии достигалось при использовании функции~(\ref{eq37}). 
 В~остальных~16~случаях выигрыш, обусловленный использованием функции вида~(\ref{eq38}) 
 со значением $\alpha\hm\neq 0$, оказался ничтожным и~не превысил~0,008~б/п 
 ни в~одном из случаев. При этом уменьшение квазиэнтропии для томограммы в~целом 
 как результат использования для его детальных составляющих оптимальных функций 
 вида~(\ref{eq38}) оказалось еще меньше: оно не превысило~0,002~б/п. 
 Поэтому использование функций вида~(\ref{eq38}) в~процессе построения 
 со\-сто\-яний для детальных со\-став\-ля\-ющих было признано нецелесообразным.

Рассмотрим теперь уже упомянутое выше изменение процедуры построения кодовых
 вероятностей, используемое как при сжатии приближений, так и~при сжатии 
 детальных составляющих, но затрагивающее лишь фоновое состояние. 
 Необходимость данного изменения обусловлена тем, что отсчеты любой из компонент 
 ДВП могут иметь отрицательные значения. Это не сказывается на построении кодовых 
 вероятностей для нефоновых состояний, поскольку в~этом случае предусмотрено 
 использование процедуры симметризации. Для фонового состояния ситуация отличается. 
 С~одной стороны, для любой компоненты частотное распределение значений в~фоновом 
 со\-сто\-янии по-преж\-не\-му несимметрично и~использование сим\-мет\-ри\-за\-ции привело бы 
 к~неоправданным издержкам. С~другой стороны, вероятности отрицательных значений 
 в~фоновом состоянии, вообще говоря, не равны нулю и~для таких значений должны быть 
 построены ненулевые кодовые вероятности. Поэтому при построении аппроксимации для 
 фонового состояния нельзя ограничиться диапазоном ${\cal I}_0^+\hm=
 [1,\,a_{\max}^+({\cal S}_0)]$, а~необходимо добавить диапазон ${\cal I}_0^-\hm=
 [a_{\min}^-({\cal S}_0),\,-1]$, где $a_{\min}^-({\cal S}_0)$~--- 
 минимальное\linebreak значение, которое принимают отсчеты рассмат\-риваемой компоненты 
 в~фоновом состоянии,\linebreak и~построить соответствующие условные кодовые вероятности. Вклад 
 избыточности кодирования нового диапазона $R(\mathbf{X}|{\cal I}_0^-,{\cal S}_0)$ 
 в~общую избыточность кодирования составляет 
 $f({\cal I}_0^-|{\cal S}_0)R(\mathbf{X}|{\cal I}_0^-,{\cal S}_0)$, где $\mathbf{X}$~--- 
 любая из компонент ДВП.

Задача аппроксимации для отрицательного диапазона~${\cal I}^-$ сводится к~уже 
рассмотренной ранее задаче для положительного диапазона ${\cal I}^+\hm=
-{\cal I}^-$. Пусть $f(x)$, $x\hm\in{\cal I}^-$,~--- 
заданное в~отрицательном диапазоне условное частотное распределение. 
Функция~$\varphi(x)\hm= f(-x)$, $x\hm\in{\cal I}^+$,~--- 
условное распределение в~положительном диапазоне, поэтому рассмотренная 
в~подразд.~4.3 процедура позволяет построить для него 
аппроксимацию $q_\varphi(x)$, $x\hm\in{\cal I}^+$. В~качестве искомого условного 
кодового распределения будем использовать функцию $q(x)\hm= 
q_\varphi(-x)$, $x\hm\in{\cal I}^-$.

% Table 7
\begin{table*}[b]\small
\vspace*{-6pt}
\begin{center}
\Caption{Оптимальные/квазиоптимальные пороги и~квазиэнтропия компонент ДВП}
\label{tab7}
\vspace{2ex}

\begin{tabular}{|c|cc|cc|cc|}
\hline
&&&&&&\\[-9pt]
 X & $\hat{\frak T}^2;$ & $\hat{H}^2$ $(T=2)$ & $\hat{\frak T}^3;$ &
  $\hat{H}^3$ $(T=3)$ & 
$\tilde{\frak T}^3;$ & $\tilde{H}^3$ $(T=3)$ \\
%Data
\hline
A1 & \{23,182\}; &  4,880032 & ~\{20,73,271\}; &  4,847027 & ~\{14,29,182\}; &  4,853077 \\
%\hline
V1 & \{12,34\}; &  4,010317 & ~~~\{9,21,42\}; &  3,997519 & ~~\{12,28,87\}; &  3,998197\\
%\hline
H1 & \{20,81\}; &  4,523917 & \{11,30,115\}; &  4,489731 & \{12,41,116\}; &  4,490816\\
%\hline
D1 & \{13,27\}; &  4,018969 & \{13,26,52\}; &  4,008997 & \{11,19,49\}; &  4,010081\\
%\hline
A2 & \{28,175\}; &  5,124079 &\{23,85,312\}; &  5,087575 &\{13,35,175\}; &  5,096258\\
%\hline
V2 & \{17,42\}; &  4,242733 & \{13,30,69\}; &  4,227881 & \{13,26,69\}; &  4,228058\\
%\hline
H2 & \{17,63\}; &  4,725712 & \{17,51,150\}; &  4,696555 &\{17,51,150\}; &  4,696555\\
%\hline
D2 & \{15,36\}; &  4,292950 & \{11,21,43\}; &  4,281943 & \{11,21,43\}; &  4,281943\\
%\hline
A3 & \{56,241\}; &  6,491513 & \{51,193,584\}; &  6,449757 & \{42,115,428\}; &  6,454966\\
%\hline
V3 & \{68,191\}; &  6,064261 & \{58,148,335\}; &  6,045233 & \{51,104,205\}; &  6,046978\\
%\hline
H3 & \{78,229\}; &  6,223272 & \{61,155,318\}; &  6,199585 & \{61,155,318\}; &  6,199585\\
%\hline
D3 & \{64,145\}; &  6,119812 & \{61,117,233\}; &  6,105214 & ~\{61,99,217\}; &  6,106998\\
%\hline
A4 & \{62,276\}; &  6,448976 & \{53,166,577\}; &  6,406838 & \{48,150,562\}; &  6,408289\\
%\hline
V4 & \{68,203\}; &  5,985891 & \{48,107,283\}; &  5,967160 & \{48,107,283\}; &  5,967160\\
%\hline
H4 & \{78,230\}; &  6,130190 & \{57,140,347\}; &  6,103864 & \{57,140,347\}; &  6,103864\\
%\hline
D4 & \{69,147\}; &  6,024354 & \{63,133,321\}; &  6,011689 & \{42,83,175\}; &  6,012719\\
%\hline
A5 & \{12,92\}; &  4,942235 & \{1,16,100\}; &  4,856658 & \{1,22,116\}; &  4,863834\\
%\hline
V5 & \{13,39\}; &  3,891974 & \{10,24,87\}; &  3,866175 & \{10,21,71\}; &  3,867450\\
%\hline
H5 & \{11,32\}; &  3,636817 & \{6,17,60\}; &  3,614506 & \{6,17,60\}; &  3,614506\\
%\hline
D5 & \{8,17\}; &  3,092293 & \{8,15,35\}; &  3,082019 & \{6,12,35\}; &  3,082257\\
%\hline
A6 & \{16,87\}; &  4,661787 &\{15,47,163\}; &  4,627884 & \{12,22,89\}; &  4,640066\\
%\hline
V6 & \{11,39\}; &  3,686671 &\{9,20,79\}; &  3,664356 & \{8,15,63\}; &  3,666472\\
%\hline
H6 & \{10,26\}; &  3,474091 &\{7,13,41\}; &  3,458590 & \{7,12,41\}; &  3,458997\\
%\hline
D6 & \{5,10\}; &  2,865786 & \{5,8,20\}; &  2,855021 & \{5,8,20\}; &  2,855021\\
\hline
\end{tabular}
\end{center}
\end{table*}

\subsection{Оценки минимальной скорости и~избыточности кодирования}


Оценки минимальной скорости кодирования (квазиэнтропии) компонент 
ДВП томограмм T1--T6, отвечающие множествам состояний, построенных 
с~использованием двух и~трех оптимальных и~трех квазиоптимальных порогов, 
приведены в~табл.~\ref{tab7}. Там же приведены и~соответствующие значения 
порогов. Первый столбец таблицы определяет рассматриваемое изображение 
(указаны тип компоненты ДВП и~номер томограммы). Во втором и~третьем столбцах 
приведены значения двух и~трех оптимальных порогов и~соответствующие значения 
квазиэнтропии. Последний столбец содержит те же данные для квазиоптимальных 
порогов. В~соответствии со сказанным в~подразд.~5.2, в~случае 
приближений (компоненты A1--A6) множества состояний строились для ошибок 
предсказания с~использованием функции~(\ref{eq22}); оценки квазиэнтропии 
получены также для ошибок предсказания. В~случае детальных составляющих 
(остальные компоненты) множества состояний строились для значений компонент 
с~использованием функции~(\ref{eq37}); оценки квазиэнтропии получены также 
для значений компонент.

Анализ приведенных данных кроме всего прочего еще раз демонстрирует 
эффективность использования трех квазиоптимальных порогов.
%
Поскольку минимальная скорость кодирования (квазиэнтропия) томограммы 
в~целом равна среднему арифметическому значений квазиэнтропии всех компонент ДВП, 
она может легко быть вычислена на основе данных табл.~\ref{tab7}. Соответст\-ву\-ющие 
результаты для томограмм~T1--T6 приведены в~табл.~8.

Сравнение приведенных результатов с~аналогичными результатами 
из подразд.~4.2 показывает, что квазиэнтропия кодирования компонент 
ДВП томограммы всегда оказывается меньше соответствующей квазиэнтропии 
кодирования ошибок предсказания той же томограммы. Достигаемый выигрыш 
варьируется от нескольких сотых долей для томограмм легких до нескольких 
десятых долей для томограмм мозга.

Построение кодовых распределений и~оценка избыточности арифметического 
кодирования для любой из компонент ДВП осуществляется так, как описано 
в~подразд.~4.3. Единственное отличие~--- использование дополнительного 
отрицательного диапазона для фонового состояния~--- подробно рассмотрено 
в~подразд.~5.2.



% Table 9
\setcounter{table}{8}

\begin{table*}[b]\small
\vspace*{-9pt}
\begin{center}
\Caption{Избыточность кодирования компонент ДВП (процедура уравнивания)}
\label{tab9}
\vspace{2ex}

\tabcolsep=11pt
\begin{tabular}{|c|c|c|c|c|c|}
\hline
&&&&&\\[-9pt]
X & $R_G$ &  $R$ & $R_{\mathrm{T}}$ & $R+R_{\mathrm{T}}$ & 
$(R+R_{\mathrm{T}})/\tilde{H}^3$ \\
\hline
A1 & 0,021421 & 0,052898 & 0,013824 & 0,066723 & 0,013749\\
%\hline
V1 & 0,007909 & 0,019402 & 0,010681 & 0,030083 & 0,007524\\
%\hline
H1 & 0,013685 & 0,027911 & 0,011398 & 0,039309 & 0,008753\\
%\hline
D1 & 0,005286 & 0,014373 & 0,009247 & 0,023619 & 0,005890\\
%\hline
A2 & 0,023268 & 0,054725 & 0,013824 & 0,068549 & 0,013451\\
%\hline
V2 & 0,007788 & 0,019579 & 0,009964 & 0,029543 & 0,006987\\
%\hline
H2 & 0,012490 & 0,026999 & 0,010681 & 0,037680 & 0,008023\\
%\hline
D2 & 0,005202 & 0,014967 & 0,009964 & 0,024931 & 0,005822\\
%\hline
A3 & 0,036772 & 0,084136 & 0,013824 & 0,097960 & 0,015176\\
%\hline
V3 & 0,016965 & 0,042210 & 0,012833 & 0,055043 & 0,009102\\
%\hline
H3 & 0,022136 & 0,048502 & 0,012833 & 0,061335 & 0,009893\\
%\hline
D3 & 0,014539 & 0,033993 & 0,011398 & 0,045392 & 0,007433\\
%\hline
A4 & 0,041553 & 0,093684 & 0,013824 & 0,107509 & 0,016776\\
%\hline
V4 & 0,018914 & 0,045172 & 0,012833 & 0,058004 & 0,009721\\
%\hline
H4 & 0,022429 & 0,051096 & 0,012833 & 0,063929 & 0,010473\\
%\hline
D4 & 0,014004 & 0,033910 & 0,012115 & 0,046025 & 0,007655\\
%\hline
A5 & 0,022703 & 0,052829 & 0,012360 & 0,065189 & 0,013403\\
%\hline
V5 & 0,007669 & 0,018427 & 0,011398 & 0,029825 & 0,007712\\
%\hline
H5 & 0,012084 & 0,022112 & 0,011398 & 0,033511 & 0,009271\\
%\hline
D5 & 0,006518 & 0,014612 & 0,010681 & 0,025293 & 0,008206\\
%\hline
A6 & 0,023082 & 0,053538 & 0,013824 & 0,067363 & 0,014518\\
%\hline
V6 & 0,012474 & 0,022548 & 0,011398 & 0,033947 & 0,009259\\
%\hline
H6 & 0,016700 & 0,025930 & 0,011398 & 0,037328 & 0,010792\\
%\hline
D6 & 0,010453 & 0,017056 & 0,009247 & 0,026303 & 0,009213\\
\hline
\end{tabular}
\end{center}
%\end{table*}
% Table 10
%\begin{table*}\small
\begin{center}
\Caption{Избыточность кодирования в~целом (процедура уравнивания)}
\label{tab10}
\vspace{2ex}

\tabcolsep=10pt
\begin{tabular}{|c|c|c|c|c|c|}
\hline
&&&&&\\[-9pt]
T & $R_G$ &  $R$ & $R_{\rm T}$ & $R+R_{\rm T}$ & $(R+R_{\rm T})/\tilde{H}^3$\\
\hline
T1 & 0,012075 & 0,028646 & 0,011288 & 0,039934 & 0,009205\\
%\hline
T2 & 0,012187 & 0,029068 & 0,011108 & 0,040176 & 0,008780\\
%\hline
T3 & 0,022603 & 0,052210 & 0,012722 & 0,064933 & 0,010469\\
%\hline
T4 & 0,024225 & 0,055966 & 0,012901 & 0,068867 & 0,011247\\
%\hline
T5 & 0,012244 & 0,026995 & 0,011459 & 0,038455 & 0,009970\\
%\hline
T6 & 0,015677 & 0,029768 & 0,011467 & 0,041235 & 0,011281\\
\hline
\end{tabular}
\end{center}
\end{table*}

В табл.~\ref{tab9} представлены оценки избыточности кодирования 
компонент ДВП томограмм~T1--T6, полученные с~использованием процедуры 
урав-\linebreak\vspace*{-12pt}

\pagebreak

{\small   %tabl8
 \noindent
{{\tablename~8}\ \ \small{Квазиэнтропия в~целом (оптимальные/квазиоптимальные пороги)}}
%\vspace*{2ex}

\begin{center}
\tabcolsep=10pt
\begin{tabular}{|c|c|c|c|}
\hline
&&&\\[-9pt]
T  & $\hat{H}^2$ $(T=2)$ & $\hat{H}^3$ $(T=3)$ &  $\tilde{H}^3$ $(T=3)$\\
\hline
T1 & 4,358309 & 4,335819 & 4,338043 \\
%\hline
T2 & 4,596369 & 4,573489 & 4,575704 \\
%\hline
T3 & 6,224715 & 6,199947 & 6,202132 \\
%\hline
T4 & 6,147353 & 6,122388 & 6,123008 \\
%\hline
T5 & 3,890830 & 3,854840 & 3,857012 \\
%\hline
T6 & 3,672084 & 3,651463 & 3,655139 \\
\hline
\end{tabular}
\end{center}
\vspace*{16pt}
}



\noindent
нивания при построении кодовых распределений. Параметры процедуры 
выбирались так же, как и~ранее ($\rho\hm= 0{,}004$ и~$I_{\max}^+\hm=4$). 
Как и~для ошибок предсказания, использовано квазиоптимальное множество 
состояний (свое для каждой компоненты ДВП). Структура таблицы аналогична 
структу\-ре табл.~\ref{tab4} и~\ref{tab6}. В~первом столбце таблицы указаны 
тип компоненты ДВП и~номер томограммы. В~остальных столбцах~--- 
соответствующие значения из\-бы\-точ\-ности симметризации, индивидуальной из\-бы\-точ\-ности 
арифметического кодирования (с~учетом симметризации), избыточности передачи, общей 
избыточности кодирования метода и~относительной общей избыточности кодирования. 
Аккуратный подсчет числа битов, необходимых для передачи параметров декодеру, 
может быть произведен так же, как 
в~подразд.~4.3. Подсчет должен производиться отдельно для компоненты каждого типа.




Поскольку значение избыточности кодирования томограммы в~целом равно 
среднему арифметическому значений избыточности всех компонент ДВП, 
она может легко быть вычислена на основе данных табл.~\ref{tab9}. 
Соответствующие результаты для томограмм~T1--T6 приведены в~табл.~\ref{tab10}.

% Table 11
\begin{table*}[b]\small
\vspace*{-9pt}
\begin{center}
\Caption{Скорости кодирования томограмм~T1--T6}
\label{tab11}
\vspace{2ex}

\begin{tabular}{|c|c|c|c|c|}
\hline
&Ошибки&&\multicolumn{2}{c|}{JP2}\\
\cline{4-5}
\multicolumn{1}{|c|}
{\raisebox{6pt}[0pt][0pt]{ T}} &  предсказания & 
\multicolumn{1}{c|}
{\raisebox{6pt}[0pt][0pt]{\tabcolsep=0pt\begin{tabular}{c}Компоненты\\ ДВП\end{tabular}}} 
& Вариант 1 & Вариант~2\\
\hline
T1 & 4,505450 & 4,377977 & 4,696686 & 4,481018\\
%\hline
T2 & 4,789562 & 4,615880 & 4,925232 & 4,706512\\
%\hline
T3 & 6,319010 & 6,267065 & 6,672668 & 6,481934\\
%\hline
T4 & 6,229735 & 6,191875 & 6,603790 & 6,408722\\
%\hline
T5 & 4,237956 & 3,895467 & 4,252380 & 3,941132\\
%\hline
T6 & 3,978116 & 3,696374 & 3,959625 & 3,703583\\
\hline
\end{tabular}
\end{center}
\end{table*}


Сравнение приведенных данных с~аналогичными данными табл.~\ref{tab6}, 
относящимися к~кодированию ошибок предсказания, обнаруживает более чем 
двукратное увеличение общей избыточности кодирования метода. Это вызвано и~увеличением 
избыточности арифметического кодирования, и~увеличением избыточности передачи. 

Причинами увеличения избыточности арифметического кодирования являются увеличение 
избыточности симметризации и~снижение качества\linebreak аппроксимации. И~то, и~другое 
обусловлено четырехкратным %\linebreak 
уменьшением числа отсчетов компонент 
ДВП в~сравнении с~исходным числом отсчетов. В~ре\-зуль\-та\-те частотные распределения\linebreak 
стано\-вят\-ся менее <<представительными>>, что увеличивает асим\-мет\-рию и~снижает 
качество аппрокси\-мации. 
{ %\looseness=1

}

Увеличение избыточности передачи связано, главным 
образом, с~необходимостью вместо одного набора параметров передать четыре 
набора, по одному для каждой компоненты. Кроме того, снижение качества 
аппроксимации приводит к~увеличению числа диапазонов, которые строятся в~процессе 
применения процедуры уравни\-вания.
{\looseness=1

}

Несмотря на все сказанное, общая относительная избыточность кодирования метода 
находится на уровне~1\%, что с~практической точки зрения может быть признано 
более чем хорошим результатом.

\vspace*{-6pt}

\section{Заключение}

В работе построены и~исследованы два метода обратимого сжатия компьютерных 
томограмм. Первый метод предполагает кодирование ошибок предсказания томограммы, 
второй~--- кодирование компонентов ДВП томограммы. Оба метода получены как результат 
адаптации к~конкретному типу данных некоторого общего метода (подхода), основанного 
на применении универсального арифметического кодирования. Для каждого из методов 
получены эффективные индивидуальные оценки скорости кодирования.

Томограммы представляют собой полутоновые изображения, поэтому их обратимое 
сжатие может быть осуществлено методом, входящим в~стандарт JPEG~2000. Сравнение 
скоростей кодирования рассмотрен\-ных в~работе методов и~алгоритма обратимого 
сжатия JPEG~2000 позволяет вынести суждения как об эффективности конкретных 
предложенных методов, так и~о~потенциальных возможностях предложенного общего 
подхода в~целом.

В сводной табл.~\ref{tab11} представлены экспериментально полученные оценки 
скорости кодирования (в битах на пиксель) для томограмм~T1--T6. Во второй 
колонке приведены результаты, отвеча\-ющие кодированию значений ошибок предсказания\linebreak 
(см.\ разд.~4), в~третьей~--- кодированию значений компонент ДВП (см.\ разд.~5). 
В~обоих случаях приведенные скорости кодирования соответствуют алгоритмам, 
использующим квазиоптимальное множество состояний и~процедуру уравнивания 
при построении кодовых вероятностей, т.\,е.\ алгоритмам, допускающим <<быструю>> 
реализацию. В~чет\-вер\-той колонке приведена скорость кодирования в~случае применения 
алгоритма JPEG~2000 к~исходным данным томограммы (JP2, вариант~1), в~пятой --- 
скорость кодирования в~случае применения алгоритма JPEG~2000 к~данным, 
полученным после амплитудного преобразования~(\ref{eq21}) (JP2, вариант~2). 
В~последних двух случаях для кодирования использована эталонная реализация (Jasper) 
стандарта JPEG~2000 (библиотека доступна по адресу 
{\sf http://www.ece.uvic.ca/$\sim$mdadams/jasper}).


Приведенные в~табл.~\ref{tab11} результаты показывают, что 
наименьшая скорость кодирования (наибольшая степень сжатия) 
достигается при использовании метода кодирования компонент ДВП. Отметим, 
что применение к~данным преобразования~(\ref{eq21}) заметно уменьшает 
скорость кодирования данных алгоритмом JPEG~2000. 

Можно констатировать, 
что разработанные для сжатия томограмм методы кодирования весьма эффективны, 
хотя выигрыш, который достигается по сравнению с~алгоритмом JPEG~2000, 
по-ви\-ди\-мо\-му, недостаточен, чтобы рекомендовать их практическое внедрение. 
Более важным является то, что полученные результаты демонстрируют большие потенциальные 
возможности общего метода (подхода), основанного на применении универсального 
арифметического кодирования, который можно с~успехом адаптировать для построения 
методов сжатия таких данных, которые не являются изображениями и~где алгоритм JPEG~2000 
неприменим. Это может стать целью дальнейших исследований. В~част\-ности, большой 
интерес представляет задача построения метода обратимого сжатия карт 
силовых кривых атом\-но-си\-ло\-вой микроскопии.

%\vspace*{-9pt}


{\small\frenchspacing
 {%\baselineskip=10.8pt
 \addcontentsline{toc}{section}{References}
 \begin{thebibliography}{9}
\bibitem{b01}
\Au{Сушко Д.\,В., Штарьков~Ю.\,М.} О~сжатии томографических данных~// 
Информационные процессы, 2008. Т.~8. №\,4. С.~240--255.
\bibitem{b02}
\Au{Witten I.\,H., Neal R.\,M., Cleary~J.\,G.} Arithmetic coding for data compression~// 
Commun. ACM, 1987. Vol.~30. No.\,6. P.~520--540.
\bibitem{b03}
\Au{Сушко Д.\,В.} Выбор состояний источника при сжатии томограмм~// 
Информационные процессы, 2010. Т.~10. №\,3. С.~237--244.
\bibitem{b04}
\Au{Sushko D.\,V.}  Choice of source states for compression of tomograms~// 
J.~Commun. Technol. El., 2011. Vol.~56. No.\,6. P.~716--721.
\bibitem{b05}
\Au{Сушко Д.\,В., Штарьков~Ю.\,М.}  Вей\-в\-лет-пре\-об\-ра\-зо\-ва\-ния 
и~сжатие компьютерных томограмм~// Информационные процессы, 2009. Т.~9. №\,2. С.~105--115.
\bibitem{b06}
\Au{Добеши И.} Десять лекций по вейвлетам~/ Пер. с~англ.~--- 
Ижевск: Регулярная и~хаотическая динамика, 2001. 464~с. 
(\Au{Daubechies~I.} Ten lectures on wavelets.~--- CBMS-NSF regional conference 
ser. in applied mathematics.~--- SIAM, 1992. Vol.~61. 377~p.)
\bibitem{b07}
\Au{Le Gall D., Tabatabai~A.}  Sub-band coding of digital images using symmetric short 
kernel filters and arithmetic coding techniques~// IEEE  Conference  (International) on
Acoustics,  Speech, and Signal Processing Proceedings, 1988. P.~761--764.
\bibitem{b08}
\Au{Sweldens~W.} The lifting scheme: A~custom-design construction 
of biorthogonal wavelets~//  Appl. Comput. Harmon. Anal., 1996.  
Vol.~3. No.\,2. P.~186--200.
 \end{thebibliography}

 }
 }

\end{multicols}

\vspace*{-9pt}

\hfill{\small\textit{Поступила в~редакцию 30.11.16}}

\vspace*{6pt}

%\newpage

%\vspace*{-24pt}

\hrule

\vspace*{2pt}

\hrule

%\vspace*{8pt}


\def\tit{REVERSIBLE DATA COMPRESSION BY~UNIVERSAL~ARITHMETIC~CODING}

\def\titkol{Reversible data compression by~universal arithmetic coding}

\def\aut{A.\,I.~Stefanovich and D.\,V.~Sushko}

\def\autkol{A.\,I.~Stefanovich and D.\,V.~Sushko}

\titel{\tit}{\aut}{\autkol}{\titkol}

\vspace*{-9pt}


 \noindent
Institute of Informatics Problems, 
Federal Research Center ``Computer Science and Control'' 
of the Russian Academy of Sciences, 44-2~Vavilov Str., Moscow 119333, 
Russian Federation 



\def\leftfootline{\small{\textbf{\thepage}
\hfill INFORMATIKA I EE PRIMENENIYA~--- INFORMATICS AND
APPLICATIONS\ \ \ 2017\ \ \ volume~11\ \ \ issue\ 1}
}%
 \def\rightfootline{\small{INFORMATIKA I EE PRIMENENIYA~---
INFORMATICS AND APPLICATIONS\ \ \ 2017\ \ \ volume~11\ \ \ issue\ 1
\hfill \textbf{\thepage}}}

\vspace*{3pt}



\Abste{The paper considers the general approach to the reversible (lossless) 
digital data compression problem, which is based on universal arithmetic coding 
of data with unknown statistics. A~model of a source with calculable sequence 
of states is used for data description. Within the approach, the tasks of obtaining 
specific compression methods and algorithms for particular data types are set up. 
The authors use computed tomography data (tomograms) as the object of the study 
and present two methods of lossless compression of tomograms. The first method 
encodes prediction errors of tomograms; the second method encodes components 
of discrete wavelet transform of tomograms. These methods are examined in 
details, effective compression algorithms are constructed, and individual 
estimates of bit rate are obtained for the algorithms. The bit rates of the 
constructed algorithms and the lossless compression algorithms of the JPEG~2000~standard 
are compared. The results demonstrate high quality of the 
constructed algorithms and indicate great potential of the approach in general.}

\KWE{reversible data compression; lossless compression; universal coding; 
arithmetic coding; computed tomography}

\DOI{10.14357/19922264170103}

%\vspace*{-9pt}

%\Ack



%\vspace*{3pt}

  \begin{multicols}{2}

\renewcommand{\bibname}{\protect\rmfamily References}
%\renewcommand{\bibname}{\large\protect\rm References}

{\small\frenchspacing
 {%\baselineskip=10.8pt
 \addcontentsline{toc}{section}{References}
 \begin{thebibliography}{9}
\bibitem{1-sh-1}
\Aue{Sushko, D.\,V., and Yu.\,M.~Shtar'kov}. 2008. 
O~szhatii tomograficheskikh dannykh [On tomography data compression]. 
\textit{Informatsionnye protsessy} [Information Processes] 8(4):240--255.
\bibitem{2-sh-1}
\Aue{Witten, I.\,H., R.\,M.~Neal, and J.\,G.~Cleary}. 1987. 
Arithmetic coding for data compression. \textit{Commun. ACM} 30(6):520--540.
\bibitem{3-sh-1}
\Aue{Sushko, D.\,V.} 2010. Vybor sostoyaniy istochnika pri szhatii tomogramm 
[Choice of source states for compression of tomograms]. 
\textit{Informatsionnye protsessy} [Information Processes] 10(3):237--244.
\bibitem{4-sh-1}
\Aue{Sushko, D.\,V.} 2011. Choice of source states for compression of tomograms. 
\textit{J.~Commun. Technol. El.} 56(6):716--721.
\bibitem{5-sh-1}
\Aue{Sushko, D.\,V., and Yu.\,M.~Shtar'kov}. 2009. Veyvlet-preobrazovaniya 
i~szhatie komp'yuternykh tomogramm [Wavelet transforms and computed tomogram 
compression]. \textit{Informatsionnye protsessy} [Information Processes] 9(2):105--115.
\bibitem{6-sh-1}
\Aue{Daubechies, I.} 1992.
\textit{Ten lectures on wavelets}. 
CBMS-NSF regional conference ser. in applied mathematics. SIAM. Vol.~61. 377~p.
\bibitem{7-sh-1}
\Aue{Le Gall,~D., and A.~Tabatabai}. 1988. 
Sub-band coding of digital images using symmetric short kernel filters 
and arithmetic coding techniques. 
\textit{IEEE Conference (International) on Acoustics,  Speech,
and Signal Processing Proceedings}. 761--764.
\bibitem{8-sh-1}
\Aue{Sweldens, W.} 1996. The lifting scheme: 
A~custom-design construction of biorthogonal wavelets. 
\textit{Appl. Comput. Harmon. Anal.} 3(2):186--200.
\end{thebibliography}

 }
 }

\end{multicols}

\vspace*{-3pt}

\hfill{\small\textit{Received November 30, 2016}}


\Contr

\noindent
\textbf{Stefanovich Alexei I.} (b.\ 1983)~---
scientist, Institute of Informatics Problems, Federal Research Center 
``Computer Science and Control'' of the Russian Academy of Sciences, 44-2~Vavilov
Str., Moscow 119333, Russian Federation; \mbox{astefanovich@ipiran.ru} 

\vspace*{3pt}

\noindent
\textbf{Sushko Dmitry V.} (b.\ 1962)~---
Candidate of Science (PhD) in physics and mathematics, senior scientist, Institute 
of Informatics Problems, Federal Research Center ``Computer Science
and Control'' of the Russian Academy of Sciences, 44-2~Vavilov Str., Moscow 119333, 
Russian Federation; \mbox{dsushko@ipiran.ru} 
\label{end\stat}


\renewcommand{\bibname}{\protect\rm Литература}  %3
%\newcommand{\eol}{\end{enumerate}\setlength{\itemsep}{-\parsep}}
%\newcommand{\ang}[1]{\langle{#1}\rangle}
%\newcommand{\infinity}{\infty}
%\newcommand{\mess}[1]{\mbox{\tt #1}}
%\newcommand{\var}[1]{\mbox{\it #1}}
%\newcommand{\order}[1]{\stackrel{#1}\fa}
%\newcommand{\orderr}[1]{\stackrel{#1}\Longrightarrow}
%\newcommand{\infrel}[1]{\stackrel{#1}\Longrightarrow}
%\newcommand{\prog}{\mbox{\tt Prog}}
%\newcommand{\comment}[1]{}
%\newcommand{\set}[1]{\{#1\}}
%\newcommand{\pair}[2]{\langle #1,#2 \rangle}
%\newcommand{\remove}[1]{}
%\renewcommand{\qed}{\hfill\rule{2mm}{2mm}}
%\newcommand{\bull}[1]{\begin{itemize}\item{#1}\end{itemize}}
%\newcommand{\marg}[1]{\marginpar{\small #1}}


\renewcommand{\figurename}{\protect\bf Figure}
\renewcommand{\tablename}{\protect\bf Table}

\def\stat{frenkel}


\def\tit{SEAMLESS ROUTE UPDATES IN SOFTWARE-DEFINED NETWORKING 
VIA QUALITY OF~SERVICE COMPLIANCE VERIFICATION}

\def\titkol{Seamless route updates in software-defined networking via 
quality of service compliance verification}

\def\autkol{S.\,L.~Frenkel and~D.~Khankin}

\def\aut{S.\,L.~Frenkel$^1$ and~D.~Khankin$^2$}

\titel{\tit}{\aut}{\autkol}{\titkol}

%{\renewcommand{\thefootnote}{\fnsymbol{footnote}}
%\footnotetext[1] {The 
%research of Yuri Kabanov was done under partial financial support of the grant 
%of RSF No.\,14-49-00079.}}

\renewcommand{\thefootnote}{\arabic{footnote}}
\footnotetext[1]{Institute of Informatics Problems, Federal Research 
Center ``Computer Science and Control'' of the Russian Academy of Sciences,
 44-2~Vavilov Str., Moscow 119333, Russian Federation, \mbox{fsergei51@gmail.com}}
\footnotetext[2]{Computer Science Department, Ben-Gurion University of the Negev, 
Beer-Sheva 84105, Israel, \mbox{danielkh@post.bgu.ac.il}}


\index{Frenkel S.\,L.}
\index{Khankin D.}
\index{Френкель С.}
\index{Ханкин Д.}

\def\leftfootline{\small{\textbf{\thepage}
\hfill INFORMATIKA I EE PRIMENENIYA~--- INFORMATICS AND
APPLICATIONS\ \ \ 2018\ \ \ volume~12\ \ \ issue\ 4}
}%
 \def\rightfootline{\small{INFORMATIKA I EE PRIMENENIYA~---
INFORMATICS AND APPLICATIONS\ \ \ 2018\ \ \ volume~12\ \ \ issue\ 4
\hfill \textbf{\thepage}}}

\vspace*{4pt}

\Abste{In software-defined networking (SDN), the control plane and the data 
plane are decoupled. This allows high flexibility by providing abstractions 
for network management applications and being directly programmable. 
However, reconfiguration and updates of a~network are sometimes inevitable due 
to topology changes, maintenance, or failures. In the scenario,  
a~current route~$C$ and a set of possible new routes~$\{N_i\}$, where one of the 
new routes is required to replace the current route, are given. There is a chance that 
a~new route $N_i$ is longer than a~different new route $N_j$, but $N_i$ is 
a~more reliable one and it will update faster or perform better after the update 
in terms of quality of service (QoS) demands. 
Taking into account the random nature of the network functioning, 
the present authors supplement the recently proposed algorithm by Delaet
\textit{et al}.\ for route updates with 
a~technique based on Markov chains (MCs). As such, an enhanced algorithm 
for complying QoS demands during route updates is proposed
in a~seamless fashion. First, 
an extension to the update algorithm of Delaet \textit{et al}.\ 
that describes the transmission of packets through a~chosen route and compares 
the update process for all possible alternative routes is suggested. Second, several 
methods for choosing a~combination of preferred subparts of new routes, resulting 
in an optimal, in the sense of QoS compliance, new route is provided.} 

\KWE{software-defined networking; Markov chains; quality of service}

\DOI{10.14357/19922264180408}


\vspace*{8pt}


\vskip 12pt plus 9pt minus 6pt

 \thispagestyle{myheadings}

 \begin{multicols}{2}

 \label{st\stat}

\section{Introduction}
\label{s:Intro}

\noindent
Software-defined networking is an emerging network paradigm, in which the 
control plane is decoupled from the data plane enabling centralized control 
logic. Such a~dynamic network may require frequent modifications and updates to 
the network topology and configuration. 
Also, the network topology is available to the centralized control entity, there, 
due to this flexibility, it is possible to perform offline optimized calculations.

Network functions virtualization (NFV) allows replacing traditional network 
devices with software that is running on commodity servers. This software 
implements the functionality that was previously provided by dedicated hardware. 
Network functions virtualization
 allows services to be composed of virtual network functions (VNF) hosted on 
different data centers. Software-defined networking, 
when applied to NFV, helps in addressing challenges 
of dynamic resource management and intelligent service 
orchestration~\cite{rao_sdn_2014}. Sometimes, traffic is often required to pass 
through and be processed by an ordered sequence of possibly remote 
VNFs~\cite{ghaznavi_service_2016}. For example, traffic may be required to pass 
through intrusion detection system, proxy, load balancer, or a~firewall. 
Such concatenation of services is called \textit{service function chaining} 
(SFC).

Consider, for example, two communicating parties in a~network featuring complex 
network topology (e.\,g., Small-world network), and the communication flow is 
passed over a~series of VNFs. It may be the case that the network operator is 
required to move the communicating flow to a~different path due to QoS 
requirements or other possible cost considerations. We are interested 
to model the anticipated expected number of steps until the update is complete 
given a~possible new route following the required QoS demands, e.\,g., 
delay, communication rounds, cost, etc. 

%Aforesaid dynamic networking requires frequent modifications and updates to the network. 
Let us consider a pair $(C, \{N_i\})$ where a~current route~$C$ from~$s$ to~$d$ 
is scheduled to be replaced by a new route from the set~$\{N_i\}$, each from~$s$ 
to~$d$ either. Let us model each route as an ordered list of network elements, such 
as VNFs (SFCs) or generally saying routers. Each new route~$N_i$ is constructed 
during the update process, and thus, certain delays may be introduced due to
 initial packet processing or due to possible losses. 
 %There, the eventual arrival of packets along the new route during the update process is critical for successful route update. Another possible example is when the routes are SFCs, and the requirement is to update a current chain to a new one, different service chains may exhibit different delays. 

The design goals must be achieved by constructing effective algorithms for 
efficient packet QoS routing in NFV/SDN computer network. Depending on the 
QoS metric, the lower (e.\,g., for reliability) or upper (e.\,g., for a~delay) 
constraints represent the desired bounds that the orchestration must meet. 
Since different configurations could meet these bounds, the designer should also 
optimize against a~specific metric by using these both ends of the extreme. 

Methods based on integer linear programming (ILP) were proposed in several works 
(see section~\ref{sec:related_work}). The difficulty of using tools based on ILP 
 in the operational work of an administrator is that in view of the possible 
 infeasibility of the resulting solution, it may take not a~few resources (time, efforts) 
 until acceptable QoS values can be ensured.

We consider the use of ``design via verification'' approach, suggesting a~method 
for complying QoS demands. The method is based on augmenting the update algorithm with
a~verification logic. Namely, we suggest the use of 
\textit{Probabilistic real-time Computation Tree Logic} 
(PCTL)~\cite{hansson_logic_1994} for expressing real-time and probability in systems. 
Using PCTL, we can express the probability for a~process to complete after 
a~certain number of steps along an execution path and verify the selected route 
for the update.


%Assume that packets are sent from a source node $s$ to a destination node $d$ along the current route. After the update process is finished, packets will be forwarded from $s$ to $d$ along the new route. 
Delaet \textit{et al.}\ proposed a~multicast-based scheme for seamlessly updating 
a~current route to a~new one~\cite{delaet_seamless_2015}. 
According to the multicast scheme, the controller instructs 
a~router to temporarily have two $(s,d)$ entries in the routing table. When 
a~router $r \neq d$ receives a~packet from~$s$ to~$d$, it sends the packet 
according to the forwarding instructions of all of its $(s,d)$ routing 
table entries. When two identical copies of a~packet that was multicasted 
over the current and new portion of a~route arrive, the controller can dismantle 
the current route, as the new route is ready. During the update process, packets 
should not be lost, no cycles should be formed, and communication should not 
be disrupted.

%Taking into account the random nature of the network functioning, we supplement the algorithm for route updates introduced by Delaet et al. in \cite{delaet_seamless_2015}, with a technique based on Markov chains. In our extension of the algorithm, we describe the transmission of packets through a chosen route and compare the update process for all the possible alternative routes that are candidates for replacement. 

Our contribution is a model for a successful route update, including its 
intermediate steps, as MC states, each with 
a~given probability. With our model, we are able to characterize the quality of 
an update by expected number of steps in the~MC. 
%We use Markov chains to characterize the quality of the update service, and represent the expected number of steps in the Markov chain as the quality of a successful update. While, the probability for an update event 

We suggest an enhanced update method for the network administrator to augment 
his decision regarding QoS demands in terms of various network parameters and 
possible failure of the update process. Moreover, in contrast to other works, 
we are able to provide a~version of an algorithm that can perform real-time QoS
 assessment during a~route update, for each subpart of a~route. At last, using 
 our method, it is possible that the active new route will be comprised of subparts 
 of different new routes, providing optimal route update service in regard of 
 required network QoS. 

%We assume that each new route is legal. 
%However, mixing subroutes belonging to different routes may result in inconsistent state or a cycle formed in the network. We use different 
%
%
%
%We model the update process as a service, namely as a VNF, and we use Markov chains to characterize the quality of the update service. Using the expected number of steps in the Markov chain representing the update, we abstract the quality of the update service. We calculate for each possible new (sub-)route the expected number of steps required to update an old (sub-)route successfully. Subsequently, the old route is updated to the new route which requires less number of steps with high probability. We supplement the seamless update algorithm proposed by the authors of \cite{delaet_seamless_2015} with the model in this work.

%The virtualized service implementing the update algorithm will provide a recommendation for an optimal choice of a route, based on the performed calculations. Fundamentally, we create a QoS VNF for seamlessly updating a route, regarding network parameters, and taking into consideration the complexity and possible failures of updating a route. In case there exist several alternatives for a route update, there is a chance that one of the possible new routes is much longer, however, a more reliable one, and as such will update faster. 
%
%
%One of the important requirements to modification process is that the update process should not form congestion in the network, nor result in time delays, and not lose any packets. 
%
%
%Additionally, we provide an enhanced version of an algorithm that can perform the quality of service assessment during the update process, for each subpart of the new route. 
%
%We propose a directed graph $G=(V,E)$, for representing the possible legal combinations of sub-routes. The set of common nodes to $(C, \{N_i\})$ subdivides the old route and each of the new routes to sub-routes. For two sub-routes represented by the nodes $u,v \in V$, the sub-route $v$ can be launched after $u$ if and only if there exists a directed edge $(u,v) \in E$. Otherwise, the launch of $v$ after $u$ is forbidden and can result in a cycle formed in the network.


%The results of this work helped to develop an operating strategy for a network administrator, supporting both, seamlessly updating a route, and providing QoS requirements. 

Extended abstract of this work appeared as a conference paper 
in~\cite{frenkel_predicting_2017} which presented preliminary results. 
In this work, we describe in detail the system settings and bring new results 
by providing two additional algorithms.
{\looseness=1

}

In the following section, we overview the related work. Next, we provide 
the required definitions and the system settings and describe the MC 
characterization of the network. Further, we describe different update setting, 
accordingly accompanying algorithms and data structures, used for QoS assessment 
during route updates.

\vspace*{-9pt}

\section{Related Work}
\label{sec:related_work}

\vspace*{-2pt}
%The design goals must be achieved by constructing effective algorithms for efficient packet QoS routing in NFV/SDN computer network. %These algorithms, which must enable an administrator to orchestrate the existing services exported by remote providers, were considered in \cite{martins_clickos_2014, zaalouk_orchsec:_2014}. Likewise, the functional behavior (e.g., services being deprecated by their providers), as well as changes in the non-functional behavior of the orchestrated services (e.g., an increased execution time) were also considered.

%Depending on the QoS metric, the lower (e.g., for reliability) or upper (e.g., for delay) constraints represent the desired bounds that the orchestration must meet. Since different configurations could meet these bounds, the designer must also optimize against a specific metric by using these both ends of extreme.

\noindent
Quality of service routing using multipath was proposed in~\cite{devi_approach_2015}. 
The routing algorithm, initially, eliminates all links that do not meet the 
bandwidth requirements. Then, it finds disjoint shortest paths based on 
the residual network graph in each iteration.

The work~\cite{egilmez_distributed_2012} proposed a~QoS optimized routing 
over multidomain OpenFlow networks managed by a~distributed control plane, 
where each controller performs optimal routing within its domain. 
The QoS routing problem was posed as a~constrained shortest path (CSP) problem, 
and the proposed solution computes a~near-optimal route, based on LARAC 
(Lagrange relaxation based aggregated cost)
algorithm~\cite{juttner_lagrange_2001}. The proposed algorithm is an approximation 
algorithm; it always gives a~suboptimal solution.

For traditional network architecture, a~routing strategy approach based on 
ILP was introduced in~\cite{yu_efficient_2013}.
 The main disadvantage of using ILP is that the problem is NP-hard. 
 Additionally, ILP cannot be applied to probabilistic values. 
 Using linear programming (not limited to integers) rounded to integer solutions 
 will not yield an optimal solution.
 

Route updates are extensively researched in SDN~\cite{foerster_survey_2016}, 
standing on the work by Reitblatt \textit{et al.}\ where requirements for SDN 
updates were examined. This work focused on per-packet consistency property, 
stating that packets have to be forwarded either using the initial configuration 
or the final configuration but never a~mixture of them, throughout the update 
process~\cite{reitblatt_consistent_2011}. The authors proposed 
a~2-phase commit technique which relies on packets tagging so that either of 
the rules is applied. However, such technique wastes critical network resources 
and complications are formed due to packet tagging~\cite{foerster_survey_2016}. 
Further, Delaet \textit{et al.}\ showed in~\cite{delaet_seamless_2015} 
that using a~careful multicast during route updates provides 
a~better working solution.

Hogan and Esposito propose in~\cite{hogan_stochastic_2017} the use of
 Bayesian networks for delay estimation as a~traffic engineering tool and model 
 the path selection problem using a~risk minimization technique. 
 However, the authors state that the accuracy of their model is limited by its 
 ability to correctly identify dependencies in the data. In our work, 
 we suggest a~general tool for probabilistic verification of any network parameter, 
 which does not depend on variance within the dataset.
 
 

In~\cite{mcgeer_safe_2012}, an update protocol proposed where packets are 
sent to the controller during updates; such approach adds 
a~significant cost to the control plane bandwidth~\cite{delaet_seamless_2015}. 
In~\cite{mcgeer_correct_2013}, an algorithm to find 
a~safe update sequence expressed as a~logic circuit has been proposed. 
However, the algorithm 
requires a~dedicated protocol which is not currently 
supported~\cite{foerster_survey_2016}. The authors 
of~\cite{katta_incremental_2013} propose to perform the 2-phase update 
scheme from~\cite{reitblatt_consistent_2011} incrementally, making the update longer. 
%For a thorough review of route updates, the reader is referred to \cite{foerster_survey_2016}.






Software-defined networking allows the involvement of the network administrator into the network 
management during route udpdates and, in particular, during packet transmission. 
Thus, it would be highly desirable to support the decision making process 
with the right tools. Our novelty is exactly such tool, for augmenting 
online decision making of the network administrator during network management 
in a~stochastic environment.
%In this work, we propose a technique to optimize the update process by selecting the preferred (sub-)route in order to reduce the update time. We use the expected number of steps for successfully completing the update as a QoS metric, and extend the algorithm by Delaet~et~al. with Discrete Time Markov Chains (DTMC) for finding (sub-)routes which are preferred in terms of QoS. % As such, we propose to use the route updates algorithm from \cite{delaet_seamless_2015} as a virtual service for network updates per QoS requirements.

%The interaction of software components have a greater weight in NFV context, which may lead to stochastic-like behavior 

%At present, certain routing algorithms (including $k$ Edge-Disjoint) are based on the shortest path (SP) problem solution \cite{wood_toward_2015}. However, the method proposed by Wood et al. is generic and valuable only in the case of request arrival, and do not consider certain additional important requirements, such as removal or priorities of requests. 

%Several approaches for efficient SP-based QoS routing have been recently proposed in \cite{buchbinder_improved_2006}, where the authors introduce and analyze a centralized algorithm for an online scheduling and routing of arbitrary sequence of communication requests. 

%Unsplittable (single-path) assignment for each request of QoS routing is probably competitive with the best possible splittable (multipath assignment).

The work by Delaet \textit{et al.}~[4] introduced the Make\&Activate-Before-Break 
approach for seamless
route update in SDN. The authors described in a~high-level the multicasting-based 
update, which we
employ in this work. Also, they introduced a~controller-based method for 
verifying the correctness
of a~new route before the traffic redirection. Dinitz \textit{et al.}~[16] 
extended the work~[4] to the general
case of several dependent (via shared links) routes pairs. The routes update 
problem was proved to
be NP-hard~\cite{17-aaa}. The authors of~[16] suggested the use of 
artificial intelligence (AI) methods for 
solving the problem. As a~basis for AI-based solutions, Dinitz 
\textit{et al.}\ proposed a dependence graph model describing the current
state of the problem instance at any replacement stage. 
In addition, route readiness verification similar
to that in~[4] was implemented in~[16] as a high-level network protocol.

In this work, we investigate a different problem; we consider the route updates 
problem from a~QoS
perspective and describe in high-level both the prediction and the update processes.

\vspace*{-9pt}

\section{Preliminaries and Definitions}

\vspace*{-2pt}

\noindent
The basic system settings are as follows. 
For a~(route) sequence~$X$, we denote by~$x_i$ the $i$th element in it.
In a~(directed) communication network, 
we are given a~route~$C$ from source~$s$ to destination~$d$. 
Additionally, we are given a~set of different new routes~$N_i$, each going from~$s$ 
to~$d$. We model each route as an ordered set of network nodes connected by network 
links. We assume that neither of the routes contains cycles. 
Each router in a~route matches a~packet from~$s$ to~$d$ 
and forwards the packet to the next router in order. After the update 
is complete, each router in the new route should forward the packets from~$s$ 
to~$d$ to the next router in order along the new route. 

In our work, we consider the route replacement problem as a~sequence of 
subroutes replacements.
The routes replacement subsystem was in great detail described by Dinitz 
\textit{et al.} in~\cite{dinitz_dependence_2017}. We borrow
from~[16] the relevant parts which we briefly describe here.

\smallskip

\noindent
\textbf{Definition~1.} We  define a~subset from $a\in X$ to $b\in X$ of an ordered
set~$X$, when $a$ precedes~$b$, as~a~subroute from~$a$ to~$b$, and denote such subroute by
$[a,b]$.

\smallskip

 

\textbf{Subroutes.} The current route~$C$ subdivides each new route 
to~$k$~common subroutes (a~subroute may consist of one router in the simplest case) 
and $k-1$ noncommon subroutes. 
For illustration, see Fig.~1.
In Fig.~1 and figures below, the current route is depicted
in a~light grey color full nodes, connected with
solid edges. The new route is depicted in white colored nodes, connected with
dashed edges. The common nodes are depicted as shaded. 
If there are several new
routes, the nodes of each route are filled with a~designating pattern. 
Additionally, for easier reading,
when it is possible, we denote subroutes of some route~$X$ as~$X^\prime$, $X^{\prime\prime}$, 
etc. In other cases, a~subroute~$j$
of a~new (current) route~$i$ is denoted as $N_j^i (C_i^j)$. 
Similarly, routers of some route~$X$ are denoted by~$r^\prime$,
$r^{\prime\prime}$, etc.

 { \begin{center}  %fig1
\vspace*{1pt}
 \mbox{%
 \epsfxsize=78.631mm 
 \epsfbox{fre-1.eps}
 }


\vspace*{3pt}


\noindent
{{\figurename~1}\ \ \small{Route $C$ with two possible new routes sharing a~link}}
\end{center}
}

\vspace*{6pt}






In the example in Fig.~1, 
noncommon new subroutes 
of route~$N_1$ are denoted by~$N^1_1=[s,r_2]$ and~$N^2_1=[r_2,d]$, while the noncommon new 
subroutes of~$N_2$ are denoted by~$N^1_2=[s,r_1]$, $N^2_2=[r_1,r_3]$, 
$N^3_2=[r_3,r_2]$, and~$N^4_2=[r_2,d]$. 

Note that in general, the order of common subroutes along~$C$ and along~$N$ 
can be different. See, for example, the common subroutes of~$C$ and~$N_2$ in 
%Figure \ref{fig:two_routes}.
Fig.~1.

\smallskip

\noindent
\textbf{Definition~2.} A~new noncommon subroute of~$N$ from router~$a$
to router~$b$ is legitimate for update only if~$a$ precedes~$b$ on the route~$C$.

\smallskip

Definition~2 guides us on which subroutes can be launched without creating routing cycles in the
network system. (See~[4] for details.)


When an update of a~subroute~$N^\prime$ from router~$r$ to~$r^\prime$ is finished, 
the update flow goes along~$C$ from~$s$ to~$r$, continues along~$N^\prime$ up to~$r^\prime$, 
and finishes along~$C$ from~$r^\prime$
 to~$d$. 
For illustration, see the result of launching~$N^4_2$ in Fig.~2.

 { \begin{center}  %fig2
\vspace*{-1pt}
 \mbox{%
 \epsfxsize=78.631mm 
 \epsfbox{fre-2.eps}
 }


\vspace*{3pt}


\noindent
{{\figurename~2}\ \ \small{$N^4_2$ was launched}}
\end{center}
}

\vspace*{4pt}


 

 Note that launching a~currently nonlegitimate new subroute, for example,~$N^3_2$ 
 in Fig.~1, is forbidden since it will form a~cycle 
 resulting in packets circulating and overwhelming the network. 

\textbf{Dynamics of the system.}
%\label{sec:dynamics} 
Dinitz \textit{et al.}\ performed a~detailed analysis on the dynamics of a~subroutes
system. After an update of a~subroute is complete, the set of current subroutes~$C$ 
and the set
of new subroutes~$N$ are recalculated. This may result in different system of subroutes. For example,
see Fig.~2 where after the launch of $N^4_2$ from the example in Fig.~1, 
the sets of subroutes are
recalculated. As a~result, we obtain different subroutes (for clarity, the previous labels are kept). See
also~[16] for details and extensive analysis.

\vspace*{-4pt}

\subsection{Markov chain characterization of~the~network~states}

\noindent
We characterize execution of some (sub)route in the network by 
a~packet delay time between the (sub)route's common sender and common destination 
routers as well the probability of a~packet drop. Let us for now define our 
network routing model (conceptual model) informally in the following terms. 
Delay of a~packet is obtained using a~physical delay and the total processing 
time in the router. We consider that transmission of packets in 
a~network can have a~random behavior, caused by the random character of both, 
the input, and possible loss of packets. There we are interested in 
a~probabilistic model, namely, a~Markov model. In order to fully characterize 
the network as an~MC, the internal state of each router 
(and, in particular, the buffer occupancies), as well as the characteristics
 of all flows, need to be expressed as states in the chain. 

However, such approach would result in an enormous and intractable number of states. 
Therefore, to simplify these computations, let us characterize the delay time as 
an abstract variable~$t$. This abstract variable can be interpreted in different ways, 
e.\,g., the current processing queue length and a~packet transmission rate of the link, 
or possibly a~fixed value, such as an interval between the beginning of 
a~packet transmission after being processed in some node and the end of processing 
at the next node. 

We describe the functioning of the network in the transmission of packets 
as transitions of a~discrete-time MC (DTMC). The state space corresponds to the set 
of nodes such that 
the transmission of a~packet from a~node that has finished processing the packet 
to the next node corresponds to the transition of the chain to the next state.


Discrete-time MC is defined as a~tuple $D\linebreak =(S, s_0, P)$. In the tuple, $S$ is 
the finite set of states, $s_0\in S$ is the initial
state, $P:S \times S \rightarrow [0, 1]$ is the transition probability matrix in 
which $\forall s\in S$, $\sum\nolimits_{s' \in S} P(s,s') = 1$. 
For any two states $s, s' \in S$, if $P(s,s')>0$, then~$s'$ is the successor of~$s$. 
For a~subset of states $T \subseteq S$, the probability of moving from a~state~$s$ 
to any state $t \in T$ in a~single step is denoted by $P(s, T)$ and is given by 
$P(s,T)=\sum\nolimits_{t \in T} P(s, t)$. 
%The row $P(s,:)$, in the transition matrix $P$, contains the probabilities of moving from $s$ to its successors, while the column $P(:, s)$ contains the probabilities of entering the state $s$ from any other state.

\vspace*{-6pt}

\subsection{Verification syntax}

\noindent
For implementation of our PCTL-based model, we use PRISM~--- 
probabilistic model checker~\cite{kwiatkowska_prism_2011}. There, we follow 
PRISM property specification language. Here, we briefly describe the essential 
syntax while more details can be found in~\cite{noauthor_prism_nodate}.

Given a property~$\Psi$, we say that~$\Psi$ is true with probability~$p$ 
and write that as
$P_p [ \Psi ]$. If the probability~$p$ is unknown, PRISM allows, for DTMC, 
writing properties queries of the form $P_{=?}[ \Psi ]$, meaning 
``what is the probability that~$\Psi$ is true?''. Additionally, it is possible 
to use a~time bound and write properties queries such as 
$P_{=?}[F^{\leq T} \Psi]$, meaning ``what is the probability that~$\Psi$ 
is true after less than~$T$~steps?''. At last, it is possible to compute 
properties such as expected time or expected number of steps. 
For example, $R_{=?}[F \Psi]$, meaning ``what is the expected number of 
steps until $\Psi$ is true?''. 
%\section{Model Settings}
%, and a subroute of route $X$ from router $a$ to router $b$ is specified by $[a,b]_X$

%When a new subroute of $N$ that is scheduled to update a current sub-route of $C_i$ is launched, the route $C$ is updated such that the updated sub-route is replaced by launched sub-route, and the new sub-route is now part of the current route $C$.

\setcounter{figure}{3}
\begin{figure*}[b] %fig4
\vspace*{-6pt}
 \begin{center}
 \mbox{%
 \epsfxsize=149.177mm 
 \epsfbox{fre-3.eps}
 }
 \end{center}
\vspace*{-9pt}

 \Caption{New routes~$N_1$~(\textit{a}) and $N_2$~(\textit{b}) and
 MC states for~$N_1$~(\textit{c}) 
and~$N_2$~(\textit{d})}
 \label{fig:routes_dtmc_example}
\end{figure*}



\vspace*{-6pt}

\section{Prediction of Preferred Update}
%\section{Prediction of Preferred Update}
\label{sec:dtmc}

\noindent
The states of a~DTMC describe the nodes in the new route and the transition 
probabilities in the chain represent the possible delay or 
a~packet loss in the routers along the new route. The
states are defined as 
$\{s_1, \ldots , s_n\}$ where~$n$ is the number
  of nodes in the new route. 
The network achieves the state~$s_i$ if a packet has reached the $i$th node. 
For example, in Fig.~3, the self-transition 
edge represents the probability for a~delay due to packet loss, rules installation 
at the router, or congestion on the router-controller link, while the 
forward transition edge represents the probability for 
a~successful transition to the next state. These probabilities can be estimated 
from network statistics (see, for example,~\cite{hogan_stochastic_2017}). 
The labels on edges are the probability values, when edge has no label
 means probability~1.
 
 The initial probability distribution of states is given by the vector~$P_0$ of size~$n$. 
We can determine the prob-\linebreak\vspace*{-12pt}
 
 %\linebreak\vspace*{-12pt}

{ \begin{center}  %fig3
\vspace*{-0.5pt}
  \mbox{%
 \epsfxsize=77.518mm 
 \epsfbox{fre-4.eps}
 }


\end{center}

\vspace*{-3pt}

\noindent
{{\figurename~3}\ \ \small{Probability as a~function of number of steps to update routes~$N_1$~(\textit{1})
 and~$N_2$~(\textit{2})}}
}

\vspace*{12pt}



\noindent
ability that a~particular route delays the update process 
by~$k$, that is, the number of steps required for a~successful update is given by 
$p(k)=P_0 P^k$. Using this characteristic, which is, in fact, the 
probability distribution of the number of steps $P(k < x)$, one can 
calculate various properties like average delay time for the new route, 
maximum or minimum number of steps to update, etc.
 
 Consider the example illustrated in Fig.~4. 
Figure~4\textit{a} illustrates the current route~$C$ and a candidate new route~$N_1$. 
Figure~4\textit{b} shows the same current route~$C$ with another candidate 
new route~$N_2$. 
Figures~4\textit{c} and~4\textit{d} 
show the MCs for new routes~$N_1$ and~$N_2$, accordingly, with given transition 
probabilities.

During the update process, packets are sent along the current and the new routes. 
Since the new route is\linebreak\vspace*{-9.5pt}

\columnbreak

\noindent
 not operational yet, packets can be delayed due to 
congestion on certain nodes or due to switch configurations. 
%
For example, if routing rules have not yet been installed in some switch, then an 
arriving packet is sent to the controller~\cite{onf_openflow_2015}. The controller 
then decides reactively on further actions whether to install an appropriate rule 
for the packet. Also, the controller may be busy with other work and not respond 
immediately. Those packet processing actions may delay the update process. 
In the case buffer becomes full, for example, if the network is being congested, 
packets may be dropped. There, the transition to the next state during the 
update process depends on the likelihood of a~delay or a~loss of a~packet in the 
current state. 

In the example, the number of steps required for launching~$N_2$ is smaller than 
the number of steps required for launching~$N_1$. However, due to a higher likelihood 
of delays along the route~$N_2$, it is possible that~$N_1$ is preferred having 
a~higher probability for a~successful update. The network administrator may ask 
which new route is recommended for the update process, considering the expected 
number of steps required for the update. 
%
That is, updating paths requires the operator to decide 
on the possible choice of a~subroute for the next step. 
One should consider the possibility of including a~decision tool augmenting the 
controller during route updates. 

There were many attempts to use the LP/ILP 
approach, as it was already mentioned above (see, e.\,g.,~\cite{juttner_lagrange_2001}), 
but they have encountered the same difficulties, especially when taking 
into account online implementation. We show that it is possible to describe 
the routing process as DTMC. Thus, taking into consideration~$O(n^3)$ worst case 
computation complexity, we consider using the ``design via verification'' 
mentioned above based on PCTL verification, similar to the one used in 
PRISM~\cite{kwiatkowska_prism_2011}.


We have calculated the probability for a~successful update as a~function of 
number of steps for routes~$N_1$ and~$N_2$ from the example in 
Fig.~\ref{fig:routes_dtmc_example}. See Fig.~3 
where this function is shown. Curve~\textit{1}
represents the plot for~$N_1$ and curve~\textit{2} represents
 the plot for~$N_2$. 

Observe that after~20~steps, both new routes will be launched with probability~1 
which can be written as 
$$
P_{1}\left[F^{>20}N_1\right]=P_{1}\left[F^{>20}N_2\right]=1\,.
$$
The expected number of steps required for~$N_1$ is smaller than the required for~$N_2$:
$$
R \left[F~N_1\right] < R \left[F~N_2\right]\,.
$$
However, the probability for successfully updating in less than~15~steps 
is higher for route~$N_2$ ($0.55 \pm 0.040$ for~$N_1$ and 
$0.717 \pm 0.036$ for~$N_2$, based on~99\% confidence level):
$P_{0.717 \pm 0.036}\left[F^{\leq 15} N_2 \right].$

\vspace*{-6pt}


\section{Route Updates per~Quality~of~Service}
\label{sec:updates_qos}

\vspace*{-2pt}

\noindent
In this section, we show algorithm that we propose for various settings. 
First, we show an enhancement for the sequential update algorithm 
from~\cite{delaet_seamless_2015}, which during the update process decides on 
preferred subroute from the set of possible subroutes as part of QoS requirements. 
In the multicast-based update, several methods were proposed 
in~\cite{delaet_seamless_2015} for eliminating duplicated packets. 
In the case the common destination router is not able to immediately eliminate 
duplicated packets, the algorithm begins the update from the end, 
ensuring a~correct update process~[4].



\begin{algorithm*} %alg1
 \setlength{\algowidth}{100mm}
 \setlength{\hsize}{\algowidth}
 \caption{Update per QoS Algorithm}
 \label{alg:update_per_qos}

%\hrule
%\vspace*{2pt}
%\centerline
%{\textbf{Algorithm~1:} Update per QoS Algorithm}\par

%\vspace*{2pt}

%\hrule
 \small
 
 %\Input
 {directed graph $G$} 
 
 \BlankLine
 \tcc{$A$ is a collection of nodes} $A \leftarrow$ choose nodes from $G$ with in-degree $0$ \\
 
 \Repeat {out-degree of node $N^t_i > 0$}
 {
 \ForEach{$v \in A$ \label{alg:inner_loop}}
 {
 calculate $R[F~v]$ \\
% calculate the expected QoS for this node as described in Section \ref{sec:updates_qos} \\
 }\label{alg:end_inner_loop}
 
% $N^t_i \leftarrow$ choose the node that maximizes QoS \label{alg:choose_qos}\\ 
 $N^t_i \leftarrow \argmax_{v} (R[F~v])$ \label{alg:choose_qos} \\
 launch $N^t_i$ \\
 update $C$ accordingly \\
 merge any new and common subroutes as described in section~3 \\ 
 $A \leftarrow$ choose nodes neighboring to $N^t_i$ \\ 
 }
 
 \BlankLine 
 
\end{algorithm*}





 
%The algorithm starts from any node with in-degree 0 since it means that such node has no precedence dependence. Updating is completed when the algorithm arrives to a node with out-degree zero, which would be the last subroute to launch.


After that, we show an algorithm that chooses the subroutes for update arbitrary, 
assuming that the common destination node will not leak duplicated packets. 
However, the packets sending rate along the new subroute need to be temporarily limited~[4].

At last, we present a supplementing algorithm that suggests which subroutes can 
be updated in parallel.

%The set of common nodes for each pair of routes subdivides the routes to sub-routes relatively to each other. 

\vspace*{12pt}

\subsection{Sequential update}

\noindent
Let us begin the update from the end, namely, from the last alternative 
subroute of any new route. Provably, this prevents the formation of 
cycles~\cite{delaet_seamless_2015}. In order to represent all possible choices 
of a~path from a current state of the update process to the end of the update process, 
we propose to use a directed graph which nodes are the new, legitimate for launching, 
subroutes of the network. The edges of the graph represent a~legal order of launching 
new subroutes. Each path in this graph from a~current node to the last node in 
the path represents a~legal combination of chosen subroutes. The update process is 
continued as long as there is a~possible node to transition to. 

Let us examine the two possible new routes~$N_1$ and~$N_2$ that can replace the 
current route~$C$ from the example depicted in Fig.~1. 
The new route~$N_1$ is composed of~$N^1_1$ and~$N^2_1$, while the new route~$N_2$ 
composed of~$N^1_2$, $N^2_2$, $N^3_2$, and~$N^4_2$. Starting from the end, the only 
new subroutes that are allowable to launch are~$N^2_1$ and~$N^4_2$. 
Assume that based on the DTMC calculations performed as described in section~4, 
the subroute~$N^4_2$ is chosen for update. After the update of the subroute is 
complete, the current route~$C$ is composed of not updated yet part of the old 
route and~$N^4_2$. See Fig.~2 where the change in~$C$ 
is depicted.

After the subroute~$N^4_2$ is launched, we arrive at a~smaller problem in which 
less subroutes are left to update. Due to dynamics of the system 
(see section~3), some new subroutes can merge into a~single new subroute.
See Fig.~2 where after~$N^4_2$ was launched, the 
new subroutes~$N^3_2$ and~$N^2_2$ are merged into a~single subroute. Now, one 
can launch either~$N^1_1$ or~$N^2_2$ merged with~$N^3_2$. Assume that we choose to 
launch~$N^1_1$, which launch
 finishes the update. The route~$C$ updated to~$N^1_1$ 
and~$N^4_2$. See Fig.~5 illustrating that.


Figure~6 shows the directed graph that represents 
the possible update sequences. Initially, the subroutes that %\linebreak\vspace*{-12pt}
 are legal 
for launch are~$N^2_1$ and~$N^4_2$. As such, these are
the only subroutes that
 have in-degree~0. Launching~$N^3_2$
 is forbidden; hence, there is no node in the 
 graph~$G$ that represents this subroute. After launching~$N^4_2$, we\linebreak\vspace*{-12pt}
 
 \setcounter{figure}{4}

{ \begin{center}  %fig5
\vspace*{12pt}
 \mbox{%
 \epsfxsize=78.631mm 
 \epsfbox{fre-5.eps}
 }


\vspace*{3pt}


\noindent
{{\figurename~5}\ \ \small{$N^1_1$ was launched}}
\end{center}
}

\vspace*{6pt}

{ \begin{center}  %fig6
\vspace*{1pt}
 \mbox{%
 \epsfxsize=36.428mm 
 \epsfbox{fre-6.eps}
 }


\end{center}


\noindent
{{\figurename~6}\ \ \small{Graph 
representation for possible update paths for routes update example from Fig.~1}}

}

%\vspace*{6pt}

\noindent
  can 
 proceed by launching~$N^1_1$ or~$N^2_2$. However, if~$N^2_1$ was launched first, 
 it would be forbidden to launch~$N^2_2$ since it shares a~common edge with~$N^2_1$. 
 This is reflected in the graph~$G$ by not having a~directed edge from the
  node~$N^2_1$ to the node~$N^2_2$. We finish the update process
 by arriving either 
 to~$N^1_1$ or to~$N^1_2$. Notably, these nodes have out-degree~0.

 Algorithm~1 updates subroutes according to calculated QoS for each new subroute, by
 choosing at each step the new subroute that maximizes QoS.


The algorithm starts by selecting the initial set of subroute nodes. 
These are nodes with in-degree~0. The algorithm continues traversing the graph up 
to arrival at a node with out-degree~0 which would be the last subroute to launch. 
The inner loop at lines~\ref{alg:inner_loop}--\ref{alg:end_inner_loop} 
calculates the QoS for each neighboring node. Afterward, at 
line~\ref{alg:choose_qos}, the algorithm chooses the node that maximizes QoS. 
Then launches this node and updates the route~$C$, accordingly (see 
Figs.~1--5 for illustration). 
Afterward, the algorithm selects the next neighboring nodes.

After execution of Algorithm~1, the resulting new route maximally complies QoS 
requirements.

%\vspace*{12pt}

\subsection{Arbitrary subroutes selection} 
%\label{sec:arbitrary}

%\vspace*{-12pt}

\noindent
In this subsection, we assume that immediate duplicate packets elimination is possible. 
It may be that some of the subroutes are not ready for an update yet. 
Thus, meanwhile, the administrator may want to proceed with the update process 
to other subroutes or see possible variations of the update. 
For such scenario, we provide an algorithm which can select a~subroute for 
update arbitrary and continue the update process from there. 
We create a~forest graph of all possible update combinations from which the 
desired update sequence can be chosen. 
{\looseness=1

}
 


Figure~7 shows all possible combinations from example 
in Fig.~1. Noticeable, as mentioned earlier, some\linebreak\vspace*{-12pt}

{ \begin{center}  %fig7
\vspace*{1pt}
  \mbox{%
 \epsfxsize=71.694mm 
 \epsfbox{fre-7.eps}
 }


\end{center}


\noindent
{{\figurename~7}\ \ \small{Forest graph representing execution combinations for example from 
 Fig.~1}}
}

\vspace*{12pt}


\noindent
 combinations 
exhibit fewer steps, though possible that its QoS compliance is worse than others.



Algorithm~2 starts by iterating over all roots of the forest graph and 
calculating QoS using Algorithm~1 each tree. Afterward, launch the update 
of the tree that maximizes QoS.

\begin{algorithm*} %alg2
\setlength{\algowidth}{100mm}
 \setlength{\hsize}{\algowidth}
 \caption{Arbitrary Selection Update}
 \label{alg:arbitrary_update}
 \small
 
% \Input
{directed graph $G$} 
 
 %\BlankLine
 
 $A_0 \leftarrow$ choose nodes from $G$ with in-degree $0$ \\
 $Q \leftarrow \{\}$ \\
 
 \BlankLine
 \tcc{iterate over all roots of trees in the forest $G$}
 \ForEach{$v_r \in A_0$}
 {
 $q \leftarrow$ get the expected QoS using Algorithm~1 for $v_r$ \\
 $Q \leftarrow Q \cup \{q \rightarrow \mathrm{root} \}$ \\
 }

 \BlankLine
 $q_{\max} \leftarrow \max_{\mathrm{QoS}}(Q)$ \\
 launch maximizing QoS update order in $\mathrm{root}=Q[q_{\max}]$ \\ 
 
 
\end{algorithm*}


%\columnbreak

\vspace*{12pt}





\subsection{Parallel update}

\noindent
In certain cases, it is possible to update in parallel several subroutes 
and, as such, decrease update time. However, launching subroutes in parallel 
is not always possible
 since subroute may share a~link and, thus, leads to congestion 
during the update process, close a~cycle, or lead to an inconsistent state of the 
system. In~\cite{delaet_seamless_2015}, it was shown that two new subroutes~$N'$ 
from~$a$ to~$b$ and~$N''$ from~$c$ to~$d$ can be launched in parallel only if~$c$ 
succeeds~$b$ or~$a$ succeeds~$d$.



%\begin{proposition}
% Let $N'$ from $a$ to $b$ and $N''$ from $c$ to $d$ be two legitimate new subroutes. $N'$ and $N''$ can be launched in parallel only if $c$ succeeds $b$ or $a$ succeeds $d$.
%%Two subroutes that are each legitimate can be launched in parallel only if they share at most one common subroute.
%\end{proposition}
%\begin{proof}
% \textbf{Direction}: $\Rightarrow$ Let $N'$ from router $a$ to $b$ and $N''$ from router $c$ to $d$, be two new legitimate sub-routes. The only way for them to share more than one common sub-route is if $b$ succeeds $c$ on $C$. In such case, launching $N'$ will eliminate the part of $C$ from $c$ to $b$ with no proper connection from $b$ to $c$, which leaves the system in an inconsistent state. The same occurs if $N''$ is launched. \\
% \textbf{Direction}: $\Leftarrow$ Let $N'$ from router $a$ to $b$ and $N''$ from router $c$ to $d$, be two new sub-routes, not necessary part of the same new route, such that $b$ precedes $c$ or $b=c$. If $a$ precedes $b$, than $N'$ is legal for launching independently of $N''$. Similarly, if $c$ precedes $b$, than $N''$ is legal for launching independently of $N'$. Thus, since $N'$ can be launched independently from $N''$, they can be launched in parallel. Symmetric considerations lead to same result in case $a$ succeeds $d$.
% 
%\noindent Generalization to more than two sub-routes is trivial.
%\end{proof}



\begin{algorithm*}[b] %[t] %alg3
\setlength{\algowidth}{100mm}
 \setlength{\hsize}{\algowidth}
 \caption{Parallel Update}
 \label{alg:parallel_update}
 \small
 
 %\Input
 {weighted graph $G_S$} 
 
 \BlankLine
 
 \While{there are still current subroutes to update}
 {
 $A \leftarrow$ find maximum-weight independent set in $G_S$ \\
 
 \BlankLine 
 \tcc{do in parallel} 
 \ForEach{$N^t_i \in A$} 
 { 
 launch $N^t_i$ \\
 }
 }
 
 \vspace*{6pt}
 
\end{algorithm*}

We create a supplementary graph~$G_S$, in which nodes are the new legitimate 
for launching subroutes, and edges represent restrictions on parallel 
launching of subroutes. See Fig.~8 for illustration, 
depicting subroutes from example in Fig.~1 and their parallel 
restrictions. For example, $N^4_2$ and~$N^1_2$ can be launched in parallel since 
there is no edge connecting them.

Clearly, any independent set of subroutes from the supplementary 
graph contains subroutes that can be launched in parallel. 
This can be further enhanced by setting QoS calculated values as weights 
on nodes of the graph and finding the subroutes that can be launched 
in parallel by finding a~maximum-weight independent set of the graph~$G_S$. 
Since~$G_S$ has few
 number of nodes (several tens), it is possible to find 
the
 maximum-weight independent set even by enumerating
 all possible independent 
sets~\cite{wu_review_2015} and comparing their total weights.
{\looseness=-1



{ \begin{center}  %fig8
\vspace*{12pt}
  \mbox{%
 \epsfxsize=36.666mm 
 \epsfbox{fre-8.eps}
 }


\end{center}


\noindent
{{\figurename~8}\ \ \small{Supplementary graph of the example in 
 Fig.~1, showing which subroutes cannot be run in parallel}
}}

%\vspace*{12pt}



} 



Important, the parallel method should not be launched on its own. 
For example, assume that at the first iteration of Algorithm~3, 
the independent sets of nodes are~$A_1$ and~$A_2$. Let us assume that~$A_1$ complies 
better to QoS demands than~$A_2$ and, thus, $A_1$ will be selected. 
Also, let us assume that~$B_1$ is the next independent set in the graph 
if~$A_1$ was selected and~$B_2$ if~$A_2$ was selected. 
Also, let us assume that~$B_1$ is
the next independent set in the graph if~$A_1$ was selected and~$B_2$ if~$A_2$ 
was selected.
It is possible that due to the dynamics of the system (see section~3), 
we could obtain overall higher QoS results if we initially launched the 
subroutes from the sets~$A_2$ and~$B_2$ afterwards than from the sets~$A_1$ and~$B_1$.
 

Therefore, the graph that we create in this section for parallelization constraints 
is a~supplementary graph which must be used in conjunction with the graphs from 
previous sections. Optimal results will be obtained when used in conjunction with 
the forest graph from subsection~5.2.

It is also important to note that, in the worst case, when there are 
no disjoint subroutes, the parallel method is reduced to the sequential 
method thought with a higher running time.

\vspace*{-12pt} 


\section{Implementation}

\noindent
We implemented the update algorithms from~\cite{delaet_seamless_2015} as 
services for our QoS verification module. The update algorithm itself 
was not modified. In other words, we treated the update itself as 
an atomic action. The route updates
 algorithms are implemented as 
applications interacting with the northbound interface of an SDN controller. 
We used POX~\cite{kaur_network_2014} as a~platform for controller development and 
Mininet~\cite{lantz_network_2010} for network topology emulation. 
Figure~9 depicts the schematic arrangement of the 
functional elements. 



We created networks with topology of random graph and small-world features. 
During each simulation trial, a~pair of common source and destination nodes $(s,d)$ 
were selected. A~path connecting~$s$ and~$d$ was selected as a~current route and 
a~set of~4~new routes connecting $(s,d)$, to replace the current route, were 
selected, possibly with shared links among themselves and the current route. 

We considered latency due to the formed congestion as QoS demands for the update, 
implemented by forming congestion on randomly selected subroutes. Route 
update was executed by the update algorithm from~\cite{delaet_seamless_2015} for 
each pair of current and new routes. Further, one of the enhanced versions 
was executed, updating to the
 preferred combination of subroutes, by identifying 
the congested subroutes (e.\,g., by estimating latency).

{ \begin{center}  %fig9
\vspace*{8pt}
  \mbox{%
 \epsfxsize=58.544mm 
 \epsfbox{fre-9.eps}
 }

\vspace*{3pt}


\noindent
{{\figurename~9}\ \ \small{Description of the system}
}
\end{center}}

%\vspace*{12pt}



%\vspace*{-45pt}

\section{Concluding Remarks}

\noindent
The study in this paper illustrates a~feasibility of modeling and 
designing the route update process via verification using DTMC. The goal was to 
strengthen the network administrator involvement in management and decision 
making during route update. In the present model, the network administrator is able 
to consider network parameters such as packet losses, delay, communication 
rounds, flow table updates, congestion, and other inherent unreliabilities of 
the network. 

We extended the updating algorithm with the ability to compute QoS as the 
MC characteristics, where the MC corresponds to the states 
of the update process. Using this MC computation ability, it is 
possible to predict the expected number of steps (delay time) required to 
complete the update process. These prediction results allow the administrator 
to make a~decision whether a~new route can satisfy the user requirements per QoS 
or a~more reliable route will be selected.

We provided sequential update algorithm and an arbitrary order algorithm 
when for the later, it is assumed that immediate duplicate packets elimination 
is possible. Further, we suggest a supplementary graph and algorithm for launching 
updates in parallel when it is possible.

This paper proposes a~conceptual approach. In future research, we will focus 
on optimization of predictions supplementing the network administrator with 
a~powerful tool which will be able to enhance the update process 
with fine grained analysis of the network.

\vspace*{-12pt}


\Ack
\noindent
The first author has partially been supported by the 
Russian Foundation for Basic Research under grants RFBR 18-07-00669 and 18-29-03100. 
The second author has partially been supported by the Rita Altura Trust Chair in
Computer Sciences; The Lynne and William Frankel Center for Computer
Science.

%\bigskip


The authors thank Prof.\ Shlomi Dolev 
for his valuable input and Prof.\ Yefim Dinitz for his comments.
 
\renewcommand{\bibname}{\protect\rmfamily References}

%\vspace*{-6pt}

\vspace*{-6pt}

{\small\frenchspacing
{\baselineskip=10.35pt
\begin{thebibliography}{99}



\bibitem{rao_sdn_2014}  %1
\Aue{Rao, S.\,K.} 2014. SDN and its use-cases~--- NV and NFV:
A~state-of-the-art survey. NEC Technologies India Ltd. 25~p.

\bibitem{ghaznavi_service_2016}  %2
\Aue{Ghaznavi, M., N.~Shahriar, R.~Ahmed, and R.~Boutaba}. 2016. 
Service function chaining simplified. {arXiv.org}. arXiv:1601.00751.

\bibitem{hansson_logic_1994}  %3
\Aue{Hansson, H., and B.~Jonsson}. 
1994. A~logic for reasoning about time and reliability. 
\textit{Form. Asp. Comput.} 6(5):512--535.

\bibitem{delaet_seamless_2015}  %4
\Aue{Delaet, S., S.~Dolev, D.~Khankin, S.~Tzur-David, and T.~Godinger}. 
2015. Seamless SDN route updates. \textit{IEEE 14th Symposium (International)
on Network Computing and Applications}. IEEE. 120--125.

\bibitem{frenkel_predicting_2017} 
\Aue{Frenkel, S., D.~Khankin, and A.~Kutsyy}. 
2017. Predicting and choosing alternatives of route updates per QoS VNF in SDN. 
\textit{IEEE 16th Symposium (International) on Network Computing and Applications}. 
IEEE. 1--6. 

\bibitem{devi_approach_2015} 
\Aue{Devi, G., and S.~Upadhyaya}. 2015. 
An approach to distributed multi-path QoS routing. 
\textit{Indian J.~Sci. Technol.} 8(20):1--14. 
doi: 10.17485/ijst/2015/v8i20/49253.

\bibitem{egilmez_distributed_2012} 
\Aue{Egilmez, H.\,E., S.~Civanlar, and A.\,M.~Tekalp}. 2012. 
A~distributed QoS routing architecture for scalable video streaming over multi-domain 
OpenFlow networks. \textit{19th IEEE Conference (International) on Image Processing}.
IEEE. 2237--2240.

\bibitem{juttner_lagrange_2001} 
\Aue{Juttner, A., B.~Szviatovski, I.~Mecs, and Z.~Rajko}. 2001. 
Lagrange relaxation based method
for the QoS routing problem. \textit{IEEE Conference on Computer Communications. 
20th Annual Joint Conference of the IEEE Computer and Communications Society
 Proceedings}. IEEE. 2:859--868.

\bibitem{yu_efficient_2013} %9
\Aue{Yu, Z., F.~Ma, J.~Liu, B.~Hu, and Z.~Zhang}. 2013. 
An efficient approximate algorithm for disjoint QoS routing.
\textit{Math. Probl. Eng.} 2013:489149. 9~p. 
doi: 10.1155/2013/489149.

\bibitem{foerster_survey_2016} 
\Aue{Foerster, K.-T., S.~Schmid, and S.~Vissicchio} 2016. 
A~survey of consistent network updates. \mbox{Arxiv.org}. \mbox{arXiv}:\linebreak 1609.02305.

\bibitem{reitblatt_consistent_2011} 
\Aue{Reitblatt, M., N.~Foster, J.~Rexford, and D.~Walker}. 
2011. Consistent updates for software-defined networks: Change you can believe in! 
\textit{10th ACM Workshop on Hot Topics in Networks Proceedings}.
New York, NY: ACM. Art.\ No.\,7. doi: 10.1145/2070562.2070569.

\bibitem{hogan_stochastic_2017} 
\Aue{Hogan, M., and F.~Esposito}. 
2017. Stochastic delay forecasts for edge traffic engineering via Bayesian networks. 
\textit{IEEE 16th Symposium (International) on Network Computing and Applications}. 
IEEE. 1--4.

\bibitem{mcgeer_safe_2012} %15
\Aue{McGeer, R.} 2012. A~safe, efficient Update Protocol for Openflow Networks. 
\textit{1st Workshop on Hot Topics in Software Defined Networks Proceedings}. 
New York, NY: ACM. 12:61--66.
\bibitem{mcgeer_correct_2013} 
\Aue{McGeer, R.} 2013. A~correct, zero-overhead protocol for network updates. 
\textit{2nd ACM SIGCOMM Workshop on Hot Topics in Software Defined Networking
Proceedings}. New York, NY: ACM. 13:161--162.
\bibitem{katta_incremental_2013} 
\Aue{Katta, N.\,P., J.~Rexford, and D.~Walker}. 
2013. Incremental consistent updates. \textit{2nd ACM SIGCOMM Workshop on Hot Topics 
in Software Defined Networking Proceedings}.
New York, NY: ACM. 13:49--54.

\bibitem{dinitz_dependence_2017}  %16
\Aue{Dinitz, Y., S.~Dolev, and D.~Khankin}. 
2017. Dependence graph and master switch for seamless dependent routes 
replacement in SDN. \textit{IEEE 16th Symposium 
(International) on Network Computing and Applications}. IEEE. 1--7.

\bibitem{17-aaa}
\Aue{Amiri, S.\,A., S.~Dudycz, S.~Schmid, and S.~Wiederrecht}.
2016. Congestion-free rerouting of flows
on DAGs. \mbox{ArXiv}.org. arXiv:1611.09296.
% [cs, math], Nov. 2016, arXiv: 1611.09296. [Online]. Available:
%http://arxiv.org/abs/1611.09296

\bibitem{kwiatkowska_prism_2011}  %17
\Aue{Kwiatkowska, M., G.~Norman, and D.~Parker}. 2011. 
PRISM~4.0: Verification of probabilistic real-time systems. 
\textit{Computer aided verification}.
Eds. G.~Gopalakrishnan and S.~Qadeer.
Lecture notes in computer science ser. Springer.
6806:585--591.

\bibitem{noauthor_prism_nodate}  %18
\Aue{Kwiatkowska, M., G.~Norman, and D.~Parker}. 2018. 
{PRISM manual}. Available at:
{\sf http://www.\linebreak prismmodelchecker.org/manual/}
(accessed December~10, 2018).

\bibitem{onf_openflow_2015} %19
{Open Networking Foundation}. 2015. 
OpenFlow Switch Specification Ver~1.5.1. 


\bibitem{wu_review_2015}  %20
\Aue{Wu, Q., and J.-K.~Hao}. 2015. 
A~review on algorithms for maximum clique problems. 
\textit{Eur. J.~Oper. Res.} 242(3):693--709.

\bibitem{kaur_network_2014}  %21
\Aue{Kaur, S., J.~Singh, and N.\,S.~Ghumman}. 2014. 
Network programmability using POX controller. 
\textit{Conference (International) on Communication, Computing and Systems}.
138.

\bibitem{lantz_network_2010}  %22
\Aue{Lantz, B., B.~Heller, and N.~McKeown}. 2010. 
A~network in a~laptop: Rapid prototyping for software-defined networks. 
\textit{9th ACM SIGCOMM Workshop on Hot Topics in Networks Proceedings}. 
New York, NY: ACM.  Art.\ No.\,19. doi: 10.1145/1868447.1868466.
\end{thebibliography} } }

\end{multicols}

\vspace*{-9pt}

\hfill{\small\textit{Received October 9, 2018}}

\vspace*{-22pt}

\Contr

\vspace*{-3pt}

\noindent
\textbf{Frenkel Sergey L.} (b.\ 1951)~--- 
Candidate of Science (PhD) in technology, associate professor, 
senior scientist, Institute of Informatics Problems, Federal Research Center 
``Computer Sciences and Control'' of the Russian Academy of Sciences, 
44-2~Vavilov Str., Moscow 119333, Russian Federation; \mbox{fsergei51@gmail.com}

\vspace*{1pt}

\noindent
\textbf{Khankin D.} (b.\ 1983)~--- MSc, doctorate student, Department of Computer 
Science, Ben-Gurion University of the Negev, Beer-Sheva 84105, Israel; 
\mbox{danielkh@post.bgu.ac.il}

\vspace*{4pt}

\hrule

\vspace*{2pt}

\hrule

\vspace*{-7pt}

%\newpage

%\vspace*{-28pt}

\def\tit{НЕПРЕРЫВНЫЕ ОБНОВЛЕНИЯ МАРШРУТА В~SDN С~ИСПОЛЬЗОВАНИЕМ ПРОВЕРКИ СООТВЕТСТВИЯ 
КАЧЕСТВУ~ОБСЛУЖИВАНИЯ$^*$\\[-7pt]}

\def\titkol{Непрерывные обновления маршрута в~SDN с~использованием проверки соответствия 
качеству обслуживания}

\def\aut{С.\,Л.~Френкель$^1$, Д.~Ханкин$^2$\\[-7pt]}

\def\autkol{С.\,Л.~Френкель, Д.~Ханкин}

{\renewcommand{\thefootnote}{\fnsymbol{footnote}} \footnotetext[1]
{Работа была частично поддержана РФФИ (гранты 18-07~00669 и~18-29-03100), 
а~также Rita Altura Trust Chair in
Computer Sciences; The Lynne and William Frankel Center for Computer
Science.}}



\titel{\tit}{\aut}{\autkol}{\titkol}

\vspace*{-22pt}

\noindent
$^1$Институт проблем информатики Федерального исследовательского центра 
<<Информатика и~управление>>\linebreak
$\hphantom{^1}$Российской академии наук
%, fsergei51@gmail.com 

\noindent
$^2$Университет им.\ Бен-Гуриона в Негеве, Беэр-Шева, Израиль
%, danielkh@post.bgu.ac.il 

\vspace*{1pt}

\def\leftfootline{\small{\textbf{\thepage}
\hfill ИНФОРМАТИКА И ЕЁ ПРИМЕНЕНИЯ\ \ \ том\ 12\ \ \ выпуск\ 4\ \ \ 2018}
}%
 \def\rightfootline{\small{ИНФОРМАТИКА И ЕЁ ПРИМЕНЕНИЯ\ \ \ том\ 12\ \ \ выпуск\ 4\ \ \ 2018
\hfill \textbf{\thepage}}}

\vspace*{-1pt}


 
\Abst{В программно-определяемой сети (SDN~--- software-defined networking) 
уровень управ\-ле\-ния 
и~уровень данных разделены. Это обеспечивает высокую гибкость эксплуатации, 
предоставляя абстракции для управления сетью приложений 
и~возможность непосредственного программирования маршрутов.
Однако из-за изменений топологии, процедуры обслуживания или происходящих 
сбоев иногда необходима реконфигурация и~обновление сети. 
В~предлагаемом сценарии рассматривается текущий маршрут~$C$
и~набор возможных новых маршрутов~~$\{N_i\}$, где для замены текущего 
маршрута требуется 
один из\linebreak\vspace*{-12pt}}

\Abstend{новых маршрутов. Существует вероятность того, что новый маршрут~$N_i$ 
окажется длиннее некоторого другого нового маршрута~$N_j$, но при этом~$N_i$ 
будет более надежным и~он будет обновляться быстрее или работать лучше 
после обновления с~точки зрения требований качества обслуживания (QoS~---
quality of service). Принимая 
во внимание случайный характер функционирования сети, авторы дополнили недавно 
предложенный алгоритм обновления маршрута Delaet с~соавт.\ методом оценки соблюдения 
требований QoS во время непрерывного обновления маршрута, основанным на 
использовании цепей Маркова. При этом, во-пер\-вых, предлагается расширить 
алгоритм передачи пакетов по выбранному маршруту, сравнивая процесс обновления 
для возможных альтернатив маршрута. Во-вто\-рых, предлагается несколько 
способов выбора комбинаций предпочтительных отрезков путей новых маршрутов, 
что приводит к оптимальному в~смысле соответствия QoS маршруту.}


\KW{программно-определяемые сети; цепи Маркова; качество обслуживания}

\DOI{10.14357/19922264180408}



%\vspace*{-3pt}


 \begin{multicols}{2}

\renewcommand{\bibname}{\protect\rmfamily Литература}
%\renewcommand{\bibname}{\large\protect\rm References}

{\small\frenchspacing
{\baselineskip=10.5pt
\begin{thebibliography}{99}
%\vspace*{-3pt}


\bibitem{2-fr-1}
\Au{Rao S.\,K.} SDN and its use-cases~--- NV and NFV: A~state-of-the-art survey.~--- 
NEC Technologies India Ltd., 2014. 25~p.
\bibitem{3-fr-1}
\Au{Ghaznavi M., Shahriar~N., Ahmed~R., Boutaba~R.} 
Service function chaining simplified~// Arxiv.org, 2016. \mbox{arXiv}:1601.00751cs.
\bibitem{4-fr-1}
\Au{Hansson H., Jonsson~B.} A~logic for reasoning about time and reliability~// 
Form. Asp. Comput., 1994. Vol.~6. No.\,5. P.~512--535.

\bibitem{1-fr-1} %4
\Au{Delaet S., Dolev~S., Khankin~D., Tzur-David~S., Godinger~T.}
Seamless SDN route updates~// IEEE 14th Symposium (International)
 on Network Computing and Applications.~--- IEEE, 2015. P.~120--125.
 
 
\bibitem{5-fr-1}
\Au{Frenkel S., Khankin D., Kutsyy~A.} Predicting and choosing alternatives 
of route updates per QoS VNF in SDN~// IEEE 16th Symposium (International)
on Network Computing and Applications.~--- IEEE, 2017. P.~1--6.
\bibitem{6-fr-1}
\Au{Devi G., Upadhyaya~S.} An approach to distributed multi-path QoS routing~// 
Indian J.~Sci. Technol., 2015. Vol.~8. Iss.~20. P.~1--14. 
doi: 10.17485/ijst/2015/v8i20/49253.
\bibitem{7-fr-1}
\Au{Egilmez H.\,E., Civanlar S., Tekalp~A.\,M.} 
A~distributed QoS routing architecture for scalable video streaming over multi-domain 
OpenFlow networks~// 19th IEEE Conference (International)
on Image Processing.~--- IEEE, 2012. P.~2237--2240.
\bibitem{8-fr-1}
\Au{Juttner A., Szviatovski B., Mecs~I., Rajko~Z.}
Lagrange relaxation based method for the QoS routing problem~// 
IEEE INFOCOM 2001 Conference on Computer Communications. 20th 
Annual Joint Conference of the IEEE Computer and Communications Society
Proceedings.~--- IEEE, 2001. Vol.~2. P.~859--868.
\bibitem{9-fr-1}
\Au{Yu Z., Ma F., Liu~J., Hu~B., Zhang~Z.}
An efficient approximate algorithm for disjoint QoS routing~// 
Math. Probl. Eng., 2013. Vol.~2013. Art.\ No.\,489149. 9~p. 
doi: 10.1155/2013/489149.
\bibitem{10-fr-1}
\Au{Foerster K.-T., Schmid S., Vissicchio~S.}
A~survey of consistent network updates~// Arxiv.org, 2016. arXiv:1609.02305.
\bibitem{11-fr-1}
\Au{Reitblatt M., Foster N., Rexford J., Walker~D.} 
Consistent updates for software-defined networks: Change you can believe in!~// 
10th ACM Workshop on Hot Topics in Networks Proceedings.~--- New York, NY, USA: ACM, 
2011. Art.\ No.\,7. doi: 10.1145/2070562.2070569.
\bibitem{12-fr-1}
\Au{Hogan M., Esposito F.} Stochastic delay forecasts for edge traffic engineering 
via Bayesian Networks~// IEEE 16th Symposium (International)
on Network Computing and Applications.~--- IEEE, 2017. P.~1--4.
\bibitem{13-fr-1}
\Au{McGeer R.} A~safe, efficient Update Protocol for Openflow Networks~// 
1st Workshop on Hot Topics in Software Defined Networks Proceedings.~--- 
New York, NY, USA: ACM, 2012. Vol.~12. P.~61--66.
\bibitem{14-fr-1}
\Au{McGeer R.} 2013. A~correct, zero-overhead protocol for network updates~// 
2nd Workshop on Hot Topics in Software Defined Networking Proceedings.~--- 
New York, NY, USA: ACM, 2013. Vol.~13. P.~161--162.
\bibitem{15-fr-1}
\Au{Katta N.\,P., Rexford J., Walker~D.} Incremental consistent updates~// 
2nd Workshop on Hot Topics in Software Defined Networking Proceedings.~--- 
New York, NY, USA: ACM, 2013. Vol.~13. P.~49--54.
\bibitem{16-fr-1}
\Au{Dinitz Y., Dolev S., Khankin~D.}
 Dependence graph and master switch for seamless dependent 
 routes replacement in SDN~// IEEE 16th Symposium 
 (International) on Network Computing and Applications.~--- IEEE, 2017. P.~1--7.
 \bibitem{17-aaa-1}
\Au{Amiri~S.\,A., Dudycz~S., Schmid~S., Wiederrecht~S}.
 Congestion-free rerouting of flows
on DAGs~// ArXiv.org, 2016. arXiv:1611.09296.
% [cs, math], Nov. 2016, arXiv: 1611.09296. [Online]. Available:
%http://arxiv.org/abs/1611.09296

\bibitem{17-fr-1}
\Au{Kwiatkowska M., Norman~G., Parker~D.}
 PRISM~4.0: Verification of probabilistic real-time systems~//
 Computer aided verification~/
 Eds. G.~Gopalakrishnan, S.~Qadeer.~---
Lecture notes in computer science ser.~--- Springer, 2011. 
 Vol.~6806. P.~585--591.
\bibitem{18-fr-1}
\Au{Kwiatkowska M., Norman G., Parker~D.}
 PRISM manual, 2018. 
{\sf http://www.prismmodelchecker.org/manual}.
\bibitem{19-fr-1}
Open Networking Foundation. OpenFlow Switch Specification Ver~1.5.1, 2015. 

\bibitem{21-fr-1}
\Au{Wu Q., Hao J.-K.} A~review on algorithms for maximum clique problems~// 
Eur. J.~Oper. Res., 2015. Vol.~242. No.\,3. P.~693--709.

\bibitem{20-fr-1}
\Au{Kaur S., Singh J., Ghumman~N.\,S.}
 Network programmability using POX controller~// Conference
 (International) on Communication, Computing and Systems, 2014. P.~138.
\bibitem{22-fr-1}
\Au{Lantz B., Heller B., McKeown~N.} 
A~network in a~laptop: Rapid prototyping for software-defined networks~// 
9th ACM SIGCOMM Workshop on Hot Topics in Networks Proceedings.~--- 
New York, NY, USA: ACM, 2010. Art.\ No.\,19. doi: 10.1145/1868447.1868466.
\end{thebibliography}
} }

\end{multicols}

 \label{end\stat}

 \vspace*{-9pt}

\hfill{\small\textit{Поступила в~редакцию 09.10.2018}}


%\renewcommand{\bibname}{\protect\rm Литература}
\renewcommand{\figurename}{\protect\bf Рис.}
\renewcommand{\tablename}{\protect\bf Таблица} %4
\def\stat{gorshenin}

\def\tit{ЗАШУМЛЕНИЕ ДАННЫХ КОНЕЧНЫМИ СМЕСЯМИ НОРМАЛЬНЫХ 
И~ГАММА-РАСПРЕДЕЛЕНИЙ\\ С~ПРИМЕНЕНИЕМ К~ЗАДАЧЕ ОКРУГЛЕНИЯ НАБЛЮДЕНИЙ$^*$}

\def\titkol{Зашумление данных конечными смесями нормальных 
и~гамма-распределений с~применением к~задаче округления} % наблюдений}

\def\aut{А.\,К.~Горшенин$^1$}

\def\autkol{А.\,К.~Горшенин}

\titel{\tit}{\aut}{\autkol}{\titkol}

\index{Горшенин А.\,К.}
\index{Gorshenin A.\,K.}


{\renewcommand{\thefootnote}{\fnsymbol{footnote}} \footnotetext[1]
{Работа выполнена при поддержке РНФ (проект 18-71-00156).}}


\renewcommand{\thefootnote}{\arabic{footnote}}
\footnotetext[1]{Институт проблем информатики Федерального исследовательского центра 
<<Информатика и~управление>> Российской академии наук, \mbox{agorshenin@frccsc.ru}}

\vspace*{-12pt}




\Abst{Во многих реальных задачах проводится статистический анализ данных, 
содержащих дополнительные ошибки измерения, в~том числе в~виде округления, 
что в~ряде ситуаций может приводить к~достаточно существенным искажениям. 
В~настоящей статье для одной из возможных моделей округления получены оценки 
для неизвестного математического ожидания наблюдений в~предположении, что 
исходные данные дополнительно зашумлены с~по\-мощью случайных величин, 
име\-ющих распределения типа конечных смесей нормальных и~гам\-ма-за\-ко\-нов. 
Построены доверительные интервалы для неизвестного математического ожидания 
с~использованием уточненной оценки для дисперсии целой части случайной величины. 
Обсуждается алгоритм определения значения параметра для искусственного шума, 
добавление которого к~исходным данным способствует повышению качества работы 
метода скользящего разделения смесей.}

\KW{зашумленные данные; округленные наблюдения; конечные смеси нормальных 
распределений; конечные смеси гам\-ма-рас\-пре\-де\-ле\-ний; доверительные интервалы;  
метод скользящего разделения смесей}

\DOI{10.14357/19922264180304}
  
\vspace*{-4pt}


\vskip 10pt plus 9pt minus 6pt

\thispagestyle{headings}

\begin{multicols}{2}

\label{st\stat}


\section{Введение}

Во многих реальных задачах данные, являющиеся непрерывными по своей сути, 
регистрируются с~помощью инструментов, вносящих дополнительные ошибки 
измерения, в~том чис\-ле в~виде округления. Таким образом, статистический 
анализ проводится не для исходных, а для преобразованных некоторым 
случайным образом наблюдений, что в~ряде ситуаций может приводить к~достаточно
 существенным искажениям.

Для преодоления данной проблемы развивались различные подходы, в~том числе 
на основе смешанных моделей (см., например, статью~\cite{Wright2003}, в~которой 
различные компоненты  используются для пред\-став\-ле\-ния уровней округления). 
В~работе~\cite{Bai2009} приводятся результаты для моделей авторегрессии и~скользящего 
среднего для округленных данных, а~в~статье~\cite{Zhang2010} эти результаты 
развиваются и~исследуются их асимптотические свойства. 
В~статье~\cite{Zhao2012} исследован метод оценивания па\-ра\-мет\-ров конечных смесей 
вероятностных распределений (в~том чис\-ле, и~многомерных) 
на основе использования EM (expectation-maximization) 
алгоритма~\cite{Korolev2011-i} с~\mbox{целью} получения состоятельных 
и~асимптотически нормальных оценок.

В настоящей статье развиваются результаты для моделей округления, 
описанных в~работах~\cite{Ushakov2015,Ushakov2017a,Ushakov2017b}. 
В~их рамках будут получены оценки для неизве\-ст\-ного математического ожидания 
наблюдений в~предположении, что исходные данные зашумлены с~по\-мощью случайных 
величин, имеющих распределения типа конечных смесей нормальных и~гам\-ма-за\-ко\-нов. 
Это позволяет учесть большее количество случайных факторов, влия\-ющих на величину 
<<дополнительной>> ошибки. Также будут построены доверительные интервалы для 
неизвестного математического ожидания. Выражения для гам\-ма-рас\-пре\-де\-ле\-ний 
получены впервые. Также обсуждается алгоритм определения значения па\-ра\-мет\-ра для 
искусственного шума, добавление которого к~исходным данным способствует 
повышению качества работы метода скользящего разделения смесей~\cite{Gorshenin2016}.

\vspace*{-12pt}

\section{Предположения и~базовые отношения}

Для сокращения формулировок теорем в~сле\-ду\-ющих разделах сделаем ряд 
предположений, на которые будем ссылаться в~дальнейшем. Итак, пусть:
\begin{itemize}
\item[(A)] $X_1,X_2,\ldots$~--- независимые одинаково распределенные 
случайные величины с~неизвестным математическим ожиданием ${\sf E}_X\hm<+\infty$;
\item[(B)] $\varepsilon_1,\varepsilon_2,\ldots$~--- независимые одинаково 
распределенные случайные величины с~математическим ожиданием 
${\sf E}_\varepsilon\hm<+\infty$; %\label{B}
\item[(C)] $X_1,X_2,\ldots$ и~$\varepsilon_1,\varepsilon_2,\ldots$ 
являются независимыми;
\item[(D)] $Y_j=\left[X_j+\varepsilon_j+1/2\right]$ для всех $j\hm=1,2,\ldots$ 
представляют собой округление значения суммы случайных величин $X_j\hm+\varepsilon_j$ 
до ближайшего целого сверху (при этом запись~$[\cdot]$ соответствует целой 
части выражения).
\end{itemize}

В рамках данных предположений в~статье будут рассмотрены вопросы качества 
приближения неизвестного математического ожидания~${\sf E}_X$ для исходных данных 
в~ситуации, когда наблюдения для анализа получены с~аддитивной ошибкой c известными 
распределениями (см.\ предположение~(B)) и~дополнительно округляются до 
ближайшего целого (см.\ предположение~(D)).

Заметим, что в~силу усиленного закона больших чисел справедливы следующие выражения:
\begin{multline}
\fr{1}{n}\sum\limits_{j=1}^n Y_j\xrightarrow[n\to\infty]{\text{п.н.}}
{\sf E}_Y\equiv\mathbb{E}\left[X_1+\varepsilon_1+\fr{1}{2}\right]={}\\
{}=\mathbb{E}\left(X_j+\varepsilon_j+\fr{1}{2}\right)-\mathbb{E}
\left\{X_j+\varepsilon_j+\fr{1}{2}\right\}={}\\
{}={\sf E}_X+{\sf E}_\varepsilon+\fr{1}{2}-\mathbb{E}\left\{X_j+\varepsilon_j+\fr{1}{2}\right\}. 
\label{Law}
\end{multline}

Запись $\{\cdot\}$ в~формуле~\eqref{Law} соответствует дробной 
части выражения, а~п.н.\ обозначает сходимость в~смысле почти наверное.

Для доказательства результатов в~дальнейшем потребуется следующее 
представления для дробной части  абсолютно непрерывной случайной величины~$Z$ 
с~абсолютно  интегрируемой характеристической функцией~$\varphi_Z(t)$
 (см., например, Лемму~4 в~работе~\cite{Ushakov2017b}):
\begin{equation}
\label{Fract}
\mathbb{E}\{Z\}=\fr{1}{2}-\sum\limits_{n=1}^\infty 
\fr{\mathrm{Im}\left (\varphi_Z(2\pi n)\right)}{\pi n}\,.
\end{equation}

Через $\mathrm{Im}\,(\cdot)$ в~формуле~\eqref{Fract} обозначена мнимая часть 
соответствующей функции.

При построении доверительных интервалов в~дальнейшем будет 
использована следующая оценка, справедливая для любой случайной величины~$Z$:
\begin{equation}
\mathbb{D}[Z]\leqslant \left(\sqrt{\mathbb{D} Z}+\fr{1}{2}\right)^2.
\label{Var}
\end{equation}
Она может быть проверена непосредственно с~учетом представления 
$\mathbb{D} [Z]\hm=\mathbb{D}\left(Z\hm-\{Z\}\right)$, неравенства 
Ко\-ши--Бу\-ня\-ков\-ско\-го для ковариации и~соотношения 
 $\mathbb{D}\{Z\}\hm\leqslant 1/4$, справедливого для любой случайной величины~$Z$ 
 (см., например, статью~\cite{Ushakov2017b}). Отметим, что данная оценка 
 является более точной по сравнению с~использованным для аналогичных 
 целей в~работе~\cite{Ushakov2017b} соотношением 
 $\mathbb{D} [Z]\hm\leqslant 2\mathbb{D} Z\hm+1/2$. Действительно,
\begin{equation*}
2\mathbb{D} Z+\fr{1}{2}-\left(\sqrt{\mathbb{D} Z}+\fr{1}{2}\right)^2=
\left(\sqrt{\mathbb{D} Z}-\fr{1}{2}\right)^2\geqslant0\,,
\end{equation*}
причем для всех $\sqrt{\mathbb{D} Z}\hm\neq {1}/{2}$ 
данное неравенство является строгим.

\section{Конечные смеси нормальных законов}

Для случайной величины~$X$, имеющей распределение типа 
конечной смеси нормальных законов~\cite{Korolev2011-i} с~параметрами 
${\bf a}\hm=(a_1,\ldots, a_k)$, $a_j\hm\in \mathbb{R}$, 
$\boldsymbol{\sigma}\hm=(\sigma_1,\ldots, \sigma_k)$, 
$\sigma_j\hm>0$, ${\bf p}\hm=(p_1,\ldots, p_k)$, $p_j\hm\geqslant 0$, 
$\sum\nolimits_{j=1}^{k}p_j\hm=1$, плот\-ность которого задается выражением
\begin{equation}
f_X(x)=\sum\limits_{j=1}^{k}\fr{p_j}{\sigma_j\sqrt{2\pi}}\,e^{-(x-a_j)^2/(2\sigma_j^2)}\,,
\label{FinNormMixt}
\end{equation}
характеристическая функция имеет вид:
\begin{equation}
\varphi_X(t)=\int\limits_{-\infty}^{+\infty}\!\!e^{itx} f_X(x)\, dx = 
\sum\limits_{j=1}^{k}p_j e^{ita_j-\sigma_j^2 t^2/2}.
\label{ChiFinNormMixt}
\end{equation}

Абсолютная интегрируемость  $\varphi_X(t)$ вытекает из свойств 
характеристической функции нормального распределения. 
Заметим, что в~точке $t\hm=2\pi n$ выражение~\eqref{ChiFinNormMixt} принимает 
сле\-ду\-ющий вид:
\begin{equation}
\label{ChiFinNormMixt2npi}
\varphi_X(2\pi n)= \sum\limits_{j=1}^{k}p_j e^{-2\pi^2 \sigma_j^2 n^2}\,.
\end{equation}

Рассмотрим вопрос точности оценивания неизвестного математического ожидания~${\sf E}_X$ 
при до\-бав\-ле\-нии зашумления.

\smallskip

\noindent
\textbf{Теорема~1.}\ 
\textit{Пусть выполнены предположения}~(A)--(D), 
\textit{причем случайные величины~$\varepsilon_j$, $j\hm=1,2,\ldots$, 
имеют распределение типа конечной $k$-ком\-по\-нент\-ной смеси нормальных законов 
вида}~\eqref{FinNormMixt} \textit{с~па\-ра\-мет\-ра\-ми~${\bf a}$, $\boldsymbol{\sigma}$ 
и~${\bf p}$. Тогда}
\begin{equation}
\label{Th1Eq}
\left\lvert {\sf E}_Y-{\sf E}_X\right\rvert \leqslant 
A+\fr{1}{\pi}\left(1+\fr{1}{4\pi^2\sigma^2}\right)e^{-2\pi^2\sigma^2}\,, 
\end{equation}
\textit{где} $A=\max(|a_1|,\ldots,|a_k|)$, $\sigma\hm=\min(\sigma_1,\ldots,\sigma_k)$.

\smallskip


\noindent
Д\,о\,к\,а\,з\,а\,т\,е\,л\,ь\,с\,т\,в\,о\,.\ \
С~учетом пред\-став\-ле\-ний~\eqref{Law},~\eqref{Fract} и~\eqref{ChiFinNormMixt2npi}, 
ограниченности модуля характеристической функции, а~также не\-за\-ви\-си\-мости 
случайных величин~$X_j$ и~$\varepsilon_j$ имеем:
\begin{multline*}
\left\lvert {\sf E}_Y-{\sf E}_X\right\rvert =
\left\lvert {\sf E}_\varepsilon+\fr{1}{2}-\mathbb{E}\left\{X_j+
\varepsilon_j+\fr{1}{2}\right\}\right\rvert={}\\
{}=\left\lvert {\sf E}_\varepsilon+\sum\limits_{n=1}^\infty
\fr{\mathrm{Im} \left(\varphi_{X_j}(2\pi n)\varphi_{\varepsilon_j}(2\pi n)
\varphi_{1/2}(2\pi n)\right)}{\pi n}\right\rvert={}\\
=\left\lvert 
\vphantom{\fr{(-1)^n\sum\nolimits_{j=1}^{k}p_j e^{-2\pi^2 \sigma_j^2 n^2} 
\mathrm{Im} \left(\varphi_{X_j}(2\pi n)\right)}{\pi n}}
{\sf E}_\varepsilon+{}\right.\\
\left.{}+\sum\limits_{n=1}^\infty
\fr{\mathrm{Im} \left(\varphi_{X_j}(2\pi n) 
\sum\nolimits_{j=1}^{k}p_j e^{-2\pi^2 \sigma_j^2 n^2} 
e^{\pi n}\right)}{\pi n}\right\rvert={}\\
{}=\left\lvert 
\vphantom{\fr{(-1)^n\sum\nolimits_{j=1}^{k}p_j e^{-2\pi^2 \sigma_j^2 n^2} 
\mathrm{Im} \left(\varphi_{X_j}(2\pi n)\right)}{\pi n}}
{\sf E}_\varepsilon+{}\right.\\
\left.{}+\sum\limits_{n=1}^\infty
\fr{(-1)^n\sum\nolimits_{j=1}^{k}p_j e^{-2\pi^2 \sigma_j^2 n^2} 
\mathrm{Im} \left(\varphi_{X_j}(2\pi n)\right)}{\pi n}\right\rvert\leqslant{}\\
{}\leqslant \left\lvert {\sf E}_\varepsilon\right\rvert+\left\lvert\
\sum\limits_{j=1}^{k}p_j\sum\limits_{n=1}^\infty 
\fr{1}{\pi n} e^{-2\pi^2 \sigma_j^2 n^2}\right\rvert\leqslant {}\\
\\
{}\leqslant
\max\left(|a_1|,\ldots,|a_k|\right)+{}\\
{}+\sum\limits_{j=1}^{k} 
\fr{p_j}{\pi} \left(\!1+\fr{1}{4\pi^2\sigma_j^2}\!\right)\!e^{-2\pi^2\sigma_j^2}\leqslant{}\\
{}\leqslant
A+\fr{1}\pi\left(1+\fr{1}{4\pi^2\sigma^2}\right)e^{-2\pi^2\sigma^2}\,.
\end{multline*}

Справедливость использованной оценки 
\begin{equation*}
\sum\limits_{n=1}^\infty
\fr{e^{-2\pi^2 \sigma_j^2 n^2}}{n}\leqslant 
\left(1+\fr{1}{4\pi^2\sigma_j^2}\right)e^{-2\pi^2\sigma_j^2}
\end{equation*}
может быть проверена непосредственно (например, см.\ доказательство Теоремы~6 
в~статье~\cite{Ushakov2017b}).~\hfill$\square$

\smallskip

\noindent
\textbf{Замечание~1.}
В~случае, если зашумление производится нормально распределенными случайными 
величинами c нулевыми средними (т.\,е.\ в~формуле~\eqref{Th1Eq} необходимо считать 
$A\hm=0$, $k\hm=1$), то Тео\-ре\-ма~1 совпадает с~результатом, 
полученным в~работе~\cite{Ushakov2017b}.


\smallskip

Рассмотрим вопросы построения доверительного интервала для неизвестного 
математического ожидания~${\sf E}_X$ в~предположении, что случайные величины~$X_j$ не 
содержат ошибок измерения, а~все погрешности учтены исключительно в~за\-шум\-ля\-ющих 
элементах~$\varepsilon_j$.

\smallskip

\noindent
\textbf{Теорема~2.}\ 
\textit{Пусть выполнены предположения}~(A)--(D), 
\textit{причем случайные величины~$\varepsilon_j$, $j\hm=1,2,\ldots$, имеют 
распределение типа конечной $k$-ком\-по\-нент\-ной смеси нормальных законов 
вида}~\eqref{FinNormMixt} \textit{с~параметрами~${\bf a}$, $\boldsymbol{\sigma}$ 
и~${\bf p}$, а~случайные величины} $X_j\stackrel{\text{п.н.}}{=}{\sf E}_X$, $j\hm=1,2,\ldots$ 
\textit{Тогда доверительный интервал для~${\sf E}_X$ при условии $0\hm<\alpha\hm<1$ имеет вид}:
\begin{equation} 
\label{Th2Eq}
\hat{{\sf E}}_X - f({\bf a},\boldsymbol{\sigma},\alpha,n) 
\leqslant {\sf E}_X \leqslant  \hat{{\sf E}}_X + f({\bf a},\boldsymbol{\sigma},\alpha,n),
\end{equation}
\textit{где}

\vspace*{-2pt}

\noindent
\begin{align}
\hat{{\sf E}}_X&=\fr{1}{n} \sum\limits_{j=1}^{n} Y_j\,; \label{Th2hatE}\\
f({\bf a},\boldsymbol{\sigma},\alpha,n)&=
\fr{z_{1-{\alpha}/2}}{\sqrt{n}} \left(\sqrt{A^2+\Sigma^2}+\fr{1}{2}\right) +{}\notag\\
&{}+A+\fr{1}\pi\left(1+\fr{1}{4\pi^2\sigma^2}\right)e^{-2\pi^2\sigma^2}\,;
  \label{Th2f}
\end{align}
\textit{$z_{1-{\alpha}/2}$~--- $\left(1-{\alpha}/2\right)$-кван\-тиль 
стандартного нормального распределения; $A\hm=\max(|a_1|,\ldots,|a_k|)$; 
$\Sigma\hm=\max(\sigma_1,\ldots,\sigma_k)$; $\sigma\hm=\min(\sigma_1,\ldots,\sigma_k)$}. 


\smallskip

\noindent
\noindent
Д\,о\,к\,а\,з\,а\,т\,е\,л\,ь\,с\,т\,в\,о\,.\ \
Из центральной предельной тео\-ре\-мы с~учетом условия~(A) следует, 
что величина~$\hat{{\sf E}}_X$~\eqref{Th2hatE} асимптотически нормальна с~математическим 
ожиданием 
\begin{equation}
{\sf E}_Y\equiv \mathbb{E}\left[{\sf E}_X+\varepsilon_1+\fr{1}{2}\right] \label{EY}
\end{equation}
и дисперсией
\begin{equation}
\fr{1}{n} {\sf D}_Y\equiv \fr{1}{n}\mathbb{D}\left[{\sf E}_X+\varepsilon_1+
\fr{1}{2}\right]. \label{DY}
\end{equation}

Воспользовавшись оценкой~\eqref{Var}, получим:

\vspace*{-2pt}

\noindent
\begin{multline*}
{\sf D}_Y \leqslant  \left(\sqrt{\mathbb{D} \left({\sf E}_X+\varepsilon_1+\fr{1}{2}\right)}+
\fr{1}{2}\right)^2={}\\
{}=
\left(\sqrt{\mathbb{D}\varepsilon_1}+\fr{1}{2}\right)^2= {}\\
{}= \left(\sqrt{\sum\limits_{j=1}^{k}p_j\left(\left(a_j-\sum\limits_{t=1}^{k}
p_t a_t\right)^2+\sigma_j^2\right)}+\fr{1}{2}\right)^2\leqslant {}\\ 
{}\leqslant \left(\sqrt{A^2+\Sigma^2}+\fr{1}{2}\right)^2\,.
\end{multline*}
Тогда доверительный интервал уровня $1\hm-\alpha$ для математического ожидания~${\sf E}_Y$ 
имеет вид:
\begin{equation*}
\mathbb{P}\left(\left\lvert \hat{{\sf E}}_X-{\sf E}_Y\right\rvert \leqslant 
\fr{z_{1-{\alpha}/2}}{\sqrt{n}} 
\left(\sqrt{A^2+\Sigma^2}+\fr{1}{2}\right)\right)\geqslant 1-\alpha\,.
\end{equation*}

\begin{table*}[b]\small
\begin{center}

\begin{tabular}{|c|c|c|c|c|c|c|c|}
\multicolumn{7}{p{100mm}}{Численные решения уравнений~\eqref{f1} и~\eqref{f2} относительно 
параметра~$\sigma$ для некоторых значений~$n$ и~$\alpha$}\\
\multicolumn{7}{c}{\ }\\[-6pt]
\hline
\multicolumn{1}{|c|}{Размер}  & \multicolumn{2}{c|}{Уровень $\alpha=0{,}1$}& 
\multicolumn{2}{c|}{Уровень $\alpha=0{,}05$}& 
\multicolumn{2}{c|}{Уровень $\alpha=0{,}01$}\\
\cline{2-7}
\multicolumn{1}{|c|}{выборки $n$}&$\sigma_1$&$\sigma_2$&$\sigma_1$&$\sigma_2$&$\sigma_1$&$\sigma_2$\\
\hline
$\hphantom{000}100$&$0{,}4302$&$0{,}435$&$0{,}419$&$0{,}425$&$0{,}4002$&$0{,}408$\\
%\hline
$\hphantom{000}200$&$0{,}452$&$0{,}455$ &$0{,}441$&$0{,}445$&$0{,}424$&$0{,}429$\\
%\hline
$\hphantom{00}1000$&$0{,}499$&$0{,}499$ &$0{,}489$&$0{,}489$&$0{,}473$&$0{,}475$\\
%\hline
$\hphantom{0}10000$&$0{,}558$&$0{,}556$ &$0{,}549$&$0{,}547$&$0{,}536$&$0{,}534$\\
%\hline
$100000$&$0{,}611$&$0{,}607$ &$0{,}603$&$0{,}599$&$0{,}591$&$0{,}588$\\
\hline
\end{tabular}
\end{center}
\end{table*}


\noindent
Откуда следует справедливость соотношения~\eqref{Th2Eq} c~уче\-том 
очевидного неравенства

\pagebreak

\noindent
\begin{equation*}
\left\lvert \hat{{\sf E}}_X-{\sf E}_X\right\rvert \leqslant 
\left\lvert \hat{{\sf E}}_X-{\sf E}_Y\right\rvert +\left\lvert {\sf E}_Y-{\sf E}_X\right\rvert 
\end{equation*}
и оценки~\eqref{Th1Eq} из Теоремы~1.~\hfill$\square$

\smallskip

\noindent
\textbf{Замечание~2.}
В~работе~\cite{Gorshenin2016} было продемонстрировано повышение точ\-ности 
работы метода скользящего разделения конечных нормальных смесей за счет 
введения дополнительной компоненты, имеющей нормальное 
распределение $\mathcal{N}(0,\sigma^2)$ с~математическим ожиданием, равным~$0$, 
и~стандартным отклонением~$\sigma$. При этом была отмечена сложность выбора 
параметра~$\sigma$ для сохранения структуры выборки, близкой к~исходной. 
Результат Теоремы~2 может быть использован с~данной целью, если положить $k\hm=1$, 
$a_j\hm=0$ для всех $j\hm=1,2,\ldots$ и~выбирать величину~$\sigma$ как 
минимизирующую длину доверительного интервала~\eqref{Th2Eq}. Для 
этого необходимо найти производную функции $f(0,\sigma,\alpha,n)$~\eqref{Th2f} 
и~численно решить уравнение
\begin{multline}
f_\sigma'(0,\sigma,\alpha,n)\equiv \fr{z_{1-{\alpha}/2}}{\sqrt{n}} - {}\\
{}-
e^{-2\pi^2\sigma^2}\left(4\pi\sigma+\fr{1}{2\pi^3\sigma^3}+
\fr{1}{\pi\sigma}\right)=0
\label{f1}
\end{multline}
относительно неизвестного параметра~$\sigma$ при выбранных значениях величин~$n$ 
и~$\alpha$. В~качестве альтернативы можно использовать вид доверительного интервала 
из статьи~\cite{Ushakov2017b}, полученный с~помощью неравенства $\mathbb{D} [Z]
\hm\leqslant 2\mathbb{D} Z\hm+{1}/{2}$, и~искать решение уравнения вида:
\begin{multline}
\hspace*{-2.90578pt}\fr{2\sigma z_{1-{\alpha}/2}}{\sqrt{n (2\sigma^2+{1}/{2})}} -
 e^{-2\pi^2\sigma^2}\left(4\pi\sigma+\fr{1}{2\pi^3\sigma^3}+
 \fr{1}{\pi\sigma}\right)={}\\
 {}=0\,.\label{f2}
\end{multline}

Примеры найденных значений~$\sigma$ для типичных размеров выборок в~методе 
скользящего разделения смесей (учитываются как возможная ширина окна, 
так и~общее количество наблюдений в~анализируемом ряде) приведены в~таблице 
(использован метод оптимизации \verb"Trust-Region Dogleg" пакета \verb"MATLAB" 
c~настройками по умолчанию), в~которой через~$\sigma_1$ обозначено приближенное  
решение уравнения~\eqref{f1}, a~через $\sigma_2$~--- уравнения~\eqref{f2}.


Проверка практической эффективности данного подхода в~качестве 
критерия выбора параметров зашумляющего распределения для повышения 
точности работы метода скользящего разделения смесей может быть отмечена 
как задача для дальнейших исследований.


\section{Конечные смеси гамма-распределений}

Для случайной величины~$X$, имеющей распределение типа конечной смеси 
гам\-ма-рас\-пре\-де\-ле\-ний с~параметрами ${\bf r}\hm=(r_1,\ldots, r_k)$,
 $r_j\hm>0$, $\boldsymbol{\lambda}\hm=(\lambda_1,\ldots, \lambda_k)$, $\lambda_j\hm>0$, 
 ${\bf p}\hm=(p_1,\ldots, p_k)$, $p_j\hm\geqslant 0$, $\sum\nolimits_{j=1}^{k}p_j\hm=1$, 
 плот\-ность которого задается выражением
\begin{equation}
f_X(x)=\sum\limits_{j=1}^{k}p_j\fr{\lambda_j^{r_j} e^{-\lambda_j x}}
{\Gamma(r_j)}\,x^{r_j-1}\,,
\label{FinGammaMixt}
\end{equation}
характеристическая функция имеет следующий вид:
%характеристическая функция задается следующим выражением:
\begin{equation}
\varphi_X(t)=\!\int\limits_{-\infty}^{+\infty}\!\!\!e^{itx} f_X(x)\, dx = \!
\sum\limits_{j=1}^{k}p_j \left(\!1-\fr{it}{\lambda_j}\right)^{-r_j}\!.\!
\label{ChiFinGammaMixt}
\end{equation}

Отметим, что подобные модели зашумления разумно использовать в~случае, 
если известно, что данные сосредоточены на положительной полуоси, например 
при анализе различных информационных потоков (см., в~част\-ности, 
 работу~\cite{Gorshenin2013}). 

Проверим абсолютную интегрируемость функции $\varphi_X(t)$~\eqref{ChiFinGammaMixt}. 
Имеем:
\begin{multline*}
\int\limits_{-\infty}^{+\infty}\left\lvert\varphi_X(t)\right\rvert\, dt\leqslant 
\sum\limits_{j=1}^{k}p_j \int\limits_{-\infty}^{+\infty}\left\lvert \left(
1-\fr{it}{\lambda_j}\right)^{-r_j}\right\rvert \, dt={}\\
{}=\sum\limits_{j=1}^{k}p_j \int\limits_{-\infty}^{+\infty} \left\lvert\left(
\fr{\lambda_j(\lambda_j+it)}{\lambda_j^2+t^2}\right)^{r_j}\right\rvert\, dt \leqslant{}\\
{}\leqslant\sum\limits_{j=1}^{k}p_j \lambda_j \int\limits_{-\infty}^{+\infty}\left(
1+y^2\right)^{-{r_j}/{2}}\, dy\,.
\end{multline*}

Подынтегральное выражение при $r_j\hm\geqslant 2$ может быть оценено сверху 
функцией $1/({1+y^2})$, при этом соответствующий интеграл равен~$\pi$, что влечет 
абсолютную интегрируемость характеристической функции для конечной смеси 
гам\-ма-рас\-пре\-де\-ле\-ний. Поэтому в~дальнейшем будем предполагать,
 что $r_j\hm\geqslant 2$ для всех возможных значений $j\hm=1,2,\ldots$

Рассмотрим вопрос точ\-ности оценивания неизвестного математического ожидания ${\sf E}_X\hm>0$ 
при добавлении зашумления.

\smallskip

\noindent
\textbf{Теорема~3.}
\textit{Пусть выполнены предположения}~(A)--(D), 
\textit{причем случайные величины~$\varepsilon_j$, $j\hm=1,2,\ldots$, имеют 
распределение типа конечной $k$-ком\-по\-нент\-ной смеси 
гам\-ма-рас\-пре\-де\-ле\-ний вида}~\eqref{FinGammaMixt} 
\textit{с~па\-ра\-мет\-ра\-ми~${\bf r}$, $\boldsymbol{\lambda}$ и~${\bf p}$. Тогда}
\begin{equation}
\label{Th3Eq}
\left\lvert {\sf E}_Y-{\sf E}_X\right\rvert \leqslant \fr{R}{\lambda}+
\fr{\Lambda^{R}}{2^{r}\pi^{r+1}}\left(1+\frac1{r}\right)\,,
\end{equation}
\textit{где} $r=\min(r_1, \ldots,r_k)$; $R\hm=\max(r_1, \ldots,r_k)$; 
$\lambda\hm=\max(\lambda_1, \ldots,\lambda_k)$; 
$\Lambda\hm=\max(\lambda_1, \ldots,\lambda_k)$.

\smallskip

\noindent
Д\,о\,к\,а\,з\,а\,т\,е\,л\,ь\,с\,т\,в\,о\,.\ \
С~учетом пред\-став\-ле\-ний~\eqref{Law} и~\eqref{Fract}, ограниченности 
модуля характеристической функции, перехода от тригонометрической к~показательной 
записи комплексных чисел, а~также независимости случайных величин~$X_j$ 
и~$\varepsilon_j$ \mbox{имеем}:
\begin{multline*}
\left\lvert {\sf E}_Y-{\sf E}_X\right\rvert
\leqslant \left\lvert {\sf E}_\varepsilon\right\rvert+ {}\\
{}+\left\lvert\sum\limits_{n=1}^\infty
\left(
(-1)^n\mathrm{Im} \left(\sum\limits_{j=1}^{k}p_j \varphi_{X_j}(2\pi n)\left(
\vphantom{\fr{2\pi n}{\lambda_j}}
1-{}\right.\right.\right.\right.\\
\left.\left.\left.\left.{}-i\left(\fr{2\pi n}{\lambda_j}\right)\right)^{-r_j}\right)
\Bigg/ ({\pi n})
\vphantom{\sum\limits_{j=1}^{k}}
\right)\right\rvert={}\\
{}=\left\lvert {\sf E}_\varepsilon\right\rvert+ 
\left\lvert\sum\limits_{n=1}^\infty
\left(\!(-1)^n\mathrm{Im} \!\left(\sum\limits_{j=1}^{k}p_j \left(\!
1+\fr{4\pi^2 n^2}{\lambda_j^2}\right)^{- {r_j}/2}\!\times{}\right.\right.\right.\hspace*{-2.8663pt}\\
\left.\left.\left.{}\times \varphi_{X_j}(2\pi n)\,
e^{-ir_j\mathrm{arctan}\,({{t}/{\lambda_j}})}\right)
\Bigg/
({\pi n})
\vphantom{\left(
1+\fr{4\pi^2 n^2}{\lambda_j^2}\right)^{- {r_j}/2}}
\right)\right\rvert\leqslant{}\\
{}\leqslant \left\lvert {\sf E}_\varepsilon\right\rvert+\sum\limits_{j=1}^{k}
p_j\sum\limits_{n=1}^\infty\fr{1}{\pi n}\left(
1+\fr{4\pi^2 n^2}{\lambda_j^2}\right)^{-{r_j}/2}\leqslant{}\\
{}\leqslant  \fr{R}\lambda + \sum\limits_{j=1}^{k}p_j
\sum\limits_{n=1}^\infty\left(\fr{1}{\pi n}\,
\fr{\lambda_j^{r_j}}{(2\pi)^{r_j} n^{r_j}}\right)\leqslant {}
\\
{}\leqslant  \fr{R}{\lambda} + \sum\limits_{j=1}^{k}p_j 
\fr{\lambda_j^{r_j}}{2^{r_j}\pi^{r_j+1}}\left(1+\int\limits_{1}^{\infty}
\fr{1}{ x^{r_j+1}}\,dx\right)
\leqslant{}\\
{}\leqslant \fr{R}{\lambda}+\fr{\Lambda^{R}}{2^{r}\pi^{r+1}}\left(1+\fr{1}{r}\right).
\end{multline*}

При переходе от суммы к~интегралу используется факт убывания функции как переменной~$n$ 
(или~$x$).~\hfill$\square$


\smallskip

\noindent
\textbf{Замечание~3.}\
Теорема~3 описывает соответ\-ст\-ву\-ющий результат для гам\-ма-рас\-пре\-де\-лен\-ных 
за\-шум\-ля\-ющих случайных величин, если положить $k\hm=1$ в~выражении~\eqref{Th3Eq}. 
При этом, очевидно, $r\hm\equiv R$ и~$\lambda\hm\equiv \Lambda$.


\smallskip

Рассмотрим вопросы построения доверительного интервала для неизвестного 
математического ожидания ${\sf E}_X\hm>0$ в~предположении, что случайные величины~$X_j$ 
не содержат ошибок измерения, а все погрешности учтены исключительно в~за\-шум\-ля\-ющих 
элементах~$\varepsilon_j$.

\smallskip

\noindent
\textbf{Теорема~4.}
\textit{Пусть выполнены предположения}~(A)--(D), 
\textit{причем случайные величины~$\varepsilon_j$, $j\hm=1,2,\ldots$, имеют 
распределение типа конечной $k$-ком\-по\-нент\-ной смеси 
гам\-ма-рас\-пре\-де\-ле\-ний вида}~\eqref{FinGammaMixt} 
\textit{с~па\-ра\-мет\-ра\-ми~${\bf r}$, $\boldsymbol{\lambda}$ и~${\bf p}$, 
а~случайные величины} $X_j\stackrel{\text{п.н.}}{=}{\sf E}_X$, $j=1,2,\ldots$ 
\textit{Тогда доверительный интервал для~${\sf E}_X$ при условии $0\hm<\alpha\hm<1$ имеет вид}:
\begin{equation} 
\label{Th4Eq}
\left\lvert {\sf E}_X - \hat{{\sf E}}_X\right\rvert \leqslant  
f({\bf r},\boldsymbol{\lambda},\alpha,n),
\end{equation}
\textit{где}

\vspace*{-9pt}

\noindent
\begin{align}
\hat{{\sf E}}_X&=\fr{1}{n} \sum\limits_{j=1}^{n} Y_j\,; \label{Th4hatE}\\[-4pt]
f({\bf r}, \boldsymbol{\lambda},\alpha,n)&=\fr{z_{1-{\alpha}/2}}{\sqrt{n}} \left(
\sqrt{\fr{R(R+1)}{\lambda^2}-\fr{r^2}{\Lambda^2}}+\fr{1}{2}\right) +{}\notag\\[-1pt]
&\hspace*{20mm}{}+
\fr{R}{\lambda}+\fr{\Lambda^{R}}{2^{r}\pi^{r+1}}\left(1+\fr{1}{r}\right); \notag
\end{align}
\textit{$z_{1-{\alpha}/2}$~--- $\left(1-{\alpha}/2\right)$-кван\-тиль 
стандартного нормального распределения; $r\hm=\min(r_1, \ldots,r_k)$; 
$R\hm=\max(r_1, \ldots,r_k)$; $\lambda\hm=\max(\lambda_1, \ldots,\lambda_k)$; 
$\Lambda\hm=\max(\lambda_1, \ldots,\lambda_k)$}. 

\smallskip

\noindent
Д\,о\,к\,а\,з\,а\,т\,е\,л\,ь\,с\,т\,в\,о\,.\ \
Из центральной предельной теоремы с~учетом условия~(A) 
следует, что величина~$\hat{{\sf E}}_X$~\eqref{Th4hatE} асимптотически нормальна 
с~математическим ожиданием~${\sf E}_Y$~\eqref{EY} и~дисперсией $(1/n){\sf D}_Y$~\eqref{DY}. 
Пользуясь определением и~свойствами гам\-ма-функ\-ции, а~также оценкой~\eqref{Var} 
получим:

\noindent
\begin{multline*}
{\sf D}_Y \leqslant \left(\sqrt{\sum\limits_{j=1}^k p_j
\fr{\lambda_j^{r_j}}{\Gamma(r_j)} \int\limits_{0}^{+\infty} 
e^{\lambda_j x}x^{r_j+1}\, dx}+\fr{1}{2}\right)^2= {}\\[-0.5pt]
= \left(\sqrt{\sum\limits_{j=1}^{k}p_j
\fr{r_j(r_j+1)}{\lambda_j^2}-\left(\sum\limits_{j=1}^{k}p_j
\fr{r_j}{\lambda_j}\right) ^2}+\fr{1}{2}\right)^2\leqslant {}\\[-1.5pt]
{}\leqslant \left(\sqrt{\fr{R(R+1)}{\lambda^2}-\fr{r^2}{\Lambda^2}}+\fr{1}{2}\right)^2\,.
\end{multline*}

Аналогично доказательству Тео\-ре\-мы~2 с~учетом оценки~\eqref{Th3Eq} 
отсюда следует справедливость соотношения~\eqref{Th4Eq}.~\hfill$\square$

\vspace*{-12pt}

\section{Заключение}

Итак, в~работе получены оценки для математического ожидания наблюдений в~предположении 
зашумления конечными смесями нормальных\linebreak (Тео\-ре\-ма~1) 
и~гам\-ма-рас\-пре\-де\-ле\-ний (Тео\-ре\-ма~3). 
%
Построены доверительные интервалы 
для неизвестного математического ожидания в~этих случаях с~использованием 
уточненной оценки~\eqref{Var} 
(Тео\-ре\-мы~2 и~4 соответственно). Отметим, что соответствующие соотношения 
зависят только от <<экстремальных>> значений параметров смесей, но не от числа 
компонент и~весов в~распределении зашумляющих наблюдений. 
%
Замечание~2 
предлагает подход, который  может быть использован для определения неизвестного 
параметра искусственно добавляемого к~исходным данным шума для улучшения качества 
работы метода скользящего разделения смесей.

\smallskip
Автор выражает признательность доктору фи\-зи\-ко-ма\-те\-ма\-ти\-че\-ских наук, 
профессору Виктору Юрьевичу Королеву за идею использования оценки 
дисперсии вида~\eqref{Var} и~другие полезные обсуждения в~рамках 
работы над данной статьей.

\vspace*{-12pt}

{\small\frenchspacing
 {%\baselineskip=10.8pt
 \addcontentsline{toc}{section}{References}
 \begin{thebibliography}{99}
\bibitem{Wright2003} \Au{Wright~D.\,E., Bray~I.} 
A~mixture model for rounded data~// J.~Roy. Stat. Soc.~D 
Sta., 2003. Vol.~52. P.~3--13.

\columnbreak

\bibitem{Bai2009} \Au{Bai~Z., Zheng~S., Zhang~B., Hu~G.} 
Statistical analysis for rounded data~// J.~Stat. Plan.  Infer., 2009. 
Vol.~139. Iss.~8. P.~2526--2542.

\bibitem{Zhang2010} \Au{Zhang~B., Liu~T., Bai~Z.\,D.} 
Analysis of rounded data from dependent sequences~// 
Ann. I.~Stat. Math., 2010. Vol.~62. Iss.~6. P.~1143--1173.

\bibitem{Zhao2012} \Au{Zhao~N., Bai~Z.} 
Analysis of rounded data in mixture normal model~// Stat. Pap., 2012. 
Vol.~53. P.~895--914.

\bibitem{Korolev2011-i} \Au{Королев~В.\,Ю.} 
Ве\-ро\-ят\-но\-ст\-но-ста\-ти\-сти\-че\-ские методы декомпозиции волатильности 
хаотических процессов.~--- М.: Изд-во Моск. ун-та, 2011. 512~с.

\bibitem{Ushakov2015} \Au{Ушаков В.\,Г., Ушаков Н.\,Г.} 
Об усреднении округленных данных~// Информатика и~её применения, 2015. Т.~9. 
Вып.~4. С.~106--109.

\bibitem{Ushakov2017a} \Au{Ушаков~В.\,Г., Ушаков~Н.\,Г.} 
Границы точ\-ности восстановления информации, 
теряемой при округлении результатов наблюдений~// 
Вестник Московского университета. Серия~15: Вычислительная математика и~кибернетика, 
2017. №\,2. С.~26--30.

\bibitem{Ushakov2017b} \Au{Ushakov~N.\,G., Ushakov~V.\,G.} 
Statistical analysis of rounded data: Recovering of information lost due to rounding~// 
J.~Korean Stat. Soc., 2017.  Vol.~46. No.\,3. P.~426--437.

\bibitem{Gorshenin2016} \Au{Gorshenin~A.\,K., Korolev~V.\,Yu.} 
A~noising method for the identification of the stochastic structure of 
information flows~// Comm. Com. Inf. Sc., 2017. 
Vol.~678. P.~279--289.

\bibitem{Gorshenin2013} 
\Au{Gorshenin~A., Korolev~V.} Modelling of statistical
fluctuations of information flows by mixtures of gamma distributions~// 
27th European Conference on Modelling and Simulation Proceedings.~--- 
Dudweiler, Germany: Digitaldruck Pirrot GmbHP, 2013. P.~569--572.
 \end{thebibliography}

 }
 }

\end{multicols}

\vspace*{-6pt}

\hfill{\small\textit{Поступила в~редакцию 03.08.18}}

\vspace*{6pt}

%\newpage

%\vspace*{-24pt}

\hrule

\vspace*{2pt}

\hrule

\vspace*{-2pt}


\def\tit{DATA NOISING BY FINITE NORMAL AND~GAMMA MIXTURES WITH~APPLICATION 
TO~THE~PROBLEM OF~ROUNDED OBSERVATIONS}


\def\titkol{Data noising by finite normal and~gamma mixtures with~application 
to~the~problem of~rounded observations}



\def\aut{A.\,K.~Gorshenin}

\def\autkol{A.\,K.~Gorshenin}

\titel{\tit}{\aut}{\autkol}{\titkol}

\vspace*{-11pt}


\noindent
Institute of Informatics Problems, Federal Research Center ``Computer Science and
Control'' of the Russian Academy of Sciences, 44-2~Vavilov Str., Moscow 119333,
Russian Federation


\def\leftfootline{\small{\textbf{\thepage}
\hfill INFORMATIKA I EE PRIMENENIYA~--- INFORMATICS AND
APPLICATIONS\ \ \ 2018\ \ \ volume~12\ \ \ issue\ 3}
}%
 \def\rightfootline{\small{INFORMATIKA I EE PRIMENENIYA~---
INFORMATICS AND APPLICATIONS\ \ \ 2018\ \ \ volume~12\ \ \ issue\ 3
\hfill \textbf{\thepage}}}

\vspace*{3pt}



\Abste{In many real problems, statistical analysis of data containing additional 
measurement errors, including rounding, is performed, which in some situations can 
lead to sufficiently significant distortions. In this paper, estimates for an 
unknown expectation of observations are obtained for one of the possible 
rounding models under the assumption that the original data are additionally 
noised with random variables having distributions of the type of finite 
mixtures of normal and gamma laws. Confidence intervals for an 
unknown expectation are constructed using the refined estimate for 
the variance of the integer part of the random variable. An algorithm 
for determining the value of the parameter of artificial noise, which 
can be added to the initial data to improve the quality of the 
method of moving separation of mixtures, is discussed.}


\KWE{noisy data; rounded data; finite normal mixtures; finite gamma mixtures; 
confidence intervals; moving separation of mixtures}



\DOI{10.14357/19922264180304}

%\vspace*{-14pt}

\Ack
\noindent
The research was supported by the Russian Science Foundation (project 18-71-00156).



%\vspace*{6pt}

  \begin{multicols}{2}

\renewcommand{\bibname}{\protect\rmfamily References}
%\renewcommand{\bibname}{\large\protect\rm References}

{\small\frenchspacing
 {%\baselineskip=10.8pt
 \addcontentsline{toc}{section}{References}
 \begin{thebibliography}{99}
\bibitem{1-gor-1}
\Aue{Wright,~D.\,E., and I.~Bray.} 2003.
A~mixture model for rounded data.  \textit{J.~Roy. Stat. Soc.~D Sta.} 52:3--13.

\bibitem{2-gor-1}
\Aue{Bai,~Z., S.~Zheng, B.~Zhang, and G.~Hu.} 2009. 
Statistical analysis for rounded data. \textit{J.~Stat. Plan. 
Infer.} 139(8):2526--2542.

\bibitem{3-gor-1}
\Aue{Zhang,~B., T.~Liu, and Z.\,D.~Bai.} 2010. 
Analysis of rounded data from dependent sequences. 
\textit{Ann. I.~Stat. Math.} 62(6):1143--1173.

\bibitem{4-gor-1}
\Aue{Zhao,~N., and Z.~Bai.} 2012. Analysis of rounded data in mixture normal model. 
\textit{Stat. Pap.} 53:895--914.

\bibitem{5-gor-1}
\Aue{Korolev, V.\,Yu.} 2011. 
\textit{Veroyatnostno-statisticheskie metody dekompozitsii volatil'nosti 
khaoticheskikh protsessov} [Probabilistic and statistical methods of 
decomposition of volatility of chaotic processes]. 
Moscow: Moscow University Publishing House. 512~p.

\bibitem{6-gor-1}
\Aue{Ushakov, V.\,G., and N.\,G.~Ushakov.} 
2015. Ob usrednenii okruglennykh dannykh [On averaging of rounded data].
\textit{Informatika i~ee Primeneniya~--- Inform. Appl.} 9(4):106--109.

\bibitem{7-gor-1}
\Aue{Ushakov,~V.\,G., and N.\,G.~Ushakov.} 2017. 
Boundaries of the precision of restoring information lost after rounding
 the results from observations. 
 \textit{Moscow University Computational Math. Cybernetics} 41(2):76--80.

\bibitem{8-gor-1}
\Aue{Ushakov,~N.\,G., and  V.\,G.~Ushakov.} 2017. 
Statistical analysis of rounded data: Recovering of information lost due to rounding. 
\textit{J.~Korean Stat. Soc.} 46(3):426--437.

\bibitem{9-gor-1}
\Aue{Gorshenin,~A.\,K., and V.\,Yu.~Korolev.} 2016. 
A~noising method for the identification of the stochastic structure of information 
flows. \textit{Comm. Com. Inf. Sc.} 678:279--289.

\bibitem{10-gor-1}
\Aue{Gorshenin,~A., and V.~Korolev.} 2013.  Modelling of statistical fluctuations of
information flows by mixtures of gamma distributions. 
\textit{27th European Conference on Modelling and Simulation Proceedings}. 
Dudweiler, Germany: Digitaldruck Pirrot GmbHP. 569--572.

\end{thebibliography}

 }
 }

\end{multicols}

\vspace*{-6pt}

\hfill{\small\textit{Received August 3, 2018}}

%\pagebreak

%\vspace*{-18pt}

\Contrl

\noindent
\textbf{Gorshenin Andrey K.} (b.\ 1986)~--- Candidate of Science (PhD) in physics and
mathematics, associate professor, leading scientist, Institute of Informatics Problems,
Federal Research Center ``Computer Science and Control'' of the Russian Academy of
Sciences, 44-2 Vavilov Str., Moscow 119333, Russian Federation; 
\mbox{agorshenin@frccsc.ru}
\label{end\stat}

\renewcommand{\bibname}{\protect\rm Литература}       %5
 \def\stat{dokukin}

\def\tit{МНОГОУРОВНЕВЫЕ МОДЕЛИ РЕШЕНИЯ МНОГОКЛАССОВЫХ ЗАДАЧ РАСПОЗНАВАНИЯ$^*$}

\def\titkol{Многоуровневые модели решения многоклассовых задач распознавания}

\def\aut{А.\,A.~Докукин$^1$, В.\,В.~Рязанов$^2$, О.\,В.~Шут$^3$}

\def\autkol{А.\,A.~Докукин, В.\,В.~Рязанов, О.\,В.~Шут}

\titel{\tit}{\aut}{\autkol}{\titkol}

\index{Докукин А.\,A.}
\index{Рязанов В.\,В.}
\index{Шут О.\,В.}
\index{Dokukin A.\,A.}
\index{Ryazanov V.\,V.}
\index{Shut O.\,V.}


{\renewcommand{\thefootnote}{\fnsymbol{footnote}} \footnotetext[1]
{Работа выполнена при финансовой поддержке РФФИ (проект 15-51-04028) 
и~БРФФИ (проект Ф15РМ-037).}}


\renewcommand{\thefootnote}{\arabic{footnote}}
\footnotetext[1]{Федеральный исследовательский центр <<Информатика и~управ\-ле\-ние>> 
Российской академии наук, \mbox{dalex@ccas.ru}}
\footnotetext[2]{Московский физико-технический институт (государственный университет), 
\mbox{vasyarv@mail.ru}}
\footnotetext[3]{Белорусский государственный университет, \mbox{olgashut@tut.by}}

\vspace*{2pt}

\Abst{Проблема поиска набора бинарных подзадач для многоклассовых задач 
распознавания рассмотрена с~точки зрения алгебраического и~логического 
подходов к~распознаванию.
При этом теоретически исследованы границы применимости указанных подходов.
Так, рассмотрена связь корректности алгоритмов первого и~второго уровня, 
получено достаточное условие.
Кроме того, показана правомерность использования метода объектных резолюций 
для построения новых объектов на основе информации, заданной прецедентным способом.
В~качестве прикладных результатов предлагаются две модификации метода ECOC 
(error-correcting output codes~--- 
коды, исправляющие ошибки).
Первая заключается в~оптимизации набора бинарных подзадач с~учетом качества 
решающих их алгоритмов.
Вторая представляет собой развитие метода нечеткой объектной резолюции, 
где в~качестве кодового описания класса используется
 мультимножество кодов обучающих объектов.
Предложенные модификации позволяют в~различных условиях 
улучшать качество исходного метода, что продемонстрировано с~по\-мощью 
модельных и прикладных задач.}


\KW{распознавание; многоклассовая задача; ECOC; многоуровневый метод; корректность; 
алгебраический подход; логический подход; кодовое описание класса}

\DOI{10.14357/19922264170106}  

%\vspace*{-4pt}


\vskip 10pt plus 9pt minus 6pt

\thispagestyle{headings}

\begin{multicols}{2}

\label{st\stat}
  

\section{Введение}

В~настоящей статье рассматривается задача распознавания со~многими классами.
Будем использовать стандартную постановку задачи из~\cite{zhur1}.

\smallskip

\noindent
\textbf{Определение~1.}\
Назовем задачей распознавания~$Z$ следующую задачу.
Пусть задана обучающая выборка $\tilde{S}_t(Z)\hm=\{S_1,\ldots,S_m\}$, 
описанная векторами вещественных признаков, $S_i\hm=(a_{i1}, \ldots, a_{in})$, 
$i\hm=1,\ldots,m$.
Выборка разбита на~$l$~классов $K_1,\dots,K_l$.
Классификация объектов обучающей выборки задается 
информационными векторами ${\alpha_i}\hm=(\alpha_{i1}, \ldots, \alpha_{il})$, 
где $\alpha_{ij}$~--- значение предиката <<$S_i\hm\in K_j$>>.
Необходимо построить алгоритм~$A$, позволяющий вычислить классификацию нового 
объекта~$S$.

\smallskip

Если классы не пересекаются, то классификацию объектов можно 
задавать одним числом $\alpha_i\hm\in\{1,\ldots,l\}$, 
и~в~дальнейшем будет использоваться именно такая нотация.

Многоклассовой задачей распознавания будем называть задачу с $l\hm>2$.
Особенностью такой постановки является тот факт, что не все эффективные методы 
распознавания способны непосредственно решать многоклассовые задачи.
В~отличие от, например, метода ближайших соседей или алгоритма вычисления 
оценок~\cite{zhur1, zhur2}, такие методы, как метод опорных векторов~\cite{svm} 
или статистически взвешенные синдромы~\cite{sws}, приходится применять 
в~несколько этапов.
Сначала решается набор дихотомических подзадач, после чего их результаты 
объединяются и~интерпретируются в~терминах исходного набора классов.

Некоторые из~таких многоуровневых подходов достаточно очевидны.
Это так называемые <<один против всех>> (one-vs-all)~\cite{svm} 
и~<<каждый с~каждым>> (one-vs-one)~\cite{knerr}.
Применяются и более общие подходы.
Так, в~методе ECOC~\cite{Dietterich} используются произвольные разбиения 
множества классов на пары метаклассов.
Каждый класс при этом получает двоичный код, как и каждый распознаваемый объект.
Решение о~классификации принимается на~основании близости кодов.
Этот метод был, в~свою очередь, обобщен в~\cite{Allwein}.
Отличие заключается в~том, что при построении бинарных задач 
допускается исключение из рассмотрения исходных классов.
Таким образом, коды классов становятся троичными.
Это важное дополнение позволяет включить схему <<каждый с~каж\-дым>> в~общий метод.

В~настоящей работе предпринята попытка предложить свой метод построения 
набора бинарных подзадач и~обосновать его.
Первые главы посвяще-\linebreak\vspace*{-12pt}

\pagebreak

\noindent
ны теоретическому исследованию вопроса 
с~точки зрения алгебраического и~логического подходов к~распознаванию.
Затем предлагаются прикладные методы и~проводится тестирование.

\section{Корректность многоуровневого алгоритма} %\label{chapter:correctness}

Вопрос корректности является центральным в~алгебраической теории распознавания, 
созданной академиком Ю.\,И.~Журавлевым в~1970-х~гг.~\cite{zhur1, zhur2}.
Под корректностью понимается способность алгоритма безошибочно распознать 
заданную контрольную выборку.
Этому вопросу посвящено большое количество теоретических исследований.
Так, сам Ю.\,И.~Журавлев сформулировал теорему существования корректного 
алгоритма для задачи распознавания в~алгебраическом замыкании 
семейства алгоритмов вычисления оценок (АВО)~\cite{zhur1,zhur2} 
и~оценил его сложность, т.\,е.\ степень корректного полинома.
Эта оценка постепенно уточнялась его учениками, и~в~итоге 
точная оценка была получена А.\,Г.~Дьяконовым~\cite{djakonov}.
Таким образом, вопрос корректности в~одноуровневых схемах можно считать 
закрытым, как минимум, для семейства алгоритмов вычисления оценок.
Однако представляет интерес корректность многоуровневых схем, рассматриваемых 
в~данной работе.
Перейдем к~рассмотрению связи корректности алгоритмов на~разных уровнях 
и~для начала запишем несколько формальных определений.

\smallskip

\noindent
\textbf{Определение~2.}\
Пусть задана задача распознавания~$Z$ и контрольная выборка 
объектов $\tilde{S}_r(Z)\hm=\{S^1,\ldots,S^q\}$ с~известной классификацией 
$\alpha^t\hm\in\{1,\ldots,l\}$, т.\,е.\ выполняется предикат 
<<$S^t\hm\in K_{\alpha^t}$>>, $t\hm=1,\ldots,q$.
Будем называть алгоритм~$A$ корректным для задачи~$Z$ 
и~контрольной выборки $\tilde{S}_r(Z)$, если $A(S^t)\hm=\alpha^t$ для всех 
$t\hm=1,\ldots,q$.
Здесь $A(S^t)\hm\in\{1,\ldots,l,\Delta\}$~--- ответ алгоритма 
о~классификации объекта~$S^t$, который соответствует номеру класса или отказу 
от распознавания~---~$\Delta$.


\smallskip

\noindent
\textbf{Определение~3.}\
Пусть задана задача распознавания~$Z$ и~два непересекающихся подмножества 
множества классов $K^0\subset\{K_1,\ldots,K_l\}$, $K^1\hm\subset\{K_1,\ldots,K_l\}$, 
$K^0\cap K^1\hm=\emptyset$.
Назовем бинарной (дихотомической) подзадачей~$Z$ задачу распознавания~$Z^\prime$ 
следующего вида: $\tilde{S}_t(Z^\prime)\hm=\tilde{S}_t(Z)\cap (K^0\cup K^1)$,
 $\tilde{S}_r(Z^\prime)\hm=\tilde{S}_r(Z)\cap (K^0\cup K^1)$, классы соответствуют 
 метаклассам~$K^0$ и~$K^1$.
Будем говорить, что класс~$K_i$ активен в~бинарной подзадаче~$Z^\prime$, 
если $K_i\hm\in (K^0\cup K^1)$.
Бинарную подзадачу, в~которой все классы активны, будем называть полной.
Число активных классов будем называть рангом бинарной подзадачи~$r(Z_i)$.

\smallskip

Основным условием теоремы существования\linebreak
 корректного полинома над алгоритмами 
вы\-чис\-ления оценок~\cite{zhur1, zhur2} является попарная неизоморфность 
контрольных объектов, т.\,е.\ наличие для\linebreak любой пары контрольных объектов
такого обуча\-юще\-го, что хотя~бы по одному признаку расстояния от 
этих контрольных до него не~равны:
$\forall\ S^i,S^j\hm\in\tilde{S}_r(Z)$, $\exists\ S_k\in\tilde{S}_t(Z)$, 
$p\hm\in\{1,\ldots,n\}$, такие что $|a_{kp}\hm-a^i_p|\hm\neq|a_{kp}\hm-a^j_p|$.
Оно и попарное неравенство классов являются достаточным условием 
существования корректного алгоритма в~алгебраическом замыкании семейства 
АВО~\cite{dokukin2001}.
С~точки зрения теоремы существования рассматриваемая связь достаточно очевидна.
Несложно показать, что выполнение достаточных условий для исходной 
многоклассовой задачи не~гарантирует их выполнения для бинарных подзадач.
Подробно этот вопрос и~другие доказательства рассмотрены в~\cite{patrec}.

При этом также достаточно очевидно, что в~обратную сторону следствие выполняется.
Если набор бинарных подзадач содержит все контрольные объекты 
и~для них выполняются достаточные условия, то~для исходной многоклассовой 
задачи эти условия также выполняются и~корректный алгоритм существует.
Однако этот факт сам по~себе не~дает дополнительных конструктивных 
средств построения такого корректного алгоритма, кроме уже имеющихся 
в~теореме существования.
Исходное~же предположение состоит в~том, что двухуровневая схема позволит 
упростить такое построение.
Поэтому перейдем к~рассмотрению методов построения корректного 
многоклассового алгоритма на базе корректных двухклассовых слагаемых.

Рассмотрим следующую общую схему двухуровневого распознавания.


\smallskip

\noindent
\textbf{Определение~4.}\
Пусть задана задача распознавания~$Z$ и набор из~$W$ ее бинарных 
подзадач $Z_1,\ldots,Z_W$.
Назовем алгоритмом первого уровня алгоритмы~$A_i$, решающие соответственно 
подзадачи~$Z_i$, $i\hm=1,\ldots,W$.
Назовем алгоритмом второго уровня алгоритм~$A$, решающий задачу~$Z$ 
и~использующий для этого выходы алгоритмов первого уровня.

При этом будем называть вектор~$\gamma(K_i)$, где 
$$
\gamma(K_i)_j=
\begin{cases}
1\,,  &\ \mbox{если } K_i \in K^0_j\,;\\
-1\,, &\ \mbox{если } K_i \in K^1_j\,;\\ 
0  &\ \mbox{в~остальных~случаях}\,,
\end{cases}
$$
кодом класса~$K_i$, 
$i\hm=1,\ldots,l$, $j\hm=1,\ldots,W$.
Рангом класса $r(K_i)$ будем называть число бинарных подзадач, 
в~которых он активен, или $|\{\gamma(K_i)_j \,|\, \gamma(K_i)_j\hm\neq0$,
$j\hm=1,\ldots,W\}|$.

Аналогично определим код объекта $\gamma(S^t)$: 
$$
\gamma(S^t)_j= 
\begin{cases}
1\,,  &\ \mbox{если } K_{A_j(S^t)} \in K^0_j\,;\\
-1\,,  &\ \mbox{если } K_{A_j(S^t)} \in K^1_j\,;\\
0\,, &\ \mbox{если~произошел~отказ}\\
&\ \hspace*{17mm}\mbox{от~распознавания}\,,
\end{cases}
$$
$t\hm=1,\ldots,q$, $j\hm=1,\ldots,W$.


\smallskip

Рассмотрим многоклассовую задачу распознавания~$Z$
 и~набор бинарных подзадач $Z_1,\ldots,Z_W$.
Очевидно, если коды классов на~данном наборе попарно различаются 
и~бинарные подзадачи полны, то алгоритм, построенный по~схеме ECOC, будет корректен.

Сложность представляют неполные бинарные подзадачи, поскольку объекты 
игнорируемых классов будут получать произвольные оценки.
При этом если допустить в~наборе бинарных подзадач полные, то этот 
недостаток также легко исправить за~счет полных вспомогательных задач.

Таким образом, наибольшую сложность представляет случай, 
когда все бинарные подзадачи неполны.
К~тому же он представляет и наибольший интерес, поскольку позволяет 
упростить бинарные подзадачи за~счет сокращения количества объектов.
Перейдем к~его рассмотрению и~для опре\-де\-лен\-ности будем считать, 
что ранг всех бинарных подзадач одинаков.

\smallskip

\noindent
\textbf{Определение~5.}\
Назовем расстоянием между двумя классами $d(K_i, K_j)$ число бинарных 
подзадач, в~которых они оба активны и принадлежат различным метаклассам:
\begin{multline*}
d(K_i, K_j) = \left| \left\{ t\in\{1,\ldots,W\} \;|\; 
\gamma(K_i)_t\neq\gamma(K_j)_t,\right.\right.\\
\left.\left. \gamma(K_i)_t\neq 0,\; 
\gamma(K_j)_t\neq 0 \right\} \right|\,.
\end{multline*}


Сформулируем достаточное условие коррект\-ности алгоритма второго уровня.
Пусть задана задача распознавания~$Z$ и~набор бинарных подзадач $Z_1,\ldots,Z_W$, 
ранг всех бинарных подзадач одинаков и~равен $r\hm<l$.
Пусть также алгоритмы первого уровня $A_1,\ldots,A_W$ являются корректными 
на соответствующих бинарных подзадачах.


\smallskip
\noindent
\textbf{Теорема~1.}\
\textit{Если для любых двух классов разность их рангов меньше расстояния 
между ними, т.\,е.\ выполнено неравенство}:
\begin{equation*}
r(K_j) - r(K_i) < d(K_j,K_i),\ \forall\ i,j=1,\ldots,l,\; i\neq j\,,
\end{equation*}
\textit{то алгоритм второго уровня~$A$ является корректным}.


\smallskip

Заметим, что выполнение условия теоремы 
возможно только при положительном расстоянии между любыми двумя классами.

Обратное утверждение выполняется только для случая трех классов.
Действительно, есть следующие типы наборов бинарных подзадач, 
которые не~удовлетворяют условию утверждения: $\{\{1\}\mbox{--}\{2\}\}$, 
$\{\{1\}\mbox{--}\{2\}, \{1\}\mbox{--}\{3\}\}$ 
и~$\{\{1\}\mbox{--}\{2\}$, $\{1\}\mbox{--}\{2,3\}\}$.
Легко убедиться, что во~всех случаях алгоритм второго уровня будет некорректным.

Для случая четырех классов можно привести контрпример.
Возьмем следующие обучающие объекты: $S_1\hm=(-2, 2)\hm\in K_1$, 
$S_2\hm=(2, 2)\hm\in K_2$, $S_3\hm=(-2, -2)\hm\in K_3$ и~$S_4\hm=(2, -2)\hm\in K_4$.
Контрольные построим похожим образом: $S^1\hm=(-1, 1)\hm\in K_1$, 
$S^2=(1, 1)\hm\in K_2$, $S^3\hm=(-1, -1)\hm\in K_3$ и~$S^4\hm=(1, -1)\hm\in K_4$.
В~качестве набора бинарных подзадач возьмем схему <<каждый с~каждым>> 
и~исключим пару $\{\{1\}\mbox{--}\{4\}\}$, чтобы нарушить достаточное условие.
При этом достаточно несложно построить корректный алгоритм второго уровня.

Рассмотрим теперь случай, когда алгоритмы первого уровня некорректны.
Если ошибок незначительное количество, то можно модифицировать 
теорему~1 и~потребовать большего расстояния между классами, чтобы их исправить.
Кроме того, в~достаточном условии фигурирует верхняя оценка голосов 
за~чужие классы, что может позволить получить корректный результирующий алгоритм 
в~реальной ситуации при более равномерном распределении ошибок между классами.

Если~же ошибок значительное количество, то возникают следующие соображения.
Во-пер\-вых, теряется смысл использования теоремы~1 
и~появляется необходимость рассматривать уже не~коды классов, 
а~коды отдельных объектов, чтобы делать выводы о~корректности.
Во-вто\-рых, теряется смысл рассмотрения неполных подзадач.
Учитывая эти соображения, рассмотрим предельную, в~некотором смысле, ситуацию.


Пусть дана задача распознавания~$Z$ и~набор полных бинарных подзадач %\linebreak
 $Z_1,\ldots,Z_W$.
Пусть также дан набор алгоритмов первого уровня $A_1,\ldots,A_W$, решающих эти подзадачи.
Для простоты рас\-смот\-рим случай, когда эти алгоритмы не~дают отказов на~объектах
задачи~$Z$. %\linebreak
Построим следующую задачу~$Z^\prime$.
Число признаков~--- $W$, при этом все признаки бинарные.
Обуча\-ющую выборку составляют коды классов.
Объектами контрольной выборки являются коды объектов исходной задачи~$\tilde{S}_r(Z)$, 
полученные алгоритмами первого уровня.
При этом в~рамках одного класса повторяющиеся объекты исключаются.
Тогда справедлива следующая тео\-рема.

\smallskip

\noindent
\textbf{Теорема~2.}\
\textit{Пусть в~обучающей выборке~$\tilde{S}_t(Z^\prime)$ попарно различны классы,
тогда для существования алгоритма~$A^\prime$, корректного для задачи~$Z^\prime$, 
необходимо и~достаточно попарное различие объектов контрольной 
выборки}~$\tilde{S}_r(Z^\prime)$.


\smallskip

Таким образом, ошибки в результирующем алгоритме возникают, 
если невозможно предложить такую бинарную подзадачу, в~которой два 
объекта из~разных классов относились~бы к~разным метаклассам.
Объекты, неправильно классифициру\-емые по~этой причине, логично считать 
выбросами и~исключать из~рассмотрения.

Очевидно, такой подход обладает всеми недостатками теоремы существования.
Хотя решение и~строится конструктивно, оно громоздко, 
а~получаемый алгоритм склонен к~переобучению.
Прикладной метод двухуровневого распознавания, основанный на~оптимизации 
набора бинарных подзадач, будет предложен в~разд.~4 и~испытан на~практике.

\section{Метод резолюций}

В~задачах распознавания образов часто используются два способа представления информации:
 логический, представляющий собой описание объектов с~использованием логических 
 формул или правил,
 и~прецедентный, заключающийся в~непосредственном перечислении объектов и~классов, 
 которым принадлежат эти объекты.
Первый из~них\linebreak
 применяется в~продукционных экспертных системах~\cite{Giarratano}, 
второй характерен для большинства задач распознавания с~обучением.
Для решения \mbox{задач}, информация в~которых представлена логическим или прецедентным 
способом, соответственно используются метод резолюций и многочисленные алгоритмы 
распознавания, примером которых может служить семейство алгоритмов, 
описанное в~\cite{krasn1998}.
Существуют задачи, в~которых используются оба способа одновременно,~--- например, 
это задача медицинской диагностики~\cite{ablam2011}.
В~данной работе предлагается использовать аналогичный подход для построения 
многоуровневых схем распознавания.
При этом на~первом уровне могут использоваться любые алгоритмы, 
а~объединение их результатов будет производиться в~рамках логического подхода.

Переформулируем задачу распознавания образов~$Z$ в~общей постановке~\cite{ablam2011} 
с~использованием принятых в~логическом подходе обозначений.

На множестве объектов~$X$ произвольной природы заданы 
подмножества $ X_{1},\ldots,X_{l} $, на\-зы\-ва\-емые классами.
Задана также начальная информация~$I_{0}$ о~классах $X _{1},\ldots,X_{l} $.
Требуется указать алгоритм~$A$, определенный на~всем множестве~$X$, 
вычисля\-ющий на~основании информации~$I_{0}$ для произвольного объекта $ x \hm\in X $ 
результат, который может быть интерпретирован в~терминах принадлежности~$x$ 
к~классам~$X _{1},\ldots,X_{l} $.

Введем систему предикатов, характеризующую принадлежность произвольного 
объекта $x \hm\in X $ классам $X _{1},\ldots,X_{l} $:
$$
P_{i}(x)=\begin{cases}
1\,, &\ x \in X_{i}\,; \\
0\,, &\ x \notin X_{i}\,,
\end{cases}\qquad i=1,\ldots,l \,. 
$$

Информацию $I_{0}$ представим в~виде:
$$
I_{0}=\left\lbrace (x,P(x))|x \in X, P(x)=(P_{1}(x),\ldots,P_{l}(x)) 
\right\rbrace \,, 
$$
где $ P(x) $ называется информационным вектором, который 
сопоставляется объектам $x \hm\in X$~\cite{zhur1}.
Для каждого такого объекта, входящего в~описание~$I_{0}$, 
информационный вектор считается известным.
В~распознавании образов множество таких объектов называется выборкой.
Обычно ее принято разделять на~две части~\cite{zhur1}.
Первая часть называется обучающей выборкой 
и~используется для определения параметров или настройки процесса 
обучения алгоритмов распознавания.
Вторая часть называется контрольной выборкой 
и~используется для оценки качества работы алгоритмов.
Обозначим эти части через~$X^{0}$ и~$X^{q}$ соответственно.
К~введенным выборкам чаще всего предъявляется требование 
$X^{0} \cap X^{q}\hm= \emptyset$.

Будем говорить, что любой алгоритм~$A$, реша\-ющий задачу~$Z$, 
строит классификационный вектор $ P^{A}(x)\hm=(P^{A}_{1}(x),\ldots,P^{A}_{l}(x)) $,
 где $ P^{A}_{i}(x)\hm\in \left\lbrace 0,1\right\rbrace $.
Если $ P^{A}_{i}(x)\hm=1 $, то результат алгоритма интерпретируется как 
$x \hm\in X_{i} $; если $ P^{A}_{i}(x)\hm=0 $, то $x \notin X_{i} $.

Для оценки качества работы алгоритма~$A$ вводится функционал качества $\Phi_{A}(X)$, 
значения которого легко интерпретировались бы в~терминах совпадения или близости 
$P_{i}(x)$ и~$P^{A}_{i}(x)$.
В~общем случае чем ближе значение $\Phi_{A}(X^{q})$ к~$1$, тем 
меньше ошибок допускает алгоритм~$A$.
Поэтому из~нескольких алгоритмов предпочтительным считается тот, который имеет 
наибольшее значение функционала~$\Phi_{A}$.
В~предельном случае, если $\Phi_{A}(X^{q})\hm=1$, то алгоритм~$A$ решает задачу~$Z$ 
безошибочно.
Такие алгоритмы называются корректными в~алгебраической теории распознавания 
(см.\ определение~2).

Введем следующие определения и обозначения.
Пусть $S\hm=\left\lbrace s_{1},\ldots,s_{n} \right\rbrace $~--- 
множество всех признаков в~предметной области задачи~$Z$, где $n\hm<\infty $;
 $D_{j}$~--- множество значений признака~$s_{j}\hm\in S$.
Не нарушая общности, можно считать, что $D_{j}\hm=\left\lbrace 0,1,\ldots,|D_{j}|-1 
\right\rbrace $.
Обозначим
$$
 D=\left\lbrace 0,1,\ldots,\max\limits_{j} \left\lbrace |D_{j}|-1\right\rbrace 
 \right\rbrace =\left\lbrace 0,1,\ldots,k-1 \right\rbrace \,. 
 $$

В~дальнейшем предполагается, что все признаки принимают значения 
из~множества~$D$, где $k\hm\neq 1 $.
Объектом назовем отображение

\pagebreak

\noindent
$$
p\left(s_{1},\ldots,s_{n}\right)=\left(D^{p}_{1},\ldots,D^{p}_{n}\right)\,, 
$$
где $D^{p}_{j}\subset D $~--- множество значений признака $s_{j}\hm\in S $ объекта~$p$, 
причем $ D^{p}_{j}\hm\neq\emptyset $.
Объекты называются равными, если $ \forall\ j\, D^{p}_{j}\hm = D^{q}_{j} $.

Если $ D^{p}_{j}\hm=\emptyset$, то считается, что~$p$ не обладает признаком~$s_{j}$ 
и~потому не~рассматривается в~рамках задачи~$Z$.
В~общем случае для произвольного признака существует $ |\rho(D)|\hm=2^{k} $ 
возможных комбинаций его допустимых значений, где $\rho(D)$~--- множество 
всех подмножеств~$D$.
Поэтому будем считать, что $X\hm=(\rho(D))^{n} $.

Для удобства рассуждений назовем множество объектов набором.

В~\cite{patrec} показано существование кодировки, использование которой 
правомерно для описания как нормализованных, так и~ненормализованных объектов,
откуда следует эквивалентность прецедентного 
и~логического способов представления информации в~задаче~$Z$.

Опишем применение метода резолюций для решения задачи~$Z$.
Рассмотрим метод резолюций, исходными данными для которого являются 
не логические формулы, а~объекты из~$X$.
Этот модифицированный метод назовем методом объектных резолюций.

Объект~$r$ называется объектной резольвентой, построенной по~объектам~$p$ 
и~$q$, если значения признаков~$r$ удовлетворяют следующему условию:
$$
 D_{j}^{r}= \begin{cases}
D_{j}^{p} \bigcup D_{j}^{q}\,, & j=h\,; \\
D_{j}^{p} \bigcap D_{j}^{q}\,, & j \neq h\,,
\end{cases}
$$
где $h$~--- номер произвольного признака~$ s_{h}\hm\in S $.

Операцию построения объектной резольвенты обозначим $r\hm=Or_{h}(p,q) $.

Также в~\cite{patrec} показана правомерность использования метода объектных 
резолюций для построения новых объектов на~основе информации, заданной 
прецедентным способом.

Рассмотрим алгоритм использования метода объектных резолюций для решения задачи~$Z$.
Зафиксируем номер~$i$ класса~$X_{i}$ и~определим, принадлежит ли объект~$x$ 
этому классу.
Пусть $ X_{i}^{0}\hm=X^{0} \bigcap X_{i} $.

Алгоритм объектных резолюций~$A_{1} $:
\begin{description}
\item[Шаг~1.] Введем множество $Y_{i}\hm=X_{i}^{0} $.

\item[Шаг~2.] Если $ x \hm\in Y_{i} $, то переходим к~шагу~6, иначе~--- к~шагу~3.

\item[Шаг~3.] Выбираем из~$Y_{i}$ нерассмотренную тройку $(p,q,h) $, где~$p$ и~$q$~--- 
объекты; $h$~--- номер признака.
Если все такие тройки уже рассматривались, то переходим к~шагу~6.

\item[Шаг~4.] Вычисляем $r\hm=Or_{h}(p,q) $. Если $\exists\ j \ D_{j}^{r}\hm=\emptyset$, 
возвращаемся к~шагу~3.

\item[Шаг~5.] Если $r \notin Y_{i} $, то $ Y_{i} := Y_{i} \bigcup \left\lbrace r 
\right\rbrace$. Возвращаемся к~шагу~2.

\item[Шаг~6.] Алгоритм завершает работу.
\end{description}

Алгоритм~$ A_{1} $ можно применять как для прямого, так и для обратного вывода.
В~последнем случае на~шаге~1 вводится множество $ Y_{i}\hm=X_{i}^{0} \bigcup 
\left\lbrace x \right\rbrace$, а~в~качестве объекта~$x$ рассматривается несуществующий 
объект~$o$: $\exists\ j \ D_{j}^{o}\hm=\emptyset$.
Обратный вывод может использоваться, например, в~случае, когда число классов~$l$ 
рав\-но~2.
В~зависимости от~типа вывода результат работы алгоритма~$A_{1}$ можно 
интерпретировать следующим образом.
\begin{enumerate}[1.]
\item Прямой вывод: если алгоритм закончил работу из-за получения объекта~$x$, 
это значит, что набор $\mathrm{Norm}(X_{i}^{0})$ содержит объект~$x$.
Поэтому  $x \hm\in X_{i}$.

\item Обратный вывод: если алгоритм закончил работу из-за 
получения объекта~$o$, это значит, что $ \mathrm{Norm}(Y_{i})\hm=X $ , т.\,е.~$Y_{i}$ 
потенциально содержит все объекты из~$X$.
Поэтому $x \hm\notin X_{i}$.
\end{enumerate}

Если ни~один из~этих результатов не~получен, то с~по\-мощью данного 
алгоритма нельзя сделать никаких выводов о~принадлежности~$x$ классу~$X_{i}$.

Приведем результаты алгоритма~$A_{1}$ к~численному виду.
По~результатам работы алгоритма построим классификационный вектор
$$
A_{1}(x)=\left(P_{1}^{A_{1}}(x),\ldots,P_{l}^{A_{1}}(x)\right) \,,
$$
где 
$$
P_{i}^{A_{1}}(x) = \begin{cases}
1\,, &\ x \in X_{i}\,; \\
0\,, &\ x \notin X_{i}\,.
\end{cases} 
$$

Решение о~принадлежности~$x$ классу~$X_{i}$ принимается алгоритмом~$A_{1}$.

Опишем применение метода резолюций на~втором уровне многоуровневой схемы.
Для этого также построим новую задачу с~$W$ бинарными признаками, где 
объектами являются коды исходных объектов, полученные алгоритмами первого уровня.
Только, в~отличие от~разд.~2, не~будем исключать повторы~--- это позволит 
в~дальнейшем оценить степень вхождения объекта в~класс с~по\-мощью 
метода нечеткой резолюции,
 т.\,е.\ метода резолюций для решения задачи~$Z$ 
 в~случае, когда информация~$I_{0}$ задана с~помощью функций нечеткой логики.

Пусть $E $~--- произвольное множество.
Введем характеристическую функцию~$\mu_{E}(x) $,
 значения которой описывают степень принадлежности элемента~$x$ множеству~$E$: 
 $\mu_{E}(x) \hm\in [0,1]$.

Пусть $ E_{1}$ и~$ E_{2} $~--- нечеткие подмножества~$E$.
Рассмотрим следующие операции нечеткой логики~\cite{kofman}:
\begin{enumerate}[(1)]
\item дополнение:
$$
\mu_{\overline{E_{1}}} (x)= 1 - \mu_{E_{1}}(x)\,;
$$

\item пересечение:
$$
 \mu_{E_{1} \bigcap E_{2}}(x) = \min\{{\mu_{E_{1}}(x), \mu_{E_{2}}(x)}\}\,;
 $$ 
    
\item объединение:
$$ \mu_{E_{1} \bigcup E_{2}}(x) = \max\{{\mu_{E_{1}}(x), \mu_{E_{2}}(x)}\}\,.
$$
\end{enumerate}

Припишем каждому набору $V \subset X$ характеристическую функцию~$\mu_{V}(p)$, 
значение которой описывает степень принадлежности объекта~$p$ набору~$V$.
Приведем пример такой функции.
Пусть\linebreak
 объекты могут повторяться в~обучающей выборке~$X^{0}$.
Обозначим через~$N^{p}$ общее количество вхождений объекта~$p$ в~$X^{0}$, 
а~через $ N_{i}^{p} $~--- число вхождений~$p$ в~$X_{i}^{0} $.
Таким образом, $N^{p} \hm= \sum\nolimits_{i=1}^{l} N_{i}^{p}$.
Определим $\mu_{X_{i}}(p)$ следующим образом:
$$
\mu_{X_{i}}(p) = \fr {N_{i}^{p}} {N^{p}} \,.
$$
Тогда $\mu_{X_{i}}(p)\hm \in [0,1] $, что и~требуется.

Поскольку алгебра нормализованных объектов~$G^{\mathrm{norm}}$ 
изоморфна алгебре логических функций~$L_{k}$, то операции нечеткой 
логики справедливы и~для наборов:
\begin{align*}
\mu_{\overline{V}} (p) &= 1 - \mu_{V}(p) \,; \\
\mu_{V \bigcap W}(p) &= \min\{{\mu_{V}(x), \mu_{W}(p)}\}  \,; \\
\mu_{V \bigcup W}(p) &= \max\{{\mu_{V}(x), \mu_{W}(p)}\}  \,.
\end{align*}
Здесь $ V \subset X $ и~$W \subset X $~--- произвольные наборы.

Для нечеткой логики также существуют аналоги метода резолюций.
Один из~примеров такого аналога приведен в~\cite{lee}.
Рассмотрим обобщение метода объектных резолюций для случая 
нечеткого описания объектов.


\smallskip

\noindent
\textbf{Теорема~3.}\
\textit{Пусть заданы наборы $x_{1}$ и~$x_{2}$.
Построим $ r\hm=Or_{h}(x_1,x_2)$.
Тогда существует такое значение} $t \hm\in [0,1]$, 
\textit{что $ \mu_{x_{1}}(p)\hm > t$, $\mu_{x_{2}}(p) \hm> t$ и $\mu_{r}(p) \hm> t$}.


\smallskip

Несложно видеть, что утверждение данной тео\-ре\-мы можно обобщить для 
произвольного числа объектов.
Таким образом, для любых набора объектов~$V$ 
и~класса~$X_{i}$ существует такое $t \hm\in [0,1]$, 
которое можно принять в~качестве порогового значения, интерпретируемого 
в~терминах принадлежности заданному классу:
$$
 \forall\ p \in V (p \in X_{i} \Leftrightarrow \mu_{X_{i}}(p) > t) \,.
 $$

В~\cite{lee} показано, что при выполнении условий $\mu_{E}(x_{1}) \hm> 0{,}5$ 
и~$ \mu_{E}(x_{2})\hm > 0{,}5 $ справедливо следу\-ющее утверждение:
$$
\mu_{E}(r) > \mu_{E}\left(x_{1} \wedge x_{2}\right) \,, 
$$
а~следовательно, $\mu_{E}(r)\hm > 0{,}5$.
Поэтому пороговое значение $t\hm=0{,}5$ 
особенно удобно для применения на~практике.
%, так как в~этом случае 
%не~требуется знать значения $\mu_{z_{1}}(p)$ и~$\mu_{z_{2}}(p)$.

Опишем алгоритм использования нечеткого метода объектных резолюций для решения 
задачи~$Z$.
Пусть задано пороговое значение $t \hm\in [0,1]$.

\smallskip

Алгоритм~$ A_{1}^{f} $:
\begin{description}
\item[Шаг~1.] Применим алгоритм~$A_{1} $ к~$ X $.
Пусть $Y_{i} $~--- набор объектов, которые по~результатам выполнения алгоритма~$A_{1}$ 
считаются принадлежащими классу~$X_{i}$, $i\hm=1,\ldots,l $, т.\,е.\
 $\forall\ p \hm\in Y_{i}$, $\mu_{X_{i}}(p)\hm>t$.

\item[Шаг~2.] Для каждого объекта
 $ p\hm \in X \backslash \left(\bigcup\limits_{i=1}^{l} Y_{i}\right)$
выполним шаги~3--4.

\item[Шаг~3.] Для каждого класса~$X_{i}$ вычислим $\mu_{X_{i}}(p)$.

\item[Шаг~4.] Пусть $\{\mu_{Y_{v}}(p) \} \hm= \max\limits_{i} \{ \mu_{Y_{i}}(p) \}$,
 $w\hm = \max\limits_{v} \{v\}$.
Если $\mu_{X_{w}}(p) \hm> t$, добавим объект~$p$ в~$Y_{w}$:
$$ 
Y_{w} = Y_{w} \bigcup p\,. 
$$

\item[Шаг~5.] Алгоритм завершает работу.

\end{description}

Результаты алгоритма~$ A_{1}^{f}$ интерпретируются следующим образом: если 
по окончании работы алгоритма $p \hm\in Y_{i}$, то $p \hm\in X_{i}$.

Таким образом, алгоритм~$ A_{1}^{f}$ относит объект~$p$ к~классу, 
в~котором функция принадлежности этого объекта принимает максимальное значение, 
а~если таких классов несколько, выбирает среди них класс с~максимальным номером.

Покажем, что алгоритм~$ A_{1}^{f} $ работает не хуже алгоритма~$A_{1}$.
Определим $ P_{i}^{A_{1}^{f}}(p) $ следующим образом:
$$
 P_{i}^{A_{1}^{f}}(p) = \begin{cases}
1\,, &\ p \in X_{i}\,; \\
0\,, &\ p \notin X_{i}\,.
\end{cases} 
$$

В~определении $ P_{i}^{A_{1}^{f}}(p)$ решение о~принадлеж\-ности~$p$ 
классу~$X_{i}$ принимается алгоритмом~$A_{1}^{f}$.


\smallskip

\noindent
\textbf{Теорема~4.}\
$\Phi_{A_{1}^{f}}(X^{q}) \hm\geqslant \Phi_{A_{1}}(X^{q})$.

\smallskip


Метод нечеткой объектной резолюции с~некоторыми модификациями будет 
использован в~сле\-ду\-ющем разделе для объединения результатов алгоритмов первого уровня.
Описанием класса при этом будет служить мультимножество кодов 
объектов обучающей выборки, полученное набором алгоритмов, 
настроенных на~выбранном наборе бинарных подзадач.

\section{Прикладная реализация}

Перейдем к~рассмотрению прикладного метода построения наборов бинарных подзадач.
Будем учитывать, во-первых, полученные выше теоретические результаты.
Так, потребуем выполнения достаточного условия теоремы~1.
Для полностью определенных задач это будет означать различимость кодов классов.

Построим работу метода следующим образом.
Сначала получим некоторый набор бинарных подзадач.
Единственным условием на~этом этапе является различимость классов.
Затем найдем веса этих наборов, исходя из~оптимальности рас\-сто\-яний между 
кодами классов.
На~практике веса дихотомий часто обнуляются, что позволяет сократить исходный набор.

Исходный набор бинарных подзадач строится несколькими способами.
Самый очевидный из~них~--- конструировать случайные дихотомии, т.\,е.\
 полагать вектор $\beta \hm=(\beta_{1} ,\beta_{2},\ldots,\beta_{N})
 \hm\in \{ 0,1\}$ случайным.
Качество полученных дихотомий, или вероятность правильной классификации на~два 
класса, будет различаться.
При этом в~итоговом наборе выгодно иметь алгоритмы лучшего качества, 
совершающие меньшее число ошибок.

Для построения таких подзадач, называемых в~дальнейшем <<оптимальными>>, 
будем использовать метод наискорейшего спуска.
В~качестве начального приближения будет использоваться случайная дихотомия 
$\beta \hm=(\beta _{1} ,\beta _{2} ,\ldots,\beta _{N} )\in \{ 0,1\}$.\linebreak
Далее находится и изменяется компонента век\-тора~$\beta _{i}$ дихотомии, 
дающая максимальное увеличение критерия качества.
После единичной итерации вектор изменяется следующим образом:\linebreak 
$\beta \hm=(\beta_{1},\beta_{2},\ldots,\beta_{i-1}, 1-\beta_{i},
\beta_{i+1},\ldots,\beta_{N})$~--- и~процесс повторяется.
Если требуемой компоненты~$\beta_{i}$ не~существует, то процесс оптимизации 
заверша\-ется.

Исходный набор бинарных подзадач составляется из~оптимальных дихотомий.
В~ряде случаев алгоритм оптимизации оказывается не~в~состоянии обеспечить 
разделимость классов.
Тогда кодовая матрица пополняется случайными.

Предлагается использовать два подхода к обработке первичного набора
 бинарных подзадач.
В~рамках первого этот набор еще раз оптимизируется с~целью обеспечения 
максимального расстояния между кодами классов.
Для этого вводятся веса дихотомий и рассматривается следующая задача оптимизации.
Пусть ${\left\| \alpha _{ij} \right\|_{l \times W}}$~--- кодовая матрица, 
полученная путем поиска дихотомий, где $l$~--- число классов исходной выборки, 
а~$W$~--- число дихотомий:

\noindent
\begin{align*}
\sum\limits_{j = 1}^W {\left| \alpha _{\nu j} - \alpha _{\mu j} \right|} {x_j} &\ge y 
\enskip
 \forall\ \nu,\mu;\,\nu  > \mu;\enskip \nu,\,\mu  = 1,\ldots,l\,; \\
\sum\limits_{j = 1}^W {{x_j}} & = W\,; \\
y &\rightarrow  \max\,.
\end{align*}
Кроме отбора существенных бинарных подзадач, т.\,е.\
 подзадач, у~которых $x_i\hm>0$, найденные веса будем использовать непосредственно 
 при расчете функции близости кода объекта к~коду класса:
\begin{equation*}
d(S^t,K_j) = \sum\limits_{j = 1}^W {\left| {{\alpha _{i j}} - 
{\beta _{j}}} \right|} {x_j}\,,
\end{equation*}
где $\beta$~--- код объекта~$S^t$, $\gamma(S^t)\hm=\beta_j$.
Этот метод будем называть методом оптимизации дихотомий с~весами (ОДВ).

Второй подход основан на~методе НОР (нечеткой объектной резолюции).
Вместо кода класса\linebreak введем его кодовое описание, вычисленное как 
мультимножество кодов обучающих объектов, полученных тем~же набором алгоритмов.
Пусть класс~$K_j$ описан набором пар $\{\gamma_{ji}, \nu_{ji}\}$, $i\hm=1,\ldots,W_j$,
 где $\gamma_{ji}\hm=\gamma(S)$; $S\hm\in K_j\cap \tilde{S}_t(Z)$~--- коды 
 объектов класса~$K_j$ обучающей выборки;
 $\nu_{ji}$~--- час\-то\-та кода~$\gamma_{ji}$ в~описании класса~$K_j$: 
 $$
 \nu_{ji}=\fr{{|\{S|S\hm\in K_j\cap \tilde{S}_t(Z),
 \gamma(S)=\gamma_{ji}\}|}}{{|K_j\cap \tilde{S}_t(Z)|}}\,;
 $$
 $W_j$~--- число различных кодов в~описании класса~$K_j$.
Тогда оценку произвольного объекта~$S$ за~класс~$K_j$ будем вычислять по~формуле:
$$
\Gamma_j(S) = \sum\limits_{i=1}^{W_j}\nu_{ji}\fr{1}
{\left(1+d(\gamma(S),\gamma_{ji})\right)^2}\,,
$$
где $d(\gamma_1, \gamma_2)$ обозначает расстояние Хэмминга между 
кодами~$\gamma_1$ и~$\gamma_2$.



Этот подход будем называть методом кодовых описаний классов, или КОК.

Эксперименты для оценки эффективности предлагаемых подходов проводились
  с~модельной задачей model2 (12~классов) и двумя реальными задачами:
  предсказание года (10~классов, year)
  и~распознавание букв (26~классов, letter recognition, UCI Machine Learning 
  Repository~\cite{uci}).
При этом для каждой задачи генерировались случайные наборы бинарных подзадач 
заданных мощностей, которые затем использовались для обучения 
и~проверки на~независимой выборке предлагаемых подходов: ОДВ и~КОК.
В~качестве ориентира используется алгоритм, основанный на~поиске ближайшего
 кода\linebreak\vspace*{-12pt}
 
 \pagebreak
 
{ \small \begin{center}  %fig2
\begin{tabular}{|c|c|c|c|c|}
\multicolumn{5}{c}{Результаты экспериментов}\\
\multicolumn{5}{c}{\ }\\[-6pt]
  \hline
    Задача & \tabcolsep=0pt
    \begin{tabular}{c}Число\\ подзадач\end{tabular} & ECOC & КОК & ОДВ \\
  \hline
 & 20 & 68,0 & 68,9 & 66,4 \\
    & 40 & 69,1 & 69,9 & 69,7 \\
    model2  & 60 & 70,2 & 70,1 & 69,7 \\
    & 80 & 70,5 & 70,6 & 70,9 \\
    & 100\hphantom{9} & 70,8 & 70,7 & 71,0 \\
  \hline
 & 20 & 85,0 & 85,3 & 84,2 \\
    & 40 & 86,2 & 86,2 & 85,5 \\
  letter    & 60 & 86,5 & 86,5 & 86,3 \\
    & 80 & 86,7 & 86,8 & 86,7 \\
    & 100\hphantom{9} & 87,0 & 87,0 & 86,8 \\
  \hline
 \multicolumn{1}{|c|}{\raisebox{-18pt}[0pt][0pt]{ year }}  & 20 & 41,8 & 40,4 & 44,8 \\
    & 40 & 40,1 & 40,4 & 40,5 \\
    & 60 & 39,7 & 47,4 & 40,3 \\
    & 80 & 42,1 & 43,6 & 44,5 \\
  \hline
\end{tabular}
\end{center}}

\vspace*{6pt}


 
 \noindent
  класса, т.\,е.\ реализация метода ECOC, выполненная авторами работы.
Для каждой задачи и каждой мощности генерировались по~три случайных набора, 
после чего результаты усреднялись.
Бинарные подзадачи решались с~помощью метода опорных векторов из~пакета 
scikit-learn~\cite{scikit}.

Модельная задача заслуживает отдельного упоминания.
На двумерной плоскости генерировалось~20~выборок из~нормального распределения 
с~центрами, расположенными в~пять столбцов и~четыре строки.
После чего некоторые пары и~тройки получившихся скоплений точек объединялись 
в~один класс. Всего таких классов было~12.

Результаты экспериментов показаны в~таблице.
Задачи model2 и~letter демонстрируют похожую тенденцию~--- при небольшом 
количестве закономерностей метод ОДВ обычно отстает, но с~его ростом 
отставание уменьшается и~даже сменяется опережением.
Такое поведение метода достаточно ожидаемо, так как 
чем больше исходного материала для оптимизации, тем проще отобрать хорошие подзадачи.
В~этих экспериментах отбиралось примерно две трети подзадач.


Вместе с~тем при небольшом числе исходных случайных подзадач метод КОК 
демонстрирует преимущество перед простым методом.
Затем,\linebreak с~рос\-том их числа, преимущество ослабевает.
Исключение составила задача~{year}, в~которой качество 
менялось непредсказуемо.
Причем даже в~рамках одного метода и~одного набора подзадач 
дисперсия качества распознавания была очень высока.
С~другой стороны, это позволило получить наибольший абсолютный 
выигрыш от~применения пред\-ла\-га\-емых подходов.
Такое поведение алгоритмов на~данной задаче требует дополнительного 
исследования и~объяснения.

{\small\frenchspacing
 {%\baselineskip=10.8pt
 \addcontentsline{toc}{section}{References}
 \begin{thebibliography}{99}
\bibitem{zhur1}
    \Au{Журавлёв~Ю.\,И.}
{Корректные алгебры над множеством некорректных (эвристических) алгоритмов~I}~//
    Кибернетика,  1977. №\,4. С.~14--21.
\bibitem{zhur2}
    \Au{Журавлёв~Ю.\,И.}
{Корректные алгебры над множеством некорректных (эвристических) алгоритмов~II}~//
    Кибернетика,  1977. №\,6. С.~21--27.
\bibitem{svm}
\Au{Cortes~C., Vapnik~V.}
{Support-vector networks}~//
    Mach. Learn.,  1995. Vol.~20. No.\,3. P.~273--297.
\bibitem{sws}
    \Au{Кузнецов~В.\,А., Сенько~О.\,В., Кузнецова~А.\,В., Семенова~Л.\,П., 
    Алещенко~А.\,В., Гладышева~Т.\,Б., Ившина~А.\,В.} 
    Распознавание нечетких сис\-тем по методу статистически взвешенных синдромов 
    и~его применение для иммуногематологической характеристики нормы 
    и~хронической патологии~// Хим. физика, 1996.  Т.~15. №\,1.  С.~81--100.


\bibitem{knerr}
    \Au{Knerr~S., Personnaz~L., Dreyfus~G.}
{ Single-layer learning revisited: A~stepwise procedure for building and training neural network}~//
    Neurocomputing: Algorithms, architectures and applications~/
    Eds. F.\,F.~Soulie, J.~Herault.~--- NATO ASI subser. F.~---  Berlin--Heidelberg:
    Springer-Verlag, 1990. Vol.\,68. 
    P.~41--50.
\bibitem{Dietterich}
 \Au{Dietterich~T.\,G., Bakiri~G.}
{Solving multiclass learning problems via error-correcting output codes}~//
    J.~Artif. Intell. Res., 1995. No.\,2. P.~263--286.
\bibitem{Allwein}
  \Au{Allwein~E., Shapire~R., Singer~Y.}
{Reducing multi-class to binary: A unifying approach for margin classifiers}~//
    J.~Mach. Learn. Res., 2000. Vol.~1. No.\,1. P.~113--141.
\bibitem{djakonov}
    \Au{Дьяконов~А.\,Г.}
    {Алгебра над алгоритмами вычисления оценок: минимальная степень корректного алгоритма}~//
    Ж.~вычисл. матем. и матем. физ.,  2005. Т.~45. №\,6. С.~1134--1145.
\bibitem{dokukin2001}
    \Au{Докукин~А.\,А.}
{О~построении в~алгебраическом замыкании одного алгоритма распознавания}~//
    Ж.~вычисл. матем. и матем. физ., 2001. Т.~41. №\,12. С.~1811--1815.
\bibitem{patrec}
    \Au{Dokukin~A., Ryazanov~V., Shut~O.}
{Multilevel models for solution of multiclass recognition problems}~//
    Pattern Recognition Image Anal., 2016. Vol.~26. No.\,3. P.~461--473.

\bibitem{Giarratano} %11
    \Au{Джарратано~Дж., Райли~Г.}
Экспертные системы: принципы разработки и программирование~/
Пер. с~англ.~--- М.: Вильямс, 2007. 
1152~с. (\Au{Giarratano~J.\,C., Riley~G.\,D.} {Expert systems: 
Principles and programming}.~---  Boston, MA, USA: PWS Publ. Co., 2004. 856~p.)
\bibitem{krasn1998}
    \Au{Краснопрошин~В.\,В., Образцов~В.\,А.}
{Распознавание с~обучением как задача выбора}~//
    Цифровая обработка изображений, 1998. С.~80--94.
\bibitem{ablam2011}
    \Au{Абламейко~С.\,В., Краснопрошин~В.\,В., Образцов~В.\,А.}
{Модели и технологии распознавания образов с~приложением в~интеллектуальном анализе данных}~//
    Вестник БГУ. Сер.~1: Физика. Математика. Информатика, 2011. №\,3. С.~62--72.
\bibitem{kofman}
    \Au{Кофман~А.}
Введение в~теорию нечетких множеств~/
Пер. с~англ.~--- М.: Радио и связь, 1982. 432~с.
(\Au{Kaufman~A.} {Introduction to the theory of fuzzy subsets.}~---
New York, NY, USA: Academic Press, 1975.
432~p.)
\bibitem{lee}
    \Au{Lee~R.\,C.\,T.}
{Fuzzy logic and the resolution principle}~//
    J.~ACM, 1972. Vol.~19. No.\,1. P.~109--119.

%\bibitem{berger}
% \Au{Berger~A.}
%    {Error-correcting output coding for text classification}~//
%    IJCAI'99: Workshop on Machine Learning for Information Filtering Proceedings, 1999. 8~p.


\bibitem{uci}
    \Au{Lichman~M.}
    {{UCI} machine learning repository}.~--- Irvine, CA, USA: University of 
    California, School of Information and Computer Science, 2013. 
    {\sf http://archive. ics.uci.edu/ml}.
\bibitem{scikit}
    \Au{Pedregosa~F., Varoquaux~G., Gramfort~A., Michel~V., Thirion~B., Grisel~O., 
    Blondel~M., Prettenhofer~P.,
         Weiss~R., Dubourg~V., Vanderplas~J., Passos~A., Cournapeau~D., Brucher~M., 
         Perrot~M., Duchesnay~E.}
{Scikit-learn: Machine learning in {P}ython}~//
{J.~Mach. Learn. Res.}, 2011. Vol.~12. P.~2825--2830.


 \end{thebibliography}

 }
 }

\end{multicols}

\vspace*{-6pt}

\hfill{\small\textit{Поступила в~редакцию 2.08.16}}

\vspace*{8pt}

%\newpage

%\vspace*{-24pt}

\hrule

\vspace*{2pt}

\hrule

%\vspace*{8pt}


\def\tit{MULTILEVEL MODELS FOR~PATTERN RECOGNITION TASKS WITH~MULTIPLE CLASSES}

\def\titkol{Multilevel models for~pattern recognition tasks with~multiple classes}

\def\aut{A.\,A.~Dokukin$^1$, V.\,V.~Ryazanov$^2$, and~O.\,V.~Shut$^3$}

\def\autkol{A.\,A.~Dokukin, V.\,V.~Ryazanov, and~O.\,V.~Shut}

\titel{\tit}{\aut}{\autkol}{\titkol}

\vspace*{-9pt}

\noindent
$^1$Federal Research Center 
``Computer Science and Control'' of Russian Academy of Sciences,
 40~Vavilov Str.,\linebreak
 $\hphantom{^1}$Moscow, 119333, Russian Federation
 
 \noindent
 $^2$Moscow Institute of Physics and Technology, 9~Institutskiy Per., 
Dolgoprudny, Moscow Region 141700, Russian\linebreak
 $\hphantom{^1}$Federation


 \noindent
$^3$Belarusian State University, 4~Nezavisimosti Av., 
Minsk 220030, Republic of Belarus



\def\leftfootline{\small{\textbf{\thepage}
\hfill INFORMATIKA I EE PRIMENENIYA~--- INFORMATICS AND
APPLICATIONS\ \ \ 2017\ \ \ volume~11\ \ \ issue\ 1}
}%
 \def\rightfootline{\small{INFORMATIKA I EE PRIMENENIYA~---
INFORMATICS AND APPLICATIONS\ \ \ 2017\ \ \ volume~11\ \ \ issue\ 1
\hfill \textbf{\thepage}}}

\vspace*{3pt}



\Abste{The problem of choosing binary subtasks for recognition tasks with multiple 
classes is considered from the points of view of the algebraic and logical approaches 
to recognition. The limits of their applicability were studied theoretically. The 
sufficient condition of correctness of algorithms is stated as a~result of research 
of dependency between the first and the second level algorithms. Additionally, 
the paper proves that the object resolution method is applicable to constructing 
new objects using the precedent information. As an applied result, two modifications 
of the ECOC (error-correcting output codes)
method are proposed. The first one is based on optimization of the binary 
subtasks set. The second one develops ideas of the fuzzy object resolution method 
with classes described by multisets of codes of their precedents. The proposed 
modifications make it possible to increase the initial method's quality in 
various situations, which is demonstrated by the example of model and 
real-world tasks.}

\KWE{classification; multiclass task; ECOC; multilevel method; correctness; algebraic
approach; logical approach; code class description}



\DOI{10.14357/19922264170106}  

\vspace*{-9pt}

\Ack
\noindent
The work is supported by the Russian Foundation for Basic Research 
(grant No.\,15-51-04028) and BRFBR (grant No.\,F15PM-037).



%\vspace*{3pt}

  \begin{multicols}{2}

\renewcommand{\bibname}{\protect\rmfamily References}
%\renewcommand{\bibname}{\large\protect\rm References}

{\small\frenchspacing
 {%\baselineskip=10.8pt
 \addcontentsline{toc}{section}{References}
 \begin{thebibliography}{99}

\bibitem{1-dok-1}
\Aue{Zhuravlev, Yu.\,I.} 1977. Korrektnye algebry nad mno\-zhe\-st\-vom nekorrektnykh 
(evristicheskikh) algoritmov~I 
[Correct algebras over sets of incorrect (heuristic) algorithms~I]. 
\textit{Kibernetika} [Cybernetics] 4:14--21.
\bibitem{2-dok-1}
\Aue{Zhuravlev, Yu.\,I.} 1977. Korrektnye algebry nad mno\-zhe\-st\-vom nekorrektnykh 
(evristicheskikh) algoritmov~II 
[Correct algebras over sets of incorrect (heuristic) algorithms~II].
\textit{Kibernetika} [Cybernetics] 6:21--27.
\bibitem{3-dok-1}
\Aue{Cortes, C., and V.~Vapnik}. 1995. Support-vector networks. 
\textit{Mach. Learn.} 3(20):273--297.
\bibitem{4-dok-1}
\Aue{Kuznetsov, V.\,A., O.\,V.~Senko, A.\,V.~Kuznetsova, L.\,P.~Semenova,
A.\,V.~Aleshchenko, T.\,B.~Gladysheva, and
A.\,V.~Ivshina.} 1996. 
Raspoznavanie nechetkikh sistem po metodu statisticheski vzveshennykh sindromov 
i~ego primenenie dlya immunogematologicheskoy kha\-rak\-te\-ri\-sti\-ki normy i~khronicheskoy 
patologii
[Recognition of fuzzy systems by method of statistically weighed syndromes 
and its using for immunological and hematological norm and chronic pathology].
\textit{Khim. Fiz.} 15(1):81--100.
\bibitem{5-dok-1}
\Aue{Knerr, S., L.~Personnaz, and G.~Dreyfus}. 1990. 
Single-layer learning revisited: A~stepwise procedure for building and training 
neural network. 
\textit{Neurocomputing: Algorithms, architectures and applications}. Eds. F.\,F.~Soulie and 
J.~Herault.
NATO ASI subser. F. Berlin--Heidelberg:
    Springer-Verlag. 68:41--50.

\bibitem{6-dok-1}
\Aue{Dietterich, T.\,G., and G.~Bakiri}. 1995. Solving multiclass 
learning problems via error-correcting output codes. 
\textit{J.~Artif. Intell. Res.} 2:263--286.
\bibitem{7-dok-1}
\Aue{Allwein, E., R.~Shapire, and Y.~Singer}. 2000. Reducing multi-class to binary: 
A~unifying approach for margin classifiers. 
\textit{J.~Mach. Learn. Res.} 1(1):113--141.
\bibitem{8-dok-1}
\Aue{D'yakonov, A.\,G.} 2005. Algebra over estimation algorithms: 
The minimal degree of correct algorithms.
\textit{Comp. Math. Math. Phys.} 45(6):1095--1106.
\bibitem{9-dok-1}
\Aue{Dokukin, A.\,A.} 2001. The construction of a recognition algorithm in 
the algebraic closure. 
\textit{Comp. Math. Math. Phys.} 41(12):1907--1911.
\bibitem{10-dok-1}
\Aue{Dokukin, A., V.~Ryazanov, and O.~Shut}. 2016. Multilevel
 models for solution of multiclass recognition problems. 
 \textit{Pattern Recognition Image Anal.} 26(3):461--473.
\bibitem{11-dok-1}
\Aue{Giarratano, J.\,C., and G.\,D.~Riley}. 2004. \textit{Expert systems: 
Principles and programming}.  Boston, MA: PWS Publ. Co. 856~p.
\bibitem{12-dok-1}
\Aue{Krasnoproshin, V.\,V., and V.\,A.~Obraztsov}. 1998. 
Ras\-po\-zna\-va\-nie s~obucheniem kak zadacha vybora
[Supervised recognition as selection problem]. 
\textit{Tsifrovaya obrabotka izobrazheniy} [Digital image processing]. 
Minsk: ITK. 80--94.
\bibitem{13-dok-1}
\Aue{Ablamejko, S.\,V., V.\,V.~Krasnoproshin, and V.\,A.~Ob\-raz\-tsov}. 
2011. Modeli i~tehnologii ras\-po\-zna\-va\-niya obrazov s~prilozheniem v~intellektual'nom
analize dannykh [\mbox{Models} and technologies of pattern recognition with 
application to data mining]. 
\textit{Vestnik BGU. Ser.~1, Fizika, Ma\-te\-ma\-ti\-ka, Informatika}
[BSU Herald. Ser.~1, Physics, Mathematics, Informatics] 3:62--72.
\bibitem{14-dok-1}
\Aue{Kaufman, A.} 1975. \textit{Introduction to the theory of fuzzy subsets.}
New York, NY: Academic Press.
432~p.
\bibitem{15-dok-1}
\Aue{Lee, R.\,C.\,T.} 1972. Fuzzy logic and the resolution principle. 
\textit{J.~ACM} 19(1):109--119.
%\bibitem{16-dok-1}
%\Aue{Berger, A.} 1999. Error-correcting output coding for text classification. 
%\textit{IJCAI: Workshop on Machine Learning For Information Filtering Proceedings}. 
%Stockholm. 8~p. 
\bibitem{17-dok-1}
\Aue{Lichman, M.} 2013. \textit{UCI machine learning repository}. 
Irvine, CA: University of California, School of Information and Computer Science.
Available at: {\sf http://archive.ics.uci.edu/ml}
(accessed March~3, 2017). 

\bibitem{18-dok-1}
\Aue{Pedregosa, F., G.~Varoquaux, A.~Gramfort, V.~Michel, B.~Thirion, O.~Grisel, 
M.~Blondel, P.~Prettenhofer, R.~Weiss, V.~Dubourg, J.~Vanderplas, A.~Passos, 
D.~Cournapeau, M.~Brucher, M.~Perrot, and E.~Duchesnay}. 2011. Scikit-learn: 
Machine learning in Python. \textit{J.~Mach. Learn. Res.} 12:2825--2830.
\end{thebibliography}

 }
 }

\end{multicols}

\vspace*{-3pt}

\hfill{\small\textit{Received August 2, 2016}}

\Contr

\noindent
\textbf{Dokukin Alexander A.}\ (b.\ 1980)~--- 
PhD in physics and mathematics, senior scientist, Federal Research Center 
``Computer Science and Control'' of the Russian Academy of Sciences,
 40~Vavilov Str., Moscow, 119333, Russian Federation; \mbox{dalex@ccas.ru} 
 
 \vspace*{3pt}

\noindent
\textbf{Ryazanov Vasiliy V.}\ (b.\ 1991)~--- 
PhD student, Moscow Institute of Physics and Technology, 9~Institutskiy Per., 
Dolgoprudny, Moscow Region 141700, Russian Federation; \mbox{vasyarv@mail.ru}

\vspace*{3pt}


\noindent
\textbf{Shut Olga V.}\ (b.\ 1987)~--- 
assistant professor, Belarusian State University, 4~Nezavisimosti Av., 
Minsk 220030, Republic of Belarus; \mbox{olgashut@tut.by}


\label{end\stat}


\renewcommand{\bibname}{\protect\rm Литература}  %6
 \def\stat{gor+lub}

\def\tit{АЛГОРИТМ ПРЕОБРАЗОВАНИЯ ОДНОГО ГРАФА В~ДРУГОЙ С~МИНИМАЛЬНОЙ 
ЦЕНОЙ$^*$}

\def\titkol{Алгоритм преобразования одного графа в~другой с~минимальной 
ценой}

\def\aut{К.\,Ю.~Горбунов$^1$, В.\,А.~Любецкий$^2$}

\def\autkol{К.\,Ю.~Горбунов, В.\,А.~Любецкий}

\titel{\tit}{\aut}{\autkol}{\titkol}

\index{К.\,Ю.~Горбунов$^1$, В.\,А.~Любецкий$^2$}


{\renewcommand{\thefootnote}{\fnsymbol{footnote}} \footnotetext[1]
{Работа выполнена при финансовой поддержке Российского 
научного фонда (проект 14-50-00150).}}


\renewcommand{\thefootnote}{\arabic{footnote}}
\footnotetext[1]{Институт проблем передачи информации им.\ А.\,А.~Харкевича Российской академии наук, 
\mbox{gorbunov@iitp.ru}}
\footnotetext[2]{Институт проблем передачи информации им.\ А.\,А.~Харкевича Российской академии наук;
ме\-ха\-ни\-ко-ма\-те\-ма\-ти\-че\-ский факультет Московского государственного университета 
им.\ М.\,В.~Ломоносова, \mbox{lyubetsk@iitp.ru}}


\Abst{Рассматриваются ориентированные графы, состоящие из любого числа дизъюнктных 
цепей и~циклов, реб\-рам графов приписаны без повторений их имена~--- натуральные числа. 
Фиксирован список операций, каждая из которых по-сво\-ему преобразует один граф 
в~другой, ей приписано число~--- цена данной операции. Нужно найти минимальную по 
суммарной цене последовательность операций, которая для двух данных графов преобразует 
один в~другой. Эта задача самым широким образом применяется в~прикладных вопросах.  
По-ви\-ди\-мо\-му, она является NP-трудной и~поэтому может быть эффективно решена 
только при том или ином условии на цены или при некотором ограничении на графы. Ее 
решение при достаточно широких условиях получено в~виде линейных по времени 
и~памяти алгоритмов, для которых доказана точность (неэвристичность), т.\,е.\ доказано, что 
они всегда находят минимальную по цене последовательность операций. Задача давно 
решается многими эвристическими алгоритмами, которые тестировались на разных данных, 
но предлагаемые авторами решения~--- первые среди точных.}

\KW{ориентированный граф из цепей и~циклов; преобразование графов с~минимальной 
ценой; точное линейное решение; условие на графы; условие на цены; условно кратчайшее 
решение}

\DOI{10.14357/19922264170107}  


\vskip 10pt plus 9pt minus 6pt

\thispagestyle{headings}

\begin{multicols}{2}

\label{st\stat}

  \section{Введение. Постановка задачи}
   
  \subsection{CC-графы}
  
  В работе рассмотрена следующая  
ком\-би\-на\-тор\-но-оп\-ти\-ми\-за\-ци\-он\-ная задача. Назовем  
\textit{СС-гра\-фом} ориентированный граф, состоящий из любого чис\-ла 
дизъюнктных цепей и~циклов, включая петли, реб\-рам которого приписаны без 
повторений натуральные числа (\textit{имена} ре\-бер). Цепи и~циклы являются 
(без учета ориентации) компонентами связности такого графа, которые будем 
называть \textit{компонента\-ми}. Фиксирован список операций, которые 
преобразуют один СС-граф в~другой такой же граф. 

Можно рассматривать более 
общий случай графов и~любой список операций, но в~длительной истории 
исследования этой задачи (по разным, прежде всего прикладным, причинам) 
сформировалось указанное определение графа и~тот список операций, который 
приведен в~подразд.~1.2. Каж\-дой операции приписано число, которое 
называется ее \textit{ценой}. В~прикладных задачах цены являются строго 
положительными рациональными числами, но, разумеется, теоретически их 
можно считать натуральными числами. Любой последовательности операций, 
применяемых друг за другом, начиная с~данного СС-гра\-фа~$a$ и~заканчивая 
некоторым результирующим СС-гра\-фом~$b$, приписывается 
\textit{суммарная цена}~--- сумма цен всех операций в~этой 
последовательности. 

\begin{figure*}[b] %fig1
\vspace*{12pt}
\begin{center}
\mbox{%
\epsfxsize=101.723mm
\epsfbox{lub-1.eps}
}
\end{center}
\vspace*{-9pt}
  \Caption{Четыре стандартные операции над CC-графом:
  (\textit{а})~двойная переклейка;
  (\textit{б})~полуторная переклейка;
  (\textit{в})~разрез и~склейка}
  %\end{figure*} 
  %\begin{figure*} %fig2
\vspace*{12pt}
\begin{center}
\mbox{%
\epsfxsize=95.006mm
\epsfbox{lub-2.eps}
}
\end{center}
\vspace*{-9pt}
  \Caption{Две дополнительные операции над CC-графом:
  удаление и~вставка}
  \end{figure*}
  
  Итак, пусть даны два СС-гра\-фа~$a$ и~$b$. Требуется найти минимальную 
по функционалу суммарной цены последовательность операций, которая 
преобразует~$a$ в~$b$. Такую последовательность называют 
\textit{кратчайшей}, а~ее цену~--- \textit{кратчайшей ценой}. Предполагается, 
хотя это не доказано, что задача на\-хож\-де\-ния кратчайшей последовательности 
или кратчайшей цены для переменных~$a$, $b$ и~переменных цен операций 
является NP-труд\-ной. Задача остается таковой, если фиксировать 
произвольные (случайные) цены операций. Поскольку практический интерес 
представляют линейные или, во всяком случае, полиномиальные алгоритмы 
низкой степени, для их поиска приходится накладывать условия на 
соотношение цен или на вид графов. В~части второго популярно такое 
ограничение: в~последовательности операций, которая преобразует~$a$ в~$b$, 
включая и~сами~$a$ и~$b$, присутствует один и~тот же постоянный набор 
имен. Задачу с~этим ограничением назовем задачей с~\textit{постоянным  
со\-ста\-вом} (имен ребер). В отсутствие этого ограничения задачу назовем 
задачей с~\textit{переменным составом}. В~разд.~2 приводится схема 
линейного по времени и~памяти алгоритма ее решения в~трех случаях: два 
относятся к~постоянному составу и~один к~переменному составу. Все нюансы 
работы алгоритма, со\-про\-вож\-да\-емые рисунками, а также детали доказательств 
приведены в~\cite{1-gor, 2-gor}.
  
  \subsection{Операции над CC-графами}
  
   Фиксируются следующие операции, на\-зы\-ва\-емые \textit{стандартными}:
   \begin{itemize} 
\item разрезать две вершины, имеющиеся в~графе, и~по-но\-во\-му отождествить 
(склеить) четыре образовавшихся края (\textit{двойная переклейка}) 
(рис.~1,\,\textit{a});
\item разрезать вершину и~по-но\-во\-му отождествить (склеить) 
один образовавшийся край с~ка\-ким-то свободным краем в~графе 
(\textit{полуторная переклейка}) (рис.~1,\,\textit{б}); 
\item разрезать вершину или 
отождествить два свободных края (одинарные переклейки~--- \textit{разрез} 
и~\textit{склейка}) (рис.~1,\,\textit{в}).
\end{itemize}
  

  
  Также фиксируются две \textit{дополнительные операции}, называемые 
соответственно \textit{вставкой} и~\textit{удалением}: до\-ба\-вить/уда\-лить 
цепь~$X$ в~граф или из графа (рис.~2).
  

  
  Для вставки это означает: добавить в~граф саму цепь~$X$ или ее же 
с~отождествлением ее концов, т.\,е.\ добавить новую цепь или новый цикл; или 
добавить цепь~$X$ в~цепь, уже имеющуюся в~графе, вместо ее конца или ее 
внутренней вершины; аналогично для вставки в~цикл. И~симметрично для 
операции удаления. При этом добавлять можно цепь, имена которой содержатся 
в~$b\backslash a$, а удалять можно цепь, имена которой содержатся 
в~$a\backslash b$. В~\cite{3-gor} предложенный авторами алгоритм был 
применен в~конкретном прикладном исследовании, и~там содержатся более 
подробные, чем в~подразд.~1.3, сведения по истории задачи, однако содержание 
работы~\cite{3-gor} не используется в~данной статье.
  
  \subsection{Обзор непосредственно примыкающей литературы}
  
  Приведем несколько математических результатов других авторов по этой 
задаче. Таких результатов немного. В~случае \textit{постоянного} состава 
и~\textit{различных} цен алгоритм решения с~его математическим 
обосно\-ва\-ни\-ем не был предложен. В~работах~[4, 5] рассматривается сразу 
\textit{переменный состав}. В~\cite{4-gor} на цены накладывается условие: 
у~всех стандартных операций они равны~1, а~у~операций встав-\linebreak ки и~удаления 
равны и~не больше~1. В~\cite{5-gor} на цены накладывается другое, более 
слабое условие~--- опущено <<не больше~1>>, но зато рассматриваются  
CC-гра\-фы только из циклов. По существу, последнее условие эквивалентно 
тому, что из стандартных операций допускается только двойная переклейка. 
Насколько авторы понимают, результат в~\cite{4-gor} не содержит корректных 
доказательств, так что указанный случай не получил обоснования. Подходы, 
которые предложены в~[4,~5] (т.\,е.\ определения вспомогательных графов~--- 
прием, который используется во всех работах на эту тему, включая и~данную), 
отличаются от предлагаемого авторами статьи.
{\looseness=1

} 
  
  В~\cite{2-gor} авторы описали линейный по времени и~памяти алгоритм 
построения кратчайшей последовательности операций, преобразующих один 
CC-граф~$a$ в~другой CC-граф~$b$ для случая \textit{переменного} состава, 
если цены всех операций равны. Если все цены равны, кратчайшую 
последовательность называют \textit{минимальной}, а~кратчайшую цену~--- 
\textit{минимальной ценой}. Конечно, главный интерес представляет общий 
случай задачи, в~котором цены не равны. 

  
  Как и~в~\cite{2-gor}, в~разд.~2 изложение строится по такому плану. 
Определяется понятие общего графа $a\hm+b$, иное, нежели  
в~\cite{4-gor, 5-gor}; показано, что исходная задача эквивалентна приведению 
$a\hm+b$ к~специальному виду, который называется \textit{финальным}, 
аналогами операций, которые определены в~подразд.~1.2. Затем описывается 
алгоритм и~приводится доказательство его точности (неэвристичности, 
корректности), т.\,е.\ доказательство того, что он действительно находит 
минимум суммарной цены. Как отмечалось выше, некоторые технически 
громоздкие детали в~описании и~доказательстве приведены в~\cite{1-gor}. 
Вместо <<неэвристический алгоритм>> или <<неэвристическое решение>> 
иногда говорят \textit{точный алгоритм} или \textit{точное решение} 
соответственно.

  
  В исследованиях этой задачи используется различная терминология. Так, 
в~\cite{2-gor} CC-граф называется \textit{структурой}; для согласования с~этой 
работой будем далее использовать последний, более привычный термин. 
В~\cite{3-gor} и~в~других работах CC-граф (структура) называется 
хромосомной структурой, его реб\-ра называются генами, а~компоненты~--- 
хромосомами. Это связано с~тем, что задача возникла в~контексте 
биоинформатики, где для ее решения разработано большое число более или 
менее эвристических алгоритмов. В~\cite{3-gor} приводится краткий обзор по 
истории исследований задачи. 

  
  Напомним, что минимальная последовательность и~минимальная цена 
являются частными случаями кратчайшей последовательности и~кратчайшей 
цены. 
  
  \section{Решение задачи }
  
  \subsection{Общий граф и~идея алгоритма}
  
  \textit{Общий граф} $a\hm+b$ двух структур~$a$ и~$b$ имеет следующие 
вершины. \textit{Обычные} вершины~--- имена краев одноименных ребер в~$a$ 
и~$b$; например, начало реб\-ра с~именем~3 будет иметь имя~3$_1$. 
И~\textit{особые вершины}~--- максимальные по включению связные участки 
из ребер, принадлежащих лишь одной из структур, которые называют 
\textit{блоками}. Блок принадлежит одной из структур, и~соответствующая 
особая вершина помечается как $a$- или $b$-вер\-ши\-на. Реб\-ра общего графа 
следующие. \textit{Обычное} реб\-ро соединяет две обычных вершины, если 
соответствующие им края отождествлены (склеены) в~$a$ или в~$b$. 
А~\textit{особое} реб\-ро соединяет обычную вершину с~особой, если в~$a$ или 
в~$b$ край, со\-от\-вет\-ст\-ву\-ющий обычной вершине, отож\-дествлен (\textit{склеен}) 
с~краем блока, со\-от\-вет\-ст\-ву\-юще\-го особой вершине. Такое реб\-ро помечается как 
$a$- или $b$-реб\-ро. \textit{Петля} в~$a\hm+b$ соответствует циклу, который 
является блоком; иными словами, особая вершина этого блока соединяется 
с~собой. \textit{Висячим} называется реб\-ро, инцидентное особой вершине 
степени~1. Пример двух структур и~их общего графа приведен в~\cite{3-gor} на 
рис.~1 и~2. Таким образом, общий граф несет информацию о~склейках 
одновременно в~$a$\linebreak и~в~$b$.
  
  Общий граф~--- неориентированный, он состоит из связных компонент~--- 
также цепей и~циклов. Невисячие особые реб\-ра присутствуют в~нем парами~--- 
реб\-ра\-ми, инцидентными одной особой вершине; такую пару удобно считать за 
одно двойное ребро. Поэтому \textit{размером компоненты} назовем сумму 
в~ней числа обычных ребер с~половиной числа особых невисячих ребер 
(в~\cite{2-gor} эта величина названа длиной компоненты, что вызывает 
путаницу с~обычной длиной цепи или цикла). Для изолированных обычных 
вершин и~петель размер считаем равным~0, для изолированных особых вершин 
(не петель)~--- равным~$-1$. Общий граф называется 
\textit{финального вида}, если каждая его компонента~--- изолированная 
обычная вершина или цикл без особых ребер размера (или в~данном случае то 
же самое~--- длины)~2, одно ребро из~$a$ и~другое из~$b$. 

\begin{figure*}[b] %fig3
\vspace*{12pt}
\begin{center}
\mbox{%
\epsfxsize=82.146mm
\epsfbox{lub-3.eps}
}
\end{center}
\vspace*{-9pt}
\Caption{Операции, разрешенные над общим графом:
(\textit{а})~двойная переклейка;
(\textit{б})~полуторная переклейка;
(\textit{в})~разрез и~склейка;
(\textit{г})~$a$- или $b$-удаление. Большой кружок показывает особую 
вершину. Отметим: операция вставки оказывается ненужной, но зато операция удаления 
применяется в~двух вариантах: $a$-уда\-ле\-ния и~$b$-уда\-ле\-ния особой вершины; 
первая~--- 
по цене удаления, вторая~--- по цене вставки}
\end{figure*}
  
  Эти определения приведены в~\cite{2-gor, 3-gor} и~частично в~\cite{6-gor} 
и~здесь повторяются для удобства читателя. 
  
  В~\cite{2-gor} доказано: для любых структур~$a$ и~$b$ существует 
кратчайшая последовательность, в~которой все удаления предшествуют всем 
вставкам и~все операции сохраняют блоки без изменения. Хотя там считалось, 
что цены всех операций равны, легко проверить, что приведенное 
доказательство сохраняется без изменения, если равны только цены\linebreak 
стандартных операций. Действительно, в~том доказательстве произвольная 
кратчайшая последовательность преобразуется в~последовательность 
указанного вида, при этом стандартная операция перехо\-дит в~стандартную, 
удаление~--- в~удаление, вставка~--- во вставку. Поэтому как там, так и~здесь 
суммарная цена (иногда будем говорить~--- цена) последовательности не 
меняется.


  
  Из этого \textit{утверждения} следует: в~предположении одинаковых цен 
стандартных операций задача поиска кратчайшей последовательности для 
структур~$a$ и~$b$ эквивалентна задаче приведения этих структур  
к~ка\-кой-ни\-будь одной структуре~$c$ двумя суммарно кратчайшими 
последовательностями, причем вставка не используется, а~цена удаления, 
применяемого в~преобразованиях~$b$, равна цене вставки. 

Действительно, 
если~$c$~--- структура в~кратчайшей последовательности, полученная после 
выполнения всех удалений и~до всех вставок, то можно преобразовать к~ней 
структуру~$a$ (прямыми операциями) и~структуру~$b$ (обратными 
операциями). А~последняя задача эквивалентна задаче преобразования общего 
графа~$a\hm+b$ к~финальному виду~$c\hm+c$ следующими аналогами 
исходных операций (рис.~3).
  

  
  \textit{Двойная переклейка} (рис.~3,\,\textit{a}): удаление двух одинаково 
помеченных ребер общего графа и~соединение четырех образовавшихся концов 
двумя новыми неинцидентными реб\-ра\-ми с~той же пометкой. Если при этом 
образуется ребро с~особыми концами (оба относятся к~$a$ или оба к~$b$), то 
оно заменяется одной особой вершиной, которой приписано объединение 
блоков двух исходных особых вершин. 

\textit{Полуторная переклейка}
(рис.~3,\,\textit{б}): удаление реб\-ра общего графа и~соединение ребром с~той же 
пометкой, скажем~$a$, одного из его концов с~обычной вершиной, не 
инцидентной ребру с~этой пометкой, или с~особой вершиной степени не 
больше~1 с~той же пометкой (с~возможным последующим отождествлением 
двух особых вершин). 

\textit{Склейка} (рис.~3,\,\textit{в}): добавление реб\-ра 
(скажем, с~пометкой~$a$) между вершинами, каждая из которых является или 
обычной, не инцидентной ребру с~пометкой~$a$, или особой, степени не 
больше~1, с~той же пометкой (с~возможным последующим отождествлением 
двух особых вершин). 

\textit{Разрез} (рис.~3,\,\textit{в}): удаление любого 
реб\-ра. 

\textit{Удаление особой вершины} (рис.~3,\,\textit{г}): если ее степень~2, 
то она удаляется и~инцидентные ей реб\-ра склеиваются в~одно реб\-ро с~той же 
пометкой; если ее степень~1, то она удаляется вместе с~инцидентным ей 
ребром; если ее степень~0 или это пет\-ля, то вершина с~пет\-лей удаляются. 
Удаление особой $a$-вер\-ши\-ны получает цену операции удаления, удаление 
особой $b$-вер\-ши\-ны~--- цену операции встав\-ки. Условимся далее 
в~выражении особая~$a$- или $b$-вер\-ши\-на опускать слово <<особая>>.
  
  Забегая вперед, опишем \textit{идею предлагаемого алгоритма}. В~случае 
\textit{постоянного} состава в~общем графе имеются лишь обычные вершины 
и~реб\-ра. Алгоритм приводит его к~финальному виду в~два этапа. Первый 
  этап~--- двойными переклейками разбить все циклы на циклы длины~2. 
Второй этап~--- обработка цепей: если цена двойной переклейки меньше цены 
полуторной, замкнуть цепи в~циклы и~обработать их, как на этапе~1. Иначе 
нужно полуторными переклейками от цепи пошагово отщеплять циклы 
длины~2.
  
  В случае \textit{переменного} состава алгоритм из компонент вырезает 
обычные реб\-ра, замыкая их в~циклы длины~2. Затем обрабатываются цепи. Это 
связано с~тем, что совместная обработка цепей позволяет экономить число 
операций по сравнению с~тем их числом, которое получилось бы при обработке 
каждой цепи в~отдельности; это~--- основная идея алгоритма. Совместная 
обработка цепей описана в~\cite{2-gor}, где доказано, что она приводит 
к~максимально возможной экономии числа операций, т.\,е.\ к~минимизации 
числа операций. 

Ниже в~описании алгоритма совместная обработка цепей 
выполняется на шаге~3, каждый пункт которого соответствует определенной 
совместной обработке (можно сказать~--- взаимодействию) цепей. После 
шага~3 общий граф может еще содержать цепи, а также в~нем остаются 
исходные циклы. Между этими компонентами уже невозможны 
взаимодействия, которые экономили бы число операций. Но возможны 
взаимодействия, заменяющие дорогое удаление $b$-вер\-ши\-ны на операцию 
с~меньшей ценой. Эти взаимодействия описаны на шаге~4 алгоритма.
  
  \subsection{Приведение общего графа в~случае постоянного состава 
и~разных цен операций}

\vspace*{-2pt}
  
   Рассмотрим случай постоянного состава, вставки и~удаления отсутствуют, 
цены операций различны. Точнее, предполагается, что цены операций 
удовлетворяют одному из двух условий: $c_2\hm\leq c_1\hm\leq 
c_1^\prime\hm\leq c_{1{,}5}$ (\textit{циклический} вариант) и~$c_1\hm\leq 
c_1^\prime\hm\leq c_{1{,}5}\hm\leq c_2$ (\textit{линейный} вариант). Здесь 
указаны \mbox{соотношения} между ценами разреза~$c_1$, склейки~$c_1^\prime$, 
полуторной переклейки~$c_{1{,}5}$, двойной переклейки~$c_2$. 
  
  В этом случае предлагается решение несколько \textbf{измененной задачи}: 
кратчайшая последовательность ищется среди всех минимальных 
последовательностей. Она называется \textit{условно кратчайшей}. Решение 
задачи в~этом смысле будем называть \textit{условной оптимизацией}. Авторы 
не знают, существует ли полиномиальный по времени алгоритм решения 
безусловной задачи даже при одном из указанных соотношений цен, если 
только не все цены равны. Конечно, если они равны, то \textit{условная} 
оптимизация совпадает с~\textit{безусловной}, т.\,е.\ с~решением исходной 
задачи. 
  
  В рассматриваемом случае общий граф состоит из циклов и~цепей, в~которых 
чередуются~$a$- и~$b$-реб\-ра. В~случае постоянного состава 
\textit{качеством} $H(a+b)$ общего графа $a\hm+b$ назовем число циклов, 
сложенное с~половиной числа четных цепей в~нем. Четной называется цепь 
с~четным числом ребер, а~так\-же цепь размера ноль; цепи нечетного размера не 
учитываются; понятия размера и~длины в~этом пункте совпадают. Пусть 
структуры~$a$ и~$b$ имеют по~$n$~ребер. Для построения минимальной 
последовательности решающее значение имеет возрастание качества от 
значения $H(a+b)$ до значения~$n$ на~$+1$ при выполнении каждой 
операции; таким образом, минимальную длину можно указать сразу: она равна 
$n\hm- H(a\hm+b)$. 
  
  \smallskip
    
  \noindent
  \textbf{Лемма~1.}
  \begin{enumerate}[1.]
\item  \textit{Каждая стандартная операция изменяет качество общего графа 
на~$0$ или}~$\pm1$. %\\[-15pt]
  \item \textit{Для нефинального графа существует операция, увеличивающая 
его качество на}~1. %\\[-15pt]
  \item\textit{Граф $a+b$ финальный, если и~только если $a\hm=b$; для 
финального графа $a\hm+b$ выполняется} $H(a\hm+b)\hm=n$. %\\[-15pt] 
  \item  \textit{Как безусловная, так и~условная задачи для~$a$ и~$b$ 
эквивалентны соответствующим задачам о~приведении общего 
графа~$a\hm+b$ к~финальному виду}. %\\[-15pt]
\item \textit{Существует последовательность операций, преобразующая 
$a\hm+b$ к~финальному виду, на каждом шаге которой качество 
увеличивается ровно на~$1$; ее длина равна} $k \hm= n\hm- H(a\hm+b)$.
  \item \textit{Минимальная длина равна}~$k$.
  \item \textit{Минимальными последовательностями для $a\hm+b$ являются 
в~точности те, у~которых каждая операция увеличивает качество общего 
графа на~$1$. Их длины равны}~$k$.
  \end{enumerate}
  
  \noindent
  Простое д\,о\,к\,а\,з\,а\,т\,е\,л\,ь\,с\,т\,в\,о\ леммы~1 приведено  
в~\cite[п.~3]{1-gor}.
  
  \smallskip
  
  Опишем точный линейный алгоритм приведения к~финальному виду 
в~случаях циклического и~линейного соотношений цен. Пункт~3 в~\cite{1-gor} 
содержит рисунки, наглядно поясняющие его работу. Для \textbf{циклического 
варианта} он состоит из трех шагов.
  \begin{description}
  \item[Шаг~1.] Если имеется цикл длины, строго большей двух, двойной 
переклейкой разбиваем его на два цикла, один из которых имеет длину~2.
  \item[Шаг~2.] Склейкой каждую нечетную цепь замыкаем в~цикл, после чего 
применяем шаг~1.
  \item[Шаг~3.] Полуторной переклейкой каждую ненулевую четную цепь 
замыкаем в~цикл, один край цепи становится нулевой цепью. Затем применяем 
шаг~1.
  \end{description}
  
  Алгоритм решения условной задачи для \textbf{линейного варианта} состоит 
также из трех шагов.
  \begin{description}
  \item[Шаг~1.] Тот же, что и~в предыдущем алгоритме. 
  \item[Шаг~2.] Разрезом от каждой нечетной цепи отделяем крайнюю 
вершину, получаем четную цепь на~1~меньшей длины и~нулевую цепь.
  \item[Шаг~3.] Полуторной переклейкой каждую ненулевую четную цепь 
укорачиваем на~2~реб\-ра и~замыкаем два ее крайних реб\-ра в~цикл, пока 
в~общем графе не останется ненулевых цепей.
  \end{description}
  
  Если ни один шаг не применим, то общий граф уже имеет финальный вид 
и~к~нему применяется пустая последовательность операций.~$\square$
  
  Отметим: от~\cite{4-gor, 5-gor} приведенное ниже доказательство теорем~1 
и~2 отличается другими условиями на цены, использованием другого 
вспомогательного графа, меньшего размера, и~индукцией по величине~$C(G)$ 
общего графа, которая и~со\-став\-ля\-ет суть приводимых доказательств (не говоря 
об отсутствии полного доказательства в~\cite{4-gor, 5-gor}).
  
  \smallskip
  
  \noindent
  \textbf{Теорема~1.}\ \textit{Указанные линейные алгоритмы точно решают 
задачу условной оптимизации для циклического и~линейного вариантов цен.}
  
\columnbreak


  \noindent
  С\,х\,е\,м\,а\ д\,о\,к\,а\,з\,а\,т\,е\,л\,ь\,с\,т\,в\,а\,.\ \ Минимальность 
полученной последовательности следует из леммы~1. Из нее же следует 
линейность алгоритма по вре\-мени. 
  
  Докажем, что полученная последовательность~--- кратчайшая. Для этого 
выразим суммарную цену $c(G)$ в~полученной алгоритмом последовательности 
через числовые характеристики графа~$G$ (подробности приведены 
в~\cite[п.~3]{1-gor}). Кратчайшую цену для приведения графа~$G$ 
к~финальному виду \textit{обозначим}~$C(G)$. Индукцией по 
величине~$C(G)$ покажем, что для всех графов~$G$ выполняется неравенство 
$c(G)\hm\leq C(G)$. Отсюда $c(G)\hm=C(G)$, что и~требуется. Число вершин 
в~графе~$G$ фиксировано, поэтому множество минимальных цен конечно. 
Индукция идет по естественному порядку в~этом множестве цен. Если 
$C(G)\hm= 0$, то граф~$G$ финального вида и~$c(G)\hm=0$.
  
  \textbf{Индуктивный шаг.} Пусть для всех графов~$G^\prime$, у~которых 
$C(G^\prime)\hm<C(G)$, выполняется неравенство $c(G)\hm\leq C(G)$. 
Докажем его для~$G$. Рассмотрим приводящую последовательность для~$G$. 
Обозначим через~$o$ ее первую операцию, $c(o)$~--- ее цену, $o(G)$~--- 
результат ее применения к~$G$. Достаточно проверить неравенство 
$c(o^\prime)\hm\geq c(G)\hm- c(o^\prime(G))$ для каждой операции~$o^\prime$. 
Действительно, по предположению индукции имеем: $c(o(G))\hm\leq C(o(G))$. 
Отсюда $c(G)\hm\leq c(o(G))\hm+ c(o) \hm\leq C(o(G))\hm +c(o)\hm= C(G)$. 
Подробности проверки приведены в~\cite[п.~3]{1-gor}.~$\square$

%\vspace*{-9pt}
  
  \subsection{Приведение общего графа в~случае переменного состава 
и~разных цен операций}
  
   Дан общий граф $a+b$ и~число~$\varepsilon$, $0\hm\leq\varepsilon\hm\leq1$. 
Разрешены все операции, т.\,е.\ в~силу утверждения в~подразд.~1.1 стандартные 
операции, удаления~$a$- и~$b$-вер\-шин (особых вершин с~пометкой~$a$ 
или~$b$). Пусть цены стандартных операций и~$a$-уда\-ле\-ния равны~1, а цена 
$b$-уда\-ле\-ния вершины равна $1\hm+\varepsilon$. Термин \textit{конец}, 
естественно, относится к~концу реб\-ра или к~изолированной вершине в~общем 
графе. 
  
  Предлагаемый алгоритм компьютерно тестировался в~общем случае, если 
цена $b$-уда\-ле\-ния больше цены всех других операций. Как правило, 
алгоритм находил ответ, близкий к~кратчайшей последовательности. Эта более 
общая ситуация здесь не рассматривается, но с~учетом возможного 
эвристического использования в~описание алгоритма, которое приведено ниже, 
включены соответствующие пояснения; они не используются в~доказательстве, 
которое также приводится ниже. 
  
\pagebreak 
  
  \textbf{Краткое описание алгоритма}
  
  \begin{description}
  \item[Шаг~1.] Удалить особые $a$-петли. 
  \item[Шаг~2.] Вырезать все обычные реб\-ра, не входящие в~2-цик\-лы (т.\,е.\ 
циклы размера~2), замыкая их в~финальные~2-цик\-лы двойной (если ребро не 
крайнее) или полуторной (если оно крайнее) переклейками или склейкой (если 
оно изолированное). В~\cite[п.~4]{1-gor} содержится подробное описание 
работы алгоритма на шагах~2 и~3 с~рисунками.
  \item[Шаг~3.] Фактически этот шаг тот же, что и~в~\cite{2-gor} (более 
подробно он описан в~\cite[п.~4]{1-gor}). Напомним его смысл. В~множестве 
цепей общего графа (после шага~2) выделяются небольшие попарно 
непересекающиеся подмножества мощ\-ности от~2 до~4. Внутри каждого 
подмножества~$M$ производится ($\vert M\vert \hm-1$) операций между 
цепями (взаимодействий) так, что если каждую цепь из~$M$ приводить 
к~финальному виду автономно (т.\,е.\ без взаимодействий с~другими 
компонентами), то число требуемых операций будет строго больше числа 
операций, тре\-бу\-емых, если сначала провести данное взаимодействие. 
Доказывается, что описанное множество взаимодействий дает максимально 
возможную экономию числа операций.
  \item[Шаг~4.] На этом шаге в~определенном порядке производятся 
взаимодействия между связными компонентами общего графа. Каждое 
взаимодействие производится до тех пор, пока есть компоненты, которые могут 
служить его аргументами. Эти взаимодействия не уменьшают общее число 
операций (точнее, сохраняют его), но позволяют заменить <<дорогую>> 
операцию удаления $b$-вер\-ши\-ны на другую, более дешевую операцию. 
Например, если удалить две $b$-пет\-ли по отдельности, будет произведено две 
операции удаления $b$-вер\-ши\-ны, если же сначала двойной переклейкой 
слить эти две петли в~одну (это частный случай взаимодействия~4.1  
из~\cite{1-gor}), то одно удаление заменится на двойную переклейку. Подробно 
шаг~4 описан в~\cite[п.~4]{1-gor}.
  \item[Шаг~5.] Удаляем изолированные особые вершины и~петли. Из 
оставшихся цепей удаляем особые вершины. Из циклов размера, большего~2, 
вырезаем~2-цик\-лы так, чтобы происходило отож\-де\-ст\-вле\-ние двух  
$b$-вер\-шин (соответственно,\linebreak в~2-цикл включается $a$-вер\-ши\-на).  
Из~2-цик\-лов удаляем особые вершины.
  \end{description}
  
  Конец описания алгоритма.~$\square$
  \smallskip
  
  Д\,о\,к\,а\,ж\,е\,м\ теорему о~минимальности суммарной цены 
последовательности операций, которая получается в~алгоритме, т.\,е.\ 
о~точ\-ности (корректности) алгоритма. 
  
  Пусть $B^\prime$~--- число циклов в~графе $a\hm+b$, содержащих  
$b$-вер\-ши\-ну, но не содержащих $a$-вер\-ши\-ну (назовем их  
$b$-цик\-ла\-ми). Напомним обозначения из~\cite{2-gor}: $B$~--- число особых 
вершин в~$a\hm+b$; $S$~--- сумма целых частей половин числа ребер (назовем 
число \textit{длиной}) максимальных отрезков (\textit{сегментов}) в~$a\hm+b$, 
которые состоят из обычных ребер, плюс число нечетных (т.\,е.\ нечетной 
длины) крайних сегментов минус число циклических сегментов. 
\textit{Крайним} называется сегмент, расположенный с~краю цепи, включая 
и~случай целой цепи. Обычной называется пара, состоящая из одной из 
стандартных операций вместе с~ее аргументом, результат которой не меняет 
число особых вершин. Далее <<обычная>> относится к~операции, а ее аргумент 
подразумевается заданным. \textit{Дефект} цепи (или цикла) равен 
минимальному числу обычных операций в~последовательности, которая 
приводит ее (или его) к~финальному виду, не считая вырезания обычных ребер 
на шаге~2; в~последовательности могут встречаться и~операции с~их 
аргументами, которые не являются обычными; назовем их \textit{особыми}. 
В~\cite{2-gor} приведена зависимость дефекта от типа компоненты. 
Обозначим~$D$ сумму дефектов компонент графа $a\hm+b$. Обозначим~$P$ 
разность величин~$D$, вычисленных до и~после применения шага~3 
алгоритма. Заметим, что в~любой последовательности операций, 
финализирующих общий граф, число особых операций равно числу особых 
вершин в~нем, так что экономия числа операций может относиться лишь 
к~обычным операциям. Поскольку все операции на шаге~3 особые, 
величина~$P$ равна числу операций, сэкономленных на шаге~3. 
Величина~$\varepsilon$ определена выше. Пусть $C\hm=B\hm+S\hm+D\hm- 
P \hm+\varepsilon(B^\prime\hm+1)$. 
  
  \smallskip
  
  \noindent
  \textbf{Теорема~2.}\ \textit{Алгоритм строит последовательность 
операций, суммарная цена которой равна одному из трех значений $C\hm-
\varepsilon$, $C$, $C\hm+\varepsilon$. Минимально возможная суммарная цена 
последовательности операций, приводящей граф $a\hm+b$ к~финальному виду, 
также равна одному из этих значений. Время работы алгоритма линейное по 
порядку.}
  \smallskip
  
  В~доказательстве теоремы будут использованы нижеследующие леммы~2 
и~3.
  
  \smallskip
  
  \noindent
  \textbf{Лемма~2.}\ \textit{После выполнения шага~$4$ остается~$0$, $1$ 
или~$2$ связных компоненты, имеющих $b$-вер\-ши\-ну и~не являющихся 
исходными $b$-цик\-лами.}
  
  Простое доказательство леммы приведено в~\cite[п.~4]{1-gor} (там это 
лемма~3). 
  
  \smallskip
  
  \noindent
  \textbf{Лемма~3.} \textit{Число обычных операций в~алгоритме равно} 
$S\hm+D\hm-P$.
  
  \smallskip
  
  \noindent
  Д\,о\,к\,а\,з\,а\,т\,е\,л\,ь\,с\,т\,в\,о\,.\ \ Напомним~\cite{2-gor}, что 
минимальное число обычных операций, требуемых для приведения компоненты 
(после шага~2) к~финальному виду без использования других компонент, равно 
ее дефекту. Настоящий алгоритм отличается от описанного в~\cite{2-gor} 
наличием шага~4. Любая операция шага~4 либо особая и~не меняющая дефект 
результата по сравнению с~суммарным дефектом аргументов, либо обычная 
и~уменьшающая его на~1. Поэтому обычных операций в~алгоритме столько же, 
сколько и~раньше, т.\,е.\ $S\hm+D\hm- P$.~$\square$
  
  \smallskip
  
  \noindent
  Д\,о\,к\,а\,з\,а\,т\,е\,л\,ь\,с\,т\,в\,о\ \ теоремы~2. На шаге~5 для каждой 
компоненты, имеющей $b$-вер\-ши\-ны, применяется ровно одна операция 
удаления $b$-вер\-ши\-ны. По лемме~2 общее число таких операций равно 
$B^\prime\hm+n$, где~$n$ равно~0, 1 или~2. Всего особых операций~$B$. 
В~силу леммы~3 суммарная цена операций алгоритма равна 
$(1\hm+\varepsilon)(B^\prime\hm+n)\hm+(B\hm- B^\prime\hm- n)\hm+(S\hm+D\hm- 
P)\hm=B\hm+S\hm+ D\hm- P\hm+\varepsilon(B^\prime \hm+n)$, откуда следует 
первое утверждение теоремы. 
  
  Второе утверждение теоремы докажем индукцией по минимальной 
суммарной цене~$M$ операций, приводящих общий граф к~финальному виду; 
имеется лишь конечное число возможных значений~$M$ на любом 
ограниченном отрезке, которые рассматриваем по их возрастанию. Рассуждая 
так же, как и~в~доказательстве теоремы~1, видим, что достаточно для любой 
операции~$o$, примененной к~произвольному общему графу~$G$, проверить, 
что цена $c(o)$ операции~$o$ не меньше $C(G)\hm- C(o(G))$, где $C(G)$~--- 
величина~$C$, определенная в~формулировке теоремы~2. Подробности 
проверки приведены в~\cite[п.~4]{1-gor}.~$\square$
  
  \smallskip
  
  \noindent
  \textbf{Следствие.}\ Цена последовательности операций, которую строит 
описанный алгоритм, отличается от цены кратчайшей последовательности не 
более чем на~$\varepsilon$.
  
  \smallskip
  \noindent
  Д\,о\,к\,а\,з\,а\,т\,е\,л\,ь\,с\,т\,в\,о\ \ приведено в~\cite{7-gor}.~$\square$
  
  \section{Обобщение: задача с~повторением имен}
  
  \subsection{Постановка задачи}
  
  Важное в~прикладных вопросах обобщение рассмот\-ренной выше задачи 
состоит в~том, что в~структурах разрешается повторение имен. Это обобщение 
назовем \textit{задачей с~повторениями} (или по историческим причинам 
говорят: задачей с~паралогами). Пусть~$a$ и~$b$~--- такие структуры. 
Например, имеются в~$a$ три реб\-ра с~именем~$k$ и~в~$b$ два реб\-ра с~тем же 
именем~$k$. Нужно найти биекцию меньшего из этих двух множеств ре\-бер 
в~большее (для данного имени~$k$); и~аналогично для каждого имени~$k$, 
если ему соответствуют два таких множества, одно в~$a$ и~другое в~$b$, 
с~разным числом элементов. Итак, нужно найти семейство биекций, 
индексированное~$k$, при котором кратчайшая цена достигает минимального 
значения. Более детально на этом примере: нужно приписать этим пяти реб\-рам 
индекс~$i$ к~их имени~$k$ (получатся \textit{полные имена}, которые имеют 
вид~$k.i$, где~$i$ меняется от~1 до~3) так, чтобы с~новыми именами у всех 
повторяющихся ребер достигалось минимальное значение кратчайшей цены 
кратчайшего преобразования~$a$ в~$b$. Индекс~$i$ определяет частичное 
соответствие между бывшими одноименными реб\-ра\-ми в~$a$ и~$b$ 
и,~в~частности, определяет, какие реб\-ра общие и~какие особые для этих 
структур. Например, эти три реб\-ра можно индексировать~$k.1$, $k.2$, $k.3$, 
а~два других реб\-ра индексировать~$k.2$ и~$k.3$, тогда ребро~$k.1$ особое, 
а~остальные реб\-ра общие. Полные имена позволяют перейти от \textit{задачи 
с~повторениями к~задаче без повторений} (имен), последняя рассматривалась 
в~разд.~1 и~2.
  
  В силу NP-труд\-ности задачи с~повторениями, нельзя найти точный 
полиномиальный алгоритм ее решения. Однако ниже будет показано, как 
математически строго свести ее к~задаче целочисленного линейного 
программирования (ЦЛП). Как известно, для задач ЦЛП доступны 
компьютерные программы, выдающие, как правило, точное решение за время, 
близкое к~линейному, и~имеются соответствующие математические результаты. 
  
  Для краткости рассмотрим здесь только случай \textit{одинаковых цен всех 
операций}. В~\cite{3-gor} авторы описали сведение задачи с~повторениями 
к~задаче ЦЛП в~случае, если структуры состоят только из циклических 
компонент. При этом число переменных и~ограничений в~соответствующей 
задаче ЦЛП не более чем квадратично от размера исходных структур, что, 
конечно, принципиально важно. Далее будет \textit{описано такое сведение 
в~общем случае с~сохранением той же оценки на число переменных и~число 
ограничений}. 
  
  \subsection{Решение задачи}
  
  В исходных структурах~$a$ и~$b$ выберем произвольно второй индекс 
у~всех повторяющихся имен; структуры с~такими полными именами 
\textit{обозначим}~$a^\prime$ и~$b^\prime$; в~них (полные) имена уже не 
повторяются.
  
  Рассмотрим булевы переменные~$z_{abkij}$, для которых $z_{abkij}\hm=1$, если 
ребро~$k.i$ в~$a^\prime$ по искомой биекции соответствует ребру~$k.j$ 
в~$b^\prime$, иначе $z_{abkij}\hm=0$. Таким образом, значения этих переменных 
определяют соответствие ребер в~$a^\prime$ и~$b^\prime$. Переименуем реб\-ра 
в~$b^\prime$ по этому соответствию, результат обозначим~$a^\prime(z)$ 
и~$b^\prime(z)$. С~по\-мощью ограничений на эти переменные легко выразить 
понятия в~$a^\prime(z)$ и~$b^\prime(z)$: <<особое ребро>> и~<<цикл, 
состоящий из особых ребер>> (назовем его \textit{особым циклом}). Конечно, 
здесь описан только смысл переменной~$z$, который выражен в~рамках задачи 
ЦЛП.
  
  Каждой паре~$s$ различных краев ребер в~$a^\prime$ (или в~$b^\prime$) 
сопоставим булеву переменную~$t_{as}$ (соответственно~$t_{bs}$), ограничения 
на которые обеспечат следующие три свойства у~$a^\prime(z)$ 
(и~у~$b^\prime(z)$): если край из~$s$ принадлежит общему ребру или лежит 
в~особом цикле, то $t_{as}\hm=0$; для каждого края существует не более одного 
края, для которых $t_{as}\hm=1$ на этой паре в~качестве~$s$; для каждого края 
особого реб\-ра, не принадлежащего особому циклу, существует край, для 
которого $t_{as}\hm=1$ на этой паре в~качестве~$s$. И~аналогично для~$t_{bs}$. 
  
  По значениям переменных~$t_{as}$ и~$t_{bs}$ определим новые вершины 
и~реб\-ра в~$a^\prime(z)$ и~$b^\prime(z)$, результат обозначим соответственно 
$a^\prime(z, t)$ и~$b^\prime(z, t)$. Все особые реб\-ра из~$b^\prime(z)$, не 
содержащиеся в~особых циклах, добавим в~$a^\prime(z)$; края новых ребер 
склеим, если $t_{bs}\hm=1$ на этой паре в~качестве~$s$. Аналогично особые реб\-ра 
из~$a^\prime(z)$, не содержащиеся в~особых циклах, добавим в~$b^\prime(z)$; 
их края аналогично склеим, если $t_{as}\hm=1$. Таким образом, в~структурах 
$a^\prime(z, t)$ и~$b^\prime(z, t)$ все реб\-ра общие, кроме принадлежащих 
особым циклам. Как раз эти особые циклы удалим из $a^\prime(z, t)$ 
и~$b^\prime(z, t)$, результат обозначим теми же буквами; получены структуры 
с~постоянным составом. Из условий на~$t_{as}$ и~$t_{bs}$ следует: каждое новое 
ребро входит в~цикл из новых ребер.
  
  \textit{Обозначим} 
$G^\prime\hm=G^\prime(z, t)\hm=a^\prime(z, t)\hm+b^\prime(z, t)$. Для 
графа~$G^\prime$ вычислим значение $C_1\hm+0{,}5C_2$, где~$C_1$ 
и~$C_2$~--- число циклов и~четных цепей в~этом графе соответственно.
  
  Число~$C_1$ вычисляется так же, как в~\cite{3-gor} для общего графа 
вычисляется число~$S_2$ циклов, состоящих из обычных ребер (с~учетом 
цепей и~новых ре\-бер). 
{\looseness=1

}
  
  Для вычисления величины $0{,}5C_2$ каждому краю~$p$~реб\-ра в~$a$ 
и~$b$ сопоставим переменную~$r_p$, принимающую значения~0, 1 или~$-1$. 
Ограничения обеспечат условия: если для пары склеенных краев 
в~$a^\prime(z, t)$ или $b^\prime(z, t)$ одна из переменных рав\-на~1, то вторая 
равна~$-1$, а~если первая переменная рав\-на~0, то вторая рав\-на~0 или~$-1$. 
  
  Переменные $r_p$ будут входить в~минимизиру\-емую функцию~$F$, которую 
определим чуть ниже, с~отрицательным коэффициентом, поэтому они равны~1 
на изолированных вершинах в~$G^\prime$. На вершинах из циклов 
в~$G^\prime$ значения~1 и~$-1$ переменных~$r_p$ чередуются или все эти 
значения нулевые; в~любом случае сумма всех~$r_p$ вдоль цикла равна~0. Эти 
значения чередуются и~на ненулевой четной цепи, причем на ее краях они 
равны~1; так что их сумма вдоль цепи равна~1. На нечетных цепях это 
чередование перемежается с~нулевыми значениями; в~любом случае их сумма 
вдоль такой цепи равна~0. Отсюда вытекает, что полусумма всех 
значений~$r_p$ равна~$C_2$.
  
  Итак, минимизируемая целевая функция \textit{определяется как} 
$F\hm=C_0\hm+n\hm+s_a\hm+s_b\hm- C_1\hm- 0{,}5C_2$, где~$C_0$~--- сумма чисел 
особых циклов в~$a^\prime(z)$ и~$b^\prime(z)$; $n$~--- число (однократно 
учитываемых) общих ребер в~них; $s_a$ и~$s_b$~--- число особых ребер 
в~$a^\prime(z)$ и~$b^\prime(z)$, не входящих в~особые циклы. Значение 
$C_0\hm+n\hm+s_a\hm+s_b$ линейно выражается через введенные 
переменные~$z$. Значение~$F$ линейно выражается через переменные~$z$, 
$t$ и~$r_p$.
  
  Итак, исходная задача сведена к~задаче ЦЛП. Указанная оценка числа 
переменных и~ограничений очевидна. Корректность такого сведения формально 
доказана в~работе авторов, которая пред\-став\-ле\-на в~печать. В~этом 
доказательстве решающий шаг состоит в~том, что минимальное число операций в~последовательности, преобразующей~$a^\prime(z)$ в~$b^\prime(z)$, 
равно~$F$. Действительно, минимальная последовательность, преобразующая 
$a^\prime(z, t)$ в~$b^\prime(z, t)$, имеет длину~$F$, что следует из результата 
в~\cite{2-gor}. Она индуцирует последовательность той же длины, 
преобразующую~$a^\prime(z)$ в~$b^\prime(z)$. И~обратно: кратчайшая 
последовательность, преобразующая~$a^\prime(z)$ в~$b^\prime(z)$, индуцирует 
последовательность той же длины, преобразующую~$a^\prime(z, t)$ 
в~$b^\prime(z, t)$. 
  
  Идея такого соответствия последовательностей состоит в~том, что операции 
удаления участка ребер ставится в~соответствие стандартная операция, 
вырезающая и~зацикливающая этот участок, а~операции вставки участка 
ребер~--- стандартная операция, вставляющая этот зацикленный участок в~то же 
место. Такая идея была предложена в~\cite{8-gor}.
  
{\small\frenchspacing
 {%\baselineskip=10.8pt
 \addcontentsline{toc}{section}{References}
 \begin{thebibliography}{9}
\bibitem{1-gor}
\Au{Горбунов К.\,Ю., Любецкий~В.\,А.} Линейный алгоритм кратчайшей перестройки графов 
при разных ценах операций~// Информационные процессы, 2016. Т.~16. №\,2. C.~223--236.
\bibitem{2-gor}
\Au{Горбунов К.\,Ю., Любецкий~В.\,А.} Линейный алгоритм минимальной перестройки 
структур~// Проблемы передачи информации, 2017  (в печати). Т.~53. Вып.~1.
\bibitem{3-gor}
\Au{Lyubetsky V.\,A., Gershgorin~R.\,A., Seliverstov~A.\,V., Gorbunov~K.\,Yu.} Algorithms for 
reconstruction of chromosomal structures~// BMC Bioinformatics, 2016. Vol.~17. P.~40.1--40.23.
\bibitem{4-gor}
\Au{Da Silva P.\,H., Machado~R., Dantas~S., and Braga~M.\,D.\,V.} DCJ-indel and  
DCJ-substitution distances with distinct operation costs~// Algorithm. Mol. Biol., 2013. 
Vol.~8. P.~21.1--21.15.
\bibitem{5-gor}
\Au{Compeau P.\,E.\,C.} A~generalized cost model for DCJ-indel sorting~// 
Algorithms in bioinformatics~/ Eds. D.\,G.~Brown, B.~Morgenstern.~---
Lecture notes in 
computer science ser.~--- Springer, 2014. Vol.~8701. P.~38--51.
\bibitem{6-gor}
\Au{Горбунов К.\,Ю., Гершгорин~Р.\,А., Любецкий~В.\,А.} Перестройка и~реконструкция 
хромосомных структур~// Молекулярная биология, 2015. Т.~49. №\,3. С.~372--383.
\bibitem{7-gor}
\Au{Горбунов К.\,Ю., Любецкий~В.\,А.} Модифицированный алгоритм преобразования 
хромосомных структур: условия абсолютной точ\-ности~// Современные информационные 
технологии и~ИТ-обра\-зо\-ва\-ние, 2016. Т.~12. №\,1. С.~162--172.
\bibitem{8-gor}
\Au{Compeau P.\,E.\,C.} DCJ-indel sorting revisited~// Algorithm. Mol. Biol., 2013. 
Vol.~8. P.~6.1--6.9.
 \end{thebibliography}

 }
 }

\end{multicols}

\vspace*{-6pt}

\hfill{\small\textit{Поступила в~редакцию 26.04.16}}

\vspace*{6pt}

%\newpage

%\vspace*{-24pt}

\hrule

\vspace*{2pt}

\hrule

\vspace*{-2pt}


\def\tit{ALGORITHM OF~TRANSFORMATION OF~A~GRAPH\\ INTO~ANOTHER ONE WITH~MINIMAL COST}

\def\titkol{Algorithm of transformation of a graph into another one with minimal cost}

\def\aut{K.\,Yu.~Gorbunov$^1$ and~V.\,A.~Lyubetsky$^{1,2}$}

\def\autkol{K.\,Yu.~Gorbunov and~V.\,A.~Lyubetsky}

\titel{\tit}{\aut}{\autkol}{\titkol}

\vspace*{-9pt}


\noindent
$^1$A.\,A.~Kharkevich Institute for Information Transmission Problems of the Russian Academy of 
Sciences,\linebreak
$\hphantom{^1}$19-1 Bolshoy Karetny Per., Moscow 127051, Russian Federation 

\noindent
$^2$Faculty of Mechanics and Mathematics, M.\,V.~Lomonosov Moscow State 
University, Main Building, Leninskiye\linebreak 
$\hphantom{^1}$Gory, GSP-1, Moscow 119991, Russian 
Federation



\def\leftfootline{\small{\textbf{\thepage}
\hfill INFORMATIKA I EE PRIMENENIYA~--- INFORMATICS AND
APPLICATIONS\ \ \ 2017\ \ \ volume~11\ \ \ issue\ 1}
}%
 \def\rightfootline{\small{INFORMATIKA I EE PRIMENENIYA~---
INFORMATICS AND APPLICATIONS\ \ \ 2017\ \ \ volume~11\ \ \ issue\ 1
\hfill \textbf{\thepage}}}

\vspace*{3pt}



\Abste{The authors study orgraphs with any number of chains and cycles. Edges of 
orgraphs have unique names~--- natural numbers. There is a fixed list of operations 
that transform one graph into another. A~cost is assigned to each operation. The task is 
to find the path of transformations with minimal total cost. This problem has a~wide 
range of practical applications. The task is probably NP-hard and, thus, can be solved 
only under constraints imposed on costs or graphs. Such solutions are proposed in the 
study. The corresponding algorithms are linear in time and memory and are proved to 
be exact (nonheuristic), i.\,e., to find the path of transformations with minimal cost. 
Many heuristic algorithms solving this problem are known and tested on various data, 
but the proposed solutions are the first exact solutions.}

\KWE{orgraph with chains and cycles; graph transformation; graph transformation 
with minimal total cost; exact linear algorithm; graph constraint; cost constraint; 
conditional shortest solution}

\DOI{10.14357/19922264170107}  

\vspace*{-24pt}

\Ack
\noindent
The study was supported by the Russian Science Foundation (project  
No.\,14-50-00150).



%\vspace*{3pt}

  \begin{multicols}{2}

\renewcommand{\bibname}{\protect\rmfamily References}
%\renewcommand{\bibname}{\large\protect\rm References}

{\small\frenchspacing
 {%\baselineskip=10.8pt
 \addcontentsline{toc}{section}{References}
 \begin{thebibliography}{9}
\bibitem{1-gor-1}
\Aue{Gorbunov, K.\,Yu., and V.\,A.~Lyubetsky}. 2016. Lineynyy algoritm 
kratchayshey perestroyki grafov pri raznykh tsenakh operatsiy [A linear algorithm of 
the shortest transformation of graphs under different operation costs]. 
\textit{Informatsionnye protsessy} [Information Processes] 16(2):223--236.
\bibitem{2-gor-1}
\Aue{Gorbunov, K.\,Yu., and V.\,A.~Lyubetsky}. 2017 (in press). Lineynyy algoritm 
minimal'noy perestroyki struktur [Linear algorithm of the minimal reconstruction of 
structures under different operation costs]. \textit{Problemy peredachi informatsii} 
[Problems of Information Transmission] 53(1). 
\bibitem{3-gor-1}
\Aue{Lyubetsky, V.\,A., R.\,A.~Gershgorin, A.\,V.~Seliverstov, and 
K.\,Yu.~Gorbunov}. 2016. Algorithms for reconstruction of chromosomal structures. 
\textit{BMC Bioinformatics} 17:40.1--40.23.
\bibitem{4-gor-1}
\Aue{Da Silva, P.\,H., R.~Machado, S.~Dantas, and M.\,D.\,V.~Braga}. 2013.  
DCJ-indel and DCJ-substitution distances with distinct operation costs. 
\textit{Algorithm. Mol. Biol}. 8:21.1--\linebreak 21.15.
\bibitem{5-gor-1}
\Aue{Compeau, P.\,E.\,C.} 2014. A~generalized cost model for DCJ-indel sorting. 
\textit{Algorithms in bioinformatics}. Eds.\ D.\,G.~Brown and B.~Morgenstern.
{Lecture notes in computer science ser.} Springer. 8701:38--51.
\bibitem{6-gor-1}
\Aue{Gorbunov, K.\,Yu., R.\,A.~Gershgorin, and V.\,A.~Lyubetsky}. 2015. 
Rearrangement and 
inference of chromosome structures. \textit{Mol. 
Biol.} 49(3):327--338. 
\bibitem{7-gor-1}
\Aue{Gorbunov, K.\,Yu., and V.\,A.~Lyubetsky}. 2016.  
Modifitsirovannyy algoritm preobrazovaniya khromosomnykh struktur: Usloviya 
absolyutnoy tochnosti [Modified algorithm for transformation of chromosome structures: 
Conditions of absolute accuracy]. \textit{Sovremennye informatsionnye tekhnologii 
i~IT-obrazovanie} [Modern Information Technologies and IT Education] 12(1):162--172.
\bibitem{8-gor-1}
\Aue{Compeau, P.\,E.\,C.} 2013. DCJ-indel sorting revisited. \textit{Algorithm.  
Mol. Biol.} 8:6.1--6.9.
\end{thebibliography}

 }
 }

\end{multicols}

\vspace*{-9pt}

\hfill{\small\textit{Received April 26, 2016}}

\pagebreak

\Contr

\noindent
\textbf{Gorbunov Konstantin Yu.} (b.\ 1965)~--- Candidate of Science (PhD) in 
physics and mathematics, leading scientist, A.\,A.~Kharkevich Institute for Information Transmission 
Problems of the Russian Academy of Sciences, 
\mbox{19-1}~Bolshoy Karetny Per., Moscow 127051, Russian Federation; 
\mbox{gorbunov@iitp.ru}

\vspace*{3pt}

\noindent
\textbf{Lyubetsky Vassily~A.} (b.\ 1945)~--- Doctor of Science in physics and 
mathematics, professor, Head of Laboratory, A.\,A.~Kharkevich Institute for Information 
Transmission Problems of the Russian Academy of Sciences, 
\mbox{19-1}~Bolshoy Karetny Per., Moscow 127051, Russian Federation; professor, 
Faculty of Mechanics and Mathematics, M.\,V.~Lomonosov Moscow State 
University, Main Building, Leninskiye Gory, GSP-1, Moscow 119991, Russian 
Federation; \mbox{lyubetsk@iitp.ru}
\label{end\stat}


\renewcommand{\bibname}{\protect\rm Литература}  %7  
\def\stat{alex}

\def\tit{ПРИМЕНЕНИЕ КОНТЕКСТНО-СВОБОДНЫХ ГРАММАТИК ДЛЯ~ИЗВЛЕЧЕНИЯ ОНТОЛОГИИ ИЗ ТЕКСТОВ 
КОРОТКИХ ОПИСАНИЙ СТАТЕЙ БИОЛОГИЧЕСКОЙ ТЕМАТИКИ$^*$}

\def\titkol{Применение КС-грамматик для извлечения онтологии из текстов 
коротких описаний статей биологической тематики}

\def\aut{Д.\,А.~Алексеевский$^1$}

\def\autkol{Д.\,А.~Алексеевский}

\titel{\tit}{\aut}{\autkol}{\titkol}

{\renewcommand{\thefootnote}{\fnsymbol{footnote}} \footnotetext[1]
{Работа выполнена при частичной поддержке РФФИ (проект 15-07-09306).}}


\renewcommand{\thefootnote}{\arabic{footnote}}
\footnotetext[1]{НИУ Высшая школа экономики, dalexeyevsky@hse.ru}

\vspace*{-6pt}

  
  \Abst{Обработка текстов биологической и~медицинской тематики представляет интерес 
как с~точки зрения биологии, для которой она предоставляет ценные результаты, так 
и~в~качестве источника более сложных задач для обработки текстов. Одной из важных 
задач автоматической обработки текстов является построение онтологий. 
Предложен метод построения онтологий промежуточного уровня по корпусу текстов на 
ограниченном подмножестве английского языка. Онтологии промежуточного уровня служат 
одним из инструментов решения задачи установления соответствия между фактами 
в~априорных онтологиях и~фрагментами текста. Предложен новый подход, основанный на 
расширенном определении кон\-текст\-но-сво\-бод\-ных (КС) грам\-ма\-тик, позволяющий порождать онтологии, 
обладающие указанным свойством. Показаны преимущества использования 
корпусов на ограниченном подмножестве естественного языка для построения таких 
онтологий.}
  
  \KW{КС-грамматики; построение онтологий; биомедицинские тексты}
  
  \DOI{10.14357/19922264160111} %

%\vspace*{-4pt}

\vskip 10pt plus 9pt minus 6pt

\thispagestyle{headings}

\begin{multicols}{2}

\label{st\stat}
  
  \section{Введение}
  
%  \vspace*{-2pt}
  
  За последние десятилетия биология, а следом за ней и~медицина претерпели 
несколько научных переворотов, каждый из которых приводил к~бурному росту 
числа публикаций, а также и~прочих текстов этих тематик. Многие полученные 
данные были собраны в~базы данных, которые играют большую роль в~этих 
науках. В~то же время с~ростом объема опубликованных текстов 
обнаруживаются новые виды данных, доступные в~текстовом виде, требующие 
структурирования и~верификации. Этим объясняется растущая актуальность 
темы извлечения фактов из текстов биологической и~медицинской тематики. 
Следует заметить, что эта тема имеет существенные отличия от автоматической 
обработки текстов в~целом, что обусловливает выделение ее в~отдельную 
область.
  
  Задачам автоматической обработки текстов медицинской и~биологической 
тематики посвящено много работ. Среди современных направлений 
исследова\-ний: извлечение и~нормализация именованных сущностей~[1], 
извлечение событий и~состав\-ных отношений~[2], анализ дискурса 
и~ко\-референции~[3], построение и~пополнение онтологий и~баз данных~[4]. 
Среди наиболее широко\linebreak используемых биологических баз данных встречаются 
ресурсы, совмещающие структурированные данные (ссылки на другие базы 
данных, чис\-ло\-вые характеристики объектов, номенклатурные на\-звания 
объектов и~т.\,п.), неструктурированные тексто\-вые данные (текстовые 
описания, цитаты из статей и~эн\-цик\-ло\-пе\-дий) и~час\-тич\-но формализованные 
текстовые данные (описания на ограниченном подмножестве языка 
с~использованием контролируемых словарей)~[5, 6].
  
  Наряду с~задачей извлечения фактов, соответ\-ст\-вующих заранее заданной 
онтологии, для неко\-торых областей актуальна задача определения 
онтологической структуры и~извлечения самих онтологиче\-ских элементов. 
%
В~настоящей статье {пред\-ло\-жен} метод преобразования частично 
структурированных текстовых описаний в~онтологии, основанный на 
использовании гетерогенных час\-тот\-ных списков и~семантически 
ориентированных КС (СОКС) грамматик.
  
  Для иллюстрации работы метода выбраны краткие аннотации статей, 
используемые в~одной из баз данных (см.\ подразд.~2.4). Приведена 
последовательность действий по преобразованию аннотаций в~онтологическое 
представление, дана оценка применимости метода в~выбранном примере. 
В~насто\-ящее время указанные краткие аннотации заполняются кураторами 
вручную, но затем автоматически посредством простых шаблонов по ним 
определяется уровень доверия к~записи в~базе данных. Приведение таких 
аннотаций к~онтологическому представлению является необходимым первым 
ша-\linebreak\vspace*{-12pt}

\pagebreak

\noindent
гом для последующего автоматического постро\-ения аннотаций по тексту 
статьи.

\vspace*{-6pt}
  
  \section{Контекст работы}
  
  \vspace*{-3pt}
  
  \subsection{Специфика обработки биологических текстов}
  
  \vspace*{-1pt}
  
  Медицинская и~биологическая тематика текстов привносит особенности во 
многие этапы их обработки. В~значительной мере именно это и~обуслов\-ливает 
выделение bionlp как отдельной предметной области.
  
  Один из часто используемых шагов обработки текстов~--- идентификация 
фрагментов текста, соответствующих известным сущностям в~базах данных, 
так называемое извлечение именованных сущностей.
{\looseness=-1

}
  
  В биологических текстах этот шаг обработки усложнен несколькими 
обстоятельствами:
  \begin{itemize}
\item именованная сущность может являться лишь частью слова, например 
в~предложении <<The acid-promoted expression of the PmrD protein was  
phoPQ-dependent, which is in agreement with the fact that PhoP is the only known 
direct transcriptional activator of pmrD (Kox \textit{et al.}, 2000)>> в~слове  
phoPQ-dependent выделяют двухбелковый комплекс <<phoPQ>>, состоящий из 
белка <<phoP>> и~белка~<<Q>>;\\[-15pt]
\item некоторые сущности, такие как белки или химические соединения, имеют множество синонимичных 
названий, при этом в~текстах могут использоваться не полные названия, а~их сокращения, смысл которых 
возможно восстановить лишь из контекста. Например, белок\footnote{UniProt AC P30233, {\sf 
http://www.uniprot.org/uniprot/P30233}.}, имеющий в~базе данных названия <<Sweet protein 
mabinlin-2>>, <<Mabinlin~II>>, <<MAB~II>>, <<Sweet protein mabinlin-2 chain~A>>, 
<<Sweet protein 
\mbox{mabinlin-2} chain~B>>, может встречаться в~стать\-ях как <<heat-stable sweet protein, mabinlin-II>>, 
<<mabinlin>> (в~пределах текста одной статьи это название может в~разных контекстах обозначать как 
название класса белков, так и~конкретный белок), <<Cm-MaIIA>> (обозначение одной цепочки 
модифицированного белка, введенное в~статье)~[7];\\[-15pt]
\item для некоторых сущностей после определения их названия необходимо 
точнее идентифицировать сущность, о которой идет речь: например, одно и~то 
же название белка может иметь несколько аллелей в~одном организме, белок 
может различаться или не различаться в~зависимости от ткани, для которой 
проводился эксперимент, одно имя могут иметь схожие, но различные белки из 
разных организмов. Для каждого белка, имеющего то же имя, имеется 
отдельная запись в~базе данных, и~необходимо определить, о какой именно 
записи идет речь.
\end{itemize}

\vspace*{-10pt}

  \subsection{Задача построения онтологий}
  
  В литературе встречаются разнообразные определения понятия онтологии 
в~зависимости от темы и~специфики выбранной задачи. Встречающиеся 
определения этого понятия в~контексте извлечения фактов описывают способы 
представления знаний, как правило, состоящие из описаний сущностей, их 
свойств, классификации, связей между ними и~логических правил пополнения 
их свойств и~связей~[8].
  
  Выделяют онтологии, построенные априори путем логической 
классификации и~лексические, в~которых отражаются семантические связи 
между языковыми единицами~[9]. Они обладают разными свойствами: 
априорная точнее отражает предметную область и~позволяет применять 
богатые механизмы логического вывода, в~то время как сущности лексических 
онтологий, как правило, проще выделять в~тексте. В~связи с~этим одна из часто 
возникающих задач состоит в~установлении соответствия между сущностями 
лексической и~априорной онтологии~[10].
  
  Подход, предлагаемый в~настоящей статье, позволяет построить онтологию, 
занимающую промежуточное положение. Такая онтология строится частично 
по базам данных как априорная, частично по корпусу текстов как лексическая. 
Это определяет ее главное достоинство: она содержит как ссылки на 
конкретные сущности из базы данных, так и~их текстовое представление.

\vspace*{-10pt}
  
  \subsection{Семантически-ориентированные контекстно-свободные грамматики}
  
  Для настоящей работы в~качестве формализма для описания синтаксической 
структуры предложения были выбраны КС-грам\-ма\-ти\-ки. Контекстно-свободная
грам\-ма\-ти\-ка~--- это способ описания структуры 
предложения в~виде иерархии составляющих частей~[11]. Дадим ей 
формальное определение.
  
  \vspace*{1pt}
  
  \noindent
  \textbf{Определение~1.} Кон\-тек\-ст\-но-сво\-бод\-ной грамматикой 
называется четверка $G\hm=(V,\Sigma, R, S)$, где $V$~--- конечное множество 
нетерминальных символов; $\Sigma$~--- конечное множество терминальных 
символов; $R\subset \{V\times (V\cup \Sigma)^*\}$~--- множество правил вывода 
вида $v\hm\to a_1a_2\cdots$, где $v\hm\in V$, $a_i\hm\in V\cup \Sigma$; $S\hm\in 
V$~--- начальный символ.
  
\pagebreak
  
  Формальным определением для описания онтологии в~настоящей работе 
было выбрано следующее: онтология~--- это ориентированный граф, в~котором 
каждая вершина и~каждое ребро со\-про\-вож\-да\-ют\-ся пометой. С~помощью пометы 
множество вершин делится на вер\-ши\-ны-клас\-сы  
и~вер\-ши\-ны-эк\-земп\-ля\-ры. Пометы на ребрах устанавливают тип 
отношений, в~которых находятся две выбранные вершины. Приведем более 
формализованное определение.
  
  \smallskip
  
  \noindent
  \textbf{Определение 2.} Онтологией называется пара $O\hm= (G_O,L_O)$ из 
ориентированного графа и~меток к~нему. В~свою очередь, граф 
$G_O\hm=(E_O,R_O)$ состоит из множества вершин~$E_O$, называемого 
множеством сущностей, и~множества ребер $R_O\subset E_O\times E_O$, 
называемого множеством отношений; метки $L_O\hm= (T_E, T_R, L_E, L_R)$ 
задаются алфавитом возможных меток для вершин~$T_E$, алфавитом 
возможных меток для ребер~$T_R$, отображением $L_E: E_O\hm\to T_E$ 
вершины на ее метку и~отображением $L_R: R_O\hm\to T_R$ ребра на его 
метку.

  \smallskip
  
  Одно из свойств предлагаемого в~настоящей работе алгоритма состоит 
в~простоте выделения онтологических фактов из деревьев синтаксического 
разбора. Такой алгоритм требует введения нового понятия: семантически 
ориентированной КС-грам\-ма\-ти\-ки. Контекстно-свободная грам\-ма\-ти\-ка является 
семантически ориентированной для данной онтологии, если часть ее правил 
описывает сущности и~отношения в~онтологии. Предлагается следующее 
определение.
  
  \smallskip
  
  \noindent
  \textbf{Определение~3.} Семантически ориентированной  
КС-грамматикой называется тройка $S\hm=(G,O,M)$ 
из КС-грам\-ма\-ти\-ки $G\hm=(V,\Sigma,\ldots)$, 
онтологии $O\hm= ((E-O, R_O), (T_E, T_R, L_E, L_R))$ и~отображения~$M$ 
между ними. Отображение $M\hm= (M_E, M_R)$ состоит из отображения 
$M_E\subset (\Sigma \cup V, E_O)$, где $\forall (v,e), (v^\prime, e^\prime)\hm\in 
M_E: v\hm= v^\prime \Leftrightarrow e\hm=e^\prime$; символов грамматики на 
вершины онтологии и~отображения $M_R\subset (V,L_R)$, где $\forall (v,r), 
(v^\prime, r^\prime)\hm\in M_R:\ v\hm=v^\prime\Leftrightarrow r\hm=r^\prime$.

  \smallskip
  
  Терминальный символ грамматики может быть отображен на  
вер\-ши\-ну-класс или вершину-эк\-земп\-ляр либо не использоваться в~онтологии. 
В~последнем случае терминальный символ будем называть синтаксическим по 
последнему этапу обработки текста, в~котором он используется. 
Нетерминальный символ может быть отображен на вер\-ши\-ну-класс, метку 
ребра (тип отношения), в~том числе одновременно, либо не использоваться 
в~онтологии. Как и~в случае с~терминальными вершинами, в~последнем случае 
такой нетерминал будет называться синтаксическим.

%\pagebreak

%\vspace*{-6pt}
  
  \subsection{База данных UniProt}
  
  Материалом для разработки и~тестирования предлагаемой процедуры 
построения онтологий послужила свободно распространяемая база UniProt~[6].
  
  UniProt является хранилищем аминокислотных последовательностей белков 
наряду с~их краткими описаниями. База содержит ссылки на другие базы 
данных, посвященные исследованиям белков специфическими методами. 
Кроме того, частью описания белка в~базе является список литературы, 
описывающей белок.
  
  Для каждого белка база содержит:
  \begin{itemize}
\item описание его аминокислотной последовательности (поле <<\ \ >>~--- два 
пробела);
\item обозначения белка согласно различным номенклатурам (поля <<DE>> 
и~<<GN>>);
\item идентификаторы в~различных биологических базах данных самого белка 
(поля <<ID>>, <<AC>> и~<<DR>>) и~его носителя (<<OC>> и~<<OX>>);
\item биологический контекст белка (поля <<OS>>, <<OG>> и~<<OH>>);
\item библиографическую информацию (поля <<RN>>, <<RP>>, <<RC>>, 
<<RX>>, <<RG>>, <<RA>>, <<RT>> и~<<RL>>);
\item описания известных свойств белка: текстовые (поле <<CC>>), на 
ограниченном подмножестве английского языка (поля <<RP>> и~<<KW>>), 
формализованные (поле <<FT>>);
\item уровень доверия данной записи (поле <<PE>>);
\item прочую служебную информацию (поля <<DT>> и~<<SQ>>).
\end{itemize}

  Значение <<PE>> уровня достоверности записи базы данных определяется 
тем, какими экспериментальными средствами установлен факт существования 
белка и~его соответствия представленным данным. Описания того, какие 
экспериментальные средства применялись к~белку, хранятся в~базе в~полях 
<<CC>>, <<RP>> и~<<KW>>, для некоторых методов факт их применения 
можно опознать по свойствам в~поле <<FT>>. В~базе заданы формальные 
правила выставления значения уровня доверия (<<PE>>) в~зависимости от 
наличия некоторых шаблонных выражений в~этих полях~\cite{12-al}.



  \begin{figure*}[b] %fig1
  \begin{center}
  
  {\small
  \begin{boxedverbatim}
X-RAY CRYSTALLOGRAPHY (1.80 ANGSTROMS) OF 44-480 OF WILD-TYPE AND MUTANTS TYR-118;
ARG-168 AND ALA-309 IN ACTIVE AND RESTING STATES AND IN COMPLEX WITH PEPTIDE SUBSTRATE, 
FUNCTION, CATALYTIC ACTIVITY, ENZYME REGULATION, SUBSTRATE SPECIFICITY, SUBUNIT, DOMAIN, 
PROTEOLYTIC AUTO-CLEAVAGE, ACTIVE SITES, SITES, DISRUPTION PHENOTYPE, MUTAGENESIS
OF VAL-118; ARG-168; SER-309 AND GLN-338, AND PDZ DOMAIN DELETION MUTANT.
\end{boxedverbatim}

}
\begin{center}
{\small (\textit{а})}
\end{center}

{\small
\begin{boxedverbatim}
> X-RAY CRYSTALLOGRAPHY (1.80 ANGSTROMS) OF 44-480 OF WILD-TYPE AND MUTANTS TYR-118;
ARG-168 AND ALA-309 IN ACTIVE AND RESTING STATES AND IN COMPLEX WITH PEPTIDE SUBSTRATE
> FUNCTION
> CATALYTIC ACTIVITY
> ENZYME REGULATION
> SUBSTRATE SPECIFICITY
> SUBUNIT
> DOMAIN
> PROTEOLYTIC AUTO-CLEAVAGE
> ACTIVE SITES
> SITES
> DISRUPTION PHENOTYPE
  > MUTAGENESIS OF VAL-118; ARG-168; SER-309
\end{boxedverbatim}

}

\begin{center}
{\small (\textit{б})}
\end{center}
  \end{center}
  
  \vspace*{-14pt}
  
  \begin{center}
  \Caption{Примеры описаний в~поле <<RP>>: (\textit{а})~полное описание, (\textit{б}) 
атомарные факты}
  \end{center}
  \vspace*{-12pt}
  \end{figure*}
  
  
  
  База данных UniProt состоит из двух час\-тей: UniProt/TrEMBL, пополняемой 
полностью автоматически, и~UniProt/Swiss-Prot, по\-пол\-ня\-емой кураторами 
вручную на основе материалов UniProt/TrEMBL, существующих публикаций 
и~материалов других баз данных. Поля <<KW>> и~<<FT>> получают 
начальные значения автоматически в~базе данных UniProt/TrEMBL, хотя затем 
могут быть изменены в~процессе курирования. Поля <<CC>> и~<<RP>> 
заполняются только кураторами вручную.
  
  Этими обстоятельствами обусловлено то, что в~качестве материала для 
настоящей работы был собран корпус предложений в~поле <<RP>> из базы 
данных UniProt/Swiss-Prot.

\vspace*{-9pt}
  
  \section{Материалы и~методы}
  
  Материалом исследований послужили данные из базы UniProt/Swiss-Prot 
версии 2015\_01. Собранный корпус уникальных атомарных причин 
цитирования в~поле <<RP>> имеет размер 173\,212~предложений.
  
  Всего база UniProt/Swiss-Prot 2015\_1 содержит:
  \begin{itemize}
\item 547\,357 записей (одна запись описывает один белок);
\item 1\,092\,817 ссылок на литературу и, соответственно, всего предложений 
в~поле <<RP>>, включая повторяющиеся, среди них;
\item 179\,616 уникальных предложений в~поле <<RP>>, состоящих из одного или 
нескольких атомарных описаний (в~свою очередь также вклю\-ча\-ющих 
повторения);
\item 173\,212 уникальных атомарных описаний.
\end{itemize}

  Для дальнейшей работы использовался описанный корпус уникальных 
атомарных описаний, с~тем чтобы наиболее полно покрыть максимально 
возможное количество особых случаев в~языке.

%\vspace*{-9pt}
  
  \subsection{Особенности предложений в~поле <<RP>>}
  
    \vspace*{-2pt}
  
  Предложения в~поле <<RP>> являются полуструктурированными, так как несут 
признаки как структурированных, так и~естественных языковых данных. 
Предложения порождаются кураторами. Для них не существует 
формализованного описания структуры или инструмента для валидации. 
Существует находящаяся на данный момент в~стадии разработки
инициатива по унификации представления названий различных классов 
сущностей в~таких предложениях с~помощью внедрения контро\-ли\-ру\-емых 
словарей~\cite{13-al}. Наряду с~этим для кураторов существует инструкция по заполнению, 
вклю\-ча\-ющая в~себя примеры представления большого числа типов 
фактов~\cite{14-al}.
  
  Каждое атомарное описание является именной группой. Важно заметить, что 
для краткости описания не содержат упоминаний описываемого объекта. 
Объект описания устанавливается из факта принадлежности описания записи 
в~базе данных (рис.~1). 
  
\vspace*{-14pt}
  
  \subsection{Словники}
  
  \vspace*{-6pt}
  
  Для извлечения именованных сущностей и~насыщения списка примитивных 
фактов были использованы словники.
  
  Словник имен белков был построен по значениям в~поле <<DE>>, подразделам 
RecName и~AltName
 базы данных UniProt и~полям Full, Short, Name, Synonyms 
в~них. Суммарный объем словника составил 308\,370 словосочетаний.
  
  Некоторые названия белков совпадают с~общезначимыми словами 
английского языка. Для того чтобы исключить ошибки второго рода в~таких 
случаях, из словника имен белков были удалены все слова, являющиеся 
словами английского языка. Для этой фильтрации был использован словник 
общеупотребительной лексики американского анг\-лий\-ско\-го языка~\cite{15-al} 
объемом 99\,171~словоформа, содержащий все падежные формы слов.

\vspace*{-6pt}
  
  \subsection{Методы}
  
  Для сегментации текста на слова был использован токенизатор, 
сохраняющий все знаки пунктуации, включая дефисы, как отдельные токены. 
Токенизатор был разработан на основе пакета re языка Python~\cite{16-al}.
  
  Для построения СОКС-грамматик был использован парсер Эрли с~проходом 
снизу вверх из пакета nltk~\cite{17-al} для языка Python.
  
  Для построения частотных списков использовались средства shell script 
и~сопутствующие программы текстовой обработки из базового комплекта операционной системы 
GNU: cat, sort, uniq, grep, sed, head, tail, less.

\vspace*{-6pt}
  
  \section{Алгоритм разработки онтологии с~помощью контекстно-свободных грамматик}
  
  Задача алгоритма состоит в~том, чтобы за наименьшее время преобразовать 
наибольшую часть заранее заданного корпуса фактов, представленного в~виде 
полуструктурированных текстовых данных, в~онтологическое представление.
  
  Основная идея алгоритма состоит в~итерационном применении и~пополнении 
КС-грам\-ма\-ти\-ки.\linebreak\vspace*{-12pt}


\noindent После каждого применения грамматики предложения 
корпуса преобразуются в~гетерогенную последовательность из токенов 
и~нетерминальных \mbox{символов} грамматики. Полученный корпус гетерогенных 
последовательностей используется для того, чтобы определить, какое правило 
нужно добавить в~корпус для получения наибольшего прироста количества 
предложений, разбор которых доведен до не\-тер\-ми\-на\-ла-вер\-шины.
  
  При построении КС-грам\-ма\-ти\-ки терминальными символами грамматики 
являются токены из корпуса, множество нетерминальных символов является 
объединением из множества типов сущностей в~онтологии и~множества 
вспомогательных нетерминальных символов.
  
  Входными данными для построения онтологии являются:
  \begin{itemize}
\item корпус разбираемых текстов;\\[-15pt]
\item базы данных и~словники, позволяющие выделять в~тексте релевантные 
именованные сущности.
\end{itemize}

  Алгоритм состоит из пяти шагов:
  \begin{enumerate}[1.]
\item Подготовить начальную грамматику.\\[-15pt]
\item Применить к~корпусу текстов правила грамматики, заменив покрытые 
правилами фрагменты текста соответствующими нетерминалами.\\[-15pt]
\item Оценить покрытие корпуса текстов нетерминалами и~выбрать метод 
пополнения грамматики (см.\ ниже).\\[-15pt]
\item Пополнить грамматику новым правилом (см.\ ниже).\\[-15pt]
\item Перейти на шаг~2.
  \end{enumerate}
  
  Начальная грамматика содержит заранее определенный  
не\-тер\-ми\-нал-вер\-ши\-ну; множество нетерминальных символов, состоящее 
только из не-\linebreak тер\-ми\-на\-ла-вер\-ши\-ны; множество терминальных симво\-лов, 
совпадающее с~множеством токенов корпуса; множество правил, являющееся 
пустым.
  
  Оценка покрытия может производиться одним из двух способов.
  \begin{enumerate}[1.]
\item Выбрать из корпуса случайным образом~100~предложений, среди них 
найти наиболее частую синтаксическую конструкцию или тип именованной 
сущности, который еще не покрыт правилами грамматики.\\[-15pt]
\item Построить частотный список предложений, выбрать из них наиболее 
частое, для которого может быть написано правило СОКС-грам\-ма\-ти\-ки, 
не имеющее ложных срабатываний.
\end{enumerate}

  В результате оценки должно быть порождено правило одного из трех видов:
  \begin{enumerate}[(1)]
\item синтаксическое упрощение;\\[-15pt]
\item создание или пополнение газетира;\\[-15pt]
\item семантическое правило.
\end{enumerate}

  \textit{Синтаксическими упрощениями} называются прави\-ла грамматики, 
которые не отображаются в~результирующей онтологии, но обобщают 
однородные конструкции и~упрощают последующее расширение грамматики.
  
  К этому типу правил относятся, например,
  
\noindent
{\small
  \begin{verbatim}
and -> 'AND' | ',' | ',' 'AND' | ';' | ';' 'AND'
и
det -> 'A' | 'AN' | 'THE'
\end{verbatim}

}

  Необходимость создания или пополнения газетира возникает в~тех случаях, 
когда наиболее частым\linebreak\vspace*{-12pt}

\pagebreak

\noindent
 не покрытым нетерминалами явлением в~корпусе 
оказываются названия именованных сущностей, принадлежащие к~одному 
классу.
  
  Например, в~предложении
  \begin{verbatim}
PALMITOYLATION AT CYS-11, AND MUTAGENESIS
 OF SER-2; ARG-6 AND CYS-11.
\end{verbatim}
\noindent
четыре раза встречаются названия конкретных аминокислотных остатков в~
белке, представленные как название аминокислоты и~номер ее позиции, 
записанные через дефис. В тот момент, когда в~корпусе такие случаи 
становятся самыми частотными из неразобранных, необходимо пополнить 
газетир списком названий аминокислот.

  Третий вариант действий состоит в~том, чтобы пополнить  
СОКС-грам\-ма\-ти\-ку \textit{семантическим правилом}. Для этого 
необходимо выявить самую час\-тот\-ную конструкцию, такую что в~ней нет 
токенов, которые могли бы войти в~именованную сущность; в~ней нет лексики, 
играющей исключительно синтаксическую роль; она не сведена к~нетерминалу, 
являющемуся вершиной онтологии.
  
  Такая конструкция может являться предложением целиком, в~этом случае из 
нее будет образовано новое правило для СОКС-грамматики, в~левой час\-ти 
которого будет находиться вершина онтологии:
  \begin{verbatim}
feature -> 'STRUCTURE' 'BY' method
feature -> modification 'AT' range
\end{verbatim}


  Пример предложений, использующих приведенный фрагмент грамматики:
  \begin{verbatim}
STRUCTURE BY ELECTRON MICROSCOPY
 (9.4 ANGSTROMS).
PHOSPHOPANTETHEINYLATION AT SER-37.
\end{verbatim}

  Такая конструкция может одновременно быть предложением и~сводиться 
к~нетерминалу, который при этом не является вершиной онтологии, например:
  \begin{verbatim}
feature -> method
feature -> interaction
\end{verbatim}

  Пример предложений, использующих приведенный фрагмент грамматики:
  \begin{verbatim}
IDENTIFICATION BY MASS SPECTROMETRY.
CALMODULIN-BINDING.
\end{verbatim}

  Такая конструкция может являться частью предложения, в~этом случае 
нетерминал в~левой части правила не будет являться вершиной онтологии, 
например:
  \begin{verbatim}
interaction -> interaction 'WITH' protein
\end{verbatim}

  Пример предложений, использующих приведенный фрагмент грамматики:
  \begin{verbatim}
INTERACTION WITH MPK6
\end{verbatim}
  
  \subsection{Преобразование деревьев~синтаксического разбора 
в~онтологическое представление данных}
  
  В результате работы СОКС-пар\-се\-ра предложения исходного текста 
преобразуются в~деревья синтаксического разбора. Например, предложение
  \begin{verbatim}
FUNCTION, AND INTERACTION WITH RBM8A;
NXF1 AND THE EXON JUNCTION COMPLEX.
\end{verbatim}
после разбора преобразуется в~следующее дерево:
\begin{verbatim}
(description
 (feature
  (feature FUNCTION)
  (and , AND)
  (feature
    (interaction
     (interaction INTERACTION)
     WITH
     (protein
      (protein (protein RBM8A) (and ;)
       (protein NXF1))
      (and AND)
      (protein (det THE)
       (protein (words EXON JUNCTION)
        COMPLEX))))))
 .)
\end{verbatim}

  Такое дерево содержит набор связей, которые в~точности соответствуют 
онтологическим. Помимо таких связей в~дереве имеются связи и~узлы, 
имеющие синтаксическую роль (сочетание и~детерминанты). Кроме того, связи, 
отвечающие за сочетание, представлены здесь не как однородные связи внутри 
одного объекта, а~как вложенная рекурсивная цепочка связей.

  \begin{figure*}[b] %fig2
  \vspace*{6pt}
\begin{center}

{\small
  \begin{boxedverbatim}
> [X - RAY CRYSTALLOGRAPHY [1 . 80 ANGSTROMS]resolution OF [44 - 480]range 
OF [WILD - TYPE AND MUTANTS [TYR - 118 ; ARG - 168 AND ALA - 309]range]variant
IN [ACTIVE AND RESTING STATES]form AND IN [COMPLEX WITH [PEPTIDE SUBSTRATE]chemical]chemenv]feature
> [FUNCTION]feature
> [CATALYTIC ACTIVITY]feature
> [ENZYME REGULATION]feature
> [SUBSTRATE SPECIFICITY]feature
> [SUBUNIT]feature
> [DOMAIN]feature
> [PROTEOLYTIC AUTO - CLEAVAGE]feature
> [ACTIVE SITES]feature
> [SITES]feature
> [DISRUPTION PHENOTYPE]feature
  > [MUTAGENESIS OF [VAL - 118 ; ARG - 168 ; SER - 309 AND GLN - 338]range]feature AND
  \end{boxedverbatim}
  
  }
  
  \vspace*{5pt}
  
  \Caption{Разбор описания}
  \end{center}
  \end{figure*}
  
  Для преобразования деревьев такого вида в~онтологические факты 
необходимо:
  \begin{itemize}
\item заменить текстовое описание именованных сущностей на 
идентификатор базы данных (например, заменить \verb"RBM8A" на 
\verb"Q9Y5S9; RBM8A" является названием белка, общего для многих 
видов, база данных UniProt содержит 64~белка с~идентичным названием, 
текст данного предложения получен из описания белка, извлеченного из 
h.sapiens; следовательно, нас интересуют и~белки \verb"RBM8A" только из 
h.sapiens, такой белок только один);
\item нормализовать числовые значения (например, заменить 
на~4.2~поддерево 
\begin{verbatim}
(float (digits 4) . (digits 2)));
\end{verbatim}
\item раскрыть случаи сочетания необходимым для данного онтологического 
класса способом;
\item удалить нетерминалы, играющие синтаксическую роль (например, 
поддерево: \verb"(det THE)");
\item в~случаях, когда несколько аргументов обозначаются одним и~тем же 
нетерминалом, дать аргументам различные имена;
\item преобразовать правила грамматики в~объявление онтологических 
классов, отношений класс--под\-класс и~объявлений свойств;
\item преобразовать газетиры в~объявление онтологических индивидов и~отношений класс--ин\-дивид;
\item преобразовать дерево разбора в~объявление набора онтологических 
индивидов, объявление их отношения к~соответствующим онтологическим 
классам и~отношений часть--це\-лое и~атрибут для этих индивидов.
\end{itemize}

  Для приведенного примера фрагмент грамматики (вместе с~вставленными 
  в~него для на\-гляд\-ности фрагментами необходимых газетиров) выглядит 
следующим образом:

        \vspace*{-2pt}
        
        \noindent
  \begin{verbatim}
description -> feature '.'
feature -> feature and feature
feature -> interaction
feature -> 'FUNCTION'
interaction -> interaction 'WITH' protein
interaction -> 'INTERACTION'
protein -> protein and protein
protein -> words 'COMPLEX'
protein -> det protein
protein -> Q9Y5S9 | Q9UBU9
and -> 'AND' | ',' | ',' 'AND' | ';' |
 ';' 'AND'
        \end{verbatim}
        
%        \vspace*{-9pt}
        
  Он однозначным образом преобразуется в~набор определений (здесь авторы 
используют OWL2 functional notation~\cite{18-al}:
  \begin{verbatim}
Declaration(Class(:Description))
Declaration(Class(:Feature))
Declaration(Class(:Function))
Declaration(Class(:Interaction))
Declaration(Class(:Protein))
Declaration(ObjectProperty
 (:InteractionWith))
ObjectPropertyDomain(:InteractionWith 
 :Protein)
        
SubClassOf(:Feature :Description)
SubClassOf(:Interaction :Feature)
SubClassOf(:Function :Feature)
      
Declaration(NamedIndividual(:Q9Y5S9))
ClassAssertion(:Protein :Q9Y5S9)
Declaration(NamedIndividual(:Q9UBU9))
ClassAssertion(:Protein :Q9UBU9)
\end{verbatim}
        
  При этом приведенное описание трансформируется в~набор онтологических 
объектов:
  \begin{verbatim}
Declaration(NamedIndividual(:function1))
ClassAssertion(:Function :function1)

Declaration(NamedIndividual(:interaction1))
ClassAssertion(:Interaction :interaction1)

ObjectPropertyAssertion(:InteractionWith 
:interaction1 :Q9Y5S9)
ObjectPropertyAssertion(:InteractionWith 
:interaction1 :Q9UBU9)

Declaration(NamedIndividual(:protein1))
AnnotationAssertion( rdfs:comment 
:protein1 "EXON JUNCTION COMPLEX" )
ClassAssertion(:Protein :protein1)
\end{verbatim}

\vspace*{-16pt}

\section{Результаты и~обсуждение}
  

  
  В ходе работы была построена СОКС-грам\-ма\-ти\-ка, 
содержащая~179~правил (рис.~2).

\vspace*{-8pt}
  
  \subsection{Оценка покрытия}
  
  Для оценки была выбрана случайным образом тестовая выборка из 
100~предложений, 96~из них уникальные. Тестовая выборка содержит 
205~атомарных причин цитирования, 135~из них уникальные.
  
  Задача построения газетиров находится за пределами настоящей работы, 
поэтому в~тестовой выборке перед тестированием сущности, входящие 
в~газетиры, были вручную заменены на соответствующие им нетерминалы. 
Дополнительно в~грамматику были добавлены правила, позволяющие 
обрабатывать такие предобработанные входные данные.
  
  В тех случаях, где в~тестирующей выборке одна и~та же сущность могла быть 
описана более длинной или более короткой цепочкой, использовалась более 
короткая цепочка. Таким образом вручную были размечены классы: белок, 
вещество, болезнь, лекарство, химическая модификация.
  
  Полученные в~результате тестирования оценки покрытия представлены 
в~таблице.

{\small
  \begin{center}
  \tabcolsep=3pt
  \begin{tabular}{|l|c|}
  \multicolumn{2}{c}{Результаты тестирования покрытия}\\
  \multicolumn{2}{c}{\ }\\[-6pt]
  \hline
  \multicolumn{1}{|c|}{Тестирование} & Доля\\
  \hline
  Все атомарные причины цитирования&73\%\\
  Уникальные атомарные причины цитирования&43\%\\
  Все предложения&54\%\\
  Уникальные предложения&52\%\\
  \hline
  \end{tabular}
  \end{center}
}

\vspace*{6pt}
  


  
  Следует обратить внимание на значительный (в~1,7~раза) прирост покрытия 
при отключении процедуры удаления дубликатов из корпуса ато-\linebreak марных 
причин цитирования. Это является кос\-венным следствием большого числа 
дубликатов, которые, в~свою очередь, являются следствием огра\-ни\-чен\-ности 
выбранного языка (он использует только именные группы) и~его лексической 
огра\-ни\-чен\-ности (кураторы следуют инструкции, рег\-ламентирующей 
используемую лексику). Такие ограничения приводят к~значительному объему 
дуб\-ли\-рования в~корпусе. Это дает возможность при меньшем числе правил 
в~грамматике добиваться более высокого покрытия корпуса, что и~предложено 
в~настоящей статье.

 
  
  Очевидно, что более сложные конструкции обладают б$\acute{\mbox{о}}$льшим 
разнообразием и,~следовательно, меньшей степенью дублирования, что 
и~продемонстрировано на оценке покрытия полных предложений. Таким 
образом, для более сложных или менее ограниченных языков кажется 
осмысленным в~качестве предобработки выделять наиболее узко лишь такие 
конструкции, которые имеют сущности, значимые для составляемой онтологии. 
Для построения онтологий, описывающих объекты и~их свойства, такой 
предобработкой может служить выделение именных групп.

\vspace*{-6pt}
  
  \section{Заключение}
  
  В работе поставлена актуальная задача разработки новых онтологий на 
основе корпусных данных и~предложен подход к~ее решению. Для составления 
онтологий в~работе дано определение и~представлен алгоритм составления 
семантически ориентированных КС-грам\-ма\-тик. Важным аспектом подхода 
является использование в~качестве материала для построения онтологии 
корпуса предложений на ограниченном подмножестве естественного языка.
{\looseness=1

}
  
  Алгоритм опробован для текстов именных групп ограниченного языка, 
используемого в~базе UniProt для описания причин цитирования статьи, 
в~результате чего составлена грамматика и~разработан синтаксический 
анализатор таких причин цитирования.

\vspace*{-6pt}

{\small\frenchspacing
 {%\baselineskip=10.8pt
 \addcontentsline{toc}{section}{References}
 \begin{thebibliography}{99}
 
 \vspace*{-2pt}
\bibitem{1-al}
\Au{\mbox{Do{\!\!\ptb{\u{g}}}an} R.\,I., Leaman R., Lu~Zh.} Ncbi disease corpus: 
A~resource for 
disease name recognition and concept normalization~// J.~Biomed. Inform., 2014. 
Vol.~47. P.~1--10. doi: 10.1016/j.jbi.2013.12.006.
\bibitem{2-al}
\Au{Li Ch., Song R., Liakata~M., Vlachos~A., Seneff~S., Zhang~X.} Using word embedding for 
bio-event extraction~// 2015 Workshop on Biomedical Natural Language Processing 
(BioNLP 2015) Proceedings.~--- Beijing, China: ACL, 2015. P.~121--126. 
\bibitem{3-al}
\Au{Kim  J., Nguyen N., Wang~Yu., Tsujii~J., Takagi~T., Yonezawa~A.} The genia event and 
protein coreference tasks of the BioNLP shared task 2011~// BMC Bioinformatics, 2012. 
Vol.~13. Suppl.~11:S1. doi:10.1186/1471-2105-13-S11-S1.
\bibitem{4-al}
\Au{N$\acute{\mbox{e}}$dellec C., Bossy~R., Kim~Ji., Kim~Ju., Ohta~To., Pyysalo~S., 
Zweigenbaum~P.} Overview of BioNLP shared task 2013~// BioNLP Shared Task 2013 
Workshop (BioNLP-ST 2013) Proceedings.~---  ACL, 2013. P.~1--7.

\bibitem{6-al} %5
\Au{Tanabe M., Kanehisa~M.} Unit~1--12 using the KEGG database resource~// Current protocols in 
bioinformatics.~--- John Wiley\,\&\,Sons, Inc., 2012. P.~1.12.1--1.12.43.  doi: 10.1002/0471250953.bi0112s38.

\bibitem{5-al} %6
The UniProt Consortium. UniProt: A~hub for protein information~// Nucleic Acids Res., 
2015.  Vol.~43. P.~D204--D212. doi: 10.1093/nar/gku989.

\bibitem{7-al}
\Au{Tonkon M.\,J., Miller R.\,R., DeMaria~A.\,N., Vismara~L.\,A., Amsterdam~E.\,A., 
Mason~D.\,T.} Multifactor evaluation of the determinants of ischemic electrocardiographic 
response to maximal treadmill testing in coronary disease~// Am. J.~Med., 1977. Vol.~62. 
Iss.~3. P.~339--346. doi: 10.1016/0002-9343(77)90830-0.
\bibitem{8-al}
\Au{Giaretta P., Guarino~N.} Ontologies and knowledge bases towards a terminological 
clarification~// Towards very large knowledge bases.~--- Amsterdam: IOS Press.  
P.~25--32.
\bibitem{9-al}
\Au{Jones D., Bench-Capon~T., Visser~P.} Methodologies for ontology development~//  
IT\&KNOWS Conference, XV IFIP World Computer Congress Proceedings.~--- Budapest, 
1998.
\bibitem{10-al}
\Au{Reed S.\,L., Lenat D.\,B.} Mapping ontologies into Cyc~// AAAI 2002 Conference Workshop 
on Ontologies For The Semantic Web, 2002. P.~1--6.

\columnbreak


\bibitem{11-al}
\Au{Хомский Н.} Аспекты теории синтаксиса~/ Пер. В.\,А.~Звегинцева.~--- М.: Изд-во Моск. 
ун-та, 1972.  258~с. (\Au{Chomsky~N.} {Aspects of the theory of Syntax}.~---  MIT Press, 
1969. 261~p.)
\bibitem{12-al}
Criteria used to assign the pe level of entries. {\sf http:// www.uniprot.org/docs/pe\_criteria}.
\bibitem{13-al}
Controlled vocabulary. {\sf http://www.uniprot.org/help/\linebreak controlled\_vocabulary}.
\bibitem{14-al}
\textit{UniProt Consortium}. UniProt manual curation sop. {\sf 
http:// www.uniprot.org/docs/sop\_manual\_curation.pdf}.
\bibitem{15-al}
\Au{Beale A.} Spell checker oriented word lists. 1999--2015. 
{\sf http://wordlist.aspell.net/12dicts-readme}.
\bibitem{16-al}
\Au{Van~Rossum  G.} Python programming language~// USENIX Annual Technical 
Conference, 2007.
\bibitem{17-al}
\Au{Bird  S., Klein~E., Loper~E.} Natural language processing with Python.~--- O'Reilly Media, 
2009. 512~p.
\bibitem{18-al}
\Au{Horridge M., Patel-Schneider~P.\,F.} OWL 2 Web Ontology Language Manchester  
Syntax.~--- 2nd ed.~--- W3C Working Group Note, 2009. 
{\sf http://www.w3.org/TR/owl2-manchester-syntax}.
\end{thebibliography}

 }
 }

\end{multicols}

\vspace*{-3pt}

\hfill{\small\textit{Поступила в~редакцию 23.09.15}}

\vspace*{8pt}

%\newpage

%\vspace*{-24pt}

\hrule

\vspace*{2pt}

\hrule

%\vspace*{8pt}



\def\tit{BioNLP ONTOLOGY EXTRACTION FROM A~RESTRICTED LANGUAGE CORPUS 
WITH CONTEXT-FREE GRAMMARS}

\def\titkol{BioNLP ontology extraction from a~restricted language corpus 
with context-free grammars}

\def\aut{D.\,A.~Alexeyevsky}

\def\autkol{D.\,A.~Alexeyevsky}

\titel{\tit}{\aut}{\autkol}{\titkol}

\vspace*{-9pt}

\noindent
National Research University Higher School of Economics; 20~Myasnitskaya 
Str., Moscow 101000, Russian Federation

\def\leftfootline{\small{\textbf{\thepage}
\hfill INFORMATIKA I EE PRIMENENIYA~--- INFORMATICS AND
APPLICATIONS\ \ \ 2016\ \ \ volume~10\ \ \ issue\ 1}
}%
 \def\rightfootline{\small{INFORMATIKA I EE PRIMENENIYA~---
INFORMATICS AND APPLICATIONS\ \ \ 2016\ \ \ volume~10\ \ \ issue\ 1
\hfill \textbf{\thepage}}}

\vspace*{3pt}

  
  
    
  
\Abste{BioNLP is an emerging area of NLP that brings new challenging objects for 
language processing and new valuable resources for bioinformatics and medicine. 
One notable task in BioNLP is creating de-novo ontologies. This is generally 
a~tedious process; however, in some cases, it is possible to automate it to some extent. 
One such case is when a corpus of texts in a restricted subset of natural language is 
available. This paper presents a simple approach to automate ontology creation in 
such cases. The approach is aimed to simplify mapping of entities in natural texts to 
predefined ontologies wherever possible. The paper discusses which properties of the 
corpus enable the approach presented.}

\KWE{BioNLP; ontology creation; context-free grammar}


\DOI{10.14357/19922264160111}

\vspace*{-12pt}

\Ack
\noindent
The work was partly supported by the Russian Foundation for Basic
Research (project 15-07-09306).



%\vspace*{3pt}

  \begin{multicols}{2}

\renewcommand{\bibname}{\protect\rmfamily References}
%\renewcommand{\bibname}{\large\protect\rm References}

{\small\frenchspacing
 {%\baselineskip=10.8pt
 \addcontentsline{toc}{section}{References}
 \begin{thebibliography}{99}
\bibitem{1-al-1}
\Aue{Do{\!\!\ptb{\!\u{g}}}an, R. I., R.~Leaman, and Zh.~Lu.} 2014. Ncbi disease 
corpus: A~resource for disease name and concept normalization. 
\textit{J.~Biomed. Inform.} 47:1--10. doi: 10.1016/j.jbi.2013.12.006.
\bibitem{2-al-1}
\Aue{Li, Ch., R. Song, M.~Liakata, A.~Vlachos, S.~Seneff, and X.~Zhang}. 
2015. Using word embedding for bio-event extraction. \textit{2015 Workshop on 
Biomedical Natural Language Processing (BioNLP 2015) Proceedings}. Beijing, 
China: ACL. 121--126.
\bibitem{3-al-1}
\Aue{Kim,  J., N. Nguyen, Yu.~Wang,  J.~Tsujii, T.~Takagi, and A.~Yonezawa}. 
2012. The genia event and protein coreference tasks of the BioNLP shared task 
2011. \textit{BMC Bioinformatics} 13(Suppl.~11:S1).  
doi: 10.1186/1471-2105-13-S11-S1.
\bibitem{4-al-1}
\Aue{N$\acute{\mbox{e}}$dellec, C., R.~Bossy, Ji.~Kim, Ju.~Kim, To.~Ohta, 
S.~Pyysalo, and P.~Zweigenbaum}. 2013. Overview of BioNLP shared task 2013. 
\textit{BioNLP Shared Task 2013 Workshop (BioNLP-ST 2013) Proceedings}. 
ACL. 1--7.

\bibitem{6-al-1} %5
\Aue{Tanabe, M., and M.~Kanehisa}. 2012. Unit~1--12 using the KEGG database resource. 
\textit{Current protocols in bioinformatics}. 
John Wiley\,\&\,Sons, Inc. 1.12.1--1.12.43.   doi: 
10.1002/0471250953.bi0112s38.

\bibitem{5-al-1} %6
The UniProt Consortium. 2015. Uniprot: A~hub for protein information. 
\textit{Nucleic Acids Res.} 43:D204--D212. doi:10.1093/nar/gku989.
\bibitem{7-al-1}
\Aue{Tonkon, M.\,J., R.\,R.~Miller, A.\,N.~DeMaria, L.\,A.~Vismara, 
E.\,A.~Amsterdam, and D.\,T.~Mason}. 1977. Multifactor evaluation of the 
determinants of ischemic electrocardiographic response to maximal treadmill 
testing in coronary disease. \textit{Am. J.~Med.} 62(3):339--346. doi: 
10.1016/0002-9343(77)90830-0.
\bibitem{8-al-1}
\Aue{Giaretta, P., and N.~Guarino}. 1995. Ontologies and knowledge bases 
towards a terminological clarification. \textit{Towards very large knowledge 
bases}. Amsterdam: IOS Press. 25--32.
\bibitem{9-al-1}
\Aue{Jones, D., T.~Bench-Capon, and P.~Visser}. 1998. Methodologies for 
ontology development. \textit{IT\&KNOWS Conference, XV IFIP World 
Computer Congress Proceedings}. Budapest. 62--75.
\bibitem{10-al-1}
\Aue{Reed, S.\,L., and D.\,B.~Lenat.} 2002. Mapping ontologies into Cyc. 
\textit{AAAI 2002 Conference Workshop on Ontologies For The Semantic Web}. 
1--6.
\bibitem{11-al-1}
\Aue{Chomsky, N.} 1969. \textit{Aspects of the theory of Syntax}.  MIT Press. 
261~p. 
\bibitem{12-al-1}
Criteria used to assign the pe level of entries. Available at: {\sf 
http://www.uniprot. org/docs/pe\_criteria} (accessed January~21, 2016).
\bibitem{13-al-1}
Controlled vocabulary. Available at: {\sf 
http://www.uniprot. org/help/controlled\_vocabulary} (accessed January~21, 
2016).
\bibitem{14-al-1}
UniProt Consortium. Uniprot manual curation sop. Available at: {\sf 
http://www.uniprot.org/docs/sop\_manual\_\linebreak curation.pdf} (accessed January~21, 
2016).
\bibitem{15-al-1}
\Aue{Beale, A.} 1999--2015. \textit{Spell checker oriented word lists}. Available 
at: {\sf http://wordlist.aspell.net/12dicts-readme} (accessed January~21, 2016).
\bibitem{16-al-1}
\Aue{Van Rossum, G.} 2007. Python programming language. \textit{USENIX 
Annual Technical Conference}.
\bibitem{17-al-1}
\Aue{Bird, S., E.~Klein, and E.~Loper}. 2009. \textit{Natural language 
processing with Python}. O'Reilly Media. 512~p.
\bibitem{18-al-1}
\Aue{Horridge, M., and P.\,F.~Patel-Schneider}. 2009. OWL~2 Web Ontology 
Language Manchester Syntax. 2nd ed. W3C Working Group Note. Available at: 
{\sf http://www.w3.org/TR/owl2-manchester-syntax} (accessed January~21, 
2016).

\end{thebibliography}

 }
 }

\end{multicols}

\vspace*{-3pt}

\hfill{\small\textit{Received September 23, 2015}}

\Contrl

\noindent
  \textbf{Alexeyevsky Daniil A.} (b.\ 1983)~--- PhD student, Faculty of 
Humanities, National Research University Higher School of Economics; 
20~Myasnitskaya Str., Moscow 101000, Russian Federation; 
dalexeyevsky@hse.ru

   
\label{end\stat}


\renewcommand{\bibname}{\protect\rm Литература}   %8
\def\stat{zatsman}

\def\tit{ТРАНСФОРМАЦИИ ОБЪЕКТОВ ПЕРВОГО И~ВТОРОГО ПОРЯДКА 
В~ЛЕКСИКОГРАФИЧЕСКОЙ ИНФОРМАЦИОННОЙ СИСТЕМЕ$^*$}

\def\titkol{Трансформации объектов первого и~второго порядка 
в~лексикографической информационной системе}

\def\aut{И.\,М.~Зацман$^1$}

\def\autkol{И.\,М.~Зацман}

\titel{\tit}{\aut}{\autkol}{\titkol}

\index{Зацман И.\,М.}
\index{Zatsman I.\,M.}


{\renewcommand{\thefootnote}{\fnsymbol{footnote}} \footnotetext[1]
{Исследование выполнено в~ФИЦ ИУ РАН за счет гранта Российского научного фонда №\,24-18-00155, {\sf 
https://rscf.ru/project/24-18-00155}. Работа выполнялась с~использованием инфраструктуры Центра 
коллективного пользования <<Высокопроизводительные вычисления и~большие данные>> (ЦКП 
<<Информатика>>) ФИЦ ИУ РАН (г.\ Москва).}}


\renewcommand{\thefootnote}{\arabic{footnote}}
\footnotetext[1]{ Федеральный исследовательский центр <<Информатика и~управление>> Российской академии наук, 
\mbox{izatsman@yandex.ru}}

\vspace*{-12pt}


  
  \Abst{Рассматриваются теоретические основания проектирования информационных 
технологий (ИТ) интеграции двуязычных словарей и~параллельных корпусов. Дано описание 
первых результатов создания третьего уровня классификации трансформаций объектов 
предметной области информатики, которую предполагается использовать при создании 
концепции лексикографической информационной системы, обеспечивающей интеграцию. 
Все сущности информатики в~статье разделены на два глобальных класса: объекты и~их 
трансформации. Для каждого такого класса конструируется своя классификация. Ранее были 
описаны два верхних уровня классификации трансформаций объектов предметной области. 
В~данной статье рассматривается третий уровень этой классификации. Основанием для 
построения самого верхнего ее уровня служило деление предметной области информатики 
на среды (ментальная, сенсорно воспринимаемая, цифровая и~ряд других сред), каждая из 
которых по определению включает объекты одной природы. Основанием для построения 
второго уровня классификации трансформаций объектов служила типология знаковых  
сис\-тем А.~Соломоника. Цель статьи состоит в~систематизации трансформаций первого 
и~второго порядка объектов предметной области на третьем уровне этой классификации. 
Основанием для систематизации служит средовая версия иерархии Акоффа.}
  
  \KW{объекты предметной области; трансформации объектов; классификация; данные; 
информация; знание; лексикографическая информационная сис\-тема}

\DOI{10.14357/19922264240211}{VZTGVV}
  
\vspace*{3pt}


\vskip 10pt plus 9pt minus 6pt

\thispagestyle{headings}

\begin{multicols}{2}

\label{st\stat}
  
\section{Введение}

\vspace*{-9pt}

  Возникновение параллельных корпусов, в~которых предложениям 
оригинального текста со\-по\-став\-ле\-ны предложения его перевода, обеспечило 
возможность контрастивного лингвистического\linebreak \mbox{анализа} на принципиально 
новом уровне полноты и~точности, недостижимом в~докорпусную эпоху. 
Пионерскими в~этой области стали работы \mbox{1990-х~гг}. Стига Йоханссона  
с~анг\-ло-нор\-веж\-ским корпусом~[1]. В России параллельные корпусы стали 
формироваться в~начале XXI~века в~рамках Национального корпуса русского 
языка~[2].
  
  Создатели двуязычных словарей используют параллельные корпусы для 
сбора материала и~эмпирической проверки своих гипотез, касающихся 
межъязы\-ко\-вой эквивалентности. Ценность параллельных корпусов 
определяется тем, что в~лингвистике этап сбора исходного материала считается 
наиболее трудоемким и~наименее творческим, а~параллельные корпусы 
позволяют значительно сэкономить время и~силы для творческого этапа 
создания словарей~[3].
 % 
  При этом двуязычные словари, создаваемые на основе исходного материала, 
извлеченного из параллельных корпусов, сейчас формируются без связей с~их 
текстами. Другими словами, онлайновые связи созданных словарей 
с~параллельными корпусами, которые служили источниками исходного 
материала, отсутствуют. 

Параллельные корпусы постоянно пополняются 
новыми текстами, в~предложениях которых можно обнаружить новые значения 
слов и~устойчивых словосочетаний. Однако при этом отсутствуют методы 
и~средства оперативного обновления словарей по корпусным данным. 
В~настоящее время проблема установления связей между двуязычными 
словарями и~параллельными корпусами (далее~--- проблема интеграции) 
находится на стадии поиска концептуальных подходов к~их интеграции на 
уровне значений.
  
  Подход к~решению проблемы интеграции, предлагаемый в~статье, учитывает 
  и~появление новых значений слов и~устойчивых словосочетаний, и~динамику 
смысловых значений, которая обусловлена развитием и~пополнением знания 
лингвистов, фиксирующих эти значения в~результате семантического анализа 
пополняемых корпусных данных. Проведенные эксперименты показали, что 
обнаружение нового лингвистического знания обусловливает и~формирование 
дефиниций новых значений, и~пересмотр уже существующих дефиниций~[4, 5].
  
  Например, в~проведенных экспериментах с~использованием ЦКП 
<<Информатика>> ФИЦ ИУ РАН фиксировалась эволюция значений немецких 
модальных глаголов, исходное состояние значений которых было описано 
в~не\-мец\-ко-рус\-ском словаре. В~экспериментальном массиве текстов как 
потенциальных источниках нового знания 16\,268 предложений содержали 
немецкие модальные глаголы и~в~2041 из них встречался глагол sollen. 
В~начале эксперимента в~словаре были описаны~12~значений этого модального 
глагола. По окончании эксперимента лингвисты обнаружили два новых его 
значения, согласовали их дефиниции и~описали эволюцию дефиниций~[6, 7].
  
  Таким образом, для решения проблемы интеграции требуется фиксировать 
новое знание, обнаруженное лингвистами в~текстовых данных параллельных 
корпусов, отслеживать эволюцию знания, представленного в~виде дефиниций 
значений слов и~устойчивых словосочетаний, и,~соответственно, 
актуализировать электронные двуязычные словари. Предлагаемый 
концептуальный подход к~интеграции, который планируется реализовать 
в~процессе проектирования лексикографической информационной сис\-те\-мы, 
фиксирующей эволюцию лингвистического знания, основан на решении 
следующих задач:\\[-14pt]
  \begin{itemize}
  \item категоризация трех базовых понятий информатики, включенных 
  в~иерархию Акоффа~[8] (данные, информация, знание), на объекты 
проектируемой сис\-те\-мы, которая необходима, чтобы фиксировать 
<<кванты>> нового знания и~отслеживать его эволюцию в~этой сис\-теме;\\[-15pt]
  \item  систематизация трансформаций объектов этой сис\-темы.\\[-14pt]
  \end{itemize}
  
  Цель статьи и~состоит в~решении двух задач: категоризации трех базовых 
понятий информатики на объекты лексикографической информационной  
сис\-те\-мы и~сис\-те\-ма\-ти\-за\-ции трансформаций первого и~второго порядка 
ее объектов.
  
  Трансформациями первого порядка, о которых сказано в~формулировке цели 
статьи, называются взаимные преобразования между двумя объектами  
сис\-те\-мы одной природы. Например, перевод в~сис\-те\-ме текста с~русского 
языка на английский относится к~ним. Трансформациями второго порядка 
и~выше называются взаимные преобразования между двумя и~более объектами 
разной природы. Например, кодирование символов текс\-та компьютерными 
кодами и~их декодирование относятся по определению к~трансформациям 
второго порядка.

%\vspace*{-9pt}
  
\section{Процессы трансформаций в~информатике}

%\vspace*{-3pt}

Процессы трансформаций, рассматриваемые в~статье, относятся к~теоретическому ядру информатики, а~не 
только к~проектированию лексикографической информационной сис\-те\-мы. Например, из трех основных 
подходов к~описанию предметной об\-ласти информатики\footnote{В статье предметная область информатики 
трактуется согласно концепции полиадического компьютинга Пола Розенблума~\cite{9-zac}.} (объектный, 
трансформационный и~синтетический) сис\-те\-ма\-ти\-за\-ция трансформаций ближе всего ко второму 
подходу. Примерами первого подхода, в~рамках которого основное внимание уделяется объектам предметной 
области информатики и~в~меньшей степени отношениям\linebreak между ними, могут служить  
работы~\cite{8-zac, 10-zac, 11-zac}; \mbox{примерами} второго подхода, в~рамках которого основное внимание 
уделяется трансформациям и~в~меньшей степени трансформируемым объектам,~---  
работы~\cite{12-zac, 13-zac}; примерами третьего, синтетического подхода, в~котором уделяется внимание 
и~объектам предметной об\-ласти информатики, и~отношениям между ними, могут служить работы~\cite{14-zac, 
15-zac, 16-zac, 17-zac, 18-zac}.

  Таким образом, для описания трансформаций объектов лексикографической 
информационной\linebreak системы предпочтительнее всего трансформационный 
подход, который упоминается и~в определениях информатики. Например, 
в~2009~г.\ П.~Деннинг и~П.~Розенблум сформулировали суть \mbox{информатики} как 
компьютинга следующим образом: <<$\ldots$информатика~--- это не просто 
алгоритмы и~структуры данных; это преобразования [трансформации] 
представлений>>~\cite{12-zac}. Чуть позже, в~контексте краткого описания 
парадигмы информатики как компьютинга, П.~Деннинг и~П.~Фриман изменили 
эту формулировку на такую: <<Центральный объект внимания в~информатике 
можно определить как информационные процессы~--- \textit{естественные или 
искусственные процессы, преобразующие информацию} (курсив мой~--- 
И.\,З.)>>~\cite{13-zac}. Согласно парадигме, предлагаемой авторами этой 
статьи, на начальном этапе проектирования автоматизированных систем 
базовыми элементами моделей их функционирования служат 
\textit{информационные про\-цессы}.
  
  Однако если 15~лет назад в~формулировке из работы~\cite{13-zac} шла речь 
о~процессах, преобразующих информацию, то в~последние~10~лет в~спектр 
процессов трансформаций все чаще стали включать процессы, преобразующие 
не только информацию, но также и~другие объекты автоматизированных 
систем, в~первую очередь данные и~знания~[19--21]. Например, Виктория 
Стодден, позиционируя науку о~данных как одну из дисциплин информатики, 
говорит, что центральный объект исследований в~науке о~данных~--- это 
<<изучение обобщаемого извлечения знания из данных>>~\cite{21-zac}. 
Увеличение и~чис\-ла объектов, и~спект\-ра процессов их трансформаций 
в~автоматизированных сис\-те\-мах обуслов\-ли\-ва\-ет не\-об\-хо\-ди\-мость 
систематизации и~объектов, и~процессов их трансформаций на начальном этапе 
проектирования сис\-тем.
  
  Для создания концепции лексикографической информационной сис\-те\-мы 
и~проектирования ИТ, обеспечивающих интеграцию 
двуязычных словарей и~параллельных корпусов, сначала выполним 
категоризацию на объекты этой сис\-те\-мы трех базовых понятий информатики 
(данные, информация, знание) в~контексте построения классификаций 
сущностей ее предметной об\-ласти.
  
  Необходимость использования классификаций информатики в~процессе 
создания концепции проиллюстрируем, используя иерархию  
Акоффа~\cite{8-zac}. Он использовал принцип их вертикального размещения 
в~иерархии снизу вверх: данные, информация и~знание. Еще в~ней есть термин 
<<мудрость>>, который в~статье не рассматривается. Такое размещение Акофф 
прокомментировал так: <<Каждое из пе\-ре\-чис\-лен\-ных понятий [кроме данных] 
содержит в~себе нижестоящие$\ldots$>>~\cite{8-zac}.
  
  Этому принципу размещения и~комментарию Акоффа свойственны 
недостатки, проанализированные, в~частности, в~работе~\cite{10-zac}. Главный 
вывод, к~которому пришла Роули после изучения иерархии Акоффа, 
заключается в~следующем: <<$\ldots$информация определяется в~терминах 
данных, знание~--- в~терминах информации$\ldots$ но существует меньше 
консенсуса в~описании трансформаций, которые преобразуют сущности, 
расположенные ниже в~иерархии, в~те, которые находятся над ними, что 
приводит к~их терминологической неопределенности>>~\cite{10-zac}. Причина 
этой неопределенности, скорее всего, в~том, что базовые понятия информатики 
включены в~иерархию Акоффа изолированно от общего контекста 
классификаций сущностей ее предметной об\-ласти.

%\vspace*{-9pt}
  
\section{Классификации сущностей информатики}


%\vspace*{-2pt}

  Все сущности предметной области информатики в~работах~[22, 23] 
разделены на два глобальных класса: ее объекты и~их трансформации. Для 
каждого такого класса была предложена своя классификация. 
В~работе~\cite{22-zac} дано описание классификации объектов предметной 
области информатики, первый уровень которой содержит базовые понятия ее 
предметной области (данные, информация, знания и~др.).  
В~работе~\cite{23-zac} дано описание двух верхних уровней классификации 
трансформаций объектов предметной об\-ласти (см.\ рисунок 
в~работе~\cite{23-zac}). Основанием для построения самого верхнего ее уровня послужило деление 
предметной области информатики на среды\footnote{В~работе~\cite{24-zac} дано описание пяти сред 
предметной области информатики (ментальная; сенсорно воспринимаемая, или информационная; 
цифровая; нейро- и~ДНК-среда), каждая из которых по определению включает объекты одной и~той же 
природы.} и~степень разнообразия природы объектов, вовлеченных в~трансформации:
\begin{itemize}
\item  первый класс верхнего уровня классификации включает 
трансформации объектов в~пределах среды только одной природы 
(трансформации первого порядка);
\item  второй класс включает трансформации объектов, относящихся 
к~двум средам разной природы (трансформации второго порядка);
\item третий и~последующие классы включают трансформации объектов, 
относящихся к~трем и~более средам разной природы (трансформации 
третьего и~более высоких порядков).
\end{itemize}

  В работе~\cite{23-zac} были приведены примеры для трех первых классов 
трансформаций, включая пример трансформаций объектов, относящихся 
к~двум средам разной природы (компьютерное кодирование символов текстов 
с~по\-мощью таб\-лиц Unicode).
  
Основанием для построения второго уровня классификации трансформаций объектов послужила типология 
знаковых сис\-тем А.~Соломоника~\cite[c.~131]{25-zac}: естественные знаковые сис\-те\-мы, образные,  
ес\-тест\-вен\-но-язы\-ко\-в$\acute{\mbox{ы}}$е,  
вер\-баль\-но-не\-сло\-вес\-ные сис\-те\-мы записи\footnote{Под системой записи понимается знаковая 
система, сочетающая вербальные знаки с~несловесными (языки нотной записи, карт, таблиц и~др.).} 
и~формализованные знаковые сис\-те\-мы, включая математические. Введем понятие обобщенного текста~--- 
это текст, который может быть создан в~любой из перечисленных знаковых систем. Тогда обобщенные тексты 
могут быть естественными, образными, ес\-тест\-вен\-но-язы\-ко\-в$\acute{\mbox{ы}}$\-ми,  
вер\-баль\-но-не\-сло\-вес\-ны\-ми и~формализованными. Второй уровень классификации трансформаций 
охватывает не все виды объектов предметной  
об\-ласти информатики, а~только перечисленные~5~видов текс\-тов и~их представления, вовлеченные 
в~процессы трансформаций в~одной или более средах вместе с~данными, знанием и~его концептами.

\begin{figure*}[b] %fig1
\vspace*{6pt}
      \begin{center}
     \mbox{%
\epsfxsize=121.191mm 
\epsfbox{zac-1.eps}
}
\end{center}
\vspace*{-6pt}
\Caption{Средовая версия иерархии Акоффа}
\end{figure*}

\section{Классификация трансформаций: построение~третьего 
уровня}

  Основанием для систематизации трансформаций первого и~второго порядка 
на третьем уровне этой классификации служит иерархия Акоффа~\cite{8-zac}, 
на основе которой и~была создана ее средов$\acute{\mbox{а}}$я версия~[26, 
27]. Для создания средов$\acute{\mbox{о}}$й версии была выполнена 
категоризация трех базовых понятий информатики (данные, информация, 
знания) на объекты лексикографической информационной сис\-те\-мы 
в~процессе создания ее концепции\linebreak (рис.~1).
  


  В отличие от классической иерархии Акоффа, в~ее 
средов$\acute{\mbox{о}}$й версии различаются три вида данных: сенсорно 
воспринимаемые, цифровые и~те данные, которые генерируются 
искусственными нейронными сетями (ИНС) в~системах искусственного интеллекта 
(далее~--- ИИ-дан\-ные). Последний вид данных необходим, например, для 
различения входа и~выхода процесса применения обученной 
ИНС в~цифровой модели генерации знания, описанию которой 
посвящена работа~\cite{27-zac}.
  
  Также предлагается различать два вида информации: сенсорно 
воспринимаемая и~цифровая. Кроме знания в~средов$\acute{\mbox{у}}$ю 
версию добавлены концепты и~ментальные образы сенсорно воспринимаемых 
данных. Последние служат промежуточной сущностью между сенсорно 
воспринимаемыми данными и~генерируемым знанием при описании процессов 
извлечения знания из текстовых данных лексикографической информационной 
системы. Описание объектов средов$\acute{\mbox{о}}$й версии иерархии 
Акоффа (см.\ рис.~1) и~отношений между ними дано в~работах~\cite{26-zac, 28-zac}.
  
  В средов$\acute{\mbox{о}}$й версии число объектов равно восьми. Если 
учитывать направления трансформаций, то между восемью объектами на 
рис.~1 она включает~16 их видов (трансформации на границе между сенсорно 
воспринимаемыми данными и~информацией, обозначенные символом~<<?>>, 
в~статье не рас\-смат\-ри\-ва\-ют\-ся). В~будущем число объектов 
в~средов$\acute{\mbox{о}}$й версии, которая выбрана как основание для 
сис\-те\-ма\-ти\-за\-ции трансформаций первого и~второго порядка, может быть 
увеличено. Для построения классификации трансформаций 
важ\-но не возможное увеличение числа объектов 
и~трансформаций между ними, а то, что их виды в~средов$\acute{\mbox{о}}$й 
версии распределены между трансформациями первого и~второго порядка. Из 
16~видов на рис.~1 шесть относятся к~трансформациям первого порядка, это\linebreak 
виды с~номерами~7, 8, 13--16 (далее~--- типология трансформаций первого 
порядка), а~десять~--- к~трансформациям второго порядка, это виды 
с~\mbox{номерами}~1--6 и~9--12 (далее~--- типология трансформаций второго 
порядка). Разместим обе типологии на третьем уровне классификации (см.\ ее 
схему на рис.~2). Перечислим виды трансформаций первой типологии, вводя 
в~скобках их краткие названия, используемые ниже на рис.~3:
  \begin{description}
  \item[\,] 7~--- членение знания на концепты с~помощью одной или нескольких 
знаковых систем (далее~--- членение знания);
  \item[\,] 8~--- формирование знания на основе концептов (формирование 
знания);
  \item[\,] 13~--- обучение ИНС;
  \end{description}
  
  \vspace*{-6pt}
  
  \pagebreak
  
  \end{multicols}
  
  \begin{figure*} %fig2
\vspace*{1pt}
      \begin{center}
     \mbox{%
\epsfxsize=127.513mm 
\epsfbox{zac-2.eps}
}
\end{center}
\vspace*{-9pt}
\Caption{Схема трех верхних уровней классификации трансформаций объектов (объединены 
по три слоя и~для второго, и~для третьего уровней этой классификации)}
\end{figure*}
  
  \begin{multicols}{2}
  
  \noindent
  \begin{description}
  \item[\,] 14~--- восстановление обучающей информации на основе 
содержания обученной ИНС (обращение ИНС);
  \item[\,] 15~--- использование обученной ИНС (использование ИНС);



  \item[\,] 16~--- восстановление исходных данных, соответствующих 
полученным результатам работы обучен\-ной ИНС (восстановление исходных данных 
по результатам ИНС).
  \end{description}
  
  
  Не все виды трансформаций 13--16 поддерживаются в~конкретных системах 
искусственного интеллекта, но с~теоретической точки зрения все их 
предлагается включить в~первую типологию для полноты спектра видов 
трансформаций.
  
  Перечислим виды трансформаций второй типологии:
  \begin{description}
  \item[\,] 1~--- декодирование цифровых данных в~компьютерных системах 
(декодирование данных);
  \item[\,]  2~--- кодирование сенсорно воспринимаемых данных (кодирование 
данных);
  \item[\,] 3~--- ментальное копирование сенсорно воспринимаемых данных 
(ментальное копирование);
  \item[\,] 4~--- восстановление сенсорно воспринимаемых данных по 
ментальным образам (восстановление по образам);
  \item[\,] 5~--- смысловая интерпретация без деления на концепты ментальных 
образов сенсорно воспринимаемых данных (смысловая интерпретация);
  \item[\,] 6~--- восстановление ментальных образов (восстановление образов);
  \item[\,] 9~--- представление концептов в~виде сенсорно воспринимаемой 
информации, например текс\-та\-ми, формулами, таблицами, рисунками и~т.\,д.\ 
(представление концептов);
  \item[\,] 10~--- понимание смысла сенсорно воспринимаемой информации 
(понимание смысла);
  \item[\,] 11~--- кодирование сенсорно воспринимаемой информации 
(кодирование информации);
\end{description}

\vspace*{-6pt}

\pagebreak

\end{multicols}

\begin{figure*} %fig3
\vspace*{1pt}
      \begin{center}
     \mbox{%
\epsfxsize=163mm 
\epsfbox{zac-3.eps}
}
\end{center}
\vspace*{-9pt}
\Caption{Схема частного случая классификации трансформаций объектов (трансформации 
пронумерованы согласно рис.~1)}
\end{figure*}

\begin{multicols}{2}

\noindent
\begin{description}

  \item[\,] 12~--- декодирование цифровой информации (декодирование 
информации).
  \end{description}
  
  Отметим, что в~существующих ИТ
  и~компьютерных системах наиболее часто используются виды 
трансформаций~13 и~15 типологии первого порядка и~1, 2, 11 и~12 типологии 
второго порядка. На рис.~2 в~первом слое третьего уровня классификации 
показаны типологии первого порядка без указания числа трансформаций в~них 
и~без детализации трансформируемых объектов.
  
  Во втором слое третьего уровня классификации условно (без названий) 
показаны типологии второго порядка. Также на рис.~2 в~третьем слое третьего 
уровня классификации условно (также без названий) показаны типологии 
третьего порядка, которые планируется рассмотреть в~отдельной статье. По 
определению они должны включать трансформации между тремя объектами 
разной природы, но средов$\acute{\mbox{а}}$я версия иерархии Акоффа 
включает трансформации только между двумя объектами разной природы. 
Поэтому потребуется другое основание для их систематизации (ранее были 
рассмотрены отдельные примеры трансформаций третьего 
порядка\footnote{Далеко не всегда трансформации третьего и~более высоких порядков можно 
рассматривать как последовательность трансформаций второго порядка. Примером этого могут 
служить трансформации в~процессе обучения пациента пользованию роботизированной рукой, 
охватывающие личностные концепты пациента, релевантные его намерениям, сигналы активности 
мозга как объекты нейросреды и~компьютерные коды~\cite{29-zac}.}~\cite{29-zac}).

\section{Классификация трансформаций: частный~случай}

  Выше было отмечено, что в~будущем число объектов 
в~средов$\acute{\mbox{о}}$й версии иерархии Акоффа может быть увеличено. 
Это означает, что увеличатся и~чис\-ло объектов, и~чис\-ло трансформаций между 
ними в~классификации трансформаций, так как эта средов$\acute{\mbox{а}}$я 
версия служит по определению основанием для систематизации 
трансформаций первого и~второго порядка. Поэтому на третьем уровне рис.~2 
указаны типологии без детализации объектов и~без указания числа 
трансформаций в~каждой из них. С~одной стороны, при таком подходе 
получаем достаточно общий вид этой классификации, так как она не зависит от 
числа объектов в~том или ином варианте средов$\acute{\mbox{о}}$й версии 
(и~это существенно упрощает рис.~2). С~другой стороны, на третьем уровне 
такой общей классификации подразумевается, но не эксплицируется природа 
трансформируемых объектов и~их возможные сочетания в~трансформациях. 

При проектировании лексикографической информационной системы важно 
эксплицировать природу трансформируемых объектов и~их возможные 
сочетания.
  %
  Поэтому в~парадигму информатики~\cite{30-zac} кроме общей 
классификации трансформаций предлагается включать и~ее частные случаи, 
эксплицирующие природу трансформируемых объектов. 

В~этом разделе 
рассмотрим один частный случай, когда используются только естественные 
знаковые сис\-те\-мы из типологии А.~Соломоника~\cite{25-zac} вместе 
с~данными, знанием и~его концептами. Чис\-ло естественных языков при этом не 
ограничено. И~этот частный случай классификации включает только три 
класса природных трансформаций (первого, второго и~третьего порядка, см.\ 
схему классификации на рис.~3).
  
  Первый и~второй уровни схемы общей классификации (см.\ рис.~2) можно 
объединить в~один уровень в~этом частном случае. Ниже этого уровня 
приведено содержание типологий первого и~второго порядка без содержания 
типологий третьего по\-рядка.




  Наполнение типологий первого и~второго порядка соответствует 
средов$\acute{\mbox{о}}$й версии иерархии Акоффа на рис.~1, содержащей 
6~видов трансформаций типологии первого порядка и~10~видов 
трансформаций типологии второго порядка (на рис.~3 стрелки указывают 
направления трансформаций согласно средов$\acute{\mbox{о}}$й версии на рис.~1).
  
  Таким образом, частный случай классификации содержит для этих двух 
типологий 16~теоретически возможных трансформаций, 6 из которых 
в~настоящее время в~существующих ИТ применяются наиболее часто: виды 
трансформаций~1, 2, 11 и~12 типологии второго порядка реализуются 
с~помощью тех или иных методов ко\-ди\-ро\-ва\-ния/де\-ко\-ди\-ро\-ва\-ния 
(например, с~использованием таблиц Unicode), а~виды трансформаций~13 и~15
 в~типологии первого порядка реализуются полностью с~по\-мощью процессов 
цифровой обработки компьютерами.
  
  Остальные виды трансформаций или применяются намного реже (это 
виды~3, 5, 7, 9 и~10), или находятся в~стадии поиска и~разработки (14 и~16) или 
в~настоящее время носят только теоретический характер, обеспечивая полноту 
первой и~второй типологий (4, 6 и~8). Знаком~<<?>> обозначены те виды 
трансформаций, которые по определению не существуют в~используемой 
парадигме информатики~\cite{30-zac}. Однако возможно, что в~других 
будущих подходах к~построению ее парадигмы эти виды трансформаций будут 
существовать.
  
\section{Заключение}

  На сегодняшний день процесс построения классификаций объектов 
предметной области информатики~\cite{22-zac} и~их  
трансформаций~\cite{23-zac} еще не завершен. Однако первые результаты их 
построения уже используются для создания концепции лексикографической 
информационной сис\-те\-мы, обеспечивающей интеграцию двуязычных 
словарей и~параллельных корпусов.
  
  \bigskip
  
  
  Автор признателен рецензентам за помощь в~улучшении статьи.
  
{\small\frenchspacing
 { %\baselineskip=10.6pt
 %\addcontentsline{toc}{section}{References}
 \begin{thebibliography}{99}
\bibitem{1-zac}
\Au{Aijmer K., Altenberg~B.} Advances in corpus-based contrastive linguistics. Studies in honour 
of Stig Johansson.~--- Amsterdam: John Benjamins, 2013. 295~p.  doi: 10.1075/scl.54.
\bibitem{2-zac}
\Au{Добровольский Д.\,О., Кретов~А.\, А., Шаров~С.\,А.} Корпус параллельных текстов~// 
Научная и~техническая информация. Сер.~2: Информационные процессы и~сис\-те\-мы, 2005. 
№\,6. С.~16--27.
\bibitem{3-zac}
\Au{Добровольский Д.\,О.} Корпус параллельных текстов и~сопоставительная 
лексикология~// Труды Института русского языка им.\ В.\,В.~Виноградова, 2015. №\,6. 
С.~413--449. EDN: VJQBHP.
\bibitem{4-zac}
\Au{Гончаров А.\,А., Зацман~И.\,М., Кружков~М.\,Г.} Эволюция классификаций 
в~надкорпусных базах данных~// Информатика и~её применения, 2020. Т.~14. Вып.~4. 
С.~108--116. doi: 10.14357/19922264200415.  
EDN: \mbox{GKWBZT}.
\bibitem{5-zac}
\Au{Гончаров А.\, А., Зацман И. \,М., Кружков~М.\, Г}. Представление новых 
лексикографических знаний в~динамических классификационных сис\-те\-мах~// 
Информатика и~её применения, 2021. Т.~15. Вып.~1. С.~86--93.  doi: 10.14357/19922264210112. EDN: OPEFXW.
\bibitem{6-zac}
\Au{Zatsman I.} Finding and filling lacunas in linguistic typologies~// 15th Forum (International) 
on Knowledge Asset Dynamics Proceedings.~--- Matera, Italy: Institute of Knowledge Asset 
Management, 2020. P.~780--793.
\bibitem{7-zac}
\Au{Zatsman I.} Three-dimensional encoding of emerging meanings in AI-systems~// 21st 
European Conference on Knowledge Management Proceedings.~--- Reading, U.K.: Academic 
Publishing International Ltd., 2020. P.~878--887.
\bibitem{8-zac}
\Au{Ackoff R.} From data to wisdom~// J.~Applied Systems Analysis, 1989. Vol.~16. No.\,1. P.~3--9.
\bibitem{9-zac}
\Au{Rosenbloom P.\,S.} On computing: The fourth great scientific domain.~--- Cambridge, MA, 
USA: MIT Press, 2013. 307~p.
\bibitem{10-zac}
\Au{Rowley J.} The wisdom hierarchy: Representations of the DIKW hierarchy~// J.~Inf. 
Sci., 2007. Vol.~33. Iss.~2. P.~163--180. doi: 10.1177/0165551506070706.
\bibitem{11-zac} 
\Au{Frick$\acute{\mbox{e}}$~M.\,H.} Data--Information--Knowledge--Wisdom (DIKW) pyramid, 
framework, continuum~// Encyclopedia of big data~/ Eds. L.~Schintler, C.~McNeely.~--- Cham: 
Springer, 2018. 4~p. doi: 10.1007/978-3-319-32001-4\_331-1.
\bibitem{12-zac}
\Au{Denning P., Rosenbloom~P.} Computing: The fourth great domain of science~// Commun. 
ACM, 2009. Vol.~52. Iss.~9. P.~27--29.
\bibitem{13-zac}
\Au{Denning P., Freeman~P.} Computing's paradigm~// Commun.  ACM, 2009. Vol.~52. 
Iss.~12. P.~28--30. doi: 10.1145/ 1610252.1610265.
\bibitem{17-zac} %14
\Au{Farradane J.} Knowledge, information, and information science~// J.~Inf. Sci., 
1980. Vol.~2. Iss.~2. P.~75--80. doi: 10.1177/01655515800020020.

\bibitem{15-zac}
\Au{Шрейдер Ю.\,А.} Информация и~знание~// Сис\-тем\-ная концепция информационных 
процессов.~--- М.: ВНИИСИ, 1988. С.~47--52.
\bibitem{16-zac}
\Au{Ingwersen P.} Information and information science~// Enclyclopaedie of library and 
information science~/ Eds. J.\,D.~McDonald, 
M.~Levine-Clark.~--- New York, NY, USA: Marcel Dekker Inc., 1992. Vol.~56. Sup.~19. 
P.~137--174.

\bibitem{14-zac} %17
Информатика как наука об информации: Информационный, документальный, 
технологический, экономический, социальный и~организационный аспекты~/ Под ред. 
Р.\,С.~Гиляревского.~--- М.: Фаир-Пресс, 2006. 592~с.

\bibitem{18-zac}
\Au{Hjorland B.} Library and information science: practice, theory, and philosophical basis~// 
Inform. Process. Manag., 2000. Vol.~36. Iss.~3. P.~501--531. doi:  
10.1016/S0306-\mbox{4573(99)00038-2}.
\bibitem{19-zac}
Deep shift~--- technology tipping points and societal impact.~--- Geneva: WE Forum, 2015. 44~p. 
{\sf http://www3.weforum.org/docs/WEF\_GAC15\_ Technological\_Tipping\_Points\_report\_2015.pdf}.
\bibitem{20-zac}
\Au{Berman F., Rutenbar~R., Hailpern~B., Christensen~H., Davidson~S., Estrin~D., 
Franklin~M., Martonosi~M., Raghavan~P., Stodden~V., Szalay~A.\,S.} Realizing the potential of 
data science~// Commun.  ACM, 2018. Vol.~61. Iss.~4. P.~67--72. doi: 10.1145/3188721.

\bibitem{21-zac}
\Au{Stodden V.} The data science life cycle: A~disciplined approach to advancing data science as 
a~science~// Commun.  ACM, 2020. Vol.~63. Iss.~7. P.~58--66. doi: 10.1145/ 3360646.


\bibitem{23-zac} %22
\Au{Зацман И.\,М.} Научная парадигма информатики: классификация трансформаций 
объектов предметной об\-ласти~// Системы и~средства информатики, 2023. Т.~33. №\,4. 
С.~126--138. doi: 10.14357/08696527230412. EDN: ZIKUWO.

\bibitem{22-zac} %23
\Au{Зацман И.\,М.} Научная парадигма информатики: классификация объектов предметной  
об\-ласти~// Информатика и~её применения, 2023. Т.~17. Вып.~4. С.~96--103. doi: 
10.14357/19922264230413. EDN: FIUQAT.

\bibitem{24-zac}
\Au{Зацман И.\,М.} О~научной парадигме информатики: верхний уровень классификации 
объектов ее предметной об\-ласти~// Информатика и~её применения, 2022. Т.~16. Вып.~4. 
С.~73--79. doi: 10.14357/ 19922264220411. EDN: XZNKVI.

\bibitem{25-zac}
\Au{Соломоник А.\,Б.} Философия знаковых систем и~язык.~--- М.: ЛКИ, 2011. 408~с.
\bibitem{26-zac}
\Au{Зацман И.\,М.} Трансформация иерархии Акоффа в~научной парадигме информатики~// 
Информатика и~её применения, 2023. Т.~17. Вып.~3. С.~107--113. doi: 
10.14357/19922264230315. EDN: UMVRRV.

\bibitem{27-zac}
\Au{Zatsman I.} Building digital spiral models of knowledge generation~// 19th Forum 
(International) on Knowledge Asset Dynamics Proceedings.~--- Matera, Italy: Arts for Business 
Institute, 2024. P.~2185--2196.
\bibitem{28-zac}
\Au{Zatsman I.} Digital spiral model of knowledge creation and encoding its dynamics~// 18th 
Forum (International) on Knowledge Asset Dynamics Proceedings.~--- Matera, Italy: Arts for 
Business Institute, 2023. P.~581--596.
\bibitem{29-zac}
\Au{Зацман И.\,М.} Интерфейсы третьего порядка в~информатике~// Информатика и~её 
применения, 2019. Т.~13. Вып.~3. С.~82--89. doi: 10.14357/19922264190312. EDN: 
EHRQLF.

\bibitem{30-zac}
\Au{Зацман И.\,М.} Научная парадигма информатики как третьей культуры~//  
На\-уч\-но-тех\-ни\-че\-ская информация. Сер.~1: Организация и~методика информационной 
работы, 2023. №\,11. С.~1--14.

\end{thebibliography}

 }
 }

\end{multicols}

\vspace*{-9pt}

\hfill{\small\textit{Поступила в~редакцию 14.04.24}}

\vspace*{4pt}

%\pagebreak

%\newpage

%\vspace*{-28pt}

\hrule

\vspace*{2pt}

\hrule



\def\tit{OBJECT TRANSFORMATIONS OF~THE~FIRST AND~SECOND ORDER
IN~A~LEXICOGRAPHIC INFORMATION SYSTEM\\[-5pt]}


\def\titkol{Object transformations of~the~first and~second order
in~a~lexicographic information system}


\def\aut{I.\,M.~Zatsman}

\def\autkol{I.\,M.~Zatsman}

\titel{\tit}{\aut}{\autkol}{\titkol}

\vspace*{-13pt}


\noindent
Federal Research Center ``Computer Science and Control'' of the Russian Academy of Sciences, 
44-2~Vavilov Str., Moscow 119133, Russian Federation


\def\leftfootline{\small{\textbf{\thepage}
\hfill INFORMATIKA I EE PRIMENENIYA~--- INFORMATICS AND
APPLICATIONS\ \ \ 2024\ \ \ volume~18\ \ \ issue\ 2}
}%
 \def\rightfootline{\small{INFORMATIKA I EE PRIMENENIYA~---
INFORMATICS AND APPLICATIONS\ \ \ 2024\ \ \ volume~18\ \ \ issue\ 2
\hfill \textbf{\thepage}}}

\vspace*{2pt}



\Abste{The theoretical foundations of the design of information technologies used for 
the integration of bilingual dictionaries and parallel corpora are considered. The 
description of the first outcomes of the creation of the third\linebreak\vspace*{-12pt}}

\Abstend{ level of object 
transformations classification in the subject domain of informatics, which is supposed 
to be used
in creating the lexicographic information system providing integration, is 
given. All the entities of informatics are divided into two global classes: objects and 
their transformations. For each such class, its own classification is constructed. 
Previously, the two upper levels of the object transformation classification in the subject 
domain have been described. The present paper discusses the third level of this classification. The 
basis for the construction of its highest level was the division of the subject domain of 
informatics into media (mental, sensory, digital, and a~number of other media), each 
of which by definition includes objects of the same nature. The Solomonick's 
typology of sign systems served as the basis for constructing the second level of the 
object transformation classification. The aim of the paper is to systematize object 
transformations of the first and second orders at the third level of this classification. 
The basis for systematization is the medium version of the Ackoff's hierarchy.}

\KWE{subject domain objects; object transformations; classification; data; 
information; knowledge; lexicographic information system}


\DOI{10.14357/19922264240211}{VZTGVV}

\vspace*{-12pt}

\Ack

\vspace*{-3pt}


\noindent
The reported study was funded by the Russian Science Foundation, project  
No.\,24-18-00155, {\sf 
https://rscf.ru/project/24-18-00155}. The research was carried out using the infrastructure of the Shared 
Research Facilities ``High Performance Computing and Big Data'' (CKP 
``Informatics'') of FRC CSC RAS (Moscow) .
   


  \begin{multicols}{2}

\renewcommand{\bibname}{\protect\rmfamily References}
%\renewcommand{\bibname}{\large\protect\rm References}

{\small\frenchspacing
 {%\baselineskip=10.8pt
 \addcontentsline{toc}{section}{References}
 \begin{thebibliography}{99} 
\bibitem{1-zac-1}
\Aue{Aijmer, K., and B.~Altenberg.} 2013. \textit{Advances in corpus-based 
contrastive linguistics. Studies in honour of Stig Johansson}. Amsterdam: John 
Benjamins. 295~p. doi: 10.1075/scl.54.
\bibitem{2-zac-1}
\Aue{Dobrovolskiy, D.\,O., A.\,A.~Kretov, and S.\,A.~Sharov.} 2005. Korpus 
parallel'nykh tekstov [Corpus of parallel texts]. \textit{Nauchnaya i~tekhnicheskaya 
informatsiya. Ser. 2. Informatsionnye protsessy i~sistemy} [Scientific and Technical 
Information. Ser.~2: Information Processes and Systems] 6:16--27.
\bibitem{3-zac-1}
\Aue{Dobrovolskiy, D.\,O.} 2015. Korpus parallel'nykh tekstov i~sopostavitel'naya 
leksikologiya [The corpus of parallel texts and contrastive lexicology]. \textit{Trudy 
Instituta russkogo yazyka im. V.\,V.~Vinogradova} [Proceedings of the 
V.\,V.~Vinogradov Russian Language Institute] 6:413--449. EDN: VJQBHP.
\bibitem{4-zac-1}
\Aue{Goncharov, A.\,A., I.\,M.~Zatsman, and M.\,G.~Kruzhkov.} 2020. Evolyutsiya 
klassifikatsiy v~nadkorpusnykh ba\-zakh dannykh [Evolution of classifications in 
supracorpora databases]. \textit{Informatika i~ee Primeneniya~--- Inform. \mbox{Appl.}}  
14(4):108--116. doi: 10.14357/19922264200415.  
EDN: GKWBZT.
\bibitem{5-zac-1}
\Aue{Goncharov, A.\,A., I.\,M.~Zatsman, and M.\,G.~Kruzhkov.} 2021. 
Predstavlenie novykh leksikograficheskikh znaniy v~dinamicheskikh 
klassifikatsionnykh sistemakh [Representation of new lexicographical knowledge in 
dynamic classification systems]. \textit{Informatika i~ee Primeneniya~--- Inform. 
Appl.} 15(1):86--93. doi: 10.14357/19922264210112. EDN: OPEFXW.
\bibitem{6-zac-1}
\Aue{Zatsman, I.} 2020. Finding and filling lacunas in linguistic typologies. 
\textit{15th Forum (International) on Knowledge Asset Dynamics Proceedings}. 
Matera, Italy: Institute of Knowledge Asset Management. 780--793.
\bibitem{7-zac-1}
\Aue{Zatsman, I.} 2020. Three-dimensional encoding of emerging meanings in  
AI-systems. \textit{21st European Conference on Knowledge Management 
Proceedings}. Reading, U.K.: Academic Publishing International Ltd. 878--887.
\bibitem{8-zac-1}
\Aue{Ackoff, R.} 1989. From data to wisdom. \textit{J.~Applied Systems Analysis} 
16(1):3--9.
\bibitem{9-zac-1}
\Aue{Rosenbloom, P.\,S.} 2013. \textit{On computing: The fourth great scientific 
domain}. Cambridge, MA: MIT Press. 307~p.
\bibitem{10-zac-1}
\Aue{Rowley, J.} 2007. The wisdom hierarchy: Representations of the DIKW 
hierarchy. \textit{J.~Inf. Sci.} 33(2):163--180. doi: 10.1177/0165551506070706.
\bibitem{11-zac-1}
\Aue{Frick$\acute{\mbox{e}}$, M.\,H.} 2018.  
Data-Information-Knowledge-Wisdom (DIKW) pyramid, framework, continuum. 
\textit{Encyclopedia of big data}. Eds. L.~Schintler and C.~McNeely. Cham: 
Springer. 4~p. doi: 10.1007/978-3-319-32001- 4\_331-1.
\bibitem{12-zac-1}
\Aue{Denning, P., and P.~Rosenbloom.} 2009. Computing: The fourth great domain 
of science. \textit{Commun. ACM} 52(9):27--29.
\bibitem{13-zac-1}
\Aue{Denning, P., and P.~Freeman.} 2009. Computing's paradigm. \textit{Commun. 
ACM} 52(12):28--30. doi: 10.1145/ 1610252.1610265.

\bibitem{17-zac-1} %14
\Aue{Farradane, J.} 1980. Knowledge, information, and information science. 
\textit{J.~Inf. Sci.} 2(2):75--80. doi: 10.1177/ 01655515800020020.

\bibitem{15-zac-1}
\Aue{Shreyder, Yu.\,A.} 1988. Informatsiya i~znanie [Information and knowledge]. 
\textit{Sistemnaya kontseptsiya in\-for\-ma\-tsi\-on\-nykh protsessov} [System concept of 
information processes]. Moscow: VNIISI. 47--52.
\bibitem{16-zac-1}
\Aue{Ingwersen, P.} 1995. Information and information science. 
\textit{Encyclopedia of library and information science}. Eds. J.\,D.~McDonald and 
M.~Levine-Clark. New York, NY: Marcel Dekker Inc. 56(19):137--174.

\bibitem{14-zac-1} %17
Gilyarevskiy, R.\,S., ed. 2006. \textit{Informatika kak nauka ob informatsii: 
informatsionnyy, dokumental'nyy, tekh\-no\-lo\-gi\-che\-skiy, ekonomicheskiy, sotsial'nyy 
i~organizatsionnyy aspekty} [Informatics as information science: Informational, 
documentary, technological, economic, social, and organizational dimensions]. 
Moscow: FAIR-PRESS. 592~p.

\bibitem{18-zac-1}
\Aue{Hjorland, B.} 2000. Library and information science: Practice, theory, and 
philosophical basis. \textit{Inform. Process. Manag.} 36(3):501--531. doi:  
10.1016/S0306-\mbox{4573(99)00038-2}.
\bibitem{19-zac-1}
Deep shift~--- technology tipping points and societal impact. 2015. \textit{World Economic 
Forum}. Geneva. 44~p. Available at: {\sf 
http://www3.weforum.org/docs/WEF\_ GAC15\_Technological\_Tipping\_Points\_report\_2015.pdf} (accessed May~20, 
2024).
\bibitem{20-zac-1}
\Aue{Berman, F., R.~Rutenbar, B.~Hailpern, H.~Christensen, S.~Davidson, 
D.~Estrin, M.~Franklin, M.~Martonosi, P.~Raghavan, V.~Stodden, and 
A.\,S.~Szalay.} 2018. Realizing the potential of data science. \textit{Commun. ACM} 
61(4):67--72. doi: 10.1145/3188721.
\bibitem{21-zac-1}
\Aue{Stodden, V.} 2020. The data science life cycle: A~disciplined approach to 
advancing data science as a~science. \textit{Commun. ACM} 
 63(7):58--66. doi: 10.1145/3360646.

\bibitem{23-zac-1} %22
\Aue{Zatsman, I.\,M.} 2023. Nauchnaya paradigma informatiki: klassifikatsiya 
transformatsiy ob''ektov predmetnoy oblasti [Scientific paradigm of informatics: 
Transformation classification of domain objects]. \textit{Sistemy i~Sredstva 
Informatiki~--- Systems and Means of Informatics} 33(4):126--138. doi: 
10.14357/08696527230412. EDN: ZIKUWO.

\bibitem{22-zac-1} %23
\Aue{Zatsman, I.\,M.} 2023. Nauchnaya paradigma informatiki: klassifikatsiya 
ob''ektov predmetnoy oblasti [Scientific paradigm of informatics: Classification of 
domain objects]. \textit{Informatika i~ee Primeneniya~--- Inform. Appl.} 
 17(4):96--103. doi: 10.14357/19922264230413. EDN: FIUQAT.
 
\bibitem{24-zac-1}
\Aue{   Zatsman, I.\,M.} 2022. O nauchnoy paradigme informatiki: verkhniy uroven' 
klassifikatsii ob''ektov ee predmetnoy oblasti [On the scientific paradigm of 
informatics: The classification high level of its objects]. \textit{Informatika i~ee 
Primeneniya~--- Inform. Appl.} 16(4):73--79. doi: 10.14357/19922264220411. EDN: 
XZNKVI.
\bibitem{25-zac-1}
\Aue{Solomonick, A.\,B.} 2011. \textit{Filosofiya znakovykh system i~yazyk} 
[Philosophy of sign systems and language]. Moscow: LKI. 408~p.
\bibitem{26-zac-1}
\Aue{Zatsman, I.\,M.} 2023. Transformatsiya ierarkhii Akoffa v~nauchnoy 
paradigme informatiki [Transformation of the Ackoff's hierarchy in the scientific 
paradigm of informatics]. \textit{Informatika i~ee Primeneniya~--- Inform. \mbox{Appl.}} 
17(3):107--113. doi: 10.14357/19922264230315. EDN: UMVRRV.
\bibitem{27-zac-1}
\Aue{Zatsman, I.} 2024. Building digital spiral models of knowledge 
generation. \textit{19th Forum (International) on Knowledge Asset Dynamics 
Proceedings}. Matera, Italy: Arts for Business Institute. 2185--2196.
\bibitem{28-zac-1}
\Aue{Zatsman, I.} 2023. Digital spiral model of knowledge creation and encoding its 
dynamics. \textit{18th Forum (International) on Knowledge Asset Dynamics 
Proceedings}. Matera, Italy: Arts for Business Institute. 581--596.
\bibitem{29-zac-1}
\Aue{Zatsman, I.\,M.} 2019. Interfeysy tret'ego poryadka v~informatike 
 [Third-order interfaces in informatics]. \textit{Informatika i~ee Primeneniya~--- 
Inform. Appl.} 13(3):82--89. doi: 10.14357/19922264190312. EDN: EHRQLF.
\bibitem{30-zac-1}
\Aue{Zatsman, I.} 2023. Scientific paradigm of informatics as a~third culture. 
\textit{Scientific Technical Information Processing} 50(4):246--258. doi: 
10.3103/S0147688223040111. EDN: CKHMYS.

\end{thebibliography}

 }
 }

\end{multicols}

\vspace*{-6pt}

\hfill{\small\textit{Received April 14, 2024}} 


\vspace*{-12pt}


\Contrl

\vspace*{-3pt}

\noindent
\textbf{Zatsman Igor M.} (b.\ 1952)~--- Doctor of Science in technology, head of 
department, Federal Research Center ``Computer Science and Control'' of the 
Russian Academy of Sciences, 44-2~Vavilov Str., Moscow 119333, Russian 
Federation; \mbox{izatsman@yandex.ru}





\label{end\stat}

\renewcommand{\bibname}{\protect\rm Литература}   %9
\renewcommand{\figurename}{\protect\bf Figure}
\renewcommand{\tablename}{\protect\bf Table}

\def\stat{kabanov}


\def\tit{ON UNIQUENESS OF CLEARING VECTORS REDUCING~THE~SYSTEMIC RISK}

\def\titkol{On uniqueness of clearing vectors reducing the systemic risk}

\def\autkol{Kh.\ El Bitar,  Yu.~Kabanov, and~R.~Mokbel}

\def\aut{Kh.\ El Bitar$^1$,  Yu.~Kabanov$^2$, and~R.~Mokbel$^3$}

\titel{\tit}{\aut}{\autkol}{\titkol}

%{\renewcommand{\thefootnote}{\fnsymbol{footnote}}
%\footnotetext[1] {The 
%research of Yuri Kabanov was done under partial financial support   of the grant 
%of  RSF No.\,14-49-00079.}}

\renewcommand{\thefootnote}{\arabic{footnote}}
\footnotetext[1]{Laboratoire de Math$\acute{\mbox{e}}$matiques, Universit$\acute{\mbox{e}}$ de 
Franche-Comt$\acute{\mbox{e}}$, 16~Route de Gray, 25030 \mbox{Besan{\!\ptb{\c{c}}}on}, CEDEX, France, 
\mbox{khalilbitar\_aw@hotmail.com}}
\footnotetext[2]{Laboratoire de 
Math$\acute{\mbox{e}}$matiques, Universit$\acute{\mbox{e}}$ de
 Franche-Comt$\acute{\mbox{e}}$, 16~Route de Gray, 25030 
\mbox{Besan{\!\ptb{\c{c}}}on}, CEDEX, France; 
Institute of Informatics Problems, Federal Research 
Center ``Computer Science and Control'' of the Russian Academy of Sciences, 
44-2~Vavilov Str., Moscow 119333, Russian Federation; 
National Research University 
``MPEI,'' 14~Krasnokazarmennaya Str., Moscow, 111250, Russian Federation, 
\mbox{Youri.Kabanov@univ-fcomte.fr}}
\footnotetext[3]{Laboratoire de 
Math$\acute{\mbox{e}}$matiques, Universit$\acute{\mbox{e}}$ de 
Franche-Comt$\acute{\mbox{e}}$, 
16~Route de Gray, 25030  \mbox{Besan{\!\ptb{\c{c}}}on}, CEDEX, France,
\mbox{ritamokbel@hotmail.com}}

\index{El Bitar Kh.}
\index{Kabanov Yu.}
\index{Mokbel R.}
\index{Эль Битар Х.}
\index{Кабанов Ю.}
\index{Мокбель Р.}


\vspace*{-12pt}

\def\leftfootline{\small{\textbf{\thepage}
\hfill INFORMATIKA I EE PRIMENENIYA~--- INFORMATICS AND APPLICATIONS\ \ \ 2017\ \ \ volume~11\ \ \ issue\ 1}
}%
 \def\rightfootline{\small{INFORMATIKA I EE PRIMENENIYA~--- INFORMATICS AND APPLICATIONS\ \ \ 2017\ \ \ volume~11\ \ \ issue\ 1
\hfill \textbf{\thepage}}}




\Abste{Clearing of financial system, i.\,e., of a~network of interconnecting banks, is 
a~procedure of simultaneous repaying debts to reduce their total volume. The 
vector whose components are  repayments of each bank
is called clearing vector.  In  simple models  considered  by Eisenberg and Noe 
(2001) and, independently,  by Suzuki (2002), it was shown that
the  clearing  to the minimal value of debts  accordingly to natural rules  can 
be formulated as fixpoint problems.
The existence
of their solutions, i.\,e., of clearing vectors,  is rather straightforward and can 
be obtained by a~direct reference to the Knaster--Tarski or Brouwer theorems.  
The uniqueness of clearing vectors is a~more delicate problem which was solved 
by Eisenberg and Noe  using a~graph structure of the financial network.  
The uniqueness  results have been proved in two generalizations of the  Eisenberg--Noe model:  
in the Elsinger model with seniority of liabilities and in the Amini--Filipovic--Minca 
type model with several
types of illiquid assets whose firing sale has a~market impact.}

\KWE{systemic risk;  financial networks; clearing; Knaster--Tarski 
theorem; Eisenberg--Noe model; debt seniority; price impact}

\DOI{10.14357/19922264170110} 

\vspace*{7pt}


\vskip 12pt plus 9pt minus 6pt

      \thispagestyle{myheadings}

      \begin{multicols}{2}

                  \label{st\stat}


\section{Introduction}

\noindent
To explain the clearing problem, let us start with the simplest example of 
a~financial
system with two agents each having in a~cash 10 dollars. The first agent gets 
from the second a~credit of~1M  dollars, the second gets from the first  a~credit 
of~1~M and 1~dollars. Apparently, as a~result, both agents have a~huge liabilities with 
respect to each other. Of course, the agents can be asked to reduce their 
liabilities by reimbursing credits partially (e.\,g., to the levels~0.5~M and 
0.5~M\;+\;1 in liabilities and~10~dollars both in cash) or completely, with zero 
liabilities and cash reserves~11 and~9~dollars, respectively. Intuitively, the 
situation where the liability is reduced (i.\,e., the system is cleared) seems to 
be less risky: if one of the agents became bankrupt and only the percentage of the 
huge debt value  can be reimbursed, the creditor's losses will be also huge. For 
complex financial systems involving large numbers of agents
with chains of borrowing,  the clearing problem, that is, the reduction of 
absolute values by reimbursement, looks much more complicated.
{ %\looseness=1

}

In the influential paper~\cite{Eisenberg-Noe} published in 2001, Eisenberg and 
Noe suggested a~clearing procedure in the model describing a~financial system 
composed by~$N$~banks (under ``banks''  can be understood  various financial 
institutions); a~more general model was introduced independently at the same 
time by Suzuki~\cite{Suzuki}.   The assets of the bank are cash and interbank 
exposures which are, in turn, liabilities for its debtors.  The clearing 
consists in simultaneous paying all debts. Each bank pays to its counterparties 
the debts \textit{pro rata} of their relative volume using its cash reserve and 
money collected from the credited banks. The rule is: either all debts are payed 
in full or the zero level of the equity is attained and the bank defaults. The 
totals reimbursed by banks form an $N$-dimensional clearing vector. A~remarkable 
feature is that this vector is a~fixed point of a~monotone mapping of a~complete 
lattice into itself and its existence follows immediately from the 
Knaster--Tarski  theorem, a~beautiful and fairy simple result which proof needs only 
a~few lines of arguments~\cite{Tarski}. The uniqueness of the clearing vector is a~more 
delicate 
result involving the graph structure of the system.

The ideas of the  Eisenberg--Noe paper happened to be very fruitful and their 
model
was generalized in many directions having not only financial importance but 
posing  interesting mathematical questions. One of them is the question on 
uniqueness of clearing vector or   equilibrium  on financial market.

The first theorem provides a~new sufficient condition for the Elsinger model of 
clearing with debts priority structure. This model is given by a~set of 
liability matrices corresponding to each seniority. The idea of the present approach is 
to use the largest clearing vector which always exists to construct a~new 
liability matrix generating a~graph structure with which one can work in 
a~similar way as in the Eisenberg--Noe model.
The second theorem deals with the uniqueness of  equilibrium in a~clearing  
model with several illiquid assets and a~market impact.  In the presence of 
several illiquid assets,  the banks are faced the choice of  asset selling 
strategies. The proportional scheme of selling similar to that in the 
paper by Cont--Wagalath~\cite{Cont-Wag} has been used 
leaving game-theoretical versions for 
future studies.  In the case of one illiquid asset, the
obtained result is close to that  
of the study by Amini--Filipovic--Minca~\cite{AFM}, but the present definition of the 
equilibrium is different (but equivalent).

The structure of the note is as follows. In the introductory section~2, 
 the general principle and results are discussed briefly in the framework of the 
Eisenberg--Noe model. To facilitate the comparison with further development, 
also, short proofs are provided.
In section~3, a~uniqueness of the clearing vector for the Elsinger model 
where senior  liabilities should be reimbursed before the juniors ones. Section~4 
contains
the sufficient condition  for the uniqueness of the equilibrium in the model
where clearing requires selling of the illiquid assets with price impact.  
Economically speaking, it is  oriented to the recovering of the market  after 
fire sales.  For the reader convenience,  in Appendix, 
a~short information about the Knaster--Tarski theorem adapted to 
the present authors' needs is provided.



\noindent
\textbf{Notations.}\ The partial ordering in~$\mathbb{R}^n$ and its 
subsets  induced by the cone~$\mathbb{R}^n_+$ is denoted by $\ge$. In other words, the inequality $y\ge x$ 
is understood componentwise. Also, the symbols $x\wedge y$ and $x\vee y$ mean, 
respectively, the componentwise minimum and maximum, $x^+:=x\vee 0$ and\linebreak 
$x^-:=(- x)^+$.
The notation $[x,z]$ is used for the order interval, i.\,e.,
$[x,z]=\{y\in \mathbb{R}^n:\ x\le y\le z \}$.
If $A\subseteq [x,z]$, then $\inf A$ is the unique element $\underline y\in 
[x,z]$ such
that $\underline y\le y$ for all $y\in A$ and for any $\tilde y$  such that 
$\tilde y\le y$ for all $y\in A$, one has $\tilde y\le \underline y$, that 
is, the component $\underline y^i=\inf \{y^i:\ y\in A\}$ for  $i=1,\dots,n$.

The matrix notations are used where the vectors are columns, $'$ is the symbol of 
transpose, and\linebreak  ${\bf 1}':=(1,\dots,1)$ (the dimension of the vector is supposed 
to be clear from the context).

\vspace*{-9pt}

\section{The Eisenberg--Noe Model}

\noindent
In~\cite{Eisenberg-Noe}, Eisenberg and Noe investigated the model 
describing a~financial system composed of $N$ banks (under ``banks"  can be 
understood  various financial institutions). In the aggregate oversimplified  
form, the balance sheet of the bank $i$ can be split into two parts: assets and 
liabilities. The assets are of two types:  interbank assets (exposures)~$\tilde X^i$ 
and cash~$e^i$.  The liabilities are: interbank debts (liabilities)~$\tilde L^i$ 
and the equity~$C ^i$ (or proper capital reserve) equalizing the two sides 
of the balance sheet:
\vspace*{2pt}

\noindent
$$
e^i+\tilde X^i= \tilde L^i + C^i\,.
$$

\vspace*{-2pt}

\noindent
All these values are assumed to be greater or equal to zero. The condition that 
$C^i\ge 0$ means that the bank is solvent.

More detailed balance sheet provides the information on the values  of 
liabilities of the bank  $i$ to the bank $j$, namely,  vectors 
$(L^{i1},\ldots,L^{iN})'$ of liabilities and  $(X^{i1},\ldots,X^{iN})$ of exposures.
 With this, one 
has  $\tilde X^i=X^{i1}+\cdots+X^{iN}$ and $\tilde L^i =L^{i1}+\cdots+L^{iN}$.

The matrix $L=(L^{ij})$ with positive entries and zero diagonal defines the
total interbank exposures. Since the value of the exposure of~$i$ to~$j$ is the 
value of the liability of~$j$ to~$i$, one has that $L'=X$.  So, 
the matrix $L$ and the vector~$e$ give a~description of a~financial system in 
this model.

Put

\vspace*{-3pt}

\noindent
$$
\Pi^{ij}:=
\begin{cases}
\fr {L^{ij}}{\tilde L^i}=\displaystyle \fr {L^{ij}}{\sum\nolimits_j L^{ij}} 
&\ \mbox{if } \tilde L^i\neq 0\,; \\
\delta^{ij} &\  \mbox{otherwise}
\end{cases}
$$
where the Kronecker symbol $\delta^{ij}=0$ for $i\neq j$ and $\delta^{ii}=1$.
Then,~$\Pi^{ij}$  describes the proportion of the value debtor $i$ due to the 
creditor~$j$ of the total interbank debt of~$i$; $\Pi=(\Pi^{ij})$  is called 
relative liabilities matrix. Note that in this definition, to get a~stochastic 
matrix $\Pi$, we deviate from~\cite{Eisenberg-Noe} where $\Pi^{ii}=0$ when 
$L^i= 0$.

%As an example consider the simplest system with two banks where 
%$L^{12}=L^{21}+\varepsilon$ where $\e<0$ can be thought small with respect to~$L^{21}$.  
%After paying debts in the cleared system the matrix of liabilities will have the 
%entries $L^{12}_{c}=\e$, $L^{21}_c=0$. That is the values of debts are reduced 
%and so are eventual values of losses in the case of defaults of a~partner.

In general, financial system   may have a~complicated structure with cyclical 
interdependences and  banks may have large exposures within cycles. To reduce 
them, one can impose a~clearing mechanism satisfying several natural 
requirements: limited liability and proportionality. Formally,  this  leads to 
the concept of a~\textit{clearing payment vector} $p^*\in \prod_i[0,\tilde L^i]$ 
satisfying the following properties:
\begin{itemize}
\item[$a.$] \textit{Limiting liability}. For every $i$,
$$
p_i^*\le e^i+\sum\limits_j\Pi^{ji}p_j^*\,.
$$

\item[$b.$] \textit{Absolute priority.} For every $i$, either $p^*_i=\tilde L^i$, or
$$
p_i^*= e^i+ \sum\limits_j\Pi^{ji}p_j^*.
$$
\end{itemize}
One may think that the  central clearing authority forces  each bank to make 
a~``fair'' payment of debts in such\linebreak\vspace*{-12pt}

\pagebreak

\noindent
 a~way that, having  the total payment~$p_i^*$, 
the bank~$i$ remains solvent and  pays to~$j$ the fraction $p_i^*\Pi^{ij}$ in 
such a~way that either its total debts are paid,  or all the resources are 
exhausted.

Alternatively, the conditions~$a$ and~$b$ can be written in the following way:
\begin{equation}
\label{p^*} p^*=\min \left\{ e+\Pi' p^*, \tilde L\right\}
\end{equation}
where the minimum is understood in the componentwise sense, i.\,e., accordingly to 
the partial ordering defined by the cone~${\mathbb{R}}^N_+$.

The main result of Eisenberg and Noe asserts that the set of clearing vectors is 
nonempty. Moreover, there are the minimal and the maximal clearing vectors,  
denoted here~$\underline p$ and~$\bar p$, respectively.  This assertion follows
immediately from the Knaster--Tarski fixed point theorem: the monotone mapping 
$f:p\mapsto (e+\Pi'p)\wedge \tilde L$ of a~complete lattice $[0,\tilde L]$ into 
itself has the largest and the smallest fixed points (for 
details, see section~5). The set $[0,\tilde L]$ is convex and compact and~$f$ is a~continuous 
mapping. So, the existence of its fixed point follows also from the classical  
Brouwer theorem.

Using the obvious identity $(x-y)^+=x -x\wedge y$, one can rewrite 
Eq.~(\ref{p^*}) in the following equivalent form:
\begin{equation}
\label{alt1}
\left(e+\Pi'p^*-\tilde L\right)^+=e+\Pi'p^*-p^*
\end{equation}
where the left-hand side is the equity vector of the system after clearing.

%After clearing by an arbitrary clearing (outflow) vector $p^*$ the equity 
%vector of the system  is
%$$(e+\Pi'p^*-\tilde L)^+=e+\Pi'p^*-p^*; $$
%this equality is nothing but the equation (\ref{p^*}) written in an equivalent 
%form.

An important but simple observation: {\it the equity (after clearing) does not 
depend on the clearing vector}.
Indeed,~$\Pi$~being  a~stochastic matrix, ${\bf 1}'\Pi'={\bf 1}'$.  Therefore, 
multiplying  the above representation~(\ref{alt1}) from the left by~${\bf 1}'$, 
one gets that  the sum of equities
$$
{\bf 1}'\left(e+\Pi'p^*-\tilde L\right)^+={\bf 1}'e
$$
is equal to the sum of the initial cash reserves, that is, invariant with respect 
to the choice of the clearing vector.
On the other hand, by monotonicity, one has that
$$
\left(e+\Pi'p^*-\tilde L\right)^+\le \left(e+\Pi'\bar p-\tilde L\right)^+.
$$
If the both sides here are not equal, then
$$
{\bf 1}'\left(e+\Pi'p^* - \tilde L\right)^+< {\bf 1}'\left(e+\Pi'\bar p-\tilde L\right)^+$$
in contradiction with the invariance of the  total of equities.

\smallskip

\noindent
\textbf{Sufficient condition for the uniqueness of the clearing vector.}
As in~\cite{Eisenberg-Noe}, let us assume for simplicity that $\tilde L^i>0$ 
for all~$i$.

For a~stochastic matrix~$\Pi$,  we say that
$I\subseteq  \{1,\ldots,N\}$ is
a~($\Pi$-)\textit{surplus set} if $\Pi^{ij}=0$ for all $i\in I$, $j\in I^c$, 
and~$\sum_{j\in I}e^j>0$.

\columnbreak

Recall that~$j$ is the creditor of $i$ if $\Pi^{ij}>0$ (i.\,e., $\Pi^{ij}>0$); in 
this case, let us use, as in the  theory of Markov chains or in the graph 
theory,  the notation $i\to j$.

Let us denote by $o(i)$ {\it the orbit of $i$} that is the set of all~$j$ for which 
there is a~directed path 
$$
i\to i_1\to i_2\to\cdots\to j\,,$$ 
i.\,e.,  $o(i)$ is the set 
of all direct or indirect creditors of~$i$.

Note that the orbit $o(i)$ with $\sum_{j\in I}e^j>0$ is a~surplus set. Indeed,
if\ $\Pi^{jj'}>0$ for some $j\in o(i)$, $j'\notin o(i)$, i.\,e.,  $j\to j'$, then 
there is
a~path 
$$i\to i_1\to i_2\to\cdots\to j \to  j'\,.
$$


\noindent
\textbf{Lemma~1.}\
%\label{equity>0}
 \textit{Suppose that the market is cleared by a~vector $p^*\in [0,\tilde L]$. Let~$I$ 
be a~surplus set.  Then, at least one node of~$I$ has a~strictly positive equity 
value}.

\textit{In particular,
any orbit~$o(i)$ with $\sum_{j\in o(i)}e^j>0$ has an element with strictly  
positive equity value}.

\smallskip

\noindent
P\,r\,o\,o\,f\,.\ \  Multiplying the identity~(\ref{alt1}) by~${\bf 1}'_I$ and noticing 
that
$({\bf 1}'_I\Pi')^i=1$ for $i\in I$,
one obtains that
$$
{\bf 1}'_I \left(e+\Pi'p^*-\tilde L\right)^+\ge {\bf 1}'_I e>0
$$
implying the claim.~$\square$

\smallskip

A financial system is called \textit{regular} if for  every~$i$, the orbit~$o(i)$ is 
a~surplus set.

\smallskip

\noindent
\textbf{Theorem~1.}\
%\label{uni1}
\textit{Suppose that the financial system is regular.
Then}, $\underline p=\bar p$.

\smallskip

\noindent
P\,r\,o\,o\,f\,.\ \  Suppose that~$\underline p$ and~$\bar p$ are not equal, i.\,e., 
$\underline p\le \bar p$ but for some~$i$, one 
has the strict inequality  $\underline p^i<\bar p^i$.
Denote by~$C$ the vector of equities (it is common for all clearing vectors).
By assumption, the orbit~$o(i)$ is a~surplus set and by Lemma~1, it 
contains an element~$m$ with the equity value $C^m>0$. By definition of the 
orbit, there is a~path $i\to i_1\to \cdots \to m$ and one may assume without loss of 
generality that in this path,~$m$ is  the first node with strictly positive 
equity value.

First, let us prove that  one may consider only the case where the path
consists of one step,  i.\,e., $i\to m$.  To this end, let us check that
$\underline p^{i_1}<\bar p^{i_1}$ if $i_1\neq m$. In other words, the property 
that $\underline p^i\neq \bar p^i$ propagates along the path.

Suppose that $\bar p^{i_1}< \tilde L^{i_1}$. Then, also, $\underline p^{i_1}< 
\tilde L^{i_1}$.  In such a~case,

\vspace*{3pt}

\noindent
$$
 \underline p^{i_1}=e^{i_1}+ \sum\limits_j\Pi^{ji_1}\underline p^j\,, \enskip \bar 
p^{i_1}=e^{i_1}+\sum\limits_j\Pi^{ji_1}\bar p^j
$$
and one has  that

\vspace*{3pt}

\noindent
$$
\bar p^{i_1}-\underline p^{i_1}=\sum\limits_j\Pi^{ji_1}\left(\bar p^j-\underline p^j\right)>0
$$

\vspace*{-6pt}

\noindent
because   $\Pi^{ii_1}>0$, that is, $\underline p^{i_1}<  \bar p^{i_1}$. This 
inequality also holds trivially, if
$\bar p^{i_1}= \tilde L^{i_1}$ but $\underline p^{i_1}< \tilde L^{i_1}$.
 The remaining\linebreak\vspace*{-12pt}
 
 \pagebreak
 
 \noindent
  case where
$\underline p^{i_1}=\bar p^{i_1}=\tilde L^{i_1}$ is excluded as it is supposed that 
$C^{i_1}=0$.  Indeed, according to~(\ref{alt1}),  this leads to the equalities:
$$
e^{i_1}+ \sum\limits_j\Pi^{ji_1}\bar p^j - \tilde L^{i_1}=0\,;\enskip
e^{i_1}+  \sum\limits_j\Pi^{ji_1}\underline p^j - \tilde L^{i_1}=0\,,
$$
implying the identity
$$
\sum\limits_j\Pi^{ji_1}\left(\bar p^j-\underline p^j\right)=0
$$
which cannot be true since in the above sum, the term corresponding to $j=i$ is 
strictly positive.

So, it is sufficient to consider only one-step case. Since $C^m>0$, one has the 
representations:
\begin{align*}
C^m&=e^{m}+ \sum\limits_j\Pi^{jm}\underline p^j - \tilde L^{m}\,; \\
C^m&=e^{m}+ \sum\limits_j\Pi^{jm}\bar p^j- \tilde L^{m}\,.
\end{align*}
As above, one again obtains the impossible equality:
$$
\sum\limits_j\Pi^{jm}\left(\bar p^j-\underline p^j\right)=0\,.
$$
Therefore, the  assumption $\underline p^i<\bar p^i$ leads to a~contradiction. 
The
uniqueness of clearing vector is proven.~$\square$


\smallskip

\noindent
\textbf{Remark~1.}\
The above  theorem reveals that the problem to find a~clearing 
vector is ill-posed. Indeed, adding an infinitesimally small amount $\varepsilon>0$ 
(say,  one cent) to the initial endowments leads to a~unique clearing vector. Similar 
effect will have small increase in liabilities. One can think that the ``true'' 
liability matrix has all elements strictly positive and that in the model matrix, zero 
elements appeared because liabilities are neglected.
These phenomena are related to the ill-posedness of the spectral problem for 
stochastic matrices. Another question is which clearing vector is natural.


\smallskip


The above proof  is rather straightforward and uses graph-theoretical language.  
One can get another one  appealing to the contraction property of the mapping 
$f:p\mapsto (e+\Pi'p)\wedge \tilde L$ defined on the set $[0,\tilde L]$ equipped 
with $l_1$-distance $|p-\tilde p|_1$.

\smallskip

\noindent
\textbf{Proposition.}\
For every $p,\tilde p\in [0,\tilde L]$
\begin{equation*}
%\label{non-exp}
\left\vert f(p)-f(\tilde p)\right\vert_1\le \left\vert\Pi' (p-\tilde p)
\right\vert_1\le \left\vert p-\tilde p\right\vert_1\,.
\end{equation*}
Moreover, the first relation above is the equality if and only if the
union of subsets $A:=\{i:\ (\Pi'p)^i=(\Pi'\tilde p)^i\}$ and $B:=\{i:\ 
(\Pi'p)^i,(\Pi'\tilde p)^i\le \tilde L^i-e^i\}$ is the set of indices 
$\{1,\dots, N\}$.

\smallskip
%Moreover, if for each $i$ the sum $\sum_{j\in o(i)} e^i>0$, then the mapping 
%$f$ is a~contraction %on the set ${\rm Fix}_f$, i.e. the above inequality is 
%strict when the fixed points $p\neq \tilde p$.

\noindent
P\,r\,o\,o\,f\,.\ \ Using the elementary inequality $|a\wedge c-b\wedge c|$\linebreak $\le |a-b|$ 
which holds as  the
equality if and only if when\linebreak $a=b$ or $a,b\le c$, one obtains that
$|f(p)-f(\tilde p)|_1$\linebreak $\le |\Pi'p-\Pi'\tilde p|_1$
where the equality holds if and only if for every~$i$, one has 
$(\Pi'p)^i=(\Pi'\tilde p)^i$ or
$(\Pi'p)^i,(\Pi'\tilde p)^i$\linebreak $\le \tilde L^i-e^i$. Since $|\Pi'y|_1\le 
|\Pi'|_1|y|_1$ and $|\Pi'|_1=1$, one has the claim.~$\square$

\smallskip

Let us consider  the case where the matrix~$\Pi$ is irreducible. Suppose that 
${\bf 1}'e>0$ and~$p$ and~$\tilde p$ are two different fixed points of the 
mapping~$f$. According to above proposition,
$$
\sum\limits_{j\in B}\Pi^{ji}\left(p^j-\tilde p^j\right)=p^ i-\tilde p^i\,, \enskip i\in B\,.
$$
This means that  the nonzero vector with the coordinates $p^ i-\tilde p^i$, 
$i\in B$, is a~left eigenvector of the matrix
$(\Pi^{ij})_{i,j\in B}$ corresponding to unit eigenvalue. This is possible only 
if the latter matrix coincides with~$\Pi$. Thus, $p=f(p)=e+\Pi'p$. Since  
${\bf 1}'\Pi'p={\bf 1}'p$, one gets that ${\bf 1}'e=0$
which is a~contradiction.  Using the decomposition of the matrix~$\Pi$ on the 
irreducible component, one gets that  the clearing vector  is unique if for any 
irreducible component, there is a~node with strictly positive initial endowment.



\section{The Elsinger Model}

\noindent
In the present paper,  a~simplified version of the Elsinger model
introduced in~\cite{Elsinger2011}, where the interbank debts may be 
senior and junior, is considered. In this model, the system of~$N$ banks is described by the 
vector
of cash reserves and by~$M$~matrices $L_1=(L^{ij}_1), \ldots, L_M=(L^{ij}_M)$ 
representing the hierarchy of liabilities with decreasing seniority,  that is, 
the element~$L^{ij}_1$ represents the debt of the bank~$i$ to the bank~$j$ of the 
highest seniority, etc.,  $\sum_jL^{ij}_S$ is the total of  debts of the bank~$i$ 
of the seniority~$S$.

The relative liabilities are defined by  the matrix~$\Pi_S$ with
$$
\Pi_S^{ij}=\fr {L_S^{ij}}{\tilde L_S^i}=\fr {L_S^{ij}}{\sum\nolimits_j L_S^{ij}}\,.
$$
The clearing procedure requires the complete reimbursement of the debts starting 
from the highest priority and for each seniority level, the distribution is 
proportional
to the volume of debts of this seniority. For the bank~$i$, let us denote  by $p^i_S$ 
the value distributed to cover the debts of the seniority~$S$. So, the clearing 
can be described by the set of vectors~$p_S$, $S=1,\ldots, M$, which can be 
considered as a~``long'' vector from~$(\mathbb{R}^N)^M$  satisfying the system of 
equations:
\begin{equation*}
p_{1}^{i}=\min \left\{e^i+\sum\limits_S \sum\limits_j\Pi_S^{ji}p_{S}^{j}, \tilde L_1^i 
\right\}\,;
\end{equation*}
\begin{align*}
p_{S}^{i}&=\min\left\{\left(e^i+\sum\limits_S \sum\limits_j\Pi_S^{ji}p_{S}^{j}-
\sum\limits_{r<S}\tilde 
L_r^i\right)^+, \tilde L_S^i \right\}\,,  \\
&\hspace*{57mm}1<S\le M\,.
\end{align*}
In a~vector form, these equations can be written as follows:

\vspace*{-4pt}

\noindent
\begin{multline}
\label{SM}
p_{S}^{}=\left(e+\sum\limits_S \hspace*{-1.2pt}
\Pi_S'p_{S}-\sum\limits_{r<S}\hspace*{-1.2pt}\tilde L_r\right)^+\wedge  
\tilde  L_S\,,  \\ S=1,\ldots,M\,.
\end{multline}
It is clear that for the partial ordering in~$(\mathbb{R}^N)^M$ induced by the 
cone~$(\mathbb{R}^N_+)^M$, the function

\vspace*{-4pt}

\noindent
\begin{multline*}
\left(p_1,\ldots,p_M\right)\mapsto \left(
\left(e+\sum\limits_S \Pi_S'p_{S}^* \right)^+\wedge \tilde L_1 
,\ldots\right.\\
\left.\ldots,\left(e+\sum\limits_S \Pi_S'p_{S}^*-\sum\limits_{r<M}\tilde L_r\right)^+ 
\wedge L_M 
\right)
\end{multline*}
is a~monotone mapping of the order interval 
$$
[0,\tilde L_1] \times\cdots\times 
[0,\tilde L_M]\subset (\mathbb{R}^N)^M
$$ 
into itself.
 Thus, according to the Knaster--Tarski theorem, the set of fixed points of this 
mapping, i.\,e., the solutions of Eq.~(\ref{SM}), is nonempty and has the 
maximal and the minimal elements.

In the case of liabilities of different seniority after clearing by the vector 
$p\in (\mathbb{R}^N)^M$,  the equity vector $C\in \mathbb{R}^N$ has the form:
$$
C=\left(e+\sum\limits_S \Pi_S'p_{S}-\sum\limits_S \tilde L_S\right)^+\,.
$$

%\smallskip

\noindent
\textbf{Lemma~2.}\
\textit{The equity vector does not depend on the clearing vector}.

\vspace*{2pt}

\noindent
P\,r\,o\,o\,f\,.\ \  Note that
$$
\left(e+\sum\limits_S\Pi'_Sp_S\right)\wedge \sum\limits_S \tilde L^i_S=\sum\limits_S p_S\,.
$$
Therefore,
$$
\left(e+\sum\limits_S \Pi_S'p_{S}-\sum\limits_S \tilde L_S\right)^+=
e+\sum\limits_S \Pi_S'p_{S}-\sum\limits_S  p_{S}\,.
$$
With this identity, the reasoning is analogous to that with a~single seniority 
class.~$\square$

\vspace*{2pt}

The aim of this section is to provide a~sufficient condition for the uniqueness 
of clearing vector using a~specific graph structure induced by the matrices~$\Pi_S$.

For a~given clearing vector~$p$, let us define the \textit{default index}~$d^i$ of the 
node~$i$ as the smallest~$r$  such that
$$
\bar p_r^i=e^i+ \sum\limits_S \sum\limits_j\Pi_S^{ji}\bar p_{S}^j-\sum\limits_{r'< r}\tilde 
L_{r'}^{i}\,.
$$
In another words,~$d^i$ is the lowest seniority for which the bank equity after 
clearing is equal to zero. Define the matrix $\Delta=\Delta(p)$ by putting 
$$
\Delta^{ij}=
\begin{cases}
1 &\ \mbox{if\ \ } \Pi_{d(i)}^{ij}>0\,;\\
0 &\ \mbox{otherwise}.
\end{cases}
$$

%\columnbreak

\noindent
Let us use 
the notation $i\leadsto j$ if $\Delta^{ij}=1$ and  denote by $O(i)$ \textit{the 
$\Delta $-orbit of $i$} that is the set of all~$j$ for which there is 
a~directed path $i\leadsto i_1\leadsto i_2\leadsto\cdots\leadsto j$.

\vspace*{2pt}

\noindent
\textbf{Theorem~2.}\
\textit{Suppose that for the clearing vector $\bar p$, any $\Delta $-orbit is a~surplus 
set.
Then, the clearing vector is unique}.

\vspace*{2pt}

\noindent
P\,r\,o\,o\,f\,.\ \  By definition, the default index
$$
d^i:=\min\left\{r:\ \bar p_r^i=e^i+ \sum\limits_S \sum\limits_j\Pi_S^{ji}
\bar p_{S}^j-\sum\limits_{r'<  r}\tilde L_{r'}^{i}\right \}\,.
$$
It follows that $\bar p_r^i=0$; hence,  $\underline p_r^i=0$ for every $r>d^i$.
Suppose that
$\underline p_{d^i}^i<\bar p_{d^i}^i$ and consider a~path 
$$
i\leadsto  i_1\leadsto i_2\leadsto\cdots \leadsto m
$$ 
ending up at the node with strictly  positive equity value.

First, let us show that at least for one seniority $\underline p^{i_1}_S<\bar 
p^{i_1}_S$.

Let $r':=d^{i_1}$.  By definition, one has: 
$$
\bar p^{i_1}_r=\begin{cases}
\tilde L^{i_1}_r\,, & r\le r'\,;\\
0\,,  & r>r'\,.
\end{cases}
$$
 The claim 
holds, if  $\underline p^{i_1}_r<\tilde L^{i_1}_r$
for some $r<r'$. Thus, it remains to consider only the case where $\underline 
p^{i_1}_r=\bar p^{i_1}_r = \tilde L^{i_1}_r$
for all $r<r'$ and prove that  $\underline p^{i_1}_{r'}<\bar p^{i_1}_{r'}$.
One has the alternative: either $\underline p^{i_1}_{r'}<\bar p^{i_1}_{r'}\le  
\tilde L^{i_1}_r$ (what is needed), or
$\underline p^{i_1}_{r'}=\bar p^{i_1}_{r'}\le  \tilde L^{i_1}_r$. The second 
case is impossible, since the equalities

\noindent
\begin{align*}
\bar p^{i_1}_{r'}&=e^{i_1}+ \sum\limits_S \sum\limits_j\Pi_S^{ji_1}\bar p_{S}^j-
\sum\limits_{r<  r'}\tilde L_{r}^{i_1}\,;\\
\underline p^{i_1}_{r'}&=e^{i_1}+ \sum\limits_S 
\sum\limits_j\Pi_S^{ji_1}\underline p_{S}^j-
\sum\limits_{r< r'}\tilde L_{r}^{i_1}
\end{align*}
imply that

\noindent
\begin{multline*}
\bar p^{i_1}_{r'}-\underline p^{i_1}_{r'}=\sum\limits_S \sum\limits_j\Pi_S^{ji_1}
\left(\bar  p_{S}^j-\underline  p_{S}^j\right)\\
{}\ge \Pi_{d^i}^{ii_1}
\left(\bar p_{d^i}^i-\underline   p_{d^i}^i\right)>0\,.
\end{multline*}
This is a~contradiction.

\pagebreak

The above argument reduces the problem to the case $i\leadsto m$ and the node~$m$ 
has a~strictly positive equity.  The equity~$C^m$ does not depend on the 
clearing vector.  Therefore,

\noindent
\begin{align*}
C^m&=e^{m}+ \sum\limits_S \sum\limits_j\Pi_S^{jm}\bar p_{S}^j-
\sum\limits_{S}\tilde L_{S}^{m}\,;\\
C^m&=e^{m}+ \sum\limits_S \sum\limits_j\Pi_S^{jm}\underline p_{S}^j-
\sum\limits_{S}\tilde L_{S}^{m}\,.
\end{align*}


\noindent
It follows that
$$
0=\sum\limits_S \sum\limits_j\Pi_S^{jm}\left(\bar p_{S}^j-\underline p_{S}^j\right)\ge 
\Pi_{d^i}^{im}\left(\bar p_{d^i}^i-\underline  p_{d^i}^i\right)>0\,.
$$
This contradiction shows that $\underline p=\bar p$.

\subsection{Example~1}

\noindent
Let us consider the system consisting of~3~nodes with the initial cash 
endowments
given by the vector $e=(0.1,0,0)$ and the liability and the "distribution"  
matrices corresponding to senior and junior debts:
\begin{alignat*}{2}
L_S&=
\begin{pmatrix}
0 & 1 & 0\\
1 & 0 & 1\\
0 & 2 & 0
\end{pmatrix}\,; &\enskip
L_J&=\begin{pmatrix}
0 & 0 & 0\\
0& 0 & 2\\
0 & 0 & 0
\end{pmatrix}\,;
\\[9pt]
\Pi_S&=
\begin{pmatrix}
0 & 1 & 0\\
0.5 & 0 & 0.5\\
0 & 1 & 0
\end{pmatrix}\,; &\enskip
\Pi_J&=\begin{pmatrix}
0 & 0 & 0\\
0& 0 & 1\\
0 & 0 & 0
\end{pmatrix}.
\end{alignat*}
For this model, the vectors of total liabilities corresponding to the senior and 
junior debts are, respectively, $\tilde L_S=(1,2,2)$ and   $\tilde L_J=(0,2,0)$.

The equations for clearing vectors are:
\begin{align*}
p_S^1 & =  \left(0.1+0.5\, p_S^2 \right)\wedge 1\,;\\
p_S^2 & =  \left(p_S^1+p_S^3 \right)\wedge 2\,;\\
p_S^3 & =  \left(0.5\, p_S^2+p_J^2\right)\wedge 2\,;\\
p_J^1 & = 0\,;\\
p_J^2 & = \left(p_S^1+p_S^3-2\right)^+\wedge 2\,;\\
p_J^3 & = 0.
\end{align*}
It is not difficult to check that there are infinite set of clearing vectors.
Namely, one has that $p_S=(1,2,1+t)$ and $p_J=(0,t,0)$ where $t\in [0,1]$.
The minimal clearing vector corresponds to $t=0$ and the maximal corresponds to 
$t=1$.

\subsection{Example~2}

\noindent
The vector of cash endowments and the matrix of the senior debts  is the same as 
in Example~1. The junior debts matrix $L_J$ and the corresponding 
distribution matrix~$\Pi_J$ are now:
$$
L_J=\begin{pmatrix}
0 & 0 & 0\\
0.4& 0 & 1.6\\
0 & 0 & 0
\end{pmatrix}\,;
 \enskip
\Pi_J=\begin{pmatrix}
0 & 0 & 0\\
0.2& 0 & 0.8\\
0 & 0 & 0
\end{pmatrix}\,.
$$
We are looking for positive solutions of the following  equations:
\begin{align*}
p_S^1 & =  \left(0.1+0.5\, p_S^2 + 0.2\, p_J^2\right)\wedge 1\,;\\
p_S^2 & =  \left(p_S^1+p_S^3 \right)\wedge 2\,;\\
p_S^3 & =  \left(0.5\, p_S^2+0.8\, p_J^2\right)\wedge 2\,;\\
p_J^1 & = 0\,;\\
p_J^2 & = \left(p_S^1+p_S^3-2\right)^+\wedge 2\,;\\
p_J^3 & = 0\,.
\end{align*}
Note that $p_S^1\le 1$ and $p_S^2\le 2$; hence, $p_J^2\le 1$ and the 3rd equation 
is linear:
\begin{equation}
\label{pS3}
p_S^3  =  0.5\, p_S^2+0.8\, p_J^2.
\end{equation}
Substituting into the 2nd equation this expression for~$p_S^3$ and the 
expression for~$p_S^1$ from the 1st equation, one gets that
\begin{equation*}
p_S^2 \!=\!\left(\!\left(0.1+0.5\, p_S^2 + 0.2\, p_J^2\right)\wedge 1+
0.5\, p_S^2+0.8\, p_J^2 \right)\wedge 2.
\end{equation*}
The inequality $p_S^1< 1$ is impossible since in this case, $0.1+0.5\, p_S^2 + 
0.2\, p_J^2<1$, implying that
$$
p_S^2 =\left(0.1+p_S^2 + p_J^2\right)\wedge 2\,.
$$
For positive values of unknown variables, the last equality may hold only if  
$p_S^2=2$ but then, the 1st equation tells one that  $p_S^1=1$.

Thus, it was determined that $p_S^1=1$.

Combining the 2nd equation with~(\ref{pS3}), one obtains the equality
$$
p_S^2  =  \left(1+0.5\, p_S^2+0.8\, p_J^2\right)\wedge 2
$$
implying that $p_S^2=2$.

Available information allows one to reduce
the 5th equation to the simple one of the  form
$p_J^2  = 0.8\left(p_J^2\right)^+\wedge 2$ having the unique solution  $p_J^2=0$.

Summarizing, one gets that  $p_S=(1,2,1)$ and $p_J=(0,0,0)$.

\smallskip

\noindent
\textbf{Comment.} In the first example, the bank 1 has met all liabilities and 
finished with a~positive equity,  the bank~2 has payed the senior liabilities 
but defaulted on the junior debts, the bank~3 has defaulted already at the 
senior debts; and the 
bank~2 has no junior liabilities with the bank~1.  So, the $\Delta$-orbit of the 
banks~2 and~3 are not surplus sets and there are infinite many clearing vectors. 
In the second example, the bank~2 has a~junior debt to bank~1, 
all  $\Delta$-orbits are surplus sets, and the clearing vector is unique.


\section{Models with Illiquid Assets and~a~Price Impact}

\noindent
Let us consider the clearing problem without seniority structure where the bank~$i$ 
owns not only cash~$e^i$ but also~$K$~illiquid assets, in quantities 
$y^{i1},\dots y^{iK}$ represented in  the model by the row~$i$ of the matrix 
$Y=(y^{im})$, $i\le N$, $m\le K$. The nominal prices per unit  of illiquid 
assets are strictly positive  numbers $Q^1,\ldots,Q^K$.  The clearing might  
require their partial   sale  influencing   the market price. If the bank sells  
$u^{im}\in [0,y^{im}]$ units of the $m$th assets for the price~$q_m$, its 
total increase in cash is
$$
(Uq)^i=\sum\limits_{m=1}^K u^{im}q^m\,.
$$

\textbf{The price formation}  is modeled by the inverse demand function 
$F_0:\mathbb{R}^K\to \mathbb{R}^K$ assumed to be continuous and monotone 
decreasing ($F_0(z)\le F_0(x)$ when $z\ge x$ in the sense of partial ordering 
defined by~$\mathbb{R}^K_+$) and 
such that $F_0(0)=Q$ and $F^m_0(Y'{\bf 1})>0$ for $m=1,\ldots , K$.  The first 
condition means that in the absence
of supply, the prices are just the nominal prices while  the second one shows 
that even in the case of total sale, the prices of illiquid assets remain strictly 
positive.


\textbf{The clearing rules:} each bank pays  debts in accordance to the matrix of 
relative liabilities
and sells illiquid assets if it has insufficient amount of cash. The result of 
clearing should be: all
debts of the bank are covered or its equity falls down  to zero.



In the case of several illiquid assets,  there is a~problem how the banks chose 
their strategies of selling. In principle, one can imagine the situation that 
they have  full freedom and, acting in the noncooperative way, drop down the 
market of  illiquid assets because of an excessive supply. It seems reasonable 
that the central authority may  impose extra rules on selling illiquid assets. 
Let us suppose that this is done by prescribing that the bank~$i$ must sell all 
assets in the same proportion~$\alpha^{i}$:
\begin{equation*}
\alpha^i(q)=\fr{\left(\tilde L^i -e^i - \sum\nolimits_j\Pi^{ji}p^j \right)^+}
{ \sum\nolimits_k  y^{ik} q^k}\,\wedge 1\,,\enskip i=1,\dots, N\,.
\end{equation*}
This formula means that for a~fixed market price, the bank does not sell illiquid 
assets
if its  cash reserve together with collected debts covers the liabilities.
In the another extreme case where
$$
\tilde L^i -e^i - \sum\limits_j\Pi^{ji}p^j \ge \sum\limits_k y^{ik} q^k=(Yq)^i\,,
$$
all illiquid assets have to be sold and the bank defaults. In the intermediate
case, the bank sells a~share $\alpha^i\in ]0,1[$ of the $m$th asset adding to its 
cash an extra amount
$(({\tilde L^i -e^i - \sum\nolimits_j\Pi^{ji}p^j})/{\sum\nolimits_k y^{ik} 
q^k})\,y^{im}q_m$.
The total increase in cash allows to cover the liabilities.

Under such a~rule, the  $i$th bank sells~$u^{im}:=u^{im}(p,q)$ units of the $m$th asset where
\begin{equation*}
u^{im}
{}:=\fr{y^{im}\left(\tilde L^i -e^i - \sum\nolimits_j\Pi^{ji}p^j 
\right)^+}{ \sum\nolimits_k y^{ik} q^k}\,\wedge y^{im}.
\end{equation*}
The total supply of the illiquid assets is given by the vector ${\bf 1}'U(p,q)$ 
where
$U(p,q)$ is the matrix with entries given by the above formula.

Define the equilibrium vector 
$$
\left(p^*,q^*\right)\in \left[0,\tilde L\right] \times \left[ F_0(1Y),Q\right]
$$ 
as 
the solution of the system of $N+K$ equations written in the matrix form as
\begin{align}
\label{firstM}
p&=(e+U(p,q)q+\Pi'p)\wedge \tilde L\,;\\
\label{secondM}
q&=F_0(U'(p,q){\bf 1})\,.
\end{align}
The existence of the equilibrium is easy. Indeed,
check that 
\begin{gather*}
U'(p,q){\bf 1}\ge U'\left(\tilde p,\tilde q\right){\bf 1}\,;\\
U(p,q)q+\Pi'p\le  U\left(\tilde p,\tilde q\right)\tilde q+\Pi'\tilde p
\end{gather*}
when $(\tilde p,\tilde q)\ge (p,q)$. Denoting  $F(p,q)$ the right-hand side 
of the first equation, one obtains that  
$$
(p,q)\mapsto \left(F(p,q),F_0\left(U'(p,q)\right){\bf  1}\right)
$$ 
is a~monotone  mapping of the order interval $[0,\tilde L]\linebreak\times [ F_0(1Y),Q]$ into 
itself.  According to Knaster--Tarski theorem, the set of its fixed points is nonempty 
and contains the minimal and maximal elements $(\underline p^*, \underline q^*)$ 
and $(\bar p^*,\bar q^*)$.

For a~fixed $q$, the function $p\to F(p,q)$ is monotone. Thus, by the 
Knaster--Tarski theorem, the set of solutions of Eq.~(\ref{firstM}) is nonempty 
and contains, in particular, the maximal element~$\bar p(q)$.

For any fixed $q\in [F_0(Y),Q]$, the largest solution $\bar p=\bar p(q)$ 
of~(\ref{firstM}) is given by formula:
$$
\bar p=\sup\left\{p\in [0,\tilde L]:\ p\le \left(e+U(p,q)q+\Pi'p\right)\wedge \tilde L\right\}
$$
implying that $q\mapsto \bar p(q)$ is an increasing (and continuous) function on 
$[F_0(Y),Q]$.  It follows that the supply function
$$
q\mapsto \zeta(q):=U'(\bar p(q),q){\bf 1}
$$
is decreasing and, therefore, $q\mapsto F_0(\zeta(q))$ is an increasing 
(and continuous) mapping of the interval  $[F_0(Y),Q]$ into itself and, 
therefore, it has  the minimal and maximal fixed points that will be denoted by~$q_1$ 
and~$q_2$.

\smallskip

\noindent
\textbf{Lemma~3.}\
\textit{Suppose that the scalar function $x\to x'F_0(x)$ is strictly increasing on 
$[F_0(Y),Q]$. Then, the
solution of the equation  $q=F_0(\zeta(q))$ is unique, i.\,e.}, $q_1=q_2$.

\smallskip

\noindent
P\,r\,o\,o\,f\,.\ \
Arguing by contradiction, suppose that  $q_1\neq q_2$.     Since $q_1\le q_2$ 
and $\zeta(\cdot)$ is decreasing,   $\zeta(q_1)\ge \zeta(q_2)$. Moreover, 
$\zeta(q_1)\neq \zeta(q_2)$ as the values of~$F_0$ at these points are~$q_1$ 
and~$q_2$.
 The assumed strict monotonicity  implies that
 $$
 \zeta'(q_1)F_0( \zeta(q_1))> \zeta'(q_2)F_0( \zeta(q_2)).
 $$
It follows that
$$
\zeta'\left(q_1\right) q_1> \zeta'\left(q_2\right)q_2\,.
 $$
To get a~contradiction, it is sufficient to show that
$$
\Delta:= \zeta'\left(q_2\right)q_2-\zeta'\left(q_1\right)q_1\ge 0\,.
$$
Let $\bar p_k:=\bar p(q_k)$ and let
$$
D_k:=\left\{i:\ \left(\tilde L-e-\Pi'\bar p\left(q_k\right)\right)^i\ge 
\left(Yq_k\right)^i\right\}\,,
$$
i.\,e., $D_k$ is the set of banks that are forced to sell all their illiquid assets 
for the price~$q_k$, $k=1,2$. Since~$\bar p(\cdot)$ is increasing, $D_2\subseteq D_1$.  
With the 
notation~${\bf 1}'_{A}$ for the row-vector representing the indicator function
of the subset $A\subseteq \{1,\dots, N\}$, one has, taking into account that 
$a^+=a+a^-$:
\begin{multline*}
\zeta'\left(q_k\right)q_k={\bf 1}'_{D_k}Yq_k\\
{}+{\bf 1}'_{D_k^c}\left(\tilde L-e-\Pi'\bar 
p_k\right)+{\bf 1}'_{D_k^c}\left(\tilde L-e-\Pi'\bar p_k\right)^-.
\end{multline*}
This formula leads to the representation:
\begin{multline*}
\Delta={\bf 1}'_{D_2}Y(q_2-q_1)-{\bf 1}'_{D_1\setminus D_2}Yq_1\\
{}- {\bf 1}'_{D_1^c}
\Pi'\left(\bar p_2-  \bar p_1\right)+{\bf 1}'_{D_2^c\setminus D_1^c}
\left(\tilde L-e -\Pi'\bar p_2\right)\\
{}+ {\bf 1}'_{D_1^c}\left(\left(\tilde L-e -\Pi'\bar p_2\right)^- -
\left(\tilde L-e -\Pi'\bar p_1\right)^-\right)\\
+
{\bf 1}'_{D_2^c\setminus D_1^c}\left(\tilde L-e -\Pi'\bar p_2\right)^-.
\end{multline*}
Since the function $x\to x^-$ (on ${\mathbb{R}}^N$) is positive and decreasing, the 
last two terms in the right-hand side are positive. Regrouping  the third and 
the forth  terms, one gets that
\begin{multline}
\label{ineq1}
\Delta\ge{\bf 1}'_{D_2}Y\left(q_2-q_1\right)-{\bf 1}'_{D_1\setminus D_2}q_1Y
- {\bf 1}'_{D_2^c}\Pi'(\bar p_2-
 \bar p_1)\\
 {}+{\bf 1}'_{D_1\setminus D_2}\left(\tilde L-e -\Pi'\bar p_1\right)\,.
\end{multline}
From Eq.~(\ref{firstM}), it follows that
\begin{multline*}
{\bf 1}'\Pi'\left(\bar p_2-  \bar p_1\right)=
{\bf 1}'\left(\bar p_2-  \bar p_1\right)={\bf 1}'_{D_1}\left(\bar p_2-  \bar p_1\right)\\
{}={\bf 1}'_{D_2}\left(q_2u\left(\bar p_2,q_2\right)-q_1u
\left(\bar p_1,q_1\right)+\Pi'\left(\bar p_2-  \bar p_1\right)\right)\\
{}+{\bf 1}'_{D_1\setminus D_2}\left(\tilde L -\left(e+q_1u\left(\bar p_1,q_1
\right) +\Pi'\bar p_1\right)\right)
\end{multline*}
implying that

\columnbreak

\noindent
\begin{multline*}
 {\bf 1}'_{D_2^c}\Pi'\left(\bar p_2-
 \bar p_1\right)={\bf 1}'_{D_2}\left(U\left(\bar p_2,q_2\right)q_2\right.\\
 \left.{}-
 U\left(\bar p_1,q_1\right)q_1\right)-{\bf  1}'_{D_1\setminus D_2}
 U\left(\bar p_1,q_1\right)q_1\\
{}+{\bf 1}'_{D_1\setminus D_2}\left(\tilde L-e -\Pi'\bar p_1\right)\,.
\end{multline*}
Substituting this expression in~(\ref{ineq1}), one has:
\begin{multline*}
\Delta\ge{\bf 1}'_{D_2}Y\left(q_2-q_1\right)-{\bf 1}'_{D_1\setminus D_2}Yq_1\\
{}-{\bf 1}'_{D_2}\left(U\left(
\bar p_2,q_2\right)q_2-U\left(\bar p_1,q_1\right)q_1\right)\\
{}+
{\bf 1}'_{D_1\setminus D_2}q_1u\left(\bar p_1,q_1\right)=0
\end{multline*}
since the cash increment $(U(\bar p_2,q_2)q_2)^i=(Yq)^i$ for the bank $i\in D_2$ 
and $(U(\bar p_1,q_1)q_1)^i=(Yq_1)^i$ for $i\in D_1\supseteq D_2$.~$\square$


\smallskip

\noindent
\textbf{Theorem~3.}\
\textit{Suppose that the scalar function $x\to x'F_0(x)$ is strictly increasing on 
$[F_0(Y),Q]$. Then, there is $q^*$ such that  the set of solutions of the 
system}~(\ref{firstM}),  (\ref{secondM})
\textit{is contained in the interval  with the extremities $(\underline p(q^*),q^*)$ and 
$(\bar p(q^*),q^*)$.
In particular, if for each~$q$ the solution of}~(\ref{firstM}) \textit{is unique, then 
the solution of the system is also unique}.

\smallskip

\noindent
P\,r\,o\,o\,f\,.\ \ 
Let~$\Gamma$ be the set of~$q$ for which $(p,q)$ is a~solution  of 
the system~(\ref{firstM}),  (\ref{secondM}). If $q^*\in \Gamma$, then $(\bar 
p(q^*),q^*)$
is the solution of~(\ref{firstM}),  (\ref{secondM}). According to  
lemma~3, the point~$q^*$ is uniquely defined. This implies the result.~$\square$

\smallskip

Note that the uniqueness of the solution of~(\ref{firstM}) is guarantied if  for 
each~$i$,
the orbit of~$i$ contains an element with positive cash reserve.

\smallskip

\noindent
\textbf{Remark~2.}
In~\cite{AFM},  it was considered  a~model coinciding with studied 
above
in the case of a~single illiquid asset. The difference is that in the cited 
paper, the equilibrium is defined  as a~vector $(p,q)$ satisfying the
system of equations:
\begin{align}
\label{firstAFM}
p&=\left(e+qy+ \Pi'p\right)^+\wedge \tilde L\,; \\
%\label{secondAFM}
q&=F_0\left({\bf 1}'\left(\left(q^{-1}
\left(\tilde L-e-\Pi'p\right)^+\right)\wedge y\right)\right).\notag
\end{align}
To our opinion, the definition of the equilibrium given 
by the system~(\ref{firstM}), 
(\ref{secondM}), which is in the one liquid asset case has the  form:
\begin{align}
p&=\left(e+\left(\tilde L-e-\Pi'p\right)^+\wedge (qy)+ 
\Pi'p\right)\wedge \tilde L\,; \label{firstAFM1}
\\
%\label{secondAFM1}
q&=F_0\left({\bf 1}'\left(\left(q^{-1}\left(\tilde L-e-\Pi'p\right)^+
\right)\wedge y\right)\right), \notag
\end{align}
 is more natural.  In fact, the   right-hand sides of~(\ref{firstAFM}) 
 and~(\ref{firstAFM1}) as functions $R_1(p,q)$ and $R_2(p,q)$ defined
 on $[0,\tilde L]\times [ F_0(1Y),Q]$ coincide.  To see this, fix~$i$ and  
consider the three possible cases.
\begin{enumerate}[1.]
\item  Let  $e^i+qy+ (\Pi'p)^i\le \tilde L^i$. Then, the expressions for 
$R^i_1(p,q)$ and $R^i_2(p,q)$ have the same form  $e^i+qy+ (\Pi'p)^i$.

\item Let $e^i+qy+ (\Pi'p)^i> \tilde L^i$ and $\tilde L^i-e^i - (\Pi'p)^i\ge 0$. 
Then, the values $R^i_1(p,q)$ and $R^i_2(p,q)$ are equal to~$\tilde L^i$.

\item Let $e^i+qy+ (\Pi'p)^i> \tilde L^i$ and $\tilde L^i-e^i - (\Pi'p)^i<0$. 
Then, the value of $R^i_1(p,q)$ is $\tilde L^i$ and the value of $R^2_1(p,q)$ is 
$(e^i + (\Pi'p)^i)\wedge \tilde L^i=\tilde L^i$.
\end{enumerate}

\vspace*{-18pt}


{\small
\section*{\raggedleft Appendix}

%\vspace*{-6pt}

\subsection*{Knaster--Tarski Fixpoint Theorem}
%\label{app}

\noindent
Let $X$ be a~set with a~partial ordering~$\ge$ and let~$A$ be its nonempty 
subset.
By definition,~$\sup A$ is an element~$\bar x$ such that $\bar x\ge x$ for all 
$x\in A$ and if~$\bar x'$ is such that  $\bar x'\ge x$ for all $x\in A$, then 
$\bar x'\ge \bar x$. The definition of~$\inf A$ follows the same pattern but 
with the dual ordering~$\le$.  A~partially ordered set~$X$ is  {\it complete 
lattice} if for any its nonempty subset~$A$,
there exist~$\inf A$ and~$\sup A$.

\smallskip

\noindent
\textbf{Theorem~4.}\
\textit{Let $X$ be a~complete lattice and let $f : X \mapsto X$ be an order-preserving 
mapping, $L:=\{x:\  f(x)\le x\}$, $U:=\{x:\ f(x)\ge x\}$.   The set
$L\cap U$ of fixed points of~$f$
is nonempty and has the smallest and the largest fixed points  which are, 
respectively, $\underline x:=\inf L$ and}   $\bar x:=\sup U$.

\smallskip

 \noindent
P\,r\,o\,o\,f\,.\ \  
Note that $L\neq \emptyset$ since it contains the element~$\sup X$.
Take arbitrary $x\in L$. Then, $\underline x\le x$
implying that $f(\underline x)\le f(x) \le x$. Thus, $f(\underline x)\le 
\underline x$ as~$\underline x$ is~$\inf L$. So, $\underline x\in L$. 
Since $f(L)\subseteq L$, 
also $f(\underline x)\in L$; hence,  $\underline x\le f(\underline x)$, i.\,e., 
$\underline x= f(\underline x)$. All fixed points belong to~$L$ and, 
therefore,~$\underline x$ is the smallest one.

The proof of the statement for the largest fixed point is analogous.~$\square$

\smallskip

 \noindent
 \textbf{Corollary.}\
\textit{Let $f(\cdot;y)$ be an order-preserving mapping of a~complete lattice $(X,\ge)$ into 
itself, depending on the parameter~$y$ from a~partially ordered set 
$(Y,\succeq)$.
Suppose that $f(\cdot,y)$ is increasing in~$y$, that is, $f(x,y')\ge f(x,y)$ for all 
$x\in X$ when $y'\succeq y$.  Then, the smallest and the largest fixed points are 
also increasing in}~$y$.

\smallskip

\noindent
P\,r\,o\,o\,f\,.\ \ The claim is obvious because the sets   
$$
L(y):=\{x:\  f(x,y)\le x\}
$$ 
are decreasing and the sets 
$$
U(y):=\{x:\ f(x,y)\ge x\}$$ 
are increasing in~$y$
(see~\cite{Milgrom-Roberts}).

These general results are applied to the order intervals $[a,b]\subset \mathbb{R}^d$
with the ordering induced by~$\mathbb{R}^d_+$.

}

\vspace*{-6pt}

\Ack
\noindent
The 
research of Yuri Kabanov was done under partial financial support   of the grant 
of  the Russian Science Foundation No.\,14-49-00079.


\renewcommand{\bibname}{\protect\rmfamily References}

\vspace*{-6pt}

{\small\frenchspacing
{%\baselineskip=10.8pt
\begin{thebibliography}{9}

\bibitem{Eisenberg-Noe} %1
\Aue{Eisenberg, L., and T.\,H.~Noe}. 2001. Systemic risk in financial systems. 
\textit{Manag. Sci.} 47(2):236--249.

\bibitem{Suzuki} %2
\Aue{Suzuki, T.} 2002. Valuing corporate debt: The effect of cross-holdings of stock 
and debt. \textit{J.~Oper. Res. Soc. Japan} 45(2):123--144.

\bibitem{Tarski} %3
\Aue{Tarski, A.} 1955. A~lattice-theoretical fixpoint theorem and its applications. 
\textit{Pacific J.~Math.} 5(2):285--309.


\bibitem{Cont-Wag} %4
\Aue{Cont, R., and L.~Wagalath}. 2015. Fire sale forensics: Measuring endogenous risk. 
\textit{Math. Finance} 26:835--866. %doi: 10.1111/mafi.12071.

\bibitem{AFM} %5
\Aue{Amini, H., D.~Filipovi$\acute{\mbox{c}}$, and A.~Minca.} 2015. To fully net or not to net: Adverse 
effects of partial multilateral netting. %Swiss Finance Institute Research Paper  series. No.~14-63. Forthcoming in ``
\textit{Oper. Res.} 64(5):1135--1142.

\bibitem{Elsinger2011} %6
\Aue{Elsinger, H.} 2009. Financial networks, cross holdings, and limited liability. 
Working paper from Oesterreichische Nationalbank.

\bibitem{Milgrom-Roberts} %7
\Aue{Milgrom, J., and J.~Roberts.} 1994. Comparing equilibria. 
\textit{Am. Econ. Rev.}  84:441--454.




\end{thebibliography} } }

\end{multicols}

\vspace*{-6pt}

\hfill{\small\textit{Received September 25, 2016}}

\vspace*{-18pt}

\Contr

%\vspace*{-3pt}

\noindent
\textbf{El Bitar  Khalil} (b.\ 1981)~--- 
PhD student, Laboratoire de Mathematiques, Universite de Franche-Comte, 
16~Route de Gray, 25030, \mbox{Besan{\!\ptb{\c{c}}}on}, CEDEX, France; 
\mbox{khalilbitar\_aw@hotmail.com}  

 \vspace*{1pt}
 
 \noindent
 \textbf{Kabanov Yuri M.} (b.\ 1948)~---
  professor, Laboratoire de Mathematiques, Universite de Franche-Comte, 
  16~Route de Gray, 25030, Besancon, CEDEX, France; leading scientist, 
  Institute of Informatics Problems, Federal Research Center 
  ``Computer Science and Control'' of the Russian Academy of Sciences,  
  44-2~Vavilov Str., Moscow 119333, Russian Federation; 
  National Research University ``MPEI,'' 14~Krasnokazarmennaya Str., 
  Moscow 111250, Russian Federation; \mbox{Youri.Kabanov@univ-fcomte.fr} 

\vspace*{1pt}
 
 \noindent
 \textbf{Mokbel Rita} (b.\ 1981)~--- 
 PhD student, Laboratoire de Mathematiques, Universite de Franche-Comte, 
 16~Route de Gray, 25030, Besancon, CEDEX, France; \mbox{ritamokbel@hotmail.com}




%\vspace*{8pt}

%\hrule

%\vspace*{2pt}

%\hrule

\newpage

\vspace*{-24pt}



\def\tit{О~ЕДИНСТВЕННОСТИ КЛИРИНГОВЫХ ВЕКТОРОВ, РЕДУЦИРУЮЩИХ 
СИСТЕМНЫЙ РИСК$^*$}

\def\aut{Х.~Эль Битар$^1$, Ю.~Кабанов$^{1,2,3}$, Р.~Мокбель$^1$}


\def\titkol{О~единственности клиринговых векторов, редуцирующих 
системный риск}

\def\autkol{Х.~Эль Битар, Ю.~Кабанов, Р.~Мокбель}

{\renewcommand{\thefootnote}{\fnsymbol{footnote}}
\footnotetext[1]{Представленные в настоящей статье результаты исследований, проведенных 
Ю.\,М.~Кабановым, были получены при частичной финансовой поддержке 
Российского научного фонда (проект №\,14-49-00079).}}


\titel{\tit}{\aut}{\autkol}{\titkol}

\vspace*{-12pt}

\noindent
$^1$Лаборатория математики Университета Франш-Кон\-те, г.~Безансон, Франция

\noindent
$^2$Институт проблем информатики Федерального исследовательского
центра <<Информатика и~управление>>\linebreak
$\hphantom{^1}$Российской академии наук, Российский
университет дружбы народов

\noindent
$^3$Национальный исследовательский университет <<МЭИ>>

\vspace*{6pt}

\def\leftfootline{\small{\textbf{\thepage}
\hfill ИНФОРМАТИКА И ЕЁ ПРИМЕНЕНИЯ\ \ \ том\ 11\ \ \ выпуск\ 1\ \ \ 2017}
}%
 \def\rightfootline{\small{ИНФОРМАТИКА И ЕЁ ПРИМЕНЕНИЯ\ \ \ том\ 11\ \ \ выпуск\ 1\ \ \ 2017
\hfill \textbf{\thepage}}}


\Abst{В~финансовых системах, т.\,е.\ в сети взаимосвязанных банков, 
процедура взаимозачета, или клиринга, состоит в~одновременной выплате 
задолженностей с~целью уменьшения общей их суммы в~системе. Вектор, компоненты 
которого есть суммарные выплаты каждого банка системы, называется клиринговым 
вектором. В~простых моделях, предложенных Айзенбергом и Ноэ (2001) и~независимо 
Судзуки (2002) было показано, что полный клиринг описывается вектором, который 
является неподвижной точкой некоторого отображения. Существование клирингового 
вектора может быть получено прямой ссылкой на теоремы о~неподвижной точке 
Кнас\-те\-ра--Тар\-скo\-го или Брауэра. Вопрос о~его единственности является более 
деликатным. Айзенберг и Ноэ получили достаточное условие единственности 
в~терминах графа связей финансовой системы. В~настоящей работе доказывается 
единственность для двух более общих моделей: модели Эльсингера с~приоритетами 
долгов и~модели типа Ами\-ни--Фи\-ли\-по\-ви\-ча--Мин\-ки, 
в~которой банки имеют неликвидные 
активы, продажа которых влияет на их рыночную цену.}

\KW{системный риск; финансовые сети; клиринг; теорема 
Кнас\-те\-ра--Тар\-ско\-го; модель Ай\-зен\-бер\-га--Ноэ; приоритет финансовых обязательств; 
влияние на ценообразование}



\DOI{10.14357/19922264170110}

%\vspace*{6pt}


 \begin{multicols}{2}

\renewcommand{\bibname}{\protect\rmfamily Литература}
%\renewcommand{\bibname}{\large\protect\rm References}

{\small\frenchspacing
{%\baselineskip=10.8pt
\begin{thebibliography}{9}
\bibitem{3-kab} %1
\Au{Eisenberg L., Noe~T.\,H.} Systemic risk in financial systems~// 
Manag. Sci., 2001. Vol.~47. No.\,2. P.~236--249.
\bibitem{6-kab} %2
\Au{Suzuki T.} Valuing corporate debt: The effect of cross-holdings of stock and debt~// 
J.~Oper. Res. Soc. Japan, 2002. Vol.~45. No.\,2. P.~123--144.
\bibitem{7-kab} %3
\Au{Tarski A.} A~lattice-theoretical fixpoint theorem and its applications~// 
Pacific J.~Math., 1955. Vol.~5. No.\,2. P.~285--309.

\bibitem{2-kab} %4
\Au{Cont R., Wagalath~L.} Fire sale forensics: Measuring endogenous risk~// 
Math.  Finance, 2015. Vol.~26. P.~835--866. %doi: 10.1111/mafi.12071.
\bibitem{1-kab} %5
\Au{Amini H., Filipovi$\acute{\mbox{c}}$~D., Minca~A.} To fully net or not to net: 
Adverse effects of partial multilateral netting~// Oper. Res., 2015. Vol.~62.
No.\,5. P.~1135--1142.

\bibitem{4-kab} %6
\Au{Elsinger H.} Financial networks, cross holdings, and limited liability. 
Working paper from Oesterreichische Nationalbank, 2009.
\bibitem{5-kab} %7
\Au{Milgrom J., Roberts~J.} Comparing equilibria~// Am. Econ. Rev., 1994. 
Vol.~84. P.~441--454.


\end{thebibliography}
} }

\end{multicols}

 \label{end\stat}

 \vspace*{-3pt}

\hfill{\small\textit{Поступила в редакцию  25.09.2016}}
%\renewcommand{\bibname}{\protect\rm Литература}
\renewcommand{\figurename}{\protect\bf Рис.}
\renewcommand{\tablename}{\protect\bf Таблица}  %10
\renewcommand{\figurename}{\protect\bf Figure}
\renewcommand{\tablename}{\protect\bf Table}

\def\stat{self-mul}

\def\tit{INFORMATICS AND~ITS~ROLE FOR~THE~STUDY OF~GENESIS 
AND~PROPERTIES OF~COMPLEX NATURAL SYSTEMS$^*$}

\def\titkol{Informatics and its role for the study of genesis 
and properties of complex natural systems}

\def\autkol{R.\,B.~Seyful-Mulyukov}

\def\aut{R.\,B.~Seyful-Mulyukov$^1$}

\titel{\tit}{\aut}{\autkol}{\titkol}


\index{Seyful-Mulyukov R.\,B.}
\index{Сейфуль-Мулюков Р.\,Б.}

{\renewcommand{\thefootnote}{\fnsymbol{footnote}}
\footnotetext[1] { The investigation was carried out according to the Program 
``Informatics 
methods in development of the petroleum origin theory and elaboration of new 
technologies for exploring petroleum and gas accumulations and providing energy 
security of the Russian Federation'' under the general theme ``Society 
informatization and information security.''}}

\renewcommand{\thefootnote}{\arabic{footnote}}
\footnotetext[1]{Institute of Informatics Problems, Federal Research Center 
``Computer Science and Control'' of the Russian Academy of 
Sciences, 44-2~Vavilov Str.,  Moscow 119333, Russian Federation, \mbox{rust@ipiran.ru}}




\def\leftfootline{\small{\textbf{\thepage}
\hfill INFORMATIKA I EE PRIMENENIYA~--- INFORMATICS AND APPLICATIONS\ \ \ 2017\ \ \ volume~11\ \ \ issue\ 1}
}%
 \def\rightfootline{\small{INFORMATIKA I EE PRIMENENIYA~--- INFORMATICS AND APPLICATIONS\ \ \ 2017\ \ \ volume~11\ \ \ issue\ 1
\hfill \textbf{\thepage}}}



\Abste{The paper considers the history of cognition of information as a~phenomenon and 
informatics as its quantitative and qualitative development. The logical connection between such 
notions as information, informatics, complexity, and complex natural self-organizing systems is 
investigated. It is considered that information, besides its usual traditional meaning, is one of the 
main properties of matter. Informatics is considered as an instrument for cognition of 
development and structure of complex natural systems. Petroleum is chosen as an example of 
such system. It is proved that petroleum, as well as each its hydrocarbon molecule, possesses 
corpuscular properties, and petroleum as a~whole has information volume calculated in bits. 
A~new approach is proposed for petroleum accumulations exploration. It is based on the fact 
that petroleum generation is a~discrete process. Consequently, the process of discovering 
petroleum accumulations has two stages. The first stage is characterized by static 
uncertainty
and the second stage is characterized by dynamic uncertainty. Both types of uncertainty need to 
be removed. The paper presents technologies and methods of solving these problems.}

\KWE{informatics; informatization; complex natural system; petroleum origin; petroleum 
exploration; static uncertainty; dynamic uncertainty}

\DOI{10.14357/19922264170111} 

%\vspace*{9pt}


\vskip 10pt plus 9pt minus 6pt

      \thispagestyle{myheadings}

      \begin{multicols}{2}

                  \label{st\stat}

\noindent
   Historically, ``information'' was understood as data about an event, a~state, or 
other characteristics of a~phenomenon that living species transmitted to each other. 
In particular, information is a~light, heat, or sound signal from a~natural 
phenomenon, which plants receive or to which they react. Information is a~signal 
of different types that terrestrial animals, pests, birds, as well as marine creatures 
receive or exchange with each other.
{\looseness=1

}
   
   In economics, main data are expressed with digits. In industry and agriculture, 
``information'' is the name of specifications of goods and services or unit 
measurements, such as weight, mass, volume, size, distance, and others. In 
education, ``information'' is all totality of the basic knowledge about nature, 
history, and laws. This knowledge has vital importance for development of 
mankind. In each sphere of activities, ``information" is understood in a~specific 
way.
   
   The notion of ``informatization'' appeared in the information theory and 
different spheres of information applications. Its appearance was caused by a~rapid 
increase of flow of information, which was used in everyday life, industry, science, 
culture, education, social, and other spheres.
   
   Information has to be transmitted, received, processed, interpreted, stored, and 
undergo many other manipulations. It is necessary for the right positioning of an 
individual or a~community in a~society and has vital importance for economic 
independency and national security. All of these activities are informatization.
   
   Informatization is not a~one-time campaign. Everyday activities introduced the 
public consciousness to the necessity of informatization of all spheres of social 
activities. K.~Kolin proved that informatization has to be perceived in the public 
consciousness as a~powerful instrument for qualitative modification of education, 
science development, new technologies application, improving management, and 
other activities. All of these activities have vital importance for development and 
national security~[1].
   
   This paper presents application of informatization to solving one of 
fundamental problems of natural sciences~--- the problem of genesis of petroleum 
and natural hydrocarbon gas. For this reason, the paper contains a~short introduction 
to the history of cognition of the notion of ``information.'' It does not mean only 
social phenomena or informatics as an instrument, which provides information 
storage, usage, transmission, and processing. ``Information'' also means the 
properties of matter, which are associated with the notion of a~``complex natural 
system.'' These systems can be cognized using the laws of informatics.
   
   Understanding of the notion of ``information'' depends on means of its 
transmission, usage, application, and many other factors and is always subjective. 
There is no exact definition of ``information,'' which would be universally 
recognized, and such definition is not possible in principle. The development of 
civilization and growth of our knowledge about matter, movement, time, and space 
resulted in a~new deeper and more comprehensive understanding of ``information.'' 
The most important achievement was the identification of ``information'' and 
``uncertainty'' that were measured by the C.~Shannon's mathematical theory of 
information~[2]. It was a~qualitative definition of ``information.'' For the first time 
in the history of information science, A.~Ursul presented an integrated 
philosophical definition of ``information'' that shows the relation between its 
quantitative and qualitative content~[3].
   
   The revolution of information cognition was the result of detection of its new 
meaning. Physicists M.~Planck and L.~de~Broglie~[4] investigated matter on 
atomic and subatomic levels and proved that besides its usual meaning as 
something existing in our consciousness, information is also one of the main 
properties of matter that exists beyond men's consciousness or wish. A.~Zeilinger 
proved that each elementary particle of an atom contains one bit of 
information~[5]. 
   
   Further investigation of qualitative and quantitative characteristics of 
information caused the appearance of new disciplines. First among them was 
cybernetics that N.~Wiener defined as a~``scientific study of control and 
communication in animals and machines''~[6]. The practical application of this 
idea was the computer. Later, the term ``informatics'' was introduced by 
K.~Steinbuch~[7]. Since~1966, informatics was positioned as a~science about 
collection, storage, distribution, retrieval, and use of scientific and technical 
information~[8].
    
    Russian and American scientists continued to investigate informatics and 
information science theory and applied problems, which are based on 
achievements of mathematics, physics, cybernetics, and philosophy. The approach 
of American information specialists and their understanding of ``information'' and 
``information science'' is best described in two monographs. The first one considers 
information science as a~metascience~[9]. Its integral parts are mathematics, 
linguistics, psychology, library science, engineering science, and computer 
science~[10]. Physics and cybernetics were predecessors of information science; 
therefore, they are not present in this list. However, these disciplines proved 
physical nature of information as one of its main properties.
    
    Another monograph is an official publication of the American Society for 
Information Science and Technology (ASIS\&T). The monograph follows the 
ideas of Otten and Debone. Information science is considered mainly as 
investigation of mental perception and interpretation of information as 
a~phenomenon existing in our consciousness. A~considerable part of the 
monograph is devoted to information in economics. In the case of market 
economy, ``information'' (in other words, ``uncertainty'' or ``information entropy'') 
is related to the notion of ``value.'' It means that the Shannon's theory, which deals 
with quantity of information, represents the quantitative side of information with 
its value. In this context, information has value since it can be bought or sold in 
order to decrease uncertainty and create a~product with a~larger value.
   
   The development of the information theory relates to I.~Gurevich~[11]. Basing 
on the postulate that information is one of the main fundamental properties of 
matter, he calculated information content of each chemical element of the 
Mendeleev table in bits. Besides, he proved that laws of fundamental sciences 
including informatics make it possible to understand development of complex 
natural and social systems.
{\looseness=1

}
   
   An indicative example, which demonstrates effectiveness of applying 
informatics laws to the study of complex natural systems, is the problem of origin 
of petroleum and natural hydrocarbon gas. According to the currently dominating 
model, petroleum origin is mainly the problem of petroleum geology and 
geochemistry. This model has become insufficient, which results in a~considerable 
decrease of effectiveness of exploration works. Surprisingly, application of 
informatics laws solves this problem. In informatics, the nature and properties of 
petroleum are represented as direct or indirect consequences of the fact that 
petroleum is a~complex self-developing natural system~[12, 13]. Its nature, main 
properties, and genesis become understandable, if one considers them in the 
context of the phenomenon of complexity and informatics laws.
   
The new understanding of deep inorganic nature of petroleum as a~phenomenon 
that is not related to the biosphere in any way leads to the necessity of changing the 
ideology of exploration of its commercial accumulations. The current methodology 
is based on the assumption that remainders of the biosphere's plants and animals 
served as raw material for petroleum and gas generation. However, nobody has 
ever proved the existence of mother rocks generating petroleum and their ability to 
transform organic matter into hydrocarbon molecules for sure. Nobody has ever 
proved that petroleum is able to migrate through geological rocks, so media keeps 
its initial hydrocarbon composition. The deep inorganic nature of petroleum proves 
its place in the hierarchy of organization of matter on the Earth. There are three 
main forms of organization of matter. The higher form is represented by billions of 
\textit{cells} of living species. The intermediary form is represented by thousands 
of petroleum hydrocarbon \textit{molecules}. The lower form is represented by 
several \textit{atoms} of chemical elements composing crystals of rocks. 

The most simple petroleum hydrocarbon molecules transform into the most 
complex ones, which is a~self-organizing discrete process. As the result, a~complex 
natural system appears, which possesses several properties. This paper considers 
the properties, which are the most significant for petroleum generation, namely, 
\textit{existence}, \textit{development}, and \textit{cognoscibility}~[14]. These 
properties are typical for any complex natural system.

System uniqueness is one of existence features. Petroleum is a~unique phenomenon 
that is created by a~unique set of geological, geochemical, fluid-dynamic, and other 
natural conditions. They are always different for each period of the Earth's history. 
Petroleum composition and structure correspond completely to the current period 
of the Earth's development. 
     
     System constant movement is one of \textit{development} features. 
Petroleum is permanently migrating and therefore changing. Petroleum, which is 
conserved in one place by geological media and keeps its original hydrocarbon 
composition for hundreds of millions of years, does not exist anymore. 
Consequently, inside the Earth, there is no Devonian (360~M years old), 
Carboniferous (300~M years old), Permian (251~M years old), and Jurassic 
(145~M years old) petroleum. Its generation does not take millions of years, but 
hundreds or thousands of years. Petroleum generation has the following stages: 
hydrocarbon radicals and methane molecules appear in the Earth's upper mantle, 
migrate through the Earth's crust, interact with rocks of geological media, 
transform into hydrocarbon molecules of different kinds, and finally, accumulate 
within a~reservoir as petroleum.
{\looseness=1

}
     
     \textit{Cognoscibility} of a~complex system pertains to the gnoseological 
aspect of petroleum nature. There is a~principal question: Is our perception of 
petroleum adequate to its real nature and age or nonadequate? There are two 
possible answers to this question. One answer corresponds to the organic paradigm 
of petroleum genesis, which considers that composition, properties, and other 
features of petroleum just pumped from a~reservoir are related to the phenomena, 
which were generated hundreds of millions years ago. It means that animals and 
plants of the previous epoch, which served as a~source for kerogen generation, had 
the same composition as they have now. Another opposite idea is that petroleum 
composition and genesis correspond completely to the modern epoch of the Earth's 
development. 
     
     It is absolutely clear that we cognize composition and features of a~complex 
system as a~natural phenomenon, which is generated by all totality of modern 
geological, geochemical, thermodynamic, and other conditions. Otherwise, one has 
to accept that~300, 200, and~100~M years ago geology, geochemistry, 
thermodynamics, and many other natural conditions of the Earth's interior were the 
same as they are now. It is impossible in principle. So, according to the 
\textit{cognoscibility} feature, petroleum is a~modern complex abiogenic natural 
system, which consists of several thousands of hydrocarbon molecules, which are 
not related to the biosphere.
{\looseness=1

}
     
     The petroleum example allows demonstrating one more important feature. 
The necessity of informatization of scientific research follows from the fact that 
information is one of the main properties of matter. Petroleum characteristics have 
never been considered before in this context. Petroleum possesses physical and 
chemical properties, as well as information content, which corresponds to the 
quantity of carbon and hydrogen atoms, which compose hydrocarbon 
molecules~\cite{13-seif}. In this case, one has to consider that any atom has 
information volume that is calculated in bits. Petroleum consists of hydrocarbon 
molecules on~95\%. They are divided into three main groups: paraffin  
(30\%--35\%), naphthenic (25\%--75\%), and aromatic (10\%--15\%). There are 
different kinds of petroleum with different quantities of hydrocarbon molecules of 
different kinds: light oil (C$_{32}$H$_{66}$SN), low-gravity oil 
(C$_{32}$H$_{66}$OSN), and bitumen (C$_{45}$H$_{51}$O$_2$SN). 
Correspondingly, their information volume is~16\,224, 16\,729, and~17\,789~bits.
     
     Application of the complex natural system ideology and informatics laws 
opens new possibilities for petroleum, such as exploration in different geological 
media. Therefore, the procedure of exploration of commercial petroleum 
accumulations requires reconsideration. Static and dynamic uncertainty are typical 
for a~complex natural self-organizing system. Both types of uncertainty have to be 
removed in order to find the place where petroleum is accumulated. This 
exploration strategy was proposed for the first time in~\cite{12-seif}.
     
     Static uncertainty applies to geological elements, whose location in the 
Earth's interior has not changed during all the time of petroleum generation and 
accumulation. These geological elements are trap, reservoir, impermeable 
covering, and canal for petroleum migration. Petroleum accumulation could not 
have happened without these geological elements. They must be present together, 
and their static uncertainty has to be removed in the Aristotelian logic terms~--- 
``yes'' or ``no'' only. The units used to express exact characteristics of a~given 
geological element are its size, density, permeability, fracturing, degree, as well as 
petrographic and chemical composition of reservoir, trap, and others. All of these 
characteristics can be successfully determined by seismic survey technologies. 
However, problems arise when all geological elements, provided by three-dimensional (3D) 
mathematical models of petroleum bearing basins, are present, but exploration 
wells turn out to be dry. Currently, the average statistical percent of successful 
exploration does not exceed~30\%. 
     
     Therefore, besides static uncertainty, there exists another type of uncertainty 
that does not deal with stable geological elements, but with results of a~dynamic 
discrete process. The result is a~petroleum accumulation, which is located 
sometimes under~10~km of sedimentary rocks. Fixing the stages of petroleum 
generation and the trajectory of its migration in time and space is impossible. It 
means that one cannot present a~discrete process as a~Cantorian set and the result 
of a~discrete process cannot be determined unambiguously by the Boolean classic 
logic, since mathematical calculation of this logic needs alternating values~``1'' 
or~``0'' only. 
     
     Hence, the problem of removing dynamic uncertainty and deciding whether 
a~petroleum accumulation exists or not does not have a~unique unambiguous 
solution. If a~problem does not have a~unique solution in principle, one can pass to 
its multiple-valued solution. In this case, intermediate values are introduced. The 
probability of accumulation presence in deep strata can be expressed with values 
0.1, 0.2, 0.25,\ $\ldots$\,,\ 0.7,\ $\ldots$\ until~0.95. It means that if static uncertainty was 
removed for sure, then the probability of removing its dynamic uncertainty could 
be calculated by applying the fuzzy set theory equations~\cite{15-seif}. Petroleum 
geologists and geophysicists possess all totality of information about the Earth's 
interior, including its 3D mathematical model, as well as experience and intuition. 
However, in many cases, this information is insufficient for planning well drilling 
in a~certain point. One can get additional data on that point calculated 
mathematically to be sure that the probability of discovering a~petroleum 
accumulation is equal to~25\%, 75\%, or even~95\%.
     
     The fuzzy set can consist of multiple direct and indirect geochemical indices 
of hydrocarbon molecules detected in different elements of environment. Light 
gasiform hydrocarbon molecules belong only to petroleum, which migrated from 
the pool to the surface and accumulated in the soil, snow, plants, air located close 
to the surface, and other elements of environment. Hydrocarbon molecules can be 
detected by modern geochemical methods.
     
     The new approach described in the 
paper made it possible to propose a~solution of the fundamental problems of 
petroleum and gas genesis and of exploration of their commercial accumulations in 
a~nontraditional, nontrivial way. 
    
\renewcommand{\bibname}{\protect\rmfamily References}


{\small\frenchspacing
{%\baselineskip=10.8pt
\begin{thebibliography}{99}

\bibitem{1-seif}
\Aue{Kolin, K.\,K.} 2010. \textit{Filosofskie problemy informatiki} 
[Philosophic problems of informatics]. Moscow: BINOM. 259~p.
\bibitem{2-seif}
\Aue{Shannon, C.\,E.} 1948. A~mathematical theory of communication. 
\textit{Bell Syst. Tech.~J.} 27:379--423, 623--656.
\bibitem{3-seif}
\Aue{Ursul, A.\,D.} 2010. \textit{Priroda informatsii: Filosofskiy ocherk} 
[The nature of information: A~philosophic essay]. 2nd ed. Chelyabinsk: 
CHGAKI. 231~p.
\bibitem{4-seif}
\Aue{De Broiglie, L.} 1927. Wave mechanics and the atomic structure of 
matter and radiation. \textit{J.~Phys. Radium} 8(5):225--241.
\bibitem{5-seif}
\Aue{Zeilinger, A.} 1999. A~foundation principal for quantum mechanics. 
\textit{Found. Phys.} 29(4):631--643.
\bibitem{6-seif}
\Aue{Wiener, N.} 1961. \textit{Cybernetics, or control and communication 
in the animal and the machine}. 2nd rev. ed. Cambridge: MIT Press. 
232~p.
\bibitem{7-seif}
\Aue{Steinbuch, K.} 1957. Informatik. \textit{Automatische 
Informationsverarbeitung, SEG-Nachrichten} (Technische Mitteilunger der 
Standard Elektrik Gruppe). No.~4. 171~p.
\bibitem{8-seif}
\Aue{Mikhaylov, A.\,I., A.\,I.~Chernyy, and R.\,S.~Gilyarevskiy}. 1968. 
\textit{Osnovy informatiki} [The foundations of informatics]. Moscow: 
Nauka. 425~p.
\bibitem{9-seif}
\Aue{Otten, K., and A.~Debons}. 1970. Towards a~metascience of 
information: Informatology. \textit{J.~Am. Soc. Inform. 
Sci.} 21:89--94.
\bibitem{10-seif}
\Aue{Norton, M.\,J.} 2010. \textit{Introductory concepts in information 
science}. 2nd ed. ASIST monograph ser. Information Today. 210~p.
\bibitem{11-seif}
\Aue{Gurevich, I.\,M.} 2007. \textit{Zakony informatiki~--- osnova 
stroeniya i~poznaniya slozhnykh system} [The laws of informatics as a~basis 
of cognizing complex systems]. Moscow: TORUS PRESS. 399~p.
\bibitem{12-seif}
\Aue{Heylighen, F.} 2008. Сomplexity and self-organization. 
\textit{Encyclopedia of library and information sciences}. Eds. M.\,J.~Bates 
and M.\,N.~Maack. Taylor \& Frances. 9~p. Available at: {\sf 
http://pespmc1.vub.ac.be/Papers/ELIS-complexity.pdf} (accessed 
February~8, 2017).
\bibitem{13-seif}
\Aue{Seyful-Mulyukov, R.\,B.} 2012. \textit{Neft' i~gaz. Glubinnaya priroda 
i~ee prikladnoe znachenie} [Petroleum and gas. Deep nature and its applied 
meaning].  Moscow: TORUS PRESS. 214~p. 
\bibitem{14-seif}
\Aue{Seyful-Mulyukov, R.\,B.} 2010. \textit{Neft'~--- uglevodorodnye 
posledovatel'nosti: Analiz modeley genezisa i~evolyutsii} [Petroleum~--- 
hydrocarbon sequences: An analysis of models of genesis and evolution]. 
Moscow: 11~format. 173~p.
\bibitem{15-seif}
\Aue{Zadeh, L.\,A.} 1975. The concept of linguistic variable and its 
application to approximate reasoning. \textit{Inform. Sci.} 8:199--249,  
310--357; 9:43--80.
\end{thebibliography} }
 }

\end{multicols}

\vspace*{-6pt}

\hfill{\small\textit{Received October 11, 2016}}

\vspace*{-24pt}

\Contrl


\vspace*{3pt}

\noindent
\textbf{Seyful-Mulyukov Rustem B.} (b.\ 1928)~--- Doctor of Science in 
geology, professor, Head of Laboratory, Institute of Informatics Problems, 
Federal Research Center ``Computer Science and Control'' of the Russian 
Academy of Sciences, 44-2~Vavilov Str,  Moscow 119333, Russian Federation; 
\mbox{rust@ipiran.ru}


%\vspace*{8pt}

%\hrule

%\vspace*{2pt}

%\hrule

\newpage

\vspace*{-24pt}



\def\tit{ИНФОРМАТИКА И~ЕЕ РОЛЬ В~ПОЗНАНИИ 
ОБРАЗОВАНИЯ И~СВОЙСТВ СЛОЖНОЙ ПРИРОДНОЙ 
СИСТЕМЫ}

\def\aut{Р.\,Б.~Сейфуль-Мулюков}


\def\titkol{Информатика и~ее роль в~познании 
образования и~свойств сложной природной 
системы}

\def\autkol{Р.\,Б.~Сейфуль-Мулюков}

%{\renewcommand{\thefootnote}{\fnsymbol{footnote}}
%\footnotetext[1]{Работа проводится при финансовой поддержке Программы
%стратегического развития Петрозаводского государственного университета в~рамках
%на\-уч\-но-ис\-сле\-до\-ва\-тель\-ской деятельности.}}


\titel{\tit}{\aut}{\autkol}{\titkol}

\vspace*{-12pt}

\noindent
Институт проблем информатики Федерального исследовательского центра <<Информатика 
и~управление>>
Российской академии наук

\vspace*{6pt}

\def\leftfootline{\small{\textbf{\thepage}
\hfill ИНФОРМАТИКА И ЕЁ ПРИМЕНЕНИЯ\ \ \ том\ 11\ \ \ выпуск\ 1\ \ \ 2017}
}%
 \def\rightfootline{\small{ИНФОРМАТИКА И ЕЁ ПРИМЕНЕНИЯ\ \ \ том\ 11\ \ \ выпуск\ 1\ \ \ 2017
\hfill \textbf{\thepage}}}



\Abst{Рассматривается история познания феномена <<информации>> 
и~информатики как междисциплинарной науки, изучающей качественные 
и~количественные особенности ее практических приложений. 
Представляется логическая связь таких широко распространенных понятий, 
как информация, информатика, сложность, сложные природные 
самоорганизующиеся системы. Принимается во внимание, что информация 
кроме традиционного, общепринятого значения является одним из свойств 
материи. Информатика наряду с~другими особенностями является 
инструментом познания развития и~строения сложных природных 
самоорганизующихся систем. В~качестве примера такой системы выбрана 
нефть. Доказывается, что нефть обладает корпускулярными свойствами 
и~каждая молекула углеводорода имеет объем информации в~битах. 
Предлагается новый подход к~поиску месторождений нефти, основанный на 
том факте, что ее образование~--- это дискретный процесс. Соответственно, 
обнаружение его результата является раскрытием статической 
и~динамической неопределенности. Рассматриваются методы и~технологии 
их раскрытия.}

\KW{информатика; информатизация; природные сложные системы; 
образование нефти; поиски нефти; статическая неопределенность; 
динамическая неопределенность} 
   
\DOI{10.14357/19922264160111} 

%\vspace*{18pt}


 \begin{multicols}{2}

\renewcommand{\bibname}{\protect\rmfamily Литература}
%\renewcommand{\bibname}{\large\protect\rm References}

{\small\frenchspacing
{%\baselineskip=10.8pt
\begin{thebibliography}{99}
\bibitem{1-seif-1}
\Au{Колин К.\,К.} Философские проблемы информатики.~--- М.: БИНОМ, 
2010. 259~с.
\bibitem{2-seif-1}
\Au{Shannon C.\,E.} A~mathematical theory of communication~// Bell Syst. 
Tech.~J., 1948. Vol.~27. P.~379--423, 623--656.
\bibitem{3-seif-1}
\Au{Урсул А.\,Д.} Природа информации: Философский очерк.~--- 2-е изд.~--- 
Челябинск: ЧГАКИ, 2010. 231~с.
\bibitem{4-seif-1}
\Au{De Broiglie L.} Wave mechanics and the atomic structure of matter and 
radiation~// J.~Phys. Radium, 1927. Vol.~8. No.\,5. P.~225--241.
\bibitem{5-seif-1}
\Au{Zeilinger A.} A~foundation principal for quantum mechanics~// Found. 
Phys., 1999. Vol.~29. No.\,4. P.~631--643.
\bibitem{6-seif-1}
\Au{Wiener N.} Cybernetics, or control and communication in the animal and the 
machine.~--- 2nd rev. ed.~--- Cambridge: MIT Press, 1961. 232~p.
\bibitem{7-seif-1}
\Au{Steinbuch K.} Informatik~// Automatische Informationsverarbeitung,  
SEG-Nachrichten (Technische Mitteilunger der Standard Elektrik Gruppe), 1957. 
No.\,4. 171~p.
\bibitem{8-seif-1}
\Au{Михайлов А.\,И., Черный~А.\,И., Гиляревский~Р.\,С.} Основы 
информатики.~--- М.: Наука, 1968. 425~с.
\bibitem{9-seif-1}
\Au{Otten K., Debons~A.} Towards a~metascience of information: 
Informatology~// J.~Am. Soc. Inform. Sci., 1970. Vol.~21.  
P.~89--94.
\bibitem{10-seif-1}
\Au{Norton M.\,J.} Introductory concepts in information science.~--- 2nd ed.~--- 
ASIST monograph ser.~--- Information Today, 2010. 210~p.
\bibitem{11-seif-1}
\Au{Гуревич И.\,М.} Законы информатики~--- основа строения и познания 
сложных систем.~--- М.: ТОРУС ПРЕСС, 2007. 399~с.
\bibitem{12-seif-1}
\Au{Heylighen F.} Сomplexity and self-organization~// Encyclopedia of library 
and information sciences~/ Eds. M.\,J.~Bates, M.\,N.~Maack.~--- Taylor \& 
Frances, 2008. 20~p. {\sf http://pespmc1.vub.ac.be/Papers/ELIS-complexity.pdf}.
\bibitem{13-seif-1}
\Au{Сейфуль-Мулюков Р.\,Б.} Нефть и газ. Глубинная природа и~ее 
прикладное значение.~--- М: ТОРУС ПРЕСС, 2012. 214~с.
\bibitem{14-seif-1}
\Au{Сейфуль-Мулюков Р.\,Б.} Нефть~--- углеводородные последовательности: 
анализ моделей генезиса и эволюции.~--- М.: 11~формат, 2010. 173~с.
\bibitem{15-seif-1}
\Au{Заде Л.} Понятие лингвистической переменной и~его применение 
к~принятию приближенных решений.~--- М.: Мир, 1976. 176~с.

\end{thebibliography}
} }

\end{multicols}

 \label{end\stat}

 \vspace*{-3pt}

\hfill{\small\textit{Поступила в~редакцию  11.10.2016}}
%\renewcommand{\bibname}{\protect\rm Литература}
\renewcommand{\figurename}{\protect\bf Рис.}
\renewcommand{\tablename}{\protect\bf Таблица}   %11



%%%%%%%%%%%%%%%%%%%%%%%%%%%%%%%%%%%%%%%%%%%%%%%

%\def\stat{rez}
{%\hrule\par
%\vskip 7pt % 7pt
\raggedleft\Large \bf%\baselineskip=3.2ex
Р\,Е\,Ц\,Е\,Н\,З\,И\,И \vskip 17pt
    \hrule
    \par
\vskip 6pt plus 6pt minus 3pt }

%\thispagestyle{headings} %с верхним колонтитулом
%\thispagestyle{myheadings} %с нижним колонтитулом, но в верхнем РЕЦЕНЗИИ

\def\tit{НОВАЯ КНИГА И.\,Н.~СИНИЦЫНА, А.\,С.~ШАЛАМОВА <<ЛЕКЦИИ ПО ТЕОРИИ 
ИНТЕГРИРОВАННОЙ ЛОГИСТИЧЕСКОЙ ПОДДЕРЖКИ>> (М.: ТОРУС ПРЕСС, 2012. 624~с.)}

%1
\def\aut{Д.ф.-м.н., профессор С.\,Я.~Шоргин}

\def\auf{\ }

\def\leftkol{\ % РЕЦЕНЗИИ
}

\def\rightkol{ \ } 

%\def\leftkol{\ } % ENGLISH ABSTRACTS}

%\def\rightkol{\ } %ENGLISH ABSTRACTS}

%\def\leftkol{РЕЦЕНЗИИ}

%\def\rightkol{РЕЦЕНЗИИ}

\titele{\tit}{\aut}{\auf}{\leftkol}{\rightkol}
\vspace*{-18pt}


     \label{st\stat}

     \begin{multicols}{2}
     {\small
     {\baselineskip=10.1pt
     

      В книге представлено системное изложение теоретических основ одного из новейших 
направлений в \mbox{об\-ласти} экономики послепродажного обслуживания изделий наукоемкой 
продукции (ИНП) длительного пользования~--- интегрированной логистической поддержки
(ИЛП). 
{\looseness=1

}

Приведены также результаты новых работ, выполненных в Институте проблем информатики 
Российской академии наук в рамках научного направления <<Информационные технологии и 
анализ сложных сис\-тем>>.
 {%\looseness=1

}
     
      Излагаемые в книге научные подходы позво\-ляют карди\-наль\-но реформировать 
существующие системы производства и эксплуатации ИНП путем создания и внед\-ре\-ния 
методов рационального и оптимального управ\-ле\-ния процессами расходования 
вре\-мен\-н$\acute{\mbox{ы}}$х, 
мате\-ри\-аль\-ных, трудовых и других ресурсов на всех стадиях жизненного цикла изделий (ЖЦИ) по 
критериям экономической целесообразности и эф\-фек\-тив\-ности.
  {\looseness=1

}
    
      В книге приведен краткий обзор причин возник\-новения и
      развития CALS-методологии как основы 
современных международных стандартов по созданию и функционированию глобальных 
ин\-фор\-ма\-ци\-он\-но-ком\-му\-ни\-ка\-ци\-он\-ных систем, ее ключевых возможностей и эффективности 
результатов ее использования. 
Авторы %\linebreak 
предлагают ряд научных обоснований для разработки 
единой теории проектирования и управления систем ИЛП для полноценного использования 
преимуществ %\linebreak
 суще\-ст\-ву\-ющей методологии, определяют \mbox{общую} структурную схему 
комплексной системы <<ИНП-СППО>> и необходимость разработки для ее описания 
гибридных стохастических моделей.
{%\looseness=1

}

%\columnbreak
      
      Книга состоит из пяти частей, где последовательно излагается материал по каждой из 
следующих тем: <<Интегрированная логистическая поддержка>>, <<Теория гибридных 
стохастических систем и компьютерная поддержка исследований и разработок>>, <<Основы 
математического моделирования, анализа и синтеза систем послепродажного обслуживания>>, 
<<Определение и анализ показателей экспортного потенциала ИНП при проектировании>>, 
<<Задачи управления поддержкой послепродажного обслуживания>>, а также 
<<Моделирование инвестиционных процессов ИЛП в условиях неравновесных финансовых 
рынков>>. 
   
      В конце каждой главы приведены выводы и даны вопросы и задания для 
самоконтроля. В~приложениях содержатся основные определения по программам работ по 
анализу ИЛП, логистическим базам данных и компьютерным решениям, эквивалентной статистической 
линеаризации нелинейных преобразований ИЛП, справочный материал, а также развернутые 
уравнения для вероятностных характеристик.


      \def\leftkol{РЕЦЕНЗИИ}

\def\rightkol{РЕЦЕНЗИИ} 

      
      Книга заинтересует широкий круг специалистов и может быть использована научными 
проектными организациями в сфере промышленного производства ИНП. Большое количество 
иллюстраций, примеров и вопросов, обращенных к читателю, позволяет использовать книгу 
также в качестве учебного пособия для студентов и аспирантов машиностроительных, 
транспортных и~других специальностей, а также для самостоятельного изучения. 
{%\looseness=-1

}

Книга 
представляет несомненный интерес для специалистов и студентов в области прикладной 
математики и информатики.
    

}

}
\end{multicols}

%\newpage

\def\stat{authorsrus}
{%\hrule\par
%\vskip 7pt % 7pt
\raggedleft\Large \bf%\baselineskip=3.2ex
О\,Б\ \ А\,В\,Т\,О\,Р\,А\,Х \vskip 17pt
    \hrule
    \par
\vskip 21pt plus 8pt minus 4pt }


\def\tit{\ }

\def\aut{\ }

\def\auf{\ }

\def\leftkol{\ } % ENGLISH ABSTRACTS}

\def\rightkol{ОБ АВТОРАХ} %ENGLISH ABSTRACTS}

\titele{\tit}{\aut}{\auf}{\leftkol}{\rightkol}
      
            \label{st\stat}



\vspace*{24pt}

\begin{multicols}{2}




\noindent
\textbf{Архипов Олег Петрович} (р.\ 1948)~---
кандидат технических наук, директор Орловского филиала Института проб\-лем информатики
Российской академии наук
%302025, г.Орел, Московское шоссе, д.137

\vspace*{3pt}

\noindent
\textbf{Бирюкова Татьяна Константиновна} (р.\ 1968)~---
кандидат фи\-зи\-ко-ма\-те\-ма\-ти\-че\-ских наук, старший научный сотрудник Института проб\-лем информатики
Российской академии наук

\vspace*{3pt}

\noindent 
\textbf{Бобков  Сергей Геннадьевич} (р.\ 1955)~---
доктор технических наук,  заведующий отделением На\-уч\-но-ис\-сле\-до\-ва\-тель\-ско\-го 
института системных исследований Российской академии наук
%117218, Москва, Нахимовский просп., 36, к.1 

\vspace*{3pt}

\noindent \textbf{Васильев Николай Семенович} (р.\ 1952)~--- доктор 
фи\-зи\-ко-ма\-те\-ма\-ти\-че\-ских наук, профессор, 
МГТУ им.\ Н.\,Э.~Баумана 
%, Москва 105005, 2-я Бауманская ул., д.~5,

\vspace*{3pt}

\noindent
\textbf{Гершкович Максим Михайлович} (р.\ 1968)~---
старший научный сотрудник Института проб\-лем информатики
Российской академии наук

\vspace*{3pt}

\noindent 
\textbf{Дьяченко Юрий Георгиевич} (р.\ 1958)~--- кандидат технических наук, 
старший научный сотрудник Института проб\-лем информатики
Российской академии наук

\vspace*{3pt}

\noindent 
\textbf{Ерошенко Александр Андреевич} (р.\ 1989)~--- аспирант кафедры 
математической статистики факультета вычисли\-тельной математики и кибернетики 
Московского государственного университета им.\ М.\,В.~Ломоносова
%119991, Москва ГСП-1, Ленинские горы, д.\ 1, стр. 52

\vspace*{3pt}
 
\noindent 
\textbf{Захаров Виктор Николаевич} (р.\ 1948)~--- 
доктор технических наук, доцент, ученый секретарь Института проб\-лем информатики
Российской академии наук

\vspace*{3pt}

\noindent
\textbf{Зейфман Александр Израилевич} (р.\ 1954)~---
доктор фи\-зи\-ко-ма\-те\-ма\-ти\-че\-ских наук, профессор, 
заведующий кафедрой Вологодского государственного университета; 
старший научный сотрудник Института проб\-лем информатики
Российской академии наук; главный научный сотрудник ИСЭРТ Российской академии наук

\vspace*{3pt}

\noindent
\textbf{Зыкин Сергей Владимирович} (р.\ 1959)~--- 
доктор технических наук, профессор, заведующий лабораторией Института математики 
им.\ С.\,Л.~Соболева Сибирского отделения Российской академии наук, Новосибирск 
%630090, пр.\ ак.\ Коптюга, 4 

\vspace*{4pt}

\noindent
\textbf{Киреев Владимир Иванович} (р.\ 1938)~---
доктор фи\-зи\-ко-ма\-те\-ма\-ти\-че\-ских наук, профессор Московского 
государственного горного университета
%Адрес: Россия, 119991, г. Москва, Ленинский проспект, д. 6

%\columnbreak

\vspace*{4pt}

\noindent
\textbf{Козеренко Елена Борисовна} (р.\ 1959)~---
кандидат филологических наук, заведующая лабораторией Института проб\-лем информатики
Российской академии наук

\vspace*{4pt}

\noindent
\textbf{Королев Виктор Юрьевич} (р.\ 1954)~--- доктор
фи\-зи\-ко-ма\-те\-ма\-ти\-че\-ских наук, профессор кафедры математической 
статистики факультета вычисли\-тельной математики и кибернетики 
Московского государственного университета; 
ведущий научный сотрудник Института проб\-лем информатики
Российской академии наук

\vspace*{4pt}

\noindent
\textbf{Коротышева Анна Владимировна} (р.\ 1988)~---
старший преподаватель Вологодского государственного университета

\vspace*{4pt}

\noindent 
\textbf{Кун Де Турк} (р.\ 1981)~--- научный сотрудник 
исследовательской группы SMACS факультета телекоммуникаций и обработки информации
Университета Гента, Бельгия
%В-9000 Гент, Бельгия

\vspace*{4pt}

\noindent
\textbf{Лупенцов Олег Сергеевич} (р.\ 1986)~---
аспирант Омского государственного института сервиса
%Омск 644043, ул.\ Певцова 13

\vspace*{4pt}

\noindent
\textbf{Лучко Олег Николаевич} (р.\ 1961)~---
кандидат педагогических наук, профессор, заведующий кафедрой 
Омского государственного института сервиса
%Омск 644043, ул.\ Певцова 13

\vspace*{4pt}

\noindent
\textbf{Малашенко Юрий Евгеньевич} (р.\ 1946)~---
доктор фи\-зи\-ко-ма\-те\-ма\-ти\-че\-ских наук, заведующий сектором 
Вычислительного центра им.\ А.\,А.~Дородницына Российской академии наук
%Адрес: 119333, Москва, ул. Вавилова, 40,

\vspace*{4pt}

\noindent
\textbf{Маньяков Юрий Анатольевич} (р.\ 1984)~---
кандидат технических наук, научный сотрудник Орловского филиала Института проб\-лем информатики
Российской академии наук
%302025, г.Орел, Московское шоссе, д.137

\vspace*{4pt}

\noindent
\textbf{Маренко Валентина Афанасьевна} (р.\ 1951)~---
кандидат технических наук, доцент, старший научный сотрудник 
Института математики им.\ С.\,Л.~Соболева Сибирского отделения Российской академии наук
%Новосибирск 630090, пр. ак. Коптюга, 4 

\vspace*{3pt}

\noindent 
\textbf{Морозов Евсей Викторович} (р.\ 1947)~--- доктор 
фи\-зи\-ко-ма\-те\-ма\-ти\-че\-ских, профессор, ведущий научный сотрудник 
Института прикладных математических исследований Карельского научного центра Российской
академии наук; 
%%185910 Россия, Республика Карелия, г.\ Петрозаводск, ул.\ Пушкинская, 11
профессор Петрозаводского государственного университета, Петрозаводск
%185910 Россия, Республика Карелия, г.\ Петрозаводск, пр.\ Ленина, 33

%\pagebreak

\vspace*{3pt}

\noindent
\textbf{Назарова Ирина Александровна} (р.\ 1966)~---
кандидат фи\-зи\-ко-ма\-те\-ма\-ти\-че\-ских наук, 
научный сотрудник Вычислительного центра им.\ А.\,А.~Дородницына Российской академии наук 
%Адрес: 119333, Москва, ул. Вавилова, 40

\vspace*{3pt}

\noindent
\textbf{Павлов Игорь Валерианович} (р.\ 1945)~--- 
доктор фи\-зи\-ко-ма\-те\-ма\-ти\-че\-ских наук, профессор МГТУ им.\ Н.\,Э.~Баумана 
%Москва 105005, 2-я Бауманская ул., д.~5 

%\pagebreak

\vspace*{3pt}

\noindent 
\textbf{Потахина Любовь Викторовна} (р.\ 1989)~--- аспирантка
Института прикладных математических исследований Карельского научного центра
Российской академии наук; 
%%185910 Россия, Республика Карелия, г.\ Петрозаводск, ул.\ Пушкинская, 11
инженер Петрозаводского государственного университета, Петрозаводск
%185910 Россия, Республика Карелия, г.\ Петрозаводск, пр.\ Ленина, 33

\vspace*{3pt}

\noindent 
\textbf{Рождественский Юрий Владимирович} (р.\ 1952)~--- 
кандидат технических наук, заведующий сектором Института проб\-лем информатики
Российской академии наук

\vspace*{3pt}

\noindent 
\textbf{Синицын Игорь Николаевич} (р.\ 1940)~--- доктор технических наук,
профессор, заслуженный деятель\linebreak\vspace*{-12pt}

\columnbreak

\noindent
 науки РФ, заведующий отделом Института проб\-лем информатики
Российской академии наук

\vspace*{7pt}


\noindent
\textbf{Сиротинин Денис Олегович} (р.\ 1984)~---
кандидат технических наук, научный сотрудник Орловского филиала Института проб\-лем информатики
Российской академии наук
%302025, г.Орел, Московское шоссе, д.137

\vspace*{7pt}

%\columnbreak

\noindent 
\textbf{Соколов  Игорь Анатольевич} (р.\ 1954)~--- академик (действительный член) Российской 
академии наук, доктор технических наук, директор Института проб\-лем информатики
Российской академии наук

\vspace*{7pt}

\noindent
\textbf{Степченков Юрий Афанасьевич} (р.\ 1951)~---
кандидат технических наук, заведующий отделом Института проб\-лем информатики
Российской академии наук

\vspace*{7pt}

\noindent
\textbf{Сурков Алексей Викторович} (р.\ 1978)~--- 
старший научный сотрудник На\-уч\-но-ис\-сле\-до\-ва\-тель\-ско\-го 
института системных исследований Российской академии наук
%117218, Москва, Нахимовский просп., 36, к.1 

\vspace*{7pt}

\noindent 
\textbf{Шестаков Олег Владимирович} (р.\ 1976)~--- доктор 
фи\-зи\-ко-ма\-те\-ма\-ти\-че\-ских, доцент кафедры математической статистики 
факультета вычисли\-тельной математики и кибернетики Московского 
государственного университета им.\ М.\,В.~Ломоносова; 
%119991, Москва ГСП-1, Ленинские горы, д.\ 1, стр. 52
старший научный сотрудник Института проб\-лем информатики
Российской академии наук
%, Москва 119333, ул. Вавилова, д.~44, корп.~2

\vspace*{7pt}

\noindent 
\textbf{Шоргин Сергей Яковлевич} (р.\ 1952.)~--- доктор
фи\-зи\-ко-ма\-те\-ма\-ти\-че\-ских наук, профессор, заместитель директора Института 
проб\-лем информатики Российской академии наук





%%%%%%%%%%%%%%%%%%%%%%%%%%%%%%%%%%%%%%%%%%%%%%%%%%%%%%%%%%%%%%%%%%%%%%%%%%%%%%%




%\def\rightkol{ОБ АВТОРАХ}
%\def\leftkol{ОБ АВТОРАХ}

 \label{end\stat}





%\def\leftfootline{\small{\textbf{\thepage}
%\hfill ИНФОРМАТИКА И ЕЁ ПРИМЕНЕНИЯ\ \ \ том~7\ \ \ выпуск~1\ \ \ 2013}
%}%
% \def\rightfootline{\small{ИНФОРМАТИКА И ЕЁ ПРИМЕНЕНИЯ\ \ \ том~7\ \ \ выпуск~1\ \ \ 2013
%\hfill \textbf{\thepage}}}


%\thispagestyle{myheadings}



\end{multicols}

\newpage

%\end{document}

%
\def\stat{rekl}
%\label{preobr}

%\def\tit{АКАДЕМИК ПУГАЧЁВ  ВЛАДИМИР СЕМЁНОВИЧ\\
%25.03.1911--25.03.1998}


%   \vspace*{-48pt}
%   \begin{center}\LARGE
%Академик Пугачёв  Владимир Семёнович\\ (25.03.1911--25.03.1998)
%   \end{center}

   %\vspace*{2.5mm}

   \begin{center}

{\prgsh\LARGE
ЮБИЛЕИ}

\end{center}
%\hrule

\vspace*{6pt}


   \vspace*{8mm}

   \thispagestyle{empty}


%\def\stat{emel}


\section*{К 70-летию заместителя директора ИПИ РАН,\\ члена редколлегии журнала
<<Информатика и её применения>>\\ доктора технических наук В.\,И.~Будзко}

\vspace*{18pt}




          \begin{multicols}{2}

%            \label{st\stat}

\begin{center}
\vspace*{1pt}
\mbox{%
\epsfxsize=78mm
\epsfbox{bud-1.eps}
}
\end{center}

\vspace*{12pt}

      14 августа 2014~г.\ исполнилось 70~лет за\-мес\-ти\-те\-лю директора ИПИ РАН по
научной работе доктору технических наук Владимиру Игоревичу Будзко.

      Владимир Игоревич Будзко родился в г.~Москве. Высшее образование получил на факультете
элект\-рон\-но-вы\-чис\-ли\-тель\-ных устройств в Московском
ин\-же\-нер\-но-фи\-зи\-че\-ском институте
(МИФИ), который он окончил в 1968~г., после чего был на\-прав\-лен для прохождения
службы в одну из войс\-ко\-вых частей, где прошел путь от инженера до первого заместителя
командира войсковой части.

      С приходом В.\,И.~Будзко в ИПИ РАН (2001~г.)\ в институте
сформировалось новое научное на\-прав\-ле\-ние теоретических исследований~--- <<Постро\-ение
ин\-фор\-ма\-ци\-он\-но-те\-ле\-ком\-му\-ни\-ка\-ци\-он\-ных\linebreak сис\-тем
высокой до\-ступ\-ности>>. В~рамках этого
направления выполнен широкий круг фундаментальных исследований по поиску подходов и
определению принципов построения средств обеспечения доступности, конфиденциальности
и целостности современных крупномасштабных
ин\-фор\-ма\-ци\-он\-но-те\-ле\-ком\-му\-ни\-ка\-ци\-он\-ных
сис\-тем (ИТС). Разработаны основные сис\-тем\-но-тех\-ни\-че\-ские принципы и базовые
архитектурные решения построения перспективных для условий России ИТС с
централизованной обработкой и хранением информации, сочетающих в себе свойства
высокой доступности, отказо- и катастрофоустойчивости, информационной защищенности.
Определены принципы, методы и математические основы рационального построения и
оптимизации средств восстановления функционирования центров обработки данных (ЦОД)
после возникновения отказов и катастроф, передачи и хранения данных, обеспечения
информационной безопасности при достижении минимальной совокупной стоимости
владения такими системами. Результаты нашли практическое воплощение при реализации
проектов в интересах ряда отечественных государственных и негосударственных
организаций, таких как Банк России (БР), Внешторгбанк, ОАО <<ГМК <<Норильский Никель>>,
<<Газпром>>, Минэкономразвития России, Правительство Москвы, а также ряд силовых
ведомств.

      Под руководством В.\,И.~Будзко начиная с 2001~г.\ выполнен комплекс
      на\-уч\-но-ис\-сле\-до\-ва\-тель\-ских и
      опыт\-но-кон\-ст\-рук\-тор\-ских работ (свыше 100~проектов),
направленных на развитие электронной информационной технологии БР.
Разработаны концепции развития ИТС БР сначала до 2008~г., а затем до 2013~г., которые
были приняты в качестве основы проведения технической политики. За реализацию проекта
<<Катастрофоустойчивая тер\-ри\-то\-ри\-аль\-но-рас\-пре\-де\-лен\-ная
      ин\-фор\-ма\-ци\-он\-но-те\-ле\-ком\-му\-ни\-ка\-ци\-он\-ная сис\-те\-ма централизованной
обработки банковской информации>> В.\,И.~Будзко удостоен Премии Правительства РФ в
области науки и техники за 2010~г.

      В.\,И.~Будзко возглавлял и возглавляет работы по ряду других прикладных проектов,
связанных с созданием, совершенствованием и развитием крупномасштабных ИТС.

      В.\,И.~Будзко~--- генерал-майор, доктор технических наук, член-кор\-рес\-пон\-дент
Академии криптографии РФ, известный ученый в области информатики и применения
информационных технологий при построении территориально распределенных ИТС
различного назначения. Является автором свыше 250~научных работ, опубликованных в
на\-уч\-но-тех\-ни\-че\-ских и специальных изданиях.

    \thispagestyle{empty}

      В.\,И.~Будзко уделяет большое внимание подготовке научных кадров. Под его
руководством защищено 6~диссертаций на соискание ученой степени кандидата
технических наук. Свыше 30~лет он читает лекции в ИКСИ Академии ФСБ, профессор
кафедры НИЯУ МИФИ. Является членом двух диссертационных советов, главным
редактором журнала <<Системы высокой доступности>> и членом редколлегии журнала
<<Информатика и её применения>>.

      \bigskip

      Редакционный совет и Редакционная коллегия журнала <<Информатика и её
применения>> сердечно поздравляют Владимира Игоревича Будзко с 70-ле\-ти\-ем и желают
крепкого здоровья и новых научных достижений.

\end{multicols}

\def\stat{cont}
{%\hrule\par
%\vskip 7pt % 7pt
\raggedleft\Large \bf%\baselineskip=3.2ex
А\,В\,Т\,О\,Р\,С\,К\,И\,Й\ \ У\,К\,А\,З\,А\,Т\,Е\,Л\,Ь\ \ З\,А\ \ 2\,0\,1\,0 г. \vskip 17pt
    \hrule
    \par
\vskip 21pt plus 6pt minus 3pt }

\label{st\stat}

\def\tit{\ }

\def\aut{\ }
\def\auf{\ }

\def\leftkol{\ } % ENGLISH ABSTRACTS}

\def\rightkol{\ } %АВТОРСКИЙ УКАЗАТЕЛЬ ЗА 2010 г.} %ENGLISH ABSTRACTS}

\titele{\tit}{\aut}{\auf}{\leftkol}{\rightkol}

\vspace*{-12pt}

{\tabcolsep=3pt
\begin{tabular}{p{388pt}rr}
&\textbf{Выпуск} & \textbf{Стр.}\\[6pt]
\hangindent=23pt\noindent\textbf{Арутюнян~А.\,Р.} Моделирование влияния деформаций отпечатков пальцев на 
точность\linebreak
\vspace*{-12pt}\\
\hspace*{23pt}дактилоскопической идентификации$\dotfill$&1&51\\
\hangindent=23pt\noindent\textbf{Архипов~О.\,П., Зыкова~З.\,П.} Интеграция гетерогенной информации о цветных 
пикселях\linebreak
\vspace*{-12pt}\\
\hspace*{23pt}и их цветовосприятии$\dotfill$&4&15\\
\hangindent=23pt\noindent\textbf{Баранов~С.\,И., Френкель~С.\,Л., Захаров~В.\,Н.} Полуформальная верификация 
цифрового устройства с конвейером, основанная на использовании алгоритмических машин\linebreak
\vspace*{-12pt}\\
\hspace*{23pt}состояния$\dotfill$&4&49\\
\textbf{Бекетова~И.\,В.} см.~Каратеев~С.\,Л.&&\\
\textbf{Белоусов~В.\,В.} см.~Синицын~И.\,Н.&&\\
\hangindent=23pt\noindent\textbf{Бенинг~В.\,Е., Королев~Р.\,А.} О предельном поведении мощностей критериев в 
случае\linebreak
\vspace*{-12pt}\\
\hspace*{23pt}распределения Лапласа$\dotfill$&2&63\\
\hangindent=23pt\noindent\textbf{Бенинг~В.\,Е., Сипина~А.\,В.} Асимптотическое разложение для мощности 
критерия,\linebreak
\vspace*{-12pt}\\
\hspace*{23pt}основанного на выборочной медиане, в случае распределения Лапласа$\dotfill$&1&18\\
\textbf{Бондаренко~А.\,В.} см.~Каратеев~С.\,Л.&&\\
\hangindent=23pt\noindent\textbf{Бородина~А.\,В., Морозов~Е.\,В.} Об оценивании асимптотики вероятности 
большого\linebreak
\vspace*{-12pt}\\
\hspace*{23pt}уклонения стационарной регенеративной очереди с одним прибором$\dotfill$&3&29\\
\hangindent=23pt\noindent\textbf{Бунтман~Н.\,В., Минель~Ж.-Л., Ле~Пезан~Д., Зацман~И.\,М.} Типология и 
компьютерное\linebreak
\vspace*{-12pt}\\
\hspace*{23pt}моделирование трудностей перевода$\dotfill$&3&77\\
\textbf{Визильтер~Ю.\,В.} см.~Каратеев~С.\,Л.&&\\
\hangindent=23pt\noindent\textbf{Гавриленко~С.\,В.} Оценки скорости сходимости распределений случайных сумм с 
безгранично делимыми индексами к нормальному закону$\dotfill$&4&81\\
\hangindent=23pt\noindent\textbf{Григорьева~М.\,Е., Шевцова~И.\,Г.} Уточнение неравенства 
Каца--Берри--Эссеена$\dotfill$&2&75\\
\hangindent=23pt\noindent\textbf{Грушо~А.\,А., Грушо~Н.\,А., Тимонина~Е.\,Е.} Поиск конфликтов в политиках 
безопасности: модель случайных графов$\dotfill$&3&38\\
\textbf{Грушо~Н.\,А.} см.~Грушо~А.\,А.&&\\
\hangindent=23pt\noindent\textbf{Гудков~В.\,Ю.} Математические модели изображения отпечатка пальца на основе 
описания линий$\dotfill$&1&58\\
\textbf{Гуртов~А.\,В.} см.~Лукьяненко~А.\,С.&&\\
\textbf{Желтов~С.\,Ю.} см.~Каратеев~С.\,Л.&&\\
\hangindent=23pt\noindent\textbf{Захаров~А.\,А., Серебряков~В.\,А.} Система управления электронной библиотекой 
LibMeta$\dotfill$&4&2\\
\textbf{Захаров~В.\,Н.} см.~Баранов~С.\,И.&&\\
\textbf{Захарова~Т.\,В.} см.~Матвеева~С.\,С.&&\\
\hangindent=23pt\noindent\textbf{Зацаринный~А.\,А., Чупраков~К.\,Г.} Некоторые аспекты выбора технологии для 
постро-\linebreak
\vspace*{-12pt}\\
\hspace*{23pt}ения систем отображения информации ситуационного центра$\dotfill$&3&59\\
\textbf{Зацман~И.\,М.} см.~Бунтман~Н.\,В.&&\\
\hangindent=23pt\noindent\textbf{Зейфман~А.\,И., Коротышева~А.\,В., Сатин~Я.\,А., Шоргин~С.\,Я.} Об 
устойчивости нестаци-\linebreak
\vspace*{-12pt}\\
\hspace*{23pt}онарных систем обслуживания с катастрофами$\dotfill$&3&9\\
\textbf{Зыкова~З.\,П.} см.~Архипов~О.\,П.&&\\
\hangindent=23pt\noindent\textbf{Илюшин~Г.\,Я., Соколов~И.\,А.} Организация управляемого доступа пользователей 
к\linebreak
\vspace*{-12pt}\\
\hspace*{23pt}разнородным ведомственным информационным ресурсам$\dotfill$&1&24\\
\hangindent=23pt\noindent\textbf{Кавагучи~Ю., Ульянов~В.\,В., Фуджикоши~Я.} Приближения для статистик, 
описывающих\linebreak
\vspace*{-12pt}\\
\hspace*{23pt}геометрические свойства данных большой размерности, с оценками 
ошибок$\dotfill$&1&12\\
\hangindent=23pt\noindent\textbf{Каратеев~С.\,Л., Бекетова~И.\,В., Ососков~М.\,В., Князь~В.\,А., 
Визильтер~Ю.\,В., Бондаренко~А.\,В., Желтов~С.\,Ю.} Автоматизированный контроль 
качества цифровых\linebreak
\vspace*{-12pt}\\
\hspace*{23pt}изображений для персональных документов$\dotfill$&1&65\\
\end{tabular}
}

\pagebreak

\def\leftkol{АВТОРСКИЙ УКАЗАТЕЛЬ ЗА 2010 г.} % ENGLISH ABSTRACTS}

\def\rightkol{АВТОРСКИЙ УКАЗАТЕЛЬ ЗА 2010 г.} %ENGLISH ABSTRACTS}

{\tabcolsep=3pt
\begin{tabular}{p{388pt}rr}
&\textbf{Выпуск} & \textbf{Стр.}\\[3pt]
\hangindent=23pt\noindent\textbf{Козеренко~Е.\,Б.} Лингвистические фильтры в статистических моделях машинного\linebreak
\vspace*{-12pt}\\
\hspace*{23pt}перевода$\dotfill$&2&83\\
\hangindent=23pt\noindent\textbf{Козеренко~Е.\,Б., Кузнецов~И.\,П.} Когнитивно-лингвистические представления в 
систе-\linebreak
\vspace*{-12pt}\\
\hspace*{23pt}мах обработки текстов$\dotfill$&3&69\\
\textbf{Князь~В.\,А.} см.~Каратеев~С.\,Л.&&\\
\hangindent=23pt\noindent\textbf{Колесников~А.\,В., Солдатов~С.\,А.} Алгоритм координации для гибридной 
интеллектуальной системы решения сложной задачи оперативно-производственного\linebreak
\vspace*{-12pt}\\
\hspace*{23pt}планирования$\dotfill$&4&61\\
\hangindent=23pt\noindent\textbf{Коновалов~М.\,Г.} О планировании потоков в системах вычислительных 
ресурсов$\dotfill$&2&3\\
\textbf{Конушин~А.\,С.} см.~Конушин~В.\,С.&&\\
\hangindent=23pt\noindent\textbf{Конушин~В.\,С., Кривовязь~Г.\,Р., Конушин~А.\,С.} Алгоритм распознавания людей 
в видео-\linebreak
\vspace*{-12pt}\\
\hspace*{23pt}последовательности по одежде$\dotfill$&1&74\\
\textbf{Корепанов~Э.\, Р.} см.~Синицын~И.\,Н.&&\\
\textbf{Королев~В.\,Ю.} см.~Соколов~И.\,А.&&\\
\textbf{Королев~Р.\,А.} см.~Бенинг~В.\,Е.&&\\
\textbf{Коротышева~А.\,В.} см.~Зейфман~А.\,И.&&\\
\hangindent=23pt\noindent\textbf{Кривенко~М.\,П.} Непараметрическое оценивание элементов байесовского 
клас\-си-\linebreak
\vspace*{-12pt}\\
\hspace*{23pt}фикатора$\dotfill$&2&13\\
\textbf{Кривовязь~Г.\,Р.} см.~Конушин~В.\,С.&&\\
\textbf{Крылов~А.\,С.} см.~Павельева~Е.\,А.&&\\
\hangindent=23pt\noindent\textbf{Крылов~В.\,А.} Моделирование и классификация многоканальных дистанционных\linebreak
\vspace*{-12pt}\\
\hspace*{23pt}изображений с использованием копул$\dotfill$&4&34\\
\hangindent=23pt\noindent\textbf{Крючин~О.\,В.} Разработка параллельных эвристических алгоритмов подбора 
весовых\linebreak
\vspace*{-12pt}\\
\hspace*{23pt}коэффициентов искусственной нейтронной сети$\dotfill$&2&53\\
\hangindent=23pt\noindent\textbf{Кудрявцев~А.\,А., Шоргин~С.\,Я.} Байесовские модели массового обслуживания и 
надеж-\linebreak
\vspace*{-12pt}\\
\hspace*{23pt}ности: характеристики среднего числа заявок в системе $M\vert M \vert 1\vert 
\infty$$\dotfill$&3&16\\
\hangindent=23pt\noindent\textbf{Кузнецов~А.\,А.} Связь между временными и структурно-топологическими 
характери-\linebreak
\vspace*{-12pt}\\
\hspace*{23pt}стиками диаграмм ритма сердца здоровых людей$\dotfill$&4&39\\
\textbf{Кузнецов~И.\,П.} см.~Козеренко~Е.\,Б.&&\\
\textbf{Ле~Пезан~Д.} см.~Бунтман~Н.\,В.&&\\
\hangindent=23pt\noindent\textbf{Лукьяненко~А.\,С., Морозов~Е.\,В., Гуртов~А.\,В.} Анализ сетевого протокола с общей 
функ-\linebreak
\vspace*{-12pt}\\
\hspace*{23pt}цией расширения окна передачи сообщения при конфликтах$\dotfill$&2&46\\
\hangindent=23pt\noindent\textbf{Лямин~О.\,О.} О предельном поведении мощностей критериев в случае обобщенного\linebreak
\vspace*{-12pt}\\
\hspace*{23pt}распределения Лапласа$\dotfill$&3&47\\
\hangindent=23pt\noindent\textbf{Маркин~А.\,В., Шестаков~О.\,В.} Асимптотики оценки риска при пороговой 
обработке\linebreak
\vspace*{-12pt}\\
\hspace*{23pt}вейвлет-вейглет коэффициентов в задаче томографии$\dotfill$&2&36\\
\hangindent=23pt\noindent\textbf{Матвеева~С.\,С., Захарова~Т.\,В.} Сети массового обслуживания с наименьшей 
длиной\linebreak
\vspace*{-12pt}\\
\hspace*{23pt}очереди$\dotfill$&3&22\\
\hangindent=23pt\noindent\textbf{Матюшенко~С.\,И.} Стационарные характеристики двухканальной системы 
обслужива-\linebreak
\vspace*{-12pt}\\
\hspace*{23pt}ния с переупорядочиванием заявок и распределениями фазового типа$\dotfill$&4&68\\
\textbf{Минель~Ж.-Л.} см.~Бунтман~Н.\,В.&&\\
\textbf{Морозов~Е.\,В.} см.~Бородина~А.\,В.&&\\
\textbf{Морозов~Е.\,В.} см.~Лукьяненко~А.\,С.&&\\
\textbf{Ососков~М.\,В.} см.~Каратеев~С.\,Л.&&\\
\hangindent=23pt\noindent\textbf{Павельева~Е.\,А., Крылов~А.\,С.} Поиск и анализ ключевых точек радужной 
оболочки\linebreak
\vspace*{-12pt}\\
\hspace*{23pt}глаза методом преобразования Эрмита$\dotfill$&1&79\\
\textbf{Печинкин~А.\,В.} см.~Френкель~С.\,Л.,&&\\
\hangindent=23pt\noindent\textbf{Протасов~В.\,И.} Составление субъективного портрета с использованием 
эволюционно-\linebreak
\vspace*{-12pt}\\
\hspace*{23pt}го морфинга и квалиметрия метода$\dotfill$&1&83\\
\hangindent=23pt\noindent\textbf{Рудаков~К.\,В., Торшин~И.\,Ю.} Вопросы разрешимости задачи распознавания 
вторичной\linebreak
\vspace*{-12pt}\\
\hspace*{23pt}структуры белка$\dotfill$&2&25\\
\textbf{Сатин~Я.\,А.} см.~Зейфман~А.\,И.&&\\
\hangindent=23pt\noindent\textbf{Сейфуль-Мулюков~Р.\,Б.} Нефть как носитель информации о своем 
происхождении,\linebreak
\vspace*{-12pt}\\
\hspace*{23pt}структуре и эволюции$\dotfill$&1&41\\
\end{tabular}
}

{\tabcolsep=3pt
\begin{tabular}{p{388pt}rr}
&\textbf{Выпуск} & \textbf{Стр.}\\[6pt]
\textbf{Семендяев~Н.\,Н.} см.~Синицын~И.\,Н.&&\\
\textbf{Серебряков~В.\,А.} см.~Захаров~А.\,А.&&\\
\textbf{Синицын~В.\,И.} см.~Синицын~И.\,Н.&&\\
\hangindent=23pt\noindent\textbf{Синицын~И.\,Н., Синицын~В.\,И., Корепанов~Э.\, Р., Белоусов~В.\,В., 
Семендяев~Н.\,Н.} Оперативное построение информационных моделей движения полюса 
Земли\linebreak
\vspace*{-12pt}\\
\hspace*{23pt}методами линейных и линеаризованных фильтров$\dotfill$&1&2\\
\textbf{Сипина~А.\,В.} см.~Бенинг~В.\,Е.&&\\
\hangindent=23pt\noindent\textbf{Соколов~И.\,А.} О работах заслуженного деятеля науки Российской Федерации 
И.\,Н.~Синицына в области информационных технологий и автоматизации (к 70-летию\linebreak
\vspace*{-12pt}\\
\hspace*{23pt}со дня рождения)$\dotfill$&3&84\\
\textbf{Соколов~И.\,А.} см.~Илюшин~Г.\,Я.&&\\
\hangindent=23pt\noindent\textbf{Соколов~И.\,А., Королев~В.\,Ю.} Предисловие$\dotfill$&2&2\\
\textbf{Солдатов~С.\,А.} см.~Колесников~А.\,В.&&\\
\hangindent=23pt\noindent\textbf{Степанов~С.\,Ю.} Использование координатного метода фрагментации 
коммутаторной\linebreak
\vspace*{-12pt}\\
\hspace*{23pt}нейронной сети для сокращения трафика$\dotfill$&2&57\\
\textbf{Тимонина~Е.\,Е.} см.~Грушо~А.\,А.&&\\
\textbf{Торшин~И.\,Ю.} см.~Рудаков~К.\,В.&&\\
\textbf{Ульянов~В.\,В.} см.~Кавагучи~Ю.&&\\
\textbf{Фазекаш~И.} см.~Чупрунов~А.\,Н.&&\\
\textbf{Френкель~С.\,Л.} см.~Баранов~С.\,И.&&\\
\hangindent=23pt\noindent\textbf{Френкель~С.\,Л., Печинкин~А.\,В.} Оценка времени самовосстановления в 
цифровых\linebreak
\vspace*{-12pt}\\
\hspace*{23pt}системах после сбоев, вызываемых переходными помехами$\dotfill$&3&2\\
\textbf{Фуджикоши~Я.} см.~Кавагучи~Ю.&&\\
\hangindent=23pt\noindent\textbf{Цискаридзе~А.\,К.} Математическая модель и метод восстановления позы человека 
по\linebreak
\vspace*{-12pt}\\
\hspace*{23pt}стереопаре силуэтных изображений$\dotfill$&4&27\\
\hangindent=23pt\noindent\textbf{Чупраков~К.\,Г.} К вопросу о размещении коллективных средств отображения в 
ситуа-\linebreak
\vspace*{-12pt}\\
\hspace*{23pt}ционном зале с заданными параметрами$\dotfill$&4&89\\
\textbf{Чупраков~К.\,Г.} см.~Зацаринный~А.\,А.&&\\
\hangindent=23pt\noindent\textbf{Чупрунов~А.\,Н., Фазекаш~И.} Законы повторного логарифма для числа 
безошибочных\linebreak
\vspace*{-12pt}\\
\hspace*{23pt}блоков при помехоустойчивом кодировании$\dotfill$&3&42\\
\textbf{Шевцова~И.\,Г.} см.~Григорьева~М.\,Е.&&\\
\hangindent=23pt\noindent\textbf{Шестаков~О.\,В.} Аппроксимация распределения оценки риска пороговой 
обработки вейвлет-коэффициентов нормальным распределением при использовании 
выбо-\linebreak
\vspace*{-12pt}\\
\hspace*{23pt}рочной дисперсии$\dotfill$&4&73\\
\textbf{Шестаков~О.\,В.} см.~Маркин~А.\,В.&&\\
\textbf{Шоргин~С.\,Я.} см.~Зейфман~А.\,И.&&\\
\textbf{Шоргин~С.\,Я.} см.~Кудрявцев~А.\,А.&&\\
\end{tabular}
}

%\thispagestyle{myheadings}
\def\leftfootline{\small{\textbf{\thepage}
\hfill ИНФОРМАТИКА И ЕЁ ПРИМЕНЕНИЯ\ \ \ том~4\ \ \ выпуск~4\ \ \ 2010}
}%
 \def\rightfootline{\small{ИНФОРМАТИКА И ЕЁ ПРИМЕНЕНИЯ\ \ \ том~4\ \ \ выпуск~4\ \ \ 2010
 \hfill \textbf{\thepage}}}
 \label{end\stat}





%Том 10 Выпуск 1-4 Год 2016

\def\stat{cont-e}
{%\hrule\par
%\vskip 7pt % 7pt
\raggedleft\Large \bf%\baselineskip=3.2ex
2\,0\,1\,6\ \ A\,U\,T\,H\,O\,R\ \ I\,N\,D\,E\,X \vskip 17pt
 \hrule
 \par
\vskip 21pt plus 6pt minus 3pt }

\label{st\stat}

\def\tit{\ }

\def\aut{\ }
\def\auf{\ }

\def\leftkol{\ } %2016 AUTHOR INDEX} % ENGLISH ABSTRACTS}

\def\rightkol{\ } %2016 AUTHOR INDEX} %ENGLISH ABSTRACTS}

\titele{\tit}{\aut}{\auf}{\leftkol}{\rightkol}

\def\leftfootline{\small{\textbf{\thepage}
\hfill INFORMATIKA I EE PRIMENENIYA~--- INFORMATICS AND APPLICATIONS\ \ \ 2016\
\ \ volume~10\ \ \ issue\ 4}
}%
 \def\rightfootline{\small{INFORMATIKA I EE PRIMENENIYA~--- INFORMATICS AND APPLICATIONS\ \ \ 2016\ \ \ volume~10\ \ \ issue\ 4
\hfill \textbf{\thepage}}}

\vspace*{-12pt}
\vspace*{-18pt}

{\tabcolsep=2.8pt
\begin{tabular}{p{382pt}cc}
&\textbf{Issue} & \textbf{Page}\\[6pt]
\Avtors{Agalarov~M.\,Ya.} see~Agalarov~Ya.\,M.&&\\
\Avtors{Agalarov~Ya.\,M., Agalarov~M.\,Ya., and
Shorgin~V.\,S.} About the optimal threshold of queue\linebreak
\\[-12pt]
\hspace*{23pt}length in a~particular problem of profit maximization
in the $M/G/1$ queuing system&2&70--79\\
\Avtors{Alexeyevsky~D.\,A.} BioNLP ontology extraction from 
a~restricted language corpus with\linebreak
\\[-12pt]
\hspace*{23pt}context-free grammars&1&119--128\\
\Avtors{Andreev~S.\,D.} see~Gaidamaka~Yu.\,V.&&\\
\Avtors{Andreev~S.\,D.} see~Ometov~A.\,Ya.&&\\
\Avtors{Arkhipov~O.\,P., Arkhipov~P.\,O., and Sidorkin~I.\,I.} The
option to create a~local coordinate\linebreak
\\[-12pt]
\hspace*{23pt}system for synchronization of selected images&3&91--97\\
\Avtors{Arkhipov~P.\,O.} see~Arkhipov~O.\,P.&&\\
\Avtors{Belousov~V.\,V.} see~Shnurkov~P.\,V.&&\\
\Avtors{Belousov~V.\,V.} see~Shnurkov~P.\,V.&&\\
\Avtors{Bening~V.\,E.} Calculation of~the~asymptotic deficiency
of~some statistical procedures based\linebreak
\\[-12pt]
\hspace*{23pt}on~samples with~random sizes&4&34--45\\
\Avtors{Borisov~A.\,V., Bosov~A.\,V., and Miller~G.\,B.} Modeling and
monitoring of VoIP connection&2&\hphantom{1}2--13\\
\Avtors{Bosov~A.\,V.} see~Borisov~A.\,V.&&\\
\Avtors{Briukhov~D.\,O.} see~Stupnikov~S.\,A.&&\\
\Avtors{Callaos~N.\,K.\ and Seyful-Mulyukov~R.\,B.} Complexity and
its information content&1&129--139\\
\Avtors{Chertok~A.\,V., Kadaner~A.\,I., Khazeeva~G.\,T., and
Sokolov~I.\,A.} Regime switching detection\linebreak
\\[-12pt]
\hspace*{23pt}for~the~Levy driven
Ornstein--Uhlenbeck process using CUSUM methods&4&46--56\\
\Avtors{Chichagov~V.\,V.} Asymptotic expansions of mean absolute
error of uniformly minimum variance unbiased and maximum likelihood
estimators on the one-parameter exponential\linebreak
\\[-12pt]
\hspace*{23pt}family model of lattice distributions&3&66--76\\
\Avtors{Danishevsky~V.\,I.} see~Kolesnikov A.\,V.&&\\
\Avtors{Fazliev~A.\,Z.} see~Kalinichenko~L.\,A.&&\\
\Avtors{Fedoseev~A.\,A.} What is behind the concept of ``knowledge in
small packages''&3&105--110\\
\Avtors{Gaidamaka~Yu.\,V., Andreev~S.\,D., Sopin~E.\,S.,
Samouylov~K.\,E., and Shorgin~S.\,Ya.} Interference analysis
of~the~device-to-device communications model with~regard to~a~signal\linebreak
\\[-12pt]
\hspace*{23pt}propagation environment&4&\hphantom{1}2--10\\
\Avtors{Gasilov~A.\,V.} see~Yakovlev~O.\,A.&&\\
\Avtors{Goncharov~A.\,V.\ and Strijov~V.\,V.} Metric time series
classification using weighted dynamic\linebreak
\\[-12pt]
\hspace*{23pt}warping relative to centroids of classes&2&36--47\\
\Avtors{Gordov~E.\,P.} see~Kalinichenko~L.\,A.&&\\
\Avtors{Gorshenin~A.\,K.} Concept of online service for stochastic
modeling of real processes&1&72--81\\
\Avtors{Gorshenin~A.\,K.} see~Shnurkov~P.\,V.&&\\
\Avtors{Gorshenin~A.\,K.} see~Shnurkov~P.\,V.&&\\
\Avtors{Grusho~A.\,A., Grusho~N.\,A., Zabezhailo~M.\,I., and
Timonina~E.\,E.} Integration of statistical and\linebreak
\\[-12pt]
\hspace*{23pt}deterministic methods for
analysis of information security&3&2--8\\
\Avtors{Grusho~A.\,A., Zabezhailo~M.\,I., and Zatsarinny~A.\,A.} On
the advanced procedure to reduce\linebreak
\\[-12pt]
\hspace*{23pt}calculation of Galois closures&4&\hphantom{1}96--104\\
\Avtors{Grusho~N.\,A.} see~Grusho~A.\,A.&&\\
\Avtors{Havanskov~V.\,A.} see~Minin~V.\,A.&&\\
\Avtors{Inkova~O.\,Yu.} see~Zatsman~I.\,M.&&\\
\Avtors{Isachenko~R.\,V.\ and Strijov~V.\,V.} Metric learning in
multiclass time series classification\linebreak
\\[-12pt]
\hspace*{23pt}problem&2&48--57\\
\end{tabular}
}
\pagebreak

\def\leftfootline{\small{\textbf{\thepage}
\hfill INFORMATIKA I EE PRIMENENIYA~--- INFORMATICS AND APPLICATIONS\ \ \ 2016\
\ \ volume~10\ \ \ issue\ 4}
}%
 \def\rightfootline{\small{INFORMATIKA I EE PRIMENENIYA~---
INFORMATICS AND APPLICATIONS\ \ \ 2016\ \ \ volume~10\ \ \ issue\ 4
\hfill \textbf{\thepage}}}

\def\leftkol{2016 AUTHOR INDEX} % ENGLISH ABSTRACTS}

\def\rightkol{2016 AUTHOR INDEX} %ENGLISH ABSTRACTS}


{\tabcolsep=2.83pt
\begin{tabular}{p{382pt}cc}
&\textbf{Issue} & \textbf{Page}\\[6pt]
\Avtors{Kadaner~A.\,I.} see~Chertok~A.\,V.&&\\[.255pt]
\Avtors{Kalinichenko~L.\,A., Volnova~A.\,A., Gordov~E.\,P.,
Kiselyova~N.\,N., Kovaleva~D.\,A., Malkov~O.\,Yu., Okladnikov~I.\,G.,
Podkolodnyy~N.\,L., Pozanenko~A.\,S., Ponomareva~N.\,V.,
Stupnikov~S.\,A.,} \textbf{and Fazliev~A.\,Z.} Data access challenges for data
intensive\linebreak
\\[-12pt]
\hspace*{23pt}research in Russia&1& 2--22\\[.255pt]
\Avtors{Karasikov~M.\,E.\ and Strijov~V.\,V.} Feature-based
time-series classification&4&121--131\\[.255pt]
\Avtors{Khazeeva~G.\,T.} see~Chertok~A.\,V.&&\\[.255pt]
\Avtors{Khokhlov~Yu.\,S.} Multivariate fractional Levy motion and its
applications&2&\hphantom{1}98--106\\[.255pt]
\Avtors{Kirikov~I.\,A., Kolesnikov~A.\,V., Listopad~S.\,V., and
Rumovskaya~S.\,B.} Fine-grained hybrid\linebreak
\\[-12pt]
\hspace*{23pt}intelligent systems. Part 2:
Bidirectional hybridization&1&\hphantom{1}96--105\\[.255pt]
\Avtors{Kirikov~I.\,A., Kolesnikov~A.\,V., Listopad~S.\,V., and
Rumovskaya~S.\,B.} ``Virtual council''~---\linebreak
\\[-12pt]
\hspace*{23pt}source environment
supporting complex diagnostic decision making&3&81--90\\[.255pt]
\Avtors{Kiselyova~N.\,N.} see~Kalinichenko~L.\,A.&&\\[.255pt]
\Avtors{Kolesnikov A.\,V., Listopad~S.\,V., Rumovskaya~S.\,B., and
Danishevsky~V.\,I.} Informal axiomatic\linebreak
\\[-12pt]
\hspace*{23pt}theory of~the~role visual models&4&114--120\\[.255pt]
\Avtors{Kolesnikov~A.\,V.} see~Kirikov~I.\,A.&&\\[.255pt]
\Avtors{Kolesnikov~A.\,V.} see~Kirikov~I.\,A.&&\\[.255pt]
\Avtors{Kolin~K.\,K.} Humanitarian aspects of information
security&3&111--121\\[.255pt]
\Avtors{Konovalov~M.\,G.\ and Razumchik~R.\,V.} Dispatching
to~two parallel nonobservable queues using\linebreak
\\[-12pt]
\hspace*{23pt}only static
information&4&57--67\\[.255pt]
\Avtors{Korchagin~A.\,Yu.} see~Korolev~V.\,Yu.&&\\[.255pt]
\Avtors{Korchagin~A.\,Yu.} see~Korolev~V.\,Yu.&&\\[.255pt]
\Avtors{Korepanov~E.\,R.} see~Sinitsyn~I.\,N.&&\\[.255pt]
\Avtors{Korepanov~E.\,R.} see~Sinitsyn~I.\,N.&&\\[.255pt]
\Avtors{Korolev~V.\,Yu., Korchagin~A.\,Yu., and Zeifman~A.\,I.} The
Poisson theorem for Bernoulli trials\linebreak
\\[-12pt]
\hspace*{23pt}with~a~random probability
of~success and~a~discrete analog of~the~Weibull distribution&4&11--20\\[.255pt]
\Avtors{Korolev~V.\,Yu., Zeifman~A.\,I., and Korchagin~A.\,Yu.}
Asymmetric Linnik distributions as~limit\linebreak
\\[-12pt]
\hspace*{23pt}laws for~random sums
of~independent random variables with~finite variances&4&21--33\\[.255pt]
\Avtors{Koucheryavy~E.\,A.} see~Ometov~A.\,Ya.&&\\[.255pt]
\Avtors{Kovaleva~D.\,A.} see~Kalinichenko~L.\,A.&&\\[.255pt]
\Avtors{Kovalyov~S.\,P.} Metaprogramming to increase
manufacturability of large-scale software-\linebreak
\\[-12pt]
\hspace*{23pt}intensive systems&1&56--66\\[.255pt]
\Avtors{Krivenko~M.\,P.} Significance tests of feature selection for
classification&3&32--40\\[.255pt]
\Avtors{Kruzhkov~M.\,G.} see~Zalizniak~Anna~A.&&\\[.255pt]
\Avtors{Kruzhkov~M.\,G.} see~Zatsman~I.\,M.&&\\[.255pt]
\Avtors{Kudryavtsev~A.\,A.} Bayesian queueing and reliability models:
\textit{A~priori} distributions with\linebreak
\\[-12pt]
\hspace*{23pt}compact support&1&67--71\\[.255pt]
\Avtors{Kudryavtsev~A.\,A.} Characteristics dependent on the balance
coefficient in Bayesian models\linebreak
\\[-12pt]
\hspace*{23pt}with compact support of \textit{a priori}
distributions&3&77--80\\[.255pt]
\Avtors{Kudryavtsev~A.\,A.\ and Palionnaia~S.\,I.} Bayesian recurrent
model of reliability growth:\linebreak
\\[-12pt]
\hspace*{23pt}Parabolic distribution of parameters&2&80--83\\[.255pt]
\Avtors{Kudryavtsev~A.\,A.\ and Titova~A.\,I.} Bayesian queuing
and~reliability models: Degenerate-\linebreak
\\[-12pt]
\hspace*{23pt}Weibull case&4&68--71\\[.255pt]
\Avtors{Leontyev~N.\,D.\ and Ushakov~V.\,G.} Analysis of a queueing
system with autoregressive arrivals\linebreak
\\[-12pt]
\hspace*{23pt}and nonpreemptive priority&3&15--22\\[.255pt]
\Avtors{Listopad~S.\,V.} see~Kirikov~I.\,A.&&\\[.255pt]
\Avtors{Listopad~S.\,V.} see~Kirikov~I.\,A.&&\\[.255pt]
\Avtors{Listopad~S.\,V.} see~Kolesnikov A.\,V.&&\\[.255pt]
\Avtors{Malkov~O.\,Yu.} see~Kalinichenko~L.\,A.&&\\[.255pt]
\Avtors{Markov~A.\,S., Monakhov~M.\,M., and
Ulyanov~V.\,V.} Generalized Cornish--Fisher expansions\linebreak
\\[-12pt]
\hspace*{23pt}for distributions of statistics based on samples
of random size&2&84--91\\[.255pt]
\Avtors{Melnikov~A.\,K.\ and Ronzhin~A.\,F.} Generalized statistical
method of~text analysis based\linebreak
\\[-12pt]
\hspace*{23pt}on~calculation of~probability distributions
of~statistical values&4&89--95\\
\end{tabular}
}
\pagebreak

\def\leftfootline{\small{\textbf{\thepage}
\hfill INFORMATIKA I EE PRIMENENIYA~--- INFORMATICS AND APPLICATIONS\ \ \ 2016\
\ \ volume~10\ \ \ issue\ 4}
}%
 \def\rightfootline{\small{INFORMATIKA I EE PRIMENENIYA~---
INFORMATICS AND APPLICATIONS\ \ \ 2016\ \ \ volume~10\ \ \ issue\ 4
\hfill \textbf{\thepage}}}

\def\leftkol{2016 AUTHOR INDEX} % ENGLISH ABSTRACTS}

\def\rightkol{2016 AUTHOR INDEX} %ENGLISH ABSTRACTS}


{\tabcolsep=3pt
\begin{tabular}{p{381pt}cc}
&\textbf{Issue} & \textbf{Page}\\[6pt]
\Avtors{Meykhanadzhyan~L.\,A.} Stationary characteristics of the finite
capacity queueing system with\linebreak
\\[-12pt]
\hspace*{23pt}inverse service order and generalized
probabilistic priority&2&123--131\\[.23pt]
\Avtors{Miller~G.\,B.} see~Borisov~A.\,V.&&\\[.23pt]
\Avtors{Minin~V.\,A., Zatsman~I.\,M., Havanskov~V.\,A., and
Shubnikov~S.\,K.} Intensity of citation of scientific publications in
inventions on information and computer technologies patented\linebreak
\\[-12pt]
\hspace*{23pt}in Russia by domestic and foreign applicants&2&107--122\\[.23pt]
\Avtors{Monakhov~M.\,M.} see~Markov~A.\,S.&&\\[.23pt]
\Avtors{Naumov~V.\,A.\ and Samouylov~K.\,E.} On relationship
between queuing systems with resources\linebreak
\\[-12pt]
\hspace*{23pt}and Erlang networks&3&\hphantom{1}9--14\\[.23pt]
\Avtors{Okladnikov~I.\,G.} see~Kalinichenko~L.\,A.&&\\[.23pt]
\Avtors{Ometov~A.\,Ya., Andreev~S.\,D., Turlikov~A.\,M., and
Koucheryavy~E.\,A.} Performance analysis of\linebreak
\\[-12pt]
\hspace*{23pt}a wireless data
aggregation system with contention for contemporary sensor
networks&3&23--31\\[.23pt]
\Avtors{Palionnaia~S.\,I.} see~Kudryavtsev~A.\,A.&&\\[.23pt]
\Avtors{Podkolodnyy~N.\,L.} see~Kalinichenko~L.\,A.&&\\[.23pt]
\Avtors{Ponomareva~N.\,V.} see~Kalinichenko~L.\,A.&&\\[.23pt]
\Avtors{Popkova~N.\,A.} see~Zatsman~I.\,M.&&\\[.23pt]
\Avtors{Pozanenko~A.\,S.} see~Kalinichenko~L.\,A.&&\\[.23pt]
\Avtors{Razumchik~R.\,V.} see~Konovalov~M.\,G.&&\\[.23pt]
\Avtors{Ronzhin~A.\,F.} see~Melnikov~A.\,K.&&\\[.23pt]
\Avtors{Rumovskaya~S.\,B.} see~Kirikov~I.\,A.&&\\[.23pt]
\Avtors{Rumovskaya~S.\,B.} see~Kirikov~I.\,A.&&\\[.23pt]
\Avtors{Rumovskaya~S.\,B.} see~Kolesnikov A.\,V.&&\\[.23pt]
\Avtors{Samouylov~K.\,E.} see~Gaidamaka~Yu.\,V.&&\\[.23pt]
\Avtors{Samouylov~K.\,E.} see~Naumov~V.\,A.&&\\[.23pt]
\Avtors{Serebryanskii~S.\,M.} see~Tyrsin~A.\,N.&&\\[.23pt]
\Avtors{Seyful-Mulyukov~R.\,B.} see~Callaos~N.\,K.&&\\[.23pt]
\Avtors{Shestakov~O.\,V.} Statistical properties of the denoising method
based on the stabilized hard\linebreak
\\[-12pt]
\hspace*{23pt}thresholding&2&65--69\\[.23pt]
\Avtors{Shestakov~O.\,V.} The strong law of large numbers for the risk
estimate in the problem of\linebreak
\\[-12pt]
\hspace*{23pt}tomographic image reconstruction from
projections with a correlated noise&3&41--45\\[.23pt]
\Avtors{Shestakov~O.\,V.} see~Zakharova~T.\,V.&&\\[.23pt]
\Avtors{Shnurkov~P.\,V., Gorshenin~A.\,K., and Belousov~V.\,V.}
Analytical solution of~the~optimal control\linebreak
\\[-12pt]
\hspace*{23pt}task of~a~semi-Markov
process with~finite set of~states&4&72--88\\[.23pt]
\Avtors{Shnurkov~P.\,V., Zasypko~V.\,V., Belousov~V.\,V., and
Gorshenin~A.\,K.} Development of the algorithm of numerical solution
of the optimal investment control problem\linebreak
\\[-12pt]
\hspace*{23pt}in the closed dynamical model of three-sector economy&1&82--95\\[.23pt]
\Avtors{Shorgin~S.\,Ya.} see~Gaidamaka~Yu.\,V.&&\\[.23pt]
\Avtors{Shorgin~V.\,S.} see~Agalarov~Ya.\,M.&&\\[.23pt]
\Avtors{Shubnikov~S.\,K.} see~Minin~V.\,A.&&\\[.23pt]
\Avtors{Sidorkin~I.\,I.} see~Arkhipov~O.\,P.&&\\[.23pt]
\Avtors{Sinitsyn~I.\,N.} Analytical modeling of processes in stochastic
systems with complex fractional\linebreak
\\[-12pt]
\hspace*{23pt}order Bessel nonlinearities&3&55--65\\[.23pt]
\Avtors{Sinitsyn~I.\,N.} Orthogonal supoptimal filters for nonlinear
stochastic systems on manifolds&1&34--44\\[.23pt]
\Avtors{Sinitsyn~I.\,N.\ and Korepanov~E.\,R.} Normal Pugachev
conditionally-optimal filters and extra-\linebreak
\\[-12pt]
\hspace*{23pt}polators for state linear stochastic systems&2&14--23\\[.23pt]
\Avtors{Sinitsyn~I.\,N.\ and Sinitsyn~V.\,I.} Analytical modeling of
distributions in stochastic systems on\linebreak
\\[-12pt]
\hspace*{23pt}manifolds based on ellipsoidal approximation&1&45--55\\[.23pt]
\Avtors{Sinitsyn~I.\,N., Sinitsyn~V.\,I., and
Korepanov~E.\,R.} Ellipsoidal suboptimal filters for nonlinear\linebreak
\\[-12pt]
\hspace*{23pt}stochastic systems on manifolds&2&24--35\\[.23pt]
\Avtors{Sinitsyn~V.\,I.} see~Sinitsyn~I.\,N.&&\\[.23pt]
\Avtors{Sinitsyn~V.\,I.} see~Sinitsyn~I.\,N.&&\\[.23pt]
\Avtors{Skvortsov~N.\,A.} see~Stupnikov~S.\,A.&&\\[.23pt]
\Avtors{Sokolov~I.\,A.} see~Chertok~A.\,V.&&\\
\end{tabular}
}
\pagebreak

\def\leftfootline{\small{\textbf{\thepage}
\hfill INFORMATIKA I EE PRIMENENIYA~--- INFORMATICS AND APPLICATIONS\ \ \ 2016\
\ \ volume~10\ \ \ issue\ 4}
}%
 \def\rightfootline{\small{INFORMATIKA I EE PRIMENENIYA~---
INFORMATICS AND APPLICATIONS\ \ \ 2016\ \ \ volume~10\ \ \ issue\ 4
\hfill \textbf{\thepage}}}

\def\leftkol{2016 AUTHOR INDEX} % ENGLISH ABSTRACTS}

\def\rightkol{2016 AUTHOR INDEX} %ENGLISH ABSTRACTS}


{\tabcolsep=3pt
\begin{tabular}{p{382pt}cc}
&\textbf{Issue} & \textbf{Page}\\[6pt]
\Avtors{Sopin~E.\,S.} see~Gaidamaka~Yu.\,V.&&\\
\Avtors{Strijov~V.\,V.} see~Goncharov~A.\,V.&&\\
\Avtors{Strijov~V.\,V.} see~Isachenko~R.\,V.&&\\
\Avtors{Strijov~V.\,V.} see~Karasikov~M.\,E.&&\\
\Avtors{Stupnikov~S.\,A., Briukhov~D.\,O., and Skvortsov~N.\,A.}
Co-lending systemic risk analysis over\linebreak
\\[-12pt]
\hspace*{23pt}heterogeneous data collections&1&23--33\\
\Avtors{Stupnikov~S.\,A.} see~Kalinichenko~L.\,A.&&\\
\Avtors{Suchkov~A.\,P.} see~Zatsarinny~A.\,A.&&\\
\Avtors{Timonina~E.\,E.} see~Grusho~A.\,A.&&\\
\Avtors{Titova~A.\,I.} see~Kudryavtsev~A.\,A.&&\\
\Avtors{Turlikov~A.\,M.} see~Ometov~A.\,Ya.&&\\
\Avtors{Tyrsin~A.\,N.\ and Serebryanskii~S.\,M.} Recognition of
dependences on the basis of inverse\linebreak
\\[-12pt]
\hspace*{23pt}mapping&2&58--64\\
\Avtors{Ulyanov~V.\,V.} see~Markov~A.\,S.&&\\
\Avtors{Ushakov~V.\,G.} Queueing system with working vacations and
hyperexponential input stream&2&92--97\\
\Avtors{Ushakov~V.\,G.} see~Leontyev~N.\,D.&&\\
\Avtors{Volnova~A.\,A.} see~Kalinichenko~L.\,A.&&\\
\Avtors{Yakovlev~O.\,A.\ and Gasilov~A.\,V.} Speeded-up stereo
matching using geodesic support weights&3&\hphantom{1}98--104\\
\Avtors{Zabezhailo~M.\,I.} see~Grusho~A.\,A.&&\\
\Avtors{Zabezhailo~M.\,I.} see~Grusho~A.\,A.&&\\
\Avtors{Zakharova~T.\,V.\ and Shestakov~O.\,V.} Precision analysis of
wavelet processing of aerodynamic\linebreak
\\[-12pt]
\hspace*{23pt}flow patterns&3&46--54\\
\Avtors{Zalizniak~Anna~A.\ and Kruzhkov~M.\,G.} Database
of~Russian impersonal verbal constructions&4&132--141\\
\Avtors{Zasypko~V.\,V.} see~Shnurkov~P.\,V.&&\\
\Avtors{Zatsarinny~A.\,A.\ and Suchkov~A.\,P.} Systems engineering
approaches to~the~establishment of\linebreak
\\[-12pt]
\hspace*{23pt}a~system for~decision support based
on~situational analysis&4&105--113\\
\Avtors{Zatsarinny~A.\,A.} see~Grusho~A.\,A.&&\\
\Avtors{Zatsman~I.\,M., Inkova~O.\,Yu., Kruzhkov~M.\,G., and
Popkova~N.\,A.} Representation of cross-\linebreak
\\[-12pt]
\hspace*{23pt}lingual knowledge about
connectors in supracorpora databases&1&106--118\\
\Avtors{Zatsman~I.\,M.} see~Minin~V.\,A.&&\\
\Avtors{Zeifman~A.\,I.} see~Korolev~V.\,Yu.&&\\
\Avtors{Zeifman~A.\,I.} see~Korolev~V.\,Yu.&&\\
\end{tabular}
}

%\thispagestyle{myheadings}
\def\leftfootline{\small{\textbf{\thepage}
\hfill INFORMATIKA I EE PRIMENENIYA~--- INFORMATICS AND APPLICATIONS\ \ \ 2016\
\ \ volume~10\ \ \ issue\ 4}
}%
 \def\rightfootline{\small{INFORMATIKA I EE PRIMENENIYA~---
INFORMATICS AND APPLICATIONS\ \ \ 2016\ \ \ volume~10\ \ \ issue\ 4
\hfill \textbf{\thepage}}}

 \label{end\stat}

\newpage

%\def\stat{rekl}
%\label{preobr}

%\def\tit{АКАДЕМИК ПУГАЧЁВ  ВЛАДИМИР СЕМЁНОВИЧ\\
%25.03.1911--25.03.1998}


%   \vspace*{-48pt}
%   \begin{center}\LARGE
%Академик Пугачёв  Владимир Семёнович\\ (25.03.1911--25.03.1998)
%   \end{center}
   
   %\vspace*{2.5mm}
   
   \begin{center}

{\prgsh\LARGE
ОБЪЯВЛЕНИЯ О КОНФЕРЕНЦИЯХ}

\end{center}
%\hrule

\vspace*{6pt}

   
   \vspace*{10mm}
   
   \thispagestyle{empty}

\noindent
\begin{tabular}{cc}
%\begin{center}
\multicolumn{1}{c}{\raisebox{-40pt}[0pt][0pt]{\mbox{%
\epsfxsize=33mm
\epsfbox{vspu.eps}
}}}
%\end{center}
&
\tabcolsep=0pt\begin{tabular}{c}
{\prg{\Large\textbf{XII Всероссийское совещание}}}\\[6pt]
{\prg{\Large\textbf{по проблемам управления}}}\\[12pt]
{\prg{\large 16--19 июня 2014~г.}}\\[6pt] 
{\prg{\large Институт проблем управления имени В.\,А.~Трапезникова РАН}}\\[6pt]
{\prg{\large Москва, Россия}}
\end{tabular}
\end{tabular}

\vspace*{60pt}

     
 { %\large    
 XII Всероссийское совещание по проблемам управления (ВСПУ XII), посвященное 75-летию 
Института проблем управления (ИПУ) имени В.\,А.~Трапезникова РАН, проводится 16--19~июня 
2014~г.\ 
в ИПУ РАН (г.~Москва, Россия). ВСПУ XII организуется ИПУ РАН при поддержке РФФИ, Отделения 
энергетики, машиностроения, механики и процессов управления Российской академии наук, 
Российского 
национального комитета по автоматическому управлению, Академии навигации и управ\-ле\-ния 
движением, 
Научного совета РАН по комплексным проблемам управления и автоматизации, Совета по 
мехатронике и робототехнике РАН. Официальный язык Совещания~--- русский.

\vspace*{24pt}
     
     \textbf{Направления работы}
     \begin{enumerate}[1.]
\item Теория систем управления
\item Управление подвижными объектами и навигация
\item Интеллектуальные системы управления
\item Управление в промышленности, транспортом и логистикой
\item Управление системами междисциплинарной природы
\item Средства измерения, вычислений и контроля в управлении
\item Системный анализ и принятие решений в задачах управления
\item Информационные технологии в управлении
\item Проблемы образования в области управления: современное содержание и технологии обучения
\end{enumerate}

\vspace*{24pt}

     Подробная информация о Совещании находится на сайте {\sf http://vspu2014.ipu.ru}. Срок 
окончательной подачи докладов через систему подачи докладов на сайте~--- \textbf{30~ноября} 
2013~г.
}

%\include{rekl-1}

%\end{document}

%   \vspace*{-48pt}

\begin{center}
\vspace*{6pt}
\mbox{%
\epsfxsize=53.502mm
\epsfbox{foto-1.eps}
}
\end{center}

\vspace*{6pt} %Академик


   \begin{center}
\fbox{\Large\textbf{Профессор Игорь Алексеевич Ушаков}}\\[12pt]
\textbf{\large 22.01.1935--27.02.2015}
   \end{center}


   %\vspace*{2.5mm}

   \vspace*{5mm}

   \thispagestyle{empty}

%\

%\vspace*{-12pt}


Редакционный совет и редакционная коллегия журнала <<Информатика и~её применения>> с~глубоким прискорбием извещают, что 27~февраля 2015~г.\ после тяжелой
и~продолжительной болезни скончался Игорь Алексеевич Ушаков~--- доктор технических наук, профессор, член редколлегии журнала <<Информатика и ее применения>>.

Игорь Алексеевич Ушаков окончил Московский авиационный институт, в~1963~г.\ защитил кандидатскую, а~в~1968~г.~--- докторскую диссертацию. С~1958 по 1989~гг.\ работал в~ряде научно-исследовательских организаций СССР, в~том числе руководил отделами в~НИИ АА и~ВЦ АН СССР; с 1969 по 1989 гг. преподавал в~МФТИ (был профессором, а~затем заведующим кафедрой) и~в~МЭИ. С~1989~г.~---- в~США: являлся профессором университета Дж.\ Вашингтона, университета Дж.\ Мэйсона и~Калифорнийского университета, сотрудником компаний MCI, Qualcomm и Hughes.

И.\,А.~Ушаков с момента основания журнала <<Надежность и~контроль качества>> был заместителем ответственного редактора, а~затем на протяжении многих лет членом редколлегии. В~2006~г.\ основал электронный международный журнал ``Reliability: Theory \& Application'', главным редактором которого оставался до конца жизни.

Учебниками и справочниками по теории надежности, написанными И.\,А.~Ушаковым, пользовались и~пользуются несколько поколений ученых и~специалистов в~разных странах мира.

Игорь Алексеевич всегда уделял огромное внимание работе с~молодежью; более~50 его учеников защитили докторские и~кандидатские диссертации.

И.\,А.~Ушаков вел активную научно-про\-све\-ти\-тель\-скую деятельность. В~частности, он был одним из организаторов и~руководителей Московского кабинета качества и~надежности при Политехническом музее (целью этого Кабинета было оказание консультаций работникам промышленных предприятий и~чтение курсов лекций для инженеров, занимающихся проблемой надежности). Находясь в~США, И.\,А.~Ушаков создал международный ин\-тер\-нет-фо\-рум им.\ Б.\,В.~Гнеденко, объединивший около~400~видных специалистов по приложениям теории вероятностей и~математической статистики, преимущественно в~об\-ласти теории надежности и~анализа риска, из десятков стран мира; коллективным членов этого Форума является и~наш журнал. Цели Форума~--- содействие контактам между специалистами из разных стран, организация обмена профессиональными 
новостями и~информацией (новые публикации, предстоящие события и~др.). Также необходимо отметить большое число на\-уч\-но-по\-пу\-ляр\-ных работ, опубликованных И.\,А.~Ушаковым.

И.\,А.~Ушаков обладал большим личным обаянием, имел широкий круг интересов. Все знавшие И.\,А.~Ушакова всегда будут помнить его как замечательного ученого и~прекрасного человека.

\bigskip

Редакционный совет и редакционная коллегия журнала <<Информатика и~её применения>> 
выражают глубокие соболезнования родным и близким покойного, всем, кто его знал и~работал с~ним.



%\end{document}

%\include{IPPM-25}

\def\stat{cont-rus}
{%\hrule\par
%\vskip 7pt % 7pt
\vspace*{-24pt}
\raggedleft\Large \bf%\baselineskip=3.2ex
Правила подготовки рукописей  для публикации в журнале
<<Информатика~и~её~применения>> \vskip 8pt
    \hrule
    \par
\vskip 14pt plus 6pt minus 3pt }

\label{st\stat}

\def\tit{\ }

\def\aut{\ }
\def\auf{\ }

\def\leftkol{\ }
% Правила подготовки рукописей  для публикации в журнале
%<<Информатика и её применения>>

\def\rightkol{\ }
%Правила подготовки рукописей  для публикации в журнале
%<<Информатика и её применения>>}


\titele{\tit}{\aut}{\auf}{\leftkol}{\rightkol}


\vspace*{-60pt}
{ %\small

Журнал <<Информатика и её применения>>
публикует теоретические, обзорные и дискуссионные статьи,
посвященные научным исследованиям и разработкам в области
информатики и ее приложений.

Журнал издается на русском языке. По специальному решению
редколлегии отдельные статьи могут печататься на английском языке.

Тематика журнала охватывает следующие направления:
\begin{itemize}
\item теоретические основы информатики;\\[-15pt]
      \item
математические методы исследования сложных систем и процессов;\\[-15pt]
           \item
информационные системы и сети;\\[-15pt]
                \item
информационные технологии;\\[-15pt]
                     \item
архитектура и программное обеспечение вычислительных комплексов и сетей.\\[-15pt]
\end{itemize}


\noindent
\begin{enumerate}[1.]
\item В журнале печатаются статьи, содержащие результаты, ранее не опубликованные и
не предназначенные к одновременной публикации в других изданиях.

%Публикация не должна нарушать закон об авторских правах.
Публикация предоставленной автором(ами) рукописи не должна нарушать 
положений глав~69, 70 раздела~VII части~IV Гражданского кодекса, 
которые определяют права на результаты интеллектуальной деятельности 
и~средства индивидуализации, в~том числе авторские права, в~РФ.

Ответственность за нарушение авторских прав, в~случае предъявления претензий к~редакции журнала,  
несут авторы статей.



Направляя рукопись в редакцию, авторы сохраняют свои права на данную
рукопись и при этом передают учредителям и редколлегии журнала неисключительные права на
издание статьи на русском языке 
(или на языке статьи, если он отличен от рус\-ско\-го) и~на перевод ее на английский
язык, а~также на
ее распространение в России и за рубежом. 
Каждый автор должен представить в~редакцию подписанный 
с~его стороны <<Лицензионный договор о~передаче неисключительных прав 
на использование произведения>>, текст которого размещен по адресу 
{\sf http://www.ipiran.ru/publications/licence.doc}. 
Этот договор может быть пред\-став\-лен в~бумажном (в~2-х экз.)\ 
или в~электронном виде (отсканированная копия заполненного и~подписанного документа).




Редколлегия вправе запросить у авторов экспертное заключение о возможности
пуб\-ли\-ка\-ции пред\-став\-лен\-ной статьи в открытой печати.\\[-13.5pt]

\item К статье прилагаются данные автора (авторов) (см.\ п.~8). При наличии нескольких
авторов указывается фамилия автора, ответственного за переписку с редакцией.\\[-13.5pt]

\item Редакция журнала осуществляет экспертизу присланных статей в соответствии с
принятой в журнале процедурой рецензирования.

Возвращение рукописи на доработку не означает ее принятия к печати.

Доработанный вариант с ответом на замечания рецензента необходимо прислать в
редакцию.\\[-13.5pt]

\item Решение редколлегии о публикации статьи или ее отклонении сообщается авторам.

Редколлегия может также направить авторам текст рецензии на их статью. Дискуссия по
поводу отклоненных статей не ведется.\\[-13.5pt]

%\pagebreak

\item Редактура статей высылается авторам для просмотра. Замечания к редактуре должны
быть присланы авторами в кратчайшие сроки.\\[-13.5pt]

\item Рукопись предоставляется в электронном виде в форматах MS WORD (.doc или
.docx) или \LaTeX\  (.tex), дополнительно~--- в формате .pdf, на дискете, лазерном диске
или электронной почтой. Предоставление бумажной рукописи необязательно.\\[-13.5pt]

\item При подготовке рукописи в MS Word рекомендуется использовать следующие
настройки.

Параметры страницы:
формат~--- А4; ориентация~--- книжная; поля (см): внутри~--- 2,5, снаружи~--- 1,5,
сверху~--- 2, снизу~--- 2, от края до нижнего колонтитула~--- 1,3.

Основной текст: стиль~--- <<Обычный>>, шрифт~--- Times New Roman, размер~---
14~пунк\-тов, абзацный отступ~--- 0,5~см, 1,5~интервала, выравнивание~--- по ширине.

\pagebreak

\def\leftkol{Правила подготовки рукописей  для публикации в журнале
<<Информатика и её применения>>}

\def\rightkol{Правила подготовки рукописей  для публикации в журнале
<<Информатика и её применения>>}



Рекомендуемый объем рукописи~--- не свыше 10~страниц указанного формата.
При превышении указанного объема редколлегия вправе потребовать от 
автора сокращения объема рукописи.


Сокращения слов, помимо стандартных, не допускаются. Допускается минимальное
количество аббревиатур.


Все страницы рукописи нумеруются.

Шаблоны оформления представлены в интернете:

\noindent
 {\sf
http://www.ipiran.ru/journal/template\_iiep\_ssi\_2024.zip}\\[-14pt]

\item Статья должна содержать следующую информацию на {\bfseries\textit{русском и
английском языках}}:\\[-16pt]

\begin{itemize}
\item название статьи;\\[-15pt]
\item Ф.И.О.\ авторов, на английском можно только имя и фамилию;\\[-15pt]
\item место работы, с указанием почтового адреса организации и электронного адреса каждого
автора;\\[-15pt]
\item сведения об авторах, в соответствии с форматом, образцы которого
представлены на страницах:



\def\leftfootline{\small{\textbf{\thepage}
\hfill ИНФОРМАТИКА И ЕЁ ПРИМЕНЕНИЯ\ \ \ том\ 18\ \ \ выпуск\ 3\ \ \ 2024}
}%
 \def\rightfootline{\small{ИНФОРМАТИКА И ЕЁ ПРИМЕНЕНИЯ\ \ \ том\ 18\ \ \ выпуск\ 3\ \ \ 2024
\hfill \textbf{\thepage}}}



{\sf http://www.ipiran.ru/journal/issues/2013\_07\_01/authors.asp} и

{\sf http://www.ipiran.ru/journal/issues/2013\_07\_01\_eng/authors.asp};
\item аннотация (не менее 100~слов на каждом из языков). Аннотация~--- это краткое
резюме работы, которое может публиковаться отдельно. Она является основным
источником информации в~ин\-фор\-ма\-ци\-он\-ных системах и базах данных. Английская
аннотация должна быть оригинальной, может не быть дословным переводом русского
текста и должна быть написана хорошим английским языком. В~аннотации не должно
быть ссылок на литературу и, по возможности, формул;\\[-15pt]
\item ключевые слова~--- желательно из принятых в мировой
на\-уч\-но-тех\-ни\-че\-ской литературе тематических тезаурусов. Предложения не
могут быть ключевыми словами;\\[-15pt]
\item источники финансирования работы (ссылки на гранты, проекты,
поддерживающие организации и~т.\,п.).
\end{itemize}



%\pagebreak

\item  Требования к спискам литературы.\\[-14pt]

Ссылки на литературу в тексте статьи нумеруются (в квадратных скобках) и
располагаются в каждом из списков литературы в порядке  первых упоминаний. Если источник имеет DOI и/или EDN,
то их необходимо указывать.

Списки литературы представляются в двух вариантах:\\[-14pt]


\noindent
\begin{enumerate}[(1)]
\item \textbf{Список литературы к русскоязычной части}. Русские и английские
работы~---  на языке и в алфавите оригинала;\\[-14.5pt]
\item  \textbf{References}. Русские работы и работы на других языках~--- в латинской
транслитерации с переводом на английский язык; английские работы и работы на других
языках~--- на языке оригинала.
\end{enumerate}

Необходимо для составления списка ``References'' пользоваться размещенной на сайте
{\sf http://www. translit.net/ru/bgn/} бесплатной программой транслитерации русского
 текста в~латиницу. %, при этом в~за\-клад\-ке <<варианты\ldots>> следует выбратьопцию BGN.

Список литературы ``References'' приводится полностью отдельным блоком, повторяя все
позиции из списка литературы к русскоязычной части, независимо от того, имеются или
нет в нем иностранные источники. Если в списке литературы к русскоязычной части есть
ссылки на иностранные публикации, набранные латиницей, они полностью повторяются в
списке ``References''.

Ниже приведены примеры ссылок на различные виды публикаций в списке ``References''.

\def\leftfootline{\small{\textbf{\thepage}
\hfill ИНФОРМАТИКА И ЕЁ ПРИМЕНЕНИЯ\ \ \ том\ 18\ \ \ выпуск\ 3\ \ \ 2024}
}%
 \def\rightfootline{\small{ИНФОРМАТИКА И ЕЁ ПРИМЕНЕНИЯ\ \ \ том\ 18\ \ \ выпуск\ 3\ \ \ 2024
\hfill \textbf{\thepage}}}

{\small

\noindent
\textbf{Описание статьи из журнала:}

\Aue{Zagurenko, A.\,G., V.\,A.~Korotovskikh, A.\,A.~Kolesnikov, A.\,V.~Timonov, and D.\,V.~Kardymon}. 2008.
Tekhniko-ekonomicheskaya optimizatsiya dizayna gidrorazryva plasta [Technical and
economic optimization of the design
of hydraulic fracturing]. \textit{Neftyanoe hozyaystvo} [\textit{Oil Industry}] 11:54--57.

\Aue{Zhang, Z., and D.~Zhu}. 2008. Experimental research on the localized
electrochemical micromachining. \textit{Russ. J.~Electrochem.}  44(8):926--930.
{\sf doi:10.1134/S1023193508080077}.

\noindent
\textbf{Описание статьи из электронного журнала:}

\Aue{Swaminathan, V., E.~Lepkoswka-White, and B.\,P.~Rao}. 1999. Browsers or buyers in cyberspace? An
investigation of electronic factors influencing electronic exchange. \textit{JCMC}
5(2). Available at: {\sf http://www.ascusc.org/jcmc/vol5/issue2/} (accessed April~28, 2011).

\def\leftkol{Правила подготовки рукописей  для публикации в журнале
<<Информатика и её применения>>}

\def\rightkol{Правила подготовки рукописей  для публикации в журнале
<<Информатика и её применения>>}


\noindent
\textbf{Описание статьи из продолжающегося издания (сборника трудов):}

\Aue{Astakhov, M.\,V., and T.\,V.~Tagantsev}. 2006. Eksperimental'noe
issledovanie prochnosti soedineniy ``stal'--kompozit'' [Experimental study of
the strength of joints ``steel--composite'']. \textit{Trudy MGTU
``Matematicheskoe modelirovanie slozhnykh tekh\-ni\-che\-skikh sistem''}
[\textit{Bauman MSTU ``Mathematical Modeling of Complex Technical
Systems'' Proceedings}]. 593:125--130.


\pagebreak



\noindent
\textbf{Описание материалов конференций:}

\Aue{Usmanov, T.\,S., A.\,A.~Gusmanov, I.\,Z.~Mullagalin, R.\,Ju.~Muhametshina, A.\,N.~Chervyakova, and
A.\,V.~Sveshnikov}. 2007. Osobennosti proektirovaniya razrabotki mestorozhdeniy
s primeneniem gidrorazryva
plasta [Features of the design of field development with the use of hydraulic fracturing].
\textit{Trudy 6-go
Mezhdu\-na\-rod\-no\-go Simpoziuma ``Novye resursosberegayushchie tekhnologii nedropol'zovaniya i povysheniya
neftegazootdachi''} [\textit{6th  Symposium (International) ``New Energy Saving Subsoil Technologies and
the Increasing of the Oil and Gas Impact'' Proceedings}]. Moscow. 267--272.



\def\leftfootline{\small{\textbf{\thepage}
\hfill ИНФОРМАТИКА И ЕЁ ПРИМЕНЕНИЯ\ \ \ том\ 18\ \ \ выпуск\ 3\ \ \ 2024}
}%
 \def\rightfootline{\small{ИНФОРМАТИКА И ЕЁ ПРИМЕНЕНИЯ\ \ \ том\ 18\ \ \ выпуск\ 3\ \ \ 2024
\hfill \textbf{\thepage}}}



\noindent
\textbf{Описание книги (монографии, сборники):}



Lindorf, L.\,S., and L.\,G.~Mamikoniants, eds. 1972.
\textit{Ekspluatatsiya turbogeneratorov s neposredstvennym
okhlazhdeniem} [\textit{Operation of turbine generators with direct cooling}].
Moscow: Energy Publs. 352~p.


\Aue{Latyshev, V.\,N.} 2009. \textit{Tribologiya rezaniya. Kn.~1: Friktsionnye protsessy
pri rezanii metallov}
[\textit{Tribology of cutting. Vol.~1: Frictional processes in metal cutting}]. Ivanovo: Ivanovskii
State Univ. 108~p.

\def\leftkol{Правила подготовки рукописей  для публикации в журнале
<<Информатика и её применения>>}

\def\rightkol{Правила подготовки рукописей  для публикации в журнале
<<Информатика и её применения>>}

\noindent
\textbf{Описание переводной книги}
(в списке литературы к русскоязычной части необходимо указать:~/ Пер.\ с англ.~---
после названия книги, а в конце ссылки указать оригинал книги в круглых скобках):
\begin{enumerate}[1.]
\item  В русскоязычной части:

\def\leftfootline{\small{\textbf{\thepage}
\hfill ИНФОРМАТИКА И ЕЁ ПРИМЕНЕНИЯ\ \ \ том\ 18\ \ \ выпуск\ 3\ \ \ 2024}
}%
 \def\rightfootline{\small{ИНФОРМАТИКА И ЕЁ ПРИМЕНЕНИЯ\ \ \ том\ 18\ \ \ выпуск\ 3\ \ \ 2024
\hfill \textbf{\thepage}}}

\Au{Тимошенко С.\,П., Янг Д.\,Х., Уивер~У.}
Колебания в инженерном деле~/ Пер.\ с англ.~--- М.: Машиностроение, 1985. 472~с.
(\Au{Timoshenko~S.\,P., Young~D.\,H., Weaver~W.}
Vibration problems in engineering.~--- 4th ed.~--- New York, NY, USA: Wiley, 1974. 521~p.)\\[-13.5pt]
\item  В англоязычной части:

\Aue{Timoshenko, S.\,P., D.\,H.~Young, and W.~Weaver}.
1974. \textit{Vibration problems in engineering}. 4th ed. New York: 
Wiley. 521~p.
\end{enumerate}

\vspace*{-3pt}


\noindent
\textbf{Описание неопубликованного документа:}


\Aue{Latypov, A.\,R., M.\,M.~Khasanov, and V.\,A.~Baikov}.
2004 (unpubl.). Geologiya i~dobycha (NGT GiD) [Geology and production (NGT GiD)]. Certificate on official registration of the computer program
No.\,2004611198. 

\noindent
\textbf{Описание интернет-ресурса:}


Pravila tsitirovaniya istochnikov [Rules for the citing of sources]. Available at: {\sf
http://www.scribd.com/doc/1034528/} (accessed February~7, 2011).

%\pagebreak

\noindent
\textbf{Описание диссертации или автореферата диссертации:}

\Aue{Semenov, V.\,I.}
2003. Matematicheskoe modelirovanie plazmy v sisteme kompaktnyy tor [Mathematical
modeling of the plasma in the compact torus].  Moscow.  D.Sc.\ Diss. 272~p.

\Aue{Kozhunova, O.\,S.} 2009. Tekhnologiya razrabotki semanticheskogo
slovarya informatsionnogo monitoringa [Technology of development of
semantic dictionary of information monitoring system].  Moscow: IPI RAN. PhD Thesis. 23~p.


\noindent
\textbf{Описание ГОСТа:}

GOST 8.586.5-2005. 2007. Metodika vypolneniya izmereniy. Izmerenie raskhoda i~kolichestva zhidkostey i~gazov
s~pomoshch'yu standartnykh suzhayushchikh ustroystv [Method of measurement.
Measurement of flow rate and volume of liquids and gases by means of orifice devices]. Moscow:
Standardinform  Publs. 10~p.

\noindent
\textbf{Описание патента:}

\Aue{Bolshakov, M.\,V., A.\,V.~Kulakov, A.\,N.~Lavrenov, and M.\,V.~Palkin}.
2006. Sposob orientirovaniya po krenu letatel'nogo
apparata s opti\-che\-skoy golovkoy
samonavedeniya [The way to orient on the roll of aircraft with optical homing head].
Patent RF No.\,2280590.
}

\item Присланные в редакцию материалы авторам не возвращаются.\\[-13.5pt]

\item При отправке файлов по электронной почте просим придерживаться следующих
правил:
\begin{itemize}
\item указывать в поле subject (тема) название журнала и фамилию автора;\\[-13.5pt]
\item указывать в тексте письма название статьи, авторов и~журнал, в~который направляется статья;\\[-13.5pt]
\item использовать attach (присоединение);\\[-13.5pt]
\item в состав электронной версии статьи должны входить: файл, содержащий текст
статьи, и файл(ы), содержащий(е) иллюстрации.\\[-13.5pt]
\end{itemize}

\item Журнал <<Информатика и её применения>> является некоммерческим изданием.
Плата за публикацию не взимается, гонорар авторам не выплачивается.
\end{enumerate}



\def\leftfootline{\small{\textbf{\thepage}
\hfill ИНФОРМАТИКА И ЕЁ ПРИМЕНЕНИЯ\ \ \ том\ 18\ \ \ выпуск\ 3\ \ \ 2024}
}%
 \def\rightfootline{\small{ИНФОРМАТИКА И ЕЁ ПРИМЕНЕНИЯ\ \ \ том\ 18\ \ \ выпуск\ 3\ \ \ 2024
\hfill \textbf{\thepage}}}


\vspace*{-1mm}

\begin{center}

\textbf{Адрес редакции журнала <<Информатика и её применения>>:} \\




Москва 119333, ул.~Вавилова, д.~44, корп.~2, ФИЦ ИУ РАН\\[-10pt]

\

Тел.: +7\,(499)\,135-86-92\ \ Факс:  +7\,(495)\,930-45-05\\[-10pt]

 \

e-mail:   {\sf iiep@frccsc.ru} (Стригина Светлана Николаевна)\\[-10pt]

\

{\sf http://www.ipiran.ru/journal/issues/}
\end{center}
}


\def\leftkol{Правила подготовки рукописей  для публикации в журнале
<<Информатика и её применения>>}

\def\rightkol{Правила подготовки рукописей  для публикации в журнале
<<Информатика и её применения>>}


\def\leftfootline{\small{\textbf{\thepage}
\hfill ИНФОРМАТИКА И ЕЁ ПРИМЕНЕНИЯ\ \ \ том\ 18\ \ \ выпуск\ 3\ \ \ 2024}
}%
 \def\rightfootline{\small{ИНФОРМАТИКА И ЕЁ ПРИМЕНЕНИЯ\ \ \ том\ 18\ \ \ выпуск\ 3\ \ \ 2024
\hfill \textbf{\thepage}}} 
\def\stat{podg-e}
{%\hrule\par
%\vskip 7pt % 7pt
\vspace*{-24pt}
\raggedleft\Large \bf%\baselineskip=3.2ex
Requirements for manuscripts submitted to Journal
``Informatics~and~Applications'' \vskip 8pt
    \hrule
    \par
\vskip 21pt plus 6pt minus 3pt }

\label{st\stat}

\def\tit{\ }

\def\aut{\ }
\def\auf{\ }

\def\leftkol{\ }

\def\rightkol{\ }
%Requirements for manuscripts submitted to Journal
%``Informatics~and~Applications''}

\titele{\tit}{\aut}{\auf}{\leftkol}{\rightkol}

\def\leftfootline{\small{\textbf{\thepage}
\hfill INFORMATIKA I EE PRIMENENIYA~--- INFORMATICS AND APPLICATIONS\ \ \ 2019\
\ \ volume~13\ \ \ issue\ 4}
}%
 \def\rightfootline{\small{INFORMATIKA I EE PRIMENENIYA~--- INFORMATICS AND APPLICATIONS\ \ \ 2019\ \ \ volume~13\ \ \ issue\ 4
\hfill \textbf{\thepage}}}

\vspace*{-60pt}

{\small

\noindent
Journal ``Informatics and Applications'' (Inform.\ Appl.)
publishes theoretical, review, and discussion
articles on the research and development in the
field of informatics and its applications.

The journal is published in Russian.
By a special decision of the editorial
board, some articles can be published in English.


The topics covered include the following areas:
\begin{itemize}
               \item
     theoretical fundamentals of informatics; \\[-14pt]
\item
mathematical methods for studying complex systems and processes; \\[-14pt]
\item
information systems and networks;\\[-14pt]
\item
information technologies; and \\[-14pt]
\item
architecture and software of computational complexes and networks. \\[-14pt]
\end{itemize}

\noindent
\begin{enumerate}[1.]
\item The Journal publishes original articles which have not been published before and are not
intended for simultaneous publication in other editions. An article submitted to the Journal must not violate the
Copyright law. Sending the manuscript to the Editorial Board, the authors retain all rights of the
owners of the manuscript and transfer the nonexclusive rights to publish the article in Russian
(or the language of the article, if not Russian) and its distribution in Russia and abroad to the
Founders and the Editorial Board. Authors should submit a letter to the Editorial Board in the
following form:

{\bfseries\textit{Agreement on the transfer of rights to publish:}}

``\textit{We, the undersigned authors of the manuscript ``\ldots'', pass to the
Founder and the Editorial Board of the Journal ``Informatics and Applications''
the nonexclusive right to publish the manuscript of the article in Russian (or
in English) in both print and electronic versions of the Journal. We affirm
that this publication does not violate the Copyright of other persons or
organizations.}

\textit{Author(s) signature(s): (name(s), address(es), date).}

This agreement should be submitted in paper form or in the form of a scanned copy (signed by
the authors).


%The Editorial Board has the right to request from the authors an official expert conclusion that
%the submitted article has no secret data prohibited for publication. \\[-13.5pt]
\item
A submitted article should be attached with \textbf{the data on the author(s)} (see item~8). If
there are several authors, the contact person should be indicated who is responsible for
correspondence with the Editorial Board and other authors about revisions and final approval
of the proofs.\\[-13.5pt]

\item The Editorial Board of the Journal examines the article according to the established
reviewing procedure. If the authors receive their article for correction after reviewing, it does not
mean that the article is approved for publication. The corrected article should be sent to the
Editorial Board for the subsequent review and approval.\\[-13.5pt]

\item The decision on the article publication or its rejection is communicated to the authors. The
Editorial Board may also send the reviews on the submitted articles to the authors. Any
discussion upon the rejected articles is not possible.\\[-13.5pt]

\item The edited articles will be sent to the authors for proofread. The comments of the authors
to the edited text of the article should be sent to the Editorial Board as soon as possible.\\[-13.5pt]

\item The manuscript of the article should be presented electronically in the MS WORD (.doc or
.docx) or \LaTeX\ (.tex) formats, and additionally in the .pdf format. All documents
 may be sent
by e-mail or provided on a CD or diskette. A~hard copy submission is not necessary.\\[-13.5pt]

\item The recommended typesetting instructions for manuscript.

Pages parameters: format A4, portrait orientation, document margins (cm): left~--- 2.5, right~---
1.5, above~--- 2.0, below~--- 2.0, footer 1.3.

Text: font~---Times New Roman, font size~--- 14, paragraph indent~--- 0.5, line spacing~--- 1.5,
justified alignment.

The recommended manuscript size: not more than 15~pages of the specified format.
If the specified size exceeded, the editorial board is entitled to require the author
to reduce the manuscript.

Use only standard abbreviations. Avoid  abbreviations in the title and
abstract. The full term for which an abbreviation stands should precede
its first use in the text unless it is a standard unit of measurement.

All pages of the manuscript should be numbered.

The templates for the manuscript typesetting are presented on site: {\sf
http://www.ipiran.ru/journal/template.doc}.\\[-13.5pt]


%\def\leftkol{Requirements for manuscripts submitted to Journal
%``Informatics~and~Applications''}

\item The articles should enclose data both in \textbf{Russian and English}:
\begin{itemize}
\item title;\\[-13.5pt]
\item author's name and surname;\\[-13.5pt]
\item affiliation~--- organization, its address with ZIP code, city, country, and
official e-mail address;\\[-13.5pt]
\item data on authors according to the format: (see site)

{\sf http://www.ipiran.ru/journal/issues/2013\_07\_01/authors.asp}  and

{\sf  http://www.ipiran.ru/journal/issues/2013\_07\_01\_eng/authors.asp};\\[-13.5pt]

\pagebreak

\def\leftfootline{\small{\textbf{\thepage}
\hfill INFORMATIKA I EE PRIMENENIYA~--- INFORMATICS AND APPLICATIONS\ \ \ 2019\
\ \ volume~13\ \ \ issue\ 4}
}%
 \def\rightfootline{\small{INFORMATIKA I EE PRIMENENIYA~--- INFORMATICS AND APPLICATIONS\ \ \ 2019\ \ \ volume~13\ \ \ issue\ 4
\hfill \textbf{\thepage}}}


%\def\leftkol{Requirements for manuscripts submitted to Journal
%``Informatics~and~Applications''}

%\def\rightkol{Requirements for manuscripts submitted to Journal
%``Informatics~and~Applications''}



\item abstract (not less than 100 words) both in Russian and in English. Abstract is a short
summary of the article that can be published separately. The abstract is the
main source of information on the article and it could be included in leading information
systems and data bases. The abstract in English has to be an original text and should
not be an exact translation of the Russian one. Good English is required.
In abstracts, avoid references and formulae;\\[-13.5pt]
\item indexing is performed on the basis of keywords. The use of keywords from the
internationally accepted thematic Thesauri is recommended.

%\def\leftkol{Requirements for manuscripts submitted to Journal
%``Informatics~and~Applications''}

%\def\rightkol{Requirements for manuscripts submitted to Journal
%``Informatics~and~Applications''}

Important! Keywords must not be sentences;
\item Acknowledgments.
\end{itemize}

\item References. Russian references have to be presented both in English translation and Latin
transliteration (refer {\sf http://www.translit.net/ru/bgn/}).

Please take into account the following examples of Russian references appearance:

\noindent
\textbf{Article in journal:}

\Aue{Zhang, Z., and D.~Zhu}. 2008. Experimental research on the localized electrochemical
micromachining.
\textit{Rus. J.~Electrochem.}  44(8):926--930. {\sf doi:10.1134/S1023193508080077}.


\noindent
\textbf{Journal article in electronic format:}

\Aue{Swaminathan, V., E.~Lepkoswka-White, and B.\,P.~Rao}. 1999. Browsers or buyers in
cyberspace? An
investigation of electronic factors influencing electronic exchange. \textit{JCMC}
5(2). Available at: {\sf http://www.ascusc.org/jcmc/vol5/issue2/} (accessed April~28, 2011).




\noindent
\textbf{Article from the continuing publication (collection of works, proceedings):}

\Aue{Astakhov, M.\,V., and T.\,V.~Tagantsev}. 2006. Eksperimental'noe
issledovanie prochnosti soedineniy ``stal'--kompozit'' [Experimental study of
the strength of joints ``steel--composite'']. \textit{Trudy MGTU
``Matematicheskoe modelirovanie slozhnykh tekh\-ni\-che\-skikh sistem''}
[\textit{Bauman MSTU ``Mathematical Modeling of Complex Technical
Systems'' Proceedings}]. 593:125--130.

\def\leftfootline{\small{\textbf{\thepage}
\hfill INFORMATIKA I EE PRIMENENIYA~--- INFORMATICS AND APPLICATIONS\ \ \ 2019\
\ \ volume~13\ \ \ issue\ 4}
}%
 \def\rightfootline{\small{INFORMATIKA I EE PRIMENENIYA~--- INFORMATICS AND APPLICATIONS\ \ \ 2019\ \ \ volume~13\ \ \ issue\ 4
\hfill \textbf{\thepage}}}

\def\leftkol{Requirements for manuscripts submitted to Journal
``Informatics~and~Applications''}

\def\rightkol{Requirements for manuscripts submitted to Journal
``Informatics~and~Applications''}

\noindent
\textbf{Conference proceedings:}

\Aue{Usmanov, T.\,S., A.\,A.~Gusmanov, I.\,Z.~Mullagalin, R.\,Ju.~Muhametshina,
A.\,N.~Chervyakova, and
A.\,V.~Sveshnikov}. 2007. Osobennosti proektirovaniya razrabotki mestorozhdeniy
s primeneniem gidrorazryva
plasta [Features of the design of field development with the use of hydraulic fracturing].
\textit{Trudy 6-go
Mezhdu\-na\-rod\-no\-go Simpoziuma ``Novye resursosberegayushchie tekhnologii
nedropol'zovaniya i povysheniya
neftegazootdachi''} [\textit{6th  Symposium (International) ``New Energy Saving Subsoil
Technologies and
the Increasing of the Oil and Gas Impact'' Proceedings}]. Moscow. 267--272.


\noindent
\textbf{Books and other monographs:}




Lindorf, L.\,S., and L.\,G.~Mamikoniants, eds. 1972.
\textit{Ekspluatatsiya turbogeneratorov s neposredstvennym
okhlazhdeniem} [\textit{Operation of turbine generators with direct cooling}].
Moscow: Energy Publs. 352~p.


%\Aue{Latyshev, V.\,N.} 2009. \textit{Tribologiya rezaniya. Kn.~1: Frikcionnye prosessy
%pri rezanii metallov}
%[\textit{Tribology of cutting. Vol.~1: Frictional processes in metal cutting}]. Ivanovo: Ivanovskii
%State Univ. 108~p.


%\noindent
%\textbf{Unpublished material:}

%\Aue{Latypov, A.\,R., M.\,M.~Khasanov, and V.\,A.~Baikov}.
%2004. Geology and production (NGT GiD). Certificate on official registration of the computer
%program
%No.\,2004611198. (In Russian, unpubl.)

%\noindent
%\textbf{Internet-source:}

%APA Style. 2011. Available at: {\sf http://www.apastyle.org/apa-style-help.aspx} (accessed
%February~5, 2011).

%Pravila citirovaniya istochnikov [Rules for the citing of sources]. Available at: {\sf
%http://www.scribd.com/doc/1034528/} (accessed February~7, 2011).


\noindent
\textbf{Dissertation and Thesis:}

%\Aue{Semenov, V.\,I.}
%2003. Matematicheskoe modelirovanie plazmy v sisteme kompaktnyy tor. [Mathematical
%modeling of the plasma in the compact torus]. D.Sc.\ Diss. Moscow. 272~p.

\Aue{Kozhunova, O.\,S.} 2009. Tekhnologiya razrabotki semanticheskogo
slovarya informatsionnogo monitoringa [Technology of development of
semantic dictionary of information monitoring system]. PhD Thesis. Moscow: IPI RAN. 23~p.


\noindent
\textbf{State standards and patents:}

GOST 8.586.5-2005. 2007. Metodika vypolneniya izmereniy. Izmerenie raskhoda i~kolichestva
zhidkostey i gazov 
s~pomoshch'yu standartnykh suzhayushchikh ustroystv [Method of measurement.
Measurement of flow rate and volume of liquids and gases by means of orifice devices]. M.:
Standardinform
Publs. 10~p.

%\noindent
%\textbf{Patent:}

\Aue{Bolshakov, M.\,V., A.\,V.~Kulakov, A.\,N.~Lavrenov, and M.\,V.~Palkin}.
2006. Sposob orientirovaniya po krenu letatel'nogo
apparata s opti\-che\-skoy golovkoy
samonavedeniya [The way to orient on the roll of aircraft with optical homing head].
Patent RF No.\,2280590.

References in Latin transcription are presented in the original language.

References in the text are numbered according to the order of their
first appearance; the number is
placed in square brackets. All items from the reference list should be
cited.\\[-13.5pt]

\item Manuscripts and additional materials are not returned to Authors by the Editorial Board.\\[-13.5pt]

\item Submissions of files by e-mail must include:\\[-13.5pt]
\begin{itemize}
\item   the journal title and author's name in the ``Subject'' field; \\[-13.5pt]
\item   an article and additional materials have to be attached using the ``attach'' function;\\[-13.5pt]
\item   an electronic version of the article should contain the file with the text and a separate file
with figures.\\[-13.5pt]
\end{itemize}

\item ``Informatics and Applications'' journal is not a profit publication. There are no
charges for the authors as well as there are no royalties.\\[-13.5pt]
\end{enumerate}

\def\leftfootline{\small{\textbf{\thepage}
\hfill INFORMATIKA I EE PRIMENENIYA~--- INFORMATICS AND APPLICATIONS\ \ \ 2019\
\ \ volume~13\ \ \ issue\ 4}
}%
 \def\rightfootline{\small{INFORMATIKA I EE PRIMENENIYA~--- INFORMATICS AND APPLICATIONS\ \ \ 2019\ \ \ volume~13\ \ \ issue\ 4
\hfill \textbf{\thepage}}}

\def\leftkol{Requirements for manuscripts submitted to Journal
``Informatics~and~Applications''}

\def\rightkol{Requirements for manuscripts submitted to Journal
``Informatics~and~Applications''}


%\vspace*{5mm}


\begin{center}
\textbf{Editorial Board address:} \\

%ABOUT AUTHORS



FRC CSC RAS, 44, block~2, Vavilov Str., Moscow 119333, Russia\\[-10pt]

\

Ph.: +7\,(499)\,135\,86\,92,\ \ Fax: +7\,(495)\,930\,45\,05\\[-10pt]

\

 e-mail: {\sf rust@ipiran.ru} (to Prof.\ Rustem Seyful-Mulyukov)\\[-10pt]

\

 {\sf http://www.ipiran.ru/english/journal.asp}
\end{center}
 }
%\thispagestyle{myheadings}

\def\leftkol{Requirements for manuscripts submitted to Journal
``Informatics~and~Applications''}

\def\rightkol{Requirements for manuscripts submitted to Journal
``Informatics~and~Applications''}

\def\leftfootline{\small{\textbf{\thepage}
\hfill INFORMATIKA I EE PRIMENENIYA~--- INFORMATICS AND APPLICATIONS\ \ \ 2019\
\ \ volume~13\ \ \ issue\ 4}
}%
 \def\rightfootline{\small{INFORMATIKA I EE PRIMENENIYA~--- INFORMATICS AND APPLICATIONS\ \ \ 2019\ \ \ volume~13\ \ \ issue\ 4
\hfill \textbf{\thepage}}}

 \label{end\stat}

\newpage

%\vspace*{-60pt} {\small
{\baselineskip=9.1pt
\section*{Правила подготовки рукописей статей для публикации в журнале
<<Информатика и её применения>>}

\thispagestyle{empty}

 Журнал <<Информатика и её применения>> публикует
теоретические, обзорные и дискуссионные статьи, посвященные научным
исследованиям и разработкам в области информатики и ее приложений. Журнал
издается на русском языке. По специальному решению редколлегии отдельные статьи,
в виде исключения, могут печататься на английском языке.
Тематика журнала охватывает следующие направления:
\begin{itemize}
\item теоретические основы информатики; %\\[-13.5pt]
\item математические методы исследования сложных систем и процессов; %\\[-13.5pt]
\item информационные системы и сети; %\\[-13.5pt]
\item информационные технологии; %\\[-13.5pt]
\item архитектура и программное
обеспечение вычислительных комплексов и сетей.
\end{itemize}
\begin{enumerate}
\item В журнале печатаются результаты, ранее не
опубликованные и не предназначенные к одновременной публикации в других
изданиях. Публикация не должна нарушать закон об авторских правах. Направляя
свою рукопись в редакцию, авторы автоматически передают учредителям и
редколлегии неисключительные права на издание данной статьи на русском языке и
на ее распространение в России и за рубежом. При этом за авторами сохраняются
все права как собственников данной рукописи. В связи с этим авторами должно
быть представлено в редакцию письмо в следующей форме:
Соглашение о передаче права на публикацию:

\textit{<<Мы, нижеподписавшиеся, авторы рукописи <<$\qquad\qquad$>>, передаем
учредителям и редколлегии журнала <<Информатика и её применения>>
неисключительное право опубликовать данную рукопись статьи на русском языке как
в печатной, так и в электронной версиях журнала. Мы подтверждаем, что данная
публикация не нарушает авторского права других лиц или организаций. Подписи
авторов: (ф.\,и.\,о., дата, адрес)>>.}

Указанное соглашение может быть представлено 
как в бумажном виде, так и в виде отсканированной копии (с подписями авторов).


Редколлегия вправе запросить у авторов экспертное заключение о возможности
опубликования представленной статьи в открытой печати. %\\[-13.5pt]
\item Статья
подписывается всеми авторами. На отдельном листе представляются данные автора
(или всех авторов): фамилия, полные имя и отчество, телефон, факс, e-mail,
почтовый адрес. Если работа выполнена несколькими авторами, указывается фамилия
одного из них, ответственного за переписку с редакцией. %\\[-13.5pt]
\item Редакция журнала
осуществляет самостоятельную экспертизу присланных статей. Возвращение рукописи
на доработку не означает, что статья уже принята к печати. Доработанный вариант
с ответом на замечания рецензента необходимо прислать в редакцию. %\\[-13.5pt]
\item Решение
редакционной коллегии о принятии статьи к печати или ее отклонении сообщается
авторам. Редколлегия не обязуется направлять рецензию авторам отклоненной
статьи. %\\[-13.5pt]
\item Корректура статей высылается авторам для просмотра. Редакция
просит авторов присылать свои замечания в кратчайшие сроки. %\\[-13.5pt]
\item При
подготовке рукописи в MS Word рекомендуется использовать следующие настройки.
Параметры страницы: формат~--- А4; ориентация~--- книжная; поля (см): внутри~---
2,5, снаружи~--- 1,5, сверху~--- 2, снизу~--- 2, от края до нижнего
колонтитула~--- 1,3. Основной текст: стиль~--- <<Обычный>>: шрифт Times New
Roman, размер 14~пунктов, абзацный отступ~--- 0,5~см, 1,5 интервала,
выравнивание~--- по ширине. Рекомендуемый объем рукописи~--- не свыше
25~страниц указанного формата. Ознакомиться с шаблонами, содержащими примеры
оформления, можно по адресу в Интернете:
\textsf{http://www.ipiran.ru/journal/template.doc}.
\item К рукописи, предоставляемой в 2-х
экземплярах, обязательно прилагается электронная версия статьи (как правило, в
форматах MS WORD (.doc) или \LaTeX\ (.tex), а также~--- дополнительно~--- в
формате .pdf) на дискете, лазерном диске или по электронной почте. Сокращения
слов, кроме стандартных, не применяются. Все страницы рукописи должны быть
пронумерованы. %\\[-13.5pt]
\item Статья должна содержать следующую информацию на русском и
английском языках: название, Ф.И.О. авторов, места работы авторов и их
электронные адреса, подробные сведения об авторах, оформленные в соответствии с форматом, 
определяемым файлами {\sf http://www.ipiran.ru/journal/issues/2011\_05\_01/authors.asp} и 
{\sf http://www.ipiran.ru/journal/issues/2011\_01\_eng/authors.asp},
аннотация (не более 100~слов), ключевые слова. Ссылки на
литературу в тексте статьи нумеруются (в квадратных скобках) и располагаются в
порядке их первого упоминания. В~списке литературы не должно быть позиций, на которые нет ссылки в тексте статьи.
Все фамилии авторов, заглавия статей, названия
книг, конференций и~т.\,п.\ даются на языке оригинала, если этот язык
использует кириллический или латинский алфавит. %\\[-13.5pt]
\item Присланные в редакцию материалы авторам не возвращаются.
\item При отправке файлов по электронной
почте просим придерживаться следующих правил:
\begin{itemize}
\item указывать в поле subject (тема) название журнала и фамилию автора; %\\[-13.5pt]
\item использовать attach (присоединение); %\\[-13.5pt]
\item в случае больших объемов информации возможно
использование общеизвестных архиваторов (ZIP, RAR); %\\[-13.5pt]
\item в состав электронной версии статьи должны входить: файл, содержащий текст статьи, и файл(ы),
содержащий(е) иллюстрации. %\\[-13.5pt]
\end{itemize}
\item Журнал <<Информатика и её применения>> является некоммерческим изданием. 
Плата за публикацию с авторов не взимается, гонорар авторам не выплачивается.
\end{enumerate}
\thispagestyle{empty}
\textbf{Адрес редакции:} Москва 119333,
ул.~Вавилова, д.~44, корп.~2, ИПИ РАН\\
\hphantom{\textbf{Адрес редакции:} }Тел.: +7 (499) 135-86-92\ \
Факс:  +7 (495) 930-45-05\ \  E-mail:   rust@ipiran.ru }
}

%\include{ipi-ind}

%\tableofcontents

\end{document}


%\tableofcontents

%\end{document}





%\def\stat{cont}
{%\hrule\par
%\vskip 7pt % 7pt
\raggedleft\Large \bf%\baselineskip=3.2ex
А\,В\,Т\,О\,Р\,С\,К\,И\,Й\ \ У\,К\,А\,З\,А\,Т\,Е\,Л\,Ь\ \ З\,А\ \ 2\,0\,0\,7 г. \vskip 17pt
    \hrule
    \par
\vskip 21pt plus 6pt minus 3pt }

\label{st\stat}

\def\tit{\ }

\def\aut{\ }
\def\auf{\ }

\def\leftkol{\ } % ENGLISH ABSTRACTS}

\def\rightkol{\ } %ENGLISH ABSTRACTS}

\titele{\tit}{\aut}{\auf}{\leftkol}{\rightkol}


\contentsline {chapter}{\ }{Выпуск \quad Стр.} 
\contentsline {section}{\textbf{Батракова Д.\,А., Королев В.\,Ю., Шоргин С.\,Я.}\ \ Новый метод вероятностно-ста\-ти\-сти\-че\-ско\-го анализа информационных потоков в\nobreakspace {}телекоммуникационных сетях}{\qquad 1 \qquad 40} 
\contentsline {section}{\textbf{Борисов А.\,В.}\ \ Байесовское оценивание в системах наблюдения с\nobreakspace {}марковскими скачкообразными процессами: игровой подход}{\qquad 2 \qquad 65}
\contentsline {section}{\textbf{Босов А.\,В., Иванов А.\,В.}\ \ Программная инфраструктура информационного Web-пор\-тала}{\qquad 2 \qquad 50}
\contentsline {section}{\textbf{Захаров В.\,Н., Калиниченко Л.\,А., Соколов И.\,А., Ступников С.\,А.}\ \ Конструирование канонических информационных моделей для интегрированных информационных систем}{\qquad 2 \qquad 15}
\contentsline {section}{\textbf{Захаров В.\,Н., Козмидиади В.\,А.}\ \ Средства обеспечения отказоустойчивости при\-ло\-жений}{\qquad 1 \qquad 14} 
\contentsline {section}{\textbf{Иванов А.\,В.}\ \ см. Босов А.\,В.\hfill\hfill\hfill\hfill\hfill\hfill\hfill\hfill\hfill\hfill\hfill\hfill\hfill\hfill\hfill\hfill\hfill\hfill\hfill\hfill\hfill\hfill\hfill\hfill\hfill\hfill\hfill\hfill\hfill\hfill\hfill\hfill\hfill\hfill\hfill}{\ }
\contentsline {section}{\textbf{Ильин В.\,Д., Соколов И.\,А.}\ \ Символьная модель системы знаний информатики в\nobreakspace {}че\-ло\-ве\-ко-автоматной среде}{\qquad 1 \qquad 66} 
\contentsline {section}{\textbf{Калиниченко Л.\,А.}\ \ см. Захаров В.\,Н.\hfill\hfill\hfill\hfill\hfill\hfill\hfill\hfill\hfill\hfill\hfill\hfill\hfill\hfill\hfill\hfill\hfill\hfill\hfill\hfill\hfill\hfill\hfill\hfill\hfill\hfill\hfill\hfill\hfill\hfill\hfill\hfill\hfill\hfill\hfill}{\ }
\contentsline {section}{\textbf{Козеренко Е.\,Б.}\ \ Лингвистическое моделирование для систем машинного перевода и обработки знаний}{\qquad 1 \qquad 54} 
\contentsline {section}{\textbf{Козмидиади В.\,А.}\ \ см. Захаров В.\,Н.\hfill\hfill\hfill\hfill\hfill\hfill\hfill\hfill\hfill\hfill\hfill\hfill\hfill\hfill\hfill\hfill\hfill\hfill\hfill\hfill\hfill\hfill\hfill\hfill\hfill\hfill\hfill\hfill\hfill\hfill\hfill\hfill\hfill\hfill\hfill }{\ } 
\contentsline {section}{\textbf{Королев В.\,Ю.}\ \ см. Батракова Д.\,А.\hfill\hfill\hfill\hfill\hfill\hfill\hfill\hfill\hfill\hfill\hfill\hfill\hfill\hfill\hfill\hfill\hfill\hfill\hfill\hfill\hfill\hfill\hfill\hfill\hfill\hfill\hfill\hfill\hfill\hfill\hfill\hfill\hfill\hfill\hfill}{\ } 
\contentsline {section}{\textbf{Кудрявцев А.\,А., Шоргин С.\,Я.}\ \ Байесовский подход к\nobreakspace {}анализу систем массового обслуживания и\nobreakspace {}показателей надежности}{\qquad 2 \qquad 76}
\contentsline {section}{\textbf{Печинкин А.\,В., Соколов И.\,А., Чаплыгин В.\,В.}\ \ Многолинейная система массового обслуживания с конечным накопителем и ненадежными приборами}{\qquad 1 \qquad 27} 
\contentsline {section}{\textbf{Печинкин А.\,В., Соколов И.\,А., Чаплыгин В.\,В.}\ \ Стационарные характеристики многолинейной\nobreakspace {}системы массового обслуживания с\nobreakspace {}одновременными отказами приборов}{\qquad 2 \qquad 39}
\contentsline {section}{\textbf{Синицын И.\,Н.}\ \ Корреляционные методы построения аналитических информационных моделей флуктуаций полюса Земли по априорным данным}{\qquad 2 \qquad \hphantom{9}2}
\contentsline {section}{\textbf{Синицын И.\,Н.}\ \ Развитие теории фильтров Пугачева для оперативной обработки информации в стохастических системах}{{\qquad 1 \qquad \hphantom{9}3}} 
\contentsline {section}{\textbf{Соколов И.\,А.}\ \ см. Захаров В.\,Н.\hfill\hfill\hfill\hfill\hfill\hfill\hfill\hfill\hfill\hfill\hfill\hfill\hfill\hfill\hfill\hfill\hfill\hfill\hfill\hfill\hfill\hfill\hfill\hfill\hfill\hfill\hfill\hfill\hfill\hfill\hfill\hfill\hfill\hfill\hfill}{\ }
\contentsline {section}{\textbf{Соколов И.\,А.}\ \ см. Ильин В.\,Д.\hfill\hfill\hfill\hfill\hfill\hfill\hfill\hfill\hfill\hfill\hfill\hfill\hfill\hfill\hfill\hfill\hfill\hfill\hfill\hfill\hfill\hfill\hfill\hfill\hfill\hfill\hfill\hfill\hfill\hfill\hfill\hfill\hfill\hfill\hfill}{\ } 
\contentsline {section}{\textbf{Соколов И.\,А.}\ \ см. Печинкин А.\,В.\hfill\hfill\hfill\hfill\hfill\hfill\hfill\hfill\hfill\hfill\hfill\hfill\hfill\hfill\hfill\hfill\hfill\hfill\hfill\hfill\hfill\hfill\hfill\hfill\hfill\hfill\hfill\hfill\hfill\hfill\hfill\hfill\hfill\hfill\hfill}{\ } 
\contentsline {section}{\textbf{Соколов И.\,А.}\ \ см. Печинкин А.\,В.\hfill\hfill\hfill\hfill\hfill\hfill\hfill\hfill\hfill\hfill\hfill\hfill\hfill\hfill\hfill\hfill\hfill\hfill\hfill\hfill\hfill\hfill\hfill\hfill\hfill\hfill\hfill\hfill\hfill\hfill\hfill\hfill\hfill\hfill\hfill}{\ }
\contentsline {section}{\textbf{Ступников С.\,А.}\ \ см. Захаров В.\,Н.\hfill\hfill\hfill\hfill\hfill\hfill\hfill\hfill\hfill\hfill\hfill\hfill\hfill\hfill\hfill\hfill\hfill\hfill\hfill\hfill\hfill\hfill\hfill\hfill\hfill\hfill\hfill\hfill\hfill\hfill\hfill\hfill\hfill\hfill\hfill}{\ }
\contentsline {section}{\textbf{Чаплыгин В.\,В.}\ \ см. Печинкин А.\,В.\hfill\hfill\hfill\hfill\hfill\hfill\hfill\hfill\hfill\hfill\hfill\hfill\hfill\hfill\hfill\hfill\hfill\hfill\hfill\hfill\hfill\hfill\hfill\hfill\hfill\hfill\hfill\hfill\hfill\hfill\hfill\hfill\hfill\hfill\hfill}{\ } 
\contentsline {section}{\textbf{Чаплыгин В.\,В.}\ \ см. Печинкин А.\,В.\hfill\hfill\hfill\hfill\hfill\hfill\hfill\hfill\hfill\hfill\hfill\hfill\hfill\hfill\hfill\hfill\hfill\hfill\hfill\hfill\hfill\hfill\hfill\hfill\hfill\hfill\hfill\hfill\hfill\hfill\hfill\hfill\hfill\hfill\hfill}{\ }
\contentsline {section}{\textbf{Шоргин С.\,Я.}\ \ см. Батракова Д.\,А.\hfill\hfill\hfill\hfill\hfill\hfill\hfill\hfill\hfill\hfill\hfill\hfill\hfill\hfill\hfill\hfill\hfill\hfill\hfill\hfill\hfill\hfill\hfill\hfill\hfill\hfill\hfill\hfill\hfill\hfill\hfill\hfill\hfill\hfill\hfill}{\ } 
\contentsline {section}{\textbf{Шоргин С.\,Я.}\ \ см. Кудрявцев А.\,А.\hfill\hfill\hfill\hfill\hfill\hfill\hfill\hfill\hfill\hfill\hfill\hfill\hfill\hfill\hfill\hfill\hfill\hfill\hfill\hfill\hfill\hfill\hfill\hfill\hfill\hfill\hfill\hfill\hfill\hfill\hfill\hfill\hfill\hfill\hfill}{\ }
%\thispagestyle{myheadings}
\def\leftfootline{\small{\textbf{\thepage}
\hfill ИНФОРМАТИКА И ЕЁ ПРИМЕНЕНИЯ\ \ \ том~1\ \ \ выпуск~2\ \ \ 2007}
}%
 \def\rightfootline{\small{ИНФОРМАТИКА И ЕЁ ПРИМЕНЕНИЯ\ \ \ том~1\ \ \ выпуск~2\ \ \ 2007
 \hfill \textbf{\thepage}}}
 \label{end\stat}

%\def\stat{cont-e}
{%\hrule\par
%\vskip 7pt % 7pt
\raggedleft\Large \bf%\baselineskip=3.2ex
2\,0\,0\,7\ \ A\,U\,T\,H\,O\,R\ \ I\,N\,D\,E\,X \vskip 17pt
    \hrule
    \par
\vskip 21pt plus 6pt minus 3pt }

\label{st\stat}

\def\tit{\ }

\def\aut{\ }
\def\auf{\ }

\def\leftkol{\ } % ENGLISH ABSTRACTS}

\def\rightkol{\ } %ENGLISH ABSTRACTS}

\titele{\tit}{\aut}{\auf}{\leftkol}{\rightkol}


\contentsline {chapter}{\ }{Issue \quad Page} 
\contentsline {subsection}{\textbf{Batrakova D.\,A., Korolev V.\,Yu., Shorgin S.\,Ya.}\ \ A New Method for the Probabilistic and Statistical Analysis of Information Flows in Telecommunication Networks}{\qquad 1 \qquad 40} 
\contentsline {subsection}{\textbf{Borisov A.\,V.}\ \ Bayesian Estimation in\nobreakspace {}Observation Systems with\nobreakspace {}Markov Jump Processes: Game-Theoretic Approach}{\qquad 2 \qquad 65} 
\contentsline {subsection}{\textbf{Bosov A.\,V., Ivanov A.\,V.}\ \ Linguistic Simulation for Machine Translation and Knowledge Management Systems}{\qquad 2 \qquad 50} 
\contentsline {subsection}{\textbf{Chaplygin V.\,V.} see Pechinkin A.\,V.\hfill\hfill\hfill\hfill\hfill\hfill\hfill\hfill\hfill\hfill\hfill\hfill\hfill\hfill\hfill\hfill\hfill\hfill\hfill\hfill\hfill\hfill\hfill\hfill\hfill\hfill\hfill\hfill\hfill\hfill\hfill\hfill\hfill\hfill\hfill}{\ }
\contentsline {subsection}{\textbf{Chaplygin V.\,V.} see Pechinkin A.\,V.\hfill\hfill\hfill\hfill\hfill\hfill\hfill\hfill\hfill\hfill\hfill\hfill\hfill\hfill\hfill\hfill\hfill\hfill\hfill\hfill\hfill\hfill\hfill\hfill\hfill\hfill\hfill\hfill\hfill\hfill\hfill\hfill\hfill\hfill\hfill}{\ }
\contentsline {subsection}{\textbf{Ilyin V.\,D., Sokolov I.\,A.}\ \ The Symbol Model of Informatics Knowledge System in Human-Automaton Environment}{\qquad 1 \qquad 66} 
\contentsline {subsection}{\textbf{Ivanov A.\,V.} see Bosov A.\,V.\hfill\hfill\hfill\hfill\hfill\hfill\hfill\hfill\hfill\hfill\hfill\hfill\hfill\hfill\hfill\hfill\hfill\hfill\hfill\hfill\hfill\hfill\hfill\hfill\hfill\hfill\hfill\hfill\hfill\hfill\hfill\hfill\hfill\hfill\hfill}{\ }
\contentsline {subsection}{\textbf{Kalinichenko L.\,A.} see Zakharov V.\,N.\hfill\hfill\hfill\hfill\hfill\hfill\hfill\hfill\hfill\hfill\hfill\hfill\hfill\hfill\hfill\hfill\hfill\hfill\hfill\hfill\hfill\hfill\hfill\hfill\hfill\hfill\hfill\hfill\hfill\hfill\hfill\hfill\hfill\hfill\hfill}{\ }
\contentsline {subsection}{\textbf{Korolev V.\,Yu.} see Batrakova D.\,A.\hfill\hfill\hfill\hfill\hfill\hfill\hfill\hfill\hfill\hfill\hfill\hfill\hfill\hfill\hfill\hfill\hfill\hfill\hfill\hfill\hfill\hfill\hfill\hfill\hfill\hfill\hfill\hfill\hfill\hfill\hfill\hfill\hfill\hfill\hfill}{\ }
\contentsline {subsection}{\textbf{Kozerenko E.\,B.}\ \ Linguistic Simulation for Machine Translation and Knowledge Management Systems}{\qquad 1 \qquad 54} 
\contentsline {subsection}{\textbf{Kozmidiady V.\,A.} see Zakharov V.\,N.\hfill\hfill\hfill\hfill\hfill\hfill\hfill\hfill\hfill\hfill\hfill\hfill\hfill\hfill\hfill\hfill\hfill\hfill\hfill\hfill\hfill\hfill\hfill\hfill\hfill\hfill\hfill\hfill\hfill\hfill\hfill\hfill\hfill\hfill\hfill}{\ }
\contentsline {subsection}{\textbf{Kudryavtsev A.\,A., Shorgin S.\,Ya.}\ \ Bayesian Approach to Queueing Systems and Reliability Characteristics}{\qquad 2 \qquad 76} 
\contentsline {subsection}{\textbf{Pechinkin A.\,V., Sokolov I.\,A., Chaplygin V.\,V.}\ \ Multichannel Queuing System with Finite Buffer and Unreliable Servers}{\qquad 1 \qquad 27} 
\contentsline {subsection}{\textbf{Pechinkin A.\,V., Sokolov I.\,A., Chaplygin V.\,V.}\ \ Stationary Characteristics of a Multichannel Queueing System with\nobreakspace {}Simultaneous Refusals of Servers}{\qquad 2 \qquad 39} 
\contentsline {subsection}{\textbf{Shorgin S.\,Ya.} see Batrakova D.\,A.\hfill\hfill\hfill\hfill\hfill\hfill\hfill\hfill\hfill\hfill\hfill\hfill\hfill\hfill\hfill\hfill\hfill\hfill\hfill\hfill\hfill\hfill\hfill\hfill\hfill\hfill\hfill\hfill\hfill\hfill\hfill\hfill\hfill\hfill\hfill}{\ }
\contentsline {subsection}{\textbf{Shorgin S.\,Ya.} see Kudryavtsev A.\,A.\hfill\hfill\hfill\hfill\hfill\hfill\hfill\hfill\hfill\hfill\hfill\hfill\hfill\hfill\hfill\hfill\hfill\hfill\hfill\hfill\hfill\hfill\hfill\hfill\hfill\hfill\hfill\hfill\hfill\hfill\hfill\hfill\hfill\hfill\hfill}{\ }
\contentsline {subsection}{\textbf{Sinitsyn I.\,N.}\ \ Correlational Methods for Analytical Informational Models of the Earth Pole Fluctuations Design Based on a priori Data}{\qquad 2 \qquad \hphantom{9}2}
\contentsline {subsection}{\textbf{Sinitsyn I.\,N.}\ \ Development of Pugachev Filtering for Stochastic Systems}{\qquad 1 \qquad \hphantom{9}3}
\contentsline {subsection}{\textbf{Sokolov I.\,A.} see Ilyin V.\,D.\hfill\hfill\hfill\hfill\hfill\hfill\hfill\hfill\hfill\hfill\hfill\hfill\hfill\hfill\hfill\hfill\hfill\hfill\hfill\hfill\hfill\hfill\hfill\hfill\hfill\hfill\hfill\hfill\hfill\hfill\hfill\hfill\hfill\hfill\hfill}{\ }
\contentsline {subsection}{\textbf{Sokolov I.\,A.} see Pechinkin A.\,V.\hfill\hfill\hfill\hfill\hfill\hfill\hfill\hfill\hfill\hfill\hfill\hfill\hfill\hfill\hfill\hfill\hfill\hfill\hfill\hfill\hfill\hfill\hfill\hfill\hfill\hfill\hfill\hfill\hfill\hfill\hfill\hfill\hfill\hfill\hfill}{\ }
\contentsline {subsection}{\textbf{Sokolov I.\,A.} see Pechinkin A.\,V.\hfill\hfill\hfill\hfill\hfill\hfill\hfill\hfill\hfill\hfill\hfill\hfill\hfill\hfill\hfill\hfill\hfill\hfill\hfill\hfill\hfill\hfill\hfill\hfill\hfill\hfill\hfill\hfill\hfill\hfill\hfill\hfill\hfill\hfill\hfill}{\ }
\contentsline {subsection}{\textbf{Sokolov I.\,A.} see Zakharov V.\,N.\hfill\hfill\hfill\hfill\hfill\hfill\hfill\hfill\hfill\hfill\hfill\hfill\hfill\hfill\hfill\hfill\hfill\hfill\hfill\hfill\hfill\hfill\hfill\hfill\hfill\hfill\hfill\hfill\hfill\hfill\hfill\hfill\hfill\hfill\hfill}{\ }
\contentsline {subsection}{\textbf{Stupnikov S.\,A.} see Zakharov V.\,N.\hfill\hfill\hfill\hfill\hfill\hfill\hfill\hfill\hfill\hfill\hfill\hfill\hfill\hfill\hfill\hfill\hfill\hfill\hfill\hfill\hfill\hfill\hfill\hfill\hfill\hfill\hfill\hfill\hfill\hfill\hfill\hfill\hfill\hfill\hfill}{\ }
\contentsline {subsection}{\textbf{Zakharov V.\,N., Kalinichenko L.\,A., Sokolov I.\,A., Stupnikov S.\,A.}\ \ Development of Canonical Information Models for Integrated Information Systems}{\qquad 2 \qquad 15} 
\contentsline {subsection}{\textbf{Zakharov V.\,N., Kozmidiady V.\,A.}\ \ Means Providing Applications Fault Tolerance}{\qquad 1 \qquad 14} 
\def\leftfootline{\small{\textbf{\thepage}
\hfill ИНФОРМАТИКА И ЕЁ ПРИМЕНЕНИЯ\ \ \ том~1\ \ \ выпуск~2\ \ \ 2007}
}%
 \def\rightfootline{\small{ИНФОРМАТИКА И ЕЁ ПРИМЕНЕНИЯ\ \ \ том~1\ \ \ выпуск~2\ \ \ 2007
 \hfill \textbf{\thepage}}}
 \label{end\stat}


%\tableofcontents


\end{document}

\newcommand{\Ack}{\subsection*{\protect\large\bf Acknowledgments}}