 \def\stat{gor+lub}

\def\tit{АЛГОРИТМ ПРЕОБРАЗОВАНИЯ ОДНОГО ГРАФА В~ДРУГОЙ С~МИНИМАЛЬНОЙ 
ЦЕНОЙ$^*$}

\def\titkol{Алгоритм преобразования одного графа в~другой с~минимальной 
ценой}

\def\aut{К.\,Ю.~Горбунов$^1$, В.\,А.~Любецкий$^2$}

\def\autkol{К.\,Ю.~Горбунов, В.\,А.~Любецкий}

\titel{\tit}{\aut}{\autkol}{\titkol}

\index{К.\,Ю.~Горбунов$^1$, В.\,А.~Любецкий$^2$}


{\renewcommand{\thefootnote}{\fnsymbol{footnote}} \footnotetext[1]
{Работа выполнена при финансовой поддержке Российского 
научного фонда (проект 14-50-00150).}}


\renewcommand{\thefootnote}{\arabic{footnote}}
\footnotetext[1]{Институт проблем передачи информации им.\ А.\,А.~Харкевича Российской академии наук, 
\mbox{gorbunov@iitp.ru}}
\footnotetext[2]{Институт проблем передачи информации им.\ А.\,А.~Харкевича Российской академии наук;
ме\-ха\-ни\-ко-ма\-те\-ма\-ти\-че\-ский факультет Московского государственного университета 
им.\ М.\,В.~Ломоносова, \mbox{lyubetsk@iitp.ru}}


\Abst{Рассматриваются ориентированные графы, состоящие из любого числа дизъюнктных 
цепей и~циклов, реб\-рам графов приписаны без повторений их имена~--- натуральные числа. 
Фиксирован список операций, каждая из которых по-сво\-ему преобразует один граф 
в~другой, ей приписано число~--- цена данной операции. Нужно найти минимальную по 
суммарной цене последовательность операций, которая для двух данных графов преобразует 
один в~другой. Эта задача самым широким образом применяется в~прикладных вопросах.  
По-ви\-ди\-мо\-му, она является NP-трудной и~поэтому может быть эффективно решена 
только при том или ином условии на цены или при некотором ограничении на графы. Ее 
решение при достаточно широких условиях получено в~виде линейных по времени 
и~памяти алгоритмов, для которых доказана точность (неэвристичность), т.\,е.\ доказано, что 
они всегда находят минимальную по цене последовательность операций. Задача давно 
решается многими эвристическими алгоритмами, которые тестировались на разных данных, 
но предлагаемые авторами решения~--- первые среди точных.}

\KW{ориентированный граф из цепей и~циклов; преобразование графов с~минимальной 
ценой; точное линейное решение; условие на графы; условие на цены; условно кратчайшее 
решение}

\DOI{10.14357/19922264170107}  


\vskip 10pt plus 9pt minus 6pt

\thispagestyle{headings}

\begin{multicols}{2}

\label{st\stat}

  \section{Введение. Постановка задачи}
   
  \subsection{CC-графы}
  
  В работе рассмотрена следующая  
ком\-би\-на\-тор\-но-оп\-ти\-ми\-за\-ци\-он\-ная задача. Назовем  
\textit{СС-гра\-фом} ориентированный граф, состоящий из любого чис\-ла 
дизъюнктных цепей и~циклов, включая петли, реб\-рам которого приписаны без 
повторений натуральные числа (\textit{имена} ре\-бер). Цепи и~циклы являются 
(без учета ориентации) компонентами связности такого графа, которые будем 
называть \textit{компонента\-ми}. Фиксирован список операций, которые 
преобразуют один СС-граф в~другой такой же граф. 

Можно рассматривать более 
общий случай графов и~любой список операций, но в~длительной истории 
исследования этой задачи (по разным, прежде всего прикладным, причинам) 
сформировалось указанное определение графа и~тот список операций, который 
приведен в~подразд.~1.2. Каж\-дой операции приписано число, которое 
называется ее \textit{ценой}. В~прикладных задачах цены являются строго 
положительными рациональными числами, но, разумеется, теоретически их 
можно считать натуральными числами. Любой последовательности операций, 
применяемых друг за другом, начиная с~данного СС-гра\-фа~$a$ и~заканчивая 
некоторым результирующим СС-гра\-фом~$b$, приписывается 
\textit{суммарная цена}~--- сумма цен всех операций в~этой 
последовательности. 

\begin{figure*}[b] %fig1
\vspace*{12pt}
\begin{center}
\mbox{%
\epsfxsize=101.723mm
\epsfbox{lub-1.eps}
}
\end{center}
\vspace*{-9pt}
  \Caption{Четыре стандартные операции над CC-графом:
  (\textit{а})~двойная переклейка;
  (\textit{б})~полуторная переклейка;
  (\textit{в})~разрез и~склейка}
  %\end{figure*} 
  %\begin{figure*} %fig2
\vspace*{12pt}
\begin{center}
\mbox{%
\epsfxsize=95.006mm
\epsfbox{lub-2.eps}
}
\end{center}
\vspace*{-9pt}
  \Caption{Две дополнительные операции над CC-графом:
  удаление и~вставка}
  \end{figure*}
  
  Итак, пусть даны два СС-гра\-фа~$a$ и~$b$. Требуется найти минимальную 
по функционалу суммарной цены последовательность операций, которая 
преобразует~$a$ в~$b$. Такую последовательность называют 
\textit{кратчайшей}, а~ее цену~--- \textit{кратчайшей ценой}. Предполагается, 
хотя это не доказано, что задача на\-хож\-де\-ния кратчайшей последовательности 
или кратчайшей цены для переменных~$a$, $b$ и~переменных цен операций 
является NP-труд\-ной. Задача остается таковой, если фиксировать 
произвольные (случайные) цены операций. Поскольку практический интерес 
представляют линейные или, во всяком случае, полиномиальные алгоритмы 
низкой степени, для их поиска приходится накладывать условия на 
соотношение цен или на вид графов. В~части второго популярно такое 
ограничение: в~последовательности операций, которая преобразует~$a$ в~$b$, 
включая и~сами~$a$ и~$b$, присутствует один и~тот же постоянный набор 
имен. Задачу с~этим ограничением назовем задачей с~\textit{постоянным  
со\-ста\-вом} (имен ребер). В отсутствие этого ограничения задачу назовем 
задачей с~\textit{переменным составом}. В~разд.~2 приводится схема 
линейного по времени и~памяти алгоритма ее решения в~трех случаях: два 
относятся к~постоянному составу и~один к~переменному составу. Все нюансы 
работы алгоритма, со\-про\-вож\-да\-емые рисунками, а также детали доказательств 
приведены в~\cite{1-gor, 2-gor}.
  
  \subsection{Операции над CC-графами}
  
   Фиксируются следующие операции, на\-зы\-ва\-емые \textit{стандартными}:
   \begin{itemize} 
\item разрезать две вершины, имеющиеся в~графе, и~по-но\-во\-му отождествить 
(склеить) четыре образовавшихся края (\textit{двойная переклейка}) 
(рис.~1,\,\textit{a});
\item разрезать вершину и~по-но\-во\-му отождествить (склеить) 
один образовавшийся край с~ка\-ким-то свободным краем в~графе 
(\textit{полуторная переклейка}) (рис.~1,\,\textit{б}); 
\item разрезать вершину или 
отождествить два свободных края (одинарные переклейки~--- \textit{разрез} 
и~\textit{склейка}) (рис.~1,\,\textit{в}).
\end{itemize}
  

  
  Также фиксируются две \textit{дополнительные операции}, называемые 
соответственно \textit{вставкой} и~\textit{удалением}: до\-ба\-вить/уда\-лить 
цепь~$X$ в~граф или из графа (рис.~2).
  

  
  Для вставки это означает: добавить в~граф саму цепь~$X$ или ее же 
с~отождествлением ее концов, т.\,е.\ добавить новую цепь или новый цикл; или 
добавить цепь~$X$ в~цепь, уже имеющуюся в~графе, вместо ее конца или ее 
внутренней вершины; аналогично для вставки в~цикл. И~симметрично для 
операции удаления. При этом добавлять можно цепь, имена которой содержатся 
в~$b\backslash a$, а удалять можно цепь, имена которой содержатся 
в~$a\backslash b$. В~\cite{3-gor} предложенный авторами алгоритм был 
применен в~конкретном прикладном исследовании, и~там содержатся более 
подробные, чем в~подразд.~1.3, сведения по истории задачи, однако содержание 
работы~\cite{3-gor} не используется в~данной статье.
  
  \subsection{Обзор непосредственно примыкающей литературы}
  
  Приведем несколько математических результатов других авторов по этой 
задаче. Таких результатов немного. В~случае \textit{постоянного} состава 
и~\textit{различных} цен алгоритм решения с~его математическим 
обосно\-ва\-ни\-ем не был предложен. В~работах~[4, 5] рассматривается сразу 
\textit{переменный состав}. В~\cite{4-gor} на цены накладывается условие: 
у~всех стандартных операций они равны~1, а~у~операций встав-\linebreak ки и~удаления 
равны и~не больше~1. В~\cite{5-gor} на цены накладывается другое, более 
слабое условие~--- опущено <<не больше~1>>, но зато рассматриваются  
CC-гра\-фы только из циклов. По существу, последнее условие эквивалентно 
тому, что из стандартных операций допускается только двойная переклейка. 
Насколько авторы понимают, результат в~\cite{4-gor} не содержит корректных 
доказательств, так что указанный случай не получил обоснования. Подходы, 
которые предложены в~[4,~5] (т.\,е.\ определения вспомогательных графов~--- 
прием, который используется во всех работах на эту тему, включая и~данную), 
отличаются от предлагаемого авторами статьи.
{\looseness=1

} 
  
  В~\cite{2-gor} авторы описали линейный по времени и~памяти алгоритм 
построения кратчайшей последовательности операций, преобразующих один 
CC-граф~$a$ в~другой CC-граф~$b$ для случая \textit{переменного} состава, 
если цены всех операций равны. Если все цены равны, кратчайшую 
последовательность называют \textit{минимальной}, а~кратчайшую цену~--- 
\textit{минимальной ценой}. Конечно, главный интерес представляет общий 
случай задачи, в~котором цены не равны. 

  
  Как и~в~\cite{2-gor}, в~разд.~2 изложение строится по такому плану. 
Определяется понятие общего графа $a\hm+b$, иное, нежели  
в~\cite{4-gor, 5-gor}; показано, что исходная задача эквивалентна приведению 
$a\hm+b$ к~специальному виду, который называется \textit{финальным}, 
аналогами операций, которые определены в~подразд.~1.2. Затем описывается 
алгоритм и~приводится доказательство его точности (неэвристичности, 
корректности), т.\,е.\ доказательство того, что он действительно находит 
минимум суммарной цены. Как отмечалось выше, некоторые технически 
громоздкие детали в~описании и~доказательстве приведены в~\cite{1-gor}. 
Вместо <<неэвристический алгоритм>> или <<неэвристическое решение>> 
иногда говорят \textit{точный алгоритм} или \textit{точное решение} 
соответственно.

  
  В исследованиях этой задачи используется различная терминология. Так, 
в~\cite{2-gor} CC-граф называется \textit{структурой}; для согласования с~этой 
работой будем далее использовать последний, более привычный термин. 
В~\cite{3-gor} и~в~других работах CC-граф (структура) называется 
хромосомной структурой, его реб\-ра называются генами, а~компоненты~--- 
хромосомами. Это связано с~тем, что задача возникла в~контексте 
биоинформатики, где для ее решения разработано большое число более или 
менее эвристических алгоритмов. В~\cite{3-gor} приводится краткий обзор по 
истории исследований задачи. 

  
  Напомним, что минимальная последовательность и~минимальная цена 
являются частными случаями кратчайшей последовательности и~кратчайшей 
цены. 
  
  \section{Решение задачи }
  
  \subsection{Общий граф и~идея алгоритма}
  
  \textit{Общий граф} $a\hm+b$ двух структур~$a$ и~$b$ имеет следующие 
вершины. \textit{Обычные} вершины~--- имена краев одноименных ребер в~$a$ 
и~$b$; например, начало реб\-ра с~именем~3 будет иметь имя~3$_1$. 
И~\textit{особые вершины}~--- максимальные по включению связные участки 
из ребер, принадлежащих лишь одной из структур, которые называют 
\textit{блоками}. Блок принадлежит одной из структур, и~соответствующая 
особая вершина помечается как $a$- или $b$-вер\-ши\-на. Реб\-ра общего графа 
следующие. \textit{Обычное} реб\-ро соединяет две обычных вершины, если 
соответствующие им края отождествлены (склеены) в~$a$ или в~$b$. 
А~\textit{особое} реб\-ро соединяет обычную вершину с~особой, если в~$a$ или 
в~$b$ край, со\-от\-вет\-ст\-ву\-ющий обычной вершине, отож\-дествлен (\textit{склеен}) 
с~краем блока, со\-от\-вет\-ст\-ву\-юще\-го особой вершине. Такое реб\-ро помечается как 
$a$- или $b$-реб\-ро. \textit{Петля} в~$a\hm+b$ соответствует циклу, который 
является блоком; иными словами, особая вершина этого блока соединяется 
с~собой. \textit{Висячим} называется реб\-ро, инцидентное особой вершине 
степени~1. Пример двух структур и~их общего графа приведен в~\cite{3-gor} на 
рис.~1 и~2. Таким образом, общий граф несет информацию о~склейках 
одновременно в~$a$\linebreak и~в~$b$.
  
  Общий граф~--- неориентированный, он состоит из связных компонент~--- 
также цепей и~циклов. Невисячие особые реб\-ра присутствуют в~нем парами~--- 
реб\-ра\-ми, инцидентными одной особой вершине; такую пару удобно считать за 
одно двойное ребро. Поэтому \textit{размером компоненты} назовем сумму 
в~ней числа обычных ребер с~половиной числа особых невисячих ребер 
(в~\cite{2-gor} эта величина названа длиной компоненты, что вызывает 
путаницу с~обычной длиной цепи или цикла). Для изолированных обычных 
вершин и~петель размер считаем равным~0, для изолированных особых вершин 
(не петель)~--- равным~$-1$. Общий граф называется 
\textit{финального вида}, если каждая его компонента~--- изолированная 
обычная вершина или цикл без особых ребер размера (или в~данном случае то 
же самое~--- длины)~2, одно ребро из~$a$ и~другое из~$b$. 

\begin{figure*}[b] %fig3
\vspace*{12pt}
\begin{center}
\mbox{%
\epsfxsize=82.146mm
\epsfbox{lub-3.eps}
}
\end{center}
\vspace*{-9pt}
\Caption{Операции, разрешенные над общим графом:
(\textit{а})~двойная переклейка;
(\textit{б})~полуторная переклейка;
(\textit{в})~разрез и~склейка;
(\textit{г})~$a$- или $b$-удаление. Большой кружок показывает особую 
вершину. Отметим: операция вставки оказывается ненужной, но зато операция удаления 
применяется в~двух вариантах: $a$-уда\-ле\-ния и~$b$-уда\-ле\-ния особой вершины; 
первая~--- 
по цене удаления, вторая~--- по цене вставки}
\end{figure*}
  
  Эти определения приведены в~\cite{2-gor, 3-gor} и~частично в~\cite{6-gor} 
и~здесь повторяются для удобства читателя. 
  
  В~\cite{2-gor} доказано: для любых структур~$a$ и~$b$ существует 
кратчайшая последовательность, в~которой все удаления предшествуют всем 
вставкам и~все операции сохраняют блоки без изменения. Хотя там считалось, 
что цены всех операций равны, легко проверить, что приведенное 
доказательство сохраняется без изменения, если равны только цены\linebreak 
стандартных операций. Действительно, в~том доказательстве произвольная 
кратчайшая последовательность преобразуется в~последовательность 
указанного вида, при этом стандартная операция перехо\-дит в~стандартную, 
удаление~--- в~удаление, вставка~--- во вставку. Поэтому как там, так и~здесь 
суммарная цена (иногда будем говорить~--- цена) последовательности не 
меняется.


  
  Из этого \textit{утверждения} следует: в~предположении одинаковых цен 
стандартных операций задача поиска кратчайшей последовательности для 
структур~$a$ и~$b$ эквивалентна задаче приведения этих структур  
к~ка\-кой-ни\-будь одной структуре~$c$ двумя суммарно кратчайшими 
последовательностями, причем вставка не используется, а~цена удаления, 
применяемого в~преобразованиях~$b$, равна цене вставки. 

Действительно, 
если~$c$~--- структура в~кратчайшей последовательности, полученная после 
выполнения всех удалений и~до всех вставок, то можно преобразовать к~ней 
структуру~$a$ (прямыми операциями) и~структуру~$b$ (обратными 
операциями). А~последняя задача эквивалентна задаче преобразования общего 
графа~$a\hm+b$ к~финальному виду~$c\hm+c$ следующими аналогами 
исходных операций (рис.~3).
  

  
  \textit{Двойная переклейка} (рис.~3,\,\textit{a}): удаление двух одинаково 
помеченных ребер общего графа и~соединение четырех образовавшихся концов 
двумя новыми неинцидентными реб\-ра\-ми с~той же пометкой. Если при этом 
образуется ребро с~особыми концами (оба относятся к~$a$ или оба к~$b$), то 
оно заменяется одной особой вершиной, которой приписано объединение 
блоков двух исходных особых вершин. 

\textit{Полуторная переклейка}
(рис.~3,\,\textit{б}): удаление реб\-ра общего графа и~соединение ребром с~той же 
пометкой, скажем~$a$, одного из его концов с~обычной вершиной, не 
инцидентной ребру с~этой пометкой, или с~особой вершиной степени не 
больше~1 с~той же пометкой (с~возможным последующим отождествлением 
двух особых вершин). 

\textit{Склейка} (рис.~3,\,\textit{в}): добавление реб\-ра 
(скажем, с~пометкой~$a$) между вершинами, каждая из которых является или 
обычной, не инцидентной ребру с~пометкой~$a$, или особой, степени не 
больше~1, с~той же пометкой (с~возможным последующим отождествлением 
двух особых вершин). 

\textit{Разрез} (рис.~3,\,\textit{в}): удаление любого 
реб\-ра. 

\textit{Удаление особой вершины} (рис.~3,\,\textit{г}): если ее степень~2, 
то она удаляется и~инцидентные ей реб\-ра склеиваются в~одно реб\-ро с~той же 
пометкой; если ее степень~1, то она удаляется вместе с~инцидентным ей 
ребром; если ее степень~0 или это пет\-ля, то вершина с~пет\-лей удаляются. 
Удаление особой $a$-вер\-ши\-ны получает цену операции удаления, удаление 
особой $b$-вер\-ши\-ны~--- цену операции встав\-ки. Условимся далее 
в~выражении особая~$a$- или $b$-вер\-ши\-на опускать слово <<особая>>.
  
  Забегая вперед, опишем \textit{идею предлагаемого алгоритма}. В~случае 
\textit{постоянного} состава в~общем графе имеются лишь обычные вершины 
и~реб\-ра. Алгоритм приводит его к~финальному виду в~два этапа. Первый 
  этап~--- двойными переклейками разбить все циклы на циклы длины~2. 
Второй этап~--- обработка цепей: если цена двойной переклейки меньше цены 
полуторной, замкнуть цепи в~циклы и~обработать их, как на этапе~1. Иначе 
нужно полуторными переклейками от цепи пошагово отщеплять циклы 
длины~2.
  
  В случае \textit{переменного} состава алгоритм из компонент вырезает 
обычные реб\-ра, замыкая их в~циклы длины~2. Затем обрабатываются цепи. Это 
связано с~тем, что совместная обработка цепей позволяет экономить число 
операций по сравнению с~тем их числом, которое получилось бы при обработке 
каждой цепи в~отдельности; это~--- основная идея алгоритма. Совместная 
обработка цепей описана в~\cite{2-gor}, где доказано, что она приводит 
к~максимально возможной экономии числа операций, т.\,е.\ к~минимизации 
числа операций. 

Ниже в~описании алгоритма совместная обработка цепей 
выполняется на шаге~3, каждый пункт которого соответствует определенной 
совместной обработке (можно сказать~--- взаимодействию) цепей. После 
шага~3 общий граф может еще содержать цепи, а также в~нем остаются 
исходные циклы. Между этими компонентами уже невозможны 
взаимодействия, которые экономили бы число операций. Но возможны 
взаимодействия, заменяющие дорогое удаление $b$-вер\-ши\-ны на операцию 
с~меньшей ценой. Эти взаимодействия описаны на шаге~4 алгоритма.
  
  \subsection{Приведение общего графа в~случае постоянного состава 
и~разных цен операций}

\vspace*{-2pt}
  
   Рассмотрим случай постоянного состава, вставки и~удаления отсутствуют, 
цены операций различны. Точнее, предполагается, что цены операций 
удовлетворяют одному из двух условий: $c_2\hm\leq c_1\hm\leq 
c_1^\prime\hm\leq c_{1{,}5}$ (\textit{циклический} вариант) и~$c_1\hm\leq 
c_1^\prime\hm\leq c_{1{,}5}\hm\leq c_2$ (\textit{линейный} вариант). Здесь 
указаны \mbox{соотношения} между ценами разреза~$c_1$, склейки~$c_1^\prime$, 
полуторной переклейки~$c_{1{,}5}$, двойной переклейки~$c_2$. 
  
  В этом случае предлагается решение несколько \textbf{измененной задачи}: 
кратчайшая последовательность ищется среди всех минимальных 
последовательностей. Она называется \textit{условно кратчайшей}. Решение 
задачи в~этом смысле будем называть \textit{условной оптимизацией}. Авторы 
не знают, существует ли полиномиальный по времени алгоритм решения 
безусловной задачи даже при одном из указанных соотношений цен, если 
только не все цены равны. Конечно, если они равны, то \textit{условная} 
оптимизация совпадает с~\textit{безусловной}, т.\,е.\ с~решением исходной 
задачи. 
  
  В рассматриваемом случае общий граф состоит из циклов и~цепей, в~которых 
чередуются~$a$- и~$b$-реб\-ра. В~случае постоянного состава 
\textit{качеством} $H(a+b)$ общего графа $a\hm+b$ назовем число циклов, 
сложенное с~половиной числа четных цепей в~нем. Четной называется цепь 
с~четным числом ребер, а~так\-же цепь размера ноль; цепи нечетного размера не 
учитываются; понятия размера и~длины в~этом пункте совпадают. Пусть 
структуры~$a$ и~$b$ имеют по~$n$~ребер. Для построения минимальной 
последовательности решающее значение имеет возрастание качества от 
значения $H(a+b)$ до значения~$n$ на~$+1$ при выполнении каждой 
операции; таким образом, минимальную длину можно указать сразу: она равна 
$n\hm- H(a\hm+b)$. 
  
  \smallskip
    
  \noindent
  \textbf{Лемма~1.}
  \begin{enumerate}[1.]
\item  \textit{Каждая стандартная операция изменяет качество общего графа 
на~$0$ или}~$\pm1$. %\\[-15pt]
  \item \textit{Для нефинального графа существует операция, увеличивающая 
его качество на}~1. %\\[-15pt]
  \item\textit{Граф $a+b$ финальный, если и~только если $a\hm=b$; для 
финального графа $a\hm+b$ выполняется} $H(a\hm+b)\hm=n$. %\\[-15pt] 
  \item  \textit{Как безусловная, так и~условная задачи для~$a$ и~$b$ 
эквивалентны соответствующим задачам о~приведении общего 
графа~$a\hm+b$ к~финальному виду}. %\\[-15pt]
\item \textit{Существует последовательность операций, преобразующая 
$a\hm+b$ к~финальному виду, на каждом шаге которой качество 
увеличивается ровно на~$1$; ее длина равна} $k \hm= n\hm- H(a\hm+b)$.
  \item \textit{Минимальная длина равна}~$k$.
  \item \textit{Минимальными последовательностями для $a\hm+b$ являются 
в~точности те, у~которых каждая операция увеличивает качество общего 
графа на~$1$. Их длины равны}~$k$.
  \end{enumerate}
  
  \noindent
  Простое д\,о\,к\,а\,з\,а\,т\,е\,л\,ь\,с\,т\,в\,о\ леммы~1 приведено  
в~\cite[п.~3]{1-gor}.
  
  \smallskip
  
  Опишем точный линейный алгоритм приведения к~финальному виду 
в~случаях циклического и~линейного соотношений цен. Пункт~3 в~\cite{1-gor} 
содержит рисунки, наглядно поясняющие его работу. Для \textbf{циклического 
варианта} он состоит из трех шагов.
  \begin{description}
  \item[Шаг~1.] Если имеется цикл длины, строго большей двух, двойной 
переклейкой разбиваем его на два цикла, один из которых имеет длину~2.
  \item[Шаг~2.] Склейкой каждую нечетную цепь замыкаем в~цикл, после чего 
применяем шаг~1.
  \item[Шаг~3.] Полуторной переклейкой каждую ненулевую четную цепь 
замыкаем в~цикл, один край цепи становится нулевой цепью. Затем применяем 
шаг~1.
  \end{description}
  
  Алгоритм решения условной задачи для \textbf{линейного варианта} состоит 
также из трех шагов.
  \begin{description}
  \item[Шаг~1.] Тот же, что и~в предыдущем алгоритме. 
  \item[Шаг~2.] Разрезом от каждой нечетной цепи отделяем крайнюю 
вершину, получаем четную цепь на~1~меньшей длины и~нулевую цепь.
  \item[Шаг~3.] Полуторной переклейкой каждую ненулевую четную цепь 
укорачиваем на~2~реб\-ра и~замыкаем два ее крайних реб\-ра в~цикл, пока 
в~общем графе не останется ненулевых цепей.
  \end{description}
  
  Если ни один шаг не применим, то общий граф уже имеет финальный вид 
и~к~нему применяется пустая последовательность операций.~$\square$
  
  Отметим: от~\cite{4-gor, 5-gor} приведенное ниже доказательство теорем~1 
и~2 отличается другими условиями на цены, использованием другого 
вспомогательного графа, меньшего размера, и~индукцией по величине~$C(G)$ 
общего графа, которая и~со\-став\-ля\-ет суть приводимых доказательств (не говоря 
об отсутствии полного доказательства в~\cite{4-gor, 5-gor}).
  
  \smallskip
  
  \noindent
  \textbf{Теорема~1.}\ \textit{Указанные линейные алгоритмы точно решают 
задачу условной оптимизации для циклического и~линейного вариантов цен.}
  
\columnbreak


  \noindent
  С\,х\,е\,м\,а\ д\,о\,к\,а\,з\,а\,т\,е\,л\,ь\,с\,т\,в\,а\,.\ \ Минимальность 
полученной последовательности следует из леммы~1. Из нее же следует 
линейность алгоритма по вре\-мени. 
  
  Докажем, что полученная последовательность~--- кратчайшая. Для этого 
выразим суммарную цену $c(G)$ в~полученной алгоритмом последовательности 
через числовые характеристики графа~$G$ (подробности приведены 
в~\cite[п.~3]{1-gor}). Кратчайшую цену для приведения графа~$G$ 
к~финальному виду \textit{обозначим}~$C(G)$. Индукцией по 
величине~$C(G)$ покажем, что для всех графов~$G$ выполняется неравенство 
$c(G)\hm\leq C(G)$. Отсюда $c(G)\hm=C(G)$, что и~требуется. Число вершин 
в~графе~$G$ фиксировано, поэтому множество минимальных цен конечно. 
Индукция идет по естественному порядку в~этом множестве цен. Если 
$C(G)\hm= 0$, то граф~$G$ финального вида и~$c(G)\hm=0$.
  
  \textbf{Индуктивный шаг.} Пусть для всех графов~$G^\prime$, у~которых 
$C(G^\prime)\hm<C(G)$, выполняется неравенство $c(G)\hm\leq C(G)$. 
Докажем его для~$G$. Рассмотрим приводящую последовательность для~$G$. 
Обозначим через~$o$ ее первую операцию, $c(o)$~--- ее цену, $o(G)$~--- 
результат ее применения к~$G$. Достаточно проверить неравенство 
$c(o^\prime)\hm\geq c(G)\hm- c(o^\prime(G))$ для каждой операции~$o^\prime$. 
Действительно, по предположению индукции имеем: $c(o(G))\hm\leq C(o(G))$. 
Отсюда $c(G)\hm\leq c(o(G))\hm+ c(o) \hm\leq C(o(G))\hm +c(o)\hm= C(G)$. 
Подробности проверки приведены в~\cite[п.~3]{1-gor}.~$\square$

%\vspace*{-9pt}
  
  \subsection{Приведение общего графа в~случае переменного состава 
и~разных цен операций}
  
   Дан общий граф $a+b$ и~число~$\varepsilon$, $0\hm\leq\varepsilon\hm\leq1$. 
Разрешены все операции, т.\,е.\ в~силу утверждения в~подразд.~1.1 стандартные 
операции, удаления~$a$- и~$b$-вер\-шин (особых вершин с~пометкой~$a$ 
или~$b$). Пусть цены стандартных операций и~$a$-уда\-ле\-ния равны~1, а цена 
$b$-уда\-ле\-ния вершины равна $1\hm+\varepsilon$. Термин \textit{конец}, 
естественно, относится к~концу реб\-ра или к~изолированной вершине в~общем 
графе. 
  
  Предлагаемый алгоритм компьютерно тестировался в~общем случае, если 
цена $b$-уда\-ле\-ния больше цены всех других операций. Как правило, 
алгоритм находил ответ, близкий к~кратчайшей последовательности. Эта более 
общая ситуация здесь не рассматривается, но с~учетом возможного 
эвристического использования в~описание алгоритма, которое приведено ниже, 
включены соответствующие пояснения; они не используются в~доказательстве, 
которое также приводится ниже. 
  
\pagebreak 
  
  \textbf{Краткое описание алгоритма}
  
  \begin{description}
  \item[Шаг~1.] Удалить особые $a$-петли. 
  \item[Шаг~2.] Вырезать все обычные реб\-ра, не входящие в~2-цик\-лы (т.\,е.\ 
циклы размера~2), замыкая их в~финальные~2-цик\-лы двойной (если ребро не 
крайнее) или полуторной (если оно крайнее) переклейками или склейкой (если 
оно изолированное). В~\cite[п.~4]{1-gor} содержится подробное описание 
работы алгоритма на шагах~2 и~3 с~рисунками.
  \item[Шаг~3.] Фактически этот шаг тот же, что и~в~\cite{2-gor} (более 
подробно он описан в~\cite[п.~4]{1-gor}). Напомним его смысл. В~множестве 
цепей общего графа (после шага~2) выделяются небольшие попарно 
непересекающиеся подмножества мощ\-ности от~2 до~4. Внутри каждого 
подмножества~$M$ производится ($\vert M\vert \hm-1$) операций между 
цепями (взаимодействий) так, что если каждую цепь из~$M$ приводить 
к~финальному виду автономно (т.\,е.\ без взаимодействий с~другими 
компонентами), то число требуемых операций будет строго больше числа 
операций, тре\-бу\-емых, если сначала провести данное взаимодействие. 
Доказывается, что описанное множество взаимодействий дает максимально 
возможную экономию числа операций.
  \item[Шаг~4.] На этом шаге в~определенном порядке производятся 
взаимодействия между связными компонентами общего графа. Каждое 
взаимодействие производится до тех пор, пока есть компоненты, которые могут 
служить его аргументами. Эти взаимодействия не уменьшают общее число 
операций (точнее, сохраняют его), но позволяют заменить <<дорогую>> 
операцию удаления $b$-вер\-ши\-ны на другую, более дешевую операцию. 
Например, если удалить две $b$-пет\-ли по отдельности, будет произведено две 
операции удаления $b$-вер\-ши\-ны, если же сначала двойной переклейкой 
слить эти две петли в~одну (это частный случай взаимодействия~4.1  
из~\cite{1-gor}), то одно удаление заменится на двойную переклейку. Подробно 
шаг~4 описан в~\cite[п.~4]{1-gor}.
  \item[Шаг~5.] Удаляем изолированные особые вершины и~петли. Из 
оставшихся цепей удаляем особые вершины. Из циклов размера, большего~2, 
вырезаем~2-цик\-лы так, чтобы происходило отож\-де\-ст\-вле\-ние двух  
$b$-вер\-шин (соответственно,\linebreak в~2-цикл включается $a$-вер\-ши\-на).  
Из~2-цик\-лов удаляем особые вершины.
  \end{description}
  
  Конец описания алгоритма.~$\square$
  \smallskip
  
  Д\,о\,к\,а\,ж\,е\,м\ теорему о~минимальности суммарной цены 
последовательности операций, которая получается в~алгоритме, т.\,е.\ 
о~точ\-ности (корректности) алгоритма. 
  
  Пусть $B^\prime$~--- число циклов в~графе $a\hm+b$, содержащих  
$b$-вер\-ши\-ну, но не содержащих $a$-вер\-ши\-ну (назовем их  
$b$-цик\-ла\-ми). Напомним обозначения из~\cite{2-gor}: $B$~--- число особых 
вершин в~$a\hm+b$; $S$~--- сумма целых частей половин числа ребер (назовем 
число \textit{длиной}) максимальных отрезков (\textit{сегментов}) в~$a\hm+b$, 
которые состоят из обычных ребер, плюс число нечетных (т.\,е.\ нечетной 
длины) крайних сегментов минус число циклических сегментов. 
\textit{Крайним} называется сегмент, расположенный с~краю цепи, включая 
и~случай целой цепи. Обычной называется пара, состоящая из одной из 
стандартных операций вместе с~ее аргументом, результат которой не меняет 
число особых вершин. Далее <<обычная>> относится к~операции, а ее аргумент 
подразумевается заданным. \textit{Дефект} цепи (или цикла) равен 
минимальному числу обычных операций в~последовательности, которая 
приводит ее (или его) к~финальному виду, не считая вырезания обычных ребер 
на шаге~2; в~последовательности могут встречаться и~операции с~их 
аргументами, которые не являются обычными; назовем их \textit{особыми}. 
В~\cite{2-gor} приведена зависимость дефекта от типа компоненты. 
Обозначим~$D$ сумму дефектов компонент графа $a\hm+b$. Обозначим~$P$ 
разность величин~$D$, вычисленных до и~после применения шага~3 
алгоритма. Заметим, что в~любой последовательности операций, 
финализирующих общий граф, число особых операций равно числу особых 
вершин в~нем, так что экономия числа операций может относиться лишь 
к~обычным операциям. Поскольку все операции на шаге~3 особые, 
величина~$P$ равна числу операций, сэкономленных на шаге~3. 
Величина~$\varepsilon$ определена выше. Пусть $C\hm=B\hm+S\hm+D\hm- 
P \hm+\varepsilon(B^\prime\hm+1)$. 
  
  \smallskip
  
  \noindent
  \textbf{Теорема~2.}\ \textit{Алгоритм строит последовательность 
операций, суммарная цена которой равна одному из трех значений $C\hm-
\varepsilon$, $C$, $C\hm+\varepsilon$. Минимально возможная суммарная цена 
последовательности операций, приводящей граф $a\hm+b$ к~финальному виду, 
также равна одному из этих значений. Время работы алгоритма линейное по 
порядку.}
  \smallskip
  
  В~доказательстве теоремы будут использованы нижеследующие леммы~2 
и~3.
  
  \smallskip
  
  \noindent
  \textbf{Лемма~2.}\ \textit{После выполнения шага~$4$ остается~$0$, $1$ 
или~$2$ связных компоненты, имеющих $b$-вер\-ши\-ну и~не являющихся 
исходными $b$-цик\-лами.}
  
  Простое доказательство леммы приведено в~\cite[п.~4]{1-gor} (там это 
лемма~3). 
  
  \smallskip
  
  \noindent
  \textbf{Лемма~3.} \textit{Число обычных операций в~алгоритме равно} 
$S\hm+D\hm-P$.
  
  \smallskip
  
  \noindent
  Д\,о\,к\,а\,з\,а\,т\,е\,л\,ь\,с\,т\,в\,о\,.\ \ Напомним~\cite{2-gor}, что 
минимальное число обычных операций, требуемых для приведения компоненты 
(после шага~2) к~финальному виду без использования других компонент, равно 
ее дефекту. Настоящий алгоритм отличается от описанного в~\cite{2-gor} 
наличием шага~4. Любая операция шага~4 либо особая и~не меняющая дефект 
результата по сравнению с~суммарным дефектом аргументов, либо обычная 
и~уменьшающая его на~1. Поэтому обычных операций в~алгоритме столько же, 
сколько и~раньше, т.\,е.\ $S\hm+D\hm- P$.~$\square$
  
  \smallskip
  
  \noindent
  Д\,о\,к\,а\,з\,а\,т\,е\,л\,ь\,с\,т\,в\,о\ \ теоремы~2. На шаге~5 для каждой 
компоненты, имеющей $b$-вер\-ши\-ны, применяется ровно одна операция 
удаления $b$-вер\-ши\-ны. По лемме~2 общее число таких операций равно 
$B^\prime\hm+n$, где~$n$ равно~0, 1 или~2. Всего особых операций~$B$. 
В~силу леммы~3 суммарная цена операций алгоритма равна 
$(1\hm+\varepsilon)(B^\prime\hm+n)\hm+(B\hm- B^\prime\hm- n)\hm+(S\hm+D\hm- 
P)\hm=B\hm+S\hm+ D\hm- P\hm+\varepsilon(B^\prime \hm+n)$, откуда следует 
первое утверждение теоремы. 
  
  Второе утверждение теоремы докажем индукцией по минимальной 
суммарной цене~$M$ операций, приводящих общий граф к~финальному виду; 
имеется лишь конечное число возможных значений~$M$ на любом 
ограниченном отрезке, которые рассматриваем по их возрастанию. Рассуждая 
так же, как и~в~доказательстве теоремы~1, видим, что достаточно для любой 
операции~$o$, примененной к~произвольному общему графу~$G$, проверить, 
что цена $c(o)$ операции~$o$ не меньше $C(G)\hm- C(o(G))$, где $C(G)$~--- 
величина~$C$, определенная в~формулировке теоремы~2. Подробности 
проверки приведены в~\cite[п.~4]{1-gor}.~$\square$
  
  \smallskip
  
  \noindent
  \textbf{Следствие.}\ Цена последовательности операций, которую строит 
описанный алгоритм, отличается от цены кратчайшей последовательности не 
более чем на~$\varepsilon$.
  
  \smallskip
  \noindent
  Д\,о\,к\,а\,з\,а\,т\,е\,л\,ь\,с\,т\,в\,о\ \ приведено в~\cite{7-gor}.~$\square$
  
  \section{Обобщение: задача с~повторением имен}
  
  \subsection{Постановка задачи}
  
  Важное в~прикладных вопросах обобщение рассмот\-ренной выше задачи 
состоит в~том, что в~структурах разрешается повторение имен. Это обобщение 
назовем \textit{задачей с~повторениями} (или по историческим причинам 
говорят: задачей с~паралогами). Пусть~$a$ и~$b$~--- такие структуры. 
Например, имеются в~$a$ три реб\-ра с~именем~$k$ и~в~$b$ два реб\-ра с~тем же 
именем~$k$. Нужно найти биекцию меньшего из этих двух множеств ре\-бер 
в~большее (для данного имени~$k$); и~аналогично для каждого имени~$k$, 
если ему соответствуют два таких множества, одно в~$a$ и~другое в~$b$, 
с~разным числом элементов. Итак, нужно найти семейство биекций, 
индексированное~$k$, при котором кратчайшая цена достигает минимального 
значения. Более детально на этом примере: нужно приписать этим пяти реб\-рам 
индекс~$i$ к~их имени~$k$ (получатся \textit{полные имена}, которые имеют 
вид~$k.i$, где~$i$ меняется от~1 до~3) так, чтобы с~новыми именами у всех 
повторяющихся ребер достигалось минимальное значение кратчайшей цены 
кратчайшего преобразования~$a$ в~$b$. Индекс~$i$ определяет частичное 
соответствие между бывшими одноименными реб\-ра\-ми в~$a$ и~$b$ 
и,~в~частности, определяет, какие реб\-ра общие и~какие особые для этих 
структур. Например, эти три реб\-ра можно индексировать~$k.1$, $k.2$, $k.3$, 
а~два других реб\-ра индексировать~$k.2$ и~$k.3$, тогда ребро~$k.1$ особое, 
а~остальные реб\-ра общие. Полные имена позволяют перейти от \textit{задачи 
с~повторениями к~задаче без повторений} (имен), последняя рассматривалась 
в~разд.~1 и~2.
  
  В силу NP-труд\-ности задачи с~повторениями, нельзя найти точный 
полиномиальный алгоритм ее решения. Однако ниже будет показано, как 
математически строго свести ее к~задаче целочисленного линейного 
программирования (ЦЛП). Как известно, для задач ЦЛП доступны 
компьютерные программы, выдающие, как правило, точное решение за время, 
близкое к~линейному, и~имеются соответствующие математические результаты. 
  
  Для краткости рассмотрим здесь только случай \textit{одинаковых цен всех 
операций}. В~\cite{3-gor} авторы описали сведение задачи с~повторениями 
к~задаче ЦЛП в~случае, если структуры состоят только из циклических 
компонент. При этом число переменных и~ограничений в~соответствующей 
задаче ЦЛП не более чем квадратично от размера исходных структур, что, 
конечно, принципиально важно. Далее будет \textit{описано такое сведение 
в~общем случае с~сохранением той же оценки на число переменных и~число 
ограничений}. 
  
  \subsection{Решение задачи}
  
  В исходных структурах~$a$ и~$b$ выберем произвольно второй индекс 
у~всех повторяющихся имен; структуры с~такими полными именами 
\textit{обозначим}~$a^\prime$ и~$b^\prime$; в~них (полные) имена уже не 
повторяются.
  
  Рассмотрим булевы переменные~$z_{abkij}$, для которых $z_{abkij}\hm=1$, если 
ребро~$k.i$ в~$a^\prime$ по искомой биекции соответствует ребру~$k.j$ 
в~$b^\prime$, иначе $z_{abkij}\hm=0$. Таким образом, значения этих переменных 
определяют соответствие ребер в~$a^\prime$ и~$b^\prime$. Переименуем реб\-ра 
в~$b^\prime$ по этому соответствию, результат обозначим~$a^\prime(z)$ 
и~$b^\prime(z)$. С~по\-мощью ограничений на эти переменные легко выразить 
понятия в~$a^\prime(z)$ и~$b^\prime(z)$: <<особое ребро>> и~<<цикл, 
состоящий из особых ребер>> (назовем его \textit{особым циклом}). Конечно, 
здесь описан только смысл переменной~$z$, который выражен в~рамках задачи 
ЦЛП.
  
  Каждой паре~$s$ различных краев ребер в~$a^\prime$ (или в~$b^\prime$) 
сопоставим булеву переменную~$t_{as}$ (соответственно~$t_{bs}$), ограничения 
на которые обеспечат следующие три свойства у~$a^\prime(z)$ 
(и~у~$b^\prime(z)$): если край из~$s$ принадлежит общему ребру или лежит 
в~особом цикле, то $t_{as}\hm=0$; для каждого края существует не более одного 
края, для которых $t_{as}\hm=1$ на этой паре в~качестве~$s$; для каждого края 
особого реб\-ра, не принадлежащего особому циклу, существует край, для 
которого $t_{as}\hm=1$ на этой паре в~качестве~$s$. И~аналогично для~$t_{bs}$. 
  
  По значениям переменных~$t_{as}$ и~$t_{bs}$ определим новые вершины 
и~реб\-ра в~$a^\prime(z)$ и~$b^\prime(z)$, результат обозначим соответственно 
$a^\prime(z, t)$ и~$b^\prime(z, t)$. Все особые реб\-ра из~$b^\prime(z)$, не 
содержащиеся в~особых циклах, добавим в~$a^\prime(z)$; края новых ребер 
склеим, если $t_{bs}\hm=1$ на этой паре в~качестве~$s$. Аналогично особые реб\-ра 
из~$a^\prime(z)$, не содержащиеся в~особых циклах, добавим в~$b^\prime(z)$; 
их края аналогично склеим, если $t_{as}\hm=1$. Таким образом, в~структурах 
$a^\prime(z, t)$ и~$b^\prime(z, t)$ все реб\-ра общие, кроме принадлежащих 
особым циклам. Как раз эти особые циклы удалим из $a^\prime(z, t)$ 
и~$b^\prime(z, t)$, результат обозначим теми же буквами; получены структуры 
с~постоянным составом. Из условий на~$t_{as}$ и~$t_{bs}$ следует: каждое новое 
ребро входит в~цикл из новых ребер.
  
  \textit{Обозначим} 
$G^\prime\hm=G^\prime(z, t)\hm=a^\prime(z, t)\hm+b^\prime(z, t)$. Для 
графа~$G^\prime$ вычислим значение $C_1\hm+0{,}5C_2$, где~$C_1$ 
и~$C_2$~--- число циклов и~четных цепей в~этом графе соответственно.
  
  Число~$C_1$ вычисляется так же, как в~\cite{3-gor} для общего графа 
вычисляется число~$S_2$ циклов, состоящих из обычных ребер (с~учетом 
цепей и~новых ре\-бер). 
{\looseness=1

}
  
  Для вычисления величины $0{,}5C_2$ каждому краю~$p$~реб\-ра в~$a$ 
и~$b$ сопоставим переменную~$r_p$, принимающую значения~0, 1 или~$-1$. 
Ограничения обеспечат условия: если для пары склеенных краев 
в~$a^\prime(z, t)$ или $b^\prime(z, t)$ одна из переменных рав\-на~1, то вторая 
равна~$-1$, а~если первая переменная рав\-на~0, то вторая рав\-на~0 или~$-1$. 
  
  Переменные $r_p$ будут входить в~минимизиру\-емую функцию~$F$, которую 
определим чуть ниже, с~отрицательным коэффициентом, поэтому они равны~1 
на изолированных вершинах в~$G^\prime$. На вершинах из циклов 
в~$G^\prime$ значения~1 и~$-1$ переменных~$r_p$ чередуются или все эти 
значения нулевые; в~любом случае сумма всех~$r_p$ вдоль цикла равна~0. Эти 
значения чередуются и~на ненулевой четной цепи, причем на ее краях они 
равны~1; так что их сумма вдоль цепи равна~1. На нечетных цепях это 
чередование перемежается с~нулевыми значениями; в~любом случае их сумма 
вдоль такой цепи равна~0. Отсюда вытекает, что полусумма всех 
значений~$r_p$ равна~$C_2$.
  
  Итак, минимизируемая целевая функция \textit{определяется как} 
$F\hm=C_0\hm+n\hm+s_a\hm+s_b\hm- C_1\hm- 0{,}5C_2$, где~$C_0$~--- сумма чисел 
особых циклов в~$a^\prime(z)$ и~$b^\prime(z)$; $n$~--- число (однократно 
учитываемых) общих ребер в~них; $s_a$ и~$s_b$~--- число особых ребер 
в~$a^\prime(z)$ и~$b^\prime(z)$, не входящих в~особые циклы. Значение 
$C_0\hm+n\hm+s_a\hm+s_b$ линейно выражается через введенные 
переменные~$z$. Значение~$F$ линейно выражается через переменные~$z$, 
$t$ и~$r_p$.
  
  Итак, исходная задача сведена к~задаче ЦЛП. Указанная оценка числа 
переменных и~ограничений очевидна. Корректность такого сведения формально 
доказана в~работе авторов, которая пред\-став\-ле\-на в~печать. В~этом 
доказательстве решающий шаг состоит в~том, что минимальное число операций в~последовательности, преобразующей~$a^\prime(z)$ в~$b^\prime(z)$, 
равно~$F$. Действительно, минимальная последовательность, преобразующая 
$a^\prime(z, t)$ в~$b^\prime(z, t)$, имеет длину~$F$, что следует из результата 
в~\cite{2-gor}. Она индуцирует последовательность той же длины, 
преобразующую~$a^\prime(z)$ в~$b^\prime(z)$. И~обратно: кратчайшая 
последовательность, преобразующая~$a^\prime(z)$ в~$b^\prime(z)$, индуцирует 
последовательность той же длины, преобразующую~$a^\prime(z, t)$ 
в~$b^\prime(z, t)$. 
  
  Идея такого соответствия последовательностей состоит в~том, что операции 
удаления участка ребер ставится в~соответствие стандартная операция, 
вырезающая и~зацикливающая этот участок, а~операции вставки участка 
ребер~--- стандартная операция, вставляющая этот зацикленный участок в~то же 
место. Такая идея была предложена в~\cite{8-gor}.
  
{\small\frenchspacing
 {%\baselineskip=10.8pt
 \addcontentsline{toc}{section}{References}
 \begin{thebibliography}{9}
\bibitem{1-gor}
\Au{Горбунов К.\,Ю., Любецкий~В.\,А.} Линейный алгоритм кратчайшей перестройки графов 
при разных ценах операций~// Информационные процессы, 2016. Т.~16. №\,2. C.~223--236.
\bibitem{2-gor}
\Au{Горбунов К.\,Ю., Любецкий~В.\,А.} Линейный алгоритм минимальной перестройки 
структур~// Проблемы передачи информации, 2017  (в печати). Т.~53. Вып.~1.
\bibitem{3-gor}
\Au{Lyubetsky V.\,A., Gershgorin~R.\,A., Seliverstov~A.\,V., Gorbunov~K.\,Yu.} Algorithms for 
reconstruction of chromosomal structures~// BMC Bioinformatics, 2016. Vol.~17. P.~40.1--40.23.
\bibitem{4-gor}
\Au{Da Silva P.\,H., Machado~R., Dantas~S., and Braga~M.\,D.\,V.} DCJ-indel and  
DCJ-substitution distances with distinct operation costs~// Algorithm. Mol. Biol., 2013. 
Vol.~8. P.~21.1--21.15.
\bibitem{5-gor}
\Au{Compeau P.\,E.\,C.} A~generalized cost model for DCJ-indel sorting~// 
Algorithms in bioinformatics~/ Eds. D.\,G.~Brown, B.~Morgenstern.~---
Lecture notes in 
computer science ser.~--- Springer, 2014. Vol.~8701. P.~38--51.
\bibitem{6-gor}
\Au{Горбунов К.\,Ю., Гершгорин~Р.\,А., Любецкий~В.\,А.} Перестройка и~реконструкция 
хромосомных структур~// Молекулярная биология, 2015. Т.~49. №\,3. С.~372--383.
\bibitem{7-gor}
\Au{Горбунов К.\,Ю., Любецкий~В.\,А.} Модифицированный алгоритм преобразования 
хромосомных структур: условия абсолютной точ\-ности~// Современные информационные 
технологии и~ИТ-обра\-зо\-ва\-ние, 2016. Т.~12. №\,1. С.~162--172.
\bibitem{8-gor}
\Au{Compeau P.\,E.\,C.} DCJ-indel sorting revisited~// Algorithm. Mol. Biol., 2013. 
Vol.~8. P.~6.1--6.9.
 \end{thebibliography}

 }
 }

\end{multicols}

\vspace*{-6pt}

\hfill{\small\textit{Поступила в~редакцию 26.04.16}}

\vspace*{6pt}

%\newpage

%\vspace*{-24pt}

\hrule

\vspace*{2pt}

\hrule

\vspace*{-2pt}


\def\tit{ALGORITHM OF~TRANSFORMATION OF~A~GRAPH\\ INTO~ANOTHER ONE WITH~MINIMAL COST}

\def\titkol{Algorithm of transformation of a graph into another one with minimal cost}

\def\aut{K.\,Yu.~Gorbunov$^1$ and~V.\,A.~Lyubetsky$^{1,2}$}

\def\autkol{K.\,Yu.~Gorbunov and~V.\,A.~Lyubetsky}

\titel{\tit}{\aut}{\autkol}{\titkol}

\vspace*{-9pt}


\noindent
$^1$A.\,A.~Kharkevich Institute for Information Transmission Problems of the Russian Academy of 
Sciences,\linebreak
$\hphantom{^1}$19-1 Bolshoy Karetny Per., Moscow 127051, Russian Federation 

\noindent
$^2$Faculty of Mechanics and Mathematics, M.\,V.~Lomonosov Moscow State 
University, Main Building, Leninskiye\linebreak 
$\hphantom{^1}$Gory, GSP-1, Moscow 119991, Russian 
Federation



\def\leftfootline{\small{\textbf{\thepage}
\hfill INFORMATIKA I EE PRIMENENIYA~--- INFORMATICS AND
APPLICATIONS\ \ \ 2017\ \ \ volume~11\ \ \ issue\ 1}
}%
 \def\rightfootline{\small{INFORMATIKA I EE PRIMENENIYA~---
INFORMATICS AND APPLICATIONS\ \ \ 2017\ \ \ volume~11\ \ \ issue\ 1
\hfill \textbf{\thepage}}}

\vspace*{3pt}



\Abste{The authors study orgraphs with any number of chains and cycles. Edges of 
orgraphs have unique names~--- natural numbers. There is a fixed list of operations 
that transform one graph into another. A~cost is assigned to each operation. The task is 
to find the path of transformations with minimal total cost. This problem has a~wide 
range of practical applications. The task is probably NP-hard and, thus, can be solved 
only under constraints imposed on costs or graphs. Such solutions are proposed in the 
study. The corresponding algorithms are linear in time and memory and are proved to 
be exact (nonheuristic), i.\,e., to find the path of transformations with minimal cost. 
Many heuristic algorithms solving this problem are known and tested on various data, 
but the proposed solutions are the first exact solutions.}

\KWE{orgraph with chains and cycles; graph transformation; graph transformation 
with minimal total cost; exact linear algorithm; graph constraint; cost constraint; 
conditional shortest solution}

\DOI{10.14357/19922264170107}  

\vspace*{-24pt}

\Ack
\noindent
The study was supported by the Russian Science Foundation (project  
No.\,14-50-00150).



%\vspace*{3pt}

  \begin{multicols}{2}

\renewcommand{\bibname}{\protect\rmfamily References}
%\renewcommand{\bibname}{\large\protect\rm References}

{\small\frenchspacing
 {%\baselineskip=10.8pt
 \addcontentsline{toc}{section}{References}
 \begin{thebibliography}{9}
\bibitem{1-gor-1}
\Aue{Gorbunov, K.\,Yu., and V.\,A.~Lyubetsky}. 2016. Lineynyy algoritm 
kratchayshey perestroyki grafov pri raznykh tsenakh operatsiy [A linear algorithm of 
the shortest transformation of graphs under different operation costs]. 
\textit{Informatsionnye protsessy} [Information Processes] 16(2):223--236.
\bibitem{2-gor-1}
\Aue{Gorbunov, K.\,Yu., and V.\,A.~Lyubetsky}. 2017 (in press). Lineynyy algoritm 
minimal'noy perestroyki struktur [Linear algorithm of the minimal reconstruction of 
structures under different operation costs]. \textit{Problemy peredachi informatsii} 
[Problems of Information Transmission] 53(1). 
\bibitem{3-gor-1}
\Aue{Lyubetsky, V.\,A., R.\,A.~Gershgorin, A.\,V.~Seliverstov, and 
K.\,Yu.~Gorbunov}. 2016. Algorithms for reconstruction of chromosomal structures. 
\textit{BMC Bioinformatics} 17:40.1--40.23.
\bibitem{4-gor-1}
\Aue{Da Silva, P.\,H., R.~Machado, S.~Dantas, and M.\,D.\,V.~Braga}. 2013.  
DCJ-indel and DCJ-substitution distances with distinct operation costs. 
\textit{Algorithm. Mol. Biol}. 8:21.1--\linebreak 21.15.
\bibitem{5-gor-1}
\Aue{Compeau, P.\,E.\,C.} 2014. A~generalized cost model for DCJ-indel sorting. 
\textit{Algorithms in bioinformatics}. Eds.\ D.\,G.~Brown and B.~Morgenstern.
{Lecture notes in computer science ser.} Springer. 8701:38--51.
\bibitem{6-gor-1}
\Aue{Gorbunov, K.\,Yu., R.\,A.~Gershgorin, and V.\,A.~Lyubetsky}. 2015. 
Rearrangement and 
inference of chromosome structures. \textit{Mol. 
Biol.} 49(3):327--338. 
\bibitem{7-gor-1}
\Aue{Gorbunov, K.\,Yu., and V.\,A.~Lyubetsky}. 2016.  
Modifitsirovannyy algoritm preobrazovaniya khromosomnykh struktur: Usloviya 
absolyutnoy tochnosti [Modified algorithm for transformation of chromosome structures: 
Conditions of absolute accuracy]. \textit{Sovremennye informatsionnye tekhnologii 
i~IT-obrazovanie} [Modern Information Technologies and IT Education] 12(1):162--172.
\bibitem{8-gor-1}
\Aue{Compeau, P.\,E.\,C.} 2013. DCJ-indel sorting revisited. \textit{Algorithm.  
Mol. Biol.} 8:6.1--6.9.
\end{thebibliography}

 }
 }

\end{multicols}

\vspace*{-9pt}

\hfill{\small\textit{Received April 26, 2016}}

\pagebreak

\Contr

\noindent
\textbf{Gorbunov Konstantin Yu.} (b.\ 1965)~--- Candidate of Science (PhD) in 
physics and mathematics, leading scientist, A.\,A.~Kharkevich Institute for Information Transmission 
Problems of the Russian Academy of Sciences, 
\mbox{19-1}~Bolshoy Karetny Per., Moscow 127051, Russian Federation; 
\mbox{gorbunov@iitp.ru}

\vspace*{3pt}

\noindent
\textbf{Lyubetsky Vassily~A.} (b.\ 1945)~--- Doctor of Science in physics and 
mathematics, professor, Head of Laboratory, A.\,A.~Kharkevich Institute for Information 
Transmission Problems of the Russian Academy of Sciences, 
\mbox{19-1}~Bolshoy Karetny Per., Moscow 127051, Russian Federation; professor, 
Faculty of Mechanics and Mathematics, M.\,V.~Lomonosov Moscow State 
University, Main Building, Leninskiye Gory, GSP-1, Moscow 119991, Russian 
Federation; \mbox{lyubetsk@iitp.ru}
\label{end\stat}


\renewcommand{\bibname}{\protect\rm Литература} 