 \def\stat{zatsman}

\def\tit{НАДКОРПУСНАЯ БАЗА ДАННЫХ КОННЕКТОРОВ:\\ ПОСТРОЕНИЕ 
СИСТЕМЫ ТЕРМИНОВ$^*$}

\def\titkol{Надкорпусная база данных коннекторов: построение 
системы терминов}

\def\aut{Анна А.~Зализняк$^1$, И.\,М.~Зацман$^2$, О.\,Ю.~Инькова$^3$}

\def\autkol{Анна А.~Зализняк, И.\,М.~Зацман, О.\,Ю.~Инькова}

\titel{\tit}{\aut}{\autkol}{\titkol}

\index{Зализняк Анна А.}
\index{Зацман И.\,М.}
\index{Инькова О.\,Ю.}
\index{Zaliznyak Anna A.}
\index{Zatsman I.\,M.}
\index{Inkova O.\,Yu.}


{\renewcommand{\thefootnote}{\fnsymbol{footnote}} \footnotetext[1]
{Работа выполнена в~Институте проб\-лем информатики ФИЦ ИУ РАН при поддержке РФФИ 
(проект №\,16-24-41002) и~ШННФ (проект №\,IZLRZ1\_164059).}}


\renewcommand{\thefootnote}{\arabic{footnote}}
\footnotetext[1]{Институт языкознания Российской академии наук; Институт проб\-лем информатики 
Федерального исследовательского центра <<Информатика и~управ\-ле\-ние>> Российской 
академии наук, \mbox{anna.zalizniak@gmail.com}}
\footnotetext[2]{Институт проблем информатики Федерального исследовательского 
центра <<Информатика 
и~управ\-ле\-ние>> Российской академии наук, \mbox{izatsman@yandex.ru}}
\footnotetext[3]{Женевский университет, \mbox{Olga.Inkova@unige.ch}}

\vspace*{-2pt}

  
   \Abst{Рассматривается задача контрастивного исследования
    коннекторов русского языка с~по\-мощью надкорпусной базы данных (НБД), которая представляет собой новую категорию 
информационных лингвистических ресурсов. Надкорпусная база данных содержит параллельные выровненные 
тексты, в~которых одному или нескольким предложениям поставлены в~соответствие одно 
или несколько предложений их перевода. Первая характерная черта НБД заключается 
в~возможности аннотирования исследуемых языковых единиц (ЯЕ), в~данном случае 
коннекторов. Вторая со\-сто\-ит в~том, что решение задачи аннотирования привело к~появлению 
широкого спектра новых сущностей и~понятий как в~информатике, так и~в лингвистике. Для 
их описания предлагается сис\-те\-ма терминов, носящая междисциплинарный характер. 
С~одной стороны, эти термины используются лингвистами для представления новых 
фундаментальных знаний, полученных ими в~процессе контрастивного исследования 
коннекторов русского языка. С~другой стороны, они применяются при разработке 
архитектуры и~функциональных подсистем НБД, а также для разработки информационного, 
лингвистического и~программного видов обеспечения. Кроме того, эта сис\-те\-ма терминов 
необходима для сопоставления полученных результатов с~имеющимися отечественными 
и~зарубежными аналогами.}
   
   \KW{надкорпусная база данных; система терминов; аннотирование коннекторов; 
параллельные тексты; корпусная лингвистика; хронотипическая фасетная классификация}

\DOI{10.14357/19922264170109}  

\vspace*{-5pt}



\vskip 10pt plus 9pt minus 6pt

\thispagestyle{headings}

\begin{multicols}{2}

\label{st\stat}
   
\section{Введение}
    
   Аннотирование коннекторов с~помощью НБД 
является одной из задач совместного рос\-сий\-ско-швей\-цар\-ско\-го проекта 
<<Контрастивное корпусное исследование коннекторов русского языка>>, 
который в~настоящее время выполняется в~Институте проблем информатики 
ФИЦ ИУ РАН и~на филологическом факультете Женевского университета. 
Одной из целей аннотирования является контрастивное исследование 
и~статистический анализ с~помощью НБД универсальных 
и~лингвоспецифичных черт семантики русских коннекторов, т.\,е.\ служебных 
слов различных грамматических клас\-сов (сочинительные и~подчинительные 
союзы, некоторые наречия, предлоги и~частицы, а~также\linebreak единицы комплексной 
грамматической природы, получившие название <<дискурсивные  
слова>>), выполняющих связующую функцию~[1--4].

   
    Основная цель статьи заключается в~описании системы терминов, 
разрабатываемой совместными усилиями лингвистов, переводчиков 
и~специалистов в~области информатики в~рамках этого проекта. Определяемые 
термины, с~одной стороны, необходимы лингвистам для разработки принципов\linebreak 
аннотирования и~представления новых фундаментальных знаний, полученных 
в~процессе конт\-рас\-тив\-но\-го исследования коннекторов русского языка 
и~статистического анализа результатов их аннотирова\-ния. С~другой стороны, 
они необходимы для описания требований к~архитектуре, функциональным 
подсистемам, информационному, программному и~лингвистическому видам 
обеспечения~НБД. 

В~совместной работе лингвистов и~специалистов по 
информатике, не владеющих, как правило, профессиональной лексикой друг 
друга, эта система терминов выполняет функцию \textit{лингва франка}. Она 
создается ими с~целью обеспечить когнитивную интероперабельность 
(взаимное понимание) участ\-ников в~процессе выполнения ими проекта, 
в~котором лингвистическая и~ин\-фор\-ма\-ци\-он\-но-ком\-пью\-тер\-ная  
со\-став\-ля\-ющие вплетены друг в~друга. 

Кроме того, создаваемая система 
терминов необходима для сопоставления получаемых в~проекте результатов 
с~имеющимися отечественными и~зарубежными аналогами.

\section{Надкорпусная база данных}

    Надкорпусная база данных, обеспечивающая возможность 
аннотирования анализируемых ЯЕ, относится к~новой категории 
информационных лингвистических ресурсов~[5, 6]. В~ней одному или 
нескольким предложениям текста на русском языке поставлены в~соответствие 
одно или несколько предложений их перевода на французский 
язык\footnote{Использование в~качестве единицы аннотирования последовательности из двух 
и~более предложений обусловлено спецификой коннекторов как ЯЕ, 
устанавливающих ло\-ги\-ко-се\-ман\-ти\-че\-ские отношения как между частями одного 
предложения, так и~между предложениями.}. Поставленные в~соответствие предложения 
(или их последовательности) называются \textit{текстовыми парами}. Надкорпусная
база данных 
обеспечивает поиск пар по словам, сочетаниям слов и~меткам их 
морфологической разметки (табл.~1).
    

     
    В настоящее время в~НБД загружены тексты параллельного французского 
подкорпуса Национального корпуса русского языка. Общий объем 
этого подкорпуса составляет около~3,5~млн сло\-во\-упо\-треб\-ле\-ний. В~НБД 
предложениям на русском языке поставлены в~соответствие их переводы на 
французский язык, всего~104\,471~пара, а~предложениям на французском 
языке~--- их переводы на русский язык, всего~13\,402~пары. В~обоих случаях 
объектом исследования являются коннекторы русского языка. При этом 
в~первом случае одновременно\linebreak %\vspace*{-12pt}

\columnbreak

\noindent
 анализируются и~их функционально 
эквивалентные фрагменты (ФЭФ~--- термин Д.~Добровольского~[7]), а~во 
втором случае~--- те \textit{стимулы} французского текста, переводами 
которых на русский язык являются исследуемые коннекторы\footnote{Термин 
\textit{стимул перевода} в~том значении, в~котором он применяется в~данных исследованиях 
ЯЕ, в~том числе коннекторов, введен в~работе~\cite{8-zat} (см.\  
также~\cite{2-zat, 9-zat}).}~\cite{2-zat, 8-zat, 9-zat}. Поэтому в~первом случае будем 
говорить о~\textit{прямом переводе}, а~во втором~--- 
о~\textit{реверсивном}\footnote{ Термин \textit{реверсивный перевод} предложен 
Н.\,В.~Бунтман при обсуждении этой статьи.}.
    
    Первое отличие НБД от параллельных корпусов текстов заключается 
в~возможности аннотирования исследуемых ЯЕ в~обоих 
направлениях перевода. В~параллельных корпусах такая функция отсутствует. 
Более того, сам замысел электронных корпусов текстов не предполагает 
реализации  в~них подобных функций. Второе отличие состоит в~том, что 
в~процессе аннотирования конкретных ЯЕ в~НБД программно 
формируются информационные объекты разной степени генерализации. Их 
формирование позволяет проводить многоаспектный статистический анализ 
ис\-сле\-ду\-емых~ЯЕ.
    
    Для описания результатов аннотирования и~порождаемых 
информационных объектов на первом этапе выполнения проекта стала 
создаваться сис\-тема терминов. Отдельные термины, например 
\textit{двуязычный кортеж}, \textit{фасетная классификация
(ФК) ЯЕ}, были определены в~работе~\cite{10-zat}. Фасетная классификация, 
используемая при аннотировании анализируемых ЯЕ
в~НБД, была названа \textit{хронотипической} (термин вводится впервые); это 
обозначение отражает\, тот\, факт,\, что производимые экспертами по\linebreak\vspace*{-36pt}



\end{multicols}



    \begin{table*}[h]\small % tabl1
    \vspace*{-12pt}
    \begin{center}
     \Caption{Первые четыре текстовые пары найденных в~НБД коннекторов по сочетанию 
слов <<не только>> в~русском тексте}
\vspace*{2ex}
    
    \begin{tabular}{|p{78.5mm}|p{78.5mm}|}
    \hline
    \multicolumn{1}{|c|}{Оригинальный текст} & 
\multicolumn{1}{c|}{Перевод}\\
    \hline
     Захар не старался изменить \textbf{не только} данного ему Богом образа, 
     но и~своего 
костюма, в~котором ходил в~деревне&Zakhar n'avait rien fait pour changer 
l'apparence que Dieu 
lui avait donn$\acute{\mbox{e}}$e ni le costume qu'il avait port$\acute{\mbox{e}}$ 
{\ptb{\`{a}}}~la campagne\\
     \hline
     Он его представлял себе чем-то вроде второго отца, который только 
     и~дышит тем, как 
бы за дело и~не за дело, сплошь да рядом, награждать своих подчиненных и~заботиться 
\textbf{не только} о~их нуждах, но и~об удовольствиях&Il se l'imaginait 
comme une sorte de 
second \mbox{p{\!\ptb{\`{e}}}re} qui ne pensait \mbox{qu'{\!\ptb{\`{a}}}} distribuer 
des primes {\ptb{\`{a}}}~ses 
employ$\acute{\mbox{e}}$s, qu'ils le m$\acute{\mbox{e}}$ritent ou \mbox{non,\,{\ptb{\`{a}}}} 
tort \mbox{et\,{\ptb{\`{a}}}~travers}, \mbox{et~qu'{\!\ptb{\`{a}}}} veiller 
non seulement {\ptb{\`{a}}}~leurs 
besoins mais aussi {\ptb{\`{a}}}~leur bien-$\hat{\mbox{e}}$tre\\
     \hline
     Это происходило, как заметил Обломов впоследствии, оттого, что есть такие 
начальники, которые в~испуганном до одурения лице подчиненного, выскочившего к~ним 
навстречу, видят \textbf{не только} почтение к~себе, но даже ревность, а~иногда 
и~способности к~службе&Comme Oblomov le remarqua plus tard, la cause en 
$\acute{\mbox{e}}$tait que certains sup$\acute{\mbox{e}}$rieurs voyaient dans la mine 
effray$\acute{\mbox{e}}$e d'un employ$\acute{\mbox{e}}$ qui 
\mbox{s'empressait~{\ptb{\`{a}}}~leur} 
rencontre, non seulement une preuve de respect pour eux, mais aussi un signe de 
\mbox{z{\!\ptb{\`{e}}}le} et 
m$\hat{\mbox{e}}$me d'aptitude au service\\
     \hline
     Со времени смерти стариков хозяйственные дела в~деревне \textbf{не только} не 
улучшились, но, как видно из письма старосты, становились хуже&Depuis la mort des parents, 
les affaires du domaine non seulement ne s'am$\acute{\mbox{e}}$lioraient pas, mais, 
\mbox{{\!\ptb{\`{a}}}}~en croire la lettre du r$\acute{\mbox{e}}$gisseur, empiraient\\
          \hline
     \end{tabular}
          \end{center}
         \end{table*}
         

         
  \begin{table*}[b]\small %tabl2
   \begin{center}
   \Caption{Пример русско-французской двуязычной аннотации}
   \vspace*{2ex}
   
   \begin{tabular}{|p{41mm}|p{26mm}|p{41mm}|p{28mm}|}
   \hline
Со времени смерти стариков хозяйственные дела в~деревне \textbf{не только} не 
улучшились, \textbf{но}, как видно из письма старосты, становились хуже 
&\textbf{не} \textbf{только}$\|$\textbf{но}\newline
$\langle$~TBD~$\rangle$\newline 
$\langle$~CNT p CNT q~$\rangle$\newline
$\langle$~CNT~$\rangle$\newline 
$\langle$~Дистант~$\rangle$
&Depuis la mort des parents, les affaires du domaine \textbf{non seulement} ne 
s'am$\acute{\mbox{e}}$lioraient pas, \textbf{mais},\,{\ptb{\`{a}}}~en 
croire la lettre du 
r$\acute{\mbox{e}}$gisseur, empiraient&\textbf{non seulement}$\|$\textbf{mais}\newline
$\langle$~TBD~$\rangle$\newline 
$\langle$~CNT p CNT q~$\rangle$\newline 
$\langle$~CNT~$\rangle$\newline
$\langle$~Дистант~$\rangle$\\
\hline
\end{tabular}
\end{center}
\end{table*}     
         

                     \begin{multicols}{2} 
    
     
              
\noindent
ходу работы 
изменения числа и~состава рубрик классификации фиксируются в~форме 
\textit{хронотипов}~--- состояний классификационных систем 
в~последовательные дискретные моменты времени.
    
   Следующий раздел содержит дефиниции основных терминов разработанной 
системы, которые относятся к~моно- и~двуязычным аннотациям, а~так\-же 
иллюстрирующие их примеры. В~разд.~4 рассматриваются 
поливариантные и~полиязычные аннотации.

\vspace*{-9pt}

\section{Моно- и~двуязычные аннотации}
    
   Выше уже были определены термины: \textit{текстовая пара}, 
\textit{прямой} и~\textit{реверсивный перевод}, \textit{хронотипическая 
классификация} (\textit{классификационная система}) и~ее \textit{хронотипы}. 
Определение других терминов будет во многом основано на понятиях 
<<\textit{аннотация>>} и~<<\textit{аннотирование}>> исследуемых 
ЯЕ, их ФЭФ, а также переводных соответствий ЯЕ и~ФЭФ. 
Проиллюстрируем значение дефиниции понятия <<\textit{аннотация}>> 
в~контексте сопоставления ожидаемых результатов проекта с~имеющимися 
аналогами. 
   
   Наиболее близким аналогом НБД совместного проекта является база данных 
Penn Discourse Treebank (PDTB), содержащая тексты на английском языке, 
объем которых равен приблизительно~1~млн 
словоупотреблений~\cite{11-zat}.  База данных PDTB была создана для исследования 
дискурсивных отношений и~коннекторов как средств их выражения как внутри 
предложения, так и~между предложениями. В~PDTB 
сформировано~18\,459~аннотаций коннекторов. В~рас\-смат\-ри\-ва\-емом 
совместном проекте на момент его завершения запланировано 
сформировать~5000~аннотаций коннекторов русского языка и~их французских 
ФЭФ, полученных в~процессе выполнения профессионального перевода, 
который будем называть \textit{референтным переводом} (РП), так как в~НБД 
он является эталоном при исследовании качества машинного перевода (МП) 
коннекторов~\cite{12-zat, 13-zat}. Одновременно будут 
сформированы~5000~аннотаций для МП, т.\,е.\ всего~10\,000, 
при этом для МП берутся те же предложения 
оригинального текста, что и~для РП. Тогда общее число аннотаций в~НБД 
(10\,000) будет почти в~два раза меньше, чем в~PDTB (18\,459). Однако реальное 
соотношение количества аннотаций в~двух сравниваемых базах данных совсем 
иное, поскольку аннотации в~PDTB являются моноязычными (МА), а~в~НБД~--- 
двуязычными (анализируются и~описываются одновременно исследуемые 
коннекторы и~их ФЭФ).
   
   Приведенный пример иллюстрирует необходимость четкого определения 
как минимум трех видов аннотации: моно-, дву- и~полиязычной (определения~1 
и~2 даны без учета \textit{вариантности} переводов; это понятие вводится 
в~следующем разделе).
   
   \vspace*{3pt}
   
   \noindent
   \textbf{Определение~1.}\ \textit{Моноязычной аннотацией}  
исследуемой ЯЕ в~НБД называется совокупность рубрик 
ФК ЯЕ, сформированная 
лингвистами в~процессе семантического анализа ЯЕ, которой 
они приписывают структурированный текстовый комментарий; при этом 
идентификатор автора и~дата создания аннотации формируются программно.
   
   \smallskip
   
   Аналогично определяется МА для ФЭФ исследуемой 
ЯЕ с~использованием рубрик соответству\-ющей ФК
(ФК ФЭФ).

\begin{table*}\small %tabl3
\begin{center}
\Caption{Пример русско-французской поливариантной двуязычной аннотации}
\vspace*{2ex}

\begin{tabular}{|p{36mm}|p{31mm}|p{43mm}|p{31mm}|}
\hline
Со времени смерти стариков хозяйственные дела в~деревне \textbf{не только} не 
улучшились, \textbf{но}, как видно из письма старосты, становились хуже &\textbf{не 
только}$\|$\textbf{но}\newline  
$\langle$~неединственности~$\rangle$\newline 
$\langle$~CNT p CNT q~$\rangle$\newline  
$\langle$~CNT~$\rangle$\newline  
$\langle$~Дистант~$\rangle$
&Depuis la mort des parents, les affaires du domaine \textbf{non seulement} ne 
s'am$\acute{\mbox{e}}$lioraient pas, \textbf{mais},~{\ptb{\`{a}}}~en croire la lettre du 
r$\acute{\mbox{e}}$gisseur, empiraient\newline 
\newline
Avec le temps de la mort des personnes $\hat{\mbox{a}}$g$\acute{\mbox{e}}$es les 
affaires 
$\acute{\mbox{e}}$conomiques dans le village, \textbf{non seulement} ne s'est pas 
am$\acute{\mbox{e}}$lior$\acute{\mbox{e}}$e, \textbf{mais}, comme on peut le voir 
{\ptb{\`{a}}}~partir de la lettre des chefs, de s'aggraver 
&\textbf{non seulement}$\|$\textbf{mais}\newline  
$\langle$~неединственности~$\rangle$\newline 
$\langle$~CNT p CNT q~$\rangle$\newline  
$\langle$~CNT ~$\rangle$\newline 
$\langle$~Дистант~$\rangle$\newline\newline  
\textbf{non 
seulement}$\|$\textbf{mais}\newline  
$\langle$~неединственности~$\rangle$\newline 
$\langle$~CNT p CNT q~$\rangle$\newline
$\langle$~CNT~$\rangle$\newline
$\langle$~AgramTotal~$\rangle$\\
\hline
\end{tabular}
\end{center}
\vspace*{6pt}
\end{table*}
   
      \vspace*{3pt}
   
   \noindent
   \textbf{Определение~2.}\ \textit{Двуязычной аннотацией} исследуемой 
ЯЕ и~ее ФЭФ в~НБД называется кортеж, т.\,е.\ упорядоченная 
пара вида $\langle$МА ЯЕ; МА ФЭФ ЯЕ$\rangle$, которому лингвисты 
приписывают рубрики ФК кортежа и~структурированный 
текстовый комментарий; при этом идентификатор автора и~дата создания 
записи формируются программно для всего кортежа, а не для каждой из 
двух~МА.

   \smallskip
   
   Пример русско-фран\-цуз\-ской двуязычной аннотации, сформированной в~НБД в~результате семантического анализа ЯЕ <<\textit{не только}$\ldots$, 
\textit{но}>> и~ее контекста из четвертой текстовой пары табл.~1, приведен 
в~табл.~2 (без рубрик кортежа, комментария, идентификатора и~даты). Моноязычная аннотация 
ЯЕ 
<<\textit{не только}$\ldots$, \textit{но}>> состоит из первого и~второго 
столбцов этой таблицы, МА ФЭФ ЯЕ~--- из третьего и~четвертого столбцов. 
Третий столбец содержит ФЭФ <<\textit{non seulement}$\ldots$ \textit{mais}>> 
в~контексте перевода всего русского предложения на французский язык. 
Второй и~чет\-вер\-тый столбцы содержат рубрики, которые выбраны лингвистом 
в~процессе аннотирования этой ЯЕ и~ее ФЭФ из ФК ЯЕ и~ФК ФЭФ 
соответственно. В~обеих МА выделяется главная руб\-ри\-ка ФК, которая 
характеризует форму и~смыс\-ло\-вое содержание ЯЕ 
(\textbf{не} \textbf{только}$\|$\textbf{но}) и~ее ФЭФ (\textbf{non 
seulement}$\|$\textbf{mais}). Кроме этих главных рубрик обе МА содержат по 
четыре дополнительные руб\-ри\-ки, которые в~этом примере у них одинаковые,\linebreak 
а~\mbox{именно:}
   \begin{enumerate}[(1)]
\item специальная руб\-ри\-ка-мет\-ка TBD (to be defined), которая говорит 
о~том, что отношение, выраженное коннектором, еще не определено; %\\[-15pt]
\item рубрика $\langle$CNT $p$ CNT $q$$\rangle$ говорит о~том, что 
элементы многокомпонентного коннектора находятся в~каждом из 
соединяемых фрагментов текста~$p$ и~$q$; %\\[-15pt]
\item рубрика $\langle$CNT$\rangle$ говорит о~том, что кортеж построен для всего 
коннектора, а~не для отдельных составляющих его блоков или 
элементов; %\\[-15pt]
\item рубрика $\langle$Дистант$\rangle$ говорит о~том, что части 
коннектора разделены текстом.
\end{enumerate}

   Отметим, что в~общем случае рубрики во втором и~четвертом столбцах 
могут не совпадать.
   

     
  
  В этом примере специальная рубрика-метка TBD говорит о том, что 
  ло\-ги\-ко-се\-ман\-ти\-че\-ское отношение, выражаемое ЯЕ 
  <<\textit{не только}$\ldots$, 
\textit{но}>>, в~момент формирования этой рус\-ско-фран\-цуз\-ской 
двуязычной аннотации не было определено. Одна из возможных причин 
простановки этой пометки заключается в~том, что в~ФК в~момент 
формирования этой аннотации отсутствовала релевантная руб\-ри\-ка. Это служит 
сигналом того, что со\-от\-вет\-ст\-ву\-ющий фасет используемой классификации 
необходимо расширить. После добавления новой рубрики в~ФК эта аннотация 
будет отредактирована следующим образом: руб\-ри\-ку-мет\-ку TBD 
впоследствии заменит рубрика отношения <<неединственности>> (см.\ табл.~3, 
где используется очередной <<хронотип>> ФК, уже включающий данную 
рубрику). Более подробное описание этого примера, а~так\-же фасетов 
используемых классификаций и~их рубрик приведено в~работе~\cite{10-zat}.

\section{Поливариантные и~полиязычные аннотации}
    
  При наличии в~НБД двух и~более вариантов перевода одного и~того же текста 
появляется новый вид аннотации, который определим сле\-ду\-ющим образом.
  
  \smallskip
  
  \noindent
   \textbf{Определение~3.}\ \textit{Поливариантной двуязычной аннотацией} 
исследуемой ЯЕ и~ее ФЭФ в~НБД называется кортеж вида 
$\langle\mathrm{МА\ ЯЕ}; \{ \mathrm{МА\ ФЭФ}_1\ \mathrm{ЯЕ}, \ldots\linebreak
\ldots, 
\mathrm{МА\ ФЭФ}_n\ \mathrm{ЯЕ} \}\rangle$, где $\mathrm{ФЭФ}_1, \ldots, 
\mathrm{ФЭФ}_n$ соответствуют~$n$ разным вариантам перевода на один 
и~тот же язык, при этом лингвисты кортежу вида $\langle\mathrm{МА\ ЯЕ;\ 
МА\ ФЭФ}_i\ \mathrm{ЯЕ}\rangle$, где $i\hm=1, 2, \ldots, n$, присваивают 
рубрики и~комментарий, а идентификатор автора и~дата формирования 
приписываются программно каждому кортежу.
   
   \smallskip
  
  Из этого определения следует, что в~НБД поливариантная двуязычная 
аннотация формируется поэтапно (на каждом этапе создается, как правило, 
один кортеж вида $\langle\mathrm{МА\ ЯЕ;\ МА}_i\ \mathrm{ФЭФ\ 
ЯЕ}\rangle$). При наличии в~НБД переводов одних и~тех же текстов на два 
и~более языков появляется еще один вид аннотации.

\begin{table*} %tabl4
   \begin{center}
   \Caption{Пример французско-русской РМЭ}
\vspace*{2ex}

\begin{tabular}{|p{37mm}|p{37mm}|p{37mm}|p{35mm}|}
\hline
Joli calembour. Il est  
\textbf{non-seulement} \mbox{tr{\!\ptb{\`{e}}}s}-fort, \textbf{mais encore}  
\mbox{tr{\!\ptb{\`{e}}}s}-spirituel &\textbf{non seulement}$\|$\textbf{mais encore}\newline 
$\langle$~неединственности~$\rangle$\newline 
$\langle$~CNT p CNT q~$\rangle$\newline 
$\langle$~CNT~$\rangle$\newline 
$\langle$~Дистант~$\rangle$
&Прекрасный каламбур, \textbf{не только} складный, \textbf{но и}~остроумный &\textbf{не 
только}$\|$\textbf{но и}\newline 
$\langle$~неединственности~$\rangle$\newline 
$\langle$~CNT p CNT q~$\rangle$\newline 
$\langle$~CNT~$\rangle$\newline 
$\langle$~Дистант~$\rangle$\newline\\
\hline
\end{tabular}
\end{center}
\vspace*{9pt}
\end{table*}
  
   \smallskip
   
   \noindent
   \textbf{Определение~4.}\ \textit{Полиязычной аннотацией} исследу\-емой 
ЯЕ и~ее ФЭФ в~НБД называется кортеж вида 
$\langle\mathrm{МА\ ЯЕ};\ \{\mathrm{МА}_1\ \mathrm{ФЭФ\ ЯЕ}, \ldots, 
\mathrm{МА}_n\ \mathrm{ФЭФ\ ЯЕ}\}\rangle$, где $\mathrm{МА}_1, \ldots, 
\mathrm{МА}_n$ соответствуют~$n$~разным языкам перевода; при этом лингвисты 
кортежу вида $\langle\mathrm{МА\ ЯЕ};\ \mathrm{МА}_i\ \mathrm{ФЭФ\ 
ЯЕ}\rangle$, где $i\hm=1, 2, \ldots, n$, присваивают рубрики и~комментарий, 
а~идентификатор и~дата формируются программно для каждого кортежа.
   
   \smallskip
   
   В~совместном проекте по исследованию коннекторов русского языка 
формируются только моно- и~поливариантные двуязычные аннотации согласно 
определениям~2 и~3. Полиязычные аннотации для переводов на два и~более 
языков в~проекте не используются и~далее в~статье не рассматриваются.


   
   Пример русско-французской поливариантной двуязычной аннотации, 
сформированной в~НБД в~результате семантического анализа ЯЕ <<\textit{не 
только}$\ldots$, \textit{но}>>, ее контекста в~четвертой текстовой паре  
в~табл.~1, одного ее РП и~одного МП приведен в~табл.~3 (без рубрик кортежей, 
комментариев, идентификаторов и~дат формирования кортежей).
   
   Отметим, что определения~2 и~3 применимы только к~прямому переводу, 
поэтому два определяемых ими термина будем называть <<прямая двуязычная 
аннотация>> (кратко~--- моноэквиваленция или МЭ) и~<<прямая 
поливариантная двуязычная аннотация>> (кратко~--- полиэквиваленция или 
ПЭ)\footnote{Принцип представления отношения переводной эквивалентности в~форме 
двухместного кортежа как единицы базы данных, а~также термины <<моноэквиваленция>> 
и~<<полиэквиваленция>> предложены Анной А.~Зализняк в~работе~\cite{8-zat}.}.



   
   Для реверсивного перевода определим еще два вида аннотации.
   
   \smallskip
   
   \noindent
   \textbf{Определение~5.}\ \textit{Реверсивной двуязычной аннотацией} 
(реверсивной МЭ, или РМЭ) исследуемой ЯЕ
и~ее ФЭФ в~НБД называется кортеж вида $\langle$МА ФЭФ ЯЕ; МА 
ЯЕ$\rangle$, в~котором лингвист приписывает рубрики ФК
 и~структурированный текстовый комментарий для ФЭФ и~для 
ЯЕ, а~идентификатор и~дата формируются программно; при этом в~качестве 
ФЭФ выступает стимул для появления исследуемой ЯЕ в~переводе.
   
   Пример французско-рус\-ской РМЭ, сформированной в~НБД в~результате 
семантического анализа стимула для появления исследуемой ЯЕ <<\textit{не 
только}$\ldots$, \textit{но и}>> в~переводе, а также его контекста, приведен 
в~табл.~4. Моноязычная аннотация ЯЕ <<\textit{не только}$\ldots$, \textit{но и}>>  
со\-сто\-ит из 
третьего и~четвертого столбцов этой таблицы, МА ФЭФ ЯЕ, т.\,е.\ стимула,~--- 
из первого и~второго столб\-цов. Первый стол\-бец содержит ФЭФ <<\textit{non 
seulement}$\ldots$\ \textit{mais encore}>> в~контексте всего французского 
предложения, а~третий стол\-бец содержит перевод этого предложения на 
русский язык. Второй и~четвертый столб\-цы содержат руб\-ри\-ки, которые 
выбраны лингвистом в~процессе аннотирования стимула для появления 
ис\-сле\-дуе\-мой ЯЕ и~аннотирования самой ЯЕ из ФК ФЭФ и~ФК ЯЕ 
соответственно.
   
   

   
   \smallskip
   
   \noindent
   \textbf{Определение~6.}\ \textit{Реверсивной поливариантной двуязычной 
аннотацией} (реверсивной ПЭ, или РПЭ) ЯЕ, 
исследуемой в~двух и~более переводах одного и~того же текста на русский язык, 
и~ее ФЭФ в~НБД называется кортеж вида $\langle\mathrm{МА\ ФЭФ\ ЯЕ};\ 
\{\mathrm{МА}_1\ \mathrm{ЯЕ}, \ldots, \mathrm{МА}_n\ \mathrm{ЯЕ}\}\rangle$, 
где $\mathrm{МА}_1,\ldots, \mathrm{МА}_n$ соответствуют~$n$~разным 
вариантам перевода на русский язык; при этом ФЭФ ЯЕ является стимулом для 
появления в~переводах одной и~той же исследуемой ЯЕ, но в~$n$~разных ее 
контекстах.
   
   \smallskip
   
   Для каждой из прямых и~реверсивных МЭ и~ПЭ в~работе~\cite{10-zat} были 
определены три частных случая аннотации коннекторов, обозначаемых в~НБД 
следующим образом:
   \begin{description}
\item[\,] I тип~--- аннотация для всего коннектора в~целом (см.\ МЭ 
I~типа в~табл.~2, ПЭ I~типа в~табл.~3 и~РМЭ I~типа в~табл.~4),
\item[\,] II~тип~--- аннотация для компонента (блока) коннектора, 
состоящего из двух и~более неделимых элементов (табл.~5),
\item[\,] III тип~--- аннотация для неделимого элемента коннектора 
(табл.~6).
\end{description}

   Таким образом, с~учетом деления на три типа всего в~статье 
определено~12~видов прямых и~реверсивных МЭ и~ПЭ (см.\ нижний ряд на 
рисунке), которые являются основой создаваемой системы терминов, развитие 
и~уточнение которой планируется продолжить в~процессе выполнения 
совместного проекта и~наполнения НБД.

 

 

\pagebreak

\end{multicols}
   
\begin{table*}\small %tabl5
\begin{center}
\Caption{Пример русско-французской МЭ II типа}
\vspace*{2ex}

\begin{tabular}{|p{44mm}|p{25mm}|p{40mm}|p{25mm}|}
\hline
Со времени смерти стариков хозяйственные дела в~деревне \textbf{не только} не 
улучшились, но, как видно из письма старосты, становились хуже &\textbf{не 
только}\newline 
$\langle$~с предикацией~$\rangle$\newline 
$\langle$~неначальная~$\rangle$\newline 
$\langle$~Part~$\rangle$\newline 
$\langle$~Контакт~$\rangle$
&Depuis la mort des parents, les affaires du domaine \textbf{non seulement} ne 
s'am$\acute{\mbox{e}}$lioraient pas, \mbox{mais,~{\ptb{\`{a}}}~en} croire la lettre du 
r$\acute{\mbox{e}}$gisseur, empiraient &\textbf{non seulement}\newline 
$\langle$~неначальная~$\rangle$\newline 
$\langle$~с предикацией~$\rangle$\newline 
$\langle$~Part~$\rangle$\newline 
$\langle$~Контакт~$\rangle$\\
\hline
\multicolumn{4}{p{148mm}}{\footnotesize \textbf{Примечания.}
   \begin{enumerate}[1.]
   \item Рубрика <<Part>> говорит о~том, что аннотация сформирована для части (здесь~--- 
блока) <<не только>> коннектора <<не только$\ldots$, но~и>>.
   \item Рубрика <<с~предикацией>> говорит о~том, что аннотируемая часть коннектора 
маркирует фрагмент текста, представляющий собой предикативную единицу.
   \item Рубрика <<неначальная>> говорит о~том, что аннотируемая часть коннектора 
занимает в~маркируемом им фрагменте текста неначальную позицию.
   \item Рубрика <<контакт>> говорит о~том, что элементы, составляющие аннотируемую 
часть коннектора, находятся в~контактном расположении.
   \end{enumerate}}
\end{tabular}
\end{center}
%\end{table*} 
%\begin{table*}\small %tabl6
\vspace*{-18pt}
\begin{center}
\Caption{Пример русско-французской МЭ III типа}
\vspace*{2ex}

\begin{tabular}{|p{58mm}|p{10mm}|p{58mm}|p{7mm}|}
\hline
Петрович, $[\ldots]$ был совсем заспавшись; \textbf{но} при всем том, как только узнал, 
в~чем дело, 
точно как будто его черт толкнул &\textbf{но}\newline 
$\langle$~Part~$\rangle$ 
&P$\acute{\mbox{e}}$trovitch $[\ldots]$ semblait tout endormi. 
Malgr$\acute{\mbox{e}}$ cela, 
\mbox{\mbox{d{\!\ptb{\`{e}}}s}} qu'il eut compris de quoi il s'agissait,
 ce fut comme si quelque diable 
l'e$\hat{\mbox{u}}$t pouss$\acute{\mbox{e}}$ &\textbf{zero}\\ 
   \hline
   \multicolumn{4}{p{146mm}}{\footnotesize 
   \textbf{Примечания.}\begin{enumerate}[1.]
   \item Рубрика <<Part>> говорит о~том, что аннотация сформирована для части 
   (здесь~--- 
неделимого элемента) <<но>> коннектора <<но при всем том>>.
   \item Специальная рубрика <<\textbf{zero}>> говорит о~том, что неделимый элемент 
коннектора здесь не переведен.
   \end{enumerate}}
   \end{tabular}
   \end{center}
   %\vspace*{2pt}
   %\end{table*}
%\begin{figure*}
\vspace*{-3pt}
\begin{center}
\mbox{%
\epsfxsize=136.171mm
\epsfbox{zac-2.eps}
}

\vspace*{6pt}

\noindent
{\small Виды аннотаций, определенные в~статье, с~учетом деления их на типы}
\end{center}
\end{table*}

\begin{multicols}{2}

В заключение этого раздела определим тот процесс, который выполняется 
лингвистами при наполнении ими НБД.
   
  
   
  \smallskip
   
   \noindent
   \textbf{Определение~7.}\ Формирование аннотации любого вида: 
моноязычной, полиязычной, прямых и~реверсивных МЭ и~ПЭ всех трех  
типов~--- будем называть \textit{аннотированием} ЯЕ, при необходимости 
соответственно эксплицируя частные его случаи, например: \textit{реверсивное 
поливариантное двуязычное аннотирование I~типа}.



\section{Заключение}

\vspace*{-24pt}
    
   Решение задачи двуязычного аннотирования в~НБД коннекторов, других 
исследуемых ЯЕ и~их  переводных соответствий привело к~появлению
широкого 
спектра новых сущностей и~понятий как  в~информатике, так и~в лингвистике. 
Их появление свидетельствует о том, что НБД стали новой категорией 
информационных лингвистических ресурсов, обеспечивающих 
целенаправленную генерацию новых фундаментальных знаний~[14--18].
   
   Определение и~типологизация перечисленных видов аннотаций, а~также 
формализация самого процесса аннотирования служат одновременно 
нескольким целям. Рассмотренные дефиниции обеспечивают единое понимание 
содержания целей, задач и~ожидаемых результатов совместного проекта как 
лингвистами, так и~специалистами в~области\linebreak информатики, а~так\-же 
когнитивную интероперабельность участников проекта. Создаваемая сис\-тема 
терминов необходима для согласованного понима\-ния методики аннотирования 
и~описания результатов, получаемых в~процессе выполнения проекта, а~также 
для их сопоставления с~отечественными и~зарубежными аналогами.
   
   Для исполнителей совместного проекта, отве\-чающих за его 
   ин\-фор\-ма\-ци\-он\-но-ком\-пью\-тер\-ное обеспечение, рассмотренные дефиниции 
   служат основой 
создания и~развития архитектуры НБД в~соответствии с~целями и~задачами 
проекта. Задача представления в~НБД прямых и~реверсивных МЭ и~ПЭ всех 
трех типов, которые состоят из двух и~более МА
исследуемых ЯЕ и~их ФЭФ, во многом определяет основные принципы 
проектирования НБД. В~частности, количество оснований ФК
и~состав рубрик по каждому основанию может изменяться 
в~процессе наполнения НБД, что требует соответствующих архитектурных 
решений.
   
   Описание взаимосвязей основных принципов проектирования НБД 
с~разрабатываемой системой терминов выходит за рамки этой статьи и~должно 
стать предметом отдельного исследования.

%\vspace*{-9pt}
   
{\small\frenchspacing
 {%\baselineskip=10.8pt
 \addcontentsline{toc}{section}{References}
 \begin{thebibliography}{99}
 
 \bibitem{3-zat} %1
\Au{Баранов А.\,Н., Плунгян~В.\,А., Рахилина~Е.\,В.} Путеводитель по 
дискурсивным словам русского языка.~--- М.:
Помовсий и~Партнеры, 1993. 207~с.
\bibitem{4-zat} %2
Дискурсивные слова русского языка. Опыт  
кон\-текст\-но-се\-ман\-ти\-че\-ско\-го описания~/ Под ред.\ К.~Киселевой, 
Д.~Пайллард.~--- М.: Метатекст, 1998. 446~с.
\bibitem{1-zat} %3
\Au{Инькова-Манзотти О.\,Ю.} Коннекторы противопоставления во 
французском и~русском языках. Сопоставительное исследование.~--- М.: 
Информэлектро, 2001. 434~с.
\bibitem{2-zat} %4
\Au{Зализняк Анна А.} База данных межъязыковых эквиваленций как 
инструмент лингвистического анализа~// Компьютерная лингвистика 
и~интеллектуальные технологии: По мат-лам ежегодной Междунар. конф. 
<<Диалог>> (2016).~--- М.: РГГУ, 2016. С.~763--775.



\bibitem{5-zat}
\Au{Кружков М.\,Г.} Информационные ресурсы контрастивных 
лингвистических исследований: электронные корпуса текстов~// Системы 
и~средства информатики, 2015. Т.~25. №\,2. С.~140--159.

\columnbreak

\bibitem{6-zat}
\Au{Зализняк Анна~А., Зацман~И.\,М., Инькова~О.\,Ю., Кружков~М.\,Г.} 
Надкорпусные базы данных как лингвистический ресурс~// Корпусная 
лингвистика-2015: Труды 7-й Междунар. конф.~--- СПб.: СПбГУ, 2015. 
С.~211--218.
\bibitem{7-zat}
\Au{Добровольский Д.\,О., Кретов~А.\,А., Шаров~С.\,А.} Корпус параллельных 
текстов: архитектура и~возможности использования~// Национальный корпус 
русского языка: 2003--2005.~--- М.: Индрик, 2005. С.~263--296.
\bibitem{8-zat}
\Au{Loiseau S., Sitchinava~D.\,V., Zalizniak~Anna~A., Zatsman~I.\,M.} Information 
technologies for creating the database of equivalent verbal forms in the  
Russian-French multivariant parallel corpus~// Информатика и~её применения, 
2013. Т.~7. Вып.~2. С.~100--109.
\bibitem{9-zat}
\Au{Сичинава Д.\,В.} Использование параллельного корпуса для 
количественного изучения лингвоспецифичной лексики~// Язык, литература, 
культура: актуальные проблемы изучения и~преподавания.~--- М.: МАКС 
ПРЕСС, 2014. Вып.~10. С.~37--44.
\bibitem{10-zat}
\Au{Зацман И.\,М., Инькова~О.\,Ю., Кружков~М.\,Г., Попкова~Н.\,А.} 
Представление кроссязыковых знаний о~коннекторах в~надкорпусных базах 
данных~// Информатика и~её применения, 2016. Т.~10. Вып.~1. С.~106--118.
\bibitem{11-zat}
\Au{Prasad R., Dinesh~N., Lee~A., Miltsakaki~E., Robaldo~L., Joshi~A., 
Webber~B.} The Penn Discourse TreeBank 2.0~// 6th Conference (International) on 
Language Resources and Evaluation (LREC) Proceedings.~--- Paris: European 
Language Resources Association (ELRA), 2008. P.~2961--2968.
\bibitem{12-zat}
Learning machine translation~/ Eds. C.~Goutte, N.~Cancedda, M.~Dymetan, 
G.~Foster.~---  London: MIT Press, 2009. 316~p.
\bibitem{13-zat}
\Au{Lo C., Wu~D.} MEANT: An inexpensive, high-accuracy, semi-automatic metric 
for evaluating translation utility via semantic frames~// Human Language 
Technologies: 49th Annual Meeting of the Association for Computational Linguistics 
Proceedings.~---  Stroudsburg: Association for Computational Linguistics, 2011. 
Vol.~1. P.~220--229.
\bibitem{14-zat}
\Au{Zatsman~I.} Tracing emerging meanings by computer: Semiotic 
framework~//  13th European Conference on Knowledge Management 
Proceedings.~---  Reading, U.K.: Academic Publishing International Ltd., 2012. 
Vol.~2. P.~1298--1307.
\bibitem{15-zat}
\Au{Zatsman I., Buntman~N., Kruzhkov~M., Nuriev~V., Zalizniak~Anna~A.} 
Conceptual framework for development of computer technology supporting  
cross-linguistic knowledge discovery~// 15th European Conference on Knowledge 
Management Proceedings.~---  Reading, U.K.: Academic Publishing International Ltd., 
2014. Vol.~3. P.~1063--1071.
\bibitem{16-zat}
\Au{Zatsman I., Buntman~N.} Outlining goals for discovering new knowledge and 
computerised tracing of emerging meanings discovery~//  16th European Conference 
on Knowledge Management Proceedings.~--- Reading, U.K.: Academic Publishing 
International Ltd., 2015. P.~851--860.

\pagebreak

\bibitem{17-zat}
\Au{Зацман И.\,М.} Процессы целенаправленной генерации и~развития 
кроссязыковых экспертных знаний: семиотические основания 
моделирования~// Информатика и~её применения, 2015. Т.~9. Вып.~3.\linebreak  
С.~106--123.
\bibitem{18-zat}
\Au{Zatsman I., Buntman~N., Coldefy-Faucard~ A., Nuriev~V.} WEB knowledge 
base for asynchronous brainstorming~// 17th European Conference on Knowledge 
Management Proceedings.~--- Reading, U.K.: Academic Publishing International Ltd., 
2016. P.~976--983.
 \end{thebibliography}

 }
 }

\end{multicols}

\vspace*{-3pt}

\hfill{\small\textit{Поступила в~редакцию 17.01.17}}

\vspace*{8pt}

%\newpage

%\vspace*{-24pt}

\hrule

\vspace*{2pt}

\hrule

%\vspace*{8pt}



\def\tit{SUPRACORPORA DATABASE ON~CONNECTIVES: TERM~SYSTEM~DEVELOPMENT}

\def\titkol{Supracorpora database on~connectives: Term system 
development}

\def\aut{Anna A.~Zaliznyak$^{1,2}$, I.\,M.~Zatsman$^2$, and~O.\,Yu.~Inkova$^3$}

\def\autkol{Anna A.~Zaliznyak, I.\,M.~Zatsman, and~O.\,Yu.~Inkova}

\titel{\tit}{\aut}{\autkol}{\titkol}

\vspace*{-9pt}


\noindent
   $^1$Institute of Linguistics, Russian Academy of Sciences, 1-1~Bolshoy Kislovskiy Per., 
Moscow 125009, Russian\linebreak
$\hphantom{^1}$Federation
   
   \noindent
   $^2$Institute of Informatics Problems, Federal Research Center ``Computer Science and 
Control'' of the Russian\linebreak
$\hphantom{^1}$Academy of Sciences, 44-2~Vavilov Str., Moscow 119333, Russian 
Federation

   \noindent
   $^3$University of Geneva, 22~Bd des Philosophes, CH-1205 Geneva~4, 
Switzerland



\def\leftfootline{\small{\textbf{\thepage}
\hfill INFORMATIKA I EE PRIMENENIYA~--- INFORMATICS AND
APPLICATIONS\ \ \ 2017\ \ \ volume~11\ \ \ issue\ 1}
}%
 \def\rightfootline{\small{INFORMATIKA I EE PRIMENENIYA~---
INFORMATICS AND APPLICATIONS\ \ \ 2017\ \ \ volume~11\ \ \ issue\ 1
\hfill \textbf{\thepage}}}

\vspace*{3pt}

   
   
   \Abste{The article considers a supracorpora database (SCDB)~--- a~new type of 
   linguistic information resource. The SCDB contains aligned parallel texts wherein 
   source language sentences are aligned with target language sentences. 
   One distinctive feature of the SCDB is that it supports annotating the 
   examined linguistic items (in this case, connectives). Another important 
   feature is that cross-linguistic annotating makes it possible to reveal 
   a~wide spectrum of new entities and concepts, both in informatics and linguistics. 
   For description of these entities and concepts, a~new multidisciplinary term 
   system is proposed. On the one hand, the proposed terms are used by linguists 
   for description of new basic knowledge generated as a~result of contrastive 
   analysis of Russian connectives. On the other hand, the design of architecture 
   and functional subsystems of the SCDB is based on these terms, and they are 
   used for the development of respective information, linguistic and software tools. 
   Finally, the term system is required for comparison of the presented 
   outcomes of the project with similar results of other projects.}
   
   \KWE{supracorpora database; term system; connectives; linguistic annotation; 
   parallel texts; 
   corpus linguistics; chronotypical faceted classification}
   
  
   
  \DOI{10.14357/19922264170109}  

%\vspace*{-9pt}

 \Ack
   \noindent
   This research was performed in the Institute of Informatics 
   Problems, Federal Research Center 
``Computer Science and Control'' of the Russian Academy of Sciences, 
with financial support of the 
Russian Foundation for Basic Research (project No.\,16-24-41002) 
and Swiss National Science Foundation (project No.\,IZLRZ1\_164059).



%\vspace*{3pt}

  \begin{multicols}{2}

\renewcommand{\bibname}{\protect\rmfamily References}
%\renewcommand{\bibname}{\large\protect\rm References}

{\small\frenchspacing
 {%\baselineskip=10.8pt
 \addcontentsline{toc}{section}{References}
 \begin{thebibliography}{99}
 \bibitem{3-zat-1} %1
\Aue{Baranov, A.\,N., V.\,A.~Plungyan, and E.\,V.~Rakhilina.} 1993. 
\textit{Putevoditel' po diskursivnym slovam russkogo yazyka} [Guide to the 
Russian discourse words]. Мoscow: Pomovskiy i~Partnery. 207~p.
\bibitem{4-zat-1} %2
Kiseleva, K., and D.~Paillard, eds. 1998. \textit{Diskursivnye slo\-va russkogo 
yazyka. Opyt kontekstno-semanticheskogo opisaniya} [Russian discourse 
words: A~contextual-semantic description]. Мoscow: Metatext. 446~p.
\bibitem{1-zat-1} %3
\Aue{Inkova-Manzotti, O.\,Yu.} 2001. \textit{Konnektory protivopostavleniya 
vo frantsuzskom i~russkom yazykakh: Sopostavitel'noe issledovanie} 
[Connectives of opposition in French and Russian: A~comparative study]. 
Moscow: Informelektro. 434~p.
\bibitem{2-zat-1} %4
\Aue{Zaliznyak, Anna~A.} 2016. Baza dannykh mezh\-yazy\-ko\-vykh 
ekvivalentsiy kak instrument lingvisticheskogo ana\-li\-za [Database of  
cross-linguistic equivalences as a tool for linguistic analysis]. 
\textit{Computer Linguistics and 
Intellectual Technologies: Conference (International) ``Dialog'' Proceedings}. 
Moscow: RGGU. 763--775.

\bibitem{5-zat-1}
\Aue{Kruzhkov, M.\,G.} 2015. Informatsionnye resursy kontrastivnykh 
lingvisticheskikh issledovaniy: Elektronnye korpusa tekstov [Information 
resources for contrastive studies: Digital text corpora]. \textit{Sistemy 
i~Sredstva Informatiki~--- Systems and Means of Informatics}  
25(2):140--159.
\bibitem{6-zat-1}
\Aue{Zaliznyak, Anna~A., I.\,M.~Zatsman, O.\,Yu.~Inkova, and 
M.\,G.~Kruzhkov.} 2015. Nadkorpusnye bazy dannykh kak lingvisticheskiy 
resurs [Supracorpora databases as linguistic resource]. \textit{7th Conference 
(International) on Corpus Linguistics Proceedings}. St.\ Petersburg: SPbGU.\linebreak 
211--218.
\bibitem{7-zat-1}
\Aue{Dobrovol'skiy, D.\,O., A.\,A.~Kretov, and S.\,A.~Sharov.} 2005. Korpus 
parallel'nykh tekstov: Arkhitektura i~voz\-mozh\-nosti ispol'zovaniya [Corpus of 
parallel texts: Architecture and applications]. \textit{Natsional'nyy korpus 
russkogo yazyka: 2003--2005} [Russian National Corpus: 2003--2005]. 
Moscow: Indrik. 263--296.
\bibitem{8-zat-1}
\Aue{Loiseau, S., D.\,V. Sitchinava, Anna~A.~Zalizniak, and 
I.\,M.~Zatsman.} 2013. Information technologies for creating the database of 
equivalent verbal forms in the Russian-French multivariant parallel corpus. 
\textit{Informatika i ee Primeneniya~--- Inform. Appl.} 7(2):100--109.
\bibitem{9-zat-1}
\Aue{Sitchinava, D.\,V.} 2014. Ispol'zovanie parallel'nogo korpusa dlya 
kolichestvennogo izucheniya lingvospetsifichnoy leksiki [Using a parallel 
corpus for the quantitative study of language-specific units]. \textit{Yazyk, 
literatura, kul'tura: Aktual'nye problemy izucheniya i~prepodavaniya} 
[Language, literature, culture: Actual problems of research and teaching]. 
Moscow: MAKS PRESS. 10:37--44.
\bibitem{10-zat-1}
\Aue{Zatsman, I.\,M., O.\,Yu.~Inkova, M.\,G.~Kruzhkov, and 
N.\,A.~Popkova.} 2016. Predstavlenie krossyazykovykh znaniy 
o~konnektorakh v~nadkorpusnykh bazakh dannykh [Representation of  
cross-lingual knowledge about connectives in supracorpora databases]. 
\textit{Informatika i~ee Primeneniya~--- Inform. Appl.} 10(1):106--118.
\bibitem{11-zat-1}
\Aue{Prasad, R., N.~Dinesh, A.~Lee, E.~Miltsakaki, L.~Robaldo, A.~Joshi, 
and B.~Webber.} 2008. The Penn Discourse TreeBank~2.0. \textit{6th 
Conference (International) on Language Resources and Evaluation (LREC) 
Proceedings}. Paris: European Language Resources Association (ELRA). 
2961--2968.

\columnbreak
\bibitem{12-zat-1}
Goutte, C., N. Cancedda, M.~Dymetan, and G.~Foster, eds. 2009. 
\textit{Learning machine translation}. London: MIT Press. 316~p.
\bibitem{13-zat-1}
\Aue{Lo, C., and D.~Wu.} 2011. MEANT: An inexpensive, high-accuracy, 
semi-automatic metric for evaluating translation utility via semantic frames. 
\textit{Human Language Technologies: 49th Annual Meeting of the 
Association for Computational Linguistics Proceedings}.  Stroudsburg: 
Association for Computational Linguistics. 1:220--229.
\bibitem{14-zat-1}
\Aue{Zatsman, I.} 2012. Tracing emerging meanings by computer: Semiotic 
framework. \textit{13th European Conference on Knowledge Management 
Proceedings}.  Reading, U.K.: Academic Publishing International Ltd.  
2:1298--1307.
\bibitem{15-zat-1}
\Aue{Zatsman, I., N.~Buntman, M.~Kruzhkov, V.~Nuriev, and 
Anna~A.~Zalizniak.} 2014. Conceptual framework for development of 
computer technology supporting cross-linguistic knowledge discovery. 
\textit{15th European Conference on Knowledge Management Proceedings}.  
Reading, U.K.: Academic Publishing International Ltd. 3:1063--1071.
\bibitem{16-zat-1}
\Aue{Zatsman, I., and N.~Buntman.} 2015. Outlining goals for discovering 
new knowledge and computerised tracing of emerging meanings discovery. 
\textit{16th European Conference on Knowledge Management Proceedings}. 
Reading, U.K.: Academic Publishing International Ltd.  
851--860.
\bibitem{17-zat-1}
\Aue{Zatsman, I.} 2015. Protsessy tselenapravlennoy ge\-ne\-ra\-tsii %\linebreak
 i~razvitiya 
krossyazykovykh ekspertnykh znaniy: Se\-mi\-o\-ti\-che\-skie osnovaniya 
modelirovaniya [Goal-oriented\linebreak processes of cross-lingual expert knowledge 
creation: Semiotic foundations for modeling]. \textit{Informatika i~ee 
Primeneniya~--- Inform. Appl.} 9(3):106--123.
\bibitem{18-zat-1}
\Aue{Zatsman, I., N.~Buntman, A.~Coldefy-Faucard, and V.~Nuriev.} 2016. 
WEB knowledge base for asynchronous brainstorming. \textit{17th European 
Conference on Knowledge Management Proceedings}. Reading, U.K.: Academic 
Publishing International Ltd. 976--983.
\end{thebibliography}

 }
 }

\end{multicols}

\vspace*{-3pt}

\hfill{\small\textit{Received January 17, 2017}}


\Contr


\noindent
\textbf{Zaliznyak Anna A.} (b.\ 1959)~--- Doctor of Science in philology, leading 
scientist, Institute of Linguistics, Russian Academy of Sciences, 1-1~Bolshoy Kislovskiy 
Per., Moscow 125009, Russian Federation; leading scientist, Institute of Informatics 
Problems, Federal Research Center ``Computer Science and Control'' of the Russian 
Academy of Sciences, 44-2~Vavilov Str., Moscow 119333, Russian Federation; 
\mbox{anna.zalizniak@gmail.com}

\vspace*{3pt}

\noindent
\textbf{Zatsman Igor M.} (b.\ 1952)~--- Doctor of Science in technology, Head of 
Department, Institute of Informatics Problems, Federal Research Center ``Computer 
Science and Control'' of the Russian Academy of Sciences, 44-2~Vavilov Str., Moscow 
119333, Russian Federation; \mbox{izatsman@yandex.ru}

\vspace*{3pt}

\noindent
\textbf{Inkova Olga Yu.} (b.\ 1965)~--- Doctor of Science in philology, Faculty member, University 
of Geneva, 22~Bd des Philosophes, CH-1205 Geneva 4, Switzerland; 
%senior scientist, Institute of Informatics Problems, Federal Research Center ``Computer 
%Science and Control'' of the Russian Academy of Sciences, 44-2~Vavilov Str., Moscow 119333, 
%Russian Federation;  
\mbox{Olga.Inkova@unige.ch}
   
\label{end\stat}


\renewcommand{\bibname}{\protect\rm Литература} 
    
    