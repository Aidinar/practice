 \def\stat{gorshenin}

\def\tit{О НЕКОТОРЫХ МАТЕМАТИЧЕСКИХ И~ПРОГРАММНЫХ МЕТОДАХ ПОСТРОЕНИЯ СТРУКТУРНЫХ 
МОДЕЛЕЙ ИНФОРМАЦИОННЫХ ПОТОКОВ$^*$}

\def\titkol{О некоторых математических и~программных методах построения структурных 
моделей информационных потоков}

\def\aut{А.\,К.~Горшенин$^1$}

\def\autkol{А.\,К.~Горшенин}

\titel{\tit}{\aut}{\autkol}{\titkol}

\index{Горшенин А.\,К.}
\index{Gorshenin A.\,K.}


{\renewcommand{\thefootnote}{\fnsymbol{footnote}} \footnotetext[1]
{Работа выполнена при частичной поддержке
РФФИ (проект 15-37-20851 мол\_а\_вед) и~Программы Президиума РАН №\,I.33П (проект 063-2016-0015).}}


\renewcommand{\thefootnote}{\arabic{footnote}}
\footnotetext[1]{Институт проблем информатики Федерального исследовательского центра 
<<Информатика и~управление>> Российской академии наук, \mbox{agorshenin@frccsc.ru}}


\Abst{Потоки событий в~современных информационных сис\-те\-мах не являются регулярными, 
поэтому методы анализа, основанные на классических теоремах, справедливых при 
определенных условиях регулярности, могут приводить к~некорректным выводам, 
включая недооценку рисков экстремальных событий. При практическом 
моделировании и~анализе нестационарных информационных потоков ключевой 
задачей становится выбор статистических методов оценивания неизвестных 
параметров моделей. В~исследованиях, проводимых в~рамках научной школы профессора 
В.\,Ю.~Королева, традиционно для этих целей принято использовать так называемый метод 
скользящего разделения смесей (СРС-ме\-тод), 
основанный на специальном разбиении исходной выборки 
на подвыборки (окна) и~дальнейшем анализе поведения данных на каждом окне, в~рамках 
смешанных вероятностных моделей. Описанные методы анализа стохастических 
данных на основе смешанных вероятностных моделей позволяют повысить эффективность 
исследования сложных информационных сис\-тем. Развитие и~использование предложенных 
методов может оказаться весьма полезным в~соответствующих областях прикладной 
математики и~компьютерных наук.}

\KW{информационные системы; смешанные вероятностные модели; метод 
скользящего разделения смесей; статистический анализ данных; 
экстремальные наблюдения; зашумленные данные; пороговое значение; метод 
Peak Over Threshold; теорема Пи\-канд\-са\,--\,Бал\-ке\-мы\,--\,Де Ха\-ана; 
теорема Реньи; он\-лайн-комп\-лекс; матричные вычисления}

\DOI{10.14357/19922264170105}  


\vskip 10pt plus 9pt minus 6pt

\thispagestyle{headings}

\begin{multicols}{2}

\label{st\stat}

\section{Введение}

Потоки событий в~современных информационных, телекоммуникационных и~вычислительных\linebreak 
сис\-те\-мах не являются регулярными (стационарными, однородными), поэтому методы 
анализа, основанные на классических теоремах, спра\-вед\-ливых при определенных 
условиях регулярности, могут приводить к~некорректным выводам, включая 
недо\-оцен\-ку рисков экстремальных событий. При практическом моделировании и~анализе 
нестационарных информационных потоков ключевой задачей становится выбор 
статистических методов оценивания неизвестных параметров моделей. 

В~исследованиях, 
проводимых в~рамках научной школы профессора, доктора фи\-зи\-ко-ма\-те\-ма\-ти\-че\-ских
наук  В.\,Ю.~Королева 
(ВМК МГУ им.\ М.\,В. Ломоносова, ИПИ ФИЦ ИУ РАН), традиционно для этих 
целей принято использовать так называемый СРС-ме\-тод~\cite{Korolev2011}, основанный на специальном разбиении 
исходной выборки на подвыборки (окна) и~дальнейшем анализе поведения данных на 
каж\-дом окне, в~рамках смешанных вероятностных моделей. Поэтому особое внимание в~данной 
работе уделяется подходам, способствующим повышению эффективности СРС-ме\-тода.

Статья посвящена обзору следующих основных результатов исследований, 
соответствующих тематике проекта РФФИ 15-37-20851 мол\_а\_вед 
<<Математические и~программные методы построения структурных моделей 
информационных потоков>> (руководитель~--- А.\,К.~Горшенин):
\begin{itemize}
\item развитие и~апробация методологии определения экстремальных наблюдений в~данных 
(как на основе СРС-ме\-то\-да~\cite{Gorshenin2015Ocean}, так и~с~использованием теорем 
Пи\-канд\-са\,--\,Бал\-ке\-мы\,--\,Де Ха\-ана и~Реньи 
о~редеющих потоках~\cite{Gorshenin2016CCIS1});
\item изучение методологии искусственного зашумления исходных данных 
для повышения корректности результатов анализа СРС-ме\-то\-да~\cite{Gorshenin2016CCIS2};
\item разработка программной реализации сеточных методов разделения конечных 
смесей на базе архитектуры \verb"NVIDIA CUDA"~\cite{Gorshenin2016CUDA};
\item реализация анализа моментных характеристик конечных нормальных 
смесей~\cite{Gorshenin2016Concept};
\item анализ стохастического подхода к~верификации времени выполнения 
программного кода~\cite{Gorshenin2016ICNAAM2};
\item разработка архитектуры~\cite{Gorshenin2015ICUMT2} специализированного 
он\-лайн-комп\-лек\-са стохастического моделирования реальных процессов и~создание 
бе\-та-вер\-сии вычислительного портала {\sf http://msm-analysis.com} с~сис\-те\-мой 
пользовательской регистрации, тестовым набором методов обработки данных, 
развиваемых в~проекте, и~выводом графических результатов~\cite{Gorshenin2016ICNAAM1};
\item использование полученных инструментов для анализа реальных 
данных (турбулентные потоки тепла между океаном и~атмосферой~\cite{Gorshenin2015CCIS}, 
отчеты профилировщика программного кода~\cite{Gorshenin2016ICNAAM2}, информация 
об осадках~\cite{Gorshenin2016Soft3,Gorshenin2016Soft4}).
\end{itemize}

\section{Методология определения экстремальных наблюдений в~данных}

В статье~\cite{Gorshenin2015Ocean} продемонстрирован подход на основании 
упорядочения весов и~дисперсий, получаемых в~рамках аппроксимации исходных 
данных моделью типа конечной смеси нормальных законов и~применения СРС-ме\-то\-да. 
Для этого полученные компоненты смеси упорядочиваются по убыванию значения дисперсий, 
а~наблюдения, соответствующие компоненте с~наибольшим значением, могут 
интерпретироваться как экстремальные относительно других. Данный метод 
решения задачи определения доли таких наблюдений в~исходной выборке 
не требует привлечения теории экстремальных значений (так как соответствует 
па\-ра\-мет\-ру веса соответствующей компоненты), однако для исходных данных 
(в~указанной работе рассмотрен пример турбулентных потоков тепла между 
океаном и~атмосферой) необходимо заранее определить модель, пусть и~достаточно 
общего вида.

В работе~\cite{Gorshenin2016CCIS1} предложена альтернативная описанной выше 
методология определения экстремальных пороговых значений в~неотрицательных данных, 
которые характерны для многих информационных сис\-тем. Стоит отметить, что данный 
подход развивает популярную технику Peak Over Threshold~\cite{Leadbetter1991}.

Согласно предельной теореме Реньи~\cite{Gnedenko1996} для редеющих потоков, 
стационарный точечный процесс при специальном прореживании (каждая точка 
удаляется с~вероятностью $1\hm-p$, где $p \hm\to 0$) сходится к~пуассоновскому 
процессу. При этом интервалы между скачками имеют экспоненциальное распределение. 
Таким образом, проверяя гипотезу экспоненциальности, можно определить пороговое 
значение. Согласно тео\-ре\-ме 
Пи\-канд\-са\,--\,Бал\-ке\-мы\,--\,Де Ха\-ана~\cite{Balkema1974,Pickands1975} распределение 
превышений данного порога должно стремиться к~обобщенному распределению Парето 
с~функцией распределения $F_{\xi, \sigma, \mu}(x)$, а~именно: 
для достаточно широкого класса распределений случайной величины~$X$
\begin{equation*}
F(X-u\leqslant y \mid X>u)\to F_{\xi, \sigma, \mu}(y) \ \mbox{при } 
u\to\infty\,, 
\end{equation*}
где $F_{\xi, \sigma, \mu}(x)$ имеет следующий вид:
\begin{multline}
F_{\xi, \sigma, \mu}(y)={}\\
{}=
\begin{cases}
1-\left(1+\fr{\xi(y-\mu)}{\sigma}\right)^{-1/\xi},& \mbox{если }\xi\neq 0\,;\\
1-e^{-({y-\mu})/{\sigma}}& \mbox{иначе\,.}
\end{cases}
\label{GPD}
\end{multline}

Успешность подгонки реальных данных проверяется по $\chi^2$-кри\-те\-рию. 
Подобная процедура со\-став\-ля\-ет суть восходящего метода определения порогового значения, 
так как <<уровень>>поднимается от нулевого значения данных с~некоторым шагом, 
пока проверяемая гипотеза об экспоненциаль\-ности отвергается (поясняющие графики 
можно найти в~работе~\cite{Gorshenin2016CCIS1}). Допустимо движение и~в~обратном 
направлении~--- так реализуется нисходящий метод определения порогового значения.
В~статье~\cite{Gorshenin2016CCIS1} так\-же приведены наглядные блок-схе\-мы 
для каж\-до\-го из методов, поз\-во\-ля\-ющие реализовать процедуру на любом удобном 
языке программирования. Стоит отметить, что в~качестве данных могут выступать 
наблюдения любой природы, удовле\-тво\-ря\-ющие усло\-вию неотрицательности.

Предположим, что наблюдения могут быть разделены на группы строго положительных 
и~нулевых наблюдений, следующих друг за другом, т.\,е.\ 
исходные данные можно записать в~виде~$X_{n,j}$, где~$n$~--- 
номер соответствующей группы с~ненулевыми данными, а~$j$~соответствует 
позиции элемента внут\-ри данной группы. Пусть~$t$~--- время начала одной из групп. 
Тогда можно рассмотреть данные, пред\-став\-ля\-ющие собой кумулятивные суммы:
\begin{itemize}
\item $X_{n,1}$ в~момент времени~$t$;
\item  $X_{n,1}+X_{n,2}$ в~момент времени~$t\hm+1$;\\
$\ldots\ldots\ldots\ldots\ldots\ldots\ldots\ldots\ldots\ldots\ldots\ldots\ldots\ldots\ldots$
\item $X_{n,1}+\cdots+X_{n,k_n}$ в~момент времени~$t\hm+k_n$,
па\-ра\-метр~$k_n$ соответствует размеру группы.
\end{itemize}
 Превышение порога 
происходит в~момент времени~$t\hm+j$, который определяется выполнением сле\-ду\-ющих условий:
\begin{align*}
X_{n,1}+\cdots+X_{n,j-1}&\leqslant u\,;\\
X_{n,1}+\cdots+X_{n,j}&>u\,.
\end{align*}

В случае если $X_{n,j}$ имеют гамма- или Па\-ре\-то-рас\-пре\-де\-ле\-ние, 
представление кумулятивных данных в~виде случайной суммы со случайным чис\-лом 
слагаемых $\sum\nolimits_{j=1}^{N_n} X_{n,j}$, где~$N_n$ есть 
случайная величина с~отрицательным биномиальным распределением, приводит к~новым 
смешанным вероятностным моделям (например, пуассоновским)~\cite{Korolev2011}, 
что позволяет более тонко исследовать структуру изучаемых процессов.

Описанная методология опирается на две фундаментальные теоремы теорий 
случайных про\-цессов и~экстремальных значений без априорных предположе\-ний о~данных, 
что позволяет, с~одной стороны, адаптивно учитывать их из\-ме\-ня\-ющу\-юся структуру, 
а~с~другой~--- ослабить эмпирические\linebreak предположения при построении 
ве\-ро\-ят\-ност\-но-стати\-сти\-че\-ских моделей. Результаты могут быть 
использованы при построении вероятностных прогно\-зов экстремальных событий, 
связанных с~нестационарными потоками событий в~различных информационных сис\-те\-мах, 
в~том числе и~для анализа климатических изменений в~регионах с~потенциальными 
угрозами наводнений и~затопления жилых, промышленных и~иных стратегически важных 
объектов.

\section{Методология искусственного зашумления исходных данных 
для~повышения корректности результатов анализа метода скользящего
разделения смесей}

Одной из наиболее важных характеристик процессов, происходящих в~информационных 
сис\-те\-мах, является интенсивность (трафика, событий и~т.\,д.)~--- 
величина, которая является неотрицательной. В~этой ситуации использование СРС-ме\-то\-да 
для конечных сдвиг-мас\-штаб\-ных смесей пред\-став\-ля\-ет\-ся не вполне корректным, так 
как данный класс распределений сосредоточен на всей действительной оси. 
Для решения данной проблемы в~работе~\cite{Gorshenin2016CCIS2} предложено 
использование искусственного зашумления выборки с~по\-мощью добавления к~данным 
случайной величины с~известным распределением, а~именно: нулевым 
математическим ожиданием и~некоторой дисперсией. При этом известно, 
что конечные нормальные смеси являются достаточно адекватной аппроксимацией 
для нормальных дис\-пер\-си\-он\-но-сдви\-го\-вых, выступающих в~качестве 
предельных распределений для сумм случайных величин со случайным числом слагаемых. 
Таким образом, можно получить разумные асимптотические аппроксимации статистических 
закономерностей в~рассматриваемых данных.

Отметим, что методология искусственного зашумления успешно
применяется, напри\-мер, в~области статистической 
обработки сигналов (см., 
например,~\cite{Brey1996,Kosko2003}). Кроме того, как продемонстрировано 
в~работе~\cite{Osoba2013}, такая техника может способствовать некоторому 
повышению вычислительной эффективности широко известного метода полу\-чения 
оценок максимального правдоподобия~---\linebreak EM-ал\-го\-ритма.

В рамках работы~\cite{Gorshenin2016CCIS2} было проведено за\-шум\-ление 
строго положительных данных, новая выбор\-ка, в~которой появились отрицательные 
на\-блю\-дения, исследовалась с~помощью классической \mbox{модели}\linebreak конечных сдвиг-мас\-штаб\-ных 
смесей нормальных распределений в~рамках СРС-ме\-то\-да, а~затем из полученных 
результатов была удалена известная шумовая компонента (при этом были корректно 
пересчитаны веса оставшихся, чтобы в~сумме по-преж\-не\-му получалась единица),
\begin{figure*}[b] %fig1
 \vspace*{12pt}
\begin{center}
\mbox{%
\epsfxsize=160mm
\epsfbox{gor-1.eps}
}
\end{center}
\vspace*{-9pt}
\Caption{Изменение значений логарифма функции правдоподобия при изменении 
параметрической сетки  (график создан средствами он\-лайн-пор\-та\-ла 
{\sf http://msm-analysis.com})}
\label{FigLikelihood}
\end{figure*}
а именно: для каж\-дого из наблюдений $X_j$ зашумление предполагает проведение 
замены вида
\begin{equation*}
X_j \to X_j +\varepsilon_j
\end{equation*}
для всех $j=1,\ldots,N$, где~$N$~--- чис\-ло исходных наблюдений. 
В~качестве зашумляющих использованы случайные величины 
$\varepsilon_j\hm\sim\mathcal{N}(0,\sigma^2)$ с~нормальным распределением 
с~параметрами~$0$ и~$\sigma$, при этом выбор параметра~$\sigma$ 
позволяет варьировать изменение начальных данных. Неудачный выбор~$\sigma$ 
(например, в~случае превышения значения выборочного среднеквадратического отклонения) 
может при\-вес\-ти к~существенным изменениям в~данных и~потере структурных компонент 
в~рамках анализа СРС-ме\-то\-дом. Случайные величины~$X_j$ и~$\varepsilon_j$ 
предполагаются независимыми. Очевидно, что данная операция зашумления 
не изменяет математические ожидания как компонент, так и~самой смеси, 
а~для дисперсии справедливо представление 
(см.\ подробнее~\cite{Korolev2011,Gorshenin2016CCIS2}):
\begin{multline*}
\mathbb{D}_{\mathrm{mixt}}=
\sum\limits_{i=1}^{k}p_i(t)\left[a_i(t)-\sum\limits_{i=1}^{k}p_i(t)a_i(t)\right]^2+{}\\[2pt]
{}+
\sum\limits_{i=1}^{k}p_i(t)(\sigma_i^2(t)+\sigma^2)={}\\[2pt]
{}=\sum\limits_{i=1}^{k}p_i(t)\left[
a_i(t)-\sum\limits_{i=1}^{k}p_i(t)a_i(t)\right]^2+{}\\[2pt]
{}+\sum\limits_{i=1}^{k}p_i(t)
\sigma_i^2(t) +\sigma^2.
\end{multline*}

Полученные в~ходе такого анализа результаты продемонстрировали значительно 
лучшее качество выделения структурных компонент по сравнению с~ситуацией, 
когда в~качестве входа СРС-ме\-то\-да использовалась исходная неотрицательная вы\-борка. 
{\looseness=1

}

Дальнейшие исследования в~данном вопросе могут быть направлены на установление 
критериев, которые могли бы использоваться для автоматизации процесса зашумления 
исходных данных, т.\,е.\ выбора значения параметра~$\sigma$. В~част\-ности, 
может быть использован подход на основе статьи~\cite{Ushakov2015} в~рамках 
изучения связи между ошибкой измерения, погрешностью округления и~точностью 
восстановления измеряемой величины.

\vspace*{-9pt}

\section{Реализация сеточных методов разделения конечных смесей 
на~базе архитектуры NVIDIA CUDA}

\vspace*{-2pt}


Стремительный рост популярности решений на базе платформ, ориентированных 
на задействование в~процессе проведения вычислений специализированных графических 
видеокарт, наблюдается в~широком спектре актуальных исследовательских областей: 
в~медицине для реконструкции изображений, в~задачах геостатистического имитационного 
моделирования для изучения запасов нефти и~газа, в~фармакокинетическом анализе данных 
(более подробный обзор с~указанием ссылок приведен в~статье~\cite{Gorshenin2016CUDA}). 

\begin{figure*}[b] %fig2
\vspace*{-8pt}
\begin{center}
\mbox{%
\epsfxsize=109.871mm
\epsfbox{gor-2.eps}
}
\end{center}
\vspace*{-13pt}
\Caption{Прогнозирование {\sf MAD} с~помощью модели {\sf ARIMAX}:
(\textit{а})~500~шагов (\textit{1}~--- наблюдения; \textit{2}~--- прогноз; 
\textit{3}~---
измерения); (\textit{б})~модуль отклонения прогноза от истинного значения}
\label{FigARIMAX} 
\end{figure*}


В~работе~\cite{Gorshenin2015ICNAAM1} были предложены концептуальные решения 
на базе архитектуры \verb"NVIDIA CUDA", направленные на повышение эффективности 
реализации сеточных модификаций методов поиска оценок неизвестных параметров 
в~смешанных вероятностных моделях. Была разработана про\-грам\-мная реализация 
сеточного метода разделения смесей нормальных распределений с~фиксированными 
компонентами на базе архитектуры \verb"NVIDIA CUDA", описанного 
в~статье~\cite{22-a}. Данное решение подходит для обработки 
данных в~режиме реального времени (или в~близком к~нему). Аналогичные реализации 
в~рамках библиографического поиска в~базе Web of Science обнаружены не были.

Пример графического представления результатов работы данного метода с~по\-мощью 
средств портала {\sf http://msm-analysis.com} продемонстрирован на 
рис.~\ref{FigLikelihood}. В~работе~\cite{Gorshenin2016CUDA} можно найти 
аналогичный график, созданный в~среде \verb"MATLAB".

Отметим, что теоретическое обоснование корректности использования 
сеточных методов раз-\linebreak деления конечных сдвиг-мас\-штаб\-ных смесей нор\-маль\-ных законов 
для ряда достаточно общих\linebreak моде\-лей можно найти 
в~статьях~\cite{Gorshenin2012aStab,Gorshenin2012bStab}, некоторые результаты 
для дис\-пер\-си\-он\-но-сдви\-го\-вых смесей нормальных законов приведены 
в~статье~\cite{Korolev2015}.

\section{Реализация анализа моментных характеристик конечных нормальных смесей}

В работе~\cite{Gorshenin2016Concept} значительное внимание уделяется 
получению явных формул для ряда моментных характеристик, отражающих 
динамику изменения\linebreak смеси в~процессе использования СРС-ме\-то\-да, преж\-де всего 
в~матричном представлении,~--- математического ожидания, дисперсии, 
коэффициентов асимметрии и~эксцесса.

Для отыскания подобных характеристик требуется вычисление значений 
центральных моментов порядка $\mu\hm\geqslant 1$, $\mu\hm\in \mathbb{N}$,
$\mathbb{E} \left(Z_t-\mathbb{E} Z_t\right)^\mu$,
где случайная величина~$Z_t$ имеет распределение типа конечной сдвиг-мас\-штаб\-ной 
смеси нормальных законов, а~индекс~$t$ указывает на то обстоятельство, что 
параметры распределения могут изменяться в~процессе работы СРС-ме\-то\-да. 
Очевидно, что для получения явного представления требуется отыскание 
соответствующих начальных моментов, матричные формулы для первых четырех 
из которых приведены ниже:
\begin{multline*}
\mathbb{E} Z_t^m={}\\
{}=\begin{cases}
P_t\cdot A_t^{\mathrm{T}}, & m=1;\\
P_t\left(D_{A,\,t}\cdot A_t^{\mathrm{T}} + D_{\Sigma,\,t} \cdot \Sigma_t^{\mathrm{T}}\right), & m=2;\\
P_t\cdot D_{A,\,t}\left(D_{A,\,t}\cdot A_t^{\mathrm{T}}+3 D_{\Sigma,\,t}
\cdot\Sigma_t^{\mathrm{T}}\right), & m=3;\\
P_t\left( D_{A,\,t}^3\cdot A_t^{\mathrm{T}} +6 D_{\Sigma,\,t}^2\cdot D_{A,\,t}
\cdot A_t^{\mathrm{T}}+{}\right.&\\
\left.\hspace*{30mm}{}+3 D_{\Sigma,\,t}^3\cdot\Sigma_t^{\mathrm{T}}\right), & m=4;
\end{cases}\hspace*{-2.42pt}
\label{NonCentralZMatrix}
\end{multline*}
матричная запись для компонент смеси пред\-став\-ле\-на в~виде соответствующих векторов
\begin{align*}
P_t&=\left(p_1,\ldots,p_{k(t)}\right)\,;\\
A_t&=\left(a_1,\ldots,a_{k(t)}\right)\,;\\
\Sigma_t&=\left(\sigma_1,\ldots,\sigma_{k(t)}\right)\,; \\
D_{A,\,t}&=\mathrm{diag}\left\{a_1,\ldots,a_{k(t)}\right\}\,;\\
D_{\Sigma,\,t}&=\mathrm{diag}\left\{\sigma_1,\ldots,\sigma_{k(t)}\right\},
\end{align*}
а обозначение $\mathrm{diag}\{\cdots\}$ использовано для записи диагональных 
матриц с~указанными элементами. Использование таких выражений позволяет 
значительно повысить эффективность вычислений для ряда современных архитектур 
(см., например,~\cite{Gorshenin2015ICNAAM3}).

Необходимо отметить, что работа в~данном направлении может быть развита с~точки 
зрения прогнозирования данных характеристик. На рис.~\ref{FigARIMAX},\,\textit{а}
 приведен 
пример прогнозирования величины среднего абсолютного отклонения (\verb"MAD", 
median absolute deviation) для математического ожидания смеси по 
известным~8650~наблюдениям (кривая~\textit{1}) на~500~шагов (кривая~\textit{2}) 
с~по\-мощью интегрированной модели авторегрессии~--- 
скользящего среднего \verb"ARIMAX". Рисунок~\ref{FigARIMAX},\,\textit{б} 
демонстрирует величину абсолютного отклонения от истинных наблюдений 
(кривая~\textit{3} на рис.~2,\,\textit{а}) для этих~500~шагов. Видно, что в~данном случае можно 
успешно прогнозировать порядка~150~наблюдений, что важно в~случае необходимости 
расчетов в~реальном времени (получается больший временной слот на обновление 
оценок па\-ра\-мет\-ров смешанной вероятностной модели).


\section{Cтохастический подход к~верификации времени выполнения программного кода}

Важной задачей при разработке программного\linebreak обеспечения, в~том числе 
критического по быст\-ро\-дей\-ст\-вию, является контроль накладных расходов, 
возникающих из-за различных причин (в~част\-ности, из-за изменения среды окружения, 
\mbox{в~которой} запускается программа). Для стохастической верификации времени 
выполнения программ в~работе~\cite{Gorshenin2016ICNAAM2} было предложено 
использование техники на основе применения СРС-ме\-тода.

Анализ функционирования программы в~среде выполнения со случайными факторами 
может выполняться по следующему алгоритму. Один и~тот же программный код многократно 
запускается для формирования тестовой выборки необходимого размера, которая 
содержит сведения о~времени его работы. В~качестве тестового примера были 
использованы~10\,000~запусков одной из встроенных функций (\verb"surf") 
среды \verb"MATLAB", предназначенной для рисования трехмерных поверхностей. 
С~по\-мощью СРС-ме\-то\-да для конечных сдвиг-мас\-штаб\-ных 
смесей нормальных законов были выявлены структурные компоненты неизвестного 
формирующего процесса, которые затем сопоставлялись с~результатами профилирования, 
в~част\-ности для поиска соответствия с~конкретными элементами программного кода, 
на которые в~процентном отношении тратилось время, эквивалентное весам 
соответствующих компонент в~аппроксимирующей смеси. Был выявлен участок в~программном 
коде, временн$\acute{\mbox{ы}}$е затраты на работу которого составляют около~80\% общего времени, 
еще один блок с~8\% затрат и~три участка с~долей около~3\% каждый. 
Эти данные хорошо согласуются с~результатами СРС-ме\-то\-да, 
с~по\-мощью которого была определена главная компонента с~большим весом, 
одна компонента с~умеренным весом, а~также~2--3~<<шумовые>> компоненты.

Подобный подход к~анализу производитель\-ности программного обеспечения 
был впервые предложен в~работе~\cite{Gorshenin2016ICNAAM2}. При этом возможен 
анализ и~исходных данных профилировщика, однако здесь следует учитывать, что 
они представляют собой строго положительные значения, поэтому может оказаться 
весьма полезной описанная выше методология искусственного зашумления наблюдений. 
Разработанная методология может быть использована для оценивания производительности 
программного кода и~соответствующей структуры временн$\acute{\mbox{ы}}$х затрат в~рамках 
специализированной он\-лайн-сис\-те\-мы {\sf http://msm-analysis.com}, например 
для определения эффективности отдельных ее модулей.

\section{Анализ реальных данных с~помощью разработанных подходов}

В статье~\cite{Gorshenin2015Ocean} СРС-ме\-тод
использован в~применении к~задаче статистического моделирования закономерностей в~явных 
и~скрытых турбулентных теп\-ло\-вых потоках между океаном и~атмосферой. \mbox{Также} для 
повышения эффективности и~качества анализа предложена специальная версия EM-ал\-го\-рит\-ма, 
основанная на максимизации функции правдоподобия в~классе конечных смесей 
нормальных законов. В~качестве исходных данных используются результаты 
шестичасовых наблюдений в~Атлантике за период с~1948 по~2008~гг. 
Показано, что в~эволюции теп\-ло\-вых потоков основная компонента с~небольшой дисперсией 
может сопровождаться стохастически развивающимися и~исчезающими компонентами с~большой 
дис\-пер\-си\-ей, при этом вклад в~общую дисперсию ее чис\-то стохастической диффузионной 
компоненты превосходит динамическую со\-став\-ля\-ющую. Математическое ожидание заметно 
ко\-леб\-лет\-ся во времени с~изменяющейся амплитудой, при этом для каждого периода 
амплитуда меньше в~периоды сезонного увеличения общего среднего значения. Вклад в~общую 
дисперсию чис\-то стохастической диффузионной компоненты дис\-пер\-сии больше, чем 
динамической со\-став\-ля\-ющей. При этом эксцесс распределения максимален во время 
периода <<спокойствия>>.

В большинстве работ, посвященных изучению метеорологических данных, используемые 
аналитические модели для наблюдаемых статистиче-\linebreak ских
 закономерностей в~осадках 
далеко не всегда\linebreak являются корректными. Например, традиционно счи\-та\-ет\-ся, что 
продолжительность <<дождливого>>\linebreak перио\-да (чис\-ло подряд идущих дней, во время 
которых наблюдались осадки) имеет геометриче\-ское распределение (см., например, 
статью~\cite{Zolina2013}). Возмож\-но, это предположение основано на\linebreak традиционной 
интерпретации геометрического распределения в~терминах испытаний Бернулли: 
распределения числа подряд идущих дней с~осадками (<<успехи>>) до первого дня 
без осадков (<<неудача>>). Однако необходимые для такой интерпретации условия 
(например, независимость испытаний) на практике нарушаются.

В рамках исследований был проведен статистический анализ для определения 
следующих закономерностей для процесса выпадения осадков:
\begin{itemize}
\item распределение длительностей временн$\acute{\mbox{ы}}$х интервалов, в~которые наблюдались осадки;
\item распределение длительностей временн$\acute{\mbox{ы}}$х интервалов без осадков;
\item распределение объема выпавших осадков;
\item распределение интенсивностей осадков, выпавших за дождливый период.
\end{itemize}

Данный анализ был автоматизирован, в~част\-ности для 
среды \verb"MATLAB"~\cite{Gorshenin2016Soft3}, с~по\-мощью специально 
разработанного для этих целей интерактивного скрипта. Для каждой величины 
используется аппроксимация различными семействами вероятностных распределений 
(отрицательное биномиальное, экспоненциальное, гамма-распределение, обобщенное 
распределение Парето~\eqref{GPD}), при этом качество аппроксимации оценивается 
по критерию хи-квад\-рат. Стоит отметить, что адекватные модели закономерностей 
процессов для осадков крайне\linebreak\vspace*{-12pt}
{ \begin{center}  %fig3
 \vspace*{1pt}
 \mbox{%
\epsfxsize=77.981mm
\epsfbox{gor-3.eps}
}
\end{center}

%\vspace*{-3pt}


\noindent
{{\figurename~3}\ \ \small{Пример для распределения длительностей дождливых интервалов:
\textit{1}~--- исходные данные; \textit{2}~--- отрицательное биноминальное 
распределение с параметрами $r\hm=0{,}876$ и~$p\hm= 0{,}489$}}
}


\vspace*{12pt}

\noindent
 важны для успешного прогнозирования стихийных 
бедствий (например, наводнений или длительных засушливых периодов).

На рис.~3 приведен пример подгонки распределения длительностей 
дождливых интервалов с~помощью отрицательного биномиального распределения. 
Стоит отметить как высокое визуальное\linebreak соответствие, так и~результаты сравнения 
по $\chi^2$-кри\-те\-рию ($p$-зна\-че\-ние оказалось равно~0,1238).



Исследуемая случайная величина, соответствующая длительности 
дождливого периода, могла принимать целые значения не меньше единицы, в~то время 
как классическое определение отрицательного биномиального 
распределения с~па\-ра\-мет\-ра\-ми $r\hm>0$ и~$p\hm\in(0,1)$ подразумевает для 
случайной величины~$Y\hm\sim NB(r,p)$ выполнение следующего равенства:
\begin{equation*}
\mathbb{P}(Y=k)=\fr{\Gamma(r+k)}{\Gamma(r)\Gamma(k+1)}\, p^r (1-p)^{k},
\enskip k=0,1,2,\ldots
\end{equation*}
Однако легко заметить, что для случайной величины $X\hm=Y\hm+1$ справедливы соотношения 
(с~учетом условия $X\geqslant 1$):
\begin{multline*}
\mathbb{P}(X=k)=\mathbb{P}(X-1=k-1)={}\\
{}=\fr{\Gamma(r+k-1)}{\Gamma(r)\Gamma(k-1+1)} p^r
(1-p)^{k-1}={}\\
{}=\fr{\Gamma(r+k-1)}{\Gamma(r)\Gamma(k)} p^r
(1-p)^{k-1}\,, \enskip k=1,2,3,\ldots
\end{multline*}
Поэтому для проверки соответствующей гипотезы достаточно вычесть 
из исходных данных единицу (см.\ гистограммы 
на рис.~3).


Кроме того, была разработана и~реализована методология предсказания осадков 
на основе исторических паттернов~\cite{Gorshenin2016Soft4}: часовые исторические 
наблюдения процесса выпадения осадков представляются в~виде последовательности 
нулей (осадков не было) и~единиц (наблюдались осадки, независимо от объема). 
На основе этих данных формируются паттерны, определяющие картину на\-ли\-чия/от\-сут\-ст\-вия 
осадков за период времени заданной длины (по дням). Рассчитывается частота 
(вероятность) появления каждого паттерна по историческим данным, при этом 
максимальный размер для паттернов задается пользователем в~качестве входного 
параметра модуля. Эти частоты используются для вероятностного прогноза выпадения 
осадков в~рамках заданного периода времени в~будущем, включая оценку вероятности 
выпадения осадков через фиксированное пользователем число дней.

\section{Заключение}

Многие сложные методы вероятностного анализа реализуются в~виде отдельных 
биб\-ли\-о\-тек или программ для специализированных пакетов или предполагают 
соответствие специальным требованиям характеристик аппаратного обеспечения. 
В~качестве альтернативы~\cite{Gorshenin2016Concept} такому подходу предполагается 
их интеграция в~специализированную он\-лайн-сис\-те\-му {\sf http://msm-analysis.com} 
для распределенных вычислений. Это может оказаться весьма полезным, например, 
для исследователей, не имеющих возможности использовать оборудование от \verb"NVIDIA". 
Подобный сервис, предоставляющий широчайшие возможности для моделирования процессов 
в~информационных сис\-те\-мах, снабженный веб-ин\-тер\-фей\-сом и~инструментарием 
для распределенной пользовательской обработки, является новым, не имеет прямых 
аналогов и~способен заинтересовать исследователей различных предметных областей.

В качестве специализированной опции для он\-лайн-комп\-лек\-са с~точки зрения 
обеспечения дополнительного уровня безопасности (помимо сис\-те\-мы пользовательской 
авторизации) был разработан тестовый модуль графической идентификации 
пользователей~\cite{Gorshenin2016Soft5}. Программа получает набор фотографических 
изображений с~камеры и~сравнивает с~заранее заданным для данного пользователя 
образцом. Для этого используется гибридный алгоритм на основе методов сравнения 
контуров и~хэш-таб\-лиц. В~модуле для обработки изображений использованы функции 
библиотеки компьютерного зрения с~открытым исходным кодом \verb"OpenCV". 
Данная разработка является экспериментальной и~на текущем этапе не включена 
в~доступный для пользователей функционал портала.

Разработанные и~реализованные программно в~рамках проекта методы анализа 
стохастических данных на основе смешанных вероятностных моделей позволяют 
повысить эффективность исследования сложных информационных сис\-тем, а~так\-же
развить ранее полученные результаты в~ряде важных прикладных областей~\cite{29-a}. 
Двухлетний план работ по проекту 
РФФИ 15-37-20851 мол\_а\_вед <<Математические и~программные методы построения 
структурных моделей информационных потоков>> выполнен в~полном объеме, 
запланированные цели достигнуты. Развитие и~использование предложенных 
методов может оказаться весьма полезным при проведении исследований в~соответствующих 
областях прикладной математики и~компьютерных наук.

\bigskip

Автор выражает признательность доктору фи\-зи\-ко-ма\-те\-ма\-ти\-че\-ских наук, профессору Виктору Юрьевичу
 Королеву за полезные обсуждения в~рамках совместных исследований проблематики 
 смешанных вероятностных моделей, а~также Виктору Кузьмину за плодотворное участие 
 в~разработке ар\-хи\-тек\-тур\-но-про\-грам\-мных решений для вычислительного портала 
 {\sf http://msm-analysis.com}.

{\small\frenchspacing
 {%\baselineskip=10.8pt
 \addcontentsline{toc}{section}{References}
 \begin{thebibliography}{99}
\bibitem{Korolev2011} 
\Au{Королев~В.\,Ю.} Ве\-ро\-ят\-ност\-но-ста\-ти\-сти\-че\-ские 
методы декомпозиции волатильности хаотических процессов.~--- М.: Изд-во Моск. ун-та, 
2011. 512~с.

\bibitem{Gorshenin2015Ocean} 
\Au{Королев~В.\,Ю., Горшенин~А.\,К., Гулев~С.\,К., Беляев~К.\,П.} Статистическое 
моделирование турбулентных потоков тепла между океаном и~атмосферой 
с~по\-мощью метода скользящего разделения конечных нормальных смесей~// Информатика 
и~её применения, 2015. Т.~9. Вып.~4. C.~3--13.

\bibitem{Gorshenin2016CCIS1} 
\Au{Gorshenin~A.\,K., Korolev~V.\,Yu.} A~methodology for the identification 
of extremal loading in data flows in information systems~// 
Comm. Com. Inf. Sc., 2016. Vol.~638. P.~94--103.

\bibitem{Gorshenin2016CCIS2} 
\Au{Gorshenin~A.\,K., Korolev~V.\,Yu.} A~noising method for the identification 
of the stochastic structure of information flows~// Comm. Com. Inf. Sc., 
2016. Vol.~678. P.~1--11.

\bibitem{Gorshenin2016CUDA} 
\Au{Горшенин~А.\,К., Кузьмин~В.\,Ю.} Применение архитектуры CUDA 
при реализации сеточных алгоритмов для метода скользящего разделения смесей~// 
Системы и~средства информатики,
2016. Т.~26. Вып.~4. С.~60--73.

\bibitem{Gorshenin2016Concept} 
\Au{Горшенин~А.\,К.} Концепция он\-лайн-комп\-лек\-са 
для стохастического моделирования реальных процессов~// Информатика и~её 
применения, 2016. Т.~10. Вып.~1. C.~72--81.

\bibitem{Gorshenin2016ICNAAM2} 
\Au{Gorshenin~A., Frenkel~S., Korolev~V.} On a~stochastic approach to 
a~code performance estimation~// AIP Conference Proceedings, 2016. Vol.~1738. 
220010. 4~p.

\bibitem{Gorshenin2015ICUMT2} 
\Au{Gorshenin~A., Kuzmin~V.} Online system for the construction of structural
 models of information flows~// 7th  Congress (International) on 
 Ultra Modern Telecommunications and Control Systems and Workshops (ICUMT) Proceedings.~-- 
 Piscataway, NJ, USA: IEEE, 2015. P.~216--219.

\bibitem{Gorshenin2016ICNAAM1} 
\Au{Gorshenin~A., Kuzmin~V.} On an interface of the online system for 
a~stochastic analysis of the varied information flows~// AIP Conference Proceedings, 
2016. Vol.~1738. 220009. 4~p.

\bibitem{Gorshenin2015CCIS} 
\Au{Korolev~V.\,Yu., Gorshenin~A.\,K., Gulev~S.\,K., Belyaev~K.\,P.} 
Statistical modeling of air--sea turbulent heat fluxes by finite mixtures of 
Gaussian distributions~// Comm. Com. Inf. Sc., 2015. Vol.~564. P.~152--162.

\bibitem{Gorshenin2016Soft3} 
\Au{Горшенин~А.\,К.} Программный модуль анализа статистических характеристик осадков. 
Свидетельство о~государственной регистрации программы для ЭВМ 
№\,2016618864 от 09.08.2016.

\bibitem{Gorshenin2016Soft4} \Au{Горшенин~А.\,К., Королев~В.\,Ю.} 
Программный модуль предсказания осадков на основе исторических паттернов. 
Свидетельство о~государственной регистрации программы для ЭВМ №\,2016618887 
от 09.08.2016.

\bibitem{Leadbetter1991} \Au{Leadbetter M.\,R.} 
On a~basis for ``Peaks over Threshold'' modeling~// Stat. Probabil. Lett., 
1991. Vol.~12. No.\,4. P. 357--362.

\bibitem{Gnedenko1996} \Au{Gnedenko~B.\,V., Korolev~V.\,Yu.} Random summation: Limit 
theorems and applications.~--- Boca Raton, FL, USA: CRC Press, 1996. 288~p.

\bibitem{Balkema1974} \Au{Balkema~A., de~Haan~L.} 
Residual life time at great age~// Ann. Probab., 1974. Vol.~2. No.\,5. P.~792--804.

\bibitem{Pickands1975} \Au{Pickands~J., III.} Statistical inference
 using extreme order statistics~// Ann. Stat., 1975. Vol.~3. No.\,1. P.~119--131.

\bibitem{Brey1996} \Au{Brey~J.\,J., Prados~A.} Stochastic resonance 
in a~one-dimension Ising model~// Phys. Lett.~A, 1996. Vol.~216. P.~240--246.

\bibitem{Kosko2003} \Au{Kosko~B., Mitaim~S.} Stochastic resonance 
in noisy threshold neurons~// Neural Networks, 2003. Vol.~16. No.\,5. P.~755--761.

\bibitem{Osoba2013} \Au{Osoba~O., Mitaim~S., Kosko~B.} The noisy 
Expectation-Maximization algorithm~// Fluct. Noise Lett., 2013. Vol.~12. No.\,3. 1350012. 30~p.

\bibitem{Ushakov2015} \Au{Ушаков В.\,Г., Ушаков Н.\,Г.} Об усреднении округленных
 данных~// Информатика и~её применения, 2015. Т.~9. Вып.~4. С.~106--109.

\bibitem{Gorshenin2015ICNAAM1} \Au{Kuzmin~V.\,Yu., Gorshenin~A.\,K., Ostroumov~D.\,S., 
Uglitskaya~M.\,G.} Application of GPU and parallel programming on grid methods~// 
AIP Conference Proceedings, 2015. Vol.~1648. 250006. 4~p.

\bibitem{22-a}
\Au{Gorshenin~A., Korolev~V., Kuzmin~V., Zeifman~A.} 
Coordinate-wise versions of the grid method for the analysis of intensities of 
non-stationary information flows by moving separation of mixtures of 
gamma-distribution~// 27th European Conference on Modelling and Simulation
Proceedings.~--- Dudweiler, Germany: Digitaldruck Pirrot GmbH, 2013. P.~565--568.


\bibitem{Gorshenin2012aStab}\Au{Горшенин~А.\,К.} Устойчивость масштабных 
смесей нормальных законов относительно смешивающего распределения~// 
Системы и~средства информатики, 2012. Т.~22. №\,1. С.~136--148.

\bibitem{Gorshenin2012bStab}\Au{Горшенин~А.\,К.} Об устойчивости сдвиговых 
смесей нормальных законов по отношению к~изменениям смешивающего распределения~// 
Информатика и~её применения, 2012. Т.~6. Вып.~2. С.~22--28.

\bibitem{Korolev2015} \Au{Королев В., Корчагин А., Горшенин А.} 
Некоторые свойства дис\-пер\-си\-он\-но-сдви\-го\-вых смесей нормальных законов~// 
Статистические методы оценивания и~проверки гипотез, 2015. Вып.~26. С.~134--153.

\bibitem{Gorshenin2015ICNAAM3} \Au{Gorshenin~A.\,K.} 
On implementation of EM-type algorithms in the 
stochastic models for a matrix computing on GPU~// AIP Conference Proceedings, 
2015. Vol.~1648. 250008. 4~p.

\bibitem{Zolina2013} \Au{Zolina~O., Simmer~C., Belyaev~K., Gulev~S., Koltermann~P.} 
Changes in the duration of European wet and dry spells during the last~60~years~// 
J.~Climate, 2013. Vol.~26. No.\,6. P.~2022--2047.

\bibitem{Gorshenin2016Soft5} \Au{Горшенин~А.\,К., Кагерманов~Ш.\,Ш.} 
Модуль графической идентификации пользователей веб-сер\-ви\-са. 
Свидетельство о~государственной регистрации программы для ЭВМ 
№\,2016661021 от 28.09.2016.

\bibitem{29-a}
\Au{Batanov~G.\,M., Gorshenin~A.\,K., Korolev~V.\,Yu., Ma\-la\-khov~D.\,V., 
Skvortsova~N.\,N.} 
The evolution of probability characteristics of low-frequency plasma turbulence~// 
Math. Models Computer Simulations, 2012. Vol.~4. Iss.~1. P.~10--25.

 \end{thebibliography}

 }
 }

\end{multicols}

\vspace*{-3pt}

\hfill{\small\textit{Поступила в~редакцию 15.01.17}}

%\vspace*{8pt}

\newpage

\vspace*{-24pt}

%\hrule

%\vspace*{2pt}

%\hrule

%\vspace*{8pt}



\def\tit{ON SOME MATHEMATICAL AND~PROGRAMMING METHODS FOR~CONSTRUCTION 
OF~STRUCTURAL MODELS OF~INFORMATION FLOWS}

\def\titkol{On some mathematical and programming methods for~construction of structural models of~information flows}

\def\aut{A.\,K.~Gorshenin}

\def\autkol{A.\,K.~Gorshenin}

\titel{\tit}{\aut}{\autkol}{\titkol}

\vspace*{-9pt}


 \noindent
Institute of Informatics Problems,
Federal Research Center ``Computer Science and Control'' of the Russian Academy 
of Sciences, 44-2~Vavilova Str., Moscow 119333, Russian Federation



\def\leftfootline{\small{\textbf{\thepage}
\hfill INFORMATIKA I EE PRIMENENIYA~--- INFORMATICS AND
APPLICATIONS\ \ \ 2017\ \ \ volume~11\ \ \ issue\ 1}
}%
 \def\rightfootline{\small{INFORMATIKA I EE PRIMENENIYA~---
INFORMATICS AND APPLICATIONS\ \ \ 2017\ \ \ volume~11\ \ \ issue\ 1
\hfill \textbf{\thepage}}}

\vspace*{3pt}



\Abste{The flows of events in the modern information systems are not regular; 
so, the methods of analysis based on the classical theorems that are correct only 
under certain regularity conditions can lead to false conclusions including 
underestimation of risks of extreme events. The key problem of practical modeling 
and analysis of nonstationary information flows is selection of 
statistical methods for estimation of the unknown model parameters.
For these purposes, the so-called method of moving the separation of 
mixtures based on a~special decomposition of the original sample into subsamples 
(windows) and data analysis for each window within the framework of the mixed 
probability models is traditionally used by the members of Prof.\ V.\,Yu.~Korolev's 
Scientific School. The
paper describes the methods of stochastic data analysis 
based on the mixed probability models that can enhance the\linebreak\vspace*{-12pt}}

\Abstend{effectiveness of 
complex information systems research. The development and application of 
the proposed methods can be useful for the 
appropriate areas of applied mathematics and computer sciences.}


\KWE{information system; mixed probability models; moving separation of
mixtures; statistical data analysis; extremal values; noisy data; threshold; 
Peak Over Threshold; Pickands\,--\,Balkema\,--\,de Haan theorem; R$\acute{\mbox{e}}$nyi 
theorem; online software; matrix computing}


\DOI{10.14357/19922264170105}  

%\vspace*{-9pt}

\Ack
\noindent
The research was supported by the Russian Foundation for Basic Research 
(project 15-37-20851) and the RAS Presidium Program No.\,I.33P (project 0063-2016-0015). 

 

\vspace*{6pt}

  \begin{multicols}{2}

\renewcommand{\bibname}{\protect\rmfamily References}
%\renewcommand{\bibname}{\large\protect\rm References}

{\small\frenchspacing
 {%\baselineskip=10.8pt
 \addcontentsline{toc}{section}{References}
 \begin{thebibliography}{99}

\bibitem{1-g}
\Aue{Korolev, V.\,Yu.} 2011. 
\textit{Veroyatnostno-statisticheskie metody dekompozitsii volatil'nosti 
khaoticheskikh protsessov} [Probabilistic and statistical methods of decomposition 
of volatility of chaotic processes]. Moscow: Moscow University Publishing House. 512~p.

\bibitem{2-g}
\Aue{Korolev, V.\,Yu., A.\,K.~Gorshenin, S.\,K.~Gulev, and K.\,P.~Belyaev}. 2015. 
Statisticheskoe modelirovanie turbulentnykh potokov tepla mezhdu okeanom i~at\-mo\-sfe\-roy 
s~pomoshch'yu metoda skol'zyashchego razdeleniya konechnykh normal'nykh smesey 
[Statistical modeling of air--sea turbulent heat fluxes by the method of 
moving separation of finite normal mixtures]. 
\textit{Informatika i~ee Primeneniya~---  Inform. Appl.} 9(4):3--13.

\bibitem{3-g}
\Aue{Gorshenin,~A.\,K., and V.\,Yu.~Korolev.} 2016. 
A~methodology for the identification of extremal loading in data flows 
in information systems. \textit{Comm. Com. Inf. Sc.} 638:94--103.

\bibitem{4-g}
\Aue{Gorshenin,~A.\,K., and V.\,Yu.~Korolev}. 2016. A~noising method for the 
identification of the stochastic structure of information flows. 
\textit{Comm. Com. Inf. Sc.} 678:1--11.

\bibitem{5-g}
\Aue{Gorshenin,~A.\,K., and V.\,Yu.~Kuzmin}. 2016. Primenenie arkhitektury CUDA 
pri realizatsii setochnykh algoritmov dlya metoda skol'zyashchego razdeleniya 
smesey [Application of the CUDA architecture for implementation of grid-based 
algorithms for the method of moving separation of mixtures]. 
\textit{Sistemy i~Sredstva Informatiki~--- Systems and Means of Informatics} 26(4):60--73.

\bibitem{6-g}
\Aue{Gorshenin,~A.\,K.} 2016. Kontseptsiya onlayn-kompleksa dlya stokhasticheskogo 
modelirovaniya real'nykh pro\-tses\-sov [Concept of online service for stochastic 
modeling of real processes]. \textit{Informatika i~ee Primeneniya~--- Inform. Appl.} 
10(1):72--81.

\bibitem{7-g}
\Aue{Gorshenin,~A., S.~Frenkel, and V.~Korolev.} 2016. On a~stochastic approach to 
a~code performance estimation. \textit{AIP Conference Proceedings}. 1738:220010. 4~p.

\bibitem{8-g}
\Aue{Gorshenin, A.\,K., and V.~Kuzmin.} 2015. Online system for the construction of
structural models of information flows. \textit{7th 
 Congress (International) on Ultra Modern Telecommunications and Control Systems 
and Workshops (ICUMT) Proceedings}. Piscataway, NJ. 216--219.

\bibitem{9-g}
\Aue{Gorshenin, A.\,K., and V.~Kuzmin.} 2015. On an interface of the online system 
for a~stochastic analysis of the varied information flows. 
\textit{AIP Conference Proceedings}. 1738:220010. 4~p.

\bibitem{10-g}
\Aue{Korolev, V.\,Yu., A.\,K.~Gorshenin, S.\,K.~Gulev, and K.\,P.~Belyaev.} 2015. 
Statistical modeling of air--sea turbulent heat fluxes by finite mixtures 
of Gaussian distributions. \textit{Comm. Com. Inf. Sc.} 564:152--162.

\bibitem{11-g}
\Aue{Gorshenin,~A.\,K.} 2016.  Programmnyy modul' ana\-li\-za statisticheskikh 
kharakteristik osadkov [The software module to analyze the statistical 
characteristics of precipitation]. Certificate RF of State Registration 
of Computer Programs No.\,2016618864.

\bibitem{12-g}
\Aue{Gorshenin,~A.\,K., and V.\,Yu.~Korolev.} 2016. Programmnyy modul' predskazaniya 
osadkov na osnove istoricheskikh patternov  [The software module for precipitation 
prediction based on historical patterns]. Certificate RF of State Registration of 
Computer Programs No.\,2016618887.

\bibitem{13-g}
\Aue{Leadbetter, M.\,R.} 1991. On a~basis for ``Peaks over Threshold'' modeling. 
\textit{Stat. Probabil. Lett.} 12(4):357--362.

\bibitem{14-g}
\Aue{Gnedenko,~B.\,V., and V.\,Yu.~Korolev.} 1996. 
\textit{Random summation: Limit theorems and applications}. Boca Raton, FL: 
CRC Press, 1996. 288~p.

\bibitem{15-g}
\Aue{Balkema,~A., and L.~de Haan.} 1974. Residual life time at great age. 
\textit{Ann. Probab.} 2(5):792--804.

\bibitem{16-g}
\Aue{Pickands,~J., III.} 1975. Statistical inference using extreme order statistics. 
\textit{Ann. Stat.} 3(1):119--131.

\bibitem{17-g}
\Aue{Brey,~J.\,J., and A.~Prados.} 1996. Stochastic resonance in 
a~one-dimension Ising model. \textit{Phys. Lett.~A} 216:240--246.

\bibitem{18-g}
\Aue{Kosko,~B., and S.~Mitaim}. 2003. Stochastic resonance in noisy threshold neurons. 
\textit{Neural Networks} 16(5):755--761.

\bibitem{19-g}
\Aue{Osoba,~O., S.~Mitaim, and B.~Kosko.} 2013. The noisy Expectation-Maximization 
algorithm. \textit{Fluct. Noise Lett.} 12(3):1350012. 30~p.

\bibitem{20-g}
\Aue{Ushakov, V.\,G., and N.\,G.~Ushakov}. 2015. Ob usrednenii okruglennykh dannykh 
[On averaging of rounded data]
\textit{Informatika i~ee Primeneniya~---  Inform. Appl.} 9(4):106--109.

\bibitem{21-g}
\Aue{Kuzmin,~V.\,Yu., A.\,K.~Gorshenin, D.\,S.~Ostroumov, and M.\,G.~Uglitskaya.} 2015. 
Application of GPU and parallel programming on grid methods.  
\textit{AIP Conference Proceedings}. 1648:250006. 4~p.

\bibitem{22-a-g}
\Aue{Gorshenin,~A., V.~Korolev, V.~Kuzmin, and A.~Zeifman.}
 2013. Coordinate-wise versions of the grid method for the analysis of intensities 
 of non-stationary information flows by moving separation of mixtures of 
 gamma-distribution. \textit{27th European Conference on Modelling and Simulation
 Proceedings}. Dudweiler, Germany. 565--568.


\bibitem{22-g}
\Aue{Gorshenin,~A.\,K.} 2012. Ustoychivost' masshtabnykh smesey normal'nykh zakonov 
otnositel'no sme\-shi\-va\-yushche\-go raspredeleniya [Stability of normal scale mixtures 
with respect to variations in mixing distribution]. 
\textit{Sistemy i~Sredstva Informatiki~--- Systems and Means of Informatics} 
22(1):136--148.

\bibitem{23-g}
\Aue{Gorshenin,~A.\,K.} 2012. Ob ustoychivosti sdvigovykh smesey normal'nykh zakonov 
po otnosheniyu k~izmeneniyam smeshivayushchego raspredeleniya [On stability of 
normal location mixtures with respect to variations in mixing distribution]. 
\textit{Informatika i~ee Primeneniya~--- Inform. Appl.} 6(2):22--28.

\bibitem{24-g}
\Aue{Korolev, V.\,Yu., A.\,Yu.~Korchagin, and A.\,K.~Gorshenin.} 2015. 
Nekotorye svoystva dispersionno-sdvigovykh smesey normal'nykh zakonov 
[Some properties of variance-mean normal mixtures]. 
\textit{Statisticheskie metody otsenivaniya i~proverki gipotez} 
[Statistical Methods of Estimation and Hypothesis Testing] 26:134--153.

\bibitem{25-g}
\Aue{Gorshenin, A.\,K.} 2015. On implementation of EM-type algorithms in the 
stochastic models for a matrix computing on GPU. \textit{AIP Conference Proceedings}.
 1648:250008. 4~p.

\bibitem{26-g}
\Aue{Zolina,~O., C.~Simmer, K.~Belyaev, S.~Gulev, and P.~Koltermann.} 
2013. Changes in the duration of European wet and dry spells during the last~60~years. 
\textit{J.~Climate} 26(6):2022--2047.

\bibitem{27-g}
\Aue{Gorshenin,~A.\,K., and Sh.\,Sh.~Kagermanov.} 2016. Modul' graficheskoy 
identifikatsii pol'zovateley veb-servisa [The module for web service user's 
graphical identification]. Certificate RF of State Registration Of Computer 
Programs No.\,2016661021.

\bibitem{29-g-1}
\Aue{Batanov,~G.\,M., A.\,K.~Gorshenin, V.\,Yu.~Korolev, D.\,V.~Malakhov, 
and N.\,N.~Skvortsova.} 2012. 
The evolution of probability characteristics of low-frequency plasma turbulence. 
\textit{Math. Models Computer Simulations} 4(1):\linebreak 10--25.
\end{thebibliography}

 }
 }

\end{multicols}

\vspace*{-3pt}

\hfill{\small\textit{Received January 15, 2017}}



\Contrl

\noindent
\textbf{Gorshenin Andrey K.} (b.\ 1986)~--- 
Candidate of Science (PhD) in physics and mathematics, associate professor,
leading scientist,
Institute of Informatics Problems,
Federal Research Center ``Computer Science and Control'' 
of the Russian Academy of Sciences, 44-2~Vavilova Str., Moscow 119333, 
Russian Federation; \mbox{agorshenin@frccsc.ru}


\label{end\stat}


\renewcommand{\bibname}{\protect\rm Литература} 