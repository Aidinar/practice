\def\stat{ushakovi}

\def\tit{ОБ УСРЕДНЕНИИ ОКРУГЛЕННЫХ ДАННЫХ$^*$}

\def\titkol{Об усреднении округленных данных}

\def\aut{В.\,Г.~Ушаков$^1$,  Н.\,Г.~Ушаков$^2$}

\def\autkol{В.\,Г.~Ушаков,  Н.\,Г.~Ушаков}

\titel{\tit}{\aut}{\autkol}{\titkol}

{\renewcommand{\thefootnote}{\fnsymbol{footnote}} \footnotetext[1]
{Работа выполнена при финансовой поддержке РНФ
(проект 14-11-00364).}}


\renewcommand{\thefootnote}{\arabic{footnote}}
\footnotetext[1]{Факультет вычислительной математики и~кибернетики Московского государственного 
университета имени М.\,В.~Ломоносова;  
Институт проблем информатики Федерального исследовательского
центра <<Информатика и~управление>> Российской академии наук,
vgushakov@mail.ru}
\footnotetext[2]{Институт проблем
технологии микроэлектроники и~особочистых материалов Российской академии наук,
Черноголовка; Норвежский на\-уч\-но-тех\-но\-ло\-ги\-че\-ский университет,
Тронхейм, ushakov@math.ntnu.no}

\vspace*{6pt}


\Abst{Значения каждой наблюдаемой величины регистрируются с~конечной точностью, определяемой 
разрешающей способностью
измерительного инструмента. Естественно ожидать, что ошибки округления могут существенно повлиять
на точность восстановления математического ожидания наблюдаемой величины. С~другой 
стороны, часто исследователь имеет возможность воздействовать на наблюдения перед 
их регистрацией, например добавлять аддитивную или мультипликативную шумовую 
составляющую. В~работе изучается связь между ошибкой измерения,
погрешностью округления и~точностью восстановления измеряемой величины при
усреднении многократно проводимых измерений. Показано, что при одном и~том же
уровне округления точность восстановления тем выше, чем больше, в~определенном смысле,
ошибка измерения.}

\KW{округленные данные; закон больших чисел; полная вариация; разложение вероятностных распределений}

\DOI{10.14357/19922264150412}

\vspace*{6pt}

\vskip 14pt plus 9pt minus 6pt

\thispagestyle{headings}

\begin{multicols}{2}

\label{st\stat}

\section{Введение}

Округление экспериментальных данных является неизбежным этапом их статистической 
обработки. Так как аналоговые сигналы хранятся и~обрабатываются в~цифровом виде,
они подвергаются ана\-ло\-го-циф\-ро\-во\-му преобразованию. Разрядность современных 
ана\-ло\-го-циф\-ро\-вых преобразователей большая, но конечная.

Исследования показали, что игнорирование этого факта может привести 
к~значительным ошибкам при решении  задач статистической обработки данных. 
В~последнее время
появилось много работ, где исследуется влияние погрешностей округления на 
решение различных статистических задач и~предлагаются методы, которые можно 
применять в~этой ситуации. В~работах~[1--5] можно найти как решения задач 
статистического анализа округленных данных, так и~обширную библиографию.

В настоящей работе рассматривается следу\-ющая задача. Предположим, что помимо 
случайных ошибок, сопутствующих каждому реальному измерению, наблюдаемые
 значения содержат еще одну погрешность, связанную с~ограниченной точностью 
 измеряющего прибора (т.\,е.\ подвергаются округлению). При отсутствии этой 
 погрешности и~при наличии
 достаточно большого числа независимых измерений можно сколь угодно точно 
 определить измеряемую величину, если распределение случайной ошибки таково, 
 что выполняется
 закон больших чисел. Можно ли надеяться на это, если есть еще и~погрешность 
 округления? В~работе дается положительный ответ на этот вопрос. Средство достижения 
 неограниченной
 точности кажется парадоксальным: наблюдаемая величина должна быть дополнительно 
 зашумлена до округления, причем дополнительный шум не должен зависеть от наблюдаемой 
 величины и~чем больше его дисперсия, тем большей точности можно достигнуть.

Существуют различные типы округления: к~меньшему, к~большему, 
к~ближайшему, случайное округление, чередующееся округление и~др.
Будем рассматривать округление к~ближайшему. Кроме того, не
ограничивая общности, будем считать, что шаг округления равен
единице. Таким образом, для действительного числа~$x$ с~целой частью
$[x]$ и~дробной частью $\{x\}$ округленное значение (обозначим его~$x^\ast$) 
равно $[x]$, если $\{x\}\hm<0{,}5$, и~$[x]+1$, если
$\{x\}\hm\geqslant0{,}5$. Заметим, что $x^\ast\hm=[x\hm+1/2]$.


Итак,  пусть случайные величины $X_1,\ldots,X_n$ являются результатами 
измерений величины~$\mu$. Предположим теперь, что сами случайные величины 
$X_1,\ldots,X_n$
наблюдаются с~некоторыми дополнительными ошибками $\varepsilon_1,\ldots,\varepsilon_n$, 
а~затем еще регистрируются
в~округленном виде. Таким образом, наблюдения имеют вид 
$(X_1\hm+\varepsilon_1)^\ast,\ldots,(X_n+\varepsilon_n)^\ast$.


\section{Вспомогательные результаты}

В данном разделе докажем несколько вспомогательных результатов.
Будем говорить, что функция распределения $G(x)$ является компонентой
функции распределения $F(x)$, если существует функция распределения $H(x)$
такая, что
$$
F(x)=\int\limits_{-\infty}^\infty G(x-t)\,dH(t)\,.
$$

Полной вариацией действительной функции $f(x)$, определенной на конечном или
бесконечном отрезке, называется величина
$$
{V}(f)=\sup\sum\limits_{i=0}^n|f(x_{i+1})-f(x_i)|\,,
$$
где верхняя грань берется по всем значениям~$n$ и~всевозможным наборам
$x_0\hm<x_1<\cdots<x_n$.

\smallskip

\noindent
\textbf{Лемма 1.}\ \textit{Если равномерное на отрезке $[0,1]$ распределение является
компонентой распределения случайной величины~$X$, то дробная часть~$\{X\}$
имеет равномерное на $[0,1)$ распределение.}

\smallskip

\noindent
Д\,о\,к\,а\,з\,а\,т\,е\,л\,ь\,с\,т\,в\,о\,.\ \
 Обозначим функции распределения случайных величин~$X$
и~$\{X\}$ соответственно $F(x)$ и~$F_1(x)$, а равномерную на $[0,1]$ функцию
распределения~--- $U(x)$.
По условию леммы существует функция распределения $G(x)$ такая, что
$$
F(x)=\int\limits_{-\infty}^\infty G(x-t)\,dU(t)=\int\limits_0^1G(x-t)\,dt\,.
$$
Следовательно, для любых $a\hm<b$
\begin{multline*}
{\sf P}(a\leqslant X\leqslant b)=F(b)-F(a)={}\\
{}=\int\limits_0^1G(b-t)\,dt-\int\limits_0^1G(a-t)\,dt
\end{multline*}
(функция распределения $F(x)$ непрерывна, поскольку содержит в~качестве
компоненты равномерное распределение). Пусть $0\hm<x\hm<1$. Легко видеть, что
$$
F_1(x)=\sum\limits_{k=-\infty}^\infty {\sf P}(k\leqslant X\leqslant k+x)\,,
$$
поэтому
\begin{multline*}
F_1(x)= {}\\
{}=\sum\limits_{k=-\infty}^\infty \left[
\int\limits_0^1G(k+x-t)\,dt-\int\limits_0^1G(k-t)\,dt
\right]={}
\end{multline*}

\noindent
\begin{multline*}
{}=\sum\limits_{k=-\infty}^\infty
\left[
\int\limits_{\,x+k-1}^{x+k}G(y)\,dy-\int\limits_{k-1}^{k}G(y)\,dy
\right]={}\\
{}=\sum\limits_{k=-\infty}^\infty
\left[
\int\limits_{k}^{k+x}G(y)\,dy-\int\limits_{k-1}^{k-1+x}G(y)\,dy
\right]={}\\
{}=\lim_{m\to\infty}\sum\limits_{k=-\infty}^m
\left[
\int\limits_{k}^{k+x}G(y)\,dy-\int\limits_{k-1}^{k-1+x}G(y)\,dy
\right]={}\\
{}=\lim_{m\to\infty}\int\limits_{m}^{m+x}G(y)\,dy=x\,,
\end{multline*}
т.\,е.\ $F_1(x)$ равномерна на $[0,1)$.
%\square

\smallskip

\noindent
\textbf{Лемма 2.}\ \textit{Пусть $X$~--- абсолютно непрерывная случайная величина
с непрерывной плотностью распределения $f(x)$. Тогда}

\noindent
$$
\left|{\sf E}\{X\}-\fr{1}{2}\right|\leqslant\fr{{V}(f)}{2}\,.
$$


\noindent
Д\,о\,к\,а\,з\,а\,т\,е\,л\,ь\,с\,т\,в\,о\,.\ \ Обозначим

\noindent
$$
p_k=\int\limits_k^{k+1}f(x)\,dx=\int\limits_0^1f(x+k)\,dx\,,\enskip k=0,\pm1,\pm2,\ldots
$$
Для каждого $k$ пусть $x'_k$, $x''_k$ и~$x^{(0)}_k$~--- точки интервала
$[0,1]$ такие, что
\begin{align*}
f(x'_k+k)&=\min\limits_{0\leqslant x\leqslant1}f(x+k)\,;\\
f(x''_k+k)&=\max\limits_{0\leqslant x\leqslant 1}f(x+k)\,;\\
\int\limits_0^1xf(x+k)\,dx&=f(x^\ast_k+k)\int\limits_0^1x\,dx=
\fr{f(x^\ast_k+k)}{2}\,.
\end{align*}
В силу равенства
$$
{\sf P}\left(\{X\}\leqslant x\right)=\sum\limits_{k=-\infty}^\infty 
{\sf P}(k\leqslant X\leqslant k+x)
$$
при $0<x<1$, получаем
$$
{\sf E}\{X\}=\sum\limits_{k=-\infty}^\infty\int\limits_0^1xf(x+k)\,dx
$$
и, следовательно,
\begin{multline*}
\left|{\sf E}{\{X\}}-\fr{1}{2}\right|={}\\
{}=
\left|\sum\limits_{k=-\infty}^\infty\left(\int\limits_0^1xf(x+k)\,dx-
p_k\int\limits_0^1x\,dx\right)\right|\leqslant{}\\
{}\leqslant\sum\limits_{k=-\infty}^\infty\left|\int\limits_0^1xf(x+k)\,dx-
p_k\int\limits_0^1xdx\right|={}
\end{multline*}

\noindent
\begin{multline*}
{}=
\fr{1}{2}\sum\limits_{k=-\infty}^\infty|f(x^\ast_k+k)-p_k|\leqslant{}\\
{}\leqslant\fr{1}{2}\sum\limits_{k=-\infty}^\infty|f(x'_k+k)-f(x''_k+k)|\leqslant
\fr{{V}(f)}{2}\,.
\end{multline*}
%\square

\vspace*{-12pt}

\section{ Закон больших чисел}

Итак, имеется последовательность независимых и~одинаково распределенных
случайных величин $X_1,X_2,\ldots$ с~математическим ожиданием~$\mu$ и~последовательность
независимых (и независимых от $X_1,X_2,\ldots$) и~одинаково распределенных абсолютно непрерывных
случайных величин
$\varepsilon_1,\varepsilon_2,\ldots$ с~нулевым математическим ожиданием.

Следующая теорема дает простое достаточное условие, когда~$\mu$~может быть 
восстановлено
с неограниченной точностью (разумеется, при неограниченном числе измерений).

\smallskip

\noindent
\textbf{Теорема~1.}\  \textit{Если равномерное на отрезке единичной длины распределение
является компонентой распределения случайных величин $\varepsilon_i$, то}
$$
\fr{1}{n}\sum\limits_{i=1}^n(X_i+\varepsilon_i)^\ast
\stackrel{\mathrm{a.s.}}{\longrightarrow}\mu
$$
при $n\hm\to\infty$.

\vspace{2pt}

\noindent
Д\,о\,к\,а\,з\,а\,т\,е\,л\,ь\,с\,т\,в\,о\,.\ \ 
Из условия теоремы следует, что равномерное на отрезке $[0,1]$
распределение является компонентой распределения случайных величин
$X_i\hm+\varepsilon_i+1/2$; следовательно, в~силу леммы~1 распределение дробной части
$\{X_i\hm+\varepsilon_i+1/2\}$ равномерно на $[0,1)$ и

\vspace*{2pt}

\noindent
$$
{\sf E}\left\{X_i+\varepsilon_i+\fr{1}{2}\right\}=\fr{1}{2}\,.
$$
С учетом этого получаем

\vspace*{-4pt}

\noindent
\begin{multline*}
{\sf E}\left(X_i+\varepsilon_i\right)^\ast=
{\sf E}\left[X_i+\varepsilon_i+\fr{1}{2}\right]={}\\
{}={\sf E}\left(X_i+\varepsilon_i+\fr{1}{2}\right)-
{\sf E}\left\{X_i+\varepsilon_i+\fr{1}{2}\right\}=\mu\,.
\end{multline*}
Результат теперь следует из классического усиленного закона больших чисел.
%\square

\vspace*{2pt}

\noindent
\textbf{Теорема~2.}\ \textit{Если плотность распределения случайных 
величин~$\varepsilon_i$
имеет ограниченную вариацию и~$\lambda\hm\to\infty$, то}

\noindent
$$
\fr{1}{n}\sum\limits_{i=1}^n(X_i+\lambda\varepsilon_i)^\ast
\stackrel{\mathrm{a.s.}}{\longrightarrow}\mu
$$
\textit{при $n\to\infty$, т.\,е.\ предел (почти наверное) левой час\-ти при $n\hm\to\infty$
стремится к~$\mu$ при стремлении $\lambda$ к~бесконечности.}

\smallskip

\noindent
Д\,о\,к\,а\,з\,а\,т\,е\,л\,ь\,с\,т\,в\,о\,.\ \ 
В~силу усиленного закона больших чисел теорема будет доказана,
если  показать, что
\begin{equation}
\lim\limits_{\lambda\to\infty}{\sf E}\left(X_i+\lambda\varepsilon_i\right)^\ast=\mu\,.
\label{e1-us}
\end{equation}
В свою очередь, в~силу равенства

\vspace*{-2pt}

\noindent
\begin{multline*}
{\sf E}\left(X_i+\lambda\varepsilon_i\right)^\ast=
{\sf E}\left(X_i+\lambda\varepsilon_i+\fr{1}{2}\right)-{}\\
{}-
{\sf E}\left\{X_i+\lambda\varepsilon_i+\fr{1}{2}\right\}
\end{multline*}

\noindent
уравнение~(1)~будет доказано, если  показать, что

\noindent
\begin{equation}
\lim\limits_{\lambda\to\infty}{\sf E}\left\{X_i+\lambda\varepsilon_i+
\fr{1}{2}\right\}=\fr{1}{2}\,.
\label{e2-us}
\end{equation}

Пусть $f(x)$ и~$f_\lambda(x)$~--- плотности распределения случайных величин
$X_i+\lambda\varepsilon_i+1/2$ и~$\lambda\varepsilon_i$ соответственно.
Поскольку

\vspace*{2pt}

\noindent
$$
f_\lambda(x)=\fr{1}{\lambda}\,f_1\left(\fr{x}{\lambda}\right)\,,
$$
имеем

\noindent
$$
{V}(f_\lambda)=\fr{1}{\lambda}\,{V}(f_1)\,.
$$
С другой стороны, ${V}(f)\hm\leqslant {V}(f_\lambda)$. Таким образом,

\vspace*{3pt}

\noindent
$$
\lim\limits_{\lambda\to\infty} {V}(f)=0\,.
$$

\vspace*{-2pt}

\noindent
Отсюда в~силу леммы~2 следует~(2) и,~следовательно,~(1).

%\square

Отметим, что дисперсия дополнительного зашумления пропорциональна~$\lambda^2$.
Таким образом, верхняя граница точности восстановления оказывается тем меньше,
чем больше дисперсия ошибки. Цена, которую приходится за это платить,~---
необходимость увеличения числа наблюдений.

\vspace*{-7pt}

{\small\frenchspacing
 {%\baselineskip=10.8pt
 \addcontentsline{toc}{section}{References}
 \begin{thebibliography}{9}
 
 \vspace*{-2pt}
 
\bibitem{1-us}
\Au{Hall P.} The influence of rounding errors on some nonparametric estimators
of a density and its derivatives~// SIAM J. Appl. Math., 1982.
Vol.~42. P.~390--399.
\bibitem{5-us} %2
\Au{Zhidong B., Shurong~Z., Baoxue~Z., Guorong~H.} Statistical analysis for
rounded data~// J.~Stat. Plan. Infer., 2009. Vol.~139. P.~2526--2542.

\bibitem{3-us} %3
\Au{Schneeweiss H., Komlos~J., Ahmad A.\,S.} Symmetric and asymmetric rounding:
A~review and some new results~// AStA Adv. Stat. Anal., 2010.
Vol.~94. P.~247--271.
\bibitem{4-us} %4
\Au{Weiming L., Tianqing L., Zhidong~B.} Rounded data analysis based on ranked
set sample~// Stat. Papers, 2012. Vol.~53. P.~439--455.
\bibitem{2-us} %5
\Au{Ningning Z., Zhidong~B.} Analysis of rounded data in mixture normal model~//
Stat. Papers, 2012. Vol.~53. P.~895--914.

 \end{thebibliography}

 }
 }

\end{multicols}

\vspace*{-10pt}

\hfill{\small\textit{Поступила в~редакцию 10.11.15}}

%\vspace*{8pt}

\newpage

\vspace*{-24pt}

%\hrule

%\vspace*{2pt}

%\hrule

%\vspace*{8pt}

\def\tit{ON AVERAGING OF ROUNDED DATA}

\def\titkol{On averaging of rounded data}

\def\aut{V.\,G.~Ushakov$^{1,2}$ and N.\,G.~Ushakov$^{3,4}$}

\def\autkol{V.\,G.~Ushakov and N.\,G.~Ushakov}

\titel{\tit}{\aut}{\autkol}{\titkol}

\vspace*{-9pt}


\noindent
$^1$Department of Mathematical 
Statistics, Faculty of Computational Mathematics and Cybernetics,\linebreak 
$\hphantom{^1}$M.\,V.~Lomonosov Moscow State University, 1-52 Leninskiye Gory, Moscow 119991, 
GSP-1, Russian\linebreak 
$\hphantom{^1}$Federation

\noindent
$^2$Institute of Informatics Problems, 
Federal Research Center ``Computer Science and Control'' 
of Russian Academy\linebreak 
$\hphantom{^1}$of Sciences, 44-2 Vavilov Str., Moscow 119333, Russian Federation

\noindent
$^3$Institute of 
 Microelectronics Technology and High-Purity Materials of the Russian Academy of 
 Sciences,\linebreak 
$\hphantom{^1}$6~Academician Osipyan Str., Chernogolovka, Moscow Region 142432, 
 Russian Federation
 
\noindent
 $^4$Norwegian University of Science and Technology, 
 15A S.P.\ Andersensvei, Trondheim 7491, Norway


\def\leftfootline{\small{\textbf{\thepage}
\hfill INFORMATIKA I EE PRIMENENIYA~--- INFORMATICS AND
APPLICATIONS\ \ \ 2015\ \ \ volume~9\ \ \ issue\ 4}
}%
 \def\rightfootline{\small{INFORMATIKA I EE PRIMENENIYA~---
INFORMATICS AND APPLICATIONS\ \ \ 2015\ \ \ volume~9\ \ \ issue\ 4
\hfill \textbf{\thepage}}}

\vspace*{3pt}

\Abste{Each observed value is registered with finite accuracy which is determined by the
sensitivity of the equipment. It is expected that rounding errors could play 
an important role in the estimation of the mean of
the observed value. On the other hand, the researcher usually
has a possibility to affect the observation before its registration, for example,
to intensify it or to add some additional component. This paper
studies the relationship between the measurement error, rounding error, 
and the accuracy of the reconstruction of the observed value for
the case of averaging
of repeated measurements. It is demonstrated that under a fixed rounding level, in some sense, the greater the measurement error, the higher the reconstruction accuracy.}

\KWE{rounded data; law of large numbers; total variation; decomposition of probability distributions}

\DOI{10.14357/19922264150412}

\Ack
    \noindent
The work was supported by the Russian Science Foundation
(project 14-11-00364).


%\vspace*{3pt}

  \begin{multicols}{2}

\renewcommand{\bibname}{\protect\rmfamily References}
%\renewcommand{\bibname}{\large\protect\rm References}

{\small\frenchspacing
 {%\baselineskip=10.8pt
 \addcontentsline{toc}{section}{References}
 \begin{thebibliography}{9}

\bibitem{1-us-1}
\Aue{Hall, P.} 1982. The influence of rounding errors on some nonparametric estimators
of a density and its derivatives. \textit{SIAM J.~Appl. Math.} 42:390--399.

\bibitem{5-us-1} %2
\Aue{Zhidong, B., Z.~Shurong, Z.~Baoxue, and H.~Guorong.} 2009. Statistical analysis for
rounded data. \textit{J.~Stat. Plan. Infer.} 139:2526--2542.

\bibitem{3-us-1}
\Aue{Schneeweiss, H., J.~Komlos, and A.\,S.~Ahmad.} 2010. 
Symmetric and asymmetric rounding:
A~review and some new results. \textit{AStA Adv. Stat. Anal.} 94:247--271.
\bibitem{4-us-1}
\Aue{Weiming, L., L.~Tianqing, and B.~Zhidong.} 2012. 
Rounded data analysis based on ranked
set sample. \textit{Stat. Papers} 53:439--455.

\bibitem{2-us-1} %5
\Aue{Ningning, Z., and B.~Zhidong.} 2012. Analysis of rounded data in mixture normal model.
\textit{Stat. Papers} 53:895--914.

\end{thebibliography}

 }
 }

\end{multicols}

\vspace*{-3pt}

\hfill{\small\textit{Received November 10, 2015}}


\Contr

\noindent
\textbf{Ushakov Vladimir G.}\ (b.\ 1952)~---
Doctor of Science in physics and mathematics, professor, Department of Mathematical 
Statistics, Faculty of Computational Mathematics and Cybernetics, 
M.\,V.~Lomonosov Moscow State University, 1-52 Leninskiye Gory, Moscow 119991, 
GSP-1, Russian Federation; senior scientist, Institute of Informatics Problems, 
Federal Research Center ``Computer Science and Control'' 
of Russian Academy of Sciences, 44-2 Vavilov Str., Moscow 119333, Russian Federation; 
vgushakov@mail.ru 

\vspace*{4pt} 

\noindent
\textbf{Ushakov Nikolai G.}\ (b.\ 1952)~---
 Doctor of Science in physics and mathematics; leading scientist, Institute of 
 Microelectronics Technology and High-Purity Materials of the Russian Academy of 
 Sciences, 6~Academician Osipyan Str., Chernogolovka, Moscow Region 142432, 
 Russian Federation; professor, Norwegian University of Science and Technology, 
 15A S.P.\ Andersensvei, Trondheim 7491, Norway; ushakov@math.ntnu.no 


\label{end\stat}


\renewcommand{\bibname}{\protect\rm Литература}