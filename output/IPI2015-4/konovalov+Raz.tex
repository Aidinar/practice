\def\stat{kon-raz}

\def\tit{ОБЗОР МОДЕЛЕЙ И~АЛГОРИТМОВ РАЗМЕЩЕНИЯ ЗАДАНИЙ 
В~СИСТЕМАХ С~ПАРАЛЛЕЛЬНЫМ ОБСЛУЖИВАНИЕМ$^*$}

\def\titkol{Обзор моделей и~алгоритмов размещения заданий 
в~системах с~параллельным обслуживанием}

\def\aut{М.\,Г. Коновалов$^1$, Р.\,В.~Разумчик$^2$}

\def\autkol{М.\,Г. Коновалов, Р.\,В.~Разумчик}

\titel{\tit}{\aut}{\autkol}{\titkol}

{\renewcommand{\thefootnote}{\fnsymbol{footnote}} \footnotetext[1]
{Работа выполнена при частичной поддержке РФФИ (проекты 15-07-03406 и~13-07-00223).}}


\renewcommand{\thefootnote}{\arabic{footnote}}
\footnotetext[1]{Институт проблем информатики Федерального исследовательского
центра <<Информатика и~управление>> Российской академии наук,
mkonovalov@ipiran.ru}
\footnotetext[2]{Институт проблем информатики Федерального исследовательского
центра <<Информатика и~управление>> Российской академии наук;
Российский университет дружбы народов, 
rrazumchik@ipiran.ru}

    \Abst{Представлен обзор зарубежных публикаций по проблеме эффективного распределения заданий 
    в~системах обслуживания (проблема диспетчеризации). Отобранные для обзора модели касаются систем 
с~независимыми параллельными безотказными серверами и~случайными потоками заданий, не имеющих 
внутренней структуры. Общая трактовка проблемы сформулирована как задача оптимизации, которая имеет 
многочисленные варианты в~зависимости от дополнительных предположений. Изложение моделей следует 
классификации этих предположений по объему и~характеру априорной информации, возможностям 
наблюдения в~процессе функционирования системы и~критериям эффективности. Приведено описание 
основных отмеченных в~литературе и~используемых на практике алгоритмов диспетчеризации и~их 
сравнительных свойств. Отмечены основные методы, используемые при анализе моделей и~построении 
стратегий размещения заданий. Обзор призван акцентировать внимание на одной из важных и~далеких от 
окончательного разрешения проблем обработки больших объемов информации.}

\KW{системы с~параллельным обслуживанием; стратегии размещения заданий; диспетчеризация}

\DOI{10.14357/19922264150406}

\vskip 14pt plus 9pt minus 6pt

\thispagestyle{headings}

\begin{multicols}{2}

\label{st\stat}

\section{Введение}

    Многие современные технические системы представляют собой параллельно 
работающие обслуживающие ресурсы~--- серверы, на которые поступают заявки на 
выполнение заданий. Каж\-дое\linebreak зада\-ние может быть выполнено на ка\-ком-ли\-бо сервере, 
однако выбор заранее однозначно не определен. Требуется построить процедуру выбора 
сервера для выполнения задания. Необходимость в~таких процедурах возникает 
в~разнообразных компьютерных, сетевых, производственных и~иных сис\-те\-мах, т.\,е.\ 
практически везде, где есть необходимость или возможность параллельного 
обслуживания. Появление новых возможностей (услуг, сервисов и~др.)\ влечет 
усложнение техники, пред\-остав\-ля\-ющей эти возможности, и~одновременно рост числа 
пользовательских запросов. Качественная обработка запросов невозможна без 
управляющих механизмов, которые определяют, какой запрос, где и~когда должен быть 
выполнен. Приведем несколько примеров реальных систем, в~которых возникает 
потребность в~эффективных алгоритмах размещения заданий:
    \begin{itemize}
\item распределенные компьютерные системы передачи, обработки и~хранения данных: 
телекоммуникационные сети, грид- и~клауд-сис\-те\-мы, базы данных;
\item узлы, представляющие собой совокупности серверов, соединенных сетью передачи 
данных и~работающих как единое целое (веб-фер\-, цент\-ры обработки данных, 
компьютерные кластеры);
\item разнородные объекты: суперкомпьютеры, веб-про\-вай\-де\-ры, машины 
с~распределенной па\-мятью, хеш-таб\-ли\-цы, когнитивные ра\-дио\-сис\-темы;
\item человеко-машинные системы: транспортные, ремонтные и~иные обслуживающие 
системы.
\end{itemize}

    В таких системах механизмы размещения заданий являются неотъемлемой составной 
частью, причем для соблюдения требований к~качеству обслуживания распределение 
заданий должно происходить как можно быстрее. Например, компании Google и~Amazon 
сообщили, что увеличение среднего времени отклика для веб-поиска на 500~мс приводит 
к потере 1,2\% пользователей~[1].
     
    Важность рассматриваемой проблемы подтверждается большим числом публикаций, 
в настоящее время почти исключительно зарубежных. Сама же проблема является 
сложной, и~относящиеся к~ней задачи чаще всего не удается решить исключительно 
математическими методами. Тем не менее, признавая практическую ценность 
и~плодотворность эвристических идей и~инженерного подхода, для настоящего обзора 
отбирались работы, в~которых математическая модель все же присутствует. Во всех 
отобранных моделях рассматриваемые системы состоят из нескольких параллельных 
и~независимых серверов; их функционирование рассматривается без учета возможных 
отказов; отсутствует внутренняя структура у поступающих заданий. Общая постановка 
задачи сформулирована в~разд.~2. В~разд.~3 содержится описание наиболее известных 
и~употребительных на практике алгоритмов. Изучаемые системы разделяются на два класса 
в зависимости от того, составлены ли они из одинаковых или не\-оди\-на\-ко\-вых серверов. 
Обзор моделей, относящихся к~этим классам, помещен соответственно в~разд.~4 и~5.

\vspace*{-6pt}

\section{Общая задача о размещении заданий}

\vspace*{-2pt}

    Рассматривается модель системы, состоящей из~$N$~параллельных серверов 
$\mathbb{R}_1,\ldots, \mathbb{R}_N$, где $\mathbb{R}$~--- набор параметров $n$-го сервера. 
В~систему поступают задания в~моменты, порожденные входными потоками 
$\mathbb{F}_1,\ldots, \mathbb{F}_M$, где $\mathbb{F}_m$~--- набор параметров, 
определяющий $m$-й поток. В~момент поступления задание направляется в~очередь 
одного из серверов, и~в~дальнейшем, если не оговорено противное, будем считать, что 
переход заявок из одной очереди в~другую невозможен. Каждый сервер работает 
с~поступившими заданиями автономно от остальной части системы.
    
    В число параметров сервера входят дисциплина обслуживания заданий в~очереди 
(английские термины: scheduling, service discipline), а~также характеристики его 
производительности. В~число параметров потока входят характеристики случайного 
процесса, задающего моменты поступления заданий, а также характеристики заданий, 
поступающих из этого потока.
    
    Время пребывания задания в~системе совпадает со временем его пребывания на 
выбранном сер\-вере и~складывается из времени ожидания в~очереди и~времени 
(непосредственного) выполнения. \mbox{Время} ожидания зависит от дисциплины обслуживания. 
Время выполнения задания зависит от характеристик задания и~сервера. Таким образом, 
рассматриваемая модель задается набором пара\-мет\-ров $\mathbb{S}\hm= 
(\mathbb{F},\ldots,\mathbb{F}_M;\mathbb{R},\ldots , \mathbb{R}_N)$.
    
    Распределение заданий по серверам в~системе~$\mathbb{S}$ осуществляется согласно 
стратегии размещения заданий, которую будем называть также диспетчеризацией 
(dispatching) и~которая является основным предметом обсуждения в~этой статье.
    
    Система эволюционирует в~непрерывном времени. Ее полная предыстория 
складывается из моментов поступления заданий, из принятых решений о выборе сервера 
и~из моментов начала и~окончания выполнения заданий. Пусть $\mathfrak{h}_k$~--- 
наблюдаемая часть полной предыстории системы к~моменту поступления $k$-го по счету 
задания. Диспетчеризация в~самом общем виде представляется набором правил 
назначения сервера для по\-сту\-па\-юще\-го задания в~зависимости от наблюдаемой 
предыс\-тории:
%\noindent
    $$
    \sigma=\left\{ \sigma_k(\mathfrak{h}_k),\ k=1,2,\ldots\right\}\,,
    $$
где правило $\sigma_k(\mathfrak{h}_k)$~--- это вероятностное распределение на 
множестве $\{1,\ldots, N\}$. Пусть $\Xi$~---  задан-\linebreak ное подмножество 
систем~$\mathbb{S}$;
    $\Sigma$~--- заданное подмножество диспетчеризаций~$\sigma$, которое назовем 
множест\-вом допустимых стратегий размещения;
    $W\hm= W(\mathbb{S},\sigma)$~--- заданный числовой показатель, который 
характеризует затраты, связанные с~применением диспетчеризации~$\sigma$ 
к~системе~$\mathbb{S}$.
    
    Проблему диспетчеризации можно сформулировать как задачу минимизации, т.\,е.\ 
нахождения такой стратегии размещения~$\sigma^*$, которая обеспечивала бы 
выполнение неравенства
%\vspace*{2pt}
%\noindent
    $$
    W(\sigma^*) \leq \mathop{\mathrm{inf}}\limits_\Sigma W(\sigma)+\varepsilon
    $$
    
%  \vspace*{-4pt}
    
    \noindent
 для любой системы из множества~$\Xi$ и~для заданного значения  $\varepsilon\hm\geq 0$.
    
    Задание множества~$\Xi$ равносильно указанию априорной информации о системе, 
которая имеется при конструировании стратегии размещения заданий. Например, если 
множество~$\Xi$ состоит из одного элемента, то это означает, что при построении 
диспетчеризации можно считать известными все параметры системы~$\mathbb{S}$. 
Напротив, неполнота априорной информации о~параметрах из набора~$\mathbb{S}$ 
означает, что диспетчер, раз\-ме\-ща\-ющий задания, может иметь дело с~любой из систем 
в~диапазоне неопределенности, заданном множеством~$\Xi$, и~при этом не знать 
в~точности, каким именно объектом он управляет. Согласно определению, оптимальная 
диспетчеризация~$\sigma^*$ должна в~этом случае обладать определенной 
<<инвариантностью>> по отношению к~множеству систем~$\Xi$.
    
    Задание множества~$\Sigma$ тесно связано с~описанием возможностей наблюдения 
в~процессе функционирования системы. Например, нельзя считать до\-пус\-ти\-мой 
стратегию, опирающуюся на длины очередей, если эти очереди невозможно отслеживать 
в~реальном времени. Однако происхождение множества~$\Sigma$ вызвано не только 
фактором наблюда\-емости. Ограничение множества допустимых стра-\linebreak\vspace*{-12pt}

\pagebreak

\noindent
тегий может быть 
вызвано техническими или иными дополнительными соображениями, которые\linebreak могут 
потребовать, скажем, использовать только стратегии с~минимальным объемом 
сопутствующих вычислений и~ограниченными требованиями к~памяти.
    
    Конкретные задачи диспетчеризации содержат дополнительные предположения, 
которые фактически выделяют множества~$\Xi$ и~$\Sigma$, обычно не указывая эти 
множества явно. Перечислим основные условия, которые обычно накладываются на 
общую модель, с~тем чтобы получить более доступную для решения задачу.
    
    Последовательность моментов поступления заданий предполагается чаще всего 
порожденной единственным пуассоновским процессом. В~более общем варианте 
рассматривается рекуррентный поток.
    
    Задания, как правило, характеризуются функцией распределения~$F$, определяющей 
размер задания или время его выполнения. Относительно функции~$F$ делаются 
дополнительные предположения.
    
Серверы часто представляют собой готовые технические изделия со встроенной схемой управления 
очередью, которую нельзя легко изменить. Наиболее употребительными дисциплинами обслуживания 
очереди являются\footnote{Более широко класс наиболее часто рассматриваемых дисциплин можно 
определить так: рассматриваются консервативные (сохраняющие работу) дисциплины, которые не 
допускают прерывания уже выполняемого задания.}:
\begin{itemize}
\item FIFO (First In First Out);
    \item LIFO (Last In First Out) или LIFO-PR (Last In First Out Preemptive Resume);
    \item PS (Processor Sharing).
    \end{itemize}
    
    Производительность сервера характеризуется иногда распределением времени 
выполнения задания, а иногда скоростью $u\hm>0$. В~последнем случае время 
выполнения задания размером~$s$ равно $s/u$.
    
    Большая вариативность имеет место в~части предположений о возможностях 
наблюдения. Диапазон этих предположений простирается от полной наблюдаемости до 
фактически полной невозможности что-либо наблюдать, кроме моментов выбора серверов 
(или, что то же самое, кроме моментов поступления заданий) и~номеров выбора. Впрочем, 
условия на наблюдаемость часто вообще явно не формулируются, а~только отражаются 
в~том объеме информации, которую используют рассматри\-ва\-емые стратегии.
    
    Хотя сформулированную оптимизационную\linebreak задачу можно назвать основной 
в~проблематике диспетчеризации, ее постановка, разумеется, не ис\-черпывает всех 
вопросов, которые возникают в~рассматриваемой области. Существенное вни-\linebreak\vspace*{-12pt}

\columnbreak

\noindent
мание 
уделяется также, например, анализу стационарных характеристик системы, которые 
имеют место при той или иной стратегии размещения заданий.
    
    Надо также отметить, что изложенный взгляд на проблему диспетчеризации с~чисто 
математической точки зрения является до некоторой степени идеализированным, 
поскольку формальная задача, как правило, весьма сложна как для аналитического, так 
и~для численного решения. Этим объясняется малое число работ, в~которых оптимальная 
диспетчеризация была бы найдена как решение задачи оптимизации (по этому поводу см., 
например, краткий обзор в~\cite{2-kon}). Гораздо чаще используется <<инженерный>> 
подход, когда вначале на основании, как правило, интуитивно ясных эвристических 
соображений создаются алгоритмы размещения заданий, а~затем исследуются 
и~сравниваются их свойства, в~том числе оптимизационные. По этой причине обзор методов 
диспетчеризации начинается в~следующем разделе с~описания наиболее употребительных 
алгоритмов.

\vspace*{-9pt}

\section{Основные стратегии размещения заданий}

\vspace*{-2pt}

    Простейшая диспетчеризация Random состоит в~том, что каждый раз сервер 
выбирается равновероятно. Она не использует никакой информации о системе~--- ни 
априорной, ни текущей (кроме значения~$N$). Диспетчеризация Random является 
частным случаем более общей диспетчеризации PAP (Probabilistic Allocation Policy), 
которая параметризована дискретным распределением $p\hm= (p_1,\ldots, p_N)$. Задание 
направляется на сервер~$n$ с~вероятностью~$p_n$. Реализация стратегии не требует 
наблюдений. Однако для того, чтобы найти, хотя бы приближенно, наилучшее 
распределение~$p$, надо знать априорную информацию о~параметрах заданий и~серверов. 
Известен также подход, в~котором PAP оптимизируется в~процессе работы системы, но 
в~этом случае необходимы дополнительные наблюдения за траекторией системы. Обе 
стратегии Random и~PAP относятся к~классу, который в~литературе часто называется 
Bernoulli splitting и~характеризуется тем, что для принятия решения не нужно знания 
текущего состояния системы. Естественным обобщением стратегии Bernoulli splitting для 
систем с~несколькими входящими потоками является стратегия MCPAP (MultiClass 
Probabilistic Allocation Policy), когда задание класса~$i$ направляется на сервер~$j$ 
с~вероятностью~$p_{ij}$.
    
    Детерминированным аналогом Random является диспетчеризация Round-Robin, 
которая действует циклически, направляя $k$-е задание на сервер $k\,\mathrm{mod}\,N$. 
Для ее реализации необходимо помнить предыдущий выбор.
    
    Диспетчеризация JSQ (Join the Shortest Queue) направляет задание на сервер 
    с~минимальной очередью. Диспетчеризация JSQ($d$) действует аналогичным образом, но 
не на всем множестве серверов, а в~группе из случайно выбранных~$d$~серверов. Если же 
группа формируется так: сначала случайным образом формируется группа серверов 
размера $M\hm\leq N$, а затем из этой группы наугад выбираются~$d$~серверов, то 
диспетчеризация называется HJSQ($d$) (Hybrid JSQ($d$)).
    
    Диспетчеризация MEST (Minimum Expected Service Time) направляет задание в~конец 
очереди с~наименьшим ожидаемым временем начала обслуживания. Эта стратегия неявно 
предполагает, что размер задания становится известным только при его поступлении на 
обработку на сервер и~требует, чтобы была доступна следующая информация:
    \begin{itemize}
    \item размер очереди, т.\,е.\ число заданий, ожидающих выполнения;
    \item производительность серверов;
    \item распределение времени обслуживания задания;
    \item среднее остаточное время выполнения задания при условии, что оно уже 
выполняется в~течение некоторого времени.
    \end{itemize}
    
    Динамическая диспетчеризация MERL (Minimum Expected Residual Load), также 
известная, как LWL (Least Work Left), направляет задание на сервер с~наименьшим 
объемом невыполненной нагрузки, т.\,е.\ с~минимальной остаточной работой, включая 
заявки в~очереди. Стратегия неявно предполагает, что размер задания становится 
известным в~момент его поступления в~систему, и~в~остальном требует той же 
информации и~тех же наблюдений, что и~стратегия MEST.
    
    Диспетчеризация Myopic предполагает, что вновь поступающее задание поступает 
в~ту очередь, которая минимизирует время, необходимое для освобождения от работы 
всей системы целиком, в~предположении, что в~дальнейшем задания в~систему не 
поступают.
    
    Три предыдущие стратегии, по сути, используют одинаковую идею: посылать задание 
туда, где <<лучше>> с~сиюминутной точки зрения. Они представляют собой 
разновидности из известного семейства <<жадных>> алгоритмов.
    
    Диспетчеризация SITA-E (Size Interval Task Assignment with Equal Load) устроена так, 
что каждый сервер обслуживает задания, размер которых попадает в~назначенный для 
данного сервера диапазон. Выбор диапазонов для серверов осуществляется так, чтобы 
средняя нагрузка была одинакова. Более точно, пусть область определения функции~$F$ 
сосредоточена в~диапазоне от~$x_0$ до~$x_h$. Тогда разбиение $0\hm\leq x_0\hm< 
x_1<\cdots < x_h\hm\leq \infty$ выбирается так, чтобы выполнялись равенства: 
    $$
    \int\limits_{x_i-1}^{x_i} dF(x) =\fr{1}{h} \int_{x_0}^{x_h} x\,dF(x)\,.
    $$
    
    Диспетчеризация SITA-V (Size Interval Task Assignment with Variable Load) является 
модификацией стратегии SITA-E. Ее реализация на примере системы из двух серверов 
выглядит следующим образом. Пусть размер заданий принимает значения из отрезка 
$[k,p]$ и~пусть~$x$~--- некоторая точка из этого отрезка. Маршрутизация происходит 
таким образом, что задания, размер которых меньше~$x$, на\-прав\-ля\-ют\-ся на первый 
сервер, а остальные задания~--- на второй сервер. Таким образом, идея заключается в~том, 
чтобы рассредоточивать многочисленные короткие задания и~редкие длинные задания. 
Отметим, что при этом нарушается так называемый принцип балансировки нагрузки, 
который соблюдается в~той или иной степени в~стратегиях Random, Round-Robin, MERL 
и~SITA-E. Рандомизированная балансировка является очень простой и~одновременно 
эффективной с~точки зрения уменьшения коллизий, времени ожидания, очередей и~т.\,д.
    
    Если вместо размеров заданий рассматривается их <<значимость>> (т.\,е.\ некоторая 
числовая характеристика), то диспетчеризация, устроенная по принципу SITA, носит 
название VITA (Value Interval Task Assignment). Другими диспетчеризациями, 
использующими информацию о значимости заданий (value of job), являются C-MU 
и~LAVA  (Length and Value Aware). Обе диспетчеризации для принятия решения используют информацию только 
о~находящихся в~момент принятия решения заданиях в~сис\-те\-ме. Согласно 
диспетчеризации C-MU задание направляется на сервер с~максимальным значением 
произведения $c_i\mu_i$, где $c_i$~--- <<стоимость хранения>> (holding cost), 
а~$\mu_i$~--- интенсивность обслуживания на сервере~$i$. Точнее, задание 
с~характеристикой значимости~$\nu$ направляется на сервер 
$\mathop{\mathrm{argmin}}\limits_i\,(\overline{\nu}_i\hm-\nu)\mu_i/(n_i+1)$, где 
$\overline{\nu}_i\hm= \sum_{j=1}^{n_i} \nu_{i,j}/n_i$, $\nu_{i,j}$~--- значимость 
выполнения задачи~$j$ на сервере~$i$, $n_i$~--- число заявок на сервере~$i$ в~момент 
принятия решения. Если же задание с~характеристикой~$\nu$ направляется на сервер 
$\mathop{\mathrm{argmin}}\limits_i \left((n_i\hm+2)\nu\hm+\sum_{j=1}^{n_i}
\nu_{i,j}\right)/(2\mu_i)$, то это диспетчеризация LAVA.
    
    Диспетчеризация TDP (Threshold Dispatching Policy) предполагает использование 
<<порогового>> правила для принятия решения. Согласно TDP в~момент поступления 
задания оно направляется на сервер с~номером 
$\mathop{\mathrm{argmin}}\limits_i(u_i+\xi_i)$, где $u_i$~--- время (начиная с~момента 
поступления), необходимое для завершения всей имеющейся работы на сервере~$i$, 
а~$\xi_i$~--- некоторые константы (пороги). От значения порогов~$\xi_i$ и~зависит 
значение целевой функции. Отметим, что применение данного класса пороговых 
диспетчеризаций можно часто наблюдать на практике (например, в~оборудовании сетей 
связи). Однако доказать его оптимальность удается лишь в~некоторых весьма частных 
случаях.
    
    Диспетчеризация FPI (First Policy Iteration), в~отличие от всех изложенных выше, 
основана не на интуитивных соображениях, а на методах марковского процесса принятия 
решений. Для ее построения используется первый шаг итеративного\linebreak алгоритма 
улучшения марковской стратегии. За\linebreak начальное приближение берется одна из 
диспетчеризаций, например Random, или VITA, или\linebreak любая~дру\-гая, при которой входные 
потоки на серверы остаются пуассоновскими. Последнее об\-стоя\-тельство позволяет 
выписать явные формулы, необходимые для одного итеративного шага (по\-сле\-ду\-ющие 
шаги технически невыполнимы). Согласно FPI поступившая заявка направляется в~узел
    $\mathop{\mathrm{argmin}}\limits_i  
    (\gamma\lambda_i {\sf 
M}\nu_i(n_i+1)/(\mu_i-\lambda_i)\hm+\sum_{j=1}^{n_i} \nu_{i,j}\hm+ 
(n_i\hm+2)\nu)/(2\mu_i\hm-\lambda_i)$,
где $\gamma$~--- коэффициент дисконтирования.

    Следующие четыре стратегии допускают переход между очередями для заданий 
и~прерывание обслуживания.
    
    Диспетчеризация TAGS (Task Assignment by Guessing Size) каждому серверу 
назначает допустимый временной интервал ожидания обслуживания, т.\,е.\ интервал, 
в~течение которого задание может ожидать обработки. Длины интервалов различны для 
разных серверов. Если задание не выполнено в~течение отведенного времени, то его 
выполнение прерывается и~задание отправляется в~очередь другого сервера, на котором 
будет выполняться заново. Подробнее, пусть $\Delta_n$~--- время, которое задание может 
провести на сервере~$n$, включая ожидание в~очереди и~непосредственное выполнение, 
и~пусть $\Delta_1\hm< \Delta_2<\cdots < \Delta_N$. Все поступающие в~сис\-те\-му задания 
сначала отправляются на сервер~$1$. Если задание не выполнено за время~$\Delta_1$, то 
оно на\-прав\-ля\-ет\-ся на сервер~$2$. Там оно может пробыть время, не 
превышающее~$\Delta_2$, и~в~случае невыполнения отправляется на сервер~$3$ и~т.\,д.
    
    Диспетчеризация TAPTF (Task Assignment based on Prioritizing Traffic Flows) является 
модификацией стратегии TAGS. Для этой стратегии аналогичным образом определяются 
числа~$\Delta_n$. Но у каждого сервера имеется не одна, а две очереди: основная 
и~вспомогательная (так называемая очередь повторных заданий). Когда новое задание поступает 
в~сис\-те\-му, оно немедленно направляется в~основную очередь одного из серверов 
в~соответствии с~некоторым статическим правилом. Если сервер не успел выполнить 
задание за отведенный интервал времени, задание перемещается во вспомогательную 
очередь того сервера, у которого допустимый интервал ожидания выше. При этом задания 
в~основной очереди имеют более высокий приоритет обслуживания, чем во 
вспомогательной.
    
    Диспетчеризация TAPTF-WC (Task Assignment based on Prioritizing Traffic Flows with 
Work-Conserving migration) отличается от предыдущей тем, что задания из очереди 
повторных заданий выполняются с~того места, на котором было прервано их 
обслуживание на предыдущем сервере.
    
    Диспетчеризация MTTMEL (MultiTier Task assignment policy with Minimum Excess 
Load) является обобщением предыдущих трех стратегий. В~этом алгоритме все серверы 
разбиваются на группы различного размера. Допустимый временной интервал ожидания 
обслуживания теперь назначается не каж\-до\-му серверу в~отдельности, а~сразу на всю 
группу. Новое задание направляется на один из серверов группы наименьшего размера. 
Если там оно не выполнено за отведенное время, то оно переходит на один из серверов 
следующей по размеру группы и~т.\,д.

\section{Модели с~одинаковыми серверами}

    Наиболее изучены в~теоретическом отношении задачи, в~которых все серверы 
идентичны по своим параметрам. В~этом разделе представлены несколько моделей 
с~одинаковыми серверами, которые отличаются дисциплинами обслуживания очередей, 
функцией распределения времени выполнения заданий~$F$ и~--- главное отличие~--- 
возможностями наблюдения. Каноническим методом для эффективного распределения 
заданий между ресурсами в~случае одинаковых серверов является балансировка нагрузки. 
Согласно этому принципу действуют стратегии Random, Round-Robin, JSQ, JSQ($d$), 
MERL и~SITA-E.

\subsection{Наблюдаемые очереди}
    
    4.1.1 Случай, когда неизвестен размер заданий, приводит к~системам, носящим иногда 
название <<модели супермаркета>>. Эти модели изучались в~[3--10] при различных 
дополнительных условиях. Существующая методология исследования заключается 
в~следующем:
    \begin{itemize}
    \item  система описывается с~помощью марковского процесса, для которого 
устанавливается наличие предельного распределения;
    \item система описывается с~помощью дифференциальных уравнений, для которых 
исследуют вопросы существования и~единственности решения, дающего стационарное 
распределение очередей.
    \end{itemize}
    
    Такой подход становится затруднителен, когда распределения времен обслуживания 
не экспоненциальные, поскольку приходится усложнять пространство состояний 
марковской цепи. В~\cite{10-kon} предлагается способ выхода из затруднения. В~этой 
работе дополнительно предполагается, что
    \begin{itemize}
    \item интенсивность пуассоновского входящего потока прямо пропорциональна числу 
серверов;
    \item дисциплина обслуживания очереди одинаковая для всех серверов;
    \item функция распределения времени выполнения задания произвольная;
    \item наблюдаемы размеры очередей.
    \end{itemize}
    
    Основой подхода является диспетчеризация JSQ($d$) и~гипотеза об асимптотической 
независи\-мости размеров очередей при больших значениях~$N$. Для системы 
с~дисциплиной обслуживания очереди LIFO-PR и~функцией распределения времени 
обслуживания общего вида с~помощью предложенного подхода доказано, что при 
$n\hm\to\infty$ вероятность~$P_k$ того, что в~очереди к~любому серверу будет не 
менее~$k$ заданий, удовлетворяет асимптотическому соотношению $P_k\sim \lambda (d^k\hm-
1)/(d\hm-1)$. Для системы с~дисциплиной обслуживания очере-\linebreak ди \mbox{FIFO} зависимость~$P_k$ 
от~$d$ имеет пороговый харак\-тер. При различных значениях~$d$ характер зависимости 
может быть полиномиальным, экспоненциальным и~дважды экспоненциальным. При 
неограниченном рос\-те числа серверов совместное распределение числа заявок в~очередях 
(если используются дисциплины обслуживания PS или LIFO-PR) являются 
инвариантными относительно распределения времени обслуживания. В~вычислительном 
отношении использованный в~\cite{10-kon} подход позволил уменьшить на два порядка 
время для имитационного моделирования.
    
    \smallskip
    
    4.1.2\ Известны характеристики времени выполнения заданий, в~частности доступны 
оценки оста-\linebreak\vspace*{-12pt}

\columnbreak

\noindent
точных времен выполнения заданий~[11, 12]. Дополнительные 
предположения:
    \begin{itemize}
    \item  входной поток~--- пуассоновский;
    \item дисциплина обслуживания очереди~--- FIFO;
    \item интенсивность обслуживания $h(t)\hm= f(t)/(1\hm-F(t))$ является неубывающей 
функцией.
    \end{itemize}
    
    Анализ диспетчеризации JSQ для системы из параллельных серверов проведен 
в~\cite{13-kon, 14-kon} в~предположении пуассоновского входного потока и~
экспоненциального распределения времени обслуживания. В~этих предположениях ее 
оптимальность доказана в~\cite{11-kon} для серверов с~дисциплиной FIFO.  
В~\cite{16-kon} JSQ анализируется для обслуживания с~дисциплиной PS, 
где показано (с~по\-мощью различного рода приближений 
и~численных экспериментов), что в~системах серверов с~произвольным распределением 
времени обслуживания схема JSQ хотя и~не является оптимальной в~смысле минимума 
задержки, но имеет хорошую производительность в~сравнении с~другими, более 
сложными стратегиями. Оптимальность JSQ для некоторых более сложных систем 
доказана в~\cite{17-kon}.
    
    В~\cite{12-kon} изучается диспетчеризация MEST, которая оказывается оптимальной 
в смысле стохастического упорядочивания. Это означает, что она максимизи\-рует 
вероятность того, что для произвольных~$k$ и~$t$ не менее чем~$k$~заявок будут 
обслужены за время~$t$.
    
    \smallskip
    
    4.1.3\ Модели с~известным размером заданий (или с~известным временем выполнения 
заданий) ценны тем, что в~них делаются реалистичные предположения о~функции 
распределения размера задания. Данное предположение имеет практическую 
предпосылку. В~некоторых случаях (например, в~сетевых запросах на скачивание файлов) 
удается оценить распределение размеров (времен выполнения) поступающих 
заданий~\cite{19-kon}. Анализ статистики в~реальных сис\-те\-мах показал, что 
распределение длины задач в~современных информационно-те\-ле\-ком\-му\-ни\-ка\-ци\-он\-ных 
системах хорошо описывается степенной функцией~$x^{-\alpha}$, где $0{,}9\hm\leq 
\alpha \hm \leq 2$~[17--21].
    
    Стратегия <<наименьшей очереди>> (JSQ), применяемая при неизвестных размерах 
задач, не всегда является оптимальной~\cite{23-kon}, как, например, в~случае 
распределения времени обслуживания с~тяжелым хвостом и~с большой дисперсией. 
В~\cite{19-kon, 24-kon} проведен подробный сравнительный анализ нескольких 
диспетчеризаций для таких функций~$F$ при следующих дополнительных 
предположениях:
    \begin{itemize}
    \item  входной поток~--- пуассоновский;
    \item  дисциплина обслуживания очереди~--- FIFO;
    \item  распределение времени обслуживания~$F$ имеет плотность
    $$
    f(x) =\fr{\alpha k^u}{1-(k/p)^\alpha}\,x^{-\alpha-1}\,,\ k\leq x\leq p\,;
    $$
    \item  размер задания известен в~момент поступления.
    \end{itemize}
    
    Рассмотрены следующие диспетчеризации: Random, Round-Robin, Dynamic, SITA-E 
и~SITA-V. Выводы относительно производительности стра\-тегий сделаны на основании 
методов теории мас\-сового обслуживания и~результатов численных эксперимен\-тов по 
критериям среднего времени выполнения задач и~среднего значения для отношения 
времени выполнения заданий к~их размеру. По\-следний критерий считается важным 
в~рас\-смат\-ри\-ва\-емых моделях и~носит специальное название: замедление (slowdown). 
Лейтмотивом всего исследования служит то очевидное соображение, что разные критерии 
(разные целевые функции) могут требовать разных стратегий. Установлены, в~частности, 
следующие факты.
    
    Первые две стратегии обладают свойством выравнивать нагрузку. Однако если 
тестировать стратегии для различных времен между поступлениями, то результаты для 
циклической стратегии обладают меньшим разбросом, чем результаты для стратегии 
Random.
    
    Привязка сервера к~определенному диапазону размера задач позволяет избежать 
дискриминации коротких задач. 
    %
    Динамическая стратегия <<оптимальна>> с~точки зрения отдельной задачи, а~также 
с~точки зрения мгновенного выравнивания нагрузки во всей сис\-теме.
    %
    Если дисперсия размера задач не очень высока, то предпочтительнее динамическая 
стратегия, в~противном случае лучшей является стратегия SITA-E.
    
    Различие между четырьмя стратегиями может быть велико. Стратегии Random 
и~Round-Robin хуже, чем SITA-E и~Dynamic на несколько порядков. SITA-E превосходит 
Dynamic почти в~два раза.
    
    Преимущество SITA-E объясняют следующим образом:
    \begin{itemize}
    \item уменьшается дисперсия поступающих на отдельный сервер задач;
    \item если нагрузка сбалансирована, то большинство задач направляется на сервер 
с~<<лучшими условиями>>;
    \item улучшается среднее значение замедления.
    \end{itemize}
    
    Стратегия SITA-E улучшает замедление, однако при этом увеличивает среднее время 
обслуживания. Вычислительные эксперименты показали целесообразность применения 
алгоритма SITA-V на веб-фер\-мах.
    
    Диспетчеризации SITA-E и~SITA-V являются представителями целого класса стратегий 
SITA, которые регулируют размер заданий, поступающих на отдельные серверы (более 
подробно см.~\cite{25-kon, 26-kon}).

\vspace*{-1pt}
   
   \subsection{Ненаблюдаемые очереди}
   
   \vspace*{-1pt}
   
    При неизвестном заранее времени обслуживания диспетчеризация может быть либо 
самой простой (вроде Random), либо может использовать каким-то достаточно сложным 
образом наблюдения за реальным временем выполнения заданий. В~[26--29] 
следуют вторым путем. В~этих работах в~предположении, что все очереди 
управляются согласно дисциплине FIFO, последовательно предложены четыре 
нетривиальных диспетчеризации (все на основе эвристических соображений): TAGS, 
TAPTF, TAPTF-WC и~MTTMEL. Во всех четырех случаях необходимо задать 
значения~$\Delta_n$ времени, которое может провести задание на сервере~$n$, включая 
ожидание в~очереди и~непосредственное выполнение (описание диспетчеризаций~--- 
см.\ разд.~3). Эти значения, выбранные изначально наугад или по экспертной оценке, 
актуализируются путем наблюдения как за системой, так и~за поведением выбранной 
целевой функции. Оптимизация заключается в~выборе значений~$\Delta_n$, при которых 
целевая функция достигает значения, близкого к~оптимальному.
    
    Упомянутые диспетчеризации снижают дис\-пер\-сию времени выполнения заданий. На 
практике это означает, что на каждом сервере скапливаются примерно одинаковые по 
времени выполнения задания. Хорошие результаты обнаруживаются для функций~$F$, 
имеющих тяжелые хвосты (при этом существенно, что времена обслуживания 
неизвестны). Однако имеются и~серьезные с~практической точки зрения недостатки. 
Реализация таких стратегий связана с~<<накладными расходами>>: необходимы механизмы, 
обеспечивающие аккуратное прерывание заданий, перенаправление на другие серверы 
и~возобновление обслуживания. Кроме того, возможны плохие результаты при высокой 
загрузке системы. Например, TAGS вносит несправедливость в~распределение заданий, 
так как длинные задания чаще прерываются.

\vspace*{-2pt}

\section{Модели с~неодинаковыми серверами}

%\vspace*{-1pt}

\subsection{Наблюдаемые очереди}

%\vspace*{-1pt}

В~\cite{31-kon} исследуется система, в~которой каждое задание характеризуется парой независимых в~
совокупности случайных величин $(X,V)$, где $X$~--- размер (время обслуживания), имеющий 
экспоненциальное распределение, $V$~--- значимость задачи (некоторая числовая характеристика), 
распределение которой задано заранее. Значение~$V$ наблюдается в~момент поступления задания 
в~систему. Напротив, значения~$X$ не наблюдаются\footnote{На самом деле, даже если они наблюдаются, то 
в силу предположения об экспоненциальности размеров заданий не несут никакой информации из-за 
свойства отсутствия последействия.}. В~работе с~помощью аналитических методов теории массового 
обслуживания, численных методов и~имитационного моделирования проведен сравнительный анализ 
диспетчеризаций, оптимизирующих среднее значение произведения~$VT$, где $T$~--- время пребывания 
задания в~системе. В~экспериментах диспетчеризации Random и~VITA уступали JSQ, в~то время как C-MU 
и~LAVA превосходили ее. Чаще всего наилучшие результаты показывала LAVA. Однако применение более 
сложной диспетчеризации FPI позволило улучшить LAVA, особенно заметно в~случае большой загрузки. 
Аналитический раздел~\cite{31-kon} содержит ряд утверждений, касающихся режимов устойчивости 
диспетчеризаций, а также сравнения LAVA с~JSQ в~частном случае двух одинаковых серверов.
    
    В~\cite{32-kon} рассматривается система, в~которую могут поступать несколько 
пуассоновских потоков задач интенсивности~$\lambda_i$, $i\hm= 1,\ldots, M$. Поступив 
на сервер, они обслуживаются одинаково, вне за\-ви\-си\-мости от типа потока. Каждый 
сервер~$j$ описывается функцией $D_j\hm= D_j(\gamma_j)$, которая характеризует 
<<затраты>> на обслуживание на нем задания, включая ожидание в~очереди, 
в~зависимости от нагрузки~$\gamma_j$ на этот сервер. С каждым классом заданий~$i$ 
связано число~$\beta_i$, которое является количественной (экспертной) оценкой того, 
насколько критично качество обслуживания для заданий класса~$i$. В~работе показано 
с~помощью методов математического программирования, что диспетчеризация MCPAP 
доставляет оптимальное значение целевой функции: 
$$
W(P)  = \sum\limits_{i=1}^M 
\sum\limits_{j=1}^N \beta_i \lambda_i p_{ij} D_j(\gamma_j)\,.
$$
 В~связи с~интересом 
к~децентрализованному управ\-ле\-нию наряду с~минимизацией целевой функции 
рас\-смат\-ри\-ва\-ет\-ся и~игровая постановка. Целью обслуживания заданий из потока~$i$ 
объявляется выбор такого сервера~$j$, который минимизировал бы собственные затраты 
$V_{ij} \hm= c_j\hm+ \beta_i D_j$ ($c_j\hm>0$~--- заданные коэффициенты). Вводится 
понятие равновесной стратегии, обладающей свойством $\forall\,i,j,k:\ p_{ij}\hm> 
0\hm\Rightarrow V_{ij}\hm\leq V_{ik}$. Если задания размещаются в~соответствии 
с~равновесной стратегией, то сервер, на который направляются задания с~положительной 
вероятностью, должен обходиться не дороже, чем любой другой сервер. Равновесная 
стратегия всегда существует как частный случай равновесия по Нэшу.
    
    Похожая постановка задачи рассмотрена в~\cite{33-kon}. Основной результат работы 
тот же: оптимальной с~точки зрения минимизации функции типа $W(P)$ является 
диспетчеризация MCPAP. Кроме того, эксперименты показали, что при росте нагрузки 
имеет значение дисциплина обслуживания на серверах. Так, дисциплина, в~соответствии 
с~которой заявки на сервере переупорядочиваются некоторым определенным образом, 
оказалась лучше, чем FIFO.
    
    Исследование систем, в~которых имеется возможность заранее задать 
производительность $c_i$ каж\-до\-го сервера (обслуживающего задания по дисциплине PS), 
проведено в~\cite{1-kon}. Задание, направленное на сервер~$i$, обслуживается на нем со 
скоростью $c_i/x_i(t)$, где $x_i(t)$~--- общее число задач на этом сервере в~момент~$t$. 
Используя в~качестве целевой функции среднее время пребывания задания в~сис\-те\-ме, 
авторы показали, что нахождение оптимальной диспетчеризации сводится к~задаче 
математического программирования. В~качестве наилучших вариантов предлагается 
использовать MCPAP, JSQ и~HSJQ.
    
    В~\cite{34-kon} рассмотрена модель, в~которой точные размеры (будущие времена 
обслуживания) заданий предполагаются известными в~моменты их поступления 
в~систему. Целевая функция~--- среднее время пребывания в~системе. Для случая двух 
серверов доказано, что оптимальная диспетчеризация найдется среди стратегий TDP. 
Заметим, что в~случае одинаковых серверов оптимальной является стратегия LWL вне 
зависимости от числа серверов~\cite{35-kon, 36-kon}. Для нахождения порога предложен 
алгоритм типа FPI. Для большего числа серверов вопрос оп\-ти\-маль\-ности TDP не 
исследован, но эксперименты показывают, что TDP может давать результаты, близкие 
к~оптимальным. Однако вопрос нахождения точных пороговых значений остается 
открытым~\cite{37-kon}.

\vspace*{-6pt}

\subsection{Ненаблюдаемые очереди}

\vspace*{-2pt}

    В~\cite{15-kon} рассмотрена модель с~известными скоростями серверов и~размерами 
поступающих заданий. При этом нет возможности наблюдать помимо очередей еще 
и~остаточные (прошедшие) времена обслуживания заданий на серверах. Дисциплина 
обслуживания на каждом сервере~--- \mbox{FIFO}. Опи-\linebreak\vspace*{-12pt}

\pagebreak

\noindent
сан класс легко реализуемых статических 
диспетчеризаций, позволяющих достигать близких к~оптимальным значений времени 
ожидания начала обслуживания (и~времени пребывания в~сис\-те\-ме). В~работе 
сравниваются диспетчеризации Random, SITA и~NSI (Nested Size-Interval). Последний 
вариант диспетчеризации, NSI, отличается от SITA лишь тем, что интервалы, которые 
определяют, задания какого размера (времени выполнения) поступают на каждый сервер, 
могут как перекрываться, так и~быть вложенными друг в~друга. Диспетчеризация SITA, 
являясь оптимальной среди всех статистических стратегий в~случае одинаковых серверов, 
уже не является таковой в~случае различных серверов. В~\cite{15-kon} доказано, что 
в~данном случае оптимальная статистическая стратегия находится в~классе 
диспетчеризаций NSI, и~предложен метод ее на\-хож\-дения, использующий аппарат 
теории массового обслуживания (результаты для системы массового обслуживания 
типа $M$/$G$/1).

\section{Заключение}

    Проведенный обзор свидетельствует о наличии заметного раздела современной науки,  
который сформировался в~связи с~проблемами развития информационных технологий. 
Многообразие реальных приложений и~ситуаций порождает большое число моделей 
в~области оптимального размещения заданий в~системах с~параллельным обслуживанием. 
Основные успехи в~создании алгоритмов диспетчеризации на данном этапе связаны 
с~реализацией эвристических идей. Эти достижения вполне ощутимы, и~в~качестве 
примера действующих реализаций механизмов для размещения заданий можно привести 
такие системы, как Cisco's Local Director, IBM's Network Dispatcher, F5's Big 
    IP~[38--42]. Однако общий подход, который 
связал бы отдельные многочисленные постановки задач в~единую теорию, пока 
отсутствует. Среди аналитических методов, использованных в~рассмотренных моделях, 
выделяются методы теории массового обслуживания и~теории марковского процесса 
принятия решений. Применяются также методы оптимизации (линейное, нелинейное 
и~целочисленное программирование), униформизация (переход от дискретного времени 
к~непрерывному). В~некоторых постановках, не отмеченных явным образом в~статье, 
оказываются полезными методы теории чисел (цепные дроби, последовательности Фарея 
и~др.). Можно, однако, наблюдать, как перечень применимых аналитических методов 
анализа стремительно сокращается по мере усложнения рассматриваемой системы 
и~остается единственный подход~--- метод статистических испытаний. Со 
статистическим моделированием связаны исследования по созданию 
самонастраивающихся алгоритмов, корректирующих диспетчеризацию по наблюдениям 
за траекторией поведения системы. Это направление представляется перспективным 
и~требует отдельного дополнительного освещения.

{\small\frenchspacing
 {%\baselineskip=10.8pt
 \addcontentsline{toc}{section}{References}
 \begin{thebibliography}{99}
    \bibitem{1-kon}
    \Au{Mukhopadhyay A., Mazumdar R.\,R.} Analysis of load balancing in large 
heterogeneous processor sharing systems. {\sf http://arxiv.org/abs/1311.5806}.
    \bibitem{2-kon}
    \Au{Hyyti$\ddot{\mbox{a}}$~E.} Optimal routing of fixed size jobs to two parallel 
servers~// INFOR: Inform. Syst. Oper. Res., 2013. Vol.~51. No.\,4. 
P.~215--224.
\bibitem{7-kon} %3
    \Au{Mitzenmacher M.} The power of two choices in randomized load balancing.~--- Berkeley, 1996.
    Ph.D.\  Thesis.
    \bibitem{8-kon} %4
    \Au{Vvedenskaya N.\,D., Dobrushin~R.\,L., Karpelevich~F.\,I.} Queueing system with 
selection of the shortest of two queues: An asymptotic approach~// Probl. Inf. Transm., 1996. 
Vol.~32. No.\,1. P.~20--34.
\bibitem{6-kon} %5
    \Au{Martin J.\,B., Suhov Y.\,M.} Fast jackson networks~// Ann. Appl. Probab., 1999. Vol.~9. 
No.\,4. P.~840--854.
    \bibitem{3-kon} %6
    \Au{Graham C.} Chaoticity on path space for a queueing network with selection of the 
shortest queue among several~// J.~Appl. Probab., 2000. Vol.~37. P.~198--211.
 \bibitem{9-kon} %7
    \Au{Mitzenmacher M., Richa~A., Sitaraman~R.\,K.} The power of two random choices: 
A~survey of techniques and results~// Handbook of randomized computing~/ Eds. 
S.~Rajasekaran, P.\,M.~Pardalos, J.\,H.~Reif, J.~Rolim.~--- Norwell, MA, USA: Kluwer Academic 
Publs., 2001.  Vol.~1. P.~255--312.
    \bibitem{4-kon} %8
    \Au{Luczak M., McDiarmid C.} On the power of two choices: Balls and bins in continuous 
time~// Ann. Appl. Probab., 2005. Vol.~15. No.\,3. P.~1733--1764.
    \bibitem{5-kon} %9
    \Au{Luczak M., McDiarmid C.} On the maximum queue length in the supermarket model~// 
Ann. Probab., 2006. Vol.~34. No.\,2. P.~493--527.
        \bibitem{10-kon} %10
    \Au{Bramson M., Lu~Y., Prabhakar~B.} Randomized load balancing with general service 
time distributions~// ACM Special Interest Group on Computer Systems Performance, 
SIGMETRICS Proceedings, 2010. Vol.~38. Iss.~1. P.~275--286.
    \bibitem{11-kon}
    \Au{Winston W.} Optimality of the shortest line discipline~// J.~Appl. Probab., 1977. Vol.~14. 
P.~181--189.
    \bibitem{12-kon}
    \Au{Weber R.} On the optimal assignment of customers to parallel servers~// J.~Appl. 
Probab., 1978. Vol.~15. P.~406--413.
    \bibitem{14-kon} %13
    \Au{Haight A.} Two queues in parallel~// Biometrika, 1958. Vol.~45. P.~401--410.
        \bibitem{13-kon} %14
    \Au{Kingman J.\,F.\,C.} Two similar queues in parallel~// Biometrika, 1961. Vol.~48. 
P.~1316--1323.


    \bibitem{16-kon} %15
    \Au{Gupta V., Harchol-Balter~M., Sigman~K., Whitt~W.} Analysis of 
    join-the-shortest-queue routing for Web Server Farms~// Perform. Evaluation, 2007. 
Vol.~64. P.~1062--1081.
    \bibitem{17-kon} %16
    \Au{Akgun O., Righter~R., Wolff~R.} Multiple server system with flexible arrivals~// 
Adv. Appl. Probab., 2011. Vol.~43. P.~985--1004.
 \bibitem{19-kon} %17
    \Au{Harchol-Balter~M., Crovella~M., Murta~C.} On choosing a~task assignment policy 
for a~distributed server system~// J.~Parallel Distr. Comp., 1999. Vol.~59. Iss.~2. 
P.~204--228.
 \bibitem{21-kon} %18
    \Au{Paxson V., Floyd~S.} Wide area traffic: The failure of Poisson modeling~// IEEE/ACM 
Trans. Netw., 1995. Vol.~3. No.\,3. P.~226--244.
    \bibitem{22-kon} %19
    \Au{Peterson D., Adams~D.} Fractal patterns in DASD I/O traffic~// 22nd Computer 
Measurement Group Conference (International) Proceedings.~--- San Diego, CA, USA, 1996. 
P.~560--571.
    \bibitem{18-kon} %20
    \Au{Crovella M., Taqqu~M., Bestavros~A.} Heavy-tailed probability distributions in the 
World Wide Web~// A~practical guide to heavy tails~/ Eds. R.\,J.~Adler, R.\,E.~Feldman, and 
M.\,S.~Taqqu.~--- Cambridge, MA, USA: Birkhauser Boston Inc., 1998. P.~3--25.
   
    \bibitem{20-kon} %21
    \Au{Feitelson D.} Workload modeling for computer systems performance evaluation.~--- 
Cambridge, MA, USA: Cambridge University Press, 2015. 597~p.
   
    \bibitem{23-kon} %22
    \Au{Whitt W.} Deciding which queue to join: Some counterexamples~// Oper. Res., 
1986. Vol.~34. No.\,1. P.~55--62.
    \bibitem{24-kon} %23
    \Au{Crovella M., Harchol-Balter~M., Murta~C.} Task assignment in a~distributed system: 
Improving performance by unbalancing load.~--- Boston: Boston University, 1997.
Boston University Computer Science 
Department Technical Reports. BUCS-TR-1997-018.
    \bibitem{25-kon} %24
    \Au{Crovella M., Harchol-Balter~M., Murta~C.} Task assignment in a distributed system: 
Improving performance by unbalancing load~// ACM Sigmetrics '98 Conference on 
Measurement and Modeling of Computer Systems Poster Session Proceedings.~--- Madison, WI, USA, 
1998. P.~268--269.
    \bibitem{26-kon} %25
    \Au{Harchol-Balter M., Crovella~M., Murta~C.} On choosing a~task assignment policy for 
a~distributed server system~// J.~Parallel Distr. Comp., 1999. Vol.~59. P.~204--228.
    \bibitem{27-kon} %26
    \Au{Harchol-Balter M.} Task assignment with unknown duration~// J.~ACM, 2002. 
Vol.~49. P.~260--288.
    \bibitem{28-kon} %27
    \Au{Broberg J., Tari~Z., Zeephongsekul~P.} Task assignment based on prioritising traffic 
flows~// Principles of distributed systems~/
Ed. T.~Higashino.~--- Lectire notes in
computer science ser.~--- Grenoble, France:  Springer, 2004. Vol.~3544. P.~415--430.
    \bibitem{29-kon} %28
    \Au{Broberg J., Tari~Z., Zeephongsekul~P.} Task assignment with work-conserving 
migration~// J.~Parallel Computing, 2006. Vol.~32. P.~808--830.
    \bibitem{30-kon} %29
    \Au{Jayasinghe M., Tari~Z., Zeephongsekult~P.} A~scalable multi-tier task assignment 
policy with minimum excess load~//  IEEE Symposium on Computers and Communications 
Proceedings.~--- Riccione, Italy: IEEE, 2010. P.~913--918.
    \bibitem{31-kon} %30
    \Au{Doroudi S., Hyyti$\ddot{\mbox{a}}$~E., Harchol-Balter~M.} Value driven load 
balancing~// Perform. Evaluation, 2014. Vol.~79. P.~306--327.
    \bibitem{32-kon} %31
    \Au{Bodas T., Ganesh~A., Manjunath~D.} Tolls and welfare optimization for multiclass 
traffic in multiqueue systems. {\sf http://arxiv.org/abs/1409.7195}.
    \bibitem{33-kon} %32
    \Au{Becker K., Gaver~D., Glazebrook~K., Jacobs~P., Lawphongpanich~S.} Allocation of 
tasks to specialized processors: A~planning approach~// Eur. J.~Oper. Res., 
2000. Vol.~126. P.~80--88.
    \bibitem{34-kon} %33
    \Au{Hyyti$\ddot{\mbox{a}}$E.} Optimal routing of fixed size jobs to two parallel servers~// 
INFOR: Inform. Syst. Oper. Res., 2013. Vol.~51. No.\,4. P.~215--224.
    \bibitem{35-kon} %34
    \Au{Harchol-Balter M., Crovella~M., Murta~C.} On choosing a~task assignment policy for 
a~distributed server system~// J.~Parallel Distr. Comp., 1999. Vol.~59. P. 204--228.
    \bibitem{36-kon} %35
    \Au{Hyyti$\ddot{\mbox{a}}$E., Penttinen~A., Aalto~S., Virtamo~J.} Dispatching problem 
with fixed size jobs and processor sharing discipline~// 23rd Teletraffic Congress (International) 
(ITC'23).~--- San Fransisco, CA, USA, 2011. P.~190--197.
    \bibitem{37-kon} %36
    \Au{Konovalov M., Razumchik~R.} Iterative algorithm for threshold calculation in the 
problem of routing fixed size jobs to two parallel servers~// J.~Telecommunications Inform. 
Technol., 2015. Vol.~3. P.~32--38.
 \bibitem{15-kon} %37
    \Au{Feng H., Misra~V., Rubenstein~D.} Optimal state-free, size-aware dispatching for 
heterogeneous $M$/$G$/-type systems~// Performance Evaluation, 2005. Vol.~62. P.~475--492.
    \bibitem{38-kon} %38
    Cisco LocalDirector 400 series. {\sf 
    http://www.cisco.com/ c/en/us/products/routers/localdirector-400-series}.
    \bibitem{39-kon}
    \Au{Pistoia M., Letilley~C.} IBM web sphere performance pack: Load balancing with IBM 
secure way network dispatcher~// IBM Redbooks, 1999.
    \bibitem{40-kon}
    F5 Products. Big-IP. {\sf http://www.f5.com/products/big-ip}.
    
    \bibitem{42-kon} %41
    \Au{Schurman E., Brutlag~J.} The user and business impact on server delays, additional 
bytes and http chunking in web search~// O'Reilly Velocity Web Performance and 
Operations Conference, 2009. {\sf http://velocityconf. com/velocity2009/public/schedule/detail/8523}.
\bibitem{41-kon} %42
    Microsoft sharepoint 2010 load balancer. 
    {\sf http://loadba lancer.org/applications/microsoft-apps/microsoft-share point}. 
 \end{thebibliography}

 }
 }

\end{multicols}

\vspace*{-3pt}

\hfill{\small\textit{Поступила в~редакцию 19.10.15}}

%\vspace*{8pt}

\newpage

%\vspace*{-24pt}

%\hrule

%\vspace*{2pt}

%\hrule

\vspace*{-24pt}

\def\tit{METHODS AND ALGORITHMS FOR JOB SCHEDULING IN SYSTEMS 
WITH PARALLEL SERVICE: A~SURVEY}

\def\titkol{Methods and algorithms for job scheduling in systems 
with parallel service: A~survey}

\def\aut{M.\,G.~Konovalov$^1$ and R.\,V.~Razumchik$^{1,2}$}

\def\autkol{M.\,G.~Konovalov and R.\,V.~Razumchik}

\titel{\tit}{\aut}{\autkol}{\titkol}

\vspace*{-9pt}


\noindent
   $^1$Institute of Informatics Problems, 
    Federal Research Center ``Computer Science and Control'' of the Russian\linebreak
    $\hphantom{^1}$Academy 
of Sciences, 44-2 Vavilov Str., Moscow 119333, Russian Federation

\noindent
    $^2$Peoples' Friendship University of Russia, 6 Miklukho-Maklaya Str., Moscow 117198, Russian 
Federation


\def\leftfootline{\small{\textbf{\thepage}
\hfill INFORMATIKA I EE PRIMENENIYA~--- INFORMATICS AND
APPLICATIONS\ \ \ 2015\ \ \ volume~9\ \ \ issue\ 4}
}%
 \def\rightfootline{\small{INFORMATIKA I EE PRIMENENIYA~---
INFORMATICS AND APPLICATIONS\ \ \ 2015\ \ \ volume~9\ \ \ issue\ 4
\hfill \textbf{\thepage}}}

\vspace*{3pt}


\Abste{The review of research papers devoted to the analysis of the dispatching problem in queueing systems is 
presented. The analysis is restricted to the class of systems with independent, operating in parallel, fully reliable 
servers, stochastic incoming flows of customers without any preceding constraints. 
The general goal of the analysis 
carried out in most of the papers is the solution of an optimization problem, which specification heavily depends on 
additional assumptions made. The models considered in the review are classified into several classes depending on 
the amount of \textit{a~priori} information and observability at decision times and performance criteria. The description of 
the dispatching algorithms most commonly found in literature and their properties is given. The main methods 
used for the analysis of the systems under these dispatching algorithms are reviewed. This review is intended to 
draw attention of the research community to one of the important problems in the field of information processing.}

\KWE{dispatching; scheduling; parallel service; queueing system; optimization}


\DOI{10.14357/19922264150406}

\Ack
\noindent
The research was partly supported by the Russian Foundation for Basic Research (projects 15-07-03406
 and 13-07-00223).


%\vspace*{3pt}

  \begin{multicols}{2}

\renewcommand{\bibname}{\protect\rmfamily References}
%\renewcommand{\bibname}{\large\protect\rm References}

{\small\frenchspacing
 {%\baselineskip=10.8pt
 \addcontentsline{toc}{section}{References}
 \begin{thebibliography}{99}
    \bibitem{1-kon-1}
    \Aue{Mukhopadhyay, A., and R.\,R.~Mazumdar}. Analysis of load balancing in large 
heterogeneous processor sharing systems. Available at: {\sf http://arxiv.org/abs/1311.5806} 
(accessed October~17, 2015).
    \bibitem{2-kon-1}
    \Aue{Hyyti$\ddot{\mbox{a}}$,~E.} 2013. Optimal routing of fixed size jobs to two parallel 
servers. \textit{INFOR: Inform. Syst. Oper. Res.} 51(4):215--224.
\bibitem{7-kon-1} %3
    \Aue{Mitzenmacher, M.} 1996. The power of two choices in randomized load balancing. 
Berkeley. Ph.D. Thesis. 
    \bibitem{8-kon-1} %4
    \Aue{Vvedenskaya, N.\,D., R.\,L.~Dobrushin, and F.\,I.~Karpelevich}. 1996. Queueing 
system with selection of the shortest of two queues: An asymptotic approach. \textit{Probl. Inf. 
Transm.} 32(1):20--34.
 \bibitem{6-kon-1} %5
    \Aue{Martin, J.\,B., and Y.\,M.~Suhov}. 1999. Fast jackson networks. \textit{Ann. Appl. 
Probab.}  9(4):840--854.
    \bibitem{3-kon-1} %6
    \Aue{Graham, C.} 2000. Chaoticity on path space for a queueing network with selection of 
the shortest queue among several. \textit{J.~Appl. Probab.} 37:198--211.
\bibitem{9-kon-1} %7
    \Aue{Mitzenmacher, M., A.~Richa, and R.~Sitaraman}. 2001. The power of two random 
choices: A survey of techniques and results. \textit{Handbook of randomized computing}. 
 Eds.\  S.~Rajasekaran, P.\,M.~Pardalos,  J.\,H.~Reif, and J.~Rolim. 
 Norwell, MA: Kluwer Academic Publs. 1:255--312.
    \bibitem{4-kon-1} %8
    \Aue{Luczak, M., and C.~McDiarmid}. 2005. On the power of two choices: Balls and bins 
in continuous time. \textit{Ann. Appl. Probab.} 15(3):1733--1764.
    \bibitem{5-kon-1} %9
    \Aue{Luczak, M., and C.~McDiarmid}. 2006. On the maximum queue length in the 
supermarket model. \textit{Ann. Probab.} 34(2):493--527.
   
    
    
    \bibitem{10-kon-1}
    \Aue{Bramson, M., Y.\,Lu, and B.~Prabhakar}. 2010. Randomized load balancing with 
general service time distributions. \textit{ACM Special Interest Group on Computer Systems 
Performance, SIGMETRICS Proceedings}. 38(1):275--286.
    \bibitem{11-kon-1}
    \Aue{Winston, W.} 1977. Optimality of the shortest line discipline. \textit{J.~Appl. Probab.} 
14:181--189.
    \bibitem{12-kon-1}
    \Aue{Weber, R.} 1978. On the optimal assignment of customers to parallel servers. \textit{J. 
Appl. Probab.}  15:406-413.

    
    \bibitem{14-kon-1} %13
    \Aue{Haight, A.} 1958. Two queues in parallel. \textit{Biometrika} 45:401--410.
    
    \bibitem{13-kon-1} %14
    \Aue{Kingman, J.\,F.\,C.} 1961. Two similar queues in parallel. \textit{Biometrika} 
48:1316--1323.

   
    \bibitem{16-kon-1} %15
    \Aue{Gupta, V., M.~Harchol-Balter, K.~Sigman, and W.~Whitt}. 2007. Analysis of 
    join-the-shortest-queue routing for Web Server Farms. \textit{Performance Evaluation} 
64:1062--1081.
    \bibitem{17-kon-1} %16
    \Aue{Akgun, O., R.~Righter, and R.~Wolff}. 2011. Multiple server system with flexible 
arrivals. \textit{Adv. Appl. Probab.} 43:985--1004.


    
    \bibitem{19-kon-1} %17
    \Aue{Harchol-Balter, M., M.~Crovella, and C.~Murta}. 1999. On choosing a~task 
assignment policy for a~distributed server system. \textit{J.~Parallel Distr. Comp.} 
59(2):204--228.
\bibitem{21-kon-1} %18
    \Aue{Paxson, V., and S.~Floyd}. 1995. Wide area traffic: The failure of Poisson modeling. 
\textit{IEEE/ACM Trans. Netw.} 3(3):226--244.
    \bibitem{22-kon-1} %19
    \Aue{Peterson, D., and D.~Adams}. 1996. Fractal patterns in DASD I/O traffic. 
\textit{22nd Computer  Measurement Group Conference (International) Proceedings}. 
San Diego, CA. 560--571.
\bibitem{18-kon-1} %20
    \Aue{Crovella, M., M.~Taqqu, and A.~Bestavros}. 1998. Heavy-tailed probability 
distributions in the World Wide Web. \textit{A~practical guide to heavy tails}. Eds.\ 
R.\,J.~Adler, R.\,E.~Feldman, and M.\,S.~Taqqu. Cambridge, MA: Birkhauser Boston 
Inc. 3--25.
    \bibitem{20-kon-1} %21
    \Aue{Feitelson, D.} 2015. \textit{Workload modeling for computer systems performance 
evaluation}. Cambridge, MA: Cambridge University Press. 597~p.
    
    \bibitem{23-kon-1} %22
    \Aue{Whitt, W.} 1986. Deciding which queue to join: Some counterexamples. 
\textit{Oper. Res.} 34(1):55--62.
    \bibitem{24-kon-1} %23
    \Aue{Crovella, M., M.~Harchol-Balter, and C.~Murta}. 1997. Task assignmentin 
a~distributed system: Improving performance by unbalancing load. Boston:
Boston University. Boston University 
Computer Science Department Technical Reports. BUCS-TR-1997-018.
    \bibitem{25-kon-1} %24
    \Aue{Crovella, M., M.~Harchol-Balter, and C.~Murta}. 1998. Task assignment in a 
distributed system: Improving performance by unbalancing load. \textit{ACM Sigmetrics'98 
Conference on Measurement and Modeling of Computer Systems Poster Session Proceedings}.  
Madison, WI. 268--269.
    \bibitem{26-kon-1} %25
    \Aue{Harchol-Balter, M., M.~Crovella, and C.~Murta}. 1999. On choosing a task 
assignment policy for a distributed server system. \textit{J.~Parallel Distr. Comp.}  
59:204--228.
    \bibitem{27-kon-1} %26
    \Aue{Harchol-Balter, M.} 2002. Task assignment with unknown duration. \textit{J.~ACM} 
49:260--288.
    \bibitem{28-kon-1} %27
    \Aue{Broberg, J., Z. Tari, and P.~Zeephongsekul}. 2004. Task assignment based on 
prioritising traffic flows. \textit{Principles of distributed systems}.
Ed.\ T.~Higashino. Lecture notes in computer science ser.
 Grenoble, France:   Springer. 3544:415--430.
    \bibitem{29-kon-1} %28
    \Aue{Broberg, J., Z. Tari, and P.~Zeephongsekul}. 2006. Task assignment with work-
conserving migration. \textit{J.~Parallel Computing} 32:808--830.
    \bibitem{30-kon-1} %29
    \Aue{Jayasinghe, M., Z.~Tari, and P.~Zeephongsekult}. 2010. A~scalable multi-tier task 
assignment policy with minimum excess load. \textit{IEEE Symposium on Computers and 
Communications Proceedings}. Riccione, Italy: IEEE.  913--918.
    \bibitem{31-kon-1} %30
    \Aue{Doroudi, S., E.~Hyyti$\ddot{\mbox{a}}$, and M.~Harchol-Balter}. 2014. Value 
driven load balancing. \textit{Perform. Evaluation} 79:306--327.
    \bibitem{32-kon-1} %31
    \Aue{Bodas, T., A. Ganesh, and D.~Manjunath}. 2014. Tolls and welfare optimization for 
multiclass traffic in multiqueue systems. Available at: {\sf http://arxiv.org/abs/1409.7195} 
(accessed October~17, 2015).
    \bibitem{33-kon-1} %32
    \Aue{Becker, K., D. Gaver, K.~Glazebrook, P.~Jacobs, and S.~Lawphongpanich}. 2000. 
Allocation of tasks to specialized processors: A~planning approach. \textit{Eur. 
J.~Oper. Res.} 126:80--88.
    \bibitem{34-kon-1} %33
    \Aue{Hyyti$\ddot{\mbox{a}}$, E.} 2013. Optimal routing of fixed size jobs to two parallel 
servers. \textit{INFOR: Inform. Syst. Oper. Res.} 51(4):215--224.
    \bibitem{35-kon-1} %34
    \Aue{Harchol-Balter, M., M.~Crovella, and C.~Murta}. 1999. On choosing a task 
assignment policy for a distributed server system. \textit{J.~Parallel  Distr. Comp.} 
59:204--228.
    \bibitem{36-kon-1} %35
    \Aue{Hyyti$\ddot{\mbox{a}}$, E., A. Penttinen, S.~Aalto., and J.~Virtamo}. 2011. 
Dispatching problem with fixed size jobs and processor sharing discipline. \textit{23rd 
Teletraffic Congress (International) (ITC'23)}. San Fransisco, CA. 190--197.
    \bibitem{37-kon-1} %36
    \Aue{Konovalov, M., and R. Razumchik}. 2015. Iterative algorithm for threshold calculation 
in the problem of routing fixed size jobs to two parallel servers. \textit{J.~Telecommunications  
Inform. Technol.} 3:32--38.
 \bibitem{15-kon-1} %37
    \Aue{Feng, H., V.~Misra, and D.~Rubenstein}. 2005. Optimal state-free, size-aware 
dispatching for heterogeneous $M$/$G$/-type systems. \textit{Perform. Evaluation} 62:475--492.
    \bibitem{38-kon-1}
    Cisco LocalDirector 400 Series. Available at: {\sf 
http://www. cisco.com/c/en/us/products/routers/localdirector-400-series}
(accessed December~2, 2015). 
    \bibitem{39-kon-1}
    \Aue{Pistoia, M., and C.~Letilley}. 1999. IBM websphere performance pack: Load 
balancing with IBM secure way network dispatcher. \textit{IBM Redbooks}.
    \bibitem{40-kon-1}
    F5 Products, Big-IP. Available at: {\sf http://www.f5.com/ products/big-ip}
    (accessed December~2, 2015).
   
    \bibitem{42-kon-1} %41
    \Aue{Schurman, E., and J.~Brutlag}. 2009. The user and business impact on server delays, 
additional bytes and http chunking in web search. \textit{O'Reilly Velocity Web 
Performance and Operations Conference}. Available at:
{\sf 
http://\linebreak velocityconf.com/velocity2009/public/schedule/detail/ 8523}
(accessed December~2, 2015).
 \bibitem{41-kon-1} %42
    Microsoft sharepoint 2010 load balancer. Available at: {\sf 
http:// loadbalancer.org/applications/microsoft-apps/ microsoft-sharepoint}
(accesse December~2, 2015). 
    \end{thebibliography}

 }
 }

\end{multicols}

\vspace*{-3pt}

\hfill{\small\textit{Received October 19, 2015}}
    
    \Contr
    
    \noindent
    \textbf{Konovalov Mikhail G.} (b.\ 1950)~--- Doctor of Science in technology, 
    Head of Laboratory, Institute of Informatics Problems, Federal Research Ceter 
    ''Computer Science and Control'' of the Russian Academy of Sciences, 44-2 Vavilov 
    Str., Moscow 119333, Russian Federation; mkonovalov@ipiran.ru
    
    \vspace*{3pt}
    
    \noindent
    \textbf{Razumchik Rostislav V.} (b.\ 1984)~--- Candidate of Science (PhD) in 
    physics and mathematics, senior scientist, Institute of Informatics Problems, 
    Federal Research Center ``Computer Science and Control'' of the Russian Academy 
    of Sciences, 44-2 Vavilov Str., Moscow 119333, Russian Federation; 
    associate professor, Peoples' Friendship University of Russia, 
    6 Miklukho-Maklaya Str., Moscow 117198, Russian Federation; rrazumchik@ipiran.ru 
    
    
\label{end\stat}


\renewcommand{\bibname}{\protect\rm Литература}