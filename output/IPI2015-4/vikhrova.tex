\def\stat{vikhrova}

\def\tit{К АНАЛИЗУ ПОКАЗАТЕЛЕЙ КАЧЕСТВА ОБСЛУЖИВАНИЯ 
В~СОВРЕМЕННЫХ БЕСПРОВОДНЫХ СЕТЯХ}

\def\titkol{К анализу показателей качества обслуживания 
в~современных беспроводных сетях}

\def\aut{О.\,Г. Вихрова$^1$, К.\,Е.~Самуйлов$^2$, Э.\,С.~Сопин$^3$, С.\,Я.~Шоргин$^4$}

\def\autkol{О.\,Г. Вихрова, К.\,Е.~Самуйлов, Э.\,С.~Сопин, С.\,Я.~Шоргин}

\titel{\tit}{\aut}{\autkol}{\titkol}

%{\renewcommand{\thefootnote}{\fnsymbol{footnote}} \footnotetext[1]
%{Работа выполнена при частичной финансовой поддержке РФФИ (проект 13-01-00215).}}


\renewcommand{\thefootnote}{\arabic{footnote}}
\footnotetext[1]{Российский университет дружбы народов, o.vikhrova@gmail.com}
\footnotetext[2]{Российский университет дружбы народов, ksam@sci.pfu.edu.ru}
\footnotetext[3]{Российский университет дружбы народов, esopin@sci.pfu.edu.ru}
\footnotetext[4]{Федеральный исследовательский центр <<Информатика и~управление>> Российской академии 
наук, sshorgin@ipiran.ru}
    
  \Abst{Аналитики предсказывают, что в~период с~2014 по~2019~гг.\ темпы роста мирового 
мобильного трафика втрое превысят темпы роста мирового фиксированного трафика. Число 
мобильных пользователей увеличится с~4,1~млрд до~4,9~млрд, число мобильных 
устройств и~подключений может достигнуть~10~млрд. К~2019~г.\ скорость передачи 
в~мировых мобильных сетях может увеличиться с~1,7 до~4,0~Мбит/с. 
Особое внимание стоит обратить на долю мобильного видео, которая составит~72\%~мирового 
мобильного трафика. Тенденции роста числа мобильных подключений и~объема 
мобильного трафика на сегодняшний день ставят перед операторами мобильных услуг 
задачу эффективного и~адаптивного использования доступных радиоресурсов. В~связи 
с~этим в~работе исследуется упрощенная математическая модель, которая позволяет 
аналитически оценить вероятность блокировки системы и~среднее значение занятого ресурса 
в~соответствии с политикой распределения ресурсов технологии LTE-Advanced.}
  
  \KWE{LTE-Advanced; политика распределения ресурсов; СМО с ограниченными 
ресурсами}

\DOI{10.14357/19922264150405}

\vskip 14pt plus 9pt minus 6pt

\thispagestyle{headings}

\begin{multicols}{2}

\label{st\stat}

\section{Введение}
    
  Популярность мультимедийных услуг обуслов\-ле\-на стремительным ростом 
числа разнообразных мобильных устройств, подключенных к~сети. 
Пользователи все чаще предпочитают просматривать медиаконтент на 
смартфонах, планшетах и~отдают предпочтение беспроводному типу связи. 
Провайдеры телекоммуникационных услуг вынуждены искать способы 
эффективного и~адаптивного распределения и~использования ограниченных 
радиоресурсов. Технология высокоскоростной беспроводной передачи данных 
LTE-Advanced позволяет выделять каждой новой сессии диапазоны частот, 
основываясь на принципах мультидиапазонности, принимая во внимание тип 
запрашиваемой пользователем услуги и~учитывая географическую зону 
обслуживания. 
  
  Каждое мобильное устройство сети LTE может передавать данные базовой 
станции с определенной скоростью. Мгновенная скорость в~этом случае 
определяется политикой распределения ресурсов и~ограничена максимальным 
значением, которое определяется по формуле $c_i^{\max} \hm= 
w\mathrm{log}_2(1\hm+\gamma_ip_{\max})$ для некоторой $i$-й сес\-сии, где 
$w$~--- доступная ширина полосы частот; $p_{\max}$~--- мощность передачи 
сигнала усилителем; $\gamma_i$~--- отношение сигнала к~шуму для данной 
сессии, которое зависит от расстояния между устройством и~станцией. Сессия 
может быть принята только в~том случае, если ресурсов системы достаточно, т.\,е.\ 
должно выполняться неравенство $\sum\nolimits_i c_i/c_i^{\max}\hm\leq 1$, 
где $c_i$~--- запрашиваемая $i$-й сессией скорость передачи данных. Для 
анализа вероятности блокировки вызова и~доли используемых радиоресурсов 
системы в~данной работе построена математическая модель в~терминах систем 
массового обслуживания (СМО) с ограниченными ресурсами~\cite{5-v}.

\vspace*{-6pt}

\section{Система массового обслуживания с~дискретным~ресурсом}

\vspace*{-2pt}
    
  В работе~\cite{2-v} рассматривается многолинейная сис\-те\-ма без очереди, 
  в~которой поступившая заявка занимает некоторые ресурсы на все время ее 
обслуживания. По завершении обслуживания весь занятый данной заявкой 
объем ресурсов осво\-бож\-да\-ет\-ся (рис.~1). Если в~системе недостаточно 
свободных ресурсов, необходимых для обслуживания поступившей заявки, то 
такая заявка теряется. Предполагается, что случайные векторы, описывающие 
требования заявок к~ресурсам, не зависят от процессов поступления и~
обслуживания заявок, независимы в~совокупности и~одинаково распределены. 
При анализе данной системы необходимо для каждой заявки запоминать 
вектор занятых ею ресурсов. 
  \begin{figure*} %fig1
           \vspace*{1pt}
 \begin{center}
 \mbox{%
 \epsfxsize=155.162mm
 \epsfbox{vih-1.eps}
 }
 \end{center}
 \vspace*{-11pt}
  \Caption{Функционирование исходной СМО: (\textit{а})~в системе одна заявка: занят 
один прибор и~занято~$r_1$~единиц ресурса;
  (\textit{б})~в~системе две заявки: заняты два прибора; 1-я заявка 
  занимает~$r_1$~единиц ресурса и~2-я заявка занимает $r_2$ единиц ресурса;
  (\textit{в})~обслужилась вторая заявка: занят один прибор и~1-я заявка 
занимает~$r_1$~единиц ресурса}
  \end{figure*}
  
  Для упрощения анализа модели предлагается отслеживать только суммарный 
объем ресурсов, занятый всеми заявками. В~связи с тем, что в~упрощенной 
системе не отслеживаются объемы ресурсов, занятых каждой заявкой, объемы 
ресурсов, освобождаемых по завершении обслуживания, могут отличаться от 
тех, что были выделены заявкам при поступлении. При заданных суммарных 
объемах занятых ресурсов и~числе заявок в~системе в~момент окончания 
обслуживания объемы освобождаемых ресурсов не зависят от поведения 
системы до этого момента и~имеют функцию распределения, вы\-чис\-ля\-емую по 
формуле Байеса. 
  
  Средние значения объема занятого ресурса в~исходной и~упрощенной 
системе оказались очень близки при пуассоновском входящем потоке 
и~экспоненциальном обслуживании~\cite{2-v}. Позднее в~\cite{4-v} было 
показано, что близкие значения имеют не только средние, но и~стационарные 
распределения объема занимаемых ресурсов. И~наконец, в~статье~\cite{3-v} 
было строго доказано, что распределение стационарных вероятностей 
исходной и~упрощенной модели совпадают. 
  
  Рассматривается упрощенная СМО заявок с~$N$ 
приборами и~ограниченным дискретным ресурсом объемом~$R$. В~систему 
поступает пуассоновский поток заявок с параметром~$\lambda$. Длительности 
обслуживания заявок независимы между собой, также не зависят от 
поступающего потока, экспоненциально распределены с параметром~$\mu$. 
В~системе в~некоторый момент времени~$t$ находится~$\xi(t)$~заявок, при 
этом занято~$\delta(t)$ ресурса, $\delta(t)\hm<R$. Поступающей в~систему для 
обслуживания $i$-й заявке необходимо выделить целое число единиц ресурса 
$r_i\hm\geq 0$ (рис.~2). Случайные величины~$r_i$ независимы 
в~совокупности и~одинаково распределены с функцией распределения $F(x)$, 
средним значением~$m$ и~дисперсией~$\sigma^2$. Значение выделяемого 
объема ресурса не зависит от процессов поступления и~обслуживания заявок. 
  
  Поступившая $i$-я заявка теряется, если на момент ее поступления объем 
доступного ресурса меньше объема ресурса, необходимого для ее 
обслуживания: $R\hm- \delta(t)\hm< r_i$, а~также в~случае, если все приборы 
заняты: $\xi(t)\hm=N$. Объем занятого ресурса системы~$\delta(t)$ 
увеличивается на величину ресурса $r_i\hm\geq 0$, выделенного $i$-й заявке 
в~момент начала ее обслуживания.
  
  Объем занятых ресурсов $\delta(\tau_i)$ уменьшается на случайную 
величину~$v_i$ в~момент~$\tau_i$ завершения обслуживания $i$-й заявки. 
Случайная величина~$v_i$ при некотором заданном числе заявок в~системе 
$\xi(\tau_i)\hm= k$ и~объеме занятого ресурса $\delta(\tau_i)\hm=y$ не зависит 
от поведения системы в~прошлом и~согласно формуле Байеса имеет функцию 
распределения $f_k(x\vert y) \hm= P(v_i\leq x\vert \xi(\tau_i)\hm= k; 
\delta(\tau_i)\hm=y)$, $0\hm\leq x\hm\leq y$. 
  
  Случайные величины~$r_i$ требуемого заявке ресурса принимают значения 
$j\hm= \overline{0,R}$. Пусть $p_j\hm= P(r_i=j)$, $j\hm= \overline{0,R}$,~--- 
распределение случайных величин~$r_i$.
  
  
  \begin{figure*} %fig2
             \vspace*{1pt}
 \begin{center}
 \mbox{%
 \epsfxsize=163.663mm
 \epsfbox{vih-2.eps}
 }
 \end{center}
 \vspace*{-11pt}
  \Caption{Функционирование упрощенной СМО:
  (\textit{а})~в системе одна заявка: занят один прибор и~занято~$r_1$~единиц ресурса;
  (\textit{б})~в системе три заявки: заняты три прибора и~занято $r_1\hm+r_2\hm+r_3$ 
единиц ресурса; (\textit{в})~две заявки обслужились: занят один прибор и~занято 
$r_1\hm+r_2\hm+r_3\hm-v_1\hm-v_2$ единиц ресурса}
  \end{figure*}
  
  Введем обозначение~$p_j^{(k)}$, $j\hm= \overline{0,R}$, для $k$-крат\-ной 
свертки этого распределения, которое позволяет определить, какие значения 
принимает случайная величина  $y\hm= \sum\nolimits_{i=0}^k r_i$.

 Тогда 
функция распределения объема всего занятого ресурса системы~$k$~заявками 
$F_k(x\vert j)$ является\linebreak\vspace*{-12pt}
\begin{center}  %fig3
\vspace*{-1pt}
\mbox{%
 \epsfxsize=78.119mm
 \epsfbox{vih-3.eps}
 }
 

\vspace*{3pt}


{{\figurename~3}\ \ \small{Пространство состояний СП $X(t)$}}
\end{center}
 
 \vspace*{12pt}
 \begin{center}  %fig4
\vspace*{-1pt}
\mbox{%
 \epsfxsize=77.584mm
 \epsfbox{vih-4.eps}
 }
 
\end{center}

%\vspace*{9pt}

\noindent
{{\figurename~4}\ \ \small{Диаграмма переходов между состояниями системы}}
 
 \vspace*{12pt}


\noindent
 ку\-соч\-но-по\-сто\-ян\-ной на всей области ее 
определения, в~точке $x\hm=i$, $i\hm= \overline{1,R}$, имеет скачок на 
величину
  $$
  p_{ij}^k = \fr{p_i p_{j-i}^{(k-1)}}{p_j^{(k)}}\,,\enskip 0<i\leq j\,,\ 
i=\overline{0,R}\,.
  $$
  
  Свертки вероятностей $p_j^{(k)}$, $k\hm\geq 2$, в~общем случае могут быть 
вычислены на основе известного распределения~$p_i$, $0\hm\leq i\hm\leq R$, 
по рекуррентной формуле

\noindent
  \begin{equation}
  p_j^{(k)} = \sum\limits_{i=0}^j p_i p_{j-i}^{(k-1)}\,.
  \label{e1-v}
  \end{equation}
  
  Случайный процесс (СП) $X(t)\hm= (\xi(t),\delta(t))$ на множестве состояний 
$X_k\hm= \{ (k,i)\vert 0\hm\leq i\hm\leq R, p_i^{(k)}\hm>0\}$ является цепью 
Маркова, заданной на пространстве состояний $X\hm= \mathop{\cup}\limits_{k=0}^N 
X_k$, где $X_k\hm= \{ (k,i(\vert 0\hm\leq i \hm\leq R, p_i^{(k)}\hm>0\}$, и~
изображенной на рис.~3.
  
  На рис.~4 представлена диаграмма переходов между состояниями системы 
при условии, что $1\hm\leq i\hm< N$, $0\hm\leq s\hm\leq j\hm\leq q\hm\leq R$.
  
 
  
  Матрица интенсивностей переходов 
  $$
  A= [a((i,j),(k,r))]
  $$
   является блочной 
трехдиагональной матрицей c~диагональными блоками $\Psi_0, \Psi_1, \ldots , 
\Psi_N$, наддиагональными блоками $\Lambda_1,\ldots, \Lambda_N$ 
и~поддиагональными блоками ${M}_0, \ldots , {M}_{N-1}$, где

\noindent
  \begin{align}
  \Psi_0 &= -\lambda P_R\,;\label{e2-v}\\
  \Lambda_1 &= (\lambda_{0,j})_{(j\vert(1,j)\in X_1)} = \lambda p_j\,; 
\label{e3-v}\\
  {M}_0 &= (\mu_{i,0})_{(i\vert (1,i)\in X_1)} = \mu\,;\label{e4-v}\\
  \Psi_n &= (\psi_{i,j})_{(i,j\vert(n,i),(n,j)\in X_k}
  \begin{cases} -(\lambda P_{R-i} +n\mu), &\\
  &\hspace*{-9mm}i=j;\\
  0\,, &\hspace*{-9mm}i\not=j
  \end{cases}\!\!
  \label{e5-v}
  \end{align}
при $n=\overline{1,N-1}$;
\pagebreak

\noindent
\begin{align}
\Psi_N&= (\psi_{i,j})_{(i,j\vert(N,i),(N,j)\in X_N)}\begin{cases}
-N\mu\,, &\ i=j;\\
0\,, &\ i\not=j;
\end{cases}
\label{e6-v}\\
\Lambda_n &= {}\notag\\
&\hspace*{-9mm}{}=(\lambda_{i,j})_{(i,j\vert(n,i)\in X_{k-1},(n,j)\in X_k)} 
\begin{cases}
\lambda p_{j-i}\,, &\\
&\hspace*{-5mm} i\leq j\leq R;\\
0 &
\end{cases}
\end{align}
при $n=\overline{2,N}$;
\begin{multline}
{M}_n ={}\\
\hspace*{-1mm}{}= (\mu_{i,j})_{(i,j\vert(n,i)\in X_{k+1},(n,j)\in X_k)}\!
\begin{cases}\!
\!(n+1)\mu\fr{p_j^{(n)} p_{i-j}}{p_i^{(n+1)}}, &\\
&\hspace*{-20mm} j\leq i\leq R;\\
\!0 &
\end{cases}\!\!\!\!\!\!\!\!\!
\label{e8-v}
\end{multline}
при $n=\overline{1,N-1}$.

  Решив систему уравнений равновесия (СУР) 
  с~матрицей~$A$, можно найти стационарное распределение 
процесса $X(t)\hm= (\xi(t),\delta(t))$, где 
  \begin{align*}
  q_{0,0}& = \lim\limits_{t\to\infty} P\{\xi(t)=0,\delta(t)=0\}\,;\\
  q_{k,i} &=\lim\limits_{t\to\infty} P\{\xi(t)=k,\delta(t)=i\}\,,\\
  &\hspace*{15mm} 1\leq k\leq N\,,\ 0\leq 
i\leq R\,.
  \end{align*}
  
  На основании формул~(\ref{e2-v})--(\ref{e8-v}) СУР можно записать в~следующем виде:
  \begin{gather*}
  \lambda P_R q_{0,0} -\mu \sum\limits_{j=0}^R q_{1,j}=0\,;\\
  \hspace*{-20mm}(\lambda P_{R-j}+i\mu)q_{i,j}-\lambda \sum\limits_{s=0}^j p_{j-s} q_{i-1,s} -{}\\
 \hspace*{10mm} {}- 
(i+1) \mu \sum\limits_{s=j}^R \fr{p_j^{(i)} p_{s-j}}{p_s^{(i+1)}}\,q_{i+1,s}=0\,,\\
  \hspace*{20mm}0\leq j\leq R\,,\enskip 1\leq i\leq N-1\,;\\
  N\mu q_{N,j} -\lambda \sum\limits_{s=0}^j p_{j-s} q_{N-1,s}=0\,,\ 0\leq j\leq 
R\,.
  \end{gather*}
  
  Введем подвекторы стационарных вероятностей $q_0\hm= q_{0,0}$, 
$q_i\hm= (q_{i,j})_{j=\overline{0,R}}$, $i\hm= 1,\ldots ,N$. Сис\-те\-ма урав\-не\-ний
равновесия в~мат\-рич\-ном 
виде с учетом блоч\-но-трех\-диа\-го\-наль\-но\-го вида матрицы 
интенсивностей переходов~$A$ выглядит следующим образом:
  \begin{align*}
  q_0\Psi_0-q_1^{\mathrm{T}} {M}_0&=0\,;\\
  q_i^{\mathrm{T}} \Psi_i -q_{i+1}^{\mathrm{T}} {M}_i -q_{i-1}^{\mathrm{T}} \Lambda_i &=0\,,\ i=1,\ldots, 
N-1\,;\\
  q_N^{\mathrm{T}} \Psi_N-q_{N-1}^{\mathrm{T}} \Lambda_N&=0\,.
  \end{align*}
  
  Матрица~$A$ является неразложимой консервативной трехдиагональной 
матрицей по по\-стро\-ению, поэтому для решения СУР применим метод  
UL-раз\-ло\-же\-ния~\cite{1-v}
  $$
  b= \sum\limits_{k=0}^N b_k \sum\limits_{i=k}^N q_{k,i}\,,
  $$
  где $b_k$~--- условное среднее число занятого ресурса при~$k$~заявках 
в~системе:
  $$
  b_k= \fr{\sum\nolimits_{i=0}^N i p_i^{(k)}}{\sum\nolimits_{i=0}^N 
p_i^{(k)}}\,.
  $$
  
  Вероятность потери заявки из-за нехватки ресурса или свободных приборов 
для обслуживания определяется согласно формуле:
  \begin{equation}
  B=1-\sum\limits_{k=0}^N \sum\limits_{i=k}^N q_{k,i} \sum\limits_{j=1}^{N-i} 
p_j\,.
  \label{e9-v}
  \end{equation}
  
  Среднее число заявок в~системе $\overline{N}$ можно определить по 
формуле:
  \begin{equation}
  \overline{N} = \sum\limits_{i=0}^N iq_i\,,
  \label{e10-v}
  \end{equation}
где $q_i= \sum\nolimits_{j=0}^R q_{i,j}$.

  В работе~\cite{3-v} было найдено аналитическое решение  
СУР~(\ref{e2-v})--(\ref{e8-v}) в~общем виде. Для частного случая, когда объем 
выделяемых заявке ресурсов имеет дискретное распределение, авторами были 
получены более простые формулы для определения стационарных 
вероятностей:
  \begin{align*}
  q_{k,\bullet} &=\lim\limits_{t\to\infty} P\{ \xi(t)=k\} = p_0
\fr{\rho^k}{k!}\sum\limits_{i=0}^R p_i^{(k)}\,,\ 0<k<N\,;\\
  q_{k,j} &=\lim\limits_{t\to\infty} P\{\xi(t)=k;\delta(t)=j\} = p_0 \fr{\rho^k}{k!}\, 
p_j^{(k)}\,,\\
& \hspace*{38mm}0\leq j\leq R\,,\ 0<k\leq N\,,
  \end{align*}
где 
$$
p_0 = \left( 1+\sum\limits_{k=1}^N \fr{\rho^k}{k!} \sum\limits_{i=0}^R 
p_i^{(k)}\right)^{-1}\,.
$$

  Тогда средний объем занятых ресурсов~$b$ можно найти по формуле:
  \begin{equation}
  b=\sum\limits_{k=0}^N b_k q_{k,\bullet} = p_0\sum\limits_{k=0}^N 
\fr{\rho^k}{k!} \sum\limits_{i=0}^R i p_i^{(k)}\,.
  \label{e11-v}
  \end{equation}
  
  Вероятность потери заявки определяется согласно формуле: 
  \begin{multline}
  B=1-\sum\limits_{k=0}^{N-1} \sum\limits_{i=0}^R q_{k,i} 
\sum\limits_{j=0}^{R-i} p_j={}\\
{}=1-p_0 \sum\limits_{k=0}^{N-1} \fr{\rho^k}{k!} 
\sum\limits_{i=0}^R p_i^{(k+1)}\,.
  \label{e12-v}
  \end{multline}

\section{Численный анализ}
    
  Рассмотрим различные частные случаи, когда объем ресурса, выделяемого 
заявке, имеет биномиальное, смещенное биномиальное и~геометрическое 
распределение, а~общий объем ресурса сис\-те\-мы~$R$ совпадает с количеством 
приборов~$N$.
   
  Для вычисления вероят\-ност\-но-вре\-мен\-н$\acute{\mbox{ы}}$х характеристик  
модели~(\ref{e9-v})--(\ref{e12-v}) необходимо вы\-чис\-лить свертку 
распределения вероятностей $p_j^{(k)}$ по формуле~(1) для выбранного 
дискретного распределения объема занимаемых заявкой ресурсов.
  
  Рассмотрим частный случай, когда объем ресурса, выделяемого одной заявке 
$r_j\hm\geq 0$, имеет биномиальное распределение 
$$
p_i= \begin{pmatrix}
r\\ 
i\end{pmatrix} p^i(1-p)^{r-i}\,,\enskip 
0\leq i\leq r\,,
$$
 где $p\hm= m/r$;
$$
p_j^{(k)} = \begin{pmatrix} 
kr\\ j\end{pmatrix} p^j (1- p)^{kr-j}\,.
$$ 

Тогда 
распределение объема ресурса, который освободится после завершения 
обслуживания $i$-й заявки 
\begin{multline*}
p_{ij}^k\hm= \fr{p_i p_{j-i}^{(k-1)}}{p_j^{(k)}} = 
\begin{pmatrix}r\\i\end{pmatrix}\begin{pmatrix} kr-r\\ j-i\end{pmatrix}\Bigg/ 
\begin{pmatrix} kr\\ j\end{pmatrix}\,,\\ 0\leq i\leq r\,,\ \ i\leq j\,,\ \ 0\leq 
j\leq kr\,,
\end{multline*}
 зависит только от максимального числа~$r$ ресурса, доступного 
одной заявке.
  
  В важном для приложений частном случае, когда объем выделяемого заявке 
ресурса больше нуля, $r_j\hm>0$ имеет смещенное биномиальное 
распределение 
$$
p_i\hm= \begin{pmatrix} r-1\\ i-1\end{pmatrix} p^{i-1}(1-p)^{r-i}\,,\enskip 
1\leq i\leq r\,,
$$
где $p\hm=(m\hm-1)/(r\hm-1)$, $m\hm= np\hm+ (1\hm-p)$ 
и $\sigma^2\hm= rp(1\hm-p)$;
$$
p_j^{(k)}\hm= \begin{pmatrix} k(r-1)\\ j-
k\end{pmatrix} p^{j-k} (1\hm-p)^{kr-j}\,.
$$

 Соответственно, распределение 
  \begin{multline*}
  \hspace*{-0.26108pt}p_{ij}^k \hm= {\begin{pmatrix} r-1\\ i-1\end{pmatrix} \begin{pmatrix} k(r-1)-
(r-1)\\ (j-k) -(i-1)\end{pmatrix}}\Bigg/ {\begin{pmatrix} k(r-1)\\ j-k\end{pmatrix}},
\\ 
1< i\leq r,\ i\leq j,\ k\leq j \leq k(r-1),\ 1<k\leq j.
\end{multline*}
  
  Для сравнения полученных для первых двух частных случаев результатов 
рассмотрим пример, когда число единиц ресурса, выделяемого поступившей 
заявке, распределено согласно геометрическому закону: 
$$
p_i= p^i(1-p)\,, \enskip
1\leq i\leq r\,,
$$
где 
\begin{gather*}
m= \fr{1-p}{p}\,;\quad \sigma^2= \fr{1-p}{p^2}\,; \\
p_j^{(k)} = \begin{pmatrix} k+j-1\\ k\end{pmatrix} p^j (1-p)^k\,, 
\end{gather*}
распределение 
\begin{multline*}
p_{ij}^k = \begin{pmatrix} k+j-i-2\\ k-1\end{pmatrix} \Bigg/ 
\begin{pmatrix} k+j-1\\ k\end{pmatrix}\,,\\
i\leq j\,,\enskip k\leq j\,,\enskip 1< k\leq j\,.
\end{multline*}
  
  В качестве исходных данных для численного анализа были выбраны 
значения $N\hm=R\hm=100$, $\mu\hm= 1$, $\lambda\hm= \{12, 13,\ldots, 20\}$. 
На риc.~5 пред\-став\-ле\-ны графики зависимости вероятностных характеристик 
от нагрузки~$\rho$ для выбранных дискретных распределений объема ресурса. 
Параметры распределений были подобраны таким образом, что 
математическое ожидание $m\hm=5{,}4$ и~объем доступного одной заявке 
ресурса $r\hm=18$ для всех трех распределений, а~дисперсии были 
различными. 
 
  
  
  
  На рис.~5,\,\textit{а} заметим, что для случаев, когда выделяемые заявкам ресурсы 
имеют геометрическое распределение, среднее число заявок в~системе при 
$\rho\hm>1$ растет быстрее, чем в~случаях с биномиальными распределениями. 
График зависимости среднего числа занятого ресурса для геометрического 
распределения, наоборот, приближается к~кривой для биномиального 
распределения. Так как геометрическое распределение имеет б$\acute{\mbox{о}}$льшую 
дисперсию, чем биномиальное и~смещенное биномиальное распределения, то 
была исследована зависимость тех же вероятностных характеристик при 
различной нагрузке от дисперсии.
  
  На рис.~6 представлены графики зависи\-мости среднего числа заявок, 
среднего числа занятых ресурсов системы и~вероятности блокировки СМО от 
дисперсии~$\sigma^2$. С~ростом дисперсии гео\-мет\-ри\-че\-ско-\linebreak\vspace*{-12pt}

\pagebreak

\setcounter{figure}{4}

\end{multicols}

 \begin{figure} %5-6
 \vspace*{1pt}
 \begin{center}
 \mbox{%
 \epsfxsize=162mm
 \epsfbox{vih-5-6.eps}
 }
 \end{center}
 \vspace*{-16pt}
\begin{minipage}[t]{80mm}
\Caption{Зависимость среднего числа заявок в~системе~(\textit{а}),
  среднего числа занятого ресурса~(\textit{б}) и~вероятности блокировки
  СМО~(\textit{в}) от нагрузки для геометрического~(\textit{1}); 
  биноминального~(\textit{2}) и~смещенного биноминального~(\textit{3}) распределений}
\end{minipage}
\hfill
\begin{minipage}[t]{80mm}
\Caption{Зависимость среднего числа заявок в~системе~(\textit{а}), 
среднего числа занятого ресурса~(\textit{б})
  и~вероятности блокировки СМО~(\textit{в}) от дисперсии: \textit{1}~--- 
  $\rho\hm= 0{,}9$; \textit{2}~--- 0,8; 
\textit{3}~--- $\rho\hm=0{,}7$}
\end{minipage}
\vspace*{-4pt}
\end{figure}
 

 
 \begin{multicols}{2}

\noindent
го распределения 
среднее число занятых приборов увеличивается, а~среднее число занятого 
ресурса уменьшается. 

Заявки, которые требуют больше ресурсов для 
обслуживания, будут блокироваться, 
и~в~систему будет поступать больше 
заявок, требующих меньшего объема ресурса.

\columnbreak

%\columnbreak
  

  
\section{Заключение}


    
  В работе исследуется упрощенная математи\-ческая модель установления 
сессий между мо\-биль\-ными устройствами конечных пользователей 
мультимедий\-ных услуг и~базовой станцией с распределением ограниченных 
ресурсов беспроводной высокоскоростной сети LTE. 
%
Для расчетов предложены 
численный и~аналитический способы расчета вероятностных характеристик 
системы. 

Также был проведен сравнительный анализ зависимости 
вероятностных характеристик системы от параметров и~типов распределений. 

Было замечено, что при увеличении дисперсии геометрического распределения 
объема выделя\-емо\-го поступившей заявке ресурса среднее чис\-ло заявок 
в~системе и~среднее число занятого ресур\-са уменьшаются. В~условиях 
нагрузки, близкой к~единице, $\rho\hm= \{0{,}7; 0{,}8; 0{,}9\}$, на 
об\-слу\-жи\-ва\-ние будут приниматься заявки, которые для обслу\-жи\-ва\-ния требуют 
наименьшего объема ресурсов. Более требовательные заявки будут 
бло\-ки\-ро\-ваться. 
{\looseness=1

}
  
  Дальнейшие исследования будут включать разработку эффективных 
вычислительных алгоритмов для расчета ве\-ро\-ят\-ност\-но-вре\-мен\-н$\acute{\mbox{ы}}$х 
характеристик модели.
  
  
{\small\frenchspacing
 {%\baselineskip=10.8pt
 \addcontentsline{toc}{section}{References}
 \begin{thebibliography}{9}
 
 \bibitem{5-v}
\Au{Pyattaev A., Johnsson K., Surak~A., Florea~R., Andreev~S., Koucheryavy~Y.} 
Network-assisted D2D communications: Implementing a technology prototype for 
cellular traffic offloading~// 2014 IEEE Wireless Communications and Networking 
Conference (WCNC).~--- Istambul: IEEE, 2014. P.~3266--3271.
 \bibitem{2-v} %1
\Au{Наумов В.\,А., Самуйлов К.\,Е.} О~моделировании систем массового 
обслуживания с множественными ресурсами~// Вестник РУДН. Сер. 
Математика. Информатика. Физика, 2014. №\,3. С.~58--62.
\bibitem{4-v} %2
\Au{Naumov V., Samouylov K., Sopin~E., Andreev~S.} Two approaches to analysis of 
queuing systems with limited resources~// 6th Congress (International) on Ultra 
Modern Telecommunications and Control Systems (ICUMT-2014) Proceedings.~--- 
СПб: IEEE, 2014. P.~485--488.
\bibitem{3-v} %3
\Au{Наумов В.\,А., Самуйлов К.\,Е., Самуйлов~А.\,К.} О~суммарном объеме 
ресурсов, занимаемых заявками~// Автоматика и~телемеханика, 2015 (в печати).

\bibitem{1-v} %4
\Au{Наумов В.\,А.} Численные методы анализа марковских систем.~--- М.: РУДН, 1985. 
36~с.



 \end{thebibliography}

 }
 }

\end{multicols}

\vspace*{-12pt}

\hfill{\small\textit{Поступила в~редакцию 5.10.15}}

\vspace*{4pt}

%\newpage

%\vspace*{-24pt}

\hrule

\vspace*{2pt}

\hrule

\vspace*{-4pt}

\def\tit{ON PERFORMANCE ANALYSIS OF~MODERN WIRELESS NETWORKS\\[-5pt]}

\def\titkol{On performance analysis of modern wireless networks}

\def\aut{O.\,G. Vikhrova$^1$, K.\,E. Samouylov$^1$, E.\,S.~Sopin$^1$, 
and~S.\,Ya.~Shorgin$^2$\\[-11pt]}

\def\autkol{O.\,G. Vikhrova, K.\,E. Samouylov, E.\,S.~Sopin, 
and~S.\,Ya.~Shorgin}

\titel{\tit}{\aut}{\autkol}{\titkol}

\vspace*{-9pt}


\noindent
$^1$Peoples' Friendship University of Russia, 6~Miklukho-Maklaya Str., Moscow 
117198, Russian Federation

\noindent
$^2$Federal Research Center ``Computer Science and Control''' of the Russian 
Academy of Sciences, 44-2~Vavilov\linebreak
$\hphantom{^1}$Str., Moscow 119333, Russian Federation


\def\leftfootline{\small{\textbf{\thepage}
\hfill INFORMATIKA I EE PRIMENENIYA~--- INFORMATICS AND
APPLICATIONS\ \ \ 2015\ \ \ volume~9\ \ \ issue\ 4}
}%
 \def\rightfootline{\small{INFORMATIKA I EE PRIMENENIYA~---
INFORMATICS AND APPLICATIONS\ \ \ 2015\ \ \ volume~9\ \ \ issue\ 4
\hfill \textbf{\thepage}}}

\vspace*{3pt}



\Abste{Analytics predict that worldwide mobile traffic growth rate will exceed fixed traffic 
approximately three times from~2014 to~2019. Number of mobile users will increase up 
to~4.9~billions and mobile devices number will exceed 10~billions. The average mobile network 
connection speed (1.7~Mbps in~2014) will reach nearly~4.0~Mbps by~2019. 
Special attention should be paid to mobile video traffic that will reach three-fourths of the whole 
mobile traffic by~2019. These tendencies bring new challenges for mobile communication 
providers to increase efficiency and additivity of radio resource allocation. In this connection, the 
paper analyzes a simplified model that allows one to obtain analytical estimates of the blocking 
probability and the average value of occupied resources according to the resource allocation policy 
of the LTE-Advanced technology.}

\KWE{LTE-Advance; resource allocation policy; limited resource queue}

\DOI{10.14357/19922264150405}

%\Ack
%\noindent




\vspace*{-9pt}

  \begin{multicols}{2}

\renewcommand{\bibname}{\protect\rmfamily References}
%\renewcommand{\bibname}{\large\protect\rm References}

{\small\frenchspacing
 {%\baselineskip=10.8pt
 \addcontentsline{toc}{section}{References}
 \begin{thebibliography}{9}
 
 \bibitem{5-v-1}
\Aue{Pyattaev, A., K. Johnsson, A.~Surak, R.~Florea, S.~Andreev, and 
Y.~Koucheryavy}.  2014. Network-assisted D2D communications: Implementing 
a~technology prototype for cellular traffic offloading. \textit{Wireless 
Communications and Networking Conference (WCNC)}.  Istanbul: IEEE. 3266--3271.

\bibitem{2-v-1} %1
\Aue{Naumov, V.\,A., and K.\,E.~Samouylov}. 2014. O~mo\-de\-li\-rovanii sistem 
massovogo obsluzhivaniya s~mnozhestvennymi resursami [Modelling queuing 
systems with different types of resources]. \textit{Vestnik RUDN. Ser. Matematika. 
Informatika. Fizika} [PFUR Bulletin, Mathematics, Informatics, Physics]. Moscow.  
3:58--62.
\bibitem{4-v-1} %2
\Aue{Naumov, V., K. Samouylov, E.~Sopin, and S.~Andreev}. 2014. Two approaches 
to analysis of queuing systems with limited resources. \textit{6th
Congress (International) on Ultra Modern 
Telecommunications and Control Systems and Workshops (ICUMT) Proceedings}.  
St.\ Petersburg: IEEE. 485--488.
\bibitem{3-v-1} %3
\Aue{Naumov, V.\,A., K.\,E.~Samouylov, and A.\,K.~Samuylov}. 2015 (in press). 
O~summarnom ob''eme resursov, zanimaemykh zayavkami [Resources aggregation 
in queuing system]. \textit{Automation and Remote Control}.
\bibitem{1-v-1} %4
\Aue{Naumov, V.\,A.} 1985. \textit{Chislennye metody analiza markovskikh 
sistem} [Numerical analysis of Markov systems]. Moscow: PFUR Publs. 36~p.

\end{thebibliography}

 }
 }

\end{multicols}

\vspace*{-12pt}

\hfill{\small\textit{Received October~5, 2015}}

\Contr

\noindent
\textbf{Vikhrova Olga G.} (b.\ 1990)~--- PhD student, Peoples' Friendship 
University of Russia, 6 Miklukho-Maklaya Str., Moscow 117198, Russian 
Federation; o.vikhrova@gmail.ru

\vspace*{3pt}

\noindent
\textbf{Samouylov Konstantin E.} (b.\ 1955)~---
Doctor of Science in technology, professor, Head of Department, Peoples' 
Friendship University of Russia, 6 Miklukho-Maklaya Str., Moscow 117198, 
Russian Federation; ksam@sci.pfu.edu.ru

\vspace*{3pt}

\noindent
\textbf{Sopin Eduard S.} (b.\ 1986)~---
Candidate of Science (PhD) in physics and mathematics, associate professor, 
Peoples' Friendship University of Russia, 6 Miklukho-Maklaya Str., Moscow 
117198, Russian Federation; esopin@sci.pfu.edu.ru

\vspace*{3pt}

\noindent
\textbf{Shorgin Sergey Ya.} (b.\ 1952)~---
Doctor of Science in physics and mathematics, professor, Deputy Director, Federal 
Research Center ``Computer Science and Control''' of the Russian Academy of 
Sciences, 44-2 Vavilov Str., Moscow 119333, Russian Federation; 
sshorgin@ipiran.ru

\label{end\stat}


\renewcommand{\bibname}{\protect\rm Литература}