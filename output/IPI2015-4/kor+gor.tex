\def\stat{kor+gor}

\def\tit{СТАТИСТИЧЕСКОЕ МОДЕЛИРОВАНИЕ ТУРБУЛЕНТНЫХ ПОТОКОВ ТЕПЛА МЕЖДУ
ОКЕАНОМ И~АТМОСФЕРОЙ С~ПОМОЩЬЮ МЕТОДА СКОЛЬЗЯЩЕГО РАЗДЕЛЕНИЯ
КОНЕЧНЫХ НОРМАЛЬНЫХ СМЕСЕЙ$^*$}

\def\titkol{Статистическое моделирование турбулентных потоков тепла между
океаном и~атмосферой с~помощью СРС-метода} %скользящего разделения конечных нормальных смесей}

\def\aut{В.\,Ю.~Королев$^1$, А.\,К.~Горшенин$^2$, С.\,К.~Гулев$^3$, К.\,П.~Беляев$^4$}

\def\autkol{В.\,Ю.~Королев, А.\,К.~Горшенин, С.\,К.~Гулев, К.\,П.~Беляев}

\titel{\tit}{\aut}{\autkol}{\titkol}

{\renewcommand{\thefootnote}{\fnsymbol{footnote}} \footnotetext[1]
{Работа выполнена при поддержке
гранта Президента РФ МК-4103.2014.9 и~РФФИ (проекты
15-07-02652 и~15-37-20851).}}


\renewcommand{\thefootnote}{\arabic{footnote}}
\footnotetext[1]{Факультет вычислительной математики и~кибернетики
Московского государственного университета им. М.\,В.~Ломоносова;
Институт проблем информатики Федерального исследовательского
центра <<Информатика и~управление>> Российской академии наук, 
vkorolev@cs.msu.su }
\footnotetext[2]{Институт проблем информатики Федерального исследовательского
центра <<Информатика и~управление>> Российской академии наук; 
Московский государственный университет
информационных технологий, радиотехники и~электроники,
a.k.gorshenin@gmail.com}
\footnotetext[3]{Институт
океанологии им.\ П.\,П.~Ширшова РАН; географический факультет
Московского государственного
университета им.\ М.\,В.~Ломоносова; Кильский университет, Германия,
gul@sail.msk.ru}
\footnotetext[4]{Институт океанологии им.\ П.\,П.~Ширшова РАН; Федеральный университет штата Баиа, 
Сальвадор, Бразилия,\linebreak kosbel55@gmail.com}

\vspace*{-6pt}


\Abst{Метод скользящего разделения смесей применяется 
к~задаче статистического моделирования закономерностей в~явных 
и~скрытых турбулентных тепловых потоках. В~качестве исходных данных
используются результаты шестичасовых наблюдений в~Атлантике
(NCEP-NCAR, 1948--2008~гг.). В~качестве математической модели
используется аппроксимация конечными смесями нормальных
распределений с~параметрами, зависящими от времени. Применение
методологии скользящего разделения смесей направлено на выявление
закономерности в~изменении этих параметров и~определение
изменчивости, которая может быть ассоциирована как с~трендом, так 
и~с~нерегулярной изменчивостью. Предложен соответствующий подход 
к~определению доли экстремальных наблюдений в~исходной выборке.}

\KW{конечные смеси нормальных распределений;
метод скользящего разделения смесей; вероятностные модели;
интеллектуальный анализ данных}

\DOI{10.14357/19922264150401}



\vskip 14pt plus 9pt minus 6pt

\thispagestyle{headings}

\begin{multicols}{2}

\label{st\stat}

\section{Введение}

Изучение явных и~скрытых поверхностных турбулентных тепловых
потоков между атмосферой и~океаном, определяющих их
взаимодействие, является чрезвычайно важным во многих областях
наук о~Земле. Данные о потоках могут быть получены из нескольких
источников, каждый из которых обладает собственными
достоинствами и~недостатками. Наиболее подробные временн$\acute{\mbox{ы}}$е
ряды (с~периодами  наблюдений в~100 и~более лет)  доступны\linebreak 
с~по\-мощью программы наблюдений Voluntary\linebreak Observing Ship
(VOS)~\cite{Berry2009}, в~то время как данные за последние несколько
десятилетий собираются с~помощью спутников, повторного анализа 
и~комбинированных продуктов (например,
OA-Flux~\cite{Yu2007a,Yu2007b}) с~высоким пространственным 
и~временн$\acute{\mbox{ы}}$м разрешением.

Информация об изменчивости поверхностных турбулентных тепловых
потоков в~большинстве случаев ограничивается первым (в~некоторых\linebreak
случаях еще и~вторым) моментом вероятностного распределения
потоков. Они традиционно вы\-чис\-ля\-ются по временн$\acute{\mbox{о}}$му ряду,
соответствующему потоку, и~составляют основу для
климатологических исследований и~проверок~\cite{Josey2001,Bouras2006}. 
Тем не менее детальная
оценка характеристик теплового потока, в~том числе определение
экстремальных значений, требует точного знания вероятностного
распределения, а~также анализа изменений его параметров во
времени и~пространстве. Отсутствие подобных знаний при построении
океанологических и~климатических моделей серьезно снижает
качество основанного на них прогнозирования.

\begin{figure*}[b] %fig1
         \vspace*{-5pt}
 \begin{center}
 \mbox{%
 \epsfxsize=162.281mm
 \epsfbox{kor-1.eps}
 }
 \end{center}
 \vspace*{-11pt}
\Caption{Гистограммы, построенные по выборкам
$\xi_1,\ldots,\xi_n$ (\textit{а}) и~$\zeta_1,\ldots,\zeta_n$~(\textit{б})
\label{fig1}
с~па\-ра\-мет\-ра\-ми $n\hm=1000$ и~$a\hm=2$}
\end{figure*}

Еще одна важная причина для изучения вероятностных распределений
турбулентных потоков теп\-ла~--- необходимость количественного
оценивания и~минимизации выборочных ошибок в~продуктах,
построенных по базам VOS~\cite{Gulev2007a,Gulev2007b}. Большие по
величине ошибки в~выборках влияют как на оценки средних величин
потоков, так и~на характеристики экстремальных потоков.


Попытка выбрать подходящий тип вероятностного распределения для
турбулентных потоков была осуществлена в~работе~\cite{Gulev2012}, 
в~которой показано, что в~качестве достаточно хорошей аппроксимации
для данных может быть использовано так назы\-ва\-емое
двухпараметрическое распределение Фи\-ше\-ра--Тип\-пе\-та
(FT-рас\-пре\-де\-ле\-ние). В~указанной работе были оценены параметры сдвига
и~масштаба данного распределения, осуществлена проверка гипотез о
качестве подбора модели, а~также была предложена глобальная
климатологическая интерпретация параметров FT-рас\-пре\-де\-ле\-ния. Кроме
того, FT-рас\-пре\-де\-ле\-ние было использовано для анализа временн$\acute{\mbox{ы}}$х рядов
значительного объема для поверхностных турбулентных потоков,
реконструированных по наблюдениям из базы VOS с~1880~г.~\cite{Gulev2013}. 
Тем не менее многие вопросы, связанные 
с~вероятностным распределением поверхностных турбулентных потоков, все
еще остаются открытыми. В~част\-ности, с~по\-мощью FT-рас\-пре\-де\-ле\-ния
часто не удается корректно аппроксимировать экстремальные
турбулентные потоки тепла, а~также в~полной мере учесть случай так
называемых <<тяжелых хвостов>> в~распределениях потоков.




В настоящей работе используется идея представления
вероятностного распределения приращений теплового потока
(разности первого порядка) в~виде конечной смеси нескольких
нормальных (гауссовских) распределений с~параметрами,
зависящими от времени. В~частности, вес каждого из распределений
в~смеси также может изменяться во времени. Для поиска оценок
неизвестных па\-ра\-мет\-ров модели используется метод скользящего
разделения смесей. Данный подход позволяет выявить
закономерности в~изменении параметров и~определить изменчивость,
которая может быть ассоциирована с~трендом, а~также динамику,
связанную с~диффузией или нерегулярной изменчивостью.

\section{Однородность данных}

Статистический анализ стохастических закономерностей 
в~наблюдаемых временн$\acute{\mbox{ы}}$х рядах традиционно подразумевает работу
со всеми име\-ющи\-мися данными без ка\-кой-ли\-бо предварительной\linebreak
обработки с~целью получения однородных данных. Например, 
в~работе~\cite{Gulev2012} FT-рас\-пре\-де\-ле\-ние применялось для
аппроксимации исходного временн$\acute{\mbox{о}}$го ряда. Однако такой подход,
вероятно, не может быть использован для анализа очень длинных
временн$\acute{\mbox{ы}}$х рядов и~эволюции параметров распределения во времени.
При этом выборка, используемая для статистического анализа, не
является однородной, так как отдельные ее элементы не являются
независимыми. Чтобы пояснить это обстоятельство, рассмотрим
следующий модельный пример.

Всюду в~дальнейшем обозначения $\varphi(x)$ и~$\Phi(x)$ будут
использованы для плотности и~функции распределения стандартного
нормального закона, а~именно:

\noindent
\begin{equation*}
\varphi(x)=\fr{1}{\sqrt{2\pi}}\,e^{-x^2/2}\,;\enskip
\Phi(x)=\int\limits_{-\infty}^{x}\varphi(z)\,dz\,,\enskip x\in\mathbb{R}\,.
\end{equation*}

Предположим, что $n$~--- натуральное число,
$\xi_1,\xi_2,\ldots,\xi_n$~--- независимые одинаково распределенные
случайные величины с~общей функцией распределения
$F(x)\hm=\Phi(x\hm-a)$
(т.\,е.\ каждая случайная величина~$\xi_j$ имеет нормальное
распределение со средним~$a$ и~единичной дисперсией). Определим
новый набор случайных величин $\zeta_1,\zeta_2,\ldots,\zeta_n$
следующим образом:

\vspace*{2pt}

\noindent
\begin{equation*}
\zeta_k=\xi_1+\cdots+\xi_k\,,\enskip k=1,\ldots,n\,.
\end{equation*}

\vspace*{-2pt}

Очевидно, что для каждого $k\in\{1,\ldots,n\}$ элемент~$\zeta_k$
имеет нормальное распределение со средним $ka$ и~дисперсией~$k$.
Таким образом, выборка $\zeta_1,\zeta_2,\ldots,\zeta_n$ не является
однородной и~независимой.

На рис.~\ref{fig1} данный эффект проиллюстрирован с~помощью
гистограмм, построенных по смоделированной выборке
$\xi_1,\ldots,\xi_n$  с~$n\hm=1000$ и~$a\hm=2$ (см.\ рис.~1,\,\textit{а}) 
и~соответствующей выборке $\zeta_1,\ldots,\zeta_n$ (см.\ рис.~1,\,\textit{б}). 
Гистограмма на рис.~1,\,\textit{б} существенно скошена вправо с~малым
числом отрицательных значений. Данная картина в~точности
соответствует форме распределения, предложенного в~\cite{Gulev2012}.


Стохастический характер~$\zeta_k$  в~значительной степени
определяется суммами $\xi_1+\cdots+\,\xi_{k-1}$ и~слабо зависит от
$\xi_k$. Чем больше величина~$k$, тем меньший вклад случайной
величины~$\xi_k$ в~$\zeta_k$. Таким образом, любой анализ
статистических закономерностей~$\xi_i$, $i\hm=1,\ldots,n$,
непосредственно по выборке $\zeta_1,\zeta_2,\ldots,\zeta_n$ может
выполняться только в~рамках серь\-ез\-ных дополнительных допущений.
Кроме того, с~математической точки зрения стандартные
статистические процедуры для выборки
$\zeta_1,\zeta_2,\ldots,\zeta_n$ не применимы. Практическая ценность
результатов, основанных на таком анализе, вызывает существенные
вопросы.

Чтобы избежать влияния упомянутых проблем, возникающих при
использовании традиционных способов, следует проанализировать
преобразованный временной ряд, рассмотрев ряд \textit{приращений} 
турбулентных тепловых потоков.

\section{Оcновы применения метода скользящего разделения смесей
на~основе конечных нормальных смесей 
для~анализа временных рядов}

Для выявления структурных изменений в~наблюдаемых стохастических
процессах с~течением времени может быть успешно использован метод
скользящего разделения смесей (СРС-ме\-тод),\linebreak предложенный 
в~книге~\cite{Korolev2011}. В~статьях~[11--13] приводятся примеры
эффективной работы этого метода при анализе финансовых рынков,
трафика в~информацион\-ных системах и~процессов в~турбулентной плазме.
Ключевая особенность СРС-ме\-то\-да состоит в~представлении изменчивости
временн$\acute{\mbox{ы}}$х рядов динамической и~диффузионной компонентами.

В рамках этого метода одномерные распределения приращений основного
процесса аппрок\-симируются конечными смесями нормальных
распреде\-лений. Теоретические основы для этих моделей основаны на
утверждении, что конечные нормальные смеси могут эффективно
приблизить сдвиг-мас\-штаб\-ные нормальные смеси общего вида или
нормальные дис\-пер\-си\-он\-но-сдви\-го\-вые смеси. Они появляются в~качестве
предельных законов для распределений сумм случайного числа
независи\-мых случайных величин или неоднородных и~нестационарных
случайных блужданий (подробнее
см.~\cite{Korolev2011,Korolev2014,ZaksKorolev2013}).

Для проведения анализа изменений в~стохастическом процессе
необходимо последовательно решать задачу статистического
оценивания неизвестных параметров распределений для
движущегося отрезка выборки (некоторого заранее заданного
размера), представляющего собой подвыборку для дальнейшего
анализа. Оценивая параметры для данного отрезка (также
называемого  окном), можно получить временной ряд
этих параметров, которой может быть использован для анализа
эволюции диффузной и~динамической компонент процесса во
времени.

Предположим, что функция распределения для части данных (окна) 
в~момент времени~$t$ может быть представлена в~виде
\begin{multline}
F_t(x)={}\\
{}=\sum\limits_{i=1}^k
\fr{p_i(t)}{\sigma_i(t)\sqrt{2\pi}}\int\limits_{-\infty}^{x}\exp
\left\{-\fr{(t-a_i
(t))^2}
{2\sigma_i^2(t)}\right\}\,dt, 
\label{Mixture}
\end{multline}
где 
\begin{equation}
\sum\limits_{i=1}^{k}p_{i}(t)=1\,, \enskip
p_{i}(t)\geqslant 0 \label{Weights}
\end{equation}
(для любых $x \in \R$,  $a_i(t)\hm\in\R$, $\sigma_i(t)\hm>0$,
$i\hm=1,\ldots,k$). Модель~\eqref{Mixture} называется конечной смесью
нормальных законов. Параметры $p_{1}(t),\ldots,p_{k}(t)$,
удовлетворяющие соотношению~\eqref{Weights}, называются весами,
параметр~$k$ определяет число компонент в~смеси.

Параметры модели~\eqref{Mixture} существенным образом зависят от
времени. Данный факт проиллюстрирован на рис.~\ref{fig2},
изображающем гистограммы, построенные на различных окнах
(размер каждого окна составляет~200~элементов), и~приближающие
плотности вида~\eqref{Mixture}, параметры которых оценены на
соответствующих окнах.



Параметр  $k$ также может зависеть от времени. Тем не менее 
и~с~точки зрения выбора эффективных
 настроек метода, и~с~точки зрения
интерпретации результатов желательно задать предельно допустимое
значение~$k$~заранее. Как правило, типичное чис\-ло компонент не
превышает~6 или~7~--- именно  такое количество обеспечивает
качественное приближение любой модели. При оценивании параметров
модели~\eqref{Mixture} в~режиме скользящего окна не-\linebreak\vspace*{-12pt}

\pagebreak

\end{multicols}

\begin{figure} %fig2
\vspace*{1pt}
 \begin{center}
 \mbox{%
 \epsfxsize=162.699mm
 \epsfbox{kor-2.eps}
 }
 \end{center}
 \vspace*{-12pt}
\Caption{Гистограммы, построенные для различных
положений окна, с~подгонкой плотностями типа конечных смесей
нормальных законов
\label{fig2}}
\vspace*{-3pt}
\end{figure}

\begin{multicols}{2}

\noindent
которые веса
могут быть достаточно близки к нулевому значению. Это влечет
необходимость отбрасывания из рассмотрения соответствующей
компоненты и~уменьшение общего числа компонент в~смеси.

Параметры $a_{1}(t),\ldots,a_{k}(t)$ ассоциированы с~динамической
компонентой внутренней изменчи\-вости процесса, параметры
$\sigma_{1}(t),\ldots,\sigma_{k}(t)$  связаны с~диффузионной
компонентой~\cite{Korolev2011}. Если переменная~$Z_t$ представляет
собой случайную величину с~функцией распределения, определяемой
выражением~\eqref{Mixture}, то ее дисперсия может быть
представлена в~виде суммы двух компонент:
\begin{equation}
\label{VolDec}
{\sf D}Z_t=\sum\limits_{i=1}^{k}p_i(t)\left[a_i(t)-\overline
a(t)\right]^2+\sum\limits_{i=1}^{k}p_i(t)\sigma_i^2(t)\,,
\end{equation}
где
\begin{equation*}
\overline a(t)=\sum\limits_{i=1}^{k}p_i(t)a_i(t)\,.
\end{equation*}

Первый член в~правой части выражения~\eqref{VolDec} зависит только
от весов $p_i(t)$ и~математических ожиданий $a_i(t)$ компонент
смеси~\eqref{Mixture}. Так как~$Z_t$ представляет собой приращения
исходного процесса, то $a_i(t)$ является математическим ожиданием
значения приращения, т.\,е.\ <<трендовой>> компонен-\linebreak той. 

Таким
образом, первое слагаемое определяет часть общей дисперсии
(изменчивости), обуслов\-лен\-ной локальными элементарными
трендами. Его обычно называют \textit{динамической} компонентой
дисперсии. Кроме того, $a_i(t)$ является математическим ожиданием
случайной величины, распределение которой %\linebreak 
соответствует $i$-й
компоненте\linebreak смеси~\eqref{Mixture}. По построению эта случайная
величина является приращением исходного процесса на {\it
единичном} временн$\acute{\mbox{о}}$м интервале, т.\,е.\ $a_i(t)$ представляет
собой среднюю  скорость изменения  \mbox{$i$-й} компоненты.
%
 Таким
образом, множество пар $(a_1(t),p_1(t)),\ldots,(a_k(t),p_k(t))$
определяет распределение скоростей изменения локальных трендов
в~момент времени~$t$.
{ %\looseness=1

}

Второе слагаемое в~правой части выражения~\eqref{VolDec} зависит
только от весов $p_i(t)$ и~дисперсий $\sigma_i^2(t)$ компонентов 
и~представляет собой чисто стохастическую \textit{диффузионную}
составляющую общей дис\-персии.

\section{Оценивание неизвестных параметров распределений}

Для оценивания параметров модели~\eqref{Mixture} на каж\-дом окне
можно воспользоваться классическим EM (expectation-maximization) ал\-го\-рит\-мом~---
итерационным численным методом для максимизации функции
правдоподобия нескольких переменных. Данный метод был
предложен в~\cite{Dempster1977} и~подробно описан
в~\cite{Korolev2011}. Хотя существует значительное чис\-ло
всевозможных модификаций этого алгоритма, классический
EM-ал\-го\-ритм остается весьма надежным инструментом для поиска
оценок параметров смеси конечных нормальных
законов~\eqref{Mixture}.

При этом данный алгоритм не лишен и~ряда существенных недостатков.
Вероятно, основной из них~--- его неустойчивость по отношению 
к~выбору начального приближения. Функция правдоподобия для модели
конечной нормальной смеси  имеет существенно негладкую поверхность 
с~пиками, впадинами или разрывами. Поэтому, будучи <<жадным>>
алгоритмом, EM-ал\-го\-ритм сходится к~тому локальному максимуму,
который является ближайшим к~начальному приближению. Стандартные
способы преодоления данной особенности представлены ниже:
\begin{itemize}
\item начальное приближение следует выбирать случайным образом;
\item нужно выбрать несколько начальных приближений, а~оценку на
окне определить как усреднение результатов нескольких независимых
запусков EM-ал\-го\-ритма;
\item  следует выбрать несколько начальных приближений, а~затем 
в~качестве оценки на окне выбрать набор параметров,
максимизирующих значение функции правдоподобия.
\end{itemize}

При использовании EM-алгоритма в~скользящем режиме (как это
сделано в~данной работе) значительное внимание должно быть
уделено визуализации результа\-тов. Для удобной интерпретации
результа\-тов полученные кривые, изобража\-ющие эволюцию во
времени параметров $p_i(t)$, $a_i(t)$, $\sigma_i(t)$, $i\hm=1,\ldots,k$,
должны быть гладкими. В~частности, это может быть достигнуто 
в~случае использования оценок EM-ал\-го\-рит\-ма на текущем окне 
в~качестве значений начальных приближений для следующего окна.
Однако такой способ увеличивает риск выбора локального
экстремума вместо глобального.

\begin{figure*} %fig3
\vspace*{1pt}
 \begin{center}
 \mbox{%
 \epsfxsize=148.824mm
 \epsfbox{kor-3.eps}
 }
 \end{center}
 \vspace*{-11pt}
\Caption{Графики, изображающие оценки параметров,
полученные в~трех режимах: (\textit{а})~ЕМ-ал\-го\-ритм со случайным выбором начальных приближений;
(\textit{б})~<<классический>> ЕМ-ал\-го\-ритм со случайным выбором начальных значений для
весов; (\textit{в})~ЕМ-ал\-го\-ритм с~выбором случайного начального приближения
на каж\-дом окне
\label{fig3}}
\end{figure*}

Результаты, полученные для трех вышеуказанных способов выбора
начального приближения, представлены на рис.~\ref{fig3}. 

На рис.~3,\,\textit{а}
(верхний график) представлена эволюция во времени параметров $a_i(t)$ для
локальных трендов, оцененных с~по\-мощью EM-ал\-го\-рит\-ма со случайным
выбором начальных приближений. На каждом окне расчеты с~помощью
EM-ал\-го\-рит\-ма проводятся пять раз, при этом начальные значения
выбираются случайным образом для каждого запуска. Результаты
усредняются по всему набору значений, полученному при работе
алгоритмов на данном окне. Веса компонентов изоб\-ра\-же\-ны 
в~соответствии с~цветовой шкалой, изоб\-ра\-жен\-ной справа вне графиков.
Рисунок~3,\,\textit{а} (нижний график) демонстрирует изменение во времени локальных
параметров диффузии $\sigma_i^2(t)$, оценива\-емых с~по\-мощью описанной
версии EM-ал\-го\-рит\-ма. 

На рис.~3,\,\textit{б}
(верхний график) продемонстрирована эволюция
во времени параметров $a_i(t)$ для локальных трендов, оценива\-емых 
с~по\-мощью <<классического>> EM-ал\-го\-рит\-ма со случайным выбором
начальных значений для весов. На каждом окне начальные приближения
для параметров сдвига и~масштаба выбираются единым образом (как
среднее и~выборочная дисперсия для текущего окна соответственно). На
рис.~3,\,\textit{б} (нижний график) приведены изменения во времени параметров
локальных диффузий $\sigma_i^2(t)$, оцениваемых с~по\-мощью
<<классического>> EM-ал\-го\-рит\-ма. 

На рис.~3,\,\textit{в} (верхний график) приведены изменения
во времени параметров $a_i(t)$ для локальных трендов, оцененных 
с~по\-мощью EM-ал\-го\-рит\-ма с~выбором случайного начального приближения на
каждом окне. Для каждого положения окна EM-ал\-го\-ритм запускается пять
раз, начальные значения выбираются каждый раз случайным образом. При
этом в~качестве оценки выбирается набор параметров, соответствующий
функции правдоподобия с~наибольшим значением среди всех запусков.
Рисунок~3,\,\textit{в} (нижний график) демонстрирует изменение во времени локальных
параметров диффузии $\sigma_i^2(t)$, оцениваемых с~по\-мощью описанной
версии EM-ал\-го\-ритма.

Можно сделать вывод, что третья версия EM-ал\-го\-рит\-ма со случайным
выбором начальных приближений и~поиском набора,
соответствующего максимальному значению функции правдоподобия
на каждом окне, дает наиболее четкие результаты.

\begin{figure*} %fig4
\vspace*{1pt}
 \begin{center}
 \mbox{%
 \epsfxsize=106.544mm
 \epsfbox{kor-4.eps}
 }
 \end{center}
 \vspace*{-12pt}
\Caption{Квантили вероятностного распределения приращений процесса потоков тепла:
\textit{1}~--- 0,05; \textit{2}~--- 0,1; \textit{3}~--- 0,2; \textit{4}~--- 0,3;
\textit{5}~--- 0,4, \textit{6}~--- 0,5; \textit{7}~--- 0,6;
\textit{8}~--- 0,7; \textit{9}~--- 0,8; \textit{10}~--- 0,9; \textit{11}~--- 0,95
\label{fig4}}
\vspace*{-3pt}
\end{figure*}
\begin{figure*}[b] %fig5
\vspace*{-3pt}
 \begin{center}
 \mbox{%
 \epsfxsize=166mm %.439mm
 \epsfbox{kor-5.eps}
 }
 \end{center}
 \vspace*{-12pt}
\Caption{Моментные характеристики вероятностного
распределения приращений тепловых потоков: (\textit{а})~математическое ожидание;
(\textit{б})~дисперсия (\textit{1}~--- дисперсия аппроксимирующей смеси нормальных
распределений; \textit{2}~-- 
динамическая компонента дисперсии;
\textit{3}~--- диффузионная компонента дисперсии);
(\textit{в})~асимметрия; (\textit{г})~эксцесс
\label{fig5} }
\end{figure*}


Данный алгоритм был использован для анализа временн$\acute{\mbox{о}}$й
изменчивости параметров распределения приращений тепловых
потоков. Полученные квантили распределения представлены на
рис.~\ref{fig4} изотопными линиями (горизонтальная ось времени
соответствует периоду длительностью око-\linebreak\vspace*{-12pt}

\pagebreak

\noindent
ло~3,5~лет). На графиках
явно выделяется сезонная периодичность.

\begin{figure*} %fig6
\vspace*{1pt}
 \begin{center}
 \mbox{%
 \epsfxsize=144.858mm
 \epsfbox{kor-6.eps}
 }
 \end{center}
 \vspace*{-11pt}
\Caption{Определение доли экстремальных наблюдений:
(\textit{а})~вес; (\textit{б})~СКО \label{fig6}
($\sigma_1\hm\leqslant\sigma_2$)}
\end{figure*}

На рис.~\ref{fig5} представлена эволюция моментных характеристик
распределения вероятностей приращений процесса тепловых потоков.
Видно, что математическое ожидание заметно колеблется во
времени с~изменяющейся амплитудой. Кроме того, для каждого
периода амплитуда меньше для периода сезонного увеличения
общего среднего по сравнению с~амплитудой колебаний для периода
сезонного снижения общего среднего значения. Хорошо
прослеживается сезонный характер изменения дисперсии.
Достаточно интересным представляется и~тот факт, что
вклад в~общую дис\-пер\-сию чисто стохастической диффузионной
компоненты дис\-пер\-сии (см.\ второе слагаемое в~правой части
соотношения~\eqref{VolDec})~--- зеленая кривая на правом верхнем
графике~---  больше, чем динамической составляющей (см.\ первый
член в~правой части соотношения~\eqref{VolDec}), изображенной
синей кривой. Можно отметить, что правый хвост распределения
приращений тяжелее левого. Еще одно интересное наблюдение
заключается в~том, что эксцесс этого распределения максимален во
время периода <<спокой\-ствия>>.
{\looseness=1

}


\section{Определение доли экстремальных наблюдений в~данных}

На основании рассмотренной выше методологии в~настоящем
разделе обсудим возможный способ решения задачи определения
доли экстремальных наблюдений в~исходной выборке, не
\mbox{тре\-бу\-ющий} привлечения теории экстремальных значений.

Предположим, что в~модели~\eqref{Mixture} параметр~$k$ выбирается
равным двум (с одной стороны, для ускорения вычислений, с~другой~---  
для получения более контрастной общей картины). На
рис.~\ref{fig6} пред\-став\-ле\-ны два графика для рассмотренного ранее
ряда. На верхнем изображена эволюция весов каждой из компонент, 
а~на нижнем~--- соответствующие им среднеквадратические отклонения (СКО).
При этом справедливо соотношение $\sigma_1\hm\leqslant\sigma_2$
(отметим, что один параметр превосходит другой минимум в~2~раза
для каждого положения скользящего окна), а~цвета значков на обоих
графиках соответствуют друг другу: серым обозначены графики для
компоненты с~наибольшей дисперсией на каждом окне, черным~--- 
с~наименьшей.



Наблюдения, соответствующие компоненте с~наибольшей
дисперсией, могут быть проинтерпретированы как
экстремальные (относительно наблюдений другой компоненты).
Стоит отметить, что веса этой компоненты лежат в~диапазоне
0,2--0,4, т.\,е.\ можно сказать, что указанные наблюдения
составляют примерно треть от общего числа. Отметим, что указанная
картина сохраняется для всех сезонов, при этом характер
экстремальных наблюдений меняется.

\section{Выводы}

В данной работе метод скользящего разделения смесей был
использован для анализа статистических закономерностей во
временн$\acute{\mbox{о}}$й эволюции тепловых потоков. Вычисления проводились 
с~по\-мощью нескольких модификаций EM-алгоритма, из которых 
в~качестве наиболее подходящей была выбрана специальная версия,
основанная на максимизации функции правдоподобия в~классе
конечных смесей нормальных законов. Было показано, что 
в~стохастическом поведении эволюции тепловых потоков основная
компонента с~небольшой дисперсией может сопровождаться
стохастически развивающимися и~исчезающими компонентами 
с~большой дисперсией. Также отмечен ряд закономерностей во
временн$\acute{\mbox{о}}$й изменчивости моментных характеристик приращений
значений процесса теп\-ло\-вых потоков. На основании упорядочивания
весов и~дисперсий предложен метод определения доли
экстремальных наблюдений в~рас\-смат\-ри\-ва\-емых временн$\acute{\mbox{ы}}$х рядах.

{\small\frenchspacing
 {%\baselineskip=10.8pt
 \addcontentsline{toc}{section}{References}
 \begin{thebibliography}{99}
\bibitem{Berry2009} 
\Au{Berry~D.\,I., Kent~E.\,C.} A~new air--sea
interaction gridded dataset from ICOADS with uncertainty
estimates~// Bull. Am. Meteorol. Soc., 2009.
Vol.~90. No.\,5. P.~645--656.

\bibitem{Yu2007a} 
\Au{Yu~L.} Global variations in oceanic evaporation
(1958--2005): The role of the changing wind speed~// 
J.~Climate, 2007. Vol.~20. P.~5376--5390.

\bibitem{Yu2007b} 
\Au{Yu~L., Weller~R.\,A.} Objectively analyzed air--sea
heat fluxes for the global ice-free oceans (1981--2005)~// Bull.
Am. Meteorol. Soc., 2007. Vol.~88. P.~527--539.

\bibitem{Josey2001} 
\Au{Josey~S.\,A.} A~comparison of ECMWF,
NCEP-NCAR and SOC surface heat fluxes with moored buoy measurements
in the subduction region of the Northeast Atlantic~// J.~Climate, 2001. Vol.~14. P.~1780--1789.

\bibitem{Bouras2006} 
\Au{Bouras~D.} Comparison of five
satellite-derived latent heat flux products to moored buoy data~//
J.~Climate, 2006. Vol.~19. P.~6291--6313.

\bibitem{Gulev2007a} 
\Au{Gulev~S.\,K., Jung~T., Ruprecht~E.}
Estimation of the impact of sampling errors in the VOS observations
on air--sea fluxes. Part I. Uncertainties in climate means~// J.~Climate, 2007. Vol.~20. P.~279--301.

\bibitem{Gulev2007b}  
\Au{Gulev~S.\,K., Jung~T., Ruprecht~E.} Estimation
of the impact of sampling errors in the VOS observations on air--sea
fluxes. Part~II. Impact on trends and interannual variability~//
J.~Climate, 2007. Vol.~20. P.~302--315.

\bibitem{Gulev2012} 
\Au{Gulev~S.\,K., Belyaev~K.\,P.} Probability
distribution characteristics for surface air--sea turbulent heat
fluxes over the global ocean~// J.~Climate, 2012. Vol.~25.
No.\,1. P.~184--206.

\bibitem{Gulev2013} 
\Au{Gulev~S.\,K., Latif~M., Keenlyside~N.,
Park~W., Koltermann~K.\,P.} North Atlantic Ocean control on surface
heat flux on multidecadal timescales~// Nature, 2013. Vol.~499.
P.~464--467.

\bibitem{Korolev2011} 
\Au{Королев~В.\,Ю.}
Ве\-ро\-ят\-но\-ст\-но-ста\-ти\-сти\-че\-ские методы декомпозиции волатильности
хаотических процессов.~--- М.: Изд-во Моск. ун-та, 2011. 512~с.

\bibitem{Gorshenin2008} 
\Au{Горшенин~А.\,К., Королев~В.\,Ю.,
Турсунбаев~А.\,М.} Медианные  модификации EM- и~SEM-ал\-го\-рит\-мов
для разделения смесей вероятностных распределений и~их
применение к декомпозиции волатильности финансовых временных
рядов~// Информатика и~её применения, 2008. Т.~2. Вып.~4. C.~12--47.

\bibitem{Gorshenin2013Alesund1} 
\Au{Gorshenin~A., Korolev~V.}
Modeling of statistical fluctuations of information flows by
mixtures of gamma distributions~// 27th European
Conference on Modelling and Simulation Proceedings.~--- Dudweiler, Germany:
Digitaldruck Pirrot GmbHP, 2013. P.~569--572.

\bibitem{Gorshenin2014} 
\Au{Горшенин~А.\,К.} Информационная
технология исследования тонкой структуры хаотических процессов 
в~плазме с~помощью анализа спектров~// Системы и~средства
информатики, 2014. Т.~24. Вып.~1. С.~116--127.

\bibitem{Korolev2014} 
\Au{Королев~В.\,Ю.} Обобщенные
гиперболические распределения как предельные для случайных сумм~// 
Теория вероятностей и~ее применения, 2013. T.~58. Вып.~1.
С.~117--132.

\bibitem{ZaksKorolev2013}  
\Au{Королев~В.\,Ю., Закс~Л.\,М.}
Обобщенные дисперсионные гам\-ма-рас\-пре\-де\-ле\-ния как предельные для
случайных сумм~// Информатика и~её применения, 2013. Т.~7. Вып.~1.
С.~105--115.

\bibitem{Dempster1977} 
\Au{Dempster~A., Laird~N., Rubin~D.}
Maximum likelihood estimation from incompleted data~// J.~Roy. 
Stat. Soc. B, 1977. Vol.~39. No.\,1. P.~1--38.
 \end{thebibliography}

 }
 }

\end{multicols}

\vspace*{-3pt}

\hfill{\small\textit{Поступила в~редакцию 04.09.15}}

%\vspace*{8pt}

\newpage

\vspace*{-24pt}

%\hrule

%\vspace*{2pt}

%\hrule

%\vspace*{8pt}

\def\tit{STATISTICAL MODELING OF~AIR--SEA TURBULENT HEAT FLUXES BY~THE~METHOD 
OF~MOVING SEPARATION OF~FINITE NORMAL MIXTURES}

\def\titkol{Statistical modeling of~air--sea turbulent heat fluxes by~the~method 
of~moving separation of~finite normal mixtures}

\def\aut{V.\,Yu.~Korolev$^{1,2}$, A.\,K.~Gorshenin$^{2,3}$, S.\,K.~Gulev$^{4,5,6}$, 
and~K.\,P.~Belyaev$^{5,7}$}

\def\autkol{V.\,Yu.~Korolev, A.\,K.~Gorshenin, S.\,K.~Gulev, 
and~K.\,P.~Belyaev}

\titel{\tit}{\aut}{\autkol}{\titkol}

\vspace*{-9pt}


\noindent
$^1$Faculty of Computational 
Mathematics and Cybernetics, M.\,V.~Lomonosov Moscow State University,\linebreak 
$\hphantom{^1}$1-52 Leninskiye Gory, GSP-1, Moscow 119991, Russian Federation

\noindent
$^2$Institute of Informatics Problems, Federal Research Center ``Computer 
Sciences and Control'' of the Russian\linebreak 
$\hphantom{^1}$Academy of Sciences, 44-2 Vavilov Str., Moscow 119333, Russian Federation

\noindent
$^3$Moscow State University of Information Technologies, Radioengineering, and
Electronics, 78~Vernadskogo Ave.,\linebreak
$\hphantom{^1}$Moscow 119454,  Russian Federation

\noindent
$^4$Faculty of Geography, M.\,V.~Lomonosov Moscow State University, 1~Leninskiye Gory, GSP-1,
Moscow 119991,\linebreak
$\hphantom{^1}$Russian Federation

\noindent
$^5$P.\,P.~Shirshov Institute 
of Oceanology, 36~Nakhimovski Prosp., Moscow 117997, Russian Federation

\noindent
$^6$University of Kiel, Christian-Albrechts-Universit$\ddot{\mbox{a}}$t 
zu Kiel, 4~Christian-Albrechts-Platz, Kiel 24098, Germany

\noindent
$^7$Federal University of Bahia, Rua Adhemar de Barros, no 500, Ondina, 40.710-110, 
Salvador, Bahia, Brazil


\def\leftfootline{\small{\textbf{\thepage}
\hfill INFORMATIKA I EE PRIMENENIYA~--- INFORMATICS AND
APPLICATIONS\ \ \ 2015\ \ \ volume~9\ \ \ issue\ 4}
}%
 \def\rightfootline{\small{INFORMATIKA I EE PRIMENENIYA~---
INFORMATICS AND APPLICATIONS\ \ \ 2015\ \ \ volume~9\ \ \ issue\ 4
\hfill \textbf{\thepage}}}

\vspace*{3pt}



\Abste{The method of moving separation of mixtures is
applied to the problem of statistical modeling of regularities in
explicit and latent turbulent heat fluxes. The six-hour
observations in the Atlantic region (NCEP-NCAR, 1948--2008) are
used as initial data. The basic approximate mathematical model 
is a~finite normal mixture with parameters depending on time. The
methodology of moving separation of mixtures allows one to analyze the
regularities in the variation of parameters and to capture the
variability which can be associated with the trend as well as the
irregular variation. An approach is proposed to the determination of
the proportion of extreme observations in the original sample.}

\KWE{finite normal mixtures; moving separation of
mixtures; probabilistic models; data mining}


\DOI{10.14357/19922264150401}

\vspace*{-18pt}

\Ack
    \noindent
The work was supported by the President of the Russian Federation
(grant MK-4103.2014.9) and the Russian Foundation for Basic
Research (projects
15-07-02652 and 15-37-20851).



%\vspace*{3pt}

  \begin{multicols}{2}

\renewcommand{\bibname}{\protect\rmfamily References}
%\renewcommand{\bibname}{\large\protect\rm References}

{\small\frenchspacing
 {%\baselineskip=10.8pt
 \addcontentsline{toc}{section}{References}
 \begin{thebibliography}{99}
\bibitem{1-kg-1}
\Aue{Berry, D.\,I., and E.\,C.~Kent}. 2009. A~new air--sea interaction gridded
dataset from ICOADS with uncertainty estimates. \textit{Bull.
Am. Meteorol. Soc.}  90(5):645--656.

\bibitem{2-kg-1}
\Aue{Yu,~L.} 2007. Global variations in oceanic evaporation (1958--2005):
The role of the changing wind speed. \textit{J.~Climate}
20:5376--5390.

\bibitem{3-kg-1}
\Aue{Yu,~L., and ~R.\,A.~Weller}. 2007. Objectively analyzed air--sea heat
fluxes for the global ice-free oceans (1981--2005). \textit{Bull.
Am. Meteorol. Soc.} 88:527--539.

\bibitem{4-kg-1}
\Aue{Josey,~S.\,A.} 2001. A~comparison of ECMWF, NCEP-NCAR and
SOC surface heat fluxes with moored buoy measurements in the subduction
region of the Northeast Atlantic. \textit{J.~Climate} 14:1780--1789.

\bibitem{5-kg-1}
\Aue{Bouras,~D.} 2006. Comparison of five satellite-derived latent heat flux
products to moored buoy data. \textit{J.~Climate} 19:6291--6313.

\bibitem{6-kg-1}
\Aue{Gulev,~S.\,K., T.~Jung, and E.~Ruprecht}. 2007. Estimation of the impact of
sampling errors in the VOS observations on air--sea fluxes. Part~I.
Uncertainties in climate means. \textit{J.~Climate} 20:279--301.

\bibitem{7-kg-1}
\Aue{Gulev,~S.\,K., T.~Jung, and E.~Ruprecht}. 2007. Estimation of the impact of
sampling errors in the VOS observations on air--sea fluxes. Part~II. Impact
on trends and interannual variability. \textit{J.~Climate} 20:302--315.

\bibitem{8-kg-1}
\Aue{Gulev,~S.\,K., and K.\,P.~Belyaev}. 2012. Probability distribution
characteristics for surface air--sea turbulent heat fluxes over the global
ocean. \textit{J.~Climate} 25(1):184--206.

\bibitem{9-kg-1}
\Aue{Gulev,~S.\,K., M.~Latif, N.~Keenlyside, W.~Park, and
K.\,P.~Koltermann}. 2013. North Atlantic Ocean control on surface heat flux
on multidecadal timescales. \textit{Nature} 499:464--467.

\bibitem{10-kg-1}
\Aue{Korolev, V.\,Yu.} 2011. \textit{Veroyatnostno-statisticheskie metody
dekompozitsii volatil'nosti khaoticheskikh protsessov} [Probabilistic and
statistical methods of decomposition of volatility of chaotic processes].
Moscow: Moscow University Publs. 512~p.

\bibitem{11-kg-1}
\Aue{Gorshenin, A.\,K.,  V.\,Yu.~Korolev, D.\,V.~Malakhov and
A.\,M.~Tursunbaev}. 2008.   Mediannye modifikatsii EM- i~SEM-algoritmov
dlya razdeleniya smesey ve\-ro\-yat\-no\-st\-nykh raspredeleniy i~ikh primenenie 
k~dekompozitsii volatil'nosti finansovykh vremennykh ryadov [Median
modification of EM- and SEM-algorithms for separation of mixtures of
probability distributions and their application to the decomposition of volatility
of financial time series]. \textit{Informatika i~ee Primeneniya}~---
\textit{Inform. Appl.} 2(4):12--47.

\bibitem{12-kg-1}
\Aue{Gorshenin, A.\,K., and V.\,Yu.~Korolev}. 2013.Modeling of statistical
fluctuations of information flows by mixtures of gamma distributions.
\textit{27th European Conference on Modelling and
Simulation Proceedings}. Dudweiler, Germany:
Digitaldruck Pirrot GmbHP. 569--572.

\bibitem{13-kg-1}
\Aue{Gorshenin, A.\,K.} 2014. Informatsionnaya tekhnologiya issledovaniya
tonkoy struktury khaoticheskikh protsessov v~plazme s~po\-moshch'yu analiza
spektrov [The information technology to research the fine structure of
chaotic processes in plasma by the analysis of the spectra]. \textit{Sistemy
i~Sredstva Informatiki}~--- \textit{Systems and Means of Informatics}
24(1):116--127.

\bibitem{14-kg-1}
\Aue{Korolev, V.\,Yu.} 2013. Obobshchennye giperbolicheskie raspredeleniya
kak predel'nye dlya sluchaynykh summ [Generalized hyperbolic laws as limit
distributions for random sums]. \textit{Teoriya Veroyatnostey i~ee
Primeneniya} [{Theory of Probability and Its Applications}]
58(1):117--132.

\bibitem{15-kg-1}
\Aue{Korolev, V.\,Yu., and L.\,M.~Zaks}. 2013. Obobshchennye dispersionnye
gamma-raspredeleniya kak predel'nye dlya sluchaynykh summ
[Variance-generalized-gamma-distributions as limit laws for random sums].
\textit{Informatika i~ee Primeneniya}~---\textit{Inform. Appl.}
7(1):105--115.

\bibitem{16-kg-1}
\Aue{Dempster, A., N.~Laird,  and D.~Rubin}. 1977. Maximum likelihood
estimation from incompleted data. \textit{J.~Roy. Stat.
Soc. B} 39(1):1--38.
\end{thebibliography}

 }
 }

\end{multicols}

\vspace*{-3pt}

\hfill{\small\textit{Received September 04, 2015}}

\Contr

\noindent
\textbf{Korolev Victor Yu.} (b.\ 1954)~---
Doctor of Science in physics and mathematics, professor, 
Head of the Department of Mathematical Statistics, Faculty of Computational 
Mathematics and Cybernetics, M.\,V.~Lomonosov Moscow State University, 
1-52 Leninskiye Gory, GSP-1, Moscow 119991, Russian Federation; 
leading scientist, Institute of Informatics Problems, Federal Research Center 
``Computer Science and Control'' of the Russian Academy of Sciences, 
44-2 Vavilov Str., Moscow 119333, Russian Federation; vkorolev@cs.msu.su


\vspace*{3pt}

\noindent
\textbf{Gorshenin Andrey K.}  (b.\ 1986)~---
Candidate of Science (PhD) in physics and mathematics, senior scientist, 
Federal Research Center 
``Computer Science and Control'' of the Russian Academy of Sciences, 
44-2 Vavilov Str., Moscow 119333, Russian Federation; associate professor, 
Moscow State University of Information Technologies, Radioengineering,
and Electronics, 78~Vernadskogo Ave., Moscow 119454, 
Russian Federation;  a.k.gorshenin@gmail.com


\vspace*{3pt}

\noindent
\textbf{Gulev Sergey K.} (b.\ 1958)~---
Corresponding Member of the Russian Academy of Sciences, Doctor of Science 
in physics and mathematics, Head of Laboratory, P.\,P.~Shirshov Institute 
of Oceanology, 36~Nakhimovski Prosp., Moscow 117997, Russian Federation; 
professor, Department of Oceanology, Faculty of Geography,
M.\,V.~Lomonosov Moscow State University, 1~Leninskiye Gory, GSP-1,
Moscow 119991, Russian Federation; 
professor, University of Kiel, Christian-Albrechts-Universit$\ddot{\mbox{a}}$t 
zu Kiel, 4~Christian-Albrechts-Platz, Kiel 24098, Germany; gul@sail.msk.ru

\vspace*{3pt}

\noindent
\textbf{Belyaev Konstantin P.} (b.\ 1955)~---
Doctor of Science in physics and mathematics, leading scientist, 
P.\,P.~Shirshov Institute of Oceanology, 36~Nakhimovski Prosp., 
Moscow 117997, Russian Federation; visiting professor, 
Federal University of Bahia, Rua Adhemar de Barros, no 500, Ondina, 40.710-110, 
Salvador, Bahia, Brazil; kosbel55@gmail.com
\label{end\stat}


\renewcommand{\bibname}{\protect\rm Литература}