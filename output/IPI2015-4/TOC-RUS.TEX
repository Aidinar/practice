%Том 9   Выпуск 1-4   Год 2015

\def\stat{cont}
{%\hrule\par
%\vskip 7pt % 7pt
\raggedleft\Large \bf%\baselineskip=3.2ex
А\,В\,Т\,О\,Р\,С\,К\,И\,Й\ \ У\,К\,А\,З\,А\,Т\,Е\,Л\,Ь\ \ З\,А\ \ 2\,0\,1\,5 г. \vskip 17pt
    \hrule
    \par
\vskip 21pt plus 6pt minus 3pt }

\label{st\stat}

\def\tit{\ }

\def\aut{\ }
\def\auf{\ }

\def\leftkol{\ } % ENGLISH ABSTRACTS}

\def\rightkol{\ } %АВТОРСКИЙ УКАЗАТЕЛЬ ЗА 2015 г.} %ENGLISH ABSTRACTS}

\titele{\tit}{\aut}{\auf}{\leftkol}{\rightkol}

\vspace*{-12pt}
\vspace*{-18pt}

{\tabcolsep=3pt
\begin{tabular}{p{388pt}rr}
&\textbf{Выпуск} & \textbf{Стр.}\\[6pt]
\textbf{Абрамов М.\,О., Катаев М.\,Ю.} Влияние морфологических операций
на распознавание\linebreak
\\[-12pt]
\hspace*{21pt}фигуры движущегося человека по потоку
изображений&3&124\\
\hangindent=21pt\noindent\textbf{Агаларов Я.\,М.} Пороговая стратегия ограничения доступа
к ресурсам в сис\-те\-ме массового обслуживания $M/D/1$ с функцией штрафов за
несвоевременное обслужива-\linebreak
\\[-12pt]
\hspace*{21pt}ние заявок&3&55\\
\textbf{Беляев К.\,П.} см.\ Королев В.\,Ю.&&\\
\textbf{Беляев К.\,П., Тучкова Н.\,П.} Предельные распределения для
характеристик при усво-\linebreak
\\[-12pt]
\hspace*{21pt}ении данных наблюдений в стационарном режиме&2&50\\
\textbf{Беляев Ю.\,К., Кристрём Б.} Анализ обзорных обследований,
содержащих цензуриро-\linebreak
\\[-12pt]
\hspace*{21pt}ванные данные в округленных интервалах&3&2\\
\textbf{Березин С.\,В., Заяц О.\,И.} Применение уравнения
Пугачёва--Свешникова к решению\linebreak
\\[-12pt]
\hspace*{21pt}задачи Бакстера о длительности выбросов&2&39\\
\textbf{Боровский П.} см.\ Френкель С.&&\\
\textbf{Бронштейн Е.\,М., Вагапова Д.\,М.} Сравнительный анализ
применения эвристического\linebreak
\\[-12pt]
\hspace*{21pt}и метаэвристического алгоритмов к задаче о
школьном автобусе&2&56\\
\textbf{Вагапова Д.\,М.} см.\ Бронштейн Е.\,М.&&\\
\textbf{Васильев Н.\,С.} Коалиционно устойчивые эффективные равновесия
в моделях коллек-\linebreak
\\[-12pt]
\hspace*{21pt}тивного поведения с обменом информацией&2&2\\
\textbf{Вихрова О.\,Г., Самуйлов К.\,Е., Сопин Э.\,С., Шоргин С.\,Я.}
К~анализу показателей\linebreak
\\[-12pt]
\hspace*{21pt}качества обслуживания в~современных беспроводных
сетях&4&48\\
\textbf{Гайдамака Ю.\,В., Самуйлов А.\,К.} Метод расчета характеристик
интерференции двух\linebreak
\\[-12pt]
\hspace*{21pt}взаимодействующих устройств в беспроводной
гетерогенной сети&1&9\\
\hangindent=21pt\noindent\textbf{Галина И.\,В., Козеренко Е.\,Б., Морозова Ю.\,И., Сомин Н.\,В.,
Шарнин М.\,М.} Ассоциативные портреты предметной области~--- инструмент
автоматизированного построения сис\-тем big data для извлечения знаний:
теория, методика, визуализа-\linebreak
\\[-12pt]
\hspace*{21pt}ция, возможное применение&2&93\\
\textbf{Горбунова А.\,В., Зарядов И.\,С., Матюшенко С.\,И., Самуйлов
К.\,Е., Шоргин С.\,Я.} Ап-\linebreak
\\[-12pt]
\hspace*{21pt}проксимация времени отклика сис\-те\-мы облачных
вычислений&3& 32\\
\textbf{Горшенин А.\,К.} см.\ Королев В.\,Ю.&&\\
\textbf{Грушо А.\,А., Грушо Н.\,А., Тимонина Е.\,Е.} Оценки скорости
передачи информации\linebreak
\\[-12pt]
\hspace*{21pt}и~пропускной способности в~скрытых каналах
с~метками&4&85\\
\textbf{Грушо А.\,А., Забежайло М.\,И., Зацаринный А.\,А.} Контроль
и~управление информаци-\linebreak
\\[-12pt]
\hspace*{21pt}онными потоками в~облачной среде&4&91\\
\textbf{Грушо Н.\,А.} см.\ Грушо А.\,А.&&\\
\textbf{Гулев С.\,К.} см.\ Королев В.\,Ю.&&\\
\textbf{Долев Ш., Коган-Садецкая М.} Эвристические сертификаты
посредством приближений&1&15\\
\textbf{Жаворонкова Ю.\,В., Кудрявцев А.\,А., Шоргин С.\,Я.}
Байесовская рекуррентная модель\linebreak
\\[-12pt]
\hspace*{21pt}роста надежности: бета-равномерное
распределение параметров&1&98\\
\textbf{Забежайло М.\,И.} см.\ Грушо А.\,А.&&\\
\textbf{Зарядов И.\,С.} см.\ Горбунова А.\,В.&&\\
\textbf{Зацаринный А.\,А.} см.\ Грушо А.\,А.&&\\
\textbf{Зацман И.\,М.} Процессы целенаправленной генерации и развития
кросс-языковых\linebreak
\\[-12pt]
\hspace*{21pt}экспертных знаний: семиотические основания
моделирования&3&106\\
\textbf{Зацман И.\,М.} см.\ Минин В.\,А.&&\\
\textbf{Заяц О.\,И.} см.\ Березин С.\,В.&&\\
\textbf{Змеев Д.\,Н., Климов А.\,В., Левченко Н.\,Н., Окунев А.\,С.,
Стемпковский А.\,Л.} Потоковая\linebreak
\\[-12pt]
\hspace*{21pt}модель вычислений как парадигма
программирования будущего&4&29\\
\end{tabular}
}

\pagebreak

\def\leftkol{АВТОРСКИЙ УКАЗАТЕЛЬ ЗА 2015 г.} % ENGLISH ABSTRACTS}

\def\rightkol{АВТОРСКИЙ УКАЗАТЕЛЬ ЗА 2015 г.} %ENGLISH ABSTRACTS}

%\thispagestyle{myheadings}
\def\leftfootline{\small{\textbf{\thepage}
\hfill ИНФОРМАТИКА И ЕЁ ПРИМЕНЕНИЯ\ \ \ том~9\ \ \ выпуск~4\ \ \ 2015}
}%
 \def\rightfootline{\small{ИНФОРМАТИКА И ЕЁ ПРИМЕНЕНИЯ\ \ \ том~9\ \ \ выпуск~4\ \ \ 2015
 \hfill \textbf{\thepage}}}


{\tabcolsep=3pt
\begin{tabular}{p{388pt}rr}
&\textbf{Выпуск} & \textbf{Стр.}\\[2.3pt]
\textbf{Зыкин В.\,С.} Ссылочная целостность данных в корпоративных
информационных\linebreak
\\[-12.35pt]
\hspace*{21pt}сис\-те\-мах&3&97\\[-.75pt]
\textbf{Калиниченко Л.\,А., Ковалев Д.\,Ю., Ковалева Д.\,А., Малков
О.\,Ю.} Методы и~средства\linebreak
\\[-12.35pt]
\hspace*{21pt}поддержки исследований, движимых гипотезами:
обзор&1&28\\[-.75pt]
\textbf{Кантор О.\,Г.} см.\ Спивак С.\,И.&&\\[-.75pt]
\textbf{Катаев М.\,Ю.} см.\ Абрамов М.\,О.&&\\[-.75pt]
\textbf{Кириков И.\,А., Колесников А.\,В., Листопад С.\,В., Румовская
С.\,Б.} Мелкозернистые\linebreak
\\[-12.35pt]
\hspace*{21pt}гибридные интеллектуальные сис\-те\-мы. Часть~1:
Лингвистический подход&4&98\\[-.75pt]
\textbf{Климов А.\,В.} см.\ Змеев Д.\,Н.&&\\[-.75pt]
\textbf{Ковалев Д.\,Ю.} см.\ Калиниченко Л.\,А.&&\\[-.75pt]
\textbf{Ковалева Д.\,А.} см.\ Калиниченко Л.\,А.&&\\[-.75pt]
\textbf{Ковалёв С.\,П.} Формальный аксиоматический подход к
аспектно-ориентированному\linebreak
\\[-12.35pt]
\hspace*{21pt}расширению технологий программирования&1&55\\[-.75pt]
\textbf{Коган-Садецкая М.} см.\ Долев Ш.&&\\[-.75pt]
\textbf{Козеренко Е.\,Б.} см.\ Галина И.\,В.&&\\[-.75pt]
\textbf{Колесников А.\,В.} см.\ Кириков И.\,А.&&\\[-.75pt]
\textbf{Колесов С.\,В.} см.\ Спивак С.\,И.&&\\[-.75pt]
\textbf{Коновалов М.\,Г., Разумчик Р.\,В.} Обзор моделей и~алгоритмов
размещения заданий\linebreak
\\[-12.35pt]
\hspace*{21pt}в~сис\-те\-мах с~параллельным обслуживанием&4&56\\[-.75pt]
\textbf{Копицкая М.} см.\ Френкель С.&&\\[-.75pt]
\textbf{Корепанов Э.\,Р.} см.\ Синицын И.\,Н.&&\\[-.75pt]
\textbf{Корепанов Э.\,Р.} см.\ Синицын И.\,Н.&&\\[-.75pt]
\textbf{Корепанов Э.\,Р.} см.\ Синицын И.\,Н.&&\\[-.75pt]
\textbf{Корепанов Э.\,Р.} см.\ Синицын И.\,Н.&&\\[-.75pt]
\hangindent=21pt\noindent\textbf{Королев В.\,Ю., Горшенин А.\,К., Гулев С.\,К., Беляев К.\,П.}
Статистическое моделирование турбулентных потоков тепла между океаном
и~атмосферой с~помощью метода\linebreak
\\[-12.35pt]
\hspace*{21pt}скользящего разделения конечных
нормальных смесей&4&3\\[-.75pt]
\hangindent=21pt\noindent\textbf{Королев В.\,Ю., Корчагин А.\,Ю., Соколов И.\,А.} Обобщенные
дисперсионные гамма-рас-\linebreak
\\[-12.35pt]
\hspace*{21pt}пре\-де\-ле\-ния как модели статистических
закономерностей на финансовых рынках&4&14\\[-.75pt]
\textbf{Корчагин А.\,Ю.} см.\ Королев В.\,Ю.&&\\[-.75pt]
\textbf{Корчажкина О.\,М.} К оценке эффективности учебно-познавательной
деятельности\linebreak
\\[-12.35pt]
\hspace*{21pt}учащихся с использованием информационных
технологий&1&106\\[-.75pt]
\textbf{Кривенко М.\,П.} Модели для представления и обработки
референсных значений&2&63\\[-.75pt]
\textbf{Кристрём Б.} см.\ Беляев Ю.\,К.&&\\[-.75pt]
\textbf{Кудрявцев А.\,А.} см.\ Жаворонкова Ю.\,В.&&\\[-.75pt]
\textbf{Кузнецов С.\,И.} см.\ Спивак С.\,И.&&\\[-.75pt]
\textbf{Лебедев А.\,В.} Экстремальные индексы в схеме серий и их
приложения&3&39\\[-.75pt]
\textbf{Левченко Н.\,Н.} см.\ Змеев Д.\,Н.&&\\[-.75pt]
\textbf{Лери М.\,М.} Пожар на конфигурационном графе со случайными
переходами огня\linebreak
\\[-12.35pt]
\hspace*{21pt}по ребрам&3&65\\[-.75pt]
\textbf{Листопад С.\,В.} см.\ Кириков И.\,А.&&\\[-.75pt]
\textbf{Малков О.\,Ю.} см.\ Калиниченко Л.\,А.&&\\[-.75pt]
\textbf{Матюшенко С.\,И.} см.\ Горбунова А.\,В.&&\\[-.75pt]
\hangindent=21pt\noindent\textbf{Мейханаджян Л.\,А., Милованова Т.\,А., Разумчик Р.\,В.}
Время ожидания в сис\-те\-ме обслуживания с инверсионным порядком обслуживания
и обобщенным вероят-\linebreak
\\[-12.35pt]
\hspace*{21pt}ностным приоритетом&2&14\\[-.75pt]
\textbf{Милованова Т.\,А.} см.\ Мейханаджян Л.\,А.&&\\[-.75pt]
\hangindent=21pt\noindent\textbf{Минин В.\,А., Зацман И.\,М., Хавансков В.\,А., Шубников С.\,К.}
Индикаторы тематических взаимосвязей отраслей науки и
информационно-компьютерных технологий в~на-\linebreak
\\[-12.35pt]
\hspace*{21pt}ча\-ле XXI~века&2&111\\[-.75pt]
\textbf{Миронов А.\,М.} Реализуемость вероятностных реакций конечными
вероятностными\linebreak
\\[-12.35pt]
\hspace*{21pt}автоматами&3&85\\[-.75pt]
\textbf{Молотковский Р.} см.\ Френкель С.&&\\[-.75pt]
\textbf{Морозова Ю.\,И.} см.\ Галина И.\,В.&&\\[-.75pt]
\textbf{Окунев А.\,С.} см.\ Змеев Д.\,Н.&&\\[-.75pt]
\textbf{Печинкин А.\,В., Разумчик Р.\,В.} Совместное стационарное
распределение числа заявок\linebreak
\\[-12.35pt]
\hspace*{21pt}в $m$ очередях в $N$-канальной сис\-те\-ме
обслуживания с переупорядочением заявок&3&25\\[-.75pt]
\end{tabular}
}

\pagebreak

\def\leftkol{АВТОРСКИЙ УКАЗАТЕЛЬ ЗА 2015 г.} % ENGLISH ABSTRACTS}

\def\rightkol{АВТОРСКИЙ УКАЗАТЕЛЬ ЗА 2015 г.} %ENGLISH ABSTRACTS}

%\thispagestyle{myheadings}
\def\leftfootline{\small{\textbf{\thepage}
\hfill ИНФОРМАТИКА И ЕЁ ПРИМЕНЕНИЯ\ \ \ том~9\ \ \ выпуск~4\ \ \ 2015}
}%
 \def\rightfootline{\small{ИНФОРМАТИКА И ЕЁ ПРИМЕНЕНИЯ\ \ \ том~9\ \ \ выпуск~4\ \ \ 2015
 \hfill \textbf{\thepage}}}


{\tabcolsep=3pt
\begin{tabular}{p{388pt}rr}
&\textbf{Выпуск} & \textbf{Стр.}\\[3pt]
\textbf{Попова М.\,С., Стрижов В.\,В.} Выбор оптимальной модели
классификации физической\linebreak
\\[-12pt]
\hspace*{21pt}активности по измерениям
акселерометра&1&76\\[-.35pt]
\hangindent=21pt\noindent\textbf{Разумчик Р.\,В.} Алгебраический метод приближенного расчета
стационарного распределения в~сис\-те\-ме обслуживания конечной емкости
с~отрицательными заявками\linebreak
\\[-12pt]
\hspace*{21pt}и~двумя очередями&4&68\\[-.35pt]
\textbf{Разумчик Р.\,В.} см.\ Коновалов М.\,Г.&&\\[-.35pt]
\textbf{Разумчик Р.\,В.} см.\ Мейханаджян Л.\,А.&&\\[-.35pt]
\textbf{Разумчик Р.\,В.} см.\ Печинкин А.\,В.&&\\[-.35pt]
\textbf{Румовская С.\,Б.} см.\ Кириков И.\,А.&&\\[-.35pt]
\textbf{Самуйлов А.\,К.} см.\ Гайдамака Ю.\,В.&&\\[-.35pt]
\textbf{Самуйлов К.\,Е.} см.\ Вихрова О.\,Г.&&\\[-.35pt]
\textbf{Самуйлов К.\,Е.} см.\ Горбунова А.\,В.&&\\[-.35pt]
\textbf{Синицын В.\,И.} см.\ Синицын И.\,Н.&&\\[-.35pt]
\textbf{Синицын В.\,И.} см.\ Синицын И.\,Н.&&\\[-.35pt]
\textbf{Синицын И.\,Н.} Аналитическое моделирование процессов
в~динамических сис\-те\-мах\linebreak
\\[-12pt]
\hspace*{21pt}с~цилиндрическими бесселевыми
нелинейностями&4&37\\[-.35pt]
\textbf{Синицын И.\,Н.} Аналитическое моделирование распределений
методом ортогональных\linebreak
\\[-12pt]
\hspace*{21pt}разложений в нелинейных стохастических сис\-те\-мах на
многообразиях&3&17\\[-.35pt]
\hangindent=21pt\noindent\textbf{Синицын И.\,Н., Корепанов Э.\,Р.} Нормальные
условно-оптимальные фильтры Пугачёва для дифференциальных стохастических сис\-тем,
линейных относительно состо-\linebreak
\\[-12pt]
\hspace*{21pt}яния&2&30\\[-.35pt]
\textbf{Синицын И.\,Н., Корепанов Э.\,Р.} Устойчивые линейные условно
оптимальные фильтры\linebreak
\\[-12pt]
\hspace*{21pt}и экстраполяторы для стохастических сис\-тем с
мультипликативными шумами&1&70\\[-.35pt]
\textbf{Синицын И.\,Н., Синицын В.\,И., Корепанов Э.\,Р.}
Моделирование нормальных процессов\linebreak
\\[-12pt]
\hspace*{21pt}в стохастических сис\-те\-мах со сложными
иррациональными нелинейностями&1&2\\[-.35pt]
\textbf{Синицын И.\,Н., Синицын В.\,И., Корепанов Э.\,Р.}
Моделирование нормальных процессов\linebreak
\\[-12pt]
\hspace*{21pt}в стохастических сис\-те\-мах со сложными
трансцендентными нелинейностями&2&23\\[-.35pt]
\textbf{Соколов И.\,А.} см.\ Королев В.\,Ю.&&\\[-.35pt]
\textbf{Сомин Н.\,В.} см.\ Галина И.\,В.&&\\[-.35pt]
\textbf{Сопин Э.\,С.} см.\ Вихрова О.\,Г.&&\\[-.35pt]
\textbf{Спивак С.\,И., Кантор О.\,Г., Юнусова Д.\,С., Кузнецов С.\,И.,
Колесов С.\,В.} Оценка\linebreak
\\[-12pt]
\hspace*{21pt}погрешности и значимости измерений для линейных
моделей&1&87\\[-.35pt]
\textbf{Стемпковский А.\,Л.} см.\ Змеев Д.\,Н.&&\\[-.35pt]
\textbf{Стенина М.\,М., Стрижов В.\,В.} Согласование прогнозов при
решении задач прогнози-\linebreak
\\[-12pt]
\hspace*{21pt}рования иерархических временных рядов&2&75\\[-.35pt]
\textbf{Стрижов В.\,В.} см.\ Попова М.\,С.&&\\[-.35pt]
\textbf{Стрижов В.\,В.} см.\ Стенина М.\,М.&&\\[-.35pt]
\textbf{Тимонина Е.\,Е.} см.\ Грушо А.\,А.&&\\[-.35pt]
\textbf{Тучкова Н.\,П.} см.\ Беляев К.\,П.&&\\[-.35pt]
\textbf{Ушаков В.\,Г., Ушаков Н.\,Г.} Об усреднении округленных
данных&4&106\\[-.35pt]
\textbf{Ушаков Н.\,Г.} см.\ Ушаков В.\,Г.&&\\[-.35pt]
\textbf{Френкель С., Копицкая М., Молотковский Р., Боровский П.}
Улучшение производитель-\linebreak
\\[-12pt]
\hspace*{21pt}ности алгоритма сжатия данных Лемпеля--Зива--Уэлча&4&78\\[-.35pt]
\textbf{Хавансков В.\,А.} см.\ Минин В.\,А.&&\\[-.35pt]
\hangindent=21pt\noindent\textbf{Чичагов В.\,В.} Асимптотические
разложения высокого порядка для
несмещенных оценок и их дисперсий в модели однопараметрического
экспоненциального\linebreak
\\[-12pt]
\hspace*{21pt}се\-мей\-ст\-ва&3&72\\[-.35pt]
\textbf{Шарнин М.\,М.} см.\ Галина И.\,В.&&\\[-.35pt]
\textbf{Шестаков О.\,В.} Непараметрическое оценивание многомерной
плотности с помощью\linebreak
\\[-12pt]
\hspace*{21pt}вейвлет-оценок одномерных проекций&2&88\\[-.35pt]
\textbf{Шоргин С.\,Я.} см.\ Вихрова О.\,Г.&&\\[-.35pt]
\textbf{Шоргин С.\,Я.} см.\ Горбунова А.\,В.&&\\[-.35pt]
\textbf{Шоргин С.\,Я.} см.\ Жаворонкова Ю.\,В.&&\\[-.35pt]
\textbf{Шубников С.\,К.} см.\ Минин В.\,А.&&\\[-.35pt]
\textbf{Юнусова Д.\,С.} см.\ Спивак С.\,И.&&\\[-.35pt]
\end{tabular}
}

%\thispagestyle{myheadings}
\def\leftfootline{\small{\textbf{\thepage}
\hfill ИНФОРМАТИКА И ЕЁ ПРИМЕНЕНИЯ\ \ \ том~9\ \ \ выпуск~4\ \ \ 2015}
}%
 \def\rightfootline{\small{ИНФОРМАТИКА И ЕЁ ПРИМЕНЕНИЯ\ \ \ том~9\ \ \ выпуск~4\ \ \ 2015
 \hfill \textbf{\thepage}}}

 \label{end\stat}


%\linebreak
%\\[-12pt]
%\hspace*{21pt}

%\hangindent=21pt\noindent