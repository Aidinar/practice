

\def\stat{kor+korchagin}

\def\tit{ОБОБЩЕННЫЕ ДИСПЕРСИОННЫЕ ГАММА-РАСПРЕДЕЛЕНИЯ
КАК~МОДЕЛИ СТАТИСТИЧЕСКИХ ЗАКОНОМЕРНОСТЕЙ НА~ФИНАНСОВЫХ
РЫНКАХ$^*$}

\def\titkol{Обобщенные дисперсионные гамма-распределения
как модели статистических закономерностей} % на финансовых рынках}

\def\aut{В.\,Ю.~Королев$^1$, А.\,Ю.~Корчагин$^2$, И.\,А.~Соколов$^3$}

\def\autkol{В.\,Ю.~Королев, А.\,Ю.~Корчагин, И.\,А.~Соколов}

\titel{\tit}{\aut}{\autkol}{\titkol}

{\renewcommand{\thefootnote}{\fnsymbol{footnote}} \footnotetext[1]
{Работа выполнена при поддержке Российского фонда
фундаментальных исследований (проект 14-07-00041а.)}}


\renewcommand{\thefootnote}{\arabic{footnote}}
\footnotetext[1]{Факультет вычислительной математики и кибернетики
Московского государственного университета имени М.\,В.~Ломоносова;
Институт проблем информатики Федерального исследовательского
центра <<Информатика и~управление>> Российской академии наук, vkorolev@cs.msu.su} 
\footnotetext[2]{Факультет вычислительной математики и кибернетики
Московского государственного университета имени М.\,В.~Ломоносова,
sasha.korchagin@gmail.com}
\footnotetext[3]{Федеральный исследовательский
центр <<Информатика и~управление>> Российской академии наук, ISokolov@ipran.ru}



\vspace*{-4pt}

\Abst{Обсуждаются различные вопросы, связанные 
с~применением обобщенных дисперсионных гам\-ма-рас\-пре\-де\-ле\-ний для
моделирования статистических закономерностей на финансовых рынках.
Описаны простейшие свойства обобщенных дисперсионных
гам\-ма-рас\-пре\-де\-ле\-ний как специальных дис\-пер\-си\-он\-но-сдви\-го\-вых смесей
нормальных законов, в~которых смешивающими являются обобщенные
гам\-ма-рас\-пре\-де\-ле\-ния. Приведены предельные теоремы для сумм
случайного числа независимых случайных величин~--- аналоги закона
больших чисел и центральной предельной теоремы,~--- обосновывающие
возможность использования обобщенных дисперсионных
гам\-ма-рас\-пре\-де\-ле\-ний в~качестве асимптотических аппроксимаций.
Приведены результаты практической подгонки обобщенных дисперсионных
гам\-ма-рас\-пре\-де\-ле\-ний к реальным данным о поведении финансовых
индексов и обобщенных гам\-ма-рас\-пре\-де\-ле\-ний к наблюдаемым
интенсивностям информационных потоков в~современных финансовых
информационных системах. Результаты сравнения обобщенных
дисперсионных гам\-ма-мо\-де\-лей с обобщенными гиперболическими моделями
свидетельствуют о~преимуществе первых над вторыми. Также обсуждаются
методы оценивания параметров обобщенных дисперсионных гам\-ма-мо\-де\-лей
и~их применение при прогнозировании процессов, протекающих на
финансовых рынках.}

\KW{обобщенные дисперсионные гамма-распределения;
дис\-пер\-си\-он\-но-сдви\-го\-вые смеси нормальных законов; обобщенные
гамма-распределения; суммы случайного числа случайных величин; закон
больших чисел; центральная предельная теорема}

\DOI{10.14357/19922264150402}

\vspace*{3pt}

\vskip 14pt plus 9pt minus 6pt

\thispagestyle{headings}

\begin{multicols}{2}

\label{st\stat}

\section{Введение}

Обобщенные дисперсионные гам\-ма-рас\-пре\-де\-ле\-ния (GVG-distributions)
являются специальными дисперсионно-сдвиговыми смесями нормальных
законов. Понятие дис\-пер\-си\-он\-но-сдви\-го\-вой смеси нормальных законов
(normal variance-mean mixture) введено в~1970--1980~гг.\ 
в~работах \mbox{О.-Е.}~Барн\-дорфф-Ниль\-се\-на и его
коллег~[1--3] как довольно гибкое обобщение нормального распределения.

Пусть $\beta\in\mathbb{R}$, $\alpha\hm\in\mathbb{R}$,
$0\hm<\sigma\hm<\infty$, $\Phi(x)$~--- стандартная нормальная функция
распределения, $G(x)$~--- функция распределения, все точки роста
которой сосредоточены на~$\mathbb{R}_+$. Дис\-пер\-си\-он\-но-сдви\-го\-вой
смесью нормальных законов называется функция распределения
\begin{equation}
F(x)=\int\limits_{0}^{\infty}\Phi\left(\fr{x-\beta-\alpha
z}{\sigma\sqrt{z}}\right)\,dG(z)\,,\enskip x\in \mathbb{R}\,.
\label{e1-kk}
\end{equation}
Обратим внимание, что в~соотношении~(1) смешивание происходит
одновременно и по параметру сдвига, и по параметру масштаба, но так
как эти параметры в~(1) связаны жесткой зависимостью, при которой
параметры положения ({\it сдвига}) смешиваемых нормальных законов
пропорциональны их {\it дисперсиям}, то фактически смесь~(1)
является однопараметрической. Именно поэтому смеси вида~(1)
называются {\it дис\-пер\-си\-он\-но-сдви\-го\-выми}.

Вероятностные модели типа дис\-пер\-си\-он\-но-сдви\-го\-вых смесей нормальных
законов рассматриваются в~качестве базовых во многих практических
задачах, связанных с изучением статистических закономерностей
поведения информационных потоков. Наиболее известными
представителями таких моделей являются {\it обобщенные
гиперболические распределения}, введенные в~упомянутых работах 
\mbox{О.-Е.}~Барн\-дорфф-Ниль\-се\-на 
и~его коллег. Обобщенные гиперболические модели
имеют вид~(1), в~кото-\linebreak\vspace*{-12pt}

\pagebreak

\noindent
ром $G(z)\hm=P_{\mathrm{GIG}}(z;\nu,\mu,\lambda)$~---
функция обобщенного обратного гауссовского распределения,
соответствующая плотности
\begin{multline*}
p_{\mathrm{GIG}}(x;\nu,\mu,\lambda)=\fr{\lambda^{\nu/2}}{2\mu^{\nu/2}\,
K_{\nu}\left(\sqrt{\mu\lambda}\right)}\cdot
x^{\nu-1}\times{}\\
{}\times\exp\left\{-\fr{1}{2}\left(\fr{\mu}{x}+\lambda
x\right)\right\},\enskip x>0\,.
\end{multline*}
Здесь $K_{\nu}(z)$~-- модифицированная бесселева функция третьего
рода порядка~$\nu$,
\begin{multline*}
K_{\nu}(z)=\fr{1}{2}\int\limits_{0}^{\infty}
y^{\nu-1}\exp\left\{-\fr{z}{2}\left(y+\fr{1}{y}\right)\right\}\,dy\,,\\
z\in\mathbb{C}\,,\ \mathrm{Re}\,z>0\,.
\end{multline*}
При этом $\nu\in\mathbb{R}$, и если $\nu\hm<0$, то $\mu\hm>0$, $\lambda\hm\geqslant0$; если
$\nu\hm=0$, то $\mu\hm>0$, $\lambda\hm>0$; если $\nu\hm>0$, то $\mu\hm\geqslant0$,
$\lambda\hm>0$. Обобщенные гиперболические модели демонстрируют очень
высокую адекватность при анализе финансовых данных.

В работе~\cite{KorolevSokolov2012} было введено еще одно семейст-\linebreak во
дис\-пер\-си\-он\-но-сдви\-го\-вых смесей нормальных \mbox{законов} вида~(1)~---
семейство {\it обобщенных дис\-пер\-си\-он\-ных гам\-ма-рас\-пре\-де\-ле\-ний}, 
в~котором $G(z)\hm=P_{\mathrm{GG}}(z;\nu,\kappa,\delta)$~--- функция обобщенного
гам\-ма-рас\-пре\-де\-ле\-ния, соответствующая плот\-ности
\begin{multline}
p_{\mathrm{GG}}(x;\nu,\kappa,\delta)=\fr{|\nu|}{\delta^{k|\nu|}\Gamma(\kappa)}\,
x^{\kappa\nu-1}\exp\left\{-\fr{x^{\nu}}{\delta^{|\nu|}}\right\}\,,\\
 x\geqslant0\,,
 \label{e2-kk}
\end{multline}
с параметрами $\nu\hm\in\mathbb{R}$,  $\kappa,\,\delta\hm\in{\mathbb R}^+$,
отвечающими соответственно за {\it степень, форму и масштаб}.
Функцию обобщенного гам\-ма-дис\-пер\-си\-он\-но\-го распределения будем
обозначать $P_{\mathrm{GVG}}(x;\alpha,\sigma,\nu,\kappa,\delta)$,
\begin{multline}
P_{\mathrm{GVG}}(x;\alpha,\sigma,\nu,\kappa,\delta)={}\\
\hspace*{-2mm}{}=
\int\limits_{0}^{\infty}\Phi\left(\fr{x-\beta-\alpha
z}{\sigma\sqrt{z}}\right)p_{\mathrm{GG}}(x;\nu,\kappa,\delta)\,dz,\enskip
x\in\mathbb{R}.\!\!\!\!
\label{e3-kk}
\end{multline}

Класс обобщенных гамма-рас\-пре\-де\-ле\-ний (GG-рас\-пре\-де\-ле\-ний) был впервые
описан как единое семейство в~1962~г.\ в~работе~\cite{Stacy1962} 
в~качестве семейства вероятностных моделей, включающего в~себя
одновременно гам\-ма-рас\-пре\-де\-ле\-ния и распределения Вейбулла. Семейство
GG-рас\-пре\-де\-ле\-ний содержит практически все наиболее популярные
абсолютно непрерывные распределения. В~частности, семейство
GG-рас\-пре\-де\-ле\-ний содержит гам\-ма-рас\-пре\-де\-ле\-ние $(\nu\hm=1)$ и его
частные случаи: показательное (экспоненциальное) распределение
($\nu\hm=1$, $\kappa\hm=1$), распределение Эрланга ($\nu\hm=1$,
$\kappa\hm\in\mathbb{N}$), распределение хи-квад\-рат ($\nu\hm=1$,
$\delta\hm=2$); распределение Накагами ($\nu\hm=2$); полунормальное
распределение, иначе называемое сложенным (folded) нормальным
(распределение максимума винеровского процесса на отрезке $[0,1]$)
($\nu\hm=2$, $\kappa\hm=1/2$); распределение Рэлея ($\nu\hm=2$,
$\kappa\hm=1$); xи-рас\-пре\-де\-ле\-ние ($\nu\hm=2$, $\delta\hm=\sqrt{2}$);
распределение Максвелла (распределение модулей скоростей движения
молекул в~разреженном газе) ($\nu\hm=2$, $\kappa\hm=3/2$);
распределение Вей\-бул\-ла--Гне\-ден\-ко ($\kappa\hm=1$, $\nu\hm>0$); обратное
гам\-ма-рас\-пре\-де\-ле\-ние ($\nu\hm=-1$); распределение Леви (одностороннее
устойчивое распределение с характеристическим показателем
$\alpha\hm=1/2$~--- распределение времени достижения процессом
броуновского движения единичного уровня) ($\nu\hm=-1$,
$\kappa\hm=1/2$); логнормальное распределение ($\kappa\hm\to\infty$) и~другие законы.

Обощенные гамма-распределения широко применяются во многих прикладных задачах,
прежде всего связанных с обработкой сигналов и изображений. Работы,
посвященные применению GG-рас\-пре\-де\-ле\-ний в~качестве моделей реально
наблюдаемых закономерностей, исчисляются сотнями.\linebreak Большая
популярность GG-рас\-пре\-де\-ле\-ний, по-ви\-ди\-мо\-му, обуслов\-ле\-на возможностью
их ис\-поль\-зо\-вания в~качестве адекватных асимптотических
аппроксимаций, поскольку все перечисленные\linebreak вы\-ше отдельные
представители GG-рас\-пре\-де\-ле\-ний выступают в~качестве предель\-ных 
в~различных предель\-ных тео\-ре\-мах тео\-рии вероятностей. Эти свойства
GG-распределений обосновывают, в~част\-ности, целесообразность
моделирования с~их по\-мощью распределения случайной интенсив\-ности
потока запросов в~информационных системах. Ниже будет сформулирована
предельная теорема для случайных сумм независимых случайных величин,
в которой предельными законами оказываются GG-рас\-пре\-де\-ле\-ния в~общем их виде.

Необходимость рассмотрения GG-рас\-пре\-де\-ле\-ний в~качестве смешивающих 
в~дис\-пер\-си\-он\-но-сдви\-го\-вых смесях нормальных законов обусловлена тем,
что, в~отличие от GIG-рас\-пре\-де\-ле\-ний, класс GG-рас\-пре\-де\-ле\-ний содержит
законы, хвосты которых убывают вейбулловским образом. В~частности,
при $0\hm<\nu\hm<1$ GG-рас\-пре\-де\-ле\-ния за\-ни\-ма\-ют промежуточное место
между распределениями с экспоненциальным убыванием хвостов
(показательное распределение, гам\-ма-рас\-пре\-де\-ле\-ние) 
и~<<тяжелохвостыми>> распределениями со степенным убыванием хвостов
типа Цип\-фа--Па\-ре\-то. В~част\-ности, возможно, именно поэтому
распределения Вей\-бул\-ла--Гне\-ден\-ко с $\nu\hm<1$ в~некоторых работах (см.,
например,~[6--8]) называются {\it растянутыми $($stretched$)$
показательными распределениями}). Вместе с тем, как показано 
в~работе~\cite{AntonovKoksharov2006}, асимптотическое поведение
хвостов смесей нормальных законов в~определенном смысле совпадает 
с~аналогичным поведением хвостов смешивающих законов. Поэтому, 
в~отличие от обобщенных гиперболических законов, класс обобщенных
дисперсионных гам\-ма-рас\-пре\-де\-ле\-ний содержит распределения с хвостами,
убывающими экс\-по\-нен\-ци\-аль\-но-сте\-пен\-н$\acute{\mbox{ы}}$м образом. Такие особенности
эмпирических распределений характерны для финансовых индексов 
с~нерегулярным (перемежающимся) поведением, когда на фоне <<обычных>>
стохастических флуктуаций возможны (кратковременные) периоды
повышенной активности (<<волатильности>>) или отдельные выбросы.

Статья построена следующим образом. В~разд.~2 приведены
элементарные свойства дисперси\-он-\linebreak но-сдвиговых смесей нормальных
законов. Основная часть этого раздела посвящена обсуждению\linebreak
возможности использования GVG-рас\-пре\-де\-ле\-ний в~качестве
асимптотических аппроксимаций. Высокую адекватность моделей типа
дис\-пер\-си\-он\-но-сдви\-го\-вых смесей нормальных законов можно\linebreak легко
объяснить довольно большим числом настраиваемых параметров 
в~указанных моделях. Однако на самом деле их адекватность имеет
гораздо более глубокие теоретические обоснования. В~прикладной
теории вероятностей принято считать, что та или иная модель может
быть в~достаточной мере обоснованной (адекватной) только тогда,
когда она является {\it асимптотической аппроксимацией}, т.\,е.\
когда существует довольно простая предельная схема (например, схема
максимума или схема суммирования) и соответствующая предельная
теорема, в~которой рассматриваемая модель выступает в~качестве
предельного распределения. Наличие такой формальной асимптотической
схемы может дать дополнительную информацию о реальных механизмах,
формирующих те или иные наблюдаемые статистические закономерности.
Оказывается, что дис\-пер\-си\-он\-но-сдви\-го\-вые смеси нормальных законов
являются предельными законами в~довольно простых предельных теоремах
для случайно остановленных случайных блужданий. Такие теоремы
позволяют однозначно связать конкретный смешивающий закон 
в~дис\-пер\-си\-он\-но-сдви\-го\-вых смесях с поведением интенсивности потока
информативных событий, в~результате которых накапливаются данные,
характеризующие анализируемый случайный процесс. Тем самым эти
теоремы как бы позволяют разделить вклады внешних и внутренних
факторов в~случайность поведения анализируемого процесса. 
В~разд.~2 приведены и прокомментированы довольно простые предельные тео\-ре\-мы
для случайных сумм типа центральной предельной тео\-ре\-мы, в~которых
предельными законами являются обобщенные дисперсионные
гам\-ма-рас\-пре\-де\-ле\-ния. Особое внимание в~этом разделе уделено
механизму формирования GG-рас\-пре\-де\-ле\-ний общего вида как
асимптотических аппроксимаций. В~качестве такого механизма предложен
аналог закона больших чисел для отрицательных биномиальных случайных
сумм независимых неодинаково распределенных случайных величин.

В разд.~3 приведены примеры подгонки GG-рас\-пре\-де\-ле\-ний к реальным
данным об интенсивности потоков событий на финансовых рынках, 
а~также результаты сравнения таких моделей с GIG-рас\-пре\-де\-ле\-ниями.

Для оценивания параметров упомянутых выше моделей недавно
разработан, реализован и~тео\-ре\-ти\-чески и экспериментально исследован
новый комбинированный сеточный метод разделения
дис\-пер\-си\-он\-но-сдви\-го\-вых смесей нормальных законов,\linebreak описываемый 
в~разд.~4. Этот метод превосходит извест\-ные методы по точности 
и~эффективности. Последнее обстоятельство особенно важно при
применении этого метода для анализа реальных процессов с целью
прогнозирования со\-пут\-ст\-ву\-ющих рисков, возможности и результаты
которого на примере индекса Shanghai Composite также обсуждаются 
в~разд.~4.

\section{Обобщенные дисперсионные гамма-распределения 
как~асимптотические аппроксимации}

\subsection{Элементарные свойства дисперсионно-сдвиговых смесей
нормальных законов}

Будем считать, что все случайные величины, о~которых пойдет речь
ниже, заданы на одном вероятностном пространстве $(\Omega,\,
\mathfrak{A},\,{\sf P})$. Если~$Y$ и $U$~--- независимые случайные
величины с функциями распределения $\Phi(x)$ и~$G(x)$ (при этом,
очевидно, ${\sf P}(U\geqslant0)\hm=1$), a~$Z$~--- случайная величина,
удовлетворяющая соотношению $Z\eqd \sigma Y\sqrt{U}\hm+\alpha U
\hm+\beta$, то ${\sf P}(Z<x)\hm=F(x)$, где $F(x)$~--- функция распределения
дис\-пер\-си\-он\-но-сдви\-го\-вой смеси нормальных законов, определяемая
соотношением~(1), $x\hm\in\mathbb{R}$.

Легко убедиться, что если ${\sf E}U\hm<\infty$, то
\begin{equation}
{\sf E}Z=\beta+\alpha{\sf E}U\,,
\label{e4-kk}
\end{equation}
а если при этом и ${\sf E}U^2\hm<\infty$, то ${\sf
E}Z^2\hm=\beta^2\hm+(\sigma^2\hm+2\alpha\beta){\sf E}U\hm+\alpha^2{\sf E}U^2$,
${\sf D}Z\hm=\alpha^2{\sf D}U\hm+\sigma^2{\sf E}U$.

При вычислении моментов более высоких порядков можно использовать
следующее соотношение между характеристическими функциями
$\mathfrak{f}_Z$ и~$\mathfrak{f}_U$ случайных величин~$Z$ и~$U$:
$$
\mathfrak{f}_Z(t)=e^{it\beta}\mathfrak{f}_U\left(\alpha
t+\fr{1}{2}\,i\sigma^2t^2\right)\,,\enskip t\in\mathbb{R}\,.
$$
Чтобы убедиться в~справедливости последнего соотношения, достаточно
заметить, что по теореме Фубини
\begin{multline*}
\mathfrak{f}_Z(t)={\sf E}e^{itZ}={\sf
E}\exp\{it(\beta+\sigma\sqrt{U}\cdot Y+\alpha
U)\}={}\\
{}=\int\limits_{0}^{\infty}{\sf E}\exp\{it\beta+\sigma\sqrt{u}\cdot
Y+\alpha u)\}\,dG(u)={}
\\
{}=e^{it\beta}\int\limits_{0}^{\infty}\exp\left\{it\alpha u-
\fr{1}{2}\,\sigma^2t^2u\right\}\,dG(u)={}\\
{}=e^{it\beta}\int\limits_{0}^{\infty}\exp\left\{iu\left(\alpha t+
\fr{1}{2}\,i\sigma^2t^2\right)\right\}\,dG(u)={}
\\
{}=e^{it\beta}\mathfrak{f}_U\left(\alpha
t+\fr{1}{2}\,i\sigma^2t^2\right)\,,\enskip t\in\mathbb{R}\,.
\end{multline*}


\subsection{Критерий сходимости распределений случайных сумм 
к~обобщенным дисперсионным гамма-распределениям}

Пусть $\{X_{n,j}\}_{j\geqslant1},$ $n\hm=1,2,\ldots,$~--- семейство
последовательностей одинаково распределенных в~каж\-дой
последовательности (при каждом фиксированном~$n$) случайных величин.
Пусть $\{N_n\}_{n\geqslant1}$~--- последовательность целочисленных
неотрицательных случайных величин, таких что пpи каждом $n\hm\geqslant1$
случайные величины $N_n,X_{n,1},X_{n,2},\ldots$ независимы. Положим
$S_{n,k}\hm=X_{n,1}+\cdots +X_{n,k}$. Во избежание недоразумений
полагаем $\sum_{j=1}^0\hm=0$. Символ~$\Longrightarrow$ будет обозначать
слабую сходимость (сходимость по распределению). Всюду далее
сходимость подразумевается при $n\hm\to\infty$. Будем говорить, что
$N_n\hm\to\infty$ по вероятности, если ${\sf P}(N_n\hm\leqslant m)\hm\to 0$ для
любого $m\hm>0$.

В работе~\cite{Korolev2013} доказана следующая базовая теорема.

\smallskip

\noindent
\textbf{Теорема~1}.\ \textit{Предположим, что существуют последовательность
натуральных чисел $\{k_n\}_{n\geqslant1}$ и число $\alpha\hm\in\mathbb{R}$
такие, что}
\begin{equation}
{\sf P}\left(S_{n,k_n}<x\right)\Longrightarrow \Phi(x-\alpha)\,.\label{e5-kk}
\end{equation}
\textit{Предположим, что $N_n\hm\to\infty$ по вероятности. Для того чтобы имела
место сходимость распределений случайных сумм к некоторой функции
распределения}~$F(x)$:
\begin{equation*}
{\sf P}\big(S_{n,N_n}<x\big)\Longrightarrow F(x)\,, %\label{e6-kk}
\end{equation*}
\textit{необходимо и достаточно, чтобы существовала функция распределения~$G(x)$ 
такая, что} 
\begin{gather*}
G(0)=0\,;\\
F(x)=\int\limits_{0}^{\infty}\Phi\left(\fr{x-\alpha
z}{\sqrt{z}}\right)\,dG(z)\,; %\label{e7-kk}
\\
{\sf P}(N_n<xk_n)\Longrightarrow G(x)\,.
%\label{e8-kk}
\end{gather*}


\smallskip

Условие~(\ref{e5-kk}) выполняется в~следующей довольно общей ситуации.
Предположим, что случайные величины~$X_{n,j}$ имеют конечные
дисперсии. Также предположим, что величины~$X_{n,j}$ могут быть
представлены в~виде $X_{n,j}\hm=X_{n,j}^*\hm+\alpha_n$, где
$\alpha_n\hm\in\mathbb{R}$, a $X_{n,j}^*$~--- случайная величина с ${\sf
E} X_{n,j}^*\hm=0$, ${\sf D} X_{n,j}^*\hm=\sigma_n^2<\infty$, так что
${\sf E} X_{n,1}\hm=\alpha_n$ и ${\sf D} X_{n,1}\hm=\sigma_n^2$.
Предположим, что $\alpha_nk_n\hm\to a$ и $k_n\sigma_n^2\hm\to 1$ при
$n\hm\to\infty$. Тогда вследствие хорошо известного результата 
о~необходимых и достаточных условиях сходимости к нормальному закону
распределений сумм независимых случайных величин с конечными
дисперсиями в~схеме серий (см., например,~\cite{GnedenkoKolmogorov1949}) 
можно заметить, что соотношение~(\ref{e5-kk})
имеет место тогда и только тогда, когда выполнено условие
Линдеберга:
$$
\lim\limits_{n\to\infty}k_n{\sf
E}(X_{n,1}^*)^2\mathbb{I}(|X_{n,1}^*|\geqslant\varepsilon)=0
$$
для любого $\varepsilon>0$ (здесь $\mathbb{I}(A)$~--- индикаторная
функция множества (события)~$A$), т.\,е.\ квадратичные хвосты
распределений слагаемых должны убывать достаточно быстро.

Следующее следствие теоремы~1, приведенное в~\cite{KorolevZaks2013},
представляет собой критерий схо\-ди\-мости распределений случайных сумм
к~обобщенным дис\-пер\-си\-он\-но-сдви\-го\-вым распределениям.

\smallskip

\noindent
\textbf{Следствие~1}.\ \textit{Предположим, что существуют
последовательность натуральных чисел $\{k_n\}_{n\geqslant1}$ и число
$\alpha\hm\in\mathbb{R}$ такие, что имеет место сходимость~$(\ref{e5-kk})$.
Предположим, что $N_n\hm\to\infty$ по вероятности. Для того чтобы имела
место сходимость распределений случайных сумм к обобщенным
дисперсионным гам\-ма-рас\-пре\-де\-ле\-ниям$:$
\begin{equation}
{\sf P}\left(S_{n,N_n}<x\right)\Longrightarrow
P_{\mathrm{GVG}}(x;\alpha,\sigma,\nu,\kappa,\delta)\,,\label{e9-kk}
\end{equation}
необходимо и достаточно, чтобы
\begin{equation}
{\sf P}(N_n<xk_n)\Longrightarrow
P_{\mathrm{GG}}(x;\nu,\kappa,\delta)\,,\label{e10-kk}
\end{equation}
где $P_{\mathrm{GG}}(x;\nu,\kappa,\delta)$~--- функция распределения
обобщенного гам\-ма-рас\-пре\-де\-ле\-ния, соответствующая плотности
$p_{\mathrm{GG}}(x;\nu,\kappa,\delta)$ $($см.\ $(\ref{e2-kk}))$.}

\smallskip

В соотношениях~(\ref{e5-kk})--(\ref{e10-kk}) предельные функции распределения
непрерывны, поэтому в~них сходимость по распределению
эквивалентна равномерной сходимости функций распределения.

\subsection{Сходимость распределений отрицательных биномиальных
случайных сумм к~обобщенным гамма-распределениям}

В следствии~1 ключевым условием является соотношение~(\ref{e10-kk}). В~данном
параграфе будет приведен пример асимптотической конструкции, в~рамках
которой справедлива предельная теорема, влекущая сходимость~(\ref{e10-kk}).

Рассмотрим случайную величину $\overline N_{r,p}$, име\-ющую
отрицательное биномиальное распределение с параметрами $r\hm>0$ и~$p\hm\in(0,1)$:
\begin{multline}
{\sf P}(\overline N_{r,p}=k)=C_{r+k-2}^{k-1}p^r(1-p)^{k-1}\,,\\
k=1,2,\ldots,\label{e11-kk}
\end{multline}
где для нецелых~$r$
$$
C_{r+k-2}^{k-1}=\fr{\Gamma(r+k-1)}{(k-1)!\Gamma(r)}\,,
$$
а $\Gamma(x)$~--- эйлерова гам\-ма-функ\-ция. В~таком случае ${\sf
E}\overline N_{r,p}\hm=r/p$.

\smallskip

\noindent
\textbf{Лемма~1.}\ \textit{Eсли $p\hm\to0$, то}
\begin{equation}
\hspace*{-1mm}\lim\limits_{p\to 0}\sup\limits_x\left\vert {\sf P}(p\overline
N_{r,p}<x)-\fr{1}{\Gamma(r)}\int\limits_{0}^{x}z^{r-1}e^{-z}dz\right\vert =0.\!\!\!\!
\label{e12-kk}
\end{equation}

\smallskip

\noindent
Д\,о\,к\,а\,з\,а\,т\,е\,л\,ь\,с\,т\,в\,о\,.\ \ Характеристическая функция $f_{r,p}(t)$
случайной величины $\overline N_{r,p}$ имеет вид:
$$
f_{r,p}(t)=\left(\fr{pe^{it}}{1-(1-p)e^{it}}\right)^r,\enskip
t\in\mathbb{R}\,,
$$
поэтому для каждого $t\hm\in\mathbb{R}$
\begin{multline}
{\sf
E}e^{itpN_{r,p}}=f_{r,p}(pt)=\left(\fr{pe^{ipt}}{1-(1-p)e^{ipt}}\right)^r={}\\
{}=
\left(\fr{p}{e^{-ipt}-1+p}\right)^r=\left(\fr{p}{p-ipt+o(p)}\right)^r={}
\\
{}=[1-it+o(1)]^{-r}\longrightarrow (1-it)^{-r}\label{e13-kk}
\end{multline}
при $p\to 0$. Но правая часть~(\ref{e13-kk})~--- это характеристическая функция
гам\-ма-рас\-пре\-де\-ле\-ния с~параметром формы~$r$ и~параметром масштаба
$\lambda\hm=1$. Таким образом, по теореме непрерывности имеет место
поточечная сходимость функций распределения, участвующих в~(\ref{e12-kk}),
которая является равномерной в~силу непрерывности предельной функции
распределения. Лемма доказана.

\smallskip

Заметим, что случайная величина $\overline N_{1,p}$ имеет
геометрическое распределение, а предельным в~(\ref{e12-kk}) является
стандартное показательное распределение.

Рассмотрим последовательность случайных величин
$\zeta_1,\zeta_2,\ldots$ Пусть $\tau_1,\tau_2,\ldots$~---
на\-ту\-раль\-но\-знач\-ные случайные величины такие, что при каждом~$n$ 
случайная величина~$\tau_n$ независима от последовательности
$\zeta_1,\zeta_2,\ldots$ В~следующей лемме сходимость подразумевается
при $n\hm\to\infty$.

\smallskip

\noindent
\textbf{Лемма~2.}\ \textit{Предположим, что существуют неограниченно
возрастающая $($убывающая к нулю$)$ последовательность положительных
чисел $\{b_n\}_{n\geqslant1}$ и случайная величина~$\zeta$ такие, что}
$$
b_n^{-1}\zeta_n\Longrightarrow\zeta.
$$
\textit{Если существуют неограниченно возрастающая $($убывающая к нулю$)$
последовательность по\-ло\-жи\-тельных чисел $\{d_n\}_{n\geqslant1}$ 
и~случайная величина~$N$ такие, что}
\begin{equation}
d_n^{-1}b_{\tau_n}\Longrightarrow N\,,\label{e14-kk}
\end{equation}
то
\begin{equation}
d_n^{-1}\zeta_{\tau_n}\Longrightarrow \zeta N\,,\label{e15-kk}
\end{equation}
\textit{причем случайные сомножители в~правой части $(\ref{e15-kk})$ независимы. Если
дополнительно $\tau_n\hm\longrightarrow\infty$ по вероятности 
и~семейство масштабных смесей функции распределения случайной величины~$\zeta$ 
идентифици\-ру\-емо, то условие~$(\ref{e14-kk})$ не только достаточно для~$(\ref{e15-kk})$, 
но и необходимо.}

\smallskip

\noindent
Д\,о\,к\,а\,з\,а\,т\,е\,л\,ь\,с\,т\,в\,о\ \ см.~в~\cite{Korolev1994} (случай
$b_n,d_n\hm\to\infty$), \cite{Korolev1995} (случай $b_n,d_n\hm\to 0$) 
или~\cite{BeningKorolev2002} (теорема 3.5.5).

\smallskip

Перейдем к постановке задачи, в~рамках которой GG-распределения в
общем их виде~(\ref{e2-kk}) оказываются предельными для случайных сумм
независи\-мых случайных величин. Пусть $M_1,M_2,\ldots$~--- независимые
случайные величины. Не ограничивая общности, можно считать, что эти
величины принимают натуральные значения. Предположим, что случайная
величина $\overline N_{r,p_n}$ имеет отрицательное биномиальное
распределение~(\ref{e3-kk}) с параметрами $r\hm>0$ и $p_n\hm\in(0,1)$, причем при
каж\-дом $n\hm\in\mathbb{N}$ случайная величина $\overline N_{r,p_n}$
независима от последовательности независимых случайных величин~$\{M_j\}_{j\geqslant1}$ 
и~$p_n\hm\to 0$ при $n\hm\to\infty$. В~качестве
случайной величины~$N_n$ рассмотрим {\it отрицательную биномиальную
случайную сумму}
\begin{equation}
N_n=\sum\limits_{j=1}^{\overline N_{r,p_n}}M_j\,.\label{e16-kk}
\end{equation}
В частном случае, когда $r\hm=1$, $N_n$~является {\it геометрической
суммой}, свойства которой хорошо изучены для ситуации, в~которой все
величины $M_1,M_2,\ldots$ имеют одинаковое распределение~\cite{Kalashnikov1997}. 
Особо отметим, что здесь не предполагается,
что распределения случайных величин $M_1,M_2,\ldots$ совпадают.
Используя лемму~2, получаем следующее утверждение.

\smallskip

\noindent
\textbf{Теорема~3.}\ \textit{Предположим, что случайные величины~$M_j$,
$j\hm\geqslant1$, удовлетворяют условию статистической устойчивости}:
\begin{equation}
\fr{1}{n^{1/\nu}}\sum\limits_{j=1}^nM_j\longrightarrow 1
\label{e17-kk}
\end{equation}
\textit{по вероятности при $n\to\infty$ с некоторым $\nu>0$. Предположим,
что случайная величина $\overline N_{r,p_n}$ имеет отрицательное
биномиальное распределение $(3)$ с параметрами $r>0$ и $p_n\to 0$ и
при каждом $n\in\mathbb{N}$ независима от последовательности
$\{M_j\}_{j\geqslant1}$. Пусть случайные величины $N_n$ имеют вид $(16)$.
Тогда}
\begin{multline*}
\hspace*{-13.27338pt}\lim\limits_{n\to\infty}\sup\limits_{x>0}\left\vert {\sf
P}\left(p_n^{1/\nu}N_n<x\right)-\fr{\nu}{\Gamma(r)}\int\limits_{0}^{x}z^{\nu
r-1}e^{-z^{\nu}}\,dz\right\vert={}\\
{}=0\,.
\end{multline*}

\smallskip

\noindent
Д\,о\,к\,а\,з\,а\,т\,е\,л\,ь\,с\,т\,в\,о\,.\ \ 
Воспользуемся леммой~2, в~которой для каждого
$n\hm\in\mathbb{N}$ положим $b_n\hm=n^{1/\nu}$, $d_n\hm=p_n^{-1/\nu}$,
$\zeta_n\hm=M_1+\cdots+M_n$, $\tau_n\hm=\overline N_{r,p_n}$. Тогда
$\zeta_{\tau_n}\hm=N_n$ и~в~силу~(\ref{e17-kk}) $\zeta\hm=1$. Очевидно, что
\begin{multline*}
{\sf P}(d_n^{-1}b_{\tau_n}<x)={\sf
P}\left((p_n\tau_n)^{1/\nu}<x\right)={}\\
{}={\sf P}\left(p_n\overline
N_{r,p_n}<x^{\nu}\right)\,,
\end{multline*}
поэтому в~соответствии с леммами~2 и~1
\begin{multline*}
\lim\limits_{n\to\infty}{\sf P}\left(p_n^{1/\nu}N_n<x\right)=
\lim\limits_{n\to\infty}{\sf
P}\left(p_n\overline
N_{r,p_n}<x^{\nu}\right)={}\\
{}=\fr{1}{\Gamma(r)}\int\limits_{0}^{x^{\nu}}z^{r-1}e^{-z}\,dz=
\fr{\nu}{\Gamma(r)}\int\limits _{0}^{x}z^{\nu r-1}e^{-z^{\nu}}\,dz
\end{multline*}
равномерно по $x\hm>0$. Лемма доказана.

\smallskip

В частном случае $r\hm=1$ распределение случайной величины $\overline
N_{r,p_n}$ является геометрическим, а~предельное обобщенное
гам\-ма-рас\-пре\-де\-ле\-ние в~тео\-ре\-ме~3 является распределением Вейбулла 
с~параметром~$\nu$. Таким образом, теорема~3, по сути, представляет
собой закон больших чисел для отрицательных биномиальных случайных
сумм необязательно одинаково распределенных независимых случайных
величин и устанавливает условия сходимости распределений таких сумм
к~обобщенным гам\-ма-рас\-пре\-де\-ле\-ни\-ям. Тем самым оно обобщает известные
утверждения о~сходимости гео\-мет\-ри\-че\-ских сумм (см., например,~\cite{Kalashnikov1997}). 
В~част\-ности, при $r\hm=1$ тео\-ре\-ма~3 описывает
сходимость гео\-мет\-ри\-че\-ских случайных сумм неодинаково распределенных
независимых слагаемых к распределению Вей\-булла.

Пусть $k_n$~--- натуральные числа, фигурирующие в~условии~(\ref{e11-kk}).
Поскольку параметр~$p_n$ в~тео\-ре\-ме~3 может быть произвольным,
полагая $p_n\hm=k_n^{-\nu}$, можно получить конкретный пример ситуации,
в~которой асимптотическим распределением числа слагаемых в~сумме,
фигурирующий в~следствии~1, являет\-ся GG-рас\-пре\-де\-ле\-ние. В~этом
примере одним из важнейших является условие статистической
устойчивости~(\ref{e17-kk}), смысл которого в~том, что интенсивность событий
имеет нетривиальный тренд (затухание или, наоборот, возрастание).



\section{Подгонка моделей обощенных гамма-распределений к~реальным данным об~интенсивности
потоков событий на~финансовых рынках}

В данном разделе будут приведены результаты статистического анализа
конкретных финансовых данных, подтверждающие перспективность
применения GVG-рас\-пре\-де\-ле\-ний в~качестве моде\-лей наблюдаемых
статистических закономерностей. Из общей теории формирования
финансовых индексов, основанной на предельных теоремах для случайных
блуж\-да\-ний со случайными интенсивностями скачков~\cite{Korolev2011-2},
вытекает, что распределения\linebreak (логариф\-мических) приращений хорошо
аппроксимируются дис\-пер\-си\-он\-но-сдви\-го\-вы\-ми смесями нормальных законов.
Этот вывод подтвержден многочисленными работами, в~которых указанные
закономерности хорошо описываются с~по\-мощью конкретных смесей,
например обобщенных гиперболических распределений или их\linebreak частных
случаев~[3, 18--28]. 
Как показано в~теореме~1 и следствии~1,
основным условием (критерием) адекватности той или иной
дис\-пер\-си\-он\-но-сдви\-го\-вой смеси нормальных законов является
адекватность конкретного смешивающего распределения как модели
распределения случайной интенсивности торгов. Здесь на нескольких
примерах будет продемонстрировано, что GG-рас\-пре\-де\-ле\-ния лучше
описывают статистические закономерности поведения случайной
интенсивности торгов, нежели GIG-мо\-де\-ли, и,~следовательно,
обобщенные\linebreak\vspace*{-12pt}

\pagebreak

\end{multicols}

\begin{figure} %fig1
         \vspace*{1pt}
 \begin{center}
 \mbox{%
 \epsfxsize=117.768mm
 \epsfbox{ko2-1.eps}
 }
 \end{center}
 \vspace*{-11pt}
\Caption{Интенсивность торгов по акциям
Сбербанка в~течение дня}
\label{SBER_1}
%\end{figure*}
%\begin{figure*} %fig2
 \vspace*{1pt}
 \begin{center}
 \mbox{%
 \epsfxsize=160.052mm
 \epsfbox{ko2-2.eps}
 }
 \end{center}
 \vspace*{-11pt}
\Caption{Сравнение GG~(\textit{1}) и~GIG~(\textit{2}) мо\-де\-лей по
<<склеенным>> данным~(\textit{а}) и~на одном <<активном>> периоде~(\textit{б})}
\label{SBER_2}
\end{figure}

\begin{multicols}{2}

\noindent
 дисперсионные гам\-ма-рас\-пре\-де\-ле\-ния лучше описывают
статистические закономерности поведения (логарифмических) приращений
соответствующих финансовых индексов.

Рассмотрим интенсивности потоков заявок (ордеров) на торгах акциями
Сбербанка. Эти акции торгуются не очень активно, но торгуются
интересно~--- в~течение дня заметны <<пробелы>> с нулевой
интенсивностью  (рис.~1).



Если формально убрать периоды <<отдыха>> торгующихся, <<склеить>>
оставшиеся интервалы и вычислить интенсивность с окном в~60~с,
то получается картина, представленная на рис.~2,\,\textit{а}, из которой
видно, что GG-рас\-пре\-де\-ле\-ние (кривая~\textit{1}) дает слегка лучшее
приближение, чем GIG (кривая~\textit{2}) ($P$-зна\-че\-ния критерия хи-квад\-рат составляют
соответственно~0,283 и~0,278).



Если же взять отдельно период активности (выделенный на рис.~1
фигурной скобкой снизу) и~вычислить для него интенсивность торгов 
с~окном в~5~с, то преимущество GG-мо\-де\-ли станет не\-оспо\-римым (см.\
рис.~2,\,\textit{б}). Видно, что, хотя визуально GG- и GIG-мо\-де\-ли близки,
$P$-значение для GIG-рас\-пре\-де\-ле\-ния практически равно нулю, поскольку
хвост подогнанного GIG-рас\-пре\-де\-ле\-ния убывает\linebreak\vspace*{-12pt}
\begin{center}  %fig3
\vspace*{-1pt}
\mbox{%
 \epsfxsize=77.404mm
 \epsfbox{ko2-3.eps}
 }



\vspace*{3pt}

\noindent
{{\figurename~3}\ \ \small{Текущая интенсивность потока заявок на
LSE}}
 
\end{center}

\vspace*{6pt}


\addtocounter{figure}{1}

\noindent
 слишком быстро по
сравнению с GG-мо\-делью, для которой $P$-зна\-че\-ние равно~0,393.

Аналогичным образом были также проанализированы очень
высокочастотные данные о~торгах среднеактивным инструментом на
лондонской бирже LSE (London Stock Exchange). Здесь поток заявок гораздо плотнее, данные 
о~текущей интенсивности представлены на рис.~3. Из-за
высокой плотности заявок за все время торгов интенсивность, даже
вычисленная для 5-се\-кунд\-ных окон, не опускается до нуля. На рис.~4,\,\textit{а}
представлены\linebreak\vspace*{-12pt}

\pagebreak

\begin{center}  %fig4
\vspace*{-12pt}
\mbox{%
 \epsfxsize=78.233mm
 \epsfbox{ko2-3-a.eps}
 }

\end{center}

\vspace*{-6pt}

\noindent
{{\figurename~4}\ \ \small{Результаты подгонки GG~(\textit{1}) и~GIG~(\textit{2}) мо\-де\-лей к~данным
об интенсивности потока заявок, вычисленных по 5-се\-кунд\-ным окнам для
интервала длиной 1~ч с~10:00 до~11:00~(\textit{а}) и~к~данным 
об интенсивности потока заявок
для всего торгового дня, вычисленных по 5-се\-кунд\-ным~(\textit{б}) 
и~60-се\-кунд\-ным~(\textit{в}) окнам}}
 


\vspace*{24pt}


\addtocounter{figure}{1}




\noindent
 результаты подгонки GG- и GIG-мо\-де\-лей 
к~данным об интенсивности потока заявок, вычисленных по 5-се\-кунд\-ным
окнам для интервала длиной~1~ч с~10:00 до 11:00. 

На рис.~4,\,\textit{б} и~4,\,\textit{в}
приведены результаты подгонки GG- и GIG-мо\-де\-лей к~данным об
интенсивности потока заявок для всего торгового дня, вычисленных по
5-се\-кунд\-ным~(\textit{б}) и~60-се\-кунд\-ным~(\textit{в}) окнам. Везде заметно
преимущество GG-мо\-де\-лей, подкрепленное б$\acute{\mbox{о}}$льшими $P$-зна\-че\-ниями.



\section{Прогнозирование финансовых рисков с~помощью модифицированного
сеточного метода скользящего разделения обобщенных дисперсионных
гамма-распределений}

Одной из важнейших практических задач на финансовых рынках является
задача прогнозирования рисков, связанных с превышением теми или
иными показателями критических порогов. Помимо непосредственного
исследования распределений любая финансовая организация
заинтересована в~получении бо\-лее-ме\-нее достоверных прогнозов на
основе наблюдаемых данных. Прогнозирование несет в~себе большой
спекулятивный фактор, но некоторые жесткие требования к любому
осмысленному методу прогнозирования известны заранее: метод должен
работать достаточно быстро, чтобы прогноз оставлял время для
принятия решения, и он должен показывать хорошие результаты на
случайно выбранных исторических данных.

Задача прогнозирования рисков сводится к прогнозированию
распределений. Для упрощения задачи оценки и прогнозирования
распределений час\-то используется подход снижения размерности путем
априорного сужения классов допустимых смесей. 
%
В~данном разделе
описывается алгоритм\linebreak прогнозирования параметров
дис\-пер\-си\-он\-но-сдви\-го\-вых смесей (в~част\-ности, оценки рисков) на\linebreak
примере обобщенных дисперсионных гам\-ма-рас\-пре\-де\-ле\-ний. Этот алгоритм
был ранее отработан на\linebreak обобщенных гиперболических распределениях~\cite{KorolevKorchagin2015}.

Возьмем интересующий нас временной ряд и~применим к~нему стандартный
подход разделения смесей нормальных законов с~использованием
скользящего окна. Для этого зафиксируем размер окна~$w$ и сдвиг окна
$s\hm\leqslant w$. Далее на каждом окне применим модифицированный двухэтапный
сеточный метод разделения дис\-пер\-си\-он\-но-сдви\-го\-вых смесей нормальных
законов, предложенный в~работе~\cite{kk2014}.

Без ограничения общности будем считать, что параметр сдвига~$\beta$
(см.~(\ref{e1-kk})) равен нулю.

На первом этапе на положительной полупрямой выделим основную часть
носителя смешивающего распределения, т.\,е.\ ограниченный интервал,
вероятность которого, вычисленная в~соответствии со смешивающим
распределением, практически равна единице. На этот интервал накинем
конечную сетку, содержащую, возможно, очень много {\it известных}
узлов $u_1,\ldots,u_K$. Приблизим искомое обобщенное дисперсионное
гам\-ма-рас\-пре\-де\-ле\-ние конечной смесью нормальных законов:
\begin{multline}
P_{\mathrm{GVG}}(x;\alpha,\sigma,\nu,\kappa,\delta)\approx
\sum_{i=1}^Kp_i\Phi\left(\fr{x-\alpha
u_i}{\sqrt{u_i}}\right)\,,\\
 x\in\mathbb{R}\,.\label{e18-kk}
\end{multline}
В смеси, стоящей в~правой части соотношения~(\ref{e18-kk}), неизвестными
являются только па\-ра\-мет\-ры $p_1,\ldots,p_{K-1},\alpha$. 

Пусть
$x_1,\ldots,x_n$~--- анализируемая выборка значений случайной
величины с~оценива\-емым обобщенным гиперболическим распределением.
Итерационный процесс, определяющий сеточный ЕМ (expectation-maximization) ал\-го\-ритм для данной
задачи, задается следующим образом. Пусть
$p_1^{(m)},\ldots,p_{K-1}^{(m)}, \alpha^{(m)}$~--- оценки параметров
$p_1,\ldots,p_{K-1}$ и $\alpha$ на $m$-й итерации,
$p_K^{(m)}\hm=1-p_1^{(m)}-\cdots-p_{K-1}^{(m)}$. Обозначим
\begin{align*}
\phi_{ij}^{(m)}&=\fr{1}{\sqrt{u_i}}\,\phi\left(\fr{x_j-\alpha^{(m)}u_i}{\sqrt{u_i}}\right)\,;\\
g_{ij}^{(m)}&=\fr{p_i^{(m)}\phi_{ij}^{(m)}}
{\sum\nolimits_{r=1}^Kp_r^{(m)}\phi_{rj}^{(m)}},\
\  i=1,\ldots,K;\ j=1,\ldots,n.
\end{align*}
Тогда, используя стандартные рассуждения, определяющие
вычислительные формулы EM-ал\-го\-рит\-ма для параметров конечной смеси
нормальных законов (см., например,~\cite[разд.~5.3.7--5.3.8]{Korolev2011-2}), 
следует положить
\begin{equation}
p_i^{(m+1)}=\fr{1}{n}\sum\limits_{j=1}^ng_{ij}^{(m)}\,, \enskip
i=1,\ldots,K\,.\label{e19-kk}
\end{equation}
Обозначим $\overline{x}=(1/n)\sum\nolimits_{j=1}^nx_j$. Используя
соотношение~(5.3.24) в~\cite{Korolev2011-2}, с учетом очевидного
равенства $\sum\nolimits_{i=1}^Kg_{ij}^{(m)}\hm=1$ можно заметить, что
уточненная оценка параметра $\alpha$ имеет вид:
\begin{equation}
\alpha^{(m+1)}=\fr{\overline{x}}{\sum\nolimits_{i=1}^Ku_ip_i^{(m+1)}}\,,
\label{e20-kk}
\end{equation}
т.\,е.\ равна отношению генерального выборочного среднего и текущего
эмпирического среднего смешивающего распределения, что вполне
согласуется с соотношением~(\ref{e4-kk}).

Как известно, классический ЕМ-ал\-го\-ритм обладает свойством
монотонности. Поэтому если узлы $u_1,\ldots,u_K$ сетки различны,
неотрицательны и известны, то итерационный процесс~(\ref{e19-kk}),~(\ref{e20-kk})
является монотонным, т.\,е.\ каждая его итерация не уменьшает
целевую сеточную функцию правдоподобия

\noindent
\begin{multline*}
L(p_1,\ldots,p_K,\alpha;x_1,\ldots,x_n)={}\\
{}=
\prod\limits_{j=1}^n\left[\sum\limits_{i=1}^K
\fr{p_i}{\sqrt{u_i}}\,\phi\left(\fr{x_j-\alpha^{(m)}u_i}{\sqrt{u_i}}\right)\right]\,.
\end{multline*}
В~\cite[разд.~5.7.4]{Korolev2011-2} показано, что при каждом
фиксированном значении параметра~$\alpha$ сеточная функция
правдоподобия $L(p_1,\ldots,p_{K-1},\alpha;\,x_1,\ldots,x_n)$
вогнута по аргументам $p_1,\ldots,p_{K-1}$. Поэтому на каждом шаге
итерационного процесса вмес\-то соотношения~(\ref{e3-kk}) можно использовать
любой более быстрый алгоритм максимизации функции
$L(p_1,\ldots,p_{K-1},\alpha^{(m)};\,x_1,\ldots,x_n)$ по переменным
$p_1,\ldots,p_{K-1}$. Например, оценки весов $p_1,\ldots,p_K$ можно
искать методом условного градиента~\cite{Korolev2011-2, kn2010}.

Таким образом, на первом этапе получаются оценки параметра~$\alpha$
и~весов всех узлов~$u_i$ конечной сетки, накинутой на носитель
смешивающего обобщенного гам\-ма-рас\-пре\-де\-ле\-ния
$P_{\mathrm{GG}}(x;\nu,\kappa,\delta)$.

На втором этапе остается применить ка\-кой-ли\-бо стандартный метод
подгонки GG-рас\-пре\-де\-ле\-ния $P_{\mathrm{GG}}(x;\nu,\kappa,\delta)$ 
к~эмпирическим данным типа гистограммы $(u_1, p_1),\ldots, (u_K, p_K)$. 
Например, параметры~$\nu$, $\kappa$ и~$\delta$ можно оценить,
решая задачу наименьших квадратов
\begin{multline*}
\left(\nu^*,\kappa^*,\delta^*\right)={}\\
{}=
\mathop{\mathrm{arg\,min}}\limits_{\nu,\kappa,\delta}
\sum\limits_{i=1}^K\left[p_i-\hspace*{-2mm}
\int\limits_{(u_{i-1}+u_i)/2}^{((u_i+u_{i+1})/2}\hspace*{-8mm}p_{\mathrm{GG}}
(u;\,\nu,\kappa,\delta)\,du\right]^2,
\end{multline*}
где $u_0=0$, $u_{K+1}\hm=\infty$. Или же эти параметры можно найти из
условия
$$
(\nu^*,\kappa^*,\delta^*)=\mathop{\mathrm{arg\,min}}\limits_{\nu,\kappa,\delta}D_{\mathrm{KL}}
\left[p_{\mathrm{GG}}(u;\,\nu,\kappa,\delta),\,h(u)\right]\,,
$$
где $h(u)$~--- гистограмма, построенная по значениям $(u_1, p_1),\ldots, (u_K,
p_K)$:
\begin{multline*}
h(u)={}\\
\!\!{}=\!\begin{cases}\!
0\,, &\  u\leqslant\fr{1}{2}\,u_1;\\
\fr{u_{i+1}-u_{i-1}}{2p_i}, &\!\!\! \!\fr{1}{2}\left(u_{i-1}+u_i\right)<
u\leqslant\fr{1}{2}\left(u_{i}+u_{i+1}\right);\hspace*{-6.32454pt}\\ 
0\,, & u>\fr{1}{2}\left(3u_{K}-u_{K-1}\right),
\end{cases}\!
\end{multline*}
 а $D_{\mathrm{KL}}\left[p_{\mathrm{GG}}(u;\,\nu,\kappa,\delta),\,h(u)\right]$~---
расстояние (дивергенция) Куль\-ба\-ка--Лейб\-лера.

Конкретный алгоритм выбора сетки на первом этапе описан в~работе~\cite{kk2014}.

\begin{figure*} %fig5
 \vspace*{1pt}
 \begin{center}
 \mbox{%
 \epsfxsize=162.687mm
 \epsfbox{ko2-4.eps}
 }
 \end{center}
 \vspace*{-11pt}
 \Caption{Результаты подгонки
GVG~(\textit{1}) и~GH~(\textit{2}) рас\-пре\-де\-ле\-ний к приращениям логарифмов индекса Shanghai
Composite при двух разных положениях скользящего окна}
\label{Shanghai_1}
\end{figure*}

В качестве входных данных для метода прогнозирования используется
результат работы описанного выше двухэтапного сеточного метода
разделения дис\-пер\-си\-он\-но-сдви\-го\-вых смесей нормальных законов, т.\,е.\
ряд оценок параметров распределений, полученных на~$\hat N$
известных окнах с историческими наблюдениями, $\theta_1, \theta_2,
\ldots, \theta_N$, где каждое $\theta_i\hm =
(\alpha_i,\nu_i,\kappa_i,\delta_i)^{\mathrm{T}}$.

Конечная задача прогнозирования~--- получить оценки $\theta_{n+1},
\theta_{n+2}, \ldots$ для окон, которые будут час\-тично или полностью
состоять из будущих наблюдений. Для этого рассмотрим соотношение
$$
\tilde \theta_{i+1} = F_1  \theta_i +  F_2  \theta_{i-1} + \cdots +
F_r \theta_{i-r+1}\,,
$$
где $r\in\N$~--- заранее фиксированный параметр, имеющий смысл
порядка прогноза; $F_j \hm\in \mathbb{R}^{5 \times 5}$~--- мат\-ри\-цы-рег\-рес\-соры.

Процедура поиска параметров модели имеет вид
\begin{equation}
(F_1, \ldots, F_r) = \mathop{\mathrm{arg\,min}} \sum\limits_{i=r+1}^{N-1}\left(
\theta_{i+1}- \tilde \theta_{i+1} \right)^2\!.\! \label{e21-kk}
\end{equation}

По сути, имеем авторегрессионную модель порядка~$r$, где поиск
матриц~$F_j$ (обучение модели) производится с использованием
минимизации суммарного RSS (residual sum of squares)
по $\hat N-r$ предсказаниям модели на
известных данных. Для простоты обозначений примем $N \hm= \hat N - r$.

Соотношение~(\ref{e21-kk}) представляет собой разновидность линейной
регрессии, и ее программная реализация не представляется сложной.
Более того, во многих статистических пакетах есть встроенные функции
расчета матриц~$F_j$ для случаев $r\hm=1, 2$.


В работе~\cite{kya2015} детально описаны особенности использования
модифицированного сеточного алгоритма и приведены практические
советы для получения входного ряда $\theta_1, \theta_2, \ldots,
\theta_n$.

В качестве исходных данных возьмем индекс Shanghai Composite~---
основной индикатор китайской биржи, включающий в~себя все
торгующиеся на этой бирже компании. Индекс является ключевым для
китайского рынка, поэтому он достаточно часто подвергается анализу 
и~попыткам предсказания своей динамики.

Будет рассматриваться изменение логарифмов значений индекса (цены) 
с~частотой (тиком), равной~1~мин, на протяжении трех рабочих дней,
начиная с~открытия биржи~5~января 2015~г. Размер окна установим
равным~3~ч: $w\hm=180$, сдвиг окна минимальный~--- одно наблюдение:
$s\hm=1$ тик.

Для начала проверим целесообразность использования именно
GVG-моделей для описания статистических закономерностей поведения
при\-ра\-щений логарифмов индекса Shanghai Composite. Результаты этой
проверки иллюстрируются на рис.~5, где приведены результаты подгонки
GVG и GH (generalized hyperbolic) рас\-пре\-де\-ле\-ний к указанным приращениям при двух разных
положениях скользящего окна. Видно, что оба распределения
удовлетворительно описывают данные, но $P$-зна\-че\-ние для
GVG-рас\-пре\-де\-ле\-ния в~обоих случаях выше.
Аналогичная проверка была произведена для~410~положений окна. 
В~403~случаях $P$-зна\-че\-ние для GVG-рас\-пре\-де\-ле\-ния оказалось выше, чем для
GH-рас\-пре\-де\-ления.





Применяя модифицированный сеточный алгоритм к исходным данным,
получаем ряд оценок параметров $\theta_i \hm=
(\alpha_i,\nu_i,\kappa_i,\delta_i)^{\mathrm{T}}$. Так как алгоритм устойчив 
к~входным данным, а сдвиг окна минимальный, ряд~$\theta_i$ дает
сглаженную картину. Далее применяется алгоритм прогнозирования,
описанный в~\cite{KorolevKorchagin2015}. Результаты прогнозирования
для разных временн$\acute{\mbox{ы}}$х горизонтов представлены на рис.~6, где
показано изменение истинного и прогнозируемого распределения 
с~расширением горизонта прогно-\linebreak\vspace*{-12pt}

\pagebreak

\end{multicols}

\begin{figure} %fig6
\vspace*{1pt}
 \begin{center}
 \mbox{%
 \epsfxsize=161.931mm
 \epsfbox{ko2-5.eps}
 }
 \end{center}
 \vspace*{-11pt}
\Caption{Прогнозируемое~(\textit{1}) и~истинное~(\textit{2})
распределения для следующих горизонтов: (\textit{a})~10~мин, $T+10$; 
(\textit{б})~30~мин, $T+30$; (\textit{в})~1~ч, $T+60$; 
(\textit{г})~1,5~ч, $T+90$; (\textit{д})~2~ч, $T+120$; (\textit{е})~3~ч, $T+180$}
\label{T+120}
\end{figure}

\begin{multicols}{2}



\noindent
зирования, а также приведены значения
прогнозированных и <<истинных>> параметров указанных распределений.



Особый интерес представляет сравнение динамики отдельных параметров
распределения прогноза и~истинных исторических значений без
использования самого вида распределения. На рис.~7 отдельно показаны
прогнозируемые значения параметра~$\nu$ (после момента времени
$T\hm=100$) и~его истинные значения. 

Обратим внимание, что и~истинные,
и~прогнозируемые значения па\-ра\-мет\-ра~$\nu$ лежат между нулем 
и~единицей, т.\,е.\ подогнанные распределения занимают промежуточное
место между законами с~экс-\linebreak\vspace*{-12pt}

\begin{center}  %fig7
\vspace*{-6pt}
\mbox{%
 \epsfxsize=77.88mm
 \epsfbox{ko2-6.eps}
 }

\end{center}

%\vspace*{3pt}

\noindent
{{\figurename~7}\ \ \small{Прогнозируемые~(\textit{1}) и~истинные~(\textit{2}) значения параметра~$\nu$}}
 


\vspace*{6pt}


\addtocounter{figure}{1}



\begin{center}  %fig8
\vspace*{-1pt}
\mbox{%
 \epsfxsize=78.102mm
 \epsfbox{ko2-7.eps}
 }

\end{center}

%\vspace*{3pt}

\noindent
{{\figurename~8}\ \ \small{Прогнозируемые значения медианы~(\textit{1}),
математического значения~(\textit{2}) и~квантилей логарифмических приращений~(\textit{3})
индекса Shanghai Composite, а~также истинные значения индекса~(\textit{4}) и~квантилей~(\textit{5})}}
 


\vspace*{17pt}


\addtocounter{figure}{1}


\noindent
поненциально убывающими хвостами 
и~распределениями с~хвостами, убывающими степенн$\acute{\mbox{ы}}$м образом.




Наконец, на рис.~8 приведены прогнозируемые значения
базовых числовых характеристик процесса (среднее, медиана, квантили
порядков~0,025 и~0,975), а~также истинные значения процесса.


На основании полученных результатов можно сделать вывод, что модель
типа GVG-рас\-пре\-де\-лений хорошо подходит для прогнозирования как
минимум на~2~ч вперед. Важно отметить, что найден\-ная модель
достаточно точно оценивает хвос\-ты распределений, позволяя полагаться
на полученные прогнозы при оценке рисков.

При практическом использовании модели на поступающих в~реальном
времени данных постепенно становятся доступными новые исходные
параметры, что позволяет проводить повторное обуче\-ние модели
(пересчитывая регрессионные мат\-ри\-цы). Это особенно удобно, поскольку
поиск необходимых мат\-риц происходит очень быстро и не представляет
большой вычислительной слож\-ности.

{\small\frenchspacing
 {%\baselineskip=10.8pt
 \addcontentsline{toc}{section}{References}
 \begin{thebibliography}{99}
\bibitem{BN1977} %1
\Au{Barndorff-Nielsen~O.\,E.} Exponentially decreasing distributions
for the logarithm of particle size~// Proc. Roy. Soc. Lond. A,
1977. Vol.~353. P.~401--419.

\bibitem{BN1982} %2
\Au{Barndorff-Nielsen~O.\,E., Kent~J., \mbox{S{\!\ptb{\!\o}}\,ren}\-sen~M.} Normal
variance-mean mixtures and $z$-distributions~// Int.
Statist. Rev., 1982. Vol.~50. No.~2. P.~145--159.

\bibitem{BN1978} %3
\Au{Barndorff-Nielsen O.\,E.} Hyperbolic distributions and
distributions of hyperbolae~// Scand. J.~Statist., 1978. Vol.~5.
P.~151--157.



\bibitem{KorolevSokolov2012} %4
\Au{Королев В.\,Ю., Соколов И.\,А.} Скошенные распределения
Стьюдента, дисперсионные гам\-ма-рас\-пре\-де\-ле\-ния и их обобщения как
асимптотические аппроксимации~// Информатика и~её применения, 2012.
Т.~6. Вып.~1. С.~2--10.

\bibitem{Stacy1962} %5
\Au{Stacy E.\,W.} A~generalization of the gamma
distribution~// Ann. Math. Stat., 1962. Vol.~33.
P.~1187--1192.

\bibitem{LaherrereSornette1998} %6
\Au{Laherr\!{\!\ptb{\!\`e}}re J., Sornette D.}
Stretched exponential distributions in nature and economy: ``Fat
tails'' with characteristic scales~// Eur. Phys. J.~B,
1998. Vol.~2. P.~525--539.

\bibitem{Sornette_et_al2005} 
\Au{Malevergne Y., Pisarenko V.,
Sornette~D.} Empirical distributions of stock returns: Between the
stretched exponential and the power law?~// Quantitative Finance,
2005. Vol.~5. P.~379--401.

\bibitem{Sornette_et_al2006} 
\Au{Malevergne Y., Pisarenko~V.,
Sornette~D.} On the power of generalized extreme value (GEV) and
generalized Pareto distribution (GDP) estimators for empirical
distributions of stock returns~// Appl. Financial Economics, 2006.
Vol.~16. P.~271--289.

\bibitem{AntonovKoksharov2006} 
\Au{Антонов С.\,Н., Кокшаров~С.\,Н.} Об
асимптотическом поведении хвостов масштабных смесей нормальных
распределений~// Статистические методы оценивания и проверки
гипотез.~--- Пермь: Изд-во Пермского ун-та, 2006. С.~90--105.

\bibitem{Korolev2013} \Au{Королев В.\,Ю.} Обобщенные гиперболические законы
как предельные для распределений случайных сумм~// Теория
вероятностей и ее применения, 2013. Т.~58. Вып.~1. С. 117--132.

\bibitem{GnedenkoKolmogorov1949}
\Au{Гнеденко Б.\,В., Колмогоров~А.\,Н.} Предельные распределения для
сумм независимых случайных величин.~--- М.-Л.: ГИТТЛ, 1949. 264~с.

\bibitem{KorolevZaks2013} \Au{Закс Л.\,М., Королев~В.\,Ю.} Обобщенные
дисперсионные гамма-распределения как предельные для случайных сумм~// 
Информатика и её применения, 2013. Т.~7. Вып.~1. С. 105--115.

\bibitem{Korolev1994} \Au{Королев В.\,Ю.} Сходимость случайных
последовательностей с независимыми случайными индексами. I~// Теория
вероятностей и ее применения, 1994. Т.~39. Вып.~2. С.~313--333.

\bibitem{Korolev1995} \Au{Королев В.\,Ю.} Сходимость случайных
последовательностей с независимыми случайными индексами. II~//
Теория вероятностей и ее применения, 1995. Т.~40. Вып.~4. С.~907--910.

\bibitem{BeningKorolev2002} \Au{Bening V.\,E., Korolev V.\,Yu.}
Generalized Poisson models and their applications in insurance and
finance.~--- Utrecht: VSP, 2002. 456~p.

\bibitem{Kalashnikov1997} 
\Au{Kalashnikov V.\,V.} Geometric sums: Bounds for rare events with
applications.~--- Dordrecht: Kluwer Academic Publs., 1997. 270~p.

\bibitem{Korolev2011-2}
\Au{Королев В.\,Ю.} Ве\-ро\-ят\-ност\-но-ста\-ти\-сти\-че\-ские методы декомпозиции
волатильности хаотических процессов.~--- М.: Изд-во Московского
ун-та, 2011. 510~с.

\bibitem{BN1979}
\Au{Barndorff-Nielsen O.\,E.} Models for non-Gaussian variation,
with applications to turbulence~// Proc. Roy. Soc. Lond. A,
1979. Vol.~368. P.~501--520.

\bibitem{MadanSeneta1990}
\Au{Madan D.\,B., Seneta E.} The variance gamma $($V.G.$)$ model
for share market return~// J.~Business, 1990. Vol.~63. P.~511--524.

\bibitem{EberleinKeller1995}
\Au{Eberlein E., Keller~U.} Hyperbolic distributions in finance~//
Bernoulli, 1995. Vol.~1. No.\,3. P. 281--299.

\bibitem{Prause1997}
\Au{Prause K.} Modeling financial data using generalized hyperbolic
distributions.~--- Freiburg: Universit$\ddot{\mbox{a}}$t Freiburg, Institut
f$\ddot{\mbox{u}}$r Mathematische Stochastic, 1997. Preprint No.\,48. 14~p.

\bibitem{CarrMadanChang1998}
\Au{Carr P.\,P., Madan D.\,B., Chang~E.\,C.} The Variance Gamma
process and option pricing~// Eur. Finance Rev., 1998. Vol.~2.
P.~79--105.

\bibitem{EberleinKellerPrause1998}
\Au{Eberlein E., Keller~U., Prause~K.} New insights into smile,
mispricing and value at risk: The hyperbolic model~// J.~Business, 1998. Vol.~71. P.~371--405.

\bibitem{BarndorffNielsen1998}
\Au{Barndorff-Nielsen O.\,E.} Processes of normal inverse Gaussian
type~// Finance Stochastics, 1998. Vol.~2. P.~41--68.

\bibitem{EberleinPrause1998}
\Au{Eberlein E., Prause~K.} The generalized hyperbolic model:
Financial derivatives and risk measures.~--- Freiburg:
Universit$\ddot{\mbox{a}}$t Freiburg, Institut f$\ddot{\mbox{u}}$r 
Mathematische Stochastic, 1998. Preprint No.\,56. 168~p.

\bibitem{Shiryaev1998}
\Au{Ширяев А.\,Н.} Основы стохастической финансовой математики. Т.~1. Факты. Модели.~--- 
М.: Фазис, 1998. 512~с.

\bibitem{Eberlein1999}
\Au{Eberlein E.} Application of generalized hyperbolic Levy
motions to finance.~--- Freiburg: Universit$\ddot{\mbox{a}}$t Freiburg, Institut
f$\ddot{\mbox{u}}$r Mathematische Stochastic, 1999. Preprint No.\,64. P.~319--336.

\bibitem{BNBlaesildSchmiegel2004}
\Au{Barndorff-Nielsen O.\,E., Bl$\protect\ae$sild P., Schmiegel J.}   
A~parsimonious and universal description of turbulent velocity
increments~// Eur. Phys.~J., 2004. Vol.~B41. P.~345--363.

\bibitem{KorolevKorchagin2015} 
\Au{Корчагин А.\,Ю., Королев~В.\,Ю.} Прогнозирование финансовых
рисков с~по\-мощью модифицированного сеточного метода скользящего
разделения дис\-пер\-си\-он\-но-сдви\-го\-вых смесей нормальных законов~//
Вестник Карагандинского ун-та, 2015. Т.~78. Вып.~2.\linebreak С.~65--74.

\bibitem{kk2014} \Au{Королев В.\,Ю., Корчагин~А.\,Ю.} Модифицированный
сеточный метод разделения дис\-пер\-си\-он\-но-сдви\-го\-вых смесей нормальных
законов~// Информатика и~её применения, 2014. Т.~8. Вып.~4. С.~11--19.

\bibitem{kn2010} \Au{Королев В.\,Ю., Назаров~А.\,Л.} Разделение смесей
вероятностных распределений при помощи сеточных методов моментов 
и~максимального правдоподобия~// Автоматика и~телемеханика, 2010.
Вып.~3. С.~98--116.

\bibitem{kya2015} \Au{Корчагин А.\,Ю., Ярошенко~И.\,И.} О~практическом
использовании модифицированного сеточного метода разделения
дис\-пер\-си\-он\-но-сдви\-го\-вых смесей нормальных законов~// Информатика и~её
применения, 2015. Т.~9. Вып.~1. С.~2--10.
 \end{thebibliography}

 }
 }

\end{multicols}

\vspace*{-3pt}

\hfill{\small\textit{Поступила в~редакцию 10.11.15}}

\vspace*{8pt}

%\newpage

%\vspace*{-24pt}

\hrule

\vspace*{2pt}

\hrule

\vspace*{8pt}

\def\tit{MODELING OF~STATISTICAL REGULARITIES IN~FINANCIAL MARKETS
BY~GENERALIZED VARIANCE GAMMA DISTRIBUTIONS}

\def\titkol{Modeling of~statistical regularities in~financial markets
by~generalized variance gamma distributions}

\def\aut{V.\,Yu.~Korolev$^{1,2}$,  A.\,Yu.~Korchagin$^2$, and~I.\,A.~Sokolov$^3$}

\def\autkol{V.\,Yu.~Korolev,  A.\,Yu.~Korchagin, and~I.\,A.~Sokolov}

\titel{\tit}{\aut}{\autkol}{\titkol}

\vspace*{-9pt}


\noindent
$^1$Faculty of Computational Mathematics and Cybernetics, 
 M.\,V.~Lomonosov Moscow State University, 1-52\linebreak
 $\hphantom{^1}$Leninskiye Gory, GSP-1, Moscow 119991, 
 Russian Federation
 
\noindent
 $^2$Institute of Informatics Problems, 
 Federal Research Center ``Computer Science and Control'' of the Russian\linebreak
  $\hphantom{^1}$Academy of 
 Sciences, 44-2 Vavilov Str., Moscow 119333,  Russian Federation
 
 \noindent
 $^3$Federal Research Center ``Computer Science and Control'' of the Russian Academy of 
 Sciences, 44-2 Vavilov\linebreak
  $\hphantom{^1}$Str., Moscow 119333,  Russian Federation


\def\leftfootline{\small{\textbf{\thepage}
\hfill INFORMATIKA I EE PRIMENENIYA~--- INFORMATICS AND
APPLICATIONS\ \ \ 2015\ \ \ volume~9\ \ \ issue\ 4}
}%
 \def\rightfootline{\small{INFORMATIKA I EE PRIMENENIYA~---
INFORMATICS AND APPLICATIONS\ \ \ 2015\ \ \ volume~9\ \ \ issue\ 4
\hfill \textbf{\thepage}}}

\vspace*{3pt}





\Abste{Some aspects of the application of generalized
variance gamma distributions for modeling statistical regularities
in financial markets are discussed. The paper describes elementary properties
of generalized variance gamma distributions as special
normal variance-mean mixtures in which mixing distributions are
the generalized gamma laws. Limit theorems for sums of a random number
of independent random variables are presented that are analogs of
the law of large numbers and the central limit theorem. These
theorems give grounds for the possibility of using generalized
variance gamma distributions as asymptotic approximations. The
paper presents\linebreak\vspace*{-12pt}}

\Abstend{the results of practical fitting of generalized variance
gamma distributions to real data concerning the behavior of
financial indexes as well as of fitting generalized gamma
distributions to the observed intensities of information flows in
contemporary financial information systems. The results of
comparison of generalized gamma models with generalized hyperbolic
models demonstrate the superiority of the former over the latter.
The methods for parameter estimation of generalized gamma models 
are also discussed as well as their application for predicting processes
in financial markets.}

\KWE{random sum; normal mixture; normal variance-mean
mixture; generalized hyperbolic distribution; generalized
variance-gamma distribution; generalized gamma distribution; law of
large numbers; central limit theorem}

\DOI{10.14357/19922264150402}

\Ack
\noindent
The work was partly supported by the Russian Foundation for Basic Research 
(project No.\,14-07-00041a).





%\vspace*{3pt}

  \begin{multicols}{2}

\renewcommand{\bibname}{\protect\rmfamily References}
%\renewcommand{\bibname}{\large\protect\rm References}

{\small\frenchspacing
 {%\baselineskip=10.8pt
 \addcontentsline{toc}{section}{References}
 \begin{thebibliography}{99}

\bibitem{BN1977eng-1}  %1
\Aue{Barndorff-Nielsen, O.\,E.} 1977.
Exponentially decreasing distributions for the logarithm of particle size.
\textit{Proc. Roy. Soc. Lond. A} 353:401--419.

\bibitem{BN1982eng} %2
\Aue{Barndorff-Nielsen, O.\,E., J.~Kent, and M.\,M.~\mbox{S\!{\ptb{\o}}ren}\-sen}. 1982.
Normal variance-mean mixtures and \mbox{$z$-distributions}.
\textit{Int. Statist. Rev.} 50(2):145--159.

\bibitem{BN1978eng-1} %3
\Aue{Barndorff-Nielsen, O.\,E.} 1978.
Hyperbolic distributions and distributions of hyperbolae.
\textit{Scand. J.~Statist.} 5:151--157.



\bibitem{ks2012eng} 
\Aue{Korolev, V.\,Yu., and I.\,A.~Sokolov}. 2012.
Skoshennye raspredeleniya St'yudenta, dispersionnye gamma-raspredeleniya 
i~ikh obobshcheniya kak asimptoticheskie approksimatsii
[Skewed Student's distributions, variance gamma distributions and their 
generalizations as asymptotic approximations].
\textit{Informatika i~ee Primeneniya}~--- \textit{Inform. Appl.} 6(1):2--10.

\bibitem{Stacy1962eng} \Aue{Stacy, E.\,W.} 1962.
A~generalization of the gamma distribution.
\textit{Ann. Math. Stat.} 33:1187--1192.

\bibitem{LaherrereSornette1998eng} 
\Aue{Laherr\mbox{{\ptb{\!\`e}\!}\,}re, J., and D.~Sornette}. 1998.
Stretched exponential distributions in nature and economy: ``Fat tails'' 
with characteristic scales.
\textit{Eur. Phys. J.~B} 2:525--539.

\bibitem{Sornette_et_al2005eng} 
\Aue{Malevergne, Y., V.~Pisarenko, and D.~Sornette}. 2005.
Empirical distributions of stock returns: Between the
stretched exponential and the power law? 
\textit{Quantitative Finance} 5:379--401.

\bibitem{Sornette_et_al2006eng} 
\Aue{Malevergne, Y., V.~Pisarenko, and D.~Sornette}. 2006.
On the power of generalized extreme value (GEV) and
generalized Pareto distribution (GDP) estimators for empirical
distributions of stock returns. \textit{Appl. Financial Economics}
16:271--289.

\bibitem{AntonovKoksharov2006eng} 
\Aue{Antonov, S.\,N., and S.\,N.~Koksharov}. 2006.
Ob asimptoticheskom povedenii khvostov masshtabnykh smesey normalnykh raspredeleniy
[On asimptotic behavior of tails of normal variance mixtures].
\textit{Statisticheskie metody otsenivaniya i~proverki gipotez} 
[Statistical methods for estimation and hypothesis validation]. 
Perm': Perm' University Press. 2--10.

\bibitem{Korolev2013eng} \Aue{Korolev, V.\,Yu.} 2013.
Obobshchennye giperbolicheskie raspredeleniya kak predel'nye dlya sluchaynykh summ
[Generalized hyperbolic distributions as limiting for random sums]
\textit{Theory Probab. Appl.} 58(1):117--132.

\bibitem{GnedenkoKolmogorov1949eng} 
\Aue{Gnedenko, B.\,V., and A.\,N.~Kolmogorov}. 1949. 
\textit{Predel'nye raspredeleniya dlya summ nezavisimykh sluchaynykh velichin} 
[Limit distibutions for sums of independent random variables]. Moscow--Leningrag: GITTL.
264~p.

\bibitem{KorolevZaks2013eng} \Aue{Zaks, L.\,M., and V.\,Yu.~Korolev}. 2013.
Obobshchennye dispersionnye gamma-raspredeleniya kak predel'nye dlya sluchaynykh summ
[Generalized variance gamma distributions as limiting for random sums].
\textit{Informatika i~ee Primeneniya}~--- \textit{Inform. Appl.} 7(1):105--115.

\bibitem{Korolev1994eng} \Aue{Korolev, V.\,Yu.} 1994. 
Skhodimost' sluchaynykh posledovatel'nostey s~nezavisimymi sluchaynymy indeksami.~I 
[Convergence of random sequences with independent random indeces.~I]. 
\textit{Theory Probab. Appl.} 39(2):313--333.

\bibitem{Korolev1995eng} 
\Aue{Korolev, V.\,Yu.} 1995. Skhodimost' sluchaynykh posledovatel'nostey 
s~nezavisimymi sluchaynymy indeksami.~II 
[Convergence of random sequences with independent random indeces.~II]. 
\textit{Theory Probab. Appl.} 40(4):907--910.

\bibitem{BeningKorolev2002eng} 
\Aue{Bening, V.\,E., and V.\,Yu.~Korolev}. 2002. 
\textit{Generalized Poisson models and their applications in insurance and finance}. 
Utrecht: VSP. 456~p.

\bibitem{Kalashnikov1997_eng} 
\Aue{Kalashnikov, V.\,V.} 1997. \textit{Geometric sums: Bounds for 
rare events with applications}. {Dordrecht: Kluwer Academic Publs}. 270~p.

\bibitem{k2011eng} \Aue{Korolev, V.\,Yu.} 2011.
\textit{Veroyatnostno-statisticheskie metody dekompozitsii volatil'nosti 
khaoticheskikh protsessov}
[Probabilistic and statistical methods for the decomposition of volatility of 
chaotic processes].
Moscow: Moscow University Press. 510~p.

\bibitem{BN1979eng} \Aue{Barndorff-Nielsen, O.\,E.} 1979. 
Models for non-Gaussian variation,
with applications to turbulence. \textit{Proc. Roy. Soc. Lond. A} 368:501--520.

\bibitem{MadanSeneta1990eng} \Aue{Madan, D.\,B., and E.~Seneta}. 1990. 
The variance gamma $($V.G.$)$ model
for share market return. \textit{J.~Business} 63:511--524.

\bibitem{EberleinKeller1995eng} \Aue{Eberlein, E., and U.~Keller}. 1995. 
Hyperbolic distributions in finance. \textit{Bernoulli} 1(3):281--299.

\bibitem{Prause1997eng} 
\Aue{Prause, K.} 1997. Modeling financial data using generalized hyperbolic
distributions. 
{Freiburg: Universit$\ddot{\mbox{a}}$t Freiburg, Institut f$\ddot{\mbox{u}}$r 
Mathematische Stochastic}. Preprint No.\,48. 14~p.

\bibitem{CarrMadanChang1998eng} 
\Aue{Carr, P.\,P., D.\,B. Madan, and E.\,C.~Chang}. 1998. The Variance Gamma
process and option pricing. \textit{Eur. Finance Rev.} 2:79--105.

\bibitem{EberleinKellerPrause1998eng} 
\Aue{Eberlein, E., U. Keller, and K.~Prause}. 1998. New insights into smile,
mispricing and value at risk: The hyperbolic model. \textit{J.~Business} 71:371--405.

\bibitem{BarndorffNielsen1998eng} 
\Aue{Barndorff-Nielsen, O.\,E.} 1998. Processes of normal inverse Gaussian
type. \textit{Finance Stochastics} 2:41--68.

\bibitem{EberleinPrause1998eng} 
\Aue{Eberlein, E., and K.~Prause}. 1998. The generalized hyperbolic model:
Financial derivatives and risk measures. 
{Freiburg: Universit$\ddot{\mbox{a}}$t Freiburg, Institut f$\ddot{\mbox{u}}$r 
Mathematische Stochastic}. Preprint No.\,56. 168~p.

\bibitem{Shiryaev1998eng} 
\Aue{Shiryaev, A.\,N.} 1998. \textit{Osnovy stokhasticheskoy finansovoy matematiki. Т.~1. 
Fakty. Modeli}
[Basics of stochastic financial mathematics. Vol.~1. Facts. Models]. 
{Moscow: Fazis}. 512~p.

\bibitem{Eberlein1999eng} 
\Aue{Eberlein, E.} 1999. Application of generalized hyperbolic Levy
motions to finance. 
{Freiburg: Universit$\ddot{\mbox{a}}$t Freiburg, Institut f$\ddot{\mbox{u}}$r 
Mathematische Stochastic}. Preprint No.\,64. 319--336.

\bibitem{BNBlaesildSchmiegel2004eng} 
\Aue{Barndorff-Nielsen, O.\,E., P.~Bl$\protect\ae$sild, and J.~Schmiegel}. 2004.
A~parsimonious and universal description of turbulent velocity increments. 
\textit{Eur. Phys.~J.} B41:345--363.

\bibitem{KorolevKorchagin2015eng} 
\Aue{Korchagin, A.\,Yu., and V.\,Yu.~Korolev}. 2015. 
Prognozirovanie finansovykh riskov s~pomoshch'yu mo\-di\-fi\-tsi\-ro\-van\-no\-go setochnogo 
metoda skol'zyashchego razdeleniya dispersionno-sdvigovykh smesey normal'nykh zakonov 
[Forecasting financial risks using modified grid method of moving window decomposition 
of variance-mean normal mixtures]. 
\textit{Vestnik Karagandinskogo Un-ta} 78(2):65--74.

\bibitem{kk2014eng} 
\Aue{Korolev, V.\,Yu., and A.\,Yu.~Korchagin}. 2014. Modi\-fi\-tsi\-ro\-van\-nyy 
setochnyy metod razdeleniya dispersionno-sdvigovykh smesey normal'nykh zakonov 
[Modified grid-based method for decomposition of variance-mean\linebreak normal mixtures]. 
\textit{Informatika i~ee Primeneniya}~---
\textit{Inform. Appl.} 8(4):11--19.

\bibitem{kn2010eng} 
\Aue{Korolev, V.\,Yu., and A.\,L.~Nazarov}. 2010.
{Razdelenie smesey veroyatnostnykh raspredeleniy pri pomoshchi setochnykh 
metodov momentov i~maksimal'nogo pravdopodobiya} [Separation of mixtures using 
grid moment-based methods and maximum likelihood].
\textit{Avtomatika i~Telemekhanika} [Automatics Telemechanics] 3:98--116.

\bibitem{kya2015eng} 
\Aue{Korchagin, A.\,Yu., and  I.\,I.~Yaroshenko}. 2015.
O~prak\-ti\-che\-skom ispol'zovanii modifitsirovannogo setochnogo metoda razdeleniya 
dispersionno-sdvigovykh smesey normal'nykh zakonov [On practical usage of modified 
grid method of decomposition of normal variance-mean mixtures.] 
\textit{Informatika i~ee Primeneniya}~--- \textit{Informatics Appl.} 9(1):2--10.
\end{thebibliography}

 }
 }

\end{multicols}

\vspace*{-3pt}

\hfill{\small\textit{Received November 10, 2015}}

\Contr


\noindent
\textbf{Korolev Victor Yu.} (b.\ 1954)~---
Doctor of Science in physics and mathematics, professor, Head of the Department of 
Mathematical Statistics, Faculty of Computational Mathematics and Cybernetics, 
 M.\,V.~Lomonosov Moscow State University, 1-52 Leninskiye Gory, GSP-1, Moscow 119991, 
 Russian Federation; leading scientist, Institute of Informatics Problems, 
 Federal Research Center ``Computer Science and Control'' of the Russian Academy of 
 Sciences, 44-2 Vavilov Str., Moscow 119333,  Russian Federation; 
 vkorolev@cs.msu.su

\vspace*{3pt}

\noindent
\textbf{Korchagin Alexander Yu.} (b.\ 1989)~---
junior scientist, Faculty of Computational Mathematics and Cybernetics, 
M.\,V.~Lomonosov Moscow State University, 1-52 Leninskiye Gory, GSP-1, Moscow 119991, 
Russian Federation; sasha.korchagin@gmail.com


\vspace*{3pt}

\noindent
\textbf{Sokolov Igor A.} (b.\ 1954)~--- Academician of the Russian Academy of Sciences, 
Doctor of Science in technology, Director, Federal Research Center 
``Computer Science and Control'' of the Russian Academy of Sciences; 
isokolov@ipiran.ru

\label{end\stat}


\renewcommand{\bibname}{\protect\rm Литература}