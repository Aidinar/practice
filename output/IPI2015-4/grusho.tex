\def\stat{grusho}

\def\tit{ОЦЕНКИ СКОРОСТИ ПЕРЕДАЧИ ИНФОРМАЦИИ И~ПРОПУСКНОЙ 
СПОСОБНОСТИ В~СКРЫТЫХ~КАНАЛАХ~С~МЕТКАМИ$^*$}

\def\titkol{Оценки скорости передачи информации и~пропускной 
способности в~скрытых каналах с~метками}

\def\aut{А.\,А.~Грушо$^1$, Н.\,А.~Грушо$^2$, Е.\,Е.~Тимонина$^3$}

\def\autkol{А.\,А.~Грушо, Н.\,А.~Грушо, Е.\,Е.~Тимонина}

\titel{\tit}{\aut}{\autkol}{\titkol}

{\renewcommand{\thefootnote}{\fnsymbol{footnote}} \footnotetext[1]
{Работа выполнена при частичной финансовой поддержке РФФИ (проект 13-01-00215).}}


\renewcommand{\thefootnote}{\arabic{footnote}}
\footnotetext[1]{Институт проблем информатики Федерального исследовательского
центра <<Информатика и~управление>> Российской академии наук,
grusho@yandex.ru}
\footnotetext[2]{Институт проблем информатики Федерального исследовательского
центра <<Информатика и~управление>> Российской академии наук,
info@itake.ru}
\footnotetext[3]{Институт проблем информатики Федерального исследовательского
центра <<Информатика и~управление>> Российской академии наук,
eltimon@yandex.ru}

  \Abst{Работа посвящена оценкам пропускной способности и~скорости передачи 
информации в~скрытых каналах специального вида. Эти скрытые каналы порождаются 
с~по\-мощью сигналов (меток), не несущих самостоятельной информации, но легко 
выделяемых на приемном конце. Такие метки служат своего рода запретами с~точки зрения 
допустимых значений всех параметров, связанных с~передачей легальной информации. 
Скрываемая информация кодируется длинами выделяемых метками фрагментов данных 
в~легальной передаче.}
  
\KW{скрытый канал; теоретико-вероятностные модели скрытых каналов; скрытые каналы, 
порожденные метками; пропускная способность скрытого канала; скорость передачи} 

\DOI{10.14357/19922264150409}



\vskip 14pt plus 9pt minus 6pt

\thispagestyle{headings}

\begin{multicols}{2}

\label{st\stat}
     
\section{Введение}
  
  Впервые понятие скрытого канала было введено в~работе Лэмпсона~[1] 
в~1973~г. Канал называется скрытым, если он не проектировался, не 
предполагался для передачи информации в~электронной системе обработки 
данных. 
  
  Свойствам скрытых каналов посвящено большое количество научных работ. 
История иссле\-дований скрытых каналов отражена, например, в~обзоре~[2]. 
Основные проблемы, связанные со скрытыми каналами, следующие:
  \begin{itemize}
\item разработка способа передачи и~получения скрываемой информации;
\item обеспечение невыявляемости (<<невидимости>>) скрытого канала;
\item оценка пропускной способности и~скорости передачи информации 
по скрытому каналу;
\item обеспечение устойчивости скрытой передачи информации к методам 
защиты от скрытых каналов. 
\end{itemize}

  Рассматриваемый в~статье класс скрытых каналов относится к каналам по 
времени~[2]. В~каналах этого типа используются метки, т.\,е.\ не несущие 
самостоятельной информации сигналы, позволяющие выделять (помечать) 
символы в~легальной передаче данных. Метки могут быть отделены от 
символов легальной передачи данных, но появление меток должно 
синхронизироваться с~появлением символов легальной передачи~[3, 4].
  
  Для организации скрытых каналов в~компьютерных системах и~сетях 
необходимо, как правило, наличие про\-граммно-ап\-па\-рат\-ных агентов 
(сущностей) отправителя и~получателя скрываемой информации. В~рамках 
данной статьи предполагается наличие таких агентов. Кроме задач по 
организации взаимодействия с~помощью скрытого канала агенты должны 
обладать единым способом кодирования информации на передающем конце 
и~декодирования на приемном конце. 
  
  Передача информации по скрытому каналу, использующему метки, основана 
на том, что агент-от\-пра\-ви\-тель может устанавливать метку синхронно 
с~появлением выбранного символа легальной передачи данных,  
а~агент-по\-лу\-ча\-тель должен выявлять присланную метку и~на основании 
этого выделять символ, к которому она относится. Реализация передачи 
основана на системе счетчиков, которые вычисляют длины фрагментов 
легальной передачи данных, определяемых метками. Эти длины могут 
записываться в~память без декодирования и~передаваться по другим скрытым 
каналам с~использованием, может быть, других меток. Таким образом может 
быть организован транзит скрытой передачи в~различных средах. 
  
  Пример скрытого канала данного типа внут\-ри компьютерной системы 
приведен в~работе~\cite{3-gru}. В~этом примере в~качестве меток 
использовались изменения формы электрических сигналов, с~помощью 
которых передавались данные в~компьютере. Передающие и~принимающие 
микропро\-цессоры могут формировать и~идентифицировать изменение формы 
электрических сигналов. Однако в~цифровых данных, используемых 
компьютером, такие изменения не идентифицируются. 
  
  Исследованиям пропускной способности скрытых каналов также посвящено 
значительное число работ. Наиболее близкой к проблематике данной статьи 
является работа~[5]. В~ней также рассматривается организация скрытого 
канала с~помощью генерации объектов синхронизации. Однако сам канал 
строится на других принципах, а~именно: на принципах удаления и~вставки бит 
в битовую легальную передачу данных. 
  
\section{Скрытые каналы с~метками}
  
  Предположим, что время дискретно и~легальная информация представляет 
собой последовательность данных, каждый элемент которых является буквой 
некоторого конечного алфавита. Буквы на разных местах передаваемой 
последовательности могут принадлежать различным алфавитам. Метки 
определяются некоторым способом выделения символов в~передаваемой 
последовательности легальных данных. Метки разбивают последовательность 
данных на фрагменты. В~скрытых каналах с~метками информация передается 
такими фрагментами~[3, 4]. Длина фрагмента~--- число знаков в~нем. 
  
  Определим код в~рассматриваемом скрытом канале. Кодируются слова 
и~символы некоторого языка, описывающего скрываемые данные. Эти 
элементы будем называть кодовыми величинами (КВ). Будем считать, что код 
состоит из~$N$~КВ. Каждая КВ кодируется наборами длин 
фрагментов $L\hm= (l_1,\ldots, l_k)$, которые будем называть кодовыми 
значениями (КЗ). Длины отсчитываются от одного начала фрагмента. Для 
каждой КВ выбирается свой набор длин, и~число их, вообще говоря, может 
быть различным. 
  
  Для простоты в~рамках данной статьи будем считать, что число~$k$ для всех 
КЗ одинаково и~все $l_i\hm\leq n$, $i\hm= 1,\ldots, k$. Тогда передача КЗ 
с~помощью скрытого канала описывается блочным кодом~\cite{6-gru} 
с~длиной блока~$n$. Таким образом, начало всех КЗ в~последовательности 
точно определено. 
  
  В скрытом канале с~метками предполагается наличие шума, который 
выражается в~случайном появлении меток в~случайных местах. Будем считать, 
что метки появляются случайно и~независимо друг от друга с~одинаковой 
вероятностью~$p$. Появление на одном месте двух меток: случайной и~неслучайной~--- 
равносильно появлению одной метки. 
  
  Согласно~\cite{6-gru}, скорость передачи одной КВ в~данной модели 
равна~$n$. В~работе строится оценка сверху для минимальной ско\-рости 
передачи при заданном ограничении~$\varepsilon$ на вероятность ошибочного 
декодирования. 
  
  Определим подробнее процедуру кодирования и~декодирования в~скрытом 
канале с~метками. Каж\-дая из~$N$~возможных КВ источника скрытого 
сообщения по заданной таблице кодируется вектором $L^{(i)} \hm= (l_1^{(i)}, 
\ldots, l_k^{(i)})$, $i\hm= 1,\ldots, N$. При этом выполняется следующая 
цепочка неравенств:
  $$
  l_1^{(i)}<l_2^{(i)}<\cdots <l_k^{(i)} \leq n\,,\enskip i=1,\ldots, N\,.
  $$
  
  В передаваемой легальной последовательности данных, начиная от точки 
отсчета, кратной~$n$ (начало блока), ставятся неслучайные метки на символах, 
расположенных от начала блока на расстояниях $l_1^{(i)},\ldots, l_k^{(i)}$, 
если передается КЗ~$L^{(i)}$. На приемном конце известны точки начала 
каждого КЗ. Тогда приемник скрытого сообщения определяет наличие меток на 
местах $l_1^{(i)},\ldots, l_k^{(i)}$ и~идентифицирует полученное КЗ~$L^{(i)}$. 
  
  Возможный шум может породить ложное принятое КЗ. Возникшая 
неопределенность декодирования в~рассматриваемом случае считается 
ошибкой при передаче по скрытому каналу. Именно вероятность появления 
ложных КЗ за счет шума ограничивается заданным параметром~$\varepsilon$, 
\mbox{$0\hm< \varepsilon\hm< 1$.}\linebreak Следовательно, истинное КЗ обязательно 
выявляется в~рассматриваемой модели, но вероятность возникновения 
неопределенности при передаче каж\-до\-го КЗ в~скрытом канале 
ограничена~$\varepsilon$. 
  
\section{Кодирование с~помощью непересекающихся длин 
в~кодовых значениях}

  Рассмотрим простейшую модель кода в~скрытом канале с~метками 
и~построим оценку скорости передачи в~нем. 
  
  Код записывается в~виде матрицы $\mathcal{L}\hm= \left\| a_{ij}\right\|$ 
размера $N\times n$, где $a_{ij}\hm=1$ тогда и~только тогда, когда~$L^{(i)}$ 
имеет фрагмент длины~$j$. В~противном случае $a_{ij}\hm=0$. В~каж\-дой 
строке матрицы~$\mathcal{L}$ находится ровно~$k$~единиц. 
  
  Простейший код характеризуется тем, что в~каж\-дом столбце 
матрицы~$\mathcal{L}$ не больше одной единицы. Это условие означает, что 
длины фрагментов в~кодовых словах для различных КЗ не пересекаются. Легко 
подсчитать число матриц~$\mathcal{L}$, удовлетворя\-ющих таким условиям:
${n!}/({(k!)^N(n-kN)!})$.
  
  Таким образом, множество возможных кодов не пусто, когда выполнено 
следующее неравенство:
  \begin{equation}
  n\geq Nk\,.
  \label{e1-gru}
  \end{equation}
  
  Вероятность случайного порождения данного КЗ равна~$p^k$. 
  
  Из непересекаемости длин фрагментов в~кодовых словах следует 
независимость случайного возникновения различных ложных КЗ, поэтому 
вероятность появления хотя бы одного ложного КЗ равна
  $1-(1-p^k)^{N-1}$.
  
  По условию эта вероятность ограничена па\-ра\-мет\-ром~$\varepsilon$; 
следовательно, получаем уравнение для~$k$, обеспечивающего заданную 
малость вероятности ошибки:
  $$
  \varepsilon = 1-\left( 1-p^k\right)^{N-1}\,.
  $$
  Отсюда $k$ имеет следующий вид:
  \begin{equation}
  k=\left[ \fr{\ln \left(1-(1-\varepsilon)^{1/(N-1)}\right)}{\ln p}\right]\,,
  \label{e2-gru}
  \end{equation}
где $[x]$ означает наименьшее целое, большее или равное~$x$. 

  Из формул~(\ref{e1-gru}) и~(\ref{e2-gru}) следует оценка скорости передачи по скрытому каналу:
  \begin{equation*}
  n= N\left[ \fr{\ln \left(1-(1-\varepsilon)^{1/(N-1)}\right)}{\ln p}\right]\,.
%  \label{e3-gru}
  \end{equation*}
  
  Приведем примеры численных расчетов характеристик скрытого канала 
в~зависимости от мощности алфавита скрытого канала~$N$, допустимой 
вероятности неоднозначного декодирования~$\varepsilon$ и~вероятности 
случайного появления меток~$p$.
  

 
  

  
  Из рисунка следует, что построение скрытого канала указанного типа 
возможно в~достаточно широком спектре значений определяющих па\-ра\-мет\-ров.
  
  Алгоритм декодирования в~рассматриваемом случае можно уточнить 
следующим образом. 
  \begin{enumerate}[1.]
\item От начала блока отсчитываются расстояния $l_1^{(1)},\ldots, l_1^{(N)}$.

\item Выделяются фрагменты, в~которых на этих расстояниях есть метки. 
Пусть эти фрагменты определяются длинами $l_1^{(i_1)},\ldots, l_1^{(i_j)}$. 
Верхние индексы определяют потенциальные КЗ: 
$$
\mathcal{L}_1= \left\{ L^{(i_1)}, \ldots, L^{(i_j)}\right\}\,.
$$
\end{enumerate} 
%\begin{figure*} %fig1
         \vspace*{-8pt}
 \begin{center}
 \mbox{%
 \epsfxsize=77.853mm
 \epsfbox{gru-1-1.eps}
 }
 \end{center}
% \vspace*{-3pt}
{\small Число знаков легальной передачи на один знак скрытого канала при размере 
кода скрытой передачи $N\hm=16$~(\textit{а}), 32~(\textit{б}) и~64~(\textit{в}): \textit{1}~--- $\varepsilon\hm=0{,}01$; 
\textit{2}~--- 0,005; \textit{3}~--- 0,001; \textit{4}~--- 0,0005; \textit{5}~---
0,0001}

\vspace*{28pt}

  %\end{figure*}
  \noindent
  \begin{enumerate}[1.]
  \setcounter{enumi}{2}
\item Для выделенных КЗ находятся метки на местах $l_2^{(i_1)},\ldots, 
l_2^{(i_j)}$. Эта процедура выделяет новый набор КЗ: 
$$
\mathcal{L}_2= \left\{ L^{(t_1)}, \ldots , L^{(t_s)}\right\} \subseteq 
\mathcal{L}_1\,.
$$
%\end{enumerate}
%\begin{enumerate}[4.]
\item Продолжая такой отсев, можно получить единственное кодовое слово, 
которое было передано, или несколько КЗ, что означает ошибку. 
\end{enumerate}

  В зависимости от свойств исходных сообщений (например, перечисление 
некоторых свойств) в~одном блоке можно передавать несколько КЗ. В~этом 
случае следует считать, что алгоритм декодирования определяет несколько КЗ, 
не считающихся ошибкой. Для этого случая вероятность ошибки оценивается 
несколько иначе. 
  
  Пусть допускается передача в~одном блоке ров\-но~$r$~КЗ. Тогда среднее 
появлений хотя бы одного ложного КЗ равна $r(1-(1-p^k)^{N-r})$.
  %
  Приравняв эту величину к~уровню допустимой ошибки~$\varepsilon$, 
получим:
  \begin{equation}
  k=\left[ \fr{\ln\left(1-\left(1-\varepsilon/r\right)^{1/(N-r)}\right)}{\ln p}\right]\,.
  \label{e4-gru}
  \end{equation}
  Из (\ref{e1-gru}) и~(\ref{e4-gru}) следует оценка скорости передачи в~скрытом 
канале:
  $$
  n=kN=N\left[ \fr{\ln\left(1-\left(1-\varepsilon/r\right)^{1/(N-r)}\right)}{\ln p}\right]\,.
  $$
  
  Из возможности передачи (используя особенности семантики) одновременно 
нескольких кодовых слов в~одном блоке следует, что полученная оценка 
скорости передачи может оказаться строгой оценкой сверху. Полученные 
оценки являются оценками пропускной способности~\cite{6-gru}. 
  
\section{Оценки скорости передачи скрытого канала 
при~использовании специальных кодов}

  Пропускную способность скрытого канала с~метками можно увеличить, если 
строить код в~соответствии с~некоторыми дополнительными ограничениями.
  
  \smallskip
  
  \noindent
  \textbf{Определение~1}~\cite{7-gru}. \textit{Уравновешенной неполной  
блок-схе\-мой} (далее~--- просто блок-схе\-мой) \textit{называется такое 
размещение~$n$~различных элементов по~$N$~блокам, что каждый блок 
содержит точно~$k$~различных элементов, каждый элемент появляется точно 
в~$r$~различных блоках и~каждая пара различных элементов появляется точно 
в~$\lambda$~блоках.} 
  
  \smallskip
  
  \noindent
  \textbf{Определение~2}~\cite{7-gru}. \textit{Блок-схе\-ма называется 
сим\-мет\-рич\-ной, если $n = N$}. 
  
  \smallskip
  
  
  Для симметричных блок-схем~\cite{7-gru} $k\hm= r$ и~любые два 
различных блока имеют точно $\lambda$ общих элементов. 
  
  Пусть кодовые слова~$\mathcal{L}$ образуют симметричную блок-схе\-му. 
Тогда вероятность того, что кроме истинного КЗ появится случайное ложное 
кодовое слово, равна $p^{k-\lambda}$. Используя неравенство Маркова, 
получим оценку вероятности~$P$ появления хотя бы одного ложного КЗ:
  $$
  P\leq (N-1) p^{k-\lambda}\,.
  $$
  
  Приравнивая $P$ к~$\varepsilon$, получим следующую оценку для~$k$:
  \begin{equation}
  k\geq \fr{\ln(\varepsilon/(N-1))}{\ln p} +\lambda\,.
  \label{e5-gru}
  \end{equation}
  Эта оценка справедлива для тех~$p$, для которых  $(N-1)p^{k-\lambda} 
\hm<1$.
  
  Дополнительным условием неравенства~(\ref{e5-gru}) служит ограничение 
  \begin{equation}
  k\leq n=N\,.
  \label{e6-gru}
  \end{equation}
  
  Из вида матрицы~$\mathcal{L}$ следует~\cite{8-gru}, что она представима 
в~виде суммы подстановочных матриц. Каж\-дая такая подстановочная матрица 
однозначно определяет каждое кодовое слово, а~$k$~обеспечивает 
необходимый уровень ошибки.
  
  Условие~(\ref{e6-gru}) позволяет получить дополнительное ограничение на 
вероятность~$p$, при котором задача передачи с~заданной ошибкой разрешима. 
Из~(\ref{e5-gru}) и~(\ref{e6-gru}) получаем
  \begin{equation*}
  N\geq \fr{\ln(\varepsilon/(N-1))}{\ln p} +\lambda\,.
%  \label{e7-gru}
  \end{equation*}
    Отсюда получаем еще одну оценку для~$p$:
  $$
  p\leq \left( \fr{\varepsilon}{N-1}\right)^{1/N} e^{\lambda/N}\,.
  $$
  
  В проведенных расчетах не учитывались дополнительные соотношения 
между параметрами симметричных блок-схем, при которых эти блок-схе\-мы 
существуют. Например, если в~симметричной блок-схе\-ме~$n$~чет\-но, то 
$k\hm-\lambda$ есть точный квадрат~\cite{7-gru}. В~противном случае искомой 
блок-схе\-мы не существует. 
  
  Это и~другие ограничения делают задачу построения кода со скоростью 
передачи~$N$~достаточно сложной. К~примеру, существует симметричная 
блок-схе\-ма с~параметрами $N\hm=n\hm=16$, $k \hm= 6$, $\lambda\hm = 2$. 
Для этого кода уровень неоднозначного декодирования $\varepsilon\hm = 
0{,}0001$ при $p\hm = 0{,}1$ и~$\varepsilon\hm = 0{,}001$ при $p\hm = 0{,}2$.
  
  Подробнее с~этими вопросами можно ознакомиться в~книге~\cite{7-gru}. 
  
  Еще один класс ограничений накладывает требование <<невидимости>> 
скрытого канала. Число искусственных меток должно быть на уровне 
стандартного отклонения от числа случайных меток, т.\,е.~$k$~имеет порядок 
$\sqrt{np(1-p)}$.
  
  Единственным способом защиты от рас\-смот\-рен\-но\-го класса скрытых каналов 
является нарушение согласования счетчиков на передающем и~прием\-ном 
концах скрытого канала. Такое рассогласование можно создать с~помощью 
вставок неинформативных символов в~легальное сообщение. Если изъятие этих 
вставок происходит перед декодированием скрытой передачи, то 
рассогласование длин может помешать правильному декодированию 
информации в~скрытом канале. 
  
  Однако в~указанном методе борьбы с~рас\-смот\-рен\-ным скрытым каналом 
необходимо учитывать следующее. Если размеры вставок малы, то за счет 
соотношения между длинами $l_1^{(i)},\ldots , l_k^{(i)}$ 
и~увеличения~$k$~можно нейтрализовать влияние защитных вставок. Если 
вставки делать достаточно длинными, то резко снижается пропускная 
способность легальной передачи, что противоречит функциональности 
легального канала. 
  
\section{Заключение}

  В работе получены оценки пропускной способности скрытых каналов, 
построенных с~помощью меток. Полученные оценки показывают 
соизмеримость передаваемой по скрытому каналу информации с~объемом 
информации, передаваемой легально. Полученные оценки являются оценками 
сверху, так как  скрытые каналы допускают одновременную передачу 
нескольких кодовых слов в~одном блоке. 
  
  Обсуждаются вопросы защиты от скрытых каналов рассматриваемого класса. 
  
{\small\frenchspacing
 {%\baselineskip=10.8pt
 \addcontentsline{toc}{section}{References}
 \begin{thebibliography}{9}
    \bibitem{1-gru}
    \Au{Lampson B.\,W.} A~note of the confinement problem~// 
Commun. ACM, 1973. Vol.~16. No.\,10. P.~613--615.
    \bibitem{2-gru}
    \Au{Тимонина Е.} Скрытые каналы (обзор)~// Jet Info, 2002. Т.~14. 
Вып.~114. С.~3--11.
    \bibitem{3-gru}
    \Au{Грушо Н.} Скрытые каналы, основанные на метках~// Системы 
и~средства информатики, 2013. Т.~23. №\,1. С.~7--13.
    \bibitem{4-gru}
    \Au{Грушо А., Грушо~Н., Тимонина~Е.} Анализ меток в~скрытых 
каналах~// Информатика и~её применения, 2014. Т.~8. Вып.~4. С.~12--16.
    \bibitem{5-gru}
    \Au{Wang Z., Lee R.\,B.} Capacity estimation of non-synchronous 
covert channels~// 2nd Workshop (International) on Security in Distributed 
Computing Systems (SDCS'05) Proceedings.~--- Columbus, OH, USA, 2005. 
P.~170--176. 
    \bibitem{6-gru}
    \Au{Чисар И., Кёрнер Я.} Теория информации: теоремы кодирования для 
дискретных схем без памяти~/ Пер. с~англ.~--- М.: Мир, 1985. 400~с. 
(\Au{Csiszar~I., Korner~J.} Information theory: Coding theorems for discrete 
memoryless systems.~--- Budapest: Academiai Kiado, 1981. 472~p.).
    \bibitem{7-gru}
    \Au{Холл М.} Комбинаторика~/ Пер. с~англ.~--- М.: Мир, 1970. 424~с. 
(\Au{Hall~M., Jr.} Combinatorial theory.~--- Waltham, MA\,--\,Toronto\,--\,London: 
Blaisdell Publishing Co., 1967. 310~p.) 
    \bibitem{8-gru}
    \Au{Сачков В.\,Н., Тараканов~В.\,Е.} Комбинаторика неотрицательных 
мат\-риц.~--- М.: ТВП, 2000. 448~c.
 \end{thebibliography}

 }
 }

\end{multicols}

\vspace*{-3pt}

\hfill{\small\textit{Поступила в~редакцию 25.07.15}}

\vspace*{8pt}

%\newpage

%\vspace*{-24pt}

\hrule

\vspace*{2pt}

\hrule

\vspace*{8pt}

\def\tit{RATE OF INFORMATION TRANSFER AND~CAPACITY 
IN~COVERT~CHANNELS DEFINED BY~TAGS}

\def\titkol{Rate of information transfer and capacity 
in covert channels defined by tags}

\def\aut{A.\,A.~Grusho, N.\,A.~Grusho, and E.\,E.~Timonina}

\def\autkol{A.\,A.~Grusho, N.\,A.~Grusho, and E.\,E.~Timonina}

\titel{\tit}{\aut}{\autkol}{\titkol}

\vspace*{-9pt}


\noindent
Institute of Informatics Problems,
Federal Research Center ``Computer Science and Control'' of
the Russian Academy of Sciences, 44-2 Vavilov Str.,
Moscow 119333, Russian Federation


\def\leftfootline{\small{\textbf{\thepage}
\hfill INFORMATIKA I EE PRIMENENIYA~--- INFORMATICS AND
APPLICATIONS\ \ \ 2015\ \ \ volume~9\ \ \ issue\ 4}
}%
 \def\rightfootline{\small{INFORMATIKA I EE PRIMENENIYA~---
INFORMATICS AND APPLICATIONS\ \ \ 2015\ \ \ volume~9\ \ \ issue\ 4
\hfill \textbf{\thepage}}}

\vspace*{3pt}




    \Abste{The paper is devoted to estimations of capacity and rate of information 
transfer in covert channels of a special type. These covert channels are generated with 
the help of signals (tags) which do not bear semantic information, but are easily 
allocated on the reception end. Such tags are a kind of bans from the point of view of 
admissible values of all parameters connected with transfer of legal information. The 
covert information is coded by lengths of fragments of the data in legal transfer 
allocated by tags.}
  
  \KWE{covert channels; probability-theoretic models of covert channels; covert 
channels generated by tags; capacity of a covert channel; rate of information transfer} 
    
\DOI{10.14357/19922264150409}

\Ack
    \noindent
The paper was partly supported by the Russian Foundation for Basic Research  (project 13-01-00215).



%\vspace*{3pt}

  \begin{multicols}{2}

\renewcommand{\bibname}{\protect\rmfamily References}
%\renewcommand{\bibname}{\large\protect\rm References}

{\small\frenchspacing
 {%\baselineskip=10.8pt
 \addcontentsline{toc}{section}{References}
 \begin{thebibliography}{9}
    \bibitem{1-gru-1}
    \Aue{Lampson, B.\,W.} 1973. A~note of the confinement problem. 
\textit{Commun. ACM} 16(10):613--615.
    \bibitem{2-gru-1}
    \Aue{Timonina, E.} 2002. Skrytye kanaly (obzor) [Covert channels (review)]. 
\textit{Jet Info} 14(114):3--11.
    \bibitem{3-gru-1}
    \Aue{Grusho, N.} 2013. Skrytye kanaly, osnovannye na metkakh [The covert 
channels based on tags]. \textit{Sistemy i~Sredstva Informatiki}~--- \textit{Systems 
and Means of Informatics} 23(1):7--13.
    \bibitem{4-gru-1}
    \Aue{Grusho,~A., N.~Grusho, and E.~Timonina}. 2014. Analiz metok 
v~skrytykh kanalakh [The analysis of tags in the covert channels]. \textit{Informatika 
i~ee Primeneniya}~--- \textit{Inform. Appl.} 8(4):12--16.
    \bibitem{5-gru-1}
    \Aue{Wang,~Z., and R.\,B.~Lee.} 2005.  Capacity estimation of  
non-synchronous covert channels. \textit{2nd Workshop (International) on Security in 
Distributed Computing Systems (SDCS) Proceedings}. Columbus, OH. 170--176. 
    \bibitem{6-gru-1}
    \Aue{Csiszar, I.,  and J.~Korner}. 1981. \textit{Information theory: Coding 
theorems for discrete memoryless systems}. Budapest: Academiai Kiado. 472~p.
    \bibitem{7-gru-1}
    \Aue{Hall, M.,~Jr.} 1967. \textit{Combinatorial theory}. Waltham,
    MA\,--\,Toronto\,--\,London: Blaisdell Publishing Co. 310~p. 
    \bibitem{8-gru-1}
    \Aue{Sachkov, V.\,N., and V.\,E.~Tarakanov}. 2000. \textit{Kombinatorika 
neotritsatel'nykh matrits} [Combination theory of nonnegative matrixes]. Moscow: 
TVP. 448~p.
\end{thebibliography}

 }
 }

\end{multicols}

\vspace*{-3pt}

\hfill{\small\textit{Received July 25, 2015}}
    
    \Contr
    
    \noindent
\textbf{Grusho Alexander A.} (b.\ 1946)~--- 
Doctor of Science in physics and mathematics, professor, leading scientist, Institute of 
Informatics Problems, Federal Research Center ``Computer Sciences and Control'' of the 
Russian Academy of Sciences,
44-2 Vavilov 
    Str., Moscow 119333, Russian Federation;  grusho@yandex.ru

\vspace*{3pt}

\noindent
\textbf{Grusho Nick A.} (b.\ 1982)~--- Candidate of Science (PhD) in physics and mathematics, 
senior scientist, Institute of Informatics Problems, Federal Research Center ``Computer 
Sciences and Control'' of the Russian Academy of Sciences,
44-2 Vavilov 
    Str., Moscow 119333, Russian Federation; info@itake.ru

\vspace*{3pt}

\noindent
\textbf{Timonina Elena E.} (b.\ 1952)~--- Doctor of Science in technology, professor, leading 
scientist, Institute of Informatics Problems, Federal Research Center ``Computer Sciences and 
Control'' of the Russian Academy of Sciences,
44-2~Vavilov Str., Moscow 119333, Russian Federation; eltimon@yandex.ru
    

\label{end\stat}


\renewcommand{\bibname}{\protect\rm Литература}