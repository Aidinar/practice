\def\ctn{{\mathrm{CБH}}}
\def\tn{{\mathrm{БH}}}
\def\an{{\mathrm{AH}}}

\def\stat{sinitsin}


\def\tit{АНАЛИТИЧЕСКОЕ МОДЕЛИРОВАНИЕ ПРОЦЕССОВ В~ДИНАМИЧЕСКИХ СИСТЕМАХ
С~ЦИЛИНДРИЧЕСКИМИ БЕССЕЛЕВЫМИ 
НЕЛИНЕЙНОСТЯМИ$^*$}

\def\titkol{Аналитическое моделирование процессов в~динамических системах
с~%цилиндрическими 
бесселевыми  нелинейностями}

\def\aut{И.\,Н.~Синицын$^1$}

\def\autkol{И.\,Н.~Синицын}

\titel{\tit}{\aut}{\autkol}{\titkol}

{\renewcommand{\thefootnote}{\fnsymbol{footnote}} \footnotetext[1]
{Работа выполнена при поддержке РФФИ (проект 15-07-02244).}}


\renewcommand{\thefootnote}{\arabic{footnote}}
\footnotetext[1]{Институт проблем информатики Федерального исследовательского
центра <<Информатика и~управление>> Российской академии наук,
sinitsin@dol.ru}



\label{st\stat}

                  \thispagestyle{headings}

\Abst{Рассмотрены  методы аналитического  моделирования (МАМ) процессов 
в~динамических системах со сложными бесселевыми нелинейностями (БН) при гармонических 
и~стохастических, узкополосных и~широкополосных возмущениях. Даны необходимые 
сведения из теории бесселевых функций и~сложных БН. 
Приведено методическое и~алгоритмическое обеспечение МАМ на основе методов 
статистической линеаризации (МСЛ) и~нормальной аппроксимации (МНА) для стохастических 
широкополосных процессов типа белого шума. Рассмотрены особенности МАМ для гармонических 
и~узкополосных стохастических процессов (СтП). В~приложении приведены формулы для 
коэффициентов МСЛ для типовых БН. В~качестве тестовых примеров 
рассмотрены задачи МАМ процессов в~одномерных системах с~аддитивными 
и~мультипликативными белыми шумами. Особое внимание уделено процессам в~бесселевом 
осцилляторе в~условиях различных возмущений.  Заключение содержит основные выводы 
и~некоторые обобщения.}

\KW{бесселева нелинейность (БН); гармонический процесс;
метод аналитического моделирования (МАМ); метод нормальной аппроксимации (МНА);
метод статистической линеаризации (МСЛ); осциллятор Бесселя;
сложные бесселевы нелинейности (СБН); стохастический нормальный гауссовский процесс;
стохастические системы на многообразиях (МСтС); узкополосный стохастический процесс;
фазовый портрет; формула Гиббса; функции  Куммера}

\DOI{10.14357/19922264150404}



\vskip 14pt plus 9pt minus 6pt

\thispagestyle{headings}

\begin{multicols}{2}

%\label{st\stat}

\section{Введение}


В~[1--4] рассмотрены вопросы  аналитического моделирования процессов 
в~дифференциальных  стохастических системах (СтС) на многообразиях (МСтС) 
с~винеровскими и~пуассоновскими шумами, а~также со сложными трансцендентными 
нелинейностями (СТН). Такие модели описывают\linebreak поведение многих современных нано- 
и~квантовооптических  технических средств информатики и~управления.
Приведены уравнения МНА 
и~МСЛ для моделирования нестационарных 
и~стационарных нормальных процессов в~МСтС.
Рассмотрены методы вычисления типовых интегралов для одно- и~многомерных  ТН 
скалярного и~векторного аргумента, получающихся из суперпозиции типовых ТН. 
Обсуждается алгоритмическое обеспечение МАМ.

Продолжим исследования~[1--4] применительно к~МСтС с~СТН в~виде (цилиндрических) 
БН. В~разд.~2 приводятся необходимые сведения из 
теории бесселевых функций первого рода и~целого порядка. В~разд.~3 
излагается МСЛ для МСтС с~БН, а~соответствующий МАМ на основе МНА и~МСЛ 
дается в~разд.~4. В~разд.~5 рассматриваются особенности МАМ для гармонических 
и~узкополосных стохастических процессов. Приложение~1 содержит формулы для 
коэффициентов МСЛ для типовых БН. Приложение~2 содержит тестовые примеры. 
Особое внимание уделено моделированию процессов в~осцилляторе Бесселя в~условиях 
различных возмущений.

\section{Цилиндрические функции Бесселя и~бесселевы нелинейности}


Многие сложные трансцендентные нелинейные преобразования вида
    \begin{equation*}
    z=\vrp^{\mathrm{СТН}} (y)
    %\label{e1-sin}
    \end{equation*}
выражаются через функции Бесселя $J_n (y)$ первого рода порядка~$n$ 
$(n\hm=0, 1, 2,\ldots)$.
Функции Бесселя, порядок которых равен половине нечетного целого числа, 
выражаются через элементарные функции~[5--7].

Как известно~[5--7], функции $J_n(y)$ целого порядка~$n$ удобно определить 
следующим бесконечным рядом:
\begin{equation*}
    J_n (y) = \left(\fr{y}{2}\right)^n \sss_{k=0}^\infty 
    \fr{1}{k! (n+k+1)!} \left( -\fr{y^2}{4}\right)^k\,, %\label{e2-sin}
    \end{equation*}
причем $J_{-n} \hm= (-1)^n J_n(y)$.

Функции $J_n =J_n (y)$ являются решением следующего линейного дифференциального 
уравнения с~переменными коэффициентами:
\begin{equation*}
y^2 \fr{d^2 J_n}{d y^2} + y \fr{dJ_n}{dy} + \left(y^2 - n^2\right) J_n =0\,, %\label{e3-sin}
\end{equation*}
допускают интегральные представления вида
\begin{multline*}
J_n (y) = \fr{(y/2)^n}{\sqrt{\pi} \Gamma (n+1/2)} 
\iii_0^\pi \cos (y \cos \theta) \sin^{2n} \theta\, d\theta={}\\
{}=\fr{2(y/2)^n}{\sqrt{\pi} \Gamma ( n+1/2)} \iii_0^1 (1- t^2)^{n-1/2} \cos yt\, dt 
\\ 
\left(\mathrm{Re}\ n>-\fr{1}{2}\right)\,, %\label{e4-sin}
\end{multline*}
а~также удовлетворяют рекуррентным соотноше\-ниям
   \begin{equation}
   \left.
   \begin{array}{rl}
    J_{n-1} (y) +J_{n+1} (y) &= \fr{2n}{y}\,J_n (y)\,;\\[6pt]
    J_{n-1} (y) - J_{n+1}(y) &=2J_n' (y)\,;
    \\[6pt]
    J_n' (y) &= J_{n-1} (y) - \fr{n}{y}\, J_n (y)\,;\\[6pt]
    J_n' (y) &= - J_{n+1} (y) + \fr{n}{y}\, J_n (y)\,,
    \end{array}
    \right\}
    \label{e5-sin}
    \end{equation}
где $J_n' (y) = dJ_n (y)/dy $; $\Gamma (n)$~--- гам\-ма-функ\-ция.
%
При этом верхние границы для функции $J_n(y)$ определяются следующими неравенствами:
    \begin{equation}
    \left.
    \begin{array}{c}
    \lv J_n (y) \rv < 1 \enskip (\forall n \ge 0)\,; \\[6pt]
 \lv J_n (y) \rv \le \fr{1}{\sqrt{2}} \enskip (n\ge 1)\,;\\[6pt]
    0<J_n (y) < \fr{2^{1/3}}{3^{2/3} \Gamma (2/3) n^{1/3}} \enskip (n>0)\,;\\[6pt]
    \lv J_n (ny) \rv \le \lv y^n \displaystyle\exp \fr{n\sqrt{1-y^2}}{\lk 1+ \sqrt{1-y^2}\rk^n}\rv.
    \end{array}
    \right\}
    \label{e6-sin}
    \end{equation}

Нули $j_{n,s}$ и~$j_{n,s}'$ для  $J_n (y)$  и~$J_n'(y)\hm=dJ_n(y)/dy$ действительны, 
причем они разделяются для $y\hm\ne 0$:

\noindent
\begin{equation}
\left.
\begin{array}{c}
    \hspace*{-2mm}0<j_{n,1} <j_{n+1,1} < j_{n,2} < j_{n+1, 2} < \cdots \ (n>0);\\[6pt]
    n\le j_{n,1}' <  j_{n,1}<  j_{n,2}' <  j_{n,2}< \cdots \enskip (n>0)\,;\\[6pt]
    j_{0,1}' =0\,;\quad  j_{0,s}' =  j_{1,s-1} \enskip ( s=2,3,\ldots)\,;\\[6pt]
    j_{n,s} =\rho_n(s)\,;\quad j_{n,s}' = \si_n (s-1)\,,
    \end{array}\!
    \right\}\!\!
    \label{e7-sin}
    \end{equation}
где $\rho_n$ и~$\si_n$~--- нули $J_n(y)$ и~$J_n'(y)$, 
$\rho_n (0)\hm=0$, $\si_n (0) \hm= j_{n,1}'$.

Теоремы умножения и~сложения для функции $J_n(y)$ записываются в~виде:
\begin{gather}
\hspace*{-3mm}J_n(\la y) = \la^n \sss_{k=0}^\infty \fr{(-1)^k (\la^2 -1)^k (y/2)^k}{k!}\,
 J_{n+k} (y);\!\!\label{e8-sin}\\
J_n (y_1\pm y_2) =\sss_{k=-\infty}^\infty J_{n\mp k} (y_1) J_k (y_2)\,.\notag %\label{e9-sin}
\end{gather}
Дифференцирование $J_n(y)$ проводится согласно следующим формулам:
    \begin{gather*}
    J_0' (y) =-J_1 (y)\,;\enskip 
    yJ_n' (y) = n J_n(y) - y J_{n+1} (y)\,;\\ 
    \fr{2n}{y}\, J_n (y) = J_{n-1} (y) + J_{n+1} (y)\,;\\
    \fr{d}{dy} \lk y^n J_n (y)\rk = y^n J_{n-1} (y)\,;
\end{gather*}

\vspace*{-12pt}

\noindent
    \begin{multline*}
4J_n'' (y) = 2 J_{n-1}' (y) - 2 J_{n+1}' (y) = {}\\
{}=J_{n-2} (y) - 2 J_n (y) + J_{n+2} (y)\,;
\end{multline*}

\vspace*{-12pt}

\noindent
\begin{multline*}
2^s J_n^{(s)} (y) = \left[ J_{n-s} (y) - C_s^1 J_{n-s+2} (y) + {}\right.\\
\left.{}+C_s^2 J_{n-s+4} (y) 
-\cdots - (-1)^s J_{n+s} (y)\right] \\ 
(s=0,1,2,\ldots)\,. %\label{e10-sin}
\end{multline*}

В основе интегрирования функции $J_n(y)$ лежат следующие формулы:
\begin{equation}
\left.
\hspace*{-2mm}\begin{array}{rl}
\displaystyle\iii_0^y J_{2r} (\eta) \,d\eta &= \displaystyle\iii_0^y J_0 (\eta)\, d\eta - 
2 \sss_{k=0}^{r-1} J_{2k+1} (y)\,;\\[6pt]
\displaystyle\iii_0^y J_{2r+1} (\eta) \,d\eta &= 1- J_0 (y) - 2 \displaystyle\sss_{k=0}^r J_{2k} (y)\,;\\[6pt]
\displaystyle\iii_0^y J_{n+1}(\eta) \,d\eta &= \displaystyle\iii_0^y J_{n-1} (\eta) \,d\eta - 2 J_n (y)\,.
\end{array}
\right\}\!\!
\label{e11-sin}
\end{equation}

\noindent
\textbf{Замечание~1.}\
Цилиндрической функцией Бесселя мнимого аргумента порядка~$n$ называется функция 
вида $I_n (y) \hm= (i)^{-n} J_n (iy)$.


\vspace*{2pt}

\noindent
\textbf{Замечание~2.}\
Примерами СБН, получаемых посредством сумм типовых БН, могут служить следующие:

\noindent
    \begin{align*}
    \vrp^\ctn (Y,t) &=\sss_{r=1}^n l_{rt} \vrp^\tn_r (Y)\,;
\\ 
    \vrp^\ctn (Y,t) &=\sss_{r=1}^n l_{rt} \vrp^\tn_r (Y)\vrp_r^\an (Y),
  \end{align*}
а также дробно-ра\-цио\-наль\-ные представления:

\pagebreak

\noindent
    \begin{align*}
    \vrp^\ctn (Y,t) &=
    \fr{\sum\nolimits_{r=1}^{n'} l_{rt}' {\vrp'}^\tn_r (Y)}
    {\sum\nolimits_{r=1}^{n''} l_{rt}'' {\vrp''}^\tn_r (Y)}\,;\
    \\
     \vrp^\ctn (Y,t) &=\fr{\sum\nolimits_{r=1}^{n'} l_{rt}' {\vrp'}^\tn_r (Y){\vrp'}_r^\an (Y)}
    {\sum\nolimits_{r=1}^{n''} l_{rt}'' {\vrp''}^\tn_r (Y){\vrp''}_r^\an (Y)}\,,
  \end{align*}
где ${\vrp}^\tn_r (Y)$, ${\vrp'}^\tn_r (Y)$ и~${\vrp''}^\tn_r (Y)$~--- 
типовые бесселевы функции; $ l_{rt}$,  $l_{rt}'$ и~$l_{rt}''$~--- 
коэффициенты, зависящие от времени~$t$; ${\vrp}_r^\an (Y)$, 
${\vrp'}_r^\an (Y)$ и~${\vrp''}_r^\an (Y)$~--- 
алгебраические нелинейности (многочлены, степен\-ные, иррациональные, 
дроб\-но-ра\-ци\-о\-наль\-ные и~другие функции).

Другими примерами СБН являются нелинейности, получаемые путем соответствующего 
преобразования аргумента:
    \begin{align*}
    \vrp^\ctn (Y,t) &=\vrp^\an (\psi^\tn (Y,t), t)\,;\\
      \vrp^\ctn (Y,t) &=\vrp^\tn (\psi^\an (Y,t), t)\,,
    \end{align*}
где $\vrp^\an$, $\psi^\an$, $\vrp^\tn (Y,t)$ и~$\psi^\tn (Y,t)$~--- 
типовые алгебраические и~бесселевы трансцендентные нелинейности.


\noindent
\textbf{Замечание~3.}
В качестве примеров скалярных СБН векторного аргумента $Y\hm=\lk Y_1\cdots Y_p\rk^{\mathrm{T}}$ 
рассмотрим следующие:
    \begin{align*}
    \vrp^\ctn (Y,t) &=\sss_{r=1}^n \prod_{h=1}^H l_{rh,t} \vrp_{rh}^\tn (Y_h)\,;\\
        \vrp^\ctn (Y,t) &=\fr{\sss_{r=1}^{n'} \prod_{h=1}^{H'} l_{rh,t}' {\vrp'}_{rh}^\tn 
    (Y_h)}{\sss_{r=1}^{n''} \prod_{h=1}^{H''} l_{rh,t}'' {\vrp''}_{rh}^\tn (Y_h)}\,.
    \end{align*}
В случае векторных и~матричных СБН приведенные формулы имеют место 
для соответствующих компонент.


Функции $J_n(y)$ для неотрицательных $n\hm=0,1,2,\ldots$ представляют собой однозначные 
целые функции~$y$ и~определяются как коэффициенты следующих рядов Фурье:
    \begin{align*}
    \cos (y\sin\xi) &=  J_0 (y) + 2 \sss_{k=1}^\infty J_{2k} (y) \cos 
    2 k\xi\,;\\
    \sin (y\sin\xi) &= 2 \sss_{k=1}^\infty J_{2k-1} (y) 
    \sin (2k-1)\xi\,.
%    \label{e12-sin}
    \end{align*}

Пусть $\rho_i$ и~$\rho_k$~--- два нуля функции Бесселя $J_n(y)$ целого порядка, 
тогда имеют место следующие условия ортогональности:

\noindent
    $$
    \iii_0^1 J_n (\rho_i y) J_n (\rho_k y)\, dy = \begin{cases}
        0 &\hspace*{-25mm} \mbox{при } i\ne k\,;\\
        \fr{1}{2} \lk J_n' (\rho_i) \rk^2 = \fr{1}{2} \lk J_{n+1}(\rho_i)\rk^2 
        &\\
        &\hspace*{-25mm} \mbox{при } i= k\,.\end{cases}\hspace*{-8.89503pt}
        $$

При достаточно общих условиях нелинейную функцию  $\vrp^\ctn(y)$ 
можно разложить в~следующий ряд Фурье--Бес\-се\-ля~[5--7]:
    \begin{equation}
    \left.
    \begin{array}{rl}
    \vrp(y) &= \displaystyle\sss_{k=1}^\infty \alpha_k J_n (\rho_k y)\,;\\[6pt]
     \alpha_k &= \fr{2}{\left[ J_{n+1} (\rho_k)\right]^2} 
     \displaystyle\iii_0^1 \xi\vrp^\ctn (\xi) J_n (\rho_k \xi)\, d \xi\,.
     \end{array}
     \right\}
     \label{e13-sin}
     \end{equation}
Здесь $\rho_1, \rho_2,\ldots$~--- положительные нули функции $J_n(y)$, 
расположенные в~порядке их возрастания, а~$\alpha_k$~--- коэффициенты разложения.

\section{Статистическая линеаризация бесселевых нелинейностей}

Рассмотрим статистическую линеаризацию БН по Казакову~\cite{8-sin}:
    \begin{equation}
    Z= J_n(Y)
    \label{e14-sin}
    \end{equation}
при несимметричном гауссовском стохастическом входном сигнале~$Y$:
\begin{equation}
Y_t = Y(t) = m_y + Y_t^0\,,
\label{e15-sin}
\end{equation}
где $m_y$~--- математическое ожидание, %а~$D_y$~--- дисперсия, 
$Y_t^0 \hm= Y^0(t) \hm= Y(t) - m_y$. В~соответствии с~МСЛ 
зависимость аппроксимируется следующим выражением~[8--10]:
\begin{equation}
Z_t =  \vrp_0 (m_y, D_y) + k_1(m_y, D_y) Y_t^0\,.\label{e16-sin}
\end{equation}
Здесь $\vrp_0$ и~$k_1$~--- коэффициенты статистической линеаризации, зависящие 
от~$m_y$, дисперсии~$D_y$  и~определяемые по формулам:
\begin{multline}
\vrp_0 = \vrp_0 (m_y, D_y) ={}\\
{}= \fr{1}{\sqrt{2\pi D_y}} 
    \iin J_n(\eta) e^{-(\eta-m_y)^2/2D_y}\, d\eta\,;\label{e17-sin}
    \end{multline}
    
    \vspace*{-12pt}
    
    \noindent
\begin{multline}
k_1 = k_1(m_y, D_y) ={}\\
{}= \fr{1}{\sqrt{2\pi D_y}} \iin 
(\eta-m_y) J_n(\eta) e^{-(\eta-m_y)^2/2D_y} d\eta = {}\\
{}=
\fr{\prt \vrp_0 (m_y, D_y)}{\prt m_y}\,.
\label{e18-sin}
\end{multline}
При $n=2r+1$ для функции $J_{2r+1}(Y)$ в~силу ее нечетности имеем:
 \begin{equation}
 \vrp_0^{J_{2r+1}} (m_y, D_y) = k_0^{J_{2r+1}} (m_y, D_y) m_y\,;\label{e19-sin}
 \end{equation}
 
 \vspace*{-12pt}
 
 \noindent
\begin{multline}
 k_0^{J_{2r+1}}(m_y, D_y) = {}\\
 {}=\fr{1}{m_y \sqrt{2\pi D_y}} \iin J_{2r+1} 
 (\eta) e^{-(\eta-m_y)^2/ 2D_y} d\eta\,.
 \label{e20-sin}
 \end{multline}

\noindent
\textbf{Теорема~1.}\ \textit{Если существуют вероятностные моменты второго порядка 
для уравнений}~(\ref{e15-sin}) \textit{и}~(\ref{e16-sin}), 
\textit{то в~основе МСЛ для БН четного порядка $(n\hm=2r)$ 
лежат уравнения}~(\ref{e15-sin})--(\ref{e18-sin}), 
\textit{а~для БН нечетного порядка} $(n\hm=2r+1)$~--- (\ref{e16-sin}), 
(\ref{e19-sin}), (\ref{e20-sin}).

\smallskip

Коэффициенты статистической линеаризации  типовых БН приведены в~приложении~П1.


\section{Аналитическое моделирование гауссовских процессов 
в~стохастических системах с~бесселевыми нелинейностями}

Как известно~[8--10],  уравнения конечномерных непрерывных 
нелинейных систем со стохастическими возмущениями путем расширения вектора 
состояния СтС могут быть записаны в~виде сле\-ду\-юще\-го векторного стохастического 
дифференциального уравнения Ито:
  \begin{multline}
  dY_t = a(Y_t, t) \,dt + b (Y_t, t)\, dW_0+{}\\
  {}+ \iii_{R_0} c (Y_t, t, v) P^0 (dt, dv)\,,\enskip 
  Y(t_0) = Y_0\,.\label{e21-sin}
  \end{multline}
Здесь $Y_t$~--- $(p\times 1)$-мер\-ный вектор состояния, $Y_t \hm\in \Delta_y$ 
($\Delta_y$~--- многообразие состояний);  $a\hm=a(Y_t, t)$ и~$b\hm=b(y_t, t)$~--- 
известные  $(p\times 1)$-мер\-ная и~$(p\times m)$-мер\-ная функции~$Y_t$ и~$t$;  
$W_0\hm= W_0(t)$~--- $(r\times 1)$-мер\-ный винеровский СтП 
интенсивности  $\nu_0 \hm= \nu_0(t)$; $c(Y_t, t, v)$~--- $(p\times 1)$-мер\-ная 
функция~$Y_t$, $t$ и~вспомогательного $(q\times 1)$-мер\-но\-го параметра~$v$; 
$\iii_{\Delta}\,dP^0 (t, A)$~--- центрированная пуассоновская мера, определя\-емая
    $$
    \iii_{\Delta} dP^0 (t, A) = \iii_{\Delta} dP (t,A) =\iii_{\Delta} \nu_P (t,A) \,dt\,.
    $$
При этом принято: $\iii_{\Delta}$~--- число скачков пуассоновского
СтП в~интервале времени  $\Delta = (t_1, t_2]$; $\nu_P (t, A)$~---
интенсивность пуассоновского СтП  $P(t,A)$; $A$~--- некоторое
борелевское множество пространства~$R_0^q$ с~выколотым началом.
Начальное значение~$Y_0$ представляет собой случайную величину, не
зависящую от приращений $W_0(t)$ и~$P(t,A)$ на интервалах времени,
следующих за~$t_0$, $t_0\hm \le t_1\hm\le t_2$, для любого множества~$A$.
Предполагается, что элементы функций $a(y,t)$, $b(y,t)$ и~$c(y,t,v)$:
$a_i (y_t, t)$, $b_{ij} (y_t, t)$ и~$ c_i (y_t, t, v)$~--- допускают
разложения Фурье--Бес\-се\-ля вида~(\ref{e13-sin}):
\begin{align*}
a_i (y_t, t) &= \sss_k \prod_l \alpha_{lk}^{a_i} J_n (\rho_k y_{kt})\,;\\
b_{ij} (y_t, t) &= \sss_k \prod_l \beta_{lk}^{b_{ij}} J_n (\rho_k y_{kt})\,;\\
c_i (y_t, t,v) &= \sss_k \prod_l \gamma_{lk}^{c_i} J_n (\rho_k y_{kt})\,.
\end{align*}

\noindent
\textbf{Замечание~4.}
Для аддитивных гауссовских (нормальных) и~обобщенных пуассоновских возмущений 
уравнение~(\ref{e21-sin}) принимает  вид~[8--10]:
\begin{equation}
\dot Y = a(Y_t,t)+ b_0 (t) V\,, \  V = \dot W\,,\  Y(t_0) = Y_0\,.
\label{e22-sin}
\end{equation}
Здесь $W$~--- СтП с~независимыми приращениями, представляющий собой смесь 
нормального и~обобщенного пуассоновского СтП.


Если предположить существование конечных вероятностных моментов второго 
порядка для моментов времени~$t_1$ и~$t_2$, то уравнения МНА примут следующий 
вид~[8--10]:
\begin{itemize}
\item  для характеристических функций:
    \begin{equation}
    \left.
    \begin{array}{c}
    g_1^N (\la;t) =\exp \lk i\la^{\mathrm{T}} m_t - \fr{1}{2}\, \la^{\mathrm{T}} K_t \la\rk\,;\\[6pt]
   \hspace*{-35mm}g_{t_1, t_2}^N (\la_1, \la_2;t_1, t_2 ) ={}\\
    \hspace*{10mm}{}=\exp \lk i\bar \la^{\mathrm{T}} \bar m_2 - 
    \fr{1}{2} \,\bar \la^{\mathrm{T}} \bar K_2 \la\rk\,,
    \end{array}
    \right\}
    \label{e23-sin}
    \end{equation}
    где
\begin{gather*}
\bar \la =\lk \la_1^{\mathrm{T}}\la_2^{\mathrm{T}}\rk^{\mathrm{T}}\,; \enskip
\bar m_2 = \lk m_{t_1}^{\mathrm{T}} m_{t_2}^{\mathrm{T}}\rk^{\mathrm{T}}\,;\\ 
\bar K_2= \begin{bmatrix}
    K(t_1, t_1)& K(t_1, t_2)\\[2pt]
    K(t_2, t_1)& K(t_2, t_2)\end{bmatrix}\,;
    \end{gather*}

\item для математических ожиданий~$m_t$, ковариационной матрицы~$K_t$ 
и~матрицы ковариационных функций $K(t_1, t_2)$:
 \begin{equation}
 \left.
 \hspace*{-3mm}\begin{array}{rl}
 \dot m_t &= a_1 (m_t, K_t, t)\,,\enskip m_0 = m(t_0)\,;\\[6pt]
\dot K_t &= a_2 (m_t, K_t, t)\,,\enskip K_0 = K(t_0)\,;\\[6pt]
\fr{\prt K(t_1, t_2)}{\prt t_2 }&= K(t_1, t_2) a_{21} (m_{t_2}, K_{t_2}, t_2)^{\mathrm{T}}\,,
\\[6pt]
 &\hspace*{20mm}K(t_1, t_1) = K_{t_1}\,.
 \end{array}
 \right\}\!
\label{e24-sin}
\end{equation}
    \end{itemize}
Здесь приняты следующие обозначения:
    $$
    m_t = {\sf M}_{\Delta_y}^N Y_t\,;\enskip Y_t^0 = Y_t - m_t\,;
    $$
    $$
    K_t = {\sf M}_{\Delta_y}^N Y_t^0 Y_t^{\mathrm{0T}}\,;\enskip K(t_1, t_2) = 
    M_{\Delta_y}^N Y_{t_1}^{0} Y_{t_2}^{\mathrm{0T}}\,;
    $$
    $$
    a_1 = a_1 (m_t, K_t, t) = {\sf M}_{\Delta_y}^N a (Y_t, t)\,;
    $$
    
    \vspace*{-12pt}
    
    \noindent
    \begin{multline*}
     a_2 (m_t, K_t, t) = a_{21} (m_t, K_t, t)+{}\\
     {}+ a_{21} (m_t, K_t, t)^{\mathrm{T}} +
    a_{22}(m_t, K_t, t)\,;
    \end{multline*}
    $$
    a_{21} = a_{21}(m_t, K_t, t)=  {\sf M}_{\Delta_y}^N a(Y_t, t) Y_{t}^{\mathrm{0T}}\,;
    $$
    $$
    a_{22} = a_{22}(m_t, K_t, t)= {\sf M}_{\Delta_y}^N \bar\si (Y_t, t)\,;
    $$
    $$
    \si (Y_t, t)=b(Y_t, t) \nu_0(t) b(Y_t, t)^{\mathrm{T}}\,;
    $$

    \vspace*{-12pt}
    
    \noindent
    \begin{multline}
    \bar\si (Y_t, t) ={}\\
    {}= \si (Y_t, t)+\iii_{R_0^q} c (Y_t, t, v) 
    c(Y_t, t,v)^{\mathrm{T}} \nu_P (t, dv)\,,\label{e25-sin}
    \end{multline}
где ${\sf M}_{\Delta_y}^N$~--- символ вычисления математического ожидания для 
нормальных распределений~(\ref{e23-sin}).

Для стационарных СтС нормальные стационарные СтП~--- если они
существуют, то  $m_t \hm= m^*$, $ K_t \hm= K^*$, $K(t_1, t_2) \hm=
k(\tau)$ $(\tau \hm= t_1-t_2)$~--- определяются уравнениями~[8--10]:
    \begin{equation}
    \left.
    \begin{array}{c}
    a_1 (m^*, K^*) =0\,;\enskip a_2 (m^*, K^*)=0\,;\\[6pt]
    \dot k_\tau (\tau) = a_{21} (m^*, K^*) {K^*}^{-1} k(\tau)\,;\\[6pt] 
    k(0) =K^* \enskip (\forall \tau >0)\,; \\[6pt] 
    k(\tau) = k(-\tau)^{\mathrm{T}} \enskip (\forall\tau <0)\,.
    \end{array}
    \right\}
    \label{e26-sin}
    \end{equation}
При этом необходимо, чтобы матрица  $a_{21} ( m^*, K^*)\hm= a^*_{21}$ 
была асимптотически устойчивой.

Уравнения МНА в~случае СтС~(\ref{e22-sin}) переходят в~уравнения 
МСЛ~[8--10], если принять
 \begin{equation}
 \left.
 \hspace*{-2mm}\begin{array}{c}
    a(Y_t,t) = a_0 (m_t, K_t) + k_1^a (m_t, K_t) Y_t^0\,;
   \\[6pt]
    k_1^a (m_t, K_t, t) =\lk \left(\fr{\prt}{\prt m_t} \right)
    a_0 (m_t, K_t, t)^{\mathrm{T}}\rk^{\mathrm{T}}\,;
\\[6pt]
    b(Y_t,t) = b_0 (t)\,;
    \\[6pt]
    \si(Y_t, t)= b_0(t) \nu(t) b_0(t)^{\mathrm{T}} = \si_0(t)\,;
   \\[6pt]
    \dot m_t = a_1 (m_t, K_t, t)\,,\enskip m_0 = m(t_0)\,;
    \\[6pt]
    \dot K_t = k_1^a (m_t, K_t, t) K_t + K_t k_1^a (m_t, K_t, t)^{\mathrm{T}} +{}\\
    {}+     \si_0(t)\,,\enskip 
    K_0 = K(t_0)\,;\\[6pt]
    \fr{\prt K(t_1, t_2)}{\prt t_2} = K(t_1, t_2) K_{t_2} k_1^a (m_{t_2}, 
    K_{t_2}, t_2)^{\mathrm{T}}\,,\\[6pt]
     K(t_1, t_2) = K_{t_1}\,.
     \end{array}
     \right\}\!\!
    \label{e27-sin}
    \end{equation}

Для стационарных СтС~(\ref{e22-sin}) при условии асимптотической устойчивости 
матрицы $k_1^a (m^*, K^*)$ в~основе МСЛ лежат уравнения~(\ref{e26-sin}), 
записанные в~виде:
    \begin{equation}
    \left.
    \hspace*{-2mm}\begin{array}{c}
    a_0 (m^*, K^*) =0\,; \\[6pt]
     k_1^a ( m^*,  K^*)  K^* +  
    K^* k_1^a ( m^*,  K^*)^{\mathrm{T}} +\bar \si_0 =0\,;
\\[6pt]
\dot k_\tau (\tau) = k_1^a ( m^*, K^*)k(\tau)\,;\\[6pt]
k(0) =K^* \enskip (\forall \tau >0)\,;\\[6pt] 
k(\tau) = k (-\tau)^{\mathrm{T}} \enskip (\forall \tau <0)\,.
\end{array}
\right\}\!\!
\label{e28-sin}
\end{equation}

\noindent
\textbf{Теорема~2.} \textit{Если существуют интегралы}~(\ref{e25-sin}), 
\textit{то уравнения}~(\ref{e24-sin}) \textit{лежат в~основе МАМ на базе МСЛ для 
СтС}~(\ref{e21-sin}), \textit{а~уравнения}~(\ref{e27-sin})~--- 
\textit{на базе МСЛ для СтС}~(\ref{e22-sin}). 
\textit{Для моделирования стационарных СтП согласно МНА служат соотношения}~(\ref{e26-sin}), 
\textit{а~МСЛ}~--- (\ref{e28-sin}) 
\textit{при условии асимптотической устойчивости мат\-ри\-цы~$a_{21}^*$}.

\smallskip

Для алгоритмизации МНА необходимо уметь вычислять следующие интегралы:
\begin{multline}
I_0^a = I_0^a (m_t, K_t, t) ={}\\
{}=
 a_1 (m_t, K_t, t)= {\sf M}_{\Delta_y}^N a(Y_t, t);\label{e29-sin}
 \end{multline}
 
 \vspace*{-12pt}
 
 \noindent
 \begin{multline}
I_1^a = I_1^a (m_t, K_t, t)={}\\
{}=
 a_{21}(m_t, K_t, t)= {\sf M}_{\Delta_y}^N a(Y_t , t) Y_t^{\mathrm{0T}}\,;
 \label{e29a-sin}
  \end{multline}
 
 \vspace*{-12pt}
 
 \noindent
 \begin{multline}
I_0^{\bar \si} = I_0^{\bar \si} (m_t, K_t, t) ={}\\
{}=
a_{22}(m_t, K_t, t) = {\sf M}_N \bar \si (Y_t, t)\,,
\label{e29b-sin}
\end{multline}
а для МСЛ достаточно вычислить интеграл в~(\ref{e29-sin}), 
причем интеграл~$I_1^a$ вычисляется по формуле~[8--10]:
\begin{equation*}
k_1^a = k_1^a (m_t, K_t, t)=\lk \left( \fr{\prt}{\prt m_t}\right) 
I_0^a (m_t, K_t, t)^{\mathrm{T}}\rk^{\mathrm{T}}\,.
%\label{e30-sin}
\end{equation*}

Численные методы определения интегралов типа~(\ref{e29-sin})--(\ref{e29b-sin}) 
описаны в~\cite{7-sin, 11-sin}.

\smallskip

{ %\small
\noindent
\textbf{Замечание~5.}
Важно иметь в~виду, что уравнения МНА (МСЛ) содержат интегралы~$I_0^a$, 
$I_1^a$ и~$I_0^\si$ в~виде соответствующих коэффициентов, поэтому процедура 
вычисления интегралов должна быть согласована с~методом численного решения 
обыкновенных дифференциальных уравнений для~$m_t$, $K_t$ и~$K(t_1, t_2)$. 
Эти коэффициенты допускают дифференцирование по~$m_t$ и~$K_t$, так как под 
интегралом стоит сглаживающая нормальная плотность.
}

\smallskip

В~\cite{1-sin} изложены алгоритмы дискретного аналитического и~статистического
моделирования типовых распределений (в~том числе нормальных)\linebreak в~нелинейных
МСтС. Алгоритмы дискретного аналитического и~статистического
моделирования для СтС с~СБН, а~также смешанные алгоритмы различной
степени точности относительно шага интегрирования также представлены в~\cite{1-sin}.

\section{Особенности аналитического моделирования гармонических 
и~узкополосных стохастических процессов}

Рассмотрим сначала гармоническую линеаризацию бесселевой функции $J_n (y)$ 
целого порядка~(\ref{e14-sin})
при несимметричном гармоническом входном процессе~$Y$ вида
\begin{equation}
Y= A_0 + A \sin \psi\,, \enskip \psi =\omega t\,.
\label{e31-sin}
\end{equation}
В соответствии с~методом гармонической линеаризации (МГЛ)~\cite{8-sin} 
зависимость~(\ref{e14-sin}) при условии~(\ref{e31-sin}) аппроксимируется функцией
\begin{equation}
Z\approx A_{z0} + q_z A\sin \psi + q_z' A\cos \psi\,.
\label{e32-sin}
\end{equation}
Входящие в~(\ref{e32-sin}) коэффициенты МГЛ определяются формулами:
\begin{equation}
\hspace*{-2mm}A_{z0} = A_{z0} (A_0, A) = \fr{1}{2\pi} \iii_0^{2\pi} J_n 
(A_0+ A\sin \psi)\, d\psi,\!\!\!
\label{e33-sin}
\end{equation}
\begin{equation}
\left.
\begin{array}{rl}
\hspace*{-2mm}q_z=q_z(A_0, A) &={}\\[6pt]
&\hspace*{-15mm}{}=\displaystyle \fr{1}{\pi A} \iii_0^{2\pi} J_n (A_0+A\sin \psi) \sin \psi\, d\psi\,;\\[6pt]
\hspace*{-2mm}q_z'=q_z'(A_0, A) &={}\\[6pt]
&\hspace*{-15mm}{}=\displaystyle \fr{1}{\pi A} \iii_0^{2\pi} J_n (A_0+A\sin \psi) \cos \psi d\psi\,.
\end{array}
\right\}
\label{e34-sin}
\end{equation}
Для $n=2r+1$, учитывая нечетность функции $J_{2r+1} (y)$, вместо~$A_{z0}$ 
вводят следующий коэффициент:
\begin{multline}
q_{z0} = q_{z0} ( A_0, A) = {}\\
{}=\fr{1}{2\pi A_0} \iii_0^{2\pi} J_{2r+1} 
(A_0 + A\sin \psi)\, d\psi\,.\label{e35-sin}
\end{multline}

\noindent
\textbf{Теорема~3.} \textit{Если существуют интегралы}~(\ref{e33-sin}) 
\textit{и}~(\ref{e34-sin}), \textit{то уравнения}~(\ref{e32-sin}) 
\textit{при условиях}~(\ref{e33-sin}) \textit{и}~(\ref{e34-sin}) 
\textit{лежат в~основе МГЛ для БН}~(\ref{e14-sin}) 
\textit{четного порядка $J_{2r} (y)$, а}~(\ref{e34-sin}) и~(\ref{e35-sin})~--- 
\textit{для БН нечетного порядка $J_{2r+1}(y)$}.

\smallskip

Рассмотрим конечномерную динамическую сис\-те\-му вида
\begin{equation}
\dot Y_t = F (Y_t, X_t, t)\,,\enskip Y(t_0) = Y_0\,.
\label{e36-sin}
\end{equation}
Здесь $Y_t$~--- вектор состояния размерности~$p^y$; $X_t \hm= A_0 \hm+
A_1\sin \psi$ $(\psi \hm=\omega t)$~--- несимметричный гармонический процесс 
размерности~$p^x$ ($A_0$ и~$A_1$~--- постоянные или медленно меняющиеся 
векторные функции на интервале $2\pi/\omega$); 
$F(Y_t, X_t, t)$~--- векторная нелинейная функция, допускающая разложение 
Фурье--Бес\-се\-ля вида~(\ref{e13-sin}):
\begin{equation}
F(Y_t, X_t, t) = \sss_l\prod_k \alpha_{lk} (X_{l,k}, t) J_n (\rho_l Y_{lk})\,.
\label{e37-sin}
\end{equation}
Здесь $\alpha_{lk}$~--- известные коэффициенты; $\rho_l$~--- положительные 
корни $J_n (y)$, расположенные в~порядке возрастания.

Используя формулы~(\ref{e5-sin})--(\ref{e8-sin}) и~применяя метод медленно 
меняющихся амплитуд, положив
    \begin{equation}
    Y_t = B_0 + B_1 \sin\psi + B_2 \cos \psi \enskip ( \psi = \omega t)\,,
    \label{e38-sin}
    \end{equation}
придем к~следующим дифференциальным уравнениям для векторов~$B_i$ $(i\hm=0,1,2)$:
    \begin{equation}
    \left.
    \begin{array}{rl}
    \dot B_0 &= F_0 (A_0, A_1, t)\,;\\[6pt] 
    \dot B_1 &= B_2 \omega + F_1 (A_0, A_1, t)\,;\\[6pt] 
    \dot B_2 &=-B_1\omega + F_2 (A_0, A_1, t)\,.
    \end{array}
    \right\}
    \label{e39-sin}
    \end{equation}
Здесь $F_i = F_i (A_0, A_1, t)$ $(i\hm=0,1,2)$~--- векторные коэффициенты МГЛ, 
вычисляемые согласно разд.~3 по формулам:

\noindent
  \begin{equation}
  \left.
  \begin{array}{rl}
  \hspace*{-3mm}F_0 (A_0, A_1, t)&=\fr{1}{2\pi}
  \displaystyle\iii_0^{2\pi} F(B_0+B_1\sin \psi + {}\\[6pt]
&\hspace*{-5mm}  {}+
  B_2\cos \psi, A_0 + A_1 \sin\psi,t)\, d\psi\,;\\[6pt]
  \hspace*{-3mm}F_1 (A_0, A_1, t)&=\fr{1}{\pi}
\displaystyle\iii_0^{2\pi} F(B_0+B_1\sin \psi + {}\\[6pt]
&\hspace*{-10mm}{}+
B_2\cos \psi, A_0 + A_1 \sin\psi,t)\sin\psi \,d\psi\,;\\[6pt]
   \hspace*{-3mm}F_2 (A_0, A_1, t)&=\fr{1}{\pi}
 \displaystyle\iii_0^{2\pi} F(B_0+
 B_1\sin \psi +{}\\[6pt]
 &\hspace*{-10mm}{}+ B_2\cos \psi, A_0 + A_1 \sin\psi,t)\cos\psi\, d\psi\,.
 \end{array}
 \right\}
 \label{e40-sin}
 \end{equation}
Для определения стационарных $B_0^*$, $B_1^*$ и~$B_2^*$ следует приравнять 
правые части~(\ref{e40-sin}) нулю:
\begin{equation}
F_i (B_0^*, B_1^*, B_2^*) =0 \enskip (i=0,1,2)\,.\label{e41-sin}
\end{equation}

\noindent
\textbf{Теорема~4.}\ \textit{Если существуют интегралы}~(\ref{e40-sin}), 
\textit{то в~основе МАМ для мед\-лен\-но\-ме\-ня\-ющих\-ся гармонических процессов 
в~динамической системе}~(\ref{e36-sin}) \textit{при}\linebreak \textit{условии}~(\ref{e37-sin}) 
\textit{лежат уравнения}~(\ref{e38-sin})--(\ref{e40-sin}), 
\textit{а~для стационарных гармонических процессов~--- уравнения}~(\ref{e41-sin}).

\smallskip

Теперь рассмотрим узкополосный гауссовский СтП, допускающий представление вида~\cite{12-sin}:
\begin{equation*}
X_t = A_t \sin \lk \omega t + \Theta (t)\rk\,.
%\label{e42-sin}
\end{equation*}
Здесь $A_t$ и~$\Theta_t$~--- медленноменяющиеся процессы на интервале времени 
$2\pi /\omega$ и~в~диапазоне частот $\Delta \omega \hm\ll \omega$. Одномерные 
плотности $a\hm= A_t$ и~$\theta \hm= \Theta_t$ определяются следующими формулами:
\begin{align*}
f(\dot a) &= \fr{1}{\Delta \omega \sqrt{2\pi D}} \exp \left( 
-\fr{\dot a}{2\Delta^2 D} \right)\,;\\
f(\dot\theta) &= \fr{\Delta^2}{2\left(\dot\theta^2 + \Delta^2\right)^{3/2}}\,.
%\label{e43-sin}
\end{align*}
В этом случае непосредственно применяется метод медленноменяющихся 
амплитуд~(\ref{e31-sin})--(\ref{e41-sin}) совместно с~формулами вычисления функций 
случайного аргумента~[8--10].

В приложении~П2 приведены тестовые примеры к~разд.~4 и~5.

\vspace*{-6pt}

\section{Заключение}

\vspace*{-2pt}

На основе приближенных методов МНА и~МГЛ 
развито методическое и~алгоритмическое обеспечение 
аналитического моделирования процессов в~динамических системах со сложными 
 (цилиндрическими первого рода и~целого порядка) БН
в~условиях гармонических, стохастических узкополосных и~широкополосных возмущений.

В приложениях П1 и~П2 приведены выражения для коэффициентов МНА (МСЛ) 
для типовых БН, а~также тестовые примеры. Особое 
внимание уделено аналитическому моделированию процессов в~осцилляторе Бесселя 
в~условиях различных возмущений.

Для учета негауссовости стохастической динамики могут использоваться методы 
моментов~\cite{9-sin, 10-sin, 13-sin} и~ортогональных разложений~\cite{9-sin, 10-sin, 14-sin, 15-sin}.

Алгоритмы положены в~основу разрабатываемого инструментального программного 
обеспечения для решения задач надежности и~безопасности сис\-тем и~средств 
информатики и~управ\-ления.

Результаты допускают обобщение на случай\linebreak других бесселевских нелинейностей, 
а~также не\-пре\-рыв\-но-дис\-крет\-ных, дискретных бесселевых ди\-на\-ми\-ческих сис\-тем, 
в~том числе с~автокоррелированными возмущениями.

\vspace*{-12pt}

{\small
\section*{\raggedleft Приложения}



\subsection*{П1.\ Коэффициенты статистической \hphantom{1.\ }и~гармонической 
линеаризации \hphantom{1.\ }типовых 
бесселевых нелинейностей}


Формулы~(\ref{e17-sin})--(\ref{e20-sin}) для коэффициентов статистической 
линеаризации $\vrp_0\hm=\vrp_0(m_y, D_y)$, $k_0\hm=k_0(m_y, D_y)$ 
и~$k_1\hm=k_1(m_y, D_y)$ с~учетом замечаний~2 и~3 при $m_y\hm\ne 0$, $D_y\hm \ne 0$ 
позволяют проводить расчеты для СБН численными методами~\cite{11-sin}, 
используя в~качестве параметров~$D_y$ и~отношение сиг\-нал--шум $\zeta \hm= m_y D_y^{-1/2}$.

Для случая $m_y=0$, используя известные формулы для определенных интегралов от 
типовых БН~\cite{6-sin, 7-sin}
\begin{multline*}
\iii_0^\infty e^{-a^2\eta^2} \eta^{\mu-1} J_n (b\eta)\, d\eta = {}\\[2pt]
{}=\fr{\Gamma \left((n+\mu)/2\right) \left(b/(2a)\right)^n}{2a^\mu \Gamma (n+1)}\,
\mathcal{K} \left( \fr{n+\mu}{2}, n+1, -\fr{b^2}{4a^2}\right)\\[2pt]
 \left(\mathrm{Re}\,(\mu+n) >o\,,\enskip
    \mathrm{Re}\, a^2 >0\right)\,;
%    \label{e1.1-sin}
    \end{multline*}

\vspace*{-12pt}

\noindent
\begin{multline*}
\iii_0^\infty e^{-a^2\eta^2} J_n^2 (b\eta) \,d\eta = \fr{1}{2a^2} \exp  
\left(-\fr{b^2}{2a^2}\right)I_n \left( \fr{b^2}{2a^2}\right) \\[2pt]
    \left(\mathrm{Re} \, n>1\,,\enskip b>0\right)
%\label{e1.2=sin}
\end{multline*}
($\mathcal{K} (x,y,z)$~--- вырожденная функция Куммера~\cite{7-sin},  
$I_n (x)$~--- функция Бесселя мнимого аргумента (замечание~1)), 
приходим к~следующим результатам:
\begin{multline}
Z= J_n (Y) \,;\enskip \vrp_0 (0,D) ={}\\[2pt]
\hspace*{-5mm}{}= \fr{\Gamma \left(({n+1})/2\right) }{\sqrt{\pi} \Gamma(n+1)} \left( 
\fr{D}{2}\right)^{\!n/2}\! \mathcal{K}
     \left( \fr{n+1}{2} , n+1, -\fr{D}{2}\right);
     \label{e1.3-sin}
     \end{multline}
     
     \vspace*{-12pt}
     
     \begin{multline*}
Z= J_n (bY) \,;\enskip \vrp_0 (0,D) = {}\\[2pt]
{}=\fr{\Gamma \left((n+1)/2\right) }
{\sqrt{\pi} \Gamma(n+1)} \left( b\sqrt{\fr{D}{2}}\right)^{\!n}\! \mathcal{K}
     \left( \fr{n+1}{2} , n+1, -\fr{b^2D}{2}\right);
%     \label{e1.4-sin}
     \end{multline*}
     
          \vspace*{-12pt}
     
     \begin{multline*}
    Z= Y^{\mu-1}J_n (bY)\,;\enskip \vrp_0 (0,D) ={}\\[2pt]
    \hspace*{-5mm}{}= \fr{2^{(\mu-1)/2}}{\sqrt{\pi} \Gamma(n+1)}D^{(\mu-1)/2}\, \mathcal{K}
     \left( \fr{n+\mu}{2} , n+1, -\fr{b^2D}{2}\right);
     \end{multline*}
     
     \vspace*{-12pt}
     
     \noindent
\begin{multline}
k_1 (0,D)={}\\[2pt]
{}= \fr{2^{\mu/2} D^\mu}{\sqrt{\pi} \Gamma(n+1)}\,\mathcal{K}
\left( \fr{n+\mu+1}{2} , n+1, -\fr{b^2D}{2}\right)\,; 
\label{e1.5-sin}
\end{multline}

\vspace*{-9pt}

\begin{equation*}
Z= J_n^2 (Y) \,;\enskip \vrp_0 (0,D) = \sqrt{\fr{2D}{\pi}}\,e^{-D} I_n (D)\,;
%\label{e1.6-sin}
\end{equation*}
\begin{equation}
Z= J_n^2 (bY) \,;\enskip \vrp_0 (0,D) = \sqrt{\fr{2D}{\pi}}\,e^{-b^2D} I_n (b^2D)\,.
\label{e1.7-sin}
\end{equation}

Для расчета коэффициентов гармонической линеаризации БН и~СБН используются 
формулы~(\ref{e33-sin})--(\ref{e35-sin}) и~(\ref{e40-sin}). 
При этом обычно используются численные методы~\cite{11-sin}.




\subsection*{П2.\ Тестовые примеры}


\noindent
\textbf{Пример~1.}\
Для одномерной бесселевой системы с~аддитивным гауссовским шумом
  \begin{equation*}
  \dot Y_t = \alpha +\beta J_{2r+1} (b Y_t) + \gamma V
%  \label{e2.1-sin}
  \end{equation*}
($\alpha$, $\beta$, $b$ и~$\gamma$~--- постоянные; $V$~--- белый шум интенсивности~$\nu$) 
МАМ на основе МСЛ (теорема~2) приводит к~следующим результатам:
   \begin{equation}
   \left.
   \begin{array}{rl}
   \dot m_y &= \alpha +\beta \vrp_0^{2r+1} (m_y, D_y)\,;\\[6pt] 
   \dot D_y &= 2\beta k_1^{2r+1}(m_y, D_y) D_y +\si\,.
   \end{array}
   \right\}
   \label{e2.2-sin}
   \end{equation}
Здесь введены следующие обозначения:
\begin{align*}
    \si&= \gamma^2 \nu\,;\\
   % k_1^{2r+1} (m_y, D_y) &=\fr{\prt}{\prt m_y} \lk 
%    k_0^{2r+1}(m_y, D_y) m_y\rk\,;\\
    \vrp_0^{2r+1} (m_y, D_y) &={}\\
    &\hspace*{-10pt} {}=\fr{1}{\sqrt{2\pi D_y}} \iin J_{2r+1} (b\eta) 
    e^{-(\eta-m_y)^2/2D_y}\, d\eta\,;
  \\
k_1^{2r+1}(m_y, D_y) &=\fr{\prt \vrp_0^{2r+1}(m_y, D_y)}{\prt m_y}\,.
%\label{e2.3-sin}
\end{align*}

При $\beta<0$, приравнивая нулю правые части~(\ref{e2.2-sin}), 
получим уравнения для стационарных $m^*$ и~$D^*$:
\begin{equation*}
\alpha+\beta \vrp_0^{2r+1}(m^*, D^*) =0;\enskip 
2\beta k_1^{2r+1}(m^*, D^*) D^* + \gamma^2 \nu^* =0.
%\label{e2.4-sin}
\end{equation*}
Отсюда при $\alpha=0$ находим, что  $m^*\hm=0$, а~$D^*$ с~учетом~(\ref{e1.5-sin}) 
при $\mu\hm=1$ определяется из уравнения
\begin{equation*}
k^{2r+1}(0,D^*)D^* + \delta^* =0\enskip \left(\delta^* = \fr{\gamma^2 \nu^* }{2\beta}\right).
%\label{e2.5-sin}
\end{equation*}

\noindent
\textbf{Пример~2.}\
Для одномерной бесселевой системы с~мультипликативным шумом
\begin{equation*}
\dot Y_t = \alpha + \beta Y_t + \gamma J_{2r+1} (bY_t) V %\label{e2.6-sin}
\end{equation*}
($\alpha$, $\beta$, $\gamma$ и~$b$~--- постоянные; 
$V$~--- гауссовский белый шум интенсивности~$\nu$) МАМ на основе МНА (теорема~2) 
приводит к~следующим результатам:
\begin{equation}
\left.
\begin{array}{rl}
    \dot m_y &= \alpha+\beta m_y\,;\\[6pt]
    \dot D_y &= 2 \beta D_y + \si(m_y, D_y)\,;
\\[6pt]
    \si(m_y, D_y) &= {}\\[6pt]
    &\hspace*{-15mm}{}=\fr{\gamma^2 \nu}{\sqrt{2\pi D_y}} 
   \displaystyle \iin J_{2r+1}^2 (b\eta) e^{-(\eta- m_y)^2/(2D_y)}\, d\eta\,.
    \end{array}
    \right\}
    \label{e2.7-sin}
    \end{equation}
При $\beta<0$, приравнивая нулю правые части~(\ref{e2.7-sin}), придем 
к~следующим уравнениям для стационарных~$m^*$ и~$D^*$:
\begin{equation*}
\alpha+\beta m^*=0\,;\enskip 2\beta D^* +\si(m^*, D^*) =0\,.
%\label{e2.8-sin}
\end{equation*}
Отсюда при  $\alpha\hm=0$ следует, что  $m^*\hm=0$,  а~$D^*$ с~учетом~(\ref{e1.7-sin}) 
определяется из уравнения
\begin{equation*}
\sqrt{D^*} I_{2r+1}(b^2 D^*) e^{-b^2D^*} = \sqrt{\fr{\pi}{2}}\,. 
%\label{e2.9-sin}
\end{equation*}

\noindent
\textbf{Пример~3.}\
Рассмотрим сначала двумерную СтС, описывающую стохастическую динамику осциллятора 
Бесселя:
\begin{equation*}
\left.
\begin{array}{c}
\ddot Y +\omega_0^2 J_{2r+1} (Y)=0 \mbox{ при }
    Y_1=Y\,,\ \dot Y_1 = Y_2\,; \\[6pt]
     \dot Y_2 = \Pi' (Y_1)\,; \ Y_1(t_0) = Y_{10}\,;\ 
     Y_2(t_0) = Y_{20}\,.
     \end{array}
     \right\}
%     \label{e2.10-sin}
     \end{equation*}
Здесь $Y_1$ и~$Y_2$~--- координата и~скорость; 
$\Pi_{Y_1}' (Y_1) \hm= \prt  \Pi (Y_1)/ \prt Y_1\hm= -\omega_0^2 J_{2r+1} (Y_1)$;
$\Pi'' (Y_1) \hm=- \omega_0^2 J_{2r+1}'(Y_1) \hm= - (\omega_0^2/2) \lk J_{2r} (Y_1) \hm- 
J_{2r+2}(Y_1)\rk$.

Пользуясь формулами~(\ref{e11-sin}) для интеграла от бесселевой функции, находим 
точное выражение для потенциальной энергии $\Pi (Y_1)$:
\begin{multline*}
    \Pi_{2r+1} (Y_1) =-\omega_0^2 \iii_0^{Y_1} J_{2r+1}(\eta) \,d\eta={}\\
    {}= -\omega_0^2 \lk 1- J_0 (Y_1) - 2 \sss_{k=0}^r J_{2k} (Y_1)\rk.
%    \label{e2.11-sin}
    \end{multline*}
При этом функция Гамильтона будет иметь вид:
\begin{equation}
\mathcal{H} (Y_1, Y_2) = \fr{Y_2^2}{2} -\omega_0^2 \lk 1- J_0 (Y_1) - 
2 \sss_{k=0}^r J_{2k} (Y_1)\rk. \label{e2.12-sin}
\end{equation}
Нули $J_n(Y_1)$ действительны, разделяются нулями $J_{n+1} (Y_1)$ 
и~$J_n'(Y_1)$. При этом выполняются условия~(\ref{e7-sin}).

Фазовый портрет осциллятора Бесселя при произвольных начальных 
условиях определяется следующими дифференциальными или интегральными соотноше\-ниями:
\begin{equation*}
\fr{dY_2}{dY_1} = \fr{ -\omega_0^2 J_{2r+1} (Y_1)}{Y_2}\,; %\label{e2.13-sin}
\end{equation*}
\begin{equation*}
Y_2^2 = h- 2\Pi_{2r+1} (Y_1)\,,\enskip 
h = Y_{20}^2 + 2 \omega_0^2 \iii_0^{Y_{10}} J_{2r+1} (\eta)\, d\eta\,. %\label{e2.14-sin}
\end{equation*}
Особыми точками являются множества $\{Y_2\hm=0$, $J_{2r+1} (Y_1) \hm=0\}$. Для их 
устойчивости достаточно, чтобы $\Pi'' (Y_1) \hm>0$.

Полученные соотношения лежат в~основе МАМ при детерминированных 
и~случайных начальных условиях.


Наличие сил сопротивления (возбуждения), зависящих от параметра~$U$,
в~виде постоянной состав\-ля\-ющей $L_0 \hm= L_0 (U)$ и~переменной составляющей 
$L_1\hm=L_1(Y_1, Y_2,U)$ в~уравнениях осциллятора, когда
\begin{equation*}
    \dot Y_1 = Y_2\,;\enskip \dot Y_2 = \Pi' (Y_1, U) + L_0 (U) -L_1(Y_1, Y_2, U)\,,
%    \label{e2.15-sin}
    \end{equation*}
приводит к~следующему дифференциальному уравнению фазового портрета:
\begin{equation*}
\fr{dY_2}{dY_1} =\fr{-\omega_0^2 J_{2r+1} (Y_1) + L_0(U) - L_1 (Y_1, Y_2, U)}{Y_2}\,.
%\label{e2.16-sin}
\end{equation*}

Особыми точками фазового портрета являются множества $\{Y_2 \hm=0$, 
$-\omega_0^2 J_{2r+1} (Y_1)\hm + L_0 (U)\hm - L_1 (Y_1, Y_2, U)\hm =0\}$.
Полученные соотношения лежат в~основе МАМ при детерминированных 
и~случайных значениях начальных условий~$Y_{10}$, $Y_{20}$ и~параметра~$U$.

\pagebreak

Теперь рассмотрим аналитическое моделирование процессов в~осцилляторе 
Бесселя при наличии сил вязкого
трения и~гармонического возмущения, основываясь на уравнениях
\begin{equation}
\dot Y_1 = Y_2\,,\enskip \dot Y_2 = - \omega_0^2 J_{2r+1} (Y_1) - 2 \varepsilon 
\omega_0 Y_2 +A \sin \omega t\,.\label{e2.17-sin}
\end{equation}
Выполним в~(\ref{e2.17-sin}) замену переменных
$Y_1 \hm= B \sin \psi$ и~$Y_2 \hm= B\omega \cos \psi$ $(\psi \hm= \omega t \hm+\theta)$ 
и~проведем гармоническую линеаризацию $J_{2r+1} (Y_1)$, положив
    $$
    J_{2r+1} (Y_1) \approx q_{2r+1} (B) B \sin \psi\,;
    $$
    
    \vspace*{-12pt}
    
    \noindent
    \begin{multline*}
q_{2r+1} (B) = \fr{1}{\pi B} \iii_0^{2\pi} J_{2r+1} (B \sin \psi) \sin \psi\, d\psi={}\\
{}=
 \fr{1}{\pi B} \iii_0^{2\pi} \left[ \fr{(A\sin\psi)^{2r+1}}{2^{2r+1}} 
\sss_{k=0}^\infty \left(-\fr{A^2 \sin^2\psi}{4}\right)^k\right.\times{}\\
{}\times
\left.    \fr{1}{k! (2r+1+k)!}\right]\, d\psi\,. %\label{e2.18-sin}
    \end{multline*}
В результате уравнения~(\ref{e2.17-sin}) примут следующий вид:
\begin{equation}
\left.
\begin{array}{c}
\dot B = L_c \cos^2 \psi + L_s \sin \psi \cos\psi\,;\\[6pt] 
B\dot\theta =-L_c \sin\psi \cos\psi - L_s \sin^2\psi\,,
\end{array}
\right\}
\label{e2.19-sin}
\end{equation}
где
\begin{align*}
L_c &= -\fr{2\varepsilon \omega_0}{\omega} B + \fr{A}{\omega} 
\sin \theta\,;\\
L_s &= \fr{\omega^2 - \omega_0^2}{\omega} q_{2r+1} (B) B + \fr{A}{\omega} \cos \theta\,.
\end{align*}

Далее после осреднения по~$\psi$ правых частей~(\ref{e2.19-sin}) 
придем к~искомым уравнениям для осредненных~$B$ и~$\theta$:
\begin{equation}
\dot B =\fr{1}{2}\langle L_c (B, \theta, A,\omega)\rangle\,;\enskip 
\dot \theta =-\fr{1}{2}\,\fr{\langle L_s (B,\theta,A,\omega)\rangle}{B}\,,\label{e2.20-sin}
\end{equation}
где $\langle\cdots\rangle$~--- символ осреднения по~$\psi$.
Для определения стационарных~$B^*$ и~$\theta^*$ имеем конечные уравнения:
\begin{equation*}
\langle L_c (B^*,\theta^*,A,\omega)\rangle=0\,;\enskip 
\langle L_s (B^*,\theta^*,A,\omega)\rangle =0\,.
%\label{e2.21-sin}
\end{equation*}
При этом, как следует из~(\ref{e2.20-sin}), резонансное соотношение имеет следующий вид:
\begin{equation}
\left(\fr{\omega_0}{\omega}\right)^2 = q_{2r+1} (B)\,.
\label{e2.22-sin}
\end{equation}

Полученные соотношения лежат в~основе МАМ вынужденных гармонических колебаний.

Рассмотрим уравнения
\begin{equation}
\dot Y_1 = Y_2\,;\enskip 
\dot Y_2 = -\omega_0^2 J_{2r+1} (Y_1)+ L_0 - 2\varepsilon \omega_0 Y_2 +V\,,
\label{e2.23-sin}
\end{equation}
описывающие процессы в~осцилляторе Бесселя при наличии постоянного возмущения~$L_0$,
сил вязкого трения и~широкополосного гауссовского стохастического процесса вида  
белого шума интенсивности $\nu^V\hm=\nu$.

При $L_0=0$ и~$\varepsilon \omega_0\hm >0$ точное стационарное распределение~$Y_1$ 
и~$Y_2$ определяется формулой Гиббса~\cite{9-sin, 10-sin} для функции 
Гамильтона~(\ref{e2.12-sin}):
\begin{equation*}
f^* (y_1, y_2) =c \exp \lk - \fr{4\varepsilon\omega_0}{\nu}\,\mathcal{H} 
(y_1, y_2)\rk, %\label{e2.24-sin}
\end{equation*}
где $c$~--- постоянная нормировки плотности.
Отсюда следует, что распределение~$Y_2$~--- гауссовское 
с~параметрами $m_{y_2}^* \hm=0$, $D_2^* \hm= \nu/(2\varepsilon\omega_0)$, 
а~распределение~$Y_1$~---  негауссовское:

\noindent
    \begin{equation*}
    f_{y_1}^* (y_1) = c \exp \lk - \fr{4\varepsilon \omega_0}{\nu} \Pi_{2r+1} (y_1) \rk.
%    \label{e2.25-sin}
\end{equation*}

Точные значения математического ожидания и~дисперсии для~$Y_1$ равны:

\noindent
\begin{equation*}
m_{y_1}^* =0\,;\enskip D_1^* = c \iin \eta \exp \lk - \fr{4\varepsilon\omega_0}{\nu}\,
\Pi_{2r+1} (\eta) \rk \,d\eta\,. %\label{e2.26-sin}
\end{equation*}

Для уравнений~(\ref{e2.22-sin}) при  $L_0\hm\ne 0$, согласно теореме~2, находим:
\begin{equation}
\left.
\hspace*{-3mm}\begin{array}{rl}
\dot m_1 &= m_2\,;\\[6pt] 
\dot m_2 &=-\omega_0^2 \vrp_0^{2r+1} (m_1, D_1) + L_0 - 2 \varepsilon \omega_0 m_2\,;
\\[6pt]
\dot D_1 &= 2 K_{12}\,;\\[6pt] 
\dot D_2 &= - 2 \lk \omega_0^2 k_1^{2r+1} (m_1, D_1) K_{12} + 2\varepsilon \omega_0 D_2\rk + \nu\,;
\\[6pt]
\dot K_{12} &= D_2 - \omega_0^2 k_1^{2r+1}(m_1, D_1) D_1 - 2\varepsilon \omega_0 K_{12}\,.
\end{array}
\right\}
\label{e2.27-sin}
\end{equation}
%\label{e2.28-sin}
Здесь

\noindent
   \begin{multline*}
    \vrp_0^{2r+1}(m_1, D_1) ={}\\[-1pt]
    {}= \fr{1}{\sqrt{2\pi D_1}} 
    \iin J_{2r+1}(\eta) e^{-(\eta- m_y)^2/(2D_y)}\, d\eta\,;
\end{multline*}

\vspace*{-9pt}

\noindent
    $$
    k_1^{2r+1} (m_1 D_1) = \fr{\prt \vrp_0^{2r+1} (m_1, D_1)}{\prt m_1}\,.
    $$
Отсюда при $\varepsilon \hm>0$ и~$L_0 \hm=0$ в~стационарном случае имеем:
\begin{equation}
\left.
\begin{array}{c}
m_1^* =0\,; \enskip m_2^* =0\,;\enskip K_{12}^* =0\,;\\[6pt]
D_2^* =  \fr{\nu^*}{4 \varepsilon \omega_0}\,;\enskip
    k_1^{2r+1} (0, D_1^*)D_1^* = \fr{\nu^* }{4\varepsilon \omega_0^2}\,.
    \end{array}
    \right\}
    \label{e2.29-sin}
    \end{equation}
Для входящего в~(\ref{e2.29-sin}) коэффициента статистической линеаризации 
в~силу~(\ref{e1.3-sin}) находим следующее выражение:
    \begin{multline}
    k_1^{2r+1} (h) = \fr{4\Gamma ((2r+3)/2) }{\Gamma (2r+2) (2h)^{2r+3}} \,
    \mathcal{K} \left(\fr{2r+3}{2}, 2r+2, \xi\right)\\
    \left( h^2 =\fr{ 1}{2D_1^*}\right).
    \label{e2.30-sin}
    \end{multline}

Пользуясь асимптотическими разложениями функции Куммера~\cite{7-sin}, 
из~(\ref{e2.30-sin}) получаем следующие формулы:
   \begin{equation}
    \hspace*{-2mm}k_1^{2r+1} (h) \sim \!
    \begin{cases}
    \fr{4\Gamma ((2r+3)/2)}{\Gamma(2r+2) (2h)^{2r+3}}&\! \mbox{при}\ 
    D_1\to 0\,;\\
%     \label{e2.31-sin}
\fr{\Gamma (2r+2)}{\Gamma((2r+1)/2)} \cdot
{(2h)^{2r+3}}&\! \mbox{при}\  D_1\to \infty\,,\end{cases}\!\!\!\!\!\!
     \label{e2.32-sin}
    \end{equation}
где $\Gamma(n+1) \hm= n!$; $h^2 \hm= 1/(2D_1)$.

\pagebreak

Формулы~(\ref{e2.32-sin}) для $n\hm=2r\hm+1$ при $r\hm=0$ 
и~1 имеют соответственно вид:
    \begin{alignat*}{2}
    k_1^{(1)} &\sim \fr{2\sqrt{\pi}}{(2h)^3}\,; &\ 
    k_1^{(1)} &\sim \fr{2}{\sqrt{\pi}} (2h)^3\,,\\
    k_1^{(3)} &\sim \fr{\sqrt{\pi}}{12(2h)^5}\,; &\quad
    k_1^{(3)} &\sim \fr{12}{\sqrt{\pi}} (2h)^5\,.
    \end{alignat*}

На основе уравнений~(\ref{e2.27-sin}) 
строятся эллипсы рассеивания для зависимостей
$D_1 \hm= D_1 (t)$, $D_2\hm = D_2 (t)$ и~$K_{12}\hm = K_{12} (t)$ 
в~переходном и~стационарном режимах.

\noindent
\textbf{Замечание~6.}
Аналогично МНА (МСЛ) используется для моделирования динамики осциллятора Бесселя 
в~случае аддитивных и~мультипликативных возмущений вида 
$[ 1\hm+\gamma (Y_1, Y_2, U)]V$ в~(\ref{e2.23-sin}).



}


{\small\frenchspacing
{%\baselineskip=10.8pt
\addcontentsline{toc}{section}{Литература}
\begin{thebibliography}{99}
\bibitem{1-sin}
\Au{Синицын И.\,Н.}
Параметрическое статистическое и~аналитическое моделирование распределений 
в~нелинейных стохастических системах на многообразиях~// 
Информатика и~её применения, 2013. Т.~7. Вып.~2. С.~4--16.

\bibitem{2-sin}
\Au{Синицын И.\,Н., Синицын В.\,И. }
Аналитическое моделирование нормальных процессов 
в~стохастических системах со сложными нелинейностями~// Информатика и~её 
применения, 2014. Т.~8. Вып.~3. С.~12--18.

\bibitem{3-sin}
\Au{Синицын И.\,Н., Синицын В.\,И., Сергеев~И.\,В., Белоусов~В.\,В., Шоргин~В.\,С.}
Математическое обеспечение аналитического моделирования стохастических сис\-тем 
со сложными нелинейностями~// Системы и~средства информатики, 2014. 
Т.~24. №\,3. С.~4--17.

\bibitem{4-sin}
\Au{Синицын И.\,Н., Синицын~В.\,И., Корепанов~Э.\,Р.}
Моделирование процессов в~стохастических системах 
со сложными трансцендентными нелинейностями~// Информатика и~ее применения, 2015. 
Т.~9. Вып.~2. С.~23--29.

\bibitem{5-sin}
\Au{Грей Э., Мэтьюз Г.}
Функции Бесселя и~их приложения к~физике и~механике.~--- 
М.: ИЛ, 1953. 371~с.

\bibitem{6-sin}
\Au{Градштейн И.\,С., Рыжик~И.\,М. }
Таблицы интегралов, сумм, рядов и~произведений.~--- М.: ГИФМЛ, 1963. 1100~с.

\bibitem{7-sin}
Справочник по специальным функциям~/ Под ред. М.~Абрамовича, И.~Стигана.~--- 
М.: Наука, 1979. 832~с.

\bibitem{8-sin}
\Au{Синицын И.\,Н.,  Синицын В.\,И. }
Лекции по нормальной и~эллипсоидальной аппроксимации распределений 
в~стохастических системах.~--- М.: ТОРУС ПРЕСС, 2013. 488~с.

\bibitem{9-sin}
\Au{Пугачёв В.\,С., Синицын~И.\,Н.}
Стохастические дифференциальные системы. Анализ и~фильтрация.~--- М.:
Наука,  1990.  632~с. (\Au{Pugachev~V.\,S., Sinitsyn~I.\,N.}
Stochastic differential systems.
Analysis and filtering.~--- Chichester\,--\,New York, NY, USA: Jonh Wiley, 1987.
549~p.)

\bibitem{10-sin}
\Au{Пугачёв В.\,С., Синицын~И.\,Н.}
Теория стохастических систем.~--- М.: Логос, 2000; 2004. 1000~с.
(\Au{Pugachev~V.\,S., Sinitsyn~I.\,N.} Stochastic systems. Theory and  applications.~---
Singapore: World Scientific, 2001. 908~p.)

\bibitem{11-sin}
\Au{Попов Б.\,А., Теслер~Г.\,С. }
Вычисление функций на ЭВМ: Справочник.~--- Киев: Наукова Думка, 1984. 599~с.

\bibitem{12-sin}
\Au{Свешников А.\,А.}
Прикладные методы теории случайных функций.~--- М.: Наука, 1968. 462~с.

\bibitem{13-sin}
\Au{Синицын И.\,Н.}
Методы моментов в~задачах аналитического моделирования распределений 
в~нелинейных стохастических системах на многообразиях~// Системы и~средства 
информатики, 2015. Т.~25. №\,3. С.~24--43.

\bibitem{14-sin}
\Au{Синицын И.\,Н.}
Аналитическое моделирование распределений методом ортогональных разложений 
в~нелинейных стохастических системах на многообразиях~// Информатика и~её 
применения, 2015. Т.~9. Вып.~3. С.~17--24.

\bibitem{15-sin}
\Au{Синицын И.\,Н.}
Применение ортогональных разложений для аналитического моделирования 
многомерных распределений в~нелинейных стохастических системах на многообразиях~// 
Системы и~средства информатики, 2015. Т.~25. №\,3. С.~4--23. 
     \end{thebibliography}

 }
 }

\end{multicols}

\vspace*{-12pt}

\hfill{\small\textit{Поступила в~редакцию 13.08.15}}

\vspace*{8pt}

%\newpage

%\vspace*{-24pt}

\hrule

\vspace*{2pt}

\hrule

%\vspace*{8pt}

\def\tit{ANALYTICAL MODELING OF PROCESSES
IN~DYNAMICAL SYSTEMS WITH CYLINDRIC
BESSEL NONLINEARITIES}

\def\titkol{Analytical modeling of processes
in~dynamical systems with cylindric
Bessel nonlinearities}

\def\aut{I.\,N.~Sinitsyn}

\def\autkol{I.\,N.~Sinitsyn}

\titel{\tit}{\aut}{\autkol}{\titkol}

\vspace*{-9pt}


\noindent
Institute of Informatics Problems,
Federal Research Center ``Computer Science and Control'' of
the Russian Academy of Sciences, 44-2 Vavilov Str.,
Moscow 119333, Russian Federation


\def\leftfootline{\small{\textbf{\thepage}
\hfill INFORMATIKA I EE PRIMENENIYA~--- INFORMATICS AND
APPLICATIONS\ \ \ 2015\ \ \ volume~9\ \ \ issue\ 4}
}%
 \def\rightfootline{\small{INFORMATIKA I EE PRIMENENIYA~---
INFORMATICS AND APPLICATIONS\ \ \ 2015\ \ \ volume~9\ \ \ issue\ 4
\hfill \textbf{\thepage}}}

\vspace*{3pt}




    \Abste{Methods of analytical modeling (MAM) for processes in dynamical systems with 
complex Bessel nonlinearities with
harmonically and stochastically wide and narrow band disturbances are given. 
Neccessary elements of the cylindric Bessel
functions theory and complex Bessel nonlinearities are presented. Methodological
and algorithmical support for MAM  based on the
 statistical linearization method (SLM) and the normal approximation method  
 for wide-band stochastic processes (white noise) is developed.
 Pecularities of MAM for\linebreak\vspace*{-12pt}}

\Abstend{harmonical and narrow band stochastic processes are discussed. 
 Test examples of one-dimensional systems with additive and
 multiplicative noises and Bessel nonlinearities and  for Bessel oscillator with 
 various disturbances are given. Conclusions and some generalizations are mentioned.}
  
  \KWE{Bessel nonliearity;
Bessel oscillator;
complex Bessel nonlinearity;
Gibbs formula; harmonical process; Kummer function;
method of analytical modeling; narrow band stochastic processes;
normal approximation method (NAM); statistical linearization method (SLM);
stochastic system on manifold (MStS); white noise; wide band stochastic process} 
    
    
    
\DOI{10.14357/19922264150404}

\vspace*{-12pt}

\Ack
    \noindent
The work was supported by the Russian Foundation for Basic Research (project 15-07-02244).



%\vspace*{-6pt}

  \begin{multicols}{2}

\renewcommand{\bibname}{\protect\rmfamily References}
%\renewcommand{\bibname}{\large\protect\rm References}



{\small\frenchspacing
 {%\baselineskip=10.8pt
 \addcontentsline{toc}{section}{References}
 \begin{thebibliography}{99}



\bibitem{1-sin-1}
\Aue{Sinitsyn, I.\,N.} 2013.
Parametricheskoe statisticheskoe i~analiticheskoe modelirovanie raspredeleniy 
v~nelineynykh stokhasticheskikh sistemakh na mno\-go\-ob\-ra\-zi\-yakh
[Parametrical statistical and analytical modeling of distributions 
in nonlinear stochastic systems on manifolds].
\textit{Informatika i~ee Primeneniya}~--- \textit{Inform. Appl.} 7(2):4--16.

\bibitem{2-sin-1}
\Aue{Sinitsyn, I.\,N., and V.\,I.~Sinitsyn}.
2014.
Analiticheskoe modelirovanie normal'nykh protsessov 
v~sto\-kha\-sti\-che\-skikh sistemakh so slozhnymi nelineynostyami
[Analytical modeling of normal processes in stochastic systems with complex nonlinearities].
\textit{Informatika i~ee Primeneniya}~--- \textit{Inform. Appl.}  8(3):12--18.

\bibitem{3-sin-1}
\Aue{Sinitsyn, I.\,N., V.\,I.~Sinitsyn, I.\,V.~Sergeev, V.\,V.~Belousov, and
V.\,S.~Shorgin.} 2014.
Matematicheskoe obespechenie analiticheskogo modelirovaniya stokhasticheskikh 
sistem so slozhnymi nelineynostyami
[Mathematical software tools for analytical modeling 
in stochastic systems with complex nonlinearities]. 
\textit{Sistemy i~Sredstva Informatiki}~---
\textit{Systems and Means of Informatics} 24(3):4--17.

\bibitem{4-sin-1}
\Aue{Sinitsyn, I.\,N., V.\,I.~Sinitsyn, and E.\,R.~Korepanov.} 2015.
Modelirovanie protsessov v~stokhasticheskikh sistemakh so slozhnymi 
transtsendentnymi nelineynostyami
[Modeling of normal processes in stochastic systems with 
complex transcendental  nonlinearities]. \textit{Informatika i~ee Primeneniya}~---
\textit{Inform. Appl.} 9(2):23--29.

\bibitem{5-sin-1}
\Aue{Grey, E., and G.~Met'yuz.} 1953.
\textit{Funktsii Besselya i~ikh prilozheniya k~fizike i~mekhanike}.~--- 
Moscow: IL. 371~p.

\bibitem{6-sin-1}
\Aue{Gradshteyn, I.\,S., and I.\,M.~Ryzhik.} 1963.
\textit{Tablitsy integralov, summ, ryadov i~proizvedeniy}
[Tables of integrals, sums, series and products]. Moscow: GIFML.  1100~p.

\bibitem{7-sin-1}
Abramovich,~M., and I.~Stigan, eds. 1979.
\textit{Spravochnik po spetsial'nym funktsiyam}
[Handbook of mathematical functions]. Moscow: Nauka. 832~p.

\bibitem{8-sin-1}
\Aue{Sinitsyn, I.\,N.,  and V.\,I.~Sinitsyn.} 2013.
\textit{Lektsii po normal'noy i~ellipsoidal'noy approksimatsii raspredeleniy 
v~stokhasticheskikh sistemakh}
[Lectures on normal and ellipsoidal approximation of distributions in stochastic 
systems]. Moscow: TORUS PRESS.  488~p.

\bibitem{9-sin-1}
\Aue{Pugachev, V.\,S., and I.\,N.~Sinitsyn.} 
1987. \textit{Stochastic differential systems.
Analysis and filtering}. Chichester\,--\,New York, NY: Jonh Wiley.
549~p.

\bibitem{10-sin-1}
\Aue{Pugachev, V.\,S., and I.\,N.~Sinitsyn.} 
 2001.  \textit{Stochastic systems. Theory and  applications}.
Singapore: World Scientific. 908~p.

\bibitem{11-sin-1}
\Aue{Popov, B.\,A., and G.\,S.~Tesler.} 1984.
\textit{Vychislenie funktsiy na EVM: Spravochnik}
[Calculation of functions on the computer: Handbook]. Kiev: Naukova Dumka.  599~p.

\bibitem{12-sin-1}
\Aue{Sveshnikov, A.\,A. } 1968.
\textit{Prikladnye metody teorii sluchaynykh funktsiy}
[Applied methods in theory of random functions].  Moscow: Nauka.  462~p.

\bibitem{13-sin-1}
\Aue{Sinitsyn, I.\,N.} 2015.
Metody momentov v~zadachakh analiticheskogo modelirovaniya raspredeleniy 
v~nelineynyh stokhasticheskikh sistemakh na mno\-go\-ob\-ra\-zi\-yakh
[Moments methods for analytical modeling of stochastic systems on manifolds].
\textit{Sistemy i~Sredstva Informatiki}~---
\textit{Systems and Means of Informatics} 25(3):24--43.

\bibitem{14-sin-1}
\Aue{Sinitsyn, I.\,N.} 2015.
Analiticheskoe modelirovanie raspre\-deleniy metodom ortogonal'nykh razlozheniy 
v~ne\-li\-ney\-nykh stokhasticheskikh sistemakh na mno\-go\-ob\-ra\-zi\-yakh
[Analytical modeling in~stochastic systems on manifolds based on orthogonal expansions].
\textit{Informatika i~ee Primeneniya}~---
\textit{Inform Appl.} 9(3):17--24.

\bibitem{15-sin-1}
\Aue{Sinitsyn, I.\,N.} 2015.
Primenenie ortogonal'nykh raz\-lo\-zhe\-niy dlya analiticheskogo modelirovaniya 
mno\-go\-mer\-nykh raspredeleniy v~nelineynykh stokhasticheskikh sistemakh 
na mnogoobraziyakh
[Applications of orthogonal expansions for analytical modeling
of multidimensional distributions in stochastic systems on manifold].
\textit{Sistemy i~Sredstva Informatiki}~---
\textit{Systems and Means of Informatics} 25(3):4--23.
\end{thebibliography}

 }
 }

\end{multicols}

\vspace*{-6pt}

\hfill{\small\textit{Received August 13, 2015}}

\vspace*{-12pt}
    

\Contrl

\noindent
\textbf{Sinitsyn Igor N.} (b.\ 1940)~---
Doctor of Science in technology, professor,
Honored scientist of RF, Head of Department, 
Institute of Informatics Problems, 
Federal Research Center ``Computer Science and
Control'' of the Russian Academy of Sciences, 44-2 Vavilov Str.,
Moscow 119333, Russian Federation; sinitsin@dol.ru

 \label{end\stat}


\renewcommand{\bibname}{\protect\rm Литература}