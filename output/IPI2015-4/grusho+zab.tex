\def\stat{gr+zab}

\def\tit{КОНТРОЛЬ И УПРАВЛЕНИЕ ИНФОРМАЦИОННЫМИ ПОТОКАМИ 
В~ОБЛАЧНОЙ СРЕДЕ$^*$}

\def\titkol{Контроль и~управление информационными потоками 
в~облачной среде}

\def\aut{А.\,А.~Грушо$^1$, М.\,И.~Забежайло$^2$, А.\,А.~Зацаринный$^3$}

\def\autkol{А.\,А.~Грушо, М.\,И.~Забежайло, А.\,А.~Зацаринный}

\titel{\tit}{\aut}{\autkol}{\titkol}

{\renewcommand{\thefootnote}{\fnsymbol{footnote}} \footnotetext[1]
{Работа поддержана РФФИ (проект 15-29-07981).}}


\renewcommand{\thefootnote}{\arabic{footnote}}
\footnotetext[1]{Институт проблем информатики Федерального исследовательского
центра <<Информатика и~управление>> Российской академии наук,
grusho@yandex.ru}
\footnotetext[2]{ВИНИТИ РАН, m.zabezhailo@yandex.ru}
\footnotetext[3]{Федеральный исследовательский
центр <<Информатика и~управление>> Российской академии наук, azatsarinny@ipiran.ru}


  \Abst{Контроль и~управление информационными потоками в~облачной среде служат 
важным механизмом информационной безопасности. Основная проблема в~реализации 
такого механизма~--- скорость коммутации пакетов в~соответствии с~заданными правилами. 
В ~статье рассмотрены методы контроля и~управления информационными потоками 
в~облачной среде, базирующиеся на про\-граммно-кон\-фи\-гу\-ри\-ру\-емых сетях. Построены оценки 
скорости информационного обмена в~рамках облачных вычислений. Обсуждается система 
процедур, позволяющая реконфигурировать  коммутационные таб\-ли\-цы (КТ) в~соответствии с~текущими 
изменениями топологии сетевых взаимодействий в~облачной среде.} 

\KW{облачные вычисления; про\-граммно-кон\-фи\-гу\-ри\-ру\-емые сети; скорость 
управления и~безопасность информационных потоков в~облаке; управление 
масштабируемостью и~динамическими изменениями в~топологии сети} 

\DOI{10.14357/19922264150410}

\vskip 14pt plus 9pt minus 6pt

\thispagestyle{headings}

\begin{multicols}{2}

\label{st\stat}
     
\section{Введение}

  Сегодня cloud computing~--- один из наиболее востребованных и~интенсивно 
развиваемых комплексов информационных технологий. В~русскоязычной 
специальной литературе наиболее распространенным термином для понятия 
\textit{cloud computing} стали \textit{облачные вычисления}, хотя в~этой 
формулировке в~явном виде, к сожалению, не прослеживаются некоторые 
другие важные смысловые оттенки исходного понятия. Например, восприятие 
существительного \textit{cloud} как \textit{покров} (\textit{cover}, 
\textit{covering}) и~как \textit{множество} (\textit{multiplicity}, \textit{plurality}, 
\textit{variety}) позволяет указать на одну из принципиально важных 
функциональных особенностей \textit{облачных вычислений~--- 
масштабируемость} (так называемую \textit{вычислительную эластичность}) 
таких решений. 
  
  Ответом на потребность в~виртуализации трех базовых ресурсов~--- 
вычислений, хранения данных и~сетевых взаимодействий~--- стал целый спектр 
но-\linebreak вых подходов, например Software Defined Networks (SDN~--- 
про\-граммно-кон\-фи\-гу\-ри\-ру\-емые сети)~[1] и~Network Function 
Virtualization (NFV~--- виртуализация сетевых функций)~[2], во\-пло\-тив\-ши\-еся 
в~эффективные прикладные ин\-фор\-ма\-ци\-он\-но-тех\-но\-ло\-ги\-че\-ские решения 
нового поколения. Они вошли в~проекты Internet2~[3] и~раз\-ви\-ва\-ющу\-юся 
в~настоящий момент уже как международная кооперация американскую 
национальную программу GENI (Global Environment for Network 
Innovations)~[4]. 
  
  Критически важными характеристиками облачных решений оказались 
\textit{скорость} и~\textit{защищенность} информационного обмена. При этом 
названные два показателя оказываются <<противоположно направленными>>. 
Высокий уровень защищенности, как правило, существенным образом снижает 
скорость облачных взаимодействий, так как  понижает скорость обмена 
данными. Наоборот, высокие скорости информационного обмена предъявляют 
весьма жесткие требования к архитектуре и~функциональности 
соответствующих систем информационной безопасности. 
  
  Традиционно решения по безопасности в~больших информационных 
системах основаны на разграничении доступа. Например, политика 
безопасности RBAC (Role Based Access Control~--- контроль доступа, 
основанный на ролях)~[5] может быть построена с~по\-мощью 
изолированных доменов, отвечающих дереву привилегий в~организации.
  
  Вторым важнейшим механизмом обеспечения информационной 
безопасности может стать контроль и~управление информационными потоками 
(см., например,~[6]).
  
  Отметим, что с~помощью SDN можно контролировать информационные 
потоки и~управлять ими. С~помощью контроля информационных потоков 
можно создавать изолированные домены. Таким образом, SDN может 
обеспечить важнейшие функции защиты от несанкционированного доступа.
  
  Еще одной важной функцией в~системе безопасности информационной 
системы является мониторинг событий безопасности~[6]. Одним из самых 
эффективных механизмов решения этой проблемы служит DPI (Deep Packet 
Inspection~--- глубокий анализ содержимого пакетов). Отметим, что DPI на 
уровне контроллера SDN решает также задачу защиты системы мониторинга.
  
  Приведем некоторые технологические решения, позволяющие SDN решать 
перечисленные выше задачи обеспечения информационной без\-опас\-ности 
облачных вычислений. За счет специ\-ального  
(проб\-лем\-но-ори\-ен\-ти\-ро\-ван\-но\-го) разделения уровня исполнения 
трафика данных и~уровня управления этим трафиком на две области можно 
сформировать дополнительные по сравнению с~использованием классических 
сетевых технологий возможности управления сетевыми взаимодействиями. Так, 
в~част\-ности, удается получить существенные преимущества за счет переноса 
ряда важных функций на SDN-конт\-рол\-лер, который (как программная 
система, обеспечивающая виртуализацию необхо\-димых функций по 
управлению компьютерной \mbox{сетью}) может располагаться на отдельной (в~том 
числе распределенной) вычислительной установке требуемой 
производительности. Как следствие, открываются возможности гибкого 
управления выполняемыми на контроллере вы\-чис\-ле\-ни\-ями и~без\-опас\-ностью.
  
  Интерес, в~первую очередь, представляют возможности вынести ряд важных, 
однако второстепенных по отношению к~собственно передаче пакетов 
и~потоков данных вычислительных процедур\linebreak на внешний по отношению 
к~механизмам организации трафика контроллер. Это позволяет реализовать ряд 
ресурсоемких процедур управления трафиком вне механизма организации 
трафика (вне\linebreak SDN-ком\-му\-та\-то\-ра), используя внешний по отношению 
к~нему вычислитель необходимой про\-из\-во\-ди\-тель\-ности. Назовем его  
SDN-конт\-рол\-ле\-ром. Такой прием позволяет разделить решение ряда\linebreak задач 
организации и~управ\-ле\-ния облачным трафиком, <<чувствительных>> 
к~объемам требуемых вы\-чис\-ле\-ний, на две части:
  \begin{enumerate}[(1)]
\item однократное исполнение требующей существенных объемов 
вычислений специальной части целостного комплекса процедур анализа 
и~управления трафиком;
\item многократно выполняемый набор действий, которые достаточно 
малозатратны в~части объемов реализующих их вычислений.
\end{enumerate}

  Первая из них может выполняться на внешнем SDN-конт\-рол\-ле\-ре, 
а~вторая~--- в~том числе и~на SDN-ком\-му\-та\-торе.
  
  В качестве иллюстративного примера такого разделения можно обратиться 
к~задаче поиска заголовка входного пакета в~ТК (по\-дроб\-но 
эта задача будет рассмотрена в~следующих разделах). 
Исполнению здесь подлежит достаточно простая по постановке формальная 
процедура~--- проверить, содержится ли входной булев вектор фиксированной 
длины~$k$ в~содержащей~$n$~строк заданной КТ 
булевых векторов той же длины~$k$. Однако при проверке таких вхождений 
приходится учитывать следующие существенные соображения. В~реальных 
приложениях КТ оказываются очень большими. На текущий момент 
известны предлагаемые рынку промышленные коммутаторы, оперирующие 
с~КТ суммар\-ным размером в~150--200~тыс.\ строк. При этом необходима 
высокая скорость решения задачи.
{\looseness=-1

}
  
  При решении этой формальной задачи \mbox{можно} использовать ряд достаточно 
хорошо изученных вариан\-тов ее разделения на две подзадачи, одна из которых 
будет решаться однократно, порождая приведение КТ к некоторому 
специальному виду и~требуя при этом существенно б$\acute{\mbox{о}}$льших (по сравнению 
с~решением второй подзадачи) объемов вычисле\-ний. Вторая подзадача 
обеспечит многократную быструю проверку, использующую этот специальный 
вид КТ, собственно вложимости входного заголовка~$h$ в~текущее 
состояние~КТ.
  
  Хорошо известными (см., в~частности, классическую работу~[7]) примерами 
подобных разделений являются:
\begin{itemize}
\item лексикографическое упорядочение исходной КТ, когда проверка 
вложимости~$h$ в~КТ (как и~внесение динамических изменений в~текущее 
состояние КТ~--- о~чем речь пойдет чуть позже) оказывается достаточно 
простой вычислительной процедурой; 
\item приведение КТ к виду бинарного дерева (диаграммы), когда 
порождение такого дерева оказывается в~общем случае экспоненциально 
сложной вычислительной процедурой, а собственно проверка вложимости 
входного заголовка~$h$ в~уже построенное дерево представляет собою 
процедуру, которая требует всего лишь линейно быстро растущего с~ростом 
длины вектора~$h$ объема вычислений.
\end{itemize}

  Однако при переходе от анализа заголовков пересылаемых пакетов данных 
также и~к~анализу их\linebreak  содержания (DPI, IDS/IPS (Intrusion Detection/\linebreak Protection 
Systems~--- системы обна\-ру\-же\-ния/пред\-от\-вра\-ще\-ния вторжений)) 
рас\-смат\-ри\-ва\-емая ситуация существенно усложняется. Множество признаков, 
проверяемых на встречаемость в~<<теле>> переда\-ва\-емых пакетов, многократно 
возрастает. Например, актуальные базы сигнатур вирусов сегодня оперируют 
уже сотнями тысяч объектов. На <<техническом>> уровне эта проблема 
<<размерности>> объекта анализа отражается в~том, что
\begin{itemize}
\item формализация ее в~уже представленной выше <<технике>> операций 
с~булевыми векторами приводит к необходимости работать с~объектами 
очень большой длины, содержащими при этом существенно больше нулей, 
чем единиц;
\item в~подобных булевых векторах вся полезная информация уже 
представлена элементами неупорядоченных множеств.
\end{itemize}

  Ранее~[8] уже рассматривались некоторые варианты статического механизма 
контроля сетевого трафика с~использованием возможностей SDN-под\-хо\-да. 
В~данной работе предлагается целенаправлен\-ное расширение представленного 
подхода на случай управления динамическими изменениями актуальной 
конфигурации SDN-се\-ти, учи\-ты\-ва\-ющий помимо обработки заголовков 
пакетов данных также и~потребности в~организации быстрого анализа их 
содержания. При этом, как и~прежде, основная роль в~предлагаемой 
процедурной конструкции отводится эффективной организации (на 
функционирующем по предлагаемым специальным правилам  
SDN-конт\-рол\-ле\-ре) вспомогательных вычислений, которые и~обеспечивают 
поддержку принятия решений по реконфигурированию в~со\-от\-вет\-ст\-ву\-ющей 
сети обмена информацией.

\vspace*{-6pt}

\section{Методы проверки вложимости векторов в~коммутационной таблице}

\vspace*{-2pt}
  
  В индустриальных приложениях <<ширина>> КТ (число ее столбцов)~$k$, 
как правило, существенно меньше числа ее строк~$n$. Однако следует иметь 
в~виду, что $k\hm\leq \mathrm{log}_2n$. Заметим, что прямой поиск~--- 
перебор строк коммутационной таблицы~--- также может быть оптимизирован 
за счет использования лексикографических упорядочений строк исходной КТ 
или же, например, приведения исходной КТ к~виду бинарного дерева. 
Рассмотрим эти соображения более подробно.
  
  Пусть $L(\mathrm{КТ})$ есть лексикографически упорядоченная по строкам версия 
анализируемой КТ. Обозначим число операций для лексикографического 
упорядочения КТ через~$S_L^1$. Несложно убедиться, что процедура 
приведения КТ к~целевому виду требует при порождении $L(\mathrm{КТ})$ $S_L^1 
\hm\leq C_L^1 kn \mathrm{log}_2 n$ операций, где $C_L^1\hm= const$. Таким 
образом, имеет место следующая теорема. 
  
  \smallskip
  
  \noindent
  \textbf{Теорема~1.}\ \textit{Сложность приведения к виду $L(\mathrm{КТ})$ таб\-ли\-цы 
коммутаций, име\-ющей~$n$~строк и~$k$~столбцов, ограничена сверху величиной 
$C_L^1kn \mathrm{log}_2 n$.}
  
  \smallskip
  
  При проверке вложимости входного заголовка $h_i\hm= \{a^1_{i_1}, 
a^2_{i_2},\ldots , a^k_{i_k}\}$ в~$L(\mathrm{КТ})$ в~каждом из ее столбцов с~номером 
$j\hm+1$, $j\hm\in\{1, 2, \ldots, k-1\}$, придется искать соответствующую пару 
содержащих только~1 и~только~0~<<блоков>> (множеств строк KT), на 
которую <<распадается>> множество значений~$a^j_{i_j}$ в~предыду\-щем 
столбце номер~$j$. Чтобы сократить объем перебора значений в~каждом из 
столбцов~$K$ (по сравнению с~прямым просмотром каждого столбца целиком), 
можно воспользоваться алгоритмом последовательного разбиения множеств 
элементов столбца, например пополам или же по правилу золотого сечения. 
Однако число таких блоков с~рос\-том номера~$j$ будет расти как 
$\mathrm{log}_2 j$, т.\,е.\ не быстрее, чем $C_L^2\mathrm{log}_2 n$ для 
некоторой константы~$C_L^2$. 
  
  Пусть $S_L^2$~--- объем перебора при проверке вложимости заголовка~$h$ 
в~$L(\mathrm{КТ})$. Таким образом, имеет место следующая теорема.
  
  \smallskip
  
  \noindent
  \textbf{Теорема~2.} \textit{Объем перебора при проверке вложимости 
заголовка~$h$ в~$L(\mathrm{КТ})$  таблицы коммутаций можно оценить как} 
$S_L^2\hm\leq C_L^2 \mathrm{log}_2 n$.
  
  \smallskip
  
  Определим бинарное дерево, соответствующее КТ. Корень дерева находится 
на уровне (слое)~0, и~из каждой вершины в~следующий слой могут идти не 
более двух ребер, помеченных~0 и~1. Каждая вершина, кроме корневой, имеет 
метку~0 или~1, соответствующую метке входящего ребра. Таким образом, 
каждый путь из корня в~вершину слоя~$i$ взаимно однозначно соответствует 
двоичному вектору длины~$i$ (как по меткам ребер, так и~по меткам вершин). 
Число слоев в~бинарном дереве, пред\-став\-ля\-ющем КТ, равно~$k$. 
Максимальное число вершин в~слое~$k$ равно~$2^k$, а~число всех вершин 
в~полном бинарном дереве равно $2^{k+1}-1$. 
  
  Выделим пути в~полном бинарном дереве с~$k$~слоями, соответствующие 
векторам из КТ. Полученное дерево обозначим $D(\mathrm{КТ})$. Проверка наличия 
вектора в~КТ с~использованием построенного дерева $D(\mathrm{КТ})$ состоит 
в~прохождении пути длины~$k$, причем из каждой вершины переход 
осуществляется по ребру с~меткой, соответствующей очередному знаку 
в~проверяемом векторе. Если на очередном шаге необходимое ребро 
отсутствует, то это означает, что рассматриваемого вектора в~КТ нет. Отметим, 
что по построению дерева $D(\mathrm{КТ})$, если КТ не пуста, из любой достигнутой 
вершины есть хотя бы одно выходящее ребро. Проверка с~помощью дерева 
$D(\mathrm{КТ})$ присутствия векторов в~КТ не превосходит~$k$~шагов.
  
  Для построения $D(\mathrm{КТ})$ на некотором макете полного двоичного дерева 
необходимо $2^{k+1}\hm-1$~элементов памяти и~$nk$~операций выделения 
фрагментов $D(\mathrm{КТ})$ в~этом макете.
  
  \smallskip
  
  \noindent
  \textbf{Теорема~3.}\ \textit{Определение принадлежности вектора таблице 
коммутаций с~помощью $D(\mathrm{КТ})$ требует не более~$k$~операций прохождения по 
ребрам $D(\mathrm{КТ})$, но построение $D(\mathrm{КТ})$ требует~$nk$~операций выделения 
фрагментов $D(\mathrm{КТ})$ в~макете полного двоичного дерева с~$k$~слоями.}
  
  \smallskip
  
  Положим $k=\alpha \mathrm{log}_2n$, где $\alpha\geq1$. Тогда объем памяти 
для макета полного двоичного дерева равен $2n^\alpha\hm-1$. Сложность 
построения $D(\mathrm{КТ})$ равна $\alpha n \mathrm{log}_2n$, а~проверка 
принадлежности к КТ потребует не больше чем $\alpha\mathrm{log}_2 n$ 
операций. 
  
  Метод бинарного дерева можно усовершенствовать так, что скорость 
проверки вложимости существенно сократится при небольших ухудшениях 
остальных параметров. 
  
  \textit{Наименьшим запретом в~двоичном корневом дереве} $D(\mathrm{КТ})$ назовем 
путь (соответственно, вектор) из корня, заканчивающийся отсутствующим 
реб\-ром. Запрет~--- это путь (вектор), соответствующий в~$D(\mathrm{КТ})$ 
отсутствующему пути в~концевую вершину на уровне~$k$. Для каждого 
запрета в~полном двоичном дереве существует наименьший запрет. 
Вложимость вектора в~КТ можно проверять по дереву наименьших запретов 
$M(\mathrm{КТ})$. Если начало проверяемого вектора совпадает с~со\-от\-вет\-ст\-ву\-ющим 
наименьшим запретом в~дереве $M(KT)$, то проверяемый вектор не лежит 
в~КТ. 
  
  На простейшей модели оценим влияние коротких и~длинных наименьших 
запретов. Предположим, что $\overline{x}$~--- проверяемый вектор, а КТ 
формируется случайно с~помощью независимого равновероятно\-го выбора 
векторов длины~$k$. Обозначим через~$P_i$ вероятность того, что случайный 
вектор войдет в~противоречие с~$\overline{x}$ до координаты с~индексом~$i$. 
Тогда $1\hm- P_i$ означает вероятность совпадения случайного вектора 
с~вектором~$\overline{x}$ по всем~$i$~первым координатам, при этом $1\hm-
P_i \hm= 1/2^i$. Вероятность того, что различие векторов произойдет на 
координате~$i$, равна~$2^{-(i+1)}$. Чис\-ло векторов, для которых различие 
происходит в~координате с~индексом~$i$, распределено по биномиальному 
закону. Отсюда следует, что среднее чис\-ло векторов из~$n$~независимых 
случайных векторов, в~которых различие происходит на шаге $i\hm+1$, равно 
$n/2^{i+1}$. Это чис\-ло убывает с~ростом~$i$, что служит доводом в~пользу 
того, что основное количество наименьших запретов имеет маленькую длину. 
Эти рас\-суж\-де\-ния показывают целесообразность использования дерева 
наименьших запретов для определения принадлежности вектора~КТ. 
  
  Скорость определения принадлежности вектора КТ с~помощью бинарных 
деревьев можно повысить с~помощью распараллеливания поиска наименьших 
запретов. С~этой целью рассмотрим \mbox{пару}\linebreak полных корневых бинарных деревьев 
с~$k$~слоями, которые будем называть <<левое>> и~<<правое>>. Коммутационная таблица 
порождает в~левом полном дереве поддерево $D_L(\mathrm{КТ})$, как это было описано 
выше при по\-стро\-ении  $D(\mathrm{КТ})$. Аналогично по КТ\linebreak на основе правого 
полного дерева строится\linebreak дерево $D_R(\mathrm{КТ})$. Разница между деревьями состоит 
в том, что пути в~$D_R(\mathrm{КТ})$ соответствуют векторам из КТ, читаемым справа 
налево. Соответственно строятся два дерева наименьших запретов $M_L(\mathrm{КТ})$ 
и~$M_R(\mathrm{КТ})$. Если вектор~$\overline{x}$ не входит в~КТ, то в~$M_L(\mathrm{КТ})$ 
и~$M_R(KT)$ есть два пути, соответствующие вектору~$\overline{x}$ при его 
чтении слева направо в~$M_L(\mathrm{КТ})$ и~справа налево в~$M_R(\mathrm{КТ})$. 
Вектору~$\overline{x}$ соответствуют наименьшие запреты~$\overline{x}_L$ 
и~$\overline{x}_R$. Поиск наименьшего запрета для проверяемого 
вектора~$\overline{x}$ параллельно по $M_L(\mathrm{КТ})$ и~$M_R(\mathrm{КТ})$ может 
ускорить процесс отбраковки. 
  
  При наличии других признаков отбраковки (требования политики 
безопасности) можно также рассматривать параллельно несколько деревьев 
наименьших запретов, позволяющих ускорить процесс разрешения 
информационного потока.

\vspace*{-8pt} 
  
\section{Процедура реконфигурирования коммутационной таблицы}

\vspace*{-2pt}
  
  Пусть на момент времени~$T$ КТ имела конфигурацию $H(T)$. Будем 
считать, что через промежуток времени~$\Delta T_1$ в~таблице произошли ее 
первые структурные изменения:
  \begin{itemize}
\item удалены строки (заголовки), образующие множество $H^-(T\hm+\Delta 
T_1)$;\\[-14pt]
\item добавлены строки, образующие множество $H^+(T\hm+ \Delta T_1)$.
\end{itemize}
При этом с~формальной точки зрения представляется естественным считать, 
что хотя бы одно из множеств $H^+(T\hm+\Delta T_1)$ и~$H^-(T\hm+\Delta 
T_1)$ не является пустым. 
Таким образом,
$$
H(T+\Delta T_1) = (H(T)\backslash H^-(T+\Delta T_1) )\cup H^+ (T+\Delta T_1)\,.
$$
  
  Исключение векторов или включение новых в~лексикографически 
упорядоченную КТ по сложности равносильно поиску векторов в~такой 
таблице. Это замечание определяет сложность вставки или удаления новых 
векторов из лексикографически упорядоченной~КТ. 
  
  Удаление пути из дерева равносильно удалению ближайшего к~корню ребра 
и~всех его продолжений. Добавление пути в~дерево равносильно добавлению 
новых ребер. Отсюда сложность оного изменения не превосходит~$k$.

\vspace*{-9pt}
  
\section{Контроль и~управление информационными потоками 
рассмотренными методами}

\vspace*{-2pt}
  
  Несложно убедиться, что мониторинг контента и~управление потоками 
данных в~облачной среде могут быть основаны не только на анализе структуры 
заголовков (булевых векторов фиксированной длины). В~частности, можно 
учитывать также те или иные характеристики содержания переда\-ва\-емых 
пакетов. В~КТ обсуждалась обработка заголовков, где необходимо было 
оперировать векторами фиксированной и~при этом достаточно малой по 
сравнению с~размерами пакета данных длины. В~случае\linebreak учета контентных 
характеристик пакетов возни\-кает необходимость обрабатывать подмножества,\linebreak 
вообще говоря, достаточно большого множества признаков, например 
множества контентных характеристик пакетов, выделяемых DPI- или же 
IDS/IPS-сред\-ст\-ва\-ми. Таким образом, появляется возможность обрабатывать 
в едином процессе множество параметров информационной безопасности.
{ %\looseness=-1

}
  
  Все эти признаки можно собрать в~некоторый шаблон описания, где наличие 
конкретного признака у~конкретного пакета данных будет кодироваться 
единицей, а~его отсутствие~--- нулем. Однако размер такого шаблона (число 
позиций в~соответствующем этому шаблону булевом векторе) может оказаться 
большим. 
  
  Представленные возможности могут оказаться весьма полезными при 
оптимизации функци\-о\-нирования подсистемы обеспечения информа\-ционной 
безопасности в~облачном ландшафте, отвеча\-ющей, например, за мониторинг 
исполнения соответствующей политики безопасности компьютерной сети, 
в~том числе за контроль:\\[-14pt]
  \begin{itemize}
\item[(a)] <<непересекаемости>> заданных подсетей обла-\linebreak ка (т.\,е.\ 
физического отделения используемой каж\-дой из двух соответствующих 
подсетей сис\-тем\-но-тех\-ни\-че\-ских ресурсов, если это \textit{дано полити\-кой 
безопасности}, при имеющих мес\-то динамических изменениях топологии 
виртуальной инфраструктуры облака);
\item[(б)] доступности (в~том числе соответствия \textit{заданной политике 
безопасности}) конкретных сетевых взаимодействий типа 
$\langle${узел}~$X$\,--\,{узел}~$Y$$\rangle$ или 
$\langle${мно\-же\-ст\-во узлов}~$Q$\,--\,{множество 
узлов}~$R$$\rangle$;
\item[(в)] доступности пакетов заданного типа лишь в~наперед заданном 
перечне узлов.
\end{itemize}

  При динамических изменениях текущей топологии виртуальной архитектуры 
облачной сети недостаточно только отслеживать (и~пресекать) возможные 
нарушения заранее заданных требо\-ваний политики безопасности. К~примеру, 
функция~(в) может поддерживаться средствами DPI- или же  
IDS/IPS-ре\-ше\-ний, которые удобно запускать именно на SDN-конт\-рол\-ле\-ре. Не 
менее важно выполнять быстрое, т.\,е.\ не порождающее критичных задержек 
трафика в~сети, реконфигурирование КТ к~виду, который обеспечивает решение 
текущих задач в~данном облачном ландшафте и~при этом удовлетворяет 
требованиям заранее сформулированной политики безопасности.

\vspace*{-6pt}
  
\section{Заключение}

\vspace*{-2pt}
  
  Разумеется, для эффективного использования в~тех или иных реальных 
приложениях предложенная процедурная схема управления 
реконфигурированием КТ в~каждом конкретном случае требует аккуратной 
настройки, которая учитывала бы доступные в~имеющемся  
сис\-тем\-но-тех\-ни\-че\-ском ландшафте соответствующей облачной среды 
функциональные возможности. 
  
  Для определения значений таких <<калибровочных>> параметров требуется 
проведение специальных тестовых экспериментов на соответствующем 
компьютерном оборудовании. Развитие пред\-став\-лен\-ных здесь работ именно 
в~этом направлении экспериментальных исследований будет сле\-ду\-ющей целью 
в~ближайшем будущем.
  
  Еще одним не менее интересным направлением перспективного развития 
обсуждаемого здесь подхода является изучение возможностей 
распараллеливания вычислений при реализации предложенных процедур 
реконфигурирования таблиц коммутации.
  
  Наконец, определенные перспективы имеет и~углубленное изучение тех 
дополнительных возможностей, которые предлагаемый подход к~управ\-ле\-нию 
вычислительными ресурсами облачной среды предоставляет для 
аутентификации <<статуса>> и~<<полномочий>> передаваемых пакетов, 
а~также контроля корректности их адресации как важных функций 
поддержания устойчивости и~без\-опас\-ности информационного обмена 
в~облачной среде вычислений при динамическом реконфигурировании ее 
ресурсов.

{\small\frenchspacing
 {%\baselineskip=10.8pt
 \addcontentsline{toc}{section}{References}
 \begin{thebibliography}{9}
  \bibitem{1-gr}
      Сетевые технологии SDN~--- Software Defined Networking. 
{\sf http://habrahabr.ru/company/muk/blog/251959}.
\bibitem{2-gr}
    ETSI. Network functions virtualization. 
    {\sf http://www. etsi.org/technologies-clusters/technologies/nfv}.
\bibitem{3-gr}
   The Internet2 community: Enabling the future. {\sf http:// www.internet2.edu}.
\bibitem{4-gr}
 GENI: Exploring networks of the future. {\sf https:// www.geni.net}.

\bibitem{6-gr}
\Au{Ferraiolo D.\,F., Kuhn~D.\,R.} Role-based access controls~// 15th National 
Computer Security Conference. Baltimore, 1992. P.~554--563.
{\sf http://csrc.nist.gov/rbac/rbacSTD-\linebreak ACM.pdf}.
\bibitem{7-gr}
ГОСТ Р ИСО/МЭК 15408-1-2008. Информационная технология. Методы 
и средства обеспечения без\-опас\-ности. Критерии оценки безопасности 
информационных технологий. Часть~1. Введение и~\mbox{общая} модель. 
{\sf http://docs.cntd.ru/document/gost-r-iso-mek-\linebreak 15408-1-2008}.
\bibitem{8-gr}
\Au{Кнут Д.} Искусство программирования. Т.~3. Сортировка и~поиск~/
Пер. с~англ.~--- М: 
Вильямс, 2007. 832~с. (\Au{Knuth~D.\,E.} 1998. 
{The art of computer programming. Vol.~3: 
  Sorting and searching}.~--- 2nd ed.~--- Reading, MA, USA: Addison-Wesley, 1998. 794~p.)
\bibitem{9-gr}
\Au{Забежайло М.\,И.} К~задаче анализа вложимости подслов в~заголовки 
пакетов данных~// Системы и~средства информатики, 2013. Т.~23. №\,1. С.~58--68.
 \end{thebibliography}

 }
 }

\end{multicols}

\vspace*{-6pt}

\hfill{\small\textit{Поступила в~редакцию 23.10.15}}

\vspace*{8pt}

%\newpage

%\vspace*{-24pt}

\hrule

\vspace*{2pt}

\hrule

\vspace*{8pt}

\def\tit{INFORMATION FLOW MONITORING AND CONTROL 
IN~THE~CLOUD COMPUTING ENVIRONMENT}

\def\titkol{Information flow monitoring and control 
in the cloud computing environment}

\def\aut{A.\,A.~Grusho$^1$, M.\,I.~Zabezhailo$^2$, and A.\,A.~Zatsarinny$^3$}

\def\autkol{A.\,A.~Grusho, M.\,I.~Zabezhailo, and A.\,A.~Zatsarinny}

\titel{\tit}{\aut}{\autkol}{\titkol}

\vspace*{-9pt}


\noindent
$^1$Institute of Informatics Problems, 
Federal Research Center ``Computer Sciences and Control'' of the Russian\linebreak
$\hphantom{^1}$Academy 
of Sciences, 44-2 Vavilov 
    Str., Moscow 119333, Russian Federation

\noindent
$^2$All-Russian Institute for Scientific 
and Technical Information of Russian Academy of Sciences, 20~Usievicha Str.,\linebreak
$\hphantom{^1}$Moscow 125190, Russian Federation

\noindent
$^3$Federal Research Center ``Computer Sciences and Control'' of the Russian Academy 
of Sciences, 44-2 Vavilov\linebreak
$\hphantom{^1}$Str., Moscow 119333, Russian Federation


\def\leftfootline{\small{\textbf{\thepage}
\hfill INFORMATIKA I EE PRIMENENIYA~--- INFORMATICS AND
APPLICATIONS\ \ \ 2015\ \ \ volume~9\ \ \ issue\ 4}
}%
 \def\rightfootline{\small{INFORMATIKA I EE PRIMENENIYA~---
INFORMATICS AND APPLICATIONS\ \ \ 2015\ \ \ volume~9\ \ \ issue\ 4
\hfill \textbf{\thepage}}}

\vspace*{3pt}  
  


\Abste{Control of information flows is one of the most important mechanisms of 
security implementation in cloud computing. Some SDN (Software Defined Networks) 
based technologies for packet 
forwarding in the cloud computing environment are presented and analyzed. 
The special problem-oriented procedure for monitoring and quick control of forwarding 
table reconfiguration is discussed. This procedure is responsible for control of 
requirements of network security policy in the dynamically changing cloud computing 
environment. The presented control procedures are based on Lexical Ordering (LO) of 
forwarding table (FT) and FT transformation in Binary Tree (BT). Some estimations 
of computational complexity of the presented header analysis by BT and LO are discussed. 
Possibilities to extend the presented algorithmic approach from header analysis to Deep 
Packet Inspection (DPI) and some related problems are announced and evaluated.}

\KWE{cloud computing; software defined networks; scalability of network resources; 
dynamic changes in network topology; 
information security in cloud computing environment; header analysis}

\DOI{10.14357/19922264150410}

\vspace*{-12pt}

\Ack

\vspace*{-2pt}

    \noindent
The work was supported by the Russian Foundation for Basic Research (project 15-29-07981).



%\vspace*{3pt}

  \begin{multicols}{2}

\renewcommand{\bibname}{\protect\rmfamily References}
%\renewcommand{\bibname}{\large\protect\rm References}

{\small\frenchspacing
 {%\baselineskip=10.8pt
 \addcontentsline{toc}{section}{References}
 \begin{thebibliography}{9}
\bibitem{1-gr-1}
Software Defined Networking. Available at: 
{\sf http://\linebreak habrahabr.ru/company/muk/blog/251959/}
(accessed December~11, 2015).
\bibitem{2-gr-1}
 ETSI. Network functions virtualization. Available at: 
{\sf http://www.etsi.org/technologies-clusters/\linebreak technologies/nfv}
(accessed December~11, 2015).
\bibitem{3-gr-1}
The Internet2 community: Enabling the future.  Available at: 
{\sf http://www.internet2.edu/}
(accessed December~11, 2015).
\bibitem{4-gr-1}
 GENI: Exploring networks of the future. Available at: {\sf https://www.geni.net}
 (accessed December~11, 2015).

\bibitem{6-gr-1}
\Aue{Ferraiolo, D.\,F., and D.\,R.~Kuhn}. 1992. 
Role-based access controls. \textit{15th National 
Computer Security Conference}. Baltimore. 554--563.
Available at: 
{\sf http://csrc. nist.gov/rbac/rbacSTD-ACM.pdf} (accessed December~11, 2015).
\bibitem{7-gr-1}
GOST R ISO/IEC 15408-1-2008. Informatsionnaya tekhnologiya. Metody 
i~sredstva obespetcheniya infor\-ma\-tsi\-on\-noy bez\-opas\-nosti. Kriterii otsenki bezopasnosti 
informatsionnykh tekhnologiy. Chast'~1. Vvedenie i~ob\-shchaya model' [Information 
technology. Security techniques. Evaluation criteria for IT security. Part~1. 
Introduction and general model]. Available\linebreak\vspace*{-12pt}

\columnbreak

\noindent
 at: 
{\sf http://docs.cntd.ru/document/gost-r-iso-mek-15408-1-2008}
(accessed December~11, 2015).
  \bibitem{8-gr-1}
  \Aue{Knuth,~D.\,E.} 1998. \textit{The art of computer programming. Vol.~3: 
  Sorting and 
searching}. 2nd ed. Reading, MA: Addison-Wesley. 794~p.
\bibitem{9-gr-1}
\Aue{Zabezhailo, M.\,I.} 2013. K~zadache analiza vlozhimosti podslov 
v~zagolovki paketov dannykh [On the problem of subsequences inclusion into the data 
packages headers]. \textit{Sistemy i~Sredstva Informatiki}~---
\textit{Systems and Means of Informatics} 23(1):58--68.
 \end{thebibliography}

 }
 }

\end{multicols}

\vspace*{-3pt}

\hfill{\small\textit{Received October 23, 2015}}
    
    
\Contr

\noindent
\textbf{Grusho Alexander A.} (b.\ 1946)~--- Doctor of Science in physics and 
mathematics, professor, leading scientist, Institute of Informatics Problems, 
Federal Research Center ``Computer Sciences and Control'' of the Russian Academy 
of Sciences, 44-2 Vavilov 
    Str., Moscow 119333, Russian Federation;
grusho@yandex.ru 

\vspace*{3pt}

\noindent
\textbf{Zabezhailo Michael I.} (b.\ 1956)~--- Candidate of Science (PhD) in 
physics and mathematics, senior scientist, All-Russian Institute for Scientific 
and Technical Information of the Russian Academy of Sciences, 20~Usievicha Str., 
Moscow 125190, Russian Federation; m.zabezhailo@yandex.ru

\vspace*{3pt}

\noindent
\textbf{Zatsarinny Alexander A.} (b.\ 1951)~--- Doctor of Science in 
technology, professor, Deputy Director, Federal Research Center ``Computer 
Sciences and Control'' of the Russian Academy of Sciences, 44-2 Vavilov 
    Str., Moscow 119333, Russian Federation;
azatsarinny@ipiran.ru
    
\label{end\stat}


\renewcommand{\bibname}{\protect\rm Литература}

 
 
 
 