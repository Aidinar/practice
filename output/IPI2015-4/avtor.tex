\def\stat{authorsrus}
{%\hrule\par
%\vskip 7pt % 7pt
\raggedleft\Large \bf%\baselineskip=3.2ex
О\,Б\ \ А\,В\,Т\,О\,Р\,А\,Х \vskip 17pt
    \hrule
    \par
\vskip 21pt plus 8pt minus 4pt }


\def\tit{\ }

\def\aut{\ }

\def\auf{\ }

\def\leftkol{ОБ АВТОРАХ}

\def\rightkol{\ }

\titele{\tit}{\aut}{\auf}{\leftkol}{\rightkol}

            \label{st\stat}



\vspace*{-24pt}

\begin{multicols}{2}

\noindent
\textbf{Беляев Константин Павлович} (р.\ 1955)~---
доктор фи\-зи\-ко-ма\-те\-ма\-ти\-че\-ских наук, ведущий научный сотрудник 
Института океанологии им.\ П.\,П.~Ширшова; приглашенный профессор, 
Федеральный универси\-тет штата Баиа, Сальвадор, Бразилия

\noindent
\textbf{Боровский Павел} (р.\ 1986)~---
магистрант кафедры программной инженерии Академического инженерного колледжа 
Шамуна, Беер-Шева, Израиль 

\noindent
\textbf{Вихрова Ольга Геннадиевна} (р.\ 1990)~---
аспирантка Российского университета дружбы народов


\noindent
\textbf{Горшенин Андрей Константинович} (р.\ 1986)~---
кандидат фи\-зи\-ко-ма\-те\-ма\-ти\-че\-ских наук, старший научный 
сотрудник Института проб\-лем информатики Федерального исследовательского центра 
<<Информатика и~управ\-ле\-ние√ Российской академии наук; доцент Московского государственного университета информационных технологий,
радиотехники и~электроники


\noindent
\textbf{Грушо Александр Александрович} (р.\ 1946)~--- 
доктор фи\-зи\-ко-ма\-те\-ма\-ти\-че\-ских наук, профессор, 
ведущий научный сотрудник Института проб\-лем информатики 
Федерального исследовательского\linebreak центра <<Информатика и~управ\-ление>> 
Российской академии наук

\noindent
\textbf{Грушо Николай Александрович} (р.\ 1982)~--- кандидат 
фи\-зи\-ко-ма\-те\-ма\-ти\-че\-ских наук наук, старший научный сотрудник 
Института проб\-лем информатики 
Федерального исследовательского центра <<Информатика и~управ\-ление>> 
Российской академии наук

\noindent
\textbf{Гулев Сергей Константинович} (р.\ 1958)~---
член-кор\-рес\-пон\-дент Российской академии наук, доктор 
фи\-зи\-ко-ма\-те\-ма\-ти\-че\-ских наук, заведующий лабораторией Института 
океанологии им.\ П.\,П.~Ширшова; профессор кафедры
океанологии географического факультета
Московского государственного университета имени М.\,В.~Ломоносова; 
профессор Кильского университета, Германия 

\noindent
\textbf{Забежайло Михаил Иванович} (р.\ 1956)~---
кандидат фи\-зи\-ко-ма\-те\-ма\-ти\-че\-ских наук, доцент, старший научный 
сотрудник Всероссийского института научной и~технической информации Российской 
академии наук (ВИНИТИ РАН) 

\noindent
\textbf{Зацаринный Александр Алексеевич} (р.\ 1951)~---
доктор технических наук, профессор, заместитель директора 
Федерального исследовательского цент-\linebreak\vspace*{-12pt}

\columnbreak

\noindent
ра <<Информатика и~управление>> 
Российской академии наук 



\noindent
\textbf{Змеев Дмитрий Николаевич} (р.\ 1978)~---
 научный сотрудник Институт проб\-лем проектирования 
 в~мик\-ро\-электронике Российской академии наук
 
 \noindent
\textbf{Кириков Игорь Александрович} (р.\ 1955)~--- 
кандидат технических наук, директор Калининградского филиала 
Федерального исследовательского центра <<Информатика и~управ\-ле\-ние>> 
Российской академии наук

\noindent
\textbf{Климов Аркадий Валентинович} (р.\ 1953)~---
 старший научный сотрудник Институт проб\-лем проектирования в микроэлектронике Российской академии наук

\noindent
\textbf{Колесников Александр Васильевич} (р.\ 1948)~---
доктор технических наук, профессор кафедры телекоммуникаций 
Российского государственного университета имени Иммануила Канта, 
старший научный\linebreak сотрудник Калининградского филиала Федерального 
исследовательского центра <<Информатика и~управ\-ле\-ние>> Российской академии наук

\noindent
\textbf{Коновалов Михаил Григорьевич} (р.\ 1950)~--- 
доктор технических наук, заведующий сектором Института 
проб\-лем информатики Федерального 
исследовательского центра <<Информатика и~управ\-ле\-ние>> Российской академии наук 

\noindent
\textbf{Копицкая Марина} (р.\ 1966)~---
PhD по информатике,  старший лектор кафедры программной инженерии 
Академического инженерного колледжа Шамуна, Беер-Шева, Израиль 

\noindent
\textbf{Королев Виктор Юрьевич} (р.\ 1954)~---
доктор фи\-зи\-ко-ма\-те\-ма\-ти\-че\-ских наук, профессор,
заведующий кафедрой 
математической статистики факультета вычислительной математики и~кибернетики Московского
государственного университета 
имени М.\,В.~Ломоносова;
ведущий научный сотрудник Института проб\-лем информатики Федерального 
исследовательского центра <<Информатика и~управ\-ле\-ние>> Российской академии наук

\noindent
\textbf{Корчагин Александр Юрьевич} (р.\ 1989)~---
младший научный сотрудник факультета вычислительной математики и~кибернетики 
Московского государствен\-но\-го университета имени М.\,В.~Ломоносова



\noindent
\textbf{Левченко Николай Николаевич} (р.\ 1978)~--- 
кандидат технических наук, заведующий отделом Институт проб\-лем проектирования 
в микроэлектронике Российской академии наук

\noindent
\textbf{Листопад Сергей Викторович} (р.\ 1984)~---
кандидат технических наук, научный сотрудник Калининградского филиала Федерального 
исследовательского центра <<Информатика и~управ\-ле\-ние>> Российской академии наук

\noindent
\textbf{Молотковский Роман} (р.\ 1989)~---
студент кафедры программной инженерии Академического инженерного колледжа Шамуна, 
Беер-Шева, Израиль

\noindent
\textbf{Окунев Анатолий Семенович} (р.\ 1951)~---
 кандидат технических наук, ведущий научный сотрудник Институт проб\-лем 
 проектирования в микроэлектронике Российской академии наук
 
  \noindent
\textbf{Разумчик Ростислав Валерьевич} (р.\ 1984)~---
кандидат фи\-зи\-ко-ма\-те\-ма\-ти\-че\-ских наук,
старший научный сотрудник Института проб\-лем 
информатики Федерального исследовательского центра <<Информатика и~управ\-ление>> 
Российской академии наук, доцент
Российского университета дружбы народов


 
 \noindent
\textbf{Румовская София Борисовна} (р.\ 1985)~---
программист 1-й категории Калининградского филиала Федерального исследовательского 
центра <<Информатика и~управ\-ле\-ние>> Российской академии наук

\noindent
\textbf{Самуйлов Константин Евгеньевич} (р.\ 1955)~---
доктор технических наук, профессор, заведующий ка\-фед\-рой Российского 
университета дружбы народов 

\noindent
  \textbf{Синицын Игорь Николаевич} (р.\ 1940)~--- доктор технических наук,
  профессор, заслуженный деятель науки РФ, заведующий отделом Института
  проб\-лем информатики Федерального исследовательского центра
  <<Информатика и~управ\-ле\-ние>> Российской академии наук
  
  \noindent
\textbf{Соколов Игорь Анатольевич} (р.\ 1954)~---
академик Российской академии наук, доктор технических наук, директор Федерального 
исследовательского центра <<Информатика и~управ\-ле\-ние>> Российской академии наук

\columnbreak 
  
  \noindent
\textbf{Сопин Эдуард Сергеевич} (р.\ 1987)~--- кандидат 
фи\-зи\-ко-ма\-те\-ма\-ти\-че\-ских наук, доцент Российского университета дружбы народов
 
 \noindent
\textbf{Стемпковский Александр Леонидович} (р.\ 1950)~--- 
доктор технических наук, профессор, академик Российской академии наук, директор Института проб\-лем 
проектирования в микроэлектронике Российской академии наук

\noindent
\textbf{Тимонина Елена Евгеньевна} (р.\ 1952)~--- 
доктор технических наук, профессор, ведущий научный сотруд\-ник 
Института проб\-лем информатики 
Федерального исследовательского центра <<Информатика и~управ\-ление>> 
Российской академии наук

\noindent
\textbf{Ушаков Владимир Георгиевич} (р.\ 1952)~--- 
доктор фи\-зи\-ко-ма\-те\-ма\-ти\-че\-ских наук, 
профессор кафедры математической статистики факультета вычислительной математики 
и~кибернетики Московского государственного университета имени М.\,В.~Ломоносова; 
старший научный сотрудник Института проб\-лем информатики Федерального исследовательского 
центра <<Информатика и~управ\-ле\-ние>> Российской академии наук

 
\noindent
\textbf{Ушаков Николай Георгиевич} (р.\ 1954)~---
доктор фи\-зи\-ко-ма\-те\-ма\-ти\-че\-ских наук, ведущий научный сотруд\-ник 
Института проблем технологии микроэлектроники и особо чистых материалов 
Российской академии наук (Черноголовка); профессор Норвежского 
на\-уч\-но-тех\-но\-ло\-ги\-че\-ско\-го университета (г.\ Тронхейм)

\noindent
\textbf{Френкель Сергей Лазаревич} (р.\ 1951)~---
кандидат технических наук, старший научный сотрудник\linebreak 
Института проб\-лем информатики 
Федерального исследовательского центра <<Информатика и~управ\-ле\-ние>> 
Российской академии наук; 
доцент Московского государственного университета информационных технологий,
радиотехники и~электроники

\noindent
\textbf{Шоргин Сергей Яковлевич} (р.\ 1952)~---
доктор фи\-зи\-ко-ма\-те\-ма\-ти\-че\-ских наук, профессор, заместитель директора 
Федерального исследовательского цент\-ра <<Информатика и~управ\-ле\-ние>> 
Российской академии наук


%%%%%%%%%%%%%%%%%%%%%%%%%%%%%%%%%%%%

%\def\rightkol{ОБ АВТОРАХ}
%\def\leftkol{ОБ АВТОРАХ}

 \label{end\stat}





%\def\leftfootline{\small{\textbf{\thepage}
%\hfill ИНФОРМАТИКА И ЕЁ ПРИМЕНЕНИЯ\ \ \ том~9\ \ \ выпуск~4\ \ \ 2015}
%}%
% \def\rightfootline{\small{ИНФОРМАТИКА И ЕЁ ПРИМЕНЕНИЯ\ \ \ том~9\ \ \ выпуск~4\ \ \ 2015
%\hfill \textbf{\thepage}}}


%\thispagestyle{myheadings}



\end{multicols}

\newpage