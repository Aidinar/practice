\def\stat{kirikov}

\def\tit{МЕЛКОЗЕРНИСТЫЕ ГИБРИДНЫЕ ИНТЕЛЛЕКТУАЛЬНЫЕ 
СИСТЕМЫ. ЧАСТЬ~1:~ЛИНГВИСТИЧЕСКИЙ ПОДХОД}

\def\titkol{Мелкозернистые гибридные интеллектуальные 
системы. Часть~1: Лингвистический подход}

\def\aut{И.\,А.~Кириков$^1$, А.\,В.~Колесников$^2$, С.\,В.~Листопад$^3$, 
С.\,Б.~Румовская$^4$}

\def\autkol{И.\,А.~Кириков, А.\,В.~Колесников, С.\,В.~Листопад, 
С.\,Б.~Румовская}

\titel{\tit}{\aut}{\autkol}{\titkol}

%{\renewcommand{\thefootnote}{\fnsymbol{footnote}} \footnotetext[1]
%{Работа выполнена при частичной финансовой поддержке РФФИ (проект 13-01-00215).}}


\renewcommand{\thefootnote}{\arabic{footnote}}
\footnotetext[1]{Калининградский филиал Федерального исследовательского
центра <<Информатика и~управ\-ление>> Российской академии наук, baltbipiran@mail.ru}
\footnotetext[2]{Балтийский федеральный университет им.\ И.~Канта; 
Калининградский филиал Федерального исследовательского
центра <<Информатика и~управ\-ление>> Российской академии наук, avkolesnikov@yandex.ru}
\footnotetext[3]{Калининградский филиал Федерального исследовательского
центра <<Информатика и~управ\-ление>> Российской академии наук, ser-list-post@yandex.ru}
\footnotetext[4]{Калининградский филиал Федерального исследовательского
центра <<Информатика и~управ\-ле\-ние>> Российской академии наук, sophiyabr@gmail.com}


 
  \Abst{Рассматривается проблематика междисциплинарных инструментариев и~свойства 
<<зернистости>> гибридов в~информатике. Результаты исследований показаны в~рамках 
лингвистического подхода, суть которого состоит в~трансформации вербализованной 
информации об объ\-ек\-тах-ори\-ги\-на\-лах (сложных задачах) и~объ\-ек\-тах-про\-то\-ти\-пах 
(методах моделирования), имеющейся в~полиязыках профессиональной деятельности, 
в~объ\-ек\-ты-ре\-зуль\-та\-ты (функциональные гибридные интеллектуальные системы (\mbox{ФГиИС})). 
Трансформация направляется эвристиками~--- схемами ролевых концептуальных моделей  (КМ)
в~неформальной аксиоматической теории. Категориальное ядро теории~---  
<<ре\-сурс--свой\-ст\-во--дей\-ст\-вие--от\-но\-ше\-ние>>. Над его расширением 
специфицированы одно-, двух- и~трехролевые конструкты~--- основные элементы, из 
которых по правилам склеивания построены схемы отображения информации о~ресурсах, 
действиях, ситуациях, состояниях объекта управления, сложных задачах субъекта 
управления и~мелкозернистых гибридах субъекта моделирования.} 
  
  \KW{логико-математический интеллект; гибридные интеллектуальные системы; 
лингвистический подход; теория ролевых концептуальных моделей}

\DOI{10.14357/19922264150411}

\vskip 14pt plus 9pt minus 6pt

\thispagestyle{headings}

\begin{multicols}{2}

\label{st\stat}

\section{Введение}

  Идея построения многомодельной семиотической системы с~логико-лингвистической, 
имитационной и~аналитической компонентой впервые высказана Я.\,А.~Гельфандбейном, 
А.\,В.~Колесниковым и~И.\,Д.~Рудинским (Рига, Калининград, СССР) в~1981~г.\ в~проекте 
<<Ситуация>> Научного совета по комплексной проблеме <<Кибернетика>> при 
Президиуме АН СССР. Четыре года понадобилось на реализацию и~апробацию  
смен\-но-су\-точ\-но\-го планирования. Спустя более чем десять лет, в~1994--1995~гг.\ 
в~работах L.~Medsker (Вашингтон, США)\linebreak были анонсированы гибридные 
интеллектуальные \linebreak 
системы (ГиИС), по существу совпадающие с~интеллектуальными гиб\-рид\-ны\-ми 
системами~[1], гиб\-рид\-ны\-ми интегрированными сис\-те\-ма\-ми~[2, 3],\linebreak гиб\-рид\-ны\-ми 
информационными системами~[4] и~гиб\-рид\-ны\-ми интеллектуальными адаптивными 
сис\-те\-ма\-ми~[5]. 

Параллельно развивалась и~на\-уч\-но-прак\-ти\-че\-ская <<платформа>> 
методов ГиИС~--- компьютерные сис\-те\-мы поддержки 
принятия решений (\mbox{СППР})~[2]. На рубеже XX и~XXI~вв.\ знания в~этой междисциплинарной 
об\-ласти изучены, обобщены и~опубликованы в~работах А.\,Н.~Борисова, 
А.\,В.~Гаврилова~\cite{6-kir}, А.\,В.~Колесникова~\cite{7-kir}, 
Д.\,А.~Поспелова, Г.\,В.~Рыбиной~\cite{3-kir}, В.\,Б.~Тарасова,  
Н.\,Г.~Ярушкиной~\cite{8-kir}, S.~Goonatilake и~S.~Khebbal~\cite{1-kir},  
L.~Medsker~\cite{4-kir}. В~России сложилось несколько школ в~этой об\-ласти 
междисциплинарных знаний: Д.\,А.~Поспелова\,--\,В.\,Б.~Тарасова;  
В.\,Н.~Вагина\,--\,А.\,П.~Еремеева; Г.\,В.~Рыбиной в~Москве; Н.\,Г.~Ярушкиной 
в~Ульяновске; В.\,Ф.~Пономарева\,--\,А.\,В.~Колесникова в~Калининграде. 
  
  Обобщение многолетнего опыта автоматизированного решения сложных практических 
задач методами междисциплинарных инструментариев показывает их сегодняшнюю 
проблематику:
\begin{enumerate}[(1)]
\item наличие противоречия между свойствами объективной реальности, 
процессами, явлениями и~событиями окружающего мира и~научной картиной мира как части 
мировоззрения ученых и~специалистов, работающих в~информатике, управлении 
и~проектировании; 
\item отображение только ло\-ги\-ко-ма\-те\-ма\-ти\-че\-ско\-го интеллекта, языковой 
коммуникации, левосторонней составляющей рассуждений экспертов и~лица, принимающего 
коллективные решения в~условиях неоднородности и~неопределенности информации, в~то 
время как специалисты из разных предметных областей убеж\-де\-ны, что именно сочетание 
естественного и~визуального языков в~ло\-ги\-ко-ма\-те\-ма\-ти\-че\-ских  
и~ви\-зу\-аль\-но-про\-стран\-ст\-вен\-ных, правосторонних рассуждениях, соответственно, 
сочетание языковой и~визуальной коммуникации над проблемами релевантно феномену 
человеческого мышления; 
\item несмотря на более чем 15-лет\-ние исследования, мало изучены 
свойства <<зернистости>> гиб\-ри\-дов и, как следствие, перспективные цели гиб\-ри\-ди\-за\-ции, 
в~частности мелкозернистые ФГиИС  (МФГиИС); 
\item не исследовано свойство устойчивости ГиИС управ\-ле\-ния, 
в~частности применением методов гомеостатики 
к~взаимодействующим, балансирующим и~компенсирующим недостатки друг друга  
эле\-мен\-там-мо\-де\-лям экспертов и~лиц, принимающих решения (ЛПР), управ\-ле\-нию 
их внут\-рен\-ним противоречием, что 
приведет к~созданию более совершенных с~точки зрения живучести, по\-ме\-хо\-устой\-чи\-вости 
структур сис\-тем управ\-ле\-ния; 
\item информатика не завершила выработку принципов, 
переводящих создание гибридов из уникальной ремесленной мастерской 
  в~про\-ект\-но-кон\-ст\-рук\-тор\-скую деятельность; 
  \item малочисленность конкурирующих 
методологий гибридизации в~информатике; 
\item неразвитость, а~для МФГиИС и~отсутствие 
технологий гибридизации и~инструментальных средств поддержки; 
\item чрезвычайно узкий 
охват автоматизацией только задач пассивных фаз управления~\cite{7-kir}: учета, контроля, 
реже анализа, нормирования, прогнозирования,~--- и~непонимание актуальности 
автоматизированного решения сложных\linebreak задач организации, планирования 
и~регулиро-\linebreak вания, 
а~отсюда~--- слабая вос\-тре\-бо\-ван\-ность\linebreak гиб\-ри\-ди\-за\-ции практикой компьютерных  
\mbox{СППР}; положение усугубляется терминологической путаницей, 
компьютерным жаргоном, несоблюдением нормативов стандартизации. 
\end{enumerate}
  
  Рассмотренная проблематика частично затрагивалась в~работах~[1, 4, 6--9], 
  четвертая проб\-ле\-ма~--- известная, ее решение открывает путь к~новому 
классу информационных сис\-тем~--- гомеостатическим ГиИС. 
  
  Настоящая работа состоит из двух частей. В~первой части, представленной в~данной 
статье, в~рамках лингвистического подхода для молодых ученых, специалистов по 
информатике и~на\-уч\-но-прак\-ти\-че\-ско\-го сообщества, а~также практиков 
СППР исследуется <<зернистость>> гибридов. Во второй части~[10] 
 формулируется в~теории схем ролевых КМ класс 
мелкозернистых ГиИС, представлены результаты 
исследования свойств функциональной, инструментальной неоднородности сложных задач и~
введено понятие двунаправленной гибридизации.

\section{Лингвистический подход к~гибридизации}

  Разработку ГиИС называют гибридизацией~\cite{4-kir, 7-kir}. Это трудоемкий процесс, 
требующий широкого спектра знаний о~предметной области, задачах, методах их решения, 
длительной по времени, сложной обработки информации и~экспериментов. Два\-дца\-ти\-лет\-ний опыт 
авторов статьи показывает, что на разработку <<вручную>> ФГиИС, 
решающей сложную задачу, затрачивается 3--5~чел.-лет.
  
  Выполняя гибридизацию, разработчик имеет дело с~тремя сущностями: 
  \begin{enumerate}[(1)]
  \item сложной задачей 
(объ\-ек\-том-ори\-ги\-на\-лом);
\item  несколькими методами решения задач  
(объ\-ек\-та\-ми-про\-то\-ти\-па\-ми);
\item моделью ГиИС (объ\-ек\-том-ре\-зуль\-та\-том).
\end{enumerate}
 Первые 
два объекта~--- источники информации для гибридизации, а~последний~--- основа 
направленного формообразования, изначально программирующая то, что должно появиться 
в~конце, после трансформации информации. 
  
  Взаимодействие субъектов в~процессе решения сложных задач может осуществляться 
только с~использованием механизма обеспечения такого взаимодействия~--- языка. Вместе 
с~системами формирования и~манипулирования образами, запахами\linebreak
 и~т.\,п.\ язык формирует 
изученный к~настоящему времени в~науке инструментарий повседневной и~
профессиональной деятельности человека. Будучи одновременно средством мысленного 
эксперимента и~моделью внешнего мира, язык профессиональной деятельности (ЯПД)~--- 
объект анализа и~получения информации для гибридизации при разработке ФГиИС. 
В~этой связи излагаемый ниже подход к~разработке ГиИС будем называть 
\textit{лингвистическим}.
  
  Профессиональные языки управления и~моделирования~--- это полиязыки, 
сформировавшиеся в~результате совместной деятельности различных профессиональных 
групп. Трансформация информации об объ\-ек\-тах-ори\-ги\-на\-лах с~полиязыка 
профессиональной коллективной деятельности по принятию решений и~профессионального 
языка разработчика об объ\-ек\-тах-про\-то\-ти\-пах в~объ\-ек\-ты-ре\-зуль\-та\-ты на сегодняшний 
день не изучается математикой, и~ожидать в~обозримом будущем гибри\-ди\-за\-ции по принципу 
трансляции текстов ори\-ги\-на\-лов и~прототипов во внутримашинное представление 
с~получением исполняемой про\-грам\-мы-при\-ло\-же\-ния не приходится. Поэтому 
современная методология предусматривает последовательное, а~зачастую итерационное, 
моделирование на все более и~более искусственных языках, свободных от недостатков ЯПД.
  
  Один из таких искусственных языков постро\-ения моделей~--- язык 
КМ. Концептуальная модель (англ.\ \textit{concept}~--- понятие, идея; общее 
представление; концепция)~--- модель предметной области, состоящая из перечня всех 
понятий, используемых для описания этой области, вместе со свойствами 
и~характеристиками, классификацией этих понятий по типам, ситуациям, признакам 
в~данной области и~законов протекания процессов в~ней~\cite{7-kir}. Такой смысл близок 
к~термину <<концепция>> (от лат.\ \textit{conceptio}~--- понимание, система) как 
определенный способ трактовки ка\-ко\-го-ли\-бо предмета, процесса, явления. 

Таким 
образом, КМ имеет двоякую сущность. С~одной стороны, она 
ограничивает предметную область как совокупность реальных или абстрактных объектов 
(сущностей), связей и~отношений между объектами, а~также процедур %\linebreak
 преобразования этих 
объектов при решении задач. С~другой стороны, она предполагает внесение в~гибридизацию 
субъективных представлений %\linebreak 
разработчика в~виде его знаний и~опыта~--- концепций. Это 
эвристики, специфицирующие по\-нятия ресурса, свойства, действия, структуры, со\-сто\-яния, 
поведения объекта управления, цели, оценки, критерия, плана, задачи субъекта управления, а~
также %\linebreak 
метода, модели, алгоритма, программы субъекта моделирования и~раскрывающие 
содержание, логическую связанность отдельных видов де\-я\-тель\-ности при гибридизации. 
Отсутствие таких эвристик у~разработчика делает гибридизацию 
не\-це\-ле\-на\-прав\-лен\-ной, 
вносит в~эту сложную де\-я\-тель\-ность стохастичность. 
Извлечение, пред\-став\-ле\-ние, хранение, 
тиражирование и~применение %\linebreak
 таких эвристик чрезвычайно важны при гиб\-ри\-ди\-за\-ции.  
Ес\-те\-ст\-вен\-но-язы\-ко\-вые концепции воз\-буж\-да\-ют %\linebreak 
 и~правополу\-шарное мышление, 
хранящее кон\-цеп\-ции-об\-ра\-зы. Следуя Д.\,А.~Поспелову~\cite{10-kir},
 предложен и~развит 
класс ролевых КМ~\cite{7-kir}, а~следуя %\linebreak 
А.\,И.~Уемо\-ву~\cite{11-kir}, 
триада <<вещь--свой\-ст\-во--от\-но\-ше\-ние>> положена в~основу структурирования ЯПД 
экспертов и~ЛПР, ЯПД разработчика, а~также 
неформальной аксиоматической теории концептуального моделирования функциональных 
\mbox{ГиИС}~\cite{12-kir}. 
  
  Концептуализация основывается на описании предметной области в~виде неформальной 
(содержательной) аксиоматической теории. Большинство аксиоматических теорий~--- 
неформальные в~том смысле, что они предполагают использование результатов общей 
теории множеств и~следуют правилам вывода в~логике.
  
  Неформальная аксиоматическая теория Те схем ролевых КМ
включает в~себя четыре списка~\cite{13-kir}: 
  \begin{equation}
  \mathrm{Те}= \left\langle \mathrm{Те}^{\mathrm{ТН}}, 
  \mathrm{Те}^{\mathrm{ТО}}, \mathrm{Те}^{\mathrm{А}}, \mathrm{Те}^{\mathrm{ВТ}}\right\rangle\,,
  \label{e1-kir}
  \end{equation}
    где $\mathrm{Те}^{\mathrm{ТН}}$, $\mathrm{Те}^{\mathrm{ТО}}$~--- 
    неопределяемые и~определяемые термины соответственно; 
$\mathrm{Те}^{\mathrm{А}}$~--- аксиомы; $\mathrm{Те}^{\mathrm{ВТ}}$~--- 
вы\-ска\-зы\-ва\-ния-тео\-ре\-мы, выводимые из~$\mathrm{Те}^{\mathrm{А}}$ 
по некоторым фиксированным логическим правилам. 
  
  Упорядоченная пара $\langle \mathrm{Те}^{\mathrm{ТН}}, 
  \mathrm{Те}^{\mathrm{А}}\rangle$~--- формулировка теории, которая 
совместно с~$\mathrm{Те}^{\mathrm{ВТ}}$ определяет ее предметную область. В~концептуальном 
моделировании понятийная структура предметной области описывается неформальными 
аксиоматическими теориями, формулировки которых называются концептуальными 
моделями.
  
  \noindent
  \textbf{Категориальное ядро теории} {\boldmath{($\mathrm{Те}^{\mathrm{ТН}}$, 
  $\mathrm{Те}^{\mathrm{ТО}}$)}}. Один из результатов 
в~теории систем и~системном анализе получен А.\,И.~Уемовым~\cite{11-kir}, 
предложившим в~качестве базиса системного подхода к~анализу и~синтезу сложных объектов 
триаду категорий <<вещь--свой\-ст\-во--от\-но\-ше\-ние>> (рис.~1,\,\textit{а}).
  
  Главная особенность триады <<вещь--свой\-ст\-во--от\-но\-ше\-ние>>~--- 
соотносительность, все эти категории специфицируются друг через друга, причем 
центральная, основная категория~--- категория вещи~\cite{11-kir}, и~в~категориальном ядре 
не действует\linebreak правило о~запрещении круга в~определениях. Непо\-средственное применение 
ядра <<вещь--свой\-ст\-во--от\-но\-ше\-ние>> к~системному анализу предметной области 
затруднительно в~силу нескольких причин: 
\begin{enumerate}[(1)]
\item категория вещи~--- верхний предел 
абстракции, что вызывает трудности с~ее интерпретацией участниками коллективного 
принятия решений и~разработчиком; 
\item трудности с~интерпретацией понятия 
<<зависимость>> между категориями ядра; поскольку зависимость между ними~--- это 
отношение, то возникает неоднозначность в~толковании отношений; 




\item категория 
<<отношение>> носит общий характер, и~целесообразно выбрать категорию, которая, 
являясь отношением, была бы в~то же время пред\-мет\-но-ори\-ен\-ти\-ро\-ван\-ной. 
\end{enumerate}

\begin{center}  %fig1
\vspace*{-1pt}
\mbox{%
 \epsfxsize=77.662mm
 \epsfbox{kir-1.eps}
 }

\end{center}

\vspace*{-4pt}

\noindent
{{\figurename~1}\ \ \small{Категориальный базис аксиоматической теории ролевых КМ: 
(\textit{а})~триада <<вещь--свой\-ст\-во--от\-но\-ше\-ние>>; (\textit{б})~тетрада  
<<ре\-сурс--свой\-ст\-во--дей\-ст\-вие--от\-но\-ше\-ние>>}}

 \vspace*{10pt}
 
 \noindent
Трудности преодолены следующим образом. Множество вещей ограничено  
ве\-ща\-ми-ре\-сур\-са\-ми (ре-\linebreak сурса\-ми), поскольку прагматичны вещи из\linebreak арсенала ЛПР для 
решения задач (природные явления~--- также ресурс). Множество отношений ограничим 
 отноше\-ни\-ями-дей\-ст\-ви\-ями (действиями), подчеркивая интерес к~отношениям на 
ресурсах, установление или разрыв которых может изменить состояние объекта управления, 
т.\,е.\ операциям и~процессам.


  
 
  Концептуальная модель на рис.~1,\,\textit{б} дает прос\-тое и~прозрачное структурирование 
внешнего мира, утверждая, что это есть мир ресурсов, действий их свойств и~девяти классов 
отношений <<ре\-сурс--ре\-сурс>>, <<дей\-ст\-вие--дей\-ст\-вие>>,  
<<свой\-ст\-во--свой\-ст\-во>>, <<ре\-сурс--свой\-ст\-во>> и~<<свой\-ст\-во--ре\-сурс>>, 
<<ре\-сурс--дей\-ст\-вие>> и~<<дей\-ст\-вие--ре\-сурс>>, а~также  
<<дей\-ст\-вие--свой\-ст\-во>> и~<<свой\-ст\-во--дей\-ст\-вие>>, которые ниже обозначены 
как~$r$, а~их множества~--- $R$. Особенности этой модели~--- ролевой характер 
и~инвариантность к~специфике предметной области. 
  
  Введем множество $X\hm= \{\mathrm{RES}, \mathrm{PR}, \mathrm{ACT}\}$, где 
  $\mathrm{RES}\hm = X^{\mathrm{res}}$, $\mathrm{PR}\hm= 
X^{\mathrm{pr}}$, $\mathrm{ACT}\hm= X^{\mathrm{act}}$~--- множества базисных 
понятий, обозначающих ресурсы, 
свойства и~действия соответственно, и~построим на~$X$ полный граф $G\hm= \langle 
X,R\rangle$. Тогда\linebreak имеем классификацию базовых отношений $R\hm=$\linebreak $= 
\left\{ R^{\mathrm{res}\,\mathrm{res}}, 
R^{\mathrm{pr}\,\mathrm{pr}}, R^{\mathrm{act}\,\mathrm{act}}, R^{\mathrm{res}\,\mathrm{pr}}, R^{\mathrm{pr}\,\mathrm{res}}, R^{\mathrm{res}\,\mathrm{act}},\right.$\linebreak
$R^{\mathrm{act}\,\mathrm{res}}, R^{\mathrm{act}\,\mathrm{pr}}, 
\left.R^{\mathrm{pr}\,\mathrm{act}}\right\}$, где 
$R^{\mathrm{res}\,\mathrm{res}}$, $R^{\mathrm{pr}\,\mathrm{pr}}$, 
$R^{\mathrm{act}\,\mathrm{act}}$, $R^{\mathrm{res}\,\mathrm{pr}}$, 
$R^{\mathrm{pr}\,\mathrm{res}}$, $R^{\mathrm{res}\,\mathrm{act}}$, 
$R^{\mathrm{act}\,\mathrm{res}}$, $R^{\mathrm{act}\,\mathrm{pr}}$, 
$R^{\mathrm{pr}\,\mathrm{act}}$~--- множества 
отношений <<ре\-сурс--ре\-сурс>>, <<свой\-ст\-во--свой\-ст\-во>>, <<дей\-ст\-вие--дей\-ст\-вие>>, 
<<ре\-сурс--свой\-ст\-во>>, <<свой\-ст\-во--ре\-сурс>>, <<ре\-сурс--дей\-ст\-вие>>, 
<<дей\-ст\-вие--ре\-сурс>>, <<дей\-ст\-вие--свой\-ст\-во>>, <<свой\-ст\-во--дей\-ст\-вие>> 
соответственно.
  
  Определив граф $G$, имеем $\mathrm{Те}^{\mathrm{ТН}}\hm= \{$<<вещь>>, <<свойство>>, 
<<отношение>>\}, $\mathrm{Те}^{\mathrm{ТО}}\hm= \{$<<вещь>>, <<свойство>>, <<отношение>>, 
<<ресурс>>, <<действие>>\} и~можно ввести аксиоматику теории~(1). 
  
  \subsection*{Аксиоматика ролевых конструктов ($\mathrm{Те}^{\mathrm{А}}$)}
  
  \noindent
  \textbf{Аксиома 1.}\ \textit{Если $X^\alpha\subseteq X$, а~$R^{\alpha\alpha}\subseteq R$, то 
существует одноролевой конструкт $\mathrm{co}^1\hm= 
R^{\alpha\alpha}(X^\alpha,X^\alpha)$, где 
$\alpha\hm\in \{\mathrm{res}, \mathrm{pr}, \mathrm{act}\}$, 
имеющий в~качестве ролей одну и~ту же категорию ядра. 
Одноролевой конструкт~--- это точечная диаграмма. Например, в~ЯПД диспетчера морского 
порта есть одноролевые конструкты}: {\bfseries\textit{склад рядом причал, камера часть 
холодильника}}. \textit{Семантическая сила одноролевых конструктов невелика. Они 
выражают смысл на абстрактном уровне.}
  
  \smallskip
  
  \noindent
  \textbf{Аксиома 2.}\ \textit{Если $X^\alpha, X^\beta\hm\subseteq X$, а~$R^{\alpha\alpha}$, 
$R^{\beta\beta}$, $R^{\alpha\beta}l\subseteq$\linebreak $\subseteq R$, то существует двухролевой конструкт 
$\mathrm{co}^2\hm= R^{\alpha\alpha}(X^\alpha, X^\alpha)\circ 
R^{\beta\beta}(X^{\beta}, X^{\beta})\circ 
R^{\alpha\beta}(X^\alpha,X^\beta)$, где 
$\alpha,\beta\hm\in \{\mathrm{res}, \mathrm{pr},\mathrm{act}\}$; $\alpha\not=\beta$; 
$\circ$~--- конкатенация. Двухролевой конструкт включает два одноролевых конструкта, 
связанных отношениями~$R^{\alpha\beta}$. Двухролевых конструктов шесть, по числу 
ребер на графе~$G$. Переход к~двухролевой модели~--- это переход от точечной к~линейной 
диаграмме. В~ЯПД диспетчера морского порта есть двухролевые конструкты}: 
{\bfseries\textit{судно Агат стоит у причала №\,16, ускоренная выгрузка используется на 
судне Топаз}}. \textit{Линейная диаграмма, какой бы длинной она ни была, не обеспечивает 
целостную картину внешнего мира.}
  
  \smallskip
  
  \noindent
  \textbf{Аксиома 3.} \textit{Если $X^\alpha, X^\beta, X^\gamma\subseteq X$, а~
$R^{\alpha\alpha}$, $R^{\beta\beta}$, $R^{\gamma\gamma}$,  $R^{\alpha\gamma}$, 
$R^{\beta\alpha}$, $R^{\gamma\beta}\subseteq R$, то существует трехролевой 
конструкт~$\mathrm{co}^3$, которым назовем выражение $\mathrm{co}^3\hm= R^{\alpha\alpha} 
(X^\alpha,X^\alpha)\circ 
R^{\beta\beta}(X^\beta,X^\beta)\circ R^{\gamma\gamma} 
(X^\gamma,X^\gamma) \circ$\linebreak $\circ R^{\alpha\gamma}(X^\alpha,X^\gamma)\circ 
R^{\beta\alpha}(X^\beta,X^\alpha) \circ R^{\gamma\beta}(X^\gamma, X^\beta)$, где 
$\alpha,\beta,\gamma\hm\in \{\mbox{``}\mathrm{res}\mbox{''}, \mbox{``}\mathrm{pr}\mbox{''}, 
\mbox{``}\mathrm{act}\mbox{''}\}$; $\alpha\not=\beta\not=\gamma$. Трехролевых конструктов 
девять. Переход от двухролевой\linebreak модели к~трехролевой~--- это переход от линейной 
разомкнутой диаграммы к~фигурной (треугольной) замкнутой диаграмме, концептуально 
полной для триады <<ре\-сурс--свой\-ст\-во--дей\-ст\-вие>> картине мира. Напри\-мер, в~ЯПД 
диспетчера есть трехролевой конструкт}: {\bfseries\textit{судно стоит у причала №\,15, и~на 
нем выполняется ускоренная выгрузка мороженой рыбы.}}
  
  \textit{В итоге имеем $\mathrm{Те}^{\mathrm{А}}\hm= \{A_1, A_2, A_3\}$ и~три множества конструктов 
$\mathrm{CO}^1\hm= \{\mathrm{co}_1^1, \ldots, \mathrm{co}_3^1\}$, 
$\mathrm{CO}^2\hm= \{\mathrm{co}_1^2, \ldots, \mathrm{co}_6^2\}$, 
$\mathrm{CO}^3\hm=\{\mathrm{co}^3_1, \ldots, \mathrm{co}_9^3\}$. На категориальном ядре, т.\,е.\ множестве~$X$, 
невозможно задать никаких других конструктов.}
  
  \smallskip
  
  \noindent
  \textbf{Расширение категориального ядра} {\boldmath{($\mathrm{Те}^{\mathrm{ТН}},\,\mathrm{Те}^{\mathrm{ТО}}$)}}. 
  Для моделирования 
предметных знаний об объекте управления категориальное ядро расширим $\mathrm{PAR}\hm= 
X^{\mathrm{par}}\hm\subseteq X^{\mathrm{pr}}$~--- 
физическими свойствами (пара\-мет\-рами), $\mathrm{CH}\hm= 
X^{\mathrm{ch}}\hm\subseteq X^{\mathrm{pr}}$~--- 
характеристическими свойствами (характеристиками), 
$\mathrm{NAM}\hm= X^{\mathrm{nam}}\hm\subseteq X^{\mathrm{pr}}$~--- 
именными свойствами (именами), $\mathrm{MES}\hm= 
X^{\mathrm{mes}}$~--- единицами измерения\linebreak
 (мерами), $\mathrm{VAL}\hm= X^{\mathrm{val}}$~--- значениями 
и~$\mathrm{ST}\hm= X^{\mathrm{st}}$~--- состояниями. Для моделирования предмет\-ных знаний о~субъекте 
управления в~$X$ включены $\mathrm{EST}\hm= X^{\mathrm{est}}$~--- оценки 
и~$\mathrm{PRB}\hm= X^{\mathrm{prb}}\hm= 
\{\mathrm{prb}~\mbox{---}~\mbox{<<задача>>}\}$, а~о~субъекте моделирования~--- 
$\mathrm{MET}\hm= X^{\mathrm{met}}$~---  
<<метод>>, mod~--- <<модель>>, $\mathrm{prg}$~--- <<программа>>$\}$. Чтобы иметь 
возможность расширять категориальное ядро, введем и~$\mathrm{EX}\hm= 
X^{\mathrm{ex}}$~--- экзотические 
понятия.

\smallskip
  
 \noindent
  \textbf{Правило склеивания ролевых конструктов.} Для по\-строения множества  
вы\-ска\-зы\-ва\-ний-тео\-рем~$\mathrm{Те}^{\mathrm{ВТ}}$ теории~(1) 
сформулируем на~$X$ правило 
склеивания (во всех правилах $\alpha,\beta\hm\in \{\mathrm{res}, 
\mathrm{pr}, \mathrm{act},$ $\mathrm{mes}, \mathrm{val}, \mathrm{st}, 
\mathrm{est}, \mathrm{prb}, \mathrm{met}, \mathrm{mod},
\mathrm{prg},\mathrm{par}, %$\linebreak $  
\mathrm{ch},  \mathrm{nam}\}$) следующим образом: 
\begin{enumerate}[(1)]
\item трехролевой конструкт $\mathrm{co}^3\hm\in \mathrm{CO}^3$ есть схема 
$\mathrm{sch}\hm\in \mathrm{SCH}$ ролевых КМ; 
\item если $\mathrm{sch}\hm= A_1\circ 
R^{\alpha\,\alpha}(X^\alpha,X^\alpha)\circ B_1$~--- схема ролевых КМ, где $A_1$ и~$B_1$~--- 
любые фрагменты (в~том числе и~пустые) модели, не содержащие 
$R^{\alpha\,\alpha}(X^\alpha, X^\alpha)$, исключая случай $A_1\hm= B_1\hm= \varnothing$, 
и~$\mathrm{co}\hm= A_2\circ R^{\alpha\,\alpha} (X^\alpha,X^\alpha)\circ B_2$~--- двух- или 
трехролевой конструкт, где $A_2$ и~$B_2$~--- любые фрагменты (в~том числе и~пустые), не 
содержащие $R^{\alpha\,\alpha}(X^\alpha,X^\alpha)$, исключая случай $A_2\hm= B_2\hm= 
\varnothing$, то $\mathrm{sch}\hm= A_1\circ R^{\alpha\,\alpha}(X^\alpha, X^\alpha)\circ B_1
\circ A_2\circ B_2$~--- также схема ролевых КМ (склеивание треугольников вершиной); 
\item если 
$\mathrm{sch}_1\hm= A_1\circ R^{\alpha\,\beta}(X^\alpha, X^\beta)\circ B_1$ 
и~$\mathrm{sch}_2\hm= A_2\circ  R^{\beta\alpha}(X^\beta, X^\alpha)\circ B_2$~--- схемы ролевых КМ, причем 
$A_1\hm= A_2$ и~$B_1\hm=B_2$~--- не\-пус\-тые фрагменты, то $\mathrm{sch}_3\hm= A_1\circ R^{\alpha,\beta}(X^\alpha,
 X^\beta)\circ$\linebreak $\circ 
R^{\beta\,\alpha} (X^\beta, X^\alpha)\circ B_1$~--- также схема ролевых КМ (добавление 
обратного отношения); 
\item если $\mathrm{sch}_1\hm= A_1\circ R^{\alpha\,\beta} (X^\alpha, X^\beta)\circ 
B_1$ и~$\mathrm{sch}_2\hm= A_2\circ R^{\alpha\,\beta} (X^\alpha, X^\beta)\circ B_2$~--- схемы ролевых 
КМ, причем $A_1$, $B_1$, $A_2$ и~$B_2$~--- непустые фрагменты, то $\mathrm{sch}_3\hm= A_1\circ 
A_2\circ R^{\alpha\,\beta}(X^\alpha, X^\beta)\circ B_1 \circ B_2$~--- также схема 
ролевых КМ 
(склеивание треугольников стороной); 
\item в схеме ролевых КМ не может быть более одного 
несвязанного одноролевого конструкта; 
\item никаких других схем ролевых КМ нет.
\end{enumerate}
  
  Таким образом, схемы ролевых КМ $\mathrm{sch}\hm\in \mathrm{SCH}$ строятся из  
кон\-струк\-тов-ак\-си\-ом, и~есть вы\-ска\-зы\-ва\-ния-тео\-ре\-мы теории~(1), полученные по 
правилу склеивания. Схемы~--- это эвристики, знания и~опыт
 разработчика в~виде 
концепций. Они, с~одной стороны, повторяют структурирование ЯПД коллективного 
интеллекта, вырабатывающего и~принимающего решения, специфицируют понятия ресурса,
свойства, действия, структуры, состояния, поведения объекта управ\-ле\-ния, цели, оценки, 
критерия,\linebreak\vspace*{-12pt}

\columnbreak

\begin{center}  %fig2
\vspace*{-1pt}
 \mbox{%
 \epsfxsize=76.312mm
 \epsfbox{kir-2.eps}
 }



\vspace*{6pt}

\noindent
{{\figurename~2}\ \ \small{Лингвистический подход к~гибридизации}}

\end{center}

 \vspace*{6pt}

\noindent
 плана, задачи субъекта управ\-ле\-ния. С~другой стороны, повторяют 
структурирование ЯПД коллективного интеллекта разработчика и~специфицируют понятия 
метода, модели, алгоритма, программы и,~с~третьей стороны, повторяют объекты 
конструирования и~целеполагания~--- мо\-дель-эле\-мен\-та, гетерогенное модельное поле, 
функциональную ГиИС, мелкозернистую ГиИС, поведение ГиИС, раскрывают содержание, 
логическую связанность деятельности при гибридизации.
  
  Суть лингвистического подхода к~гибридизации показана на рис.~2, а~основные 
положения изложены ниже.
  

\noindent
\begin{enumerate}[1.]
\item Базируется на компьютерном моделировании ло\-ги\-ко-математического интеллекта~--- 
одного из множества интеллектов человека по Гард\-неру.
\item Данные, знания, опыт, связанные с~решением сложных задач  
(объ\-ек\-та\-ми-ори\-ги\-на\-ла\-ми) в~систе\-ме управ\-ле\-ния, должны быть 
вербализованы, извлекаемы из экспертов, ЛПР и~других источников~--- текстов на ЯПД, баз 
данных. Поиск, извлечение знаний об объ\-ек\-тах-ори\-ги\-на\-лах и~конструирование знаков 
информационного языка выполняются по соответствующим схе\-мам-эври\-сти\-кам 
теории~(1).
\item Данные, знания, опыт, связанные с~методами моделирования  
(объ\-ек\-та\-ми-про\-то\-ти\-па\-ми) и~их применением 
в~на\-уч\-но-ис\-сле\-до\-ва\-тель\-ских работах,  
про\-ект\-но-кон\-ст\-рук\-тор\-ской и~эксплуатационной деятельности, должны быть 
вербализованы, извлекаемы из экспертов и~других источников~--- научной,  
учеб\-но-ме\-то\-ди\-че\-ской литературы, про\-ект\-но-кон\-ст\-рук\-тор\-ской документации. 
\item  Данные, знания, опыт, связанные с~ГиИС (объ\-ек\-та\-ми-ре\-зуль\-та\-та\-ми) и~их 
применением для автоматизированного решения сложных задач, должны быть 
вербализованы, извлекаемы из экспертов и~представлены методологическим учением 
(методологией). Объ\-ек\-ты-ре\-зуль\-та\-ты конструируются, а~их поведение специфицируется как 
знаки информационного языка по соответствующим схе\-мам-эври\-сти\-кам теории~(1). 
\item Данные, знания, опыт, связанные с~действиями по преобразованию вербальной 
информации об объ\-ек\-тах-ори\-ги\-на\-лах (функциональных %\linebreak
 родительских признаках) 
и~объ\-ек\-тах-прото\-типах (инструментальных родительских %\linebreak 
при\-знаках)  
в~объ\-ек\-ты-ре\-зуль\-та\-ты (целевые комбинации 
родительских функциональных 
и~инструментальных признаков), должны быть вербализованы, извлекаемы из экспертов 
и~представлены методологическим учением (методологией)~--- ло\-ги\-че\-ски увя\-зан\-ной 
и~це\-ле\-на\-прав\-лен\-ной последовательностью действий.
\end{enumerate}



\section{Заключение}



  В рамках лингвистического подхода к~гибридизации рассмотрено моделирование 
МФГиИС в~неформальной 
аксиоматической теории схем ролевых КМ. Такие системы обладают 
значительным потенциалом имитации рассуждений специалистов, решающих сложные 
задачи. С~одной стороны, модельная архитектура МФГиИС отображает на 
компьютере функциональную структуру сложной задачи, области ее однородных 
параметров, представленных множествами подзадач, что позволяет имитировать несколько 
линий рассуждений. С~другой стороны, в~таких системах для решения подзадач 
используются не только известные в~тео\-рии принятия решений, исследовании операций, 
искусственном интеллекте, математической статистике методы моделирования  
(ме\-то\-ды-ро\-ди\-те\-ли), но и~конструируются из набора инструментальных средств 
(зерен) ме\-то\-ды-по\-том\-ки, лишенные родительских недостатков, что позволяет вести 
рас\-суж\-де\-ния эффективнее. 


  
{\small\frenchspacing
 {%\baselineskip=10.8pt
 \addcontentsline{toc}{section}{References}
 \begin{thebibliography}{99}
 

 
\bibitem{1-kir}
\Au{ Goonatilake S., Khebbal S.} Intelligent hybrid systems~// 1st Singapore Conference 
(International) on Intelligent Systems Proceedings, 1992. P.~356--364.
\bibitem{2-kir}
\Au{Поспелов Г.\,С.} Искусственный интеллект~--- основа новой информационной 
технологии.~--- М.: Наука, 1988. 280~c.
\bibitem{3-kir}
\Au{Рыбина Г.\,В.} Интегрированные экспертные системы: современное состояние, 
проблемы и~тенденции~// Известия РАН. Теория и~системы управления, 2002. №\,5.  
С.~111--126.
\bibitem{4-kir}
\Au{Medsker L.\,R.} Hybrid intelligent systems.~--- Kluwer Academic Publ., 1995. 295~p.
\bibitem{5-kir}
\Au{Kasabov N., Kozma R.} Hybrid intelligent adaptive systems: A~framework and a~case study 
on speech recognition~// Int. J.~Intell. Syst., 1998. Vol.~13. Iss.~6. P.~455--466.
\bibitem{6-kir}
\Au{Гаврилов А.\,В.} Гибридные интеллектуальные сис\-те\-мы.~--- Новосибирск: НГТУ, 2003. 
164~с.
\bibitem{7-kir}
\Au{Колесников А.\,В.} Гибридные интеллектуальные сис\-те\-мы. Теория и~технология 
разработки~/ Под ред. А.\,М.~Яшина.~--- СПб: СПбГТУ, 2001. 711~c.
\bibitem{8-kir}
\Au{Ярушкина Н.\,Г.} Основы теории нечетких и~гибридных сис\-тем.~--- М.: Финансы 
и~статистика, 2004. 320~с.
\bibitem{9-kir}
\Au{Колесников А.\,В., Кириков~И.\,А.} Методология и~технология решения сложных задач 
методами функциональных гибридных интеллектуальных систем.~--- М.: ИПИ РАН, 2007. 
387~с.
\bibitem{9-1-kir}
\Au{Кириков И.\,А., Колесников~А.\,В., Листопад~С.\,В., 
Румовская~С.\,Б.} Мелкозернистые гибридные интеллектуальные системы. Часть~2. 
Двунаправленная 
гибридизация>>~// <<Информатика и~её применения>>, 2016. Вып.~1  (в~печати).
\bibitem{10-kir}
\Au{Поспелов Д.\,А.} Ситуационное управление: теория и~практика.~--- М.: Наука, 
1986. 288~с.
\bibitem{11-kir} %12
\Au{Уемов А.\,И.} Вещи, свойства, отношения.~--- М.: Институт философии АН СССР, 1963. 
184~с.
\bibitem{12-kir} %13
\Au{Колесников А.\,В.} Моделирование естественных гетерогенных систем 
коллективного принятия решений~// Системный анализ и~информационные 
технологии (САИТ-2015): Тр. 6-й Междунар. конф.~--- М.: ИСА РАН, 2015. Т.~1.  
С.~7--16.
\bibitem{13-kir}
\Au{Лелюк В.\,А.} Концептуальное проектирование систем с~базами знаний.~--- Харьков: 
Основа, 1990. 144~с.
 \end{thebibliography}

 }
 }

\end{multicols}

\vspace*{-3pt}

\hfill{\small\textit{Поступила в~редакцию 20.09.15}}

%\vspace*{8pt}

\newpage

%%\vspace*{-24pt}

%\hrule

%\vspace*{2pt}

%\hrule

\vspace*{-24pt}

\def\tit{FINE-GRAINED HYBRID INTELLIGENT
  SYSTEMS.\\ PART~1: LINGUISTIC~APPROACH}

\def\titkol{Fine-grained hybrid intelligent
  systems. Part~1:~Linguistic~approach}

\def\aut{I.\,А.~Kirikov$^1$, А.\,V.~Kolesnikov$^{1,2}$, S.\,V.~Listopad$^1$, 
and~S.\,B.~Rumovskaya$^1$}

\def\autkol{I.\,А.~Kirikov, А.\,V.~Kolesnikov, S.\,V.~Listopad, 
and~S.\,B.~Rumovskaya}

\titel{\tit}{\aut}{\autkol}{\titkol}

\vspace*{-9pt}


\noindent
$^1$Kaliningrad Branch of the Federal Research Center ``Computer Science and
Control'' of the Russian Academy\linebreak
$\hphantom{^1}$of Sciences, 5~Gostinaya Str.,     Kaliningrad 
236000,  Russian Federation
   
   \noindent
   $^2$Immanuel Kant Baltic Federal University, 14~Nevskogo Str., Kaliningrad 236041,
Russian Federation


\def\leftfootline{\small{\textbf{\thepage}
\hfill INFORMATIKA I EE PRIMENENIYA~--- INFORMATICS AND
APPLICATIONS\ \ \ 2015\ \ \ volume~9\ \ \ issue\ 4}
}%
 \def\rightfootline{\small{INFORMATIKA I EE PRIMENENIYA~---
INFORMATICS AND APPLICATIONS\ \ \ 2015\ \ \ volume~9\ \ \ issue\ 4
\hfill \textbf{\thepage}}}

\vspace*{3pt}


   
  
  \Abste{The paper considers the problematic of interdisciplinary tools and the property of 
``grain'' for hybrids in informatics. The results are presented within the linguistic approach, the core 
of which is transformation of the verbalized information about objects-originals (complex subjects) 
and objects-prototypes (modeling approaches) to objects-results (functional hybrid intelligent 
system). It exists in polylanguages of professional activity. The transformation is directed by 
heuristics, which are the schemes of the conceptual role models in the informal axiomatic theory. 
The category core of the theory is ``resource--property--operation--relation.'' Mono-, bi-, 
and tri-role 
constructs are specified over its extension, which are the basic elements. On the basis of these 
elements the schemes of reflection of information about resources, operations, situations, state of 
object of management, complex tasks of management entity, and fine-grained hybrids of modeling 
entity are built. }
   
  \KWE{logical-mathematical intelligence; hybrid intelligent systems; 
  linguistic approach; theory 
of role conceptual models}

\DOI{10.14357/19922264150411}

%\Ack
%    \noindent


%\vspace*{3pt}

  \begin{multicols}{2}

\renewcommand{\bibname}{\protect\rmfamily References}
%\renewcommand{\bibname}{\large\protect\rm References}

{\small\frenchspacing
 {%\baselineskip=10.8pt
 \addcontentsline{toc}{section}{References}
 \begin{thebibliography}{99}
\bibitem{1-kir-1}
\Aue{Goonatilake, S., and S.~Khebbal}. 1992. Intelligent hybrid systems. 
\textit{1st Singapore  Conference (International) on Intelligent Systems
Proceedings}. 356--364.
\bibitem{2-kir-1}
\Aue{Pospelov, G.\,S.} 1988. \textit{Iskusstvennyy intellekt~--- osnova novoy informatsionnoy 
tekhnologii} [Artificial intelligence~--- the base of the new information processing technology]. 
Moscow: Nauka. 280~p.
\bibitem{3-kir-1}
\Aue{Rybina, G.\,V.} 2002. Integrirovannye ekspertnye sistemy: Sovremennoe sostoyanie, 
problemy i~tendentsii [Integrated expert systems: Actual status, problems and scientific trends]. 
\textit{Izvestiya RAN. Теoriya i~Sistemy Upravleniya} [J.~Computer Systems Sciences 
International] 5:111--126.
\bibitem{4-kir-1}
\Aue{Medsker, L.\,R.} 1995. \textit{Hybrid intelligent systems}. Kluwer Academic Publ. 
295~p.
\bibitem{5-kir-1}
\Aue{Kasabov, N., and R. Kozma}. 1998. Hybrid intelligent adaptive systems: A~framework and 
a~case study on speech recognition. \textit{Int. J.~Intell. Syst.} 13(6):455--466.
\bibitem{6-kir-1}
\Aue{Gavrilov, A.\,V.} 2003. \textit{Gibridnye intellektual'nye sistemy} [Hybrid intelligent 
systems]. Novosibirsk: NNSTU Publs.  164~p.
\bibitem{7-kir-1}
\Aue{Kolesnikov, A.\,V.} 2001. \textit{Gibridnye intellektual'nye sistemy. Теoriya i~tekhnologiya 
razrabotki} [Hybrid artificial systems. Theory and development technology]. 
St.\ Petersburg:  \mbox{SPbGTU}. 711~p.
\bibitem{8-kir-1}
\Aue{Jarushkina, N.\,G.} 2004. \textit{Osnovy teorii nechetkikh i~gibridnykh sistem} [Foundations 
of theory of fuzzy and hybrid systems]. Moscow: Finance and Statistics. 320~p.
\bibitem{9-kir-1}
\Aue{Kolesnikov, A.\,V., and I.\,A.~Kirikov}. 2007. \textit{Metodologiya i~tekhnologiya 
resheniya slozhnykh zadach metodami funk\-tsi\-o\-nal'nykh gibridnykh intellektual'nykh sistem} 
[Methodology and technology for solving of complex problems using the methodology of 
functional hybrid artificial systems]. Moscow: IPI RAN. 387~p.
\bibitem{9-1-kir-1}
\Aue{Kirikov, I.\,A., A.\,V.~Kolesnikov, S.\,V.~Listopad, 
and S.\,B.~Rumovskaya}. 2016 (in press). 
Melkozernistye gibridnye intellektual'nye sistemy. Chast'~2:
Dvunapravlennaya gibridizatsiya 
[Fine-grained hybrid intelligent
  systems. Part~2: Bidirectional hybridization]. 
  \textit{Informatika i~ee Primeneniya}~--- \textit{Inform. Appl.} 1.
\bibitem{10-kir-1}
\Aue{Pospelov, D.\,A.} 1986. \textit{Situatsionnoe upravlenie: Теoriya i~praktika} [Situation 
control: Theory and practice]. Moscow: Nauka. 288~p.
\bibitem{11-kir-1}
\Aue{Uemov, A.\,I.} 1963. \textit{Veshchi, svoystva, otnosheniya} [Items, properties, relations]. 
Moscow: Institute of Philosophy of the Academy of Sciences of the USSR. 184~p.
\bibitem{12-kir-1}
\Aue{Kolesnikov, A.\,V.} 2015. Modelirovanie estestvennykh geterogennykh sistem kollektivnogo 
prinyatiya resheniy [Modeling of the natural heterogeneous systems of collective decision-making]. 
\textit{6th 
Conference (International) ``Systems Analysis and Information Теchnology'' Proceedings}. 
Moscow: ISA RAS. 1:7--16.
\bibitem{13-kir-1}
\Aue{Leljuk, V.\,A.} 1990. \textit{Kontseptual'noe proektirovanie sistem s~bazami znaniy} 
[Substructured developmet of the systems with knowledge base]. Kharkov: Osnova. 144~p.
\end{thebibliography}

 }
 }

\end{multicols}

\vspace*{-3pt}

\hfill{\small\textit{Received September 20, 2015}}

\Contr

\noindent
\textbf{Kirikov Igor A.}\ (b.\ 1955)~---
Candidate of  Sciences (PhD) in technology; director, Kaliningrad Branch of the 
Federal Research Center ``Computer Science and Control'' of the Russian Academy 
of Sciences, 5~Gostinaya Str., Kaliningrad 236000,  Russian Federation; baltbipiran@mail.ru

\vspace*{3pt}

\noindent
\textbf{Kolesnikov Alexander V.}\ (b.\ 1948)~---
Doctor of Sciences in technology; professor, Department of telecommunications, 
Immanuel Kant Baltic Federal University, 14~Nevskogo Str., Kaliningrad 236041,
Russian Federation; senior scientist, Kaliningrad Branch of 
the Federal Research Center ``Computer Science and Control'' of the Russian 
Academy of Sciences, 5~Gostinaya Str., Kaliningrad 236000,  Russian Federation; avkolesnikov@yandex.ru

\vspace*{3pt}

\noindent
\textbf{Listopad Sergey V.}\ (b.\ 1984)~---
Candidate of  Sciences (PhD) in technology; scientist, Kaliningrad Branch of the 
Federal Research Center ``Computer Science and Control'' of the Russian Academy 
of Sciences, 5~Gostinaya Str., Kaliningrad 236000,  Russian Federation;  ser-list-post@yandex.ru

\vspace*{3pt}

\noindent
\textbf{Rumovskaya Sophiya B.}\ (b.\ 1985)~--- programmer~I, Kaliningrad Branch 
of the Federal Research Center ``Computer Science and Control'' of the Russian 
Academy of Sciences, 5~Gostinaya Str., Kaliningrad 236000,  Russian Federation; sophiyabr@gmail.com
\label{end\stat}


\renewcommand{\bibname}{\protect\rm Литература}