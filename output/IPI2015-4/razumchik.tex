\renewcommand{\figurename}{\protect\bf Figure}
\renewcommand{\tablename}{\protect\bf Table}

\def\stat{razumchik}


\def\tit{ALGEBRAIC METHOD FOR APPROXIMATING JOINT STATIONARY DISTRIBUTION IN FINITE CAPACITY
QUEUE WITH NEGATIVE CUSTOMERS AND TWO QUEUES$^*$}

\def\titkol{Algebraic method for approximating joint stationary distribution in finite capacity
queue with negative customers and two queues}

\def\autkol{R.~Razumchik}

\def\aut{R.~Razumchik$^1$}

\titel{\tit}{\aut}{\autkol}{\titkol}

{\renewcommand{\thefootnote}{\fnsymbol{footnote}}
\footnotetext[1] {This work was supported in part by the
Russian Foundation for Basic Research (grants 15-07-03007 and 13-07-00223).}}

\renewcommand{\thefootnote}{\arabic{footnote}}
\footnotetext[1]{Institute of Informatics Problems, Federal Research Center 
``Computer Science and Control'' of the Russian Academy of Sciences,
44-2 Vavilov Str., Moscow 119333, Russian Federation;
Peoples' Friendship University, 6 Miklukho-Maklaya Str., Moscow 117198, Russian Federation,
rrazumchik@ipiran.ru}


\vspace*{-9pt}

\def\leftfootline{\small{\textbf{\thepage}
\hfill INFORMATIKA I EE PRIMENENIYA~--- INFORMATICS AND APPLICATIONS\ \ \ 2015\ \ \ volume~9\ \ \ issue\ 4}
}%
 \def\rightfootline{\small{INFORMATIKA I EE PRIMENENIYA~--- INFORMATICS AND APPLICATIONS\ \ \ 2015\ \ \ volume~9\ \ \ issue\ 4
\hfill \textbf{\thepage}}}


\Abste{Consideration is given to the single-server
queueing system (QS) with a~Poisson flow of (ordinary)
customers and Poisson flow of negative customers.
There is a~queue of capacity~$k$ ($0<k<\infty$), where ordinary customers
wait for service. 
If an ordinary customer finds the queue full upon an arrival, it is considered to be lost.
Each negative customer upon arrival 
moves one ordinary customer from the queue, if it not empty, to 
another queue (bunker) of capacity~$r$ ($0<r<\infty$) and after that it leaves the system. 
If upon arrival of a negative customer the queue is not empty 
and the bunker is full, the negative customer and one ordinary customer from
the queue leave the system. In all other cases, an
arrival of a~negative customer has no effect on the system.
Customers from bunker are served with relative priority (i.\,e.,
a customer from bunker enters server if only there 
are no customers in the queue to be served).
Service times of customers from both the queue and the bunker 
are exponentially distributed with the same parameter. 
Purely algebraic method based on generating functions, 
Chebyshev and Gegenbauer polynomials
for approximate calculation of
joint stationary probability distribution
is presented for the case $k=r$. 
Numerical examples, showing both pros and cons of the method are provided.}


\KWE{queueing system; negative customers; Gegenbauer polynomials; 
stationary distribution; approximation}

\DOI{10.14357/19922264150407} 

\vspace*{6pt}


\vskip 12pt plus 9pt minus 6pt

      \thispagestyle{myheadings}

      \begin{multicols}{2}

                  \label{st\stat}


\section{Introduction}

\noindent
Queueing systems and networks with negative customers have received
significant attention from researches since they appeared in
early 1990s~[1--4].
This topic has already collected a big body of knowledge
and it is neither possible nor intended to state here
the comprehensive list of related papers.
For review of the publications up to~2011, one
can be referred to~[5--7].
Detailed bibliography up to~2011 can be found in~\cite{tien}.
Among the latest research results, one can mention, 
for example,~[7--19]. %\cite{newnew}--\cite{mikl}. %~\cite{aaa3}, 
%\cite{aaa1}--\cite{aaa4}.
The novel approach to the analysis of networks with negative customers one
can find in~\cite{z1, z2}.

The concept of negative customer turned out to be versatile.
Quite often, it is assumed that the arrival of a~negative customer leads to
permanent removal of other (ordinary) customers residing
in the system. It can also be the case that 
negative arrivals move ordinary customers
between queues or nodes in the network.
One of the problems' statements introduced in~\cite{ros-cas} implies that negative
customers do not remove ordinary customers
waiting for service in the buffer 
from the system, but delay their service by
displacing them into another queue (bunker)
wherefrom they are served according to a certain discipline
(say, for example, relative priority).
Latest results of the analysis of QS with such type of
negative customers are presented in~\cite{new8, new7, new9, mikl}.
Particular interest in the analysis of
such QS presents the case
when queues' capacities (of buffer and bunker) are finite.
{\looseness=-1

}

It is shown in~\cite{aaa6} that apart from matrix analytic approach,
Gaussian elimination method, and known numerical methods,
it is possible to obtain joint stationary distribution
utilizing properties of generating functions
and special functions (Chebyshev and Gegenbauer polynomials).
Note that such approach is not new as it was already used in~\cite{avrach} 
(and later in some other papers, for example, in~\cite{peter})
but for the analysis of another types of QS.
The method presented in~\cite{aaa6}
which allows one to do it is direct and suitable for exact arithmetics implementation
though still suffers from computational problem.
{It is expressed} in the need to efficiently perform basic operations with arbitrary big numbers.

In this paper, the modification of
method from~\cite{aaa6} is presented, which makes it, from the one hand, approximate
and, from the other hand, allows one to carry out the computations
of stationary performance characteristics
for higher values of queues' capacities almost without
loosing accuracy. One interesting feature of the proposed method revealed during numerical
experiments (see section~4) indicates that
depending on initial values, one can obtain exact values for several
performance characteristics using worst-case approximation (which substantially cuts computation time).


The rest of the paper is organized as follows. Section~2 is devoted to
the detailed description of the system, equilibrium equations for
joint stationary probabilities, and several useful relations
that follow from rate-in-rate-out principle.
In section~3, it is shown how to use
generating functions and Chebyshev and Gegenbauer polynomials in order
to obtain expressions for joint stationary probability distribution
and to calculate stationary performance characteristics.
Computational results are provided in section~4. In conclusion,
the summary of the paper and future research plans are given.

\section{Description of~the~System}

\noindent
Consideration is given to the QS with
incoming Poisson flow of ordinary customers of intensity~$\lambda$
and Poisson flow of negative customers of intensity~$\lambda^-$.
For ordinary customers, there is a finite buffer of size~$k$.
If upon arrival of ordinary customer buffer is full, it is lost.
A~negative customer upon arrival moves one ordinary customer from buffer
(if it is not empty; negative customers cannot remove customer which is
in service) to another queue of finite capacity~$r$ (bunker).
If upon arrival of a negative customer the queue in the buffer is not empty but
bunker is full, the customer that is displaced from buffer is lost.
Customers from both queues are served according
to exponential distribution with the same parameter~$\mu$,
FIFO (first in, first out) discipline, but the customers from 
the bunker are served with relative
priority. This means that the customer from the bunker goes to the server
only when server finished service of all customers from the buffer.

Denote by $p_{ij}$ stationary probability of the fact that
there are~$i$~customers in buffer, $j$~customers in bunker,
and server is busy. By~$p_0$, denote stationary probability of empty system.
Stationary distribution satisfies the following system of equilibrium equations:
\begin{equation}    
\lambda p_0 = \mu p_{00}\,;
\label{eq31}
\end{equation}
\begin{equation}    
(\lambda+\mu)p_{00}=\lambda p_0 + \mu p_{10} + \mu p_{01}\,,\ \ i=0\,, \ j=0\,;
\label{eq32}
\end{equation}

\vspace*{-12pt}

\noindent
\begin{multline}
\left(\lambda+\mu+\lambda^-\right)
p_{i0}=\lambda p_{i-1,0} + \mu p_{i+1,0}\,, \\ 
i=1\,, \dots, k-1\,, \ j=0\,;
\label{eq33}
\end{multline}
%%%%%%%%%%%%%%%%

\vspace*{-6pt}

\noindent
\begin{equation}
\left(\mu+\lambda^-\right)p_{kj}=\lambda p_{k-1,j}\,,\
i=k\,, \ j=0\,, \dots, r\,;
\label{eq34}
\end{equation}
%%%%%%%%%%%%%%%%

\vspace*{-12pt}

\noindent
\begin{multline}
\left(\lambda+\mu\right) p_{0j}=\mu p_{1j} + \lambda^- p_{1,j-1} + 
\mu p_{0,j+1}\,,\\ 
i=0\,, \ j=1\,, \dots , r-1\,;
\label{eq35}
\end{multline}

\vspace*{-12pt}

\noindent
\begin{multline}
\left(\lambda+\mu\right) p_{0r}=(\mu + \lambda^- ) p_{1r}+ \lambda^- p_{1,r-1}\,,\\
i=0\,, \ j=r\,;
\label{eq36}
\end{multline}

%\vspace*{-12pt}

\noindent
\begin{multline}
\left(\lambda+\mu+\lambda^-\right)p_{ij}=\lambda p_{i-1,j} + \mu p_{i+1,j} +
 \lambda^- p_{i+1,j-1}\,,\\  
 i=1,\dots , k-1\,, \ j=1,\dots ,r-1\,;
\label{eq37}
\end{multline}

\vspace*{-12pt}

\noindent
\begin{multline}
\left(\lambda+\mu+\lambda^-\right)p_{ir}=\lambda p_{i-1,r}+ \lambda^- p_{i+1,r-1}\\ +
\left(\mu + \lambda^-\right) p_{i+1,r}\,,\  
i=1 ,\dots , k-1\,, \ j=r\,,
\label{eq38}
\end{multline}
%%%%%%%%%%%%%%%%
with the normalization condition
\begin{equation*}
p_{0} + \sum\limits_{i=0}^k \sum\limits_{j=0}^r p_{ij} = 1\,.
%\label{eq39}
\end{equation*}

Although for the sake of clarity further the case $k=r$
and $r \ge 6$ will be considered, there appears nothing new (but only technical details) 
in the general case $0<k<\infty$, $0<r<\infty$.

One of the key relations that allows the computation
of stationary joint probability distribution is the one which shows how
probability of empty system~$p_0$ depends on boundary probabilities
$\{ p_{0j}, \ j=1,\dots ,r \}$, $\{ p_{i0}, \ i=1,\dots ,r-1 \}$,
$\{ p_{ir}, \ i=1,\dots ,r-1 \}$, and $\{ p_{rj}, \ j=0 ,\dots , r \}$.
It is shown in~\cite{aaa6} using rate-in-rate-out principle
that for probability~$p_0$, it holds
\begin{equation}
p_0 = 1- \fr{ (1-q^{r+1} )\rho }{(1-\rho + \lambda^-/\mu) q^{r+1}}
 \sum\limits_{j=0}^{r}p_{rj} 
\label{eq54}
\end{equation}
where here and henceforth $\rho=\lambda/\mu$, $q=\lambda/(\mu+\lambda^-)$.
 Thus, for the considered QS, the probability~$p_0$, 
 besides initial parameters $r$, $\lambda$, $\lambda^-$, and $\mu$, depends only on the boundary 
probabilities $\{ p_{rj}, \ j=0 ,\dots , r \}$.
In \cite{aaa6}, it is shown how it is possible to find such values~$X_{rj}$ 
that $p_{rj}=X_{rj} p_0$ and how then, using~(\ref{eq54}), one can compute 
the probability~$p_0$ and, consequently, the whole joint probability distribution. 
In the next section, the author follows the argumentation from~\cite{aaa6}
and suggests modifications for several steps that lead to approximate solution. 

\section{Approximation of~Joint Stationary Probability Distribution}

\noindent
Introduce double probability generating function (PGF)
\begin{equation*}
P(u,v) = \sum\limits_{i=0}^r \sum\limits_{j=0}^r p_{ij} u^i v^j\,,
\ \ 0 \le u \le 1\,, \ \ 0 \le v \le 1\,.
\end{equation*}
Multiplying~(\ref{eq31})--(\ref{eq38}) by $u^i v^j$ and summing over all
values of~$i$ and~$j$, having collected common terms, one obtains
\begin{multline*}
- \left[\lambda u^2 -\left(\lambda +\mu +\lambda^-\right) u + \mu + \lambda^- v \right] P(u,v)
\\
{}=\fr{\mu u (v-1)}{v}\, p_{00} + \lambda u^{r+1} (1-u) \sum\limits_{j=0}^{r} p_{rj} v^j
\end{multline*}

\noindent
\begin{multline}
{}+ \fr{(\mu + \lambda^- v)(u-v)}{v} \sum\limits_{j=0}^{r} p_{0j} v^j{}\\
+ \lambda^- v^r (1-v) \sum\limits_{i=1}^{r} p_{ir} u^i\,.
\label{eq519}
\end{multline}
The expression in the square brackets in the left part of the Eq.~(\ref{eq519})
is a~polynomial of a second degree in~$u$.
Its roots have the form: 
\begin{multline*}
u_{1,2} = u_{1,2} (v)\\
{}= \fr{\lambda + \mu + \lambda^- \mp
\sqrt{(\lambda + \mu + \lambda^-)^2 -
4\lambda (\mu + \lambda^- v)}}{2 \lambda}\,.
\end{multline*}

It can be seen that $u_2(v)>1$ and $0<u_1(v) \le 1$ for $0 \le v \le 1$.
Generating function $P(u,v)$ is the ratio of two polynomial functions.
For each value of~$v$, PGF $P(u,v)$ is the continuous function of~$u$
on the whole set~$\mathbf{R}$ of real numbers.
Then, as left part in~(\ref{eq519})
vanishes at points $(u_1(v),v)$ and $(u_2(v),v)$,
the right part must vanish at these points too. Therefore, one obtains two equations:
\begin{multline}
%\label{(5-1n)}
\label{eq520}
\fr{\mu u_1 (v-1)}{v}\, p_{00} +
\lambda u_1^{r+1} (1-u_1) \sum\limits_{j=0}^{r} p_{rj} v^j 
{}\\
{}+
\fr{(\mu + \lambda^- v)(u_1-v)}{v} \sum\limits_{j=0}^{r} p_{0j} v^j \\
{}+
\lambda^- v^r (1-v) \sum\limits_{i=1}^{r} p_{ir} u_1^i = 0\,;
\end{multline}

\vspace*{-12pt}

\noindent
\begin{multline}
%\label{(5-2n)}
\label{eq521}
\fr{\mu u_2 (v-1)}{v}\, p_{00} + \lambda u_2^{r+1} (1-u_2) \sum\limits_{j=0}^{r} p_{rj} v^j
 \\
{}+ \fr{(\mu + \lambda^- v)(u_2-v)}{v} \sum\limits_{j=0}^{r} p_{0j} v^j\\
{} +
\lambda^- v^r (1-v) \sum\limits_{i=1}^{r} p_{ir} u_2^i =0\,.
\end{multline}
If one now expresses the term with~$p_{00}$ from~(\ref{eq520})
and put it into~(\ref{eq521}), after collecting common terms,
one gets the following equation:
\begin{multline}
%\label{(5-3n)}
\label{eq522}
\left (
\fr{u_2^r -u_1^r }{u_2 - u_1} -
\fr{u_2^{r+1} -u_1^{r+1}}{u_2 - u_1} \right ) \sum\limits_{j=0}^{r} p_{rj} v^j +{}
\\
{}+
\fr{\lambda^- v^r (1-v)}{\lambda } \sum\limits_{i=1}^{r} p_{ir} 
\fr{u_2^{i-1} -u_1^{i-1}}{u_2 - u_1}\\
{} +\sum\limits_{j=0}^{r} p_{0j} v^j = 0\,.
\end{multline}
But if one expresses term with
$\sum_{j=0}^{r} p_{0j} v^j$ from~(\ref{eq520}) and puts it into~(\ref{eq521}), one obtains:

\noindent
\begin{multline}
\label{eq523}
%\label{(5-4n)}
\mu (1-v)p_{00} + \mu \left (
\fr{u^r_2 - u^r_1 }{u_2-u_1} -
\fr{u^{r+1}_2 - u^{r+1}_1}{u_2-u_1} \right ) \sum\limits_{j=0}^{r} p_{rj} v^j  \\
{}+ \left (
\lambda^- \fr{u^r_2 - u^r_1}{ u_2-u_1} -
\left(\lambda + \lambda^-\right)\fr{u^{r+1}_2 - u^{r+1}_1}{u_2-u_1}\right.\\
\left. +
\lambda \fr{u^{r+2}_2 - u^{r+2}_1}{u_2-u_1} \right) 
\sum\limits_{j=0}^{r} p_{rj} v^{j+1}
\\
{}+ \fr{\mu \lambda^- v^r (1-v)}{\lambda}
\sum\limits_{i=1}^{r} p_{ir} \fr{u_2^{i-1} - u_1^{i-1} }{ u_2-u_1}\\
{}- \lambda^- v^{r+1} (1-v) \sum\limits_{i=1}^{r} p_{ir} 
\fr{u_2^{i} - u_1^{i}}{u_2-u_1} + {} \\
{}+ \fr{(\lambda^-)^2 v^{r+1} (1-v)}{\lambda}
\sum\limits_{i=1}^{r} p_{ir} \fr{ u_2^{i-1} - u_1^{i-1} }{u_2-u_1} = 0\,.
\end{multline}

Though the expressions for the roots $u_{1}(v)$ and $u_{2}(v)$ are known,
use the  Lagrange interpolating polynomials ($L^1_{n}(v)$ and $L^2_{n}(v)$,
respectively), which coincide with
$u_{1}(v)$ and $u_{2}(v)$ at $n+1$ different points.
It is known (see, e.\,g.,~\cite{suetin})
that the {approximation error $u_{1}(v) \approx L^1_{n}(v)$
(and $u_{2}(v) \approx L^2_{n}(v)$)}
is minimum if the interpolation nodes $\overline{v}_k$  are
the zeros of the Chebyshev polynomials $T_{n+1}(v)$ of the first kind
of degree $n+1$, i.\,e.,

\noindent
$$
\overline{v}_k = \cos \left (
\fr{(2k-1) \pi }{2n+2}\right )\,, \ k=1,2, \dots ,n+1\,.
$$
Given that Chebyshev polynomials of the first kind are defined on $[-1,1]$
and functions $u_{1,2}(v)$ are defined
for $v \in [0,1]$, then the values of nodes must
be recalculated using linear transformation, i.\,e., the new nodes have the values

\noindent
$$
v_k = \fr{\overline{v}_k+1}{2}\,, \ k=1,2, \dots , n+1\,.
$$
Now, the expressions $(u_2^i - u_1^i)/(u_2-u_1)$, $i \ge 1$,
that enter Eqs.~(\ref{eq522}) and~(\ref{eq523}),
can be rewritten in the form:

\noindent
\begin{multline*}
%\label{(5-6n)}
\fr{u_2^i - u_1^i}{ u_2-u_1}=
\fr{u_2(v)^i - u_1(v)^i }{u_2(v)-u_1(v)}=
\fr{L^2_n(v)^i - L^1_n(v)^i}{L^2_n(v)-L^1_n(v)}\\
{}=
f_{in}(v)\,, \ i \ge 1\,, v \in [0,1]\,.
\end{multline*}
\noindent 
It is known (see, e.\,g.,~\cite{cheb})
that the function $f_{in}(v)$
that interpolates the fraction $(u_2^i - u_1^i)/(u_2-u_1)$
at $n+1$ zeros of Chebyshev polynomials of the first kind
can be written as their combination, that is,

\noindent
\begin{equation}
\label{funap}
f_{in}(v) \approx \sum\limits_{k=0}^n w_{ik} T_{k}(2v-1)
\end{equation}
where the coefficients $w_{k}$

\pagebreak

\noindent
\begin{align*}
w_{i1}&=\fr{\sum\nolimits_{j=1}^{n+1} f_{in}(v_j) T_{1}(\overline{v}_j)
}{n+1}\,; 
\\
w_{ik}&=
\fr{2 \sum\nolimits_{j=1}^{n+1} f_{in}(v_j) T_{k}(\overline{v}_j)
}{n+1}\,, \ \ k=1 ,\dots , n \,.
\end{align*}
In~\cite{aaa6}, there was found the closed-form expression
for $(u_2^i - u_1^i)/(u_2-u_1)$
which is a polynomial of integer degree $\lfloor{i/2}\rfloor$ in~$v$ with real coefficients.
{Here is the main point of modification.}
In~(\ref{funap}), the \textbf{same} number of interpolation nodes~$n$ for 
every $i>1$ in  such way that $n < \lfloor{r/2}\rfloor$. This reduces 
the degrees of the polynomials
 that need to be computed and, consequently, leads to the simplification of
 calculations (of course, at the expense of  {accuracy loss}).

Using the exact relations between the Chebyshev polynomials $T_{k}(v)$ 
and $U_{k}(v)$ of the first and the second kind, respectively, 
$T_{k}(v) = U_{k}(v) - v U_{k-1}(v)$ and noticing that $U_{-1}(v)=0$,
one can rewrite~(\ref{funap}) in the form:
\begin{multline}
\label{funap1}
f_{in}(v)\approx \sum\limits_{k=0}^n w_{ik} U_{k}(2v-1)\\
{} -
(2v-1) \sum\limits_{k=0}^{n-1} w_{i,k+1} U_{k}(2v-1)\,.
\end{multline}

Denote by $U^{(m)}_{n}(v)$ the $m$th derivative of $U_n(v)$ at point~$v$.
Derivatives $U^{(m)}_{n}(v)$ are related with Gegenbauer polynomials $C^m_n(v)$
by equality (see, e.\,g.,~\cite[p.\,186]{erdelyi})
\begin{equation}
U^{(m)}_{n}(v)= 2^m m! C_{n-m}^{m+1} (v)\,.
\label{eq47}
\end{equation}
The values of $C^m_n(v)$ at point $v=0$ are given by (see, 
e.\,g.,~\cite[p.\,175]{erdelyi}):
\begin{equation*}
%\label{eq48}
C_{n}^{m} (0)= 
\begin{cases}
    0, &\ n \ne 2l\,;  \\
   \fr{(-1)^{n/2} {\Gamma} (m + {n/2})}{({n/2})! {\Gamma} (m)}, &\ n=2l\,.
\end{cases}
%\label{(14n)}
\end{equation*}
Here, ${\Gamma} (n)$ denotes gamma function. Expanding in~(\ref{funap1}) 
functions $U_{k}(2v-1)$ in Maclaurin series and using~(\ref{eq47}), one obtains
\begin{multline*}
f_{in}(v) \approx \sum\limits_{k=0}^n w_{ik} \sum\limits_{m=0}^{k}
2^m (2v-1)^m C_{k-m}^{m+1} (0)\\
{} -
\sum\limits_{k=0}^{n-1} w_{i,k+1} \sum\limits_{m=0}^{k} 2^m (2v-1)^{m+1} 
C_{k-m}^{m+1} (0)\,.
\end{multline*}
The latter expression
is the polynomial of integer degree~$n$ in~$v$, that is,
\begin{equation}
\label{funap2}
f_{in}(v) \approx \sum\limits_{j=0}^n h_{ij} v^j
\end{equation}
where the coefficients~$h_j$ are computed using the following relations:
\begin{align*}
h_{i0}&=\! \sum\limits_{m=0}^{n-1} \!(-2)^m \sum\limits_{k=m}^{n-1} \left[w_{ik}
C_{k-m}^{m+1} (0) + w_{i,k+1} C_{k-m}^{m+1} (0)\right]\\
& \hspace*{30mm}{}+ \sum\limits_{m=0}^{n} (-2)^m w_{i,n} C_{n-m}^{m+1} (0)\,;
\\
h_{ij}&= 2^j \left( \sum\limits_{m=j}^{n} 2^m \begin{pmatrix}{m}\\{j}\end{pmatrix}
(-1)^{m-j} \sum\limits_{k=m}^n w_{ik} C_{k-m}^{m+1} (0)\right.\\
&\hspace*{20mm}{}- 2^{j-1} \sum\limits_{k=j-1}^{n-1} w_{i,k+1} C_{k-j+1}^{j} (0) 
\\
&\hspace*{-7mm}\left. {}-\sum\limits_{m=j}^{n-1}\! 2^m \begin{pmatrix}{m+1}\\{j}\end{pmatrix}
 (-1)^{m+1-j} \sum\limits_{k=m}^{n-1}\!
w_{i,k+1} C_{k-m}^{m+1} (0) \right)\!;
\\
h_{in}&= w_{in} 2^{2n-1} \left ( 2 C_{0}^{n+1} (0) -
C_{0}^{n} (0) \right)\,.
\end{align*}
Here, $\begin{pmatrix}{n}\\{k}\end{pmatrix}$ is the number of the $k$th combination from~$n$.

Now, let go back to the simplification of equalities~(\ref{eq522}) and~(\ref{eq523}).
The coefficient in front of the sum in the first term 
of~(\ref{eq522}) can be represented, with respect to~(\ref{funap2}), as
\begin{multline}
\label{(5-9n)}
\fr{u_2^r -u_1^r}{u_2 - u_1}- \fr{u_2^{r+1} -u_1^{r+1}}{u_2 - u_1}\\
{}=\sum\limits_{j=0}^n h_{rj} v^j- \sum\limits_{j=0}^n h_{r+1,j} v^j
=
\sum\limits_{j=0}^n A_j v^j 
\end{multline}
where $A_j=h_{rj} - h_{r+1,j}$. The second term in~(\ref{eq522})
can be rewritten as 
\begin{equation}
\label{(5-10n)}
\sum\limits_{i=1}^{r} p_{ir} \fr{u_2^{i-1} -u_1^{i-1} }{u_2 - u_1}
=p_{2r} + \sum\limits_{j=0}^n B_j v^j
\end{equation}
where $B_j=\sum_{i=3}^{r} p_{ir} h_{i-1,j}$. Now, if one puts~(\ref{(5-9n)})
and~(\ref{(5-10n)}) into~(\ref{eq522}) and
collects the common terms, one obtains the following equation:
\begin{multline*}
\sum\limits_{m=0}^{n}  v^m \left[ \sum\limits_{i=0}^m
p_{r,m-i} A_i + p_{0m}\right]\\
{} + \sum\limits_{m=n+1}^{r}  v^m \left[\sum_{i=0}^n
p_{r,m-i} A_i + p_{0m}\right]  \\
{}+
\sum\limits_{m=r+1}^{r+n}  v^m \left[ \sum\limits_{i=m-r}^n p_{r,m-i} A_i +
\fr{\lambda^- }{\lambda }B_{m-r}\right]\\
{} +
\fr{\lambda^-}{\lambda }\, v^r \left[p_{2r}+B_0\right]  - 
\fr{\lambda^-}{\lambda }\, v^{r+1} \left[p_{2r}+B_0\right]\\
{} -
\fr{\lambda^- }{\lambda } \sum\limits_{m=r+2}^{r+n+1} B_{m-r-1} v^{m}=0\,.
\end{multline*}
This is polynomial in a single variable~$v$ of integer degree $r+n+1$.
From the fact that it equals zero $\forall v \in [0,1]$ and,
therefore, its coefficients are all equal to zero, one obtains the following system
of algebraic equations with constant coefficients:
\begin{equation}
\label{(5-18n)}
\sum\limits_{i=0}^m p_{r,m-i} A_i + p_{0m}=0\,, \enskip 0 \le m \le n\,;
\end{equation}
\begin{equation}
\sum\limits_{i=0}^n p_{r,m-i} A_i + p_{0m}=0\,, \enskip n+1 \le m \le r-1\,;
\label{5-18na}
\end{equation}
\begin{equation}
\sum\limits_{i=0}^n p_{r,r-i} A_i + p_{0r} + \fr{\lambda^-}{\lambda }
\left[p_{2r}+B_0\right] =0\,, \enskip m=r\,; 
\label{5-18nb}
\end{equation}

\vspace*{-12pt}

\noindent
\begin{multline}
\sum\limits_{i=1}^n p_{r,r+1-i} A_i + \fr{\lambda^- }{\lambda }\,B_{1}
- \fr{\lambda^-}{\lambda } \left[p_{2r}+B_0\right] =0\,,
\\ m = r+1\,;
\label{e5-18nc}
\end{multline}
\vspace*{-12pt}

\noindent
\begin{multline}
\sum\limits_{i=m-r}^n p_{r,m-i} A_i + \fr{\lambda^-}{\lambda }B_{m-r}
- \fr{\lambda^- }{\lambda } \,B_{m-r-1} =0\,, \\
 r+2 \le m \le r+n\,;
\label{e5-18nd}
\end{multline}
\begin{equation}
-\fr{\lambda^-}{\lambda }\, B_{n}=0\,, \enskip  m = r+n+1\,. 
\label{e5-18ne}
\end{equation}

If one sums up all equations starting from $m=r+1$, one obtains
\begin{equation}
\label{conn}
\fr{\lambda^- }{\lambda } \left[p_{2r}+B_0\right] =
\sum\limits_{m=r+1}^{r+n} \sum\limits_{i=m-r}^n p_{r,m-i} A_i\,.
\end{equation}

Now, simplify relation~(\ref{eq523}) by analogy with~(\ref{eq522}).
It is straightforward to show that the coefficient in front of the sum
in the third term, with respect to~(\ref{funap2}), equals
\begin{multline}
\label{da}
\lambda^- \fr{u^r_2 - u^r_1 }{u_2-u_1} -
\left(\lambda + \lambda^-\right) \fr{u^{r+1}_2 - u^{r+1}_1}{u_2-u_1}\\
{}+
\lambda \fr{u^{r+2}_2 - u^{r+2}_1}{u_2-u_1} = \sum\limits_{j=0}^n
D_j v^j 
\end{multline}
where $D_j = \lambda^- h_{rj} - (\lambda + \lambda^-) h_{r+1,j} +
\lambda h_{r+2,j}$. The last term in~(\ref{eq523}) simply equals
\begin{equation}
\label{da2}
\sum\limits_{i=1}^{r} p_{ir} \fr{ u_2^{i} - u_1^{i} }{u_2-u_1} = p_{1r} +
 \sum\limits_{j=0}^n E_j v^j
\end{equation}
where $E_j= \sum_{i=2}^{r} p_{ir} h_{ij}$.
By putting the representations~(\ref{(5-9n)}), (\ref{(5-10n)}), (\ref{da}) 
and~(\ref{da2}) in~(\ref{eq523}), having collected the common terms, one arrives at the
following equation: 
\begin{multline*}
\mu (1-v)p_{00} + \mu \left (
\sum\limits_{m=0}^n \!v^m \sum\limits_{i=0}^{m} \!p_{r,m-i} A_i\right.\\
\left.{} +
\sum\limits_{m=n+1}^r\!\!\! v^m \!
\sum\limits_{i=0}^{n} p_{r,m-i} A_i + 
\sum\limits_{m=r+1}^{n+r}\!\! \!v^m \! \sum\limits_{i=m-r}^{n} p_{r,m-i} A_i
\right)\hspace*{-9.41606pt}
\end{multline*}

\noindent
\begin{multline*}
\hspace*{-3mm}{} + \left(
\sum\limits_{m=1}^{n+1} v^{m}\! \sum\limits_{i=0}^{m-1} p_{r,m-i-1} D_i
+\!\!\!\sum\limits_{m=n+2}^{r+1}\! \!\!v^{m} \!\sum\limits_{i=0}^{n} p_{r,m-i-1} D_i\right.\\
\left.{}+ \sum\limits_{m=r+2}^{n+r+1} v^{m}\!\! \!
\sum\limits_{i=m-r-1}^{n}\! p_{r,m-i-1} D_i\!
\right)
 +
\fr{\mu \lambda^-}{\lambda}\,v^r \left[
p_{2r}+B_0\right ]\\
+v^{r+1} \lambda^- \left (
\left[p_{2r} + B_0\right] \fr{(\lambda^- - \mu) }{\lambda}
-\left[p_{1r} + E_0\right] \right) \\
{}+ v^{r+2} \lambda^- \left (
p_{1r} + E_0 - \fr{\lambda^-}{\lambda} \left[ p_{2r} + B_0\right]
\right)\\
{}+ \fr{\mu \lambda^-}{\lambda} \sum\limits_{j=1}^n B_j v^{r+j} 
-
\fr{\mu \lambda^-  }{\lambda} \sum\limits_{j=1}^n
B_j v^{r+j+1}\\
{} + \fr{(\lambda^-)^2  }{\lambda} \sum\limits_{j=1}^n B_j v^{r+j+1}
-\fr{(\lambda^-)^2 }{\lambda} \sum\limits_{j=1}^n B_j v^{r+j+2} 
\\
{}-
\lambda^- \sum\limits_{j=1}^n  E_j v^{r+j+1} +  \lambda^- \sum\limits_{j=1}^n  E_j v^{r+j+2}
=0\,.
\end{multline*}
This is a polynomial in a single-variable~$v$ of integer degree $r+n+2$.
From the fact that it equals zero $\forall v \in [0,1]$ and,
therefore, its coefficients are all equal to zero,
one can obtain the system of algebraic equations with constant coefficients. 
Only
first (starting from the lowest degree of~$v$) $r+1$ equations which are
\begin{equation}
\label{sys2}
\mu p_{00} + \mu p_{r0} A_0=0\,, \enskip m=0\,;
\end{equation}
\begin{equation}
\label{sys2a}
-\mu p_{00} + \mu  \sum\limits_{i=0}^{1} p_{r,1-i} A_i +
p_{r0} D_0 =0\,, \enskip m=1\,; 
\end{equation}

\vspace*{-12pt}

\noindent
\begin{multline}
\label{sys2b}
\mu \sum\limits_{i=0}^{m} p_{r,m-i} A_i + \sum\limits_{i=0}^{m-1} p_{r,m-i-1} D_i
=0\,, \\ 2 \le m \le n\,;
\end{multline}

\vspace*{-12pt}

\noindent
\begin{multline}
\label{sys2c}
\mu \sum\limits_{i=0}^{n} p_{r,n+1-i} A_i
+ \sum\limits_{i=0}^{n} p_{r,n-i} D_i =0\,, \\ m=n+1\,;
\end{multline}

\vspace*{-12pt}

\noindent
\begin{multline}
\label{sys2d}
\mu \sum\limits_{i=0}^{n} p_{r,m-i} A_i +
\sum\limits_{i=0}^{n} p_{r,m-i-1} D_i =0\,, \\ n+2 \le m \le r-1\,;
\end{multline}

\vspace*{-12pt}

\noindent
\begin{multline}
\label{sys2e}
\hspace*{-4.53584pt}\mu \sum\limits_{i=0}^{n} p_{r,r-i} A_i +
\sum\limits_{i=0}^{n} p_{r,r-i-1} D_i + \fr{\mu \lambda^- }{\lambda}
\left[ p_{2r} + B_0\right] =0\,, \\ 
m=r\,,
\end{multline}

\noindent
are needed. The last term in Eqs.~(\ref{sys2e})
depends on~$p_{ir}$, $i= 2 ,\dots , r$.
But it can be replaced according to~(\ref{conn}) by the expression
that contains only
 probabilities~$p_{rj}$, $j= 0 ,\dots , r$,
thus making the whole system~(\ref{sys2})--(\ref{sys2e}) solvable (up to a constant).
Let one  now express from Eq.~(\ref{sys2}) probability~$p_{r0}$,
from the Eq.~(\ref{sys2a})~--- probability~$p_{r1}$,
from the Eq.~(\ref{sys2b})~--- probability~$p_{r2}$, etc.
Consequently, one obtains the following recursive formulas for
probabilities~$p_{rj}$, $j= 0 ,\dots , r$:

\pagebreak

\noindent
\begin{equation}
\label{fprj}
\left.
  \begin{array}{rl}
p_{r0}& = - \fr{\lambda}{\mu A_0}\, p_0 \,, \ m=0\,; \\[6pt]
p_{r1} &= -\fr{1}{\mu A_0} \left (
 \mu A_1+D_0 + \mu  A_0 \right ) p_{r0} \,, \\[6pt]
 &\hspace*{35mm} m=1\,; \\[6pt]
p_{rm} &=\displaystyle  -\fr{1}{\mu A_0} \sum\limits_{i=0}^{m-1} p_{r,m-i-1} 
(\mu A_{i+1} + D_i) \,, \\[6pt]
& \hspace*{35mm}2 \le m \le n\,; \\[6pt]
p_{r,n+1} &=\displaystyle-\fr{1}{\mu A_0} \left 
( \mu \sum\limits_{i=0}^{n-1} p_{r,n-i} A_{i+1}\right.\\[6pt]
&\left.{} +
\displaystyle\sum\limits_{i=0}^{n} p_{r,n-i} D_i \right )\,, \ m=n+1\,; \\[6pt]
p_{rm} &=\displaystyle-\fr{1}{\mu A_0} \left ( \mu \sum\limits_{i=1}^{n} p_{r,m-i} A_i\right.\\[6pt]
&\hspace*{-10mm}\left. {}  + \displaystyle\sum\limits_{i=0}^{n} p_{r,m-i-1} D_i \right )\,, \
n+2 \le m \le r-1\,; \\[6pt]
p_{rr} &=\displaystyle-\fr{1}{\mu \sum\limits_{m=0}^{n} A_{m}} \left (
\sum\limits_{i=0}^{n} p_{r,r-i-1} D_i\right.\\[6pt]
&\left.{} + \displaystyle\mu \sum\limits_{m=0}^{n-1} 
\sum\limits_{i=1}^{n-m} p_{r,r-i} A_{m+i} \right ) \,,\  m=r\,. 
  \end{array}
\right\}
\end{equation}


\noindent
Notice that here,~$A_i$ and~$D_i$
are the constants defined in~(\ref{(5-9n)}) and~(\ref{da}),
and their values depend on the quality of approximation~(\ref{funap2}).

From Eqs.~(\ref{(5-18n)})--(\ref{5-18nb}), taking into account~(\ref{conn}),
one immediately obtains
recurrence relations for probabilities $\{p_{0j}, \ j= 0 ,\dots , r \}$:
\begin{equation}
\label{ot2}
\left.
  \begin{array}{rl}
p_{0,m}&= \displaystyle- \sum\limits_{i=0}^m p_{r,m-i} A_i \,, \ 0 \le m \le n\,; \\[6pt]
p_{0,m}&= \displaystyle-  \sum\limits_{i=0}^n p_{r,m-i} A_i\,, \ n+1 \le m \le r-1\,; \\[6pt]
p_{0,r}& = \displaystyle - \sum\limits_{m=r}^{r+n} 
\sum\limits_{i=m-r}^n p_{r,m-i} A_i \,. 
  \end{array}
\right\}
\end{equation}

Now, the algorithm for the computation of the stationary
joint probability distribution~$p_{ij}$ is 
straightforward\footnote{In fact, it is the same as in~\cite{aaa6}; 
so, here, it is just repeated for the sake of completeness.}:
\begin{itemize}
 \item
 compute values $X_{rj}=p_{rj}/p_0$, $j= 0 ,\dots , r$,
 using Eqs.~(\ref{fprj});

 \item
 compute probability~$p_0$ from~(\ref{eq54}) using representation $p_{rj}=p_0X_{rj}$;

 \item
compute  probabilities~$\{p_{rj}$, $j= 0 ,\dots , r \}$,
using equation $p_{rj}=X_{rj} p_0$;

  \item
 compute probabilities $\{p_{0j}$, $j=0 ,\dots , r \}$,
 using Eqs.~(\ref{ot2});
   \item
 compute probabilities $\{p_{r-1,j}$, $j=0 ,\dots , r \}$  using Eq.~(\ref{eq34});

     \item
     compute probabilities $\{p_{i0}$, $j=r-2 ,\dots , 1 \}$  using Eq.~(\ref{eq33});

      \item
     compute probabilities $\{p_{r-2,j}$, $j= r-1 ,\dots , 1 \}$  using Eq.~(\ref{eq37});
     and

    \item
    for each $i=r-2 ,\dots , 1$, compute probabilities $\{p_{i-1,j}$, $j= r-1 ,\dots , 1\}$
     using Eq.~(\ref{eq37}) and probabilities~$p_{i,r}$ using Eq.~(\ref{eq38}).
\end{itemize}

For some important performance characteristics of the system,
one needs to perform only the first three steps of the algorithm,
For example, the probability that the arriving customer
is lost due to the full buffer equals $\pi_{1}=\sum\nolimits_{j=0}^{r}p_{rj}$.
The probability~$\pi_{2}$ that the ordinary customer
will be lost due to the full bunker equals
$\pi_{2}=\lambda^- \sum_{i=1}^{r}p_{ir} /  (\lambda (1-\pi_{1})).$
Other quantities such as the expected number and the variance of  
number of customers in the buffer and in the bunker can be computed
using the corresponding formulas given in~\cite{aaa6}.

In the next section, the author presents some numerical examples
which show how performance characteristics behave when
one varies the quality of approximation~(\ref{funap2}).

\vspace*{-6pt}

\section{Numerical Examples}

\noindent
Here, the results of numerical computations of the following stationary 
performance characteristics are presented:
the probability that the arriving customer is lost due to the full buffer~($\pi_{1}$),
probability customer displaced from buffer is lost due to the full bunker~($\pi_{2}$), 
mean waiting time in buffer ($\omega_{\mathrm{buff}}$) and bunker~($\omega_{\mathrm{bunk}}$).

Four different combinations of initial parameters are considered: 
\begin{enumerate}[(1)]
  \item $r=26$, $\lambda=7$, $\lambda^-=5$, $\mu=10$ (Table~1);
  \item $r=26$, $\lambda=12$, $\lambda^-=5$, $\mu=10$ (Table~2);
  \item $r=26$, $\lambda=20$, $\lambda^-=5$, $\mu=10$ (Table~3); and
  \item $r=26$, $\lambda=20$, $\lambda^-=12$, $\mu=10$ (Table~4).
\end{enumerate}
For each of these cases, numerical computations were carried out
using different number of interpolation nodes~$n$, specifically, $3 \le n < \lfloor r /2\rfloor$.
The results are presented in Tables~1--4.
Last line in each table states exact values of performance
characteristics, obtained using algorithm from~\cite{aaa6}.


From the results in the tables, one can see that {it is possible 
to obtain (though not always)} the accurate
values for the considered performance characteristics if
the number of interpolation points is less than
the degree of polynomials used in direct method (i.\,e., less
than~$\lfloor{r/2}\rfloor$). Numerical experiments
show that compu-\linebreak\vspace*{-12pt}

\pagebreak

%\begin{center}  %fig1

\noindent
%\noindent
{{\tablename~1}\ \ \small{Performance characteristics for $r=26$, $\lambda=7$, $\lambda^-=5$, and $\mu=10$}}
 
 \vspace*{-3pt}

{\small\begin{center}
\tabcolsep=6.8pt
\begin{tabular}{ccrcr}
\hline
$n$ & $\pi_{1}$ & \multicolumn{1}{c}{$\pi_{2}$} & $\omega_{\mathrm{buff}}$ & 
\multicolumn{1}{c}{$\omega_{\mathrm{bunk}}$}\\
\hline
3&      0.000000&  $ -0.426874$&      0.124851&   $-40.620061$\\
4&      0.000000&   $-0.427355$&  0.124894&   $-0.053369$\\
5&      0.000000&   $-0.743357$&      0.152544&   3.467104\\
6&      0.000000&   $-0.084505$&      0.094894&   1.839610\\
7&      0.000000&   0.007251&       0.086866&   0.364169\\
8&      0.000000&   $-0.000211$&      0.087518&   0.637415\\
9&      0.000000&   0.000004&       0.087500&   0.624550\\
\hline
&   0.000000&   0.000007&   0.087499&   0.624747\\
\hline
\end{tabular}
\end{center}}
%\end{table*}

%%%2


\noindent
{{\tablename~2}\ \ \small{Performance characteristics for $r=26$, $\lambda=12$, $\lambda^-=5$, and
$\mu=10$}}
 
 \vspace*{-3pt}

{\small\begin{center}
\tabcolsep=7.5pt
\begin{tabular}{ccccr}
\hline
$n$ & $\pi_{1}$ & $\pi_{2}$ & $\omega_{\mathrm{buff}}$ & 
\multicolumn{1}{c}{$\omega_{\mathrm{bunk}}$}\\
\hline
3&  0.000606&   0.166082&   0.328110    &$-14.288452$\\
4&  0.000606&   0.166201&       0.328063&   $-14.598858$\\
5&      0.000605&   0.166759&   0.327844    &$-15.622581$\\
6&      0.000606&   0.166190&   0.328068    &3.012986\\
7&      0.000606&   0.166627&   0.327896    &12.247301\\
8&      0.000606&   0.166654&   0.327885    &11.335891\\
9&      0.000606&   0.166656&   0.327884    &11.366810\\
\hline
& 0.000606& 0.166656    &   0.327884    &11.366303\\
\hline
\end{tabular}
\end{center}}
%\end{table*}

\noindent
{{\tablename~3}\ \ \small{Performance characteristics for $r=26$, $\lambda=20$, $\lambda^-=5$, 
and $\mu=10$}}
 
 \vspace*{-3pt}

{\small\begin{center}
\tabcolsep=7.1pt
\begin{tabular}{ccccr}
\hline
$n$ & $\pi_{1}$ & $\pi_{2}$ & $\omega_{\mathrm{buff}}$ & 
\multicolumn{1}{c}{$\omega_{\mathrm{bunk}}$}\\
\hline
3&  0.250106 &  0.333239 &  1.534312    & $-14.854582$ \\
4&  0.250106 &  0.333240 &      1.534311 & $ -12.811167$ \\
5&      0.250106 &  0.333239 &  1.534312    & $-19.689504 $\\
6&      0.250106 &  0.333239 &  1.534312    & 36.261106 \\
7&      0.250106 &  0.333239 &  1.534312    &$ -892.561062$ \\
8&      0.250106 &  0.333239 &  1.534312    & 10237.695931 \\
9&      0.250106 &  0.333239 &  1.534312    & 18727.217279 \\
\hline
& 0.250106 &    0.333239    &   1.534312    &18414.922135\\
\hline
\end{tabular}
\end{center}}


\noindent
{{\tablename~4}\ \ \small{Performance characteristics for $r=26$, $\lambda=20$, $\lambda^-=12$, 
and $\mu=10$}}
 
 \vspace*{-3pt}

{\small\begin{center}
\tabcolsep=7.9pt
\begin{tabular}{ccccr}
\hline
$n$ & $\pi_{1}$ & $\pi_{2}$ & $\omega_{\mathrm{buff}}$ & 
\multicolumn{1}{c}{$\omega_{\mathrm{bunk}}$}\\
\hline
3&  0.008258  & 0.495837  & 0.391757    & $-3.781687$ \\
4&  0.008258  & 0.495837 &      0.391757 &  $-3.828509$ \\
5&      0.008258 &  0.495837  & 0.391756    &$ -3.784435 $\\
6&      0.008258 &  0.495837 &  0.391757    & $-3.584416$ \\
7&      0.008258 &  0.495837 &  0.391757    & $-5.855355$ \\
8&      0.008258 &  0.495837  & 0.391757    & 12.740076 \\
9&      0.008258  & 0.495837  & 0.391757    &  26.932748  \\
\hline
& 0.008258 &    0.495826    &   0.391757    & 23.646577 \\
\hline
\end{tabular}
\end{center}}
%\end{table*}

\vspace*{10pt}


\noindent
tational time decreases substantially
with slow decrease of~$n$. Nevertheless, it remains an
open questing how to choose appropriate number of nodes~$n$
so as not to lose much in accuracy but noticeably gain in computation time
and whether the value of~$n$ is the same for all performance characteristics
independently of combination of initial parameters.



\section{Concluding Remarks}


\noindent
In this paper, the algebraic method for
approximate calculation of joint stationary probability distribution
in single server QS with Poisson flows of ordinary and negative customers,
finite buffer, and finite bunker for customers that were displaced from 
the buffer is presented.
Obtained relations were verified by comparing with the results provided by direct methods.
The method heavily relies on the opportunity to  solve the equilibrium equations recursively.
This opportunity is established by considering additional equations obtained from 
the properties of the generation function of  the joint distribution.
Though the results are presented for the special case
when the queues' capacities are equal, it is straightforward 
to generalize the method for different (finite) capacities  and heterogeneous servers.
Future research efforts are worth concentrating in two directions.
The first one is the in-depth analysis of the approximation and
the study of cases in which it works either well or bad.
The second one is the application of the proposed method for the analysis
of more complex QS which behavior 
can be represented as multidimensional Markov processes with finite-state space.

\renewcommand{\bibname}{\protect\rmfamily References}



{\small\frenchspacing
{%\baselineskip=10.8pt
\begin{thebibliography}{99}

\bibitem{new11} %1
\Aue{Gelenbe, E., P. Glynn, and K.~Sigman}. 1991. Queues with negative arrivals. 
\textit{J.~Appl. Probab.} 28:245--250.
\bibitem{z3} %8???@
\Aue{Gelenbe, E.} 1993. G-networks with instantaneous customer movement. 
\textit{J.~Appl. Probab.} 30(3):742--748.
\bibitem{new33} %2
\Aue{Harrison, P.\,G., and E.~Pitel}. 1993. 
Sojourn times in single-server queues with negative customers.
\textit{J.~Appl. Probab.} 30:943--963.
\bibitem{new3} %3
\Aue{Fourneau, J., E. Gelenbe, and R.~Suros}. 1996. 
\mbox{G-networks} with multiple classes of negative and positive customers.
\textit{Theor. Comput. Sci.} 155(1):141--156.



\bibitem{new1} %4
\Aue{Artalejo, J.\,R.} 2000. G-networks: A~versatile approach for work 
removal in queueing systems. 
\textit{Eur. J.~Oper. Res.} 126:233--249.



\bibitem{new10} %5
\Aue{Bocharov, P.\,P., and V.\,M.~Vishnevskii}. 2003. 
G-networks: Development of the theory of multiplicative networks.
\textit{Automat. Rem. Contr.} 64(5):714--739.

\bibitem{newnew} %6
Boucherie,~R.\,J., and N.~van~Dijk, eds. 2011. \textit{Queueing networks: A~fundamental approach}. 
International ser. in operations research and management science.
New York, NY: Springer. 154. 800~p.

\bibitem{tien} %7
\Aue{Van Do, T.} 2011. An initiative for a classified bibliography on G-networks.
\textit{Perform. Evaluation} 68(4):385--394.



\bibitem{suetin} %8
\Aue{Suetin, P.\,K.} 1979. \textit{Klassicheskie ortogonal'nye mnogochleny} 
[{Classical orthogonal polynomials}]. Moscow: Nauka. 416~p. 


\bibitem{ros-cas} %9
\Aue{Manzo, R., N. Cascone, and R.\,V.~Razumchik}. 2008. 
Exponential queuing system with negative customers and bunker for ousted customers.
\textit{Automat. Rem. Contr.} 69:1542--1551.

\bibitem{aaa1} %10
\Aue{Klimenok, V., and A.~Dudin}. 2012. A~BMAP/PH/$N$ queue 
with negative customers and partial protection of service. 
\textit{Commun. Stat. Simulat.} 41(7):1062--1082.

\bibitem{new8} %11
\Aue{Pechinkin, A.\,V., and R.\,V.~Razumchik}. 2012. Stationary waiting time
distribution in queueing system with negative customers and bunker
for ousted customers under Last-LIFO-LIFO service discipline.
\textit{J.~Commun. Technol. El.} 57(12):1331--1339.

\bibitem{new7} %12
\Aue{Pechinkin, A.\,V., and R.\,V.~Razumchik}. 2012. A~method for calculating stationary
queue distribution in a queuing system with flows of ordinary and negative
claims and a bunker for superseded claims. \textit{J.~Commun. Technol.
El.} 57(8):882--891.

\bibitem{aaa3} %4-1@
\Aue{Dao-Thi, T., J.~Fourneau, and M.~Tran}. 2013. Networks of 
order independent queues with signals. \textit{21st  Symposium 
(International) on Modeling, Analysis and Simulation of Computer and 
Telecommunication Systems Proceedings}. 131--140.


\bibitem{aaa2} %13
\Aue{Krishna Kumar,~B., S.~Pavai Madheswari, and S.\,R.~Anantha Lakshmi}. 2013. 
An $M$/$G$/1 Bernoulli feedback retrial queueing system with negative customers.
\textit{Oper. Res.} 13(2):187--210.


\bibitem{new9} %14
\Aue{Razumchik, R.\,V.} 2013. Statsionarnoe raspredelenie vremeni ozhidaniya 
v~sisteme obsluzhivaniya s~otritsatel'nymi zayavkami, bunkerom dlya vytesnennykh 
zayavok, raz\-lich\-ny\-mi intensivnostyami obsluzhivaniya pri discipline 
First-FIFO-FIFO [Stationary waiting time distribution in queueing system with negative customers and bunker for ousted customers 
under First-FIFO-FIFO service discipline]. 
\textit{Informatika i~ee Primenenija}~---
\textit{Inform. Appl.} 7(2):34--39.

\bibitem{aaa6} %15
\Aue{Razumchik, R.\,V.} 2014. Analysis of finite capacity queue with negative customers and bunker for ousted customers using chebyshev and gegenbauer polynomials. 
\textit{Asia Pac. J.~Oper. Res.} 31(4):1450029. 21~p.

\bibitem{aaa4} %16
\Aue{Van Do, T., D. Papp, R.~Chakka, J.~Sztrik, and J.~Wang}. 2014. 
$M$/$M$/1 retrial queue with working vacations and negative customer arrivals.
\textit{Int. J.~Adv. Intelligence Paradigms} 6(1):52--65.

\bibitem{mikl} %17
\Aue{Razumchik, R., and M.~Telek}. 2015. 
Delay analysis of a queue with re-sequencing buffer and Markov environment.
\textit{Queueing Syst.} 22~p. doi: 10.1007/S11134-015-9444-Z.


\bibitem{z1} %18
\Aue{Balsamo, S., P.\,G.~Harrison, and A.~Marin}. 2010. 
A~unifying approach to product-forms in networks with finite capacity constraints. 
\textit{SIGMETRICS} 38(1):25--36.

\bibitem{z2} %19
\Aue{Harrison, P.\,G., and A.~Marin}. 2014. Product-forms in multi-way synchronisations. 
\textit{Comput.~J.} 57(11):1693--1710.

\bibitem{avrach} %20
\Aue{Avrachenkov, K.\,E., N.\,O.~Vilchevsky, and G.\,L.~Shevljakov}. 2003.
Priority queueing with finite buffer size and randomized push-out mechanism. 
\textit{ACM  Conference (International) on Measurement and Modelling of Computer
Proceedings}. San Diego. 324--335.

\bibitem{peter} %21
\Aue{Ilyashenko, A., O. Zayats, V.~Muliukha, and L.~Laboshin}. 2014. 
Further investigations of the priority queuing system with preemptive priority 
and randomized push-out mechanism. 
\textit{Internet of things, smart spaces, and next generation networks and systems}. 
Eds.\ S.~Balandin, S.\,D.~Andreev, and  Y.~Koucheryavy. 
Lecture notes in computer science ser. 
Springer. 8638:433--443.

\bibitem{cheb} %22
\Aue{Gil, A., J. Seguram, and N.\,M.~Temme}. 2007. 
\textit{Numerical methods for special functions}. Philadelphia, PA: 
Society for Industrial and Applied Mathematics. 431~p.

\bibitem{erdelyi} %23
\Aue{Erdelyi, A., and H.~Bateman}. 1985. 
\textit{Higher transcendental functions.} II. Malabar: Robert
E.~Krieger Publishing Co. 396~p.


\end{thebibliography} } }

\end{multicols}

\vspace*{-6pt}

\hfill{\small\textit{Received October 19, 2015}}

\vspace*{-12pt}

\Contrl

\vspace*{-3pt}

\noindent
\textbf{Razumchik Rostislav V.} (b.\ 1984)~--- Candidate of Science (PhD) in 
physics and mathematics, senior scientist, Institute of Informatics Problems, 
Federal Research Center ``Computer Science and Control'' of the Russian Academy 
of Sciences, 44-2 Vavilov Str., Moscow 119333, Russian Federation; 
associate professor, Peoples' Friendship University of Russia, 
6 Miklukho-Maklaya Str., Moscow 117198, Russian Federation; rrazumchik@ipiran.ru 


%\vspace*{8pt}

%\hrule

%\vspace*{2pt}

%\hrule

\newpage

\vspace*{-24pt}



\def\tit{АЛГЕБРАИЧЕСКИЙ МЕТОД ПРИБЛИЖЕННОГО РАСЧЕТА СТАЦИОНАРНОГО РАСПРЕДЕЛЕНИЯ 
В~СИСТЕМЕ ОБСЛУЖИВАНИЯ КОНЕЧНОЙ ЕМКОСТИ С~ОТРИЦАТЕЛЬНЫМИ ЗАЯВКАМИ И~ДВУМЯ ОЧЕРЕДЯМИ}

\def\aut{Р.\,В.~Разумчик}


\def\titkol{Алгебраический метод приближенного расчета стационарного распределения 
в~системе обслуживания} % конечной емкости с~отрицательными заявками и двумя очередями}

\def\autkol{Р.\,В.~Разумчик}

%{\renewcommand{\thefootnote}{\fnsymbol{footnote}}
%\footnotetext[1]{Работа проводится при финансовой поддержке Программы
%стратегического развития Петрозаводского государственного университета в рамках
%на\-уч\-но-ис\-сле\-до\-ва\-тель\-ской деятельности.}}


\titel{\tit}{\aut}{\autkol}{\titkol}

\vspace*{-12pt}

\noindent
Институт проблем информатики Федерального исследовательского
центра <<Информатика и~управление>> Российской академии наук, Российский
университет дружбы народов, rrazumchik@ipiran.ru

\vspace*{6pt}

\def\leftfootline{\small{\textbf{\thepage}
\hfill ИНФОРМАТИКА И ЕЁ ПРИМЕНЕНИЯ\ \ \ том\ 9\ \ \ выпуск\ 4\ \ \ 2015}
}%
 \def\rightfootline{\small{ИНФОРМАТИКА И ЕЁ ПРИМЕНЕНИЯ\ \ \ том\ 9\ \ \ выпуск\ 4\ \ \ 2015
\hfill \textbf{\thepage}}}

\Abst{Рассматривается система массового обслуживания с пуассоновским потоком
обычных и~пуассоновским потоком отрицательных заявок. Для обычных заявок
имеется накопитель конечной емкости~$k$. Если обычная заявка при поступлении
застает накпитель полностью заполненным, она теряется. Отрицательная
заявка при поступлении вытесняет одну обычную заявку
из очереди в~накопителе (если он не пуст) в другую очередь
(бункер) конечной емкости~$r$, после чего покидает систему, 
не оказывая на нее никакого воздействия. 
Если в~момент вытеснения обычной заявки из накопителя бункер
полностью заполнен, обе заявки (обычная и отрицательная) 
покидают систему. В~других случаях поступления отрицательной заявки не оказывают
влияния на функционирование системы. Заявки из бункера обслуживаются
с~относительным приоритетом. Времена обслуживания заявок как из накопителя, 
так и~из бункера имеют экспоненциальное распределение
с~одинаковым параметром. Предложен алгебраический метод 
приближенного расчета совместного стационарного распределения очередей 
для случая $k\hm=r$. Представлены некоторые результаты численных экспериментов, 
показывающие достоинства и~недостатки метода.}

\KW{система обслуживания; отрицательные заявки; многочлены Гегенбауэра; 
стационарное распределение}


\DOI{10.14357/19922264150407}

\vspace*{6pt}


 \begin{multicols}{2}

\renewcommand{\bibname}{\protect\rmfamily Литература}
%\renewcommand{\bibname}{\large\protect\rm References}

{\small\frenchspacing
{%\baselineskip=10.8pt
\begin{thebibliography}{99}
\bibitem{f2} %1
\Au{Gelenbe E., Glynn P., Sigman~K.} Queues with negative arrivals~// 
J.~Appl. Probab., 1991. Vol.~28. P.~245--250.

\bibitem{f3} %2
\Au{Gelenbe E.} 
G-networks with instantaneous customer movement~// J.~Appl. Probab., 1993. 
Vol.~30. No.\,3. P.~742--748.
\bibitem{f5} %3
\Au{Harrison P.\,G., Pitel E.} 
Sojourn times in single-server queues with negative customers~// 
{J.~Appl. Probab.}, 1993. Vol.~30. P.~943--963.
\bibitem{f1} %4
\Au{Fourneau J.,  Gelenbe~E.,  Suros~R.}
G-networks with multiple classes of negative and positive customers~// 
Theor. Comput. Sci., 1996. Vol.~155. No.\,1. P.~141--156.

\bibitem{d1} %5
\Au{Artalejo J.\,R.} G-networks: A~versatile approach for work removal in queueing systems~// 
Eur. J.~Oper. Res., 2000. Vol.~126. P.~233--249.

\bibitem{d3} %6
\Au{Bocharov P.\,P., Vishnevskii~V.\,M.}
G-networks: Development of the theory of multiplicative networks~// 
Automat. Rem. Contr., 2003. Vol.~64. No.\,5. P.~714--739.

\bibitem{h3} %7
Queueing networks: A~fundamental approach~/
Eds.\ Buncherie~R.\,J., N.~Van Dijk.~--- 
International ser. in operations research and management science.~---
New York, NY, USA: Springer, 2011.  Vol.~154. 800~p.

\bibitem{h4} %8
\Au{Van Do T.} An initiative for a classified bibliography on G-networks~// 
{Perform. Evaluation}, 2011. Vol.~68. No.\,4. P.~385--394.

\bibitem{h2} %9
\Au{Суетин П.\,К.} Классические ортогональные многочлены.~---
М.: Наука, 1979. 416~с.







\bibitem{g2} %10
\Au{Manzo R., Cascone~N., Razumchik~R.\,V.}
Exponential queuing system with negative customers and bunker for ousted customers~// 
{Automat. Rem. Contr.}, 2008. 
Vol.~69. P.~1542--1551.

\bibitem{f6} %11
\Au{Klimenok  V., Dudin A.} A~BMAP/PH/$N$ queue with negative customers and partial protection of service~//
{Commun. Stat. Simul.}, 2012. Vol.~41. No.\,7. P.~1062--1082.

\bibitem{g3} %12
\Au{Pechinkin A.\,V., Razumchik~R.\,V.} Stationary waiting time
distribution in queueing system with negative customers and bunker
for ousted customers under Last-LIFO-LIFO service discipline~//
{J.~Commun. Technol. El.}, 2012. Vol.~57. No.\,12. P.~1331--1339.

\bibitem{g4} %13
\Au{Pechinkin A.\,V., Razumchik R.\,V.} A~method for calculating stationary
queue distribution in a queuing system with flows of ordinary and negative
claims and a~bunker for superseded claims // {J.~Commun. Technol.
El.}, 2012. Vol.~57. No.\,8. P.~882--891.

\bibitem{d4} %14
\Au{Dao-Thi T., Fourneau~J., Tran~M.}
Networks of order independent queues with signals~// 
21st  Symposium (International) on Modeling, Analysis and Simulation of Computer 
and Telecommunication Systems Proceedings, 2013. P.~131--140.

\bibitem{g1} %15
\Au{Krishna Kumar B., Pavai Madheswari~S.,  Anantha Lakshmi~S.\,R.}  
An $M$/$G$/1 Bernoulli feedback retrial queueing system with negative customers~// 
{Oper. Res.}, 2013. Vol.~13. No.\,2. P.~187--210.

\bibitem{g6} %16
\Au{Разумчик Р.\,В.}
Стационарное распределение времени ожидания в~системе обслуживания 
с~отрицательными заявками, бункером для вытесненных заявок, различными интенсивностями 
обслуживания при дисциплине First-FIFO-FIFO~// Информатика и~её\linebreak применения, 2013. 
Т.~7. Вып.~2. С.~34--39. 

\bibitem{g5} %17
\Au{Razumchik R.\,V.}
Analysis of finite capacity queue with negative customers and bunker for ousted customers using chebyshev and gegenbauer polynomials~//
{Asia Pac. J.~Oper. Res.}, 2014. Vol.~31. No.\,1450029. 21~p.



\bibitem{h5} %18
\Au{Van Do T., Papp D., Chakka~R.,  Sztrik~J.,  Wang~J.} 
$M$/$M$/1 retrial queue with working vacations and negative customer arrivals~// 
{Int. J.~Adv. Intelligence Paradigms}, 2014. Vol.~6. No.\,1. P.~52--65.

\bibitem{h1} %19
\Au{Razumchik R., Telek~M.}
Delay analysis of a~queue with re-sequencing buffer and Markov environment~// 
{Queueing Syst.}, 2015. 22~p. doi: 10.1007/S11134-015-9444-Z.


\bibitem{h6} %20
\Au{Balsamo S., Harrison~P.\,G.,  Marin~A.} 
A~unifying approach to product-forms in networks with finite capacity constraints~// 
{SIGMETRICS}, 2010. Vol.~38. No.\,1. P.~25--36.

\bibitem{h7} %21
\Au{Harrison P.\,G., Marin~A.} Product-forms in multi-way synchronisations~// 
{Comput.~J.}, 2014. Vol.~57. No.\,11. P.~1693--1710.
\bibitem{d2} %22
\Au{Avrachenkov K.\,E., Vilchevsky~N.\,O., Shevljakov~G.\,L.} 
Priority queueing with finite buffer size and randomized push-out mechanism~// 
ACM  Conference (International) on Measurement and Modelling of Computer
Proceedings. San Diego, 2003. P.~324--335.

\bibitem{h8} %23
\Au{Ilyashenko A., Zayats~O., Muliukha~V., Laboshin~L.}
Further investigations of the priority queuing system with preemptive priority 
and randomized push-out mechanism~// 
{Internet of things, smart spaces, and next generation networks and systems}~/
Eds.\ S.~Balandin, S.\,D.~Andreev, and  Y.~Koucheryavy.~--- 
Lecture notes in computer science ser.~--- 
Springer, 2014. Vol.~8638. P.~433--443.

\bibitem{f4} %24
\Au{Gil A., Seguram J., Temme~N.\,M.}
Numerical methods for special functions.~--- 
Philadelphia, PA, USA: Society for Industrial and Applied Mathematics, 2007. 431~p.

\bibitem{d5} %25
\Au{Erdelyi A., Bateman H.} Higher transcendental functions. II.~--- Malabar: Robert
E.~Krieger Publishing Co., 1985. 396~p.
\end{thebibliography}
} }

\end{multicols}

 \label{end\stat}

 \vspace*{-3pt}

\hfill{\small\textit{Поступила в редакцию  19.10.2015}}
%\renewcommand{\bibname}{\protect\rm Литература}
\renewcommand{\figurename}{\protect\bf Рис.}
\renewcommand{\tablename}{\protect\bf Таблица}