\def\stat{arkhipov}

\def\tit{ВАРИАНТ СОЗДАНИЯ ЛОКАЛЬНОЙ СИСТЕМЫ КООРДИНАТ 
ДЛЯ~СИНХРОНИЗАЦИИ ИЗОБРАЖЕНИЙ ВЫБРАННЫХ СНИМКОВ}

\def\titkol{Вариант создания локальной системы координат 
для~синхронизации изображений выбранных снимков}

\def\aut{О.\,П.~Архипов$^1$, П.\,О.~Архипов$^2$, И.\,И.~Сидоркин$^3$}

\def\autkol{О.\,П.~Архипов, П.\,О.~Архипов, И.\,И.~Сидоркин}

\titel{\tit}{\aut}{\autkol}{\titkol}

\index{Архипов О.\,П.}
\index{Архипов П.\,О.}
\index{Сидоркин И.\,И.}
\index{Arkhipov O.\,P.}
\index{Arkhipov P.\,O.}
\index{Sidorkin I.\,I.}


%{\renewcommand{\thefootnote}{\fnsymbol{footnote}} \footnotetext[1]
%{Работа выполнена при частичной поддержке РФФИ (проект 16-07-00272 А).}}


\renewcommand{\thefootnote}{\arabic{footnote}}
\footnotetext[1]{Орловский филиал Федерального исследовательского центра <<Информатика и~управление>> 
Российской академии наук, \mbox{arkhipov12@yandex.ru}}
\footnotetext[2]{Орловский филиал Федерального исследовательского центра <<Информатика и~управление>> 
Российской академии наук, \mbox{arpaul@mail.ru}}
\footnotetext[3]{Орловский филиал Федерального исследовательского центра <<Информатика и~управление>> 
Российской академии наук, \mbox{voronecburgsiti@mail.ru}}

\Abst{Рассмотрены проблемы сравнения пар изображений, имеющих 
искажения поворота и~сдвига сцен друг относительно друга. Разработан 
алгоритм создания локальной системы координат (ЛСК) для пар сравниваемых 
изображений.}

\KW{алгоритм; методика; локальная система координат; цветное 
изображение; синхронизация; пиксель; цветное пятно; фильтрация}

\DOI{10.14357/19922264160312} 


\vskip 12pt plus 9pt minus 6pt

\thispagestyle{headings}

\begin{multicols}{2}

\label{st\stat}

  \section{Введение}
  
  Важным этапом обработки кадров видеопотока является построение 
ЛСК для синхронизации обрабатываемых 
изображений. Под синхронизацией понимается процедура совмещения пары 
обрабатываемых кадров путем смещения одного изображения относительно 
другого для достижения совпадения одинаковых устойчивых робастных 
структур. В качестве общих робастных структур могут выступать границы 
объектов, имеющихся на полутоновых изображениях, и~центры одинаковых 
по площади цветных пятен соответствующих цветных изображений. 
В~случае необходимости сравнения пары кадров, полученных с~различных 
точек съемки либо с~отличным углом съемки, в~результате чего изображения 
оказались смещены относительно друг друга, синхронизация может стать 
единственно возможным решением для осуществления возможности 
машинного сравнения изображений. 

В~данной статье описывается процесс 
создания ЛСК для синхронизации пар 
обрабатываемых изоб\-ра\-же\-ний. Актуальность работы обусловлена 
необходимостью сравнения пар изоб\-ра\-же\-ний, которые были получены 
с~разных точек съемки, что привело к~искажениям поворота и~смещения. 

Целью данной работы является разработка варианта создания 
ЛСК для синхронизации пар изображений выбранных 
снимков. Основная идея работы состоит в~том, что для синхронизации двух 
изображений необходимо отыскать на этих изображениях робастные 
структуры, которые повторялись бы на каждом из этих изображений, а~затем 
выполнить создание ЛСК с~сохранением лишь 
общей части обрабатываемой пары изображений. Предполагается, что даже 
будучи смещенными друг относительно друга и/или повернутыми на 
произвольный угол, данные изображения, имеющие общую сов\-па\-да\-ющую 
часть, могут быть синхронизированы путем создания ЛСК
 и~преобразованием одного из изображений. Независимо от угла 
поворота и~смещения изображений, имеющих общую часть, робастные 
структуры данных изображений будут сов\-падать. 

\vspace*{-6pt}

  \section{Обзор аналогов}
  
  \vspace*{-2pt}
  
  Одними из наиболее распространенных методов определения 
геометрического рассогласования изображений являются корреляционные 
методы~[1, 2]. Данные методы позволяют рассчитать коэффициент 
корреляции для всех возможных вариантов смещения изображений друг 
относительно друга и~выбрать одно пиковое значение, которое будет 
соответствовать наибольшему совпадению двух сравниваемых изображений. 
Еще одним примером определения взаимного сдвига изображений являются 
статические методы, в~основе которых лежит процесс вычисления 
евклидовой меры взаимного рассогласования изображений~[3]. Однако 
данные методы являются весьма чувствительными к~шумам на 
изображениях, которые являются их неотъем-\linebreak\vspace*{-12pt}

\pagebreak

\noindent
лемой частью, и,~что более 
существенно, они не позволяют выполнить согласование изображений, 
имеющих искажение поворота. 
  
  Так как при съемке изображений нестационарной камерой получаемые 
изображения имеют именно искажения сдвига и~поворота, то пе\-ре\-чис\-лен\-ные 
выше методы не могут быть использованы для синхронизации таких 
изображений. В~данной статье предлагается метод, основанный на 
выявлении робастных характеристик, имеющих сходство на обоих 
обрабатываемых изображениях, который позволит выполнять 
синхронизацию изображений, подвергнутых искажениям сдвига и~поворота. 

\vspace*{-6pt}

  \section{Создание локальной системы координат 
для~синхронизации изображений выбранных снимков}

\vspace*{-2pt}
  
  Цветное изображение представляется в~виде двумерной 
последовательности пикселей вида
  \begin{multline*}
  \mathrm{Image}_i = \!\left\{\!
  \begin{matrix}
  p_{i,1,1}(\mathrm{R,G,B}), & p_{i,1,2}(\mathrm{R,G,B}), &\ldots\\ 
  \ldots  &\ldots  &\ldots\\
  p_{i,h,1}(\mathrm{R,G,B}), & p_{i,h,2}(\mathrm{R,G,B}), &\ldots\end{matrix}\right.\\ 
\hspace*{40mm}\left.\begin{matrix}\ldots, &p_{i,1,w}(\mathrm{R,G,B})\\
\ldots &\ldots\\
\ldots, &p_{i,h,w}(\mathrm{R,G,B})
  \end{matrix}\!
  \right\}\!\!,\hspace*{-0.7966pt}\\
   i\in [1, 2]\,,\enskip
  w\in [1, W_i]\,,\enskip h\in [1, H_i]\,,
%  \label{e1-ar}
  \end{multline*}
где Image$_i$~--- изображение снимка~$i$;
$p$~--- пиксели с~цветовыми координатами (R, G, B);
$W_i$ и~$H_i$~--- ширина и~высота изображения снимка~$i$ в~пикселях. 

  Для сравнения цветных изображений предлагается использовать цветные 
пятна изображений и~робастные характеристики этих изображений. Для 
этого необходимо выполнить процедуру получения полутоновых 
изображений для получения наборов робастных характеристик каждого из 
обра\-ба\-ты\-ва\-емых изображений вида

\noindent
  \begin{multline*}
  \mathrm{Im}_i ={}\\
  {}=Q\left(\varphi_{a,1}(\mathrm{Image}_i),   
\varphi_{a,2 }(\mathrm{Image}_i), 
\varphi_{a,3}(\mathrm{Image}_i)\right)\,,\\
  i\in [1, 2]\,,
%  \label{e2-ar}
  \end{multline*}
где Im$_i$~--- полутоновое изоб\-ра\-же\-ние снимка~$i$;
$Q$~--- функция объединения полутоновых преобразований;
$\varphi$~--- функция выполнения полутоновых преобразований~\cite{4-ar}.
  
  Перед выполнением сегментации цветных изоб\-ра\-же\-ний необходимо 
выполнить огрубление цветовых составляющих изображений до~256~цветов, 
что позволит получить более удобные для сегментации
изображения 
с~четким контрастированием цвето-\linebreak\vspace*{-12pt}

\columnbreak

\noindent вых пятен~\cite{5-ar}. Процедура 
аппроксимации изображений выполняется в~два этапа:
\begin{enumerate}[(1)]
\item  аппроксимация 
изображений до~4096~цветов вида

\noindent
  \begin{equation*}
  \mathrm{Img}_i=\left\{
  \Psi_{\mathrm{app}_{4096}} (\mathrm{Image}_i ,
\mathrm{Pal}_{4096})\right\}\,,\enskip i\in [1, 2]\,,
%  \label{3-ar}
  \end{equation*}
где Img$_i$~--- аппроксимированное до~4096~цветов изображение 
снимка~$i$;
$\Psi_{\mathrm{app}_{4096}}$~--- функция получения множества  
(R, G, B)-пик\-се\-лей в~результате аппроксимации к~4096~цветам;
$\mathrm{Pal}_{4096}$~--- палитра~4096~цветов;
\item 
аппроксимация изображений до~256~цветов вида

\noindent
\begin{equation*}
\mathrm{Img}_i=\left\{
\Psi_{\mathrm{app}_{256}}(\mathrm{Image}_i, 
\mathrm{Pal}_{256})\right\}\,,\enskip i\in [1, 2]\,,
%\label{e4-ar}
\end{equation*}
где $\Psi_{\mathrm{app}_{256}}$~--- функция получения множества  
(R, G, B)-пик\-се\-лей в~результате аппроксимации к~256~цветам;
Pal$_{256}$~--- палитра 256~цветов.
\end{enumerate}
  
  Полученные в~результате выполнения двух этапов аппроксимации 
изображения должны быть сегментированы с~целью формирования 
последовательности цветных пятен каждого изображения. 
Последовательность цветных пятен изображений можно представить в~виде: 
  \begin{multline*}
  \hspace*{-5pt}\Psi_{\mathrm{segm}_i}=\{\Psi_{i,j}\} = \left\{
  \begin{matrix}
  \psi_{i,1}(p_{i,1,1}),&\ldots,& \psi_{i,1}(p_{i,1,t}),\ldots\\
  \ldots&\ldots&\ldots\\
  \psi_{i,n}(p_{i,h,1}),&\ldots, &\psi_{i,n}(p_{i,h,t}),\ldots\end{matrix}\right.\\
\left.\begin{matrix}
\ldots, &\psi_{i,u}(p_{i,1,w-g}),&\ldots ,& \psi_{i,k}(p_{i,1,w})\\
  \ldots&\ldots&\ldots&\ldots\\
\ldots,& \psi_{i,j}(p_{i,h,w-d}),&\ldots ,& \psi_{i,j}(p_{i,h,w})
  \end{matrix} \right\}
%  \label{e5-ar}
  \end{multline*}
  
  \vspace*{-12pt}
  
  \noindent
  \begin{gather*}
  i\in [1,  2]\,,\enskip
  j\in [0,  J_j]\,,\enskip
  t\in [1, T_i]\,,\\
  d\in [1, D_i]\,,\enskip
  g\in [1, G_i]\,,\enskip u\in [1, U_i]\,,\\
  w\in [1, W_i]\,,\enskip h\in [1, H_i]\,,\enskip
  n\in [1, N_i]\,,\\
  N_i\leq J_i\,,\enskip U_i\leq J_i\,, \enskip T_i\leq J_i\,,\enskip
  D_i<W_i\,,
  \end{gather*}
где $\Psi_{\mathrm{segm}_i}$~--- множество сегментов изоб\-ра\-же\-ния 
снимка~$i$;
$\psi_{i,j}$~--- сегмент с~номером~$j$ цветного изоб\-ра\-же\-ния снимка~$i$;
$J_i$~--- максимальное количество цветных сегментов изоб\-ра\-же\-ния 
снимка~$i$.
  
  Полученное множество цветных пятен представляет собой 
последовательность из всех цветоразличимых сегментов обрабатываемых 
изоб\-ра\-же\-ний~[6--8]. Для осуществления синхронизации 
изображений путем определения и~сопоставления робастных структур пары 
обрабатываемых изображений необходимо выполнить фильтрацию 
полученного множества цветных пятен. Сравнительно маленькие и~большие 
по площади пятна на изоб\-ра\-же\-ни\-ях не позволяют выполнить синхронизацию 
изображений из-за того, что маленькие пятна могут с~большой вероятностью 
повторяться на паре обрабатываемых изоб\-ра\-же\-ний либо вовсе пропа-\linebreak\vspace*{-12pt}

\pagebreak

\noindent
дать, 
а~большие пятна могут оказаться на границах изоб\-ра\-же\-ний, что приведет 
к~невозможности точного определения их центров. Следовательно, 
необходимо выбрать для дальнейшего рассмотрения только цветные пятна 
средних размеров, имеющих граничные точки, полученные в~результате 
полутоновых преобразований.
  
  Средняя величина пятен определяется как суммарная величина значений 
площадей всех цветных пятен изображения, деленная на количество цветных 
пятен данного изображения. 
  
  Ввиду того что точное совпадение размеров пятен маловероятно, выбирать 
следует при фильт\-ра\-ции пятна, площадь которых будет принадлежать 
промежутку от $\mathrm{AveSize}/2$ до 3AveSize, где\linebreak AveSize~--- средний 
размер цветного пятна изоб\-ра\-же\-ния. Выбор данного интервала увеличивает 
количество цветных пятен, подлежащих рассмотрению, и~повышает 
вероятность успешной синхронизации изображений.
  
  Отфильтрованная последовательность цветных пятен обрабатываемых 
изображений может быть представлена в~виде:

\noindent
  \begin{multline*}
  \theta_{\mathrm{segm}_i}= \{\theta_{i,k}\}\subset \Psi_{\mathrm{segm}_i}:\ i\in 
[1, 2]\,,\\
  k\in [0,  K]\,,\enskip j\in [0,  J_i]\,,\enskip K_i\leq J_i\,,
%  \label{e6-ar}
  \end{multline*}
где $\theta_{\mathrm{segm}_i}$~--- множество цветных сегментов, оставшихся 
после фильтрации цветных сегментов изоб\-ра\-же\-ния снимка~$i$.

  После того как получена последовательность цветных пятен, 
удовлетворяющая всем условиям фильтрации: площадь и~наличие граничных 
точек, необходимо выполнить процедуру сравнения множеств~$\theta_{\mathrm{segm}_i}$ 
для первого и~второго изображений. Сравнение 
производится путем определения совпадения площадей цветных пятен 
и~взаимного удаления данных пятен от других на каждом изоб\-ра\-же\-нии. При 
этом, найдя цветное пятно $\theta_{1,b}:\ b\hm\leq K_1$, принадлежащее 
первому изображению, и~цветное пятно $\theta_{2,v}:\ v\hm\leq K_2$, 
площадь которого соответствует~$\theta_{1,b}$, необходимо выполнить 
проверку удаленности от всех цветных пятен на каждом изображении 
относительно данных пятен. Если $Y_{\mathrm{segm}}\hm=\emptyset$, то данная пара 
заносится в~множество как совпадающие друг с~другом сегменты. Если 
расстояния до большей части пятен, которые были занесены в~$Y_{\mathrm{segm}}$, 
совпадают, то необходимо добавить в~множество~$Y_{\mathrm{segm}}$ данную пару 
цветных пятен, иначе они будут признаны как ошибочно выбранные 
совпадающими друг с~другом. Получа\-емую последовательность  можно 
представить в~виде:
  \begin{multline*}
  Y_{\mathrm{segm}} =\{\tau_m\}\subset \theta_{\mathrm{segm}_i}:\
  i\in [1, 2]\,,\\
  m\in [0,M_i]\,,\enskip K_1\geq M_i\leq K_2\,,
%  \label{e7-ar}
  \end{multline*}
  
  \columnbreak
  
\noindent
где $Y_{\mathrm{segm}}$~--- множество цветных сегментов, выбранных в~качестве 
совпадающих на паре изображений выбранных снимков.

  Поскольку первый элемент множества~$Y_{\mathrm{segm}}$ не может быть 
проверен на удаленность от других цветных пятен, а второй и~последующий 
сравниваются только с~предыдущими элементами, то необходимо выполнить 
дополнительную проверку\linebreak уда\-лен\-ности каждого элемента 
множества~$Y_{\mathrm{segm}}$ для каждого изоб\-ра\-же\-ния, тем самым исключив 
возможные случайные ошибки. Результирующее множество сегментов, 
имеющих соответствие на паре обрабатываемых изоб\-ра\-же\-ний, может быть 
представлено в~виде:
  \begin{equation*}
  S_{\mathrm{segm}}= \{s_c\}\subset Y_{\mathrm{segm}}:\ c\in [0,C_i]\,,\ K_1\geq C_i\leq K_2\,,
%  \label{e8-ar}
  \end{equation*}
  
  \vspace*{-2pt}
  
  \noindent
где $S_{\mathrm{segm}}$~--- множество цветных сегментов, оставшихся после 
фильтрации цветных сегментов, выбранных в~качестве совпадающих на паре 
изображений выбранных снимков.

  Успешная синхронизация изображений возможна, если имеется три 
и~более сегментов, име\-ющих соответствие на паре обрабатываемых 
изоб\-ра\-же\-ний. При этом чем дальше данные цветовые\linebreak пятна будут 
располагаться друг от друга, тем выше точность синхронизации и~создания 
ЛСК. Если число соответствующих друг другу 
сегментов три и~более, то выполняется процедура создания 
ЛСК для обрабатываемой пары изображений путем 
определения угла поворота одного изображения относительно другого 
и~вычисления расстояний до краев общей области с~переносом пикселей 
каждого из изображений. Таким образом, при выполнении условия наличия 
минимум трех цветных пятен, которые совпадают на паре обрабатываемых 
изображений, за счет выполнения процедуры

\vspace*{2pt}

\noindent
  \begin{equation*}
  \mathrm{Im}\_s_i= \mathrm{Qs}_i(\mathrm{Image}, S_{\mathrm{segm}})\,,\enskip i\in [1, 
2]\,,
%  \label{e9-ar}
  \end{equation*}
  
  \vspace*{-2pt}
  
  \noindent
где Im\_s$_i$~--- изображения, преобразованные к~ЛСК;
Qs$_i$~--- процедура построения ЛСК для 
изоб\-ра\-же\-ния снимка~$i$;
получаем два новых изображения, которые будут иметь новые 
ЛСК, совпадающие на обоих изоб\-ра\-же\-ниях.
  
  На рис.~1 приведена схема алгоритма создания 
ЛСК для синхронизации изображений выбранных кадров.



  Процессы на рис.~1:
{1}~--- получение полутоновых представлений изоб\-ра\-же\-ний
Image$_1$ и~Image$_2$;
  {2}~--- аппроксимация изоб\-ра\-же\-ний Image$_1$ и~Image$_2$
  к~палитре~4096~цветов;
  {3}~--- аппроксимация изобра\-же\-ний Img$_1$ и~Img$_2$ 
к~палитре~256~цветов;
  {4}~---  сегментация изоб\-ра\-же\-ний Img$_1$ и~Img$_2$;
  {5}~--- фильт\-ра\-ция цветных сегментов $\Psi_\mathrm{segm_1}$ 
и~$\Psi_\mathrm{segm_2}$;\linebreak\vspace*{-12pt} 

  
  \pagebreak
  
  \end{multicols}
  
  \begin{figure} %fig1
\vspace*{1pt}
 \begin{center}  
\mbox{%
 \epsfxsize=140.988mm
 \epsfbox{arh-1.eps}
 }
\end{center} 
\vspace*{-9pt}
\Caption{Схема алгоритма создания ЛСК для синхронизации 
изображений выбранных кадров}
\end{figure}
  
  \begin{multicols}{2}
  
  \noindent
  {6}~--- определение соответствия цветных сегментов 
$\theta_{\mathrm{segm}_1}$ и~$\theta_{\mathrm{segm}_2}$ пары изоб\-ра\-же\-ний
Img$_1$ и~Img$_2$ вне зависимости от угла поворота и~смещения;
  {7}~--- фильт\-ра\-ция обнаруженных сегментов множества~$Y_{\mathrm{segm}}$;
  {8}~--- определение числа совпадающих пятен на паре сравниваемых 
изоб\-ра\-же\-ний и~возможности построения ЛСК;
  {9}~--- построение ЛСК;
  {10}~--- выдача ошибки построения ЛСК;
  {11}~--- завершение работы.
  
\vspace*{-6pt}

    \section{Результаты вычислительных экспериментов}
    
  Для тестирования предлагаемого варианта создания 
ЛСК для синхронизации изображений выбранных снимков была 
выбрана пара кадров, представленных на рис.~2. 
     
\begin{figure*} %fig2
\vspace*{1pt}
 \begin{center}  
\mbox{%
 \epsfxsize=156.221mm
 \epsfbox{arh-2.eps}
 }
\end{center} 
\vspace*{-9pt}
      \Caption{Изображения первого~(\textit{а}) и~второго~(\textit{а}) снимков}
      \end{figure*}
      
  После выполнения сегментации и~фильтрации сегментов оставшиеся 
сегменты на изображениях первого и~второго снимков выделены зеленым 
цветом. Для наглядности вручную были отмечены красными линиями 
и~стрелками части изображений, по которым наиболее отчетливо видно 
относительное смещение изображений (рис.~3).
  
\begin{figure*} %fig3
\vspace*{1pt}
 \begin{center}  
\mbox{%
 \epsfxsize=156.304mm
 \epsfbox{arh-3.eps}
 }
\end{center} 
\vspace*{-9pt}
  \Caption{Сегменты для синхронизации, визуализация относительного смещения 
изображений}
  \end{figure*}
  
  После выполнения вышеописанных процедур было 
обнаружено~6~совпадающих цветных сегментов, за счет которых была 
построена ЛСК и~преобразованы оба изображения к~виду, пригодному для 
сравнения. Полученные изображения пред\-став\-ле\-ны на
рис.~4,\,\textit{а} и~4,\,\textit{б}.
  
  \begin{figure*} %fig4
  \vspace*{1pt}
 \begin{center}  
\mbox{%
 \epsfxsize=162.337mm
 \epsfbox{arh-4.eps}
 }
\end{center} 
\vspace*{-9pt}
  \Caption{Преобразованные изображения первого~(\textit{а}) и~второго~(\textit{б}) снимков}
  \end{figure*}
  
  \vspace*{-12pt}
     
  \section{Заключение}
  
  В данной статье был предложен вариант создания 
ЛСК для синхронизации изображений выбранных снимков.
  %
  В результате его тестирования были получены положительные результаты 
для изображений, имеющих искажение сдвига и~поворота. 
  
{\small\frenchspacing
 {%\baselineskip=10.8pt
 \addcontentsline{toc}{section}{References}
 \begin{thebibliography}{9}


\bibitem{2-ar}
\Au{Прэтт У.} Цифровая обработка изображений~/ Пер. с~англ.~--- М.: Мир, 
1982. (Pratt~W.\,K. Digital image processing.~---Wiley-Interscience Publication, 
1978. 750~p.)
\bibitem{1-ar}
\Au{Форсайт Д., Понс Ж.} Компьютерное зрение. Современный подход.~--- 
М.: Вильямс, 2004. 928~с. 
\bibitem{3-ar}
\Au{Гонсалес Р., Вудс Р.} Цифровая обработка изображений~/ Пер. с~англ.~--- 
М.: Техносфера, 2005. 1070~с. (Gonzalez~R.\,C.,  Woods~R.\,E. Digital image 
processing.~--- Wiley-Interscience Publication, 2002. 793~p.)
\bibitem{4-ar}
\Au{Архипов О.\,П., Зыкова З.\,П.} Применение полутоновых представлений 
при анализе изменений цветных изоб\-ра\-же\-ний~// Информатика и~её 
применения, 2014. Т.~8. Вып.~3. С.~90--99.
\bibitem{5-ar}
\Au{Архипов О.\,П., Зыкова З.\,П.} Интеграция гетерогенной информации 
о~пикселях и~их цветовосприятии~// Информатика и~её применения, 2010. 
T.~4. Вып.~4. С.~14--25.
\bibitem{6-ar}
\Au{Архипов О.\,П., Зыкова З.\,П.} Функциональное описание 
индивидуального цветовосприятия~// Известия ОрелГТУ. Сер. 
Информационные системы и~технологии, 2010. №\,5. С.~5--12.
\bibitem{7-ar}
\Au{Архипов О.\,П., Зыкова З.\,П.} RGB-ха\-рак\-те\-ри\-за\-ция пространства 
цветовосприятия~// Системы и~средства информатики, 2010. Вып.~20. №\,1. 
С.~72--89.
\bibitem{8-ar}
\Au{Архипов О.\,П., Зыкова З.\,П.} Равноконтрастные градационные 
преобразования ступенчатых тоновых шкал~// Информационные системы 
и~технологии, 2011. №\,4. С.~39--46.
\end{thebibliography}

 }
 }

\end{multicols}

\vspace*{-6pt}

\hfill{\small\textit{Поступила в~редакцию 12.07.16}}

\vspace*{10pt}

%\newpage

%\vspace*{-24pt}

\hrule

\vspace*{2pt}

\hrule

%\vspace*{8pt}



\def\tit{THE OPTION TO CREATE A~LOCAL COORDINATE SYSTEM 
FOR~SYNCHRONIZATION OF~SELECTED IMAGES}

\def\titkol{The option to create a~local coordinate system 
for~synchronization of selected images}

\def\aut{O.\,P.~Arkhipov, P.\,O.~Arkhipov, and~I.\,I.~Sidorkin}

\def\autkol{O.\,P.~Arkhipov, P.\,O.~Arkhipov, and~I.\,I.~Sidorkin}

\titel{\tit}{\aut}{\autkol}{\titkol}

\vspace*{-9pt}

\noindent
Orel Branch of the 
Federal Research Center ``Computer Science and Control'' of the Russian Academy 
of Sciences, 137~Moskovskoe Sh., Orel 302025, Russian Federation


\def\leftfootline{\small{\textbf{\thepage}
\hfill INFORMATIKA I EE PRIMENENIYA~--- INFORMATICS AND
APPLICATIONS\ \ \ 2016\ \ \ volume~10\ \ \ issue\ 3}
}%
 \def\rightfootline{\small{INFORMATIKA I EE PRIMENENIYA~---
INFORMATICS AND APPLICATIONS\ \ \ 2016\ \ \ volume~10\ \ \ issue\ 3
\hfill \textbf{\thepage}}}

\vspace*{9pt}



\Abste{While comparing pairs of images, in most cases, the problem of 
misalignment of images arises in which one image is distortion of translation and 
rotation relative to another image. Such an image is quite difficult to compare in 
automatic mode. Existing methods of image pairs  synchronize a~large 
number of constraints due to which most of them are rarely used. The proposed 
option to create a~local coordinate system for synchronizing images is based on the 
analysis of color spots presented on the images. It is assumed that successful 
synchronization of two images on their total amount of colored spots is to be 
found that match on the data images. For comparison, it is suggested to use
colored spots and 
distances between spots. In order to successfully synchronize, one needs at 
least three colored spots, which would coincide in all modes of filtration. The 
experiments show acceptable results of synchronization.}

\KWE{algorithm; local coordinate system; color image; synchronization; pixel; 
colored spot; filtration}

\DOI{10.14357/19922264160312} 

\vspace*{9pt}

%\Ack
%\noindent



%\vspace*{6pt}

  \begin{multicols}{2}

\renewcommand{\bibname}{\protect\rmfamily References}
%\renewcommand{\bibname}{\large\protect\rm References}

{\small\frenchspacing
 {%\baselineskip=10.8pt
 \addcontentsline{toc}{section}{References}
 \begin{thebibliography}{9}

  
\bibitem{2-ar-1}
\Aue{Pratt, W.\,K.} 1978. \textit{Digital image processing}. Wiley-Interscience 
Publication. 750~p.
\bibitem{1-ar-1}
  \Aue{Forsyth, D.\,A., and J.~Ponce}. 2002. \textit{Computer vision: A~modern 
approach}. Prentice Hall Professional Technical Reference. 720~p.
\bibitem{3-ar-1}
\Aue{Gonzalez, R.\,C., and R.\,E.~Woods}. 2002. \textit{Digital image 
processing}. Wiley-Interscience Publication. 793~p.
  \bibitem{4-ar-1}
  \Aue{Arhipov, O.\,P., and Z.\,P.~Zykova}. 2014. Primenenie polutonovykh 
predstavleniy pri analize izmeneniy tsvetnykh izobrazheniy [The use of half-tone 
representations in the analysis of changes in color images]. \textit{Informatika i~ee 
Primeneniya~--- Inform. Appl.} 8(3):90--99.
  \bibitem{5-ar-1}
  \Aue{Arhipov, O.\,P., and Z.\,P.~Zykova}. 2010. Integratsiya geterogennoy 
informatsii o pikselyakh i~ikh tsvetovospriyatii [Integration of heterogeneous 
information about pixels and their color perception]. \textit{Informatika i~ee 
Primeneniya~--- Inform. Appl.} 4(4):14--25.
  \bibitem{6-ar-1}
  \Aue{Arhipov, O.\,P., and Z.\,P.~Zykova}. 2010. Funktsional'noe opisanie 
individual'nogo tsvetovospriyatiya [Characteristics of color perceptual space]. 
\textit{Izvestiya OrTGU. Ser. Informatsionnye sistemy i~tekhnologii} [Herald
of Oryol Technical State University. Ser. 
information systems and technologies] 5:5--12.

\pagebreak

  \bibitem{7-sr-1}
  \Aue{Arhipov, O.\,P., and Z.\,P.~Zykova}. 2010. RGB-kharakterizatsiya 
prostranstva tsvetovospriyatiya [RGB-characterization of color space]. 
\textit{Sistemy i~Sredstva Informatiki~--- Systems and Means of Informatics} 
1(20):\linebreak 72--89.
  \bibitem{8-ar-1}
  \Aue{Arhipov, O.\,P., and Z.\,P.~Zykova}. 2011. Ravnokontrastnye 
gradatsionnye preobrazovaniya stupenchatykh tonovykh shkal [Equal contrast 
graded transformation of step tinted scales]. \textit{Informatsionnye Sistemy 
i~Tekhnologii} [Information Systems and Technologies] 4:39--46.
\end{thebibliography}

 }
 }

\end{multicols}

\vspace*{-6pt}

\hfill{\small\textit{Received July 12, 2016}}

\vspace*{-3pt}


\Contr

\noindent
\textbf{Arkhipov Oleg P.}\ (b.\ 1948)~--- Candidate of Science (PhD) in technology, Director, Oryol Branch of  
Federal Research Center  ``Computer Science  and Control'' of the 
Russian Academy of Sciences, 137~Moskovskoe Sh., Oryol 
302025, Russian Federation; \mbox{arkhipov12@yandex.ru}

\vspace*{4pt}

\noindent
\textbf{Arkhipov Pavel O.}\ (b.\ 1979)~--- Candidate of Science (PhD) in technology, senior scientist, Oryol Branch 
of  Federal Research Center  ``Computer Science  and Control'' of the Russian 
Academy of Sciences, 137~Moskovskoe Sh., Oryol 
302025, Russian Federation; \mbox{arpaul@mail.ru}

\vspace*{4pt}

\noindent
\textbf{Sidorkin Ivan I.}\ (b.\ 1990)~--- engineer-researcher, Orel Branch of the 
Federal Research Center ``Computer Science and Control'' of the Russian Academy 
of Sciences, 137~Moskovskoe Sh., Orel 302025, Russian Federation; 
\mbox{voronecburgsiti@mail.ru}
  \label{end\stat}
  
  
  \renewcommand{\bibname}{\protect\rm Литература}