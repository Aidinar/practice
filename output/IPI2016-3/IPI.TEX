\documentclass[10pt]{book}
\usepackage[utf8]{inputenc}

\usepackage{latexsym,amssymb,amsfonts,amsmath,indentfirst,shapepar,%fleqn,%
picinpar,shadow,floatflt,enumerate,multicol,colortbl,moreverb,ipi}

\usepackage{rotating}
\usepackage{mathrsfs}
\usepackage[noend]{algorithmic}
\usepackage{ulem}
%\usepackage{graphicx}
%\usepackage{algorithm2e}

\input{epsf}

%\nofiles

%\includeonly{avtor} %+pdf
%\includeonly{obchak,avtor}
%\includeonly{pred}      %
%\includeonly{podgot-rus,podgot-eng}  %+pdf
%\includeonly{ocherk} %+
%\includeonly{nekrol} %+
%\includeonly{ipi-ind}




%\includeonly{grusho} %pdf %1
%\includeonly{naumov} %2pdf
%\includeonly{fedoseev} %pdf
%\includeonly{kirikov} %pdf
%\includeonly{sinits} %pdf
%\includeonly{kudr}   %pdf
%\includeonly{krivenko} %pdf
%\includeonly{yakovlev} %pdf?
%\includeonly{shest}  %pdf
%\includeonly{kolin} %pdf
%\includeonly{ometov} %pdf
%\includeonly{arkhipov} %цветpdf
%\includeonly{zah-shest} %pdf
%\includeonly{chichagov} %pdf
%\includeonly{Leon_Ush} %pdf




%\includeonly{toc-rus, toc-en}
%\includeonly{obchak} %,toc-en}

%\includeonly{rekl}
%\includeonly{rekl-1}
%\includeonly{reshal}  %
%\includeonly{eng-index}
%\includeonly{cover3}

\usepackage{acad}
%\usepackage{courier}
\usepackage{decor}
\usepackage{newton}
\usepackage{pragmatica}
\usepackage{zapfchan}
\usepackage{petrotex}
\usepackage{bm}                     % полужирные греческие буквы
\usepackage{upgreek}                % прямые греческие буквы
\usepackage{eufrak}
\usepackage{verbatim}

\renewcommand{\bottomfraction}{0.99}
\renewcommand{\topfraction}{0.99}
\renewcommand{\textfraction}{0.01}

\setcounter{secnumdepth}{1} %здесь - 3 + chapter = 4

\arraycolsep=1.5pt

%\usepackage[pdftex]{graphicx}

%\usepackage{oz}

%NEW COMMANDS


\renewcommand*{\hm}[1]{#1\nobreak\discretionary{}%
            {\hbox{$\mathsurround=0pt #1$}}{}} %% Дублирует знаки операций
                               %при переносе в формуле (перед знаком, который
                               %надо продублировать ставится команда \hm)

%\newcommand{\endproof}{\hfill$\Box$}
%\renewcommand{\r}{\mathbb{R}}
\newcommand{\I}{{\rm I\hspace{-0.7mm}I}}
%\newcommand{\Ikl}{{\tt{1}}\hspace*{-1.44mm}\mathtt{1}}
\newcommand{\Ik}{\mbox{{\small \tt {1}}\hspace{-1.3mm}{\tt 1}}}
\newcommand{\argmin}{\mathop{\mathrm{arg}\,\mathrm{min}}}
\newcommand{\argmax}{\mathop{\mathrm{arg}\,\mathrm{max}}}
%\newcommand{\capr}{\mathop{\cap\,}}
%\newcommand{\cupr}{\mathop{\cup\,}}
%\def\argmin{\mathop{arg\,min}}

\def\vrp{\varphi}
\def\prt{\partial}
\def\mm{{\sf M}}
\def\modnop#1{\mathop{#1}\limits_{n}}
\def\eam{\mathbin{{\mathop{=}\limits^{\mathrm{def}}}}}
\def\dey#1#2{#1 (#2)}
\def\deyc#1#2{#1 \cdot  #2}
\def\ra#1{\;\mathop{\to}\limits^{#1}\;}
\def\raz#1{\;\mathop{\longrightarrow}\limits^{\!\!\!#1}\;}
\def\ral#1{\;\mathop{\longrightarrow}\limits^{#1}\;}

\newcommand{\Nor}{\mathcal{N}}
\newcommand{\T}{\mathbb{T}}
\newcommand{\Z}{\mathbb{Z}}



\newcommand{\il}[2]{\int\limits_{#1}^{#2}}%интеграл с пределами #1 и #2

\def\sm2{\mathop {\sum\limits^{n^\Theta}\sum\limits^{n^\Theta}}}
\def\sss{\sum\limits}
\def\tr{,\,\ldots\,,\,}
\def\rk{\right]}
\def\lk{\left[}
\def\rf{\right\}}
\def\lf{\left\{}
\def\lv{\,\left\vert}
\def\rv{\right\vert\,}
\def\iii{\int\limits}
\def\iin{\int\limits_{-\infty}^\infty}
\def\rrv{\right\vert}


\def\ee{{\cal E}}
\def\ww{{\cal W}}
\def\yy{{\cal Y}}
\def\vv{{\cal V}}

\newcommand{\R}{\mathbb R}
\newcommand{\E}{\mathbb E}
\newcommand{\N}{\mathbb N}

\renewcommand{\P}{\mathbb{P}}

\newcommand{\h}{{\bf H}}
\newcommand{\p}{{\sf P}}  % вероятность

\newcommand{\e}{{\sf E}}  % мат. ожидание
\newcommand{\D}{{\sf D}}  % дисперсия
\newcommand{\eps}{\varepsilon}
\newcommand{\vp}{{\mathbf p}}
\newcommand{\vz}{{\mathbf z}}
\newcommand{\vx}{{\mathbf x}}
\newcommand{\vf}{{\mathbf f}}
\newcommand{\F}{{\mathcal F}}
\def\ap{{\mathrm{ЭР}}}
\newcommand{\ud}{\Delta_n} %uniform ditance
\newcommand{\nud}{\Delta_n(x)}
\renewcommand{\Re}{\mathrm{Re}\,}

\newcommand{\abs}[1]{\left\vert#1\right\vert}

\newcommand{\norm}[1]{\left\Vert#1\right\Vert}
\def\da{(\Delta_t,A)}

\newcommand{\corr}{\mathrm{corr}}

\newcommand{\cov}{\mathrm{cov}}
\newcommand{\Expect}{\mathbb{E}}

\def\w{\omega}
\def\W{\Omega}

\def\inh{\int\limits_{nh}^{(n+1)h}}

\def\sumin{\sum_{i=1}^N}


\def\bxt{(Y,t)}
\def\xt{(y,t)}

\def\ovth{{\fr{\tau-nh}{h}}}
\def\ov{\overline}
\def\tm{\tilde m}
\def\tl{\tilde\lambda}
\def\tB{\widetilde B}
\def\tb{\tilde b}
\def\ld{\ldots}
\def\cd{\cdots}


\DeclareMathOperator{\sign}{sign}

%\newcommand{\gr}{{\geqslant}}


\newcommand{\g}{\mbox{\textit{g}}}

\renewcommand{\la}{\lambda}
\newcommand{\si}{\sigma}
\newcommand{\alp}{\alpha}

%\newcommand{\pto}{\stackrel{P}{\longrightarrow}} % сходимость по веpоятности

\newcommand{\eqd}{\stackrel{\mathrm{d}}{=}} % равенство по pаспpеделению
\newcommand{\eqdelta}{\stackrel{\Delta}{=}} % равенство по pаспpеделению

\def\be#1{\begin{equation}\label{#1}}
\def\ee{\end{equation}}
\def\re#1{(\ref{#1})}

\def\bn{\begin{enumerate}}
\def\en{\end{enumerate}}
\def\bi{\begin{itemize}}
\def\ei{\end{itemize}}
%\def\i{\item}

%\newcommand{\kp}{\kappa}
%\def\Q{{\cal Q}} \def\H{{\cal H}}
%\newcommand{\bet}{\beta_{2+\delta}}


%\newtheorem{definition}{Определение}
%\renewcommand{\thedefinition}{\arabic{definition}.}
%END NEW COMMANDS

%\renewcommand{\baselinestretch}{1.2}

%\pagestyle{myheadings}

\setlength{\textwidth}{167mm}      % 122mm
\setlength{\textheight}{658pt}
%\setlength{\textheight}{635.6pt}
\setlength{\columnsep}{4.5mm}

\setcounter{secnumdepth}{4}

%\addtolength{\headheight}{2pt}
%\addtolength{\headsep}{-2mm}

\addtolength{\topmargin}{-7mm}  % for printing


%\hoffset=-30mm  % From Yap
\hoffset=-23mm  % From Acrobat

%\voffset=0mm % From Yap
\voffset=-5mm   % From Acrobat

%\addtolength{\evensidemargin}{-2.5mm} % for printing
%\addtolength{\oddsidemargin}{2.5mm}  % for printing

\addtolength{\evensidemargin}{-12mm} % for printing
\addtolength{\oddsidemargin}{8mm}  % for printing

%\renewcommand{\thefootnote}{\fnsymbol{footnote}}
%\renewcommand{\thefootnote}{\arabic{footnote}}
\renewcommand{\figurename}{\protect\bf Рис.}
\renewcommand{\tablename}{\protect\bf Таблица}

\newcommand{\Caption}[1]{\caption{\protect\small %\baselineskip=2.5ex
#1}}

\renewcommand{\thefigure}{\arabic{figure}}
\renewcommand{\thetable}{\arabic{table}}
\renewcommand{\theequation}{\arabic{equation}}
\renewcommand{\thesection}{\arabic{section}}

\renewcommand{\contentsname}{СОДЕРЖАНИЕ}
\newcommand{\fr}[2]{\displaystyle\frac{\displaystyle #1\mathstrut}{\displaystyle #2\mathstrut}}

%\renewcommand{\thefootnote}{\fnsymbol{footnote}}
%\newcommand{\g}{\mbox{\textit{g}}}

%\newcommand{\Caption}[1]{\caption{\protect\small\baselineskip=2ex #1}}
\newcounter{razdel}
\setcounter{razdel}{0}


\newcommand{\titel}[4]{%
\

\vspace*{5pt}

\ifodd\therazdel {\raggedright\noindent\Large\textrm\textbf
 \lineskip .75em
  \baselineskip=3.2ex #1 \par}
\vskip 1em {\noindent\large\textrm\textbf #2 \par}
\addcontentsline{toc}{subsection}{{\textrm\textbf #3}\protect\newline #1}
\def\rightheadline{\underline{\noindent\hbox to \textwidth{\hfill\small\textrm{#4}
%\hfill \large\bf\thepage
}}}
\def\leftheadline{\underline{\noindent\parbox{\textwidth}{
%\raggedleft\large\bf\thepage \hfill
\small\textit{#3}\hfill}}}
\def\leftfootline{\small{\textbf{\thepage}
\hfill ИНФОРМАТИКА И ЕЁ ПРИМЕНЕНИЯ\ \ \ том~10\ \ \ выпуск 3\ \ \ 2016}
}%
 \def\rightfootline{\small{ИНФОРМАТИКА И ЕЁ ПРИМЕНЕНИЯ\ \ \ том~10\ \ \ выпуск~3\ \ \ 2016
\hfill \textbf{\thepage}}}
\vskip 2em \setcounter{figure}{0}
\setcounter{table}{0}
\setcounter{equation}{0}
\setcounter{section}{0}
\setcounter{subsection}{0}
\setcounter{subsubsection}{0}
\setcounter{footnote}{0}
\setcounter{razdel}{0}
%\end{flushleft}
\else {
 \raggedright\noindent\Large\textrm\textbf
 \lineskip .75em
\baselineskip=3.2ex #1 \par} \vskip 1em
%\begin{flushleft}
{\noindent\large\textrm\textbf #2 \par}
\addcontentsline{toc}{subsection}{{\textrm\textbf #3}\protect\newline #1}
\def\rightheadline{\underline{\noindent\hbox to \textwidth{\hfill\small\textrm{#4}
%\hfill \large\bf\thepage
}}}
\def\leftheadline{\underline{\noindent\parbox{\textwidth}{%\raggedleft\large\bf\thepage \hfill
\small\textit{#3}\hfill}}}
\def\leftfootline{\small{\textbf{\thepage}
\hfill ИНФОРМАТИКА И ЕЁ ПРИМЕНЕНИЯ\ \ \ том~10\ \ \ выпуск~3\ \ \ 2016}
}%
 \def\rightfootline{\small{ИНФОРМАТИКА И ЕЁ ПРИМЕНЕНИЯ\ \ \ том~10\ \ \ выпуск~3\ \ \ 2016
\hfill \textbf{\thepage}}} \vskip 2em \setcounter{figure}{0}
\setcounter{table}{0} \setcounter{equation}{0} \setcounter{section}{0}
\setcounter{subsection}{0} \setcounter{subsubsection}{0}
\setcounter{footnote}{0}
%\end{flushleft}
\fi}

\newcommand{\titelr}[2]{%
\

\vspace*{5pt}

\ifodd\therazdel {\raggedright\noindent%\Large\textrm\textbf
 \lineskip .75em
  \baselineskip=3.2ex #1 \par}
\vskip 1em {\noindent\normalsize\textrm\textbf #2 \par}
\else {
 \raggedright\noindent\Large\textrm\textbf
 \lineskip .75em
\baselineskip=3.2ex #1 \par} \vskip 1em
%\begin{flushleft}
{\noindent\large\textrm\textbf #2 \par
%\noindent\normalsize\textrm\textbf #2 \par
} \fi}

\newcommand{\titele}[5]{%
\

%\vspace*{5pt}

\ifodd\therazdel {\raggedright\noindent\large
\textrm\textbf
 \lineskip .75em
%  \baselineskip=3.2ex
#1 \par}
\vskip .5em {\noindent\large\textrm\textbf #2 \par}
\vskip .5em
 {\noindent\textrm #3 \par}
\addcontentsline{toc}{subsection}{{\textrm\textbf #1}\protect\newline #2}
\def\rightheadline{\underline{\noindent\hbox to \textwidth{\hfill\small\textrm{#4}
%\hfill \large\bf\thepage
}}}
\def\leftheadline{\underline{\noindent\parbox{\textwidth}{
%\raggedleft\large\bf\thepage \hfill
\small\textrm{#5}\hfill}}}
\def\leftfootline{\small{\textbf{\thepage}
\hfill ИНФОРМАТИКА И ЕЁ ПРИМЕНЕНИЯ\ \ \ том~10\ \ \ выпуск~3\ \ \ 2016}
}%
 \def\rightfootline{\small{ИНФОРМАТИКА И ЕЁ ПРИМЕНЕНИЯ\ \ \ том~10\ \ \ выпуск~3\ \ \ 2016
\hfill \textbf{\thepage}}} \vskip 1em \setcounter{figure}{0}
\setcounter{table}{0} \setcounter{equation}{0} \setcounter{section}{0}
\setcounter{subsection}{0} \setcounter{subsubsection}{0}
\setcounter{footnote}{0} \setcounter{razdel}{0}
%\end{flushleft}
\else {
 \raggedright\noindent\large
 \textrm\textbf
 \lineskip .75em
%\baselineskip=3.2ex
#1 \par} \vskip .5em
%\begin{flushleft}
{\noindent\large\textrm\textbf #2 \par} \vskip .5em
 {\noindent\textrm #3 \par}
\addcontentsline{toc}{subsection}{{\textrm\textbf #1}\protect\newline #2}
\def\rightheadline{\underline{\noindent\hbox to \textwidth{\hfill\small\textrm{#4}
%\hfill \large\bf\thepage
}}}
\def\leftheadline{\underline{\noindent\parbox{\textwidth}{%\raggedleft\large\bf\thepage \hfill
\small\textrm{#5}\hfill}}}
\def\leftfootline{\small{\textbf{\thepage}
\hfill ИНФОРМАТИКА И ЕЁ ПРИМЕНЕНИЯ\ \ \ том~10\ \ \ выпуск~3\ \ \ 2016}
}%
 \def\rightfootline{\small{ИНФОРМАТИКА И ЕЁ ПРИМЕНЕНИЯ\ \ \ том~10\ \ \ выпуск~3\ \ \ 2016
\hfill \textbf{\thepage}}} \vskip 1em \setcounter{figure}{0}
\setcounter{table}{0} \setcounter{equation}{0} \setcounter{section}{0}
\setcounter{subsection}{0} \setcounter{subsubsection}{0}
\setcounter{footnote}{0}
%\end{flushleft}
\fi}

\def\Abst#1{
\begin{center}\small\nwt
\parbox{150mm}{%\baselineskip=2.5ex
\textbf{Аннотация:}\ \
%\hspace*{\parindent}
#1}
\end{center}}
\def\Abste#1{
\begin{center}\small\nwt
\parbox{150mm}{%\baselineskip=2.5ex
\textbf{Abstract:}\ \
%\hspace*{\parindent}
#1}
\end{center}}

\def\DOI#1{
\begin{center}\small\nwt
\parbox{150mm}{%\baselineskip=2.5ex
\textbf{DOI:}\ \
%\hspace*{\parindent}
#1}
\end{center}}

\def\Abstend#1{
\begin{center}\small\nwt
\parbox{150mm}{%\baselineskip=2.5ex
%\hspace*{\parindent}
#1}
\end{center}}


\def\KW#1{
\begin{center}\small\nwt
\parbox{150mm}{%\baselineskip=2.5ex
\textbf{Ключевые слова:}\ \ #1}
\end{center}}

\def\KWE#1{
\begin{center}\small\nwt
\parbox{150mm}{%\baselineskip=2.5ex
\textbf{Keywords:}\ \ #1}
\end{center}}


\def\KWN#1{
%\begin{center}
%\small
%\parbox{150mm}\end{center}
}

\renewcommand{\thesubsection}{\thesection.\arabic{subsection}\hspace*{-5pt}}
\renewcommand{\thesubsubsection}{\thesubsection\hspace*{5pt}.\arabic{subsubsection}\hspace*{-3pt}}

\newcommand{\Ack}{\section*{\protect\rmfamily Acknowledgments}\noindent}
\newcommand{\Contr}{\section*{\protect\rmfamily Contributors}\noindent}
\newcommand{\Contrl}{\section*{\protect\rmfamily Contributor}\noindent}

\makeindex


\begin{document}
\Rus

\nwt
%\ptb


%\renewcommand{\contentsname}{\protect\Large\bf Содержание}

\setcounter{tocdepth}{2}

%\tableofcontents

\renewcommand{\bibname}{\protect\rmfamily Литература}
  \def\Au#1{{\it #1}}
    \def\Aue#1{{#1}}

%\newcommand{\No}{№}
  \newcommand{\tg}{\,\mathrm{tg}\,}
    \newcommand{\ctg}{\,\mathrm{ctg}\,}
  \newcommand{\arctg}{\,\mathrm{arctg}\,}

\def\forallb{\mathop{\forall}}
\def\cupb{\mathop{\cup}}
\def\existsb{\mathop{\exists}}


\newpage
\addtocounter{razdel}{1}
%\def\razd{РЕГУЛИРУЕМЫЙ ЭЛЕКТРОПРИВОД ДЛЯ ЭЛЕКТРОЭНЕРГЕТИКИ}


\setcounter{page}{2}

%   { %\Large  
   { %\baselineskip=16.6pt
   
   \vspace*{-48pt}
   \begin{center}\LARGE
   \textit{Предисловие}
   \end{center}
   
   %\vspace*{2.5mm}
   
   \vspace*{25mm}
   
   \thispagestyle{empty}
   
   { %\small 

    
Вниманию читателей журнала <<Информатика и её применения>> предлагается 
очередной тематический выпуск <<Вероятностно-статистические методы и 
задачи информатики и информационных технологий>>. Предыдущие тематические 
выпуски журнала по данному направлению вышли в 2008~г.\ (т.~2, вып.~2), 
в 2009~г.\ (т.~3, вып.~3) и в 2010~г.\ (т.~4, вып.~2). 

Статьи, собранные в данном журнале, посвящены разработке новых вероятностно-статистических 
методов, ориентированных на применение к решению конкретных задач информатики и информационных 
технологий, а также~--- в ряде случаев~--- и других прикладных задач. Проблематика, охватываемая 
публикуемыми работами, развивается в рамках научного сотрудничества между Институтом проблем 
информатики Российской академии наук (ИПИ РАН) и Факультетом вычислительной математики и 
кибернетики Московского государственного университета им.\ М.\,В.~Ломоносова в ходе работ 
над совместными научными проектами (в том числе в рамках функционирования 
Научно-образовательного центра <<Вероятностно-статистические методы анализа рисков>>). 
Многие из авторов статей, включенных в данный номер журнала, являются активными участниками 
традиционного международного семинара по проблемам устойчивости стохастических моделей, 
руководимого В.\,М.~Золотаревым и В.\,Ю.~Королевым; регулярные сессии этого семинара 
проводятся под эгидой МГУ и ИПИ РАН (в 2011~г.\ указанный семинар проводится в октябре 
в Калининградской области РФ). 

Наряду с представителями ИПИ РАН и МГУ в число авторов данного выпуска журнала входят 
ученые из Научно-исследовательского института системных исследований РАН, Института 
проблем технологии микроэлектроники и особочистых материалов РАН, Института 
прикладных математических исследований Карельского НЦ РАН, Московского 
авиационного института, Вологодского государственного педагогического университета, 
НИИММ им.\ Н.\,Г.~Чеботарева, Казанского государственного университета, Дебреценского 
университета (Венгрия).

Несколько статей выпуска посвящено разработке и применению стохастических методов и 
информационных технологий для решения различных прикладных задач. В~работе В.\,Г.~Ушакова 
и О.\,В.~Шестакова рассмотрена задача определения вероятностных характеристик случайных 
функций по распределениям интегральных преобразований, возникающих в задачах эмиссионной 
томографии. В~статье Д.\,О.~Яковенко и М.\,А.~Целищева рассмотрены некоторые вопросы 
математической теории риска и предложен новый подход к диверсификации инвестиционных 
портфелей. Работа И.\,А.~Кудрявцевой и А.\,В.~Пантелеева посвящена построению и 
исследованию математической модели, описывающей динамику сильноионизованной плазмы. 
В~статье П.\,П.~Кольцова изучается качество работы ряда алгоритмов сегментации изображений. 
Статья А.\,Н.~Чупрунова и И.~Фазекаша посвящена вероятностному анализу числа без\-оши\-бочных 
блоков при помехоустойчивом кодировании; получены усиленные законы больших чисел для указанных 
величин.

В данном выпуске традиционно присутствует тематика, весьма активно разрабатываемая в течение 
многих лет специалистами ИПИ РАН и МГУ,~--- методы моделирования и управления для 
информационно-телекоммуникационных и вычислительных систем, в частности методы 
теории массового обслуживания. В~статье А.\,И.~Зейфмана с соавторами рассматриваются 
модели обслуживания, описываемые марковскими цепями с непрерывным временем в случае 
наличия катастроф. В~работе М.\,М.~Лери и И.\,А.~Чеплюковой рассматриваются случайные 
графы Интернет-типа, т.\,е.\ графы, степени вершин которых имеют степенные распределения; 
такие задачи находят применение при исследовании глобальных сетей передачи данных. 
Работа Р.\,В.~Разумчика посвящена исследованию систем массового обслуживания специального 
вида~--- с отрицательными заявками и хранением вытесненных заявок.

Ряд статей посвящен развитию перспективных теоретических 
вероятностно-статистических методов, которые находят широкое применение в различных 
задачах информатики и информационных технологий. В~работе В.\,Е.~Бенинга, А.\,К.~Горшенина 
и В.\,Ю.~Королева рассмотрена задача статистической проверки гипотез о числе компонент 
смеси вероятностных распределений, приводится конструкция асимптотически наиболее мощного 
критерия. Результаты этой работы найдут применение в ряде прикладных задач, использующих 
математическую модель смеси вероятностных распределений (в информатике, моделировании 
финансовых рынков, физике турбулентной плазмы и~т.\,д.). В~статье В.\,Ю.~Королева, 
И.\,Г.~Шевцовой и С.\,Я.~Шоргина строится новая, улучшенная оценка точности нормальной 
аппроксимации для пуассоновских случайных сумм; как известно, указанные случайные суммы 
широко используются в качестве моделей многих реальных объектов, в том числе в информатике, 
физике и других прикладных областях. Работа В.\,Г.~Ушакова и Н.\,Г.~Ушакова посвящена 
исследованию ядерной оценки плотности распределения; эти результаты могут применяться, 
в част\-ности, при анализе трафика в телекоммуникационных системах. Серьезные приложения 
в статистике могут получить результаты работы О.\,В.~Шестакова, в которой доказаны оценки 
скорости сходимости распределения выборочного абсолютного медианного отклонения к нормальному 
закону. 

\smallskip

Редакционная коллегия журнала выражает надежду, что данный тематический  выпуск 
будет интересен специалистам в области теории вероятностей и математической статистики 
и их применения к решению задач информатики и информационных технологий.
     
     %\vfill 
     \vspace*{20mm}
     \noindent
     Заместитель главного редактора журнала <<Информатика и её 
применения>>,\\
     директор ИПИ РАН, академик  \hfill
     \textit{И.\,А.~Соколов}\\
     
     \noindent
     Редактор-составитель тематического выпуска,\\
     профессор кафедры математической статистики факультета\\
      вычислительной математики и кибернетики МГУ им.\ М.\,В.~Ломоносова,\\
     ведущий научный сотрудник ИПИ РАН,\\ 
доктор физико-математических наук \hfill
      \textit{В.\,Ю.~Королев}
     
     } }
     }





\def\stat{grusho}

\def\tit{АРХИТЕКТУРНЫЕ РЕШЕНИЯ В~ЗАДАЧЕ ВЫЯВЛЕНИЯ МОШЕННИЧЕСТВА ПРИ~АНАЛИЗЕ 
ИНФОРМАЦИОННЫХ ПОТОКОВ В~ЦИФРОВОЙ ЭКОНОМИКЕ$^*$}

\def\titkol{Архитектурные решения в~задаче выявления мошенничества при~анализе 
информационных потоков в
%~цифровой 
экономике}

\def\aut{А.\,А.~Грушо$^1$, М.\,И.~Забежайло$^2$, Н.\,А.~Грушо$^3$, 
Е.\,Е.~Тимонина$^4$}

\def\autkol{А.\,А.~Грушо, М.\,И.~Забежайло, Н.\,А.~Грушо, 
Е.\,Е.~Тимонина}

\titel{\tit}{\aut}{\autkol}{\titkol}

\index{Грушо А.\,А.}
\index{Забежайло М.\,И.}
\index{Грушо Н.\,А.}
\index{Тимонина Е.\,Е.}
\index{Grusho A.\,A.}
\index{Zabezhailo M.\,I.}
\index{Grusho N.\,A.}
\index{Timonina E.\,E.}


{\renewcommand{\thefootnote}{\fnsymbol{footnote}} \footnotetext[1]
{Работа частично поддержана РФФИ (проекты 18-29-03081 и~18-07-00274).}}


\renewcommand{\thefootnote}{\arabic{footnote}}
\footnotetext[1]{Институт проблем информатики Федерального исследовательского центра <<Информатика и~управление>> 
Российской академии наук, grusho@yandex.ru}
\footnotetext[2]{Институт проблем информатики Федерального исследовательского центра <<Информатика и~управление>> 
Российской академии наук, m.zabezhailo@yandex.ru}
\footnotetext[3]{Институт проблем информатики Федерального исследовательского центра <<Информатика и~управление>> 
Российской академии наук, info@itake.ru}
\footnotetext[4]{Институт проблем информатики Федерального исследовательского центра <<Информатика и~управление>> 
Российской академии наук, eltimon@yandex.ru}

\vspace*{-12pt}
   

 
  
  \Abst{Cформулирован подход к~исследованию некоторых видов мошенничества в~цифровой 
экономике с~использованием причинно-следственных связей. Во всех видах рассматриваемых 
мошенничеств должно наблюдаться несоответствие между целями финансовых транзакций 
и~реальной стоимостью достижения этих целей. Данные о транзакциях можно собирать, 
наблюдая информационные потоки, в~которых отражаются эти транзакции. Архитектура сбора 
данных и~их анализа может быть организована с~помощью распределенных реестров 
с~централизованным консенсусом, что позволяет создать аналог электронной бухгалтерской 
книги, фиксирующей финансово-экономическую деятельность субъектов цифровой экономики в~регионе. 
  Рассматриваемые методы выявления мошенничества основаны на противоречиях 
между действиями, описанными в~транзакциях, и~информацией, содержащейся в~планах, 
стандартах, прецедентах и~др. Рассмотрен метод, основанный на некоторой упрощенной схеме 
реализации абстрактного проекта. Для выявления противоречий необходимо проводить анализ 
от следствия к~причине, т.\,е.\ искать аномалии в~информации, описывающей порождение 
наблюдаемых следствий. 
  Показано, как в~реализации проекта можно выделять простые <<необходимые условия>> 
нарушения при\-чин\-но-след\-ст\-вен\-ных связей, т.\,е.\ множество <<необходимых условий>>, 
нарушение которых свидетельствует о наличии мошенничества. Это множество <<необходимых 
условий>> можно назвать метаданными для контроля проекта на выявление мошенничества.} 
 
 
  \KW{цифровая экономика; информационные потоки; при\-чин\-но-след\-ст\-вен\-ные связи; 
выявление мошеннических схем} 

\DOI{10.14357/19922264190204}
  
\vspace*{-4pt}


\vskip 10pt plus 9pt minus 6pt

\thispagestyle{headings}

\begin{multicols}{2}

\label{st\stat}

\section{Введение}

\vspace*{3pt}

  В работе сформулирован подход к~исследованию некоторых видов 
мошенничества в~цифровой экономике с~использованием  
при\-чин\-но-след\-ст\-вен\-ных связей. Рассматриваются три вида мошенничества, 
а именно:
  \begin{enumerate}[(1)]
\item отмыв денег; 
\item обман при выполнении договорных обязательств при реализации 
технических проектов (строительные проекты и~др.); 
\item незаконный вывод денег. 
\end{enumerate}

  Названные виды мошенничества могут быть сведены к~решению одного типа 
задач. Для отмывания денег источник должен заключать фиктивные контракты, 
в~соответствии с~которыми будут переводиться средства за заведомо ненужную 
работу и~материалы. 
  
  Мошенничество, связанное с~невыполнением договорных обязательств, связано 
со снижением качества услуг, качества и~количества закупаемых 
материалов, выполнением работ с~ненадлежащим качеством. 
  
  Вывод денег связан с~переводом средств фир\-мам-од\-но\-днев\-кам, которые 
заведомо не могут выполнить обязательства по контрактам, за которые им 
переводятся средства. 
  
  Таким образом, во всех трех видах рассматриваемых мошенничеств должно 
наблюдаться несоответствие между целями финансовых транзакций и~реальной 
стоимостью достижения этих целей. Данные о транзакциях можно собирать, 
наблюдая информационные потоки, в~которых отражаются эти транзакции. 
  
  Однако для наблюдения таких информационных потоков необходимо создавать 
архитектуру\linebreak телекоммуникационной системы, позволяющей перехватывать 
и~собирать данные о всех транзакциях. Например, такая архитектура может быть 
организована с~помощью распределенных реестров с~централизованным 
консенсусом, т.\,е.\ все информационные потоки, сформированные в~цифровой 
экономике и~несущие информацию о транзакциях, проходят через некоторый 
центральный узел, запоминающий их в~форме распределенного реестра. Такие 
реестры могут дублироваться в~аналогичных центрах различных регионов, что 
позволяет создать аналог электронной бухгалтерской книги, фиксирующей 
фи\-нан\-со\-во-эко\-но\-ми\-че\-скую деятельность субъектов цифровой экономики. Такой 
подход предложено реализовать на базе системы ситуационных центров, что 
отражено в~работах~[1, 2].
  
  Собранная из информационных потоков информация о~транзакциях, т.\,е.\ 
о~контрактах, договорах, платежах, отчетах, закупленных материалах, 
характеристиках исполнителей работ и~др., собирается в~базе данных в~указанном 
центре. Согласно теории интеллектуальных сис\-тем~[3], эту базу данных можно 
называть базой фактов (БФ). Базу фактов можно представить как бинарную мат\-ри\-цу, 
строки которой описывают характеристики, входящие в~транзакции, а столбцы 
нумеруются характеристиками. Строки матрицы будем называть 
\textit{объектами}~[4, 5]. 
  
  Рассматриваемые в~работе методы выявления мошенничества будут основаны 
на противоречиях между действиями, описанными в~транзакциях, и~информацией, 
содержащейся в~планах, стандартах, прецедентах и~др. Для нахождения 
противоречий в~архитектуре центра предусмотрена другая база данных~--- база 
знаний (БЗ)~\cite{3-gr, 6-gr}, которая устроена так же, как БФ. 
  
  Информация в~БЗ собирается на основе положительного опыта или расчетов. 
Используя БЗ, можно выводить факты нарушения при\-чин\-но-след\-ст\-вен\-ных 
связей. Нарушения при\-чин\-но-след\-ст\-вен\-ных связей будем называть 
\textit{аномалиями}. 
  
  Для упрощения дальнейшее изложение будет вестись в~рамках поиска 
противоречий при выполнении некоторого абстрактного проекта. Выявление 
аномалий будет происходить на основе фактов из БФ с~помощью знаний из БЗ 
методами искусственного интеллекта и~интеллектуального анализа 
данных~\cite{6-gr}. 

\vspace*{-10pt}
  
  \section{Модели}
  
  \vspace*{-3pt}
  
  Наиболее сложная из рассмотренных выше задач~--- выявление противоречий, 
т.\,е.\ использование БЗ для получения новых знаний и~выявление аномалий из 
полученных фактов. 
  
  Все способы выявления противоречий основаны на определении 
  причинно-следственных связей. При этом противоречия в~параметрах транзакций по 
отношению к~требуемым в~БЗ составляют сущность аномалий. 
  
   Далее будет рассмотрен метод, основанный на некоторой упрощенной схеме 
реализации абстрактного проекта. 
  
  Каждый проект имеет цель: например, цель представляет собой построение 
некоторой системы. Воспользуемся структурным подходом, который позволяет 
строить проект на основе разбиения системы на подсистемы и~определения 
взаимодействий подсистем~\cite{7-gr}. При этом каждая подсистема также 
представима структурной моделью. 
  
  Как сама система, так и~каждая ее подсистема имеют свой функционал 
и~спецификацию, па\-ра\-мет\-ры настройки и~домены параметров настройки. Кроме 
этих характеристик существует множество характеристик, связанных 
с~<<жизненным циклом>> создания системы. Сюда входят работы, ресурсы, 
сроки выполнения работ по созданию подсистем и~самой системы, стоимости 
компонентов и~материалов, стоимости работ, схемы поставок, договорные 
обязательства и~др. Все характеристики связаны между собой, поэтому можно 
говорить о стоимости и~времени изготовления структурных компонентов системы. 
  
  Одной из важнейших характеристик является смета (система смет для 
подсистем). Смета сопоставляет каждому компоненту системы стоимость его 
изготовления и~настройки. 
  
  Схема построения системы может быть пред\-став\-ле\-на диаграммой, 
изображенной на рис.~1. 

{ \begin{center}  %fig1
 \vspace*{9pt}
   \mbox{%
 \epsfxsize=79mm 
 \epsfbox{gru-1.eps}
 }


\vspace*{9pt}


\noindent
{{\figurename~1}\ \ \small{Диаграмма достижения цели}}
\end{center}
}

\vspace*{9pt}

\addtocounter{figure}{1}
  
  


  Представленная на рис.~1 диаграмма позволяет описать основные классы 
возможных противоречий при достижении цели. Противоречия возникают, когда 
данные БФ не соответствуют требуемым характеристикам. 
  
  
  \section{Потенциальные классы аномалий при~достижении цели}
  
  Выделим четыре потенциальных класса противоречий, которые показывают, 
каким образом нужно искать эти противоречия.
  
 
  Противоречие цели и~проекта (рис.~2) возникает при отсутствии обоснования 
или в~случае логического противоречия между возможностями проектируемого 
функционала и~целью системы. Отметим, что в~проект входят сроки, перечень 
работ, материалы, настройки, которые описываются соответствующими 
параметрами и~допустимыми значениями этих параметров. Проект формируется 
на основе БЗ и~расчетов, исходя из информации, полученной по аналогии 
с~другими проектами и~решениями, которые считаются апробированными. 
  
  Отметим, что цель порождает проект и~в этом смысле является причиной 
проекта. Однако для анализа противоречий необходимо двигаться по штриховой 
стрелке диаграммы (см.\ рис.~2) от проекта к~цели. В~самом деле, любой компонент 
проекта направлен на теоретическое достижение цели. Цель~--- сложный объект, 
поэтому в~проекте могут возникнуть характеристики, противоречащие хотя бы 
некоторым характеристикам цели. Это делает проект противоречивым, но вывод 
об этом может быть сделан только на уровне описания цели. 
  

  Противоречия между проектом и~его реализацией, исключая настройки 
(рис.~3), могут возникать, например, при закупке исполнителем материалов более 
низкого качества по более низким ценам, при попытках достижения требуемых 
сроков работы за счет снижения качества выполнения работ, за счет нахождения 
<<объективных>> причин для увеличения сроков работы и,~следовательно, 
увеличения цены реализации проекта. 


  Для выявления указанных противоречий необходимо двигаться по диаграмме 
(см.\ рис.~3) в~обратную сторону в~соответствии со~штриховыми стрелками. 
Действительно, выявить противоречия между характеристиками закупленных 
материалов и~требуемыми по проекту можно только при обращении к~проекту 
и~его спецификациям. Манипуляции со сроками работы также можно выявить 
только при обращении к~соответствующим расчетам в~проекте. Задержки в~сроках 
работы, связанные с~поставками материалов, можно определить только на 
предыдущем этапе диаграммы (см.\ рис.~3) в~описании проекта. 


  


  Противоречия между реализацией проекта и~его настройкой (рис.~4) возникает, 
когда не удается добиться требуемых значений параметров функционала, не 
удается обеспечить необходимый уровень\linebreak\vspace*{-12pt}

{ \begin{center}  %fig2
 \vspace*{-6pt}
   \mbox{%
 \epsfxsize=16mm 
 \epsfbox{gru-2.eps}
 }


\vspace*{6pt}


\noindent
{{\figurename~2}\ \ \small{Противоречия цели и~проекта}}
\end{center}
}

%\vspace*{9pt}

\addtocounter{figure}{1}

{ \begin{center}  %fig3
 \vspace*{6pt}
    \mbox{%
 \epsfxsize=79mm 
 \epsfbox{gru-3.eps}
 }


\end{center}

\vspace*{-2pt}


\noindent
{{\figurename~3}\ \ \small{Противоречия проекта и~его реализации (без настройки)}}
}

\vspace*{6pt}

\addtocounter{figure}{1}

{ \begin{center}  %fig4
 \vspace*{1pt}
   \mbox{%
 \epsfxsize=54.5mm 
 \epsfbox{gru-4.eps}
 }


\end{center}


\noindent
{{\figurename~4}\ \ \small{Противоречия реализации проекта и~его на\-стройки}}
}

%\vspace*{9pt}

\addtocounter{figure}{1}

{ \begin{center}  %fig5
 \vspace*{5pt}
    \mbox{%
 \epsfxsize=79mm 
 \epsfbox{gru-5.eps}
 }


\end{center}



\noindent
{{\figurename~5}\ \ \small{Противоречия цели и~достигнутой реализации проекта}}
}

\vspace*{6pt}

\addtocounter{figure}{1}

\noindent
 качества реализации проекта. Для 
определения противоречия в~настройках надо опять же двигаться по диаграмме 
(см.\ рис.~4) в~обратную сторону по штриховым стрелкам, так как для выявления 
характеристик результатов работы, которые не дают возможности реализации 
определенного функционала, необходимо иметь информацию о результатах этой 
работы. 


  



  Противоречие между целью и~достигнутой реализацией проекта (рис.~5) 
возникает, когда реализованная система не позволяет достичь цели. В~этом случае 
опять противоречие нужно искать, двигаясь от цели к~реальному достигнутому 
функционалу по штриховой стрелке (см.\ рис.~5).
  
  Суммируя положения, изложенные в~данном разделе, приходим к~выводу, что 
для выявления противоречий необходимо проводить анализ от следствия 
к~причине, т.\,е.\ искать аномалии в~информации, описывающей порождение 
наблюдаемых следствий. 
  
  
  \section{Связь противоречий и~причин}
  
  Прежде чем построить связь между причинами и~противоречиями, кратко 
опишем простейшую модель связи этих понятий. Причины и~противоречия будут 
сформулированы для представления компонентов системы как объектов, 
обладающих наборами известных характеристик~\cite{4-gr, 5-gr}. 
  
  Пусть $U\hm=\{\alpha, \beta, \ldots\}$~--- совокупность характеристик 
(пространство характеристик). Согласно~\cite{4-gr} \textit{объектом}~$O$ 
называется любое подмножество характеристик $O\hm\subseteq U$. Рассмотрим 
последовательность объектов, возможно в~различных пространствах 
характеристик. 
  
  \smallskip
  
  \noindent
  \textbf{Определение~1.}\ Объект~$P$ с~числом характеристик, большим или 
равным~2, является \textit{причиной} объекта (\textit{свойства})~$B$ в~цепочке 
наблюдаемых объектов тогда и~только тогда, когда выполнены следующие 
условия:
  \begin{enumerate}[(1)]
\item для каждого объекта~$C$, если $P\hm\subseteq C$, то $C\mapsto B$, где 
$C\mapsto B$ означает, что объект~$B$ присутствует в~объекте, следующем за 
объектом~$C$;
\item объект~$P$ является минимальным объектом, удовлетворяющим 
условию~1, а~именно: $\forall \alpha\hm\in P$ объект~$P\backslash \{\alpha\}$ 
не является причиной, т.\,е.\ $\exists C:\ \alpha\not\in C$, $P\backslash 
\{\alpha\}\hm\subseteq C$ и~$C\not\mapsto B$, где $C\not\mapsto B$ означает, 
что~$B$ не может содержаться в~объекте, следующем за объектом~$C$. 
\end{enumerate}

  Приведенное определение причины является упрощением причин, 
возникающих в~реальном мире. Например, реальные причины могут возникать\linebreak 
как совокупность характеристик из разных пространств. Одно следствие может 
порождаться разными причинами или возникать из внешних\linebreak и~ненаблюдаемых 
характеристик. Однако пред\-став\-лен\-ная далее формализация позволяет доступно 
изложить при\-чин\-но-след\-ст\-вен\-ные истоки противоречий, которые 
инициируют в~дальнейшем глубокое исследование рассматриваемых процессов.
  
  Будем считать, что для любого интересующего нас свойства~$B$ существует 
причина. Тогда справедлива следующая теорема.
  
  \smallskip
  
  \noindent
  \textbf{Теорема~1.}\ \textit{Для любого свойства~$B$ существует 
единственная причина}. 
  
  \smallskip
  
  \noindent
  Д\,о\,к\,а\,з\,а\,т\,е\,л\,ь\,с\,т\,в\,о\,.\ \ Доказательство будем вести от противного, 
т.\,е.\ предположим, что существуют две причины свойства~$B$: $P$ 
и~$P^\prime$, $P\hm\not= P^\prime$. Тогда существует $\alpha\hm\in U$, которое 
удовлетворяет одному из двух условий:
  \begin{itemize}
\item[(а)] $\alpha\in P$, $\alpha\notin P^\prime$;
\item[(б)] $\alpha\notin P$, $\alpha \in P^\prime$.
\end{itemize}

  Пусть выполняется условие~(б). Тогда $P^\prime\backslash \{\alpha\}$ не 
является причиной по условию~2 определения~1, т.\,е.\ $\exists C$ такое, что 
$\alpha\notin C$, $P^\prime\backslash \{\alpha\}\hm\subseteq C$ и~$C\not\mapsto B$. 
Но если~$B$ произошло и~$P$ его причина, то $C\mapsto B$, что противоречит 
предположению. Теорема~1 доказана.
  
  \smallskip
  
  \noindent
  \textbf{Лемма.} \textit{Если $P$~--- причина появления свойства~$B$, то 
объект~$B$ определяет существование свойства~$P$ в~объекте, 
предшествующем~$B$. }
  
  \smallskip
  
  \noindent
  Д\,о\,к\,а\,з\,а\,т\,е\,л\,ь\,с\,т\,в\,о\,.\ \ Из предположения, что у~каж\-до\-го 
свойства~$B$ есть причина, и~условия, что~$P$ является причиной~$B$, следует, 
что при появлении в~данных свойства~$B$ объект~$C$, предшествующий 
появлению~$B$, содержит как часть объект~$P$. Это следует из теоремы~1 
и~определения причины. 
  
  Докажем принцип <<необходимого условия>>, который, несмотря на простоту 
доказательства, будет играть в~дальнейшем существенную роль.
  
  \smallskip
  
  \noindent
  \textbf{Теорема~2.} \textit{Если~$P$~--- причина появления свойства~$B$ 
и~$A\hm\subseteq P$, то объект~$B$ определяет наличие свойства~$A$ 
в~объекте, предшествующем~$B$}. 
  
  \smallskip
  
  \noindent
  Д\,о\,к\,а\,з\,а\,т\,е\,л\,ь\,с\,т\,в\,о\,.\ \ Пусть в~данных имеется объект~$B$ 
и~$P\mapsto B$, тогда в~силу существования и~единственности причины~$B$ 
в~данных должен существовать объект~$C$, предшествующий~$B$ 
и~содержащий причину~$P$. Поскольку $A\hm\subseteq P$ и~$B$ содержит 
причину~$P$, то $B\mapsto A$. С~учетом леммы теорема~2 доказана.
  
  \smallskip
  
  Пусть даны пространства $U_1, U_2,\ldots$ и~имеется последовательность 
данных (процесс выполнения этапов проекта в~соответствии с~рис.~1) $A, B, 
\ldots$, где каждый объект является подмножеством некоторого 
пространства~$U_i$, $i\hm=1,\ldots$ Тогда в~объекте~$A$ присутствует 
причина~$P$ появления интересующего нас свойства~$C$ в~объекте~$B$. Пусть 
$P\hm\subseteq A$, тогда по теореме~2 $\forall \alpha\hm\in P$:  
$C\mapsto \{\alpha\}$, т.\,е.\ из появления~$C$ следует появление 
характеристики~$\alpha$ в~предшествующем объекте. Это необходимое условие 
того, что~$C$ удовлетворяет причинно-следственным связям развития процесса 
выполнения проекта. Если для~$C$ нет характеристики~$\alpha$, которую можно 
отнести к~причине~$C$, то можно считать, что нарушена  
при\-чин\-но-след\-ст\-вен\-ная связь и~$C$~--- аномальный объект. 
  
  \smallskip
  
  \noindent
  \textbf{Пример.} Если объект~$C$ состоит в~получении суммы~$a$ 
фирмой~$K$, то согласно теореме~2 в~пред\-шест\-ву\-ющем объекте должна 
существовать причина перевода суммы~$a$ на фирму~$K$. Если эта причина 
в~проекте отсутствует, то это можно считать признаком мошеннической схемы. 
Все проекты по предположению собираются из <<кубиков>>, содержащихся в~БЗ. 
Тогда можно сравнить цену объекта~$C$, породившего получение суммы~$a$, 
и~сумму, присутствующую в~смете проекта. Если разница велика, то это либо 
ошибка проекта, либо признак мошеннической схемы.
  
  \section{Поиск противоречий на~основе~принципа <<необходимых~условий>>}
   
  Как было показано в~разд.~3, нахождение противоречий соответствуют 
движению от следствия к~причине. Для каждого объекта в~наблюдаемых данных 
выявление причин его появления является трудоемкой задачей. Кроме того, при 
реализации контроля соблюдения при\-чин\-но-след\-ст\-вен\-ных связей на 
большом множестве участников экономической деятельности задача анализа 
причин становится трудоемкой. Поэтому процедуру контроля необходимо разбить 
на два этапа, где первый этап состоит в~анализе простых <<необходимых 
условий>> проявления мошенничества, когда используется хотя бы одна 
известная характеристика причины. Второй этап (в~режиме офлайн) состоит 
в~выявлении причин, позволяющих провести анализ источников мошеннических 
схем. 
  
  Один из подходов к~выбору <<необходимых условий>> состоит в~построении 
множества подцелей исходной цели проекта (структурный метод построения 
проекта~\cite{7-gr}). Каждая подцель описывается диаграммой на рис.~1, 
и~реализации подцелей должны образовывать полный функционал цели. Это 
является необходимым, но не достаточным условием достижения цели, так как 
при таком подходе отсутствует компонент согласования всех подцелей в~единую 
систему. Однако такой подход значительно упрощает анализ выполнения проекта 
на предмет поиска мошенничества. Если признаки мошенничества будут 
обнаружены в~реализации хотя бы одной из подцелей, то это значит, что 
мошенничество присутствует в~реализации всего проекта. 
  
  Аналогично в~реализации каждого этапа в~любой из подцелей можно выделять 
простые <<необходимые условия>> нарушения при\-чин\-но-след\-ст\-венн\-ых 
связей. 
  
  Таким образом, получается множество <<необходимых условий>>, нарушение 
которых свидетельствует о наличии мошенничества. Это множество 
<<необходимых условий>> можно назвать метаданными~[8, 9] для контроля 
проекта на выявление мошенничества. 
  
  
  \section{Заключение }
  
  В поиске противоречий необходимо от транзакций, соответствующих 
следствиям при\-чин\-но-след\-ст\-вен\-ных связей, переходить к~анализу причин 
наблюдаемых следствий. Это сложная задача, которая связана с~описанием причин 
определенных свойств. 
  
  В работе представлена модель, позволяющая строить множество необходимых 
условий соответствия наблюдаемого следствия вызвавшей его причине. Этот 
подход делает поиск противоречий вполне вычислимой задачей, но не гарантирует 
успех. 
  
  {\small\frenchspacing
 {%\baselineskip=10.8pt
 \addcontentsline{toc}{section}{References}
 \begin{thebibliography}{9}
\bibitem{1-gr}
\Au{Грушо А.\,А., Зацаринный~А.\,А., Тимонина~Е.\,Е.} Блокчейны цифровой экономики на базе 
системы ситуационных центров и~централизованного консенсуса~// Радиолокация, навигация, 
связь: Мат-лы XXV Междунар. научн.-технич. конф.~---
Воронеж: Издательский дом ВГУ, 2019. Т.~6. С.~183--191. 
\bibitem{2-gr}
\Au{Grusho A., Zatsarinny~A., Timonina~E.} A~system approach to information security in 
distributed ledgers on the situational centers platform.~---
Lecture notes in computer science ser.~--- Springer, 2019 
(in press).
\bibitem{3-gr}
\Au{Финн В.\,К.} Искусственный интеллект: Методология, применения, философия.~--- М.: 
Красанд, 2011. 448~с.

\bibitem{5-gr} %4
\Au{Аншаков~О.\,М., Фабрикантова~Е.\,Ф.} ДСМ-ме\-тод автоматического порождения 
гипотез: Логические и~эпистемологические основания.~--- М.: Либроком, 2009. 432~с.

\bibitem{4-gr} %5
\Au{Poelmans J., Elzinga~P., Viaene~S., Dedene~G.} Formal concept analysis in knowledge 
discovery: A~survey~// Conceptual structures: From information to intelligence~/ Eds.\ M.~Croitoru, 
S.~Ferr$\acute{\mbox{e}}$, and D.~Lukose.~--- Lecture notes in computer science 
ser.~--- Berlin--Heidelberg: Springer, 2010. Vol.~6208.  P.~139--153.

\bibitem{6-gr}
\Au{Панкратова~Е.\,С., Финн~В.\,К.} Автоматическое по\-рож\-де\-ние гипотез в~интеллектуальных 
системах.~--- М.: Либроком, 2009. 528~с. 
\bibitem{7-gr}
\Au{Денисов А.\,А., Колесников~Д.\,Н.} Теория больших систем управления.~--- Л.: Энергоиздат, 1982. 488~с.

\bibitem{9-gr}
\Au{Грушо А.\,А., Грушо Н.\,А., Забежайло~М.\,И., Смирнов~Д.\,В., Тимонина~Е.\,Е.} 
Параметризация в~прикладных задачах поиска эмпирических причин~// Информатика и~её 
применения, 2018. Т.~12. Вып.~3. С.~62--66.

\bibitem{8-gr}
\Au{Грушо А.\,А., Грушо Н.\,А., Левыкин~М.\,В., Тимонина~Е.\,Е.} Методы идентификации 
захвата хоста в~распределенной ин\-фор\-ма\-ци\-он\-но-вы\-чис\-ли\-тель\-ной сис\-те\-ме, 
защищенной с~помощью метаданных~// Информатика и~её применения, 2018. Т.~12. Вып.~4. 
С.~41--45.

 \end{thebibliography}

 }
 }

\end{multicols}

\vspace*{-3pt}

\hfill{\small\textit{Поступила в~редакцию 03.04.19}}

%\vspace*{8pt}

%\pagebreak

\newpage

\vspace*{-28pt}

%\hrule

%\vspace*{2pt}

%\hrule

%\vspace*{-2pt}

\def\tit{ARCHITECTURAL DECISIONS IN~THE~PROBLEM 
OF~IDENTIFICATION OF~FRAUD IN~THE~ANALYSIS 
OF~INFORMATION FLOWS IN~DIGITAL ECONOMY\\[-5pt]}


\def\titkol{Architectural decisions in~the~problem 
of~identification of~fraud in~the~analysis 
of~information flows in~digital economy}

\def\aut{A.\,A.~Grusho, M.\,I.~Zabezhailo, N.\,A.~Grusho, and~E.\,E.~Timonina}

\def\autkol{A.\,A.~Grusho, M.\,I.~Zabezhailo, N.\,A.~Grusho, and~E.\,E.~Timonina}

\titel{\tit}{\aut}{\autkol}{\titkol}

\vspace*{-13pt}


 \noindent
   Institute of Informatics Problems, Federal Research Center ``Computer Sciences and 
Control'' of the Russian Academy of Sciences; 44-2~Vavilov Str., Moscow 119133, 
Russian Federation

\def\leftfootline{\small{\textbf{\thepage}
\hfill INFORMATIKA I EE PRIMENENIYA~--- INFORMATICS AND
APPLICATIONS\ \ \ 2019\ \ \ volume~13\ \ \ issue\ 2}
}%
 \def\rightfootline{\small{INFORMATIKA I EE PRIMENENIYA~---
INFORMATICS AND APPLICATIONS\ \ \ 2019\ \ \ volume~13\ \ \ issue\ 2
\hfill \textbf{\thepage}}}

\vspace*{3pt}


   
     
   \Abste{An approach to a~research of some types of fraud in digital economy with the usage of relationships of 
cause and effect is formulated. In all types of the considered frauds, the discrepancy between the 
purposes of financial transactions and actual cost of achievement of these purposes
has to be observed. Data on 
transactions can be collected by observing information flows in which these transactions are reflected. 
The architecture of data collection and their analysis can be organized by means of the distributed 
ledgers with the centralized consensus that allows creating an analog of the electronic account book 
fixing financial and economic activity of subjects of digital economy in the region. 
   The methods of fraud identification considered are based on the contradictions 
between actions described in transactions and information, which is contained in plans, standards, 
precedents, etc. 
   The method based on a~simplified scheme of implementation of the abstract project is considered. 
For identification of contradictions, it is necessary to carry out the analysis from the effect to the cause, 
i.\,e., to look for anomalies in information describing the generation of the observed effects. 
   It is shown how in implementation of the project it is possible to allocate simple ``necessary 
conditions'' of violation of cause and effect relationships, i.\,e., a~set of ``necessary conditions'' 
violation of which demonstrates fraud existence. It is possible to call this set of "necessary conditions" 
by metadata for control of the project for fraud identification.} 
   
   \KWE{digital economy; information flows; relationships of reason and effect; detection of 
fraudulent schemes}
   
  

 \DOI{10.14357/19922264190204}

\vspace*{-20pt}

 \Ack
   \noindent
   The work was partially supported by the Russian Foundation for Basic Research (projects  
18-29-03081 and 18-07-00274).



%\vspace*{6pt}

  \begin{multicols}{2}

\renewcommand{\bibname}{\protect\rmfamily References}
%\renewcommand{\bibname}{\large\protect\rm References}

{\small\frenchspacing
 {\baselineskip=10.5pt
 \addcontentsline{toc}{section}{References}
 \begin{thebibliography}{9}
\bibitem{1-gr-1}
\Aue{Grusho, A.\,A., A.\,A.~Zatsarinny, and E.\,E.~Timonina.} 2019. Blokcheyny tsifrovoy ekonomiki 
na baze sistemy situatsionnykh tsentrov i~tsentralizovannogo konsensusa [Blockchains of digital 
economy on the basis of the system of the situational centres and the centralized consensus]. 
\textit{25th Scientific and Technical Conference (International) ``Radar-Location, Navigation, 
Communication'' Proceedings}. Voronezh: VSU Publs. 6:183--191.
\bibitem{2-gr-1}
\Aue{Grusho, A., A.~Zatsarinny, and E.~Timonina.} 2019 (in press). 
A~system approach to information security 
in distributed ledgers on the situational centers platform. 
Lecture notes in computer science ser. Springer.
\bibitem{3-gr-1}
\Aue{Finn, V.\,K.} 2011. \textit{Iskusstvennyy intellekt: Metodologiya, primeneniya, filosofiya} 
[Artificial intelligence: Methodology, applications, philosophy]. Moscow: KRASAND. 448~p.

\bibitem{5-gr-1}
\Aue{Anshakov, O.\,M., and E.\,F.~Fabrikantova}. 2009. \textit{DSM-metod avtomaticheskogo porozhdeniya gipotez: Logicheskie 
i~epistemologicheskie osnovaniya} [JSM-method of automatic hypothesis generation: Logical and 
epistemological]. Moscow: KD LIBROKOM. 432~p.
\bibitem{4-gr-1} %5
\Aue{Poelmans, J., P.~Elzinga, S.~Viaene, and G.~Dedene.} 2010. Formal concept analysis in 
knowledge discovery: A~survey. \textit{Conceptual structures: From information to intelligence}. 
Eds.\ M.~Croitoru, S.~Ferr$\acute{\mbox{e}}$, and D.~Lukose. Lecture notes in 
computer science ser. Berlin--Heidelberg: Springer. 6208:139--153.

\bibitem{6-gr-1}
\Aue{Pankratov, E.\,S., and V.\,K.~Finn}. 
2009. \textit{Avtomaticheskoe porozhdenie gipotez v~intellektual'nykh 
sistemakh} [Automatic hypotheses generation in intelligent systems]. Moscow: KD 
\mbox{LIBROKOM}.  528~p. 
\bibitem{7-gr-1}
\Aue{Denisov, A.\,A., and D.\,N.~Kolesnikov.} 1982. \textit{Teoriya bol'shikh 
sistem upravleniya} [Theory of big control systems]. Leningrad: Energoizdat. 488~p.

\bibitem{9-gr-1}
\Aue{Grusho, A.\,A., N.\,A.~Grusho, M.\,I.~Zabezhailo, D.\,V.~Smirnov, and 
E.\,E.~Timonina.} 2018. 
Parametrizatsiya v~prikladnykh zadachakh poiska empiricheskikh prichin 
[Parametrization in applied 
problems of search of the empirical reasons]. 
\textit{Informatika i~ee Primeneniya~--- 
Inform. Appl.} 12(3):62--66.

\bibitem{8-gr-1}
\Aue{Grusho, A.\,A., N.\,A.~Grusho, M.\,V.~Levykin, and E.\,E.~Timonina.} 2018. Metody 
identifikatsii zakhvata khosta v~raspredelennoy informatsionno-vychislitel'noy sisteme, 
zashchishchennoy s~pomoshch'yu metadannykh [Methods of identification of host capture 
in the  distributed information system which is protected on the base of meta data].
\textit{Informatika i~ee 
Primeneniya~--- Inform. Appl.} 12(4):41--45.
{ %\looseness=1

}

\end{thebibliography}

 }
 }

\end{multicols}

\vspace*{-12pt}

\hfill{\small\textit{Received April 3, 2019}}

%\pagebreak

%\vspace*{-18pt}

\Contr

\noindent
\textbf{Grusho Alexander A.} (b.\ 1946)~--- Doctor of Science in physics and 
mathematics, professor, principal scientist, Institute of Informatics Problems, 
Federal Research Center ``Computer Sciences and Control'' of the Russian 
Academy of Sciences; 44-2~Vavilov Str., Moscow 119133, Russian Federation; 
\mbox{grusho@yandex.ru} 

\vspace*{3pt}

\noindent
\textbf{Zabezhailo Michael I.} (b.\ 1956)~--- Doctor of Science in physics and 
mathematics, principal scientist, Institute of Informatics Problems, Federal Research 
Center ``Computer Sciences and Control'' of the Russian Academy of Sciences;  
44-2~Vavilov Str., Moscow 119133, Russian Federation; 
\mbox{m.zabezhailo@yandex.ru} 

\vspace*{3pt}


\noindent
\textbf{Grusho Nikolai A.} (b.\ 1982)~--- Candidate of Science (PhD) in physics 
and mathematics, senior scientist, Institute of Informatics Problems, Federal 
Research Center ``Computer Sciences and Control'' of the Russian Academy of 
Sciences; 44-2~Vavilov Str., Moscow 119133, Russian Federation; 
\mbox{info@itake.ru} 

\vspace*{3pt}


\noindent
\textbf{Timonina Elena E.} (b.\ 1952)~--- Doctor of Science in technology, 
professor, leading scientist, Institute of Informatics Problems, Federal Research 
Center ``Computer Sciences and Control'' of the Russian Academy of Sciences;  
44-2~Vavilov Str., Moscow 119133, Russian Federation; 
\mbox{eltimon@yandex.ru} 

\label{end\stat}

\renewcommand{\bibname}{\protect\rm Литература}   %1
\def\stat{naumov}

\def\tit{О МАРКОВСКИХ И~РАЦИОНАЛЬНЫХ ПОТОКАХ 
СЛУЧАЙНЫХ СОБЫТИЙ.~II$^*$} % Часть~2$^*$}

\def\titkol{О марковских и рациональных потоках случайных 
событий. II} %Часть 2}

\def\aut{В.\,А.~Наумов$^1$, К.\,Е.~Самуйлов$^2$}

\def\autkol{В.\,А.~Наумов, К.\,Е.~Самуйлов}

\titel{\tit}{\aut}{\autkol}{\titkol}

\index{Наумов В.\,А.}
\index{Самуйлов К.\,Е.}
\index{Naumov V.\,A.}
\index{Samouylov К.\,Е.}


{\renewcommand{\thefootnote}{\fnsymbol{footnote}} \footnotetext[1]
{Исследование выполнено при финансовой поддержке РФФИ в рамках научного проекта №\,19-17-50126.}}


\renewcommand{\thefootnote}{\arabic{footnote}}
\footnotetext[1]{Исследовательский институт инноваций, г.~Хельсинки, Финляндия, 
\mbox{valeriy.naumov@pfu.fi}}
\footnotetext[2]{Российский университет дружбы народов; Институт проблем информатики Федерального 
исследовательского центра <<Информатика и~управ\-ле\-ние>> Российской академии наук, \mbox{samouylov-ke@rudn.ru}}

%\vspace*{6pt}

  \Abst{Статья представляет собой вторую часть обзора, выполненного в рамках проекта 
РФФИ 
  №\,19-17-50126. Цель обзора~--- ознакомление заинтересованных читателей с основами 
теории марковских потоков событий для более подробного изучения и облегчения 
применения этих моделей на практике. В~первой части приведены свойства общих 
марковских потоков событий и показана их связь с марковскими аддитивными процессами и 
процессами марковского восстановления. Во второй части обзора рассмотрены важные для 
приложений частные случаи таких потоков~--- подклассы марковских потоков событий, а~именно:
 простые и групповые потоки однородных и неоднородных событий. Показано, 
как свойства марковских потоков событий связаны с мультипликативностью стационарных 
распределений марковских систем. Обсуждаются  
мат\-рич\-но-экс\-по\-нен\-ци\-аль\-ные распределения и рациональные потоки событий, 
расширяющие возможности марковских потоков для моделирования сложных систем, при 
этом сохраняющие удобство их анализа с помощью вычислительной техники.}
  
  \KW{марковские процессы; марковские аддитивные процессы; потоки без последействия; 
  МС-по\-то\-ки}
  
\DOI{10.14357/19922264200406} 
  
\vspace*{6pt}


\vskip 10pt plus 9pt minus 6pt

\thispagestyle{headings}

\begin{multicols}{2}

\label{st\stat}


\section{Введение}

Настоящий обзор, состоящий из двух частей, включает изложение основ 
теории марковских потоков и снабжен ссылками на большое число работ, 
посвященных марковским и~рациональным потокам событий. Он начался с 
рассмотрения в первой части случайных величин фазового типа, определения 
марковских потоков общего вида и их связи с~марковскими аддитивными 
процессами и процессами марковского восстановления. Во второй части 
обзора  перейдем к важным для приложений подклассам марковских потоков 
однородных и неоднородных событий в разд.~2, а~в~завершение в~разд.~3 
обсудим  
мат\-рич\-но-экс\-по\-нен\-ци\-аль\-ные распределения и~в~разд.~4 
рациональные потоки событий, которые расширяют возможности марковских 
потоков для моделирования сложных систем и~при этом сохраняют удобство 
их анализа. 

Как и в первой части обзора, далее в работе жирные строчные буквы 
обозначают векторы, а~жирные прописные буквы обозначают матрицы. 
Кроме того, используются следующие обозначения: 
$$
\delta(i,j)= \begin{cases}
1, &\mbox{если } i=j\,;\\
0 & \mbox{в~противном\ случае};
\end{cases}
$$
 у~вектора~$\mathbf{e}_i$ 
$i$-я координата равна единице, а остальные равны нулю; $\mathbf{I}\hm= \left[ 
\delta(i,j)\right]$~--- единичная матрица; $\mathbf{u}$~---  
век\-тор-стол\-бец из единиц; $\boldsymbol{\mathcal{N}}^K$~--- множество 
неотрицательных целочисленных векторов длины~$K$, 
$\boldsymbol{\mathcal{N}}^K _0\hm= \boldsymbol{\mathcal{N}}^K \backslash 
\{\mathbf{0}\}$. Для краткости вмес\-то <<наступило $n_1$ событий типа~1, 
$n_2$ событий ти-\linebreak па~2,~\ldots , $n_K$ событий типа~$K$>> будем писать 
<<наступило $\mathbf{n}$ событий>>, где $\mathbf{n}\hm= \left( n_1, n_2, 
\ldots , n_K\right)$.

\section{Важные для~приложений частные случаи марковских потоков 
событий}

\subsection{Простой марковский поток однородных событий}

  Рассмотрим некоторый поток случайных неоднородных событий и 
обозначим через $N_k(t)$ чис\-ло событий типа~$k$, наступивших за время~$t$, 
$\mathbf{N}(t)\hm= (N_1(t), N_2(t), \ldots, N_K(t))$. Поток случайных событий 
называется марковским, если для некоторого случайного процесса~$X(t)$ с 
конечным \mbox{множеством} состояний $\boldsymbol{\mathcal{X}}\hm= \{1,2,\ldots , 
L\}$ процесс $\xi(t)\hm= (X(t), \mathbf{N}(t))$ является марковским процессом, 
однородным во времени и по второй компоненте, т.\,е.\ если для любых~$t, 
h\hm>0$ справедливы равенства
  \begin{multline*}
  {\sf P}\left(X(h+t)=j, \mathbf{N}(h+t)=\mathbf{k}+\mathbf{n}\vert X(h)=i, \right.\\
\left.\mathbf{N}(h)=\mathbf{k}\right)=p_{\mathbf{n}}(i,j,t)\,,\enskip
  \mathbf{k}, \mathbf{n} \in \boldsymbol{\mathcal{N}}^K,\enskip i,j\in 
\boldsymbol{\mathcal{X}}\,.
  \end{multline*}
Матрицы вероятностей переходов $\mathbf{P}_{\mathbf{n}}(t)\hm= 
[p_{\mathbf{n}}(i,j,t)]$ однозначно определяются матрицами интенсивностей 
переходов $\mathbf{A}_{\mathbf{n}}\hm= \left[ a_{\mathbf{n}}(i,j)\right]$, 
$\mathbf{n}\hm\geq \mathbf{0}$, где
\begin{align*}
a_{\mathbf{0}}(i,j) &=\lim\limits_{t\to0} \fr{1}{t}\left( p_{\mathbf{0}}(i,j,t)-\delta(i,j)\right)\,,\enskip
 i,j\in  \boldsymbol{\mathcal{X}}\,;\\
a_{\mathbf{n}}(i,j) &=\lim\limits_{t\to0} \fr{1}{t}\, p_{\mathbf{n}}(i,j,t)\,,\enskip i,j\in 
\boldsymbol{\mathcal{X}}\,,\enskip \mathbf{n}\in \boldsymbol{\mathcal{N}}^K_0,
\end{align*}
при этом фазовый процесс~$X(t)$ является однородным марковским 
процессом с матрицей интенсивностей переходов $\mathbf{A}\hm= 
\sum\nolimits_{\mathbf{n}\in \boldsymbol{\mathcal{N}}^K} 
\mathbf{A}_{\mathbf{n}}$.
  
  В первой части обзора определен процесс марковского восстановления 
$(X_l,\boldsymbol{\sigma}_l, \tau_l)$, где $X_l\hm= X(t_l)$~--- состояния 
фазового процесса~$X(t)$ марковского потока в моменты после наступления\linebreak 
событий потока, $X(t)\hm\in \boldsymbol{\mathcal{X}} \hm= \{1,2,\ldots ,L\}$, 
$0\hm< t_1\hm< t_2
  <\cdots$~--- моменты наступления событий, также называемые 
вызывающими моментами; $\tau_l\hm= t_l\hm- t_{l-1}$~--- длины интервалов 
между \mbox{моментами} наступления событий; $\boldsymbol{\sigma}_l$~--- вектор, 
$\boldsymbol{\sigma}_l\hm= (\sigma_{l,1}, \ldots , \sigma_{l,K})$, 
в~котором~$\sigma_{l,k}$ есть размер группы событий типа~$k$, наступивших 
в~момент~$t_l$, $l\hm=1, 2, \ldots$ Матрицы $\mathbf{G}_{\mathbf{n}}(x)\hm= 
[G_{\mathbf{n}}(i,j,x)]$, описывающие связанный с марковским потоком 
процесс марковского восстановления $(X_l, \boldsymbol{\sigma}_l, \tau_l)$, и 
их преобразования Лап\-ла\-са--Стилть\-еса имеют следующий вид:

\noindent
  \begin{align}
  \mathbf{G}_{\mathbf{n}}(x)&=\int\limits_0^x \exp 
(z\mathbf{A}_0)\mathbf{A}_{\mathbf{n}}\,dz={}\notag\\
&\hspace*{-10mm}{}=\left( \exp 
(x\mathbf{A}_{\mathbf{0}}))-\mathbf{I}\right)\mathbf{A}_0^{-1} \mathbf{A}_{\mathbf{n}}\,,\ \mathbf{n}\in 
\boldsymbol{\mathcal{N}}_0^K\,;
  \label{e1-nau}\\
  \int\limits_0^x e^{-\nu x}d\mathbf{G}_{\mathbf{n}}(x)&= (\nu\mathbf{I}-
\mathbf{A}_{\mathbf{0}})^{-1}\mathbf{A}_{\mathbf{n}}\,,\ \mathbf{n}\in 
\boldsymbol{\mathcal{N}}_0^K\,.
  \label{e2-nau}
  \end{align}
Используя матрицы $\mathbf{G}_{\mathbf{n}}(x)$, можно найти совместное 
распределение числа~$\boldsymbol{\sigma}_l$ наступивших событий и 
длин~$\tau_l$ интервалов между вызывающими моментами 
\begin{multline}
F_{\mathbf{k}_1, \mathbf{k}_2, \ldots , \mathbf{k}_m} \left(x_1, x_2, \ldots , 
x_m\right)={}\\
{}={\sf P}\left(
\boldsymbol{\sigma}_l=\mathbf{k}_l\,, \tau_l<x_l\,, l=1,2,\ldots, m\right)={}\\
{}=\bm{\alpha}\mathbf{G}_{\mathbf{k}_1}(x_1) \mathbf{G}_{\mathbf{k}_2}(x_2)\cdots 
\mathbf{G}_{\mathbf{k}_m}(x_m)\mathbf{u}\,,
\label{e3-nau}
\end{multline}
а также плотность этого распределения

\columnbreak

\noindent
\begin{multline}
f_{\mathbf{k}_1, \mathbf{k}_2, \ldots , \mathbf{k}_m} (x_1, x_2, \ldots , 
x_m)={}\\
{}=\bm{\alpha}\exp \left( x_1\mathbf{A}_{\mathbf{0}}\right) 
\mathbf{A}_{\mathbf{k}_1}\exp \left( x_2\mathbf{A}_{\mathbf{0}}\right) 
\mathbf{A}_{\mathbf{k}_2}\cdots\\
\cdots \exp \left( x_m\mathbf{A}_{\mathbf{0}}\right) 
\mathbf{A}_{\mathbf{k}_m}\mathbf{u}\,,\quad
\mathbf{k}_1, \mathbf{k}_2, \ldots , \mathbf{k}_m\in 
\boldsymbol{\mathcal{N}}_0^K\,,\\
 x_0, x_1, \ldots , x_m>0\,,\enskip m=1,2,\ldots
\label{e4-nau}
\end{multline}

\vspace*{-6pt}

\noindent
где $\bm{\alpha}$~--- начальное распределение фазового про\-цесса.


  
  Простой марковский поток однородных событий~--- это марковский поток 
событий одного типа, причем в каждый вызывающий момент наступает ровно 
одно событие. Он характеризуется двумя мат\-ри\-ца\-ми интенсивностей 
переходов $\mathbf{S}\hm= \mathbf{A}_0$ и~$\mathbf{R}\hm= \mathbf{A}_1$, 
а~остальные матрицы~$\mathbf{A}_k$, $k\hm\geq 2$, для такого потока~--- 
нулевые. Первыми работами, посвященными простым марковским потокам 
однородных событий, стали~[1--5]. Их применение к~решению задач теории 
телетрафика рассматривается  
в~\cite{6-nau, 7-nau}. Поток вызывающих моментов любого марковского 
потока~--- это простой марковский поток, характеризуемый матрицами 
$\mathbf{S}\hm= \mathbf{A}\hm-\mathbf{R}$ и~$\mathbf{R}\hm= 
\sum\nolimits_{\mathbf{n}\in \boldsymbol{\mathcal{N}}_0^K} 
\mathbf{A}_{\mathbf{n}}$. К~простым марковским потокам относятся также 
процессы восстановления фазового типа~\cite{8-nau}. Для таких потоков ранг 
матрицы~$\mathbf{R}$ равен единице и~она имеет вид $\mathbf{R}\hm= 
\mathbf{sq}$, где $\mathbf{s}\hm= -\mathbf{Su}$. Верно и~обратное~\cite{7-nau}. 
В~англоязычной литературе простые марковские потоки называют 
Markovian arrival process и~используют для их обозначения сокращение МАР 
или MArP.
  
  Простой марковский поток однородных событий является 
полумарковским, поскольку последовательность $(X_l, \tau_l)$, $l\hm=1, 
2,\ldots,$~--- процесс марковского восстановления. Из~(1) и~(2) вытекают 
следующие формулы для полумарковской матрицы $\mathbf{G}(x)\hm= \left[ 
G(i,j,x)\right]$ процесса $(X_l,\tau_l)$ марковского восстановления с 
элементами 

\vspace*{3pt}

\noindent
  $$
  G(i,j,x)={\sf P} \left( X_l=j,\ \tau_l<x\vert X_{l-1}=i\right)
  $$
  
  \vspace*{-1pt}
  
  \noindent
 и для ее преобразования Лап\-ла\-са--Стилть\-еса:
 
 \vspace*{2pt}
 
 \noindent
\begin{equation}
\left.
\begin{array}{rl}
\mathbf{G}(x)&=\left( \exp (x\mathbf{S})-\mathbf{I}\right) \mathbf{S}^{-
1}\mathbf{R}\,;\\
\displaystyle\int\limits_0^x e^{-\nu x}d\mathbf{G}(x)&=(\nu\mathbf{I}-\mathbf{S})^{-1}\mathbf{R}\,.
\end{array}
\right\}
\label{e5-nau}
\end{equation}

\vspace*{-2pt}
  
  Из~(\ref{e4-nau}) вытекает следующее выражение для плотности функции 
распределения длин интервалов~$\tau_l$ между моментами наступления 
событий простого марковского потока однородных событий:

\vspace*{-8pt}

\noindent
  \begin{multline}
  f\left( x_1, x_2, \ldots, x_m\right)={}\\
  {}=\bm{\alpha}\exp \left(x_1\mathbf{S}\right) 
\mathbf{R}\exp \left( x_2\mathbf{S}\right)\mathbf{R}\cdots \exp \left( 
x_m\mathbf{S}\right) \mathbf{Ru}\,,\\
  x_0, x_1,\ldots , x_m>0\,,\enskip m=1,2,\ldots
  \label{e6-nau}
  \end{multline}
  
  \vspace*{-2pt}
  
  Поскольку простой марковский поток является полумарковским, при 
анализе систем массового обслуживания с такими поступающими потоками 
можно использовать результаты, полученные для систем с полумарковским 
входящим потоком,  
например~[9--12].
   
  В первом разделе обзора указано, что стационарные распределения 
$\mathbf{q}\hm=[q(i)]$ и $\mathbf{q}_{\mathbf{n}}\hm= [q_{\mathbf{n}}(i)]$, 
$\mathbf{n}\hm\in \boldsymbol{\mathcal{N}}_0^K$, вложенных цепей 
Маркова~$X_l$ и~$(X_l, \boldsymbol{\sigma}_l)$ связаны со стационарным 
распределением~$\mathbf{p}$ фазового процесса~$X(t)$ следующими 
равенствами:
\begin{multline*}
  \mathbf{q}=\fr{1}{\lambda}\,\mathbf{p}\boldsymbol{\Lambda}\,,\
  \mathbf{p}=-\lambda \mathbf{q}\mathbf{A}_0^{-1}\,,\ 
  \mathbf{q}=\sum\limits_{\mathbf{n}\in \boldsymbol{\mathcal{N}}_0^K} 
\mathbf{q}_{\mathbf{n}}\,,\\
 \mathbf{q}_{\mathbf{n}}=\fr{1}{\lambda}\,\mathbf{p}
  \mathbf{A}_{\mathbf{n}}\,,\enskip \mathbf{n}\in \boldsymbol{\mathcal{N}}_0^K\,.
  \end{multline*}

  Если вектор из единиц~$\mathbf{u}$ является правым собственным 
вектором каждой из матриц~$\mathbf{A}_{\mathbf{n}}$ и выполняются 
равенства 
  \begin{equation}
  \mathbf{A}_{\mathbf{n}}\mathbf{u}=\lambda_{\mathbf{u}}\mathbf{u}\,,\quad
  \mathbf{n}\in \boldsymbol{\mathcal{N}}_0^K\,,
  \label{e7-nau}
  \end{equation}
то из~(\ref{e3-nau}) следует, что при любом начальном 
распределении~$\mathbf{s}$ марковский поток будет стационарным потоком 
без последействия. Аналогично, если вектор стационарных 
вероятностей~$\mathbf{p}$ является левым собственным вектором 
матриц~$\mathbf{A}_{\mathbf{n}}$ и выполняются равенства 
\begin{equation}
\mathbf{pA}_{\mathbf{n}}=\lambda_{\mathbf{n}}\mathbf{p}\,,\quad
\mathbf{n}\in \boldsymbol{\mathcal{N}}_0^K\,.
\label{e8-nau}
\end{equation}
    
Условия~(\ref{e7-nau}) и~(\ref{e8-nau}), достаточные для того чтобы 
марковский поток был пуассоновским, для простого марковского потока 
приобретают вид $\mathbf{Ru}\hm= \lambda\mathbf{u}$ и~$\mathbf{pR}\hm= 
\lambda\mathbf{p}$ соответственно, где $\lambda\hm= \mathbf{pRu}$~--- 
интенсивность потока. Проверка необходимых и~достаточных условий 
пуассоновости простого марковского потока более сложна и~требует знания 
собственных векторов матрицы~$\mathbf{S}$~\cite{13-nau}.
  
  Считающий процесс $N(t)$ стационарной версии простого марковского 
потока является асимптотически нормальным с~математическим ожиданием 
${\sf M}(t)\hm=\lambda t$ и дисперсией
  $$
  {\sf D}(t)=\left( 2\mathbf{d}_1\mathbf{s}-\lambda\right) t +2\left( 
\mathbf{d}_2\mathbf{s}-\lambda\right) +o(1)\,,
  $$
где векторы-столб\-цы~$\mathbf{d}_1$ и~$\mathbf{d}_2$~--- единственные 
решения систем линейных уравнений~\cite{2-nau}:
\begin{alignat*}{2}
\mathbf{d}_1\mathbf{A} &=\mathbf{p}(\lambda \mathbf{I}-\mathbf{R})\,,&\quad
\mathbf{d}_1\mathbf{u}&=1\,;\\
\mathbf{d}_2\mathbf{A}&=\mathbf{d}_1 -\mathbf{p}\,, &\quad
\mathbf{d}_2\mathbf{u}&=1\,.
\end{alignat*}
    
\subsection{Простой марковский поток неоднородных событий}

  Простой марковский поток неоднородных событий~--- это марковский 
поток событий нескольких типов, в каждый вызывающий момент которого 
наступает ровно одно событие. Такой поток характеризуется $K\hm+1$ 
матрицами интенсивностей переходов $\mathbf{S}\hm= \mathbf{A}_0$ и 
$\mathbf{R}_k\hm= \mathbf{A}_{\mathbf{e}_k}$, $k\hm=1,2,\ldots ,K$, 
а~остальные матрицы~$\mathbf{A}_{\mathbf{n}}$~--- нулевые. При этом поток 
событий одного типа, например типа~$i$, является простым марковским 
потоком однородных событий, описываемым матрицами 
$\mathbf{S}_i\hm=\mathbf{A}\hm- \mathbf{A}_{\mathbf{e}_i}$ 
и~$\mathbf{R}_i$. Первыми работами, посвященными прос\-тым марковским 
потокам неоднородных событий, считаются~[14--16]. В~англоязычной 
литературе такой поток называют Markovian Arrival Process with marked arrivals 
и~используют для его обозначения сокращение ММАР.  
Из~(\ref{e5-nau}) вытекает следующее выражение для плотности совместного 
распределения ${\sf P}(\omega_l=k_l,\tau_l<x_l, l\hm=1,2,\ldots ,m)$ типов 
$\omega_l$ событий, наступивших в~момент~$t_l$, и~длин~$\tau_l$ интервалов 
между вызывающими моментами: 
  \begin{multline}
  f_{{k}_1, {k}_2, \ldots , {k}_m}\left( x_1, x_2, \ldots , x_m\right)={}\\
  {}=\bm{\alpha}\exp \left( x_1\mathbf{S}\right)\mathbf{R}_{k_1}\exp\left( 
x_2\mathbf{S}\right) \mathbf{R}_{k_2}\cdots\\
\cdots \exp \left( x_m\mathbf{S}\right) 
\mathbf{R}_{k_m}\mathbf{u}\,,\quad
 1\leq k_1, k_2, \ldots , k_m\leq K\,,\\
x_0, x_1, \ldots , x_m>0\,,\quad  m=1,2,\ldots
 \label{e9-nau}
\end{multline}

\subsection{Марковский поток групп однородных событий}

  Марковский поток групп однородных событий~--- это марковский поток 
событий одного типа, в каждый вызывающий момент которого \mbox{может} 
наступить несколько событий. Такие марковские потоки впервые 
исследовались в~\cite{8-nau, 17-nau, 18-nau}, а их описание с помощью 
матриц~$\mathbf{A}_{\mathbf{n}}$ впервые появилось в~\cite{19-nau}. 
В~англоязычной литературе такой поток сейчас называют batch Markovian 
arrival process и используют для его обозначения сокращение BMAP. 
В~\cite{20-nau} получены формулы и асимптотики для первых двух моментов 
считающего процесса~$N(t)$, а~в~\cite{21-nau}~--- для старших моментов~$N(t)$.

\section{Матрично-экспоненциальные распределения}

  Функция распределения $F(t)$ неотрицательной случайной величины 
называется мат\-рич\-но-экс\-по\-нен\-ци\-аль\-ной, если $F(0)\hm<1$ и она 
представима в~виде 
  \begin{equation}
  F(t)=1-\mathbf{q}\exp (t\mathbf{S})\mathbf{u}
  \label{e10-nau}
  \end{equation}
с некоторым вектором~$\mathbf{q}$ и матрицей~$\mathbf{S}$, име\-ющей 
собственные числа лишь с отрицательными действительными частями. Для 
того чтобы функция распределения~$F(t)$ неотрицательной случайной 
величины была  
мат\-рич\-но-экс\-по\-нен\-ци\-аль\-ной, необходимо и достаточно, чтобы она 
имела рациональное преобразование 
Лап\-ла\-са--Стилть\-еса $\tilde{F}(\nu)$. Минимальный порядок 
матрицы~$\mathbf{S}$  
в~мат\-рич\-но-экс\-по\-нен\-ци\-аль\-ном представлении~(\ref{e10-nau}) равен 
чис\-лу полюсов функции $\tilde{F}(\nu)$ с учетом их кратности. Представление 
с~матрицей~$\mathbf{S}$ минимального порядка называется минимальным. 

  В некоторых работах по мат\-рич\-но-экс\-по\-нен\-ци\-аль\-ным функциям  
распределения~\cite{22-nau, 23-nau, 24-nau}, а~также в книгах~\cite{25-nau, 26-nau}, 
чтобы подчеркнуть аналогию с экспоненциальными функциями 
\mbox{распределения},  
вмес\-то~(\ref{e10-nau}) использовалось пред\-став\-ле\-ние $F(t)\hm= 1\hm - 
\mathbf{q}\exp (-t\mathbf{B})\mathbf{u}$ со знаком минус перед~$t$ 
и~мат\-ри\-цей~$\mathbf{B}$, име\-ющей собственные чис\-ла с~положительными 
действительными частями. В~настоящее\linebreak время используются только 
представления вида~(\ref{e10-nau}). Иногда допускается, что 
вектор~$\mathbf{u}$ в~(\ref{e10-nau}) может быть любым, а~не состоящим из 
единиц, как в~рас\-смат\-ри\-ва\-емом случае. Однако в~\cite{24-nau, 27-nau} 
было показано, что всегда можно подобрать мат\-рич\-но-экс\-по\-нен\-ци\-аль\-ное 
пред\-став\-ле\-ние с~$\mathbf{u}\hm=(1,1,\ldots , 1)$. 
  
  Идея матрично-экс\-по\-нен\-ци\-аль\-ных функций распределения восходит 
к работе~\cite{28-nau}, в которой показано, что рациональные преобразования  
Лап\-ла\-са--Стилть\-еса неотрицательных функций распределения 
представимы в виде:
  $$
  \tilde{F}(s)=p_0+\sum\limits^L_{l=1} q_0\cdots q_{l-1} p_l \prod\limits^l_{i=1} 
\fr{\lambda_i}{\lambda_{i}+s}\,,
  $$
где $p_i+q_i\hm=1$, $i\hm=1, \ldots , L$, $p_L\hm=1$, и~$-\lambda_i$, $i\hm=1, 
\ldots , L$,~--- полюсы~$\tilde{F}(s)$. Такое представление можно записать в 
мат\-рич\-но-экс\-по\-нен\-ци\-аль\-ном виде~(\ref{e10-nau}), полагая 
\begin{align*}
\mathbf{q}&=(1,0,\ldots ,0)\,;\\
\mathbf{S}&=\begin{bmatrix}
-\lambda_1&q_1\lambda_1&0&\cdots&0\\
0&-\lambda_2&q_2\lambda_2&\ddots &\vdots\\
0&0&\ddots& \ddots& 0\\
\vdots& \ddots& \ddots& -\lambda_{L-1}&q_{L-1}\lambda_{L-1}\\
0&\cdots & 0&0&-\lambda_L
\end{bmatrix}\,,
\end{align*}
%
  при этом элементы матрицы~$\mathbf{S}$ могут быть комплексными. 
В~\cite{22-nau} показано, что вектор~$\mathbf{q}$ и~мат\-ри\-ца~$\mathbf{S}$  
в~мат\-рич\-но-экс\-по\-нен\-ци\-аль\-ном  
пред\-ставлении~(\ref{e10-nau}) всегда могут быть выбраны действительными. 
  
  Из~(\ref{e10-nau}) вытекают формулы для начальных моментов
  \begin{equation*}
  \int\limits_0^\infty t^n dF(t)=n! \mathbf{q}(-\mathbf{S})^{-n}\mathbf{u}\,,\enskip 
n=1,2,\ldots
  %\label{e11-nau}
  \end{equation*}
и для преобразования Лап\-ла\-са--Стилть\-еса функции распределения~$F(t)$ 
\begin{multline*}
\tilde{F}(\nu)=\int\limits_0^\infty e^{-\nu t}dF(t)={}\\
{}=1-
\mathbf{q}\mathbf{u}+\mathbf{q}(\nu\mathbf{I}-\mathbf{S})^{-1} \mathbf{s}=1-
\nu\mathbf{q}(\nu\mathbf{I}-\mathbf{S})^{-1}\mathbf{u}\,,
%\label{e12-nau}
\end{multline*}
где $\mathbf{s}=-\mathbf{Su}$. Кроме того,  
мат\-рич\-но-экс\-по\-нен\-ци\-аль\-ные функции распределения обладают 
сле\-ду\-ющи\-ми свойствами~\cite{24-nau}.
\begin{enumerate}[1.]
\item Пусть $F_i(t)=1\hm- \mathbf{q}_i\exp(t\mathbf{S}_i) \mathbf{u}$, 
$i\hm=1,2$,~--- мат\-рич\-но-экс\-по\-нен\-ци\-аль\-ные функции 
распределения и $p_1\hm+p_2\hm=1$. Тогда
\begin{align*}
p_1F_1(t)+p_2F_2(t)&={}\\
&\hspace*{-15mm}{}=1-(p_1\mathbf{q}_1, p_2\mathbf{q}_2)\exp \left( 
t\begin{bmatrix} \mathbf{S}_1 & \mathbf{0}\\
\mathbf{0}& \mathbf{S}_2\end{bmatrix}
\right) \mathbf{u}\,; %\label{e13-naum}
\\
\left( F_1*F_2\right) (t) &={}\\
&\hspace*{-25mm}{}= 1-\left(\mathbf{q}_1,F_1(0) \mathbf{q}_2\right) \exp 
\left( t \begin{bmatrix}
\mathbf{S}_1 & -\mathbf{S}_1\mathbf{uq}_2\\
\mathbf{0} & \mathbf{S}_2\end{bmatrix} \right) \mathbf{u}\,.
%\label{e14-nau}
\end{align*}
\item Пусть $\tau$ и~$\gamma$~--- независимые неотрицательные случайные 
величины с функциями распределения~$F(t)$ и~$G(t)$ соответственно, 
причем~$F(t)$ имеет  
мат\-рич\-но-экс\-по\-нен\-ци\-аль\-ное представление~(\ref{e10-nau}). Тогда 
функция распределения~$H(t)$ случайной величины  $(\tau\hm-\gamma)^+$ 
имеет  
мат\-рич\-но-экс\-по\-нен\-ци\-аль\-ное пред\-став\-ление 
\begin{equation*}
H(t)=1-\mathbf{qU}\exp (t\mathbf{S})\mathbf{u}\,,
%\label{e15-nau}
\end{equation*}
где 
\begin{equation}
\mathbf{U}=\int\limits_0^\infty e^{t\mathbf{S}}dG(t)\,.
\label{e16-nau}
\end{equation}

\item Пусть $F(t)$ имеет мат\-рич\-но-экс\-по\-нен\-ци\-аль\-ное  
представление~(\ref{e10-nau}), а~у~квад\-рат\-ной мат\-ри\-цы~$\mathbf{V}$ 
все собственные числа имеют неотрицательные вещественные части. Тогда
\begin{multline*}
\int\limits_0^\infty e^{-t\mathbf{V}}dF(t)=(1-\mathbf{qu}) \mathbf{I}+
(\mathbf{q}\otimes \mathbf{I})\boldsymbol{\Psi}(\mathbf{Su}\otimes 
\mathbf{I})={}\\
{}=\mathbf{I}-(\mathbf{q}\otimes 
\mathbf{I})\boldsymbol{\Psi}(\mathbf{u}\otimes \mathbf{V})\,,
%\label{e17-nau}
\end{multline*}
где $\boldsymbol{\Psi}=(\mathbf{I}\otimes \mathbf{V}\hm- \mathbf{S}\otimes 
\mathbf{I})^{-1}$. 
  \end{enumerate}
  
  Последнее свойство можно использовать для вычисления 
матриц~$\mathbf{U}$ в~(\ref{e16-nau}) для мат\-рич\-но-экс\-по\-нен\-ци\-аль\-ных 
функций распределения~$G(t)$.
  
  Ясно, что функции распределения фазового типа являются  
мат\-рич\-но-экс\-по\-нен\-ци\-аль\-ны\-ми. Однако их  
мат\-рич\-но-экс\-по\-нен\-ци\-аль\-ные представления 
  \begin{multline*}
  F(t)-1-\mathbf{q}\exp (t\mathbf{S})\mathbf{u}\,,\enskip
  %\label{e18-nau}
    F(0)=1-\mathbf{qu}\,,\\
     \fr{d}{dt}\,F(t)= \mathbf{q}\exp 
(t\mathbf{S})\mathbf{s}\,,\ t>0\,,
  \end{multline*}
с ограничениями
\begin{equation}
\hspace*{-2mm}\left.
\begin{array}{rlrlrl}
\!\!\displaystyle 0<\sum\limits_{j\in \boldsymbol{\mathcal{X}}} q(j)&\leq 1\,,&\ q(i)&\geq0\,,&\ i&\in 
\boldsymbol{\mathcal{X}};
\\[9pt]
\!\!\displaystyle \sum\limits_{j\in \boldsymbol{\mathcal{X}}} \!\!s(i,j)&\leq 0\,,&\ s(i,j)&\geq 0\,,&\ 
i&\not= j\,,\ i, j\in \boldsymbol{\mathcal{X}},
\end{array}\!
\right\}\!\!\!\!
\label{e19-nau}
\end{equation}
где $\mathbf{S}=[s(i,j)]$, следует отличать от мат\-рич\-но-экс\-по\-нен\-ци\-аль\-ных 
представлений этих же функций, но без ограничений~(\ref{e19-nau}). 
Порядок  
мат\-рич\-но-экс\-по\-нен\-ци\-аль\-но\-го представления, удовлетворяющего 
ограничениям~(\ref{e19-nau}), будем называть числом этапов этого 
представления, а~порядок мат\-рич\-но-экс\-по\-нен\-ци\-аль\-но\-го 
представления, не удовлетворяющего этим ограничениям, следуя~\cite{28-nau}, 
будем называть\linebreak
 числом \textit{фиктивных} этапов. Необходимые и 
достаточные условия того, чтобы для функции распределения 
с~рациональным преобразованием Лап\-ла\-са--Стилть\-еса существовало 
представление, \mbox{удовлетворяющее} ограничениям~(\ref{e19-nau}), получены 
в~\cite{29-nau}. Для этого надо, чтобы (а)~функция распределения имела 
непрерывную положительную плотность на правой полуоси и~(б)~ее 
преобразование Лап\-ла\-са--Стилть\-еса имело единственный полюс 
с~максимальной вещественной частью. 

\section{Рациональные потоки событий}

  Рациональный поток групп неоднородных событий 
$(t_l,\boldsymbol{\sigma}_l)$, $l\hm=1,2,\ldots$, можно определить как поток, 
для которого совместное распределение чис\-ла~$\boldsymbol{\sigma}_l$ 
наступивших событий и~длин~$\tau_l$ интервалов между моментами~$t_l$ 
наступления событий дается формулами~(\ref{e1-nau}) и~(\ref{e3-nau}) 
с~матрицами~$\mathbf{A}_{\mathbf{n}}$, $\mathbf{n}\hm\in 
\boldsymbol{\mathcal{N}}^K $, обладающими следующими свойствами:
  \begin{enumerate}[(1)]
\item действительные части собственных чисел мат\-ри\-цы~$\mathbf{A}_{\mathbf{0}}$ 
отрицательны;
\item действительные части собственных чисел мат\-ри\-цы 
$\mathbf{A}\hm= \sum\nolimits_{\mathbf{n}\in 
\boldsymbol{\mathcal{N}}^K} \mathbf{A}_{\mathbf{n}}$ 
неположительны;
\item $\mathbf{Au}=\mathbf{0}$.
\end{enumerate}
  
  Для стационарных версий рациональных потоков дополнительно требуется, 
чтобы начальный вектор~$\bm{\alpha}$ совпадал с решением~$\mathbf{p}$ 
системы линейных уравнений $\mathbf{pA}\hm=0$, $\mathbf{pu}\hm=1$.
  
  Простой рациональный поток однородных событий, также называемый  
мат\-рич\-но-экс\-по\-нен\-циальным потоком~\cite{30-nau},~--- это поток 
событий одного типа, в каждый вызывающий момент которого наступает 
ровно одно событие и для которого плотность совместного распределения 
длин~$\tau_l$ интервалов между моментами наступления событий дается 
формулой~(\ref{e6-nau}) с матрицами~$\mathbf{S}$ и~$\mathbf{R}$, 
обладающими следующими свойствами~\cite{31-nau}:
\begin{itemize}
\item[(а)] вещественные части собственных чисел матрицы~$\mathbf{S}$ 
отрицательны;
\item[(б)] вещественные части собственных чисел матрицы 
$\mathbf{S}\hm+\mathbf{R}$ неположительны; 
\item[(в)] $(\mathbf{S}+\mathbf{R})\mathbf{u}=\mathbf{0}$. 
  \end{itemize}
  
  Примерами простых рациональных потоков однородных событий могут 
служить полумарковские потоки~\cite{22-nau} и процессы 
восстановления~\cite{27-nau}  
с~мат\-рич\-но-экс\-по\-нен\-ци\-аль\-ны\-ми функциями распределения длин 
интервалов между наступлениями событий.
  
  Рациональный поток неоднородных событий~--- это поток событий 
нескольких типов, в каждый вызывающий момент которого наступает ровно 
одно событие. Для такого потока совместное распределение типов 
наступивших событий~$\omega_l$ и длин~$\tau_l$ интервалов между 
моментами наступления событий дается формулой~(\ref{e9-nau}), а на 
матрицы~$\mathbf{S}$ 
и~$\mathbf{R}\hm=\mathbf{R}_1\hm+\mathbf{R}_2+\cdots  + \mathbf{R}_K$ 
накладываются перечисленные выше ограничения~(a)--(в)~\cite{32-nau}. 

\section{Заключение}

  Метод этапов Эрланга~\cite{33-nau} более 100~лет применяется при анализе 
стохастических систем. К~его широкому распространению привело открытие  
мат\-рич\-но-экс\-по\-нен\-ци\-аль\-но\-го представления для функций 
распределения фазового типа~\cite{34-nau} и моделей марковских потоков 
событий~\cite{1-nau, 17-nau}. Эти модели хорошо подходят для анализа 
стохастических систем с~по\-мощью вычислительной техники, 
приспособленной к обработке векторов и матриц, что привело к развитию 
специальных матричных методов анализа стохастических систем.
  
  Метод фиктивных этапов, предложенный в~\cite{28-nau}, позволил 
распространить метод Эрланга на любые распределения с рациональным 
преобразованием 
  Лап\-ла\-са--Стилть\-еса. Использование мат\-рич\-но-экс\-по\-нен\-ци\-аль\-ных 
  представлений для функций распределения~\cite{22-nau, 23-nau, 25-nau} 
  и~потоков случайных событий~\cite{31-nau} с произвольными рациональными 
преобразованиями 
  Лап\-ла\-са--Стилть\-еса упрощает применение метода фиктивных этапов. 
Формальное применение метода фиктивных этапов приводит\linebreak 
к~решению, 
в~котором вероятности, со\-от\-вет\-ст\-ву\-ющие фиктивным этапам, могут оказаться 
отрицательными, б$\acute{\mbox{о}}$льшими единицы или даже 
комплекс\-ны\-ми. Однако вероятности, соответствующие \mbox{не\-фик\-тив\-ным} 
состояниям, будут неотрицательными числами, не превосходящими единицы. 
Существуют различные интерпретации понятий отрицательных вероятностей 
и интенсивностей \mbox{переходов} \cite{35-nau, 36-nau, 37-nau, 38-nau}. Более 
детально ознакомиться с~{марковским} и~рациональным потоками событий, 
а~также с~матричными методами анализа стохастических систем можно 
 в~обзорах~\cite{39-nau, 40-nau, 41-nau, 42-nau, 43-nau, 44-nau}  
и~монографиях~[18, 25, 26, 45--57]. 

{\small\frenchspacing
 {%\baselineskip=10.8pt
 %\addcontentsline{toc}{section}{References}
 \begin{thebibliography}{99}

\bibitem{1-nau}
\Au{Наумов В.\,А.} О~независимой работе подсистем сложной системы~// Тр.~III 
Всесоюзной  
шко\-лы-се\-ми\-на\-ра по теории массового обслуживания.~--- 
М.: МГУ, 1976. №\,2. С.~169--177.
\bibitem{2-nau}
\Au{Бочаров П.\,П., Наумов В.\,А.} Анализ гиперэкспоненциальной двухфазной системы с 
ограниченным накопителем~// Информационные сети и их структура.~--- М.: Наука, 
1976.  
С.~168--180.
\bibitem{3-nau}
\Au{Наумов В.\,А.} Об обслуженной и избыточной нагрузках полнодоступного пучка с 
ограниченной очередью~// Численные методы решения задач математической физики и 
теории систем.~--- М.: УДН, 1977. С.~51--55.
\bibitem{4-nau}
\Au{Наумов В.\,А.} Исследование некоторых многофазных систем массового 
обслуживания: Дис.\ \ldots\ канд. физ.-мат. наук.~--- М.: УДН, 1978.  98~с.
\bibitem{5-nau}
\Au{Lucantoni D.\,M., Meier-Hellstern~K., Neuts M.\,F.} A~single-server queue with server 
vacations and a class of non-renewal arrival processes~// Adv. Appl. Probab., 1990. 
Vol.~22. Iss.~3. P.~676--705.
\bibitem{6-nau}
\Au{Башарин Г.\,П., Кокотушкин~В.\,А., Наумов~В.\,А.} О~методе эквивалентных замен 
расчета фрагментов сетей связи для ЦВМ~// Изв. АН \mbox{СССР}. Техническая кибернетика, 1979. №\,6. С.~92--99.
\bibitem{7-nau}
\Au{Basharin G., Naumov V.} Simple matrix description of peaked and smooth traffic and 
its applications~// 3rd ITC Specialist Seminar on Fundamentals of Teletraffic Theory.~--- M.: 
VINITI, 1984. P.~38--44. 
\bibitem{8-nau}
\Au{Neuts M.\,F.} Renewal processes of phase type~// Nav. Res. Logist.~Q., 1978. 
Vol.~25. Iss.~3. P.~445--454.
\bibitem{9-nau}
\Au{Cinlar E.} Queues with semi-Markovian arrivals~// J.~Appl. Probab., 1967. Vol.~4. Iss.~2.  
P.~365--379.
\bibitem{10-nau}
\Au{Franken P.} Erlangsche Formeln f$\ddot{\mbox{u}}$r semimarkowschen Eingang // 
Elektronische Informationsverarbeitung Kybernetik, 1968. Vol.~4. Iss.~3. P.~197--204.
\bibitem{11-nau}
\Au{Franken P., Kerstan~J.} Bedienungssysteme mit unendlich vielen Bedienungsapparaten~// 
Operationsforschung Mathematische Statistik.~--- Berlin: Akademie-Verlag, 1968. Vol.~I. 
P.~67--76.
\bibitem{12-nau}
\Au{Neuts M.\,F., Chen~S.-Z.} The infinite server queue with semi-Markovian arrivals and negative 
exponential services~// J.~Appl. Probab., 1972. Vol.~9. Iss.~1. P.~178--184.
\bibitem{13-nau}
\Au{Bean N.\,G., Green D.\,A., Taylor~P.\,G.} When is a MAP poisson?~// 2nd Australia--Japan 
Workshop on Stochastic Models in Engineering,
Technology 
and Management Proceedings~/ Eds. J.~Wilson, D.\,N.\,P.~Murthy, S.~Osaki.~--- 
Brisbane: Technology Management Center, University of Queensland, 1996. P.~34--43.
\bibitem{14-nau}
\Au{Наумов В.\,А.} Матричный аналог формулы Эрланга~// Модели распределения 
информации и методы их анализа.~--- М.: ВИНИТИ, 1988. C.~39--43.
\bibitem{15-nau}
\Au{He Q.-M.} Queues with marked customers~// Adv. Appl. Probab., 1996. Vol.~28. 
Iss.~2. P.~567--587.
\bibitem{16-nau}
\Au{He Q.-M., Neuts M.\,F.} Markov chains with marked transitions~// Stoch. Proc. 
Appl., 1998. Vol.~74. P.~37--52.
\bibitem{17-nau}
\Au{Neuts M.\,F.} A versatile Markovian point process.~--- Newark, DE: 
University of Delaware, Department of Statistics and Computer Science, 1977.
 Technical Report 77/13. 29~p.
\bibitem{18-nau}
\Au{Neuts M.\,F.} Structured stochastic matrices of $M/G/1$ type and their applications.~--- New 
York, NY, USA: Marcel Dekker, 1989. 512~p.
\bibitem{19-nau}
\Au{Lucantoni D.\,M.} New results on the single server queue with a batch Markovian arrival 
process~// Communications Statistics. Stochastic Models, 1991. Vol.~7. Iss.~1. P.~1--46. 
\bibitem{20-nau}
\Au{Narayana S., Neuts M.\,F.} The first two moment matrices of the counts for the Markovian 
arrival processes~// Communications Statistics. Stochastic Models, 1992. Vol.~8. Iss.~3. P.~459--477. 
\bibitem{21-nau}
\Au{Nielsen B.\,F., Nilsson L.\,A.\,F., Thygesen~U.\,H., Beyer~J.\,E.} Higher order moments and 
conditional asymptotics of the batch Markovian arrival process~// Stoch. Models, 2007. Vol.~23. 
Iss.~1. P.~1--26.
\bibitem{22-nau}
\Au{Бочаров П.\,П., Наумов В.\,А.} O~некоторых системах массового обслуживания 
конечной емкости~// Проб\-ле\-мы передачи информации, 1977. Т.~13. №\,4. С.~96--104.
\bibitem{23-nau}
\Au{Наумов В.\,А.} Об однолинейной системе с ограниченным накопителем и заявками 
нескольких видов~// Модели систем распределения информации и их анализ.~--- М.: 
Наука, 1982. C.~77--82.
\bibitem{24-nau}
\Au{Наумов В.\,А.} О~функциях распределения с рациональным преобразованием  
Лап\-ла\-са--Стилть\-еса~// Анализ информационно-вычислительных систем.~--- М.: 
УДН, 1986. С.~47--56.
\bibitem{25-nau}
\Au{Бочаров П.\,П., Печинкин~А.\,В.} Теория массового обслуживания.~--- М.: РУДН, 
1995. 528~с.
\bibitem{26-nau}
\Au{Bocharov P.\,P., D'Apice~C., Pechinkin~A.\,V., Salerno~S.} Queueing theory.~--- Utrecht--Boston: 
VSP, 2004. 446~p.
\bibitem{27-nau}
\Au{Asmussen S., Bladt M.} Renewal theory and queueing algorithms for matrix-exponential 
distributions~// Matrix-analytic methods in stochastic models~/
Eds. A.\,S.~Alfa, S.~Chakravarty.~--- New York, NY, USA: Marcel 
Dekker, 1996. P.~313--341.
\bibitem{28-nau}
\Au{Cox D.\,R.} A use of complex probabilities in the theory of stochastic processes~// Math. 
Proc. Cambridge, 1955. Vol.~51. Iss.~2. P.~313--319. 
\bibitem{29-nau}
\Au{O'Cinneide C.\,A.} Characterization of phase-type distributions~// Communications Statistics. 
Stochastic Models, 1990. Vol.~6. Iss.~1. P.~1--57.
\bibitem{30-nau}
\Au{Bodrog L., Horv$\acute{\mbox{a}}$th~A., Telek~M.} On the properties of moments of matrix 
exponential distributions and matrix exponential processes~// Dagstuhl Seminar Proceedings, 2008. 
Vol.~07461. Paper~1394.
\bibitem{31-nau}
\Au{Asmussen S., Bladt M.} Point processes with finite-dimensional conditional probabilities~// 
Stoch. Proc. \mbox{Appl.}, 1999. Vol.~82. Iss.~1. P.~127--142.
\bibitem{32-nau}
\Au{Horvath G., Telek M.} Acceptance-rejection methods for generating random variables from 
matrix exponential distribution and rational arrival processes~// Matrix-analytic methods in stochastic 
models~/ Eds. G.~Latouche, V.~Ramaswami, J.~Sethuraman, \textit{et al.}~--- 
New York, NY, USA: Springer, 2012. P.~123--144.
\bibitem{33-nau}
\Au{Erlang A.\,K.} \mbox{L{\!\ptb{\o}}sning} af nogle Problemer fra Sandsynlighedsregningen af 
Betydning for de automatiske Telefoncentraler~// Elektroteknikeren, 1917. Iss.~13. P.~5--13.
\bibitem{34-nau}
\Au{Neuts M.\,F.} Probability distribution of phase type~// Liber Amicorum Professor Emeritus 
H.~Florin.~--- Ottignies-Louvain-la-Neuve, Belgium: University of Louvain, Department of Mathematics, 
1975. P.~173--206.
\bibitem{35-nau}
\Au{Bartlett M.\,S.} Negative probability~// Math. Proc. Cambridge, 1945. Vol.~41. Iss.~1. P.~71--73.
\bibitem{36-nau}
\Au{Cox D.\,R.} The analysis of non-Markovian stochastic processes by the inclusion of 
supplementary variables~// Math. Proc. Cambridge, 1955. Vol.~51. Iss.~3. P.~433--441. 
\bibitem{37-nau}
\Au{Bladt M., Neuts M.\,F.} Matrix-exponential distributions: Calculus and interpretations via flows~// 
Stoch. Models, 2003. Vol.~19. Iss.~1. P.~113--124.
\bibitem{38-nau}
\Au{Khrennikov A.} Interpretations of probability.~--- 2nd ed.~--- Berlin: Walter de Gruyter, 2009. 
237~p.
\bibitem{39-nau}
\Au{Наумов В.\,А.} Марковские модели потоков требований~// Системы массового 
обслуживания и информатика.~--- М.: УДН, 1987. С.~67--73.
\bibitem{40-nau}
\Au{Asmussen S.} Matrix-analytic models and their analysis~// Scand. J.~Stat., 2000. 
Vol.~27. Iss.~2. P.~193--226.
\bibitem{41-nau}
\Au{Bladt M.} A~review on phase-type distributions and their use in risk theory~// ASTIN Bull., 
2005. Vol.~35. No.\,1. P.~145--161.
\bibitem{42-nau}
\Au{Artalejo J.\,R., G$\acute{\mbox{o}}$mez-Corral~A.} Markovian arrivals in stochastic 
modelling: A~survey and some new results~// SORT~--- Stat. Oper. Res.~T., 2010. 
Vol.~34. Iss.~2. P.~101--144.
\bibitem{43-nau}
\Au{Вишневский В.\,М., Дудин~А.\,Н.} Системы массового обслуживания с 
коррелированными входными потоками и их применение для моделирования 
телекоммуникационных сетей~// Автоматика и телемеханика, 2017. №\,8. С.~3--59.
\bibitem{44-nau}
\Au{Basharin G., Naumov~V., Samouylov~K.} On Markovian modelling of arrival processes~// 
Stat. Pap., 2018. Vol.~59. Iss.~4. P.~1533--1540. 
\bibitem{45-nau}
\Au{Neuts M.\,F.} Matrix-geometric solutions in stochastic models: An algorithmic approach.~--- 
Baltimore, MA, USA: The John Hopkins University Press, 1981. 332~p.
\bibitem{46-nau}
\Au{Latouche G., Ramaswami~V.} Introduction to matrix analytic methods in stochastic modeling.~--- 
Philadelphia, PA, USA: ASA \& SIAM, 1999. 334~p.
\bibitem{47-nau}
\Au{Asmussen S.} Applied probability and queues.~--- New York, NY, USA: Springer, 2003. 
438~p.
\bibitem{48-nau}
\Au{Breuer L., Baum D.} An introduction to queueing theory and matrix-analytic methods.~--- 
Dordrecht: Springer, 2005. 272~p.
\bibitem{49-nau}
\Au{Bini D.\,A., Latouche~G., Meini~B.} Numerical methods for structured Markov chains.~--- 
New York, NY, USA: Oxford University Press, 2005. 336~p.
\bibitem{50-nau}
\Au{Asmussen S., O'Cinneide~C.\,A.} Matrix-exponential distributions~// Encyclopedia of statistical 
sciences~/ Eds. S.~Kotz, C.\,B.~Read, N.~Balakrishnan, 
B.~Vidakovic, N.\,L.~Johnson.~--- Hoboken, NJ, USA: John Wiley \& Sons, 2006. Vol.~3. P.~1--5.
doi: 10.1002/0471667196.ess1092.
\bibitem{51-nau}
\Au{Li Q.-L.} Constructive computation in stochastic models with applications.~--- Berlin: Springer-Verlag, 2009. 650~p.
\bibitem{52-nau}
\Au{Lipsky L.} Queueing theory: A~linear algebraic approach.~--- 2nd ed.~--- New York, NY, 
USA: Springer, 2009. 548~p.
\bibitem{53-nau}
\Au{Alfa A.\,S.} Queueing theory for telecommunications.~--- New York, NY, USA: Springer, 2010. 
238~p.
\bibitem{54-nau}
\Au{He Q.-M.} Fundamentals of matrix-analytic methods.~--- New York, NY, USA: Springer, 2014. 
349~p.
\bibitem{55-nau}
\Au{Buchholz P., Kriege~J., Felko~I.} Input modeling with phase-type distributions and Markov 
models. Theory and applications.~--- New York, NY, USA: Springer, 2014. 127~p.
\bibitem{56-nau}
\Au{Наумов В.\,А., Самуйлов~В.\,А., Гайдамака~Ю.\,В.} Мультипликативные решения 
конечных цепей Маркова.~--- М.: РУДН, 2015. 159~с.
\bibitem{57-nau}
\Au{Bladt M., Nielsen B.\,F.} Matrix-exponential distributions in applied probability.~--- Boston, MA, USA: 
Springer, 2017. 736~p.
\end{thebibliography}

 }
 }

\end{multicols}

\vspace*{-12pt}

\hfill{\small\textit{Поступила в~редакцию 02.07.20}}

\vspace*{8pt}

%\pagebreak

\newpage

\vspace*{-28pt}

%\hrule

%\vspace*{2pt}

%\hrule

%\vspace*{-2pt}

\def\tit{ON MARKOVIAN AND RATIONAL ARRIVAL PROCESSES.~II}


\def\titkol{On Markovian and rational arrival processes.~II}


\def\aut{V.\,A.~Naumov$^1$ and~К.\,Е.~Samouylov$^{2,3}$}

\def\autkol{V.\,A.~Naumov and~К.\,Е.~Samouylov}

\titel{\tit}{\aut}{\autkol}{\titkol}

\vspace*{-11pt}


   \noindent
   $^1$Service Innovation Research Institute, 8A Annankatu, Helsinki 00120, Finland

\noindent
$^2$Peoples' Friendship University of Russia (RUDN University), 6~Miklukho-Maklaya Str., Moscow 
117198, Russian\linebreak
$\hphantom{^1}$Federation

\noindent
$^3$Institute of Informatics Problems, Federal Research Center ``Computer Science and Control'' 
of the Russian\linebreak
$\hphantom{^1}$Academy of Sciences, 44-2~Vavilov Str., Moscow 119333, Russian Federation

  

\def\leftfootline{\small{\textbf{\thepage}
\hfill INFORMATIKA I EE PRIMENENIYA~--- INFORMATICS AND
APPLICATIONS\ \ \ 2020\ \ \ volume~14\ \ \ issue\ 4}
}%
 \def\rightfootline{\small{INFORMATIKA I EE PRIMENENIYA~---
INFORMATICS AND APPLICATIONS\ \ \ 2020\ \ \ volume~14\ \ \ issue\ 4
\hfill \textbf{\thepage}}}

\vspace*{3pt} 
  
  
   
   
  \Abste{This article is the second part of the review carried out within the framework of the RFBR 
project No.\,19-17-50126. The purpose of this review is to get the interested readers familiar with the 
basics of the theory of Markovian arrival processes to facilitate the application of these models in practice 
and, if necessary, to study them in detail. In the first part of the review, the properties of the general 
Markovian arrival processes are presented and their relationship with Markov additive processes and 
Markov renewal processes is shown. In the second part of the review, the important for applications 
subclasses of Markovian arrival processes, i.\,e., simple and batch arrival processes of homogeneous and 
heterogeneous arrivals, are considered. It is shown how the properties of Markovian arrival processes are 
associated with the product form of stationary distributions of Markov systems. In conclusion, 
matrix-exponential distributions and rational arrival processes are discussed that expand the capabilities of 
Markovian arrival processes for modeling complex systems, while preserving the convenience of analyzing 
them using computations.}
  
  \KWE{Markov chain; Markovian arrival process; Markov additive process; MAP; MArP}
  
  
\DOI{10.14357/19922264200406} 

%\vspace*{-20pt}

  \Ack
  \noindent
  The reported study was funded by RFBR, project No.\,19-17-50126. 
  

%\vspace*{6pt}

  \begin{multicols}{2}

\renewcommand{\bibname}{\protect\rmfamily References}
%\renewcommand{\bibname}{\large\protect\rm References}

{\small\frenchspacing
 {%\baselineskip=10.8pt
 \addcontentsline{toc}{section}{References}
 \begin{thebibliography}{99}
  
  \bibitem{1-nau-1}
  \Aue{Naumov, V.\,A.} 1976. O~nezavisimoy rabote podsistem slozhnoy sistemy [About independent 
operation of subsystems of a complex system]. \textit{Tr. III Vsesoyuznoy shkoly-seminara po teorii 
massovogo obsluzhivaniya} [3th All-Russian School-Seminar of Queuing Theory Proceedings]. 
Moscow. 2:169--177.
  \bibitem{2-nau-1}
  \Aue{Bocharov, P.\,P., and V.\,A.~Naumov.} 1976. Analiz gipereksponentsial'noy dvukhfaznoy sistemy 
s~ogranichennym nakopitelem [Analysis of a hyperexponential two-phase system with a limited storage]. 
\textit{Informatsionnye seti i~ikh struktura} [Information networks and their structure]. Moscow: 
Nauka. 
  168--180.
  \bibitem{3-nau-1}
  \Aue{Naumov, V.\,A.} 1977. Ob obsluzhennoy i~izbytochnoy nagruzkakh polnodostupnogo puchka 
s~ogranichennoy ochered'yu [About serviced and excessive loads of a fully accessible bundle with a limited 
queue]. \textit{Chislennye metody resheniya zadach matematicheskoy fiziki i~teorii system} 
[Numerical methods for solving problems of mathematical physics and systems theory]. Moscow: UDN. 
51--55.
  \bibitem{4-nau-1}
  \Aue{Naumov, V.\,A.} 1978. Issledovanie nekotorykh mnogofaznykh sistem massovogo obsluzhivaniya 
[Research of some multiphase queuing systems].  Moscow: UDN.  PhD Thesis. 98~p.
  \bibitem{5-nau-1}
  \Aue{Lucantoni, D.\,M., K.~Meier-Hellstern, and M.\,F.~Neuts.} 1990. A single-server queue with 
server vacations and a~class of non-renewal arrival processes. \textit{Adv. Appl. Probab.} 
22(3):676--705.
  \bibitem{6-nau-1}
  \Aue{Basharin, G.\,P., V.\,A.~Kokotushkin, and V.\,A.~Naumov.} 1979. O~metode ekvivalentnykh 
zamen rascheta fragmentov setey svyazi dlya TsVM [On the method of equivalent substitutions for 
calculating fragments of communication networks for a central computer]. \textit{Engineering Cybernetics}
 6:92--99.
  \bibitem{7-nau-1}
  \Aue{Basharin, G.\,P., and V.\,A.~Naumov.} 1984. Simple matrix description of peaked and smooth 
traffic and its applications. \textit{3rd ITC Specialist Seminar on Fundamentals of Teletraffic Theory}. 
Moscow: VINITI. 
  38--44. 
  \bibitem{8-nau-1}
  \Aue{Neuts, M.\,F.} 1978. Renewal processes of phase type. \textit{Nav. Res. Logist.~Q.} 25(3):445--454.
  \bibitem{9-nau-1}
  \Aue{Cinlar, E.} 1967. Queues with semi-Markovian arrivals. \textit{J.~Appl. Probab.} 4(2):365--379.
  \bibitem{10-nau-1}
  \Aue{Franken, P.} 1968. Erlangsche Formeln f$\ddot{\mbox{u}}$r semimarkowschen Eingang. 
\textit{Elektronische Informationsverarbeitung  Kybernetik} 4(3):197--204.
  \bibitem{11-nau-1}
  \Aue{Franken, P., and J.~Kerstan.} 1968. Bedienungssysteme mit unendlich vielen 
Bedienungsapparaten. \textit{Operationsforschung Mathematische Statistik} 1:67--76.
  \bibitem{12-nau-1}
  \Aue{Neuts, M.\,F., and S.-Z.~Chen.} 1972. The infinite server queue with semi-Markovian arrivals 
and negative exponential services. \textit{J.~Appl. Probab.} 9(1):178--184.
  \bibitem{13-nau-1}
  \Aue{Bean, N.\,G., D.\,A.~Green, and P.\,G.~Taylor.} 1996. When is a MAP poisson? \textit{2nd 
  Australia--Japan Workshop on Stochastic Models in Engineering, Technology and Management 
Proceedings}. Eds. J.~Wilson, D.\,N.\,P.~Murthy, and S.~Osaki. 
Brisbane: Technology Management Center, University of Queensland. 34--43.
  \bibitem{14-nau-1}
  \Aue{Naumov, V.\,A.} 1988. Matrichnyy analog formuly Erlanga [The matrix analogue of a formula of 
Erlang]. \textit{Modeli raspredeleniya informatsii i~metody ikh analiza} [Information distribution 
models and methods for their analysis]. Moscow: VINITI. 39--43.
  \bibitem{15-nau-1}
  \Aue{He, Q.-M.} 1996. Queues with marked customers. \textit{Adv. Appl. Probab.} 
28(2):567--587.
  \bibitem{16-nau-1}
  \Aue{He, Q.-M., and M.\,F. Neuts.} 1998. Markov chains with marked transitions. \textit{Stoch. 
Proc. Appl.} 74:37--52.
  \bibitem{17-nau-1}
  \Aue{Neuts, M.\,F.} 1977. A~versatile Markovian point process.  
Newark, DE: University of Delaware, Department of Statistics and Computer Science.
Technical Report 77/13. 29~p.
  \bibitem{18-nau-1}
  \Aue{Neuts, M.\,F.} 1989. \textit{Structured stochastic matrices of $M/G/1$ type and their 
applications}. New York, NY: Marcel Dekker. 512~p.
  \bibitem{19-nau-1}
  \Aue{Lucantoni, D.\,M.} 1991. New results on the single server queue with a batch Markovian arrival 
process. \textit{Communications Statistics. Stochastic Models} 7(1):1--46. 
  \bibitem{20-nau-1}
  \Aue{Narayana, S., and M.\,F.~Neuts.} 1992. The first two moment matrices of the counts for the 
Markovian arrival processes. \textit{Communications Statistics. Stochastic Models} 8(3):459--477. 
  \bibitem{21-nau-1}
  \Aue{Nielsen, B.\,F., L.\,A.\,F.~Nilsson, U.\,H.~Thygesen, and J.\,E.~Beyer}. 2007. Higher order 
moments and conditional asymptotics of the batch Markovian arrival process. \textit{Stoch. Models} 
23(1):1--26.
  \bibitem{22-nau-1}
  \Aue{Bocharov, P.\,P., and V.\,A.~Naumov.} 1977. O~nekotorykh sistemakh massovogo 
obsluzhivaniya konechnoy emkosti [On some queueing systems of finite capacity]. \textit{Problemy 
peredachi informatsii} [Problems of Information Transmission] 13(4):96--104.
  \bibitem{23-nau-1}
  \Aue{Naumov, V.\,A.} 1982. Ob odnolineynoy sisteme s~ogranichennym nakopitelem i~zayavkami 
neskol'kikh vidov [About a single-line system with limited storage and multiple types of requests]. 
\textit{Modeli sistem raspredeleniya informatsii i~ikh analiz} [Models of information distribution 
systems and methods for their analysis]. Moscow: Nauka. 77--82.
  \bibitem{24-nau-1}
  \Aue{Naumov, V.\,A.} 1986. O~funktsiyakh raspredeleniya s~ratsio\-nal'nym preobrazovaniem  
Laplasa--Stilt'esa [On distribu\-tion functions with rational Laplace--Stiltjes transformation]. \textit{Analiz 
  informatsionno-vychislitel'nykh \mbox{system}}
   [\mbox{Analysis} of information and computing systems]. Moscow: 
UDN. 47--56.
  \bibitem{25-nau-1}
  \Aue{Bocharov, P.\,P., and A.\,V.~Pechinkin.} 1995. \textit{Teoriya massovogo obsluzhivaniya} 
[Queueing theory]. Moscow: RUDN. 528~p.
  \bibitem{26-nau-1}
  \Aue{Bocharov, P.\,P., C.~D'Apice, A.\,V.~Pechinkin, and S.~Salerno.} 2004. \textit{Queueing 
theory}. Utrecht--Boston: VSP. 446~p.
  \bibitem{27-nau-1}
  \Aue{Asmussen, S., and M.~Bladt}. 1996. Renewal theory and queueing algorithms for 
  matrix-exponential distributions. \textit{Matrix-analytic methods in stochastic models}. 
  Eds. A.\,S.~Alfa and 
S.~Chakravarty. New York, NY: Marcel Dekker. 313--341.
  \bibitem{28-nau-1}
  \Aue{Cox, D.\,R.} 1955. A~use of complex probabilities in the theory of stochastic processes. 
\textit{Math. Proc. Cambridge} 51(2):313--319.
  \bibitem{29-nau-1}
  \Aue{O'Cinneide, C.\,A.} 1990. Characterization of phase-type distributions. \textit{Communications 
Statistics. Stochastic Models} 6(1):1--57.
  \bibitem{30-nau-1}
  \Aue{Bodrog, L., A.~Horv$\acute{\mbox{a}}$th, and M.~Telek.} 2008. On the properties of 
moments of matrix exponential distributions and matrix exponential processes. 
\textit{Dagstuhl Seminar Proceedings} 07461:1394.
  \bibitem{31-nau-1}
  \Aue{Asmussen, S., and M.~Bladt.} 1999. Point processes with finite-dimensional conditional 
probabilities. \textit{Stoch. Proc. Appl.} 82(1):127--142.
  \bibitem{32-nau-1}
  \Aue{Horvath, G., and M.~Telek.} 2012. Acceptance-rejection methods for generating random 
variables from matrix exponential distribution and rational arrival processes. \textit{Matrix-analytic 
methods in stochastic models.} Eds. G.~Latouche, V.~Ramaswami, J.~Sethuraman, \textit{et al.} New York, NY: Springer. 123--144.
  \bibitem{33-nau-1}
  \Aue{Erlang, A.\,K.} 1917. \mbox{L{\!\ptb{\o}}sning} af nogle Problemer fra 
Sandsynlighedsregningen af Betydning for de automatiske Telefoncentraler. \textit{Elektroteknikeren} 
13:5--13.
  \bibitem{34-nau-1}
  \Aue{Neuts, M.\,F.} 1975. Probability distribution of phase type. \textit{Liber Amicorum Professor 
Emeritus H.~Florin}. Ottignies-Louvain-la-Neuve, Belgium: University of Louvain, Department of 
Mathematics.  
173--206.
  \bibitem{35-nau-1}
  \Aue{Bartlett, M.\,S.} 1945. Negative probability. \textit{Math. Proc. 
Cambridge} 41(1):71--73.
  \bibitem{36-nau-1}
  \Aue{Cox, D.\,R.} 1955. The analysis of non-Markovian stochastic processes by the inclusion of 
supplementary variables. \textit{Math. Proc. Cambridge} 
51(3):433--441.
  \bibitem{37-nau-1}
  \Aue{Bladt, M., and M.\,F.~Neuts.} 2003. Matrix-exponential distributions: Calculus and 
interpretations via flows. \textit{Stoch. Models} 19(1):113--124.
  \bibitem{38-nau-1}
  \Aue{Khrennikov, A.} 2009. \textit{Interpretations of probability}. 2nd ed. Berlin: Walter de 
Gruyter. 237~p.
  \bibitem{39-nau-1}
  \Aue{Naumov, V.\,A.} 1987. Markovskie modeli potokov trebovaniy [Markov models of demand 
flows]. \textit{Sistemy massovogo obsluzhivaniya i~informatika} [Queuing systems and computer 
science]. Moscow: UDN. 67--73.
  \bibitem{40-nau-1}
  \Aue{Asmussen, S.} 2000. Matrix-analytic models and their analysis. \textit{Scand. 
J.~Stat.} 27(2):193--226.
  \bibitem{41-nau-1}
  \Aue{Bladt, M.} 2005. A~review on phase-type distributions and their use in risk theory. \textit{ASTIN 
Bull.} 35(1):145--161.
  \bibitem{42-nau-1}
  \Aue{Artalejo, J.\,R., and A.~G$\acute{\mbox{o}}$mez-Corral.} 2010. Markovian arrivals in 
stochastic modelling: A~survey and some new results. \textit{SORT~--- Stat. Oper. Res.~T.}  
34(2):101--144.
  \bibitem{43-nau-1}
  \Aue{Vishnevskiy, V.\,M., and A.\,N.~Dudin.} 2017. Queueing systems with correlated arrival flows 
and their applications to modeling telecommunication networks. \textit{Automat. Rem. Contr.} 
78(8):1361--1403.
  \bibitem{44-nau-1}
  \Aue{Basharin, G., V.~Naumov, and K.~Samouylov.} 2018. On Markovian modelling of arrival 
processes. \textit{Stat. Pap.} 59(4):1533--1540.
  \bibitem{45-nau-1}
  \Aue{Neuts, M.\,F.} 1981. \textit{Matrix-geometric solutions in stochastic models: An algorithmic 
approach.} Baltimore, MA: The John Hopkins University Press. 332~p.
  \bibitem{46-nau-1}
  \Aue{Latouche, G., and V.~Ramaswami.} 1999. \textit{Introduction to matrix analytic methods in 
stochastic modeling}. Philadelphia, PA: ASA \& SIAM. 334~p.
  \bibitem{47-nau-1}
  \Aue{Asmussen, S.} 2003. \textit{Applied probability and queues}. New  York, NY: Springer. 
438~p.
  \bibitem{48-nau-1}
  \Aue{Breuer, L., and D.~Baum.} 2005. \textit{An introduction to queueing theory and 
  matrix-analytic methods.} Dordrecht: Springer. 272~p.
  \bibitem{49-nau-1}
  \Aue{Bini, D.\,A., G.~Latouche, and B.~Meini.} 2005. \textit{Numerical methods for structured 
Markov chains}. New  York, NY: Oxford University Press. 336~p.
  \bibitem{50-nau-1}
  \Aue{Asmussen, S., and C.\,A.~O'Cinneide}. 2006. Matrix-exponential distributions. 
\textit{Encyclopedia of statistical sciences.} Eds. S.~Kotz, C.\,B.~Read, N.~Balakrishnan, 
B.~Vidakovic, and N.\,L.~Johnson. Hoboken, NJ: John Wiley \&~Sons. 3:1--5. doi: 10.1002/0471667196.ess1092.pub2.
  \bibitem{51-nau-1}
  \Aue{Li, Q.-L.} 2009. \textit{Constructive computation in stochastic models with applications}. 
Berlin: 
  Springer-Verlag. 650~p.
  \bibitem{52-nau-1}
  \Aue{Lipsky, L.} 2009. \textit{Queueing theory: A~linear algebraic approach}. 2nd ed. New York, 
NY: Springer. 548~p.
  \bibitem{53-nau-1}
  \Aue{Alfa, A.\,S.} 2010. \textit{Queueing theory for telecommunications}. New York, NY: 
Springer. 238 p.
  \bibitem{54-nau-1}
  \Aue{He, Q.-M.} 2014. \textit{Fundamentals of matrix-analytic methods.} New York, NY: 
Springer. 349 p.
  \bibitem{55-nau-1}
  \Aue{Buchholz, P., J.~Kriege, and I.~Felko.} 2014. \textit{Input modeling with phase-type 
distributions and Markov models. Theory and applications.} New York, NY: Springer. 127~p.
  \bibitem{56-nau-1}
  \Aue{Naumov, V.\,A., K.\,E.~Samuylov, and Yu.\,V.~Gaidamaka.} 2015. \textit{Mul'tiplikativnye 
resheniya konechnykh tsepey Markova} [Multiplicative solutions of finite Markov chains]. Moscow: 
RUDN. 159~p.
  \bibitem{57-nau-1}
  \Aue{Bladt, M., and B.\,F.~Nielsen.} 2017. \textit{Matrix-exponential distributions in applied 
probability}. Boston, MA: Springer. 736~p.
\end{thebibliography}

 }
 }

\end{multicols}

\vspace*{-3pt}

\hfill{\small\textit{Received July 2, 2020}}

%\pagebreak

  %\vspace*{-24pt}
  
  \Contr
  
  \noindent
  \textbf{Naumov Valeriy A.} (b.\ 1950)~--- Candidate of Science (PhD) in physics and mathematics, 
scientific director, Service Innovation Research Institute, 8A~Annankatu, Helsinki 00120, Finland; 
\mbox{valeriy.naumov@pfu.fi}
  
  \vspace*{3pt}
  
  \noindent
  \textbf{Samouylov Konstantin E.} (b.\ 1955)~--- Doctor of Science in technology, professor, Head of 
Department,  Peoples' Friendship 
University of Russia (RUDN University), 6~Miklukho-Maklaya Str., Moscow 117198, Russian 
Federation; senior scientist, Institute of Informatics Problems, Federal Research Center ``Computer 
Science and Control'' of the Russian Academy of Sciences, 44-2~Vavilov Str., Moscow 119333, Russian 
Federation; 
  \mbox{samuylov-ke@rudn.university}
  
\label{end\stat}

\renewcommand{\bibname}{\protect\rm Литература} 
  
   %2
\def\stat{ushakov+leon}

\def\tit{АНАЛИЗ СИСТЕМЫ ОБСЛУЖИВАНИЯ С~ВХОДЯЩИМ ПОТОКОМ АВТОРЕГРЕССИОННОГО 
ТИПА И~ОТНОСИТЕЛЬНЫМ ПРИОРИТЕТОМ$^*$}

\def\titkol{Анализ системы обслуживания с~входящим потоком авторегрессионного 
типа и~относительным приоритетом}

\def\aut{Н.\,Д.~Леонтьев$^1$, В.\,Г.~Ушаков$^2$}

\def\autkol{Н.\,Д.~Леонтьев, В.\,Г.~Ушаков}

\titel{\tit}{\aut}{\autkol}{\titkol}

\index{Леонтьев Н.\,Д.}
\index{Ушаков В.\,Г.}
\index{Leontyev N.\,D.}
\index{Ushakov V.\,G.}


{\renewcommand{\thefootnote}{\fnsymbol{footnote}} \footnotetext[1]
{Работа выполнена при финансовой поддержке РФФИ (проект 15-07-02354).}}


\renewcommand{\thefootnote}{\arabic{footnote}}
\footnotetext[1]{Факультет вычислительной математики и~кибернетики Московского 
государственного университета им.\ М.\,В.~Ломоносова, \mbox{ndleontyev@gmail.com}}
\footnotetext[2]{Факультет вычислительной математики и~кибернетики 
Московского государственного университета им.\ М.\,В.~Ломоносова; 
Институт проблем информатики ФИЦ ИУ РАН, \mbox{vgushakov@mail.ru}}

\vspace*{-12pt}

\Abst{Рассматривается одноканальная система массового обслуживания 
с~неограниченным числом мест для ожидания, в~которую поступают два потока требований: 
первый поток~--- пуассоновский, а~второй~--- неординарный пуассоновский (т.\,е.\ 
пуассоновский поток групп требований). Требования из первого потока имеют 
относительный приоритет перед требованиями второго потока. Особенностью системы 
является авторегрессионная зависимость размеров групп требований второго потока: 
размер $n$-й поступившей в~систему группы требований либо с~некоторой 
фиксированной вероятностью равен размеру $(n-1)$-й поступившей в~систему группы 
требований, либо с~дополнительной вероятностью является независимой от него 
случайной величиной. Длительности обслуживания требований каждого потока являются 
независимыми случайными величинами с~произвольным распределением. Найдена 
производящая функция совместного распределения числа требований каждого 
потока в~системе в~произвольный момент времени.}


\KW{теория массового обслуживания; нестационарный режим; системы с~групповым 
поступлением требований; относительный приоритет}

\DOI{10.14357/19922264160303} 

%\vspace*{-3pt}
  

\vskip 12pt plus 9pt minus 6pt

\thispagestyle{headings}

\begin{multicols}{2}

\label{st\stat}

\section{Введение}

При моделировании передачи данных в~телекоммуникационных сетях важно учитывать 
природу потоков информации и~характеристики потоков в~зависимости от приложений. 
В~простейших моделях предполагается, что все пакеты информации имеют фиксированную 
конечную длину, а~размеры пакетов независимы. Однако в~ряде случаев необходимо 
рассматривать более сложные конструкции входящего потока, которые позволяют 
учитывать неоднородную и~коррелированную природу потоков данных. 
В~данной работе рассматривается сис\-те\-ма массового обслуживания типа $M|G|1$ 
с~двумя потоками требований, один из которых является приоритетным, 
а~другой~--- коррелированным потоком групп требований.

Настоящая работа обобщает результаты \mbox{статьи}~[1] на случай двух потоков, 
требования одного из которых имеют относительный приоритет.

\vspace*{-9pt}

\section{Описание системы}

Рассматривается одноканальная система обслуживания, в~которую поступают 
два потока требований: 
\begin{itemize}
\item первый поток~--- пуассоновский с~интенсив\-ностью~$a_1$, 
\item второй поток является пуассоновским потоком групп требований.
\end{itemize}
Интенсивность 
поступления групп требований равна~$a_2$. Группа состоит из случайного числа 
требований и~содержит~$k$ требований с~вероятностью~$h_k$, $k\hm=1,\ldots,M$. 
Между размерами двух последо\-ва\-тельно по\-сту\-па\-ющих групп требований имеется 
следующая зависимость: размер $n$-й по\-сту\-па\-ющей группы требований либо 
с~ве\-ро\-ят\-ностью $0\hm\leq p\hm<1$ равен размеру $(n-1)$-й группы, либо 
с~ве\-ро\-ят\-ностью $1-p$ является независимой от него случайной величиной. Требования 
из первого потока имеют относительный приоритет по отношению к~требованиям из 
второго потока. Иными словами, прерывание уже начатого обслуживания не допускается 
и~требования из второго потока могут поступать на обслуживание только при отсутствии 
в~очереди требований из первого по\-тока. 
%
Будем считать, что число мест для ожидания 
не ограничено, а~длительность обслуживания требований $i$-го потока имеет функцию 
распределения $B_i(x)$ с~плотностью $b_i(x)$ и~преобразованием Лапласа~$\beta_i(s)$.

\pagebreak

Определим следующие случайные процессы:
\begin{description}
\item[\,] $I(t)$ --- номер потока, требование из которого обслуживается в~момент~$t$;
\item[\,]$L_i(t)$ --- число требований $i$-го потока в~системе в~момент времени~$t$;
\item[\,]$X(t)$ --- время, прошедшее с~начала обслуживания требования, находящегося на 
обслуживании в~момент~$t$\footnote{В~случае, когда система свободна, можно 
для определенности положить $I(t)\hm=0$ и~$X(t)\hm=0$.};
\item[\,]$N(t)$ --- размер последней поступившей в~систему до момента~$t$ 
группы требований $2$-го потока.
\end{description}
Положим
\begin{align*}
P_i(n_1,n_2,k,x,t)&=\fr{\partial}{\partial x}
\mathbf{P}\left(I(t)=i,L_1(t)=n_1,\right.\\
&\hspace*{14pt}\left.L_2(t)=n_2,N(t)=k,X(t)<x\right);\\
P(0,k,t)&=\mathbf{P}(L_1(t)=0,L_2(t)=0,N(t)=k).\hspace*{-2pt}
\end{align*}

Обозначим
\begin{align*}
\pi_1(z_1,z_2,k,x,s)&=\\
&\hspace*{-46pt}{}=\sum\limits_{n_1=1}^{\infty}z_1^{n_1}
\sum\limits_{n_2=0}^{\infty}z_2^{n_2}
\int\limits_0^{\infty}e^{-st}P_1\left(n_1,n_2,k,x,t\right)\,dt\,;\\
\pi_2(z_1,z_2,k,x,s)&=\\
&\hspace*{-47pt}{}=\sum\limits_{n_1=0}^{\infty}z_1^{n_1}
\sum\limits_{n_2=1}^{\infty}z_2^{n_2}
\int\limits_0^{\infty}e^{-st}P_2\left(n_1,n_2,k,x,t\right)\,dt\,;\\
\pi_0(k,s)&=\int\limits_0^{\infty}e^{-st} P(0,k,t)\,dt
\end{align*}
при $|z_1|\leq1$, $|z_2|\hm\leq1$, $\mathrm{Re}\,(s)\hm>0$.

\section{Основные результаты}

Соотношения для определения $\pi_i(z_1,z_2,k,x,s)$  содержатся в~следующей 
основной теореме.

\smallskip

\noindent
\textbf{Теорема.}\ \textit{Функции $\pi_i(z_1,z_2,k,x,s)$ при $|z_1|\hm<1$,
$|z_2|<1$, $k=1,\ldots,M$, $x\geq 0$, $\mathrm{Re}\,(s)>0$ определяются по
следующим формулам}:
\begin{multline*}
\pi_i(z_1,z_2,k,x,s)=(1-B_i(x))\sum\limits_{n=1}^{M}C_{i,n}(z_1,z_2,s)\times{}\\
{}\times\fr{(1-p)h_k a_2 z_2^k}{\widetilde{\lambda}_n(z_2)-pa_2 z_2^k}
\exp\left(-\left(
\vphantom{\widetilde{\lambda}_n(z_2)}
s+a_1-a_1 z_1+a_2-{}\right.\right.\\
\left.\left.{}-\widetilde{\lambda}_n(z_2)\right)x\right)\,,
\end{multline*}
\textit{где}
\begin{multline*}
C_{1,n}\left(z_1,z_2,s\right)={}\\
{}=\fr{1}{1-z_1^{-1}\beta_1
\left(s+a_1-a_1 z_1+a_2-\widetilde{\lambda}_n(z_2)\right)}\times{}\\
{}\times\fr{\prod\limits_{i=1}^M(\widetilde{\lambda}_n\left(z_2\right)-pa_2 z_2^i)}
{\prod\limits_{\substack{j=1\\j\neq n}}^M(\widetilde{\lambda}_n(z_2)-
\widetilde{\lambda}_j(z_2))}
\sum\limits_{m=1}^M\fr{1}{\widetilde{\lambda}_n(z_2)-pa_2 z_2^m}\times{}\\
{}\times\left(
\vphantom{\fr{1-\tilde{\Lambda}_2^{-1}}{1-z_2^{-1}}}
b_m\left(z_1,z_2,s\right)-{}\right.\\
{}-\fr{1-z_2^{-1}\beta_2\left(s+a_1-a_1 z_1+a_2-
\widetilde{\lambda}_n(z_2)\right)}
{1-z_2^{-1}\beta_2(s+a_1-a_1 z_{1,n}(z_2,s)+a_2-\widetilde{\lambda}_n(z_2))}\times{}\\
\left.{}\times b_m\left(z_{1,n}\left(z_2,s\right),z_2,s\right)\right)\,;
\end{multline*}

\vspace*{-12pt}

\noindent
\begin{multline*}
C_{2,n}\left(z_1,z_2,s\right)={}\\
{}=\fr{1}{1-z_2^{-1}\beta_2(s+a_1-a_1 z_{1,n}
(z_2,s)+a_2-\widetilde{\lambda}_n(z_2))}\times{}\\
{}\times\fr{\prod\limits_{i=1}^M(\widetilde{\lambda}_n(z_2)-pa_2 z_2^i)}
{\prod\limits_{\substack{j=1\\j\neq n}}^M\left(\widetilde{\lambda}_n(z_2)-
\widetilde{\lambda}_j\left(z_2\right)\right)}
\sum\limits_{m=1}^M\fr{b_m(z_{1,n}(z_2,s),z_2,s)}
{\widetilde{\lambda}_n(z_2)-pa_2 z_2^m}\,;\hspace*{-0.39255pt}
\end{multline*}

\vspace*{-9pt}

\noindent
\begin{multline*}
\!\!\!\!b_m\!\left(z_1,z_2,s\right)=-\left(s+a_1-a_1 z_1+a_2\right)\!\pi_0(m,s)+h_m+{}\\
{}+\left[p\pi_0
(m,s)a_2 z_2^m+(1-p)\sum\limits_{n=1}^{M}\pi_0(n,s)h_m a_2 z_2^m\right]\,;
\end{multline*}
$\widetilde{\lambda}_1(z_2),\ldots,\widetilde{\lambda}_M(z_2)$ 
определяются из уравнения
\begin{multline*}
\prod\limits_{i=1}^M\left(pa_2 z_2^i-\widetilde{\lambda}\right)+{}\\
{}+
\sum\limits_{i=1}^M \left(1-p\right) h_i a_2 z_2^i
\prod\limits_{\substack{j=1\\j\neq i}}^M
\left(pa_2 z_2^j-\widetilde{\lambda}\right)=0\,,
\end{multline*}
а $\pi_0(1,s),\ldots,\pi_0(M,s)$~--- из системы
\begin{multline*}
\sum\limits_{m=1}^M
\prod\limits_{\substack{j=1\\
j\neq m}}^M(\widetilde{\lambda}_n-pa_2 z_{2,n}^j)
\left(
\vphantom{\tilde{\lambda}_n}
-\left(s+a_1-a_1 z_{1,n}+{}\right.\right.\\
\left.\left.{}+a_2\right)
\pi_0(m,s)+h_m+\widetilde{\lambda}_n(z_{2,n})\pi_0(m,s)\right)=0\,,
\end{multline*}
\textit{в которой $z_{1,n}\hm=z_{1,n}(z_{2,n},s)$ и~$z_{1,n}(z_2,s)$~--- 
решение функционального уравнения}
\begin{equation*}
z_{1,n}=\beta_1\left(s+a_1-a_1 z_{1,n}+a_2-\widetilde{\lambda}_n\left(z_2\right)\right)\,,
\end{equation*}
\textit{а $z_{2,n}=z_{2,n}(s)$~--- решение функционального урав\-нения}
\begin{equation*}
z_{2,n}=\beta_2\left(s+a_1-a_1 z_{1,n}(z_{2,n},s)+a_2-\widetilde{\lambda}_n
\left(z_{2,n}\right)\right)\,.
\end{equation*}

\noindent
Д\,о\,к\,а\,з\,а\,т\,е\,л\,ь\,с\,т\,в\,о\,.\ \
 Функции $P_i(n_1,n_2,k,x,t)$ и~$P(0,k,t)$ удовлетворяют соотношениям:
\begin{multline}
P_1\left(n_1,n_2,k,x+\Delta,t+\Delta\right)={}\\[3pt]
{}=P_1\left(n_1,n_2,k,x,t\right)
\left[1-(a_1+a_2+\eta_1(x))\Delta\right]+{}\\[3pt]
{}+\mathbf{1}_{\{n_1>1\}}P_1\left(n_1-1,n_2,k,x,t\right)a_1\Delta+{}\\[3pt]
{}+\mathbf{1}_{\{n_2\geq k\}}\left[
\vphantom{\sum\limits_{m=1}^{M}}
pP_1\left(n_1,n_2-k,k,x,t\right)a_2\Delta+{}\right.\\
\left.{}+(1-p)
\sum\limits_{m=1}^{M}P_1(n_1,n_2-k,m,x,t)h_k a_2\Delta\right]\,;\label{before1}
\end{multline}

\vspace*{-10pt}

\noindent
\begin{multline}
P_2\left(n_1,n_2,k,x+\Delta,t+\Delta\right)={}\\[3pt]
{}=
P_2\left(n_1,n_2,k,x,t\right)\left[1-(a_1+a_2+\eta_2(x))\Delta\right]+{}\\[3pt]
{}+\mathbf{1}_{\{n_1\geq 1\}}P_2\left(n_1-1,n_2,k,x,t\right)a_1\Delta+{}\\[3pt]
{}+\mathbf{1}_{\{n_2>k\}}\left[
\vphantom{\sum\limits_{m=1}^{M}}
pP_2(n_1,n_2-k,k,x,t)a_2\Delta+{}\right.\\
\left.{}+(1-p)
\sum\limits_{m=1}^{M}P_2\left(n_1,n_2-k,m,x,t\right)h_k a_2\Delta\right]\,;
\label{before2}
\end{multline}

\vspace*{-10pt}

\noindent
\begin{multline}
P\left(0,k,t+\Delta\right)=
P(0,k,t)\left[1-\left(a_1+a_2\right)\Delta\right]+{}\\
{}+
\int\limits_0^{\infty}P_1(1,0,k,x,t)\eta_1(x)\,dx\Delta+{}\\
{}+\int\limits_0^{\infty}P_2(0,1,k,x,t)\eta_2(x)\,dx\Delta\,;
\label{before3}
\end{multline}

\vspace*{-10pt}

\noindent
\begin{multline}
\int\limits_0^\Delta P_1\left(n_1,n_2,k,u,t+\Delta\right)\,du={}\\
{}=
\int\limits_0^{\infty}P_1\left(n_1+1,n_2,k,x,t\right)\eta_1(x)\,dx\Delta+{}\\
{}+\int\limits_0^{\infty}P_2\left(n_1,n_2+1,k,x,t\right)\eta_2(x)\,dx\Delta+{}\\
{}+
\delta_{n_1,1}\delta_{n_2,0}P(0,k,t)a_1\Delta\,;
\label{before4}
\end{multline}

%\vspace*{-12pt}

\noindent
\begin{multline}
\int\limits_0^\Delta P_2\left(0,n_2,k,u,t+\Delta\right)\,du={}\\
{}=
\int\limits_0^{\infty}P_1\left(1,n_2,k,x,t\right)\eta_1(x)\,dx\Delta+{}
\\
{}+\int\limits_0^{\infty}P_2\left(0,n_2+1,k,x,t\right)\eta_2(x)\,dx\Delta+{}\\
{}+
\delta_{n_2,k}\left[
\vphantom{\sum\limits_{m=1}^{M}}
pP(0,k,t)a_2\Delta+{}\right.\\
\left.{}+(1-p)\sum\limits_{m=1}^{M}
P(0,m,t)h_k a_2\Delta\right]\,,
\label{before5}
\end{multline}
где $\eta_i(x)=b_i(x)/(1-B_i(x))$.

Переходя к~пределу при $\Delta\to 0$ в~(\ref{before1})---(\ref{before5}), имеем:
\begin{multline}
\fr{\partial P_1(n_1,n_2,k,x,t)}{\partial t}+
\fr{\partial P_1(n_1,n_2,k,x,t)}{\partial x}={}\\[3pt]
{}=
-\left(a_1+a_2+\eta_1(x)\right)P_1\left(n_1,n_2,k,x,t\right)+{}\\[3pt]
{}+\mathbf{1}_{\{n_1>1\}}P_1\left(n_1-1,n_2,k,x,t\right)a_1+{}\\[3pt]
{}+\mathbf{1}_{\{n_2\geq k\}}\left[
\vphantom{\sum\limits_{m=1}^{M}}
pP_1\left(n_1,n_2-k,k,x,t\right)a_2+{}\right.\\
\left.{}+(1-p)\sum\limits_{m=1}^{M}
P_1\left(n_1,n_2-k,m,x,t\right)h_k a_2\right]\,;
\label{after1}
\end{multline}

\vspace*{-10pt}

\noindent
\begin{multline}
\fr{\partial P_2(n_1,n_2,k,x,t)}{\partial t}+
\fr{\partial P_2(n_1,n_2,k,x,t)}{\partial x}={}\\[3pt]
{}=
-\left(a_1+a_2+\eta_2(x)\right)P_2(n_1,n_2,k,x,t)+{}\\[3pt]
{}+\mathbf{1}_{\{n_1\geq 1\}}P_2\left(n_1-1,n_2,k,x,t\right)a_1+{}\\[3pt]
{}+\mathbf{1}_{\{n_2>k\}}\left[
\vphantom{\sum\limits_{m=1}^{M}}
pP_2\left(n_1,n_2-k,k,x,t\right)a_2+{}\right.\\
\left.{}+(1-p)\sum\limits_{m=1}^{M}
P_2\left(n_1,n_2-k,m,x,t\right)h_k a_2\right]\,;
\label{after2}
\end{multline}

\vspace*{-12pt}

\noindent
\begin{multline}
\fr{\partial P(0,k,t)}{\partial t}=
-\left(a_1+a_2\right)P(0,k,t)+{}\\
{}+\int\limits_0^{\infty}P_1(1,0,k,x,t)\eta_1(x)\,dx+{}\\
{}+\int\limits_0^{\infty}P_2(0,1,k,x,t)\eta_2(x)\,dx\,;
\label{after3}
\end{multline}

%\vspace*{-12pt}

\noindent
\begin{multline}
P_1\left(n_1,n_2,k,0,t\right)={}\\
{}=
\int\limits_0^{\infty}P_1\left(n_1+1,n_2,k,x,t\right)\eta_1(x)\,dx+{}\\
{}+\int\limits_0^{\infty}P_2\left(n_1,n_2+1,k,x,t\right)\eta_2(x)\,dx+{}\\
{}+
\delta_{n_1,1}\delta_{n_2,0}P(0,k,t)a_1\,;
\label{after4}
\end{multline}

\vspace*{-3pt}

\noindent
\begin{equation}
P_2\left(n_1,n_2,k,0,t\right)=0\,,\enskip n_1>0\,;
\label{after5}
\end{equation}

\vspace*{-12pt}

\noindent
\begin{multline}
P_2\left(0,n_2,k,0,t\right)=\int\limits_0^{\infty}P_1(1,n_2,k,x,t)\eta_1(x)\,dx+{}\\
{}+\int\limits_0^{\infty}P_2\left(0,n_2+1,k,x,t\right)
\eta_2(x)\,dx+{}\\
{}+\delta_{n_2,k}\left[
\vphantom{\sum\limits_{m=1}^{M}}
pP(0,k,t)a_2+{}\right.\\
\left.{}+(1-p)
\sum\limits_{m=1}^{M}P(0,m,t)h_k a_2\right].\label{after6}
\end{multline}

Переходя в~уравнениях (\ref{after1})---(\ref{after6}) 
к~производящим функциям и~преобразованиям Лапласа по~$t$, получим:
\begin{multline*}
\fr{\partial \pi_i(z_1,z_2,k,x,s)}{\partial x}={}\\
{}=
-\left(s+a_1-a_1 z_1+a_2+\eta_i(x)\right)\pi_i\left(z_1,z_2,k,x,s\right)+{}\\
{}+\left[
\vphantom{\sum\limits_{m=1}^{M}}
p\pi_i\left(z_1,z_2,k,x,s\right)
a_2 z_2^k+{}\right.\\
\left.{}+(1-p)\sum\limits_{m=1}^{M}
\pi_i\left(z_1,z_2,m,x,s\right)h_k a_2 z_2^k\right]\,;
%\label{first}
\end{multline*}

\vspace*{-12pt}

\noindent
\begin{multline}
\left(s+a_1+a_2\right)\pi_0(k,s)-h_k={}\\
{}=
\int\limits_0^{\infty}\int\limits_0^{\infty}P_1(1,0,k,x,t)\eta_1(x)\,dx e^{-st}\,dt+{}\\
{}+\int\limits_0^{\infty}\int\limits_0^{\infty}P_2(0,1,k,x,t)\eta_2(x)\,dx e^{-st}\,dt\,;
\label{second}
\end{multline}

\vspace*{-12pt}

\noindent
\begin{multline*}
\pi_1\left(z_1,z_2,k,0,s\right)=
z_1^{-1}\hspace*{-4pt}\int\limits_0^{\infty}
\hspace*{-4pt}\pi_1\left(z_1,z_2,k,x,s\right)\eta_1(x)\,dx-{}\hspace*{-0.78543pt}\\
{}-\sum\limits_{n_2=0}^{\infty}z_2^{n_2}
\int\limits_0^{\infty}\int\limits_0^{\infty}P_1(1,n_2,k,x,t)\eta_1(x)\,dx e^{-st}\,dt+{}\\
{}+
z_2^{-1}\int\limits_0^{\infty}\pi_2\left(z_1,z_2,k,x,s\right)\eta_2(x)\,dx-{}
\end{multline*}

\noindent
\begin{multline}
{}-z_2^{-1}\sum\limits_{n_2=1}^{\infty}\hspace*{-3pt}z_2^{n_2}
\int\limits_0^{\infty}\int\limits_0^{\infty}\!P_2(0,n_2,k,x,t)\eta_2(x)\,dx e^{-st}\,dt+{}\\
{}+
\pi_0(k,s)az_1\,;
\label{third}
\end{multline}

\vspace*{-9pt}

\noindent
\begin{equation}
\sum\limits_{n_1=1}^{\infty}z_1^{n_1}
\sum\limits_{n_2=1}^{\infty}z_2^{n_2}\int\limits_0^{\infty}
P_2\left(n_1,n_2,k,0,t\right)e^{-st}\,dt=0\,;
\label{fourth}
\end{equation}

\vspace*{-12pt}

\noindent
\begin{multline}
\sum\limits_{n_2=1}^{\infty}z_2^{n_2}
\int\limits_0^{\infty}P_2\left(0,n_2,k,0,t\right)e^{-st}\,dt={}\\
{}=
\sum\limits_{n_2=1}^{\infty}z_2^{n_2}\int\limits_0^{\infty}
\int\limits_0^{\infty}P_1(1,n_2,k,x,t)\eta_1(x)\,dxe^{-st}\,dt+{}\\
{}+z_2^{-1}\sum\limits_{n_2=1}^{\infty}z_2^{n_2}
\int\limits_0^{\infty}\int\limits_0^{\infty}P_2(0,n_2,k,x,t)\eta_2(x)\,dxe^{-st}\,dt-{}\\
{}-
\int\limits_0^{\infty}\int\limits_0^{\infty}P_2(0,1,k,x,t)\eta_2(x)\,dxe^{-st}\,dt+{}\\
\!\!\!{}+\left[p\pi_0(k,s)a_2 z_2^{k}+(1-p)\!
\sum\limits_{m=1}^M\!\pi_0(m,s)h_k a_2 z_2^k\right].\!\!\!
\label{fifth}
\end{multline}

Суммируя уравнения (\ref{second})---(\ref{fifth}), приходим к~сис\-те\-ме:
\begin{multline}
\fr{\partial \pi_i(z_1,z_2,k,x,s)}{\partial x}
={}\\
{}= -\left(s+a_1-a_1 z_1+a_2+\eta_i(x)\right)\pi_i\left(z_1,z_2,k,x,s\right)+{}\\
{}+\left[
\vphantom{\sum\limits_{m=1}^{M}}
p\pi_i\left(z_1,z_2,k,x,s\right)a_2 z_2^k+{}\right.\\
\left.{}+(1-p)\sum\limits_{m=1}^{M}
\pi_i\left(z_1,z_2,m,x,s\right)h_k a_2 z_2^k\right]\,;
\label{newfirst}
\end{multline}

\vspace*{-12pt}

\noindent
\begin{multline}
\pi_1\left(z_1,z_2,k,0,s\right)+
\pi_2\left(z_1,z_2,k,0,s\right)={}\\
{}=
z_1^{-1}\int\limits_0^{\infty}\pi_1\left(z_1,z_2,k,x,s\right)\eta_1(x)\,dx+{}\\
{}+z_2^{-1}\int\limits_0^{\infty}\pi_2(z_1,z_2,k,x,s)\eta_2(x)\,dx-{}\\
{}-
\left(s+a_1-a_1 z_1+a_2\right)\pi_0(k,s)+h_k+{}\\
\!\!\!{}+\left[p\pi_0(k,s)a_2 z_2^k+(1-p)\!\sum\limits_{m=1}^{M}\!
\pi_0(m,s)h_k a_2 z_2^k\right].\!\!
\label{newsecond}
\end{multline}

\vspace*{-12pt}

\pagebreak

Обозначим
$$
\pi_i\left(z_1,z_2,k,x,s\right)=\left(1-B_i(x)\right)\widetilde{\pi}_i\left(z_1,z_2,k,x,s\right).
$$
В новых обозначениях~(\ref{newfirst}) примет вид:
\begin{multline}
\fr{\partial\widetilde{\pi}_i(z_1,z_2,k,x,s)}{\partial x}={}\\
{}=
-\left(s+a_1-a_1 z_1+a_2\right)\widetilde{\pi}_i\left(z_1,z_2,k,x,s\right)+{}\\
{}+\left[
\sum\limits_{m=1}^{M}
p\widetilde{\pi}_i\left(z_1,z_2,k,x,s\right)a_2 z_2^k+{}\right.\\
\left.{}+(1-p)
\sum\limits_{m=1}^{M}\widetilde{\pi}_i\left(z_1,z_2,m,x,s\right)
h_k a_2 z_2^k\right].
\label{system}
\end{multline}
Это линейная система дифференциальных уравнений первого порядка 
с~постоянными коэффициентами, решение которой можно записать в~виде:
\begin{multline}
\widetilde{\pi}_i\left(z_1,z_2,k,x,s\right)={}\\
{}=\sum\limits_{n=1}^{M}C_{i,n}\left(z_1,z_2,s\right)u_{kn}(z_2)
\exp\left(-\left(
\vphantom{\tilde{\lambda}_n}
s+a_1-{}\right.\right.\\
\left.\left.{}-a_1 z_1+a_2-\widetilde{\lambda}_n(z_2)\right)x\right)\,,
\label{solution}
\end{multline}
где $\widetilde{\lambda}_k(z_2)\hm=\lambda_k(z_1,z_2,s)\hm+(s\hm+a_1\hm-a_1 z_1\hm+a_2)$, 
$k\hm=1,\ldots,M$; $\lambda_1(z_1,z_2,s),\ldots,\lambda_M(z_1,z_2,s)$~--- 
собственные значения матрицы системы: 
$u_1(z_2)\hm=(u_{11}(z_2),\ldots,u_{M1}(z_2))^{\mathrm{T}},\ldots,u_M(z_2)
\hm=(u_{1M}(z_2),\ldots,u_{MM}(z_2))^{\mathrm{T}}$~--- соответствующие собственные векторы. 
Заметим, что матрицы систем дифференциальных уравнений для 
$\widetilde{\pi}_1(z_1,z_2,k,x,s)$ и~$\widetilde{\pi}_2(z_1,z_2,k,x,s)$ одинаковы, 
а~следовательно, собственные значения и~собственные векторы в~записи решений совпадают.

Функции $\widetilde{\lambda}_1(z_2),\ldots,\widetilde{\lambda}_M(z_2)$ являются 
решениями характеристического уравнения:
\begin{multline}
\prod_{i=1}^M\left(pa_2 z_2^i-\widetilde{\lambda}\right)+
\sum\limits_{i=1}^M(1-p)h_i a_2 z_2^i
\prod\limits_{\substack{j=1\\j\neq i}}^M(pa_2 z_2^j-
\widetilde{\lambda})={}\\
{}=0\,.
\label{determinant}
\end{multline}

Подставляя~(\ref{solution}) в~(\ref{system}), находим:
\begin{equation}
u_{mn}(z_2)=\fr{(1-p)h_m a_2 z_2^m}{\widetilde{\lambda}_n(z_2)-pa_2 z_2^m}\,.
\label{eigenvectors}
\end{equation}

Подставив~(\ref{solution}) в~(\ref{newsecond}), получим:
\begin{multline*}
\sum\limits_{n=1}^{M}C_{1,n}\left(z_1,z_2,s\right)u_{mn}(z_2)+{}\\
{}+
\sum\limits_{n=1}^{M}C_{2,n}\left(z_1,z_2,s\right)u_{mn}(z_2)={}
\end{multline*}

\noindent
\begin{multline*}
{}=z_1^{-1}\sum\limits_{n=1}^{M}C_{1,n}\left(z_1,z_2,s\right)u_{mn}
\left(z_2\right)\times{}\\
{}\times\beta_1\left(s+a_1-a_1 z_1+a_2-\widetilde{\lambda}_n\left(z_2\right)\right)+{}\\
{}+z_2^{-1}\sum\limits_{n=1}^{M}C_{2,n}\left(z_1,z_2,s\right)u_{mn}(z_2)\times{}\\
{}\times \beta_2
\left(s+a_1-a_1 z_1+a_2-\widetilde{\lambda}_n\left(z_2\right)\right)-{}\\
{}-\left(s+a_1-a_1 z_1+a_2\right)
\pi_0(m,s)+h_m+{}\\
{}+\left[p\pi_0(m,s)a_2 z_2^m+(1-p)
\sum\limits_{n=1}^{M}\pi_0(n,s)h_m a_2 z_2^m\right]\,.
\end{multline*}
Перепишем это уравнение в~виде:
\begin{multline}
\sum\limits_{n=1}^{M}\left[
C_{1,n}(z_1,z_2,s)\left(
\vphantom{\tilde{\lambda}_n^{-1}}
1-{}\right.\right.\\
\left.{}-z_1^{-1}\beta_1
\left(s+a_1-a_1 z_1+a_2-\widetilde{\lambda}_n(z_2)\right)\right)+{}\\
{}+C_{2,n}\left(z_1,z_2,s\right)\left(1-
\vphantom{\tilde{\lambda}_n^{-1}}
{}\right.\\
\hspace*{-7.5pt}\left.\left.{}-z_2^{-1}\beta_2
\left(s+a_1-a_1 z_1+a_2-\widetilde{\lambda}_n(z_2)\right)\right)\right]
u_{mn}\left(z_2\right)={}\\
{}=b_m\left(z_1,z_2,s\right)\,,
\label{linear}
\end{multline}
где
\begin{multline*}
\!\!\!\!b_m\!\left(z_1,z_2,s\right)=
-\left(s+a_1-a_1 z_1+a_2\right)\!\pi_0(m,s)+h_m+{}\\
{}+
\left[p\pi_0(m,s)a_2 z_2^m+(1-p)
\sum\limits_{n=1}^{M}\pi_0(n,s)h_m a_2 z_2^m\right].
\end{multline*}

Подставим~(\ref{eigenvectors}) в~(\ref{linear}) и~поделим обе части полученного 
уравнения на $(1\hm-p)h_m a_2 z_2^m$:
\begin{multline*}
\sum\limits_{n=1}^{M}\left[C_{1,n}(z_1,z_2,s)\left(1-
\vphantom{\tilde{\lambda}_n^{-1}}
{}\right.\right.\\
\left.{}-z_1^{-1}\beta_1
\left(s+a_1-a_1 z_1+a_2-\widetilde{\lambda}_n\left(z_2\right)\right)\right)+{}\\
{}+C_{2,n}\left(z_1,z_2,s\right)\left(
\vphantom{\tilde{\lambda}_n^{-1}}
1-{}\right.\\
\left.\left.{}-z_2^{-1}\beta_2
\left(s+a_1-a_1 z_1+a_2-\widetilde{\lambda}_n(z_2)\right)\right)
\right]\times{}\\
{}\times \fr{1}{\widetilde{\lambda}_n(z_2)-pa_2 z_2^m}=
\fr{b_m(z_1,z_2,s)}{(1-p)h_m a_2 z_2^m}\,.
\end{multline*}
Это система линейных алгебраических уравнений с~матрицей Коши. 
Ее решение записывается в~виде\footnote{Про обращение матриц Коши см.~[2].}:

\noindent
\begin{multline*}
C_{1,n}(z_1,z_2,s)\left(
\vphantom{\tilde{\lambda}_n^{-1}}
1-{}\right.\\
\left.{}-z_1^{-1}\beta_1
\left(s+a_1-a_1 z_1+a_2-\widetilde{\lambda}_n(z_2)\right)\right)+{}\\
{}+C_{2,n}\left(z_1,z_2,s\right)
\left(
\vphantom{\tilde{\lambda}_n^{-1}}
1-{}\right.\\
\left.{}-z_2^{-1}\beta_2\left(s+a_1-a_1 z_1+a_2-\widetilde{\lambda}_n(z_2)\right)\right)={}\\
{}=\fr{\prod\limits_{i=1}^M(\widetilde{\lambda}_n(z_2)-pa_2 z_2^i)}
{\prod\limits_{\substack{j=1\\j\neq n}}^M
(\widetilde{\lambda}_n(z_2)-\widetilde{\lambda}_j(z_2))}
\sum\limits_{m=1}^M
\fr{\prod\limits_{\substack{i=1\\i\neq n}}^M
(pa_2 z_2^m-\widetilde{\lambda}_i(z_2))}
{\prod\limits_{\substack{j=1\\j\neq m}}^M(pa_2 z_2^m-pa_2 z_2^j)}\times{}\\
{}\times
\fr{b_m(z_1,z_2,s)}{(1-p)h_m a_2 z_2^m}.
\end{multline*}
Далее, поскольку функции $\widetilde{\lambda}_m(z_2)$, $m\hm=1,\ldots,M$, 
являются решениями уравнения~(\ref{determinant}), можно записать:
\begin{multline}
\prod_{i=1}^M(pa_2 z_2^i-\widetilde{\lambda})+
\sum\limits_{i=1}^M
(1-p)h_i a_2 z_2^i
\prod\limits_{\substack{j=1\\j\neq i}}^M\left(pa_2 z_2^j-\widetilde{\lambda}\right)={}\\
{}=
\prod_{j=1}^M\left(\widetilde{\lambda}_j(z_2)-\widetilde{\lambda}\right).
\label{polynom}
\end{multline}
Подставляя в~(\ref{polynom}) $\widetilde{\lambda}\hm=pa_2 z_2^m$, получим:
\begin{multline*}
(1-p)h_m a_2 z_2^m
\prod\limits_{\substack{j=1\\j\neq m}}^M
\left(pa_2 z_2^j-pa_2 z_2^m\right)={}\\
{}=\prod_{j=1}^M
\left(\widetilde{\lambda}_j(z_2)-pa_2 z_2^m\right).
\end{multline*}
Отсюда
\begin{multline}
C_{1,n}\left(z_1,z_2,s\right)\left(
\vphantom{\tilde{\lambda}_n^{-1}}
1-{}\right.\\
\left.{}-z_1^{-1}\beta_1
\left(s+a_1-a_1 z_1+a_2-\widetilde{\lambda}_n\left(z_2\right)\right)\right)+{}\\
{}+C_{2,n}\left(z_1,z_2,s\right)\left(
\vphantom{\tilde{\lambda}_n^{-1}}
1-{}\right.\\
\left.{}-z_2^{-1}\beta_2\left(s+a_1-a_1 z_1+a_2-\widetilde{\lambda}_n\left(
z_2\right)\right)\right)={}\\
{}=\fr{\prod\limits_{i=1}^M(\widetilde{\lambda}_n(z_2)-pa_2 z_2^i)}
{\prod\limits_{\substack{j=1\\j\neq n}}^M(\widetilde{\lambda}_n(z_2)-
\widetilde{\lambda}_j(z_2))}
\sum\limits_{m=1}^M
\fr{b_m(z_1,z_2,s)}{\widetilde{\lambda}_n(z_2)-pa_2 z_2^m}\,.
\label{C}
\end{multline}

Рассмотрим уравнение:
\begin{equation}
z_1=\beta_1\left(s+a_1-a_1 z_1+a_2-\widetilde{\lambda}_n(z_2)\right).
\label{functional}
\end{equation}
Обе части уравнения являются аналитическими в~области $|z_1|\hm\leq 1$ функциями. 
Имеем:
\begin{multline*}
\left\vert \beta_1\left(s+a_1-a_1 z_1+a_2-\widetilde{\lambda}_n
\left(z_2\right)\right)\right\vert
\leq{}\\
{}\leq \beta_1\left( 
\mathrm{Re}\,\left(s+a_1-a_1 z_1+a_2-\widetilde{\lambda}_n\left(z_2\right)\right)
\right)\leq{}\\
{}\leq 
\beta_1(\mathrm{Re}\,(s))<1=\left\vert z_1\right\vert
\end{multline*}
при $|z_1|=1$. В~силу теоремы Руше отсюда следует, что функциональное 
уравнение~(\ref{functional}) имеет единственное решение $z_1\hm=z_{1,n}(z_2,s)$, 
причем функция $z_{1,n}(z_2,s)$ является аналитической в~области $|z_2|\hm\leq 
1\times\mathrm{Re}\,(s)\hm>0$.

Подставляя $z_1\hm=z_{1,n}(z_2,s)$ в~уравнение~(\ref{C}), получим:
\begin{multline}
C_{2,n}\left(z_{1,n}(z_2,s),z_2,s\right)
\left(\vphantom{\tilde{\lambda}_n}
1-{}\right.\\
\left.{}-z_2^{-1}\beta_2\left(s+a_1-a_1 z_{1,n}(z_2,s)+a_2-\widetilde{\lambda}_n\left(z_2\right)
\right)\right)={}\\
{}=\fr{\prod\limits_{i=1}^M(\widetilde{\lambda}_n(z_2)-pa_2 z_2^i)}
{\prod\limits_{\substack{j=1\\j\neq n}}^M(\widetilde{\lambda}_n(z_2)-
\widetilde{\lambda}_j(z_2))}\times{}\\
{}\times
\sum\limits_{m=1}^M
\fr{b_m(z_{1,n}(z_2,s),z_2,s)}{\widetilde{\lambda}_n(z_2)-pa_2 z_2^m}\,,
\label{findpi1}
\end{multline}
откуда
\begin{multline*}
C_{2,n}\left(z_{1,n}\left(z_2,s\right),z_2,s\right)={}\\
{}=
\fr{1}{1-z_2^{-1}\beta_2(s+a_1-a_1 z_{1,n}(z_2,s)+
a_2-\widetilde{\lambda}_n(z_2))}\times{}\\
{}\times
\fr{\prod\limits_{i=1}^M(\widetilde{\lambda}_n(z_2)-pa_2 z_2^i)}
{\prod\limits_{\substack{j=1\\j\neq n}}^M(\widetilde{\lambda}_n(z_2)-
\widetilde{\lambda}_j(z_2))}
\sum\limits_{m=1}^M
\fr{b_m(z_{1,n}(z_2,s),z_2,s)}{\widetilde{\lambda}_n(z_2)-pa_2 z_2^m}\,.
\end{multline*}
Заметим, что $\widetilde{\pi}_2(z_1,z_2,k,x,s)$, а~значит, 
и~$C_{2,n}(z_1,z_2,s)$, не зависит от~$z_1$. Из этого факта вытекает, 
что можно записать:
\begin{multline}
C_{2,n}\left(z_1,z_2,s\right)={}\\
{}=\!
\fr{1}{1\!-\!z_2^{-1}\,\beta_2\!\left(\!s+a_1-a_1 z_{1,n}(z_2,s)+a_2-\widetilde{\lambda}_n
\left(z_2\right)\!\right)}\!\times{}\\
\hspace*{-4mm}{}\times
\fr{\prod\limits_{i=1}^M(\widetilde{\lambda}_n(z_2)-pa_2 z_2^i)}
{\prod\limits_{\substack{j=1\\j\neq n}}^M(\widetilde{\lambda}_n(z_2)-
\widetilde{\lambda}_j(z_2))}
\!\sum\limits_{m=1}^M\!
\fr{b_m(z_{1,n}(z_2,s),z_2,s)}{\widetilde{\lambda}_n(z_2)-pa_2 z_2^m}.\!\!\!
\label{C2}
\end{multline}
Подставляя~(\ref{C2}) в~(\ref{C}), получим
\begin{multline*}
C_{1,n}(z_1,z_2,s)\left(
\vphantom{\tilde{\lambda}_n^{-1}}
1-{}\right.\\
\left.{}-z_1^{-1}\beta_1
\left(s+a_1-a_1 z_1+a_2-\widetilde{\lambda}_n\left(z_2\right)\right)\right)={}\\
{}=\fr{\prod\limits_{i=1}^M(\widetilde{\lambda}_n(z_2)-pa_2 z_2^i)}
{\prod\limits_{\substack{j=1\\j\neq n}}^M(\widetilde{\lambda}_n(z_2)-
\widetilde{\lambda}_j(z_2))}
\sum\limits_{m=1}^M
\fr{1}{\widetilde{\lambda}_n(z_2)-pa_2 z_2^m}\times{}\\
{}\times\left(b_m\left(z_1,z_2,s\right)-{}\right.\\
{}-
\fr{1-z_2^{-1}\beta_2(s+a_1-a_1 z_1+a_2-\widetilde{\lambda}_n(z_2))}
{1-z_2^{-1}\beta_2(s+a_1-a_1 z_{1,n}(z_2,s)+a_2-\widetilde{\lambda}_n(z_2))}\times{}\\
\left.{}\times b_m\left(z_{1,n}\left(z_2,s\right),z_2,s\right)\right)\,,
%\label{C1}
\end{multline*}
т.\,е.\ \\[-17pt]
\begin{multline*}
C_{1,n}\left(z_1,z_2,s\right)={}\\
{}=
\fr{1}{1-z_1^{-1}\,\beta_1\left(s+a_1-a_1 z_1+a_2-
\widetilde{\lambda}_n\left(z_2\right)\right)}\times{}\\
{}\times
\fr{\prod\limits_{i=1}^M
(\widetilde{\lambda}_n(z_2)-pa_2 z_2^i)}
{\prod\limits_{\substack{j=1\\j\neq n}}^M(\widetilde{\lambda}_n(z_2)-
\widetilde{\lambda}_j(z_2))}
\sum\limits_{m=1}^M
\fr{1}{\widetilde{\lambda}_n(z_2)-pa_2 z_2^m}\times{}\\
{}\times\left( b_m\left(z_1,z_2,s\right)-{}\right.\\
{}-
\fr{1-z_2^{-1}\beta_2(s+a_1-a_1 z_1+a_2-\widetilde{\lambda}_n(z_2))}
{1-z_2^{-1}\beta_2(s+a_1-a_1 z_{1,n}(z_2,s)+a_2-\widetilde{\lambda}_n(z_2))}\times{}\\
\left.{}\times b_m\left(z_{1,n}\left(z_2,s\right),z_2,s\right)\right).
\end{multline*}

Остается найти $\pi_0(m,s)$, $m\hm=1,\ldots,M$. Рас\-смот\-рим уравнение:
\begin{equation}
z_2=\beta_2\left(s+a_1-a_1 z_{1,n}\left(z_2,s\right)+
a_2-\widetilde{\lambda}_n\left(z_2\right)\!\right).\!\!
\label{functional2}
\end{equation}
Обе части уравнения являются аналитическими в~области $|z_2|\hm\leq 1$ 
функциями. Имеем:
\begin{multline*}
\left\vert\beta_2\left(s+a_1-a_1 z_{1,n}\left(z_2,s\right)+a_2-
\widetilde{\lambda}_n\left(z_2\right)\right)\right\vert
\leq{}\\
{}\leq\beta_2\left(\mathrm{Re}\left(s+a_1-a_1 z_{1,n}\left(z_2,s\right)+
a_2-\widetilde{\lambda}_n\left(z_2\right)\!\right)\!\right)\leq{}\\
{}\leq\beta_2(\mathrm{Re}\,(s))<1=\left|z_2\right|
\end{multline*}
при $|z_2|\hm=1$. В~силу теоремы Руше отсюда следует, что функциональное 
уравнение~(\ref{functional2}) имеет единственное решение $z_2\hm=z_{2,n}(s)$, 
причем функция $z_{2,n}(s)$ является аналитической в~области $\mathrm{Re}\,(s)\hm>0$.

Подставляя $z_2=z_{2,n}(s)$ в~уравнение (\ref{findpi1}), 
приходим после ряда преобразований 
к~уравнению\footnote{В дальнейшем будем для краткости писать $z_{1,n}$ 
вместо $z_{1,n}(z_{2,n}(s),s)$, $z_{2,n}$ вместо $z_{2,n}(s)$ 
и~$\widetilde{\lambda}_n$ вместо $\widetilde{\lambda}_n(z_{2,n})$.}:
\begin{equation}
\sum\limits_{m=1}^M
\prod\limits_{\substack{j=1\\j\neq m}}^M(\widetilde{\lambda}_n-
pa_2 z_{2,n}^j)b_m(z_{1,n},z_{2,n},s)=0\,.
\label{findpi2}
\end{equation}
Вспомним, что
\begin{multline*}
\hspace*{-5.5pt}b_m\!\left(z_1,z_2,s\right)=-\left(s+a_1-a_1 z_1+a_2\right)
\!\pi_0(m,s)+h_m+{}\\
{}+\left[p\pi_0(m,s)a_2 z_2^m+(1-p)
\sum\limits_{n=1}^{M}\pi_0(n,s)h_m a_2 z_2^m\right].
\end{multline*}
С учетом уравнения~(\ref{determinant}) будем иметь:
\begin{multline*}
\sum\limits_{m=1}^M\prod\limits_{\substack{j=1\\j\neq m}}^M
\left(\widetilde{\lambda}_n-pa_2 z_{2,n}^j\right)
\left[
\vphantom{\sum\limits_{k=1}^{M}}
p\pi_0(m,s)a_2 z_{2,n}^m+{}\right.\\
\left.{}+(1-p)
\sum\limits_{k=1}^{M}\pi_0(k,s)h_m a_2 z_{2,n}^m\right]={}\\
{}=\sum\limits_{m=1}^M
\prod\limits_{\substack{j=1\\j\neq m}}^M
\left(\widetilde{\lambda}_n-pa_2 z_{2,n}^j\right)\times{}\\
{}\times
\left[
\vphantom{\sum\limits_{k=1}^{M}}
\left(pa_2 z_{2,n}^m-\widetilde{\lambda}_n\right)
\pi_0(m,s)+\widetilde{\lambda}_n \pi_0(m,s)+{}\right.\\
\left.{}+(1-p)
\sum\limits_{k=1}^{M}\pi_0(k,s)h_m a_2 z_{2,n}^m\right]={}\\
{}=-\prod\limits_{j=1}^M\left(\widetilde{\lambda}_n-pa_2 z_{2,n}^j\right)
\sum\limits_{m=1}^M \pi_0(m,s)+{}\\
{}+
\sum\limits_{m=1}^M
\prod\limits_{\substack{j=1\\j\neq m}}^M\left(\widetilde{\lambda}_n-pa_2 z_{2,n}^j\right)
\widetilde{\lambda}_n \pi_0(m,s)+{}\\
{}+\prod\limits_{j=1}^M
\left(\widetilde{\lambda}_n-pa_2 z_{2,n}^j\right)
\sum\limits_{m=1}^M \pi_0(m,s)={}\\
{}=
\sum\limits_{m=1}^M
\prod\limits_{\substack{j=1\\j\neq m}}^M\left(\widetilde{\lambda}_n-pa_2 z_{2,n}^j\right)
\widetilde{\lambda}_n \pi_0(m,s)\,.
\end{multline*}
Возвращаясь к~уравнению~(\ref{findpi2}), получим:
\begin{multline*}
\sum\limits_{m=1}^M
\prod\limits_{\substack{j=1\\j\neq m}}^M\left(\widetilde{\lambda}_n-pa_2 z_{2,n}^j\right)
\left(
\vphantom{\widetilde{\lambda}_n}
-\left(s+a_1-a_1 z_{1,n}+{}\right.\right.\\
\left.\left.{}+a_2\right)\pi_0(m,s)+
h_m+\widetilde{\lambda}_n \pi_0(m,s)\right)=0\,.
\end{multline*}

Там самым доказательство теоремы завершено.

{\small\frenchspacing
 {%\baselineskip=10.8pt
 \addcontentsline{toc}{section}{References}
 \begin{thebibliography}{9}
\bibitem{1-us}
\Au{Леонтьев Н.\,Д., Ушаков В.\,Г.} 
Анализ системы обслуживания с~входящим потоком авторегрессионного типа~// 
Информатика и~её применения, 2014. Т.~8. Вып.~3. С.~39--44.
\bibitem{2-us}
\Au{Schechter S.} On the inversion of certain matrices~// 
Math. Tab. Aids Comput., 1959. Vol.~13. No.\,66. P.~73--77.
\end{thebibliography}

 }
 }

\end{multicols}

\vspace*{-6pt}

\hfill{\small\textit{Поступила в~редакцию 11.05.16}}

\vspace*{8pt}

%\newpage

%\vspace*{-24pt}

\hrule

\vspace*{2pt}

\hrule

%\vspace*{8pt}



\def\tit{ANALYSIS OF~A~QUEUEING SYSTEM WITH~AUTOREGRESSIVE ARRIVALS 
AND~NONPREEMPTIVE PRIORITY}

\def\titkol{Analysis of~a~queueing system with~autoregressive arrivals 
and~nonpreemptive priority}

\def\aut{N.\,D.~Leontyev$^1$ and V.\,G.~Ushakov$^{1,2}$}

\def\autkol{N.\,D.~Leontyev and V.\,G.~Ushakov}

\titel{\tit}{\aut}{\autkol}{\titkol}

\vspace*{-9pt}

\noindent
$^1$Department of Mathematical 
Statistics, Faculty of Computational Mathematics and Cybernetics,\linebreak
$\hphantom{^1}$M.\,V.~Lomonosov 
Moscow State University, 1-52~Leninskiye Gory, Moscow 119991, GSP-1, Russian\linebreak 
$\hphantom{^1}$Federation

\noindent
$^2$Institute of Informatics Problems, Federal 
Research Center ``Computer Science and Control'' of the Russian\linebreak 
$\hphantom{^1}$Academy of Sciences, 
44-2~Vavilov Str., Moscow 119333, Russian Federation

\def\leftfootline{\small{\textbf{\thepage}
\hfill INFORMATIKA I EE PRIMENENIYA~--- INFORMATICS AND
APPLICATIONS\ \ \ 2016\ \ \ volume~10\ \ \ issue\ 3}
}%
 \def\rightfootline{\small{INFORMATIKA I EE PRIMENENIYA~---
INFORMATICS AND APPLICATIONS\ \ \ 2016\ \ \ volume~10\ \ \ issue\ 3
\hfill \textbf{\thepage}}}

\vspace*{3pt}



\Abste{The paper studies a single server queueing system with infinite capacity 
and with two arrival streams, one of which is Poisson and the other is batch Poisson. 
The customers from the first stream have nonpreemptive priority over the customers 
from the second. A~feature of the system under study is autoregressive dependence 
of the sizes of the batches from the second arrival stream: the size of the 
$n$th batch is equal to the size of the $(n-1)$st batch with a~fixed probability 
and is an independent random variable with complementary probability. Service 
times of the customers from each stream are supposed to be independent random 
variables with specified distributions. The main object of the study is the 
number of the customers from each stream in the system at an arbitrary moment. 
The relations derived make it possible
to find Laplace transorm in time of probability 
generating function of the transient queue length and also a~number of 
additional characteristics.}

\KWE{queueing theory; transient behavior; batch arrivals; nonpreemptive priority}

\DOI{10.14357/19922264160303}

%\vspace*{-9pt}

\Ack
\noindent
This work was supported by the Russian Foundation for Basic 
Research (project No.\,15-07-02354).


%\vspace*{3pt}

  \begin{multicols}{2}

\renewcommand{\bibname}{\protect\rmfamily References}
%\renewcommand{\bibname}{\large\protect\rm References}

{\small\frenchspacing
 {%\baselineskip=10.8pt
 \addcontentsline{toc}{section}{References}
 \begin{thebibliography}{9}
\bibitem{1-us-1}
\Aue{Leontyev, N.\,D., and V.\,G.~Ushakov}. 
2014. Analiz sistemy obsluzhivaniya s~vkhodyashchim potokom avtoregressionnogo tipa 
[Analysis of queueing system with autoregressive arrivals]. 
\textit{Informatika i~ee Primeneniya~--- Inform. Appl.} 8(3):39--44.

 
\bibitem{2-us-1}
\Aue{Schechter, S.} 1959. On the inversion of certain matrices. 
\textit{Math. Tab. Aids Comput.} 13(66):73--77.
   \end{thebibliography}

 }
 }

\end{multicols}

\vspace*{-3pt}

\hfill{\small\textit{Received May 11, 2016}}


\Contr

\noindent
\textbf{Leontyev Nikolai D.} (b.\ 1988)~--- PhD student,
Department of Mathematical 
Statistics, Faculty of Computational Mathematics and Cybernetics, M.\,V.~Lomonosov 
Moscow State University, 1-52~Leninskiye Gory, Moscow 119991, GSP-1, Russian 
Federation; \mbox{ndleontyev@gmail.com}

\vspace*{3pt}

\noindent
\textbf{Ushakov Vladimir G.} (b.\ 1952)~---
Doctor of Science in physics and mathematics, professor, Department of Mathematical 
Statistics, Faculty of Computational Mathematics and Cybernetics, M.\,V.~Lomonosov 
Moscow State University, 1-52~Leninskiye Gory, Moscow 119991, GSP-1, Russian 
Federation; senior scientist, Institute of Informatics Problems, Federal 
Research Center ``Computer Science and Control'' of the Russian Academy of Sciences, 
44-2~Vavilov Str., Moscow 119333, Russian Federation; \mbox{vgushakov@mail.ru} 
\label{end\stat}


\renewcommand{\bibname}{\protect\rm Литература} %3
\def\stat{ometov}

\def\tit{АНАЛИЗ ПРОИЗВОДИТЕЛЬНОСТИ БЕСПРОВОДНОЙ 
СИСТЕМЫ АГРЕГАЦИИ ДАННЫХ С~СОСТЯЗАНИЕМ 
ДЛЯ~СОВРЕМЕННЫХ СЕНСОРНЫХ СЕТЕЙ$^*$}

\def\titkol{Анализ производительности беспроводной 
системы агрегации данных с~состязанием} 
%для современных сенсорных сетей}

\def\aut{А.\,Я.~Омётов$^1$, С.\,Д.~Андреев$^2$, А.\,М.~Тюрликов$^3$, 
Е.\,А.~Кучерявый$^4$}

\def\autkol{А.\,Я.~Омётов, С.\,Д.~Андреев, А.\,М.~Тюрликов, 
Е.\,А.~Кучерявый}

\titel{\tit}{\aut}{\autkol}{\titkol}

\index{Омётов А.\,Я.}
\index{Андреев С.\,Д.}
\index{Тюрликов А.\,М.} 
\index{Кучерявый Е.\,А.}
\index{Ometov A.\,Ya.}
\index{Andreev S.\,D.}
\index{Turlikov A.\,M.}
\index{Koucheryavy E.\,A.}


{\renewcommand{\thefootnote}{\fnsymbol{footnote}} \footnotetext[1]
{Исследование выполнено при частичной финансовой поддержке РФФИ (проект 15-07-03051), 
а~также 
Фонда содействия развитию малых форм предприятий в~на\-уч\-но-тех\-ни\-че\-ской сфере в~рамках 
программы <<УМНИК>>  по договору №\,8268ГУ2015 от~02.12.2015.}}


\renewcommand{\thefootnote}{\arabic{footnote}}
\footnotetext[1]{Санкт-Петербургский государственный университет телекоммуникаций им.\ М.\,А.~Бонч-Бруевича, 
\mbox{alexander.ometov@gmail.com}}
\footnotetext[2]{Российский университет дружбы народов, \mbox{serge.andreev@gmail.com}}
\footnotetext[3]{Санкт-Петербургский государственный университет аэрокосмического приборостроения, 
\mbox{turlikov@vu.spb.ru}}
\footnotetext[4]{Национальный исследовательский университет <<Высшая школа экономики>>, 
\mbox{ykoucheryavy@hse.ru}}

\vspace*{-12pt}


\Abst{Рассматривается беспроводная система связи, 
учитывающая особенности современных сенсорных сетей, в~которых 
устройства передают свои данные на множество промежуточных 
агрегирующих узлов, имеющих подключение к~сети Интернет по 
технологии IEEE 802.11-2014 (WiFi). Предполагается, что агрегатор 
осуществляет пересылку данных от многих сенсоров, участвуя при этом 
в~состязании за общий канал связи с~другими агрегаторами. Предлагается 
аналитическая модель такого состязания, учитывающая специфику 
алгоритма разрешения коллизий, характеристики протокола доступа 
к~каналу, а~также возможность потери данных на узле агрегации. 
Полученные аналитические результаты сопоставляются с~данными 
имитационного моделирования, и~вычисляется максимальное количество 
поддерживаемых системой связи сенсоров.}

\KW{Интернет вещей; беспроводные сенсорные сети; регенеративный 
анализ; БЛВС; стандарт IEEE 802.11-2014}

 \DOI{10.14357/19922264160304} 
  


\vskip 10pt plus 9pt minus 6pt

\thispagestyle{headings}

\begin{multicols}{2}

\label{st\stat}
    
  \section{Введение}
     
    В последнее время все более усиливается влияние беспроводных 
технологий на современное общество, что, в~свою очередь, предвещает 
рост научного интереса к~данной тематике в~ближайшие\linebreak годы и~влечет за 
собой потенциальную возможность установления беспроводного 
соединения в~любом месте и~в любое время~[1]. Данная возможность 
является привлекательной для внедрения концепции <<Интернета 
вещей>> (Internet of Things, IoT)~[2]. При интеграции IoT многие 
устройства могут быть оборудованы сенсорами и~расширяющими 
модулями, с~помощью которых появляется возможность обрабатывать 
и~передавать информацию без вмешательства человека. Данные 
беспроводные технологии открывают дорогу для широкого спектра 
сервисов, начиная с~удаленного наблюдения и~заканчивая 
здравоохранением. 

Основной целью текущих исследований является 
разработка системы связи, использующей подходящие  
IoT-тех\-но\-ло\-гии для реализации подключения разнородных сенсоров. 
{\looseness=-1

}
    
    Исторически беспроводные сети разрабатывались для использования 
людьми, а для использования их машинами необходима значительная 
оптимизация, что определяет необходимость разработки или улучшения 
технологий связи для поддержки большого числа устройств~[3]. 
Основными требованиями к~таким технологиям остаются низкая 
сложность обработки данных, дешевизна в~производстве и~высокая 
энергоэффективность. 
    
    Беспроводные локальные сети на основе стандарта IEEE 802.11 
(WiFi) являются одним из самых распространенных технических 
решений для ор\-ганизации беспроводного доступа в~домах и~на 
предприятиях. Благодаря их высокой пропускной спо\-соб\-ности, 
относительно низкой сто\-и\-мости и~повсеместной распространенности 
использование WiFi для сценариев IoT является все более 
привлекательным. 

\begin{figure*}[b] %fig1
\vspace*{1pt}
 \begin{center}  
\mbox{%
 \epsfxsize=114.889mm
 \epsfbox{ome-1.eps}
 }
\end{center} 
\vspace*{-9pt}
\Caption{Предполагаемая топология беспроводной сети}
     \end{figure*}

В~данной работе производится исследование 
принципиальной возможности и~эффективности использования 
современной технологии WiFi с~учетом ее технических характеристик 
для типовых сценариев IoT. В~частности, предлагается аналитическая 
модель для учета особенностей работы технологии WiFi (состязание 
между агрегаторами, особенности протокола доступа, режимы передачи 
данных и~т.\,д.), основанная на теории регенерирующих процессов. 
Аналитические результаты сопоставляются с~данными, полученными 
имитационным моделированием, и~делается вывод о наибольшем 
возможном количестве сенсоров, поддерживаемых такой системой связи.

\vspace*{-6pt}
   
   \section{Модель системы и~анализ}
   
   \vspace*{-6pt}
   
\subsection{Описание сценария и~протокола}

\vspace*{-1pt}
    
    В данной работе рассматривается изолированный сегмент (кластер) 
беспроводной сети для IoT-при\-ло\-же\-ний со статичным размещением~$M$ 
агрегирующих узлов, в~котором отсутствуют <<скрытые>> станции. 
Топология данной сети представлена на рис.~1. 

Агрегаторы оборудованы 
двумя модулями беспроводной связи: WiFi и~ZigBee. Сенсоры передают 
собранную ими информацию на агрегаторы посредством ZigBee, и~далее 
поток данных перенаправляется в~канал WiFi, соединяющий агрегаторы 
с~сетью Интернет. Агрегаторы взаимодействуют в~нелицензированном 
частотном диапазоне WiFi и~используют протокол случайного 
множественного доступа для передачи накопленных данных в~общий 
канал связи. При передаче более чем от одного агрегатора единовременно 
возникает наложение таких передач в~канале~--- коллизия. В~случае если 
передавало только одно устройство, передача считается успешной 
в~предположении, что в~канале отсутствует шум. Третье возможное 
событие~--- пус\-той слот~--- происходит в~случае, если ни один из 
агрегаторов не осуществлял передачу. В~то же время известные решения 
для сотовых сетей в~данной работе рассматриваться не будут~[4].
    

     
    В данной работе рассматривается система среднего/большого 
производства, где в~каждом помещении находится один агрегатор, 
обслуживающий не более~20~узлов. Интерференция и~конфликты 
в~канале сен\-сор--аг\-ре\-га\-тор для данного исследования не являются 
критическими, так как алгоритмы беспроводной связи ближнего радиуса 
действия не подвержены интерференции от кластеров из соседних 
помещений, в~то время как агрегаторы могут конфликтовать в~связи 
с~более широким радиусом действия беспроводного покрытия. 

Таким 
образом, в~работе предполагается худший случай насыщенного трафика 
между всеми агрегирующими узлами~[5]. Иными словами, на уровне 
управления доступом к~среде можно наблюдать целиком заполненный 
буфер исходящих сообщений. Более высокие уровни модели связи 
в~данной работе не рассматриваются ввиду предположения о~прос\-то\-те 
сенсоров. Данное допущение дает возможность оценивать наихудший 
сценарий загрузки сети и~поз\-во\-ля\-ет производить оценку с~точки зрения 
пропускной способности насыщения~$S$.
    
    Согласно спецификации стандарта IEEE 802.11, процесс доступа 
агрегаторов к~общему каналу связи основан на алгоритме двоичной 
экспоненциальной отсрочки (ДЭО) и~состоит в~выборе случайного 
интервала отсрочки перед передачей из некоторого окна отсрочки 
(CW). В~процессе работы алгоритма ДЭО в~случае передачи 
единовременно двумя или более агрегаторами происходит коллизия. Она 
отслеживается на стороне получателя (точки доступа), а~каждому из 
передающих абонентов пред\-остав\-ля\-ет\-ся возможность повторной 
передачи, если это позволит сделать счетчик повторных передач (RC).


\pagebreak


%\begin{table*}
\noindent
{\small
\begin{center}

\begin{tabular}{|l|c|}
\multicolumn{2}{c}{Основные обозначения, использованные в~работе}\\
\multicolumn{2}{c}{\ }\\[-4pt]
\hline
\multicolumn{1}{|c|}{Параметр}&Обозначение\\
\hline
Максимальная длительность доступа&$T_{\mathrm{TXOP}}$\\
Количество агрегаторов&$M$\\
Текущее окно отсрочки&$W_i$, CW\\
Начальное окно отсрочки&$W_0$\\
Длительность пустого слота&$\Sigma$\\
Длительность AIFS&$T_{\mathrm{AIFS}}$\\
Длительность SIFS&$T_{\mathrm{SIFS}}$\\
Длительность BA&$T_{\mathrm{BA}}$\\
Длительность RTS&$T_{\mathrm{RTS}}$\\
Длительность CTS&$T_{\mathrm{CTS}}$\\
Длительность CFE&$T_{\mathrm{CFE}}$\\
Средняя длительность блока данных&$T_{E[P]}$\\
Длительность передачи преамбулы $P$&$T_P$\\
\hline
\end{tabular}
\end{center}}

\vspace*{18pt}
%\end{table*}

\noindent 
В~данном случае значение окна отсрочки CW будет удвоено 
($W_{i+1}\hm= 2W_i$) с~целью уменьшения вероятности повторной 
коллизии. В~то же время будет уменьшен счетчик повторных передач 
RC. Возможный рост~CW ограничен максимальным значением 
($\mathrm{CW}_{\max}\hm= 2^mW_0$), где~$m$ определяется как <<степень>> 
отсрочки. Основные используемые в~работе сокращения представлены 
в~таблице.
    

    
    Данная схема передачи может использовать два альтернативных 
механизма доступа, подробно описанных в~спецификации IEEE 802.11. 
При использовании механизма базового доступа в~канал (Basic) пакет 
данных (либо агрегированная группа пакетов с~единой преамбулой~$P$) 
передается незамедлительно после ожидания регуляционного 
межкадрового интервала (AIFS) и~случайного времени отсрочки 
(BOT). Информация об успешной передаче пакета/блока данных 
передается в~блоковом под\-тверж\-де\-нии (BA). Механизм доступа 
<<запрос на от\-прав\-ку\,/\,раз\-ре\-ше\-ние отправки>> (RTS/CTS) 
использует алгоритм <<двукратного рукопожатия>> при передаче 
сообщения. Иными словами, канал резервируется за определенным 
агрегатором на время его предполагаемой передачи.
    
    В результате агрегации пакетов данных на физическом уровне, на 
высоких скоростях передачи влияние служебной информации на 
пропускную способность становится менее значительно. Однако 
длительность нового пакета данных после агрегирования не должна 
превышать интервала захва\-та среды передачи (TXOP), включая 
необходимые межкадровые интервалы (SIFS), блоковое под\-тверж\-де\-ние 
(BA) и~(опционально) RTS/CTS. При не\-об\-хо\-ди\-мости агрегатор может 
также преждевременно закончить передачу, в~случае если длительность 
передачи пакета данных оказалась меньше TXOP, отправив сообщение об 
освобождении канала (CFE).
    
    Согласно стандарту после первого интервала AIFS значение 
счетчика отсрочки (BC) выбирается как равномерно распределенная 
величина~$W_i$ в~промежутке между~0 и~$W_0\hm-1$, где~$W_i$ 
является окном отсрочки~CW. После каждого пустого слота~BC 
уменьшается на единицу. Как только~BC достигает нуля, выбранный 
агрегатор пытается передать. При возникновении коллизии происходит 
повторная передача, в~случае если счетчик повторных передач~RC не 
равен нулю. В~случае повторной передачи $W_i\hm= 2W_{i-1}$. Рост 
CW также ограничен максимальным значением~$W_{\max}$, но 
агрегатор может продолжать передавать повторно, пока~RC не 
достигнет нуля. В~момент, когда пакет успешно передан или принято 
решение об отказе от передачи, CW устанавливается в~некоторое 
начальное значение~$W_0$. Эквивалентно CW$_{\max}\hm= 2^m W_0$, 
где $m$~--- степень отсрочки.
    
    Далее представлен анализ работы системы доступа к~каналу, 
функционирующей на основе алгоритма ДЭО с~потерями согласно 
описанию выше. В~работе~[6] предложно рассматривать два параметра 
функционирования системы на основании циклов регенерации: 
вероятность передачи~$p_t$ пользователем в~конкретный временной слот 
и~вероятность ошибки~$p_c$ в~случае осуществленной передачи. 
Данные параметры предполагаются неизменными на всем протяжении 
работы насыщенной системы. Вследствие данного допущения весь 
кластер может быть рассмотрен с~точки зрения одного <<меченого>> 
абонента этой системы. Влияние всех прочих факторов учитывается 
в~значении вероятности ошибки~$p_c$. Необходимо отметить, что 
данная замена допустима только в~случае справедливой системы, т.\,е.\ 
когда все пользователи получают приблизительно равную долю времени 
доступа к~каналу~\cite{7-om}.

    
\subsection{Общие понятия для~системы без~потерь}
    
    Рассмотрим модель, представленную на рис.~2, основываясь на 
концепции \textit{циклов регенерации}. 
    
\begin{figure*} %fig2
\vspace*{1pt}
 \begin{center}  
\mbox{%
 \epsfxsize=136.118mm
 \epsfbox{ome-2.eps}
 }
\end{center} 
\vspace*{-9pt}
\Caption{Упрощенная модель алгоритма ДЭО}
     \end{figure*}
     
    Данная модель рассматривает равные временн$\acute{\mbox{ы}}$е слоты, где начало 
передачи пакета совпадает с~началом слота. Каждая передача занимает 
в~точности один слот. Подобное упрощение типично для моделирования 
протоколов случайного множественного доступа~\cite{9-om} 
и~позволяет легко масштабировать модель согласно требуемым 
временн$\acute{\mbox{ы}}$м характеристикам конкретного протокола, что будет показано 
в~следующем подразделе. Отметим, что меченый агрегатор имеет 
сле\-ду\-ющую вероятность конфликта в~произвольно взятом слоте:

\pagebreak

\noindent
    \begin{equation}
    p_c=1-(1-p_t)^{M-1}\,.
    \label{e1-om}
    \end{equation}
    %
    Важно также обратить внимание, что вероятность передачи в~канал 
для данного агрегатора может быть рассчитана как отношение числа 
попыток передачи на пакет $B^{(i)}$ в~течение некоторого цикла 
регенерации к~длительности данного цикла в~слотах~$D^{(i)}$:
    \begin{equation}
    p_t=\lim\limits_{n\to\infty} \fr{\sum\nolimits_{i=1}^n 
B^{(i)}}{\sum\nolimits_{i=1}^n D^{(i)}}=\fr{E[B]}{E[D]}\,.
    \label{e2-om}
    \end{equation}
    
    Предполагая, что система находится в~насыщении и~не имеет потерь, 
можно с~легкостью получить~$E[B]$:
    \begin{multline}
    E[B]= \sum\limits_{i=1}^\infty \mathrm{Pr}\,\{B=i\} ={}\\
    {}=\left(1-p_c\right) 
\sum\limits_{i=1}^\infty ip_c^{i-1} =\fr{1}{1-p_c}\,.
    \label{e3-om}
    \end{multline}
    
    Выражение для $E[D]$ может быть получено аналогичным 
способом~\cite{10-om}:
    \begin{multline}
    E[D]= \sum\limits_{i=1}^\infty  D^{(i)} \mathrm{Pr}\,\{D=i\}={}\\
    {}=\left(1-p_c\right) 
\sum\limits_{i=1}^\infty D^{(i)} p_c^{i-1}\,,
    \label{e4-om}
    \end{multline}
где $D^{(i)}$~--- длина цикла регенерации при условии, что было 
произведено~$i$~попыток передачи.

    Основываясь на зависимости~$i$ и~$m$, можно также получить 
следующие правила вычисления величины $D^{(i)}$:
    \begin{equation}
    D^{(i)}= \begin{cases}
    2^{i-1}W_0-\fr{W_0-1}{2}\,, &\\
    &\hspace*{-30mm} \mbox{если } 1\leq i\leq m+1\,;\\
    2^{m-1} W_0(i-m+1) - \fr{W_0-i}{2}\,, &\\
    &\hspace*{-30mm} \mbox{если } i>m+1\,.
    \end{cases}
    \label{e5-om}
    \end{equation} 
    
    Подставив~(\ref{e5-om}) в~(\ref{e4-om}), после преобразования 
получим:

\noindent
    \begin{multline}
    E[D]={}\\
    {}= \fr{(1-2p_c) (W_0+1) +p_c W_0 (1-(2p_c)^m)}{2(1-2p_c) (1-
p_c)}\,.
    \label{e6-om}
    \end{multline}
    
    Далее, подставляя~(\ref{e3-om}) и~(\ref{e6-om}) в~(\ref{e2-om}), 
имеем:
    \begin{equation}
    p_t= \fr{2(1-2p_c)} {(1-2p_c) (W_0+1) +p_c W_0 (1-(2p_c)^m)}\,.
    \label{e7-om}
    \end{equation}
    
    Следует отметить, что аналогичные результаты были получены 
в~известной работе~\cite{8-om}, где аналитические расчеты были 
основаны на двумерных цепях Маркова, которые сложно 
масштабировать на случай дополнительных параметров системы. Здесь 
же использован иной математический подход, яв\-ля\-ющий\-ся более 
простым, но не менее эффективным. Выражения~(\ref{e1-om})  
и~(\ref{e7-om}) составляют систему двух нелинейных уравнений 
с~неизвестными~$p_c$ и~$p_t$, которую можно решить численно. 
    
\subsection{Система с~потерями}
     
    В данном подразделе представлен анализ ис\-ходной системы, 
работающей с~потерями, т.\,е.\linebreak с~определенным максимальным числом 
повторных передач~$K$ для отдельно взятого пакета. Для 
вы\-чис\-ле\-ния~$p_t$ в~данной системе определяем $E[B]$ и~$E[D]$ 
согласно~(\ref{e2-om}), но выражение для расчета среднего числа 
попыток передачи в~цикле $E[B]$ необходимо модифицировать 
следующим образом:
    \begin{multline}
    E[B]= \sum\limits_{i=1}^{K+1} i Pr\{B=i\} = {}\\
    {}=\left(1-p_c\right) 
\sum\limits_{i=1}^{K+1} i p_c^{i-1} (K+1) p_c^{K+1} = \fr{1-
p_c^{K+1}}{1-p_c}\,.
    \label{e8-om}
    \end{multline}
    
    Далее вычисляем среднюю длительность цикла регенерации как
    \begin{multline*}
    E[D]= \sum\limits_{i=1}^{K+1} D(i) \mathrm{Pr}\,\{B=i\} = {}\\
    {}=\left(1-p_c\right) 
\sum\limits_{i=1}^{K+1} D(i) p_c^{i-1} +D(K+1) p_c^{K+1}\,.
%    \label{e9-om}
    \end{multline*}
    
    Возможны две ситуации, зависящие от соотношения~$m$ и~$K$: 
когда $K\hm\leq m$ и~$K\hm>m$. В~первом случае
    \begin{multline}
    E\left[ D^\prime\right] = (1-p_c) \left[ \sum\limits_{i=1}^{K+1}\! \left( 
2^{i-1} W_0 -\fr{W_0-2}{2}\right) p_c^{i-1}\right] +{}\\
{}+
    p_c^{K-1} \left( 2^K W_0 -\fr{W_0-(K+1)}{2}\right)\,.
    \label{e10-om}
    \end{multline}
    
    Соответственно, вероятность выхода в~канал~$p^\prime_t$ может 
быть получена при расчете выражения~(\ref{e2-om}) в~результате 
подстановки~$E[B]$ из~(\ref{e8-om}) и~$E[D]$ (в~данном случае 
$E[D^\prime]$) из~(\ref{e10-om}):
    \begin{multline}
    p^\prime_t = 2(1-2p_c) \left(1-2p_c^{K+1}\right)\!\Big /\!
    \left[ \left(1-2p_c\right) 
\left( \vphantom{\left(2p_c\right)^{K+1}}
1-{}\right.\right.\\
\hspace*{-5mm}\left.\left.{}-\left(2p_c\right)^{K+1}\right) 
+ W_0 \left(1-p_c\right) \left(1-\left(2p_c\right)^{K+1}\right)\right]\,.
    \label{e11-om}
    \end{multline}
    
    Чтобы рассчитать второй случай (когда $K\hm>m$), необходимо 
получить соответствующее значение~$E[D^{\prime\prime}]$:
    \begin{multline}
   \! E[D^{\prime\prime}] = (1-p_c) \left[ \sum\limits_{i=1}^{m+1} \left( 
2^{i-1}W_0 -\fr{W_0-i}{2}\right) p_c^{i-1}+ {}\right.\\
\left.{}+
    \sum\limits_{i=1}^{m+1} \left( 2^{i-1}W_0(i-m+1)-\fr{W_0-i}{2}\right) 
p_c^{i-1}\right] + {}\\
{}+p_c^{K-1} \left( 2^KW_0 - \fr{W_0-(K+1)}{2}\right)\,.
\label{e12-om}
    \end{multline}
     
     Аналогично можно получить вероятность 
передачи~$p^{\prime\prime}_t$, подставляя $E[B]$ из~(\ref{e8-om}) 
и~$E[D]$ (в данном случае $E[D^{\prime\prime}]$) из~(\ref{e12-om}) 
в~(\ref{e2-om}):
    \begin{multline}
    p^{\prime\prime}_t =   2\left(1-2p_c\right)\left(1-2p_c^{K+1}\right)\Big / 
    \left[ \vphantom{\left(2p_c\right)^m}
    (1-2p_c)\times{}\right.\\
   {}\times 
\left(W_0\left(1-\left(2p_c\right)^{K+1}\right)\right)+
\left(1-\left(2p_c\right)^{K+1}\right) +{}\\
\left. {}+W_0 p_c\left(1-\left(2p_c\right)^m\right)\right]\,.
    \label{e13-om}
    \end{multline}
    
    Итак, вероятность выхода в~канал~$p_t$ в~системе с~потерями может 
быть получена двумя способами: как~$p^\prime_t$ из~(\ref{e11-om}) или 
как~$p_t^{\prime\prime}$ из~(\ref{e13-om}), в~зависимости от 
соотношения~$K$ и~$m$. Решая нелинейную систему 
уравнений~(\ref{e8-om}), получаем итоговое значение~$p_t$.
    
    Также в~данной работе были использованы фактические 
длительности служебных сообщений\linebreak
 WiFi для вычисления пропускной 
способности насыщения, которая может быть воспроизведена по 
аналогии с~\cite{9-om, 8-om, 11-om} и~множеством других работ. Основное 
отличие от вышеприведенной модели заключается в~том, что 
используются разные длительности слотов для различных событий 
в~канале, которые соответствуют текущей спецификации стандарта IEEE 
802.11-2014. Далее рассматривается работа механизма доступа 
<<RTS/CTS>>. Размеры слотов разной длительности следующие: 
$\sigma$ соответствует длительности пустого слота, $T_S$~--- 
длительности успешной передачи, а $T_C$~--- длительности коллизии. 
Длительности успеха и~коллизии могут быть рассчитаны 
согласно~\cite{7-om} как
    \begin{align*}
       \hspace*{-2mm}T_S &= T_{\mathrm{RTS}}+ T_{\mathrm{SIFS}}+ T_{\mathrm{CTS}}+ T_{\mathrm{SIFS}} +{}       \hspace*{2mm}\\
    &\hspace*{13mm}{}+T_P+ T_{E[P]}  +T_{\mathrm{SIFS}} +T_{\mathrm{BA}}+T_{\mathrm{AIFS}}\,,       \\
          \hspace*{-2mm} T_C &= T_{\mathrm{RTS}} +T_{\mathrm{AIFS}}\,.       \hspace*{2mm}
   %    \label{e14-om}
    \end{align*}
    
    В итоге можно получить пропускную способность насыщения~$S$:
    \begin{equation*}
    S= \fr{P_t P_S E[P]}{(1-P_t)\sigma +P_t P_S T_S +P_t(1-P_S) T_C}\,,
%    \label{e15-om}
    \end{equation*}
где $P_t= 1-(1\hm- p_t)^M$~--- вероятность того, что в~системе данные 
передавал только один агрегатор, $P_S\hm= Mp_t(1\hm- p_t)^{M-1}P_t^{-
1}$~--- вероятность успешной передачи (при условии, что передавал 
один агрегатор). 

    В системе с~потерями можно также выписать вероятность того, что 
агрегатор отказывается от передачи пакета после~$K$~неуспешных 
повторных передач, вызванных коллизиями: $P_d\hm= p_c^{K-1}$.

\section{Результаты и~выводы}
     
    В данном разделе представлены результаты моделирования для 
современной версии протокола IEEE~802.11. Также рассмотрено их 
сопоставление с~аналитическими результатами, полученными выше. 
Реализованная система имитационного моделирования является гибким 
программным инструментом, включающим различные сценарии 
взаимодействия WiFi-агре\-га\-то\-ров, а также набор необходимых 
механизмов для управления доступом к~среде. Система моделирования 
была откалибрована по результатам, представленным 
    в~работе~\cite{8-om} (зависимость пропускной способности~$S$ от 
количества абонентов~$M$ в~канале). Для этого был оцифрован 
соответствующий график и~произведено наложение полученных 
результатов на воспроизведенные данные. Результаты можно наблюдать 
на рис.~3,\,\textit{а}. 

\begin{figure*} %fig3
\vspace*{1pt}
 \begin{center}  
\mbox{%
 \epsfxsize=162.694mm
 \epsfbox{ome-3.eps}
 }
\end{center} 
\vspace*{-9pt}
\Caption{Пропускная способность насыщения при $T\hm= 1$~(\textit{а}) и 65~Мбит/с~(\textit{б}) 
и~$K\hm\to\infty$: \textit{1}~--- Basic, $W_0=32$, $m=3$;
 \textit{2}~--- Basic, $W_0=32$, $m=5$;
      \textit{3}~--- Basic, $W_0=128$, $m=3$;
      \textit{4}~--- RTS/CTS, $W_0=32$, $m=3$; 
      \textit{5}~--- RTS/CTS, $W_0=128$, $m=3$}
      \vspace*{8pt}
      \end{figure*}
           \begin{figure*}[b] %fig4
     \vspace*{8pt}
 \begin{center}  
\mbox{%
 \epsfxsize=103.26mm
 \epsfbox{ome-5.eps}
 }
\end{center} 
\vspace*{-9pt}
\Caption{Максимальное число сенсоров с~агрегацией в~канале WiFi: 
\textit{1}~--- Basic, $W_0=128$, $m=3$;
      \textit{2}~--- RTS/CTS, $W_0=32$, $m=3$; 
      \textit{3}~--- RTS/CTS, $W_0=128$, $m=5$}
      \end{figure*}
     
    Следует отметить, что результаты работы~\cite{8-om} 
и~проведенного в~данной работе моделирования совпадают 
в~аналогичных условиях и~с~максимальной скоростью передачи 
$T\hm=1$~Мбит/с (система без потерь). 
    
    Далее рассмотрим сценарий, в~котором~$M$~агрегаторов в~кластере 
используют реалистичные длительности слотов согласно текущей 
спецификации IEEE 802.11-2014.\ Для этого установим более вы\-сокую 
скорость передачи $T\hm=65$~Мбит/с и~сопоставим анализ (кривые) 
с~результатами моделирования (символы) на рис.~3,\,\textit{б}.

    
    Рассмотрение системы, работающей без потерь (в~которой 
количество попыток повторной передачи не ограничено), воссоздает 
ситуацию, когда~$M$~агрегаторов отправляют сообщения к~точке 
доступа WiFi в~насыщенном режиме. С~по\-мощью\linebreak предлагаемого 
подхода получена максимально достижимая пропускная способность 
сис\-те\-мы. В~целом зависимость на рис.~3,\,\textit{б} аналогична результатам, 
приведенным на рис.~3,\,\textit{а}, но значение\linebreak 
достижимой пропускной 
способности в~случае $T\hm=65$~Мбит/с значительно выше. 
    
    В заключение рассмотрим систему предложенной топологии (см.\ 
рис.~1) с~различным числом агрегаторов (5--16) и~определим 
максимально возможное число сенсоров, которые могут быть обслужены 
такой системой.
 
Результаты на рис.~4 получены при условии, что каждый 
сенсор передает пакеты на агрегатор со средней скоростью~256~бит/с, 
а~максимальные пропускные способности в~восходящем канале (к~точке 
доступа WiFi) взяты согласно результатам, пред\-став\-лен\-ным на рис.~3,\,\textit{б}. 
В~работе~\cite{12-om} пред\-став\-ле\-на статистика реальной плот\-ности 
размещения сенсоров в~городских условиях, что для типового радиуса 
покрытия точки доступа WiFi (до~300~м) дает около~1000~устройств на 
агрегатор. При этом на рис.~4 видно, что при использовании 
предложенной в~текущем исследовании топологии сети, использующей 
WiFi-агре\-га\-цию, удается достичь расчетного числа поддерживаемых 
сенсоров даже при достаточно большом количестве агрегирующих 
устройств.


   
 \section{Заключение}
     
    В данной работе проведен анализ системы передачи данных от 
сенсорных устройств с~их промежуточной агрегацией и~пересылкой по 
технологии WiFi (IEEE 802.11-2014). Предполагается, что 
соответствующая топология станет типовой для многих  
IoT-при\-ло\-же\-ний с~большим количеством сенсоров, 
и~рассматривается этап состязания между WiFi-аг\-ре\-га\-то\-ра\-ми за выход 
в~беспроводной канал связи. 

Предлагается аналитическая модель, 
построенная на основе теории регенерирующих процессов 
и~учитывающая основные особенности работы протокола доступа к~каналу, 
а~также алгоритма разрешения коллизий. 
    %
    В частности, предполагается, что данные на агрегаторе могут быть 
потеряны, если они превысили определенное количество попыток 
повторной передачи. Значение пропускной способности насыщения, 
полученное в~рамках предложенной модели, сопоставляется с~данными 
имитационного моделирования. Делается вывод об их совпадении 
и~обосновывается максимальное число сенсоров, которые могут быть 
обслужены системой с~рас\-смат\-ри\-ва\-емой топологией. Полученные 
значения существенно превышают ожидаемое число сенсоров 
в~городских условиях даже при достаточно большом количестве 
агрегаторов, что подтверждает целесообразность использования 
технологии WiFi в~беспроводных системах агрегации данных с~большим 
числом сенсоров.



{\small\frenchspacing
 {%\baselineskip=10.8pt
 \addcontentsline{toc}{section}{References}
 \begin{thebibliography}{99}

\bibitem{1-om}
\Au{Ahmadian A., Galinina O.\,S., Gudkova~I.\,A., Andreev~S.\,D., 
Shorgin~S.\,Ya., Samouylov~K.\,E.} On capturing spatial diversity of joint 
M2M/H2H dynamic uplink transmissions in 3GPP LTE cellular system~// 
Next Generation Teletraffic and Wired/Wireless Advanced Networking 
Conference (International) Proceedings.~---  Lecture notes in computer 
science ser.~--- St.\ Petersburg, Russia, 2015. Vol.~9247. P.~407--421.
\bibitem{2-om}
\Au{Кучерявый А.\,Е.} Самоорганизующиеся сети и~новые услуги~// 
Электросвязь, 2009. Вып.~1. С.~19--23.
\bibitem{3-om}
\Au{Восков Л.\,С.} Беспроводные сенсорные сети и~прикладные 
проекты~// Автоматизация и~IT в~энергетике, 2009. №\,2-3. С.~44--49.
\bibitem{4-om}
\Au{Косинов М.\,И., Шорин О.\,А.} Повышение емкости сотовой системы 
связи при использовании зон перекрытия~// Электросвязь, 2003. Вып.~1. 
С.~18--20.
\bibitem{5-om}
\Au{Гайдамака Ю.\,В., Печинкин~А.\,В., Разумчик~Р.\,В., 
Самуйлов~А.\,К., Самуйлов~К.\,Е., Соколов~И.\,А., Сопин~Э.\,С., 
Шоргин~С.\,Я.} Распределение времени выхода из множества состояний 
перегрузки в~системе $M|M|1|\langle L,H \rangle |\langle H,R \rangle$ 
с~гистерезисным управлением нагрузкой~// Информатика и~её 
применения, 2013. Т.~7. Вып.~4. С.~20--33.
\bibitem{6-om}
\Au{Bianchi G.} Performance analysis of the IEEE 802.11 distributed 
coordination function~// IEEE J.~Sel. Area. Comm., 2000. 
Vol.~18. No.\,3. P.~535--547.
\bibitem{7-om}
\Au{Skordoulis D., Ni~Q., Chen~H.\,H., Stephens~A.\,P., Liu~C., 
Jamalipour~A.} IEEE 802.11n MAC frame aggregation mechanisms for  
next-generation high-throughput WLANs~// IEEE Wirel. Commun., 
2008. Vol.~15. No.\,1. P.~40--47.

\bibitem{9-om} %8
\Au{Sharma G., Ganesh A., Key~P., Needham~R.} Performance analysis of 
contention based medium access control protocols~// IEEE Trans. 
Inform. Theory, 2009. Vol.~55. No.\,4. P.~1665--1682.
\bibitem{10-om} %9
\Au{Malone D., Duffy~K., Leith~D.} Modeling the 802.11 distributed 
coordenation function in non-saturated heterogeneous conditions~// 
IEEE/ACM Trans. Networks, 2007. Vol.~15. No.\,1. P.~159--172.
\bibitem{8-om} %10
\Au{Bordenave C., McDonald D., Proutire~A.} Random multi-access 
algorithms~--- a mean field analysis~// Rapport de Recherche, 2005. Vol.~5632. 
P.~1--12.
\bibitem{11-om}
\Au{Andreev S., Koucheryavy~Y., Sousa~L.} Calculation of transmission 
probability in heterogeneous ad hoc networks~// IEEE Baltic Congress on 
Future Internet and Communications (BCFIC) Proceedings, 2011. P.~75--82.
\bibitem{12-om}
\Au{Ortiz S.} IEEE 802.11n: The road ahead~// IEEE Computer, 2009. 
Vol.~42. No.\,7. P.~13--15.
\end{thebibliography}

 }
 }

\end{multicols}

\vspace*{-6pt}

\hfill{\small\textit{Поступила в~редакцию 06.04.16}}

%\vspace*{8pt}

\newpage

\vspace*{-30pt}

%\hrule

%\vspace*{2pt}

%\hrule

%\vspace*{8pt}



\def\tit{PERFORMANCE ANALYSIS OF~A~WIRELESS DATA 
AGGREGATION SYSTEM WITH~CONTENTION 
FOR~CONTEMPORARY SENSOR NETWORKS}

\def\titkol{Performance analysis of~a~wireless data 
aggregation system with~contention 
for~contemporary sensor networks}

\def\aut{A.\,Ya.~Ometov$^1$, S.\,D.~Andreev$^2$, 
A.\,M.~Turlikov$^3$, and~E.\,A.~Koucheryavy$^4$}

\def\autkol{A.\,Ya.~Ometov, S.\,D.~Andreev, 
A.\,M.~Turlikov, and~E.\,A.~Koucheryavy}

\titel{\tit}{\aut}{\autkol}{\titkol}

\vspace*{-9pt}

\noindent
        $^1$Saint-Petersburg State University of Telecommunications, 
22B~Pr.~Bolshevikov,  St.\ Petersburg 193232, Russian\linebreak
$\hphantom{^1}$Federation 
        
        \noindent
        $^2$Peoples' Friendship University of Russia, 3~Ordzhonikidze Str., 
Moscow 115419, Russian Federation
     
     \noindent
        $^3$State University of Aerospace Instrumentation, 67~Bolshaya 
Morskaya Str., St.\ Petersburg 190000, Russian\linebreak
$\hphantom{^1}$Federation
     
     \noindent
        $^4$National Research University Higher School of Economics, 
30~Myasnitskaya Str.,  Moscow 101000, Russian\linebreak
$\hphantom{^1}$Federation


\def\leftfootline{\small{\textbf{\thepage}
\hfill INFORMATIKA I EE PRIMENENIYA~--- INFORMATICS AND
APPLICATIONS\ \ \ 2016\ \ \ volume~10\ \ \ issue\ 3}
}%
 \def\rightfootline{\small{INFORMATIKA I EE PRIMENENIYA~---
INFORMATICS AND APPLICATIONS\ \ \ 2016\ \ \ volume~10\ \ \ issue\ 3
\hfill \textbf{\thepage}}}

\vspace*{3pt}
       
    
        
     \Abste{The paper considers a wireless communication 
system with a number of sensing devices that transmit their data to 
multiple aggregating nodes connected to Internet via IEEE  
802.11-2012 (WiFi) technology. It is assumed that an aggregator 
retransmits data from many sensors by competing with other 
aggregators for the shared channel. The paper proposes an 
analytical model taking into account the features of the collision 
resolution algorithm, the properties of the channel access protocol, 
as well as the possibility to discard data at the aggregator. The 
obtained analytical results are compared with the simulation data, 
and the maximum number of supported sensors in the 
communication system is estimated.}
     
     \KWE{Internet of Things; wireless sensor networks; 
saturated system; regenerative analysis; WLAN; IEEE 802.11-2014 
standard}


\DOI{10.14357/19922264160304}

\vspace*{-9pt}

\Ack
\noindent
This work is supported by the Russian Foundation for 
Basic Research (project No.\,15-07-03051) and the Foundation for Assistance to 
Small Innovative Enterprises (FASIE) within the program ``UMNIK'' 
under grant 8268GU2015 (02.12.2015).


\vspace*{9pt}

  \begin{multicols}{2}

\renewcommand{\bibname}{\protect\rmfamily References}
%\renewcommand{\bibname}{\large\protect\rm References}

{\small\frenchspacing
 {%\baselineskip=10.8pt
 \addcontentsline{toc}{section}{References}
 \begin{thebibliography}{99}
\bibitem{1-om-1}
\Aue{Ahmadian, A.\,M., O.\,S.~Galinina, I.\,A.~Gudkova, S.\,D.~Andreev, 
S.\,Ya.~Shorgin, and K.\,E.~Samouylov}. 2015. On capturing spatial diversity 
of joint M2M/H2H dynamic uplink transmissions in 3GPP LTE cellular 
system. \textit{International Next Generation Teletraffic and Wired/Wireless 
Advanced Networking Conference Proceedings}. 
Lecture notes in computer science ser. St.\ Petersburg, Russia.  9247:407--421.
\bibitem{2-om-1}
\Aue{Koucheryavy, A.\,E.} 2009. Samoorganizuyushchiesya seti i~novye 
uslugi [Ad hoc networks and new services]. \textit{Elektrosvyaz'} 
[Telecommunications] 1:19--23.
\bibitem{3-om-1}
\Aue{Voskov, L.\,S.} 2009. Besprovodnye sensornye seti i~prikladnye proekty 
[Wireless networks and application projects]. \textit{Avtomatizatsiya i~IT 
v~energetike} [IT automatization in Enegetics] 2-3:44--49.
\bibitem{4-om-1}
\Aue{Kosinov, M.\,I., and O.\,A.~Shorin}. 2003. Povyshenie emkosti sotovoy 
sistemy svyazi pri ispol'zovanii zon perekrytiya [Increasing cellular network 
capacity utilizing the overlapping zones]. \textit{Elektrosvyaz'} 
[Telecommunications] 1:18--20.
\bibitem{5-om-1}
\Aue{Gaydamaka, Yu.\,V., A.\,V.~Pechinkin, R.\,V.~Razumchik, 
A.\,K.~Samuylov, K.\,E.~Samuylov, I.\,A.~Sokolov, E.\,S.~Sopin, and 
S.\,Ya.~Shorgin}. 2013. Raspredelenie vremeni vykhoda iz mnozhestva 
sostoyaniy peregruzki v~sisteme $M|M|1|\langle L,H \rangle |\langle H,R 
\rangle$ s~gisterezisnym upravleniem nagruzkoy [The distribution of the return 
time from the set of overload states to the set of normal load states in a~system 
$M|M|1|\langle L,H \rangle |\langle H,R \rangle$ with hysteretic load control]. 
\textit{Informatika i~ee Primeneniya~--- Inform. Appl.} 7(4):\linebreak 20--33.
\bibitem{6-om-1}
\Aue{Bianchi, G.} 2000. Performance analysis of the IEEE 802.11 distributed 
coordination function. \textit{IEEE J.~Sel. Area. Comm}. 
18(3):535--547.
\bibitem{7-om-1}
\Aue{Skordoulis, D., Q.~Ni, H.\,H.~Chen, A.\,P.~Stephens, C.~Liu, and 
A.~Jamalipour}. 2008. IEEE 802.11~n~MAC frame aggregation mechanisms 
for next-generation high-throughput WLANs. \textit{IEEE Wirel. 
Commun.} 15(1):40--47.


\bibitem{9-om-1} %8
\Aue{Sharma, G., A.~Ganesh, P.~Key, and R.~Needham}. 2009. Performance 
analysis of contention based medium access control protocols. \textit{IEEE 
Trans. Inform. Theory} 55(4):1665--1682.

\pagebreak

\bibitem{10-om-1} %9
\Au{Malone, D., K.~Duffy, and D.~Leith}. 2007. Modeling the 802.11 
distributed coordenation function in non-saturated heterogeneous conditions.  
\textit{IEEE/ACM Trans. Networks} 15(1):159--172.
\bibitem{8-om-1} %10
\Aue{Bordenave, C., D.~McDonald, and A.~Proutire}. 2005. Random  
multi-access algorithms~--- a~mean field analysis. \textit{Rapport de 
Recherche}  5632:1--12.

%\columnbreak

\bibitem{11-om-1}
\Aue{Andreev, S., Y.~Koucheryavy, and L.~Sousa}. 2011. Calculation of 
transmission probability in heterogeneous ad hoc networks. \textit{IEEE Baltic 
Congress on Future Internet and Communications (BCFIC) Proceedings}.  
75--82.
\bibitem{12-om-1}
\Aue{Ortiz, S.} 2009. IEEE 802.11n: The road ahead. \textit{IEEE 
Computer} 42(7):13--15.
\end{thebibliography}

 }
 }

\end{multicols}

\vspace*{-3pt}

\hfill{\small\textit{Received April 06, 2016}}

\Contr

\noindent
\textbf{Ometov Aleksandr Ya.} (b.\ 1991)~--- PhD student, St.\ Petersburg State University 
of Telecommunications, 22B~Bolshevikov Pr., St.\ Petersburg 193232, Russian Federation; 
\mbox{alеxаnder.omеtov@gmаil.com}

\vspace*{4pt}

\noindent
\textbf{Andreev Sergey D.} (b.\ 1984)~--- Candidate of Sciences, PhD; associate professor, 
Peoples' Friendship University of Russia, 3~Ordzhonikidze Str., Moscow 115419, Russian 
Federation; \mbox{serge.аndeev@gmаil.com}

\vspace*{4pt}

\noindent
\textbf{Turlikov Andrey M.} (b.\ 1957)~--- Doctor of Sciences, professor; Head of 
Department, St.\ Petersburg State University of Aerospace Instrumentation, 67~Bolshaya 
Morskaya Str., St.\ Petersburg 190000, Russian Federation; \mbox{turlikоv@vu.spb.ru}

\vspace*{4pt}

\noindent
\textbf{Koucheryavy Evgeni A.} (b.\ 1974)~--- Candidate of Sciences, PhD; professor, 
National Research University Higher School of Economics, 20 Myasnitskaya Str., Moscow 
101000, Russian Federation; \mbox{ykоucheryаvy@hsе.ru}

\label{end\stat}


\renewcommand{\bibname}{\protect\rm Литература}
      %4
\def\stat{krivenko}

\def\tit{МНОГОМЕРНЫЙ РЕФЕРЕНСНЫЙ РЕГИОН\\ ВЫСОКОЙ ПЛОТНОСТИ}

\def\titkol{Многомерный референсный регион высокой плотности}

\def\aut{М.\,П.~Кривенко$^1$}

\def\autkol{М.\,П.~Кривенко}

\titel{\tit}{\aut}{\autkol}{\titkol}

\index{Кривенко М.\,П.}
\index{Krivenko M.\,P.}


%{\renewcommand{\thefootnote}{\fnsymbol{footnote}} \footnotetext[1]
%{Работа выполнена при финансовой поддержке РФФИ (проекты 16-07-00677 
%и~15-37-20611-мол\_а\_вед).}}


\renewcommand{\thefootnote}{\arabic{footnote}}
\footnotetext[1]{Институт проблем информатики Федерального исследовательского центра <<Информатика и~управление>> Российской академии наук,
\mbox{mkrivenko@ipiran.ru}}

\vspace*{4pt}



\Abst{Рассматриваются принципы построения многомерных референсных регионов
(MRR~--- multivariate reference region). 
Предложен оригинальный метод построения региона на основе областей с~высокой 
плотностью точек и~аппроксимации распределения данных с~помощью смеси нормальных 
распределений. Для оценки порога для плотности распределения используется  
бут\-стреп-ме\-тод. В~качестве эксперимента рассмотрена задача построения 
и~использования эталонной области для прогнозирования типа мочевого камня. Обработка 
реальных данных продемонстрировала преимущества предлагаемых решений.}

\KW{многомерный референсный регион; область высокой плотности; бут\-стреп-ме\-тод; 
смесь многомерных нормальных распределений}

\vspace*{6pt}

\DOI{10.14357/19922264170207} 


\vskip 10pt plus 9pt minus 6pt

\thispagestyle{headings}

\begin{multicols}{2}

\label{st\stat}

\section{Введение}

     Многомерный референсный регион 
был предложен в~литературе по клинической химии в~начале 1970-х~гг.\ как 
альтернатива одномерным референсным интервалам~[1]. Там излагались 
преимущества предлагаемых множественных тестов, хоть и~имеющих 
упрощенный вид, но снижающих (по отношению к~одномерным вариантам) 
число ложных положительных результатов. Появление MRR оказалось 
особенно привлекательным для интерпретации результатов наборов 
медицинских тестов. Тем не менее возникали трудности в~построении 
и~использовании процедур многомерного анализа (см., например,~[2]), 
связанные, в~частности, с~быстрым увеличением числа параметров, которые 
должны быть оценены. Немногие лаборатории использовали MRR в~своей 
практике, причем в~экспериментальном режиме, и,~как следствие, на 
сегодняшний день имеется относительно малое количество соответствующих 
публикаций. 

\vspace*{-6pt}

\section{Многомерный референсный регион на основе расстояния Махалонобиса}

\vspace*{-2pt}

     Одномерный референсный интервал, полученный статистическим путем, 
использует центральную часть значений анализируемого показателя, обычно 
соответствующую~95\% некоторой популяции~--- совокупности особей 
определенного вида (например, здоровой части населения определенного пола 
из некоторого диапазона возрастов). Одномерные референсные интервалы 
применялись в~течение многих лет в~качестве стандартного приема 
интерпретации лабораторных данных. Они легко формируются, хранятся, 
извлекаются и~передаются в~лабораторных информационных системах, просты 
в~понимании, хорошо воспринимаются медицинским сообществом в~ходе 
длительного использования. Тем не менее одномерные референсные интервалы 
при классификации данных могут дать большое число ложно аномальных 
результатов. Этот далеко не единственный недостаток однофакторного 
референсного интервала может быть полностью или частично устранен 
с~помощью MRR.
     
     Простейшим и~весьма распространенным способом построения MRR 
является использование прямого произведения отдельных референсных 
интервалов в~предположении, что они статистически независимы. Пусть 
$(1\hm-\alpha)$~--- вероятность попадания в~MRR, а~$p_0$~--- вероятность 
попадания в~референсный интервал для любого из~$d$~признаков, тогда 
$p_0\hm= \sqrt[d]{1-\alpha}$. С~ростом размерности~$d$ значения~$p_0$ 
быстро приближаются к~1, что фактически лишает смысла применение MRR.
     
     Как и~в одномерном случае, отправной точкой для построения MRR 
может стать нормальное распределение. Идеи центрального расположения 
референсного региона и~заданной вероятности попадания в~него приводят для 
$d$-мер\-но\-го нормального распределения, имеющего плотность 
распределения
     \begin{multline*}
     \varphi(y,\mu,\Sigma) ={}\\
     {}=(2\pi)^{-d/2}\vert\Sigma\vert^{-1/2}\exp \left( -\fr{\left(y-
\mu\right)^{\mathrm{T}} \Sigma^{-1}(y-\mu)}{2}\right),
   \end{multline*}
где величина $(y-\mu)^{\mathrm{T}} \Sigma^{-1} (y-\mu)$ есть квадрат так 
называемого расстояния Махаланобиса между~$y$ и~$\mu$, к~использованию 
многомерного эллипсоида
\begin{multline*}
(2\pi)^{-d/2}\vert\Sigma\vert^{-1/2}\exp \left( -\fr{\left(y-\mu\right)^{\mathrm{T}}
\Sigma^{-1} 
(y-\mu)}{2}\right) ={}\\
{}=const
\end{multline*}
или, что то же самое, 
$$ 
(y-\mu)^{\mathrm{T}} \Sigma^{-1}(y-\mu)=const\,.
$$
Его называют эллипсоидом равной плотности распределения (или просто 
эллипсоидом равной вероятности). 
     
     Если задаться вероятностью $(1\hm-\alpha)$ попадания в~эллипсоид 
равной вероятности вида $(y\hm-\mu)^{\mathrm{T}}\Sigma^{-1} (y\hm-\mu)\hm= 
\rho$, то параметр~$\rho$ удовлетворяет уравнению $\mathrm{Pr}\left\{ 
\chi_d^2\leq \rho\right\} \hm=1\hm-\alpha$.
     
     Использование эллипсоида в~качестве MRR будет оправдано только 
тогда, когда исходное распределение данных есть многомерное нормаль-\linebreak ное. 
Поэтому становятся актуальными критерии\linebreak подгонки, а~также использование 
процедур норма\-ли\-зации распределения данных в~многомерном\linebreak случае.
 Если 
с~помощью тестов выявляется, что распределение не является нормальным, то 
Международная федерация клинической химии и~лабораторной медицины 
рекомендует, согласно~[3], использовать двухступенчатую процедуру 
нормализации. Следует обратить внимание, что многошаговость здесь 
относится не к~многомерности, а касается лишь покоординатного 
преобразования распределения данных к~нормальному.
     
     Первые же попытки применения MRR на основе расстояния 
Махалонобиса (фактически это означает принятие модели нормального 
распределения референсных значений) выявили ряд недостатков (более 
подробно смотри в~\cite[разд.~6.2]{4-kri}):
     \begin{itemize}
\item проявление <<проклятий>> размерности при механическом 
увеличении~$d$, в~особенности если игнорируется этап анализа состава 
признаков~[1, 5, 6];
\item из-за небольших объемов обучающей выборки невысокая устойчивость 
при применении, в~частности чувствительность к~увеличению неточностей 
измерений после того, как регион был установлен~\cite{5-kri, 7-kri}. 
\item предположение о нормальном распределении и~попытки <<подправить>> 
действительность с~помощью преобразований реальных данных для их 
нормализации при увеличении размерности данных становятся все более 
шаткими~\cite{5-kri};
\item представление и~интерпретация выводов на основе MRR трудно 
понимаемы не только для специалистов в~предметной области~[8].
\end{itemize}

\vspace*{-9pt}

\section{Многомерный референсный регион высокой плотности}

\vspace*{-2pt}

     Заметим, что в~случае нормального распределения референсных значений 
для точек внут\-ри построенного эллипсоида значения плотности\linebreak распределения 
больше, чем на границе, а~вне~--- меньше. Это замечание позволяет 
предложить другой подход к~построению MRR.
     
     \smallskip
     
     \noindent
     \textbf{Определение.}\ Eсли плотность распределения референсных 
значений есть $f(y)$, то MRR есть область $A_t\hm= \left\{ y\in 
\mathcal{R}^d\vert f(y)\hm\geq t\right\}$ для некоторого порогового 
значения~$t$. 
     
     \smallskip
     
     Для нормального распределения это уже упомянутый эллипсоид равной 
вероятности. Если задается вероятность $(1\hm-\alpha)$ попадания в~$A_t$, то 
пороговое значение~$t$ есть решение уравнения $\int\nolimits_{A_t} 
f(u)\,du\hm=1\hm-\alpha$, получить которое аналитически в~случае 
произвольной плотности распределения вряд ли возможно. Здесь присутствуют 
две проблемы: вычисление многомерного интеграла и~зависимость области 
интегрирования от неизвестного значения. Для решения их предлагается 
привлечь метод моделирования.
     
     Сгенерируем выборку из $f(y)$, которую обозначим как $Y^f\hm= \left\{ 
y_1^f, \ldots, y_m^f\right\}$. Для оценки $\int\nolimits_{A_t} f(u)\,du$ 
используем отношение:

\noindent
\begin{multline*}
     \fr{\left\vert \left\{ y_i^f\vert y_i^f\in A_t\right\}\right\vert }{m} =
      \fr{\left\vert\left\{ y_i^f\vert 
f\left(y_i^f\right) \geq t\right\}\right\vert }{m} ={}\\
{}= 1-\fr{\left\vert \left\{ y_i^f\vert f(y_i^f)<t\right\}\right\vert }{m}=1-
F_m(t)\,,
     \end{multline*}
где $F_m(t)$~--- эмпирическая функция распределения случайной 
величины~$f(y)$, т.\,е.\ случайной величины, являющейся результатом 
преобразования с~помощью функции~$f(\cdot)$ случайной величины, име\-ющей 
плотность распределения~$f(u)$.

     Таким образом, искомая оценка~$t^*$ должна удовле\-тво\-рять уравнению 
$F_m(t^*)\hm=\alpha$ и~может быть получена как непараметрическая оценка 
квантиля\linebreak\vspace*{-12pt}

\pagebreak

\noindent
 порядка~$\alpha$ из распределения $F_m(\cdot)$. Если обозначить 
$f_i\hm= f(y_i^f)$, то~$t^*$ есть~$f_{(r)}$, где
     $$
     r= \begin{cases}
     m\alpha, &\ m\alpha~\mbox{---~целое}\,;\\
     \lfloor m\alpha+1\rfloor\,, & m\alpha~\mbox{--- не целое}\,.
     \end{cases}
     $$
     Заметим, что для такой оценки можно указать доверительный интервал.
     
     Для построения MRR необходимо знать распределение данных. При 
реализации принципа точек высокой плотности в~первую очередь следует 
обратиться к~параметрическим моделям, в~част\-ности к~смеси нормальных 
распределений, име\-ющей плотность распределения
     $$
     f(u) =\sum\limits_{j=1}^k p_j \varphi\left (u,\mu_j, \Sigma_j\right)\,.
     $$
Если $\hat{f}(u)$~--- оценка смеси, то~$t^*$ строится сле\-ду\-ющим образом:
\begin{itemize}
\item генерируется выборка $\left\{ y_1^f,\ldots , y_m^f\right\}$ из $\hat{f}(u)$ и~
для каждого ее $i$-го элемента подсчитывается значение $\hat{f}\left( 
y_i^f\right)$;
\item в~качестве~$t^*$ берется непараметрическая оценка квантиля 
порядка~$\alpha$ (в случае необходимости дополнительно находится 
непараметрическая оценка доверительного интервала для~$t^*$, что 
может характеризовать правильность выбранного объема для 
генерируемой выборки).
\end{itemize}

     Пусть для $f(u)$ имеется~$A_t$, а также получена $\hat{f}(u)$ 
и~соответствующий MRR вида~$\hat{A}_t$. Качество аппроксимации~$A_t$ 
с~по\-мощью~$\hat{A}_t$ можно оценить с~по\-мощью вероятности совпадения 
этих областей, т.\,е. 
     $$
     P_c= \int\limits_{\{ u\in A_t\}\cup \{u\in \hat{A}_t\}} \hspace*{-6mm}
f(u)\,du+\int\limits_{\{u\not\in A_t\} \cup\{ u\not\in \hat{A}_t\}}\hspace*{-6mm} f(u)\,du\,.
     $$
     
     Для оценки  $P_c$ можно использовать величину
     \begin{multline*}
     \hat{P}_c= \fr{\left\vert \left\{ 
     y_i^f\vert y_i^f \in \left\{\left\{ y_i^f\in A_t\right\}\cup \left\{y_i^f\in 
\hat{A}_t\right\}\right\}\right\}\right\vert}{m}+{}\\
{}+\fr{\left\vert \left\{ y_i^f\vert y_i^f \in \left\{\left\{ y_i^f\not\in A_t\right\}\cup 
\left\{ y_i^f\not\in \hat{A}_t\right\}\right\}\right\}\right\vert}{m}\,.
     \end{multline*}
     
     Использование MRR высокой плотности для диагностирования сводится 
к~реализации так называемого слабого критерия значимости для наблюденного 
значения~$x$: нулевая гипотеза заключается в~том, что $x\hm\in A_t$, 
статистика критерия есть $\hat{f}(x)$ и~решение о~принадлежности 
критической об\-ласти~$A_t$ принимается при больших значениях~$\hat{f}(x)$.
     
     Для медицинской практики важна возможность использования 
референсного региона при интерпретации результатов обследования 
некоторого пациента с~вектором признаков~$x$. В~подобных случаях 
сложившейся практикой для слабых критериев значимости является 
использование критического уровня~$\alpha_{\mathrm{cr}}$ (более распространенным 
в~медицине является употребление термина $p$-зна\-че\-ние)  $\alpha_{\mathrm{cr}}\hm= 
\mathrm{Pr}\left\{ \hat{f}(y)\hm\leq \hat{f}(x)\right\}$, где $y$~--- случайная 
величина, имеющая плотность распределения~$\hat{f}(u)$, а $\hat{f}(x)$~--- 
значение плотности распределения~$\hat{f}(u)$ в~точке~$x$. Эта 
характеристика дает представление о~том, насколько сильно данное 
наблюденное значение~$x$ противоречит гипотезе (или подкрепляет ее) 
о~принадлежности данных MRR. При выбранном же заранее уровне 
значимости с~помощью~$\alpha_{\mathrm{cr}}$ сразу же можно принять конкретное 
решение. 

\vspace*{-9pt}

\section{Эксперименты}

\vspace*{-2pt}

     Для демонстрации возможностей MRR использовались данные по 
прогнозу химического состава мочевых камней по метаболическим 
показателям мочи и~сыворотки крови, а также антропологическим 
характеристикам пациентов~[9]. В качестве исходной классификации камней 
рассматривалась следующая: чисто оксалатные (далее обозначены как O), чисто 
уратные (U), чисто фосфатные (P), смесь только оксалатных и~уратных (OU), 
смесь только оксалатных и~фосфатных (OP), смесь только уратных 
и~фосфатных (UP), все остальные. Данная классификация была построена 
в~[10] на основе доминирующих частот встречаемости основных компонентов. 
В~качестве референсных значений рассматривались наборы метаболических 
и~антропологических показателей (их всего было~14), соответствующих 
определенному классу камней.

\begin{table*}\small
\begin{center}


\begin{tabular}{|c|c|c|c|c|c|c|}
\multicolumn{7}{c}{Качество классификации с~помощью MRR}\\
\multicolumn{7}{c}{\ }\\[-6pt]
\hline
\multicolumn{1}{|c|}{\raisebox{-6pt}[0pt][0pt]{\tabcolsep=0pt\begin{tabular}{c}Тип\\ камня\end{tabular}}}&
\multicolumn{1}{c|}{\raisebox{-6pt}[0pt][0pt]{$N$}}&$(1-\alpha)$, 
&\multicolumn{2}{c|}{MRR(5)}&\multicolumn{2}{c|}{MRR(1)}\\
\cline{4-7}
&&&&&&\\[-9pt]
&&\%&$(1-\hat{\alpha})$, \%&$\hat{\beta}$, \%&$(1-\hat{\alpha})$, \%&$\hat{\beta}$, \%\\
\hline
\multicolumn{1}{|c|}{\raisebox{-18pt}[0pt][0pt]{O}}&
\multicolumn{1}{c|}{\raisebox{-18pt}[0pt][0pt]{82}}
&95&100\hphantom{9}&71&90&24\\
&&85&96&78&89&36\\
&&75&91&85&77&44\\
&&65&76&88&74&50\\
\hline
\multicolumn{1}{|c|}{\raisebox{-18pt}[0pt][0pt]{U}}&
\multicolumn{1}{c|}{\raisebox{-18pt}[0pt][0pt]{76}}&95&100\hphantom{9}&75&91&24\\
&&85&99&85&80&35\\
&&75&82&89&74&48\\
&&65&71&91&68&56\\
\hline
\multicolumn{1}{|c|}{\raisebox{-18pt}[0pt][0pt]{P}}&
\multicolumn{1}{c|}{\raisebox{-18pt}[0pt][0pt]{83}}&95&100\hphantom{9}&66&87&25\\
&&85&94&78&86&33\\
&&75&86&82&82&41\\
&&65&77&87&75&47\\
\hline
\end{tabular}
\end{center}
\end{table*}
     
     
     Для каждого из основных классов O, U, P, OU, OP и~UP перед построением 
MRR проводилась селекция признаков и~принималось то значение размерности 
признакового пространства~$d$ и~соответствующий набор показателей, 
которые позволяли прогнозировать состав камней без потери качества 
(методика описана в~\cite{9-kri} и~привела к~значению $d\hm=9$). В~качестве 
модели данных в~первую очередь рассматривалась смесь многомерных 
нормальных распределений из пяти элементов (подбор числа элементов смеси 
проводился с~по\-мощью AIC~--- Akaike information criterion), для соответствующего региона было принято 
обозначение MRR(5). Для сравнения также использовалась модель 
нормального распределения, которой соответствовал MRR(1). Полученные 
результаты приводятся час\-тич\-но в~таблице, где $N$~--- объем 
классифицируемых данных; $\hat{\alpha}$~--- оценка для~$\alpha$; 
$\hat{\beta}$~--- оценка мощности критерия при определении типа камня на 
основании MRR.


     Одной из базовых характеристик является вероятность попадания в~MRR 
$(1\hm-\alpha)$ и~ее оценка $(1\hm-\hat{\alpha})$. Сравнение соответствующих 
столбцов с~учетом значений~$N$ и~ориентировочных значений разброса 
(стандартные отклонения на основе биномиального распределения) не 
позволило выявить явных отклонений. Необходимо, правда, отметить, что во 
всех проанализированных случаях для MRR(5) оказалось, что $1\hm-
\hat{\alpha}\hm\geq 1\hm-\alpha$.
     
     Назначение MRR, заключающееся в~сжатом представлении референсных 
значений, в~многомерном случае практически не проявляется. Для задания 
MRR(5) необходимо указать следующие величины: $1\hm-\alpha$, $t$, 
$p_1,\ldots, p_{k-1}$, $\mu_1, \Sigma_1,\ldots , \mu_k,\Sigma_k$, общее 
количество которых равно  $[2\hm+ (k\hm-1)\hm+ k(d\hm+ d(d\hm+1)/2)]$ 
и,~в~частности, в~рассматриваемых экспериментах~--- 276. Для MRR(1) это 
значение меньше и~равно~56. При этом для обрабатываемой обучающей 
выборки в~зависимости от класса камней речь идет о~порядка~10$^2$ векторах 
данных (см.\ столбец со значениями~$N$), что приблизительно 
дает~10$^3$~скалярных величин.
     
     Другое назначение MRR состоит в~его использовании для 
диагностирования (классификации). В~этой связи в~первую очередь 
проводился сравнительный анализ MRR(1) (фактически это означает, что 
построение региона осуществляется на основе расстояния Махаланобиса) 
и~MRR(5) (модель смеси нормальных распределений и~предложенный 
в~данной работе метод оценивания па\-ра\-мет\-ров региона). Показателем 
информативности метода построения многомерного региона выступала 
мощность соответствующего слабого критерия значимости, а~именно: 
вероятность не попасть в~MRR при условии, что данные берутся из дополнения 
к~классу, для которого построена MRR. Сравнение соответствующих столбцов 
говорит о~явном преимуществе двух предложенных моментов: усложнение 
модели данных путем перехода от нормального распределения к~смеси 
нормальных распределений и~построение региона высокой плотности.
     
     Использование критического уровня можно продемонстрировать  
с~по\-мощью зависимости результатов сравнения двух классов от того, какой 
класс взять за основу. Введем для возможных значений $p$-ве\-ли\-чи\-ны три 
интервала: $(-\infty, 1\%)$, $[1\%, 5\%)$, $[5\%, 100\%)$ с~соответствующей 
интерпретацией положения наблюденного набора показателей для пациента 
относительно построенного MRR: уверенное непопадание, неуверенное 
попадание, уверенное попадание. Если MRR построить для оксалатных камней, 
то результаты для анализа пациентов с~фосфатными камнями дадут следующий 
вектор относительных частот попадания $p$-ве\-ли\-чин в~указанные 
интервалы: $(60\%, 18\%, 22\%)$. Если же MRR строить для фосфатных 
камней, то получим $(71\%, 5\%, 24\%)$. Таким образом, для классификации 
указанных камней при приблизительно одинаковых частотах попадания в~MRR 
(22\% или~24\%) уверенный отказ от референсного региона происходит чаще, 
если принять за базовый MRR регион для фосфатных камней. Построение 
шкалы, подобной рассмотренной, является прерогативой специалистов 
в~предметной области, в~данной работе она использовалась только для 
иллюстрации. 

\vspace*{-6pt}

\section{Заключение}

\vspace*{-2pt}

     На настоящий момент имеется относительно мало примеров применения 
MRR в~клинической практике. Тому есть несколько причин. Математическое 
обеспечение, необходимое для получения и~применения MRR, не отвечает 
возможностям большинства клинических лабораторий. Лаборатории слабо 
оснащены программными средствами\linebreak для реализации достаточно сложного 
математического аппарата многомерного анализа, а~еще важнее, что 
отсутствуют методики, инструкции по\linebreak использованию соответствующих 
средств. Лишь немногие клинические применения демонстрируют 
преимущества MRR, хотя свидетельств неудачных попыток больше.
     
     Несмотря на сложности внедрения мно\-го\-мерно\-го анализа референсных 
значений, можно сформулировать некоторые рекомендации по иссле\-до\-ва\-нию 
и~разработке MRR. Во-пер\-вых, эффективная размерность в~MRR должна 
быть как можно меньше, чтобы избежать затенения диагностически полезной 
информации тестами, со\-зда\-ющи\-ми шум. Низкая размерность также должна 
уменьшить неблагоприятные последствия увеличения неточности результатов 
в~связи с~ростом числа анализируемых показателей. Во-вто\-рых, показатели 
(тес\-ты), включенные в~MRR, должны быть физиологически релевантными 
исследуемому кругу расстройств, чтобы максимизировать информацию, 
полученную от MRR. В-треть\-их, чтобы учесть эффекты долгосрочной 
лабораторной из\-мен\-чи\-вости, данные, используемые для получения MRR, 
долж\-ны быть собраны и~проанализированы в~течение достаточно большого 
периода времени (от нескольких недель до нескольких месяцев).  
В-чет\-вер\-тых, представление результатов лабораторных исследований 
следует осуществлять в~графическом виде, чтобы помочь врачам лучше понять 
MRR. Различные подходы к~уменьшению размерности помогут выполнить это 
требование.
     
     Необходима дальнейшая разработка пояснительных инструментов, 
способных воспринять результаты анализа MRR. При этом дополнительно 
необходима информация о~том, какие именно тес\-ты являются важнейшими 
факторами нарушения нормы. Надо признать, что соответствующий 
математический аппарат еще предстоит разработать. Решение перечисленных 
вопросов играет важную роль для обеспечения постоянного клинического 
применения MRR. 

\vspace*{-6pt}
     
{\small\frenchspacing
 {%\baselineskip=10.8pt
 \addcontentsline{toc}{section}{References}
 \begin{thebibliography}{99}
 
 \vspace*{-2pt}
 
\bibitem{1-kri}
\Au{Boyd J.\,C.} Reference regions of two or more dimensions~// Clin. Chem. Lab. 
Med., 2004. Vol.~42. No.\,7. P.~739--746.
\bibitem{2-kri}
\Au{Winkel P.} Patterns and clusters~--- multivariate approach for interpreting 
clinical chemistry results~// Clin. Chem., 1973. Vol.~19. No.\,12. P.~1329--1333.
\bibitem{3-kri}
IFCC. Expert panel on theory of reference values. Approved recommendation on the 
theory of reference values. Part~5. Statistical treatment of collected reference values. 
Determination of reference limits~// J.~Clin. Chem. Clin. Biochem., 1987. Vol.~25. 
No.\,9. P.~645--656.
\bibitem{4-kri}
\Au{Кривенко М.\,П.} Статистические методы представления и~предварительной 
обработки референсных значений.~--- М.: ФИЦ ИУ РАН, 2016. 160~с.
\bibitem{5-kri}
\Au{Boyd J.\,C., Lacher~D.\,A.} The multivariate reference range: An alternative 
interpretation of multi-test profiles~// Clin. Chem., 1982. Vol.~28. No.\,2.  
P.~259--265.
\bibitem{6-kri}
\Au{Albert A., Harris~E.\,K.} Multivariate interpretation of clinical laboratory  
data.~--- New York, NY, USA: CRC Press, 1987. 328~p.
\bibitem{7-kri}
\Au{Linnet K.} Influence of sampling variation and analytical errors on the 
performance of the multivariate reference region~// Meth. Inf. Med., 1988. Vol.~27. 
No.\,1. P.~37--42.
\bibitem{8-kri}
\Au{Durbridge T.\,C.} Clinical acceptance of a multi-test reference region for 
biochemical-panel results~// Clin. Chem., 1983. Vol.~29. No.\,10. P.~1724--1726.
\bibitem{9-kri}
\Au{Кривенко М.\,П.} Критерии значимости отбора признаков классификации~// 
Информатика и~её применения, 2016. Т.~10. Вып.~3. С.~32--40.
\bibitem{10-kri}
\Au{Кривенко М.\,П., Голованов~С.\,А., Сивков~А.\,В.} Анализ однородности 
данных о химическом составе камней при уролитиазе~// Информатика и~её 
применения, 2013. Т.~7. Вып.~4. С.~94--104.
 \end{thebibliography}

 }
 }

\end{multicols}

\vspace*{-10pt}

\hfill{\small\textit{Поступила в~редакцию 5.12.16}}

\vspace*{4pt}

%\newpage

%\vspace*{-24pt}

\hrule

\vspace*{2pt}

\hrule

\vspace*{-3pt}


\def\tit{HIGH-DENSITY MULTIVARIATE REFERENCE REGION\\[-5pt]}

\def\titkol{High-density multivariate reference region}

\def\aut{M.\,P.~Krivenko\\[-7pt]}

\def\autkol{M.\,P.~Krivenko}

\titel{\tit}{\aut}{\autkol}{\titkol}

\vspace*{-16pt}


\noindent
Institute of Informatics Problems, Federal Research Center 
``Computer Science and Control'' of the Russian
Academy of Sciences,  44-2~Vavilov Str., Moscow 119333, Russian Federation



\def\leftfootline{\small{\textbf{\thepage}
\hfill INFORMATIKA I EE PRIMENENIYA~--- INFORMATICS AND
APPLICATIONS\ \ \ 2017\ \ \ volume~11\ \ \ issue\ 2}
}%
 \def\rightfootline{\small{INFORMATIKA I EE PRIMENENIYA~---
INFORMATICS AND APPLICATIONS\ \ \ 2017\ \ \ volume~11\ \ \ issue\ 2
\hfill \textbf{\thepage}}}

\vspace*{2pt}




\Abste{The paper considers the principles of construction of multivariate 
reference regions. An original method of construction of 
a~region on the basis of areas of high density of points and approximation 
of data distribution with a~mixture of normal distributions is suggested. 
To estimate the threshold for the probability density, the bootstrap method is used. 
As an experiment, the paper considers the problem of description and use of 
the reference region for predicting the type of urinary stones. 
Real data treatment demonstrated the benefits of the proposed solutions.}

\KWE{multivariate reference region; high-density region; bootstrap method; 
multivariate normal mixture}

\DOI{10.14357/19922264170207} 

%\vspace*{-18pt}

%\Ack
%\noindent



%\vspace*{3pt}

  \begin{multicols}{2}

\renewcommand{\bibname}{\protect\rmfamily References}
%\renewcommand{\bibname}{\large\protect\rm References}

{\small\frenchspacing
 {%\baselineskip=10.8pt
 \addcontentsline{toc}{section}{References}
 \begin{thebibliography}{99}
\bibitem{1-kri-1}
\Aue{Boyd, J.\,C.} 2004. Reference regions of two or more dimensions. \textit{Clin. 
Chem. Lab. Med.} 42(7):739--746.

\bibitem{2-kri-1}
\Aue{Winkel, P.} 1973. Patterns and clusters~--- multivariate approach for interpreting 
clinical chemistry results. \textit{Clin. Chem.} 19(12):1329--1333.
\bibitem{3-kri-1}
IFCC. 1987. Expert panel on theory of reference values. Approved recommendation on the 
theory of reference values. Part~5. Statistical treatment of collected reference values. 
Determination of reference limits. \textit{J.~Clin. Chem. Clin. Biochem.} 
25(9):645--656.
\bibitem{4-kri-1}
\Aue{Krivenko, M.\,P.} 2016. \textit{Statisticheskie metody predstavleniya 
i~predvaritel'noy obrabotki referensnykh znacheniy}
[Statistical methods for representation and preliminary processing of
reference values]. Moscow: FRC CSC RAS. 160~p.

\bibitem{5-kri-1}
\Aue{Boyd, J.\,C., and D.\,A.~Lacher.} 1982. The multivariate reference range: An 
alternative interpretation of multi-test profiles. \textit{Clin. Chem.}  
28(2):259--265.
\bibitem{6-kri-1}
\Aue{Albert, A., and E.\,K.~Harris.} 1987. \textit{Multivariate interpretation of 
clinical laboratory data}. New York, NY: CRC Press. 328~p.
\bibitem{7-kri-1}
\Aue{Linnet, K.} 1988. Influence of sampling variation and analytical errors on the 
performance of the multivariate reference region. \textit{Meth. Inf. Med.}  
27(1):37--42.
\bibitem{8-kri-1}
\Aue{Durbridge, T.\,C.} 1983. Clinical acceptance of a multi-test reference region 
for biochemical-panel results. \textit{Clin. Chem.} 29(10):1724--1726.
\bibitem{9-kri-1}
\Aue{Krivenko, M.\,P.} 2016. Kriterii znachimosti otbora priznakov klassifikatsii
[Significance tests of feature selection for~classification]. \textit{Informatika i~ee 
Primeneniya~--- Inform. Appl.} 10(3):32--40.
\bibitem{10-kri-1}
\Aue{Krivenko, M.\,P., S.\,A.~Golovanov, and A.\,V.~Sivkov}. 2013. Analiz 
odnorodnosti dannykh o~khimicheskom sostave kamney pri urolitiaze
[Analysis of data homogeneity of~the~chemical compositions 
of~stones in~case of~urolithiasis]. \textit{Informatika i~ee Primeneniya~---
Inform Appl.} 7(4):94--104.
\end{thebibliography}

 }
 }

\end{multicols}

\vspace*{-3pt}

\hfill{\small\textit{Received December 5, 2016}}


\Contrl

\noindent
\textbf{Krivenko Michail P.} (b.\ 1946)~--- Doctor of Science in technology, 
professor, leading scientist, Institute of Informatics Problems, Federal Research 
Center ``Computer Science and Control'' of the Russian Academy of Sciences, 
\mbox{44-2}~Vavilov Str., Moscow 119333, Russian Federation; \mbox{mkrivenko@ipiran.ru}

\label{end\stat}


\renewcommand{\bibname}{\protect\rm Литература}  %5
\def\stat{shestakov}

\def\tit{ОБРАЩЕНИЕ ОДНОРОДНЫХ ОПЕРАТОРОВ С~ПОМОЩЬЮ
СТАБИЛИЗИРОВАННОЙ ЖЕСТКОЙ ПОРОГОВОЙ ОБРАБОТКИ
ПРИ~НЕИЗВЕСТНОЙ ДИСПЕРСИИ ШУМА$^*$}

\def\titkol{Обращение однородных операторов с~помощью
стабилизированной жесткой пороговой обработки}
%при~неизвестной дисперсии шума}

\def\aut{О.\,В.~Шестаков$^1$}

\def\autkol{О.\,В.~Шестаков}

\titel{\tit}{\aut}{\autkol}{\titkol}

\index{Шестаков О.\,В.}
\index{Shestakov O.\,V.}


{\renewcommand{\thefootnote}{\fnsymbol{footnote}} \footnotetext[1]
{Работа выполнена при частичной финансовой поддержке РФФИ (проект 19-07-00352).}}


\renewcommand{\thefootnote}{\arabic{footnote}}
\footnotetext[1]{Московский государственный университет им.\ М.\,В.~Ломоносова, 
кафедра математической статистики факультета вычислительной математики и~кибернетики; 
Институт проб\-лем информатики Федерального исследовательского центра 
<<Информатика и~управ\-ле\-ние>> Российской академии наук, \mbox{oshestakov@cs.msu.su}}


\vspace*{-6pt}


\Abst{При обращении линейных однородных операторов обычно необходимо использовать 
методы регуляризации, поскольку наблюдаемые данные, как правило, зашумлены. 
Для подавления шума часто используется пороговая обработка 
вейвлет-ко\-эф\-фи\-ци\-ен\-тов функции наблюдаемого сигнала. 
Пороговая обработка стала популярным инструментом подавления 
шума благодаря своей простоте, вы\-чис\-ли\-тель\-ной эффективности и~воз\-мож\-ности 
адаптации к~функциям, имеющим на разных участках разную степень регулярности. 
Рассматривается предложенный недавно стабилизированный метод жесткой 
пороговой обработки, в~котором устранены основные недостатки мягкой и~жесткой 
пороговой обработки, и~исследуются статистические свойства этого метода. 
В~модели данных с~аддитивным гауссовским шумом с~неизвестной дисперсией 
проведен анализ несмещенной оценки среднеквадратичного риска и~показано, 
что при определенных условиях данная оценка является асимптотически нормальной, 
при этом дисперсия предельного распределения зависит от способа оценивания 
дисперсии шума.}

\KW{вейвлеты; пороговая обработка; несмещенная оценка риска; 
асимптотическая нормальность; сильная состоятельность}

\DOI{10.14357/19922264190107}
  
%\vspace*{4pt}


\vskip 10pt plus 9pt minus 6pt

\thispagestyle{headings}

\begin{multicols}{2}

\label{st\stat}

\section{Введение}

В медицинских, физических, астрономических и~других научных проблемах часто 
возникает задача получить представление об объекте, который описывается 
некоторой функцией~$f$, имея возможность наблюдать только функцию~$Kf$, где~$K$~--- 
некоторый линейный оператор. При этом часто нельзя просто применить 
к~наблюдаемым данным обратный оператор~$K^{-1}$, поскольку эти данные, как правило, 
содержат шум и~задача обращения оператора~$K$ некорректно поставлена. 
К~тому же обычно дис\-пер\-сия шума неизвестна и~ее необходимо оценивать 
по наблюдаемым данным. 

Одним из популярных инструментов при регуляризации 
процедуры обращения служит вейв\-лет-раз\-ло\-же\-ние с~последующей 
пороговой обработкой вейв\-лет-ко\-эф\-фи\-ци\-ен\-тов. Наиболее распростра\-нен\-ные 
виды пороговой обработки~--- жесткая и~мягкая. В~работе~\cite{HL10} 
был предложен метод стабилизированной жесткой пороговой обработки, который 
объединяет в~себе преимущества этих двух видов. 
В~ситуации, когда дисперсия шума предполагается известной, в~работе~\cite{SH18} 
доказана асимптотическая нормальность оценки среднеквадратичного риска пороговой 
обработки. 

В~данной работе исследуется влияние способов оценивания дисперсии шума 
на характеристики предельного распределения оценки среднеквадратичного риска. 
Для метода мягкой пороговой обработки подобные исследования проводились 
в~работах~\cite{KS11-1, KS11-2}.

\section{Обращение линейных однородных операторов с~помощью вейглет-вейвлет-разложения}

В данной работе рассматривается метод обращения линейных однородных операторов, 
основанный на вейг\-лет-вейв\-лет-раз\-ло\-же\-нии~\cite{AS98}. Линейный оператор~$K$ 
называется однородным, если
$$
K\left[f\left(a\left(x-x_0\right)\right)\right]=a^{-\alpha}(Kf)\left[a\left(x-x_0\right)\right]
$$
для любого $x_0$ и~любого $a\hm>0$. Параметр~$\alpha$ называется показателем 
однородности. Примерами линейных однородных операторов служат оператор 
интегрирования, преобразование Гильберта и~преобразование Абеля.

Относительно наблюдаемой функции~$Kf$ будем предполагать, что она определена на 
конечном отрезке и~равномерно регулярна по Липшицу с~некоторым показателем $\gamma\hm>0$. 
Вейв\-лет-разложение~$Kf$ представляет собой ряд по ортонормированному базису
\begin{equation}
\label{wavelet_decomp}
Kf = \sum\limits_{j,k \in Z} \langle Kf,\psi_{j,k} \rangle \psi_{j,k}\,,
\end{equation}
где $\psi(t)$~--- некоторая материнская вейв\-лет-функ\-ция, 
а~$\psi_{j,k}(t) \hm= 2^{j/2}\psi(2^jt \hm- k)$. Индекс~$j$ в~(\ref{wavelet_decomp}) 
называется масштабом, а~индекс~$k$~--- сдвигом. Если вейв\-лет-функ\-ция 
обладает определенными свойствами регулярности~\cite{Mal99}, 
то для коэффициентов разложения в~(\ref{wavelet_decomp}) справедливо
\begin{equation}
\label{wavelet_decay}
\abs{\langle Kf, \psi_{j,k} \rangle} \leqslant \fr{C_f}
{2^{j \left( \gamma + 1/2 \right)}}\,,
\end{equation}
где $C_f$~--- некоторая положительная константа.

Поскольку оператор~$K$ линеен и~однороден, существуют такие функции~$u_{j,k}$, 
что $\langle f,u_{j,k}\rangle\hm=\langle Kf,\psi_{j,k}\rangle$. При этом функция~$f$ 
представляется в~виде ряда
\begin{equation}
\label{VWD}
f = \sum\limits_{j,k \in Z}\beta_{j,k}\langle Kf,\psi_{j,k}\rangle u_{j,k},
\end{equation}
где $u_{j,k} = K^{-1}\psi_{j,k}/\beta_{j,k}$, $\beta_{j,k}\hm=2^{\alpha j}\beta_{00}$, 
$\beta_{00} \hm= \norm{K^{-1}\psi}$ (функции~$u_{j,k}$, как и~$\psi_{j,k}$, 
представляют собой сдвиги и~растяжения одной материнской функции~$u$ и~называются 
вейглетами). При соответствующем выборе~$\psi(t)$ последовательность~$\{u_{j,k}\}$ 
образует устойчивый базис~\cite{L97}. Формула~(\ref{VWD}) и~есть основа метода 
вейг\-лет-вейв\-лет-раз\-ло\-же\-ния.

\section*{Пороговая обработка эмпирических коэффициентов}

При фактических измерениях значения функции сигнала регистрируются 
в~дискретных отсчетах, при этом такие значения, как правило, зашумлены. 
Рассмотрим сле\-ду\-ющую модель данных \mbox{с~шумом}:
\begin{equation*}
%\label{Data_Model}
X_i = (Kf)_i + \epsilon_i\,, \enskip i = 1, \dots, 2^J\,, %\notag
\end{equation*}
где $2^J$~--- число отсчетов; $(Kf)_i$~--- незашумленные значения функции сигнала; 
$\epsilon_i$~--- независимые нормально распределенные случайные величины с~нулевым 
средним и~дисперсией~$\sigma^2$.
После применения дискретного вейв\-лет-пре\-об\-ра\-зо\-ва\-ния 
получается следующая модель зашумленных вейв\-лет-ко\-эф\-фи\-ци\-ен\-тов:
\begin{equation*}
Y_{j,k}=\mu_{j,k}+\epsilon^W_{j,k},\enskip 
j=0,\ldots,J-1,\ k=0,\ldots,2^{j}-1\,,
\end{equation*}
где $\epsilon^W_{j,k}$ независимы и~распределены так же, как и~$\epsilon_i$, 
а~$\mu_{j,k}\hm= 2^{J/2}\langle Kf,\psi_{j,k}\rangle$~\cite{Mal99}.

Для подавления шума и~построения оценки функции сигнала к~коэффициентам~$Y_{j,k}$ 
обычно применяется функция жесткой пороговой обработки 
$\rho_{H}(y,T)\hm=x\textbf{I}(\abs{y}>T)$ или мягкой пороговой 
обработки $\rho_{S}(y,T)\hm=\textbf{sgn}(x)\left(\abs{y}-T\right)_{+}$ 
с~порогом~$T$. При таком подходе обнуляются коэффициенты, абсолютная величина 
которых ниже порога, так как в~силу~(\ref{wavelet_decay}) основная часть
 полезного сигнала содержится в~относительно небольшом числе больших по 
 модулю коэффициентов.

Каждому из этих видов пороговой обработки присущи свои недостатки. 
Жесткая пороговая функция разрывна, и~это приводит к~отсутствию устойчивости 
при выборе порога~\cite{B96} и~невозможности построения несмещенной оценки 
среднеквадратичного риска~\cite{J01}. При мягкой пороговой обработке в~оценке 
функции появляется дополнительное смещение. Чтобы частично избежать этих недостатков, 
в~работе~\cite{HL10} был предложен новый вид пороговой обработки, представляющий 
собой сглаженный (стабилизированный) аналог жесткой пороговой обработки. 
В~этом методе оценки~$\mu_{j,k}$ вычисляются по формулам:
\begin{equation*}
\widehat{\mu}_{j,k}=\Expect 
\left[\rho_{H}(Y_{j,k}+\lambda\xi_{j,k},T_j)|Y_{j,k}\right], %\notag
\end{equation*}
где случайные величины~$\xi_{j,k}$ имеют стандартное нормальное распределение и~не 
зависят от~$Y_{j,k}$, а~$\lambda\hm>0$~--- 
параметр стабилизации, отвечающий за степень сглаживания. Вычисляя математическое 
ожидание, получаем:
\begin{multline*}
\hspace*{-8.37947pt}\widehat{\mu}_{j,k}=Y_{j,k}\left[\Phi\!\left(-\fr{T_j+Y_{j,k}}
{\lambda}\right)+1-\Phi\left(\fr{T_j-Y_{j,k}}{\lambda}\right)\!\right]+{}\\
{}+
\lambda\left[\phi\left(\fr{T_j-Y_{j,k}}{\lambda}\right)-
\phi\left(\fr{T_j+Y_{j,k}}{\lambda}\right)\right]. %\notag
\end{multline*}
Достоинством такого метода является бесконечная дифференцируемость~$\widehat{\mu}_{j,k}$ 
по~$Y_{j,k}$, что приводит к~более робастным оценкам~\cite{HL10}. Заметим также, 
что при $\lambda\hm\to0$ получается обычный метод жесткой пороговой обработки. 
В~данной работе параметр~$\lambda$ предполагается фиксированным, а~в~качестве~$T_j$ 
для каждого масштаба~$j$ выбирается порог $T_j\hm=\sigma\sqrt{2\ln 2^j}$. 
Такой порог получил название <<универсальный>>, так как он не зависит 
от наблюдаемых данных. И~при жесткой, и~при мягкой пороговой обработке этот 
порог обеспечивает близость среднеквадратичного риска к~минимальному~\cite{Mal99}.

\section{Несмещенная оценка среднеквадратичного риска}

Среднеквадратичный риск метода пороговой обработки определяется по формуле:
\begin{equation}
\label{Risk}
R_J(\sigma)=\sum\limits_{j=0}^{J-1}\sum\limits_{k=0}^{2^j-1}\beta^2_{j,k}
\Expect\left(\widehat{\mu}_{j,k}(\sigma)-\mu_{j,k}\right)^2.
\end{equation}
В~\cite{HL10} показано, что при стабилизированной жесткой пороговой обработке
\begin{multline*}
\Expect\left(\widehat{\mu}_{j,k}(\sigma)-\mu_{j,k}\right)^2={}\\
{}=
\Expect\left[(Y_{j,k}-\widehat{\mu}_{j,k}(\sigma))^2+
2\sigma^2\fr{\partial}{\partial Y_{j,k}}\,\widehat{\mu}_{j,k}(\sigma)\right]-
\sigma^2, %\notag
\end{multline*}
где
\begin{multline*}
\fr{\partial}{\partial Y_{j,k}}\widehat{\mu}_{j,k}(\sigma)={}\\
{}=\Phi\left(-\fr{T_j+Y_{j,k}}{\lambda}\right)+1-
\Phi\left(\fr{T_j-Y_{j,k}}{\lambda}\right)+{}\\
{}+
\fr{T_j}{\lambda}\left[\phi\left(\fr{T_j-Y_{j,k}}{\lambda}\right)+
\phi\left(\fr{T_j+Y_{j,k}}{\lambda}\right)\right]. %\notag
\end{multline*}
Таким образом, величина
\begin{multline}
\label{Risk_Estimate}
\widehat{R}_J(\sigma)=\sum\limits_{j=0}^{J-1}\sum\limits_{k=0}^{2^j-1}
\beta^2_{j,k}
\Bigg[
\left(
Y_{j,k}-
\widehat{\mu}_{j,k}(\sigma)\right)^2+{}\\
{}+2\sigma^2\fr{\partial}{\partial Y_{j,k}}\,\widehat{\mu}_{j,k}(\sigma)-
\sigma^2
\Bigg]
\end{multline}
является несмещенной оценкой~$R_J$, не зависящей от ненаблюдаемых значений~$\mu_{j,k}$.

В работе~\cite{SH18} доказано следующее утверждение, устанавливающее 
асимптотическую нормальность оценки~(\ref{Risk_Estimate}) и~позволяющее строить 
асимптотические доверительные интервалы для риска~(\ref{Risk}).

\smallskip

\noindent
\textbf{Теорема 1.} 
\textit{Пусть $K$~--- линейный однородный оператор с~показателем 
однородности $\alpha\hm>0$, а~$Kf$ задана на конечном отрезке и~равномерно 
регулярна по Липшицу с~показателем $\gamma\hm>0$. Тогда}
\begin{equation*}
%\label{Normality}
{\sf P}\left(\fr{\widehat{R}_J(\sigma)-
R_J(\sigma)}{D_J}<x\right)\Rightarrow\Phi(x)\,, %\notag
\end{equation*}
\textit{где}
$$
D^2_J=\fr{2\sigma^4\beta_{0,0}^4}{2^{4\alpha+1}-1}2^{(4\alpha+1)J}\,.
$$

\section{Виды оценок дисперсии шума}

Как правило, дисперсия~$\sigma^2$ неизвестна и~вместо ее точного значения 
необходимо использовать некоторую оценку~$\hat{\sigma}^2$, которая обычно 
строится по половине всех вейв\-лет-ко\-эф\-фи\-ци\-ен\-тов для $j\hm=J\hm-1$, 
так как в~силу~(\ref{wavelet_decay}) эти коэффициенты фактически содержат только шум. 
При этом порог вычисляется по формуле $\hat{T}_j\hm=\hat{\sigma}\sqrt{2\ln 2^j}$.

В качестве оценки~$\sigma^2$ (или $\sigma$) в~данной работе 
рассматривается выборочная дисперсия
\begin{equation}
\label{SampleVarianceDef}
\widehat{\sigma}_S^2=\fr{1}{2^{J-1}}
\sum\limits_{k=0}^{2^{J-1}-1}Y_{J-1,k}^2-\overline{Y}^2,
\end{equation}
где
\begin{equation*}
\overline{Y}=\fr{1}{2^{J-1}}\sum\limits_{k=0}^{2^{J-1}-1}Y_{J-1,k}\,,
\end{equation*}
а также соответствующим образом нормированный выборочный интерквартильный 
размах~$\widehat{\sigma}_{R}$ и~выборочное абсолютное медианное 
отклонение~$\widehat{\sigma}_{M}$, которые определяются сле\-ду\-ющим образом:
\begin{align}
\widehat{\sigma}_{R}&=\fr{Y_{(J-1,3/4)}-Y_{(J-1,1/4)}}{2\xi_{3/4}}\,;
\label{IQR_Definition}
\\
\widehat{\sigma}_{M}&=\fr{\mathop{\mbox{med}}\limits_{0\leqslant k\leqslant 2^{J-1}-1}|Y_{J-1,k}-\mathop{\mbox{med}}\limits_{0\leqslant l\leqslant 2^{J-1}-1} Y_{J-1,l}|}{\xi_{3/4}}\,.
\label{MAD_Definition}
\end{align}
Здесь $Y_{(J-1,1/4)}$ и~$Y_{(J-1,3/4)}$~--- выборочные квантили порядка~$1/4$ и~$3/4$, 
построенные по выборке из половины всех вейв\-лет-ко\-эф\-фи\-ци\-ен\-тов при 
$j\hm=J\hm-1$; $\xi_{3/4}$~--- теоретическая квантиль порядка~$3/4$ 
стандартного нормального распределения ($\xi_{3/4}\hm\approx0,6745$); $\mbox{med}$ 
обозначает выборочную медиану.

Выборочная дисперсия служит самой популярной оценкой величины~$\sigma^2$, и~в~случае 
отсутствия выбросов она наиболее предпочтительна. Однако в~случае, когда 
оценка дисперсии строится по выборке сигнала, естественно ожидать, 
что выборка не будет однородной. Преимущество использования последних 
двух оценок заключается в~их ро\-баст\-ности, т.\,е.\ нечувствительности к~выбросам.

\section{Предельная дисперсия оценки среднеквадратичного риска}

Способ оценивания дисперсии шума влияет на вид предельной дисперсии 
оценки среднеквадратичного риска. Подобный эффект наблюдается и~при 
мягкой пороговой обработке~[4].

\noindent
\textbf{Теорема~2.}\ \textit{Пусть $Kf$ задана на конечном отрезке и~равномерно 
регулярна по Липшицу с~показателем $\gamma\hm>1/4$, а оценка дисперсии 
шума задана соотношением}~\eqref{SampleVarianceDef}. \textit{Тогда}
\begin{equation}
\label{CLT_Operator_SampleVar_Sigma}
\mathsf{P}\left(\frac{\widehat{R}_J(\widehat{\sigma}_S)-R_J(\sigma)}{D_J}<x\right)
\Rightarrow \Phi_{\Upsilon_1}(x),\notag
\end{equation}
\textit{где $\Phi_{\Upsilon_1}(x)$~--- функция распределения нормального 
закона с~нулевым средним и~дисперсией}
$$
\Upsilon_1^2=\fr{1}{2^{4\alpha+1}}+
\fr{2^{4\alpha+1}-1}{2^{4\alpha+1}\left(2^{2\alpha+1}-1\right)^2}\,.
$$

\noindent
Д\,о\,к\,а\,з\,а\,т\,е\,л\,ь\,с\,т\,в\,о\,.\ \ Обозначим
\begin{multline*}
\widehat{U}_J(\sigma)=\sum\limits_{j=0}^{J-1}\sum\limits_{k=0}^{2^j-1}
\beta^2_{j,k}\Bigg[
\left(Y_{j,k}-\widehat{\mu}_{j,k}(\sigma)\right)^2+{}\\
{}+2\sigma^2\fr{\partial}{\partial Y_{j,k}}\widehat{\mu}_{j,k}(\sigma)\Bigg] %\notag
\end{multline*}
и запишем $\widehat{R}_J(\hat{\sigma}_S)-R_J(\sigma)$ в~виде
\begin{multline*}
%\label{Three_Sums}
\widehat{R}_J(\hat{\sigma}_S)-R_J(\sigma)={}\\
{}=\left[\widehat{U}_J(\hat{\sigma}_S)-\widehat{U}_J(\sigma)\right]+
\left[\widehat{R}_J(\sigma)-R_J(\sigma)\right]+{}\\
{}+
\fr{2^{(2\alpha+1)J}-1}{2^{2\alpha+1}-1}(\sigma^2-\hat{\sigma}^2_S)
\equiv S_1+S_2+S_3\,.
\end{multline*}

Повторяя рассуждения из работ~\cite{KS11-1, KS11-2} и~учитывая, что если $\gamma\hm>1/4$, 
то выполнено $2^{J/2}\overline{Y}^2\stackrel{{\sf P}}{\to} 0$ при 
$J\hm\rightarrow\infty$~\cite{KS11-2}, можно показать, что
\begin{equation*}
{\sf P}\left(\fr{S_2+S_3}{D_J}<x\right)\Rightarrow\Phi_{\Upsilon_1}(x)\,.%\notag
\end{equation*}
% на самом деле с~условием Линдеберга чуть по-другому (без ограниченности слагаемых). Но дисперсия равномерно ограничена -- значит выполнено.

Докажем, что $D_J^{-1}S_1\stackrel{{\sf P}}{\to}0$ при $J\hm\rightarrow\infty$. 
Пусть $C_\delta\hm>0$~--- некоторая константа, а $\delta_J\hm=C_\delta J^{1/2}2^{-J/2}$. 
Запишем
\begin{multline*}
S_1=\mathbf{1}\left(\abs{\sigma^2-\hat{\sigma}^2_S}>\delta_J\right)S_1+{}\\
{}+
\mathbf{1}\left(\abs{\sigma^2-\hat{\sigma}^2_S}\leqslant\delta_J\right)
S_1\equiv S'_1+S''_1. %\notag
\end{multline*}
Для произвольного $\varepsilon\hm>0$
\begin{equation*}
{\sf P}\left(S'_1>\varepsilon\right)\leqslant{\sf P}
\left(\abs{\sigma^2-\hat{\sigma}^2_S}>\delta_J\right). %\notag
\end{equation*}
При выполнении условий теоремы, если константа~$C_\delta$ достаточно велика, 
то найдется константа~$\tilde{C}_\delta>0$ такая, что~\cite{KS11-2}
\begin{equation*}
{\sf P}\left(\abs{\sigma^2-\hat{\sigma}^2_S}>\delta_J\right)
\leqslant\tilde{C}_\delta2^{-J/2}. %\notag
\end{equation*}
%% комментарии по поводу этого неравенства и~загрязнения выборки есть в~диссертации
Следовательно, $S'_1\stackrel{P}{\to}0$ при $J\hm\rightarrow\infty$.

Обозначим слагаемые в~сумме~$S''_1$ через~$F_{j,k}(\hat{\sigma}_S)$. Пусть 
$A_j\hm=\sqrt{A\ln 2^j}$, где $0\hm<A\hm<2(\sigma^2\hm-\delta_J)$. Имеем:

\noindent
\begin{multline*}
\hspace*{-9.9pt}\sum\limits_{j=0}^{J-1}\sum\limits_{k=0}^{2^j-1}F_{j,k}\left(\hat{\sigma}_S\right)=
\sum\limits_{j=0}^{J-1}\sum\limits_{k=0}^{2^j-1}
\mathbf{1}(\abs{Y_{j,k}}\leqslant A_j)F_{j,k}(\hat{\sigma}_S)+{}\\
{}+
\sum\limits_{j=0}^{J-1}\sum\limits_{k=0}^{2^j-1}
\mathbf{1}\left(\abs{Y_{j,k}}>A_j\right)F_{j,k}(\hat{\sigma}_S)
\equiv  W_1+W_2. %\notag
\end{multline*}
Рассмотрим $W_1$. Учитывая определения $\widehat{\mu}_{j,k}(\sigma)$, 
$({\partial}/{\partial Y_{j,k}})\widehat{\mu}_{j,k}(\sigma)$ и~$A_j$, 
можно убедиться, что найдут\-ся константы $C_1\hm>0$ и~$\theta\hm>0$ такие, что
\begin{equation*}
\abs{\mathbf{1}\left(\abs{Y_{j,k}}\leqslant A_J\right)
F_{j,k}(\hat{\sigma}_S)}\leqslant C_1 
J^{5/2}2^{(2\alpha-\theta)j-J/2}\;\;\mbox{п.в.} %\notag
\end{equation*}
% поскольку выполнено \mathbf{1}(\abs{\sigma^2-\hat{\sigma}^2_S}\leqslant\delta_J). В логарифме степень: от Y идет 1, от T идет 1, от \delta_J идет 1/2 но для J, а не для j, поэтому берем для всех J^{5/2}. В степени 2: 2\alpha от \beta{j,k}, \theta из-за выбора A, J/2 от \delta_J
Следовательно, $D_J^{-1}W_1\hm\rightarrow 0$ п.в.\ при $J\hm\rightarrow\infty$.

Далее для слагаемых~$W_2$ имеем:
\begin{multline*}
\left\vert \mathbf{1}\left(
\left\vert Y_{j,k}\right\vert
> A_J\right)F_{j,k}
\left(\hat{\sigma}_S\right)\right\vert
\leqslant{}\\
{}\leqslant C_2 J^{3/2}2^{2\alpha j-J/2} 
\mathbf{1}\left( \left\vert Y_{j,k}\right\vert > A_J\right) 
\left\vert Y_{j,k}\right\vert^2\;\;\mbox{п.в.},
%\notag
\end{multline*}
% поскольку выполнено \mathbf{1}(\abs{\sigma^2-\hat{\sigma}^2_S}\leqslant\delta_J). В логарифме от T идет 1, от \delta_J идет 1/2.
где $C_2>0$~--- некоторая константа. Учитывая распределение~$Y_{j,k}$, 
нетрудно убедиться, что
\begin{equation*}
\Expect\frac{1}{D_J} \sum\limits_{j=0}^{J-1}
\sum\limits_{k=0}^{2^j-1} J^{3/2}2^{2\alpha j-J/2} 
\mathbf{1}\left(\abs{Y_{j,k}}> A_j\right)
\abs{Y_{j,k}}^2\to 0
\end{equation*}
при $J\rightarrow\infty$. %\notag
Следовательно, используя неравенство Маркова, получаем, что
\begin{equation*}
D_J^{-1}W_2\stackrel{{\sf P}}{\to}0\;\;\mbox{при}\;J\rightarrow\infty\,. %\notag
\end{equation*}
Таким образом, $D_J^{-1}S_1\stackrel{{\sf P}}{\to}0$ при $J\hm\rightarrow\infty$.

Теорема доказана.

\smallskip

Рассмотрим теперь ситуацию, когда в~качестве оценки~$\sigma$ используется 
величина~$\widehat{\sigma}_{R}$ или~$\widehat{\sigma}_{M}$. 
В~этом случае повышаются требования к~гладкости функции сигнала.

\smallskip

\noindent
\textbf{Теорема~3.}\
\textit{Пусть~$Kf$ задана на конечном отрезке и~равномерно регулярна по 
Липшицу с~показателем $\gamma\hm>1/2$, а оценка дисперсии шума~$\hat{\sigma}$ 
задана соотношением}~\eqref{IQR_Definition} 
\textit{или соотношением}~\eqref{MAD_Definition}. \textit{Тогда}
\begin{equation*}
\label{CLT_Operator_RobVar_Sigma}
\mathsf{P}\left(\fr{\widehat{R}_J(\widehat{\sigma})-R_J(\sigma)}{D_J}<x\right)
\Rightarrow \Phi_{\Upsilon_2}(x)\,, %\notag
\end{equation*}
где $\Phi_{\Upsilon_2}(x)$~--- функция распределения нормального закона 
с~нулевым средним и~дисперсией
\begin{multline*}
\Upsilon_2^2=1+\fr{2^{4\alpha+1}-1}{4(2^{2\alpha+1}-1)^2
\xi_{3/4}^2(\phi(\xi_{3/4}))^2}-{}\\
{}-
\fr{2^{4\alpha+1}-1 }{2^{2\alpha-1}(2^{2\alpha+1}-1)}\,.
\end{multline*}

\noindent
Д\,о\,к\,а\,з\,а\,т\,е\,л\,ь\,с\,т\,в\,о\,.\ \
Как и~в~предыдущей теореме, запишем
$\widehat{R}_J(\hat{\sigma})\hm-R_J(\sigma)\hm=S_1\hm+S_2\hm+S_3.$
Учитывая,\linebreak\vspace*{-12pt}

\pagebreak

\noindent
 что $\gamma\hm>1/2$, и~поступая, как в~работах~\cite{SH18, KS11-2, SH12}, 
с~использованием разложения Бахадура для выборочных квантилей~\cite{S80} и~выборочного 
абсолютного медианного отклонения~\cite{SM09}, можно показать, что
\begin{equation*}
{\sf P}\left(\fr{S_2+S_3}{D_J}<x\right)\Rightarrow\Phi_{\Upsilon_2}(x)\,. %\notag
\end{equation*}
% на самом деле с~условием Линдеберга чуть по-другому (без ограниченности слагаемых). Но дисперсия равномерно ограничена -- значит выполнено.

Используя экспоненциальные неравенства для выборочных квантилей~\cite{S80} 
и~выборочного абсолютного медианного отклонения~\cite{SM09}, получаем, что при 
выполнении условий теоремы найдется такая константа $C_\delta\hm>0$, что при 
$\delta_J\hm=C_\delta J^{1/2}2^{-J/2}$ для некоторой константы~$\widetilde{C}_\delta>0$ 
выполнено:
\begin{align*}
\mathsf{P}\left(\abs{\widehat{\sigma}_{R}-\sigma}>\delta_J\right)
&\leqslant\widetilde{C}_\delta2^{-J/2}\,;
\\
\mathsf{P}\left(\abs{\widehat{\sigma}_{M}-\sigma}>\delta_J\right)
&\leqslant\widetilde{C}_\delta2^{-J/2}\,. %\notag
\end{align*}
%% комментарии по поводу этого неравенства и~загрязнения выборки есть в~диссертации
Далее, повторяя рассуждения предыдущей теоремы, заключаем, что 
$D_J^{-1}S_1\stackrel{{\sf P}}{\to}0$ при $J\hm\rightarrow\infty$.


Теорема доказана.



{\small\frenchspacing
 {%\baselineskip=10.8pt
 \addcontentsline{toc}{section}{References}
 \begin{thebibliography}{99}

\bibitem{HL10}
\Au{Huang H.-C., Lee~T.\,C.\,M.} 
Stabilized thresholding with generalized sure for image denoising~// 
IEEE 17th  Conference (International) on Image Processing
Proceedings.~--- IEEE, 2010. P.~1881--1884.

\bibitem{SH18}
\Au{Shestakov O.\,V.} 
Nonlinear regularization of inverse problems for linear homogeneous transforms 
by the stabilized hard thresholding~// J.~Math. Sci., 2018. Vol.~234. No.\,6. P.~780--785.

\bibitem{KS11-1}
\Au{Кудрявцев А.\,А., Шестаков~О.\,В.} 
Асимптотика оценки риска при вейг\-лет-вейв\-лет разложении наблюдаемого сигнала~// 
T-Comm~--- телекоммуникации и~транспорт, 2011. №\,2. С.~54--57.

\bibitem{KS11-2}
\Au{Кудрявцев А.\,А., Шестаков~О.\,В.} 
Асимптотическое распределение оценки риска пороговой обработки 
вейг\-лет-ко\-эф\-фи\-ци\-ен\-тов сигнала при неизвестном уровне шума~// 
T-Comm~--- телекоммуникации и~транспорт, 2011. №\,5. С.~24--30.

\bibitem{AS98}
\Au{Abramovich F., Silverman~B.\,W.} 
Wavelet decomposition approaches to statistical inverse problems~// 
Biometrika, 1998. Vol.~85. No.\,1. P. 115--129.

\bibitem{Mal99}
\Au{Mallat S.} A~Wavelet tour of signal processing.~--- 
New York, NY, USA: Academic Press, 1999. 857~p.

\bibitem{L97}
\Au{Lee N.} Wavelet-vaguelette decompositions and homogenous equations.~--- 
West Lafayette, IN, USA: Purdue University, 1997.  PhD Thesis. 103~p.

\bibitem{B96}
\Au{Breiman L.} Heuristics of instability and stabilization in model selection~// 
Ann. Stat., 1996. Vol.~24. No.\,6. P.~2350--2383.

\bibitem{J01}
\Au{Jansen M.} Noise reduction by wavelet thresholding.~--- 
Lecture notes in statistics ser.~--- New York, NY, USA: Springer Verlag,
2001. Vol.~161. 196~p.

\bibitem{SH12}
\Au{Шестаков О.\,В.} О~скорости сходимости оценки риска пороговой обработки 
вейв\-лет-ко\-эф\-фи\-ци\-ен\-тов к~нормальному закону при использовании 
робастных оценок дисперсии~// Информатика и~её применения, 2012. Т.~6. Вып.~2. 
С.~122--128.

\bibitem{S80}
\Au{Serfling R.} Approximation theorems of mathematical statistics.~--- 
New York, NY, USA: John Wiley \& Sons, 1980. 371~p.

\bibitem{SM09}
\Au{Serfling R., Mazumder~S.} 
Exponential probability inequality and convergence results for the median 
absolute deviation and its modifications~// Stat. Probabil. Lett., 2009. 
Vol.~79. No.\,16. P.~1767--1773.
 \end{thebibliography}

 }
 }

\end{multicols}

\vspace*{-3pt}

\hfill{\small\textit{Поступила в~редакцию 14.12.18}}

\vspace*{8pt}

%\pagebreak

%\newpage

%\vspace*{-28pt}

\hrule

\vspace*{2pt}

\hrule

%\vspace*{-2pt}

\def\tit{INVERSION OF~HOMOGENEOUS OPERATORS USING~STABILIZED HARD THRESHOLDING 
WITH~UNKNOWN NOISE VARIANCE}

\def\titkol{Inversion of~homogeneous operators using~stabilized hard thresholding 
with~unknown noise variance}

\def\aut{O.\,V.~Shestakov}

\def\autkol{O.\,V.~Shestakov}

\titel{\tit}{\aut}{\autkol}{\titkol}

\vspace*{-11pt}


\noindent
Department of Mathematical Statistics, Faculty of Computational Mathematics and Cybernetics, M.V. Lomonosov Moscow State University, 1-52 Leninskiye Gory, GSP-1, Moscow 119991, Russian Federation
Institute of Informatics Problems, Federal Research Center 
``Computer Science and Control'' of the Russian Academy of Sciences, 44-2~Vavilov Str., 
Moscow 119333, Russian Federation

\def\leftfootline{\small{\textbf{\thepage}
\hfill INFORMATIKA I EE PRIMENENIYA~--- INFORMATICS AND
APPLICATIONS\ \ \ 2019\ \ \ volume~13\ \ \ issue\ 1}
}%
 \def\rightfootline{\small{INFORMATIKA I EE PRIMENENIYA~---
INFORMATICS AND APPLICATIONS\ \ \ 2019\ \ \ volume~13\ \ \ issue\ 1
\hfill \textbf{\thepage}}}

\vspace*{6pt}



\Abste{When inverting linear homogeneous operators, it is necessary to use 
regularization methods, since observed data are usually noisy. For noise suppression, 
threshold processing of  wavelet coefficients of the observed signal function 
is often used. Threshold processing has become a~popular noise suppression tool 
due to its simplicity, computational efficiency, and ability to adapt to functions 
that have different degrees of regularity at different domains. The paper 
discusses the recently proposed stabilized hard thresholding method that eliminates 
the main
drawbacks of soft and hard thresholding methods and studies statistical 
properties of this method. In the data model\linebreak\vspace*{-12pt}}

\Abstend{with an additive Gaussian noise with 
unknown variance, an unbiased estimate of the mean square risk is analyzed and it 
is shown that under certain conditions, this estimate is asymptotically normal and 
the variance of the limit distribution depends on the type of estimate of noise variance.}


\KWE{wavelets; threshold processing; unbiased risk estimate; asymptotic normality;
strong consistency}




\DOI{10.14357/19922264190107}

%\vspace*{-14pt}

\Ack
\noindent
This research was partly supported by the Russian  
Foundation for Basic Research (project No.\,19-07-00352).




%\vspace*{6pt}

  \begin{multicols}{2}

\renewcommand{\bibname}{\protect\rmfamily References}
%\renewcommand{\bibname}{\large\protect\rm References}

{\small\frenchspacing
 {%\baselineskip=10.8pt
 \addcontentsline{toc}{section}{References}
 \begin{thebibliography}{99}
\bibitem{1-sh-1}
\Aue{Huang, H.-C., and T.\,C.\,M.~Lee.} 2010. 
Stabilized thresholding with generalized sure for image denoising. 
\textit{IEEE 17th Conference (International) on Image Processing}. IEEE. 1881--1884.

 

\bibitem{2-sh-1}
\Aue{Shestakov, O.\,V.} 2018. 
Nonlinear regularization of inverse problems for linear homogeneous transforms 
by the stabilized hard thresholding. 
\textit{J.~Math. Sci.} 234(6):780--785.

\bibitem{3-sh-1}
\Aue{Kudryavtsev, A.\,A., and O.\,V.~Shestakov.} 2011. Аsimptotika otsenki riska pri 
veyglet-veyvlet razlozhenii nablyuda\-emo\-go signala [The average risk assessment 
of the wavelet decomposition of the signal].
\textit{T-Comm~--- Telecommunications and Their Application in
Transport Industry} 2:54--57.

\bibitem{4-sh-1}
\Aue{Kudryavtsev, A.\,A., and O.\,V.~Shestakov.} 2011. Аsimptoticheskoe raspredelenie 
otsenki riska porogovoy ob\-ra\-bot\-ki veyglet-koeffitsientov signala pri 
neizvestnom urovne shuma [Asymptotic distribution of the risk estimate of 
the signal vaguelette coefficients thresholding at the unknown noise level]. 
\textit{T-Comm~--- Telecommunications and Their Application in
Transport Industry} 5:24--30.

\bibitem{5-sh-1}
\Aue{Abramovich, F., and B.\,W.~Silverman.} 1998. Wavelet 
decomposition approaches to statistical inverse problems. 
\textit{Biometrika} 85(1):115--129.

\bibitem{6-sh-1}
\Aue{Mallat, S.} 1999. \textit{A~wavelet tour of signal processing.} New York, NY: 
Academic Press. 857 p.

\bibitem{7-sh-1}
\Aue{Lee, N.} 1997. Wavelet-vaguelette decompositions and homogenous equations. 
 West Lafayette, IN: Purdue University. PhD Thesis. 103~p.

\bibitem{8-sh-1}
\Aue{Breiman, L.} 1996. 
Heuristics of instability and stabilization in model selection. 
\textit{Ann. Stat.} 24(6):2350--2383.

\bibitem{9-sh-1}
\Aue{Jansen, M.} 2001. \textit{Noise reduction by wavelet thresholding.} 
Lecture notes in statistics ser.
New York, NY: Springer Verlag.  Vol.~161. 196~p.

\bibitem{10-sh-1}
\Aue{Shestakov, O.\,V.} 2012. O~skorosti skhodimosti otsenki riska porogovoy 
obrabotki veyvlet-koeffitsientov k~nor\-mal'\-no\-mu zakonu pri ispol'zovanii robastnykh 
otsenok dispersii [On the rate of convergence to the normal law of risk estimate for 
wavelet coefficients thresholding when using robust variance estimates]. 
\textit{Informatika i~ee Primeneniya~--- Inform. Appl.}  6(2):122--128.

\bibitem{11-sh-1}
\Aue{Serfling, R.} 1980. \textit{Approximation theorems of mathematical statistics}.
New York, NY: John Wiley \& Sons. 371~p.

\bibitem{12-sh-1}
\Aue{Serfling, R., and S.~Mazumder.} 2009. Exponential probability inequality 
and convergence results for the median absolute deviation and its modifications. 
\textit{Stat. Probabil. Lett.} 79(16):1767--1773.
\end{thebibliography}

 }
 }

\end{multicols}

\vspace*{-6pt}

\hfill{\small\textit{Received December 14, 2018}}

%\pagebreak

%\vspace*{-18pt}  

\Contrl

\noindent
\textbf{Shestakov Oleg V.} (b.\ 1976)~--- 
Doctor of Science in physics and mathematics, professor, Department of 
Mathematical Statistics, Faculty of Computational Mathematics and Cybernetics, 
M.\,V.~Lomonosov Moscow State University, 1-52~Leninskiye Gory, GSP-1, Moscow 119991, 
Russian Federation; senior scientist, Institute of Informatics Problems, 
Federal Research Center ``Computer Science and Control'' 
of the Russian Academy of Sciences, 44-2~Vavilov Str., Moscow 119333, 
Russian Federation; \mbox{oshestakov@cs.msu.su}
\label{end\stat}

\renewcommand{\bibname}{\protect\rm Литература} 
     %6
\def\stat{zah+shestakov}

\def\tit{АНАЛИЗ ТОЧНОСТИ ВЕЙВЛЕТ-ОБРАБОТКИ АЭРОДИНАМИЧЕСКИХ КАРТИН ОБТЕКАНИЯ}

\def\titkol{Анализ точности вейвлет-обработки аэродинамических картин обтекания}

\def\aut{Т.\,В.~Захарова$^1$, О.\,В.~Шестаков$^2$}

\def\autkol{Т.\,В.~Захарова, О.\,В.~Шестаков}

\titel{\tit}{\aut}{\autkol}{\titkol}

\index{Шестаков О.\,В.}
\index{Shestakov O.\,V.}
\index{Захарова Т.\,В.}
\index{Zakharova T.\,V.}


%{\renewcommand{\thefootnote}{\fnsymbol{footnote}} \footnotetext[1]
%{Работа выполнена при частичной финансовой поддержке РФФИ (проект 15-07-02652).}}


\renewcommand{\thefootnote}{\arabic{footnote}}
\footnotetext[1]{Московский государственный университет им.\ М.\,В.~Ломоносова,
факультет вычислительной математики и~кибернетики; Институт проб\-лем
информатики Федерального исследовательского центра <<Информатика и~управление>> 
Российской академии наук, \mbox{lsa@cs.msu.ru}} 
\footnotetext[2]{Московский государственный университет им.\ М.\,В.~Ломоносова, 
кафедра математической статистики факультета вычислительной математики и~кибернетики; 
Институт проблем информатики Федерального исследовательского центра 
<<Информатика и~управление>> Российской академии наук, \mbox{oshestakov@cs.msu.su}}



\Abst{Предлагается новый метод обработки зашумленных аэродинамических картин 
обтекания, основанный на технике вейвлет-ана\-ли\-за. Вейв\-лет-ме\-то\-ды 
подавления шума, основанные на процедуре пороговой обработки, широко используются 
при анализе сигналов и~изображений. Их привлекательность заключается, во-пер\-вых, 
в~быстроте алгоритмов построения оценок, а~во-вто\-рых, в~возможности лучшей, чем 
линейные методы, адаптации к~функциям, имеющим на разных участках различную степень 
регулярности. Анализ погрешностей этих методов представляет собой важную практическую 
задачу, поскольку он позволяет оценить качество как самих методов, так и~используемого 
оборудования.
Проведена верификация метода на основе сравнительного анализа с~результатами 
обработки ранее разработанным дискриминантным методом. Рассчитанная оценка 
погрешности обработки при этом согласуется с~теоретическими результатами для 
этой оценки.}

    \KW{вейвлет-анализ; пороговая обработка; несмещенная оценка риска; 
    аэродинамический поток}


\DOI{10.14357/19922264160307} 
  
  %\vspace*{6pt}


\vskip 12pt plus 9pt minus 6pt

\thispagestyle{headings}

\begin{multicols}{2}

\label{st\stat}


\section{Введение}

Обработка результатов экспериментов в~аэродинамической трубе требует
исключительной точ\-ности. Результаты таких испытаний влияют на
кинематические возможности проектируемой техники, экономичность 
и~безопасность ее использования. Повысить точность можно путем
услож\-не\-ния экспериментальной установки либо\linebreak использованием
специализированной обработки результатов эксперимента. Первый путь,
как правило, весьма дорог и~сложен: для аэродинамических
экспериментов это означает улучшение системы фильтрации воздуха 
и~поддержания постоянства его температуры, влажности и~т.\,д.,
усложнение оптических систем для снятия характеристик обтекания.
Второй требует разработки и~верификации методики оценки истинных
значений по имеющимся данным~--- зачастую это сделать проще 
и~дешевле; таким образом, разработка специальных алгоритмов обработки
данных весьма актуальна в~данной области.

В рамках данной работы непосредственные физические характеристики
газовой струи заменены условным эквивалентом --- цветом на теневой
картине обтекания. Задача состоит в~очистке данного сигнала от шума,
возникающего вследствие наличия в~газовой струе и~на оптике
аппаратуры фото/видеосъемки пыли и~т.\,п.~\cite{holder, krasnov}. Для
решения данной задачи в~работе используются методы вейв\-лет-анализа,
которые прекрасно зарекомендовали себя при анализе и~обработке
нестационарных сигналов и~изображений~\cite{smolentsev, posobie}.


\section{Вейвлет-методы обработки изображений}

\subsection{Пороговая обработка вейвлет-коэффициентов изображения}

Большинство алгоритмов подавления шума, использующие методы вейв\-лет-ана\-ли\-за, 
действуют по принципу 
<<вейв\-лет-пре\-об\-ра\-зо\-ва\-ние\,--\,обработка вейв\-лет-ко\-эф\-фи\-ци\-ен\-тов\,--\,об\-рат\-ное 
вейв\-лет-пре\-об\-ра\-зо\-ва\-ние>>. В~данной работе рассматривается один из 
способов обработки вейв\-лет-ко\-эф\-фи\-ци\-ен\-тов~--- 
так называемая мягкая пороговая обработка.

Пусть $N=2^J$ для некоторого $J\hm>0$; $f(i,j)$, $0\hm\leqslant i,j \hm< N,$~--- 
матрица изображения; $\{W(i,j)\}_{0\leqslant i,j < N}$~--- 
помехи (шум), являющиеся реализацией некоторого
случайного процесса. В~данной работе предполагается, 
что $\{W(i,j)\}_{0\leqslant i,j < N}$~--- независимые
одинаково распределенные нормальные случайные величины с~нулевым 
средним и~дисперсией~$\sigma^2$. Искаженный
сигнал определяется формулой:
\begin{equation*}
    X(i,j)=f(i,j)+W(i,j)\,, \enskip 0 \leqslant i,j < N\,.
\end{equation*}
После применения дискретного вейв\-лет-пре\-об\-ра\-зо\-ва\-ния 
$Y\hm=W(X)$~\cite{posobie} 
получается следующая модель эмпирических вейв\-лет-ко\-эф\-фи\-ци\-ентов:
\begin{multline*}
    Y^{[\lambda]}(j,k,l)=f^{[\lambda]}_W(j,k,l)+
    Z^{[\lambda]}(j,k,l)\,, \\ 
    \lambda=1,2,3\,,\enskip 0\leqslant j<J\,,\enskip 0\leqslant k,l< 2^j\,,
\end{multline*}
где $f_W(j,k,l)$~--- вейвлет-ко\-эф\-фи\-ци\-ен\-ты <<чис\-то\-го>> изображения, 
а~шумовые коэффициенты $Z^{[\lambda]}(j,k,l)$ в~силу ортогональности 
вейв\-лет-пре\-об\-ра\-зо\-ва\-ния имеют такую же статистическую структуру, 
как $W(i,j)$. Индекс~$j$ называется масштабом и~отвечает за <<размер>> 
вейв\-лет-ко\-эф\-фи\-ци\-ен\-та (размер участка изображения, за который 
отвечает данный коэффициент), индексы $k$ и~$l$~--- сдвиги, отвечающие за 
<<положение>> вейв\-лет-ко\-эф\-фи\-ци\-ен\-та (расположение участка изображения, 
за который отвечает данный коэффициент), а~индекс $\lambda$ описывает тип 
вейв\-лет-ко\-эф\-фи\-ци\-ен\-та: при $\lambda\hm=1$ 
вейв\-лет-ко\-эф\-фи\-ци\-ен\-ты называются вертикальными, при $\lambda\hm=2$~--- 
горизонтальными, а~при $\lambda\hm=3$~--- диагональными~\cite{smolentsev, bogges, malla}.


Оценки вейвлет-коэффициентов изображения $\hat{f}_W(j,k,l)$
вычисляются покомпонентно, т.\,е.
$$
\hat{f}^{[\lambda]}_W(j,k,l)=\rho_T(Y^{[\lambda]}(j,k,l))\,,
$$
где $\rho_T(x)$~--- функция мягкой пороговой обработки с~порогом $T\hm>0$, 
задаваемая формулой:
\begin{equation*}
    \label{soft_function}
        \rho_T(x)=
            \begin{cases}
                x-T, &\mbox{ если }x > T\,;\\
                x+T, &\mbox{ если }x < -T\,;\\
                0, &\mbox{ если } \left|x\right|  \leqslant T\,.
            \end{cases}
\end{equation*}
Таким образом, в~результате пороговой обработки обнуляются те 
вейв\-лет-ко\-эф\-фи\-ци\-ен\-ты, абсолютная величина которых не превосходит порога, 
а~абсолютная величина остальных вейв\-лет-ко\-эф\-фи\-ци\-ен\-тов 
уменьшается на величину порога.

Риск (среднеквадратичная погрешность) мягкой пороговой обработки 
определяется формулой:
\begin{equation}
    \label{risk_def}
        r_f= \sum\limits_{\lambda=1}^{3}\sum\limits_{j=0}^{J-1}
        \sum\limits_{i,j=0}^{2^j-1}{\mathbb{E}\left|\hat{f}^{[\lambda]}_W(j,k,l)
        -f^{[\lambda]}_W(j,k,l)\right|^2}\,.
\end{equation}
Выражение (\ref{risk_def}) содержит неизвестные величины 
<<чистых>> вейв\-лет-ко\-эф\-фи\-ци\-ен\-тов $f^{[\lambda]}_W(j,k,l)$, 
поэтому на практике вычислить его нельзя. Однако его можно оценить, используя 
только известные эмпирические вейв\-лет-ко\-эф\-фи\-ци\-ен\-ты~$Y^{[\lambda]}(j,k,l)$. 
В~каждом слагаемом если $\left|Y^{[\lambda]}(j,k,l)\right|\hm>T$, 
то вклад этого слагаемого в~риск со\-став\-ля\-ет $\sigma^2\hm+T^2$, 
а~если $\left|Y^{[\lambda]}(j,k,l)\right|\hm\leqslant T$, то вклад 
со\-став\-ля\-ет $[f^{[\lambda]}_W(j,k,l)]^2$. Поскольку $\mathbb{E}[Y^{[\lambda]}(j,k,l)]^2
\hm=\sigma^2+[f^{[\lambda]}_W(j,k,l)]^2$, величину $[f^{[\lambda]}_W(j,k,l)]^2$ 
можно оценить разностью $[Y^{[\lambda]}(j,k,l)]^2\hm-\sigma^2$.

Таким образом, в~качестве оценки риска можно использовать следующую величину:
\begin{equation}
\label{RiskEstimateDefinition}
\hat{r}_f=\sum\limits_{\lambda=1}^{3}\sum\limits_{j=0}^{J-1}
\sum\limits_{i,j=0}^{2^j-1}F[[Y^{[\lambda]}(j,k,l)]^2]\,,
\end{equation}
где $F[x]=(x-\sigma^2)\Ik(|x|\hm\leqslant
T^2)\hm+(\sigma^2\hm+T^2)\Ik(|x|\hm>T^2).$ Донохо и~Джонстон
показали~\cite{DonJ2, DonJ}, что при мягкой пороговой обработке для
оценки риска~(\ref{RiskEstimateDefinition}) справедливо следующее
утверждение. 

\smallskip

\noindent
\textbf{Лемма~1}. \textit{ %{\ulemma{\label{Unbiased_Estimte}
$\mathbb{E}\hat{r}_f\hm=r_f$, т.\,е. $\hat{r}_f$ является несмещенной
оценкой для~$r_f$.}

\smallskip


Исследуем теперь вопрос выбора порога.

\subsection{Предпосылки выбора порога вейвлет-обработки}

\noindent
\textbf{Лемма~2.}
\textit{Пусть $z_1, z_2, \ldots, z_n$~--- независимые случайные величины, 
имеющие нормальное распределение с~нулевым средними и~дисперсией~$\sigma^2$;
\begin{align*}
    A_n &= \left\{\max\limits_{1 \leqslant i \leqslant n}{\left|z_i\right|}>
    \sigma\sqrt{2\ln{n}}\right\}\,;\\
    B_n(t)&=\left\{\max\limits_{1 \leqslant i \leqslant n}{\left| z_i \right|} 
    > \sigma t+\sigma\sqrt{2\ln{n}}\right\}.
\end{align*}
Тогда}
\begin{equation*}
    \label{pi_n}
        P\left(A_n\right)\to 0 \mbox{ при }n \to \infty\,;\enskip 
P\left(B_n(t)\right) < e^{-{t^2}/{2}}\,.
\end{equation*}


Порог $T=\sigma\sqrt{2\ln{n}}$ согласно лемме~2 с~большой вероятностью 
удаляет основной шум,
а~оставшийся шум будет незначительным, так как вероятность~$B_n$ экспоненциально 
убывает.

На практике часто приходится оценивать значение~$\sigma$. Обычно в~качестве 
такой оценки используется величина
\begin{equation}
    \label{sigma_hat}
        \hat{\sigma}=\fr{1}{C_{3/4}}\,\mathrm{MAD}\,,
\end{equation}
где MAD (Median Absolute Deviation)~--- выборочное медианное абсолютное 
отклонение, построенное по вейв\-лет-ко\-эф\-фи\-ци\-ен\-там при $j\hm=J\hm-1$ 
(считается, что на этом масштабе коэффициенты содержат только шум~\cite{malla}), 
а~$C_{3/4}$~--- $3/4$-кван\-тиль стандартного нормального распределения.

В силу приведенных рассуждений в~данной работе используется порог

\begin{equation*}
    T_U=\hat{\sigma}\sqrt{2 \ln{n}}, \quad n=N^2,
\end{equation*}
который называется <<универсальным>>.


В работах \cite{MA09, MSH10} доказаны следующие утверждения, 
позволяющие статистически оценить погрешность метода пороговой обработки, 
используя только известные эмпирические вейв\-лет-ко\-эф\-фи\-ци\-енты.

\smallskip

\noindent
\textbf{Теорема~1.}
\textit{Пусть функция, описывающая изображение, является ку\-соч\-но-ре\-гу\-ляр\-ной по 
Липшицу с~показателем $\gamma\hm>0$. Тогда при использовании мягкой пороговой 
обработки с~порогом $T_U$}
\begin{equation}\label{Universal_Consistency}
\fr{\hat{r}_f-r_f}{N^2}\rightarrow 0\enskip \mbox{при}\enskip N\rightarrow\infty\,.
\end{equation}


\noindent
\textbf{Теорема~2.}
\textit{Пусть функция, описывающая изображение, является ку\-соч\-но-ре\-гу\-ляр\-ной 
по Липшицу с~показателем $\gamma\hm>1/2$. Тогда при использовании мягкой пороговой 
обработки с~порогом~$T_U$}
\begin{equation}
\label{Universal_Normality_Sigma_Robust}
\mathsf{P}\left(\fr{\hat{r}_f-r_f}{\sqrt{2}\hat{\sigma}^2N}<x\right)\Rightarrow
\Phi_{\Upsilon}(x)\ \mbox{ при }\ N\rightarrow\infty\,.
\end{equation}
\textit{Здесь $\Phi_{\Upsilon}(x)$~--- 
функция распределения нормального закона с~нулевым средним и~дисперсией 
$\Upsilon^2\hm=[2C_{3/4}\phi(C_{3/4})]^{-2}\hm-1$, 
а~$\phi(x)$~--- плот\-ность стандартного нормального распределения.}

\smallskip


Теорема~1 говорит о том, что $\hat{r}_f$ является состоятельной оценкой, 
а~теорема~2 дает возможность строить асимптотические доверительные интервалы 
для величины~$r_f$ и~оценивать отклонение~$\hat{r}_f$ от~$r_f$.


Рассмотрим применение этих результатов на примере обработки аэродинамических 
изображений.

\section{Обработка цветных теневых картин аэродинамического эксперимента}

\subsection{Постановка задачи}

В аэродинамической трубе установлено тело (использовались данные,
полученные с~цилиндра, ось которого была сначала установлена
перпендикулярно набегающему потоку и~параллельно оптической оси
регистрирующей аппаратуры, а~затем~--- перпендикулярно набегающему
потоку и~оптической оси регистрирующей аппаратуры). 

Исследуется
структура обтекающей его газовой струи. При помощи теневого прибора
производится снятие характеристик обтекания, при этом на выходе
получается так называемая цветная теневая картина обтекания, т.\,е.\
изображение, на котором соответствующими цветами фиксируется
плотность потока~\cite{holder,krasnov,Eng_dis_m, Rus_dis_m}.

Дешифровка цветной теневой картины в~значительной мере осложняется
влиянием зашумления. Оно обусловливается многими факторами, например
наличием частиц пыли в~набегающем потоке и~на оптических элементах
фотоаппаратуры, при помощи которой ведется съемка теневой картины,
т.\,е.\ как собственно возникающими в~исследуемой газовой струе
(внутренними), так и~вносимыми измерительной установкой (внешними).
Естественным в~этом случае можно считать предположение о~том, что
их влияние на результат измерения описывается совокупностью
независимых одинаково распределенных  нормальных случайных величин.
Вообще говоря, следует заметить, что цветорегистрирующая аппаратура
может зашумлять цвет из разных частей спектра по-разному, в~этом
случае нужно обладать соответствующими данными о зависимости
точности работы установки от входного сигнала. При отсутствии
указанных сведений будем считать, что систематической ошибки
измерительная аппаратура не создает, а~шум, вносимый ею в~данные, от
самого сигнала не зависит.

В изначальной постановке необходимо создать методику оценки
плотности по цветной теневой картине потока в~аэродинамической
трубе. Алгоритм должен быть устойчив как к~внутренним, так 
и~к~внешним шумам и~с достаточной точностью оценивать численные
характеристики обтекания. 
{\looseness=1

}

В~такой постановке задача несколько лет
назад была успешно решена при помощи так называемого дискриминантного метода~[11--14], 
он будет кратко описан ниже.

В данной работе исследуется вопрос о~возможности реализовать
процедуру обработки цветной теневой картины методами
вейв\-лет-ана\-ли\-за. При этом, ввиду отсутствия прямого (без
использования результатов работы дискриминантного метода) алгоритма
дешифровки цветной теневой картины, исследуется вопрос
восстановления цветности изоб\-ра\-же\-ния, а~не собственно плотности
потока. 

Особое внимание уделено точности обработки об\-ласти перед
телом, установленным в~аэродинамической трубе, так как дискриминантный
метод оптимизирован для анализа этой об\-ласти.



\subsection{Решение задачи при~помощи дискриминантного метода}

Входная цветная теневая картина представляет собой графический файл
в формате bmp. Размер изображения $512 \times 512$ пикселей. Каждый
пиксель изображения~--- элемент цветового пространства $(R, G, B)$.
Отобразим его на единичную сферу, т.\,е.\ осуществим следующее
преобразование:
{ %\small
\begin{multline}
(r,g,b)\to{}\\
\!\to\!\left(\!\fr{r}{\sqrt{r^2+g^2+b^2}},\!\fr{g}{\sqrt{r^2+g^2+b^2}},\!
\fr{b}{\sqrt{r^2+g^2+b^2}}\!\right)\!\!=\\
\label{rgbtocos}
=\left(\cos{\alpha},\cos{\beta},\cos{\gamma}\right)\,.
\end{multline}
}

Всюду далее будем называть цветовым пространством единичную сферу, 
на которой расположены векторы
$\left(\cos{\alpha},\cos{\beta},\cos{\gamma}\right)$~--- элементы 
пространства $(R, G, B)$ после преобразования~(\ref{rgbtocos}).

Все цветовое пространство разобъем на так называемые эталонные классы:
произведем разби\-ение полосы спектра, зафиксированной измерительным
прибором, на прямоугольные сегменты,\linebreak для каждого из них в~терминах
косинусов (из приведенного выше преобразования) вычислим дисперсию
$\{{\sf D}_{i1},{\sf D}_{i2},{\sf D}_{i3}\}$ и~выборочное среднее
$\{z_{i1},z_{i2},z_{i3}\}$ (так называемый центр) класса с~номером~$i$.
Фактически эти характеристики параметризуют зависимость погрешности
определения прибором цвета в~разных частях спектра. Каждый класс
эквивалентности характеризуется также чис\-лом~$X_{i}$~--- индексом
цветности.

Введем дискриминантную функцию
\begin{multline*}
    d_{i}(v)={}\\
    {}=\!\sqrt{a_{i1}\left(v_1-z_{i1}\right)^2+a_{i2}\left(v_2-z_{i2}\right)^2+
    a_{i3}\left(v_3-z_{i3}\right)^2},\hspace*{-7.9pt}
\end{multline*}
где
$v\hm=(v_1,v_2,v_3)$~--- произвольный вектор цветового пространства;
$$
a_{ik}=\fr{\min_{j \neq i}(z_{ik}-z_{jk})}{D_{ik}},\enskip k\hm=1,2,3\,;$$
$D_{ik}$~--- выборочная дисперсия $v_k$ на $i$-м эталонном классе.


При помощи функций $d_i(v)$ по сути производится вычисление расстояний~$\rho$ 
от центров эталонных классов
до точек цветового пространства~$v$. Теоретически точка~$v$ находится 
в~$i$-м эталонном классе, если $d_i(v)\hm=\min_{j}d_j(v)$.

Для каждого пикселя анализируемого изображения вычислим 
два наиболее близких класса (для простоты
проиндексируем их: 1~--- ближайший и~2~--- второй в~порядке удаления) и~на 
основе расстояний~$\rho_1$ и~$\rho_2$
до них и~индексов цветности соответственно~$X_1$ и~$X_2$ будем вычислять 
цветность~$X$ вектора~$v$ по формуле
\begin{equation*}
    X=X_1+\fr{\rho_1}{\rho_1+\rho_2}\left|X_1-X_2\right|\,.
\end{equation*}

Далее производится восстановление плотности  и~цветной теневой картины 
на основе сформированных индексов
цветности, сведений об эталонных классах и~параметров измерительной установки.

Дискриминантный метод показал хорошую устойчивость как к~внешним,
так и~к~внутренним шумам и~позволил с~необходимой точностью
восстановить истинные значения плотности потока и,~соответственно,
цвета на теневой картине~[12--14].

Нужно отметить тот факт, что константы настройки дискриминантного 
метода установлены таким образом, чтобы с~максимальной точностью оценивать 
плотность потока в~отдельном участке спектра, соответствующем области 
набегающего потока прямо перед столкновением его с~телом.



\subsection{Решение задачи средствами вейвлет-анализа}

Как уже отмечалась, общая концепция любого алгоритма обработки изображений, 
основанного на вейв\-лет-пре\-об\-ра\-зо\-ва\-нии, заключается в~сле\-ду\-ющем:
\begin{enumerate}[(1)]
    \item  преобразование (декомпозиция) изображения;
    \item  обработка массива вейв\-лет-ко\-эф\-фи\-ци\-ен\-тов;
    \item  реконструкция изображения.
\end{enumerate}
Декомпозиция и~реконструкция~--- взаимосвязанные процессы, так как
должны проводиться с~использованием наперед заданного вида вейвлета
и~глубины разложения. Таким образом, сразу появляются два параметра
алгоритма: используемый вейвлет и~глубина декомпозиции. По
результатам тестов с~использованием различных типов вейвлетов
наилучший результат был получен для так называемого обратного биортогонального
семейства. Глубина разложения сказывается на размере деталей,
которые все еще можно считать шумовыми, этот параметр подбирался
исключительно экспериментально: заранее можно было лишь сказать,
что, исходя из информационного смысла вейв\-лет-ко\-эф\-фи\-ци\-ен\-тов,
задействовать слишком большую глубину разложения не имело смысла.

Отметим тот факт, что отсутствуют ка\-кие-ли\-бо априорные сведения о степени 
зашумления изображения. Поскольку в~качестве способа обработки массива 
вейв\-лет-ко\-эф\-фи\-ци\-ен\-тов заявлена пороговая обработка, то в~силу 
леммы~2 получаем, что достаточно оценить дисперсию шума (порог выбран универсальным).

Цветная теневая картина~--- изображение, где каждый пиксель кодируется 
триплетом $(R,G,B)$. Предлагаемый алгоритм вейв\-лет-об\-ра\-бот\-ки 
оценивает пороги для каждой цветовой компоненты в~отдельности, тем 
самым отчасти учитывается тот факт, что зашумление может по-раз\-но\-му 
проявляться в~разных частях спектра. Эти принципы вполне соответствуют специфике 
обрабатываемого сигнала: в~ходе специальной проверки были зафиксированы отличия 
в~зашумлении для разных цветовых компонент.

Одной из главных характеристик сигнала является его энергия. 
В~терминах изображения она вычисляется как
\begin{equation}
    \left\|C_0\right\|=\sum\limits_{i,j=1}^{N}
    \left({r_{i,j}^2+g_{i,j}^2+b_{i,j}^2}\right)\,,
\label{image_energy}
\end{equation}
где $(r_{i,j},g_{i,j},b_{i,j})$~--- пиксель изображения.

Результат сравнения энергий, заключающихся в~аппроксимации и~детализации 
изображения при глубине разложения не более~4, показал, что значительная доля 
сигнала заключена в~аппроксимирующих коэффициентах, поэтому малые изменения, 
вносимые в~них, могут серьезно повлиять на результат обработки. 
Вследствие этого было принято решение о невнесении ка\-ких-ли\-бо изменений 
в~коэффициенты аппроксимации. Этот вывод вполне согласуется с~физическим смыслом: 
низкочастотный вейв\-лет-фильтр выделяет главные особенности сигнала, по сути, 
создавая сглаженный и~уменьшенный его вариант, изменять его~--- 
подвергаться значительному риску потери информации о сигнале. Коэффициенты 
детализации же как раз характеризуют отличия аппроксимации от оригинала, 
большие по величине вряд ли соответствуют шуму (в~терминах постановки задачи), 
в~то время как малые (они и~отвечают зашумлению) можно приравнять к~0, 
что как раз соответствует пороговой обработке сигнала.

Дополнительное исследование вейв\-лет-ко\-эф\-фи\-ци\-ен\-тов показало, 
что в~качестве оценки~$\sigma$ можно взять величину~$\hat{\sigma}$, определяемую 
формулой~(\ref{sigma_hat}) для каждого набора коэффициентов детализации.


Описанный алгоритм вейв\-лет-об\-ра\-бот\-ки показал высокую точность
восстановления цвет\-ности теневых аэродинамических картин, что
подтверждается проведенным сравнительным анализом с~теневыми
картинами, обработанными дискриминантным методом~[11--14], для
верификации которого использовалась система оптических \mbox{клиньев}.


\subsection{Примеры обработки изображений с~использованием дискриминантного 
и~вейвлет-методов}

В данном разделе в~качестве примеров обработки изображений приводятся результаты 
обработки\linebreak картин обтекания цилиндра. Так как дискриминантный метод оптимизирован 
(т.\,е.\ демонст\-ри\-рует наивысшую точность результатов) для обра\-ботки  зоны 
цветового спектра соответствующей\linebreak плот\-ности потока перед исследуемым телом, 
то особый акцент будет сделан на сравнении получаемых результатов именно в~этой 
об\-ласти изображения.

\subsubsection{Исходные данные и~сравнение результатов обработки}

Для данной публикации все приводимые ниже рисунки конвертированы в~формат 
<<Градации серого>>. Исходные цифровые цветные изображения в~формате~$\mathrm{RGB}$, 
взятые для анализа и~в~дальнейшем обработанные обоими методами, приведены 
в~приложении~[15].

На рис.~\ref{pic1},\,\textit{а} четко заметно возмущение, возникающее в~поле плот\-ности 
набегающего потока перед телом. Уточнение значений плотности газа в~данной области~--- 
основная цель создания дискриминантного метода. На рис.~\ref{pic1},\,\textit{б} 
данная особенность не просматривается.
\begin{figure*} %fig1
\vspace*{1pt}
 \begin{center}  
\mbox{%
 \epsfxsize=157.726mm
 \epsfbox{zah-1.eps}
 }
\end{center} 
\vspace*{-9pt}
\Caption{Изображения~1~(\textit{а}) и~2~(\textit{б}) до обработки}
\label{pic1}
\end{figure*}


Поскольку особое значение имеет обработка области перед телом, приведем также 
соответствующим образом кадрированное изображение, полученное на основе 
первого (рис.~2).



Введем обозначения, которые будут использоваться далее:
\begin{itemize}
    \item $r$, $g$, $b$~--- значения соответствующих цветовых компонент 
    пикселя из диапазона $0\ldots255$;
    \item пара нижних индексов для цветовых компонент~--- координаты пикселя 
    в~изображении от верхнего левого угла;
    \item изображение имеет размер $N \times N$ пикселей ($N\hm=512$);
    \item верхний индекс~discr свидетельствует о принадлежности значения 
    к~результатам обработки дискриминантным методом, wave~--- 
    вейв\-лет-ал\-го\-рит\-мом;
    \item под элементом изображения понимается некоторая цветовая компонента пикселя.
\end{itemize}
 \noindent
 \begin{center}  %fig2
 \vspace*{1pt}
 \mbox{%
 \epsfxsize=78mm
 \epsfbox{zah-3.eps}
 }



\vspace*{3pt}

\noindent
{{\figurename~2}\ \ \small{Кадрированное изображение~1 до обработки}}

\end{center} 

 \vspace*{9pt}
 
 \addtocounter{figure}{1}



За критерий сравнения работы дискриминантного метода и~вейв\-лет-об\-ра\-бот\-ки 
выберем
число элементов изображения, различающихся не более чем на~2~единицы, 
и~обозначим его через~$\delta$. Тогда
\begin{multline*}
\delta=\sum\limits_{i,j=1,\ldots ,N}\left(
\mathbb{I}\left\{\left|r_{i,j}^{\mathrm{discr}}-r_{i,j}^{\mathrm{wave}}\right|
\leqslant2\right\}+{}\right.\\
{}+\mathbb{I}\left\{\left|g_{i,j}^{\mathrm{discr}}-
g_{i,j}^{\mathrm{wave}}\right|\leqslant2\right\}+{}\\
\left.{}+
\mathbb{I}\left\{\left|b_{i,j}^{\mathrm{discr}}-b_{i,j}^{\mathrm{wave}}\right|\leqslant2\right\}\right).
\end{multline*}


\subsubsection{Результаты обработки вейвлет-методом}

Результаты обработки исходных теневых картин 
вейв\-лет-ме\-то\-дом представлены на рис.~3 и~4.
\begin{figure*} %fig3
\vspace*{1pt}
 \begin{center}  
\mbox{%
 \epsfxsize=157.815mm
 \epsfbox{zah-4.eps}
 }
\end{center} 
\vspace*{-9pt}
\Caption{Изображения~1~(\textit{а}) и~2~(\textit{б})
после вейв\-лет-об\-ра\-бот\-ки с~глубиной 
декомпозиции~3 и~мягким порогом}
\label{pic1_w}
\vspace*{12pt}
\end{figure*}

 



Расчет значения критерия~$\delta$ показал, что около~80\%~элементов 
изображений обрабатываются одинаково обоими рассматриваемыми методами.

Значения нормированных оценок средне\-квад\-ра\-тичной погрешности 
для изображения~1 равны~24,7, 30,1 и~41,6 для цветовых компонент~$r$, $g$ и~$b$ 
соответственно. Эти же значения для изображения~2 равны~23,9, 29,7 и~40,9. 
На основе теоремы~2 можно сделать вывод, что значения нормированных оценок 
среднеквадратичной по\-греш\-ности, которые вычисляются с~использованием только 
известных эмпирических вейв\-лет-ко\-эф\-фи\-ци\-ен\-тов, с~вероятностью~0,95 
отличаются от истинных среднеквадратичных погрешностей не более чем на~1,1, 1,34 
и~1,99 для изображения~1 и~не более чем на~1,05, 1,32 и~1,9 для изображения~2.


Таким образом, вейв\-лет-ме\-тод, несмотря на кардинально другой подход 
к~способу оценивания зашумления, дает результаты, схожие с~теми, что получа\-ются 
при обработке изображений дискри\-минантным методом. Наличие параметризации 
вейв\-лет-ал\-го\-рит\-ма позволяет применять его к~изоб\-ра\-же\-ниям с~различной 
степенью и~характером зашумле\-ния. Например, в~случае увеличения разрешения 
входного сигнала необходимо будет изменить глубину разложения, чтобы 
величина деталей, среди которых алгоритм отсеивает\linebreak

\noindent
 \begin{center}  %fig4
 \vspace*{1pt}
 \mbox{%
 \epsfxsize=78mm
 \epsfbox{zah-6.eps}
 }



\end{center} 

\noindent
{{\figurename~4}\ \ \small{Изображение~1 после вейв\-лет-об\-ра\-бот\-ки 
с~глубиной декомпозиции~3 и~мягким порогом и~последующего кадрирования}}



 \vspace*{24pt}
 
 \addtocounter{figure}{1}
 
 \noindent
  шумовые, 
соответствующим образом перемасштабировалась. Выбранный в~качестве 
оптимального для тестовой группы изображений вейвлет из обратного 
биортогонального семейства безусловно не является гарантированно 
лучшим для обработки всех теневых картин, он показывал наилучшие 
результаты в~рамках имевшегося набора изображений. Вмес\-те с~тем 
значения параметров гладкости исходного и~итогового сигнала 
вполне соответствуют ожи\-да\-емым значениям с~точки зрения физического 
смыс\-ла цветной теневой картины и~характеристик тестовых изображений: 
вследствие зашумленности исходный сигнал должен обладать минимальной 
гладкостью, в~то время как генерируемый~--- значительно большей.

\section{Выводы}
В ходе решения данной задачи был разработан алгоритм, 
основанный на средствах, предлагаемых вейв\-лет-ана\-ли\-зом. 
Для верификации результатов было проведено сравнение с~дискриминантным 
методом обработки аэродинамических картин.

В ходе тестирования была показана возможность настройки вейв\-лет-ал\-го\-рит\-ма 
для обработки имевшейся группы цветных теневых картин с~достаточной точностью 
относительно результатов обработки дискриминантным методом.

Был проведен расчет нормированной оценки среднеквадратичной погрешности 
по экспериментальным данным для каждого цветового канала. Полученные 
эмпирические значения погрешности хорошо согласуются с~теоретическими 
значениями, рассчитанными на основе свойств ее предельного распределения.

{\small\frenchspacing
 {%\baselineskip=10.8pt
 \addcontentsline{toc}{section}{References}
 \begin{thebibliography}{99}
    \bibitem{holder} %1
    \Au{Холдер Д., Норт Р.} Теневые методы в~аэродинамике~/
    Пер с англ.~--- М.: Мир, 1966.
    180~с. (\Au{Holder~D.\,W., North~R.\,J.} Schlieren methods.~---
    National Physics Laboratory. Notes on applied science No.\,31.~---
    London: H.M.S.O., 1963. 106~p.)
    \bibitem{krasnov}  %2
    \Au{Краснов Н.\.Ф.} Аэродинамика. Т.~2: Методы аэродинамического расчета.~---
     4-е изд.~--- М.: Высшая школа, 2010. 416~с.
         \bibitem{smolentsev}  %3
    \Au{Смоленцев Н.\,К.} 
    Основы теории вейв\-ле\-тов. Вейвлеты в~Matlab.~--- М.: ДМК Пресс, 2005. 157~с.
    \bibitem{posobie} %4
    \Au{Захарова Т.\,В., Шестаков~О.\,В.} 
    Вей\-в\-лет-ана\-лиз и~его приложения.~--- 2-е изд., перераб. и~доп.~--- 
    М.: ИНФРА-М, 2012. 157~с.
    \bibitem{bogges} %5
    \Au{Bogges A., Narkovich~F.\,A.} 
    A~first course in wavelets with Fourier analysis.~--- Prentice Hall, 2001. 293~p.

    \bibitem{malla} %6
    \Au{Малла С.} 
    Вэйвлеты в~обработке сигналов~/ Пер. с~англ.~--- М.: Мир, 2005. 671~с.
    (\Au{Mallat~S.} A~wavelet tour of signal processing.~--- 2nd ed.~---
    Elsevier, 1999. 661~p.)
    \bibitem{DonJ2} %7
\Au{Donoho D., Johnstone~I.} Ideal spatial adaptation via wavelet shrinkage~// 
Biometrika, 1994. Vol.~81. P.~425--455.
    
\bibitem{DonJ}  %8
\Au{Donoho D., Johnstone~I.} 
Adapting to unknown smoothness via wavelet shrinkage~// J.~Am. Statist. Assoc., 1995. Vol.~90. P.~1200--1224.


\bibitem{MA09} %9
\Au{Маркин А.\,В.} Предельное распределение оценки риска при пороговой обработке 
вей\-в\-лет-ко\-эф\-фи\-ци\-ен\-тов~// Информатика и~её применения, 2009. Т.~3. Вып.~4. С.~57--63.

\bibitem{MSH10} %10
\Au{Маркин А.\,В., Шестаков~О.\,В.} 
О~состоятельности оценки риска при пороговой обработке 
вей\-в\-лет-ко\-эф\-фи\-ци\-ен\-тов~// Вестн. Моск. ун-та. Сер.~15. 
Вычисл. матем. и~киберн., 2010. Вып.~1. C.~26--34.
    \bibitem{Eng_dis_m} %11
    \Au{Zakharova T.\,V., Berezentsev~M.\,V.} 
    About a method of supervision classification in the decision of aerodinamic 
    problems~// 24th Seminar (International)
    on Stability Problems for Stochastic Models Transactions.~--- Riga: TTI, 2004. P.~353--356.
    
    \bibitem{Rus_dis_m} %12 
    \Au{Захарова Т.\,В.} Метод распознавания для восстановления изображений 
    цветных теневых картин~// Обозрение прикладной и~промышленной математики, 2005. 
    Т.~12. Вып.~4. С.~967--968.
    \bibitem{aero} 
    \Au{Захарова Т.\,В.,  Шагиров~Э.\,А.} 
    Определение плот\-ности аэродинамического потока обтекания методом цветовой 
    фильтрации~// Математическое моделирование, 2013. Т.~25. Вып.~12. С.~103--109.
    \bibitem{aero2} 
    \Au{Захарова Т.\,В., Шагиров~Э.\,А.} 
    Оптимизация метода цветовой фильтрации для решения задач аэродинамики~// 
    Обозрение прикладной и~промышленной математики, 2013. Т.~20. Вып.~4. С.~545--548.
    \bibitem{addition} 
    \Au{Захарова Т.\,В., Шестаков~О.\,В.} 
    Приложение \mbox{к~статье} <<Анализ точности вей\-в\-лет-об\-ра\-бот\-ки 
    аэродинамических картин обтекания>>. 
    {\sf http://www.ipiran.ru/ \mbox{publications/pictures.pdf}}.
\end{thebibliography}

 }
 }

\end{multicols}

\vspace*{-6pt}

\hfill{\small\textit{Поступила в~редакцию 07.05.16}}

\vspace*{8pt}

%\newpage

%\vspace*{-24pt}

\hrule

\vspace*{2pt}

\hrule

%\vspace*{8pt}



\def\tit{PRECISION ANALYSIS OF~WAVELET PROCESSING OF~AERODYNAMIC FLOW PATTERNS}

\def\titkol{Precision analysis of wavelet processing of aerodynamic flow patterns}

\def\aut{T.\,V.~Zakharova$^{1,2}$ and O.\,V.~Shestakov$^{1,2}$}

\def\autkol{T.\,V.~Zakharova and O.\,V.~Shestakov}

\titel{\tit}{\aut}{\autkol}{\titkol}

\vspace*{-9pt}

\noindent
$^1$Department of Mathematical Statistics, Faculty of Computational Mathematics 
and Cybernetics,\linebreak
$\hphantom{^1}$M.\,V.~Lomonosov Moscow State University, 1-52~Leninskiye Gory, 
GSP-1, Moscow 119991, Russian\linebreak
$\hphantom{^1}$Federation

\noindent
$^2$Institute of Informatics Problems, Federal Research Center 
``Computer Science and Control''
of the Russian\linebreak
$\hphantom{^1}$Academy of Sciences, 44-2~Vavilov Str., Moscow 119333,  Russian Federation



\def\leftfootline{\small{\textbf{\thepage}
\hfill INFORMATIKA I EE PRIMENENIYA~--- INFORMATICS AND
APPLICATIONS\ \ \ 2016\ \ \ volume~10\ \ \ issue\ 3}
}%
 \def\rightfootline{\small{INFORMATIKA I EE PRIMENENIYA~---
INFORMATICS AND APPLICATIONS\ \ \ 2016\ \ \ volume~10\ \ \ issue\ 3
\hfill \textbf{\thepage}}}

\vspace*{3pt}


\Abste{This paper is devoted to a new method of aerodynamic flow pattern processing
based on the wavelet analysis. Wavelet thresholding techniques are widely used in 
signal and image processing. These methods are easily implemented through fast 
algorithms; so, they are very appealing in practical situations. Besides, they adapt 
to function classes with different amounts of smoothness in different locations more 
effectively than the usual linear methods. Wavelet thresholding risk analysis is 
an important practical task, because it allows determining the quality 
of the techniques themselves and the equipment which is being used. Comparative analysis 
using the discriminant method was carried out to verify the new method. 
The empirical estimated error of processing is consistent with theoretical 
results for this estimate.}

\KWE{wavelet analysis; thresholding; unbiased risk estimate; aerodynamic flow}

\DOI{10.14357/19922264160307} 

%\vspace*{-3pt}

\pagebreak

%\Ack
%\noindent



%\vspace*{3pt}

  \begin{multicols}{2}

\renewcommand{\bibname}{\protect\rmfamily References}
%\renewcommand{\bibname}{\large\protect\rm References}

{\small\frenchspacing
 {%\baselineskip=10.8pt
 \addcontentsline{toc}{section}{References}
 \begin{thebibliography}{99}

\bibitem{1-zah}
\Aue{Holder, D.\,W., and R.\,J.~North.} 1963.
\textit{Schlieren methods}.
    National Physics Laboratory. Notes on applied science No.\,31.
    London: H.M.S.O. 106~p.
\bibitem{2-zah}
\Aue{Krasnov, N.\,F.} 2010. \textit{Aerodinamika. T.~2: Metody aerodinamicheskogo rascheta}
[Aerodynamics. Vol.~2: The aerodynamic calculation methods].  4th ed. 
Moscow: Vysshaya Shkola. 416 p.

\bibitem{4-zah} %3
\Aue{Smolentsev, N.\,K.} 2008. 
\textit{Osnovy teorii veyvletov. Veyvlety v~Matlab} 
[Foundations of the theory of wavelets. Wavelets in Matlab]. Moscow: DMK Press. 448~p.
\bibitem{3-zah} %4
\Aue{Zakharova, T.\,V., and O.\,V.~Shestakov}. 2012.
\textit{Veyvlet-analiz i~ego prilozheniya} [Wavelet analysis and its applications]. 
2nd ed. Moscow: INFRA-M. 157~p.

\bibitem{6-zah} %5
\Aue{Boggess, A., and F.\,A.~Narkovich}. 2001. 
\textit{A~first course in wavelets with Fourier analysis}. Prentice Hall. 293~p.
\bibitem{5-zah} %6
\Aue{Mallat, S.} 1999. 
\textit{Wavelet tour of signal processing}. Elsevier. 661~p.

\bibitem{8-zah} %7
\Aue{Donoho, D., and I.~Johnstone}. 1994.  
Ideal spatial adaptation via wavelet shrinkage. \textit{Biometrika} 81:425--455.

\bibitem{7-zah} %8
\Aue{Donoho, D., and I.~Johnstone}. 1995. 
Adapting to unknown smoothness via wavelet shrinkage. 
\textit{J.~Am. Statist. Assoc.} 90:1200--1224.


\bibitem{10-zah} %9
\Aue{Markin, A.\,V.} 2009. 
Predel'noe raspredelenie otsenki riska pri porogovoy obrabotke veyvlet-koeffitsientov 
[Limit distribution of risk estimate of wavelet coefficient thresholding].
\textit{Informatika i~ee Primeneniya~--- Inform. Appl.} 3(4):57--63.
\bibitem{9-zah} %10
\Aue{Markin, A.\,V., and O.\,V.~Shestakov}. 
2010. O~sostoyatel'nosti otsenki riska pri porogovoy obrabotke 
veyvlet-koefficientov [Consistency of risk estimation with thresholding 
of wavelet coefficients]. \textit{Vestn. Mosk. un-ta. Ser.~15. 
Vychisl. matem. i~kibern.} [Moscow University Computational Mathematics and Cybernetics]
1:26-34.

\bibitem{11-zah}
\Aue{Zakharova, T.\,V., and M.\,V.~Berezentsev}. 
2004. About a~method of supervision classification in the decision of 
aerodinamic problems. \textit{24th Seminar 
(International) on Stability Problems for Stochastic Models Transactions}. Riga: TTI. 353--356.
\bibitem{12-zah}
\Aue{Zakharova, T.\,V.} 2005. 
Metod raspoznavaniya dlya vosstanovleniya izobrazheniy tsvetnykh tenevykh kartin 
[Recognition method for recovery of nonferrous shadow paintings images]. 
\textit{Obozrenie Prikladnoy i~Promyshlennoy Matematiki} 
[Review of Applied and Industrial Mathematics] 12(4):967--968.
\bibitem{13-zah}
\Aue{Zakharova, T.\,V., and E.\,A.~Shagirov}. 2013. 
Opredelenie plotnosti aerodinamicheskogo potoka obtekaniya metodom tsvetovoy 
fil'tratsii [Density determination of aerodynamic flow color flow filtration method]. 
\textit{Matematicheskoe modelirovanie} [Mathematical Modeling] 25(12):103--109.
\bibitem{14-zah}
\Aue{Zakharova, T.\,V., and E.\,A.~Shagirov}.  
2013. Optimizatsiya metoda tsvetovoy fil'tratsii dlya resheniya zadach 
aerodinamiki [Optimization of the color filter method for solving the problems 
of aerodynamics]. \textit{Obozrenie Prikladnoy i~Promyshlennoy Matematiki} 
[Review of Applied and Industrial Mathematics] 20(4):545--548.
\bibitem{15-zah}
\Aue{Zakharova, T.\,V., and O.\,V.~Shestakov.}
Appendix to the paper ``Precision analysis of wavelet processing of aerodynamic 
flow patterns.'' 
Available at: {\sf http://www.ipiran.ru/publications/pictures.pdf}
(accessed August~29, 2016).
\end{thebibliography}

 }
 }

\end{multicols}

\vspace*{-3pt}

\hfill{\small\textit{Received May 07, 2016}}

\Contr

\noindent
\textbf{Zakharova Tatiana V.} (b.\ 1962)~--- Candidate of Science (PhD) in 
physics and mathematics; senior lecturer, Department of Mathematical Statistics, 
Faculty of Computational Mathematics and Cybernetics, M.\,V.~Lomonosov Moscow 
State University, 1-52~Leninskiye Gory, GSP-1, Moscow 119991, Russian Federation; 
senior scientist, Institute of Informatics Problems, Federal Research Center 
``Computer Science and Control'' of the Russian Academy of Sciences, 
44-2~Vavilov Str., Moscow 119333, Russian Federation; \mbox{lsa@cs.msu.ru}

\vspace*{3pt}

\noindent
\textbf{Shestakov Oleg V.} (b.\ 1976)~--- 
Doctor of Science in physics and mathematics, associate professor, 
Department of Mathematical Statistics, Faculty of Computational Mathematics 
and Cybernetics, M.\,V.~Lomonosov Moscow State University, 1-52~Leninskiye Gory, 
GSP-1, Moscow 119991, Russian Federation; senior scientist, 
Institute of Informatics Problems, Federal Research Center 
``Computer Science and Control''
of the Russian Academy of Sciences, 44-2~Vavilov Str., Moscow 119333, 
Russian Federation; \mbox{oshestakov@cs.msu.su}
\label{end\stat}


\renewcommand{\bibname}{\protect\rm Литература} %7
\def\stat{sinits}

\def\tit{АНАЛИТИЧЕСКОЕ МОДЕЛИРОВАНИЕ
НОРМАЛЬНЫХ ПРОЦЕССОВ В~СТОХАСТИЧЕСКИХ СИСТЕМАХ СО~СЛОЖНЫМИ~НЕЛИНЕЙНОСТЯМИ}

\def\titkol{Аналитическое моделирование
нормальных процессов в~стохастических системах со~сложными нелинейностями}

\def\aut{И.\,Н.~Синицын$^1$, В.\,И.~Синицын$^2$}

\def\autkol{И.\,Н.~Синицын, В.\,И.~Синицын}

\titel{\tit}{\aut}{\autkol}{\titkol}

\renewcommand{\thefootnote}{\arabic{footnote}}
\footnotetext[1]{Институт проблем
информатики Российской академии наук, sinitsin@dol.ru}
\footnotetext[2]{Институт проблем
информатики Российской академии наук, vsinitsin@ipiran.ru}


\Abst{Рассматриваются конечномерные дифференциальные стохастические системы
(ДСтС) и эредитарные (интегродифференциальные) стохастические системы  (ЭСтС)
с винеровскими и пуассоновскими шумами, приводимые к ДСтС со сложными конечными,
дифференциальными и интегральными нелинейностями. Такие модели функционирования
описывают поведение многих современных нано- и кван\-то\-во-оп\-ти\-че\-ских
технических средств информатики. Приводятся уравнения методов нормальной
аппроксимации (МНА) и статистической линеаризации (МСЛ) для аналитического
моделирования нестационарных и стационарных нормальных (гауссовских) процессов
в нелинейных ДСтС и  нелинейных ЭСтС путем аппроксимации эредитарных ядер
линейными операторными уравнениями для дифференцируемых нелинейностей и
сингулярными ядрами для недифференцируемых нелинейностей. Рассматриваются
методы вычисления типовых интегралов МНА (МСЛ) для сложных (многомерных и
векторного аргумента) конечных и дифференциальных нелинейностей. Особое
внимание уделяется иррациональным и дробно-рациональным нелинейностям
скалярного аргумента. Приводятся примеры вычисления интегралов. Подробно
рассматриваются вопросы вычисления типовых интегралов МНА (МСЛ) для сложных
интегральных нелинейностей.}

\KW{аналитическое моделирование;
дифференциальные стохастические системы с винеровскими и пуассоновскими шумами (ДСтС);
метод нормальной аппроксимации (МНА);
метод статистической линеаризации (МСЛ);
сложные иррациональные нелинейности;
сложные конечные, дифференциальные и интегральные нелинейности;
эредитарные стохастические системы (ЭСтС), приводимые к дифференциальным}

\DOI{10.14357/19922264140302}

\vspace*{9pt}

\vskip 16pt plus 9pt minus 6pt

\thispagestyle{headings}

\begin{multicols}{2}

\label{st\stat}


\section{Введение}


Моделями функционирования многих современных технических сис\-тем информатики
служат стохастические системы (СтС), описываемые дифференциальными, интегральными
и интегродифференциальными уравнениями со сложными дроб\-но-ра\-ци\-о\-наль\-ны\-ми,
иррациональными и интегральными нелинейностями. В~[1] дано систематическое
изложение МНА и МСЛ для ДСтС и ЭСтС, приводимых к дифференциальным.

Обобщая~[2--7], рассмотрим развитие МНА и МСЛ для аналитического моделирования
нормальных стохастических процессов (СтП) на случай СтС со сложными конечными,
дифференциальными и интегральными нелинейностями.

Как показано в~\cite{4-sin}, альтернативным подходом к аналитическому моделированию
СтП в ДСтС и ЭСтС служит подход, основанный на дискретизации стохастических
дифференциальных уравнений на основе использования обобщенной формы Ито и
кратных стохастических интегралов от винеровских и пуассоновских СтП с
последующим применением дискретных версий МНА (МСЛ).

Статья состоит из введения, пяти разделов и заключения.

В~разд.~2 и~3
приводятся уравнения МНА и МСЛ для аналитического моделирования одно- и
двумерных распределений стационарных и нестационарных СтП в ДСтС и ЭСтС,
приводимых к ДСтС.

Типовые интегралы МНА и МСЛ рассматриваются в разд.~4.

Особенности аналитического моделирования в ДСтС со сложными конечными и
дифференциальными нелинейностями обсуждаются в разд.~5.

Раздел~6
посвящен аналитическому моделированию СтП в ДСтС со сложными интегральными
нелинейностями.

Приводятся примеры.


\section{Уравнения методов нормальной~аппроксимации и~статистической
линеаризации для~дифференциальных стохастических систем}

Как известно~\cite{2-sin, 3-sin},  уравнения конечномерных непрерывных нелинейных сис\-тем
со стохастическими возмущениями путем расширения вектора состояния ДСтС
могут быть записаны в виде следующего векторного стохастического
дифференциального уравнения Ито:
    \begin{multline}
    dY_t = a(Y_t, t)\, dt + b (Y_t, t) \,dW_0+{}\\
    {}+ \iii_{R_0} c (Y_t, t, v) P^0
    (dt, dv)\,,\enskip Y(t_0) = Y_0\,.\label{e2.1-sin}
    \end{multline}
Здесь $a=a(Y_t, t)$ и $b\hm=b(y_t, t)$~--- известные
$(p\times 1)$-мер\-ная и  $(p\times m)$-мер\-ная функции~$Y_t$ и~$t$;
$W_0\hm= W_0(t)$~--- $r$-мер\-ный винеровский СтП интенсивности
$\nu_0 \hm= \nu_0(t)$; $c(Y_t, t, v)$~--- $(p\times 1)$-мер\-ная функция  $Y_t, t$
и вспомогательного $(q\times 1)$-мер\-но\-го па\-ра\-мет\-ра~$v$;
$\iii_{\Delta} dP^0 (t, A)$~--- центрированная пуассоновская мера,
определяемая
\begin{equation*}
\iii_{\Delta} dP^0 (t, A) = \iii_{\Delta} dP (t,A) =
\iii_{\Delta} \nu_P (t,A)\, dt\,. %\label{e2.2-sin}
\end{equation*}
В~(\ref{e2.1-sin}) принято: $\iii_{\Delta}$~-- число скачков пуассоновского
СтП в интервале времени  $\Delta \hm= (t_1, t_2]$; $\nu_P (t, A)$~---
интенсивность пуассоновского СтП  $P(t,A)$; $A$~--- некоторое борелевское
множество пространства  $R_0^q$ с выколотым началом.
Начальное значение~$Y_0$ представляет собой случайную величину, не зависящую
от приращений СтП  $W_0(t)$ и $P(t,A)$ на интервалах времени, следующих
за~$t_0$, $t_0 \hm\le t_1\hm\le t_2$ для любого множества~$A$.

В случае аддитивных нормальных (гауссовских) и обобщенных
пуассоновских возмущений уравнение~(\ref{e2.1-sin}) имеет вид:
\begin{equation}
\dot Y_t = a(Y_t,t)+ b_0 (t) V\,, \enskip
V = \dot W\,,\enskip Y(t_0) = Y_0\,.\label{e2.3-sin}
\end{equation}
Здесь $W$~--- СтП с независимыми приращениями, представляющий собой
смесь нормального и обобщенного пуассоновского СтП.

Если предположить существование конечных вероятностных
моментов второго порядка для моментов времени~$t_1$ и~$t_2$, то уравнения
МНА примут следующий вид~\cite{2-sin, 3-sin}:
\begin{itemize}
\item  для характеристических функций
    \begin{equation}
    g_1^N (\la;t) =\exp \lk i\la^{\mathrm{T}} m_t - \fr{1}{2}\, \la^{\mathrm{T}} K_t \la\rk\,;\label{e2.4-sin}
    \end{equation}
\begin{equation}
\hspace*{-7.5mm}g_{t_1, t_2}^N (\la_1, \la_2;t_1, t_2 ) =\exp \lk i\bar \la^{\mathrm{T}} \bar m_2 -
\fr{1}{2}\, \bar \la^{\mathrm{T}} \bar K_2 \la\rk\,,\!\!\label{e2.5-sin}
\end{equation}
где
    \begin{gather*}
    \bar \la =\lk \la_1^{\mathrm{T}}\la_2^{\mathrm{T}}\rk^{\mathrm{T}}\,; \quad
        \bar m_2 = \lk m_{t_1}^{\mathrm{T}} m_{t_2}^{\mathrm{T}}\rk^{\mathrm{T}}\,;\\
        \bar K_2= \begin{bmatrix}
    K(t_1, t_1)& K(t_1, t_2)\\
    K(t_2, t_1)& K(t_2, t_2)
    \end{bmatrix}\,;
    \end{gather*}

\item для математических ожиданий  $m_t$, ковариационной матрицы~$K_t$ и
матрицы ковариационных функций $K(t_1, t_2)$:
    \begin{equation}
    \dot m_t = a_1 (m_t, K_t, t)\,,\enskip m_0 = m(t_0)\,;\label{e2.6-sin}
    \end{equation}
\begin{equation}
\dot K_t = a_2 (m_t, K_t, t)\,,\enskip K_0 = K(t_0)\,;\label{e2.7-sin}
\end{equation}

\vspace*{-12pt}

\noindent
\begin{multline}
\fr{\prt K(t_1, t_2)}{\prt t_2 }= K(t_1, t_2) a_{21} (m_{t_2}, K_{t_2}, t_2)^{\mathrm{T}}\,;\\
K(t_1, t_1) = K_{t_1}\,.
\label{e2.8-sin}
\end{multline}
    \end{itemize}
Здесь приняты следующие обозначения:
\begin{equation}
a_1 = a_1 (m_t, K_t, t) = M_N a (Y_t, t)\,;\label{e2.9-sin}
\end{equation}

\vspace*{-12pt}

\noindent
\begin{multline}
a_2 = a_2 (m_t, K_t, t) = a_{21} (m_t, K_t, t)+{}\\
{}+ a_{21} (m_t, K_t, t)^{\mathrm{T}} +
a_{22}(m_t, K_t, t)\,;\label{e2.10-sin}
\end{multline}

\vspace*{-12pt}

\noindent

\begin{equation}
a_{21} = a_{21}(m_t, K_t, t)=  M_N a(Y_t, t) Y_{t}^{0\mathrm{T}}\,;\label{e2.11-sin}
\end{equation}
\begin{equation*}
a_{22} = a_{22}(m_t, K_t, t)= M_N \sigma (Y_t, t)\,;
%\label{e2.12-sin}
\end{equation*}

\vspace*{-12pt}

\noindent
\begin{multline*}
\sigma (Y_t, t) = b(Y_t, t) \nu_0(t) b(Y_t, t)^{\mathrm{T}} +{}\\
{}+
\iii_{R_0^q} c (Y_t, t, v) c(Y_t, t,v)^{\mathrm{T}}
\nu_P (t, dv)\,; %\label{e2.13-sin}
\end{multline*}

\vspace*{-12pt}

\begin{gather*}
m_t = MY_t\,,\quad Y_t^0 = Y_t - m_t\,,\\
K_t = M_N Y_0^0 Y_t^{0\mathrm{T}}\,,\quad K(t_1, t_2) =
M_N Y_{t_1}^0 Y_{t_2}^0\,; %\label{e2.14-sin}
\end{gather*}
$M_N$~--- символ вычисления математического ожидания для нормальных
распределений~(\ref{e2.4-sin}) и~(\ref{e2.5-sin}).

Для стационарных ДСтС нормальные стационарные СтП~--- если они существуют,
то  $m_t \hm=\bar m$, $ K_t \hm=\bar K$, $K(t_1, t_2) \hm= k(\tau)$
$(\tau \hm= t_1\hm-t_2)$,~--- определяются уравнениями~\cite{2-sin, 3-sin}:
   \begin{equation}
   a_1 (\bar m, \bar K) =0\,;\enskip a_2 (\bar m, \bar K)=0\,;\label{e2.15-sin}
   \end{equation}
   \begin{equation}
   \left.
   \hspace*{-2.8mm}\begin{array}{l}
  \dot k_\tau (\tau) = a_{21} (\bar m, \bar K)\bar K^{-1} k(\tau)\,;\\[9pt]
  k(0) =\bar K \enskip (\forall \tau >0)\,, \
  k(\tau) = k(-\tau)^{\mathrm{T}} \enskip
  (\forall\tau <0)\,.
  \end{array}\!\!
  \right\}\!\!
  \label{e2.16-sin}
  \end{equation}
При этом необходимо, чтобы матрица  $a_{21} (\bar m, \bar K)\hm=\bar a_{21}$
была бы асимптотически устойчивой.

Для ДСтС~(\ref{e2.3-sin}) уравнения МНА переходят в уравнения МСЛ
Казакова~\cite{2-sin, 3-sin}, если принять
\begin{equation}
a(Y_t,t) = a_1 (m_t, K_t) + k_1^a (m_t, K_t) Y_t^0\,;\label{e2.17-sin}
\end{equation}
\begin{equation}\left.
\begin{array}{rl}
b(Y_t,t) &= b_0 (t)\,;\\[9pt]
    \si(Y_t, t)&= b_0(t) \nu(t) b_0(t)^{\mathrm{T}} = \si_0(t)\,,
    \end{array}
    \right\}\label{e2.18-sin}
    \end{equation}
    \begin{equation}
k_1^a (m_t, K_t, t) =\lk \left(\fr{\prt}{\prt m_t} \right)
    a_0 (m_t, K_t, t)^{\mathrm{T}}\rk^{\mathrm{T}}\,;\label{e2.19-sin}
    \end{equation}
    \begin{equation}
\dot m_t = a_1 (m_t, K_t, t) \,,\enskip m_0 = m(t_0)\,,\label{e2.20-sin}
\end{equation}

\vspace*{-12pt}

\noindent
\begin{multline}
\dot K_t = k_1^a (m_t, K_t, t) K_t + K_t k_1^a (m_t, K_t, t)^{\mathrm{T}}
    +\si_0(t)\,;\\
    K_0 = K(t_0)\,;
    \label{e2.21-sin}
    \end{multline}

    \vspace*{-12pt}

    \noindent
\begin{multline}
\fr{\prt K(t_1, t_2)}{\prt t_2} =
    K(t_1, t_2) k_{t_2} k_1^a (m_{t_2}, K_{t_2}, t_2)^{\mathrm{T}}\,;\\
    K(t_1, t_2) = K_{t_1}\,.
    \label{e2.22-sin}
\end{multline}

Для стационарных ДСтС~(\ref{e2.3-sin})
при условии асимптотической устойчивости матрицы $k_1^a (\bar m, \bar K)$
в основе МСЛ лежат уравнения~(\ref{e2.15-sin}), записанные в виде:
    \begin{gather}
    a_1 (\bar m, \bar K) =0\,; \label{e2.23-sin}\\
k_1^a (\bar m, \bar K) \bar K + \bar K k_1^a
(\bar m, \bar K)^{\mathrm{T}} +\bar \si_0 =0\,;\label{e2.24-sin}
\end{gather}

\vspace*{-12pt}

\noindent
\begin{multline}
k_\tau (\tau) = k_1^a (\bar m, \bar K)k(\tau)\,,\enskip
k(0) =\bar K \enskip (\forall \tau >0)\,,\\
k(\tau) = k (-\tau)^{\mathrm{T}} \enskip (\forall \tau <0)\,.
\label{e2.25-sin}
\end{multline}

Уравнения~(\ref{e2.4-sin})--(\ref{e2.8-sin})
лежат в основе МНА для ДСтС~(\ref{e2.1-sin}), а уравнения~(\ref{e2.17-sin})--(\ref{e2.22-sin})~---
в основе МСЛ для ДСтС~(\ref{e2.3-sin}). Для определения стационарных СтП
согласно МНА служат соотношения~(\ref{e2.15-sin}) и~(\ref{e2.16-sin}),
а МСЛ~--- (\ref{e2.17-sin})--(\ref{e2.25-sin}).

\section{Уравнения методов нормальной~аппроксимации и~статистической линеаризации
для~эредитарных стохастических систем, приводимых к~дифференциальным}

Рассмотрим ЭСтС, описываемую интегродифференциальным уравнением Ито
следующего вида~\cite{7-sin}:

\noindent
\begin{multline}
dX_t = \lk a(X_t, t) +\iii_{t_0}^t a_1 (X(\tau) ,\tau, t)\,d\tau\rk dt+{}\\
{}+\lk b(X_t, t) +\iii_{t_0}^t b_1 (X(\tau) ,\tau, t)\,d\tau\rk dW_0+{}\\
\hspace*{-1.5mm}{}+\!\!\iii_{R_0^q}\!\!\lk c(X_t, t,v) +\!\iii_{t_0}^t\! c_1 (X(\tau) ,\tau, t,v)\,d\tau\!\rk\! dP^0 (t, dv)
\!\!\!\!\label{e3.1-sin}
\end{multline}
с начальным условием  $X(t_0) = X_0$. В~(\ref{e3.1-sin})
сохранены обозначения разд.~2.

Функции $a=a(X_t, t)$, $a_1\hm = a_1(X (\tau),\tau, t)$,
$b\hm=b(X_t, t)$, $b_1\hm = b_1(X (\tau),\tau, t)$,
$c\hm=c(X_t,t,v)$ и $c_1\hm = c_1(X (\tau),\tau, t,v)$ имеют
соответственно размерности $p\times 1$, $p\times 1$, $p\times r$,
$p\times r$, $p\times 1$ и $p\times 1$ и допускают представления следующего вида:
\begin{equation}
\left.
\begin{array}{rl}
a_1&=A(t,\tau) \vrp (X(\tau) , \tau)\,;\\[9pt]
b_1&=B(t,\tau) \psi (X(\tau) ,  \tau)\,;\\[9pt]
c_1&=C(t,\tau) \chi (X(\tau) ,  \tau, v)\,.
\end{array}
\right\}
\label{e3.2-sin}
\end{equation}
Здесь эредитарные ядра $A\hm=A(t,\tau)\hm=\lk A_{ij}(t,\tau)\rk$
$(i,j\hm=\overline{1,p})$,
$B\hm=B(t,\tau)=\lk B_{i l}(t,\tau)\rk$ $(i\hm=\overline{1,p}$;
$l\hm=\overline{1,r})$ и $C\hm=C(t,\tau)=\lk C_{ij}(t,\tau)\rk$
$(i,j\hm=\overline{1,p})$ имеют соответственно размерности
$p\times p$, $p\times r$ и $p\times p$. Они удовлетворяют следующим условиям
физической реализуемости и асимптотического затухания:
\begin{multline}
A_{ij}(t,\tau)=0;\enskip B_{i l}(t,\tau)=0;\\[1pt]
C_{ij}(t,\tau)=0\enskip \forall \tau >t;\label{e3.3-sin}
\end{multline}

\vspace*{-12pt}

\begin{equation}
\left.
\hspace*{-3mm}\begin{array}{c}
\displaystyle\iin\! \lv A_{ij} (t,\tau) \rv d\tau <\infty\,;\
\displaystyle\iin\! \lv B_{i l} (t,\tau) \rv d\tau <\infty \,;\\[9pt]
\displaystyle\iin \!\lv C_{ij} (t,\tau) \rv d\tau <\infty\,.
\end{array}\!
\right\}\!
\label{e3.4-sin}
\end{equation}

В дальнейшем ограничимся случаем, когда эредитарные ядра удовлетворяют
линейным операторным уравнениям~\cite{6-sin, 5-sin, 7-sin}.

Нелинейные в общем случае функции $\vrp\hm=\vrp(X(\tau),\tau)$,
$\psi \hm=\psi(X(\tau), \tau)$, $\chi \hm=\chi (X(\tau),  \tau, v)$
отражают нелинейные свойства ЭСтС, зависят от  $X(\tau)$ и имеют размерности
$p\times 1$, $p\times p$, $p\times 1$ соответственно.

Важный класс  эредитарных ядер представляют собой
сингулярные (вырожденные) ядра, когда имеют место представления:
\begin{equation}
\left.
\hspace*{-3mm}\begin{array}{rl}
A_{ij} (t,\tau) &= A_{ij}^+(t) A_{ij}^-(\tau)\,;\\[9pt]
B_{i l} (t,\tau)& = B_{il}^+(t) B_{il}^-(\tau)\,;\\[9pt]
C_{ij} (t,\tau) &= C_{ij}^+ ( t) C_{ij}^- (\tau)\
(i,l= \overline{1,p}, j=\overline{1,r}).
\end{array}\!
\right\}\!\!
\label{e3.5-sin}
\end{equation}

В~\cite{6-sin, 5-sin, 7-sin} показано, что для дифференцируемых нелинейных
функций~$\vrp$, $\psi$, $\chi$ путем расширения вектора состояния за счет
инструментальных переменных, аппроксимируемых линейными операторными уравнениями,
определяющими эредитарные ядра в ЭСтС, (\ref{e3.1-sin})--(\ref{e3.4-sin})
приводятся к ДСтС вида~(\ref{e2.1-sin}) или~(\ref{e2.3-sin}).
В~случае недифференцируемых нелинейных функций~$\vrp$, $\psi$, $\chi$
ЭСтС~(\ref{e3.1-sin})--(\ref{e3.4-sin}) приводятся к~(\ref{e2.1-sin}) или~(\ref{e2.3-sin})
на основе аппроксимации вырожденными (сингулярными) ядрами~\cite{6-sin, 5-sin, 7-sin}.

Таким образом, после приведения ЭСтС~(\ref{e3.1-sin}) к ДСтС~(\ref{e2.1-sin})
или~(\ref{e2.3-sin}) можно воспользоваться уравнениями МНА и МСЛ разд.~2.

\section{Типовые интегралы методов нормальной аппроксимации и~статистической
линеаризации}

Как следует из уравнений~(\ref{e2.9-sin})--(\ref{e2.11-sin}),
для МНА необходимо уметь вычислять следующие интегралы:
\begin{multline}
I_0^a = I_0^a (m_t, K_t, t) = a_1 (m_t, K_t, t)={}\\
{}= M_N a(Y_t, t)\,;
\label{e4.1-sin}
\end{multline}

\vspace*{-12pt}

\noindent
\begin{multline}
I_1^a = I_1^a (m_t, K_t, t)= a_{21}(m_t, K_t, t)= {}\\
{}=M_N a(Y_t , t) Y_t^{0\mathrm{T}}\,;\label{e4.2-sin}
\end{multline}

\vspace*{-12pt}

\noindent
\begin{multline}
I_0^{\bar \si} = I_0^{\bar \si} (m_t, K_t, t) = a_{22}(m_t, K_t, t) ={}\\
{}= M_N \bar \si (Y_t, t)\,.\label{e4.3-sin}
\end{multline}
Для МСЛ достаточно вычислить интеграл~(\ref{e4.1-sin}),
причем интеграл~$I_1^a$ вычисляется по формуле~\cite{2-sin, 3-sin, 4-sin}:
\begin{equation*}
k_1^a = k_1^a (m_t, K_t, t)=\lk \left( \fr{\prt}{\prt m_t}\right)
I_0^a (m_t, K_t, t)^{\mathrm{T}}\rk^{\mathrm{T}}. %\label{e4.4-sin}
\end{equation*}

\medskip

\noindent
\textbf{Пример 1.} В~[1] для типовых степенных, тригоно\-мет\-ри\-че\-ских,
показательных и ку\-соч\-но-по\-сто\-ян\-ных нелинейностей $Z_t \hm=\vrp (Y_t, t)$
скалярного и векторного аргумента приведены формулы для интегралов
$I_0^\vrp \hm= I_0^\vrp (m_t^y, K_t^y, t)$, а также
$k_1^\vrp \hm= k_1^\vrp (m_t^y, K_t^y, t)$.

\medskip

\noindent
\textbf{Замечание.}
 Важно иметь в виду, что уравнения МНА (МСЛ) содержат интегралы
 $I_0^a$, $I_1^a$, $I_0^\si$ в виде соответствующих коэффициентов.
 Поэтому процедура вычисления интегралов должна быть согласована с
 методом численного решения обыкновенных дифференциальных уравнений для
 $m_t$, $K_t$ и $K(t_1, t_2)$. Эти коэффициенты допускают дифференцирование
 по~$m_t$ и~$K_t$, так как под интегралом стоит сглаживающая нормальная плотность.

\section{Сложные конечные и~дифференциальные нелинейности}

Важный класс сложных конечных нелинейностей (многомерных и векторного аргумента)
представляют собой сложные функции вида:
    \begin{equation*}
    \xi =\vrp (X_t, Y_t, t)\,,\enskip X_t =\psi (Y_t, t)\,. %\label{e5.1-sin}
    \end{equation*}
В~этом случае вычисление интегралов (см.\ разд.~4) проводится по совокупности
переменных  $\lk X_t^{\mathrm{T}} Y_t^{\mathrm{T}}\rk^{\mathrm{T}}$.
К таким нелинейностям, например, относятся дроб\-но-ра\-ци\-о\-наль\-ные,
иррациональные  нелинейности, выражаемые специальными функциями, многозначные
нелинейности, зависящие от СтП~$X_t$ и его производных~$\dot X_t$,  $\ddot X_t$
и~др.

\medskip

\noindent
\textbf{Пример 2.}
Рассмотрим вычисление интегралов~(\ref{e4.1-sin}) и~(\ref{e4.2-sin})
для сложных одномерных иррациональных нелинейностей скалярного аргумента
\begin{equation}
\vrp (Y_t, t) =\lv Y_t\rrv^{\alpha-1}\, \mathrm{sgn}\, Y_t
\label{e5.2-sin}
\end{equation}
($\alpha$~--- нецелый показатель).

Пользуясь~(\ref{e2.16-sin}) и~(\ref{e2.19-sin}), представим~(\ref{e5.2-sin}) в виде
\begin{equation*}
\vrp(Y_t, t) = \vrp_0 (m_t, D_t, t) + k_1^\vrp(m_t, D_t, t) Y_t^0. %\label{e5.3-sin}
\end{equation*}
Здесь введены следующие обозначения:
\begin{gather*}
\vrp_0(m_t, D_t, t) =\Gamma(\alpha) D_t^{1/2} e^{-\xi^2/4} D_{-\alpha} (\xi)\,;%\label{e5.4-sin}
\\
k_1^a (m_t, D_t, t) =\fr {\prt \vrp_0(m_t, D_t, t)}{\prt m_t}\,,%\label{e5.5-sin}
\end{gather*}
где  $\Gamma(\alpha)$~--- гамма-функция,  $\xi \hm= m_t/\sqrt{D_t}$~---
отношение <<сиг\-нал--шум>>; $D_{-\alpha} (\xi)$~---
функция параболического цилиндра~\cite{9-sin}.
При вычислении были учтены следующие соотношения~\cite{9-sin, 8-sin}:
\begin{multline}
\iii_0^\infty x^{\alpha-1} e^{-\beta x^2 - \gamma x} \,dx ={}\\
{}=
(2\beta)^{-\alpha/2} \Gamma(\alpha) \exp \left(\fr{\gamma^2}{8\beta}\right)
D_{-\alpha} \left(\fr{\gamma}{\sqrt{2\beta}}\right)\,;\label{e5.6-sin}
\end{multline}

\vspace*{-12pt}

\noindent
\begin{multline}
\fr{dD_\rho(\xi)}{d\xi} =
   -\fr{\xi}{2}\, D_\rho (\xi) -\rho D_{\rho-1} (\xi) =
   \fr{\xi}{2}\, D_\rho (\xi) -{}\\
   {}- D_{\rho+1} (\xi) \enskip
   (\mathrm{Re}\, \beta>0\,,\enskip \mathrm{Re}\,\alpha>0\,,\enskip
   \rho=-\alpha)\,.\label{e5.7-sin}
   \end{multline}

Соотношения~(\ref{e5.6-sin}) и~(\ref{e5.7-sin})
могут быть использованы также для вычисления интегралов~(\ref{e4.3-sin}).

\medskip

\noindent
\textbf{Замечание.}
Для вычисления интегралов $I_0^a$, $I_1^a$ и $I_0^{\bar \si}$
применительно к типовым иррациональным нелинейностям вида
    $\lv Y_t\rrv^{\alp-1} e^{\delta Y_t}$, $\lv Y_t\rrv^{\alp-1}  \cos \w Y_t$,
    $\lv Y_t\rrv^{\alp-1}  \sin \w Y_t$
и более общим нелинейностям \mbox{вида}
    \begin{equation*}
    \vrp (Y_t, t) =\Phi^\vrp \left( \lv Y_t\rrv^{\alpha-1}, t\right) %\label{e5.8-sin}
    \end{equation*}
можно рекомендовать известные численные методы вычисления функций на ЭВМ~\cite{8-sin}.

\smallskip

\noindent
\textbf{Пример 3.}
Для нелинейной дроб\-но-ра\-ци\-о\-наль\-ной функции

\noindent
\begin{equation*}
\vrp (Y_t, t) = \fr{a}{(b+Y_t)^2} %\label{e5.9-sin}
\end{equation*}
имеем

\vspace*{-3pt}

\noindent
\begin{gather*}
\vrp_0 (m_t, D_t, t) =a b^{-2} \lk 1+ \chi (m_t, D_t, t)\rk\,; %\label{e5.10-sin}
\\
k_1^\vrp (m_t, D_t, t) =  a b^{-2}\fr{\prt \chi (m_t, D_t, t)}{\prt m_t}\,. %\label{e5.11-sin}
\end{gather*}
Здесь

\vspace*{-3pt}

\noindent
\begin{multline*}
\chi (m_t, D_t, t) ={}\\
{}=\sss_{n=1}^\infty \sss_{l=0}^{E(n/2)}
\fr{(-1)^n (n+1) n!}{(n-2l)! (2l)!}\, b^{-n} m_t^n \left( \fr{D_t}{ 2 m_t^2}
\right)^l, %\label{e5.12-sin}
\end{multline*}
где  $E(n/2)$~--- целая часть~$n/2$; $a\hm=a(t)$; $b\hm= b(t)$.

\vspace*{-6pt}

\section{Сложные интегральные нелинейности}

\vspace*{-2pt}

Пусть сначала векторно-матричная нелинейность имеет эредитарный характер, т.\,е.\
\begin{equation}
\underline{\vrp} (Y_t, t) =\iii_{t_0}^t A(t,\tau) \vrp (Y(\tau), \tau) \,d\tau\,.
\label{e6.1-sin}
\end{equation}
Тогда, как показано в~\cite{6-sin, 5-sin, 7-sin}, следует соответст\-ву\-ющие
интегродифференциальные соотношения путем введения  инструментальных
переменных привести к дифференциальным соотношениям.  Для
дифференцируемых функций~$\vrp$ и асимптотически устойчивых ядер
$A(t,\tau)$ зависимость~(\ref{e3.5-sin}) имеет следующий дифференциальный вид:
\begin{equation*}
F^A (t, D) \underline{\vrp} (Y_t, t) = H^A (t, D) \vrp (Y_t, t)\,. %\label{e6.2-sin}
\end{equation*}
Здесь $F^A (t, D)$ и  $H^A (t, D)$~--- линейные дифференциальные операторы $(D\hm= d/dt)$.

Для недифференцируемых функций~$\vrp$ и асимптотически устойчивых
сингулярных ядер~(\ref{e3.5-sin}) используются соотношения:
\begin{equation*}
\underline{\vrp} (Y_t, t) = A^+ Z\,,\enskip
\dot Z = A^- \vrp\,,\enskip
Z(t_0)=0\,. %\label{e6.3-sin}
\end{equation*}

Многочисленные примеры аналитического моделирования ЭСтС можно найти
в~[1--3, 5, 7, 10, 11].

Как отмечалось в~\cite{3-sin}, часто наряду с интегральными
нелинейностями~(\ref{e6.1-sin}) рассматривают нелинейности вида:

\columnbreak

\noindent
\begin{equation*}
Z_s =\sss_{\rho=1}^R \mathcal{A}_\rho \vrp_\rho (Y_{t_1}\tr Y_{t_r})\,, %\label{e6.2-sin}
\end{equation*}
где $\mathcal{A}_1 \tr \mathcal{A}_R$~--- произвольные линейные операторы,
действующие над функциями~$r$ переменных  $t_1\tr t_r$; $\vrp_\rho
\hm=\vrp_\rho (Y_{t_1} \tr Y_{t_r})$~--- линейные функции отмеченных
переменных. Такие нелинейности называются приводимыми к линейным.
Важным частным случаем~(\ref{e6.1-sin}) являются интегральные нелинейности вида:

\noindent
\begin{gather}
Z_s =\iii_T \vrp (Y_t, t, s)\, dt\,; \notag%\label{e6.3-sin}
\\
Z_s =\!\iii_T \!\cdots\!\iii_T\! \vrp (Y_{t_1}\tr Y_{t_r}; t_1\tr t_r, s)\,dt_1
\ldots dt_r,\notag %\label{e6.4-sin}
\end{gather}
В этом случае используется МСЛ по совокупности переменных  $Y_{t_1} \tr Y_{t_r}$.

\vspace*{-9pt}

\section{Заключение}

\vspace*{-2pt}

Разработаны методы и алгоритмы МНА и МСЛ для ДСтС и ЭСтС,
приводимых к ДСтС со сложными конечными, дроб\-но-ра\-ци\-о\-наль\-ны\-ми,
иррациональными, а также дифференциальными и интегральными нелинейностями.
Приведены примеры.

Результаты допускают обобщение на случай ДСтС и ЭСтС со
стохастическими нелинейностями, заданными каноническими разложениями и
интегральными каноническими  представлениями~\cite{1-sin, 3-sin, 11-sin}.

\vspace*{-9pt}

{\small\frenchspacing
 {%\baselineskip=10.8pt
 \addcontentsline{toc}{section}{References}
 \begin{thebibliography}{99}

 \vspace*{-2pt}

\bibitem{1-sin}
\Au{Синицын И.\,Н.,  Синицын~В.\,И.}
Лекции по нормальной и эллипсоидальной аппроксимации распределений в
стохастических сис\-те\-мах.~--- М.: ТОРУС ПРЕСС, 2013. 488~с.

\bibitem{6-sin} %2
\Au{Синицын И.\,Н. }
Stochastic hereditary control systems~// Проблемы управления и
теории информации, 1986. Т.~15. №\,4. С.~287--298.

\bibitem{2-sin} %3
\Au{Пугачев В.\,С., Синицын~И.\,Н.}
Стохастические дифференциальные сис\-те\-мы. Анализ и фильтрация.~--- М.:
Наука,  1990.  632~с. [Англ. пер.
 Stochastic differential systems.
Analysis and filtering.~--- Chichester, New York: Jonh Wiley, 1987.
549~p.].

\bibitem{5-sin} %4
\Au{Синицын И.\,Н. }
Конечномерные распределения процессов в стохастических интегральных
и интегродифференциальных системах~// Preprints of the 2nd IFAC
Symposium on Stochastic Control.~--- Vilnius: Pergamon Press,
1987.  Vol.~1. P.~144--153.

\bibitem{3-sin} %5
\Au{Пугачев В.\,С., Синицын~И.\,Н.}
Теория стохастических систем.~--- М.: Логос, 2000; 2004. 1000~с.
[Англ. пер.\linebreak\vspace*{-12pt}

\pagebreak

\noindent Stochastic systems. Theory and  applications.~---
Singapore: World Scientific, 2001. 908~p.].

\bibitem{4-sin} %6
\Au{Синицын И.\,Н.}
Параметрическое статистическое и аналитическое моделирование распределений
в нелинейных стохастических сис\-те\-мах на многообразиях~//
Информатика и её применения, 2013. Т.~7. Вып.~2. С.~4--16.

\bibitem{7-sin} %7
\Au{Синицын И.\,Н. }
Анализ и моделирование распределений в эредитарных стохастических
сис\-те\-мах~// Информатика и её применения, 2014. Т.~8. Вып.~1.\linebreak
С.~2--11.



\bibitem{9-sin} %8
\Au{Градштейн И.\,С., Рыжик~И.\,М.}
Таблицы интегралов, сумм, рядов и произведений.~--- М.: ГИФМЛ, 1963. 1100~с.

\bibitem{8-sin} %9
\Au{Попов Б.\,А., Теслер~Г.\,С. }
Вычисление функций на ЭВМ: Справочник.~--- Киев: Наукова Думка, 1984. 599~с.


\bibitem{11-sin} %10
\Au{Синицын И.\,Н.}
Канонические представления случайных функций и их применение в
задачах компьютерной поддержки научных исследований.~--- М.: ТОРУС
ПРЕСС, 2009. 768~с.

\bibitem{10-sin} %11
\Au{Синицын И.\,Н., Синицын~В.\,И., Корепанов~Э.\,Р., Белоусов~В.\,В.,
Сергеев~И.\,В., Басилашвили~Д.\,А.}
Опыт моделирования эредитарных стохастических сис\-тем~//
Кибернетика и высокие технологии XXI века: Сб. докл.  XIII Междунар.
науч.-технич. конф.~--- Воронеж: Саквоее, 2012. Т.~2. C.~346--357.

 \end{thebibliography}

 }
 }

\end{multicols}

\vspace*{-9pt}

\hfill{\small\textit{Поступила в редакцию 05.05.14}}

%\newpage

\vspace*{12pt}

\hrule

\vspace*{2pt}

\hrule

\vspace*{12pt}

\def\tit{ANALYTICAL MODELING OF NORMAL PROCESSES
 IN~STOCHASTIC SYSTEMS WITH~COMPLEX NONLINEARITIES}

\def\titkol{Analytical modeling of normal processes
 in~stochastic systems with~complex nonlinearities}

\def\aut{I.\,N.~Sinitsyn and V.\,I.~Sinitsyn}

\def\autkol{I.\,N.~Sinitsyn and V.\,I.~Sinitsyn}

\titel{\tit}{\aut}{\autkol}{\titkol}

\vspace*{-9pt}

\noindent
Institute of Informatics Problems, Russian Academy of Sciences,
44-2 Vavilov Str., Moscow 119333, Russian Federation


\def\leftfootline{\small{\textbf{\thepage}
\hfill INFORMATIKA I EE PRIMENENIYA~--- INFORMATICS AND
APPLICATIONS\ \ \ 2014\ \ \ volume~8\ \ \ issue\ 3}
}%
 \def\rightfootline{\small{INFORMATIKA I EE PRIMENENIYA~---
INFORMATICS AND APPLICATIONS\ \ \ 2014\ \ \ volume~8\ \ \ issue\ 3
\hfill \textbf{\thepage}}}

\vspace*{6pt}

\Abste{Differential stochastic systems (DStS) with Wiener and Poisson
noises and complex finite, differential, and  integral nonlinearities and
hereditary StS reducible to DStS are considered. Equations and algorithms
of analytical modeling based on the normal approximation method (NAM) and the
statistical linearization method (SLM) are given. The case of complex
continuous and discontinuous nonlinearities of scalar and vector arguments
is considered. Special attention is paid to NAM (SLM) typical integrals
for finite rational and irrational nonlinear and integral scalar and vector
nonlinear functions. The general case of integral nonlinearities reducible to
linear is considered. Test examples are given.}

\KWE{analytical modeling;
complex finite differential and integral nonlinearities;
complex irrational nonlinerarites
differential stochastic system with Wiener and Poisson noises;
method of normal approximation;
method of statistical linearization;
hereditary stochastic systems reducible to differential}

\DOI{10.14357/19922264140302}

  \begin{multicols}{2}

\renewcommand{\bibname}{\protect\rmfamily References}
%\renewcommand{\bibname}{\large\protect\rm References}

{\small\frenchspacing
 {%\baselineskip=10.8pt
 \addcontentsline{toc}{section}{References}
 \begin{thebibliography}{99}



\bibitem{1-sin-1}
\Aue{Sinitsyn, I.\,N., and  V.\,I.~Sinitsyn}.  2013.
Lektsii po normal'noy i ellipsoidal'noy approksimatsii raspredeleniy
v stokhasticheskikh sistemakh [Lectures on normal and ellipsoidal
approximation of distributions in stochastic systems].
Moscow: TORUS PRESS. 488~p.

\bibitem{6-sin-1} %2
\Aue{Sinitsyn, I.\,N.}  1986.
{Stochastic hereditary control systems}.
\textit{Problems Control Inform. Theory} 15(4):287--298.

\bibitem{2-sin-1} %3
\Aue{Pugachev, V.\,S., and  I.\,N.~Sinitsyn}.  1987.
\textit{Stochastic differential systems. Analysis and filtering.}
Chichester, New York: Jonh Wiley. 549~p.

\bibitem{5-sin-1} %4
\Aue{Sinitsyn, I.\,N.}  1987.
Konechnomernye raspredeleniya protsessov v stokhasticheskikh integral'nykh
i in\-teg\-ro\-dif\-fe\-ren\-tsial'nykh sistemakh [Finite dimensional distributions
of processes in stochastic integral and integrodifferential systems].
\textit{2nd  Symposium (International) IFAC on Stochastic Control
Preprints}. Vilnius: Pergamon Press. 1:144--153.

\bibitem{3-sin-1} %5
\Aue{Pugachev, V.\,S., and I.\,N.~Sinitsyn}. 2001.
\textit{Stochastic systems. Theory and  applications}.
Singapore: World Scientific. 908~p.

\bibitem{4-sin-1} %6
\Aue{Sinitsyn, I.\,N.}  2013.
Parametricheskoe statisticheskoe i analiticheskoe modelirovanie
raspredeleniy v nelineynykh stokhasticheskikh sistemakh na mnogoobraziyakh
[Parametric statistical and analytical modeling of distributions in
stochastic systems on manifolds].
\textit{Informatika i ee Primeneniya}~--- \textit{Inform. Appl.} 7(2):4--16.


\bibitem{7-sin-1} %7
\Aue{Sinitsyn, I.\,N.}  2014.
Analiz i modelirovanie raspredeleniy v ereditarnykh stokhasticheskikh sistemakh
[Analysis and modeling of distributions in hereditary stochastic systems].
\textit{Informatika i ee Primeneniya}~--- \textit{Inform. Appl.} 8(1):2--11.

\bibitem{9-sin-1} %8
\Aue{Gradshteyn, I.\,S., and I.\,M.~Ryzhik}.  1963.
\textit{Tablitsy integralov, summ, ryadov i proizvedeniy}
[Tables of integrals, sums, series, and products]. Moscow:  GIFML.   1100~p.

\pagebreak

\bibitem{8-sin-1} %9
\Aue{Popov, B.\,A., and G.\,S.~Tesler}.  1984.
\textit{Vychislenie funktsiy na EVM}. Spravochnik [Computing of functions].
Kiev: Naukova Dumka.  599~p.


\bibitem{11-sin-1} %10
\Au{Sinitsyn, I.\,N.} 2009.
\textit{Kanonicheskie predstavleniya sluchaynykh funktsiy i ikh primenenie v
zadachakh komp'yuternoy podderzhki nauchnykh issledovaniy}
[Canonical expansions of random functions and its application to
scientific computer-aided support]. Moscow: TORUS PRESS. 768~p.

\bibitem{10-sin-1} %11
\Aue{Sinitsyn, I.\,N., V.\,I.~Sinitsyn, E.\,R.~Korepanov,
V.\,V.~Belousov, I.\,V.~Sergeev, and D.\,A.~Basilashvili}.
2012. Opyt modelirovaniya ereditarnykh stokhasticheskikh sistem
[Experience of modeling in hereditary stochastic systems].
\textit{Kibernetika i Vysokie Tekhnologii XXI~Veka:
Sbornik dokladov  XIII Mezhdunar. nauch.-tekhnich. konf.}
[Cybernatics ans High Technologies of the XXI Century: Materials of
XIII  Scientific and Technological Conference (International)].
Voronezh: Sakvoee. 2:346--357.

\end{thebibliography}

 }
 }

\end{multicols}

\vspace*{-6pt}

\hfill{\small\textit{Received May 05, 2014}}

\vspace*{-18pt}

\Contr

\noindent
\textbf{Sinitsyn Igor N.} (b.\ 1940)~---
Doctor of Science in technology, professor, Honored scientist of RF, Head of Department, Institute of
Informatics Problems, Russian Academy of Sciences,
44-2 Vavilov Str., Moscow 119333, Russian
Federation; sinitsin@dol.ru

\vspace*{3pt}

\noindent
\textbf{Sinitsyn Vladimir I.} (b.\ 1968)~--- Doctor of Science in physics
and mathematics, associate professor, Head of Department, Institute of
Information Problems, Russian Academy of Sciences,
44-2 Vavilov Str., Moscow 119333, Russian Federation; VSinitsin@ipiran.ru




\label{end\stat}

\renewcommand{\bibname}{\protect\rm Литература} %8
\include{chichagov} %9
\def\stat{kudr}

\def\tit{ПРИБЛИЖЕННЫЕ МЕТОДЫ РЕШЕНИЯ ЗАДАЧИ ДИАГНОСТИКИ ПЛОСКИМ 
ЗОНДОМ СИЛЬНОИОНИЗОВАННОЙ ПЛАЗМЫ С~УЧЕТОМ КУЛОНОВСКИХ 
СТОЛКНОВЕНИЙ}

\def\titkol{Приближенные методы решения задачи диагностики плоским 
зондом сильноионизованной плазмы} %с~учетом Кулоновских  столкновений}

\def\autkol{И.\,А.~Кудрявцева, А.\,В.~Пантелеев}
\def\aut{И.\,А.~Кудрявцева$^1$, А.\,В.~Пантелеев$^2$}

\titel{\tit}{\aut}{\autkol}{\titkol}

%{\renewcommand{\thefootnote}{\fnsymbol{footnote}}\footnotetext[1]
%{Работа поддержана Российским фондом фундаментальных исследований
%(проекты 11-01-00515а и 11-07-00112а), а также Министерством
%образования и науки РФ в рамках ФЦП <<Научные и
%научно-педагогические кадры инновационной России на 2009--2013~годы>>.}}


\renewcommand{\thefootnote}{\arabic{footnote}}
\footnotetext[1]{Московский авиационный институт, irina.home.mail@mail.ru}
\footnotetext[2]{Московский авиационный институт, avpanteleev@inbox.ru}

\vspace*{-2pt}

\Abst{Сформирована математическая модель, описывающая динамику сильноионизованной 
плазмы с учетом столкновений заряженных частиц вблизи плоского зонда. Модель включает уравнение 
Фоккера--Планка и уравнение Пуассона. Предложено два подхода к решению задачи: на основе метода 
статистических испытаний Мон\-те-Кар\-ло и на основе композиции метода крупных частиц и метода 
расщепления.} 

\vspace*{-2pt}

\KW{телекоммуникационные системы; метод Монте-Карло; метод крупных частиц; метод 
расщепления; зонд; уравнение Фоккера--Планка; уравнение Пуассона} 

\vspace*{-4pt}

 \vskip 8pt plus 9pt minus 6pt

      \thispagestyle{headings}

      \begin{multicols}{2}
      
            \label{st\stat}

\section{Введение}

В настоящее время в области телекоммуникаций все более востребованными становятся 
информационные технологии, основанные на использовании математических моделей и численных 
методов физики плазмы. Поэтому особенно актуальным является решение разнообразных задач анализа 
поведения плазмы, включающих в себя формирование новых моделей и методов их исследования. 
Помимо этого, в разработке телекоммуникационного оборудования эффективно используются 
собственно физические свойства плазмы. В~частности, изготовлена антенна, работа которой основана 
на газовом разряде низкотемпературной плазмы~[1], интенсивно ведутся разработки по созданию и 
усовершенствованию источников бесперебойного питания на основе плазменных элементов~[2, 3]. 
      
      Одним из наиболее перспективных направлений для построения систем оптической 
беспроводной связи является использование лазеров~\cite{4-k, 5-k}. В~этой связи большое внимание 
уделяется использованию плазмы при разработке импульсных сильноточных коммутаторов~\cite{6-k}, 
так как практическое применение подобных разработок требует повышения уровня надежности и 
быстродействия лазерных систем.
      
      Исследования низкотемпературной плазмы также связаны с разработками в области дальней 
космической связи, так как моделирование процессов взаимодействия заряженного тела с верхними 
слоями атмосферы позволяет предлагать способы улучшения существующих систем радиосвязи с 
космическими летательными аппаратами~\cite{7-k}. 
      
      Наряду с этим актуальными также являются задачи диагностики плазмы, поскольку перспективы 
ее использования в области телекоммуникаций после более полного изучения физических свойств 
могут значительно расшириться. 

Для диагностики плазмы применяют зондовые методы исследования~[8--11]. Эти методы относятся к 
классу контактных методов; как следствие, возникает сложность в исследовании пристеночной области 
вблизи зонда, которая характеризуется достаточно сложным распределением потенциала и функциями 
распределения, отличными от максвелловских. 

Данная работа посвящена исследованию переходного режима обтекания заряженного тела плазмой. Для 
переходного режима выполняется следующее условие: длина свободного пробега иона до столкновения 
с нейтральным атомом или другим ионом невелика по сравнению с характерными размерами тела. 
В~этом случае возникает необходимость учета столкновений заряженных частиц с нейтральными 
атомами и кулоновских столкновений. В~работах~\cite{10-k, 11-k} подробно рассмотрена модель с 
учетом столкновений заряженных частиц с нейтральными атомами. В~настоящей статье представлена 
теоретическая модель, описывающая влияния ион-ионных и ион-элек\-т\-рон\-ных столкновений на 
измеряемые характеристики плазмы, что ранее детально не исследовалось.
      
      В~рамках данной работы предлагается модель, описывающая динамику сильноионизованной 
плазмы с учетом кулоновских столкновений. Эта модель учитывает такие процессы взаимодействия, 
как перенос частиц и столкновения между заряженными частицами типа <<ион--ион>> и 
      <<ион--электрон>> под влиянием макроскопического электрического поля. Перечисленные 
процессы описываются самосогласованной системой уравнений, включающей уравнение 
      Фок\-ке\-ра--План\-ка и уравнение Пуассона~[12].
      
      Вычислительная модель задачи строится на основе двух методов: метода статистических 
испытаний Мон\-те-Кар\-ло и композиции метода крупных частиц и метода расщепления. Приведены 
результаты численного моделирования, полученные с использованием вышеперечисленных методов.

\vspace*{-4pt}

\section{Постановка задачи}

\vspace*{-2pt}

Рассматривается следующая физическая постановка зондовой задачи~[11]. В~невозмущенную 
бесконечно протяженную плазму, состоящую из электронов и однозарядных ионов, внесена большая\linebreak 
заряженная до потенциала $\varphi_p$ плоскость. Плоскость, расположенная поперек потока плазмы, 
является идеально поглощающей для электронов. Ионы при ударе о плоскость нейтрализуются. 
Предполагается, что частицы в плазме движутся под действием внешнего электрического поля, 
магнитное поле отсутствует. Концентрации ионов $n_{i\infty}$ и электронов $n_{e\infty}$, а также 
температуры данных час\-тиц~$T_{i\infty}$ 
и~$T_{e\infty}$ в невозмущенной плазме заданы. За начальные 
функции распределения обоих типов час\-тиц принимаются функции распределения Максвелла. 
      
      Требуется с учетом столкновений между заряженными частицами найти напряженность 
самосогласованного электрического поля $\vec{E}(\vec{r},t)$, функции распределения однозарядных 
ионов $f_i(\vec{r}, \vec{v}, t)$ и электронов $f_e(\vec{r}, \vec{v}, t)$, 
а также их моменты (плотности 
токов ионов и электронов  $j_i(\vec{r},t)\hm
=q\int f_i(\vec{r}, \vec{v}, t)\vec{v}\,d\vec{v}$, $j_e(\vec{r},t) 
\hm={\sf e}\int f_e(\vec{r},\vec{v},t)\vec{v}\,d\vec{v}$, где $q=Z_i{\sf e}$, $Z_i=1$~--- заряд иона, ${\sf 
e}$~--- заряд электрона; концентрации ионов и электронов $n_i(\vec{r},t)\hm=\int 
f_i(\vec{r},\vec{v},t)\,d\vec{v}$, $n_e(\vec{r},t)\hm=\int f_e(\vec{r},\vec{v}, t)\,d\vec{v}$). 
Поведение частиц во 
времени~$t$ характеризуется ра\-ди\-ус-век\-то\-ром~$\vec{r}$ и вектором скорости~$\vec{v}$.
      
      Математическая модель, соответствующая данной физической постановке задачи, имеет 
вид~\cite{11-k, 13-k}:

\noindent
      \begin{equation}
      \left.
      \begin{array}{c}
      \fr{\partial f_\alpha (\vec{r},\vec{v},t)}{\partial t}+
      \vec{v}\fr{\partial f_\alpha (\vec{r},\vec{v},t)}{ 
\partial \vec{r}}+
\fr{\vec{F}_\alpha(\vec{r},t)}{m_\alpha}\times{}\\[4pt]
{}\times\fr{\partial f_\alpha(\vec{r},\vec{v},t)}{ \partial 
\vec{v}}=
\left(\fr{\partial f_\alpha(\vec{r},\vec{v},t)}{ \partial t}\right)_{\mathrm{с}}+S_\alpha 
(\vec{r},\vec{v},t)\,;\\[6pt]
      \Delta\varphi(\vec{r},t)=-\fr{{\sf e}}{\varepsilon_0}\left( n_i(\vec{r},t)-n_e(\vec{r},t)\right)\,;\\[6pt]
      \vec{E}(\vec{r},t)=-\nabla \varphi(\vec{r},t)\,.
      \end{array}\!\!
      \right\}\!\!
      \label{e1-k}
      \end{equation}
Здесь первое уравнение~--- уравнение Фок\-ке\-ра--План\-ка для частиц сорта~$\alpha$ ($\alpha=i,e$), 
второе~--- уравнение Пуассона для самосогласованного электрического поля; 
$f_\alpha(\vec{r},\vec{v},t)$~--- функция\linebreak
распределения час\-тиц сорта~$\alpha$; $(\partial 
f_\alpha(\vec{r},\vec{v},t)/\partial t)_{\mathrm{с}}$~--- 
оператор столкновений Фок\-ке\-ра--План\-ка; 
функция~$S_\alpha(\vec{r},\vec{v},t)$ описывает источники или стоки\linebreak
 час\-тиц; 
$\vec{F}_\alpha(\vec{r},t)=q_\alpha\vec{E}(\vec{r},t)$, где $\vec{E}(\vec{r},t)$~--- напряженность 
самосогласованного электрического поля, 
$$
q_\alpha =
\begin{cases}
-{\sf e}\,, & \alpha=e\,,\\
{\sf e}\,, & \alpha=i\,;
\end{cases}
$$
$\varphi(\vec{r},t)$~--- потенциал самосогласованного электрического поля; $n_\alpha(\vec{r},t)$ ($\alpha 
\hm=i,e$)~--- концентрация частиц сорта~$\alpha$; $m_\alpha$~--- масса частицы сорта~$\alpha$; 
$\varepsilon_0$~--- электрическая постоянная. 

Оператор столкновений Фок\-ке\-ра--План\-ка имеет вид~\cite{13-k, 14-k}
\begin{multline*}
\fr{1}{\Gamma_\alpha}\left( \fr{\partial f_\alpha}{\partial t}\right)_{\mathrm{с}} 
=\fr{1}{2}\,\nabla_v\nabla_v:\left(f_\alpha\nabla_v\nabla_vg_\alpha(\vec{r},\vec{v},t)\right)-{}\\
{}-
\nabla_v\cdot\left(f_\alpha\nabla_v h_\alpha\right)\,,
\end{multline*}
где $\nabla_v\nabla_v g_\alpha(\vec{r},\vec{v},t)$~--- ковариантная тензорная производная второго ранга, 
знак двоеточия ($:$) обозначает операцию двойного суммирования:
\begin{gather*}
\Gamma_\alpha=\fr{Z_\alpha^4 {\sf e}^4}{4\pi \varepsilon_0^2 m^2_\alpha}\,\ln D_\alpha\,;
\\
D_\alpha =\fr{12\pi\varepsilon_0 kT_{\alpha\infty}}{Z_\alpha^2 {\sf e}^2}\left( \fr{\varepsilon_0 k 
T_{e\infty}}{n_{e\infty} {\sf e}^2}\right)^{1/2}\,;\\
g_\alpha (\vec{r},\vec{v},t)=\sum\limits_{b=i,e}\left( \fr{Z_b}{Z_\alpha}\right) \int f_b 
(\vec{r},{\vec{v}}^{\,\prime},t)\left\vert \vec{v}-{\vec{v}}^{\,\prime}\right\vert\,d\vec{v}^{\,\prime}\,;\\
h_\alpha (\vec{r},\vec{v},t)=\sum\limits_{b=i,e} \fr{m_\alpha+m_b}{m_b} 
\left(\fr{Z_b}{Z_\alpha}\right)
\int
\fr{f_b(\vec{r},{\vec{v}}^{\,\prime}, t)}{\vert \vec{v}-{\vec{v}}^{\,\prime}\vert}
\,d{\vec{v}}^{\,\prime}\,;\\
Z_\alpha =1\,, \quad \alpha=i,e\,.
\end{gather*}
 
К системе уравнений~(\ref{e1-k}) необходимо добавить начальные и краевые условия:
\begin{equation}
\!\left.
\begin{array}{rrl}
t=0:\ & f_\alpha(\vec{r},\vec{v},0)&=f_\alpha^{\mathrm{maksv}}\,,\enskip \alpha=i,e;\\[9pt]
\vec{r}\in \Omega_p:\ & f_\alpha(\vec{r},\vec{v},t)\big\vert_{\vec{r}\in\Omega_p}&=0\,,\enskip \alpha=i,e\,;\\[9pt]
&\varphi(\vec{r},t)\big\vert_{\vec{r}\in\Omega_p}&=\varphi_p\,;\\[9pt]
\vec{r}\in\Omega_\infty:\ & 
f_\alpha(\vec{r},\vec{v},t)\big\vert_{\vec{r}\in\Omega_\infty}&= %{}\\[9pt]
f_\alpha^{\mathrm{maksv}}\,,\enskip \alpha=i,e\,;\\[9pt]
&\varphi(\vec{r},t)\big\vert_{\vec{r}\in\Omega_\infty}&=0\,,
\end{array}\!\!
\right\}\!\!\!\!
\label{e2-k}
\end{equation}
    где 
    
    \noindent
    \begin{multline*}
    f_\alpha^{\mathrm{maksv}}=n_{\alpha\infty}\left(\fr{m_\alpha}{2k\pi T_{\alpha\infty}}\right)^{3/2}\times{}\\
    {}\times
    \exp\left( -
\fr{m_\alpha}{2kT_{\alpha\infty}}\left\vert\vec{v}-\vec{v}_\infty\right\vert^2\right)\,,
\enskip \alpha=i, e\,;
\end{multline*} 
$\Omega_p$ и $\Omega_\infty$~--- множество радиус-векторов час\-тиц, концы которых принадлежат плоскости зонда и 
границе возмущенной зоны соответственно.

Для решения поставленной задачи введем декартову систему координат таким образом, чтобы 
заряженная плоскость совпала с плоскостью~$0xz$. Тогда положение частицы в пространстве будет 
определяться координатами $x,y,z$, а скорость~--- координатами $v_x, v_y, v_z$. В~силу того что 
плоскость является бесконечно большой в сравнении с характерным размером задачи, функции 
распределения частиц будут зависеть только от переменных $y, v_y, t$.

Поставленную задачу предлагается решать независимо двумя методами. Первый метод основывается на 
методе статистических испытаний Мон\-те-Кар\-ло, второй метод является композицией метода 
расщепления и метода крупных частиц.

\section{Применение метода Монте-Карло}

Запишем самосогласованную систему уравнений~(\ref{e1-k}) и~(\ref{e2-k}) в декартовой системе 
координат с учетом сделанных предположений:
\begin{equation}
\left.
\begin{array}{l}
\fr{\partial f_\alpha}{\partial t}+
v_y\fr{\partial f_\alpha}{\partial y}+\fr{F_y^\alpha}{m_\alpha}\,\fr{\partial 
f_\alpha}{\partial v_y}=\fr{1}{2}\,\fr{\partial^2 }{\partial [v_y]^2}\times{}\\
{}\times \left( 
f_\alpha\fr{\partial^2 g_\alpha  }{\partial [v_y]^2}\right) -
\fr{\partial}{\partial v_y}\left( f_\alpha\fr{\partial h_\alpha}{\partial v_y}\right)\,,
\enskip \alpha=i,e\,;\\[6pt]
    \fr{\partial^2\varphi}{\partial y^2} =-\fr{{\sf e}}{\varepsilon_0}\left(n_i-n_e\right)\,;
    \enskip E_y=-
\fr{\partial\varphi}{\partial y}\,;\\[6pt]
\hspace*{3.1mm}    t=0:\  \hspace*{2.6mm}f_\alpha(y,v_y,0)=f_\alpha^{\mathrm{maksv}}\,,\ \alpha=i,e\,;\\[9pt]
\hspace*{2.9mm} y=0:\ \hspace*{2.8mm}f_\alpha(0,v_y,t)=0\,,\ \alpha=i,e\,;\\[9pt]
\hspace*{24.3mm}\varphi(0,t)=\varphi_p\,;\\[9pt]
y=y_\infty:\ f_\alpha(y_\infty, v_y, t)=f_\alpha^{\mathrm{maksv}}\,,\ \alpha=i,e\,;\\[9pt]
\hspace*{21.5mm}\varphi(y_\infty, t)=0\,.
\end{array}
\right \}
\label{e3-k}
\end{equation}

В полученной системе уравнений~(\ref{e3-k}) перейдем к безразмерным величинам, применив 
соотношение $X=M_X \hat{X}$, где $M_X$~--- масштаб размерной величины~$X$, $\hat{X}$~--- 
безразмерная величина~$X$. В~качестве используемых масштабов были взяты следующие: радиус 
Дебая, скорость теплового движения частиц, концентрация частиц в невозмущенной плазме, потенциал, 
возникающий при разделении зарядов в дебаевской сфере, и производные от них величины.

Система безразмерных уравнений имеет следующий вид:
%\noindent
\begin{equation}
\left.
\begin{array}{l}
\fr{\partial 
\hat{f}_\alpha}{\partial\hat{t}}+A_\alpha\fr{\partial\hat{f}_\alpha}{\partial\hat{y}}+
B_\alpha\hat{E}_y\fr{\partial\hat{f}_\alpha}{\partial \hat{v}_y}={}\\
\!{}=
\fr{\partial^2}{\partial[\hat{v}_y]^2}\left(D_\alpha 
\hat{f}_\alpha\right)-\fr{\partial}{\partial\hat{v}_y}\left(K_\alpha \hat{f}_\alpha\right),\enskip 
\alpha=i,e;\\[9pt]
\fr{\partial^2\hat{\varphi}}{\partial\hat{y}^2}=-\left(\hat{n}_i-\hat{n}_e\right)\,;\enskip \hat{e}_y=-
\fr{\partial\hat\varphi}{\partial\hat{y}}\,;\\[9pt]
\hspace*{3.1mm}\hat{t}=0:\ \hspace*{2.6mm}\hat{f}_\alpha(\hat{y},\hat{v}_y,0)=\hat{f}_\alpha^{\mathrm{maksv}}\,,\enskip \alpha-i,e\,;\\[9pt]
\hspace*{2.9mm}\hat{y}=0:\ \hspace*{2.8mm}\hat{f}_\alpha(0,\hat{v}_y,\hat{t})=0\,,\enskip \alpha=i,e\,;\\[9pt]
\hspace*{24.3mm}\hat\varphi(0,\hat{t})=\hat{\varphi}_p\,;\\[9pt]
\hat{y}=\hat{y}_\infty:\ \hat{f}_\alpha(\hat{y}_\infty, \hat{v}_y, \hat{t})=\hat{f}^{\mathrm{maksv}}_\alpha\,,\enskip 
\alpha=i,e\,;\\[9pt]
\hspace*{21.5mm}\hat\varphi(\hat{y}_\infty,\hat{t})=0\,.
\end{array}
\right\}
\label{e4-k}
\end{equation}
Здесь 

\vspace*{-2pt}

\noindent
\begin{gather*}
A_\alpha=\sqrt{\delta_\alpha }\,\hat{v}_y\,;\enskip 
B_\alpha=\sqrt{\delta_\alpha}\,\fr{z_\alpha}{2\varepsilon_\alpha}\,;\\
\delta_\alpha=\fr{\varepsilon_\alpha}{\mu_\alpha}\,;\enskip 
\varepsilon_\alpha=\fr{T_{\alpha\infty}}{T_{i\infty}}\,;\\
\mu_\alpha=\fr{m_\alpha}{m_i}\,;\enskip 
D_\alpha=A_g^\alpha\fr{\partial^2\hat{g}_\alpha}{\partial  [\hat{v}_y]^2}\,;\\
K_\alpha=A_h^\alpha \fr{\partial \hat{h}_\alpha}{\partial \hat{v}_y}\,,\enskip \alpha=i,e\,,
\end{gather*}
где $A_g^\alpha$ и $A_h^\alpha$~--- коэффициенты, определяемые характерными параметрами 
задачи~\cite{15-k}.

Поиск решения самосогласованной системы уравнений~(\ref{e4-k}) осуществляется по следующей 
схе-\linebreak ме. Вначале находятся значения напряженности\linebreak
 электрического поля по значениям потенциала, 
полученным из граничной задачи для уравнения Пуассона. Далее, используя найденные значения 
напряженности, решается уравнение Фок\-ке\-ра--План\-ка путем перехода к стохастическому 
дифференциальному уравнению (СДУ) Ито:

\noindent
\begin{multline*}
d\Theta_\alpha(\hat{t}) = a_\alpha \left(\hat{t},\Theta_\alpha(\hat{t})\right)+{}\\
{}+\sigma\left(
\hat{t},\Theta_\alpha(\hat{t})\right)\,dW(\hat{t})\,,\quad \alpha=i,e\,,
%\label{e5-k}
\end{multline*}
где 

\noindent
\begin{align*}
\Theta_\alpha(\hat{t})&=\begin{bmatrix}
\hat{y}(\hat{t})\\ \hat{v}_y(\hat{t})
\end{bmatrix}\,;\\
a_\alpha\left(\hat{t},\Theta_\alpha(\hat{t})\right)&=\begin{bmatrix}
-A_\alpha\\ -K_\alpha -B_\alpha \hat{E}_y
\end{bmatrix}\,;\\
\sigma_\alpha\left(\hat{t},\Theta_\alpha(\hat{t})\right)\sigma_\alpha^{\mathrm{T}}\left( 
\hat{t},\Theta_\alpha(\hat{t})\right)&=D_\alpha\,,\enskip \alpha=i,e\,;
\end{align*} 
$W(\hat{t})$~--- стандартный винеровский случайный процесс.
\pagebreak

Для нахождения значений вектора состояния~$\Theta_\alpha(\hat{t})$ применим явную разностную 
схему стохастического метода Эйлера~\cite{16-k}:
\begin{multline*}
\Theta_\alpha^{n+1}=\Theta_\alpha^n +h_\tau a_\alpha \left( \hat{t}_n, \Theta_\alpha^n\right)+\sigma_\alpha 
\left( \hat{t}_n, \Theta_\alpha^n\right)\Delta W_n\,,\\ 
n=0,\ldots , N\,,\ \alpha=i,e\,,
%\label{e6-k}
\end{multline*}
где $\Theta_\alpha^n$, $n=0,\ldots , N$,~--- приближенное значение вектора 
состояния~$\Theta_\alpha(\hat{t})$, $\alpha=i,e$, в момент времени $\hat{t}\hm=\hat{t}_n$, 
$\hat{t}_n\hm=n h_\tau$, $n=0,\ldots , N$; $h_\tau$~--- достаточно малый шаг интегрирования; $\Delta 
W_n$, $n=0,\ldots ,N$,~--- величина приращения винеровского процесса~$W(\hat{t})$ на отрезке $\left[ 
\hat{t}_n,\,\hat{t}_{n+1}\right]$, по определению независимая от~$\Theta_\alpha^0$, 
$\Delta W_0,\ldots , 
\Delta W_{n-1}$: $\Delta W_n\hm=W(\hat{t}_{n-1})\hm-W(\hat{t}_n)$; $\Delta W_n\hm\sim N(0,\,h_\tau)$, 
т.\,е.\ $\Delta W_n$ представляют собой гауссовские случайные величины с нулевыми математическими 
ожиданиями и дисперсиями, равными шагу интегрирования; $\Theta_\alpha^0$~--- значение вектора 
состояния $\Theta_\alpha(\hat{t})$, $\alpha\hm=i,e$, в момент времени $\hat{t}=0$, 
$\Theta_\alpha^0\hm\sim \hat{f}_\alpha^{\mathrm{maksv}}$. 

Частные производные $\partial^2\hat{g}_\alpha/\partial[\hat{v}_y]^2$ и $\partial \hat{h}_\alpha/\partial 
\hat{v}_y$, являющиеся составляющими матрицы $\sigma_\alpha (\hat{t}_n, 
\Theta_\alpha^n)\sigma_\alpha^{\mathrm{T}}(\hat{t}_n,\Theta_\alpha^n)$ и вектора $a_\alpha(\hat{t}_n, 
\Theta_\alpha^n)$ соответственно, аппроксимируются со вторым порядком точности на трехточечном 
шаблоне на основе значений~$\hat{g}_\alpha$ и~$\hat{h}_\alpha$~\cite{17-k}.
      
      В выражения для функций~$\hat{g}_\alpha$ и~$\hat{h}_\alpha$ входят интегралы, которые 
вычисляются методом Мон\-те-Кар\-ло с использованием набора значений скоростной компоненты 
вектора состояния~$\hat{v}_y$, полученных из решения СДУ Ито:
      \begin{equation*}
      \int \hat{f}_\alpha \left\vert \hat{v}_y-
\hat{v}_y^\prime\right\vert\,dv_y^\prime=M\left(\zeta\left(\hat{V}_y\right)\right)\,,
\end{equation*}
где
$$
      \zeta\left(\hat{V}_y\right)=\left\vert \hat{v}_y-\hat{V}_y\right\vert\,,\enskip \hat{V}_y\sim 
\hat{f}_\alpha\,.
  $$
      
      Для вычисления напряженности самосогласованного электрического поля $\hat{E}_y=-
\partial\hat{\varphi}/\partial\hat{y}$, входящей в вектор $a_\alpha(\hat{t}_n, \Theta_\alpha^n)$, необходимо 
аналогично аппроксимировать со вторым порядком точности производную 
$\partial\hat{\varphi}/\partial\hat{y}$ на трехточечном шаблоне с использованием значений 
потенциала~$\hat{\varphi}$~\cite{17-k}. Значения потенциала~$\hat\varphi$ находятся из решения 
уравнения Пуассона. 
      
      Граничную задачу для уравнения Пуассона 
      \begin{align*}
      \fr{\partial^2 \hat\varphi}{\partial \hat{y}^2} & = -\left(\hat{n}_i-\hat{n}_e\right)\,;\\
      \hat{\varphi}\big|_{\hat{y}=0} &=\hat{\varphi}_p\,;\\
      \hat{\varphi}\big|_{\hat{y}_\infty=0} &=0
      \end{align*}
    предлагается решать путем перехода к конечно-разностной системе с последующим ее решением 
методом прогонки~\cite{17-k}:

\noindent
\begin{gather*}
\hat{\varphi}^n_{l-1}+2\hat{\varphi}_l^n+\hat{\varphi}^n_{l+1}=
h_y\hat{\delta}_l^n\,,\enskip l=1,\ldots , 
N_y\,;\\
\hat{\delta}_l^n=-\left( \hat{n}^n_{i,l}-\hat{n}^n_{e,l}\right)\,;\enskip 
\hat{\varphi}_0=\hat{\varphi}_p\,;\enskip \hat{\varphi}_{N_y}=0\,,
\end{gather*}
где $N_y$~--- число шагов по переменной~$\hat{y}$, $h_y$~--- величина шагов разбиения по~$\hat{y}$. 
      
      Концентрации $\hat{n}_\alpha$, $\alpha=i,e$, и плотности токов частиц на зонд~$\hat{f}_\alpha$, 
$\alpha=i,e$, вычисляются согласно описанному выше методу Мон\-те-Карло.

\section{Применение метода расщепления и~метода крупных~частиц}

Решение задачи в данном случае предлагается начать с записи правой части уравнения 
Фок\-ке\-ра--План\-ка в декартовой системе координат в виде:
$$
\mathbf{Q} f_\alpha = \fr{1}{2}\,\fr{\partial^2 f_\alpha}{\partial [v_y]^2}\,\fr{\partial^2 g_\alpha}{\partial 
[v_y]^2}+\fr{\partial f_\alpha}{\partial v_y}\,\fr{\partial C_\alpha}{\partial v_y}+H_\alpha\,,\enskip 
\alpha=i,e\,,
$$  
где 
\begin{align*}
C_\alpha(\vec{r},\vec{v},t)&=
\begin{cases}
\fr{1-\gamma}{Z_i^2}\int\fr{f_e(\vec{r},{\vec{v}}^{\,\prime},t)}{|\vec{v}-{\vec{v}}^{\,\prime} |}\,d{\vec{v}}^{\,\prime}\,, 
&\alpha=i\,;\\[9pt]
\fr{Z_i^2(\gamma-1)}{\gamma}\int \fr{f_i(\vec{r},{\vec{v}}^{\,\prime}, t)}
{|\vec{v}-{\vec{v}}^{\,\prime} 
|}\,d{\vec{v}}^{\,\prime}\,, &\alpha=e\,;
\end{cases} 
\\
H_\alpha&=
\begin{cases}
4\pi \left( \fr{\gamma f_e}{Z_i^2}+f_i\right)f_i\,, & \alpha=i\,;\\[9pt]
4\pi\left(\fr{Z_i^2 f_i}{\gamma}+f_e\right)f_e\,, &\alpha=e\,.
\end{cases}
\end{align*}
Тогда при переходе к безразмерным величинам (см.\ разд.~3) система~(\ref{e1-k}) запишется 
следующим образом:
      \begin{equation}
      \left.
\!\!\begin{array}{l}
      \fr{\partial 
\hat{f}_\alpha}{\partial\hat{t}}+A_\alpha\fr{\partial\hat{f}_\alpha}{\partial\hat{y}}+
B_\alpha  \hat{E}_y
\fr{\partial\hat{f}_\alpha}{\partial\hat{v}_\alpha}=\tilde{\mathbf{Q}}\hat{f}_\alpha\,,\enskip 
\alpha=i,e;\\[9pt]
      \fr{\partial^2\hat{\varphi}}{\partial\hat{y}^2}=-\left( \hat{n}_i-\hat{n}_e\right)\,,\enskip \hat{E}_y=-
\fr{\partial\hat\varphi}{\partial\hat{y}}\,,\\[9pt]
\hspace*{3.1mm}\hat{t}=0:\ \hspace*{2.6mm}\hat{f}_\alpha(\hat{y},\hat{v}_y, 0)=\hat{f}_\alpha^{\mathrm{maksv}}\,,\enskip \alpha=i,e\,,\\[9pt]
\hspace*{2.9mm} \hat{y}=0:\ \hspace*{2.8mm}\hat{f}_\alpha(0,\hat{v}_y,\hat{t})=0\,,\enskip \alpha=i,e\,;\\[9pt]
\hspace*{24.3mm}\hat\varphi(0,\hat{t})=\hat{\varphi}_p\,;\\[9pt]
      \hat{y}=\hat{y}_\infty:\ \hat{f}_\alpha(\hat{y}_\infty, 
\hat{v}_y,\hat{t})=\hat{f}_\alpha^{\mathrm{maksv}}\,,\enskip \alpha=i,e\,;\\[9pt]
\hspace*{21.5mm}\hat{\varphi}(\hat{y}_\infty,\hat{t})=0\,,\\[9pt]
    \end{array}
\right\}\!\!
\label{e7-k}
\end{equation}
где 
\begin{gather*}
\tilde{\mathbf{Q}} \hat{f}_\alpha=D_\alpha\fr{\partial^2\hat{f}_\alpha}{\partial 
[\hat{v}_y]^2}+K_\alpha\fr{\partial\hat{f}_\alpha}{\partial\hat{v}_y}+H_\alpha\,;\\
D_\alpha=A_g^\alpha\fr{\partial^2\hat{g}_\alpha}{\partial [\hat{v}_y]^2}\,;\enskip 
K_\alpha=A_h^\alpha \fr{\partial \hat{h}_\alpha}{\partial\hat{v}_y}\,,\ \alpha=i,e\,.
\end{gather*}

Для решения системы уравнений~(\ref{e7-k}) применяется модификация метода 
расщепления~\cite{17-k}, согласно которой исходная задача разбивается на две вспомогательные. Такое 
разбиение можно осуществить, переписав уравнение Фок\-ке\-ра--План\-ка в следующем виде:
$$
\fr{\partial\hat{f}_\alpha}{\partial\hat{t}} =
\tilde{\mathbf{Q}}_1\hat{f}_\alpha+\tilde{\mathbf{Q}}_2\hat{f}_\alpha\,,
$$
где 
\begin{align*}
\tilde{\mathbf{Q}}_1\hat{f}_\alpha &=-
\left(A_\alpha\fr{\partial\hat{f}_\alpha}{\partial\hat{y}}+
B_\alpha\fr{\partial\hat{f}_\alpha}{\partial\hat{y}}
\right)\,;\\
\tilde{\mathbf{Q}}_2\hat{f}_\alpha 
&=\left(D_\alpha\fr{\partial^2\hat{f}_\alpha}{\partial[\hat{v}_y]^2}+K_\alpha\fr{\partial 
\hat{f}_\alpha}{\partial\hat{v}_y}+H_\alpha\right)\,.
\end{align*}

      Правая часть уравнения Фок\-ке\-ра--План\-ка представляет собой сумму двух операторов, 
первый из которых отвечает за перенос частиц, второй~--- за столкновения заряженных частиц. 
В~результате образуются следующие задачи, которые решаются последовательно:
      \begin{itemize}
\item первая задача:
\begin{align*}
&\fr{\partial w_\alpha(\hat{y},\hat{v}_y,\hat{t})}{\partial\hat{t}} =\mathbf{Q}_1 
w_\alpha(\hat{y},\hat{v}_y,\hat{t})\,,\enskip \alpha=i,e\,;\\[9pt]
&\fr{\partial^2\hat\varphi}{\partial\hat{y}^2}=-\left(\hat{n}_i-\hat{n}_e\right)\,;\enskip
\hat{E}_y=-
\fr{\partial\hat\varphi}{\partial\hat{y}}\,;\\[9pt]
&w_\alpha(\hat{y},\hat{v}_y,\hat{t}^n)=\hat{f}_\alpha(\hat{y},\hat{v}_y,\hat{t}^n)\,,\enskip n=0,\ldots ,N-
1\,;\\[9pt]
&\hspace{2.9mm}\hat{y}=0:\ \hspace*{2.9mm}w_\alpha(0,\hat{v}_y,\hat{t})=0\,,\enskip \alpha=i,e\,;\\[9pt]
&\hspace*{25.1mm}\hat\varphi(0,\hat{t})=\hat{\varphi}_p\,;\\[9pt]
&\hat{y}=\hat{y}_\infty:\ w_\alpha(\hat{y}_\infty, \hat{v}_y, \hat{t})=
\hat{f}_\alpha^{\mathrm{maksv}}\,,\enskip 
\alpha=i,e\,;\\[9pt]
&\hspace*{22.5mm}\hat\varphi(\hat{y}_\infty,\hat{t})=0\,;
\end{align*}
\item вторая задача:
\begin{align*}
\!\!\!\!\!\!\!\fr{\partial s_\alpha(\hat{y},\hat{v}_y,\hat{t})}{\partial \hat{t}} &=\mathbf{Q}_2 
s_\alpha(\hat{y},\hat{v}_y,\hat{t})\,, & \alpha&=i,e\,;\\
\!\!\!\!\!\!\!s_\alpha (\hat{y},\hat{v}_y,\hat{t}^n) &=w_\alpha (\hat{y},\hat{v}_y, \hat{t}^{n+1}),& n&=0,\ldots ,N-
1.
\end{align*}
\end{itemize}

Первая задача представляет собой систему безразмерных уравнений Вла\-со\-ва--Пуас\-со\-на. Для ее 
решения применяется метод крупных частиц~\cite{18-k}. Согласно этому методу решение задачи 
осуществляется путем расщепления на два этапа: на первом этапе не учитываются конвективные члены 
и решение получается обычным интегрированием на неподвижной эйлеровой сетке, а на втором этапе 
рассматривается система, которая описывает перенос частиц в лагранжевой системе координат. Кроме 
того, на первом этапе необходимо решить уравнение Пуассона для получения значений потенциала 
самосогласованного электрического поля. Для этого применяется метод, описанный в разд.~3. 

Вторая задача решается путем перехода к ко\-неч\-но-раз\-ност\-ной сис\-те\-ме. При этом частные 
производные $\partial^2\hat{g}_\alpha/\partial[\hat{v}_y]^2$ и $\partial\hat{h}_\alpha/\partial\hat{v}_y$ 
аппроксимируются со вторым порядком точности с использованием трехточечного шаблона, а 
производная $\partial s_\alpha/\partial\hat{t}$ аппроксимируется на двухточечном шаблоне с первым 
порядком точности~\cite{16-k}. К~полученной системе разностных уравнений предлагается применить 
один из классических методов решения систем линейных уравнений, например метод 
Гаусса~\cite{19-k}.
      
      Решением первой задачи является функция $w_\alpha(\hat{y}, \hat{v}_y, \hat{t}^n)$, 
$n\hm=0,\ldots ,N$, , которая дает начальное условие для второй задачи. Решая вторую задачу, находим 
функцию $s_\alpha(\hat{y},\hat{v}_y,\hat{t}^n)\hm=\hat{f}_\alpha(\hat{y},\hat{v}_y,\hat{t}^n)$, 
$n=1,\ldots ,N$, $\alpha=i,e$, которая определяет решение $\hat{f}_\alpha(\hat{y},\hat{v}_y,\hat{t}^n)$, 
$\alpha=i,e$, исходной системы~(\ref{e7-k}) для рассматриваемых моментов времени $n=1,\ldots ,N$.

Моменты функций распределения $\hat{f}_\alpha$, $\alpha=i,e$, находятся с помощью методов 
численного интегрирования, например метода трапеций~\cite{19-k}.

\section{Результаты численного моделирования}

Для двух описанных выше методов реализованы две отдельные программы в среде {Matlab~7.0}. 
Эти программы позволяют по заданным значениям концентраций и температур частиц $n_{i\infty}$, 
$n_{e\infty}$, $T_{i\infty}$ и~$T_{e\infty}$ в невозмущенной плазме, а также потенциала~$\varphi_p$, 
подаваемого на зонд, изучить эволюцию во времени плотностей тока частиц~$j_i$ и~$j_e$, концентраций 
частиц~$n_i$  и~$n_e$ в произвольной точке пространства в возмущенной зоне, а также динамику 
изменения напряженности~$E_y$ самосогласованного электрического поля во времени и пространстве.

С использованием разработанных программ проведены серии расчетных экспериментов, в которых 
значение концентраций варьировалось в пределах $n_{i\infty} \hm = n_{e\infty}\hm =10^{18}\div 
10^{22}$~м$^{-3}$. Значение температур было выбрано неизменным и равным $T_{i\infty}\hm = 
T_{e\infty}\hm=3000$~K, а значения потенциала, подаваемого на зонд, изменялись в пределах 
$\varphi_p\hm=0\div 2{,}6$~В.

На рис.~1  и~2 приведены графики изменения напряженности самосогласованного электрического
 поля (см.\ рис.~1) и плотности токов ионов (см.\linebreak\vspace*{-12pt}

\pagebreak

\end{multicols}

\begin{figure} %fig1
\vspace*{1pt}
\begin{center}
\mbox{%
\epsfxsize=162.594mm
\epsfbox{kud-1.eps}
}
\end{center}
\vspace*{-9pt}
\Caption{Динамика изменения плотности тока ионов во времени в фиксированной точке возмущенной 
зоны для значений потенциала: \textit{1}~--- $\varphi_p=-6$; 
\textit{2}~--- $\varphi_p=-16$; \textit{3}~--- $\varphi_p=- 30$ 
в случае применения методов Монте-Карло~(\textit{а}) 
и крупных частиц~(\textit{б})}
\end{figure}

\begin{figure} %fig2
\vspace*{1pt}
\begin{center}
\mbox{%
\epsfxsize=162.713mm
\epsfbox{kud-2.eps}
}
\end{center}
\vspace*{-9pt}
\Caption{Динамика изменения напряженности электрического поля во времени в фиксированной точке 
возмущенной зоны для значений потенциала: 
\textit{1}~--- $\varphi_p=-6$; \textit{2}~--- $\varphi_p=-16$; 
\textit{3}~--- $\varphi_p=-30$ в случае применения методов Монте-Карло~(\textit{а}) и
крупных частиц~(\textit{б})
}
\end{figure}

\begin{multicols}{2}

\noindent
 рис.~2) во времени в фиксированной точке пространства 
возмущенной зоны в случае применения обоих разработанных алгоритмов.


На основании полученных результатов можно отметить похожее поведение зависимостей 
напряженности электрического поля и плотности тока от времени в двух рассматриваемых случаях. 
Графики кривых сначала убывают, затем начинают возрастать, выходя в некоторый момент 
времени~$t^\prime$ (момент установления) на стационарные значения. 

Одинаковое поведение 
напряженности и плот\-ности тока можно объяснить из следующих соображений: плотность тока ионов в 
данной области пространства равна произведению концентрации ионов на их направленную скорость и 
на заряд иона. Скорость ионов, в свою очередь, зависит от заряда, массы и напряженности 
электрического поля. 
%\columnbreak

При внесении в плазму отрицательно заряженного зонда возникает электрическое поле, которое 
нарушает квазинейтральность плазмы. Для того чтобы компенсировать действие внешнего 
электрического поля, ионы устремляются к зонду, а электроны~--- от зонда. Это приводит к дисбалансу 
концентраций вблизи зонда и, как следствие, к увеличению разности потенциалов; график 
напряженности электрического поля убывает. Вскоре разделение зарядов компенсирует внешнее 
электрическое поле; график выходит на стационарное значение. 

Также можно отметить, что значения 
напряженности электрического поля и плотности тока частиц на зонд в момент установления для двух 
методов совпадают. 

Момент установления~$t^\prime$ зависит от при\-ме\-ня\-емо\-го метода решения. В~случае метода 
Мон\-те-Кар\-ло $t^\prime=3{,}5\div 4$~ед., а для метода крупных частиц совместно с методом 
расщепления $t^\prime\hm=5\div 5{,}5$~ед. Используя ко\-неч\-но-раз\-ност\-ный метод, можно 
получить динамику изменения функций распределения частиц~$f_\alpha$, $\alpha=i,e$, во времени и 
пространстве. Функции распределения позволяют наглядно представить влияние на картину 
распределения частиц вблизи зонда самой поверхности зонда и электрического поля.

\section{Заключение}
      
      В работе найдено решение задачи диагностики плоским зондом сильноионизованной плазмы с 
учетом столкновений заряженных частиц. Разработана математическая модель исследуемого явления, 
описываемая уравнениями Фок\-ке\-ра--План\-ка и Пуассона. Решение получено двумя методами:\linebreak 
статистическим и ко\-неч\-но-раз\-ност\-ным на основе\linebreak сформированных алгоритмов. Приведены 
резуль-\linebreak таты численного моделирования при различных\linebreak характерных параметрах задачи.
 Из  проведенных 
вычислительных экспериментов вытекает, что искомые величины: напряженность 
электрического поля, плотности токов частиц на зонд, концентрации частиц вблизи зонда~--- как по 
характеру зависимости, так и по числовым значениям совпадают. При применении метода 
      Мон\-те-Кар\-ло момент установления наступает быстрее по сравнению с конечно-разностным 
методом, однако конечно-разностный метод позволяет получить более наглядные результаты.

{\small\frenchspacing
{%\baselineskip=10.8pt
\addcontentsline{toc}{section}{Литература}
\begin{thebibliography}{99}

\bibitem{1-k}
\Au{Alexeff I., Anderson T.}
Experimental and theoretical results with plasma antenna~// IEEE Trans. Plasma Sci., 2006. Vol.~34. 
No.\,2. P.~166--172.

\bibitem{2-k}
\Au{Сысун В.\,И.}
Сильноионизованная низкотемпературная плазма в приборах электронной техники: Методы 
исследования, свойства, применение. Дисс. \ldots д-ра физ.-мат. наук в форме науч. докл.: 
01.04.08.~--- Пет\-ро\-за\-водск, 1996.

\bibitem{3-k}
\Au{Тухас В.\,А.}
Методология создания средств измерений и испытаний на устойчивость к кондуктивным помехам~// 
Мат-лы VI Междунар. симп. по электромагнитной совместимости и 
электромагнитной экологии.~--- СПб., 2005. С.~231--234.

\bibitem{4-k}
\Au{Гудзенко Л.\,И., Яковленко С.\,И.}
Плазменные лазеры.~--- М.: Атомиздат, 1978.  256~с.

\bibitem{5-k}
\Au{Звелто О.}
Принципы лазеров.~--- М.: Мир, 1990.  560~с.

\bibitem{6-k}
\Au{Сысун В.\,И., Хромой Ю.\,Д.}
Расширение канала мощного импульсного разряда в парах ртути~// Электронная техника, 1974. 
Сер.~4. Вып.~10. С.~80--85. 

\bibitem{7-k}
\Au{Винклер Дж.\,Р.}
Искусственные пучки частиц в космической плазме.~--- М.: Мир, 1985.  451~с.

\bibitem{8-k}
\Au{Bernstein I.\,B., Rabinowitz I.\,N.}
Theory of electrostatic probes in low-density plasma~// Phys. Fluids, 1959. Vol.~2. No.\,2. P.~112--121. 

\bibitem{9-k}
\Au{Альперт Я.\,Л., Гуревич А.\,В., Питаевский~Л.\,П.}
Искусственные спутники в разреженной плазме.~--- М.: Наука, 1964.  282~с.

\bibitem{10-k}
\Au{Чан П., Тэлбот Л., Турян~К.}
Электрические зонды в неподвижной и движущейся плазме.~--- М.: Мир, 1978.  202~с.

\bibitem{11-k}
\Au{Алексеев Б.\,В., Котельников В.\,А.}
Зондовый метод диагностики плазмы.~--- М.: Энергоатомиздат, 1989.  240~с.

\bibitem{12-k}
\Au{Пантелеев А.\,В., Кудрявцева И.\,А.}
Формирование математической модели двухкомпонентной плазмы с учетом столкновений 
заряженных частиц в случае плоского зонда~// Теоретические вопросы вычислительной техники и 
программного обеспечения: Межвузовский сб. научн. тр.~--- М.: МИРЭА, 2006. С.~11--21.

\bibitem{13-k}
\Au{Олдер Б.}
Вычислительные методы в физике плазмы.~--- М.: Мир, 1974.  111~с.

\bibitem{14-k}
\Au{Montgomery D.\,C., Tidman D.\,A.}
Plasma kinetic theory.~--- New York, 1964. 

\bibitem{15-k}
\Au{Кудрявцева И.\,А., Пантелеев А.\,В.}
Применение метода Мон\-те-Кар\-ло для анализа поведения двухкомпонентной плазмы с учетом 
столкновений между заряженными частицами~// Теоретические вопросы\linebreak
вычислительной техники и 
программного обеспечения: Межвузовский сб. научн. тр.~--- М.: МИРЭА, 2008. С.~122--128. 

\bibitem{16-k}
\Au{Семенов В.\,В., Пантелеев А.\,В., Руденко~Е.\,А., Бор\-та\-ков\-ский~А.\,С.}
Методы описания, анализа и синтеза нелинейных систем управления.~--- М.: МАИ, 1993.  312~с.

\bibitem{17-k}
\Au{Киреев В.\,И., Пантелеев А.\,В.}
Численные методы в примерах и задачах.~--- М.: Высшая школа, 2006.  480~с.

\bibitem{18-k}
\Au{Белоцерковский О.\,М., Давыдов~Ю.\,М.}
Метод крупных частиц в газовой динамике. Вычислительный эксперимент.~--- М.: Наука, 
Физматгиз, 1982.

\label{end\stat}

\bibitem{19-k}
\Au{Вержбицкий В.\,М.}
Основы численных методов.~--- М.: Высшая школа, 2002.  840~с.
 \end{thebibliography}
}
}


\end{multicols}        %10
\def\stat{kirikov}

\def\tit{<<ВИРТУАЛЬНЫЙ КОНСИЛИУМ>>~--- ИНСТРУМЕНТАЛЬНАЯ 
СРЕДА ПОДДЕРЖКИ ПРИНЯТИЯ 
  СЛОЖНЫХ ДИАГНОСТИЧЕСКИХ РЕШЕНИЙ$^*$}

\def\titkol{<<Виртуальный консилиум>>~--- инструментальная 
среда поддержки принятия сложных диагностических решений}

\def\aut{И.\,А.~Кириков$^1$, А.\,В.~Колесников$^2$, С.\,В.~Листопад$^3$, 
С.\,Б.~Румовская$^4$}

\def\autkol{И.\,А.~Кириков, А.\,В.~Колесников, С.\,В.~Листопад, 
С.\,Б.~Румовская}

\titel{\tit}{\aut}{\autkol}{\titkol}

\index{Кириков И.\,А.}
\index{Колесников А.\,В.}
\index{Листопад С.\,В.} 
\index{Румовская С.\,Б.}
\index{Kirikov I.\,А.}
\index{Kolesnikov А.\,V.}
\index{Listopad S.\,V.}
\index{Rumovskaya S.\,B.}


{\renewcommand{\thefootnote}{\fnsymbol{footnote}} \footnotetext[1]
{Работа выполнена при частичной поддержке РФФИ (проект 16-07-00272 А).}}


\renewcommand{\thefootnote}{\arabic{footnote}}
\footnotetext[1]{Калининградский филиал Федерального исследовательского центра <<Информатика и~управление>> 
Российской академии наук, \mbox{baltbipiran@mail.ru}}
\footnotetext[2]{Балтийский Федеральный университет
имени  И.~Канта, Калининградский филиал Федерального 
исследовательского центра <<Информатика и~управление>> Российской академии наук, 
\mbox{avkolesnikov@yandex.ru}}
\footnotetext[3]{Калининградский филиал Федерального исследовательского центра <<Информатика и~управление>> 
Российской академии наук, \mbox{ser-list-post@yandex.ru}}
\footnotetext[4]{Калининградский филиал Федерального исследовательского центра <<Информатика 
и~управление>> Российской академии наук, \mbox{sophiyabr@gmail.com}}
 
 \vspace*{-3pt}
 
  \Abst{Рассматривается проблема принятия индивидуального решения при диагностике 
пациентов в~ам\-бу\-ла\-тор\-но-по\-ли\-кли\-ни\-че\-ских учреждениях на примере 
диагностики артериальной гипертензии (АГ). Предлагается повысить качество принятия 
индивидуального решения за счет консультаций с~системой поддержки принятия  
решения~--- <<Виртуальным консилиумом>>, моделирующим коллективный интеллект 
врачей стационара многопрофильного больничного учреждения. Приведены результаты 
проектирования и~экспериментального исследования лабораторного прототипа 
<<Виртуального консилиума>>.}

  \KW{система поддержки принятия решения; виртуальный консилиум; функциональная 
гибридная интеллектуальная система}

\DOI{10.14357/19922264160311} 


\vskip 10pt plus 9pt minus 6pt

\thispagestyle{headings}

\begin{multicols}{2}

\label{st\stat}
  

\section{Введение}

  Степень исследования, понимания и~качества диагностики проблемных сред и~их 
окружения отражена в~научной картине мира, онтологи\-зи\-ру\-ющей его представления 
и~делающей рассуждения и~целенаправленную деятельность <<зависимыми>> от них. 
В~искусственном интеллекте понятию <<картина мира>> соответствует понятие <<модель 
внешнего мира>> М.\,Г.~Га\-азе-Рап\-по\-пор\-та и~Д.\,А.~Поспелова~[1]. 
  
  Новая картина мира складывается из многочисленных теорий и~взглядов: <<ноосфера>>, 
<<разумный мир>> (В.\,И.~Вернадский, Н.\,Н.~Моисеев, А.\,В.~Поздняков); <<мир 
диалектики>>~--- мир диалога разных логик (Е.\,Л.~Доценко); социальная парадигма 
искусственного интеллекта (<<The society of mind>>) М.~Минского;  
сис\-тем\-но-ор\-га\-ни\-за\-ци\-он\-ный подход в~искусственном интеллекте 
В.\,Б.~Тарасова; теория иерархических многоуровневых систем М.~Месаровича, Д.~Мако 
и~И.~Такахары и~др.~--- и~укладывается в~семь постулатов~[2]: (1)~признание 
гетерогенности мира и~любого объекта, разнообразия жизни; (2)~неопределенность границ 
объектов и~связь <<всего со всем>>; (3)~относительность любой иерархии и~горизонтальные 
связи; (4)~дополнительность и~сотрудничество; (5)~полицентризм; (6)~относительность 
знания; (7)~соответствие управления сложности объекта. 
  
  Сложная задача диагностики АГ (СЗДАГ)~---
  за\-да\-ча-сис\-те\-ма, вклю\-ча\-ющая диагностические и~технологические подзадачи, 
повышающие эффективность обработки симптоматической информации о пациенте. 
Разнообразие подзадач СЗДАГ с~различными характеристическими свойствами требует 
разнообразия соответствующих методов принятия решений, системного анализа, 
искусственного интеллекта и~инженерии знаний. 
  
  Анализ результатов влияния новой картины мира на ментальную составляющую 
врачебной практики и~медицинской информатики~[3] показал, что, несмотря на стремление 
биомедицины к~гетерогенности восприятия организма человека и~процесса его диагностики 
в~рамках семипостулатной картины мира, человек по-преж\-не\-му остается 
<<расчлененным>> объектом познания, что сформировало <<узких>> специалистов, 
поглощенных решением частных задач. Новый тип ученого <<праг\-ма\-ти\-ка-фак\-то\-ло\-га>> 
утратил системное мышление, перестал задумываться над тем, что делается <<вокруг>> 
и~какое значение могут иметь добытые им факты для понимания работы организма в~целом. 
В~этой связи\linebreak\vspace*{-12pt}

\pagebreak

\noindent
 очевидна необходимость перехода от методов <<конкурентной>> диагностики 
к системному мышлению и~методам гетерогенной диагностики.
  
  В~[3--5] представлены результаты системного анализа СЗДАГ, следуя 
  проблемно-структурной (ПС) методологии, этапы~1--5~[6]: идентификация, редукция сложной задачи, 
спецификация диагностических подзадач, выбор методов их решения, а~также проверка 
неоднородности сложной задачи диагностики. Работы~[3--5] подтвердили релевантность 
применения междисциплинарных инструментариев для решения 
СЗДАГ, мо\-де\-ли\-ру\-ющих разнообразие информации, 
сотрудничество, дополнительность и~относительность знаний, сочетающих методы 
и~методики системного анализа диагностической проблемы с~динамическим синтезом 
метода ее решения и~имитацией работы искусственного гетерогенного коллектива~--- 
<<виртуального консилиума>>.
  
  Разнообразие~--- признак, проявление гетерогенности. Следствие закона необходимого 
разнообразия У.\,Р.~Эшби констатирует, что управ\-ле\-ние обеспечивается, если разнообразие 
средств управ\-ля\-юще\-го не меньше разнообразия управ\-ля\-емой им ситуации. Для отображения 
в информатике ситуативного разнообразия в~естественных гетерогенных системах в~[6] 
введены модели <<гетерогенная, неоднородная задача>> и~<<гомогенная, однородная 
задача>>, а~сам закон трактуется так: только разнообразная, скоординированная клиническая 
деятельность, элементы которой в~комбинации решают одну задачу, сделает результат 
диагностики качественно лучше в~обществе с~новой научной картиной мира. Специфике 
такой работы соответствует коллективный труд экспертов в~малых группах за круглым 
столом~--- консилиумы, совещания, естественные гетерогенные системы для решения 
сложных задач~\cite{3-kir}, где на первый план выходят знания и~опыт лица, принимающего 
решения (ЛПР), и~экспертов.
  
  \begin{figure*} %fig1
\vspace*{1pt}
 \begin{center}  
\mbox{%
 \epsfxsize=147.497mm
 \epsfbox{kir-1.eps}
 }
\end{center} 
%\vspace*{-9pt}
%\Caption{Концептуальная модель процесса диагностики артериальной гипертензии: в~многопрофильном 
%стационарном больничном учреждении~(\textit{а}); в~амбулаторно-поликлиническом~(\textit{б})}
  \end{figure*}

  \addtocounter{figure}{1}
  
  Настоящая работа~--- продолжение работ~[3--5,\linebreak 7] и~имеет целью представить: (1)~результаты 
исследования процесса диагностики АГ  
в~ле\-чеб\-но-про\-фи\-лак\-ти\-че\-ских больничных учреждениях (ЛПУ) широкого 
профиля~--- предлагается повысить эффективность и~качество индивидуальных 
диагностических решений в~ЛПУ широкого профиля ам\-бу\-ла\-тор\-но-по\-ли\-кли\-ни\-че\-ско\-го 
характера (рис.~1,\,\textit{а}) за счет внедрения информационной технологии 
<<Виртуальный консилиум>>, моделирующей коллективное обсуждение; 
(2)~архитектуру <<Виртуального консилиума>> и~результаты лабораторных экспериментов с~
его интегрированными моделями (первые результаты лабораторных экспериментов 
приведены в~[7]).

\section{Диагностика артериальной гипертензии в~многопрофильном 
стационарном больничном учреждении и~в~амбулаторно-поликлиническом 
учреждении}

\vspace*{-9pt}


  В~[8, 9] представлены результаты исследования процесса диагностики 
АГ в~Калининградской клинической областной больнице (КОКБ) 
(см.\ рис.~1,\,\textit{б}) и~ее Диагностическом центре (см.\ рис.~1,\,\textit{а}). 

Для формирования 
полного дифференциального диагноза АГ коллективом врачей во главе с~лечащим врачом, 
ЛПР-кар\-дио\-ло\-гом, в~стационаре привлекаются до тринадцати вра\-чей-экс\-пер\-тов~--- носителей 
знаний из различных разделов медицины: невролог, нефролог, сосудистый хирург, уролог, 
психолог, педиатр, аку\-шер-ги\-не\-ко\-лог, онколог, окулист, врачи функциональной 
диагностики, эндокринолог, терапевт, кардиолог. 

Для исследований выбраны шесть 
специалистов (см.\ рис.~1,\,\textit{б}), решающих двенадцать функциональных подзадач 
(рис.~\ref{f2-kir}), возникающих в~90\%~случаев диагностики АГ, 
каждый из которых формирует промежуточные заключения о~состоянии объекта 
диагностики в~своей области медицинских зна\-ний. 
{\looseness=1

}

Полученные исходные данные об объекте 
диагностики разнородны (содержатся в~медицинской карте): количественные,  
ви\-зу\-аль\-но-графиче\-ские параметры (детерминированные переменные),\linebreak 
лингвистические четкие и~нечеткие переменные. Лицо, при\-ни\-ма\-ющее решение, изучает в~медицинской карте 
симптомы и~частные диагностические мнения вра\-чей-экс\-пер\-тов, множество которых 
подбирает сам, и~ставит заключительный диагноз. Вра\-чам-экс\-пер\-там доступны симптомы 
и~мнения других врачей-экспертов из медицинской карты.
\mbox{Лицо}, при\-ни\-ма\-ющее решение, и~вра\-чи-экс\-пер\-ты 
обследуют пациента и~формируют диагностические заключения согласно нормативным 
документам, например~[10]. В~ЛПУ широкого профиля (см.\ рис.~1,\,\textit{а}) ЛПР~--- это врач 
общей практики или терапевт (иногда кардиолог, но зачастую без опыта работы, к~которому 
направляет терапевт сразу же при выявлении повышенного артериального давления), это 
врач <<праг\-ма\-тик-фак\-то\-лог>>~\cite{9-kir}, объединяющий в~себе роли вра\-ча-ЛПР  
и~вра\-чей-экс\-пер\-тов узкой специализации.

\end{multicols}

\begin{figure} %fig2
\vspace*{1pt}
 \begin{center}  
\mbox{%
 \epsfxsize=163.044mm
 \epsfbox{kir-2.eps}
 }
\end{center} 
\vspace*{-9pt}
\Caption{Архитектура ВКДАГ }
\label{f2-kir}
\vspace*{3pt}
\end{figure}

\begin{multicols}{2}
  

  Исследования диагностического процесса на материалах Диагностического центра КОКБ 
по модели на рис.~1,\,\textit{а} показали, что~70\%~пациентов с~АГ 
амбулаторно-поликлинического учреждения не знают о своем заболевании, в~то время как в~стационарных 
медицинских учреждениях (см.\ рис.~1,\,\textit{б}) практически в~100\%~случаев имеет место 
как адекватное проведение, так и~отображение в~медицинских картах симптоматических 
данных обследования с~подтверждением диагноза  
ла\-бо\-ра\-тор\-но-ин\-ст\-ру\-мен\-таль\-ны\-ми методами исследования. 
  
  В этой связи предлагается повысить эффективность и~качество индивидуальных 
диагностических решений в~ЛПУ широкого профиля амбула\-тор\-но-по\-ли\-кли\-ни\-че\-ско\-го 
характера (см.\ рис.~1,\,\textit{а}) за счет внед\-ре\-ния информационной технологии 
<<Виртуальный консилиум>> (см.\ рис.~\ref{f2-kir}), моделирующей коллективное обсуждение, 
обладающего синергией, опытом и~знаниями в~решении подзадач диагностики 
АГ в~стационаре (см.\ рис.~1,\,\textit{б}). 


  

  
\section{Инструментальная среда <<Виртуальный консилиум для~диагностики 
артериальной гипертензии>>}

\vspace*{-18pt}

  Инструментальная среда <<Виртуальный консилиум>>, архитектура которой 
представлена на рис.~\ref{f2-kir}, а~структура в~\cite{7-kir}, ограничена пациентами 
стар\-ше~18~лет, без особых состояний, нет распознавания снимков, не предусматривается 
назначение лечения и~не диагностируется ряд симптоматических артериальных гипертензий. 

Архитектура <<Виртуального консилиума для диагностики артериальной гипертензии>> 
(ВКДАГ) включает межмодульные интерфейсы~$\zeta^u$ для модулей, реализованных 
посредством различных методологий гибридных интеллектуальных сис\-тем (\mbox{ГиИС}) 
(генетические алгоритмы ($g$), нечеткие 
сис-\linebreak\vspace*{-12pt}

\pagebreak

\end{multicols}

\begin{table*}\small
%\vspace*{-12pt}
\begin{center}
\Caption{Описание блоков архитектуры ВКДАГ}
\vspace*{2ex}

\begin{tabular}{|p{30mm}|p{40mm}|p{39mm}|p{39mm}|}
\hline
\multicolumn{1}{|c|}{Наименование блока}&\multicolumn{1}{c|}{Функции}&\multicolumn{1}{c|}{Вход}&\multicolumn{1}{c|} 
{Выход}\\
\hline
Технологический модуль $i$-й&
Организация эффективной обработки данных и~знаний, выбирается для 
включения в~функциональную \mbox{ГиИС}~--- построение информативного набора 
признаков для диагностики&Популяция 
индивидуумов, накладывающихся как маска на $i$-й функциональный модуль&
Наилучшая особь с~оптимальным набором признаков~--- накладывается как 
маска на $i$-й функциональный модуль\\
\hline
Функциональный модуль $i$-й&Классификация состояния здоровья пациента в~рамках 
\mbox{$i$-й} диагностической 
подзадачи, выбирается для включения в~функциональную \mbox{ГиИС} &
Подмножество $i$-е симптомов с~интерфейса 
пользователя&Частное $i$-е заключение о~со\-сто\-янии здоровья пациента\\
\hline
Функциональный модуль {HCCCC}, моделирующий ЛПР&
Формирование заключительного диагноза 
АГ (всегда в~составе <<Виртуального консилиума>>)&Подмножество симптомов 
с~интерфейса пользователя, множество выходов функциональных модулей&
Заключительный диагноз АГ \\
\hline
Функциональный модуль {ИНСРЭКГ}&Классификация патологического состояния пациента по его 
электрокардиограмме&\multicolumn{2}{p{60mm}|}{Рассмотрены подробно в~\cite{4-kir}}\\
\cline{1-2}
Функциональный модуль {ИНССМАД}&Прогноз нормальных зна\-чений суточного мониторирования 
артериального давле\-ния и~вычисление отклонения &\multicolumn{2}{c|}{\ }\\
\hline
Интерфейс модификации структуры {ВКДАГ}&Исключение из диагностики модулей, решающих не 
интересующие пользователя подзадачи &
Выбранные пользователем подзадачи диагностики &
Функциональная ГиИС, 
синтезированная посредством алгоритма из~\cite{4-kir}\\
\hline
Интерфейс пользователя <<Диагноз>>&Визуализация результатов диагностики и~корректировка их 
пользователем &Заключительный диагноз от функционального модуля НСССС&Отчет, содержащий 
множество симптомов и~диагноз\\
\hline
Интерфейс пользователя &Ввод информации о~со\-сто\-янии здоровья пациента &
Множество значений 
показателей состояния здоровья пациента&
Показатели состояния здоровья пациента, распределенные по 
функциональным модулям \\
\hline
Модификация интерфейса пользователя&Деактивация элементов на интерфейсе пользователя для ввода 
значений показателей состояния здоровья&Множество выходов технологических модулей&Частично 
деактивированный интерфейс пользователя \\
\hline
\end{tabular}
\end{center}
\end{table*}

\begin{multicols}{2}

\noindent 
те\-мы ($f$), искусственные нейронные сети ($n$)).
 В~библиотеке модулей диагностики 
и~препро\-цессии хранятся заранее инициализированные\linebreak в~программной среде 
функциональные и~технологические модели. 
По умолчанию все модули включены 
в~структуру <<Виртуального консилиума>>, их описание пред\-став\-ле\-но в~табл.~1. %\\[-15pt]
%
      <<Виртуальный консилиум>> (см.\ рис.~\ref{f2-kir}) запускает интерфейс пользователя, 
ЛПР-вра\-ча~--- <<{Интерфейс модификации структуры ВКДАГ}>>, посредством 
которого включаются функциональные 
 и~технологические модули в~работу сис\-те\-мы: модуль 
<<Анализ СМАД>>, модуль <<Распознавание ЭКГ>>, модули технологических подзадач из 
группы <<Построение информативного набора признаков\linebreak (симптомов) при диагностике 
заболеваний>> и~модули подзадач из группы <<Диагностика критериев оценки 
сер\-деч\-но-со\-су\-ди\-сто\-го риска и~вторичной АГ у~пациента>> ({ДАГ}$_1$, \ldots , {ДАГ}$_9$): 
диагностики\linebreak поражений ор\-га\-нов-ми\-ше\-ней, факторов риска, цереброваскулярных 
болезней, метаболического синд\-ро\-ма и~сахарного диабета, заболеваний периферических 
артерий, ишемической болезни сердца,\linebreak эндокринной АГ, паренхиматозной нефропатии 
и~реноваскулярной АГ соответственно. Все выбранные $i$-е технологические модули 
запускаются, решают соответствующую подзадачу и~передают информацию на блок 
<<{Модификация интерфейса пользователя}>>. Он деактивирует показатели 
со\-сто\-яния здоровья на <<{Интерфейсе пользователя для\linebreak ввода значений показателей 
состояния здоровья пациента}>> и~корректирует работу $i$-го функционального модуля 
подзадач {ДАГ}$_1$, \ldots\linebreak \ldots , {ДАГ}$_9$. Далее активируется откорректированный 
интерфейс, вводятся симптомы, которые передаются функциональным нечетким модулям, 
решающим подзадачи {ДАГ}$_1$, \ldots , {ДАГ}$_9$\linebreak (моделируют принятие 
решения экспертами, врачами смежных специальностей~--- кардиологом как экспертом, 
неврологом, нефрологом, терапевтом, эндокринологом, урологом). Последние в~свою 
очередь передают информацию о~патологиях, выявленных ими у~пациента, 
функциональному модулю {НСССС} (моделирует принятие решения ЛПР~---  
вра\-чом-кар\-дио\-ло\-гом), решающему подзадачу <<Оценка степени и~стадии 
артериальной гипертензии, степени риска сер\-дечно-сосу\-ди\-стых заболеваний>>. 

В~библиотеке ВКДАГ есть еще два функциональных модуля (см.\ табл.~1), вклю\-ча\-ющих\-ся 
в~работу консилиума посредством <<{Интерфейса модификации структуры 
ВКДАГ}>>: 
      \begin{enumerate}[(1)]
      \item {ИНСРЭКГ}, передающий информацию на модули диагностики поражений 
ор\-га\-нов-ми\-ше\-ней (на рис.~\ref{f2-kir}~--- это {НСДАГ}$_1$), цереброваскулярных 
болезней ({НСДАГ}$_3$) и~ишемической болезни сердца ({НСДАГ}$_6$); 
      \item {ИНССМАД}, формирующий информацию о~нормальных значениях 
суточного артериального давления на функциональный модуль {НСССС}.
      \end{enumerate}
      
\section{Экспериментальное лабораторное исследование программной 
реализации прототипа инструментальной среды <<Виртуальный консилиум>>}
  
  Экспериментальное лабораторное исследование программной реализации 
исследовательского прототипа функциональной гибридной интеллектуальной системы 
ВКДАГ для поддержки принятия сложных диагностических решений необходимо для 
подтверждения его релевантности~[3--5, 7] реальной ситуации диагностики АГ. В~[4] 
пред\-став\-ле\-на информация по особенностям функциональных и~технологических моделей 
гетерогенного модельного поля ВКДАГ, а~в~[7]~--- информация по их инициализации 
в~среде MATLAB-Simulink, результаты исследований качества работы каждой модели 
гетерогенного модельного поля <<Виртуального консилиума>> автономно, а~также 
подтверждена их релевантность работе экспертов~--- врачей узкой специализации, что 
предотвращает распространение ошибок работы автономных моделей на работу 
интегрированной модели. 

В~настоящей работе приведены результаты исследования качества 
интегрированных моделей, синтезированных <<Виртуальным консилиумом>>\linebreak 
и~моделирующих дополнительность и~сотрудничество, которые имитируют коллективные 
рас\-суж\-де\-ния специалистов при постановке диагноза. 

В~табл.~2 представлены критерии 
и~результаты тес\-ти\-ро\-ва\-ния интегрированных моделей <<Виртуального консилиума>> 
с~различными комбинациями знаний врачей, классифицирующих патологическое состояние 
пациента. Порядок работы моделей гетерогенного модельного поля \mbox{ВКДАГ}: запускаются 
модели первой очереди~--- модели технологических элементов {ГАППС}$_{1\mbox{--}9}$, 
корректирующие множества входных переменных моделей {НСДАГ}$_{1\mbox{--}9}$ 
и~{НСССС}; обработка информации передается функциональным элементам: модели 
второй очереди <<отправляют>> информацию на модели третьей, пятой, шес\-той и~седьмой 
очередей~--- \mbox{ИНСРЭКГ} (модель, решающая задачу распознавания электрокардиограммы (ЭКГ)), 
{ИНССМАД} (формирует оптимальные множества показателей суточного давления), 
{НСДАГ}$_9$, {НСДАГ}$_2$ и~{НСДАГ}$_6$; третья\linebreak очередь содержит 
модели НСДАГ$_4$ и~НСДАГ$_5$, передающие выходную информацию на вход моделей четвертой 
и~седьмой очередей; четвертая очередь содержит модель {НСДАГ}$_8$, пе\-ре\-да\-ющую 
информацию  модели пятой очереди {НСДАГ}$_1$, которая в~свою очередь передает 
информацию\linebreak {НСДАГ}$_3$ (шес\-тая очередь); от {НСДАГ}$_3$ передается 
информация {НСДАГ}$_7$ (седьмая очередь); последней запускается модель 
{НСССС}, формирующая заключительный диагноз, на вход которой передается 
выходная информация функциональных моделей вто\-рой--седь\-мой очередей.
  
  Таким образом: (1)~без знаний кардиолога, или нефролога, или эндокринолога 
сред\-не\-квад\-ратическая ошибка наибольшая~--- 0,697; 0,448 и~0,211 соответственно, 
и~объясняется это тем, что кардиолог играет ключевую роль в~обработке ин\-формации, 
поступающей от других врачей\linebreak\vspace*{-12pt}


\pagebreak

\end{multicols}

\begin{table}\small
\begin{center}
\Caption{Параметры и~результаты тестирования интегрированных моделей }
\vspace*{2ex}

\begin{tabular}{|p{66mm}|p{88mm}|}
\hline
\multicolumn{1}{|c|}{\tabcolsep=0pt\begin{tabular}{c}Наименование параметров\\ 
и результатов тестирования\end{tabular}}&
\multicolumn{1}{c|}{Значения параметров и~результатов 
тестирования}\\
\hline
Объем тестовой выборки ВКДАГ, интегрирующего знания всех шести врачей&800 наблюдений~--- 500 с~
диагнозами эссенциальной АГ и~300 с~диагнозами вторичной АГ\\
\hline
Объем тестовой выборки ВКДАГ, интегрирующего знания менее шести врачей&400 наблюдений~--- 200 с~
диагнозами эссенциальной АГ и~200 с~диагнозами вторичной АГ\\
\hline
Источник формирования тестовой вы\-борки&Архив медицинских карт пациентов 1-го кардиологического 
отделения КОКБ\\
\hline
Элемент тестирующей последова\-тель\-ности&
Содержит множество нечетких лингвистических переменных и~вектор образа электрокардиограммы (может отсутствовать)\\
\hline
Эталонный диагноз&Результаты деятельности лечащего вра\-ча-кар\-дио\-ло\-га, подводящего общий итог~--- 
дифференциальный диагноз АГ\\
\hline
Критерии тестирования&Среднеквадратическая ошибка $f$ классификации состояния здоровья пациента~[7]\\
\hline
$f$(шесть врачей)&0,0837\\
\hline
$f$(без кардиолога)&0,697\\
\hline
$f$(без нефролога)&0,448 (в остальных 55,2\% случаях диагноз не вызовет доверия)\\
\hline
$f$(без терапевта)&0,151\\
\hline
$f$(без невролога)&0,149\\
\hline
$f$(без эндокринолога)&0,211 (в остальных 78,9\% случаях диагноз не вызовет доверия)\\
\hline
$f$(без сосудистого хирурга)&0,0798\\
\hline
$f$(без знаний терапевта, невролога, неф\-ро\-ло\-га, эндокринолога, сосудистого хирурга)&0,711\\
\hline
$f$(без знаний терапевта, невролога, эндокринолога, сосудистого хирурга)&0,485\\
\hline
$f$(без знаний невролога, эндокринолога, сосудистого хирурга)&0,334\\
\hline
$f$(без знаний невролога, сосудистого хи\-рурга)&0,167\\
\hline
\end{tabular}
\end{center}
\end{table}

\begin{multicols}{2}


\noindent
 и~от ла\-бораторных исследований, и~в~постановке
заключительного диагноза, а~нефролог и~эндокринолог~--- в~исключении вторичной 
АГ; (2)~знания врача~--- сосудистого хирурга не влияют на 
результаты работы <<Виртуального консилиума>>, и~объясняется это тем, что знания 
сосудистого хирурга, касающиеся диагностики АГ, составляют только~20\% базы знаний 
нечеткой системы, распознающей заболевания периферических артерий (ассоциативные 
клинические состояния), встречающихся не более чем у~10\% населения~\cite{11-kir}, 
и~в~тес\-то\-вую выборку не попала ни одна карта с~данными заболеваниями; (3)~чем больше 
численный состав <<Виртуального консилиума>>, тем с~меньшей среднеквадратической 
ошибкой он классифицирует состояние здоровья пациента; (4)~<<Виртуальный консилиум>> 
в~со\-ста\-ве шести врачей диагностирует АГ со среднеквадратической 
ошибкой постановки диагноза $f = 0{,}0837$, т.\,е.\ дает диагноз, верный в~84\% слу\-чаях. 
{\looseness=1

}
  
  Поскольку <<Виртуальный консилиум>> разра\-ботан на основе всероссийских~\cite{9-kir} 
и~между\-народных рекомендаций по диагностике АГ и~со\-пут\-ст\-ву\-ющих заболеваний, 
которых должен придерживать\-ся каж\-дый врач в~своей практике, при переносе \mbox{ВКДАГ} 
в~другое больничное учреж\-де\-ние необходимо пред\-оста\-вить врачам данного учреждения 
протоколы подтверждения диагностических правил всех баз знаний экспериментальными 
данными из архива КОКБ для ознакомления 
и~внесения при необходимости коррективов в~связи с~возможными особенностями их 
контингента пациентов, а~также возможных требований по устранению ограничений 
системы со стороны персонала нового больничного учреждения. Значительной 
корректировки баз знаний не потребуется.
  
  Таким образом, лабораторные эксперименты с~прототипом <<Виртуального 
консилиума>> дали обнадеживающие результаты. 

Верное решение получено в~84\% 
случаев. В~ам\-бу\-ла\-тор\-но-кли\-ни\-че\-ских учреждениях диагноз не 
выявляется у~70\% пациентов в~основном по причине инертности врачей, недостатка опыта 
врачей узкой специализации и~нехватки кадров в~ЛПУ
широкого профиля, что по результатам экспериментов может быть устранено с~по\-мощью 
применения \mbox{ВКДАГ} во время приема пациентов с~подозрением на АГ.

\section{Заключение}

  Лабораторно подтверждена эффективность предлагаемого подхода для проектирования 
диагностических систем как гетерогенных искусственных диагностических систем со 
свойствами дополнительности, сотрудничества и~относительности\linebreak
 знаний, синтезирующих 
интегрированные методы и~модели, разнообразие которых устраняет разнообразие 
диагностической информации об организме человека~--- <<Виртуальных консилиумов>>,\linebreak 
моделиру\-ющих работу коллектива врачей в~многопрофильном стационарном больничном 
учреждении (на примере КОКБ) и~внедрение 
которых повыша\-ет эффективность и~качество индивидуальных диагностических решений 
в~ам\-бу\-ла\-тор\-но-по\-ли\-кли\-ни\-че\-ском учреждении широкого профиля (на примере 
Диагностического центра КОКБ), где заключение о состоянии больного из-за проблемы 
с~кадрами узкой специализации принимает чаще всего один специалист~--- терапевт или 
врач общей практики, иногда кардиолог, но без опыта работы.

{\small\frenchspacing
 {%\baselineskip=10.8pt
 \addcontentsline{toc}{section}{References}
 \begin{thebibliography}{99}
\bibitem{1-kir}
\Au{Гаазе-Раппопорт М.\,Г., Поспелов~Д.\,А.} От амебы до робота: модели поведения.~--- 
М.: Наука, 1987. 288~с.
\bibitem{2-kir}
\Au{Колесников А.\,В., Кириков~И.\,А., Листопад~С.\,В. %Румовская~С.\,Б. 
и~др.} Решение 
сложных задач коммивояжера методами функциональных гибридных интеллектуальных 
сис\-тем.~--- М.: ИПИ РАН, 2011. 295~с.
\bibitem{3-kir}
\Au{Кириков И.\,А., Колесников~А.\,В., Румовская~С.\,Б.} Исследование сложной задачи 
диагностики артериальной гипертензии в~методологии искусственных гетерогенных  
сис\-тем~// Системы и~средства информатики, 2013. Т.~23. №\,2. С.~81--99. doi: 
10.14357/08696527130208.
\bibitem{4-kir}
\Au{Кириков И.\,А., Колесников~А.\,В., Румовская~С.\,Б.} Функциональная гибридная 
интеллектуальная система для поддержки принятия решений при диагностике артериальной 
гипертензии~// Системы и~средства информатики, 2014. Т.~24. №\,1. С.~153--179. doi: 
10.14357/08696527140110.
\bibitem{5-kir}
\Au{Колесников А.\,В., Румовская~С.\,Б., Листопад~С.\,В., Кириков~И.\,А.} Системный 
анализ в~решении сложных диагностических задач~// Системный анализ и~информационные 
технологии (САИТ-2015): Тр. VI~Междунар. конф.~--- М.: 
ИСА РАН, 2015. Т.~1. С.~157--167.
\bibitem{6-kir}
\Au{Колесников А.\,В., Кириков~И.\,А.} Методология и~технология решения сложных задач 
методами функциональных гибридных интеллектуальных систем.~--- М.: ИПИ РАН, 2007. 
387~с.
\bibitem{7-kir}
\Au{Кириков И.\,А., Колесников~А.\,В., Румовская~С.\,Б.} Исследование лабораторного 
прототипа искусственной гетерогенной системы для диагностики артериальной 
гипертензии~// Системы и~средства информатики, 2014. Т.~24. №\,3. С.~131--143. doi: 
10.14357/08696527140309.
\bibitem{8-kir}
\Au{Румовская С.\,Б.} Методы и~средства информатики для диагностики 
артериальной гипертензии в~ле\-чеб\-но-про\-фи\-лак\-ти\-че\-ских учреждениях 
широкого профиля~// Задачи современной информатики (ЗСИ-2015): Тр. 2-й 
молодежной научной конф.~--- М.: ФИЦ ИУ РАН, 2015. 
С.~168--174.
\bibitem{9-kir}
\Au{Кириков~И.\,А., Румовская~С.\,Б.} Гетерогенная диагностика артериальной 
гипертензии~// Информатика, управление и~системный анализ (ИУСА-2016): Тр. 
4-й Всеросс. научной конф. молодых ученых с~международным участием.~--- 
Тверь: ТвГТУ, 2016. Т.~1. С.~180--188.
\bibitem{10-kir}
Комитет экспертов ВНОК. Диагностика и~лечение артериальной гипертензии. 
Российские рекомендации~// Системные гипертензии, 2010. Вып.~3. С.~5--26.
\bibitem{11-kir}
\Au{Галимзянов Ф.\,В.} Заболевания периферических артерий (клиника, 
диагностика, лечение)~// Международный журнал экспериментального образования, 
2014. Вып.~8. С.~113--114. 

\end{thebibliography}

 }
 }

\end{multicols}

\vspace*{-6pt}

\hfill{\small\textit{Поступила в~редакцию 18.06.16}}

\vspace*{8pt}

%\newpage

%\vspace*{-24pt}

\hrule

\vspace*{2pt}

\hrule

%\vspace*{8pt}



\def\tit{``VIRTUAL COUNCIL''~--- SOURCE ENVIRONMENT SUPPORTING 
COMPLEX DIAGNOSTIC DECISION MAKING}

\def\titkol{``Virtual council''~--- source environment supporting 
complex diagnostic decision making}

\def\aut{I.\,А.~Kirikov$^1$, А.\,V.~Kolesnikov$^{1,2}$, S.\,V.~Listopad$^1$, and 
S.\,B.~Rumovskaya$^1$}

\def\autkol{I.\,А.~Kirikov, А.\,V.~Kolesnikov, S.\,V.~Listopad, and 
S.\,B.~Rumovskaya}

\titel{\tit}{\aut}{\autkol}{\titkol}

\vspace*{-9pt}

\noindent
$^1$Kaliningrad Branch of the Federal Research Center ``Computer Science and 
Control'' of the Russian Academy\linebreak
$\hphantom{^1}$of Sciences, 5~Gostinaya Str., Kaliningrad 236000, 
Russian Federation
   
   \noindent
   $^2$Immanuel Kant Baltic Federal University, 14~Nevskogo Str., Kaliningrad 236041, 
Russian Federation


\def\leftfootline{\small{\textbf{\thepage}
\hfill INFORMATIKA I EE PRIMENENIYA~--- INFORMATICS AND
APPLICATIONS\ \ \ 2016\ \ \ volume~10\ \ \ issue\ 3}
}%
 \def\rightfootline{\small{INFORMATIKA I EE PRIMENENIYA~---
INFORMATICS AND APPLICATIONS\ \ \ 2016\ \ \ volume~10\ \ \ issue\ 3
\hfill \textbf{\thepage}}}

\vspace*{3pt}
  
    
  
\Abste{The paper considers the problem of individual decision making during 
diagnostics of 
patients in outpatient clinics by the example of arterial 
hypertension diagnostics. It is proposed to 
raise the quality of individual decision\linebreak\vspace*{-12pt}}

\Abstend{making by means of consultations with the ``Virtual council'' 
decision support system, which models the work of physician councils in inpatient multifield 
clinics. The results of development and experimental research of the 
laboratory prototype of ``Virtual council'' are presented.}

\KWE{decision support system; virtual council; functional hybrid intellectual system}

\DOI{10.14357/19922264160311} 

\vspace*{-9pt}

\Ack
\noindent
The work was performed with partial support of the Russian
Foundation for Basic Research (grant No.\,16-07-00272~А).


%\vspace*{3pt}

  \begin{multicols}{2}

\renewcommand{\bibname}{\protect\rmfamily References}
%\renewcommand{\bibname}{\large\protect\rm References}

{\small\frenchspacing
 {%\baselineskip=10.8pt
 \addcontentsline{toc}{section}{References}
 \begin{thebibliography}{99}
\bibitem{1-kir-1}
\Aue{Gaaze-Rappoport, M.\,G., and D.\,A.~Pospelov}. 1987. \textit{Ot ameby do robota: Modeli 
povedeniya} [From ameba to robotic mashine: Behavior model] Moscow: Nauka. 288~p.
\bibitem{2-kir-1}
\Aue{Kolesnikov,~A.\,V., I.\,A.~Kirikov, S.\,V.~Listopad, \textit{et al.}}. 2011. \textit{Reshenie 
slozhnykh zadach kommivoyazhera metodami funktsional'nykh gibridnykh intellektual'nykh 
sistem} [Solving of the complex traveling salesman problem by means of functional hybrid 
intellectual systems]. Moscow: IPI RAN. 295~p.
\bibitem{3-kir-1}
\Aue{Kirikov, I.\,A., A.\,V.~Kolesnikov, and S.\,B.~Rumovskaya}.\linebreak
 2013. Issledovanie slozhnoy 
zadachi diagnostiki arterial'noy gipertenzii v~metodologii iskusstvennykh geterogennykh sistem 
[Research of the complex problem at\linebreak diagnosing of the arterial hypertension within the 
methodology of artificial heterogeneous systems]. \textit{Sistemy i~Sredstva Informatiki~--- 
Systems and Means of Informatics} 23(2):81--99. doi: 10.14357/08696527130208.
\bibitem{4-kir-1}
\Aue{Kirikov, I.\,A., A.\,V.~Kolesnikov, and S.\,B.~Rumovskaya}.\linebreak
 2014. Funktsional'naya 
gibridnaya intellektual'naya sistema dlya podderzhki prinyatiya resheniya pri diagnostike 
arterial'noy gipertenzii [Functional hybrid intelligent decision support system for diagnosing of the 
\mbox{arterial} hypertension]. \textit{Sistemy i~Sredstva Informatiki~--- Systems and Means of Informatics} 
24(1):153--179. doi: 10.14357/08696527140110. 
\bibitem{5-kir-1}
\Aue{Kolesnikov, A.\,V., I.\,A.~Kirikov, S.\,V.~Listopad, and S.\,B.~Rumovskaya}. 2015. 
Sistemnyy analiz v~reshenii slozhnykh diagnosticheskikh zadach [Systems analysis for solving 
complex diagnostic tasks]. \textit{Tr. 6-y Mezhdunar. konf. ``Sistemnyy analiz i~informatsionnye 
tekhnologii''} [6th Conference (International) ``Systems Analysis and Information Technology'' 
Proceedings]. Moscow.  1:157--167.
\bibitem{6-kir-1}
\Au{Kolesnikov, A.\,V., and I.\,A.~Kirikov}. 2007. \textit{Metodologiya i~tekhnologiya resheniya 
slozhnykh zadach metodami funk\-tsi\-o\-nal'\-nykh gibridnykh intellektual'nykh sistem} [Methodology 
and technology for solving of complex problems using the methodology of functional hybrid 
artificial systems]. Moscow: IPI RAN. 387~p.
\bibitem{7-kir-1}
\Aue{Kirikov, I.\,A., A.\,V.~Kolesnikov, and S.\,B.~Rumovskaya}. 2014. Issledovanie 
laboratornogo prototipa iskusstvennoy geterogennoy sistemy dlya diagnostiki arterial'noy 
gipertenzii [Research of the laboratory prototype of the artificial heterogeneous system for 
diagnosing of the arterial hypertension]. \textit{Sistemy i~Sredstva informatiki~--- Systems and 
Means of Informatics} 24(3):131--143. doi: 10.14357/08696527140309.
\bibitem{8-kir-1}
\Au{Rumovskaya, S.\,B.} 2015. Metody i~sredstva informatiki dlya diagnostiki 
arterial'noy gipertenzii v~lechebno-profilakticheskikh uchrezhdeniyakh shirokogo profilya 
[Methods and tools of informatics for diagnostics of arterial hypertension in multiskilled 
medical preventive institution]. \textit{Tr. 2-y molodezhnoy nauchnoy konf. ``Zadachi 
sovremennoy informatiki''} [2nd Youth Conference ``Tasks of Modern Informatics'' 
Proceedings]. Moscow: FRC ``Computer Science and Control'' RAS. 168--174.
\bibitem{9-kir-1}
\Aue{Kirikov, I.\,A., and S.\,B.~Rumovskaya}. 2016. Geterogennaya diagnostika arterial'noy 
gipertenzii [Heterogeneous diagnostics of arterial hypertension]. \textit{Tr. 4-y Vseross. 
nauchnoy konf. molodykh uchenykh s~mezhdunarodnym uchastiem ``Informatika, 
upravlenie i~sistemnyy analiz''} [4th Youth Conference (International) ``Informatics, Control 
and Systems Analysis'' Proceedings]. Tver: Tver State Technical University. 1:180--188.
\bibitem{10-kir-1}
Komitet ekspertov VNOK [Committee of experts of All-Russia Scientific Society of Сardiologists]. 
2010. Diagnostika i~lechenie arterial'noy gipertenzii. Rossiyskie 
rekomendatsii [Diagnosing and treatment of arterial 
hypertension. Russian recommenation]. 
\textit{Sistemnye gipertenzii} [Systemic Hypertension] 3:5--26. 
\bibitem{11-kir-1}
\Aue{Galimzyanov, F.\,V.} 2014. Zabolevaniya perifericheskikh arteriy (Klinika, 
diagnostika, lechenie) [Peripheral vascular disease (Clinic, diagnostics, treatment]. 
\textit{Mezhdunarodnyy zhurnal eksperimental'nogo obrazovaniya} [Int. J.~Research 
Education] 8:113--114. 
   \end{thebibliography}

 }
 }

\end{multicols}

\vspace*{-9pt}

\hfill{\small\textit{Received June 18, 2016}}

\vspace*{-3pt}
    
  
  \Contr
  
  \noindent
  \textbf{Kirikov Igor A.}\ (b.\ 1955)~---
  Candidate of  Sciences (PhD) in technology; director, Kaliningrad Branch of the 
  Federal Research Center ``Computer Science and Control'' of the Russian Academy 
  of Sciences, 5~Gostinaya Str., Kaliningrad 236000,  Russian Federation; 
baltbipiran@mail.ru
  
  \pagebreak
%  \vspace*{3pt}
  
  \noindent
  \textbf{Kolesnikov Alexander V.}\ (b.\ 1948)~---
  Doctor of Sciences in technology; professor, 
Department of Telecommunications, 
 Immanuel Kant Baltic Federal University, 14~Nevskogo Str., Kaliningrad 236041, Russian Federation; senior scientist, Kaliningrad Branch of 
  the Federal Research Center ``Computer Science and Control'' of the Russian 
  Academy of Sciences, 5~Gostinaya Str., Kaliningrad 236000,  Russian Federation; 
  avkolesnikov@yandex.ru
  
  \vspace*{4pt}
  
  \noindent
  \textbf{Listopad Sergey V.}\ (b.\ 1984)~---
  Candidate of  Sciences (PhD) in technology; scientist, Kaliningrad Branch of the 
  Federal Research Center ``Computer Science and Control'' of the Russian Academy 
  of Sciences, 5~Gostinaya Str., Kaliningrad 236000,  Russian Federation;   
ser-list-post@yandex.ru
  
  \vspace*{4pt}
  
  \noindent
  \textbf{Rumovskaya Sophiya B.}\ (b.\ 1985)~--- programmer~I, Kaliningrad Branch 
  of the Federal Research Center ``Computer Science and Control'' of the Russian 
  Academy of Sciences, 5~Gostinaya Str., Kaliningrad 236000,  Russian Federation; 
  sophiyabr@gmail.com
  \label{end\stat}
  
  
  \renewcommand{\bibname}{\protect\rm Литература} %11
\def\stat{arkhipov}

\def\tit{ВАРИАНТ СОЗДАНИЯ ЛОКАЛЬНОЙ СИСТЕМЫ КООРДИНАТ 
ДЛЯ~СИНХРОНИЗАЦИИ ИЗОБРАЖЕНИЙ ВЫБРАННЫХ СНИМКОВ}

\def\titkol{Вариант создания локальной системы координат 
для~синхронизации изображений выбранных снимков}

\def\aut{О.\,П.~Архипов$^1$, П.\,О.~Архипов$^2$, И.\,И.~Сидоркин$^3$}

\def\autkol{О.\,П.~Архипов, П.\,О.~Архипов, И.\,И.~Сидоркин}

\titel{\tit}{\aut}{\autkol}{\titkol}

\index{Архипов О.\,П.}
\index{Архипов П.\,О.}
\index{Сидоркин И.\,И.}
\index{Arkhipov O.\,P.}
\index{Arkhipov P.\,O.}
\index{Sidorkin I.\,I.}


%{\renewcommand{\thefootnote}{\fnsymbol{footnote}} \footnotetext[1]
%{Работа выполнена при частичной поддержке РФФИ (проект 16-07-00272 А).}}


\renewcommand{\thefootnote}{\arabic{footnote}}
\footnotetext[1]{Орловский филиал Федерального исследовательского центра <<Информатика и~управление>> 
Российской академии наук, \mbox{arkhipov12@yandex.ru}}
\footnotetext[2]{Орловский филиал Федерального исследовательского центра <<Информатика и~управление>> 
Российской академии наук, \mbox{arpaul@mail.ru}}
\footnotetext[3]{Орловский филиал Федерального исследовательского центра <<Информатика и~управление>> 
Российской академии наук, \mbox{voronecburgsiti@mail.ru}}

\Abst{Рассмотрены проблемы сравнения пар изображений, имеющих 
искажения поворота и~сдвига сцен друг относительно друга. Разработан 
алгоритм создания локальной системы координат (ЛСК) для пар сравниваемых 
изображений.}

\KW{алгоритм; методика; локальная система координат; цветное 
изображение; синхронизация; пиксель; цветное пятно; фильтрация}

\DOI{10.14357/19922264160312} 


\vskip 12pt plus 9pt minus 6pt

\thispagestyle{headings}

\begin{multicols}{2}

\label{st\stat}

  \section{Введение}
  
  Важным этапом обработки кадров видеопотока является построение 
ЛСК для синхронизации обрабатываемых 
изображений. Под синхронизацией понимается процедура совмещения пары 
обрабатываемых кадров путем смещения одного изображения относительно 
другого для достижения совпадения одинаковых устойчивых робастных 
структур. В качестве общих робастных структур могут выступать границы 
объектов, имеющихся на полутоновых изображениях, и~центры одинаковых 
по площади цветных пятен соответствующих цветных изображений. 
В~случае необходимости сравнения пары кадров, полученных с~различных 
точек съемки либо с~отличным углом съемки, в~результате чего изображения 
оказались смещены относительно друг друга, синхронизация может стать 
единственно возможным решением для осуществления возможности 
машинного сравнения изображений. 

В~данной статье описывается процесс 
создания ЛСК для синхронизации пар 
обрабатываемых изоб\-ра\-же\-ний. Актуальность работы обусловлена 
необходимостью сравнения пар изоб\-ра\-же\-ний, которые были получены 
с~разных точек съемки, что привело к~искажениям поворота и~смещения. 

Целью данной работы является разработка варианта создания 
ЛСК для синхронизации пар изображений выбранных 
снимков. Основная идея работы состоит в~том, что для синхронизации двух 
изображений необходимо отыскать на этих изображениях робастные 
структуры, которые повторялись бы на каждом из этих изображений, а~затем 
выполнить создание ЛСК с~сохранением лишь 
общей части обрабатываемой пары изображений. Предполагается, что даже 
будучи смещенными друг относительно друга и/или повернутыми на 
произвольный угол, данные изображения, имеющие общую сов\-па\-да\-ющую 
часть, могут быть синхронизированы путем создания ЛСК
 и~преобразованием одного из изображений. Независимо от угла 
поворота и~смещения изображений, имеющих общую часть, робастные 
структуры данных изображений будут сов\-падать. 

\vspace*{-6pt}

  \section{Обзор аналогов}
  
  \vspace*{-2pt}
  
  Одними из наиболее распространенных методов определения 
геометрического рассогласования изображений являются корреляционные 
методы~[1, 2]. Данные методы позволяют рассчитать коэффициент 
корреляции для всех возможных вариантов смещения изображений друг 
относительно друга и~выбрать одно пиковое значение, которое будет 
соответствовать наибольшему совпадению двух сравниваемых изображений. 
Еще одним примером определения взаимного сдвига изображений являются 
статические методы, в~основе которых лежит процесс вычисления 
евклидовой меры взаимного рассогласования изображений~[3]. Однако 
данные методы являются весьма чувствительными к~шумам на 
изображениях, которые являются их неотъем-\linebreak\vspace*{-12pt}

\pagebreak

\noindent
лемой частью, и,~что более 
существенно, они не позволяют выполнить согласование изображений, 
имеющих искажение поворота. 
  
  Так как при съемке изображений нестационарной камерой получаемые 
изображения имеют именно искажения сдвига и~поворота, то пе\-ре\-чис\-лен\-ные 
выше методы не могут быть использованы для синхронизации таких 
изображений. В~данной статье предлагается метод, основанный на 
выявлении робастных характеристик, имеющих сходство на обоих 
обрабатываемых изображениях, который позволит выполнять 
синхронизацию изображений, подвергнутых искажениям сдвига и~поворота. 

\vspace*{-6pt}

  \section{Создание локальной системы координат 
для~синхронизации изображений выбранных снимков}

\vspace*{-2pt}
  
  Цветное изображение представляется в~виде двумерной 
последовательности пикселей вида
  \begin{multline*}
  \mathrm{Image}_i = \!\left\{\!
  \begin{matrix}
  p_{i,1,1}(\mathrm{R,G,B}), & p_{i,1,2}(\mathrm{R,G,B}), &\ldots\\ 
  \ldots  &\ldots  &\ldots\\
  p_{i,h,1}(\mathrm{R,G,B}), & p_{i,h,2}(\mathrm{R,G,B}), &\ldots\end{matrix}\right.\\ 
\hspace*{40mm}\left.\begin{matrix}\ldots, &p_{i,1,w}(\mathrm{R,G,B})\\
\ldots &\ldots\\
\ldots, &p_{i,h,w}(\mathrm{R,G,B})
  \end{matrix}\!
  \right\}\!\!,\hspace*{-0.7966pt}\\
   i\in [1, 2]\,,\enskip
  w\in [1, W_i]\,,\enskip h\in [1, H_i]\,,
%  \label{e1-ar}
  \end{multline*}
где Image$_i$~--- изображение снимка~$i$;
$p$~--- пиксели с~цветовыми координатами (R, G, B);
$W_i$ и~$H_i$~--- ширина и~высота изображения снимка~$i$ в~пикселях. 

  Для сравнения цветных изображений предлагается использовать цветные 
пятна изображений и~робастные характеристики этих изображений. Для 
этого необходимо выполнить процедуру получения полутоновых 
изображений для получения наборов робастных характеристик каждого из 
обра\-ба\-ты\-ва\-емых изображений вида

\noindent
  \begin{multline*}
  \mathrm{Im}_i ={}\\
  {}=Q\left(\varphi_{a,1}(\mathrm{Image}_i),   
\varphi_{a,2 }(\mathrm{Image}_i), 
\varphi_{a,3}(\mathrm{Image}_i)\right)\,,\\
  i\in [1, 2]\,,
%  \label{e2-ar}
  \end{multline*}
где Im$_i$~--- полутоновое изоб\-ра\-же\-ние снимка~$i$;
$Q$~--- функция объединения полутоновых преобразований;
$\varphi$~--- функция выполнения полутоновых преобразований~\cite{4-ar}.
  
  Перед выполнением сегментации цветных изоб\-ра\-же\-ний необходимо 
выполнить огрубление цветовых составляющих изображений до~256~цветов, 
что позволит получить более удобные для сегментации
изображения 
с~четким контрастированием цвето-\linebreak\vspace*{-12pt}

\columnbreak

\noindent вых пятен~\cite{5-ar}. Процедура 
аппроксимации изображений выполняется в~два этапа:
\begin{enumerate}[(1)]
\item  аппроксимация 
изображений до~4096~цветов вида

\noindent
  \begin{equation*}
  \mathrm{Img}_i=\left\{
  \Psi_{\mathrm{app}_{4096}} (\mathrm{Image}_i ,
\mathrm{Pal}_{4096})\right\}\,,\enskip i\in [1, 2]\,,
%  \label{3-ar}
  \end{equation*}
где Img$_i$~--- аппроксимированное до~4096~цветов изображение 
снимка~$i$;
$\Psi_{\mathrm{app}_{4096}}$~--- функция получения множества  
(R, G, B)-пик\-се\-лей в~результате аппроксимации к~4096~цветам;
$\mathrm{Pal}_{4096}$~--- палитра~4096~цветов;
\item 
аппроксимация изображений до~256~цветов вида

\noindent
\begin{equation*}
\mathrm{Img}_i=\left\{
\Psi_{\mathrm{app}_{256}}(\mathrm{Image}_i, 
\mathrm{Pal}_{256})\right\}\,,\enskip i\in [1, 2]\,,
%\label{e4-ar}
\end{equation*}
где $\Psi_{\mathrm{app}_{256}}$~--- функция получения множества  
(R, G, B)-пик\-се\-лей в~результате аппроксимации к~256~цветам;
Pal$_{256}$~--- палитра 256~цветов.
\end{enumerate}
  
  Полученные в~результате выполнения двух этапов аппроксимации 
изображения должны быть сегментированы с~целью формирования 
последовательности цветных пятен каждого изображения. 
Последовательность цветных пятен изображений можно представить в~виде: 
  \begin{multline*}
  \hspace*{-5pt}\Psi_{\mathrm{segm}_i}=\{\Psi_{i,j}\} = \left\{
  \begin{matrix}
  \psi_{i,1}(p_{i,1,1}),&\ldots,& \psi_{i,1}(p_{i,1,t}),\ldots\\
  \ldots&\ldots&\ldots\\
  \psi_{i,n}(p_{i,h,1}),&\ldots, &\psi_{i,n}(p_{i,h,t}),\ldots\end{matrix}\right.\\
\left.\begin{matrix}
\ldots, &\psi_{i,u}(p_{i,1,w-g}),&\ldots ,& \psi_{i,k}(p_{i,1,w})\\
  \ldots&\ldots&\ldots&\ldots\\
\ldots,& \psi_{i,j}(p_{i,h,w-d}),&\ldots ,& \psi_{i,j}(p_{i,h,w})
  \end{matrix} \right\}
%  \label{e5-ar}
  \end{multline*}
  
  \vspace*{-12pt}
  
  \noindent
  \begin{gather*}
  i\in [1,  2]\,,\enskip
  j\in [0,  J_j]\,,\enskip
  t\in [1, T_i]\,,\\
  d\in [1, D_i]\,,\enskip
  g\in [1, G_i]\,,\enskip u\in [1, U_i]\,,\\
  w\in [1, W_i]\,,\enskip h\in [1, H_i]\,,\enskip
  n\in [1, N_i]\,,\\
  N_i\leq J_i\,,\enskip U_i\leq J_i\,, \enskip T_i\leq J_i\,,\enskip
  D_i<W_i\,,
  \end{gather*}
где $\Psi_{\mathrm{segm}_i}$~--- множество сегментов изоб\-ра\-же\-ния 
снимка~$i$;
$\psi_{i,j}$~--- сегмент с~номером~$j$ цветного изоб\-ра\-же\-ния снимка~$i$;
$J_i$~--- максимальное количество цветных сегментов изоб\-ра\-же\-ния 
снимка~$i$.
  
  Полученное множество цветных пятен представляет собой 
последовательность из всех цветоразличимых сегментов обрабатываемых 
изоб\-ра\-же\-ний~[6--8]. Для осуществления синхронизации 
изображений путем определения и~сопоставления робастных структур пары 
обрабатываемых изображений необходимо выполнить фильтрацию 
полученного множества цветных пятен. Сравнительно маленькие и~большие 
по площади пятна на изоб\-ра\-же\-ни\-ях не позволяют выполнить синхронизацию 
изображений из-за того, что маленькие пятна могут с~большой вероятностью 
повторяться на паре обрабатываемых изоб\-ра\-же\-ний либо вовсе пропа-\linebreak\vspace*{-12pt}

\pagebreak

\noindent
дать, 
а~большие пятна могут оказаться на границах изоб\-ра\-же\-ний, что приведет 
к~невозможности точного определения их центров. Следовательно, 
необходимо выбрать для дальнейшего рассмотрения только цветные пятна 
средних размеров, имеющих граничные точки, полученные в~результате 
полутоновых преобразований.
  
  Средняя величина пятен определяется как суммарная величина значений 
площадей всех цветных пятен изображения, деленная на количество цветных 
пятен данного изображения. 
  
  Ввиду того что точное совпадение размеров пятен маловероятно, выбирать 
следует при фильт\-ра\-ции пятна, площадь которых будет принадлежать 
промежутку от $\mathrm{AveSize}/2$ до 3AveSize, где\linebreak AveSize~--- средний 
размер цветного пятна изоб\-ра\-же\-ния. Выбор данного интервала увеличивает 
количество цветных пятен, подлежащих рассмотрению, и~повышает 
вероятность успешной синхронизации изображений.
  
  Отфильтрованная последовательность цветных пятен обрабатываемых 
изображений может быть представлена в~виде:

\noindent
  \begin{multline*}
  \theta_{\mathrm{segm}_i}= \{\theta_{i,k}\}\subset \Psi_{\mathrm{segm}_i}:\ i\in 
[1, 2]\,,\\
  k\in [0,  K]\,,\enskip j\in [0,  J_i]\,,\enskip K_i\leq J_i\,,
%  \label{e6-ar}
  \end{multline*}
где $\theta_{\mathrm{segm}_i}$~--- множество цветных сегментов, оставшихся 
после фильтрации цветных сегментов изоб\-ра\-же\-ния снимка~$i$.

  После того как получена последовательность цветных пятен, 
удовлетворяющая всем условиям фильтрации: площадь и~наличие граничных 
точек, необходимо выполнить процедуру сравнения множеств~$\theta_{\mathrm{segm}_i}$ 
для первого и~второго изображений. Сравнение 
производится путем определения совпадения площадей цветных пятен 
и~взаимного удаления данных пятен от других на каждом изоб\-ра\-же\-нии. При 
этом, найдя цветное пятно $\theta_{1,b}:\ b\hm\leq K_1$, принадлежащее 
первому изображению, и~цветное пятно $\theta_{2,v}:\ v\hm\leq K_2$, 
площадь которого соответствует~$\theta_{1,b}$, необходимо выполнить 
проверку удаленности от всех цветных пятен на каждом изображении 
относительно данных пятен. Если $Y_{\mathrm{segm}}\hm=\emptyset$, то данная пара 
заносится в~множество как совпадающие друг с~другом сегменты. Если 
расстояния до большей части пятен, которые были занесены в~$Y_{\mathrm{segm}}$, 
совпадают, то необходимо добавить в~множество~$Y_{\mathrm{segm}}$ данную пару 
цветных пятен, иначе они будут признаны как ошибочно выбранные 
совпадающими друг с~другом. Получа\-емую последовательность  можно 
представить в~виде:
  \begin{multline*}
  Y_{\mathrm{segm}} =\{\tau_m\}\subset \theta_{\mathrm{segm}_i}:\
  i\in [1, 2]\,,\\
  m\in [0,M_i]\,,\enskip K_1\geq M_i\leq K_2\,,
%  \label{e7-ar}
  \end{multline*}
  
  \columnbreak
  
\noindent
где $Y_{\mathrm{segm}}$~--- множество цветных сегментов, выбранных в~качестве 
совпадающих на паре изображений выбранных снимков.

  Поскольку первый элемент множества~$Y_{\mathrm{segm}}$ не может быть 
проверен на удаленность от других цветных пятен, а второй и~последующий 
сравниваются только с~предыдущими элементами, то необходимо выполнить 
дополнительную проверку\linebreak уда\-лен\-ности каждого элемента 
множества~$Y_{\mathrm{segm}}$ для каждого изоб\-ра\-же\-ния, тем самым исключив 
возможные случайные ошибки. Результирующее множество сегментов, 
имеющих соответствие на паре обрабатываемых изоб\-ра\-же\-ний, может быть 
представлено в~виде:
  \begin{equation*}
  S_{\mathrm{segm}}= \{s_c\}\subset Y_{\mathrm{segm}}:\ c\in [0,C_i]\,,\ K_1\geq C_i\leq K_2\,,
%  \label{e8-ar}
  \end{equation*}
  
  \vspace*{-2pt}
  
  \noindent
где $S_{\mathrm{segm}}$~--- множество цветных сегментов, оставшихся после 
фильтрации цветных сегментов, выбранных в~качестве совпадающих на паре 
изображений выбранных снимков.

  Успешная синхронизация изображений возможна, если имеется три 
и~более сегментов, име\-ющих соответствие на паре обрабатываемых 
изоб\-ра\-же\-ний. При этом чем дальше данные цветовые\linebreak пятна будут 
располагаться друг от друга, тем выше точность синхронизации и~создания 
ЛСК. Если число соответствующих друг другу 
сегментов три и~более, то выполняется процедура создания 
ЛСК для обрабатываемой пары изображений путем 
определения угла поворота одного изображения относительно другого 
и~вычисления расстояний до краев общей области с~переносом пикселей 
каждого из изображений. Таким образом, при выполнении условия наличия 
минимум трех цветных пятен, которые совпадают на паре обрабатываемых 
изображений, за счет выполнения процедуры

\vspace*{2pt}

\noindent
  \begin{equation*}
  \mathrm{Im}\_s_i= \mathrm{Qs}_i(\mathrm{Image}, S_{\mathrm{segm}})\,,\enskip i\in [1, 
2]\,,
%  \label{e9-ar}
  \end{equation*}
  
  \vspace*{-2pt}
  
  \noindent
где Im\_s$_i$~--- изображения, преобразованные к~ЛСК;
Qs$_i$~--- процедура построения ЛСК для 
изоб\-ра\-же\-ния снимка~$i$;
получаем два новых изображения, которые будут иметь новые 
ЛСК, совпадающие на обоих изоб\-ра\-же\-ниях.
  
  На рис.~1 приведена схема алгоритма создания 
ЛСК для синхронизации изображений выбранных кадров.



  Процессы на рис.~1:
{1}~--- получение полутоновых представлений изоб\-ра\-же\-ний
Image$_1$ и~Image$_2$;
  {2}~--- аппроксимация изоб\-ра\-же\-ний Image$_1$ и~Image$_2$
  к~палитре~4096~цветов;
  {3}~--- аппроксимация изобра\-же\-ний Img$_1$ и~Img$_2$ 
к~палитре~256~цветов;
  {4}~---  сегментация изоб\-ра\-же\-ний Img$_1$ и~Img$_2$;
  {5}~--- фильт\-ра\-ция цветных сегментов $\Psi_\mathrm{segm_1}$ 
и~$\Psi_\mathrm{segm_2}$;\linebreak\vspace*{-12pt} 

  
  \pagebreak
  
  \end{multicols}
  
  \begin{figure} %fig1
\vspace*{1pt}
 \begin{center}  
\mbox{%
 \epsfxsize=140.988mm
 \epsfbox{arh-1.eps}
 }
\end{center} 
\vspace*{-9pt}
\Caption{Схема алгоритма создания ЛСК для синхронизации 
изображений выбранных кадров}
\end{figure}
  
  \begin{multicols}{2}
  
  \noindent
  {6}~--- определение соответствия цветных сегментов 
$\theta_{\mathrm{segm}_1}$ и~$\theta_{\mathrm{segm}_2}$ пары изоб\-ра\-же\-ний
Img$_1$ и~Img$_2$ вне зависимости от угла поворота и~смещения;
  {7}~--- фильт\-ра\-ция обнаруженных сегментов множества~$Y_{\mathrm{segm}}$;
  {8}~--- определение числа совпадающих пятен на паре сравниваемых 
изоб\-ра\-же\-ний и~возможности построения ЛСК;
  {9}~--- построение ЛСК;
  {10}~--- выдача ошибки построения ЛСК;
  {11}~--- завершение работы.
  
\vspace*{-6pt}

    \section{Результаты вычислительных экспериментов}
    
  Для тестирования предлагаемого варианта создания 
ЛСК для синхронизации изображений выбранных снимков была 
выбрана пара кадров, представленных на рис.~2. 
     
\begin{figure*} %fig2
\vspace*{1pt}
 \begin{center}  
\mbox{%
 \epsfxsize=156.221mm
 \epsfbox{arh-2.eps}
 }
\end{center} 
\vspace*{-9pt}
      \Caption{Изображения первого~(\textit{а}) и~второго~(\textit{а}) снимков}
      \end{figure*}
      
  После выполнения сегментации и~фильтрации сегментов оставшиеся 
сегменты на изображениях первого и~второго снимков выделены зеленым 
цветом. Для наглядности вручную были отмечены красными линиями 
и~стрелками части изображений, по которым наиболее отчетливо видно 
относительное смещение изображений (рис.~3).
  
\begin{figure*} %fig3
\vspace*{1pt}
 \begin{center}  
\mbox{%
 \epsfxsize=156.304mm
 \epsfbox{arh-3.eps}
 }
\end{center} 
\vspace*{-9pt}
  \Caption{Сегменты для синхронизации, визуализация относительного смещения 
изображений}
  \end{figure*}
  
  После выполнения вышеописанных процедур было 
обнаружено~6~совпадающих цветных сегментов, за счет которых была 
построена ЛСК и~преобразованы оба изображения к~виду, пригодному для 
сравнения. Полученные изображения пред\-став\-ле\-ны на
рис.~4,\,\textit{а} и~4,\,\textit{б}.
  
  \begin{figure*} %fig4
  \vspace*{1pt}
 \begin{center}  
\mbox{%
 \epsfxsize=162.337mm
 \epsfbox{arh-4.eps}
 }
\end{center} 
\vspace*{-9pt}
  \Caption{Преобразованные изображения первого~(\textit{а}) и~второго~(\textit{б}) снимков}
  \end{figure*}
  
  \vspace*{-12pt}
     
  \section{Заключение}
  
  В данной статье был предложен вариант создания 
ЛСК для синхронизации изображений выбранных снимков.
  %
  В результате его тестирования были получены положительные результаты 
для изображений, имеющих искажение сдвига и~поворота. 
  
{\small\frenchspacing
 {%\baselineskip=10.8pt
 \addcontentsline{toc}{section}{References}
 \begin{thebibliography}{9}


\bibitem{2-ar}
\Au{Прэтт У.} Цифровая обработка изображений~/ Пер. с~англ.~--- М.: Мир, 
1982. (Pratt~W.\,K. Digital image processing.~---Wiley-Interscience Publication, 
1978. 750~p.)
\bibitem{1-ar}
\Au{Форсайт Д., Понс Ж.} Компьютерное зрение. Современный подход.~--- 
М.: Вильямс, 2004. 928~с. 
\bibitem{3-ar}
\Au{Гонсалес Р., Вудс Р.} Цифровая обработка изображений~/ Пер. с~англ.~--- 
М.: Техносфера, 2005. 1070~с. (Gonzalez~R.\,C.,  Woods~R.\,E. Digital image 
processing.~--- Wiley-Interscience Publication, 2002. 793~p.)
\bibitem{4-ar}
\Au{Архипов О.\,П., Зыкова З.\,П.} Применение полутоновых представлений 
при анализе изменений цветных изоб\-ра\-же\-ний~// Информатика и~её 
применения, 2014. Т.~8. Вып.~3. С.~90--99.
\bibitem{5-ar}
\Au{Архипов О.\,П., Зыкова З.\,П.} Интеграция гетерогенной информации 
о~пикселях и~их цветовосприятии~// Информатика и~её применения, 2010. 
T.~4. Вып.~4. С.~14--25.
\bibitem{6-ar}
\Au{Архипов О.\,П., Зыкова З.\,П.} Функциональное описание 
индивидуального цветовосприятия~// Известия ОрелГТУ. Сер. 
Информационные системы и~технологии, 2010. №\,5. С.~5--12.
\bibitem{7-ar}
\Au{Архипов О.\,П., Зыкова З.\,П.} RGB-ха\-рак\-те\-ри\-за\-ция пространства 
цветовосприятия~// Системы и~средства информатики, 2010. Вып.~20. №\,1. 
С.~72--89.
\bibitem{8-ar}
\Au{Архипов О.\,П., Зыкова З.\,П.} Равноконтрастные градационные 
преобразования ступенчатых тоновых шкал~// Информационные системы 
и~технологии, 2011. №\,4. С.~39--46.
\end{thebibliography}

 }
 }

\end{multicols}

\vspace*{-6pt}

\hfill{\small\textit{Поступила в~редакцию 12.07.16}}

\vspace*{10pt}

%\newpage

%\vspace*{-24pt}

\hrule

\vspace*{2pt}

\hrule

%\vspace*{8pt}



\def\tit{THE OPTION TO CREATE A~LOCAL COORDINATE SYSTEM 
FOR~SYNCHRONIZATION OF~SELECTED IMAGES}

\def\titkol{The option to create a~local coordinate system 
for~synchronization of selected images}

\def\aut{O.\,P.~Arkhipov, P.\,O.~Arkhipov, and~I.\,I.~Sidorkin}

\def\autkol{O.\,P.~Arkhipov, P.\,O.~Arkhipov, and~I.\,I.~Sidorkin}

\titel{\tit}{\aut}{\autkol}{\titkol}

\vspace*{-9pt}

\noindent
Orel Branch of the 
Federal Research Center ``Computer Science and Control'' of the Russian Academy 
of Sciences, 137~Moskovskoe Sh., Orel 302025, Russian Federation


\def\leftfootline{\small{\textbf{\thepage}
\hfill INFORMATIKA I EE PRIMENENIYA~--- INFORMATICS AND
APPLICATIONS\ \ \ 2016\ \ \ volume~10\ \ \ issue\ 3}
}%
 \def\rightfootline{\small{INFORMATIKA I EE PRIMENENIYA~---
INFORMATICS AND APPLICATIONS\ \ \ 2016\ \ \ volume~10\ \ \ issue\ 3
\hfill \textbf{\thepage}}}

\vspace*{9pt}



\Abste{While comparing pairs of images, in most cases, the problem of 
misalignment of images arises in which one image is distortion of translation and 
rotation relative to another image. Such an image is quite difficult to compare in 
automatic mode. Existing methods of image pairs  synchronize a~large 
number of constraints due to which most of them are rarely used. The proposed 
option to create a~local coordinate system for synchronizing images is based on the 
analysis of color spots presented on the images. It is assumed that successful 
synchronization of two images on their total amount of colored spots is to be 
found that match on the data images. For comparison, it is suggested to use
colored spots and 
distances between spots. In order to successfully synchronize, one needs at 
least three colored spots, which would coincide in all modes of filtration. The 
experiments show acceptable results of synchronization.}

\KWE{algorithm; local coordinate system; color image; synchronization; pixel; 
colored spot; filtration}

\DOI{10.14357/19922264160312} 

\vspace*{9pt}

%\Ack
%\noindent



%\vspace*{6pt}

  \begin{multicols}{2}

\renewcommand{\bibname}{\protect\rmfamily References}
%\renewcommand{\bibname}{\large\protect\rm References}

{\small\frenchspacing
 {%\baselineskip=10.8pt
 \addcontentsline{toc}{section}{References}
 \begin{thebibliography}{9}

  
\bibitem{2-ar-1}
\Aue{Pratt, W.\,K.} 1978. \textit{Digital image processing}. Wiley-Interscience 
Publication. 750~p.
\bibitem{1-ar-1}
  \Aue{Forsyth, D.\,A., and J.~Ponce}. 2002. \textit{Computer vision: A~modern 
approach}. Prentice Hall Professional Technical Reference. 720~p.
\bibitem{3-ar-1}
\Aue{Gonzalez, R.\,C., and R.\,E.~Woods}. 2002. \textit{Digital image 
processing}. Wiley-Interscience Publication. 793~p.
  \bibitem{4-ar-1}
  \Aue{Arhipov, O.\,P., and Z.\,P.~Zykova}. 2014. Primenenie polutonovykh 
predstavleniy pri analize izmeneniy tsvetnykh izobrazheniy [The use of half-tone 
representations in the analysis of changes in color images]. \textit{Informatika i~ee 
Primeneniya~--- Inform. Appl.} 8(3):90--99.
  \bibitem{5-ar-1}
  \Aue{Arhipov, O.\,P., and Z.\,P.~Zykova}. 2010. Integratsiya geterogennoy 
informatsii o pikselyakh i~ikh tsvetovospriyatii [Integration of heterogeneous 
information about pixels and their color perception]. \textit{Informatika i~ee 
Primeneniya~--- Inform. Appl.} 4(4):14--25.
  \bibitem{6-ar-1}
  \Aue{Arhipov, O.\,P., and Z.\,P.~Zykova}. 2010. Funktsional'noe opisanie 
individual'nogo tsvetovospriyatiya [Characteristics of color perceptual space]. 
\textit{Izvestiya OrTGU. Ser. Informatsionnye sistemy i~tekhnologii} [Herald
of Oryol Technical State University. Ser. 
information systems and technologies] 5:5--12.

\pagebreak

  \bibitem{7-sr-1}
  \Aue{Arhipov, O.\,P., and Z.\,P.~Zykova}. 2010. RGB-kharakterizatsiya 
prostranstva tsvetovospriyatiya [RGB-characterization of color space]. 
\textit{Sistemy i~Sredstva Informatiki~--- Systems and Means of Informatics} 
1(20):\linebreak 72--89.
  \bibitem{8-ar-1}
  \Aue{Arhipov, O.\,P., and Z.\,P.~Zykova}. 2011. Ravnokontrastnye 
gradatsionnye preobrazovaniya stupenchatykh tonovykh shkal [Equal contrast 
graded transformation of step tinted scales]. \textit{Informatsionnye Sistemy 
i~Tekhnologii} [Information Systems and Technologies] 4:39--46.
\end{thebibliography}

 }
 }

\end{multicols}

\vspace*{-6pt}

\hfill{\small\textit{Received July 12, 2016}}

\vspace*{-3pt}


\Contr

\noindent
\textbf{Arkhipov Oleg P.}\ (b.\ 1948)~--- Candidate of Science (PhD) in technology, Director, Oryol Branch of  
Federal Research Center  ``Computer Science  and Control'' of the 
Russian Academy of Sciences, 137~Moskovskoe Sh., Oryol 
302025, Russian Federation; \mbox{arkhipov12@yandex.ru}

\vspace*{4pt}

\noindent
\textbf{Arkhipov Pavel O.}\ (b.\ 1979)~--- Candidate of Science (PhD) in technology, senior scientist, Oryol Branch 
of  Federal Research Center  ``Computer Science  and Control'' of the Russian 
Academy of Sciences, 137~Moskovskoe Sh., Oryol 
302025, Russian Federation; \mbox{arpaul@mail.ru}

\vspace*{4pt}

\noindent
\textbf{Sidorkin Ivan I.}\ (b.\ 1990)~--- engineer-researcher, Orel Branch of the 
Federal Research Center ``Computer Science and Control'' of the Russian Academy 
of Sciences, 137~Moskovskoe Sh., Orel 302025, Russian Federation; 
\mbox{voronecburgsiti@mail.ru}
  \label{end\stat}
  
  
  \renewcommand{\bibname}{\protect\rm Литература} %12
\def\stat{yakovlev}

\def\tit{УСКОРЕННЫЙ АЛГОРИТМ СТЕРЕОСОПОСТАВЛЕНИЯ НА~ОСНОВЕ 
ГЕОДЕЗИЧЕСКИХ ВСПОМОГАТЕЛЬНЫХ КОЭФФИЦИЕНТОВ}

\def\titkol{Ускоренный алгоритм стереосопоставления на~основе 
геодезических вспомогательных коэффициентов}

\def\aut{О.\,А.~Яковлев$^1$, А.\,В.~Гасилов$^2$}

\def\autkol{О.\,А.~Яковлев, А.\,В.~Гасилов}

\titel{\tit}{\aut}{\autkol}{\titkol}

\index{Яковлев О.\,А.}
\index{Гасилов А.\,В.}
\index{Yakovlev O.\,A.}
\index{Gasilov A.\,V.}


%{\renewcommand{\thefootnote}{\fnsymbol{footnote}} \footnotetext[1]
%{Работа выполнена при частичной поддержке РФФИ (проект 16-07-00272 А).}}


\renewcommand{\thefootnote}{\arabic{footnote}}
\footnotetext[1]{Орловский филиал Федерального исследовательского центра <<Информатика и~управление>> Российской 
академии наук, \mbox{maucra@gmail.com}}
\footnotetext[2]{Орловский филиал Федерального исследовательского центра <<Информатика и~управление>> Российской 
академии наук, \mbox{gasilov.av@yandex.ru}}

\vspace*{-12pt}
  
  \Abst{Среди локальных алгоритмов стереосопоставления качественные результаты дают 
алгоритмы, использующие концепцию вспомогательных коэффициентов. В~данной работе 
предложена модификация алгоритма стереосопоставления на основе геодезических 
расстояний. Предлагаемая модификация существенно снижает вычислительные затраты, 
незначительно уступая в~качестве сопоставления оригинальному подходу, что 
подтверждается приведенными результатами экспериментов. Рассматриваемый 
алгоритм опирается на сегментацию одного из изображений стереопары, используя 
геодезические вспомогательные коэффициенты для вычисления цвета каждого сегмента. 
Такое преобразование изображения делает возможным применение принципа частичных 
сумм при вычислении стоимости сопоставления, что и~является основным источником 
прироста производительности.}
  
  \KW{стереосопоставление; сегментация; геодезические вспомогательные коэффициенты}
  
\DOI{10.14357/19922264160313} 


\vskip 10pt plus 9pt minus 6pt

\thispagestyle{headings}

\begin{multicols}{2}

\label{st\stat}

\section{Введение}

  Задача стереосопоставления двух изображений состоит в~установлении для 
каждой точки одного изображения соответствующей ей точки на втором\linebreak 
изображении. В~данной работе рассматривается самый распространенный 
вариант исходных данных~--- пара ректифицированных изображений. В~паре 
таких изображений естественным образом можно выделить левое ($I_L$) 
и~правое ($I_R$). На ректифицированных изображениях эпиполярные прямые 
параллельны оси~$Ox$, следовательно, соответствующие точки имеют равные 
координаты по оси~$Oy$. Пусть $p\hm\in I_L$ и~$p^\prime\hm\in I_R$~--- 
соответствующие точки, тогда диспарантностью пикселя~$p$ будем называть 
величину $\mathrm{disp}\,(p)\hm= p_x\hm-p^\prime_x$. Значения $\mathrm{disp}\,(p)$ для каждого 
$p\hm\in I_L$ образуют матрицу, называемую картой диспарантности. Карта 
диспарантности выступает результатом сте\-рео\-со\-по\-став\-ле\-ния изображений 
и~может быть использована, например, для трехмерной реконструкции сцены.
  
  Среди алгоритмов стереосопоставления выделяют глобальные и~локальные 
алгоритмы. Глобальные алгоритмы решают задачу оптимизации,\linebreak вы\-чис\-ляя 
оптимальную по некоторому критерию\linebreak карту диспарантности, что обеспечивает 
высокое качество результата. Однако вычислительная\linebreak сложность глобальных 
алгоритмов не позволяет применять их в~системах реального времени. 
Локальные алгоритмы являются представителями класса жадных алгоритмов, 
имеют высокую производительность, но качество вычисленных карт 
дис\-па\-рант\-ности у~них значительно ниже, чем у~глобальных алгоритмов. 
В~связи с~постоянным совершенствованием %\linebreak 
аппарат\-ных средств большинство 
исследований на\-прав\-ле\-но на улучшение качест-\linebreak ва
 сопоставления изображений. 
Как следствие, многие локальные алгоритмы приобрели черты глобальных, и~их 
применение в~условиях ограниченных вычислительных ресурсов становится 
невозможным. Алгоритм сопоставления изображений с~помощью 
геодезических вспомогательных коэффициентов~[1] является ярким примером 
локального алгоритма, %\linebreak 
обладающего как высоким качеством, так и~высокой 
вычислительной сложностью. В~настоящей работе рассматривается 
модификация этого алгоритма, позволяющая добиться роста 
производительности с~незначительной потерей качества.

\vspace*{-9pt}
  
\section{Стереосопоставление на~основе геодезических 
вспомогательных коэффициентов}

\vspace*{-2pt}
  
  Рассмотрим неориентированный граф, вершинами которого являются 
пиксели изображения и~каждая вершина соединена ребром с~8~соседними. 
Стоимостью ребра будем считать евклидово расстояние между двумя 
пикселями как RGB-век\-то\-ра\-ми. Стоимость кратчайшего пути между двумя 
вершинами такого графа будем называть геодезическим расстоянием между 
соответствующими пикселями изображения.
  
  В работе~[1] предлагается использовать в~качестве вспомогательного 
коэффициента величину, обратную геодезическому расстоянию:
  \begin{equation}
  w(p,q) =e^{-g(p,q)/\gamma}\,,
  \label{1-ya}
  \end{equation}
     где $g(p,q)$~--- геодезическое расстояние между~$p$ и~$q$;  
$\gamma$~--- положительный коэффициент.

Коэффициент~$\gamma$ позволяет регулировать скорость убывания $w(p,q)$ 
при возрастании $g(p,q)$. Рассмотрим функцию $f(x)\hm= e^{-x/\gamma}$. Ее 
производная $f^\prime(x)\hm= -(1/\gamma) e^{-x/\gamma}$. Таким образом, чем 
меньше~$\gamma$, тем быстрее убывает~$f(x)$.
  
  Для вычисления $\mathrm{disp}\,(p)$ с~помощью вспомогательных коэффициентов 
используется ска\-ни\-ру\-ющее окно, размер которого подбирается 
экспериментально. Пусть Win$_p$~--- сканирующее окно\linebreak с~центром 
в~точке~$p$, тогда стоимость диспарантности~$d$ для пикселя~$p$ можно 
вычислить как
  $$
  c(p,d) =\sum\limits_{q\in \mathrm{Win}_p} w(p,q)  \mathrm{diff} \left(q,\left(q_x-d, q_y\right ) 
\right)\,,
  $$
  где $\mathrm{diff}\,(p,q)$~--- величина цветоразличия пикселя~$p$ левого изображения 
и~пикселя~$q$ правого.
  
  Для вычисления диспарантности каждого пикселя будем использовать 
локальную оптимизацию: в~качестве $disp(p)$ возьмем такое значение~$d$, при 
котором достигается минимум стоимостной функции~$c(p,d)$, т.\,е.
  $$
  \mathrm{disp}\,(p)=\mathop{\mathrm{argmin}}\limits_{d\leq D} c(p,d)\,,
  $$
  где $D$~--- максимально допустимая диспарантность.
  
  Пусть $W$ и~$H$~--- ширина и~высота изображения соответственно; $s$~--- 
размер сканирующего окна. Тогда вычислительная сложность построения карты 
диспарантности описанным способом составит $O(W  H  s^2 D)$.

\section{Ускоренный алгоритм стереосопоставления}

  Нетрудно понять, что вспомогательные коэффициенты позволяют оценить 
вероятность принадлежности двух пикселей одной поверхности сцены. Более 
грубую оценку принадлежности двух пикселей одной поверхности дает 
сегментация изображения~[2]. Сегментация и~вспомогательные коэффициенты 
могут быть скомбинированы, что позволит снизить вычислительную сложность 
сопоставления изображений.
  
  Пусть $S_1\cup S_2\cup S_3\cup\cdots \cup S_m$~--- сегментация левого 
изображения. Опорной точкой pivot$_i$ сегмента~$S_i$ будем называть его 
медиану как упорядоченного по значениям координат списка точек. Цветом 
сегмента~$i$ будем называть средневзвешенный вспомогательными 
коэффициентами цвет составляющих его пикселей:
  $$
  C_i= \fr{\sum\nolimits_{p\in S_i} w(p,\mathrm{pivot}_i) I_L(p)} {\sum\nolimits_{p\in S_i} 
w(p,\mathrm{pivot}_i)}\,,
  $$
  где $I_L(p)$~--- цвет пикселя~$p$  на изображении.
  
  На данном этапе предполагается, что такой вариант использования 
коэффициентов принадлежности не приведет к~значительному снижению 
качества по сравнению с~оригинальным алгоритмом. Параметр~$\gamma$ 
в~выражении~(1) должен быть согласован с~уровнем сегментации, иначе 
теряется смысл применения вспомогательных коэффициентов~--- 
значение~$C_i$ будет близко к~среднему цвету сегмента. Высокому уровню 
сегментации соответствует малое значение~$\gamma$ и~наоборот. Рисунок~1 
иллюстрирует влияние параметра~$\gamma$ на значения вспомогательных 
коэффициентов.
  
  
  \begin{figure*} %fig1
  \vspace*{1pt}
 \begin{center}  
\mbox{%
 \epsfxsize=115mm
 \epsfbox{yak-1.eps}
 }
\end{center} 
\vspace*{-9pt}
\Caption{Вспомогательные коэффициенты при различных значениях~$\gamma$: 
(\textit{а})~$\gamma\hm=10$; (\textit{б})~15; (\textit{в})~30; (\textit{г})~$\gamma\hm=50$}
\end{figure*}

  Пусть $s(p)$~--- номер сегмента, которому принадлежит пиксель~$p$, тогда 
в~качестве стоимостной функции будем рассматривать величину
  \begin{equation}
  c(p,d) =\sum\limits_{q\in \mathrm{Win}_p} \left\| C_{s(q)} -I_R(q_x-d, q_y)\right\|_{2}\,,
  \label{e2-ya}
  \end{equation}
  где $I_R(x,y)$~--- цвет пикселя $(x,y)$ на правом изображении.
  
Используя стоимостную функцию~(2), можно значительно сократить 
вычислительные затраты на построение карты диспарантности~--- стоимостная 
функция для каждой позиции сканирующего окна может быть вычислена за 
$O(1)$ с~помощью частичных сумм.

  Рассмотрим метод частичных сумм. Имеется матрица $A\hm= (a_{i,j})$, 
состоящая из $n\times m$ элементов, и~необходимо многократно вычислять 
сумму подматриц этой матрицы. Пусть
  $$
  \mathrm{ps}_{r,c} =\sum\limits_{i=1}^r \sum\limits_{j=1}^c a_{i,j}\,.
  $$
Используя принцип динамического программирования, величину ps можно 
рассчитать за время порядка $O(n m)$ согласно соотношению:
\begin{equation}
\mathrm{ps}_{r,c}= \mathrm{ps}_{r-1,c} +\mathrm{ps}_{r,c-1} -
\mathrm{ps}_{r-1,c-1}+a_{r,c}\,.
\label{e3-ya}
\end{equation}
  
  Сумму на произвольной подматрице $(r_1,c_1)\hm- (r_2,c_2)$ можно 
вычислить по формуле вклю\-че\-ний-ис\-клю\-че\-ний:
  \begin{multline}
  \mathrm{sum}\left(r_1,c_1,r_2,c_2\right) = {}\\
  {}=\mathrm{ps}_{r_2, c_2} -\mathrm{ps}_{r_2, c_1-1} - \mathrm{ps}_{r_1-1, 
c_2} + \mathrm{ps}_{r_1-1, c_1-1}\,.
  \label{e4-ya}
  \end{multline}
  
  Воспользуемся методом частичных сумм для вычисления стоимостной 
функции $c(p,d)$. Пусть sum$_d(x_1, y_1, x_2, y_2)$~--- суммарная стоимость 
диспарантности~$d$ для всех пикселей прямоугольника $(x_1,y_1)\mbox{--}(x_2,y_2)$, т.\,е.
 \begin{multline*}
  \mathrm{sum}_d \left( x_1, y_1, x_2, y_2\right) = {}\\
  {}=\sum\limits_{i=x_1}^{x_2} 
\sum\limits_{j=y_1}^{y_2} \left\| C_{s(i,j)} -I_R (x-d,y)\right\|_2\,,
  \end{multline*}
  где $s(i,j)$~--- номер сегмента, которому принадле-\linebreak жит пиксель~($i,j$).
  
    Обозначив 
  $$
  \mathrm{ps}_d(x,y)= \sum\limits_{i\leq x} \sum\limits_{j\leq y} \left\| 
C_{s(i,j)} - I_R (x- d, y)\right\|_2\,,
$$
 согласно выражению~(\ref{e4-ya})  имеем:
  \begin{multline*}
  \mathrm{sum}_d\left( x_1, y_1, x_2, y_2\right) ={}\\
  {}= \mathrm{ps}_d\left( x_2, y_2\right)  - \mathrm{ps}_d\left( x_2-
1, y_1-1\right) - {}\\
{}-\mathrm{ps}_d \left( x_1-1, y_2-1\right) +\mathrm{ps}_d \left( x_1-1, y_1-1\right)\,.
  \end{multline*}
  
  С помощью выражения~(\ref{e3-ya}) величину~ps$_d$ можно вычислить за 
время порядка $O(W  H)$. Значение стоимостной функции для 
сканирующего окна размером $s\hm= 2h\hm+1$ выражается через~sum$_d$:

\vspace*{2pt}

\noindent
  $$
  c(p,d) = \mathrm{sum}_d \left( p_x -h, p_y -h, p_x+h, p_y+h\right)\,.
  $$
  
  \vspace*{-2pt}
  
  Таким образом, стоимостная функция для всевозможных позиций 
сканирующего окна при фиксированной диспарантности рассчитывается за 
время~$O(W  H)$; следовательно, вычислительная сложность построения 
карты диспарантности есть $O(W H D)$.
  
  Для определения ошибочных значений диспарантности воспользуемся 
стандартным критерием согласованности. Пусть Disp$_{L,R}$~--- карта 
диспарантности для левого изображения относительно правого, 
а~Disp$_{R,L}$~--- карта диспарантности для правого изображения 
относительно левого. В~соответствии с~приведенным ранее определением, 
значения диспарантности для правого изображения должны быть 
неположительными.
  
  Пусть Disp$_{L,R}(x,y)$ и~Disp$_{R,L} (x^\prime, y)$, где $x^\prime \hm= 
x\hm- \mathrm{Disp}_{L,R}$, являются корректными значениями диспарантности. Тогда 
выполняется равенство:

\noindent
  $$
  \mathrm{Disp}_{L,R} (x,y) =- \mathrm{Disp}_{R,L}\left(x^\prime, y\right)\,.
  $$
На основании этого равенства будем производить фильтрацию карты 
диспарантности: если равенство  для некоторого пикселя не выполняется, будем 
считать, что его диспарантность не установлена.
\begin{table*}
{\small \begin{center}
\Caption{Результаты расчета геодезических расстояний алгоритмом Borgefors}
\vspace*{2ex}

\begin{tabular}{|c|c|c|c|c|}
\hline
\tabcolsep=0pt\begin{tabular}{c}Число\\ итераций\end{tabular}&
Средняя ошибка&
\tabcolsep=0pt\begin{tabular}{c}Максимальная\\ ошибка\end{tabular}&
\tabcolsep=0pt\begin{tabular}{c}Время\\ расчета, с\end{tabular}&
\tabcolsep=0pt\begin{tabular}{c}Время расчета\\ алгоритмом Дейкстры, с\end{tabular}\\
\hline
2&0,0022&0,35&0,22&0,15\\
3&$4{,}5\cdot 10^{-4}$&0,15&0,32&\\
5&$4\cdot  
10^{-5}$&\hphantom{9}0,057 &0,53&\\
10\hphantom{9}&10$^{-8}$&10$^{-4}$&1,05&\\
\hline
\end{tabular}
\end{center}}
\end{table*}
\begin{figure*}[b] %fig2
%\renewcommand{\tablename}{\protect\bf Рис.}
%\setcounter{table}{1}
\vspace*{1pt}
 \begin{center}  
\mbox{%
 \epsfxsize=160mm
 \epsfbox{yak-2.eps}
 }
\end{center} 
\vspace*{-9pt}
\Caption{Результаты стереосопоставления: (\textit{а})~левое изображение стереопары; 
(\textit{б})~результат работы ускоренного алгоритма; (\textit{в})~результат работы 
оригинального алгоритма}
\end{figure*}
  


%  \renewcommand{\figurename}{\protect\bf Рис.}
%\renewcommand{\tablename}{\protect\bf Таблица}

%\addtocounter{figure}{1}
%\addtocounter{table}{-1}

  Ввиду того что при вычислении диспарант\-ности задействуется некоторая 
окрестность пикселя, вполне вероятным является случай незначительного 
рассогласования значений Disp$_{L,R}$ и~Disp$_{R,L}$. Для корректной 
обработки таких случаев следует ослабить критерий согласованности до
  $$
  \left\vert \mathrm{Disp}_{L,R} (x,y) +\mathrm{Disp}_{R,L} 
  \left( x^\prime, y\right) \right\vert \leq 
1\,.
  $$

\section{Вычисление геодезических расстояний}

  В работе~[1] для вычисления геодезических расстояний используется 
итерационный алгоритм Borgefors~\cite{3-ya}, позволяющий найти 
приближение геодезического расстояния с~заданной точностью. 
Вычислительная сложность алгоритма Borgefors линейна относительно размера 
сегмента.
  
  Для точного вычисления геодезических расстояний можно воспользоваться 
алгоритмом Дейкстры, который позволяет вычислить длины кратчайших путей 
от одной вершины графа до остальных. В~качестве вершин графа будем 
использовать пиксели одного сегмента, а~ребро между двумя вершинами будет 
существовать, если соответствующие им пиксели являются смежными в~восьми 
направлениях. Стоимость ребра~--- евклидово расстояние между цветами 
пикселей в~системе RGB. Очевидно, что длина кратчайшего пути в~таком графе 
будет геодезическим расстоянием по определению.
  
  Описанный граф обладает важным свойством: число ребер линейно зависит 
от числа вершин. Такое свойство дает возможность реализовать алгоритм 
Дейкстры с~использованием двоичной кучи~[4], снизив вычислительную 
сложность до $O(\vert V\vert \log \vert V \vert )$, где $\vert V\vert$~--- чис\-ло 
вершин графа, т.\,е.\ размер сегмента.
  
  В ходе экспериментов было выявлено, что асимптотически лучшее 
быстродействие алгоритма Borgefors проявляется лишь на областях достаточно 
большого размера, наличие которых в~сегментации реальных снимков 
маловероятно. В~табл.~1 представлены результаты расчета геодезических 
расстояний алгоритмом Borgefors с~разным числом итераций для изображения 
размером $450\times 375$.




  Таким образом, применение алгоритма Borgefors не оправдано при расчете 
геодезических расстояний внутри сегмента.

\vspace*{-12pt}

\section{Результаты}

  Алгоритм был реализован на языке C++ в~соответствии с~приведенным 
описанием. Реализация алгоритма доступна в~открытом 
репозитории ({\sf https:// github.com/helgui/FastGSW.}). Оценка результата 
стереосопоставления проводилась по методике, описанной в~работе~[5], 
с~по\-мощью предоставленного авторами работы программного обеспе\-чения 
Middlebury Stereo 
Evaluation SDK. %\linebreak\vspace*{-12pt}
  Набор входных данных~[6] состоит из~14~стереопар различной 
конфигурации. В~ходе экспериментов проводилось сравнение ускоренного 
алгорит-\linebreak\vspace*{-12pt}



\pagebreak

\end{multicols}

\begin{table}\small %tabl2
\begin{center}
\Caption{Параметры запуска алгоритмов}
\vspace*{2ex}

\begin{tabular}{|l|c|c|}
\hline
\multicolumn{1}{|c|}{Параметр}&
\tabcolsep=0pt\begin{tabular}{c}Значение\\ для оригинального\\ алгоритма\end{tabular}&
\tabcolsep=0pt\begin{tabular}{c}Значение\\ для ускоренного\\ алгоритма\end{tabular}\\
\hline
Размер сканирующего окна&31&15\\
Коэффициент~$\gamma$&10&5\\
Алгоритм сегментации&Не используется&Сегментация на основе графа~[7]\\
\hline
\end{tabular}
\end{center}
%\end{table*}
%\begin{table*}\small %tabl3
\begin{center}
\Caption{Результаты экспериментов}
\vspace*{2ex}

\begin{tabular}{|l|l|c|c|c|}
\hline
\multicolumn{1}{|c|}{Название теста}&\multicolumn{1}{c|}{
\tabcolsep=0pt\begin{tabular}{c}Используемый \\ алгоритм\end{tabular}}&
\tabcolsep=0pt\begin{tabular}{c}Значения\\ диспарантности\\ с~ошибкой\\ более 2 единиц,
 \%\end{tabular}&
 \tabcolsep=0pt\begin{tabular}{c}Неустановленные\\  значения\\ диспарантности,\\ \%\end{tabular}&
 \tabcolsep=0pt\begin{tabular}{c}Время\\ выполнения,\\ с\end{tabular}\\
\hline
\multicolumn{1}{|l|}{\raisebox{-6pt}[0pt][0pt]{Adirondack}}&
Оригинальный&\hphantom{9}9,98&66,72&562,19\\
&Ускоренный&10,26&60,44&\hphantom{99}6,39\\
\hline
\multicolumn{1}{|l|}{\raisebox{-6pt}[0pt][0pt]{Jadeplant}}&Оригинальный&
\hphantom{9}4,05&40,03&1063,23\hphantom{9}\\
&Ускоренный&11,79&60,23&\hphantom{99}6,31\\
\hline
\multicolumn{1}{|l|}{\raisebox{-6pt}[0pt][0pt]{Motorcycle}}&Оригинальный&
\hphantom{9}5,72&31,98&557,48\\
&Ускоренный&\hphantom{9}6,92&39,03&\hphantom{99}6,58\\
\hline
\multicolumn{1}{|l|}{\raisebox{-6pt}[0pt][0pt]{Motorcycle E}}&Оригинальный&
14,46&82,6\hphantom{9}&556,19\\
&Ускоренный&\hphantom{9}8,31&84,86&\hphantom{99}6,77\\
\hline
\multicolumn{1}{|l|}{\raisebox{-6pt}[0pt][0pt]{Piano}}&Оригинальный&13,71&26,21&478,00\\
&Ускоренный&\hphantom{9}8,55&47,49&\hphantom{99}6,17\\
\hline
\multicolumn{1}{|l|}{\raisebox{-6pt}[0pt][0pt]{Piano L}}&Оригинальный&10,77&81,84&479,30\\
&Ускоренный&\hphantom{9}7,49&84,33&\hphantom{99}6,17\\
\hline
\multicolumn{1}{|l|}{\raisebox{-6pt}[0pt][0pt]{Pipes}}&Оригинальный&
\hphantom{9}5,24&36,68&578,16\\
&Ускоренный&10,7\hphantom{9}&38,23&\hphantom{99}6,45\\
\hline
\multicolumn{1}{|l|}{\raisebox{-6pt}[0pt][0pt]{Playroom}}&Оригинальный&10,19&61,9\hphantom{9}&597,20\\
&Ускоренный&11,08&47,81&\hphantom{99}6,02\\
\hline
\multicolumn{1}{|l|}{\raisebox{-6pt}[0pt][0pt]{Playtable}}&Оригинальный&10,86&42,08&497,01\\
&Ускоренный&10,04&54,52&\hphantom{99}5,67\\
\hline
\multicolumn{1}{|l|}{\raisebox{-6pt}[0pt][0pt]{Playtable P}}&Оригинальный&
\hphantom{9}7,15&36,33&498,76\\
&Ускоренный&\hphantom{9}9,98&54,4\hphantom{9}&\hphantom{99}5,83\\
\hline
\multicolumn{1}{|l|}{\raisebox{-6pt}[0pt][0pt]{Recycle}}&Оригинальный&
\hphantom{9}8,78&31,9\hphantom{9}&491,86\\
&Ускоренный&8,4&53,29&\hphantom{99}6,35\\
\hline
\multicolumn{1}{|l|}{\raisebox{-6pt}[0pt][0pt]{Shelves}}&Оригинальный&16,64&42,83&475,90\\
&Ускоренный&10,19&58,21&\hphantom{99}6,52\\
\hline
\multicolumn{1}{|l|}{\raisebox{-6pt}[0pt][0pt]{Teddy}}&Оригинальный&
\hphantom{9}3,11&30,12&233,55\\
&Ускоренный&7,1&34,48&\hphantom{99}3,36\\
\hline
\multicolumn{1}{|l|}{\raisebox{-6pt}[0pt][0pt]{Vintage}}&Оригинальный&13,06&32,35&
1283,93\hphantom{9}\\
&Ускоренный&7,9&52,3\hphantom{9}&\hphantom{99}6,38\\
\hline
\end{tabular}
\end{center}
\end{table}

\begin{multicols}{2}





\noindent
ма с~оригинальным (рис.~2). Параметры запуска алгоритмов приведены 
в~табл.~2. Следует заметить, что параметры для оригинального алгоритма 
заданы в~соответствии с~рекомендациями авторов. Результаты экспериментов 
сведены в~табл.~3.
  


  Главный критерий оценки качества~--- доля пикселей от общего числа, 
вычисленное значение диспарантности для которых отличается более чем 
на~2~единицы от истинного значения. Разница между средними значениями 
этого параметра для оригинального и~ускоренного алгоритмов 
составила~0,33\%.
  
  При оценке также использовался дополнительный критерий~--- доля 
пикселей, отброшенных при фильтрации. Разница между средними значениями 
составила~8,4\% в~пользу оригинального алго-\linebreak ритма.
  
  Из табл.~3 видно, что ускоренный алгоритм имеет многократное 
превосходство в~производительности. Исходя из приведенных ранее 
асимптотических оценок сложности, нетрудно понять, что разни\-ца во времени 
расчета при увеличении размера сканирующего окна будет расти 
пропорционально квадрату этого размера.

%\vspace*{-6pt}

\section{Заключение}

  В данной работе была описана модификация алгоритма~[1] и~представлена 
его реализация. Результаты экспериментов позволили убедиться в~том, что 
путем незначительной потери качества был получен существенный прирост 
производительности. Также очевидным преимуществом рассмотренного метода 
является эффективность интеграции с~сис\-те\-ма\-ми компьютерного зрения, 
которые уже используют сегментацию при анализе изображений.
  
  Единственным недостатком предложенного подхода является меньшая 
плотность вычисленных карт диспарантности, что может свидетельствовать 
о~недостаточной совместимости алгоритма с~фильт\-ра\-ци\-ей по критерию 
согласованности.

\vspace*{-6pt}
  
{\small\frenchspacing
 {%\baselineskip=10.8pt
 \addcontentsline{toc}{section}{References}
 \begin{thebibliography}{9}
\bibitem{1-ya}
\Au{Hosni A., Bleyer M., Gelautz~M., Rhemann~C.} Local stereo matching using 
geodesic support weights~// 16th IEEE Conference (International) on Image 
Processing.~--- IEEE Press, 2009. P.~2093--2096.
\bibitem{2-ya}
\Au{Shapiro L.\,G., Stockman G.\,C.} Computer vision.~--- Upper Saddle River, NJ, 
USA: Prentice Hall, 2001. 580~p.
\bibitem{3-ya}
\Au{Borgefors G.} Distance transformations in digital images~// Comput. Vision 
Graph. Image Proc., 1986. Vol.~34. No.\,3. P.~344--371.
\bibitem{4-ya}
\Au{Кормен Т.\,Х., Лейзерсон~Ч.\,И., Ривест~Р.\,Л., Штайн~К.} Алгоритмы: 
построение и~анализ~/ Пер. с~англ.~--- 3-е изд.~--- М.: Вильямс, 2013. 1328~с. 
(\Au{Cormen~T.\,H., Leiserson~С.\,E., Rivest~R.\,L., Stein~C.} Introduction to 
algorithms.~--- 3rd ed.~--- MIT Press, 2009. 1312~p.).
\bibitem{5-ya}
\Au{Scharstein D., Szeliski~R.} A~taxonomy and evaluation of dense two-frame 
stereo correspondence algorithms~// IJCV, 2002. Vol.~47. No.\,1/2/3. P.~7--42.
\bibitem{6-ya}
\Au{Scharstein D., Hirschm$\ddot{\mbox{u}}$ller~H., Kitajima~Y., Krathwohl~G., 
Nesic~N., Wang~X., Westling~P.} High-resolution stereo datasets with  
subpixel-accurate ground truth~// German Conference on Pattern Recognition.~--- 
Springer, 2014. P.~31--42.
\bibitem{7-ya}
\Au{Felzenszwalb P.\,F., Huttenlocher~D.\,P.} Efficient graph-based image 
segmentation~// IJCV, 2004. Vol.~59. No.\,2. P.~167--181.
\end{thebibliography}

 }
 }

\end{multicols}

%\vspace*{-6pt}

\hfill{\small\textit{Поступила в~редакцию 14.07.16}}

\vspace*{10pt}

%\newpage

%\vspace*{-24pt}

\hrule

\vspace*{2pt}

\hrule

\vspace*{8pt}



\def\tit{SPEEDED-UP STEREO MATCHING USING~GEODESIC SUPPORT 
WEIGHTS}

\def\titkol{Speeded-up stereo matching using~geodesic support 
weights}

\def\aut{O.\,A.~Yakovlev and A.\,V.~Gasilov}

\def\autkol{O.\,A.~Yakovlev and A.\,V.~Gasilov}

\titel{\tit}{\aut}{\autkol}{\titkol}

\vspace*{-9pt}

\noindent
Orel Branch of the Federal Research Center ``Computer Science and Control'' of the 
Russian Academy of Sciences, 137~Moskovskoe Sh., Orel 302025, Russian 
Federation


\def\leftfootline{\small{\textbf{\thepage}
\hfill INFORMATIKA I EE PRIMENENIYA~--- INFORMATICS AND
APPLICATIONS\ \ \ 2016\ \ \ volume~10\ \ \ issue\ 3}
}%
 \def\rightfootline{\small{INFORMATIKA I EE PRIMENENIYA~---
INFORMATICS AND APPLICATIONS\ \ \ 2016\ \ \ volume~10\ \ \ issue\ 3
\hfill \textbf{\thepage}}}

\vspace*{9pt}


 
\Abste{In local stereo matching, the algorithms based on the adaptive support weights 
have good-quality results. This paper presents a modified version of the local 
matching with geodesic support. The proposed algorithm considerably reduces 
computation time at the cost of insignificant loss of quality that is experimentally 
confirmed with Middlebury Stereo Evaluation SDK. The key idea is to combine 
geodesic support weights and image segmentation for recoloring the reference image. 
This transformation makes it possible to use partial sums for matching cost 
computation.}

\vspace*{3pt}

\KWE{stereo matching; segmentation; geodesic support weights}


\vspace*{3pt}

\DOI{10.14357/19922264160313} 

%\vspace*{-9pt}

%\Ack
%\noindent



\vspace*{18pt}

  \begin{multicols}{2}

\renewcommand{\bibname}{\protect\rmfamily References}
%\renewcommand{\bibname}{\large\protect\rm References}

{\small\frenchspacing
 {%\baselineskip=10.8pt
 \addcontentsline{toc}{section}{References}
 \begin{thebibliography}{9}
\bibitem{1-ya-1}
\Aue{Hosni, A., M. Bleyer, M.~Gelautz, and C.~Rhemann}. 2009. Local stereo 
matching using geodesic support weights. \textit{16th IEEE Conference 
(International) on Image Processing}. IEEE Press. 2093--2096.
\bibitem{2-ya-1}
\Aue{Shapiro, L.\,G., and G.\,C.~Stockman}. 2001. \textit{Computer vision}.  
Upper Saddle River, NJ: Prentice Hall. 580~p.
\bibitem{3-ya-1}
\Aue{Borgefors, G.} 1986. Distance transformations in digital images.  
\textit{Comput. Vision Graph. Image Proc.} 34:344--371.
\bibitem{4-ya-1}
\Aue{Cormen, T.\,H., Сh.\,E.~Leiserson, R.\,L.~Rivest, and C.~Stein}. 2009. 
\textit{Introduction to algorithms}. 3rd ed. MIT Press. 1312~p.

%\pagebreak

\bibitem{5-ya-1}
\Aue{Scharstein, D., and R.~Szeliski}. 2002. A~taxonomy and evaluation of 
dense two-frame stereo correspondence algorithms. \textit{IJCV}  
47(1-3):7--42.
\bibitem{6-ya-1}
\Aue{Scharstein, D., H.~Hirschm$\ddot{\mbox{u}}$ller, Y.~Kitajima, 
G.~Krathwohl, N.~Nesic, X.~Wang, and P.~Westling}. 2014. High-resolution 
stereo datasets with subpixel-accurate ground truth. \textit{German Conference 
on Pattern Recognition}. Springer. 31--42.
\bibitem{7-ya-1}
\Aue{Felzenszwalb, P.\,F., and D.\,P.~Huttenlocher}. 2004. Efficient 
graph-based image segmentation. \textit{IJCV} 59:167--181.
   \end{thebibliography}

 }
 }

\end{multicols}

\vspace*{-3pt}

\hfill{\small\textit{Received July 14, 2016}}

\vspace*{-3pt}

\Contr

\noindent
\textbf{Yakovlev Oleg A.} (b.\ 1992)~--- research engineer, Orel Branch of the 
Federal Research Center ``Computer Science and Control'' of the Russian Academy 
of Sciences, 137~Moskovskoe Sh., Orel 302025, Russian Federation; 
\mbox{maucra@gmail.com}


\vspace*{5pt}

\noindent
\textbf{Gasilov Artur V.} (b.\ 1992)~--- research engineer, Orel Branch of the Federal 
Research Center ``Computer Science and Control'' of the Russian Academy of 
Sciences, 137~Moskovskoe Sh., Orel 302025, Russian Federation; 
\mbox{gasilov.av@ya.ru}

  \label{end\stat}
  
  
  \renewcommand{\bibname}{\protect\rm Литература} %13
\def\stat{fedoseev}

\def\tit{К ВОПРОСУ ОБ УМЕНЬШЕНИИ ОБЪЕМА ПОРЦИЙ УЧЕБНОГО МАТЕРИАЛА 
ПРИ~ЭЛЕКТРОННОМ ОБУЧЕНИИ$^*$}

\def\titkol{К вопросу об уменьшении объема порций учебного материала 
при~электронном обучении}

\def\aut{А.\,А.~Федосеев$^1$}

\def\autkol{А.\,А.~Федосеев}

\titel{\tit}{\aut}{\autkol}{\titkol}

\index{Федосеев А.\,А.}
\index{Fedoseev A.\,A.}


{\renewcommand{\thefootnote}{\fnsymbol{footnote}} \footnotetext[1]
{Работа выполнена в~рамках Программы фундаментальных научных исследований 
в~Российской Федерации на долгосрочный период (2013--2020~годы). Тема №\,34.2. 
Когнитивные мультимедиа и~интерактивность в~образовании в~условиях мобильного 
Интернета.}}


\renewcommand{\thefootnote}{\arabic{footnote}}
\footnotetext[1]{Институт проблем информатики Федерального исследовательского
центра <<Информатика и~управление>> Российской академии наук, 
\mbox{a.fedoseev@ipiran.ru}}



\Abst{Предпринята попытка анализа электронного предъявления учебного 
материала как автоматизированного процесса. Проанализированы причины 
сокращения продолжительности видеолекций для массовых открытых 
онлайновых курсов, а также аналогичного сокращения необходимого времени 
работы с~мультимедиа электронными образовательными ресурсами 
и~параграфами электронных учебников. Показано, что причиной для таких 
сокращений является не столько сама продолжительность, сколько объем 
предъявляемого учебного материала, который может быть усвоен за один 
сеанс. Для определения пределов этого объема количество предъявляемой 
информации, измеренное в~новых понятиях и~связанных с~ними уже усвоенных 
по\-ня\-тий-лин\-ков, сравнивается с~предельным количеством элементов, 
обрабатываемых в~оперативной памяти человека одновременно. В~результате 
делается вывод о~том, что за требованием сокращения продолжительности 
лекций стоит необходимое ограничение объема предъявляемой учебной 
информации. Учет этого обстоятельства позволил сформулировать понятие 
комплекта заданий и~сделать предложение относительно процедуры 
автоматизированного обучения. Статья публикуется в~порядке обсуждения.}

\KW{электронные средства обучения; микрообучение; понятиe; линк; 
<<кошелек Миллера>>; комплект заданий; автоматизированное обучение}

  
\DOI{10.14357/19922264160314} 
  
  \vspace*{3pt}


\vskip 10pt plus 9pt minus 6pt

\thispagestyle{headings}

\begin{multicols}{2}

\label{st\stat}
  
\section{Введение}

  Нормальная продолжительность лекции в~российской высшей школе~--- два 
академических часа, что составляет 90~мин. Бывают сдвоенные лекции 
продолжительностью 180~мин. В~других странах примерно то же самое, но 
предполагается некоторое пространство для маневра продолжительностью 
лекции.
  
  Если вузовскую лекцию записать на видео, то, казалось бы, должна 
получиться основа учебного материала для одного из новых Массовых 
открытых онлайновых курсов (МООК), которые приобрели известность 
и~популярность начиная с~2012~г.~[1]. Чтобы лекционная основа 
превратилась в~заготовку такого курса, следует добавить к~ней механизм 
обратной связи для фиксации факта освоения материала лекции слушателями. 
Для этого в~зависимости от целей и~материала курса применяются различные 
методы.
  
  Однако опыт применения МООК и~электронных ресурсов (в~том числе 
электронных учебников) показал, что продолжительные видеолекции или 
параграфы электронных учебников, тре\-бу\-ющие значительного по времени 
внимания, для дистанционного обучения не годятся. Эмпирически оказалось, 
что учебные материалы должны быть сформированы таким образом, чтобы 
работа с~ними не превышала~15~мин, а~лучше~--- еще меньше. Эта ситуация 
породила термин microlearning (мик\-ро\-обуче\-ние)~[2] и~метафоры 
<<информация на один укус>> и~<<внимание на чайную ложку>> 
(в~примерном переводе). При этом материал для микрообучения~--- это не 
порубленные на части длинные лекции, а~самостоятельные, логически полные 
и~связные короткие видеоролики. Таким образом, считается, что для заочного 
обучения, основанного на ин\-фор\-ма\-ци\-он\-но-ком\-му\-ни\-ка\-ци\-он\-ных 
технологиях (ИКТ), новые знания должны подаваться мелкими порциями 
(microlearning), работа с~каждой из которых занимает не более нескольких 
минут. 
  
  Попытки объяснить необходимость предъявления новых знаний мелкими 
порциями сводятся к~соображению, что при более длинных лекциях внимание 
слушателя рассеивается и~материал перестает усваиваться. Так, исследование 
лекций почти пятисот МООК показало~[3], что интерес к~лекции резко 
уменьшается уже на шестой минуте про\-смот\-ра. Не опровергая это 
исследование, попробуем не согласиться с~тем, что именно продолжительность 
лекции является критической в~дистанционном обучении. Внимание студентов 
точно так же рассеивается и~во время очной лекции в~аудитории, однако 
девяностоминутный стандарт существует столетиями, и~ни\-кто не собирается 
изменять его в~угоду непоседливости студентов. И~это при том, что отвлечение 
внимания во время очной лекции чревато полной утерей понимания изложения 
и невозможностью его восстановления, поскольку лекция существует только 
в~момент ее прочтения. Что касается заочных учебных материалов, 
доставляемых средствами ИКТ, потеря внимания не наносит существенного 
вреда слушателю, поскольку существует возможность получить повторно ту 
часть материала, которая не была воспринята с~первой попытки.  
По-ви\-ди\-мо\-му, дело не только в~продолжительности лекции, а в~чем-то еще.
  
\section{Отличие автоматизированного процесса от~<<ручного>>}

  Предъявление слушателю, студенту или учащемуся учебного материала 
с~использованием средств ИКТ является автоматизированным процессом 
в~отличие от изложения материала лектором или учителем. Учитель или 
лектор, излагая материал, самостоятельно определяет необходимую 
продолжительность цикла обучения, так же как и~условия перехода 
к~изложению следующего материала. Тео\-ре\-ти\-че\-ски это означает, что учитель 
должен убедиться в~готовности учащихся к~дальнейшему учению. На практике 
это не всегда возможно. Учитель не в~состоянии опросить всех учащихся, не 
все из них присутствовали на прошлом уроке, кое-кто, возможно, не выполнил 
домашнее задание. Тем не менее учитель хорошо знаком со своими 
подопечными и~в~состоянии принять решение о~моменте, когда можно 
предъявлять следующий учебный материал в~рамках существующего учебного 
плана. Ситуация в~высшей школе несколько более свободна, поскольку 
предполагается (не всегда обоснованно) ответственное отношение студентов 
к~учению.
  
  Если какая-то операция осуществляется автоматизированно, в~данном случае 
это операция предъявления средствами ИКТ учебного материала слушателям 
(учащимся, студентам) для усвоения, то естественно потребовать наличия 
автоматического сигнала о~ее нормальном завершении. Этот принцип удачно 
реализован в~Академии Салмана Хана ({\sf https://www.khanacademy.org}). 
После короткой видеолекции учащемуся предлагается выполнить ряд заданий. 
При возникновении сложностей предусмотрена возможность воспользоваться 
подсказками. Только после того, как все задания выполнены~--- что является 
автоматическим сигналом о~завершении процесса восприятия и~закрепления 
учебного материала, учащемуся становится доступной следующая порция 
видеоматериалов. 
  
  По договору с~Академией Хана студенты Массачусетского технологического 
института должны были готовить учебные видеоматериалы для их 
использования Академией. В~прочитанной для студентов в~2012~г.\ лекции ({\sf 
https://www.youtube. com/watch?v=VA273i3z7Mk\&nohtml5=False}) Салман Хан, 
в~частности, настаивал, что ролики должны быть по возможности короткими. 
Некоторые лекции ему самому удалось сделать трехминутными, но, 
к~сожалению, не все. То же самое относится и~к~заданиям, выполнение 
которых должно засвидетельствовать освоение материала микролекции. 
Задания также должны быть прос\-ты\-ми и~быстровыполнимыми. Как видим, 
подход Академии Хана вполне согласуется с~концепцией микрообучения.
  
  У организаторов дистанционного курса нет иной возможности понять, 
воспринята ли видеолекция, как получить сигнал о~том, что все задания, 
относящиеся к~этой лекции, выполнены. Поскольку МООК относятся 
к~высшему или дополнительному образованию, заранее предполагается 
большая мотивированность и~ответственность слушателей. Что касается 
электронных ресурсов и~учебников для общеобразовательной школы, то там 
ситуация другая. Полагаться на сознательность и~ответственность учащихся не 
приходится. Поэтому и~методы контроля усвоения знаний должны быть более 
строгими. Очевидно, что учитель может существовать (и~на самом деле 
существует) в~условиях, когда часть его учеников знает весь заданный 
материал, кто-то знает его частично, а некоторые вообще ничего не знают. Он 
примерно представляет уровень знаний каждого ученика и, поскольку их не так 
уж много, может своим индивидуальным вниманием (применяя различные 
педагогические приемы, в~том числе формирующее обучение) до некоторой 
степени компенсировать разноуровневость знаний учащихся.
  
  Если предъявление нового учебного материала передается технологиям, то 
ситуация изменяется и~учащихся приходится подгонять (автоматизированно) 
под единый уровень. В противном случае от автоматизации не будет толка. Что 
с того, что ученикам предъявили некий электронный образовательный  
ресурс~--- кто-то посмотрел и~изучил, а~кто-то и~компьютер не включал,~--- 
подчищать все придется учителю вручную.
  
  Таким образом, автоматизация предъявления учебного материала неизбежно 
тянет за собой и~автоматизированный процесс контроля усвоения этого 
материала. Иначе автоматизированный процесс оказывается незавершенным 
и~восприятие учебного материала приходится проверять традиционными 
способами (контрольный опрос учащихся), т.\,е.\ учебный процесс 
возвращается к~своей традиционной форме и~смысл автоматизации пропадает.
  
\section{Как человек воспринимает новую информацию}

  У каждого человека со временем складывается собственная система знаний. 
Как показано в~[4], сначала закладываются первичные элементарные понятия, 
затем по мере поступления новой информации присутствующие в~ней понятия 
(факты, образы, связи, категории~--- все что угодно) выражаются через уже 
имеющиеся, усвоенные элементы, после чего новая информация встраивается 
в~сис\-те\-му знаний. Для объяснения этого процесса все уже устоявшиеся 
элементы, нужные для объяснения нового понятия, в~[4] предлагается называть 
линками (в переводе~--- связями). Таким образом, каждое новое понятие 
оказывается связанным с~некоторым количеством линков и~в~таком виде 
остается в~долговременной памяти человека. Когда изучается новое понятие, 
например закон Ома, линками являются понятия напряжения, тока 
и~сопротивления. Однако, когда закон Ома встроится в~сис\-те\-му знаний, 
в~памяти он будет представлен одним укрупненным понятием, определяющим 
весь закон. Если обладателю знания закона Ома понадобится использовать его 
для формирования ка\-ко\-го-ли\-бо нового понятия, то этот закон будет 
использован как линк. Более крупным по\-ня\-ти\-ем-лин\-ком может быть вся 
электротехника и~даже вся физика. Здесь важно, что при восприятии новой 
информации каждое понятие обрастает соответствующими линками и~таким 
образом укрупняется. Кстати, при изучении закона Ома учащимся потребуется 
старый комплексный линк под названием <<алгебраические преобразования>>, 
иначе никак не справиться с~вычислением тока или, наоборот, напряжения. 
Естественно, что этот линк уже находится в~памяти как единое целое, 
объединяющее все изученные ал\-геб\-ра\-и\-че\-ские преобразования.
  
  В западной литературе, например в~[5], для объяснения аналогичных 
процессов используется понятие чанка (chunk~--- кусок, ломоть). 
  
  Особенности восприятия информации человеком в~соответствии 
с~открытием, сделанным Джорджем Миллером~[6] (остроумно названным 
<<кошелек Миллера>>), не позволяют обрабатывать в~оператив\-ной памяти 
одновременно более семи плюс-ми\-нус двух элементов. Таким образом, 
количество элементов предназначенного для усвоения учебного материала 
должно быть в~пределах семи (плюс-ми\-нус два). В~[7] на большом 
практическом материале показано, что количество элементов, предъявляемых 
в~новом учебном материале, должно быть не менее трех и~не более пяти. Если 
количество таких элементов три и~менее, то интерес слушателей не 
пробуждается, поскольку они воспринимают материал слишком простым 
и~потому не заслуживающим внимания. Если число элементов учебного 
материала более пяти, то интерес учащихся пропадает уже из-за того, что они 
теряют смысл предъявляемого материала и~не понимают его. Очевидно, что 
количество одновременно обрабатываемых элементов в~диапазоне от трех до 
пяти вполне согласуется с~законом Миллера.
  
  Отождествляя элементы обрабатываемой оперативной памятью человека 
информации с~понятиями и~линками, можно заключить, что эффективное 
восприятие может произойти, если их количество в~новом учебном материале 
не менее трех и~не более пяти. Проблема в~том, что в~понятиях и~линках никто 
учебный материал не измеряет. Интуитивно отмечается, что более короткий 
материал усваивается успешнее, причем проверка усвоения осуществляется 
сравнительно просто. Сколько понятий и~связанных с~ними линков можно 
ввести на лекции продолжительностью~3~мин и~сколько~--- за 90~мин? 
И~далее: как убедиться, что весь~90-ми\-нут\-ный материал усвоен? Каков объем 
выполненных заданий должен убедить в~том, что материал на самом деле 
усвоен? 
  
  Таким образом, теперь можно быть уверенным, что дело не только 
в~регулировании продолжительности лекции, но и~в~ограничении объема 
предъявляемой информации.
  
\section{Понятие комплекта заданий}
  Следующим шагом на пути автоматизации учебного процесса становится 
предъявление учащемуся таких заданий, выполнение которых гарантированно 
свидетельствовало бы о~полном усвоении этого материала. Как упомянуто 
ранее, рекомендации на этот счет касаются только количества и~слож\-ности 
заданий: заданий должно быть по возможности много, но они должны быть 
простыми. Попробуем связать это с~теми понятиями и~линками~[4], количество 
которых определяет предельный объем учебного материала. В~примере про 
закон Ома есть новое понятие~--- закон Ома, есть три линка: напряжение, ток 
и~сопротивление~--- и~есть комплексный линк <<алгебраические 
преобразования>>. Всего~--- пять. Никакие параллельные и~последовательные 
соединения, электродвижущая сила и~полное сопротивление цепи, а также удельное 
сопротивление и~поперечное сечение проводника не должны входить 
в~материал о~законе Ома, поскольку это дополнительные понятия и~им место 
в~дальнейших порциях учебного материала, которые будут предъявляться 
позже. 
  
  Какими должны быть задания, чтобы пол\-ностью удостовериться в~усвоении 
материала <<закон Ома>>? Во-пер\-вых, это вопросы на понимание каж\-до\-го 
нового понятия. Новое понятие одно: закон Ома. Во-вто\-рых, вопросы на 
понимание действия каждого уже известного понятия (линка) в~законе Ома. 
Таких понятий три: напряжение, ток и~сопротивление. У алгебраических 
преобразований нет специфических взаимодействий с~законом Ома. Этот линк 
проявит себя, когда понадобится со\-став\-лять формулы для вычислений.  
В-третьих, вопросы на понимание взаимодействия каждой пары величин 
в~рамках изучаемого закона. Таких пар три: ток и~напряжение, ток 
и~сопротивление, напряжение и~сопротивление. В-чет\-вер\-тых, задания на 
вычисление каждой величины при известных двух %\linebreak
 других. Их тоже три.  
В-пя\-тых, определение\linebreak зависимостей каждой величины от изменений двух 
других. Их может быть шесть, если одна из независимых величин является 
аргументом, а~вторая~--- параметром. Итого в~данном примере насчитывается 
минимум~16~заданий пяти различных типов.\linebreak Почему минимум? Потому что 
меньше нельзя: не все аспекты порции знаний будут проверены. 
А~больше~16~заданий вполне может быть. Например, чтобы проверить 
усвоение новых понятий, может понадобиться более одного вопроса на 
понимание для каждого понятия. Задачи на вычисления могут быть 
сформулированы, что называется, <<в~лоб>>, а могут иметь завуалированную 
структуру, с~тем чтобы учащийся догадался, как решить задачу. 
  
  Таким образом, опираясь на использованные в~порции учебного материла 
понятия и~линки, можно сформировать некоторый набор заданий, которыми 
можно проверить усвоение всех аспектов заключенного в~этой порции знания. 
При этом рекомендации оказываются соблюденными: количество заданий 
существенно превышает обычную норму. Так, количество вопросов для 
самопроверки, размещаемых после параграфа с~учебным материалом, как 
правило, не превышает трех--че\-ты\-рех. Число задач, задаваемых на дом, не 
превышает пяти--шести. И~это при том, что количество информации 
в~параграфах учебников не нормировано понятиями и~связанными с~ними 
линками и~существенно превышает рекомендуемый уровень.
  
  Будем этот необходимый объем заданий, определенный по понятиям 
и~линкам нового учебного материала, называть комплектом. Каждой порции 
учебного материала соответствует свой комплект заданий.
  
  Количество необходимых заданий для выявления полного усвоения 
материала~--- еще один довод в~пользу мелких порций электронных лекций, 
электронных образовательных мультимедиаресурсов и~параграфов 
электронных учебников. При увеличении числа понятий и~линков учебного 
материала стремительно возрастает размер необходимого комплекта заданий, 
поскольку необходимо проверять понимание взаимодействия каждого 
с~каждым.
  
\section{Автоматизированная диагностика }

  Поскольку комплект заданий охватывает проверку усвоения всех аспектов 
соответствующего учебного материала, то возникает возможность 
автоматизиро\-ванной диагностики возникших пробелов в~знании. Таким 
образом, практическое применение понятия комплекта заданий позволяет 
осуществлять автоматизированную диагностику пробелов в~знаниях учащихся. 
Поскольку заданиями охвачены все аспекты учебного материала по 
возможности с~исчерпывающей полнотой, невыполнение отдельных заданий 
с~очевидностью указывает на те разделы материала, к~которым они относятся.
  
\section{Автоматизированное обучение}

  Согласно~[8], обучение~--- это процесс, при котором исправляются ошибки 
восприятия учащихся, не позволившие им выполнить задания с~первой\linebreak 
попытки. В~этом смысле обучение является индивидуальным и~итеративным 
процессом, в~котором на каждой итерации в~соответствии с~допущенными 
учащимися ошибками каждый из них должен получить новый учебный 
материал, специально разработанный для устранения его пробелов 
в~восприятии. После проработки этого материала\linebreak вновь совершается попытка 
выполнить задания. Если задания оказываются снова не выполненными  
в~ка\-кой-то части, то должна осуществиться следующая итерация создания 
нового учебного материала и~предъявления его учащемуся для проработки. 
Этот процесс не применяется в~полной мере на практике в~учебных заведениях, 
поскольку в~них не предусмотрено систематических индивидуальных занятий. 
В~ка\-кой-то степени эти положения теории отрабатываются репетиторами или 
родственниками учащихся.
  
  Разработчик образовательного ресурса или электронного учебника, 
доставляемого средствами ИКТ, имеет возможность заранее предусмотреть 
и~заготовить специальные дополнительные учебные материалы, направленные 
на преодоление непонимания в~выполнении заданий. Рассмотрим для примера, 
какие должны быть заготовлены корректирующие материалы по закону Ома. 
Во-пер\-вых, это материалы, направленные на формирование правильного 
понимания каждого из участвующих в~законе Ома линков. Их три. Это 
напряжение, ток и~сопротивление. Во-вто\-рых, должны также присутствовать материалы, 
корректирующие неправильное понимание пар линков. Их тоже три. 
Неправильное понимание самого закона Ома также должно исправляться 
корректирующим материалом. И~наконец, в-третьих, неправильное использование 
алгебраических %\linebreak 
преобразований должно привести к~отсылке ученика 
к~соответствующему разделу математики. Итого получилось семь 
корректирующих материалов. Надо сказать, что в~той или иной степени эти 
разделы излагались в~основном учебном материале. Речь идет только о~более 
подробном их изложении для полной ликвидации неправильного понимания. 
Эти материалы должны быть предъявлены каждому учащемуся соответственно 
до\-пущенным им ошибкам. После проработки специального материала 
учащийся вновь получает соответств\-ующий комплект заданий. При 
выполнении всех заданий делается заключение об усво\-ении материала, а~при 
невыполнении~--- повторяется процедура предъявления специального учебного 
материала, соответствующего новым допущенным ошибкам. По-ви\-ди\-мо\-му, 
для предотвращения возможности бесконечного цикла следует предусмотреть 
общение с~учителем после нескольких циклов с~одними и~теми же типами 
ошибок.
  
  По существу, описанная процедура является не чем иным, как 
автоматизированным обучением. А~это означает, что электронная доставка 
учебных материалов мелкими порциями с~выделением элементов информации, 
включая комплекты заданий, дает возможность не только убедиться в~усвоении 
или неусвоении предъявленного материала, но и~обеспечить приемлемый 
уровень его усвоения всеми учащимися.
  
\section{Заключение}

  Анализ причин сокращения продолжитель\-ности учебных материалов, 
предъявляемых средствами ИКТ, показал, что дело скорее в~объеме 
предъявляемой информации, которая может быть воспринята за один сеанс, 
чем в~продолжительности как таковой. Организаторы МООК, создатели 
электронных образовательных ресурсов и~электронных учебников интуитивно 
уменьшают порции предъявляемой учебной информации, поскольку 
автоматизированный процесс не позволяет откладывать усвоение материала 
<<на потом>>. Введение в~практику электронного обучения регулирования 
количества по\-ня\-тий-лин\-ков и~формирования комплектов заданий может 
позволить довести процедуры электронного обучения до полного усво\-ения 
материала всеми слушателями или учащимися в~темпе учебного процесса и~тем 
самым повысить эффективность и~результативность методов дистанционного 
обуче\-ния.
  
{\small\frenchspacing
 {%\baselineskip=10.8pt
 \addcontentsline{toc}{section}{References}
 \begin{thebibliography}{9}
\bibitem{1-fed}
\Au{Богданова Д.\,А.} Большой прорыв: от открытых образовательных 
ресурсов~--- к~Массовым Открытым Онлайновым Курсам~// Дистанционное 
и~виртуальное обучение, 2013. №\,4. С.~35--47.
\bibitem{2-fed}
\Au{Fernandez J.} The microlearning trend: Accommodating cultural and 
cognitive shifts~// Learning Solutions Magazine, 2014. December~1. {\sf 
http:// www.learningsolutionsmag.com/articles/1578/the-microlearning-trend-accommodating-cultural-and-cognitive-shifts}.
\bibitem{3-fed}
\Au{Guo P.\,J., Kim J., Rubin~R.} How video production affects student 
engagement: An empirical study of MOOC videos.~--- MIT Computer Science 
and Artificial Intelligence Laboratory, 2014. 10~p. {\sf 
https://groups.csail.mit.edu/uid/other-pubs/las2014-pguo-engagement.pdf}.
\bibitem{4-fed}
\Au{Карпенко М.\,П.} Телеобучение.~--- М.: СГА, 2008. 800~с.
\bibitem{5-fed}
\Au{Chase W.\,G., Simon H.\,A.} Perception in chess~// Cognitive Psychol., 
1973. No.\,4. P.~55--61.
\bibitem{6-fed}
\Au{Miller G.\,A.} The magical number seven, plus or minus two: Some limits on 
our capacity for processing information~// Psychol. Rev., 1956. Vol.~63. 
P.~81--97.
\bibitem{7-fed}
\Au{Петрова В.} Метод 3-4-5, чтобы все запоминать! Освойте новую 
технологию запоминания.~--- Montreal: Accent Graphics Communications, 
2014. 169~с.
\bibitem{8-fed}
\Au{Писарев В.\,Е., Писарева Т.\,Е.} Теория педагогики.~--- Воронеж: 
Кварта, 2009. 611~c.
\end{thebibliography}

 }
 }

\end{multicols}

\vspace*{-6pt}

\hfill{\small\textit{Поступила в~редакцию 27.04.16}}

%\vspace*{8pt}

\newpage

\vspace*{-24pt}

%\hrule

%\vspace*{2pt}

%\hrule

%\vspace*{8pt}



\def\tit{WHAT IS BEHIND THE~CONCEPT OF~``KNOWLEDGE IN~SMALL 
PACKAGES''}

\def\titkol{What is behind the~concept of~``knowledge in small 
packages''}

\def\aut{A.\,A.~Fedoseev}

\def\autkol{A.\,A.~Fedoseev}

\titel{\tit}{\aut}{\autkol}{\titkol}

\vspace*{-9pt}

\noindent
Institute of Informatics Problems, Federal Research Center 
``Computer Science and Control'' of the Russian\linebreak
Academy of Sciences,
44-2~Vavilov Str., Moscow 119333, Russian Federation


\def\leftfootline{\small{\textbf{\thepage}
\hfill INFORMATIKA I EE PRIMENENIYA~--- INFORMATICS AND
APPLICATIONS\ \ \ 2016\ \ \ volume~10\ \ \ issue\ 3}
}%
 \def\rightfootline{\small{INFORMATIKA I EE PRIMENENIYA~---
INFORMATICS AND APPLICATIONS\ \ \ 2016\ \ \ volume~10\ \ \ issue\ 3
\hfill \textbf{\thepage}}}

\vspace*{3pt}


\Abste{An attempt has been made to analyze the electronic presentation of 
educational material as automated process. The reasons for the reduction of the 
length of video lectures for massive open online courses and other educational 
electronic resources as well as the requirement of reducing the time required for 
multimedia electronic educational resources and paragraphs of electronic tutor 
books are analyzed. It is shown that the reason for these requirements is not the 
resource duration, but rather the volume of the educational material that can be 
learned in one session. To determine the limits of this volume the amount of 
presented information, measured in terms of new concepts and related already 
learned concepts-links, was compared with the limited number of items processed 
in the human memory simultaneously. As a result, it is concluded that the 
requirement of reducing the length of lectures is necessary to limit the scope of the 
educational information presented. This circumstance has made it possible to 
formulate the concept of a~task set and to make a proposal for automated training 
procedures. The paper is published in order to discuss this problem.}

\KWE{e-learning tools; microlearning; concept; link; ``Miller purse;'' task set; 
automated training}


\DOI{10.14357/19922264160314} 

\vspace*{-9pt}

\Ack
\noindent
The work was done under the Program of Fundamental Scientific Research 
in the Russian Federation for the long term (2013--2020). 
Subject No.\,34.2. Cognitive multimedia and interactivity in education 
in the conditions of mobile Internet.

%Работа выполнена в~рамках Программы фундаментальных научных исследований 
%в~Российской Федерации на долгосрочный период (2013--2020~годы). Тема №\,34.2. 
%Когнитивные мультимедиа и~интерактивность в~образовании в~условиях мобильного 
%интернета.


%\vspace*{3pt}

  \begin{multicols}{2}

\renewcommand{\bibname}{\protect\rmfamily References}
%\renewcommand{\bibname}{\large\protect\rm References}

{\small\frenchspacing
 {%\baselineskip=10.8pt
 \addcontentsline{toc}{section}{References}
 \begin{thebibliography}{9}
\bibitem{1-fed-1}
\Aue{Bogdanova, D.\,A.} 2013. Bol'shoy proryv: Ot otkrytykh obrazovatel'nykh 
resursov~--- k Massovym Otkrytym Onlaynovym Kursam [The big breakthrough: 
From open educational resources~--- to massive open online courses]. 
\textit{Distantsionnoe i~virtual'noe obuchenie} [Distance and Virtual Learning]  
4:35--47.
\bibitem{2-fed-1}
\Aue{Fernandez, J.} December 1, 2014. The microlearning trend: Accommodating cultural and 
cognitive shifts. \textit{Learning Solutions Magazine}.  Available at: 
{\sf http://www. learningsolutionsmag.com/articles/1578/the-microlearning-trend-accommodating-cultural-and-cognitive-shifts} (accessed March~31, 2016).
\bibitem{3-fed-1}
\Aue{Guo, P.\,J., J.~Kim, and R.~Rubin}. 2014. How video production affects 
student engagement: An empirical study of MOOC videos. MIT Computer 
Science and Artificial Intelligence Laboratory. 10~p. Available at: {\sf 
https://groups.csail.mit.edu/uid/other-pubs/las2014-pguo-engagement.pdf} 
(accessed March~31, 2016).
\bibitem{4-fed-1}
\Aue{Karpenko, M.\,P.} 2008. \textit{Teleobuchenie} [Teleeducation]. Moscow: 
SGA. 800~p.
\bibitem{5-fed-1}
\Aue{Chase, W.\,G., and H.\,A.~Simon}. 1973. Perception in chess. 
\textit{Cognitive Psychol.} (4):55--61.
\bibitem{6-fed-1}
\Au{Miller, G.\,A.} 1956. The magical number seven, plus or minus two: Some 
limits on our capacity for processing information. \textit{Psychol. Rev.} 
63:81--97.
\bibitem{7-fed-1}
\Aue{Petrova, V.} 2014. \textit{Metod 3-4-5, chtoby vse zapominat'!\ Osvoy\-te 
novuyu tekhnologiyu zapominaniya} [Method 3-4-5 to remember everything! 
Master the new memory technology]. Montreal: Accent Graphics 
Communications. 169~p.
\bibitem{8-fed-1}
\Aue{Pisarev, V.\,E., and T.\,E. Pisareva}. 2009. \textit{Teoriya pedagogiki} 
[Pedagogy theory]. Voronezh: KVARTA.  611~p.
   \end{thebibliography}

 }
 }

\end{multicols}

\vspace*{-3pt}

\hfill{\small\textit{Received April 27, 2016}}
  
  \Contrl
  
 \noindent
  \textbf{Fedoseev Andrei A.}\ (b.\ 1946)~--- Candidate of Science (PhD) in
technology, leading scientist,  Institute of Informatics Problems, Federal Research 
Center ``Computer Science and Control'' of the Russian Academy of  
Sciences,~44-2~Vavilova Str., Moscow 119333, Russian Federation; 
\mbox{a.fedoseev@ipiran.ru}

  
\label{end\stat}


\renewcommand{\bibname}{\protect\rm Литература} %14
\def\stat{kolin}

\def\tit{ГУМАНИТАРНЫЕ АСПЕКТЫ ПРОБЛЕМЫ ИНФОРМАЦИОННОЙ БЕЗОПАСНОСТИ}

\def\titkol{Гуманитарные аспекты проблемы информационной безопасности}

\def\aut{К.\,К.~Колин$^1$}

\def\autkol{К.\,К.~Колин}

\titel{\tit}{\aut}{\autkol}{\titkol}

\index{Колин К.\,К.}
\index{Kolin K.\,K.}


%{\renewcommand{\thefootnote}{\fnsymbol{footnote}} \footnotetext[1]
%{Работа выполнена в~рамках Программы фундаментальных научных исследований 
%в~Российской Федерации на долгосрочный период (2013--2020~годы). Тема №\,34.2. 
%Когнитивные мультимедиа и~интерактивность в~образовании в~условиях мобильного 
%Интернета.}}


\renewcommand{\thefootnote}{\arabic{footnote}}
\footnotetext[1]{Институт проблем информатики Федерального исследовательского
центра <<Информатика и~управление>> Российской академии наук, 
\mbox{kolinkk@mail.ru}}

\vspace*{-6pt}

\Abst{Анализируются гуманитарные аспекты проблемы информационной безопасности (ИБ), 
которая рассматривается как важнейший компонент национальной и~глобальной 
безопасности. Показано, что в~современных условиях становления глобального 
информационного общества и~усиления геополитического противоборства 
в~информационном пространстве ИБ государства, человека 
и~общества становится глобальной проблемой дальнейшего развития цивилизации, при 
этом гуманитарные компоненты этой проблемы выдвигаются на первый план. 
Рассмотрена структура гуманитарных проблем ИБ
и~первоочередные меры по их решению в~России.}

\KW{глобальная безопасность; гуманитарные проблемы; информационная безопасность; 
информационная культура; информационная этика; национальная безопасность}

\DOI{10.14357/19922264160315} 
  
%  \vspace*{-6pt}


\vskip 12pt plus 9pt minus 6pt

\thispagestyle{headings}

\begin{multicols}{2}

\label{st\stat}

\section{Информационная безопасность как~гуманитарная 
проблема}

   Исследования показывают, что обеспечение ИБ
государства, человека и~общества сегодня становится одной из 
глобальных и~стратегически важных проблем дальнейшего развития 
цивилизации в~XXI~в., при этом на первый план выдвигаются 
гуманитарные аспекты этой проблемы, которые необходимо обязательно 
учитывать при ее анализе и~решении~[1--3].
   %
   Возрастание роли гуманитарных аспектов данной проблемы обусловлено 
следующими тенденциями развития современного общества:
   \begin{enumerate}[1.]
\item Процесс информатизации общества принял глобальный характер 
и~сегодня охватывает практически все страны и~регионы мира, при этом 
новые средства и~технологии для работы с~информацией получают 
массовое распространение и~становятся атрибутами профессиональной 
и~бытовой культуры для все большей части населения. Их использование 
повышает качество жизни, дает существенную экономию социального 
времени, создает новые стереотипы поведения и~общения миллионов 
людей, изменяет их традиционные представления о~личном 
и~общественном богатстве~[4] и~даже о~пространстве и~времени.\\[-14pt]
\item По оценкам ряда специалистов, человечество вступило 
в~информационную эпоху своего развития~[5], оно активно формирует 
новую среду обитания и~в результате этого само изменяется вместе с~этой 
средой, оказывающей на человека существенно большее влияние, чем 
это ожидалось ранее~[6].
\item Глобальная информатизация создает для государства, человека 
и~общества не только новые возможности, но также и~новые проблемы, 
одной из которых является проблема ИБ. 
Исследования показывают, что эта проблема является многоаспектной 
и~комплексной, а~ее гуманитарные аспекты недостаточно исследованы, не 
учитываются в~принятой ООН стратегии устойчивого развития~[7], 
а~в~системе образования на необходимом уровне не изучаются~[8].
\end{enumerate}

   Гуманитарный аспект проблемы ИБ состоит 
в~том, что именно человек является творцом всех информационных 
ресурсов, систем и~технологий информационного общества. Поэтому их 
качество и~безопасность использования во многом определяются качествами 
самого человека. При этом речь идет не только о~надежности 
и~эффективности работы этих средств и~систем, но и~об их 
воздействии на человека, общество и~окружающую природу.

\begin{table*}[b]\small %tabl1
\begin{center}
\Caption{Структура гуманитарных проблем ИБ}
\vspace*{2ex}

\begin{tabular}{|l|l|}
\hline
\multicolumn{1}{|c|}{Группа проблем}&\multicolumn{1}{c|}{Краткое содержание 
проблемы}\\
\hline
1.\ Геополитические проблемы&\tabcolsep=0pt\begin{tabular}{l}Технологии <<мягкой 
силы>> в~геополитике~[11, 12]\\
Электронная слежка за политическими лидерами\\
<<Глобальное наблюдение>> за населением\\
Информационные и~<<гибридные>> войны~\cite{13-kol} \end{tabular}\\
\hline
2.\ Социальные проблемы&\tabcolsep=0pt\begin{tabular}{l}Информационная 
преступность\\
Информационное неравенство\\
Манипуляции общественным сознанием~\cite{20-kol}\\
Виртуализация общества \end{tabular}\\
\hline
3.\ Культурологические проблемы&\tabcolsep=0pt\begin{tabular}{l}Глобализация 
и~культура\\
Новая информационная культура общества\\
Электронная культура\\
Многоязычие в~киберпространстве \end{tabular}\\
\hline
4.\ Антропологические проблемы&\tabcolsep=0pt\begin{tabular}{l}Энергоинформационная 
безопасность\\
Интеллектуальная безопасность\\
Информационные факторы деструктивного поведения\\
Информационные болезни\\
Информационная видеоэкология\end{tabular}\\
\hline
\end{tabular}
\end{center}
\end{table*}

\vspace*{-7pt}

\section{Онтологическая двойственность гуманитарных проблем 
информационной безопасности и~их антропологические аспекты}

\vspace*{-2pt}

   Исследования показали, что отличительной особенностью гуманитарных 
проблем ИБ является их \textit{онтологическая 
двойственность}. Она состоит в~том, что человек в~этой проблеме выступает 
не только как \textit{объект защиты} от внешних информационных угроз, но 
также и~как основной \textit{источник этих угроз} для своего внешнего 
окружения. 
   %
   Кроме того, в~современных информационных сис\-те\-мах различного 
назначения, слож\-ность которых неуклонно возрастает, именно человек 
становится основным фактором риска для их безопасного 
функционирования. Эта особенность данной проб\-ле\-мы также является 
принципиально важной для ее понимания и~исследования.
   
   Необходимо отметить, что в~последние годы гуманитарные аспекты 
проблемы ИБ стали все\linebreak более заметно проявлять 
себя не только на социальном и~психологическом, но также и~на 
биологическом уровне природы человека. Так, например, исследования 
американских, немецких\linebreak и~российских ученых показали, что воздействие на 
человека интенсивных потоков информации, которые являются 
характерными для информационного общества, приводят к~изменениям 
нейронной структуры головного мозга человека, которые существенным 
образом изменяют его интеллектуальные и~психические способности, 
социальное поведение, коммуникабельность и~самооценку своих 
поступков~[6].
 %  
   Это означает, что проблемы ИБ сегодня 
необходимо изучать комплексно, с~учетом также и~антропологических 
аспектов этих проб-\linebreak лем. 
   
   Автору представляется, что этому должно содействовать формирование 
\textit{информационной антропологии}~--- новой научной дисциплины, 
которая начала изучаться в~России с~2011~г. Структура предметной 
области этой дисциплины рассмотрена в~работах~[9, 10].

\section{Структура гуманитарных проблем информационной 
безопасности}
   
   Структура основных гуманитарных проблем 
ИБ в~сжатом виде представлена в~табл.~1. В~ней отражены 
четыре группы этих проблем, каждая их которых связана с~определенным 
видом деятель\-ности современного общества.



\section{Социальные проблемы информационной безопасности}

   \textbf{Информационная преступность.} В~числе социальных проблем 
ИБ проблема информационной преступности 
стала изучаться одной из первых. При этом она связывалась, главным 
образом, с~проб\-ле\-мой несанкционированного доступа к~информации, 
хранящейся и~циркулирующей в~компьютерных информационных сис\-те\-мах. 
Эта проб\-ле\-ма\linebreak стала проявлять себя уже в~начале~1990-х~гг.\ в~связи 
с~развитием процесса информатизации общества и~его распространением на 
фи\-нан\-со\-во-эко\-но\-ми\-че\-скую сферу. 
   
   Для противодействия этой угрозе достаточно быст\-ро стали создаваться 
различные системы информационной защиты компьютерных сис\-тем и~сетей, 
которые широко используются и~в настоящее время. Тем не менее 
информационная преступность остается актуальной проблемой и~сейчас, 
причем наибольшую опасность представляют уже не столько атаки хакеров 
с~целью хищения финансовых средств из банков, сколько 
несанкционированный доступ к~конфиденциальной информации 
и~персональным данным отдельных категорий граж\-дан в~компьютерных 
системах, их копирование и~последующее распространение.
   
   \textbf{Проблема информационного неравенства.} Анализ основных 
тенденций развития глобального процесса информатизации общества 
показал, что этот процесс создает для развития цивилизации не только новые 
возможности, но также и~новые проблемы. Одной из них является 
\textbf{проблема информационного неравенства}~\cite{17-kol}.
   
    Суть этой проблемы заключается в~том, что в~процессе становления 
информационного общества электронные информационные ресурсы, а~также 
новые средства, сети и~информационные\linebreak техноло\-гии оказываются 
в~различной степени доступными для отдельных людей, организаций, стран 
и~регионов мирового сообщества. При этом те люди, организации, страны 
и~регионы, которые оказываются способными эффективно использовать 
возможности новой информационной среды общества для своего 
интеллектуального развития и~решения других проблем, получают 
существенные преимущества перед другими субъектами мирового 
сообщества, которые при этом вытесняются на обочину процесса развития 
цивилизации.
    
    Так, например, объем продаж товаров и~услуг через сеть Интернет еще 
в~2000~г.\ превысил сумму в~1~трлн долл.\ США. Однако 
основную долю прибыли от этих продаж получили лишь те страны, 
в~которых эта сеть была в~достаточной степени развитой и~доступной для 
населения.
    
    Что же касается современных средств информатики и~новых 
информационных технологий, то их массовое использование создает 
беспрецедентные возможности не только для на\-уч\-но-тех\-ни\-че\-ско\-го, 
но и~для со\-ци\-аль\-но-эко\-но\-ми\-че\-ско\-го развития общества. При этом 
формируется совершенно новый, информационный уклад жизни 
и~производственной деятельности многих миллионов людей.
    
    Системные исследования проблемы информационного неравенства 
проводятся в~Институте проб\-лем информатики РАН уже более 25~лет. Их 
результаты опубликованы в~ряде статей и~монографий~[15--17] 
и~неоднократно докладывались на международных конференциях. 
   % 
    На основе этих результатов в~1997~г.\ в~\mbox{ЮНЕСКО} была направлена 
аналитическая записка, в~которой была представлена российская концепция 
трактовки содержания проблемы информационного неравенства как новой 
комплексной проблемы глобального масштаба. В~ней было показано, что 
принятый в~тот период времени  
ин\-ст\-ру\-мен\-таль\-но-тех\-но\-ло\-ги\-че\-ский подход к~этой проблеме 
является недостаточным, так как он не учитывает целого ряда важных 
факторов гуманитарного характера. В~их числе такие факторы, как уровень 
информационной, в~том числе лингвистической, культуры человека 
и~общества, а~также уровень их общей образованности, который 
в~значительной мере определяет мотивацию активной деятельности людей 
в~новом информационном пространстве.
    
    Дальнейшее развитие процесса информатизации общества показало, что 
эта концепция является более адекватной реальности, и~поэтому она сегодня 
находит все больше сторонников как в~России, так и~в других странах. Так, 
например, если\linebreak в~1997~г., когда в~ежегодном докладе Программы\linebreak развития 
ООН было введено понятие <<информационной бедности>>, она 
определялась исходя из возможностей доступа людей к~современным  
ин\-фор\-ма\-ци\-он\-но-те\-ле\-ком\-му\-ни\-ка\-ци\-он\-ным технологиям, то 
в~2005~г., на втором этапе Международной встречи по проб\-ле\-мам 
глобального информационного общества в~Тунисе, эта проб\-ле\-ма 
трактовалась уже не как проблема <<цифрового разрыва>> (digitaldivide), 
а~именно как \textit{глобальная проблема информационного неравенства}, 
с~учетом указанных выше ее гуманитарных аспектов.
    
     В работе~\cite{19-kol} показано, что в~структуре этой проблемы 
целесообразно различать следующие три основных аспекта.
\begin{enumerate}[1.]
\item \textit{Личностно-социальный аспект}, который связан с~проблемой 
социальной адаптации человека в~новой, быстро изменяющейся 
информационной среде. Именно здесь возникает новая форма социального 
неравенства людей~--- \textit{информационное неравенство}. Снизить 
остроту этой проблемы призвана перспективная система образования, 
которая должна предоставить возможность всем членам общества получать 
необходимые знания и~умения, для того чтобы правильно ориентироваться 
в~новом информационном пространстве и~эффективно использовать его 
возможности.
    \item \textit{Социально-экономический аспект}, который связан 
с~национальной политикой той или иной страны в~области развития 
информационной среды отдельных регионов и~страны в~целом, их 
информационной инфраструктуры, средств и~методов доступа 
к~информационным ресурсам и~информационным коммуникациям, а~также 
в~области развития и~практического использования информационных 
технологий и~информационного законодательства. Решение этих проблем 
должно являться одним из важнейших направлений государственной 
политики в~на\-уч\-но-тех\-ни\-че\-ской, экономической и~социальной сферах 
современного общества.
     \item  \textit{Геополитический аспект}, который связан 
с~неравномерностью развития процесса информатизации в~различных 
странах и~регионах мира, что объясняется не только различиями  
в~на\-уч\-но-тех\-ни\-че\-ском и~экономическом потенциалах\linebreak
 этих стран, но 
также и~уровнем развития об\-разования в~этих странах, а~также степенью 
понима\-ния их политическими лидерами основных тенденций 
и~закономерностей современного этапа развития цивилизации.
    \end{enumerate}
    
При изучении проблемы информационного неравенства в~контексте задач 
обеспечения ИБ необходимо учитывать, что 
процесс информатизации общества оказывает на него как позитивное, так 
и~негативное воздействие. С одной стороны, он повышает эффективность 
общественного производства и~содействует созданию новых рабочих мест, 
в~том числе для людей с~ограниченными возможностями, повышает 
качество жизни населения. Но, с~другой стороны, появление все более 
сложной информационной техники и~технологий, электронных офисов 
и~роботизированных производств требует от людей более высокого уровня 
квалификации и~интеллекта. А~поскольку система образования во многих 
странах не обеспечивает этих требований, происходит дальнейшее 
социальное расслоение общества, которое усиливает в~нем социальную 
напряженность.
%
     Именно поэтому проб\-ле\-ма информационного неравенства и~должна 
сегодня квалифицироваться как одна из актуальных глобальных проблем, 
тесно связанных с~обеспечением национальной и~глобальной без\-опас\-ности.
{\looseness=1

} 
     
     Для решения этой проблемы необходимо проведение адекватной 
государственной и~международной политики в~области развития 
информационной инфраструктуры общества, в~правовой сфере, а~также 
в~сфере образования и~культуры. Во многих странах это сегодня 
осуществляется на уровне целевых национальных программ развития 
информационного общества.
     
    \textbf{Проблема виртуализации общества.} В~последние годы 
    в~обществе стала все более заметной принципиально новая тенденция 
социальных изменений, которая получила название \textit{виртуализации 
общества}. Суть ее заключается в~том, что во многих жизненно важных 
сферах общества~--- в~экономике, политике, культуре, науке и~образовании~--- 
происходит замещение реальных вещей и~действий их симулякрами~--- 
искусственными образами, которые являются лишь символами этих вещей 
и~действий~[18--20].

    Другими словами, современное человечество активно формирует вокруг 
себя новый, иллюзорный мир символов, который существует параллельно с~
реальным физическим миром и~становится такой же неотъемлемой частью 
нашего бытия, как и~физическая реальность.
    
    Казалось бы, ну и~что здесь плохого? Ведь на то и~дано природой 
человеку сознание и~развитое воображение, чтобы он мог при помощи этих 
двух своих особенных качеств моделировать процессы реального мира 
и~таким образом лучше познавать этот мир, прогнозировать возможное 
развитие в~нем различных процессов. 
    
    Оказывается, все гораздо сложнее. Погружаясь все глубже в~мир 
виртуальности, человек не только подменяет реальные вещи и~действия их 
образами и~символами, но также и~\textit{формирует новые ценности}, 
которые затем оказывают влияние на него самого. А~это уже принципиально 
новый со\-ци\-аль\-но-пси\-хо\-ло\-ги\-че\-ский феномен, и,~как показывают 
исследования, его прогнозируемые последствия далеко неоднозначны.
    
    \textbf{Понятие виртуальности.} Термин <<виртуальный>> 
происходит от латинского слова \textit{virtualis}~--- возможный, вероятный, т.\,е.\ 
такой, который может проявиться при определенных условиях, но реально не 
существует~\cite{22-kol}.
     
     В современном русском языке понятие <<виртуальный>> имеет 
несколько смысловых значений. Сначала это понятие использовали физики 
для обозначения элементарных частиц, имеющих очень малое время 
существования. Затем этот термин стал\linebreak проникать на страницы научной 
и~популярной литературы для обозначения искусственной реаль-\linebreak ности, 
создаваемой в~сознании человека при помо-\linebreak щи новейших средств 
компьютерной техники и~киберне\-тических систем. Эта искусственная 
реальность и~получила название \textit{виртуальной реальности}.
{\looseness=-1

}
     
     Однако в~данной работе речь идет о~совсем\linebreak другом феномене, который 
напрямую не связан с~компьютерной техникой и~кибернетическими 
устройствами. Имеется в~виду то новое явление общественной жизни, 
которое проявляется в~устойчивой тенденции отхода все большего числа 
людей от традиционных условий своего существования, основанных на 
личном общении с~другими людьми. Оно подменяется принципиально 
новыми процессами информационных коммуникаций, где присутствуют 
лишь символы и~образы реального мира, которые постепенно заменяют 
человеку этот мир и~все больше изолируют его от этого мира.
    
    \textbf{Виртуализация общества как глобальный процесс.} Феномен 
виртуализации общества стал объектом внимания ученых совсем недавно, не 
более~15~лет тому назад. Попытки его анализа практически 
одновременно предприняли А.~Бюль и~М.~Поэту в~Германии, а~также 
канадские ученые М.~Вейнстен и~А.~Крокер. В~России одним из первых эту 
проблему стал изучать социолог из Санкт-Пе\-тер\-бург\-ско\-го 
государственного университета Д.\,В.~Иванов~\cite{23-kol}. Результаты 
исследований показали, что здесь мы имеем дело с~принципиально новым 
процессом глобального масштаба, который отражает новые трансформации 
в~современном обществе. Эти трансформации еще мало изучены, но уже 
сегодня понятно, что они имеют достаточно серьезные последствия.
     
     Каковы же причины возникновения процесса виртуализации общества? 
На этот счет сегодня существуют различные точки зрения. Западные ученые 
эти причины связывают в~основном с~развитием процессов информатизации 
общества и~все более широким распространением новых информационных 
и~телекоммуникационных технологий. Так, например, согласно мнению 
Бюля, виртуализация общества представляет собой технический процесс 
создания своеобразного \textit{виртуального общества}, которое существует 
как бы <<параллельно>> с~реальным обществом, не оказывая при этом на 
него существенного влияния. 
     
     Принципиально иной позиции придерживается Иванов, который 
считает, что причины виртуализации общества находятся в~нем самом 
и~заключаются в~\textit{изменении социальной природы самого общества}. 
Что же касается информатизации, компьютеризации и~виртуализации 
общества, то эти процессы являются следствиями, а~не причинами 
вышеуказанных изменений. Именно поэтому виртуализация общества 
и~должна рассматриваться как некая глубинная социальная тенденция 
трансформации самого общества, связанная с~общими закономерностями его 
развития, а~вовсе не как результат развития научно-технического прогресса. 
     
     В соответствии с~этой точкой зрения, которую разделяет и~автор 
настоящей работы, изучение процессов виртуализации общества и~их 
возможных последствий является сегодня весьма актуальной проблемой. Ее 
решение позволило бы не только лучше понять существо и~закономерности 
тех глобальных процессов, которые происходят сегодня в~мировом 
сообществе, но также и~выработать рациональную стратегию адаптации 
человека и~общества к~новым условиям их существования в~XXI~в., которые 
становятся все более динамичными.
     
\section{Культурологические проблемы информационной 
безопасности}

\vspace*{-3pt}

     \textbf{Глобализация и~культура.} Исследования показали, что 
процессы глобализации общества оказывают существенное влияние на его 
культуру~\cite{24-kol}. Развитию процессов глобализации общества 
содействует его все более масштабная информатизация, которая несет за 
собой не только новые средства и~технологии стран Запада, но также и~их 
языки, манеру одеваться, стереотипы поведения и~общения. 
     
     В работе~\cite{25-kol} показано, что с~информационной\linebreak точки зрения 
процессы глобализации общества оказывают на него двоякое воздействие. 
С~од-\linebreak ной стороны, развитие информационных коммуникаций существенно 
повышает \textit{информационную \linebreak связанность} мирового сообщества, 
содействует распространению новых знаний и~технологий, способов 
организации производства и~борьбы с~болезнями. И~этот результат является 
позитивным с~точки зрения перспектив дальнейшего безопасного развития 
цивилизации.
     
     Но, с~другой стороны, деградация национальных культур снижает 
уровень \textit{культурного разнообразия} общества, делает его более 
однородным и,~следовательно, менее приспособленным к~противодействию 
глобальным вызовам и~угрозам XXI~в. 
     
     Кроме того, разрушаются духовные ценности национальных культур, 
а~вместо них насаждаются новые ценности потребительского общества. Этот 
процесс является одной из глобальных угроз для безопасного развития 
цивилизации, что уже признается не только российскими, но и~западными 
учеными~\cite{25-kol, 26-kol}.

\begin{table*}\small
\begin{center}
\Caption{Культурологические проблемы ИБ}
\vspace*{2ex}

\tabcolsep=3.7pt
\begin{tabular}{|l|l|}
\hline
\multicolumn{1}{|c|}{Группа проблем}&\multicolumn{1}{c|}{Краткое содержание 
проблемы}\\
\hline
1.\ Глобализация и~культура&\tabcolsep=0pt\begin{tabular}{l}Деградация национальных 
культур\\
Этнос и~нация в~культурологической перспективе\\
Национальное единство в~условиях глобализации\\
Развитие человеческих ресурсов в~информационном обществе\end{tabular}\\
\hline
2.\ Человек в~информационном обществе&\tabcolsep=0pt\begin{tabular}{l}Новая структура 
занятости населения\\
Усиление технократии\\
Новые формы информационного неравенства\\
Урбанизация в~информационном обществе\end{tabular}\\
\hline
3.\ Языки в~новом информационном 
пространстве&\tabcolsep=0pt\begin{tabular}{l}Информационная бедность 
и~лингвистическая культура\\
Сокращение мирового русскоязычного пространства\\
Многоязычие в~киберпространстве\\
Технологии автоматизированного перевода текстов и~речи\end{tabular}\\
\hline
4.\ Электронная культура&\tabcolsep=0pt\begin{tabular}{l}Безопасность электронной 
информационной техники\\
Массовое обучение пользователей\\
Формирование культуры ИБ\end{tabular}\\
\hline
\end{tabular}
\end{center}
\vspace*{6pt}
\end{table*}
     
     \textbf{Новая информационная культура общества.} Проб\-ле\-ма 
формирования новой информационной культуры общества, которая должна 
быть адекватной условиям жизни и~деятельности людей в~новой 
информационной среде их обитания, была по\-став\-ле\-на в~России академиком 
А.\,П.~Ершовым еще в~1988~г.~\cite{26-kol}. Он показал, что эта проблема 
будет глобальной, стратегически важной и~социально значимой для развития 
цивилизации в~XXI~в. Однако системные исследования этой проблемы 
начались лишь в~2011~г., когда в~Германии на русском языке была издана 
первая монография, специально посвященная этой проблеме~\cite{27-kol}. 
В~ней было показано, что для комплексного изучения проблем становления 
и~развития новой информационной культуры общества должна быть 
сформирована специальная научная дисциплина~--- \textit{информационная 
культурология}.
     
     В данной монографии была предложена структура предметной об\-ласти 
этой дисциплины, рассмотрены ее основные задачи и~перспективы\linebreak  развития, 
показана их связь с~проблемами обеспечения~ИБ.
     
     В 2015~г.\ эта монография в~существенно переработанном виде была 
издана и~в~России~\cite{28-kol}. При этом культурологическим аспектам 
проблемы ИБ в~ней посвящен отдельный раздел, 
вклю\-ча\-ющий пять глав. Состав этих проблем представлен в~табл.~2. 
     
     Таким образом, можно утверждать, что Россия сегодня является 
лидером в~об\-ласти изучения проб\-лем информационной культуры 
в~комплексной постановке с~учетом взаимосвязи с~проблемами 
ИБ.



     
\section{Антропологические проблемы информационной 
безопасности}



    \textbf{Энергоинформационная безопасность в~информационном 
обществе.} Современная промышленная и~технологическая революция 
существенным образом изменили энергоинформационное поле нашей 
планеты. Мощные электростанции, крупные\linebreak промышленные производства, 
высоковольтные\linebreak линии электропередачи, городские здания и~сооружения~--- 
все эти объекты создают вокруг себя достаточно интенсивные 
электромагнитные поля, которые постоянно окружают современного 
человека и~воздействуют на его организм. 
    
    Развитие информационного общества усиливает это воздействие 
и~делает его глобальным. Ведь средства телевидения и~мобильной связи 
сегодня имеются практически в~каждой семье и~регулярно используются как 
взрослыми, так и~детьми, причем их количество и~интенсивность 
использования продолжают возрастать. 
    
    Какое воздействие оказывает электромагнитное излучение этих средств 
на организм человека?\linebreak
 Каков допустимый уровень этого воздействия на 
детей, взрослых, а~также на зародышей, еще находящихся в~утробе матери? 
Какими могут быть последствия этого воздействия? На все эти вопросы пока 
нет удовлетворительных ответов, так как данная проблема системно не 
изучается. А~ведь она является глобальной и~представляет серьезную угрозу 
не только для человека, но и~для всей биосферы нашей планеты. 
    
    Так, например, одним из тревожных признаков является сокращение 
численности пчел, которое в~последние годы наблюдается во многих странах, 
но причина его пока не выявлена. Возможно, это связано с~развитием средств 
мобильной связи.
    
    Впервые проблема энергоинформационной без\-опас\-ности была 
поставлена автором настоящей статьи в~работе~\cite{19-kol}. Эта публикация 
не привлекла к~себе внимания специалистов в~области глобальных 
экологических проблем, однако надеяться, что она сама собой решится, 
также нет оснований. Ведь подавляющая часть объектов энергетики и~связи 
находится сегодня в~собственности частных компаний, заинтересованных 
главным образом в~получении прибыли, а~не в~решении проблем 
энергоинформационной безопасности человека и~общества. 
    
    \textbf{Поколение Next и~новая проблема интеллектуальной 
безопасности.} Исследования последних лет показывают, что 
информатизация общества оказывает сильное воздействие не только на 
социальные аспекты повседневной жизни и~профессиональной деятельности 
людей, но также на их психику, образ мышления и~даже на развитие 
головного мозга. Так, например, американские психологи Г.~Смолл 
и~Г.~Ворган в~своей монографии~\cite{6-kol} утверждают, что новое поколение людей 
информационной эпохи, которое уже получило название <<поколения 
Next>>, будет обладать совсем другой психикой и~образом мышления по 
сравнению с~людьми старшего поколения. При этом весьма вероятно, что 
нейронная структура головного мозга у этих людей будет отличаться от той, 
которая существует в~настоящее время.
    
    Свою гипотезу авторы указанной монографии аргументируют 
следующим образом. Согласно тео\-рии эволюции Чарльза Дарвина, развитие 
головного мозга человека происходит в~результате его приспособления 
к~изменениям окружающей среды. Эта общая закономерность действует 
и~сегодня, в~условиях стремительного развития процесса информатизации 
и~формирования глобального информационного общества. А~поскольку 
наиболее радикальные и~быстрые перемены происходят именно 
в~информационной сфере общества, мозг человека начинает 
приспосабливаться к~этим изменениям путем адекватных изменений 
в~организации своей структуры. И~этот феномен является вполне 
закономерным. Вероятнее всего, в~ближайшие годы он будет только 
нарастать.
    
    Проблема здесь заключается в~том, что указанные изменения в~природе 
человека происходят слишком быстро, на протяжении жизни одного 
поколения людей. Для психологов это оказалось полной неожиданностью. 
Ведь таких радикальных изменений природа человека не испытывала 
никогда, а~по своей значимости они сопоставимы, пожалуй, лишь 
с~феноменом появления членораздельной речи. 
    
    Но ведь и~масштабы современной информационной революции также 
являются беспрецедентными в~истории человечества. Их значимость 
и~возможные последствия еще в~необходимой мере не исследованы. Это нам 
еще предстоит сделать в~будущем. 
    
    \textbf{Отличительные черты людей эпохи Интернета.} Всех нас 
удивляет, как быстро и~легко дети осваивают современную достаточно 
сложную информационную технику и~новые информационные технологии. 
Специалисты по возрастной психологии знают, что мозг ребенка является 
очень пластичным, поэтому дети легко осваивают и~новые языки, и~новую 
технику, и~новые стереотипы поведения людей в~информационном обществе. 
При этом у~них вырабатываются совсем другие, отличные от традиционных, 
формы мыслительной дея\-тель\-ности, обусловленные повседневным 
использованием новых информационных технологий. Их мозг становится 
способным к~обработке больших объемов информации, а~также к~быстрой 
реакции на зрительные образы. 
    
    Развитию этих способностей в~значительной мере содействует активное 
использование компьютерных поисковых систем, компьютерные игры, 
а~также общение по электронной почте. Социологические исследования 
развития интеллектуального уровня людей показывают, что IQ среднего 
человека в~XXI~в.\ стремительно растет. 
    
    Вполне возможно, что новые информационные технологии развивают 
интеллект точно так же, как это делают головоломки, игра в~шахматы 
и~изучение новых языков. Наблюдения показывают, что люди, часто 
использующие Интернет, как правило, быстрее находят выход из сложных 
положений и~в повседневной жизни. Ведь каждый день, отыскивая для себя 
в~сети нужную информацию, они тренируют те мозговые центры, которые 
связаны с~оперативным решением практических задач. 
    
    Однако с~развитием интеллекта и~логического мышления у нового 
поколения людей эпохи Интернет не так все однозначно. Здесь есть 
и~достаточно серьезные негативные факторы.
    
    \textbf{Угроза психологического расслоения человечества 
в~информационном обществе.} Исследования показы\-вают, что постоянное 
использование компьютерных информационных технологий влечет за собой 
не только положительные, но и~отрицательные последствия для психики 
человека. Одно из них~--- это так называемое <<клиповое 
мышление>>~\cite{6-kol}. 
    
    Суть этого феномена состоит в~том, что частое использование сети 
Интернет уменьшает способность человека к~концентрации мысли, 
созерцанию и~абстрактному мышлению. Его мозг начинает постепенно 
привыкать к~получению информации в~готовом виде, которую уже не нужно 
анализировать, поэтому и~процесс мышления у~таких людей становится 
фрагментарным, <<клиповым>>. 
    
    Таким образом, вместо мыслителя человек превращается в~своего рода 
сортировщика готовой\linebreak информации, при этом те зоны мозга, которые 
отвеча\-ют за абстрактное мышление, постепенно деградируют, и~в~будущем, 
вполне возможно, они могут совсем атрофироваться. Как же он сможет 
решать те новые глобальные проблемы XXI~в.~\cite{29-kol}, которые, как 
снежная лавина, нарастают уже сегодня? В~этом, по мнению автора, 
и~состоит суть новой глобальной проблемы \textit{интеллектуальной 
без\-опас\-ности}~\cite{17-kol}.
    
    Тревогу вызывает тот факт, что указанные изменения психики чаще 
всего наблюдаются у~молодого поколения людей, вырастающих 
в~современную информационную эпоху. Так, например, в~Японии, одной из 
наиболее информационно развитых стран мира, многие школьники младших 
классов сегодня не умеют считать в~уме, так как вместо этого используют 
калькуляторы в~своих смартфонах или компьютерах. На это уже обратили 
свое внимание японские преподаватели, которые специально заставляют 
таких школьников считать в~уме и~даже сдавать соответствующие экзамены. 
    
    Таким образом, на наших глазах вырастает новое поколение людей, 
которые будут обладать совсем другой психикой и~другим типом мышления. 
Их отличительной чертой будет рассеянное внимание, когда человек следит 
за всем сразу, ни на чем не сосредотачиваясь. Они будут хуже нас общаться 
между собой в~обычной, не компьютерной реальности, так как их мозг будет 
все больше утрачивать те базовые механизмы, которые управляют 
контактами с~другими людьми. 
    
    Их память будет все меньше использоваться для запоминания 
фактографической и~другой информации, так как ее <<кибернетическими 
протезами>> станут персональные компьютеры, смартфоны и~электронные 
базы данных сети Интернет. Эти люди, вероятнее всего, будут запоминать не 
саму информацию, а~метаинформацию, т.\,е.\ информацию о~том, в~какой 
папке компьютерной памяти она хранится или же в~какой электронной 
библиотеке ее можно найти. 
    
    Следует ожидать, что представители <<поколения Next>> будут еще 
меньше, чем наши современники, читать художественную литературу, 
в~особенности классическую, историческую и~научную. Зачем им это делать, 
если есть Интернет и~Википедия? Теат\-ры, консерватории и~музеи, скорее 
всего, им также будут неинтересны. Ведь с~их содержанием можно будет 
познакомиться в~электронной сети, не выходя из дома. 
    
    В информационном обществе важную роль играет \textit{сетевое 
общение}, которое существенным образом расширяет возможность контактов 
с~другими людьми, повышает их оперативность и~экономит массу 
социального времени. Но ведь при этом возникает риск забыть о~том, что на 
самом деле представляет собой дружба между людьми в~реальном мире, для 
которой необходимо реальное общение. 
    
    Вполне возможно, что в~информационном обществе появится также 
и~\textit{новый вид одиночества}. Это ситуация, когда телевизор и~другие 
средства информационной техники выключены и~человек остается один 
в~уже мало привычном для него реальном мире. Ведь уже давно известно, 
что нигде люди не чувствуют себя так одинокими, как в~большом городе, 
когда они часто не знакомы даже с~теми, кто живет в~соседней квартире. 
    
    \textbf{Информационные факторы деструктивного поведения 
человека.} В~числе новых направлений исследования антропологических 
аспектов проблем ИБ следует отметить работы 
российского композитора В.\,С.~Дашкевича, где рассматривается влияние на 
деятельность головного мозга человека той акустической и,~в~част\-ности, 
музыкальной среды, в~которой он обитает. В~них показано, что одной из 
причин повышения уровня деструктивности поведения людей в~современном 
обществе, является его \textit{музыкальная культура}~\cite{30-kol}. 
     
     Таким образом, музыкальная культура общества также должна стать 
одним из объектов тех перспективных исследований, которые должны 
проводиться в~интересах изучения гуманитарных проблем ИБ.
    
    \section{Заключение}
     
     \textbf{Доктрина и~Стратегия информационной безопасности 
Российской Федерации.} В~настоящее время концептуальные основы 
ИБ России определяет Доктрина 
информационной без\-опас\-ности РФ, принятая еще в~2000~г. По своему 
содержанию она представляет собой развернутый и~достаточно хорошо 
продуманный документ, многие положения которого являются актуальными 
и~в~настоящее время. Тем не менее он требует корректировки, так как за 
последние годы ситуация в~данной области изменилась в~худшую для России 
сторону и, кроме того, появились новые информационные вызовы и~угрозы 
как национального, так и~глобального масштаба. 
     %
     Поэтому еще в~декабре 2015~г.\ Советом Безопасности РФ был 
подготовлен проект новой Доктрины информационной безопасности, 
утверждение которой ожидается в~2016~г. Однако до сих пор текст этого 
документа не опубликован и~на общественном уровне еще не обсуждался.
     
     Представляется, что для эффективного противодействия комплексу 
современных угроз для ИБ России необходимо 
также разработать и~принять \textit{Стратегию информационной 
безопасности РФ на период до 2030~г.} В~ней должны быть определены 
конкретные задачи в~этой области, сроки их решения и~количественные 
показатели, необходимое правовое, организационное и~кадровое 
обеспе\-чение. 
     
     Отметим, что вопрос о~необходимости разработки и~принятия 
Стратегии информационной безопас\-ности России на среднесрочный период 
неоднократно ставился Институтом проблем информатики РАН как 
в~научных публикациях, так и~в~Аналитических материалах, которые 
Российская академия наук представляла для включения в~ежегодный Доклад 
Президенту России <<О~состоянии национальной безопасности РФ и~мерах 
по ее укреплению>>. Сегодня пришло время, когда Стратегия 
информационной безопасности России является крайне необходимой.
     
     Необходимо также пересмотреть содержание научных дисциплин ВАК 
РФ по тематике ИБ, предусмотрев в~них 
наиболее актуальные гуманитарные аспекты, некоторые из которых были 
рассмотрены в~данной работе.

{\small\frenchspacing
 {%\baselineskip=10.8pt
 \addcontentsline{toc}{section}{References}
 \begin{thebibliography}{99}
 \bibitem{2-kol} %1
\Au{Колин К.\,К.} Информационная безопасность как гуманитарная проб\-ле\-ма~// 
Открытое образование, 2006. №\,1(54). С.~86--93.
\bibitem{1-kol} %2
\Au{Соколов И.\,А., Колин К.\,К.} Развитие информационного общества в~России и~
актуальные проблемы информационной безопасности~// Информационное общество, 
2009. №\,4-5. С.~98--106.

\bibitem{3-kol}
\Au{Колин К.\,К.} Гуманитарные проблемы информационной безопасности~// 
Информационные технологии, 2012. №\,12. Приложение. С.~1--32.
\bibitem{4-kol}
\Au{Тоффлер Э.} Революционное богатство.~--- М.: ACT Москва, 2008. 569~с.
\bibitem{5-kol}
\Au{Кастельс М.} Информационная эпоха: экономика, общество и~культура.~--- М.: 
ГУ ВШЭ, 2000. 458~с.
\bibitem{6-kol}
\Au{Смолл Г., Врган Г.} Мозг онлайн. Человек в~эпоху Интернета.~--- М.: Колибри, 
2011. 352~с.
\bibitem{7-kol}
\Au{Колин К.\,К.} Половинчатая стратегия: критический анализ новой Стратегии ООН 
в~об\-ласти устойчивого развития~// Партнерство цивилизаций, 2016. №\,1-2.  
С.~33--41.
\bibitem{8-kol}
\Au{Соколов И.\,А., Колин К.\,К.} Новый этап информатизации общества и~актуальные 
проблемы образования~// Информатика и~её применения, 2008. Т.~2. Вып.~1.  
С.~67--76.
\bibitem{9-kol}
\Au{Колин К.\,К.} Информационная антропология: поколение Next и~угроза 
психологического расслоения человечества в~информационном обществе~// Вестник 
Челябинской государственной академии культуры и~искусств, 2011. №\,4. С.~32--36.
\bibitem{10-kol}
\Au{Колин К.\,К.} Информационная антропология: предмет и~задачи нового 
направления в~науке и~образовании~// Вестник Кемеровского государственного ун-та 
культуры и~искусств, 2011. №\,17. С.~17--32.
\bibitem{11-kol}
\Au{Смирнов А.\,И., Кохтюлина И.\,Н.} Глобальная безопасность и~<<мягкая сила 
2.0>>: вызовы и~возможности для России.~--- М.: ВНИИГеосистем, 2012. 276~с.
\bibitem{12-kol}
\Au{Шабалов М.\,П.} <<Мягкая сила>> в~современной геополитике~// Стратегические 
приоритеты, 2014. №\,4. С.~27--43.
\bibitem{13-kol}
\Au{Кошкин Р.\,П.} Россия и~мир: новые приоритеты в~геополитике.~--- М.: 
Стратегические приоритеты, 2015. 236~с.
%\bibitem{14-kol}
%\Au{Колин К.\,К.} Новая Военная доктрина и~гуманитарные проблемы национальной 
%безопасности России~// Стратегические приоритеты, 2015. №\,1(5). С.~30--47.
%\bibitem{15-kol}
%Стратегия национальной безопасности Российской Федерации. Утверждена Указом 
%Президента РФ от~31~декабря 2015~г. №\,683.
%\bibitem{16-kol}
%\Au{Бетелин В.\,Б.} О проблеме импортозамещения и~альтернативной модели 
%экономического развития России~// Стратегические приоритеты, 2016. №\,1(9).  
%С.~11--21.
\bibitem{20-kol} %14
\Au{Кара-Мурза С.\,Г.} Манипуляции сознанием.~--- М.: Алгоритм, 2000. 688~с.
\bibitem{17-kol} %15
\Au{Колин К.\,К.} Информационное неравенство~--- новая проблема XXI~века~// 
Социология, социальность, современность.~--- М.: Союз, 1998. Вып.~5. 
С.~99--101.

\bibitem{19-kol} %16
\Au{Колин К.\,К.} Информационные проб\-ле\-мы  
со\-ци\-аль\-но-эко\-но\-ми\-че\-ско\-го развития общества.~--- М.: Союз, 1995. 72~с.
\bibitem{18-kol} %17
\Au{Колин К.\,К.} Глобальные проблемы информатизации: информационное 
неравенство~// Alma mater (Вестник высшей школы), 2000. №\,6. С.~27--32.
\bibitem{23-kol} %18
\Au{Иванов Д.\,В.} Виртуализация общества.~--- СПб.: Петербургское востоковедение, 
2000. 96~с.

\bibitem{21-kol} %19
\Au{Колин К.\,К.} Виртуализация общества~--- новая угроза для его стабильности~// 
Синергетическая парадигма. Человек и~общество в~условиях нестабильности.~--- 
М.: РАГС, 2003. С.~449--462.
\bibitem{22-kol} %20
\Au{Колин К.\,К.} Виртуализация общества~// Большая Российская энциклопедия, 
2006. Т.~5. С.~370.

\bibitem{24-kol} %21
\Au{Колин К.\,К.} Глобализация и~культура~// Вестник Биб\-ли\-о\-течной Ассамблеи 
Евразии, 2004. №\,1. С.~12--15.
\bibitem{25-kol} %22
\Au{Колин К.\,К.} Информатизация общества и~глобализация.~--- Красноярск, 
СФУ, 2011. 52~с.
\bibitem{26-kol} %23
\Au{Ершов А.\,П.} Информатизация: от компьютерной грамотности школьников 
к~информационной культуре общества~// Коммунист, 1988. №\,2. С.~82--92. 
\bibitem{27-kol} %24
\Au{Колин К.\,К., Урсул А.\,Д.} Информационная культурология: предмет и~задачи 
нового научного направления.~--- Saarbruchen, Germany: LAP Lambert Academic 
Publishing, 2011. 249~с.
\bibitem{28-kol} %25
\Au{Колин К.\,К., Урсул А.\,Д.} Информация и~культура. Введение в~информационную 
культурологию.~--- М.: Стратегические приоритеты, 2015. 300~с.
\bibitem{29-kol} %26
\Au{Колин К.\,К.} Глобальные угрозы развитию цивилизации в~XXI~веке~// 
Стратегические приоритеты, 2014. №\,1. С.~6--30.
\bibitem{30-kol} %27
\Au{Дашкевич В.\,С.} Великое культурное одичание. Арт-ана\-лиз.~--- М.: Russian 
Chess House, 2013. 717~с.
\end{thebibliography}

 }
 }

\end{multicols}

\vspace*{-3pt}

\hfill{\small\textit{Поступила в~редакцию 18.07.16}}

\vspace*{8pt}

\newpage

\vspace*{-30pt}

%\hrule

%\vspace*{2pt}

%\hrule

%\vspace*{8pt}



\def\tit{HUMANITARIAN ASPECTS OF~INFORMATION SECURITY}

\def\titkol{Humanitarian aspects of information security}

\def\aut{K.\,K.~Kolin}

\def\autkol{K.\,K.~Kolin}

\titel{\tit}{\aut}{\autkol}{\titkol}

\vspace*{-9pt}

\noindent
Institute of Informatics Problems, Federal Research Center 
``Computer Science and Control'' of the Russian\linebreak
Academy of Sciences,
44-2~Vavilov Str., Moscow 119333, Russian Federation


\def\leftfootline{\small{\textbf{\thepage}
\hfill INFORMATIKA I EE PRIMENENIYA~--- INFORMATICS AND
APPLICATIONS\ \ \ 2016\ \ \ volume~10\ \ \ issue\ 3}
}%
 \def\rightfootline{\small{INFORMATIKA I EE PRIMENENIYA~---
INFORMATICS AND APPLICATIONS\ \ \ 2016\ \ \ volume~10\ \ \ issue\ 3
\hfill \textbf{\thepage}}}

\vspace*{3pt}

\Abste{The paper analyzes humanitarian aspects of information security, which is 
regarded as the most important component of national and global security. It is 
shown that in modern conditions, the formation of the global information society 
and strengthening of geopolitical confrontation in the information space of 
information security of state, individual, and society is becoming a global problem 
for the further development of civilization. The humanitarian component of this 
problem comes to the forefront. The structure of humanitarian problems of 
information security is described. Priority measures for their solution in Russia are 
proposed.}

\KWE{global security; humanitarian issues; information security; information 
culture; information ethics; national security}

\DOI{10.14357/19922264160315} 

%\vspace*{-9pt}

%\Ack
%\noindent


%\vspace*{3pt}

  \begin{multicols}{2}

\renewcommand{\bibname}{\protect\rmfamily References}
%\renewcommand{\bibname}{\large\protect\rm References}

{\small\frenchspacing
 {%\baselineskip=10.8pt
 \addcontentsline{toc}{section}{References}
 \begin{thebibliography}{99}
      
      
      \bibitem{2-kol-1}
      \Aue{Kolin, K.\,K.} 2006. Informatsionnaya bezopasnost' kak gumanitarnaya problema 
[Information security as a humanitarian problem]. \textit{Otkrytoe Obrazovanie} [Open 
Education] 1:86--93.

\bibitem{1-kol-1}
      \Aue{Sokolov, I.\,A., and K.\,K.~Kolin}. 2009. Razvitie informatsionnogo obshchestva 
v~Rossii i~aktual'nye problemy informatsionnoy bezopasnosti [Information society 
development in Russia and problems of information security]. \textit{Informatsionnoe 
Obshchestvo} [Information Society] 4-5:98--106.

      \bibitem{3-kol-1}
      \Aue{Kolin, K.\,K.} 2012. Gumanitarnye problemy infor\-ma\-tsi\-onnoy bezopasnosti [Since 
the humanitarian problems of information security]. \textit{Informatsionnye Tekhnologii}  
[Information Technology] 12(App.):1--32.
\bibitem{4-kol-1}
\Aue{Toffler, E.} 2008. \textit{Revolyutsionnoe bogatstvo} [Revolutionary wealth]. Moscow: 
ACT Moskva. 569~p.
      \bibitem{5-kol-1}
      \Aue{Kastel's, M.} 2000. \textit{Informatsionnaya epokha: Ekonomika, obshchestvo 
i~kul'tura} [The information age: Economy, society, and culture]. Moscow: GU VShE. 458~p.
      \bibitem{6-kol-1}
      \Aue{Smoll, G., and G.~Vrgan}. 2011. \textit{Mozg onlayn. Chelovek v~epokhu 
Interneta} [Brain online. People in the Internet age]. Moscow: Kolibri. 352~p.
      \bibitem{7-kol-1}
      \Aue{Kolin, K.\,K.} 2016. Polovinchataya strategiya: Kriticheskiy analiz novoy Strategii 
OON v~oblasti ustoychivogo razvitiya [Since a half-hearted strategy: A~critical analysis of the 
new UN Strategy on sustainable development]. \textit{Partnerstvo Tsivilizatsiy}  
[Partnership of  Civilizations] 1-2:33--41.
      \bibitem{8-kol-1}
      \Aue{Sokolov, I.\,A., and K.\,K.~Kolin}. 2008. Novyy etap informatizatsii obshchestva 
i~aktual'nye problemy obrazovaniya 
[The new stage of the society informatization and actual 
problems of education]. \textit{Informatika i~ee Primeneniya~--- Inform. Appl.} 
2(1):67--76.
      \bibitem{9-kol-1}
      \Aue{Kolin, K.\,K.} 2011. Informatsionnaya antropologiya: Pokolenie Next i~ugroza 
psikhologicheskogo rassloeniya chelovechestva v~informatsionnom obshchestve [Information 
anthropology: The Next generation and the threat of psychological stratification of humanity in 
the information society]. \textit{Vestnik Chelyabinskoy gosudarstvennoy akademii kul'tury 
i~iskusstv}  [Bulletin of the Chelyabinsk State Academy of Culture and Arts] 4:32--36.
      \bibitem{10-kol-1}
      \Aue{Kolin, K.\,K.} 2011. Informatsionnaya antropologiya: Predmet i~zadachi novogo 
napravleniya v~nauke i~obrazovanii [Information anthropology: The subject and objectives of the 
new direction in science and education]. \textit{Vestnik Kemerovskogo gosudarstvennogo un-ta 
kul'tury i~iskusstv}  [Bulletin of Kemerovo State University of Culture and Arts] 17:17--32.
      \bibitem{11-kol-1}
      \Aue{Smirnov, A.\,I., and I.\,N.~Kokhtyulina}. 2012. Global'naya bezopasnost' 
i~``myagkaya sila 2.0:'' Vyzovy i~vozmozhnosti dlya Rossii [Global security and ``soft power 
2.0:'' Challenges and opportunities for Russia]. Moscow: \mbox{VNIIGeosistem}. 276~p.
      \bibitem{12-kol-1}
      \Au{Shabalov, M.\,P.} 2014. ``Myagkaya sila'' v~sovremennoy geopolitike [``Soft 
power'' in contemporary geopolitics]. \textit{Strategicheskie Prioritety}  [Strategic Priorities] 
4:27--43.
      \bibitem{13-kol-1}
      \Au{Koshkin, R.\,P.} 2015. Rossiya i~mir: Novye prioritety v~geopolitike [Russia and the 
world: New priorities in geopolitics]. Moscow: Strategicheskie prioritety [Strategic Priorities]. 
236~p.
     % \bibitem{14-kol-1}
%      \Aue{Kolin, K.\,K.} 2015. Novaya voennaya doktrina i~gumanitarnye problemy 
%natsional'noy bezopasnosti Rossii [Since the New Military doctrine and humanitarian problems 
%of national security of Russia]. \textit{Strategicheskie Prioritety}  [Strategic Priorities]  
%1(5):30--47.
%      \bibitem{15-kol-1}
%       Strategiya natsional'noy bezopasnosti Rossiyskoy Federatsii. 
%       2015. Utverzhdena Ukazom 
%Prezidenta RF ot~31~dekabrya 2015~g. No.\,683 [The national security strategy of the Russian 
%Federation. Approved by the Decree of the President of the Russian Federation from 
%December~31, 2015, No.\,683].
%      \bibitem{16-kol-1}
%      \Aue{Betelin, V.\,B.} 2016. O~probleme importozameshcheniya i~al'ternativnoy modeli 
%ekonomicheskogo razvitiya Rossii [On the problem of import substitution and alternative models 
%of economic development of Russia]. \textit{Strate\-gi\-che\-skie prioritety} [Strategic Priorities] 
%1(9):11--21.
 \bibitem{20-kol-1} %14
      \Aue{Kara-Murza, S.\,G.} 2000. \textit{Manipulyatsii soznaniem} [Manipulation of 
consciousness]. Moscow: Algoritm. 688~p.
      \bibitem{17-kol-1} %15
      \Aue{Kolin, K.\,K.} 1998. Informatsionnoe neravenstvo~--- novaya problema 
XXI~veka [Information inequality is a new problem of the XXI century]. \textit{Sotsiologiya, 
sotsial'nost', sovremennost'} [Sociology, sociality, modernity]. 
Moscow: Soyuz. 5:99--101.
 \bibitem{19-kol-1} %16
      \Aue{Kolin, K.\,K.} 1995. \textit{Informatsionnye problemy sotsial'no-ekonomicheskogo 
razvitiya obshchestva} [Information problems of socio-economic development of society]. 
Moscow: Soyuz. 72~p.
      \bibitem{18-kol-1} %17
      \Aue{Kolin, K.\,K.} 2000. Global'nye problemy informatizatsii: Informatsionnoe 
neravenstvo [The global problem of informatization: Information inequality]. \textit{Alma mater 
(Vestnik vysshey shkoly)}  [Alma Mater (Bulletin of High School)] 6:27--32.
     
     
\bibitem{23-kol-1} %18
      \Aue{Ivanov, D.\,V.} 2000. Virtualizatsiya obshchestva 
      [Virtualization of society]. Saint Petersburg: 
Peterburgskoe vostokovedenie. 96~p.
      \bibitem{21-kol-1} %19
      \Aue{Kolin, K.\,K.} 2003. Virtualizatsiya obshchestva~-- novaya ugroza dlya ego 
stabil'nosti [Since virtualization companies~--- a new threat to its stability]. 
\textit{Sinergeticheskaya paradigma. Chelovek i~obshchestvo v~usloviyakh 
nestabil'nosti} [Synergetic paradigm. Man and society in conditions of instability]. 
Moscow: RAGS. 449--462.
      \bibitem{22-kol-1} %20
      \Aue{Kolin, K.\,K.} 2006. Virtualizatsiya obshchestva [Virtualization of society]. 
\textit{Bol'shaya Rossiyskaya entsiklopediya} [Great Russian Encyclopedia]. 5:370.
      
      \bibitem{24-kol-1} %21
      \Aue{Kolin, K.\,K.} 2004. Globalizatsiya i~kul'tura [Globalization and culture]. 
\textit{Vestnik Bibliotechnoy Assamblei Evrazii}  
[Bulletin of the Library Assembly of Eurasia] 
1:12--15.
      \bibitem{25-kol-1} %22
      \Aue{Kolin, K.\,K.} 2011. \textit{Informatizatsiya obshchestva i~glo\-ba\-li\-za\-tsiya}
      [Information society and globalization]. Krasnoyarsk: SFU. 52~p.
      \bibitem{26-kol-1} %23
      \Aue{Ershov, A.\,P.} 1998. Informatizatsiya: Ot komp'yuternoy gramotnosti shkol'nikov 
k~informatsionnoy kul'ture obshchestva [Informatization: From computer literacy to information 
culture society]. \textit{Kommunist} [Communist] 2:82--92.
      \bibitem{27-kol-1} %24
      \Aue{Kolin, K.\,K., and A.\,D.~Ursul}. 2011. \textit{Informatsionnaya kul'turologiya: 
Predmet i~zadachi novogo nauchnogo napravleniya} [Information studies: Subject and problems 
of a new scientific direction]. Saarbruchen, Germany: LAP Lambert Academic Publishing. 
249~p.
      \bibitem{28-kol-1} %25
      \Aue{Kolin, K.\,K., and A.\,D.~Ursul}. 2015. \textit{Informatsiya i~kul'tura. Vvedenie 
v~informatsionnuyu kul'turologiyu} [Information and culture. Introduction to information 
studies]. Moscow: Strategicheskie prioritety. 300~p.
      \bibitem{29-kol-1} %26
      \Aue{Kolin, K.\,K.} 2014. Global'nye ugrozy razvitiyu tsivilizatsii v~XXI~veke [The 
global threat to the development of civilization in the XXI~century]. \textit{Strategicheskie 
Prioritety}  [Strategic Priorities] 1:6--30.
      \bibitem{30-kol-1} %27
      \Aue{Dashkevich, V.\,S.} 2013. \textit{Velikoe kul'turnoe odichanie. Art-analiz} [The 
Great cultural savagery. Art-analysis]. Moscow: Russian Chess House. 717~p.
   \end{thebibliography}

 }
 }

\end{multicols}

\vspace*{-3pt}

\hfill{\small\textit{Received July 18, 2016}}

\Contrl

\noindent
\textbf{Kolin Konstantin K.} (b.\ 1935)~--- Doctor of Science in technology, professor, Honored scientist 
of RF, principal scientist, Institute of Informatics Problems, Federal Research Center ``Computer Science 
and Control'' of the Russian Academy of Sciences, 44-2~Vavilov Str., Moscow 119333, Russian 
Federation; \mbox{kolinkk@mail.ru}
\label{end\stat}


\renewcommand{\bibname}{\protect\rm Литература} %15



%%%%%%%%%%%%%%%%%%%%%%%%%%%%%%%%%%%%%%%%%%%%%%%

%\def\stat{rez}
{%\hrule\par
%\vskip 7pt % 7pt
\raggedleft\Large \bf%\baselineskip=3.2ex
Р\,Е\,Ц\,Е\,Н\,З\,И\,И \vskip 17pt
    \hrule
    \par
\vskip 6pt plus 6pt minus 3pt }

%\thispagestyle{headings} %с верхним колонтитулом
%\thispagestyle{myheadings} %с нижним колонтитулом, но в верхнем РЕЦЕНЗИИ

\def\tit{НОВАЯ КНИГА И.\,Н.~СИНИЦЫНА, А.\,С.~ШАЛАМОВА <<ЛЕКЦИИ ПО ТЕОРИИ 
ИНТЕГРИРОВАННОЙ ЛОГИСТИЧЕСКОЙ ПОДДЕРЖКИ>> (М.: ТОРУС ПРЕСС, 2012. 624~с.)}

%1
\def\aut{Д.ф.-м.н., профессор С.\,Я.~Шоргин}

\def\auf{\ }

\def\leftkol{\ % РЕЦЕНЗИИ
}

\def\rightkol{ \ } 

%\def\leftkol{\ } % ENGLISH ABSTRACTS}

%\def\rightkol{\ } %ENGLISH ABSTRACTS}

%\def\leftkol{РЕЦЕНЗИИ}

%\def\rightkol{РЕЦЕНЗИИ}

\titele{\tit}{\aut}{\auf}{\leftkol}{\rightkol}
\vspace*{-18pt}


     \label{st\stat}

     \begin{multicols}{2}
     {\small
     {\baselineskip=10.1pt
     

      В книге представлено системное изложение теоретических основ одного из новейших 
направлений в \mbox{об\-ласти} экономики послепродажного обслуживания изделий наукоемкой 
продукции (ИНП) длительного пользования~--- интегрированной логистической поддержки
(ИЛП). 
{\looseness=1

}

Приведены также результаты новых работ, выполненных в Институте проблем информатики 
Российской академии наук в рамках научного направления <<Информационные технологии и 
анализ сложных сис\-тем>>.
 {%\looseness=1

}
     
      Излагаемые в книге научные подходы позво\-ляют карди\-наль\-но реформировать 
существующие системы производства и эксплуатации ИНП путем создания и внед\-ре\-ния 
методов рационального и оптимального управ\-ле\-ния процессами расходования 
вре\-мен\-н$\acute{\mbox{ы}}$х, 
мате\-ри\-аль\-ных, трудовых и других ресурсов на всех стадиях жизненного цикла изделий (ЖЦИ) по 
критериям экономической целесообразности и эф\-фек\-тив\-ности.
  {\looseness=1

}
    
      В книге приведен краткий обзор причин возник\-новения и
      развития CALS-методологии как основы 
современных международных стандартов по созданию и функционированию глобальных 
ин\-фор\-ма\-ци\-он\-но-ком\-му\-ни\-ка\-ци\-он\-ных систем, ее ключевых возможностей и эффективности 
результатов ее использования. 
Авторы %\linebreak 
предлагают ряд научных обоснований для разработки 
единой теории проектирования и управления систем ИЛП для полноценного использования 
преимуществ %\linebreak
 суще\-ст\-ву\-ющей методологии, определяют \mbox{общую} структурную схему 
комплексной системы <<ИНП-СППО>> и необходимость разработки для ее описания 
гибридных стохастических моделей.
{%\looseness=1

}

%\columnbreak
      
      Книга состоит из пяти частей, где последовательно излагается материал по каждой из 
следующих тем: <<Интегрированная логистическая поддержка>>, <<Теория гибридных 
стохастических систем и компьютерная поддержка исследований и разработок>>, <<Основы 
математического моделирования, анализа и синтеза систем послепродажного обслуживания>>, 
<<Определение и анализ показателей экспортного потенциала ИНП при проектировании>>, 
<<Задачи управления поддержкой послепродажного обслуживания>>, а также 
<<Моделирование инвестиционных процессов ИЛП в условиях неравновесных финансовых 
рынков>>. 
   
      В конце каждой главы приведены выводы и даны вопросы и задания для 
самоконтроля. В~приложениях содержатся основные определения по программам работ по 
анализу ИЛП, логистическим базам данных и компьютерным решениям, эквивалентной статистической 
линеаризации нелинейных преобразований ИЛП, справочный материал, а также развернутые 
уравнения для вероятностных характеристик.


      \def\leftkol{РЕЦЕНЗИИ}

\def\rightkol{РЕЦЕНЗИИ} 

      
      Книга заинтересует широкий круг специалистов и может быть использована научными 
проектными организациями в сфере промышленного производства ИНП. Большое количество 
иллюстраций, примеров и вопросов, обращенных к читателю, позволяет использовать книгу 
также в качестве учебного пособия для студентов и аспирантов машиностроительных, 
транспортных и~других специальностей, а также для самостоятельного изучения. 
{%\looseness=-1

}

Книга 
представляет несомненный интерес для специалистов и студентов в области прикладной 
математики и информатики.
    

}

}
\end{multicols}

%\newpage

\def\stat{authorsrus}
{%\hrule\par
%\vskip 7pt % 7pt
\raggedleft\Large \bf%\baselineskip=3.2ex
О\,Б\ \ А\,В\,Т\,О\,Р\,А\,Х \vskip 17pt
    \hrule
    \par
\vskip 21pt plus 8pt minus 4pt }


\def\tit{\ }

\def\aut{\ }

\def\auf{\ }

\def\leftkol{\ } % ENGLISH ABSTRACTS}

\def\rightkol{ОБ АВТОРАХ} %ENGLISH ABSTRACTS}

\titele{\tit}{\aut}{\auf}{\leftkol}{\rightkol}
      
            \label{st\stat}



\vspace*{24pt}

\begin{multicols}{2}




\noindent
\textbf{Архипов Олег Петрович} (р.\ 1948)~---
кандидат технических наук, директор Орловского филиала Института проб\-лем информатики
Российской академии наук
%302025, г.Орел, Московское шоссе, д.137

\vspace*{3pt}

\noindent
\textbf{Бирюкова Татьяна Константиновна} (р.\ 1968)~---
кандидат фи\-зи\-ко-ма\-те\-ма\-ти\-че\-ских наук, старший научный сотрудник Института проб\-лем информатики
Российской академии наук

\vspace*{3pt}

\noindent 
\textbf{Бобков  Сергей Геннадьевич} (р.\ 1955)~---
доктор технических наук,  заведующий отделением На\-уч\-но-ис\-сле\-до\-ва\-тель\-ско\-го 
института системных исследований Российской академии наук
%117218, Москва, Нахимовский просп., 36, к.1 

\vspace*{3pt}

\noindent \textbf{Васильев Николай Семенович} (р.\ 1952)~--- доктор 
фи\-зи\-ко-ма\-те\-ма\-ти\-че\-ских наук, профессор, 
МГТУ им.\ Н.\,Э.~Баумана 
%, Москва 105005, 2-я Бауманская ул., д.~5,

\vspace*{3pt}

\noindent
\textbf{Гершкович Максим Михайлович} (р.\ 1968)~---
старший научный сотрудник Института проб\-лем информатики
Российской академии наук

\vspace*{3pt}

\noindent 
\textbf{Дьяченко Юрий Георгиевич} (р.\ 1958)~--- кандидат технических наук, 
старший научный сотрудник Института проб\-лем информатики
Российской академии наук

\vspace*{3pt}

\noindent 
\textbf{Ерошенко Александр Андреевич} (р.\ 1989)~--- аспирант кафедры 
математической статистики факультета вычисли\-тельной математики и кибернетики 
Московского государственного университета им.\ М.\,В.~Ломоносова
%119991, Москва ГСП-1, Ленинские горы, д.\ 1, стр. 52

\vspace*{3pt}
 
\noindent 
\textbf{Захаров Виктор Николаевич} (р.\ 1948)~--- 
доктор технических наук, доцент, ученый секретарь Института проб\-лем информатики
Российской академии наук

\vspace*{3pt}

\noindent
\textbf{Зейфман Александр Израилевич} (р.\ 1954)~---
доктор фи\-зи\-ко-ма\-те\-ма\-ти\-че\-ских наук, профессор, 
заведующий кафедрой Вологодского государственного университета; 
старший научный сотрудник Института проб\-лем информатики
Российской академии наук; главный научный сотрудник ИСЭРТ Российской академии наук

\vspace*{3pt}

\noindent
\textbf{Зыкин Сергей Владимирович} (р.\ 1959)~--- 
доктор технических наук, профессор, заведующий лабораторией Института математики 
им.\ С.\,Л.~Соболева Сибирского отделения Российской академии наук, Новосибирск 
%630090, пр.\ ак.\ Коптюга, 4 

\vspace*{4pt}

\noindent
\textbf{Киреев Владимир Иванович} (р.\ 1938)~---
доктор фи\-зи\-ко-ма\-те\-ма\-ти\-че\-ских наук, профессор Московского 
государственного горного университета
%Адрес: Россия, 119991, г. Москва, Ленинский проспект, д. 6

%\columnbreak

\vspace*{4pt}

\noindent
\textbf{Козеренко Елена Борисовна} (р.\ 1959)~---
кандидат филологических наук, заведующая лабораторией Института проб\-лем информатики
Российской академии наук

\vspace*{4pt}

\noindent
\textbf{Королев Виктор Юрьевич} (р.\ 1954)~--- доктор
фи\-зи\-ко-ма\-те\-ма\-ти\-че\-ских наук, профессор кафедры математической 
статистики факультета вычисли\-тельной математики и кибернетики 
Московского государственного университета; 
ведущий научный сотрудник Института проб\-лем информатики
Российской академии наук

\vspace*{4pt}

\noindent
\textbf{Коротышева Анна Владимировна} (р.\ 1988)~---
старший преподаватель Вологодского государственного университета

\vspace*{4pt}

\noindent 
\textbf{Кун Де Турк} (р.\ 1981)~--- научный сотрудник 
исследовательской группы SMACS факультета телекоммуникаций и обработки информации
Университета Гента, Бельгия
%В-9000 Гент, Бельгия

\vspace*{4pt}

\noindent
\textbf{Лупенцов Олег Сергеевич} (р.\ 1986)~---
аспирант Омского государственного института сервиса
%Омск 644043, ул.\ Певцова 13

\vspace*{4pt}

\noindent
\textbf{Лучко Олег Николаевич} (р.\ 1961)~---
кандидат педагогических наук, профессор, заведующий кафедрой 
Омского государственного института сервиса
%Омск 644043, ул.\ Певцова 13

\vspace*{4pt}

\noindent
\textbf{Малашенко Юрий Евгеньевич} (р.\ 1946)~---
доктор фи\-зи\-ко-ма\-те\-ма\-ти\-че\-ских наук, заведующий сектором 
Вычислительного центра им.\ А.\,А.~Дородницына Российской академии наук
%Адрес: 119333, Москва, ул. Вавилова, 40,

\vspace*{4pt}

\noindent
\textbf{Маньяков Юрий Анатольевич} (р.\ 1984)~---
кандидат технических наук, научный сотрудник Орловского филиала Института проб\-лем информатики
Российской академии наук
%302025, г.Орел, Московское шоссе, д.137

\vspace*{4pt}

\noindent
\textbf{Маренко Валентина Афанасьевна} (р.\ 1951)~---
кандидат технических наук, доцент, старший научный сотрудник 
Института математики им.\ С.\,Л.~Соболева Сибирского отделения Российской академии наук
%Новосибирск 630090, пр. ак. Коптюга, 4 

\vspace*{3pt}

\noindent 
\textbf{Морозов Евсей Викторович} (р.\ 1947)~--- доктор 
фи\-зи\-ко-ма\-те\-ма\-ти\-че\-ских, профессор, ведущий научный сотрудник 
Института прикладных математических исследований Карельского научного центра Российской
академии наук; 
%%185910 Россия, Республика Карелия, г.\ Петрозаводск, ул.\ Пушкинская, 11
профессор Петрозаводского государственного университета, Петрозаводск
%185910 Россия, Республика Карелия, г.\ Петрозаводск, пр.\ Ленина, 33

%\pagebreak

\vspace*{3pt}

\noindent
\textbf{Назарова Ирина Александровна} (р.\ 1966)~---
кандидат фи\-зи\-ко-ма\-те\-ма\-ти\-че\-ских наук, 
научный сотрудник Вычислительного центра им.\ А.\,А.~Дородницына Российской академии наук 
%Адрес: 119333, Москва, ул. Вавилова, 40

\vspace*{3pt}

\noindent
\textbf{Павлов Игорь Валерианович} (р.\ 1945)~--- 
доктор фи\-зи\-ко-ма\-те\-ма\-ти\-че\-ских наук, профессор МГТУ им.\ Н.\,Э.~Баумана 
%Москва 105005, 2-я Бауманская ул., д.~5 

%\pagebreak

\vspace*{3pt}

\noindent 
\textbf{Потахина Любовь Викторовна} (р.\ 1989)~--- аспирантка
Института прикладных математических исследований Карельского научного центра
Российской академии наук; 
%%185910 Россия, Республика Карелия, г.\ Петрозаводск, ул.\ Пушкинская, 11
инженер Петрозаводского государственного университета, Петрозаводск
%185910 Россия, Республика Карелия, г.\ Петрозаводск, пр.\ Ленина, 33

\vspace*{3pt}

\noindent 
\textbf{Рождественский Юрий Владимирович} (р.\ 1952)~--- 
кандидат технических наук, заведующий сектором Института проб\-лем информатики
Российской академии наук

\vspace*{3pt}

\noindent 
\textbf{Синицын Игорь Николаевич} (р.\ 1940)~--- доктор технических наук,
профессор, заслуженный деятель\linebreak\vspace*{-12pt}

\columnbreak

\noindent
 науки РФ, заведующий отделом Института проб\-лем информатики
Российской академии наук

\vspace*{7pt}


\noindent
\textbf{Сиротинин Денис Олегович} (р.\ 1984)~---
кандидат технических наук, научный сотрудник Орловского филиала Института проб\-лем информатики
Российской академии наук
%302025, г.Орел, Московское шоссе, д.137

\vspace*{7pt}

%\columnbreak

\noindent 
\textbf{Соколов  Игорь Анатольевич} (р.\ 1954)~--- академик (действительный член) Российской 
академии наук, доктор технических наук, директор Института проб\-лем информатики
Российской академии наук

\vspace*{7pt}

\noindent
\textbf{Степченков Юрий Афанасьевич} (р.\ 1951)~---
кандидат технических наук, заведующий отделом Института проб\-лем информатики
Российской академии наук

\vspace*{7pt}

\noindent
\textbf{Сурков Алексей Викторович} (р.\ 1978)~--- 
старший научный сотрудник На\-уч\-но-ис\-сле\-до\-ва\-тель\-ско\-го 
института системных исследований Российской академии наук
%117218, Москва, Нахимовский просп., 36, к.1 

\vspace*{7pt}

\noindent 
\textbf{Шестаков Олег Владимирович} (р.\ 1976)~--- доктор 
фи\-зи\-ко-ма\-те\-ма\-ти\-че\-ских, доцент кафедры математической статистики 
факультета вычисли\-тельной математики и кибернетики Московского 
государственного университета им.\ М.\,В.~Ломоносова; 
%119991, Москва ГСП-1, Ленинские горы, д.\ 1, стр. 52
старший научный сотрудник Института проб\-лем информатики
Российской академии наук
%, Москва 119333, ул. Вавилова, д.~44, корп.~2

\vspace*{7pt}

\noindent 
\textbf{Шоргин Сергей Яковлевич} (р.\ 1952.)~--- доктор
фи\-зи\-ко-ма\-те\-ма\-ти\-че\-ских наук, профессор, заместитель директора Института 
проб\-лем информатики Российской академии наук





%%%%%%%%%%%%%%%%%%%%%%%%%%%%%%%%%%%%%%%%%%%%%%%%%%%%%%%%%%%%%%%%%%%%%%%%%%%%%%%




%\def\rightkol{ОБ АВТОРАХ}
%\def\leftkol{ОБ АВТОРАХ}

 \label{end\stat}





%\def\leftfootline{\small{\textbf{\thepage}
%\hfill ИНФОРМАТИКА И ЕЁ ПРИМЕНЕНИЯ\ \ \ том~7\ \ \ выпуск~1\ \ \ 2013}
%}%
% \def\rightfootline{\small{ИНФОРМАТИКА И ЕЁ ПРИМЕНЕНИЯ\ \ \ том~7\ \ \ выпуск~1\ \ \ 2013
%\hfill \textbf{\thepage}}}


%\thispagestyle{myheadings}



\end{multicols}

\newpage

%\end{document}

%
\def\stat{rekl}
%\label{preobr}

%\def\tit{АКАДЕМИК ПУГАЧЁВ  ВЛАДИМИР СЕМЁНОВИЧ\\
%25.03.1911--25.03.1998}


%   \vspace*{-48pt}
%   \begin{center}\LARGE
%Академик Пугачёв  Владимир Семёнович\\ (25.03.1911--25.03.1998)
%   \end{center}

   %\vspace*{2.5mm}

   \begin{center}

{\prgsh\LARGE
ЮБИЛЕИ}

\end{center}
%\hrule

\vspace*{6pt}


   \vspace*{8mm}

   \thispagestyle{empty}


%\def\stat{emel}


\section*{К 70-летию заместителя директора ИПИ РАН,\\ члена редколлегии журнала
<<Информатика и её применения>>\\ доктора технических наук В.\,И.~Будзко}

\vspace*{18pt}




          \begin{multicols}{2}

%            \label{st\stat}

\begin{center}
\vspace*{1pt}
\mbox{%
\epsfxsize=78mm
\epsfbox{bud-1.eps}
}
\end{center}

\vspace*{12pt}

      14 августа 2014~г.\ исполнилось 70~лет за\-мес\-ти\-те\-лю директора ИПИ РАН по
научной работе доктору технических наук Владимиру Игоревичу Будзко.

      Владимир Игоревич Будзко родился в г.~Москве. Высшее образование получил на факультете
элект\-рон\-но-вы\-чис\-ли\-тель\-ных устройств в Московском
ин\-же\-нер\-но-фи\-зи\-че\-ском институте
(МИФИ), который он окончил в 1968~г., после чего был на\-прав\-лен для прохождения
службы в одну из войс\-ко\-вых частей, где прошел путь от инженера до первого заместителя
командира войсковой части.

      С приходом В.\,И.~Будзко в ИПИ РАН (2001~г.)\ в институте
сформировалось новое научное на\-прав\-ле\-ние теоретических исследований~--- <<Постро\-ение
ин\-фор\-ма\-ци\-он\-но-те\-ле\-ком\-му\-ни\-ка\-ци\-он\-ных\linebreak сис\-тем
высокой до\-ступ\-ности>>. В~рамках этого
направления выполнен широкий круг фундаментальных исследований по поиску подходов и
определению принципов построения средств обеспечения доступности, конфиденциальности
и целостности современных крупномасштабных
ин\-фор\-ма\-ци\-он\-но-те\-ле\-ком\-му\-ни\-ка\-ци\-он\-ных
сис\-тем (ИТС). Разработаны основные сис\-тем\-но-тех\-ни\-че\-ские принципы и базовые
архитектурные решения построения перспективных для условий России ИТС с
централизованной обработкой и хранением информации, сочетающих в себе свойства
высокой доступности, отказо- и катастрофоустойчивости, информационной защищенности.
Определены принципы, методы и математические основы рационального построения и
оптимизации средств восстановления функционирования центров обработки данных (ЦОД)
после возникновения отказов и катастроф, передачи и хранения данных, обеспечения
информационной безопасности при достижении минимальной совокупной стоимости
владения такими системами. Результаты нашли практическое воплощение при реализации
проектов в интересах ряда отечественных государственных и негосударственных
организаций, таких как Банк России (БР), Внешторгбанк, ОАО <<ГМК <<Норильский Никель>>,
<<Газпром>>, Минэкономразвития России, Правительство Москвы, а также ряд силовых
ведомств.

      Под руководством В.\,И.~Будзко начиная с 2001~г.\ выполнен комплекс
      на\-уч\-но-ис\-сле\-до\-ва\-тель\-ских и
      опыт\-но-кон\-ст\-рук\-тор\-ских работ (свыше 100~проектов),
направленных на развитие электронной информационной технологии БР.
Разработаны концепции развития ИТС БР сначала до 2008~г., а затем до 2013~г., которые
были приняты в качестве основы проведения технической политики. За реализацию проекта
<<Катастрофоустойчивая тер\-ри\-то\-ри\-аль\-но-рас\-пре\-де\-лен\-ная
      ин\-фор\-ма\-ци\-он\-но-те\-ле\-ком\-му\-ни\-ка\-ци\-он\-ная сис\-те\-ма централизованной
обработки банковской информации>> В.\,И.~Будзко удостоен Премии Правительства РФ в
области науки и техники за 2010~г.

      В.\,И.~Будзко возглавлял и возглавляет работы по ряду других прикладных проектов,
связанных с созданием, совершенствованием и развитием крупномасштабных ИТС.

      В.\,И.~Будзко~--- генерал-майор, доктор технических наук, член-кор\-рес\-пон\-дент
Академии криптографии РФ, известный ученый в области информатики и применения
информационных технологий при построении территориально распределенных ИТС
различного назначения. Является автором свыше 250~научных работ, опубликованных в
на\-уч\-но-тех\-ни\-че\-ских и специальных изданиях.

    \thispagestyle{empty}

      В.\,И.~Будзко уделяет большое внимание подготовке научных кадров. Под его
руководством защищено 6~диссертаций на соискание ученой степени кандидата
технических наук. Свыше 30~лет он читает лекции в ИКСИ Академии ФСБ, профессор
кафедры НИЯУ МИФИ. Является членом двух диссертационных советов, главным
редактором журнала <<Системы высокой доступности>> и членом редколлегии журнала
<<Информатика и её применения>>.

      \bigskip

      Редакционный совет и Редакционная коллегия журнала <<Информатика и её
применения>> сердечно поздравляют Владимира Игоревича Будзко с 70-ле\-ти\-ем и желают
крепкого здоровья и новых научных достижений.

\end{multicols}

%\def\stat{cont}
{%\hrule\par
%\vskip 7pt % 7pt
\raggedleft\Large \bf%\baselineskip=3.2ex
А\,В\,Т\,О\,Р\,С\,К\,И\,Й\ \ У\,К\,А\,З\,А\,Т\,Е\,Л\,Ь\ \ З\,А\ \ 2\,0\,1\,0 г. \vskip 17pt
    \hrule
    \par
\vskip 21pt plus 6pt minus 3pt }

\label{st\stat}

\def\tit{\ }

\def\aut{\ }
\def\auf{\ }

\def\leftkol{\ } % ENGLISH ABSTRACTS}

\def\rightkol{\ } %АВТОРСКИЙ УКАЗАТЕЛЬ ЗА 2010 г.} %ENGLISH ABSTRACTS}

\titele{\tit}{\aut}{\auf}{\leftkol}{\rightkol}

\vspace*{-12pt}

{\tabcolsep=3pt
\begin{tabular}{p{388pt}rr}
&\textbf{Выпуск} & \textbf{Стр.}\\[6pt]
\hangindent=23pt\noindent\textbf{Арутюнян~А.\,Р.} Моделирование влияния деформаций отпечатков пальцев на 
точность\linebreak
\vspace*{-12pt}\\
\hspace*{23pt}дактилоскопической идентификации$\dotfill$&1&51\\
\hangindent=23pt\noindent\textbf{Архипов~О.\,П., Зыкова~З.\,П.} Интеграция гетерогенной информации о цветных 
пикселях\linebreak
\vspace*{-12pt}\\
\hspace*{23pt}и их цветовосприятии$\dotfill$&4&15\\
\hangindent=23pt\noindent\textbf{Баранов~С.\,И., Френкель~С.\,Л., Захаров~В.\,Н.} Полуформальная верификация 
цифрового устройства с конвейером, основанная на использовании алгоритмических машин\linebreak
\vspace*{-12pt}\\
\hspace*{23pt}состояния$\dotfill$&4&49\\
\textbf{Бекетова~И.\,В.} см.~Каратеев~С.\,Л.&&\\
\textbf{Белоусов~В.\,В.} см.~Синицын~И.\,Н.&&\\
\hangindent=23pt\noindent\textbf{Бенинг~В.\,Е., Королев~Р.\,А.} О предельном поведении мощностей критериев в 
случае\linebreak
\vspace*{-12pt}\\
\hspace*{23pt}распределения Лапласа$\dotfill$&2&63\\
\hangindent=23pt\noindent\textbf{Бенинг~В.\,Е., Сипина~А.\,В.} Асимптотическое разложение для мощности 
критерия,\linebreak
\vspace*{-12pt}\\
\hspace*{23pt}основанного на выборочной медиане, в случае распределения Лапласа$\dotfill$&1&18\\
\textbf{Бондаренко~А.\,В.} см.~Каратеев~С.\,Л.&&\\
\hangindent=23pt\noindent\textbf{Бородина~А.\,В., Морозов~Е.\,В.} Об оценивании асимптотики вероятности 
большого\linebreak
\vspace*{-12pt}\\
\hspace*{23pt}уклонения стационарной регенеративной очереди с одним прибором$\dotfill$&3&29\\
\hangindent=23pt\noindent\textbf{Бунтман~Н.\,В., Минель~Ж.-Л., Ле~Пезан~Д., Зацман~И.\,М.} Типология и 
компьютерное\linebreak
\vspace*{-12pt}\\
\hspace*{23pt}моделирование трудностей перевода$\dotfill$&3&77\\
\textbf{Визильтер~Ю.\,В.} см.~Каратеев~С.\,Л.&&\\
\hangindent=23pt\noindent\textbf{Гавриленко~С.\,В.} Оценки скорости сходимости распределений случайных сумм с 
безгранично делимыми индексами к нормальному закону$\dotfill$&4&81\\
\hangindent=23pt\noindent\textbf{Григорьева~М.\,Е., Шевцова~И.\,Г.} Уточнение неравенства 
Каца--Берри--Эссеена$\dotfill$&2&75\\
\hangindent=23pt\noindent\textbf{Грушо~А.\,А., Грушо~Н.\,А., Тимонина~Е.\,Е.} Поиск конфликтов в политиках 
безопасности: модель случайных графов$\dotfill$&3&38\\
\textbf{Грушо~Н.\,А.} см.~Грушо~А.\,А.&&\\
\hangindent=23pt\noindent\textbf{Гудков~В.\,Ю.} Математические модели изображения отпечатка пальца на основе 
описания линий$\dotfill$&1&58\\
\textbf{Гуртов~А.\,В.} см.~Лукьяненко~А.\,С.&&\\
\textbf{Желтов~С.\,Ю.} см.~Каратеев~С.\,Л.&&\\
\hangindent=23pt\noindent\textbf{Захаров~А.\,А., Серебряков~В.\,А.} Система управления электронной библиотекой 
LibMeta$\dotfill$&4&2\\
\textbf{Захаров~В.\,Н.} см.~Баранов~С.\,И.&&\\
\textbf{Захарова~Т.\,В.} см.~Матвеева~С.\,С.&&\\
\hangindent=23pt\noindent\textbf{Зацаринный~А.\,А., Чупраков~К.\,Г.} Некоторые аспекты выбора технологии для 
постро-\linebreak
\vspace*{-12pt}\\
\hspace*{23pt}ения систем отображения информации ситуационного центра$\dotfill$&3&59\\
\textbf{Зацман~И.\,М.} см.~Бунтман~Н.\,В.&&\\
\hangindent=23pt\noindent\textbf{Зейфман~А.\,И., Коротышева~А.\,В., Сатин~Я.\,А., Шоргин~С.\,Я.} Об 
устойчивости нестаци-\linebreak
\vspace*{-12pt}\\
\hspace*{23pt}онарных систем обслуживания с катастрофами$\dotfill$&3&9\\
\textbf{Зыкова~З.\,П.} см.~Архипов~О.\,П.&&\\
\hangindent=23pt\noindent\textbf{Илюшин~Г.\,Я., Соколов~И.\,А.} Организация управляемого доступа пользователей 
к\linebreak
\vspace*{-12pt}\\
\hspace*{23pt}разнородным ведомственным информационным ресурсам$\dotfill$&1&24\\
\hangindent=23pt\noindent\textbf{Кавагучи~Ю., Ульянов~В.\,В., Фуджикоши~Я.} Приближения для статистик, 
описывающих\linebreak
\vspace*{-12pt}\\
\hspace*{23pt}геометрические свойства данных большой размерности, с оценками 
ошибок$\dotfill$&1&12\\
\hangindent=23pt\noindent\textbf{Каратеев~С.\,Л., Бекетова~И.\,В., Ососков~М.\,В., Князь~В.\,А., 
Визильтер~Ю.\,В., Бондаренко~А.\,В., Желтов~С.\,Ю.} Автоматизированный контроль 
качества цифровых\linebreak
\vspace*{-12pt}\\
\hspace*{23pt}изображений для персональных документов$\dotfill$&1&65\\
\end{tabular}
}

\pagebreak

\def\leftkol{АВТОРСКИЙ УКАЗАТЕЛЬ ЗА 2010 г.} % ENGLISH ABSTRACTS}

\def\rightkol{АВТОРСКИЙ УКАЗАТЕЛЬ ЗА 2010 г.} %ENGLISH ABSTRACTS}

{\tabcolsep=3pt
\begin{tabular}{p{388pt}rr}
&\textbf{Выпуск} & \textbf{Стр.}\\[3pt]
\hangindent=23pt\noindent\textbf{Козеренко~Е.\,Б.} Лингвистические фильтры в статистических моделях машинного\linebreak
\vspace*{-12pt}\\
\hspace*{23pt}перевода$\dotfill$&2&83\\
\hangindent=23pt\noindent\textbf{Козеренко~Е.\,Б., Кузнецов~И.\,П.} Когнитивно-лингвистические представления в 
систе-\linebreak
\vspace*{-12pt}\\
\hspace*{23pt}мах обработки текстов$\dotfill$&3&69\\
\textbf{Князь~В.\,А.} см.~Каратеев~С.\,Л.&&\\
\hangindent=23pt\noindent\textbf{Колесников~А.\,В., Солдатов~С.\,А.} Алгоритм координации для гибридной 
интеллектуальной системы решения сложной задачи оперативно-производственного\linebreak
\vspace*{-12pt}\\
\hspace*{23pt}планирования$\dotfill$&4&61\\
\hangindent=23pt\noindent\textbf{Коновалов~М.\,Г.} О планировании потоков в системах вычислительных 
ресурсов$\dotfill$&2&3\\
\textbf{Конушин~А.\,С.} см.~Конушин~В.\,С.&&\\
\hangindent=23pt\noindent\textbf{Конушин~В.\,С., Кривовязь~Г.\,Р., Конушин~А.\,С.} Алгоритм распознавания людей 
в видео-\linebreak
\vspace*{-12pt}\\
\hspace*{23pt}последовательности по одежде$\dotfill$&1&74\\
\textbf{Корепанов~Э.\, Р.} см.~Синицын~И.\,Н.&&\\
\textbf{Королев~В.\,Ю.} см.~Соколов~И.\,А.&&\\
\textbf{Королев~Р.\,А.} см.~Бенинг~В.\,Е.&&\\
\textbf{Коротышева~А.\,В.} см.~Зейфман~А.\,И.&&\\
\hangindent=23pt\noindent\textbf{Кривенко~М.\,П.} Непараметрическое оценивание элементов байесовского 
клас\-си-\linebreak
\vspace*{-12pt}\\
\hspace*{23pt}фикатора$\dotfill$&2&13\\
\textbf{Кривовязь~Г.\,Р.} см.~Конушин~В.\,С.&&\\
\textbf{Крылов~А.\,С.} см.~Павельева~Е.\,А.&&\\
\hangindent=23pt\noindent\textbf{Крылов~В.\,А.} Моделирование и классификация многоканальных дистанционных\linebreak
\vspace*{-12pt}\\
\hspace*{23pt}изображений с использованием копул$\dotfill$&4&34\\
\hangindent=23pt\noindent\textbf{Крючин~О.\,В.} Разработка параллельных эвристических алгоритмов подбора 
весовых\linebreak
\vspace*{-12pt}\\
\hspace*{23pt}коэффициентов искусственной нейтронной сети$\dotfill$&2&53\\
\hangindent=23pt\noindent\textbf{Кудрявцев~А.\,А., Шоргин~С.\,Я.} Байесовские модели массового обслуживания и 
надеж-\linebreak
\vspace*{-12pt}\\
\hspace*{23pt}ности: характеристики среднего числа заявок в системе $M\vert M \vert 1\vert 
\infty$$\dotfill$&3&16\\
\hangindent=23pt\noindent\textbf{Кузнецов~А.\,А.} Связь между временными и структурно-топологическими 
характери-\linebreak
\vspace*{-12pt}\\
\hspace*{23pt}стиками диаграмм ритма сердца здоровых людей$\dotfill$&4&39\\
\textbf{Кузнецов~И.\,П.} см.~Козеренко~Е.\,Б.&&\\
\textbf{Ле~Пезан~Д.} см.~Бунтман~Н.\,В.&&\\
\hangindent=23pt\noindent\textbf{Лукьяненко~А.\,С., Морозов~Е.\,В., Гуртов~А.\,В.} Анализ сетевого протокола с общей 
функ-\linebreak
\vspace*{-12pt}\\
\hspace*{23pt}цией расширения окна передачи сообщения при конфликтах$\dotfill$&2&46\\
\hangindent=23pt\noindent\textbf{Лямин~О.\,О.} О предельном поведении мощностей критериев в случае обобщенного\linebreak
\vspace*{-12pt}\\
\hspace*{23pt}распределения Лапласа$\dotfill$&3&47\\
\hangindent=23pt\noindent\textbf{Маркин~А.\,В., Шестаков~О.\,В.} Асимптотики оценки риска при пороговой 
обработке\linebreak
\vspace*{-12pt}\\
\hspace*{23pt}вейвлет-вейглет коэффициентов в задаче томографии$\dotfill$&2&36\\
\hangindent=23pt\noindent\textbf{Матвеева~С.\,С., Захарова~Т.\,В.} Сети массового обслуживания с наименьшей 
длиной\linebreak
\vspace*{-12pt}\\
\hspace*{23pt}очереди$\dotfill$&3&22\\
\hangindent=23pt\noindent\textbf{Матюшенко~С.\,И.} Стационарные характеристики двухканальной системы 
обслужива-\linebreak
\vspace*{-12pt}\\
\hspace*{23pt}ния с переупорядочиванием заявок и распределениями фазового типа$\dotfill$&4&68\\
\textbf{Минель~Ж.-Л.} см.~Бунтман~Н.\,В.&&\\
\textbf{Морозов~Е.\,В.} см.~Бородина~А.\,В.&&\\
\textbf{Морозов~Е.\,В.} см.~Лукьяненко~А.\,С.&&\\
\textbf{Ососков~М.\,В.} см.~Каратеев~С.\,Л.&&\\
\hangindent=23pt\noindent\textbf{Павельева~Е.\,А., Крылов~А.\,С.} Поиск и анализ ключевых точек радужной 
оболочки\linebreak
\vspace*{-12pt}\\
\hspace*{23pt}глаза методом преобразования Эрмита$\dotfill$&1&79\\
\textbf{Печинкин~А.\,В.} см.~Френкель~С.\,Л.,&&\\
\hangindent=23pt\noindent\textbf{Протасов~В.\,И.} Составление субъективного портрета с использованием 
эволюционно-\linebreak
\vspace*{-12pt}\\
\hspace*{23pt}го морфинга и квалиметрия метода$\dotfill$&1&83\\
\hangindent=23pt\noindent\textbf{Рудаков~К.\,В., Торшин~И.\,Ю.} Вопросы разрешимости задачи распознавания 
вторичной\linebreak
\vspace*{-12pt}\\
\hspace*{23pt}структуры белка$\dotfill$&2&25\\
\textbf{Сатин~Я.\,А.} см.~Зейфман~А.\,И.&&\\
\hangindent=23pt\noindent\textbf{Сейфуль-Мулюков~Р.\,Б.} Нефть как носитель информации о своем 
происхождении,\linebreak
\vspace*{-12pt}\\
\hspace*{23pt}структуре и эволюции$\dotfill$&1&41\\
\end{tabular}
}

{\tabcolsep=3pt
\begin{tabular}{p{388pt}rr}
&\textbf{Выпуск} & \textbf{Стр.}\\[6pt]
\textbf{Семендяев~Н.\,Н.} см.~Синицын~И.\,Н.&&\\
\textbf{Серебряков~В.\,А.} см.~Захаров~А.\,А.&&\\
\textbf{Синицын~В.\,И.} см.~Синицын~И.\,Н.&&\\
\hangindent=23pt\noindent\textbf{Синицын~И.\,Н., Синицын~В.\,И., Корепанов~Э.\, Р., Белоусов~В.\,В., 
Семендяев~Н.\,Н.} Оперативное построение информационных моделей движения полюса 
Земли\linebreak
\vspace*{-12pt}\\
\hspace*{23pt}методами линейных и линеаризованных фильтров$\dotfill$&1&2\\
\textbf{Сипина~А.\,В.} см.~Бенинг~В.\,Е.&&\\
\hangindent=23pt\noindent\textbf{Соколов~И.\,А.} О работах заслуженного деятеля науки Российской Федерации 
И.\,Н.~Синицына в области информационных технологий и автоматизации (к 70-летию\linebreak
\vspace*{-12pt}\\
\hspace*{23pt}со дня рождения)$\dotfill$&3&84\\
\textbf{Соколов~И.\,А.} см.~Илюшин~Г.\,Я.&&\\
\hangindent=23pt\noindent\textbf{Соколов~И.\,А., Королев~В.\,Ю.} Предисловие$\dotfill$&2&2\\
\textbf{Солдатов~С.\,А.} см.~Колесников~А.\,В.&&\\
\hangindent=23pt\noindent\textbf{Степанов~С.\,Ю.} Использование координатного метода фрагментации 
коммутаторной\linebreak
\vspace*{-12pt}\\
\hspace*{23pt}нейронной сети для сокращения трафика$\dotfill$&2&57\\
\textbf{Тимонина~Е.\,Е.} см.~Грушо~А.\,А.&&\\
\textbf{Торшин~И.\,Ю.} см.~Рудаков~К.\,В.&&\\
\textbf{Ульянов~В.\,В.} см.~Кавагучи~Ю.&&\\
\textbf{Фазекаш~И.} см.~Чупрунов~А.\,Н.&&\\
\textbf{Френкель~С.\,Л.} см.~Баранов~С.\,И.&&\\
\hangindent=23pt\noindent\textbf{Френкель~С.\,Л., Печинкин~А.\,В.} Оценка времени самовосстановления в 
цифровых\linebreak
\vspace*{-12pt}\\
\hspace*{23pt}системах после сбоев, вызываемых переходными помехами$\dotfill$&3&2\\
\textbf{Фуджикоши~Я.} см.~Кавагучи~Ю.&&\\
\hangindent=23pt\noindent\textbf{Цискаридзе~А.\,К.} Математическая модель и метод восстановления позы человека 
по\linebreak
\vspace*{-12pt}\\
\hspace*{23pt}стереопаре силуэтных изображений$\dotfill$&4&27\\
\hangindent=23pt\noindent\textbf{Чупраков~К.\,Г.} К вопросу о размещении коллективных средств отображения в 
ситуа-\linebreak
\vspace*{-12pt}\\
\hspace*{23pt}ционном зале с заданными параметрами$\dotfill$&4&89\\
\textbf{Чупраков~К.\,Г.} см.~Зацаринный~А.\,А.&&\\
\hangindent=23pt\noindent\textbf{Чупрунов~А.\,Н., Фазекаш~И.} Законы повторного логарифма для числа 
безошибочных\linebreak
\vspace*{-12pt}\\
\hspace*{23pt}блоков при помехоустойчивом кодировании$\dotfill$&3&42\\
\textbf{Шевцова~И.\,Г.} см.~Григорьева~М.\,Е.&&\\
\hangindent=23pt\noindent\textbf{Шестаков~О.\,В.} Аппроксимация распределения оценки риска пороговой 
обработки вейвлет-коэффициентов нормальным распределением при использовании 
выбо-\linebreak
\vspace*{-12pt}\\
\hspace*{23pt}рочной дисперсии$\dotfill$&4&73\\
\textbf{Шестаков~О.\,В.} см.~Маркин~А.\,В.&&\\
\textbf{Шоргин~С.\,Я.} см.~Зейфман~А.\,И.&&\\
\textbf{Шоргин~С.\,Я.} см.~Кудрявцев~А.\,А.&&\\
\end{tabular}
}

%\thispagestyle{myheadings}
\def\leftfootline{\small{\textbf{\thepage}
\hfill ИНФОРМАТИКА И ЕЁ ПРИМЕНЕНИЯ\ \ \ том~4\ \ \ выпуск~4\ \ \ 2010}
}%
 \def\rightfootline{\small{ИНФОРМАТИКА И ЕЁ ПРИМЕНЕНИЯ\ \ \ том~4\ \ \ выпуск~4\ \ \ 2010
 \hfill \textbf{\thepage}}}
 \label{end\stat}

%
%Том 10 Выпуск 1-4 Год 2016

\def\stat{cont-e}
{%\hrule\par
%\vskip 7pt % 7pt
\raggedleft\Large \bf%\baselineskip=3.2ex
2\,0\,1\,6\ \ A\,U\,T\,H\,O\,R\ \ I\,N\,D\,E\,X \vskip 17pt
 \hrule
 \par
\vskip 21pt plus 6pt minus 3pt }

\label{st\stat}

\def\tit{\ }

\def\aut{\ }
\def\auf{\ }

\def\leftkol{\ } %2016 AUTHOR INDEX} % ENGLISH ABSTRACTS}

\def\rightkol{\ } %2016 AUTHOR INDEX} %ENGLISH ABSTRACTS}

\titele{\tit}{\aut}{\auf}{\leftkol}{\rightkol}

\def\leftfootline{\small{\textbf{\thepage}
\hfill INFORMATIKA I EE PRIMENENIYA~--- INFORMATICS AND APPLICATIONS\ \ \ 2016\
\ \ volume~10\ \ \ issue\ 4}
}%
 \def\rightfootline{\small{INFORMATIKA I EE PRIMENENIYA~--- INFORMATICS AND APPLICATIONS\ \ \ 2016\ \ \ volume~10\ \ \ issue\ 4
\hfill \textbf{\thepage}}}

\vspace*{-12pt}
\vspace*{-18pt}

{\tabcolsep=2.8pt
\begin{tabular}{p{382pt}cc}
&\textbf{Issue} & \textbf{Page}\\[6pt]
\Avtors{Agalarov~M.\,Ya.} see~Agalarov~Ya.\,M.&&\\
\Avtors{Agalarov~Ya.\,M., Agalarov~M.\,Ya., and
Shorgin~V.\,S.} About the optimal threshold of queue\linebreak
\\[-12pt]
\hspace*{23pt}length in a~particular problem of profit maximization
in the $M/G/1$ queuing system&2&70--79\\
\Avtors{Alexeyevsky~D.\,A.} BioNLP ontology extraction from 
a~restricted language corpus with\linebreak
\\[-12pt]
\hspace*{23pt}context-free grammars&1&119--128\\
\Avtors{Andreev~S.\,D.} see~Gaidamaka~Yu.\,V.&&\\
\Avtors{Andreev~S.\,D.} see~Ometov~A.\,Ya.&&\\
\Avtors{Arkhipov~O.\,P., Arkhipov~P.\,O., and Sidorkin~I.\,I.} The
option to create a~local coordinate\linebreak
\\[-12pt]
\hspace*{23pt}system for synchronization of selected images&3&91--97\\
\Avtors{Arkhipov~P.\,O.} see~Arkhipov~O.\,P.&&\\
\Avtors{Belousov~V.\,V.} see~Shnurkov~P.\,V.&&\\
\Avtors{Belousov~V.\,V.} see~Shnurkov~P.\,V.&&\\
\Avtors{Bening~V.\,E.} Calculation of~the~asymptotic deficiency
of~some statistical procedures based\linebreak
\\[-12pt]
\hspace*{23pt}on~samples with~random sizes&4&34--45\\
\Avtors{Borisov~A.\,V., Bosov~A.\,V., and Miller~G.\,B.} Modeling and
monitoring of VoIP connection&2&\hphantom{1}2--13\\
\Avtors{Bosov~A.\,V.} see~Borisov~A.\,V.&&\\
\Avtors{Briukhov~D.\,O.} see~Stupnikov~S.\,A.&&\\
\Avtors{Callaos~N.\,K.\ and Seyful-Mulyukov~R.\,B.} Complexity and
its information content&1&129--139\\
\Avtors{Chertok~A.\,V., Kadaner~A.\,I., Khazeeva~G.\,T., and
Sokolov~I.\,A.} Regime switching detection\linebreak
\\[-12pt]
\hspace*{23pt}for~the~Levy driven
Ornstein--Uhlenbeck process using CUSUM methods&4&46--56\\
\Avtors{Chichagov~V.\,V.} Asymptotic expansions of mean absolute
error of uniformly minimum variance unbiased and maximum likelihood
estimators on the one-parameter exponential\linebreak
\\[-12pt]
\hspace*{23pt}family model of lattice distributions&3&66--76\\
\Avtors{Danishevsky~V.\,I.} see~Kolesnikov A.\,V.&&\\
\Avtors{Fazliev~A.\,Z.} see~Kalinichenko~L.\,A.&&\\
\Avtors{Fedoseev~A.\,A.} What is behind the concept of ``knowledge in
small packages''&3&105--110\\
\Avtors{Gaidamaka~Yu.\,V., Andreev~S.\,D., Sopin~E.\,S.,
Samouylov~K.\,E., and Shorgin~S.\,Ya.} Interference analysis
of~the~device-to-device communications model with~regard to~a~signal\linebreak
\\[-12pt]
\hspace*{23pt}propagation environment&4&\hphantom{1}2--10\\
\Avtors{Gasilov~A.\,V.} see~Yakovlev~O.\,A.&&\\
\Avtors{Goncharov~A.\,V.\ and Strijov~V.\,V.} Metric time series
classification using weighted dynamic\linebreak
\\[-12pt]
\hspace*{23pt}warping relative to centroids of classes&2&36--47\\
\Avtors{Gordov~E.\,P.} see~Kalinichenko~L.\,A.&&\\
\Avtors{Gorshenin~A.\,K.} Concept of online service for stochastic
modeling of real processes&1&72--81\\
\Avtors{Gorshenin~A.\,K.} see~Shnurkov~P.\,V.&&\\
\Avtors{Gorshenin~A.\,K.} see~Shnurkov~P.\,V.&&\\
\Avtors{Grusho~A.\,A., Grusho~N.\,A., Zabezhailo~M.\,I., and
Timonina~E.\,E.} Integration of statistical and\linebreak
\\[-12pt]
\hspace*{23pt}deterministic methods for
analysis of information security&3&2--8\\
\Avtors{Grusho~A.\,A., Zabezhailo~M.\,I., and Zatsarinny~A.\,A.} On
the advanced procedure to reduce\linebreak
\\[-12pt]
\hspace*{23pt}calculation of Galois closures&4&\hphantom{1}96--104\\
\Avtors{Grusho~N.\,A.} see~Grusho~A.\,A.&&\\
\Avtors{Havanskov~V.\,A.} see~Minin~V.\,A.&&\\
\Avtors{Inkova~O.\,Yu.} see~Zatsman~I.\,M.&&\\
\Avtors{Isachenko~R.\,V.\ and Strijov~V.\,V.} Metric learning in
multiclass time series classification\linebreak
\\[-12pt]
\hspace*{23pt}problem&2&48--57\\
\end{tabular}
}
\pagebreak

\def\leftfootline{\small{\textbf{\thepage}
\hfill INFORMATIKA I EE PRIMENENIYA~--- INFORMATICS AND APPLICATIONS\ \ \ 2016\
\ \ volume~10\ \ \ issue\ 4}
}%
 \def\rightfootline{\small{INFORMATIKA I EE PRIMENENIYA~---
INFORMATICS AND APPLICATIONS\ \ \ 2016\ \ \ volume~10\ \ \ issue\ 4
\hfill \textbf{\thepage}}}

\def\leftkol{2016 AUTHOR INDEX} % ENGLISH ABSTRACTS}

\def\rightkol{2016 AUTHOR INDEX} %ENGLISH ABSTRACTS}


{\tabcolsep=2.83pt
\begin{tabular}{p{382pt}cc}
&\textbf{Issue} & \textbf{Page}\\[6pt]
\Avtors{Kadaner~A.\,I.} see~Chertok~A.\,V.&&\\[.255pt]
\Avtors{Kalinichenko~L.\,A., Volnova~A.\,A., Gordov~E.\,P.,
Kiselyova~N.\,N., Kovaleva~D.\,A., Malkov~O.\,Yu., Okladnikov~I.\,G.,
Podkolodnyy~N.\,L., Pozanenko~A.\,S., Ponomareva~N.\,V.,
Stupnikov~S.\,A.,} \textbf{and Fazliev~A.\,Z.} Data access challenges for data
intensive\linebreak
\\[-12pt]
\hspace*{23pt}research in Russia&1& 2--22\\[.255pt]
\Avtors{Karasikov~M.\,E.\ and Strijov~V.\,V.} Feature-based
time-series classification&4&121--131\\[.255pt]
\Avtors{Khazeeva~G.\,T.} see~Chertok~A.\,V.&&\\[.255pt]
\Avtors{Khokhlov~Yu.\,S.} Multivariate fractional Levy motion and its
applications&2&\hphantom{1}98--106\\[.255pt]
\Avtors{Kirikov~I.\,A., Kolesnikov~A.\,V., Listopad~S.\,V., and
Rumovskaya~S.\,B.} Fine-grained hybrid\linebreak
\\[-12pt]
\hspace*{23pt}intelligent systems. Part 2:
Bidirectional hybridization&1&\hphantom{1}96--105\\[.255pt]
\Avtors{Kirikov~I.\,A., Kolesnikov~A.\,V., Listopad~S.\,V., and
Rumovskaya~S.\,B.} ``Virtual council''~---\linebreak
\\[-12pt]
\hspace*{23pt}source environment
supporting complex diagnostic decision making&3&81--90\\[.255pt]
\Avtors{Kiselyova~N.\,N.} see~Kalinichenko~L.\,A.&&\\[.255pt]
\Avtors{Kolesnikov A.\,V., Listopad~S.\,V., Rumovskaya~S.\,B., and
Danishevsky~V.\,I.} Informal axiomatic\linebreak
\\[-12pt]
\hspace*{23pt}theory of~the~role visual models&4&114--120\\[.255pt]
\Avtors{Kolesnikov~A.\,V.} see~Kirikov~I.\,A.&&\\[.255pt]
\Avtors{Kolesnikov~A.\,V.} see~Kirikov~I.\,A.&&\\[.255pt]
\Avtors{Kolin~K.\,K.} Humanitarian aspects of information
security&3&111--121\\[.255pt]
\Avtors{Konovalov~M.\,G.\ and Razumchik~R.\,V.} Dispatching
to~two parallel nonobservable queues using\linebreak
\\[-12pt]
\hspace*{23pt}only static
information&4&57--67\\[.255pt]
\Avtors{Korchagin~A.\,Yu.} see~Korolev~V.\,Yu.&&\\[.255pt]
\Avtors{Korchagin~A.\,Yu.} see~Korolev~V.\,Yu.&&\\[.255pt]
\Avtors{Korepanov~E.\,R.} see~Sinitsyn~I.\,N.&&\\[.255pt]
\Avtors{Korepanov~E.\,R.} see~Sinitsyn~I.\,N.&&\\[.255pt]
\Avtors{Korolev~V.\,Yu., Korchagin~A.\,Yu., and Zeifman~A.\,I.} The
Poisson theorem for Bernoulli trials\linebreak
\\[-12pt]
\hspace*{23pt}with~a~random probability
of~success and~a~discrete analog of~the~Weibull distribution&4&11--20\\[.255pt]
\Avtors{Korolev~V.\,Yu., Zeifman~A.\,I., and Korchagin~A.\,Yu.}
Asymmetric Linnik distributions as~limit\linebreak
\\[-12pt]
\hspace*{23pt}laws for~random sums
of~independent random variables with~finite variances&4&21--33\\[.255pt]
\Avtors{Koucheryavy~E.\,A.} see~Ometov~A.\,Ya.&&\\[.255pt]
\Avtors{Kovaleva~D.\,A.} see~Kalinichenko~L.\,A.&&\\[.255pt]
\Avtors{Kovalyov~S.\,P.} Metaprogramming to increase
manufacturability of large-scale software-\linebreak
\\[-12pt]
\hspace*{23pt}intensive systems&1&56--66\\[.255pt]
\Avtors{Krivenko~M.\,P.} Significance tests of feature selection for
classification&3&32--40\\[.255pt]
\Avtors{Kruzhkov~M.\,G.} see~Zalizniak~Anna~A.&&\\[.255pt]
\Avtors{Kruzhkov~M.\,G.} see~Zatsman~I.\,M.&&\\[.255pt]
\Avtors{Kudryavtsev~A.\,A.} Bayesian queueing and reliability models:
\textit{A~priori} distributions with\linebreak
\\[-12pt]
\hspace*{23pt}compact support&1&67--71\\[.255pt]
\Avtors{Kudryavtsev~A.\,A.} Characteristics dependent on the balance
coefficient in Bayesian models\linebreak
\\[-12pt]
\hspace*{23pt}with compact support of \textit{a priori}
distributions&3&77--80\\[.255pt]
\Avtors{Kudryavtsev~A.\,A.\ and Palionnaia~S.\,I.} Bayesian recurrent
model of reliability growth:\linebreak
\\[-12pt]
\hspace*{23pt}Parabolic distribution of parameters&2&80--83\\[.255pt]
\Avtors{Kudryavtsev~A.\,A.\ and Titova~A.\,I.} Bayesian queuing
and~reliability models: Degenerate-\linebreak
\\[-12pt]
\hspace*{23pt}Weibull case&4&68--71\\[.255pt]
\Avtors{Leontyev~N.\,D.\ and Ushakov~V.\,G.} Analysis of a queueing
system with autoregressive arrivals\linebreak
\\[-12pt]
\hspace*{23pt}and nonpreemptive priority&3&15--22\\[.255pt]
\Avtors{Listopad~S.\,V.} see~Kirikov~I.\,A.&&\\[.255pt]
\Avtors{Listopad~S.\,V.} see~Kirikov~I.\,A.&&\\[.255pt]
\Avtors{Listopad~S.\,V.} see~Kolesnikov A.\,V.&&\\[.255pt]
\Avtors{Malkov~O.\,Yu.} see~Kalinichenko~L.\,A.&&\\[.255pt]
\Avtors{Markov~A.\,S., Monakhov~M.\,M., and
Ulyanov~V.\,V.} Generalized Cornish--Fisher expansions\linebreak
\\[-12pt]
\hspace*{23pt}for distributions of statistics based on samples
of random size&2&84--91\\[.255pt]
\Avtors{Melnikov~A.\,K.\ and Ronzhin~A.\,F.} Generalized statistical
method of~text analysis based\linebreak
\\[-12pt]
\hspace*{23pt}on~calculation of~probability distributions
of~statistical values&4&89--95\\
\end{tabular}
}
\pagebreak

\def\leftfootline{\small{\textbf{\thepage}
\hfill INFORMATIKA I EE PRIMENENIYA~--- INFORMATICS AND APPLICATIONS\ \ \ 2016\
\ \ volume~10\ \ \ issue\ 4}
}%
 \def\rightfootline{\small{INFORMATIKA I EE PRIMENENIYA~---
INFORMATICS AND APPLICATIONS\ \ \ 2016\ \ \ volume~10\ \ \ issue\ 4
\hfill \textbf{\thepage}}}

\def\leftkol{2016 AUTHOR INDEX} % ENGLISH ABSTRACTS}

\def\rightkol{2016 AUTHOR INDEX} %ENGLISH ABSTRACTS}


{\tabcolsep=3pt
\begin{tabular}{p{381pt}cc}
&\textbf{Issue} & \textbf{Page}\\[6pt]
\Avtors{Meykhanadzhyan~L.\,A.} Stationary characteristics of the finite
capacity queueing system with\linebreak
\\[-12pt]
\hspace*{23pt}inverse service order and generalized
probabilistic priority&2&123--131\\[.23pt]
\Avtors{Miller~G.\,B.} see~Borisov~A.\,V.&&\\[.23pt]
\Avtors{Minin~V.\,A., Zatsman~I.\,M., Havanskov~V.\,A., and
Shubnikov~S.\,K.} Intensity of citation of scientific publications in
inventions on information and computer technologies patented\linebreak
\\[-12pt]
\hspace*{23pt}in Russia by domestic and foreign applicants&2&107--122\\[.23pt]
\Avtors{Monakhov~M.\,M.} see~Markov~A.\,S.&&\\[.23pt]
\Avtors{Naumov~V.\,A.\ and Samouylov~K.\,E.} On relationship
between queuing systems with resources\linebreak
\\[-12pt]
\hspace*{23pt}and Erlang networks&3&\hphantom{1}9--14\\[.23pt]
\Avtors{Okladnikov~I.\,G.} see~Kalinichenko~L.\,A.&&\\[.23pt]
\Avtors{Ometov~A.\,Ya., Andreev~S.\,D., Turlikov~A.\,M., and
Koucheryavy~E.\,A.} Performance analysis of\linebreak
\\[-12pt]
\hspace*{23pt}a wireless data
aggregation system with contention for contemporary sensor
networks&3&23--31\\[.23pt]
\Avtors{Palionnaia~S.\,I.} see~Kudryavtsev~A.\,A.&&\\[.23pt]
\Avtors{Podkolodnyy~N.\,L.} see~Kalinichenko~L.\,A.&&\\[.23pt]
\Avtors{Ponomareva~N.\,V.} see~Kalinichenko~L.\,A.&&\\[.23pt]
\Avtors{Popkova~N.\,A.} see~Zatsman~I.\,M.&&\\[.23pt]
\Avtors{Pozanenko~A.\,S.} see~Kalinichenko~L.\,A.&&\\[.23pt]
\Avtors{Razumchik~R.\,V.} see~Konovalov~M.\,G.&&\\[.23pt]
\Avtors{Ronzhin~A.\,F.} see~Melnikov~A.\,K.&&\\[.23pt]
\Avtors{Rumovskaya~S.\,B.} see~Kirikov~I.\,A.&&\\[.23pt]
\Avtors{Rumovskaya~S.\,B.} see~Kirikov~I.\,A.&&\\[.23pt]
\Avtors{Rumovskaya~S.\,B.} see~Kolesnikov A.\,V.&&\\[.23pt]
\Avtors{Samouylov~K.\,E.} see~Gaidamaka~Yu.\,V.&&\\[.23pt]
\Avtors{Samouylov~K.\,E.} see~Naumov~V.\,A.&&\\[.23pt]
\Avtors{Serebryanskii~S.\,M.} see~Tyrsin~A.\,N.&&\\[.23pt]
\Avtors{Seyful-Mulyukov~R.\,B.} see~Callaos~N.\,K.&&\\[.23pt]
\Avtors{Shestakov~O.\,V.} Statistical properties of the denoising method
based on the stabilized hard\linebreak
\\[-12pt]
\hspace*{23pt}thresholding&2&65--69\\[.23pt]
\Avtors{Shestakov~O.\,V.} The strong law of large numbers for the risk
estimate in the problem of\linebreak
\\[-12pt]
\hspace*{23pt}tomographic image reconstruction from
projections with a correlated noise&3&41--45\\[.23pt]
\Avtors{Shestakov~O.\,V.} see~Zakharova~T.\,V.&&\\[.23pt]
\Avtors{Shnurkov~P.\,V., Gorshenin~A.\,K., and Belousov~V.\,V.}
Analytical solution of~the~optimal control\linebreak
\\[-12pt]
\hspace*{23pt}task of~a~semi-Markov
process with~finite set of~states&4&72--88\\[.23pt]
\Avtors{Shnurkov~P.\,V., Zasypko~V.\,V., Belousov~V.\,V., and
Gorshenin~A.\,K.} Development of the algorithm of numerical solution
of the optimal investment control problem\linebreak
\\[-12pt]
\hspace*{23pt}in the closed dynamical model of three-sector economy&1&82--95\\[.23pt]
\Avtors{Shorgin~S.\,Ya.} see~Gaidamaka~Yu.\,V.&&\\[.23pt]
\Avtors{Shorgin~V.\,S.} see~Agalarov~Ya.\,M.&&\\[.23pt]
\Avtors{Shubnikov~S.\,K.} see~Minin~V.\,A.&&\\[.23pt]
\Avtors{Sidorkin~I.\,I.} see~Arkhipov~O.\,P.&&\\[.23pt]
\Avtors{Sinitsyn~I.\,N.} Analytical modeling of processes in stochastic
systems with complex fractional\linebreak
\\[-12pt]
\hspace*{23pt}order Bessel nonlinearities&3&55--65\\[.23pt]
\Avtors{Sinitsyn~I.\,N.} Orthogonal supoptimal filters for nonlinear
stochastic systems on manifolds&1&34--44\\[.23pt]
\Avtors{Sinitsyn~I.\,N.\ and Korepanov~E.\,R.} Normal Pugachev
conditionally-optimal filters and extra-\linebreak
\\[-12pt]
\hspace*{23pt}polators for state linear stochastic systems&2&14--23\\[.23pt]
\Avtors{Sinitsyn~I.\,N.\ and Sinitsyn~V.\,I.} Analytical modeling of
distributions in stochastic systems on\linebreak
\\[-12pt]
\hspace*{23pt}manifolds based on ellipsoidal approximation&1&45--55\\[.23pt]
\Avtors{Sinitsyn~I.\,N., Sinitsyn~V.\,I., and
Korepanov~E.\,R.} Ellipsoidal suboptimal filters for nonlinear\linebreak
\\[-12pt]
\hspace*{23pt}stochastic systems on manifolds&2&24--35\\[.23pt]
\Avtors{Sinitsyn~V.\,I.} see~Sinitsyn~I.\,N.&&\\[.23pt]
\Avtors{Sinitsyn~V.\,I.} see~Sinitsyn~I.\,N.&&\\[.23pt]
\Avtors{Skvortsov~N.\,A.} see~Stupnikov~S.\,A.&&\\[.23pt]
\Avtors{Sokolov~I.\,A.} see~Chertok~A.\,V.&&\\
\end{tabular}
}
\pagebreak

\def\leftfootline{\small{\textbf{\thepage}
\hfill INFORMATIKA I EE PRIMENENIYA~--- INFORMATICS AND APPLICATIONS\ \ \ 2016\
\ \ volume~10\ \ \ issue\ 4}
}%
 \def\rightfootline{\small{INFORMATIKA I EE PRIMENENIYA~---
INFORMATICS AND APPLICATIONS\ \ \ 2016\ \ \ volume~10\ \ \ issue\ 4
\hfill \textbf{\thepage}}}

\def\leftkol{2016 AUTHOR INDEX} % ENGLISH ABSTRACTS}

\def\rightkol{2016 AUTHOR INDEX} %ENGLISH ABSTRACTS}


{\tabcolsep=3pt
\begin{tabular}{p{382pt}cc}
&\textbf{Issue} & \textbf{Page}\\[6pt]
\Avtors{Sopin~E.\,S.} see~Gaidamaka~Yu.\,V.&&\\
\Avtors{Strijov~V.\,V.} see~Goncharov~A.\,V.&&\\
\Avtors{Strijov~V.\,V.} see~Isachenko~R.\,V.&&\\
\Avtors{Strijov~V.\,V.} see~Karasikov~M.\,E.&&\\
\Avtors{Stupnikov~S.\,A., Briukhov~D.\,O., and Skvortsov~N.\,A.}
Co-lending systemic risk analysis over\linebreak
\\[-12pt]
\hspace*{23pt}heterogeneous data collections&1&23--33\\
\Avtors{Stupnikov~S.\,A.} see~Kalinichenko~L.\,A.&&\\
\Avtors{Suchkov~A.\,P.} see~Zatsarinny~A.\,A.&&\\
\Avtors{Timonina~E.\,E.} see~Grusho~A.\,A.&&\\
\Avtors{Titova~A.\,I.} see~Kudryavtsev~A.\,A.&&\\
\Avtors{Turlikov~A.\,M.} see~Ometov~A.\,Ya.&&\\
\Avtors{Tyrsin~A.\,N.\ and Serebryanskii~S.\,M.} Recognition of
dependences on the basis of inverse\linebreak
\\[-12pt]
\hspace*{23pt}mapping&2&58--64\\
\Avtors{Ulyanov~V.\,V.} see~Markov~A.\,S.&&\\
\Avtors{Ushakov~V.\,G.} Queueing system with working vacations and
hyperexponential input stream&2&92--97\\
\Avtors{Ushakov~V.\,G.} see~Leontyev~N.\,D.&&\\
\Avtors{Volnova~A.\,A.} see~Kalinichenko~L.\,A.&&\\
\Avtors{Yakovlev~O.\,A.\ and Gasilov~A.\,V.} Speeded-up stereo
matching using geodesic support weights&3&\hphantom{1}98--104\\
\Avtors{Zabezhailo~M.\,I.} see~Grusho~A.\,A.&&\\
\Avtors{Zabezhailo~M.\,I.} see~Grusho~A.\,A.&&\\
\Avtors{Zakharova~T.\,V.\ and Shestakov~O.\,V.} Precision analysis of
wavelet processing of aerodynamic\linebreak
\\[-12pt]
\hspace*{23pt}flow patterns&3&46--54\\
\Avtors{Zalizniak~Anna~A.\ and Kruzhkov~M.\,G.} Database
of~Russian impersonal verbal constructions&4&132--141\\
\Avtors{Zasypko~V.\,V.} see~Shnurkov~P.\,V.&&\\
\Avtors{Zatsarinny~A.\,A.\ and Suchkov~A.\,P.} Systems engineering
approaches to~the~establishment of\linebreak
\\[-12pt]
\hspace*{23pt}a~system for~decision support based
on~situational analysis&4&105--113\\
\Avtors{Zatsarinny~A.\,A.} see~Grusho~A.\,A.&&\\
\Avtors{Zatsman~I.\,M., Inkova~O.\,Yu., Kruzhkov~M.\,G., and
Popkova~N.\,A.} Representation of cross-\linebreak
\\[-12pt]
\hspace*{23pt}lingual knowledge about
connectors in supracorpora databases&1&106--118\\
\Avtors{Zatsman~I.\,M.} see~Minin~V.\,A.&&\\
\Avtors{Zeifman~A.\,I.} see~Korolev~V.\,Yu.&&\\
\Avtors{Zeifman~A.\,I.} see~Korolev~V.\,Yu.&&\\
\end{tabular}
}

%\thispagestyle{myheadings}
\def\leftfootline{\small{\textbf{\thepage}
\hfill INFORMATIKA I EE PRIMENENIYA~--- INFORMATICS AND APPLICATIONS\ \ \ 2016\
\ \ volume~10\ \ \ issue\ 4}
}%
 \def\rightfootline{\small{INFORMATIKA I EE PRIMENENIYA~---
INFORMATICS AND APPLICATIONS\ \ \ 2016\ \ \ volume~10\ \ \ issue\ 4
\hfill \textbf{\thepage}}}

 \label{end\stat}

\newpage

%\def\stat{rekl}
%\label{preobr}

%\def\tit{АКАДЕМИК ПУГАЧЁВ  ВЛАДИМИР СЕМЁНОВИЧ\\
%25.03.1911--25.03.1998}


%   \vspace*{-48pt}
%   \begin{center}\LARGE
%Академик Пугачёв  Владимир Семёнович\\ (25.03.1911--25.03.1998)
%   \end{center}
   
   %\vspace*{2.5mm}
   
   \begin{center}

{\prgsh\LARGE
ОБЪЯВЛЕНИЯ О КОНФЕРЕНЦИЯХ}

\end{center}
%\hrule

\vspace*{6pt}

   
   \vspace*{10mm}
   
   \thispagestyle{empty}

\noindent
\begin{tabular}{cc}
%\begin{center}
\multicolumn{1}{c}{\raisebox{-40pt}[0pt][0pt]{\mbox{%
\epsfxsize=33mm
\epsfbox{vspu.eps}
}}}
%\end{center}
&
\tabcolsep=0pt\begin{tabular}{c}
{\prg{\Large\textbf{XII Всероссийское совещание}}}\\[6pt]
{\prg{\Large\textbf{по проблемам управления}}}\\[12pt]
{\prg{\large 16--19 июня 2014~г.}}\\[6pt] 
{\prg{\large Институт проблем управления имени В.\,А.~Трапезникова РАН}}\\[6pt]
{\prg{\large Москва, Россия}}
\end{tabular}
\end{tabular}

\vspace*{60pt}

     
 { %\large    
 XII Всероссийское совещание по проблемам управления (ВСПУ XII), посвященное 75-летию 
Института проблем управления (ИПУ) имени В.\,А.~Трапезникова РАН, проводится 16--19~июня 
2014~г.\ 
в ИПУ РАН (г.~Москва, Россия). ВСПУ XII организуется ИПУ РАН при поддержке РФФИ, Отделения 
энергетики, машиностроения, механики и процессов управления Российской академии наук, 
Российского 
национального комитета по автоматическому управлению, Академии навигации и управ\-ле\-ния 
движением, 
Научного совета РАН по комплексным проблемам управления и автоматизации, Совета по 
мехатронике и робототехнике РАН. Официальный язык Совещания~--- русский.

\vspace*{24pt}
     
     \textbf{Направления работы}
     \begin{enumerate}[1.]
\item Теория систем управления
\item Управление подвижными объектами и навигация
\item Интеллектуальные системы управления
\item Управление в промышленности, транспортом и логистикой
\item Управление системами междисциплинарной природы
\item Средства измерения, вычислений и контроля в управлении
\item Системный анализ и принятие решений в задачах управления
\item Информационные технологии в управлении
\item Проблемы образования в области управления: современное содержание и технологии обучения
\end{enumerate}

\vspace*{24pt}

     Подробная информация о Совещании находится на сайте {\sf http://vspu2014.ipu.ru}. Срок 
окончательной подачи докладов через систему подачи докладов на сайте~--- \textbf{30~ноября} 
2013~г.
}

%\include{rekl-1}

%\end{document}

%   \vspace*{-48pt}

\begin{center}
\vspace*{6pt}
\mbox{%
\epsfxsize=53.502mm
\epsfbox{foto-1.eps}
}
\end{center}

\vspace*{6pt} %Академик


   \begin{center}
\fbox{\Large\textbf{Профессор Игорь Алексеевич Ушаков}}\\[12pt]
\textbf{\large 22.01.1935--27.02.2015}
   \end{center}


   %\vspace*{2.5mm}

   \vspace*{5mm}

   \thispagestyle{empty}

%\

%\vspace*{-12pt}


Редакционный совет и редакционная коллегия журнала <<Информатика и~её применения>> с~глубоким прискорбием извещают, что 27~февраля 2015~г.\ после тяжелой
и~продолжительной болезни скончался Игорь Алексеевич Ушаков~--- доктор технических наук, профессор, член редколлегии журнала <<Информатика и ее применения>>.

Игорь Алексеевич Ушаков окончил Московский авиационный институт, в~1963~г.\ защитил кандидатскую, а~в~1968~г.~--- докторскую диссертацию. С~1958 по 1989~гг.\ работал в~ряде научно-исследовательских организаций СССР, в~том числе руководил отделами в~НИИ АА и~ВЦ АН СССР; с 1969 по 1989 гг. преподавал в~МФТИ (был профессором, а~затем заведующим кафедрой) и~в~МЭИ. С~1989~г.~---- в~США: являлся профессором университета Дж.\ Вашингтона, университета Дж.\ Мэйсона и~Калифорнийского университета, сотрудником компаний MCI, Qualcomm и Hughes.

И.\,А.~Ушаков с момента основания журнала <<Надежность и~контроль качества>> был заместителем ответственного редактора, а~затем на протяжении многих лет членом редколлегии. В~2006~г.\ основал электронный международный журнал ``Reliability: Theory \& Application'', главным редактором которого оставался до конца жизни.

Учебниками и справочниками по теории надежности, написанными И.\,А.~Ушаковым, пользовались и~пользуются несколько поколений ученых и~специалистов в~разных странах мира.

Игорь Алексеевич всегда уделял огромное внимание работе с~молодежью; более~50 его учеников защитили докторские и~кандидатские диссертации.

И.\,А.~Ушаков вел активную научно-про\-све\-ти\-тель\-скую деятельность. В~частности, он был одним из организаторов и~руководителей Московского кабинета качества и~надежности при Политехническом музее (целью этого Кабинета было оказание консультаций работникам промышленных предприятий и~чтение курсов лекций для инженеров, занимающихся проблемой надежности). Находясь в~США, И.\,А.~Ушаков создал международный ин\-тер\-нет-фо\-рум им.\ Б.\,В.~Гнеденко, объединивший около~400~видных специалистов по приложениям теории вероятностей и~математической статистики, преимущественно в~об\-ласти теории надежности и~анализа риска, из десятков стран мира; коллективным членов этого Форума является и~наш журнал. Цели Форума~--- содействие контактам между специалистами из разных стран, организация обмена профессиональными 
новостями и~информацией (новые публикации, предстоящие события и~др.). Также необходимо отметить большое число на\-уч\-но-по\-пу\-ляр\-ных работ, опубликованных И.\,А.~Ушаковым.

И.\,А.~Ушаков обладал большим личным обаянием, имел широкий круг интересов. Все знавшие И.\,А.~Ушакова всегда будут помнить его как замечательного ученого и~прекрасного человека.

\bigskip

Редакционный совет и редакционная коллегия журнала <<Информатика и~её применения>> 
выражают глубокие соболезнования родным и близким покойного, всем, кто его знал и~работал с~ним.



%\end{document}

%\include{IPPM-25}

\def\stat{cont-rus}
{%\hrule\par
%\vskip 7pt % 7pt
\vspace*{-24pt}
\raggedleft\Large \bf%\baselineskip=3.2ex
Правила подготовки рукописей  для публикации в журнале
<<Информатика~и~её~применения>> \vskip 8pt
    \hrule
    \par
\vskip 14pt plus 6pt minus 3pt }

\label{st\stat}

\def\tit{\ }

\def\aut{\ }
\def\auf{\ }

\def\leftkol{\ }
% Правила подготовки рукописей  для публикации в журнале
%<<Информатика и её применения>>

\def\rightkol{\ }
%Правила подготовки рукописей  для публикации в журнале
%<<Информатика и её применения>>}


\titele{\tit}{\aut}{\auf}{\leftkol}{\rightkol}


\vspace*{-60pt}
{ %\small

Журнал <<Информатика и её применения>>
публикует теоретические, обзорные и дискуссионные статьи,
посвященные научным исследованиям и разработкам в области
информатики и ее приложений.

Журнал издается на русском языке. По специальному решению
редколлегии отдельные статьи могут печататься на английском языке.

Тематика журнала охватывает следующие направления:
\begin{itemize}
\item теоретические основы информатики;\\[-15pt]
      \item
математические методы исследования сложных систем и процессов;\\[-15pt]
           \item
информационные системы и сети;\\[-15pt]
                \item
информационные технологии;\\[-15pt]
                     \item
архитектура и программное обеспечение вычислительных комплексов и сетей.\\[-15pt]
\end{itemize}


\noindent
\begin{enumerate}[1.]
\item В журнале печатаются статьи, содержащие результаты, ранее не опубликованные и
не предназначенные к одновременной публикации в других изданиях.

%Публикация не должна нарушать закон об авторских правах.
Публикация предоставленной автором(ами) рукописи не должна нарушать 
положений глав~69, 70 раздела~VII части~IV Гражданского кодекса, 
которые определяют права на результаты интеллектуальной деятельности 
и~средства индивидуализации, в~том числе авторские права, в~РФ.

Ответственность за нарушение авторских прав, в~случае предъявления претензий к~редакции журнала,  
несут авторы статей.



Направляя рукопись в редакцию, авторы сохраняют свои права на данную
рукопись и при этом передают учредителям и редколлегии журнала неисключительные права на
издание статьи на русском языке 
(или на языке статьи, если он отличен от рус\-ско\-го) и~на перевод ее на английский
язык, а~также на
ее распространение в России и за рубежом. 
Каждый автор должен представить в~редакцию подписанный 
с~его стороны <<Лицензионный договор о~передаче неисключительных прав 
на использование произведения>>, текст которого размещен по адресу 
{\sf http://www.ipiran.ru/publications/licence.doc}. 
Этот договор может быть пред\-став\-лен в~бумажном (в~2-х экз.)\ 
или в~электронном виде (отсканированная копия заполненного и~подписанного документа).




Редколлегия вправе запросить у авторов экспертное заключение о возможности
пуб\-ли\-ка\-ции пред\-став\-лен\-ной статьи в открытой печати.\\[-13.5pt]

\item К статье прилагаются данные автора (авторов) (см.\ п.~8). При наличии нескольких
авторов указывается фамилия автора, ответственного за переписку с редакцией.\\[-13.5pt]

\item Редакция журнала осуществляет экспертизу присланных статей в соответствии с
принятой в журнале процедурой рецензирования.

Возвращение рукописи на доработку не означает ее принятия к печати.

Доработанный вариант с ответом на замечания рецензента необходимо прислать в
редакцию.\\[-13.5pt]

\item Решение редколлегии о публикации статьи или ее отклонении сообщается авторам.

Редколлегия может также направить авторам текст рецензии на их статью. Дискуссия по
поводу отклоненных статей не ведется.\\[-13.5pt]

%\pagebreak

\item Редактура статей высылается авторам для просмотра. Замечания к редактуре должны
быть присланы авторами в кратчайшие сроки.\\[-13.5pt]

\item Рукопись предоставляется в электронном виде в форматах MS WORD (.doc или
.docx) или \LaTeX\  (.tex), дополнительно~--- в формате .pdf, на дискете, лазерном диске
или электронной почтой. Предоставление бумажной рукописи необязательно.\\[-13.5pt]

\item При подготовке рукописи в MS Word рекомендуется использовать следующие
настройки.

Параметры страницы:
формат~--- А4; ориентация~--- книжная; поля (см): внутри~--- 2,5, снаружи~--- 1,5,
сверху~--- 2, снизу~--- 2, от края до нижнего колонтитула~--- 1,3.

Основной текст: стиль~--- <<Обычный>>, шрифт~--- Times New Roman, размер~---
14~пунк\-тов, абзацный отступ~--- 0,5~см, 1,5~интервала, выравнивание~--- по ширине.

\pagebreak

\def\leftkol{Правила подготовки рукописей  для публикации в журнале
<<Информатика и её применения>>}

\def\rightkol{Правила подготовки рукописей  для публикации в журнале
<<Информатика и её применения>>}



Рекомендуемый объем рукописи~--- не свыше 10~страниц указанного формата.
При превышении указанного объема редколлегия вправе потребовать от 
автора сокращения объема рукописи.


Сокращения слов, помимо стандартных, не допускаются. Допускается минимальное
количество аббревиатур.


Все страницы рукописи нумеруются.

Шаблоны оформления представлены в интернете:

\noindent
 {\sf
http://www.ipiran.ru/journal/template\_iiep\_ssi\_2024.zip}\\[-14pt]

\item Статья должна содержать следующую информацию на {\bfseries\textit{русском и
английском языках}}:\\[-16pt]

\begin{itemize}
\item название статьи;\\[-15pt]
\item Ф.И.О.\ авторов, на английском можно только имя и фамилию;\\[-15pt]
\item место работы, с указанием почтового адреса организации и электронного адреса каждого
автора;\\[-15pt]
\item сведения об авторах, в соответствии с форматом, образцы которого
представлены на страницах:



\def\leftfootline{\small{\textbf{\thepage}
\hfill ИНФОРМАТИКА И ЕЁ ПРИМЕНЕНИЯ\ \ \ том\ 18\ \ \ выпуск\ 3\ \ \ 2024}
}%
 \def\rightfootline{\small{ИНФОРМАТИКА И ЕЁ ПРИМЕНЕНИЯ\ \ \ том\ 18\ \ \ выпуск\ 3\ \ \ 2024
\hfill \textbf{\thepage}}}



{\sf http://www.ipiran.ru/journal/issues/2013\_07\_01/authors.asp} и

{\sf http://www.ipiran.ru/journal/issues/2013\_07\_01\_eng/authors.asp};
\item аннотация (не менее 100~слов на каждом из языков). Аннотация~--- это краткое
резюме работы, которое может публиковаться отдельно. Она является основным
источником информации в~ин\-фор\-ма\-ци\-он\-ных системах и базах данных. Английская
аннотация должна быть оригинальной, может не быть дословным переводом русского
текста и должна быть написана хорошим английским языком. В~аннотации не должно
быть ссылок на литературу и, по возможности, формул;\\[-15pt]
\item ключевые слова~--- желательно из принятых в мировой
на\-уч\-но-тех\-ни\-че\-ской литературе тематических тезаурусов. Предложения не
могут быть ключевыми словами;\\[-15pt]
\item источники финансирования работы (ссылки на гранты, проекты,
поддерживающие организации и~т.\,п.).
\end{itemize}



%\pagebreak

\item  Требования к спискам литературы.\\[-14pt]

Ссылки на литературу в тексте статьи нумеруются (в квадратных скобках) и
располагаются в каждом из списков литературы в порядке  первых упоминаний. Если источник имеет DOI и/или EDN,
то их необходимо указывать.

Списки литературы представляются в двух вариантах:\\[-14pt]


\noindent
\begin{enumerate}[(1)]
\item \textbf{Список литературы к русскоязычной части}. Русские и английские
работы~---  на языке и в алфавите оригинала;\\[-14.5pt]
\item  \textbf{References}. Русские работы и работы на других языках~--- в латинской
транслитерации с переводом на английский язык; английские работы и работы на других
языках~--- на языке оригинала.
\end{enumerate}

Необходимо для составления списка ``References'' пользоваться размещенной на сайте
{\sf http://www. translit.net/ru/bgn/} бесплатной программой транслитерации русского
 текста в~латиницу. %, при этом в~за\-клад\-ке <<варианты\ldots>> следует выбратьопцию BGN.

Список литературы ``References'' приводится полностью отдельным блоком, повторяя все
позиции из списка литературы к русскоязычной части, независимо от того, имеются или
нет в нем иностранные источники. Если в списке литературы к русскоязычной части есть
ссылки на иностранные публикации, набранные латиницей, они полностью повторяются в
списке ``References''.

Ниже приведены примеры ссылок на различные виды публикаций в списке ``References''.

\def\leftfootline{\small{\textbf{\thepage}
\hfill ИНФОРМАТИКА И ЕЁ ПРИМЕНЕНИЯ\ \ \ том\ 18\ \ \ выпуск\ 3\ \ \ 2024}
}%
 \def\rightfootline{\small{ИНФОРМАТИКА И ЕЁ ПРИМЕНЕНИЯ\ \ \ том\ 18\ \ \ выпуск\ 3\ \ \ 2024
\hfill \textbf{\thepage}}}

{\small

\noindent
\textbf{Описание статьи из журнала:}

\Aue{Zagurenko, A.\,G., V.\,A.~Korotovskikh, A.\,A.~Kolesnikov, A.\,V.~Timonov, and D.\,V.~Kardymon}. 2008.
Tekhniko-ekonomicheskaya optimizatsiya dizayna gidrorazryva plasta [Technical and
economic optimization of the design
of hydraulic fracturing]. \textit{Neftyanoe hozyaystvo} [\textit{Oil Industry}] 11:54--57.

\Aue{Zhang, Z., and D.~Zhu}. 2008. Experimental research on the localized
electrochemical micromachining. \textit{Russ. J.~Electrochem.}  44(8):926--930.
{\sf doi:10.1134/S1023193508080077}.

\noindent
\textbf{Описание статьи из электронного журнала:}

\Aue{Swaminathan, V., E.~Lepkoswka-White, and B.\,P.~Rao}. 1999. Browsers or buyers in cyberspace? An
investigation of electronic factors influencing electronic exchange. \textit{JCMC}
5(2). Available at: {\sf http://www.ascusc.org/jcmc/vol5/issue2/} (accessed April~28, 2011).

\def\leftkol{Правила подготовки рукописей  для публикации в журнале
<<Информатика и её применения>>}

\def\rightkol{Правила подготовки рукописей  для публикации в журнале
<<Информатика и её применения>>}


\noindent
\textbf{Описание статьи из продолжающегося издания (сборника трудов):}

\Aue{Astakhov, M.\,V., and T.\,V.~Tagantsev}. 2006. Eksperimental'noe
issledovanie prochnosti soedineniy ``stal'--kompozit'' [Experimental study of
the strength of joints ``steel--composite'']. \textit{Trudy MGTU
``Matematicheskoe modelirovanie slozhnykh tekh\-ni\-che\-skikh sistem''}
[\textit{Bauman MSTU ``Mathematical Modeling of Complex Technical
Systems'' Proceedings}]. 593:125--130.


\pagebreak



\noindent
\textbf{Описание материалов конференций:}

\Aue{Usmanov, T.\,S., A.\,A.~Gusmanov, I.\,Z.~Mullagalin, R.\,Ju.~Muhametshina, A.\,N.~Chervyakova, and
A.\,V.~Sveshnikov}. 2007. Osobennosti proektirovaniya razrabotki mestorozhdeniy
s primeneniem gidrorazryva
plasta [Features of the design of field development with the use of hydraulic fracturing].
\textit{Trudy 6-go
Mezhdu\-na\-rod\-no\-go Simpoziuma ``Novye resursosberegayushchie tekhnologii nedropol'zovaniya i povysheniya
neftegazootdachi''} [\textit{6th  Symposium (International) ``New Energy Saving Subsoil Technologies and
the Increasing of the Oil and Gas Impact'' Proceedings}]. Moscow. 267--272.



\def\leftfootline{\small{\textbf{\thepage}
\hfill ИНФОРМАТИКА И ЕЁ ПРИМЕНЕНИЯ\ \ \ том\ 18\ \ \ выпуск\ 3\ \ \ 2024}
}%
 \def\rightfootline{\small{ИНФОРМАТИКА И ЕЁ ПРИМЕНЕНИЯ\ \ \ том\ 18\ \ \ выпуск\ 3\ \ \ 2024
\hfill \textbf{\thepage}}}



\noindent
\textbf{Описание книги (монографии, сборники):}



Lindorf, L.\,S., and L.\,G.~Mamikoniants, eds. 1972.
\textit{Ekspluatatsiya turbogeneratorov s neposredstvennym
okhlazhdeniem} [\textit{Operation of turbine generators with direct cooling}].
Moscow: Energy Publs. 352~p.


\Aue{Latyshev, V.\,N.} 2009. \textit{Tribologiya rezaniya. Kn.~1: Friktsionnye protsessy
pri rezanii metallov}
[\textit{Tribology of cutting. Vol.~1: Frictional processes in metal cutting}]. Ivanovo: Ivanovskii
State Univ. 108~p.

\def\leftkol{Правила подготовки рукописей  для публикации в журнале
<<Информатика и её применения>>}

\def\rightkol{Правила подготовки рукописей  для публикации в журнале
<<Информатика и её применения>>}

\noindent
\textbf{Описание переводной книги}
(в списке литературы к русскоязычной части необходимо указать:~/ Пер.\ с англ.~---
после названия книги, а в конце ссылки указать оригинал книги в круглых скобках):
\begin{enumerate}[1.]
\item  В русскоязычной части:

\def\leftfootline{\small{\textbf{\thepage}
\hfill ИНФОРМАТИКА И ЕЁ ПРИМЕНЕНИЯ\ \ \ том\ 18\ \ \ выпуск\ 3\ \ \ 2024}
}%
 \def\rightfootline{\small{ИНФОРМАТИКА И ЕЁ ПРИМЕНЕНИЯ\ \ \ том\ 18\ \ \ выпуск\ 3\ \ \ 2024
\hfill \textbf{\thepage}}}

\Au{Тимошенко С.\,П., Янг Д.\,Х., Уивер~У.}
Колебания в инженерном деле~/ Пер.\ с англ.~--- М.: Машиностроение, 1985. 472~с.
(\Au{Timoshenko~S.\,P., Young~D.\,H., Weaver~W.}
Vibration problems in engineering.~--- 4th ed.~--- New York, NY, USA: Wiley, 1974. 521~p.)\\[-13.5pt]
\item  В англоязычной части:

\Aue{Timoshenko, S.\,P., D.\,H.~Young, and W.~Weaver}.
1974. \textit{Vibration problems in engineering}. 4th ed. New York: 
Wiley. 521~p.
\end{enumerate}

\vspace*{-3pt}


\noindent
\textbf{Описание неопубликованного документа:}


\Aue{Latypov, A.\,R., M.\,M.~Khasanov, and V.\,A.~Baikov}.
2004 (unpubl.). Geologiya i~dobycha (NGT GiD) [Geology and production (NGT GiD)]. Certificate on official registration of the computer program
No.\,2004611198. 

\noindent
\textbf{Описание интернет-ресурса:}


Pravila tsitirovaniya istochnikov [Rules for the citing of sources]. Available at: {\sf
http://www.scribd.com/doc/1034528/} (accessed February~7, 2011).

%\pagebreak

\noindent
\textbf{Описание диссертации или автореферата диссертации:}

\Aue{Semenov, V.\,I.}
2003. Matematicheskoe modelirovanie plazmy v sisteme kompaktnyy tor [Mathematical
modeling of the plasma in the compact torus].  Moscow.  D.Sc.\ Diss. 272~p.

\Aue{Kozhunova, O.\,S.} 2009. Tekhnologiya razrabotki semanticheskogo
slovarya informatsionnogo monitoringa [Technology of development of
semantic dictionary of information monitoring system].  Moscow: IPI RAN. PhD Thesis. 23~p.


\noindent
\textbf{Описание ГОСТа:}

GOST 8.586.5-2005. 2007. Metodika vypolneniya izmereniy. Izmerenie raskhoda i~kolichestva zhidkostey i~gazov
s~pomoshch'yu standartnykh suzhayushchikh ustroystv [Method of measurement.
Measurement of flow rate and volume of liquids and gases by means of orifice devices]. Moscow:
Standardinform  Publs. 10~p.

\noindent
\textbf{Описание патента:}

\Aue{Bolshakov, M.\,V., A.\,V.~Kulakov, A.\,N.~Lavrenov, and M.\,V.~Palkin}.
2006. Sposob orientirovaniya po krenu letatel'nogo
apparata s opti\-che\-skoy golovkoy
samonavedeniya [The way to orient on the roll of aircraft with optical homing head].
Patent RF No.\,2280590.
}

\item Присланные в редакцию материалы авторам не возвращаются.\\[-13.5pt]

\item При отправке файлов по электронной почте просим придерживаться следующих
правил:
\begin{itemize}
\item указывать в поле subject (тема) название журнала и фамилию автора;\\[-13.5pt]
\item указывать в тексте письма название статьи, авторов и~журнал, в~который направляется статья;\\[-13.5pt]
\item использовать attach (присоединение);\\[-13.5pt]
\item в состав электронной версии статьи должны входить: файл, содержащий текст
статьи, и файл(ы), содержащий(е) иллюстрации.\\[-13.5pt]
\end{itemize}

\item Журнал <<Информатика и её применения>> является некоммерческим изданием.
Плата за публикацию не взимается, гонорар авторам не выплачивается.
\end{enumerate}



\def\leftfootline{\small{\textbf{\thepage}
\hfill ИНФОРМАТИКА И ЕЁ ПРИМЕНЕНИЯ\ \ \ том\ 18\ \ \ выпуск\ 3\ \ \ 2024}
}%
 \def\rightfootline{\small{ИНФОРМАТИКА И ЕЁ ПРИМЕНЕНИЯ\ \ \ том\ 18\ \ \ выпуск\ 3\ \ \ 2024
\hfill \textbf{\thepage}}}


\vspace*{-1mm}

\begin{center}

\textbf{Адрес редакции журнала <<Информатика и её применения>>:} \\




Москва 119333, ул.~Вавилова, д.~44, корп.~2, ФИЦ ИУ РАН\\[-10pt]

\

Тел.: +7\,(499)\,135-86-92\ \ Факс:  +7\,(495)\,930-45-05\\[-10pt]

 \

e-mail:   {\sf iiep@frccsc.ru} (Стригина Светлана Николаевна)\\[-10pt]

\

{\sf http://www.ipiran.ru/journal/issues/}
\end{center}
}


\def\leftkol{Правила подготовки рукописей  для публикации в журнале
<<Информатика и её применения>>}

\def\rightkol{Правила подготовки рукописей  для публикации в журнале
<<Информатика и её применения>>}


\def\leftfootline{\small{\textbf{\thepage}
\hfill ИНФОРМАТИКА И ЕЁ ПРИМЕНЕНИЯ\ \ \ том\ 18\ \ \ выпуск\ 3\ \ \ 2024}
}%
 \def\rightfootline{\small{ИНФОРМАТИКА И ЕЁ ПРИМЕНЕНИЯ\ \ \ том\ 18\ \ \ выпуск\ 3\ \ \ 2024
\hfill \textbf{\thepage}}} 
\def\stat{podg-e}
{%\hrule\par
%\vskip 7pt % 7pt
\vspace*{-24pt}
\raggedleft\Large \bf%\baselineskip=3.2ex
Requirements for manuscripts submitted to Journal
``Informatics~and~Applications'' \vskip 8pt
    \hrule
    \par
\vskip 21pt plus 6pt minus 3pt }

\label{st\stat}

\def\tit{\ }

\def\aut{\ }
\def\auf{\ }

\def\leftkol{\ }

\def\rightkol{\ }
%Requirements for manuscripts submitted to Journal
%``Informatics~and~Applications''}

\titele{\tit}{\aut}{\auf}{\leftkol}{\rightkol}

\def\leftfootline{\small{\textbf{\thepage}
\hfill INFORMATIKA I EE PRIMENENIYA~--- INFORMATICS AND APPLICATIONS\ \ \ 2019\
\ \ volume~13\ \ \ issue\ 4}
}%
 \def\rightfootline{\small{INFORMATIKA I EE PRIMENENIYA~--- INFORMATICS AND APPLICATIONS\ \ \ 2019\ \ \ volume~13\ \ \ issue\ 4
\hfill \textbf{\thepage}}}

\vspace*{-60pt}

{\small

\noindent
Journal ``Informatics and Applications'' (Inform.\ Appl.)
publishes theoretical, review, and discussion
articles on the research and development in the
field of informatics and its applications.

The journal is published in Russian.
By a special decision of the editorial
board, some articles can be published in English.


The topics covered include the following areas:
\begin{itemize}
               \item
     theoretical fundamentals of informatics; \\[-14pt]
\item
mathematical methods for studying complex systems and processes; \\[-14pt]
\item
information systems and networks;\\[-14pt]
\item
information technologies; and \\[-14pt]
\item
architecture and software of computational complexes and networks. \\[-14pt]
\end{itemize}

\noindent
\begin{enumerate}[1.]
\item The Journal publishes original articles which have not been published before and are not
intended for simultaneous publication in other editions. An article submitted to the Journal must not violate the
Copyright law. Sending the manuscript to the Editorial Board, the authors retain all rights of the
owners of the manuscript and transfer the nonexclusive rights to publish the article in Russian
(or the language of the article, if not Russian) and its distribution in Russia and abroad to the
Founders and the Editorial Board. Authors should submit a letter to the Editorial Board in the
following form:

{\bfseries\textit{Agreement on the transfer of rights to publish:}}

``\textit{We, the undersigned authors of the manuscript ``\ldots'', pass to the
Founder and the Editorial Board of the Journal ``Informatics and Applications''
the nonexclusive right to publish the manuscript of the article in Russian (or
in English) in both print and electronic versions of the Journal. We affirm
that this publication does not violate the Copyright of other persons or
organizations.}

\textit{Author(s) signature(s): (name(s), address(es), date).}

This agreement should be submitted in paper form or in the form of a scanned copy (signed by
the authors).


%The Editorial Board has the right to request from the authors an official expert conclusion that
%the submitted article has no secret data prohibited for publication. \\[-13.5pt]
\item
A submitted article should be attached with \textbf{the data on the author(s)} (see item~8). If
there are several authors, the contact person should be indicated who is responsible for
correspondence with the Editorial Board and other authors about revisions and final approval
of the proofs.\\[-13.5pt]

\item The Editorial Board of the Journal examines the article according to the established
reviewing procedure. If the authors receive their article for correction after reviewing, it does not
mean that the article is approved for publication. The corrected article should be sent to the
Editorial Board for the subsequent review and approval.\\[-13.5pt]

\item The decision on the article publication or its rejection is communicated to the authors. The
Editorial Board may also send the reviews on the submitted articles to the authors. Any
discussion upon the rejected articles is not possible.\\[-13.5pt]

\item The edited articles will be sent to the authors for proofread. The comments of the authors
to the edited text of the article should be sent to the Editorial Board as soon as possible.\\[-13.5pt]

\item The manuscript of the article should be presented electronically in the MS WORD (.doc or
.docx) or \LaTeX\ (.tex) formats, and additionally in the .pdf format. All documents
 may be sent
by e-mail or provided on a CD or diskette. A~hard copy submission is not necessary.\\[-13.5pt]

\item The recommended typesetting instructions for manuscript.

Pages parameters: format A4, portrait orientation, document margins (cm): left~--- 2.5, right~---
1.5, above~--- 2.0, below~--- 2.0, footer 1.3.

Text: font~---Times New Roman, font size~--- 14, paragraph indent~--- 0.5, line spacing~--- 1.5,
justified alignment.

The recommended manuscript size: not more than 15~pages of the specified format.
If the specified size exceeded, the editorial board is entitled to require the author
to reduce the manuscript.

Use only standard abbreviations. Avoid  abbreviations in the title and
abstract. The full term for which an abbreviation stands should precede
its first use in the text unless it is a standard unit of measurement.

All pages of the manuscript should be numbered.

The templates for the manuscript typesetting are presented on site: {\sf
http://www.ipiran.ru/journal/template.doc}.\\[-13.5pt]


%\def\leftkol{Requirements for manuscripts submitted to Journal
%``Informatics~and~Applications''}

\item The articles should enclose data both in \textbf{Russian and English}:
\begin{itemize}
\item title;\\[-13.5pt]
\item author's name and surname;\\[-13.5pt]
\item affiliation~--- organization, its address with ZIP code, city, country, and
official e-mail address;\\[-13.5pt]
\item data on authors according to the format: (see site)

{\sf http://www.ipiran.ru/journal/issues/2013\_07\_01/authors.asp}  and

{\sf  http://www.ipiran.ru/journal/issues/2013\_07\_01\_eng/authors.asp};\\[-13.5pt]

\pagebreak

\def\leftfootline{\small{\textbf{\thepage}
\hfill INFORMATIKA I EE PRIMENENIYA~--- INFORMATICS AND APPLICATIONS\ \ \ 2019\
\ \ volume~13\ \ \ issue\ 4}
}%
 \def\rightfootline{\small{INFORMATIKA I EE PRIMENENIYA~--- INFORMATICS AND APPLICATIONS\ \ \ 2019\ \ \ volume~13\ \ \ issue\ 4
\hfill \textbf{\thepage}}}


%\def\leftkol{Requirements for manuscripts submitted to Journal
%``Informatics~and~Applications''}

%\def\rightkol{Requirements for manuscripts submitted to Journal
%``Informatics~and~Applications''}



\item abstract (not less than 100 words) both in Russian and in English. Abstract is a short
summary of the article that can be published separately. The abstract is the
main source of information on the article and it could be included in leading information
systems and data bases. The abstract in English has to be an original text and should
not be an exact translation of the Russian one. Good English is required.
In abstracts, avoid references and formulae;\\[-13.5pt]
\item indexing is performed on the basis of keywords. The use of keywords from the
internationally accepted thematic Thesauri is recommended.

%\def\leftkol{Requirements for manuscripts submitted to Journal
%``Informatics~and~Applications''}

%\def\rightkol{Requirements for manuscripts submitted to Journal
%``Informatics~and~Applications''}

Important! Keywords must not be sentences;
\item Acknowledgments.
\end{itemize}

\item References. Russian references have to be presented both in English translation and Latin
transliteration (refer {\sf http://www.translit.net/ru/bgn/}).

Please take into account the following examples of Russian references appearance:

\noindent
\textbf{Article in journal:}

\Aue{Zhang, Z., and D.~Zhu}. 2008. Experimental research on the localized electrochemical
micromachining.
\textit{Rus. J.~Electrochem.}  44(8):926--930. {\sf doi:10.1134/S1023193508080077}.


\noindent
\textbf{Journal article in electronic format:}

\Aue{Swaminathan, V., E.~Lepkoswka-White, and B.\,P.~Rao}. 1999. Browsers or buyers in
cyberspace? An
investigation of electronic factors influencing electronic exchange. \textit{JCMC}
5(2). Available at: {\sf http://www.ascusc.org/jcmc/vol5/issue2/} (accessed April~28, 2011).




\noindent
\textbf{Article from the continuing publication (collection of works, proceedings):}

\Aue{Astakhov, M.\,V., and T.\,V.~Tagantsev}. 2006. Eksperimental'noe
issledovanie prochnosti soedineniy ``stal'--kompozit'' [Experimental study of
the strength of joints ``steel--composite'']. \textit{Trudy MGTU
``Matematicheskoe modelirovanie slozhnykh tekh\-ni\-che\-skikh sistem''}
[\textit{Bauman MSTU ``Mathematical Modeling of Complex Technical
Systems'' Proceedings}]. 593:125--130.

\def\leftfootline{\small{\textbf{\thepage}
\hfill INFORMATIKA I EE PRIMENENIYA~--- INFORMATICS AND APPLICATIONS\ \ \ 2019\
\ \ volume~13\ \ \ issue\ 4}
}%
 \def\rightfootline{\small{INFORMATIKA I EE PRIMENENIYA~--- INFORMATICS AND APPLICATIONS\ \ \ 2019\ \ \ volume~13\ \ \ issue\ 4
\hfill \textbf{\thepage}}}

\def\leftkol{Requirements for manuscripts submitted to Journal
``Informatics~and~Applications''}

\def\rightkol{Requirements for manuscripts submitted to Journal
``Informatics~and~Applications''}

\noindent
\textbf{Conference proceedings:}

\Aue{Usmanov, T.\,S., A.\,A.~Gusmanov, I.\,Z.~Mullagalin, R.\,Ju.~Muhametshina,
A.\,N.~Chervyakova, and
A.\,V.~Sveshnikov}. 2007. Osobennosti proektirovaniya razrabotki mestorozhdeniy
s primeneniem gidrorazryva
plasta [Features of the design of field development with the use of hydraulic fracturing].
\textit{Trudy 6-go
Mezhdu\-na\-rod\-no\-go Simpoziuma ``Novye resursosberegayushchie tekhnologii
nedropol'zovaniya i povysheniya
neftegazootdachi''} [\textit{6th  Symposium (International) ``New Energy Saving Subsoil
Technologies and
the Increasing of the Oil and Gas Impact'' Proceedings}]. Moscow. 267--272.


\noindent
\textbf{Books and other monographs:}




Lindorf, L.\,S., and L.\,G.~Mamikoniants, eds. 1972.
\textit{Ekspluatatsiya turbogeneratorov s neposredstvennym
okhlazhdeniem} [\textit{Operation of turbine generators with direct cooling}].
Moscow: Energy Publs. 352~p.


%\Aue{Latyshev, V.\,N.} 2009. \textit{Tribologiya rezaniya. Kn.~1: Frikcionnye prosessy
%pri rezanii metallov}
%[\textit{Tribology of cutting. Vol.~1: Frictional processes in metal cutting}]. Ivanovo: Ivanovskii
%State Univ. 108~p.


%\noindent
%\textbf{Unpublished material:}

%\Aue{Latypov, A.\,R., M.\,M.~Khasanov, and V.\,A.~Baikov}.
%2004. Geology and production (NGT GiD). Certificate on official registration of the computer
%program
%No.\,2004611198. (In Russian, unpubl.)

%\noindent
%\textbf{Internet-source:}

%APA Style. 2011. Available at: {\sf http://www.apastyle.org/apa-style-help.aspx} (accessed
%February~5, 2011).

%Pravila citirovaniya istochnikov [Rules for the citing of sources]. Available at: {\sf
%http://www.scribd.com/doc/1034528/} (accessed February~7, 2011).


\noindent
\textbf{Dissertation and Thesis:}

%\Aue{Semenov, V.\,I.}
%2003. Matematicheskoe modelirovanie plazmy v sisteme kompaktnyy tor. [Mathematical
%modeling of the plasma in the compact torus]. D.Sc.\ Diss. Moscow. 272~p.

\Aue{Kozhunova, O.\,S.} 2009. Tekhnologiya razrabotki semanticheskogo
slovarya informatsionnogo monitoringa [Technology of development of
semantic dictionary of information monitoring system]. PhD Thesis. Moscow: IPI RAN. 23~p.


\noindent
\textbf{State standards and patents:}

GOST 8.586.5-2005. 2007. Metodika vypolneniya izmereniy. Izmerenie raskhoda i~kolichestva
zhidkostey i gazov 
s~pomoshch'yu standartnykh suzhayushchikh ustroystv [Method of measurement.
Measurement of flow rate and volume of liquids and gases by means of orifice devices]. M.:
Standardinform
Publs. 10~p.

%\noindent
%\textbf{Patent:}

\Aue{Bolshakov, M.\,V., A.\,V.~Kulakov, A.\,N.~Lavrenov, and M.\,V.~Palkin}.
2006. Sposob orientirovaniya po krenu letatel'nogo
apparata s opti\-che\-skoy golovkoy
samonavedeniya [The way to orient on the roll of aircraft with optical homing head].
Patent RF No.\,2280590.

References in Latin transcription are presented in the original language.

References in the text are numbered according to the order of their
first appearance; the number is
placed in square brackets. All items from the reference list should be
cited.\\[-13.5pt]

\item Manuscripts and additional materials are not returned to Authors by the Editorial Board.\\[-13.5pt]

\item Submissions of files by e-mail must include:\\[-13.5pt]
\begin{itemize}
\item   the journal title and author's name in the ``Subject'' field; \\[-13.5pt]
\item   an article and additional materials have to be attached using the ``attach'' function;\\[-13.5pt]
\item   an electronic version of the article should contain the file with the text and a separate file
with figures.\\[-13.5pt]
\end{itemize}

\item ``Informatics and Applications'' journal is not a profit publication. There are no
charges for the authors as well as there are no royalties.\\[-13.5pt]
\end{enumerate}

\def\leftfootline{\small{\textbf{\thepage}
\hfill INFORMATIKA I EE PRIMENENIYA~--- INFORMATICS AND APPLICATIONS\ \ \ 2019\
\ \ volume~13\ \ \ issue\ 4}
}%
 \def\rightfootline{\small{INFORMATIKA I EE PRIMENENIYA~--- INFORMATICS AND APPLICATIONS\ \ \ 2019\ \ \ volume~13\ \ \ issue\ 4
\hfill \textbf{\thepage}}}

\def\leftkol{Requirements for manuscripts submitted to Journal
``Informatics~and~Applications''}

\def\rightkol{Requirements for manuscripts submitted to Journal
``Informatics~and~Applications''}


%\vspace*{5mm}


\begin{center}
\textbf{Editorial Board address:} \\

%ABOUT AUTHORS



FRC CSC RAS, 44, block~2, Vavilov Str., Moscow 119333, Russia\\[-10pt]

\

Ph.: +7\,(499)\,135\,86\,92,\ \ Fax: +7\,(495)\,930\,45\,05\\[-10pt]

\

 e-mail: {\sf rust@ipiran.ru} (to Prof.\ Rustem Seyful-Mulyukov)\\[-10pt]

\

 {\sf http://www.ipiran.ru/english/journal.asp}
\end{center}
 }
%\thispagestyle{myheadings}

\def\leftkol{Requirements for manuscripts submitted to Journal
``Informatics~and~Applications''}

\def\rightkol{Requirements for manuscripts submitted to Journal
``Informatics~and~Applications''}

\def\leftfootline{\small{\textbf{\thepage}
\hfill INFORMATIKA I EE PRIMENENIYA~--- INFORMATICS AND APPLICATIONS\ \ \ 2019\
\ \ volume~13\ \ \ issue\ 4}
}%
 \def\rightfootline{\small{INFORMATIKA I EE PRIMENENIYA~--- INFORMATICS AND APPLICATIONS\ \ \ 2019\ \ \ volume~13\ \ \ issue\ 4
\hfill \textbf{\thepage}}}

 \label{end\stat}

\newpage

%\vspace*{-60pt} {\small
{\baselineskip=9.1pt
\section*{Правила подготовки рукописей статей для публикации в журнале
<<Информатика и её применения>>}

\thispagestyle{empty}

 Журнал <<Информатика и её применения>> публикует
теоретические, обзорные и дискуссионные статьи, посвященные научным
исследованиям и разработкам в области информатики и ее приложений. Журнал
издается на русском языке. По специальному решению редколлегии отдельные статьи,
в виде исключения, могут печататься на английском языке.
Тематика журнала охватывает следующие направления:
\begin{itemize}
\item теоретические основы информатики; %\\[-13.5pt]
\item математические методы исследования сложных систем и процессов; %\\[-13.5pt]
\item информационные системы и сети; %\\[-13.5pt]
\item информационные технологии; %\\[-13.5pt]
\item архитектура и программное
обеспечение вычислительных комплексов и сетей.
\end{itemize}
\begin{enumerate}
\item В журнале печатаются результаты, ранее не
опубликованные и не предназначенные к одновременной публикации в других
изданиях. Публикация не должна нарушать закон об авторских правах. Направляя
свою рукопись в редакцию, авторы автоматически передают учредителям и
редколлегии неисключительные права на издание данной статьи на русском языке и
на ее распространение в России и за рубежом. При этом за авторами сохраняются
все права как собственников данной рукописи. В связи с этим авторами должно
быть представлено в редакцию письмо в следующей форме:
Соглашение о передаче права на публикацию:

\textit{<<Мы, нижеподписавшиеся, авторы рукописи <<$\qquad\qquad$>>, передаем
учредителям и редколлегии журнала <<Информатика и её применения>>
неисключительное право опубликовать данную рукопись статьи на русском языке как
в печатной, так и в электронной версиях журнала. Мы подтверждаем, что данная
публикация не нарушает авторского права других лиц или организаций. Подписи
авторов: (ф.\,и.\,о., дата, адрес)>>.}

Указанное соглашение может быть представлено 
как в бумажном виде, так и в виде отсканированной копии (с подписями авторов).


Редколлегия вправе запросить у авторов экспертное заключение о возможности
опубликования представленной статьи в открытой печати. %\\[-13.5pt]
\item Статья
подписывается всеми авторами. На отдельном листе представляются данные автора
(или всех авторов): фамилия, полные имя и отчество, телефон, факс, e-mail,
почтовый адрес. Если работа выполнена несколькими авторами, указывается фамилия
одного из них, ответственного за переписку с редакцией. %\\[-13.5pt]
\item Редакция журнала
осуществляет самостоятельную экспертизу присланных статей. Возвращение рукописи
на доработку не означает, что статья уже принята к печати. Доработанный вариант
с ответом на замечания рецензента необходимо прислать в редакцию. %\\[-13.5pt]
\item Решение
редакционной коллегии о принятии статьи к печати или ее отклонении сообщается
авторам. Редколлегия не обязуется направлять рецензию авторам отклоненной
статьи. %\\[-13.5pt]
\item Корректура статей высылается авторам для просмотра. Редакция
просит авторов присылать свои замечания в кратчайшие сроки. %\\[-13.5pt]
\item При
подготовке рукописи в MS Word рекомендуется использовать следующие настройки.
Параметры страницы: формат~--- А4; ориентация~--- книжная; поля (см): внутри~---
2,5, снаружи~--- 1,5, сверху~--- 2, снизу~--- 2, от края до нижнего
колонтитула~--- 1,3. Основной текст: стиль~--- <<Обычный>>: шрифт Times New
Roman, размер 14~пунктов, абзацный отступ~--- 0,5~см, 1,5 интервала,
выравнивание~--- по ширине. Рекомендуемый объем рукописи~--- не свыше
25~страниц указанного формата. Ознакомиться с шаблонами, содержащими примеры
оформления, можно по адресу в Интернете:
\textsf{http://www.ipiran.ru/journal/template.doc}.
\item К рукописи, предоставляемой в 2-х
экземплярах, обязательно прилагается электронная версия статьи (как правило, в
форматах MS WORD (.doc) или \LaTeX\ (.tex), а также~--- дополнительно~--- в
формате .pdf) на дискете, лазерном диске или по электронной почте. Сокращения
слов, кроме стандартных, не применяются. Все страницы рукописи должны быть
пронумерованы. %\\[-13.5pt]
\item Статья должна содержать следующую информацию на русском и
английском языках: название, Ф.И.О. авторов, места работы авторов и их
электронные адреса, подробные сведения об авторах, оформленные в соответствии с форматом, 
определяемым файлами {\sf http://www.ipiran.ru/journal/issues/2011\_05\_01/authors.asp} и 
{\sf http://www.ipiran.ru/journal/issues/2011\_01\_eng/authors.asp},
аннотация (не более 100~слов), ключевые слова. Ссылки на
литературу в тексте статьи нумеруются (в квадратных скобках) и располагаются в
порядке их первого упоминания. В~списке литературы не должно быть позиций, на которые нет ссылки в тексте статьи.
Все фамилии авторов, заглавия статей, названия
книг, конференций и~т.\,п.\ даются на языке оригинала, если этот язык
использует кириллический или латинский алфавит. %\\[-13.5pt]
\item Присланные в редакцию материалы авторам не возвращаются.
\item При отправке файлов по электронной
почте просим придерживаться следующих правил:
\begin{itemize}
\item указывать в поле subject (тема) название журнала и фамилию автора; %\\[-13.5pt]
\item использовать attach (присоединение); %\\[-13.5pt]
\item в случае больших объемов информации возможно
использование общеизвестных архиваторов (ZIP, RAR); %\\[-13.5pt]
\item в состав электронной версии статьи должны входить: файл, содержащий текст статьи, и файл(ы),
содержащий(е) иллюстрации. %\\[-13.5pt]
\end{itemize}
\item Журнал <<Информатика и её применения>> является некоммерческим изданием. 
Плата за публикацию с авторов не взимается, гонорар авторам не выплачивается.
\end{enumerate}
\thispagestyle{empty}
\textbf{Адрес редакции:} Москва 119333,
ул.~Вавилова, д.~44, корп.~2, ИПИ РАН\\
\hphantom{\textbf{Адрес редакции:} }Тел.: +7 (499) 135-86-92\ \
Факс:  +7 (495) 930-45-05\ \  E-mail:   rust@ipiran.ru }
}

%\include{ipi-ind}

\end{document}


%\tableofcontents

%\end{document}





%\def\stat{cont}
{%\hrule\par
%\vskip 7pt % 7pt
\raggedleft\Large \bf%\baselineskip=3.2ex
А\,В\,Т\,О\,Р\,С\,К\,И\,Й\ \ У\,К\,А\,З\,А\,Т\,Е\,Л\,Ь\ \ З\,А\ \ 2\,0\,0\,7 г. \vskip 17pt
    \hrule
    \par
\vskip 21pt plus 6pt minus 3pt }

\label{st\stat}

\def\tit{\ }

\def\aut{\ }
\def\auf{\ }

\def\leftkol{\ } % ENGLISH ABSTRACTS}

\def\rightkol{\ } %ENGLISH ABSTRACTS}

\titele{\tit}{\aut}{\auf}{\leftkol}{\rightkol}


\contentsline {chapter}{\ }{Выпуск \quad Стр.} 
\contentsline {section}{\textbf{Батракова Д.\,А., Королев В.\,Ю., Шоргин С.\,Я.}\ \ Новый метод вероятностно-ста\-ти\-сти\-че\-ско\-го анализа информационных потоков в\nobreakspace {}телекоммуникационных сетях}{\qquad 1 \qquad 40} 
\contentsline {section}{\textbf{Борисов А.\,В.}\ \ Байесовское оценивание в системах наблюдения с\nobreakspace {}марковскими скачкообразными процессами: игровой подход}{\qquad 2 \qquad 65}
\contentsline {section}{\textbf{Босов А.\,В., Иванов А.\,В.}\ \ Программная инфраструктура информационного Web-пор\-тала}{\qquad 2 \qquad 50}
\contentsline {section}{\textbf{Захаров В.\,Н., Калиниченко Л.\,А., Соколов И.\,А., Ступников С.\,А.}\ \ Конструирование канонических информационных моделей для интегрированных информационных систем}{\qquad 2 \qquad 15}
\contentsline {section}{\textbf{Захаров В.\,Н., Козмидиади В.\,А.}\ \ Средства обеспечения отказоустойчивости при\-ло\-жений}{\qquad 1 \qquad 14} 
\contentsline {section}{\textbf{Иванов А.\,В.}\ \ см. Босов А.\,В.\hfill\hfill\hfill\hfill\hfill\hfill\hfill\hfill\hfill\hfill\hfill\hfill\hfill\hfill\hfill\hfill\hfill\hfill\hfill\hfill\hfill\hfill\hfill\hfill\hfill\hfill\hfill\hfill\hfill\hfill\hfill\hfill\hfill\hfill\hfill}{\ }
\contentsline {section}{\textbf{Ильин В.\,Д., Соколов И.\,А.}\ \ Символьная модель системы знаний информатики в\nobreakspace {}че\-ло\-ве\-ко-автоматной среде}{\qquad 1 \qquad 66} 
\contentsline {section}{\textbf{Калиниченко Л.\,А.}\ \ см. Захаров В.\,Н.\hfill\hfill\hfill\hfill\hfill\hfill\hfill\hfill\hfill\hfill\hfill\hfill\hfill\hfill\hfill\hfill\hfill\hfill\hfill\hfill\hfill\hfill\hfill\hfill\hfill\hfill\hfill\hfill\hfill\hfill\hfill\hfill\hfill\hfill\hfill}{\ }
\contentsline {section}{\textbf{Козеренко Е.\,Б.}\ \ Лингвистическое моделирование для систем машинного перевода и обработки знаний}{\qquad 1 \qquad 54} 
\contentsline {section}{\textbf{Козмидиади В.\,А.}\ \ см. Захаров В.\,Н.\hfill\hfill\hfill\hfill\hfill\hfill\hfill\hfill\hfill\hfill\hfill\hfill\hfill\hfill\hfill\hfill\hfill\hfill\hfill\hfill\hfill\hfill\hfill\hfill\hfill\hfill\hfill\hfill\hfill\hfill\hfill\hfill\hfill\hfill\hfill }{\ } 
\contentsline {section}{\textbf{Королев В.\,Ю.}\ \ см. Батракова Д.\,А.\hfill\hfill\hfill\hfill\hfill\hfill\hfill\hfill\hfill\hfill\hfill\hfill\hfill\hfill\hfill\hfill\hfill\hfill\hfill\hfill\hfill\hfill\hfill\hfill\hfill\hfill\hfill\hfill\hfill\hfill\hfill\hfill\hfill\hfill\hfill}{\ } 
\contentsline {section}{\textbf{Кудрявцев А.\,А., Шоргин С.\,Я.}\ \ Байесовский подход к\nobreakspace {}анализу систем массового обслуживания и\nobreakspace {}показателей надежности}{\qquad 2 \qquad 76}
\contentsline {section}{\textbf{Печинкин А.\,В., Соколов И.\,А., Чаплыгин В.\,В.}\ \ Многолинейная система массового обслуживания с конечным накопителем и ненадежными приборами}{\qquad 1 \qquad 27} 
\contentsline {section}{\textbf{Печинкин А.\,В., Соколов И.\,А., Чаплыгин В.\,В.}\ \ Стационарные характеристики многолинейной\nobreakspace {}системы массового обслуживания с\nobreakspace {}одновременными отказами приборов}{\qquad 2 \qquad 39}
\contentsline {section}{\textbf{Синицын И.\,Н.}\ \ Корреляционные методы построения аналитических информационных моделей флуктуаций полюса Земли по априорным данным}{\qquad 2 \qquad \hphantom{9}2}
\contentsline {section}{\textbf{Синицын И.\,Н.}\ \ Развитие теории фильтров Пугачева для оперативной обработки информации в стохастических системах}{{\qquad 1 \qquad \hphantom{9}3}} 
\contentsline {section}{\textbf{Соколов И.\,А.}\ \ см. Захаров В.\,Н.\hfill\hfill\hfill\hfill\hfill\hfill\hfill\hfill\hfill\hfill\hfill\hfill\hfill\hfill\hfill\hfill\hfill\hfill\hfill\hfill\hfill\hfill\hfill\hfill\hfill\hfill\hfill\hfill\hfill\hfill\hfill\hfill\hfill\hfill\hfill}{\ }
\contentsline {section}{\textbf{Соколов И.\,А.}\ \ см. Ильин В.\,Д.\hfill\hfill\hfill\hfill\hfill\hfill\hfill\hfill\hfill\hfill\hfill\hfill\hfill\hfill\hfill\hfill\hfill\hfill\hfill\hfill\hfill\hfill\hfill\hfill\hfill\hfill\hfill\hfill\hfill\hfill\hfill\hfill\hfill\hfill\hfill}{\ } 
\contentsline {section}{\textbf{Соколов И.\,А.}\ \ см. Печинкин А.\,В.\hfill\hfill\hfill\hfill\hfill\hfill\hfill\hfill\hfill\hfill\hfill\hfill\hfill\hfill\hfill\hfill\hfill\hfill\hfill\hfill\hfill\hfill\hfill\hfill\hfill\hfill\hfill\hfill\hfill\hfill\hfill\hfill\hfill\hfill\hfill}{\ } 
\contentsline {section}{\textbf{Соколов И.\,А.}\ \ см. Печинкин А.\,В.\hfill\hfill\hfill\hfill\hfill\hfill\hfill\hfill\hfill\hfill\hfill\hfill\hfill\hfill\hfill\hfill\hfill\hfill\hfill\hfill\hfill\hfill\hfill\hfill\hfill\hfill\hfill\hfill\hfill\hfill\hfill\hfill\hfill\hfill\hfill}{\ }
\contentsline {section}{\textbf{Ступников С.\,А.}\ \ см. Захаров В.\,Н.\hfill\hfill\hfill\hfill\hfill\hfill\hfill\hfill\hfill\hfill\hfill\hfill\hfill\hfill\hfill\hfill\hfill\hfill\hfill\hfill\hfill\hfill\hfill\hfill\hfill\hfill\hfill\hfill\hfill\hfill\hfill\hfill\hfill\hfill\hfill}{\ }
\contentsline {section}{\textbf{Чаплыгин В.\,В.}\ \ см. Печинкин А.\,В.\hfill\hfill\hfill\hfill\hfill\hfill\hfill\hfill\hfill\hfill\hfill\hfill\hfill\hfill\hfill\hfill\hfill\hfill\hfill\hfill\hfill\hfill\hfill\hfill\hfill\hfill\hfill\hfill\hfill\hfill\hfill\hfill\hfill\hfill\hfill}{\ } 
\contentsline {section}{\textbf{Чаплыгин В.\,В.}\ \ см. Печинкин А.\,В.\hfill\hfill\hfill\hfill\hfill\hfill\hfill\hfill\hfill\hfill\hfill\hfill\hfill\hfill\hfill\hfill\hfill\hfill\hfill\hfill\hfill\hfill\hfill\hfill\hfill\hfill\hfill\hfill\hfill\hfill\hfill\hfill\hfill\hfill\hfill}{\ }
\contentsline {section}{\textbf{Шоргин С.\,Я.}\ \ см. Батракова Д.\,А.\hfill\hfill\hfill\hfill\hfill\hfill\hfill\hfill\hfill\hfill\hfill\hfill\hfill\hfill\hfill\hfill\hfill\hfill\hfill\hfill\hfill\hfill\hfill\hfill\hfill\hfill\hfill\hfill\hfill\hfill\hfill\hfill\hfill\hfill\hfill}{\ } 
\contentsline {section}{\textbf{Шоргин С.\,Я.}\ \ см. Кудрявцев А.\,А.\hfill\hfill\hfill\hfill\hfill\hfill\hfill\hfill\hfill\hfill\hfill\hfill\hfill\hfill\hfill\hfill\hfill\hfill\hfill\hfill\hfill\hfill\hfill\hfill\hfill\hfill\hfill\hfill\hfill\hfill\hfill\hfill\hfill\hfill\hfill}{\ }
%\thispagestyle{myheadings}
\def\leftfootline{\small{\textbf{\thepage}
\hfill ИНФОРМАТИКА И ЕЁ ПРИМЕНЕНИЯ\ \ \ том~1\ \ \ выпуск~2\ \ \ 2007}
}%
 \def\rightfootline{\small{ИНФОРМАТИКА И ЕЁ ПРИМЕНЕНИЯ\ \ \ том~1\ \ \ выпуск~2\ \ \ 2007
 \hfill \textbf{\thepage}}}
 \label{end\stat}

%\def\stat{cont-e}
{%\hrule\par
%\vskip 7pt % 7pt
\raggedleft\Large \bf%\baselineskip=3.2ex
2\,0\,0\,7\ \ A\,U\,T\,H\,O\,R\ \ I\,N\,D\,E\,X \vskip 17pt
    \hrule
    \par
\vskip 21pt plus 6pt minus 3pt }

\label{st\stat}

\def\tit{\ }

\def\aut{\ }
\def\auf{\ }

\def\leftkol{\ } % ENGLISH ABSTRACTS}

\def\rightkol{\ } %ENGLISH ABSTRACTS}

\titele{\tit}{\aut}{\auf}{\leftkol}{\rightkol}


\contentsline {chapter}{\ }{Issue \quad Page} 
\contentsline {subsection}{\textbf{Batrakova D.\,A., Korolev V.\,Yu., Shorgin S.\,Ya.}\ \ A New Method for the Probabilistic and Statistical Analysis of Information Flows in Telecommunication Networks}{\qquad 1 \qquad 40} 
\contentsline {subsection}{\textbf{Borisov A.\,V.}\ \ Bayesian Estimation in\nobreakspace {}Observation Systems with\nobreakspace {}Markov Jump Processes: Game-Theoretic Approach}{\qquad 2 \qquad 65} 
\contentsline {subsection}{\textbf{Bosov A.\,V., Ivanov A.\,V.}\ \ Linguistic Simulation for Machine Translation and Knowledge Management Systems}{\qquad 2 \qquad 50} 
\contentsline {subsection}{\textbf{Chaplygin V.\,V.} see Pechinkin A.\,V.\hfill\hfill\hfill\hfill\hfill\hfill\hfill\hfill\hfill\hfill\hfill\hfill\hfill\hfill\hfill\hfill\hfill\hfill\hfill\hfill\hfill\hfill\hfill\hfill\hfill\hfill\hfill\hfill\hfill\hfill\hfill\hfill\hfill\hfill\hfill}{\ }
\contentsline {subsection}{\textbf{Chaplygin V.\,V.} see Pechinkin A.\,V.\hfill\hfill\hfill\hfill\hfill\hfill\hfill\hfill\hfill\hfill\hfill\hfill\hfill\hfill\hfill\hfill\hfill\hfill\hfill\hfill\hfill\hfill\hfill\hfill\hfill\hfill\hfill\hfill\hfill\hfill\hfill\hfill\hfill\hfill\hfill}{\ }
\contentsline {subsection}{\textbf{Ilyin V.\,D., Sokolov I.\,A.}\ \ The Symbol Model of Informatics Knowledge System in Human-Automaton Environment}{\qquad 1 \qquad 66} 
\contentsline {subsection}{\textbf{Ivanov A.\,V.} see Bosov A.\,V.\hfill\hfill\hfill\hfill\hfill\hfill\hfill\hfill\hfill\hfill\hfill\hfill\hfill\hfill\hfill\hfill\hfill\hfill\hfill\hfill\hfill\hfill\hfill\hfill\hfill\hfill\hfill\hfill\hfill\hfill\hfill\hfill\hfill\hfill\hfill}{\ }
\contentsline {subsection}{\textbf{Kalinichenko L.\,A.} see Zakharov V.\,N.\hfill\hfill\hfill\hfill\hfill\hfill\hfill\hfill\hfill\hfill\hfill\hfill\hfill\hfill\hfill\hfill\hfill\hfill\hfill\hfill\hfill\hfill\hfill\hfill\hfill\hfill\hfill\hfill\hfill\hfill\hfill\hfill\hfill\hfill\hfill}{\ }
\contentsline {subsection}{\textbf{Korolev V.\,Yu.} see Batrakova D.\,A.\hfill\hfill\hfill\hfill\hfill\hfill\hfill\hfill\hfill\hfill\hfill\hfill\hfill\hfill\hfill\hfill\hfill\hfill\hfill\hfill\hfill\hfill\hfill\hfill\hfill\hfill\hfill\hfill\hfill\hfill\hfill\hfill\hfill\hfill\hfill}{\ }
\contentsline {subsection}{\textbf{Kozerenko E.\,B.}\ \ Linguistic Simulation for Machine Translation and Knowledge Management Systems}{\qquad 1 \qquad 54} 
\contentsline {subsection}{\textbf{Kozmidiady V.\,A.} see Zakharov V.\,N.\hfill\hfill\hfill\hfill\hfill\hfill\hfill\hfill\hfill\hfill\hfill\hfill\hfill\hfill\hfill\hfill\hfill\hfill\hfill\hfill\hfill\hfill\hfill\hfill\hfill\hfill\hfill\hfill\hfill\hfill\hfill\hfill\hfill\hfill\hfill}{\ }
\contentsline {subsection}{\textbf{Kudryavtsev A.\,A., Shorgin S.\,Ya.}\ \ Bayesian Approach to Queueing Systems and Reliability Characteristics}{\qquad 2 \qquad 76} 
\contentsline {subsection}{\textbf{Pechinkin A.\,V., Sokolov I.\,A., Chaplygin V.\,V.}\ \ Multichannel Queuing System with Finite Buffer and Unreliable Servers}{\qquad 1 \qquad 27} 
\contentsline {subsection}{\textbf{Pechinkin A.\,V., Sokolov I.\,A., Chaplygin V.\,V.}\ \ Stationary Characteristics of a Multichannel Queueing System with\nobreakspace {}Simultaneous Refusals of Servers}{\qquad 2 \qquad 39} 
\contentsline {subsection}{\textbf{Shorgin S.\,Ya.} see Batrakova D.\,A.\hfill\hfill\hfill\hfill\hfill\hfill\hfill\hfill\hfill\hfill\hfill\hfill\hfill\hfill\hfill\hfill\hfill\hfill\hfill\hfill\hfill\hfill\hfill\hfill\hfill\hfill\hfill\hfill\hfill\hfill\hfill\hfill\hfill\hfill\hfill}{\ }
\contentsline {subsection}{\textbf{Shorgin S.\,Ya.} see Kudryavtsev A.\,A.\hfill\hfill\hfill\hfill\hfill\hfill\hfill\hfill\hfill\hfill\hfill\hfill\hfill\hfill\hfill\hfill\hfill\hfill\hfill\hfill\hfill\hfill\hfill\hfill\hfill\hfill\hfill\hfill\hfill\hfill\hfill\hfill\hfill\hfill\hfill}{\ }
\contentsline {subsection}{\textbf{Sinitsyn I.\,N.}\ \ Correlational Methods for Analytical Informational Models of the Earth Pole Fluctuations Design Based on a priori Data}{\qquad 2 \qquad \hphantom{9}2}
\contentsline {subsection}{\textbf{Sinitsyn I.\,N.}\ \ Development of Pugachev Filtering for Stochastic Systems}{\qquad 1 \qquad \hphantom{9}3}
\contentsline {subsection}{\textbf{Sokolov I.\,A.} see Ilyin V.\,D.\hfill\hfill\hfill\hfill\hfill\hfill\hfill\hfill\hfill\hfill\hfill\hfill\hfill\hfill\hfill\hfill\hfill\hfill\hfill\hfill\hfill\hfill\hfill\hfill\hfill\hfill\hfill\hfill\hfill\hfill\hfill\hfill\hfill\hfill\hfill}{\ }
\contentsline {subsection}{\textbf{Sokolov I.\,A.} see Pechinkin A.\,V.\hfill\hfill\hfill\hfill\hfill\hfill\hfill\hfill\hfill\hfill\hfill\hfill\hfill\hfill\hfill\hfill\hfill\hfill\hfill\hfill\hfill\hfill\hfill\hfill\hfill\hfill\hfill\hfill\hfill\hfill\hfill\hfill\hfill\hfill\hfill}{\ }
\contentsline {subsection}{\textbf{Sokolov I.\,A.} see Pechinkin A.\,V.\hfill\hfill\hfill\hfill\hfill\hfill\hfill\hfill\hfill\hfill\hfill\hfill\hfill\hfill\hfill\hfill\hfill\hfill\hfill\hfill\hfill\hfill\hfill\hfill\hfill\hfill\hfill\hfill\hfill\hfill\hfill\hfill\hfill\hfill\hfill}{\ }
\contentsline {subsection}{\textbf{Sokolov I.\,A.} see Zakharov V.\,N.\hfill\hfill\hfill\hfill\hfill\hfill\hfill\hfill\hfill\hfill\hfill\hfill\hfill\hfill\hfill\hfill\hfill\hfill\hfill\hfill\hfill\hfill\hfill\hfill\hfill\hfill\hfill\hfill\hfill\hfill\hfill\hfill\hfill\hfill\hfill}{\ }
\contentsline {subsection}{\textbf{Stupnikov S.\,A.} see Zakharov V.\,N.\hfill\hfill\hfill\hfill\hfill\hfill\hfill\hfill\hfill\hfill\hfill\hfill\hfill\hfill\hfill\hfill\hfill\hfill\hfill\hfill\hfill\hfill\hfill\hfill\hfill\hfill\hfill\hfill\hfill\hfill\hfill\hfill\hfill\hfill\hfill}{\ }
\contentsline {subsection}{\textbf{Zakharov V.\,N., Kalinichenko L.\,A., Sokolov I.\,A., Stupnikov S.\,A.}\ \ Development of Canonical Information Models for Integrated Information Systems}{\qquad 2 \qquad 15} 
\contentsline {subsection}{\textbf{Zakharov V.\,N., Kozmidiady V.\,A.}\ \ Means Providing Applications Fault Tolerance}{\qquad 1 \qquad 14} 
\def\leftfootline{\small{\textbf{\thepage}
\hfill ИНФОРМАТИКА И ЕЁ ПРИМЕНЕНИЯ\ \ \ том~1\ \ \ выпуск~2\ \ \ 2007}
}%
 \def\rightfootline{\small{ИНФОРМАТИКА И ЕЁ ПРИМЕНЕНИЯ\ \ \ том~1\ \ \ выпуск~2\ \ \ 2007
 \hfill \textbf{\thepage}}}
 \label{end\stat}


%\tableofcontents


\end{document}

\newcommand{\Ack}{\subsection*{\protect\large\bf Acknowledgments}}