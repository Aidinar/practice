%\newcommand{\il}[2]{\int\limits_{#1}^{#2}}


\newcommand{\fl}{f_\lambda}
\newcommand{\fm}{f_\mu}

\def\stat{kudr}

\def\tit{ЗАВИСИМЫЕ ОТ КОЭФФИЦИЕНТА БАЛАНСА 
ХАРАКТЕРИСТИКИ В~БАЙЕСОВСКИХ МОДЕЛЯХ 
С~КОМПАКТНЫМ НОСИТЕЛЕМ  АПРИОРНЫХ РАСПРЕДЕЛЕНИЙ$^*$}

\def\titkol{Зависимые от коэффициента баланса 
характеристики в~байесовских моделях 
с %компактным 
носителем  априорных распределений}

\def\aut{А.\,А.~Кудрявцев$^1$}

\def\autkol{А.\,А.~Кудрявцев}

\titel{\tit}{\aut}{\autkol}{\titkol}

\index{Кудрявцев А.\,А.}
\index{Kudryavtsev A.\,A.}


{\renewcommand{\thefootnote}{\fnsymbol{footnote}} \footnotetext[1]
{Исследование выполнено при поддержке Российского научного фонда (проект 14-11-00397).}}


\renewcommand{\thefootnote}{\arabic{footnote}}
\footnotetext[1]{Московский государственный университет им.~М.\,В.~Ломоносова, 
факультет вычислительной математики и~кибернетики; Институт 
проблем информатики Федерального исследовательского центра 
<<Информатика и~управ\-ле\-ние>> Российской академии наук, \mbox{nubigena@mail.ru}}



\Abst{Приведены вероятностные распределения некоторых характеристик, 
зависящих от коэффициента баланса, т.\,е.\ от отношения параметра, <<препятствующего>> 
функционированию сис\-те\-мы, к~параметру, <<способствующему>> функционированию. 
В~теории массового обслуживания для модели $M|M|1$ такие характеристики интерпретируются 
как среднее число заявок в~сис\-те\-ме, коэффициент готовности, вероятность <<непотери>> 
вызова, а~для дискретных экспоненциальных моделей роста надежности~--- 
как предельная надежность системы. В~рамках байесовского подхода предполагается, 
что исходные параметры случайны и~имеют априорные распределения с~компактным носителем.}

\KW{байесовский подход; системы массового обслуживания; надежность; смешанные
распределения; распределения с~компактным носителем}

\DOI{10.14357/19922264160310} 


\vskip 12pt plus 9pt minus 6pt

\thispagestyle{headings}

\begin{multicols}{2}

\label{st\stat}

\section{Введение}

В работе~\cite{K2016} был предложен подход, позволя\-ющий частично унифицировать 
терминологию, используемую при постановке задач, относящихся к~байесовским 
моделям теории массового обслуживания и~надежности. Суть этого подхода 
заключается в~определении параметров, <<препятствующих>> и~<<способствующих>> 
функционированию некоторой сложной системы, и~рассмотрению их частного 
(называемого коэффициентом баланса), от величины которого зависит эффективность 
работы системы.
В классических задачах обслуживания и~надежности коэффициент баланса 
входит в~большинство основных формул, характеризующих функционирование системы.

Обозначим один из параметров, <<препятству\-ющих>>/<<способствующих>> функционированию, 
через~$\lambda$, а второй~--- через~$\mu$. Через $\rho\hm=\lambda/\mu$ обозначим 
коэффициент баланса. В~рамках байесовского подхода предполагается, что~$\lambda$ 
и~$\mu$~--- случайные величины с~некоторыми известными априорными распределениями. 
Ниже будут рассмотрены вероятностные характеристики случайных величин 
$\pi\hm=1/(1\hm+\rho)$ и~$N\hm=\rho/(1\hm-\rho)$.

В задачах массового обслуживания для моделей $M|M|1$ величину~$\rho$, равную 
отношению па\-ра\-мет\-ра входящего потока к~па\-ра\-мет\-ру обслуживания, принято называть 
коэффициентом загрузки, величину~$\pi$~--- коэффициентом готовности и~ве\-ро\-ят\-ностью 
<<непотери>> вызова, величину~$N$~--- средним числом заявок в~системе. 
В~дискретных экспоненциальных моделях роста надежности~$\rho$ имеет смысл отношения 
параметра <<эффективности>> к~па\-ра\-мет\-ру <<дефективности>>, а~под величиной~$\pi$ 
понимают предельную надежность системы.

Далее приводятся результаты для распределения величин~$\pi$ и~$N$ в~случае, 
когда носителями распределений~$\lambda$ и~$\mu$ являются отрезки на положительной 
полупрямой. При применении изложенных ниже результатов к~надежностным постановкам 
необходимо ограничивать правые концы носителей распределений единицей~\cite{KuSh2015}.

\section{Основные результаты}

Пусть $\lambda$ и~$\mu$~--- независимые абсолютно непрерывные случайные 
величины, причем ${\sf P}(\lambda\hm\in[a_\lambda,b_\lambda])\hm=1$, $0\hm<a_\lambda\hm<b_\lambda$, 
и~не существует мно\-же\-ст\-ва $S\subset[a_\lambda,b_\lambda]$ положительной меры Лебега такого, что 
${\sf P}(\lambda\in S)\hm=0$, а~для случайной величины $\mu$ выполнены 
аналогичные требования с~параметрами~$a_\mu$ и~$b_\mu$. Во всех последующих выкладках 
будем предполагать, что $x\hm>0$.

В статье~\cite{K2016} было сформулировано следующее утверждение.

\smallskip

\noindent
\textbf{Теорема~1.} 
\textit{Пусть независимые абсолютно непрерывные случайные величины~$\lambda$ и~$\mu$ 
имеют соответственно носители распределений $[a_\lambda,b_\lambda]$ и~$[a_\mu,b_\mu]$, 
$0\hm<a_\lambda\hm<b_\lambda$, $0\hm<a_\mu\hm<b_\mu$, и~плотности $\fl(x)$ и~$\fm(x)$. Тогда 
случайная величина $\rho\hm=\lambda/\mu$ имеет функцию распределения}
\begin{multline}
F_\rho(x)=
\Ik\left(\fr{a_\lambda}{b_\mu}<x\le 
\min\left\{\fr{a_\lambda}{a_\mu},\fr{b_\lambda}{b_\mu}\right\}\right)
\times{}\\
{}\times\il{a_\lambda/x}{b_\mu}
\il{a_\lambda}{xy}\fl(u)\fm(y)\, dudy+{}\\
{}+\Ik\left(\frac{a_\lambda}{a_\mu}<x\le \fr{b_\lambda}{b_\mu}\right)\il{a_\mu}{b_\mu}\il{a_\lambda}
{xy}\fl(u)\fm(y)\, dudy+{}\\
{}+\Ik\left(\fr{b_\lambda}{b_\mu}<x\le \fr{a_\lambda}{a_\mu}\right)\times{}\\
{}\times\left[
 \il{a_\lambda/x}{b_\lambda/x}\il{a_\lambda}{xy}\fl(u)\fm(y)\, dudy +
 \il{b_\lambda/x}{b_\mu}\fm(y)\, dy\right]+{}\\
{}+\Ik\left(\max\left\{\fr{a_\lambda}{a_\mu},\fr{b_\lambda}{b_\mu}\right\}<x\le
\fr{b_\lambda}{a_\mu}\right)\times{}\\
{}\times
\left[ \il{a_\mu}{b_\lambda/x}\il{a_\lambda}{xy}\fl(u)\fm(y)\, dudy+
\il{b_\lambda/x}{b_\mu}\fm(y)\, dy\right]+{}\\
{}+\Ik\left(x>\fr{b_\lambda}{a_\mu}\right)\,.
\label{F_rho}
\end{multline}


Основываясь на теореме~1 и~соотношении
$$
F_\pi(x)=1-F_\rho\left(\fr{1-x}{x}\right)\,,
$$
несложно убедиться в~справедливости следующего утверждения.

\smallskip

\noindent
\textbf{Следствие~1.} 
Пусть коэффициент баланса $\rho\hm=\lambda/\mu$ имеет функцию распределения~(\ref{F_rho}). 
Тогда случайная величина $\pi\hm=1/(1\hm+\rho)$ имеет функцию распределения
\begin{multline*}
F_\pi(x)=1-\Ik\left(x<\fr{a_\mu}{b_\lambda+a_\mu}\right)-{}\\
{}-\Ik\left(\fr{a_\mu}{b_\lambda+a_\mu}\le x<\min\left\{\fr{a_\mu}{a_\lambda+a_\mu},
\fr{b_\mu}{b_\lambda+b_\mu}\right\}\right)\times{}\\
{}\times\left[\,
\il{a_\mu}{{b_\lambda x}/({1-x})}
\il{a_\lambda}{({1-x})y/{x}}\hspace*{-5mm}\fl(u)\fm(y)\, dudy+\hspace*{-5mm}
\il{{b_\lambda x}/({1-x})}{b_\mu}\hspace*{-5mm}\fm(y)\, dy\right]-{}\\
{}-\Ik\left(\fr{a_\mu}{a_\lambda+a_\mu}\le x<\fr{b_\mu}{b_\lambda+b_\mu}\right)\times{}
\end{multline*}

\noindent
\begin{multline*}
{}\times\left[ \,
\il{{a_\lambda x}/({1-x})}{{b_\lambda x}/({1-x})}
\il{a_\lambda}{({1-x})y/{x}}\hspace*{-13pt}\fl(u)\fm(y)\, dudy +\hspace*{-14pt}
\il{{b_\lambda x}/({1-x})}{b_\mu}\hspace*{-16pt}\fm(y)\, dy\right]-{}\\
{}-\Ik\left(\fr{b_\mu}{b_\lambda+b_\mu}\le x<\fr{a_\mu}{a_\lambda+a_\mu}\right)\times{}\\
{}\times
\il{a_\mu}{b_\mu}
\il{a_\lambda}{({1-x})y/{x}}\fl(u)\fm(y)\, dudy-{}\\
{}-\Ik\left(\max\left\{\fr{a_\mu}{a_\lambda+a_\mu},\frac{b_\mu}{b_\lambda+b_\mu}\right\}\le 
x<\fr{b_\mu}{a_\lambda+b_\mu}\right)\times{}\\
{}\times
\il{{a_\lambda x}/({1-x})}{b_\mu}
\il{a_\lambda}{({1-x})y/{x}}\fl(u)\fm(y)\, dudy\,.
\end{multline*}


\noindent
\textbf{Замечание~1.}\ Для вычисления моментов случайной величины~$\pi$ удобно 
использовать формулу для плотности $f_\pi(x)$, которая получается 
аналогично следствию~1 при помощи соотношения
$$
f_\pi(x)=\fr{1}{x^2}f_\rho\left(\fr{1-x}{x}\right)
$$
из соответствующего утверждения статьи~\cite{K2016}. Для сокращения изложения 
опустим это следствие.

\smallskip

\noindent
\textbf{Замечание~2.} Особый интерес представляет случай, в~котором плотности 
априорных распределений с~ограниченным носителем могут быть представлены в~виде 
полинома
$$
f(x)=\sum\limits_{i=0}^{n}c_{i}\, x^i\cdot\Ik(x\in[a,b])\,.
$$
В работах \cite{KuSoSh, KuPa2016} можно соответственно найти примеры 
вычисления вероятностных и~моментных характеристик~$\pi$ в~случаях, когда 
рассматриваются смеси двух равномерных ($n\hm=0$) и~параболических распределений ($n\hm=2$).

\smallskip

Для вычисления функции распределения случайной величины $N\hm=\rho/(1\hm-\rho)$ 
достаточно воспользоваться теоремой~1 и~формулой
$$
F_N(x)=F_\rho\left(\fr{x}{1+x}\right)\,.
$$
При этом, в~отличие от следствия~1, в~общем случае нельзя сформулировать 
соответствующее утверж\-де\-ние в~терминах индикаторов неравенств относительно 
аргумента функции распределения~$x$.
Это связано с~тем, что, как показано в~\cite{KuSh10}, функция распределения $F_N(x)$ 
может быть несобственной, т.\,е.\ $\lim\limits_{x\to+\infty}F_N(x)\hm<1$. При рассмотрении 
характеристик распределения случайной величины~$N$, таким образом, существенную 
роль играет величина <<дефекта>>
$$
\delta=1-F_N(+\infty)\equiv{\sf P}(\rho\ge1)\equiv 1-F_\rho(1)\,,
$$
для которого справедливо следующее утверждение.

\smallskip

\noindent
\textbf{Теорема~2.} 
\textit{Пусть независимые абсолютно непрерывные случайные величины~$\lambda$ и~$\mu$ 
имеют соответственно носители распределений $[a_\lambda,b_\lambda]$ и~$[a_\mu,b_\mu]$, 
$0\hm<a_\lambda\hm<b_\lambda$, $0\hm<a_\mu\hm<b_\mu$. Тогда для <<дефекта>>~$\delta$ функции 
распределения $F_N(x)$ справедливо}
$$
\delta=\begin{cases}
1\,,&  \ \mbox{если }b_\mu<a_\lambda; \\
0\,,& \ \mbox{если }b_\lambda<a_\mu; \\
1-{\sf E} F_\lambda(\mu)& \ \mbox{в противном случае}.
\end{cases}
$$


\noindent
Д\,о\,к\,а\,з\,а\,т\,е\,л\,ь\,с\,т\,в\,о\,.\ \ 
Для нахождения величины <<дефекта>>~$\delta$ нужно для всевозможных 
комбинаций взаимного расположения чисел~$a_\lambda$, $b_\lambda$, $a_\mu$, $b_\mu$ 
вычислить $1\hm-F_\rho(1)$, воспользовавшись теоремой~1. Случаи, 
когда отрезки $[a_\lambda,b_\lambda]$ и~$[a_\mu,b_\mu]$ не пересекаются,~--- тривиальные.

Пусть $a_\lambda<a_\mu\hm<b_\lambda\hm<b_\mu$. Тогда
\begin{multline*}
1-\delta=F_\rho(1)=
\Ik\left(\max\left\{\fr{a_\lambda}{a_\mu},\fr{b_\lambda}{b_\mu}\right\}<1\le
\fr{b_\lambda}{a_\mu}\right)\times{}\\
{}\times
\left[ \il{a_\mu}{b_\lambda}\il{a_\lambda}{y}\fl(u)\fm(y)\, dudy+
\il{b_\lambda}{b_\mu}\fm(y)\, dy\right]={}\\
{}=\il{a_\mu}{b_\lambda}F_\lambda(y)\fm(y)\, dy+\il{b_\lambda}{b_\mu}\fm(y)\, dy={}\\
{}={\sf E} F_\lambda(\mu)-\il{b_\lambda}{b_\mu}F_\lambda(y)\fm(y)\, dy+
\il{b_\lambda}{b_\mu}\fm(y)\, dy=
{\sf E} F_\lambda(\mu).\hspace*{-5.63pt}
\end{multline*}

Остальные случаи взаимного расположения концов отрезков 
рассматриваются аналогично и~приводят к~тому же результату.

\smallskip

\noindent
\textbf{Замечание 3.} При помощи классических методов теории вероятностей 
имеет смысл рассматривать только случай $\delta\hm=0$. В~частности, при помощи 
соотношения
$$
f_N(x)=\fr{1}{(1+x)^2}f_\rho\left(\fr{x}{1+x}\right)
$$
можно найти плотность и~моменты случайной величины~$N$. 
В~остальных случаях изучать свойства распределения~$N$ можно при помощи квантилей 
порядка $(0,1-\delta)$.


{\small\frenchspacing
 {%\baselineskip=10.8pt
 \addcontentsline{toc}{section}{References}
 \begin{thebibliography}{9}
\bibitem{K2016}
\Au{Кудрявцев А.\,А.}
Байесовские модели массового обслуживания и~надежности: 
априорные распределения с~компактным носителем~//
Информатика и~её применения, 2016. Т.~10. Вып.~1. С.~67--71.

\bibitem{KuSh2015}
\Au{Кудрявцев А.\,А., Шоргин С.\,Я.}
Байесовские модели\linebreak в~тео\-рии массового обслуживания и~надежности.~--- 
М.: ФИЦ ИУ РАН, 2015. 76~с.

\bibitem{KuSoSh}
\Au{Кудрявцев А.\,А., Соколов И.\,А., Шоргин~С.\,Я.} Байесовская
рекуррентная модель роста надежности: равномерное распределение
параметров~// Информатика и~её применения, 2013. Т.~7. Вып.~2.
С.~55--59.

\bibitem{KuPa2016}
\Au{Кудрявцев А.\,А., Палионная С.\,И.} Байесовская
рекуррентная модель роста надежности: параболическое распределение
параметров~// Информатика и~её применения, 2016. Т.~10. Вып.~2.
С.~80--83.

\bibitem{KuSh10}
\Au{Кудрявцев А.\,А., Шоргин С.\,Я.} Байесовские модели
массового обслуживания и~надежности: характеристики среднего числа
заявок в~системе $M|M|1|\infty$~// Информатика и~её применения, 2010. 
Т.~4. Вып.~3. С.~16--21.
\end{thebibliography}

 }
 }

\end{multicols}

\vspace*{-6pt}

\hfill{\small\textit{Поступила в~редакцию 29.06.16}}

\vspace*{8pt}

%\newpage

%\vspace*{-24pt}

\hrule

\vspace*{2pt}

\hrule

%\vspace*{8pt}



\def\tit{CHARACTERISTICS DEPENDENT ON~THE~BALANCE COEFFICIENT IN~BAYESIAN MODELS 
WITH~COMPACT~SUPPORT~OF~\textit{A~PRIORI} DISTRIBUTIONS}

\def\titkol{Characteristics dependent on~the~balance coefficient in~bayesian models 
with~compact support of~\textit{a~priori} distributions}

\def\aut{A.\,A.~Kudryavtsev$^{1,2}$}

\def\autkol{A.\,A.~Kudryavtsev}

\titel{\tit}{\aut}{\autkol}{\titkol}

\vspace*{-9pt}

\noindent
$^1$Department of Mathematical Statistics, Faculty of 
Computational Mathematics and Cybernetics,\linebreak 
$\hphantom{^1}$M.\,V.~Lomonosov Moscow State University, 
1-52 Leninskiye Gory, GSP-1, Moscow 119991, Russian\linebreak 
$\hphantom{^1}$Federation

\noindent
$^2$Institute of Informatics Problems, Federal Research Center ``Computer Science 
and Control" of the Russian\linebreak 
$\hphantom{^1}$Academy of Sciences, 44-2 Vavilov Str., 
Moscow 119333, Russian Federation


\def\leftfootline{\small{\textbf{\thepage}
\hfill INFORMATIKA I EE PRIMENENIYA~--- INFORMATICS AND
APPLICATIONS\ \ \ 2016\ \ \ volume~10\ \ \ issue\ 3}
}%
 \def\rightfootline{\small{INFORMATIKA I EE PRIMENENIYA~---
INFORMATICS AND APPLICATIONS\ \ \ 2016\ \ \ volume~10\ \ \ issue\ 3
\hfill \textbf{\thepage}}}

\vspace*{3pt}



\Abste{Distributions of some characteristics dependent 
on the balance coefficient, which is defined as a~ratio
of two parameters that 
are interpreted as the parameter ``obstructing'' the functioning of the 
system and the\linebreak\vspace*{-12pt}}

\Abstend{parameter ``conducing'' the functioning of the system, are presented. 
In the queuing theory for $M|M|1$ models, such characteristics are interpreted 
as an average amount of claims in the system, the readiness coefficient, 
the probability that the claim will not be lost and as the marginal system's 
reliability for the discrete exponential reliability model. In the framework 
of the Bayesian approach, it is supposed that initial parameters are 
random and have \textit{a~priori} distributions with compact support.}

\KWE{Bayesian approach; mass service theory; reliability theory; mixed distributions; 
distributions with compact support}

\DOI{10.14357/19922264160310} 

%\vspace*{-9pt}

\Ack
\noindent
This work was financially supported by the Russian Science Foundation 
(grant No.\,14-11-00397).


%\vspace*{3pt}

  \begin{multicols}{2}

\renewcommand{\bibname}{\protect\rmfamily References}
%\renewcommand{\bibname}{\large\protect\rm References}

{\small\frenchspacing
 {%\baselineskip=10.8pt
 \addcontentsline{toc}{section}{References}
 \begin{thebibliography}{9}



\bibitem{1-kudr}
\Aue{Kudryavtsev, A.\,A.} 2016. Bayesovskie modeli massovogo 
obsluzhivaniya i~nadezhnosti: Apriornye raspredeleniya s~kompaktnym nositelem 
[Bayesian queueing and reliability models: \textit{A~priori} distributions with compact support]. 
\textit{Informatika i~ee Primeneniya~--- Inform. Appl.} 10(1):67--71.
\bibitem{2-kudr}
\Aue{Kudryavtsev, A.\,A., and S.\,Ya.~Shorgin.} 2015. \textit{Bayesovskie modeli 
v~teorii massovogo obsluzhivaniya i~nadezhnosti} [Bayesian models in mass service 
and reliability theories]. Moscow: FIC IU RAN. 76~p.
\bibitem{3-kudr}
\Aue{Kudryavtsev, A.\,A., I.\,A.~Sokolov, and S.\,Ya.~Shorgin}. 2013. 
Bayesovskaya rekurrentnaya model rosta nadezhnosti: Ravnomernoe raspredelenie 
parametrov [Bayesian recurrent model of reliability growth: 
Homogeneous distribution of parameters]. \textit{Informatika i~ee Primeneniya~---
Inform. Appl.} 7(2):55--59.
\bibitem{4-kudr}
\Aue{Kudryavtsev, A.\,A., and S.\,I.~Palionnaia}. 2016. Bayesovskaya rekurrentnaya 
model rosta nadezhnosti: Parabolicheskoe raspredelenie parametrov 
[Bayesian recurrent model of reliability growth: Parabolic distribution of parameters].
\textit{Informatika i~ee Primeneniya~---
Inform. Appl.} 10(2):80--83.
\bibitem{5-kudr}
\Aue{Kudryavtsev, A.\,A., and S.\,Ya.~Shorgin}. 2010. Bayesovskie 
modeli massovogo obsluzhivaniya i~nadezhnosti: Kharakteristiki srednego chisla zayavok 
v~sisteme  $M|M|1|\infty$ [Bayesian queueing and reliability models: 
Average number of claims characteristics in $M|M|1|\infty$ system]. 
\textit{Informatika i~ee Primeneniya~--- Inform. Appl.} 4(3):16--21.
 \end{thebibliography}

 }
 }

\end{multicols}

\vspace*{-3pt}

\hfill{\small\textit{Received June 29, 2016}}


\Contrl

\noindent
\textbf{Kudryavtsev Alexey A.} (b.\ 1978)~--- Candidate of Sciences (PhD) 
in physics and mathematics, associate professor, Department of Mathematical 
Statistics, Faculty of Computational Mathematics and Cybernetics, M.\,V.~Lomonosov 
Moscow State University, 1-52 Leninskiye Gory, GSP-1, Moscow 119991, 
Russian Federation; Institute of Informatics Problems, Federal Research Center 
``Computer Science and Control'' of the Russian Academy of Sciences, 
44-2~Vavilov Str., Moscow 119333, Russian Federation; \mbox{nubigena@mail.ru}

\label{end\stat}


\renewcommand{\bibname}{\protect\rm Литература} 



 