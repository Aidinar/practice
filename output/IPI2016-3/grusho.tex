\def\stat{grusho}

\def\tit{ИНТЕГРАЦИЯ СТАТИСТИЧЕСКИХ И~ДЕТЕРМИНИСТСКИХ МЕТОДОВ АНАЛИЗА 
ИНФОРМАЦИОННОЙ БЕЗОПАСНОСТИ$^*$}

\def\titkol{Интеграция статистических и~детерминистских методов анализа 
информационной безопасности}

\def\aut{А.\,А.~Грушо$^1$, Н.\,А.~Грушо$^2$, М.\,И.~Забежайло$^3$, 
Е.\,Е.~Тимонина$^4$}

\def\autkol{А.\,А.~Грушо, Н.\,А.~Грушо, М.\,И.~Забежайло, 
Е.\,Е.~Тимонина}

\titel{\tit}{\aut}{\autkol}{\titkol}

\index{Грушо А.\,А.}
\index{Грушо Н.\,А.}
\index{Забежайло М.\,И.} 
\index{Тимонина Е.\,Е.}
\index{Grusho A.\,A.}
\index{Grusho N.\,A.}
\index{Zabezhailo M.\,I.}
\index{Timonina E.\,E.}


{\renewcommand{\thefootnote}{\fnsymbol{footnote}} \footnotetext[1]
{Работа поддержана РФФИ (проекты 15-29-07981 и~15-07-02053).}}


\renewcommand{\thefootnote}{\arabic{footnote}}
\footnotetext[1]{Институт проблем информатики Федерального исследовательского
центра <<Информатика и~управление>> Российской академии наук, \mbox{grusho@yandex.ru}}
\footnotetext[2]{Институт проблем информатики Федерального исследовательского
центра <<Информатика и~управление>> Российской академии наук, \mbox{info@itake.ru}}
\footnotetext[3]{Институт проблем информатики Федерального исследовательского
центра <<Информатика и~управление>> Российской академии наук, \mbox{m.zabezhailo@yandex.ru}}
\footnotetext[4]{Институт проблем информатики Федерального исследовательского
центра <<Информатика и~управление>> Российской академии наук, \mbox{eltimon@yandex.ru}}

\vspace*{-3pt}


  \Abst{Статья посвящена разработке методов автоматического анализа и~управления 
механизмами информационной безопасности в~облачных вычислительных средах. 
Рассмотренные подходы основаны на синтезе ве\-ро\-ят\-но\-ст\-но-ста\-ти\-сти\-че\-ских 
и~детерминистских методов анализа ситуаций, встречающихся в~задачах информационной 
безопасности. 
  Статистический анализ позволяет сформировать множество объектов для 
детерминистского (логического) анализа. Поскольку детерминистские методы требуют 
больших объемов вычислений, предварительная статистическая обработка позволяет 
выделить для детерминистского (логического) анализа доступные для вычисления объемы 
данных. 
  В~работе детерминистские методы представлены аналогами поиска  
при\-чин\-но-след\-ст\-вен\-ных связей. Применение эвристик и~правдоподобных 
рассуждений может породить недостоверные выводы, которые связаны со случайным 
характером исходных данных, поэтому проводится анализ возможностей 
случайного порождения заключений детерминистского характера. Предложенные методы 
анализа ориентированы на двухуровневую архитектуру построения информационной 
безопасности в~облачных вычислительных средах. В~этой архитектуре автоматический 
интеллектуальный анализ данных порождает на верхнем уровне быструю реакцию для 
разрешения конфликтов в~вычислительных процессах или выявление функционирования 
вредоносного кода.}
  
\KW{облачные вычислительные среды; информационная безопасность;  
ве\-ро\-ят\-но\-ст\-но-ста\-ти\-сти\-че\-ские и~детерминистские (логические) методы анализа; 
эвристические алгоритмы; взаимное влияние данных}

\DOI{10.14357/19922264160301} 


\vskip 10pt plus 9pt minus 6pt

\thispagestyle{headings}

\begin{multicols}{2}

\label{st\stat}
  

\section{Введение}

  Изложим идеологию взаимного дополнения статистических и~логических 
методов. Статистическая обработка данных основана на использовании 
технологии усреднения (сглаживания). Тогда значимые отклонения от 
средних сглаженных данных порождают исходные данные для получения 
новой информации~[1]. 

Эта идея основана на том, что значимые отклонения могут 
иметь закономерную структуру, которая отсутствует в~основной массе данных, 
использованных для сглаживания. В~качестве примера таких задач можно 
указать способы определения и~выявления признаков атак с~по\-мощью анализа 
логов~[2, 3]. 
  
  В работе~[4] предлагается искать локальные соотношения, которым 
удовлетворяют случайные последовательности. 

Таким образом статистические 
методы позволяют выделить множество фрагментов данных, отличающихся от 
средних сглаженных данных. Таких фрагментов немного, поэтому далее 
использовать статистические методы анализа неэффективно. 
  
  Вместе с~тем эти фрагменты можно рассматривать как множество объектов 
для применения логических методов анализа~[5, 6]. Эти методы позволяют 
строить эвристические закономерности, анализировать  
при\-чин\-но-след\-ст\-вен\-ные связи, автоматически порождать гипотезы о 
структурных зависимостях в~данных~\cite{6-gru}. 
  
  Логические методы, использующие эвристику и~правдоподобные 
рассуждения, при всей строгости логического анализа не гарантируют 
достоверности выводов. Поэтому целесообразно сравнивать полученные 
выводы с~результатами оценок, полученных вероятностными методами, 
в~предположении стохастического характера результатов логической 
обработки данных. Это особенно важно, когда методы интеллектуального 
анализа данных применяются многократно на больших объемах данных. 
В~таких многократных случаях требуется автоматическая интеллектуальная 
обработка данных, так как объем необходимых вычислений существенно 
превышает человеческие возможности. 
  
  В продолжение идей ДСМ (Джон Стюарт Милль) ме\-то\-да~\cite{6-gru} в~работе приведена 
<<смягченная>> модель анализа при\-чин\-но-след\-ст\-вен\-ных связей в~данных 
и~простейшая стохастическая модель ее валидации. 
  
\section{Влияние характеристик процессов на~их~целевые свойства} 

  Рассмотрим $r+1$ процессов $\xi_0,\ldots, \xi_r$, $r\hm\geq 0$. Пусть 
$\xi_0$~--- целевой процесс, его состояния содержат интересующее нас 
состояние (свойство)~$p$. Задача состоит в~выявлении влияния остальных 
процес\-сов $\xi_1,\ldots, \xi_r\ldots$ на появление свойства~$p$ в~процессе~$\xi_0$. 
В~данной работе понятие влияния отличается от известных функций 
влияния~\cite{7-gru}. Пусть время дискретно, что позволяет точно 
синхронизировать все процессы. Пусть $t_1, t_2, \ldots , t_n, \ldots$~--- моменты 
появления свойства~$p$ в~процессе~$\xi_0$. Для простоты исследуем влияние 
процесса~$\xi_1$ на появление свойства~$p$. Для этого рассмотрим фрагменты 
процесса~$\xi_1$ в~промежутках времени $t_1, t_2,\ldots , t_n, \ldots$ Будем 
считать, что процесс~$\xi_1$ принимает значения в~алфавите  $A\hm= \{a_1, 
\ldots ,a_m\}$. 
  
  В усложненной схеме поиска влияния процесса~$\xi_1$ на появление 
свойства~$p$ следует рассматривать подпоследовательности длин, б$\acute{\mbox{о}}$льших 
либо равных~$l$, для выделенных фрагментов. Параметр~$l$ характеризует 
степень положительного влияния~$\xi_1$ на появление свойства~$p$. Наличие 
достаточно большого количества фрагментов, содержащих одинаковые 
подпоследовательности, можно рассматривать как довод в~пользу влияния 
процесса~$\xi_1$ на появление свойства~$p$ процесса~$\xi_0$. 
  
  В более простой схеме следует рассматривать множества элементов 
в~каждом фрагменте и~выбирать такие фрагменты для определения влияния, 
в~которых число элементов в~пересечении множеств этих фрагментов больше 
либо рав\-но~$l$. 
  
  Выбор между упрощенной или усложненной схемами связан со сложностью 
вычислений и/или со схемами расчета вероятностей ошибочного (случайного) 
решения о наличии влияния~$\xi_1$ на появление свойства~$p$. 
  
  Рассмотрим упрощенную схему и~обозначим через $T_0\hm<T_1$ два 
пороговых значения времени. Пусть $\Phi^+(i)$~--- множество различных 
элементов, расположенных на расстоянии, не превосходящем порог~$T_0$, 
от~$p$ в~$i$-м фрагменте. Поскольку на появление свойства~$p$ могут влиять 
различные множества элементов из~$\xi_1$, то выбираем 
подпоследовательность из~$k$ фрагментов $i_1, i_2, \ldots , i_k$, $k\hm\geq 2$, 
таких что
  $
  \left\vert \mathop{\bigcap}\limits_{i\in \{i_1, \ldots ,i_k\}}
  \Phi^+(i)\right\vert \geq l.$ 
  
  \smallskip
  
  \noindent
  \textbf{Замечание~1.} В~случае исследования влияния~$r$~процессов на 
появление свойства~$p$ надо брать пересечение множеств~$\Phi^+_j(i)$, 
$j\hm= 1, \ldots ,r$. 
  
  \smallskip
  
  В данной схеме необходимо оценить возможность ошибочного решения 
о~влиянии процесса~$\xi_1$ на появление свойства~$p$. Рассмотрим матрицу 
$K^+ \hm= \| k^+_{ij}\|$, столбцы которой пронумерованы чис\-ла\-ми от~1 
до~$m$, а~строки пронумерованы номерами фрагментов процесса~$\xi_1$, 
определяемыми моментами времени $t_1, t_2, \ldots , t_n,\ldots$, 
элементы~$k_{ij}^+$ матрицы~$K^+$ равны~1 или~0, при этом 
$k^+_{ij}\hm=1$, если элемент $a_j\hm\in \Phi^+(i)$. Матрица~$K_n^+$ 
получается из матрицы~$K^+$ ограничением на первые~$n$~строк. 
  
  Число единиц в~каждой строке матрицы~$K^+$ зависит от длин фрагментов 
и вероятностей появления букв алфавита~$A$. В~простейшей модели 
ошибочного влияния будем считать, что элементы матрицы~$K^+$ появляются 
независимо друг от друга с~одинаковой вероятностью~$q$ появления единицы. 
Вероятность появления заданного подмножества мощности~$l$ в~данной 
строке равна~$q^l$. Вероятность~$P_s^+$ того, что данное подмножество 
встретится в~не менее чем~$s$~строках матрицы~$K_n^+$, равна
  $$
  P_s^+ = \sum\limits_{v=s}^n \begin{pmatrix}
  n\\ v
  \end{pmatrix}
  q^{lv} \left( 1-q^l\right)^{n-v}\,.
  $$
  
  Используя теорему Муавра--Лапласа~\cite{8-gru}, получаем оценку~$s$, при 
которой вероятность случайного влияния данного подмножества стремится к~0 
при $n\hm\to\infty$:
  $$
  s\geq q^l n +\ln n \sqrt{nq^l \left( 1-q^l\right)}\,.
  $$
  
  В таблице представлены численные расчеты значений~$s$ для различных 
значений~$q$, $l$ и~$n$.
  

  
  Используя неравенство Маркова~\cite{9-gru}, получим, что вероятность 
случайного появления какого-либо множества, содержащего число элементов, 
большее или равное~$l$, и~влияющего на появление свойства~$p$ 
в~процессе~$\xi_1$, стремится к~нулю. 
  
  Выше рассматривалось положительное влияние на появление свойства~$p$. 
Однако согласно~\cite{6-gru} следует рассматривать также и~негативное 
влияние. Множество~$\Phi^-(i)$ негативно влияет на появление\linebreak\vspace*{-12pt}

\pagebreak

 % \begin{table*}
 {\small
  \begin{center}
  \begin{tabular}{|r|c|c|r|c|c|}
  \multicolumn{6}{p{76mm}}{Оценка~$s$, при которой вероятность случайного влияния 
подмножества мощности~$l$ стремится к~0}\\
  \multicolumn{6}{c}{\ }\\[-6pt]
  \hline
  \multicolumn{3}{|c|}{\tabcolsep=0pt\begin{tabular}{c}Вероятность появления\\ единицы  
$q = 0{,}5$\end{tabular}}&  \multicolumn{3}{c|}{\tabcolsep=0pt\begin{tabular}{c}
Вероятность появления\\ единицы  
$q = 0{,}9$\end{tabular}}\\
\hline
\multicolumn{1}{|c|}{\ \ \ \ $n$\ \ \ \ }&\ \ \ \  $l$\ \ \ \  &$s$&
\multicolumn{1}{c|}{\ \ \ \ $n$\ \ \ \ }&\ \  \ \ $L$\ \ \ \  &$s$\\
\hline
100&\hphantom{9}3&19&100&\hphantom{9}3&\hphantom{9}79\\
400&\hphantom{9}5&20&400&\hphantom{9}5&245\\
10000&10&21&10000&10&3500\hphantom{9}\\
\hline
\end{tabular}
\end{center}}
%\end{table*}

\vspace*{24pt}

\noindent
 свойства~$p$, 
если для каждого состояния $a_j\hm\in \Phi^-(i)$ расстояние от~$a_j$ до~$p$ 
больше порога~$T_1$. Тогда негативное влияние на появление свойства~$p$ 
существует, если существуют фрагменты $i_1, i_2,\ldots , i_k$, $k\hm\geq 2$, 
такие что $\left\vert \mathop{\bigcap}\limits_{i\in \{i_1, \ldots, i_k\}} \Phi^-(i)
\right\vert  \hm\geq l$. Параметр~$l$ характеризует степень негативного влияния~$\xi_1$ 
на появление свойства~$p$. 
  
  Для ускорения поиска негативного влияния рассмотрим расстояние $d(p, 
p^\prime)$ между свойствами~$p$ и~$p^\prime$. Выберем те 
свойства~$p^\prime$, для которых $d(p, p^\prime)\hm>T_1$. Для 
таких~$p^\prime$ найдем положительное влияние $\Phi^+(p^\prime)$. Тогда 
расстояние $d(p,\Phi^+(p^\prime))\hm\geq T_1$, так как 
$\Phi^+(p^\prime)$ предшествует появлению свойства~$p^\prime$. 
  
  Влияние $\Phi^{(0)}(i)$~--- это нулевое влияние, если существуют 
фрагменты $i_1,i_2,\ldots, i_k$, $k\hm\geq2$, такие что  $d 
\left( \mathop{\bigcap}\limits_{i\in\{i_1,\ldots,i_k\}} \Phi^{(0)}(i),p\right)\hm< T_0$, 
и~существуют фрагменты $j_1, j_2, \ldots ,j_v$, $v\hm\geq 2$, такие что $d 
\left( \mathop{\bigcap}\limits_{i\in\{ j_1,\ldots, j_v\}} \Phi^{(0)}(i),p\right)\hm\geq 
T_1$. Возможность случайного появления негативного влияния~$\Phi^-(i)$ или 
нулевого влияния~$\Phi^{(0)}(i)$ оценивается аналогично тому, как это было 
сделано для положительного влияния~$\Phi^+(i)$. 
  
\section{Графы влияния}

  Рассмотренную в~предыдущем разделе схему влияния можно обобщить 
с~помощью графов влияния. Вершинами графа влияния являются свойства 
объектов влияния (события, участие в~по\-рож\-де\-нии информационного потока, 
вызов функции компьютерной системы, элемент решения некоторой задачи, 
фрагмент вычислительного процесса и~т.\,д.), а~дуги графа отражают влияние 
одних свойств на другие. Каждая вершина состоит из двух частей <<$+$>> 
и~<<$-$>>. Дуга графа из вершины~$i$ в~часть <<$+$>> вершины~$j$ означает 
положительное влияние~$i$ на~$j$, т.\,е.\ участие в~порождении свойства~$j$. 
Дуга графа из вершины~$i$ в~часть <<$-$>> вершины~$j$ означает 
отрицательное влияние~$i$ на~$j$, т.\,е.\ препятствует появлению свойства~$j$. 
Если из одной вершины~$i$ идут две дуги в~<<$+$>> и~<<$-$>> вершины~$j$, 
то этот случай обозначим~<<0>>, что значит противоречие (неправильное 
понимание влияния). Если дуги из~$i$ в~$j$ нет, то этот случай будем 
обозначать через~<<$\tau$>>, что означает отсутствие влияния или отсутствие 
знания о влиянии. 
  
  Дуги в~графах влияния можно маркировать степенью влияния или силой 
влияния. Степень влияния ранжируется между максимальным влиянием, что 
является причиной свойства или отсутствия свойства, и~минимальным 
влиянием, что означает невлияние на появление данного свойства. 
  
  В связи с~графами влияния возникают следу\-ющие задачи.
  \begin{enumerate}[1.]
\item \textbf{Порождение графов влияния.} В~разд.~2 рас\-смот\-рен случай выявления 
влияний <<$+$>>, <<$-$>>, <<0>> и~структуры этих влияний в~виде свойства 
процесса~$\xi_1$.\\[-5pt] 
\item \textbf{Преобразование графов влияния.} Ясно, что транзитивное замыкание 
двух последовательных дуг из~$x$ в~$y$ и~из~$y$ в~$z$ означает 
возможность влияния~$x$ на~$z$, но более слабое, чем влияние~$x$ на~$y$ 
и~$y$ на~$z$. Таким образом, введение дополнительной вершины позволяет 
снижать влияние одного свойства на другое. Этот вывод использовался 
в~работе~\cite{10-gru}. Наоборот, возможность агрегирования всех вершин, 
имеющих влияние на данную, так, что из агрегированного множества не 
выходит других дуг, кроме как в~данную вершину, означает усиление 
влияния до причины (в смысле Д.\,С.~Милля~\cite{6-gru}).\\[-5pt] 
\item \textbf{Пути использования графов влияния в~задачах обеспечения 
информационной безопасности.}
\end{enumerate}
  
\section{Примеры описания влияний}

  \subsection{Однородная простая цепь Маркова}
  
  Введенные понятия проще всего объяснить на примере простой однородной 
цепи Маркова на множестве состояний  $A\hm= \{a_1, \ldots ,a_m\}$ с~матрицей 
переходных вероятностей~$P$. Если в~матрице переходных вероятностей~$P$ 
все ненулевые вероятности заменить на~1, то получится часть матрицы 
смежности графа влияний на переходы из состояния в~состояние, 
описывающего влияние~<<$+$>> для $T_0\hm=1$. 
  
  Пусть из вершины $x$ в~вершину~$y$ ведет единственная дуга, других дуг 
из вершины~$x$ нет и~нет других дуг в~вершину~$y$. Тогда состояние~$x$ 
является причиной состояния~$y$ (рис.~1). 

  \pagebreak
   

 \noindent
 \begin{center}  %fig1
 \vspace*{1pt}
 \mbox{%
 \epsfxsize=77.822mm
 \epsfbox{gru-1.eps}
 }



\vspace*{3pt}

\noindent
{{\figurename~1}\ \ \small{Состояние $x$ является причиной состояния~$y$}}

\end{center} 

 \vspace*{9pt}
 
 \noindent
 \begin{center}  %fig2
 \vspace*{1pt}
 \mbox{%
 \epsfxsize=71.086mm
 \epsfbox{gru-2.eps}
 }



\vspace*{3pt}

\noindent
{{\figurename~2}\ \ \small{$x$ и~$y$ влияют на~<<$+$>>-свой\-ст\-ва~$z$}}

\end{center} 


 \vspace*{9pt}
 
 \noindent
 \begin{center}  %fig3
 \vspace*{1pt}
 \mbox{%
 \epsfxsize=74.212mm
 \epsfbox{gru-3.eps}
 }



\vspace*{3pt}

\noindent
{{\figurename~3}\ \ \small{Влияние <<$-$>>-свой\-ст\-ва~$x$ на свойство~$y$}}
\end{center} 

 \vspace*{16pt}



\addtocounter{figure}{3}
   

Если из вершин $x$ и~$y$ ведут дуги в~вершину~$z$, то $x$ влияет 
на~<<$+$>> вершины~$z$ и~$y$ влияет на~<<$+$>> вершины~$z$ (рис.~2).


  Если из вершины $x$ нет дуги в~вершину~$y$, то это можно рассматривать 
как влияние <<$-$>>-свой\-ст\-ва~$x$ на свойство y для $T_1\hm=2$ (рис.~3).
  
  
  В случае заданной однородной цепи Маркова невозможно появление~<<0>> 
и~<<$\tau$>>. 
  
  Для однородной цепи Маркова с~одним ациклическим эргодическим классом 
транзитивное замыкание графа влияний на определенное число шагов 
позволяет обосновать влияние~<<$+$>> каждого свойства~$x$ на каждое 
свойство~$y$. При этом для каждого шага замыкания дуги 
  в~час\-ти~<<$-$>>-свойств строятся отдельно от предыдущих 
и~последующих шагов итераций. 
  
  \subsection{Однородная сложная цепь Маркова}
  
  Для однородной сложной цепи Маркова глубины~2 переходы определяются 
графом, вершинами которого являются пары из множества~$A^2$, причем 
допустимы только переходы вида ($x, y$) в~($y, z$). В~таком графе 
влияние~<<$+$>> на свойство~$z$ будет определяться ориентированными 
цепочками длины~2. 
  
  При переходе на влияние множеств может возникнуть влияние~<<0>>. 

\section{Использование графов влияния в~задачах обеспечения 
информационной безопасности}

\vspace*{-1pt}
  
  \subsection{Идентификация безопасного обращения к~ресурсу}
  
  \vspace*{-1pt}
  
  Одним из признаков функционирования вредоносного кода в~программной 
среде является обращение к~набору библиотек, функции которых позволяют 
реализовать вредоносное воздействие (критические функции). Это не означает, 
что обращение к~этим библиотекам недопустимо для легальных программ. При 
этом даже в~легальном вычислительном процессе обращение к~критическим 
функциям не всегда является необходимым. 

Классификация программ по 
возможностям обращения к~критическим функциям является сложной задачей. 
  %
  Однако можно выделить признаки (влияние), позволяющие 
идентифицировать возможность обращения к~критическим функциям. 
Появление таких признаков в~легальном вычислительном процессе позволяет 
сделать вывод о том, что обращение к~критическим функциям будет легальным. 
При отсутствии указанных признаков обращение к~критическим функциям 
может означать функционирование враждебного кода. Тогда такое обращение 
следует рассматривать как событие без\-опас\-ности и~принимать меры по 
предотвращению враждебного воздействия. 
  
  Исследование влияния на возможность обращения к~критическим функциям 
можно проводить предварительно, а~в~ходе вычислительного процесса 
использовать сигнатуры влияния в~легальных вычислительных процессах. 

\vspace*{-8pt}
  
  \subsection{Разрешение конфликтов в~вычислительных процессах 
с~помощью влияния}

\vspace*{-1pt}
  
  Конфликты в~вычислительных процессах могут возникать по различным 
причинам. Например, выполнение процесса блокируется нарушением  
ка\-ко\-го-ли\-бо требования политики безопасности одного из участников 
облачных вычислений. Возникает задача <<отката>> вычислительного 
процесса и~построения новой траектории процесса, не позволяющей снова 
выйти на конфликт. 
  
  Рассмотрим эвристический алгоритм решения этой задачи с~использованием 
влияний. Пусть~$p$ является причиной конфликта, $F$~--- влияние на 
появление~$p$. Задача состоит в~том, чтобы найти новые, близкие к~$F$ 
траектории, не ведущие к~$p$. Поскольку~$F$ описывает влияние на 
появление~$p$, то возможны случаи появления~$F$ без по\-сле\-ду\-юще-\linebreak\vspace*{-12pt}

\pagebreak

\noindent
го 
появления~$p$. Обозначим через~$X$ множество состояний процесса, которые 
следуют за~$F$, но не совпадают с~$p$. 
  
  Рассмотрим влияние на каждый из элементов множества~$X$. 
И~пусть~$F^\prime$ принадлежит этому множеству, т.\,е.\ $F^\prime$ 
описывает влияние на появление какого-то элемента из~$X$, который не 
совпадает с~$p$. Множество~$F^\prime$ не совпадает с~$F$, но является 
влиянием для свойства, которому может предшествовать~$F$. Таким образом, 
$F^\prime$ может рассматриваться как признак альтернативной траектории 
вычислительного процесса, не приводящей к~конфликту. 
  
  Возможны дополнительные ограничения на альтернативные траектории 
вычислительного процесса, а~именно: начало альтернативной траектории 
должно совпадать с~точкой <<отката>> первоначальной траектории, приведшей 
к~конфликту. В~отброшенном фрагменте первоначальной траектории 
вычислительного процесса могли быть выполнены необходимые функции, не 
входившие в~множество влияния~$F$ на появление конфликта. Эти 
выполненные функции можно оставить в~новом варианте траектории 
вычислительного процесса. Таким образом, влияние позволяет построить 
эвристический алгоритм разрешения конфликтов в~вычислительном процессе, 
отличный от полного перебора. 

\vspace*{-8pt}
  
\section{Заключение}

  В работе рассматриваются пути совместного использования статистических 
и логических методов анализа больших стохастических данных. 
  
  Статистический анализ позволяет сформировать множество объектов для 
детерминистского (логического) анализа. Поскольку детерминистские методы 
требуют больших объемов вычислений, то предварительная статистическая 
обработка позволяет выделить доступные для вычисления объемы данных для 
детерминистского (логического) анализа. 
  
  В работе детерминистские методы представлены простейшим аналитическим 
поиском при\-чин\-но-след\-ст\-вен\-ных связей~\cite{6-gru}. Применение 
эвристических и~правдоподобных рассуждений может породить недостоверные 
выводы, которые связаны со случайным характером исходных данных, 
поэтому требуется проверка возможности получения подобных выводов 
случайно в~простейших вероятностных моделях. В~работе предложенные 
методы анализа ориентированы на построение системы анализа 
информационной безопасности в~облачных вычислительных средах. Они 
позволяют построить механизмы разрешения конфликтов в~вычислительных 
процессах и~идентифицировать легальное использование критических ресурсов. 
  
  Рассмотренные методы интеллектуального анализа данных предполагают 
двухуровневую архитектуру решения проблем информационной без\-опас\-ности, 
при которой сложные задачи решаются\linebreak в~режиме офлайн и~реализуются в~виде 
скоростных алгоритмов принятия решений на основе сигнатур.

\vspace*{-8pt}
  
{\small\frenchspacing
 {%\baselineskip=10.8pt
 \addcontentsline{toc}{section}{References}
 \begin{thebibliography}{99}
\bibitem{1-gru}
\Au{Тьюки Дж.} Анализ результатов наблюдений. Разведочный анализ~/ Пер. 
с~англ.~--- М.: Мир, 1981. 696~с. (\Au{Tukey J.\,W.} Exploratory data analysis.~--- 
Addison-Wesley Pub. Co, 1977. 688~p.)

\bibitem{3-gru}
\Au{Норткат С., Купер~М., Фирноу~М., Фредерик~К.} Анализ типовых 
нарушений безопасности в~сетях~/ Пер. с~англ.~--- М.: Вильямс, 2001. 464~с. 
(\Au{Nortcutt~S., Cooper~M., Fearnow~M., Frederik~K.} Intrusion signatures and 
analysis.~--- New Readers Pub., 2001. 408~p.)

\bibitem{2-gru}
\Au{Грушо А., Забежайло~М., Зацаринный~А.} Контроль и~управление 
информационными потоками в~облачной среде~// Информатика и~её 
применения, 2015. Т.~9. Вып.~4. С.~95--101.
\bibitem{4-gru}
\Au{Бендат Дж., Пирсол~А.} Прикладной анализ случайных данных~/ Пер. 
с~англ.~--- М.: Мир, 1989. 540~с. (\Au{Bendat~J.\,S., Piersol~A.\,G.} Random data: 
Analysis and measurement procedures.~--- 2nd ed.~--- New York, NY, USA: John 
Wiley and Sons, 1986. 566~p.)
\bibitem{5-gru}
\Au{Журавлев Ю.\,И.} Корректные алгебры над множеством некорректных 
(эвристических) алгоритмов~// Кибернетика, 1977. Ч.~I. №\,4. С.~5--17; Ч.~II.  
№\,6. С.~21--27; Кибернетика,  1978. Ч.~III. №\,2. С.~35--43.
\bibitem{6-gru}
\Au{Панкратова Е.\,С., Финн~В.\,К.} Автоматическое по\-рож\-де\-ние гипотез~/
Под общ. ред. В.\,К.~Финна.~--- 
М.: Либроком, 2009. 528~с. 
\bibitem{7-gru}
\Au{Хампель Ф., Рончетти~Э., Рауссеу~П., Штаэль~В.} Робастность 
в~статистике. Подход на основе функций влияния~/ Пер. с~англ.~--- М.: Мир, 
1989. 519~с. (\Au{Hampel~F.\,R., Ronchetti~E.\,M., Rousseeuw~P.\,J., Stahel~W.\,A.} 
Robust statistics. The approach based on influence functions.~--- New York, NY, 
USA: John Willey and Sons, 1986. 509~p.)
\bibitem{8-gru}
\Au{Феллер В.} Введение в~теорию вероятностей и~ее приложения~/ Пер. 
с~англ.~--- М.: Мир, 1967. Т.~1. 499~с. (\Au{Feller~W.} An introduction to 
probability theory and its applications.~--- 2nd ed.~--- New York, NY, USA: John 
Wiley and Sons, 1950. Vol.~1. 520~p.)
\bibitem{9-gru}
\Au{Лоэв М.} Теория вероятностей~/ Пер. с~англ.~-- М.: ИЛ, 
1962. 720~с. (\Au{\mbox{Lo\!{\!\ptb{\`{e}}}ve~M.}} Probability theory.~--- 
Princeton, NJ, USA: D~Van Nostrand, 1955. 701~p.)
\bibitem{10-gru}
\Au{Грушо А.\,А., Грушо Н.\,А., Тимонина~Е.\,Е., Шоргин~С.\,Я.}  Безопасные 
архитектуры распределенных сис\-тем~// Системы и~средства информатики, 
2014. Т.~24. №\,3. С.~18--31.
\end{thebibliography}

 }
 }

\end{multicols}

\vspace*{-12pt}

\hfill{\small\textit{Поступила в~редакцию 27.06.16}}

%\vspace*{-8pt}

\newpage

\vspace*{-24pt}

%\hrule

%\vspace*{2pt}

%\hrule

%\vspace*{8pt}


\def\tit{INTEGRATION OF STATISTICAL AND~DETERMINISTIC METHODS 
FOR~ANALYSIS OF~INFORMATION SECURITY}

\def\titkol{Integration of statistical and~deterministic methods 
for~analysis of~information security}

\def\aut{A.\,A.~Grusho, N.\,A.~Grusho, M.\,I.~Zabezhailo, and E.\,E.~Timonina}

\def\autkol{A.\,A.~Grusho, N.\,A.~Grusho, M.\,I.~Zabezhailo, and E.\,E.~Timonina}

\titel{\tit}{\aut}{\autkol}{\titkol}

\vspace*{-9pt}

\noindent
Institute of Informatics Problems, Federal Research Center 
``Computer Science and Control'' of the Russian\linebreak
Academy of Sciences,
44-2~Vavilov Str., Moscow 119333, Russian Federation


\def\leftfootline{\small{\textbf{\thepage}
\hfill INFORMATIKA I EE PRIMENENIYA~--- INFORMATICS AND
APPLICATIONS\ \ \ 2016\ \ \ volume~10\ \ \ issue\ 3}
}%
 \def\rightfootline{\small{INFORMATIKA I EE PRIMENENIYA~---
INFORMATICS AND APPLICATIONS\ \ \ 2016\ \ \ volume~10\ \ \ issue\ 3
\hfill \textbf{\thepage}}}

\vspace*{3pt}

  
  \Abste{The paper is devoted to the methods of automatic analysis of information security 
and control mechanisms in cloudy computing environments. The considered approaches are based 
on synthesis of probabilistic and statistical and deterministic methods. 
  The statistical analysis allows creating a set of objects for the deterministic (logical) analysis. As 
deterministic methods demand large volumes of calculations, preliminary statistical processing 
allows to reduce volumes of data for the deterministic (logical) analysis. 
  In the paper, deterministic methods are presented by analogs of search of causal relationships. 
Application of heuristic and plausible reasonings can generate doubtful conclusions which are 
connected with random character of source data. Therefore, the analysis of random 
generation of deterministic conclusions is considered. The suggested methods of the analysis are 
focused on two-level architecture of information security system in cloudy computing environments. 
In this architecture, a~slow automatic data mining generates at the top level fast reaction for 
resolution of conflicts in computing processes or identification of malicious code functioning.}

  \KWE{cloudy computing environments; information security; probabilistic and statistical 
and deterministic (logical) methods of the analysis; heuristic algorithms; mutual influence of data}
    
\DOI{10.14357/19922264160301}

\vspace*{-9pt}

\Ack
\noindent
The paper was supported by the Russian Foundation for Basic Research 
(projects 15-29-07981 and 15-07-02053).


%\vspace*{3pt}

  \begin{multicols}{2}

\renewcommand{\bibname}{\protect\rmfamily References}
%\renewcommand{\bibname}{\large\protect\rm References}

{\small\frenchspacing
 {%\baselineskip=10.8pt
 \addcontentsline{toc}{section}{References}
 \begin{thebibliography}{99}
\bibitem{1-gru-1}
\Aue{Tukey, J.\,W.} 1977. \textit{Exploratory data analysis}. Addison-Wesley Pub. 
Co. 688~p.

\bibitem{3-gru-1}
\Aue{Nortcutt, S., M.~Cooper, M.~Fearnow, and K.~Frederik}. 2001. 
\textit{Intrusion signatures and analysis}. New Readers Pub. 408~p. 

\bibitem{2-gru-1}
\Aue{Grusho, A., M.~Zabezhaylo, and A.~Zatsarinny}. 2015. Kontrol' i upravlenie 
informatsionnymi potokami v~oblachnoy srede [Control and management of 
information streams in the cloudy environment]. \textit{Informatika i~ee 
Primeneniya~--- Inform. Appl.} 9(4):95--101. 

\bibitem{4-gru-1}
\Aue{Bendat, J.\,S., and A.\,G.~Piersol}. 1986. \textit{Random data: Analysis and 
measurement procedures}. 2nd ed. New York, NY: John Wiley and Sons. 566~p.
\bibitem{5-gru-1}
\Aue{Zhuravlev, Yu.\,I.} 1977--1978. Korrektnye algebry nad mnozhestvom 
nekorrektnykh (evristicheskikh) algoritmov. Ch.~I (1977); Ch.~II (1977); 
Ch.~III (1978) [Correct algebras over a set of incorrect (heuristic) algorithms. 
Part~I (1977), Part~II (1977), Part~III (1978)]. \textit{Kibernetika} [Cybernetics] 
I(4):5--17; II(6):21--27; III(2):35--43. 
\bibitem{6-gru-1}
\Aue{Pankratova, E.\,S., and V.\,K.~Finn}. 2009.
\textit{Avtomaticheskoe porozhdenie gipotez} [Automatic hypothesis generation in 
intelligent systems]. Ed.\ V.\,K.~Finn. Moscow:  Librokom. 528~p. 
\bibitem{7-gru-1}
\Aue{Hampel~F.\,R., E.\,M.~Ronchetti, P.\,J.~Rousseeuw, and W.\,A.~Stahel.} 1986. 
\textit{Robust statistics. The approach based on influence functions}. John Willey 
and Sons, Inc. 509~p.
\bibitem{8-gru-1}
\Aue{Feller, W.} 1950. \textit{An introduction to probability theory and its 
applications}. 2d ed. New York, NY: John Wiley and Sons, Inc. Vol.~1. 520~p.
\bibitem{9-gru-1}
\Aue{\mbox{Lo{\!\ptb{\`{e}}}ve}, M.} 1955. \textit{Probability theory}. Princeton, NJ: 
D~Van Nostrand. 701~p. 
\bibitem{10-gru-1}
\Aue{Grusho,~A., N.~Grusho, E.~Timonina, and S.~Shorgin}. 2014. Bezopasnye 
arkhitektury raspredelennykh sistem [Secure architecture of the distributed systems]. 
\textit{Sistemy i~Sredstva Informatiki~--- Systems and Means of Informatics} 
24(3):18--31.
   \end{thebibliography}

 }
 }

\end{multicols}

\vspace*{-3pt}

\hfill{\small\textit{Received June 27, 2016}}

\vspace*{-12pt}

\Contr

\noindent
\textbf{Grusho Alexander A.} (b.\ 1946)~---  Doctor of Science in physics and mathematics, 
professor, Head of Laboratory, Institute of Informatics Problems, Federal Research Center 
``Computer Sciences and Control'' of the Russian Academy of Sciences, 44-2~Vavilov Str., Moscow 
119333, Russian Federation; \mbox{grusho@yandex.ru} 

\pagebreak

\noindent
\textbf{Grusho Nick A.} (b.\ 1982)~---  Candidate of Science (PhD) in physics and mathematics, 
senior scientist, Institute of Informatics Problems, Federal Research Center ``Computer Sciences 
and Control'' of the Russian Academy of Sciences, 44-2~Vavilov Str., Moscow 119333, Russian 
Federation; \mbox{info@itake.ru}

\vspace*{4pt}

\noindent
\textbf{Zabezhailo Michael I.} (b.\ 1956)~--- Candidate of Science (PhD) in physics and 
mathematics, associate professor, Head of Laboratory, Institute of Informatics Problems, Federal 
Research Center ``Computer Sciences and Control'' of the Russian Academy of Sciences,  
44-2~Vavilov Str., Moscow 119333, Russian Federation; \mbox{m.zabezhailo@yandex.ru} 

\vspace*{4pt}

\noindent
\textbf{Timonina Elena E.} (b.\ 1952)~--- Doctor of Science in technology, professor, leading 
scientist, Institute of Informatics Problems, Federal Research Center ``Computer Sciences and 
Control'' of the Russian Academy of Sciences, 44-2~Vavilov Str., Moscow 119333, Russian 
Federation; \mbox{eltimon@yandex.ru}

\label{end\stat}


\renewcommand{\bibname}{\protect\rm Литература}