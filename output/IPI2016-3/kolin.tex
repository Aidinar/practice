\def\stat{kolin}

\def\tit{ГУМАНИТАРНЫЕ АСПЕКТЫ ПРОБЛЕМЫ ИНФОРМАЦИОННОЙ БЕЗОПАСНОСТИ}

\def\titkol{Гуманитарные аспекты проблемы информационной безопасности}

\def\aut{К.\,К.~Колин$^1$}

\def\autkol{К.\,К.~Колин}

\titel{\tit}{\aut}{\autkol}{\titkol}

\index{Колин К.\,К.}
\index{Kolin K.\,K.}


%{\renewcommand{\thefootnote}{\fnsymbol{footnote}} \footnotetext[1]
%{Работа выполнена в~рамках Программы фундаментальных научных исследований 
%в~Российской Федерации на долгосрочный период (2013--2020~годы). Тема №\,34.2. 
%Когнитивные мультимедиа и~интерактивность в~образовании в~условиях мобильного 
%Интернета.}}


\renewcommand{\thefootnote}{\arabic{footnote}}
\footnotetext[1]{Институт проблем информатики Федерального исследовательского
центра <<Информатика и~управление>> Российской академии наук, 
\mbox{kolinkk@mail.ru}}

\vspace*{-6pt}

\Abst{Анализируются гуманитарные аспекты проблемы информационной безопасности (ИБ), 
которая рассматривается как важнейший компонент национальной и~глобальной 
безопасности. Показано, что в~современных условиях становления глобального 
информационного общества и~усиления геополитического противоборства 
в~информационном пространстве ИБ государства, человека 
и~общества становится глобальной проблемой дальнейшего развития цивилизации, при 
этом гуманитарные компоненты этой проблемы выдвигаются на первый план. 
Рассмотрена структура гуманитарных проблем ИБ
и~первоочередные меры по их решению в~России.}

\KW{глобальная безопасность; гуманитарные проблемы; информационная безопасность; 
информационная культура; информационная этика; национальная безопасность}

\DOI{10.14357/19922264160315} 
  
%  \vspace*{-6pt}


\vskip 12pt plus 9pt minus 6pt

\thispagestyle{headings}

\begin{multicols}{2}

\label{st\stat}

\section{Информационная безопасность как~гуманитарная 
проблема}

   Исследования показывают, что обеспечение ИБ
государства, человека и~общества сегодня становится одной из 
глобальных и~стратегически важных проблем дальнейшего развития 
цивилизации в~XXI~в., при этом на первый план выдвигаются 
гуманитарные аспекты этой проблемы, которые необходимо обязательно 
учитывать при ее анализе и~решении~[1--3].
   %
   Возрастание роли гуманитарных аспектов данной проблемы обусловлено 
следующими тенденциями развития современного общества:
   \begin{enumerate}[1.]
\item Процесс информатизации общества принял глобальный характер 
и~сегодня охватывает практически все страны и~регионы мира, при этом 
новые средства и~технологии для работы с~информацией получают 
массовое распространение и~становятся атрибутами профессиональной 
и~бытовой культуры для все большей части населения. Их использование 
повышает качество жизни, дает существенную экономию социального 
времени, создает новые стереотипы поведения и~общения миллионов 
людей, изменяет их традиционные представления о~личном 
и~общественном богатстве~[4] и~даже о~пространстве и~времени.\\[-14pt]
\item По оценкам ряда специалистов, человечество вступило 
в~информационную эпоху своего развития~[5], оно активно формирует 
новую среду обитания и~в результате этого само изменяется вместе с~этой 
средой, оказывающей на человека существенно большее влияние, чем 
это ожидалось ранее~[6].
\item Глобальная информатизация создает для государства, человека 
и~общества не только новые возможности, но также и~новые проблемы, 
одной из которых является проблема ИБ. 
Исследования показывают, что эта проблема является многоаспектной 
и~комплексной, а~ее гуманитарные аспекты недостаточно исследованы, не 
учитываются в~принятой ООН стратегии устойчивого развития~[7], 
а~в~системе образования на необходимом уровне не изучаются~[8].
\end{enumerate}

   Гуманитарный аспект проблемы ИБ состоит 
в~том, что именно человек является творцом всех информационных 
ресурсов, систем и~технологий информационного общества. Поэтому их 
качество и~безопасность использования во многом определяются качествами 
самого человека. При этом речь идет не только о~надежности 
и~эффективности работы этих средств и~систем, но и~об их 
воздействии на человека, общество и~окружающую природу.

\begin{table*}[b]\small %tabl1
\begin{center}
\Caption{Структура гуманитарных проблем ИБ}
\vspace*{2ex}

\begin{tabular}{|l|l|}
\hline
\multicolumn{1}{|c|}{Группа проблем}&\multicolumn{1}{c|}{Краткое содержание 
проблемы}\\
\hline
1.\ Геополитические проблемы&\tabcolsep=0pt\begin{tabular}{l}Технологии <<мягкой 
силы>> в~геополитике~[11, 12]\\
Электронная слежка за политическими лидерами\\
<<Глобальное наблюдение>> за населением\\
Информационные и~<<гибридные>> войны~\cite{13-kol} \end{tabular}\\
\hline
2.\ Социальные проблемы&\tabcolsep=0pt\begin{tabular}{l}Информационная 
преступность\\
Информационное неравенство\\
Манипуляции общественным сознанием~\cite{20-kol}\\
Виртуализация общества \end{tabular}\\
\hline
3.\ Культурологические проблемы&\tabcolsep=0pt\begin{tabular}{l}Глобализация 
и~культура\\
Новая информационная культура общества\\
Электронная культура\\
Многоязычие в~киберпространстве \end{tabular}\\
\hline
4.\ Антропологические проблемы&\tabcolsep=0pt\begin{tabular}{l}Энергоинформационная 
безопасность\\
Интеллектуальная безопасность\\
Информационные факторы деструктивного поведения\\
Информационные болезни\\
Информационная видеоэкология\end{tabular}\\
\hline
\end{tabular}
\end{center}
\end{table*}

\vspace*{-7pt}

\section{Онтологическая двойственность гуманитарных проблем 
информационной безопасности и~их антропологические аспекты}

\vspace*{-2pt}

   Исследования показали, что отличительной особенностью гуманитарных 
проблем ИБ является их \textit{онтологическая 
двойственность}. Она состоит в~том, что человек в~этой проблеме выступает 
не только как \textit{объект защиты} от внешних информационных угроз, но 
также и~как основной \textit{источник этих угроз} для своего внешнего 
окружения. 
   %
   Кроме того, в~современных информационных сис\-те\-мах различного 
назначения, слож\-ность которых неуклонно возрастает, именно человек 
становится основным фактором риска для их безопасного 
функционирования. Эта особенность данной проб\-ле\-мы также является 
принципиально важной для ее понимания и~исследования.
   
   Необходимо отметить, что в~последние годы гуманитарные аспекты 
проблемы ИБ стали все\linebreak более заметно проявлять 
себя не только на социальном и~психологическом, но также и~на 
биологическом уровне природы человека. Так, например, исследования 
американских, немецких\linebreak и~российских ученых показали, что воздействие на 
человека интенсивных потоков информации, которые являются 
характерными для информационного общества, приводят к~изменениям 
нейронной структуры головного мозга человека, которые существенным 
образом изменяют его интеллектуальные и~психические способности, 
социальное поведение, коммуникабельность и~самооценку своих 
поступков~[6].
 %  
   Это означает, что проблемы ИБ сегодня 
необходимо изучать комплексно, с~учетом также и~антропологических 
аспектов этих проб-\linebreak лем. 
   
   Автору представляется, что этому должно содействовать формирование 
\textit{информационной антропологии}~--- новой научной дисциплины, 
которая начала изучаться в~России с~2011~г. Структура предметной 
области этой дисциплины рассмотрена в~работах~[9, 10].

\section{Структура гуманитарных проблем информационной 
безопасности}
   
   Структура основных гуманитарных проблем 
ИБ в~сжатом виде представлена в~табл.~1. В~ней отражены 
четыре группы этих проблем, каждая их которых связана с~определенным 
видом деятель\-ности современного общества.



\section{Социальные проблемы информационной безопасности}

   \textbf{Информационная преступность.} В~числе социальных проблем 
ИБ проблема информационной преступности 
стала изучаться одной из первых. При этом она связывалась, главным 
образом, с~проб\-ле\-мой несанкционированного доступа к~информации, 
хранящейся и~циркулирующей в~компьютерных информационных сис\-те\-мах. 
Эта проб\-ле\-ма\linebreak стала проявлять себя уже в~начале~1990-х~гг.\ в~связи 
с~развитием процесса информатизации общества и~его распространением на 
фи\-нан\-со\-во-эко\-но\-ми\-че\-скую сферу. 
   
   Для противодействия этой угрозе достаточно быст\-ро стали создаваться 
различные системы информационной защиты компьютерных сис\-тем и~сетей, 
которые широко используются и~в настоящее время. Тем не менее 
информационная преступность остается актуальной проблемой и~сейчас, 
причем наибольшую опасность представляют уже не столько атаки хакеров 
с~целью хищения финансовых средств из банков, сколько 
несанкционированный доступ к~конфиденциальной информации 
и~персональным данным отдельных категорий граж\-дан в~компьютерных 
системах, их копирование и~последующее распространение.
   
   \textbf{Проблема информационного неравенства.} Анализ основных 
тенденций развития глобального процесса информатизации общества 
показал, что этот процесс создает для развития цивилизации не только новые 
возможности, но также и~новые проблемы. Одной из них является 
\textbf{проблема информационного неравенства}~\cite{17-kol}.
   
    Суть этой проблемы заключается в~том, что в~процессе становления 
информационного общества электронные информационные ресурсы, а~также 
новые средства, сети и~информационные\linebreak техноло\-гии оказываются 
в~различной степени доступными для отдельных людей, организаций, стран 
и~регионов мирового сообщества. При этом те люди, организации, страны 
и~регионы, которые оказываются способными эффективно использовать 
возможности новой информационной среды общества для своего 
интеллектуального развития и~решения других проблем, получают 
существенные преимущества перед другими субъектами мирового 
сообщества, которые при этом вытесняются на обочину процесса развития 
цивилизации.
    
    Так, например, объем продаж товаров и~услуг через сеть Интернет еще 
в~2000~г.\ превысил сумму в~1~трлн долл.\ США. Однако 
основную долю прибыли от этих продаж получили лишь те страны, 
в~которых эта сеть была в~достаточной степени развитой и~доступной для 
населения.
    
    Что же касается современных средств информатики и~новых 
информационных технологий, то их массовое использование создает 
беспрецедентные возможности не только для на\-уч\-но-тех\-ни\-че\-ско\-го, 
но и~для со\-ци\-аль\-но-эко\-но\-ми\-че\-ско\-го развития общества. При этом 
формируется совершенно новый, информационный уклад жизни 
и~производственной деятельности многих миллионов людей.
    
    Системные исследования проблемы информационного неравенства 
проводятся в~Институте проб\-лем информатики РАН уже более 25~лет. Их 
результаты опубликованы в~ряде статей и~монографий~[15--17] 
и~неоднократно докладывались на международных конференциях. 
   % 
    На основе этих результатов в~1997~г.\ в~\mbox{ЮНЕСКО} была направлена 
аналитическая записка, в~которой была представлена российская концепция 
трактовки содержания проблемы информационного неравенства как новой 
комплексной проблемы глобального масштаба. В~ней было показано, что 
принятый в~тот период времени  
ин\-ст\-ру\-мен\-таль\-но-тех\-но\-ло\-ги\-че\-ский подход к~этой проблеме 
является недостаточным, так как он не учитывает целого ряда важных 
факторов гуманитарного характера. В~их числе такие факторы, как уровень 
информационной, в~том числе лингвистической, культуры человека 
и~общества, а~также уровень их общей образованности, который 
в~значительной мере определяет мотивацию активной деятельности людей 
в~новом информационном пространстве.
    
    Дальнейшее развитие процесса информатизации общества показало, что 
эта концепция является более адекватной реальности, и~поэтому она сегодня 
находит все больше сторонников как в~России, так и~в других странах. Так, 
например, если\linebreak в~1997~г., когда в~ежегодном докладе Программы\linebreak развития 
ООН было введено понятие <<информационной бедности>>, она 
определялась исходя из возможностей доступа людей к~современным  
ин\-фор\-ма\-ци\-он\-но-те\-ле\-ком\-му\-ни\-ка\-ци\-он\-ным технологиям, то 
в~2005~г., на втором этапе Международной встречи по проб\-ле\-мам 
глобального информационного общества в~Тунисе, эта проб\-ле\-ма 
трактовалась уже не как проблема <<цифрового разрыва>> (digitaldivide), 
а~именно как \textit{глобальная проблема информационного неравенства}, 
с~учетом указанных выше ее гуманитарных аспектов.
    
     В работе~\cite{19-kol} показано, что в~структуре этой проблемы 
целесообразно различать следующие три основных аспекта.
\begin{enumerate}[1.]
\item \textit{Личностно-социальный аспект}, который связан с~проблемой 
социальной адаптации человека в~новой, быстро изменяющейся 
информационной среде. Именно здесь возникает новая форма социального 
неравенства людей~--- \textit{информационное неравенство}. Снизить 
остроту этой проблемы призвана перспективная система образования, 
которая должна предоставить возможность всем членам общества получать 
необходимые знания и~умения, для того чтобы правильно ориентироваться 
в~новом информационном пространстве и~эффективно использовать его 
возможности.
    \item \textit{Социально-экономический аспект}, который связан 
с~национальной политикой той или иной страны в~области развития 
информационной среды отдельных регионов и~страны в~целом, их 
информационной инфраструктуры, средств и~методов доступа 
к~информационным ресурсам и~информационным коммуникациям, а~также 
в~области развития и~практического использования информационных 
технологий и~информационного законодательства. Решение этих проблем 
должно являться одним из важнейших направлений государственной 
политики в~на\-уч\-но-тех\-ни\-че\-ской, экономической и~социальной сферах 
современного общества.
     \item  \textit{Геополитический аспект}, который связан 
с~неравномерностью развития процесса информатизации в~различных 
странах и~регионах мира, что объясняется не только различиями  
в~на\-уч\-но-тех\-ни\-че\-ском и~экономическом потенциалах\linebreak
 этих стран, но 
также и~уровнем развития об\-разования в~этих странах, а~также степенью 
понима\-ния их политическими лидерами основных тенденций 
и~закономерностей современного этапа развития цивилизации.
    \end{enumerate}
    
При изучении проблемы информационного неравенства в~контексте задач 
обеспечения ИБ необходимо учитывать, что 
процесс информатизации общества оказывает на него как позитивное, так 
и~негативное воздействие. С одной стороны, он повышает эффективность 
общественного производства и~содействует созданию новых рабочих мест, 
в~том числе для людей с~ограниченными возможностями, повышает 
качество жизни населения. Но, с~другой стороны, появление все более 
сложной информационной техники и~технологий, электронных офисов 
и~роботизированных производств требует от людей более высокого уровня 
квалификации и~интеллекта. А~поскольку система образования во многих 
странах не обеспечивает этих требований, происходит дальнейшее 
социальное расслоение общества, которое усиливает в~нем социальную 
напряженность.
%
     Именно поэтому проб\-ле\-ма информационного неравенства и~должна 
сегодня квалифицироваться как одна из актуальных глобальных проблем, 
тесно связанных с~обеспечением национальной и~глобальной без\-опас\-ности.
{\looseness=1

} 
     
     Для решения этой проблемы необходимо проведение адекватной 
государственной и~международной политики в~области развития 
информационной инфраструктуры общества, в~правовой сфере, а~также 
в~сфере образования и~культуры. Во многих странах это сегодня 
осуществляется на уровне целевых национальных программ развития 
информационного общества.
     
    \textbf{Проблема виртуализации общества.} В~последние годы 
    в~обществе стала все более заметной принципиально новая тенденция 
социальных изменений, которая получила название \textit{виртуализации 
общества}. Суть ее заключается в~том, что во многих жизненно важных 
сферах общества~--- в~экономике, политике, культуре, науке и~образовании~--- 
происходит замещение реальных вещей и~действий их симулякрами~--- 
искусственными образами, которые являются лишь символами этих вещей 
и~действий~[18--20].

    Другими словами, современное человечество активно формирует вокруг 
себя новый, иллюзорный мир символов, который существует параллельно с~
реальным физическим миром и~становится такой же неотъемлемой частью 
нашего бытия, как и~физическая реальность.
    
    Казалось бы, ну и~что здесь плохого? Ведь на то и~дано природой 
человеку сознание и~развитое воображение, чтобы он мог при помощи этих 
двух своих особенных качеств моделировать процессы реального мира 
и~таким образом лучше познавать этот мир, прогнозировать возможное 
развитие в~нем различных процессов. 
    
    Оказывается, все гораздо сложнее. Погружаясь все глубже в~мир 
виртуальности, человек не только подменяет реальные вещи и~действия их 
образами и~символами, но также и~\textit{формирует новые ценности}, 
которые затем оказывают влияние на него самого. А~это уже принципиально 
новый со\-ци\-аль\-но-пси\-хо\-ло\-ги\-че\-ский феномен, и,~как показывают 
исследования, его прогнозируемые последствия далеко неоднозначны.
    
    \textbf{Понятие виртуальности.} Термин <<виртуальный>> 
происходит от латинского слова \textit{virtualis}~--- возможный, вероятный, т.\,е.\ 
такой, который может проявиться при определенных условиях, но реально не 
существует~\cite{22-kol}.
     
     В современном русском языке понятие <<виртуальный>> имеет 
несколько смысловых значений. Сначала это понятие использовали физики 
для обозначения элементарных частиц, имеющих очень малое время 
существования. Затем этот термин стал\linebreak проникать на страницы научной 
и~популярной литературы для обозначения искусственной реаль-\linebreak ности, 
создаваемой в~сознании человека при помо-\linebreak щи новейших средств 
компьютерной техники и~киберне\-тических систем. Эта искусственная 
реальность и~получила название \textit{виртуальной реальности}.
{\looseness=-1

}
     
     Однако в~данной работе речь идет о~совсем\linebreak другом феномене, который 
напрямую не связан с~компьютерной техникой и~кибернетическими 
устройствами. Имеется в~виду то новое явление общественной жизни, 
которое проявляется в~устойчивой тенденции отхода все большего числа 
людей от традиционных условий своего существования, основанных на 
личном общении с~другими людьми. Оно подменяется принципиально 
новыми процессами информационных коммуникаций, где присутствуют 
лишь символы и~образы реального мира, которые постепенно заменяют 
человеку этот мир и~все больше изолируют его от этого мира.
    
    \textbf{Виртуализация общества как глобальный процесс.} Феномен 
виртуализации общества стал объектом внимания ученых совсем недавно, не 
более~15~лет тому назад. Попытки его анализа практически 
одновременно предприняли А.~Бюль и~М.~Поэту в~Германии, а~также 
канадские ученые М.~Вейнстен и~А.~Крокер. В~России одним из первых эту 
проблему стал изучать социолог из Санкт-Пе\-тер\-бург\-ско\-го 
государственного университета Д.\,В.~Иванов~\cite{23-kol}. Результаты 
исследований показали, что здесь мы имеем дело с~принципиально новым 
процессом глобального масштаба, который отражает новые трансформации 
в~современном обществе. Эти трансформации еще мало изучены, но уже 
сегодня понятно, что они имеют достаточно серьезные последствия.
     
     Каковы же причины возникновения процесса виртуализации общества? 
На этот счет сегодня существуют различные точки зрения. Западные ученые 
эти причины связывают в~основном с~развитием процессов информатизации 
общества и~все более широким распространением новых информационных 
и~телекоммуникационных технологий. Так, например, согласно мнению 
Бюля, виртуализация общества представляет собой технический процесс 
создания своеобразного \textit{виртуального общества}, которое существует 
как бы <<параллельно>> с~реальным обществом, не оказывая при этом на 
него существенного влияния. 
     
     Принципиально иной позиции придерживается Иванов, который 
считает, что причины виртуализации общества находятся в~нем самом 
и~заключаются в~\textit{изменении социальной природы самого общества}. 
Что же касается информатизации, компьютеризации и~виртуализации 
общества, то эти процессы являются следствиями, а~не причинами 
вышеуказанных изменений. Именно поэтому виртуализация общества 
и~должна рассматриваться как некая глубинная социальная тенденция 
трансформации самого общества, связанная с~общими закономерностями его 
развития, а~вовсе не как результат развития научно-технического прогресса. 
     
     В соответствии с~этой точкой зрения, которую разделяет и~автор 
настоящей работы, изучение процессов виртуализации общества и~их 
возможных последствий является сегодня весьма актуальной проблемой. Ее 
решение позволило бы не только лучше понять существо и~закономерности 
тех глобальных процессов, которые происходят сегодня в~мировом 
сообществе, но также и~выработать рациональную стратегию адаптации 
человека и~общества к~новым условиям их существования в~XXI~в., которые 
становятся все более динамичными.
     
\section{Культурологические проблемы информационной 
безопасности}

\vspace*{-3pt}

     \textbf{Глобализация и~культура.} Исследования показали, что 
процессы глобализации общества оказывают существенное влияние на его 
культуру~\cite{24-kol}. Развитию процессов глобализации общества 
содействует его все более масштабная информатизация, которая несет за 
собой не только новые средства и~технологии стран Запада, но также и~их 
языки, манеру одеваться, стереотипы поведения и~общения. 
     
     В работе~\cite{25-kol} показано, что с~информационной\linebreak точки зрения 
процессы глобализации общества оказывают на него двоякое воздействие. 
С~од-\linebreak ной стороны, развитие информационных коммуникаций существенно 
повышает \textit{информационную \linebreak связанность} мирового сообщества, 
содействует распространению новых знаний и~технологий, способов 
организации производства и~борьбы с~болезнями. И~этот результат является 
позитивным с~точки зрения перспектив дальнейшего безопасного развития 
цивилизации.
     
     Но, с~другой стороны, деградация национальных культур снижает 
уровень \textit{культурного разнообразия} общества, делает его более 
однородным и,~следовательно, менее приспособленным к~противодействию 
глобальным вызовам и~угрозам XXI~в. 
     
     Кроме того, разрушаются духовные ценности национальных культур, 
а~вместо них насаждаются новые ценности потребительского общества. Этот 
процесс является одной из глобальных угроз для безопасного развития 
цивилизации, что уже признается не только российскими, но и~западными 
учеными~\cite{25-kol, 26-kol}.

\begin{table*}\small
\begin{center}
\Caption{Культурологические проблемы ИБ}
\vspace*{2ex}

\tabcolsep=3.7pt
\begin{tabular}{|l|l|}
\hline
\multicolumn{1}{|c|}{Группа проблем}&\multicolumn{1}{c|}{Краткое содержание 
проблемы}\\
\hline
1.\ Глобализация и~культура&\tabcolsep=0pt\begin{tabular}{l}Деградация национальных 
культур\\
Этнос и~нация в~культурологической перспективе\\
Национальное единство в~условиях глобализации\\
Развитие человеческих ресурсов в~информационном обществе\end{tabular}\\
\hline
2.\ Человек в~информационном обществе&\tabcolsep=0pt\begin{tabular}{l}Новая структура 
занятости населения\\
Усиление технократии\\
Новые формы информационного неравенства\\
Урбанизация в~информационном обществе\end{tabular}\\
\hline
3.\ Языки в~новом информационном 
пространстве&\tabcolsep=0pt\begin{tabular}{l}Информационная бедность 
и~лингвистическая культура\\
Сокращение мирового русскоязычного пространства\\
Многоязычие в~киберпространстве\\
Технологии автоматизированного перевода текстов и~речи\end{tabular}\\
\hline
4.\ Электронная культура&\tabcolsep=0pt\begin{tabular}{l}Безопасность электронной 
информационной техники\\
Массовое обучение пользователей\\
Формирование культуры ИБ\end{tabular}\\
\hline
\end{tabular}
\end{center}
\vspace*{6pt}
\end{table*}
     
     \textbf{Новая информационная культура общества.} Проб\-ле\-ма 
формирования новой информационной культуры общества, которая должна 
быть адекватной условиям жизни и~деятельности людей в~новой 
информационной среде их обитания, была по\-став\-ле\-на в~России академиком 
А.\,П.~Ершовым еще в~1988~г.~\cite{26-kol}. Он показал, что эта проблема 
будет глобальной, стратегически важной и~социально значимой для развития 
цивилизации в~XXI~в. Однако системные исследования этой проблемы 
начались лишь в~2011~г., когда в~Германии на русском языке была издана 
первая монография, специально посвященная этой проблеме~\cite{27-kol}. 
В~ней было показано, что для комплексного изучения проблем становления 
и~развития новой информационной культуры общества должна быть 
сформирована специальная научная дисциплина~--- \textit{информационная 
культурология}.
     
     В данной монографии была предложена структура предметной об\-ласти 
этой дисциплины, рассмотрены ее основные задачи и~перспективы\linebreak  развития, 
показана их связь с~проблемами обеспечения~ИБ.
     
     В 2015~г.\ эта монография в~существенно переработанном виде была 
издана и~в~России~\cite{28-kol}. При этом культурологическим аспектам 
проблемы ИБ в~ней посвящен отдельный раздел, 
вклю\-ча\-ющий пять глав. Состав этих проблем представлен в~табл.~2. 
     
     Таким образом, можно утверждать, что Россия сегодня является 
лидером в~об\-ласти изучения проб\-лем информационной культуры 
в~комплексной постановке с~учетом взаимосвязи с~проблемами 
ИБ.



     
\section{Антропологические проблемы информационной 
безопасности}



    \textbf{Энергоинформационная безопасность в~информационном 
обществе.} Современная промышленная и~технологическая революция 
существенным образом изменили энергоинформационное поле нашей 
планеты. Мощные электростанции, крупные\linebreak промышленные производства, 
высоковольтные\linebreak линии электропередачи, городские здания и~сооружения~--- 
все эти объекты создают вокруг себя достаточно интенсивные 
электромагнитные поля, которые постоянно окружают современного 
человека и~воздействуют на его организм. 
    
    Развитие информационного общества усиливает это воздействие 
и~делает его глобальным. Ведь средства телевидения и~мобильной связи 
сегодня имеются практически в~каждой семье и~регулярно используются как 
взрослыми, так и~детьми, причем их количество и~интенсивность 
использования продолжают возрастать. 
    
    Какое воздействие оказывает электромагнитное излучение этих средств 
на организм человека?\linebreak
 Каков допустимый уровень этого воздействия на 
детей, взрослых, а~также на зародышей, еще находящихся в~утробе матери? 
Какими могут быть последствия этого воздействия? На все эти вопросы пока 
нет удовлетворительных ответов, так как данная проблема системно не 
изучается. А~ведь она является глобальной и~представляет серьезную угрозу 
не только для человека, но и~для всей биосферы нашей планеты. 
    
    Так, например, одним из тревожных признаков является сокращение 
численности пчел, которое в~последние годы наблюдается во многих странах, 
но причина его пока не выявлена. Возможно, это связано с~развитием средств 
мобильной связи.
    
    Впервые проблема энергоинформационной без\-опас\-ности была 
поставлена автором настоящей статьи в~работе~\cite{19-kol}. Эта публикация 
не привлекла к~себе внимания специалистов в~области глобальных 
экологических проблем, однако надеяться, что она сама собой решится, 
также нет оснований. Ведь подавляющая часть объектов энергетики и~связи 
находится сегодня в~собственности частных компаний, заинтересованных 
главным образом в~получении прибыли, а~не в~решении проблем 
энергоинформационной безопасности человека и~общества. 
    
    \textbf{Поколение Next и~новая проблема интеллектуальной 
безопасности.} Исследования последних лет показывают, что 
информатизация общества оказывает сильное воздействие не только на 
социальные аспекты повседневной жизни и~профессиональной деятельности 
людей, но также на их психику, образ мышления и~даже на развитие 
головного мозга. Так, например, американские психологи Г.~Смолл 
и~Г.~Ворган в~своей монографии~\cite{6-kol} утверждают, что новое поколение людей 
информационной эпохи, которое уже получило название <<поколения 
Next>>, будет обладать совсем другой психикой и~образом мышления по 
сравнению с~людьми старшего поколения. При этом весьма вероятно, что 
нейронная структура головного мозга у этих людей будет отличаться от той, 
которая существует в~настоящее время.
    
    Свою гипотезу авторы указанной монографии аргументируют 
следующим образом. Согласно тео\-рии эволюции Чарльза Дарвина, развитие 
головного мозга человека происходит в~результате его приспособления 
к~изменениям окружающей среды. Эта общая закономерность действует 
и~сегодня, в~условиях стремительного развития процесса информатизации 
и~формирования глобального информационного общества. А~поскольку 
наиболее радикальные и~быстрые перемены происходят именно 
в~информационной сфере общества, мозг человека начинает 
приспосабливаться к~этим изменениям путем адекватных изменений 
в~организации своей структуры. И~этот феномен является вполне 
закономерным. Вероятнее всего, в~ближайшие годы он будет только 
нарастать.
    
    Проблема здесь заключается в~том, что указанные изменения в~природе 
человека происходят слишком быстро, на протяжении жизни одного 
поколения людей. Для психологов это оказалось полной неожиданностью. 
Ведь таких радикальных изменений природа человека не испытывала 
никогда, а~по своей значимости они сопоставимы, пожалуй, лишь 
с~феноменом появления членораздельной речи. 
    
    Но ведь и~масштабы современной информационной революции также 
являются беспрецедентными в~истории человечества. Их значимость 
и~возможные последствия еще в~необходимой мере не исследованы. Это нам 
еще предстоит сделать в~будущем. 
    
    \textbf{Отличительные черты людей эпохи Интернета.} Всех нас 
удивляет, как быстро и~легко дети осваивают современную достаточно 
сложную информационную технику и~новые информационные технологии. 
Специалисты по возрастной психологии знают, что мозг ребенка является 
очень пластичным, поэтому дети легко осваивают и~новые языки, и~новую 
технику, и~новые стереотипы поведения людей в~информационном обществе. 
При этом у~них вырабатываются совсем другие, отличные от традиционных, 
формы мыслительной дея\-тель\-ности, обусловленные повседневным 
использованием новых информационных технологий. Их мозг становится 
способным к~обработке больших объемов информации, а~также к~быстрой 
реакции на зрительные образы. 
    
    Развитию этих способностей в~значительной мере содействует активное 
использование компьютерных поисковых систем, компьютерные игры, 
а~также общение по электронной почте. Социологические исследования 
развития интеллектуального уровня людей показывают, что IQ среднего 
человека в~XXI~в.\ стремительно растет. 
    
    Вполне возможно, что новые информационные технологии развивают 
интеллект точно так же, как это делают головоломки, игра в~шахматы 
и~изучение новых языков. Наблюдения показывают, что люди, часто 
использующие Интернет, как правило, быстрее находят выход из сложных 
положений и~в повседневной жизни. Ведь каждый день, отыскивая для себя 
в~сети нужную информацию, они тренируют те мозговые центры, которые 
связаны с~оперативным решением практических задач. 
    
    Однако с~развитием интеллекта и~логического мышления у нового 
поколения людей эпохи Интернет не так все однозначно. Здесь есть 
и~достаточно серьезные негативные факторы.
    
    \textbf{Угроза психологического расслоения человечества 
в~информационном обществе.} Исследования показы\-вают, что постоянное 
использование компьютерных информационных технологий влечет за собой 
не только положительные, но и~отрицательные последствия для психики 
человека. Одно из них~--- это так называемое <<клиповое 
мышление>>~\cite{6-kol}. 
    
    Суть этого феномена состоит в~том, что частое использование сети 
Интернет уменьшает способность человека к~концентрации мысли, 
созерцанию и~абстрактному мышлению. Его мозг начинает постепенно 
привыкать к~получению информации в~готовом виде, которую уже не нужно 
анализировать, поэтому и~процесс мышления у~таких людей становится 
фрагментарным, <<клиповым>>. 
    
    Таким образом, вместо мыслителя человек превращается в~своего рода 
сортировщика готовой\linebreak информации, при этом те зоны мозга, которые 
отвеча\-ют за абстрактное мышление, постепенно деградируют, и~в~будущем, 
вполне возможно, они могут совсем атрофироваться. Как же он сможет 
решать те новые глобальные проблемы XXI~в.~\cite{29-kol}, которые, как 
снежная лавина, нарастают уже сегодня? В~этом, по мнению автора, 
и~состоит суть новой глобальной проблемы \textit{интеллектуальной 
без\-опас\-ности}~\cite{17-kol}.
    
    Тревогу вызывает тот факт, что указанные изменения психики чаще 
всего наблюдаются у~молодого поколения людей, вырастающих 
в~современную информационную эпоху. Так, например, в~Японии, одной из 
наиболее информационно развитых стран мира, многие школьники младших 
классов сегодня не умеют считать в~уме, так как вместо этого используют 
калькуляторы в~своих смартфонах или компьютерах. На это уже обратили 
свое внимание японские преподаватели, которые специально заставляют 
таких школьников считать в~уме и~даже сдавать соответствующие экзамены. 
    
    Таким образом, на наших глазах вырастает новое поколение людей, 
которые будут обладать совсем другой психикой и~другим типом мышления. 
Их отличительной чертой будет рассеянное внимание, когда человек следит 
за всем сразу, ни на чем не сосредотачиваясь. Они будут хуже нас общаться 
между собой в~обычной, не компьютерной реальности, так как их мозг будет 
все больше утрачивать те базовые механизмы, которые управляют 
контактами с~другими людьми. 
    
    Их память будет все меньше использоваться для запоминания 
фактографической и~другой информации, так как ее <<кибернетическими 
протезами>> станут персональные компьютеры, смартфоны и~электронные 
базы данных сети Интернет. Эти люди, вероятнее всего, будут запоминать не 
саму информацию, а~метаинформацию, т.\,е.\ информацию о~том, в~какой 
папке компьютерной памяти она хранится или же в~какой электронной 
библиотеке ее можно найти. 
    
    Следует ожидать, что представители <<поколения Next>> будут еще 
меньше, чем наши современники, читать художественную литературу, 
в~особенности классическую, историческую и~научную. Зачем им это делать, 
если есть Интернет и~Википедия? Теат\-ры, консерватории и~музеи, скорее 
всего, им также будут неинтересны. Ведь с~их содержанием можно будет 
познакомиться в~электронной сети, не выходя из дома. 
    
    В информационном обществе важную роль играет \textit{сетевое 
общение}, которое существенным образом расширяет возможность контактов 
с~другими людьми, повышает их оперативность и~экономит массу 
социального времени. Но ведь при этом возникает риск забыть о~том, что на 
самом деле представляет собой дружба между людьми в~реальном мире, для 
которой необходимо реальное общение. 
    
    Вполне возможно, что в~информационном обществе появится также 
и~\textit{новый вид одиночества}. Это ситуация, когда телевизор и~другие 
средства информационной техники выключены и~человек остается один 
в~уже мало привычном для него реальном мире. Ведь уже давно известно, 
что нигде люди не чувствуют себя так одинокими, как в~большом городе, 
когда они часто не знакомы даже с~теми, кто живет в~соседней квартире. 
    
    \textbf{Информационные факторы деструктивного поведения 
человека.} В~числе новых направлений исследования антропологических 
аспектов проблем ИБ следует отметить работы 
российского композитора В.\,С.~Дашкевича, где рассматривается влияние на 
деятельность головного мозга человека той акустической и,~в~част\-ности, 
музыкальной среды, в~которой он обитает. В~них показано, что одной из 
причин повышения уровня деструктивности поведения людей в~современном 
обществе, является его \textit{музыкальная культура}~\cite{30-kol}. 
     
     Таким образом, музыкальная культура общества также должна стать 
одним из объектов тех перспективных исследований, которые должны 
проводиться в~интересах изучения гуманитарных проблем ИБ.
    
    \section{Заключение}
     
     \textbf{Доктрина и~Стратегия информационной безопасности 
Российской Федерации.} В~настоящее время концептуальные основы 
ИБ России определяет Доктрина 
информационной без\-опас\-ности РФ, принятая еще в~2000~г. По своему 
содержанию она представляет собой развернутый и~достаточно хорошо 
продуманный документ, многие положения которого являются актуальными 
и~в~настоящее время. Тем не менее он требует корректировки, так как за 
последние годы ситуация в~данной области изменилась в~худшую для России 
сторону и, кроме того, появились новые информационные вызовы и~угрозы 
как национального, так и~глобального масштаба. 
     %
     Поэтому еще в~декабре 2015~г.\ Советом Безопасности РФ был 
подготовлен проект новой Доктрины информационной безопасности, 
утверждение которой ожидается в~2016~г. Однако до сих пор текст этого 
документа не опубликован и~на общественном уровне еще не обсуждался.
     
     Представляется, что для эффективного противодействия комплексу 
современных угроз для ИБ России необходимо 
также разработать и~принять \textit{Стратегию информационной 
безопасности РФ на период до 2030~г.} В~ней должны быть определены 
конкретные задачи в~этой области, сроки их решения и~количественные 
показатели, необходимое правовое, организационное и~кадровое 
обеспе\-чение. 
     
     Отметим, что вопрос о~необходимости разработки и~принятия 
Стратегии информационной безопас\-ности России на среднесрочный период 
неоднократно ставился Институтом проблем информатики РАН как 
в~научных публикациях, так и~в~Аналитических материалах, которые 
Российская академия наук представляла для включения в~ежегодный Доклад 
Президенту России <<О~состоянии национальной безопасности РФ и~мерах 
по ее укреплению>>. Сегодня пришло время, когда Стратегия 
информационной безопасности России является крайне необходимой.
     
     Необходимо также пересмотреть содержание научных дисциплин ВАК 
РФ по тематике ИБ, предусмотрев в~них 
наиболее актуальные гуманитарные аспекты, некоторые из которых были 
рассмотрены в~данной работе.

{\small\frenchspacing
 {%\baselineskip=10.8pt
 \addcontentsline{toc}{section}{References}
 \begin{thebibliography}{99}
 \bibitem{2-kol} %1
\Au{Колин К.\,К.} Информационная безопасность как гуманитарная проб\-ле\-ма~// 
Открытое образование, 2006. №\,1(54). С.~86--93.
\bibitem{1-kol} %2
\Au{Соколов И.\,А., Колин К.\,К.} Развитие информационного общества в~России и~
актуальные проблемы информационной безопасности~// Информационное общество, 
2009. №\,4-5. С.~98--106.

\bibitem{3-kol}
\Au{Колин К.\,К.} Гуманитарные проблемы информационной безопасности~// 
Информационные технологии, 2012. №\,12. Приложение. С.~1--32.
\bibitem{4-kol}
\Au{Тоффлер Э.} Революционное богатство.~--- М.: ACT Москва, 2008. 569~с.
\bibitem{5-kol}
\Au{Кастельс М.} Информационная эпоха: экономика, общество и~культура.~--- М.: 
ГУ ВШЭ, 2000. 458~с.
\bibitem{6-kol}
\Au{Смолл Г., Врган Г.} Мозг онлайн. Человек в~эпоху Интернета.~--- М.: Колибри, 
2011. 352~с.
\bibitem{7-kol}
\Au{Колин К.\,К.} Половинчатая стратегия: критический анализ новой Стратегии ООН 
в~об\-ласти устойчивого развития~// Партнерство цивилизаций, 2016. №\,1-2.  
С.~33--41.
\bibitem{8-kol}
\Au{Соколов И.\,А., Колин К.\,К.} Новый этап информатизации общества и~актуальные 
проблемы образования~// Информатика и~её применения, 2008. Т.~2. Вып.~1.  
С.~67--76.
\bibitem{9-kol}
\Au{Колин К.\,К.} Информационная антропология: поколение Next и~угроза 
психологического расслоения человечества в~информационном обществе~// Вестник 
Челябинской государственной академии культуры и~искусств, 2011. №\,4. С.~32--36.
\bibitem{10-kol}
\Au{Колин К.\,К.} Информационная антропология: предмет и~задачи нового 
направления в~науке и~образовании~// Вестник Кемеровского государственного ун-та 
культуры и~искусств, 2011. №\,17. С.~17--32.
\bibitem{11-kol}
\Au{Смирнов А.\,И., Кохтюлина И.\,Н.} Глобальная безопасность и~<<мягкая сила 
2.0>>: вызовы и~возможности для России.~--- М.: ВНИИГеосистем, 2012. 276~с.
\bibitem{12-kol}
\Au{Шабалов М.\,П.} <<Мягкая сила>> в~современной геополитике~// Стратегические 
приоритеты, 2014. №\,4. С.~27--43.
\bibitem{13-kol}
\Au{Кошкин Р.\,П.} Россия и~мир: новые приоритеты в~геополитике.~--- М.: 
Стратегические приоритеты, 2015. 236~с.
%\bibitem{14-kol}
%\Au{Колин К.\,К.} Новая Военная доктрина и~гуманитарные проблемы национальной 
%безопасности России~// Стратегические приоритеты, 2015. №\,1(5). С.~30--47.
%\bibitem{15-kol}
%Стратегия национальной безопасности Российской Федерации. Утверждена Указом 
%Президента РФ от~31~декабря 2015~г. №\,683.
%\bibitem{16-kol}
%\Au{Бетелин В.\,Б.} О проблеме импортозамещения и~альтернативной модели 
%экономического развития России~// Стратегические приоритеты, 2016. №\,1(9).  
%С.~11--21.
\bibitem{20-kol} %14
\Au{Кара-Мурза С.\,Г.} Манипуляции сознанием.~--- М.: Алгоритм, 2000. 688~с.
\bibitem{17-kol} %15
\Au{Колин К.\,К.} Информационное неравенство~--- новая проблема XXI~века~// 
Социология, социальность, современность.~--- М.: Союз, 1998. Вып.~5. 
С.~99--101.

\bibitem{19-kol} %16
\Au{Колин К.\,К.} Информационные проб\-ле\-мы  
со\-ци\-аль\-но-эко\-но\-ми\-че\-ско\-го развития общества.~--- М.: Союз, 1995. 72~с.
\bibitem{18-kol} %17
\Au{Колин К.\,К.} Глобальные проблемы информатизации: информационное 
неравенство~// Alma mater (Вестник высшей школы), 2000. №\,6. С.~27--32.
\bibitem{23-kol} %18
\Au{Иванов Д.\,В.} Виртуализация общества.~--- СПб.: Петербургское востоковедение, 
2000. 96~с.

\bibitem{21-kol} %19
\Au{Колин К.\,К.} Виртуализация общества~--- новая угроза для его стабильности~// 
Синергетическая парадигма. Человек и~общество в~условиях нестабильности.~--- 
М.: РАГС, 2003. С.~449--462.
\bibitem{22-kol} %20
\Au{Колин К.\,К.} Виртуализация общества~// Большая Российская энциклопедия, 
2006. Т.~5. С.~370.

\bibitem{24-kol} %21
\Au{Колин К.\,К.} Глобализация и~культура~// Вестник Биб\-ли\-о\-течной Ассамблеи 
Евразии, 2004. №\,1. С.~12--15.
\bibitem{25-kol} %22
\Au{Колин К.\,К.} Информатизация общества и~глобализация.~--- Красноярск, 
СФУ, 2011. 52~с.
\bibitem{26-kol} %23
\Au{Ершов А.\,П.} Информатизация: от компьютерной грамотности школьников 
к~информационной культуре общества~// Коммунист, 1988. №\,2. С.~82--92. 
\bibitem{27-kol} %24
\Au{Колин К.\,К., Урсул А.\,Д.} Информационная культурология: предмет и~задачи 
нового научного направления.~--- Saarbruchen, Germany: LAP Lambert Academic 
Publishing, 2011. 249~с.
\bibitem{28-kol} %25
\Au{Колин К.\,К., Урсул А.\,Д.} Информация и~культура. Введение в~информационную 
культурологию.~--- М.: Стратегические приоритеты, 2015. 300~с.
\bibitem{29-kol} %26
\Au{Колин К.\,К.} Глобальные угрозы развитию цивилизации в~XXI~веке~// 
Стратегические приоритеты, 2014. №\,1. С.~6--30.
\bibitem{30-kol} %27
\Au{Дашкевич В.\,С.} Великое культурное одичание. Арт-ана\-лиз.~--- М.: Russian 
Chess House, 2013. 717~с.
\end{thebibliography}

 }
 }

\end{multicols}

\vspace*{-3pt}

\hfill{\small\textit{Поступила в~редакцию 18.07.16}}

\vspace*{8pt}

\newpage

\vspace*{-30pt}

%\hrule

%\vspace*{2pt}

%\hrule

%\vspace*{8pt}



\def\tit{HUMANITARIAN ASPECTS OF~INFORMATION SECURITY}

\def\titkol{Humanitarian aspects of information security}

\def\aut{K.\,K.~Kolin}

\def\autkol{K.\,K.~Kolin}

\titel{\tit}{\aut}{\autkol}{\titkol}

\vspace*{-9pt}

\noindent
Institute of Informatics Problems, Federal Research Center 
``Computer Science and Control'' of the Russian\linebreak
Academy of Sciences,
44-2~Vavilov Str., Moscow 119333, Russian Federation


\def\leftfootline{\small{\textbf{\thepage}
\hfill INFORMATIKA I EE PRIMENENIYA~--- INFORMATICS AND
APPLICATIONS\ \ \ 2016\ \ \ volume~10\ \ \ issue\ 3}
}%
 \def\rightfootline{\small{INFORMATIKA I EE PRIMENENIYA~---
INFORMATICS AND APPLICATIONS\ \ \ 2016\ \ \ volume~10\ \ \ issue\ 3
\hfill \textbf{\thepage}}}

\vspace*{3pt}

\Abste{The paper analyzes humanitarian aspects of information security, which is 
regarded as the most important component of national and global security. It is 
shown that in modern conditions, the formation of the global information society 
and strengthening of geopolitical confrontation in the information space of 
information security of state, individual, and society is becoming a global problem 
for the further development of civilization. The humanitarian component of this 
problem comes to the forefront. The structure of humanitarian problems of 
information security is described. Priority measures for their solution in Russia are 
proposed.}

\KWE{global security; humanitarian issues; information security; information 
culture; information ethics; national security}

\DOI{10.14357/19922264160315} 

%\vspace*{-9pt}

%\Ack
%\noindent


%\vspace*{3pt}

  \begin{multicols}{2}

\renewcommand{\bibname}{\protect\rmfamily References}
%\renewcommand{\bibname}{\large\protect\rm References}

{\small\frenchspacing
 {%\baselineskip=10.8pt
 \addcontentsline{toc}{section}{References}
 \begin{thebibliography}{99}
      
      
      \bibitem{2-kol-1}
      \Aue{Kolin, K.\,K.} 2006. Informatsionnaya bezopasnost' kak gumanitarnaya problema 
[Information security as a humanitarian problem]. \textit{Otkrytoe Obrazovanie} [Open 
Education] 1:86--93.

\bibitem{1-kol-1}
      \Aue{Sokolov, I.\,A., and K.\,K.~Kolin}. 2009. Razvitie informatsionnogo obshchestva 
v~Rossii i~aktual'nye problemy informatsionnoy bezopasnosti [Information society 
development in Russia and problems of information security]. \textit{Informatsionnoe 
Obshchestvo} [Information Society] 4-5:98--106.

      \bibitem{3-kol-1}
      \Aue{Kolin, K.\,K.} 2012. Gumanitarnye problemy infor\-ma\-tsi\-onnoy bezopasnosti [Since 
the humanitarian problems of information security]. \textit{Informatsionnye Tekhnologii}  
[Information Technology] 12(App.):1--32.
\bibitem{4-kol-1}
\Aue{Toffler, E.} 2008. \textit{Revolyutsionnoe bogatstvo} [Revolutionary wealth]. Moscow: 
ACT Moskva. 569~p.
      \bibitem{5-kol-1}
      \Aue{Kastel's, M.} 2000. \textit{Informatsionnaya epokha: Ekonomika, obshchestvo 
i~kul'tura} [The information age: Economy, society, and culture]. Moscow: GU VShE. 458~p.
      \bibitem{6-kol-1}
      \Aue{Smoll, G., and G.~Vrgan}. 2011. \textit{Mozg onlayn. Chelovek v~epokhu 
Interneta} [Brain online. People in the Internet age]. Moscow: Kolibri. 352~p.
      \bibitem{7-kol-1}
      \Aue{Kolin, K.\,K.} 2016. Polovinchataya strategiya: Kriticheskiy analiz novoy Strategii 
OON v~oblasti ustoychivogo razvitiya [Since a half-hearted strategy: A~critical analysis of the 
new UN Strategy on sustainable development]. \textit{Partnerstvo Tsivilizatsiy}  
[Partnership of  Civilizations] 1-2:33--41.
      \bibitem{8-kol-1}
      \Aue{Sokolov, I.\,A., and K.\,K.~Kolin}. 2008. Novyy etap informatizatsii obshchestva 
i~aktual'nye problemy obrazovaniya 
[The new stage of the society informatization and actual 
problems of education]. \textit{Informatika i~ee Primeneniya~--- Inform. Appl.} 
2(1):67--76.
      \bibitem{9-kol-1}
      \Aue{Kolin, K.\,K.} 2011. Informatsionnaya antropologiya: Pokolenie Next i~ugroza 
psikhologicheskogo rassloeniya chelovechestva v~informatsionnom obshchestve [Information 
anthropology: The Next generation and the threat of psychological stratification of humanity in 
the information society]. \textit{Vestnik Chelyabinskoy gosudarstvennoy akademii kul'tury 
i~iskusstv}  [Bulletin of the Chelyabinsk State Academy of Culture and Arts] 4:32--36.
      \bibitem{10-kol-1}
      \Aue{Kolin, K.\,K.} 2011. Informatsionnaya antropologiya: Predmet i~zadachi novogo 
napravleniya v~nauke i~obrazovanii [Information anthropology: The subject and objectives of the 
new direction in science and education]. \textit{Vestnik Kemerovskogo gosudarstvennogo un-ta 
kul'tury i~iskusstv}  [Bulletin of Kemerovo State University of Culture and Arts] 17:17--32.
      \bibitem{11-kol-1}
      \Aue{Smirnov, A.\,I., and I.\,N.~Kokhtyulina}. 2012. Global'naya bezopasnost' 
i~``myagkaya sila 2.0:'' Vyzovy i~vozmozhnosti dlya Rossii [Global security and ``soft power 
2.0:'' Challenges and opportunities for Russia]. Moscow: \mbox{VNIIGeosistem}. 276~p.
      \bibitem{12-kol-1}
      \Au{Shabalov, M.\,P.} 2014. ``Myagkaya sila'' v~sovremennoy geopolitike [``Soft 
power'' in contemporary geopolitics]. \textit{Strategicheskie Prioritety}  [Strategic Priorities] 
4:27--43.
      \bibitem{13-kol-1}
      \Au{Koshkin, R.\,P.} 2015. Rossiya i~mir: Novye prioritety v~geopolitike [Russia and the 
world: New priorities in geopolitics]. Moscow: Strategicheskie prioritety [Strategic Priorities]. 
236~p.
     % \bibitem{14-kol-1}
%      \Aue{Kolin, K.\,K.} 2015. Novaya voennaya doktrina i~gumanitarnye problemy 
%natsional'noy bezopasnosti Rossii [Since the New Military doctrine and humanitarian problems 
%of national security of Russia]. \textit{Strategicheskie Prioritety}  [Strategic Priorities]  
%1(5):30--47.
%      \bibitem{15-kol-1}
%       Strategiya natsional'noy bezopasnosti Rossiyskoy Federatsii. 
%       2015. Utverzhdena Ukazom 
%Prezidenta RF ot~31~dekabrya 2015~g. No.\,683 [The national security strategy of the Russian 
%Federation. Approved by the Decree of the President of the Russian Federation from 
%December~31, 2015, No.\,683].
%      \bibitem{16-kol-1}
%      \Aue{Betelin, V.\,B.} 2016. O~probleme importozameshcheniya i~al'ternativnoy modeli 
%ekonomicheskogo razvitiya Rossii [On the problem of import substitution and alternative models 
%of economic development of Russia]. \textit{Strate\-gi\-che\-skie prioritety} [Strategic Priorities] 
%1(9):11--21.
 \bibitem{20-kol-1} %14
      \Aue{Kara-Murza, S.\,G.} 2000. \textit{Manipulyatsii soznaniem} [Manipulation of 
consciousness]. Moscow: Algoritm. 688~p.
      \bibitem{17-kol-1} %15
      \Aue{Kolin, K.\,K.} 1998. Informatsionnoe neravenstvo~--- novaya problema 
XXI~veka [Information inequality is a new problem of the XXI century]. \textit{Sotsiologiya, 
sotsial'nost', sovremennost'} [Sociology, sociality, modernity]. 
Moscow: Soyuz. 5:99--101.
 \bibitem{19-kol-1} %16
      \Aue{Kolin, K.\,K.} 1995. \textit{Informatsionnye problemy sotsial'no-ekonomicheskogo 
razvitiya obshchestva} [Information problems of socio-economic development of society]. 
Moscow: Soyuz. 72~p.
      \bibitem{18-kol-1} %17
      \Aue{Kolin, K.\,K.} 2000. Global'nye problemy informatizatsii: Informatsionnoe 
neravenstvo [The global problem of informatization: Information inequality]. \textit{Alma mater 
(Vestnik vysshey shkoly)}  [Alma Mater (Bulletin of High School)] 6:27--32.
     
     
\bibitem{23-kol-1} %18
      \Aue{Ivanov, D.\,V.} 2000. Virtualizatsiya obshchestva 
      [Virtualization of society]. Saint Petersburg: 
Peterburgskoe vostokovedenie. 96~p.
      \bibitem{21-kol-1} %19
      \Aue{Kolin, K.\,K.} 2003. Virtualizatsiya obshchestva~-- novaya ugroza dlya ego 
stabil'nosti [Since virtualization companies~--- a new threat to its stability]. 
\textit{Sinergeticheskaya paradigma. Chelovek i~obshchestvo v~usloviyakh 
nestabil'nosti} [Synergetic paradigm. Man and society in conditions of instability]. 
Moscow: RAGS. 449--462.
      \bibitem{22-kol-1} %20
      \Aue{Kolin, K.\,K.} 2006. Virtualizatsiya obshchestva [Virtualization of society]. 
\textit{Bol'shaya Rossiyskaya entsiklopediya} [Great Russian Encyclopedia]. 5:370.
      
      \bibitem{24-kol-1} %21
      \Aue{Kolin, K.\,K.} 2004. Globalizatsiya i~kul'tura [Globalization and culture]. 
\textit{Vestnik Bibliotechnoy Assamblei Evrazii}  
[Bulletin of the Library Assembly of Eurasia] 
1:12--15.
      \bibitem{25-kol-1} %22
      \Aue{Kolin, K.\,K.} 2011. \textit{Informatizatsiya obshchestva i~glo\-ba\-li\-za\-tsiya}
      [Information society and globalization]. Krasnoyarsk: SFU. 52~p.
      \bibitem{26-kol-1} %23
      \Aue{Ershov, A.\,P.} 1998. Informatizatsiya: Ot komp'yuternoy gramotnosti shkol'nikov 
k~informatsionnoy kul'ture obshchestva [Informatization: From computer literacy to information 
culture society]. \textit{Kommunist} [Communist] 2:82--92.
      \bibitem{27-kol-1} %24
      \Aue{Kolin, K.\,K., and A.\,D.~Ursul}. 2011. \textit{Informatsionnaya kul'turologiya: 
Predmet i~zadachi novogo nauchnogo napravleniya} [Information studies: Subject and problems 
of a new scientific direction]. Saarbruchen, Germany: LAP Lambert Academic Publishing. 
249~p.
      \bibitem{28-kol-1} %25
      \Aue{Kolin, K.\,K., and A.\,D.~Ursul}. 2015. \textit{Informatsiya i~kul'tura. Vvedenie 
v~informatsionnuyu kul'turologiyu} [Information and culture. Introduction to information 
studies]. Moscow: Strategicheskie prioritety. 300~p.
      \bibitem{29-kol-1} %26
      \Aue{Kolin, K.\,K.} 2014. Global'nye ugrozy razvitiyu tsivilizatsii v~XXI~veke [The 
global threat to the development of civilization in the XXI~century]. \textit{Strategicheskie 
Prioritety}  [Strategic Priorities] 1:6--30.
      \bibitem{30-kol-1} %27
      \Aue{Dashkevich, V.\,S.} 2013. \textit{Velikoe kul'turnoe odichanie. Art-analiz} [The 
Great cultural savagery. Art-analysis]. Moscow: Russian Chess House. 717~p.
   \end{thebibliography}

 }
 }

\end{multicols}

\vspace*{-3pt}

\hfill{\small\textit{Received July 18, 2016}}

\Contrl

\noindent
\textbf{Kolin Konstantin K.} (b.\ 1935)~--- Doctor of Science in technology, professor, Honored scientist 
of RF, principal scientist, Institute of Informatics Problems, Federal Research Center ``Computer Science 
and Control'' of the Russian Academy of Sciences, 44-2~Vavilov Str., Moscow 119333, Russian 
Federation; \mbox{kolinkk@mail.ru}
\label{end\stat}


\renewcommand{\bibname}{\protect\rm Литература}