\def\stat{fedoseev}

\def\tit{К ВОПРОСУ ОБ УМЕНЬШЕНИИ ОБЪЕМА ПОРЦИЙ УЧЕБНОГО МАТЕРИАЛА 
ПРИ~ЭЛЕКТРОННОМ ОБУЧЕНИИ$^*$}

\def\titkol{К вопросу об уменьшении объема порций учебного материала 
при~электронном обучении}

\def\aut{А.\,А.~Федосеев$^1$}

\def\autkol{А.\,А.~Федосеев}

\titel{\tit}{\aut}{\autkol}{\titkol}

\index{Федосеев А.\,А.}
\index{Fedoseev A.\,A.}


{\renewcommand{\thefootnote}{\fnsymbol{footnote}} \footnotetext[1]
{Работа выполнена в~рамках Программы фундаментальных научных исследований 
в~Российской Федерации на долгосрочный период (2013--2020~годы). Тема №\,34.2. 
Когнитивные мультимедиа и~интерактивность в~образовании в~условиях мобильного 
Интернета.}}


\renewcommand{\thefootnote}{\arabic{footnote}}
\footnotetext[1]{Институт проблем информатики Федерального исследовательского
центра <<Информатика и~управление>> Российской академии наук, 
\mbox{a.fedoseev@ipiran.ru}}



\Abst{Предпринята попытка анализа электронного предъявления учебного 
материала как автоматизированного процесса. Проанализированы причины 
сокращения продолжительности видеолекций для массовых открытых 
онлайновых курсов, а также аналогичного сокращения необходимого времени 
работы с~мультимедиа электронными образовательными ресурсами 
и~параграфами электронных учебников. Показано, что причиной для таких 
сокращений является не столько сама продолжительность, сколько объем 
предъявляемого учебного материала, который может быть усвоен за один 
сеанс. Для определения пределов этого объема количество предъявляемой 
информации, измеренное в~новых понятиях и~связанных с~ними уже усвоенных 
по\-ня\-тий-лин\-ков, сравнивается с~предельным количеством элементов, 
обрабатываемых в~оперативной памяти человека одновременно. В~результате 
делается вывод о~том, что за требованием сокращения продолжительности 
лекций стоит необходимое ограничение объема предъявляемой учебной 
информации. Учет этого обстоятельства позволил сформулировать понятие 
комплекта заданий и~сделать предложение относительно процедуры 
автоматизированного обучения. Статья публикуется в~порядке обсуждения.}

\KW{электронные средства обучения; микрообучение; понятиe; линк; 
<<кошелек Миллера>>; комплект заданий; автоматизированное обучение}

  
\DOI{10.14357/19922264160314} 
  
  \vspace*{3pt}


\vskip 10pt plus 9pt minus 6pt

\thispagestyle{headings}

\begin{multicols}{2}

\label{st\stat}
  
\section{Введение}

  Нормальная продолжительность лекции в~российской высшей школе~--- два 
академических часа, что составляет 90~мин. Бывают сдвоенные лекции 
продолжительностью 180~мин. В~других странах примерно то же самое, но 
предполагается некоторое пространство для маневра продолжительностью 
лекции.
  
  Если вузовскую лекцию записать на видео, то, казалось бы, должна 
получиться основа учебного материала для одного из новых Массовых 
открытых онлайновых курсов (МООК), которые приобрели известность 
и~популярность начиная с~2012~г.~[1]. Чтобы лекционная основа 
превратилась в~заготовку такого курса, следует добавить к~ней механизм 
обратной связи для фиксации факта освоения материала лекции слушателями. 
Для этого в~зависимости от целей и~материала курса применяются различные 
методы.
  
  Однако опыт применения МООК и~электронных ресурсов (в~том числе 
электронных учебников) показал, что продолжительные видеолекции или 
параграфы электронных учебников, тре\-бу\-ющие значительного по времени 
внимания, для дистанционного обучения не годятся. Эмпирически оказалось, 
что учебные материалы должны быть сформированы таким образом, чтобы 
работа с~ними не превышала~15~мин, а~лучше~--- еще меньше. Эта ситуация 
породила термин microlearning (мик\-ро\-обуче\-ние)~[2] и~метафоры 
<<информация на один укус>> и~<<внимание на чайную ложку>> 
(в~примерном переводе). При этом материал для микрообучения~--- это не 
порубленные на части длинные лекции, а~самостоятельные, логически полные 
и~связные короткие видеоролики. Таким образом, считается, что для заочного 
обучения, основанного на ин\-фор\-ма\-ци\-он\-но-ком\-му\-ни\-ка\-ци\-он\-ных 
технологиях (ИКТ), новые знания должны подаваться мелкими порциями 
(microlearning), работа с~каждой из которых занимает не более нескольких 
минут. 
  
  Попытки объяснить необходимость предъявления новых знаний мелкими 
порциями сводятся к~соображению, что при более длинных лекциях внимание 
слушателя рассеивается и~материал перестает усваиваться. Так, исследование 
лекций почти пятисот МООК показало~[3], что интерес к~лекции резко 
уменьшается уже на шестой минуте про\-смот\-ра. Не опровергая это 
исследование, попробуем не согласиться с~тем, что именно продолжительность 
лекции является критической в~дистанционном обучении. Внимание студентов 
точно так же рассеивается и~во время очной лекции в~аудитории, однако 
девяностоминутный стандарт существует столетиями, и~ни\-кто не собирается 
изменять его в~угоду непоседливости студентов. И~это при том, что отвлечение 
внимания во время очной лекции чревато полной утерей понимания изложения 
и невозможностью его восстановления, поскольку лекция существует только 
в~момент ее прочтения. Что касается заочных учебных материалов, 
доставляемых средствами ИКТ, потеря внимания не наносит существенного 
вреда слушателю, поскольку существует возможность получить повторно ту 
часть материала, которая не была воспринята с~первой попытки.  
По-ви\-ди\-мо\-му, дело не только в~продолжительности лекции, а в~чем-то еще.
  
\section{Отличие автоматизированного процесса от~<<ручного>>}

  Предъявление слушателю, студенту или учащемуся учебного материала 
с~использованием средств ИКТ является автоматизированным процессом 
в~отличие от изложения материала лектором или учителем. Учитель или 
лектор, излагая материал, самостоятельно определяет необходимую 
продолжительность цикла обучения, так же как и~условия перехода 
к~изложению следующего материала. Тео\-ре\-ти\-че\-ски это означает, что учитель 
должен убедиться в~готовности учащихся к~дальнейшему учению. На практике 
это не всегда возможно. Учитель не в~состоянии опросить всех учащихся, не 
все из них присутствовали на прошлом уроке, кое-кто, возможно, не выполнил 
домашнее задание. Тем не менее учитель хорошо знаком со своими 
подопечными и~в~состоянии принять решение о~моменте, когда можно 
предъявлять следующий учебный материал в~рамках существующего учебного 
плана. Ситуация в~высшей школе несколько более свободна, поскольку 
предполагается (не всегда обоснованно) ответственное отношение студентов 
к~учению.
  
  Если какая-то операция осуществляется автоматизированно, в~данном случае 
это операция предъявления средствами ИКТ учебного материала слушателям 
(учащимся, студентам) для усвоения, то естественно потребовать наличия 
автоматического сигнала о~ее нормальном завершении. Этот принцип удачно 
реализован в~Академии Салмана Хана ({\sf https://www.khanacademy.org}). 
После короткой видеолекции учащемуся предлагается выполнить ряд заданий. 
При возникновении сложностей предусмотрена возможность воспользоваться 
подсказками. Только после того, как все задания выполнены~--- что является 
автоматическим сигналом о~завершении процесса восприятия и~закрепления 
учебного материала, учащемуся становится доступной следующая порция 
видеоматериалов. 
  
  По договору с~Академией Хана студенты Массачусетского технологического 
института должны были готовить учебные видеоматериалы для их 
использования Академией. В~прочитанной для студентов в~2012~г.\ лекции ({\sf 
https://www.youtube. com/watch?v=VA273i3z7Mk\&nohtml5=False}) Салман Хан, 
в~частности, настаивал, что ролики должны быть по возможности короткими. 
Некоторые лекции ему самому удалось сделать трехминутными, но, 
к~сожалению, не все. То же самое относится и~к~заданиям, выполнение 
которых должно засвидетельствовать освоение материала микролекции. 
Задания также должны быть прос\-ты\-ми и~быстровыполнимыми. Как видим, 
подход Академии Хана вполне согласуется с~концепцией микрообучения.
  
  У организаторов дистанционного курса нет иной возможности понять, 
воспринята ли видеолекция, как получить сигнал о~том, что все задания, 
относящиеся к~этой лекции, выполнены. Поскольку МООК относятся 
к~высшему или дополнительному образованию, заранее предполагается 
большая мотивированность и~ответственность слушателей. Что касается 
электронных ресурсов и~учебников для общеобразовательной школы, то там 
ситуация другая. Полагаться на сознательность и~ответственность учащихся не 
приходится. Поэтому и~методы контроля усвоения знаний должны быть более 
строгими. Очевидно, что учитель может существовать (и~на самом деле 
существует) в~условиях, когда часть его учеников знает весь заданный 
материал, кто-то знает его частично, а некоторые вообще ничего не знают. Он 
примерно представляет уровень знаний каждого ученика и, поскольку их не так 
уж много, может своим индивидуальным вниманием (применяя различные 
педагогические приемы, в~том числе формирующее обучение) до некоторой 
степени компенсировать разноуровневость знаний учащихся.
  
  Если предъявление нового учебного материала передается технологиям, то 
ситуация изменяется и~учащихся приходится подгонять (автоматизированно) 
под единый уровень. В противном случае от автоматизации не будет толка. Что 
с того, что ученикам предъявили некий электронный образовательный  
ресурс~--- кто-то посмотрел и~изучил, а~кто-то и~компьютер не включал,~--- 
подчищать все придется учителю вручную.
  
  Таким образом, автоматизация предъявления учебного материала неизбежно 
тянет за собой и~автоматизированный процесс контроля усвоения этого 
материала. Иначе автоматизированный процесс оказывается незавершенным 
и~восприятие учебного материала приходится проверять традиционными 
способами (контрольный опрос учащихся), т.\,е.\ учебный процесс 
возвращается к~своей традиционной форме и~смысл автоматизации пропадает.
  
\section{Как человек воспринимает новую информацию}

  У каждого человека со временем складывается собственная система знаний. 
Как показано в~[4], сначала закладываются первичные элементарные понятия, 
затем по мере поступления новой информации присутствующие в~ней понятия 
(факты, образы, связи, категории~--- все что угодно) выражаются через уже 
имеющиеся, усвоенные элементы, после чего новая информация встраивается 
в~сис\-те\-му знаний. Для объяснения этого процесса все уже устоявшиеся 
элементы, нужные для объяснения нового понятия, в~[4] предлагается называть 
линками (в переводе~--- связями). Таким образом, каждое новое понятие 
оказывается связанным с~некоторым количеством линков и~в~таком виде 
остается в~долговременной памяти человека. Когда изучается новое понятие, 
например закон Ома, линками являются понятия напряжения, тока 
и~сопротивления. Однако, когда закон Ома встроится в~сис\-те\-му знаний, 
в~памяти он будет представлен одним укрупненным понятием, определяющим 
весь закон. Если обладателю знания закона Ома понадобится использовать его 
для формирования ка\-ко\-го-ли\-бо нового понятия, то этот закон будет 
использован как линк. Более крупным по\-ня\-ти\-ем-лин\-ком может быть вся 
электротехника и~даже вся физика. Здесь важно, что при восприятии новой 
информации каждое понятие обрастает соответствующими линками и~таким 
образом укрупняется. Кстати, при изучении закона Ома учащимся потребуется 
старый комплексный линк под названием <<алгебраические преобразования>>, 
иначе никак не справиться с~вычислением тока или, наоборот, напряжения. 
Естественно, что этот линк уже находится в~памяти как единое целое, 
объединяющее все изученные ал\-геб\-ра\-и\-че\-ские преобразования.
  
  В западной литературе, например в~[5], для объяснения аналогичных 
процессов используется понятие чанка (chunk~--- кусок, ломоть). 
  
  Особенности восприятия информации человеком в~соответствии 
с~открытием, сделанным Джорджем Миллером~[6] (остроумно названным 
<<кошелек Миллера>>), не позволяют обрабатывать в~оператив\-ной памяти 
одновременно более семи плюс-ми\-нус двух элементов. Таким образом, 
количество элементов предназначенного для усвоения учебного материала 
должно быть в~пределах семи (плюс-ми\-нус два). В~[7] на большом 
практическом материале показано, что количество элементов, предъявляемых 
в~новом учебном материале, должно быть не менее трех и~не более пяти. Если 
количество таких элементов три и~менее, то интерес слушателей не 
пробуждается, поскольку они воспринимают материал слишком простым 
и~потому не заслуживающим внимания. Если число элементов учебного 
материала более пяти, то интерес учащихся пропадает уже из-за того, что они 
теряют смысл предъявляемого материала и~не понимают его. Очевидно, что 
количество одновременно обрабатываемых элементов в~диапазоне от трех до 
пяти вполне согласуется с~законом Миллера.
  
  Отождествляя элементы обрабатываемой оперативной памятью человека 
информации с~понятиями и~линками, можно заключить, что эффективное 
восприятие может произойти, если их количество в~новом учебном материале 
не менее трех и~не более пяти. Проблема в~том, что в~понятиях и~линках никто 
учебный материал не измеряет. Интуитивно отмечается, что более короткий 
материал усваивается успешнее, причем проверка усвоения осуществляется 
сравнительно просто. Сколько понятий и~связанных с~ними линков можно 
ввести на лекции продолжительностью~3~мин и~сколько~--- за 90~мин? 
И~далее: как убедиться, что весь~90-ми\-нут\-ный материал усвоен? Каков объем 
выполненных заданий должен убедить в~том, что материал на самом деле 
усвоен? 
  
  Таким образом, теперь можно быть уверенным, что дело не только 
в~регулировании продолжительности лекции, но и~в~ограничении объема 
предъявляемой информации.
  
\section{Понятие комплекта заданий}
  Следующим шагом на пути автоматизации учебного процесса становится 
предъявление учащемуся таких заданий, выполнение которых гарантированно 
свидетельствовало бы о~полном усвоении этого материала. Как упомянуто 
ранее, рекомендации на этот счет касаются только количества и~слож\-ности 
заданий: заданий должно быть по возможности много, но они должны быть 
простыми. Попробуем связать это с~теми понятиями и~линками~[4], количество 
которых определяет предельный объем учебного материала. В~примере про 
закон Ома есть новое понятие~--- закон Ома, есть три линка: напряжение, ток 
и~сопротивление~--- и~есть комплексный линк <<алгебраические 
преобразования>>. Всего~--- пять. Никакие параллельные и~последовательные 
соединения, электродвижущая сила и~полное сопротивление цепи, а также удельное 
сопротивление и~поперечное сечение проводника не должны входить 
в~материал о~законе Ома, поскольку это дополнительные понятия и~им место 
в~дальнейших порциях учебного материала, которые будут предъявляться 
позже. 
  
  Какими должны быть задания, чтобы пол\-ностью удостовериться в~усвоении 
материала <<закон Ома>>? Во-пер\-вых, это вопросы на понимание каж\-до\-го 
нового понятия. Новое понятие одно: закон Ома. Во-вто\-рых, вопросы на 
понимание действия каждого уже известного понятия (линка) в~законе Ома. 
Таких понятий три: напряжение, ток и~сопротивление. У алгебраических 
преобразований нет специфических взаимодействий с~законом Ома. Этот линк 
проявит себя, когда понадобится со\-став\-лять формулы для вычислений.  
В-третьих, вопросы на понимание взаимодействия каждой пары величин 
в~рамках изучаемого закона. Таких пар три: ток и~напряжение, ток 
и~сопротивление, напряжение и~сопротивление. В-чет\-вер\-тых, задания на 
вычисление каждой величины при известных двух %\linebreak
 других. Их тоже три.  
В-пя\-тых, определение\linebreak зависимостей каждой величины от изменений двух 
других. Их может быть шесть, если одна из независимых величин является 
аргументом, а~вторая~--- параметром. Итого в~данном примере насчитывается 
минимум~16~заданий пяти различных типов.\linebreak Почему минимум? Потому что 
меньше нельзя: не все аспекты порции знаний будут проверены. 
А~больше~16~заданий вполне может быть. Например, чтобы проверить 
усвоение новых понятий, может понадобиться более одного вопроса на 
понимание для каждого понятия. Задачи на вычисления могут быть 
сформулированы, что называется, <<в~лоб>>, а могут иметь завуалированную 
структуру, с~тем чтобы учащийся догадался, как решить задачу. 
  
  Таким образом, опираясь на использованные в~порции учебного материла 
понятия и~линки, можно сформировать некоторый набор заданий, которыми 
можно проверить усвоение всех аспектов заключенного в~этой порции знания. 
При этом рекомендации оказываются соблюденными: количество заданий 
существенно превышает обычную норму. Так, количество вопросов для 
самопроверки, размещаемых после параграфа с~учебным материалом, как 
правило, не превышает трех--че\-ты\-рех. Число задач, задаваемых на дом, не 
превышает пяти--шести. И~это при том, что количество информации 
в~параграфах учебников не нормировано понятиями и~связанными с~ними 
линками и~существенно превышает рекомендуемый уровень.
  
  Будем этот необходимый объем заданий, определенный по понятиям 
и~линкам нового учебного материала, называть комплектом. Каждой порции 
учебного материала соответствует свой комплект заданий.
  
  Количество необходимых заданий для выявления полного усвоения 
материала~--- еще один довод в~пользу мелких порций электронных лекций, 
электронных образовательных мультимедиаресурсов и~параграфов 
электронных учебников. При увеличении числа понятий и~линков учебного 
материала стремительно возрастает размер необходимого комплекта заданий, 
поскольку необходимо проверять понимание взаимодействия каждого 
с~каждым.
  
\section{Автоматизированная диагностика }

  Поскольку комплект заданий охватывает проверку усвоения всех аспектов 
соответствующего учебного материала, то возникает возможность 
автоматизиро\-ванной диагностики возникших пробелов в~знании. Таким 
образом, практическое применение понятия комплекта заданий позволяет 
осуществлять автоматизированную диагностику пробелов в~знаниях учащихся. 
Поскольку заданиями охвачены все аспекты учебного материала по 
возможности с~исчерпывающей полнотой, невыполнение отдельных заданий 
с~очевидностью указывает на те разделы материала, к~которым они относятся.
  
\section{Автоматизированное обучение}

  Согласно~[8], обучение~--- это процесс, при котором исправляются ошибки 
восприятия учащихся, не позволившие им выполнить задания с~первой\linebreak 
попытки. В~этом смысле обучение является индивидуальным и~итеративным 
процессом, в~котором на каждой итерации в~соответствии с~допущенными 
учащимися ошибками каждый из них должен получить новый учебный 
материал, специально разработанный для устранения его пробелов 
в~восприятии. После проработки этого материала\linebreak вновь совершается попытка 
выполнить задания. Если задания оказываются снова не выполненными  
в~ка\-кой-то части, то должна осуществиться следующая итерация создания 
нового учебного материала и~предъявления его учащемуся для проработки. 
Этот процесс не применяется в~полной мере на практике в~учебных заведениях, 
поскольку в~них не предусмотрено систематических индивидуальных занятий. 
В~ка\-кой-то степени эти положения теории отрабатываются репетиторами или 
родственниками учащихся.
  
  Разработчик образовательного ресурса или электронного учебника, 
доставляемого средствами ИКТ, имеет возможность заранее предусмотреть 
и~заготовить специальные дополнительные учебные материалы, направленные 
на преодоление непонимания в~выполнении заданий. Рассмотрим для примера, 
какие должны быть заготовлены корректирующие материалы по закону Ома. 
Во-пер\-вых, это материалы, направленные на формирование правильного 
понимания каждого из участвующих в~законе Ома линков. Их три. Это 
напряжение, ток и~сопротивление. Во-вто\-рых, должны также присутствовать материалы, 
корректирующие неправильное понимание пар линков. Их тоже три. 
Неправильное понимание самого закона Ома также должно исправляться 
корректирующим материалом. И~наконец, в-третьих, неправильное использование 
алгебраических %\linebreak 
преобразований должно привести к~отсылке ученика 
к~соответствующему разделу математики. Итого получилось семь 
корректирующих материалов. Надо сказать, что в~той или иной степени эти 
разделы излагались в~основном учебном материале. Речь идет только о~более 
подробном их изложении для полной ликвидации неправильного понимания. 
Эти материалы должны быть предъявлены каждому учащемуся соответственно 
до\-пущенным им ошибкам. После проработки специального материала 
учащийся вновь получает соответств\-ующий комплект заданий. При 
выполнении всех заданий делается заключение об усво\-ении материала, а~при 
невыполнении~--- повторяется процедура предъявления специального учебного 
материала, соответствующего новым допущенным ошибкам. По-ви\-ди\-мо\-му, 
для предотвращения возможности бесконечного цикла следует предусмотреть 
общение с~учителем после нескольких циклов с~одними и~теми же типами 
ошибок.
  
  По существу, описанная процедура является не чем иным, как 
автоматизированным обучением. А~это означает, что электронная доставка 
учебных материалов мелкими порциями с~выделением элементов информации, 
включая комплекты заданий, дает возможность не только убедиться в~усвоении 
или неусвоении предъявленного материала, но и~обеспечить приемлемый 
уровень его усвоения всеми учащимися.
  
\section{Заключение}

  Анализ причин сокращения продолжитель\-ности учебных материалов, 
предъявляемых средствами ИКТ, показал, что дело скорее в~объеме 
предъявляемой информации, которая может быть воспринята за один сеанс, 
чем в~продолжительности как таковой. Организаторы МООК, создатели 
электронных образовательных ресурсов и~электронных учебников интуитивно 
уменьшают порции предъявляемой учебной информации, поскольку 
автоматизированный процесс не позволяет откладывать усвоение материала 
<<на потом>>. Введение в~практику электронного обучения регулирования 
количества по\-ня\-тий-лин\-ков и~формирования комплектов заданий может 
позволить довести процедуры электронного обучения до полного усво\-ения 
материала всеми слушателями или учащимися в~темпе учебного процесса и~тем 
самым повысить эффективность и~результативность методов дистанционного 
обуче\-ния.
  
{\small\frenchspacing
 {%\baselineskip=10.8pt
 \addcontentsline{toc}{section}{References}
 \begin{thebibliography}{9}
\bibitem{1-fed}
\Au{Богданова Д.\,А.} Большой прорыв: от открытых образовательных 
ресурсов~--- к~Массовым Открытым Онлайновым Курсам~// Дистанционное 
и~виртуальное обучение, 2013. №\,4. С.~35--47.
\bibitem{2-fed}
\Au{Fernandez J.} The microlearning trend: Accommodating cultural and 
cognitive shifts~// Learning Solutions Magazine, 2014. December~1. {\sf 
http:// www.learningsolutionsmag.com/articles/1578/the-microlearning-trend-accommodating-cultural-and-cognitive-shifts}.
\bibitem{3-fed}
\Au{Guo P.\,J., Kim J., Rubin~R.} How video production affects student 
engagement: An empirical study of MOOC videos.~--- MIT Computer Science 
and Artificial Intelligence Laboratory, 2014. 10~p. {\sf 
https://groups.csail.mit.edu/uid/other-pubs/las2014-pguo-engagement.pdf}.
\bibitem{4-fed}
\Au{Карпенко М.\,П.} Телеобучение.~--- М.: СГА, 2008. 800~с.
\bibitem{5-fed}
\Au{Chase W.\,G., Simon H.\,A.} Perception in chess~// Cognitive Psychol., 
1973. No.\,4. P.~55--61.
\bibitem{6-fed}
\Au{Miller G.\,A.} The magical number seven, plus or minus two: Some limits on 
our capacity for processing information~// Psychol. Rev., 1956. Vol.~63. 
P.~81--97.
\bibitem{7-fed}
\Au{Петрова В.} Метод 3-4-5, чтобы все запоминать! Освойте новую 
технологию запоминания.~--- Montreal: Accent Graphics Communications, 
2014. 169~с.
\bibitem{8-fed}
\Au{Писарев В.\,Е., Писарева Т.\,Е.} Теория педагогики.~--- Воронеж: 
Кварта, 2009. 611~c.
\end{thebibliography}

 }
 }

\end{multicols}

\vspace*{-6pt}

\hfill{\small\textit{Поступила в~редакцию 27.04.16}}

%\vspace*{8pt}

\newpage

\vspace*{-24pt}

%\hrule

%\vspace*{2pt}

%\hrule

%\vspace*{8pt}



\def\tit{WHAT IS BEHIND THE~CONCEPT OF~``KNOWLEDGE IN~SMALL 
PACKAGES''}

\def\titkol{What is behind the~concept of~``knowledge in small 
packages''}

\def\aut{A.\,A.~Fedoseev}

\def\autkol{A.\,A.~Fedoseev}

\titel{\tit}{\aut}{\autkol}{\titkol}

\vspace*{-9pt}

\noindent
Institute of Informatics Problems, Federal Research Center 
``Computer Science and Control'' of the Russian\linebreak
Academy of Sciences,
44-2~Vavilov Str., Moscow 119333, Russian Federation


\def\leftfootline{\small{\textbf{\thepage}
\hfill INFORMATIKA I EE PRIMENENIYA~--- INFORMATICS AND
APPLICATIONS\ \ \ 2016\ \ \ volume~10\ \ \ issue\ 3}
}%
 \def\rightfootline{\small{INFORMATIKA I EE PRIMENENIYA~---
INFORMATICS AND APPLICATIONS\ \ \ 2016\ \ \ volume~10\ \ \ issue\ 3
\hfill \textbf{\thepage}}}

\vspace*{3pt}


\Abste{An attempt has been made to analyze the electronic presentation of 
educational material as automated process. The reasons for the reduction of the 
length of video lectures for massive open online courses and other educational 
electronic resources as well as the requirement of reducing the time required for 
multimedia electronic educational resources and paragraphs of electronic tutor 
books are analyzed. It is shown that the reason for these requirements is not the 
resource duration, but rather the volume of the educational material that can be 
learned in one session. To determine the limits of this volume the amount of 
presented information, measured in terms of new concepts and related already 
learned concepts-links, was compared with the limited number of items processed 
in the human memory simultaneously. As a result, it is concluded that the 
requirement of reducing the length of lectures is necessary to limit the scope of the 
educational information presented. This circumstance has made it possible to 
formulate the concept of a~task set and to make a proposal for automated training 
procedures. The paper is published in order to discuss this problem.}

\KWE{e-learning tools; microlearning; concept; link; ``Miller purse;'' task set; 
automated training}


\DOI{10.14357/19922264160314} 

\vspace*{-9pt}

\Ack
\noindent
The work was done under the Program of Fundamental Scientific Research 
in the Russian Federation for the long term (2013--2020). 
Subject No.\,34.2. Cognitive multimedia and interactivity in education 
in the conditions of mobile Internet.

%Работа выполнена в~рамках Программы фундаментальных научных исследований 
%в~Российской Федерации на долгосрочный период (2013--2020~годы). Тема №\,34.2. 
%Когнитивные мультимедиа и~интерактивность в~образовании в~условиях мобильного 
%интернета.


%\vspace*{3pt}

  \begin{multicols}{2}

\renewcommand{\bibname}{\protect\rmfamily References}
%\renewcommand{\bibname}{\large\protect\rm References}

{\small\frenchspacing
 {%\baselineskip=10.8pt
 \addcontentsline{toc}{section}{References}
 \begin{thebibliography}{9}
\bibitem{1-fed-1}
\Aue{Bogdanova, D.\,A.} 2013. Bol'shoy proryv: Ot otkrytykh obrazovatel'nykh 
resursov~--- k Massovym Otkrytym Onlaynovym Kursam [The big breakthrough: 
From open educational resources~--- to massive open online courses]. 
\textit{Distantsionnoe i~virtual'noe obuchenie} [Distance and Virtual Learning]  
4:35--47.
\bibitem{2-fed-1}
\Aue{Fernandez, J.} December 1, 2014. The microlearning trend: Accommodating cultural and 
cognitive shifts. \textit{Learning Solutions Magazine}.  Available at: 
{\sf http://www. learningsolutionsmag.com/articles/1578/the-microlearning-trend-accommodating-cultural-and-cognitive-shifts} (accessed March~31, 2016).
\bibitem{3-fed-1}
\Aue{Guo, P.\,J., J.~Kim, and R.~Rubin}. 2014. How video production affects 
student engagement: An empirical study of MOOC videos. MIT Computer 
Science and Artificial Intelligence Laboratory. 10~p. Available at: {\sf 
https://groups.csail.mit.edu/uid/other-pubs/las2014-pguo-engagement.pdf} 
(accessed March~31, 2016).
\bibitem{4-fed-1}
\Aue{Karpenko, M.\,P.} 2008. \textit{Teleobuchenie} [Teleeducation]. Moscow: 
SGA. 800~p.
\bibitem{5-fed-1}
\Aue{Chase, W.\,G., and H.\,A.~Simon}. 1973. Perception in chess. 
\textit{Cognitive Psychol.} (4):55--61.
\bibitem{6-fed-1}
\Au{Miller, G.\,A.} 1956. The magical number seven, plus or minus two: Some 
limits on our capacity for processing information. \textit{Psychol. Rev.} 
63:81--97.
\bibitem{7-fed-1}
\Aue{Petrova, V.} 2014. \textit{Metod 3-4-5, chtoby vse zapominat'!\ Osvoy\-te 
novuyu tekhnologiyu zapominaniya} [Method 3-4-5 to remember everything! 
Master the new memory technology]. Montreal: Accent Graphics 
Communications. 169~p.
\bibitem{8-fed-1}
\Aue{Pisarev, V.\,E., and T.\,E. Pisareva}. 2009. \textit{Teoriya pedagogiki} 
[Pedagogy theory]. Voronezh: KVARTA.  611~p.
   \end{thebibliography}

 }
 }

\end{multicols}

\vspace*{-3pt}

\hfill{\small\textit{Received April 27, 2016}}
  
  \Contrl
  
 \noindent
  \textbf{Fedoseev Andrei A.}\ (b.\ 1946)~--- Candidate of Science (PhD) in
technology, leading scientist,  Institute of Informatics Problems, Federal Research 
Center ``Computer Science and Control'' of the Russian Academy of  
Sciences,~44-2~Vavilova Str., Moscow 119333, Russian Federation; 
\mbox{a.fedoseev@ipiran.ru}

  
\label{end\stat}


\renewcommand{\bibname}{\protect\rm Литература}