\def\stat{ushakov+leon}

\def\tit{АНАЛИЗ СИСТЕМЫ ОБСЛУЖИВАНИЯ С~ВХОДЯЩИМ ПОТОКОМ АВТОРЕГРЕССИОННОГО 
ТИПА И~ОТНОСИТЕЛЬНЫМ ПРИОРИТЕТОМ$^*$}

\def\titkol{Анализ системы обслуживания с~входящим потоком авторегрессионного 
типа и~относительным приоритетом}

\def\aut{Н.\,Д.~Леонтьев$^1$, В.\,Г.~Ушаков$^2$}

\def\autkol{Н.\,Д.~Леонтьев, В.\,Г.~Ушаков}

\titel{\tit}{\aut}{\autkol}{\titkol}

\index{Леонтьев Н.\,Д.}
\index{Ушаков В.\,Г.}
\index{Leontyev N.\,D.}
\index{Ushakov V.\,G.}


{\renewcommand{\thefootnote}{\fnsymbol{footnote}} \footnotetext[1]
{Работа выполнена при финансовой поддержке РФФИ (проект 15-07-02354).}}


\renewcommand{\thefootnote}{\arabic{footnote}}
\footnotetext[1]{Факультет вычислительной математики и~кибернетики Московского 
государственного университета им.\ М.\,В.~Ломоносова, \mbox{ndleontyev@gmail.com}}
\footnotetext[2]{Факультет вычислительной математики и~кибернетики 
Московского государственного университета им.\ М.\,В.~Ломоносова; 
Институт проблем информатики ФИЦ ИУ РАН, \mbox{vgushakov@mail.ru}}

\vspace*{-12pt}

\Abst{Рассматривается одноканальная система массового обслуживания 
с~неограниченным числом мест для ожидания, в~которую поступают два потока требований: 
первый поток~--- пуассоновский, а~второй~--- неординарный пуассоновский (т.\,е.\ 
пуассоновский поток групп требований). Требования из первого потока имеют 
относительный приоритет перед требованиями второго потока. Особенностью системы 
является авторегрессионная зависимость размеров групп требований второго потока: 
размер $n$-й поступившей в~систему группы требований либо с~некоторой 
фиксированной вероятностью равен размеру $(n-1)$-й поступившей в~систему группы 
требований, либо с~дополнительной вероятностью является независимой от него 
случайной величиной. Длительности обслуживания требований каждого потока являются 
независимыми случайными величинами с~произвольным распределением. Найдена 
производящая функция совместного распределения числа требований каждого 
потока в~системе в~произвольный момент времени.}


\KW{теория массового обслуживания; нестационарный режим; системы с~групповым 
поступлением требований; относительный приоритет}

\DOI{10.14357/19922264160303} 

%\vspace*{-3pt}
  

\vskip 12pt plus 9pt minus 6pt

\thispagestyle{headings}

\begin{multicols}{2}

\label{st\stat}

\section{Введение}

При моделировании передачи данных в~телекоммуникационных сетях важно учитывать 
природу потоков информации и~характеристики потоков в~зависимости от приложений. 
В~простейших моделях предполагается, что все пакеты информации имеют фиксированную 
конечную длину, а~размеры пакетов независимы. Однако в~ряде случаев необходимо 
рассматривать более сложные конструкции входящего потока, которые позволяют 
учитывать неоднородную и~коррелированную природу потоков данных. 
В~данной работе рассматривается сис\-те\-ма массового обслуживания типа $M|G|1$ 
с~двумя потоками требований, один из которых является приоритетным, 
а~другой~--- коррелированным потоком групп требований.

Настоящая работа обобщает результаты \mbox{статьи}~[1] на случай двух потоков, 
требования одного из которых имеют относительный приоритет.

\vspace*{-9pt}

\section{Описание системы}

Рассматривается одноканальная система обслуживания, в~которую поступают 
два потока требований: 
\begin{itemize}
\item первый поток~--- пуассоновский с~интенсив\-ностью~$a_1$, 
\item второй поток является пуассоновским потоком групп требований.
\end{itemize}
Интенсивность 
поступления групп требований равна~$a_2$. Группа состоит из случайного числа 
требований и~содержит~$k$ требований с~вероятностью~$h_k$, $k\hm=1,\ldots,M$. 
Между размерами двух последо\-ва\-тельно по\-сту\-па\-ющих групп требований имеется 
следующая зависимость: размер $n$-й по\-сту\-па\-ющей группы требований либо 
с~ве\-ро\-ят\-ностью $0\hm\leq p\hm<1$ равен размеру $(n-1)$-й группы, либо 
с~ве\-ро\-ят\-ностью $1-p$ является независимой от него случайной величиной. Требования 
из первого потока имеют относительный приоритет по отношению к~требованиям из 
второго потока. Иными словами, прерывание уже начатого обслуживания не допускается 
и~требования из второго потока могут поступать на обслуживание только при отсутствии 
в~очереди требований из первого по\-тока. 
%
Будем считать, что число мест для ожидания 
не ограничено, а~длительность обслуживания требований $i$-го потока имеет функцию 
распределения $B_i(x)$ с~плотностью $b_i(x)$ и~преобразованием Лапласа~$\beta_i(s)$.

\pagebreak

Определим следующие случайные процессы:
\begin{description}
\item[\,] $I(t)$ --- номер потока, требование из которого обслуживается в~момент~$t$;
\item[\,]$L_i(t)$ --- число требований $i$-го потока в~системе в~момент времени~$t$;
\item[\,]$X(t)$ --- время, прошедшее с~начала обслуживания требования, находящегося на 
обслуживании в~момент~$t$\footnote{В~случае, когда система свободна, можно 
для определенности положить $I(t)\hm=0$ и~$X(t)\hm=0$.};
\item[\,]$N(t)$ --- размер последней поступившей в~систему до момента~$t$ 
группы требований $2$-го потока.
\end{description}
Положим
\begin{align*}
P_i(n_1,n_2,k,x,t)&=\fr{\partial}{\partial x}
\mathbf{P}\left(I(t)=i,L_1(t)=n_1,\right.\\
&\hspace*{14pt}\left.L_2(t)=n_2,N(t)=k,X(t)<x\right);\\
P(0,k,t)&=\mathbf{P}(L_1(t)=0,L_2(t)=0,N(t)=k).\hspace*{-2pt}
\end{align*}

Обозначим
\begin{align*}
\pi_1(z_1,z_2,k,x,s)&=\\
&\hspace*{-46pt}{}=\sum\limits_{n_1=1}^{\infty}z_1^{n_1}
\sum\limits_{n_2=0}^{\infty}z_2^{n_2}
\int\limits_0^{\infty}e^{-st}P_1\left(n_1,n_2,k,x,t\right)\,dt\,;\\
\pi_2(z_1,z_2,k,x,s)&=\\
&\hspace*{-47pt}{}=\sum\limits_{n_1=0}^{\infty}z_1^{n_1}
\sum\limits_{n_2=1}^{\infty}z_2^{n_2}
\int\limits_0^{\infty}e^{-st}P_2\left(n_1,n_2,k,x,t\right)\,dt\,;\\
\pi_0(k,s)&=\int\limits_0^{\infty}e^{-st} P(0,k,t)\,dt
\end{align*}
при $|z_1|\leq1$, $|z_2|\hm\leq1$, $\mathrm{Re}\,(s)\hm>0$.

\section{Основные результаты}

Соотношения для определения $\pi_i(z_1,z_2,k,x,s)$  содержатся в~следующей 
основной теореме.

\smallskip

\noindent
\textbf{Теорема.}\ \textit{Функции $\pi_i(z_1,z_2,k,x,s)$ при $|z_1|\hm<1$,
$|z_2|<1$, $k=1,\ldots,M$, $x\geq 0$, $\mathrm{Re}\,(s)>0$ определяются по
следующим формулам}:
\begin{multline*}
\pi_i(z_1,z_2,k,x,s)=(1-B_i(x))\sum\limits_{n=1}^{M}C_{i,n}(z_1,z_2,s)\times{}\\
{}\times\fr{(1-p)h_k a_2 z_2^k}{\widetilde{\lambda}_n(z_2)-pa_2 z_2^k}
\exp\left(-\left(
\vphantom{\widetilde{\lambda}_n(z_2)}
s+a_1-a_1 z_1+a_2-{}\right.\right.\\
\left.\left.{}-\widetilde{\lambda}_n(z_2)\right)x\right)\,,
\end{multline*}
\textit{где}
\begin{multline*}
C_{1,n}\left(z_1,z_2,s\right)={}\\
{}=\fr{1}{1-z_1^{-1}\beta_1
\left(s+a_1-a_1 z_1+a_2-\widetilde{\lambda}_n(z_2)\right)}\times{}\\
{}\times\fr{\prod\limits_{i=1}^M(\widetilde{\lambda}_n\left(z_2\right)-pa_2 z_2^i)}
{\prod\limits_{\substack{j=1\\j\neq n}}^M(\widetilde{\lambda}_n(z_2)-
\widetilde{\lambda}_j(z_2))}
\sum\limits_{m=1}^M\fr{1}{\widetilde{\lambda}_n(z_2)-pa_2 z_2^m}\times{}\\
{}\times\left(
\vphantom{\fr{1-\tilde{\Lambda}_2^{-1}}{1-z_2^{-1}}}
b_m\left(z_1,z_2,s\right)-{}\right.\\
{}-\fr{1-z_2^{-1}\beta_2\left(s+a_1-a_1 z_1+a_2-
\widetilde{\lambda}_n(z_2)\right)}
{1-z_2^{-1}\beta_2(s+a_1-a_1 z_{1,n}(z_2,s)+a_2-\widetilde{\lambda}_n(z_2))}\times{}\\
\left.{}\times b_m\left(z_{1,n}\left(z_2,s\right),z_2,s\right)\right)\,;
\end{multline*}

\vspace*{-12pt}

\noindent
\begin{multline*}
C_{2,n}\left(z_1,z_2,s\right)={}\\
{}=\fr{1}{1-z_2^{-1}\beta_2(s+a_1-a_1 z_{1,n}
(z_2,s)+a_2-\widetilde{\lambda}_n(z_2))}\times{}\\
{}\times\fr{\prod\limits_{i=1}^M(\widetilde{\lambda}_n(z_2)-pa_2 z_2^i)}
{\prod\limits_{\substack{j=1\\j\neq n}}^M\left(\widetilde{\lambda}_n(z_2)-
\widetilde{\lambda}_j\left(z_2\right)\right)}
\sum\limits_{m=1}^M\fr{b_m(z_{1,n}(z_2,s),z_2,s)}
{\widetilde{\lambda}_n(z_2)-pa_2 z_2^m}\,;\hspace*{-0.39255pt}
\end{multline*}

\vspace*{-9pt}

\noindent
\begin{multline*}
\!\!\!\!b_m\!\left(z_1,z_2,s\right)=-\left(s+a_1-a_1 z_1+a_2\right)\!\pi_0(m,s)+h_m+{}\\
{}+\left[p\pi_0
(m,s)a_2 z_2^m+(1-p)\sum\limits_{n=1}^{M}\pi_0(n,s)h_m a_2 z_2^m\right]\,;
\end{multline*}
$\widetilde{\lambda}_1(z_2),\ldots,\widetilde{\lambda}_M(z_2)$ 
определяются из уравнения
\begin{multline*}
\prod\limits_{i=1}^M\left(pa_2 z_2^i-\widetilde{\lambda}\right)+{}\\
{}+
\sum\limits_{i=1}^M \left(1-p\right) h_i a_2 z_2^i
\prod\limits_{\substack{j=1\\j\neq i}}^M
\left(pa_2 z_2^j-\widetilde{\lambda}\right)=0\,,
\end{multline*}
а $\pi_0(1,s),\ldots,\pi_0(M,s)$~--- из системы
\begin{multline*}
\sum\limits_{m=1}^M
\prod\limits_{\substack{j=1\\
j\neq m}}^M(\widetilde{\lambda}_n-pa_2 z_{2,n}^j)
\left(
\vphantom{\tilde{\lambda}_n}
-\left(s+a_1-a_1 z_{1,n}+{}\right.\right.\\
\left.\left.{}+a_2\right)
\pi_0(m,s)+h_m+\widetilde{\lambda}_n(z_{2,n})\pi_0(m,s)\right)=0\,,
\end{multline*}
\textit{в которой $z_{1,n}\hm=z_{1,n}(z_{2,n},s)$ и~$z_{1,n}(z_2,s)$~--- 
решение функционального уравнения}
\begin{equation*}
z_{1,n}=\beta_1\left(s+a_1-a_1 z_{1,n}+a_2-\widetilde{\lambda}_n\left(z_2\right)\right)\,,
\end{equation*}
\textit{а $z_{2,n}=z_{2,n}(s)$~--- решение функционального урав\-нения}
\begin{equation*}
z_{2,n}=\beta_2\left(s+a_1-a_1 z_{1,n}(z_{2,n},s)+a_2-\widetilde{\lambda}_n
\left(z_{2,n}\right)\right)\,.
\end{equation*}

\noindent
Д\,о\,к\,а\,з\,а\,т\,е\,л\,ь\,с\,т\,в\,о\,.\ \
 Функции $P_i(n_1,n_2,k,x,t)$ и~$P(0,k,t)$ удовлетворяют соотношениям:
\begin{multline}
P_1\left(n_1,n_2,k,x+\Delta,t+\Delta\right)={}\\[3pt]
{}=P_1\left(n_1,n_2,k,x,t\right)
\left[1-(a_1+a_2+\eta_1(x))\Delta\right]+{}\\[3pt]
{}+\mathbf{1}_{\{n_1>1\}}P_1\left(n_1-1,n_2,k,x,t\right)a_1\Delta+{}\\[3pt]
{}+\mathbf{1}_{\{n_2\geq k\}}\left[
\vphantom{\sum\limits_{m=1}^{M}}
pP_1\left(n_1,n_2-k,k,x,t\right)a_2\Delta+{}\right.\\
\left.{}+(1-p)
\sum\limits_{m=1}^{M}P_1(n_1,n_2-k,m,x,t)h_k a_2\Delta\right]\,;\label{before1}
\end{multline}

\vspace*{-10pt}

\noindent
\begin{multline}
P_2\left(n_1,n_2,k,x+\Delta,t+\Delta\right)={}\\[3pt]
{}=
P_2\left(n_1,n_2,k,x,t\right)\left[1-(a_1+a_2+\eta_2(x))\Delta\right]+{}\\[3pt]
{}+\mathbf{1}_{\{n_1\geq 1\}}P_2\left(n_1-1,n_2,k,x,t\right)a_1\Delta+{}\\[3pt]
{}+\mathbf{1}_{\{n_2>k\}}\left[
\vphantom{\sum\limits_{m=1}^{M}}
pP_2(n_1,n_2-k,k,x,t)a_2\Delta+{}\right.\\
\left.{}+(1-p)
\sum\limits_{m=1}^{M}P_2\left(n_1,n_2-k,m,x,t\right)h_k a_2\Delta\right]\,;
\label{before2}
\end{multline}

\vspace*{-10pt}

\noindent
\begin{multline}
P\left(0,k,t+\Delta\right)=
P(0,k,t)\left[1-\left(a_1+a_2\right)\Delta\right]+{}\\
{}+
\int\limits_0^{\infty}P_1(1,0,k,x,t)\eta_1(x)\,dx\Delta+{}\\
{}+\int\limits_0^{\infty}P_2(0,1,k,x,t)\eta_2(x)\,dx\Delta\,;
\label{before3}
\end{multline}

\vspace*{-10pt}

\noindent
\begin{multline}
\int\limits_0^\Delta P_1\left(n_1,n_2,k,u,t+\Delta\right)\,du={}\\
{}=
\int\limits_0^{\infty}P_1\left(n_1+1,n_2,k,x,t\right)\eta_1(x)\,dx\Delta+{}\\
{}+\int\limits_0^{\infty}P_2\left(n_1,n_2+1,k,x,t\right)\eta_2(x)\,dx\Delta+{}\\
{}+
\delta_{n_1,1}\delta_{n_2,0}P(0,k,t)a_1\Delta\,;
\label{before4}
\end{multline}

%\vspace*{-12pt}

\noindent
\begin{multline}
\int\limits_0^\Delta P_2\left(0,n_2,k,u,t+\Delta\right)\,du={}\\
{}=
\int\limits_0^{\infty}P_1\left(1,n_2,k,x,t\right)\eta_1(x)\,dx\Delta+{}
\\
{}+\int\limits_0^{\infty}P_2\left(0,n_2+1,k,x,t\right)\eta_2(x)\,dx\Delta+{}\\
{}+
\delta_{n_2,k}\left[
\vphantom{\sum\limits_{m=1}^{M}}
pP(0,k,t)a_2\Delta+{}\right.\\
\left.{}+(1-p)\sum\limits_{m=1}^{M}
P(0,m,t)h_k a_2\Delta\right]\,,
\label{before5}
\end{multline}
где $\eta_i(x)=b_i(x)/(1-B_i(x))$.

Переходя к~пределу при $\Delta\to 0$ в~(\ref{before1})---(\ref{before5}), имеем:
\begin{multline}
\fr{\partial P_1(n_1,n_2,k,x,t)}{\partial t}+
\fr{\partial P_1(n_1,n_2,k,x,t)}{\partial x}={}\\[3pt]
{}=
-\left(a_1+a_2+\eta_1(x)\right)P_1\left(n_1,n_2,k,x,t\right)+{}\\[3pt]
{}+\mathbf{1}_{\{n_1>1\}}P_1\left(n_1-1,n_2,k,x,t\right)a_1+{}\\[3pt]
{}+\mathbf{1}_{\{n_2\geq k\}}\left[
\vphantom{\sum\limits_{m=1}^{M}}
pP_1\left(n_1,n_2-k,k,x,t\right)a_2+{}\right.\\
\left.{}+(1-p)\sum\limits_{m=1}^{M}
P_1\left(n_1,n_2-k,m,x,t\right)h_k a_2\right]\,;
\label{after1}
\end{multline}

\vspace*{-10pt}

\noindent
\begin{multline}
\fr{\partial P_2(n_1,n_2,k,x,t)}{\partial t}+
\fr{\partial P_2(n_1,n_2,k,x,t)}{\partial x}={}\\[3pt]
{}=
-\left(a_1+a_2+\eta_2(x)\right)P_2(n_1,n_2,k,x,t)+{}\\[3pt]
{}+\mathbf{1}_{\{n_1\geq 1\}}P_2\left(n_1-1,n_2,k,x,t\right)a_1+{}\\[3pt]
{}+\mathbf{1}_{\{n_2>k\}}\left[
\vphantom{\sum\limits_{m=1}^{M}}
pP_2\left(n_1,n_2-k,k,x,t\right)a_2+{}\right.\\
\left.{}+(1-p)\sum\limits_{m=1}^{M}
P_2\left(n_1,n_2-k,m,x,t\right)h_k a_2\right]\,;
\label{after2}
\end{multline}

\vspace*{-12pt}

\noindent
\begin{multline}
\fr{\partial P(0,k,t)}{\partial t}=
-\left(a_1+a_2\right)P(0,k,t)+{}\\
{}+\int\limits_0^{\infty}P_1(1,0,k,x,t)\eta_1(x)\,dx+{}\\
{}+\int\limits_0^{\infty}P_2(0,1,k,x,t)\eta_2(x)\,dx\,;
\label{after3}
\end{multline}

%\vspace*{-12pt}

\noindent
\begin{multline}
P_1\left(n_1,n_2,k,0,t\right)={}\\
{}=
\int\limits_0^{\infty}P_1\left(n_1+1,n_2,k,x,t\right)\eta_1(x)\,dx+{}\\
{}+\int\limits_0^{\infty}P_2\left(n_1,n_2+1,k,x,t\right)\eta_2(x)\,dx+{}\\
{}+
\delta_{n_1,1}\delta_{n_2,0}P(0,k,t)a_1\,;
\label{after4}
\end{multline}

\vspace*{-3pt}

\noindent
\begin{equation}
P_2\left(n_1,n_2,k,0,t\right)=0\,,\enskip n_1>0\,;
\label{after5}
\end{equation}

\vspace*{-12pt}

\noindent
\begin{multline}
P_2\left(0,n_2,k,0,t\right)=\int\limits_0^{\infty}P_1(1,n_2,k,x,t)\eta_1(x)\,dx+{}\\
{}+\int\limits_0^{\infty}P_2\left(0,n_2+1,k,x,t\right)
\eta_2(x)\,dx+{}\\
{}+\delta_{n_2,k}\left[
\vphantom{\sum\limits_{m=1}^{M}}
pP(0,k,t)a_2+{}\right.\\
\left.{}+(1-p)
\sum\limits_{m=1}^{M}P(0,m,t)h_k a_2\right].\label{after6}
\end{multline}

Переходя в~уравнениях (\ref{after1})---(\ref{after6}) 
к~производящим функциям и~преобразованиям Лапласа по~$t$, получим:
\begin{multline*}
\fr{\partial \pi_i(z_1,z_2,k,x,s)}{\partial x}={}\\
{}=
-\left(s+a_1-a_1 z_1+a_2+\eta_i(x)\right)\pi_i\left(z_1,z_2,k,x,s\right)+{}\\
{}+\left[
\vphantom{\sum\limits_{m=1}^{M}}
p\pi_i\left(z_1,z_2,k,x,s\right)
a_2 z_2^k+{}\right.\\
\left.{}+(1-p)\sum\limits_{m=1}^{M}
\pi_i\left(z_1,z_2,m,x,s\right)h_k a_2 z_2^k\right]\,;
%\label{first}
\end{multline*}

\vspace*{-12pt}

\noindent
\begin{multline}
\left(s+a_1+a_2\right)\pi_0(k,s)-h_k={}\\
{}=
\int\limits_0^{\infty}\int\limits_0^{\infty}P_1(1,0,k,x,t)\eta_1(x)\,dx e^{-st}\,dt+{}\\
{}+\int\limits_0^{\infty}\int\limits_0^{\infty}P_2(0,1,k,x,t)\eta_2(x)\,dx e^{-st}\,dt\,;
\label{second}
\end{multline}

\vspace*{-12pt}

\noindent
\begin{multline*}
\pi_1\left(z_1,z_2,k,0,s\right)=
z_1^{-1}\hspace*{-4pt}\int\limits_0^{\infty}
\hspace*{-4pt}\pi_1\left(z_1,z_2,k,x,s\right)\eta_1(x)\,dx-{}\hspace*{-0.78543pt}\\
{}-\sum\limits_{n_2=0}^{\infty}z_2^{n_2}
\int\limits_0^{\infty}\int\limits_0^{\infty}P_1(1,n_2,k,x,t)\eta_1(x)\,dx e^{-st}\,dt+{}\\
{}+
z_2^{-1}\int\limits_0^{\infty}\pi_2\left(z_1,z_2,k,x,s\right)\eta_2(x)\,dx-{}
\end{multline*}

\noindent
\begin{multline}
{}-z_2^{-1}\sum\limits_{n_2=1}^{\infty}\hspace*{-3pt}z_2^{n_2}
\int\limits_0^{\infty}\int\limits_0^{\infty}\!P_2(0,n_2,k,x,t)\eta_2(x)\,dx e^{-st}\,dt+{}\\
{}+
\pi_0(k,s)az_1\,;
\label{third}
\end{multline}

\vspace*{-9pt}

\noindent
\begin{equation}
\sum\limits_{n_1=1}^{\infty}z_1^{n_1}
\sum\limits_{n_2=1}^{\infty}z_2^{n_2}\int\limits_0^{\infty}
P_2\left(n_1,n_2,k,0,t\right)e^{-st}\,dt=0\,;
\label{fourth}
\end{equation}

\vspace*{-12pt}

\noindent
\begin{multline}
\sum\limits_{n_2=1}^{\infty}z_2^{n_2}
\int\limits_0^{\infty}P_2\left(0,n_2,k,0,t\right)e^{-st}\,dt={}\\
{}=
\sum\limits_{n_2=1}^{\infty}z_2^{n_2}\int\limits_0^{\infty}
\int\limits_0^{\infty}P_1(1,n_2,k,x,t)\eta_1(x)\,dxe^{-st}\,dt+{}\\
{}+z_2^{-1}\sum\limits_{n_2=1}^{\infty}z_2^{n_2}
\int\limits_0^{\infty}\int\limits_0^{\infty}P_2(0,n_2,k,x,t)\eta_2(x)\,dxe^{-st}\,dt-{}\\
{}-
\int\limits_0^{\infty}\int\limits_0^{\infty}P_2(0,1,k,x,t)\eta_2(x)\,dxe^{-st}\,dt+{}\\
\!\!\!{}+\left[p\pi_0(k,s)a_2 z_2^{k}+(1-p)\!
\sum\limits_{m=1}^M\!\pi_0(m,s)h_k a_2 z_2^k\right].\!\!\!
\label{fifth}
\end{multline}

Суммируя уравнения (\ref{second})---(\ref{fifth}), приходим к~сис\-те\-ме:
\begin{multline}
\fr{\partial \pi_i(z_1,z_2,k,x,s)}{\partial x}
={}\\
{}= -\left(s+a_1-a_1 z_1+a_2+\eta_i(x)\right)\pi_i\left(z_1,z_2,k,x,s\right)+{}\\
{}+\left[
\vphantom{\sum\limits_{m=1}^{M}}
p\pi_i\left(z_1,z_2,k,x,s\right)a_2 z_2^k+{}\right.\\
\left.{}+(1-p)\sum\limits_{m=1}^{M}
\pi_i\left(z_1,z_2,m,x,s\right)h_k a_2 z_2^k\right]\,;
\label{newfirst}
\end{multline}

\vspace*{-12pt}

\noindent
\begin{multline}
\pi_1\left(z_1,z_2,k,0,s\right)+
\pi_2\left(z_1,z_2,k,0,s\right)={}\\
{}=
z_1^{-1}\int\limits_0^{\infty}\pi_1\left(z_1,z_2,k,x,s\right)\eta_1(x)\,dx+{}\\
{}+z_2^{-1}\int\limits_0^{\infty}\pi_2(z_1,z_2,k,x,s)\eta_2(x)\,dx-{}\\
{}-
\left(s+a_1-a_1 z_1+a_2\right)\pi_0(k,s)+h_k+{}\\
\!\!\!{}+\left[p\pi_0(k,s)a_2 z_2^k+(1-p)\!\sum\limits_{m=1}^{M}\!
\pi_0(m,s)h_k a_2 z_2^k\right].\!\!
\label{newsecond}
\end{multline}

\vspace*{-12pt}

\pagebreak

Обозначим
$$
\pi_i\left(z_1,z_2,k,x,s\right)=\left(1-B_i(x)\right)\widetilde{\pi}_i\left(z_1,z_2,k,x,s\right).
$$
В новых обозначениях~(\ref{newfirst}) примет вид:
\begin{multline}
\fr{\partial\widetilde{\pi}_i(z_1,z_2,k,x,s)}{\partial x}={}\\
{}=
-\left(s+a_1-a_1 z_1+a_2\right)\widetilde{\pi}_i\left(z_1,z_2,k,x,s\right)+{}\\
{}+\left[
\sum\limits_{m=1}^{M}
p\widetilde{\pi}_i\left(z_1,z_2,k,x,s\right)a_2 z_2^k+{}\right.\\
\left.{}+(1-p)
\sum\limits_{m=1}^{M}\widetilde{\pi}_i\left(z_1,z_2,m,x,s\right)
h_k a_2 z_2^k\right].
\label{system}
\end{multline}
Это линейная система дифференциальных уравнений первого порядка 
с~постоянными коэффициентами, решение которой можно записать в~виде:
\begin{multline}
\widetilde{\pi}_i\left(z_1,z_2,k,x,s\right)={}\\
{}=\sum\limits_{n=1}^{M}C_{i,n}\left(z_1,z_2,s\right)u_{kn}(z_2)
\exp\left(-\left(
\vphantom{\tilde{\lambda}_n}
s+a_1-{}\right.\right.\\
\left.\left.{}-a_1 z_1+a_2-\widetilde{\lambda}_n(z_2)\right)x\right)\,,
\label{solution}
\end{multline}
где $\widetilde{\lambda}_k(z_2)\hm=\lambda_k(z_1,z_2,s)\hm+(s\hm+a_1\hm-a_1 z_1\hm+a_2)$, 
$k\hm=1,\ldots,M$; $\lambda_1(z_1,z_2,s),\ldots,\lambda_M(z_1,z_2,s)$~--- 
собственные значения матрицы системы: 
$u_1(z_2)\hm=(u_{11}(z_2),\ldots,u_{M1}(z_2))^{\mathrm{T}},\ldots,u_M(z_2)
\hm=(u_{1M}(z_2),\ldots,u_{MM}(z_2))^{\mathrm{T}}$~--- соответствующие собственные векторы. 
Заметим, что матрицы систем дифференциальных уравнений для 
$\widetilde{\pi}_1(z_1,z_2,k,x,s)$ и~$\widetilde{\pi}_2(z_1,z_2,k,x,s)$ одинаковы, 
а~следовательно, собственные значения и~собственные векторы в~записи решений совпадают.

Функции $\widetilde{\lambda}_1(z_2),\ldots,\widetilde{\lambda}_M(z_2)$ являются 
решениями характеристического уравнения:
\begin{multline}
\prod_{i=1}^M\left(pa_2 z_2^i-\widetilde{\lambda}\right)+
\sum\limits_{i=1}^M(1-p)h_i a_2 z_2^i
\prod\limits_{\substack{j=1\\j\neq i}}^M(pa_2 z_2^j-
\widetilde{\lambda})={}\\
{}=0\,.
\label{determinant}
\end{multline}

Подставляя~(\ref{solution}) в~(\ref{system}), находим:
\begin{equation}
u_{mn}(z_2)=\fr{(1-p)h_m a_2 z_2^m}{\widetilde{\lambda}_n(z_2)-pa_2 z_2^m}\,.
\label{eigenvectors}
\end{equation}

Подставив~(\ref{solution}) в~(\ref{newsecond}), получим:
\begin{multline*}
\sum\limits_{n=1}^{M}C_{1,n}\left(z_1,z_2,s\right)u_{mn}(z_2)+{}\\
{}+
\sum\limits_{n=1}^{M}C_{2,n}\left(z_1,z_2,s\right)u_{mn}(z_2)={}
\end{multline*}

\noindent
\begin{multline*}
{}=z_1^{-1}\sum\limits_{n=1}^{M}C_{1,n}\left(z_1,z_2,s\right)u_{mn}
\left(z_2\right)\times{}\\
{}\times\beta_1\left(s+a_1-a_1 z_1+a_2-\widetilde{\lambda}_n\left(z_2\right)\right)+{}\\
{}+z_2^{-1}\sum\limits_{n=1}^{M}C_{2,n}\left(z_1,z_2,s\right)u_{mn}(z_2)\times{}\\
{}\times \beta_2
\left(s+a_1-a_1 z_1+a_2-\widetilde{\lambda}_n\left(z_2\right)\right)-{}\\
{}-\left(s+a_1-a_1 z_1+a_2\right)
\pi_0(m,s)+h_m+{}\\
{}+\left[p\pi_0(m,s)a_2 z_2^m+(1-p)
\sum\limits_{n=1}^{M}\pi_0(n,s)h_m a_2 z_2^m\right]\,.
\end{multline*}
Перепишем это уравнение в~виде:
\begin{multline}
\sum\limits_{n=1}^{M}\left[
C_{1,n}(z_1,z_2,s)\left(
\vphantom{\tilde{\lambda}_n^{-1}}
1-{}\right.\right.\\
\left.{}-z_1^{-1}\beta_1
\left(s+a_1-a_1 z_1+a_2-\widetilde{\lambda}_n(z_2)\right)\right)+{}\\
{}+C_{2,n}\left(z_1,z_2,s\right)\left(1-
\vphantom{\tilde{\lambda}_n^{-1}}
{}\right.\\
\hspace*{-7.5pt}\left.\left.{}-z_2^{-1}\beta_2
\left(s+a_1-a_1 z_1+a_2-\widetilde{\lambda}_n(z_2)\right)\right)\right]
u_{mn}\left(z_2\right)={}\\
{}=b_m\left(z_1,z_2,s\right)\,,
\label{linear}
\end{multline}
где
\begin{multline*}
\!\!\!\!b_m\!\left(z_1,z_2,s\right)=
-\left(s+a_1-a_1 z_1+a_2\right)\!\pi_0(m,s)+h_m+{}\\
{}+
\left[p\pi_0(m,s)a_2 z_2^m+(1-p)
\sum\limits_{n=1}^{M}\pi_0(n,s)h_m a_2 z_2^m\right].
\end{multline*}

Подставим~(\ref{eigenvectors}) в~(\ref{linear}) и~поделим обе части полученного 
уравнения на $(1\hm-p)h_m a_2 z_2^m$:
\begin{multline*}
\sum\limits_{n=1}^{M}\left[C_{1,n}(z_1,z_2,s)\left(1-
\vphantom{\tilde{\lambda}_n^{-1}}
{}\right.\right.\\
\left.{}-z_1^{-1}\beta_1
\left(s+a_1-a_1 z_1+a_2-\widetilde{\lambda}_n\left(z_2\right)\right)\right)+{}\\
{}+C_{2,n}\left(z_1,z_2,s\right)\left(
\vphantom{\tilde{\lambda}_n^{-1}}
1-{}\right.\\
\left.\left.{}-z_2^{-1}\beta_2
\left(s+a_1-a_1 z_1+a_2-\widetilde{\lambda}_n(z_2)\right)\right)
\right]\times{}\\
{}\times \fr{1}{\widetilde{\lambda}_n(z_2)-pa_2 z_2^m}=
\fr{b_m(z_1,z_2,s)}{(1-p)h_m a_2 z_2^m}\,.
\end{multline*}
Это система линейных алгебраических уравнений с~матрицей Коши. 
Ее решение записывается в~виде\footnote{Про обращение матриц Коши см.~[2].}:

\noindent
\begin{multline*}
C_{1,n}(z_1,z_2,s)\left(
\vphantom{\tilde{\lambda}_n^{-1}}
1-{}\right.\\
\left.{}-z_1^{-1}\beta_1
\left(s+a_1-a_1 z_1+a_2-\widetilde{\lambda}_n(z_2)\right)\right)+{}\\
{}+C_{2,n}\left(z_1,z_2,s\right)
\left(
\vphantom{\tilde{\lambda}_n^{-1}}
1-{}\right.\\
\left.{}-z_2^{-1}\beta_2\left(s+a_1-a_1 z_1+a_2-\widetilde{\lambda}_n(z_2)\right)\right)={}\\
{}=\fr{\prod\limits_{i=1}^M(\widetilde{\lambda}_n(z_2)-pa_2 z_2^i)}
{\prod\limits_{\substack{j=1\\j\neq n}}^M
(\widetilde{\lambda}_n(z_2)-\widetilde{\lambda}_j(z_2))}
\sum\limits_{m=1}^M
\fr{\prod\limits_{\substack{i=1\\i\neq n}}^M
(pa_2 z_2^m-\widetilde{\lambda}_i(z_2))}
{\prod\limits_{\substack{j=1\\j\neq m}}^M(pa_2 z_2^m-pa_2 z_2^j)}\times{}\\
{}\times
\fr{b_m(z_1,z_2,s)}{(1-p)h_m a_2 z_2^m}.
\end{multline*}
Далее, поскольку функции $\widetilde{\lambda}_m(z_2)$, $m\hm=1,\ldots,M$, 
являются решениями уравнения~(\ref{determinant}), можно записать:
\begin{multline}
\prod_{i=1}^M(pa_2 z_2^i-\widetilde{\lambda})+
\sum\limits_{i=1}^M
(1-p)h_i a_2 z_2^i
\prod\limits_{\substack{j=1\\j\neq i}}^M\left(pa_2 z_2^j-\widetilde{\lambda}\right)={}\\
{}=
\prod_{j=1}^M\left(\widetilde{\lambda}_j(z_2)-\widetilde{\lambda}\right).
\label{polynom}
\end{multline}
Подставляя в~(\ref{polynom}) $\widetilde{\lambda}\hm=pa_2 z_2^m$, получим:
\begin{multline*}
(1-p)h_m a_2 z_2^m
\prod\limits_{\substack{j=1\\j\neq m}}^M
\left(pa_2 z_2^j-pa_2 z_2^m\right)={}\\
{}=\prod_{j=1}^M
\left(\widetilde{\lambda}_j(z_2)-pa_2 z_2^m\right).
\end{multline*}
Отсюда
\begin{multline}
C_{1,n}\left(z_1,z_2,s\right)\left(
\vphantom{\tilde{\lambda}_n^{-1}}
1-{}\right.\\
\left.{}-z_1^{-1}\beta_1
\left(s+a_1-a_1 z_1+a_2-\widetilde{\lambda}_n\left(z_2\right)\right)\right)+{}\\
{}+C_{2,n}\left(z_1,z_2,s\right)\left(
\vphantom{\tilde{\lambda}_n^{-1}}
1-{}\right.\\
\left.{}-z_2^{-1}\beta_2\left(s+a_1-a_1 z_1+a_2-\widetilde{\lambda}_n\left(
z_2\right)\right)\right)={}\\
{}=\fr{\prod\limits_{i=1}^M(\widetilde{\lambda}_n(z_2)-pa_2 z_2^i)}
{\prod\limits_{\substack{j=1\\j\neq n}}^M(\widetilde{\lambda}_n(z_2)-
\widetilde{\lambda}_j(z_2))}
\sum\limits_{m=1}^M
\fr{b_m(z_1,z_2,s)}{\widetilde{\lambda}_n(z_2)-pa_2 z_2^m}\,.
\label{C}
\end{multline}

Рассмотрим уравнение:
\begin{equation}
z_1=\beta_1\left(s+a_1-a_1 z_1+a_2-\widetilde{\lambda}_n(z_2)\right).
\label{functional}
\end{equation}
Обе части уравнения являются аналитическими в~области $|z_1|\hm\leq 1$ функциями. 
Имеем:
\begin{multline*}
\left\vert \beta_1\left(s+a_1-a_1 z_1+a_2-\widetilde{\lambda}_n
\left(z_2\right)\right)\right\vert
\leq{}\\
{}\leq \beta_1\left( 
\mathrm{Re}\,\left(s+a_1-a_1 z_1+a_2-\widetilde{\lambda}_n\left(z_2\right)\right)
\right)\leq{}\\
{}\leq 
\beta_1(\mathrm{Re}\,(s))<1=\left\vert z_1\right\vert
\end{multline*}
при $|z_1|=1$. В~силу теоремы Руше отсюда следует, что функциональное 
уравнение~(\ref{functional}) имеет единственное решение $z_1\hm=z_{1,n}(z_2,s)$, 
причем функция $z_{1,n}(z_2,s)$ является аналитической в~области $|z_2|\hm\leq 
1\times\mathrm{Re}\,(s)\hm>0$.

Подставляя $z_1\hm=z_{1,n}(z_2,s)$ в~уравнение~(\ref{C}), получим:
\begin{multline}
C_{2,n}\left(z_{1,n}(z_2,s),z_2,s\right)
\left(\vphantom{\tilde{\lambda}_n}
1-{}\right.\\
\left.{}-z_2^{-1}\beta_2\left(s+a_1-a_1 z_{1,n}(z_2,s)+a_2-\widetilde{\lambda}_n\left(z_2\right)
\right)\right)={}\\
{}=\fr{\prod\limits_{i=1}^M(\widetilde{\lambda}_n(z_2)-pa_2 z_2^i)}
{\prod\limits_{\substack{j=1\\j\neq n}}^M(\widetilde{\lambda}_n(z_2)-
\widetilde{\lambda}_j(z_2))}\times{}\\
{}\times
\sum\limits_{m=1}^M
\fr{b_m(z_{1,n}(z_2,s),z_2,s)}{\widetilde{\lambda}_n(z_2)-pa_2 z_2^m}\,,
\label{findpi1}
\end{multline}
откуда
\begin{multline*}
C_{2,n}\left(z_{1,n}\left(z_2,s\right),z_2,s\right)={}\\
{}=
\fr{1}{1-z_2^{-1}\beta_2(s+a_1-a_1 z_{1,n}(z_2,s)+
a_2-\widetilde{\lambda}_n(z_2))}\times{}\\
{}\times
\fr{\prod\limits_{i=1}^M(\widetilde{\lambda}_n(z_2)-pa_2 z_2^i)}
{\prod\limits_{\substack{j=1\\j\neq n}}^M(\widetilde{\lambda}_n(z_2)-
\widetilde{\lambda}_j(z_2))}
\sum\limits_{m=1}^M
\fr{b_m(z_{1,n}(z_2,s),z_2,s)}{\widetilde{\lambda}_n(z_2)-pa_2 z_2^m}\,.
\end{multline*}
Заметим, что $\widetilde{\pi}_2(z_1,z_2,k,x,s)$, а~значит, 
и~$C_{2,n}(z_1,z_2,s)$, не зависит от~$z_1$. Из этого факта вытекает, 
что можно записать:
\begin{multline}
C_{2,n}\left(z_1,z_2,s\right)={}\\
{}=\!
\fr{1}{1\!-\!z_2^{-1}\,\beta_2\!\left(\!s+a_1-a_1 z_{1,n}(z_2,s)+a_2-\widetilde{\lambda}_n
\left(z_2\right)\!\right)}\!\times{}\\
\hspace*{-4mm}{}\times
\fr{\prod\limits_{i=1}^M(\widetilde{\lambda}_n(z_2)-pa_2 z_2^i)}
{\prod\limits_{\substack{j=1\\j\neq n}}^M(\widetilde{\lambda}_n(z_2)-
\widetilde{\lambda}_j(z_2))}
\!\sum\limits_{m=1}^M\!
\fr{b_m(z_{1,n}(z_2,s),z_2,s)}{\widetilde{\lambda}_n(z_2)-pa_2 z_2^m}.\!\!\!
\label{C2}
\end{multline}
Подставляя~(\ref{C2}) в~(\ref{C}), получим
\begin{multline*}
C_{1,n}(z_1,z_2,s)\left(
\vphantom{\tilde{\lambda}_n^{-1}}
1-{}\right.\\
\left.{}-z_1^{-1}\beta_1
\left(s+a_1-a_1 z_1+a_2-\widetilde{\lambda}_n\left(z_2\right)\right)\right)={}\\
{}=\fr{\prod\limits_{i=1}^M(\widetilde{\lambda}_n(z_2)-pa_2 z_2^i)}
{\prod\limits_{\substack{j=1\\j\neq n}}^M(\widetilde{\lambda}_n(z_2)-
\widetilde{\lambda}_j(z_2))}
\sum\limits_{m=1}^M
\fr{1}{\widetilde{\lambda}_n(z_2)-pa_2 z_2^m}\times{}\\
{}\times\left(b_m\left(z_1,z_2,s\right)-{}\right.\\
{}-
\fr{1-z_2^{-1}\beta_2(s+a_1-a_1 z_1+a_2-\widetilde{\lambda}_n(z_2))}
{1-z_2^{-1}\beta_2(s+a_1-a_1 z_{1,n}(z_2,s)+a_2-\widetilde{\lambda}_n(z_2))}\times{}\\
\left.{}\times b_m\left(z_{1,n}\left(z_2,s\right),z_2,s\right)\right)\,,
%\label{C1}
\end{multline*}
т.\,е.\ \\[-17pt]
\begin{multline*}
C_{1,n}\left(z_1,z_2,s\right)={}\\
{}=
\fr{1}{1-z_1^{-1}\,\beta_1\left(s+a_1-a_1 z_1+a_2-
\widetilde{\lambda}_n\left(z_2\right)\right)}\times{}\\
{}\times
\fr{\prod\limits_{i=1}^M
(\widetilde{\lambda}_n(z_2)-pa_2 z_2^i)}
{\prod\limits_{\substack{j=1\\j\neq n}}^M(\widetilde{\lambda}_n(z_2)-
\widetilde{\lambda}_j(z_2))}
\sum\limits_{m=1}^M
\fr{1}{\widetilde{\lambda}_n(z_2)-pa_2 z_2^m}\times{}\\
{}\times\left( b_m\left(z_1,z_2,s\right)-{}\right.\\
{}-
\fr{1-z_2^{-1}\beta_2(s+a_1-a_1 z_1+a_2-\widetilde{\lambda}_n(z_2))}
{1-z_2^{-1}\beta_2(s+a_1-a_1 z_{1,n}(z_2,s)+a_2-\widetilde{\lambda}_n(z_2))}\times{}\\
\left.{}\times b_m\left(z_{1,n}\left(z_2,s\right),z_2,s\right)\right).
\end{multline*}

Остается найти $\pi_0(m,s)$, $m\hm=1,\ldots,M$. Рас\-смот\-рим уравнение:
\begin{equation}
z_2=\beta_2\left(s+a_1-a_1 z_{1,n}\left(z_2,s\right)+
a_2-\widetilde{\lambda}_n\left(z_2\right)\!\right).\!\!
\label{functional2}
\end{equation}
Обе части уравнения являются аналитическими в~области $|z_2|\hm\leq 1$ 
функциями. Имеем:
\begin{multline*}
\left\vert\beta_2\left(s+a_1-a_1 z_{1,n}\left(z_2,s\right)+a_2-
\widetilde{\lambda}_n\left(z_2\right)\right)\right\vert
\leq{}\\
{}\leq\beta_2\left(\mathrm{Re}\left(s+a_1-a_1 z_{1,n}\left(z_2,s\right)+
a_2-\widetilde{\lambda}_n\left(z_2\right)\!\right)\!\right)\leq{}\\
{}\leq\beta_2(\mathrm{Re}\,(s))<1=\left|z_2\right|
\end{multline*}
при $|z_2|\hm=1$. В~силу теоремы Руше отсюда следует, что функциональное 
уравнение~(\ref{functional2}) имеет единственное решение $z_2\hm=z_{2,n}(s)$, 
причем функция $z_{2,n}(s)$ является аналитической в~области $\mathrm{Re}\,(s)\hm>0$.

Подставляя $z_2=z_{2,n}(s)$ в~уравнение (\ref{findpi1}), 
приходим после ряда преобразований 
к~уравнению\footnote{В дальнейшем будем для краткости писать $z_{1,n}$ 
вместо $z_{1,n}(z_{2,n}(s),s)$, $z_{2,n}$ вместо $z_{2,n}(s)$ 
и~$\widetilde{\lambda}_n$ вместо $\widetilde{\lambda}_n(z_{2,n})$.}:
\begin{equation}
\sum\limits_{m=1}^M
\prod\limits_{\substack{j=1\\j\neq m}}^M(\widetilde{\lambda}_n-
pa_2 z_{2,n}^j)b_m(z_{1,n},z_{2,n},s)=0\,.
\label{findpi2}
\end{equation}
Вспомним, что
\begin{multline*}
\hspace*{-5.5pt}b_m\!\left(z_1,z_2,s\right)=-\left(s+a_1-a_1 z_1+a_2\right)
\!\pi_0(m,s)+h_m+{}\\
{}+\left[p\pi_0(m,s)a_2 z_2^m+(1-p)
\sum\limits_{n=1}^{M}\pi_0(n,s)h_m a_2 z_2^m\right].
\end{multline*}
С учетом уравнения~(\ref{determinant}) будем иметь:
\begin{multline*}
\sum\limits_{m=1}^M\prod\limits_{\substack{j=1\\j\neq m}}^M
\left(\widetilde{\lambda}_n-pa_2 z_{2,n}^j\right)
\left[
\vphantom{\sum\limits_{k=1}^{M}}
p\pi_0(m,s)a_2 z_{2,n}^m+{}\right.\\
\left.{}+(1-p)
\sum\limits_{k=1}^{M}\pi_0(k,s)h_m a_2 z_{2,n}^m\right]={}\\
{}=\sum\limits_{m=1}^M
\prod\limits_{\substack{j=1\\j\neq m}}^M
\left(\widetilde{\lambda}_n-pa_2 z_{2,n}^j\right)\times{}\\
{}\times
\left[
\vphantom{\sum\limits_{k=1}^{M}}
\left(pa_2 z_{2,n}^m-\widetilde{\lambda}_n\right)
\pi_0(m,s)+\widetilde{\lambda}_n \pi_0(m,s)+{}\right.\\
\left.{}+(1-p)
\sum\limits_{k=1}^{M}\pi_0(k,s)h_m a_2 z_{2,n}^m\right]={}\\
{}=-\prod\limits_{j=1}^M\left(\widetilde{\lambda}_n-pa_2 z_{2,n}^j\right)
\sum\limits_{m=1}^M \pi_0(m,s)+{}\\
{}+
\sum\limits_{m=1}^M
\prod\limits_{\substack{j=1\\j\neq m}}^M\left(\widetilde{\lambda}_n-pa_2 z_{2,n}^j\right)
\widetilde{\lambda}_n \pi_0(m,s)+{}\\
{}+\prod\limits_{j=1}^M
\left(\widetilde{\lambda}_n-pa_2 z_{2,n}^j\right)
\sum\limits_{m=1}^M \pi_0(m,s)={}\\
{}=
\sum\limits_{m=1}^M
\prod\limits_{\substack{j=1\\j\neq m}}^M\left(\widetilde{\lambda}_n-pa_2 z_{2,n}^j\right)
\widetilde{\lambda}_n \pi_0(m,s)\,.
\end{multline*}
Возвращаясь к~уравнению~(\ref{findpi2}), получим:
\begin{multline*}
\sum\limits_{m=1}^M
\prod\limits_{\substack{j=1\\j\neq m}}^M\left(\widetilde{\lambda}_n-pa_2 z_{2,n}^j\right)
\left(
\vphantom{\widetilde{\lambda}_n}
-\left(s+a_1-a_1 z_{1,n}+{}\right.\right.\\
\left.\left.{}+a_2\right)\pi_0(m,s)+
h_m+\widetilde{\lambda}_n \pi_0(m,s)\right)=0\,.
\end{multline*}

Там самым доказательство теоремы завершено.

{\small\frenchspacing
 {%\baselineskip=10.8pt
 \addcontentsline{toc}{section}{References}
 \begin{thebibliography}{9}
\bibitem{1-us}
\Au{Леонтьев Н.\,Д., Ушаков В.\,Г.} 
Анализ системы обслуживания с~входящим потоком авторегрессионного типа~// 
Информатика и~её применения, 2014. Т.~8. Вып.~3. С.~39--44.
\bibitem{2-us}
\Au{Schechter S.} On the inversion of certain matrices~// 
Math. Tab. Aids Comput., 1959. Vol.~13. No.\,66. P.~73--77.
\end{thebibliography}

 }
 }

\end{multicols}

\vspace*{-6pt}

\hfill{\small\textit{Поступила в~редакцию 11.05.16}}

\vspace*{8pt}

%\newpage

%\vspace*{-24pt}

\hrule

\vspace*{2pt}

\hrule

%\vspace*{8pt}



\def\tit{ANALYSIS OF~A~QUEUEING SYSTEM WITH~AUTOREGRESSIVE ARRIVALS 
AND~NONPREEMPTIVE PRIORITY}

\def\titkol{Analysis of~a~queueing system with~autoregressive arrivals 
and~nonpreemptive priority}

\def\aut{N.\,D.~Leontyev$^1$ and V.\,G.~Ushakov$^{1,2}$}

\def\autkol{N.\,D.~Leontyev and V.\,G.~Ushakov}

\titel{\tit}{\aut}{\autkol}{\titkol}

\vspace*{-9pt}

\noindent
$^1$Department of Mathematical 
Statistics, Faculty of Computational Mathematics and Cybernetics,\linebreak
$\hphantom{^1}$M.\,V.~Lomonosov 
Moscow State University, 1-52~Leninskiye Gory, Moscow 119991, GSP-1, Russian\linebreak 
$\hphantom{^1}$Federation

\noindent
$^2$Institute of Informatics Problems, Federal 
Research Center ``Computer Science and Control'' of the Russian\linebreak 
$\hphantom{^1}$Academy of Sciences, 
44-2~Vavilov Str., Moscow 119333, Russian Federation

\def\leftfootline{\small{\textbf{\thepage}
\hfill INFORMATIKA I EE PRIMENENIYA~--- INFORMATICS AND
APPLICATIONS\ \ \ 2016\ \ \ volume~10\ \ \ issue\ 3}
}%
 \def\rightfootline{\small{INFORMATIKA I EE PRIMENENIYA~---
INFORMATICS AND APPLICATIONS\ \ \ 2016\ \ \ volume~10\ \ \ issue\ 3
\hfill \textbf{\thepage}}}

\vspace*{3pt}



\Abste{The paper studies a single server queueing system with infinite capacity 
and with two arrival streams, one of which is Poisson and the other is batch Poisson. 
The customers from the first stream have nonpreemptive priority over the customers 
from the second. A~feature of the system under study is autoregressive dependence 
of the sizes of the batches from the second arrival stream: the size of the 
$n$th batch is equal to the size of the $(n-1)$st batch with a~fixed probability 
and is an independent random variable with complementary probability. Service 
times of the customers from each stream are supposed to be independent random 
variables with specified distributions. The main object of the study is the 
number of the customers from each stream in the system at an arbitrary moment. 
The relations derived make it possible
to find Laplace transorm in time of probability 
generating function of the transient queue length and also a~number of 
additional characteristics.}

\KWE{queueing theory; transient behavior; batch arrivals; nonpreemptive priority}

\DOI{10.14357/19922264160303}

%\vspace*{-9pt}

\Ack
\noindent
This work was supported by the Russian Foundation for Basic 
Research (project No.\,15-07-02354).


%\vspace*{3pt}

  \begin{multicols}{2}

\renewcommand{\bibname}{\protect\rmfamily References}
%\renewcommand{\bibname}{\large\protect\rm References}

{\small\frenchspacing
 {%\baselineskip=10.8pt
 \addcontentsline{toc}{section}{References}
 \begin{thebibliography}{9}
\bibitem{1-us-1}
\Aue{Leontyev, N.\,D., and V.\,G.~Ushakov}. 
2014. Analiz sistemy obsluzhivaniya s~vkhodyashchim potokom avtoregressionnogo tipa 
[Analysis of queueing system with autoregressive arrivals]. 
\textit{Informatika i~ee Primeneniya~--- Inform. Appl.} 8(3):39--44.

 
\bibitem{2-us-1}
\Aue{Schechter, S.} 1959. On the inversion of certain matrices. 
\textit{Math. Tab. Aids Comput.} 13(66):73--77.
   \end{thebibliography}

 }
 }

\end{multicols}

\vspace*{-3pt}

\hfill{\small\textit{Received May 11, 2016}}


\Contr

\noindent
\textbf{Leontyev Nikolai D.} (b.\ 1988)~--- PhD student,
Department of Mathematical 
Statistics, Faculty of Computational Mathematics and Cybernetics, M.\,V.~Lomonosov 
Moscow State University, 1-52~Leninskiye Gory, Moscow 119991, GSP-1, Russian 
Federation; \mbox{ndleontyev@gmail.com}

\vspace*{3pt}

\noindent
\textbf{Ushakov Vladimir G.} (b.\ 1952)~---
Doctor of Science in physics and mathematics, professor, Department of Mathematical 
Statistics, Faculty of Computational Mathematics and Cybernetics, M.\,V.~Lomonosov 
Moscow State University, 1-52~Leninskiye Gory, Moscow 119991, GSP-1, Russian 
Federation; senior scientist, Institute of Informatics Problems, Federal 
Research Center ``Computer Science and Control'' of the Russian Academy of Sciences, 
44-2~Vavilov Str., Moscow 119333, Russian Federation; \mbox{vgushakov@mail.ru} 
\label{end\stat}


\renewcommand{\bibname}{\protect\rm Литература}