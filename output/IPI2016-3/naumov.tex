\def\stat{naumov}

\def\tit{О СВЯЗИ РЕСУРСНЫХ СИСТЕМ МАССОВОГО ОБСЛУЖИВАНИЯ 
С~СЕТЯМИ ЭРЛАНГА$^*$}

\def\titkol{О связи ресурсных систем массового обслуживания 
с~сетями Эрланга}

\def\aut{В.\,А.~Наумов$^1$, К.\,Е.~Самуйлов$^2$ }

\def\autkol{В.\,А.~Наумов, К.\,Е.~Самуйлов}

\titel{\tit}{\aut}{\autkol}{\titkol}

\index{Naumov V.\,A.}
\index{Samouylov K.\,E.}
\index{Наумов В.\,А.}
\index{Самуйлов К.\,Е.}


{\renewcommand{\thefootnote}{\fnsymbol{footnote}} \footnotetext[1]
{Работа выполнена при частичной финансовой поддержке РФФИ (проекты 16-07-00766,
 15-07-03051 и~15-07-03608).}}


\renewcommand{\thefootnote}{\arabic{footnote}}
\footnotetext[1]{Исследовательский институт инноваций, 
г.\ Хельсинки, Финляндия, \mbox{valeriy.naumov@pfu.fi}}
\footnotetext[2]{Российский университет дружбы народов; Институт проблем информатики Федерального 
исследовательского центра <<Информатика и~управление>> Российской академии наук,  
\mbox{ksam@sci.pfu.edu.ru}}

\vspace*{-12pt}
  
\Abst{Рассматривается модель многолинейной системы массового 
обслуживания (СМО) с~потерями, вызванными нехваткой ресурсов, 
необходимых для обслуживания заявок. Принятая на обслуживание заявка 
занимает случайные объемы ресурсов нескольких типов с~заданными 
функциями распределения. Случайные векторы, описывающие требования 
заявок к~ресурсам, не зависят от процессов поступления и~обслуживания 
заявок, независимы в~совокупности и~одинаково распределены. Интерес, как 
и~в~задаче Эрланга, представляет вычисление вероятности потери 
поступающей заявки из-за нехватки ресурсов. Показана связь между 
мультисервисными сетями Эрланга и~ресурсными СМО, что позволяет решать 
задачу вычисления вероятности потерь в~ресурсной СМО с~помощью 
известных методов, разработанных для мультисервисных сетей.}

\KW{мультисервисная сеть; сеть Эрланга; система массового обслуживания; 
ресурсная СМО; случайный объем ресурсов; вероятность потерь; решетчатая 
функция}

\DOI{10.14357/19922264160302} 


\vskip 12pt plus 9pt minus 6pt

\thispagestyle{headings}

\begin{multicols}{2}

\label{st\stat}

\section{Введение}

  Рассмотрим многолинейные СМО с~потерями, разнотипными ресурсами 
и~пуассоновским входящим потоком, которые функционируют сле\-ду\-ющим 
образом. Поступившая заявка теряется, если в~момент поступления количество 
ка\-ко\-го-ли\-бо требуемого ей ресурса превышает количество свободного 
ресурса этого типа либо если число обслуживаемых заявок достигло 
максимума. В~момент начала обслуживания заявки суммарный объем 
свободного ресурса каждого типа уменьшается на величину ресурса, 
выделенного этой заявке. В~момент окончания обслуживания заявки 
суммарный объем свободного ресурса каждого типа увеличивается на величину 
ресурса, выделенного этой заявке при поступлении. 
  
  Ресурсные СМО с~пуассоновским входящим потоком исследуются давно. 
В~\cite{3-n} получены стационарные распределения числа заявок в~системе 
и~объема занятого ресурса для СМО с~экспоненциальной функцией 
распределения длительности обслуживания и~произвольной функцией 
распределения объемов ресурса. Эти результаты обобщены в~\cite{4-n} на 
СМО с~произвольной функцией распределения длительностей обслуживания 
и~в~\cite{1-n} на системы с~множественными ресурсами. Дальнейшие 
обобщения, рассмотренные в~\cite{5-n, 6-n}, включают системы, у которых 
время обслуживания заявки и~объемы выделенных ей ресурсов являются 
зависимыми случайными величинами. В~\cite{5-n} получено стационарное 
распределение и~вероятность потери для СМО с~произвольной совместной 
функцией распределения длительности обслуживания и~объема единственного 
ресурса. Эти результаты обобщены в~\cite{6-n} на СМО, в~которых каждая 
заявка характеризуется тремя зависимыми случайными признаками: числом 
приборов, необходимых для обслуживания, объемом ресурса и~временем 
обслуживания. Применение модели ресурсных СМО к~анализу вероятностных 
характеристик беспроводных гетерогенных сетей 5-го поколения было 
предложено в~\cite{2-n}.
  
  Хорошо изучены мультисервисные сети Эрланга~\cite{7-n, 8-n} 
с~соединениями нескольких типов, в~которых каждому соединению в~каждом 
звене сети выделяется определенное число каналов, т.\,е.\ ресурсов сети, 
удерживаемое до завершения соеди-\linebreak нения. 

В~настоящей работе исследуется 
связь \mbox{между} мультисервисными сетями Эрланга и~ресурсными СМО 
с~арифметическими функциями распределения объемов требуемых ресурсов, 
которыми сколь угодно точно можно аппроксимировать любые функции 
распределения объемов ресурсов. \mbox{Целью} работы является исследование 
приближенного подхода к~вычислению вероятностных характеристик 
ресурсных СМО. Для краткости будем опускать слово <<мультисервисные>> 
в~названии сетей Эрланга.

\section{Сети Эрланга}

  Рассмотрим сеть массового обслуживания с~потерями, состоящую из 
некоторого числа узлов, соединенных звеньями. 

Пусть общее число звеньев 
сети равно~$M$, емкость $m$-го звена равна~$N_m$, $\mathbf{N}\hm= (N_1, 
N_2, \ldots ,N_M)$ и~${\sf N}(\mathbf{n}) \hm= \{ \mathbf{i}\hm\in {\sf N}^M 
\vert \mathbf{0}\leq \mathbf{i}\leq \mathbf{n}\}$, где ${\sf N}$~--- множество 
неотрицательных целых чисел. Между узлами сети могут быть установлены 
соединения~$L$~различных классов, каждый из которых однозначно 
характеризуется своими требованиями к~емкости звеньев сети. Требование 
к~числу каналов соединений $l$-го класса задается вектором 
$\mathbf{n}_l\hm= (n_{l1}, n_{l2}, \ldots , n_{lM})$, где~$n_{lj}$ есть число 
каналов, занимаемых соединением на $j$-м звене сети. 

Предположим, что 
запросы на уста\-нов\-ле\-ние в~сети соединения $l$-го класса образуют 
пуассоновский поток интенсивности~$\lambda_l$, причем средняя 
продолжительность соединений $l$-го класса равна $b_l\hm<\infty$. Если при 
поступлении запроса на уста\-нов\-ле\-ние нового соединения в~сети недостаточно 
свободных каналов или уже установлено максимально возможное 
число~$K$~соединений, происходит блокировка запроса. 
  
  Обозначим $v_l(t)$ число соединений $l$-го класса, установленных в~сети 
в~момент~$t$, $\mathbf{v}(t)\hm= (v_1(t), v_2(t),\ldots , v_L(t))$, и~${\sf K}$~--- 
пространство состо\-яний процесса $v(t)$, представляющее со-\linebreak бой множество 
неотрицательных целочисленных\linebreak векторов $\mathbf{k}\hm= (k_1, k_2, \ldots , 
k_L)$, удовлетворяющих неравенствам $k_1\hm+k_2+\cdots + k_L\hm\leq K$ 
и~$k_1\mathbf{n}_1\hm+ k_2\mathbf{n}_2+\cdots + k_L\mathbf{n}_L\hm\leq 
\mathbf{N}$. 

Стационарное распределение $\phi(\mathbf{k})\hm= 
\lim\limits_{t\to\infty} P\{ \mathbf{v}(t)=\mathbf{k}\}$ процесса $\mathbf{v}(t)$ зависит от функций 
распределения продолжительности соединений лишь посредством средних 
значений и~дается следующей формулой~\cite{7-n}:
  \begin{equation}
  \left.
  \begin{array}{rl}
  \phi(\mathbf{0}) &=  \sum\limits_{\mathbf{k}\in{\sf K}} \fr{(\lambda_1 
b_1)^{k_1}}{k_1!} \cdots \fr{(\lambda_L b_L)^{k_L}}{k_L!}\,;\\[6pt]
  \phi(\mathbf{k})& = \phi(\mathbf{0}) \fr{(\lambda_1 b_1)^{k_1}}{k_1!}\cdots 
\fr{(\lambda_L b_L)^{k_L}}{k_L!}\,,\enskip \mathbf{k}\in{\sf K}\,.
\end{array}
\right\}
  \label{e1-n}
  \end{equation}
  
  Пусть $w_m(t)$~--- число занятых в~момент~$t$ каналов $m$-го звена сети 
и~$\mathbf{w}(t)\hm= (w_1(t), w_2(t), \ldots , w_M(t))$. Зная распределение вероятностей 
$\phi(\mathbf{k})$, легко \mbox{найти} совместное распределение числа 
установленных в~сети соединений и~занятых ими каналов:
  \begin{multline*}
  \psi_k(\mathbf{i}) ={}\\[6pt]
  {}=\lim\limits_{t\to\infty} {\sf P} \{ v_1(t)+v_2(t)+\cdots +v_L(t)=k,\ 
\mathbf{w}(t)=\mathbf{i}\}={}\hspace*{-2.2pt}
\end{multline*}

\noindent
\begin{multline}
 {}= \hspace*{-20pt}
\sum\limits_{\substack{{\mathbf{k}\in{\sf K}}\\ {k_1+\cdots +
k_L=k}\\{k_1\mathbf{n}_1+\cdots +k_L\mathbf{n}_L=\mathbf{i}}}} 
\hspace*{-21pt}\phi(\mathbf{k}) =\phi(\mathbf{0}) \hspace*{-23pt}
\sum\limits_{\substack{{\mathbf{k}\in {\sf K}}\\ {k_1+\cdots +
k_L=k}\\{k_1\mathbf{n}_1+\cdots +{k}_{{L}}\mathbf{n}_{{L}}=\mathbf{i}}}} \hspace*{-20pt}
\hspace*{-4.86pt}\fr{(\lambda_1b_1)^{k_1}}{k_1!}\cdots \fr{(\lambda_L b_L)^{k_L}}{k_L!}={}\\
{}=
\phi(\mathbf{0})\fr{\rho^k}{k!} \hspace*{-20pt}
\sum\limits_{\substack{{\mathbf{k}\in {\sf K}}\\ {k_1+\cdots+ 
k_L=k}\\{k_1\mathbf{n}_1+\cdots +{k}_{{L}}\mathbf{n}_{{L}}=\mathbf{i}}}}
\hspace*{-20pt} \fr{k!}{k_1!\!\!\cdots 
k_L!} \left( \fr{\lambda_1b_1}{\rho}\right)^{k_1}\cdots \left( 
\fr{\lambda_Lb_L}{\rho}\right)^{k_L}\!, \\
  \mathbf{i}\in {\sf N}(\mathbf{N})\,, 
\enskip  k=0, 1,\ldots, K\,,
\label{e2-n}
\end{multline}
где $\rho=\lambda_1b_1\hm+ \lambda_2b_2+\cdots +\lambda_Lb_L$. 
Формула~(\ref{e2-n}) допускает простую вероятностную интерпретацию. 
Рассмотрим распределение вероятностей~$\pi(\mathbf{i})$, $\mathbf{i}\hm\in 
{\sf N}(\mathbf{N})$, случайного целочисленного вектора, с~положительной 
вероятностью принимающего лишь значения из подмножества ${\sf L} \hm= 
\{\mathbf{n}_1, \mathbf{n}_2, \ldots ,\mathbf{n}_L\}$ множества~${\sf 
N}(\mathbf{N})$, причем значение~$\mathbf{n}_l$ принимается с~вероятностью 
$\lambda_lb_l/\rho$, т.\,е. 
\begin{equation}
\pi(\mathbf{i}) =\begin{cases}
\fr{\lambda_lb_l}{\rho}\,, &\ \mathbf{i}=\mathbf{n}_l\,;\\
0\,, &\ \mathbf{i}\not={\sf L}\,.
\end{cases}
\label{e3-n}
\end{equation}
С~учетом очевидного равенства $\varphi(\mathbf{0})\hm= \psi_0(\mathbf{0})$ формулу~(\ref{e2-n}) 
можно записать следующим образом:
\begin{equation}
\psi_k(\mathbf{i}) =\psi_0 (\mathbf{0}) \pi^{(k)} (\mathbf{i}) \fr{\rho^k}{k!}\,,\enskip 
\mathbf{i}\in{\sf N}(\mathbf{N})\,,
\label{e4-n}
\end{equation}
где $\pi^{(k)}(\mathbf{i})$ есть $k$-крат\-ная свертка распределения 
вероятностей~(\ref{e3-n}).
  
  Положим $C(\mathrm{r})\hm=0$, если вектор~$\mathbf{r}$ не является 
не\-от\-ри\-ца\-тельным, а для неотрицательных векторов~$\mathbf{r}$ определим 
величины~$C(\mathbf{r})$ следующим образом:
  \begin{equation}
  C(\mathbf{r}) = 1+\sum\limits_{k=1}^K \fr{\rho^k}{k!} 
\sum\limits_{\mathbf{i}\in {\sf N}(\mathbf{r})} \pi^{(k)} (\mathbf{i})\,.
  \label{e5-n}
\end{equation}
Эти величины играют роль нормировочных констант для распределений 
вероятностей~(\ref{e1-n}) и~(\ref{e2-n}), поскольку справедливо равенство:
\begin{equation}
\phi(\mathbf{0}) =\psi_0(\mathbf{0}) =C(\mathbf{N})^{-1}\,.
\label{e6-n}
\end{equation}
  
  Условная вероятность блокировки запроса на установление соединения при 
условии, что запрашивается соединение $l$-го класса, дается следующей 
формулой~\cite{7-n}:
  \begin{equation*}
  B_l = 1- \fr{C(\mathbf{N}-\mathbf{n}_l)}{C(\mathbf{N})}\,.
%  \label{e7-n}
  \end{equation*}
Отсюда вытекает выражение для безусловной вероятности блокировки запроса 
на установление соединения: 
\begin{equation}
B= 1-\fr{1}{C(\mathbf{N})}\sum\limits_{l=1}^L \fr{\lambda_l}{\lambda}\,C\left( 
\mathbf{N} - \mathbf{n}_l\right)\,,
\label{e8-n}
\end{equation}
где $\lambda= \lambda_1\hm+ \lambda_2+\cdots+ \lambda_L$ есть интенсивность 
суммарного потока поступающих запросов.

\section{Ресурсная система массового обслуживания}

Рассмотрим ресурсную СМО с~ресурсами~$M$ типов, на которую поступает 
пуассоновский поток заявок с~параметром~$\lambda$. Обозначим~$R_m$ 
общий объем ресурса типа~$m$ и~$\mathbf{R}\hm= (R_1, R_2,\ldots , R_M)$. 
Поступившая $j$-я заявка характеризуется длительностью обслуживания~$s_j$ 
и вектором объемов необходимых ей ресурсов $\mathbf{r}_j\hm= (r_{j1}, r_{j2}, 
\ldots , r_{jM})$. Случайные векторы $(s_j,\mathbf{r}_j)$, $j\hm=1,2,\ldots$, не 
зависят от моментов поступления заявок, независимы в~совокупности и~имеют 
одинаковую совместную функцию распределения $H(t,\mathbf{x}) \hm= {\sf P}\{ 
s_j\leq t, \mathbf{r}_j\leq \mathbf{x}\}$. \mbox{Обозначим} через $F(\mathbf{x}) \hm= 
{\sf P}\{\mathbf{r}_j\leq \mathbf{x}\}$ функцию распределения объемов требуемых 
заявке ресурсов, $B(t)\hm= {\sf P}\{s_j\leq t\}$~--- функцию распределения 
дли\-тель\-ности обслуживания, $b\hm<\infty$~--- среднюю \mbox{длительность} 
обслуживания и~$\rho\hm= \lambda b$. Для простоты будем считать, что 
$F(\mathbf{R})\hm=1$, т.\,е.\ требование заявкой любого ресурса не превосходит 
его общего объема. 
  
  Состояние рассматриваемой системы в~момент~$t$ можно описать 
случайным процессом $X(t)\hm= (\xi(t),\gamma(t))$. Здесь $\xi(t)$~--- число 
заявок в~системе и~$\gamma(t)\hm= (\gamma_1(t), \ldots , \gamma_M(t))$~--- 
вектор объемов занимаемых ими ресурсов. При условии конечности среднего 
времени обслуживания~$b$ стационарное распределение процесса $X(t)$ 
  \begin{equation}
  \left.
  \begin{array}{rl}
  p_0 &=\lim\limits_{t\to\infty} {\sf P}\{ \xi(t)=0\}\,;\\[6pt] 
  P_k(\mathbf{x}) 
&=\lim\limits_{t\to\infty} {\sf P}\{ \xi(t)=k; \gamma(t)\leq \mathbf{x}\}\,,\\[6pt]
& \hspace*{5mm}\mathbf{0}\leq  \mathbf{x} \leq \mathbf{R}\,,\enskip
 k=0,1,\ldots ,K\,,
\end{array}
\right\}
  \label{e9-n}
\end{equation}
имеет следующий вид:
\begin{equation}
\left.
\begin{array}{rl}
p_0 &=\left( 1+\sum\limits_{k=1}^K G^{(k)} (\mathbf{R}) \fr{\rho^k}{k!}  
\right)^{-1}\,;\\[6pt]
P_k(\mathbf{x}) &=p_0 G^{(k)} (\mathbf{x}) \fr{\rho^k}{k!}\,,\enskip k=1, 2,\ldots, 
K\,.
\end{array}
\right\}
\label{e10-n}
\end{equation}
Здесь $K$~--- максимальное число заявок в~системе и~$G^{(k)}(\mathbf{x})$~--- 
$k$-крат\-ная свертка функции распределения
\begin{equation}
G(\mathbf{x}) = \fr{1}{b}\int\limits_{\mathbf{y}\leq \mathbf{x}} 
\int\limits_0^\infty  tH(dt, d\mathbf{y})\,.
\label{e11-n}
\end{equation}
  
  Из формул~(\ref{e9-n}), в~частности, вытекает сле\-ду\-ющее выражение для 
вероятности потери заявки в~ресурсной СМО:
  \begin{equation*}
  B=1-p_0 \left( 1+ \sum\limits_{k=1}^{K-1} \left( G^{(k)} * F\right)  (\mathbf{R}) 
\fr{\rho^k}{k!}\right)\,,
%  \label{e12-n}
  \end{equation*}
где $(G^{(k)} * F)(\mathbf{x})$ есть свертка функций распределения 
$G^{(k)}(\mathbf{x})$ и~$F(\mathbf{x})$.
  
  Справедливость формул~(\ref{e10-n}) можно установить путем очевидного 
обобщения на произвольное чис\-ло ресурсов результатов работы~\cite{6-n}. 
Ниже будет показано, как для вычисления стационарных характеристик 
ресурсных СМО с~множественными дискретными ресурсами~\cite{2-n} можно 
использовать сети Эрланга. Попутно будут доказаны формулы~(\ref{e10-n}) 
для решетчатых функций распределения объемов ресурсов~$F(\mathbf{x})$.
  
  Определение~(\ref{e11-n}) функции распределения $G(\mathbf{x})$ станет 
понятнее, если ввести условное среднее время обслуживания заявки 
$b(\mathbf{x})$ при условии, что вектор объемов необходимых ей ресурсов 
равен~$\mathbf{x}$. Это условное среднее время обслуживания можно 
определить как функцию, которая при любом векторе~$\mathbf{x}$ 
удовлетворяет следующему уравнению~\cite{9-n}:
  \begin{equation}
  \int\limits_{\mathbf{y}\leq \mathbf{x}}\int\limits_0^\infty tH(dt,d\mathbf{y})= 
\int\limits_{\mathbf{y}\leq \mathbf{x}} b(\mathbf{y}) D(d\mathbf{y})\,.
  \label{e13-n}
  \end{equation}
Используя равенство~(\ref{e13-n}), выражение~(\ref{e11-n}) для функции 
$G(\mathbf{x})$ можно переписать следующим образом:
\begin{equation}
G(\mathbf{x}) = \fr{1}{b}\sum\limits_{\mathbf{y}\leq\mathbf{x}} b(\mathbf{y}) 
F(d\mathbf{y})\,.
\label{e14-n}
\end{equation}

\begin{figure*}
\vspace*{1pt}
 \begin{center}  
\mbox{%
 \epsfxsize=147.923mm
 \epsfbox{nau-1.eps}
 }
\end{center} 
%\vspace*{-9pt}
\noindent
{\small Состояния процесса $\gamma(t)$, объема ресурсов, занятых 
в~ресурсной СМО~(\textit{а}) и~состояния процесса $\mathbf{w}(t)$, числа занятых 
каналов в~сети Келли~(\textit{б})}
\end{figure*}

\section{Связь ресурсных систем массового обслуживания с~сетями 
Эрланга}
  
  В дальнейшем будем считать, что функция распределения требуемых 
объемов ресурсов $F(\mathbf{x})$ является решетчатой с~некоторыми 
координатными шагами $\Delta_1, \Delta_2, \ldots, \Delta_M\hm>0$ и~положим 
$\mathbf{N}\hm= (N_1,N_2, \ldots ,N_M)$, где $N_m$~--- целая часть числа 
$R_m/\Delta_m$. В~этом случае векторы требуемых объемов 
ресурсов~$\mathbf{r}_j$ c~положительной вероятностью могут принимать 
лишь значения вида $\mathbf{Dk}\hm= (k_1\Delta_1, k_2\Delta_2, \ldots , 
k_M\Delta_M)$ где $\mathbf{k}\hm= (k_1, k_2, \ldots , k_M)$~--- целочисленный 
вектор, а~$\mathbf{D}$~--- диагональная матрица с~элементами~$\Delta_i$ на 
диагонали. Поэтому вместо вероятностей $P_k(\mathbf{x})$ удобнее 
рассматривать дискретное распределение вероятностей 
  \begin{multline}
    p_k(\mathbf{i}) =\lim\limits_{t\to\infty} {\sf P}\{\xi(t)=k; \enskip
\gamma(t)=\mathbf{Di}\}\,,\\
\mathbf{i}\in{\sf N}(\mathbf{N})\,,\enskip k=1,2,\ldots ,K\,,
  \label{e15-n}
 \end{multline}
зная которое, легко вычислить вероятности~(\ref{e9-n}):

\noindent
\begin{equation}
\left.
\begin{array}{rl}
p_0 &= p_0(\mathbf{0})\,;\\[6pt] 
P_k(\mathbf{x}) &= \sum\limits_{\substack{{\mathbf{i}\in{\sf 
N}(\mathbf{N})}\\ {\mathbf{Di}\leq \mathbf{x}}}} p_k(\mathbf{i})\,,\\[6pt] 
&\mathbf{0}\leq\mathbf{x}\leq\mathbf{R}\,,\enskip k=1,2,\ldots ,K\,.
\end{array}
\right\}
\label{e16-n}
\end{equation}
  
  Обозначим $f(\mathbf{i})$ вероятность того, что $j$-й заявке требуется 
вектор объемов ресурсов $\mathbf{r}_j\hm= \mathbf{Di}$, и~$b(\mathbf{i})$~--- 
условную среднюю длительность обслуживания \mbox{$j$-й} заявки при условии, что 
$\mathbf{r}_j\hm= \mathbf{Di}$. Перенумеруем все элементы множества ${\sf 
L}\hm= \{ \mathbf{i}\in {\sf N}(\mathbf{N})\vert f(\mathbf{i})>0\}$ и~обозначим 
$\mathbf{n}_l \hm= (n_{l1}, n_{l2},\ldots ,n_{lM})$ его $l$-й элемент. B~этих 
обозначениях функции распределения $F(\mathbf{x})$ и~$G(\mathbf{x})$ 
могут быть записаны следующим образом:
  \begin{equation}
  F(\mathbf{x}) = \sum\limits_{\substack{{\mathbf{i}\in{\sf L}}\\ 
{\mathbf{Di}\leq \mathbf{x}}}} f(\mathbf{i})\,;\enskip
  G(\mathbf{x}) =\fr{1}{b}\sum\limits_{\substack{{\mathbf{i}\in{\sf L}}\\ 
{\mathbf{Di}\leq \mathbf{x}}}} f(\mathbf{i}) b(\mathbf{i})\,.
  \label{e17-n}
 \end{equation}
Таким образом, $G(\mathbf{x})$ является функцией распределения некоторого 
случайного вектора, прини\-ма\-юще\-го значение $\mathbf{Di}$ с~вероятностью 
\begin{equation}
g(\mathbf{i}) = \fr{1}{b}\,f(\mathbf{i})b(\mathbf{i})\,,\enskip \mathbf{i}\in{\sf 
L}\,.
\label{e18-n}
\end{equation}
  
  Введем в~рассмотрение вспомогательную сеть Эрланга, поведение которой 
во времени синхронизировано с~поведением исходной ресурсной СМО. 
В~момент поступления в~СМО заявки, требующей вектор ресурсов 
$\mathbf{Dn}_l$ и~обслуживание в~течение времени~$\tau$, во 
вспомогательную сеть Эрланга поступает запрос на установление соединения 
класса~$l$ продолжительностью~$\tau$, а вектор~$\mathbf{n}_l$ задает 
требование к~числу каналов этого соединения. Моменты поступления заявок 
в~СМО являются моментами установления соединений в~сети, а моменты 
ухода заявок из СМО являются моментами разъединения соединений, и~только 
они. Кроме того, если поступившая в~СМО заявка теряется, то запрос на 
установление соответствующего соединения в~сети также теряется. Ясно, что 
запросы на установление во вспомогательной сети соединений $l$-го класса 
образуют пуассоновский поток интенсивности $\lambda_l\hm= \lambda 
f(\mathbf{n}_l)$, средняя продолжительность соединений $l$-го класса равна 
$b_l\hm= b(\mathbf{n}_l)$, а~распределение вероятностей~(\ref{e3-n}) 
совпадает с~распределением вероятностей~(\ref{e18-n}), т.\,е.\ $\pi(\mathbf{i}) 
\hm= g(\mathbf{i})$, $\mathbf{i}\hm\in {\sf L}$.
  


  Нетрудно видеть, что между процессами~$\xi(t)$ и~$\gamma(t)\hm= 
(\gamma_1(t), \ldots , \gamma_M(t))$, описыва\-ющи\-ми поведение ресурсной 
СМО, и~процессами $\mathbf{v}(t) \hm= (v_1(t), v_2(t), \ldots , v_L(t))$ 
и~$\mathbf{w}(t) \hm= \left(w_1(t), w_2(t), \ldots\right.$\linebreak 
$\left.\ldots, w_M(t)\right)$, описывающими 
вспомогательную сеть Эрланга, существует простая связь, 
проиллюстрированная на рисунке:
  \begin{equation*}
  \xi(t) =v_1(t)+v_2(t)+\cdots+ v_L(t)\,,\enskip \gamma(t)=\mathbf{Dw}(t)\,.
%  \label{e19-n}
  \end{equation*}
    Поэтому для распределений вероятностей~(\ref{e2-n}) и~(\ref{e15-n}) имеет 
место равенство: 
  \begin{equation*}
  p_k(\mathbf{i}) =\psi_k(\mathbf{i})\,,\enskip \mathbf{i}\in{\sf 
N}(\mathbf{N})\,,\enskip k=0,1,\ldots ,K\,,
%  \label{e20-n}
\end{equation*}
и распределение вероятностей~(\ref{e2-n}) можно записать в~следующем виде:
\begin{equation*}
p_k(\mathbf{i}) =p_0(\mathbf{0})g^{(k)} (\mathbf{i}) \fr{\rho^k}{k!}\,,\enskip 
\mathbf{i}\in{\sf N}(\mathbf{N})\,,\enskip k=0,1,\ldots, K\,,
%\label{e21-n}
\end{equation*}
где $g^{(k)}(\mathbf{i})$ есть $k$-крат\-ная свертка распределения 
вероятностей~(\ref{e18-n}). Используя равенства~(\ref{e6-n}) и~(\ref{e8-n}), 
вероятности простоя и~потери заявки можно выразить через нормировочные 
константы~(\ref{e5-n}):
\begin{equation*}
p_0=C(\mathbf{N})^{-1}\,;\enskip B=1-\fr{1}{C(\mathbf{N})}
\sum\limits_{\mathbf{i}\in{\sf N}(\mathbf{N})} C(\mathbf{i}) f(\mathbf{N}-
\mathbf{i})\,.
%\label{e22-n}
\end{equation*}
Кроме того, из формул~(\ref{e4-n}), (\ref{e5-n}), (\ref{e16-n}) и~(\ref{e17-n}) 
вытекают следующие выражения для стационарного распределения числа 
заявок в~системе и~объемов занимаемых ими ресурсов:
\begin{multline*}
p_0^{-1} =1+\sum\limits_{k=1}^K \fr{\rho^k}{k!} \sum\limits_{\mathbf{i}\in{\sf 
N}(\mathbf{N})} g^{(k)}(\mathbf{i}) ={}\\
{}=1+\sum\limits_{k=1}^K \fr{\rho^k}{k!} 
\sum\limits_{\substack{{\mathbf{i}\in{\sf N}(\mathbf{N})}\\ {\mathbf{Di}\leq 
\mathbf{R}}}} g^{(k)} (\mathbf{i}) =1+ \sum\limits_{k=1}^K G^{(k)} 
(\mathbf{R}) \fr{\rho^k}{k!}\,;
\end{multline*}

\vspace*{-12pt}

\noindent
\begin{multline*}
P_k(\mathbf{x}) = p_0 
\sum\limits_{\substack{{\mathbf{i}\in{\sf N}(\mathbf{N})}\\ {\mathbf{Di}\leq 
\mathbf{x}}}} g^{(k)} (\mathbf{i}) \fr{\rho^k}{k!} =p_0 G^{(k)} 
(\mathbf{x})\fr{\rho^k}{k!}\,,\\ k=1,2,\ldots ,K\,.
\end{multline*}

Таким образом, для решетчатых функций распределения требуемых объемов 
ресурсов $F(\mathbf{x})$ доказаны формулы~(\ref{e10-n}). 

\section{Заключение}

  В работе показано, что каждой СМО с~арифметической функцией 
распределения множественных ресурсов соответствует некоторая 
вспомогательная\linebreak сеть Эрланга. Поскольку стационарные распределения 
случайных процессов, описывающих ресурсную СМО и~соответствующую ей 
сеть Эрланга,\linebreak связаны простыми равенствами, для анализа ресурсных СМО 
имеется принципиальная возможность применения известных алгоритмов для 
анализа сетей Эрланга. 
  
  Соответствующие ресурсным СМО сети Эрланга имеют одну особенность, 
не характерную для типичных сетей Эрланга~--- число классов соединений 
вспомогательной сетей Эрланга может быть очень большим. Так, если 
$f(\mathbf{i})\hm>0$ для всех $\mathbf{i}\hm\in {\sf N}(\mathbf{N})$, то 
множество классов соединений сетей Эрланга состоит из всех целочисленных 
векторов $\mathbf{i}\hm= (i_1, i_2, \ldots , i_M)$ в~интервале $\mathbf{0}\hm\leq 
\mathbf{i}\hm\leq \mathbf{N}$ и~число классов соединений вспомогательной 
сети равно произведению $(N_1\hm+1)(N_2\hm+1)\cdots (N_M\hm+1)$. 
Поэтому необходимы дополнительные исследования для того, чтобы выяснить, 
какие точные и~приближенные методы анализа сетей Эрланга применимы 
к~ресурсным СМО.

{\small\frenchspacing
 {%\baselineskip=10.8pt
 \addcontentsline{toc}{section}{References}
 \begin{thebibliography}{9}
 \bibitem{3-n} %1
\Au{Ромм Э.\,Л., Скитович В.\,В.} Об одном обобщении задачи Эрланга~// Автоматика 
и~телемеханика, 1971. №\,6. С.~164--167.
\bibitem{4-n} %2
\Au{Тихоненко О.\,М.} Определение характеристик систем обслуживания с~ограниченной 
памятью~// Автоматика и~телемеханика, 1997. №\,6. С.~105--110.
\bibitem{1-n} %3
\Au{Наумов В.\,А., Самуйлов К.\,Е., Самуйлов~А.\,К.} О~суммарном объеме ресурсов, 
занимаемых обслуживаемыми заявками~// Автоматика и~телемеханика, 2016. №\,8.  
С.~105--110, 125--135.


\bibitem{5-n} %4
\Au{Тихоненко О.\,М., Климович К.\,Г.} Анализ систем обслуживания требований случайной 
длины при ограниченном суммарном объеме~// Проблемы передачи информации, 2001. 
Т.~37. Вып.~1. С.~78--88.
\bibitem{6-n} %5
\Au{Тихоненко О.\,М.} Обобщенная задача Эрланга для сис\-тем обслуживания 
с~ограниченным суммарным объемом~// Проблемы передачи информации, 2005. Т.~41. 
Вып.~3. С.~64--75.
\bibitem{2-n} %6
\Au{Naumov V., Samouylov K., Yarkina~N., Sopin~E., Andreev~S., Samuylov~A.} LTE 
performance analysis using queuing systems with finite resources and random requirements~// 7th 
Congress (International) on Ultra Modern Telecommunications and Control Systems ICUMT-2015  
Proceedings.~--- Piscataway, NJ, USA: IEEE, 2015. P.~100--103.
\bibitem{7-n}
\Au{Kelly F.\,P.} Loss networks~// Ann. App. Probab., 1991. Vol.~1. No.\,3. P.~319--378.
\bibitem{8-n}
\Au{Наумов В.\,А., Самуйлов К.\,Е., Гайдамака~Ю.\,В.} Мультипликативные решения 
конечных цепей Маркова.~--- М.: РУДН, 2015. 159~с.
\bibitem{9-n}
\Au{Гихман И.\,И., Скороход А.\,В.} 
Теория случайных процессов.~--- М.: Наука, 1971.  Т.~1. 664~с.
\end{thebibliography}

 }
 }

\end{multicols}

\vspace*{-3pt}

\hfill{\small\textit{Поступила в~редакцию 29.07.16}}

%\vspace*{8pt}

\newpage

\vspace*{-24pt}

%\hrule

%\vspace*{2pt}

%\hrule

%\vspace*{-24pt}


\def\tit{ON RELATIONSHIP BETWEEN QUEUING SYSTEMS WITH~RESOURCES 
AND~ERLANG NETWORKS}

\def\titkol{On relationship between queuing systems with~resources 
and~Erlang networks}

\def\aut{V.\,A.~Naumov$^1$ and K.\,E.~Samouylov$^{2, 3}$}

\def\autkol{V.\,A.~Naumov and K.\,E.~Samouylov}

\titel{\tit}{\aut}{\autkol}{\titkol}

\vspace*{-9pt}

\noindent
$^1$Service Innovation Research Institute, 30~D L$\ddot{\mbox{o}}$nnrotinkatu, Helsinki 00180, Finland

\noindent
$^2$Peoples' Friendship University of Russia, 6~Miklukho-Maklaya Str., Moscow 117198, Russian Federation

\noindent
$^3$Institute of Informatics Problems, Federal Research Center ``Computer Science and Control'' of the Russian\linebreak 
$\hphantom{^1}$Academy of Sciences, 44-2~Vavilov Str., Moscow 119333, Russian Federation


\def\leftfootline{\small{\textbf{\thepage}
\hfill INFORMATIKA I EE PRIMENENIYA~--- INFORMATICS AND
APPLICATIONS\ \ \ 2016\ \ \ volume~10\ \ \ issue\ 3}
}%
 \def\rightfootline{\small{INFORMATIKA I EE PRIMENENIYA~---
INFORMATICS AND APPLICATIONS\ \ \ 2016\ \ \ volume~10\ \ \ issue\ 3
\hfill \textbf{\thepage}}}

\vspace*{3pt}


\Abste{The paper considers a model of a multiserver queuing system (QS) with 
losses caused by the lack of resources required to service customers. During its 
service, each customer occupies a particular amount of resources of several types. 
Random vectors, describing the requirements of customers to resources, do not 
depend on the arrival process and service times and are mutually independent and 
identically distributed with the general cumulative distribution function. Like in the 
Erlang problem, the task is to calculate the probability of losses of an arriving 
customer caused by the lack of resources. The paper shows the relationship between 
multiservice loss networks and queuing systems with resources, which makes it 
possible to solve the problem of calculating the loss probability in the queuing 
systems with resources using known methods developed for multiservice loss 
networks.}

\KWE{multiservice network; Erlang network; queuing system; queuing system with 
resources; random amount of resources; loss probability; arithmetic probability 
distribution}



\DOI{10.14357/19922264160302}

\vspace*{-9pt}

\Ack
\noindent
The work was partly supported by the Russian Foundation for
Basic Research (projects
16-07-00766, 15-07-03051,   and 15-07-03608).


%\vspace*{3pt}

  \begin{multicols}{2}

\renewcommand{\bibname}{\protect\rmfamily References}
%\renewcommand{\bibname}{\large\protect\rm References}

{\small\frenchspacing
 {%\baselineskip=10.8pt
 \addcontentsline{toc}{section}{References}
 \begin{thebibliography}{9}


\bibitem{3-n-1} %1
\Aue{Romm, E., and V. Skitovitch}. 1971. On certain generalization of problem of Erlang. 
\textit{Automation Remote Control} 32(6):1000--1003.
\bibitem{4-n-1} %2
\Aue{Tikhonenko, O.} 1997. The determination of service characteristics under limited memory. 
\textit{Automation Remote Control} 58(6):969--973.
\bibitem{1-n-1} %3
\Aue{Naumov, V., K.~Samouylov, and A.~Samuylov}. 2016. On the total amount of resources 
occupied by serviced customers. \textit{Automation Remote Control} 77(8): 1419--1427.
\bibitem{5-n-1} %4
\Aue{Tikhonenko, O., and K. Klimovitch}. 2001. Analysis of queuing systems for random-length 
arrivals with limited cumulative volume. \textit{Problems Information Transmission}  
37(1):70--79.
\bibitem{6-n-1} %5
\Aue{Tikhonenko, O.} 2005. Generalized Erlang problem for queueing systems with bounded total 
size. \textit{Problems Information Transmission} 41(3):243--253.
\bibitem{2-n-1} %6
\Aue{Naumov, V., K. Samouylov, N.~Yarkina, E.~Sopin, S.~Andreev, and A.~Samuylov}. 2015. 
LTE performance analysis using queuing systems with finite resources and random requirements. 
\textit{7th  Congress (International) on Ultra Modern Telecommunications and Control Systems 
ICUMT-2015 Proceedings}. Piscataway, NJ: IEEE. 100--103.
\bibitem{7-n-1}
\Aue{Kelly, F.\,P.} 1991. Loss networks. \textit{Ann. App. Probab.} 1(3):319--378.
\bibitem{8-n-1}
\Aue{Naumov, V., Yu.~Gaidamaka, and K.~Samouylov}. 2015. Mul'tiplikativnye resheniya 
konechnykh tsepey Markova [Product form solutiuons for finite Markov' chains]. Moscow: RUDN. 
159~p.
\bibitem{9-n-1}
\Aue{Gihman, I., and A.~Skorohod}. 1971. \textit{The theory of stochastic processes}. New  
York\,--\,Heidelberg\,--\,Berlin: Springer-Verlag. Vol.~I, 1974, 574~p. 
   \end{thebibliography}

 }
 }

\end{multicols}

\vspace*{-9pt}

\hfill{\small\textit{Received July 29, 2016}}

\vspace*{-12pt}


\Contr

\noindent
\textbf{Naumov Valeriy A.} (b.\ 1950)~--- Candidate of Science (PhD) in physics and 
mathematics, Research Director, Service Innovation Research Institute, 30~D 
L$\ddot{\mbox{o}}$nnrotinkatu, Helsinki 00180, Finland; \mbox{valeriy.naumov@pfu.fi} 

\vspace*{3pt}

\noindent
\textbf{Samouylov Konstantin E.} (b.\ 1955)~--- Doctor of Science in technology, 
professor, 
Head of Department, Peoples' Friendship University of Russia, 6~Miklukho-Maklaya Str., 
Moscow 117198, Russian Federation; Institute of Informatics Problems, Federal Research 
Center ``Computer Science and Control'' of the Russian Academy of Sciences, 44-2~Vavilov 
Str., Moscow 119333, Russian Federation; \mbox{ksam@sci.pfu.edu.ru}
\label{end\stat}


\renewcommand{\bibname}{\protect\rm Литература}