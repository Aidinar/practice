\def\stat{ometov}

\def\tit{АНАЛИЗ ПРОИЗВОДИТЕЛЬНОСТИ БЕСПРОВОДНОЙ 
СИСТЕМЫ АГРЕГАЦИИ ДАННЫХ С~СОСТЯЗАНИЕМ 
ДЛЯ~СОВРЕМЕННЫХ СЕНСОРНЫХ СЕТЕЙ$^*$}

\def\titkol{Анализ производительности беспроводной 
системы агрегации данных с~состязанием} 
%для современных сенсорных сетей}

\def\aut{А.\,Я.~Омётов$^1$, С.\,Д.~Андреев$^2$, А.\,М.~Тюрликов$^3$, 
Е.\,А.~Кучерявый$^4$}

\def\autkol{А.\,Я.~Омётов, С.\,Д.~Андреев, А.\,М.~Тюрликов, 
Е.\,А.~Кучерявый}

\titel{\tit}{\aut}{\autkol}{\titkol}

\index{Омётов А.\,Я.}
\index{Андреев С.\,Д.}
\index{Тюрликов А.\,М.} 
\index{Кучерявый Е.\,А.}
\index{Ometov A.\,Ya.}
\index{Andreev S.\,D.}
\index{Turlikov A.\,M.}
\index{Koucheryavy E.\,A.}


{\renewcommand{\thefootnote}{\fnsymbol{footnote}} \footnotetext[1]
{Исследование выполнено при частичной финансовой поддержке РФФИ (проект 15-07-03051), 
а~также 
Фонда содействия развитию малых форм предприятий в~на\-уч\-но-тех\-ни\-че\-ской сфере в~рамках 
программы <<УМНИК>>  по договору №\,8268ГУ2015 от~02.12.2015.}}


\renewcommand{\thefootnote}{\arabic{footnote}}
\footnotetext[1]{Санкт-Петербургский государственный университет телекоммуникаций им.\ М.\,А.~Бонч-Бруевича, 
\mbox{alexander.ometov@gmail.com}}
\footnotetext[2]{Российский университет дружбы народов, \mbox{serge.andreev@gmail.com}}
\footnotetext[3]{Санкт-Петербургский государственный университет аэрокосмического приборостроения, 
\mbox{turlikov@vu.spb.ru}}
\footnotetext[4]{Национальный исследовательский университет <<Высшая школа экономики>>, 
\mbox{ykoucheryavy@hse.ru}}

\vspace*{-12pt}


\Abst{Рассматривается беспроводная система связи, 
учитывающая особенности современных сенсорных сетей, в~которых 
устройства передают свои данные на множество промежуточных 
агрегирующих узлов, имеющих подключение к~сети Интернет по 
технологии IEEE 802.11-2014 (WiFi). Предполагается, что агрегатор 
осуществляет пересылку данных от многих сенсоров, участвуя при этом 
в~состязании за общий канал связи с~другими агрегаторами. Предлагается 
аналитическая модель такого состязания, учитывающая специфику 
алгоритма разрешения коллизий, характеристики протокола доступа 
к~каналу, а~также возможность потери данных на узле агрегации. 
Полученные аналитические результаты сопоставляются с~данными 
имитационного моделирования, и~вычисляется максимальное количество 
поддерживаемых системой связи сенсоров.}

\KW{Интернет вещей; беспроводные сенсорные сети; регенеративный 
анализ; БЛВС; стандарт IEEE 802.11-2014}

 \DOI{10.14357/19922264160304} 
  


\vskip 10pt plus 9pt minus 6pt

\thispagestyle{headings}

\begin{multicols}{2}

\label{st\stat}
    
  \section{Введение}
     
    В последнее время все более усиливается влияние беспроводных 
технологий на современное общество, что, в~свою очередь, предвещает 
рост научного интереса к~данной тематике в~ближайшие\linebreak годы и~влечет за 
собой потенциальную возможность установления беспроводного 
соединения в~любом месте и~в любое время~[1]. Данная возможность 
является привлекательной для внедрения концепции <<Интернета 
вещей>> (Internet of Things, IoT)~[2]. При интеграции IoT многие 
устройства могут быть оборудованы сенсорами и~расширяющими 
модулями, с~помощью которых появляется возможность обрабатывать 
и~передавать информацию без вмешательства человека. Данные 
беспроводные технологии открывают дорогу для широкого спектра 
сервисов, начиная с~удаленного наблюдения и~заканчивая 
здравоохранением. 

Основной целью текущих исследований является 
разработка системы связи, использующей подходящие  
IoT-тех\-но\-ло\-гии для реализации подключения разнородных сенсоров. 
{\looseness=-1

}
    
    Исторически беспроводные сети разрабатывались для использования 
людьми, а для использования их машинами необходима значительная 
оптимизация, что определяет необходимость разработки или улучшения 
технологий связи для поддержки большого числа устройств~[3]. 
Основными требованиями к~таким технологиям остаются низкая 
сложность обработки данных, дешевизна в~производстве и~высокая 
энергоэффективность. 
    
    Беспроводные локальные сети на основе стандарта IEEE 802.11 
(WiFi) являются одним из самых распространенных технических 
решений для ор\-ганизации беспроводного доступа в~домах и~на 
предприятиях. Благодаря их высокой пропускной спо\-соб\-ности, 
относительно низкой сто\-и\-мости и~повсеместной распространенности 
использование WiFi для сценариев IoT является все более 
привлекательным. 

\begin{figure*}[b] %fig1
\vspace*{1pt}
 \begin{center}  
\mbox{%
 \epsfxsize=114.889mm
 \epsfbox{ome-1.eps}
 }
\end{center} 
\vspace*{-9pt}
\Caption{Предполагаемая топология беспроводной сети}
     \end{figure*}

В~данной работе производится исследование 
принципиальной возможности и~эффективности использования 
современной технологии WiFi с~учетом ее технических характеристик 
для типовых сценариев IoT. В~частности, предлагается аналитическая 
модель для учета особенностей работы технологии WiFi (состязание 
между агрегаторами, особенности протокола доступа, режимы передачи 
данных и~т.\,д.), основанная на теории регенерирующих процессов. 
Аналитические результаты сопоставляются с~данными, полученными 
имитационным моделированием, и~делается вывод о наибольшем 
возможном количестве сенсоров, поддерживаемых такой системой связи.

\vspace*{-6pt}
   
   \section{Модель системы и~анализ}
   
   \vspace*{-6pt}
   
\subsection{Описание сценария и~протокола}

\vspace*{-1pt}
    
    В данной работе рассматривается изолированный сегмент (кластер) 
беспроводной сети для IoT-при\-ло\-же\-ний со статичным размещением~$M$ 
агрегирующих узлов, в~котором отсутствуют <<скрытые>> станции. 
Топология данной сети представлена на рис.~1. 

Агрегаторы оборудованы 
двумя модулями беспроводной связи: WiFi и~ZigBee. Сенсоры передают 
собранную ими информацию на агрегаторы посредством ZigBee, и~далее 
поток данных перенаправляется в~канал WiFi, соединяющий агрегаторы 
с~сетью Интернет. Агрегаторы взаимодействуют в~нелицензированном 
частотном диапазоне WiFi и~используют протокол случайного 
множественного доступа для передачи накопленных данных в~общий 
канал связи. При передаче более чем от одного агрегатора единовременно 
возникает наложение таких передач в~канале~--- коллизия. В~случае если 
передавало только одно устройство, передача считается успешной 
в~предположении, что в~канале отсутствует шум. Третье возможное 
событие~--- пус\-той слот~--- происходит в~случае, если ни один из 
агрегаторов не осуществлял передачу. В~то же время известные решения 
для сотовых сетей в~данной работе рассматриваться не будут~[4].
    

     
    В данной работе рассматривается система среднего/большого 
производства, где в~каждом помещении находится один агрегатор, 
обслуживающий не более~20~узлов. Интерференция и~конфликты 
в~канале сен\-сор--аг\-ре\-га\-тор для данного исследования не являются 
критическими, так как алгоритмы беспроводной связи ближнего радиуса 
действия не подвержены интерференции от кластеров из соседних 
помещений, в~то время как агрегаторы могут конфликтовать в~связи 
с~более широким радиусом действия беспроводного покрытия. 

Таким 
образом, в~работе предполагается худший случай насыщенного трафика 
между всеми агрегирующими узлами~[5]. Иными словами, на уровне 
управления доступом к~среде можно наблюдать целиком заполненный 
буфер исходящих сообщений. Более высокие уровни модели связи 
в~данной работе не рассматриваются ввиду предположения о~прос\-то\-те 
сенсоров. Данное допущение дает возможность оценивать наихудший 
сценарий загрузки сети и~поз\-во\-ля\-ет производить оценку с~точки зрения 
пропускной способности насыщения~$S$.
    
    Согласно спецификации стандарта IEEE 802.11, процесс доступа 
агрегаторов к~общему каналу связи основан на алгоритме двоичной 
экспоненциальной отсрочки (ДЭО) и~состоит в~выборе случайного 
интервала отсрочки перед передачей из некоторого окна отсрочки 
(CW). В~процессе работы алгоритма ДЭО в~случае передачи 
единовременно двумя или более агрегаторами происходит коллизия. Она 
отслеживается на стороне получателя (точки доступа), а~каждому из 
передающих абонентов пред\-остав\-ля\-ет\-ся возможность повторной 
передачи, если это позволит сделать счетчик повторных передач (RC).


\pagebreak


%\begin{table*}
\noindent
{\small
\begin{center}

\begin{tabular}{|l|c|}
\multicolumn{2}{c}{Основные обозначения, использованные в~работе}\\
\multicolumn{2}{c}{\ }\\[-4pt]
\hline
\multicolumn{1}{|c|}{Параметр}&Обозначение\\
\hline
Максимальная длительность доступа&$T_{\mathrm{TXOP}}$\\
Количество агрегаторов&$M$\\
Текущее окно отсрочки&$W_i$, CW\\
Начальное окно отсрочки&$W_0$\\
Длительность пустого слота&$\Sigma$\\
Длительность AIFS&$T_{\mathrm{AIFS}}$\\
Длительность SIFS&$T_{\mathrm{SIFS}}$\\
Длительность BA&$T_{\mathrm{BA}}$\\
Длительность RTS&$T_{\mathrm{RTS}}$\\
Длительность CTS&$T_{\mathrm{CTS}}$\\
Длительность CFE&$T_{\mathrm{CFE}}$\\
Средняя длительность блока данных&$T_{E[P]}$\\
Длительность передачи преамбулы $P$&$T_P$\\
\hline
\end{tabular}
\end{center}}

\vspace*{18pt}
%\end{table*}

\noindent 
В~данном случае значение окна отсрочки CW будет удвоено 
($W_{i+1}\hm= 2W_i$) с~целью уменьшения вероятности повторной 
коллизии. В~то же время будет уменьшен счетчик повторных передач 
RC. Возможный рост~CW ограничен максимальным значением 
($\mathrm{CW}_{\max}\hm= 2^mW_0$), где~$m$ определяется как <<степень>> 
отсрочки. Основные используемые в~работе сокращения представлены 
в~таблице.
    

    
    Данная схема передачи может использовать два альтернативных 
механизма доступа, подробно описанных в~спецификации IEEE 802.11. 
При использовании механизма базового доступа в~канал (Basic) пакет 
данных (либо агрегированная группа пакетов с~единой преамбулой~$P$) 
передается незамедлительно после ожидания регуляционного 
межкадрового интервала (AIFS) и~случайного времени отсрочки 
(BOT). Информация об успешной передаче пакета/блока данных 
передается в~блоковом под\-тверж\-де\-нии (BA). Механизм доступа 
<<запрос на от\-прав\-ку\,/\,раз\-ре\-ше\-ние отправки>> (RTS/CTS) 
использует алгоритм <<двукратного рукопожатия>> при передаче 
сообщения. Иными словами, канал резервируется за определенным 
агрегатором на время его предполагаемой передачи.
    
    В результате агрегации пакетов данных на физическом уровне, на 
высоких скоростях передачи влияние служебной информации на 
пропускную способность становится менее значительно. Однако 
длительность нового пакета данных после агрегирования не должна 
превышать интервала захва\-та среды передачи (TXOP), включая 
необходимые межкадровые интервалы (SIFS), блоковое под\-тверж\-де\-ние 
(BA) и~(опционально) RTS/CTS. При не\-об\-хо\-ди\-мости агрегатор может 
также преждевременно закончить передачу, в~случае если длительность 
передачи пакета данных оказалась меньше TXOP, отправив сообщение об 
освобождении канала (CFE).
    
    Согласно стандарту после первого интервала AIFS значение 
счетчика отсрочки (BC) выбирается как равномерно распределенная 
величина~$W_i$ в~промежутке между~0 и~$W_0\hm-1$, где~$W_i$ 
является окном отсрочки~CW. После каждого пустого слота~BC 
уменьшается на единицу. Как только~BC достигает нуля, выбранный 
агрегатор пытается передать. При возникновении коллизии происходит 
повторная передача, в~случае если счетчик повторных передач~RC не 
равен нулю. В~случае повторной передачи $W_i\hm= 2W_{i-1}$. Рост 
CW также ограничен максимальным значением~$W_{\max}$, но 
агрегатор может продолжать передавать повторно, пока~RC не 
достигнет нуля. В~момент, когда пакет успешно передан или принято 
решение об отказе от передачи, CW устанавливается в~некоторое 
начальное значение~$W_0$. Эквивалентно CW$_{\max}\hm= 2^m W_0$, 
где $m$~--- степень отсрочки.
    
    Далее представлен анализ работы системы доступа к~каналу, 
функционирующей на основе алгоритма ДЭО с~потерями согласно 
описанию выше. В~работе~[6] предложно рассматривать два параметра 
функционирования системы на основании циклов регенерации: 
вероятность передачи~$p_t$ пользователем в~конкретный временной слот 
и~вероятность ошибки~$p_c$ в~случае осуществленной передачи. 
Данные параметры предполагаются неизменными на всем протяжении 
работы насыщенной системы. Вследствие данного допущения весь 
кластер может быть рассмотрен с~точки зрения одного <<меченого>> 
абонента этой системы. Влияние всех прочих факторов учитывается 
в~значении вероятности ошибки~$p_c$. Необходимо отметить, что 
данная замена допустима только в~случае справедливой системы, т.\,е.\ 
когда все пользователи получают приблизительно равную долю времени 
доступа к~каналу~\cite{7-om}.

    
\subsection{Общие понятия для~системы без~потерь}
    
    Рассмотрим модель, представленную на рис.~2, основываясь на 
концепции \textit{циклов регенерации}. 
    
\begin{figure*} %fig2
\vspace*{1pt}
 \begin{center}  
\mbox{%
 \epsfxsize=136.118mm
 \epsfbox{ome-2.eps}
 }
\end{center} 
\vspace*{-9pt}
\Caption{Упрощенная модель алгоритма ДЭО}
     \end{figure*}
     
    Данная модель рассматривает равные временн$\acute{\mbox{ы}}$е слоты, где начало 
передачи пакета совпадает с~началом слота. Каждая передача занимает 
в~точности один слот. Подобное упрощение типично для моделирования 
протоколов случайного множественного доступа~\cite{9-om} 
и~позволяет легко масштабировать модель согласно требуемым 
временн$\acute{\mbox{ы}}$м характеристикам конкретного протокола, что будет показано 
в~следующем подразделе. Отметим, что меченый агрегатор имеет 
сле\-ду\-ющую вероятность конфликта в~произвольно взятом слоте:

\pagebreak

\noindent
    \begin{equation}
    p_c=1-(1-p_t)^{M-1}\,.
    \label{e1-om}
    \end{equation}
    %
    Важно также обратить внимание, что вероятность передачи в~канал 
для данного агрегатора может быть рассчитана как отношение числа 
попыток передачи на пакет $B^{(i)}$ в~течение некоторого цикла 
регенерации к~длительности данного цикла в~слотах~$D^{(i)}$:
    \begin{equation}
    p_t=\lim\limits_{n\to\infty} \fr{\sum\nolimits_{i=1}^n 
B^{(i)}}{\sum\nolimits_{i=1}^n D^{(i)}}=\fr{E[B]}{E[D]}\,.
    \label{e2-om}
    \end{equation}
    
    Предполагая, что система находится в~насыщении и~не имеет потерь, 
можно с~легкостью получить~$E[B]$:
    \begin{multline}
    E[B]= \sum\limits_{i=1}^\infty \mathrm{Pr}\,\{B=i\} ={}\\
    {}=\left(1-p_c\right) 
\sum\limits_{i=1}^\infty ip_c^{i-1} =\fr{1}{1-p_c}\,.
    \label{e3-om}
    \end{multline}
    
    Выражение для $E[D]$ может быть получено аналогичным 
способом~\cite{10-om}:
    \begin{multline}
    E[D]= \sum\limits_{i=1}^\infty  D^{(i)} \mathrm{Pr}\,\{D=i\}={}\\
    {}=\left(1-p_c\right) 
\sum\limits_{i=1}^\infty D^{(i)} p_c^{i-1}\,,
    \label{e4-om}
    \end{multline}
где $D^{(i)}$~--- длина цикла регенерации при условии, что было 
произведено~$i$~попыток передачи.

    Основываясь на зависимости~$i$ и~$m$, можно также получить 
следующие правила вычисления величины $D^{(i)}$:
    \begin{equation}
    D^{(i)}= \begin{cases}
    2^{i-1}W_0-\fr{W_0-1}{2}\,, &\\
    &\hspace*{-30mm} \mbox{если } 1\leq i\leq m+1\,;\\
    2^{m-1} W_0(i-m+1) - \fr{W_0-i}{2}\,, &\\
    &\hspace*{-30mm} \mbox{если } i>m+1\,.
    \end{cases}
    \label{e5-om}
    \end{equation} 
    
    Подставив~(\ref{e5-om}) в~(\ref{e4-om}), после преобразования 
получим:

\noindent
    \begin{multline}
    E[D]={}\\
    {}= \fr{(1-2p_c) (W_0+1) +p_c W_0 (1-(2p_c)^m)}{2(1-2p_c) (1-
p_c)}\,.
    \label{e6-om}
    \end{multline}
    
    Далее, подставляя~(\ref{e3-om}) и~(\ref{e6-om}) в~(\ref{e2-om}), 
имеем:
    \begin{equation}
    p_t= \fr{2(1-2p_c)} {(1-2p_c) (W_0+1) +p_c W_0 (1-(2p_c)^m)}\,.
    \label{e7-om}
    \end{equation}
    
    Следует отметить, что аналогичные результаты были получены 
в~известной работе~\cite{8-om}, где аналитические расчеты были 
основаны на двумерных цепях Маркова, которые сложно 
масштабировать на случай дополнительных параметров системы. Здесь 
же использован иной математический подход, яв\-ля\-ющий\-ся более 
простым, но не менее эффективным. Выражения~(\ref{e1-om})  
и~(\ref{e7-om}) составляют систему двух нелинейных уравнений 
с~неизвестными~$p_c$ и~$p_t$, которую можно решить численно. 
    
\subsection{Система с~потерями}
     
    В данном подразделе представлен анализ ис\-ходной системы, 
работающей с~потерями, т.\,е.\linebreak с~определенным максимальным числом 
повторных передач~$K$ для отдельно взятого пакета. Для 
вы\-чис\-ле\-ния~$p_t$ в~данной системе определяем $E[B]$ и~$E[D]$ 
согласно~(\ref{e2-om}), но выражение для расчета среднего числа 
попыток передачи в~цикле $E[B]$ необходимо модифицировать 
следующим образом:
    \begin{multline}
    E[B]= \sum\limits_{i=1}^{K+1} i Pr\{B=i\} = {}\\
    {}=\left(1-p_c\right) 
\sum\limits_{i=1}^{K+1} i p_c^{i-1} (K+1) p_c^{K+1} = \fr{1-
p_c^{K+1}}{1-p_c}\,.
    \label{e8-om}
    \end{multline}
    
    Далее вычисляем среднюю длительность цикла регенерации как
    \begin{multline*}
    E[D]= \sum\limits_{i=1}^{K+1} D(i) \mathrm{Pr}\,\{B=i\} = {}\\
    {}=\left(1-p_c\right) 
\sum\limits_{i=1}^{K+1} D(i) p_c^{i-1} +D(K+1) p_c^{K+1}\,.
%    \label{e9-om}
    \end{multline*}
    
    Возможны две ситуации, зависящие от соотношения~$m$ и~$K$: 
когда $K\hm\leq m$ и~$K\hm>m$. В~первом случае
    \begin{multline}
    E\left[ D^\prime\right] = (1-p_c) \left[ \sum\limits_{i=1}^{K+1}\! \left( 
2^{i-1} W_0 -\fr{W_0-2}{2}\right) p_c^{i-1}\right] +{}\\
{}+
    p_c^{K-1} \left( 2^K W_0 -\fr{W_0-(K+1)}{2}\right)\,.
    \label{e10-om}
    \end{multline}
    
    Соответственно, вероятность выхода в~канал~$p^\prime_t$ может 
быть получена при расчете выражения~(\ref{e2-om}) в~результате 
подстановки~$E[B]$ из~(\ref{e8-om}) и~$E[D]$ (в~данном случае 
$E[D^\prime]$) из~(\ref{e10-om}):
    \begin{multline}
    p^\prime_t = 2(1-2p_c) \left(1-2p_c^{K+1}\right)\!\Big /\!
    \left[ \left(1-2p_c\right) 
\left( \vphantom{\left(2p_c\right)^{K+1}}
1-{}\right.\right.\\
\hspace*{-5mm}\left.\left.{}-\left(2p_c\right)^{K+1}\right) 
+ W_0 \left(1-p_c\right) \left(1-\left(2p_c\right)^{K+1}\right)\right]\,.
    \label{e11-om}
    \end{multline}
    
    Чтобы рассчитать второй случай (когда $K\hm>m$), необходимо 
получить соответствующее значение~$E[D^{\prime\prime}]$:
    \begin{multline}
   \! E[D^{\prime\prime}] = (1-p_c) \left[ \sum\limits_{i=1}^{m+1} \left( 
2^{i-1}W_0 -\fr{W_0-i}{2}\right) p_c^{i-1}+ {}\right.\\
\left.{}+
    \sum\limits_{i=1}^{m+1} \left( 2^{i-1}W_0(i-m+1)-\fr{W_0-i}{2}\right) 
p_c^{i-1}\right] + {}\\
{}+p_c^{K-1} \left( 2^KW_0 - \fr{W_0-(K+1)}{2}\right)\,.
\label{e12-om}
    \end{multline}
     
     Аналогично можно получить вероятность 
передачи~$p^{\prime\prime}_t$, подставляя $E[B]$ из~(\ref{e8-om}) 
и~$E[D]$ (в данном случае $E[D^{\prime\prime}]$) из~(\ref{e12-om}) 
в~(\ref{e2-om}):
    \begin{multline}
    p^{\prime\prime}_t =   2\left(1-2p_c\right)\left(1-2p_c^{K+1}\right)\Big / 
    \left[ \vphantom{\left(2p_c\right)^m}
    (1-2p_c)\times{}\right.\\
   {}\times 
\left(W_0\left(1-\left(2p_c\right)^{K+1}\right)\right)+
\left(1-\left(2p_c\right)^{K+1}\right) +{}\\
\left. {}+W_0 p_c\left(1-\left(2p_c\right)^m\right)\right]\,.
    \label{e13-om}
    \end{multline}
    
    Итак, вероятность выхода в~канал~$p_t$ в~системе с~потерями может 
быть получена двумя способами: как~$p^\prime_t$ из~(\ref{e11-om}) или 
как~$p_t^{\prime\prime}$ из~(\ref{e13-om}), в~зависимости от 
соотношения~$K$ и~$m$. Решая нелинейную систему 
уравнений~(\ref{e8-om}), получаем итоговое значение~$p_t$.
    
    Также в~данной работе были использованы фактические 
длительности служебных сообщений\linebreak
 WiFi для вычисления пропускной 
способности насыщения, которая может быть воспроизведена по 
аналогии с~\cite{9-om, 8-om, 11-om} и~множеством других работ. Основное 
отличие от вышеприведенной модели заключается в~том, что 
используются разные длительности слотов для различных событий 
в~канале, которые соответствуют текущей спецификации стандарта IEEE 
802.11-2014. Далее рассматривается работа механизма доступа 
<<RTS/CTS>>. Размеры слотов разной длительности следующие: 
$\sigma$ соответствует длительности пустого слота, $T_S$~--- 
длительности успешной передачи, а $T_C$~--- длительности коллизии. 
Длительности успеха и~коллизии могут быть рассчитаны 
согласно~\cite{7-om} как
    \begin{align*}
       \hspace*{-2mm}T_S &= T_{\mathrm{RTS}}+ T_{\mathrm{SIFS}}+ T_{\mathrm{CTS}}+ T_{\mathrm{SIFS}} +{}       \hspace*{2mm}\\
    &\hspace*{13mm}{}+T_P+ T_{E[P]}  +T_{\mathrm{SIFS}} +T_{\mathrm{BA}}+T_{\mathrm{AIFS}}\,,       \\
          \hspace*{-2mm} T_C &= T_{\mathrm{RTS}} +T_{\mathrm{AIFS}}\,.       \hspace*{2mm}
   %    \label{e14-om}
    \end{align*}
    
    В итоге можно получить пропускную способность насыщения~$S$:
    \begin{equation*}
    S= \fr{P_t P_S E[P]}{(1-P_t)\sigma +P_t P_S T_S +P_t(1-P_S) T_C}\,,
%    \label{e15-om}
    \end{equation*}
где $P_t= 1-(1\hm- p_t)^M$~--- вероятность того, что в~системе данные 
передавал только один агрегатор, $P_S\hm= Mp_t(1\hm- p_t)^{M-1}P_t^{-
1}$~--- вероятность успешной передачи (при условии, что передавал 
один агрегатор). 

    В системе с~потерями можно также выписать вероятность того, что 
агрегатор отказывается от передачи пакета после~$K$~неуспешных 
повторных передач, вызванных коллизиями: $P_d\hm= p_c^{K-1}$.

\section{Результаты и~выводы}
     
    В данном разделе представлены результаты моделирования для 
современной версии протокола IEEE~802.11. Также рассмотрено их 
сопоставление с~аналитическими результатами, полученными выше. 
Реализованная система имитационного моделирования является гибким 
программным инструментом, включающим различные сценарии 
взаимодействия WiFi-агре\-га\-то\-ров, а также набор необходимых 
механизмов для управления доступом к~среде. Система моделирования 
была откалибрована по результатам, представленным 
    в~работе~\cite{8-om} (зависимость пропускной способности~$S$ от 
количества абонентов~$M$ в~канале). Для этого был оцифрован 
соответствующий график и~произведено наложение полученных 
результатов на воспроизведенные данные. Результаты можно наблюдать 
на рис.~3,\,\textit{а}. 

\begin{figure*} %fig3
\vspace*{1pt}
 \begin{center}  
\mbox{%
 \epsfxsize=162.694mm
 \epsfbox{ome-3.eps}
 }
\end{center} 
\vspace*{-9pt}
\Caption{Пропускная способность насыщения при $T\hm= 1$~(\textit{а}) и 65~Мбит/с~(\textit{б}) 
и~$K\hm\to\infty$: \textit{1}~--- Basic, $W_0=32$, $m=3$;
 \textit{2}~--- Basic, $W_0=32$, $m=5$;
      \textit{3}~--- Basic, $W_0=128$, $m=3$;
      \textit{4}~--- RTS/CTS, $W_0=32$, $m=3$; 
      \textit{5}~--- RTS/CTS, $W_0=128$, $m=3$}
      \vspace*{8pt}
      \end{figure*}
           \begin{figure*}[b] %fig4
     \vspace*{8pt}
 \begin{center}  
\mbox{%
 \epsfxsize=103.26mm
 \epsfbox{ome-5.eps}
 }
\end{center} 
\vspace*{-9pt}
\Caption{Максимальное число сенсоров с~агрегацией в~канале WiFi: 
\textit{1}~--- Basic, $W_0=128$, $m=3$;
      \textit{2}~--- RTS/CTS, $W_0=32$, $m=3$; 
      \textit{3}~--- RTS/CTS, $W_0=128$, $m=5$}
      \end{figure*}
     
    Следует отметить, что результаты работы~\cite{8-om} 
и~проведенного в~данной работе моделирования совпадают 
в~аналогичных условиях и~с~максимальной скоростью передачи 
$T\hm=1$~Мбит/с (система без потерь). 
    
    Далее рассмотрим сценарий, в~котором~$M$~агрегаторов в~кластере 
используют реалистичные длительности слотов согласно текущей 
спецификации IEEE 802.11-2014.\ Для этого установим более вы\-сокую 
скорость передачи $T\hm=65$~Мбит/с и~сопоставим анализ (кривые) 
с~результатами моделирования (символы) на рис.~3,\,\textit{б}.

    
    Рассмотрение системы, работающей без потерь (в~которой 
количество попыток повторной передачи не ограничено), воссоздает 
ситуацию, когда~$M$~агрегаторов отправляют сообщения к~точке 
доступа WiFi в~насыщенном режиме. С~по\-мощью\linebreak предлагаемого 
подхода получена максимально достижимая пропускная способность 
сис\-те\-мы. В~целом зависимость на рис.~3,\,\textit{б} аналогична результатам, 
приведенным на рис.~3,\,\textit{а}, но значение\linebreak 
достижимой пропускной 
способности в~случае $T\hm=65$~Мбит/с значительно выше. 
    
    В заключение рассмотрим систему предложенной топологии (см.\ 
рис.~1) с~различным числом агрегаторов (5--16) и~определим 
максимально возможное число сенсоров, которые могут быть обслужены 
такой системой.
 
Результаты на рис.~4 получены при условии, что каждый 
сенсор передает пакеты на агрегатор со средней скоростью~256~бит/с, 
а~максимальные пропускные способности в~восходящем канале (к~точке 
доступа WiFi) взяты согласно результатам, пред\-став\-лен\-ным на рис.~3,\,\textit{б}. 
В~работе~\cite{12-om} пред\-став\-ле\-на статистика реальной плот\-ности 
размещения сенсоров в~городских условиях, что для типового радиуса 
покрытия точки доступа WiFi (до~300~м) дает около~1000~устройств на 
агрегатор. При этом на рис.~4 видно, что при использовании 
предложенной в~текущем исследовании топологии сети, использующей 
WiFi-агре\-га\-цию, удается достичь расчетного числа поддерживаемых 
сенсоров даже при достаточно большом количестве агрегирующих 
устройств.


   
 \section{Заключение}
     
    В данной работе проведен анализ системы передачи данных от 
сенсорных устройств с~их промежуточной агрегацией и~пересылкой по 
технологии WiFi (IEEE 802.11-2014). Предполагается, что 
соответствующая топология станет типовой для многих  
IoT-при\-ло\-же\-ний с~большим количеством сенсоров, 
и~рассматривается этап состязания между WiFi-аг\-ре\-га\-то\-ра\-ми за выход 
в~беспроводной канал связи. 

Предлагается аналитическая модель, 
построенная на основе теории регенерирующих процессов 
и~учитывающая основные особенности работы протокола доступа к~каналу, 
а~также алгоритма разрешения коллизий. 
    %
    В частности, предполагается, что данные на агрегаторе могут быть 
потеряны, если они превысили определенное количество попыток 
повторной передачи. Значение пропускной способности насыщения, 
полученное в~рамках предложенной модели, сопоставляется с~данными 
имитационного моделирования. Делается вывод об их совпадении 
и~обосновывается максимальное число сенсоров, которые могут быть 
обслужены системой с~рас\-смат\-ри\-ва\-емой топологией. Полученные 
значения существенно превышают ожидаемое число сенсоров 
в~городских условиях даже при достаточно большом количестве 
агрегаторов, что подтверждает целесообразность использования 
технологии WiFi в~беспроводных системах агрегации данных с~большим 
числом сенсоров.



{\small\frenchspacing
 {%\baselineskip=10.8pt
 \addcontentsline{toc}{section}{References}
 \begin{thebibliography}{99}

\bibitem{1-om}
\Au{Ahmadian A., Galinina O.\,S., Gudkova~I.\,A., Andreev~S.\,D., 
Shorgin~S.\,Ya., Samouylov~K.\,E.} On capturing spatial diversity of joint 
M2M/H2H dynamic uplink transmissions in 3GPP LTE cellular system~// 
Next Generation Teletraffic and Wired/Wireless Advanced Networking 
Conference (International) Proceedings.~---  Lecture notes in computer 
science ser.~--- St.\ Petersburg, Russia, 2015. Vol.~9247. P.~407--421.
\bibitem{2-om}
\Au{Кучерявый А.\,Е.} Самоорганизующиеся сети и~новые услуги~// 
Электросвязь, 2009. Вып.~1. С.~19--23.
\bibitem{3-om}
\Au{Восков Л.\,С.} Беспроводные сенсорные сети и~прикладные 
проекты~// Автоматизация и~IT в~энергетике, 2009. №\,2-3. С.~44--49.
\bibitem{4-om}
\Au{Косинов М.\,И., Шорин О.\,А.} Повышение емкости сотовой системы 
связи при использовании зон перекрытия~// Электросвязь, 2003. Вып.~1. 
С.~18--20.
\bibitem{5-om}
\Au{Гайдамака Ю.\,В., Печинкин~А.\,В., Разумчик~Р.\,В., 
Самуйлов~А.\,К., Самуйлов~К.\,Е., Соколов~И.\,А., Сопин~Э.\,С., 
Шоргин~С.\,Я.} Распределение времени выхода из множества состояний 
перегрузки в~системе $M|M|1|\langle L,H \rangle |\langle H,R \rangle$ 
с~гистерезисным управлением нагрузкой~// Информатика и~её 
применения, 2013. Т.~7. Вып.~4. С.~20--33.
\bibitem{6-om}
\Au{Bianchi G.} Performance analysis of the IEEE 802.11 distributed 
coordination function~// IEEE J.~Sel. Area. Comm., 2000. 
Vol.~18. No.\,3. P.~535--547.
\bibitem{7-om}
\Au{Skordoulis D., Ni~Q., Chen~H.\,H., Stephens~A.\,P., Liu~C., 
Jamalipour~A.} IEEE 802.11n MAC frame aggregation mechanisms for  
next-generation high-throughput WLANs~// IEEE Wirel. Commun., 
2008. Vol.~15. No.\,1. P.~40--47.

\bibitem{9-om} %8
\Au{Sharma G., Ganesh A., Key~P., Needham~R.} Performance analysis of 
contention based medium access control protocols~// IEEE Trans. 
Inform. Theory, 2009. Vol.~55. No.\,4. P.~1665--1682.
\bibitem{10-om} %9
\Au{Malone D., Duffy~K., Leith~D.} Modeling the 802.11 distributed 
coordenation function in non-saturated heterogeneous conditions~// 
IEEE/ACM Trans. Networks, 2007. Vol.~15. No.\,1. P.~159--172.
\bibitem{8-om} %10
\Au{Bordenave C., McDonald D., Proutire~A.} Random multi-access 
algorithms~--- a mean field analysis~// Rapport de Recherche, 2005. Vol.~5632. 
P.~1--12.
\bibitem{11-om}
\Au{Andreev S., Koucheryavy~Y., Sousa~L.} Calculation of transmission 
probability in heterogeneous ad hoc networks~// IEEE Baltic Congress on 
Future Internet and Communications (BCFIC) Proceedings, 2011. P.~75--82.
\bibitem{12-om}
\Au{Ortiz S.} IEEE 802.11n: The road ahead~// IEEE Computer, 2009. 
Vol.~42. No.\,7. P.~13--15.
\end{thebibliography}

 }
 }

\end{multicols}

\vspace*{-6pt}

\hfill{\small\textit{Поступила в~редакцию 06.04.16}}

%\vspace*{8pt}

\newpage

\vspace*{-30pt}

%\hrule

%\vspace*{2pt}

%\hrule

%\vspace*{8pt}



\def\tit{PERFORMANCE ANALYSIS OF~A~WIRELESS DATA 
AGGREGATION SYSTEM WITH~CONTENTION 
FOR~CONTEMPORARY SENSOR NETWORKS}

\def\titkol{Performance analysis of~a~wireless data 
aggregation system with~contention 
for~contemporary sensor networks}

\def\aut{A.\,Ya.~Ometov$^1$, S.\,D.~Andreev$^2$, 
A.\,M.~Turlikov$^3$, and~E.\,A.~Koucheryavy$^4$}

\def\autkol{A.\,Ya.~Ometov, S.\,D.~Andreev, 
A.\,M.~Turlikov, and~E.\,A.~Koucheryavy}

\titel{\tit}{\aut}{\autkol}{\titkol}

\vspace*{-9pt}

\noindent
        $^1$Saint-Petersburg State University of Telecommunications, 
22B~Pr.~Bolshevikov,  St.\ Petersburg 193232, Russian\linebreak
$\hphantom{^1}$Federation 
        
        \noindent
        $^2$Peoples' Friendship University of Russia, 3~Ordzhonikidze Str., 
Moscow 115419, Russian Federation
     
     \noindent
        $^3$State University of Aerospace Instrumentation, 67~Bolshaya 
Morskaya Str., St.\ Petersburg 190000, Russian\linebreak
$\hphantom{^1}$Federation
     
     \noindent
        $^4$National Research University Higher School of Economics, 
30~Myasnitskaya Str.,  Moscow 101000, Russian\linebreak
$\hphantom{^1}$Federation


\def\leftfootline{\small{\textbf{\thepage}
\hfill INFORMATIKA I EE PRIMENENIYA~--- INFORMATICS AND
APPLICATIONS\ \ \ 2016\ \ \ volume~10\ \ \ issue\ 3}
}%
 \def\rightfootline{\small{INFORMATIKA I EE PRIMENENIYA~---
INFORMATICS AND APPLICATIONS\ \ \ 2016\ \ \ volume~10\ \ \ issue\ 3
\hfill \textbf{\thepage}}}

\vspace*{3pt}
       
    
        
     \Abste{The paper considers a wireless communication 
system with a number of sensing devices that transmit their data to 
multiple aggregating nodes connected to Internet via IEEE  
802.11-2012 (WiFi) technology. It is assumed that an aggregator 
retransmits data from many sensors by competing with other 
aggregators for the shared channel. The paper proposes an 
analytical model taking into account the features of the collision 
resolution algorithm, the properties of the channel access protocol, 
as well as the possibility to discard data at the aggregator. The 
obtained analytical results are compared with the simulation data, 
and the maximum number of supported sensors in the 
communication system is estimated.}
     
     \KWE{Internet of Things; wireless sensor networks; 
saturated system; regenerative analysis; WLAN; IEEE 802.11-2014 
standard}


\DOI{10.14357/19922264160304}

\vspace*{-9pt}

\Ack
\noindent
This work is supported by the Russian Foundation for 
Basic Research (project No.\,15-07-03051) and the Foundation for Assistance to 
Small Innovative Enterprises (FASIE) within the program ``UMNIK'' 
under grant 8268GU2015 (02.12.2015).


\vspace*{9pt}

  \begin{multicols}{2}

\renewcommand{\bibname}{\protect\rmfamily References}
%\renewcommand{\bibname}{\large\protect\rm References}

{\small\frenchspacing
 {%\baselineskip=10.8pt
 \addcontentsline{toc}{section}{References}
 \begin{thebibliography}{99}
\bibitem{1-om-1}
\Aue{Ahmadian, A.\,M., O.\,S.~Galinina, I.\,A.~Gudkova, S.\,D.~Andreev, 
S.\,Ya.~Shorgin, and K.\,E.~Samouylov}. 2015. On capturing spatial diversity 
of joint M2M/H2H dynamic uplink transmissions in 3GPP LTE cellular 
system. \textit{International Next Generation Teletraffic and Wired/Wireless 
Advanced Networking Conference Proceedings}. 
Lecture notes in computer science ser. St.\ Petersburg, Russia.  9247:407--421.
\bibitem{2-om-1}
\Aue{Koucheryavy, A.\,E.} 2009. Samoorganizuyushchiesya seti i~novye 
uslugi [Ad hoc networks and new services]. \textit{Elektrosvyaz'} 
[Telecommunications] 1:19--23.
\bibitem{3-om-1}
\Aue{Voskov, L.\,S.} 2009. Besprovodnye sensornye seti i~prikladnye proekty 
[Wireless networks and application projects]. \textit{Avtomatizatsiya i~IT 
v~energetike} [IT automatization in Enegetics] 2-3:44--49.
\bibitem{4-om-1}
\Aue{Kosinov, M.\,I., and O.\,A.~Shorin}. 2003. Povyshenie emkosti sotovoy 
sistemy svyazi pri ispol'zovanii zon perekrytiya [Increasing cellular network 
capacity utilizing the overlapping zones]. \textit{Elektrosvyaz'} 
[Telecommunications] 1:18--20.
\bibitem{5-om-1}
\Aue{Gaydamaka, Yu.\,V., A.\,V.~Pechinkin, R.\,V.~Razumchik, 
A.\,K.~Samuylov, K.\,E.~Samuylov, I.\,A.~Sokolov, E.\,S.~Sopin, and 
S.\,Ya.~Shorgin}. 2013. Raspredelenie vremeni vykhoda iz mnozhestva 
sostoyaniy peregruzki v~sisteme $M|M|1|\langle L,H \rangle |\langle H,R 
\rangle$ s~gisterezisnym upravleniem nagruzkoy [The distribution of the return 
time from the set of overload states to the set of normal load states in a~system 
$M|M|1|\langle L,H \rangle |\langle H,R \rangle$ with hysteretic load control]. 
\textit{Informatika i~ee Primeneniya~--- Inform. Appl.} 7(4):\linebreak 20--33.
\bibitem{6-om-1}
\Aue{Bianchi, G.} 2000. Performance analysis of the IEEE 802.11 distributed 
coordination function. \textit{IEEE J.~Sel. Area. Comm}. 
18(3):535--547.
\bibitem{7-om-1}
\Aue{Skordoulis, D., Q.~Ni, H.\,H.~Chen, A.\,P.~Stephens, C.~Liu, and 
A.~Jamalipour}. 2008. IEEE 802.11~n~MAC frame aggregation mechanisms 
for next-generation high-throughput WLANs. \textit{IEEE Wirel. 
Commun.} 15(1):40--47.


\bibitem{9-om-1} %8
\Aue{Sharma, G., A.~Ganesh, P.~Key, and R.~Needham}. 2009. Performance 
analysis of contention based medium access control protocols. \textit{IEEE 
Trans. Inform. Theory} 55(4):1665--1682.

\pagebreak

\bibitem{10-om-1} %9
\Au{Malone, D., K.~Duffy, and D.~Leith}. 2007. Modeling the 802.11 
distributed coordenation function in non-saturated heterogeneous conditions.  
\textit{IEEE/ACM Trans. Networks} 15(1):159--172.
\bibitem{8-om-1} %10
\Aue{Bordenave, C., D.~McDonald, and A.~Proutire}. 2005. Random  
multi-access algorithms~--- a~mean field analysis. \textit{Rapport de 
Recherche}  5632:1--12.

%\columnbreak

\bibitem{11-om-1}
\Aue{Andreev, S., Y.~Koucheryavy, and L.~Sousa}. 2011. Calculation of 
transmission probability in heterogeneous ad hoc networks. \textit{IEEE Baltic 
Congress on Future Internet and Communications (BCFIC) Proceedings}.  
75--82.
\bibitem{12-om-1}
\Aue{Ortiz, S.} 2009. IEEE 802.11n: The road ahead. \textit{IEEE 
Computer} 42(7):13--15.
\end{thebibliography}

 }
 }

\end{multicols}

\vspace*{-3pt}

\hfill{\small\textit{Received April 06, 2016}}

\Contr

\noindent
\textbf{Ometov Aleksandr Ya.} (b.\ 1991)~--- PhD student, St.\ Petersburg State University 
of Telecommunications, 22B~Bolshevikov Pr., St.\ Petersburg 193232, Russian Federation; 
\mbox{alеxаnder.omеtov@gmаil.com}

\vspace*{4pt}

\noindent
\textbf{Andreev Sergey D.} (b.\ 1984)~--- Candidate of Sciences, PhD; associate professor, 
Peoples' Friendship University of Russia, 3~Ordzhonikidze Str., Moscow 115419, Russian 
Federation; \mbox{serge.аndeev@gmаil.com}

\vspace*{4pt}

\noindent
\textbf{Turlikov Andrey M.} (b.\ 1957)~--- Doctor of Sciences, professor; Head of 
Department, St.\ Petersburg State University of Aerospace Instrumentation, 67~Bolshaya 
Morskaya Str., St.\ Petersburg 190000, Russian Federation; \mbox{turlikоv@vu.spb.ru}

\vspace*{4pt}

\noindent
\textbf{Koucheryavy Evgeni A.} (b.\ 1974)~--- Candidate of Sciences, PhD; professor, 
National Research University Higher School of Economics, 20 Myasnitskaya Str., Moscow 
101000, Russian Federation; \mbox{ykоucheryаvy@hsе.ru}

\label{end\stat}


\renewcommand{\bibname}{\protect\rm Литература}
     