\def\stat{zah+shestakov}

\def\tit{АНАЛИЗ ТОЧНОСТИ ВЕЙВЛЕТ-ОБРАБОТКИ АЭРОДИНАМИЧЕСКИХ КАРТИН ОБТЕКАНИЯ}

\def\titkol{Анализ точности вейвлет-обработки аэродинамических картин обтекания}

\def\aut{Т.\,В.~Захарова$^1$, О.\,В.~Шестаков$^2$}

\def\autkol{Т.\,В.~Захарова, О.\,В.~Шестаков}

\titel{\tit}{\aut}{\autkol}{\titkol}

\index{Шестаков О.\,В.}
\index{Shestakov O.\,V.}
\index{Захарова Т.\,В.}
\index{Zakharova T.\,V.}


%{\renewcommand{\thefootnote}{\fnsymbol{footnote}} \footnotetext[1]
%{Работа выполнена при частичной финансовой поддержке РФФИ (проект 15-07-02652).}}


\renewcommand{\thefootnote}{\arabic{footnote}}
\footnotetext[1]{Московский государственный университет им.\ М.\,В.~Ломоносова,
факультет вычислительной математики и~кибернетики; Институт проб\-лем
информатики Федерального исследовательского центра <<Информатика и~управление>> 
Российской академии наук, \mbox{lsa@cs.msu.ru}} 
\footnotetext[2]{Московский государственный университет им.\ М.\,В.~Ломоносова, 
кафедра математической статистики факультета вычислительной математики и~кибернетики; 
Институт проблем информатики Федерального исследовательского центра 
<<Информатика и~управление>> Российской академии наук, \mbox{oshestakov@cs.msu.su}}



\Abst{Предлагается новый метод обработки зашумленных аэродинамических картин 
обтекания, основанный на технике вейвлет-ана\-ли\-за. Вейв\-лет-ме\-то\-ды 
подавления шума, основанные на процедуре пороговой обработки, широко используются 
при анализе сигналов и~изображений. Их привлекательность заключается, во-пер\-вых, 
в~быстроте алгоритмов построения оценок, а~во-вто\-рых, в~возможности лучшей, чем 
линейные методы, адаптации к~функциям, имеющим на разных участках различную степень 
регулярности. Анализ погрешностей этих методов представляет собой важную практическую 
задачу, поскольку он позволяет оценить качество как самих методов, так и~используемого 
оборудования.
Проведена верификация метода на основе сравнительного анализа с~результатами 
обработки ранее разработанным дискриминантным методом. Рассчитанная оценка 
погрешности обработки при этом согласуется с~теоретическими результатами для 
этой оценки.}

    \KW{вейвлет-анализ; пороговая обработка; несмещенная оценка риска; 
    аэродинамический поток}


\DOI{10.14357/19922264160307} 
  
  %\vspace*{6pt}


\vskip 12pt plus 9pt minus 6pt

\thispagestyle{headings}

\begin{multicols}{2}

\label{st\stat}


\section{Введение}

Обработка результатов экспериментов в~аэродинамической трубе требует
исключительной точ\-ности. Результаты таких испытаний влияют на
кинематические возможности проектируемой техники, экономичность 
и~безопасность ее использования. Повысить точность можно путем
услож\-не\-ния экспериментальной установки либо\linebreak использованием
специализированной обработки результатов эксперимента. Первый путь,
как правило, весьма дорог и~сложен: для аэродинамических
экспериментов это означает улучшение системы фильтрации воздуха 
и~поддержания постоянства его температуры, влажности и~т.\,д.,
усложнение оптических систем для снятия характеристик обтекания.
Второй требует разработки и~верификации методики оценки истинных
значений по имеющимся данным~--- зачастую это сделать проще 
и~дешевле; таким образом, разработка специальных алгоритмов обработки
данных весьма актуальна в~данной области.

В рамках данной работы непосредственные физические характеристики
газовой струи заменены условным эквивалентом --- цветом на теневой
картине обтекания. Задача состоит в~очистке данного сигнала от шума,
возникающего вследствие наличия в~газовой струе и~на оптике
аппаратуры фото/видеосъемки пыли и~т.\,п.~\cite{holder, krasnov}. Для
решения данной задачи в~работе используются методы вейв\-лет-анализа,
которые прекрасно зарекомендовали себя при анализе и~обработке
нестационарных сигналов и~изображений~\cite{smolentsev, posobie}.


\section{Вейвлет-методы обработки изображений}

\subsection{Пороговая обработка вейвлет-коэффициентов изображения}

Большинство алгоритмов подавления шума, использующие методы вейв\-лет-ана\-ли\-за, 
действуют по принципу 
<<вейв\-лет-пре\-об\-ра\-зо\-ва\-ние\,--\,обработка вейв\-лет-ко\-эф\-фи\-ци\-ен\-тов\,--\,об\-рат\-ное 
вейв\-лет-пре\-об\-ра\-зо\-ва\-ние>>. В~данной работе рассматривается один из 
способов обработки вейв\-лет-ко\-эф\-фи\-ци\-ен\-тов~--- 
так называемая мягкая пороговая обработка.

Пусть $N=2^J$ для некоторого $J\hm>0$; $f(i,j)$, $0\hm\leqslant i,j \hm< N,$~--- 
матрица изображения; $\{W(i,j)\}_{0\leqslant i,j < N}$~--- 
помехи (шум), являющиеся реализацией некоторого
случайного процесса. В~данной работе предполагается, 
что $\{W(i,j)\}_{0\leqslant i,j < N}$~--- независимые
одинаково распределенные нормальные случайные величины с~нулевым 
средним и~дисперсией~$\sigma^2$. Искаженный
сигнал определяется формулой:
\begin{equation*}
    X(i,j)=f(i,j)+W(i,j)\,, \enskip 0 \leqslant i,j < N\,.
\end{equation*}
После применения дискретного вейв\-лет-пре\-об\-ра\-зо\-ва\-ния 
$Y\hm=W(X)$~\cite{posobie} 
получается следующая модель эмпирических вейв\-лет-ко\-эф\-фи\-ци\-ентов:
\begin{multline*}
    Y^{[\lambda]}(j,k,l)=f^{[\lambda]}_W(j,k,l)+
    Z^{[\lambda]}(j,k,l)\,, \\ 
    \lambda=1,2,3\,,\enskip 0\leqslant j<J\,,\enskip 0\leqslant k,l< 2^j\,,
\end{multline*}
где $f_W(j,k,l)$~--- вейвлет-ко\-эф\-фи\-ци\-ен\-ты <<чис\-то\-го>> изображения, 
а~шумовые коэффициенты $Z^{[\lambda]}(j,k,l)$ в~силу ортогональности 
вейв\-лет-пре\-об\-ра\-зо\-ва\-ния имеют такую же статистическую структуру, 
как $W(i,j)$. Индекс~$j$ называется масштабом и~отвечает за <<размер>> 
вейв\-лет-ко\-эф\-фи\-ци\-ен\-та (размер участка изображения, за который 
отвечает данный коэффициент), индексы $k$ и~$l$~--- сдвиги, отвечающие за 
<<положение>> вейв\-лет-ко\-эф\-фи\-ци\-ен\-та (расположение участка изображения, 
за который отвечает данный коэффициент), а~индекс $\lambda$ описывает тип 
вейв\-лет-ко\-эф\-фи\-ци\-ен\-та: при $\lambda\hm=1$ 
вейв\-лет-ко\-эф\-фи\-ци\-ен\-ты называются вертикальными, при $\lambda\hm=2$~--- 
горизонтальными, а~при $\lambda\hm=3$~--- диагональными~\cite{smolentsev, bogges, malla}.


Оценки вейвлет-коэффициентов изображения $\hat{f}_W(j,k,l)$
вычисляются покомпонентно, т.\,е.
$$
\hat{f}^{[\lambda]}_W(j,k,l)=\rho_T(Y^{[\lambda]}(j,k,l))\,,
$$
где $\rho_T(x)$~--- функция мягкой пороговой обработки с~порогом $T\hm>0$, 
задаваемая формулой:
\begin{equation*}
    \label{soft_function}
        \rho_T(x)=
            \begin{cases}
                x-T, &\mbox{ если }x > T\,;\\
                x+T, &\mbox{ если }x < -T\,;\\
                0, &\mbox{ если } \left|x\right|  \leqslant T\,.
            \end{cases}
\end{equation*}
Таким образом, в~результате пороговой обработки обнуляются те 
вейв\-лет-ко\-эф\-фи\-ци\-ен\-ты, абсолютная величина которых не превосходит порога, 
а~абсолютная величина остальных вейв\-лет-ко\-эф\-фи\-ци\-ен\-тов 
уменьшается на величину порога.

Риск (среднеквадратичная погрешность) мягкой пороговой обработки 
определяется формулой:
\begin{equation}
    \label{risk_def}
        r_f= \sum\limits_{\lambda=1}^{3}\sum\limits_{j=0}^{J-1}
        \sum\limits_{i,j=0}^{2^j-1}{\mathbb{E}\left|\hat{f}^{[\lambda]}_W(j,k,l)
        -f^{[\lambda]}_W(j,k,l)\right|^2}\,.
\end{equation}
Выражение (\ref{risk_def}) содержит неизвестные величины 
<<чистых>> вейв\-лет-ко\-эф\-фи\-ци\-ен\-тов $f^{[\lambda]}_W(j,k,l)$, 
поэтому на практике вычислить его нельзя. Однако его можно оценить, используя 
только известные эмпирические вейв\-лет-ко\-эф\-фи\-ци\-ен\-ты~$Y^{[\lambda]}(j,k,l)$. 
В~каждом слагаемом если $\left|Y^{[\lambda]}(j,k,l)\right|\hm>T$, 
то вклад этого слагаемого в~риск со\-став\-ля\-ет $\sigma^2\hm+T^2$, 
а~если $\left|Y^{[\lambda]}(j,k,l)\right|\hm\leqslant T$, то вклад 
со\-став\-ля\-ет $[f^{[\lambda]}_W(j,k,l)]^2$. Поскольку $\mathbb{E}[Y^{[\lambda]}(j,k,l)]^2
\hm=\sigma^2+[f^{[\lambda]}_W(j,k,l)]^2$, величину $[f^{[\lambda]}_W(j,k,l)]^2$ 
можно оценить разностью $[Y^{[\lambda]}(j,k,l)]^2\hm-\sigma^2$.

Таким образом, в~качестве оценки риска можно использовать следующую величину:
\begin{equation}
\label{RiskEstimateDefinition}
\hat{r}_f=\sum\limits_{\lambda=1}^{3}\sum\limits_{j=0}^{J-1}
\sum\limits_{i,j=0}^{2^j-1}F[[Y^{[\lambda]}(j,k,l)]^2]\,,
\end{equation}
где $F[x]=(x-\sigma^2)\Ik(|x|\hm\leqslant
T^2)\hm+(\sigma^2\hm+T^2)\Ik(|x|\hm>T^2).$ Донохо и~Джонстон
показали~\cite{DonJ2, DonJ}, что при мягкой пороговой обработке для
оценки риска~(\ref{RiskEstimateDefinition}) справедливо следующее
утверждение. 

\smallskip

\noindent
\textbf{Лемма~1}. \textit{ %{\ulemma{\label{Unbiased_Estimte}
$\mathbb{E}\hat{r}_f\hm=r_f$, т.\,е. $\hat{r}_f$ является несмещенной
оценкой для~$r_f$.}

\smallskip


Исследуем теперь вопрос выбора порога.

\subsection{Предпосылки выбора порога вейвлет-обработки}

\noindent
\textbf{Лемма~2.}
\textit{Пусть $z_1, z_2, \ldots, z_n$~--- независимые случайные величины, 
имеющие нормальное распределение с~нулевым средними и~дисперсией~$\sigma^2$;
\begin{align*}
    A_n &= \left\{\max\limits_{1 \leqslant i \leqslant n}{\left|z_i\right|}>
    \sigma\sqrt{2\ln{n}}\right\}\,;\\
    B_n(t)&=\left\{\max\limits_{1 \leqslant i \leqslant n}{\left| z_i \right|} 
    > \sigma t+\sigma\sqrt{2\ln{n}}\right\}.
\end{align*}
Тогда}
\begin{equation*}
    \label{pi_n}
        P\left(A_n\right)\to 0 \mbox{ при }n \to \infty\,;\enskip 
P\left(B_n(t)\right) < e^{-{t^2}/{2}}\,.
\end{equation*}


Порог $T=\sigma\sqrt{2\ln{n}}$ согласно лемме~2 с~большой вероятностью 
удаляет основной шум,
а~оставшийся шум будет незначительным, так как вероятность~$B_n$ экспоненциально 
убывает.

На практике часто приходится оценивать значение~$\sigma$. Обычно в~качестве 
такой оценки используется величина
\begin{equation}
    \label{sigma_hat}
        \hat{\sigma}=\fr{1}{C_{3/4}}\,\mathrm{MAD}\,,
\end{equation}
где MAD (Median Absolute Deviation)~--- выборочное медианное абсолютное 
отклонение, построенное по вейв\-лет-ко\-эф\-фи\-ци\-ен\-там при $j\hm=J\hm-1$ 
(считается, что на этом масштабе коэффициенты содержат только шум~\cite{malla}), 
а~$C_{3/4}$~--- $3/4$-кван\-тиль стандартного нормального распределения.

В силу приведенных рассуждений в~данной работе используется порог

\begin{equation*}
    T_U=\hat{\sigma}\sqrt{2 \ln{n}}, \quad n=N^2,
\end{equation*}
который называется <<универсальным>>.


В работах \cite{MA09, MSH10} доказаны следующие утверждения, 
позволяющие статистически оценить погрешность метода пороговой обработки, 
используя только известные эмпирические вейв\-лет-ко\-эф\-фи\-ци\-енты.

\smallskip

\noindent
\textbf{Теорема~1.}
\textit{Пусть функция, описывающая изображение, является ку\-соч\-но-ре\-гу\-ляр\-ной по 
Липшицу с~показателем $\gamma\hm>0$. Тогда при использовании мягкой пороговой 
обработки с~порогом $T_U$}
\begin{equation}\label{Universal_Consistency}
\fr{\hat{r}_f-r_f}{N^2}\rightarrow 0\enskip \mbox{при}\enskip N\rightarrow\infty\,.
\end{equation}


\noindent
\textbf{Теорема~2.}
\textit{Пусть функция, описывающая изображение, является ку\-соч\-но-ре\-гу\-ляр\-ной 
по Липшицу с~показателем $\gamma\hm>1/2$. Тогда при использовании мягкой пороговой 
обработки с~порогом~$T_U$}
\begin{equation}
\label{Universal_Normality_Sigma_Robust}
\mathsf{P}\left(\fr{\hat{r}_f-r_f}{\sqrt{2}\hat{\sigma}^2N}<x\right)\Rightarrow
\Phi_{\Upsilon}(x)\ \mbox{ при }\ N\rightarrow\infty\,.
\end{equation}
\textit{Здесь $\Phi_{\Upsilon}(x)$~--- 
функция распределения нормального закона с~нулевым средним и~дисперсией 
$\Upsilon^2\hm=[2C_{3/4}\phi(C_{3/4})]^{-2}\hm-1$, 
а~$\phi(x)$~--- плот\-ность стандартного нормального распределения.}

\smallskip


Теорема~1 говорит о том, что $\hat{r}_f$ является состоятельной оценкой, 
а~теорема~2 дает возможность строить асимптотические доверительные интервалы 
для величины~$r_f$ и~оценивать отклонение~$\hat{r}_f$ от~$r_f$.


Рассмотрим применение этих результатов на примере обработки аэродинамических 
изображений.

\section{Обработка цветных теневых картин аэродинамического эксперимента}

\subsection{Постановка задачи}

В аэродинамической трубе установлено тело (использовались данные,
полученные с~цилиндра, ось которого была сначала установлена
перпендикулярно набегающему потоку и~параллельно оптической оси
регистрирующей аппаратуры, а~затем~--- перпендикулярно набегающему
потоку и~оптической оси регистрирующей аппаратуры). 

Исследуется
структура обтекающей его газовой струи. При помощи теневого прибора
производится снятие характеристик обтекания, при этом на выходе
получается так называемая цветная теневая картина обтекания, т.\,е.\
изображение, на котором соответствующими цветами фиксируется
плотность потока~\cite{holder,krasnov,Eng_dis_m, Rus_dis_m}.

Дешифровка цветной теневой картины в~значительной мере осложняется
влиянием зашумления. Оно обусловливается многими факторами, например
наличием частиц пыли в~набегающем потоке и~на оптических элементах
фотоаппаратуры, при помощи которой ведется съемка теневой картины,
т.\,е.\ как собственно возникающими в~исследуемой газовой струе
(внутренними), так и~вносимыми измерительной установкой (внешними).
Естественным в~этом случае можно считать предположение о~том, что
их влияние на результат измерения описывается совокупностью
независимых одинаково распределенных  нормальных случайных величин.
Вообще говоря, следует заметить, что цветорегистрирующая аппаратура
может зашумлять цвет из разных частей спектра по-разному, в~этом
случае нужно обладать соответствующими данными о зависимости
точности работы установки от входного сигнала. При отсутствии
указанных сведений будем считать, что систематической ошибки
измерительная аппаратура не создает, а~шум, вносимый ею в~данные, от
самого сигнала не зависит.

В изначальной постановке необходимо создать методику оценки
плотности по цветной теневой картине потока в~аэродинамической
трубе. Алгоритм должен быть устойчив как к~внутренним, так 
и~к~внешним шумам и~с достаточной точностью оценивать численные
характеристики обтекания. 
{\looseness=1

}

В~такой постановке задача несколько лет
назад была успешно решена при помощи так называемого дискриминантного метода~[11--14], 
он будет кратко описан ниже.

В данной работе исследуется вопрос о~возможности реализовать
процедуру обработки цветной теневой картины методами
вейв\-лет-ана\-ли\-за. При этом, ввиду отсутствия прямого (без
использования результатов работы дискриминантного метода) алгоритма
дешифровки цветной теневой картины, исследуется вопрос
восстановления цветности изоб\-ра\-же\-ния, а~не собственно плотности
потока. 

Особое внимание уделено точности обработки об\-ласти перед
телом, установленным в~аэродинамической трубе, так как дискриминантный
метод оптимизирован для анализа этой об\-ласти.



\subsection{Решение задачи при~помощи дискриминантного метода}

Входная цветная теневая картина представляет собой графический файл
в формате bmp. Размер изображения $512 \times 512$ пикселей. Каждый
пиксель изображения~--- элемент цветового пространства $(R, G, B)$.
Отобразим его на единичную сферу, т.\,е.\ осуществим следующее
преобразование:
{ %\small
\begin{multline}
(r,g,b)\to{}\\
\!\to\!\left(\!\fr{r}{\sqrt{r^2+g^2+b^2}},\!\fr{g}{\sqrt{r^2+g^2+b^2}},\!
\fr{b}{\sqrt{r^2+g^2+b^2}}\!\right)\!\!=\\
\label{rgbtocos}
=\left(\cos{\alpha},\cos{\beta},\cos{\gamma}\right)\,.
\end{multline}
}

Всюду далее будем называть цветовым пространством единичную сферу, 
на которой расположены векторы
$\left(\cos{\alpha},\cos{\beta},\cos{\gamma}\right)$~--- элементы 
пространства $(R, G, B)$ после преобразования~(\ref{rgbtocos}).

Все цветовое пространство разобъем на так называемые эталонные классы:
произведем разби\-ение полосы спектра, зафиксированной измерительным
прибором, на прямоугольные сегменты,\linebreak для каждого из них в~терминах
косинусов (из приведенного выше преобразования) вычислим дисперсию
$\{{\sf D}_{i1},{\sf D}_{i2},{\sf D}_{i3}\}$ и~выборочное среднее
$\{z_{i1},z_{i2},z_{i3}\}$ (так называемый центр) класса с~номером~$i$.
Фактически эти характеристики параметризуют зависимость погрешности
определения прибором цвета в~разных частях спектра. Каждый класс
эквивалентности характеризуется также чис\-лом~$X_{i}$~--- индексом
цветности.

Введем дискриминантную функцию
\begin{multline*}
    d_{i}(v)={}\\
    {}=\!\sqrt{a_{i1}\left(v_1-z_{i1}\right)^2+a_{i2}\left(v_2-z_{i2}\right)^2+
    a_{i3}\left(v_3-z_{i3}\right)^2},\hspace*{-7.9pt}
\end{multline*}
где
$v\hm=(v_1,v_2,v_3)$~--- произвольный вектор цветового пространства;
$$
a_{ik}=\fr{\min_{j \neq i}(z_{ik}-z_{jk})}{D_{ik}},\enskip k\hm=1,2,3\,;$$
$D_{ik}$~--- выборочная дисперсия $v_k$ на $i$-м эталонном классе.


При помощи функций $d_i(v)$ по сути производится вычисление расстояний~$\rho$ 
от центров эталонных классов
до точек цветового пространства~$v$. Теоретически точка~$v$ находится 
в~$i$-м эталонном классе, если $d_i(v)\hm=\min_{j}d_j(v)$.

Для каждого пикселя анализируемого изображения вычислим 
два наиболее близких класса (для простоты
проиндексируем их: 1~--- ближайший и~2~--- второй в~порядке удаления) и~на 
основе расстояний~$\rho_1$ и~$\rho_2$
до них и~индексов цветности соответственно~$X_1$ и~$X_2$ будем вычислять 
цветность~$X$ вектора~$v$ по формуле
\begin{equation*}
    X=X_1+\fr{\rho_1}{\rho_1+\rho_2}\left|X_1-X_2\right|\,.
\end{equation*}

Далее производится восстановление плотности  и~цветной теневой картины 
на основе сформированных индексов
цветности, сведений об эталонных классах и~параметров измерительной установки.

Дискриминантный метод показал хорошую устойчивость как к~внешним,
так и~к~внутренним шумам и~позволил с~необходимой точностью
восстановить истинные значения плотности потока и,~соответственно,
цвета на теневой картине~[12--14].

Нужно отметить тот факт, что константы настройки дискриминантного 
метода установлены таким образом, чтобы с~максимальной точностью оценивать 
плотность потока в~отдельном участке спектра, соответствующем области 
набегающего потока прямо перед столкновением его с~телом.



\subsection{Решение задачи средствами вейвлет-анализа}

Как уже отмечалась, общая концепция любого алгоритма обработки изображений, 
основанного на вейв\-лет-пре\-об\-ра\-зо\-ва\-нии, заключается в~сле\-ду\-ющем:
\begin{enumerate}[(1)]
    \item  преобразование (декомпозиция) изображения;
    \item  обработка массива вейв\-лет-ко\-эф\-фи\-ци\-ен\-тов;
    \item  реконструкция изображения.
\end{enumerate}
Декомпозиция и~реконструкция~--- взаимосвязанные процессы, так как
должны проводиться с~использованием наперед заданного вида вейвлета
и~глубины разложения. Таким образом, сразу появляются два параметра
алгоритма: используемый вейвлет и~глубина декомпозиции. По
результатам тестов с~использованием различных типов вейвлетов
наилучший результат был получен для так называемого обратного биортогонального
семейства. Глубина разложения сказывается на размере деталей,
которые все еще можно считать шумовыми, этот параметр подбирался
исключительно экспериментально: заранее можно было лишь сказать,
что, исходя из информационного смысла вейв\-лет-ко\-эф\-фи\-ци\-ен\-тов,
задействовать слишком большую глубину разложения не имело смысла.

Отметим тот факт, что отсутствуют ка\-кие-ли\-бо априорные сведения о степени 
зашумления изображения. Поскольку в~качестве способа обработки массива 
вейв\-лет-ко\-эф\-фи\-ци\-ен\-тов заявлена пороговая обработка, то в~силу 
леммы~2 получаем, что достаточно оценить дисперсию шума (порог выбран универсальным).

Цветная теневая картина~--- изображение, где каждый пиксель кодируется 
триплетом $(R,G,B)$. Предлагаемый алгоритм вейв\-лет-об\-ра\-бот\-ки 
оценивает пороги для каждой цветовой компоненты в~отдельности, тем 
самым отчасти учитывается тот факт, что зашумление может по-раз\-но\-му 
проявляться в~разных частях спектра. Эти принципы вполне соответствуют специфике 
обрабатываемого сигнала: в~ходе специальной проверки были зафиксированы отличия 
в~зашумлении для разных цветовых компонент.

Одной из главных характеристик сигнала является его энергия. 
В~терминах изображения она вычисляется как
\begin{equation}
    \left\|C_0\right\|=\sum\limits_{i,j=1}^{N}
    \left({r_{i,j}^2+g_{i,j}^2+b_{i,j}^2}\right)\,,
\label{image_energy}
\end{equation}
где $(r_{i,j},g_{i,j},b_{i,j})$~--- пиксель изображения.

Результат сравнения энергий, заключающихся в~аппроксимации и~детализации 
изображения при глубине разложения не более~4, показал, что значительная доля 
сигнала заключена в~аппроксимирующих коэффициентах, поэтому малые изменения, 
вносимые в~них, могут серьезно повлиять на результат обработки. 
Вследствие этого было принято решение о невнесении ка\-ких-ли\-бо изменений 
в~коэффициенты аппроксимации. Этот вывод вполне согласуется с~физическим смыслом: 
низкочастотный вейв\-лет-фильтр выделяет главные особенности сигнала, по сути, 
создавая сглаженный и~уменьшенный его вариант, изменять его~--- 
подвергаться значительному риску потери информации о сигнале. Коэффициенты 
детализации же как раз характеризуют отличия аппроксимации от оригинала, 
большие по величине вряд ли соответствуют шуму (в~терминах постановки задачи), 
в~то время как малые (они и~отвечают зашумлению) можно приравнять к~0, 
что как раз соответствует пороговой обработке сигнала.

Дополнительное исследование вейв\-лет-ко\-эф\-фи\-ци\-ен\-тов показало, 
что в~качестве оценки~$\sigma$ можно взять величину~$\hat{\sigma}$, определяемую 
формулой~(\ref{sigma_hat}) для каждого набора коэффициентов детализации.


Описанный алгоритм вейв\-лет-об\-ра\-бот\-ки показал высокую точность
восстановления цвет\-ности теневых аэродинамических картин, что
подтверждается проведенным сравнительным анализом с~теневыми
картинами, обработанными дискриминантным методом~[11--14], для
верификации которого использовалась система оптических \mbox{клиньев}.


\subsection{Примеры обработки изображений с~использованием дискриминантного 
и~вейвлет-методов}

В данном разделе в~качестве примеров обработки изображений приводятся результаты 
обработки\linebreak картин обтекания цилиндра. Так как дискриминантный метод оптимизирован 
(т.\,е.\ демонст\-ри\-рует наивысшую точность результатов) для обра\-ботки  зоны 
цветового спектра соответствующей\linebreak плот\-ности потока перед исследуемым телом, 
то особый акцент будет сделан на сравнении получаемых результатов именно в~этой 
об\-ласти изображения.

\subsubsection{Исходные данные и~сравнение результатов обработки}

Для данной публикации все приводимые ниже рисунки конвертированы в~формат 
<<Градации серого>>. Исходные цифровые цветные изображения в~формате~$\mathrm{RGB}$, 
взятые для анализа и~в~дальнейшем обработанные обоими методами, приведены 
в~приложении~[15].

На рис.~\ref{pic1},\,\textit{а} четко заметно возмущение, возникающее в~поле плот\-ности 
набегающего потока перед телом. Уточнение значений плотности газа в~данной области~--- 
основная цель создания дискриминантного метода. На рис.~\ref{pic1},\,\textit{б} 
данная особенность не просматривается.
\begin{figure*} %fig1
\vspace*{1pt}
 \begin{center}  
\mbox{%
 \epsfxsize=157.726mm
 \epsfbox{zah-1.eps}
 }
\end{center} 
\vspace*{-9pt}
\Caption{Изображения~1~(\textit{а}) и~2~(\textit{б}) до обработки}
\label{pic1}
\end{figure*}


Поскольку особое значение имеет обработка области перед телом, приведем также 
соответствующим образом кадрированное изображение, полученное на основе 
первого (рис.~2).



Введем обозначения, которые будут использоваться далее:
\begin{itemize}
    \item $r$, $g$, $b$~--- значения соответствующих цветовых компонент 
    пикселя из диапазона $0\ldots255$;
    \item пара нижних индексов для цветовых компонент~--- координаты пикселя 
    в~изображении от верхнего левого угла;
    \item изображение имеет размер $N \times N$ пикселей ($N\hm=512$);
    \item верхний индекс~discr свидетельствует о принадлежности значения 
    к~результатам обработки дискриминантным методом, wave~--- 
    вейв\-лет-ал\-го\-рит\-мом;
    \item под элементом изображения понимается некоторая цветовая компонента пикселя.
\end{itemize}
 \noindent
 \begin{center}  %fig2
 \vspace*{1pt}
 \mbox{%
 \epsfxsize=78mm
 \epsfbox{zah-3.eps}
 }



\vspace*{3pt}

\noindent
{{\figurename~2}\ \ \small{Кадрированное изображение~1 до обработки}}

\end{center} 

 \vspace*{9pt}
 
 \addtocounter{figure}{1}



За критерий сравнения работы дискриминантного метода и~вейв\-лет-об\-ра\-бот\-ки 
выберем
число элементов изображения, различающихся не более чем на~2~единицы, 
и~обозначим его через~$\delta$. Тогда
\begin{multline*}
\delta=\sum\limits_{i,j=1,\ldots ,N}\left(
\mathbb{I}\left\{\left|r_{i,j}^{\mathrm{discr}}-r_{i,j}^{\mathrm{wave}}\right|
\leqslant2\right\}+{}\right.\\
{}+\mathbb{I}\left\{\left|g_{i,j}^{\mathrm{discr}}-
g_{i,j}^{\mathrm{wave}}\right|\leqslant2\right\}+{}\\
\left.{}+
\mathbb{I}\left\{\left|b_{i,j}^{\mathrm{discr}}-b_{i,j}^{\mathrm{wave}}\right|\leqslant2\right\}\right).
\end{multline*}


\subsubsection{Результаты обработки вейвлет-методом}

Результаты обработки исходных теневых картин 
вейв\-лет-ме\-то\-дом представлены на рис.~3 и~4.
\begin{figure*} %fig3
\vspace*{1pt}
 \begin{center}  
\mbox{%
 \epsfxsize=157.815mm
 \epsfbox{zah-4.eps}
 }
\end{center} 
\vspace*{-9pt}
\Caption{Изображения~1~(\textit{а}) и~2~(\textit{б})
после вейв\-лет-об\-ра\-бот\-ки с~глубиной 
декомпозиции~3 и~мягким порогом}
\label{pic1_w}
\vspace*{12pt}
\end{figure*}

 



Расчет значения критерия~$\delta$ показал, что около~80\%~элементов 
изображений обрабатываются одинаково обоими рассматриваемыми методами.

Значения нормированных оценок средне\-квад\-ра\-тичной погрешности 
для изображения~1 равны~24,7, 30,1 и~41,6 для цветовых компонент~$r$, $g$ и~$b$ 
соответственно. Эти же значения для изображения~2 равны~23,9, 29,7 и~40,9. 
На основе теоремы~2 можно сделать вывод, что значения нормированных оценок 
среднеквадратичной по\-греш\-ности, которые вычисляются с~использованием только 
известных эмпирических вейв\-лет-ко\-эф\-фи\-ци\-ен\-тов, с~вероятностью~0,95 
отличаются от истинных среднеквадратичных погрешностей не более чем на~1,1, 1,34 
и~1,99 для изображения~1 и~не более чем на~1,05, 1,32 и~1,9 для изображения~2.


Таким образом, вейв\-лет-ме\-тод, несмотря на кардинально другой подход 
к~способу оценивания зашумления, дает результаты, схожие с~теми, что получа\-ются 
при обработке изображений дискри\-минантным методом. Наличие параметризации 
вейв\-лет-ал\-го\-рит\-ма позволяет применять его к~изоб\-ра\-же\-ниям с~различной 
степенью и~характером зашумле\-ния. Например, в~случае увеличения разрешения 
входного сигнала необходимо будет изменить глубину разложения, чтобы 
величина деталей, среди которых алгоритм отсеивает\linebreak

\noindent
 \begin{center}  %fig4
 \vspace*{1pt}
 \mbox{%
 \epsfxsize=78mm
 \epsfbox{zah-6.eps}
 }



\end{center} 

\noindent
{{\figurename~4}\ \ \small{Изображение~1 после вейв\-лет-об\-ра\-бот\-ки 
с~глубиной декомпозиции~3 и~мягким порогом и~последующего кадрирования}}



 \vspace*{24pt}
 
 \addtocounter{figure}{1}
 
 \noindent
  шумовые, 
соответствующим образом перемасштабировалась. Выбранный в~качестве 
оптимального для тестовой группы изображений вейвлет из обратного 
биортогонального семейства безусловно не является гарантированно 
лучшим для обработки всех теневых картин, он показывал наилучшие 
результаты в~рамках имевшегося набора изображений. Вмес\-те с~тем 
значения параметров гладкости исходного и~итогового сигнала 
вполне соответствуют ожи\-да\-емым значениям с~точки зрения физического 
смыс\-ла цветной теневой картины и~характеристик тестовых изображений: 
вследствие зашумленности исходный сигнал должен обладать минимальной 
гладкостью, в~то время как генерируемый~--- значительно большей.

\section{Выводы}
В ходе решения данной задачи был разработан алгоритм, 
основанный на средствах, предлагаемых вейв\-лет-ана\-ли\-зом. 
Для верификации результатов было проведено сравнение с~дискриминантным 
методом обработки аэродинамических картин.

В ходе тестирования была показана возможность настройки вейв\-лет-ал\-го\-рит\-ма 
для обработки имевшейся группы цветных теневых картин с~достаточной точностью 
относительно результатов обработки дискриминантным методом.

Был проведен расчет нормированной оценки среднеквадратичной погрешности 
по экспериментальным данным для каждого цветового канала. Полученные 
эмпирические значения погрешности хорошо согласуются с~теоретическими 
значениями, рассчитанными на основе свойств ее предельного распределения.

{\small\frenchspacing
 {%\baselineskip=10.8pt
 \addcontentsline{toc}{section}{References}
 \begin{thebibliography}{99}
    \bibitem{holder} %1
    \Au{Холдер Д., Норт Р.} Теневые методы в~аэродинамике~/
    Пер с англ.~--- М.: Мир, 1966.
    180~с. (\Au{Holder~D.\,W., North~R.\,J.} Schlieren methods.~---
    National Physics Laboratory. Notes on applied science No.\,31.~---
    London: H.M.S.O., 1963. 106~p.)
    \bibitem{krasnov}  %2
    \Au{Краснов Н.\.Ф.} Аэродинамика. Т.~2: Методы аэродинамического расчета.~---
     4-е изд.~--- М.: Высшая школа, 2010. 416~с.
         \bibitem{smolentsev}  %3
    \Au{Смоленцев Н.\,К.} 
    Основы теории вейв\-ле\-тов. Вейвлеты в~Matlab.~--- М.: ДМК Пресс, 2005. 157~с.
    \bibitem{posobie} %4
    \Au{Захарова Т.\,В., Шестаков~О.\,В.} 
    Вей\-в\-лет-ана\-лиз и~его приложения.~--- 2-е изд., перераб. и~доп.~--- 
    М.: ИНФРА-М, 2012. 157~с.
    \bibitem{bogges} %5
    \Au{Bogges A., Narkovich~F.\,A.} 
    A~first course in wavelets with Fourier analysis.~--- Prentice Hall, 2001. 293~p.

    \bibitem{malla} %6
    \Au{Малла С.} 
    Вэйвлеты в~обработке сигналов~/ Пер. с~англ.~--- М.: Мир, 2005. 671~с.
    (\Au{Mallat~S.} A~wavelet tour of signal processing.~--- 2nd ed.~---
    Elsevier, 1999. 661~p.)
    \bibitem{DonJ2} %7
\Au{Donoho D., Johnstone~I.} Ideal spatial adaptation via wavelet shrinkage~// 
Biometrika, 1994. Vol.~81. P.~425--455.
    
\bibitem{DonJ}  %8
\Au{Donoho D., Johnstone~I.} 
Adapting to unknown smoothness via wavelet shrinkage~// J.~Am. Statist. Assoc., 1995. Vol.~90. P.~1200--1224.


\bibitem{MA09} %9
\Au{Маркин А.\,В.} Предельное распределение оценки риска при пороговой обработке 
вей\-в\-лет-ко\-эф\-фи\-ци\-ен\-тов~// Информатика и~её применения, 2009. Т.~3. Вып.~4. С.~57--63.

\bibitem{MSH10} %10
\Au{Маркин А.\,В., Шестаков~О.\,В.} 
О~состоятельности оценки риска при пороговой обработке 
вей\-в\-лет-ко\-эф\-фи\-ци\-ен\-тов~// Вестн. Моск. ун-та. Сер.~15. 
Вычисл. матем. и~киберн., 2010. Вып.~1. C.~26--34.
    \bibitem{Eng_dis_m} %11
    \Au{Zakharova T.\,V., Berezentsev~M.\,V.} 
    About a method of supervision classification in the decision of aerodinamic 
    problems~// 24th Seminar (International)
    on Stability Problems for Stochastic Models Transactions.~--- Riga: TTI, 2004. P.~353--356.
    
    \bibitem{Rus_dis_m} %12 
    \Au{Захарова Т.\,В.} Метод распознавания для восстановления изображений 
    цветных теневых картин~// Обозрение прикладной и~промышленной математики, 2005. 
    Т.~12. Вып.~4. С.~967--968.
    \bibitem{aero} 
    \Au{Захарова Т.\,В.,  Шагиров~Э.\,А.} 
    Определение плот\-ности аэродинамического потока обтекания методом цветовой 
    фильтрации~// Математическое моделирование, 2013. Т.~25. Вып.~12. С.~103--109.
    \bibitem{aero2} 
    \Au{Захарова Т.\,В., Шагиров~Э.\,А.} 
    Оптимизация метода цветовой фильтрации для решения задач аэродинамики~// 
    Обозрение прикладной и~промышленной математики, 2013. Т.~20. Вып.~4. С.~545--548.
    \bibitem{addition} 
    \Au{Захарова Т.\,В., Шестаков~О.\,В.} 
    Приложение \mbox{к~статье} <<Анализ точности вей\-в\-лет-об\-ра\-бот\-ки 
    аэродинамических картин обтекания>>. 
    {\sf http://www.ipiran.ru/ \mbox{publications/pictures.pdf}}.
\end{thebibliography}

 }
 }

\end{multicols}

\vspace*{-6pt}

\hfill{\small\textit{Поступила в~редакцию 07.05.16}}

\vspace*{8pt}

%\newpage

%\vspace*{-24pt}

\hrule

\vspace*{2pt}

\hrule

%\vspace*{8pt}



\def\tit{PRECISION ANALYSIS OF~WAVELET PROCESSING OF~AERODYNAMIC FLOW PATTERNS}

\def\titkol{Precision analysis of wavelet processing of aerodynamic flow patterns}

\def\aut{T.\,V.~Zakharova$^{1,2}$ and O.\,V.~Shestakov$^{1,2}$}

\def\autkol{T.\,V.~Zakharova and O.\,V.~Shestakov}

\titel{\tit}{\aut}{\autkol}{\titkol}

\vspace*{-9pt}

\noindent
$^1$Department of Mathematical Statistics, Faculty of Computational Mathematics 
and Cybernetics,\linebreak
$\hphantom{^1}$M.\,V.~Lomonosov Moscow State University, 1-52~Leninskiye Gory, 
GSP-1, Moscow 119991, Russian\linebreak
$\hphantom{^1}$Federation

\noindent
$^2$Institute of Informatics Problems, Federal Research Center 
``Computer Science and Control''
of the Russian\linebreak
$\hphantom{^1}$Academy of Sciences, 44-2~Vavilov Str., Moscow 119333,  Russian Federation



\def\leftfootline{\small{\textbf{\thepage}
\hfill INFORMATIKA I EE PRIMENENIYA~--- INFORMATICS AND
APPLICATIONS\ \ \ 2016\ \ \ volume~10\ \ \ issue\ 3}
}%
 \def\rightfootline{\small{INFORMATIKA I EE PRIMENENIYA~---
INFORMATICS AND APPLICATIONS\ \ \ 2016\ \ \ volume~10\ \ \ issue\ 3
\hfill \textbf{\thepage}}}

\vspace*{3pt}


\Abste{This paper is devoted to a new method of aerodynamic flow pattern processing
based on the wavelet analysis. Wavelet thresholding techniques are widely used in 
signal and image processing. These methods are easily implemented through fast 
algorithms; so, they are very appealing in practical situations. Besides, they adapt 
to function classes with different amounts of smoothness in different locations more 
effectively than the usual linear methods. Wavelet thresholding risk analysis is 
an important practical task, because it allows determining the quality 
of the techniques themselves and the equipment which is being used. Comparative analysis 
using the discriminant method was carried out to verify the new method. 
The empirical estimated error of processing is consistent with theoretical 
results for this estimate.}

\KWE{wavelet analysis; thresholding; unbiased risk estimate; aerodynamic flow}

\DOI{10.14357/19922264160307} 

%\vspace*{-3pt}

\pagebreak

%\Ack
%\noindent



%\vspace*{3pt}

  \begin{multicols}{2}

\renewcommand{\bibname}{\protect\rmfamily References}
%\renewcommand{\bibname}{\large\protect\rm References}

{\small\frenchspacing
 {%\baselineskip=10.8pt
 \addcontentsline{toc}{section}{References}
 \begin{thebibliography}{99}

\bibitem{1-zah}
\Aue{Holder, D.\,W., and R.\,J.~North.} 1963.
\textit{Schlieren methods}.
    National Physics Laboratory. Notes on applied science No.\,31.
    London: H.M.S.O. 106~p.
\bibitem{2-zah}
\Aue{Krasnov, N.\,F.} 2010. \textit{Aerodinamika. T.~2: Metody aerodinamicheskogo rascheta}
[Aerodynamics. Vol.~2: The aerodynamic calculation methods].  4th ed. 
Moscow: Vysshaya Shkola. 416 p.

\bibitem{4-zah} %3
\Aue{Smolentsev, N.\,K.} 2008. 
\textit{Osnovy teorii veyvletov. Veyvlety v~Matlab} 
[Foundations of the theory of wavelets. Wavelets in Matlab]. Moscow: DMK Press. 448~p.
\bibitem{3-zah} %4
\Aue{Zakharova, T.\,V., and O.\,V.~Shestakov}. 2012.
\textit{Veyvlet-analiz i~ego prilozheniya} [Wavelet analysis and its applications]. 
2nd ed. Moscow: INFRA-M. 157~p.

\bibitem{6-zah} %5
\Aue{Boggess, A., and F.\,A.~Narkovich}. 2001. 
\textit{A~first course in wavelets with Fourier analysis}. Prentice Hall. 293~p.
\bibitem{5-zah} %6
\Aue{Mallat, S.} 1999. 
\textit{Wavelet tour of signal processing}. Elsevier. 661~p.

\bibitem{8-zah} %7
\Aue{Donoho, D., and I.~Johnstone}. 1994.  
Ideal spatial adaptation via wavelet shrinkage. \textit{Biometrika} 81:425--455.

\bibitem{7-zah} %8
\Aue{Donoho, D., and I.~Johnstone}. 1995. 
Adapting to unknown smoothness via wavelet shrinkage. 
\textit{J.~Am. Statist. Assoc.} 90:1200--1224.


\bibitem{10-zah} %9
\Aue{Markin, A.\,V.} 2009. 
Predel'noe raspredelenie otsenki riska pri porogovoy obrabotke veyvlet-koeffitsientov 
[Limit distribution of risk estimate of wavelet coefficient thresholding].
\textit{Informatika i~ee Primeneniya~--- Inform. Appl.} 3(4):57--63.
\bibitem{9-zah} %10
\Aue{Markin, A.\,V., and O.\,V.~Shestakov}. 
2010. O~sostoyatel'nosti otsenki riska pri porogovoy obrabotke 
veyvlet-koefficientov [Consistency of risk estimation with thresholding 
of wavelet coefficients]. \textit{Vestn. Mosk. un-ta. Ser.~15. 
Vychisl. matem. i~kibern.} [Moscow University Computational Mathematics and Cybernetics]
1:26-34.

\bibitem{11-zah}
\Aue{Zakharova, T.\,V., and M.\,V.~Berezentsev}. 
2004. About a~method of supervision classification in the decision of 
aerodinamic problems. \textit{24th Seminar 
(International) on Stability Problems for Stochastic Models Transactions}. Riga: TTI. 353--356.
\bibitem{12-zah}
\Aue{Zakharova, T.\,V.} 2005. 
Metod raspoznavaniya dlya vosstanovleniya izobrazheniy tsvetnykh tenevykh kartin 
[Recognition method for recovery of nonferrous shadow paintings images]. 
\textit{Obozrenie Prikladnoy i~Promyshlennoy Matematiki} 
[Review of Applied and Industrial Mathematics] 12(4):967--968.
\bibitem{13-zah}
\Aue{Zakharova, T.\,V., and E.\,A.~Shagirov}. 2013. 
Opredelenie plotnosti aerodinamicheskogo potoka obtekaniya metodom tsvetovoy 
fil'tratsii [Density determination of aerodynamic flow color flow filtration method]. 
\textit{Matematicheskoe modelirovanie} [Mathematical Modeling] 25(12):103--109.
\bibitem{14-zah}
\Aue{Zakharova, T.\,V., and E.\,A.~Shagirov}.  
2013. Optimizatsiya metoda tsvetovoy fil'tratsii dlya resheniya zadach 
aerodinamiki [Optimization of the color filter method for solving the problems 
of aerodynamics]. \textit{Obozrenie Prikladnoy i~Promyshlennoy Matematiki} 
[Review of Applied and Industrial Mathematics] 20(4):545--548.
\bibitem{15-zah}
\Aue{Zakharova, T.\,V., and O.\,V.~Shestakov.}
Appendix to the paper ``Precision analysis of wavelet processing of aerodynamic 
flow patterns.'' 
Available at: {\sf http://www.ipiran.ru/publications/pictures.pdf}
(accessed August~29, 2016).
\end{thebibliography}

 }
 }

\end{multicols}

\vspace*{-3pt}

\hfill{\small\textit{Received May 07, 2016}}

\Contr

\noindent
\textbf{Zakharova Tatiana V.} (b.\ 1962)~--- Candidate of Science (PhD) in 
physics and mathematics; senior lecturer, Department of Mathematical Statistics, 
Faculty of Computational Mathematics and Cybernetics, M.\,V.~Lomonosov Moscow 
State University, 1-52~Leninskiye Gory, GSP-1, Moscow 119991, Russian Federation; 
senior scientist, Institute of Informatics Problems, Federal Research Center 
``Computer Science and Control'' of the Russian Academy of Sciences, 
44-2~Vavilov Str., Moscow 119333, Russian Federation; \mbox{lsa@cs.msu.ru}

\vspace*{3pt}

\noindent
\textbf{Shestakov Oleg V.} (b.\ 1976)~--- 
Doctor of Science in physics and mathematics, associate professor, 
Department of Mathematical Statistics, Faculty of Computational Mathematics 
and Cybernetics, M.\,V.~Lomonosov Moscow State University, 1-52~Leninskiye Gory, 
GSP-1, Moscow 119991, Russian Federation; senior scientist, 
Institute of Informatics Problems, Federal Research Center 
``Computer Science and Control''
of the Russian Academy of Sciences, 44-2~Vavilov Str., Moscow 119333, 
Russian Federation; \mbox{oshestakov@cs.msu.su}
\label{end\stat}


\renewcommand{\bibname}{\protect\rm Литература}