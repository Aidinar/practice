
\def\stat{shestakov}

\def\tit{УСИЛЕННЫЙ ЗАКОН БОЛЬШИХ ЧИСЕЛ ДЛЯ~ОЦЕНКИ РИСКА
В~ЗАДАЧЕ РЕКОНСТРУКЦИИ ТОМОГРАФИЧЕСКИХ ИЗОБРАЖЕНИЙ
ИЗ~ПРОЕКЦИЙ С~КОРРЕЛИРОВАННЫМ ШУМОМ$^*$}

\def\titkol{Усиленный закон больших чисел для~оценки риска
в~задаче реконструкции томографических изображений}
%из~проекций с~коррелированным шумом}

\def\aut{О.\,В.~Шестаков$^1$}

\def\autkol{О.\,В.~Шестаков}

\titel{\tit}{\aut}{\autkol}{\titkol}

\index{Шестаков О.\,В.}
\index{Shestakov O.\,V.}


{\renewcommand{\thefootnote}{\fnsymbol{footnote}} \footnotetext[1]
{Работа выполнена при частичной финансовой поддержке РФФИ (проект 15-07-02652).}}


\renewcommand{\thefootnote}{\arabic{footnote}}
\footnotetext[1]{Московский государственный университет им.\ М.\,В.~Ломоносова, 
кафедра математической статистики факультета вычислительной математики и~кибернетики; 
Институт проблем информатики Федерального исследовательского центра 
<<Информатика и~управление>> Российской академии наук, \mbox{oshestakov@cs.msu.su}}

\vspace*{-12pt}


\Abst{Методы вейвлет-анализа, основанные на процедуре пороговой обработки коэффициентов 
разложения проекций, широко используются при решении задач реконструкции томографических 
изоб\-ра\-же\-ний, возникающих в~медицине, биологии, астрономии и~других областях. 
Их привлекательность заключается, во-пер\-вых, в~быст\-ро\-те алгоритмов, а~во-вто\-рых, 
в~возможности реконструкции локальных участков изображения по неполным проекционным 
данным, что имеет ключевое значение, например, для медицинских приложений, 
где пациента нежелательно подвергать лишней дозе облучения. Анализ погрешностей 
этих методов представляет собой важную практическую задачу, поскольку позволяет 
оценить качество как самих методов, так и~используемого оборудования. В~работе 
рассматривается метод вей\-в\-лет-вей\-глет-раз\-ло\-же\-ния при реконструкции томографических 
изображений в~модели с~коррелированным аддитивным шумом. Доказывается, что для 
несмещенной оценки среднеквадратичного риска при пороговой обработке коэффициентов 
вейв\-лет-вей\-глет-раз\-ло\-жения функции изображения выполняется усиленный закон больших 
чисел, т.\,е.\ эта оценка является сильно состоятельной.}


\KW{вейвлеты; пороговая обработка; оценка среднеквадратичного риска; преобразование Радона}

\DOI{10.14357/19922264160306} 
  
  %\vspace*{6pt}


\vskip 10pt plus 9pt minus 6pt

\thispagestyle{headings}

\begin{multicols}{2}

\label{st\stat}
  

\section{Введение}

Методы реконструкции изображений поглощающих, излучающих и~отражающих объектов и~сред 
получили широкое распространение в~самых разнообразных областях, включая медицину, 
биологию, физику плазмы, газовую динамику, геофизику, астрономию и~радиолокацию. 
В~частности, эти методы революционизировали медицинскую диагностику, поскольку позволили 
получать изоб\-ра\-же\-ния внутренних органов человека без оперативного вмешательства. 
При построении и~анализе статистической модели данных, получаемых в~ходе 
томографического эксперимента, необходимо учитывать погрешности, которые 
возникают в~реальных экспериментах из-за дискретизации исходной информации, 
несовершенства оборудования, характеристик используемого излучения, наличия 
фонового излучения и~внутренних шумов приемника и~т.\,д. В~данной работе рассматривается 
модель стационарного коррелированного шума и~исследуются свойства оценки риска в~методе 
пороговой обработки коэффициентов вей\-в\-лет-вейг\-лет-раз\-ло\-же\-ния функции 
томографического изоб\-ра\-же\-ния. Такая оценка позволяет анализировать качество 
полученного изоб\-ра\-же\-ния на основе только наблюдаемых данных, что очень важно, 
например, в~медицинских приложениях. В~работах~[1, 2] показано, что при 
определенных условиях данная оценка является состоятельной и~асимптотически 
нормальной. В~данной работе доказывается свойство сильной состоятельности.

\section{Преобразование Радона и~его обращения с~помощью метода 
вейвлет-вейглет-разложения}

Преобразование Радона на плоскости определяется как набор интегралов от функции~$f$ 
по всевозможным прямым:
\begin{equation*}
\mathrm{R f}\left(s,\theta\right) = \int\limits_{L_{s,\theta}} f(x,y)\, dl\,,
\end{equation*}
где
\begin{equation*}
L_{s,\theta} = \left\{(x,y)\colon x\cos\theta + y\sin\theta - s = 0 \right\}\,.
\end{equation*}

Задача реконструкции томографического изоб\-ра\-же\-ния состоит в~обращении преобразования~$R$. 
Эту задачу можно решать разными способами. Воспользуемся методом 
вей\-влет-вейг\-лет-раз\-ло\-же\-ния, предложенным в~работе~[3], 
который позволяет учитывать различные локальные особенности изоб\-ра\-жений.

Пусть задана масштабирующая функция $\phi(x)$ и~вей\-в\-лет-функция $\psi(x)$. 
Определим функции
\begin{equation}
\left.
\begin{array}{rl}
\psi_{j,k_1,k_2}^{[1]} (x,y)& = 2^j \phi\left(2^j x - k_1\right) \psi\left(2^j y - k_2\right)\,;\\[6pt]
\psi_{j,k_1,k_2}^{[2]} (x,y) &= 2^j \psi\left(2^j x - k_1\right) \phi\left(2^j y - k_2\right)\,;\\[6pt]
\psi_{j,k_1,k_2}^{[3]} (x,y) &= 2^j \psi\left(2^j x - k_1\right) \psi\left(2^j y - k_2\right)\,,\\
\end{array}
\right\}
\label{2d_wavelets}
\end{equation}
где $j,k_1,k_2\in \mathbb{Z}$. Семейство 
$\{\psi^{[\lambda]}_{j,k_1,k_2}\}_{\lambda,j,k_1,k_2}$
образует ортонормированный базис в~$L^2(\mathbb{R}^2)$. Индекс~$j$ 
в~(\ref{2d_wavelets}) называется масштабом, а индексы $k_1$ и~$k_2$~---
сдвигами. Функция~$\psi$ должна удовлетворять определенным
требованиям, однако ее можно выбрать таким образом, чтобы она
обладала некоторыми полезными свойствами, например была
дифференцируемой нужное число раз и~имела заданное число~$M$
нулевых моментов, т.\,е.
$$
\int\limits_{-\infty}^{\infty}x^k\psi(x)\,dx=0\,,\enskip k=0,\ldots,M-1\,.
$$
В данной работе предполагается, что используются вейвлеты Мейера~[4], 
преобразование Фурье которых обладает достаточным количеством непрерывных 
производных.

Вейвлет-разложение функции $f$ имеет вид:
\begin{equation}                                                                   
\label{waveletdecomp}
f = \sum\limits_{\lambda,j,k_1,k_2} \left\langle f,\psi^{[\lambda]}_{j,k_1,k_2}
\right\rangle \psi^{[\lambda]}_{j,k_1,k_2}\,.
\end{equation}
Идея метода вейвлет-вейг\-лет-раз\-ло\-же\-ния заключается 
в~том, чтобы заменить скалярные произведения в~(\ref{waveletdecomp}) на 
величины, выражающиеся через~Rf, а~не через~$f$. Определим последовательность 
\mbox{функций}
\begin{equation*}
\xi_{j,k_1,k_2}^{[\lambda]} = \fr{1}{4\pi}\, I^{-1} R \psi_{j,k_1,k_2}^{[\lambda]}\,,
%\label{Radon_vaguelette}
\end{equation*}
где $I^{p}$~--- потенциал Рисса, определяемый через преобразование 
Фурье по формуле $\widehat{ I^{p} g}(\omega) \hm= |w|^{-p} \widehat{g}(\omega)$. 
В~[5] показано, что последовательность функций 
$\{2^{-j/2}\xi_{j,k_1,k_2}^{[\lambda]}\}_{\lambda,j,k_1,k_2}$ образует устойчивый базис. 
При этом выполнено $\langle f,\psi_{j,k_1,k_2}^{[\lambda]}\rangle
\hm=\langle \mathrm{Rf}, \xi_{j,k_1,k_2}^{[\lambda]}\rangle $. 
Функции $2^{-j/2}\xi_{j,k_1,k_2}^{[\lambda]}$ получили название 
<<вейглеты>> за свою схожесть с~вейвлетами. Таким образом, формула обращения 
в~методе вей\-влет-вейг\-лет-раз\-ло\-же\-ния выглядит следующим образом:
\begin{equation*}
f = \sum\limits_{\lambda,j,k_1,k_2} \left\langle \mathrm{Rf}, \xi_{j,k_1,k_2}^{[\lambda]}\right\rangle 
\psi^{[\lambda]}_{j,k_1,k_2}\,.
\end{equation*}

\section{Модель данных и~оценка риска}

На практике функция~$f$ обычно задана в~дискретных отсчетах на круге. 
Без ограничения общ\-ности будем считать, что это круг единичного радиу\-са с~центром 
в~начале координат. Проекционные данные в~этом случае измеряются 
при $(s,\theta)\hm\in [-1,1]\times[0,\pi]$.
Пусть $\{e_{i,j}, i,j \in \mathbb{Z}\}$~--- стационарный гауссовский процесс 
с~ковариационной последовательностью $r_{k,p} \hm= \textbf{cov} (e_{i,j},e_{i+k,j+p})$.
Модель проекционных данных с~шумом выглядит сле\-ду\-ющим образом:

\noindent
\begin{equation*}
Y_{i,j} = \mathrm{Rf}_{i,j}+e_{i,j} \enskip i = 1, \dots, 2^J, \; j = 1, \dots, 2^J\,,
\end{equation*}
где $2^J$~--- число отсчетов.

Структура ковариации шума для преобразования Радона должна отражать 
типичную ситуацию: на практике проекции для разных углов регистрируются независимо 
друг от друга. В~рассмат\-ри\-ва\-емой модели ошибок предполагаются независимые наблюдения 
в~случае разных углов и~стационарный гауссовский шум с~нулевым математическим 
ожиданием, конечной дисперсией и~ковариационной последовательностью 
$r_\delta \hm\sim A \delta^{-\alpha}$ ($0\hm<\alpha\hm<1$) для одинаковых углов 
(так называемая долгосрочная зависимость).

При переходе к~дискретному вей\-влет-вейг\-лет-пре\-обра\-зо\-ва\-нию получается 
следующая модель дискретных коэффициентов~\cite{2-sh}:

\noindent
\begin{equation}
\label{WVD_Model}
X_{j,k_1,k_2}^{[\lambda]} = \mu_{j,k_1,k_2}^{[\lambda]} +  
2^{(1-\alpha)J/2} e_{j,k_1,k_2}^{[\lambda]}\,,
\end{equation}
где $\mu_{j,k_1,k_2}^{[\lambda]} \hm=  2^J\langle \mathrm{Rf}, \xi_{j,k_1,k_2}^{[\lambda]}
\rangle$, 
$e_{j,k_1,k_2}^{[\lambda]}\hm=\int\xi_{j,k_1,k_2}^{[\lambda]}d\mathbf{B'}_H$, 
а~$\mathbf{B'}_H(s,\theta)$~--- случайная функция, которая для каждого 
фиксированного угла~$\theta$ представляет собой дробное броуновское 
движение~$\mathbf{B}_H(s)$ с~$H \hm= 1\hm-\alpha/2$ и~имеет некоррелированные 
приращения по~$\theta$. Дисперсии  коэффициентов модели имеют вид:

\noindent
\begin{equation*}
\sigma_{\lambda,j}^2=C_{\lambda,\alpha} 2^{(1-\alpha)J}2^{j\alpha}\,,
\end{equation*}
где константы $C_{\lambda,\alpha}$ ($\lambda\hm=1,2,3$) зависят от пара\-мет\-ров~$A$, 
$\alpha$ и~выбранного вей\-влет-ба\-зиса.

Для подавления шума используется процедура\linebreak пороговой обработки. 
Ее смысл заключается в~удалении достаточно маленьких коэффициентов 
вей\-влет-вейг\-лет-раз\-ло\-же\-ния, которые считают\-ся\linebreak шумом. 
В~данной работе рассматривается так называемая мягкая пороговая обработка 
с~порогом~$T_{\lambda,j}$, зависящим от масштаба~$j$. К~каждому
ко\-эф\-фи\-ци\-ен\-ту применяется функция
$\rho_{T_{\lambda,j}}(x)\hm=\textbf{sgn}(x)\left(\abs{x}\hm -
T_{\lambda,j}\right)_{+}$, т.\,е.\ коэффициенты, которые по модулю 
меньше порога~$T_{\lambda,j}$,
обнуляются, а~абсолютные величины остальных коэффициентов
уменьшаются на величину порога. Погрешность (или риск) мягкой
пороговой обработки определяется следующим образом:

\noindent
\begin{multline}    
R_J(f) = {}\\
\!\!\!{}=\!\sum\limits_{j = 0}^{J - 1}\sum\limits_{k_1=0}^{2^j-1}
\sum\limits_{k_2=0}^{2^j-1}\sum\limits_{\lambda=1}^3 
\Expect\left(\mu_{j,k_1,k_2}^{[\lambda]} -
 \rho_{T_{\lambda,j}}(X_{j,k_1,k_2}^{[\lambda]})\right)^2\!\!.\!\!\!
 \label{Risk_def}
\end{multline}

В~[6] предложено использовать порог $T_{\lambda,j} \hm= \sqrt{2\ln 2^{2j}}
\sigma_{\lambda,j}$, названный универсальным.
В~дальнейшем будет использоваться именно такой вид порога.
В~выражении~(\ref{Risk_def}) присутствуют неизвестные величины 
$\mu_{j,k_1,k_2}^{[\lambda]}$, поэтому вычислить значение~$R_J(f)$ нельзя.
Однако его можно оценить. В~качестве оценки риска используется следующая величина~[7]:
\begin{equation}                                                                                   
\label{risk_Est}
\widehat{R}_J (f) = \sum\limits_{j=0}^{J-1} \sum\limits_{k_1=0}^{2^j-1}
\sum\limits_{k_2=0}^{2^j-1} \sum\limits_{\lambda=1}^3 F[(X_{j,k_1,k_2}^{[\lambda]})^2,T_{\lambda,j}],
\end{equation}
где $F(x,T) \hm= (x - \sigma^2)\mathbf{1}(|x|\leqslant T) \hm+ 
(\sigma^2 \hm+ T^2)\mathbf{1}(|x|\hm>T)$.
Величина $\widehat{R}_J(f)$ является несмещенной оценкой для~$R_J(f)$~[4].

\section{Сильная состоятельность оценки риска}

В работе~[1] исследовались асимптотические свойства оценки~(\ref{risk_Est}) 
в~модели с~независимым шумом.
Было показано, что при определенных условиях гладкости эта оценка является 
состоятельной и~асимптотически нормальной. В~модели с~коррелированным шумом 
аналогичные результаты получены в~работе~[2]. В~данной работе доказывается 
сильная состоятельность оценки~(\ref{risk_Est}) в~модели~(\ref{WVD_Model}).

Приведем вспомогательное утверждение~[8], в~котором оценивается вероятность 
отклонения суммы ограниченных слабозависимых случайных величин от ее 
математического ожидания.

\smallskip

\noindent
\textbf{Лемма.}\ \textit{Пусть $\{X_i,\;i\in Z\}$~--- 
последовательность случайных величин таких, что $\Expect X_i\hm=0$ и~$\abs{X_i}
\leqslant b$ п.в.\ для всех $i\hm\in Z$, где $b\hm>0$~--- 
некоторая константа. Тогда для любого $q\hm\in[1,n/2]$ и~любого} $\eps\hm>0$

\noindent
\begin{multline}
{\sf P}\left(\abs{\sum\limits_{i=1}^n X_i}>n\eps\right)
\leqslant 4\exp\left\{-\fr{\eps^2}{8b^2}q\right\}+{}\\
{}+
22\left(1+\fr{4b}{\eps}\right)^{1/2}q\alpha\left(\left[\fr{n}{2q}\right]\right)\,,
\label{Bosq_inequality}
\end{multline}
\textit{где $\alpha(k)$~--- коэффициент $\alpha$-пе\-ре\-ме\-ши\-ва\-ния 
последовательности}~$\{X_i,\;i\hm\in Z\}$.

\columnbreak

%\smallskip

Докажем теперь сильную состоятельность оценки~(\ref{risk_Est}).

\smallskip

\noindent
\textbf{Теорема.}\ 
\textit{Пусть $f\hm\in  L^2(\mathbb{R}^2)$ задана в~круге единичного 
радиуса с~центром в~начале координат. Тогда имеет место сходимость
\begin{equation}
\label{R_Conv}
\fr{\widehat{R}_J(f)-R_J(f)}{2^{\lambda J}}\rightarrow 0\; \mbox{ п.в. при } 
J\rightarrow\infty
\end{equation}
для любого $\lambda\hm>2$}.

\smallskip

\noindent
Д\,о\,к\,а\,з\,а\,т\,е\,л\,ь\,с\,т\,в\,о\,:\ \ 
Обозначим 
\begin{multline*}
f_{j,k_1,k_2}^{[\lambda]}= F\left[\left(X_{j,k_1,k_2}^{[\lambda]}\right)^2,
T_{\lambda,j}\right]-{}\\
{}-
\Expect F\left[\left(X_{j,k_1,k_2}^{[\lambda]}\right)^2,T_{\lambda,j}\right]\,.
\end{multline*} 
Выберем произвольное $0\hm<p\hm<1$ и~разобьем чис\-ли\-тель в~(\ref{R_Conv}) на две суммы:
$$
\widehat{R}_J(f)-R_J(f)=\widehat{R}_1+\widehat{R}_2\,,
$$
 где
\begin{align*}
\widehat{R}_1&=\sum\limits_{j=0}^{[pJ]-1}\sum\limits_{k_1=0}^{2^j-1}
\sum\limits_{k_2=0}^{2^j-1} \sum\limits_{\lambda=1}^3f_{j,k_1,k_2}^{[\lambda]}\,;
\\
\widehat{R}_2&=\sum\limits_{j=[pJ]}^{J-1}\sum\limits_{k_1=0}^{2^j-1}
\sum\limits_{k_2=0}^{2^j-1} \sum\limits_{\lambda=1}^3f_{j,k_1,k_2}^{[\lambda]}\,.
\end{align*}
Сначала рассмотрим~$\widehat{R}_2$. Для произвольного $\delta\hm>0$ имеем:
\begin{multline}
p_J={\sf P}\left(\abs{\widehat{R}_2}>\delta2^{\lambda J}\right)\leqslant{}\\
\!{}\leqslant\sum\limits_{j=[pJ]}^{J-1}{\sf P}
\left(\abs{\sum\limits_{k_1=0}^{2^j-1}\sum\limits_{k_2=0}^{2^j-1}
 \sum\limits_{\lambda=1}^3f_{j,k_1,k_2}^{[\lambda]}}>\delta J^{-1}2^{\lambda J}\right).\!\!
\label{PJ_inequality}
\end{multline}
Величины $F[(X_{j,k_1,k_2}^{[\lambda]})^2,T_{\lambda,j}]$ ограничены:
$$
-\sigma_{\lambda,j}^2\leqslant F\left[\left(X_{j,k_1,k_2}^{[\lambda]}\right)^2,
T_{\lambda,j}\right]\leqslant\sigma_{\lambda,j}^2+T_{\lambda,j}^2\,.
$$
Повторяя рассуждения работы~[9], можно показать, что в~силу свойств вейвлетов 
Мейера при каждом~$j$ слагаемые в~сумме под вероятностью в~(\ref{PJ_inequality}) 
удовлетворяют свойству $\rho$-пе\-ре\-ме\-ши\-ва\-ния с~коэффициентом~$\rho(k)
\hm\leqslant C k^{-M}$, где $C\hm>0$~--- некоторая константа, а $M$ зависит 
от выбранного вейвлета Мейера и~может быть сделано достаточно большим.


Для коэффициентов $\alpha$-пе\-ре\-ме\-ши\-ва\-ния и~$\rho$-пе\-ре\-ме\-ши\-ва\-ния 
справедливо неравенство $4\alpha(k)\hm \leqslant\rho(k)$~[10]. 
Применяя неравенство~(\ref{Bosq_inequality}) с~$q\hm=2^{2\theta j}$ ($\theta\hm<1$) 
для каждого~$j$ в~сумме~(\ref{PJ_inequality}) и~выбирая такой вейвлет Мейера, 
чтобы~$M$ было достаточно большим, получаем:
\begin{multline}
p_J\leqslant c_1 J\times{}\\
\!{}\times \max\limits_{[pJ]\leqslant j\leqslant J-1}\left\{\!
\exp\left[-c_2J^{-3} 2^{2(\lambda-1+\alpha)J+(2\theta-2\alpha-4)j}\right]\!\right\}+{}\\
{}+o_J\,.
\label{P_inequality}
\end{multline}
Здесь $c_1$ и~$c_2$~--- некоторые положительные константы, а $o_J$ убывает по~$J$ быстрее, 
чем~$2^{-M_0 pJ}$, где~$M_0$~--- некоторое положительное число, зависящее от~$M$.

Имеем $2\theta-2\alpha-4<0$, и~при $j\hm=J$ правая часть~(\ref{P_inequality}) не 
превосходит $c_1 J\exp\left[-c_2J^{-3} 2^{(2\lambda-6+2\theta)J}\right]$. 
Поскольку $\theta\hm<1$ можно выбрать произвольно, для того чтобы выполнялось 
неравенство $2\lambda\hm-6\hm+2\theta\hm>0$, достаточно потребовать $\lambda\hm>2$. 
При таком выборе~$\lambda$
\begin{equation*}
\sum\limits_{J=1}^{\infty}p_J<\infty
\end{equation*}
и~в~силу леммы Бо\-ре\-ля--Кан\-тел\-ли для любого $\delta>0$ событие $\{\abs{\widehat{R}_2}>\delta2^{\lambda J}\}$ осуществляется лишь конечное число раз. Следовательно, $\widehat{R}_2 2^{-\lambda J}\rightarrow 0$ п.~в.

Рассмотрим теперь $\widehat{R}_1$. Общее число сла\-га\-емых в~$\widehat{R}_1$ не 
превосходит~$2^{2[pJ]}$. Число слагаемых при каж\-дом~$j$ равно~$3\cdot2^{2j}$, 
и~при фиксированном~$j$ каждое слагаемое не превосходит по 
модулю $B J\cdot2^{J(1-\alpha)+\alpha j}$, где $B\hm>0$~--- 
некоторая константа. Следовательно, 
$\abs{\widehat{R}_1}\hm\leqslant B_1J\cdot 2^{J(1-\alpha+(\alpha+2)p)}$ п.в., 
где $B_1\hm>0$~--- некоторая константа. При любых $0\hm<\alpha\hm<1$ 
и~$\lambda\hm>2$ можно выбрать такое $0\hm<p\hm<1$, что 
$\lambda\hm-1\hm+\alpha\hm-(\alpha\hm+2)p\hm>0$. Следовательно, 
$\widehat{R}_1 \cdot 2^{-\lambda J}\rightarrow 0$ п.в. Теорема доказана.

{\small\frenchspacing
 {%\baselineskip=10.8pt
 \addcontentsline{toc}{section}{References}
 \begin{thebibliography}{99}

\bibitem{1-sh}
\Au{Маркин А.\,В., Шестаков О.\,В.} 
Асимптотики оценки риска при пороговой обработке вей\-влет-вейг\-лет 
коэффициентов в~задаче
томографии~// Информатика и~её применения, 2010. Т.~4. Вып.~2. С.~36--45.

\bibitem{2-sh}
\Au{Ерошенко А.\,А., Шестаков О.\,В.} 
Асимптотические свойства оценки риска в~задаче восстановления изображения 
с~коррелированным шумом при обращении преобразования Радона~// 
Информатика и~её применения, 2014. Т.~8. Вып.~4. C.~32--40.

\bibitem{3-sh}
\Au{Donoho D.} Nonlinear solution of linear inverse problems 
by wavelet-vaguelette decomposition~// Appl. Comput. Harmonic Anal., 1995. 
Vol.~2. P.~101--126.

\bibitem{4-sh} 
\Au{Mallat S.} A~wavelet tour of signal processing.~--- Academic Press, 1999. 857~p.

\bibitem{5-sh} 
\Au{Lee N.} Wavelet-vaguelette decompositions and homogenous equations. 
Purdue University, 1997. PhD Thesis. 93~p.

\bibitem{6-sh} 
\Au{Kolaczyk E.\,D.} Wavelet methods for the inversion of certain homogeneous 
linear operators in the presence of noisy data.  Stanford University, 1994.  
PhD Thesis. 163~p.

\bibitem{7-sh} 
\Au{Donoho D., Johnstone~I.\,M.} Adapting to unknown smoothness via wavelet shrinkage~// 
J.~Amer. Stat. Assoc., 1995. Vol.~90. P.~1200--1224.

\bibitem{8-sh} 
\Au{Bosq D.} Nonparametric statistics for stochastic processes: Estimation 
and prediction.~--- New York, NY, USA: Springer-Verlag, 1996. 169~p.

\bibitem{9-sh} 
\Au{Johnstone I.\,M.} Wavelet shrinkage for correlated data and inverse problems:
Adaptivity results~// Statistica Sinica, 1999. Vol.~9. No.~1. P.~51--83.

\bibitem{10-sh} 
\Au{Bradley R.\,C.} Basic properties of strong mixing conditions. 
A~survey and some open questions // Probab. Surveys, 2005. Vol.~2. P.~107--144.

\end{thebibliography}

 }
 }

\end{multicols}

\vspace*{-6pt}

\hfill{\small\textit{Поступила в~редакцию 01.07.16}}

\vspace*{6pt}

%\newpage

%\vspace*{-24pt}

\hrule

\vspace*{2pt}

\hrule

\vspace*{6pt}



\def\tit{THE STRONG LAW OF~LARGE NUMBERS FOR~THE~RISK ESTIMATE IN~THE~PROBLEM 
OF~TOMOGRAPHIC IMAGE RECONSTRUCTION FROM~PROJECTIONS WITH~A~CORRELATED NOISE}

\def\titkol{The strong law of large numbers for the risk estimate in~the~problem 
of~tomographic image reconstruction} % from projections with~a~correlated noise}

\def\aut{O.\,V.~Shestakov$^{1,2}$}

\def\autkol{O.\,V.~Shestakov}

\titel{\tit}{\aut}{\autkol}{\titkol}

\vspace*{-9pt}

\noindent
$^1$Department of Mathematical Statistics, Faculty of Computational Mathematics 
and Cybernetics,\linebreak
$\hphantom{^1}$M.\,V.~Lomonosov Moscow State University, 1-52~Leninskiye Gory, 
GSP-1, Moscow 119991, Russian\linebreak
$\hphantom{^1}$Federation

\noindent
$^2$Institute of Informatics Problems, Federal Research Center 
``Computer Science and Control''
of the Russian\linebreak
$\hphantom{^1}$Academy of Sciences, 44-2~Vavilov Str., Moscow 119333,  Russian Federation



\def\leftfootline{\small{\textbf{\thepage}
\hfill INFORMATIKA I EE PRIMENENIYA~--- INFORMATICS AND
APPLICATIONS\ \ \ 2016\ \ \ volume~10\ \ \ issue\ 3}
}%
 \def\rightfootline{\small{INFORMATIKA I EE PRIMENENIYA~---
INFORMATICS AND APPLICATIONS\ \ \ 2016\ \ \ volume~10\ \ \ issue\ 3
\hfill \textbf{\thepage}}}

\vspace*{3pt}



\Abste{Methods of wavelet analysis based on thresholding of coefficients 
of the projection decomposition are widely used for solving the problems 
of tomographic image reconstruction in medicine, biology, astronomy, 
and other areas. These methods are easily implemented through fast algorithms; 
so, they are very appealing in practical\linebreak\vspace*{-12pt}}

\Abstend{situations. Besides, they allow the 
reconstruction of local parts of the images using incomplete projection data, 
which is essential, for example, for medical applications, where it is not 
desirable to expose the patient to the redundant radiation dose. Wavelet 
thresholding risk analysis is an important practical task, because it 
allows determining the quality of the techniques themselves and of the 
equipment which is being used. The present paper considers the problem 
of estimating the function by inverting the Radon transform in the model 
of data with correlated noise. The paper considers the wavelet-vaguelette 
decomposition method of reconstructing tomographic images in the model 
with a correlated noise. It is proven that the unbiased mean squared error risk estimate 
for thresholding wavelet-vaguelette coefficients of the image function 
satisfies the strong law of large numbers, i.\,e., it is a strongly consistent estimate.}

\KWE{wavelets; thresholding; MSE risk estimate; Radon transform}

\DOI{10.14357/19922264160306} 

\vspace*{-9pt}

\Ack
\noindent
The work was partly supported by the Russian Foundation for 
Basic Research (project 15-07-02652).


%\vspace*{3pt}

  \begin{multicols}{2}

\renewcommand{\bibname}{\protect\rmfamily References}
%\renewcommand{\bibname}{\large\protect\rm References}

{\small\frenchspacing
 {%\baselineskip=10.8pt
 \addcontentsline{toc}{section}{References}
 \begin{thebibliography}{99}

\bibitem{1-sh-1}
\Aue{Markin, A.\,V., and O.\,V.~Shestakov}. 
2010. Asimptotiki otsenki riska pri porogovoy obrabotke veyvlet-veyglet 
koeffitsientov v zadache tomografii
[The asymptotics of risk estimate for wavelet-vaguelette 
coefficients' thresholding in the problems of tomography]. 
\textit{Informatika i~ee Primeneniya~--- Inform. Appl.} 4(2):36--45.

\bibitem{2-sh-1}
\Aue{Eroshenko, A.\,A., and O.\,V.~Shestakov}. 2014. 
Asimpto\-ti\-che\-skie svoystva otsenki riska v~zadache vosstanovleniya izob\-ra\-zhe\-niya 
s~korrelirovannym shumom pri obrashchenii preobrazovaniya Radona 
[Asymptotic properties of risk estimate in the problem of reconstructing 
images with correlated noise by inverting the Radon transform]. 
\textit{Informatika i~ee Primeneniya~--- Inform. Appl.} 8(4):32--40.

\bibitem{3-sh-1} 
\Au{Donoho, D.} 1995. 
Nonlinear solution of linear inverse problems by wavelet-vaguelette decomposition. 
\textit{Appl. Comput. Harmonic Anal.}  2:101--126.

\bibitem{4-sh-1} 
\Aue{Mallat, S.} 1999. 
\textit{A~wavelet tour of signal processing.}  New York, NY: Academic Press. 857~p.

\bibitem{5-sh1} 
\Aue{Lee, N.} 1997. {Wavelet-vaguelette decompositions and homogenous equations.
 West Lafayette: Purdue University.  PhD Thesis. 93~p.

\bibitem{6-sh-1} 
\Aue{Kolaczyk, E.\,D.} 1994. 
Wavelet methods for the inversion of certain homogeneous linear operators 
in the presence of noisy data.  Stanford: Stanford University. 
PhD Thesis. 163~p.

\bibitem{7-sh-1}} 
\Aue{Donoho, D., and I.\,M.~Johnstone}. 1995. Adapting to 
unknown smoothness via wavelet shrinkage. \textit{J.~Amer. Stat. Assoc.} 90:1200--1224.

\bibitem{8-sh-1} 
\Aue{Bosq, D.} 1996. 
\textit{Nonparametric statistics for stochastic processes: Estimation and prediction.} 
New York, NY: Springer-Verlag. 169~p.

\bibitem{9-sh-1} 
\Aue{Johnstone, I.\,M.} 1999. Wavelet shrinkage for correlated data 
and inverse problems: Adaptivity results. \textit{Statistica Sinica} 9(1):51--83.

\bibitem{10-sh-1} 
\Aue{Bradley, R.\,C.} 2005. Basic properties of strong mixing conditions. 
A~survey and some open questions. \textit{Probab. Surveys}  2:107--144.

\end{thebibliography}

 }
 }

\end{multicols}

\vspace*{-3pt}

\hfill{\small\textit{Received July 01, 2016}}

\Contrl

\noindent
\textbf{Shestakov Oleg V.} (b.\ 1976)~--- 
Doctor of Science in physics and mathematics, associate professor, 
Department of Mathematical Statistics, Faculty of Computational Mathematics 
and Cybernetics, M.\,V.~Lomonosov Moscow State University, 1-52~Leninskiye Gory, 
GSP-1, Moscow 119991, Russian Federation; senior scientist, 
Institute of Informatics Problems, Federal Research Center 
``Computer Science and Control''
of the Russian Academy of Sciences, 44-2~Vavilov Str., Moscow 119333, 
Russian Federation; \mbox{oshestakov@cs.msu.su}


\label{end\stat}


\renewcommand{\bibname}{\protect\rm Литература}