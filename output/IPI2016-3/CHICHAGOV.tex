
 \def\stat{chichagov}
 
\def\tit{АСИМПТОТИЧЕСКИЕ РАЗЛОЖЕНИЯ СРЕДНЕЙ АБСОЛЮТНОЙ ОШИБКИ 
НЕСМЕЩЕННОЙ ОЦЕНКИ С~РАВНОМЕРНО МИНИМАЛЬНОЙ ДИСПЕРСИЕЙ 
%НОРМД 
И %ОМП 
ОЦЕНКИ
 МАКСИМАЛЬНОГО ПРАВДОПОДОБИЯ 
 В~МОДЕЛИ ОДНОПАРАМЕТРИЧЕСКОГО ЭКСПОНЕНЦИАЛЬНОГО СЕМЕЙСТВА
 РЕШЕТЧАТЫХ РАСПРЕДЕЛЕНИЙ$^*$\\[-5pt]}

\def\titkol{Асимптотические разложения средней абсолютной ошибки НОРМД и~ОМП
 в~модели %однопараметрического 
 экспоненциального семейства} % решетчатых распределений}

\def\aut{В.\,В.~Чичагов$^1$\\[-7pt]}

\def\autkol{В.\,В.~Чичагов}

\titel{\tit}{\aut}{\autkol}{\titkol}

\index{Чичагов В.\,В.}
\index{Chichagov V.\,V.}


{\renewcommand{\thefootnote}{\fnsymbol{footnote}} \footnotetext[1]
{Работа выполнена при финансовой поддержке
 Министерства образования и~науки РФ (проект 2096).}}


\renewcommand{\thefootnote}{\arabic{footnote}}
\footnotetext[1]{Пермский государственный национальный исследовательский университет, 
 \mbox{chichagov@psu.ru}}
 
\vspace*{-18pt}

\Abst{Рассмотрена модель повторной выборки фиксированного объема~$n$ 
из решетчатого распределения, принадлежащего естественному однопараметрическому
 экспоненциальному семейству. При неограниченном возрастании $n$ найдены 
 асимптотические разложения
 средних абсолютных ошибок несмещенной оценки с~равномерно минимальной дисперсией 
 (НОРМД) и~оценки
 максимального правдоподобия (ОМП) заданной параметрической функции~$G[a]$.
 Отдельно исследован случай, когда $G'[a]\hm=0$, но $G''[a]\hm\neq 0.$
 В~случае распределения Пуассона для двух параметрических функций
 проведена оценка относительной погрешности вычисления разности средних
 абсолютных ошибок НОРМД и~ОМП с~помощью полученных асимптотических разложений.
 Установлено, что асимптотические результаты при достаточно большом объеме выборки
 позволяют сравнивать НОРМД и~ОМП с~помощью такого показателя качества оценок,
 как средняя абсолютная ошибка.}

\KW{экспоненциальное семейство; решетчатое распределение;
 несмещенная оценка; оценка максимального правдоподобия; асимптотическое разложение}
 
\DOI{10.14357/19922264160309} 
  
\vspace*{-8pt}

\vskip 8pt plus 9pt minus 6pt

\thispagestyle{headings}

\begin{multicols}{2}

\label{st\stat}

\section{Введение}

\vspace*{-2pt}

  Несмещенные оценки с~равномерно минимальной дисперсией и~оценки
  максимального правдоподобия  играют существенную роль в~современных
  статистических исследованиях. Выбор лучшей из них можно сделать,
  используя одну из мер близости статистической оценки к~истинному значению
  оцениваемой функции. В~данной работе в~качестве такой меры близости
  используется средняя абсолютная ошибка статистической оценки.
  
%  \smallskip

\noindent
  \textbf{Определение.} 
  Средней абсолютной ошибкой статистической оценки $T[\mathbf{X}]$ при 
  оценивании значения параметрической функции $G[a]$ будем называть функционал
  
  \noindent
  \begin{equation}
  \mathbf{MAE}\left[T[\mathbf{X}],G[a]\right]=
  \mathbf{E}\left\vert T[\mathbf{X}]-G[a]\right\vert \,,
  \label{e1-ch}
  \end{equation}
  где $\mathbf{X}=\left(X_1,X_2,\ldots,X_n\right)$~--- выборка, 
  по которой строится оценка~$T[\mathbf{X}]$.
  
  %\smallskip

  Этот показатель качества статистической оценки является более 
  естественной мерой ее точности, но в~меньшей степени поддается 
  математическому анализу по сравнению с~такой распространенной мерой точности, 
  как среднеквадратическая ошибка статистической оценки. Получить явные формулы 
  для вычисления функционала~(1) удается лишь в~редких случаях.

  Целью данной работы является получение асимптотических разложений для 
  НОРМД и~ОМП, справедливых при большом объеме выборки~$n$ в~случае 
  однопараметрического экспоненциаль-\linebreak ного семейства решетчатых распределений.
  Случай однопараметрического экспоненциального семейства абсолютно 
  непрерывных распределений \mbox{изучен} в~[1]. Задача получения асимптотических разложений
  среднеквадратических ошибок \mbox{НОРМД} и~ОМП при большом объеме выборки~$n$
  в~случае однопараметрического экспоненциального семейства распределений
  была рассмотрена в~работе~[2]. Более общие результаты, чем в~[2],
  для НОРМД представлены в~[3].
  
  \vspace*{-12pt}


\section{Описание модели наблюдений, основные предположения и~обозначения}

\vspace*{-2pt}


   Имеется $\mathbf{X}=\left(X_1,X_2,\ldots,X_n\right)$~--- повторная выборка,
  элементы которой являются независимыми случайными величинами, имеющими то 
  же самое распределение, что и~наблюдаемая случайная величина~$\xi$.

   Распределение случайной величины~$\xi$ принадлежит экспоненциальному семейству
  решетчатых распределений, определяемых набором вероятностей
  \begin{multline}
   \mathbf{P}(\xi=x) =
    \exp \left\{ \Phi_1[a] \, T[x] -\kappa[\Phi_1[a]]+ d[x ]\right\}\,,\\
   x\in {\mathbb X}_\xi\subset \mathbf{Z}\,,
   \label{e2-ch}
  \end{multline}
  с~параметром $a \hm= \mbox {\bf E} T[\xi]\in {\mathbb A}$.
  Здесь $d[x]$,  $T[x]$ и~$\Phi_1[a]$~--- заданные известные функции;
  $\kappa[\theta]$~--- кумулянтное преобразование распределения.

   Будем предполагать, что распределение~(\ref{e2-ch}) удовле\-тво\-ря\-ет следующим условиям
  регулярности.
  \begin{description}
  \item[\,] $\bf (A_1).$
   Носитель распределения~${\mathbb X}_\xi$ содержится в~$\bf Z$, но ни 
   в~какой подрешетке~$\bf Z$ и~не зависит от параметра~$a$.
  \item[\,] $\bf (A_2).$ Пусть $\widetilde{\Theta}$~--- множество 
  значений~$\theta$, удовлетворяющих соотношению
  \begin{equation*}
   \sum\limits_{x\in {\mathbb X}_G} \exp \left\{\theta  T[x]+d[x]\right\}<\infty\,.
  \end{equation*}
   Параметрическое множество ${\mathbb A}\hm= \mathbf{Int} 
   \left\{a=\kappa'[\theta]:\,\theta\in\widetilde{\Theta}\right\}$ не пусто.
  \item[\,] $\bf (A_3).$ $\Phi_1[a]$~--- бесконечное число раз 
  дифференцируемая функция, причем
      $\Phi'_1[a]\hm > 0$ при $a\hm\in {\mathbb A}$.
  \end{description}

  Далее в~работе будем придерживаться сле\-ду\-ющих обозначений:
  \begin{description}
\item[\,] $G[a]$~--- заданная параметрическая функция, $a\hm\in {\mathbb A}$;
\item[\,]
 $S_n=\sum\limits_{i=1}^n T(X_i)$~--- минимальная достаточная статистика распределения~(2);
 \item[\,]
 $\widehat{G}[a|S_n]$~--- НОРМД функции~$G[a]$;
\item[\,]
 $\widetilde{G}[a|S_n]$~--- ОМП функции~$G[a]$;
\item[\,]
 $\varphi[x]=({1}/{\sqrt{2\pi}})\exp\left[-x^2/2\right]$~--- 
 плот\-ность стандартного нормального распределения;
 \item[\,]
 $\Phi[x]=\int\limits_{-\infty}^x \varphi[t]\, dt$;
 \item[\,]
 $U^{(k)}[a]={d^k U[a]}/{da^k}$~--- производная $k$-го порядка от функции $U[a]$;
 \item[\,]
 $U[a]^j=(U[a])^j$~--- $j$-я степень функции $U[a]$;
 \item[\,]
 $\lfloor x \rfloor$~--- дробная часть~$x$;
 \item[\,]
 $\ell_0[x]=1$, $\ell_1[x]\hm=\lfloor x \rfloor\hm-{1}/{2}$,
$\ell_2[x]\hm=({1}/{2})\left(\lfloor x \rfloor^2\hm-\lfloor x \rfloor\hm+1/6\right)$,
 $0\hm\le x \hm< 1,$~--- функции, используемые в~формуле суммирования 
 Эй\-ле\-ра--Мак\-ло\-рена;
\item[\,]
 $I(D)$~--- индикатор события~$D$.
 \end{description}


\section{Основной результат и~его~доказательство}


  \noindent
  \textbf{Теорема.}\ \textit{Пусть выполнены условия~${\bf (A_1)}$--${\bf(A_3)}$, 
  а~последовательности среднеквадратических ошибок 
  $\left\{\mathbf{V} \widehat{G}[a|S_n]\right\}$ 
  и~$\left\{\mathbf{E} \left(\widetilde{G}[a|S_n]\hm-G[a]\right)^2\right\}$ 
  равномерно ограничены   начиная с~некоторого $n\hm\ge n_0$.
  Тогда при $n \hm\to \infty$ и~$G'[a]\hm\neq 0$ справедливы разложения}:
  
\noindent
   \begin{multline}
\mathbf{E} \left|\widehat{G}[a|S_n]-G[a]\right|={}\\
{}=
  \fr{2\varphi[0]\left|G'[a]\right|}{\sqrt{n\Phi_1 '[a]}}
   \left\{
   \vphantom{\fr{G^{(3)}[a]}{6 G'[a]}
  +\fr{\Phi_1^{(3)}[a]}{24 \Phi_1'[a]}}
   1-\fr{\Phi_1 '[a]}{n}\,\ell_2[na+\epsilon]
  +{}\right.
\\
{}  +\fr{1}{n\Phi_1'[a]}\left(
\vphantom{\fr{G^{(3)}[a]}{6 G'[a]}
  +\fr{\Phi_1^{(3)}[a]}{24 \Phi_1'[a]}}
  \fr{G''[a]^2}{8 G'[a]^2}
  +\fr{G''[a] \Phi_1''[a]}{4 G'[a] \Phi_1'[a]}
  -\fr{\Phi_1''[a]^2}{12\Phi_1'[a]^2}-{}\right.\\
\left.\left.  {}  -\fr{G^{(3)}[a]}{6 G'[a]}
  +\fr{\Phi_1^{(3)}[a]}{24 \Phi_1'[a]}\right)\right\}+
  \mathbf{o}\left(\fr{1}{n^{3/2}}\right)\,,
\label{e3-ch}
  \end{multline}
  
 \noindent
\textit{где}
$$
\epsilon=\fr{G''[a]}{2G'[a]\,\Phi_1 '[a]}\,;
$$

\vspace*{-12pt}

\noindent
  \begin{multline}
\mathbf{E} \left|\widetilde{G}[a|S_n]-G[a]\right|={}\\
{}=
  \fr{2\varphi[0]\left|G'[a]\right|}{\sqrt{n\Phi_1 '[a]}}
 \left\{
 \vphantom{\fr{\Phi_1^{(3)}[a]}{24 \Phi_1'[a]}}
 1 -\fr{\Phi_1 '[a]}{n}\,\ell_2[na]+{}\right.\\ 
 {} +\fr{1}{n\Phi_1'[a]}\left(
 \vphantom{\fr{\Phi_1^{(3)}[a]}{24 \Phi_1'[a]}}
  -\fr{G''[a] \Phi_1''[a]}{6 G'[a] \Phi_1'[a]}
  -\fr{\Phi_1''[a]^2}{12\Phi_1'[a]^2}
  +\fr{ G^{(3)}[a]}{3 G'[a] }
  +{}\right.\\
\left.\left.  {}+\fr{\Phi_1^{(3)}[a]}{24 \Phi_1'[a]}\right)\right\}
  +\mathbf{o}\left(\fr{1}{n^{3/2}}\right)\,.
  \label{e4-ch}
 \end{multline}
  \textit{Если}  $G'[a]= 0$, $G''[a]\hm\neq 0$, \textit{то при} $n \hm\to \infty$
  
  \noindent
  \begin{align}
 \mathbf{E} &\left|\widehat{G}[a|S_n]-G[a]\right|=
   \sqrt{\fr{2}{\pi e}}\,\fr{\left|G''[a]\right|}{ n \Phi_1'[a]}+\mathbf{o}\left(
   \fr{1}{n}\right)\,;
 \label{e5-ch}\\
 \mathbf{E} &\left|\widetilde{G}[a|S_n]-G[a]\right|=
   \fr{\left|G''[a]\right|}{2n \Phi_1'[a]}+\mathbf{o}\left(\fr{1}{n}\right)\,.
 \label{e6-ch}
  \end{align}

\noindent
\textbf{Примечание 1.}\ 
Если из разложений~(3) и~(4) исключить слагаемые, содержащие в~качестве одного 
из множителей функцию $\ell_2[\cdot]$, то в~результате получим асимптотические 
разложения средних абсолютных ошибок НОРМД и~ОМП функции $G[a]$, соответствующие 
случаю, когда распределение случайной величины~$\xi$ абсолютно непрерывно 
и~принадлежит экспоненциальному семейству. Этот результат впервые получен в~[1].

%\smallskip

\noindent
\textbf{Примечание 2.}\ Поскольку $\sqrt{{2}/({\pi e}})\hm< {1}/{2},$
 то при достаточно большом объеме выборки $n$ и~$G'[a]\hm=0$,\linebreak\vspace*{-12pt}
 
 \pagebreak
 
 \noindent 
 но $G''[a]\hm\neq 0$ средняя абсолютная ошибка НОРМД меньше средней 
 абсолютной ошибки ОМП функции $G[a]$  при любом $a\hm\in {\mathbb A}$.

\smallskip

\noindent
\textbf{Примечание 3.}\ Предположение о равномерной ограниченности 
последовательности дисперсий $\left\{\mathbf{V} \widehat{G}[a|S_n]\right\}$
 выполнено, если найдется некоторое $L\hm\in \mathbf{N}$, для которого
 $\mathbf{V} \widehat{G}[a|S_L]\hm<\infty$ [3, лемма~5.11].

\smallskip

\noindent
Д\,о\,к\,а\,з\,а\,т\,е\,л\,ь\,с\,т\,в\,о\ теоремы. 
Приведем сначала основные обозначения и~соотношения, которые будут использоваться 
для доказательства утверждения теоремы.

  Положим
 $$
 Z_n=\fr{S_n-na}{b\sqrt{n}}
 $$
 при $b=\sqrt{\mathbf{V}T[\xi]}.$

  При $|z|\le \sqrt {2\alpha \ln n}$ для любых $\alpha\hm>0,$
 $0\hm<\delta_1\hm<0{,}5$, $0\hm<\delta_2\hm<0{,}5$ верны следующие асимптотические 
 разложения функций $\widehat{G}[a;z]$ и~$\widetilde{G}[a;z]$, определяющих 
 НОРМД $\widehat{G}[a;Z_n]\hm=\widehat{G}[a|S_n]$ и~ОМП 
 $\widetilde{G}[a;Z_n]\hm=\widetilde{G}[a|S_n]$ параметрической функции $G[a]$:
 \begin{multline}
 \widehat{G}[a;z]-G[a]=
 \fr{G'[a] z}{\sqrt{n\Phi'_1[a]}}
 + \fr{G''[a] H_2[z]}{2 n\Phi'_1[a]}
 +{}\\
 \!{}+ \fr{G'''[a] H_3[z]}{6 n^{3/2}\Phi'_1[a]^{3/2}}+ \fr{G''[a]\Phi''_1[a] z}{2 n^{3/2}\Phi'_1[a]^{5/2} }
 + \mathbf{O}\left( {n^{-2+\delta_1} } \right);\!\!
 \label{e7-ch}
\end{multline}

\vspace*{-12pt}


\noindent
\begin{multline}
 \widetilde{G}[a;z]-G[a]=
 \fr{G'[a] z}{\sqrt{n\Phi'_1[a]}}
 + \fr{G''[a] z^2}{2 n\Phi'_1[a]}+{}\\
{} + \fr{G'''[a] z^3}{6 n^{3/2}\Phi'_1[a]^{3/2}}
 + \mathbf{O}\left(n^{ -2+\delta_2}\right).
 \label{e8-ch}
 \end{multline}
 Справедливость разложения~(\ref{e7-ch}) установлена в~следствии~6.2 из~[3].
 В~условиях теоремы ОМП функции $G[a]$ равна $\widetilde{G}[a;Z_n]\hm=G[S_n/n],$
 а~несмещенно оцениваемая функция $G[a]$ имеет производные всех порядков 
 (см., например,~[3, следствие~5.3]). Поэтому разложение~(\ref{e8-ch}) 
 нетрудно получить с~по\-мощью формулы Тейлора, примененной к~функции $G[s/n]$ в~точке~$a$ 
 при $z\hm=(s\hm-na)/(b\sqrt{n})$:
 \begin{multline*}
 \widetilde{G}_n[a;z]=G\left[\fr{s}{n}\right]
  ={}\\
  {}=\sum\limits_{j=0}^3 \fr{G^{(j)}[a]}{j!}\left(\fr{s}{n}-a\right)^j+
   \fr{G^{(4)}[\eta]}{4!}\left(\fr{s}{n}-a\right)^4={}\\
{}=\sum\limits_{j=0}^3 \fr{G^{(j)}[a]}{j!}\left(\fr{b z}{\sqrt{n}}\right)^j+
   \fr{G^{(4)}[\eta]}{4!}\left(\fr{b z}{\sqrt{n}}\right)^4={}
\\
  {}=\sum\limits_{j=0}^3 \fr{G^{(j)}[a]}{j!}
  \left(\fr{z}{\sqrt{n \Phi'_1[a]}}\right)^j+
   {\bf O}\left(\fr{\ln^2 n}{n^2}\right)\,,
 \end{multline*}
 где $\eta$ принадлежит интервалу с~концами $s/n$ и~$a$. 
 При этом использован тот факт, что для распределения~(\ref{e2-ch})
 $$
 b=\fr{1}{\sqrt{\Phi'_1[a]}}\,.
 $$

 Кроме того, потребуются приближения Эдж\-вор\-та 2-го и~3-го порядка
  \begin{align}
 p_2[z,n]&=\varphi[z]\left( 1 + \fr{\rho_3 H_3[z]}{ 6\sqrt {n }} +
 \fr{\rho_4 H_4[z]}{24n} +{}\right.\notag\\
 &\hspace*{40mm}\left.{}+ \fr{\rho_3^2 H_6[z]}{72n} \right)\,;
 \label{e9-ch}\\
 p_3[z,n]&=\varphi[z]\left( 1 + \fr{\rho_3 H_3[z]}{ 6\sqrt {n }} +
 \fr{\rho_4 H_4[z]}{24n} +{}\right.\notag\\
  &\hspace*{5mm}
{} + \fr{\rho_3^2 H_6[z]}{72n}+\fr{\rho_5 H_5[z]}{5! n^{3/2}}
 +\fr{\rho_3 \rho_4 H_7[z]}{3!4! n^{3/2}}+{}\notag\\
&\hspace*{35mm} \left.{}+ \fr{\rho_3^3 H_9[z]}{(3!)^4 n^{3/2}}
 \right)\,,\!
 \label{e10-ch}
 \end{align}
 где $H_j[x]$~--- полином Че\-бы\-шё\-ва--Эр\-ми\-та порядка~$j$, 
 а~$\rho_3$ и~$\rho_4$~--- коэффициенты асимметрии и~эксцесса случайной 
 величины $T[\xi],$ которые, как отмечается в~[2], определяются выражениями:
 \begin{equation}
 \rho_3  =  - \fr{\Phi''_1 [a]}{\Phi'_1[a]^{3/2}}\,;\enskip
 \rho_4  = \fr{3 \Phi''_1[a]^2 }{\Phi'_1[a]^3}
 - \fr{\Phi'''_1[a]}{\Phi'_1[a]^2 }.
 \label{e11-ch}
 \end{equation}
 Формула, определяющая явный вид нормированного кумулянта~$\rho_5$, не приводится,
 поскольку в~утверждении теоремы его нет.

  Рассмотрим сначала случай $G'[a]\hm\neq 0$. В~этом случае 
  соотношения~(\ref{e7-ch}) и~(\ref{e8-ch}) удобнее представить иначе:
 \begin{align}
  \widehat{G}[a;z]-G[a]&={}\notag\\
  &\hspace*{-15mm}{}=\fr{G'[a]}{\sqrt {n\Phi'_1 [a] }} \left\{
  h_1[z,n] + \mathbf{O}\left( {n^{- 3/2+\delta_1} } \right) \right\}\,;
\label{e12-ch}
 \\ 
 \widetilde{G}[a;z]-G[a]&= {}\notag\\
&\hspace*{-15mm}{}= \fr{G'[a]}{\sqrt {n\Phi '_1 [a]} }
    \left\{ h_2[z,n] + \mathbf{O}\left(n^{ - 3/2+\delta_2}\right)
  \right\}, \notag % \label{e13-ch}
  \end{align}
 где
 \begin{gather*}
h_1[z,n]= z
 + \fr{A H_2[z]}{\sqrt{n}}  + \fr{B_1 H_3[z]+B_2 z}{n}\,;
 \notag \\
h_2[z,n]= z + \fr{A z^2}{\sqrt {n }} + \fr{B z^3}{n}\,;\\
% \notag \\
  A = \fr{G''[a]}{2 G'[a]\sqrt {\Phi'_1[a]} }\,;\enskip
 B = \fr{G'''[a]}{6 G'[a]\Phi'_1[a]}\,;
 \label{e14-ch}\\
B_1  = \fr{G'''[a]}{6 G'[a]\Phi '_1[a]}\,;\enskip
 B_2  = \fr{G''[a]\Phi ''_1 [a]}{2 G'[a]\Phi'_1[a]^2 }\,.
% \label{e15-ch}
 \end{gather*}

 Поскольку $\mathbf{E}\left|Z_n\right|^m\hm<\infty$ при любом $m\hm\in \mathbf{N}$,
 то по теореме Аносовой~[4, с.~309] для любого $\alpha\hm>0$
 \begin{equation*}
 \mathbf{P}\left( \left| {Z_n } \right| \ge \sqrt {2\alpha \ln n} \right)
 = \fr{1}{n^\alpha \sqrt{\pi\alpha\ln n}}
\! \left(\!1+\mathbf{O}\!\left( \fr{1}{\ln n} \right)\!\right).
 \end{equation*}
 Поэтому в~силу неравенства Ко\-ши--Бу\-ня\-ков\-ско\-го при $j\hm=1,2$ и~равномерной 
 ограниченности последовательности $\left\{\mathbf{E} |Z_n^m|\right\}$
 для любого $m\hm\in \mathbf{N},$ а~также
 $\left\{\mathbf{V} \widehat{G}[a|S_n]\right\}$ 
 и~$\left\{\mathbf{E} \left(\widetilde{G}[a|S_n]\hm-G[a]\right)^2\right\}$
 начиная с~некоторого $n\hm\ge n_0$ верны оценки:
 \begin{multline}
 \mathbf{E}\left| Z_n^m  I\left( \left| {Z_n } \right| > 
 \sqrt {2\alpha \ln n} \right)\right|\le{}
 \\ 
 \le \sqrt{\mathbf{E}\left|Z_n\right|^{2m}
 \mathbf{P}\left( \left| {Z_n } \right| \ge \sqrt {2\alpha \ln n} \right)}
 ={}\\
 {}= \mathbf{o}\left(n^{-\alpha/2}\right)\,;
 \label{e16-ch}
 \end{multline}
 
 \vspace*{-12pt}
 
 \noindent
 \begin{multline}
 \mathbf{E} \left\{\left|\widehat{G}[a|S_n]-G[a]\right|
 \, I\left(|Z_n|>\sqrt{2\alpha\ln n}\right)\right\}\le{} 
\\ 
 \le\sqrt{\mathbf{V} \widehat{G}[a|S_n]
 \mathbf{P}\left( \left| {Z_n } \right| \ge \sqrt {2\alpha \ln n}\right)}
  ={}\\
  {}= \mathbf{o}\left(n^{-\alpha/2}\right)\,;
\label{e17-ch}
\end{multline}

 
 \vspace*{-12pt}
 
 \noindent
 \begin{multline}
 \mathbf{E} \left\{\left|\widetilde{G}[a|S_n]-G[a]\right|
 \, I\left(|Z_n|>\sqrt{2\alpha\ln n}\right)\right\}\le{}\\
\\
 {}\le\sqrt{\mathbf{E} \left(\widetilde{G}[a|S_n]-G[a]\right)^2
 \mathbf{P}\left( \left| {Z_n } \right| \ge \sqrt {2\alpha \ln n}\right)}
  = {}\\
  {}=\mathbf{o}\left(n^{-\alpha/2}\right)\,.
  \label{e18-ch}
 \end{multline}

 Чтобы подчеркнуть согласованность изменений переменных~$s$ и~$z$, 
 будем в~дальнейшем использовать обозначение:
 \begin{equation*}
 z_{s,n}=\fr{s-na}{b\sqrt{n}}\,.
 \end{equation*}

 При  $n\to\infty$ в~условиях ${\bf (A_1)}$--${\bf(A_3)}$ по теореме~А.4.3 из~[5]
 верны соотношения:
 \begin{multline}
 \sum\limits_{s\in \mathbf{Z}} 
 \left|z_{s,n}\right|^j \fr{\varphi[z_{s,n}]}{b\sqrt{n}}=
  \int\limits_{-\infty}^{\infty} |z|^j \varphi[z]\, dz+
\mathbf{O}
 \left(n^{-1/2}\right)\,,\\ j\in\{0,1,\ldots\}\,,
 \label{e19-ch}
 \end{multline}
 а~по теореме~22.3 из~[5]:
 \begin{equation}
 \sum\limits_{s\in \mathbf{Z}} \left|
 \mathbf{P}(S_n=s)-\fr{p_3[z_{s,n},n]}{b\sqrt{n}}\right|
 =\mathbf{o}\left(n^{-3/2}\right)\,.
 \label{e20-ch}
 \end{equation}

 Из (\ref{e9-ch}), (\ref{e10-ch}) и~(\ref{e19-ch}) следует, что
 \begin{multline*}
 \sum\limits_{s\in \mathbf{Z}}
 \left|\fr{p_3[z_{s,n},n]}{b\sqrt{n}}-\fr{p_2[z_{s,n},n]}{b\sqrt{n}}\right|
 ={}\\
 {}=\fr{1}{n^{3/2}} \sum\limits_{s\in \mathbf{Z}} \fr{\varphi[z_{s,n}]}{b\sqrt{n}}
 \left\vert\fr{\rho_5 H_5[z_{s,n}]}{5!}+\fr{\rho_3 \rho_4 H_7[z_{s,n}]}{3!4!}+{}\right.\\
\left. {}+
 \fr{\rho_3^3 H_9[z_{s,n}]}{(3!)^4} \right\vert
 =\mathbf{O}\left(n^{-3/2}\right)\,,
 \end{multline*}
а потому с~учетом~(\ref{e20-ch}) имеем:
 \begin{equation}
 \sum\limits_{s\in \mathbf{Z}} \left|\mathbf{P}(S_n=s)-
 \fr{p_{n,2}[z_{s,n}]}{b\sqrt{n}}
 \right| =\mathbf{O}\left(n^{-3/2}\right)\,.
 \label{e21-ch}
 \end{equation}
 Отсюда при $j\hm=1,2$ получим
 \begin{multline} 
 \hspace*{-0.84pt}\sum\limits_{s\in \mathbf{Z}}
 I\left( \left| z_{s,n}\right| \le \sqrt {2\alpha \ln n} \right)
 \left| 
 \vphantom{\fr{p_{n,2}[z_{s,n}]}{b\sqrt{n}}}
 h_j[z_{s,n},n]
 \left(\mathbf{P}(S_n=s)-{}\right.\right.\\
\left.\left. {}-\fr{p_{n,2}[z_{s,n}]}{b\sqrt{n}}\right)\right|=
\mathbf{O}\left(\sqrt{\ln n}\right)\times{}\\
{}\times
 \sum\limits_{s\in \mathbf{Z}}
 I\left( \left| 
 \vphantom{\fr{p_{n,2}[z_{s,n}]}{b\sqrt{n}}}
 z_{s,n}\right| \le \sqrt {2\alpha \ln n} \right)
 \left|\mathbf{P}(S_n=s)-{}\right.\\
\left. {}-\fr{p_{n,2}[z_{s,n}]}{b\sqrt{n}}\right|
 =\mathbf{O}\left(\fr{\sqrt{\ln n}} {n^{3/2}}\right)\,.
  \label{e22-ch}
 \end{multline}

 В~[1] показано, что на промежутке $[-\sqrt {2\alpha \ln n};\,\sqrt {2\alpha \ln n}]$
 функции  $h_1[z,n]$ и~$h_2[z,n]$ при $n\hm\to\infty$, $\alpha\hm>0$
 имеют единственные корни, соответственно $z_{1}\hm=A/\sqrt{n}\hm+\mathbf{O}(n^{-3/2})$
 и~$z_{2}\hm=0.$
 При этом $h_j[z,n]\hm<0,$ если $-\sqrt {2\alpha \ln n}\hm<z\hm<z_{j},$ и~$h_j[z,n]\hm>0,$
 если $z_{j}\hm<z\hm<\sqrt {2\alpha \ln n}$, $j\hm=1,2$.

 Используя упомянутые выше свойства функции  $h_1[z,n]$, а~также~(\ref{e12-ch}), 
 (\ref{e16-ch}) и~(\ref{e17-ch}), при $\alpha\hm\ge 4$ получим:
 \begin{multline}
 \mathbf{E}\left|\widehat{G}[a;Z_n]-G[a]\right|={}\\
 {}=
 \mathbf{E}\left\{\left|\widehat{G}[a;Z_n]-G[a]\right|
 I\left( \left| Z_n\right| \le \sqrt {2\alpha \ln n} \right)\right\}+{}
\\
{}+ \mathbf{E}\left\{\left|\widehat{G}[a;Z_n]-G[a]\right|
 I\left( \left| Z_n\right| > \sqrt {2\alpha \ln n} \right)\right\}={}
 \\
{} =\fr{|G'[a]|}{\sqrt {n\Phi'_1[a]}}\,
 \mathbf{E}\left\{
 \left|
 \vphantom{\mathbf{O}\left(n^{-3/2+\delta_1}\right)}
 h_1[Z_n,n]+{}\right.\right.\\
\left.\left. {}+\mathbf{O}\left(n^{-3/2+\delta_1}\right)\right|
 I\left( \left| Z_n \right| \le \sqrt {2\alpha \ln n} \right) \right\}+{}\\
 {}+
 \mathbf{o}\left(n^{-\alpha/2}\right)={}
\\
 {}=\fr{|G'[a]|}{\sqrt {n\Phi'_1[a]}}
 \mathbf{E}\left\{\left|h_1[Z_n,n]\right|\,
 I\left( |Z_n|\le \sqrt {2\alpha \ln n}\right) \right\}+{}\\
 {}+
 \mathbf{O}\left(n^{-2+\delta_1}\right),
  \label{e23-ch}
 \end{multline}
 где $0<\delta_1\hm<0{,}5.$

 Аналогично можно показать, что при $\alpha\hm\ge 4$, $0\hm<\delta_2\hm<0{,}5$
 \begin{multline}
 \mathbf{E}\left|\widetilde{G}[a;Z_n]-G[a]\right|={}
\\
{}=\fr{|G'[a]|}{\sqrt {n\Phi'_1[a]}}\,
 \mathbf{E}\left\{\left|h_2[Z_n,n]\right|\,I\left( |Z_n|\le 
 \sqrt {2\alpha \ln n}\right) \right\}+{}\\
 {}+\mathbf{O}\left(n^{-2+\delta_2}\right)\,.
 \label{e24-ch}
 \end{multline}

 Далее всюду будем предполагать, что $\alpha\hm\ge 4.$

 Найдем асимптотические разложения главных членов~(\ref{e23-ch}) и~(\ref{e24-ch}),
 воспользовавшись теоремой~А.4.3 из~[5].
 В~соответствии с~этой теоремой
 для любой функции Шварца~$f$, определенной на~$\mathbf{R},$ и~каждого
 борелевского мно\-же\-ст\-ва~$\mathbf{A}$ верна оценка:
 \begin{equation}
 \left|
 \sum\limits_{s\in {\bf S}_\mathbf{A}}  \fr{f[z_{s,n}]}{b\sqrt{n}}
 -\int\limits_{\mathbf{A}} d\Lambda[z] \right| =\mathbf{O}\left(n^{-3/2}\right),
 \label{e25-ch}
 \end{equation}
 если положить
 \begin{align*}
 \mathbf{S}_\mathbf{A}&=\left\{s\in {\bf Z}: z_{s,n}\in \mathbf{A}\right\}\,;\\
 F[z]&=\int\limits_{-\infty}^z f[x]\, dx\,;
 \\
 \Lambda[z]&=\sum\limits_{i=0}^2 \left(-\fr{1}{b\sqrt{n}}\right)^i
 \ell_i[bz\sqrt{n}+an]F^{(i)}[z]\,.
 \end{align*}

 Поскольку функции $f_j[x]\hm=h_j[x,n] p_2[x,n]$, $j\hm=1,2,$ 
 являются функциями Шварца, то,  полагая
 \begin{gather}
A_{j,1}=[z_{j};\sqrt {2\alpha \ln n}]\,;\enskip
 A_{j,2}=[-\sqrt {2\alpha \ln n};z_{j}]\,;
 \notag\\
{\bf S}_{j,1}=\left\{s\in {\bf Z}:\, z_{s,n}\in A_{j,1}\right\}\,;\notag\\
 {\bf S}_{j,2}=\left\{s\in {\bf Z}:\, z_{s,n}\in A_{j,2}\right\}\,;
\notag\\
F_j[z]=\int\limits_{-\infty}^z f_j[x] \,dx
 =\int\limits_{-\infty}^z h_j[x,n] p_2[x,n]\, dx\,;
\label{e26-ch}\\
\Lambda_j[z]=\sum\limits_{i=0}^2 \left(-\fr{1}{b\sqrt{n}}\right)^i
 \ell_i[bz\sqrt{n}+an]F_j^{(i)}[z]
 \label{e27-ch}
 \end{gather}
 и~применяя~(\ref{e22-ch}) и~(\ref{e25-ch}), получим при $j\hm=1,2$
 \begin{multline*}
 \mathbf{E}\left|h_j[Z_n,n]I\left( |Z_n|\le \sqrt {2\alpha \ln n}\right)\right|={}
 \\
 {} =\sum\limits_{s\in \mathbf{Z}}
 \left|h_j[z_{s,n},n]\right| I\left( |z_{s,n}|\le \sqrt {2\alpha \ln n}\right)
\times{}\\
{}\times \left(\mathbf{P}\left(S_n=s\right)-\fr{p_2[z_{s,n},n]}{b\sqrt{n}}\right)+{}\\
\hspace*{-2.92pt}{}+ \sum\limits_{s\in \mathbf{Z}}
 \left|h_j[z_{s,n},n]\right| I\left( |z_{s,n}|\le \sqrt {2\alpha \ln n}\right)
 \fr{p_2[z_{s,n},n]}{b\sqrt{n}}={}
\end{multline*}

\noindent
\begin{multline}
 {}=\mathbf{O}\left(\fr{\sqrt{\ln n}} {n^{3/2}}\right)
 + {}\\
 {}+\sum\limits_{s\in \mathbf{Z}}
 \left|h_j[z_{s,n},n]\right| I\left( |z_{s,n}|\le \sqrt {2\alpha \ln n}\right)
 \fr{p_2[z_{s,n},n]}{b\sqrt{n}}={}
\\
{}= \sum\limits_{s\in {\bf S}_{j,1}} h_j[z_{s,n},n] \fr{p_2[z_{s,n},n]}{b\sqrt{n}}
 - {}\\
 {}-\sum\limits_{s\in {\bf S}_{j,2}} h_j[z_{s,n},n] \fr{p_2[z_{s,n},n]}{b\sqrt{n}}
 +\mathbf{O}\left(\fr{\sqrt{\ln n}} {n^{3/2}}\right)={}
\\
{}=\int\limits_{A_{j,1}} d\Lambda_j[z]- \int\limits_{A_{j,2}} d\Lambda_j[z]
 +\mathbf{O}\left(\fr{\sqrt{\ln n}} {n^{3/2}}\right)\,.
\label{e28-ch}
 \end{multline}
 Чтобы оценить значения интегралов
 $\int\limits_{A_{j,1}} \,d\Lambda_j[z]$ 
 и~$\int\limits_{A_{j,2}} \,d\Lambda_j[z]$,
 представим подынтегральные функции в~(\ref{e26-ch}) следующим образом:
 
 \noindent
 \begin{multline*}
h_1[x,n] p_2[x,n]=\varphi[x]
 \left\{x+\fr{\rho_3 x H_3[x]}{6\sqrt{n}}+\right.{}\\
{} +\fr{\rho_4 x H_4[x]}{24n}
 + \fr{\rho_3^2 x H_6[x]}{72n}+
 \fr{A H_2[x]}{\sqrt{n}}+{}\\
\left. {}+
\fr{A\rho_3 H_2[x]H_3[x]}{6n}+\fr{B_1 H_3[x]+B_2
 x}{n}+P_1[x,n]
 \right\}={}
\\
{}=\varphi[x]
 \left\{x+\fr{\rho_3 (H_4[x]+3H_2[x])/6+AH_2[x]}{\sqrt{n}}
 +{} \right.\\
 {}+\fr{\rho_4 (H_5[x]+4H_3[x])}{24n}+\fr{\rho_3^2 (H_7[x]+6H_5[x])}{72n}
 +{}\\
 {}+\fr{A \rho_3 (H_5[x]+6H_3[x]+6x)}{6n}
 +\fr{B_1 H_3[x]+B_2 x}{n}+{}\\
\left. {}+P_1[x,n]
\vphantom{\fr{A \rho_3 (H_5[x]+6H_3[x]+6x)}{6n}}
\right\}\,;
 %\label{e29-ch}
 \end{multline*}
 
 \vspace*{-12pt}
 
 \noindent
 \begin{multline*}
h_2[x,n] p_2[x,n]=\varphi[x]
 \left\{x+\fr{\rho_3 x H_3[x]}{6\sqrt{n}}+{}\right.
\\ 
{} +\fr{\rho_4 x H_4[x]}{24n}
 +\fr{\rho_3^2 x H_6[x]}{72n}+\fr{A x^2}{\sqrt{n}}+
\fr{A\rho_3 x^2 H_3[x]}{6n}+{}\\
\left.{}+\fr{B x^3}{n}+P_2[x,n]
 \right\}={}
G'[a]\varphi[x]
 \left\{
 \vphantom{\fr{\rho_3 (H_4[x]+3H_2[x])/6+A(H_2[x]+1)}{\sqrt{n}}}
 x+{}\right.\\
 {}+\fr{\rho_3 (H_4[x]+3H_2[x])/6+A(H_2[x]+1)}{\sqrt{n}}+{}\\
{} +\fr{\rho_4 (H_5[x]+4H_3[x])}{24n}+
\fr{\rho_3^2 (H_7[x]+6H_5[x])}{72n}+{}\\
{}+
\fr{A \rho_3 (H_5[x]+7H_3[x]+6x)}{6n}
 +\fr{B (H_3[x]+3x)}{n}+{}\\
\left. {}+P_2[x,n]
\vphantom{\fr{\rho_3 (H_4[x]+3H_2[x])/6+A(H_2[x]+1)}{\sqrt{n}}}
\right\},
 %\label{e30-ch}
 \end{multline*}
 где
 \begin{align*}
 P_1[x,n]&=\fr{AH_2[x]}{n^{3/2}}
 \left( \fr{\rho_4 H_4[x]}{24}+\fr{\rho_3^2 H_6[x]}{72}\right)+{}
\\[3pt]
 &\hspace*{-12.1mm}{}+\fr{B_1 H_3[x]+B_2 x}{n^{3/2}}
 \left( \!\fr{\rho_3 H_3[x]}{6}+\fr{\rho_4 H_4[x]}{24}
 +\fr{\rho_3^2 H_6[x]}{72}\!\right)\,;
\\[3pt]
 P_2[x,n]&=\fr{A x^2}{n^{3/2}}
 \left( \fr{\rho_4 H_4[x]}{24}+\fr{\rho_3^2 H_6[x]}{72}\right)+{}
\\[3pt]
 &{}+\fr{B x^3}{n^{3/2}}
 \left( \fr{\rho_3 H_3[x]}{6}+\fr{\rho_4 H_4[x]}{24}
 +\fr{\rho_3^2 H_6[x]}{72}\right)\,.
 \end{align*}

 Учитывая, что
 \begin{align*}
\varphi[\sqrt{2\alpha\ln n}]&=\fr{1}{\sqrt{2\pi}}
 \exp\left\{-\fr{2\alpha\ln n}{2}\right\}=\mathbf{O}\left(n^{-\alpha}\right)\,;
 \\[3pt]
\varphi^{(r)}[x]&=(-1)^r H_r[x] \varphi[x] \mbox{ при } r=0,1,2,\ldots,
 \end{align*}
 получим сначала асимптотические разложения функций $F_j[z]$, $j\hm=1,2$, и~их 
 производных, справедливые при значениях аргумента $|z|\hm\le \sqrt{2\alpha\ln n}:$
 \begin{multline}
F_1[z] =-\varphi[z] \left\{1+\fr{\rho_3 (H_3[z]+3z)/6+Az}{\sqrt{n}}
 +{}\right.\\[3pt]
{}+\fr{\rho_4 (H_4[z]+4H_2[z])}{24n}+
\fr{\rho_3^2 (H_6[z]+6H_4[z])}{72n}+{}\\[3pt]
{} +\fr{A \rho_3 (H_4[z]+6H_2[z]+6)}{6n}+{}\\[3pt]
\left. {} +\fr{B_1 H_2[z]+B_2}{n}
 \right\}+\mathbf{O}\left(n^{-3/2}\right)\,;
 \label{e31-ch}
 \end{multline}
 
 \vspace*{-12pt}
 
 \noindent
 \begin{multline}
F_1'[z] ={}\\[3pt]
{}=\varphi[z]
 \left\{z+\fr{\rho_3 (H_4[z]+3H_2[z])/6+AH_2[z]}{\sqrt{n}}\right\}+{}\\[3pt]
{} +\mathbf{O}\left(n^{-1}\right)\,;
 \label{e32-ch}
 \end{multline}

%\vspace*{-3pt}

\noindent
\begin{align}
F_1''[z] &=-\varphi[z]H_2[z]+\mathbf{O}\left(n^{-1/2}\right)\,;  \label{e33-ch}\\[3pt]
 F_1^{(3)}[z] &=\varphi[z]H_3[z]+\mathbf{O}\left(n^{-1/2}\right)\,;
 \label{e33a-ch}
 \end{align}
 
 \vspace*{-6pt}
 
 \noindent
 \begin{multline*}
 \hspace*{-3.4pt}F_2[z]=\fr{A\Phi[z]}{\sqrt{n}}
 -\varphi[z] \!\left\{\!1+\fr{\rho_3 (H_3[z]+3z)/6+Az}{\sqrt{n}} +{}
 \right.
\\[3pt] 
  {}+\fr{\rho_4 (H_4[z]+4H_2[z])}{24n}+\fr{\rho_3^2 (H_6[z]+6H_4[z])}{72n}+{}
\\[3pt]
\left.
 +\fr{A \rho_3 (H_4[z]+7H_2[z]+6)}{6n}
 +\fr{B (H_2[z]+3)}{n}\right\}+{}\\
 {}+\mathbf{O}\left(n^{-3/2}\right),
% \label{e34-ch}
 \end{multline*}
 
% \vspace*{-12pt}
 
 \noindent
 \begin{multline*}
F_2'[z] =\varphi[z]
 \left\{
 \vphantom{\fr{\rho_3 (H_4[z]+3H_2[z])/6+A (H_2[z]+1)}{\sqrt{n}}}
 z+{}\right.\\
\left. {}+\fr{\rho_3 (H_4[z]+3H_2[z])/6+A (H_2[z]+1)}{\sqrt{n}}\right\}
 +{}\\
 {}+\mathbf{O}\left(n^{-1}\right)\,;
% \label{e35-ch}
 \end{multline*}
 
\vspace*{-9pt}
 
 \noindent
 \begin{align*}
F_2''[z] &=-\varphi[z]H_2[z]+\mathbf{O}\left(n^{-1/2}\right)\,;\\ % \label{e36-ch}\\
 F_2^{(3)}[z] &=\varphi[z]H_3[z]+\mathbf{O}\left(n^{-1/2}\right)\,,
% \label{e36a-ch}
 \end{align*}
 а~затем асимптотические разложения этих же функций в~точках
 $z\hm=\pm\sqrt{2\alpha\ln n}$ и~$z\hm=0:$
 
 \noindent
 \begin{multline}
 F_1[\pm\sqrt{2\alpha\ln n}] =\mathbf{O}\left(n^{-\alpha}\right)
  +\mathbf{O}\left(n^{-3/2}\right)={}\\
  {}=\mathbf{O}\left(n^{-3/2}\right)\,;
 \label{e37-ch}
 \end{multline}
 
 \vspace*{-12pt}
 
 \noindent
 \begin{multline}
F_2[\sqrt{2\alpha\ln n}] =\fr{A\Phi[\sqrt{2\alpha\ln n}]}{\sqrt{n}}
  + \mathbf{O}\left(n^{-\alpha}\right) +{}\\
  {}+\mathbf{O}\left(n^{-3/2}\right)=
  \fr{A}{\sqrt{n}}-\fr{A\varphi[\sqrt{2\alpha\ln n}]}{\sqrt{2n\alpha\ln n}}
  + \mathbf{O}\left(n^{-3/2}\right)={}\\
  {}=\fr{A}{\sqrt{n}}
  + \mathbf{O}\left(n^{-3/2}\right)\,;
\label{e38-ch}
 \end{multline}

  \vspace*{-12pt}
 
 \noindent
 \begin{multline}
F_2[-\sqrt{2\alpha\ln n}] =\fr{A\Phi[-\sqrt{2\alpha\ln n}]}{\sqrt{n}}
   +{}\\
   {}+ \mathbf{O}\left(n^{-\alpha}\right) +\mathbf{O}\left(n^{-3/2}\right)=
=\fr{A(1-\Phi[\sqrt{2\alpha\ln n}])}{\sqrt{n}}+{}\\
{}  + \mathbf{O}\left(n^{-\alpha}\right) +\mathbf{O}\left(n^{-3/2}\right)
  = \mathbf{O}\left(n^{-3/2}\right)\,;
 \label{e39-ch}
 \end{multline}
 
  \vspace*{-12pt}
 
 \noindent
 \begin{multline}
F_j'[\pm\sqrt{2\alpha\ln n}]=\mathbf{O}\left(n^{-\alpha}\right)
  +\mathbf{O}\left(n^{-1}\right)=\mathbf{O}\left(n^{-1}\right)\,, \\ j=1,2\,;
 \label{e40-ch}
 \end{multline}
 
  \vspace*{-12pt}
 
 \noindent
 \begin{multline}
F_j''[\pm\sqrt{2\alpha\ln n}] =\mathbf{O}\left(n^{-\alpha}\right)
  +\mathbf{O}\left(n^{-1/2}\right)={}\\
  {}=\mathbf{O}\left(n^{-1/2}\right)\,,\enskip j=1,2\,.
 \label{e41-ch}
 \end{multline}

 Используя равенства~(\ref{e31-ch})--(\ref{e33a-ch}), 
 оценим значения функции $F_1[z]$ и~ее производных в~точке $z\hm=0$ 
 и~точке $z\hm=\zeta$, находящейся в~интервале с~концами~$0$ и~$z_{1}$:
 

 
 \noindent
 \begin{multline}
F_1[0]={}\\
{}=
  -\varphi[0]\left\{
  1+\fr{B_2-B_1+A\rho_3/2}{n}
  +\fr{\rho_3^2- \rho_4 }{24n}
  \right\}+{}\\
  {}+\mathbf{O}\left(n^{-3/2}\right)\,;
 \label{e42-ch}
\end{multline}

\vspace*{-12pt}

\noindent
\begin{equation}
F_1'[0]= -\varphi[0]\,\fr{A}{\sqrt{n}}+\mathbf{O}\left(n^{-1}\right)\,;\label{e43-ch}
\end{equation}
\begin{align}
F_1^{(2)}[0]&=\varphi[0]+\mathbf{O}\left(\fr{1}{\sqrt{n}}\right)\,; \label{e44-ch}\\
  F_1^{(3)}[\zeta]&=\mathbf{O}\left(n^{-1/2}\right)\,.
 \label{e44a-ch}
 \end{align}

 Используя равенства~(\ref{e42-ch})--(\ref{e44a-ch}), 
 найдем разложения функции $F_1[z]$ и~ее производных
 в~точке $z\hm=z_{1}:$
  \begin{multline}
 F_1^{(2)}[z_{1}]= F_1^{(2)}[0]+F_1^{(3)}[\zeta_1] z_{1}
 =\varphi[0]+{}\\
 {}+\mathbf{O}\left(\fr{1}{\sqrt{n}}\right)\,;
 \label{e45-ch}
 \end{multline}
 
 \vspace*{-12pt}
 
 \noindent
 \begin{multline}
 F_1'[z_{1}]= F_1'[0]+F_1^{(2)}[0] z_{1}+F_1^{(3)}[\zeta_2] \fr{z^2_{1}}{2}={}
\\
  {}=-\varphi[0]\,\fr{A}{\sqrt{n}}+\mathbf{O}\left(n^{-1}\right)
  +\left(\varphi[0]+\mathbf{O}\left(n^{-1/2}\right)\right)\times{}\\
  {}\times
  \left(\fr{A}{n}+\mathbf{O}\left(n^{-3/2}\right)\right)=
  \mathbf{O}\left(n^{-1}\right)\,;
 \label{e46-ch}
 \end{multline}
 
 \vspace*{-12pt}
 
 \noindent
 \begin{multline}
 F_1[z_{1}]=F_1[0]+ F'_1[0] z_{1}+F_1^{(2)}[0] \fr{z_{1}^2}{2}
 +F_1^{(3)}[\zeta_3] \fr{z_{1}^3}{6}={}
\\
 {}=-\varphi[0]\left\{ 1+\fr{B_2-B_1}{n}+\fr{A\rho_3}{2n}
 +\fr{\rho_3^2-\rho_4  }{24n} \right\}
 +{}\\
 {}+\mathbf{O}\left(n^{-3/2}\right)+{}
\\
{} +\left(-\varphi[0]\fr{A}{\sqrt{n}}+\mathbf{O}\left(n^{-1}\right)\right)
 \left(\fr{A}{\sqrt{n}}+\mathbf{O}\left(n^{-3/2}\right)\right)+{}
 \\
{} +\left(\varphi[0]+\mathbf{O}\left(n^{-1/2}\right)\right)
 \fr{1}{2}\left(\fr{A}{\sqrt{n}}+\mathbf{O}\left(n^{-3/2}\right)\right)^2
 +{}\\
 {}+\mathbf{O}\left(n^{-2}\right)={}
\\
 {}=-\varphi[0]\left\{ 1+\fr{B_2-B_1}{n}+\fr{A\rho_3+A^2}{2n}
 +\fr{\rho_3^2-\rho_4  }{24n} \right\}+{}\\
{} +\mathbf{O}\left(n^{-3/2}\right)\,,
 \label{e47-ch}
 \end{multline}
 где $\zeta_1$, $\zeta_2$ и~$\zeta_3$~--- значения из интервала с~концами~$0$ и~$z_1$.

 Теперь оценим значения функции $F_2[z]$ и~ее производных в~точке $z=0$:
 \begin{align}
 F_2[0]&=
 \fr{A}{2\sqrt{n}} -&{}\notag\\
 &\hspace*{-7mm}{}-\varphi[0]\left\{ 1+\fr{1}{n}
 \left(2B+\fr{A\rho_3 }{3} +\fr{\rho_3^2}{24}
 -\fr{\rho_4}{24}\right)\right\}+{}\notag\\
& \hspace*{35mm}{}+\mathbf{O}\left(n^{-3/2}\right)\,;
 \label{e48-ch}\\
 F'_2[0]&=\mathbf{O}\left(n^{-1}\right)\,;
 \label{e49-ch}\\
 F''_2[0]&= \varphi[0]+\mathbf{O}\left(\fr{1}{\sqrt{n}}\right)\,.
  \label{e50-ch}
 \end{align}

 Используя оценки~(\ref{e37-ch}), (\ref{e40-ch}), (\ref{e41-ch}), 
 (\ref{e45-ch})--(\ref{e47-ch}), а~также~(\ref{e23-ch}), (\ref{e26-ch})--(\ref{e28-ch}),
 найдем асимптотическое разложение средней абсолютной ошибки НОРМД:
 
 \noindent
 \begin{multline*}
 \mathbf{E}\left|\widehat{G}[a;Z_n]-G[a]\right|=\fr{|G'[a]|}
 {\sqrt {n\Phi'_1[a]}}\times{}\\
 {}\times \sum\limits_{i=0}^2\!  
 \left(\!-\fr{1}{b\sqrt{n}}\right)^i\!\! 
 \ell_i\left[na+b\sqrt{2n\alpha\ln n}\right]\! F_1^{(i)}\!\left[\sqrt{2\alpha\ln n}\right]-{}\hspace*{-6.48pt}
 \end{multline*}
 
 \noindent
 \begin{multline}
 \hspace*{-4.159pt}{} -\fr{2|G'[a]|}{\sqrt {n\Phi'_1[a]}}\sum\limits_{i=0}^2 
\left(-\fr{1}{b\sqrt{n}}\right)^i
 \ell_i\left[na+z_{1} b\sqrt{n}\right] F_1^{(i)}[z_{1}]+{}
\\
 {}+\fr{\left\vert G'[a]\right\vert }{\sqrt {n\Phi'_1[a]}}\sum\limits_{i=0}^2  
 \left(-\fr{1}{b\sqrt{n}}\right)^i
 \ell_i\left[
 \vphantom{\sqrt{2n\alpha\ln n}}
 na-{}\right.\\
\left. {}-b\sqrt{2n\alpha\ln n}\right] F_1^{(i)}\left[-\sqrt{2\alpha\ln n}\right]
 +\mathbf{O}\left(n^{-2+\delta_1}\right)={}
\\
 {}=\fr{2\varphi[0]|G'[a]|}{\sqrt{n\Phi'_1[a]}}
 \left\{1+\fr{A^2+A\rho_3}{2n} +\fr{B_2-B_1}{n}+{}\right.\\
\left. {}+\fr{\rho_3^2-\rho_4}{24n}
 -\fr{\ell_2[Ab+na]}{b^2 n}
 \right\}+\mathbf{O}\left(n^{-2+\delta_1}\right)\,.
  \label{e51-ch}
 \end{multline}

 Проделав аналогичные выкладки, но с~использованием формул~(\ref{e38-ch})--(\ref{e41-ch}), 
 (\ref{e48-ch})--(\ref{e50-ch}), а~также~(\ref{e24-ch}), (\ref{e26-ch})--(\ref{e28-ch}),
 найдем асимптотическое разложение средней абсолютной ошибки ОМП:
 
\noindent
 \begin{multline}
 \mathbf{E}\left|\widetilde{G}[a;Z_n]-G[a]\right|={}
\\
 {}=\fr{|G'[a]|}{\sqrt {n\Phi'_1[a]}}\sum\limits_{i=0}^2  \left(-\fr{1}{b\sqrt{n}}\right)^i
 \ell_i\left[
 \vphantom{b\sqrt{2n\alpha\ln n}}
 na+{}\right.\\
\left. {}+b\sqrt{2n\alpha\ln n}\right] F_2^{(i)}\left[\sqrt{2\alpha\ln n}\right]-{}
\\
{} -\fr{2|G'[a]|}{\sqrt {n\Phi'_1[a]}}\sum\limits_{i=0}^2
 \left(-\fr{1}{b\sqrt{n}}\right)^i \ell_i[na] F_2^{(i)}[0]+{}
 \\
{} +\fr{|G'[a]|}{\sqrt {n\Phi'_1[a]}}\sum\limits_{i=0}^2  
\left(-\fr{1}{b\sqrt{n}}\right)^i
 \ell_i\left[
 \vphantom{b\sqrt{2n\alpha\ln n}}
 na-{}\right.\\
\left. {}-b\sqrt{2n\alpha\ln n}\right] F_2^{(i)}[-\sqrt{2\alpha\ln n}]
 +\mathbf{O}\left(n^{-2+\delta_2}\right)={}
\\
{} =\fr{2\varphi[0]|G'[a]|}{\sqrt{n\Phi'_1[a]}}
  \left\{1+\fr{2B}{n}+\fr{A\rho_3}{3n} +\fr{\rho_3^2-\rho_4}{24n}
  -{}\right.\\
\left.  {}-\fr{\ell_2[na]}{b^2 n} \right\}+\mathbf{O}\left(n^{-2+\delta_2}\right)\,.
 \label{e52-ch}
 \end{multline}


 Подставляя в~(\ref{e51-ch}) и~(\ref{e52-ch}) выражения коэффициентов асимметрии и~эксцесса (11),
 завершим вывод асимптотических разложений~(\ref{e3-ch}) и~(\ref{e4-ch}).

 Рассмотрим теперь случай $G'[a]\hm=0,\,G''[a]\hm\neq 0$. В этом случае можно
 ограничиться более компактным вариантом разложений~(\ref{e7-ch}) и~(\ref{e8-ch}):
   \begin{align}
 \widehat{G}[a;z]-G[a]&=\fr{ G''[a]\, H_2[z]}{2n\Phi'_1[a]}
  + \mathbf{O}\left( {n^{- 3/2+\delta_3} } \right)\,;
\label{e53-ch}
 \\ 
 \widetilde{G}[a;z]-G[a]&=  \fr{G''[a]\, z^2 }{2n\Phi'_1[a] }
  + \mathbf{O}\left( {n^{- 3/2+\delta_4} } \right)\,,
 \label{e54-ch}
 \end{align}
 где $0<\delta_3<0{,}5$, $0\hm<\delta_4\hm<0{,}5$, $|z|\hm\le \sqrt{2\alpha\ln n}.$

 Из~(\ref{e53-ch}), используя~(\ref{e17-ch}), (\ref{e21-ch}) 
 и~теорему~А.4.3 из~[5] при $n\hm\to\infty,$  последовательно получим:
 \begin{multline*}
 \mathbf{E}\left|\widehat{G}[a;Z_n]-G[a]\right|={}\\
 {}=\fr{|G''[a]|}{2n\Phi'_1 [a]}
  \mathbf{E}\left|H_2[Z_n]
   I\left(|Z_n|\le \sqrt{2\alpha\ln n} \right)\right|
    +{}\\
    {}+ \mathbf{O}\left( {n^{- 3/2+\delta_3} } \right)=
    \fr{|G''[a]|}{2n\Phi'_1 [a]} \times{}\\
    {}\times
    \sum\limits_{s\in \mathbf{Z}}
   \left|H_2[z_{s,n}] \right|  I\left(|z_{s,n}|\le \sqrt{2\alpha\ln n} \right)
   \fr{\varphi[z_{s,n}]}{b\sqrt{n}}
    + {}\\
    {}+\mathbf{O}\left( {n^{- 3/2+\delta_3} } \right)={}
    \\ 
{}   =\fr{|G''[a]|}{2n\Phi'_1 [a]}
   \int\limits_{-\sqrt{2\alpha\ln n} }^{\sqrt{2\alpha\ln n}}\hspace*{-3mm}
   \left|H_2[z] \right| \varphi[z]\, dz
   + \mathbf{O}\left( {n^{- 3/2+\delta_3} } \right)
   ={}\\
   {}=\fr{|G''[a]|}{n\Phi'_1 [a]}\sqrt{\fr{2}{\pi e}}
    + \mathbf{O}\left( {n^{- 3/2+\delta_3} } \right)\,,
  \end{multline*}
   что завершает доказательство~(\ref{e5-ch}).
   
%   \smallskip

   Справедливость~(\ref{e6-ch}) установим, используя~(\ref{e54-ch}), 
   (\ref{e18-ch}), (\ref{e21-ch}) и~теорему~А.4.3 из~[5]
   при $n\hm\to\infty$:
   
   \noindent
  \begin{multline*}
  \mathbf{E}\left|\widetilde{G}[a;Z_n]-G[a]\right|={}\\
  {}=\fr{|G''[a]|}{2n\Phi'_1 [a]}
   \mathbf{E}\left\{Z_n^2 I\left(|Z_n|\le \sqrt{2\alpha\ln n} \right)\right\}
   + {}\\
   {}+\mathbf{O}\left( {n^{- 3/2+\delta_4} } \right)=
   \fr{|G''[a]|}{2n\Phi'_1 [a]}\times{}\\
   {}\times
    \sum\limits_{s\in \mathbf{Z}}
   z_{s,n}^2 I\left(|z_{s,n}|\le \sqrt{2\alpha\ln n} \right)
   \fr{\varphi[z_{s,n}]}{b\sqrt{n}}
    + {}\\
    {}+\mathbf{O}\left( {n^{- 3/2+\delta_4} } \right)={}
  \\
{}=\fr{|G''[a]|}{2n\Phi'_1 [a]}
   \int\limits_{-\sqrt{2\alpha\ln n}}^{\sqrt{2\alpha\ln n}}\hspace*{-1mm}
   z^2 \varphi[z]\, dz
   + \mathbf{O}\left( {n^{- 3/2+\delta_4} } \right)
  ={}\\
  {}=\fr{|G''[a]|}{2n\Phi'_1 [a]}
    +\mathbf{O}\left( {n^{- 3/2+\delta_4} } \right)\,.
  \end{multline*}

   Теорема доказана.
   
   \vspace*{-9pt}

\section{Экспериментальная оценка точности асимптотических разложений средней 
абсолютной ошибки}

 С целью выяснения возможности выбора лучшей из оценок с~помощью 
 асимптотических разложений~(\ref{e3-ch}) и~(\ref{e4-ch}) 
 была проведена серия вычислений в~предположении, что наблюдаемая случайная 
 величина~$\xi$ имеет пуассоновское распределение, определяемое выражением:
  \begin{equation*}
 \mathbf{P}(\xi=x)=\fr{a^x}{x!}\,e^{-x},\enskip x=0,1,2,\dots\,,
 \end{equation*}
 с~минимальной достаточной статистикой $S_n\hm=\sum_{i=1}^n X_i,$ также 
 имеющей пуассоновское распределение:
 \begin{equation*}
 \mathbf{P}\left(S_n=s\right)=\fr{(n a)^s}{s!}\,e^{-n a}\,,\enskip s=0,1,2,\dots
 \end{equation*}
 
 

 Для двух параметрических функций $G_1[a]\hm=a^4$ и~$G_2[a]\hm=e^{-a}$
 приближенное значение выражений
 \begin{multline*}
\Delta_j[a]=\mathbf{E} \left|\widehat{G_j}[a|S_n]-G_j[a]\right|-{}\\
{}-
 \mathbf{E} \left|\widetilde{G_j}[a|S_n]-G_j[a]\right|\,,\enskip j=1,2\,,
 \end{multline*}
 вычислялось с~помощью разложений~(\ref{e3-ch}) и~(\ref{e4-ch}) по формулам:
 \begin{align*}
 \Delta_1^*[a]&=\fr{4a^{5/2}}{n^{3/2} \sqrt{2\pi}}\!
 \left(\!-\fr{25}{4} +\left\lfloor na+\fr{3}{2} \right\rfloor-
 \left\lfloor na+\fr{3}{2} \right\rfloor^2\!
 -{}\right.\\
&\hspace*{35mm}\left. {}-\left\lfloor na \right\rfloor+
 \left\lfloor na \right\rfloor^2 \right)\,;
 \\
 \Delta_2^*[a]&=\fr{e^{-a}}{n^{3/2} \sqrt{2 \pi a}}
 \left( \fr{10a-9a^2}{12}+
 \left \lfloor na-\fr{a}{2} \right\rfloor -{}\right.\\
 &\left.\hspace*{18mm}{}-
 \left\lfloor na-\fr{a}{2} \right\rfloor^2-
 \left\lfloor na\right\rfloor
  +\left\lfloor na\right\rfloor^2
 \right)\,,
 \end{align*}
 а~точное значение~--- по формулам:
 \begin{multline*}
 \Delta_1[a]  ={}\\
 {}=\sum\limits_{s=0}^\infty \left|\fr{s(s-1)(s-2)(s-3)}{n^4}-a^4 \right|
 \mathbf{P}\left(S_n=s\right)-{}\\
 {} -\sum\limits_{s=0}^\infty \left|\left(\fr{s}{n}\right)^4-a^4 \right|
 \mathbf{P}\left(S_n=s\right)
 ={}\\
 {}=2a^4\hspace*{-1mm}\sum\limits_{s=\lceil s_1 \rceil-3}^{\lceil s_1 \rceil} \hspace*{-2mm}
 \mathbf{P}\left(S_n=s\right)-
 2a^4\hspace*{-1mm}\sum\limits_{s=\lceil na \rceil-3}^{\lceil na \rceil}\hspace*{-2mm}
 \mathbf{P}\left(S_n=s\right)
  +{}
 \end{multline*}
 
 \end{multicols}
   \begin{table*}\small
 \begin{center}
\parbox{296pt}{\Caption{Значения (в  \%) относительной погрешности вычисления~(\ref{e55-ch}) 
в~зависимости от значений параметров $n,\,a$}
}

\vspace*{2ex}

\tabcolsep=12pt
\begin{tabular}{|c|c|c|c|c|c|c|}
 \hline
\multicolumn{1}{|c|}{\raisebox{-6pt}[0pt][0pt]{$a$}}&\multicolumn{6}{c|}{$n$}\\
\cline{2-7}
& 10 & 20 & 50 & 100 & 200 & 500\\
 \hline
 0{,}12 & 30{,}4 & 9{,}50 & 4{,}87 & 2{,}44 & 1{,}221 & 0{,}488 \\
% \hline
 0{,}24 & 9{,}50 & 5{,}08 & 2{,}44 & 1{,}22 & 0{,}610 & 0{,}244 \\
% \hline
 0{,}49 & 5{,}47 & 2{,}42 & 0{,}47 & 0{,}60 & 0{,}299 & 0{,}119 \\
 % \hline
 0{,}79 & 3{,}37 & 1{,}48 & 0{,}28 & 0{,}37 & 0{,}185 & 0{,}074 \\
% \hline
 1{,}00 & 2{,}93 & 1{,}46 & 0{,}59 & 0{,}29 & 0{,}146 & 0{,}059 \\
%  \hline
 2{,}29 & 1{,}15 & 0{,}50 & 0{,}09 & 0{,}13 & 0{,}064 & 0{,}026 \\
%  \hline
 4{,}93 & 0{,}63 & 0{,}15 & 0{,}04 & 0{,}06 & 0{,}030 & 0{,}012 \\
 % \hline
 10{,}0 & 0{,}29 & 0{,}15 & 0{,}06 & 0{,}03 & 0{,}015 & 0{,}006 \\
  %\hline
 20{,}0 & 0{,}15 & 0{,}07 & 0{,}03 & 0{,}01 & 0{,}007 & 0{,}003\\
 \hline
\end{tabular}
\end{center}
%\end{table*}
%\begin{table*}\small
\vspace*{18pt}
 \begin{center}
\parbox{296pt}{\Caption{Значения (в \%) относительной погрешности вычисления~(\ref{e56-ch}) 
в~зависимости от значений параметров $n,\,a$}
}

\vspace*{2ex}

\tabcolsep=12pt
\begin{tabular}{|c|c|c|c|c|c|c|}
\hline
\multicolumn{1}{|c|}{\raisebox{-6pt}[0pt][0pt]{$a$}}&\multicolumn{6}{c|}{$n$}\\
\cline{2-7}
& 10 & 20 & 50 & 100 & 200 & 500\\
 \hline
 0{,}12 & 3{,}87 & 2{,}73 & 0{,}089 & 0{,}04 & 0{,}02 & 0{,}009 \\
% \hline
 0{,}24 & 4{,}72 & 1{,}97 & 0{,}51 & 0{,}25 & 0{,}13 & 0{,}050 \\
% \hline
 0{,}49 & 5{,}21 & 2{,}75 & 1{,}31 & 0{,}49 & 0{,}24 & 0{,}098 \\
% \hline
 0{,}79 & 8{,}01 & 4{,}60 & 8{,}07 & 0{,}77 & 0{,}38 & 0{,}154 \\
% \hline
 1{,}00 & 10{,}8 & 5{,}62 & 2{,}30 & 1{,}16 & 0{,}58 & 0{,}234 \\
% \hline
 2{,}29 & 2{,}46 & 1{,}06 & 0{,}29 & 0{,}23 & 0{,}12 & 0{,}046 \\
% \hline
 4{,}93 & 5{,}19 & 2{,}86 & 1{,}16 & 0{,}57 & 0{,}29 & 0{,}115 \\
% \hline
 10{,}0 & 14{,}1 & 7{,}29 & 2{,}97 & 1{,}49 & 0{,}75 & 0{,}300 \\
% \hline
 20{,}0 & 29{,}6 & 15{,}6 & 6{,}43 & 3{,}24 & 1{,}63 & 0{,}653\\
  \hline
\end{tabular}
\end{center}
\vspace*{18pt}
\end{table*}

\begin{multicols}{2}
 
\noindent  
\begin{multline}
  {}+\fr{6a^3}{n}\left(2\sum\limits_{s=0}^{\lceil na \rceil-3} 
  \mathbf{P}\left(S_n=s\right)-1 \right)+{}
 \\ 
 {}+\fr{7a^2}{n^2}\left(2\sum\limits_{s=0}^{\lceil na \rceil-2} 
 \mathbf{P}\left(S_n=s\right)-1 \right)
 +{}\\
 {}+\fr{a}{n^3}\left(2\sum\limits_{s=0}^{\lceil na \rceil-1} 
 \mathbf{P}\left(S_n=s\right)-1 \right)\,,
 \label{e55-ch}
 \end{multline}
 
 
 \noindent
 где $\lceil x \rceil$~--- целая часть числа~$x,$ а~$s_1$~--- 
 вещественное решение уравнения 
 \begin{equation*}
 s(s-1)(s-2)(s-3)=(na)^4
 \end{equation*}
 при $s>3$  и
 \begin{multline*}
 \Delta_2[a]=
  \sum\limits_{s=0}^\infty \left|\left(\fr{n-1}{n}\right)^s-e^{-a} \right|
 \mathbf{P}\left(S_n=s\right)
  -{}\\
  {}-\sum\limits_{s=0}^\infty \left|e^{-s/n}-e^{-a} \right|
 \mathbf{P}\left(S_n=s\right)={}
 \end{multline*}
 
\noindent
  \begin{multline}
{} =2\sum\limits_{s=0}^{\lceil s_2 \rceil}
 \left\{\left(\fr{n-1}{n}\right)^s-e^{-a} \right\}\mathbf{P}\left(S_n=s\right)
 -{}\\
 {}- 2\sum\limits_{s=0}^{\lceil n a \rceil}
 \left\{e^{-s/n}-e^{-a} \right\}\mathbf{P}\left(S_n=s\right)+{}
 \\ 
{} +\exp\left\{-n a \left(1-e^{-1/n}\right) \right\}-e^{-a}\,,
 \label{e56-ch}
 \end{multline}
где $s_2=-{a}/({\ln[1-1/n]})$.


 Вычисления проводились при следующих значениях параметров:
  $n=10$, 20, 50, 100, 200 и~500;  $a\hm=0{,}12$, 0{,}24, 0{,}49, 0{,}79,
 1, 2{,}29, 4{,}93, 10 и~20. Результаты вычислений приведены в~табл.~1 и~2.
 {\looseness=1
 
 }


%\columnbreak

 Из табл. 1 и~2 видно, что относительная погрешность
 вычисления выражений $\Delta_j[a]$, $j\hm=1,2,$ с~по\-мощью асимптотических 
 разложений~(\ref{e3-ch}) и~(\ref{e4-ch}) 
 при всех значениях входных параметров не превосходит 31\%,
 причем с~ростом объема выборки убывает со скоростью~$1/n$.

 При увеличении значения оцениваемой параметрической функции также наблюдается
 тенденция к~уменьшению относительной погрешности.

 Таким образом, применение асимптотических разложений~(\ref{e3-ch}) и~(\ref{e4-ch}) 
 позволяет получить  ответ на вопрос о том, какая из двух оценок имеет меньшее значение 
 средней  абсолютной ошибки.
 
% \vspace*{6pt}


{\small\frenchspacing
 {%\baselineskip=10.8pt
 \addcontentsline{toc}{section}{References}
 \begin{thebibliography}{9}

\bibitem{1-ch}
\Au{Федосеева Н.\,П., Чичагов~В.\,В.} Сравнение UMVUE и~MLE с~по\-мощью
  абсолютной функции потерь в~случае однопараметрического экспоненциального 
  семейства непрерывных распределений~// Статистические методы оценивания 
  и~проверки гипотез.~--- Пермь: ПГНИУ, 2012. С.~96--109.
 \bibitem{2-ch}
 \Au{Hwang T.-Y., Hu C.-Y.} More comparisons of MLE with UMVUE
  for exponential families~// Ann. Inst. Statist. Math., 1990. Vol.~42. P.~65--75.
 \bibitem{3-ch}
 \Au{Чичагов В.\,В.} Асимптотические разложения высокого порядка для
  несмещенных оценок и~их дисперсий в~модели однопараметрического 
  экспоненциального семейства~// Информатика и~её 
  применения, 2015. Т.~9. Вып.~3. С.~75--87.
 \bibitem{4-ch}
 \Au{Петров В.\,В.} Суммы независимых случайных величин.~---
  М.: Наука, 1972. 416~с.
 \bibitem{5-ch}
 \Au{Бхаттачария Р.\,Н., Рао~Р.\,Р.}
  Аппроксимация нормальным распределением и~асимптотические разложения~/
  Пер. с англ.~--- 
  М.: Наука, 1982. 286~с.
  (\Au{Bhattacharya~R.\,N., Rao~R.\,R.}
{Normal approximation and asymptotic expansions}. 
New York, NY, USA: Wiley, 1976. 316~p.)
  \end{thebibliography}

 }
 }

\end{multicols}

\vspace*{-3pt}

\hfill{\small\textit{Поступила в~редакцию 22.06.16}}

\vspace*{10pt}

%\newpage

%\vspace*{-24pt}

\hrule

\vspace*{2pt}

\hrule

%\vspace*{-2pt}



\def\tit{ASYMPTOTIC EXPANSIONS OF~MEAN ABSOLUTE ERROR OF~UNIFORMLY
MINIMUM VARIANCE UNBIASED AND~MAXIMUM LIKELIHOOD ESTIMATORS
 ON~THE~ONE-PARAMETER EXPONENTIAL FAMILY MODEL OF~LATTICE DISTRIBUTIONS}

\def\titkol{Asymptotic expansions of~mean absolute error of~UMVUE and~MLE
 on~the~one-parameter exponential family model} % of~lattice distributions}

\def\aut{V.\,V.~Chichagov}

\def\autkol{V.\,V.~Chichagov}

\titel{\tit}{\aut}{\autkol}{\titkol}

\vspace*{-9pt}

\noindent
 Perm State University, 15~Bukireva Str., Perm 614990, 
 Russian Federation


\def\leftfootline{\small{\textbf{\thepage}
\hfill INFORMATIKA I EE PRIMENENIYA~--- INFORMATICS AND
APPLICATIONS\ \ \ 2016\ \ \ volume~10\ \ \ issue\ 3}
}%
 \def\rightfootline{\small{INFORMATIKA I EE PRIMENENIYA~---
INFORMATICS AND APPLICATIONS\ \ \ 2016\ \ \ volume~10\ \ \ issue\ 3
\hfill \textbf{\thepage}}}

\vspace*{3pt}


 
\Abste{The paper considers a~model of duplicate sampling with
  the fixed size~$n$ from a~lattice distribution belonging to the natural
  one-parameter exponential family.
  Asymptotic expansions of the mean absolute errors of the uniformly 
  minimum variance unbiased estimator (UMVUE) and the maximum likelihood 
  estimator (MLE) of the given parametric function are obtained in the case of 
  infinite size of the sample.
  The case when $G'[a]=0$ and $G''[a] \neq 0 $ was studied separately.
  The relative error in calculating the difference in the mean absolute
  error UMVUE and MLE  was evaluated in the case of the Poisson 
  distribution for the two parametric functions.
  This error was received via the asymptotic expansions.
  It was found that the asymptotic results with a sufficiently large 
  sample size allows one to compare UMVUE and MLE using such indicator 
  of quality assessment as   the mean absolute error.}

 \noindent
\KWE{exponential family; lattice distribution;
 unbiased estimator; maximum likelihood estimator; asymptotic expansion}


\DOI{10.14357/19922264160309} 

%\vspace*{-12pt}

\Ack
 \noindent
 The research was financially supported by the Russian Ministry of Education
 and Science (project No.~2096).


\vspace*{3pt}

  \begin{multicols}{2}

\renewcommand{\bibname}{\protect\rmfamily References}
%\renewcommand{\bibname}{\large\protect\rm References}

{\small\frenchspacing
 {%\baselineskip=10.8pt
 \addcontentsline{toc}{section}{References}
 \begin{thebibliography}{9}
 
 \vspace*{-9pt}
 
\bibitem{1-ch-1}
\Aue{Fedoseeva, N.\,P., and V.\,V.~Chichagov}. 
2012. Sravnenie UMVUE i~MLE s~pomoshch'yu absolyutnoy funktsii po\-ter' v~sluchae 
odnoparametricheskogo eksponentsial'nogo semeystva nepreryvnykh raspredeleniy 
[Comparison of\linebreak\vspace*{-8pt}

\columnbreak

\noindent
 UMVUE and MLE by the absolute loss function for the one-parameter 
exponential family of continuous distributions]. 
\textit{Statisticheskie metody otsenivaniya i~proverki gipotez} 
[Statistical methods of  estimation and testing hypotheses]. 
Perm': Perm State University Publishing House.\linebreak 
96--109.

\pagebreak


\bibitem{2-ch-1}
\Aue{Hwang, T.-Y., and C.-Y.~Hu}. 1990. More comparisons of MLE with UMVUE 
for exponential families. \textit{Ann. Inst. Statist. Math.} 42:65--75.

%\columnbreak

\bibitem{3-ch-1}
\Aue{Chichagov, V.\,V.} 2015. 
Asimptoticheskie razlozheniya vysokogo poryadka dlya nesmeshchennykh otsenok i~ikh 
dispersiy v~modeli odnoparametricheskogo eksponentsial'nogo semeystva 
[Higher-order asymptotic expansions of unbiased estimators and their variances 
on the one-parameter exponential family model]. 
\textit{Informatika i~ee Primeneniya~--- Inform. Appl}. 9(3):75--87.

%\vspace*{6pt}

\bibitem{4-ch-1}
\Aue{Petrov, V.\,V., and A.\,A.~Brown}. 1975. \textit{Sums of independent random variables}. 
Berlin: Springer-Verlag. 182~p.


%\vspace*{6pt}

\bibitem{5-ch-1}
\Aue{Bhattacharya, R.\,N., and R.\,R.~Rao}. 
1976. \textit{Normal approximation and asymptotic expansions}. New York, NY: Wiley.\linebreak 316~p.

   \end{thebibliography}

 }
 }
 


\end{multicols}

\vspace*{-7pt}

\hfill{\small\textit{Received June 22, 2016}}

\vspace*{-20pt}

\Contrl

\noindent
\textbf{Chichagov Vladimir V.} (b.\ 1955)~---
 Candidate of Science (PhD) in physics and mathematics, associate professor, 
 Perm State University, 15~Bukireva Str., Perm 614990, 
 Russian Federation; \mbox{chichagov@psu.ru}
\label{end\stat}


\renewcommand{\bibname}{\protect\rm Литература}