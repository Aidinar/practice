\def\stat{yakovlev}

\def\tit{УСКОРЕННЫЙ АЛГОРИТМ СТЕРЕОСОПОСТАВЛЕНИЯ НА~ОСНОВЕ 
ГЕОДЕЗИЧЕСКИХ ВСПОМОГАТЕЛЬНЫХ КОЭФФИЦИЕНТОВ}

\def\titkol{Ускоренный алгоритм стереосопоставления на~основе 
геодезических вспомогательных коэффициентов}

\def\aut{О.\,А.~Яковлев$^1$, А.\,В.~Гасилов$^2$}

\def\autkol{О.\,А.~Яковлев, А.\,В.~Гасилов}

\titel{\tit}{\aut}{\autkol}{\titkol}

\index{Яковлев О.\,А.}
\index{Гасилов А.\,В.}
\index{Yakovlev O.\,A.}
\index{Gasilov A.\,V.}


%{\renewcommand{\thefootnote}{\fnsymbol{footnote}} \footnotetext[1]
%{Работа выполнена при частичной поддержке РФФИ (проект 16-07-00272 А).}}


\renewcommand{\thefootnote}{\arabic{footnote}}
\footnotetext[1]{Орловский филиал Федерального исследовательского центра <<Информатика и~управление>> Российской 
академии наук, \mbox{maucra@gmail.com}}
\footnotetext[2]{Орловский филиал Федерального исследовательского центра <<Информатика и~управление>> Российской 
академии наук, \mbox{gasilov.av@yandex.ru}}

\vspace*{-12pt}
  
  \Abst{Среди локальных алгоритмов стереосопоставления качественные результаты дают 
алгоритмы, использующие концепцию вспомогательных коэффициентов. В~данной работе 
предложена модификация алгоритма стереосопоставления на основе геодезических 
расстояний. Предлагаемая модификация существенно снижает вычислительные затраты, 
незначительно уступая в~качестве сопоставления оригинальному подходу, что 
подтверждается приведенными результатами экспериментов. Рассматриваемый 
алгоритм опирается на сегментацию одного из изображений стереопары, используя 
геодезические вспомогательные коэффициенты для вычисления цвета каждого сегмента. 
Такое преобразование изображения делает возможным применение принципа частичных 
сумм при вычислении стоимости сопоставления, что и~является основным источником 
прироста производительности.}
  
  \KW{стереосопоставление; сегментация; геодезические вспомогательные коэффициенты}
  
\DOI{10.14357/19922264160313} 


\vskip 10pt plus 9pt minus 6pt

\thispagestyle{headings}

\begin{multicols}{2}

\label{st\stat}

\section{Введение}

  Задача стереосопоставления двух изображений состоит в~установлении для 
каждой точки одного изображения соответствующей ей точки на втором\linebreak 
изображении. В~данной работе рассматривается самый распространенный 
вариант исходных данных~--- пара ректифицированных изображений. В~паре 
таких изображений естественным образом можно выделить левое ($I_L$) 
и~правое ($I_R$). На ректифицированных изображениях эпиполярные прямые 
параллельны оси~$Ox$, следовательно, соответствующие точки имеют равные 
координаты по оси~$Oy$. Пусть $p\hm\in I_L$ и~$p^\prime\hm\in I_R$~--- 
соответствующие точки, тогда диспарантностью пикселя~$p$ будем называть 
величину $\mathrm{disp}\,(p)\hm= p_x\hm-p^\prime_x$. Значения $\mathrm{disp}\,(p)$ для каждого 
$p\hm\in I_L$ образуют матрицу, называемую картой диспарантности. Карта 
диспарантности выступает результатом сте\-рео\-со\-по\-став\-ле\-ния изображений 
и~может быть использована, например, для трехмерной реконструкции сцены.
  
  Среди алгоритмов стереосопоставления выделяют глобальные и~локальные 
алгоритмы. Глобальные алгоритмы решают задачу оптимизации,\linebreak вы\-чис\-ляя 
оптимальную по некоторому критерию\linebreak карту диспарантности, что обеспечивает 
высокое качество результата. Однако вычислительная\linebreak сложность глобальных 
алгоритмов не позволяет применять их в~системах реального времени. 
Локальные алгоритмы являются представителями класса жадных алгоритмов, 
имеют высокую производительность, но качество вычисленных карт 
дис\-па\-рант\-ности у~них значительно ниже, чем у~глобальных алгоритмов. 
В~связи с~постоянным совершенствованием %\linebreak 
аппарат\-ных средств большинство 
исследований на\-прав\-ле\-но на улучшение качест-\linebreak ва
 сопоставления изображений. 
Как следствие, многие локальные алгоритмы приобрели черты глобальных, и~их 
применение в~условиях ограниченных вычислительных ресурсов становится 
невозможным. Алгоритм сопоставления изображений с~помощью 
геодезических вспомогательных коэффициентов~[1] является ярким примером 
локального алгоритма, %\linebreak 
обладающего как высоким качеством, так и~высокой 
вычислительной сложностью. В~настоящей работе рассматривается 
модификация этого алгоритма, позволяющая добиться роста 
производительности с~незначительной потерей качества.

\vspace*{-9pt}
  
\section{Стереосопоставление на~основе геодезических 
вспомогательных коэффициентов}

\vspace*{-2pt}
  
  Рассмотрим неориентированный граф, вершинами которого являются 
пиксели изображения и~каждая вершина соединена ребром с~8~соседними. 
Стоимостью ребра будем считать евклидово расстояние между двумя 
пикселями как RGB-век\-то\-ра\-ми. Стоимость кратчайшего пути между двумя 
вершинами такого графа будем называть геодезическим расстоянием между 
соответствующими пикселями изображения.
  
  В работе~[1] предлагается использовать в~качестве вспомогательного 
коэффициента величину, обратную геодезическому расстоянию:
  \begin{equation}
  w(p,q) =e^{-g(p,q)/\gamma}\,,
  \label{1-ya}
  \end{equation}
     где $g(p,q)$~--- геодезическое расстояние между~$p$ и~$q$;  
$\gamma$~--- положительный коэффициент.

Коэффициент~$\gamma$ позволяет регулировать скорость убывания $w(p,q)$ 
при возрастании $g(p,q)$. Рассмотрим функцию $f(x)\hm= e^{-x/\gamma}$. Ее 
производная $f^\prime(x)\hm= -(1/\gamma) e^{-x/\gamma}$. Таким образом, чем 
меньше~$\gamma$, тем быстрее убывает~$f(x)$.
  
  Для вычисления $\mathrm{disp}\,(p)$ с~помощью вспомогательных коэффициентов 
используется ска\-ни\-ру\-ющее окно, размер которого подбирается 
экспериментально. Пусть Win$_p$~--- сканирующее окно\linebreak с~центром 
в~точке~$p$, тогда стоимость диспарантности~$d$ для пикселя~$p$ можно 
вычислить как
  $$
  c(p,d) =\sum\limits_{q\in \mathrm{Win}_p} w(p,q)  \mathrm{diff} \left(q,\left(q_x-d, q_y\right ) 
\right)\,,
  $$
  где $\mathrm{diff}\,(p,q)$~--- величина цветоразличия пикселя~$p$ левого изображения 
и~пикселя~$q$ правого.
  
  Для вычисления диспарантности каждого пикселя будем использовать 
локальную оптимизацию: в~качестве $disp(p)$ возьмем такое значение~$d$, при 
котором достигается минимум стоимостной функции~$c(p,d)$, т.\,е.
  $$
  \mathrm{disp}\,(p)=\mathop{\mathrm{argmin}}\limits_{d\leq D} c(p,d)\,,
  $$
  где $D$~--- максимально допустимая диспарантность.
  
  Пусть $W$ и~$H$~--- ширина и~высота изображения соответственно; $s$~--- 
размер сканирующего окна. Тогда вычислительная сложность построения карты 
диспарантности описанным способом составит $O(W  H  s^2 D)$.

\section{Ускоренный алгоритм стереосопоставления}

  Нетрудно понять, что вспомогательные коэффициенты позволяют оценить 
вероятность принадлежности двух пикселей одной поверхности сцены. Более 
грубую оценку принадлежности двух пикселей одной поверхности дает 
сегментация изображения~[2]. Сегментация и~вспомогательные коэффициенты 
могут быть скомбинированы, что позволит снизить вычислительную сложность 
сопоставления изображений.
  
  Пусть $S_1\cup S_2\cup S_3\cup\cdots \cup S_m$~--- сегментация левого 
изображения. Опорной точкой pivot$_i$ сегмента~$S_i$ будем называть его 
медиану как упорядоченного по значениям координат списка точек. Цветом 
сегмента~$i$ будем называть средневзвешенный вспомогательными 
коэффициентами цвет составляющих его пикселей:
  $$
  C_i= \fr{\sum\nolimits_{p\in S_i} w(p,\mathrm{pivot}_i) I_L(p)} {\sum\nolimits_{p\in S_i} 
w(p,\mathrm{pivot}_i)}\,,
  $$
  где $I_L(p)$~--- цвет пикселя~$p$  на изображении.
  
  На данном этапе предполагается, что такой вариант использования 
коэффициентов принадлежности не приведет к~значительному снижению 
качества по сравнению с~оригинальным алгоритмом. Параметр~$\gamma$ 
в~выражении~(1) должен быть согласован с~уровнем сегментации, иначе 
теряется смысл применения вспомогательных коэффициентов~--- 
значение~$C_i$ будет близко к~среднему цвету сегмента. Высокому уровню 
сегментации соответствует малое значение~$\gamma$ и~наоборот. Рисунок~1 
иллюстрирует влияние параметра~$\gamma$ на значения вспомогательных 
коэффициентов.
  
  
  \begin{figure*} %fig1
  \vspace*{1pt}
 \begin{center}  
\mbox{%
 \epsfxsize=115mm
 \epsfbox{yak-1.eps}
 }
\end{center} 
\vspace*{-9pt}
\Caption{Вспомогательные коэффициенты при различных значениях~$\gamma$: 
(\textit{а})~$\gamma\hm=10$; (\textit{б})~15; (\textit{в})~30; (\textit{г})~$\gamma\hm=50$}
\end{figure*}

  Пусть $s(p)$~--- номер сегмента, которому принадлежит пиксель~$p$, тогда 
в~качестве стоимостной функции будем рассматривать величину
  \begin{equation}
  c(p,d) =\sum\limits_{q\in \mathrm{Win}_p} \left\| C_{s(q)} -I_R(q_x-d, q_y)\right\|_{2}\,,
  \label{e2-ya}
  \end{equation}
  где $I_R(x,y)$~--- цвет пикселя $(x,y)$ на правом изображении.
  
Используя стоимостную функцию~(2), можно значительно сократить 
вычислительные затраты на построение карты диспарантности~--- стоимостная 
функция для каждой позиции сканирующего окна может быть вычислена за 
$O(1)$ с~помощью частичных сумм.

  Рассмотрим метод частичных сумм. Имеется матрица $A\hm= (a_{i,j})$, 
состоящая из $n\times m$ элементов, и~необходимо многократно вычислять 
сумму подматриц этой матрицы. Пусть
  $$
  \mathrm{ps}_{r,c} =\sum\limits_{i=1}^r \sum\limits_{j=1}^c a_{i,j}\,.
  $$
Используя принцип динамического программирования, величину ps можно 
рассчитать за время порядка $O(n m)$ согласно соотношению:
\begin{equation}
\mathrm{ps}_{r,c}= \mathrm{ps}_{r-1,c} +\mathrm{ps}_{r,c-1} -
\mathrm{ps}_{r-1,c-1}+a_{r,c}\,.
\label{e3-ya}
\end{equation}
  
  Сумму на произвольной подматрице $(r_1,c_1)\hm- (r_2,c_2)$ можно 
вычислить по формуле вклю\-че\-ний-ис\-клю\-че\-ний:
  \begin{multline}
  \mathrm{sum}\left(r_1,c_1,r_2,c_2\right) = {}\\
  {}=\mathrm{ps}_{r_2, c_2} -\mathrm{ps}_{r_2, c_1-1} - \mathrm{ps}_{r_1-1, 
c_2} + \mathrm{ps}_{r_1-1, c_1-1}\,.
  \label{e4-ya}
  \end{multline}
  
  Воспользуемся методом частичных сумм для вычисления стоимостной 
функции $c(p,d)$. Пусть sum$_d(x_1, y_1, x_2, y_2)$~--- суммарная стоимость 
диспарантности~$d$ для всех пикселей прямоугольника $(x_1,y_1)\mbox{--}(x_2,y_2)$, т.\,е.
 \begin{multline*}
  \mathrm{sum}_d \left( x_1, y_1, x_2, y_2\right) = {}\\
  {}=\sum\limits_{i=x_1}^{x_2} 
\sum\limits_{j=y_1}^{y_2} \left\| C_{s(i,j)} -I_R (x-d,y)\right\|_2\,,
  \end{multline*}
  где $s(i,j)$~--- номер сегмента, которому принадле-\linebreak жит пиксель~($i,j$).
  
    Обозначив 
  $$
  \mathrm{ps}_d(x,y)= \sum\limits_{i\leq x} \sum\limits_{j\leq y} \left\| 
C_{s(i,j)} - I_R (x- d, y)\right\|_2\,,
$$
 согласно выражению~(\ref{e4-ya})  имеем:
  \begin{multline*}
  \mathrm{sum}_d\left( x_1, y_1, x_2, y_2\right) ={}\\
  {}= \mathrm{ps}_d\left( x_2, y_2\right)  - \mathrm{ps}_d\left( x_2-
1, y_1-1\right) - {}\\
{}-\mathrm{ps}_d \left( x_1-1, y_2-1\right) +\mathrm{ps}_d \left( x_1-1, y_1-1\right)\,.
  \end{multline*}
  
  С помощью выражения~(\ref{e3-ya}) величину~ps$_d$ можно вычислить за 
время порядка $O(W  H)$. Значение стоимостной функции для 
сканирующего окна размером $s\hm= 2h\hm+1$ выражается через~sum$_d$:

\vspace*{2pt}

\noindent
  $$
  c(p,d) = \mathrm{sum}_d \left( p_x -h, p_y -h, p_x+h, p_y+h\right)\,.
  $$
  
  \vspace*{-2pt}
  
  Таким образом, стоимостная функция для всевозможных позиций 
сканирующего окна при фиксированной диспарантности рассчитывается за 
время~$O(W  H)$; следовательно, вычислительная сложность построения 
карты диспарантности есть $O(W H D)$.
  
  Для определения ошибочных значений диспарантности воспользуемся 
стандартным критерием согласованности. Пусть Disp$_{L,R}$~--- карта 
диспарантности для левого изображения относительно правого, 
а~Disp$_{R,L}$~--- карта диспарантности для правого изображения 
относительно левого. В~соответствии с~приведенным ранее определением, 
значения диспарантности для правого изображения должны быть 
неположительными.
  
  Пусть Disp$_{L,R}(x,y)$ и~Disp$_{R,L} (x^\prime, y)$, где $x^\prime \hm= 
x\hm- \mathrm{Disp}_{L,R}$, являются корректными значениями диспарантности. Тогда 
выполняется равенство:

\noindent
  $$
  \mathrm{Disp}_{L,R} (x,y) =- \mathrm{Disp}_{R,L}\left(x^\prime, y\right)\,.
  $$
На основании этого равенства будем производить фильтрацию карты 
диспарантности: если равенство  для некоторого пикселя не выполняется, будем 
считать, что его диспарантность не установлена.
\begin{table*}
{\small \begin{center}
\Caption{Результаты расчета геодезических расстояний алгоритмом Borgefors}
\vspace*{2ex}

\begin{tabular}{|c|c|c|c|c|}
\hline
\tabcolsep=0pt\begin{tabular}{c}Число\\ итераций\end{tabular}&
Средняя ошибка&
\tabcolsep=0pt\begin{tabular}{c}Максимальная\\ ошибка\end{tabular}&
\tabcolsep=0pt\begin{tabular}{c}Время\\ расчета, с\end{tabular}&
\tabcolsep=0pt\begin{tabular}{c}Время расчета\\ алгоритмом Дейкстры, с\end{tabular}\\
\hline
2&0,0022&0,35&0,22&0,15\\
3&$4{,}5\cdot 10^{-4}$&0,15&0,32&\\
5&$4\cdot  
10^{-5}$&\hphantom{9}0,057 &0,53&\\
10\hphantom{9}&10$^{-8}$&10$^{-4}$&1,05&\\
\hline
\end{tabular}
\end{center}}
\end{table*}
\begin{figure*}[b] %fig2
%\renewcommand{\tablename}{\protect\bf Рис.}
%\setcounter{table}{1}
\vspace*{1pt}
 \begin{center}  
\mbox{%
 \epsfxsize=160mm
 \epsfbox{yak-2.eps}
 }
\end{center} 
\vspace*{-9pt}
\Caption{Результаты стереосопоставления: (\textit{а})~левое изображение стереопары; 
(\textit{б})~результат работы ускоренного алгоритма; (\textit{в})~результат работы 
оригинального алгоритма}
\end{figure*}
  


%  \renewcommand{\figurename}{\protect\bf Рис.}
%\renewcommand{\tablename}{\protect\bf Таблица}

%\addtocounter{figure}{1}
%\addtocounter{table}{-1}

  Ввиду того что при вычислении диспарант\-ности задействуется некоторая 
окрестность пикселя, вполне вероятным является случай незначительного 
рассогласования значений Disp$_{L,R}$ и~Disp$_{R,L}$. Для корректной 
обработки таких случаев следует ослабить критерий согласованности до
  $$
  \left\vert \mathrm{Disp}_{L,R} (x,y) +\mathrm{Disp}_{R,L} 
  \left( x^\prime, y\right) \right\vert \leq 
1\,.
  $$

\section{Вычисление геодезических расстояний}

  В работе~[1] для вычисления геодезических расстояний используется 
итерационный алгоритм Borgefors~\cite{3-ya}, позволяющий найти 
приближение геодезического расстояния с~заданной точностью. 
Вычислительная сложность алгоритма Borgefors линейна относительно размера 
сегмента.
  
  Для точного вычисления геодезических расстояний можно воспользоваться 
алгоритмом Дейкстры, который позволяет вычислить длины кратчайших путей 
от одной вершины графа до остальных. В~качестве вершин графа будем 
использовать пиксели одного сегмента, а~ребро между двумя вершинами будет 
существовать, если соответствующие им пиксели являются смежными в~восьми 
направлениях. Стоимость ребра~--- евклидово расстояние между цветами 
пикселей в~системе RGB. Очевидно, что длина кратчайшего пути в~таком графе 
будет геодезическим расстоянием по определению.
  
  Описанный граф обладает важным свойством: число ребер линейно зависит 
от числа вершин. Такое свойство дает возможность реализовать алгоритм 
Дейкстры с~использованием двоичной кучи~[4], снизив вычислительную 
сложность до $O(\vert V\vert \log \vert V \vert )$, где $\vert V\vert$~--- чис\-ло 
вершин графа, т.\,е.\ размер сегмента.
  
  В ходе экспериментов было выявлено, что асимптотически лучшее 
быстродействие алгоритма Borgefors проявляется лишь на областях достаточно 
большого размера, наличие которых в~сегментации реальных снимков 
маловероятно. В~табл.~1 представлены результаты расчета геодезических 
расстояний алгоритмом Borgefors с~разным числом итераций для изображения 
размером $450\times 375$.




  Таким образом, применение алгоритма Borgefors не оправдано при расчете 
геодезических расстояний внутри сегмента.

\vspace*{-12pt}

\section{Результаты}

  Алгоритм был реализован на языке C++ в~соответствии с~приведенным 
описанием. Реализация алгоритма доступна в~открытом 
репозитории ({\sf https:// github.com/helgui/FastGSW.}). Оценка результата 
стереосопоставления проводилась по методике, описанной в~работе~[5], 
с~по\-мощью предоставленного авторами работы программного обеспе\-чения 
Middlebury Stereo 
Evaluation SDK. %\linebreak\vspace*{-12pt}
  Набор входных данных~[6] состоит из~14~стереопар различной 
конфигурации. В~ходе экспериментов проводилось сравнение ускоренного 
алгорит-\linebreak\vspace*{-12pt}



\pagebreak

\end{multicols}

\begin{table}\small %tabl2
\begin{center}
\Caption{Параметры запуска алгоритмов}
\vspace*{2ex}

\begin{tabular}{|l|c|c|}
\hline
\multicolumn{1}{|c|}{Параметр}&
\tabcolsep=0pt\begin{tabular}{c}Значение\\ для оригинального\\ алгоритма\end{tabular}&
\tabcolsep=0pt\begin{tabular}{c}Значение\\ для ускоренного\\ алгоритма\end{tabular}\\
\hline
Размер сканирующего окна&31&15\\
Коэффициент~$\gamma$&10&5\\
Алгоритм сегментации&Не используется&Сегментация на основе графа~[7]\\
\hline
\end{tabular}
\end{center}
%\end{table*}
%\begin{table*}\small %tabl3
\begin{center}
\Caption{Результаты экспериментов}
\vspace*{2ex}

\begin{tabular}{|l|l|c|c|c|}
\hline
\multicolumn{1}{|c|}{Название теста}&\multicolumn{1}{c|}{
\tabcolsep=0pt\begin{tabular}{c}Используемый \\ алгоритм\end{tabular}}&
\tabcolsep=0pt\begin{tabular}{c}Значения\\ диспарантности\\ с~ошибкой\\ более 2 единиц,
 \%\end{tabular}&
 \tabcolsep=0pt\begin{tabular}{c}Неустановленные\\  значения\\ диспарантности,\\ \%\end{tabular}&
 \tabcolsep=0pt\begin{tabular}{c}Время\\ выполнения,\\ с\end{tabular}\\
\hline
\multicolumn{1}{|l|}{\raisebox{-6pt}[0pt][0pt]{Adirondack}}&
Оригинальный&\hphantom{9}9,98&66,72&562,19\\
&Ускоренный&10,26&60,44&\hphantom{99}6,39\\
\hline
\multicolumn{1}{|l|}{\raisebox{-6pt}[0pt][0pt]{Jadeplant}}&Оригинальный&
\hphantom{9}4,05&40,03&1063,23\hphantom{9}\\
&Ускоренный&11,79&60,23&\hphantom{99}6,31\\
\hline
\multicolumn{1}{|l|}{\raisebox{-6pt}[0pt][0pt]{Motorcycle}}&Оригинальный&
\hphantom{9}5,72&31,98&557,48\\
&Ускоренный&\hphantom{9}6,92&39,03&\hphantom{99}6,58\\
\hline
\multicolumn{1}{|l|}{\raisebox{-6pt}[0pt][0pt]{Motorcycle E}}&Оригинальный&
14,46&82,6\hphantom{9}&556,19\\
&Ускоренный&\hphantom{9}8,31&84,86&\hphantom{99}6,77\\
\hline
\multicolumn{1}{|l|}{\raisebox{-6pt}[0pt][0pt]{Piano}}&Оригинальный&13,71&26,21&478,00\\
&Ускоренный&\hphantom{9}8,55&47,49&\hphantom{99}6,17\\
\hline
\multicolumn{1}{|l|}{\raisebox{-6pt}[0pt][0pt]{Piano L}}&Оригинальный&10,77&81,84&479,30\\
&Ускоренный&\hphantom{9}7,49&84,33&\hphantom{99}6,17\\
\hline
\multicolumn{1}{|l|}{\raisebox{-6pt}[0pt][0pt]{Pipes}}&Оригинальный&
\hphantom{9}5,24&36,68&578,16\\
&Ускоренный&10,7\hphantom{9}&38,23&\hphantom{99}6,45\\
\hline
\multicolumn{1}{|l|}{\raisebox{-6pt}[0pt][0pt]{Playroom}}&Оригинальный&10,19&61,9\hphantom{9}&597,20\\
&Ускоренный&11,08&47,81&\hphantom{99}6,02\\
\hline
\multicolumn{1}{|l|}{\raisebox{-6pt}[0pt][0pt]{Playtable}}&Оригинальный&10,86&42,08&497,01\\
&Ускоренный&10,04&54,52&\hphantom{99}5,67\\
\hline
\multicolumn{1}{|l|}{\raisebox{-6pt}[0pt][0pt]{Playtable P}}&Оригинальный&
\hphantom{9}7,15&36,33&498,76\\
&Ускоренный&\hphantom{9}9,98&54,4\hphantom{9}&\hphantom{99}5,83\\
\hline
\multicolumn{1}{|l|}{\raisebox{-6pt}[0pt][0pt]{Recycle}}&Оригинальный&
\hphantom{9}8,78&31,9\hphantom{9}&491,86\\
&Ускоренный&8,4&53,29&\hphantom{99}6,35\\
\hline
\multicolumn{1}{|l|}{\raisebox{-6pt}[0pt][0pt]{Shelves}}&Оригинальный&16,64&42,83&475,90\\
&Ускоренный&10,19&58,21&\hphantom{99}6,52\\
\hline
\multicolumn{1}{|l|}{\raisebox{-6pt}[0pt][0pt]{Teddy}}&Оригинальный&
\hphantom{9}3,11&30,12&233,55\\
&Ускоренный&7,1&34,48&\hphantom{99}3,36\\
\hline
\multicolumn{1}{|l|}{\raisebox{-6pt}[0pt][0pt]{Vintage}}&Оригинальный&13,06&32,35&
1283,93\hphantom{9}\\
&Ускоренный&7,9&52,3\hphantom{9}&\hphantom{99}6,38\\
\hline
\end{tabular}
\end{center}
\end{table}

\begin{multicols}{2}





\noindent
ма с~оригинальным (рис.~2). Параметры запуска алгоритмов приведены 
в~табл.~2. Следует заметить, что параметры для оригинального алгоритма 
заданы в~соответствии с~рекомендациями авторов. Результаты экспериментов 
сведены в~табл.~3.
  


  Главный критерий оценки качества~--- доля пикселей от общего числа, 
вычисленное значение диспарантности для которых отличается более чем 
на~2~единицы от истинного значения. Разница между средними значениями 
этого параметра для оригинального и~ускоренного алгоритмов 
составила~0,33\%.
  
  При оценке также использовался дополнительный критерий~--- доля 
пикселей, отброшенных при фильтрации. Разница между средними значениями 
составила~8,4\% в~пользу оригинального алго-\linebreak ритма.
  
  Из табл.~3 видно, что ускоренный алгоритм имеет многократное 
превосходство в~производительности. Исходя из приведенных ранее 
асимптотических оценок сложности, нетрудно понять, что разни\-ца во времени 
расчета при увеличении размера сканирующего окна будет расти 
пропорционально квадрату этого размера.

%\vspace*{-6pt}

\section{Заключение}

  В данной работе была описана модификация алгоритма~[1] и~представлена 
его реализация. Результаты экспериментов позволили убедиться в~том, что 
путем незначительной потери качества был получен существенный прирост 
производительности. Также очевидным преимуществом рассмотренного метода 
является эффективность интеграции с~сис\-те\-ма\-ми компьютерного зрения, 
которые уже используют сегментацию при анализе изображений.
  
  Единственным недостатком предложенного подхода является меньшая 
плотность вычисленных карт диспарантности, что может свидетельствовать 
о~недостаточной совместимости алгоритма с~фильт\-ра\-ци\-ей по критерию 
согласованности.

\vspace*{-6pt}
  
{\small\frenchspacing
 {%\baselineskip=10.8pt
 \addcontentsline{toc}{section}{References}
 \begin{thebibliography}{9}
\bibitem{1-ya}
\Au{Hosni A., Bleyer M., Gelautz~M., Rhemann~C.} Local stereo matching using 
geodesic support weights~// 16th IEEE Conference (International) on Image 
Processing.~--- IEEE Press, 2009. P.~2093--2096.
\bibitem{2-ya}
\Au{Shapiro L.\,G., Stockman G.\,C.} Computer vision.~--- Upper Saddle River, NJ, 
USA: Prentice Hall, 2001. 580~p.
\bibitem{3-ya}
\Au{Borgefors G.} Distance transformations in digital images~// Comput. Vision 
Graph. Image Proc., 1986. Vol.~34. No.\,3. P.~344--371.
\bibitem{4-ya}
\Au{Кормен Т.\,Х., Лейзерсон~Ч.\,И., Ривест~Р.\,Л., Штайн~К.} Алгоритмы: 
построение и~анализ~/ Пер. с~англ.~--- 3-е изд.~--- М.: Вильямс, 2013. 1328~с. 
(\Au{Cormen~T.\,H., Leiserson~С.\,E., Rivest~R.\,L., Stein~C.} Introduction to 
algorithms.~--- 3rd ed.~--- MIT Press, 2009. 1312~p.).
\bibitem{5-ya}
\Au{Scharstein D., Szeliski~R.} A~taxonomy and evaluation of dense two-frame 
stereo correspondence algorithms~// IJCV, 2002. Vol.~47. No.\,1/2/3. P.~7--42.
\bibitem{6-ya}
\Au{Scharstein D., Hirschm$\ddot{\mbox{u}}$ller~H., Kitajima~Y., Krathwohl~G., 
Nesic~N., Wang~X., Westling~P.} High-resolution stereo datasets with  
subpixel-accurate ground truth~// German Conference on Pattern Recognition.~--- 
Springer, 2014. P.~31--42.
\bibitem{7-ya}
\Au{Felzenszwalb P.\,F., Huttenlocher~D.\,P.} Efficient graph-based image 
segmentation~// IJCV, 2004. Vol.~59. No.\,2. P.~167--181.
\end{thebibliography}

 }
 }

\end{multicols}

%\vspace*{-6pt}

\hfill{\small\textit{Поступила в~редакцию 14.07.16}}

\vspace*{10pt}

%\newpage

%\vspace*{-24pt}

\hrule

\vspace*{2pt}

\hrule

\vspace*{8pt}



\def\tit{SPEEDED-UP STEREO MATCHING USING~GEODESIC SUPPORT 
WEIGHTS}

\def\titkol{Speeded-up stereo matching using~geodesic support 
weights}

\def\aut{O.\,A.~Yakovlev and A.\,V.~Gasilov}

\def\autkol{O.\,A.~Yakovlev and A.\,V.~Gasilov}

\titel{\tit}{\aut}{\autkol}{\titkol}

\vspace*{-9pt}

\noindent
Orel Branch of the Federal Research Center ``Computer Science and Control'' of the 
Russian Academy of Sciences, 137~Moskovskoe Sh., Orel 302025, Russian 
Federation


\def\leftfootline{\small{\textbf{\thepage}
\hfill INFORMATIKA I EE PRIMENENIYA~--- INFORMATICS AND
APPLICATIONS\ \ \ 2016\ \ \ volume~10\ \ \ issue\ 3}
}%
 \def\rightfootline{\small{INFORMATIKA I EE PRIMENENIYA~---
INFORMATICS AND APPLICATIONS\ \ \ 2016\ \ \ volume~10\ \ \ issue\ 3
\hfill \textbf{\thepage}}}

\vspace*{9pt}


 
\Abste{In local stereo matching, the algorithms based on the adaptive support weights 
have good-quality results. This paper presents a modified version of the local 
matching with geodesic support. The proposed algorithm considerably reduces 
computation time at the cost of insignificant loss of quality that is experimentally 
confirmed with Middlebury Stereo Evaluation SDK. The key idea is to combine 
geodesic support weights and image segmentation for recoloring the reference image. 
This transformation makes it possible to use partial sums for matching cost 
computation.}

\vspace*{3pt}

\KWE{stereo matching; segmentation; geodesic support weights}


\vspace*{3pt}

\DOI{10.14357/19922264160313} 

%\vspace*{-9pt}

%\Ack
%\noindent



\vspace*{18pt}

  \begin{multicols}{2}

\renewcommand{\bibname}{\protect\rmfamily References}
%\renewcommand{\bibname}{\large\protect\rm References}

{\small\frenchspacing
 {%\baselineskip=10.8pt
 \addcontentsline{toc}{section}{References}
 \begin{thebibliography}{9}
\bibitem{1-ya-1}
\Aue{Hosni, A., M. Bleyer, M.~Gelautz, and C.~Rhemann}. 2009. Local stereo 
matching using geodesic support weights. \textit{16th IEEE Conference 
(International) on Image Processing}. IEEE Press. 2093--2096.
\bibitem{2-ya-1}
\Aue{Shapiro, L.\,G., and G.\,C.~Stockman}. 2001. \textit{Computer vision}.  
Upper Saddle River, NJ: Prentice Hall. 580~p.
\bibitem{3-ya-1}
\Aue{Borgefors, G.} 1986. Distance transformations in digital images.  
\textit{Comput. Vision Graph. Image Proc.} 34:344--371.
\bibitem{4-ya-1}
\Aue{Cormen, T.\,H., Сh.\,E.~Leiserson, R.\,L.~Rivest, and C.~Stein}. 2009. 
\textit{Introduction to algorithms}. 3rd ed. MIT Press. 1312~p.

%\pagebreak

\bibitem{5-ya-1}
\Aue{Scharstein, D., and R.~Szeliski}. 2002. A~taxonomy and evaluation of 
dense two-frame stereo correspondence algorithms. \textit{IJCV}  
47(1-3):7--42.
\bibitem{6-ya-1}
\Aue{Scharstein, D., H.~Hirschm$\ddot{\mbox{u}}$ller, Y.~Kitajima, 
G.~Krathwohl, N.~Nesic, X.~Wang, and P.~Westling}. 2014. High-resolution 
stereo datasets with subpixel-accurate ground truth. \textit{German Conference 
on Pattern Recognition}. Springer. 31--42.
\bibitem{7-ya-1}
\Aue{Felzenszwalb, P.\,F., and D.\,P.~Huttenlocher}. 2004. Efficient 
graph-based image segmentation. \textit{IJCV} 59:167--181.
   \end{thebibliography}

 }
 }

\end{multicols}

\vspace*{-3pt}

\hfill{\small\textit{Received July 14, 2016}}

\vspace*{-3pt}

\Contr

\noindent
\textbf{Yakovlev Oleg A.} (b.\ 1992)~--- research engineer, Orel Branch of the 
Federal Research Center ``Computer Science and Control'' of the Russian Academy 
of Sciences, 137~Moskovskoe Sh., Orel 302025, Russian Federation; 
\mbox{maucra@gmail.com}


\vspace*{5pt}

\noindent
\textbf{Gasilov Artur V.} (b.\ 1992)~--- research engineer, Orel Branch of the Federal 
Research Center ``Computer Science and Control'' of the Russian Academy of 
Sciences, 137~Moskovskoe Sh., Orel 302025, Russian Federation; 
\mbox{gasilov.av@ya.ru}

  \label{end\stat}
  
  
  \renewcommand{\bibname}{\protect\rm Литература}