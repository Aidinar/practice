\def\Ai{\mathrm{Ai}\,}
\def\Bi{\mathrm{Bi}\,}
\def\ctn{{\mathrm{CБH}}}
\def\an{{\mathrm{AH}}}
\def\tn{{\mathrm{БH}}}



\def\stat{sinitsin}

\def\tit{АНАЛИТИЧЕСКОЕ МОДЕЛИРОВАНИЕ  НОРМАЛЬНЫХ ПРОЦЕССОВ В~СТОХАСТИЧЕСКИХ СИСТЕМАХ
  СО~СЛОЖНЫМИ БЕССЕЛЕВЫМИ   НЕЛИНЕЙНОСТЯМИ ДРОБНОГО ПОРЯДКА$^*$}

\def\titkol{Аналитическое моделирование  нормальных процессов в~СтС
  со~сложными БНДП} %бесселевыми нелинейностями дробного порядка}

\def\aut{И.\,Н.~Синицын$^1$}

\def\autkol{И.\,Н.~Синицын}

\titel{\tit}{\aut}{\autkol}{\titkol}

\index{Синицын И.\,Н.}
\index{Sinitsyn I.\,N.}


{\renewcommand{\thefootnote}{\fnsymbol{footnote}} \footnotetext[1]
{Работа выполнена при поддержке ОНИТ РАН (проект 0063-2015-0017~III.3).}}


\renewcommand{\thefootnote}{\arabic{footnote}}
\footnotetext[1]{Институт проблем информатики Федерального исследовательского
центра <<Информатика и~управление>> Российской академии наук, \mbox{sinitsin@dol.ru}}

 
 
\Abst{Рассматриваются методы аналитического моделирования (МАМ) нормальных 
(гауссовских) процессов в~гауссовских и~негауссовских стохастических системах (СтС) 
со сложными бесселевыми нелинейностями (БН) дробного порядка (БНДП) (сферическими, 
модифицированными сферическими и~описываемыми функциями Эйри). 
Приведены необходимые сведения из теории бесселевых функций дробного порядка. 
Даны формулы для коэффициентов статистической линеаризации 
БНДП. Особое внимание уделено алгоритмам вычисления этих 
коэффициентов, основанным на степенных разложениях. Приведены 
алгоритмы МАМ нормальных процессов в~СтС с~БНДП на основе метода нормальной 
аппроксимации (МНА) и~метода статистической линеаризации (МСЛ). 
Разработаны методы вычисления типовых интегралов для БНДП. 
Рассмотрен ряд тестовых примеров. Сделаны основные выводы 
и~некоторые обобщения.}


\KW{бесселева нелинейность дробного порядка (БНДП);
метод аналитического моделирования (МАМ);
метод нормальной аппроксимации (МНА);
метод статистической линеаризации (МСЛ);
модифицированные сферические бесселевы нелинейности;
нелинейности Эйри;
нормальный (гауссовский) стохастический процесс;
сферические бесселевы нелинейности}

\DOI{10.14357/19922264160308} 
 
 \vspace*{12pt}


\vskip 12pt plus 9pt minus 6pt

\thispagestyle{headings}

\begin{multicols}{2}

\label{st\stat}

\section{Введение}


В~[1] рассматриваются МАМ процессов 
в~динамических системах со сложными цилиндрическими БН
целого порядка при гармонических и~стохастических, узкополосных и~широкополосных 
возмущениях. Даны необходимые сведения из теории бесселевых функций и~сложных 
БН (СБН). Приведено методическое и~алгоритмическое обеспечение 
МАМ на основе МСЛ и~МНА для стохастических широкополосных процессов типа белого\linebreak шума. Рассмотрены 
особенности МАМ для гармонических и~узкополосных стохастических процессов. 
В~приложении приведены формулы для коэффициентов МСЛ для типовых бесселевых 
нелинейностей целого порядка. В~качестве тестовых примеров рассмотрены задачи 
МАМ процессов в~одномерных системах с~аддитивными и~мультипликативными белыми 
шумами. Особое внимание уделено процессам в~бесселевом осцилляторе в~условиях 
различных возмущений.  Заключение содержит основные выводы и~некоторые обобщения.

Рассмотрены особенности МАМ нормальных процессов на основе МНА (МСЛ) 
в~гауссовских и~негауссовских СтС со сложными 
БНДП.
В~разд.~2, 3 и~приложениях~П1 и~П2 даны необходимые сведения из теории 
действительных бесселевых функций и~сложных БНДП. Разделы~2--7 посвящены 
степенным алгоритмам вычисления интегралов МНА (МСЛ). В~разд.~8 
приведены алгоритмы МАМ нормальных процессов в~СтС с~БНДП, а~в~приложении~П3~--- 
тестовые примеры. Заключение содержит основные выводы и~некоторые обобщения.


\section{Функции Бесселя дробного порядка}

Как известно~[2--4], наиболее распространенными функциями Бесселя (ФБ) дробного 
порядка являются сферические и~модифицированные сферические ФБ, а~также функции 
Эйри.

Сферические ФБ являются частными решениями обыкновенного линейного дифференциального 
уравнения второго порядка следующего вида:
\begin{equation}
    z^2 \fr{d^2w}{dz^2} + 2 z \fr{dw}{dz} + 
    \lk z^2 -\nu (\nu+1)\rk w =0\,.\label{e2.1-sin}
    \end{equation}
Сферические ФБ первого и~второго рода являются действительными функциями и~определяются 
следующими формулами:
\begin{align*}
j_n (z) &= \sqrt{\fr{\pi}{2z}} \,J_{n+1/2} (z)\,;%\label{e2.2-sin}
\\
y_n (z) &= \sqrt{\fr{\pi}{2z}}\, Y_{n+1/2} (z)\,. %\label{e2.3-sin}
\end{align*}
Здесь введены следующие обозначения (см.\ приложение~П1): 
$J_\nu (z)$~--- ФБ первого рода;  $Y_\nu (z)$~--- 
ФБ второго рода (называемые также функциями Вебера или Неймана, 
причем $Y_\nu (z)\hm \equiv N_\nu (z)$).

Модифицированные сферические ФБ первого и~второго рода удовлетворяют
уравнению~(\ref{e2.1-sin}) при замене~$z$ на~$iz$:
    \begin{equation*}
    z^2 \fr{d^2 w}{dz^2} + 2 z \fr{dw}{dz} -
    \lk z^2 +\nu(\nu+1)\rk w =0 %\label{e2.4-sin}
    \end{equation*}
и при $\nu =\pm (n\pm 1/2)$ определяются следующими формулами:
\begin{gather*}
i_n (z) = \sqrt{\fr{\pi}{ 2z}}\, I_{n+1/2} (z)\,; %\label{e2.5-sin}
\\
\tilde i_n (z) + \sqrt{\fr{\pi}{2z}}\, I_{-n-1/2} (z)\,. %\label{e2.6-sin}
\end{gather*}
Здесь $I_n (z)$~--- модифицированная цилиндрическая ФБ (см.\ приложение~П1).

Функции Эйри  $\mathrm{Ai}\,(z)$  и~$\mathrm{Bi}\,(z)$ удовлетворяют 
дифференциальному уравнению вида
\begin{equation*}
\fr{d^2 w}{dz^2} - z w =0 
%\label{e2.7-sin}
\end{equation*}
и определяются следующими формулами:
\begin{equation}
\left.
\begin{array}{rl}
\Ai (z) &= c_1 f(z) - c_2 g(z)\,;\\[6pt]
\Bi(z) &= \displaystyle\sqrt{3} \lk c_1 f(z) + c_2 g(z)\rk.
\end{array}
\right\}
\label{e2.8-sin}
\end{equation}
Здесь
\begin{align*}
    f(z) &= \displaystyle\sss_{l=0}^\infty 3^l \left(\fr{1}{3}\right)_l \,
    \fr{z^{3l}}{(3l)!}\,;\\
    g(z) &= \displaystyle\sss_{l=0}^\infty 3^l \left(\fr{2}{3}\right)_l \fr{z^{3l+1}}{(3l+1)!}\,,
    %    \label{e2.9-sin}
    \end{align*}
где
\begin{gather*}
c_1 = \Ai (0) =\fr {\Bi(0)}{3} =
    \fr{3^{-2/3}}{\Gamma (2/3)} \approx 0{,}35502\,;\\ 
    c_2 = -\Ai' (0) = \fr{\Bi'(0) }{\sqrt{3}} =
    \fr{3^{-1/3}}{\Gamma (1/3)} \approx 0{,}25881\,;
\end{gather*}

\noindent
\begin{gather*}
   \left(a+\fr{1}{3} \right)_0 =1\,;\\[2pt] 
   3^l\left(a+\fr{1}{3} \right)_l = (3a+1) (3a+4) \cdots (3a + 3l-2)\,.
   \end{gather*}

В приложении П2 приведены необходимые свойства и~степенные представления 
ФБ дробного  порядка.

\section{Сложные бесселевы нелинейности}

Рассмотрим скалярное безынерционное нелинейное детерминированное преобразование вида
\begin{equation}
Z=\varphi (Y)\,.\label{e3.1-sin}
\end{equation}
Здесь под $\varphi$ понимается ФБ. Нелинейности, описываемые~(\ref{e3.1-sin}), 
следуя~[1], будем называть бесселевыми нелинейностями.

Примерами СБН~[1], получаемых посредством сумм типовых 
БН, могут служить следующие:
    \begin{align*}
    \varphi^\ctn (Y,t) &=\displaystyle\sss_{r=1}^n l_{rt} \varphi^\tn_r (Y)\,;\\[2pt]
    \varphi^\ctn (Y,t) &=\displaystyle\sss_{r=1}^n l_{rt} \varphi^\tn_r (Y)\varphi_r^\an (Y)\,,
%    \label{e3.2-sin}
    \end{align*}
а также дробно-ра\-ци\-о\-наль\-ные представления:
\begin{align*}
\varphi^\ctn (Y,t) &=\displaystyle
    \fr{\sum\nolimits_{r=1}^{n'} l_{rt}' {\varphi'}^\tn_r (Y)}{\sum\nolimits_{r=1}^{n''} l_{rt}'' 
    {\varphi''}^\tn_r (Y)}\,;\\[2pt] 
    \varphi^\ctn (Y,t) &=\displaystyle
    \fr{\sum\nolimits_{r=1}^{n'} l_{rt}' {\varphi'}^\tn_r (Y){\varphi'}_r^\an (Y)}{
    \sum\nolimits_{r=1}^{n''} l_{rt}'' {\varphi''}^\tn_r (Y){\varphi''}_r^\an (Y)}\,,
   %    \label{e3.3-sin}
\end{align*}
где ${\varphi}^\tn_r (Y)$, ${\varphi'}^\tn_r (Y)$ и~${\varphi''}^\tn_r (Y)$~--- 
типовые бес\-селевы функции; $ l_{rt}$,  $l_{rt}'$ и~ $l_{rt}''$~--- 
коэффициенты, зависящие от времени~$t$; ${\varphi}_r^\an (Y)$, ${\varphi'}_r^\an (Y)$
и~${\varphi''}_r^\an (Y)$~--- алгебраические нелинейности (многочлены, степенные, 
иррациональные, дроб\-но-ра\-цио\-наль\-ные и~другие функции).

Другими примерами СБН являются нелинейности, получаемые путем соответствующего 
преобразования аргумента:

\pagebreak

\noindent
    \begin{align*}
    \varphi^\ctn (Y,t)& =\varphi^\an \left(\psi^\tn (Y,t), t\right)\,;\\
    \varphi^\ctn (Y,t) &=\varphi^\tn \left(\psi^\an (Y,t), t\right)\,,
%    \label{e3.4-sin}
    \end{align*}
где $\varphi^\an$, $\psi^\an$, $\varphi^\tn (Y,t)$ и~$\psi^\tn (Y,t)$~--- 
типовые алгебраические и~бесселевы трансцендентные нелинейности.

В качестве примеров скалярных СБН векторного аргумента 
$Y\hm=\lk Y_1,\cdots Y_p\rk^{\mathrm{T}}$ рассмотрим следу\-ющие:
    \begin{align*}
    \varphi^\ctn (Y,t) &=\displaystyle\sss_{r=1}^n \prod\limits_{h=1}^H l_{rh,t} 
    \varphi_{rh}^\tn (Y_h)\,;    \\
    \varphi^\ctn (Y,t) &=\displaystyle\fr{\sss_{r=1}^{n'} \prod_{h=1}^{H'} l_{rh,t}' 
    {\varphi'}_{rh}^\tn (Y_h)}{\sss_{r=1}^{n''} \prod_{h=1}^{H''} l_{rh,t}'' 
    {\varphi''}_{rh}^\tn (Y_h)}\,.
 %    \label{e3.5-sin}
    \end{align*}
В случае векторных и~матричных СБН приведенные формулы имеют место 
для соответствующих компонент.

\section{Статистическая линеаризация бесселевых нелинейностей дробного порядка}


Рассмотрим статистическую линеаризацию по Казакову~[5--7] бесселевой 
нелинейности~(\ref{e3.1-sin}) (индекс <<СБН>> для краткости опускается)
при несимметричном $(m_y\hm\ne 0)$ гауссовском стохастическом входном сигнале~$Y_t$:
\begin{equation*}
Y_t = Y(t) = m_y + Y_t^0\,,
%\label{e4.1-sin}
\end{equation*}
где $m_y$~--- математическое ожидание, а~$D_y$~--- 
дисперсия, $Y_t^0\hm = Y^0(t)\hm = Y(t)\hm - m_y$. 
В~соответствии с~МСЛ зависимость 
аппроксимируется следующим выражением:
  \begin{equation*}
  Z_t =  \varphi_0 \left(m_y, D_y\right) + k_1\left(m_y, D_y\right) Y_t^0\,.
%  \label{e4.2-sin}
  \end{equation*}
Здесь $\varphi_0$ и~$k_1$~--- коэффициенты статистической линеаризации, зависящие 
от~$m_y$ и~$D_y$  и~определяемые по формулам:
    \begin{multline*}
    \varphi_0 = \varphi_0 (m_y, D_y) = {}\\
{}=\fr{1}{\sqrt{2\pi D_y}} \iin 
    \varphi(\eta) e^{-(\eta-m_y)^2/(2D_y)}\, d\eta\,; %\label{e4.3-sin}
   \end{multline*}
   
   \noindent
   \begin{multline*}
k_1 = k_1(m_y, D_y) ={}\\
{}= \fr{1}{\sqrt{2\pi D_y}} \iin (\eta-m_y) 
    \varphi(\eta) e^{-(\eta-m_y)^2/(2D_y)}\, d\eta ={}\\
{}= 
    \fr{\prt \varphi_0 (m_y, D_y)}{\prt m_y}\,. %\label{e4.4-sin}
    \end{multline*}
Для нечетных функций~$\varphi$ имеем:
 \begin{align*}
 \varphi_0 (m_y, D_y) &= k_0 (m_y, D_y) m_y\,; %\label{e4.5-sin}
 \\
 k_0(m_y, D_y) &= \fr{1}{m_y \sqrt{2\pi D_y}} \iin \hspace*{-5pt}\varphi 
    (\eta) e^{-(\eta-m_y)^2/(2D_y)}\, d\eta\,. %\label{e4.6-sin}
    \end{align*}

В зависимости от аналитической формы представления СБН различают 
следующие методы вычисления коэффициентов статистической линеаризации~\cite{4-sin}:
\begin{itemize}
\item выражения через элементарные функции;
\item  представления по формулам Рэлея;
\item степенные разложения;
\item  многочленные приближения;
\item  дробно-рациональные приближения;
\item разложения в~цепные дроби;
\item  асимптотические формулы и~приближения;
\item  рекуррентные соотношения.
\end{itemize}

В статье ограничимся разработкой алгоритмов расчетов, 
основанных на степенных разложениях.

\section{Коэффициенты статистической линеаризации сферических бесселевых 
нелинейностей~{\boldmath{$j_n (z)$}} и~модифицированных сферических 
нелинейностей~{\boldmath{$i_n (z)$}} первого~рода}


Используя степенные представления функций $i_n (z)$ и~$i_n (z)$~[2--4]
\begin{align*}
j_n (z) &= \sss_{k=0}^\infty \fr{(-1)^k z^{2k+n}}
    {(2k)!! (2k + 2n + 1)!!}\,; %\label{e5.1-sin}
    \\
i_n (z)& = \sss_{k=0}^\infty \fr{ z^{2k+n}}{(2k)!! (2k + 2n + 1)!!} %\label{e5.2-sin}
    \end{align*}
и табличные интегралы
    \begin{align*}
       \displaystyle \iii_0^\infty t^{2m+1} e^{-ct^2/2}\, dt &= 
       \fr{m!}{2^m c^{m+1}}\,; \\
 \displaystyle\iii_0^\infty t^{2m} e^{-ct^2/2}\, dt &=  \fr{(2m-1)!!}
 {2c^m}\,\sqrt{\fr{2\pi}{c}}\,,
    %    \label{e5.3-sin}
    \end{align*}
придем к~следующим степенным алгоритмам вы\-чис\-ле\-ния коэффициентов 
статистической линеаризации $\varphi_0^{j_n}, k_1^{j_n}$ 
и~$\varphi_0^{i_n}, k_1^{i_n}$:
\begin{align}
\varphi_0^{j_n} \left(m_y, D_y\right) &={}\notag\\
&\hspace*{-20mm}{}= \sss_{k=0}^\infty \fr{(-1)^k}{(2k)!! (2k+2n+1)!!}\,
    \varphi_0^{z^{2k+n}} \left(m_y, D_y\right)\,;\label{e5.4-sin}\\
\varphi_0^{i_n} (m_y, D_y) &={}\notag\\ 
&\hspace*{-20mm}{}=\sss_{k=0}^\infty 
    \fr{1}{(2k)!! (2k+2n+1)!!}\, \varphi_0^{z^{2k+n}} \left(m_y, D_y\right)\,;
    \label{e5.5-sin}
\end{align}

\begin{equation}
\left.
\begin{array}{rl}
k_1^{j_n} \left(m_y, D_y\right)& = 
\varphi_0^{zj_n} \left(m_y, D_y\right) -{}\\[6pt]
&{}- m_y \varphi_0^{j_n} \left(m_y, D_y\right) \enskip 
\left(m_y \ne 0\right)\,;\\[6pt]
k_1^{j_n} \left(0, D_y\right) &= \varphi_0^{zj_n} \left(0, D_y\right) 
 \enskip \left(m_y =0\right)\,;
 \end{array}
 \right\}
 \label{e5.6-sin}
 \end{equation}
 \begin{equation}
\left.
\begin{array}{rl}
k_1^{i_n} \left(m_y, D_y\right) &= 
\varphi_0^{zi_n} \left(m_y, D_y\right) -{}\\[6pt]
&{}- m_y \varphi_0^{i_n} 
\left(m_y, D_y\right) \enskip
\left(m_y \ne 0\right)\,;\\[6pt]
k_1^{i_n} \left(0, D_y\right) &= \varphi_0^{zi_n} \left(0, D_y\right)\enskip 
\left(m_y =0\right)\,.
\end{array}
\right\}
\label{e5.7-sin}
\end{equation}
Здесь введены обозначения:
    \begin{multline}
    \varphi_0^{z^h} \left(m_y, D_y\right) = 
    \fr{\exp (-m_y^2 /(2D_y))}{\sqrt{2\pi D_y}} \times{}\\
    {}\times
    \sss_{l=0}^\infty \fr{\lk 1+ (-1)^{h+l}\rk }{l!}\, A^{h+l} \left(D_y\right) 
    \left( \fr{m_y}{D_y}\right)^l\\ (h=1,2,\ldots)\,;\label{e5.8-sin}
    \end{multline}
    
    \vspace*{-12pt}
    
    \noindent
    \begin{equation}
    \left.
    \begin{array}{rl}
    A^{2h} \left(D_y\right) &= \displaystyle\fr{(2h-1)!!}{2}\, 
    D_y^h \sqrt{2\pi D_y}\,; \\[6pt] 
    A^{2h+1} (D_y) &= 
    \displaystyle
    \fr{h!}{2^h }\, D_y^{h+1}\enskip (h=1,2,\ldots)\,.
    \end{array}
    \right\}
    \label{e5.9-sin}
    \end{equation}

Таким образом, \textit{при $m_y \begin{matrix}
>\\  < \end{matrix} 0$, $D_y\hm>0$ в~основе степенного алгоритма 
вычисления коэффициентов статистической линеаризации сферической 
БН $j_n(z)$ лежат формулы}~(\ref{e5.4-sin}) \textit{и}~(\ref{e5.6-sin}) 
\textit{при условиях}~(\ref{e5.8-sin})  \textit{и}~(\ref{e5.9-sin}) (\textbf{теорема~1}).

\smallskip

 В основе \textbf{теоремы~2} \textit{для модифицированных сферических 
 нелинейностей $i_n(z)$ лежат формулы}~(\ref{e5.5-sin}) 
 \textit{и}~(\ref{e5.7-sin}) \textit{при условиях}~(\ref{e5.8-sin}) \textit{и} 
 (\ref{e5.9-sin}) \textit{и}~$m_y \begin{matrix} 
 >\\  <\end{matrix}  0$, $D_y \hm>0$.

\section{Коэффициенты статистической линеаризации сферических бесселевых 
нелинейностей~{\boldmath{$y_n (z)$}} и~модифицированных сферических 
нелинейностей~{\boldmath{$\tilde i_n (z)$}} второго~рода}

Используя степенные представления функций $y_n (z)$~[2--4]:
    \begin{multline*}
    y_n(z) = -\sss_{k=0}^n \fr{(2n- 2k-1)!!}
    {(2k)!!} z^{2k+n} -{}\\
    {}- \sss_{k=n}^\infty \fr{(-1)^k}{((2k)!!(2k-2n+1)!!)
     z^{2k+n}}\,, %\label{e6.1-sin}
     \end{multline*}
получим следующий алгоритм вычисления коэффициентов статистической 
линеаризации $\varphi_0^{y_n} (m_y, D_y)$ и~$k^{y_n} (m_y, D_y)$:
  \begin{multline}
  \varphi_0^{y_n} \left(m_y, D_y\right) = {}\\
  {}=
  -\sss_{k=0}^\infty \fr{(2n - 2k-1)!!}{(2k)!!}\,
     \varphi_0^{z^{2k+n}} \left(m_y, D_y\right)-{}\\
     {}-
     \sss_{k=n}^\infty \fr{(-1)^k}{(2k)!! (2k-2n+1)!!}\, 
     \varphi_0^{z^{2k+n}} \left(m_y, D_y\right)\,;
     \label{e6.2-sin}
     \end{multline}
    
%    \vspace*{-9pt}
    
    \noindent
    \begin{equation}
    \left.
    \begin{array}{l}
    k_1^{y_n} \left(m_y, D_y\right) = \varphi_0^{zy_n}  \left(m_y, D_y\right)- {}\\[6pt]
    {}-
    m_y \varphi_0^{y_n}  \left(m_y, D_y\right) = \fr{\prt \varphi_0^{y_n}}{ \prt m_y}
      \left(m_y, D_y\right) \\[6pt]   
        \hspace*{40mm}\left(m_y \ne 0\right)\,;\\[6pt]
    k_1^{y_n}  \left(0, D_y\right) = \varphi_0^{zy_n}  \left(m_y, D_y\right)\enskip 
    \left(m_y =0\right)\,.
    \end{array}
    \right\}
    \label{e6.3-sin}
    \end{equation}
Здесь $\varphi_0^{z^h}  (m_y, D_y)$ определены~(\ref{e5.8-sin})  и~(\ref{e5.9-sin}).

Аналогично имеем для~$\tilde i_n (z)$:
 \begin{multline*}
 \tilde i_n (z) = \sss_{k=0}^{n-1}
    \fr{(-1)^{n+k}(2n- 2k-1)!!}{(2k)!!} z^{2k-n-1} + {}\\
    {}+
    \sss_{k=n}^\infty \fr{(2k+1-2n)!!}{(2k)!!} z^{2k-n-1}\,; %\label{e6.4-sin}
    \end{multline*}
     
     \vspace*{-12pt}
     
\noindent
     \begin{multline}
     \varphi_0^{\tilde i_n} \left(m_y, D_y\right) = {}\\
     {}=
     \sss_{k=0}^{n-1} \fr{(-1)^{n+k}(2n - 2k)-1!!}{(2k)!!}\,
     \varphi_0^{z^{2k-n-1}} \left(m_y, D_y\right)+{}\\
    {}+\sss_{k=n}^\infty \fr{ (2k+1-2n)!!}{(2k)!!}\, \varphi_0^{z^{2k-n-1}} 
     \left(m_y, D_y\right)\,;\label{e6.5-sin}
     \end{multline}
     
     \vspace*{-12pt}
     
     \noindent
      \begin{equation}
      \left.
      \begin{array}{rl}
      k_1^{\tilde i_n}  \left(m_y, D_y\right) &= \varphi_0^{z\tilde i_n}  
      \left(m_y, D_y\right)-{}\\[6pt]
      &\hspace*{-5mm}{}- m_y  \varphi_0^{\tilde i_n}  \left(m_y, D_y\right)\enskip 
      \left(m_y \ne 0\right)\,;
     \\[6pt]
      k^{\tilde i_n} \left(0, D_y\right) &= \varphi^{z\tilde i_n} \left(0, D_y\right)\enskip 
      \left(m_y =0\right)\,.
      \end{array}
      \right\}
      \label{e6.6-sin}
      \end{equation}
Здесь $\varphi_0^{z^h} (m_y, D_y)$ определены~(\ref{e5.8-sin}) и~(\ref{e5.9-sin}).

\smallskip

В основе \textbf{теоремы~3} \textit{для алгоритма расчета коэффициентов статистической 
линеаризации сферической БН~$y_n (z)$ лежат формулы}~(\ref{e6.2-sin}) 
\textit{и}~(\ref{e6.3-sin}) \textit{при условиях}~(\ref{e5.8-sin}) \textit{и} 
(\ref{e5.9-sin}) \textit{и} $m_y \begin{matrix}
>\\ <\end{matrix} %\lessgtr
 0$, $D_y \hm>0$.
 
 \smallskip

\textit{Алгоритм расчета коэффициентов статистической линеаризации 
модифицированной сферической БН $\tilde i_n(z)$  
определяется формулами}~(\ref{e6.5-sin}) \textit{и}~(\ref{e6.6-sin}) 
\textit{при условиях}~(\ref{e5.8-sin}) \textit{и}~(\ref{e5.9-sin}) 
\textit{и} $m_y \begin{matrix}
>\\  <\end{matrix} %\lessgtr
 0$, $D_y \hm>0$ (\textbf{теорема~4}).

\section{Коэффициенты статистической линеаризации нелинейностей Эйри}

Применяя степенные разложения функций  $f(z)$ и~$g(z)$ в~(\ref{e2.8-sin}),
получим следующие алгоритмы вы\-чис\-ле\-ния коэффициентов статистической линеаризации:
  \begin{multline}
  \varphi_0^{\Ai} \left(m_y, D_y\right) = 
    c_1 \varphi_0^f \left(m_y, D_y\right) -{}\\
    {}- c_2 \varphi_0^g \left(m_y, D_y\right)\,;
    \label{e7.1-sin}
    \end{multline}
    
    \vspace*{-12pt}
    
    \noindent
    \begin{multline}
    \varphi_0^{\Bi} \left(m_y, D_y\right) = {}\\
    {}=
    \sqrt{3} \lk c_1 \varphi_0^f \left(m_y, D_y\right) +
     c_2 \varphi_0^g \left(m_y, D_y\right)\rk\,;\label{e7.2-sin}
     \end{multline}
     
%     \vspace*{-12pt}
     
     \noindent
     \begin{equation}
     \left.
     \begin{array}{l}
    k_1^{\Ai} \left(m_y, D_y\right) = \varphi_0^{z\Ai} \left(m_y, D_y\right) -{}\\[6pt]
    {}-
     m_y \varphi_0^{\Ai} \left(m_y, D_y\right) = 
     \fr{\prt \varphi_0^{\Ai}}{\prt m_y} \left(m_y, D_y\right)\\[6pt] 
    \hspace*{40mm} \left(m_y\ne 0\right)\,;\\[6pt]
         k_1^{\Ai} \left(0, D_y\right) = k_1^{z\Ai} \left(0, D_y\right)\,;
     \end{array}
     \right\}
     \label{e7.3-sin}
    \end{equation}
   
          \begin{equation}
          \left.
          \begin{array}{l}
     k_1^{\Bi} \left(m_y, D_y\right) = \varphi_0^{z\Bi} (m_y, D_y) - {}\\[6pt]
{}-     m_y \varphi_0^{\Bi} (m_y, D_y) = 
     \fr{\prt \varphi_0^{\Bi}}{\prt m_y} (m_y, D_y)\\[6pt]
     \hspace*{40mm} (m_y\ne 0)\,;\\[6pt]
    k_1^{\Bi} \left(0, D_y\right) = k_1^{z\Bi} (0, D_y)\,.
    \end{array}
    \right\}
    \label{e7.4-sin}
\end{equation}
Здесь
    \begin{equation}
    \left.
    \begin{array}{rl}
    \varphi_0^f \left(m_y, D_y\right) &= \displaystyle\sss_{l=0}^\infty 3^l 
    \left(\fr{1}{3}\right)_l 
    \fr{\varphi_0^{3l} (m_y, D_y)}{(3l)!}\,;\\[6pt]
    \varphi_0^g \left(m_y, D_y\right) &= 
\displaystyle    \sss_{l=0}^\infty 3^l \left(\fr{2}{3}\right)_l 
    \fr{\varphi_0^{3l+1} (m_y, D_y)}{(3l+1)!}\,.
    \end{array}
    \right\}
    \label{e7.5-sin}
    \end{equation}

Таким образом, \textit{в основе алгоритма расчета коэффициентов 
статистической линеаризации функции $\Ai(z)$ лежат формулы}~(\ref{e7.1-sin}), 
(\ref{e7.3-sin}) \textit{и}~(\ref{e7.5-sin}) \textit{при условиях}~(\ref{e5.8-sin}) 
\textit{и}~(\ref{e5.9-sin}) \textit{и} $m_y \begin{matrix}
>\\ < \end{matrix} 0,$%\lessgtr
$D_y \hm>0$ (\textbf{теорема~5}).
 
 \smallskip

\textit{Алгоритм расчета функции~$\Bi(z)$ определяется формулами}~(\ref{e7.2-sin}) 
\textit{и}~(\ref{e7.4-sin}) \textit{при условиях}~(\ref{e5.8-sin}) \textit{и}~(\ref{e5.9-sin}) 
\textit{и}~$m_y \begin{matrix}
> \\  <\end{matrix} %\lessgtr
 0$, $D_y \hm>0$ (\textbf{теорема~6}).


\section{Аналитическое моделирование нормальных процессов 
в~стохастических системах со~сложными бесселевыми нелинейностями дробного порядка}

Как известно~\cite{1-sin, 5-sin, 6-sin},  уравнения конечномерных непрерывных 
нелинейных систем со стохастическими возмущениями путем расширения вектора 
состояния СтС могут быть записаны в~виде сле\-ду\-юще\-го векторного стохастического 
дифференциального уравнения Ито:
    \begin{multline}
    dY_t = \displaystyle a\left(Y_t, t\right) dt + b \left(Y_t, t\right) dW_0+ {}\\
{}+\displaystyle \iii_{R_0} c 
    \left(Y_t, t, v\right) P^0 (dt, dv)\,,\enskip
    Y\left(t_0\right) = Y_0\,.
    \label{e8.1-sin}
    \end{multline}
Здесь $Y_t$~--- $(p\times 1)$-мер\-ный вектор состояния, 
$Y_t \hm\in \Delta_y$ ($\Delta_y$~--- многообразие состояний);  
$a\hm=a(Y_t, t)$ и~$b\hm=b(Y_t, t)$~--- известные  $(p\times 1)$-мер\-ная 
и~ $(p\times m)$-мер\-ная функции~$Y_t$ и~$t$;  $W_0\hm= W_0(t)$~--- 
$(r\times 1)$-мер\-ный винеровский стохастический процесс (СтП) интенсивности  
$\nu_0 \hm= \nu_0(t)$; $c(Y_t, t, v)$~--- $(p\times 1)$-мер\-ная функция~$Y_t, t$ 
и~вспомогательного $(q\times 1)$-мер\-но\-го параметра~$v$; 
$\iii_{\Delta}\, dP^0 (t, A)$~--- центрированная пуассоновская мера, определяемая
    $$
    \iii_{\Delta} \,dP^0 (t, A) = \iii_{\Delta} \,dP (t,A) =\iii_{\Delta} 
    \nu_P (t,A) \,dt\,.$$
При этом принято: $\iii_{\Delta}$~--- число скачков пуассоновского СтП
в интервале времени  $\Delta \hm= (t_1, t_2]$; $\nu_P (t, A)$~---
интенсивность пуассоновского СтП  $P(t,A)$; $A$~--- некоторое
борелевское множество пространства~$R_0^q$ с~выколотым началом.
Начальное значение~$Y_0$ представляет собой случайную величину, не зависящую 
от приращений $W_0(t)$ и~$P(t,A)$ на
интервалах времени, следующих за~$t_0$, $t_0 \hm\le t_1\hm\le t_2$, для
любого множества~$A$. Элементы век\-тор\-но-мат\-рич\-ных функций $a(Y_t,t)$, 
$b(Y_t,t)$ и~$c(Y_t,t,v)$ являются СБН дробного порядка.


Для аддитивных гауссовских (нормальных) и~обобщенных 
пуассоновских возмущений уравнение~(\ref{e8.1-sin}) принимает  вид~\cite{1-sin, 5-sin, 6-sin}:
\begin{equation}
\dot Y = a\left(Y_t,t\right)+ b_0 (t) V\,; \enskip 
    V = \dot W\,;\enskip 
    Y\left(t_0\right) = Y_0\,.
    \label{e8.2-sin}
    \end{equation}
Здесь $W$~--- СтП с~независимыми приращениями, представляющий собой смесь 
нормального и~обобщенного пуассоновского СтП.


Если предположить существование конечных вероятностных моментов второго 
порядка для моментов времени~$t_1$ и~$t_2$, то уравнения МНА примут следующий 
вид~\cite{1-sin, 5-sin, 6-sin}:
\begin{itemize}
\item для характеристических функций:
    \begin{equation}
    \left.
    \begin{array}{l}
    g_1^N (\la;t) \displaystyle=\exp \lk i\la^{\mathrm{T}} m_t - 
    \fr{1}{2}\, \la^{\mathrm{T}} K_t \la\rk;\\[6pt]
    g_{t_1, t_2}^N     \left(\la_1, \la_2;t_1, t_2 \right)={}\\[6pt]
     \hspace*{10mm}\displaystyle{} =
    \exp \lk i\bar \la^{\mathrm{T}} \bar m_2 - 
    \fr{1}{2}\,\bar \la^{\mathrm{T}} \bar K_2 \la\rk\,,
    \end{array}
    \right\}
    \label{e8.3-sin}
    \end{equation}
    где
    \begin{gather*}
    \bar \la =\lk \la_1^{\mathrm{T}}\la_2^{\mathrm{T}}\rk^{\mathrm{T}}\,; \enskip
    \bar m_2 = \lk m_{t_1}^{\mathrm{T}} m_{t_2}^{\mathrm{T}}\rk^{\mathrm{T}}\,;\\[9pt] 
    \bar K_2= \begin{bmatrix}
    K(t_1, t_1)& K(t_1, t_2)\\
    K(t_2, t_1)& K(t_2, t_2)\end{bmatrix}\,;
   \end{gather*}

\item  для математических ожиданий~$m_t$, ковариационной матрицы~$K_t$ и~матрицы 
ковариационных функций $K(t_1, t_2)$:
    \begin{equation}
    \left.
    \hspace*{-3mm}\begin{array}{l}
    \dot m_t = a_1 \left(m_t, K_t, t\right)\,,\enskip m_0 = m\left(t_0\right)\,;
   \\[6pt]
    \dot K_t = a_2 \left(m_t, K_t, t\right)\,,\enskip K_0 = K\left(t_0\right)\,;
   \\[6pt]
    \fr{\prt K(t_1, t_2)}{\prt t_2 }= K\left(t_1, t_2\right) 
    a_{21} \left(m_{t_2}, K_{t_2}, t_2\right)^{\mathrm{T}}\!,\\[6pt] 
    \hspace*{36mm}K\left(t_1, t_1\right) = K_{t_1}\,.
    \end{array}\!
    \right\}\!
    \label{e8.4-sin}
    \end{equation}
    \end{itemize}
Здесь приняты следующие обозначения:
    $$
    m_t = {\sf M}_{\Delta_y}^N \lk Y_t\rk \,;\enskip 
    Y_t^0 = Y_t - m_t\,;$$
    $$
    K_t = {\sf M}_{\Delta_y}^N \lk Y_t^0 Y_t^{0\mathrm{T}}\rk \,;\enskip 
    K\left(t_1, t_2\right) = {\sf M}_{\Delta_y}^N \lk Y_{t_1}^{0} Y_{t_2}^{0\mathrm{T}}\rk \,;
    $$
    $$
    a_1 = a_1\left(m_t, K_t, t\right) = {\sf M}_{\Delta_y}^N \lk a \left(Y_t, t\right)\rk\,;
    $$

\vspace*{-12pt}

\noindent
   \begin{multline*}
    a_2 = a_2 \left(m_t, K_t, t\right) = {}\\
    {}=a_{21} \left(m_t, K_t, t\right)+ 
    a_{21} \left(m_t, K_t, t\right)^{\mathrm{T}} +a_{22}\left( m_t, K_t, t\right)\,;
    \end{multline*}
    $$
    a_{21} = a_{21}\left(m_t, K_t, t\right)=  
    {\sf M}_{\Delta_y}^N \lk a\left(Y_t, t\right) Y_{t}^{0\mathrm{T}}\rk \,;
    $$
    $$
    a_{22} = a_{22}\left(m_t, K_t, t\right)= 
    {\sf M}_{\Delta_y}^N\lk  \bar\sigma \left(Y_t, t\right)\rk\,;
    $$
    $$
    \sigma \left(Y_t, t\right)=
    b\left(Y_t, t\right) \nu_0(t) b\left(Y_t, t\right)^{\mathrm{T}} \,;
    $$
    
    \vspace*{-12pt}
    
    \noindent
    \begin{multline}
    \bar\sigma \left(Y_t, t\right) = {}\\
    {}=\sigma \left(Y_t, t\right)+
    \iii_{R_0^q} c \left(Y_t, t, v\right) c\left(Y_t, t,v\right)^{\mathrm{T}} 
    \nu_P (t, dv)\,,\label{e8.5-sin}
    \end{multline}
где ${\sf M}_{\Delta_y}^N$~--- символ вычисления математического ожидания 
для нормальных распределений~(\ref{e8.3-sin}) на гладком многообразии~$\Delta_y$.

Для стационарных СтС нормальные стационарные СтП~--- если они
существуют, то  $m_t \hm= m^*$, $ K_t \hm= K^*$, $K(t_1, t_2)\hm =
k(\tau)$ $(\tau \hm= t_1\hm-t_2)$~--- определяются уравнениями~[1, 5, 6]:
  \begin{equation}
    \left.
    \begin{array}{l}
    a_1 \left(m^*, K^*\right) =0\,;\enskip a_2 \left(m^*, K^*\right)=0\,;
    \\[6pt]
       \dot k_\tau (\tau) = a_{21} \left(m^*, K^*\right) 
    {K^*}^{-1} k(\tau)\,;\\[6pt] 
    k(0) =K^* \enskip (\forall \tau >0)\,; \\[6pt] 
    k(\tau) = k(-\tau)^{\mathrm{T}} \enskip (\forall\tau <0)\,.
    \end{array}
    \right\}
    \label{e8.6-sin}
    \end{equation}
При этом необходимо, чтобы матрица  $a_{21} ( m^*, K^*)\hm= a^*_{21}$
 была асимптотически устойчивой.

Уравнения МНА в~случае СтС~(\ref{e8.2-sin}) переходят в~уравнения МСЛ~\cite{1-sin, 5-sin, 6-sin}:
     \begin{equation}
     \left.
     \hspace*{-3mm}\begin{array}{l}
     \dot m_t = a_1 \left(m_t, K_t, t\right)\,,\enskip 
     m_0 = m\left(t_0\right)\,;\\[6pt]
     \dot K_t = k_1^a \left(m_t, K_t, t\right) K_t + 
     K_t k_1^a \left(m_t, K_t, t\right)^{\mathrm{T}} +{}\\[6pt]
\hspace*{28mm}{}+\sigma_0(t)\,,\enskip 
     K_0 = K\left(t_0\right)\,;
     \\[6pt]
     \fr{\prt K(t_1, t_2)}{\prt t_2} = K\left(t_1, t_2\right) K_{t_2} 
   k_1^a  \left( m_{t_2}, K_{t_2}, t_2\right)^{\mathrm{T}}\,,\\[6pt] 
\hspace*{39mm}K\left(t_1, t_2\right) = K_{t_1}\,,
     \end{array}
     \right\}
     \label{e8.7-sin}
     \end{equation}
где
    $$
    a\left(Y_t,t\right) = a_0 \left(m_t, K_t\right) + 
    k_1^a \left(m_t, K_t\right) Y_t^0\,;$$
    $$ 
    k_1^a \left(m_t, K_t, t\right) =\lk \left(\fr{\prt}{\prt m_t} \right)
    a_0 \left(m_t, K_t, t\right)^{\mathrm{T}}\rk^{\mathrm{T}}\,;
    $$
    $$
    b\left(Y_t,t\right) = b_0 (t)\,;\enskip 
    \sigma\left(Y_t, t\right)= b_0(t) \nu(t) b_0(t)^{\mathrm{T}} = \sigma_0(t)\,.
    $$

Для стационарных СтС~(\ref{e8.2-sin}) при условии асимптотической устойчивости 
матрицы $k_1^a (m^*, K^*)$ в~основе МСЛ лежат уравнения~(\ref{e8.6-sin}), 
записанные в~виде:
  \begin{equation}
    \left.
    \hspace*{-2.6mm}\begin{array}{c}
    a_0 \left(m^*, K^*\right) =0\,;\\[6pt]
    k_1^a \left( m^*,  K^*\right)  K^* +  K^* k_1^a \left( m^*,  K^*\right)^{\mathrm{T}} +
    \bar \sigma_0 =0\,;
  \\[6pt]
    \dot k_\tau (\tau) = k_1^a \left( m^*, K^*\right)k(\tau)\,;\\[6pt] 
    k(0) =K^* \enskip (\forall \tau >0)\,;\\[6pt] 
    k(\tau) = k (-\tau)^{\mathrm{T}} \enskip (\forall \tau <0)\,.
    \end{array}
    \right\}\!
    \label{e8.8-sin}
    \end{equation}

\noindent
\textbf{Теорема~7.} \textit{Если существуют интегралы}~(\ref{e8.5-sin}), 
\textit{то уравнения}~(\ref{e8.3-sin}) \textit{и}~(\ref{e8.4-sin}) 
\textit{лежат в~основе нестационарных алгоритмов МАМ для негауссовских 
СтС}~(\ref{e8.1-sin}), \textit{а~уравнения}~(\ref{e8.7-sin})~---  
\textit{для негауссовских  СтС}~(\ref{e8.2-sin}).

\smallskip

\noindent
\textbf{Теорема~8.} \textit{Если СтС}~(\ref{e8.1-sin}) \textit{и}~(\ref{e8.2-sin}) 
\textit{стационарны и~существует стационарный нормальный процесс и~мат\-ри\-ца~$a_{21}^*$ 
асимптотически устойчива, то уравнения}~(\ref{e8.6-sin}) \textit{и}~(\ref{e8.8-sin}) 
\textit{лежат в~основе стационарных алгоритмов МАМ}.

\smallskip

Для гауссовских СтС теоремы~7 и~8 упрощаются, если принять  $c(Y_t, t, v)\hm\equiv 0$ 
в~(\ref{e8.1-sin}) и~$V\hm=V_0$, $\nu^V \hm= v^{V_0}$ в~(\ref{e8.2-sin}).

Для алгоритмизации МНА необходимо уметь вычислять следующие интегралы:
\begin{equation}
\left.
\begin{array}{rl}
I_0^a &= I_0^a \left(m_t, K_t, t\right) = 
    a_1 \left(m_t, K_t, t\right)={}\\[6pt]
    &\hspace*{28mm}{}={\sf M}_{\Delta_y}^N \lk a\left(Y_t, t\right)\rk\,;\\[9pt]
I_1^a &= I_1^a \left(m_t, K_t, t\right)= 
    a_{21}\left(m_t, K_t, t\right)= {}\\[6pt]
&    \hspace*{20mm}{}={\sf M}_{\Delta_y}^N \lk a\left(Y_t , t\right) 
    Y_t^{0\mathrm{T}}\rk\,;
    \\[9pt]
  I_0^{\bar \sigma} &= I_0^{\bar \sigma} \left(m_t, K_t, t\right) = a_{22}
  \left(m_t, K_t, t\right) ={}\\[6pt]
&  \hspace*{29mm}{}= {\sf M}_N \lk \bar \sigma\left(Y_t, t\right)\rk\,,
\end{array}
\right\}
  \label{e8.9-sin}
  \end{equation}
а для МСЛ достаточно вычислить первый интеграл в~(\ref{e8.9-sin}), 
причем интеграл~$I_1^a$ вычисляется по формуле~\cite{1-sin, 5-sin, 6-sin}
    \begin{equation*}
    k_1^a = k_1^a \left(m_t, K_t, t\right)=\lk \left( \fr{\prt}{\prt m_t}\right) 
    I_0^a \left(m_t, K_t, t\right)^{\mathrm{T}}\rk^{\mathrm{T}}\,.
%    \label{e8.10-sin}
    \end{equation*}

Важно иметь в~виду, что уравнения МНА (МСЛ) содержат интегралы 
$I_0^a$, $I_1^a$ и~$I_0^\sigma$ в~виде соответствующих коэффициентов. 
Поэтому процедура вычисления интегралов должна быть согласована с~методом 
численного решения обыкновенных дифференциальных уравнений для  $m_t, K_t$ 
и~$K(t_1, t_2)$. Эти коэффициенты допускают дифференцирование по~$m_t$ 
и~$K_t$, так как под интегралом стоит сглаживающая нормальная плот\-ность.


В~\cite{8-sin} изложены алгоритмы дискретного аналитического и~статистического
моделирования типовых распределений (в том числе нормальных) в~нелинейных
СтС на многообразиях. Алгоритмы дискретного аналитического и~статистического
моделирования для СтС с~СБН, а~также смешанные алгоритмы различной
степени точности относительно шага интегрирования также представлены в~\cite{8-sin}.

В приложении~П3  приведены тестовые при\-меры.

\section{Заключение}

На основе МНА (МСЛ) разработано методическое и~алгоритмическое (степенное) 
обеспечение аналитического моделирования нормальных 
процессов в~гауссовских и~негауссовских СтС с~\mbox{БНДП} (сферическими, модифицированными 
сферическими и~нелинейностеями Эйри).

Алгоритмы положены в~основу разрабатываемого инструментального программного 
обеспечения для решения задач надежности и~безопасности сис\-тем и~средств 
информатики и~управления.

Алгоритмы допускают обобщение на случай аппроксимации 
БНДП многочленными, дроб\-но-ра\-ци\-о\-наль\-ны\-ми и~другими представлениями.

Теоретический и~практический интерес представляют символьные алгоритмы.

\vspace*{12pt}


{\small

%\setcounter{equation}{0}

 \section*{\raggedleft Приложения}

 \vspace*{-6pt}


\subsection*{П1.\ Функции Бесселя первого \hphantom{П1.\ }и~второго~рода~[2--4]}


    $$
    J_\nu (z) = \sss_{k=0}^\infty \fr{(-1)^k}{k! \Gamma(k+\nu+1)} 
    \left( \fr{z}{2}\right)^{2k+\nu} \enskip (\nu \ne \pm n)\,;
    $$
    $$
    Y_\nu (z) =  \fr{1}{\sin \nu \pi} \lk J_\nu (z) \cos \nu \pi - J_{-\nu} (z)\rk 
    \enskip ( \nu \ne \pm n)\,;
    $$
   
   \vspace*{-12pt}
   
   \noindent
   \begin{multline*}
    \!Y_n (z) = \fr{2}{\pi}\, J_n(z) \ln \fr{z}{2} - 
    \fr{1}{\pi} \sss_{k=0}^{n-1} \fr{(n-k-1)!}{k!} \left( \fr{z}{2}\right)^{2k-n}\!\!\! -{}\\
\!{}-\fr{1}{\pi} \sss_{k=0}^\infty \fr{(-1)^k}{(n+k)!k!} \lk \psi(n+k+1) + 
\psi(k+1) \rk \left( \fr{z}{2}\right)^{2k+n},
\end{multline*}
где $\psi(l) = \Gamma'(l)/\Gamma (l) = (d/dl) \ln \Gamma(l)$;
    $$
    J_{-n} (z) = (-1)^n J_n (z)\,,\enskip Y_{-n} (z) = (-1)^n Y_n (z)\,;
    $$
    $$
    I_\nu (z) = \sss_{k=0}^\infty \fr{1}{k! \Gamma (k+\nu+1)} 
    \left(\fr{z}{2}\right)^{2k+\nu}\,;
    $$
    $$
    K_\nu (z) = \fr{\pi}{2 \sin \nu \pi} \lk I_{-\nu} (z) -
     I_\nu (z)\rk \enskip (\nu \ne \pm n)\,;
     $$

\vspace*{-10pt}

\noindent
\begin{multline*}
    K_n(z) ={}\\
    {}=(-1)^{n+1} I_n (z) \ln \fr{z}{2} + \fr{1}{2} 
    \sss_{k=0}^{n-1} \fr{(n-k-1)!}{k!} \left( \fr{z}{2}\right)^{2k-n} +{}\\
    {}+\fr{(-1)^n}{2} \sss_{k=0}^\infty \fr{\psi(k+n+1) + \psi(k+1)}{(n+k)! k!} 
    \left( \fr{z}{2}\right)^{n+2k}\,;
    \end{multline*}
    
%    \vspace*{-10pt}
    
    \noindent
    \begin{multline*}
J_{\pm n\pm (1/2)} (z) ={}\\
{}=(\mp)^n \sqrt{\fr{2}{\pi}}\, z^{n+(1/2)} \left( 
\fr{1}{z} \,\fr{d}{dz}\right)^n \left(\fr{1}{z} \lf{
\fr\sin z}{\cos z}\rf\right)={}\\
{}=\sqrt{\fr{2}{\pi z}} \left[  \lf \begin{matrix}
    \sin \left(z- \fr{n\pi}{2}\right)\\
    \cos \left(z+ \fr{n\pi}{2}\right)\end{matrix}\rf 
    \sss_{k=0}^{\lk n/2\rk} \fr{(-1)^k (n+2k)!}{(2k)! (n-2k)! (2z)^{2k}} \pm  \right.\\
\left. \pm \lf \begin{matrix}
    \cos \left(z- \fr{n\pi}{2}\right)\\
    \sin \left(z+ \fr{n\pi}{2}\right)\end{matrix}\rf \times{}\right.\\
\left.{}\times \sss_{k=0}^{\lk (n-1)/2\rk} \hspace*{-1mm}
    \fr{(-1)^k (n+2k+1)!}{(2k+1)! (n-2k-1)! (2z)^{2k+1}}\right];
    \end{multline*}

\noindent
$$
J_{\pm (1/2)} (z) = \sqrt{\fr{2}{\pi z}} \lf \begin{matrix}
    \sin z\\
    \cos z\end{matrix}\rf\,;
    $$
    $$
    J_{\pm (3/2)} (z) =\sqrt{\fr{2}{\pi z}}\lk \pm \fr{1}{z} \lf \begin{matrix}
    \sin z\\
    \cos z\end{matrix}\rf- \lf \begin{matrix}
    \cos z\\
    \sin z\cr\end{matrix}\rf \rk\,;
    $$
    $$
    J_{\pm (5/2)} (z) =\sqrt{\fr{2}{\pi z}}\lk \left( \fr{3}{z^2}-1 \right)\lf 
    \begin{matrix}
    \sin z\\
    \cos z\end{matrix}\rf\mp \fr{3}{z} \lf \begin{matrix}
    \cos z\\
    \sin z\end{matrix}\rf \rk\,;
    $$
    $$
    Y_{\pm n\pm 1/2} (z) =\mp (-1)^n J_{\mp n\mp 1/2} (z)\,;$$
    $$ 
    Y_{\pm 1/2} (z) = \mp \sqrt{\fr{2}{\pi z}} \lf \begin{matrix}
    \cos z\\
    \sin z\end{matrix}\rf \,;
    $$
    $$
    I_{-n} (z)= I_n (z)\,;\enskip I_n (z) = (-1)^n I_n (z)\,;
    $$
    
    \vspace*{-10pt}
    
    \noindent
    \begin{multline*}
    I_{\pm n \pm 1/2} (z) = z^n \left( \fr{1}{z}\,\fr{d}{dz}\right)^n \lk 
    \fr{1}{z} \begin{pmatrix}
    \mathrm{sh}\, z\\
    \mathrm{ch}\, z\end{pmatrix}\rk ={}\\
{}=\fr{1}{\sqrt{2\pi z}} \left[ e^z \sss_{k=0}^n \fr{(-1)^k (n+k)!}{k! (n-k)! (2z)^k} 
\pm {}\right.\\
\left.{}+\pm(-1)^{n+1} e^{-z} \sss_{k=0}^n \fr{(n+k)!}{k! (n-k)! (2z)^k}\right]\,;
\end{multline*}

\noindent
    $$
    I_{\pm 1/2} (z) = \sqrt{\fr{2}{\pi z}} \left\{\begin{matrix}
    \mathrm{sh}\, z\\
\mathrm{ch}\, z\end{matrix}\right\}\,; 
 $$
 $$
 I_{\pm 3/2} (z) = \sqrt{\fr{2}{\pi z}}\lk \left\{\begin{matrix}
    \mathrm{ch}\, z\\
    \mathrm{sh}\, z\end{matrix} \right\} \mp \fr{1}{z} 
    \left\{\begin{matrix}
    \mathrm{sh}\, z\\
    \mathrm{ch}\, z\end{matrix} \right\}\rk\,;
    $$
$$
I_{\pm 5/2} (z) = \sqrt{\fr{2}{\pi z}}\lk\left(\fr{3}{2z^2}+1\right) \left\{\begin{matrix}
    \mathrm{sh}\, z\\
    \mathrm{ch}\, z\end{matrix} \right\} - \fr{3}{z} \left\{\begin{matrix}
    \mathrm{ch}\, z\\
    \mathrm{sh}\, z\end{matrix} \right\}\rk\,;
    $$
    $$
    K_{n+1/2}(z) = \sqrt{\fr{\pi}{2 z}}\,e^{-z} \sss_{k=0}^n 
    \fr{(n+k)!}{k!(n-k)! (2z)^k}\,;
    $$
    $$
    K_{\pm 1/2} (z) = \sqrt{\fr{\pi}{2 z}}\,e^{-z}\,;\enskip 
    K_{\pm 3/2} (z) = \fr{\pi}{2z} \left(1+\fr{1}{z}\right) e^{-z}\,.
    $$
    
    


\subsection*{П2.\ Некоторые свойства функции \hphantom{П2.\ }Бесселя дробного порядка} 


{Формула Рэлея:}
    $$
    j_n (z) = z^n \left(-\fr{1}{z}\,\fr{d}{dz}\right)^n \fr{\sin^z}{z}\,;$$
    $$
    y_n (z) = -z^n \left(-\fr{1}{z}\,\fr{d}{dz}\right)^n \fr{\cos^2}{z}\,;
    $$
    $$
    i_n (z) = z^n \left(\fr{1}{z}\,\fr{d}{dz}\right)^n \fr{\mathrm{sh}\, z}{z}\,;
    $$
    $$
    \tilde i_n (z) = z^n \left(\fr{1}{z}\,\fr{d}{dz}\right)^n 
    \fr{\mathrm{ch}\, z}{z}\,.
    $$

{Производные $w_n$: $j_n, i_n, y_n, \tilde i_n$:}
    \begin{align*}
    \left(\fr{1}{z}\,\fr{d}{dz}\right)^m \lk z^{n+1} w_n (z)\rk &= 
    z^{n-m+1} w_{n-m} (z)\,;\\
    \left(\fr{1}{z}\, \fr{d}{dz}\right)^m \lk z^{-n} w_n (z)\rk& =
    (\pm)^m z^{-n-m} w_{n+m} (z)
    \\
    &\hspace*{-10mm}(n=0,\pm 1,\pm 2,\ldots,\enskip m= 1, 2,\ldots)\,.
   \end{align*}
Знак минус относится к~$j_n$ и~$y_n$, а~плюс и~минус~--- к~$i_n$.

{Основные соотношения:}
    $$
    y_n(z) = (-1)^{n+1} j_{-n-1} (z) \enskip (n=0, \pm 1, \pm 2,\ldots)\,;
    $$
    
\vspace*{-10pt}

\noindent
\begin{multline*}
    \Ai (z) = \fr{1}{3}\, \sqrt{z} \lk I_{-1/3} (b) - 
    I_{1/3}(b)\rk ={}\\
    {}= \fr{1}{\pi}\, \sqrt{\fr{z}{3}} K_{1/3} (b)\enskip 
    \left( b= \fr{2}{3} z^{3/2}\right)\,;
    \end{multline*}
    
    \noindent
    $$
    \Ai (-z) = \fr{1}{3}\, \sqrt{z} \lk J_{1/3} (b) + J_{-1/3}(b)\rk\,;
    $$
  $$
  \Bi (z) =  \sqrt{\fr{z}{3}} \lk J_{-1/3} (b) + J_{1/3}(b)\rk\,;
  $$
  $$
  \Bi (-z) =  \sqrt{\fr{z}{3}} \lk J_{-1/3} (b) - J_{1/3}(b)\rk\,;
  $$
    $$
    j_0(z) = \fr{\sin z}{z}\,;\enskip 
    j_1(z) = \fr{\sin z}{z^2}- \fr{\cos z}{z}\,;
    $$
    $$
    j_2(z) = \left(\fr{3}{z^3}-\fr{1}{z}\right) \sin z - \fr{3}{z^2}\, \cos z\,;
    $$
  $$
  y_0(z) = -j_{-1} (z) = -\fr{\cos z}{z}\,;
  $$
  $$ 
  y_1 (z) = j_{-2}(z) = - \fr{\cos z}{z^2} - \fr{\sin z}{z}\,;
  $$
  $$
  y_2(z) =-j_{-3}(z)= \left(-\fr{3}{z^3}+\fr{1}{z}\right) \cos z - 
  \fr{3}{z^2} \sin z\,;
  $$
    $$
    i_0 (z) = \fr{\mathrm{sh}\, z}{z}\,; \enskip 
    i_1 (z) = -\fr{\mathrm{sh}\, z}{z^2}+ \fr{\mathrm{ch}\, z}{z}\,;
    $$
    $$
    i_2 (z) = \left(\fr{3}{z^3} + \fr{1}{z}\right)\mathrm{sh}\, z - 
    \fr{3}{z^2}{\mathrm{ch}}\, z\,;
    $$
    $$
    \tilde i_0 (z) = \fr{\mathrm{ch}\, z}{z}\,;\enskip 
    \tilde i_1(z) = \fr{\mathrm{sh}\, z}{z} -\fr{\mathrm{ch}\, z}{z}\,;
    $$
    $$
    \tilde  i_2 (z) = -\fr{3}{z^2} \,\mathrm{sh}\, z + \left(
    \fr{3}{z^3} + \fr{1}{z}\right) \mathrm{ch}^2\, z\,.
    $$

Рекуррентные соотношения $w_n(z):\, j_n (z), y_n (z)$:
    $$
    w_{n-1} (z) + w_{n+1} (z)= \fr{2n+1}{z}\, w_n (z)\,;
    $$
    $$
    nw_{n-1} (z) -(n+1) w_{n+1} (z)= (2n+1)\fr{d}{dz}\, w_n (z)\,;
    $$
    $$
    \fr{n+1}{z}\, w_n (z) + \fr{d}{dz}\, w_n (z) = w_{n-1}(z)\,;
    $$
     $$
     \fr{n}{z}\, w_{n-1} (z) -\fr{d}{dz}\, w_n (z) = w_{n+1} (z)\,;
     $$
     $$
     i_{n-1} (z) - i_{n+1} (z) = \fr{(2n+1)}{z}\, w_n\,;
     $$
     $$
     n i_{n-1}(z) + (n+1) i_{n+1} (z) = (2n +1) \fr{d}{dz}\, i_n (z)\,;
     $$
    $$
    \fr{n+1}{z}\, i_n (z) +\fr{d}{dz}\,i_n(z) = i_{n-1} (z)\,;
     $$
    $$
    -\fr{n}{z}\, i_n(z) +\fr{d}{dz}\, i_n(z) = i_{n+1}(z)\,.
    $$

Точность вычислений может быть проверена как с~по\-мощью тождеств
    \begin{gather*}
    j_n (z) y_{n-1}(z) - j_{n-1}(z) y_n(z) = z^{-2}\,;
    \\[2pt]
    y_{n+1}(z) y_{n-1}(z) - j_{n-1}(z) y_{n+1}(z) =(2n+1) z^{-2}\,;
\\[2pt]
    i_n(z) \tilde i_n (z) - i_{n-1}(z) \tilde i_{n-1}(z) =(-1)^{n+1} z^{-2}\,,
  \end{gather*}
так и~нахождением нулей~[3--6].

\vspace*{2pt}


 
%  \vspace*{-6pt}


\subsection*{П3.\ Тестовые примеры}

\vspace*{6pt}


\noindent
\textbf{П3.1.}\ 
При $m_y =0$ для сложных нелинейностей вида
    \begin{align*}
    \varphi(Y) &= J_{1/2} (Y) 1(Y) = 1(Y)\sqrt{
    \fr{2}{\pi |Y|}}\sin Y\,;\\
    \varphi(Y) &= J_{-1/2} (Y) 1(Y) =1(Y)\sqrt{\fr{2}{\pi |Y|}}\cos Y\,,
    %\eqno(\hbox{П}3.1) 
\end{align*}
используя следующие известные табличные интегралы~[3--6]:
    \begin{multline*}
    \iii_0^\infty x^{\mu-1} e^{-\beta x^2} \sin  ax \,dx = {}\\
{}=    \fr{a e^{-a^2/4\beta}}{2^{\beta(\mu+1)/2}}\, \Gamma 
    \left(\fr{1+\mu}{2}\right) {}_1\!F_1\left(
    \fr{\mu}{2}; \fr{3}{2}; -\fr{a^2}{4\beta}\right)\\ 
    (\mathrm{Re}\, \beta>0\,,\  \mathrm{Re}\, \mu>-1)\,;
   % \eqno(\hbox{П}3.2)
   \end{multline*}
   
   \vspace*{-10pt}
   
   \noindent
\begin{multline*}
     \iii_0^\infty x^{\mu-1} e^{-\beta x^2} \cos a x\, dx = 
     \fr{ \Gamma \left({\mu/2}\right)}{2\beta^{\mu/2}} {}_1\!F_1\left(
     \fr{\mu}{2}; \fr{1}{2}; -\fr{a^2}{4\beta}\right)\\ 
     (\mathrm{Re}\, \beta>0\,,\ \mathrm{Re}\, \mu>0)\,,
    % \eqno(\hbox{П}3.3)
     \end{multline*}
при $\mu=1/2$ получаем точные выражения для  $\varphi_0 (0, D_y)$:

\columnbreak

\noindent
    $$
    \varphi_0 (0, D_y)= \fr{\Gamma \left({3/4}\right)}{(2 D_y)^{1/4}}\,
    e^{-D_y/2} {}_1\!F_1\left(\fr{1}{4}; \fr{3}{2};- \fr{D_y}{2}\right)\,;
    %\eqno(\hbox{П}3.4)
    $$
    $$
    \varphi_0 (0, D_y)= \fr{1}{2^{5/4} D_y^{1/4}}
      {}_1\!F_1\left(\fr{1}{4}; \fr{3}{2};- \fr{D_y}{2}\right)\,.
    %  \eqno(\hbox{П}3.5)
      $$
Здесь $ {}_1\!F_1\left(a; b; \xi\right)$~--- вырожденная гипергеометрическая 
функция~[3--6], допускающая представление:
   \begin{multline*}
    {}_1\!F_1\left(a; b; \xi\right) ={}\\
    {}= 1 + \fr{a}{b}\,\fr{\xi}{1!} + 
    \fr{a(a+1)}{b(b+1)} \,\fr{\xi^2}{2!} + 
    \fr{a(a+1)(a+2)}{b(b+1)(b+2)}\,\fr {\xi^3}{3!} +\cdots
 %   \eqno(\hbox{П}3.6)
    \end{multline*}

При $m_y \ne 0$ и~для $\mu=1/2$, если принять во внимание следующие 
табличные интегралы~[3--6]:
  \begin{multline*}
  \iii_0^\infty x^{\mu-1} e^{-\gamma x-\beta x^2} \sin a x \,dx = {}\\
  {}=
  -\fr{i}{2(2\beta)^{\mu/2}}\exp\left(
  \fr{\gamma^2-a^2}{8\beta}\right) \Gamma(\mu) \times{}\\
{}\times\left\{ \exp\left(-
\fr{ia\gamma}{4\beta}\right) D_{-\mu} \left( \fr{\gamma-ia}{\sqrt{2\beta}}\right)-{}\right.\\
\left.{}-
\exp\left(\fr{ia\gamma}{4\beta}\right) D_{-\mu} \left(
\fr{\gamma+ia}{\sqrt{2\beta}}\right)\right\}
%\eqno(\hbox{П}3.7)
\end{multline*}
      $$
      (i^2=-1\,,\enskip \mathrm{Re}\, \mu >-1\ ,\enskip \mathrm{Re}\, \beta>0,\enskip 
      a>0)\,;
      $$

\vspace*{-10pt}

\noindent
\begin{multline*}
\iii_0^\infty x^{\mu-1} e^{-\gamma x-\beta x^2} \cos  ax \,dx = {}\\
{}=
\fr{1}{2(2\beta)^{\mu/2}}\exp\left(
\fr{\gamma^2-a^2}{8\beta}\right) \Gamma(\mu) \times{}\\
{}\times\left\{ \exp\left(
-\fr{ia\gamma}{4\beta}\right) D_{-\mu} \left(
\fr{\gamma^2-ia}{\sqrt{2\beta}}\right)+{}\right.\\
\left.{}+\exp\left(
\fr{ia\gamma}{4\beta}\right) D_{-\mu} \left(
\fr{\gamma^2+ia}{\sqrt{2\beta}}\right)\right\}
      \\
       (\mathrm{Re}\, \mu >0\,,\ \mathrm{Re}\, \beta>0\,,\  a>0)\,,
      %\eqno(\hbox{П}3.8)
      \end{multline*}
имеем следующие точные выражения:
\begin{multline*}
\varphi_0 \left(m_y, D_y\right)= -
\fr{i(2\beta)^{1/4}}{2^{3/2}}\exp \left(-\fr{m_y^2}{2D_y}\right)\times{}\\
{}\times\left[ \exp\left( -\fr{i\gamma}{4\beta}\right)D_{-1/2} \left(
\fr{\gamma-i}{\sqrt{2\beta}}\right)-{}\right.\\
\left.{}- \exp\left( 
\fr{i\gamma}{4\beta}\right)D_{-1/2} \left(
\fr{\gamma+i}{\sqrt{2\beta}}\right)\right]\,;
%\labvel{p3.9}
\end{multline*}

\vspace*{-10pt}

\noindent
\begin{multline*}
\varphi_0 \left(m_y, D_y\right)= 
\fr{(2\beta)^{1/4}}{2^{3/2}}\exp \left(-\fr{m_y^2}{2D_y}\right)\times{}\\
{}\times\left[ \exp\left( -\fr{i\gamma}{4\beta}\right)D_{-1/2} \left(
\fr{\gamma-i}{\sqrt{2\beta}}\right)+{}\right.\\
\left.{}+ \exp\left( 
\fr{i\gamma}{4\beta}\right)D_{-1/2} \left(
\fr{\gamma+i}{\sqrt{2\beta}}\right)\right].
%\eqno(\hbox{П}3.10)
\end{multline*}
Здесь $D_\mu (\cdot )$~--- функция параболического цилиндра~[2--4];
    $$
    \gamma =\gamma \left(m_y, D_y\right) =- \fr{m_y}{D_y}\,;\enskip 
    \beta =\beta \left(D_y\right)= \fr{1}{2 D_y}\,.
    %\eqno(\hbox{П}3.11)
    $$


\noindent
\textbf{П3.2.}\ Для одномерной бесселевой системы с~аддитивным белым шумом
  $$
  \dot Y_t = \alpha +\beta w_n \left(cY_t\right) + \gamma V
  %\eqno(\hbox{П}3.12)
  $$
($\alpha$, $\beta$, $\gamma$ и~$c$~--- постоянные; 
$V$~--- гауссовский белый шум интенсивности~$\nu$, 
$w_n$: $j_n, i_n, y_n, \tilde i_n, \Ai, \Bi$) алгоритмы МАМ на основе 
МСЛ (теоремы~7 и~8) имеют  следующий вид:
    \begin{equation}
    \left.
    \begin{array}{rl}
    \dot m_y &= \alpha + \beta \varphi_0^{w_n} \left(m_y, D_y\right)\,;\\[6pt] 
    \dot D_y &= 2\beta k_1^{w_n} (m_y, D_y) + \sigma \enskip ( \sigma = \gamma^2 \nu)\,;
    \end{array}
    \right\}
    \label{p3.13}
    \end{equation}
    \begin{equation}
    \left.
    \begin{array}{rl}
    \alpha^* + \beta^* \varphi_0^{*w_n}\left(m^*, D^*\right) =0\,; \\[6pt] 
    2 \beta^* k_1^{*w_n} \left(m^*, D^*\right) D^* + \sigma^*=0 \enskip (\beta>0)\,.
    \end{array}
    \right\}
    \label{p3.14}
    \end{equation}
Коэффициенты $\varphi_0^{w_n}$ и~$\varphi_0^{*w_n}$  в~(\ref{p3.13}) и~(\ref{p3.14}) 
определяются согласно алгоритмам теорем~1--6.


\textbf{П3.3.}\
Для одномерной бесселевой системы с~мультипликативным гауссовским белым шумом
    $$
    \dot Y_t = \alp + \beta Y_t + \gamma w_n \left(cY_t\right) V
    %\eqno(\hbox{П}3.15)
    $$
алгоритм МАМ  согласно МНА определяется следующими уравнениями:
    $$
    \dot m_y = \alpha + \beta m_y \,;\enskip 
    \dot D_y = 2\beta D_y + \sigma \left( m_y, D_y\right)\,;
    %\eqno(\hbox{П}3.16)
    $$
    $$
    \alpha^* + \beta^* m^* =0\,;\enskip 
    2 \beta^* D^* + \sigma\left(m^*, D^*\right) =0 \enskip (\beta<0)\,.
    %\eqno(\hbox{П}3.17)
    $$
Здесь
    \begin{multline}
    \sigma \left(m_y, D_y\right) ={}\\
    {}= \fr{\gamma^2 \nu}{\sqrt{2\pi D_y}} 
    \iin w_n^2 (c\eta) e^{-(\eta-m_y)^2/(2D_y)} d\eta\,.
    \label{p3.18}
    \end{multline}
Интеграл~(\ref{p3.18}) вычисляют на основе теорем~1--6, 
если принять во внимание известную формулу для квадрата степенного ряда~\cite{3-sin}:
    $$
    \left( \sss_{h=0}^\infty a_h z^h\right)^2 = 
    \sss_{h=0}^\infty b_h z^h\,,
    $$
    $$
    b_0 = a_0\,,\enskip 
    b_h = \fr{1}{ha_0} \sss_{\rho=1}^h (3\rho - h) a_\rho b_{h-\rho}\,.
    $$
    
    }

{\small\frenchspacing
 {%\baselineskip=10.8pt
 \addcontentsline{toc}{section}{References}
 \begin{thebibliography}{9}


\bibitem{1-sin}
\Au{Синицын И.\,Н. }
Аналитическое моделирование процессов в~динамических системах с~цилиндрическими 
бесселевыми нелинейностями~// Информатика и~её применения, 2015. 
Т.~9. Вып.~4. С.~39--49.

\bibitem{2-sin}
\Au{Градштейн И.\,С., Рыжик~И.\,М. }
Таблицы интегралов, сумм, рядов и~произведений.~--- М.: ГИФМЛ, 1963. 1100~с.

\bibitem{3-sin}
Справочник по специальным функциям~/ Под ред. М.~Абрамовича, И.~Стигана.~--- 
М.: Наука, 1979. 832~с.

\bibitem{4-sin}
\Au{Попов Б.\,А., Теслер~Г.\,С.}
Вычисление функций на ЭВМ: Справочник.~--- Киев: Наукова Думка, 1984. 599~с.

\bibitem{5-sin} 
 \Au{Пугачёв В.\,С., Синицын~И.\,Н.}
Стохастические дифференциальные системы. Анализ и~фильтрация.~--- М.:
Наука,  1990.  632~с. 
%(Англ. пер. \Au{Pugachev~V.\,S., Sinitsyn~I.\,N.} 
%Stochastic differential systems.
%Analysis and filtering.~--- Chichester, New York: Jonh Wiley, 1987.
%549~p.)

\bibitem{6-sin}
\Au{Пугачёв В.\,С., Синицын И.\,Н.}
Теория стохастических систем.~--- М.: Логос, 2000; 2004. 1000~с.
%[Англ. пер. Stochastic Systems. Theory and  Applications. --
%Singapore: World Scientific, 2001. 908~p.].

\bibitem{7-sin}
\Au{Синицын И.\,Н.,  Синицын~В.\,И. }
Лекции по нормальной и~эллипсоидальной аппроксимации распределений 
в~стохастических системах.~--- М.: ТOРУС ПРЕСС, 2013. 488~с.

\bibitem{8-sin}
\Au{Синицын И.\,Н. }
Параметрическое статистическое и~аналитическое моделирование распределений 
в~нелинейных стохастических системах на многообразиях~// Информатика 
и~её применения, 2013. Т.~7. Вып.~2. С.~4--16.
\end{thebibliography}

 }
 }

\end{multicols}

\vspace*{-3pt}

\hfill{\small\textit{Поступила в~редакцию 22.03.16}}

\vspace*{7pt}

%\newpage

%\vspace*{-24pt}

\hrule

\vspace*{2pt}

\hrule

%\vspace*{8pt}



\def\tit{ANALYTICAL MODELING OF~PROCESSES IN~STOCHASTIC SYSTEMS WITH~COMPLEX
FRACTIONAL ORDER BESSEL NONLINEARITIES}

\def\titkol{Analytical modeling of~processes in~stochastic systems with~complex
fractional order bessel nonlinearities}

\def\aut{I.\,N.~Sinitsyn}

\def\autkol{I.\,N.~Sinitsyn}

\titel{\tit}{\aut}{\autkol}{\titkol}

\vspace*{-9pt}

\noindent
Institute of Informatics Problems, Federal Research Center 
``Computer Science and Control'' of the Russian\linebreak
Academy of Sciences,
44-2~Vavilov Str., Moscow 119333, Russian Federation


\def\leftfootline{\small{\textbf{\thepage}
\hfill INFORMATIKA I EE PRIMENENIYA~--- INFORMATICS AND
APPLICATIONS\ \ \ 2016\ \ \ volume~10\ \ \ issue\ 3}
}%
 \def\rightfootline{\small{INFORMATIKA I EE PRIMENENIYA~---
INFORMATICS AND APPLICATIONS\ \ \ 2016\ \ \ volume~10\ \ \ issue\ 3
\hfill \textbf{\thepage}}}

\vspace*{3pt}


\Abste{Methods of analytical modeling for normal (Gaussian) processes in Gaussian 
and non-Gaussian stochastic  systems  with complex fractional order Bessel 
nonlinearities (spherical, modificated spherical, and Airy) are
developed. 
Necessary information about Bessel fractional order functions 
is given. Coefficients of statistical\linebreak\vspace*{-12pt}}

\Abstend{linearization for typical fractional order 
Bessel nonlinearities are presented. Special attention is paid to the series 
algorithms. Analytical modeling algorithms have been developed for 
nonstationary and stationary normal processes. 
Test examples are presented. Main conclusions and generalizations are mentioned.}


\KWE{Airy nonlinearity; Bessel function of fractional order;
Bessel nonlinearity; method of analytical modeling; 
method of normal approximation; method of statistical linearization;
modificated spherical Bessel function; normal (Gaussian) process;
spherical Bessel function; stochastic process}


\DOI{10.14357/19922264160308} 

\vspace*{-9pt}

\Ack
\noindent
The work was supported by the Department for Nanotechnologies and Information Technologies
(ONIT) of the Russian Academy of Sciences (project 0063-2015-0017.III.3).


%\vspace*{3pt}

  \begin{multicols}{2}

\renewcommand{\bibname}{\protect\rmfamily References}
%\renewcommand{\bibname}{\large\protect\rm References}

{\small\frenchspacing
 {%\baselineskip=10.8pt
 \addcontentsline{toc}{section}{References}
 \begin{thebibliography}{9}


\bibitem{1-sin-1}
\Aue{Sinitsyn, I.\,N. } 2015.
Analiticheskoe modelirovanie protsessov v~dinamicheskikh sistemakh 
s~tsilindricheskimi besselevymi nelineynostyami [Analytical modeling of processes
in~dynamical systems with cylindric
Bessel nonlinearities]. 
\textit{Informatika i~ee Primeneniya~--- Inform. Appl.} 9(4):39--49.

\bibitem{2-sin-1}
\Aue{Gradshteyn, I.\,S., and I.\,M.~Ryzhik.} 1963.
\textit{Tablitsy integralov, summ, ryadov i~proizvedeniy}
[Tables of integrals, sums, series and products]. Moscow: GIFML.  1100~p.

\bibitem{3-sin-1}
Abramovich,~M., and I.~Stigan, eds. 1979.
\textit{Spravochnik po spetsial'nym funktsiyam}
[Handbook of mathematical functions]. Moscow: Nauka. 832~p.

\bibitem{4-sin-1}
\Aue{Popov, B.\,A., and G.\,S.~Tesler.} 1984.
\textit{Vychislenie funktsiy na EVM: Spravochnik}
[Calculation of functions on the computer: Handbook]. Kiev: Naukova Dumka.  599~p.

\bibitem{5-sin-1}
 \Aue{Pugachev, V.\,S., and I.\,N.~Sinitsyn.} 
1987. \textit{Stochastic differential systems.
Analysis and filtering}. Chichester\,--\,New York, NY: Jonh Wiley.
549~p.


\bibitem{6-sin-1}
\Aue{Pugachev, V.\,S., and I.\,N.~Sinitsyn.} 2000, 2004.
Teoriya stokhasticheskikh sistem [Stochastic systems. Theory and  applications]. 
Moscow: Logos. 1000~p.  %[Angl. per. . -- Singapore: World Scientific, 2001].


\bibitem{7-sin-1}
\Aue{Sinitsyn, I.\,N.,  and V.\,I.~Sinitsyn.} 2013.
\textit{Lektsii po normal'noy i~ellipsoidal'noy approksimatsii raspredeleniy 
v~stokhasticheskikh sistemakh} [Lectures on normal and ellipsoidal approximation
of distributions in stochastic systems].  Moscow: TORUS PRESS.  488~p.


\bibitem{8-sin-1}
\Aue{Sinitsyn, I.\,N. } 2013.
Parametricheskoe statisticheskoe i~analiticheskoe modelirovanie raspredeleniy 
v~nelineynykh stokhasticheskikh sistemakh na mnogoobraziyakh
[Parametrical statistical and analytical modeling of distributions in nonlinear stochastic
systems on manifolds].
\textit{Informatika i~ee Primeneniya~--- Inform. Appl.}  7(2):4--16.

   \end{thebibliography}

 }
 }

\end{multicols}

\vspace*{-3pt}

\hfill{\small\textit{Received March 22, 2016}}

\Contrl

\noindent
\textbf{Sinitsyn Igor N.} (b.\ 1940)~---
Doctor of Science in technology, professor,
Honored scientist of RF, Head of Department, Institute of Informatics Problems, Federal Research Center ``Computer Science and
Control'' of the Russian Academy of Sciences, 44-2 Vavilov Str.,
Moscow 119333, Russian Federation; sinitsin@dol.ru


\label{end\stat}


\renewcommand{\bibname}{\protect\rm Литература}