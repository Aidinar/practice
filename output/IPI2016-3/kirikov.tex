\def\stat{kirikov}

\def\tit{<<ВИРТУАЛЬНЫЙ КОНСИЛИУМ>>~--- ИНСТРУМЕНТАЛЬНАЯ 
СРЕДА ПОДДЕРЖКИ ПРИНЯТИЯ 
  СЛОЖНЫХ ДИАГНОСТИЧЕСКИХ РЕШЕНИЙ$^*$}

\def\titkol{<<Виртуальный консилиум>>~--- инструментальная 
среда поддержки принятия сложных диагностических решений}

\def\aut{И.\,А.~Кириков$^1$, А.\,В.~Колесников$^2$, С.\,В.~Листопад$^3$, 
С.\,Б.~Румовская$^4$}

\def\autkol{И.\,А.~Кириков, А.\,В.~Колесников, С.\,В.~Листопад, 
С.\,Б.~Румовская}

\titel{\tit}{\aut}{\autkol}{\titkol}

\index{Кириков И.\,А.}
\index{Колесников А.\,В.}
\index{Листопад С.\,В.} 
\index{Румовская С.\,Б.}
\index{Kirikov I.\,А.}
\index{Kolesnikov А.\,V.}
\index{Listopad S.\,V.}
\index{Rumovskaya S.\,B.}


{\renewcommand{\thefootnote}{\fnsymbol{footnote}} \footnotetext[1]
{Работа выполнена при частичной поддержке РФФИ (проект 16-07-00272 А).}}


\renewcommand{\thefootnote}{\arabic{footnote}}
\footnotetext[1]{Калининградский филиал Федерального исследовательского центра <<Информатика и~управление>> 
Российской академии наук, \mbox{baltbipiran@mail.ru}}
\footnotetext[2]{Балтийский Федеральный университет
имени  И.~Канта, Калининградский филиал Федерального 
исследовательского центра <<Информатика и~управление>> Российской академии наук, 
\mbox{avkolesnikov@yandex.ru}}
\footnotetext[3]{Калининградский филиал Федерального исследовательского центра <<Информатика и~управление>> 
Российской академии наук, \mbox{ser-list-post@yandex.ru}}
\footnotetext[4]{Калининградский филиал Федерального исследовательского центра <<Информатика 
и~управление>> Российской академии наук, \mbox{sophiyabr@gmail.com}}
 
 \vspace*{-3pt}
 
  \Abst{Рассматривается проблема принятия индивидуального решения при диагностике 
пациентов в~ам\-бу\-ла\-тор\-но-по\-ли\-кли\-ни\-че\-ских учреждениях на примере 
диагностики артериальной гипертензии (АГ). Предлагается повысить качество принятия 
индивидуального решения за счет консультаций с~системой поддержки принятия  
решения~--- <<Виртуальным консилиумом>>, моделирующим коллективный интеллект 
врачей стационара многопрофильного больничного учреждения. Приведены результаты 
проектирования и~экспериментального исследования лабораторного прототипа 
<<Виртуального консилиума>>.}

  \KW{система поддержки принятия решения; виртуальный консилиум; функциональная 
гибридная интеллектуальная система}

\DOI{10.14357/19922264160311} 


\vskip 10pt plus 9pt minus 6pt

\thispagestyle{headings}

\begin{multicols}{2}

\label{st\stat}
  

\section{Введение}

  Степень исследования, понимания и~качества диагностики проблемных сред и~их 
окружения отражена в~научной картине мира, онтологи\-зи\-ру\-ющей его представления 
и~делающей рассуждения и~целенаправленную деятельность <<зависимыми>> от них. 
В~искусственном интеллекте понятию <<картина мира>> соответствует понятие <<модель 
внешнего мира>> М.\,Г.~Га\-азе-Рап\-по\-пор\-та и~Д.\,А.~Поспелова~[1]. 
  
  Новая картина мира складывается из многочисленных теорий и~взглядов: <<ноосфера>>, 
<<разумный мир>> (В.\,И.~Вернадский, Н.\,Н.~Моисеев, А.\,В.~Поздняков); <<мир 
диалектики>>~--- мир диалога разных логик (Е.\,Л.~Доценко); социальная парадигма 
искусственного интеллекта (<<The society of mind>>) М.~Минского;  
сис\-тем\-но-ор\-га\-ни\-за\-ци\-он\-ный подход в~искусственном интеллекте 
В.\,Б.~Тарасова; теория иерархических многоуровневых систем М.~Месаровича, Д.~Мако 
и~И.~Такахары и~др.~--- и~укладывается в~семь постулатов~[2]: (1)~признание 
гетерогенности мира и~любого объекта, разнообразия жизни; (2)~неопределенность границ 
объектов и~связь <<всего со всем>>; (3)~относительность любой иерархии и~горизонтальные 
связи; (4)~дополнительность и~сотрудничество; (5)~полицентризм; (6)~относительность 
знания; (7)~соответствие управления сложности объекта. 
  
  Сложная задача диагностики АГ (СЗДАГ)~---
  за\-да\-ча-сис\-те\-ма, вклю\-ча\-ющая диагностические и~технологические подзадачи, 
повышающие эффективность обработки симптоматической информации о пациенте. 
Разнообразие подзадач СЗДАГ с~различными характеристическими свойствами требует 
разнообразия соответствующих методов принятия решений, системного анализа, 
искусственного интеллекта и~инженерии знаний. 
  
  Анализ результатов влияния новой картины мира на ментальную составляющую 
врачебной практики и~медицинской информатики~[3] показал, что, несмотря на стремление 
биомедицины к~гетерогенности восприятия организма человека и~процесса его диагностики 
в~рамках семипостулатной картины мира, человек по-преж\-не\-му остается 
<<расчлененным>> объектом познания, что сформировало <<узких>> специалистов, 
поглощенных решением частных задач. Новый тип ученого <<праг\-ма\-ти\-ка-фак\-то\-ло\-га>> 
утратил системное мышление, перестал задумываться над тем, что делается <<вокруг>> 
и~какое значение могут иметь добытые им факты для понимания работы организма в~целом. 
В~этой связи\linebreak\vspace*{-12pt}

\pagebreak

\noindent
 очевидна необходимость перехода от методов <<конкурентной>> диагностики 
к системному мышлению и~методам гетерогенной диагностики.
  
  В~[3--5] представлены результаты системного анализа СЗДАГ, следуя 
  проблемно-структурной (ПС) методологии, этапы~1--5~[6]: идентификация, редукция сложной задачи, 
спецификация диагностических подзадач, выбор методов их решения, а~также проверка 
неоднородности сложной задачи диагностики. Работы~[3--5] подтвердили релевантность 
применения междисциплинарных инструментариев для решения 
СЗДАГ, мо\-де\-ли\-ру\-ющих разнообразие информации, 
сотрудничество, дополнительность и~относительность знаний, сочетающих методы 
и~методики системного анализа диагностической проблемы с~динамическим синтезом 
метода ее решения и~имитацией работы искусственного гетерогенного коллектива~--- 
<<виртуального консилиума>>.
  
  Разнообразие~--- признак, проявление гетерогенности. Следствие закона необходимого 
разнообразия У.\,Р.~Эшби констатирует, что управ\-ле\-ние обеспечивается, если разнообразие 
средств управ\-ля\-юще\-го не меньше разнообразия управ\-ля\-емой им ситуации. Для отображения 
в информатике ситуативного разнообразия в~естественных гетерогенных системах в~[6] 
введены модели <<гетерогенная, неоднородная задача>> и~<<гомогенная, однородная 
задача>>, а~сам закон трактуется так: только разнообразная, скоординированная клиническая 
деятельность, элементы которой в~комбинации решают одну задачу, сделает результат 
диагностики качественно лучше в~обществе с~новой научной картиной мира. Специфике 
такой работы соответствует коллективный труд экспертов в~малых группах за круглым 
столом~--- консилиумы, совещания, естественные гетерогенные системы для решения 
сложных задач~\cite{3-kir}, где на первый план выходят знания и~опыт лица, принимающего 
решения (ЛПР), и~экспертов.
  
  \begin{figure*} %fig1
\vspace*{1pt}
 \begin{center}  
\mbox{%
 \epsfxsize=147.497mm
 \epsfbox{kir-1.eps}
 }
\end{center} 
%\vspace*{-9pt}
%\Caption{Концептуальная модель процесса диагностики артериальной гипертензии: в~многопрофильном 
%стационарном больничном учреждении~(\textit{а}); в~амбулаторно-поликлиническом~(\textit{б})}
  \end{figure*}

  \addtocounter{figure}{1}
  
  Настоящая работа~--- продолжение работ~[3--5,\linebreak 7] и~имеет целью представить: (1)~результаты 
исследования процесса диагностики АГ  
в~ле\-чеб\-но-про\-фи\-лак\-ти\-че\-ских больничных учреждениях (ЛПУ) широкого 
профиля~--- предлагается повысить эффективность и~качество индивидуальных 
диагностических решений в~ЛПУ широкого профиля ам\-бу\-ла\-тор\-но-по\-ли\-кли\-ни\-че\-ско\-го 
характера (рис.~1,\,\textit{а}) за счет внедрения информационной технологии 
<<Виртуальный консилиум>>, моделирующей коллективное обсуждение; 
(2)~архитектуру <<Виртуального консилиума>> и~результаты лабораторных экспериментов с~
его интегрированными моделями (первые результаты лабораторных экспериментов 
приведены в~[7]).

\section{Диагностика артериальной гипертензии в~многопрофильном 
стационарном больничном учреждении и~в~амбулаторно-поликлиническом 
учреждении}

\vspace*{-9pt}


  В~[8, 9] представлены результаты исследования процесса диагностики 
АГ в~Калининградской клинической областной больнице (КОКБ) 
(см.\ рис.~1,\,\textit{б}) и~ее Диагностическом центре (см.\ рис.~1,\,\textit{а}). 

Для формирования 
полного дифференциального диагноза АГ коллективом врачей во главе с~лечащим врачом, 
ЛПР-кар\-дио\-ло\-гом, в~стационаре привлекаются до тринадцати вра\-чей-экс\-пер\-тов~--- носителей 
знаний из различных разделов медицины: невролог, нефролог, сосудистый хирург, уролог, 
психолог, педиатр, аку\-шер-ги\-не\-ко\-лог, онколог, окулист, врачи функциональной 
диагностики, эндокринолог, терапевт, кардиолог. 

Для исследований выбраны шесть 
специалистов (см.\ рис.~1,\,\textit{б}), решающих двенадцать функциональных подзадач 
(рис.~\ref{f2-kir}), возникающих в~90\%~случаев диагностики АГ, 
каждый из которых формирует промежуточные заключения о~состоянии объекта 
диагностики в~своей области медицинских зна\-ний. 
{\looseness=1

}

Полученные исходные данные об объекте 
диагностики разнородны (содержатся в~медицинской карте): количественные,  
ви\-зу\-аль\-но-графиче\-ские параметры (детерминированные переменные),\linebreak 
лингвистические четкие и~нечеткие переменные. Лицо, при\-ни\-ма\-ющее решение, изучает в~медицинской карте 
симптомы и~частные диагностические мнения вра\-чей-экс\-пер\-тов, множество которых 
подбирает сам, и~ставит заключительный диагноз. Вра\-чам-экс\-пер\-там доступны симптомы 
и~мнения других врачей-экспертов из медицинской карты.
\mbox{Лицо}, при\-ни\-ма\-ющее решение, и~вра\-чи-экс\-пер\-ты 
обследуют пациента и~формируют диагностические заключения согласно нормативным 
документам, например~[10]. В~ЛПУ широкого профиля (см.\ рис.~1,\,\textit{а}) ЛПР~--- это врач 
общей практики или терапевт (иногда кардиолог, но зачастую без опыта работы, к~которому 
направляет терапевт сразу же при выявлении повышенного артериального давления), это 
врач <<праг\-ма\-тик-фак\-то\-лог>>~\cite{9-kir}, объединяющий в~себе роли вра\-ча-ЛПР  
и~вра\-чей-экс\-пер\-тов узкой специализации.

\end{multicols}

\begin{figure} %fig2
\vspace*{1pt}
 \begin{center}  
\mbox{%
 \epsfxsize=163.044mm
 \epsfbox{kir-2.eps}
 }
\end{center} 
\vspace*{-9pt}
\Caption{Архитектура ВКДАГ }
\label{f2-kir}
\vspace*{3pt}
\end{figure}

\begin{multicols}{2}
  

  Исследования диагностического процесса на материалах Диагностического центра КОКБ 
по модели на рис.~1,\,\textit{а} показали, что~70\%~пациентов с~АГ 
амбулаторно-поликлинического учреждения не знают о своем заболевании, в~то время как в~стационарных 
медицинских учреждениях (см.\ рис.~1,\,\textit{б}) практически в~100\%~случаев имеет место 
как адекватное проведение, так и~отображение в~медицинских картах симптоматических 
данных обследования с~подтверждением диагноза  
ла\-бо\-ра\-тор\-но-ин\-ст\-ру\-мен\-таль\-ны\-ми методами исследования. 
  
  В этой связи предлагается повысить эффективность и~качество индивидуальных 
диагностических решений в~ЛПУ широкого профиля амбула\-тор\-но-по\-ли\-кли\-ни\-че\-ско\-го 
характера (см.\ рис.~1,\,\textit{а}) за счет внед\-ре\-ния информационной технологии 
<<Виртуальный консилиум>> (см.\ рис.~\ref{f2-kir}), моделирующей коллективное обсуждение, 
обладающего синергией, опытом и~знаниями в~решении подзадач диагностики 
АГ в~стационаре (см.\ рис.~1,\,\textit{б}). 


  

  
\section{Инструментальная среда <<Виртуальный консилиум для~диагностики 
артериальной гипертензии>>}

\vspace*{-18pt}

  Инструментальная среда <<Виртуальный консилиум>>, архитектура которой 
представлена на рис.~\ref{f2-kir}, а~структура в~\cite{7-kir}, ограничена пациентами 
стар\-ше~18~лет, без особых состояний, нет распознавания снимков, не предусматривается 
назначение лечения и~не диагностируется ряд симптоматических артериальных гипертензий. 

Архитектура <<Виртуального консилиума для диагностики артериальной гипертензии>> 
(ВКДАГ) включает межмодульные интерфейсы~$\zeta^u$ для модулей, реализованных 
посредством различных методологий гибридных интеллектуальных сис\-тем (\mbox{ГиИС}) 
(генетические алгоритмы ($g$), нечеткие 
сис-\linebreak\vspace*{-12pt}

\pagebreak

\end{multicols}

\begin{table*}\small
%\vspace*{-12pt}
\begin{center}
\Caption{Описание блоков архитектуры ВКДАГ}
\vspace*{2ex}

\begin{tabular}{|p{30mm}|p{40mm}|p{39mm}|p{39mm}|}
\hline
\multicolumn{1}{|c|}{Наименование блока}&\multicolumn{1}{c|}{Функции}&\multicolumn{1}{c|}{Вход}&\multicolumn{1}{c|} 
{Выход}\\
\hline
Технологический модуль $i$-й&
Организация эффективной обработки данных и~знаний, выбирается для 
включения в~функциональную \mbox{ГиИС}~--- построение информативного набора 
признаков для диагностики&Популяция 
индивидуумов, накладывающихся как маска на $i$-й функциональный модуль&
Наилучшая особь с~оптимальным набором признаков~--- накладывается как 
маска на $i$-й функциональный модуль\\
\hline
Функциональный модуль $i$-й&Классификация состояния здоровья пациента в~рамках 
\mbox{$i$-й} диагностической 
подзадачи, выбирается для включения в~функциональную \mbox{ГиИС} &
Подмножество $i$-е симптомов с~интерфейса 
пользователя&Частное $i$-е заключение о~со\-сто\-янии здоровья пациента\\
\hline
Функциональный модуль {HCCCC}, моделирующий ЛПР&
Формирование заключительного диагноза 
АГ (всегда в~составе <<Виртуального консилиума>>)&Подмножество симптомов 
с~интерфейса пользователя, множество выходов функциональных модулей&
Заключительный диагноз АГ \\
\hline
Функциональный модуль {ИНСРЭКГ}&Классификация патологического состояния пациента по его 
электрокардиограмме&\multicolumn{2}{p{60mm}|}{Рассмотрены подробно в~\cite{4-kir}}\\
\cline{1-2}
Функциональный модуль {ИНССМАД}&Прогноз нормальных зна\-чений суточного мониторирования 
артериального давле\-ния и~вычисление отклонения &\multicolumn{2}{c|}{\ }\\
\hline
Интерфейс модификации структуры {ВКДАГ}&Исключение из диагностики модулей, решающих не 
интересующие пользователя подзадачи &
Выбранные пользователем подзадачи диагностики &
Функциональная ГиИС, 
синтезированная посредством алгоритма из~\cite{4-kir}\\
\hline
Интерфейс пользователя <<Диагноз>>&Визуализация результатов диагностики и~корректировка их 
пользователем &Заключительный диагноз от функционального модуля НСССС&Отчет, содержащий 
множество симптомов и~диагноз\\
\hline
Интерфейс пользователя &Ввод информации о~со\-сто\-янии здоровья пациента &
Множество значений 
показателей состояния здоровья пациента&
Показатели состояния здоровья пациента, распределенные по 
функциональным модулям \\
\hline
Модификация интерфейса пользователя&Деактивация элементов на интерфейсе пользователя для ввода 
значений показателей состояния здоровья&Множество выходов технологических модулей&Частично 
деактивированный интерфейс пользователя \\
\hline
\end{tabular}
\end{center}
\end{table*}

\begin{multicols}{2}

\noindent 
те\-мы ($f$), искусственные нейронные сети ($n$)).
 В~библиотеке модулей диагностики 
и~препро\-цессии хранятся заранее инициализированные\linebreak в~программной среде 
функциональные и~технологические модели. 
По умолчанию все модули включены 
в~структуру <<Виртуального консилиума>>, их описание пред\-став\-ле\-но в~табл.~1. %\\[-15pt]
%
      <<Виртуальный консилиум>> (см.\ рис.~\ref{f2-kir}) запускает интерфейс пользователя, 
ЛПР-вра\-ча~--- <<{Интерфейс модификации структуры ВКДАГ}>>, посредством 
которого включаются функциональные 
 и~технологические модули в~работу сис\-те\-мы: модуль 
<<Анализ СМАД>>, модуль <<Распознавание ЭКГ>>, модули технологических подзадач из 
группы <<Построение информативного набора признаков\linebreak (симптомов) при диагностике 
заболеваний>> и~модули подзадач из группы <<Диагностика критериев оценки 
сер\-деч\-но-со\-су\-ди\-сто\-го риска и~вторичной АГ у~пациента>> ({ДАГ}$_1$, \ldots , {ДАГ}$_9$): 
диагностики\linebreak поражений ор\-га\-нов-ми\-ше\-ней, факторов риска, цереброваскулярных 
болезней, метаболического синд\-ро\-ма и~сахарного диабета, заболеваний периферических 
артерий, ишемической болезни сердца,\linebreak эндокринной АГ, паренхиматозной нефропатии 
и~реноваскулярной АГ соответственно. Все выбранные $i$-е технологические модули 
запускаются, решают соответствующую подзадачу и~передают информацию на блок 
<<{Модификация интерфейса пользователя}>>. Он деактивирует показатели 
со\-сто\-яния здоровья на <<{Интерфейсе пользователя для\linebreak ввода значений показателей 
состояния здоровья пациента}>> и~корректирует работу $i$-го функционального модуля 
подзадач {ДАГ}$_1$, \ldots\linebreak \ldots , {ДАГ}$_9$. Далее активируется откорректированный 
интерфейс, вводятся симптомы, которые передаются функциональным нечетким модулям, 
решающим подзадачи {ДАГ}$_1$, \ldots , {ДАГ}$_9$\linebreak (моделируют принятие 
решения экспертами, врачами смежных специальностей~--- кардиологом как экспертом, 
неврологом, нефрологом, терапевтом, эндокринологом, урологом). Последние в~свою 
очередь передают информацию о~патологиях, выявленных ими у~пациента, 
функциональному модулю {НСССС} (моделирует принятие решения ЛПР~---  
вра\-чом-кар\-дио\-ло\-гом), решающему подзадачу <<Оценка степени и~стадии 
артериальной гипертензии, степени риска сер\-дечно-сосу\-ди\-стых заболеваний>>. 

В~библиотеке ВКДАГ есть еще два функциональных модуля (см.\ табл.~1), вклю\-ча\-ющих\-ся 
в~работу консилиума посредством <<{Интерфейса модификации структуры 
ВКДАГ}>>: 
      \begin{enumerate}[(1)]
      \item {ИНСРЭКГ}, передающий информацию на модули диагностики поражений 
ор\-га\-нов-ми\-ше\-ней (на рис.~\ref{f2-kir}~--- это {НСДАГ}$_1$), цереброваскулярных 
болезней ({НСДАГ}$_3$) и~ишемической болезни сердца ({НСДАГ}$_6$); 
      \item {ИНССМАД}, формирующий информацию о~нормальных значениях 
суточного артериального давления на функциональный модуль {НСССС}.
      \end{enumerate}
      
\section{Экспериментальное лабораторное исследование программной 
реализации прототипа инструментальной среды <<Виртуальный консилиум>>}
  
  Экспериментальное лабораторное исследование программной реализации 
исследовательского прототипа функциональной гибридной интеллектуальной системы 
ВКДАГ для поддержки принятия сложных диагностических решений необходимо для 
подтверждения его релевантности~[3--5, 7] реальной ситуации диагностики АГ. В~[4] 
пред\-став\-ле\-на информация по особенностям функциональных и~технологических моделей 
гетерогенного модельного поля ВКДАГ, а~в~[7]~--- информация по их инициализации 
в~среде MATLAB-Simulink, результаты исследований качества работы каждой модели 
гетерогенного модельного поля <<Виртуального консилиума>> автономно, а~также 
подтверждена их релевантность работе экспертов~--- врачей узкой специализации, что 
предотвращает распространение ошибок работы автономных моделей на работу 
интегрированной модели. 

В~настоящей работе приведены результаты исследования качества 
интегрированных моделей, синтезированных <<Виртуальным консилиумом>>\linebreak 
и~моделирующих дополнительность и~сотрудничество, которые имитируют коллективные 
рас\-суж\-де\-ния специалистов при постановке диагноза. 

В~табл.~2 представлены критерии 
и~результаты тес\-ти\-ро\-ва\-ния интегрированных моделей <<Виртуального консилиума>> 
с~различными комбинациями знаний врачей, классифицирующих патологическое состояние 
пациента. Порядок работы моделей гетерогенного модельного поля \mbox{ВКДАГ}: запускаются 
модели первой очереди~--- модели технологических элементов {ГАППС}$_{1\mbox{--}9}$, 
корректирующие множества входных переменных моделей {НСДАГ}$_{1\mbox{--}9}$ 
и~{НСССС}; обработка информации передается функциональным элементам: модели 
второй очереди <<отправляют>> информацию на модели третьей, пятой, шес\-той и~седьмой 
очередей~--- \mbox{ИНСРЭКГ} (модель, решающая задачу распознавания электрокардиограммы (ЭКГ)), 
{ИНССМАД} (формирует оптимальные множества показателей суточного давления), 
{НСДАГ}$_9$, {НСДАГ}$_2$ и~{НСДАГ}$_6$; третья\linebreak очередь содержит 
модели НСДАГ$_4$ и~НСДАГ$_5$, передающие выходную информацию на вход моделей четвертой 
и~седьмой очередей; четвертая очередь содержит модель {НСДАГ}$_8$, пе\-ре\-да\-ющую 
информацию  модели пятой очереди {НСДАГ}$_1$, которая в~свою очередь передает 
информацию\linebreak {НСДАГ}$_3$ (шес\-тая очередь); от {НСДАГ}$_3$ передается 
информация {НСДАГ}$_7$ (седьмая очередь); последней запускается модель 
{НСССС}, формирующая заключительный диагноз, на вход которой передается 
выходная информация функциональных моделей вто\-рой--седь\-мой очередей.
  
  Таким образом: (1)~без знаний кардиолога, или нефролога, или эндокринолога 
сред\-не\-квад\-ратическая ошибка наибольшая~--- 0,697; 0,448 и~0,211 соответственно, 
и~объясняется это тем, что кардиолог играет ключевую роль в~обработке ин\-формации, 
поступающей от других врачей\linebreak\vspace*{-12pt}


\pagebreak

\end{multicols}

\begin{table}\small
\begin{center}
\Caption{Параметры и~результаты тестирования интегрированных моделей }
\vspace*{2ex}

\begin{tabular}{|p{66mm}|p{88mm}|}
\hline
\multicolumn{1}{|c|}{\tabcolsep=0pt\begin{tabular}{c}Наименование параметров\\ 
и результатов тестирования\end{tabular}}&
\multicolumn{1}{c|}{Значения параметров и~результатов 
тестирования}\\
\hline
Объем тестовой выборки ВКДАГ, интегрирующего знания всех шести врачей&800 наблюдений~--- 500 с~
диагнозами эссенциальной АГ и~300 с~диагнозами вторичной АГ\\
\hline
Объем тестовой выборки ВКДАГ, интегрирующего знания менее шести врачей&400 наблюдений~--- 200 с~
диагнозами эссенциальной АГ и~200 с~диагнозами вторичной АГ\\
\hline
Источник формирования тестовой вы\-борки&Архив медицинских карт пациентов 1-го кардиологического 
отделения КОКБ\\
\hline
Элемент тестирующей последова\-тель\-ности&
Содержит множество нечетких лингвистических переменных и~вектор образа электрокардиограммы (может отсутствовать)\\
\hline
Эталонный диагноз&Результаты деятельности лечащего вра\-ча-кар\-дио\-ло\-га, подводящего общий итог~--- 
дифференциальный диагноз АГ\\
\hline
Критерии тестирования&Среднеквадратическая ошибка $f$ классификации состояния здоровья пациента~[7]\\
\hline
$f$(шесть врачей)&0,0837\\
\hline
$f$(без кардиолога)&0,697\\
\hline
$f$(без нефролога)&0,448 (в остальных 55,2\% случаях диагноз не вызовет доверия)\\
\hline
$f$(без терапевта)&0,151\\
\hline
$f$(без невролога)&0,149\\
\hline
$f$(без эндокринолога)&0,211 (в остальных 78,9\% случаях диагноз не вызовет доверия)\\
\hline
$f$(без сосудистого хирурга)&0,0798\\
\hline
$f$(без знаний терапевта, невролога, неф\-ро\-ло\-га, эндокринолога, сосудистого хирурга)&0,711\\
\hline
$f$(без знаний терапевта, невролога, эндокринолога, сосудистого хирурга)&0,485\\
\hline
$f$(без знаний невролога, эндокринолога, сосудистого хирурга)&0,334\\
\hline
$f$(без знаний невролога, сосудистого хи\-рурга)&0,167\\
\hline
\end{tabular}
\end{center}
\end{table}

\begin{multicols}{2}


\noindent
 и~от ла\-бораторных исследований, и~в~постановке
заключительного диагноза, а~нефролог и~эндокринолог~--- в~исключении вторичной 
АГ; (2)~знания врача~--- сосудистого хирурга не влияют на 
результаты работы <<Виртуального консилиума>>, и~объясняется это тем, что знания 
сосудистого хирурга, касающиеся диагностики АГ, составляют только~20\% базы знаний 
нечеткой системы, распознающей заболевания периферических артерий (ассоциативные 
клинические состояния), встречающихся не более чем у~10\% населения~\cite{11-kir}, 
и~в~тес\-то\-вую выборку не попала ни одна карта с~данными заболеваниями; (3)~чем больше 
численный состав <<Виртуального консилиума>>, тем с~меньшей среднеквадратической 
ошибкой он классифицирует состояние здоровья пациента; (4)~<<Виртуальный консилиум>> 
в~со\-ста\-ве шести врачей диагностирует АГ со среднеквадратической 
ошибкой постановки диагноза $f = 0{,}0837$, т.\,е.\ дает диагноз, верный в~84\% слу\-чаях. 
{\looseness=1

}
  
  Поскольку <<Виртуальный консилиум>> разра\-ботан на основе всероссийских~\cite{9-kir} 
и~между\-народных рекомендаций по диагностике АГ и~со\-пут\-ст\-ву\-ющих заболеваний, 
которых должен придерживать\-ся каж\-дый врач в~своей практике, при переносе \mbox{ВКДАГ} 
в~другое больничное учреж\-де\-ние необходимо пред\-оста\-вить врачам данного учреждения 
протоколы подтверждения диагностических правил всех баз знаний экспериментальными 
данными из архива КОКБ для ознакомления 
и~внесения при необходимости коррективов в~связи с~возможными особенностями их 
контингента пациентов, а~также возможных требований по устранению ограничений 
системы со стороны персонала нового больничного учреждения. Значительной 
корректировки баз знаний не потребуется.
  
  Таким образом, лабораторные эксперименты с~прототипом <<Виртуального 
консилиума>> дали обнадеживающие результаты. 

Верное решение получено в~84\% 
случаев. В~ам\-бу\-ла\-тор\-но-кли\-ни\-че\-ских учреждениях диагноз не 
выявляется у~70\% пациентов в~основном по причине инертности врачей, недостатка опыта 
врачей узкой специализации и~нехватки кадров в~ЛПУ
широкого профиля, что по результатам экспериментов может быть устранено с~по\-мощью 
применения \mbox{ВКДАГ} во время приема пациентов с~подозрением на АГ.

\section{Заключение}

  Лабораторно подтверждена эффективность предлагаемого подхода для проектирования 
диагностических систем как гетерогенных искусственных диагностических систем со 
свойствами дополнительности, сотрудничества и~относительности\linebreak
 знаний, синтезирующих 
интегрированные методы и~модели, разнообразие которых устраняет разнообразие 
диагностической информации об организме человека~--- <<Виртуальных консилиумов>>,\linebreak 
моделиру\-ющих работу коллектива врачей в~многопрофильном стационарном больничном 
учреждении (на примере КОКБ) и~внедрение 
которых повыша\-ет эффективность и~качество индивидуальных диагностических решений 
в~ам\-бу\-ла\-тор\-но-по\-ли\-кли\-ни\-че\-ском учреждении широкого профиля (на примере 
Диагностического центра КОКБ), где заключение о состоянии больного из-за проблемы 
с~кадрами узкой специализации принимает чаще всего один специалист~--- терапевт или 
врач общей практики, иногда кардиолог, но без опыта работы.

{\small\frenchspacing
 {%\baselineskip=10.8pt
 \addcontentsline{toc}{section}{References}
 \begin{thebibliography}{99}
\bibitem{1-kir}
\Au{Гаазе-Раппопорт М.\,Г., Поспелов~Д.\,А.} От амебы до робота: модели поведения.~--- 
М.: Наука, 1987. 288~с.
\bibitem{2-kir}
\Au{Колесников А.\,В., Кириков~И.\,А., Листопад~С.\,В. %Румовская~С.\,Б. 
и~др.} Решение 
сложных задач коммивояжера методами функциональных гибридных интеллектуальных 
сис\-тем.~--- М.: ИПИ РАН, 2011. 295~с.
\bibitem{3-kir}
\Au{Кириков И.\,А., Колесников~А.\,В., Румовская~С.\,Б.} Исследование сложной задачи 
диагностики артериальной гипертензии в~методологии искусственных гетерогенных  
сис\-тем~// Системы и~средства информатики, 2013. Т.~23. №\,2. С.~81--99. doi: 
10.14357/08696527130208.
\bibitem{4-kir}
\Au{Кириков И.\,А., Колесников~А.\,В., Румовская~С.\,Б.} Функциональная гибридная 
интеллектуальная система для поддержки принятия решений при диагностике артериальной 
гипертензии~// Системы и~средства информатики, 2014. Т.~24. №\,1. С.~153--179. doi: 
10.14357/08696527140110.
\bibitem{5-kir}
\Au{Колесников А.\,В., Румовская~С.\,Б., Листопад~С.\,В., Кириков~И.\,А.} Системный 
анализ в~решении сложных диагностических задач~// Системный анализ и~информационные 
технологии (САИТ-2015): Тр. VI~Междунар. конф.~--- М.: 
ИСА РАН, 2015. Т.~1. С.~157--167.
\bibitem{6-kir}
\Au{Колесников А.\,В., Кириков~И.\,А.} Методология и~технология решения сложных задач 
методами функциональных гибридных интеллектуальных систем.~--- М.: ИПИ РАН, 2007. 
387~с.
\bibitem{7-kir}
\Au{Кириков И.\,А., Колесников~А.\,В., Румовская~С.\,Б.} Исследование лабораторного 
прототипа искусственной гетерогенной системы для диагностики артериальной 
гипертензии~// Системы и~средства информатики, 2014. Т.~24. №\,3. С.~131--143. doi: 
10.14357/08696527140309.
\bibitem{8-kir}
\Au{Румовская С.\,Б.} Методы и~средства информатики для диагностики 
артериальной гипертензии в~ле\-чеб\-но-про\-фи\-лак\-ти\-че\-ских учреждениях 
широкого профиля~// Задачи современной информатики (ЗСИ-2015): Тр. 2-й 
молодежной научной конф.~--- М.: ФИЦ ИУ РАН, 2015. 
С.~168--174.
\bibitem{9-kir}
\Au{Кириков~И.\,А., Румовская~С.\,Б.} Гетерогенная диагностика артериальной 
гипертензии~// Информатика, управление и~системный анализ (ИУСА-2016): Тр. 
4-й Всеросс. научной конф. молодых ученых с~международным участием.~--- 
Тверь: ТвГТУ, 2016. Т.~1. С.~180--188.
\bibitem{10-kir}
Комитет экспертов ВНОК. Диагностика и~лечение артериальной гипертензии. 
Российские рекомендации~// Системные гипертензии, 2010. Вып.~3. С.~5--26.
\bibitem{11-kir}
\Au{Галимзянов Ф.\,В.} Заболевания периферических артерий (клиника, 
диагностика, лечение)~// Международный журнал экспериментального образования, 
2014. Вып.~8. С.~113--114. 

\end{thebibliography}

 }
 }

\end{multicols}

\vspace*{-6pt}

\hfill{\small\textit{Поступила в~редакцию 18.06.16}}

\vspace*{8pt}

%\newpage

%\vspace*{-24pt}

\hrule

\vspace*{2pt}

\hrule

%\vspace*{8pt}



\def\tit{``VIRTUAL COUNCIL''~--- SOURCE ENVIRONMENT SUPPORTING 
COMPLEX DIAGNOSTIC DECISION MAKING}

\def\titkol{``Virtual council''~--- source environment supporting 
complex diagnostic decision making}

\def\aut{I.\,А.~Kirikov$^1$, А.\,V.~Kolesnikov$^{1,2}$, S.\,V.~Listopad$^1$, and 
S.\,B.~Rumovskaya$^1$}

\def\autkol{I.\,А.~Kirikov, А.\,V.~Kolesnikov, S.\,V.~Listopad, and 
S.\,B.~Rumovskaya}

\titel{\tit}{\aut}{\autkol}{\titkol}

\vspace*{-9pt}

\noindent
$^1$Kaliningrad Branch of the Federal Research Center ``Computer Science and 
Control'' of the Russian Academy\linebreak
$\hphantom{^1}$of Sciences, 5~Gostinaya Str., Kaliningrad 236000, 
Russian Federation
   
   \noindent
   $^2$Immanuel Kant Baltic Federal University, 14~Nevskogo Str., Kaliningrad 236041, 
Russian Federation


\def\leftfootline{\small{\textbf{\thepage}
\hfill INFORMATIKA I EE PRIMENENIYA~--- INFORMATICS AND
APPLICATIONS\ \ \ 2016\ \ \ volume~10\ \ \ issue\ 3}
}%
 \def\rightfootline{\small{INFORMATIKA I EE PRIMENENIYA~---
INFORMATICS AND APPLICATIONS\ \ \ 2016\ \ \ volume~10\ \ \ issue\ 3
\hfill \textbf{\thepage}}}

\vspace*{3pt}
  
    
  
\Abste{The paper considers the problem of individual decision making during 
diagnostics of 
patients in outpatient clinics by the example of arterial 
hypertension diagnostics. It is proposed to 
raise the quality of individual decision\linebreak\vspace*{-12pt}}

\Abstend{making by means of consultations with the ``Virtual council'' 
decision support system, which models the work of physician councils in inpatient multifield 
clinics. The results of development and experimental research of the 
laboratory prototype of ``Virtual council'' are presented.}

\KWE{decision support system; virtual council; functional hybrid intellectual system}

\DOI{10.14357/19922264160311} 

\vspace*{-9pt}

\Ack
\noindent
The work was performed with partial support of the Russian
Foundation for Basic Research (grant No.\,16-07-00272~А).


%\vspace*{3pt}

  \begin{multicols}{2}

\renewcommand{\bibname}{\protect\rmfamily References}
%\renewcommand{\bibname}{\large\protect\rm References}

{\small\frenchspacing
 {%\baselineskip=10.8pt
 \addcontentsline{toc}{section}{References}
 \begin{thebibliography}{99}
\bibitem{1-kir-1}
\Aue{Gaaze-Rappoport, M.\,G., and D.\,A.~Pospelov}. 1987. \textit{Ot ameby do robota: Modeli 
povedeniya} [From ameba to robotic mashine: Behavior model] Moscow: Nauka. 288~p.
\bibitem{2-kir-1}
\Aue{Kolesnikov,~A.\,V., I.\,A.~Kirikov, S.\,V.~Listopad, \textit{et al.}}. 2011. \textit{Reshenie 
slozhnykh zadach kommivoyazhera metodami funktsional'nykh gibridnykh intellektual'nykh 
sistem} [Solving of the complex traveling salesman problem by means of functional hybrid 
intellectual systems]. Moscow: IPI RAN. 295~p.
\bibitem{3-kir-1}
\Aue{Kirikov, I.\,A., A.\,V.~Kolesnikov, and S.\,B.~Rumovskaya}.\linebreak
 2013. Issledovanie slozhnoy 
zadachi diagnostiki arterial'noy gipertenzii v~metodologii iskusstvennykh geterogennykh sistem 
[Research of the complex problem at\linebreak diagnosing of the arterial hypertension within the 
methodology of artificial heterogeneous systems]. \textit{Sistemy i~Sredstva Informatiki~--- 
Systems and Means of Informatics} 23(2):81--99. doi: 10.14357/08696527130208.
\bibitem{4-kir-1}
\Aue{Kirikov, I.\,A., A.\,V.~Kolesnikov, and S.\,B.~Rumovskaya}.\linebreak
 2014. Funktsional'naya 
gibridnaya intellektual'naya sistema dlya podderzhki prinyatiya resheniya pri diagnostike 
arterial'noy gipertenzii [Functional hybrid intelligent decision support system for diagnosing of the 
\mbox{arterial} hypertension]. \textit{Sistemy i~Sredstva Informatiki~--- Systems and Means of Informatics} 
24(1):153--179. doi: 10.14357/08696527140110. 
\bibitem{5-kir-1}
\Aue{Kolesnikov, A.\,V., I.\,A.~Kirikov, S.\,V.~Listopad, and S.\,B.~Rumovskaya}. 2015. 
Sistemnyy analiz v~reshenii slozhnykh diagnosticheskikh zadach [Systems analysis for solving 
complex diagnostic tasks]. \textit{Tr. 6-y Mezhdunar. konf. ``Sistemnyy analiz i~informatsionnye 
tekhnologii''} [6th Conference (International) ``Systems Analysis and Information Technology'' 
Proceedings]. Moscow.  1:157--167.
\bibitem{6-kir-1}
\Au{Kolesnikov, A.\,V., and I.\,A.~Kirikov}. 2007. \textit{Metodologiya i~tekhnologiya resheniya 
slozhnykh zadach metodami funk\-tsi\-o\-nal'\-nykh gibridnykh intellektual'nykh sistem} [Methodology 
and technology for solving of complex problems using the methodology of functional hybrid 
artificial systems]. Moscow: IPI RAN. 387~p.
\bibitem{7-kir-1}
\Aue{Kirikov, I.\,A., A.\,V.~Kolesnikov, and S.\,B.~Rumovskaya}. 2014. Issledovanie 
laboratornogo prototipa iskusstvennoy geterogennoy sistemy dlya diagnostiki arterial'noy 
gipertenzii [Research of the laboratory prototype of the artificial heterogeneous system for 
diagnosing of the arterial hypertension]. \textit{Sistemy i~Sredstva informatiki~--- Systems and 
Means of Informatics} 24(3):131--143. doi: 10.14357/08696527140309.
\bibitem{8-kir-1}
\Au{Rumovskaya, S.\,B.} 2015. Metody i~sredstva informatiki dlya diagnostiki 
arterial'noy gipertenzii v~lechebno-profilakticheskikh uchrezhdeniyakh shirokogo profilya 
[Methods and tools of informatics for diagnostics of arterial hypertension in multiskilled 
medical preventive institution]. \textit{Tr. 2-y molodezhnoy nauchnoy konf. ``Zadachi 
sovremennoy informatiki''} [2nd Youth Conference ``Tasks of Modern Informatics'' 
Proceedings]. Moscow: FRC ``Computer Science and Control'' RAS. 168--174.
\bibitem{9-kir-1}
\Aue{Kirikov, I.\,A., and S.\,B.~Rumovskaya}. 2016. Geterogennaya diagnostika arterial'noy 
gipertenzii [Heterogeneous diagnostics of arterial hypertension]. \textit{Tr. 4-y Vseross. 
nauchnoy konf. molodykh uchenykh s~mezhdunarodnym uchastiem ``Informatika, 
upravlenie i~sistemnyy analiz''} [4th Youth Conference (International) ``Informatics, Control 
and Systems Analysis'' Proceedings]. Tver: Tver State Technical University. 1:180--188.
\bibitem{10-kir-1}
Komitet ekspertov VNOK [Committee of experts of All-Russia Scientific Society of Сardiologists]. 
2010. Diagnostika i~lechenie arterial'noy gipertenzii. Rossiyskie 
rekomendatsii [Diagnosing and treatment of arterial 
hypertension. Russian recommenation]. 
\textit{Sistemnye gipertenzii} [Systemic Hypertension] 3:5--26. 
\bibitem{11-kir-1}
\Aue{Galimzyanov, F.\,V.} 2014. Zabolevaniya perifericheskikh arteriy (Klinika, 
diagnostika, lechenie) [Peripheral vascular disease (Clinic, diagnostics, treatment]. 
\textit{Mezhdunarodnyy zhurnal eksperimental'nogo obrazovaniya} [Int. J.~Research 
Education] 8:113--114. 
   \end{thebibliography}

 }
 }

\end{multicols}

\vspace*{-9pt}

\hfill{\small\textit{Received June 18, 2016}}

\vspace*{-3pt}
    
  
  \Contr
  
  \noindent
  \textbf{Kirikov Igor A.}\ (b.\ 1955)~---
  Candidate of  Sciences (PhD) in technology; director, Kaliningrad Branch of the 
  Federal Research Center ``Computer Science and Control'' of the Russian Academy 
  of Sciences, 5~Gostinaya Str., Kaliningrad 236000,  Russian Federation; 
baltbipiran@mail.ru
  
  \pagebreak
%  \vspace*{3pt}
  
  \noindent
  \textbf{Kolesnikov Alexander V.}\ (b.\ 1948)~---
  Doctor of Sciences in technology; professor, 
Department of Telecommunications, 
 Immanuel Kant Baltic Federal University, 14~Nevskogo Str., Kaliningrad 236041, Russian Federation; senior scientist, Kaliningrad Branch of 
  the Federal Research Center ``Computer Science and Control'' of the Russian 
  Academy of Sciences, 5~Gostinaya Str., Kaliningrad 236000,  Russian Federation; 
  avkolesnikov@yandex.ru
  
  \vspace*{4pt}
  
  \noindent
  \textbf{Listopad Sergey V.}\ (b.\ 1984)~---
  Candidate of  Sciences (PhD) in technology; scientist, Kaliningrad Branch of the 
  Federal Research Center ``Computer Science and Control'' of the Russian Academy 
  of Sciences, 5~Gostinaya Str., Kaliningrad 236000,  Russian Federation;   
ser-list-post@yandex.ru
  
  \vspace*{4pt}
  
  \noindent
  \textbf{Rumovskaya Sophiya B.}\ (b.\ 1985)~--- programmer~I, Kaliningrad Branch 
  of the Federal Research Center ``Computer Science and Control'' of the Russian 
  Academy of Sciences, 5~Gostinaya Str., Kaliningrad 236000,  Russian Federation; 
  sophiyabr@gmail.com
  \label{end\stat}
  
  
  \renewcommand{\bibname}{\protect\rm Литература}