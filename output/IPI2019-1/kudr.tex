\def\stat{kudr}

\def\tit{АПРИОРНЫЕ ФРЕШЕ И~МАСШТАБИРОВАННОЕ ОБРАТНОЕ ХИ-РАСПРЕДЕЛЕНИЕ В БАЙЕСОВСКИХ 
МОДЕЛЯХ БАЛАНСА$^*$}

\def\titkol{Априорные Фреше и~масштабированное обратное хи-распределение в~байесовских 
моделях баланса}

\def\aut{А.\,А.~Кудрявцев$^1$, С.\,И.~Палионная$^2$, В.\,С.~Шоргин$^3$}

\def\autkol{А.\,А.~Кудрявцев, С.\,И.~Палионная, В.\,С.~Шоргин}

\titel{\tit}{\aut}{\autkol}{\titkol}

\index{Кудрявцев А.\,А.}
\index{Палионная С.\,И.}
\index{Шоргин В.\,С.}
\index{Kudryavtsev A.\,A.}
\index{Palionnaia S.\,I.}
\index{Shorgin V.\,S.}


{\renewcommand{\thefootnote}{\fnsymbol{footnote}} \footnotetext[1]
{Работа выполнена при частичной финансовой поддержке 
РФФИ (проект 17-07-00577).}}


\renewcommand{\thefootnote}{\arabic{footnote}}
\footnotetext[1]{Московский государственный 
университет им.\ М.\,В.~Ломоносова, факультет вычислительной математики 
и~кибернетики, \mbox{nubigena@mail.ru}}
\footnotetext[2]{Московский государственный университет 
им.\ М.\,В.~Ломоносова, факультет вычислительной математики и~кибернетики, 
\mbox{sofiapalionnaya@gmail.com}}
\footnotetext[3]{Институт проблем информатики Федерального 
исследовательского центра <<Информатика и~управление>> Российской академии наук, 
\mbox{VShorgin@ipiran.ru}}

\vspace*{8pt}




\Abst{Статья продолжает ряд работ авторов в~области моделирования систем массового 
обслуживания с~применением байесовского подхода. Постановка задачи 
распространяется на более широкий класс прикладных исследований~--- изучение 
индекса баланса факторов, влияющих на функционирование системы. Предполагается, 
что параметры модели разделены на два класса, к~одному из которых относятся те, 
что оказывают позитивное влияние на функционирование сложного агрегата, 
а~к~другому~--- препятствующие функционированию. Эффективность работы исследуемой 
системы, естественно, зависит от соотношения позитивного и~негативного факторов, 
называемого индексом баланса. В~рамках байесовского подхода предполагается, что 
факторы суть случайные величины с~известными априорными распределениями. Во 
многих прикладных задачах свою адекватность демонстрируют распределения из 
гам\-ма-клас\-са. В~статье рассматриваются смеси частных случаев обобщенного 
гам\-ма-рас\-пре\-де\-ле\-ния~--- распределение Фреше и~масштабированное обратное 
хи-рас\-пре\-де\-ле\-ние.}

\KW{байесовский подход; масштабированное обратное хи-рас\-пре\-де\-ле\-ние; 
распределение Фреше; гам\-ма-экс\-по\-нен\-ци\-аль\-ная функция; модели 
баланса; смешанные распределения}

\DOI{10.14357/19922264190109}
  
\vspace*{12pt}


\vskip 10pt plus 9pt minus 6pt

\thispagestyle{headings}

\begin{multicols}{2}

\label{st\stat}

\section{Введение}

В современном мире процессы, ко\-ор\-ди\-ни\-ру\-ющие различные сферы человеческой 
де\-я\-тель\-ности, настолько усложнились, что определение критериев эффективности 
сис\-тем путем детерминированного анализа практически не представляется возможным. 
В~связи с~этим вводятся различные показатели, рейтинги и~индексы, позволяющие 
оценить состояние сис\-те\-мы, сэкономив при этом временн$\acute{\mbox{ы}}$е, материальные 
и~финансовые ре\-сурсы.

Все факторы, так или иначе влияющие на функциониро\-вание системы, можно разделить 
на позитивные (спо\-соб\-ст\-ву\-ющие корректному функционированию системы) и~негативные 
(пре\-пят\-ст\-ву\-ющие работе сис\-те\-мы). Далее для удобства записи\linebreak будем называть их 
{\it p-} и~{\it n-фак\-то\-ра\-ми} соответственно. 
Однако на эффективность 
работы системы влияют не столько абсолютные значения этих факторов, сколько их 
соотношение. Поэтому для исследования функционирования сложных агрегатов, 
включая сложные мо\-ди\-фи\-ци\-ру\-емые 
ин\-фор\-ма\-ци\-он\-но-те\-ле\-ком\-му\-ни\-ка\-ци\-он\-ные сис\-те\-мы, 
естественно перейти к~рассмотрению отношения n- к~p-фак\-то\-ру модели. Такая 
величина $\rho \hm= \lambda/\mu$, где~$\lambda$ и~$\mu$ соответственно обозначают 
n- и~p-фак\-то\-ры, называется {\it индексом баланса}~\cite{Ku2018}.

С течением времени меняется состояние среды, окружающей систему. По этой причине 
n- и~p-фак-\linebreak то\-ры рассматриваемой модели также изменяются, причем зачастую 
непредсказуемым образом.\linebreak Это дает предпосылки для рассмотрения факторов, 
а~следовательно, и~индексов баланса как случайных величин. Стоит также отметить, 
что глобальные изменения окружающей среды происходят крайне редко, поэтому можно 
считать, что законы, оказывающие влияние на факторы и~индексы конкретной модели, 
остаются неизменными. Это позволяет перейти к~применению байесовского метода 
с~заданными априорными распределениями исходных параметров.

Ниже приводятся результаты для вероятностных характеристик индекса баланса 
в~случае, когда в~качестве априорных распределений факторов рассматриваются 
распределение Фреше и~масштабированное обратное хи-рас\-пре\-де\-ле\-ние.

%\vspace*{-6pt}

\section{Основные результаты}


Рассмотрим гамма-экспоненциальную функцию~\cite{KuTi2017}:

\noindent
\begin{multline*}
{\sf Ge}_{\alpha,\, \beta} (x) = \sum\limits_{k=0}^{\infty}
\fr{x^k}{k!}\, \Gamma(\alpha k 
+ \beta)\,, \enskip x\in\mathbb{R}\,, \\
 \alpha\ge0\,, \enskip \beta> 0\,.
\end{multline*}

Для дальнейших вычислений понадобится следующее утверждение.


\smallskip

\noindent
\textbf{Лемма~1.}\
\textit{Пусть $\alpha,\theta,r,s\hm>0$. Тогда}
\begin{multline*}
\int\limits_0^{\infty}y^{-r-1}e^{-(\alpha/y)^s-(\theta/y)^2} \, dy ={}\\
{}=
 \begin{cases}
   \displaystyle \fr{\theta^{-r}}{2}\,{\sf Ge}_{s/2,\,r/2}\left(-
\left(
\fr{\alpha}{\theta}\right)^s\right), & s<2\,;\\[12pt]
   \displaystyle \fr{\alpha^{-r}}{s}\,{\sf Ge}_{2/s,\,r/s}\left(-
\left(\fr{\theta}{\alpha}\right)^{2}\right), & s>2\,;\\[12pt]
   \displaystyle \fr{(\alpha^2+\theta^2)^{-r/2}}{2}\,\Gamma
   \left(\fr{r}{2}\right), & s = 2\,.
 \end{cases}
\end{multline*}


\noindent
Д\,о\,к\,а\,з\,а\,т\,е\,л\,ь\,с\,т\,в\,о\,.\ \
Рассмотрим случай $s<2$. Используя теорему Лебега о предельном переходе, 
получаем:
\begin{multline*}
\int\limits_0^{\infty}y^{-r-1}e^{-(\alpha/y)^s-(\theta/y)^2} \, dy
={}\\
{}=\fr{\alpha^{-r}}{s} \sum\limits_{k=0}^\infty
\fr{(-1)^k}{k!} \int\limits_0^{\infty} 
t^{r/s+k-1}e^{-(\theta/\alpha t^s)^{2}} \,  dt ={}\\
{}=\fr{\theta^{-r}}{2}\sum\limits_{k=0}^\infty
\fr{(-(\alpha/\theta)^s)^{k}}{k!}\int\limits_0^{\infty}z^{(r+sk)/2-1}e^{-z} \,  dz ={}\\
{}=
\fr{\theta^{-r}}{2}\sum\limits_{k=0}^\infty
\fr{(-(\alpha/\theta)^s)^{k}}{k!}\,\Gamma\left(\fr{r+sk}{2}\right).
\end{multline*}

Случай $s>2$ рассматривается аналогично. Случай $s\hm=2$ напрямую следует из 
определения гам\-ма-функ\-ции.
Лемма доказана.

\smallskip

Рассмотрим случайную величину~$\xi$, име\-ющую масштабированное обратное 
хи-рас\-пре\-де\-ле\-ние $I\chi(q,\theta)$ с~плот\-ностью
$$
f_\xi(x)=\fr{2\theta^{q}\, e^{-\theta^2/x^2}}{\Gamma({q}/{2})x^{q +1}}\,,\enskip 
\theta>0\,,\ \ q>0\,,\ \ x>0\,,
$$
и случайную величину~$\eta$, имеющую распределение Фреше $\mathrm{Fr}\,(u,\alpha)$~\cite{Frechet1927} 
с~нулевым минимумом, плотность которого имеет вид:
$$
f_\eta(x)=\fr{u \alpha^{u}e^{-(\alpha/x)^u}}{x^{u +1}}\,,\enskip \ \alpha>0\,, \ \ 
u>0\,,\ \ x>0\,.
$$

Легко видеть, что для случайных величин~$\xi$ и~$\eta$ справедливо следующее 
утверждение.

\smallskip

\noindent
\textbf{Лемма~2.}\
\textit{Для случайных величин~$\xi$ и~$\eta$, имеющих соответственно масштабированное 
обратное хи-рас\-пре\-де\-ле\-ние $I\chi(q,\theta)$ и~распределение Фреше 
$\mathrm{Fr}\,(u,\alpha)$, для $z\hm\in\mathbb{R}$ выполняются соотношения}:
\begin{align*}
{\sf E}\xi^z&=\fr{\theta^{z} \Gamma(q/2-z/2)}{\Gamma(q/2)}\,,\ \  z<q\,;\\[6pt]
{\sf E}\eta^z&={\alpha^{z} \Gamma(1-z/u)}\,,\ \ z<u\,.
\end{align*}


\smallskip

Леммы 1 и~2 дают возможность вычислить вероятностные характеристики индекса 
баланса~$\rho$ в~байесовской модели.

\smallskip

\noindent
\textbf{Теорема~1.}\
\textit{Пусть n-фак\-тор~$\lambda$ имеет масштабированное обратное хи-рас\-пре\-де\-ле\-ние 
$I\chi(q, \theta)$, а p-фак\-тор~$\mu$ имеет распределение Фреше $\mathrm{Fr}\,(u,\alpha)$, 
причем~$\lambda$ и~$\mu$ независимы. Тогда при $x\hm>0$ плотность, функция 
распределения и~моменты индекса баланса $\rho\hm=\lambda/\mu$ имеют вид}:
\begin{multline*}%\label{f_rho_GW}
f_\rho(x) ={}\\
{}=
 \begin{cases}
   \displaystyle \fr{2\theta^{q}}{\alpha^{q}\,\Gamma(q/2)x^{q+1}}{\sf Ge}_{2/u,\, 
q/u+1}\!\left(-\!\left(\fr{\theta}{\alpha x}\right)^{\!2}\right)\!,& \!\!\! u>2;\hspace*{-8pt}\\[16pt]
   \displaystyle \fr{u\alpha^{u}x^{u-1}}{\theta^{u}\,\Gamma(q/2)}{\sf Ge}_{u/2,\, 
(q+u)/2}\left(-\left(\fr{\alpha x}{\theta}\right)^{u}\right)\!, 
& \!\!\!u<2;\hspace*{-8pt}%
\\
%   \displaystyle q\alpha^2 \theta^{q}x(\theta^2+\alpha^2 x^2)^{-q/2-1}, 
%&u = 2\,;
 \end{cases}
\end{multline*}

\vspace*{-12pt}

\noindent
\begin{multline*}
F_\rho(x) ={}\\
{}=
 \begin{cases}
   \displaystyle \fr{q }{2\Gamma(q/2)}\int\limits_{(\theta/(\alpha 
x))^{q}}^\infty \hspace*{-10pt}{\sf Ge}_{2/u,\,q/u+1}(-z^{2/q}) \, dz, & u>2;\hspace*{-2pt}\\[16pt]
   \displaystyle 1 - \fr{{\sf Ge}_{u/2,q/2}(-(\alpha x/\theta)^u)}{\Gamma(q/2)}, 
&u<2;\hspace*{-2pt}\\
 \end{cases}
\end{multline*}
$$
{\sf E}\rho^z = \fr{(\theta/\alpha)^{z} \Gamma(q/2-
z/2)\Gamma(z/u+1)}{\Gamma(q/2)}\,, \enskip z<q\,.
$$
\textit{При $u=2$ распределение индекса баланса $\rho$ совпадает с~масштабированным 
распределением Бурра}~\cite{Burr1942} 
\textit{с~параметрами} $(2,q/2,\theta/\alpha)$.

\smallskip

\noindent
Д\,о\,к\,а\,з\,а\,т\,е\,л\,ь\,с\,т\,в\,о\,.\ \ Поскольку
$$
f_\rho(x) =
\int\limits_0^\infty \fr{2u \alpha^{u}\theta^{q} e^{-(\theta/(xy))^2-
(\alpha/y)^u}}{\Gamma(q/2)x^{q +1}y^{q+u +1}}\, dy\,,
$$
вид плотности $f_\rho(x)$ вытекает из леммы~1 для всех $u\hm>0$.

Для функции распределения~$\rho$ при $u\hm>2$ справедливо для $x\hm>0$
\begin{multline*}
F_\rho(x)
={}\\
{}=\fr{2(\theta/\alpha)^q}{\Gamma(q/2)}\int\limits_{0}^{x}  y^{-q-1}{\sf Ge}_{2/u,\, 
q/u+1}\left(-\left(\fr{\theta}{\alpha y}\right)^2\right)\,dy={}\\
{}=\fr{2(\theta/\alpha)^{q}}{\Gamma(q/2)}\int\limits_0^x { y^{-q-
1}}\times{}\\
{}\times \sum\limits_{l=0}^\infty \fr{(-1)^ly^{- 2l}}{(\theta/\alpha)^{-
2l}l!}\Gamma\left(\fr{2l}{u}+\fr{q}{u}+1\right)\, dy={}\\
{}=\fr{2}{q\Gamma(q/2)}\int\limits_{(\theta/(\alpha x))^{q}
}^\infty \hspace*{-1.84pt}
\sum\limits_{l=0}^\infty \fr{\left(-
z^{2/q}\right)^l}{l!}\Gamma\left(\!\fr{2l}{u}+\fr{q}{u}+1\!\right) dz.\hspace*{-6.7852pt}
\end{multline*}
В случае $u<2$ имеем:
\begin{multline*}
F_\rho(x) =\fr{u(\alpha/\theta)^{u}}{\Gamma(q/2)}\times{}\\
{}\times\int\limits_{0}^{x} y^{u-1}
\sum\limits_{l=0}^{\infty}\fr{(-(\alpha y/\theta)^{u})^l}{l!}\, \Gamma\left(\fr{lu 
+ u+q}{2}\right)
\, dy={}\\
{}=\fr{u (\alpha/\theta)^{u}}{\Gamma(q/2)}\sum\limits_{l=0}^\infty 
\fr{(-1)^l}{(\theta/\alpha)^{u l}l!}\Gamma\left(\fr{lu+u+q}{2}\right)\times{}\\
{}\times\int\limits_0^x 
{ y^{u l+u-1}}\, dy={}\\
{}=\fr{x^{u} (\alpha/\theta)^{u}}{\Gamma(q/2)}\sum\limits_{l=0}^\infty 
\fr{(-1)^lx^{u l}}{(\theta/\alpha)^{u l}l!}\,\fr{\Gamma((lu+u+q)/2)  }{ l+ 1}={}\\
{}=1-\fr{1}{\Gamma(q/2)}\sum\limits_{m=0}^\infty \fr{(-1)^mx^{u 
m}}{(\theta/\alpha)^{u m}m!}\,\Gamma\left(\fr{mu+q}{2}\right).
\end{multline*}

Для нахождения моментов~$\rho$ достаточно воспользоваться независимостью 
случайных величин~$\lambda$ и~$\mu$ и~леммой~2.
Теорема доказана.

\smallskip

Рассмотрим симметричный случай априорных распределений.

\smallskip

\noindent
\textbf{Теорема~2.}\
\textit{Пусть n-фак\-тор~$\lambda$ имеет распределение Фреше $\mathrm{Fr}\,(v,\theta)$, 
а~p-фак\-тор~$\mu$ имеет масштабированное обратное хи-рас\-пре\-де\-ле\-ние $I\chi(p, \alpha)$, 
причем~$\lambda$ и~$\mu$ независимы. Тогда при $x\hm>0$ плотность, функция 
распределения и~моменты индекса баланса $\rho\hm=\lambda/\mu$ имеют вид}:

\noindent
\begin{multline*}%\label{f_rho_GW}
f_\rho(x) ={}\\
{}=
 \begin{cases}
   \displaystyle \fr{v\theta^{v}}{\alpha^{v}\Gamma(p/2)x^{v+1}}\,{\sf Ge}_{v/2,\, 
(v+p)/2}\left(-\left(\fr{\theta}{\alpha x}\right)^{v}\right), & \\
&\hspace*{-25pt}v<2;\hspace*{-0.96239pt}\\[3pt]
  \displaystyle \fr{2\alpha^{p}x^{p-1}}{\theta^{p}\Gamma(p/2)}\,{\sf Ge}_{2/v,\, 
p/v+1}\left(-\left(\fr{\alpha x}{\theta}\right)^2\right), &\hspace*{-25pt}v>2\,;\hspace*{-0.96239pt}\\
%   p\theta^2\alpha^{p} x^{p-1}(\theta^2+\alpha^2 x^2)^{-p/2-1}, &\text{$v=2$};
 \end{cases}
\end{multline*}

\vspace*{-12pt}

\noindent
\begin{multline*}
F_\rho(x) ={}\\
{}=
 \begin{cases}
   \displaystyle \fr{1 }{\Gamma(p/2)}\int\limits_{(\theta/(\alpha 
x))^{v}}^\infty \hspace*{-6pt}{\sf Ge}_{v/2,\,(v+p)/2}(-z) \, dz, &v<2\,;\hspace*{-1.62056pt}\\[6pt]
   \displaystyle \fr{2\alpha^{p} 
x^{p}}{v\theta^{p}\Gamma(p/2)}\,{\sf Ge}_{2/v,p/v}\left(-\left(\fr{\alpha 
x}{\theta}\right)^2\right), &v>2\,;\hspace*{-1.62056pt}
 \end{cases}
\end{multline*}
$$
{\sf E}\rho^z =\fr{(\theta/\alpha)^{z} \Gamma(p/2+z/2)\Gamma(1-
z/v)}{\Gamma(p/2)}\,, \enskip z<v\,.
$$
При $v=2$ распределение индекса баланса~$\rho$ совпадает с~распределением 
Дагума~\cite{Dagum1977} с~параметрами $(2,\theta/\alpha,p/2)$.


\smallskip

\noindent
Д\,о\,к\,а\,з\,а\,т\,е\,л\,ь\,с\,т\,в\,о\,.\ \
 Аналогично предыдущей тео\-ре\-ме для получения 
выражения для плотности~$\rho$ при всех $v\hm>0$ достаточно воспользоваться 
леммой~1.

Для функции распределения $\rho$ при $v\hm<2$ справедливо для $x\hm>0$:
\begin{multline*}
F_\rho(x)  =\fr{v(\theta/\alpha)^{v}}{ \Gamma(p/2)}\int\limits_{0}^{x}  y^{-v-1}\times{}
\\
{}\times
\sum\limits_{k=0}^{\infty}\fr{(-\theta^{v})^k}{(\alpha y)^{vk} k!} \,
\Gamma\left(\fr{kv + v+p}{2}\right)
\, dy={}\\
{}=\fr{1 }{\Gamma(p/2)}\int\limits_{(\theta/\alpha)^{v}x^{-v}}^\infty 
\sum\limits_{k=0}^\infty \fr{(-z)^k}{k!}\,\Gamma\left(\fr{kv+v+p}{2}\right) \, dz\,.
\end{multline*}
В случае $v>2$ имеем:
\begin{multline*}
F_\rho(x) =\fr{2(\alpha/\theta)^p}{\Gamma(p/2)}\int\limits_{0}^{x} y^{p-1}\times{}\\
{}\times
\sum\limits_{k=0}^{\infty}\fr{(-1)^k (\alpha y)^{2k}}{\theta^{2k}k!} \,
\Gamma\left(\fr{2k}{v} +\fr{p}{v}+1\right)
\, dy={}\\
{}=\fr{2(\alpha/\theta)^{p} }{\Gamma(p/2)}\sum\limits_{k=0}^\infty 
\fr{(-
1)^k\alpha^{ 2k}}{\theta^{ 2k}k!}\,\Gamma\left(\fr{2k}{v}+\fr{p}{v}+1\right) 
\fr{ x^{ 2k+p}}{ 2k+ p}={}\\
{}=\fr{2(\alpha/\theta)^{p}x^{p} }{v\Gamma(p/2)}\sum_{k=0}^\infty 
\fr{(-1)^kx^{ 2k}}{(\theta/\alpha)^{ 2k}k!}\,\Gamma\left(\fr{2k+p}{v}\right).
\end{multline*}

Соотношение для моментов следует из леммы~2 и~независимости случайных величин~$\lambda$ 
и~$\mu$.
Теорема доказана.

\smallskip

\noindent
\textbf{Замечание.}\ Очевидно, что теоремы~1 и~2 несложно переформулировать для 
случая смеси априорного распределения Фреше с~масштабированным обратным 
$\chi^2$-рас\-пре\-де\-ле\-ни\-ем, име\-ющим плотность
$$
f(x)=\fr{\theta^{q/2} e^{-\theta/x}}{\Gamma({q}/{2})x^{q/2 +1}}\,,\enskip 
\theta>0\,,\ \ q>0\,,\ \ x>0\,,
$$
и являющимся частным случаем обратного гамма-рас\-пре\-де\-ле\-ния, рассмотренного 
в~\cite{KuPaSh2018}.


\section{Заключение}

Методы, использованные при получении результатов данной работы, дают возможность 
исследовать отношения случайных величин, имеющих обратные распределения из 
гам\-ма-клас\-са, в~терминах гам\-ма-экс\-по\-нен\-ци\-аль\-ной 
функции~${\sf Ge}_{\alpha,\, \beta} 
(x)$ в~случаях, когда распределение смеси отлично от обобщенного бе\-та-рас\-пре\-де\-ле\-ния 
второго рода.

{\small\frenchspacing
 {%\baselineskip=10.8pt
 \addcontentsline{toc}{section}{References}
 \begin{thebibliography}{9}

\bibitem{Ku2018}
\Au{Кудрявцев~А.\,А.}
Байесовские модели баланса~// Информатика и~её применения, 2018. Т.~12. Вып.~3. С.~18--27.

\bibitem{KuTi2017}
\Au{Кудрявцев~А.\,А., Титова~А.\,И.}
Гам\-ма-экс\-по\-нен\-ци\-аль\-ная функция в~байесовских моделях массового обслуживания~// 
Информатика и~её применения, 2017. Т.~11. Вып.~4. С.~104--108.

\bibitem{Frechet1927}
\Au{Fr$\acute{\mbox{e}}$chet~M.}
Sur la loi de probabilit$\acute{\mbox{e}}$ de l'$\acute{\mbox{e}}$cart maximum~// 
Ann. Soc. Polonaise Math., 1927. Vol.~6. P.~93--116.

\bibitem{Burr1942}
\Au{Burr~I.\,W.}
Cumulative frequency functions~// Ann. Math. Stat., 1942. Vol.~13. No.\,2. P.~215--232.

\bibitem{Dagum1977}
\Au{Dagum~C.}
A~new model of personal income-distribution-specification and estimation~// 
Econ. Appl., 1977. Vol.~30. No.\,3. P.~413--437.


\bibitem{KuPaSh2018}
\Au{Кудрявцев~А.\,А., Палионная~С.\,И., Шоргин~В.\,С.}
Априорное обратное гам\-ма-рас\-пре\-де\-ле\-ние в~байесовских моделях массового 
обслуживания~// Системы и~средства информатики, 2018. Т.~28. №\,4. С.~52--58.
 \end{thebibliography}

 }
 }

\end{multicols}

\vspace*{-3pt}

\hfill{\small\textit{Поступила в~редакцию 28.12.18}}

\vspace*{8pt}

%\pagebreak

%\newpage

%\vspace*{-28pt}

\hrule

\vspace*{2pt}

\hrule

%\vspace*{-2pt}

\def\tit{\textit{A PRIORI} FRECHET AND~SCALED INVERSE CHI DISTRIBUTION IN~BAYESIAN BALANCE MODELS}


\def\titkol{\textit{A priori} Frechet and~scaled inverse chi distribution in~Bayesian balance models}

\def\aut{A.\,A.~Kudryavtsev$^1$, S.\,I.~Palionnaia$^1$, and~V.\,S.~Shorgin$^2$}

\def\autkol{A.\,A.~Kudryavtsev, S.\,I.~Palionnaia, and~V.\,S.~Shorgin}

\titel{\tit}{\aut}{\autkol}{\titkol}

\vspace*{-11pt}


\noindent
$^1$Department of Mathematical Statistics, Faculty of Computational Mathematics and 
Cybernetics, M.\,V.~Lo-\linebreak
$\hphantom{^1}$monosov Moscow State University, 1-52~Leninskiye Gory, GSP-1, 
Moscow 119991, Russian Federation


\noindent
$^2$Institute of Informatics Problems, Federal Research Center 
``Computer Science and Control'' of the Russian\linebreak
$\hphantom{^1}$Academy of Sciences, 
44-2~Vavilov Str., Moscow 119333, Russian Federation

\def\leftfootline{\small{\textbf{\thepage}
\hfill INFORMATIKA I EE PRIMENENIYA~--- INFORMATICS AND
APPLICATIONS\ \ \ 2019\ \ \ volume~13\ \ \ issue\ 1}
}%
 \def\rightfootline{\small{INFORMATIKA I EE PRIMENENIYA~---
INFORMATICS AND APPLICATIONS\ \ \ 2019\ \ \ volume~13\ \ \ issue\ 1
\hfill \textbf{\thepage}}}

\vspace*{6pt}


\Abste{This article continues a~series of authors' works in the field of modeling 
queuing systems using the Bayesian approach. The problem's statement is extended 
to a~wider class of applied research~--- the study of the balance index of factors 
affecting the functioning of the system. It is assumed that the model parameters 
are divided into two classes, one of which includes factors that have 
a~positive impact on the functioning of a~complex aggregate and the other includes 
those that interferes with the functioning. The effectiveness of the system 
under study, of course, depends on the ratio of positive and negative factors, 
called the balance index. In the framework of the Bayesian approach, it is 
assumed that the factors are random variables with known \textit{a~priori} 
distributions. In a wide range of applied problems, it is reasonable 
to use gamma-type distributions. In this paper, the mixtures of particular 
generalized gamma distribution cases~--- 
the Frechet distribution and the scaled inverse chi distribution~--- are considered.}

\KWE{Bayesian approach; scaled inverse chi distribution; 
Frechet distribution; gamma-exponential function; 
balance models; mixed distributions}



\DOI{10.14357/19922264190109}

%\vspace*{-14pt}

\Ack
\noindent
The work was partly supported by the Russian Foundation for Basic Research 
(project 17-07-00577).

\pagebreak


%\vspace*{6pt}

  \begin{multicols}{2}

\renewcommand{\bibname}{\protect\rmfamily References}
%\renewcommand{\bibname}{\large\protect\rm References}

{\small\frenchspacing
 {%\baselineskip=10.8pt
 \addcontentsline{toc}{section}{References}
 \begin{thebibliography}{9}
\bibitem{1-kudr-1}
\Aue{Kudryavtsev, A.\,A.} 2018. Bayesovskie modeli balansa 
[Bayesian balance models]. \textit{Informatika i~ee Primeneniya~--- Inform. Appl.}
12(3):18--27.

\bibitem{2-kudr-1}
\Aue{Kudryavtsev, A.\,A., and A.\,I.~Titova.} 2017. Gamma-eksponentsial'naya 
funktsiya v~bayesovskikh modelyakh massovogo obsluzhivaniya 
[Gamma-exponential function in Bayesian queuing models]. 
\textit{Informatika i~ee Primeneniya~--- Inform. Appl.} 11(4):104--108.

\bibitem{3-kudr-1}
\Aue{Fr$\acute{\mbox{e}}$chet, M.} 1927. 
Sur la loi de probabilit$\acute{\mbox{e}}$ de l'$\acute{\mbox{e}}$cart maximum. 
\textit{Ann. Soc. Polonaise 
Math.} 6:93--116.

\columnbreak

\bibitem{4-kudr-1}
\Aue{Burr, I.\,W.} 1942. Cumulative frequency functions. 
\textit{Ann. Math. Stat.} 13(2):215--232.

\bibitem{5-kudr-1}
\Aue{Dagum, C.} 1977. 
A~new model of personal income-distribution-specification and estimation. 
\textit{Econ. Appl.} 30(3):413--437.

\bibitem{6-kudr-1}
\Aue{Kudryavtsev, A.\,A., S.\,I.~Palionnaia, and V.\,S.~Shorgin.} 
2018. Apriornoe obratnoe gamma-raspredelenie v~bayesovskikh modelyakh 
massovogo obsluzhivaniya 
[\textit{A~priori} inverse gamma distribution in Bayesian queuing models]. 
\textit{Sistemy i~Sredstva Informatiki~--- Systems and Means of Informatics}
28(4):52--58.
\end{thebibliography}

 }
 }

\end{multicols}

\vspace*{-6pt}

\hfill{\small\textit{Received December 28, 2018}}

%\pagebreak

%\vspace*{-18pt}

\Contr


\noindent
\textbf{Kudryavtsev Alexey A.} (b.\ 1978)~--- 
Candidate of Science (PhD) in physics and mathematics, 
associate professor, Department of Mathematical Statistics, 
Faculty of Computational Mathematics and Cybernetics, M.\,V.~Lomonosov Moscow 
State University, 1-52~Leninskiye Gory, GSP-1, Moscow 119991, Russian Federation; 
\mbox{nubigena@mail.ru}

\vspace*{3pt}

\noindent
\textbf{Palionnaia Sofia I.} (b.\ 1995)~--- 
student, Faculty of Computational Mathematics and Cybernetics,
 M.\,V.~Lomonosov Moscow State University, 1-52~Leninskiye Gory, GSP-1, Moscow 119991, 
 Russian Federation; \mbox{sofiapalionnaya@gmail.com}
 
 \vspace*{3pt}

\noindent
\textbf{Shorgin Vsevolod S.} (b.\ 1978)~--- Candidate of Science (PhD) in technology, 
senior scientist, Institute of Informatics Problems, Federal Research Center 
``Computer Science and Control'' of the Russian Academy of Sciences, 
44-2~Vavilov Str., Moscow 119333, Russian Federation; \mbox{VShorgin@ipiran.ru}


\label{end\stat}

\renewcommand{\bibname}{\protect\rm Литература}       