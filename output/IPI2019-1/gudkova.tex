\def\stat{gudkova}

\def\tit{СРАВНИТЕЛЬНЫЙ АНАЛИЗ ПОКАЗАТЕЛЕЙ ЭФФЕКТИВНОСТИ МОДЕЛИ БЕСПРОВОДНОЙ 
СЕТИ МЕЖМАШИННОГО ВЗАИМОДЕЙСТВИЯ, РАБОТАЮЩЕЙ В~РАМКАХ ДВУХ ПОЛИТИК РАЗДЕЛЕНИЯ РАДИОРЕСУРСОВ$^*$}

\def\titkol{Сравнительный анализ показателей эффективности модели беспроводной 
сети межмашинного взаимодействия}
%, работающей в~рамках двух политик разделения радиоресурсов}

\def\aut{Е.\,В.~Маркова$^1$, А.\,А.~Гольская$^2$, И.\,Л.~Дзантиев$^3$, И.\,А.~Гудкова$^4$, 
С.\,Я.~Шоргин$^5$}

\def\autkol{Е.\,В.~Маркова, А.\,А.~Гольская, И.\,Л.~Дзантиев и др.}
%$^3$ И.\,А.~Гудкова$^4$,  С.\,Я.~Шоргин$^5$}

\titel{\tit}{\aut}{\autkol}{\titkol}

\index{Е.\,В.~Маркова$^1$, А.\,А.~Гольская$^2$, И.\,Л.~Дзантиев$^3$ И.\,А.~Гудкова$^4$, 
С.\,Я.~Шоргин$^5$}
\index{E.\,V.~Markova, A.\,A.~Golskaia, I.\,L.~Dzantiev, et al.} 
%I.\,A.~Gudkova$^{1,2}$, and~S.\,Ya.~Shorgin}


{\renewcommand{\thefootnote}{\fnsymbol{footnote}} \footnotetext[1]
{Публикация подготовлена при финансовой поддержке Минобрнауки России 
(проект 2.3397.2017/4.6).}}


\renewcommand{\thefootnote}{\arabic{footnote}}
\footnotetext[1]{Российский университет дружбы народов, markova\_ev@rudn.university}
\footnotetext[2]{Российский университет дружбы народов, golskaya\_aa@rudn.university}
\footnotetext[3]{Российский университет дружбы народов, dzantiev\_il@rudn.university}
\footnotetext[4]{Российский университет дружбы народов; Институт проблем информатики Федерального исследовательского 
центра <<Информатика и~управ\-ле\-ние>> Российской академии наук, gudkova\_ia@rudn.university}
\footnotetext[5]{Институт проблем информатики Федерального исследовательского центра 
<<Информатика и~управ\-ле\-ние>> Российской академии наук, \mbox{sshorgin@ipiran.ru}}

\vspace*{-4pt}




\Abst{В~настоящее время информационно-коммуникационные технологии (ИКТ) все глубже 
проникают во многие области современной жизни. Например, концепция интеграции ИКТ 
и~Интернета вещей (Internet of Things, IoT) для управления городской инфраструктурой 
<<Умный город>> позволяет городской власти следить за изменениями и~ситуацией в~городе 
с~помощью датчиков. При этом специализированные системы осуществляют сбор данных 
в~автоматическом режиме без участия человека. Важным параметром при определении 
показателей эффективности беспроводных сетей межмашинного взаимодействия 
(Machine-to-Machine, M2M)~--- скорости передачи данных, вероятности блокировки~--- служит 
удаленность устройства (датчика) от радиопередающей аппаратуры (базовой станции, БС). 
Поэтому при описании такой сети в~виде сис\-те\-мы массового обслуживания с~потоковым 
(гарантированная ско\-рость передачи данных) или эластичным (негарантированная скорость) 
трафиком необходимо рассматривать входящий поток запросов на передачу данных от 
M2M-устройств таким образом, чтобы учесть расположение устройств относительно БС. 
Представлена модель соты беспроводной сети со стационарными M2M-устрой\-ст\-ва\-ми, 
находящимися либо в~пассивном, либо в~активном со\-сто\-янии. Устройства описываются 
точками, случайно возникающими на плос\-кости, и~генерируют потоковый трафик, скорость 
передачи которого зависит от расположения устройства в~соте, его мощности и~уровня шума. 
Состояние системы описывает вектор переменной длины, компонентами которого служат 
расстояния от каждого активного устройства до БС. Рассмотрены две политики управ\-ле\-ния 
радиоресурсами~--- \textit{round robin} (RR) и~\textit{full power} (FP), отличающиеся друг от друга 
распределением временного интервала обслуживания M2M-устрой\-ст\-ва 
и~предоставляемой ско\-ростью передачи данных. Проведен сравнительный анализ значений 
вероятности блокировки запроса на передачу данных.}

\KW{беспроводная сеть; LTE; устройство межмашинного взаимодействия; индикатор 
качества канала; формула Шеннона; равномерное распределение; политика циклического 
обслуживания; политика обслуживания на максимальной мощности; вероятность 
блокировки}

\DOI{10.14357/19922264190115}
  
%\vspace*{4pt}


\vskip 10pt plus 9pt minus 6pt

\thispagestyle{headings}

\begin{multicols}{2}

\label{st\stat}

\section{Введение}

\vspace*{-2pt}

  В связи со стремительным ростом числа используемых M2M-устройств~[1] 
в~широком спектре отраслей они становятся все более распространенными. 
Согласно прогнозу Cisco~[2], количество M2M-со\-еди\-не\-ний составит более 
половины от общего числа подключенных устройств и~соединений (51\%) 
и~достигнет значения 14,6~млрд к~2022~г. 

Для обслуживания  
M2M-при\-ло\-же\-ний, генерирующих в~совокупности большой объем данных с~необходимым уровнем QoS (Quality of Service), требуется развитие 
существующих и~разработка новых технологий. Отметим, что для различных 
типов M2M-приложений эти технологии значительно отличаются~[3]. 

Первый 
тип приложений включает в~себя смарт-из\-ме\-ре\-ния, мобильный трекинг, 
электронное здравоохранение и~т.\,д.\ и~характеризуется малыми объемами 
данных и~длительными задержками, а~также низкой стоимостью 
и~энергопотреблением. 

Второй тип приложений относится к~связи между 
подвижными объектами и~характеризуется небольшими объемами данных, 
короткими задержками и~высокой надежностью. 

Третий тип относится 
к~видеонаблюдению, для которого характерны большие объемы данных 
и~высокая скорость их передачи. 
  
  Стремительный рост объема трафика, генерируемого  
M2M-при\-ло\-же\-ни\-ями~[4, 5], в~конечном \mbox{итоге} может привести к~нехватке 
частотного диапазона~[6]. Одним из возможных решений этой проб\-ле\-мы 
может стать более эффективное использование частотного ресурса, а~именно 
использование более высоких диапазонов час\-тот~[7]. Наиболее эффективно 
использовать частотный диапазон помогают различные политики управления 
час\-тот\-но-вре\-мен\-н$\acute{\mbox{ы}}$\-ми ресурсами, например политика, основанная на 
равномерном распределении временн$\acute{\mbox{о}}$го ресурса между всеми 
обслуживаемыми M2M-устрой\-ст\-ва\-ми (RR policy), или политика, 
основанная на передаче данных с~максимальной мощностью (FP policy). 
  
  В статье предложена модель соты беспроводной сети LTE со стационарными 
M2M-устрой\-ст\-ва\-ми, генерирующими потоковый трафик и~способными 
находиться в~двух состояниях: пассивном или активном. Устройства 
описываются точками, случайно возникающими на плоскости и~получающими 
различные значения индикатора качества канала CQI (Channel Quality Indicator) 
в~соответствии с~их удаленностью от БС~[8]. Параметр CQI может принимать 
значения от~0 до~15. Чем выше значение, тем лучше~--- тем выше скорость, 
которую может выделить БС оператора LTE. В~связи с~этим все устройства 
соты можно разделить по значению соответствующего CQI на~15~групп. 

В~работе рассмотрены две политики управления радиоресурсами, 
учитывающие значение CQI. Проведено по\-стро\-ение математической модели 
для случая конечной мощности. Для обеих политик проведен анализ 
вероятности блокировки запросов устройств на передачу данных.
  
\section{Описание модели}

  Рассмотрим соту сети беспроводной передачи данных радиуса~$R$ 
с~равномерно распределенными по территории стационарными  
M2M-устрой\-ст\-ва\-ми. С~интенсивностью~$\lambda$ устройства переходят 
из пассивного в~активное состояние и~передают данные в~восходящем канале 
(uplink channel) со средним временем передачи~$\mu^{-1}$. Пусть случайная 
величина (СВ) $\eta\hm=1,\ldots ,L$ определяет значение CQI, соответствующее 
каждому M2M-устрой\-ст\-ву. Тогда расстояние от устройства до БС 
определяется как $\xi_d(\eta)\hm=RL^{-1}\eta$. Обозначим максимальную 
мощность передачи сигнала M2M-устрой\-ст\-вом через~$p_{\max}$. Текущая 
мощ\-ность передачи пред\-став\-ля\-ет собой СВ~$\xi_p$, которая удовлетворяет 
условию $\xi_p\hm\leq p_{\max}$. Предположим, что устройства передают 
данные с~гарантированной скоростью~$r_0$. Будем считать, что максимальная 
мощность сигнала одинакова для всех устройств. Отметим, что 
рассматриваются идеальные условия, так называемая среда free  
space~\cite{9-gud}. 
  
  Предположим, что M2M-устрой\-ст\-ва распределены в~соте равномерно, 
тогда функция распределения (ФР) ${F}_{\xi_d(\eta)}(d)\hm= 
{\sf P}\{\xi_d(\eta)\leq d\}$ СВ~$\xi_d(\eta)$ равна~0 при $d\hm<0$, 
$d^2/R^2$ при $0\hm < d\hm < R$ и~1 при $d\hm > R$, 
а~соответствующая плотность распределения $f_{\xi_d(\eta)}(d)\hm= 2d/R^2$, 
$0\hm\leq d\hm\leq R$.
  
  Согласно теореме Шеннона достижимая M2M-устрой\-ст\-вом скорость 
передачи данных $r(\xi_d(\eta),\xi_p)$ зависит от ширины полосы 
частот~$\omega$ восходящего канала, мощности передачи сигнала~$\xi_p$, 
уда\-лен\-ности~$\xi_d(\eta)$ от БС и~определяется следующим образом:
  \begin{multline*}
  r\left( \xi_d(\eta),\xi_p\right) = 
  \omega\ln \left( 1+\fr{G\xi_p}{\xi_d^\kappa(\eta) 
N_0}\right) ={}\\
{}=\omega \ln \left( 1+ \fr{G\xi_p}{(R\eta/L)^\kappa N_0}\right)\,,
  %\label{e1-gud}
  \end{multline*}
где $\omega$~--- ширина полосы пропускания; $N_0$~--- мощность шума; 
$G$~--- коэффициент затухания сигнала; $\kappa$~--- степень затухания 
сигнала.    

  Рассмотрим две политики управления радиоресурсами, позволяющие учесть 
удаленность устройства от БС. Первая~--- RR policy~--- 
характеризуется разбиением временн$\acute{\mbox{о}}$го ресурса на равные интервалы. 
Вторая~--- FP policy~--- передачей сигнала с~максимальной мощностью. 
Проиллюстрируем функционирование обеих политик в~зависимости от 
расположения устройств~\cite{10-gud, 11-gud, 12-gud}, а~следовательно, 
и~значений CQI.

\setcounter{figure}{1}
\begin{figure*}[b] %fig2
\vspace*{1pt}
 \begin{center}  
  \mbox{%
 \epsfxsize=163mm 
 \epsfbox{gud-2.eps}
 }
\end{center}
\vspace*{-11pt}
\Caption{Случаи~I~(\textit{а}) и~II~(\textit{б}) для политики RR}
%\end{figure*}
%\setcounter{figure}{2}
%\begin{figure*} %fig3
%\vspace*{6pt}
 \begin{center}  
  \mbox{%
 \epsfxsize=163mm 
 \epsfbox{gud-4.eps}
 }
\end{center}
\vspace*{-11pt}
\Caption{Случаи~I~(\textit{а}) и~II~(\textit{б}) для политики FP}
\end{figure*}

  
  На рис.~1 первое устройство расположено на расстоянии 
$\xi_d(\eta_1)\hm=d_1$, второе~--- на расстоянии $\xi_d(\eta_2)\hm=d_2$ от БС. 
А~третье устройство может быть расположено либо близко к~БС, на расстоянии 
$\xi_d(\eta_1^I)\hm= d_3^I$ (случай~I), либо далеко~--- $\xi_d(\eta_1^{II})\hm= 
d_3^{II}$ (случай~II).



   Рассмотрим политику RR. Временной ресурс делится поровну 
между всеми пользовательскими устройствами, которые, в~свою очередь, 
регулируют мощность передачи для обеспечения гарантированной 
скорости~$r_0$, т.\,е.\ при переходе M2M-устрой\-ст\-ва в~активное состояние 
все остальные устройства, находящиеся также в~активном состоянии 
и~передающие данные, в~связи с~уменьшением временн$\acute{\mbox{о}}$го интервала, 
отведенного им на обслуживание, вынуждены увеличить мощность передачи 
данных для\linebreak\vspace*{-12pt}

%\pagebreak

{ \begin{center}  %fig1
 \vspace*{-3pt}
  \mbox{%
 \epsfxsize=79mm 
 \epsfbox{gud-1.eps}
 }


\end{center}


\noindent
{{\figurename~1}\ \ \small{Пример расположения M2M-устройства в~соте}}
}

\vspace*{9pt}

\addtocounter{figure}{1} 

\noindent
 достижения гарантированной скорости~$r_0$. В~противном случае 
запрос на обслуживание блокируется и~M2M-устрой\-ст\-во остается 
в~пассивном со\-сто\-янии. 
  
  Рассмотрим случай~I, когда новое устройство находится близко к~БС 
(рис.~2,\,\textit{а}). Близкое расположение позволяет увеличить мощность других 
устройств, не превышая ее максимального значения~$p_{\max}$, и~принять 
запрос на обслуживание нового устройства. В~случае сильной удаленности от 
БС (случай~II) увеличения мощности M2M-устройств до максимального 
значения~$p_{\max}$ недостаточно для достижения гарантированной ско\-рости 
и~запрос на обслуживание устройства блокируется (рис.~2,\,\textit{б}). 
  



  Далее рассмотрим поведение системы при использовании политики 
\textit{full power}, при которой все M2M-устрой\-ст\-ва всегда работают на 
максимальной мощности, а временной ресурс распределяется пропорционально 
достижимой каждым устройством скорости~$r_i$, где~$i$~--- номер 
устройства. В~случае~I временного интервала достаточно для достижения 
гарантированной скорости, поэтому возможно обслуживание нового 
устройства (рис.~3,\,\textit{а}). В~случае~II из-за удаленности от БС устройству 
недостаточно отведенного временн$\acute{\mbox{о}}$го интервала для достижения 
скорости~$r_0$ и~происходит блокировка запроса на обслуживание (рис.~3,\,\textit{б}).
  


\vspace*{-6pt}


\section{Построение математической модели}

  Введем следующие обозначения: $\xi(t)$~--- чис\-ло M2M-устройств, 
находящихся в~активном состоянии в~момент времени $t\hm> 0$; $\eta_i(t)$, 
$i\hm=1,\ldots , \xi(t)$,~--- значение параметра CQI для \mbox{$i$-го} 
 M2M-устрой\-ст\-ва в~момент времени $t\hm> 0$; $\left\{ \xi(t),\eta_1(t),\ldots , 
\eta_{\xi(t)}(t),\ t\hm\geq 0\right\}$~--- марковский случайный процесс (СП), 
описывающий поведение системы; $(k, l_1, \ldots , l_k)$, $l_i\hm=1, \ldots , L$, 
$k\hm=0,1,\ldots$,~--- состояние системы. 
  
  Для адекватного построения модели сначала рассмотрим случай, когда 
мощность передачи сигнала активными M2M-устрой\-ст\-ва\-ми не имеет 
ограничений. Тогда пространство состояний системы~${\sf L}^{\sim}$ имеет 
следующий вид:
  \begin{multline*}
  {\sf L}^{\sim}={}\\
  {}=\left\{ (0), (1,1),\ldots , (1,L), (2,1,1), \ldots , (2,L,L),\ldots\right.\\
\left.  \ldots , 
\left(k, \underbrace{1,\ldots , 1}_{k}\right), \ldots ,  \left( k, \underbrace{L,\ldots 
,L}_{k}\right), \ldots\right\}={}\\
  {}= \{ (0), (k, l_1, l_2,\ldots , l_k),\ l_i\in \{1,\ldots , L\}\,,\\
   i=1,\ldots , k\,,\ 
k=1,2,\ldots\}\,.
  %\label{e2-gud}
  \end{multline*}
  
  Предположим, что СВ~$\eta$ принимает значение, равное~$l$, 
с~вероятностью~$q_l$, $l\hm\in \{1,\ldots ,L\}$, которая с~учетом равномерного 
распределения M2M-устройств в~соте определяется как 
$$
q_l= \fr{2L-2l- 1}{L^2}\,, \enskip l=1,\ldots , L\,.
$$
  
  Пусть в~системе~$k$~активных M2M-устройств, каж\-до\-му из которых 
соответствует некоторое значение параметра CQI $16\hm- l_i$. Для упрощения 
расчетов агрегируем состояния системы по числу обслуживаемых  
M2M-устройств~$k$~\cite{13-gud}. Тогда пространство состояний~${\sf 
L}^{\sim}$ разбивается следующим образом: 
  $$
  {\sf L}^{\sim} = \mathop{\bigcup}\limits_{n=0}^\infty {\sf L}^{\sim}(k),\ 
  {\sf  L}^{\sim}(k)= \left\{ \left(k, l_1, l_2, \ldots , l_k\right) \in 
  {\sf L}^{\sim}\right\}.
  $$

\section{Сужение случайного процесса в~соответствии с~политикой управления 
радиоресурсами}

  В действительности мощность передачи сигнала не может быть бесконечной, 
а~следовательно, ограничено и~число обслуживаемых в~соте M2M-устройств. 
Эти ограничения определяются в~соответствии с~выбранной политикой 
управления радиоресурсами. В~связи с~этим рассмотрим сужение СП 
  $\left\{ \xi(t), \eta_1(t),\ldots , \eta_{\xi(t)}(t)\,, \ t\hm\geq 0\right\}$ над 
множеством ${\sf L}\hm\subset {\sf L}^{\sim}$. Для корректного определения 
множества~${\sf L}$ введем функции доступа, соответствующие каждой из 
рассматриваемых политик. Для политики RR функция доступа имеет вид:
  \begin{multline}
  g_{\xi_d(\eta)}(k, l_1, l_2,\ldots , l_k)={}\\
 \!\!{}=\begin{cases}
  1, &\!\!\! \fr{r_0}{r(\xi_d(\eta),p_{\max})}\leq \fr{1}{k+1}\,,\ i=1,\ldots , k+1;\\
  0 & \!\!\!\mbox{в~противном~случае};
  \end{cases}\!\!
  \label{e3-gud}
  \end{multline}
для политики FP
\begin{multline}
  g_{\xi_d(\eta)}(k, l_1, l_2,\ldots , l_k)={}\\
  {}=\begin{cases}
1\,, & \displaystyle \sum\limits^n_{i=1} \fr{r_0}{r(\xi_d(\eta),p_{\max})}\leq 1\,;\\
0 & \mbox{в~противном~случае}\,.
\end{cases}
\label{e4-gud}
\end{multline}
    Тогда пространство состояний ${\sf L}_{\mathrm{RR}}(k)$ или~${\sf 
L}_{\mathrm{FP}}(k)$, в~зависимости от использования политики RR или FP, 
определяется в~соответствии с~формулой~(\ref{e3-gud}) или~(\ref{e4-gud}) 
следующим образом:
  \begin{multline}
  {\sf L}_{\mathrm{RR}}(k)= \left\{
  \vphantom{\fr{1}{k}}
  0\leq d_1\leq R,\ldots , 0\leq d_k\leq R:\right.\\
\left.   \fr{r_0}{\omega\ln 
(1+Gp_{\max}/(d_i^\kappa N_0))}\leq \fr{1}{k}\,,\enskip i=1,\ldots, 
k\right\}\,,\\ {\sf L}_{\mathrm{RR}}\subset {\sf L}^{\sim}\,;
  \label{e5-gud}
  \end{multline}
  
  \vspace*{-12pt}
  
  \noindent
  \begin{multline}
  {\sf L}_{\mathrm{FP}}(k)= \left\{
  \vphantom{\sum\limits^k_{i=1}}
  0\leq d_1\leq R,\ldots , 0\leq d_k\leq R:\right.\\
   \left.\sum\limits^k_{i=1} \fr{r_0}{\omega\ln 
(1+Gp_{\max}/(d_i^\kappa N_0))}\leq 1\,,\right\}\,,\\
 {\sf  L}_{\mathrm{FP}}\subset {\sf L}^{\sim}\,.
  \label{e6-gud}
  \end{multline}
  
  Обозначим через 
  \begin{multline*}
  {\sf P}(k-1)={\sf P}\left\{ \left( k, l_1, l_2, \ldots , l_k\right)\in{}\right.\\
 \left. {}\in 
  {\sf L}(k)\vert \left(k,l_1,l_2,\ldots, l_{k-1}\right)\in {\sf  L}(k\hm-1)\right\}
\end{multline*} 
условную вероятность того, что $k$-е M2M-устрой\-ст\-во 
с~соответствующим значением CQI~$l_k$, $l_k\hm\in \{1,\ldots , L\}$, будет 
обслужено при условии, что в~системе уже обслуживается $k\hm-1$  
M2M-устройств, каждое из которых имеет соответствующее значение CQI 
$l_1,\ldots , l_{k-1}$. 
  
  Тогда стационарное распределение вероятностей~$p_k$, $k\hm=1,2,\ldots$, 
того, что в~системе ровно~$k$~активных M2M-устройств, рассчитывается по 
фор\-муле:
  \begin{multline*}
  p_k=\left( \fr{\sum\nolimits^k_{j=0}\left( \left( {\lambda}/{\mu}\right)^j 
\prod\nolimits_{i=0}^{j-1} {\sf P}(i)\right)}{j!}\right)^{-1}\times{}\\ 
{}\times \fr{(\lambda/\mu)^k \prod\nolimits_{i=0}^{k-1} {\sf P}(i)}{k!}\,,\enskip 
k=1,2,\ldots
  %\label{e7-gud}
  \end{multline*}
  
  Основной характеристикой рассматриваемой модели служит 
вероятность~$B$ блокировки передачи данных от M2M-устрой\-ст\-ва, которая 
рассчитывается по формуле:
  \begin{equation}
  B=\sum\limits^\infty_{k=1} \left( 1-{\sf P}(k-1)\right) p_k\,.
  \label{e8-gud}
  \end{equation}

\section{Расчет условной вероятности в~соответствии 
с~выбранной политикой управления радиоресурсами}

  С учетом функции доступа~(\ref{e3-gud}) и~пространства состояний~(\ref{e5-gud}) условная вероятность ${\sf P}(k-1)$ при использовании политики RR 
имеет вид:
\begin{multline*}
  {\sf P}(k-1)= \left( {\sf P}\left\{ \fr{r_0}{r(\xi_d(\eta),p_{\max})}\leq 
\fr{1}{k}\right\}\right)^k\times{}\\
 {}\times\left( {\sf P}\left\{ 
\fr{r_0}{r(\xi_d(\eta),p_{\max})}\leq \fr{1}{k-1}\right\}\right)^{-(k-1)}\,.
 % \label{e9-gud}
  \end{multline*}
  Так как СВ $\xi_d(\eta)$ является равномерно распределенной, то
  \begin{multline*}
  {\sf P}(k)=\fr{1}{R^2}\left(\fr{Gp_{\max}}{N_0}\right)^{2/\kappa} \left( 
e^{r_0k/\omega}-1\right)^{2k/\kappa}\times{} \\
{}\times
\left( e^{(r_0k+1)/\omega}-1\right) ^{-
2(k+1)/\kappa}\,.
 % \label{e10-gud}
  \end{multline*}
  
  Перейдем к~расчету условной вероятности ${\sf P}(k\hm-1)$ для политики FP, 
опираясь на формулы~(\ref{e4-gud}) и~(\ref{e6-gud}). Для случая, когда 
в~системе нет активных M2M-устройств, т.\,е.\ $k\hm=1$, получим: 
  \begin{equation*}
  {\sf P}(0)=F_{\xi_d(\eta)}\left( \min \left\{ R,\left( 
\fr{Gp_{\max}}{(e^{r_0/\omega}-1)N_0}\right)^{1/\kappa}\right\}\!\right).
 % \label{e11-gud}
  \end{equation*}
  
  Для вывода формул расчета остальных условных вероятностей ${\sf P}(k-1)$, 
$k\hm>1$, воспользуемся определением условной вероятности и~функцией 
доступа~(\ref{e6-gud}):
  \begin{multline*}
  {\sf P}(k-1)={}\\
  {}={\sf P}\left\{ \sum\limits^k_{i=1} \fr{1}{r(l_i,p_{\max})}\leq 
\fr{1}{r_0}\,, \sum_{i=1}^{k-1}\fr{1}{r(l_i,p_{\max})}\leq \fr{1}{r_0}\right\} 
\!\Bigg /\\
{} {\sf P} \left\{ \sum\limits_{i=1}^{k-1} \fr{1}{r(l_i,p_{\max})}\leq 
\fr{1}{r_0}\right\}\,,
 % \label{e12-gud}
  \end{multline*}
где согласно центральной предельной теореме
$$
{\sf P}\left\{ \sum\limits^k_{i=1} \fr{1}{r(l_i,p_{\max})}\leq 
\fr{1}{r_0}\right\}=\Phi\left(\fr{ 1-\theta k r_0}{r_0\sigma \sqrt{k}}\right)\,.
$$
Здесь  $\theta$ и~$\sigma$~--- математическое ожидание и~среднеквадратичное 
отклонение независимых и~одинаково распределенных СВ $1/r(l_i,p_{\max})$; 
$\Phi(x)\hm= (1/\sqrt{2\pi}) \int\nolimits^x_{-\infty} e^{-t/2}\,dt$~--- стандартное 
нормальное распределение.   
  
  Для дальнейшего упрощения записи введем обозначения: 
 \begin{align*}
  m_k&= kr_0E \fr{1}{r(l_i,p_{\max})}\,;\\
  \tau_k^2&= kr_0\left(
  E \left(\fr{1}{r(l_i,p_{\max})}\right)^2 - \left( E \fr{1}{r(l_i,p_{\max})}\right)^2\right)\,.
\end{align*}
 Тогда условная вероятность 
  при использовании политики FP примет вид:
  \begin{equation*}
  {\sf P}(k-1)=\fr{\Phi\left( \left(1-m_k\right)/\tau_k\right)}{
  \Phi\left(  
\left(1-m_{k-1}\right)/\tau_{k-1}\right)} \,.
%  \label{e13-gud}
  \end{equation*}
  
  Расчет математического ожидания случайных величин $1/r(l_i,p_{\max})$ 
и~$(1/r(l_i,p_{\max}))^2$ представлен в~\cite{13-gud}.

\setcounter{table}{1}
\begin{table*}\small  %tabl2
\begin{center}
\Caption{Постоянные данные}
\vspace*{2ex}

\begin{tabular}{|c|c|c|c|c|c|c|c|}
\hline
&&&&&&&\\[-9pt]
$\omega$&$L$&$d_0$&$r_0$&$\mu^{-1}$&$N_0$&$G$&$\kappa$\\
\hline
10 МГц&15&$R/15$&1~Мбит/с&0,1~с&$-60$~дБм&197,43&5\\
\hline
\end{tabular}
\end{center}
\vspace*{-6pt}
\end{table*}
  
\section{Численный анализ}

  Для проведения численного анализа веро\-ят\-ности блокировки, 
рассчитываемой по формуле~(\ref{e8-gud}) с~учетом используемой политики 
управления радиоресурсами, рассмотрим три различных сценария, в~которых 
динамически меняются: 
\begin{enumerate}[(1)]
\item радиус соты; 
\item мощность передачи; 
\item интенсивность перехода устройств в~активное состояние. 
\end{enumerate}

Сведем исходные 
данные в~табл.~1, характеризующую динамические значения параметров, 
и~табл.~2 с~постоянными значениями.
  
  



  
  Построим графики поведения вероятности блокировки для каждого из 
рассматриваемых сценариев и~в~рамках используемой политики 
распределения радиоресурсов (рис.~4). 


  
  Очевидно, что вероятность блокировки ниже при большей мощности 
передачи. При одинаковых значениях мощности вероятность блокировки 
принимает различные значения в~зависимости\linebreak\vspace*{-12pt}

%    \begin{table*}
 %tabl1
  \begin{center}
  \vspace*{6pt}
  
{{\tablename~1}\ \ \small{Динамически меняющиеся данные}}


  \vspace*{2ex}
{\small  
  \tabcolsep=4.5pt
  \begin{tabular}{|c|c|c|c|}
  \hline
Сценарий&$R$ & $p_{\max}$& $\lambda$\\
\hline
1& $200\to 400$~м &\tabcolsep=0pt\begin{tabular}{c}23 дБм (0,2~Вт)\\
42 дБм (15,85~Вт)\end{tabular} & 10\\
\hline
2& 200, 400~м &$23 \to 42$~дБм & 10\\
\hline
3 & 200, 400~м& \tabcolsep=0pt\begin{tabular}{c}23~дБм 
(0,2~Вт)\\
42~дБм (15,85~Вт)\end{tabular} & $2 \to 10$\\
\hline
\end{tabular}}
\vspace*{3pt}
\end{center}
%}
%\end{table*}

{ \begin{center}  %fig4
 \vspace*{-3.5pt}
  \mbox{%
 \epsfxsize=79mm 
 \epsfbox{gud-6.eps}
 }


\end{center}


\noindent
{{\figurename~4}\ \ \small{Вероятность блокировки в~рамках реализации сценариев~1~(\textit{а}),
2~(\textit{б}) и~3~(\textit{в})}}
}

\vspace*{9.5pt}

\addtocounter{figure}{1}


\noindent
 от используемой политики 
распределения радиоресурсов. На рис.~4,\,\textit{а} видно, что при использовании 
политики RR вероятность блокировки выше. 
  
  При увеличении мощности передачи данных вероятность блокировки 
снижается (рис.~4,\,\textit{б}). 
  


  
  С увеличением частоты поступления новых запросов на обслуживание 
устройств, вероятность блокировки увеличивается (рис.~4,\,\textit{в}). 

%\vspace*{-12pt}

\section{Заключение}

  В статье построена модель соты беспроводной\linebreak сети LTE  
с~M2M-устрой\-ст\-ва\-ми, равномерно распределенными в~соте 
и~сгруппированными по расстоянию от БС, определяемому с~учетом 
соответ\-ствующего устройству значения CQI. Устройства\linebreak генерируют 
потоковый трафик, передаваемый в~восходящем канале с~гарантированной 
скоростью. Рассмотрены две политики управления радиоресурсами, 
основанные на различных вариациях таких ресурсов, как ширина полосы 
частот, мощность передачи сигнала и~время обслуживания M2M-за\-проса.
  
  В рамках текущего исследования моделей сети с~двумя различными 
политиками распределения (FP и~RP) результаты 
численного анализа показали, что политика FP является преимущественной 
в~использовании и~наиболее эффективной, так как в~этом случае ве\-ро\-ят\-ность 
блокировки ниже.

%\vspace*{-12pt}

{\small\frenchspacing
 {%\baselineskip=10.8pt
 \addcontentsline{toc}{section}{References}
 \begin{thebibliography}{99}
\bibitem{1-gud}
\Au{Laya A., Alonso~L., Alonso-Zarate~J.} Is the random access channel of LTE and LTE-A 
suitable for M2M communications? A~survey of alternatives~// IEEE Commun. Surv. 
Tut., 2014. Vol.~16. Iss.~1. P.~4--16. doi: 10.1109/ SURV.2013.111313.00244.
\bibitem{2-gud}
Cisco Visual Networking Index: Forecast and trends, 2017--2022. White Papers, 
November~26, 2018. Document ID: 1543280537836565.
\bibitem{3-gud}
Future technology trends of terrestrial IMT systems. \mbox{ITU-R} Reports. Report M.2320,
November 2014.
\bibitem{4-gud}
\Au{Aijaz A., Tshangini~M., Nakhai~M.\,R., Chu~X., Aghvami~\mbox{A.-H.}} Energy-efficient uplink 
resource allocation in LTE networks with M2M/H2H co-existence under statistical QoS 
guarantees~// IEEE~T. Commun., 2014. Vol.~62. Iss.~7. P.~2353--2365. doi: 
10.1109/ TCOMM.2014.2328338.
\bibitem{5-gud}
\Au{Hamdoun S., Rachedi~A., Ghamri-Doudane~Y.} A~flexible M2M radio resource sharing 
scheme in LTE networks within an H2H/M2M coexistence scenario~// Conference (International) 
on Communications.~--- IEEE, 2016. P.~1--7. doi: 10.1109/ICC.2016.7511237.
\bibitem{6-gud}
Ericsson mobility report: On the pulse of the networked society.~--- Ericsson, June 2016.
\bibitem{7-gud}
Requirements, evaluation criteria and submission templates for the development of IMT-2020 // 
ITU-R Report M.2411, November 2017.
\bibitem{8-gud}
\Au{Samouylov K., Gudkova~I., Markova~E., Dzantiev~I.} On analyzing the blocking probability 
of M2M transmissions for a~CQI-based RRM scheme model in 3GPP LTE~// Information 
technologies and mathematical modelling~--- queueing theory and applications~/
Eds.\ A.~Dudin, A.~Gortsev, A.~Nazarov, R.~Yakupov.~---
Communications in computer and information science ser.~---
Springer, 2016. Vol.~638. P.~327--340. doi: 
10.1007/978-3-319-44615-8\_29.
\bibitem{9-gud}
\Au{Markova E., Dzantiev~I., Gudkova~I., Shorgin~S.} Analyzing impact of path loss models on 
probability characteristics of wireless network with licensed shared access framework~//  9th 
Congress (International) on Ultra Modern Telecommunications and Control Systems  
Proceedings.~--- Piscataway, NJ, USA: IEEE, 2017. 
P.~20--25. doi: 10.1109/ICUMT.2017.8255189.

\bibitem{12-gud} %10
\Au{Begishev V., Kovalchukov~R., Samuylov~A., Ometov~A., Moltchanov~D., Gaidamaka~Y., 
Andreev~S.} An analytical approach to SINR estimation in adjacent rectangular cells~// Internet of 
things, smart spaces, and next generation networks and systems~/
Eds.\ S.\,I.~Balandin, S.\,D.~Andreev, Y.~Koucheryavy.~---
Lecture notes in computer  science ser.~--- Springer, 2015. Vol.~9247. 
P.~446--458. doi: 10.1007/978-3-319-23126-6\_39.

\bibitem{11-gud} %11
\Au{Samuylov A., Moltchanov~D., Gaidamaka~Y., Andreev~S., Koucheryavy~Y.} Random triangle: 
A~baseline model for interference analysis in heterogeneous networks~// IEEE~T.  
Veh. Technol., 2016. Vol.~65. Iss.~8. P.~6778--6782. doi: 10.1109/TVT.2015.2481795.

\bibitem{10-gud} %12
\Au{Naumov V., Samouylov~K.} Analysis of multi-resource loss system with state-dependent arrival 
and service rates~// Probab.  Eng. Inform. Sc., 2017. Vol.~31. Iss.~4.  
P.~413--419. doi: 10.1017/S0269964817000079.

\bibitem{13-gud}
\Au{Markova E., Gudkova~I., Ometov~A., Dzantiev~I., And\-re\-ev~S., Koucheryavy~Ye., 
Samouylov~K.} Flexible spectrum management in a~smart city within licensed shared access 
framework~// IEEE Access, 2017. Vol.~5. P.~22252--22261. doi: 
10.1109/ACCESS.2017.2758840.
 \end{thebibliography}

 }
 }

\end{multicols}

\vspace*{-3pt}

\hfill{\small\textit{Поступила в~редакцию 15.01.19}}

\vspace*{12pt}

%\pagebreak

%\newpage

%\vspace*{-28pt}

\hrule

\vspace*{2pt}

\hrule

%\vspace*{-2pt}

\def\tit{COMPARATIVE ANALYSIS OF~PERFORMANCE MEASURES FOR~A~WIRELESS 
MACHINE-TO-MACHINE NETWORK MODEL OPERATING WITHIN~TWO RADIO RESOURCE MANAGEMENT POLICIES}


\def\titkol{Comparative analysis of performance measures for a~wireless 
machine-to-machine network model} % operating within two radio resource management policies}

\def\aut{E.\,V.~Markova$^1$, A.\,A.~Golskaia$^1$, I.\,L.~Dzantiev$^1$, 
I.\,A.~Gudkova$^{1,2}$, and~S.\,Ya.~Shorgin$^{1,2}$}

\def\autkol{E.\,V.~Markova, A.\,A.~Golskaia, I.\,L.~Dzantiev, et al.} 
%I.\,A.~Gudkova$^{1,2}$, and~S.\,Ya.~Shorgin$^{1,2}$}

\titel{\tit}{\aut}{\autkol}{\titkol}

\vspace*{-9pt}


\noindent
$^1$Peoples' Friendship University of Russia, 6~Miklukho-Maklaya Str., Moscow 117198, 
Russian Federation

\noindent
$^2$Institute of Informatics Problems, Federal Research Center ``Computer Science and Control'' 
of the Russian\linebreak
$\hphantom{^1}$Academy of Sciences; 44-2~Vavilov Str., Moscow 119333, Russian Federation

\def\leftfootline{\small{\textbf{\thepage}
\hfill INFORMATIKA I EE PRIMENENIYA~--- INFORMATICS AND
APPLICATIONS\ \ \ 2019\ \ \ volume~13\ \ \ issue\ 1}
}%
 \def\rightfootline{\small{INFORMATIKA I EE PRIMENENIYA~---
INFORMATICS AND APPLICATIONS\ \ \ 2019\ \ \ volume~13\ \ \ issue\ 1
\hfill \textbf{\thepage}}}

\vspace*{12pt}


    



     
     \Abste{Currently, information and communication technologies (ICT) 
     deeply penetrate into many areas of modern life. For example, the 
     concept of integrating ICT and the Internet of Things 
     for managing Smart City infrastructure allows city authorities to monitor 
     changes and the situation in the city using sensors. Thus, specialized 
     systems collect data automatically without human intervention. 
     An important parameter in determining the performance of wireless 
     networks of machine-to-machine  (M2M) interaction~--- 
     data transfer rates, blocking probabilities, becomes the distance between 
     the device (sensor) and the radio transmitting equipment (base station, BS). 
     Therefore, describing such a~network in the form of a~queuing system 
     with streaming (guaranteed data transfer rate) or elastic traffic 
     (nonguaranteed speed), it is necessary to consider the incoming stream 
     of requests for data transmission of M2M devices in such a~way as to take 
     into account the distance between devices and BS. In this paper, there 
     is built a~cell model of a wireless network with stationary M2M devices 
     that are in a~passive or active state, shown by points that appear randomly 
     on a~plane. The devices generate streaming traffic which depends on the 
     distance from the BS, the device transmit power, and the noise level. 
     The state of the system is described by the vector of variable length, 
     the components of which are the distance of each active device to the BS. 
     Two different disciplines of radio resource separation are considered~--- 
     ``round robin'' and ``full power,'' which differ\linebreak\vspace*{-12pt}}
     
     \Abstend{in the distribution of 
     the time interval for servicing an M2M device and the data transfer rate 
     provided. There is built a random process with states enlarged by the 
     number of users and a~formula for calculating 
     the probability of blocking a~data transfer request is proposed.}
     
     \KWE{wireless network; LTE; machine-to-machine communication; 
     channel quality indicator; Shannon's formula; uniform distribution; 
     round robin policy; 
     full power policy; blocking probability}


\DOI{10.14357/19922264190115}

%\vspace*{-14pt}

\Ack
\noindent
The publication was supported by the Ministry of Education and Science of the 
Russian Federation (project No.\,2.3397.2017/4.6).



%\vspace*{6pt}

  \begin{multicols}{2}

\renewcommand{\bibname}{\protect\rmfamily References}
%\renewcommand{\bibname}{\large\protect\rm References}

{\small\frenchspacing
 {%\baselineskip=10.8pt
 \addcontentsline{toc}{section}{References}
 \begin{thebibliography}{99}
\bibitem{1-gud-1}
\Aue{Laya, A., L.~Alonso, and J.~Alonso-Zarate.} 2014. Is the random access channel 
of LTE 
and LTE-A suitable for M2M communications? A~survey of alternatives. 
\textit{IEEE Commun. 
Surv. Tut.} 16(1):4--16. doi: 10.1109/ SURV.2013.111313.00244.
\bibitem{2-gud-1}
Cisco Visual Networking Index: Forecast and trends, 2017--2022.  November~26, 
2018. White Paper. Available at: 
{\sf https://www.cisco.com/c/en/us/solutions/collateral/\linebreak
service-provider/visual-networking-index-vni/white-paper-c11-741490.html}
 (accessed January~15, 2019).
\bibitem{3-gud-1}
  Future technology trends of terrestrial IMT systems. November~2014.
  ITU-R Report M.2320.
Available at: {\sf 
https://www.itu.int/pub/R-REP-M.2320-2014} (accessed January~15, 2019).
\bibitem{4-gud-1}
\Aue{Aijaz, A., M.~Tshangini, M.\,R.~Nakhai, X.~Chu, and A.-H.~Aghvami.} 
2014. Energy-efficient 
uplink resource allocation in LTE networks with M2M/H2H 
co-existence under statistical QoS 
guarantees. \textit{IEEE~T. Commun.} 62(7):2353--2365. doi: 
10.1109/TCOMM.2014.2328338.
\bibitem{5-gud-1}
\Aue{Hamdoun, S., A.~Rachedi, and Y.~Ghamri-Doudane.} 2016. 
A~flexible M2M radio resource 
sharing scheme in LTE networks within an H2H/M2M coexistence scenario. 
\textit{Conference (International) on Communications}. IEEE. 1--7. 
doi: 10.1109/ICC.2016.7511237.
\bibitem{6-gud-1}
On the pulse of the networked society. 
June 2016. Ericsson mobility report. Available at: 
{\sf https://www.\linebreak ericsson.com/assets/local/mobility-report/documents/ 2016/ericsson-mobility-report-june-2016.pdf}
 (accessed January~15, 2019) . 
\bibitem{7-gud-1}
 Requirements, evaluation criteria and submission 
templates for the development of IMT-2020. 
 November~2017.
 ITU-R Report M.2411. Available at: {\sf 
https:// www.itu.int/pub/R-REP-M.2411-2017} (accessed January~15, 2019). 
\bibitem{8-gud-1}
\Aue{Samouylov, K., I.~Gudkova, E.~Markova, and I.~Dzantiev.} 2016. 
On analyzing the blocking 
probability of M2M transmissions for a~CQI-based RRM scheme model in 3GPP LTE. 
\textit{Information 
technologies and mathematical modelling~--- queueing theory and applications}.
Eds.\ A.~Dudin, A.~Gortsev, A.~Nazarov, and R.~Yakupov.
Communications in computer and information science ser.
Springer. 638:327--340. doi: 10.1007/978-3-319-44615-8\_29.
\bibitem{9-gud-1}
\Aue{Markova, E., I.~Dzantiev, I.~Gudkova, and S.~Shorgin.} 2017. 
Analyzing impact of path loss 
models on probability characteristics of wireless network with licensed shared access 
framework. \textit{9th Congress (International) on Ultra Modern Telecommunications and 
Control Systems Proceedings}.  
Piscataway, NJ: IEEE. 20--25. doi: 10.1109/ICUMT.2017.8255189.

\bibitem{12-gud-1} %10
\Aue{Begishev, V., R.~Kovalchukov, A.~Samuylov, A.~Ometov, D.~Moltchanov, 
Y.~Gaidamaka, and  S.~Andreev.} 2015. An analytical approach to SINR 
estimation in adjacent rectangular cells. 
\textit{Internet of 
things, smart spaces, and next generation networks and systems}.
Eds.\ S.\,I.~Balandin, S.\,D.~Andreev, and Y.~Koucheryavy.
Lecture notes in computer  science ser. Springer. 9247:446--458. doi: 
10.1007/978-3-319-23126-6\_39.

\bibitem{11-gud-1}
\Aue{Samuylov, A., D.~Moltchanov, Y.~Gaidamaka, S.~Andreev, and Y.~Koucheryavy.} 2016. 
Random triangle: A~baseline model for interference analysis in heterogeneous networks. 
\textit{IEEE~T. Veh. Technol.} 65(8):6778--6782. doi: 
10.1109/TVT.2015.2481795.
\bibitem{10-gud-1} %12
\Aue{Naumov, V., and K.~Samouylov. 2017.} 
Analysis of multi-resource loss system with state-dependent arrival and service rates. 
\textit{Probab. Eng. Inform. Sc.} 31(4):413--419. 
doi: 10.1017/S0269964817000079.

\bibitem{13-gud-1}
\Aue{Markova, E., I.~Gudkova, A.~Ometov, I.~Dzantiev, S.~And\-re\-ev, Ye.~Koucheryavy, 
and~K.~Samouylov.} 2017. Flexible spectrum management in 
a~smart city within licensed shared 
access framework. \textit{IEEE Access} 5:22252--22261. doi: 10.1109/ACCESS.2017.2758840.
\end{thebibliography}

 }
 }

\end{multicols}

\vspace*{-6pt}

\hfill{\small\textit{Received January 15, 2019}}

%\pagebreak

\vspace*{-12pt}


\Contr

\noindent
\textbf{Markova Ekaterina V.} (b.\ 1987)~--- Candidate of Science (PhD) in physics and 
mathematics; associate professor, Peoples' Friendship University of Russia, 6~Miklukho-Maklaya 
Str., Moscow 117198, Russian Federation; \mbox{markova\_ev@rudn.university} 

%\vspace*{3pt}

\noindent
\textbf{Golskaia Anastasia A.} (b.\ 1994)~--- PhD student, Department of Applied Informatics 
and Probability Theory, Peoples' Friendship University of Russia, 6~Miklukho-Maklaya Str., 
Moscow 117198, Russian Federation; \mbox{golskaya\_aa@rudn.university}

\vspace*{5pt}

\noindent
\textbf{Dzantiev Iliya L.} (b.\ 1991)~--- PhD student, Department of Applied Informatics and 
Probability Theory, Peoples' Friendship University of Russia, 6 Miklukho-Maklaya Str., Moscow 
117198, Russian Federation; \mbox{dzantiev\_il@rudn.university}

\vspace*{5pt}

\noindent
\textbf{Gudkova Irina A.} (b.\ 1985)~--- Candidate of Science (PhD) in physics and 
mathematics; associate professor, 
Peoples' Friendship University of Russia, 6~Miklukho-Maklaya 
Str., Moscow 117198, Russian Federation; senior scientist, Institute of Informatics Problems, 
Federal Research Center ``Computer Science and Control'' of the Russian Academy of Sciences, 
44-2~Vavilov Str., Moscow 119333, Russian Federation; \mbox{gudkova\_ia@rudn.university}

\vspace*{5pt}

\noindent
\textbf{Shorgin Sergey Ya.} (b.\ 1952)~--- Doctor of Science in physics and mathematics, 
professor; principal scientist, Institute of Informatics Problems, Federal Research Center 
``Computer Science and Control'' of the Russian Academy of Sciences; 44-2~Vavilov Str., Moscow 
119333, Russian Federation; \mbox{sshorgin@ipiran.ru}


\label{end\stat}

\renewcommand{\bibname}{\protect\rm Литература}       