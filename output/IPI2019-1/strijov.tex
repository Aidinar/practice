\def\stat{strijov}

\def\tit{КЛАССИФИКАЦИЯ ФИЗИЧЕСКОЙ АКТИВНОСТИ ЧЕЛОВЕКА 
С~ПОМОЩЬЮ ЛОКАЛЬНЫХ АППРОКСИМИРУЮЩИХ МОДЕЛЕЙ$^*$}

\def\titkol{Классификация физической активности человека 
с~помощью локальных аппроксимирующих моделей}

\def\aut{Д.\,А.~Аникеев$^1$, Г.\,О.~Пенкин$^2$, В.\,В.~Стрижов$^3$}

\def\autkol{Д.\,А.~Аникеев, Г.\,О.~Пенкин, В.\,В.~Стрижов}

\titel{\tit}{\aut}{\autkol}{\titkol}

\index{Аникеев Д.\,А.}
\index{Пенкин Г.\,О.}
\index{Стрижов В.\,В.}
\index{Anikeyev D.\,A.}
\index{Penkin G.\,O.}
\index{Strijov V.\,V.}


{\renewcommand{\thefootnote}{\fnsymbol{footnote}} \footnotetext[1]
{Работа выполнена при частичной финансовой поддержке РФФИ (проекты 17-07-1574 и~19-07-1155) 
и~Правительства Российской Федерации (соглашение №\,05.Y09.21.0018).}}


\renewcommand{\thefootnote}{\arabic{footnote}}
\footnotetext[1]{Московский физико-технический институт, dmitriy.anikeyev@phystech.edu}
\footnotetext[2]{Московский физико-технический институт, \mbox{penkin.gr@gmail.com}}
\footnotetext[3]{Московский физико-технический институт; Вычислительный центр им.\
 А.\,А.~Дородницына Федерального исследовательского центра <<Информатика и~управление>> 
 Российской академии наук, \mbox{strijov@ccas.ru}}

\vspace*{12pt}


\Abst{Исследуется проблема классификации временных рядов 
акселерометра мобильного телефона. Классу физической активности 
соответствует сегмент временного ряда. Сегменту сопоставляется 
его признаковое описание. Оно порождается аппроксимирующим сплайном. 
Элементами вектора признаков являются коэффициенты при базисных функциях 
сплайнов. Вычислительный эксперимент находит оптимальные параметры 
аппроксимации и~параметры модели классификации 
согласно максимуму правдоподобия логистической модели классификации.}


\KW{временные ряды; классификация; сплайн; локальная аппроксимация; 
признаковое пространство}

\DOI{10.14357/19922264190106}
  
%\vspace*{4pt}


\vskip 10pt plus 9pt minus 6pt

\thispagestyle{headings}

\begin{multicols}{2}

\label{st\stat}


\section{Введение}

Цель работы заключается в~решении задачи классификации временн$\acute{\mbox{ы}}$х рядов 
сложной структуры~\cite{Karasikov2016,Farmer1987}, для которых признаковое 
описание не задано  явно. Такие временные ряды встречаются в~задачах 
классификации звуковых сигналов~\cite{myas1970}, данных 
с~акселерометра~\cite{Gary2011,ivkin2015}, построения прогностических 
финансовых моделей~\cite{Tsay2010}. В~\cite{Esl2012,Fu2011} дается обзор 
методов анализа временн$\acute{\mbox{ы}}$х рядов за последнее десятилетие.  
Задача построения признакового пространства требует выбора адекватной 
гипотезы порождения временн$\acute{\mbox{о}}$го ряда. Многие 
из ис\-сле\-ду\-емых временн$\acute{\mbox{ы}}$х рядов 
описываются моделью авторегрессии~\cite{SEMOR2004,Fatma} или 
моделью анализа сингулярного спектра.

В данной работе изучается классификация сегментов временн$\acute{\mbox{ы}}$х рядов по 
их локальному описанию. Под локальным описанием будем понимать параметры 
сплайна, аппроксимирующего сегмент ряда~\cite{Istomin2005,Tselykh2012,Det2007}.

Предполагается, что класс физической ак\-тив\-ности описывается уникальной 
аппрокси\-ми\-ру\-ющей кривой, моделью, которая задана су\-перпозицией параметрических функций.  
В~\cite{Karasikov2016,ivkin2015} показано, что использование параметров 
ап\-про\-кси\-ми\-ру\-ющих моделей в~качестве признакового описания сегмента дает 
ожидаемое качество классификации физической активности.
%
При этом остается открытой проблема использования значения параметров локальных 
аппроксимирующих моделей. Точность классификации зависит от этих па\-ра\-мет\-ров. 
Требуется найти оптимальные па\-ра\-мет\-ры локальных моделей. 

Построение моделей, 
использующих в~качестве признакового описания параметры локальных моделей, 
рассмотрено в~\cite{Gholizadeh2009}.

В данной работе для решения задачи классификации временн$\acute{\mbox{ы}}$х рядов поставлена 
задача поиска оптимальной локальной аппроксимирующей модели. Сегмент 
временн$\acute{\mbox{о}}$го 
ряда аппроксимируется сплайном (алгебраическим, сглаживающим, адаптивным 
регрессионным или сплайном с~динамическими узлами). При помощи логистической 
регрессии строится отображение из вектора признаков, состоящего из коэффициентов 
сплайна, в~метку класса.

В качестве прикладной задачи рассматривается классификация физической активности 
человека по данным с~акселерометра. В~вычислительном эксперименте сравнивается 
точность классификации в~пространствах признаков, построенных различными 
локальными моделями, найдены оптимальные параметры таких моделей.


\section{Постановка задачи классификации сегментов временного ряда}

Поставим задачу  построения признакового пространства, описывающего сегменты 
временн$\acute{\mbox{ы}}$х\linebreak рядов с~целью их классификации. 
Задачу сформулируем в~терминах построения суперпозиции функций. 
Задан временной ряд, полученный в~результате измерений акселерометра.
Для построения  выборки~$\mathfrak{D}$  ряд разбивается на сегменты.
Сегмент временн$\acute{\mbox{о}}$го ряда~--- конечная последовательность 
значений~$\mathbf{x}(t)\hm=[x_1,\dots,x_p]^\mathsf{T}$ в~моменты 
времени $\mathbf{t} \hm= [t_1, \dots, t_p]^\mathsf{T}.$ 
Задана выборка~$\mathfrak{D}\hm=\{(\mathbf{x}_i,y_i)\},$ где
$y\hm\in \mathbb{Y}$~--- метка класса  из конечного множества.
Требуется найти оптимальный классификатор~$f(\mathbf{x}_i)$ из условия
\begin{equation*}
\hat{f}={\mathop{\mathrm{arg}}\underset{f}{\min}} 
\;\mathrm{CV}\left(f,\mathfrak{D}\right),
\end{equation*}
где
$$
\mathrm{CV} (f,\mathfrak{D})=\fr{1}{R}\sum\limits_{r=1}^R \left[f(\mathbf{x}_i)\neq y_i\right]
$$
при $\mathbf{x}_i\in \mathfrak{D\setminus D}_r$.

Выборка~$\mathfrak{D}$ случайно разбивается~$R$ раз на контрольную~$\mathfrak{D}_r$ 
и~тестовую~$\mathfrak{D\setminus D}_r$. Функцией ошибки выступает 
число неверно классифицированных объектов~$\mathbf{x}_i$ на тестовой выборке. 
Предлагается представить~$f$ в~виде суперпозиции функций~$f\hm=g\left(a(t)\right)$. 
Параметризованное отображение
$$
a: (\mathbf{t},\mathbf{b}) \mapsto \mathbf{x}
$$
отображает сегмент временн$\acute{\mbox{о}}$го ряда~$\mathbf{x}\hm\in \mathbf{X}$ 
в~пространство параметров аппроксимирующего сплайна, а 
$$
g:( \mathbf{b},\mathbf{w}) \mapsto y
$$ 
является логистической регрессией и~отображает признаковое описание в~метку класса. 
Вектор~$\mathbf{b}$ назовем вектором параметров аппроксимирующей модели, 
а~$\mathbf{w}$~--- вектором параметров классификатора~$g$.
Оптимальные параметры $(\mathbf{b},\mathbf{w})$ находятся минимизацией ошибок:
\begin{align}
\label{1-str}
\hat{\mathbf{b}}&=
\mathrm{\mathop{\mathrm{arg}}\underset{\mathbf{b}}\min}\;\norm{a(\mathbf{t, b}) - \mathbf{x}}_2^2\,;
\\
\hat{\mathbf{w}} &= \mathrm{\mathop{\mathrm{arg}}\underset{\mathbf{w}}{\min}} \;\mathrm{CV}
\left(f(a,\hat{\mathbf{b}}, \mathbf{w}),\mathfrak{D}\right),\notag
\end{align}
где~$a$~--- сплайн, аппроксимирующий сегмент временн$\acute{\mbox{о}}$го ряда, 
а~$\norm{\cdot}_2^2$~--- квадратичная ошибка аппроксимации.

\section{Выбор локальных аппроксимирующих моделей}

Рассмотрим различные методы построения признакового 
пространства. Зафиксируем сегмент~$\mathbf{x}\hm=[x_1,\ldots,x_p]^\mathsf{T}$ 
временн$\acute{\mbox{о}}$го ряда, порожденный измерениями, получаемыми с~носимого 
устройства в~моменты времени~$t_1,\ldots,t_p$.

\subsection{Алгебраический сплайн}

Разобьем сегмент~$[t_1,t_p]$ на~$K$ равных отрезков. Зададим~$\{h_1(t),\ldots,
h_N(t)\}$~--- базисные функции сплайна. Гладким сплайном называется 
функция~$a(t)$. Она непрерывна и~имеет непрерывные производные вплоть 
до порядка, называемого гладкостью сплайна, причем на каждом из~$K$ отрезков
\begin{multline}
\label{2-str}
a_k(t)=\sum\limits_{n=1}^N \theta_{k,n} 
 h_n\left(t-\fr{t_k-t_{k-1}}{2}\right), \\
 k\in\{1,\ldots,K\},
\end{multline}
где коэффициенты при базисных функциях~$\theta_{k,n}\hm\in \mathbb{R}$.

Оптимальный с~точки зрения аппроксимации сплайн ищется из условия~\eqref{1-str}.
В~разновидностях сплайнов, аппроксимирующих функцию, накладываются 
дополнительные условия на функцию. Матрица~$\boldsymbol{\Theta}_{K \times N}$ 
векторизуется, образуя элемент признакового пространства.
Для алгебраического сплайна порядка~$N$ базисными функциями являются мономы
$
h_n=t^n, \enskip   n \in \{1,\dots,N\}.
$

В качестве параметров модели выступают
\begin{equation}
\label{4-str}
\mathbf{b} = \left[\boldsymbol{\Theta}_{K \times N},N,K\right].
\end{equation}
По каждому сегменту строится~$K\times (N+1)$ параметров. 
В~данной работе рассматриваются только квадратичный и~кубический сплайны.

\subsection{Сглаживающий сплайн}

Сглаживающий сплайн минимизирует на отрезке~$\left[t_{\min},t_{\max}\right]$ 
одновременно квадратичную невязку и~производную~$a(t)$ порядка~$m$.
В~данной работе используется вторая производная и~модель вида~(\ref{2-str}). 
Таким образом, минимизируется  линейная комбинация
$$
(1-\lambda)\sum\limits_{k=1}^K \left(x(t_k)-a(t_k)\right)^2+
\lambda\!\int\limits_{t_{\min}}^{t_{\max}}\!\!\!\left(D^2f(t)\right)^2\,dt 
\rightarrow \min_\mathbf{\theta},
$$
где~$\lambda$~--- параметр гладкости; $N$~--- число базисных функций; 
$K$~--- число узлов.
В качестве базисных функций выступают мономы не выше заданной степени. 
Параметрами аппроксимирующей модели являются
\begin{equation}
\label{5-str}
\mathbf{b} = \left[\boldsymbol{\Theta}_{K \times N},\lambda, K, N\right].
\end{equation}

\subsection{Адаптивные регрессионные сплайны}

В одномерном случае адаптивные регрессионные сплайны 
(МАRS) выражаются через ку\-соч\-но-ли\-ней\-ные базисные функции:
\begin{align*}
(\tau-t)_+ &=
\begin{cases}
\tau-t, &\mbox{если\ } \tau>t\,;\\
0 &\mbox{ иначе;}
\end{cases}
\\
(t-\tau)_+ &=
\begin{cases}
t-\tau, &\mbox{если } \tau<t\,;\\
0 &\mbox{иначе.}
\end{cases}
\end{align*}

Используемые для аппроксимации базисные функции имеют вид:
$$
h_j(t)=\prod\limits_{k=1}^{K_j}\left((-1)^{s(j)}(t-\tau_{k_j})\right)_+,
$$
где $K_j$~--- общее число усеченных линейных функций в~$m$-й базисной функции.
На каждом из~$M$ шагов алгоритма в~множество базисных функций 
(изначально содержащем одну) добавляется функция вида
$$
h_m(t)=\hat{C}h_j(t)\left(\tau_k-t\right)_++\hat{C'}h_j(t)\left(t-\tau_k\right)_+
$$
такая, что ошибка~\eqref{1-str} минимальна.
Таким образом, модель имеет вид:
\begin{equation}
\label{3-str}
a(t)= c_0 +  \sum\limits_{j=1}^M c_j h_j(t) +\varepsilon\,,
\end{equation}
где~$\varepsilon$~--- невязка.

В результате добавления  модель приобретает избыточное 
число базисных функций. Необходимо  удалить из множества базисных функций те, 
удаление которых  вносит наименьший вклад в~увеличение ошибки~\eqref{1-str}. 
Параметрами данной модели выступают
\begin{equation}
\label{6-str}
 \mathbf{b} = [\mathbf{H}, M].
\end{equation}
Оптимальное число базовых функций
$$
\hat{M}=\mathop{\mathop{\mathrm{arg}}\min}_{M \in \{1, \ldots, M_{\max}\}}\mathrm{GCV}\,(M)
$$
определяется с~помощью критерия обобщенного скользящего контроля
$$
\mathrm{GCV}\,(M)=\fr{1}{R}\sum\limits_{i=1}^R
\fr{\left(x\left(\tau_i\right)-a_M\left(\tau_i\right)\right)^2}{
\left(1-C(M)/R\right)^2}\,,
$$ 
где~$R$~--- размер выборки;  $C(M)$~--- оценка штрафов модели: 
$$C(M)=\mathrm{trace}\left(\mathbf{H}\left(\mathbf{H}^\mathsf{T}\mathbf{H}\right)^{-1}
\mathbf{H}^\mathsf{T}\right)+1,
$$
в которой~$\mathbf{H}$~--- матрица c элементами~$h_j(\tau_i)$.

\subsection{Сплайны с~динамическими узлами}

Пусть временной сегмент~$[t_1,t_p]$ разбит на несколько 
отрезков~$[t_1,t_{n_1}]$, $[t_{n_1},t_{n_2}],\ldots, [t_{n_{K-1}},t_p]$. 
Рассмотрим аппроксимацию сегмента временн$\acute{\mbox{о}}$го ряда на каждом
 отрезке~$[t_{n_{k-1}},t_{n_k}]$ с~помощью полинома степени~$N$. 
 Тогда~$K\times (N+1)$ коэффициентов матрицы~$\boldsymbol{\Theta}_{K \times N}$ 
 образуют элемент признакового пространства. Предлагается определять 
 концы отрезков последовательно, используя критерий Фишера. 
 Метод заключается в~последовательном определении~$t_{n_k}$ ($t_{n_0}=t_1$). 
 Положим текущую длину отрезка $l\hm=1$. Построим полином~$y_k^l(t)$ 
 соответствующей степени, минимизирующий RSS (regression sum of squares)
 на отрезке~$[t_{n_k},t_{n_{k}+l}]$. 
 Полученные значения~$x(t)$ и~$y_k^l(t)$ в~целых точках на этом отрезке 
 образуют две выборки~$X$ и~$Y$. Отношение дисперсий
 $$
 F=\fr{\hat\sigma^2_X}{\hat\sigma^2_Y}
 $$
определяет схожесть дисперсий данных выборок. 
Если дисперсии выборок близки, увеличиваем чис\-ло~$l$ на~$1$ 
(кривая хорошо аппроксимирует временной ряд). Формально определяем 
критерий остановки с~помощью~$p$-зна\-че\-ния.
Если~$p(F)$ ниже уровня значимости~$\alpha$~--- 
определяем~$n_{k+1}\hm=n_k\hm+l.$ Продолжим получать последовательность 
узлов~$t_{n_k}$, пока сегмент не закончится. 
Число~$\alpha\hm\in[0;0,5]$ будем интерпретировать как параметр метамодели.

Поиск аппроксимирующего сплайна сведем к~задаче условной 
минимизации среднеквадратичного отклонения~(\ref{1-str}) 
при соответствующих условиях гладкости функции
$$
\fr{\partial^j y_{k+1}(t_{n_k})}{\partial t^j}=
\fr{\partial^j y_k(t_{n_k})}{\partial t^j} \mbox{~для всех~} k<K\,,\enskip  n<N\,.
$$
Также рассматривается аппроксимация при условиях равенства только 
значений сплайна на концах смежных отрезков (ку\-соч\-но-глад\-кая).
Результаты аппроксимации при параметре $\alpha\hm=0{,}07$ 
и~различных степенях полинома показаны на рис.~1.


Критерий Фишера устанавливает разное число узлов на разных сегментах. 
Способ построения признакового пространства в~таком случае заключается\linebreak\vspace*{-12pt}

{ \begin{center}  %fig1
 \vspace*{-2pt}
   \mbox{%
 \epsfxsize=79mm 
 \epsfbox{str-1.eps}
 }


\end{center}


\noindent
{{\figurename~1}\ \ \small{Аппроксимация сплайном с~динамическими узлами: \textit{1}~---
    без аппроксимации;
    \textit{2}~--\textit{4}~--- степени~1--3}}
}

\vspace*{12pt}

\addtocounter{figure}{1}


\noindent
 в~дополнении 
всех векторов нулевыми компонентами до вектора одинаковой размерности. В~качестве 
альтернативного метода рассмотрим сле\-ду\-ющую перестановку компонент получившегося 
вектора.

Назовем $\norm{\mathbf{y}-\mathbf{g}}_2^2$ относительным изменением импульса 
на сегменте,
где~$\mathbf{g}\hm=(0;0;9{,}8)$~м/с$^2$~--- ускорение 
свободного падения. Отсортируем отрезки по убыванию относительных 
изменений\linebreak импульса, запишем в~соответствующем порядке\linebreak параметры 
сплайнов и~дополним все векторы ну\-левыми компонентами до векторов 
одинаковой размерности. Поскольку производится классификация данных с~акселерометра, 
такая сортировка позволяет соотносить между собой отрезки времени с~наибольшими 
изменениями импульса датчика для различных сегментов.

Параметрами модели в~этом случае являются
\begin{equation}
\label{7-str}
\mathbf{b} =\left[\boldsymbol{\Theta}_{K \times N}, \alpha, K, N\right]\,.
\end{equation}

\subsection{Альтернативные локальные модели}
Рассмотрим некоторые альтернативные методы построения признакового пространства.

\textbf{Дискретное преобразование Фурье.} 
В~качестве признакового описания временн$\acute{\mbox{о}}$го ряда берутся коэффициенты 
прямого преобразования Фурье:
\begin{align*}
 \theta(\mathbf{x})&=
 \left[\theta_1,\dots,\theta_{2p}\right];\\
 \theta_{2k-1}+i\theta_{2k}&=\sum\limits_{j=1}^p x^j 
 \exp\left(-\fr{2\pi i}{t}kj\right).
\end{align*}
Тогда обратное преобразование задает аппроксимацию: 
$$
a(x^k)=\fr{1}{t}\sum\limits_{j=1}^t\left(\theta_{2j-1}+i\theta_{2j}\right)
\exp\left( -\fr{2\pi i}{t}kj\right).
$$
Коэффициенты преобразования, образующие признаковое про\-стран\-ст\-во, 
отыскиваются с~по\-мощью линейной регрессии временн$\acute{\mbox{о}}$го 
ряда на столбцы мат\-ри\-цы Фурье. Применение этого метода для классификации 
рас\-смат\-ри\-ва\-ет\-ся в~статье~\cite{Karasikov2016}.

\textbf{Анализ сингулярного спектра.} 
Элементы признакового пространства~--- сингулярные числа 
траекторной мат\-ри\-цы~$\mathbf{X}_N$, отвечающие за величины различных час\-тот 
спект\-ра сегмента
\begin{equation}
\label{8-str}
\textbf{X}^\mathsf{T}\textbf{X}=\textbf{V}\boldsymbol{\Lambda} 
\textbf{V}^\mathsf{T},\; \boldsymbol{\Lambda}=\mathrm{diag}\{\lambda_1,\ldots,\lambda_p\}.
\end{equation}
Тогда собственные числа матрицы~$\mathbf{X}^\mathsf{T}\textbf{X}$ 
образуют вектор признаков~$\boldsymbol{\Lambda}.$
Значение~$N$ назначается равным математическому ожиданию длины сегмента 
по критерию, описанному в~разделе сегментации. 
Результаты классификации в~таком пространстве описаны в~\cite{ivkin2015}.

\section{Сегментация временных рядов}

Исходные данные представляют собой размеченные несегментированные
 отрезки произвольной длины. Для построения выборки необходимо 
 их сегментировать: разбить временной ряд~$x_1,\ldots,x_L$ на сегменты 
 вида~$[x_1,\ldots,x_p]^\mathsf{T}$ длины~$p$.
Рассматриваются два варианта разбиения квазипериодических рядов 
на синфазные сегменты. Алгоритмы выделения периодов рассмотрены 
в~\cite{vasko2002estimating,Motrenko2015Fundamental}.

\textbf{Выделение главных компонент} заключается в~разложении 
временн$\acute{\mbox{о}}$го ряда
$\mathbf{x}=\hat{\mathbf{x}}+\tilde{\mathbf{x}}+\boldsymbol{\varepsilon},
$
где~$\mathbf{\hat{x}}$~--- тренд; $\mathbf{\tilde{x}}$~--- периодическая часть; 
$\boldsymbol{\varepsilon}$~--- вектор невязок.
Построим траекторную матрицу из элементов временн$\acute{\mbox{о}}$го 
ряда~$x_1,\ldots,x_L$:
$$
\mathbf{X}=\begin{bmatrix} 
x_1 & \cdots & x_p \\ 
\vdots & \ddots & \vdots \\ 
x_{L-p+1} & \cdots & x_L \end{bmatrix}\,.
$$
По первым собственным значениям матрицы (\ref{8-str})
$$
\textbf{X}^\mathsf{T}\textbf{X}=\textbf{V}\boldsymbol{\Lambda} 
\textbf{V}^\mathsf{T},\; \boldsymbol{\Lambda}=\mathrm{diag}
\left\{\lambda_1,\ldots,\lambda_p\right\},
$$
восстановим периодическую зависимость~$\tilde {\textbf{x}}\hm=\textbf{x}_1
+$\linebreak $+\cdots+\textbf{x}_d,$ где \mbox{$\mathbf{x}_i\hm=\sqrt{\lambda_i}
\textbf{v}_i(\textbf{Xv}_i)^\mathsf{T}$}.
Искомый период ряда~$\hat{p}\hm= 
\mathop{\mathrm{arg}}\,\min\nolimits_{p} 
\norm{\textbf{x} - \tilde{\textbf{x}} - \hat{\textbf{x}}}_2^2.$

\begin{figure*}[b] %fig2
\vspace*{12pt}
 \begin{center}  
  \mbox{%
 \epsfxsize=159.635mm 
 \epsfbox{str-2.eps}
 }
\end{center}
\vspace*{-11pt}
\Caption{Матрица ковариации для кубического~(\textit{а}) и~квадратичного~(\textit{б}) сплайна}
    \label{ris:cs}
%\end{figure*}
%\begin{figure*} %fig3
\vspace*{9pt}
 \begin{center}  
  \mbox{%
 \epsfxsize=162.546mm 
 \epsfbox{str-3.eps}
 }
\end{center}
\vspace*{-11pt}
    \Caption{Зависимость точности классификации от параметра сглаживания~(\textit{а}) 
    и~от числа базисных функций~(\textit{б})}
    \label{ris:sm1}
\end{figure*}

\textbf{Метод поиска локального максимума временного ряда.}
Для заданных чисел~$s$ и~$p$ в~каждом отрезке ищется индекс 
начала сегмента согласно условию
$
\hat{s}= \mathrm{arg}\,\underset{j<s}{\max}\; x_j$ и~в~качестве 
сегмента берется ~$[x_{\hat{s}},\ldots,x_{\hat{s}+p}]^\mathsf{T}.$ 
Для получения синфазных отрезков квазипериодических рядов 
предлагается использовать параметр~$s$, больший длины периода. 
Данный метод неустойчив к~выбросам, но требует малой вычислительной 
мощности. Сложность алгоритма~--- линейная по длине исходного сегмента.

\vspace*{-6pt}

\section{Вычислительный эксперимент}

В качестве вычислительного эксперимента выбрана задача классификации 
типов физической активности человека по данным с~акселерометра.

\vspace*{-6pt}

\paragraph*{Данные WISDM.} Данные представляют собой размеченные 
несегментированные трехмерные временн$\acute{\mbox{ы}}$е ряды, полученные с~датчиков 
ак\-се\-ле\-ро\-мет\-ра. Частота измерений составляла~20~Гц.\linebreak
В~выборке представлены классы sitting~(225), standing~(275), walking~(2890), 
jogging~(1631), upstairs~(801) и~downstairs~(657).
Дробление производилось на временн$\acute{\mbox{ы}}$е сегменты длины~50. 
Для выравнивания сегментов использовался поиск максимального значения 
за следующие несколько отсчетов, который задает точку начала сегмента.

Сегментация временн$\acute{\mbox{о}}$го ряда проводилась методом поиска 
максимума с~параметрами
$m\hm=20, k\hm=100$.   В~качестве классифицирующего алгоритма использована 
логистическая регрессия из библиотеки sklearn. Результаты классификации 
показаны на рис.~\ref{ris:cs}.

\begin{table*}\small 
\begin{center}
   
   
   % \label{tab:comp}
    
    \begin{tabular}{|l|c|c|}
     \multicolumn{3}{c}{Сравнение моделей локальной аппроксимации WISDM и~USC-HAD}\\
     \multicolumn{3}{c}{\ }\\[-6pt]
        \hline
        \multicolumn{1}{|c|}{Аппроксимирующая модель}&Оптимальный параметр&Точность\\
        \hline
        \multicolumn{3}{|c|}{Выборка WISDM}\\
        \hline
        Quadratic ~(\ref{2-str})&(\ref{4-str})&\hphantom{9}0,2540\\
        Cubic ~(\ref{2-str})&(\ref{4-str})&\hphantom{9}0,7305\\
        Smoothing spline ~(\ref{2-str})&$\lambda=0{,}25$ (\ref{5-str})&0,748\\
        MARS (\ref{3-str})&$M=4$ (\ref{6-str})\hphantom{0{,}99}&0,56\hphantom{9}\\
                \hline
        \multicolumn{3}{|c|}{Выборка USC-HAD} \\
        \hline
        Quadratic ~(\ref{2-str})&(\ref{4-str})&0,587\\
        Cubic  (\ref{2-str})&(\ref{4-str})&0,926\\
        Smoothing spline ~(\ref{2-str})&$\lambda=0{,}4$ (\ref{5-str})&0,960\\
        MARS (\ref{3-str})&$M=4$ (\ref{6-str})\hphantom{{,}99}&0,835\\
        Динамические узлы Sort ~(\ref{2-str})&$K=2$, $\alpha=0{,}015$ (\ref{7-str})&0,935\\
        Динамические узлы Unsorted (\ref{2-str})&$K=2$, $\alpha=0{,}01$\hphantom{9} (\ref{7-str})&0,926\\
        \hline
    \end{tabular}
    \end{center}
\end{table*}
\setcounter{figure}{4}



Точность классификации для квадратичной ап\-прок\-си\-ма\-ции составляет~$0{,}25$, 
что ниже результатов остальных исследованных методов аппроксимации. 


Зависимость точ\-ности модели от па\-ра\-мет\-ра~$p$ для сглаживающего сплайна 
можно наблюдать на рис.~\ref{ris:sm1},\,\textit{а}. 
Как видно из данного графика, максимальная точ\-ность достигается 
при па\-ра\-мет\-ре сглаживания~$p\hm=0{,}25$.

Для построения адаптивного регрессионного сплайна была выбрана библиотека ARESLAB. 
Для того чтобы размерность пространства параметров была одинакова для всех 
сегментов, аппроксимация происходила без фазы назад. Параметр такой модели~--- 
число базисных функций. График зависимости точности классификации от параметра 
метамодели показан на рис.~\ref{ris:sm1},\,\textit{б}. Рассмотрены различные 
отображения~$h$ временн$\acute{\mbox{ы}}$х рядов в~признаковое пространство. 
Данные отображения можно сравнить с~тождественным, т.\,е.\ 
таким, при котором в~качестве матрицы плана выступает матрица значений временн$\acute{\mbox{о}}$го 
ряда в~фиксированные моменты времени. Точность и~оптимальные параметры моделей 
представлены для сравнения в~таблице. %~\ref{tab:comp}.



\vspace*{-9pt}

\paragraph*{Данные USC-HAD.} Данные представляют собой размеченные временн$\acute{\mbox{ы}}$е 
ряды различной длины с~датчиков акселерометра. Даны проекции ускорения на три 
оси с~частотой~100~Гц.
В выборке представлены классы walking forward, jumping up, 
walking left, sitting, walking right, standing, walking upstairs, 
sleeping, walking downstairs, elevator up, running forward, elevator down.


Рассмотрим теперь аппроксимацию гладкими сплайнами с~динамическими узлами  
и~ку\-соч\-но-глад\-ки\-ми, о~которых шла речь в~подразд.~3.4.
После аппроксимации ими сегментов имеется возможность менять порядок получаемых 
признаков.
Можно задать его пропорциональным длине кривой между двумя соседними 
узлами или же пропорционально относительному изменению импульса. 
Также можно оставить его без изменения.

Все вышеизложенные варианты дают всего~6~различных вариантов 
(с~точ\-ностью до степени аппроксимирующего многочлена) задания призна-\linebreak\vspace*{-12pt}

{ \begin{center}  %fig4
 \vspace*{-4pt}
   \mbox{%
 \epsfxsize=78.952mm 
 \epsfbox{str-6.eps}
 }


\end{center}


\noindent
{{\figurename~4}\ \ \small{Точность классификации для сплайнов с~динамическими узлами: пустые
    значки~--- неотсортированные сглаженные;
    черные~--- неотсортированные;
    серые значки~---  отсортированные по энергии; \textit{1}~---  линейный метод;
    \textit{2}~--- квадратичный; \textit{3}~--- кубический метод}}
}

\vspace*{8pt}

\addtocounter{figure}{1}

\noindent
кового 
пространства. На рис.~4 изображено качество 
классификации в~зависимости от значения параметра~$p$.




Чтобы результаты классификации для этой выборки можно было сравнить с~предыдущей, 
выделим  также~6~классов, по своей природе совпа\-да\-ющих или близких к~WISDM. 
Возьмем классы~1, 6, 7, 8, 9 и~10. Будем их для краткости называть walking, 
running, jumping, sitting, standing и~sleeping. Сегментация временн$\acute{\mbox{о}}$го 
ряда также производилась методом поиска максимума с~параметрами
$m\hm=50$, $k\hm=200$. Результаты сравнения моделей представлены в~таблице.


На рис.~5 показана зависимость точ\-ности 
классификации от па\-ра\-мет\-ра сглаживания (рис.~5.\,\textit{а})
и~от~чис\-ла базисных функ\-ций (рис.~5,\,\textit{б}).
Зависимость погрешности аппроксимации SMAPE от чис\-ла базисных функций для 
регрессионного сплайна показана на рис.~6.


\setcounter{figure}{4}
\begin{figure*} %fig5
\vspace*{1pt}
 \begin{center}  
  \mbox{%
 \epsfxsize=162.555mm 
 \epsfbox{str-4.eps}
 }
\end{center}
\vspace*{-13pt}
 \Caption{Зависимость точности классификации от параметра сглаживания~(\textit{а}) 
 и~от числа базисных функций~(\textit{б})}
    \label{ris:sm2}
    \vspace*{-6pt}
\end{figure*}




{ \begin{center}  %fig6
 \vspace*{-6pt}
  \mbox{%
 \epsfxsize=78.603mm 
 \epsfbox{str-5.eps}
 }


\end{center}


\noindent
{{\figurename~6}\ \ \small{Зависимость погрешности SMAPE от числа базисных функций}}
}

%\vspace*{12pt}

\addtocounter{figure}{1}

\vspace*{-9pt}

\section{Заключение}

\vspace*{-1pt}

В работе были описаны различные параметрические модели аппроксимации 
временн$\acute{\mbox{ы}}$х 
рядов. Проанализирована их эффективность при по\-стро\-ении классификатора. 
Предложен метод классификации временн$\acute{\mbox{ы}}$х рядов при помощи по\-стро\-ения 
пространства коэффициентов локальных аппроксимирующих моделей. 
Сравнивался ряд таких моделей.  Квадратичный сплайн оказался 
непригоден\linebreak для классификации, остальные отображения по\-казали результат, 
сравнимый с~тождественным отоб\-ра\-же\-ни\-ем. Оптимальные параметры классификатора 
повышают точность на 3\%--5\%. Предложенный метод позволяет получить 
классификатор без затрат времени на ручную генерацию признаков.
{\looseness=-1

}

\vspace*{-9pt}

{\small\frenchspacing
 {%\baselineskip=10.8pt
 \addcontentsline{toc}{section}{References}
 \begin{thebibliography}{99}
 
 \vspace*{-1pt}
 
 \bibitem{Farmer1987} %1
\Au{Farmer~J.\,D., Sidorowich~J.\,J.} 
Predicting chaotic time series~// Phys. Rev. Lett.,~1987. Vol.~59. No.\,8. P.~845--848.

\bibitem{Karasikov2016} %2
\Au{Карасиков~М.\,Е., Стрижов~В.\,В.} 
Классификация временных рядов в~пространстве параметров порожда\-ющих моделей~// 
Информатика и~её применения,~2008. T.~10. Вып.~4. С.~121--131.



\bibitem{myas1970}
\Au{Мясников~Л.\,Л., Мясникова~Е.\,Н.} 
Автоматическое распознавание звуковых образов.~--- Л.: Энергия, 1970.  183~с.

\bibitem{Gary2011}
\Au{Kwapisz~J.\,R., Weiss~G.\,M., Moore~S.\,A.} 
Activity recognition using cell phone accelerometers~// 
ACM SigKDD Explorations Newsletter, 2011. Vol.~12. No.\,2. P.~72--82.

\bibitem{ivkin2015}
\Au{Кузнецов~М.\,П., Ивкин~Н.\,П.} 
Алгоритм классификации временных рядов акселерометра по комбинированному 
признаковому описанию~// Машинное обучение и~анализ данных,~2015. Т.~1. №\,11. 
С.~1471--1483.

%6
\bibitem{Tsay2010} %6
\textit{Tsay~R.\,S.} Analysis of financial time series.~--- 
Hoboken, NJ, USA:  Wiley, 2006. 712~p.


\bibitem{Fu2011} %7
\Au{Fu~C.} A~review on time series data mining~// 
Eng. Appl. Artif. Intel., 2011. Vol.~24. P.~164--181.

\bibitem{Esl2012} %8
\Au{Esling~P., Agon~C.} Time series data mining~// 
ACM Comput. Surv., 2012. Vol.~45. No.\,1. P.~1--34.


\bibitem{SEMOR2004}
\Au{Coull~B.\,A., Staundenmayer~J.} 
Self modeling regression for multivariate curve data~// 
Stat. Sinica, 2004. Vol.~14. P.~695--711.

\bibitem{Fatma}
\Au{Weber G.-W., Batmaz~I., K$\ddot{\mbox{o}}$ksal~G., \textit{et al.}} 
CMARS: A~new contribution to nonparametric regression 
with multivariative adaptive regression splines supported by continuous optimisation~// 
Inverse Probl.  Sci.  En., 2012. Vol.~20. No.~3. P.~371--400.
doi: 10.1080/17415977.2011.624770.


\bibitem{Istomin2005} %11
\Au{Истомин~И.\,А., Котляров~О.\,Л., Лоскутов~А.\,Ю.} 
К~проблеме обработки временных рядов: расширение возможностей метода 
локальной аппроксимации посредством сингулярного спектрального анализа~// 
Теоретическая и~математическая физика, 2005. Т.~142. №\,1. С.~148--160.


\bibitem{Det2007} %12
\Au{Dette~H., Melas~V.\,B., Pepelyshev~A.} Optimal design for smoothing splines~// 
Ann. I.~Stat. Math., 2007. Vol.~63. No.\,5. P.~981--1003.

\bibitem{Tselykh2012} %13
\Au{Целых~В.\,Р.} Многомерные адаптивные регрессионные сплайны~// 
Машинное обучение и~анализ данных, 2012. Т.~1. №\,3. С.~272--278.


\bibitem{Gholizadeh2009}
\Au{Gholizadel~S., Salajegheh~E.} Optimal design of structures subjected 
to time history loading by swarm intelligence and an advanced metamodels~// 
Comput. Method. Appl.~M., 2009. Vol.~198. P.~2936--2949.

\bibitem{vasko2002estimating}
\Au{Vasko~K., Toivonen~H.} Estimating the number of segments in 
time series data using permutation tests~//  Conference 
(International) on Data Mining Proceedings.~--- IEEE, 2006. P.~466--473.

\bibitem{Motrenko2015Fundamental}
\Au{Motrenko A.\,P., Strijov~V.\,V.} 
Extracting fundamental periods to segment biomedical signals~// IEEE J.~Biomed. 
Health,~2015. Vol.~20. No.\,6. P.~1466--1476.
 \end{thebibliography}

 }
 }

\end{multicols}

\vspace*{-3pt}

\hfill{\small\textit{Поступила в~редакцию 23.05.18}}

\vspace*{8pt}

%\pagebreak

%\newpage

%\vspace*{-28pt}

\hrule

\vspace*{2pt}

\hrule

%\vspace*{-2pt}

\def\tit{LOCAL APPROXIMATION MODELS FOR~HUMAN PHYSICAL ACTIVITY CLASSIFICATION}


\def\titkol{Local approximation models for~human physical activity classification}

\def\aut{D.\,A.~Anikeyev$^1$, G.\,O.~Penkin$^1$, and~V.\,V.~Strijov$^{1,2}$}

\def\autkol{D.\,A.~Anikeyev, G.\,O.~Penkin, and~V.\,V.~Strijov}

\titel{\tit}{\aut}{\autkol}{\titkol}

\vspace*{-11pt}


\noindent
    $^1$Moscow Institute of Physics and Technology, 
9~Institutskiy Per., Dolgoprudny, Moscow Region 141700, Russian\linebreak
$\hphantom{^1}$Federation

%\noindent
%$^2$M.\,V.~Lomonosov Moscow State University?????

\noindent
$^2$A.\,A.~Dorodnicyn Computing Center, Federal Research Center ``Computer Science and Control'' 
of Russian\linebreak
$\hphantom{^1}$Academy of Sciences, 40~Vavilov Str., Moscow 119333, Russian Federation

\def\leftfootline{\small{\textbf{\thepage}
\hfill INFORMATIKA I EE PRIMENENIYA~--- INFORMATICS AND
APPLICATIONS\ \ \ 2019\ \ \ volume~13\ \ \ issue\ 1}
}%
 \def\rightfootline{\small{INFORMATIKA I EE PRIMENENIYA~---
INFORMATICS AND APPLICATIONS\ \ \ 2019\ \ \ volume~13\ \ \ issue\ 1
\hfill \textbf{\thepage}}}

\vspace*{6pt}




\Abste{The research is devoted to the time series classification. 
The time series is measured by an accelerometer of a~wearable device. 
A~class of physical activity is defined by its feature description of 
a~time segment. To construct this description, the authors propose to use parameters 
of various approximation splines (algebraic, smoothing, adaptive regression, 
or spline with dynamic nodes). The logistic regression is used as a~classifier. 
It delivers desired quality of the activity recognition.
        The authors analyze the space of the local approximation parameters. 
        Classification accuracy depends on the method of this space construction. 
        The computational experiment finds the optimal approximation parameters 
        and parameters of the classifier.}
        
        
        \KWE{local approximation model; time series; classification; 
        splines; feature space}


\DOI{10.14357/19922264190106}

\vspace*{-14pt}

\Ack
\noindent
The work was partially supported by the Russian Foundation for Basic Research 
(projects 17-07-1574 and~19-07-1155).



\vspace*{6pt}

  \begin{multicols}{2}

\renewcommand{\bibname}{\protect\rmfamily References}
%\renewcommand{\bibname}{\large\protect\rm References}

{\small\frenchspacing
 {%\baselineskip=10.8pt
 \addcontentsline{toc}{section}{References}
 \begin{thebibliography}{99}
 
 \bibitem{Farmer1987-1} %1
\Aue{Farmer, J.\,D., and J.\,J.~Sidorowich.} 
1987. Predicting chaotic time series. \textit{Phys. Rev. Lett.} 59(8):845--848.

\bibitem{Karasikov2016-1} %2
\Aue{Karasikov, M.\,E., and V.\,V.~Strijov.} 
2008. Klassifikatsiya vremennykh ryadov v~prostranstve 
parametrov porozhdayushchikh modeley 
[Feature-based time-series classification].
 \textit{Informatika i~ee Primeneniya~--- Inform. Appl.} 10(4):128--138.





\bibitem{myas1970-1} %3
\Aue{Myasnikov, L.\,L., and E.\,N.~Myasnikova.} 1970. \textit{Avtomaticheskoe 
raspoznavanie zvukovykh obrazov} [Automatic speech recognition]. 
Moscow: Energiya. 183~p.

\bibitem{Gary2011-1} %4
\Aue{Kwapisz, J.\,R., G.\,M.~Weiss, and S.\,A.~Moore.}
 2011. Activity recognition using cell phone accelerometers. 
 \textit{ACM SigKDD Explorations Newsletter} 12(2):72--82.

\bibitem{ivkin2015-1} %5
\Aue{Kuznetsov, M.\,P., and N.\,P.~Ivkin.} 2015. Algoritm klassifikatsii 
vremennykh ryadov akselerometra po kombinirovannomu priznakovomu opisaniyu 
[Time series classification algorithm using combined feature description]. 
\textit{Mashinnoe obuchenie i~analiz dannykh} 
[{Machine Learning Data Anal.}] 1(11):1471--1483.

\bibitem{Tsay2010-1} %6
\Aue{Tsay, R.\,S.} 2010. \textit{Analysis of financial time series.} 
Hoboken, NJ: {Wiley.} 712~p.



\bibitem{Fu2011-1} %7
\Aue{Fu, C.} 2011. A~review on time series data mining. 
\textit{Eng. Appl. Artif. Intel.} 24:164--181.

\bibitem{Esl2012-1} %8
\Aue{Esling, P., and C.~Agon.} 2012. Time series data mining. 
\textit{ACM Comput. Surv.} 45(1):1--34.

\bibitem{SEMOR2004-1} %9
\Aue{Coull, B.\,A., and J.~Staundenmayer.} 2004. Self modeling regression for
 multivariate curve data. \textit{Stat. Sinica} 14:695--711.

\bibitem{Fatma-1} %10
%\Aue{Fatma, Y.} 2011. \textit{CMARS: A~new contribution to nonparametric 
%regression with multivariative adaptive regression splines 
%supported by continuous optimisation}. Lambert. 212~p.
\Aue{Weber, G.-W., I.~Batmaz, G.~K$\ddot{\mbox{o}}$ksal, \textit{et al.}} 2012.
CMARS: A~new contribution to nonparametric regression 
with multivariative adaptive regression splines supported by continuous optimisation. 
\textit{Inverse Probl.  Sci.  En.} 20(3):371--400.
doi: 10.1080/17415977.2011.624770.

\bibitem{Istomin2005-1} %11
\Aue{Istomin, I.\,A., O.\,L.~Kotlyarov, and A.\,Yu.~Loskutov.} 2005. 
The problem of processing time series: Extending possibilities 
of the local approximation method using singular spectrum analysis. 
\textit{Theor. Math. Phys.} 142(1):128--137.

\bibitem{Det2007-1} %12
\Aue{Dette, H., V.\,B.~Melas, and A.~Pepelyshev.} 2011. Optimal design for 
smoothing splines \textit{Ann.~I.~Stat. Math.} 
63(5):981--1003.

\bibitem{Tselykh2012-1} %13
\Aue{Tselykh, V.\,R.} 2012. Mnogomernye adaptivnye regressionnye splayny 
[Multivariate adaptive regression splines]. \textit{Mashinnoe obuchenie 
i~analiz dannykh} [{Machine Learning Data Anal.}] 1(3):272--278.  



\bibitem{Gholizadeh2009-1} %14
\Aue{Gholizadel, S., and E.~Salajegheh.} 2009. 
Optimal design of structures subjected to time history loading by swarm 
intelligence and an advanced metamodels. 
\textit{Comput. Method. Appl.~M.} 198:2936--2949.



\bibitem{vasko2002estimating-1} %15
\Aue{Vasko, K., and H.~Toivonen.}
 2002. Estimating the number of segments in time series data using 
 permutation tests. \textit{Conference (International) on Data Mining
 Proceedings} 466--473.

\bibitem{Motrenko2015Fundamental-1} %16
\Aue{Motrenko, A.\,P., and V.\,V.~Strijov.} 2016. 
Extracting fundamental periods to segment biomedical signals. 
\textit{IEEE J.~Biomed. Health} 20(6):1466--1476.
\end{thebibliography}

 }
 }

\end{multicols}

\vspace*{-6pt}

\hfill{\small\textit{Received May 23, 2018}}

%\pagebreak

%\vspace*{-18pt}


\Contr

\noindent
\textbf{Anikeyev Dmitriy A.} (b.\ 1995)~--- 
student, Moscow Institute of Physics and Technology, 
9~Institutskiy Per., Dolgoprudny, Moscow Region 141700, Russian Federation; 
\mbox{dmitriy.anikeyev@phystech.edu}

\vspace*{3pt}

\noindent
\textbf{Penkin Grigoriy O.} (b.\ 1993)~--- 
PhD student, Moscow Institute of Physics and Technology, 
9~Institutskiy Per., Dolgoprudny, Moscow Region 141700, Russian Federation; 
\mbox{penkin.gr@gmail.com}

\vspace*{3pt}

\noindent
\textbf{Strijov Vadim V.} (b.\ 1967)~--- 
Doctor of Science in physics and mathematics, leading scientist, 
A.\,A.~Dorodnicyn Computing Center, Federal Research Center ``Computer Science and Control'' 
of Russian Academy of Sciences, 40~Vavilov Str., Moscow 119333, Russian Federation;
professor,  Moscow Institute of Physics and Technology, 
9~Institutskiy Per., Dolgoprudny, Moscow Region 141700, Russian Federation;
\mbox{strijov@ccas.ru}
\label{end\stat}

\renewcommand{\bibname}{\protect\rm Литература}  