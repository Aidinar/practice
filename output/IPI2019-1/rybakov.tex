

%\newcommand{\tr}{\mathop{\rm tr}}
\newcommand{\trans}{{{\mathrm{T}}}}

\def\stat{rybakov}

\def\tit{ОБ ОДНОМ КЛАССЕ ЗАДАЧ ФИЛЬТРАЦИИ НА~МНОГООБРАЗИЯХ$^*$}

\def\titkol{Об одном классе задач фильтрации на~многообразиях}

\def\aut{К.\,А.~Рыбаков$^1$}

\def\autkol{К.\,А.~Рыбаков}

\titel{\tit}{\aut}{\autkol}{\titkol}

\index{Рыбаков К.\,А.}
\index{Rybakov K.\,A.}


{\renewcommand{\thefootnote}{\fnsymbol{footnote}} \footnotetext[1]
{Работа выполнена при поддержке РФФИ (проект 17-08-00530-а).}}


\renewcommand{\thefootnote}{\arabic{footnote}}
\footnotetext[1]{Московский авиационный институт (национальный исследовательский университет), 
\mbox{rkoffice@mail.ru}}

%\vspace*{8pt}



\Abst{Цель статьи состоит в~описании стохастических дифференциальных 
систем, траектории которых находятся на гладком многообразии, в~приложении к~задаче 
оптимальной фильтрации. Дополнительным условием является принадлежность этому же 
многообразию не только траекторий системы, но и~результата оценивания 
этих траекторий на основе косвенных измерений, а~именно: решения задачи 
оптимальной фильтрации по критерию минимума среднеквадратической ошибки оценивания. 
Рас\-смат\-ри\-ва\-ют\-ся системы как диффузионного типа, так и~диф\-фу\-зи\-он\-но-скач\-ко\-об\-раз\-но\-го 
типа, т.\,е.\ при наличии как винеровских, так и~пуассоновских возмущений.
 Результатом являются условия на коэффициенты уравнения для случайного
  процесса, траектории которого требуется оценить. 
  В~основе полученных условий лежит понятие первого интеграла стохастических 
  дифференциальных уравнений, а также некоторые его свойства.}

\KW{инвариант; многообразие; оптимальная фильтрация; оценивание; 
случайный процесс; стохастическая дифференциальная система}

\DOI{10.14357/19922264190103}
  
%\vspace*{4pt}


\vskip 10pt plus 9pt minus 6pt

\thispagestyle{headings}

\begin{multicols}{2}

\label{st\stat}

\section{Введение}

В теории стохастических динамических систем задача фильтрации 
имеет важное значение, она\linebreak состоит в~нахождении оценки 
ненаблюдаемого вектора состояния системы по результатам его косвенных измерений. 
Критерии оптимальности оценки можно задавать различным образом, во многих 
приложениях ограничиваются критерием минимума среднеквадратической ошибки оценивания. 
Такие задачи решаются для непрерывных, дискретных и~не\-пре\-рыв\-но-дис\-крет\-ных 
систем, в~том числе и~в~случае, когда вектор состояния системы принадлежит 
заданному многообразию. Решению задач фильтрации на многообразиях посвящен 
целый ряд работ, опубликованных в~последнее 
время~\cite{Sin_SSI16, Sin_IP16, SinSinKor_SSI16, SinSinKor_IP17, SinSinSerKor_AiT18}.

Принадлежность вектора состояния стохастической динамической системы 
многообразию означает, что в~системе выполняется некоторый закон сохранения. 
В~более общем случае рассматривается принадлежность пары <<вре\-мя\;+\;со\-сто\-яние>> 
динамическому многообразию. Методы описания и~построения подобных непрерывных 
стохастических систем подробно изложены в~\cite{Dub_89, Dub_12, Kar_14, Kar_15}. 
Но этот закон сохранения для оценки вектора состояния при решении 
задачи фильтрации выполняться не будет, кроме специального класса систем.

Перейдем к~иллюстрирующему примеру. 
Пусть векторный случайный процесс $X(t) \hm= [  X_1(t) ~ X_2(t)  ]^\trans$ 
со значениями в~$\mathds{R}^2$ удовлетворяет линейному стохастическому 
дифференциальному уравнению Ито

\noindent
$$
  dX(t) = \underbrace{\left[ \begin{array}{cc}
    -1 & 1 \\
    -1 & -1 \\
  \end{array} \right]}_F X(t) \, dt 
  +
  \underbrace{\left[ \begin{array}{cc}
    0 & \sqrt 2 \\
    -\sqrt 2 & 0 \\
  \end{array} \right]}_S X(t) \, dW(t)\,,
$$
в котором $W(t)$~--- скалярный винеровский процесс. 
Соответствующее линейное стохастическое дифференциальное 
уравнение Стратоновича имеет вид:
\begin{multline*}
  d_{1/2} X(t) = \underbrace{\left[ \begin{array}{cc}
    0 & 1 \\
    -1 & 0 \\
  \end{array} \right]}_A X(t) \, dt +{}\\
  {}+
  \underbrace{\left[ \begin{array}{cc}
    0 & \sqrt 2 \\
    -\sqrt 2 & 0 \\
  \end{array} \right]}_S X(t) \, d_{1/2} W(t).
\end{multline*}

Аналитическое решение этих уравнений можно записать, 
используя матричные экспоненты \cite{Ave_ACMMENSP17}: 
$X(t) \hm= \exp{At} \exp{SW(t)}  X(t_0)$, $t\hm \geqslant t_0$. 
Несложно проверить, что собственные числа матриц~$A$ и~$S$~--- 
комп\-лекс\-но-со\-пря\-жен\-ные с~нулевой действительной частью, определители 
матриц~$\exp{At}$ и~$\exp{SW(t)}$ равны единице и~эти матрицы задают 
ортогональные линейные преобразования на плоскости, а~именно: повороты 
вокруг начала координат. Такие преобразования сохраняют расстояние между 
точками в~$\mathds{R}^2$: $|X(t)| \hm= |X(t_0)|$, поэтому 
траектории случайного процесса $X(t)$ принадлежат круговому цилиндру в~$\mathrm{T} \times \mathds{R}^2$, а фазовые траектории~--- это окружности 
с~цент\-ром в~начале координат (точка покоя~--- центр), радиус 
которых определяется начальными данными~$X(t_0)$.

Математическое ожидание для случайного процесса~$X(t)$ 
также можно выразить с~помощью мат\-рич\-ной экспоненты: $\mathbb{E} X(t) 
\hm= \exp{Ft} \, \mathbb{E} X(t_0)$ (здесь и~далее $\mathbb{E}$~--- 
знак математического ожидания), собственные числа матрицы~$F$~--- комп\-лекс\-но-со\-пря\-жен\-ные 
с~отрицательной действительной частью и~для математического ожидания 
верно соотношение:
$$
{e}^t |\mathbb{E} X(t)|
 ={e}^{t_0} \left\vert \mathbb{E} X(t_0)\right\vert \,. 
 $$

Таким образом, траектории случайного процесса и~его математическое 
ожидание принадлежат разным многообразиям $\mathrm{T} \times \mathds{R}^2$. 
Кривая математического ожидания принадлежит конусу 
в~$\mathrm{T} \times \mathds{R}^2$~--- динамическому многообразию, а проекция 
этой кривой на фазовую плоскость~--- это логарифмическая спираль (точка покоя~--- 
устойчивый фокус).

Если же рассмотреть задачу оптимальной фильт\-ра\-ции, т.\,е.\ оценивать траектории 
случайного процесса по результатам косвенных измерений, то результаты оценивания 
по критерию минимума\linebreak среднеквадратической ошибки не будут удовлетворять ни одному 
из указанных инвариантных соотношений и,~следовательно, закон сохранения, который 
выполняется для траекторий случайного процесса~$X(t)$, не будет выполняться для 
результатов оценивания этих траекторий. Результаты оценивания будут находиться 
в~об\-ласти, ограниченной круговым цилиндром и~конусом.

Цель статьи состоит в~описании класса стохастических дифференциальных 
систем, для которых одному и~тому же многообразию принадлежат не 
только траектории системы, но и~результат решения задачи оптимальной 
фильтрации по критерию минимума среднеквадратической ошибки оценивания.

Статья помимо введения и~заключения содержит~5~разделов. В~разд.~2 
сформулирована задача оптимальной фильтрации на гладком многообразии. 
В~разд.~3 приведены необходимые и~достаточные условия принадлежности 
траекторий стохастической системы заданному многообразию. Раздел~4 
содержит основной результат статьи~--- необходимые и~достаточные условия
 принадлежности решения задачи оптимальной фильтрации по критерию минимума 
 среднеквадратической ошибки оценивания многообразию, которому принадлежат 
 оцениваемые траектории. В~разд.~5 результаты, полученные в~разд.~4, 
 обобщаются на стохастические системы при наличии пуассоновских возмущений. 
 В~разд.~6 приводится модельный пример стохастической системы, для 
 которой одному и~тому же многообразию принадлежат не только 
 траектории этой системы, но и~результат решения задачи оптимальной фильтрации.



\section{Постановка задачи оптимальной фильтрации}

В работе рассматривается стохастическая система наблюдения, 
задаваемая стохастическими дифференциальными уравнениями Ито:

\vspace*{-2pt}

\noindent
\begin{align}
\label{eqItoX}
\hspace*{-2mm}  dX(t)& = f \left( t,X(t) \right) dt + \sigma \left( t,X(t) \right) dW(t)\,,\notag\\
  &\hspace*{44mm} X\left(t_0\right) = X_0\,;
\\
\label{eqItoY}
\hspace*{-2mm}  dY(t) &= c \left( t,X(t) \right) dt + \zeta(t) dV(t)\,,\enskip Y\left(t_0\right) = Y_0\,,
\end{align}
где $X \in \mathds{R}^n$~--- ненаблюдаемый вектор состояния; 
$Y \hm\in \mathds{R}^m$~--- вектор измерений; $t \hm\in \mathrm{T}$, $\mathrm{T}\hm = 
[t_0,T]$~--- заданный отрезок времени; $W(t)$ и~$V(t)$~--- $s$- 
и~$d$-мер\-ные независимые винеровские процессы; $f(t,x)$, $c(t,x)$,
 $\sigma(t,x)$ и~$\zeta(t)$~--- заданные век\-тор-функ\-ции и~матричные 
 функции соответствующих размеров; $X_0$ и~$Y_0$~--- начальный вектор 
 состояния и~начальный вектор измерений. Распределение~$X_0$ известно, а~$Y_0$, 
 как правило,~--- нулевой вектор. Функции $f(t,x)$, $c(t,x)$, 
 $\sigma(t,x)$ и~$\zeta(t)$ удовлетворяют условиям существования 
 и~единственности решения стохастических дифференциальных уравнений~\cite{OksSul_05}. 
 Кроме того, $\eta(t) = \zeta(t) \  \zeta^\trans(t)$~--- не\-вы\-рож\-ден\-ная мат\-ри\-ца, 
 для которой существует обратная матрица $q(t) \hm= \eta^{-1}(t)$.

Предполагается, что траектории случайного процесса~$X(t)$ 
принадлежат гладкому многообразию $\mathcal{M} \hm\subset \mathrm{T} \times 
\mathbb{R}^n$, которое определяется соотношением $\mathcal{M} \hm= 
\{(t,x) \in \mathrm{T} \times \mathbb{R}^n \colon M(t,x) \hm= C \hm= 
{const}\}$. Здесь $M(t,x)$~--- скалярная функция, не равная 
постоянной, непрерывно дифференцируемая по переменной~$t$ и~дважды 
непрерывно дифференцируемая по координатам вектора~$x$. Стохастическую систему, 
которая задается уравнением~\eqref{eqItoX}, будем называть инвариантной. 
Для инвариантной системы почти наверное $(t,X(t)) \hm\in \mathcal{M}$, если 
$(t_0,X_0)\hm \in \mathcal{M}$, т.\,е.\ $M(t,X(t))\hm = M(t_0,X_0)$. 
Подобных ограничений на случайный процесс~$Y(t)$ не накладывается.

Задача оптимальной фильтрации состоит в~нахождении 
оценки~$\hat X(t)$ по результатам измерений $Y_0^t \hm= \{Y(\tau)$,
$ \tau \hm\in [t_0,t) \}$, т.\,е.\ $\hat X(t)\hm = \psi(t,Y_0^t)$, 
где $\psi(t,Y_0^t)$~--- функция, обеспечивающая в~каждый момент времени $t \hm\in 
\mathrm{T}$ выполнение условия:

\vspace*{2pt}

\noindent
$$
  \mathbb{E} \left[ \left( X(t) - \hat X(t) \right)^\trans \left( X(t) - \hat X(t) \right) 
  \right] \rightarrow \min\limits_{\psi(t, \, \cdot \, )}\,,
$$

\vspace*{-2pt}

\noindent
т.\,е.\ решается задача оптимальной фильтрации по критерию минимума 
среднеквадратической ошибки оценивания.

Решение этой задачи записывается в~виде апостериорного математического 
ожидания~\cite{Sin_07}:

\vspace*{2pt}

\noindent
$$
  \hat X(t) = \mathbb{E} \left[ 
  X(t) | Y_0^t \right] = \int\limits_{\mathds{R}^n} x p(t,x|Y_0^t) \,dx\,,
$$


\noindent
или~\cite{BaiCri_09}:
\begin{equation}
\label{eqParticleEstimation}
  \hat X(t) = \fr{\mathbb{E}[\omega(t) X(t)]}{\mathbb{E} \, \omega(t)}\,,
\end{equation}
где $p(t,x|Y_0^t)$~--- апостериорная плот\-ность ве\-ро\-ят\-ности вектора состояния~$X$; 
$\omega(t)$~--- весовая функция:

\vspace*{-4pt}

\noindent
\begin{multline*}
  \omega(t) = \exp \left\{ \int\limits_{t_0}^t c^\trans(\tau,X(\tau)) 
  q(\tau) \,dY(\tau) -{}\right.\\
\left.  {}- \fr{1}{2} \int\limits_{t_0}^t c^\trans(\tau,X(\tau)) 
  q(\tau) c(\tau,X(\tau)) \,d\tau \right\}.
\end{multline*}

Отметим, что на последнем соотношении основан непрерывный фильтр 
частиц и~его различные варианты~\cite{BaiCri_09, Ryb_17}.

\vspace*{-4pt}

\section{Условия инвариантности}

Запишем стохастическое дифференциальное уравнение в~форме 
Стратоновича для случайного процесса~$X(t)$:

\vspace*{-2pt}

\noindent
\begin{multline}
\label{eqStr}
  d_{1/2} X(t) = a \left( t,X(t) \right) dt + \sigma \left( t,X(t) \right) 
  d_{1/2} W(t)\,,\\ 
   X\left(t_0\right) = X_0\,.
\end{multline}
В этом уравнении $a(t,x)$~--- век\-тор-функ\-ция 
той же размерности, что и~функция~$f(t,x)$:
\begin{equation}
\label{eqStr2Ito}
  a(t,x) = f(t,x) - \fr{1}{2} \sum\limits_{l = 1}^s 
  {\fr{\partial \sigma_{* l}(t,x)}{\partial x} \, \sigma_{* l}(t,x)}\,,
\end{equation}
где $\sigma_{* l}(t,x)$~--- столбец матричной функции~$\sigma(t,x)$ с~номером~$l$, 
$l \hm= 1,2,\dots,s$.

Для функций $M(t,x)$, $a(t,x)$ и~$\sigma(t,x)$, определяющих гладкое 
многообразие~$\mathcal{M}$ и~уравнение~\eqref{eqStr}, на траекториях 
случайного процесса~$X(t)$ должны выполняться следующие условия:
\begin{equation}
\label{eqCondition1}
  \fr{\partial M(t,x)}{\partial t} + \sum\limits_{i=1}^n a_i(t,x)  
  \fr{\partial M(t,x)}{\partial x_i} = 0\,;
\end{equation}
\begin{equation}
\label{eqCondition2}
  \sum\limits_{i=1}^n \sigma_{il}(t,x) 
   \fr{\partial M(t,x)}{\partial x_i} = 0\,, \enskip l = 1,2,\dots,s\,.
\end{equation}

Эти условия эквивалентны равенству нулю дифференциала 
Стратоновича для случайного процесса $M(t,X(t))$~\cite{Dub_12}. Условия для 
функции~$f(t,x)$ можно получить из равенства нулю дифференциала Ито 
для случайного процесса~$M(t,X(t))$ или подставить~\eqref{eqStr2Ito} 
в~\eqref{eqCondition1}. Такие условия приведены 
в~\cite{Dub_89, Dub_12, Kar_14, Kar_15}, в~этих же работах подробно изложена 
тео\-рия первых интегралов стохастических дифференциальных уравнений 
(функция $M(t,x)$~--- первый интеграл
уравнений~\eqref{eqItoX} и~\eqref{eqStr}), 
приложение этой тео\-рии к~за-\linebreak\vspace*{-12pt}

\columnbreak

\noindent
 дачам программного управления стохастическими 
динамическими сис\-те\-ма\-ми, там же приведены многочисленные примеры.

Отметим, что с~геометрической точки зрения~\eqref{eqCondition1}~--- 
условие ортогональности блочного вектора $[  1 \  a^\trans(t,x)]^\trans$ 
и~обобщенного градиента $\nabla_{t,x} M(t,x)$, а~\eqref{eqCondition2}~--- 
условие ортогональности каж\-до\-го столбца матрицы~$\sigma(t,x)$ 
и~градиента $\nabla M(t,x)$ в~$\mathds{R}^n$ $\forall t \hm\in \mathrm{T}$. 
В~случае $M(t,x) \hm= M(x)$ условие~\eqref{eqCondition1}~--- 
это условие ортогональности вектора~$a(t,x)$ и~градиента $\nabla M(t,x) 
\hm= \nabla M(x)$ в~$\mathds{R}^n$ $\forall t \hm\in \mathrm{T}$.

\vspace*{-4pt}

\section{Условия инвариантности в~среднем}

Опишем стохастическую систему, задаваемую стохастическим 
дифференциальным уравнением Ито~\eqref{eqItoX} и~соответствующим стохастическим 
дифференциальным уравнением Стратоновича~\eqref{eqStr}, с~дополнительным условием:
$$
  (t,\mathbb{E} X(t)), (t,\mathbb{E}[X(t)|Y_0^t]) \in \mathcal{M},
$$
т.\,е.\ заданному многообразию~$\mathcal{M}$ принадлежат не только 
траектории случайного процесса~$X(t)$, но и~априорное, а также 
апостериорное математическое ожидание случайного процесса~$X(t)$. Для такого 
класса стохастических систем решение задачи оптимальной фильтрации принадлежит 
тому же многообразию, что и~оцениваемые траектории, т.\,е.\ $(t,\hat X(t)) \hm\in 
\mathcal{M}$.

Для этого определим линейную по вектору $x \hm\in \mathds{R}^n$ функцию
\begin{equation}
\label{eqDefM}
  M(t,x) = \left( \vartheta(t),x \right) = 
  \vartheta_1(t) x_1 + \cdots + \vartheta_n(t) x_n\,,
\end{equation}
где $\vartheta(t)$~--- дифференцируемая век\-тор-функ\-ция, координаты которой 
одновременно не обращаются в~нуль: $|\vartheta(t)| \hm> 0$, $(\vartheta(t),x)$~--- 
скалярное произведение в~$\mathds{R}^n$. Функция $\vartheta(t) \hm= \nabla M(t,x)$ 
задает вектор нормали к~гиперплоскости $M(t,x) \hm= C$ в~$\mathds{R}^n$ $\forall t 
\hm\in \mathrm{T}$. В~$\mathrm{T} \times \mathds{R}^n$ многообразие $M(t,x) \hm= C$ 
ги\-пер\-плос\-костью в~общем случае не является. Тогда

\vspace*{-6pt}

\noindent
\begin{multline*}
 \! \!\!\mathbb{E} M(t,X(t)) = \mathbb{E} \left[ 
  \vartheta_1(t) X_1(t) + \cdots + \vartheta_n(t) X_n(t) \right] ={} \\
\hspace*{7pt}{}  = \vartheta_1(t) \, \mathbb{E} X_1(t) + \cdots + \vartheta_n(t) 
 \mathbb{E} X_n(t) = M(t,\mathbb{E} X(t))\,.\hspace*{-7pt}
\end{multline*}
Аналогично

\vspace*{-6pt}

\noindent
\begin{multline*}
  \mathbb{E} \left[ M(t,X(t)) | Y_0^t \right] ={}\\
  {}= 
  \mathbb{E} \left[ \vartheta_1(t) X_1(t) + \cdots + \vartheta_n(t) X_n(t) \,|\, 
  Y_0^t \right] ={} \\
{}  = \vartheta_1(t) \, \mathbb{E} \left[ X_1(t) | Y_0^t \right] + \cdots + 
\vartheta_n(t) \, \mathbb{E} \left[ X_n(t) | Y_0^t \right] = {}\\
{}=
M\left(t,\mathbb{E}\left[X(t)|Y_0^t\right]\right)\,;
\end{multline*}
следовательно, если $M(t,X(t)) = C$, 
то $M(t,\mathbb{E} X(t)) \hm= C$ и~$M(t,\mathbb{E}[X(t)|Y_0^t]) \hm= C$.

\pagebreak

Несмотря на линейность по вектору $x \hm\in \mathds{R}^n$ функции~$M(t,x)$, 
стохастическая система, траектории которой принадлежат многообразию~$\mathcal{M}$, 
может быть как линейной, так и~нелинейной. Чтобы конструктивно описать такую систему, 
определим $n\hm-1$ линейно независимых векторов  $N_1, \ldots, N_{n-1}$, 
ортогональных градиенту~$\nabla M(t,x)$. Эти векторы являются функциями переменной 
$t \hm\in \mathrm{T}$ со значениями в~$\mathds{R}^n$, зависимость от~$t$ 
для краткости опущена.

Векторы $N_1$, \dots, $N_{n-1}$ образуют базис линейного подпространства $M(t,x) \hm= 0$ 
в~$\mathds{R}^n$ $\forall t \hm\in \mathrm{T}$. Их можно выбрать, например, 
следующим образом:
$$
  N_i = \left[ E_i^\trans \  -\fr{\vartheta_i(t)}{\vartheta_n(t)} \right]^\trans,\enskip
  i = 1,\dots,n-1\,,
$$
где $E_i$~--- единичные векторы в~$\mathds{R}^{n-1}$ (столбцы единичной матрицы~$E$ 
порядка~$n\hm-1$). Чтобы такое определение было корректным, дополнительно потребуем,
 чтобы $\vartheta_n(t) \hm\neq 0$ на~$\mathrm{T}$, тем самым обеспечив и~выполнение 
 условия $|\vartheta(t)| \hm> 0$.

Кроме того, определим $n$ линейно независимых векторов $\tilde N_0, \tilde N_1, \dots, 
\tilde N_{n-1}$, ортогональных обобщенному градиенту~$\nabla_{t,x} M(t,x)$. 
Вектор~$\tilde N_0$ является функцией точки $(t,x) \hm\in \mathrm{T} \times 
\mathds{R}^n$, а остальные векторы~--- функции переменной $t \hm\in \mathrm{T}$ 
со значениями в~$\mathds{R}^{n+1}$:
\begin{multline*}
  \!\!\!\!\!\!\tilde N_0 = \begin{bmatrix} 1 &  0 &  \cdots  & 0 &  
  -\fr{\vartheta_0(t,x)}{\vartheta_n(t)} 
\end{bmatrix}^\trans\!\!; \\
  \tilde N_i = \begin{bmatrix}
   0 &  E_i^\trans &  -\fr{\vartheta_i(t)}{\vartheta_n(t)}  
\end{bmatrix}^\trans\!\!, \enskip  i = 1,\ldots,n-1\,,
\end{multline*}
где
\begin{multline*}
  \vartheta_0(t,x) = \fr{\partial M(t,x)}{\partial t} = \left( 
  \fr{\partial \vartheta(t)}{\partial t},x \right)
  = {}\\
  {}=\fr{\partial \vartheta_1(t)}{\partial t} \, x_1 + \cdots + 
  \fr{\partial \vartheta_n(t)}{\partial t} \, x_n\,.
\end{multline*}

Далее обозначим через $\mathcal{N}$ линейную оболочку векторов $N_1, \ldots, N_{n-1}$: 
$\mathcal{N} \hm= \mathrm{Lin} \{ N_1,\dots,N_{n-1} \}$,
а~через $\mathcal{N}_0$~--- линейное многообразие $N_0\hm + \mathcal{N}$:
 $\mathcal{N}_0 \hm= \{ \mathcal{V} \colon \mathcal{V}\hm = 
N_0\hm + N,\ N \hm\in \mathcal{N} \}$,
где вектор $N_0$ является функцией точки $(t,x)\hm \in \mathrm{T} \times 
\mathds{R}^n$ со значениями в~$\mathds{R}^n$:
$$
  N_0 = \begin{bmatrix} 
   0 & \cdots &~ 0~ & -\fr{\vartheta_0(t,x)}{\vartheta_n(t)} \end{bmatrix}^\trans.
$$

По построению произвольная линейная комбинация векторов $N_1, \ldots, N_{n-1}$ 
ортогональна градиенту~$\nabla M(t,x)$. Следовательно, равенство~\eqref{eqCondition2} 
можно переписать в~виде:
\begin{equation}\label{eqCondition2Geometry}
  \sigma_{* l}(t,x) \in \mathcal{N}\,,\enskip l = 1,2,\dots,s\,,
\end{equation}
или $\sigma_{* l}(t,x) \hm= q_1^l(t,x) N_1 + \cdots + q_{n-1}^l(t,x) N_{n-1}$, 
где скалярные функции $q_1^l(t,x), \ldots, q_{n-1}^l(t,x)$ 
могут быть выбраны произвольно при дополнительных условиях 
существования решения уравнения~\eqref{eqStr}, они представляют собой коэффициенты 
разложения столбца~$\sigma_{* l}(t,x)$ по линейно независимой сис\-те\-ме 
векторов $N_1, \dots, N_{n-1}$~--- базису линейного подпространства~$\mathcal{N}$.

Равенство~\eqref{eqCondition1} с~учетом введенных обозначений 
можно переписать следующим образом:
\begin{equation}
\label{eqCondition1Geometry}
  a(t,x) \in \mathcal{N}_0\,,
\end{equation}
или $a(t,x) \hm= N_0 \hm+ q_1^a(t,x) N_1 + \cdots + q_{n-1}^a(t,x) N_{n-1}$, где 
скалярные функции $q_1^a(t,x), \ldots, q_{n-1}^a(t,x)$, 
как и~ранее введенные функции~$q_r^l(t,x)$, могут быть выбраны произвольно 
при дополнительных условиях существования решения уравнения~\eqref{eqStr}. 
Если $M(t,x) \hm= M(x)$, то вектор~$N_0$ является нулевым 
и,~следовательно,~$\mathcal{N}$ и~$\mathcal{N}_0$ совпадают.

Аналог условия~\eqref{eqCondition1Geometry} для блочного вектора 
$[ 1 \  a^\trans(t,x) ]^\trans$ записывается в~виде:
$$
  [  1 \  a^\trans(t,x)]^\trans \in \tilde N_0 + \tilde{\mathcal{N}}\,,\enskip
  \tilde{\mathcal{N}} = \mathrm{Lin} \left\{ \tilde{N}_1,\dots,\tilde{N}_{n-1} \right\}.
$$

При выполнении условия~\eqref{eqCondition2Geometry} разность 
между век\-тор-функ\-ци\-ями $a(t,x)$ и~$f(t,x)$ согласно~\eqref{eqStr2Ito} 
представляет собой вектор, состоящий из суммы компонент вида:
\begin{multline*}
  \fr{1}{2} \, {\fr{\partial (q_k^l(t,x) N_k)}{\partial x} \, q_r^l(t,x) N_r} = {}\\
  {}=
  \fr{1}{2} \, q_r^l(t,x)  N_k  \left[ \nabla q_k^l(t,x) \right]^\trans  N_r\,,
\\
  k,r = 1,\dots,n-1\,, \enskip l = 1,2,\dots,s\,,
\end{multline*}
где произведение $[ \nabla q_k^l(t,x)]^\trans \, N_r$~--- 
это скалярная функция. Таким образом, если $a(t,x) \hm\in \mathcal{N}_0$ 
и~$\sigma_{* l}(t,x) \hm\in \mathcal{N}$, то $f(t,x) \hm\in \mathcal{N}_0$, или 
$f(t,x) \hm= N_0 \hm+ q_1^f(t,x) N_1 + \cdots + q_{n-1}^f(t,x) N_{n-1}$, 
где скалярные функции $q_1^f(t,x), \ldots, q_{n-1}^f(t,x)$ могут 
быть выбраны произвольно при дополнительных условиях существования 
решения уравнения~\eqref{eqItoX}. Для функции~$M(t,x)$, отличной от~\eqref{eqDefM}, 
условие $f(t,x) \hm\in \mathcal{N}_0$, вообще говоря, не выполняется; 
соответствующий пример приведен во введении.

\smallskip

\noindent
\textbf{Теорема~1.}\ 
\textit{Для того чтобы траектории стохастической дифференциальной системы, 
заданной уравнением Ито}~\eqref{eqItoX}, \textit{принадлежали многообразию~$\mathcal{M}$, 
которое определяется функцией}~\eqref{eqDefM}: 
$(t,X(t)) \hm\in \mathcal{M}$, \textit{и~при этом $(t,\mathbb{E} X(t)), 
(t,\mathbb{E}[X(t)|Y_0^t]) \hm\in \mathcal{M}$, необходимо и~достаточно, 
чтобы коэффициенты этого уравнения удовле\-тво\-ря\-ли условиям}:

\pagebreak

\noindent
$$
  f(t,x) \in \mathcal{N}_0\,; \enskip \sigma_{* l}(t,x) \in \mathcal{N}\,,\enskip
   l = 1,2,\dots,s\,,
$$
\textit{на траекториях случайного процесса}~$X(t)$.

\smallskip

\noindent
Д\,о\,к\,а\,з\,а\,т\,е\,л\,ь\,с\,т\,в\,о\ \ теоремы следует из предыдущих рассуждений. 
Если стохастическая дифференциальная система задается уравнением 
Стратоновича, то используется условие $a(t,x) \hm\in \mathcal{N}_0$.

Условия теоремы, а~также то, что векторы вида
\begin{multline*}
  \fr{\partial (q_k^l(t,x) N_k)}{\partial x} \, q_r^j(t,x) N_r = {}\\
  {}=
  q_r^l(t,x)  N_k  \left[ \nabla q_r^j(t,x) \right]^\trans  N_r\,,
\\
  k,r = 1,\ldots,n-1\,, \enskip l,j = 1,2,\ldots,s\,,
\end{multline*}
коллинеарны вектору $N_k$ $\forall t \in \mathrm{T}$, 
обеспечивают отсутствие погрешности, связанной с~отклонением 
численного решения от многообразия~$\mathcal{M}$, при чис\-лен\-ном интегрировании 
стохастического дифференциального уравнения~\eqref{eqItoX} или~\eqref{eqStr} 
и~дополнительном требовании $\vartheta(t) \hm= {const}$, т.\,е.\
 $M(t,x) \hm= M(x)$. Это связано с~видом соответствующих разностных схем, 
 для которых $X_{k+1} \hm= X_k \hm+ \Delta X_k$. Здесь $\Delta X_k$~--- 
 случайный вектор, зависящий от шага численного интегрирования~$h$, пары $(t_k,X_k)$ 
 и~принадлежащий $\mathcal{N} \hm= \mathcal{N}_0$. Например, для метода 
 Эй\-ле\-ра\,--\,Ма\-ру\-ямы
 {\looseness=1
 
 }
 
 \noindent
$$
  \Delta X_k = h  f(t_k,X_k) + \sqrt{h}  \sigma(t_k,X_k)  \Delta W_k\,,
$$
где $\Delta W_k$~--- $s$-мер\-ный случайный вектор, координаты 
которого независимы и~имеют стандартное нормальное распределение. 
Дискретные моменты времени~$t_k$ определяются при разбиении отрезка~$\mathrm{T}$ 
с~шагом~$h$. 

Погрешности такого типа анализировались в~работе~\cite{AveKarRyb_RJNAMM18} 
на примере методов Эй\-ле\-ра--Ма\-ру\-ямы, Мильштейна и~Платена, методов типа 
Рун\-ге--Кут\-ты и~типа Розенброка.



\section{Условия инвариантности при~пуассоновских возмущениях}

Рассмотрим систему наблюдения в~более общей постановке, 
которая принята в~\cite{Sin_SSI16, Sin_IP16, SinSinKor_SSI16, SinSinKor_IP17, SinSinSerKor_AiT18}, 
а~именно:
\begin{multline}
\label{eqItoXJump}
  dX(t) = f \left( t,X(t),Y(t),\rho \right) dt + {}\\
  {}+
  \sigma \left( t,X(t),Y(t),\rho \right) dW(t) + {} \\
  + \int\limits_\Theta \gamma \left( t,X(t-),Y(t-),\rho,\theta \right) 
  \nu\,(dt \times d\theta)\,,\\ X\left(t_0\right) = X_0\,;
\end{multline}

\vspace*{-12pt}

\noindent
\begin{multline}
\label{eqItoYJump}
  dY(t) = c \left( t,X(t),Y(t),\rho \right) dt + {}\\
  {}+
  \zeta \left( t,X(t),Y(t),\rho \right) dV(t) + {} \\
  + \int\limits_\Theta \delta \left( t,X(t-),Y(t-),\rho,\theta \right) \mu\,
  (dt \times d\theta)\,,\\ Y\left(t_0\right) = Y_0\,,
\end{multline}
где $X \in \mathds{R}^n$~--- ненаблюдаемый вектор состояния; 
$Y \hm\in \mathds{R}^m$~--- 
вектор измерений; $\rho \hm\in \mathrm{P} \hm\subset \mathds{R}^l$~--- 
вектор параметров; $t \hm\in \mathrm{T}$, $\mathrm{T}\hm = [t_0,T]$~--- 
заданный отрезок времени; $W(t)$ и~$V(t)$~--- $s$- и~$d$-мер\-ные 
независимые винеровские процессы; $f(t,x,y,\rho)$, $\sigma(t,x,y,\rho)$, 
$\gamma(t,x,y,\rho,\theta)$, $c(t,x,y,\rho)$, $\zeta(t,x,y,\rho)$ 
и~$\delta(t,x,y,\rho,\theta)$~--- заданные век\-тор-функ\-ции и~матричные 
функции соответствующих размеров; $\nu$ и~$\mu$~--- пуассоновские\linebreak
 меры 
на $\mathrm{T} \hm\times \Theta$, $\Theta \hm\subseteq \mathds{R}^k$, 
с~заданными характери\-сти\-че\-ски\-ми мерами, которые определяют интен\-сив\-ность 
со\-от\-вет\-ст\-ву\-ющих пуассоновских потоков и~законы распределения вектора~$\theta$ 
для уравнений~\eqref{eqItoXJump} и~\eqref{eqItoYJump}.

Наличие пуассоновской компоненты в~уравнении, описывающем ненаблюдаемый 
вектор состояния, не влияет на условия~\eqref{eqCondition2Geometry} 
и~\eqref{eqCondition1Geometry}, но требует дополнительного условия для 
функции $\gamma(t,x,y,\rho,\theta)$, которое должно выполняться на 
траекториях случайных процессов~$X(t)$ и~$Y(t)$:
$$
  M(t,x) = M \left( t,x + \gamma(t,x,y,\rho,\theta) \right).
$$

В силу линейности по вектору $x \hm\in \mathds{R}^n$ функции $M(t,x)$ 
имеем $M(t,\gamma(t,x,y,\rho,\theta))\hm = 0$, что эквивалентно ортогональности 
век\-тор-функ\-ции $\gamma(t,x,y,\rho,\theta)$ и~градиента $\nabla M(t,x)$ 
в~$\mathds{R}^n$ $\forall t \hm\in \mathrm{T}$, т.\,е.\ вектор 
$\gamma(t,x,y,\rho,\theta)$ можно разложить по линейно независимой 
системе векторов~$N_1$, \dots, $N_{n-1}$~--- базису линейного 
подпространства~$\mathcal{N}$. Переменные~$y$, $\rho$ и~$\theta$ 
входят в~это условие как параметры, они могут быть любыми с~учетом их 
области определения:
\begin{multline*}
  \gamma(t,x,y,\rho,\theta) ={}\\
  {}= q_1^\gamma(t,x,y,\rho,\theta) N_1 + 
  \cdots + q_{n-1}^\gamma(t,x,y,\rho,\theta) N_{n-1}\,.
\end{multline*}
Переменные $y$ и~$\rho$ войдут как параметры и~в условия~\eqref{eqCondition2Geometry} 
и~\eqref{eqCondition1Geometry}:
\begin{multline*}
  f(t,x,y,\rho) ={}\\
  {}= N_0 + q_1^f(t,x,y,\rho) N_1 + \cdots + q_{n-1}^f(t,x,y,\rho) N_{n-1}\,;
\end{multline*}

\vspace*{-12pt}

\noindent
\begin{multline*}
  \sigma_{* l}(t,x,y,\rho) = {}\\
  {}=q_1^l(t,x,y,\rho) N_1 + 
  \cdots + q_{n-1}^l(t,x,y,\rho) N_{n-1}\,,\\
   l = 1,2,\dots,s\,.
\end{multline*}
Если уравнения для случайных процессов $X(t)$ и~$Y(t)$ записать в~форме 
Стратоновича, то для функции $a(t,x,y,\rho)$~--- соответствующего коэффициента 
в~уравнении для случайного процесса~$X(t)$~--- будем иметь:
\begin{multline*}
  a(t,x,y,\rho) ={}\\
  {}= N_0 + q_1^a(t,x,y,\rho) N_1 + \cdots + 
  q_{n-1}^a(t,x,y,\rho) N_{n-1}\,.
\end{multline*}
Здесь все коэффициенты при векторах $N_1, \ldots$\linebreak $\ldots, N_{n-1}$ могут быть выбраны 
произвольно при дополнительных условиях существования решения 
стохастических дифференциальных уравнений.

\textbf{Теорема 2.} \textit{Для того чтобы траектории стохастической дифференциальной 
сис\-те\-мы, заданной уравнением Ито}~\eqref{eqItoXJump}, 
\textit{принадлежали многообразию~$\mathcal{M}$, которое определяется
 функцией}~\eqref{eqDefM}:\linebreak 
$(t,X(t)) \hm\!\in\! \mathcal{M}$, \textit{и~при этом $(t,\mathbb{E} X(t)), 
(t,\mathbb{E}[X(t)|Y_0^t]) \hm\in \mathcal{M}$, необходимо и~достаточно, 
чтобы коэффициенты этого уравнения удовле\-тво\-ря\-ли условиям}:
\begin{gather*}
  f(t,x,y,\rho) \in \mathcal{N}_0\,; \\
   \sigma_{* l}(t,x,y,\rho) \in \mathcal{N}\,,\enskip 
   l = 1,2,\dots,s\,;\\
   \gamma(t,x,y,\rho,\theta) \in \mathcal{N}
\end{gather*}
\textit{на траекториях случайных процессов $X(t)$ и~$Y(t)$ 
и~$\forall \rho \hm\in \mathrm{P}$, $\theta \in \Theta$}.

\smallskip

Отметим, что поскольку ограничений на случайный процесс~$Y(t)$ в~этой работе 
не накладывает\-ся, то зависимость коэффициентов уравнения\,\eqref{eqItoXJump} 
от вектора измерений, зависимость мат\-рич\-ной функции при дифференциале~$dV(t)$ от 
ненаблюдаемого вектора состояния и~вектора измерений, а~также наличие пуассоновской 
компоненты в~уравнении измерителя не усложняют условий инвариантности, 
которые сформулированы выше. Однако все перечисленные факторы усложняют 
алгоритмы нахождения оптимальной оценки~$\hat X(t)$, в~частности 
формула~\eqref{eqParticleEstimation} здесь не применима.

\vspace*{-4pt}

\section{Модельный пример}

В качестве примера рассмотрим двумерную стохастическую систему вида~\eqref{eqItoX}, 
траектории которой принадлежат гладкому многообразию 
$\mathcal{M} \hm\subset \mathrm{T} \times \mathbb{R}^2$. 
Это многообразие задается уравнением $M(t,x) \hm= -2 x_1 \hm+ \mathrm{e}^{-t} x_2 
\hm= C \hm= {const}$, т.\,е.\ функция $M(t,x)$ имеет вид~\eqref{eqDefM} при 
$n \hm= 2$, $x \hm= [  x_1 \ x_2 ]^\trans$ и~$\vartheta(t) \hm= 
[ -2 \  \mathrm{e}^{-t}]^\trans$. Таким образом,
\begin{gather*}
  \vartheta_0(t,x) = \fr{\partial M(t,x)}{\partial t} = -{e}^{-t} x_2\,;\\ 
   \vartheta_1(t) = -2\,;\quad \vartheta_2(t) = {e}^{-t}.
\end{gather*}

\begin{table*}[b]\small %[ht]
\begin{center}


\begin{tabular}{|c|c|c|}
\multicolumn{3}{c}{Отклонения от заданного многообразия}\\
\multicolumn{3}{c}{\ }\\[-6pt]
  \hline
  $h$ & $\mathbb{E} |M(1,X(1)) - M(0,X(0))|$ $\vphantom{\Bigl|}$ & 
  $\mathbb{E} |M(1,\hat X(1)) - M(0,X(0))|$ \\
  \hline
  &&\\[-9pt]
  $10^{-2}$ & 0,006488 & 0,006821 \\
  $10^{-3}$ & 0,000668 & 0,000685 \\
  $10^{-4}$ & 0,000064 & 0,000066 \\
  \hline
\end{tabular}
\end{center}
\end{table*}

Согласно методике, изложенной выше, определим два вектора:

\noindent
\begin{align*}
  N_0 &= \left[  0 \  -\fr{\vartheta_0(t,x)}{\vartheta_2(t)}  \right]^\trans 
  = [  0 \  x_2  ]^\trans;\\
  N_1 &= \left[  1 \  -\fr{\vartheta_1(t)}{\vartheta_2(t)}  \right]^\trans =
   [  1 \ 2 {e}^t ]^\trans.
\end{align*}

Опишем стохастическую дифференциальную систему, 
траектории которой принадлежат заданному гладкому 
многообразию~$\mathcal{M}$. Положим размерность винеровского процесса $s \hm= 1$. 
Тогда согласно теореме~1 имеем $f(t,x) \hm\in \mathcal{N}_0$, $\sigma(t,x)\hm \in 
\mathcal{N}$, где $\mathcal{N}_0$~--- линейное многообразие, а $\mathcal{N}$~--- 
линейное подпространство, построенные на векторах~$N_0$ и~$N_1$, а~именно:
$$
  f(t,x) = N_0 + q^f(t,x) N_1\,;\enskip \sigma(t,x) = q(t,x) N_1\,,
$$
где скалярные функции~$q^f(t,x)$ и~$q(t,x)$ могут быть выбраны произвольно 
при дополнительных условиях существования решения стохастического дифференциального 
уравнения с~коэффициентами~$f(t,x)$ и~$\sigma(t,x)$. Например, зададим эти функции 
так, чтобы уравнение~\eqref{eqItoX} было линейным: 
$$
q^f(t,x) = {e}^{-t} 
\left(x_1 - x_2\right)\,;\quad 
q(t,x) = {e}^{-t}\,.
$$
 Тогда
\begin{gather*}
  f(t,x) = \left[  {e}^{-t} \left(x_1 - x_2\right) \  \ \ \ 2 x_1 - x_2 
   \right]^\trans;\\
   \sigma(t,x) = [  {e}^{-t} \  2  ]^\trans\,.
\end{gather*}

Кроме функций $f(t,x)$ и~$\sigma(t,x)$, определяющих уравнение
 ненаблюдаемого вектора состояния сис\-те\-мы, зададим функции~$c(t,x)$ и~$\zeta(t)$, 
 которые входят в~уравнение измерителя~\eqref{eqItoY}:
$$
  c(t,x) = x_1 + x_2; \enskip \zeta(t) = 0{,}05\,.
$$

Следовательно, уравнения~\eqref{eqItoX} и~\eqref{eqItoY} 
для рас\-смат\-ри\-ва\-емо\-го примера имеют вид:
\begin{align*}
  dX(t) &= \left[ \begin{array}{cc}
    \mathrm{e}^{-t} & -\mathrm{e}^{-t} \\
    2 & -1 \\
  \end{array} \right] X(t) \, dt +
  \left[ \begin{array}{c}
    \mathrm{e}^{-t} \\
    2 \\
  \end{array} \right] dW(t), \\
 & \hspace*{30mm}X(t) = [  X_1(t) \  X_2(t) ]^\trans;
\\
  dY(t) &= \left( X_1(t) + X_2(t) \right) dt + 0{,}05\, dV(t)\,.
\end{align*}

 

С помощью выбора функций $q^f(t,x)$ и~$q(t,x)$ можно сформировать 
нелинейные уравнения стохастической системы, однако ограничимся линейным 
случаем и~воспользуемся фильтром Кал\-ма\-на--Бью\-си для нахождения 
оптимальной оценки вектора состояния по критерию минимума среднеквадратической ошибки.

Зададим отрезок времени $\mathrm{T} \hm= [0,1]$ и~нулевые начальные данные: 
$X_1(0) \hm= X_2(0) \hm= Y(0) \hm= 0$.
Результаты моделирования траектории 
случайного\linebreak\vspace*{-12pt}


{ \begin{center}  %fig1
 \vspace*{-3pt}
  \mbox{%
 \epsfxsize=79mm 
 \epsfbox{ryb-1.eps}
 }


\end{center}


\noindent
{\small{Выборочная траектория случайного процесса $X(t)$ и~ее оценка}}
}

\vspace*{9pt}

\addtocounter{figure}{1}

\noindent
  процесса $X(t)$ методом Эй\-ле\-ра--Ма\-ру\-ямы 
с~шагом численного интегрирования $h \hm= 10^{-3}$ показаны на рисунке, 
на нем же показан результат оценивания с~по\-мощью фильтра Кал\-ма\-на--Бью\-си 
(оце\-ни\-ва\-емая траектория показана черным, а~ее оценка~--- серым, ось времени 
направлена вправо, ось~$x_1$~--- влево, ось~$x_2$~--- вверх), 
а~также изображена поверхность $M(t,x) \hm= -2 x_1 \hm+ {e}^{-t} x_2 \hm= 0$ 
(нуль в~правой час\-ти является следствием нулевых начальных данных для оцениваемой 
траектории).


Кроме построения одной траектории и~на\-хож\-де\-ния ее оценки была 
проведена серия вы\-чис\-ли\-тель\-ных экспериментов: 
моделировались ан\-самб\-ли из~1000~траекторий 
случайных процессов $X(t)$ и~$Y(t)$ методом Эй\-ле\-ра--Ма\-ру\-ямы с~шагами %\linebreak 
чис\-лен\-но\-го интегрирования~$h$, равными~$10^{-2}$, $10^{-3}$ и~$10^{-4}$. 
Для каждой траектории $X(t)$ была\linebreak найде\-на оптимальная оценка $\hat X(t)$ 
с~по\-мощью фильт\-ра Кал\-ма\-на--Бью\-си по соответствующей траектории~$Y(t)$. 
На основе этих результатов моделирования были вычислены оценки среднего значения 
для величин $|M(1,X(1)) \hm- M(0,X(0))|$ и~$|M(1,\hat X(1)) \hm- M(0,X(0))|$,
 которые показывают отклонения траектории и~ее оценки, полученных численно, 
 от заданного многообразия в~момент времени $T \hm= 1$. Результаты представлены в~виде 
 таб\-лицы.



Из полученных данных видно, что при уменьшении шага численного 
интегрирования~$h$ в~10~раз средние отклонения уменьшаются почти в~10~раз. 
Это соответствует первому порядку слабой сходимости метода Эй\-ле\-ра--Ма\-ру\-ямы, 
т.\,е.\ отклонение траекторий и~их оценок от заданного многообразия вызваны 
погрешностью при численном решении стохастических дифференциальных уравнений.



\section{Заключение}

В статье описан класс стохастических дифференциальных систем, 
для которых одному и~тому же многообразию принадлежат не только 
траектории системы, но и~результат решения задачи оптимальной 
фильтрации по критерию минимума среднеквадратической ошибки оценивания, 
а~именно: приведены необходимые и~достаточные условия принадлежности траекторий 
стохастической системы и~их оценок заданному гладкому многообразию. Рассмотрены 
системы как диффузионного типа, так и~диф\-фу\-зи\-он\-но-скач\-ко\-об\-раз\-но\-го 
типа, т.\,е.\ 
при наличии как винеровских, так и~пуассоновских возмущений. Приведен модельный 
пример линейной стохастической дифференциальной системы, для которой 
траектории и~оценки этих траекторий\linebreak с~по\-мощью фильтра Кал\-ма\-на--Бью\-си 
принадлежат\linebreak динамическому многообразию. В~дальнейшем планируется рассмотреть 
многообразия меньшей размерности (здесь рассмотрены многообразия, размерность 
которых на единицу меньше размера оценива\-емо\-го вектора состояния), а~также 
обобщить полученные результаты для стохастических дифференциальных сис\-тем с~изменениями 
структуры, это даст возможность перейти к~задачам фильт\-ра\-ции на 
ку\-соч\-но-глад\-ких многообразиях.


{\small\frenchspacing
 {%\baselineskip=10.8pt
 \addcontentsline{toc}{section}{References}
 \begin{thebibliography}{99}



\bibitem{Sin_IP16}
\Au{Синицын И.\,Н.} Ортогональные субоптимальные 
фильтры для нелинейных стохастических систем на многообразиях~// Информатика и~её 
применения, 2016. Т.~10. Вып.~1. С.~34--44.

\bibitem{Sin_SSI16}
\Au{Синицын И.\,Н.} Нормальные и~ортогональные субоптимальные 
фильтры для нелинейных стохастических систем на многообразиях~// 
Системы и~средства информатики, 2016. Т.~26. №\,1. С.~199--226.

\bibitem{SinSinKor_SSI16}
\Au{Синицын И.\,Н., Синицын~В.\,И., Корепанов~Э.\,Р.} 
Эллипсоидальные субоптимальные фильтры для нелинейных стохастических 
систем на многообразиях~// Системы и~средства информатики, 2016. Т.~26. №\,2. С.~79--97.

\bibitem{SinSinKor_IP17}
\Au{Синицын И.\,Н., Синицын~В.\,И., Корепанов~Э.\,Р.} Модифицированные 
эллипсоидальные услов\-но-оп\-ти\-маль\-ные фильтры для нелинейных стохастических 
систем на многообразиях~// Информатика и~её применения, 2017. Т.~11. Вып.~2. С.~101--111.

\bibitem{SinSinSerKor_AiT18}
\Au{Синицын И.\,Н., Синицын~В.\,И., Сергеев~И.\,В., Корепанов~Э.\,Р.} 
Методы эллипсоидальной фильтрации процессов в~нелинейных стохастических системах 
на многообразиях // Автоматика и~телемеханика, 2018. №\,1. С.~147--161.

\bibitem{Dub_89}
\Au{Дубко В.\,А.} Вопросы теории и~применения стохастических дифференциальных уравнений. --- Владивосток: ДВО АН СССР, 1989. 185~с.

\bibitem{Dub_12}
\Au{Дубко В.\,А.} Стохастические дифференциальные уравнения. Избранные разделы.~--- 
Киев: Логос, 2012. 68~с.

\bibitem{Kar_14}
\Au{Карачанская Е.\,В.} Случайные процессы с~инвариантами.~--- Хабаровск: ТОГУ, 2014. 
148~с.

\bibitem{Kar_15}
\Au{Карачанская Е.\,В.} Интегральные инварианты стохастических систем 
и~программное управление с~вероятностью~1.~--- Хабаровск: ТОГУ, 2015. 148~с.

\bibitem{Ave_ACMMENSP17}
\Au{Аверина Т.\,А.} Аналитические и~численные решения 
трех систем стохастических дифференциальных уравнений с~инвариантами~// 
Аналитические и~численные методы моделирования ес\-те\-ст\-вен\-но-на\-уч\-ных 
и~социальных проблем: Тр. XII Междунар. научн.-технич. конф.~--- 
Пенза: ПГУ, 2017. С.~3--8.

\bibitem{OksSul_05}
\Au{{\ptb{\O}}\,\,ksendal B., Sulem~A.} Applied stochastic control of jump diffusions.~--- 
Berlin: Springer, 2005. 214~p.

\bibitem{Sin_07}
\Au{Синицын И.\,Н.} Фильтры Калмана и~Пугачева.~--- М.: Логос, 2007. 
776~с.

\bibitem{BaiCri_09}
\Au{Bain A., Crisan~D.} Fundamentals of stochastic filtering.~--- 
New York, NY, USA: Springer, 2009. 394~p.

\bibitem{Ryb_17}
\Au{Рыбаков К.\,А.} Статистические методы анализа и~фильтрации в~непрерывных 
стохастических системах.~--- М.: МАИ, 2017. 176~с.

\bibitem{AveKarRyb_RJNAMM18}
\Au{Averina T.\,A., Karachanskaya~E.\,V., Rybakov~K.\,A.} 
Statistical analysis of diffusion systems with invariants~// Russ. 
J.~Numer. Anal.~M., 2018. Vol.~33. Iss.~1. P.~1--13.

 \end{thebibliography}

 }
 }

\end{multicols}

\vspace*{-3pt}

\hfill{\small\textit{Поступила в~редакцию 19.04.18}}

\vspace*{8pt}

%\pagebreak

%\newpage

%\vspace*{-28pt}

\hrule

\vspace*{2pt}

\hrule

%\vspace*{-2pt}

\def\tit{ON A CLASS OF FILTERING PROBLEMS ON~MANIFOLDS}


\def\titkol{On a class of filtering problems on manifolds}

\def\aut{K.\,A.~Rybakov}

\def\autkol{K.\,A.~Rybakov}

\titel{\tit}{\aut}{\autkol}{\titkol}

\vspace*{-11pt}


\noindent
Moscow Aviation Institute (National Research University), 
4~Volokolamskoye Shosse, Moscow 125993, Russian Federation

\def\leftfootline{\small{\textbf{\thepage}
\hfill INFORMATIKA I EE PRIMENENIYA~--- INFORMATICS AND
APPLICATIONS\ \ \ 2019\ \ \ volume~13\ \ \ issue\ 1}
}%
 \def\rightfootline{\small{INFORMATIKA I EE PRIMENENIYA~---
INFORMATICS AND APPLICATIONS\ \ \ 2019\ \ \ volume~13\ \ \ issue\ 1
\hfill \textbf{\thepage}}}

\vspace*{4pt}




\Abste{The goal of the paper is to describe stochastic differential 
systems whose trajectories belong to a~smooth manifold as an application 
to the optimal filtering problem. An additional condition is that not only 
system trajectories belong to the given manifold, but also the estimation 
results for these trajectories (solution of the optimal filtering problem 
with the minimum mean-squared error) belong to this manifold. Diffusion and 
jump-diffusion systems are considered. These systems can be driven by the 
Wiener process and the Poisson process. The main result is the conditions on 
coefficients of the equation for the estimated random process. These conditions 
are obtained on the basis of the first integral concept for the stochastic 
differential equation and some of its properties.}

\KWE{invariant; estimation; manifold; optimal filtering; random process; 
stochastic differential system}




\DOI{10.14357/19922264190103}

\vspace*{-15pt}

\Ack
\noindent
The work was supported by the Russian Foundation for Basic Research 
(project 17-08-00530-а).



%\vspace*{6pt}

  \begin{multicols}{2}

\renewcommand{\bibname}{\protect\rmfamily References}
%\renewcommand{\bibname}{\large\protect\rm References}

{\small\frenchspacing
 {%\baselineskip=10.8pt
 \addcontentsline{toc}{section}{References}
 \begin{thebibliography}{99}

\vspace*{-3pt}

\bibitem{2-ryb-1}
\Aue{Sinitsyn, I.\,N.} 2016. Ortogonal'nye suboptimal'nye fil'try dlya 
nelineynykh stokhasticheskikh sistem na mno\-go\-ob\-ra\-zi\-yakh 
[Orthogonal suboptimal filters for nonlinear stochastic systems on manifolds]. 
\textit{Informatika i~ee Primeneniya~--- Inform. Appl.} 10(1):34--44.

\bibitem{1-ryb-1}
\Aue{Sinitsyn, I.\,N.} 2016. Normal'nye i~ortogonal'nye suboptimal'nye 
fil'try dlya nelineynykh stokhasticheskikh sistem na mnogoobraziyakh 
[Normal and orthogonal conditionally optimal filters for nonlinear 
stochastic systems on manifolds]. \textit{Sistemy i~Sredstva Informatiki~--- 
Systems and Means of Informatics} 26(1):199--226.

\bibitem{3-ryb-1}
\Aue{Sinitsyn, I.\,N., V.\,I.~Sinitsyn, and E.\,R.~Korepanov.}
2016. Ellipsoidal'nye suboptimal'nye fil'try dlya nelineynykh stokhasticheskikh 
sistem na mnogoobraziyakh [Ellipsoidal conditionally optimal filters for 
nonlinear stochastic systems on manifolds]. \textit{Sistemy i~Sredstva Informatiki~--- 
Systems and Means of Informatics} 26(2):79--97.

\bibitem{4-ryb-1}
\Aue{Sinitsyn, I.\,N., V.\,I.~Sinitsyn, and E.\,R.~Korepanov.} 2017. 
Modifitsirovannye ellipsoidal'nye uslovno-optinal'nye fil'try dlya
nelineynykh stokhasticheskikh sistem na mnogoobraziyakh
[Modificated ellipsoidal conditionally optimal filters for nonlinear 
stochastic systems on manifolds]. 
\textit{Informatika i~ee Primeneniya~--- Inform. Appl.} 11(2):101--111.

\bibitem{5-ryb-1}
\Aue{Sinitsyn, I.\,N., V.\,I.~Sinitsyn, I.\,V.~Sergeev, and E.\,R.~Korepanov.}
2018. Methods of ellipsoidal filtration in nonlinear stochastic systems on manifolds. 
\textit{Autom. Rem. Contr.} 79(1):117--127.

\bibitem{6-ryb-1}
\Aue{Dubko, V.\,A.} 1989. \textit{Voprosy teorii i~primeneniya sto\-kha\-sti\-che\-skikh 
differentsial'nykh uravneniy} [Problems of theory and application 
of stochastic differential equations]. Vladivostok: Akad. Nauk SSSR. 185~p.

\bibitem{7-ryb-1}
\Aue{Dubko, V.\,A.} 2012. \textit{Stokhasticheskie differentsial'nye uravneniya. 
Izbrannye razdely} [Stochastic differential equations. Selected topics]. 
Kiev: Logos. 68~p.

\bibitem{8-ryb-1}
\Aue{Karachanskaya, E.\,V.} 2014. \textit{Sluchaynye protsessy s~invariantami} 
[Random process with invariants]. Khabarovsk: Pacific National University. 148~p.

\bibitem{9-ryb-1}
\Aue{Karachanskaya, E.\,V.} 2015. \textit{Integralnye invarianty 
sto\-kha\-sti\-che\-skikh sistem 
i~programmnoe upravlenie s~ve\-ro\-yat\-nost'yu~$1$} 
[Integral invariants of stochastic systems and program control with probability~1]. 
Khabarovsk: Pacific National University. 148~p.

\bibitem{10-ryb-1}
\Aue{Averina, T.\,A.} 2017. Analiticheskie i~chislennye resheniya trekh 
sistem stokhasticheskikh differentsial'nykh uravneniy s~invariantami 
[Analytic and numerical solutions of three systems of stochastic differential 
equations with invariants]. 
\textit{12th  Scientific and Technical Conference  (International)
``Analytical and Numerical Methods for Modeling Natural-Scientific and Social Problems''
Proceedings}. Penza. 3--8.

\bibitem{11-ryb-1}
\Aue{\mbox{{\ptb{\O}}ksendal}, B., and A.~Sulem.} 2005. 
\textit{Applied stochastic control of jump diffusions}. Berlin: Springer. 214~p.

\bibitem{12-ryb-1}
\Aue{Sinitsyn, I.\,N.} 2007. \textit{Fil'try Kalmana i~Pugacheva} [Kalman and Pugachev filters]. 
Moscow: Logos. 776~p.

\bibitem{13-ryb-1}
\Aue{Bain, A., and D.~Crisan.} 2009. \textit{Fundamentals of stochastic filtering}. 
New York, NY: Springer. 394~p.

\bibitem{14-ryb-1}
\Aue{Rybakov, K.\,A.} 2017. \textit{Statisticheskie metody analiza i~fil'tratsii v~nepreryvnykh 
stokhasticheskikh sistemakh} [Statistical methods of analysis and filtering for 
continuous stochastic systems]. Moscow: MAI. 176 p.

\bibitem{15-ryb-1}
\Aue{Averina, T.\,A., E.\,V.~Karachanskaya, and K.\,A.~Rybakov.} 
2018. Statistical analysis of diffusion systems with invariants. 
\textit{Russ. J.~Numer. Anal.~M.} 33(1):1--13.
\end{thebibliography}

 }
 }

\end{multicols}

\vspace*{-6pt}

\hfill{\small\textit{Received April 19, 2018}}

%\pagebreak

%\vspace*{-18pt}



\Contrl

\noindent
\textbf{Rybakov Konstantin A.} (b.\ 1979)~---
Candidate of Sciences (PhD) in physics and mathematics, 
associate professor, Moscow Aviation Institute (National Research University), 
4~Volokolamskoye Shosse, Moscow 125993, Russian Federation; \mbox{rkoffice@mail.ru}


\label{end\stat}

\renewcommand{\bibname}{\protect\rm Литература}       