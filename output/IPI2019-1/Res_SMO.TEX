\documentclass[a4paper,12pt]{article}
\usepackage[T2A]{fontenc}                       % Поддержка русских букв
\usepackage[utf8]{inputenc}
\usepackage[utf8]{inputenc}
\usepackage[english, russian]{babel}            % Языки: русский, английский
%\usepackage[english, russian]{babel}
%\documentclass[12pt,a4paper,oneside]{article}
\usepackage[utf8]{inputenc}
\usepackage[russian]{babel}
\usepackage{amsmath}
\usepackage{amssymb}
\usepackage{euscript}
\usepackage{graphics}
\usepackage{textcomp}
\usepackage{graphicx}
\usepackage{cite}

\usepackage[left=2.5cm,right=2cm,
    top=2cm,bottom=2cm,bindingoffset=0cm]{geometry}

\begin{document}

\title{РЕСУРСНЫЕ СИСТЕМЫ МАССОВОГО ОБСЛУЖИВАНИЯ С ПРОИЗВОЛЬНЫМ ОБСЛУЖИВАНИЕМ\thanks{Публикация подготовлена при финансовой поддержке Минобрнауки России (проект 2.882.2017/4.6).}}

\date{}

\maketitle

\vspace{-2cm}

\begin{center}
А. В. Горбунова\footnote[1]{Российский университет дружбы народов, gorbunova\_av@rudn.university},
В. А. Наумов\footnote[2]{Исследовательский институт инноваций, Хельсинки, Финляндия, valeriy.naumov@pfu.fi},
Ю. В. Гайдамака\footnote[3]{Российский университет дружбы народов; Институт проблем информатики Федерального исследовательского центра «Информатика и управление» Российской академии наук, gaydamaka\_yuv@rudn.university},
К. Е. Самуйлов\footnote[4]{Российский университет дружбы народов; Институт проблем информатики Федерального исследовательского центра «Информатика и управление» Российской академии наук, samouylov\_ke@rudn.university}
\end{center}

\sloppy

\noindent
\textbf{Аннотация:}
В статье представлен обзор ресурсных систем массового обслуживания. Одному из методов их исследования посвящен целый раздел. Ценной особенностью предложенного метода является значительное упрощение анализа системы и при этом сохранение высокой точности оценки, а в отдельных случаях и отсутствие потери точности в принципе. Так, в частности, вместо исходной ресурсной модели было предложено рассматривать упрощенную, в которой после ухода заявки освобождается не тот объем ресурсов, который заявка занимала, а его некоторая случайная величина, что позволяет значительно упростить случайный процесс, описывающий поведение системы. Впоследствии для случая пуассоновского входящего потока и экспоненциального времени обслуживания была строго доказана эквивалентность результатов для исходной и упрощенной моделей. Отдельный практический интерес представляют работы с рекуррентной дисциплиной обслуживания. Результатам их анализа посвящена значительная часть работы.

\vspace{\baselineskip}     % пустая строка

\noindent
\textbf{Ключевые слова:} ресурсная система массового обслуживания; непрерывный ресурс; дискретный ресурс; ограниченный ресурс; рекуррентное обслуживание; гетерогенная сеть; стационарное распределение; полумарковский процесс


\vspace{\baselineskip}     % пустая строка

\noindent                  % убрать красную строку
\textbf{1 Введение}

Описание модели ресурсной системы массового обслуживания достаточно подробно представлено в \cite{Ch_1}. Напомним лишь, что ее отличительной чертой является то, что заявкам помимо приборов и мест ожидания требуются дополнительные ресурсы. В случае нехватки какого-либо из них заявка теряется.
Случайный объем дополнительного ресурса, занимаемого заявкой в подавляющем большинстве исследований на все время ее пребывания в системе, может носить как дискретный, так и непрерывный характер. Случаи детерминированных требований заявок к ресурсам подробно описаны в \cite{Kelly,Ross,Basharin} и здесь не рассматриваются.

Если в ресурсной модели имеется единственный тип конечного ресурса, то, как правило, таким образом моделируется ограниченность памяти некоторого технического устройства или целой информационно-вычислительной системы \cite{Romm_21_1971,Kac,Tihonenko_27_1985,Pechinkin_29_2012}. Системы с множественными ресурсами моделируют беспроводные сети, в которых разного рода радиоресурсы должны распределяться между активными абонентами и освобождаться после завершения  их обслуживания \cite{Naumov_3_2016}.
Системы с множественными ресурсами, которые выделяются не одновременно, а последовательно, могут также использоваться для моделирования беспроводных гетерогенных сетей, в которых обслуживание пользователей происходит последовательно вследствие необходимости выделения ресурсов на нескольких станциях, имеющих различные каналы передачи данных. В этом случае речь уже идет о модели многофазной ресурсной системы массового обслуживания \cite{Mois_4_2017,Mois_5_2017}.


Сеть ресурсных систем массового обслуживания (РСМО) с заявками нескольких классов и сигналами, инициирующими переходы заявок между узлами, в которой объем ресурсов, запрашиваемый клиентами, может быть как положительным, так и отрицательным, применима для моделирования автономных точек доступа, которые имеют возможность динамически увеличивать свою пропускную способность, или для моделирования гетерогенной сети, развернутой в какой-либо области \cite{Sopin_12_2017,Sopin_13_2017,Naumov_14_2017,Naumov_15_2017}. Положительные заявки в таком случае будут представлять запросы на выделение некоторого случайного объема работ, тогда как отрицательные заявки будут соответствовать увеличению объема ресурсов, доступных положительным заявкам.
Таким образом, становится возможным учитывать динамику пользователей,
%(или транспортных средств),
когда новые <<объекты>> могут предоставлять свои сетевые возможности соседним потребителям, а также возвращать их, когда они покидают территорию.

Статья организована следующим образом: в разделе 2 описываются основные типы ресурсных систем массового обслуживания с рекуррентным обслуживанием, методы их исследования и полученные результаты. Третий раздел посвящен новому подходу к исследованию ресурсных СМО ~--- методу упрощения и результатам его применения к анализу ресурсных моделей. В заключении кратко подведены итоги работы.

\vspace{\baselineskip}     % пустая строка

%\section{Ресурсные СМО с произвольным обслуживанием}
\noindent
\textbf{2 РСМО с произвольным обслуживанием}

Стоит отметить, что одним из первых в числе ученых, обративших внимание на актуальность исследования РСМО и много сделавших для ее изучения, является О.\,М.\,Тихоненко. Его авторству начиная с 1985 года \cite{Tihonenko_27_1985} принадлежит множество статей, посвященных этой тематике. В большинстве из них рассматриваются модели ресурсных систем с дисциплиной разделения процессора (Egalitarian Processor Sharing, EPS). Так, работа \cite{Tihonenko_22_2001} посвящена анализу СМО $M|G|1|m \quad (m\leq\infty)$ с разделением процессора, одним типом ресурса и существующей зависимостью между временем обслуживания заявки и объемом требуемого ей ресурса. Для этой системы определяются стационарные вероятности состояний и вероятность потери заявок.

В статье \cite{Tihonenko_24_1990} исследуются однолинейные СМО вида $M|G|1|m \quad (m\leq\infty)$ с единственным типом ограниченного объема ресурса, в которых поступающие заявки также характеризуются зависимостью между временем обслуживания заявки и случайной величиной (с. в.) требуемого ей объема ресурса. Для первой исследуемой СМО объем ресурса в системе не ограничен, в отличие от числа мест для ожидания; в результате ее анализа получено выражение для преобразования Лапласа--Стилтьеса суммарного объема ресурсов, занятого заявками. Для второй СМО, в которой объем ресурса ограничен, а накопитель ~--- бесконечной емкости, получена оценка вероятности потери заявки.

В статьях \cite{Sengupta,Tihonenko_40_2002} анализируется СМО $M|G|1|\infty$ с разделением процессора и одним типом ресурса неограниченного объема $R=\infty$, при этом заявке помимо ресурса некоторого случайного объема требуется некоторое число приборов для ее обслуживания, и эти величины зависимы, т. е. задана их совместная функция распределения. В результате было получено выражение для преобразования Лапласа--Стилтьеса суммарного объема ресурсов, занятых находящимися в системе заявками, в стационарном режиме.
В \cite{Tihonenko_40_2002} результаты были обобщены на нестационарный режим и, кроме того, было предложено использовать полученные в \cite{Sengupta} результаты для оценки вероятности потери заявок в системе с ограниченным объемом ресурса, однако такая оценка далеко не всегда приемлема для использования \cite{Tihonenko_40_2002}.
Поэтому в работе \cite{Tihonenko_25_2010} анализируется более общий вид СМО с ограничением на объем ресурса $R<\infty$: $M_{L}|G|1|m$ с разделением процессора и одним типом ресурса, т. е. в систему поступает $L$ пуассоновских потоков заявок, причем заявка класса $l, l=\overline{1,L}$ характеризуется не только необходимым ей объемом ресурса, но и константой $n_l\leq N$ ~--- числом приборов, необходимых для ее обслуживания.
Объем требуемого заявке ресурса и ее длина, под которой понимается время пребывания заявки в системе при условии отсутствия в ней других заявок (т. е. фактически время обслуживания заявки), являются зависимыми величинами. Их совместное распределение также зависит от класса заявки. Было получено стационарное распределение числа заявок в системе и вероятности потери заявок каждого класса и, кроме того,  аналогичные характеристики для некоторых частных случаев.

В \cite{Tihonenko_26_2005} фактически исследуется СМО $M|G|N|0$ с единственным типом ограниченного ресурса $R\leq \infty$, причем каждая заявка характеризуется тремя случайными признаками: числом приборов, необходимых для обслуживания, объемом ресурса и временем обслуживания, т. е. помимо некоторого случайного объема ресурсов заявке требуется еще и случайное число приборов для обслуживания. Всего имеется $N$ классов таких требований, для каждого класса заявок задана своя совместная функция распределения времени ее обслуживания и объема требуемого ей ресурса. Получено стационарное распределение числа заявок в системе и вероятности потери заявки каждого из имеющихся классов. Также в \cite{Tihonenko_26_2005} проанализированы некоторые частные случаи, например: когда объем требуемого заявке ресурса и время ее обслуживания являются независимыми с. в., когда время обслуживания заявки пропорционально объему ее ресурса, а также, как и в работе \cite{Romm_21_1971}, рассматривается частный случай, когда для обслуживания любой заявки требуется один прибор и функция распределения объема требуемых ресурсов одна и та же для всех классов заявок.

В серии работ другого авторства \cite{Pechinkin_28_2011,Pechinkin_29_2012,Pechinkin_30_1998,Pechinkin_31_1999} исследуются ресурсные системы, особенностью которых является инверсионный порядок обслуживания.
По мнению авторов, в случае с дисциплиной LIFO можно получить рекуррентные алгоритмы, пригодные для численных методов вычисления стационарных вероятностей состояний и стационарного распределения времени пребывания заявки в системе. Однако стоит заметить, что предложенные рекуррентные алгоритмы в случае с непрерывной функцией распределения объема ресурса \cite{Pechinkin_30_1998,Pechinkin_31_1999} имеют большую вычислительную сложность в силу необходимости решения довольно сложных интегральных уравнений, а условие дискретности с. в. требуемого заявке объема ресурса \cite{Pechinkin_28_2011,Pechinkin_29_2012} позволяет получить более простые и эффективные алгоритмы расчета основных стационарных характеристик СМО.

В работе \cite{Pechinkin_30_1998} рассматривается СМО типа $M_k|G|1|m,$ $0\leq m <\infty$, с инверсионным порядком обслуживания без прерывания и единственным типом ресурса объема $R$. Иными словами, на систему поступает пуассоновский входящий поток второго рода: его интенсивность $\lambda_k$ зависит от числа заявок $k$, находящихся в системе. Как и в предыдущих работах, упомянутых в этом разделе, здесь тоже существует зависимость между временем обслуживания и требуемым заявке объемом ресурса, задаваемая совместной функцией распределения. Под инверсионным порядком обслуживания без прерывания подразумевается, что каждая принятая в систему заявка становится на первое место в очереди, причем если поступающая в систему заявка застает в очереди уже $m$ заявок, но требуемый ей для обслуживания объем ресурса суммарно с необходимыми объемами ресурсов для имеющихся в системе заявок меньше $R$, то она вытесняет заявку, стоящую на приборе, и занимает ее место. Получены выражения для стационарных вероятностей различных состояний, в том числе распределение числа заявок в системе и суммарного объема занятого этими заявками ресурса, а также вероятность принятия заявки в систему, вероятность ее обслуживания; стационарное распределение времени пребывания заявки в системе. Также рассмотрены некоторые частные случаи: независимость времени обслуживания и объема необходимого ресурса; отсутствие ограничения на суммарный объем ($R=\infty$); экспоненциальное время обслуживания.

В статье \cite{Pechinkin_31_1999} исследуется СМО с марковским входящим потоком вида $MAP|G|1|m$, так же как и в \cite{Pechinkin_30_1998}, с инверсионным порядком обслуживания, но уже с прерыванием, а также с единственным типом ограниченного ресурса объема $R$ и зависимыми временем обслуживания и требуемым объемом ресурса для заявок. Получены формулы для стационарного распределения числа заявок в системе, для времени пребывания заявки в системе, а также некоторые характеристики для частных случаев: независимые время обслуживания и объем ресурса; $m=\infty$; $R=\infty$.

В работе \cite{Pechinkin_28_2011} исследуется однолинейная СМО с ограничением на суммарный объем занятых ресурсов одного типа ($R$) и геометрическим второго рода входящим потоком заявок $Geo_k|G|1|m$, функционирующая в дискретном времени с инверсионным порядком обслуживания без прерывания обслуживания и с существующей зависимостью для объема ресурса, т. е. фактически система, аналогичная СМО $M_k|G|1|m$ с непрерывным временем. Получены соотношения, которые позволяют вычислять основные стационарные характеристики (стационарные вероятности состояний и стационарное распределение времени пребывания заявки в системе).

В \cite{Pechinkin_29_2012} рассматривается система $Geo_k|G|1|\infty$, аналогичная СМО из \cite{Pechinkin_28_2011}, но объем требуемого заявке ресурса имеет дискретное распределение, потери происходят только в том случае, когда суммарный объем находящихся в системе заявок превышает величину $R$. Получены соотношения, позволяющие вычислять основные стационарные характеристики СМО.

В статьях \cite{Mois_1_2017,Mois_2_2017,Mois_3_2016,Mois_4_2017,Mois_5_2017} исследуются многофазные РСМО с непуассоновским входящим потоком, неэкспоненциальным обслуживанием, бесконечным числом приборов и неограниченным объемом выделяемого ресурса на каждой из фаз. Так, в \cite{Mois_2_2017} проведен анализ РСМО типа $MMPP|G|\infty$, в \cite{Mois_3_2016} ~--- $MAP|G|\infty$ и в \cite{Mois_1_2017} ~--- $G|G|\infty$, во всех случаях речь идет об однофазных СМО.
В статьях \cite{Mois_4_2017,Mois_5_2017} исследуются двухфазные СМО.
Перечисленные статьи объединяет общий метод исследования систем: метод асимптотического анализа в условиях растущей интенсивности входящего потока с некоторыми особенностями в зависимости от типа входящего потока. В результате анализа получены асимптотические выражения для стационарного распределения вероятностей числа заявок в системе и суммарного объема занятых ресурсов (на каждой из фаз). Проведенные исследования позволили сделать вывод о том, что во всех случаях асимптотическое распределение суммарного объема занятых ресурсов является гауссовским с соответствующими параметрами и размерностью, согласующейся со случайным процессом, описывающим поведение изучаемой системы.

В статье \cite{Naumov_18_2016} для РСМО с пуассоновским входящим потоком интенсивности $\lambda$, $M$ типами ограниченных ресурсов и $N$ приборами получены выражения для стационарных вероятностей процесса $X(t)=(\xi(t);\boldsymbol{\delta}(t))$ ($\xi(t)$ ~--- число заявок в системе, $\boldsymbol{\delta}(t)$ ~--- суммарный объем занятых ресурсов каждого типа)
%, не упрощенной СМО
в случае произвольного закона распределения времени обслуживания $B(x)$:
\begin{equation}
p_0=\lim_{t\rightarrow \infty}P\{\xi(t)=0\}=\bigg(1+\sum_{k=1}^{N}
G^{(k)}(\mathbf{R})\frac{\rho^{k}}{k!} \bigg ) ^{-1},
\end{equation}
\begin{equation}
Q_k(\mathbf{x})=\lim_{t\rightarrow \infty}P\{ \xi(t)=k; \boldsymbol{\delta}(t)\leq \mathbf{x}\}=p_0G^{(k)}(\mathbf{x}) \frac{\rho^{k}}{k!}, \mathbf{0}\leq \mathbf{x} \leq \mathbf{R}, 1\leq k \leq N.
\end{equation}
Здесь $G^{(k)}(\mathbf{x})$ ~--- $k$-кратная свертка функции распределения:
\begin{equation}
G(\mathbf{x})=\frac{1}{b}\int_{0}^{\infty}tH(dt,\mathbf{x}),
\end{equation}
где $H(t,\mathbf{x})$ ~--- совместная функция распределения длительности обслуживания и вектора объемов ресурсов, необходимых поступившей заявке, $\rho=\lambda b$, $b=\int_{0}^{\infty}tdB(t)$.
Заметим, что в формуле (11) статьи \cite{Naumov_18_2016} символ $D$ следует читать как $F$.

\vspace{\baselineskip}     % пустая строка

%\section{Метод упрощения}
\noindent
\textbf{3 Метод упрощения}

Многолинейные СМО с множественными ресурсами, рекуррентным входящим потоком и произвольными функциями распределения длительностей обслуживания и объемов ресурсов впервые исследовались в \cite{Naumov_1_2014}. В этой статье был предложен новый подход к анализу РСМО,
позволяющий значительно упростить вычисления, что принципиально важно и заслуживает отдельного внимания. Итак, поскольку анализ РСМО довольно сложен
%, по причинам упомянутым выше,
было предложено вместо исходной СМО исследовать ее упрощенный аналог, который функционирует аналогично исходной системе за исключением того, что объемы ресурсов, освобождаемые по завершении обслуживания, могут отличаться от тех, которые были выделены заявке в начале ее обслуживания.
%При заданных суммарных объемах занятых ресурсов и заданном числе заявок в системе в момент завершения обслуживания объемы освобождаемых ресурсов не зависят от поведения системы до этого момента и имеют функцию распределения специального вида, вычисляемую по формуле Байеса.
Случайный процесс, описывающий поведение упрощенной системы, легче поддается анализу, поскольку отпадает необходимость запоминания объемов ресурсов, занимаемых каждой обслуживаемой заявкой, и при анализе надо помнить лишь суммарные объемы занятых ресурсов.
В этой же работе были продемонстрированы результаты моделирования исходных и упрощенных систем при пуассоновском входящем потоке и экспоненциальном обслуживании, которое показало,  что средние значения объемов занимаемых ресурсов в стационарном режиме для них оказались очень близкими. Позднее в \cite{Naumov_2_2014}, также с помощью имитационного моделирования, было показано, что близки не только средние значения, но и стационарные распределения суммарных объемов занимаемых ресурсов.

И, наконец, в \cite{Naumov_3_2016} было строго обосновано, что переход к анализу упрощенной системы не меняет стационарного распределения суммарных объемов занятых ресурсов для случая пуассоновского входящего потока и экспоненциального времени обслуживания. Данный результат был доказан для случая непрерывной функции распределения объема требуемых заявке ресурсов.
Вообще говоря, анализируемый здесь процесс $Y(t)=(\xi(t);\boldsymbol{\delta}(t))$ не является марковским, поскольку в момент ухода заявки из системы должны быть освобождены ресурсы в объемах, равных объемам ресурсов, выделенных этой заявке при ее поступлении в систему. Однако при небольшом изменении исходной системы можно получить более простую систему с множественными ресурсами, для которой составной случайный процесс, компонентами которого являются два процесса (число заявок в системе и суммарный объем занимаемых ими ресурсов), образуют полумарковский процесс.
Итак, рассмотрим многолинейную СМО с пуассоновским входящим потоком, экспоненциальным временем обслуживания и $M$ типами ресурсов, функционирование которой подчиняется классическому набору правил \cite{Ch_1},
за исключением того, что условие:
\begin{enumerate}
\item[] в момент окончания обслуживания заявки суммарный объем занятого ресурса каждого типа уменьшается на величину ресурса, выделенного этой заявке,
\end{enumerate}
заменяется следующим:
\begin{enumerate}
\item[*] в момент окончания обслуживания заявки вектор суммарных объемов занятых ресурсов уменьшается на случайную величину, которая при заданном числе заявок в системе и векторе суммарных объемов занятых ресурсов не зависит от поведения системы до упомянутого момента времени.
\end{enumerate}

Тогда процесс $Y^*(t)=(\xi^*(t);\boldsymbol{\delta}^*(t)),$ где $\boldsymbol{\delta}^*(t)$ ~--- это те же,
что и для исходной системы, векторы суммарных объемов ресурсов, необходимых для
обслуживания заявок, а $\xi^*(t)$ ~--- число заявок в системе, является полумарковским и его предельное распределение определяется как
\begin{equation}
q_0^*=\lim_{t\rightarrow \infty}P\{\xi^*(t)=0\},
\end{equation}
\begin{equation}
Q_k^*(\mathbf{x})=\lim_{t\rightarrow \infty}P\{\xi^*(t)=k; \boldsymbol{\delta}^*(t) \leq \mathbf{x}\}, \quad 1\leq k\leq N.
\end{equation}
В результате решения соответствующей СУР получаем, что предельные распределения суммарных объемов занятых ресурсов в исходной и упрощенной СМО совпадают: $q_0^*=p_0$, $Q_k^*(\mathbf{x})=Q_k(\mathbf{x})$ \cite{Naumov_3_2016}.

В работе \cite{Sopin_16_2018} рассматривается РСМО с единственным типом пуассоновского входящего потока и $M$ типами ограниченных ресурсов.
Здесь доказана инвариантность стационарного процесса $X(t)=(\xi(t);\boldsymbol{\delta}(t))$ относительно функции распределения для упрощенной СМО. Т. е. в случае произвольного закона распределения времени обслуживания $B(x)$ стационарные вероятности упрощенной СМО $p^{*}_0$ и $Q^{*}_k(\mathbf{x})$ будут иметь тот же вид, что и в случае экспоненциального распределения времени обслуживания. А стационарные вероятности случайного процесса $(\xi(t);\boldsymbol{\delta}(t), \boldsymbol{\beta}(t))$, где $\boldsymbol{\beta}(t)=(\beta_1(t),...,\beta_{\xi(t)}(t))$ ~--- время, прошедшее с момента поступления заявок на обслуживание, примут вид
\begin{equation}
\begin{gathered}
Q_k(\mathbf{x},\tau_1,...,\tau_k)=\lim_{t\rightarrow \infty} P \{ \xi(t)=k;\boldsymbol{\delta}(t)\leq \mathbf{x}; \beta_1(t) \leq \tau_1,...,\beta_k(t) \leq \tau_k  \}=\\
=p_0 F^{(k)}(\mathbf{x})\frac{\rho^{k}}{k!}, \quad
\mathbf{0} \leq \mathbf{x} \leq \mathbf{R}, \quad 1\leq k \leq N,
\end{gathered}
\end{equation}
где $\rho=\lambda b$, $b=\int_{0}^{\infty}tdB(t)$.

Для СМО с $N$ приборами, одним типом дискретного ресурса общим объемом $R$, пуассоновским входящим потоком и экспоненциальным временем обслуживания в \cite{Sopin_4_2015,Sopin_5_2015} благодаря методу упрощения получены выражения для вероятности блокировки системы, стационарные вероятности простоя системы, маргинальные вероятности числа заявок в системе $q_{k,\cdot}$, среднее число заявок в системе, а также $q_{k,j}$ ~--– стационарные вероятности того, что число заявок в системе равно $k$, а суммарный объем занятого ресурса равен $j$:
\begin{equation}
q_{k,\cdot}=P\lim_{t\rightarrow \infty}P\{ \xi(t)=k \}=p_0\frac{\rho^k}{k!}\sum_{i=0}^{R}p_i^{(k)},
\quad 0<k\leq N,
\end{equation}
\begin{equation}
q_{k,j}=P\lim_{t\rightarrow \infty}P \{ \xi(t)=k; \delta(t)=j\}=p_0\frac{\rho^k}{k!}p_j^{(k)},
\quad 0\leq j\leq R, 0<k\leq N,
\end{equation}
где
\begin{equation}
p_0=\bigg( 1+\sum_{k=1}^{N}\frac{\rho^k}{k!}\sum_{i=0}^{R}p_i^{(k)} \bigg )^{-1},
\quad 0<k\leq N,
\end{equation}
а $p^{(k)}_i$ ~--- $k$-кратная свертка вероятности $p_i$ с. в. $\sum_{i=1}^{k}r_i$, $r_i$ ~--- с. в. объема ресурса, необходимого поступившей в систему $i$-й заявке.

В \cite{Sopin_17_2018} с помощью метода упрощения была рассмотрена РСМО с марковским входящим потоком типа $MAP|M|N|0$ и единственным типом ограниченного ресурса объема $R$. Система уравнений равновесия выведена в векторной форме и решена численно. Наряду со стационарным распределением вероятностей в статье представлены формулы для математического ожидания и дисперсии числа занятых ресурсов, а также вероятности блокировки. Результаты проиллюстрированы численным примером.

\vspace{\baselineskip}     % пустая строка

%\section{Заключение}
\noindent
\textbf{4 Заключение}\\
В настоящем обзоре кратко представлены основные разновидности ресурсных систем массового обслуживания с рекуррентным обслуживанием, существующие методы их анализа, выражения для оценки основных вероятностно-временных характеристик.

\vspace{\baselineskip}     % пустая строка

\noindent\textbf{Список литературы}
\begin{enumerate}

%1
\bibitem{Ch_1}
\textit{Горбунова А. В., Наумов В. А., Гайдамака Ю. В., Самуйлов К. Е.}
Ресурсные системы массового обслуживания как модели беспроводных систем связи // {Информатика и её применения}, 2018. Т.~12. Вып.~3. С.~48--55.

%2
\bibitem{Kelly}
\textit{Kelly F. P.}
Loss Networks // {Annals of applied probability}, 1991. No.~1. P.~319--378.

%3
\bibitem{Ross}
\textit{Ross K. W.}
Multiservice loss models for broadband telecommunication networks. ~--- {London: Springer-Verlag}, 1995. 343 p.

%4
\bibitem{Basharin}
\textit{Basharin G. P., Samouylov K. E., Yarkina N. V., Gudkova I. A.}
A new stage in mathematical teletraffic theory // {Automation and Remote Control}, 2009. Vol.~70. No.~12. P.~1954--1964.

%5
\bibitem{Romm_21_1971}
\textit{Ромм Э. Л., Скитович В. В.}
Об одном обобщении задачи Эрланга // {Автоматика и телемеханика}, 1971. №~6. С.~164--168.

%6
\bibitem{Kac}
\textit{Кац Б. А.}
Об обслуживании сообщений случайной длины // {Теория массового обслуживания: Труды 3-й Всесоюзной школы-совещания по теории массового обслуживания}. --- М.: МГУ, 1976. С.~157--168.

%7
\bibitem{Tihonenko_27_1985}
\textit{Тихоненко О. М.}
Распределение суммарного объема сообщений в однолинейной системе массового обслуживания с групповым поступлением // {Автоматика и телемеханика}, 1985. №~11. С.~78--83.

%8
\bibitem{Pechinkin_29_2012}
\textit{Печинкин А. В., Соколов И. А., Шоргин С. Я.}
Ограничение на суммарный объем заявок в дискретной системе $Geo|G|1|\infty$ // {Информатика и её применения}, 2012. Т.~6. Вып.~3. С.~107--113.

%9
\bibitem{Naumov_3_2016}
\textit{Наумов В. А., Самуйлов К. Е., Самуйлов А. К.}
О суммарном объеме ресурсов, занимаемых обслуживаемыми заявками // {Автоматика и телемеханика}, 2016. №~8. С.~105--110.

%10
\bibitem{Mois_4_2017}
\textit{Lisovskaya E., Moiseeva S., Pagano M.}
Infinite-server tandem queue with renewal arrivals and random capacity of customers // {Communications in Computer and Information Science}, 2017. Vol.~700. P.~201--216.

%11
\bibitem{Mois_5_2017}
\textit{Moiseev A., Moiseeva S., Lisovskaya E.}
Infinite-server queueing tandem with $MMPP$ arrivals and random capacity of customers // {31st European Conference on Modelling and Simulation Proceedings}. ~--- Budapest, Hungary: ECMS, 2017. P.~673--679.

%12
\bibitem{Sopin_12_2017}
\textit{Samouylov K., Sopin E., Vikhrova O.}
Analysis of queueing system with resources and signals // {Communications in Computer and Information Science}, 2017. Vol.~800. P.~358--369.

%13
\bibitem{Sopin_13_2017}
\textit{Sopin E., Vikhrova O., Samouylov K.}
LTE network model with signals and random resource requirement // {9th International Congress on Ultra Modern Telecommunications and Control Systems and Workshops (ICUMT) Proceedings}. ~--- Munich, Germany: IEEE, 2017. P.~101--106.

%14
\bibitem{Naumov_14_2017}
\textit{Наумов В. А., Самуйлов К. Е.}
Анализ сетей ресурсных систем массового обслуживания // {Автоматика и телемеханика}, 2018. №~5. С.~59--68.

%15
\bibitem{Naumov_15_2017}
\textit{Naumov V., Samouylov K.}
Analysis оf multi-resource loss system with state dependent arrival and service rates // {Probability in the Engineering and Informational Sciences}, 2017. Vol.~31. No.~4. P.~413--419.

%16
\bibitem{Tihonenko_22_2001}
\textit{Тихоненко О. М., Климович К. Г.}
Анализ систем обслуживания требований случайной длины при ограниченном суммарном объеме // {Проблемы передачи информации}, 2001. Т.~37. Вып.~1. С.~78--88.

%17
\bibitem{Tihonenko_24_1990}
\textit{Позняк Р. И., Ревинский В. В., Старовойтов А. М., Тихоненко О. М.}
Определение характеристик суммарного объема требований в однолинейных системах обслуживания с ограничениями // {Автоматика и телемеханика}, 1990. №~11. С.~182--186.

%18
\bibitem{Sengupta}
\textit{Sengupta B.}
The spatial requirement of $M|G|1$ queue or: how to design for buffer space // {Modeling and Performance Evaluation Methodology}, 1984. Vol.~60. P.~545--562.

%19
\bibitem{Tihonenko_40_2002}
\textit{Тихоненко О. М.}
Анализ системы обслуживания неоднородных требований с дисциплиной разделения процессора // {Известия Национальной академии наук Беларуси. Серия физико-математических наук}, 2002. №~2. С.~105--111.

%20
\bibitem{Tihonenko_25_2010}
\textit{Тихоненко О. М.}
Система обслуживания с разделением процессора и ограниченными ресурсами // {Автоматика и телемеханика}, 2010. №~5. С.~84--98.

%21
\bibitem{Tihonenko_26_2005}
\textit{Тихоненко О. М.}
Обобщенная задача Эрланга для систем обслуживания с ограниченным суммарным объемом // {Проблемы передачи информации}, 2005. Т.~41. Вып.~3. С.~64--75.

%22
\bibitem{Pechinkin_28_2011}
\textit{Касконе А., Манзо Р., Печинкин А. В., Шоргин С. Я.}
Система $Geo_m|G|1|n$ с дисциплиной LIFO без прерывания обслуживания и ограничением на суммарный объем заявок // {Автоматика и телемеханика}, 2011. №~1. С.~107--120.

%23
\bibitem{Pechinkin_30_1998}
\textit{Печинкин А. В.}
Система $M_i|G|1|n$ с дисциплиной $LIFO$ и ограничением на суммарный объем требований // {Автоматика и телемеханика}, 1998. №~4. С.~106--116.

%24
\bibitem{Pechinkin_31_1999}
\textit{Печинкин А. В.}
Система $MAP|G|1|n$ с дисциплиной $LIFO$ с прерыванием и ограничением на суммарный объем требований // {Автоматика и телемеханика}, 1999. №~12. С.~114--120.

%25
\bibitem{Mois_1_2017}
\textit{Лисовская Е. Ю., Моисеева С. П.}
Асимптотический анализ немарковской бесконечнолинейной системы обслуживания требований случайного объема с входящим рекуррентным потоком // {Вестник Томского государственного университета. Управление, вычислительная техника и информатика}, 2017. №~39. С.~30--38.

%26
\bibitem{Mois_2_2017}
\textit{Lisovskaya E., Moiseeva S., Pagano M., Potatueva V.}
Study of the $MMPP|GI|\infty$ queueing system with random customers' capacities // {Informatics and Applications}, 2017. Vol.~11. No.~4. P.~111--119.

%27
\bibitem{Mois_3_2016}
\textit{Кононов И. А., Лисовская Е. Ю.}
Исследование бесконечнолинейной СМО $MAP|GI|\infty$ с заявками случайного объема // {Информационные технологии и математическое моделирование (ИТММ-2016): Материалы XV Международной конференции имени А. Ф. Терпугова} ~--- Томск: ТГУ, 2016. Ч.~1. С.~67--71.

%28
\bibitem{Naumov_18_2016}
\textit{Наумов В. А., Самуйлов А. К.}
О связи ресурсных систем массового обслуживания с сетями Эрланга // {Информатика и её применения}, 2016. Т.~10. Вып.~3. С.~9--14.

%29
\bibitem{Naumov_1_2014}
\textit{Наумов В. А., Самуйлов К. Е.}
О моделировании систем массового обслуживания с множественными ресурсами // {Вестник РУДН. Серия: Математика, информатика, физика}, 2014. №~3. С.~60--64.

%30
\bibitem{Naumov_2_2014}
\textit{Naumov V., Samouylov K., Sopin E., Andreev S.}
Two approaches to analyzing dynamic cellular networks with limited resources // {6th International Congress on Ultra Modern Telecommunications and Control Systems and Workshops (ICUMT) Proceedings}. ~--- St. Petersburg, Russia: IEEE, 2014. P.~485--488.

%31
\bibitem{Sopin_16_2018}
\textit{Samouylov K., Gaidamaka Yu., Sopin E.}
Simplified analysis of queueing systems with random requirements // {Statistics and Simulation}, 2018. Vol.~231. P.~381--390.

%32
\bibitem{Sopin_4_2015}
\textit{Samouylov K., Sopin E., Vikhrova O.}
Analyzing blocking probability in LTE wireless network via queuing system with finite amount of resources // {Communications in Computer and Information Science}, 2015. Vol.~564. P.~393--403.

%33
\bibitem{Sopin_5_2015}
\textit{Вихрова О. Г., Самуйлов К. Е., Сопин Э. С., Шоргин С. Я.}
К анализу показателей качества обслуживания в современных беспроводных сетях // {Информатика и её применения}, 2015. Т.~9. Вып.~4. С.~48--55.

%34
\bibitem{Sopin_17_2018}
\textit{Sopin E., Samouylov K.}
On the analysis of the limited resources queuing system under MAP arrivals // {International Conference on Applied Mathematics, Computational Science and Systems Engineering (AMCSE 2017): Proceedings}. ~--- Athens, Greece: EDP Sciences, 2018. Vol.~16. Art. No.~01008. P.~1--4.

%====================================================

\end{enumerate}

\end{document}