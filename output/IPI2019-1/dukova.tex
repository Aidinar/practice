\def\stat{dukova}

\def\tit{О ЧИСЛЕ МАКСИМАЛЬНЫХ НЕЗАВИСИМЫХ ЭЛЕМЕНТОВ ЧАСТИЧНЫХ ПОРЯДКОВ 
(СЛУЧАЙ ЦЕПЕЙ)$^*$}

\def\titkol{О числе максимальных независимых элементов частичных порядков 
(случай цепей)}

\def\aut{Е.\,В.~Дюкова$^1$, Г.\,О.~Масляков$^2$, П.\,А.~Прокофьев$^3$}

\def\autkol{Е.\,В.~Дюкова, Г.\,О.~Масляков, П.\,А.~Прокофьев}

\titel{\tit}{\aut}{\autkol}{\titkol}

\index{Дюкова Е.\,В.}
\index{Масляков Г.\,О.}
\index{Прокофьев П.\,А.}
\index{Djukova E.\,V.}
\index{Maslyakov G.\,O.}
\index{Prokofyev P.\,A.}


{\renewcommand{\thefootnote}{\fnsymbol{footnote}} \footnotetext[1]
{Работа выполнена при частичной финансовой поддержке РФФИ (проект 
19-01-00430-а).}}


\renewcommand{\thefootnote}{\arabic{footnote}}
\footnotetext[1]{Федеральный исследовательский центр <<Информатика и~управление>> Российской академии наук; Московский 
государственный университет им.\ М.\,В.~Ломоносова, \mbox{edjukova@mail.ru}}
\footnotetext[2]{Московский государственный университет 
им.\ М.\,В.~Ломоносова, \mbox{gleb-mas@mail.ru}}
\footnotetext[3]{Институт машиноведения им.\ А.\,А.~Благонравова Российской академии наук, \mbox{p\_prok@mail.ru}}

%\vspace*{8pt}

 
  

     \Abst{Рассматривается одна из центральных труднорешаемых задач логического 
анализа данных~--- дуализация над произведением частичных порядков. Исследуется 
важный частный случай, когда каждый порядок является цепью. Если мощность каждой 
цепи равна двум, то рассматриваемая задача~--- это построение сокращенной дизъюнктивной 
нормальной формы монотонной булевой функции, заданной конъюнктивной нормальной 
формой (КНФ), что эквивалентно перечислению неприводимых покрытий булевой матрицы. При 
условии, что число строк булевой матрицы по порядку меньше числа столбцов, известна 
асимптотика типичного числа неприводимых покрытий. В~настоящей работе аналогичный 
результат получен для дуализации над произведением цепей, когда мощность каждой цепи 
больше двух. Получение подобных асимптотических оценок является технически сложной 
задачей и~необходимо, в~частности, для обоснования существования асимптотически 
оптимальных алгоритмов для задачи монотонной дуализации и~различных обобщений этой 
задачи.} 
     
     \KW{задача дуализации; произведение частичных порядков; цепь; покрытие булевой 
матрицы; упорядоченное покрытие целочисленной матрицы; асимптотически оптимальный 
алгоритм}

\DOI{10.14357/19922264190104}
  
%\vspace*{4pt}


\vskip 10pt plus 9pt minus 6pt

\thispagestyle{headings}

\begin{multicols}{2}

\label{st\stat}
     
\section{Введение}

    Логический анализ данных основан на решении сложных 
в~вычислительном плане задач, что  обусловлено применением дискретного 
аппарата. Как правило, возникают задачи, которые в~теории алгоритмической 
сложности называют труднорешаемыми. Особой сложностью отличаются 
перечислительные задачи, в~которых требуется найти (перечислить) все 
решения, при этом число решений растет экспоненциально с~рос\-том размера 
задачи (размера входа). Одной из центральных пе-\linebreak речислительных задач 
считается дуализация над произведением частичных порядков. Ниже приведена 
ее формулировка.
    
    Пусть $P\hm= P_1\times \cdots \times P_n$, где $P_1, \ldots , P_n$~--- 
конечные частично упорядоченные множества. Считается, что элемент 
$y\hm=(y_1, \ldots, y_n) \hm \in P$ следует за элементом $x\hm=(x_1, \ldots, 
x_n)\hm\in P$, если~$y_i$ следует за~$x_i$ при $i\hm=1,2, \ldots, n$. Для 
обозначения того, что $y\hm\in P$ следует за $x\hm\in P$ и~$y\not= x$, далее 
используется запись $x\prec y$. Пусть $R\hm\subseteq P$, $R^+\hm= R\cup \{ 
x\hm\in P\vert \exists a\hm\in R,\ a\prec x\}$. Задача построения двойственного 
к~$R$ множества $I(R)$, состоящего из элементов $a\hm\in P\backslash R^+$, 
таких что для любого $x\hm\in P\backslash R^+$, $x\not= a$, отношение $a\prec 
x$ не выполняется, называется дуализацией над произведением частичных 
порядков. Элементы множества~$I(R)$ называются \textit{максимальными 
независимыми от~$R$ элементами~$P$}. 
    
    Важность дуализации обусловлена большим чис\-лом приложений, среди 
которых прежде всего следует выделить логический анализ данных 
в~распознавании (машинное обучение по прецедентам) и~поиск ассоциативных 
правил в~базах данных (data mining).
    
    Одним из наиболее востребованных является случай, когда каждое~$P_i$  
представляет собой цепь, т.\,е.\ любые два элемента в~$P_i$ сравнимы. Если 
$P_i\hm= \{0,1\}$ при $i\hm\in \{1,2,\ldots ,n\}$ и~$0\prec 1$, то 
рассматриваемая задача сводится к~по\-стро\-ению сокращенной дизъюнктивной 
нормальной формы монотонной булевой функ\-ции от~$n$ переменных, 
заданной КНФ из~$\vert R\vert$ 
элементарных дизъюнкций (дуализация монотонной КНФ). Эквивалентной 
задачей является поиск неприводимых покрытий булевой мат\-ри\-цы из~$\vert 
R\vert$ строк и~$n$~столб\-цов (дуализация булевой мат\-рицы).
    
    Теоретические оценки эффективности алгоритмов дуализации базируются 
на оценке сложности одного шага~[1]. Наиболее эффективным считается 
алгоритм, который имеет полиномиальный от размера входа шаг. Однако 
полиномиальные алгоритмы удалось построить лишь для некоторых частных 
случаев дуализации монотонной КНФ, поэтому\linebreak требования к~алгоритму были 
ослаблены. Сформировались два основных направления исследований. 
    
    Первое направление нацелено на построение так называемых 
инкрементальных алгоритмов, когда алгоритму разрешено просматривать 
решения, найденные на предыдущих шагах. При этом оценка сложности шага 
алгоритма дается для худшего случая (для самого сложного варианта задачи). 
В~[2] построен инкрементальный алгоритм дуализации монотоной КНФ 
с~квазиполиномиальным шагом, определяемым фактически не только 
размером входа задачи, но и~размером ее выхода. В~[3, 4] для случая, когда 
каждое~$P_i$ является цепью и~$\vert P_i\vert \hm\geq 2$, на базе алгоритма, 
предложенного в~[2], построен квазиполиномиальный инкрементальный 
алгоритм. Подход интересен в~основном для теории, поскольку в~худшем 
случае число решений дуализации (размер выхода задачи) растет 
экспоненциально с~ростом размера ее входа. 
    
    Второе направление основано на построении асимптотически 
оптимальных алгоритмов дуализации булевой матрицы (впервые предложено 
в~[5]). В~этом случае алгоритму разрешено делать лишние полиномиальные 
шаги при условии, что их чис\-ло почти всегда должно быть достаточно мало по 
сравнению с~числом всех решений задачи (числом неприводимых покрытий 
булевой матрицы). В~результате удалось построить алгоритмы дуализации 
булевой матрицы, эффективные в~типичном случае (эффективные для почти 
всех вариантов задачи). Эти алгоритмы   лидируют по скорости счета~[6].
    
    Теоретическое обоснование асимптотически оптимальных алгоритмов 
дуализации булевой мат\-ри\-цы базируется на получении асимптотик типичных 
значений числа всех неприводимых покрытий и~числа неприводимых покрытий 
типичной длины. Технические основы получения подобных оценок заложены 
в~работах~\cite{5-duk, 7-duk}.
    
    В~[8] рассмотрен случай, когда $P_i\hm= \{0, 1, \ldots , k\hm-1\}$, 
$k\hm\geq 2$, $i\hm=1,2,\ldots ,n$, и~элементы в~$P_i$ упорядочены в~порядке 
возрастания. Показано, что задача перечисления~$I(R)$ эквивалентна 
построению некоторого специального подмножества множества неприводимых 
покрытий булевой матрицы из $\vert R\vert$ строк и~$kn$~столбцов. Для 
поиска элементов множества~$I(R)$ построен алгоритм RUNC-M+, который 
представляет собой  модификацию асимптотически оптимального алгоритма 
поиска неприводимых покрытий булевой матрицы RUNC-M из~\cite{6-duk}. 
    
    В настоящей работе обоснована асимптотическая оптимальность 
алгоритма RUNC-M+ при условии, что $\vert R\vert$ на порядок меньше~$n$ 
при $n\hm\to \infty$. С~целью получения требуемых асимптотических оценок 
введено понятие упорядоченного тупикового покрытия целочисленной 
матрицы, являющееся обобщением понятия неприводимого покрытия булевой 
матрицы. Установлено взаимно однозначное соответствие между множеством 
упорядоченных тупиковых покрытий матрицы, строками которой являются 
наборы из~$R$, и~множеством~$I(R)$. Получена асимптотика типичных 
значений числа упорядоченных тупиковых покрытий и~числа упорядоченных 
тупиковых покрытий типичной длины целочисленной матрицы в~случае 
большого числа столбцов. Из полученных асимптотических оценок, 
в~частности, следует, что величина $\vert I(R)\vert$ (здесь и~далее $\vert 
A\vert$~--- мощность~$A$) почти всегда при $n\hm\to\infty$ асимптотически 
равна числу шагов алгоритма RUNC-M+. 
    
\section{Основные понятия}

    Пусть $P_i=\{0,1,\ldots, k-1\}$, $k\hm\geq 2$, $i\hm=1,2,\ldots, n$, 
и~элементы в~$P_i$ упорядочены в~порядке воз\-рас\-тания. Введем обозначения: 
$L$~--- матрица, в~которой~$n$~столбцов и~элементы принадлежат мно\-жеству\linebreak 
$\{0,1,\ldots, k\hm-1\}$, $k\hm\geq 2$; $E_k^r$, $r\hm\leq n$,~--- множество всех 
наборов вида $(\sigma_1, \ldots, \sigma_r)$, в~которых $\sigma_i\hm\in \{0,1,\ldots 
,k\hm-1\}$, $k\hm\geq 2$, при $i\hm=1,2,\ldots, r$. 
    
    Рассмотрим $\sigma \hm\in E_k^r$, $\sigma\hm=(\sigma_1,\ldots, \sigma_r)$, 
$\sigma_i\hm<k\hm-1$, $i\hm=1,2,\ldots, r$. Через $Q_i(\sigma)$, $i\hm\in 
\{1,2,\ldots. r\}$, обозначим множество наборов $(\beta_1,\ldots, \beta_r)$ 
в~$E_k^r$, таких что $\beta_i\hm=\sigma_i\hm+1$ и~$\beta_j\hm\leq \sigma_j$ 
при $j\hm\in \{1,2,\ldots, r\}\backslash \{i\}$. 
    
    Пусть $H$~--- набор из~$r$~различных столбцов мат\-ри\-цы~$L$. 
Множество различных строк под\-мат\-ри\-цы мат\-ри\-цы~$L$, образованной 
столб\-ца\-ми набора~$H$, мож\-но рас\-смат\-ри\-вать как некоторое 
подмножество~$E^H$ наборов из~$E_k^r$. Набор столбцов~$H$ называется 
\textit{упорядоченным тупиковым $\sigma$-по\-кры\-ти\-ем} мат\-ри\-цы~$L$, 
если выполнены два сле\-ду\-ющих условия:
\begin{enumerate}[(1)]
\item $E^H$ не содержит набор 
$(\beta_1, \ldots, \beta_r)\hm\in E_k^r$, в~котором~$\beta_j\hm\leq \sigma_j$ при 
$j\hm\in \{1,2,\ldots, r\}$; 
 \item если $i\hm\in \{1,2,\ldots, r\}$, то~$E^H$ содержит 
хотя бы один набор из~$Q_i(\sigma)$.
\end{enumerate}
   
    Если выполнено условие~1, то набор столбцов~$H$ называется 
\textit{упорядоченным $\sigma$-по\-кры\-ти\-ем} матрицы~$L$. Если 
выполнено условие~2, то набор столбцов~$H$ называется 
\textit{упорядоченным $\sigma$-со\-вмести\-мым набором столбцов} 
матрицы~$L$. Упорядоченное (тупиковое) $(0, 0,\ldots, 0)$-по\-кры\-тие 
булевой матрицы называется (\textit{неприводимым}) \textit{покрытием}. 
    
    Квадратную подматрицу порядка~$r$ матрицы~$L$ назовем 
\textit{упорядоченной $\sigma$-под\-мат\-ри\-цей}, если для множества ее 
различных строк~$E$, рас\-смат\-ри\-ва\-емо\-го как некоторое подмножество наборов 
из~$E_k^r$, выполнено $E\cap Q_i(\sigma)\not= \varnothing$ при $i\hm\in 
\{1,2,\ldots, r\}$. 
    
    Обозначим через~$L_R$ матрицу, строками которой являются элементы 
множества~$R$. 
    
    Пусть $\sigma=(\sigma_1,\ldots, \sigma_n)$~--- набор из~$E_k^n$, 
в~котором элемент с~номером~$t$, $t\hm\in \{j_1,\ldots, j_r\}$, не является 
максимальным в~$P_t$, а~элемент с~номером~$t$, $t\hm\notin \{j_1, \ldots, 
j_r\}$, является максимальным в~$P_t$. Очевидным является
    
    \smallskip
    
    \noindent
    \textbf{Утверждение~1.} Набор~$\sigma$ является максимальным 
независимым от~$R$ элементом множества~$P$ тогда и~только тогда, когда 
набор столбцов матрицы~$L_R$ с~номерами $j_1,\ldots, j_r$ является 
упорядоченным тупиковым $(\sigma_{j_1}, \ldots , \sigma_{j_r})$-по\-кры\-тием.
    
    Таким образом, задача нахождения максимальных независимых от~$R$ 
элементов множества~$P$ (перечисления множества $I(R)$) сводится к~задаче 
нахождения упорядоченных тупиковых покрытий матрицы~$L_R$. 
    
    Пусть $M^k_{mn}$~--- совокупность всех матриц размера $m\times n$ 
    с~элементами из $\{0,1,\ldots k\hm-1\}$, $k\hm\geq 2$.  Представляют интерес 
типичные значения числа тупиковых упорядоченных покрытий и~длины 
тупикового упорядоченного покрытия для матрицы из~$M^k_{mn}$. 
Выявление типичной ситуации  связано с~высказыванием типа <<для почти 
всех матриц~$L$ из~$M_{mn}^k$ при $n\hm\to \infty$ выполнено 
свойство~P>>, причем свойство~P  может также иметь предельный характер. 
Например, если на матрицах из $M_{mn}^k$ заданы две функции $F(L)$ 
и~$G(L)$ с~положительными зна\-че\-ни\-ями, 
то  мож\-но говорить, что для почти всех мат\-риц~$L$ из~$M^k_{mn}$ 
выполнено~$F(L) \sim G(L)$ ($F(L)$ асимптотически равно $G(L)$), если 
существуют две положительные бесконечно малые при $n\hm\to\infty$ функции 
$\alpha(n)$ и~$\beta(n)$, такие что для всех достаточно больших~$n$ имеет 
место
    $$
    1-\fr{\vert M\vert}{\vert M_{mn}^k\vert} \leq \alpha(n)\,,
    $$
    где $M$~--- множество матриц~$L$ из~$M_{mn}^k$, для которых 
    $$
    1-\beta(n)\leq \fr{F(L)}{G(L)}\leq 1+\beta(n)\,.
    $$

\section{Асимптотика типичного числа упорядоченных тупиковых 
покрытий целочисленной матрицы в~случае большого числа 
столбцов}

    Обозначим: $\phi_d$, $d\hm>0$,~--- интервал 
    $\left( (1/2)\log_d mn\hm-
(1/2)\log_d \log_d mn-\log_d\log_d\log_d n,\right.\hspace*{-3.73233pt}$\linebreak
$\left.(1/2)\log_d mn - (1/2)\log_d 
\log_d mn\hm+ \log_d \log_d \log_d n\right)\hspace*{-1.3679pt}$;
    $E^r_{k-1}$~--- множество наборов $(\sigma_1,\ldots, \sigma_r)$ в~$E_k^r$, 
таких что $\sigma_i\hm< k\hm-1$, $i\hm=1,2,\ldots, r$; $\Pi_r(\sigma)\hm= 
(\sigma_1\hm+1)^{r-1} \cdots  (\sigma_r\hm+1)^{r-1}$, $\sigma\hm\in 
E^r_{k-1}$.
    
    Пусть $L\hm\in M^k_{mn}$, $\sigma\hm\in E^r_{k-1}$. Положим 
$B(L,\sigma)$~--- множество всех упорядоченных тупиковых  
$\sigma$-по\-кры\-тий матрицы~$L$; $S(L,\sigma)$~--- множество всех 
упорядоченных $\sigma$-под\-мат\-риц матрицы~$L$;
    \begin{align*}
    \Sigma_1(L) &= \sum\limits^n_{r=1} \sum\limits_{\sigma\in E^r_{k-1} }\vert 
B(L,\sigma)\vert\,;\\
    \Sigma_2(L)&=\sum\limits^n_{r=1}\sum\limits_{\sigma\in E^r_{k-1}} \vert 
S(L,\sigma)\vert\,.
    \end{align*}
    
    \noindent
    \textbf{Теорема~1.} \textit{Если $m^\alpha \hm\leq n\hm\leq d^m$, 
$\alpha\hm>1$, $d\hm= k/(k\hm-1)$, то для почти всех матриц~$L$ 
из~$M^k_{mn}$ при $n\hm\to \infty$ справедливо}
    $$
    \Sigma_1(L)\sim\Sigma_2(L)\sim \sum\limits_{r\notin \phi_d} 
\sum\limits_{\sigma\in E^r_{k-1}} \Pi_r(\sigma) C_n^r C_m^r r! k^{-r^2}
    $$
\textit{и~длины почти всех упорядоченных тупиковых покрытий матрицы~$L$ 
принадлежат интервалу}~$\phi_d$.

\smallskip

    
    \noindent
    Д\,о\,к\,а\,з\,а\,т\,е\,л\,ь\,с\,т\,в\,о\ \  теоремы~1 опирается на ряд 
приводимых ниже лемм~1--9.
    
    Введем обозначения: $W_r^n$, $r\hm\leq n$,~--- множество всех наборов 
вида $\{j_1,\ldots, j_r\}$, где $j_q\hm\in \{1,2,\ldots, n\}$ при $q\hm= 1,2,\ldots, 
r$ и~$j_1\hm< \cdots <j_r$; $V_r^m$, $r\hm\leq m$,~--- множество всех 
упорядоченных наборов вида $\{i_1, \ldots, i_r\}$, где $i_u\hm\in \{1,2,\ldots, 
m\}$ при $t\hm=1,2,\ldots, r$ и~$i_{u_1}\not= i_{u_2}$ при $u_1, 
u_2\hm=1,2,\ldots, r$.
    
    Пусть $\sigma\hm\in E^r_{k-1}$, $v\hm\in V_r^m$, $v\hm=\{ i_1, \ldots, i_r\}$, 
$w\hm\in W_r^n$, $M_{(v,w,\sigma)}$~--- совокупность всех матриц~$L$ 
в~$M^k_{mn}$, таких что в~подматрице, образованной столбцами с~номерами 
из~$w$, строка с~номером~$i_t$ принадлежит $Q_t(\sigma)$ при 
$t\hm=1,2,\ldots, r$.  
    
    \smallskip
    
    \noindent
    \textbf{Лемма~1.} \textit{Если $\sigma\hm\in E^r_{k-1}$, $v\hm\in V_r^m$, 
$w\hm\in W_r^n$, то} 
    $$
    \left\vert M_{(v,w,\sigma)}\right\vert=\Pi_r(\sigma) k^{mn-r^2}\,.
    $$
    
    \noindent
    Д\,о\,к\,а\,з\,а\,т\,е\,л\,ь\,с\,т\,в\,о\,.\ \ Действительно, строки 
с~номерами из~$v$ можно выбрать $\Pi_r(\sigma) k^{r(n-r)}$ способами, 
остальные строки~--- $k^{n(m-r)}$ способами. Лемма доказана.
    
    \smallskip
    
    Пусть $M_{(v,w,\sigma)}^*$~--- совокупность всех таких мат\-риц~$L$ 
в~$M_{(v,w,\sigma)}$, для которых набор столбцов с~номерами из~$w$ 
является упорядоченным $\sigma$-по\-кры\-ти\-ем и~$L\not= M_{(v^\prime, 
w,\sigma)}$ при $v^\prime\hm\not= v$. 
    
    \smallskip
    
    \noindent
    \textbf{Лемма~2.} \textit{Если $\sigma\hm\in E^r_{k-1}$, $v\hm\in V_r^m$, 
$w\hm\in W_r^n$, то} 
    $$
    M^*_{(v,w,\sigma)} \geq \Pi_r(\sigma) k^{mn-r^2}\!\left(\! 1-\fr{ (r+1)(k-
1)^r}{k^r}\!\right)^{m-r}.
    $$
    
    \noindent
    Д\,о\,к\,а\,з\,а\,т\,е\,л\,ь\,с\,т\,в\,о,.\ \ Действительно, строки 
с~номерами из~$v$ можно выбрать $\Pi_r(\sigma) k^{r(n-r)}$ способами, 
а~остальные строки можно выбрать $(k^n\hm- (r(k\hm-1)^{r-1} \hm+ (k\hm-
1)^r)k^{n-r})^{m-r}$ способами.
    
    \smallskip
    
    \noindent
    \textbf{Лемма~3.} \textit{Если $\sigma^\prime \hm\in E^r_{k-1}$, 
$\sigma^{\prime\prime} \hm\in E^l_{k-1}$, $v_1\hm\in V_r^m$, $v_2\hm\in 
V_l^m$, $w_1\hm\in W_r^n$, $w_2\hm\in W_l^n$ и~наборы~$v_1$ и~$v_2$ 
пересекаются по~$a$ $(a\hm>0)$ элементам, а~наборы~$w_1$ и~$w_2$ 
пересекаются по~$b$ $(b\hm>0)$ элементам, то}
\begin{multline*}
    \left\vert M_{(v_1, w_1,\sigma^\prime)} \cup 
M_{(v_2,w_2,\sigma^{\prime\prime})} \right\vert \leq{}\\
{}\leq \Pi_r(\sigma^\prime) \Pi_l 
(\sigma^{\prime\prime}) (k-1)^b d^{ab} k^{mn-r^2-l^2}\,.
    \end{multline*}
     
    \noindent
    Д\,о\,к\,а\,з\,а\,т\,е\,л\,ь\,с\,т\,в\,о\,.\ \ Оценим, сколькими способами 
можно построить матрицу из $M\hm= M_{(v_1, w_1, \sigma^\prime)}\cap\linebreak
\cap 
M_{(v_2, w_2, \sigma^{\prime\prime})}$. 
    
    Сначала выберем те элементы, которые расположены на пересечении 
строк с~номерами из~$v_1$ и~столбцов с~номерами из~$w_1$ (не более 
чем~$\Pi_r(\sigma^\prime)$ способов). Затем  выберем элементы, которые 
расположены на пересечении строк с~номерами из~$v_2$ и~столбцов 
с~номерами из~$w_2$, учитывая, что $ab$ из них расположены одновременно 
на пересечении строк с~номерами из~$v_1$ и~столбцов с~номерами из~$w_1$ 
(не более чем $\Pi_l(\sigma^{\prime\prime})(k\hm-1)^{b(1-a)}$ способов). 
Произвольным способом доопределим остальные элементы матрицы  
($k^{mn-r^2-l^2+ab}$способов). Из сказанного следует требуемая оценка 
для~$M$.
    
    \smallskip
    
    \noindent
    \textbf{Лемма~4.}\ \textit{Если $m^\alpha\hm\leq n\hm\leq d^m$, 
$\alpha\hm>1$, $d\hm=k/(k\hm-1)$, то имеет место} 
    \begin{multline*}
    \sum\limits^n_{r=1} \sum\limits_{\sigma\in E^r_{k-1}} \Pi_r(\sigma) C_n^r 
C_m^r r! k^{-r^2} \sim{} \\
{}\sim\sum\limits_{r\in \phi_d} \sum\limits_{\sigma\in E^r_{k-1} }
\Pi_r(\sigma) C_n^r C_m^r r! k^{-r^2}\,,\enskip n\to \infty\,.
\end{multline*}
    
    \noindent
    Д\,о\,к\,а\,з\,а\,т\,е\,л\,ь\,с\,т\,в\,о\,.\ \ Положим
    \begin{align*}
    p&=\fr{1}{2}\,\log_d mn -\fr{1}{2}\,\log_d \log_d mn\,;\\ 
    q&=\log_d \log_d  \log_d n\,;\\
    a_r &= \sum\limits_{\sigma\in E^r_{k-1}} \Pi_r(\sigma) C_n^r C_m^r r! k^{-
r^2}\,;\\ 
\Pi_r&= \sum\limits_{\sigma\in E^r_{k-1}} \Pi_r(\sigma)\,.
    \end{align*}
    
    1.~Пусть $r\geq p+q\hm-1$. Рассмотрим
    \begin{multline*}
    \left( \sigma_1+1\right)^r\cdots \left( \sigma_{r+1}+1\right)^r=
    \left( \sigma_1+1\right)^{r-1} \cdots\\
    \cdots \left(\sigma_r+1\right)^{r-
1}\left(\sigma_1+1\right) \cdots \left( \sigma_r+1\right) \left( 
\sigma_{r+1}+1\right)^r\,.
\end{multline*}
    
    Отсюда и~из того, что число членов в~$\Pi_{r+1}$ в~$k\hm-1$ раз больше 
числа членов в~сумме~$\Pi_r$, получаем:
    $$
    \fr{\Pi_{r+1}}{\Pi_r}\leq (k-1)^{2r+1}\,.
    $$
    Следовательно,
    \begin{multline*}
    \fr{a_{r+1}}{a_r}=\fr{(n-r)(m-r)}{r+1}\,d^{-2r-1}\leq{}\\
    {}\leq \fr{mn}{p}\,d^{-2p-
2q+1} \leq_n 4d^{-2q+1}\,.
    \end{multline*}
    
    2. Пусть $r\hm\leq p\hm-q\hm+1$ и~пусть $r_0(k)$~--- наименьшее 
целое~$r$, $r\hm\geq 2$, при котором
    $$
    \fr{\exp(1) r(k-2)^{r-2}}{(k-1)^{r-2}}\leq 1\,.
    $$
    
    Покажем, что при $r\hm\geq r_0(k)$ имеет место
    $$
    \Pi_{r-1}(\sigma) \leq \exp(1) (k-1)^{(r-1)(r-2)}\,.
    $$
    
    Нетрудно убедиться в~справедливости оценки при $k\hm\leq 3$ и~любом 
$r\hm\geq 2$. 
    
    Пусть $k>3$ и~для меньших значений~$k$ при указанном ограничении 
на~$r$ доказываемая оценка справедлива. Представляя сумму $\Pi_{r-
1}(\sigma)$ в~виде полинома Ньютона, получаем согласно предположению 
индукции
    \begin{multline*}
    \Pi_{r-1}(\sigma)= \left( 1+2^{r-2}+\cdots + (k-1)^{r-2}\right)^{r-1}\leq{}\\
    {}\leq 
    (k-1)^{(r-1)(r-2)} \left( 1+\fr{\exp(1) (k-2)^{r-2}}{(k-1)^{r-2}}\right)^{r-1}\leq{}\\
    {}\leq
    \exp(1) (k-1)^{(r-1)(r-2)}\,.
    \end{multline*}
        Очевидно, что при любом $k\hm\geq 2$ и~любом $r\hm\geq 1$
    $$
    (k-1)^{r(r-1)}\leq \Pi_r(\sigma)\,.
    $$
    
    Таким образом, при $r\hm\geq r_0(k)$ имеем:
    \begin{multline*}
    \fr{a_{r-1}}{a_r}= \fr{\exp(1) (k-1)^{(r-1)(r-2)} rk^{2r-1}}{(k-1)^{r(r-1)} (n-
r+1) (m-r+1)}\leq_n\\
    \leq_n \fr{\exp(1) (k-1) pd^{2p-2q+1}} {(n-p)(m-p)}\leq_n\\
    \leq_n \fr{2}{1-1/\alpha}\exp (1) (k-1) d^{-2q+1}\,.
    \end{multline*}
        При $2\leq r< r_0(k)$, пользуясь тем, что $\Pi_{r-
1}(\sigma)/\Pi_r(\sigma)\hm\leq 1$, имеем:
    $$
    \fr{a_{r-1}}{a_r}=\fr{rk^{2r-1}}{(n-p)(m-p)} \leq_n \fr{c}{mn}\,,\enskip 
c=const\,,\enskip c>0\,.
    $$
    Следовательно, 
    $$
    \sum\limits_{r\in [p+q,n]} \!\!\!\! a_r =o\left( \sum\limits_{r\in \phi_d} 
a_r\!\right);\enskip
    \sum\limits_{r\in [1, p-q]}\!\!\!\! a_r =o\left( \sum\limits_{r\in \phi_d} 
    a_r\!\right).
    $$
    
    \noindent
    \textbf{Лемма~5.} 
    \textit{Если $r, l\leq  c\log_d n$, $c\hm<1$, то имеет место}
    $$
    \sum\limits_{b=0}^{\min(r,l)} (k-1)^b d^{lb} C_n^r  C_r^b C_{n-r}^{l-b}\leq 
C_n^r C_n^l (1+\delta(n))\,,
    $$
    \textit{где} $\delta(n)\to 0$ при $n\hm\to\infty$.
    
    \noindent
    Д\,о\,к\,а\,з\,а\,т\,е\,л\,ь\,с\,т\,в\,о\,.\ \ Обозначим:
    $$
    \lambda_b = 
\fr{(k-1)^b d^{lb} C_n^r C_r^b C_{n-r}^{l-b}}{C_n^r C^l_{n-r}}\,. 
$$
    Так как
    $$
    \fr{C_r^b C_{n-r}^{l-b}}{C^l_{n-r}} \leq \left( \fr{rl}{n-r-l}\right)^b
    $$
    и~по условию $r, l\hm\leq c \log_d n$, $c\hm<1$, то 
    $$
    \lambda_b\leq \left( \fr{(k-1)\log_d^2 n}{n^{1-c} (1-2\log_d n/n)}\right)^b
    $$
и~оцениваемая сумма не превосходит $C_n^r C^l_{n-r}(1\hm+ \delta(n))$, где 
$\delta(n)\hm\to 0$, $n\hm\to \infty$. Отсюда, пользуясь неравенством $C^l_{n-
r}\hm\leq C^l_n$, получаем утверждение леммы.

\smallskip

    Будем считать $M^k_{mn}\hm=\{L\}$ пространством элементарных 
событий, в~котором каждое событие~$L$ происходит с~вероятностью $1/\vert 
M^k_{mn}\vert \hm= 1/k^{mn}$. Через ${\sf M}\,X(L)$ будем обозначать 
математическое ожидание случайной величины $X(L)$, через ${\sf D}\,X(L)$~--- дисперсию случайной величины~$X(L)$.
    
    \smallskip
    
    \noindent
    \textbf{Лемма~6}~\cite{6-duk}. \textit{Пусть для случайных величин 
$X_1(L)$ и~$X_2(L)$, определенных на~$M^k_{mn}$, выполнено 
$X_1(L)\hm\geq X_2(L)\hm\geq 0$ и~при $n\hm\to\infty$ верно}
    $$
    {\sf M}\,X_1(L)\sim {\sf M}\,X_2(L)\,;\enskip
    \fr{{\sf D}\,X_2(L)}{({\sf M}\,X_2(L))^2}\to 0\,.
    $$
    \textit{Тогда для почти всех матриц~$L$ из~$M^k_{mn}$ имеет место}: 
    $$
    X_2(L)\sim X_1(L) \sim {\sf M}\,X_2(L)\,,\enskip n\to \infty\,.
    $$
    
    Пусть $\sigma \hm\in E^r_{k-1}$, $v\hm\in V_r^m$ и~$w\hm\in W_r^n$. На 
множестве $M^k_{mn}\hm= \{L\}$ рассмотрим случайную величину 
$\eta^\sigma_{(v,w)}(L)$, равную~1, если~$L$ принадлежит 
$M_{(v,w,\sigma)}$, и~равную~0 иначе. Оценим вероятность события 
$\eta^\sigma_{(v,w)}(L)\hm=1$, обозначаемую далее через ${\sf P}\left( 
\eta^\sigma_{(v,w)}(L)\hm=1\right)$. Очевидно, в~силу леммы~1
    \begin{equation}
    {\sf P}\left( \eta^\sigma_{(v,w)} (L)=1\right) =\fr{\left\vert 
M_{(v,w,\sigma)}\right\vert}{\left\vert M^k_{mn}\right\vert} =\Pi_r(\sigma) k^{-
r^2}\,.
    \label{e1-duk}
    \end{equation} 
    
    Положим
    \begin{align*}
    \eta_1(L) &= \sum\limits^n_{r=1} \sum\limits_{v\in V_r^m} \sum\limits_{w\in 
W_r^n} \sum\limits_{\sigma\in E^r_{k-1}} \eta^\sigma_{(v,w}) (L)\,;\\
    \eta_2(L) &= \sum\limits_{r\in \phi_d} \sum\limits_{v\in V_r^m} 
\sum\limits_{w\in W_r^n} \sum\limits_{\sigma\in E^r_{k-1}} \eta^\sigma_{(v,w}) 
(L)\,.
    \end{align*}
    
    Нетрудно видеть, что $\eta_1(L)\hm= \Sigma_2(L)$ и~$\eta_2(L)\hm \leq 
\eta_1(L)$. В~силу~(1) 
    \begin{align*}
    {\sf M}\,\eta_1(L) &= \sum\limits^n_{r=1} \sum\limits_{\sigma \in E^r_{k-1} }
\Pi_r (\sigma) C_n^r C_m^r r! k^{-r^2}\,;\\
    {\sf M}\,\eta_2(L) &= \sum\limits_{r\in \phi_d} \sum\limits_{\sigma \in 
E^r_{k-1}} \Pi_r (\sigma) C_n^r C_m^r r! k^{-r^2}\,.
    \end{align*}
    
    Из полученных оценок для  ${\sf M}\, \eta_1(L)$, ${\sf M}\,\eta_2(L)$ 
и~леммы~4 сразу следует 
    
    \smallskip
    
    \noindent
    \textbf{Лемма~7.} \textit{Если $m^\alpha\hm\leq n\hm\leq d^m$, 
$\alpha\hm>1$, $d\hm=k/(k\hm-1)$, то имеет место}
    \begin{multline*}
    {\sf M}\,\eta_1(L)\sim {\sf M}\,\eta_2(L)\sim{}\\
    {}\sim  \sum\limits_{r\in \phi_d} 
\sum\limits_{\sigma\in E^r_{k-1}} \Pi_r(\sigma) C_n^r C_m^r r! k^{-r^2}\,,\enskip 
    n\to\infty\,.
\end{multline*}
    
    Пусть $\sigma \hm\in E^r_{k-1}$, $w\hm\in W_r^n$. На множестве 
$M^k_{mn}\hm= \{L\}$ рассмотрим случайную величину $\xi^\sigma_w(L)$, 
равную~1, если~$L$ принадлежит $B(L,\sigma)$, и~равную~0 иначе. 
Вероятность события $\xi_w^\sigma(L)\hm=1$ обозначим через 
${\sf P}(\xi_w^\sigma(L)\hm=1)$. 
    
    Оценим ${\sf P}(\xi_w^\sigma(L)\hm=1)$ сверху, пользуясь леммой~1. Нетрудно 
видеть, что
    \begin{multline}
    {\sf P}\left( \xi_w^\sigma(L)=1\right)\leq \sum\limits_{v\in V_r^m}\fr{\left\vert 
M_{(v,w,\sigma)}\right\vert }{
\left\vert M^k_{mn}\right\vert } ={}\\
{}=\Pi_r(\sigma) C_m^r r! k^{-r^2}\,.
    \label{e2-duk}
    \end{multline}
    
    С другой стороны, в~силу леммы~2 
    \begin{multline}
    {\sf P}\left( \xi^\sigma_w(L)=1\right) \geq{}\\
    {}\geq \sum\limits_{v\in V_r^m}\!\! \fr{\left\vert 
M^*_{(v,w,\sigma)}\right\vert}{\left\vert M^k_{mn}\right\vert}=
    \Pi_r(\sigma)C_m^r r! k^{-r^2}\times{}\\
    {}\times \left( 1- \fr{ r(k-1)^{r-1} + (k-1)^r}{
    k^r}\right)^{m-r}\,.
    \label{e3-duk}
    \end{multline}
    
    Положим
   \begin{align*}
    \xi_1(L)&=\sum\limits^n_{r=1} \sum\limits_{w\in W_r^n} 
\sum\limits_{\sigma\in E^r_{k-1}} \xi_w^\sigma (L)\,;\\
    \xi_2(L)&= \sum\limits_{r\in \phi_d} \sum\limits_{w\in W_r^n} 
\sum\limits_{\sigma\in E^r_{k-1}} \xi_w^\sigma (L)\,.
    \end{align*}
    
    Нетрудно видеть, что $\xi_1(L)\hm=\Sigma_1(L)$ и~$\xi_2(L)\hm\leq \xi_1(L)$.
    
    Имеем 
    \begin{align*}
    {\sf M}\,\xi_1(L) &= \sum\limits^n_{r=1} \sum\limits_{w\in W_r^n} 
\sum\limits_{\sigma\in E^r_{k-1}} \!{\sf P}\left( \xi_w^\sigma(L)=1\right)\,;\\
    {\sf M}\,\xi_2(L) &= \sum\limits_{r\in \phi_d} \sum\limits_{w\in W_r^n} 
\sum\limits_{\sigma\in E^r_{k-1}} \!{\sf P}\left( \xi_w^\sigma(L)=1\right)\,.
    \end{align*}
    
    Следовательно, в~силу~(\ref{e2-duk}) 
    \begin{multline}
    {\sf M}\,\xi_2(L)\leq {\sf M}\,\xi_1(L)\leq {}\\
    {}\leq \sum\limits^n_{r=1} 
\sum\limits_{\sigma\in E^r_{k-1}} \Pi_r(\sigma) C_n^r C_m^r r! k^{-r^2}\,.
    \label{e4-duk}
    \end{multline}
    
    Пользуясь~(\ref{e3-duk}) и~тем, что $mr(k\hm-1)^r/k^r\hm\leq \log_d^2 
n/n^c$, $c\hm>0$, при $r\hm\in \phi_d$, получаем:
    \begin{multline}
    {\sf M}\,\xi_1(L)\geq {\sf M}\,\xi_2(L)\geq{}\\
    {}\geq F(n) \sum\limits_{r\in 
\phi_d}\sum\limits_{\sigma\in E^r_{k-1}} \Pi_r(\sigma) C_n^r C_m^r r! k^{-r^2}\,,
    \label{e5-duk}
    \end{multline}
    где $F(n)\to 1$ при $r\hm\in \phi_d$, $n\hm\to \infty$.
    
    Из~(\ref{e4-duk}), (\ref{e5-duk}) и~леммы~4 сразу следует 
    
    \smallskip
    
    \noindent
    \textbf{Лемма~8.} \textit{Если $m^\alpha\hm\leq n\hm\leq d^m$, 
$\alpha\hm>1$, $d\hm= k/(k\hm-1)$, то имеет место}
   \begin{multline*}
    {\sf M}\,\xi_1(L) \sim {\sf M}\,\xi_2(L)\sim{}\\
    {}\sim \sum\limits_{r\in \phi_d} 
\sum\limits_{\sigma\in E^r_{k-1}}\!\! \Pi_r(\sigma) C_n^r C_m^r r! k^{-r^2}\,,\enskip 
n\to \infty\,.
\end{multline*}
    
    \smallskip
    
    \noindent
    \textbf{Лемма~9.} \textit{Если $m^\alpha\hm\leq n\hm\leq d^m$, 
$\alpha\hm>1$, $d\hm= k/(k\hm-1)$, то имеет место}
$$
    \fr{{\sf D}\,\eta_2(L)}{{\sf M}\,\eta_2(L))^2}\to 0\,;\enskip
    \fr{{\sf D}\,\xi_2(L)}{({\sf M}\,\xi_2(L))^2}\to 0\,,\enskip n\to \infty\,.
    $$
    
    \noindent
    Д\,о\,к\,а\,з\,а\,т\,е\,л\,ь\,с\,т\,в\,о\,.\ \ Нетрудно видеть, что
    
    \noindent
    $$
    {\sf M} \left( \eta_2(L)\right)^2=\sum\limits_{r,l\in\phi_d} 
    \sum\limits_{\stackrel{v_1\in 
V_r^m, v_2\in V_l^m,}{w_1\in W_r^n, w_2\in W_l^n}} 
\sum\limits_{\stackrel{\sigma^\prime 
\in E^r_{k-1},}{\sigma^{\prime\prime}\in E^l_{k-1}}} \fr{\vert M\vert}{k^{mn}}\,,
    $$
где $M\hm= M_{(v_1, w_1, \sigma^\prime)} \cap M_{(v_2, 
w_2,\sigma^{\prime\prime})}$
 Отсюда, пользуясь леммами~3 и~5, получаем
 \begin{multline} 
{\sf M}\left( \eta_2(L)\right)^2\leq{}\\
{}\leq \sum\limits_{r,l\in \phi_d} \sum\limits_{\stackrel{\sigma^\prime\in E^r_{k-1},} 
{\sigma^{\prime\prime} \in E^l_{k-1}}}\hspace*{-6pt} \Pi_r\left(\sigma^\prime\right) \Pi_l\left( 
\sigma^{\prime\prime}\right)\times{}\\
{}\times \sum\limits_{b=0}^{\min(r,l)} (k-1)^b d^{lb} C_n^r 
C_r^b C_{n-r}^{l-b} C_m^r r! C_m^l l! k^{-r^2-l^2}\leq{}\\
{}\leq
\sum\limits_{r,l\in\phi_d} \sum\limits_{\stackrel{\sigma^\prime\in E^r_{k-1},} 
{\sigma^{\prime\prime}\in E^l_{k-1}}}\hspace*{-6pt}
 \Pi_r\left(\sigma^\prime\right) \Pi_l\left( 
\sigma^{\prime\prime}\right)\times{}\\
{}\times C_n^r C_n^l C_m^r r! C_m^l l! k^{-r^2-l^2}(1+\delta(n))\,,
\label{e6-duk}
\end{multline}
где $\delta(n)\hm\to 0$ при $n\hm\to \infty$.

\columnbreak

    С другой стороны, в~силу леммы~7 при $n\hm\to \infty$ имеем:
    \begin{multline}
    \left( {\sf M}\,\eta_2(L)\right)^2 \sim \sum\limits_{r,l\in \phi_d} 
\sum\limits_{\stackrel{\sigma^\prime\in E^r_{k-1},}
{ \sigma^{\prime\prime}\in E^l_{k-1}}} 
\hspace*{-8pt}
\Pi_r \left(\sigma^\prime\right) \Pi_l\left( \sigma^{\prime\prime}\right)\times{}\\
{}\times C_n^r C_n^l 
C_m^r r! C_m^l l! k^{-r^2-l^2}.
    \label{e7-duk}
    \end{multline}

    Из~(\ref{e6-duk}), (\ref{e7-duk}) и~равенства ${\sf D}\,\eta_2(L)\hm= {\sf 
M}(\eta_2(L))^2\hm- ({\sf M}\,\eta_2(L))^2$ следует утверждение доказываемой 
леммы.  
    
    Утверждение теоремы~1 следует непосредственно из лемм~6--9. 
    
    \smallskip
    
    \noindent
    \textbf{Следствие~1.} Если $m^\alpha\hm\leq n\hm\leq 2^m$, 
$\alpha\hm>1$, то для почти всех матриц~$L$ из~$M^2_{mn}$ при $n\hm\to 
\infty$ справедливо
    $$
    \Sigma_1(L)\sim\Sigma_2(L)\sim \sum\limits_{r\in\phi_2} C_n^r C_m^r r! 
2^{-r^2}
    $$
и длины почти всех неприводимых покрытий мат\-ри\-цы~$L$ принадлежат 
интервалу~$\phi_2$.
    \smallskip
    
    Приведенные в~следствии~1 оценки типичных значений количественных 
характеристик множества неприводимых покрытий булевой матрицы 
первоначально получены в~\cite{9-duk} с~использованием понятий теории 
нормальных форм булевых функций. Идейно близки  результаты более ранней 
работы~\cite{5-duk}, в~которой предложен асимптотически оптимальный 
тестовый алгоритм распознавания и~при обосновании этого алгоритма 
получены асимптотики типичных значений числа неприводимых покрытий 
и~числа неприводимых покрытий типичной длины для матрицы из некоторого 
специального подмножества множества~$M^2_{mn}$. 
    
    \smallskip
    
    \noindent
    \textbf{Замечание~1.}\ Пусть $N_{mn}^k$~--- подмножество 
в~$M^k_{mn}$, содержащее все матрицы из~$M^k_{mn}$ с~попарно 
различными строками. В~утверждении теоремы~1 можно 
заменить~$M^k_{mn}$ на~$N^k_{mn}$, так как нетрудно показать, что при 
$m^2\hm= o(k^n)$, $n\hm\to \infty$, почти все матрицы из~$M^k_{mn}$  
представляют собой  матрицы с~попарно различными строками. 
    
    \smallskip
    
    \noindent
    \textbf{Замечание~2.} Из теоремы~1 и~замечания~1 следует, что 
асимптотически оптимальным является алгоритм дуализации над 
произведением цепей \mbox{RUNC-M+}, описанный в~\cite{8-duk}. Этот алгоритм 
перечисляет с~полиномиальной задержкой некоторое множество 
упорядоченных совместимых наборов столбцов матрицы~$L_R$, содержащее 
множество всех упорядоченных тупиковых покрытий этой матрицы. Число 
шагов алгоритма RUNC-M+ не превосходит $\Sigma_2(L_R)$.
  
  
  {\small\frenchspacing
 {%\baselineskip=10.8pt
 \addcontentsline{toc}{section}{References}
 \begin{thebibliography}{9}
     \bibitem{1-duk}
     \Au{Johnson D.\,S., Yannakakis~M., Papadimitriou~C.\,H.} On general all maximal 
independent sets~// Inform. Process. Lett., 1988. Vol.~27. Iss.~3. P.~119--123.
     \bibitem{2-duk}
     \Au{Fredman M., Khachiyan~L.} On the complexity of dualization of monotone disjunctive 
normal forms~// J.~Algorithm., 1996. Vol.~21. P.~618--628.
     \bibitem{3-duk}
     \Au{Boros E., Elbassioni~K., Gurvich~V., Khachiyan~L., Makino~K.} Dual-bounded 
generating problems: All minimal integer solutions for a monotone system of linear inequalities~// 
SIAM J.~Comput., 2002. Vol.~31. Iss.~5. Р.~1624--1643. 
     \bibitem{4-duk}
     \Au{Elbassioni K.} Algorithms for dualization over products of partially ordered sets~// 
SIAM J.~Discrete Math., 2009. Vol.~23. Iss.~1. P.~487--510.
     \bibitem{5-duk}
     \Au{Дюкова Е.\,В.} Об асимптотически оптимальном алгоритме построения тупиковых 
тестов // Докл. Акад. наук СССР, 1977. Т.~233. №\,4. С.~527--530.
     \bibitem{6-duk}
     \Au{Дюкова Е.\,В., Прокофьев~П.\,А.} Об асимптотически оптимальных алгоритмах 
дуализации~// Ж.~вычисл. матем. матем. физ., 2015. Т.~55. №\,5. 
С.~895--910.
     \bibitem{7-duk}
     \Au{Носков В.\,Н., Слепян~В.\,А.} О~числе тупиковых тестов для одного класса 
таблиц~// Кибернетика, 1972. №\,1. С.~60--65.
     \bibitem{8-duk}
     \Au{Дюкова Е.\,В., Масляков~Г.\,О., Прокофьев~П.\,А.} О~дуализации над 
произведением частичных порядков~// Машинное обучение и~анализ данных, 2017. Т.~3. 
№\,4. С.~239--249.
     \bibitem{9-duk}
     \Au{Дюкова Е.\,В.} О~сложности реализации некоторых процедур распознавания~// 
Ж.~вычисл. матем. матем. физ., 1987. Т.~27. №\,1. С.~114--127.
 \end{thebibliography}

 }
 }

\end{multicols}

\vspace*{-3pt}

\hfill{\small\textit{Поступила в~редакцию 15.11.18}}

\vspace*{8pt}

%\pagebreak

%\newpage

%\vspace*{-28pt}

\hrule

\vspace*{2pt}

\hrule

%\vspace*{-2pt}

\def\tit{ON THE NUMBER OF MAXIMAL INDEPENDENT ELEMENTS 
    OF~PARTIALLY ORDERED SETS (THE~CASE OF~CHAINS)}

\def\titkol{On the number of maximal independent elements 
    of~partially ordered sets (the~case of~chains)}

\def\aut{E.\,V.~Djukova$^{1,2}$, G.\,O.~Maslyakov$^2$, and~P.\,A.~Prokofyev$^3$}

\def\autkol{E.\,V.~Djukova, G.\,O.~Maslyakov, and~P.\,A.~Prokofyev}

\titel{\tit}{\aut}{\autkol}{\titkol}

\vspace*{-11pt}


\noindent
    $^1$Federal Research Center ``Computer Science and Control'' of the Russian 
Academy of Sciences, 42~Vavilov Str.,\linebreak
$\hphantom{^1}$Moscow 119333, Russian Federation

    \noindent
    $^2$Faculty of Computational Mathematics and Cybernetics, 
M.\,V.~Lomonosov Moscow State University, 1-52~Lenin-\linebreak 
$\hphantom{^1}$skiye Gory, GSP-1, 
Moscow 119991, Russian Federation
    
    \noindent
    $^3$Mechanical Engineering Research Institute of the Russian Academy of 
Sciences, 4~Bardina Str., Moscow 119334,\linebreak
$\hphantom{^1}$Russian Federation

\def\leftfootline{\small{\textbf{\thepage}
\hfill INFORMATIKA I EE PRIMENENIYA~--- INFORMATICS AND
APPLICATIONS\ \ \ 2019\ \ \ volume~13\ \ \ issue\ 1}
}%
 \def\rightfootline{\small{INFORMATIKA I EE PRIMENENIYA~---
INFORMATICS AND APPLICATIONS\ \ \ 2019\ \ \ volume~13\ \ \ issue\ 1
\hfill \textbf{\thepage}}}

\vspace*{6pt}
    
    
        
    \Abste{One of the central intractable problems of logical data analysis 
    is considered~--- dualization over the product of partially ordered sets. 
    The authors investigate an important special case where each order is a~chain. 
    If the power of each chain is two, then the problem under consideration 
    is to construct the reduced disjunctive normal form of a~monotone Boolean 
    function given by a~conjunctive normal form. This is equivalent to enumerating 
    irreducible covers of a~Boolean matrix. Provided the growth of the row's 
    number of the Boolean matrix to be less than the growth of the column's number, 
    the asymptotic for the typical number of irreducible covers is known. 
    In the present work, a~similar result is obtained for the dualization 
    over the product of chains when the power of each chain is more than two. 
    Obtaining such asymptotic estimates is a~technically complex task and is 
    necessary, in particular, to justify the existence of asymptotically optimal 
    algorithms for the problem 
    of monotonic dualization and various generalizations of this problem.}
    
    \KWE{problem of dualization; product of partially ordered sets; 
    chain; covering of a~Boolean matrix; 
    ordered covering of an integer matrix; asymptotically optimal algorithm}
    
   
    
\DOI{10.14357/19922264190104}

\vspace*{-8pt}

 \Ack
    \noindent
    The research was partially supported by the Russian Foundation for Basic Research 
(project 19-01-00430-a).



%\vspace*{6pt}

  \begin{multicols}{2}

\renewcommand{\bibname}{\protect\rmfamily References}
%\renewcommand{\bibname}{\large\protect\rm References}

{\small\frenchspacing
 {%\baselineskip=10.8pt
 \addcontentsline{toc}{section}{References}
 \begin{thebibliography}{9}
    \bibitem{1-duk-1}
    \Aue{Johnson, D., M.~Yannakakis, and C.~Papadimitriou.} 1988. On 
generating all maximal independent sets. \textit{Inform. Process. Lett.} 27(3):119--123.
    \bibitem{2-duk-1}
    \Aue{Fredman, M., and L.~Khachiyan.} 1996. On the complexity of dualization 
of monotone disjunctive normal forms. \textit{J.~Algorithm.} 21(3):618--628.
    \bibitem{3-duk-1}
    \Aue{Boros, E., K.~Elbassioni, V.~Gurvich, L.~Khachiyan, and K.~Makino}. 
2002. Dual-bounded generating problems: All
 minimal integer solutions for 
a~monotone system of linear inequalities. \textit{SIAM J.~Comput.}  
31(5):1624--1643. 
{\looseness=1

}
    \bibitem{4-duk-1}
    \Aue{Elbassioni, K.} 2009. Algorithms for dualization over products of partially 
ordered sets. \textit{SIAM J.~Discrete Math.} 23(1):487--510. 
    \bibitem{5-duk-1}
    \Aue{Djukova, E.} 1977. On an asymptotically optimal algoritm for 
constructing irredundant tests. \textit{Sov. Math. Dokl.} 18(2):423--426.
    \bibitem{6-duk-1}
    \Aue{Djukova, E., and P.~Prokofyev.} 2015. Asymptotically optimal 
dualization algoritms. \textit{Comp. Math. Math. Phys.} 55(5):891--905.
    \bibitem{7-duk-1}
    \Aue{Noskov, V., and V.~Slepian.} 1972. O chisle tupikovykh testov dlya 
odnogo klassa tablits [On the number of irredundant tests for a~class of tables]. 
\textit{Kibernetika} [Cybernetics] 1:60--65.
    \bibitem{8-duk-1}
    \Aue{Djukova, E., G.~Masliakov, and P.~Prokofjev.} 2017. O~dua\-li\-za\-tsii nad 
proizvedeniem chastichnykh poryadkov [About product over partially ordered sets]. 
\textit{Mashinnoe obuchenie i~analiz dannykh} [Machine Learning Data Anal.] 
3(4):239--249. 
    \bibitem{9-duk-1}
    \Aue{Dyukova, E.} 1987. On the complexity of implementation of some 
recognition procedures. \textit{Comp. Math. Math. Phys.} 27(1):74--83. 
\end{thebibliography}

 }
 }

\end{multicols}

\vspace*{-6pt}

\hfill{\small\textit{Received November 15, 2018}}

%\pagebreak

%\vspace*{-18pt}
    
    \Contr
    
    \noindent
    \textbf{Djukova Elena V.} (b.\ 1945)~--- Doctor of Science in physics and 
mathematics, principal scientist, Federal Research Center ``Computer Science and 
Control'' of the Russian Academy of Sciences, 42~Vavilov Str., Moscow 119333, 
Russian Federation; associate professor, Faculty of Computational Mathematics and 
Cybernetics, M.\,V.~Lomonosov Moscow State University, 1-52~Leninskiye Gory, 
GSP-1, Moscow 119991, Russian Federation; \mbox{edjukova@mail.ru} 
    \vspace*{3pt}
    
    \noindent
    \textbf{Maslyakov Gleb O.} (b.\ 1996)~--- student, Faculty of Computational 
Mathematics and Cybernetics, M.\,V.~Lomonosov Moscow State University,  
1-52~Leninskiye Gory, GSP-1, Moscow 119991, Russian Federation;  
\mbox{gleb-mas@mail.ru} 
    
    \vspace*{3pt}
    
    \noindent
    \textbf{Prokofyev Petr A.} (b.\ 1982)~--- Candidate of Science (PhD) in 
physics and mathematics, scientist, Mechanical Engineering Research Institute of the 
Russian Academy of Sciences, 4~Bardina Str., Moscow 119334, Russian 
Federation; \mbox{p\_prok@mail.ru}
    


\label{end\stat}

\renewcommand{\bibname}{\protect\rm Литература}       