\def\stat{gorbunova}

\def\tit{РЕСУРСНЫЕ СИСТЕМЫ МАССОВОГО ОБСЛУЖИВАНИЯ С~ПРОИЗВОЛЬНЫМ 
ОБСЛУЖИВАНИЕМ$^*$}

\def\titkol{Ресурсные системы массового обслуживания с~произвольным 
обслуживанием}

\def\aut{А.\,В.~Горбунова$^1$, В.\,А.~Наумов$^2$,  Ю.\,В.~Гайдамака$^3$,
К.\,Е.~Самуйлов$^4$}

\def\autkol{А.\,В.~Горбунова, В.\,А.~Наумов,  Ю.\,В.~Гайдамака,
К.\,Е.~Самуйлов}

\titel{\tit}{\aut}{\autkol}{\titkol}

\index{Горбунова А.\,В.}
\index{Наумов В.\,А.}
\index{Гайдамака Ю.\,В.}
\index{Самуйлов К.\,Е.}
\index{Gorbunova A.\,V.}
\index{Naumov V.\,A.}
\index{Gaidamaka Yu.\,V.} 
\index{Samouylov K.\,E.}


{\renewcommand{\thefootnote}{\fnsymbol{footnote}} \footnotetext[1]
{Публикация подготовлена при финансовой поддержке 
Минобрнауки России (проект 2.882.2017/4.6).}}


\renewcommand{\thefootnote}{\arabic{footnote}}
\footnotetext[1]{Российский университет дружбы народов, 
gorbunova\_av@rudn.university}
\footnotetext[2]{Исследовательский институт инноваций, Хельсинки, 
Финляндия, valeriy.naumov@pfu.fi}
\footnotetext[3]{Российский университет дружбы народов; Институт 
проб\-лем информатики Федерального исследовательского центра <<Информатика 
и~управ\-ле\-ние>> Российской академии наук, \mbox{gaydamaka\_yuv@rudn.university}}
\footnotetext[4]{Российский университет дружбы народов; Институт 
проб\-лем информатики Федерального исследовательского центра <<Информатика 
и~управ\-ле\-ние>> Российской академии наук, 
\mbox{samouylov\_ke@rudn.university}}

%\vspace*{8pt}



\Abst{Представлен обзор ресурсных сис\-тем массового обслуживания (РСМО). Одному из 
методов их исследования посвящен целый раздел. Ценной особенностью предложенного 
метода является значительное упрощение анализа сис\-те\-мы и~при этом сохранение 
высокой точности оценки, а~в~отдельных случаях и~отсутствие потери точности 
в~принципе. Так, в~частности, вместо исходной ресурсной модели было предложено 
рассматривать упрощенную, в~которой после ухода заявки освобождается не тот 
объем ресурсов, который заявка занимала, а~его некоторая случайная величина, что 
позволяет значительно упрос\-тить случайный процесс, описывающий поведение 
сис\-те\-мы. Впоследствии для случая пуассоновского входящего потока 
и~экспоненциального времени обслуживания была строго доказана эквивалентность 
результатов для исходной и~упрощенной моделей. Отдельный практический интерес 
представляют работы с~рекуррентной дисциплиной обслуживания. Результатам их 
анализа посвящена значительная часть работы.}

\KW{ресурсная система массового обслуживания; непрерывный 
ресурс; дискретный ресурс; ограниченный ресурс; рекуррентное обслуживание; 
гетерогенная сеть; стационарное распределение; полумарковский процесс}

\DOI{10.14357/19922264190114}
  
\vspace*{-4pt}


\vskip 10pt plus 9pt minus 6pt

\thispagestyle{headings}

\begin{multicols}{2}

\label{st\stat}

\section{Введение}

Описание модели РСМО достаточно подробно 
представлено в~\cite{Ch_1}. Напомним лишь, что ее отличительной чертой является 
то, что заявкам помимо приборов и~мест ожидания требуются дополнительные 
ресурсы. В~случае нехватки ка\-ко\-го-ли\-бо из них заявка теряется.
Случайный объем дополнительного ресурса, занимаемого заявкой в~подавляющем 
большинстве исследований на все время ее пребывания в~сис\-те\-ме, может носить как 
дискретный, так и~непрерывный характер. Случаи детерминированных требований 
заявок к~ресурсам подробно описаны в~\cite{Kelly,Ross,Basharin} и~здесь не 
рассматриваются.

Если в~ресурсной модели имеется единственный тип конечного ресурса, то, как 
правило, таким образом моделируется ограниченность памяти некоторого 
технического устройства или целой ин\-фор\-ма\-ци\-он\-но-вы\-чис\-ли\-тель\-ной 
сис\-те\-мы~\cite{Romm_21_1971,Kac,Tihonenko_27_1985,Pechinkin_29_2012}. 
%
Сис\-те\-мы 
с~множественными ресурсами моделируют беспроводные сети, в~которых разного рода 
радиоресурсы должны распределяться между активными абонентами и~освобождаться 
после завершения  их обслуживания~\cite{Naumov_3_2016}.
Системы с~множественными ресурсами, которые выделяются не одновременно, 
а~последовательно, могут также использоваться для моделирования беспроводных 
гетерогенных сетей, в~которых обслуживание пользователей происходит 
последовательно вследствие необходимости выделения ресурсов на нескольких 
станциях, имеющих различные каналы передачи данных. В~этом случае речь уже идет 
о~модели многофазной РСМО~\cite{Mois_4_2017,Mois_5_2017}.


Сеть РСМО с~заявками нескольких 
классов и~сигналами, инициирующими переходы заявок между узлами, в~которой объем 
ресурсов, запрашиваемый клиентами, может быть как положительным, так 
и~отрицательным, применима для моделирования автономных точек доступа, которые 
имеют возможность динамически увеличивать свою пропускную способность, или для 
моделирования гетерогенной сети, развернутой в~ка\-кой-ли\-бо 
области~\cite{Sopin_12_2017,Sopin_13_2017,Naumov_14_2017,Naumov_15_2017}. Положительные 
заявки в~таком случае будут представлять запросы на выделение некоторого 
случайного объема работ, тогда как отрицательные заявки будут соответствовать 
увеличению объема ресурсов, доступных положительным заявкам.
Таким образом, становится возможным учитывать динамику пользователей,
%(или транспортных средств),
когда новые <<объекты>> могут предоставлять свои сетевые возможности соседним 
потребителям, а также возвращать их, когда они покидают территорию.

Статья организована следующим образом: в~разд.~2 описываются основные типы 
РСМО\linebreak с~рекуррентным обслуживанием, методы их 
исследования и~полученные результаты. Третий раздел посвящен новому подходу 
к~исследованию РСМО~--- методу упрощения и~результатам его применения 
к~анализу ресурсных моделей. В~заключении кратко подведены итоги работы.

%\vspace*{-4pt}


\section{Ресурсная система массового обслуживания с~произвольным обслуживанием}

%\vspace*{-2pt}

Стоит отметить, что одним из первых в~числе ученых, обративших внимание на 
актуальность исследования РСМО и~много сделавших для ее изуче\-ния, является 
О.\,М.\,Тихоненко. Его авторству начиная с~1985~г.~\cite{Tihonenko_27_1985} 
принадлежит множество статей, посвященных этой тематике. В~большинстве из них 
рассматриваются модели ресурсных сис\-тем с~дисциплиной разделения процессора 
(Egalitarian Processor Sharing, EPS). Так, работа~\cite{Tihonenko_22_2001} 
посвящена анализу СМО $M|G|1|m$ $(m\leq\infty)$ с~разделением процессора, 
одним типом ресурса и~существующей зависимостью между временем обслуживания 
заявки и~объемом требуемого ей ресурса. Для этой сис\-те\-мы определяются 
стационарные вероятности состояний и~вероятность потери заявок.

В статье \cite{Tihonenko_24_1990} исследуются однолинейные СМО вида $M|G|1|m$ 
$(m\leq\infty)$ с~единственным типом ограниченного объема ресурса, 
в~которых поступающие заявки также характеризуются зависимостью между временем 
обслуживания заявки и~случайной величиной (с.в.)\ требуемого ей объема ресурса. 
Для первой исследуемой СМО объем ресурса в~сис\-те\-ме не ограничен, в~отличие от 
числа мест для ожидания; в~результате ее анализа получено выражение для 
преобразования Лап\-ла\-са--Стилть\-еса суммарного объема ресурсов, занятого заявками. 
Для второй СМО, в~которой объем ресурса ограничен, а~накопитель~--- бесконечной 
емкости, получена оценка вероятности потери заявки.

В статьях~\cite{Sengupta,Tihonenko_40_2002} анализируется СМО $M|G|1|\infty$ 
с~разделением процессора и~одним типом ресурса неограниченного объема $R\hm=\infty$, 
при этом заявке помимо ресурса некоторого случайного объема требуется некоторое 
число приборов для ее обслуживания, и~эти величины зависимы, т.\,е.\ задана их 
совместная функция распределения. В результате было получено выражение для 
преобразования Лап\-ла\-са--Стилть\-еса суммарного объема ресурсов, занятых 
находящимися в~сис\-те\-ме заявками, в~стационарном режиме.
В~\cite{Tihonenko_40_2002} результаты были обобщены на нестационарный режим 
и,~кроме того, было предложено использовать полученные в~\cite{Sengupta} результаты 
для оценки вероятности потери заявок в~сис\-те\-ме с~ограниченным объемом ресурса, 
однако такая оценка далеко не всегда приемлема для использования~\cite{Tihonenko_40_2002}.
Поэтому в~работе~\cite{Tihonenko_25_2010} анализируется более общий вид СМО 
с~ограничением на объем ресурса $R\hm<\infty$: $M_{L}|G|1|m$ с~разделением процессора и~одним типом ресурса, т.\,е.\ 
в~сис\-те\-му поступает $L$ пуассоновских потоков 
заявок, причем заявка класса~$l$, $l\hm=\overline{1,L}$, характеризуется не только 
необходимым ей объемом ресурса, но и~константой $n_l\leq N$ ~--- числом 
приборов, необходимых для ее обслуживания.
Объем требуемого заявке ресурса и~ее длина, под которой понимается время 
пребывания заявки в~сис\-те\-ме при условии отсутствия в~ней других заявок (т.\,е.\ 
фактически время обслуживания заявки), являются зависимыми величинами. Их 
совместное распределение также зависит от класса заявки. Было получено 
стационарное распределение числа заявок в~сис\-те\-ме и~вероятности потери заявок 
каждого класса и, кроме того,  аналогичные характеристики для некоторых частных 
случаев.
{\looseness=1

}

В~\cite{Tihonenko_26_2005} фактически исследуется СМО $M|G|N|0$ с~единственным 
типом ограниченного ресурса $R\hm\leq \infty$, причем каждая заявка характеризуется 
тремя случайными признаками: числом приборов, необходимых для обслуживания, 
объемом ресурса и~временем обслуживания, т.\,е.\ помимо некоторого случайного 
объема ресурсов заявке требуется еще и~случайное число приборов для 
обслуживания. Всего имеется $N$ классов таких требований, для каждого класса 
заявок задана своя совместная функция распределения времени ее обслуживания 
и~объема требуемого ей ресурса. Получено стационарное распределение числа заявок 
в~сис\-те\-ме и~вероятности потери заявки каждого из име\-ющих\-ся классов. Также 
в~\cite{Tihonenko_26_2005} проанализированы некоторые частные случаи, например: 
когда объем требуемого заявке ресурса и~время ее обслуживания являются 
независимыми с.в., когда время обслуживания заявки пропорционально объему ее 
ресурса, а~также, как и~в~работе \cite{Romm_21_1971}, рассматривается частный 
случай, когда для обслуживания любой заявки требуется один прибор и~функция 
распределения объема требуемых ресурсов одна и~та же для всех классов заявок.

В серии работ другого авторства~\cite{Pechinkin_28_2011,Pechinkin_29_2012,Pechinkin_30_1998,Pechinkin_31_1999} 
исследуются ресурсные сис\-те\-мы, особенностью которых является инверсионный 
порядок обслуживания.
По мнению авторов, в~случае с~дисциплиной LIFO (last in, first out)
можно получить рекуррентные 
алгоритмы, пригодные для численных методов вычисления стационарных вероятностей 
состояний и~стационарного распределения времени пребывания заявки в~сис\-те\-ме. 
Однако стоит заметить, что предложенные рекуррентные алгоритмы в~случае 
с~непрерывной функцией распределения объема ресурса~\cite{Pechinkin_30_1998,Pechinkin_31_1999} имеют большую вычислительную 
сложность в~силу необходимости решения довольно сложных интегральных уравнений, 
а~условие дискретности с.в.\ требуемого заявке объема ресурса~\cite{Pechinkin_28_2011,Pechinkin_29_2012} 
позволяет получить более прос\-тые и~эффективные алгоритмы расчета 
основных стационарных характеристик СМО.

В работе~\cite{Pechinkin_30_1998} рассматривается СМО типа $M_k|G|1|m,$ $0\hm\leq m 
\hm<\infty$, с~инверсионным порядком обслуживания без прерывания и~единственным 
типом ресурса объема~$R$. Иными словами, на сис\-те\-му поступает пуассоновский 
входящий поток второго рода: его интенсивность~$\lambda_k$ зависит от числа 
заявок~$k$, находящихся в~сис\-те\-ме. Как и~в предыду\-щих работах, упомянутых в~этом 
разделе, здесь тоже существует зависимость между временем обслуживания 
и~требуемым заявке объемом ресурса, задаваемая совместной функцией распределения. 
Под инверсионным порядком обслуживания без прерывания подразумевается, что 
каждая принятая в~сис\-те\-му заявка становится на первое место в~очереди, причем 
если поступающая в~сис\-те\-му заявка застает в~очереди уже~$m$~заявок, но требуемый 
ей для обслуживания объем ресурса суммарно с~необходимыми объемами ресурсов для 
имеющихся в~сис\-те\-ме заявок меньше~$R$, то она вытесняет заявку, стоящую на 
приборе, и~занимает ее место. Получены выражения для стационарных вероятностей 
различных состояний, в~том числе распределение числа заявок в~сис\-те\-ме 
и~суммарного объема занятого этими заявками ресурса, а~также вероятность принятия 
заявки в~сис\-те\-му, вероятность ее обслуживания; стационарное распределение 
времени пребывания заявки в~сис\-те\-ме. Также рассмотрены некоторые частные случаи: 
независимость времени обслуживания и~объема необходимого ресурса; отсутствие 
ограничения на суммарный объем ($R\hm=\infty$); экспоненциальное время 
обслуживания.

В статье~\cite{Pechinkin_31_1999} исследуется СМО с~марковским входящим потоком 
вида $\mathrm{MAP}|G|1|m$, так же как и~в~\cite{Pechinkin_30_1998}, с~инверсионным 
порядком обслуживания, но уже с~прерыванием, а~также с~единственным типом 
ограниченного ресурса объема $R$ и~зависимыми временем обслуживания и~требуемым 
объемом ресурса для заявок. Получены формулы для стационарного распределения 
числа заявок в~сис\-те\-ме, для времени пребывания заявки в~сис\-те\-ме, а~также 
некоторые характеристики для частных случаев: независимые время обслуживания 
и~объем ресурса; $m\hm=\infty$; $R\hm=\infty$.

В работе~\cite{Pechinkin_28_2011} исследуется однолинейная СМО с~ограничением на 
суммарный объем занятых ресурсов одного типа ($R$) и~геометрическим второго рода 
входящим потоком заявок $\mathrm{Geo}_k|G|1|m$, функционирующая в~дискретном времени 
с~инверсионным порядком обслуживания без прерывания обслуживания и~с~существующей 
зависимостью для объема ресурса, т.\,е.\ фактически сис\-те\-ма, аналогичная СМО 
$M_k|G|1|m$ с~непрерывным временем. Получены соотношения, которые позволяют 
вычислять основные стационарные характеристики (стационарные ве\-ро\-ят\-ности 
состояний и~стационарное распределение времени пребывания заявки в~сис\-теме).

В~\cite{Pechinkin_29_2012} рассматривается сис\-те\-ма $\mathrm{Geo}_k|G|1|\infty$, 
аналогичная СМО из~\cite{Pechinkin_28_2011}, но объем требуемого заявке ресурса 
имеет дискретное распределение, потери происходят только в~том случае, когда 
суммарный объем находящихся в~сис\-те\-ме заявок превышает величину~$R$. Получены 
соотношения, позволяющие вычислять основные стационарные характеристики СМО.

В статьях~\cite{Mois_1_2017,Mois_2_2017,Mois_3_2016,Mois_4_2017,Mois_5_2017} 
исследуются многофазные РСМО с~непуассоновским входящим потоком, 
неэкспоненциальным обслуживанием, бесконечным числом приборов и~неограниченным 
объемом выделяемого ресурса на каждой из фаз. Так, в~\cite{Mois_2_2017} проведен 
анализ РСМО типа $\mathrm{MMPP}|G|\infty$, в~\cite{Mois_3_2016}~--- $\mathrm{MAP}|G|\infty$ 
и~в~\cite{Mois_1_2017}~--- $G|G|\infty$, во всех случаях речь идет об однофазных 
СМО.
В~статьях \cite{Mois_4_2017,Mois_5_2017} исследуются двухфазные СМО.
Перечисленные статьи объединяет общий метод исследования сис\-тем: метод 
асимптотического анализа в~условиях растущей интенсивности входящего потока 
с~некоторыми особенностями в~зависимости от типа входящего потока. В~результате 
анализа получены асимптотические выражения для стационарного распределения 
вероятностей числа заявок в~сис\-те\-ме и~суммарного объема занятых ресурсов (на 
каждой из фаз). Проведенные исследования позволили сделать вывод о~том, что во 
всех случаях асимптотическое распределение суммарного объема занятых ресурсов 
является гауссовским с~соответствующими параметрами и~размерностью, 
согласующейся со случайным процессом, описывающим поведение изучаемой сис\-темы.

В статье~\cite{Naumov_18_2016} для РСМО с~пуассоновским входящим потоком 
интенсивности~$\lambda$, $M$ типами ограниченных ресурсов и~$N$~приборами 
получены выражения для стационарных вероятностей процесса 
$X(t)\hm=(\xi(t);\boldsymbol{\delta}(t))$ ($\xi(t)$~--- число заявок в~сис\-те\-ме; 
$\boldsymbol{\delta}(t)$~--- суммарный объем занятых ресурсов каждого типа)
%, не упрощенной СМО
в случае произвольного закона распределения времени обслуживания $B(x)$:
\begin{equation*}
p_0=\lim\limits_{t\rightarrow \infty}{\sf P}\{\xi(t)=0\}=\left(1+\sum\limits_{k=1}^{N}
G^{(k)}(\mathbf{R})\fr{\rho^{k}}{k!} \right ) ^{-1};
\end{equation*}

\vspace*{-12pt}

\noindent
\begin{multline*}
Q_k(\mathbf{x})=\lim\limits_{t\rightarrow \infty}{\sf P}\{ \xi(t)=k; 
\boldsymbol{\delta}(t)\leq \mathbf{x}\}={}\\
{}= p_0G^{(k)}(\mathbf{x}) 
\fr{\rho^{k}}{k!}\,,\enskip \mathbf{0}\leq \mathbf{x} \leq \mathbf{R},\
1\leq k \leq N.
\end{multline*}
Здесь $G^{(k)}(\mathbf{x})$~--- $k$-крат\-ная свертка функции распределения:
\begin{equation*}
G(\mathbf{x})=\fr{1}{b}\int\limits_{0}^{\infty}tH(dt,\mathbf{x})\,,
\end{equation*}
где $H(t,\mathbf{x})$~--- совместная функция распределения длительности 
обслуживания и~вектора объемов ресурсов, необходимых поступившей заявке; 
$\rho\hm=\lambda b$, $b\hm=\int\nolimits_{0}^{\infty}tdB(t)$.
Заметим, что в~формуле~(11) \mbox{статьи} \cite{Naumov_18_2016} символ~$D$ следует 
читать как~$F$.

\section{Метод упрощения}

Многолинейные СМО с~множественными ресурсами, рекуррентным входящим потоком 
и~произвольными функциями распределения длительностей обслуживания и~объемов 
ресурсов впервые\linebreak исследовались в~\cite{Naumov_1_2014}. В~этой статье был 
предложен новый подход к~анализу РСМО,
поз\-во\-ля\-ющий значительно упростить вычисления, что принципиально важно 
и~заслуживает отдельного внимания. Итак, поскольку анализ РСМО довольно сложен,
%, по причинам упомянутым выше,
было предложено вместо исходной СМО исследовать ее упрощенный аналог, который 
функционирует аналогично исходной сис\-те\-ме за исключением того,  что объемы 
ресурсов, освобождаемые по за\-вер\-шении обслуживания, могут отличаться от тех, %\linebreak 
которые были выделены заявке в~начале ее обслуживания.
%При заданных суммарных объемах занятых ресурсов и~заданном числе заявок 
%в~системе в~момент завершения обслуживания объемы освобождаемых ресурсов не 
%зависят от поведения системы до этого момента и~имеют функцию распределения 
%специального вида, вычисляемую по формуле Байеса.
Случайный процесс, опи\-сы\-ва\-ющий поведение упрощенной сис\-те\-мы, легче поддается 
анализу, поскольку отпадает необходимость запоминания объемов ресурсов, 
занимаемых каж\-дой обслуживаемой заявкой, и~при анализе надо пом\-нить лишь 
суммарные объемы занятых ресурсов.
В~этой же работе были продемонстрированы результаты моделирования исходных 
и~упрощенных сис\-тем при пуассоновском входящем потоке и~экспоненциальном 
обслуживании, которое показало,  что средние значения объемов занимаемых 
ресурсов в~стационарном режиме для них оказались очень близкими. Позднее 
в~\cite{Naumov_2_2014}, также с~помощью имитационного моделирования, было 
показано, что близ\-ки не только средние значения, но и~стационарные распределения 
суммарных объемов занимаемых ресурсов.

И,~наконец, в~\cite{Naumov_3_2016} было строго обосновано, что переход к~анализу 
упрощенной сис\-те\-мы не меняет стационарного распределения суммарных объемов 
занятых ресурсов для случая пуассоновского входящего потока и~экспоненциального 
времени обслуживания. Данный результат был доказан для случая непрерывной 
функции распределения объема требуемых заявке ресурсов.
Вообще говоря, анализируемый здесь процесс 
$Y(t)\hm=(\xi(t);\boldsymbol{\delta}(t))$ не является марковским, поскольку 
в~момент ухода заявки из сис\-те\-мы должны быть освобождены ресурсы в~объемах, равных 
объемам ресурсов, выделенных этой заявке при ее поступлении в~сис\-те\-му. Однако 
при небольшом изменении исходной системы можно получить более простую систему 
с~множественными ресурсами, для которой составной случайный процесс, компонентами 
которого являются два процесса (число заявок в~сис\-те\-ме и~суммарный объем 
занимаемых ими ресурсов), образуют полумарковский процесс.

Итак, рассмотрим многолинейную СМО с~пуассоновским входящим потоком, 
экспоненциальным временем обслуживания и~$M$ типами ресурсов, функционирование 
которой подчиняется классическому набору правил~\cite{Ch_1},
за исключением того, что условие:
\begin{enumerate}
\item[] в~момент окончания обслуживания заявки суммарный объем занятого ресурса 
каждого типа уменьшается на величину ресурса, выделенного этой заявке,
\end{enumerate}
заменяется следующим:
\begin{enumerate}
\item[*] в~момент окончания обслуживания заявки вектор суммарных объемов занятых 
ресурсов уменьшается на случайную величину, которая при заданном числе заявок 
в~сис\-те\-ме и~векторе суммарных объемов занятых ресурсов не зависит от поведения 
сис\-те\-мы до упомянутого момента времени.
\end{enumerate}

Тогда процесс $Y^*(t)\hm=(\xi^*(t);\boldsymbol{\delta}^*(t)),$ где 
$\boldsymbol{\delta}^*(t)$~--- это те же,
что и~для исходной сис\-те\-мы, векторы суммарных объемов ресурсов, необходимых для
обслуживания заявок, а~$\xi^*(t)$~--- число заявок в~сис\-те\-ме, является 
полумарковским и~его предельное распределение определяется как
\begin{align*}
q_0^*&=\lim\limits_{t\rightarrow \infty}{\sf P}\left\{\xi^*(t)=0\right\}\,;
\\
Q_k^*(\mathbf{x})&=\lim\limits_{t\rightarrow \infty}{\sf P}\left\{\xi^*(t)=k\,; 
\boldsymbol{\delta}^*(t) \leq \mathbf{x}\right\}, \enskip 1\leq k\leq N\,.
\end{align*}
В результате решения соответствующей сис\-те\-мы урав\-не\-ний 
получаем, что предельные распределе-\linebreak\vspace*{-12pt}

\pagebreak

\noindent
ния 
суммарных объемов занятых ресурсов в~исходной и~упрощенной СМО совпадают: 
$q_0^*\hm=p_0$; $Q_k^*(\mathbf{x})\hm=Q_k(\mathbf{x})$~\cite{Naumov_3_2016}.

В работе~\cite{Sopin_16_2018} рассматривается РСМО с~единственным типом 
пуассоновского входящего потока и~$M$~типами ограниченных ресурсов.
Здесь доказана инвариантность стационарного процесса 
$X(t)=(\xi(t);\boldsymbol{\delta}(t))$ относительно функции распределения для 
упрощенной СМО, т.\,е.\ в~случае произвольного закона распределения времени 
обслуживания~$B(x)$ стационарные вероятности упрощенной СМО~$p^{*}_0$ 
и~$Q^{*}_k(\mathbf{x})$ будут иметь тот же вид, что и~в случае экспоненциального 
распределения времени обслуживания. А стационарные вероятности случайного 
процесса $(\xi(t);\boldsymbol{\delta}(t), \boldsymbol{\beta}(t))$, где 
$\boldsymbol{\beta}(t)\hm=(\beta_1(t),\ldots,\beta_{\xi(t)}(t))$~--- время, 
прошедшее с~момента поступления заявок на обслуживание, примут вид:
\begin{multline*}
Q_k(\mathbf{x},\tau_1,\ldots,\tau_k)={}\\
{}=\lim_{t\rightarrow \infty} {\sf P} \left\{ 
\xi(t)=k;\boldsymbol{\delta}(t)\leq \mathbf{x}; \beta_1(t) \leq 
\tau_1,\ldots\right.\\
\left.\ldots,\beta_k(t) \leq \tau_k  \right\}=
p_0 F^{(k)}(\mathbf{x})\fr{\rho^{k}}{k!}\,, \\
\mathbf{0} \leq \mathbf{x} \leq \mathbf{R}\,, \enskip 1\leq k \leq N\,,
\end{multline*}
где $\rho=\lambda b$, $b\hm=\int\nolimits_{0}^{\infty}t\,dB(t)$.

Для СМО с~$N$ приборами, одним типом дискретного ресурса общим объемом~$R$, 
пуассоновским входящим потоком и~экспоненциальным временем обслуживания 
в~\cite{Sopin_4_2015,Sopin_5_2015} благодаря методу упрощения получены выражения 
для вероятности блокировки сис\-те\-мы, стационарные вероятности простоя системы, 
маргинальные вероятности числа заявок в~сис\-те\-ме~$q_{k,\cdot}$, среднее чис\-ло 
заявок в~сис\-те\-ме, а~также $q_{k,j}$~--- стационарные вероятности того, что 
число заявок в~сис\-те\-ме равно~$k$, а~суммарный объем занятого ресурса равен~$j$:
\begin{multline*}
q_{k,\cdot}={\sf P}\lim_{t\rightarrow \infty}{\sf P}\left\{ \xi(t)=k 
\right\}=p_0\fr{\rho^k}{k!}\sum\limits_{i=0}^{R}p_i^{(k)}\,,
\\ 0<k\leq N\,;
\end{multline*}

\vspace*{-12pt}

\noindent
\begin{multline*}
q_{k,j}={\sf P}\lim\limits_{t\rightarrow \infty}{\sf P} \left\{ \xi(t)=k; 
\delta(t)=j\right\}=p_0\fr{\rho^k}{k!}\,p_j^{(k)}\,,
\\ 0\leq j\leq R\,,\enskip 0<k\leq N\,;
\end{multline*}
где
\begin{equation*}
p_0=\left( 1+\sum\limits_{k=1}^{N}\fr{\rho^k}{k!}\sum_{i=0}^{R}p_i^{(k)} 
\right)^{-1},
\enskip 0<k\leq N\,,
\end{equation*}
а $p^{(k)}_i$~--- $k$-кратная свертка вероятности~$p_i$ с.в.\ 
$\sum\nolimits_{i=1}^{k}r_i$, $r_i$ ~--- с.в.\ объема ресурса, необходимого поступившей 
в~сис\-те\-му $i$-й заявке.

В \cite{Sopin_17_2018} с~помощью метода упрощения была рассмотрена РСМО 
с~марковским входящим потоком типа $\mathrm{MAP}|M|N|0$ и~единственным типом ограниченного 
ресурса объема $R$. Система уравнений равновесия выведена в~векторной форме 
и~решена численно. Наряду со стационарным распределением вероятностей в~статье 
представлены формулы для математического ожидания и~дисперсии числа занятых 
ресурсов, а~также вероятности блокировки. Результаты проиллюстрированы численным 
примером.

\vspace*{-9pt}

\section{Заключение}

В настоящем обзоре кратко представлены основные разновидности РСМО
 с~рекуррентным обслуживанием, существующие методы их 
анализа, выражения для оценки основных ве\-ро\-ят\-но\-ст\-но-вре\-мен\-ных характеристик.

\vspace*{-9pt}

{\small\frenchspacing
 {%\baselineskip=10.8pt
 \addcontentsline{toc}{section}{References}
 \begin{thebibliography}{99}

%1
\bibitem{Ch_1}
\Au{Горбунова~А.\,В., Наумов~В.\,А., Гайдамака~Ю.\,В., Самуйлов~К.\,Е.}
Ресурсные сис\-те\-мы массового обслуживания как модели беспроводных сис\-тем связи~// 
{Информатика и~её применения}, 2018. Т.~12. Вып.~3. С.~48--55.

%2
\bibitem{Kelly}
\Au{Kelly F.\,P.}
Loss networks~// {Ann. Appl. Probab.}, 1991. No.\,1. P.~319--378.

%3
\bibitem{Ross}
\Au{Ross K.\,W.}
{Multiservice loss models for broadband telecommunication networks}. ~--- {London: 
Springer-Verlag}, 1995. 343~p.

%4
\bibitem{Basharin}
\Au{Башарин Г.\,П., Самуйлов~К.\,Е., Яркина~Н.\,В., Гудкова~И.\,А.}
Новый этап развития математической теории телетрафика~// 
Автоматика и телемеханика, 2009.  №\,12. С.~16--28.

 

%5
\bibitem{Romm_21_1971}
\Au{Ромм Э.\,Л., Скитович~В.\,В.}
Об одном обобщении задачи Эрланга~// {Автоматика и~телемеханика}, 1971. №\,6. 
С.~164--168.

%6
\bibitem{Kac}
\Au{Кац Б.\,А.}
Об обслуживании сообщений случайной длины~// {Теория массового обслуживания: 
Труды 3-й Всесоюзн. шко\-лы-со\-ве\-ща\-ния по тео\-рии массового обслуживания}.~--- М.: 
МГУ, 1976. С.~157--168.

%7
\bibitem{Tihonenko_27_1985}
\Au{Тихоненко О.\,М.}
Распределение суммарного объема сообщений в~однолинейной сис\-те\-ме массового 
обслуживания с~групповым поступлением~// {Автоматика и~телемеханика}, 1985. 
№\,11. С.~78--83.

%8
\bibitem{Pechinkin_29_2012}
\Au{Печинкин А.\,В., Соколов~И.\,А., Шоргин~С.\,Я.}
Ограничение на суммарный объем заявок в~дискретной сис\-те\-ме $\mathrm{Geo}|G|1|\infty$~// 
{Информатика и~её применения}, 2012. Т.~6. Вып.~3. С.~107--113.

%9
\bibitem{Naumov_3_2016}
\Au{Наумов В.\,А., Самуйлов К.\,Е., Самуйлов~А.\,К.}
О~суммарном объеме ресурсов, занимаемых обслуживаемыми заявками~// {Автоматика 
и~телемеханика}, 2016. №\,8. С.~105--110.

%10
\bibitem{Mois_4_2017}
\Au{Lisovskaya E., Moiseeva~S., Pagano~M.}
Infinite-server tandem queue with renewal arrivals and random capacity of 
customers~// {Comm. Com. Inf. Sc.}, 2017. 
Vol.~700. P.~201--216.

%11
\bibitem{Mois_5_2017}
\Au{Moiseev A., Moiseeva S., Lisovskaya E.}
Infinite-server queueing tandem with MMPP arrivals and random capacity of 
customers~// {31st European Conference on Modelling and Simulation Proceedings}.~--- 
Budapest, Hungary, 2017. P.~673--679.

%12
\bibitem{Sopin_12_2017}
\Au{Samouylov K., Sopin~E., Vikhrova~O.}
Analysis of queueing system with resources and signals~// {Comm. 
Com. Inf. Sc.}, 2017. Vol.~800. P.~358--369.

%13
\bibitem{Sopin_13_2017}
\Au{Sopin~E., Vikhrova~O., Samouylov~K.}
LTE network model with signals and random resource requirement~// {9th 
 Congress (International) on Ultra Modern Telecommunications and Control Systems 
and Workshops Proceedings}. ~--- Munich, Germany: IEEE, 2017. P.~101--106.



%14
\bibitem{Naumov_15_2017}
\Au{Naumov~V., Samouylov~K.}
Analysis оf multi-resource loss system with state dependent arrival and service 
rates~// {Probab. Eng. Inform. Sc.}, 2017. 
Vol.~31. No.\,4. P.~413--419.

%15
\bibitem{Naumov_14_2017}
\Au{Наумов В.\,А., Самуйлов~К.\,Е.}
Анализ сетей ресурсных сис\-тем массового обслуживания~// {Автоматика 
и~телемеханика}, 2018. №\,5. С.~59--68.

%16
\bibitem{Tihonenko_22_2001}
\Au{Тихоненко О.\,М., Климович~К.\,Г.}
Анализ сис\-тем обслуживания требований случайной длины при ограниченном суммарном 
объеме~// {Проб\-ле\-мы передачи информации}, 2001. Т.~37. Вып.~1. С.~78--88.

%17
\bibitem{Tihonenko_24_1990}
\Au{Позняк Р.\,И., Ревинский~В.\,В., Старовойтов~А.\,М., Тихоненко~О.\,М.}
Определение характеристик суммарного объема требований в~однолинейных сис\-те\-мах 
обслуживания с~ограничениями~// {Автоматика и~телемеханика}, 1990. №\,11. С.~182--186.

%18
\bibitem{Sengupta}
\Au{Sengupta~B.}
The spatial requirement of an $M|G|1$ queue, or: How to design for buffer space~// 
{Modelling and performance evaluation methodology}~/
Eds. F.~Baccelli, G.~Fayolle.~--- Lecture notes in control and information sciences book
ser.~--- Springer, 1984. Vol.~60. P.~545--562.

%19
\bibitem{Tihonenko_40_2002}
\Au{Тихоненко О.\,М.}
Анализ сис\-те\-мы обслуживания неоднородных требований с~дисциплиной разделения 
процессора~// {Известия Национальной академии наук Беларуси. Сер. 
фи\-зи\-ко-ма\-те\-ма\-ти\-че\-ских наук}, 2002. №\,2. С.~105--111.

%20
\bibitem{Tihonenko_25_2010}
\Au{Тихоненко О.\,М.}
Система обслуживания с~разделением процессора и~ограниченными ресурсами~// 
{Автоматика и~телемеханика}, 2010. №\,5. С.~84--98.

%21
\bibitem{Tihonenko_26_2005}
\Au{Тихоненко О.\,М.}
Обобщенная задача Эрланга для сис\-тем обслуживания с~ограниченным суммарным 
объемом~// {Проб\-ле\-мы передачи информации}, 2005. Т.~41. Вып.~3. С.~64--75.



%22
\bibitem{Pechinkin_30_1998}
\Au{Печинкин А.\,В.}
Система $M_i|G|1|n$ с~дисциплиной \mbox{LIFO} и~ограничением на суммарный объем 
требований~// {Автоматика и~телемеханика}, 1998. №\,4. С.~106--116.

%23
\bibitem{Pechinkin_31_1999}
\Au{Печинкин А.\,В.}
Система $\mathrm{MAP}|G|1|n$ с~дисциплиной LIFO с~прерыванием и~ограничением на 
суммарный объем требований~// {Автоматика и~телемеханика}, 1999. №\,12. С.~114--120.

%24
\bibitem{Pechinkin_28_2011}
\Au{Касконе А., Манзо~Р., Печинкин~А.\,В., Шоргин~С.\,Я.}
Сис\-те\-ма $\mathrm{Geo}_m|G|1|n$ с~дисциплиной LIFO без прерывания обслуживания 
и~ограничением на суммарный объем заявок~// {Автоматика и~телемеханика}, 2011. 
№\,1. С.~107--120.

%25
\bibitem{Mois_3_2016}
\Au{Кононов И.\,А., Лисовская~Е.\,Ю.}
Исследование бесконечнолинейной СМО $\mathrm{MAP}|\mathrm{GI}|\infty$ с~заявками случайного объема~// 
{Информационные технологии и~математическое моделирование: 
Мат-лы XV Междунар. конф. имени А.\,Ф.~Терпугова}.~--- Томск: ТГУ, 
2016. Ч.~1. С.~67--71.

%26
\bibitem{Mois_1_2017}
\Au{Лисовская Е. Ю., Моисеева С. П.}
Асимптотический анализ немарковской бесконечнолинейной сис\-те\-мы обслуживания 
требований случайного объема с~входящим рекуррентным потоком // {Вестник 
Томского государственного университета. Управление, вычислительная техника 
и~информатика}, 2017. №~39. С.~30--38.

%27
\bibitem{Mois_2_2017}
\Au{Lisovskaya E., Moiseeva~S., Pagano~M., Potatueva~V.}
Study of the $\mathrm{MMPP}|\mathrm{GI}|\infty$ queueing system with 
random customers' capacities~// 
{Информатика 
и~её применения}, 2017. Т.~11. №\,4. С.~111--119.



%28
\bibitem{Naumov_18_2016}
\Au{Наумов В.\,А., Самуйлов~А.\,К.}
О~связи ресурсных сис\-тем массового обслуживания с~сетями Эрланга~// {Информатика 
и~её применения}, 2016. Т.~10. Вып.~3. С.~9--14.

%29
\bibitem{Naumov_1_2014}
\Au{Наумов В.\,А., Самуйлов~К.\,Е.}
О~моделировании сис\-тем массового обслуживания с~множественными ресурсами~// 
{Вестник РУДН. Сер. Математика, информатика, физика}, 2014. №\,3. С.~60--64.

%30
\bibitem{Naumov_2_2014}
\Au{Naumov V., Samouylov~K., Sopin~E., Andreev~S.}
Two approaches to analyzing dynamic cellular networks with limited resources~// 
{6th  Congress (International) on Ultra Modern Telecommunications and Control 
Systems and Workshops Proceedings}.~--- St.\ Petersburg: IEEE, 
2014. P.~485--488.

%31
\bibitem{Sopin_16_2018}
\Au{Samouylov K., Gaidamaka Yu., Sopin E.}
Simplified analysis of queueing systems with random requirements~// {Statistics and
simulation}~/
Eds.\ J.~Pilz, D.~Rasch, V.\,B.~Melas, K.~Moder.~---
Springer proceedings in mathematics \& statistics book ser.~---
Springer, 2018. Vol.~231. P.~381--390.

%32
\bibitem{Sopin_4_2015}
\Au{Samouylov K., Sopin~E., Vikhrova~O.}
Analyzing blocking probability in LTE wireless network via queuing system with 
finite amount of resources~// {Comm. Com. Inf. 
Sc.}, 2015. Vol.~564. P.~393--403.

%33
\bibitem{Sopin_5_2015}
\Au{Вихрова О.\,Г., Самуйлов~К.\,Е., Сопин~Э.\,С., Шоргин~С.\,Я.}
К анализу показателей качества обслуживания в~современных беспроводных сетях~// 
{Информатика и~её применения}, 2015. Т.~9. Вып.~4. С.~48--55.

%34
\bibitem{Sopin_17_2018}
\Au{Sopin E., Samouylov K.}
On the analysis of the limited resources queuing system under MAP arrivals~// 
{Conference (International) on Applied Mathematics, Computational Science and 
Systems Engineering Proceedings}.~--- Athens, Greece: EDP 
Sciences, 2018. Vol.~16. Art. No.\,01008. 4~p.
 \end{thebibliography}

 }
 }

\end{multicols}

\vspace*{-9pt}

\hfill{\small\textit{Поступила в~редакцию 15.01.19}}

%\vspace*{8pt}

\pagebreak

%\newpage

\vspace*{-28pt}

%\hrule

%\vspace*{2pt}

%\hrule

%\vspace*{-2pt}

\def\tit{RESOURCE QUEUING SYSTEMS\\ WITH~GENERAL SERVICE DISCIPLINE}


\def\titkol{Resource queuing systems with~general service discipline}

\def\aut{A.\,V.~Gorbunova$^1$, V.\,A.~Naumov$^2$, Yu.\,V.~Gaidamaka$^{1,3}$, 
and~K.\,E.~Samouylov$^{1,3}$}

\def\autkol{A.\,V.~Gorbunova, V.\,A.~Naumov, Yu.\,V.~Gaidamaka, 
and~K.\,E.~Samouylov}

\titel{\tit}{\aut}{\autkol}{\titkol}

\vspace*{-11pt}


\noindent
$^1$Peoples' Friendship University of Russia, 
6~Miklukho-Maklaya Str., Moscow 117198, Russian Federation

\noindent
$^2$Service Innovation Research Institute, 
8A~Annankatu, Helsinki 00120, Finland

\noindent
$^3$Institute of Informatics Problems, Federal Research Center 
``Computer Science and Control'' of the Russian\linebreak
$\hphantom{^1}$Academy of Sciences, 
44-2~Vavilov Str., Moscow 119333, Russian Federation

\def\leftfootline{\small{\textbf{\thepage}
\hfill INFORMATIKA I EE PRIMENENIYA~--- INFORMATICS AND
APPLICATIONS\ \ \ 2019\ \ \ volume~13\ \ \ issue\ 1}
}%
 \def\rightfootline{\small{INFORMATIKA I EE PRIMENENIYA~---
INFORMATICS AND APPLICATIONS\ \ \ 2019\ \ \ volume~13\ \ \ issue\ 1
\hfill \textbf{\thepage}}}

\vspace*{6pt}




\Abste{The article gives an overview of resource queuing 
systems with the concentration on the methods of their investigation.
 A~valuable part of the article is devoted to the method, which leads to 
 a~significant simplification of the system analysis while maintaining high 
 accuracy of the estimate, and in some cases without any loss of accuracy. 
 Simplification is to consider a~system with random resource amount release at 
 the instant of a customer departure instead of a~system with the exact resource 
 amount release equal to the occupied by the customer at the beginning of service. 
 Subsequently, for the case of a Poisson flow of arrivals and exponential service 
 time, the equivalence of the results for the initial and the simplified models 
 was rigorously proved. In addition, a~significant part of the paper is devoted 
 to the overview of publications on the recurrent service discipline.}

\KWE{resource queueing systems; continuous resource; discrete resource; 
limited resource; recurrent service; 
heterogeneous network; stationary distribution; semi-Markov process}
 
\DOI{10.14357/19922264190114}

%\vspace*{-14pt}

\Ack
\noindent
The publication was supported by the Ministry of Education and Science of the 
Russian Federation (project No.\,2.882.2017/4.6).



%\vspace*{6pt}

  \begin{multicols}{2}

\renewcommand{\bibname}{\protect\rmfamily References}
%\renewcommand{\bibname}{\large\protect\rm References}

{\small\frenchspacing
 {%\baselineskip=10.8pt
 \addcontentsline{toc}{section}{References}
 \begin{thebibliography}{99}
\bibitem{1-gor-1}
\Aue{Gorbunova, A.\,V., V.\,A.~Naumov, Yu.\,V.~Gaidamaka, and K.\,E.~Samouylov.} 
2018. Resursnye sistemy massovogo obsluzhivaniya kak modeli besprovodnykh 
sistem svyazi [Resource queuing systems as models of wireless communication systems]. 
\textit{Informatika i~ee Primeneniya~--- Inform. Appl.} 12(3):48--55.
\bibitem{2-gor-1}
\Aue{Kelly, F.\,P.}  1991. Loss networks.
\textit{Ann. Appl. Probab.} 1:319--378.
\bibitem{3-gor-1}
\Aue{Ross, K.\,W.} 1995. \textit{Multiservice loss models for broadband telecommunication 
networks}. London: Springer-Verlag. 343~p.
\bibitem{4-gor-1}
\Aue{Basharin, G.\,P., K.\,E.~Samouylov, N.\,V.~Yarkina, and I.\,A.~Gudkova}. 2009. 
A~new stage in mathematical teletraffic theory. 
\textit{Automat. Rem. Contr.} 70(12):1954--1964.
\bibitem{5-gor-1}
\Aue{Romm, E.\,L., and V.\,V.~Skitovitch.} 1971. Ob odnom obobshchenii zadachi Ehrlanga 
[On a~generalization of the Erlang problem]. 
\textit{Automat. Rem. Contr.} 6:164--168.
\bibitem{6-gor-1}
\Aue{Kats, B.\,A.} 1976. Ob obsluzhivanii soobshcheniy sluchaynoy dliny 
[On serving messages of random length]. 
\textit{Teoriya massovogo obsluzhivaniya: Trudy 3-y Vsesoyuzn. shkoly-soveshchaniya 
po teorii massovogo obsluzhivaniya} 
[3rd All-Union School-seminar on Queuing Theory Proceedings]. 157--168.
\bibitem{7-gor-1}
\Aue{Tikhonenko, O.\,M.} 1985. Raspredelenie summarnogo ob''ema soobshcheniy 
v~odnolineynoy sisteme massovogo obsluzhivaniya s~gruppovym postupleniem 
[Distribution of the total meddage flow in a~single-line service system]. 
\textit{Automat. Rem. Contr.} 11:78--83.
\bibitem{8-gor-1}
\Aue{Pechinkin, A.\,V., I.\,A.~Sokolov, and S.\,Ya.~Shorgin.} 
2012. Ogranichenie na summarnyy ob''em zayavok v~diskretnoy sisteme $\mathrm{Geo}|G|1|\infty$ 
[A~restriction on the total volume of demands in the discrete-time system 
$\mathrm{Geo}|G|1|\infty$]. \textit{Informatika i~ee Primeneniya~--- Inform. Appl.} 6(3):107--113.
\bibitem{9-gor-1}
\Aue{Naumov, V.\,A., K.\,E.~Samuilov, and A.\,K.~Samuilov.} 
2016. On the total amount of resources occupied by serviced customers. 
\textit{Automat. Rem. Contr}. 77(8):1419--1427.
\bibitem{10-gor-1}
\Aue{Lisovskaya, E., S.~Moiseeva, and M.~Pagano.} 
2017. Infinite-server tandem queue with renewal arrivals and random capacity 
of customers. \textit{Comm. Com. Inf. Sc.} 700:201--216.
\bibitem{11-gor-1}
\Aue{Moiseev, A., S.~Moiseeva, and E.~Lisovskaya.} 2017. 
Infinite-server queueing tandem with MMPP arrivals and random capacity of customers. 
\textit{31st European Conference on Modelling and Simulation Proceedings}. 
Budapest. 673--679.
\bibitem{12-gor-1}
\Aue{Samouylov, K., E.~Sopin, and O.~Vikhrova.} 2017. Analysis of queueing system 
with resources and signals. 
\textit{Comm. Com. Inf. Sc.} 800:358--369.
\bibitem{13-gor-1}
\Aue{Sopin, E., O. Vikhrova, and K.~Samouylov.} 2017. LTE network model with 
signals and random resource requirement. 
\textit{9th  Congress (International) on Ultra Modern Telecommunications and Control 
Systems and Workshops}. Munich, Germany: IEEE. 101--106.

\bibitem{15-gor-1} %14
\Aue{Naumov, V., and K.~Samouylov.} 2017. Analysis оf multi-resource loss 
system with state dependent arrival and service rates. 
\textit{Probab.  Eng. Inform. Sc.} 31(4):413--419.
\bibitem{14-gor-1} %15
\Aue{Naumov, V.\,A., and K.\,E.~Samuilov.} 2018. 
Analysis of networks of the resource queuing systems. 
\textit{Automat. Rem. Contr.} 79(5):822--829.

\bibitem{16-gor-1}
\Aue{Tikhonenko, O.\,M., and K.\,G.~Klimovich.} 2001. Ana\-liz sistem obsluzhivaniya 
trebovaniy sluchaynoy dliny pri ogranichennom summarnom ob''eme 
[Analysis of queuing systems for random-length arrivals with limited cumulative volume].
\textit{Probl. Inform. Transm.} 37(1):78--88.
\bibitem{17-gor-1}
\Aue{Poznyak, R.\,I., V.\,V.~Revinskiy, A.\,M.~Starovoytov, and O.\,M.~Tikhonenko.} 
1990. Opredelenie kharakteristik summarnogo ob''ema trebovaniy 
v~odnolineynykh sistemakh obsluzhivaniya s~ogranicheniyami [Calculation 
of characteristics of the total amount of customers in single-server 
queueing systems with constraints]. \textit{Automat. Rem. Contr.} 11:182--186.
\bibitem{18-gor-1}
\Aue{Sengupta, B.} 1984. The spatial requirement of an $M|G|1$ queue, or: 
How to design for buffer space. 
\textit{Modelling and performance evaluation methodology}.
Eds.\ F.~Baccelli and G.~Fayolle. Lecture notes in control
and information sciences book ser. Springer. 60:545--562.
\bibitem{19-gor-1}
\Aue{Tikhonenko, O.\,M.} 2002. Analiz sistemy obsluzhivaniya neodnorodnykh 
trebovaniy s~distsiplinoy razdeleniya pro\-tses\-so\-ra [Analysis of the queueing 
system with heterogeneous customers and processor sharing discipline]. 
\textit{Izvestiya Natsional'noy akademii nauk Belarusi. 
Ser. fiziko-matematicheskikh nauk} [Proceedings of the National Academy 
of Sciences of Belarus. Physics and Mathematics Ser.] 2:105--111.
\bibitem{20-gor-1}
\Aue{Tikhonenko, O.\,M.} 2010. 
Queuing system with processor sharing and limited resources. 
\textit{Automat. Rem. Contr.} 71(5):803--815.
\bibitem{21-gor-1}
\Aue{Tikhonenko, O.\,M.} 2005. Obobshchennaya zadacha Ehrlanga dlya 
sistem obsluzhivaniya s~ogranichennym summarnym ob''emom 
[Generalized Erlang problem for queueing systems with bounded total size]. 
\textit{Probl. Inform. Transm.}
 41(3):64--75.

\bibitem{23-gor-1} %22
\Aue{Pechinkin, A.\,V.} 1998. Sistema $M_i|G|1|n$ s~distsiplinoy LIFO 
i~ogranicheniem na summarnyy ob''em trebovaniy [$M_i|G|1|n$ 
system with LIFO discipline and constrained total amount of items]. 
\textit{Automat. Rem. Contr.} 4:106--116.
\bibitem{24-gor-1} %23
\Aue{Pechinkin, A.\,V.} 1999. Sistema $\mathrm{MAP}|G|1|n$ 
s~dis\-tsip\-li\-noy LIFO s~preryvaniem i~ogranicheniem na summarnyy ob''em trebovaniy 
[The $\mathrm{MAP}|G|1|n$ system with LIFO service discipline with 
interruptions and limitations on the total amount of requests]. 
\textit{Automat. Rem. Contr.} 12:114--120.
\bibitem{22-gor-1} %24
\Aue{Cascone, A., R.~Manzo, A.\,V.~Pechinkin, and S.\,Ya.~Shorgin.} 
2011. $\mathrm{Geo}_m|G|1|n$ 
system with LIFO discipline without interrupts and constrained total amount of 
customers. \textit{Automat. Rem. Contr.} 72(1):99--110.
\bibitem{27-gor-1} %25
\Aue{Kononov, I.\,A., and E.\,Yu.~Lisovskaya.} 
2016. Issledovanie beskonechnolineynoy SMO $\mathrm{MAP}|\mathrm{GI}|\infty$ 
s~zayavkami sluchaynogo ob''ema [Study of the infinite server queue 
$\mathrm{MAP}|\mathrm{GI}|\infty$ with customers of random volume]. 
\textit{15th Conference (International)
named after A.\,F.~Terpugov ``Information Technologies and Mathematical Modelling'' 
Proceedings}. Tomsk: Tomsk State University.  67--71.
\bibitem{25-gor-1} %26
\Aue{Lisovskaya, E.\,Yu., and S.\,P.~Moiseeva.} 2017. Asimptoticheskiy analiz 
nemarkovskoy beskonechnolineynoy sistemy obsluzhivaniya trebovaniy sluchaynogo 
ob''ema s~vkhodyashchim rekurrentnym potokom [Asymptotical analysis of 
a~non-Markovian queueing system with renewal input process and random capacity 
of customers]. \textit{Vestnik Tomskogo gosudarstvennogo universiteta. 
Upravlenie, vychislitel'naya tekhnika i~informatika} 
[Tomsk State University J.~Control Computer Science] 39:30--38.
{\looseness=1

}

\bibitem{26-gor-1} %27
\Aue{Lisovskaya, E., S.~Moiseeva, M.~Pagano, and V.~Potatueva.} 
2017. Study of the $\mathrm{MMPP}|\mathrm{GI}|\infty$ queueing system with random customers' capacities. 
\textit{Informatika i~ee Primeneniya~--- Inform. Appl.} 11(4):111--119.

\bibitem{28-gor-1}
\Aue{Naumov, V.\,A., and K.\,E.~Samouylov.} 2016. 
O~svyazi resursnykh sistem massovogo obsluzhivaniya s~setyami Ehrlanga 
[On relationship between queuing systems with resources and Erlang networks]. 
\textit{Informatika i~ee Primeneniya~--- Inform. Appl.} 10(3):9--14.
\bibitem{29-gor-1}
\Aue{Naumov, V.\,A., and K.\,E.~Samouylov.} 2014. 
O~mo\-de\-li\-ro\-va\-nii sistem massovogo obsluzhivaniya s~mno\-zhest\-ven\-ny\-mi 
resursami 
[On the modeling of queueing systems with multiple resources]. 
\textit{Vestnik RUDN. Ser. Matematika, informatika, fizika} [RUDN~J.~Mathematics 
Information Sciences Physics] 3:60--64.
\bibitem{30-gor-1}
\Aue{Naumov, V., K.~Samouylov, E.~Sopin, and S.~Andreev.} 2014. Two approaches 
to analyzing dynamic cellular networks with limited resources. 
\textit{6th  Congress (International) on Ultra Modern Telecommunications 
and Control Systems and Workshops (ICUMT) Proceedings}. St.\ Petersburg. 485--488.
\bibitem{31-gor-1}
\Aue{Samouylov, K., Yu.~Gaidamaka, and E.~Sopin.} 2018. 
Simplified analysis of queueing systems with random requirements. 
\textit{Statistics and simulation}. Eds.\ J.~Pilz, D.~Rasch, V.\,B.~Melas, and K.~Moder.
Springer proceedings in mathematics \& statistics book ser.
Springer. 231:381--390.
\bibitem{32-gor-1}
\Aue{Samouylov, K., E.~Sopin, and O.~Vikhrova.} 2015. Analyzing blocking probability 
in LTE wireless network via queuing system with finite amount of resources. 
\textit{Comm. Com. Inf. Sc.} 564:393--403.
\bibitem{33-gor-1}
\Aue{Vikhrova, O.\,G., K.\,E.~Samouylov, E.\,S.~Sopin, and S.\,Ya.~Shorgin.} 
2015. K~analizu pokazateley kachestva obsluzhivaniya 
v~sovremennykh besprovodnykh setyakh [On performance analysis of modern 
wireless networks]. \textit{Informatika i~ee Primeneniya~--- Inform. Appl.} 9(4):48--55.
{\looseness=1

}
\bibitem{34-gor-1}
\Aue{Sopin, E., and K.~Samouylov.} 2018. On the analysis of the limited resources 
queuing system under MAP arrivals. \textit{Conference (International)
on Applied Mathematics, Computational Science and Systems Engineering  
Proceedings}. 16:01008. 4~p.
\end{thebibliography}

 }
 }

\end{multicols}

\vspace*{-6pt}

\hfill{\small\textit{Received January 15, 2019}}

%\pagebreak

%\vspace*{-18pt}

\Contr


\noindent
\textbf{Gorbunova Anastasiya V.} (b.\ 1986)~--- 
Candidate of Science (PhD) in physics and mathematics, 
senior lecturer, Peoples' Friendship University of Russia, 
6~Miklukho-Maklaya Str., Moscow 117198, Russian Federation; 
\mbox{gorbunova\_av@rudn.university}

\vspace*{3pt}


\noindent
\textbf{Naumov Valeriy A.} (b.\ 1950)~--- Candidate of Science (PhD) 
in physics and mathematics, research director, 
Service Innovation Research Institute, 
8A~Annankatu, Helsinki 00120, Finland; \mbox{valeriy.naumov@pfu.fi}

\vspace*{3pt}

\noindent
\textbf{Gaidamaka Yuliya V.} (b.\ 1971)~--- Doctor of Science in physics and mathematics, 
professor, Peoples' Friendship University of Russia, 6~Miklukho-Maklaya Str., 
Moscow 117198, Russian Federation; senior scientist, 
Institute of Informatics Problems, Federal Research Center 
``Computer Science and Control'' of the Russian Academy of Sciences, 
44-2~Vavilov Str., Moscow 119333, Russian Federation; 
\mbox{gaydamaka-yuv@rudn.university}

\vspace*{3pt}

\noindent
\textbf{Samouylov Konstantin E.} (b.\ 1955)~--- 
Doctor of Science in technology, professor, Head of Department, 
Peoples' Friendship University of Russia, 6~Miklukho-Maklaya Str., 
Moscow 117198, Russian Federation; senior scientist, 
Institute of Informatics Problems, Federal Research Center 
``Computer Science and Control'' of the Russian Academy of Sciences, 
44-2~Vavilov Str., Moscow 119333, Russian Federation; 
\mbox{samuylov-ke@rudn.university}
\label{end\stat}

\renewcommand{\bibname}{\protect\rm Литература}       