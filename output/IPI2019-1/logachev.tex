\newcommand{\Tsf}{^{\mathsf T}}
\newcommand{\rank}{\mathrm{rank}\,}

\def\stat{logachev}

\def\tit{ПОЛИНОМИАЛЬНЫЕ АЛГОРИТМЫ ВЫЧИСЛЕНИЯ ЛОКАЛЬНЫХ АФФИННОСТЕЙ КВАДРАТИЧНЫХ 
БУЛЕВЫХ ФУНКЦИЙ$^*$}

\def\titkol{Полиномиальные алгоритмы вычисления локальных аффинностей квадратичных 
булевых функций}

\def\aut{О.\,А.~Логачев$^1$, А.\,А.~Сукаев$^2$, С.\,Н.~Федоров$^3$}

\def\autkol{О.\,А.~Логачев, А.\,А.~Сукаев, С.\,Н.~Федоров}

\titel{\tit}{\aut}{\autkol}{\titkol}

\index{Логачев О.\,А.}
\index{Сукаев А.\,А.}
\index{Федоров С.\,Н.}
\index{Logachev O.\,A.}
\index{Sukayev A.\,A.}
\index{Fedorov S.\,N.}


{\renewcommand{\thefootnote}{\fnsymbol{footnote}} \footnotetext[1]
{Работа выполнена при частичной поддержке РФФИ (проект 18-29-03124~мк).}}


\renewcommand{\thefootnote}{\arabic{footnote}}
\footnotetext[1]{Московский государственный университет им.\
М.\,В.~Ломоносова; Институт проб\-лем информатики Федерального исследовательского
центра <<Информатика и~управ\-ле\-ние>> Российской академии наук, \mbox{logol@iisi.msu.ru}}
\footnotetext[2]{Московский государственный университет им.\ 
М.\,В.~Ломоносова, \mbox{asukaev@gmail.com}}
\footnotetext[3]{Московский государственный университет им.\
М.\,В.~Ломоносова, \mbox{s.n.feodorov@yandex.ru}}

\vspace*{-12pt}

 
\Abst{Аффинная нормальная форма позволяет рассматривать произвольную булеву функцию на 
определенных плоскостях (так называемых локальных аффинностях) как аффинную. Данное 
пред\-став\-ле\-ние~--- по сути, аффинная аппроксимация~--- булевых функций может 
помочь в~решении систем нелинейных уравнений над полем из двух элементов. Задача 
решения таких систем (специального вида), среди прочего, используется в~ряде 
методов синтеза и~анализа средств обеспечения информационной безопасности.
В~статье описывается способ нахождения локальных аффинностей для квадратичных 
булевых функций, основанный на теореме Диксона. Тем самым решается задача 
построения аффинных нормальных форм для таких функций. Кроме того, обсуждаются 
вопросы эффективности подобных алгоритмов.
Основная цель данной статьи~--- подготовить базу для готовящейся к~публикации 
работы, предлагающей метод решения систем квадратичных булевых уравнений 
с~помощью <<аппроксимирования>> соответствующих функций их аффинными нормальными 
формами.}


\KW{булева функция; система квадратичных булевых уравнений; 
разбиение векторного пространства; плоскость; локальная аффинность; теорема 
Диксона; аффинная нормальная форма; алгебраический криптоанализ}

\DOI{10.14357/19922264190110}
  
\vspace*{-1pt}


\vskip 10pt plus 9pt minus 6pt

\thispagestyle{headings}

\begin{multicols}{2}

\label{st\stat}


\section{Введение}

\vspace*{-2pt}

Центральная идея алгебраического криптоанализа состоит в~том, чтобы описать 
используемые в~анализируемой криптосхеме преобразования сис\-те\-мой алгебраических 
уравнений (с~некоторой сек\-рет\-ной информацией в~качестве неизвестных) над\linebreak 
конечным полем и~затем решить эту систему.
В~данной статье рассматриваются только булевы системы уравнений, хотя часть 
пред\-став\-лен\-ных здесь результа\-тов может иметь место и~для сис\-тем ал\-геб\-ра\-и\-че\-ских 
уравнений над произвольными конечными полями.

Из теории сложности вычислений известно, что вычислительная задача определения 
совместности систем нелинейных булевых уравнений является NP-пол\-ной~\cite{GJ1982, GT2017}, 
а~вычислительная задача решения систем нелинейных булевых 
уравнений является NP-труд\-ной~\cite{GJ1982,GT2017}.
Однако в~специальных случаях эти задачи могут решаться эффективно (см., 
например,~\cite{GT2017,Smi2000}).

Кроме того, существуют полиномиальные алгоритмы построения по произвольной 
системе уравнений системы с~фиксированной алгебраической 
степенью~\cite[\S\;11.4.2]{Bard2009}, что позволяет, в~частности, ограничиться 
рассмотрением только квадратичных систем уравнений.

Можно выделить несколько основных классов методов, используемых в~криптоанализе 
для решения (или оценки трудоемкости решения) систем полиномиальных булевых 
уравнений:
использование базисов Грёбнера~\cite[section~12.2]{Bard2009}, применение 
программных систем поиска выполняющего набора булевой формулы 
(SAT-solvers)~\cite{BCJ2007}, вероятностные и~тео\-ре\-ти\-ко-ко\-до\-вые 
методы~\cite{LSSYa2015}, а~также  методы линеаризации~\cite[section~12.3]{Bard2009}.
Основная идея методов линеаризации состоит в~применении <<линейных>> методов 
к~нелинейным системам, т.\,е.\ в~построении сис\-тем линейных уравнений, решение 
которых дает возможность найти решение исходной нелинейной системы.

Важным параметром метода линеаризации служит число переменных в~синтезируемых 
линейных системах уравнений. Как правило, речь идет об увеличении (не~более чем 
полиномиальном) количества переменных.
Метод, основанный на рассмотренных в~данной работе идеях, по своей сути, 
осуществляет линеаризацию, но при этом он остав\-ля\-ет число переменных неизменным.

Этот метод решения квадратичных систем булевых уравнений использует локальные 
аффинности уравнений системы и~состоит из двух этапов.
Первый этап (предварительный) содержательно представляет собой описание семейств 
локальных аффинностей уравнений.
Второй этап метода заклю\-ча\-ет\-ся собственно в~решении исходной сис\-те\-мы посредством 
анализа сис\-тем линейных уравнений, полученных с~помощью этих локальных 
аффинностей.

Настоящая работа (в~силу ограниченности\linebreak объема публикации) посвящена 
исследованию первого этапа и,~в~частности, вопросам его эффективности. 
Результаты исследований с~оценкой эф\-фек\-тив\-ности и~описанием параметров второго
\mbox{этапа} предлагаемого метода предполагается опуб\-ли\-ко\-вать в~одном из сле\-ду\-ющих 
выпусков журнала.

\vspace*{-4pt}

\section{Необходимые понятия и~обозначения}

В данной работе булев куб $\{0,1\}^n$ отождествляется с~$n$-мерным векторным 
пространством~$V_n$ над полем из двух элементов~$\mathbb{F}_2$.
Векторы из $V_n$ будет удобнее записывать \textit{строками} длины~$n$. Значок~$\Tsf$ 
используется для операции транспонирования матриц.
Всюду далее~$x$ обозначает вектор $(x_1,x_2,\ldots,x_n)$.

Знак $\oplus$ будет использоваться для записи суммы по модулю~$2$ булевых 
переменных и~операций сложения в~$\mathbb{F}_2$ и~покомпонентного сложения 
в~$V_n$.

Множество всех невырожденных аффинных преобразований (отображений в~себя) 
пространства~$V_n$ обозначается через $\mathrm{GA}(V_n)$. В~матричном 
представлении действие элемента~$\alpha\in\mathrm{GA}(V_n)$ на векторах 
пространства имеет вид $\alpha(x)\hm=xA\oplus b$, где $x$~пробегает~$V_n$; $A$~--- 
невырожденная $(n\times n)$-мат\-ри\-ца над~$\mathbb{F}_2$; $b\hm\in V_n$.

Множество всех булевых функций от $n$~переменных обозначим через
$$
\mathcal{F}_n=\{f\colon V_n\to \mathbb{F}_2\}\,.
$$
Как известно, произвольную булеву функцию~$f$ от переменных $x_1,\ldots,x_n$ 
можно представить (единственным образом) в~виде полинома Жегалкина:
$$
f(x)=\bigoplus_{\varepsilon\in\{0,1\}^n} a_{\varepsilon}x^{\varepsilon}\,,
$$
где %\label{Zhegalkin}
$\varepsilon\hm=(\varepsilon_1,\ldots,\varepsilon_n)$,
$a_{\varepsilon}\hm\in\mathbb{F}_2$ и~$x^{\varepsilon}\hm=x_1^{\varepsilon_1}\cdots x_n^{\varepsilon_n}$ (считаем, 
$x_i^0\hm=1$, $x_i^1\hm=x_i$).
Далее под булевой функцией будет, как правило, подразумеваться ее запись в~виде 
полинома.

Если $\varphi$~--- некоторое преобразование пространства~$V_n$, то его действие на 
функцию~$f\hm\in\mathcal{F}_n$ будем определять и~обозначать так: 
$f^{\varphi}(x)\hm=f(\varphi(x))$.
В~частности, в~случае аффинных преобразований пространства будет рассматриваться 
множество $\mathrm{Orb}_f(\mathrm{GA}(V_n))\hm=\{f^{\varphi}\mid 
\varphi\hm\in\mathrm{GA}(V_n)\}$~--- орбита функции~$f$ относительно действия 
группы~$\mathrm{GA}(V_n)$.
Имея в~виду, что произведение~$\alpha_1\alpha_2$ элементов из $\mathrm{GA}(V_n)$ 
есть композиция $\alpha_1\circ\alpha_2(x)\hm=\alpha_1(\alpha_2(x))$, заметим, что 
действие~$\alpha_1\alpha_2$ на произвольную функцию $f\hm\in\mathcal{F}_n$ 
корректно определять следующим образом:
$$
f^{\alpha_1\alpha_2}(x)=\left(f^{\alpha_1}\right)^{\alpha_2}(x)=f^{\alpha_1}
\left(\alpha_2(x)\right)
=f\left(\alpha_1\alpha_2(x)\right),
$$
поскольку~$\alpha_i$ действуют на булеву функцию преобразованием \textit{ее 
аргумента}.

%Когда мы делаем невырожденную аффинную замену переменных $x'=\alpha(x)=xA\oplus 
%b$, функция~$f(x)$, при подставлении в~нее выражений старых переменных через 
%новые, преобразуется к~виду $f^{\alpha^{-1}}(x')$.


\textit{Алгебраической степенью} булевой функции~$f$ от $n$~переменных называют 
величину

\noindent
$$
\deg f = \max\left\{\sum\limits_{i=1}^n \varepsilon_i\mid a_{\varepsilon}=1\right\}
$$
(суммирование~--- в~$\mathbb{Z}$), т.\,е.\ максимальное число различных 
переменных в~мономах данного представления.

В множестве~$\mathcal{F}_n$ всех булевых функций от~$n$~переменных выделим 
подмножество

\noindent
$$
\mathcal{A}_n=\left\{f\in\mathcal{F}_n\mid \deg f\leqslant 1\right\}.
$$
Составляющие это подмножество функции называются линейными (в~математической 
логике и~кибернетике) или аф\-фин\-но-ли\-ней\-ны\-ми (в~ал\-геб\-ре), однако по сложившейся 
в~криптологии традиции в~данной работе они называются \textit{аффинными}, т.\,е.\ 
понимаются как частный случай аффинного \textit{отоб\-ра\-же\-ния} $n$-мер\-но\-го 
пространства в~одномерное.


Булеву функцию~$f$ c $\deg f\hm\leqslant 2$ будем называть 
\textit{квадратичной}\footnote{В~алгебре такие функции называют 
аффинно-квад\-ра\-тич\-ны\-ми. Квадратичными при этом называют функции, представляемые 
\textit{однородными} полиномами второй степени.}.
По определению квадратичная функция~$f\hm\in\mathcal{F}_n$ (ее полином Жегалкина) 
имеет вид:

\noindent
$$
f(x)= \bigoplus_{1\leqslant i<j\leqslant n} q_{ij}x_ix_j\oplus
\bigoplus_{1\leqslant k\leqslant n} 
l_kx_k \oplus c\,,
$$
где $q_{ij},l_k,c\in\mathbb{F}_2$.

В настоящей работе рассматриваются системы уравнений

\noindent
\begin{equation}
\left.
\begin{array}{c}
        f_1(x_1,\ldots,x_n)=0\,;\\
        f_2(x_1,\ldots,x_n)=0\,;\\
        \vdots\\
        f_m(x_1,\ldots,x_n)=0\\
    \end{array}
    \right\}
    \label{system}
\end{equation}
с квадратичными булевыми функциями~$f_i$, $1\hm\leqslant i\hm\leqslant m$, и~$m\hm>n$.
%Мы предполагаем, что все рассматриваемые нами системы квадратичных уравнений 
%имеют единственное решение.

\pagebreak

В матричном виде квадратичная функция записывается следующим образом:
$$
f(x)=x Q_f x\Tsf\oplus l_f x\Tsf \oplus c\,,
$$
где $Q_f$~--- верхнетреугольная $(n\times n)$-мат\-ри\-ца с~нулевой главной 
диагональю; $l_f\in\mathbb{F}_2^n$; $c\in\mathbb{F}_2$.
Рас\-смат\-ри\-ва\-ют также симметричную матрицу
$$
\tilde{Q}_f=Q_f\oplus Q_f\Tsf\,.
$$
Она определяет билинейную форму
$$
q_f(u,v)=u\tilde{Q}_f v\Tsf=f(u\oplus v)\oplus f(u) \oplus f(v)\oplus c\,,
$$
называемую \textit{ассоциированной с~квадратичной функцией~$f$}.

Булева билинейная форма $q(u,v)$, $u,v\hm\in V_n$, удовле\-тво\-ря\-ющая условиям
$$
q(u,u)=0\,;\qquad q(u,v)=q(v,u)\,,
$$
называется \textit{симплектической}.
Такие билинейные формы находятся во взаимно однозначном соответствии с~булевыми 
симметричными матрицами с~нулевой главной диагональю, называемыми 
\textit{симплектическими матрицами}.

Таким образом, для произвольной квадратичной булевой функции~$f$ 
матрица~$\tilde{Q}_f$~--- сим\-плектическая. Также очевидно, что билинейная\linebreak 
форма~$q_f(u,v)$, ассоциированная с~$f$, является симплектической.

\smallskip

\noindent
\textbf{Предложение~1}\
[6, лемма~3.3.1; 7, \S\;15.2, лемма~3].
\textit{Ранг симплектической матрицы четен.}


\smallskip

\textit{Плоскость}~$\pi$ в~$V_n$~--- это множество вида $v\hm+L$, где~$v$ и~$L$~--- 
соответственно вектор и~подпространство пространства~$V_n$. Другими словами, 
плоскость~--- аффинное подпространство в~$V_n$.
\textit{Размерность плоскости} совпадает с~размерностью соответствующего 
подпространства: $\mathrm{dim}\,\pi=\mathrm{dim}\,L$.
Как известно, любая плоскость является решением некоторой системы линейных 
уравнений, и~на\-обо\-рот: решение произвольной системы линейных уравнений~--- 
плоскость в~соответствующем пространстве.
%То есть, плоскость является линейным многообразием.

Сужение функции $f\hm\in\mathcal{F}_n$ на плоскость~$\pi$ будем обозначать 
через~$f|_{\pi}$. Таким образом, $f|_{\pi}\colon \pi\hm\to\mathbb{F}_2$ 
и~$f|_{\pi}(u)\hm=f(u)$ для всех $u\hm\in\pi$.



\section{Локальная аффинность и~аффинная нормальная форма~булевой~функции}

В этом разделе вводятся понятия, связанные с~представлением произвольной булевой 
функции совокупностью аффинных функций, заданных для определенных плоскостей 
в~векторном пространстве. Более общее изложение этой теории можно найти 
в~работе~\cite{LYaD2007}.

\textit{Локальной аффинностью} функции~$f\hm\in\mathcal{F}_n$ будем называть такую 
плоскость~$\pi$, что $f|_{\pi}$ можно продолжить до аффинной функции, т.\,е.\ 
существует $l\hm\in\mathcal{A}_n$ со свойством $f|_{\pi}\hm=l|_{\pi}$.
Очевидно, для любой булевой функции существует разбиение пространства~$V_n$ на 
ее локальные аффинности.

Возьмем произвольное разбиение $\Pi\hm=\{\pi_1,\ldots,\pi_{\lambda}\}$ 
пространства~$V_n$ на плоскости, являющиеся локальными аффинностями булевой 
функции~$f$ от $n$~переменных.
Будем называть \textit{аффинной нормальной формой} функции~$f$ выражение вида
\begin{equation}
\label{AffNF}
f(x)=\bigoplus_{j=1}^{\lambda}\chi_{\pi_j}(x) l_j(x)\,,
\end{equation}
где для каждого~$j$, $1\hm\leqslant j\hm\leqslant\lambda$, функция~$l_j$ аффинна 
и~$f|_{\pi_j}(x)\hm=l_j|_{\pi_j}(x)$, а~$\chi_{\pi_j}$~--- характеристическая 
функция (индикатор) множества~$\pi_j$.
Функции~$l_j$ из этого выражения для краткости назовем\linebreak
 \textit{аффинными 
аппроксимациями} функции~$f$.
\textit{Длиной аффинной нормальной формы} называется число плоскостей 
в~разбиении~$\Pi$, далее она будет обозначаться через~$\lambda(\Pi)$.

\smallskip

\noindent
\textbf{Замечание~1.}
  Характеристические функции плоскостей известны также под именем 
<<мультиаффинных функций>> \cite{GT2017}, играющих важную роль при описании 
классов эффективно решаемых систем булевых уравнений.


\smallskip

Характеристическая функция плоскости в~пространстве~$V_n$ имеет вполне 
определенный вид. Любая плоскость~$\pi$, как уже отмечалось, может быть задана 
как множество решений системы $d$~линейных уравнений (для некоторого~$d$):
\begin{equation}
\left.
\begin{array}{c}
        h_1(x_1,\ldots,x_n)=0\,;\\
        h_2(x_1,\ldots,x_n)=0\,;\\
        \vdots\\
        h_d(x_1,\ldots,x_n)=0\,,\\
    \end{array}
    \right\}
    \label{chi-system}
\end{equation}
где все $h_i(x)\hm\in\mathcal{A}_n$. Поскольку вектор~$x$ принадлежит 
плоскости~$\pi$ тогда и~только тогда, когда все~$h_i$, $1\hm\leqslant i\hm\leqslant d$, 
обращаются в~нуль на нем, характеристическая функция~$\pi$ выражается следующим 
образом:
$$
\chi_{\pi}(x)=\prod\limits_{i=1}^d (h_i(x)\oplus 1)\,.
$$
Если система линейных уравнений задана в~мат\-рич\-ной форме: $xH\oplus 
(b_1,\ldots,b_d)\hm=0$, $b_i\hm=h_i(0)$, то выражение будет иметь вид:

\noindent
$$
\chi_{\pi}(x)=\prod\limits_{i=1}^d (xH_i\oplus b_i\oplus 1)\,,
$$
где $H_i$~--- столбцы матрицы~$H$.

Как видно из определения, аффинная нормальная форма представляет собой 
в~некотором смыс\-ле ку\-соч\-но-аф\-фин\-ную аппроксимацию булевой функции. На каждой 
локальной аф\-фин\-ности~$\pi_j$ из разбиения $\Pi$ все, кроме одного, слагаемые 
в~выражении~\eqref{AffNF} обращаются в~нуль, и~функция принимает вид 
$f(x)\hm=\chi_{\pi_j}(x)l_j(x)\hm=l_j(x)$ для всех $x\hm\in\pi_j$.

Возможность заменить на плоскости~$\pi_j$ квадратичное уравнение $f(x)\hm=0$ 
линейным уравнением $l_j(x)\hm=0$ вместе с~дописанной к~нему системой~\eqref{chi-system} 
будет использоваться при решении систем полиномиальных уравнений 
в~следующей статье.
Как сказано в~замечании~1, функции~$\chi_{\pi}(x)$, а~также 
и~слагаемые в~аффинной нормальной форме~\eqref{AffNF} являются мультиаффинными 
функциями. Тео\-ре\-ти\-ко-слож\-ност\-ные вопросы, связанные, в~частности, с~решением 
систем мультиаффинных уравнений, а~также оценки числа таких функций 
рассматриваются в~работе~\cite{Gor1995}.

При аффинном преобразовании пространства аффинные нормальные формы сохраняются 
в~том смысле, что выражение, полученное после применения преобразования к~этой 
форме, тоже будет аффинной нормальной формой для некоторой функции.

\smallskip

\noindent
\textbf{Предложение~2.}
\textit{Пусть $\varphi\in\mathrm{GA}(V_n)$ и~$f(x)\hm=
\bigoplus_{j=1}^{\lambda(\Pi)}\chi_{\pi_j}(x) l_j(x)$~--- некоторая 
аффинная нормальная форма функции~$f$.
  Тогда} 
$$
f^{\varphi}(x)=f(\varphi(x))=\bigoplus_{j=1}^{\lambda(\Pi)}\chi_{\pi_j}(\varphi(x)) 
l_j(\varphi(x))$$ 
\textit{есть аффинная нормальная форма функции~$f^{\varphi}$}.


\smallskip

\noindent
Д\,о\,к\,а\,з\,а\,т\,е\,л\,ь\,с\,т\,в\,о\,.\ \ 
Множество $\Pi'\hm=\{\pi'_j\hm=\varphi^{-1}(\pi_j) \mid \pi_j\in\Pi\}$ является 
разбиением пространства~$V_n$ на $\lambda(\Pi)$ плоскостей, поскольку~$\varphi$~--- 
не\-вы\-рож\-ден\-ное аффинное преобразование.
Заметим,\linebreak
 что $\varphi(x)\in\pi_j$ тогда и~только тогда, когда $x\hm\in\varphi^{-1}
(\pi_j)\hm=\pi'_j$.
Поэтому $\chi^{\varphi}_{\pi_j}$~--- характеристическая функция 
плоскости~$\pi'_j$.
Выражение для~$f^{\varphi}$ в~новых обозначениях выглядит следующим образом:
$$
f^{\varphi}(x)=\bigoplus_{j=1}^{\lambda(\Pi')}\chi_{\pi'_j}(x) l_j^{\varphi}(x)\,.
$$
Так как, очевидно, функции $l_j^{\varphi}(x)\hm=l_j(\varphi(x))$ аффинны, полученное 
выражение представляет собой аффинную нормальную форму.


\section{Теорема Диксона и~приведение квадратичных функций к~каноническому 
виду}\label{Dickson}

Благодаря теореме Диксона можно для любой квадратичной булевой функции~$f$ найти 
ее каноническое представление, в~котором она выглядит наиболее просто. Как будет 
видно ниже, это представление~--- элемент из орбиты данной функции 
$\mathrm{Orb}_f(\mathrm{GA}(V_n))$.
Канонический вид квадратичной функции, в~свою очередь, подсказывает прос\-той 
способ построения ее аффинной нормальной \mbox{формы.}

\smallskip

\noindent
\textbf{Теорема~1}\ [10, \S\;199].
\textit{Для любой квад\-ра\-тич\-ной функции~$f\hm\in\mathcal{F}_n$ с~ненулевой 
матрицей~$\tilde{Q}_f$ существует аффинное 
преобразование~$\alpha\hm\in\mathrm{GA}(V_n)$, которое приводит~$f$ к~одному из  
канонических представлений}:
$$f^{\alpha}(x)=x_1x_2\oplus x_3x_4\oplus\cdots\oplus x_{2r-1}x_{2r}\oplus c
$$
\textit{или}
$$f^{\alpha}(x)=x_1x_2\oplus x_3x_4\oplus\cdots\oplus x_{2r-1}x_{2r}\oplus 
x_{2r+1}\,,
$$
где $2r=\rank \tilde{Q}_f$ и~$c\hm\in\mathbb{F}_2$.

\smallskip

Доказательство этого утверждения помимо авторского варианта можно найти также 
в~[6, \S\;3.3; 7, \S\;15.2].

На практике приведение полинома Жегалкина квадратичной булевой функции 
к~каноническому виду можно осуществить следующим способом.

Предположим, не ограничивая общности, что в~полиноме Жегалкина функции~$f$ 
присутствует моном~$x_1x_2$ (иначе с~помощью аффинного преобразования координат 
<<перенумеруем>> переменные).
Представим функцию в~виде:
\begin{multline*}
f(x)=x_1x_2\oplus x_1l_1(x_3,\ldots,x_n) \oplus x_2l_2(x_3,\ldots,x_n) \oplus{}\\
{}\oplus 
q_1(x_3,\ldots,x_n)\,,
\end{multline*}
где $l_1,l_2\in\mathcal{A}_{n-2}$, а~$q_1$~--- некоторая квадратичная функция.
Возьмем отображение~$\varphi_2$ пространства~$V_n$, задаваемое равенством:
\begin{multline*}
\varphi_2(x)=\left(x_1\oplus l_2(x_3,\ldots,x_n),\ x_2\oplus {}\right.\\
\left.{}\oplus
l_1(x_3,\ldots,x_n),\ x_3,\ldots,\ x_n\right)\,,
\end{multline*}
и рассмотрим следующую функцию:
$$
f^{(2)}(x)=x_1x_2\oplus q_2(x_3,\ldots,x_n)\,,
$$
где $q_2=q_1\oplus l_1l_2$~--- квадратичная функция.
Заметим, что $(f^{(2)})^{\varphi_2}\hm=f$.

Затем аналогично предыдущему выделим первые две переменные в~функции 
$q_2(x_3,\ldots,x_n)$.
Здесь берется отображение
\begin{multline*}
\varphi_4(x)=\bigl(x_1,\ x_2,\  x_3\oplus l_4(x_5,\ldots,x_n),\\
x_4\oplus 
l_3(x_5,\ldots,x_n),\ x_5,\ldots,\ x_n\bigr)
\end{multline*}
и функция
$$
f^{(4)}(x)=x_1x_2\oplus  x_3x_4\oplus q_4(x_5,\ldots,x_n)\,,
$$
так что $(f^{(4)})^{\varphi_4}=f^{(2)}$.

Проделываем это до тех пор, пока на некотором шаге не получим аффинную функцию
$$
q_{2r}(x_{2r+1},\ldots,x_n)=\bigoplus_{i=2r+1}^{n}b_ix_i\oplus c
$$
для некоторых $b_i,c\hm\in\mathbb{F}_2$.
Если $b_i\hm=0$ для всех~$i$, $2r\hm+1\hm\leqslant i\hm\leqslant n$, 
то искомый канонический вид 
найден: это функция~$f^{(2r)}$.
Иначе считаем, без ограничения общности, что $b_{2r+1}\hm=1$ и~полагаем
\begin{multline*}
\varphi_{2r+1}(x)=\left(x_1,\ldots,\ x_{2r},\ 
q_{2r}\left(x_{2r+1},\ldots,x_n\right),\right.\\ 
\left.x_{2r+2},\ldots,\ x_n\right)\,.
\end{multline*}
Тогда канонический вид для~$f$~--- это функция
\begin{multline*}
g(x)={}\\
{}=f^{(2r+1)}(x)=x_1x_2\oplus  x_3x_4\oplus \cdots \oplus x_{2r-1}x_{2r} 
\oplus x_{2r+1},
\end{multline*}
причем если положить $\varphi\hm=\varphi_{2r+1}\varphi_{2r}
\varphi_{2r-2}\cdots\varphi_2$, то
$$
g^{\varphi}=\left(\cdots(g^{\varphi_{2r+1}})^{\varphi_{2r}}\cdots\right)^{\varphi_2}=f.$$

%$g^{\varphi}(x)=g\bigl(\varphi_{2r+1}(\ldots\varphi_2(x)\ldots)\bigr)$.
Преобразование~$\varphi$, очевидно, аффинно, невырожденно и~имеет вид:
$$    \hspace*{-33mm}\varphi(x) ={}\hspace*{33mm}
$$
\begin{equation*}
      \begin{split}
    {}=
    x &
    {
      \begin{pmatrix}
\makebox[1.5em]{$1$} &\rule{1.5em}{0pt} & \rule{1.5em}{0pt}   & 
\rule{1.5em}{0pt}   & \rule{1.5em}{0pt}   &  & \rule{1.5em}{0pt}   & 
\rule{1.5em}{0pt}   & \rule{1.5em}{0pt}   &  \rule{1.5em}{0pt}   \\
        0 & 1 &   &   &   &   &   &   &   &   \\
        * & * &\smash[t]{\ddots}&&   &   &   &   & 
\smash[t]{\mbox{\Huge{$0$}}}  &   \\
        * & * & \smash[t]{\ddots}  & 1 &   &   &   &   &   &      \\
        * & * &\smash[t]{\ddots}   & 0 & 1 &   &   &   &   &      \\
        * & * &\smash[t]{\ddots}   & * & * & 1 &   &   &   &   \\
        * & * &\smash[t]{\ddots}   & * & * & \makebox[1.5em]{$b_{2r+2}$}  & 1 &   &  &     \\
        * & * & \smash[t]{\ddots}  & * & * & \makebox[1.5em]{$b_{2r+3}$}  & 0 
&\smash[t]{\ddots}&& \\
        \vdots  &\vdots   & \smash[t]{\ddots}  &\vdots   &\vdots   & \vdots  & \vdots  &\ddots& 
1 &     \\
        * & * &\cdots   & * & * &b_n& 0 &\cdots & 0 & 1\\
      \end{pmatrix}} \oplus \\
   \oplus &
      \;\begin{pmatrix}
      \makebox[1.5em]{$*$}&\makebox[1.5em]{$*$}&\makebox[1.5em]{$\cdots$}&\makebox[1.5
em]{$*$}&\makebox[1.5em]{$*$}&\makebox[1.5em]{$c$}&\makebox[1.5em]{$0$}&\makebox
[1.5em]{$\cdots$}&\makebox[1.5em]{$0$}&\makebox[1.5em]{$0$} \\
      \end{pmatrix}
  \end{split}
\end{equation*}
(здесь знак~$*$ заменяет собой один из элементов~$\mathbb{F}_2$, каждый раз 
свой). Соответственно, преобразование~$\alpha$ из формулировки теоремы Диксона 
является обратным к~$\varphi$.

Как будет показано в~разд.~\ref{canonic-to-ANF}, представление функций~$f_i$ 
в~таком виде, т.\,е.\ нахождение подходящих представителей орбиты 
$\mathrm{Orb}_{f_i}(\mathrm{GA}(V_n))$, позволяет легко выписать аффинные 
нормальные формы для~$f_i$.

\section{Построение аффинной нормальной формы для~квадратичной 
функции}\label{canonic-to-ANF}

Обозначим через $\varphi_i$, $1\hm\leqslant i\hm\leqslant m$, невырожденные аффинные преобразования 
пространства~$V_n$, с~помощью которых функции~$f_i$ приводятся к~каноническому 
виду~$g_i$:
$$
g_i(x)=f_i^{\varphi_i}(x)=x_1x_2\oplus\cdots\oplus x_{2r_i-1}x_{2r_i}\oplus 
b_ix_{2r_i+1}\oplus c_i\,,
$$
где $2r_i=\rank \tilde{Q}_{f_i}$, а $b_i, c_i\hm\in \mathbb{F}_2$.

Очевидно, что если среди первых~$2r_i$ переменных взять все переменные с~четными 
индексами или все с~нечетными и~зафиксировать их значения, то получится 
плоскость, являющаяся локальной аффинностью функции~$g_i$.
Рассмотрим, например, $2^{r_i}$ плоскостей, заданных уравнениями:
$$    \begin{array}{l@{\,}c@{\ }l}
        x_1&=&\delta_1;\\
        x_3&=&\delta_2;\\
        \vdots\\
        x_{2r_i-1}&=&\delta_{r_i},\\
    \end{array}
$$
где $\delta_j\in\mathbb{F}_2$, $1\hm\leqslant j\hm\leqslant r_i$.
Каждую из этих плоскостей обозначим через~$\pi'_{i,\delta}$ со сложным индексом 
$\delta\hm=(\delta_1,\dots,\delta_{r_i})\in\mathbb{F}_2^{r_i}$.
Размерность~$\pi'_{i,\delta}$ равна $n\hm-r_i$, а мощность, соответственно, 
$2^{n-r_i}$.
Нетрудно видеть, что $\Pi'_i\hm=\{\pi'_{i,\delta}\}_{\delta\in\mathbb{F}_2^{r_i}}$ 
является разбиением пространства~$V_n$.

Обозначаемое ниже через~$l'_{i,\delta}$ сужение функции~$g_i$ на каждую из 
плоскостей разбиения~--- аффинно:
\begin{multline*}
l'_{i,\delta}(x)={}\\
{}=g_i|_{\pi'_{i,\delta}}(x)=\delta_1x_2\oplus\delta_2x_4\cdots\oplus\delta_{r_i}x_{2r_i}\oplus b_ix_{2r_i+1}\oplus c_i,
\hspace*{-0.80452pt}
\end{multline*}
а характеристическая функция соответствующей плоскости имеет вид:
$$
\chi_{\pi'_{i,\delta}}(x)=\prod\limits_{k=1}^{r_i}\left(x_{2k-1}\oplus\delta_k\oplus1\right)\,.
$$

С помощью аффинной нормальной формы
$$
g_i(x)=\bigoplus_{\delta\in\mathbb{F}_2^{r_i}} 
\chi_{\pi'_{i,\delta}}(x)l'_{i,\delta}(x)
$$
для канонического представления функции~$f_i$ можно аффинным преобразованием, 
обратным к~$\varphi_i$, получить аффинную нормальную форму для исходной функции:

\vspace*{1pt}

\noindent
$$
f_i(x)=g_i^{\varphi_i^{-1}}(x) = \bigoplus_{\delta\in\mathbb{F}_2^{r_i}} 
\chi_{\pi_{i,\delta}}(x)l_{i,\delta}(x)\,,
$$

\vspace*{-3pt}

\noindent
где $\pi_{i,\delta}\hm=\varphi_i(\pi'_{i,\delta})$ 
и~$l_{i,\delta}(x)\hm=l'_{i,\delta}(\varphi_i^{-1}(x))$.

Разумеется, если алгебраическая степень какой-либо функции~$f_i$ оказалась 
равной~$1$, то искать ничего не нужно: ее полином Жегалкина является ее аффинной 
нормальной формой для тривиального разбиения $\Pi_i\hm=\{V_n\}$.

\smallskip

\noindent
\textbf{Замечание~2}.
    Подобный способ построения аффинной нормальной формы можно использовать 
    и~непосредственно для квадратичной\footnote{Для функций более высоких степеней 
такой подход тоже работает, но описать его строго гораздо сложнее и~полученные 
таким образом локальные аффинности, скорее всего, будут слишком маленькой 
размерности.} функции~$f$ в~ее исходном виде. Нужно просто фиксировать значения 
переменных так, чтобы в~каждом мономе оставалось не более одной свободной 
переменной. Для этого удобнее рассмотреть матрицу~$Q_f$, выбрать в~ней столбец 
или строку с~максимальным числом единиц среди всех столбцов и~строк (пусть это 
будет $k$-я строка) и~зафиксировать~$x_k$. Затем то же проделать, исключив из 
рассмотрения $k$-ю строку и~$k$-й столбец матрицы, и~так далее, пока единицы 
в~матрице не кончатся.
Однако, несмотря на то что здесь имеет место экономия на приведении функции 
к~каноническому виду, такой способ представляется менее эффективным в~следующем 
смысле. Канонический вид квадратичной функции содержит минимальное число мономов 
степени~$2$, поэтому для исходной (неканонической) функции придется фиксировать, 
как правило, большее число переменных. Но с~каждой дополнительно зафиксированной 
переменной размерность локальных аффинностей функции~$f$ уменьшается на~$1$, 
а~их число, соответственно, увеличивается вдвое.

\vspace*{-4pt}


\section{<<Локальные>> системы линейных уравнений}

\vspace*{-2pt}

Идея метода решения систем квадратичных булевых уравнений состоит в~следующем.
Пусть для всех~$f_i$, $1\hm\leqslant i\hm\leqslant m$, 
определены некоторые аффинные нормальные 
формы

\vspace*{1pt}

\noindent
\begin{equation*}
\label{AffNF_ij}
    f_i(x) = \bigoplus_{j=1}^{\lambda(i)} \chi_{\pi_{ij}}(x)l_{ij}(x)\,.
\end{equation*}

\vspace*{-3pt}

\noindent
Исходя из этих аффинных нормальных форм, можно для каждой пары~$i,j$ записать 
эквивалентную уравнению $f_i\hm=0$ на~$\pi_{ij}$ систему линейных уравнений:

\columnbreak

\noindent
\begin{equation*}
    \begin{array}{r@{\ }c@{\ }l}
        l_{ij}(x)&=&0;\\
        h_{ij}^1(x)&=&0;\\
        \vdots\\
        h_{ij}^{d(i,j)}(x)&=&0,\\
    \end{array}
  \label{approx}
\end{equation*}
в которой первое уравнение выражает равенство~$f_i\hm=0$ через аффинную 
аппроксимацию~$l_{ij}(x)$ функции~$f_i(x)$ на плоскости~$\pi_{ij}$, а остальные 
$d(i,j)$ уравнений задают эту плоскость.


\smallskip

\noindent
\textbf{Замечание~3.}\
Если аффинная нормальная форма получена описанным выше способом~--- через 
канонический вид квадратичной функции,~--- то характеристическая функция будет 
иметь вид:

\vspace*{1pt}

\noindent
$$
\chi_{\pi_{i,\delta}}(x)=\prod_{k=1}^{r_i}(\varphi_i^{-1}(x)e_{2k-1}
\Tsf\oplus\delta_k\oplus 1)\,,
$$

\vspace*{-3pt}

\noindent
где $e_{2k-1}$~--- $(2k-1)$-й базисный вектор, т.\,е.\ $\varphi_i^{-1}(x)e_{2k-1}
\Tsf$~--- $(2k-1)$-я компонента вектора~$\varphi_i^{-1}(x)$.
Значит, соответствующую плоскость задают уравнения
 $\{ \varphi_i^{-1}(x)e_{2k-1}\Tsf \oplus \delta_k \hm= 0 
 \mid 1\hm\leqslant k\hm\leqslant r_i \}$.

\smallskip

Таким образом, имеется набор <<локальных>> линейных систем для каждого уравнения 
исходной системы и~для каждой его локальной аффинности.
Метод состоит в~том, чтобы подобрать комбинацию <<локальных>> систем разных 
квадратичных уравнений, в~совокупности дающую решение исходной системы. Если 
решение квадратичной системы единственно (а~это естественное предположение для 
криптоанализа), ровно одна такая комбинация будет иметь решение, и~от того, как 
быстро удастся ее обнаружить, зависит эффективность метода.

\vspace*{-4pt}

\section{О~трудоемкости построения аффинной нормальной формы}

\vspace*{-2pt}

Напомним, что для функций из системы~\eqref{system} $r_i\hm=({1}/{2})\rank 
\tilde{Q}_{f_i}\hm\leqslant {n}/{2}$, $1\hm\leqslant i\hm\leqslant m$,~--- 
параметр, введенный в~разд.~\ref{canonic-to-ANF}.
Алгоритм приведения $m$~функций к~каноническому виду (см.\ разд.~4) 
имеет трудоемкость, оцениваемую выражением $O(\sum\nolimits_{i=1}^m n^2r_i)$, 
а~учитывая 
неравенство $r_i\hm\leqslant {n}/{2}$, имеем~$O(mn^3)$.

При построении аффинных нормальных форм для функции~$f_i$ в~разд.~5 
потребуется порядка $r_i2^{r_i}\hm+ n^3$ операций. 
Значит, для всех~$m$~функций имеем оценку $O(mn^3\hm+\sum\nolimits_{i=1}^m r_i2^{r_i})$.

Таким образом, в~худшем случае, когда все $r_i\hm={n}/{2}$ или даже когда хотя 
бы $r_i\hm=O(n)$ для некоторого~$i$, предложенный алгоритм экспоненциален.
Однако можно рассчитывать, что во встречающихся на практике системах 
квадратичных уравнений параметр~$r_i$ растет (с~увеличением~$n$) медленнее, 
и~тогда можно говорить о полиномиальности алгоритма построения аффинных нормальных 
форм.

В случае, когда система вида~\eqref{system} переопределенная, т.\,е.\ $n\hm\ll m$ 
(переопределенные системы достаточно часто рассматриваются в~задачах 
информатики, теории кодирования и~криптографии), можно рассчитывать на 
существование подсистемы (из~$l$~уравнений с~номерами $i_1,\ldots,i_l$), для 
которой трудоемкость построения аффинных нормальных форм меньше, чем 
экспоненциальная. Например, когда $r_{i_j}\hm=O(\sqrt{n})$, $ 1\hm\leqslant j\hm\leqslant l$, 
оценка со\-от\-вет\-ст\-ву\-ющей трудоемкости для системы~\eqref{system} имеет 
субэкспоненциальный характер.

Рассмотрим в~качестве еще одного примера класс~$\mathcal{K}_m$ систем 
$m$~квадратичных булевых уравнений от $n$~неизвестных вида~\eqref{system}, где 
$m\hm=m(n)$~--- некоторый полином от~$n$ и~где $r_i=
O(\log_2 n)$ для всех~$i$, $1\hm\leqslant i\hm\leqslant m$.


\vspace*{2pt}


\noindent
\textbf{Предложение~3.}
\textit{Для систем~\eqref{system} квадратичных булевых уравнений из 
класса~$\mathcal{K}_m$ существует полиномиальный} (\textit{по~$n$}) \textit{алгоритм построения 
аффинных нормальных форм для функций~$f_i$.}


\smallskip

Для доказательства этого утверждения достаточно рассмотреть предложенный 
в~статье алгоритм построения аффинных нормальных форм для квад\-ра\-тич\-ных булевых 
функций. В~полученной выше оценке  $O(mn^3\hm+\sum\nolimits_{i=1}^m r_i2^{r_i})$
данные в~условии ограничения на~$m$ и~на~$r_i$ дают полиномиальную оценку трудоемкости 
алгоритма.

%Отметим, что если рассматривать систему квадратичных уравнений, описывающую 
%функционирование произвольного фильтрующего генератора, то у всех уравнений 
%системы будет одно и~то же значение $r_i$, определяемое рангом матрицы 
%$\tilde{Q}_{f'}$, где $f'$ "--- ... для фильтрующей функции~$f$. Поэтому

\vspace*{-12pt}

{\small\frenchspacing
 {%\baselineskip=10.8pt
 \addcontentsline{toc}{section}{References}
 \begin{thebibliography}{99}

    \bibitem{GJ1982}
        \Au{Гэри~М., Джонсон~Д.}
        Вычислительные машины и~труднорешаемые задачи~/ Пер. с~англ.~---
        М.: Мир, 1982. 416~с.
        (\Au{Garey~M.\,R., Johnson~D.\,S.} Computers and intractability: 
A~guide to the theory of NP-completeness.~--- San Francisco, CA, USA: W.\,H.~Freeman 
and Co., 1979. 348~p.).

    \bibitem{GT2017}
        \Au{Горшков~С.\,П., Тарасов~А.\,В.}
        Сложность решения сис\-тем булевых уравнений.~---
        М.: Курс, 2017. 192~с.

    \bibitem{Smi2000}
        \Au{Смирнов~В.\,Г.}
        {Некоторые классы эффективно ре\-ша\-емых систем булевых уравнений}~//
        Труды по дискретной математике, 2000. Т.~3. С.~269--282.

    \bibitem{Bard2009}
        \Au{Bard~G.\,V.}
        Algebraic cryptanalysis.~--- Springer, 2009. 389~p.

    \bibitem{BCJ2007}
        \Au{Bard~G., Courtois~N., Jefferson~C.}
        {Efficient methods for conversion and solution of sparse systems of 
        low-degree multivariate polynomials over $\mathrm{GF}(2)$ via SAT-solvers}~//
        Cryptology ePrint Archive. Report 2007/024.
        {\sf http://eprint.iacr.org/2007/024.pdf}.

    \bibitem{LSSYa2015}
        \Au{Логачев~О.\,А., Сальников~А.\,А., Смышляев~С.\,В., 
Ященко~В.\,В.}
        Булевы функции в~теории кодирования и~крип\-то\-ло\-гии.~---
        М.: ЛЕНАНД, 2015. 576~с.

    \bibitem{MWS1979}
        \Au{Мак-Вильямс~Ф.\,Дж., Слоэн~Н.\,Дж.\,А.}
        Теория кодов, исправляющих ошибки~/ Пер. с~англ.~---
        М.: Связь, 1979. 743~с.
        (\Au{MacWilliams~F.\,J., Sloane~N.\,J.\,A.} The theory of 
        error-correcting codes.~--- 
        North-Holland mathematical library ser.~---
        North-Holland Publishing Co., 1977.  774~p.)

    \bibitem{LYaD2007}
        \Au{Logachev~O.\,A., Yashchenko~V.\,V., Denisenko~M.\,P.}
        {Local affinity of Boolean mappings}~//
        Boolean functions in cryptology and information security: Proceedings of the 
NATO Advanced Study Institute.~---
        IOS Press, 2008. P.~148--172.

    \bibitem{Gor1995}
        \Au{Горшков~С.\,П.}
        {Применение теории NP-пол\-ных задач для оценки сложности решения систем 
булевых уравнений}~//
        Обозрение прикладной и~промышленной математики, 1995. Т.~2. Вып.~3. 
С.~325--398.

    \bibitem{Dickson1901}
        \Au{Dickson~L.\,E.}
        Linear groups: With an exposition of the Galois field theory.~---
        Leipzig: B.\,G.\,Teubner, 1901. 322~p.

   % \bibitem{KSh1999}
       % \Au{Kipnis~A., Shamir~A.}
      %  {Cryptanalysis of the HFE public key cryptosystem by relinearization}~//
     %   Advances in cryptology~/
    %    Ed.\ M.\,J.~Wiener.~---
   %     Lectures notes in computer science ser.~---
   %     Springer, 1999. Vol.~1666. P.~19--30.

   % \bibitem{CShPK2000}
  %      \Au{Courtois~N., Klimov~A., Patarin~J., Shamir~A.}
 %       {Efficient algorithms for solving overdefined systems of multivariate 
%polynomial equations}~// Advances in cryptology~/
%Ed.\ B.~Preneel.~---
%         Lectures notes in computer science ser.~--- Springer, 2000. Vol.~1807. 
%P.~392--407.

   % \bibitem{FY1980}
  %      \Au{Fraenkel~A.\,S., Yesha~Y.}
 %       {Complexity of solving algebraic equations}~//
 %       Inform. Process. Lett., 1980. Vol.~10. Iss.~4-5. P.~178--179.

\end{thebibliography} 
 }
 }

\end{multicols}

\vspace*{-3pt}

\hfill{\small\textit{Поступила в~редакцию 11.01.19}}

\vspace*{8pt}

%\pagebreak

%\newpage

%\vspace*{-28pt}

\hrule

\vspace*{2pt}

\hrule

%\vspace*{-2pt}

\def\tit{POLYNOMIAL ALGORITHMS FOR~CONSTRUCTING LOCAL AFFINITIES OF~QUADRATIC BOOLEAN FUNCTIONS}

\def\titkol{Polynomial algorithms for~constructing local affinities of~quadratic Boolean functions}

\def\aut{O.\,A.~Logachev$^{1,2}$, A.\,A.~Sukayev$^1$, and~S.\,N.~Fedorov$^1$}

\def\autkol{O.\,A.~Logachev, A.\,A.~Sukayev, and~S.\,N.~Fedorov}

\titel{\tit}{\aut}{\autkol}{\titkol}

\vspace*{-11pt}


\noindent
$^1$Information Security Institute,  M.\,V.~Lomonosov Moscow State University, 
1~Michurinskiy Prosp., Moscow\linebreak
$\hphantom{^1}$119192, Russian Federation

\noindent
$^2$Institute of Informatics Problems, 
Federal Research Center ``Computer Science and Control'' 
of the Russian\linebreak
$\hphantom{^1}$Academy of Sciences, 44-2~Vavilov Str., Moscow 119333, 
Russian Federation

\def\leftfootline{\small{\textbf{\thepage}
\hfill INFORMATIKA I EE PRIMENENIYA~--- INFORMATICS AND
APPLICATIONS\ \ \ 2019\ \ \ volume~13\ \ \ issue\ 1}
}%
 \def\rightfootline{\small{INFORMATIKA I EE PRIMENENIYA~---
INFORMATICS AND APPLICATIONS\ \ \ 2019\ \ \ volume~13\ \ \ issue\ 1
\hfill \textbf{\thepage}}}

\vspace*{6pt}


\Abste{Due to the affine normal form, one can consider a~Boolean function 
as affine on certain flats in its domain~--- so-called local affinities. 
This Boolean function representation~--- affine approximation~---
could be
useful 
for solving systems of nonlinear equations over two-element field. The problem 
of solving these systems
(of a~special sort) arises, in particular, in some methods 
of the information security tools design and analysis.
The
paper describes an approach to finding local affinities for quadratic Boolean 
functions which is based on Dickson's\linebreak\vspace*{-12pt}}

\Abstend{theorem. By this, one obtains affine 
normal forms for such functions. Besides, the paper concerns the efficiency of 
corresponding algorithms.
This approach can be profitable for constructing efficient methods of solving 
systems of quadratic Boolean equations via ``approximation'' of corresponding 
Boolean functions by their affine normal forms.}

\KWE{Boolean function; system of quadratic Boolean equations; vector 
space partition; flat; local affinity; Dickson's theorem; 
affine normal form (ANF) of Boolean function; algebraic cryptanalysis}






\DOI{10.14357/19922264190110}

\vspace*{-14pt}

\Ack
\noindent
The paper was partly supported by the Russian Foundation for Basic Research 
(project 18-29-03124~mk).





  \begin{multicols}{2}

\renewcommand{\bibname}{\protect\rmfamily References}
%\renewcommand{\bibname}{\large\protect\rm References}

{\small\frenchspacing
 {%\baselineskip=10.8pt
 \addcontentsline{toc}{section}{References}
 \begin{thebibliography}{99}
\bibitem{1-log-1}
\Aue{Garey, M.\,R., and D.\,S.~Johnson.} 1979. \textit{Computers and intractability: 
A~guide to the theory of NP-completeness.} San Francisco, CA: W.\,H.~Freeman and Co. 348~p.
\bibitem{2-log-1}
\Aue{Gorshkov, S.\,P., and A.\,V.~Tarasov.} 2017. \textit{Slozhnost' re\-she\-niya 
sistem bulevykh uravneniy} [Complexity of solving the systems of 
Boolean equations]. Moscow: Kurs. 192~p.
\bibitem{3-log-1}
\Aue{Smirnov, V.\,G.} 2000. Nekotorye klassy effektivno reshaemykh 
sistem bulevykh uravneniy [Some classes of Boolean equation systems 
permitting effective solution]. 
\textit{Trudy po diskretnoy matematike} [Proceedings on Discrete Mathematics] 3:269--282.
\bibitem{4-log-1}
\Aue{Bard, G.\,V.} 2009. \textit{Algebraic cryptanalysis}. Springer. 389~p.
\bibitem{5-log-1}
\Aue{Bard, G., N.~Courtois, and C.~Jefferson.} 2007. 
Efficient methods for conversion and solution of sparse systems of 
low-degree multivariate polynomials over GF(2) via SAT-solvers. 
\textit{Cryptology ePrint Archive}. Report 2007/024. Available at: 
{\sf http://eprint.iacr.org/2007/024.pdf} (accessed August~30, 2018).
\bibitem{6-log-1}
\Aue{Logachev, O.\,A., A.\,A.~Sal'nikov, S.\,V.~Smyshlyaev, and V.\,V.~Yashchenko.} 
2015. \textit{Bulevy funktsii v~teorii kodirovaniya i~kriptologii} 
[Boolean functions in coding theory and cryptology]. Moscow: LENAND. 576~p.
\bibitem{7-log-1}
\Aue{MacWilliams, F.\,J., and N.\,J.\,A.~Sloane.} 1977. 
\textit{The theory of error-correcting codes}. 
North-Holland mathematical library ser.
North-Holland Publishing Co. 774~p.
\bibitem{8-log-1}
\Aue{Logachev, O.\,A., V.\,V.~Yashchenko, and M.\,P.~Denisenko.} 2008. 
Local affinity of Boolean mappings. 
\textit{Boolean functions in cryptology and information security: 
Proceedings of the NATO Advanced Study Institute.} IOS Press. 148--172.
\bibitem{9-log-1}
\Aue{Gorshkov, S.\,P.} 1995. Primenenie teorii NP-polnykh zadach 
dlya otsenki slozhnosti resheniya sistem bulevykh uravneniy 
[Application of the NP-complete problem theory to assessment 
of complexity of solving the systems of Boolean equations]. 
\textit{Obozrenie prikladnoy i~promyshlennoy matematiki} 
[Applied and Industrial Mathematics Review] 2(3):325--398.
\bibitem{10-log-1}
\Aue{Dickson, L.\,E.} 1901. \textit{Linear groups: 
With an exposition of the Galois field theory}. Leipzig: B.\,G.~Teubner. 322~p.
%\bibitem{11-log-1}
%\Aue{Kipnis, A., and A.~Shamir.} 1999. 
%Cryptanalysis of the HFE public key cryptosystem by relinearization. 
%\textit{Advances in cryptology}. Ed. M.\,J.~Wiener.
% Lecture notes in computer science ser.  Springer. 1666:19--30.
%\bibitem{12-log-1}
%\Aue{Courtois, N., A.~Klimov, J.~Patarin, and A.~Shamir.} 2000. 
%Efficient algorithms for solving overdefined systems of multivariate polynomial 
%equations. \textit{Advances in cryptology}. Ed.\ B.~Preneel.
%Lecture notes in computer science ser.  Springer. 1807:392--407.
%\bibitem{13-log-1}
%\Aue{Fraenkel, A.\,S., and Y.~Yesha.} 1980. 
%Complexity of solving algebraic equations. 
%\textit{Inform. Process. Lett.} 10(4-5):178--179.
\end{thebibliography}

 }
 }

\end{multicols}

\vspace*{-6pt}

\hfill{\small\textit{Received January 11, 2019}}

%\pagebreak

%\vspace*{-18pt}

\Contr

\noindent
\textbf{Logachev Oleg A.} (b.\ 1950)~--- 
Candidate of Science (PhD) in physics and mathematics, head of department, 
Information Security Institute, M.\,V.~Lomonosov Moscow State University, 
1~Michurinskiy Prosp., Moscow 119192, Russian Federation; 
senior scientist, Institute of Informatics Problems, 
Federal Research Center ``Computer Science and Control'' 
of the Russian Academy of Sciences, 44-2~Vavilov Str., Moscow 119333, 
Russian Federation; \mbox{logol@iisi.msu.ru }

 



\vspace*{3pt}

\noindent
\textbf{Sukayev Al'bert A.} (b.\ 1994)~--- 
student, Information Security Institute, Moscow State University, 
1~Michurinskiy Prosp., Moscow 119192, Russian Federation; 
\mbox{asukaev@gmail.com}

\vspace*{3pt}

\noindent
\textbf{Fedorov Sergey~N.} (b.\ 1982)~--- 
Candidate of Science (PhD) in physics and mathematics, senior scientist, 
Information Security Institute, M.\,V.~Lomonosov Moscow State University, 
1~Michurinskiy Prosp., Moscow 119192, Russian Federation; 
\mbox{s.n.feodorov@yandex.ru}
\label{end\stat}

\renewcommand{\bibname}{\protect\rm Литература}       