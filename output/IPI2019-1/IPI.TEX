\documentclass[10pt]{book}
\usepackage[utf8]{inputenc}

\usepackage{latexsym,amssymb,amsfonts,amsmath,amsxtra,dsfont,
indentfirst,shapepar,%fleqn,%
picinpar,shadow,floatflt,enumerate,multicol,colortbl,moreverb,cite,ipi}

\usepackage{rotating}
\usepackage{mathrsfs}
\usepackage[noend]{algorithmic}
\usepackage{ulem}
\usepackage{graphicx}
%\usepackage{algorithm2e}
\usepackage[linesnumbered,boxed,ruled]{algorithm2e}
%\usepackage{xypic}
\usepackage{oldgerm}
\usepackage{epic}
\usepackage{eepic}


\SetAlgorithmName{Algorithm}{алгоритм}{Список алгоритмов}

%из Дюковой

\newcommand{\algKeyword}[1]{{\bf #1}}
\newcommand{\Proc}[1]{\text{\tt #1}}
\def\CALL{\algKeyword{call}~}

\newenvironment{AlgProcedure}[1]
{
    \small
    \medskip
    %    \hrule
    \medskip
    \algKeyword{PROCEDURE} #1
    \begin{algorithmic}[1]}
    {\end{algorithmic}
    %    \hrule
    \bigskip
}

\def\CALL{\algKeyword{call}~}

%конец для Дюковой

%\RequirePackage[ruled]{algorithm}


\input{epsf}

%\nofiles

%\includeonly{avtor}             %+pdf+
%\includeonly{obchak,avtor}
%\includeonly{pred}                 %+
%\includeonly{podgot-rus-site,podgot-eng-site}  
%\includeonly{ocherk} 
%\includeonly{nekrol} 
%\includeonly{ipi-ind} 
%\includeonly{index12}
%\includeonly{toc-rus, toc-en}
%\includeonly{toc-rus}
%\includeonly{toc-en} 


%\includeonly{sinits}                                 %1pdf+
%\includeonly{bosov+stef}                             %2pdf+
%\includeonly{rybakov}                                %3pdf+
%\includeonly{dukova}                                 %4pdf+
%\includeonly{malashenko}                             %5pdf+
%\includeonly{strijov}                                %6pdf
%\includeonly{shest}                                  %7pdf+
%\includeonly{agalar}                                 %8pdf+
%\includeonly{kudr}                                   %9pdf+???
%\includeonly{logachev}                               %10pdf+
%\includeonly{gorshenin}                              %11pdf
%\includeonly{dulin}                                  %12pdf+нет замечаний
%\includeonly{zatsman}                                %13pdf+
%\includeonly{gorbunova}                              %14pdf+
%\includeonly{gudkova}                                %15pdf+
    


 



%\includeonly{nekrol}             %+


%\includeonly{obchak}
%\includeonly{rekl}
%\includeonly{rekl-1}
%\includeonly{reshal}  %
%\includeonly{cover3}

\usepackage{acad}
%\usepackage{courier}
\usepackage{decor}
\usepackage{newton}
\usepackage{pragmatica}
\usepackage{zapfchan}
\usepackage{petrotex}
\usepackage{bm}                     % полужирные греческие буквы
\usepackage{upgreek}                % прямые греческие буквы
\usepackage{eufrak}
\usepackage{verbatim}

\renewcommand{\bottomfraction}{0.99}
\renewcommand{\topfraction}{0.99}
\renewcommand{\textfraction}{0.01}

\setcounter{secnumdepth}{1} %здесь - 3 + chapter = 4

\arraycolsep=1.5pt

%\usepackage[pdftex]{graphicx}

%\usepackage{oz}

%NEW COMMANDS


\renewcommand*{\hm}[1]{#1\nobreak\discretionary{}%
            {\hbox{$\mathsurround=0pt #1$}}{}} %% Дублирует знаки операций
                               %при переносе в формуле (перед знаком, который
                               %надо продублировать ставится команда \hm)

%\newcommand{\endproof}{\hfill$\Box$}
\renewcommand{\r}{\mathbb{R}}
%\newcommand{\I}{{\rm I\hspace{-0.7mm}I}}
%\newcommand{\Ikl}{{\tt{1}}\hspace*{-1.44mm}\mathtt{1}}
\newcommand{\Ik}{\mbox{{\small \tt {1}}\hspace{-1.3mm}{\tt 1}}}
\newcommand{\argmin}{\mathop{\mathrm{arg}\,\mathrm{min}}}
\newcommand{\argmax}{\mathop{\mathrm{arg}\,\mathrm{max}}}
%\newcommand{\capr}{\mathop{\cap\,}}
%\newcommand{\cupr}{\mathop{\cup\,}}
%\def\argmin{\mathop{arg\,min}}

\def\vrp{\varphi}
\def\prt{\partial}
\def\mm{{\sf M}}
\def\modnop#1{\mathop{#1}\limits_{n}}
\def\eam{\mathbin{{\mathop{=}\limits^{\mathrm{def}}}}}
\def\dey#1#2{#1 (#2)}
\def\deyc#1#2{#1 \cdot  #2}
\def\ra#1{\;\mathop{\to}\limits^{#1}\;}
\def\raz#1{\;\mathop{\longrightarrow}\limits^{\!\!\!#1}\;}
\def\ral#1{\;\mathop{\longrightarrow}\limits^{#1}\;}

\newcommand{\Nor}{\mathcal{N}}
\newcommand{\T}{\mathbb{T}}
\newcommand{\Z}{\mathbb{Z}}



\newcommand{\il}[2]{\int\limits_{#1}^{#2}}%интеграл с пределами #1 и #2

\def\sm2{\mathop {\sum\limits^{n^\Theta}\sum\limits^{n^\Theta}}}
\def\sss{\sum\limits}
\def\tr{,\,\ldots\,,\,}
\def\rk{\right]}
\def\lk{\left[}
\def\rf{\right\}}
\def\lf{\left\{}
\def\lv{\,\left\vert}
\def\rv{\right\vert\,}
\def\iii{\int\limits}
\def\iin{\int\limits_{-\infty}^\infty}
\def\rrv{\right\vert}


\def\ee{{\cal E}}
\def\ww{{\cal W}}
\def\yy{{\cal Y}}
\def\vv{{\cal V}}

\newcommand{\R}{\mathbb R}
\newcommand{\E}{\mathbb E}
\newcommand{\N}{\mathbb N}

\renewcommand{\P}{\mathbb{P}}

\newcommand{\h}{{\bf H}}
\newcommand{\p}{{\sf P}}  % вероятность

\newcommand{\e}{{\sf E}}  % мат. ожидание
\newcommand{\D}{{\sf D}}  % дисперсия
\newcommand{\eps}{\varepsilon}
\newcommand{\vp}{{\mathbf p}}
\newcommand{\vz}{{\mathbf z}}
\newcommand{\vx}{{\mathbf x}}
\newcommand{\vf}{{\mathbf f}}
\newcommand{\F}{{\mathcal F}}
\def\ap{{\mathrm{ЭР}}}
\newcommand{\ud}{\Delta_n} %uniform ditance
\newcommand{\nud}{\Delta_n(x)}
%\renewcommand{\Re}{\mathrm{Re}\,}

\newcommand{\abs}[1]{\left\vert#1\right\vert}

\newcommand{\norm}[1]{\left\Vert#1\right\Vert}
\def\da{(\Delta_t,A)}

\newcommand{\corr}{\mathrm{corr}}

\newcommand{\cov}{\mathrm{cov}}
\newcommand{\Expect}{\mathbb{E}}

\def\w{\omega}
\def\W{\Omega}

\def\inh{\int\limits_{nh}^{(n+1)h}}

\def\sumin{\sum_{i=1}^N}


\def\bxt{(Y,t)}
\def\xt{(y,t)}

\def\ovth{{\fr{\tau-nh}{h}}}
\def\ov{\overline}
\def\tm{\tilde m}
\def\tl{\tilde\lambda}
\def\tB{\widetilde B}
\def\tb{\tilde b}
\def\ld{\ldots}
\def\cd{\cdots}


\DeclareMathOperator{\sign}{sign}

%\newcommand{\gr}{{\geqslant}}


\newcommand{\g}{\mbox{\textit{g}}}

\renewcommand{\la}{\lambda}
\newcommand{\si}{\sigma}
\newcommand{\alp}{\alpha}

\newcommand{\pto}{\stackrel{P}{\longrightarrow}} % сходимость по веpоятности

\newcommand{\eqd}{\stackrel{\mathrm{d}}{=}} % равенство по pаспpеделению
\newcommand{\eqdelta}{\stackrel{\triangle}{=}} % равенство по pаспpеделению

\def\be#1{\begin{equation}\label{#1}}
\def\ee{\end{equation}}
\def\re#1{(\ref{#1})}

\def\bn{\begin{enumerate}}
\def\en{\end{enumerate}}
\def\bi{\begin{itemize}}
\def\ei{\end{itemize}}
%\def\i{\item}

%\newcommand{\kp}{\kappa}
%\def\Q{{\cal Q}} \def\H{{\cal H}}
%\newcommand{\bet}{\beta_{2+\delta}}


%\newtheorem{definition}{Определение}
%\renewcommand{\thedefinition}{\arabic{definition}.}
%END NEW COMMANDS

%\renewcommand{\baselinestretch}{1.2}

%\pagestyle{myheadings}

\setlength{\textwidth}{167mm}      % 122mm
\setlength{\textheight}{658pt}
%\setlength{\textheight}{635.6pt}
\setlength{\columnsep}{4.5mm}

\setcounter{secnumdepth}{4}

%\addtolength{\headheight}{2pt}
%\addtolength{\headsep}{-2mm}

\addtolength{\topmargin}{-7mm}  % for printing


%\hoffset=-30mm  % From Yap
\hoffset=-23mm  % From Acrobat

%\voffset=0mm % From Yap
\voffset=-5mm   % From Acrobat

%\addtolength{\evensidemargin}{-2.5mm} % for printing
%\addtolength{\oddsidemargin}{2.5mm}  % for printing

\addtolength{\evensidemargin}{-12mm} % for printing
\addtolength{\oddsidemargin}{8mm}  % for printing

%\renewcommand{\thefootnote}{\fnsymbol{footnote}}
%\renewcommand{\thefootnote}{\arabic{footnote}}
\renewcommand{\figurename}{\protect\bf Рис.}
\renewcommand{\tablename}{\protect\bf Таблица}

\newcommand{\Caption}[1]{\caption{\protect\small %\baselineskip=2.5ex
#1}}

\renewcommand{\thefigure}{\arabic{figure}}
\renewcommand{\thetable}{\arabic{table}}
\renewcommand{\theequation}{\arabic{equation}}
\renewcommand{\thesection}{\arabic{section}}

\renewcommand{\contentsname}{СОДЕРЖАНИЕ}
\newcommand{\fr}[2]{\displaystyle\frac{\displaystyle #1\mathstrut}{\displaystyle #2\mathstrut}}

%\renewcommand{\thefootnote}{\fnsymbol{footnote}}
%\newcommand{\g}{\mbox{\textit{g}}}

%\newcommand{\Caption}[1]{\caption{\protect\small\baselineskip=2ex #1}}
\newcounter{razdel}
\setcounter{razdel}{0}


\newcommand{\titel}[4]{%
\

\vspace*{5pt}

\ifodd\therazdel {\raggedright\noindent\Large\textrm\textbf
 \lineskip .75em
  \baselineskip=3.2ex #1 \par}
\vskip 1em {\noindent\large\textrm\textbf #2 \par}
\addcontentsline{toc}{subsection}{{\textrm\textbf #1}\protect\newline #2}
\def\rightheadline{\underline{\noindent\hbox to \textwidth{\hfill\small\textrm{#4}
%\hfill \large\bf\thepage
}}}
\def\leftheadline{\underline{\noindent\parbox{\textwidth}{
%\raggedleft\large\bf\thepage \hfill
\small\textit{#3}\hfill}}}
\def\leftfootline{\small{\textbf{\thepage}
\hfill ИНФОРМАТИКА И ЕЁ ПРИМЕНЕНИЯ\ \ \ том~13\ \ \ выпуск 1\ \ \ 2019}
}%
 \def\rightfootline{\small{ИНФОРМАТИКА И ЕЁ ПРИМЕНЕНИЯ\ \ \ том~13\ \ \ выпуск~1\ \ \ 2019
\hfill \textbf{\thepage}}}
\vskip 2em \setcounter{figure}{0}
\setcounter{table}{0}
\setcounter{equation}{0}
\setcounter{section}{0}
\setcounter{subsection}{0}
\setcounter{subsubsection}{0}
\setcounter{footnote}{0}
\setcounter{razdel}{0}
%\end{flushleft}
\else {
 \raggedright\noindent\Large\textrm\textbf
 \lineskip .75em
\baselineskip=3.2ex #1 \par} \vskip 1em
%\begin{flushleft}
{\noindent\large\textrm\textbf #2 \par}
\addcontentsline{toc}{subsection}{{\textrm\textbf #1}\protect\newline #2}
\def\rightheadline{\underline{\noindent\hbox to \textwidth{\hfill\small\textrm{#4}
%\hfill \large\bf\thepage
}}}
\def\leftheadline{\underline{\noindent\parbox{\textwidth}{%\raggedleft\large\bf\thepage \hfill
\small\textit{#3}\hfill}}}
\def\leftfootline{\small{\textbf{\thepage}
\hfill ИНФОРМАТИКА И ЕЁ ПРИМЕНЕНИЯ\ \ \ том~13\ \ \ выпуск~1\ \ \ 2019}
}%
 \def\rightfootline{\small{ИНФОРМАТИКА И ЕЁ ПРИМЕНЕНИЯ\ \ \ том~13\ \ \ выпуск~1\ \ \ 2019
\hfill \textbf{\thepage}}} \vskip 2em \setcounter{figure}{0}
\setcounter{table}{0} \setcounter{equation}{0} \setcounter{section}{0}
\setcounter{subsection}{0} \setcounter{subsubsection}{0}
\setcounter{footnote}{0}
%\end{flushleft}
\fi}

\newcommand{\titelr}[2]{%
\

\vspace*{5pt}

\ifodd\therazdel {\raggedright\noindent%\Large\textrm\textbf
 \lineskip .75em
  \baselineskip=3.2ex #1 \par}
\vskip 1em {\noindent\normalsize\textrm\textbf #2 \par}
\else {
 \raggedright\noindent\Large\textrm\textbf
 \lineskip .75em
\baselineskip=3.2ex #1 \par} \vskip 1em
%\begin{flushleft}
{\noindent\large\textrm\textbf #2 \par
%\noindent\normalsize\textrm\textbf #2 \par
} \fi}

\newcommand{\titele}[5]{%
\

%\vspace*{5pt}

\ifodd\therazdel {\raggedright\noindent\large
\textrm\textbf
 \lineskip .75em
%  \baselineskip=3.2ex
#1 \par}
\vskip .5em {\noindent\large\textrm\textbf #2 \par}
\vskip .5em
 {\noindent\textrm #3 \par}
\addcontentsline{toc}{subsection}{{\textrm\textbf #1}\protect\newline #2}
\def\rightheadline{\underline{\noindent\hbox to \textwidth{\hfill\small\textrm{#4}
%\hfill \large\bf\thepage
}}}
\def\leftheadline{\underline{\noindent\parbox{\textwidth}{
%\raggedleft\large\bf\thepage \hfill
\small\textrm{#5}\hfill}}}
\def\leftfootline{\small{\textbf{\thepage}
\hfill ИНФОРМАТИКА И ЕЁ ПРИМЕНЕНИЯ\ \ \ том~13\ \ \ выпуск~1\ \ \ 2019}
}%
 \def\rightfootline{\small{ИНФОРМАТИКА И ЕЁ ПРИМЕНЕНИЯ\ \ \ том~13\ \ \ выпуск~1\ \ \ 2019
\hfill \textbf{\thepage}}} \vskip 1em \setcounter{figure}{0}
\setcounter{table}{0} \setcounter{equation}{0} \setcounter{section}{0}
\setcounter{subsection}{0} \setcounter{subsubsection}{0}
\setcounter{footnote}{0} \setcounter{razdel}{0}
%\end{flushleft}
\else {
 \raggedright\noindent\large
 \textrm\textbf
 \lineskip .75em
%\baselineskip=3.2ex
#1 \par} \vskip .5em
%\begin{flushleft}
{\noindent\large\textrm\textbf #2 \par} \vskip .5em
 {\noindent\textrm #3 \par}
\addcontentsline{toc}{subsection}{{\textrm\textbf #1}\protect\newline #2}
\def\rightheadline{\underline{\noindent\hbox to \textwidth{\hfill\small\textrm{#4}
%\hfill \large\bf\thepage
}}}
\def\leftheadline{\underline{\noindent\parbox{\textwidth}{%\raggedleft\large\bf\thepage \hfill
\small\textrm{#5}\hfill}}}
\def\leftfootline{\small{\textbf{\thepage}
\hfill ИНФОРМАТИКА И ЕЁ ПРИМЕНЕНИЯ\ \ \ том~13\ \ \ выпуск~1\ \ \ 2019}
}%
 \def\rightfootline{\small{ИНФОРМАТИКА И ЕЁ ПРИМЕНЕНИЯ\ \ \ том~13\ \ \ выпуск~1\ \ \ 2019
\hfill \textbf{\thepage}}} \vskip 1em \setcounter{figure}{0}
\setcounter{table}{0} \setcounter{equation}{0} \setcounter{section}{0}
\setcounter{subsection}{0} \setcounter{subsubsection}{0}
\setcounter{footnote}{0}
%\end{flushleft}
\fi}

\def\Abst#1{
\begin{center}\small\nwt
\parbox{150mm}{%\baselineskip=2.5ex
\textbf{Аннотация:}\ \
%\hspace*{\parindent}
#1}
\end{center}}
\def\Abste#1{
\begin{center}\small\nwt
\parbox{150mm}{%\baselineskip=2.5ex
\textbf{Abstract:}\ \
%\hspace*{\parindent}
#1}
\end{center}}

\def\DOI#1{
\begin{center}\small\nwt
\parbox{150mm}{%\baselineskip=2.5ex
\textbf{DOI:}\ \
%\hspace*{\parindent}
#1}
\end{center}}

\def\Abstend#1{
\begin{center}\small\nwt
\parbox{150mm}{%\baselineskip=2.5ex
%\hspace*{\parindent}
#1}
\end{center}}


\def\KW#1{
\begin{center}\small\nwt
\parbox{150mm}{%\baselineskip=2.5ex
\textbf{Ключевые слова:}\ \ #1}
\end{center}}

\def\KWE#1{
\begin{center}\small\nwt
\parbox{150mm}{%\baselineskip=2.5ex
\textbf{Keywords:}\ \ #1}
\end{center}}


\def\KWN#1{
%\begin{center}
%\small
%\parbox{150mm}\end{center}
}

\newcommand{\Avtors}[1]{%\smallskip
%\vspace*{.5pt}
\hangindent=23pt\noindent
%\nwt
{\bfseries#1}\
}


\renewcommand{\thesubsection}{\thesection.\arabic{subsection}\hspace*{-5pt}}
\renewcommand{\thesubsubsection}{\thesubsection\hspace*{5pt}.\arabic{subsubsection}\hspace*{-3pt}}

\newcommand{\Ack}{\section*{\protect\rmfamily Acknowledgments}\noindent}
\newcommand{\Contr}{\section*{\protect\rmfamily Contributors}\noindent}
\newcommand{\Contrl}{\section*{\protect\rmfamily Contributor}\noindent}

\makeindex


\begin{document}
\Rus

\nwt
%\ptb


%\renewcommand{\contentsname}{\protect\Large\bf Содержание}

\setcounter{tocdepth}{2}

%\tableofcontents

\renewcommand{\bibname}{\protect\rmfamily Литература}
  \def\Au#1{{\it #1}}
    \def\Aue#1{{#1}}

%\newcommand{\No}{№}
  \newcommand{\tg}{\,\mathrm{tg}\,}
    \newcommand{\ctg}{\,\mathrm{ctg}\,}
  \newcommand{\arctg}{\,\mathrm{arctg}\,}

\def\forallb{\mathop{\forall}}
\def\cupb{\mathop{\cup}}
\def\existsb{\mathop{\exists}}


\newpage
\addtocounter{razdel}{1}
%\def\razd{РЕГУЛИРУЕМЫЙ ЭЛЕКТРОПРИВОД ДЛЯ ЭЛЕКТРОЭНЕРГЕТИКИ}


\setcounter{page}{2}

%   { %\Large  
   { %\baselineskip=16.6pt
   
   \vspace*{-48pt}
   \begin{center}\LARGE
   \textit{Предисловие}
   \end{center}
   
   %\vspace*{2.5mm}
   
   \vspace*{25mm}
   
   \thispagestyle{empty}
   
   { %\small 

    
Вниманию читателей журнала <<Информатика и её применения>> предлагается 
очередной тематический выпуск <<Вероятностно-статистические методы и 
задачи информатики и информационных технологий>>. Предыдущие тематические 
выпуски журнала по данному направлению вышли в 2008~г.\ (т.~2, вып.~2), 
в 2009~г.\ (т.~3, вып.~3) и в 2010~г.\ (т.~4, вып.~2). 

Статьи, собранные в данном журнале, посвящены разработке новых вероятностно-статистических 
методов, ориентированных на применение к решению конкретных задач информатики и информационных 
технологий, а также~--- в ряде случаев~--- и других прикладных задач. Проблематика, охватываемая 
публикуемыми работами, развивается в рамках научного сотрудничества между Институтом проблем 
информатики Российской академии наук (ИПИ РАН) и Факультетом вычислительной математики и 
кибернетики Московского государственного университета им.\ М.\,В.~Ломоносова в ходе работ 
над совместными научными проектами (в том числе в рамках функционирования 
Научно-образовательного центра <<Вероятностно-статистические методы анализа рисков>>). 
Многие из авторов статей, включенных в данный номер журнала, являются активными участниками 
традиционного международного семинара по проблемам устойчивости стохастических моделей, 
руководимого В.\,М.~Золотаревым и В.\,Ю.~Королевым; регулярные сессии этого семинара 
проводятся под эгидой МГУ и ИПИ РАН (в 2011~г.\ указанный семинар проводится в октябре 
в Калининградской области РФ). 

Наряду с представителями ИПИ РАН и МГУ в число авторов данного выпуска журнала входят 
ученые из Научно-исследовательского института системных исследований РАН, Института 
проблем технологии микроэлектроники и особочистых материалов РАН, Института 
прикладных математических исследований Карельского НЦ РАН, Московского 
авиационного института, Вологодского государственного педагогического университета, 
НИИММ им.\ Н.\,Г.~Чеботарева, Казанского государственного университета, Дебреценского 
университета (Венгрия).

Несколько статей выпуска посвящено разработке и применению стохастических методов и 
информационных технологий для решения различных прикладных задач. В~работе В.\,Г.~Ушакова 
и О.\,В.~Шестакова рассмотрена задача определения вероятностных характеристик случайных 
функций по распределениям интегральных преобразований, возникающих в задачах эмиссионной 
томографии. В~статье Д.\,О.~Яковенко и М.\,А.~Целищева рассмотрены некоторые вопросы 
математической теории риска и предложен новый подход к диверсификации инвестиционных 
портфелей. Работа И.\,А.~Кудрявцевой и А.\,В.~Пантелеева посвящена построению и 
исследованию математической модели, описывающей динамику сильноионизованной плазмы. 
В~статье П.\,П.~Кольцова изучается качество работы ряда алгоритмов сегментации изображений. 
Статья А.\,Н.~Чупрунова и И.~Фазекаша посвящена вероятностному анализу числа без\-оши\-бочных 
блоков при помехоустойчивом кодировании; получены усиленные законы больших чисел для указанных 
величин.

В данном выпуске традиционно присутствует тематика, весьма активно разрабатываемая в течение 
многих лет специалистами ИПИ РАН и МГУ,~--- методы моделирования и управления для 
информационно-телекоммуникационных и вычислительных систем, в частности методы 
теории массового обслуживания. В~статье А.\,И.~Зейфмана с соавторами рассматриваются 
модели обслуживания, описываемые марковскими цепями с непрерывным временем в случае 
наличия катастроф. В~работе М.\,М.~Лери и И.\,А.~Чеплюковой рассматриваются случайные 
графы Интернет-типа, т.\,е.\ графы, степени вершин которых имеют степенные распределения; 
такие задачи находят применение при исследовании глобальных сетей передачи данных. 
Работа Р.\,В.~Разумчика посвящена исследованию систем массового обслуживания специального 
вида~--- с отрицательными заявками и хранением вытесненных заявок.

Ряд статей посвящен развитию перспективных теоретических 
вероятностно-статистических методов, которые находят широкое применение в различных 
задачах информатики и информационных технологий. В~работе В.\,Е.~Бенинга, А.\,К.~Горшенина 
и В.\,Ю.~Королева рассмотрена задача статистической проверки гипотез о числе компонент 
смеси вероятностных распределений, приводится конструкция асимптотически наиболее мощного 
критерия. Результаты этой работы найдут применение в ряде прикладных задач, использующих 
математическую модель смеси вероятностных распределений (в информатике, моделировании 
финансовых рынков, физике турбулентной плазмы и~т.\,д.). В~статье В.\,Ю.~Королева, 
И.\,Г.~Шевцовой и С.\,Я.~Шоргина строится новая, улучшенная оценка точности нормальной 
аппроксимации для пуассоновских случайных сумм; как известно, указанные случайные суммы 
широко используются в качестве моделей многих реальных объектов, в том числе в информатике, 
физике и других прикладных областях. Работа В.\,Г.~Ушакова и Н.\,Г.~Ушакова посвящена 
исследованию ядерной оценки плотности распределения; эти результаты могут применяться, 
в част\-ности, при анализе трафика в телекоммуникационных системах. Серьезные приложения 
в статистике могут получить результаты работы О.\,В.~Шестакова, в которой доказаны оценки 
скорости сходимости распределения выборочного абсолютного медианного отклонения к нормальному 
закону. 

\smallskip

Редакционная коллегия журнала выражает надежду, что данный тематический  выпуск 
будет интересен специалистам в области теории вероятностей и математической статистики 
и их применения к решению задач информатики и информационных технологий.
     
     %\vfill 
     \vspace*{20mm}
     \noindent
     Заместитель главного редактора журнала <<Информатика и её 
применения>>,\\
     директор ИПИ РАН, академик  \hfill
     \textit{И.\,А.~Соколов}\\
     
     \noindent
     Редактор-составитель тематического выпуска,\\
     профессор кафедры математической статистики факультета\\
      вычислительной математики и кибернетики МГУ им.\ М.\,В.~Ломоносова,\\
     ведущий научный сотрудник ИПИ РАН,\\ 
доктор физико-математических наук \hfill
      \textit{В.\,Ю.~Королев}
     
     } }
     }

 
\def\stat{sinits}

\def\tit{АНАЛИТИЧЕСКОЕ МОДЕЛИРОВАНИЕ
НОРМАЛЬНЫХ ПРОЦЕССОВ В~СТОХАСТИЧЕСКИХ СИСТЕМАХ СО~СЛОЖНЫМИ~НЕЛИНЕЙНОСТЯМИ}

\def\titkol{Аналитическое моделирование
нормальных процессов в~стохастических системах со~сложными нелинейностями}

\def\aut{И.\,Н.~Синицын$^1$, В.\,И.~Синицын$^2$}

\def\autkol{И.\,Н.~Синицын, В.\,И.~Синицын}

\titel{\tit}{\aut}{\autkol}{\titkol}

\renewcommand{\thefootnote}{\arabic{footnote}}
\footnotetext[1]{Институт проблем
информатики Российской академии наук, sinitsin@dol.ru}
\footnotetext[2]{Институт проблем
информатики Российской академии наук, vsinitsin@ipiran.ru}


\Abst{Рассматриваются конечномерные дифференциальные стохастические системы
(ДСтС) и эредитарные (интегродифференциальные) стохастические системы  (ЭСтС)
с винеровскими и пуассоновскими шумами, приводимые к ДСтС со сложными конечными,
дифференциальными и интегральными нелинейностями. Такие модели функционирования
описывают поведение многих современных нано- и кван\-то\-во-оп\-ти\-че\-ских
технических средств информатики. Приводятся уравнения методов нормальной
аппроксимации (МНА) и статистической линеаризации (МСЛ) для аналитического
моделирования нестационарных и стационарных нормальных (гауссовских) процессов
в нелинейных ДСтС и  нелинейных ЭСтС путем аппроксимации эредитарных ядер
линейными операторными уравнениями для дифференцируемых нелинейностей и
сингулярными ядрами для недифференцируемых нелинейностей. Рассматриваются
методы вычисления типовых интегралов МНА (МСЛ) для сложных (многомерных и
векторного аргумента) конечных и дифференциальных нелинейностей. Особое
внимание уделяется иррациональным и дробно-рациональным нелинейностям
скалярного аргумента. Приводятся примеры вычисления интегралов. Подробно
рассматриваются вопросы вычисления типовых интегралов МНА (МСЛ) для сложных
интегральных нелинейностей.}

\KW{аналитическое моделирование;
дифференциальные стохастические системы с винеровскими и пуассоновскими шумами (ДСтС);
метод нормальной аппроксимации (МНА);
метод статистической линеаризации (МСЛ);
сложные иррациональные нелинейности;
сложные конечные, дифференциальные и интегральные нелинейности;
эредитарные стохастические системы (ЭСтС), приводимые к дифференциальным}

\DOI{10.14357/19922264140302}

\vspace*{9pt}

\vskip 16pt plus 9pt minus 6pt

\thispagestyle{headings}

\begin{multicols}{2}

\label{st\stat}


\section{Введение}


Моделями функционирования многих современных технических сис\-тем информатики
служат стохастические системы (СтС), описываемые дифференциальными, интегральными
и интегродифференциальными уравнениями со сложными дроб\-но-ра\-ци\-о\-наль\-ны\-ми,
иррациональными и интегральными нелинейностями. В~[1] дано систематическое
изложение МНА и МСЛ для ДСтС и ЭСтС, приводимых к дифференциальным.

Обобщая~[2--7], рассмотрим развитие МНА и МСЛ для аналитического моделирования
нормальных стохастических процессов (СтП) на случай СтС со сложными конечными,
дифференциальными и интегральными нелинейностями.

Как показано в~\cite{4-sin}, альтернативным подходом к аналитическому моделированию
СтП в ДСтС и ЭСтС служит подход, основанный на дискретизации стохастических
дифференциальных уравнений на основе использования обобщенной формы Ито и
кратных стохастических интегралов от винеровских и пуассоновских СтП с
последующим применением дискретных версий МНА (МСЛ).

Статья состоит из введения, пяти разделов и заключения.

В~разд.~2 и~3
приводятся уравнения МНА и МСЛ для аналитического моделирования одно- и
двумерных распределений стационарных и нестационарных СтП в ДСтС и ЭСтС,
приводимых к ДСтС.

Типовые интегралы МНА и МСЛ рассматриваются в разд.~4.

Особенности аналитического моделирования в ДСтС со сложными конечными и
дифференциальными нелинейностями обсуждаются в разд.~5.

Раздел~6
посвящен аналитическому моделированию СтП в ДСтС со сложными интегральными
нелинейностями.

Приводятся примеры.


\section{Уравнения методов нормальной~аппроксимации и~статистической
линеаризации для~дифференциальных стохастических систем}

Как известно~\cite{2-sin, 3-sin},  уравнения конечномерных непрерывных нелинейных сис\-тем
со стохастическими возмущениями путем расширения вектора состояния ДСтС
могут быть записаны в виде следующего векторного стохастического
дифференциального уравнения Ито:
    \begin{multline}
    dY_t = a(Y_t, t)\, dt + b (Y_t, t) \,dW_0+{}\\
    {}+ \iii_{R_0} c (Y_t, t, v) P^0
    (dt, dv)\,,\enskip Y(t_0) = Y_0\,.\label{e2.1-sin}
    \end{multline}
Здесь $a=a(Y_t, t)$ и $b\hm=b(y_t, t)$~--- известные
$(p\times 1)$-мер\-ная и  $(p\times m)$-мер\-ная функции~$Y_t$ и~$t$;
$W_0\hm= W_0(t)$~--- $r$-мер\-ный винеровский СтП интенсивности
$\nu_0 \hm= \nu_0(t)$; $c(Y_t, t, v)$~--- $(p\times 1)$-мер\-ная функция  $Y_t, t$
и вспомогательного $(q\times 1)$-мер\-но\-го па\-ра\-мет\-ра~$v$;
$\iii_{\Delta} dP^0 (t, A)$~--- центрированная пуассоновская мера,
определяемая
\begin{equation*}
\iii_{\Delta} dP^0 (t, A) = \iii_{\Delta} dP (t,A) =
\iii_{\Delta} \nu_P (t,A)\, dt\,. %\label{e2.2-sin}
\end{equation*}
В~(\ref{e2.1-sin}) принято: $\iii_{\Delta}$~-- число скачков пуассоновского
СтП в интервале времени  $\Delta \hm= (t_1, t_2]$; $\nu_P (t, A)$~---
интенсивность пуассоновского СтП  $P(t,A)$; $A$~--- некоторое борелевское
множество пространства  $R_0^q$ с выколотым началом.
Начальное значение~$Y_0$ представляет собой случайную величину, не зависящую
от приращений СтП  $W_0(t)$ и $P(t,A)$ на интервалах времени, следующих
за~$t_0$, $t_0 \hm\le t_1\hm\le t_2$ для любого множества~$A$.

В случае аддитивных нормальных (гауссовских) и обобщенных
пуассоновских возмущений уравнение~(\ref{e2.1-sin}) имеет вид:
\begin{equation}
\dot Y_t = a(Y_t,t)+ b_0 (t) V\,, \enskip
V = \dot W\,,\enskip Y(t_0) = Y_0\,.\label{e2.3-sin}
\end{equation}
Здесь $W$~--- СтП с независимыми приращениями, представляющий собой
смесь нормального и обобщенного пуассоновского СтП.

Если предположить существование конечных вероятностных
моментов второго порядка для моментов времени~$t_1$ и~$t_2$, то уравнения
МНА примут следующий вид~\cite{2-sin, 3-sin}:
\begin{itemize}
\item  для характеристических функций
    \begin{equation}
    g_1^N (\la;t) =\exp \lk i\la^{\mathrm{T}} m_t - \fr{1}{2}\, \la^{\mathrm{T}} K_t \la\rk\,;\label{e2.4-sin}
    \end{equation}
\begin{equation}
\hspace*{-7.5mm}g_{t_1, t_2}^N (\la_1, \la_2;t_1, t_2 ) =\exp \lk i\bar \la^{\mathrm{T}} \bar m_2 -
\fr{1}{2}\, \bar \la^{\mathrm{T}} \bar K_2 \la\rk\,,\!\!\label{e2.5-sin}
\end{equation}
где
    \begin{gather*}
    \bar \la =\lk \la_1^{\mathrm{T}}\la_2^{\mathrm{T}}\rk^{\mathrm{T}}\,; \quad
        \bar m_2 = \lk m_{t_1}^{\mathrm{T}} m_{t_2}^{\mathrm{T}}\rk^{\mathrm{T}}\,;\\
        \bar K_2= \begin{bmatrix}
    K(t_1, t_1)& K(t_1, t_2)\\
    K(t_2, t_1)& K(t_2, t_2)
    \end{bmatrix}\,;
    \end{gather*}

\item для математических ожиданий  $m_t$, ковариационной матрицы~$K_t$ и
матрицы ковариационных функций $K(t_1, t_2)$:
    \begin{equation}
    \dot m_t = a_1 (m_t, K_t, t)\,,\enskip m_0 = m(t_0)\,;\label{e2.6-sin}
    \end{equation}
\begin{equation}
\dot K_t = a_2 (m_t, K_t, t)\,,\enskip K_0 = K(t_0)\,;\label{e2.7-sin}
\end{equation}

\vspace*{-12pt}

\noindent
\begin{multline}
\fr{\prt K(t_1, t_2)}{\prt t_2 }= K(t_1, t_2) a_{21} (m_{t_2}, K_{t_2}, t_2)^{\mathrm{T}}\,;\\
K(t_1, t_1) = K_{t_1}\,.
\label{e2.8-sin}
\end{multline}
    \end{itemize}
Здесь приняты следующие обозначения:
\begin{equation}
a_1 = a_1 (m_t, K_t, t) = M_N a (Y_t, t)\,;\label{e2.9-sin}
\end{equation}

\vspace*{-12pt}

\noindent
\begin{multline}
a_2 = a_2 (m_t, K_t, t) = a_{21} (m_t, K_t, t)+{}\\
{}+ a_{21} (m_t, K_t, t)^{\mathrm{T}} +
a_{22}(m_t, K_t, t)\,;\label{e2.10-sin}
\end{multline}

\vspace*{-12pt}

\noindent

\begin{equation}
a_{21} = a_{21}(m_t, K_t, t)=  M_N a(Y_t, t) Y_{t}^{0\mathrm{T}}\,;\label{e2.11-sin}
\end{equation}
\begin{equation*}
a_{22} = a_{22}(m_t, K_t, t)= M_N \sigma (Y_t, t)\,;
%\label{e2.12-sin}
\end{equation*}

\vspace*{-12pt}

\noindent
\begin{multline*}
\sigma (Y_t, t) = b(Y_t, t) \nu_0(t) b(Y_t, t)^{\mathrm{T}} +{}\\
{}+
\iii_{R_0^q} c (Y_t, t, v) c(Y_t, t,v)^{\mathrm{T}}
\nu_P (t, dv)\,; %\label{e2.13-sin}
\end{multline*}

\vspace*{-12pt}

\begin{gather*}
m_t = MY_t\,,\quad Y_t^0 = Y_t - m_t\,,\\
K_t = M_N Y_0^0 Y_t^{0\mathrm{T}}\,,\quad K(t_1, t_2) =
M_N Y_{t_1}^0 Y_{t_2}^0\,; %\label{e2.14-sin}
\end{gather*}
$M_N$~--- символ вычисления математического ожидания для нормальных
распределений~(\ref{e2.4-sin}) и~(\ref{e2.5-sin}).

Для стационарных ДСтС нормальные стационарные СтП~--- если они существуют,
то  $m_t \hm=\bar m$, $ K_t \hm=\bar K$, $K(t_1, t_2) \hm= k(\tau)$
$(\tau \hm= t_1\hm-t_2)$,~--- определяются уравнениями~\cite{2-sin, 3-sin}:
   \begin{equation}
   a_1 (\bar m, \bar K) =0\,;\enskip a_2 (\bar m, \bar K)=0\,;\label{e2.15-sin}
   \end{equation}
   \begin{equation}
   \left.
   \hspace*{-2.8mm}\begin{array}{l}
  \dot k_\tau (\tau) = a_{21} (\bar m, \bar K)\bar K^{-1} k(\tau)\,;\\[9pt]
  k(0) =\bar K \enskip (\forall \tau >0)\,, \
  k(\tau) = k(-\tau)^{\mathrm{T}} \enskip
  (\forall\tau <0)\,.
  \end{array}\!\!
  \right\}\!\!
  \label{e2.16-sin}
  \end{equation}
При этом необходимо, чтобы матрица  $a_{21} (\bar m, \bar K)\hm=\bar a_{21}$
была бы асимптотически устойчивой.

Для ДСтС~(\ref{e2.3-sin}) уравнения МНА переходят в уравнения МСЛ
Казакова~\cite{2-sin, 3-sin}, если принять
\begin{equation}
a(Y_t,t) = a_1 (m_t, K_t) + k_1^a (m_t, K_t) Y_t^0\,;\label{e2.17-sin}
\end{equation}
\begin{equation}\left.
\begin{array}{rl}
b(Y_t,t) &= b_0 (t)\,;\\[9pt]
    \si(Y_t, t)&= b_0(t) \nu(t) b_0(t)^{\mathrm{T}} = \si_0(t)\,,
    \end{array}
    \right\}\label{e2.18-sin}
    \end{equation}
    \begin{equation}
k_1^a (m_t, K_t, t) =\lk \left(\fr{\prt}{\prt m_t} \right)
    a_0 (m_t, K_t, t)^{\mathrm{T}}\rk^{\mathrm{T}}\,;\label{e2.19-sin}
    \end{equation}
    \begin{equation}
\dot m_t = a_1 (m_t, K_t, t) \,,\enskip m_0 = m(t_0)\,,\label{e2.20-sin}
\end{equation}

\vspace*{-12pt}

\noindent
\begin{multline}
\dot K_t = k_1^a (m_t, K_t, t) K_t + K_t k_1^a (m_t, K_t, t)^{\mathrm{T}}
    +\si_0(t)\,;\\
    K_0 = K(t_0)\,;
    \label{e2.21-sin}
    \end{multline}

    \vspace*{-12pt}

    \noindent
\begin{multline}
\fr{\prt K(t_1, t_2)}{\prt t_2} =
    K(t_1, t_2) k_{t_2} k_1^a (m_{t_2}, K_{t_2}, t_2)^{\mathrm{T}}\,;\\
    K(t_1, t_2) = K_{t_1}\,.
    \label{e2.22-sin}
\end{multline}

Для стационарных ДСтС~(\ref{e2.3-sin})
при условии асимптотической устойчивости матрицы $k_1^a (\bar m, \bar K)$
в основе МСЛ лежат уравнения~(\ref{e2.15-sin}), записанные в виде:
    \begin{gather}
    a_1 (\bar m, \bar K) =0\,; \label{e2.23-sin}\\
k_1^a (\bar m, \bar K) \bar K + \bar K k_1^a
(\bar m, \bar K)^{\mathrm{T}} +\bar \si_0 =0\,;\label{e2.24-sin}
\end{gather}

\vspace*{-12pt}

\noindent
\begin{multline}
k_\tau (\tau) = k_1^a (\bar m, \bar K)k(\tau)\,,\enskip
k(0) =\bar K \enskip (\forall \tau >0)\,,\\
k(\tau) = k (-\tau)^{\mathrm{T}} \enskip (\forall \tau <0)\,.
\label{e2.25-sin}
\end{multline}

Уравнения~(\ref{e2.4-sin})--(\ref{e2.8-sin})
лежат в основе МНА для ДСтС~(\ref{e2.1-sin}), а уравнения~(\ref{e2.17-sin})--(\ref{e2.22-sin})~---
в основе МСЛ для ДСтС~(\ref{e2.3-sin}). Для определения стационарных СтП
согласно МНА служат соотношения~(\ref{e2.15-sin}) и~(\ref{e2.16-sin}),
а МСЛ~--- (\ref{e2.17-sin})--(\ref{e2.25-sin}).

\section{Уравнения методов нормальной~аппроксимации и~статистической линеаризации
для~эредитарных стохастических систем, приводимых к~дифференциальным}

Рассмотрим ЭСтС, описываемую интегродифференциальным уравнением Ито
следующего вида~\cite{7-sin}:

\noindent
\begin{multline}
dX_t = \lk a(X_t, t) +\iii_{t_0}^t a_1 (X(\tau) ,\tau, t)\,d\tau\rk dt+{}\\
{}+\lk b(X_t, t) +\iii_{t_0}^t b_1 (X(\tau) ,\tau, t)\,d\tau\rk dW_0+{}\\
\hspace*{-1.5mm}{}+\!\!\iii_{R_0^q}\!\!\lk c(X_t, t,v) +\!\iii_{t_0}^t\! c_1 (X(\tau) ,\tau, t,v)\,d\tau\!\rk\! dP^0 (t, dv)
\!\!\!\!\label{e3.1-sin}
\end{multline}
с начальным условием  $X(t_0) = X_0$. В~(\ref{e3.1-sin})
сохранены обозначения разд.~2.

Функции $a=a(X_t, t)$, $a_1\hm = a_1(X (\tau),\tau, t)$,
$b\hm=b(X_t, t)$, $b_1\hm = b_1(X (\tau),\tau, t)$,
$c\hm=c(X_t,t,v)$ и $c_1\hm = c_1(X (\tau),\tau, t,v)$ имеют
соответственно размерности $p\times 1$, $p\times 1$, $p\times r$,
$p\times r$, $p\times 1$ и $p\times 1$ и допускают представления следующего вида:
\begin{equation}
\left.
\begin{array}{rl}
a_1&=A(t,\tau) \vrp (X(\tau) , \tau)\,;\\[9pt]
b_1&=B(t,\tau) \psi (X(\tau) ,  \tau)\,;\\[9pt]
c_1&=C(t,\tau) \chi (X(\tau) ,  \tau, v)\,.
\end{array}
\right\}
\label{e3.2-sin}
\end{equation}
Здесь эредитарные ядра $A\hm=A(t,\tau)\hm=\lk A_{ij}(t,\tau)\rk$
$(i,j\hm=\overline{1,p})$,
$B\hm=B(t,\tau)=\lk B_{i l}(t,\tau)\rk$ $(i\hm=\overline{1,p}$;
$l\hm=\overline{1,r})$ и $C\hm=C(t,\tau)=\lk C_{ij}(t,\tau)\rk$
$(i,j\hm=\overline{1,p})$ имеют соответственно размерности
$p\times p$, $p\times r$ и $p\times p$. Они удовлетворяют следующим условиям
физической реализуемости и асимптотического затухания:
\begin{multline}
A_{ij}(t,\tau)=0;\enskip B_{i l}(t,\tau)=0;\\[1pt]
C_{ij}(t,\tau)=0\enskip \forall \tau >t;\label{e3.3-sin}
\end{multline}

\vspace*{-12pt}

\begin{equation}
\left.
\hspace*{-3mm}\begin{array}{c}
\displaystyle\iin\! \lv A_{ij} (t,\tau) \rv d\tau <\infty\,;\
\displaystyle\iin\! \lv B_{i l} (t,\tau) \rv d\tau <\infty \,;\\[9pt]
\displaystyle\iin \!\lv C_{ij} (t,\tau) \rv d\tau <\infty\,.
\end{array}\!
\right\}\!
\label{e3.4-sin}
\end{equation}

В дальнейшем ограничимся случаем, когда эредитарные ядра удовлетворяют
линейным операторным уравнениям~\cite{6-sin, 5-sin, 7-sin}.

Нелинейные в общем случае функции $\vrp\hm=\vrp(X(\tau),\tau)$,
$\psi \hm=\psi(X(\tau), \tau)$, $\chi \hm=\chi (X(\tau),  \tau, v)$
отражают нелинейные свойства ЭСтС, зависят от  $X(\tau)$ и имеют размерности
$p\times 1$, $p\times p$, $p\times 1$ соответственно.

Важный класс  эредитарных ядер представляют собой
сингулярные (вырожденные) ядра, когда имеют место представления:
\begin{equation}
\left.
\hspace*{-3mm}\begin{array}{rl}
A_{ij} (t,\tau) &= A_{ij}^+(t) A_{ij}^-(\tau)\,;\\[9pt]
B_{i l} (t,\tau)& = B_{il}^+(t) B_{il}^-(\tau)\,;\\[9pt]
C_{ij} (t,\tau) &= C_{ij}^+ ( t) C_{ij}^- (\tau)\
(i,l= \overline{1,p}, j=\overline{1,r}).
\end{array}\!
\right\}\!\!
\label{e3.5-sin}
\end{equation}

В~\cite{6-sin, 5-sin, 7-sin} показано, что для дифференцируемых нелинейных
функций~$\vrp$, $\psi$, $\chi$ путем расширения вектора состояния за счет
инструментальных переменных, аппроксимируемых линейными операторными уравнениями,
определяющими эредитарные ядра в ЭСтС, (\ref{e3.1-sin})--(\ref{e3.4-sin})
приводятся к ДСтС вида~(\ref{e2.1-sin}) или~(\ref{e2.3-sin}).
В~случае недифференцируемых нелинейных функций~$\vrp$, $\psi$, $\chi$
ЭСтС~(\ref{e3.1-sin})--(\ref{e3.4-sin}) приводятся к~(\ref{e2.1-sin}) или~(\ref{e2.3-sin})
на основе аппроксимации вырожденными (сингулярными) ядрами~\cite{6-sin, 5-sin, 7-sin}.

Таким образом, после приведения ЭСтС~(\ref{e3.1-sin}) к ДСтС~(\ref{e2.1-sin})
или~(\ref{e2.3-sin}) можно воспользоваться уравнениями МНА и МСЛ разд.~2.

\section{Типовые интегралы методов нормальной аппроксимации и~статистической
линеаризации}

Как следует из уравнений~(\ref{e2.9-sin})--(\ref{e2.11-sin}),
для МНА необходимо уметь вычислять следующие интегралы:
\begin{multline}
I_0^a = I_0^a (m_t, K_t, t) = a_1 (m_t, K_t, t)={}\\
{}= M_N a(Y_t, t)\,;
\label{e4.1-sin}
\end{multline}

\vspace*{-12pt}

\noindent
\begin{multline}
I_1^a = I_1^a (m_t, K_t, t)= a_{21}(m_t, K_t, t)= {}\\
{}=M_N a(Y_t , t) Y_t^{0\mathrm{T}}\,;\label{e4.2-sin}
\end{multline}

\vspace*{-12pt}

\noindent
\begin{multline}
I_0^{\bar \si} = I_0^{\bar \si} (m_t, K_t, t) = a_{22}(m_t, K_t, t) ={}\\
{}= M_N \bar \si (Y_t, t)\,.\label{e4.3-sin}
\end{multline}
Для МСЛ достаточно вычислить интеграл~(\ref{e4.1-sin}),
причем интеграл~$I_1^a$ вычисляется по формуле~\cite{2-sin, 3-sin, 4-sin}:
\begin{equation*}
k_1^a = k_1^a (m_t, K_t, t)=\lk \left( \fr{\prt}{\prt m_t}\right)
I_0^a (m_t, K_t, t)^{\mathrm{T}}\rk^{\mathrm{T}}. %\label{e4.4-sin}
\end{equation*}

\medskip

\noindent
\textbf{Пример 1.} В~[1] для типовых степенных, тригоно\-мет\-ри\-че\-ских,
показательных и ку\-соч\-но-по\-сто\-ян\-ных нелинейностей $Z_t \hm=\vrp (Y_t, t)$
скалярного и векторного аргумента приведены формулы для интегралов
$I_0^\vrp \hm= I_0^\vrp (m_t^y, K_t^y, t)$, а также
$k_1^\vrp \hm= k_1^\vrp (m_t^y, K_t^y, t)$.

\medskip

\noindent
\textbf{Замечание.}
 Важно иметь в виду, что уравнения МНА (МСЛ) содержат интегралы
 $I_0^a$, $I_1^a$, $I_0^\si$ в виде соответствующих коэффициентов.
 Поэтому процедура вычисления интегралов должна быть согласована с
 методом численного решения обыкновенных дифференциальных уравнений для
 $m_t$, $K_t$ и $K(t_1, t_2)$. Эти коэффициенты допускают дифференцирование
 по~$m_t$ и~$K_t$, так как под интегралом стоит сглаживающая нормальная плотность.

\section{Сложные конечные и~дифференциальные нелинейности}

Важный класс сложных конечных нелинейностей (многомерных и векторного аргумента)
представляют собой сложные функции вида:
    \begin{equation*}
    \xi =\vrp (X_t, Y_t, t)\,,\enskip X_t =\psi (Y_t, t)\,. %\label{e5.1-sin}
    \end{equation*}
В~этом случае вычисление интегралов (см.\ разд.~4) проводится по совокупности
переменных  $\lk X_t^{\mathrm{T}} Y_t^{\mathrm{T}}\rk^{\mathrm{T}}$.
К таким нелинейностям, например, относятся дроб\-но-ра\-ци\-о\-наль\-ные,
иррациональные  нелинейности, выражаемые специальными функциями, многозначные
нелинейности, зависящие от СтП~$X_t$ и его производных~$\dot X_t$,  $\ddot X_t$
и~др.

\medskip

\noindent
\textbf{Пример 2.}
Рассмотрим вычисление интегралов~(\ref{e4.1-sin}) и~(\ref{e4.2-sin})
для сложных одномерных иррациональных нелинейностей скалярного аргумента
\begin{equation}
\vrp (Y_t, t) =\lv Y_t\rrv^{\alpha-1}\, \mathrm{sgn}\, Y_t
\label{e5.2-sin}
\end{equation}
($\alpha$~--- нецелый показатель).

Пользуясь~(\ref{e2.16-sin}) и~(\ref{e2.19-sin}), представим~(\ref{e5.2-sin}) в виде
\begin{equation*}
\vrp(Y_t, t) = \vrp_0 (m_t, D_t, t) + k_1^\vrp(m_t, D_t, t) Y_t^0. %\label{e5.3-sin}
\end{equation*}
Здесь введены следующие обозначения:
\begin{gather*}
\vrp_0(m_t, D_t, t) =\Gamma(\alpha) D_t^{1/2} e^{-\xi^2/4} D_{-\alpha} (\xi)\,;%\label{e5.4-sin}
\\
k_1^a (m_t, D_t, t) =\fr {\prt \vrp_0(m_t, D_t, t)}{\prt m_t}\,,%\label{e5.5-sin}
\end{gather*}
где  $\Gamma(\alpha)$~--- гамма-функция,  $\xi \hm= m_t/\sqrt{D_t}$~---
отношение <<сиг\-нал--шум>>; $D_{-\alpha} (\xi)$~---
функция параболического цилиндра~\cite{9-sin}.
При вычислении были учтены следующие соотношения~\cite{9-sin, 8-sin}:
\begin{multline}
\iii_0^\infty x^{\alpha-1} e^{-\beta x^2 - \gamma x} \,dx ={}\\
{}=
(2\beta)^{-\alpha/2} \Gamma(\alpha) \exp \left(\fr{\gamma^2}{8\beta}\right)
D_{-\alpha} \left(\fr{\gamma}{\sqrt{2\beta}}\right)\,;\label{e5.6-sin}
\end{multline}

\vspace*{-12pt}

\noindent
\begin{multline}
\fr{dD_\rho(\xi)}{d\xi} =
   -\fr{\xi}{2}\, D_\rho (\xi) -\rho D_{\rho-1} (\xi) =
   \fr{\xi}{2}\, D_\rho (\xi) -{}\\
   {}- D_{\rho+1} (\xi) \enskip
   (\mathrm{Re}\, \beta>0\,,\enskip \mathrm{Re}\,\alpha>0\,,\enskip
   \rho=-\alpha)\,.\label{e5.7-sin}
   \end{multline}

Соотношения~(\ref{e5.6-sin}) и~(\ref{e5.7-sin})
могут быть использованы также для вычисления интегралов~(\ref{e4.3-sin}).

\medskip

\noindent
\textbf{Замечание.}
Для вычисления интегралов $I_0^a$, $I_1^a$ и $I_0^{\bar \si}$
применительно к типовым иррациональным нелинейностям вида
    $\lv Y_t\rrv^{\alp-1} e^{\delta Y_t}$, $\lv Y_t\rrv^{\alp-1}  \cos \w Y_t$,
    $\lv Y_t\rrv^{\alp-1}  \sin \w Y_t$
и более общим нелинейностям \mbox{вида}
    \begin{equation*}
    \vrp (Y_t, t) =\Phi^\vrp \left( \lv Y_t\rrv^{\alpha-1}, t\right) %\label{e5.8-sin}
    \end{equation*}
можно рекомендовать известные численные методы вычисления функций на ЭВМ~\cite{8-sin}.

\smallskip

\noindent
\textbf{Пример 3.}
Для нелинейной дроб\-но-ра\-ци\-о\-наль\-ной функции

\noindent
\begin{equation*}
\vrp (Y_t, t) = \fr{a}{(b+Y_t)^2} %\label{e5.9-sin}
\end{equation*}
имеем

\vspace*{-3pt}

\noindent
\begin{gather*}
\vrp_0 (m_t, D_t, t) =a b^{-2} \lk 1+ \chi (m_t, D_t, t)\rk\,; %\label{e5.10-sin}
\\
k_1^\vrp (m_t, D_t, t) =  a b^{-2}\fr{\prt \chi (m_t, D_t, t)}{\prt m_t}\,. %\label{e5.11-sin}
\end{gather*}
Здесь

\vspace*{-3pt}

\noindent
\begin{multline*}
\chi (m_t, D_t, t) ={}\\
{}=\sss_{n=1}^\infty \sss_{l=0}^{E(n/2)}
\fr{(-1)^n (n+1) n!}{(n-2l)! (2l)!}\, b^{-n} m_t^n \left( \fr{D_t}{ 2 m_t^2}
\right)^l, %\label{e5.12-sin}
\end{multline*}
где  $E(n/2)$~--- целая часть~$n/2$; $a\hm=a(t)$; $b\hm= b(t)$.

\vspace*{-6pt}

\section{Сложные интегральные нелинейности}

\vspace*{-2pt}

Пусть сначала векторно-матричная нелинейность имеет эредитарный характер, т.\,е.\
\begin{equation}
\underline{\vrp} (Y_t, t) =\iii_{t_0}^t A(t,\tau) \vrp (Y(\tau), \tau) \,d\tau\,.
\label{e6.1-sin}
\end{equation}
Тогда, как показано в~\cite{6-sin, 5-sin, 7-sin}, следует соответст\-ву\-ющие
интегродифференциальные соотношения путем введения  инструментальных
переменных привести к дифференциальным соотношениям.  Для
дифференцируемых функций~$\vrp$ и асимптотически устойчивых ядер
$A(t,\tau)$ зависимость~(\ref{e3.5-sin}) имеет следующий дифференциальный вид:
\begin{equation*}
F^A (t, D) \underline{\vrp} (Y_t, t) = H^A (t, D) \vrp (Y_t, t)\,. %\label{e6.2-sin}
\end{equation*}
Здесь $F^A (t, D)$ и  $H^A (t, D)$~--- линейные дифференциальные операторы $(D\hm= d/dt)$.

Для недифференцируемых функций~$\vrp$ и асимптотически устойчивых
сингулярных ядер~(\ref{e3.5-sin}) используются соотношения:
\begin{equation*}
\underline{\vrp} (Y_t, t) = A^+ Z\,,\enskip
\dot Z = A^- \vrp\,,\enskip
Z(t_0)=0\,. %\label{e6.3-sin}
\end{equation*}

Многочисленные примеры аналитического моделирования ЭСтС можно найти
в~[1--3, 5, 7, 10, 11].

Как отмечалось в~\cite{3-sin}, часто наряду с интегральными
нелинейностями~(\ref{e6.1-sin}) рассматривают нелинейности вида:

\columnbreak

\noindent
\begin{equation*}
Z_s =\sss_{\rho=1}^R \mathcal{A}_\rho \vrp_\rho (Y_{t_1}\tr Y_{t_r})\,, %\label{e6.2-sin}
\end{equation*}
где $\mathcal{A}_1 \tr \mathcal{A}_R$~--- произвольные линейные операторы,
действующие над функциями~$r$ переменных  $t_1\tr t_r$; $\vrp_\rho
\hm=\vrp_\rho (Y_{t_1} \tr Y_{t_r})$~--- линейные функции отмеченных
переменных. Такие нелинейности называются приводимыми к линейным.
Важным частным случаем~(\ref{e6.1-sin}) являются интегральные нелинейности вида:

\noindent
\begin{gather}
Z_s =\iii_T \vrp (Y_t, t, s)\, dt\,; \notag%\label{e6.3-sin}
\\
Z_s =\!\iii_T \!\cdots\!\iii_T\! \vrp (Y_{t_1}\tr Y_{t_r}; t_1\tr t_r, s)\,dt_1
\ldots dt_r,\notag %\label{e6.4-sin}
\end{gather}
В этом случае используется МСЛ по совокупности переменных  $Y_{t_1} \tr Y_{t_r}$.

\vspace*{-9pt}

\section{Заключение}

\vspace*{-2pt}

Разработаны методы и алгоритмы МНА и МСЛ для ДСтС и ЭСтС,
приводимых к ДСтС со сложными конечными, дроб\-но-ра\-ци\-о\-наль\-ны\-ми,
иррациональными, а также дифференциальными и интегральными нелинейностями.
Приведены примеры.

Результаты допускают обобщение на случай ДСтС и ЭСтС со
стохастическими нелинейностями, заданными каноническими разложениями и
интегральными каноническими  представлениями~\cite{1-sin, 3-sin, 11-sin}.

\vspace*{-9pt}

{\small\frenchspacing
 {%\baselineskip=10.8pt
 \addcontentsline{toc}{section}{References}
 \begin{thebibliography}{99}

 \vspace*{-2pt}

\bibitem{1-sin}
\Au{Синицын И.\,Н.,  Синицын~В.\,И.}
Лекции по нормальной и эллипсоидальной аппроксимации распределений в
стохастических сис\-те\-мах.~--- М.: ТОРУС ПРЕСС, 2013. 488~с.

\bibitem{6-sin} %2
\Au{Синицын И.\,Н. }
Stochastic hereditary control systems~// Проблемы управления и
теории информации, 1986. Т.~15. №\,4. С.~287--298.

\bibitem{2-sin} %3
\Au{Пугачев В.\,С., Синицын~И.\,Н.}
Стохастические дифференциальные сис\-те\-мы. Анализ и фильтрация.~--- М.:
Наука,  1990.  632~с. [Англ. пер.
 Stochastic differential systems.
Analysis and filtering.~--- Chichester, New York: Jonh Wiley, 1987.
549~p.].

\bibitem{5-sin} %4
\Au{Синицын И.\,Н. }
Конечномерные распределения процессов в стохастических интегральных
и интегродифференциальных системах~// Preprints of the 2nd IFAC
Symposium on Stochastic Control.~--- Vilnius: Pergamon Press,
1987.  Vol.~1. P.~144--153.

\bibitem{3-sin} %5
\Au{Пугачев В.\,С., Синицын~И.\,Н.}
Теория стохастических систем.~--- М.: Логос, 2000; 2004. 1000~с.
[Англ. пер.\linebreak\vspace*{-12pt}

\pagebreak

\noindent Stochastic systems. Theory and  applications.~---
Singapore: World Scientific, 2001. 908~p.].

\bibitem{4-sin} %6
\Au{Синицын И.\,Н.}
Параметрическое статистическое и аналитическое моделирование распределений
в нелинейных стохастических сис\-те\-мах на многообразиях~//
Информатика и её применения, 2013. Т.~7. Вып.~2. С.~4--16.

\bibitem{7-sin} %7
\Au{Синицын И.\,Н. }
Анализ и моделирование распределений в эредитарных стохастических
сис\-те\-мах~// Информатика и её применения, 2014. Т.~8. Вып.~1.\linebreak
С.~2--11.



\bibitem{9-sin} %8
\Au{Градштейн И.\,С., Рыжик~И.\,М.}
Таблицы интегралов, сумм, рядов и произведений.~--- М.: ГИФМЛ, 1963. 1100~с.

\bibitem{8-sin} %9
\Au{Попов Б.\,А., Теслер~Г.\,С. }
Вычисление функций на ЭВМ: Справочник.~--- Киев: Наукова Думка, 1984. 599~с.


\bibitem{11-sin} %10
\Au{Синицын И.\,Н.}
Канонические представления случайных функций и их применение в
задачах компьютерной поддержки научных исследований.~--- М.: ТОРУС
ПРЕСС, 2009. 768~с.

\bibitem{10-sin} %11
\Au{Синицын И.\,Н., Синицын~В.\,И., Корепанов~Э.\,Р., Белоусов~В.\,В.,
Сергеев~И.\,В., Басилашвили~Д.\,А.}
Опыт моделирования эредитарных стохастических сис\-тем~//
Кибернетика и высокие технологии XXI века: Сб. докл.  XIII Междунар.
науч.-технич. конф.~--- Воронеж: Саквоее, 2012. Т.~2. C.~346--357.

 \end{thebibliography}

 }
 }

\end{multicols}

\vspace*{-9pt}

\hfill{\small\textit{Поступила в редакцию 05.05.14}}

%\newpage

\vspace*{12pt}

\hrule

\vspace*{2pt}

\hrule

\vspace*{12pt}

\def\tit{ANALYTICAL MODELING OF NORMAL PROCESSES
 IN~STOCHASTIC SYSTEMS WITH~COMPLEX NONLINEARITIES}

\def\titkol{Analytical modeling of normal processes
 in~stochastic systems with~complex nonlinearities}

\def\aut{I.\,N.~Sinitsyn and V.\,I.~Sinitsyn}

\def\autkol{I.\,N.~Sinitsyn and V.\,I.~Sinitsyn}

\titel{\tit}{\aut}{\autkol}{\titkol}

\vspace*{-9pt}

\noindent
Institute of Informatics Problems, Russian Academy of Sciences,
44-2 Vavilov Str., Moscow 119333, Russian Federation


\def\leftfootline{\small{\textbf{\thepage}
\hfill INFORMATIKA I EE PRIMENENIYA~--- INFORMATICS AND
APPLICATIONS\ \ \ 2014\ \ \ volume~8\ \ \ issue\ 3}
}%
 \def\rightfootline{\small{INFORMATIKA I EE PRIMENENIYA~---
INFORMATICS AND APPLICATIONS\ \ \ 2014\ \ \ volume~8\ \ \ issue\ 3
\hfill \textbf{\thepage}}}

\vspace*{6pt}

\Abste{Differential stochastic systems (DStS) with Wiener and Poisson
noises and complex finite, differential, and  integral nonlinearities and
hereditary StS reducible to DStS are considered. Equations and algorithms
of analytical modeling based on the normal approximation method (NAM) and the
statistical linearization method (SLM) are given. The case of complex
continuous and discontinuous nonlinearities of scalar and vector arguments
is considered. Special attention is paid to NAM (SLM) typical integrals
for finite rational and irrational nonlinear and integral scalar and vector
nonlinear functions. The general case of integral nonlinearities reducible to
linear is considered. Test examples are given.}

\KWE{analytical modeling;
complex finite differential and integral nonlinearities;
complex irrational nonlinerarites
differential stochastic system with Wiener and Poisson noises;
method of normal approximation;
method of statistical linearization;
hereditary stochastic systems reducible to differential}

\DOI{10.14357/19922264140302}

  \begin{multicols}{2}

\renewcommand{\bibname}{\protect\rmfamily References}
%\renewcommand{\bibname}{\large\protect\rm References}

{\small\frenchspacing
 {%\baselineskip=10.8pt
 \addcontentsline{toc}{section}{References}
 \begin{thebibliography}{99}



\bibitem{1-sin-1}
\Aue{Sinitsyn, I.\,N., and  V.\,I.~Sinitsyn}.  2013.
Lektsii po normal'noy i ellipsoidal'noy approksimatsii raspredeleniy
v stokhasticheskikh sistemakh [Lectures on normal and ellipsoidal
approximation of distributions in stochastic systems].
Moscow: TORUS PRESS. 488~p.

\bibitem{6-sin-1} %2
\Aue{Sinitsyn, I.\,N.}  1986.
{Stochastic hereditary control systems}.
\textit{Problems Control Inform. Theory} 15(4):287--298.

\bibitem{2-sin-1} %3
\Aue{Pugachev, V.\,S., and  I.\,N.~Sinitsyn}.  1987.
\textit{Stochastic differential systems. Analysis and filtering.}
Chichester, New York: Jonh Wiley. 549~p.

\bibitem{5-sin-1} %4
\Aue{Sinitsyn, I.\,N.}  1987.
Konechnomernye raspredeleniya protsessov v stokhasticheskikh integral'nykh
i in\-teg\-ro\-dif\-fe\-ren\-tsial'nykh sistemakh [Finite dimensional distributions
of processes in stochastic integral and integrodifferential systems].
\textit{2nd  Symposium (International) IFAC on Stochastic Control
Preprints}. Vilnius: Pergamon Press. 1:144--153.

\bibitem{3-sin-1} %5
\Aue{Pugachev, V.\,S., and I.\,N.~Sinitsyn}. 2001.
\textit{Stochastic systems. Theory and  applications}.
Singapore: World Scientific. 908~p.

\bibitem{4-sin-1} %6
\Aue{Sinitsyn, I.\,N.}  2013.
Parametricheskoe statisticheskoe i analiticheskoe modelirovanie
raspredeleniy v nelineynykh stokhasticheskikh sistemakh na mnogoobraziyakh
[Parametric statistical and analytical modeling of distributions in
stochastic systems on manifolds].
\textit{Informatika i ee Primeneniya}~--- \textit{Inform. Appl.} 7(2):4--16.


\bibitem{7-sin-1} %7
\Aue{Sinitsyn, I.\,N.}  2014.
Analiz i modelirovanie raspredeleniy v ereditarnykh stokhasticheskikh sistemakh
[Analysis and modeling of distributions in hereditary stochastic systems].
\textit{Informatika i ee Primeneniya}~--- \textit{Inform. Appl.} 8(1):2--11.

\bibitem{9-sin-1} %8
\Aue{Gradshteyn, I.\,S., and I.\,M.~Ryzhik}.  1963.
\textit{Tablitsy integralov, summ, ryadov i proizvedeniy}
[Tables of integrals, sums, series, and products]. Moscow:  GIFML.   1100~p.

\pagebreak

\bibitem{8-sin-1} %9
\Aue{Popov, B.\,A., and G.\,S.~Tesler}.  1984.
\textit{Vychislenie funktsiy na EVM}. Spravochnik [Computing of functions].
Kiev: Naukova Dumka.  599~p.


\bibitem{11-sin-1} %10
\Au{Sinitsyn, I.\,N.} 2009.
\textit{Kanonicheskie predstavleniya sluchaynykh funktsiy i ikh primenenie v
zadachakh komp'yuternoy podderzhki nauchnykh issledovaniy}
[Canonical expansions of random functions and its application to
scientific computer-aided support]. Moscow: TORUS PRESS. 768~p.

\bibitem{10-sin-1} %11
\Aue{Sinitsyn, I.\,N., V.\,I.~Sinitsyn, E.\,R.~Korepanov,
V.\,V.~Belousov, I.\,V.~Sergeev, and D.\,A.~Basilashvili}.
2012. Opyt modelirovaniya ereditarnykh stokhasticheskikh sistem
[Experience of modeling in hereditary stochastic systems].
\textit{Kibernetika i Vysokie Tekhnologii XXI~Veka:
Sbornik dokladov  XIII Mezhdunar. nauch.-tekhnich. konf.}
[Cybernatics ans High Technologies of the XXI Century: Materials of
XIII  Scientific and Technological Conference (International)].
Voronezh: Sakvoee. 2:346--357.

\end{thebibliography}

 }
 }

\end{multicols}

\vspace*{-6pt}

\hfill{\small\textit{Received May 05, 2014}}

\vspace*{-18pt}

\Contr

\noindent
\textbf{Sinitsyn Igor N.} (b.\ 1940)~---
Doctor of Science in technology, professor, Honored scientist of RF, Head of Department, Institute of
Informatics Problems, Russian Academy of Sciences,
44-2 Vavilov Str., Moscow 119333, Russian
Federation; sinitsin@dol.ru

\vspace*{3pt}

\noindent
\textbf{Sinitsyn Vladimir I.} (b.\ 1968)~--- Doctor of Science in physics
and mathematics, associate professor, Head of Department, Institute of
Information Problems, Russian Academy of Sciences,
44-2 Vavilov Str., Moscow 119333, Russian Federation; VSinitsin@ipiran.ru




\label{end\stat}

\renewcommand{\bibname}{\protect\rm Литература} %1
\def\stat{bosov+stef}

\def\tit{УПРАВЛЕНИЕ ВЫХОДОМ СТОХАСТИЧЕСКОЙ ДИФФЕРЕНЦИАЛЬНОЙ СИСТЕМЫ 
ПО~КВАДРАТИЧНОМУ КРИТЕРИЮ. II.~ЧИСЛЕННОЕ РЕШЕНИЕ УРАВНЕНИЙ 
ДИНАМИЧЕСКОГО ПРОГРАММИРОВАНИЯ$^*$}

\def\titkol{Управление выходом стохастической дифференциальной системы 
по квадратичному критерию. II %.~Численное решение уравнений  
%динамического программирования
}

\def\aut{А.\,В.~Босов$^1$, А.\,И.~Стефанович$^2$}

\def\autkol{А.\,В.~Босов, А.\,И.~Стефанович}

\titel{\tit}{\aut}{\autkol}{\titkol}

\index{Босов А.\,В.}
\index{Стефанович А.\,И.}
\index{Bosov A.\,V.}
\index{Stefanovich A.\,I.}


{\renewcommand{\thefootnote}{\fnsymbol{footnote}} \footnotetext[1]
{Работа выполнена при частичной поддержке РФФИ (проект 16-07-00677).}}


\renewcommand{\thefootnote}{\arabic{footnote}}
\footnotetext[1]{Институт проб\-лем информатики Федерального исследовательского центра 
<<Информатика и~управ\-ле\-ние>> Российской академии наук, \mbox{AVBosov@ipiran.ru}}
\footnotetext[2]{Институт проблем информатики Федерального исследовательского центра 
<<Информатика и~управ\-ле\-ние>> Российской академии наук, \mbox{AStefanovich@frccsc.ru}}

\vspace*{-8pt}

 
  

\Abst{Представлена вторая часть исследования задачи оптимального управления 
для диффузионного процесса Ито и~линейного управляемого выхода. 
Оптимальное управление выходом $dz_t\hm= a_t y_t \,dt\hm+b_t z_t \,dt\hm+ 
c_tu_t\,dt\hm+\sigma_t 
\,dw_t$ стохастической дифференциальной системы с~состоянием $dy_t\hm= 
A_t(y_t)\,dt +\Sigma_t (y_t)\, dv_t$ и~квадратичным критерием качества, 
определяемое функцией Беллмана вида $V_t(y,z)\hm= \alpha_t z^2\hm+\beta_t(y) 
z\hm+\gamma_t(y)$, рассчитывается путем приближенного решения сеточными 
методами дифференциальных уравнений для коэффициентов~$\alpha_t$, 
$\beta_t(y)$ и~$\gamma_t(y)$. Подробно рассмотрен модельный пример, 
опирающийся на простую дифференциальную модель для показателя RTT (Round-Trip Time)
сетевого протокола TCP (Transmission Control Protocol). 
Приводятся результаты численного эксперимента, 
позволяющие оценить трудности практической реализации оптимального 
решения и~обозначить задачи дальнейшего исследования.}

\KW{стохастическое дифференциальное уравнение; оптимальное управление; 
динамическое программирование; функция Беллмана; уравнение Риккати; 
линейные уравнения параболического типа}

\DOI{10.14357/19922264190102}
  
\vspace*{-4pt}


\vskip 10pt plus 9pt minus 6pt

\thispagestyle{headings}

\begin{multicols}{2}

\label{st\stat}

\section{Введение}
     
     В работе~\cite{1-b} получены аналитические соотноше\-ния, 
описывающие оптимальное решение в~задаче\linebreak управления линейным выходом 
стохастической дифференциальной системы по квадратичному критерию 
качества. Оптимизируемая динамическая система описывается двумя 
уравнениями: нелинейным стохастическим дифференциальным уравнением 
Ито для состояния и~линейным уравнением для управляемого выхода. 
Квадратичный целевой функционал обеспечил возможность решения задачи 
методом динамического программирования: получены аналитические 
выражения для функции Беллмана и, как следствие, для оптимального 
управления и~для значения функционала качества. Соответствующие 
соотношения содержат, как и~всегда в~подобных задачах, решения 
определенных дифференциальных уравнений (обыкновенных и~в~частных 
производных). Формальная постановка задачи и~основной теоретический 
результат кратко сформулированы в~следующем разделе статьи. В~данной 
работе исследование этой же задачи продолжено, но внимание сосредоточено 
на аспектах практической реализации результата.
     
     С фундаментальной точки зрения в~сравнении с~исходной постановкой 
задачи оптимизации в~стохастической системе поиск решений 
дифференциальных уравнений~--- это удовлетворительный итог 
исследования, достаточный для перехода в~практическую область, т.\,е.\ 
проведения расчетов, модельных как минимум, а то и~с реальными данными. 
На самом деле нюансы численной реализации полученного в~\cite{1-b} 
оптимального управления в~совокупности сами по себе представляют 
исследовательский вызов. В~полном объеме ответить на него пока не 
удается, но начальные результаты, приближенные алгоритмы, модельные 
расчеты получены и~составляют предмет настоящей работы. 

Итоги 
подводятся в~заключении, где в~том числе обозначены перспективы 
дальнейших исследований.

\section{Оптимальное решение задачи управления выходом}

     Следуя~\cite{1-b}, будем предполагать рас\-смат\-ри\-ва\-емые далее 
случайные функции скалярными. Скалярная модель позволяет обойти 
трудности численной реализации, неизбежно возникающие при рос\-те 
размерности. Это принципиально важно для данного этапа исследования, 
поскольку дальнейшие усилия предполагаются в~направлении 
совершенствования предложенных здесь методов приближенного численного 
решения, в~том числе с~учетом возможности роста размерности.
     
     Итак, рассматривается состояние~$y_t$ стохастической 
дифференциальной системы, описываемое нелинейным стохастическим 
дифференциальным уравнением Ито:
     \begin{equation*}
     dy_t=A_t\left( y_t\right) dt+\Sigma_t \left( y_t\right) dv_t\,,\enskip y_0=Y\,,
     %\label{e1-b}
     \end{equation*}
где $v_t$~--- стандартный винеровский процесс; $Y$~--- случайная величина 
с~конечным вторым моментом; функции $A_t$ и~$\Sigma_t$ удовлетворяют 
условиям Ито, обеспечивающим существование единственного 
решения~\cite{2-b}.

     С состоянием этой системы связан выход, который описывается 
функцией~$z_t$, линейно связанной с~$y_t$:
     \begin{equation}
     dz_t=a_t y_t\,dt+b_t z_t \,dt+c_t u_t \,dt+\sigma_t \,dw_t,\enskip z_0=Z,
     \label{e2-b}
     \end{equation}
где $w_t$~--- не зависящий от $v_t$, $Y$ и~$Z$ стандартный винеровский 
процесс; $Z$~--- случайная величина с~конечным вторым моментом; $u_t$~--- 
допустимое управ\-ле\-ние. Функции~$a_t$, $b_t$, $c_t$ и~$\sigma_t$ 
предполагаются ограниченными, процесс управления~--- допустимым 
неупреждающим~\cite{2-b}, что обеспечивает существование решения 
уравнения~(\ref{e2-b}) для любого допустимого управления.
     
     Используется целевой функционал следующего вида:
     \begin{multline*}
     \!J\left( U_0^T\right) =E\!\left\{ \int\limits_0^T\!\big( S_t\left( s_t y_t -g_t z_t-h_t 
u_t\right)^2+G_t z_t^2+{}\right.\\
\left.{}+H_t u_t^2\big)\,dt+
 S_T\left( s_T y_T-g_T z_T\right)^2+G_T z_T^2
      \vphantom{\int\limits_0^T}\right\}\,,\\
     U_0^T=\left\{ u_t, 0\leq t\leq T
     \right\}\,,
     %\label{e3-b}
     \end{multline*}
где $S_t$, $G_t$ и~$H_t$~--- неотрицательные функции.

     Решение задачи поиска $u_t^*$~--- допустимого управ\-ле\-ния, 
доставляющего минимум квадратичному функционалу $J(U_0^T)$, 
составляют следующие соотношения.
     
     Функция Беллмана $V_t(y,z)$ может быть пред\-став\-ле\-на в~виде:
     \begin{equation*}
     V_t(y,z)=\alpha_t z^2+\beta_t(y) z+\gamma_t(y)\,,
     %\label{e4-b}
     \end{equation*}
а оптимальное управление
\begin{multline}
u_t^* =u_t^*(y,z)={}\\[2pt]
{}=-\fr{1}{2}\left( S_th_t^2+H_t\right)^{-1}\left( c_t \left(
2\alpha_t z+\beta_t(y)\right)+{}\right.\\[2pt]
\left.{}+2S_t \left( s_t y-g_t z\right) h_t\right)\,,\enskip
y=y_t\,,\enskip z=z_t\,.
\label{e5-b}
\end{multline}
     
     Коэффициенты $\alpha_t$, $\beta_t(y)$ и~$\gamma_t(y)$ задаются 
следующими дифференциальными уравнениями:
     \begin{multline}
     \fr
{\partial \alpha_t}{\partial t}+
2\alpha_t
     \left( b_t-\left( S_t h_t^2+H_t\right)^{-1} c_t S_t h_t g_t\right)+{}\\[2pt]
     {}+\left( S_t-\left( S_t h_t^2+H_t\right)^{-1} S_t^2 h_t^2\right) g_t^2+{}\\[2pt]
     {}+G_t-
     \left( S_t h_t^2+H_t\right)^{-1} c_t^2 \alpha_t^2=0\,,\\[2pt]
     \alpha_T=S_T g_T^2+G_T\,;
     \label{e6-b}
     \end{multline}

\vspace*{-12pt}

\noindent
\begin{multline}
\fr{\partial \beta_t(y)}{\partial y}+\fr{1}{2}\Sigma_t^2(y)
\fr{\partial^2\beta_t(y)}{\partial y^2} 
+A_t(y)\fr{\partial \beta_t(y)}{\partial y}+{}\\[2pt]
{}+2\alpha_t\left( a_t+\left( S_t h_t^2+H_t\right)^{-1} c_t S_t h_t 
s_t\right)y+{}\\[2pt]
{}+\beta_t(y) \left( b_t-\left( S_t h_t^2+H_t\right)^{-1} c_t S_t h_t 
g_t\right)-{}\\[2pt]
{}-2\left( S_t-\left( S_t h_t^2+H_t\right)^{-1} S_t^2 h_t^2\right) s_t g_t y-{}\\[2pt]
{}-\left( 
S_t h_t^2+H_t\right)^{-1} c_t^2 \alpha_t \beta_t(y)=0\,,\\[2pt]
\beta_T(y)=-2S_T s_T g_T y\,;
\label{e7-b}
\end{multline}

\vspace*{-12pt}

\begin{multline}
\fr{\prt \gamma_t(y)}{\prt  t}+\fr{1}{2}\,\Sigma_t^2(y)\fr{\prt^2 
\gamma_t(y)}{\prt y^2} +\sigma^2_t \alpha_t +A_t(y) \fr{\prt \gamma_t(y)}{\prt 
y}+{}\\[2pt]
{}+ \beta_t(y)\left( a_t+\left( S_t h_t^2+H_t\right)^{-1} c_t S_t h_t s_t\right) 
y+{}\\[2pt]
{}+\left( S_t- \left( S_t h_t^2 +H_t\right)^{-1} S_t^2 h_t^2\right) s_t^2 y^2-{}\\[2pt]
{}-\fr{1}{4}\left( S_t h_t^2 +H_t\right)^{-1} c_t^2 \beta_t^2(y)=0\,,\\[2pt] 
\gamma_T(y)=S_T s_T^2 y^1\,.
\label{e8-b}
\end{multline}
     
     Уравнение~(\ref{e6-b}) является уравнением Риккати и~имеет 
единственное неотрицательное решение для всех $0\hm\leq t\hm\leq T$, так 
как предполагается $S_t h_t^2\hm+ H_t\hm>0$. 
Уравнения~(\ref{e7-b}) 
и~(\ref{e8-b}) представляют собой линейные уравнения в~частных 
производных второго порядка, относятся к~параболическому типу, поскольку 
$\Sigma_t^2(y)\hm>0$. Предполагается, что данные уравнения имеют 
на рассматриваемом интервале $0\hm\leq t\hm\leq T$ хотя бы одно 
ограниченное решение, которое и~определяет оптимальное решение 
рассматриваемой задачи оптимизации.
 

\section{Модельный эксперимент и~приближенное численное 
решение}
     
     Практическая реализация оптимального управ\-ле\-ния~(\ref{e5-b}), 
конечно, возможна только приближенно, рассматривать варианты, когда 
уравнения~(\ref{e6-b}) и~(\ref{e7-b}) могут быть решены аналитически, 
бесперспективно. Приближенное решение очевидным образом получается 
в~результате численного интегрирования уравнений~(\ref{e6-b})  
и~(\ref{e7-b}), а~при не\-об\-хо\-ди\-мости и~(\ref{e8-b}). Для реализации этого 
требуется модельный пример, который позволит, с~одной стороны, 
поэкспериментировать с~численными процедурами, сформировать 
представление об их точности и~ресурсоемкости. С~другой стороны, при 
выборе модельного примера следует ориентироваться на потенциальные 
прикладные области применения представленного результата. В~качестве 
такой возможной области видится область телекоммуникаций~--- 
популярный современный источник приложений для различных 
теоретических задач. Из моделей, исследуемых в~этой области, была выбрана 
простая модель для показателя RTT сетевого протокола 
TCP, предложенная в~\cite{3-b}. Опуская 
детали выбора параметров в~используемой диффузионной модели RTT, 
укажем только итоговое уравнение рассматриваемого модельного примера:
     \begin{multline}
     dy_t= \left( 1-0{,}1y_t\right) dt+0{,}5\sqrt{y_t}\,dv_t,\\ y_0=Y\sim 
N(15,9)\,.
     \label{e9-b}
     \end{multline}
Здесь $N(M,{\sf D})$~--- нормальное распределение со средним~$M$ 
и~дисперсией~${\sf D}$.
     
     Надо отметить, что такое уравнение (с~точ\-ностью до варьируемых 
параметров) изначально известно как модель Кок\-са--Ин\-гер\-со\-на--Рос\-са 
(Cox--Ingersoll--Ross model)~\cite{4-b} и~описывает эволюцию процентных 
ставок. Конечно, для описания реальных данных RTT модель  
типа~(\ref{e9-b}) представляет не более чем начальное приближение, но оно 
удачно уже потому, что дает основу для дальнейших обобщений, уточнений 
модели (см., например,~\cite{5-b}), поэтому удобна и~для целей настоящей 
работы. Свойства процесса~$y_t$ из~(\ref{e9-b}) хорошо изучены, 
в~частности имеется свойство неотрицательности выборочных значений, 
наличие эргодичности, известно предельное распределение и~переходная 
вероятность. Отметим, что начальные условия в~(\ref{e9-b}) выбраны так, 
что $M$ и~${\sf D}$ отличаются от моментных характеристик предельного 
распределения, т.\,е.\ управление ведется в~рамках переходного процесса, 
а~не в~стационарном режиме.
     
     Выход для~(\ref{e9-b}) задается уравнением
          \begin{multline}
     dz_t=0{,}1 y_t\, dt-z_t \,dt+u_t \,dt+dw_t\,,\\ z_0=Z\sim N(9, 9)\,,
     \label{e10-b}
     \end{multline}
    целевой функционал
\begin{multline*}
J\left( U_0^T\right) =E\left\{ \int\limits_0^T \left( \left( y_t-
z_t\right)^2+z_t^2+u_t^2\right) \,dt +{}\right.\\
\left.{}+\left( y_T -z_T\right)^2+z_T^2
     \vphantom{\int\limits_0^T}
     \right\}\,.
%\label{e11-b}
\end{multline*}
     
     Моделировалось $N\hm=1000$ траекторий $y_t$, $z_t^*$ и~$u_t^*$ для 
$T\hm=5$ и~$50$ (через~$z_t^*$ обозначен выход~(\ref{e10-b}), рассчитанный 
для $u_t\hm=u_t^*$). Кроме того, моделировались траектории $z_t^{\mathrm{prog}}$ 
и~$u_t^{\mathrm{prog}}$ для наилучшего программного управления $u_t^{\mathrm{prog}}\hm= E\{ 
u_t^*\}$ и~$z_t^0$ и~$u_t^0$ для неуправляемого выхода~(\ref{e10-b}), т.\,е.\ для 
$u_t\hm=0$. В~каждом случае путем осреднения по пучку траекторий 
оценивались величины~$J(U_0^t)$. Соответственно, в~результате можно 
увидеть не только конечные значения целевой функции для разных 
вариантов управления $J((U^*)_0^T)$, $J((U^{\mathrm{prog}})_0^T)$ и~$J((U^0)_0^T)$, 
но и~их формирование в~динамике.
     
     Для определения оптимального управления чис\-лен\-но решались 
уравнения для~$\alpha_t$ и~$\beta_t(y)$. Сначала было получено чис\-лен\-ное 
решение урав\-не\-ния~(\ref{e6-b}). Это уравнение Риккати, оно имеет 
единственное решение, и~вы\-чис\-ле\-ние этого решения не пред\-став\-ля\-ет труда, 
например неявным методом Эйлера.
     
     Далее решалось уравнение~(\ref{e7-b}). Как уже отмечалось, 
нахождение решений уравнений~(\ref{e7-b}) и~при  
необходимости~(\ref{e8-b}) (это уравнение можно использовать для 
определения качества оптимального управ\-ле\-ния) пред\-став\-ля\-ет 
определенные труд\-но\-сти. Для целей данной работы достаточно использовать 
традиционный подход к~чис\-лен\-но\-му решению~--- метод конечных 
разностей, тем более что варианты этого метода для урав\-не\-ния 
параболического типа дав\-но и~хорошо изучены~\cite{6-b}.
     
     Первое, что требуется сделать,~--- это ограничить область значения 
аргумента~$y$ для формирования численной схемы расчета. С~этой целью 
использовались~$N$~смоделированных методом Эйлера 
траекторий~(\ref{e9-b}): выборочные значения использовались для 
оценивания функций математического ожидания и~дисперсионной, далее 
задавалась  
$3\sigma$-труб\-ка, а~в~расчетах граница значений~$y$ задавалась путем 
добавления 10\% к~границе $3\sigma$-труб\-ки. Рисунок~1 иллюстрирует 
характерные траектории~(\ref{e9-b}), среднее значение процесса 
и~вычисленные границы (закрашенная серым область).

\begin{figure*} %fig1
\vspace*{1pt}
 \begin{center}  
  \mbox{%
 \epsfxsize=163mm 
 \epsfbox{bos-1.eps}
 }
\end{center}
\vspace*{-11pt}
\Caption{Выборочные траектории и~характеризация границы}
\end{figure*}

     Определение границы создает следующую проб\-ле\-му, обычную при 
применении любого сеточного метода, а~именно: отсутствие граничных 
условий для рассматриваемых уравнений. Этих условий нет, и~нет 
физических оснований для их обоснованного выбора. Из-за этого приходится 
выбирать граничные условия довольно волюнтаристски и~надеяться на 
устойчивость решения в~отношении этих граничных условий. 

Выбор 
граничного условия и~анализ его устойчивости выполнялся следующим 
образом. Были взяты два традиционных варианта граничных условий~--- 
в~задаче Дирихле: $\beta_t(y)\hm=0$, т.\,е.\ условие поглощения, и~в~задаче 
Неймана: $\partial \beta_t(y)/\partial y\hm=0$, т.\,е.\ условия отражения, для 
всех граничных точек. Далее уравнение решалось методом конечных 
разностей с~использованием явной и~неявной численных схем сначала для 
одного граничного условия, затем для второго, и~полученные решения 
сравнивались. Устойчивость в~таком случае означает совпадение (близость) 
решений в~заданной области за исключением границы.
     
     Для численного интегрирования в~области $[0,T]\cup \left[\, 
\underline{y}, \overline{y}\right]$ ($\underline{y}$ и~$\overline{y}$~--- 
определенные в~результате моделирования границы) формируется сетка 
$\{t_k, y_i\}$ с~шагами~$\delta_t$ (для интегрирования по переменной~$t$) 
и~$\delta_y$ (для интегрирования по переменной~$y$). С~использованием 
обозначения~$\beta_i^k$ для значения~$\beta_t(y)$ в~узле $(t_k, y_i)$ ее 
производные аппроксимировались следующим образом:
     \begin{gather*}
     \fr{\prt \beta_t(y)}{\prt t}\approx \fr{\beta_i^{k+1}-\beta_i^k}{\delta_t}\,;\quad
     \fr{\prt \beta_t(y)}{\prt y}\approx \fr{\beta^K_{i+1}-\beta^K_{i-
1}}{2\delta_y}\,;\\
     \fr{\prt^2\beta_t(y)}{\prt y^2}\approx \fr{\beta^K_{i+1}-
2\beta_i^K+\beta^K_{i-1}}{\delta_y^2}\,,
     \end{gather*}
где $K=k$ для явной схемы и~$K\hm=k\hm+1$~--- для неявной.
     
     Шаги интегрирования $\delta_t$ и~$\delta_y$ при анализе устойчивости 
выбирались разными, например явная схема обсчитывалась для 
$\delta_t\hm=0{,}0002$ и~$\delta_y\hm=0{,}1$, в~том числе с~учетом того, 
что для нее самой вопрос устойчивости ограничивает вариации шагов. 
Главное, что в~итоге подтвердилась устойчивость результатов численного 
интегрирования $\beta_t(y)$ по отношению к~граничным условиям по 
крайней мере для выбранной модели.
     
     Итоговый расчет выполнен для неявной схемы с~шагами 
$\delta_t\hm=0{,}001$ (в~том числе для приближенного 
вычисления~$\alpha_t$) и~$\delta_y\hm=0{,}01$. Результаты расчетов 
представлены на рис.~2--4, в~частности приведены примеры траекторий для 
выхода и~управлений $z_t^*, u_t^*, z_t^{\mathrm{prog}}, u_t^{\mathrm{prog}}$ и~$z_t^0, u_t^0$.



     Рисунок~4 демонстрирует ожидаемый проигрыш\linebreak неуправляемой 
системы $z_t^0, u_t^0$. Кроме того, обращает на себя внимание высокое 
качество про\-граммного управления и~выхода $z_t^{\mathrm{prog}}, u_t^{\mathrm{prog}}$. 
Объ\-ективных оснований последнее обстоятельство,\linebreak конечно, не имеет 
и~объясняется исключительно выбранными параметрами рассматриваемого 
модельного примера, прежде всего интенсивностями возмущений~--- 
коэффициентов при $v_t$ и~$w_t$, что в~свою очередь сделано для того, чтобы 
на рис.~4 изменения целевого функционала для разных стратегий управ\-ле\-ния 
можно было представить в~одном масштабе.
     
     Наконец, отметим, что существенных техниче\-ских трудностей при 
выполнении расчетов не возникло. Без дополнительных усилий сетка 
размещалась в~оперативной памяти: для интегрирования\linebreak
 с~выбранными 
шагами в~об\-ласти $[0,50]\cup [-10,40]$ было использовано всего лишь~2~ГБ 
памяти, что позволило провести расчеты на обычном персональном 
компьютере. При этом размещение сетки в~оперативной памяти обеспечило 
довольно быст\-рое проведение расчетов. Гораздо больше времен-\linebreak н$\acute{\mbox{ы}}$х ресурсов 
требуется для моделирования пучков траекторий, осред\-не\-ния~--- срав\-не\-ния 
свойств управ\-ле\-ний. Конечно, с~рос\-том размерности ситу-\linebreak\vspace*{-12pt}

\pagebreak

\end{multicols}

\begin{figure*} %fig2
\vspace*{1pt}
 \begin{center}  
  \mbox{%
 \epsfxsize=162.963mm 
 \epsfbox{bos-2.eps}
 }
\end{center}
\vspace*{-11pt}
\Caption{Выборочные траектории $z_t^*$~(\textit{1}), $z_t^{\mathrm{prog}}$~(\textit{2}) 
и~$z_t^0$~(\textit{3})}
%\end{figure*}
%\begin{figure*} %fig3
\vspace*{6pt}
 \begin{center}  
  \mbox{%
 \epsfxsize=163mm 
 \epsfbox{bos-3.eps}
 }
\end{center}
\vspace*{-11pt}
\Caption{Выборочные траектории $u_t^*$~(\textit{1}), $u_t^{\mathrm{prog}}$~(\textit{2}) 
и~$u_t^0$~(\textit{3})}
%\end{figure*}
%\begin{figure*} %fig4
\vspace*{6pt}
 \begin{center}  
  \mbox{%
 \epsfxsize=162.528mm 
 \epsfbox{bos-4.eps}
 }
\end{center}
\vspace*{-11pt}
\Caption{Динамика целевого функционала:
\textit{1}~--- $J((U^*)^t_0)$; \textit{2}~---  $J((U^0)_0^t)$; \textit{3}~--- 
$J((U^{\mathrm{prog}})^t_0)$}
\end{figure*}

\begin{multicols}{2}

\noindent
ация 
принципиально изменится и~затраты на чис\-лен\-ное интегрирование 
дифференциальных уравнений вырастут на порядки.

\vspace*{-9pt}



\section{Заключение}

\vspace*{-2pt}

     Настоящая статья содержит вторую, практическую часть исследования 
задачи оптимизации линейного выхода нелинейной дифференциальной 
системы по квадратичному критерию. В~отношении представленного 
результата следует отметить, что полученное решение представляет 
определенные вычислительные трудности при его приближенной реализации 
путем численного решения дифференциальных уравнений для 
соответствующих коэффициентов. 

Отсутствие физически обоснованных 
граничных условий потребовало значительных усилий для анализа 
численного решения на границе. Кроме того, очевидное обобщение 
постановки на многомерный случай  еще более 
усугубит вычислительные проблемы. Таким образом, затраты вы\-чис\-ли\-тель\-ных ресурсов даже на 
модельные расчеты в~скалярном случае оказались слишком значительными, 
чтобы не учитываться в~теоретической час\-ти решения. По этой причине, 
прежде чем двигаться с~рассмотренной задачей дальше, обобщать постановку 
на многомерный случай, нуж\-но предложить более действенные инструменты 
в~час\-ти чис\-лен\-ной реализации полученных точных решений. Возможности 
для этого имеются, а~соответствующие результаты должны составить 
ближайшие перспективы дальнейших исследований.
     
     Представленная в~данной работе задача оптимизации пока не 
исследовалась детально на предмет практического применения. 
Выполненные модельные расчеты~--- это, скорее, первый шаг в~на\-прав\-ле\-нии 
поиска реальных приложений. При этом основным представляется поиск 
моделей, обладающих практической ценностью и~достаточно исследованных 
в~рамках задач анализа. Источником таких моделей видится область 
инфотелекоммуникаций, активно развиваемая в~последние годы, в~част\-ности 
модели сетевых транспортных протоколов на основе скачкообразных 
марковских процессов~[7--10]. Применение рассмотренной 
оптимизационной задачи к~таким моделям также может стать 
содержательным продолжением представленного в~данной работе 
исследования.

{\small\frenchspacing
 {%\baselineskip=10.8pt
 \addcontentsline{toc}{section}{References}
 \begin{thebibliography}{99}
\bibitem{1-b}
\Au{Босов А.\,В., Стефанович~А.\,И.} Управление выходом 
стохастической дифференциальной системы по квад\-ра\-тич\-но\-му критерию. 
I.~Оптимальное решение методом динамического программирования~// 
Информатика и~её применения, 2018. Т.~12. Вып.~3. С.~99--106. doi: 
10.14357/19922264180314.
\bibitem{2-b}
\Au{Флеминг У., Ришел~Р.} Оптимальное управление 
детерминированными и~стохастическими системами~/ Пер. с~англ.~--- М.: 
Мир, 1978. 316~с. (\Au{Fleming~W.\,H., Rishel~R.\,W.} Deterministic and 
stochastic optimal control.~--- New York, NY, USA: Springer-Verlag, 1975. 
222~p.)
\bibitem{3-b}
\Au{Bohacek S., Rozovskii~B.} A~diffusion model of roundtrip time~// 
Comput. Stat. Data An., 2004. Vol.~45. Iss.~1. P.~25--50. 
doi: 10.1016/S0167-9473(03)00114-2.
\bibitem{4-b}
\Au{Cox J.\,C., Ingersoll~J.\,E., Ross~S.\,A.} A~theory of the term structure of 
interest rates~// Econometrica, 1985. Vol.~53. Iss.~2. P.~385--407. doi: 
10.2307/1911242.
\bibitem{5-b}
\Au{Bohacek S.} A~stochastic model of TCP and fair video transmission~// 
Proc. IEEE INFOCOM, 2003. P.~1134--1144. doi: 
10.1109/INFCOM.2003.1208950.
\bibitem{6-b}
\Au{Саульев В.\,К.} Интегрирование уравнений параболического типа 
методом сеток.~--- М.: Физматлит, 1960. 324~с.
\bibitem{7-b}
\Au{Борисов А.\,В., Миллер~Г.\,Б.}
Анализ и~фильтрация специальных марковских процессов в~дискретном 
времени. II.~Оптимальная фильтрация~// Автоматика и~телемеханика, 
2005. №\,7. С.~112--125. doi: 10.1007/s10513-005-0153-7.
\bibitem{8-b}
\Au{Борисов А.\,В., Миллер~Б.\,М., Семенихин~К.\,В.} Фильт\-ра\-ция 
марковского скачкообразного процесса по наблюдениям 
мультивариантного точечного процесса~// Автоматика и~телемеханика, 
2015. №\,2. С.~34--60. doi: 10.1134/S0005117915020034.
\bibitem{9-b}
\Au{Borisov A., Bosov~A., Miller~G.} Modeling and monitoring of RTP link on 
the receiver side~// Internet of things, smart spaces, and next generation
networks and systems~/ Eds. S.\,I.~Balandin, S.\,D.~Andreev,
Y.~Koucheryavy.~--- Lecture notes in computer science ser.~--- Springer, 2015. Vol.~9247. 
P.~229--241. doi: 10.1007/978-3-319-23126-6\_21.
\bibitem{10-b}
\Au{Борисов А.\,В.} Применение методов оптимальной фильтрации для 
оперативного оценивания состояний сетей массового обслуживания~// 
Автоматика и~телемеханика, 2016. №\,2. С.~115--141. doi: 
10.1134/S0005117916020053.
 \end{thebibliography}

 }
 }

\end{multicols}

\vspace*{-3pt}

\hfill{\small\textit{Поступила в~редакцию 07.06.18}}

\vspace*{8pt}

%\pagebreak

%\newpage

%\vspace*{-28pt}

\hrule

\vspace*{2pt}

\hrule

%\vspace*{-2pt}

\def\tit{STOCHASTIC DIFFERENTIAL SYSTEM OUTPUT CONTROL 
BY~THE~QUADRATIC CRITERION. II.~DYNAMIC PROGRAMMING 
EQUATIONS NUMERICAL SOLUTION}

\def\titkol{Stochastic differential system output control 
by~the~quadratic criterion. II.~Dynamic programming 
equations numerical solution}

\def\aut{A.\,V.~Bosov and~A.\,I.~Stefanovich}

\def\autkol{A.\,V.~Bosov and~A.\,I.~Stefanovich}

\titel{\tit}{\aut}{\autkol}{\titkol}

\vspace*{-11pt}


\noindent
Institute of Informatics Problems, Federal Research Center 
``Computer Science and Control'' of the 
Russian Academy of Sciences, 44-2~Vavilov Str., Moscow 119333, Russian Federation

\def\leftfootline{\small{\textbf{\thepage}
\hfill INFORMATIKA I EE PRIMENENIYA~--- INFORMATICS AND
APPLICATIONS\ \ \ 2019\ \ \ volume~13\ \ \ issue\ 1}
}%
 \def\rightfootline{\small{INFORMATIKA I EE PRIMENENIYA~---
INFORMATICS AND APPLICATIONS\ \ \ 2019\ \ \ volume~13\ \ \ issue\ 1
\hfill \textbf{\thepage}}}

\vspace*{6pt}


\Abste{The second part of the optimal control problem investigation for the Ito 
diffusion process and the controlled linear output is presented. Optimal control for 
output $dz_t= a_t y_t \,dt+b_t z_t \,dt+ c_t u_t\,dt+\sigma_t \,dw_t$ of the stochastic 
differential system $dy_t\hm= A_t(y_t)\,dt +\Sigma_t (y_t) \,dv_t$ and quadratic quality 
criterion defined by Bellman
function having form $V_t(y,z)\hm= \alpha_t 
z^2\hm+\beta_t(y) z\hm+\gamma_t(y)$ is determined numerically by an approximate 
solution to\linebreak\vspace*{-12pt}}

\Abstend{the grid methods of differential equations for the coefficients $\alpha_t$, 
$\beta_t(y)$, and~$\gamma_t(y)$. A~model experiment based on a~simple differential 
presentation for the RTT (Round-Trip Time)
parameter of the TCP (Transmission Control Protocol)
network protocol is considered in detail. 
The results of numerical simulation are given and allow one to assess the difficulties in 
the practical implementation of the optimal solution and define the tasks of further 
research.}

\KWE{stochastic differential equation; optimal control; dynamic programming; 
Bellman function; Riccati equation; linear differential equations of parabolic type}


\DOI{10.14357/19922264190102}

%\vspace*{-14pt}

\Ack
\noindent
This work was partially supported by the Russian Science Foundation (grant  
16-07-00677).



%\vspace*{6pt}

  \begin{multicols}{2}

\renewcommand{\bibname}{\protect\rmfamily References}
%\renewcommand{\bibname}{\large\protect\rm References}

{\small\frenchspacing
 {%\baselineskip=10.8pt
 \addcontentsline{toc}{section}{References}
 \begin{thebibliography}{99}
\bibitem{1-b-1}

\Aue{Bosov, A.\,V., and A.\,I.~Stefanovich.} 2018. Upravlenie vykhodom stokhasticheskoy differentsial'noy
sistemy po kvadratichnomu kriteriyu. I.~Optimal'noe reshenie
metodom dinamicheskogo programmirovaniya
[Stochastic differential system output control
by the quadratic criterion. I.~Dynamic
programming optimal solution]. \textit{Informatika i ee~Primeneniya~--- Inform. Appl.}  
12(3):99--106. doi: 
10.14357/19922264180314. 
\bibitem{2-b-1}
\Aue{Fleming, W.\,H., and R.\,W.~Rishel.} 1975. \textit{Deterministic and 
stochastic optimal control}. New York, NY: Springer-Verlag. 222~p.
\bibitem{3-b-1}
\Aue{Bohacek, S., and B.~Rozovskii.} 2004. A~diffusion model of roundtrip 
time. \textit{Comput. Stat.  Data An}. 45(1):25--50.
doi: 10.1016/S0167-9473(03)00114-2.
\bibitem{4-b-1}
\Aue{Cox, J.\,C., J.\,E.~Ingersoll, and S.\,A.~Ross.} 1985. A~theory of the 
term structure of interest rates. \textit{Econometrica} 53:385--407.
doi: 10.2307/1911242.
\bibitem{5-b-1}
\Aue{Bohacek, S.} 2003. A~stochastic model of TCP and fair video 
transmission. \textit{Proc. IEEE INFOCOM}. 1134--1144.
doi: 10.1109/INFCOM.2003.1208950.
\bibitem{6-b-1}
\Aue{Sauliev, V.\,K.} 1960. \textit{Integrirovanie uravneniy pa\-ra\-bo\-li\-che\-sko\-go 
tipa metodom setok} [Integration of parabolic equations by the grid method]. 
Moscow: Fizmatlit. 324~p.
\bibitem{7-b-1}
\Aue{Borisov, A.\,V., and G.\,B.~Miller.} 2005. Analysis and filtration of 
special discrete-time Markov processes. II.~Optimal filtration. \textit{Autom. 
Rem. Contr.} 66(7):1125--1136.
\bibitem{8-b-1}
\Aue{Borisov, A.\,V., G.\,B.~Miller, and K.\,V.~Semenikhin.} 2015. Filtering 
of the Markov jump process given the observations of multivariate point 
process. \textit{Autom. Rem. Contr.} 76(2):219--240.
\bibitem{9-b-1}
\Aue{Borisov, A., A.~Bosov, and G.~Miller.} 2015. Modeling and monitoring 
of RTP link on the receiver side. 
\textit{Internet of things, smart spaces, and next generation
networks and systems}. Eds. S.\,I.~Balandin, S.\,D.~Andreev,
and Y.~Koucheryavy. Lecture notes in computer science ser. Springer, 
2015. 9247:229--241. doi: 10.1007/978-3-319-23126-6\_21.

\bibitem{10-b-1}
\Aue{Borisov, A.\,V.} 2016. Application of optimal filtering methods for  
on-line of queueing network states. \textit{Autom. Rem. Contr.} 
77(2):277--296.
\end{thebibliography}

 }
 }

\end{multicols}

\vspace*{-6pt}

\hfill{\small\textit{Received June 7, 2018}}

%\pagebreak

%\vspace*{-18pt}
     
     \Contr
     
\noindent
\textbf{Bosov Alexey V.} (b.\ 1969)~--- Doctor of Science in technology, 
principal scientist, Institute of Informatics Problems, Federal Research 
Center ``Computer Science and Control'' of the Russian Academy of Sciences, 
44-2 Vavilov Str., Moscow 119333, Russian Federation; 
\mbox{AVBosov@ipiran.ru}

\vspace*{3pt}

\noindent
\textbf{Stefanovich Alexey I.} (b.\ 1983)~--- principal specialist, Institute of Informatics Problems, 
Federal Research Center ``Computer Science and Control'' of the Russian Academy of Sciences,  
44-2~Vavilov Str., Moscow 119333, Russian Federation; \mbox{AStefanovich@frccsc.ru}
\label{end\stat}

\renewcommand{\bibname}{\protect\rm Литература}       

         %2


%\newcommand{\tr}{\mathop{\rm tr}}
\newcommand{\trans}{{{\mathrm{T}}}}

\def\stat{rybakov}

\def\tit{ОБ ОДНОМ КЛАССЕ ЗАДАЧ ФИЛЬТРАЦИИ НА~МНОГООБРАЗИЯХ$^*$}

\def\titkol{Об одном классе задач фильтрации на~многообразиях}

\def\aut{К.\,А.~Рыбаков$^1$}

\def\autkol{К.\,А.~Рыбаков}

\titel{\tit}{\aut}{\autkol}{\titkol}

\index{Рыбаков К.\,А.}
\index{Rybakov K.\,A.}


{\renewcommand{\thefootnote}{\fnsymbol{footnote}} \footnotetext[1]
{Работа выполнена при поддержке РФФИ (проект 17-08-00530-а).}}


\renewcommand{\thefootnote}{\arabic{footnote}}
\footnotetext[1]{Московский авиационный институт (национальный исследовательский университет), 
\mbox{rkoffice@mail.ru}}

%\vspace*{8pt}



\Abst{Цель статьи состоит в~описании стохастических дифференциальных 
систем, траектории которых находятся на гладком многообразии, в~приложении к~задаче 
оптимальной фильтрации. Дополнительным условием является принадлежность этому же 
многообразию не только траекторий системы, но и~результата оценивания 
этих траекторий на основе косвенных измерений, а~именно: решения задачи 
оптимальной фильтрации по критерию минимума среднеквадратической ошибки оценивания. 
Рас\-смат\-ри\-ва\-ют\-ся системы как диффузионного типа, так и~диф\-фу\-зи\-он\-но-скач\-ко\-об\-раз\-но\-го 
типа, т.\,е.\ при наличии как винеровских, так и~пуассоновских возмущений.
 Результатом являются условия на коэффициенты уравнения для случайного
  процесса, траектории которого требуется оценить. 
  В~основе полученных условий лежит понятие первого интеграла стохастических 
  дифференциальных уравнений, а также некоторые его свойства.}

\KW{инвариант; многообразие; оптимальная фильтрация; оценивание; 
случайный процесс; стохастическая дифференциальная система}

\DOI{10.14357/19922264190103}
  
%\vspace*{4pt}


\vskip 10pt plus 9pt minus 6pt

\thispagestyle{headings}

\begin{multicols}{2}

\label{st\stat}

\section{Введение}

В теории стохастических динамических систем задача фильтрации 
имеет важное значение, она\linebreak состоит в~нахождении оценки 
ненаблюдаемого вектора состояния системы по результатам его косвенных измерений. 
Критерии оптимальности оценки можно задавать различным образом, во многих 
приложениях ограничиваются критерием минимума среднеквадратической ошибки оценивания. 
Такие задачи решаются для непрерывных, дискретных и~не\-пре\-рыв\-но-дис\-крет\-ных 
систем, в~том числе и~в~случае, когда вектор состояния системы принадлежит 
заданному многообразию. Решению задач фильтрации на многообразиях посвящен 
целый ряд работ, опубликованных в~последнее 
время~\cite{Sin_SSI16, Sin_IP16, SinSinKor_SSI16, SinSinKor_IP17, SinSinSerKor_AiT18}.

Принадлежность вектора состояния стохастической динамической системы 
многообразию означает, что в~системе выполняется некоторый закон сохранения. 
В~более общем случае рассматривается принадлежность пары <<вре\-мя\;+\;со\-сто\-яние>> 
динамическому многообразию. Методы описания и~построения подобных непрерывных 
стохастических систем подробно изложены в~\cite{Dub_89, Dub_12, Kar_14, Kar_15}. 
Но этот закон сохранения для оценки вектора состояния при решении 
задачи фильтрации выполняться не будет, кроме специального класса систем.

Перейдем к~иллюстрирующему примеру. 
Пусть векторный случайный процесс $X(t) \hm= [  X_1(t) ~ X_2(t)  ]^\trans$ 
со значениями в~$\mathds{R}^2$ удовлетворяет линейному стохастическому 
дифференциальному уравнению Ито

\noindent
$$
  dX(t) = \underbrace{\left[ \begin{array}{cc}
    -1 & 1 \\
    -1 & -1 \\
  \end{array} \right]}_F X(t) \, dt 
  +
  \underbrace{\left[ \begin{array}{cc}
    0 & \sqrt 2 \\
    -\sqrt 2 & 0 \\
  \end{array} \right]}_S X(t) \, dW(t)\,,
$$
в котором $W(t)$~--- скалярный винеровский процесс. 
Соответствующее линейное стохастическое дифференциальное 
уравнение Стратоновича имеет вид:
\begin{multline*}
  d_{1/2} X(t) = \underbrace{\left[ \begin{array}{cc}
    0 & 1 \\
    -1 & 0 \\
  \end{array} \right]}_A X(t) \, dt +{}\\
  {}+
  \underbrace{\left[ \begin{array}{cc}
    0 & \sqrt 2 \\
    -\sqrt 2 & 0 \\
  \end{array} \right]}_S X(t) \, d_{1/2} W(t).
\end{multline*}

Аналитическое решение этих уравнений можно записать, 
используя матричные экспоненты \cite{Ave_ACMMENSP17}: 
$X(t) \hm= \exp{At} \exp{SW(t)}  X(t_0)$, $t\hm \geqslant t_0$. 
Несложно проверить, что собственные числа матриц~$A$ и~$S$~--- 
комп\-лекс\-но-со\-пря\-жен\-ные с~нулевой действительной частью, определители 
матриц~$\exp{At}$ и~$\exp{SW(t)}$ равны единице и~эти матрицы задают 
ортогональные линейные преобразования на плоскости, а~именно: повороты 
вокруг начала координат. Такие преобразования сохраняют расстояние между 
точками в~$\mathds{R}^2$: $|X(t)| \hm= |X(t_0)|$, поэтому 
траектории случайного процесса $X(t)$ принадлежат круговому цилиндру в~$\mathrm{T} \times \mathds{R}^2$, а фазовые траектории~--- это окружности 
с~цент\-ром в~начале координат (точка покоя~--- центр), радиус 
которых определяется начальными данными~$X(t_0)$.

Математическое ожидание для случайного процесса~$X(t)$ 
также можно выразить с~помощью мат\-рич\-ной экспоненты: $\mathbb{E} X(t) 
\hm= \exp{Ft} \, \mathbb{E} X(t_0)$ (здесь и~далее $\mathbb{E}$~--- 
знак математического ожидания), собственные числа матрицы~$F$~--- комп\-лекс\-но-со\-пря\-жен\-ные 
с~отрицательной действительной частью и~для математического ожидания 
верно соотношение:
$$
{e}^t |\mathbb{E} X(t)|
 ={e}^{t_0} \left\vert \mathbb{E} X(t_0)\right\vert \,. 
 $$

Таким образом, траектории случайного процесса и~его математическое 
ожидание принадлежат разным многообразиям $\mathrm{T} \times \mathds{R}^2$. 
Кривая математического ожидания принадлежит конусу 
в~$\mathrm{T} \times \mathds{R}^2$~--- динамическому многообразию, а проекция 
этой кривой на фазовую плоскость~--- это логарифмическая спираль (точка покоя~--- 
устойчивый фокус).

Если же рассмотреть задачу оптимальной фильт\-ра\-ции, т.\,е.\ оценивать траектории 
случайного процесса по результатам косвенных измерений, то результаты оценивания 
по критерию минимума\linebreak среднеквадратической ошибки не будут удовлетворять ни одному 
из указанных инвариантных соотношений и,~следовательно, закон сохранения, который 
выполняется для траекторий случайного процесса~$X(t)$, не будет выполняться для 
результатов оценивания этих траекторий. Результаты оценивания будут находиться 
в~об\-ласти, ограниченной круговым цилиндром и~конусом.

Цель статьи состоит в~описании класса стохастических дифференциальных 
систем, для которых одному и~тому же многообразию принадлежат не 
только траектории системы, но и~результат решения задачи оптимальной 
фильтрации по критерию минимума среднеквадратической ошибки оценивания.

Статья помимо введения и~заключения содержит~5~разделов. В~разд.~2 
сформулирована задача оптимальной фильтрации на гладком многообразии. 
В~разд.~3 приведены необходимые и~достаточные условия принадлежности 
траекторий стохастической системы заданному многообразию. Раздел~4 
содержит основной результат статьи~--- необходимые и~достаточные условия
 принадлежности решения задачи оптимальной фильтрации по критерию минимума 
 среднеквадратической ошибки оценивания многообразию, которому принадлежат 
 оцениваемые траектории. В~разд.~5 результаты, полученные в~разд.~4, 
 обобщаются на стохастические системы при наличии пуассоновских возмущений. 
 В~разд.~6 приводится модельный пример стохастической системы, для 
 которой одному и~тому же многообразию принадлежат не только 
 траектории этой системы, но и~результат решения задачи оптимальной фильтрации.



\section{Постановка задачи оптимальной фильтрации}

В работе рассматривается стохастическая система наблюдения, 
задаваемая стохастическими дифференциальными уравнениями Ито:

\vspace*{-2pt}

\noindent
\begin{align}
\label{eqItoX}
\hspace*{-2mm}  dX(t)& = f \left( t,X(t) \right) dt + \sigma \left( t,X(t) \right) dW(t)\,,\notag\\
  &\hspace*{44mm} X\left(t_0\right) = X_0\,;
\\
\label{eqItoY}
\hspace*{-2mm}  dY(t) &= c \left( t,X(t) \right) dt + \zeta(t) dV(t)\,,\enskip Y\left(t_0\right) = Y_0\,,
\end{align}
где $X \in \mathds{R}^n$~--- ненаблюдаемый вектор состояния; 
$Y \hm\in \mathds{R}^m$~--- вектор измерений; $t \hm\in \mathrm{T}$, $\mathrm{T}\hm = 
[t_0,T]$~--- заданный отрезок времени; $W(t)$ и~$V(t)$~--- $s$- 
и~$d$-мер\-ные независимые винеровские процессы; $f(t,x)$, $c(t,x)$,
 $\sigma(t,x)$ и~$\zeta(t)$~--- заданные век\-тор-функ\-ции и~матричные 
 функции соответствующих размеров; $X_0$ и~$Y_0$~--- начальный вектор 
 состояния и~начальный вектор измерений. Распределение~$X_0$ известно, а~$Y_0$, 
 как правило,~--- нулевой вектор. Функции $f(t,x)$, $c(t,x)$, 
 $\sigma(t,x)$ и~$\zeta(t)$ удовлетворяют условиям существования 
 и~единственности решения стохастических дифференциальных уравнений~\cite{OksSul_05}. 
 Кроме того, $\eta(t) = \zeta(t) \  \zeta^\trans(t)$~--- не\-вы\-рож\-ден\-ная мат\-ри\-ца, 
 для которой существует обратная матрица $q(t) \hm= \eta^{-1}(t)$.

Предполагается, что траектории случайного процесса~$X(t)$ 
принадлежат гладкому многообразию $\mathcal{M} \hm\subset \mathrm{T} \times 
\mathbb{R}^n$, которое определяется соотношением $\mathcal{M} \hm= 
\{(t,x) \in \mathrm{T} \times \mathbb{R}^n \colon M(t,x) \hm= C \hm= 
{const}\}$. Здесь $M(t,x)$~--- скалярная функция, не равная 
постоянной, непрерывно дифференцируемая по переменной~$t$ и~дважды 
непрерывно дифференцируемая по координатам вектора~$x$. Стохастическую систему, 
которая задается уравнением~\eqref{eqItoX}, будем называть инвариантной. 
Для инвариантной системы почти наверное $(t,X(t)) \hm\in \mathcal{M}$, если 
$(t_0,X_0)\hm \in \mathcal{M}$, т.\,е.\ $M(t,X(t))\hm = M(t_0,X_0)$. 
Подобных ограничений на случайный процесс~$Y(t)$ не накладывается.

Задача оптимальной фильтрации состоит в~нахождении 
оценки~$\hat X(t)$ по результатам измерений $Y_0^t \hm= \{Y(\tau)$,
$ \tau \hm\in [t_0,t) \}$, т.\,е.\ $\hat X(t)\hm = \psi(t,Y_0^t)$, 
где $\psi(t,Y_0^t)$~--- функция, обеспечивающая в~каждый момент времени $t \hm\in 
\mathrm{T}$ выполнение условия:

\vspace*{2pt}

\noindent
$$
  \mathbb{E} \left[ \left( X(t) - \hat X(t) \right)^\trans \left( X(t) - \hat X(t) \right) 
  \right] \rightarrow \min\limits_{\psi(t, \, \cdot \, )}\,,
$$

\vspace*{-2pt}

\noindent
т.\,е.\ решается задача оптимальной фильтрации по критерию минимума 
среднеквадратической ошибки оценивания.

Решение этой задачи записывается в~виде апостериорного математического 
ожидания~\cite{Sin_07}:

\vspace*{2pt}

\noindent
$$
  \hat X(t) = \mathbb{E} \left[ 
  X(t) | Y_0^t \right] = \int\limits_{\mathds{R}^n} x p(t,x|Y_0^t) \,dx\,,
$$


\noindent
или~\cite{BaiCri_09}:
\begin{equation}
\label{eqParticleEstimation}
  \hat X(t) = \fr{\mathbb{E}[\omega(t) X(t)]}{\mathbb{E} \, \omega(t)}\,,
\end{equation}
где $p(t,x|Y_0^t)$~--- апостериорная плот\-ность ве\-ро\-ят\-ности вектора состояния~$X$; 
$\omega(t)$~--- весовая функция:

\vspace*{-4pt}

\noindent
\begin{multline*}
  \omega(t) = \exp \left\{ \int\limits_{t_0}^t c^\trans(\tau,X(\tau)) 
  q(\tau) \,dY(\tau) -{}\right.\\
\left.  {}- \fr{1}{2} \int\limits_{t_0}^t c^\trans(\tau,X(\tau)) 
  q(\tau) c(\tau,X(\tau)) \,d\tau \right\}.
\end{multline*}

Отметим, что на последнем соотношении основан непрерывный фильтр 
частиц и~его различные варианты~\cite{BaiCri_09, Ryb_17}.

\vspace*{-4pt}

\section{Условия инвариантности}

Запишем стохастическое дифференциальное уравнение в~форме 
Стратоновича для случайного процесса~$X(t)$:

\vspace*{-2pt}

\noindent
\begin{multline}
\label{eqStr}
  d_{1/2} X(t) = a \left( t,X(t) \right) dt + \sigma \left( t,X(t) \right) 
  d_{1/2} W(t)\,,\\ 
   X\left(t_0\right) = X_0\,.
\end{multline}
В этом уравнении $a(t,x)$~--- век\-тор-функ\-ция 
той же размерности, что и~функция~$f(t,x)$:
\begin{equation}
\label{eqStr2Ito}
  a(t,x) = f(t,x) - \fr{1}{2} \sum\limits_{l = 1}^s 
  {\fr{\partial \sigma_{* l}(t,x)}{\partial x} \, \sigma_{* l}(t,x)}\,,
\end{equation}
где $\sigma_{* l}(t,x)$~--- столбец матричной функции~$\sigma(t,x)$ с~номером~$l$, 
$l \hm= 1,2,\dots,s$.

Для функций $M(t,x)$, $a(t,x)$ и~$\sigma(t,x)$, определяющих гладкое 
многообразие~$\mathcal{M}$ и~уравнение~\eqref{eqStr}, на траекториях 
случайного процесса~$X(t)$ должны выполняться следующие условия:
\begin{equation}
\label{eqCondition1}
  \fr{\partial M(t,x)}{\partial t} + \sum\limits_{i=1}^n a_i(t,x)  
  \fr{\partial M(t,x)}{\partial x_i} = 0\,;
\end{equation}
\begin{equation}
\label{eqCondition2}
  \sum\limits_{i=1}^n \sigma_{il}(t,x) 
   \fr{\partial M(t,x)}{\partial x_i} = 0\,, \enskip l = 1,2,\dots,s\,.
\end{equation}

Эти условия эквивалентны равенству нулю дифференциала 
Стратоновича для случайного процесса $M(t,X(t))$~\cite{Dub_12}. Условия для 
функции~$f(t,x)$ можно получить из равенства нулю дифференциала Ито 
для случайного процесса~$M(t,X(t))$ или подставить~\eqref{eqStr2Ito} 
в~\eqref{eqCondition1}. Такие условия приведены 
в~\cite{Dub_89, Dub_12, Kar_14, Kar_15}, в~этих же работах подробно изложена 
тео\-рия первых интегралов стохастических дифференциальных уравнений 
(функция $M(t,x)$~--- первый интеграл
уравнений~\eqref{eqItoX} и~\eqref{eqStr}), 
приложение этой тео\-рии к~за-\linebreak\vspace*{-12pt}

\columnbreak

\noindent
 дачам программного управления стохастическими 
динамическими сис\-те\-ма\-ми, там же приведены многочисленные примеры.

Отметим, что с~геометрической точки зрения~\eqref{eqCondition1}~--- 
условие ортогональности блочного вектора $[  1 \  a^\trans(t,x)]^\trans$ 
и~обобщенного градиента $\nabla_{t,x} M(t,x)$, а~\eqref{eqCondition2}~--- 
условие ортогональности каж\-до\-го столбца матрицы~$\sigma(t,x)$ 
и~градиента $\nabla M(t,x)$ в~$\mathds{R}^n$ $\forall t \hm\in \mathrm{T}$. 
В~случае $M(t,x) \hm= M(x)$ условие~\eqref{eqCondition1}~--- 
это условие ортогональности вектора~$a(t,x)$ и~градиента $\nabla M(t,x) 
\hm= \nabla M(x)$ в~$\mathds{R}^n$ $\forall t \hm\in \mathrm{T}$.

\vspace*{-4pt}

\section{Условия инвариантности в~среднем}

Опишем стохастическую систему, задаваемую стохастическим 
дифференциальным уравнением Ито~\eqref{eqItoX} и~соответствующим стохастическим 
дифференциальным уравнением Стратоновича~\eqref{eqStr}, с~дополнительным условием:
$$
  (t,\mathbb{E} X(t)), (t,\mathbb{E}[X(t)|Y_0^t]) \in \mathcal{M},
$$
т.\,е.\ заданному многообразию~$\mathcal{M}$ принадлежат не только 
траектории случайного процесса~$X(t)$, но и~априорное, а также 
апостериорное математическое ожидание случайного процесса~$X(t)$. Для такого 
класса стохастических систем решение задачи оптимальной фильтрации принадлежит 
тому же многообразию, что и~оцениваемые траектории, т.\,е.\ $(t,\hat X(t)) \hm\in 
\mathcal{M}$.

Для этого определим линейную по вектору $x \hm\in \mathds{R}^n$ функцию
\begin{equation}
\label{eqDefM}
  M(t,x) = \left( \vartheta(t),x \right) = 
  \vartheta_1(t) x_1 + \cdots + \vartheta_n(t) x_n\,,
\end{equation}
где $\vartheta(t)$~--- дифференцируемая век\-тор-функ\-ция, координаты которой 
одновременно не обращаются в~нуль: $|\vartheta(t)| \hm> 0$, $(\vartheta(t),x)$~--- 
скалярное произведение в~$\mathds{R}^n$. Функция $\vartheta(t) \hm= \nabla M(t,x)$ 
задает вектор нормали к~гиперплоскости $M(t,x) \hm= C$ в~$\mathds{R}^n$ $\forall t 
\hm\in \mathrm{T}$. В~$\mathrm{T} \times \mathds{R}^n$ многообразие $M(t,x) \hm= C$ 
ги\-пер\-плос\-костью в~общем случае не является. Тогда

\vspace*{-6pt}

\noindent
\begin{multline*}
 \! \!\!\mathbb{E} M(t,X(t)) = \mathbb{E} \left[ 
  \vartheta_1(t) X_1(t) + \cdots + \vartheta_n(t) X_n(t) \right] ={} \\
\hspace*{7pt}{}  = \vartheta_1(t) \, \mathbb{E} X_1(t) + \cdots + \vartheta_n(t) 
 \mathbb{E} X_n(t) = M(t,\mathbb{E} X(t))\,.\hspace*{-7pt}
\end{multline*}
Аналогично

\vspace*{-6pt}

\noindent
\begin{multline*}
  \mathbb{E} \left[ M(t,X(t)) | Y_0^t \right] ={}\\
  {}= 
  \mathbb{E} \left[ \vartheta_1(t) X_1(t) + \cdots + \vartheta_n(t) X_n(t) \,|\, 
  Y_0^t \right] ={} \\
{}  = \vartheta_1(t) \, \mathbb{E} \left[ X_1(t) | Y_0^t \right] + \cdots + 
\vartheta_n(t) \, \mathbb{E} \left[ X_n(t) | Y_0^t \right] = {}\\
{}=
M\left(t,\mathbb{E}\left[X(t)|Y_0^t\right]\right)\,;
\end{multline*}
следовательно, если $M(t,X(t)) = C$, 
то $M(t,\mathbb{E} X(t)) \hm= C$ и~$M(t,\mathbb{E}[X(t)|Y_0^t]) \hm= C$.

\pagebreak

Несмотря на линейность по вектору $x \hm\in \mathds{R}^n$ функции~$M(t,x)$, 
стохастическая система, траектории которой принадлежат многообразию~$\mathcal{M}$, 
может быть как линейной, так и~нелинейной. Чтобы конструктивно описать такую систему, 
определим $n\hm-1$ линейно независимых векторов  $N_1, \ldots, N_{n-1}$, 
ортогональных градиенту~$\nabla M(t,x)$. Эти векторы являются функциями переменной 
$t \hm\in \mathrm{T}$ со значениями в~$\mathds{R}^n$, зависимость от~$t$ 
для краткости опущена.

Векторы $N_1$, \dots, $N_{n-1}$ образуют базис линейного подпространства $M(t,x) \hm= 0$ 
в~$\mathds{R}^n$ $\forall t \hm\in \mathrm{T}$. Их можно выбрать, например, 
следующим образом:
$$
  N_i = \left[ E_i^\trans \  -\fr{\vartheta_i(t)}{\vartheta_n(t)} \right]^\trans,\enskip
  i = 1,\dots,n-1\,,
$$
где $E_i$~--- единичные векторы в~$\mathds{R}^{n-1}$ (столбцы единичной матрицы~$E$ 
порядка~$n\hm-1$). Чтобы такое определение было корректным, дополнительно потребуем,
 чтобы $\vartheta_n(t) \hm\neq 0$ на~$\mathrm{T}$, тем самым обеспечив и~выполнение 
 условия $|\vartheta(t)| \hm> 0$.

Кроме того, определим $n$ линейно независимых векторов $\tilde N_0, \tilde N_1, \dots, 
\tilde N_{n-1}$, ортогональных обобщенному градиенту~$\nabla_{t,x} M(t,x)$. 
Вектор~$\tilde N_0$ является функцией точки $(t,x) \hm\in \mathrm{T} \times 
\mathds{R}^n$, а остальные векторы~--- функции переменной $t \hm\in \mathrm{T}$ 
со значениями в~$\mathds{R}^{n+1}$:
\begin{multline*}
  \!\!\!\!\!\!\tilde N_0 = \begin{bmatrix} 1 &  0 &  \cdots  & 0 &  
  -\fr{\vartheta_0(t,x)}{\vartheta_n(t)} 
\end{bmatrix}^\trans\!\!; \\
  \tilde N_i = \begin{bmatrix}
   0 &  E_i^\trans &  -\fr{\vartheta_i(t)}{\vartheta_n(t)}  
\end{bmatrix}^\trans\!\!, \enskip  i = 1,\ldots,n-1\,,
\end{multline*}
где
\begin{multline*}
  \vartheta_0(t,x) = \fr{\partial M(t,x)}{\partial t} = \left( 
  \fr{\partial \vartheta(t)}{\partial t},x \right)
  = {}\\
  {}=\fr{\partial \vartheta_1(t)}{\partial t} \, x_1 + \cdots + 
  \fr{\partial \vartheta_n(t)}{\partial t} \, x_n\,.
\end{multline*}

Далее обозначим через $\mathcal{N}$ линейную оболочку векторов $N_1, \ldots, N_{n-1}$: 
$\mathcal{N} \hm= \mathrm{Lin} \{ N_1,\dots,N_{n-1} \}$,
а~через $\mathcal{N}_0$~--- линейное многообразие $N_0\hm + \mathcal{N}$:
 $\mathcal{N}_0 \hm= \{ \mathcal{V} \colon \mathcal{V}\hm = 
N_0\hm + N,\ N \hm\in \mathcal{N} \}$,
где вектор $N_0$ является функцией точки $(t,x)\hm \in \mathrm{T} \times 
\mathds{R}^n$ со значениями в~$\mathds{R}^n$:
$$
  N_0 = \begin{bmatrix} 
   0 & \cdots &~ 0~ & -\fr{\vartheta_0(t,x)}{\vartheta_n(t)} \end{bmatrix}^\trans.
$$

По построению произвольная линейная комбинация векторов $N_1, \ldots, N_{n-1}$ 
ортогональна градиенту~$\nabla M(t,x)$. Следовательно, равенство~\eqref{eqCondition2} 
можно переписать в~виде:
\begin{equation}\label{eqCondition2Geometry}
  \sigma_{* l}(t,x) \in \mathcal{N}\,,\enskip l = 1,2,\dots,s\,,
\end{equation}
или $\sigma_{* l}(t,x) \hm= q_1^l(t,x) N_1 + \cdots + q_{n-1}^l(t,x) N_{n-1}$, 
где скалярные функции $q_1^l(t,x), \ldots, q_{n-1}^l(t,x)$ 
могут быть выбраны произвольно при дополнительных условиях 
существования решения уравнения~\eqref{eqStr}, они представляют собой коэффициенты 
разложения столбца~$\sigma_{* l}(t,x)$ по линейно независимой сис\-те\-ме 
векторов $N_1, \dots, N_{n-1}$~--- базису линейного подпространства~$\mathcal{N}$.

Равенство~\eqref{eqCondition1} с~учетом введенных обозначений 
можно переписать следующим образом:
\begin{equation}
\label{eqCondition1Geometry}
  a(t,x) \in \mathcal{N}_0\,,
\end{equation}
или $a(t,x) \hm= N_0 \hm+ q_1^a(t,x) N_1 + \cdots + q_{n-1}^a(t,x) N_{n-1}$, где 
скалярные функции $q_1^a(t,x), \ldots, q_{n-1}^a(t,x)$, 
как и~ранее введенные функции~$q_r^l(t,x)$, могут быть выбраны произвольно 
при дополнительных условиях существования решения уравнения~\eqref{eqStr}. 
Если $M(t,x) \hm= M(x)$, то вектор~$N_0$ является нулевым 
и,~следовательно,~$\mathcal{N}$ и~$\mathcal{N}_0$ совпадают.

Аналог условия~\eqref{eqCondition1Geometry} для блочного вектора 
$[ 1 \  a^\trans(t,x) ]^\trans$ записывается в~виде:
$$
  [  1 \  a^\trans(t,x)]^\trans \in \tilde N_0 + \tilde{\mathcal{N}}\,,\enskip
  \tilde{\mathcal{N}} = \mathrm{Lin} \left\{ \tilde{N}_1,\dots,\tilde{N}_{n-1} \right\}.
$$

При выполнении условия~\eqref{eqCondition2Geometry} разность 
между век\-тор-функ\-ци\-ями $a(t,x)$ и~$f(t,x)$ согласно~\eqref{eqStr2Ito} 
представляет собой вектор, состоящий из суммы компонент вида:
\begin{multline*}
  \fr{1}{2} \, {\fr{\partial (q_k^l(t,x) N_k)}{\partial x} \, q_r^l(t,x) N_r} = {}\\
  {}=
  \fr{1}{2} \, q_r^l(t,x)  N_k  \left[ \nabla q_k^l(t,x) \right]^\trans  N_r\,,
\\
  k,r = 1,\dots,n-1\,, \enskip l = 1,2,\dots,s\,,
\end{multline*}
где произведение $[ \nabla q_k^l(t,x)]^\trans \, N_r$~--- 
это скалярная функция. Таким образом, если $a(t,x) \hm\in \mathcal{N}_0$ 
и~$\sigma_{* l}(t,x) \hm\in \mathcal{N}$, то $f(t,x) \hm\in \mathcal{N}_0$, или 
$f(t,x) \hm= N_0 \hm+ q_1^f(t,x) N_1 + \cdots + q_{n-1}^f(t,x) N_{n-1}$, 
где скалярные функции $q_1^f(t,x), \ldots, q_{n-1}^f(t,x)$ могут 
быть выбраны произвольно при дополнительных условиях существования 
решения уравнения~\eqref{eqItoX}. Для функции~$M(t,x)$, отличной от~\eqref{eqDefM}, 
условие $f(t,x) \hm\in \mathcal{N}_0$, вообще говоря, не выполняется; 
соответствующий пример приведен во введении.

\smallskip

\noindent
\textbf{Теорема~1.}\ 
\textit{Для того чтобы траектории стохастической дифференциальной системы, 
заданной уравнением Ито}~\eqref{eqItoX}, \textit{принадлежали многообразию~$\mathcal{M}$, 
которое определяется функцией}~\eqref{eqDefM}: 
$(t,X(t)) \hm\in \mathcal{M}$, \textit{и~при этом $(t,\mathbb{E} X(t)), 
(t,\mathbb{E}[X(t)|Y_0^t]) \hm\in \mathcal{M}$, необходимо и~достаточно, 
чтобы коэффициенты этого уравнения удовле\-тво\-ря\-ли условиям}:

\pagebreak

\noindent
$$
  f(t,x) \in \mathcal{N}_0\,; \enskip \sigma_{* l}(t,x) \in \mathcal{N}\,,\enskip
   l = 1,2,\dots,s\,,
$$
\textit{на траекториях случайного процесса}~$X(t)$.

\smallskip

\noindent
Д\,о\,к\,а\,з\,а\,т\,е\,л\,ь\,с\,т\,в\,о\ \ теоремы следует из предыдущих рассуждений. 
Если стохастическая дифференциальная система задается уравнением 
Стратоновича, то используется условие $a(t,x) \hm\in \mathcal{N}_0$.

Условия теоремы, а~также то, что векторы вида
\begin{multline*}
  \fr{\partial (q_k^l(t,x) N_k)}{\partial x} \, q_r^j(t,x) N_r = {}\\
  {}=
  q_r^l(t,x)  N_k  \left[ \nabla q_r^j(t,x) \right]^\trans  N_r\,,
\\
  k,r = 1,\ldots,n-1\,, \enskip l,j = 1,2,\ldots,s\,,
\end{multline*}
коллинеарны вектору $N_k$ $\forall t \in \mathrm{T}$, 
обеспечивают отсутствие погрешности, связанной с~отклонением 
численного решения от многообразия~$\mathcal{M}$, при чис\-лен\-ном интегрировании 
стохастического дифференциального уравнения~\eqref{eqItoX} или~\eqref{eqStr} 
и~дополнительном требовании $\vartheta(t) \hm= {const}$, т.\,е.\
 $M(t,x) \hm= M(x)$. Это связано с~видом соответствующих разностных схем, 
 для которых $X_{k+1} \hm= X_k \hm+ \Delta X_k$. Здесь $\Delta X_k$~--- 
 случайный вектор, зависящий от шага численного интегрирования~$h$, пары $(t_k,X_k)$ 
 и~принадлежащий $\mathcal{N} \hm= \mathcal{N}_0$. Например, для метода 
 Эй\-ле\-ра\,--\,Ма\-ру\-ямы
 {\looseness=1
 
 }
 
 \noindent
$$
  \Delta X_k = h  f(t_k,X_k) + \sqrt{h}  \sigma(t_k,X_k)  \Delta W_k\,,
$$
где $\Delta W_k$~--- $s$-мер\-ный случайный вектор, координаты 
которого независимы и~имеют стандартное нормальное распределение. 
Дискретные моменты времени~$t_k$ определяются при разбиении отрезка~$\mathrm{T}$ 
с~шагом~$h$. 

Погрешности такого типа анализировались в~работе~\cite{AveKarRyb_RJNAMM18} 
на примере методов Эй\-ле\-ра--Ма\-ру\-ямы, Мильштейна и~Платена, методов типа 
Рун\-ге--Кут\-ты и~типа Розенброка.



\section{Условия инвариантности при~пуассоновских возмущениях}

Рассмотрим систему наблюдения в~более общей постановке, 
которая принята в~\cite{Sin_SSI16, Sin_IP16, SinSinKor_SSI16, SinSinKor_IP17, SinSinSerKor_AiT18}, 
а~именно:
\begin{multline}
\label{eqItoXJump}
  dX(t) = f \left( t,X(t),Y(t),\rho \right) dt + {}\\
  {}+
  \sigma \left( t,X(t),Y(t),\rho \right) dW(t) + {} \\
  + \int\limits_\Theta \gamma \left( t,X(t-),Y(t-),\rho,\theta \right) 
  \nu\,(dt \times d\theta)\,,\\ X\left(t_0\right) = X_0\,;
\end{multline}

\vspace*{-12pt}

\noindent
\begin{multline}
\label{eqItoYJump}
  dY(t) = c \left( t,X(t),Y(t),\rho \right) dt + {}\\
  {}+
  \zeta \left( t,X(t),Y(t),\rho \right) dV(t) + {} \\
  + \int\limits_\Theta \delta \left( t,X(t-),Y(t-),\rho,\theta \right) \mu\,
  (dt \times d\theta)\,,\\ Y\left(t_0\right) = Y_0\,,
\end{multline}
где $X \in \mathds{R}^n$~--- ненаблюдаемый вектор состояния; 
$Y \hm\in \mathds{R}^m$~--- 
вектор измерений; $\rho \hm\in \mathrm{P} \hm\subset \mathds{R}^l$~--- 
вектор параметров; $t \hm\in \mathrm{T}$, $\mathrm{T}\hm = [t_0,T]$~--- 
заданный отрезок времени; $W(t)$ и~$V(t)$~--- $s$- и~$d$-мер\-ные 
независимые винеровские процессы; $f(t,x,y,\rho)$, $\sigma(t,x,y,\rho)$, 
$\gamma(t,x,y,\rho,\theta)$, $c(t,x,y,\rho)$, $\zeta(t,x,y,\rho)$ 
и~$\delta(t,x,y,\rho,\theta)$~--- заданные век\-тор-функ\-ции и~матричные 
функции соответствующих размеров; $\nu$ и~$\mu$~--- пуассоновские\linebreak
 меры 
на $\mathrm{T} \hm\times \Theta$, $\Theta \hm\subseteq \mathds{R}^k$, 
с~заданными характери\-сти\-че\-ски\-ми мерами, которые определяют интен\-сив\-ность 
со\-от\-вет\-ст\-ву\-ющих пуассоновских потоков и~законы распределения вектора~$\theta$ 
для уравнений~\eqref{eqItoXJump} и~\eqref{eqItoYJump}.

Наличие пуассоновской компоненты в~уравнении, описывающем ненаблюдаемый 
вектор состояния, не влияет на условия~\eqref{eqCondition2Geometry} 
и~\eqref{eqCondition1Geometry}, но требует дополнительного условия для 
функции $\gamma(t,x,y,\rho,\theta)$, которое должно выполняться на 
траекториях случайных процессов~$X(t)$ и~$Y(t)$:
$$
  M(t,x) = M \left( t,x + \gamma(t,x,y,\rho,\theta) \right).
$$

В силу линейности по вектору $x \hm\in \mathds{R}^n$ функции $M(t,x)$ 
имеем $M(t,\gamma(t,x,y,\rho,\theta))\hm = 0$, что эквивалентно ортогональности 
век\-тор-функ\-ции $\gamma(t,x,y,\rho,\theta)$ и~градиента $\nabla M(t,x)$ 
в~$\mathds{R}^n$ $\forall t \hm\in \mathrm{T}$, т.\,е.\ вектор 
$\gamma(t,x,y,\rho,\theta)$ можно разложить по линейно независимой 
системе векторов~$N_1$, \dots, $N_{n-1}$~--- базису линейного 
подпространства~$\mathcal{N}$. Переменные~$y$, $\rho$ и~$\theta$ 
входят в~это условие как параметры, они могут быть любыми с~учетом их 
области определения:
\begin{multline*}
  \gamma(t,x,y,\rho,\theta) ={}\\
  {}= q_1^\gamma(t,x,y,\rho,\theta) N_1 + 
  \cdots + q_{n-1}^\gamma(t,x,y,\rho,\theta) N_{n-1}\,.
\end{multline*}
Переменные $y$ и~$\rho$ войдут как параметры и~в условия~\eqref{eqCondition2Geometry} 
и~\eqref{eqCondition1Geometry}:
\begin{multline*}
  f(t,x,y,\rho) ={}\\
  {}= N_0 + q_1^f(t,x,y,\rho) N_1 + \cdots + q_{n-1}^f(t,x,y,\rho) N_{n-1}\,;
\end{multline*}

\vspace*{-12pt}

\noindent
\begin{multline*}
  \sigma_{* l}(t,x,y,\rho) = {}\\
  {}=q_1^l(t,x,y,\rho) N_1 + 
  \cdots + q_{n-1}^l(t,x,y,\rho) N_{n-1}\,,\\
   l = 1,2,\dots,s\,.
\end{multline*}
Если уравнения для случайных процессов $X(t)$ и~$Y(t)$ записать в~форме 
Стратоновича, то для функции $a(t,x,y,\rho)$~--- соответствующего коэффициента 
в~уравнении для случайного процесса~$X(t)$~--- будем иметь:
\begin{multline*}
  a(t,x,y,\rho) ={}\\
  {}= N_0 + q_1^a(t,x,y,\rho) N_1 + \cdots + 
  q_{n-1}^a(t,x,y,\rho) N_{n-1}\,.
\end{multline*}
Здесь все коэффициенты при векторах $N_1, \ldots$\linebreak $\ldots, N_{n-1}$ могут быть выбраны 
произвольно при дополнительных условиях существования решения 
стохастических дифференциальных уравнений.

\textbf{Теорема 2.} \textit{Для того чтобы траектории стохастической дифференциальной 
сис\-те\-мы, заданной уравнением Ито}~\eqref{eqItoXJump}, 
\textit{принадлежали многообразию~$\mathcal{M}$, которое определяется
 функцией}~\eqref{eqDefM}:\linebreak 
$(t,X(t)) \hm\!\in\! \mathcal{M}$, \textit{и~при этом $(t,\mathbb{E} X(t)), 
(t,\mathbb{E}[X(t)|Y_0^t]) \hm\in \mathcal{M}$, необходимо и~достаточно, 
чтобы коэффициенты этого уравнения удовле\-тво\-ря\-ли условиям}:
\begin{gather*}
  f(t,x,y,\rho) \in \mathcal{N}_0\,; \\
   \sigma_{* l}(t,x,y,\rho) \in \mathcal{N}\,,\enskip 
   l = 1,2,\dots,s\,;\\
   \gamma(t,x,y,\rho,\theta) \in \mathcal{N}
\end{gather*}
\textit{на траекториях случайных процессов $X(t)$ и~$Y(t)$ 
и~$\forall \rho \hm\in \mathrm{P}$, $\theta \in \Theta$}.

\smallskip

Отметим, что поскольку ограничений на случайный процесс~$Y(t)$ в~этой работе 
не накладывает\-ся, то зависимость коэффициентов уравнения\,\eqref{eqItoXJump} 
от вектора измерений, зависимость мат\-рич\-ной функции при дифференциале~$dV(t)$ от 
ненаблюдаемого вектора состояния и~вектора измерений, а~также наличие пуассоновской 
компоненты в~уравнении измерителя не усложняют условий инвариантности, 
которые сформулированы выше. Однако все перечисленные факторы усложняют 
алгоритмы нахождения оптимальной оценки~$\hat X(t)$, в~частности 
формула~\eqref{eqParticleEstimation} здесь не применима.

\vspace*{-4pt}

\section{Модельный пример}

В качестве примера рассмотрим двумерную стохастическую систему вида~\eqref{eqItoX}, 
траектории которой принадлежат гладкому многообразию 
$\mathcal{M} \hm\subset \mathrm{T} \times \mathbb{R}^2$. 
Это многообразие задается уравнением $M(t,x) \hm= -2 x_1 \hm+ \mathrm{e}^{-t} x_2 
\hm= C \hm= {const}$, т.\,е.\ функция $M(t,x)$ имеет вид~\eqref{eqDefM} при 
$n \hm= 2$, $x \hm= [  x_1 \ x_2 ]^\trans$ и~$\vartheta(t) \hm= 
[ -2 \  \mathrm{e}^{-t}]^\trans$. Таким образом,
\begin{gather*}
  \vartheta_0(t,x) = \fr{\partial M(t,x)}{\partial t} = -{e}^{-t} x_2\,;\\ 
   \vartheta_1(t) = -2\,;\quad \vartheta_2(t) = {e}^{-t}.
\end{gather*}

\begin{table*}[b]\small %[ht]
\begin{center}


\begin{tabular}{|c|c|c|}
\multicolumn{3}{c}{Отклонения от заданного многообразия}\\
\multicolumn{3}{c}{\ }\\[-6pt]
  \hline
  $h$ & $\mathbb{E} |M(1,X(1)) - M(0,X(0))|$ $\vphantom{\Bigl|}$ & 
  $\mathbb{E} |M(1,\hat X(1)) - M(0,X(0))|$ \\
  \hline
  &&\\[-9pt]
  $10^{-2}$ & 0,006488 & 0,006821 \\
  $10^{-3}$ & 0,000668 & 0,000685 \\
  $10^{-4}$ & 0,000064 & 0,000066 \\
  \hline
\end{tabular}
\end{center}
\end{table*}

Согласно методике, изложенной выше, определим два вектора:

\noindent
\begin{align*}
  N_0 &= \left[  0 \  -\fr{\vartheta_0(t,x)}{\vartheta_2(t)}  \right]^\trans 
  = [  0 \  x_2  ]^\trans;\\
  N_1 &= \left[  1 \  -\fr{\vartheta_1(t)}{\vartheta_2(t)}  \right]^\trans =
   [  1 \ 2 {e}^t ]^\trans.
\end{align*}

Опишем стохастическую дифференциальную систему, 
траектории которой принадлежат заданному гладкому 
многообразию~$\mathcal{M}$. Положим размерность винеровского процесса $s \hm= 1$. 
Тогда согласно теореме~1 имеем $f(t,x) \hm\in \mathcal{N}_0$, $\sigma(t,x)\hm \in 
\mathcal{N}$, где $\mathcal{N}_0$~--- линейное многообразие, а $\mathcal{N}$~--- 
линейное подпространство, построенные на векторах~$N_0$ и~$N_1$, а~именно:
$$
  f(t,x) = N_0 + q^f(t,x) N_1\,;\enskip \sigma(t,x) = q(t,x) N_1\,,
$$
где скалярные функции~$q^f(t,x)$ и~$q(t,x)$ могут быть выбраны произвольно 
при дополнительных условиях существования решения стохастического дифференциального 
уравнения с~коэффициентами~$f(t,x)$ и~$\sigma(t,x)$. Например, зададим эти функции 
так, чтобы уравнение~\eqref{eqItoX} было линейным: 
$$
q^f(t,x) = {e}^{-t} 
\left(x_1 - x_2\right)\,;\quad 
q(t,x) = {e}^{-t}\,.
$$
 Тогда
\begin{gather*}
  f(t,x) = \left[  {e}^{-t} \left(x_1 - x_2\right) \  \ \ \ 2 x_1 - x_2 
   \right]^\trans;\\
   \sigma(t,x) = [  {e}^{-t} \  2  ]^\trans\,.
\end{gather*}

Кроме функций $f(t,x)$ и~$\sigma(t,x)$, определяющих уравнение
 ненаблюдаемого вектора состояния сис\-те\-мы, зададим функции~$c(t,x)$ и~$\zeta(t)$, 
 которые входят в~уравнение измерителя~\eqref{eqItoY}:
$$
  c(t,x) = x_1 + x_2; \enskip \zeta(t) = 0{,}05\,.
$$

Следовательно, уравнения~\eqref{eqItoX} и~\eqref{eqItoY} 
для рас\-смат\-ри\-ва\-емо\-го примера имеют вид:
\begin{align*}
  dX(t) &= \left[ \begin{array}{cc}
    \mathrm{e}^{-t} & -\mathrm{e}^{-t} \\
    2 & -1 \\
  \end{array} \right] X(t) \, dt +
  \left[ \begin{array}{c}
    \mathrm{e}^{-t} \\
    2 \\
  \end{array} \right] dW(t), \\
 & \hspace*{30mm}X(t) = [  X_1(t) \  X_2(t) ]^\trans;
\\
  dY(t) &= \left( X_1(t) + X_2(t) \right) dt + 0{,}05\, dV(t)\,.
\end{align*}

 

С помощью выбора функций $q^f(t,x)$ и~$q(t,x)$ можно сформировать 
нелинейные уравнения стохастической системы, однако ограничимся линейным 
случаем и~воспользуемся фильтром Кал\-ма\-на--Бью\-си для нахождения 
оптимальной оценки вектора состояния по критерию минимума среднеквадратической ошибки.

Зададим отрезок времени $\mathrm{T} \hm= [0,1]$ и~нулевые начальные данные: 
$X_1(0) \hm= X_2(0) \hm= Y(0) \hm= 0$.
Результаты моделирования траектории 
случайного\linebreak\vspace*{-12pt}


{ \begin{center}  %fig1
 \vspace*{-3pt}
  \mbox{%
 \epsfxsize=79mm 
 \epsfbox{ryb-1.eps}
 }


\end{center}


\noindent
{\small{Выборочная траектория случайного процесса $X(t)$ и~ее оценка}}
}

\vspace*{9pt}

\addtocounter{figure}{1}

\noindent
  процесса $X(t)$ методом Эй\-ле\-ра--Ма\-ру\-ямы 
с~шагом численного интегрирования $h \hm= 10^{-3}$ показаны на рисунке, 
на нем же показан результат оценивания с~по\-мощью фильтра Кал\-ма\-на--Бью\-си 
(оце\-ни\-ва\-емая траектория показана черным, а~ее оценка~--- серым, ось времени 
направлена вправо, ось~$x_1$~--- влево, ось~$x_2$~--- вверх), 
а~также изображена поверхность $M(t,x) \hm= -2 x_1 \hm+ {e}^{-t} x_2 \hm= 0$ 
(нуль в~правой час\-ти является следствием нулевых начальных данных для оцениваемой 
траектории).


Кроме построения одной траектории и~на\-хож\-де\-ния ее оценки была 
проведена серия вы\-чис\-ли\-тель\-ных экспериментов: 
моделировались ан\-самб\-ли из~1000~траекторий 
случайных процессов $X(t)$ и~$Y(t)$ методом Эй\-ле\-ра--Ма\-ру\-ямы с~шагами %\linebreak 
чис\-лен\-но\-го интегрирования~$h$, равными~$10^{-2}$, $10^{-3}$ и~$10^{-4}$. 
Для каждой траектории $X(t)$ была\linebreak найде\-на оптимальная оценка $\hat X(t)$ 
с~по\-мощью фильт\-ра Кал\-ма\-на--Бью\-си по соответствующей траектории~$Y(t)$. 
На основе этих результатов моделирования были вычислены оценки среднего значения 
для величин $|M(1,X(1)) \hm- M(0,X(0))|$ и~$|M(1,\hat X(1)) \hm- M(0,X(0))|$,
 которые показывают отклонения траектории и~ее оценки, полученных численно, 
 от заданного многообразия в~момент времени $T \hm= 1$. Результаты представлены в~виде 
 таб\-лицы.



Из полученных данных видно, что при уменьшении шага численного 
интегрирования~$h$ в~10~раз средние отклонения уменьшаются почти в~10~раз. 
Это соответствует первому порядку слабой сходимости метода Эй\-ле\-ра--Ма\-ру\-ямы, 
т.\,е.\ отклонение траекторий и~их оценок от заданного многообразия вызваны 
погрешностью при численном решении стохастических дифференциальных уравнений.



\section{Заключение}

В статье описан класс стохастических дифференциальных систем, 
для которых одному и~тому же многообразию принадлежат не только 
траектории системы, но и~результат решения задачи оптимальной 
фильтрации по критерию минимума среднеквадратической ошибки оценивания, 
а~именно: приведены необходимые и~достаточные условия принадлежности траекторий 
стохастической системы и~их оценок заданному гладкому многообразию. Рассмотрены 
системы как диффузионного типа, так и~диф\-фу\-зи\-он\-но-скач\-ко\-об\-раз\-но\-го 
типа, т.\,е.\ 
при наличии как винеровских, так и~пуассоновских возмущений. Приведен модельный 
пример линейной стохастической дифференциальной системы, для которой 
траектории и~оценки этих траекторий\linebreak с~по\-мощью фильтра Кал\-ма\-на--Бью\-си 
принадлежат\linebreak динамическому многообразию. В~дальнейшем планируется рассмотреть 
многообразия меньшей размерности (здесь рассмотрены многообразия, размерность 
которых на единицу меньше размера оценива\-емо\-го вектора состояния), а~также 
обобщить полученные результаты для стохастических дифференциальных сис\-тем с~изменениями 
структуры, это даст возможность перейти к~задачам фильт\-ра\-ции на 
ку\-соч\-но-глад\-ких многообразиях.


{\small\frenchspacing
 {%\baselineskip=10.8pt
 \addcontentsline{toc}{section}{References}
 \begin{thebibliography}{99}



\bibitem{Sin_IP16}
\Au{Синицын И.\,Н.} Ортогональные субоптимальные 
фильтры для нелинейных стохастических систем на многообразиях~// Информатика и~её 
применения, 2016. Т.~10. Вып.~1. С.~34--44.

\bibitem{Sin_SSI16}
\Au{Синицын И.\,Н.} Нормальные и~ортогональные субоптимальные 
фильтры для нелинейных стохастических систем на многообразиях~// 
Системы и~средства информатики, 2016. Т.~26. №\,1. С.~199--226.

\bibitem{SinSinKor_SSI16}
\Au{Синицын И.\,Н., Синицын~В.\,И., Корепанов~Э.\,Р.} 
Эллипсоидальные субоптимальные фильтры для нелинейных стохастических 
систем на многообразиях~// Системы и~средства информатики, 2016. Т.~26. №\,2. С.~79--97.

\bibitem{SinSinKor_IP17}
\Au{Синицын И.\,Н., Синицын~В.\,И., Корепанов~Э.\,Р.} Модифицированные 
эллипсоидальные услов\-но-оп\-ти\-маль\-ные фильтры для нелинейных стохастических 
систем на многообразиях~// Информатика и~её применения, 2017. Т.~11. Вып.~2. С.~101--111.

\bibitem{SinSinSerKor_AiT18}
\Au{Синицын И.\,Н., Синицын~В.\,И., Сергеев~И.\,В., Корепанов~Э.\,Р.} 
Методы эллипсоидальной фильтрации процессов в~нелинейных стохастических системах 
на многообразиях // Автоматика и~телемеханика, 2018. №\,1. С.~147--161.

\bibitem{Dub_89}
\Au{Дубко В.\,А.} Вопросы теории и~применения стохастических дифференциальных уравнений. --- Владивосток: ДВО АН СССР, 1989. 185~с.

\bibitem{Dub_12}
\Au{Дубко В.\,А.} Стохастические дифференциальные уравнения. Избранные разделы.~--- 
Киев: Логос, 2012. 68~с.

\bibitem{Kar_14}
\Au{Карачанская Е.\,В.} Случайные процессы с~инвариантами.~--- Хабаровск: ТОГУ, 2014. 
148~с.

\bibitem{Kar_15}
\Au{Карачанская Е.\,В.} Интегральные инварианты стохастических систем 
и~программное управление с~вероятностью~1.~--- Хабаровск: ТОГУ, 2015. 148~с.

\bibitem{Ave_ACMMENSP17}
\Au{Аверина Т.\,А.} Аналитические и~численные решения 
трех систем стохастических дифференциальных уравнений с~инвариантами~// 
Аналитические и~численные методы моделирования ес\-те\-ст\-вен\-но-на\-уч\-ных 
и~социальных проблем: Тр. XII Междунар. научн.-технич. конф.~--- 
Пенза: ПГУ, 2017. С.~3--8.

\bibitem{OksSul_05}
\Au{{\ptb{\O}}\,\,ksendal B., Sulem~A.} Applied stochastic control of jump diffusions.~--- 
Berlin: Springer, 2005. 214~p.

\bibitem{Sin_07}
\Au{Синицын И.\,Н.} Фильтры Калмана и~Пугачева.~--- М.: Логос, 2007. 
776~с.

\bibitem{BaiCri_09}
\Au{Bain A., Crisan~D.} Fundamentals of stochastic filtering.~--- 
New York, NY, USA: Springer, 2009. 394~p.

\bibitem{Ryb_17}
\Au{Рыбаков К.\,А.} Статистические методы анализа и~фильтрации в~непрерывных 
стохастических системах.~--- М.: МАИ, 2017. 176~с.

\bibitem{AveKarRyb_RJNAMM18}
\Au{Averina T.\,A., Karachanskaya~E.\,V., Rybakov~K.\,A.} 
Statistical analysis of diffusion systems with invariants~// Russ. 
J.~Numer. Anal.~M., 2018. Vol.~33. Iss.~1. P.~1--13.

 \end{thebibliography}

 }
 }

\end{multicols}

\vspace*{-3pt}

\hfill{\small\textit{Поступила в~редакцию 19.04.18}}

\vspace*{8pt}

%\pagebreak

%\newpage

%\vspace*{-28pt}

\hrule

\vspace*{2pt}

\hrule

%\vspace*{-2pt}

\def\tit{ON A CLASS OF FILTERING PROBLEMS ON~MANIFOLDS}


\def\titkol{On a class of filtering problems on manifolds}

\def\aut{K.\,A.~Rybakov}

\def\autkol{K.\,A.~Rybakov}

\titel{\tit}{\aut}{\autkol}{\titkol}

\vspace*{-11pt}


\noindent
Moscow Aviation Institute (National Research University), 
4~Volokolamskoye Shosse, Moscow 125993, Russian Federation

\def\leftfootline{\small{\textbf{\thepage}
\hfill INFORMATIKA I EE PRIMENENIYA~--- INFORMATICS AND
APPLICATIONS\ \ \ 2019\ \ \ volume~13\ \ \ issue\ 1}
}%
 \def\rightfootline{\small{INFORMATIKA I EE PRIMENENIYA~---
INFORMATICS AND APPLICATIONS\ \ \ 2019\ \ \ volume~13\ \ \ issue\ 1
\hfill \textbf{\thepage}}}

\vspace*{4pt}




\Abste{The goal of the paper is to describe stochastic differential 
systems whose trajectories belong to a~smooth manifold as an application 
to the optimal filtering problem. An additional condition is that not only 
system trajectories belong to the given manifold, but also the estimation 
results for these trajectories (solution of the optimal filtering problem 
with the minimum mean-squared error) belong to this manifold. Diffusion and 
jump-diffusion systems are considered. These systems can be driven by the 
Wiener process and the Poisson process. The main result is the conditions on 
coefficients of the equation for the estimated random process. These conditions 
are obtained on the basis of the first integral concept for the stochastic 
differential equation and some of its properties.}

\KWE{invariant; estimation; manifold; optimal filtering; random process; 
stochastic differential system}




\DOI{10.14357/19922264190103}

\vspace*{-15pt}

\Ack
\noindent
The work was supported by the Russian Foundation for Basic Research 
(project 17-08-00530-а).



%\vspace*{6pt}

  \begin{multicols}{2}

\renewcommand{\bibname}{\protect\rmfamily References}
%\renewcommand{\bibname}{\large\protect\rm References}

{\small\frenchspacing
 {%\baselineskip=10.8pt
 \addcontentsline{toc}{section}{References}
 \begin{thebibliography}{99}

\vspace*{-3pt}

\bibitem{2-ryb-1}
\Aue{Sinitsyn, I.\,N.} 2016. Ortogonal'nye suboptimal'nye fil'try dlya 
nelineynykh stokhasticheskikh sistem na mno\-go\-ob\-ra\-zi\-yakh 
[Orthogonal suboptimal filters for nonlinear stochastic systems on manifolds]. 
\textit{Informatika i~ee Primeneniya~--- Inform. Appl.} 10(1):34--44.

\bibitem{1-ryb-1}
\Aue{Sinitsyn, I.\,N.} 2016. Normal'nye i~ortogonal'nye suboptimal'nye 
fil'try dlya nelineynykh stokhasticheskikh sistem na mnogoobraziyakh 
[Normal and orthogonal conditionally optimal filters for nonlinear 
stochastic systems on manifolds]. \textit{Sistemy i~Sredstva Informatiki~--- 
Systems and Means of Informatics} 26(1):199--226.

\bibitem{3-ryb-1}
\Aue{Sinitsyn, I.\,N., V.\,I.~Sinitsyn, and E.\,R.~Korepanov.}
2016. Ellipsoidal'nye suboptimal'nye fil'try dlya nelineynykh stokhasticheskikh 
sistem na mnogoobraziyakh [Ellipsoidal conditionally optimal filters for 
nonlinear stochastic systems on manifolds]. \textit{Sistemy i~Sredstva Informatiki~--- 
Systems and Means of Informatics} 26(2):79--97.

\bibitem{4-ryb-1}
\Aue{Sinitsyn, I.\,N., V.\,I.~Sinitsyn, and E.\,R.~Korepanov.} 2017. 
Modifitsirovannye ellipsoidal'nye uslovno-optinal'nye fil'try dlya
nelineynykh stokhasticheskikh sistem na mnogoobraziyakh
[Modificated ellipsoidal conditionally optimal filters for nonlinear 
stochastic systems on manifolds]. 
\textit{Informatika i~ee Primeneniya~--- Inform. Appl.} 11(2):101--111.

\bibitem{5-ryb-1}
\Aue{Sinitsyn, I.\,N., V.\,I.~Sinitsyn, I.\,V.~Sergeev, and E.\,R.~Korepanov.}
2018. Methods of ellipsoidal filtration in nonlinear stochastic systems on manifolds. 
\textit{Autom. Rem. Contr.} 79(1):117--127.

\bibitem{6-ryb-1}
\Aue{Dubko, V.\,A.} 1989. \textit{Voprosy teorii i~primeneniya sto\-kha\-sti\-che\-skikh 
differentsial'nykh uravneniy} [Problems of theory and application 
of stochastic differential equations]. Vladivostok: Akad. Nauk SSSR. 185~p.

\bibitem{7-ryb-1}
\Aue{Dubko, V.\,A.} 2012. \textit{Stokhasticheskie differentsial'nye uravneniya. 
Izbrannye razdely} [Stochastic differential equations. Selected topics]. 
Kiev: Logos. 68~p.

\bibitem{8-ryb-1}
\Aue{Karachanskaya, E.\,V.} 2014. \textit{Sluchaynye protsessy s~invariantami} 
[Random process with invariants]. Khabarovsk: Pacific National University. 148~p.

\bibitem{9-ryb-1}
\Aue{Karachanskaya, E.\,V.} 2015. \textit{Integralnye invarianty 
sto\-kha\-sti\-che\-skikh sistem 
i~programmnoe upravlenie s~ve\-ro\-yat\-nost'yu~$1$} 
[Integral invariants of stochastic systems and program control with probability~1]. 
Khabarovsk: Pacific National University. 148~p.

\bibitem{10-ryb-1}
\Aue{Averina, T.\,A.} 2017. Analiticheskie i~chislennye resheniya trekh 
sistem stokhasticheskikh differentsial'nykh uravneniy s~invariantami 
[Analytic and numerical solutions of three systems of stochastic differential 
equations with invariants]. 
\textit{12th  Scientific and Technical Conference  (International)
``Analytical and Numerical Methods for Modeling Natural-Scientific and Social Problems''
Proceedings}. Penza. 3--8.

\bibitem{11-ryb-1}
\Aue{\mbox{{\ptb{\O}}ksendal}, B., and A.~Sulem.} 2005. 
\textit{Applied stochastic control of jump diffusions}. Berlin: Springer. 214~p.

\bibitem{12-ryb-1}
\Aue{Sinitsyn, I.\,N.} 2007. \textit{Fil'try Kalmana i~Pugacheva} [Kalman and Pugachev filters]. 
Moscow: Logos. 776~p.

\bibitem{13-ryb-1}
\Aue{Bain, A., and D.~Crisan.} 2009. \textit{Fundamentals of stochastic filtering}. 
New York, NY: Springer. 394~p.

\bibitem{14-ryb-1}
\Aue{Rybakov, K.\,A.} 2017. \textit{Statisticheskie metody analiza i~fil'tratsii v~nepreryvnykh 
stokhasticheskikh sistemakh} [Statistical methods of analysis and filtering for 
continuous stochastic systems]. Moscow: MAI. 176 p.

\bibitem{15-ryb-1}
\Aue{Averina, T.\,A., E.\,V.~Karachanskaya, and K.\,A.~Rybakov.} 
2018. Statistical analysis of diffusion systems with invariants. 
\textit{Russ. J.~Numer. Anal.~M.} 33(1):1--13.
\end{thebibliography}

 }
 }

\end{multicols}

\vspace*{-6pt}

\hfill{\small\textit{Received April 19, 2018}}

%\pagebreak

%\vspace*{-18pt}



\Contrl

\noindent
\textbf{Rybakov Konstantin A.} (b.\ 1979)~---
Candidate of Sciences (PhD) in physics and mathematics, 
associate professor, Moscow Aviation Institute (National Research University), 
4~Volokolamskoye Shosse, Moscow 125993, Russian Federation; \mbox{rkoffice@mail.ru}


\label{end\stat}

\renewcommand{\bibname}{\protect\rm Литература}         %3 
\def\stat{dukova}

\def\tit{О ПОИСКЕ МАКСИМАЛЬНЫХ ЧАСТЫХ И~МИНИМАЛЬНЫХ НЕЧАСТЫХ НАБОРОВ ПРОИЗВЕДЕНИЯ ЧАСТИЧНЫХ ПОРЯДКОВ}

\def\titkol{О поиске максимальных частых и~минимальных нечастых наборов произведения частичных порядков}

\def\aut{Н.\,А.~Драгунов$^1$, Е.\,В.~Дюкова$^2$}

\def\autkol{Н.\,А.~Драгунов, Е.\,В.~Дюкова}

\titel{\tit}{\aut}{\autkol}{\titkol}

\index{Драгунов Н.\,А.}
\index{Дюкова Е.\,В.}
\index{Dragunov N.\,A.}
\index{Djukova E.\,V.}


%{\renewcommand{\thefootnote}{\fnsymbol{footnote}} \footnotetext[1]
%{Работа выполнена при поддержке Министерства науки и~высшего образования Российской Федерации (проект 
%075-15-2020-799).}}


\renewcommand{\thefootnote}{\arabic{footnote}}
\footnotetext[1]{Федеральный исследовательский центр <<Информатика 
и~управ\-ле\-ние>> Российской академии наук, \mbox{nikitadragunovjob@gmail.com}}
\footnotetext[2]{Федеральный исследовательский центр <<Информатика и~управ\-ле\-ние>> 
Российской академии наук, \mbox{edjukova@mail.ru}}

\vspace*{-3pt}




\Abst{Исследованы актуальные вопросы снижения временных затрат, возникающие при 
логическом анализе данных с~элементами из декартова произведения конечных час\-тич\-но 
упорядоченных множеств. Для задачи поиска по базе транзакций максимальных час\-тых и~минимальных 
нечастых наборов произведения час\-тич\-ных порядков предложен оригинальный метод, 
основанный на решении слож\-ной дискретной задачи, называемой дуализацией 
над произведением час\-тич\-ных порядков. Метод представляет собой синтез двух других 
известных методов, один из которых достаточно очевиден, а~другой использует идею 
инкрементального пе\-ре\-чис\-ле\-ния искомых наборов и~поэтому пред\-став\-ля\-ет 
в~основном тео\-ре\-ти\-че\-ский интерес. Проведено экспериментальное исследование предложенного 
подхода к~решению рас\-смат\-ри\-ва\-емой задачи в~случае произведения конечных цепей,
 выявлены условия его эф\-фек\-тив\-ности и~для проводимого анализа данных показана 
 це\-ле\-со\-об\-раз\-ность применения асимптотически оптимальных алгоритмов дуализации 
 над произведением час\-тич\-ных порядков.}

\KW{максимальные час\-тые наборы; минимальные не\-час\-тые наборы; дуализация над 
произведением час\-тич\-ных порядков; асимп\-то\-ти\-чески оптимальный алгоритм дуализации}

\DOI{10.14357/19922264220112}
  
%\vspace*{-4pt}


\vskip 10pt plus 9pt minus 6pt

\thispagestyle{headings}

\begin{multicols}{2}

\label{st\stat}

    \section{Введение}
    
    Рас\-смат\-ри\-ва\-емая задача анализа данных занимает важ\-ное мес\-то в~об\-ласти 
    информационного поиска и~в~случае бинарных данных ставится сле\-ду\-ющим образом~\cite{4}.
    
    Дано некоторое множество элементов~$V$. Подмножества $X \hm\subseteq V$ называются наборами. Пусть~$D$~--- 
    база данных, содержащая некоторые, не обязательно различные, наборы. Наборы, 
    содержащиеся в~$D$, называются транз\-ак\-ци\-ями. Под частотой набора~$\nu(X)$ понимается доля транз\-ак\-ций в~$D$, 
    содержащих~$X$. Если $\nu(X) \hm\geq s$, $s \hm\in \left[0, 1\right]$, то набор~$X$ называется $s$-час\-тым, 
    иначе он называется $s$-не\-час\-тым. Если набор частый и~он не содержится ни в~каком другом 
    час\-том наборе, то такой набор называется максимальным час\-тым. Если набор не\-час\-тый 
    и~при этом он не содержит в~себе никакого другого не\-час\-то\-го набора, то такой набор 
    называется минимальным нечастым. Требуется найти все максимальные час\-тые и~минимальные не\-час\-тые 
    наборы при заданном~$s$.
    
    Рас\-смат\-ри\-ва\-емая задача имеет много важных приложений, одним из которых является 
    нахождение ассоциативных правил в~базах данных. В~случае бинарных данных ассоциативное правило~---
     это упорядоченная пара $ \left( X, Y \right)$ непересекающихся подмножеств множества~$V$, обо\-зна\-ча\-емая 
     $X \hm\Rightarrow Y$. Поддержкой правила $X \hm\Rightarrow Y$ называется час\-то\-та набора $Z\hm = X \cup Y$.
      Достоверностью правила $X\hm \Rightarrow Y$ называется доля транзакций, со\-дер\-жа\-щих~$Y$, 
      среди всех транзакций, содержащих~$X$. Требуется \mbox{найти} все ассоциативные правила, 
      удовле\-тво\-ря\-ющие заданным минимальной поддержке $s\hm \in [0, 1]$ и~минимальной 
      достоверности $c \hm\in [0, 1]$.  Впервые задача нахождения ассоциативных правил
       была поставлена в~\cite{1}, где она формулировалась как задача анализа по\-тре\-би\-тель\-ской корзины.

    В случае небинарных данных каждый элемент из~$V$ имеет некоторое множество чис\-ло\-вых значений 
    и~вместо наборов элементов рас\-смат\-ри\-ва\-ют\-ся наборы их значений.

    Поиск ассоциативных правил осуществляется в~два этапа. 
    На первом этапе находятся частые наборы, на втором этапе из найденных час\-тых 
    наборов формируются ассоциативные правила. При формировании правил на втором 
    этапе фактически возникает задача поиска $t$-не\-час\-тых наборов, где $t\hm > s/c$.
    
    С ростом размерности современных баз данных находить все час\-тые и~не\-час\-тые 
    наборы становится неэффективно как по времени, так и~по памяти в~силу 
    экспоненциального рос\-та чис\-ла таких наборов. Одно из решений данной проблемы 
    заключается в~поиске только максимальных час\-тых наборов и~только минимальных 
    нечастых наборов, что позволяет компактно хранить информацию о~всех час\-тых и~не\-час\-тых 
    наборах соответственно. 
    
    
    В~\cite{9} рас\-смот\-ре\-на задача поиска множеств максимальных час\-тых наборов~$X_{\max}$ 
    и~минимальных не\-час\-тых наборов~$Y_{\min}$ в~данных, пред\-став\-лен\-ных в~виде декартова 
    произведения час\-тич\-но упорядоченных множеств. Показано, что в~этом случае 
    при построении тре\-бу\-емых наборов возникают соответственно задача поиска 
    максимальных независимых элементов час\-тич\-ных порядков и~задача поиска минимальных 
    независимых элементов час\-тич\-ных порядков.  Каж\-дая из этих задач называется 
    дуализацией над произведением час\-тич\-ных порядков~\cite{8}. Обе задачи относятся к~одним 
    из цент\-раль\-ных труд\-но\-ре\-ша\-емых пе\-ре\-чис\-ли\-тель\-ных задач дис\-крет\-ной математики.
    
    Существует достаточно очевидный способ поиска максимальных час\-тых и~минимальных
     не\-час\-тых наборов произведения час\-тич\-ных порядков, основанный на по\-сле\-до\-ва\-тель\-ном 
     по\-стро\-ении указанных множеств. Одно из множеств ищется, например, алгоритмом Apriori~\cite{2},
      второе множество получается путем дуализации первого. 
      В~настоящей работе показано, что метод эффективен только в~случае, когда чис\-ло час\-тых 
      наборов существенно меньше или, наоборот, существенно больше чис\-ла не\-час\-тых наборов. 
      В~\cite{9} предложена идея со\-вмест\-но\-го пе\-ре\-чис\-ле\-ния~$X_{\max}$ и~$Y_{\min}$ с~использованием
       инкрементального алгоритма дуализации из~\cite{14}, которая автором экспериментально 
       не исследована.
    
    Основной результат настоящей работы~--- разработка нового подхода к~решению 
    поставленной задачи, который является синтезом последовательного и~совместного подходов. 
    
    Экспериментальные исследования, проведенные в~настоящей работе для случая
     произведения цепей, свидетельствуют о~том, что предложенный по\-сле\-до\-ва\-тель\-но-со\-вмест\-ный 
     метод наиболее эффективен в~случае, когда мощ\-ность множества час\-тых наборов примерно 
     равна мощ\-ности множества не\-час\-тых наборов.
     
     \vspace*{-6pt}
     
    
    \section{Постановка задачи поиска максимальных частых 
    и~минимальных нечастых наборов произведения частичных порядков}
    
         \vspace*{-2pt}
    
    Пусть $\mathcal{P} = \mathcal{P}_1 \times \dots \times \mathcal{P}_n$~--- 
    де\-кар\-то\-во произведение час\-тич\-но упорядоченных множеств. Элементы~$\mathcal{P}$ называются наборами. 
    На множестве~$\mathcal{P}$ определяется отношение частичного порядка~$\preceq$ сле\-ду\-ющим образом: 
    если $p \hm= (p_1, \dots, p_n) \hm\in \mathcal{P}$ и~$q \hm= (q_1, \dots, q_n)\hm \in \mathcal{P}$, 
    то $ p \hm\preceq q$ в~$ \mathcal{P}\hm \Leftrightarrow p_1 \hm\preceq q_1$ 
    в~$\mathcal{P}_1, \dots, p_n \hm\preceq q_n$ в~$ \mathcal{P}_n$.
    
    Пусть $\mathcal{D} (\mathcal{P})$~--- некоторая со\-во\-куп\-ность
     наборов из~$\mathcal{P}$, называемая базой данных. Наборы, на\-хо\-дя\-щи\-еся в~базе 
     данных $\mathcal{D} (\mathcal{P})$, необязательно по\-пар\-но раз\-лич\-ны и~называются транзакциями. 
     
    Введем обозначения: 
    $\vert \mathcal{D} (\mathcal{P}) \vert$~--- чис\-ло транз\-ак\-ций в~$\mathcal{D} (\mathcal{P})$; 
    $\mathcal{S}_\mathcal{D}(p)$~--- число транз\-ак\-ций в~$\mathcal{D} (\mathcal{P})$, 
    сле\-ду\-ющих за $p \hm\in \mathcal{P}$; $s \hm\in [0, 1]$. 
    
    \smallskip
    
    \noindent
    \textbf{Определение~1.}\
     Набор $p \in \mathcal{P}$ называется $s$-час\-тым, 
     если $\mathcal{S}_\mathcal{D}(p) / \vert \mathcal{D} (\mathcal{P}) \vert \hm\geq s$. Иначе набор~$p$ 
     называется $s$-не\-час\-тым.
    
    \smallskip
    
    \noindent
    \textbf{Определение~2.}\
    Набор $p \in \mathcal{P}$ называется максимальным $s$-час\-тым, если 
    он $s$-час\-тый и~никакой сле\-ду\-ющий за ним набор~$z$, $z\hm \neq p$, не является $s$-час\-тым.

    
    \smallskip
    
    \noindent
    \textbf{Определение~3.}\
    Набор $p \in \mathcal{P}$ называется минимальным $s$-не\-час\-тым, если он $s$-не\-час\-тый 
    и~никакой пред\-шест\-ву\-ющий ему набор~$z$, $z \hm\neq p$, не является $s$-не\-час\-тым.


\smallskip
    
    Далее вместо $s$-частый ($s$-не\-час\-тый) набор будем писать час\-тый (не\-час\-тый) набор. 
    Множество всех максимальных час\-тых наборов будем обозначать как $X_{\max}$, 
    а~множество всех минимальных не\-час\-тых наборов как $Y_{\min}$.
    
    Пусть $R \subset \mathcal{P}$, $R^+\hm = \{ x \in \mathcal{P} \vert \exists\, a \hm\in R, a \hm\preceq x \}$, 
    $R^- \hm= \{ x \hm\in \mathcal{P} \vert \exists\, a \hm\in R, x \hm\preceq a \}$.


    \noindent
    \textbf{Определение~4.}\
     Множество $I(R^+)$, со\-сто\-ящее из всех максимальных элементов множества~$\mathcal{P} \setminus R^+$, 
     называется максимальным независимым от~$R$.

\smallskip


   \noindent
    \textbf{Определение~5.}\
     Множество $I(R^-)$, со\-сто\-ящее из всех минимальных элементов множества~$\mathcal{P} \setminus R^-$, 
     называется минимальным независимым от~$R$.

\smallskip
    
    Каждая из задач построения $I(R^+)$ и~$I(R^-)$ 
    при заданном множестве~$R$ называется задачей дуализации над произведением час\-тич\-ных порядков.
    
    \smallskip

    \noindent
    \textbf{Утверждение~1.}\
    Если $X \hm\subset X_{\max}$, а~$y \hm\in I(X^-)$~--- не\-час\-тый набор, 
    то~$y$~--- минимальный не\-час\-тый набор.

\smallskip    
    
    \noindent
    Д\,о\,к\,а\,з\,а\,т\,е\,л\,ь\,с\,т\,в\,о\,.\  \ 
    Пусть $y \hm\notin I(X_{\max}^-)$. Так как~$y$~--- 
    нечастый набор, то в~$\mathcal{P} \setminus X^{-}_{\max}$ найдется минимальный не\-час\-тый набор~$x$ 
    такой, что $x\hm \neq y$ и~$x \hm\preceq y$. Из того, что $\mathcal{P} \setminus X^{-}_{\max} 
    \hm\subseteq \mathcal{P} \setminus X^-$, следует, что $x\hm \in \mathcal{P} \setminus X^-$, 
    что противоречит условию $y \hm\in I(X^-)$.

\smallskip

\noindent
\textbf{Утверждение~2.}\
    Пусть $X \hm\subseteq X_{\max}$, $Y\hm \subseteq Y_{\min}$. 
    Тогда $I(X^-) \hm= Y$ в~том и~только в~том случае, когда $X \hm= X_{\max}$ и~$Y \hm= Y_{\min}$.


\smallskip


  \noindent
    Д\,о\,к\,а\,з\,а\,т\,е\,л\,ь\,с\,т\,в\,о\,.\  \
    Пусть $X\! \subset\! X_{\max}, x \hm\in X_{\max}\!\setminus\!X$.
     Так как множество~$X_{\max}$~--- антицепь, то $x \hm\notin X^-$. 
     Следовательно, $x \hm\in \mathcal{P} \setminus X^{-}$.
      Но тогда существует элемент $ q \hm\in I(X^-) : q \preceq x$, 
      который является час\-тым. Однако во множестве~$Y$ частых наборов нет; следовательно, $I(X^-) \hm\neq Y$. 
      Если же $X \hm= X_{\max}$, то $I(X^-) \hm= Y_{\min}$. Таким образом, $I(X^-) \hm= Y$ тогда и~только
       тогда, когда $X \hm= X_{\max}$ и~$Y\hm = Y_{\min}$.


    
    \section{Методы построения множеств~$X_{\max}$ и~$Y_{\min}$}

    \subsection{Последовательное перечисление $X_{\max}$~и~$Y_{\min}$}

    Достаточно очевиден поиск~$X_{\max}$ и~$Y_{\min}$ при заданной $\mathcal{D} (\mathcal{P})$ 
    путем последовательного по\-стро\-ения множеств~$X_{\max}$ и~$Y_{\min}$. 
    Данный поиск осуществляется в~два этапа. На первом этапе находятся все максимальные частые 
    наборы~$X_{\max}$, например алгоритмом Apriori~\cite{2}. На втором этапе  используется свойство 
    двойственности $I \left(X_{\max}^- \right)\hm = Y_{\min}$. 
    Минимальные нечастые наборы~$Y_{\min}$ находятся путем дуализации найденного на первом этапе 
    множества~$X_{\max}$. Аналогично можно сначала искать~$Y_{\min}$ алгоритмом Apriori, а~затем 
    искать~$X_{\max}$ путем дуализации~$Y_{\min}$.

    Очевидно, что данный подход будет проявлять себя наилучшим образом в~случаях, когда 
    алгоритм Apriori или его модификации могут найти одно из искомых множеств существенно
     быст\-рее, чем другое множество, например когда мощ\-ность~$X_{\max}$ 
     существенно меньше (больше) мощ\-ности~$Y_{\min}$.
    
    \subsection{Совместное перечисление $X_{\max}$ и~$Y_{\min}$}

    В~\cite{9} предложена идея совместного перечисления множеств~$X_{\max}$ и~$Y_{\min}$. 
    На первом шаге рас\-смат\-ри\-ва\-ет\-ся некоторый случайный набор $q \hm\in \mathcal{P}$. Если $q$~--- 
    час\-тый набор, то ищется максимальный час\-тый набор, сле\-ду\-ющий за~$q$, 
    который пополняет множество $X \hm\subseteq X_{\max}$. Если $q$~---
     не\-час\-тый набор, то ищется минимальный не\-час\-тый набор, пред\-шест\-ву\-ющий~$q$, 
     который пополняет множество $Y \hm\subseteq Y_{\min}$. Пусть на шаге~$i$ ($i\hm \geq 1$) 
     построены множества $X \hm\subseteq X_{\max}$ и~$Y \hm\subseteq Y_{\min}$. Если $X \hm\neq \varnothing$, 
     $Y \hm= \varnothing$, то ищется набор~$q$ такой, что $q \hm\npreceq x, \forall x \hm\in X$. Если 
     $X \hm= \varnothing$, $Y \hm\neq \varnothing$, то ищется набор~$q$ такой, что 
     $q \hm\nsucceq y, \forall y \hm\in Y$. Если же и~$X \hm\neq \varnothing$, и~$Y \hm\neq \varnothing$, 
     то ищется набор~$q$ такой, что $q \hm\npreceq x, \forall x \hm\in X, q \hm\nsucceq y, \forall y \hm\in Y$.
      Затем, аналогично первому шагу, находится максимальный частый или минимальный нечастый набор. 
      Однако в~\cite{9} идея совместного перечисления искомых множеств экспериментально 
      не исследована и~не предложены конкретные указания по воз\-мож\-ной ее реализации.
    
    Алгоритм, основанный на совместном пе\-ре\-чис\-ле\-нии множеств~$X_{\max}$ и~$Y_{\min}$,
     реализован в~на\-сто\-ящей работе. Алгоритм строит две последовательности: $X_1 \hm\subset X_2 
     \subset \dots \subset X_{\max}$, $Y_1\hm \subset Y_2 \subset \dots \subset Y_{\min}$. 
     На первом шаге $X_1 \hm= \{x\}$, $Y_1 \hm= \{y\}$, где~$x$ и~$y$ ищутся алгоритмом Apriori.
      На шаге $i \hm+ 1$ ($i\hm \geq 1$) строится либо~$I(X^{-}_{i})$, либо~$I(Y^{+}_{i})$. Пусть на 
      шаге $i \hm+ 1$ ($i \hm\geq 1$) построено множество~$I(X^{-}_{i})$. 
      Согласно утверждениям~1 и~2, множество~$I(X^{-}_{i})$ либо не содержит час\-тых наборов 
      и~совпадает с~множеством~$Y_{\min}$ (в~этом случае $X_i \hm= X_{\max}$ 
      и~алгоритм заканчивает работу), либо~$I(X^{-}_{i})$ содержит как час\-тые, так и~не\-час\-тые наборы. 
      Каждый нечастый набор из~$I(X^{-}_{i})$ является минимальным не\-час\-тым и~пополняет множество~$Y_{i}$, 
      формируя в~результате множество~$Y_{i+1}$. Для каждого час\-то\-го набора находится один содержащий 
      его максимальный час\-тый набор путем последовательного увеличения текущего 
      частого набора в~лексикографическом порядке, который пополняет множество~$X_{i}$, 
      формируя в~результате множество~$X_{i+1}$.
      
    В~экспериментальной части работы (см.\ разд.~4) рас\-смот\-рен случай произведения цепей. 
    Задача дуализации решается с~помощью асимптотически оптимального алгоритма дуализации
     цепей \mbox{RUNC-M}+~\cite{7}. Асимптотически оптимальные алгоритмы дуализации 
     являются лидерами по ско\-рости счета~\cite{6}.

    Очевидно, что время работы совместного алгоритма в~основном зависит от чис\-ла
     минимальных не\-час\-тых и~максимальных час\-тых наборов. На\linebreak каж\-дой новой 
     итерации происходит дуализация\linebreak все б$\acute{\mbox{о}}$льших по мощ\-ности множеств~$X$ или~$Y$.\linebreak 
     Если число итераций становится достаточно\linebreak большим, то ско\-рость работы совместного 
     перечисления существенно снижается, что делает его практически неприменимым для 
     задач большой раз\-мер\-ности.
     { %\looseness=1
     
     }

    \subsection{Последовательно-совместное перечисление~$X_{\max}$ и~$Y_{\min}$}

    Предлагается следующий итеративный метод, который синтезирует идеи последовательного
     и~совместного методов, описанных выше. Положим $X_0 \hm= \varnothing$. 
     Строится одна по\-сле\-до\-ва\-тель\-ность $X_1 \hm\subset X_2 \hm\subset \dots \subset X_{\max}$. 
     На первом шаге $X_1\hm = \{x\}$, где $x$ ищется алгоритмом Apriori. На шаге $i \hm+ 1$ ($i \hm\geq 1$) 
     решается задача дуализации множества $X_{i} \setminus X_{i-1}$.

    
    
   \setcounter{figure}{1}
    \begin{figure*}[b] %fig2
  \vspace*{12pt}
  \begin{center}  
    \mbox{%
\epsfxsize=163mm
\epsfbox{duk-2.eps}
}

\end{center}
\vspace*{-9pt}
    \Caption{Зависимость времени работы алгоритмов от суммы мощностей множеств~$X_{\max}$ и~$Y_{\min}$ 
    для случая~1~(\textit{а}) и~2~(\textit{б}):
    \textit{1}~--- по\-сле\-до\-ва\-тель\-но-со\-вмест\-ный;
    \textit{2}~--- последовательный; \textit{3}~--- совместный; \textit{4}~--- Apriori}
    \label{12}
    \end{figure*}
     
    Пусть множество~$D$ есть результат дуализации $X_{i} \hm\setminus X_{i-1}$. Согласно утверждению~1, 
    множество~$D$ содержит частые наборы. Для каждого час\-то\-го набора из~$D$ 
    находится один содержащий его максимальный час\-тый набор путем последовательного 
    увеличения текущего час\-то\-го набора в~лексикографическом порядке. Все найденные максимальные
     частые наборы, которых нет в~множестве~$X_{i}$, до\-бав\-ля\-ют\-ся к~$X_{i}$, 
     и~таким образом формируется~$X_{i+1}$. Если же все найденные частые наборы уже содержатся в~$X_{i}$, 
     то решается задача дуализации множества~$X_{i}$. Если в~$I(X^{-}_{i})$ нет частых наборов, 
     то $I(X^{-}_{i})\hm = Y_{\min}$, $X_i \hm= X_{\max}$ и~алгоритм завершает работу. 
     Иначе для каждого частого набора из~$I(X^{-}_{i})$ находится один содержащий его максимальный 
     час\-тый набор, который пополняет множество~$X_{i}$, формируя в~результате множество~$X_{i+1}$.

    \section{Экспериментальное исследование}
    
    Рас\-смат\-ри\-вал\-ся случай данных, пред\-став\-лен\-ных в~виде произведения цепей мощ\-ности~5. 
    Для\linebreak таких данных проводился поиск максимальных час\-тых и~минимальных нечастых 
    наборов сле\-ду\-ющи\-ми методами: алгоритмом Apriori, модифицированным для случая 
    цепей; последовательным \mbox{методом}; совместным методом; по\-сле\-до\-ва\-тель\-но-со\-вмест\-ным методом.
    
    Все методы реализованы на языке Python~3. 
    Задача дуализации решалась алгоритмом дуализации цепей RUNC-M+~\cite{7}. 
    Эксперименты проведены на случайных базах данных различной раз\-мер\-ности. 
    Можно выделить два сле\-ду\-ющих случая соотношения мощностей множеств всех час\-тых и~не\-час\-тых наборов.
    \begin{description}
    \item[Случай 1:] мощ\-ность множества частых наборов примерно рав\-на мощ\-ности множества нечастых наборов.
    \item[Случай 2:] мощ\-ность множества частых наборов существенно меньше (больше) мощ\-ности множества 
    не\-час\-тых наборов.
    \end{description}
    
    Описанные случаи схематично изображены на рис.~1. 

    Графики зависимости времени работы тестируемых методов 
    от мощ\-ности множеств~$X_{\max}$ и~$Y_{\min}$ приведены на рис.~2.
    
    

    

    Нетрудно видеть, что в~случае~1 лучше работает по\-сле\-до\-ва\-тель\-но-со\-вмест\-ный алгоритм: 
    множества час\-тых и~не\-час\-тых наборов имеют примерно одинаковую мощ\-ность, 
    поэтому быст\-рее будет обрабатывать их по\-сле\-до\-ва\-тель\-но-со\-вмест\-ным методом. В~случае~2 
    быст\-рее работает последовательный алгоритм: быст\-рее найти множество максимальных час\-тых наборов, 
    обработав множество час\-тых наборов, и~дуализировать результат. Время поиска множеств~$X_{\max}$ 
    и~$Y_{\min}$ совместным методом и~модифицированным алгоритмом Apriori рас\-тет существенно 
    быст\-рее времени поиска по\-сле\-до\-ва\-тель\-но-со\-вмест\-ным методом в~обоих случаях.
    
    { \begin{center}  %fig1
 \vspace*{9pt}
    \mbox{%
\epsfxsize=67.963mm
\epsfbox{duk-1.eps}
}

\end{center}

\noindent
{{\figurename~1}\ \ \small{
Два случая соотношения мощностей множеств час\-тых и~не\-час\-тых наборов
}}}

%\vspace*{6pt}


    \section{Заключение}
    
Рас\-смот\-ре\-на задача поиска максимальных час\-тых и~минимальных не\-час\-тых наборов в~данных, 
представленных в~виде декартова произведения час\-тич\-ных порядков. Актуальны вопросы 
снижения временн$\acute{\mbox{ы}}$х затрат, возникающих при реализации методов нахождения искомых наборов.
 Разработан новый подход к~по\-стро\-ению множества максимальных частых наборов~$X_{\max}$ и~множества 
 минимальных не\-час\-тых наборов~$Y_{\min}$, пред\-став\-ля\-ющий собой синтез двух ранее известных 
 подходов: последовательного и~со\-вмест\-но\-го (первый достаточно очевиден, идея второго предложена в~\cite{9}). 
 Сложность последовательного, совместного и~пред\-ла\-га\-емо\-го по\-сле\-до\-ва\-тель\-но-со\-вмест\-но\-го поиска 
 обуслов\-ле\-на, в~том чис\-ле, не\-об\-хо\-ди\-мостью рас\-смат\-ри\-вать в~процессе поиска 
 труд\-но\-ре\-ша\-емую пе\-ре\-чис\-ли\-тель\-ную задачу дис\-крет\-ной математики, на\-зы\-ва\-емую дуализацией 
 над произведением час\-тич\-ных порядков.

Для случая, когда данные пред\-став\-ле\-ны в~виде произведения конечных цепей, 
приведены результаты экспериментального срав\-не\-ния названных подходов, а~так\-же независимого 
способа \mbox{по\-стро\-ения} множеств~$X_{\max}$ и~$Y_{\min}$, не тре\-бу\-юще\-го решения задачи дуализации. 
Эксперименты проводились на модельных задачах с~применением асимптотически оптимального
 алгоритма дуализации над произведением конечных цепей \mbox{RUNC-M}+~\cite{7}. 
 Результаты исследования свидетельствуют о~том, что по\-сле\-до\-ва\-тель\-но-со\-вмест\-ный 
 метод наиболее эффективен (требует меньших временн$\acute{\mbox{ы}}$х затрат по сравнению с~другими рас\-смот\-рен\-ны\-ми 
 методами) в~случае, когда мощ\-ность множества час\-тых наборов примерно равна мощ\-ности множества
  нечастых наборов. Иначе выигрывает последовательный поиск. Наихудшие показатели 
  у~независимого пе\-ре\-чис\-ле\-ния множеств~$X_{\max}$ и~$Y_{\min}$ с~использованием в~качестве
   базового алгоритма Apriori~\cite{2}, точ\-нее его модификации на тес\-ти\-ру\-емый случай. 
   Таким образом, показана це\-ле\-со\-об\-раз\-ность применения алгоритмов дуализации для 
   по\-стро\-ения множеств~$X_{\max}$ и~$Y_{\min}$.

  
  {\small\frenchspacing
 {%\baselineskip=10.8pt
 %\addcontentsline{toc}{section}{References}
 \begin{thebibliography}{9}  
    \bibitem{4}
    \Au{Aggarwal C.} 
    Frequent pattern mining.~--- Heidelberg: Springer, 2014. 467~p.
    
    \bibitem{1}
    \Au{Agrawal~R., Imielinski~T., Swami~A.} Mining association rules 
    between sets of items in large databases~// \mbox{SIGMOD} Conference (International) on Management of Data
    Proceedings.~--- New York, NY, USA: ACM, 1993. P.~207--216.
    
    \bibitem{9}
    \Au{Elbassioni K.} On finding minimal infrequent elements in multi-dimensional 
    data defined over partially ordered sets~// arXiv.org, 2014. 30~p. arXiv:1411.2275 [cs.DB].
    
    \bibitem{8}
    \Au{Elbassioni K.} Algorithms for dualization over products of partially 
    ordered sets~// SIAM J.~Discrete Math., 2009. Vol.~23. Iss.~1. P.~487--510.
    
    \bibitem{2}
    \Au{Agrawal R., Srikant~R.} 
    Fast algorithms for mining association rules in large databases~// 
    20th Conference (International) on Very Large Data Bases Proceedings.~--- San Francisco, CA, USA: 
    Morgan Kaufmann Publs. Inc., 1994. P.~487--499.
    
    \bibitem{14}
    \Au{Хачиян Л.\,Г.} Избранные труды.~--- М.: МЦНМО, 2009. 520~с.
    
    \bibitem{7}
    \Au{Дюкова Е.\,В., Масляков~Г.\,О., Прокофьев~П.\,А.} 
    О~дуализации над произведением частичных порядков~// Машинное обучение и~анализ данных, 2017. Т.~3. №\,4.  
    C.~239--249.
    
    \bibitem{6}
    \Au{Дюкова Е.\,В., Прокофьев~П.\,А.} Об асимптотически оптимальных алгоритмах дуализации~// 
    Ж.~вычисл. матем. и~матем. физ., 2015. Т.~55. №\,5. С.~895--910.
    \end{thebibliography}

 }
 }

\end{multicols}

\vspace*{-6pt}

\hfill{\small\textit{Поступила в~редакцию 15.01.21}}

\vspace*{8pt}

%\pagebreak

%\newpage

%\vspace*{-28pt}

\hrule

\vspace*{2pt}

\hrule

%\vspace*{-2pt}

\def\tit{FINDING MAXIMAL FREQUENT AND~MINIMAL INFREQUENT SETS IN~PARTIALLY ORDERED DATA}


\def\titkol{Finding maximal frequent and~minimal infrequent sets in~partially ordered data}


\def\aut{N.\,A.~Dragunov and E.\,V.~Djukova}

\def\autkol{N.\,A.~Dragunov and E.\,V.~Djukova}

\titel{\tit}{\aut}{\autkol}{\titkol}

\vspace*{-11pt}


\noindent
Federal Research Center ``Computer Science and Control'' 
of the Russian Academy of Sciences, 44-2~Vavilov Str., Moscow 119333, Russian Federation

\def\leftfootline{\small{\textbf{\thepage}
\hfill INFORMATIKA I EE PRIMENENIYA~--- INFORMATICS AND
APPLICATIONS\ \ \ 2022\ \ \ volume~16\ \ \ issue\ 1}
}%
 \def\rightfootline{\small{INFORMATIKA I EE PRIMENENIYA~---
INFORMATICS AND APPLICATIONS\ \ \ 2022\ \ \ volume~16\ \ \ issue\ 1
\hfill \textbf{\thepage}}}

\vspace*{3pt} 


\Abste{Relevant issues of time costs reducing in the logical analysis of data with elements 
from the Cartesian product of finite partially ordered sets are investigated. 
An original method based on solving a complex discrete problem called dualization
 over the product of partial orders is proposed for the problem of finding maximal 
 frequent and minimal infrequent sets in the transaction database. The proposed method 
 is a~synthesis of two other known methods, one of which is quite obvious and the other uses 
 the idea of an incremental enumeration of target\linebreak\vspace*{-12pt}}
 
 \Abstend{sets and is, therefore, mainly 
 of theoretical interest. An experimental study of the considered approaches in
  the case of the product of finite chains is carried out and conditions for
   their effectiveness are revealed. The expediency of applying 
asymptotically optimal dualization algorithms over the product of partial orders is shown.}

\KWE{maximal frequent sets; minimal infrequent sets; dualization over the product of 
partial orders; asymptotically optimal dualization algorithm}

\DOI{10.14357/19922264220112}

%\vspace*{-16pt}

%\Ack
%\noindent




%\vspace*{6pt}

  \begin{multicols}{2}

\renewcommand{\bibname}{\protect\rmfamily References}
%\renewcommand{\bibname}{\large\protect\rm References}

{\small\frenchspacing
 {%\baselineskip=10.8pt
 \addcontentsline{toc}{section}{References}
 \begin{thebibliography}{9}
\bibitem{1-dr}
\Aue{Aggarwal, C.} 2014. \textit{Frequent pattern mining}. Heidelberg: Springer. 467~p.
\bibitem{2-dr}
\Aue{Agrawal, R., T.~Imielinski, and A.~Swami.}
 1993. Mining association rules between sets of items in large databases. 
 \textit{SIGMOD  Conference (International) on Management of Data Proceedings}. New York, NY:
 ACM. 207--216. 
\bibitem{3-dr}
\Aue{Elbassioni, K.}
 2014. On finding minimal infrequent elements in multidimensional data defined over partially ordered sets. 
 arXiv.org. 30~p. Available at: 
 {\sf https://arxiv.org/\linebreak pdf/1411.2275.pdf} (accessed January~25, 2022).
\bibitem{4-dr}
\Aue{Elbassioni, K.} 2009. Algorithms for dualization over products of partially ordered sets. 
\textit{SIAM J.~Discrete Math.} 23(1):487--510.
\bibitem{5-dr}
\Aue{Agrawal, R., and R.~Srikant.}
 1994. Fast algorithms for mining association rules in large databases. 
 \textit{20th Conference (International) on Very Large Data Bases Proceedings}.
 San Francisco, CA: 
    Morgan Kaufmann Publs. Inc.  487--499.
\bibitem{6-dr}
\Aue{Khachiyan, L.\,G.} 2009. \textit{Izbrannye trudy} [Selected works]. Moscow: MCCME. 520~p.
\bibitem{7-dr}
\Aue{Djukova, E.\,V., G.\,O.~Maslyakov, and P.\,A.~Prokofyev.} 
2017. O~dualizatsii nad proizvedeniem chastichnykh poryadkov [On dualization over the product of 
partial orders]. \textit{Mashinnoe obuchenie i~analiz dannykh} [J.~Machine Learning Data Analysis] 
3(4):239--249.
\bibitem{8-dr}
\Aue{Djukova, E.\,V., and P.\,A.~Prokofyev.}
 2015. Asymptotically optimal dualization algorithms. \textit{Comp. Math.
 Math. Phys.} 55(5):891--905. 
 
 \end{thebibliography}

 }
 }

\end{multicols}

\vspace*{-6pt}

\hfill{\small\textit{Received January 15, 2021}}

%\pagebreak

%\vspace*{-18pt}

\Contr

\noindent
\textbf{Dragunov Nikita A.} (b.\ 1997)~--- 
PhD student, Federal Research Center ``Computer Science and Control'' 
of the Russian Academy of Sciences, 44-2~Vavilov Str., Moscow 119333, Russian Federation; 
\mbox{nikitadragunovjob@gmail.com}

\vspace*{3pt}

\noindent
\textbf{Djukova Elena V.} (b.\ 1945)~--- 
Doctor of Science in physics and mathematics, principal scientist, Federal Research Center
``Computer Science and Control'' of the Russian Academy of Sciences, 44-2~Vavilov Str., Moscow 119333, 
Russian Federation; \mbox{edjukova@mail.ru}




\label{end\stat}

\renewcommand{\bibname}{\protect\rm Литература}   %4
\def\stat{malashenko}

\def\tit{ПОСЛЕДОВАТЕЛЬНЫЙ АНАЛИЗ И~МЕТРИЧЕСКИЕ ОЦЕНКИ ПРЕДЕЛЬНЫХ
РАСПРЕДЕЛЕНИЙ МЕЖУЗЛОВЫХ ПОТОКОВ В~МНОГОПОЛЬЗОВАТЕЛЬСКОЙ СЕТИ}

\def\titkol{Последовательный анализ и~метрические оценки предельных
распределений межузловых потоков в %~многопользовательской 
сети}

\def\aut{Ю.\,Е. Малашенко$^1$}

\def\autkol{Ю.\,Е. Малашенко}

\titel{\tit}{\aut}{\autkol}{\titkol}

\index{Малашенко Ю.\,Е.}
\index{Malashenko Yu.\,E.}


%{\renewcommand{\thefootnote}{\fnsymbol{footnote}} \footnotetext[1]
%{Исследование выполнено при финансовой поддержке Российского научного фонда (проект 
%<<Информатика>> ФИЦ ИУ РАН, Москва).}}


\renewcommand{\thefootnote}{\arabic{footnote}}
\footnotetext[1]{Федеральный исследовательский центр <<Информатика и~управление>> Российской академии 
\mbox{mala-yur@yandex.ru}}


%\vspace*{-6pt}



\Abst{Для оценки функциональных возможностей
многопользовательской сети связи аналилизируется множество векторов межузловых потоков при предельных распределениях ресурсов
сети. В~рамках многопродуктовой модели про\-пуск\-ные спо\-соб\-ности ребер рас\-смат\-ри\-ва\-ют\-ся 
как компоненты вектора ресурсов различных
типов, которые требуются для передачи потоков различных видов.
При проведении вычислительных экспериментов на каждой итерации вычисляются нормы векторов совместно допустимых межузловых
потоков, при передаче которых полностью используется пропускная спо\-соб\-ность всех ребер сети. Полученные последовательности
метрических оценок позволяют анализировать особенности множества до\-сти\-жи\-мости и~эф\-фек\-тив\-ность использования ресурсов сети при
уравнительном распределении про\-пуск\-ной спо\-соб\-ности между корреспондентами.}

\KW{многопродуктовая потоковая сетевая
модель; множество достижимых межузловых потоков; предельные
распределения пропускной способности}

\DOI{10.14357/19922264220306} 
  
%\vspace*{-3pt}


\vskip 10pt plus 9pt minus 6pt

\thispagestyle{headings}

\begin{multicols}{2}

\label{st\stat}

\section{Введение}

Данная работа продолжает исследования функциональных характеристик
сетевых сис\-тем связи~[1]. В~настоящее время математические модели
передачи многопродуктового потока применяются для поиска
распределений потоков и~ресурсов в~многопользовательских
телекоммуникационных\linebreak сетях~[2--4]. Разрабатываются методы анализа
с~учетом вектора требований всех \textit{равноправных} 
и~невзаимозаменяемых корреспондентов~[5]. С~позиций\linebreak методологии
исследования операций изучаются справедливые распределения потоков
и~ресурсов~[6].

Соответствующие \textit{недискриминирующие} правила управления
потоками являются решениями задач на максмин и/или получаются 
в~результате использования процедур последовательной
лексикографически упорядоченной оптимизации~[7].

В~настоящей работе пути соединения корреспондентов прокладываются
через со\-от\-вет\-ст\-ву\-ющие минимальные разрезы. Указанный метод\linebreak \mbox{можно}
рассматривать как возможный вариант решения задачи о~построении
SPLIT-марш\-ру\-тов~[8,~9]. В~рамках вычислительных экспериментов\linebreak на
многопродуктовой модели анализируются распределения межузловых
потоков  и~пропускной способ\-ности сети.  Для оценки функциональных
возможностей многопользовательской сети используется вектор
совместно допустимых межузловых потоков. Под ресурсом, выделяемым
некоторой паре узлов-кор\-рес\-пон\-ден\-тов,  понимается суммарное
значение тре\-бу\-емых пропускных способностей на всех ребрах,
расположенных на всех маршрутах при прохождении межузлового\linebreak потока
данного вида.  Сумма соответствующих реберных потоков трактуется
как полная нагрузка на сеть, возникающая при передаче заданного
межузлового потока. Рас\-смат\-ри\-ва\-ют\-ся распределения пропускной
способности и~межузловых потоков при предельной загрузке сети.
При проведении вычислительных экспериментов на каждой  итерации
вычисляется норма  вектора совместно допустимых межузловых
потоков.   Для оценки величины требуемых ресурсов при соединении
корреспондентов по различным путям для каж\-дой пары узлов
определяется максимальный однопродуктовый поток. Марш\-ру\-ты передачи
всех допустимых межузловых потоков  проходят по ребрам
соответствующих минимальных разрезов. Вычислительные эксперименты
проводились  для получения последовательности  мет\-ри\-че\-ских оценок
векторов межузловых потоков, принадлежащих множеству до\-сти\-жи\-мости
многопользовательской сети.

\section{Математическая модель}

В качестве математической модели для описания
многопользовательской сетевой системы используется следующая
формальная запись условий и~ограничений, которые должны
выполняться при одновременной передаче потоков различных видов
между всеми парами улов-корреспондентов:

Сеть $G(\mathbf{d})$ задается множествами $\langle V,
R,U,P\rangle$:
\begin{itemize}
\item  узлов (вершин) сети 
$$
V=\left \{v_{1}, v_{2},\dots,v_{n},\dots,v_{N}\right\};
$$
\item  неориентированных ребер 
$$
R=\left\{r_{1}, r_{2}, \dots, r_{k}, \dots,
r_{E}\right\}.
$$
\end{itemize}

Ребро $r_{k}$ \textit{соединяет} концевые вершины~$v_{n_k}$ и~$v_{j_k}$. 
Реб\-ру~$r_{k}$ ставятся в~соответствие две
ориентированные дуги $\{u_{k},u_{k+E}\}$ из множества
ориентированных дуг $U\hm=\{u_{1}, u_{2}, \dots, u_{k}, \dots,
u_{2E}\}$. Дуги $\{u_{k}, u_{k+E}\}$ определяют прямое и~обратное
на\-прав\-ле\-ние передачи потока по реб\-ру~$r_{k}$ между концевыми
вершинами $\{v_{n_k}, v_{j_k}\}$.

В многопользовательской сети~$G(\mathbf{d})$ рассматривается
$M\hm=N(N\hm-1)$ независимых, невзаимозаменяемых и~равноправных потоков
различных видов, которые передаются между уз\-ла\-ми-кор\-рес\-пон\-ден\-та\-ми
из множества 
$$
P=\left\{p_{1}, p_{2}, \dots, p_{M}\right\}.
$$

По определению, каждой паре уз\-лов-кор\-рес\-пон\-ден\-тов~$p_{m}$
соответствуют:
\begin{itemize}
\item вершина-ис\-точ\-ник с~номером~$s_{m}$, через которую входной поток
$m$-го вида~$z_{m}$ поступает в~сеть;
\item  вершина-при\-ем\-ник с~номером~$t_{m}$, из которой поток $m$-го
вида~$z_{m}$ покидает сеть.
\end{itemize}

В множестве~$P$ выделяется подмножество $P(R^{+})$ пар
уз\-лов-кор\-рес\-пон\-ден\-тов, расположенных в~концевых вершинах
ребра~$r_{k}$, $k\hm=\overline{1,E}$. Вводятся сле\-ду\-ющие обозначения:
пусть ребро~$r_{k}$  с~номером~$k$ соединяет вершины с~номерами~$n$ и~$j$ такими, что $n\hm< j$. Для со\-от\-вет\-ст\-ву\-ющей пары
уз\-лов-кор\-рес\-пон\-ден\-тов~$p_{k}$, расположенных в~узлах $\{v_{n},
v_{j}\}$, узел~$v_{n}$ считается источником, а узел~$v_{j}$~---
приемником потока $z_{k}$ $k$-го вида, который передается из узла
c номером~$n$ в~узел с~номером~$j$ для пары~$p_{k}$ с~номером~$k$.
Для пары $p^{}_{k+E} \Longleftrightarrow \{v_{j},v_{n}\}$ узел~$v_{j}$ 
считается источником~$s_{k+E}$, а~узел $v_m$~--- приемником~$t_{k+E}$ для пары с~номером~$p_{k+E}$. Формируется
$R^+\hm=\{1,2,3,\dots,E,E+1,\dots,2E\}$~--- список номеров смежных
пар.

Пары $p_{k}$ из подмножества~$P(R^{+})$ называются
\textit{смежными} уз\-ла\-ми-кор\-рес\-пон\-ден\-та\-ми. Все остальные
\textit{несмежные} пары уз\-лов-кор\-рес\-пон\-ден\-тов относятся к~множеству~$P(R^{-})$:
\begin{equation*}
P=P(R^{+})\cup P(R^{-});\quad
P(R^{+}) \cap P(R^{-}) = \emptyset.
\end{equation*}

Введем обозначения:
\begin{description}
\item[\,]
$z_{m}$~--- величина \textit{межузлового} потока $m$-го вида,
который поступает в~сеть из узла с~номером~$s_{m }$ и~покидает из
узла с~номером~$t_{m}$;
\item[\,]
$S(v_{n})$~--- множество номеров исходящих дуг, по которым поток
покидает узел~$v_{n}$;
\item[\,]
$T(v_{n})$~--- множество номеров входящих дуг, по которым поток
поступает в~узел~$v_{n}$.
\end{description}

Во всех узлах $v_{n}\in V$, $n\hm=\overline{1,N}$, для всех видов
потоков должны выполняться условия сохранения потоков:
\begin{multline}
\label{eq1} 
\sum\limits_{i\in S(v_n )} x_{mi}-\sum\limits_{i\in T(v_n )} x_{mi}
={}\\
{}=\begin{cases}
z_m, & \mbox{если } v=v^{}_{S_m}; \\
-z_m,&\mbox{если } v=v_{t_m}; \\
0&\mbox{в остальных случаях}, \\
\end{cases}
\end{multline}
$n=\overline{1,N}$, $m\hm=\overline{1,M}$, $x_{mi}\hm\ge 0$,
$z_{m}\hm\ge0$.

Величина {z}$_{m}$ равна входному потоку $m$-го вида, который
пропускается от источника к~приемнику пары $p_{m}$ при
распределении потоков $x_{mi}$ по дугам сети.

Каждому ребру $r_{k}\hm\in R$ приписывается неотрицательное число~$d_{k}$, 
определяющее суммарный предельно допустимый поток,
который можно передать по реб\-ру~$r_{k}$ в~обоих на\-прав\-ле\-ни\-ях. 
В~исходной сети компоненты вектора про\-пуск\-ных способностей
$\mathbf{d}\hm=(d_{1}, d_{2},\dots, d_{k}, \dots, d_{E})$~--- наперед
заданные положительные числа $d_{k}
\hm> 0$. Вектором $\mathbf{d}$ определяются сле\-ду\-ющие ограничения на сумму
дуговых потоков всех видов, пе\-ре\-да\-ва\-емых по реб\-ру~$r_{k}$:
\begin{multline}
\sum\limits_{m=1}^M (x_{mk}+x_{m(k+E)}) \le d_k,\\
 x_{mk}\ge 0\,,\enskip
 x_{m(k+E)}\ge 0\,, \enskip k=\overline {1,E}\,.
 \label{eq2} 
\end{multline}
В рамках данной модели пропускная спо\-соб\-ность ребер сети~--- вектор~$\mathbf{d}$~--- трактуется как <<\textit{ресурсное ограничение}>>,
а~сумма дуговых
 потоков рас\-смат\-ри\-ва\-ет\-ся как показатель использования
<<\textit{ресурсов}>> сети при передаче межузловых потоков
различных видов.

Для всех $z_{m}$ и~$x_{mi}$, удовлетворяющих
условиям~\eqref{eq1} и~\eqref{eq2}, вычисляются суммарные потоки:
\begin{equation}
 y_{m }=\sum\limits_{i=1}^{2E} {x}_{mi},\quad
m=\overline{1,M}\,.
\label{eq3}
\end{equation}

Суммарный реберный поток~$y_{m}$ характеризует
<<\textit{нагрузку}>> на сеть при передаче межузлового потока
величины $z_{m}$ из уз\-ла-ис\-точ\-ни\-ка~$s_{m}$ в~узел-при\-ем\-ник~$t_{m}$. 
Величина~$y_{m}$ показывает, какой суммарный
\textit{ресурс}~-- пропускная спо\-соб\-ность сети~-- требуется для
передачи межузлового потока~$z_{m}$, а~отношение
$w_{m}\hm={y_m}/{z_m}$,  $m\hm=\overline{1,M},$
показывает, какие \textit{ресурсы} необходимы для передачи
единичного потока $m$-го вида между узлами~$s_{m}$ и~$t_{m}$.

Ограничения~\eqref{eq1}--\eqref{eq3} задают подмножество
допустимых значений компонент вектора межузловых потоков
$\mathbf{z}\hm=\left(z_{1}, z_{2},\dots,z_{m},\dots,z_{M}\right)$:
\begin{equation*}
{Z}(\mathbf{d})=\left\{\mathbf{z} \ge 0 \mid
(\mathbf{z},\mathbf{x},\mathbf{y}) \ \mbox{удовлетворяют~\eqref{eq1}--\eqref{eq3}}
\right\}\!,
\!\!
%\label{eq4} 
\end{equation*}
а все допустимые распределения ресурсов принадлежат подмножеству
\begin{equation*}
{Y}(\mathbf{d})=\left\{\mathbf{y} \ge 0 \mid
(\mathbf{z},\mathbf{x},\mathbf{y}) \ \mbox{удовлетворяют~\eqref{eq1}--\eqref{eq3}}\right\}\!.
%\!\!\!\label{eq5}
\end{equation*}


\section{Метрические оценки предельных распределений}

Для оценки функциональных возможностей сис\-те\-мы рассматриваются
допустимые распределения межузловых потоков при предельных
загрузках ребер сети.

В рамках данного модельного описания монопольным режимом
называется способ управления, при котором все ресурсы сети
используются для передачи потока одной выделенной пары
уз\-лов-кор\-рес\-пон\-ден\-тов $p_{a}\hm\in P(R^-)$, а для всех
остальных потоки полагаются равными нулю.

Предельно допустимый поток, который можно передать между
фиксированной парой уз\-лов-кор\-рес\-пон\-ден\-тов $p_{a}$ в~монопольном
режиме, является решением стандартной, в~данном случае
однопродуктовой, задачи о~максимальном потоке.

\smallskip

\noindent
\textbf{Задача 1.} Найти
$z_a^0\hm=\max\limits_{\langle z,x\rangle \in Z(d)} z_a
$
при условии $z_{i}=0$, $i\hm=\overline{1,M}$, $i\hm\ne a$.

При решении задачи~1 для пары $p_{a}$ вы\-чис\-ля\-ют\-ся: межузловой
поток~$z_a^0$; дуговые потоки $\{x^{0}_{ak};x^{0}_{a(k+E)}\}$,
$k\hm=\overline{1,E}$; суммарное значение реберного
потока~$y_{a}^{0}\hm=\sum\nolimits_{i=1}^{2E} {x}_{ai}^{0}$.

Поток величины $z_a^0$ является \textit{максимальным потоком},
пе\-ре\-да\-ва\-емым в~\textit{монопольном режиме} для пары
уз\-лов-кор\-рес\-пон\-ден\-тов~$p_{a}$, $p_{a}\hm\in P(R^-)$, в~сети~$G(d)$.

Задача~1 решается последовательно для всех $p_{m}\in P(R^-)$,
вы\-чис\-ля\-ют\-ся значения $z_{m}^{0}(t)$.

При проведении вычислительных экспериментов использовалась
итерационная процедура для оценки функциональных возможностей
сис\-те\-мы при передаче межузловых потоков по нескольким маршрутам.
На предварительном этапе шага~$t$ в~сети~$G(t)$ при заданных
значениях пропускной спо\-соб\-ности ребер~$d_k(t)$ для каждой \mbox{пары}
уз\-лов-кор\-рес\-пон\-ден\-тов $p_m\hm\in P(R^-)$ определяется максимально
допустимый однопродуктовый поток~$z^0_m(t)$, со\-от\-вет\-ст\-ву\-ющие
дуговые потоки $(x_{mk}^0(t),x_{m(k+E)}^0(t))$, $p_m\hm\in P(R^-)$, и~коэффициенты нормировки
$\xi_m^0(t)\hm={1}/{z_m^0(t)}$ для всех  $p_m\hm \in P(R^-)$,
таких что $z^0_m(t)\hm>0$, $y_m^0(t)\hm>0$.
Коэффициенты~$\xi_m^0(t)$ используются для поиска текущих
совместно допустимых квот на передачу потоков одновременно между
всеми парами $p_m\in P(R^-)$.

\smallskip

\noindent
\textbf{Задача 2.} Найти $\alpha^*(t)=\max\limits_\alpha \alpha$
при условиях
$$
\alpha\!\!\sum\limits_{m\in R^-}\! \xi_m^0\left(x_{mk}^0(t)+x_{m(k+E)}^0(t)\right)\le d_k(t),\enskip
k=\overline{1,E}\,.
$$

На основании $\alpha^*(t)$ вычисляются совместно допустимые
дуговые потоки:
\begin{multline*}
x_{mk}^*(t)=\alpha^*(t)\xi^0_m(t)x^0_{mk}(t),\\
x^*_{m(k+E)}(t)=\alpha^*(t)\xi^0_m(t)x^0_{m(k+E)}(t),
\\
m=\overline{1,M}\,,\enskip k=\overline{1,E}\,,
\end{multline*}
и остаточная пропускная способность ребер в~сети $G(t+1)$:
\begin{multline*}
d_k(t+1)=d_k(t)-\sum_{m\in R^-} (x_{mk}^*(t)+x_{m(k+E)}(t)),\\
k=\overline{1,E}\,,\enskip p_m\in P(R^-).
\end{multline*}
Формируется вектор допустимых межузловых потоков:
\begin{align*}
z_k^+(t)&=d_k(t+1),\enskip p_k\in P(R^+),\enskip k=\overline{1,E}\,;
\\
z_m^-(t)&=\sum\limits_{\tau=1}^t\alpha^*(\tau)\xi_m^0(\tau) z_m^0(\tau), \enskip p_m\in P(R^-).
\end{align*}

По построению, на шаге~$t$ при передаче вектора межузлового потока
$\mathbf{z}(t)=\{\mathbf{z}^+(t), \mathbf{z}^-(t)\}$ достигается
предельная загрузка, и~пропускная способность всех ребер  сети
используется полностью.

Точка с~координатами $\mathbf{z}(t)$ принадлежит множеству~$Z(d)$.

Расстояние точки от начала координат определяется как норма
соответствующего вектора:
\begin{align*}
\rho^+(t)&=\|\mathbf{z}^+(t)\|=
\left[\,\sum\limits_{k=1}(\mathbf{z}^+(t))^2\right]^{1/2};
\\
\rho^-(t)&=\|\mathbf{z}^-(t)\|= \left[\sum\limits_{p_m\in P(R^-)}(\mathbf{z}_m^-(t))^2\right]^{1/2}.
\end{align*}

Если при выполнении шага $(t+1)$ окажется, что $z_m^0(t+1)=0$ для
всех $p_m\in P(R^-)$, то про\-изойдет останов и~сформируются
массивы финальных данных:
\begin{align*}
z_m^-(T)&=\sum\limits_{\tau=1}^t \alpha^*(\tau)\xi_m^0(\tau) z_m^0(\tau),\enskip 
p_m\in P(R^-),\\
z_k^+(T)&=d_k(t+1),\enskip p_k\in P(R^+),\enskip k=\overline{1,E}\,.
\end{align*}

Вышеописанная вычислительная процедура далее обозначается как
MFPL-про\-це\-ду\-ра (от англ.\ \textit{max-flow-peak-load}).

При проведении второй серии вычислительных экспериментов
MFPL-про\-це\-ду\-ра использовалась для оценки функциональных
характеристик сис\-те\-мы при \textit{уравнительном} поэтапном
распределении пропускной способности между всеми
па\-ра\-ми-кор\-рес\-пон\-ден\-тами.

При реализации MFPL-процедуры выполнение каждого шага разбивается
на несколько этапов. На предварительном этапе шага~$t$ 
в~сети~$G(t)$ при заданных значениях пропускной способности ребер~$d_k(t)$ 
для каждой пары уз\-лов-кор\-рес\-пон\-ден\-тов $p_m\hm\in P(R^-)$
определяется максимально допустимый однопродуктовый
поток~$z_m^0(t)$, соответствующие дуговые потоки
$\left(x_{mk}^0(t),x_{m(k+E)}^0(t)\right)$, $p_m\hm\in P(R^-)$, и~суммарная
реберная нагрузка
$$
y_m^0(t)=\sum\limits_{k=1}^E (x_{mk}^0(t),x_{m(k+E)}^0(t)),\enskip p_m\in P(R^-).
$$

Для каждой пары $p_m\hm\in P(R^-)$ вычисляются коэффициенты
нормировки
$\theta_m^0(t)\hm={1}/{y_m^0(t)}$ для всех  
$p_m\hm\in P(R^-)$, таких что  $z^0_m(t)\hm>0$,
$y_m^0(t)\hm>0$.
Коэффициенты~$\theta_m^0(t)$ используются для поиска совместно
допустимых дуговых потоков для всех $p_m\hm\in P(R^-)$.

\smallskip

\noindent
\textbf{Задача 3.} Найти $\beta^*(t)=\max\nolimits_\beta \beta$ при
условиях
$$
\beta\!\!\!\!\sum\limits_{p_m\in P(R^-)}\!\!
\theta_m^0(x_{mk}^0(t)+x_{m(k+E)}^0(t))\le d_k(t),\enskip
k=\overline{1,E}\,.
$$

 С помощью $\beta^*(t)$ (решения задачи~3) вычисляются текущие допустимые значения дуговых потоков:
\begin{multline*}
x_{mk}^*(t)=\beta^*(t)\theta^0_m(t)x^0_{mk}(t),\\
x^*_{m(k+E)}(t)=\beta^*(t)\theta^0_m(t)x^0_{m(k+E)}(t), \enskip
k=\overline{1,E},
\end{multline*}
и реберных нагрузок при одновременной передаче межузловых потоков:

\noindent
\begin{multline*}
y_m^*(t)=\sum\limits_{i=1}^E
\left[x_{mi}^*(t)+x^*_{m(i+E)}(t)\right]={}\\
{}= \fr{\beta^*(t)}{y_m^0(t)} \sum\limits_{i=1}^E
\left[x_{mi}^0(t)+x^0_{m(i+E)}(t)\right]=\beta^*(t), \\
 p_m\in P(R^-).
\end{multline*}
Таким образом на каждом шаге определенная часть имеющегося ресурса
(пропускной спо\-соб\-ности) делится строго по\-ров\-ну меж\-ду всеми
корреспондентами $p_m\in P(R^-)$, для которых существует путь
передачи в~$G(t)$.

Формируется вектор допустимых межузловых потоков:
\begin{gather*}
\hspace*{-30mm}z_k^{++}(t)=d_k(t+1)={}\hspace*{10mm}\\
{}=d_k(t)-\!\!\! \sum\limits_{p_m\in P(R^-)}\!\!\!
\left(x_{mk}^*(t)+x_{m(k+E)}(t)\right),\\
\hspace*{35mm}k=\overline{1,E}, \enskip
p_k\in P(R^+);\\
z_m^{(=)}(t)\overset{\Delta}{=}\sum\limits_{\tau=1}^t\beta^*(\tau)
\theta_m^0(\tau) z_m^0(\tau), \enskip p_m\in P(R^-).
\end{gather*}

\noindent
Определяются расстояния:
\begin{align*}
\rho^{++}(t)&=\|\mathbf{z}^{++}(t)\|\overset{\Delta}{=}
\left[\sum\limits_{k=1}^E\left(d_k(t+1)\right)^2\right]^{1/2};\\
\rho^{(=)}(t)&=\|\mathbf{z}^{=}(t)\|= \left[\sum\limits_{p_m\in
P(R^-)}\left(z_m^{(=)}(t)\right)^2\right]^{1/2}.
\end{align*}

Если на предварительном этапе на шаге $(t+1)$ окажется, что в~сети~$G(t+1)$ для всех $p_m\hm\in P(R^-)$ все значения
$z_m^0(t+1)\hm=0$, то произойдет останов и~сформируются финальные
массивы:
\begin{align*}
z_k^{(++)}(T)&=d_k(t+1), \enskip
p_k\in P(R^+), \enskip k=\overline{1,E};
\\
z_m^{(=)}(t)&=\sum\limits_{\tau=1}^{t+1}\beta^*(\tau)
\theta_m^0(\tau) z_m^0(\tau), \enskip p_m\in P(R^-).
\end{align*}



\section{Вычислительный эксперимент}

Результаты вычислительных экспериментов, описанные ниже, служат
продолжением исследований, начатых в~[1]. Вычислительные
эксперименты проводились на моделях сетевых сис\-тем, пред\-став\-лен\-ных
на рис.~1 и~2. В~каждой сети~69~узлов. Пропускные спо\-соб\-но\-сти
ребер~-- значения $d_k$~-- выбирались случайным образом из отрезка
$[900,999]$ и~совпадали для ребер, при\-сут\-ст\-ву\-ющих в~обеих сетях.
В~кольцевой сети пропускная спо\-соб\-ность каждого из добавленных
ребер равнялась~900.

\begin{figure*} %fig1
\vspace*{1pt}
\begin{minipage}[t]{80mm}
  \begin{center}  
    \mbox{%
\epsfxsize=69.408mm
\epsfbox{mal-1.eps}
}

\end{center}
\vspace*{-6pt}
\Caption{Базовая сеть}
\end{minipage}
%\end{figure*}
\hfill
%\begin{figure*} %fig2
\vspace*{1pt}
\begin{minipage}[t]{80mm}
  \begin{center}  
    \mbox{%
\epsfxsize=69.408mm
\epsfbox{mal-2.eps}
}

\end{center}
\vspace*{-6pt}
\Caption{Кольцевая сеть}
\end{minipage}
\end{figure*}

\begin{table*}[b]\small %tabl1
\vspace*{-12pt}
\begin{center}

%\renewcommand{\arraystretch}{1.1}
\Caption{Базовая сеть}
\vspace*{2ex}

\begin{tabular}{|c||c|c|c||c|c|c|} 
\hline
&&&&&&\\[-9pt]
$t$  & $\rho^{-}(t)$ & $\rho^{+}(t)$ & $d^{+}(t+1)$ &
$\rho^{=}(t)$ & $\rho^{++}(t)$&  $d^{++}(t+1)$ \\ 
\hline
\hphantom{99}0  & \hphantom{99}0   & 8048&  68256&  \hphantom{9}0   &  8048&   68256\\
1  & \hphantom{9}63  & 4182&  26544&  \hphantom{9}95  &  3881&   24476\\
$\cdots$  & $\cdots$   & $\cdots$   &  $\cdots$    &  $\cdots$   &  $\cdots$   &   $\cdots$\\
11 & \hphantom{9}70  & 3975&  21469&  \hphantom{9}101\hphantom{9} &  3707&   20155\\
$\cdots$& $\cdots$   & $\cdots$   &  $\cdots$    & $\cdots$   &  $\cdots$   &  $\cdots$\\
22 & \hphantom{9}83  & 3861&  19623&  \hphantom{9}122\hphantom{9} &  3586&   18260\\
$\cdots$ & $\cdots$  & $\cdots$   &  $\cdots$   &  $\cdots$   &  $\cdots$  &   $\cdots$\\
33 & \hphantom{9}103\hphantom{9} & 3778&  18827&  \hphantom{9}139\hphantom{9} &  3522&   17601\\
$\cdots$ &$\cdots$  &$\cdots$  & $\cdots$  & $\cdots$   &  $\cdots$  &  $\cdots$\\
44 & \hphantom{9}\bf 190\hphantom{9} & \bf3553&  \bf17503&  \hphantom{9}\bf203\hphantom{9} &  \bf3285&   \bf16201\\
45 & \hphantom{9}\bf1452\hphantom{99}& \bf2166&  \hphantom{9}\bf7069 &  \hphantom{9}\bf1376\hphantom{99}&  \bf2020&   \hphantom{9}\bf6584\\
46 & \hphantom{9}\bf1498\hphantom{99}& \bf2158&  \hphantom{9}\bf6707 &  \hphantom{9}\bf1388\hphantom{99}&  \bf2017&   \hphantom{9}\bf6483\\
$\cdots$ & $\cdots$   & $\cdots$   &  $\cdots$    & $\cdots$   &  $\cdots$   &  $\cdots$\\
52 & \hphantom{9}1535\hphantom{99}& 2155&  \hphantom{9}6413 & \hphantom{9}1442\hphantom{99} &  2011&   \hphantom{9}6059\\
\hline
\end{tabular}
\end{center}
 %\end{table*}
% \begin{table*}\small %tabl2
\begin{center}
\Caption{Кольцевая сеть}
\vspace*{2ex}


\begin{tabular}{|c||c|c|c||c|c|c|} 
\hline
&&&&&&\\[-9pt]
$t$  & $\rho^{-}(t)$ & $\rho^{+}(t)$ & $d^{+}(t+1)$ &
$\rho^{=}(t)$ & $\rho^{++}(t)$&  $d^{++}(t+1)$ \\
 \hline
\hphantom{9}0  &\hphantom{99}0    & 8440  & 75456   &\hphantom{9}0      &8440   &75456\\
\hphantom{9}1  &\hphantom{9}68   & 5317  & 43038   &92     &5045   &40716 \\ 
$\cdots$ &$\cdots$    & $\cdots$     & $\cdots$   &$\cdots$      &$\cdots$      &$\cdots$      \\
11 &\hphantom{9}95   & 3608  & 20459   &124    &3397   &19080  \\
$\cdots$ &$\cdots$   & $\cdots$    & $\cdots$      &$\cdots$     &$\cdots$     &$\cdots$   \\
22 &\hphantom{9}101\hphantom{9}  & 3540  & 19530   &130    &3350   &18338 \\
$\cdots$ &$\cdots$  & $\cdots$   &$\cdots$      &$\cdots$     &$\cdots$   &$\cdots$    \\
33 &\hphantom{9}135\hphantom{9}  & 3346  & 17561   &154    &3220   &17003 \\
$\cdots$  &$\cdots$   & $\cdots$    & $\cdots$      &$\cdots$     &$\cdots$    &$\cdots$    \\
44 &\hphantom{9}234\hphantom{9}  & 3094  & 14881   &269    &2918   &13848 \\
$\cdots$ &$\cdots$   & $\cdots$    &$\cdots$      &$\cdots$     &$\cdots$     &$\cdots$    \\
50 &\hphantom{9}\bf 413\hphantom{9}  & \bf2770  & \bf12901   &\bf329    &\bf2792   &\bf13079 \\
51 &\hphantom{9}\bf1040\hphantom{99} & \bf2299  & \hphantom{9}\bf8801    &\bf334    &\bf2784   &\bf13034 \\
52 &\hphantom{9}\bf1062\hphantom{99} & \bf2297  & \hphantom{9}\bf8672    &\bf974    &\bf2262   &\hphantom{9}\bf8768  \\
$\cdots$ &$\cdots$   &$\cdots$    & $\cdots$      &$\cdots$      &$\cdots$     &$\cdots$    \\
55 &\hphantom{9}1069\hphantom{99} & 2297  & \hphantom{9}8630    &1010\hphantom{9}   &2259   &\hphantom{9}8553  \\
\hline
 \end{tabular}
\end{center}
 \end{table*}




Для базовой сети исходная сумма пропускных способностей:
$D^+(0)\hm=68\,256$, а~для кольцевой сети $D^{++}(0)=75\,456$.
Соответствующие значения $\rho^+(0)$ и~$\rho^{++}(0)$ указаны в~<<нулевой>> строке 
в~табл.~1 и~2, где собраны результаты
вычислительных экспериментов. В~ходе эксперимента при
уравнительном распределении остаточных ресурсов соблюдается
\textit{равномерное} убывание остаточной пропускной спо\-соб\-ности и~<<\textit{длины}>> вектора~$\rho^+(t)$. 
Однако между 44--46
итерациями для базовой и~50--52 для кольцевой сети наблюдается
резкий скачок величин~$\rho^-(t)$, $\rho^{=}(t)$ и~$d^+(t)$,
$d^{++}(t)$.

На указанных шагах полностью используется пропускная способность
ребер в~центральной час\-ти сети. Сеть \textit{распадается} на
несвязные компоненты, и~для $80\%$ корреспондентов пропадают пути
соединения, а~остаточный ресурс распределяется поровну между
оставшимися парами узлов.

Анализ результатов показал, что почти равные значения потоков
достигаются для~80\% корреспондентов и~требуют 60\%--70\%
ресурсов. Однако для~2\% смежных  пар узлов межузловые потоки на
два порядка выше медианных значений, а~затраты пропускной
способности  со\-став\-ля\-ют~20\%--30\%.








\section{Заключение}

Предложенный метод и~проведенные вычислительные эксперименты
показали, что уравнительное поэтапное распределение   приводит 
к~неравномерному  распределению   потоков  для разных групп\linebreak
корреспондентов.    Метрические оценки, полученные  в~ходе
экспериментов, продемонстрировали\linebreak \textit{деформацию} множества
достижимых потоков. В~рамках модели   предполагалось, что  все
корреспонденты  равноправны, а~потоки невзаимозаменяемы,  однако
при уравнительном предельном  распределении  смежные  пары узлов
оказывались в~привилегированном положении при использовании
остаточной пропускной способности. Пропускные способности  ребер
рассматривались  как вектор   ресурсов  различных типов,  которые
распределяются между корреспондентами   при передаче  потоков
различных видов.  По построению, на каж\-дом шаге норма вектора
смежных   межузловых    потоков численно равна   модулю вектора
остаточных  пропускных способностей.   Полученные мет\-ри\-че\-ские
значения  можно использовать  для   оценки функциональных
возможностей сети  в~режиме  предельной загрузки.

{\small\frenchspacing
 {%\baselineskip=10.8pt
 %\addcontentsline{toc}{section}{References}
 \begin{thebibliography}{9}

\bibitem{1-mal}
\Au{Малашенко Ю.\,Е., Назарова И.\,А.} Неоднородность
распределения   потоков при предельной  загрузке
многопользовательской сети~//  Известия РАН. Теория и~сис\-те\-мы
управления,  2022. №\,3. С.~81--96.

\bibitem{4-mal} %2
\Au{Luss H.} Equitable resource allocation: Models,
algorithms, and applications.~--- Hoboken, NJ, USA: John Wiley \& Sons, 2012.
420~p.

\bibitem{2-mal} %3
\Au{Ogryczak W., Luss~H., Pioro~M., Nace~D., Tomaszewski~A.}   Fair
optimization and networks: A~aurvey~// J.~Appl. Math., 2014. Vol.~2014. Art.~ID~612018. 25~p. doi: 10.1155/ 2014/612018.

\bibitem{3-mal} %4
\Au{Salimifard K., Bigharaz~S.} The multicommodity network
flow problem: State of the art classification, applications, and
solution methods~// J.~Oper. Res., 2020. Vol.~18. Iss.~3. P.~1--47.



\bibitem{5-mal}
\Au{Balakrishnan A., Li~G., Mirchandani~P.}  Optimal
network design with end-to-end service requirements~// Oper. Res.,
2017. Vol.~65. Iss.~3. P.~729--750.

\bibitem{6-mal}
\Au{Nace D., Doan~L.\,N., Klopfenstein~O., Bashllari~A.} Max-min
fairness in multicommodity flows~// Comput. Oper. Res., 2008.
Vol.~35. Iss.~2. P.~557--573.

\bibitem{7-mal}
\Au{Ros-Giralt J., Tsai~W.\,K.} A~lexicographic optimization
framework to the flow control problem~// IEEE T.
Inform. Theory, 2010. Vol.~56. Iss.~6. P.~2875--2886.

\bibitem{8-mal}
\Au{Baier G., Kohler~E., Skutella~M.}  The \mbox{k-splittable}
flow problem~//  Algorithmica, 2005. Vol.~42. Iss.~3-4.
P.~231--248.

\bibitem{9-mal}
\Au{Bialon P.\,A.} Randomized rounding approach to 
a~\mbox{k-splittable} multicommodity flow problem with lower path flow
bounds affording solution quality guarantees~// Telecommun. Syst.,
2017. Vol.~64. Iss.~3. P.~525--542.
\end{thebibliography}

 }
 }

\end{multicols}

\vspace*{-6pt}

\hfill{\small\textit{Поступила в~редакцию 10.06.22}}

\vspace*{8pt}

%\pagebreak

%\newpage

%\vspace*{-28pt}

\hrule

\vspace*{2pt}

\hrule

%\vspace*{-2pt}

\def\tit{SEQUENTIAL ANALYSIS AND METRIC ESTIMATES\\ OF~PEAK LOAD FLOWS IN~THE~MULTIUSER NETWORK}


\def\titkol{Sequential analysis and metric estimates of~peak load flows in~the~multiuser network}


\def\aut{Yu.\,E.~Malashenko}

\def\autkol{Yu.\,E.~Malashenko}

\titel{\tit}{\aut}{\autkol}{\titkol}

\vspace*{-8pt}


\noindent
Federal Research Center ``Computer Science and Control'' of the Russian Academy of Sciences, 
44-2~Vavilov Str., Moscow 119333, Russian Federation



\def\leftfootline{\small{\textbf{\thepage}
\hfill INFORMATIKA I EE PRIMENENIYA~--- INFORMATICS AND
APPLICATIONS\ \ \ 2022\ \ \ volume~16\ \ \ issue\ 3}
}%
 \def\rightfootline{\small{INFORMATIKA I EE PRIMENENIYA~---
INFORMATICS AND APPLICATIONS\ \ \ 2022\ \ \ volume~16\ \ \ issue\ 3
\hfill \textbf{\thepage}}}

\vspace*{3pt} 



\Abste{The set of vectors of internodal flows in a~multiuser communication network under peak load is analyzed. Within the framework of
 the multicommodity model, the throughput capacities of edges are considered as the components of a~vector of resources of various types that 
 are required for the transmission of various kinds of
 flows. When conducting computational experiments, at each iteration, the
  norms of vectors of jointly permissible internodal flows are calculated, during the transmission of which the capacity of 
  all network edges is fully used.\linebreak\vspace*{-12pt}}
 
 \Abstend{The proposed method and computational experiments have shown that the equalizing phased 
  distribution leads to an uneven distribution of flows for different groups of correspondents. Metric values obtained during experiments 
  indicate deformation of the sets of accessible flows. Within the framework of the model, all correspondents are tantamount 
  and the flows are noninterchangeable; however, in the case of an equalizing peak load distribution, adjacent pairs 
  of nodes are in a privileged position when using residual capacity. The obtained metric values can be used to 
  evaluate the functional characteristics of the transmission network in the finite capacity loading mode.}

\KWE{multicommodity flow network model; set of achievable internodal flows; peak load distribution}


\DOI{10.14357/19922264220306} 

%\vspace*{-16pt}

%\Ack
%\noindent



%\vspace*{4pt}

  \begin{multicols}{2}

\renewcommand{\bibname}{\protect\rmfamily References}
%\renewcommand{\bibname}{\large\protect\rm References}

{\small\frenchspacing
 {%\baselineskip=10.8pt
 \addcontentsline{toc}{section}{References}
 \begin{thebibliography}{9}
\bibitem{1-mal-1}
\Aue{Malashenko, Yu.\,E., and I.\,A.~Nazarova.}
2022. Heterogeneous flow distribution at the peak load in the multiuser network. \textit{J.~Comput. Sys. Sc. Int.} 61:372--387.

\bibitem{4-mal-1} %2
\Aue{Luss, H.} 2012. \textit{Equitable resource allocation: Models, algorithms, and applications}.
Hoboken, NJ: John Wiley \& Sons. 420~p.

\bibitem{2-mal-1} %3
\Aue{Ogryczak, W., H.~Luss, M.~Pioro, D.~Nace, and A.~Tomaszewski.}
 2014. Fair optimization and networks: A~survey. \textit{J.~Appl. Math.} 2014:612018. 25~p. doi: 10.1155/ 2014/612018.
\bibitem{3-mal-1} %4
\Aue{Salimifard, K., and S.~Bigharaz.}
 2020. The multicommodity network flow problem: State of the art classification, applications, and solution methods. 
 \textit{J.~Oper. Res.} 18(3):\linebreak 1--47.

\bibitem{5-mal-1}
\Aue{Balakrishnan, A., G.~Li, and P.~Mirchandani.} 2017. Optimal network design with end-to-end service requirements. 
\textit{Oper. Res.} 65(3):729--750.
\bibitem{6-mal-1}
\Aue{Nace, D., L.\,N.~Doan, O.~Klopfenstein, and A.~Bashllari.} 2008. Max-min fairness in multicommodity flows. 
\textit{Comput. Oper. Res.} 35(2):557--573.
\bibitem{7-mal-1}
\Aue{Ros-Giralt, J., and W.\,K.~Tsai.} 2010. A~lexicographic optimization framework to the flow control problem. 
\textit{IEEE T.~Inform. Theory} 56(6):2875--2886.
\bibitem{8-mal-1}
\Aue{Baier, G., E.~Kohler, and M.~Skutella.}
 2005. The k-splittable flow problem. \textit{Algorithmica} 42(3-4):231--248.
\bibitem{9-mal-1}
\Aue{Bialon, P.} 2017. A~randomized rounding approach to a~\mbox{k-splittable} multicommodity flow problem with lower path flow bounds affording solution quality guarantees. 
\textit{Telecommun. Syst.} 64(3):525--542.
 \end{thebibliography}

 }
 }

\end{multicols}

\vspace*{-6pt}

\hfill{\small\textit{Received June 10, 2022}}

\Contrl

\noindent
\textbf{Malashenko Yuri E.} (b.\ 1946)~--- 
Doctor of Science in physics and mathematics, principal scientist, Federal Research Center ``Computer Science and Control'' 
of the Russian Academy of Sciences, 44-2~Vavilov Str., Moscow 119333, Russian Federation; \mbox{malash09@ccas.ru} 


\label{end\stat}

\renewcommand{\bibname}{\protect\rm Литература}    %5
\def\stat{strijov}

\def\tit{ВОССТАНОВЛЕНИЕ МАТРИЦЫ СУПЕРПОЗИЦИИ В~ЗАДАЧЕ~СИМВОЛЬНОЙ РЕГРЕССИИ$^*$}

\def\titkol{Восстановление матрицы суперпозиции в~задаче символьной регрессии}

\def\aut{Р.\,Г.~Нейчев$^1$, И.\,А.~Шибаев$^2$, В.\,В.~Стрижов$^3$}

\def\autkol{Р.\,Г.~Нейчев, И.\,А.~Шибаев, В.\,В.~Стрижов}

\titel{\tit}{\aut}{\autkol}{\titkol}

\index{Нейчев Р.\,Г.}
\index{Шибаев И.\,А.}
\index{Стрижов В.\,В.}
\index{Neychev R.\,G.}
\index{Shibaev I.\,A.}
\index{Strijov V.\,V.}


{\renewcommand{\thefootnote}{\fnsymbol{footnote}} \footnotetext[1]
{Работа выполнена при поддержке РФФИ (проекты 20-37-90050 и~20-07-00990).}}


\renewcommand{\thefootnote}{\arabic{footnote}}
\footnotetext[1]{Московский физико-технический институт, 
\mbox{neychevr@gmail.com}}
\footnotetext[2]{Московский физико-технический институт, 
\mbox{shibaev.kesha@gmail.com}}
\footnotetext[3]{Федеральный исследовательский центр <<Информатика 
и~управ\-ле\-ние>> Российской академии наук, \mbox{strijov@phystech.edu}}

\vspace*{-12pt}
 



\Abst{Исследуется проблема порождения структуры регрессионной модели. 
Модель представляет собой суперпозицию базовых функций. Структура модели 
описывается взвешенным цвет\-ным графом. Каждая вершина графа соответствует 
некоторой базовой функции. Ребро задает суперпозицию двух функций. Вес ребра 
равен вероятности суперпозиции. Для создания оптимальной модели необходимо 
восстановить ее структуру по матрице смежности графа. Пред\-ла\-га\-емый алгоритм 
восстанавливает минимальное остовное дерево из взвешенного цветного графа. 
Пред\-став\-ле\-но новое решение, основанное на алгоритме дерева Штейнера. 
Алгоритм сравнивается с~альтернативами.}


\KW{символьная регрессия; линейное программирование; 
суперпозиция; минимальное остовное дерево; мат\-ри\-ца смеж\-ности}

\DOI{10.14357/19922264230105} 
  
\vspace*{-8pt}


\vskip 10pt plus 9pt minus 6pt

\thispagestyle{headings}

\begin{multicols}{2}

\label{st\stat}

\section{Введение}

Символьная регрессия~--- это метод по\-стро\-ения нелинейной модели, 
аппроксимирующей выборку. Структура модели определяется суперпозицией базовых 
функций. Набор базовых функций фиксируется для конкретной прикладной задачи. 
Структуры альтернативных моделей генерируются алгоритмом оптимизации для выбора 
оптимальной модели. В данной статье предлагается определять структуру модели 
с~по\-мощью вероятностного графа. Остовное дерево в~графе определяет некоторую 
суперпозицию. Для выбора оптимальной модели необходимо реконструировать 
минимальное остовное дерево по графу.

Методы генетического программирования~\cite{koza1992genetic} находят оптимальное 
подмножество в~наборе суперпозиций базовых функций, но имеют высокую 
вычислительную сложность. В~\cite{searson2010gptips} описаны методы, понижающие 
сложность. Они используют дополнительные ограничения на суперпозиции, например 
используют линейные комбинации базовых функций. Символьная регрессия, 
описанная~в~\cite{stanley2002evolving}, используется для оптимизации структуры 
суперпозиции. Методы решения задачи символьной регрессии основаны на матричном 
представлении структуры модели~\cite{bochkarev2017generation}. Однако эти методы 
не содержат ограничений на чис\-ло аргументов базовых функций и~на структуру 
графа, обеспечивающую допустимую суперпозицию. В~данной работе решается задача 
построения модели с~помощью символьной регрессии.

Требуется восстановить допустимую суперпозицию из предсказанной мат\-ри\-цы 
смежности с~вероятностями ребер. Решается задача вос\-ста\-нов\-ле\-ния~$k$-минимального 
остовного дерева $k$-MST (\textit{англ.}\ Minimum-cost Spanning Tree). Эта задача NP-слож\-ная, 
поэтому применимы только при\-бли\-жен\-ные решения~\cite{ravi1996spanning}. 
Алгоритм~$k$-MST эквивалентен проб\-ле\-ме дерева Штейнера PCST (\textit{англ.}\ 
Prize-Collecting Steiner Tree) из-за его эквивалентности ослабленной формулировке 
постановки задачи линейного программирования~\cite{chudak2004approximate}. 
В~работах~\cite{ravi1996spanning,awerbuch1998new,arora20062+} пред\-став\-ле\-ны 
приближенные решения задачи \mbox{$k$-MST}.



Предлагаемое решение основано на упрощенной версии задачи~$k$-MST, которая 
трансформируется в~задачу PCST с~постоянными призами, одинаковыми для всех 
вершин. Быст\-рый алгоритм PSCT описан в~\cite{hegde2014fast}. Альтернативное 
решение основано на алгоритме~$(2-\varepsilon)$-аппроксимации для задачи PSCT. 
Она сравнивается с~другими алгоритмами, включая алгоритмы обхода дерева в~глубину, обхода дерева в~ширину, алгоритмы Прима.

\begin{table*}[b]\small  %tabl1
\vspace*{-12pt}
\begin{center}
        \parbox{262pt}{\Caption{Вероятности суперпозиций в~матрице смежности порождают 
ориентированный граф}

}
    \label{restored_adjacency_matrix}
\vspace*{2ex}

        \begin{tabular}{|c|c|ccccccc|}
            \hline
            Арность&Функция&$\ast$&$+$&$\ln$&$\sin$&$\times$&$\exp$&$x$\\
            \hline
            $1$&$\ast$ &0,2&{\bf 0,7}&0,5&0,4&0,5&0,3&0,2\\
            $3$&$+$    &0,3&0,2&{\bf 1,0}&{\bf 0,8}&0,6&0,3&{\bf 0,7}\\
            $1$&$\ln$  &0,3&0,2&0,0&0,0&0,1&0,5&{\bf 0,5}\\
            $1$&$\sin$ &0,1&0,4&0,0&0,5&{\bf 0,9}&0,2&0,5\\
            $2$&$\times$&0,3&0,0&0,3&0,5&0,0&{\bf 0,8}&{\bf 0,6}\\
            $1$&$\exp$ &0,3&0,3&0,4&0,1&0,5&0,4&{\bf 0,4}\\
            \hline
        \end{tabular}
\end{center}
\end{table*}

\vspace*{-12pt}


\section{Задача выбора регрессионной модели}

\vspace*{-3pt}

Требуется выбрать регрессионную модель~$\varphi$ из набора альтернативных 
моделей. Модель описывает выборку~$D=\{(x_i,y_i)\}$ и~минимизирует ошибку

\noindent
\begin{equation}
\hat{\varphi}(D)=\mathop{\argmin}\limits_\varphi\sum\limits_{i=1}^m\left(\varphi(x_i)-
y_i\right)^2.
\label{task_1}
\end{equation}
Модель представляет собой суперпозицию базовых функций из некоторого заданного 
набора. На рис.~1\linebreak\vspace*{-12pt}

{ \begin{center}  %fig1
 \vspace*{-3pt}
    \mbox{%
\epsfxsize=37.447mm
\epsfbox{str-1.eps}
}

\end{center}

\vspace*{-2pt}

\noindent
{{\figurename~1}\ \ \small{Структура регрессионной модели представляет собой ориентированный 
граф
}}}

\vspace*{6pt}

\addtocounter{figure}{1}


\noindent
 показан ее пример. Структура модели~$\varphi$, 
суперпозиция, соответствует графу~$G=(V,E)$, где базовые функции находятся 
в~вершинах~$V$. {Корневая} вершина обозначается через~$\ast$. Модель:

\vspace*{1pt}

\noindent
$$
\varphi(D) =  \ln(x) + x + \sin\left( x\times \exp(x)\right).
$$

\vspace*{-4pt}

\noindent
 Еe структура в~виде матрицы 
смежности графа пред\-став\-ле\-на~в табл.~\ref{restored_adjacency_matrix}.
Базовые функции перечислены в~первой строке. Элементами матрицы являются 
вероятности ребер~$E$ дерева. Жир\-ным шриф\-том выделены ребра восстановленного 
дерева~$M$, образующие суперпозицию~$\varphi$. Для восстановления структуры 
модели~$\varphi$ как суперпозиции, заданной деревом~$M$, необходимы только 
графовое пред\-став\-ле\-ние~$G$~и~базовые функции.



Поставим задачу восстановления структуры модели. Задано множество 
выборок~$\{D_k\}$. Каждой выборке~$D_k$ соответствует своя модель. Эта модель 
имеет структуру~$M_k$. Таким образом, имеется набор пар $\{(D_k, M_k)\}$, 
выборка и~структура.
Обозначим через~$P$ отображение, которое предсказывает вероятности узлов 
в~графе~$G$ по выборке~$D$. Для выбора модели~$\varphi(D)$ необходимо восстановить 
структуру модели~$M$ по графу~$G$. Обозначим алгоритм восстановления дерева 
через~$R$. Регрессионная модель~$\hat{\varphi}(D)$, которая решает 
задачу~(\ref{task_1}), определяется формулой
$
\hat{M}=R\left(P(D)\right).
$
Поскольку дерево~$M$ играет центральную роль в~этой работе, критерий качества 
алгоритма восстановления дерева имеет вид:


\vspace*{-3pt}

\noindent
$$
\min_{M_k \in G} \fr{1}{K}\sum\limits_{k=1}^K \left[ \hat{M_k} = M_k\right].
$$

\vspace*{-4pt}

\noindent
Восстановленное дерево должно быть эквивалентно заданному дереву, следовательно, 
выбранная модель регрессии при\-бли\-жа\-ет выборку.

\vspace*{-10pt}

\section{Задача восстановления дерева суперпозиции}

\vspace*{-3pt}

Требуется восстановить дерево~$M_k$, задающее  суперпозицию и~решающее 
задачу~(\ref{task_1}). Задан ориентированный взвешенный граф~$G\hm=(V,E)$ 
с~раскрашенными вершинами~$v_i$ и~корневой вершиной~$r$. Каждая вершина~$v_i \hm\in 
V$ имеет свой цвет~$t(v_i)\hm=t_i$. Каждое реб\-ро~$e_i\in E$ имеет свой 
вес~\mbox{$w(e_i)\hm=c_i\hm\in[0,1]$}.

Требуется восстановить ориентированное дерево минимального веса с~корнем~$r$. 
Оно должно покрывать не менее~$k$ вершин в~заданном графе~$G$. Чис\-ло ребер, 
выходящих из вершины~$v_i$ дерева, должно быть меньше или равно~$t_i$. 
Корень~$r$ имеет одно ребро,~$t_r=1$.

Сформулируем это условие в~виде задачи линейного программирования 
с~целочисленными ограничениями:

\vspace*{-5pt}

\noindent
\begin{multline}
\underset{\substack{{x_e, z_S} \\ e\in E,\\ S\subseteq V\backslash 
\{r\}}}{\mbox{minimize}}  \displaystyle \sum\limits_{e\in E}c_ex_e \\[-3pt]
\mbox{s.t.}\  \displaystyle  \sum\limits_{\substack{{e\in\delta(S):}\\ e=(\ast,v_i),\\ v_i\in\delta(S)}} \!\!\!\! x_e + 
\sum\limits_{T:T\supseteq S}  \!\!\!\! z_T\geqslant 1,\enskip  S\subseteq 
V\backslash \{r\};\\[-3pt]
 \displaystyle \sum\limits_{e\in E:~e=(\ast,v)} \! x_e\leqslant 1,\enskip v\in V;\\[-3pt]
 \displaystyle \sum\limits_{e\in E:~e=(v,\ast)}x_e\leqslant t_i,\enskip  v\in V;\\[-3pt]
 \displaystyle \sum\limits_{S\subseteq V\backslash \{r\}}|S|z_S \leqslant n-k,\enskip  x_e\in\{0,1\},\enskip 
 z_S\in\{0,1\},\\[-3pt]
  e\in E,\enskip   S\subseteq V\backslash \{r\},
\label{ilp_our}
\end{multline}
где
$$
x_e =\begin{cases}
 1, &\mbox{если\ ребро}\ e\ \mbox{входит\ в~финальную}\\
 &\mbox{суперпозицию};\\
 0 & \mbox{в~противном\ случае};
 \end{cases}
 $$
  $z_S\hm = 1$ для всех вершин, исключенных из финальной 
суперпозиции. Обозначим через~$e\hm=(\ast, v)$ ориентированное ребро с~листом~$v$. 
Обозначим через $e\hm=(v, \ast)$ ориентированное ребро с~вершиной~$v$.

Первое ограничение~(\ref{ilp_our})  определяет структуру графа решения в~виде 
дерева с~корнем~$r$. Второе ограничение определяет ориентацию дерева: каждая 
вершина имеет не более одного входящего ребра. Третье ограничение определяет 
арность используемых базовых функций, поэтому число ребер, имеющих определенную 
вершину в~качестве источника, фиксировано. Четвертое ограничение говорит, что 
итоговое дерево имеет не менее~$k$ вершин. Если все веса неотрицательны, то 
четвертое ограничение на минимальное число вершин принимает более строгий вид: 
число вершин должно быть равно~$k$. Однако более слабое ограничение позволяет 
найти возможные связи с~другими оптимизационными задачами. Ограничения 
в~(\ref{ilp_our}) преследуют ту же цель.

\vspace*{-9pt}

\section{Алгоритмы восстановления дерева $k$-MST и~PCST}

\vspace*{-3pt}

\noindent
\textbf{Определение~1} (\textbf{$\bm{k}$-минимальное остовное дерево,\linebreak $\bm{k}$-MST}).
Задан взвешенный граф~$G\hm=(V,E)$ с~корнем~$r$ и~весами ребер~$w(e_i)\hm=c_i\hm\geqslant 
0$, $e_i\hm\in E$. Требуется построить ориентированное дерево минимального веса 
с~корнем~$r$, покрывающее не менее~$k$ вершин в~$G$.

\smallskip

Если та же задача ставится для ориентированных графов, то конечное дерево 
с~корнем~$r$ должно быть ориентированным. Задача линейного программирования для 
направленного~$k$-MST исключает \mbox{третье} условие в~(\ref{ilp_our}).
В~таком виде задача~$k$-MST отличается от исходной задачи восстановления 
дерева суперпозиций~(\ref{ilp_our}) отсутствием третьего ограничения на арность 
базовых функций. Это эквивалентно ограничению на число ребер, выходящих из 
вершины.

\smallskip

\noindent
\textbf{Определение~2} (\textbf{призовое дерево Штейнера, $\text{PCST}$}).\linebreak
Задан взвешенный граф $G\hm=(V,E)$ с~корнем~$r$ и~весами ребер~$w(e_i)\hm=c_i\hm\geqslant  0$, $e_i\hm\in E$, где каждой вершине~$v_i \hm\in V$ присвоен 
{приз} $\pi(v_i)\hm=\pi_i\geqslant 0$. Требуется построить дерево~$T$ с~корнем~$r$, 
которое \mbox{минимизирует} функционал
$\sum\nolimits_{e\in E}c_ex_e \hm+ \sum\nolimits_{S\subseteq V\backslash\{r\}} 
\pi(S)z_S,$
где~$x_e\in\{0, 1\}$, $x_e\hm=1$, если~$e\hm\in E$ входит в~тройку~$T$; $z_S\hm\in\{0, 1\}$, 
$z_S\hm=1$ для всех вершин, исключенных из дерева~$T$; $S \hm= V\backslash V(T)$; $\pi(S)\hm= \sum\nolimits_{v\in S}\pi(v)$.

\smallskip

В случае ориентированных графов эта задача обобщается до~асимметричной задачи 
A-PCST. Задача линейного программирования для~A-PCST принимает вид:

\vspace*{-4pt}

\noindent
\begin{multline}
\underset{\substack{x_e,z_S \\ e\in E,\\ S\subseteq V\backslash \{r\}}}{\mbox{minimize}} 
\displaystyle \sum\limits_{e\in E} c_e x_e + \sum\limits_{S\subseteq V\backslash\{r\}}  \!\!\!\!\!\pi(S)z_S \\
\mbox{s.t.}\ \displaystyle \sum\limits_{\substack{e\in\delta(S):\\e=(\ast,v_i),\\ v_i\in\delta(S)}} \!\!\!\!\!\! x_e + 
\sum\limits_{T:T\supseteq S}  \!\!\! z_T\geqslant 1,\enskip  S\subseteq  V\backslash \{r\};\\
\displaystyle \sum\limits_{e\in E:~e=(\ast,v)}\!\!\!\!  x_e\leqslant 1,\enskip
x_e\in\{0,1\},\enskip z_S\in\{0,1\},\enskip  v\in V,\\
e\in E,\enskip S\subseteq V\backslash \{r\}.
\label{ilp_pcst_ord}
\end{multline}

\vspace*{-3pt}

\noindent
Если последнее ограничение из~(\ref{ilp_our}) входит в~оптимизируемый 
функционал, задачи $k$-MST и~A-PCST имеют эквивалентные 
ограничения и~отличаются только оптимизируемым функционалом. Такое 
преобразование возможно согласно условиям Ка\-ру\-ша--Ку\-на--Так\-ке\-ра~\cite{ras2017approximate}. Если значения призов 
эквивалентны $\pi(v) \hm=  \lambda$, единственное отличие состоит в~постоянном члене~$\lambda(n\hm-k)$. Таким 
образом, задачи оптимизации~$k$-MST и~A-PCST принимают вид:

\vspace*{-4pt}

\noindent
\begin{align*}
\underset{\substack{x_e,z_S \\ e\in E,\\ S\subseteq V\backslash \{r\}}}{\mbox{minimize}} & 
\sum\limits_{e\in E}c_ex_e + \lambda\left(\sum\limits_{S\subseteq V\backslash \{r\}}|S|z_S - (n-k)\right);\\ 
\underset{\substack{x_e,z_S \\ e\in E,\\ S\subseteq V\backslash \{r\}}}{\mbox{minimize}} & 
\sum\limits_{e\in E}c_ex_e + \lambda\sum\limits_{S\subseteq V\backslash\{r\}}|S|z_S\,. 
\end{align*}
%
Константа~$\lambda$ обозначает неотрицательный множитель Лагранжа в~задаче~$k$-MST и~приз за вершину\linebreak 
в~задаче~A-PCST. 
Существуют несколько алгоритмов для решения проблемы~PCST, но не для 
решения проб\-ле\-мы A-PCST. Возможное решение~--- снять 
ограничения на ориентацию графа, чтобы\linebreak алгоритм~PCST мог позже 
восстановить ориентацию дерева.

\vspace*{-9pt}

\section{Решение задачи восстановления ограниченного леса с~помощью алгоритма 
$(2-\varepsilon)$-приближения}

\vspace*{-3pt}

Обзор методов решения задачи восстановления ограниченного леса представлен 
в~\cite{goemans1995general}. Задан взвешенный неориентированный граф~$G\hm=(V,E)$. 
Все его веса~$w(e_i)\hm=c_i\geqslant 0$, $e_i\hm\in E$. Задана некоторая 
функция~$f:2^{V}\to \{0, 1\}$. Требуется решить задачу линейного 
программирования с~целочисленными ограничениями:

\vspace*{-4pt}

\noindent
\begin{multline}
\underset{x_e:~e\in E}{\mbox{minimize}} \displaystyle \sum\limits_{e\in E}c_ex_e\\
\mbox{s.t.}\  x\left(\delta(S)\right)\geqslant f(S),\enskip  S \subset V, \enskip S \not= \emptyset,\\
 x_e\in\{0,1\},\enskip  e\in E.
\label{ilp_cfp}
\end{multline}

\vspace*{-3pt}

\noindent
Здесь
$$
x(\delta(S))=\sum\limits_{e\in \delta(S)}x_e,
$$
где $x_e\hm=1$, если 
ребро~$e$ входит в~финальное решение. Функция~$\delta(S)$ обозначает все ребра 
из~$E$ такие, что только одна из смежных вершин входит в~$S$.

Предположим, что отображение~$f$ удовлетворяет условиям

\vspace*{-3pt}

\noindent
\begin{gather*}
f(V) = 0,\\
 \underbrace{f(S)=f(V\backslash S)}_{\mathrm{симметричность}},\\
\underbrace{A,B\!\subset\! V\!: A\!\cap\! B\! =\! \emptyset, f(A)\!=\!f(B)\!=\!0\!\to\! f(A\!\cup\! B)\! =\! 0}_{\mathrm{дизъюнктивность}}.
\end{gather*}

\vspace*{-2pt}

\noindent
При выполнении этих условий~$f$ задает число ребер, начинающихся в~множестве 
вершин~$S$.

\smallskip

\noindent
\textbf{Лемма 1.}
\textit{Пусть $B\subseteq S\subset V$. Тогда $f(S) \hm= 0$ и~$f(B) \hm= 0$ приводит к}~$f(S\backslash B) \hm= 0$.

\smallskip

Задача с~таким описанием относится к~\textit{задачам поиска оптимального леса с~ограничениями}. 
Такая постановка задачи~(\ref{ilp_cfp}) с~соответствующим 
отображением~$f$ подходит для многих известных задач взвешенных графов, 
например: минимальный магистральный поиск, $st$-путь, задача Штейнера на 
минимальном дереве. Последняя задача является NP-полной, поэтому применим 
приближенный алгоритм.

\smallskip

\noindent
\textbf{Определение 3} (\textbf{алгоритм $\bm{\alpha}$-аппроксимации}).
Эвристический полиномиальный алгоритм, дающий\linebreak решение некоторой задачи 
оптимизации, называется $\alpha$-ап\-прок\-си\-ма\-ци\-ей, если он гарантирует 
удовлетворяющее ограничениям решение этой задачи оптимизации с~коэффициентом, 
меньшим или равным~$\alpha$, так что решение отличается от оптимального не более 
чем в~$\alpha$ раз по оптимизируемому функционалу.


\smallskip

Чтобы предложить приближенный алгоритм, целочисленные ограничения 
в~(\ref{ilp_cfp}) должны быть ослаблены:

\vspace*{-3pt}

\noindent
\begin{multline*}
\underset{x_e:~e\in E}{\mbox{minimize}}\  \displaystyle \sum\limits_{e\in E}c_ex_e \\
\mbox{s.t.}\  \displaystyle \sum\limits_{e\in \delta(S)}x_e\geqslant f(S),\enskip S \subset V\,, \enskip S \not= \emptyset\,,\\
 x_e>0,\enskip  e\in E,
%\label{rlp_cfp}
\end{multline*}
Двойственная задача принимает вид:

\vspace*{-4pt}

\noindent
\begin{multline}
\underset{y_S:~S \subset V, \; S \not= \emptyset}{\mbox{maximize}}\  
\displaystyle \sum\limits_{S\subset V}f(S)y_S \\
\mbox{s.t.}\  \displaystyle \sum\limits_{S:~e\in \delta(S)}y_S\leqslant c_e,\enskip  e\in E\,,\\
 y_S>0,\enskip  S \subset V, \enskip S \not= \emptyset\,,
\label{rd_cfp}
\end{multline}

\vspace*{-3pt}

\noindent
относительно дополнительного условия
$$
y_S \left(\sum\limits_{e\in \delta(S)}x_e - f(S)\right) = 0\,,\enskip S\subset  V\,.
$$

Обозначим множество вершин $A=\{v\hm\in V: f(\{v\})\hm=1\}$. Предлагается адаптивный 
жадный алгоритм $\left(2-{2}/{\vert A\vert }\right)$-ап\-прок\-си\-ма\-ции для задач 
вида~(\ref{ilp_cfp}). Алгоритм состоит из двух этапов. На первом этапе он жадно 
объединяет кластеры вершин, увеличивая двойственные переменные~$y_S$. Изначально 
каждая вершина принадлежит своему клас\-те\-ру. Если сле\-ду\-ющее реб\-ро~$e$ достигает 
равенства в~ограничениях в~(\ref{rd_cfp}), это ребро добавляется к~множеству~$S$ и~связанные клас\-те\-ры объединяются. Этот этап аналогичен алгоритму минимального 
остовного дерева Крускала. На втором этапе из конечного множества~$S$ удаляются 
некоторые ребра. Если обрезка ребра не нарушает ограничений, то это реб\-ро должно 
быть удалено.


Индекс $Z_{\mathrm{DRLP}}$ в~алгоритме~1 обозначает линейное 
программирование с~двойной релаксацией. Начальное значение $F:=\emptyset$ 
в~алгоритме~1 эквивалентно предположению $x_e \hm= 0$, $ e \hm\in E$. 
По условиям нежесткости $y_S \hm= 0$, $S \hm\subset V$,  $S \hm\not= \emptyset$.

На каждом шаге алгоритма кластер $\mathcal{C}$ содержит две компоненты 
$\mathcal{C} \hm= \mathcal{C}_i \hm\cup \mathcal{C}_a$, где $C\hm\in\mathcal{C }_a$, если 
$f(C) \hm= 1$, и~$C\hm\in\mathcal{C}_i$ в~противном случае. Назовем~$\mathcal{C}_a$ 
активным компонентом.
Переменные~$d(v)$ в~этом алгоритме связаны с~переменными~$y_S$ из~(\ref{rd_cfp}) 
соотношением
$$
d(i) = \sum\limits_{S:i\in S}y_S.
$$ 

Рассмотрим две различные компоненты $C_q$ и~$C_p$, $C_q\cap C_p\hm=\emptyset$, на 
некоторой итерации первого этапа алгоритма. Все~$y_S$ должны быть равномерно 
распределены по некоторому~$\varepsilon$ без нарушения ограничений
$$
\sum\limits_ {S:~e\in \delta(S)}y_S\leqslant c_e. 
$$
В терминах $d(v)$ это условие принимает вид:
$$
\sum\limits_{S:~e\in \delta(S)}y_S = d\left(v_1\right)+d\left(v_2\right),\enskip e=\left( v_1,v_2\right),
$$
поэтому $y_S\hm=0$ для любого~$S$ такого, что $v_1, v_2\hm\in S$, потому что 
компоненты растут только на первом этапе. Увеличение некоторых компонент на~$\varepsilon$ приводит к~уравнению
$$
d(v_1)+d(v_2)+\varepsilon \left(f(C_q)+f(C_p)\right)\leqslant 
c_e,\ e=\left(v_1,v_2\right), 
$$
что приводит к~формуле, используемой в~строке~$10$ алгоритма~1. 
В~случае когда в~состав входит следующее ребро, сумма $\sum\nolimits_{S:~e\in 
\delta (S)}y_S$ не будет увеличиваться, поэтому ограничения выполняются.

Ребра, которые можно удалить из~$F$ без добавления новых активных компонентов, 
удаляются на втором этапе алгоритма. Следующая лемма определяет свойства 
компонент связ\-ности в~$F'$.


\smallskip

\noindent
\textbf{Лемма~2.}\
\textit{Для каждой компоненты связ\-ности~$N$ из~$F'$ выполняется равенство}: $f(N)\hm=0$.

\smallskip

Следующая теорема утверж\-да\-ет, что решение, полученное с~помощью описанного 
алгоритма, удовле\-тво\-ря\-ет ограничениям исходной задачи линейного 
программирования.

\smallskip

\noindent
\textbf{Теорема~1.}
\textit{Набор ребер $F'$, полученный алгоритмом~$1$, удовлетворяет всем 
ограничениям исходной задачи}~(\ref{ilp_cfp}).


\smallskip

\noindent
\textbf{Лемма~3.}\
\textit{Обозначим граф $H$, каждая вершина которого соответствует одной из компонент 
связ\-ности $C\in\mathcal{C}$ на фиксированном шаге алгоритма. Ребро $(v_1,v_2)$ 
присутствует, если существует ребро $\hat{e}$ исходного графа, входящее в~$F'$: 
$\hat{e} \in F'$, поэтому граф $H$~--- это лес. Внут\-ри $H$ нет листовых вершин, 
со\-от\-вет\-ст\-ву\-ющих неактивным вершинам исходного графа}.

\smallskip

\noindent
\textbf{Теорема 2.}
\textit{Алгоритм~$1$ представляет собой $\alpha$-при\-бли\-жен\-ный алгоритм для 
задачи}~(\ref{ilp_cfp}) \textit{с}~$\alpha \hm= 2 - {2}/{|A|}$, \textit{где} $A\hm=\{v\  V: 
f(\{v\})=1\}$.

\smallskip

Несмотря на эту теоретическую основу, не существует подходящей функции $f$ для 
постановки задачи PCST, указанной в~(\ref{ilp_cfp}). Чтобы быть 
применимым в~этих условиях, алгоритм~1 нуждается в~нескольких 
модификациях.

\vspace*{-9pt}

\section{Модифицированная постановка задачи для~PCST}

\vspace*{-3pt}

Как и~в случае A-PCST, упрощенный вид задачи линейного 
программирования PCST принимает вид:
\begin{multline*}
\underset{\substack{x_e,s_v \\ e\in E, v\in V\backslash \{r\}}}{\mbox{minimize}}\  
\displaystyle \sum\limits_{e\in E}c_ex_e + \sum\limits_{v\in V\backslash\{r\}} \left(1-s_v\right)\pi_v \\
\mbox{s.t.}\  \displaystyle \sum\limits_{e\in\delta(S)} \!\! x_e\geqslant s_v,\enskip S\subseteq V\backslash \{r\},\enskip v\in S,\\
x_e\geqslant 0,\enskip e\in E,\enskip s_v\geqslant 0,\enskip v\in V\backslash \{r\}.
%\label{rlp_pcst_inord}
\end{multline*}
Эта постановка задачи отличается от исходной~(\ref{ilp_pcst_ord}) тем, что с~ней 
возможно согласовать задачу $k$-MST. Индикаторы~$s_v$ показывают, что 
вершина~$v$ включена в~дерево.

Двойственная задача принимает вид:

\vspace*{-3pt}

\noindent
\begin{multline*}
\underset{\substack{y_S:~S\subset V\backslash\{r\}}}{\mbox{maximize}}\ 
\displaystyle \sum\limits_{S\in V\backslash\{r\}}y_S \\
\mbox{s.t.}\  \displaystyle \sum\limits_{S:e\in\delta(S)}y_S\leqslant c_e ,\enskip e\in E;\\
 \displaystyle \sum\limits_{S\subseteq T}y_S\leqslant \sum\limits_{v\in T}\pi_v,\enskip  T\subset  V\backslash\{r\},\\
 y_S\geqslant 0,\enskip  S\subset V\backslash\{r\}.
%\label{rd_pcst_inord}
\end{multline*}

\vspace*{-3pt}

Алгоритм~2 решает эту задачу. Он похож на 
алгоритм~1. Двойные переменные должны обновляться равномерно 
с~дополнительными ограничениями. Тогда~$\varepsilon$ примет минимальное из двух 
значений в~соответствии с~обеими группами ограничений.
Более широкий анализ аппроксимационных свойств обновленного алгоритма 
представлен в~\cite{goemans1995general}. Алгоритм~2 представляет 
собой $\alpha$-приближенный алгоритм для задачи PCST с~$\alpha \hm= 2 \hm- 
{2}/({n-1})$, где $n$~--- число вершин в~графе~$G$.

\vspace*{-9pt}

\section{Вычислительный эксперимент}

\vspace*{-3pt}

Основная цель эксперимента~--- восстановить дерево суперпозиции. Алгоритмы, 
используемые для восстановления, перечислены ниже.

\vspace*{-14pt}

\paragraph*{DFS, BFS.}
Алгоритмы жадного дерева обхода в~глубину и~жадного дерева обхода в~ширину. 
Обход ребер с~наибольшим весом эквивалентен выбору наиболее вероятного пути. 
Алгоритм обхода останавливается, когда число ребер, исходящих из некоторой 
вершины, становится равным арности соответствующей функции.

\vspace*{-14pt}

\paragraph*{Алгоритм Прима.}
Алгоритм восстанавливает минимальное остовное дерево для графа с~дополнительными 
ограничениями на арность базовых функций. Эти ограничения задают минимальный вес 
ребра. После добавления вершины все лис\-то\-вые ребра этой вершины исключаются, 
чтобы сохранить направление дерева. Если число ребер, начинающихся в~какой-либо 
вершине, превышает соответствующую арность, то остальные ребра исключаются из 
множества возможных ребер в~этой вершине. Алгоритм не зависит от процедуры 
обхода. В случае небольшого шума в~матрице смежности этот алгоритм способен 
восстановить дерево суперпозиции без ошибок. 


\vspace*{-14pt}

\paragraph*{Алгоритмы на основе PCST.}
Матрица смеж\-ности~$M$ должна быть приведена к~неориентиро-\linebreak\vspace*{-12pt}

\pagebreak

\noindent
ванному виду. 
Использована квад\-рат\-ная мат\-ри\-ца~$M'$ без последнего столбца. PCST 
принимает мат\-ри\-цу смеж\-ности $1 \hm- ({1}/{2})(M' \hm+ M'^{\mathsf{T}})$ с~призовым 
значением~0,5 для каж\-дой вершины.
Призовое значение рав\-но~0,5, поскольку при меньших значениях дерево будет 
обрезано: если шум равен~0,5, некоторые вершины могут быть обрезаны по ошибке. 
В~случае больших призовых значений
дерево PCST может содержать ненужные 
вершины. Дерево восстанавливается по одному из опи-\linebreak\vspace*{-12pt}

{ \begin{center}  %fig2
 \vspace*{9pt}
    \mbox{%
\epsfxsize=79mm
\epsfbox{str-2.eps}
}
\end{center}



\noindent
{{\figurename~2}\ \ \small{Качество алгоритмов восстановления с~базовыми функциями небольших 
арностей: \textit{1}~--- DFS; \textit{2}~--- BFS; \textit{3}~--- алгоритм Прима;
\textit{4}~--- $k$-MST; \textit{5}~--- $k$-MST--DFS; \textit{6}~--- $h$-MST--BFS; \textit{7}~--- $k$-MST\,--\,ал\-го\-ритм Прима
}}}

\vspace*{6pt}

\addtocounter{figure}{1}

%\begin{table*}\small  %tabl2
\begin{center}
\parbox{75mm}{{{\tablename~2}\ \ \small{Качество алгоритмов реконструкции с~равномерным шумом, близким 
к~0,5
}}
}
    
    
\vspace*{6pt}

  {\small  \begin{tabular}{|l|ccccc|}
      \hline
                  & \multicolumn{5}{c|}{Шум}\\%& & Шум & & \\
       \cline{2-6}
        \multicolumn{1}{|c|}{\raisebox{6pt}[0pt][0pt]{Алгоритм}}                          
&0,50&0,52&0,54&0,56&0,58\\
                    \hline
      DFS        &0,20 &0,20 &0,19 &0,18 &0,16\\
      BFS        &0,60 &0,58 &0,51 &0,46 &0,40\\
      Прима    &1,00 &0,94&0,81&0,69&0,57\\
      $k$-MST     &0,17 &0,16 &0,14 &0,12 &0,10\\
      $k$-MST--DFS   &0,17 &0,16 &0,16 &0,14 &0,14 \\
      $k$-MST--BFS   &0,43 &0,40 &0,36 &0,33 &0,29 \\
      $k$-MST--Прима  &0,44 &0,39 &0,34 &0,33 &0,27 \\
      \hline
    \end{tabular}
    }
\end{center}
%\end{table*}




\noindent
 санных алгоритмов. Результаты 
$\text{PCST}$ можно использовать в~качестве априорных для других подходов, $M':=({1}/{2})(M_{\mathrm{PCST}}' + M')$,
поэтому результаты \mbox{PCST} обновляются~$M'$.


Процедура генерации данных имеет следующие допущения: арности функций 
генерируются биномиальным распределением, поэтому существуют много функций 
с~малой арностью, все базовые функции имеют только один вход. Любой случай 
с~частичной реконструкцией считается ошибкой. Качество алгоритмов реконструкции:
$$
\fr{1}{K}\sum\limits_{k=1}^K \left[ R\left( \bar{N}(M_k)\right)=M_k\right],
$$
где~$R$ ~--- алгоритм реконструкции;
$\bar{N}\hm=\left(N - \min(N)\right)/\left(\max(N)\hm-\min(N)\right)$~--- нормированная мат\-ри\-ца шума. 
Мат\-ри\-ца~$N$ генерируется как~$N(M)\hm=M\hm+U(-\alpha,\alpha)$.
Генератор случайных чисел возвращает матрицу того же вида, что и~$M$, где каждый 
элемент является независимой переменной из равномерного распределения 
в~сегменте~$[-\alpha,\alpha]$.

Вот список из семи сравниваемых алгоритмов:
DFS,
BFS,
алгоритм Прима,
$k$-MST через PCST,
$k$-MST\;+\;DFS,
$k$-MST\;+\;BFS,
$k$-MST\;+\;ал\-го\-ритм Прима.
На рис.~2 показана ошибка алгоритмов реконструкции 
с~шумом, близ\-ким к~порогу~0,5. Наилучшие результаты дает алгоритм Прима. Второе по 
точности решение основано на~$\text{BFS}$. Таб\-ли\-ца~2 
соответствует~рис.~2 и~показывает качество реконструкции 
семи алгоритмов для значений граничного шума~0,50--0,58.





\vspace*{-9pt}

\section{Заключение}

\vspace*{-3pt}

Предлагаются и~сравниваются  алгоритмы вос\-ста\-нов\-ле\-ния суперпозиции для задачи 
символьной регрессии. Алгоритм Прима дает наиболее точ\-ные результаты и~устойчив 
к~небольшому шуму в~данных. Пред\-ла\-га\-емый алгоритм дает точные результаты, но он 
более подвержен шуму в~мат\-ри\-це суперпозиции. Алгоритмы, основанные на BFS и~DFS, 
не могут вос\-ста\-но\-вить исходную суперпозицию с~зашумленными мат\-ри\-ца\-ми 
суперпозиции. Алгоритм PCST с~BFS, используемый для реконструкции мат\-ри\-цы 
суперпозиции, показывает приемлемые для практического использования результаты.

{\small\frenchspacing
 {%\baselineskip=10.8pt
 %\addcontentsline{toc}{section}{References}
 \begin{thebibliography}{99}
\bibitem{koza1992genetic}  %1
\Au{Koza J.\,R.} Genetic programming as a means for programming computers by 
natural selection~// Stat. Comput., 1994. Vol.~4. P.~87--112.

\bibitem{searson2010gptips} %2
\Au{Searson~D.\,P., Leahy~D.\,E., Willis~M.\,J.} GPTIPS: An open source 
genetic programming toolbox for multigene  symbolic regression~// 
Multiconference (International) of Engineers and Computer Scientists Proceedings, 
2010. Vol.~1. P.~77--80.

\bibitem{stanley2002evolving} %3
\Au{Stanley~K.\,O., Miikkulainen~R.} Evolving neural networks through 
augmenting topologies~// Evol. Comput., 2002. Vol.~10. 
Iss.~2. P.~99--127.

\bibitem{bochkarev2017generation}
\Au{Бочкарев~А.\,М., Софронов~И.\,Л., Стрижов~В.\,В.} По\-рож\-де\-ние экс\-перт\-но-ин\-тер\-пре\-ти\-ру\-емых 
моделей для прогноза проницаемости горной породы~// Системы и~средства информатики, 2017. Т.~27. №\,3. С.~74--87.
%

\bibitem{ravi1996spanning}
\Au{Ravi~R., Sundaram~R., Marathe~M.\,V., Rosenkrantz~D.\,J., Ravi~S.\,S.} 
Spanning trees~--- short or small~// SIAM J.~Discrete Math., 
1996. Vol.~9. Iss.~2. P.~178--200.

\bibitem{chudak2004approximate}
\Au{Chudak~F.\,A.,  Roughgarden~T., Williamson~D.\,P.} Approximate $k$-MSTS 
and $k$-Steiner trees via the primal-dual method and Lagrangean 
relaxation~// Math. Program., 2004. Vol.~100. Iss.~2. P.~411--421.

\bibitem{awerbuch1998new}
\Au{Awerbuch~B., Azar~Y., Blum~A., Vempala~S.} New approximation guarantees 
for minimum-weight $k$-trees and prize-collecting salesmen~// SIAM J. 
Comput., 1998. Vol.~28. Iss.~1. P.~254--262.

\bibitem{arora20062+}
\Au{Aror~S., Karakostas~G.} A~$2+\varepsilon$ approximation algorithm for the 
$k$-MST problem~// Math. Program., 2006. Vol.~107. 
Iss.~3. P.~491--504.

\bibitem{hegde2014fast}
\Au{Hegde~C., Indyk~P., Schmidt~L.} A~fast, adaptive variant of the 
Goemans--Williamson scheme for the prize-collecting steiner tree problem~// 11th DIMACS Implementation Challenge Workshop Proceedings, 2014. P.~1--32.
{\sf http://people. csail.mit.edu/ludwigs/papers/dimacs14\_fastpcst.pdf}.

\bibitem{ras2017approximate}
\Au{Ras~C., Swanepoel~K., Thomas~D.\,A.} Approximate Euclidean Steiner 
trees~// J.~Optimiz. Theory App., 2017. Vol.~172. 
Iss.~3. P.~845--873.

\bibitem{goemans1995general}
\Au{Goemans~M.\,X., Williamson~D.\,P.} A~general approximation technique for 
constrained forest problems~// SIAM J. Comput., 1995. Vol.~24. 
Iss.~2. P.~296--317.
\end{thebibliography}

 }
 }

\end{multicols}

\vspace*{-6pt}

\hfill{\small\textit{Поступила в~редакцию 23.01.22}}

\vspace*{8pt}

%\pagebreak

%\newpage

%\vspace*{-28pt}

\hrule

\vspace*{2pt}

\hrule

%\vspace*{-2pt}

\def\tit{OPTIMAL SPANNING TREE RECONSTRUCTION IN~SYMBOLIC~REGRESSION}


\def\titkol{Optimal spanning tree reconstruction in~symbolic regression}


\def\aut{R.\,G.~Neychev$^1$, I.\,A.~Shibaev$^1$, and~V.\,V.~Strijov$^2$}

\def\autkol{R.\,G.~Neychev, I.\,A.~Shibaev, and~V.\,V.~Strijov}

\titel{\tit}{\aut}{\autkol}{\titkol}

\vspace*{-8pt}


\noindent
$^1$Moscow Institute of Physics and Technology, 9~Institutskiy Per., Dolgoprudny, Moscow Region 141700, Russian\linebreak
$\hphantom{^1}$Federation

\noindent
$^2$Federal Research Center ``Computer Science and Control'' of the Russian Academy of Sciences, 44-2~Vavilov Str.,\linebreak
$\hphantom{^1}$Moscow 119333, Russian Federation

\def\leftfootline{\small{\textbf{\thepage}
\hfill INFORMATIKA I EE PRIMENENIYA~--- INFORMATICS AND
APPLICATIONS\ \ \ 2023\ \ \ volume~17\ \ \ issue\ 1}
}%
 \def\rightfootline{\small{INFORMATIKA I EE PRIMENENIYA~---
INFORMATICS AND APPLICATIONS\ \ \ 2023\ \ \ volume~17\ \ \ issue\ 1
\hfill \textbf{\thepage}}}

\vspace*{3pt} 



\Abste{The paper investigates the problem of regression model generation. A~model is a~superposition of primitive functions. 
The model structure is described by a~weighted colored graph. Each graph vertex corresponds to a~primitive function. 
An edge assigns a~superposition of two functions. The weight of an edge is equal to the probability of superposition. 
To generate an optimal model, one has to reconstruct its structure from its graph adjacency matrix. 
The proposed algorithm reconstructs the minimum spanning tree from the weighted colored graph. 
The paper presents a~novel solution based on the prize-collecting Steiner tree algorithm. This algorithm is compared with its alternatives.}


\KWE{symbolic regression; linear programming; superposition; minimum spanning tree; adjacency matrix}



\DOI{10.14357/19922264230105} 

\vspace*{-16pt}

\Ack

\vspace*{-3pt}


\noindent
This work was supported by the Russian Foundation for Basic Research, projects 20-37-90050 and 20-07-00990.
  

\vspace*{6pt}

  \begin{multicols}{2}

\renewcommand{\bibname}{\protect\rmfamily References}
%\renewcommand{\bibname}{\large\protect\rm References}

{\small\frenchspacing
 {%\baselineskip=10.8pt
 \addcontentsline{toc}{section}{References}
 \begin{thebibliography}{99} 

\bibitem{1-str}
\Aue{Koza, J.\,R.}
 1994. Genetic programming as a means for programming computers by natural selection. \textit{Stat. Comput.} 4:87--112.

\bibitem{2-str}
\Aue{Searson, D.\,P., D.\,E.~Leahy, and M.\,J.~Willis.}
 2010. \mbox{GPTIPS}: An open source genetic programming toolbox for multigene symbolic regression. 
 \textit{Multiconference (International) of Engineers and Computer Scientists Proceedings}. 1:77--80. 

\bibitem{3-str}
\Aue{Stanley, K.\,O., and R.~Miikkulainen.} 2002. Evolving neural networks through augmenting topologies. 
\textit{Evol. Comput.} 10(2):99--127.

\bibitem{4-str}
\Aue{Bochkarev, A.\,M., I.\,L.~Sofronov, and V.\,V.~Strijov.}
 2017. Po\-rozh\-de\-nie eks\-pert\-no-inter\-pre\-ti\-ru\-emykh mo\-de\-ley dlya prog\-no\-za pro\-ni\-tsa\-emosti gor\-noy po\-ro\-dy 
 [Generation of expertly-interpreted models for prediction of core permeability]. \textit{Sistemy i~Sredstva Informatiki~--- Systems and Means of Informatics}
  27(3):74--87.

\bibitem{5-str}
\Aue{Ravi, R., R.~Sundaram, M.\,V.~Marathe, D.\,J.~Rosenkrantz, and S.\,S.~Ravi.}
 1996. Spanning trees~--- short or small. \textit{SIAM J. Discrete Math.} 9(2):178--200.

\bibitem{6-str}
\Aue{Chudak, F.\,A., T.~Roughgarden, and D.\,P.~Williamson.}
 2004. Approximate k-MSTS and k-Steiner trees via the primal-dual method and Lagrangean relaxation. 
 \textit{Math. Program.} 100(2):411--421.

\bibitem{7-str}
\Aue{Awerbuch, B., Y.~Azar, A.~Blum, and S.~Vempala.}
 1998. New approximation guarantees for minimum-weight \mbox{k-trees} and prize-collecting salesmen.
 \textit{SIAM J. Comput.} 28(1):254--262.

\bibitem{8-str}
\Aue{Arora, S., and G.~Karakostas.} 2006. A~$2+\varepsilon$ approximation algorithm for the $k$-MST problem. 
\textit{Math. Program.} 107(3):491--504.

\bibitem{9-str}
\Aue{Hegde, C., P.~Indyk, and L.~Schmidt.} 2014. 
A~fast, adaptive variant of the Goemans--Williamson scheme for the prize-collecting Steiner tree problem. 
\textit{11th DIMACS Implementation Challenge Workshop Proceedings}. 1--32.
Available at: 
{\sf http://people.csail.mit.edu/ludwigs/papers/\linebreak dimacs14\_fastpcst.pdf} (accessed January~10, 2023).

\bibitem{10-str}
\Aue{Ras, C., K.~Swanepoel, and D.\,A.~Thomas.} 
2017. Approximate Euclidean Steiner trees. \textit{J.~Optimiz. Theory  App.} 172(3):845--873.

\bibitem{11-str}
\Aue{Goemans, M.\,X., and D.\,P.~Williamson.} 1995. 
A~general approximation technique for constrained forest problems. \textit{SIAM J. Comput.} 24(2):296--317.
 \end{thebibliography}

 }
 }

\end{multicols}

\vspace*{-6pt}

\hfill{\small\textit{Received January 23, 2022}}

\Contr

\noindent
\textbf{Neychev Radoslav G.} (b.\ 1994)~--- 
PhD student, Moscow Institute of Physics and Technology, 9~Institutskiy Per., Dolgoprudny, Moscow Region 141701, Russian Federation;
\mbox{neychev@phystech.edu}

\vspace*{3pt}

\noindent
\textbf{Shibaev Innokentii A.} (b.\ 1997)~--- 
PhD student, Moscow Institute of Physics and Technology, 9~Institutskiy Per., Dolgoprudny, Moscow Region 141701, Russian Federation; 
\mbox{shibaev.kesha@gmail.com}

\vspace*{3pt}

\noindent
\textbf{Strijov Vadim V.} (b.\ 1967)~--- 
Doctor of Science in physics and mathematics, leading scientist, A.\,A.~Dorodnicyn Computing Center, 
Federal Research Center ``Computer Science and Control'' of the Russian Academy of Sciences, 40~Vavilov Str., Moscow 119333, Russian Federation;
\mbox{strijov@phystech.edu}


\label{end\stat}

\renewcommand{\bibname}{\protect\rm Литература}  %6
\def\stat{shestakov}

\def\tit{ОБРАЩЕНИЕ ОДНОРОДНЫХ ОПЕРАТОРОВ С~ПОМОЩЬЮ
СТАБИЛИЗИРОВАННОЙ ЖЕСТКОЙ ПОРОГОВОЙ ОБРАБОТКИ
ПРИ~НЕИЗВЕСТНОЙ ДИСПЕРСИИ ШУМА$^*$}

\def\titkol{Обращение однородных операторов с~помощью
стабилизированной жесткой пороговой обработки}
%при~неизвестной дисперсии шума}

\def\aut{О.\,В.~Шестаков$^1$}

\def\autkol{О.\,В.~Шестаков}

\titel{\tit}{\aut}{\autkol}{\titkol}

\index{Шестаков О.\,В.}
\index{Shestakov O.\,V.}


{\renewcommand{\thefootnote}{\fnsymbol{footnote}} \footnotetext[1]
{Работа выполнена при частичной финансовой поддержке РФФИ (проект 19-07-00352).}}


\renewcommand{\thefootnote}{\arabic{footnote}}
\footnotetext[1]{Московский государственный университет им.\ М.\,В.~Ломоносова, 
кафедра математической статистики факультета вычислительной математики и~кибернетики; 
Институт проб\-лем информатики Федерального исследовательского центра 
<<Информатика и~управ\-ле\-ние>> Российской академии наук, \mbox{oshestakov@cs.msu.su}}


\vspace*{-6pt}


\Abst{При обращении линейных однородных операторов обычно необходимо использовать 
методы регуляризации, поскольку наблюдаемые данные, как правило, зашумлены. 
Для подавления шума часто используется пороговая обработка 
вейвлет-ко\-эф\-фи\-ци\-ен\-тов функции наблюдаемого сигнала. 
Пороговая обработка стала популярным инструментом подавления 
шума благодаря своей простоте, вы\-чис\-ли\-тель\-ной эффективности и~воз\-мож\-ности 
адаптации к~функциям, имеющим на разных участках разную степень регулярности. 
Рассматривается предложенный недавно стабилизированный метод жесткой 
пороговой обработки, в~котором устранены основные недостатки мягкой и~жесткой 
пороговой обработки, и~исследуются статистические свойства этого метода. 
В~модели данных с~аддитивным гауссовским шумом с~неизвестной дисперсией 
проведен анализ несмещенной оценки среднеквадратичного риска и~показано, 
что при определенных условиях данная оценка является асимптотически нормальной, 
при этом дисперсия предельного распределения зависит от способа оценивания 
дисперсии шума.}

\KW{вейвлеты; пороговая обработка; несмещенная оценка риска; 
асимптотическая нормальность; сильная состоятельность}

\DOI{10.14357/19922264190107}
  
%\vspace*{4pt}


\vskip 10pt plus 9pt minus 6pt

\thispagestyle{headings}

\begin{multicols}{2}

\label{st\stat}

\section{Введение}

В медицинских, физических, астрономических и~других научных проблемах часто 
возникает задача получить представление об объекте, который описывается 
некоторой функцией~$f$, имея возможность наблюдать только функцию~$Kf$, где~$K$~--- 
некоторый линейный оператор. При этом часто нельзя просто применить 
к~наблюдаемым данным обратный оператор~$K^{-1}$, поскольку эти данные, как правило, 
содержат шум и~задача обращения оператора~$K$ некорректно поставлена. 
К~тому же обычно дис\-пер\-сия шума неизвестна и~ее необходимо оценивать 
по наблюдаемым данным. 

Одним из популярных инструментов при регуляризации 
процедуры обращения служит вейв\-лет-раз\-ло\-же\-ние с~последующей 
пороговой обработкой вейв\-лет-ко\-эф\-фи\-ци\-ен\-тов. Наиболее распростра\-нен\-ные 
виды пороговой обработки~--- жесткая и~мягкая. В~работе~\cite{HL10} 
был предложен метод стабилизированной жесткой пороговой обработки, который 
объединяет в~себе преимущества этих двух видов. 
В~ситуации, когда дисперсия шума предполагается известной, в~работе~\cite{SH18} 
доказана асимптотическая нормальность оценки среднеквадратичного риска пороговой 
обработки. 

В~данной работе исследуется влияние способов оценивания дисперсии шума 
на характеристики предельного распределения оценки среднеквадратичного риска. 
Для метода мягкой пороговой обработки подобные исследования проводились 
в~работах~\cite{KS11-1, KS11-2}.

\section{Обращение линейных однородных операторов с~помощью вейглет-вейвлет-разложения}

В данной работе рассматривается метод обращения линейных однородных операторов, 
основанный на вейг\-лет-вейв\-лет-раз\-ло\-же\-нии~\cite{AS98}. Линейный оператор~$K$ 
называется однородным, если
$$
K\left[f\left(a\left(x-x_0\right)\right)\right]=a^{-\alpha}(Kf)\left[a\left(x-x_0\right)\right]
$$
для любого $x_0$ и~любого $a\hm>0$. Параметр~$\alpha$ называется показателем 
однородности. Примерами линейных однородных операторов служат оператор 
интегрирования, преобразование Гильберта и~преобразование Абеля.

Относительно наблюдаемой функции~$Kf$ будем предполагать, что она определена на 
конечном отрезке и~равномерно регулярна по Липшицу с~некоторым показателем $\gamma\hm>0$. 
Вейв\-лет-разложение~$Kf$ представляет собой ряд по ортонормированному базису
\begin{equation}
\label{wavelet_decomp}
Kf = \sum\limits_{j,k \in Z} \langle Kf,\psi_{j,k} \rangle \psi_{j,k}\,,
\end{equation}
где $\psi(t)$~--- некоторая материнская вейв\-лет-функ\-ция, 
а~$\psi_{j,k}(t) \hm= 2^{j/2}\psi(2^jt \hm- k)$. Индекс~$j$ в~(\ref{wavelet_decomp}) 
называется масштабом, а~индекс~$k$~--- сдвигом. Если вейв\-лет-функ\-ция 
обладает определенными свойствами регулярности~\cite{Mal99}, 
то для коэффициентов разложения в~(\ref{wavelet_decomp}) справедливо
\begin{equation}
\label{wavelet_decay}
\abs{\langle Kf, \psi_{j,k} \rangle} \leqslant \fr{C_f}
{2^{j \left( \gamma + 1/2 \right)}}\,,
\end{equation}
где $C_f$~--- некоторая положительная константа.

Поскольку оператор~$K$ линеен и~однороден, существуют такие функции~$u_{j,k}$, 
что $\langle f,u_{j,k}\rangle\hm=\langle Kf,\psi_{j,k}\rangle$. При этом функция~$f$ 
представляется в~виде ряда
\begin{equation}
\label{VWD}
f = \sum\limits_{j,k \in Z}\beta_{j,k}\langle Kf,\psi_{j,k}\rangle u_{j,k},
\end{equation}
где $u_{j,k} = K^{-1}\psi_{j,k}/\beta_{j,k}$, $\beta_{j,k}\hm=2^{\alpha j}\beta_{00}$, 
$\beta_{00} \hm= \norm{K^{-1}\psi}$ (функции~$u_{j,k}$, как и~$\psi_{j,k}$, 
представляют собой сдвиги и~растяжения одной материнской функции~$u$ и~называются 
вейглетами). При соответствующем выборе~$\psi(t)$ последовательность~$\{u_{j,k}\}$ 
образует устойчивый базис~\cite{L97}. Формула~(\ref{VWD}) и~есть основа метода 
вейг\-лет-вейв\-лет-раз\-ло\-же\-ния.

\section*{Пороговая обработка эмпирических коэффициентов}

При фактических измерениях значения функции сигнала регистрируются 
в~дискретных отсчетах, при этом такие значения, как правило, зашумлены. 
Рассмотрим сле\-ду\-ющую модель данных \mbox{с~шумом}:
\begin{equation*}
%\label{Data_Model}
X_i = (Kf)_i + \epsilon_i\,, \enskip i = 1, \dots, 2^J\,, %\notag
\end{equation*}
где $2^J$~--- число отсчетов; $(Kf)_i$~--- незашумленные значения функции сигнала; 
$\epsilon_i$~--- независимые нормально распределенные случайные величины с~нулевым 
средним и~дисперсией~$\sigma^2$.
После применения дискретного вейв\-лет-пре\-об\-ра\-зо\-ва\-ния 
получается следующая модель зашумленных вейв\-лет-ко\-эф\-фи\-ци\-ен\-тов:
\begin{equation*}
Y_{j,k}=\mu_{j,k}+\epsilon^W_{j,k},\enskip 
j=0,\ldots,J-1,\ k=0,\ldots,2^{j}-1\,,
\end{equation*}
где $\epsilon^W_{j,k}$ независимы и~распределены так же, как и~$\epsilon_i$, 
а~$\mu_{j,k}\hm= 2^{J/2}\langle Kf,\psi_{j,k}\rangle$~\cite{Mal99}.

Для подавления шума и~построения оценки функции сигнала к~коэффициентам~$Y_{j,k}$ 
обычно применяется функция жесткой пороговой обработки 
$\rho_{H}(y,T)\hm=x\textbf{I}(\abs{y}>T)$ или мягкой пороговой 
обработки $\rho_{S}(y,T)\hm=\textbf{sgn}(x)\left(\abs{y}-T\right)_{+}$ 
с~порогом~$T$. При таком подходе обнуляются коэффициенты, абсолютная величина 
которых ниже порога, так как в~силу~(\ref{wavelet_decay}) основная часть
 полезного сигнала содержится в~относительно небольшом числе больших по 
 модулю коэффициентов.

Каждому из этих видов пороговой обработки присущи свои недостатки. 
Жесткая пороговая функция разрывна, и~это приводит к~отсутствию устойчивости 
при выборе порога~\cite{B96} и~невозможности построения несмещенной оценки 
среднеквадратичного риска~\cite{J01}. При мягкой пороговой обработке в~оценке 
функции появляется дополнительное смещение. Чтобы частично избежать этих недостатков, 
в~работе~\cite{HL10} был предложен новый вид пороговой обработки, представляющий 
собой сглаженный (стабилизированный) аналог жесткой пороговой обработки. 
В~этом методе оценки~$\mu_{j,k}$ вычисляются по формулам:
\begin{equation*}
\widehat{\mu}_{j,k}=\Expect 
\left[\rho_{H}(Y_{j,k}+\lambda\xi_{j,k},T_j)|Y_{j,k}\right], %\notag
\end{equation*}
где случайные величины~$\xi_{j,k}$ имеют стандартное нормальное распределение и~не 
зависят от~$Y_{j,k}$, а~$\lambda\hm>0$~--- 
параметр стабилизации, отвечающий за степень сглаживания. Вычисляя математическое 
ожидание, получаем:
\begin{multline*}
\hspace*{-8.37947pt}\widehat{\mu}_{j,k}=Y_{j,k}\left[\Phi\!\left(-\fr{T_j+Y_{j,k}}
{\lambda}\right)+1-\Phi\left(\fr{T_j-Y_{j,k}}{\lambda}\right)\!\right]+{}\\
{}+
\lambda\left[\phi\left(\fr{T_j-Y_{j,k}}{\lambda}\right)-
\phi\left(\fr{T_j+Y_{j,k}}{\lambda}\right)\right]. %\notag
\end{multline*}
Достоинством такого метода является бесконечная дифференцируемость~$\widehat{\mu}_{j,k}$ 
по~$Y_{j,k}$, что приводит к~более робастным оценкам~\cite{HL10}. Заметим также, 
что при $\lambda\hm\to0$ получается обычный метод жесткой пороговой обработки. 
В~данной работе параметр~$\lambda$ предполагается фиксированным, а~в~качестве~$T_j$ 
для каждого масштаба~$j$ выбирается порог $T_j\hm=\sigma\sqrt{2\ln 2^j}$. 
Такой порог получил название <<универсальный>>, так как он не зависит 
от наблюдаемых данных. И~при жесткой, и~при мягкой пороговой обработке этот 
порог обеспечивает близость среднеквадратичного риска к~минимальному~\cite{Mal99}.

\section{Несмещенная оценка среднеквадратичного риска}

Среднеквадратичный риск метода пороговой обработки определяется по формуле:
\begin{equation}
\label{Risk}
R_J(\sigma)=\sum\limits_{j=0}^{J-1}\sum\limits_{k=0}^{2^j-1}\beta^2_{j,k}
\Expect\left(\widehat{\mu}_{j,k}(\sigma)-\mu_{j,k}\right)^2.
\end{equation}
В~\cite{HL10} показано, что при стабилизированной жесткой пороговой обработке
\begin{multline*}
\Expect\left(\widehat{\mu}_{j,k}(\sigma)-\mu_{j,k}\right)^2={}\\
{}=
\Expect\left[(Y_{j,k}-\widehat{\mu}_{j,k}(\sigma))^2+
2\sigma^2\fr{\partial}{\partial Y_{j,k}}\,\widehat{\mu}_{j,k}(\sigma)\right]-
\sigma^2, %\notag
\end{multline*}
где
\begin{multline*}
\fr{\partial}{\partial Y_{j,k}}\widehat{\mu}_{j,k}(\sigma)={}\\
{}=\Phi\left(-\fr{T_j+Y_{j,k}}{\lambda}\right)+1-
\Phi\left(\fr{T_j-Y_{j,k}}{\lambda}\right)+{}\\
{}+
\fr{T_j}{\lambda}\left[\phi\left(\fr{T_j-Y_{j,k}}{\lambda}\right)+
\phi\left(\fr{T_j+Y_{j,k}}{\lambda}\right)\right]. %\notag
\end{multline*}
Таким образом, величина
\begin{multline}
\label{Risk_Estimate}
\widehat{R}_J(\sigma)=\sum\limits_{j=0}^{J-1}\sum\limits_{k=0}^{2^j-1}
\beta^2_{j,k}
\Bigg[
\left(
Y_{j,k}-
\widehat{\mu}_{j,k}(\sigma)\right)^2+{}\\
{}+2\sigma^2\fr{\partial}{\partial Y_{j,k}}\,\widehat{\mu}_{j,k}(\sigma)-
\sigma^2
\Bigg]
\end{multline}
является несмещенной оценкой~$R_J$, не зависящей от ненаблюдаемых значений~$\mu_{j,k}$.

В работе~\cite{SH18} доказано следующее утверждение, устанавливающее 
асимптотическую нормальность оценки~(\ref{Risk_Estimate}) и~позволяющее строить 
асимптотические доверительные интервалы для риска~(\ref{Risk}).

\smallskip

\noindent
\textbf{Теорема 1.} 
\textit{Пусть $K$~--- линейный однородный оператор с~показателем 
однородности $\alpha\hm>0$, а~$Kf$ задана на конечном отрезке и~равномерно 
регулярна по Липшицу с~показателем $\gamma\hm>0$. Тогда}
\begin{equation*}
%\label{Normality}
{\sf P}\left(\fr{\widehat{R}_J(\sigma)-
R_J(\sigma)}{D_J}<x\right)\Rightarrow\Phi(x)\,, %\notag
\end{equation*}
\textit{где}
$$
D^2_J=\fr{2\sigma^4\beta_{0,0}^4}{2^{4\alpha+1}-1}2^{(4\alpha+1)J}\,.
$$

\section{Виды оценок дисперсии шума}

Как правило, дисперсия~$\sigma^2$ неизвестна и~вместо ее точного значения 
необходимо использовать некоторую оценку~$\hat{\sigma}^2$, которая обычно 
строится по половине всех вейв\-лет-ко\-эф\-фи\-ци\-ен\-тов для $j\hm=J\hm-1$, 
так как в~силу~(\ref{wavelet_decay}) эти коэффициенты фактически содержат только шум. 
При этом порог вычисляется по формуле $\hat{T}_j\hm=\hat{\sigma}\sqrt{2\ln 2^j}$.

В качестве оценки~$\sigma^2$ (или $\sigma$) в~данной работе 
рассматривается выборочная дисперсия
\begin{equation}
\label{SampleVarianceDef}
\widehat{\sigma}_S^2=\fr{1}{2^{J-1}}
\sum\limits_{k=0}^{2^{J-1}-1}Y_{J-1,k}^2-\overline{Y}^2,
\end{equation}
где
\begin{equation*}
\overline{Y}=\fr{1}{2^{J-1}}\sum\limits_{k=0}^{2^{J-1}-1}Y_{J-1,k}\,,
\end{equation*}
а также соответствующим образом нормированный выборочный интерквартильный 
размах~$\widehat{\sigma}_{R}$ и~выборочное абсолютное медианное 
отклонение~$\widehat{\sigma}_{M}$, которые определяются сле\-ду\-ющим образом:
\begin{align}
\widehat{\sigma}_{R}&=\fr{Y_{(J-1,3/4)}-Y_{(J-1,1/4)}}{2\xi_{3/4}}\,;
\label{IQR_Definition}
\\
\widehat{\sigma}_{M}&=\fr{\mathop{\mbox{med}}\limits_{0\leqslant k\leqslant 2^{J-1}-1}|Y_{J-1,k}-\mathop{\mbox{med}}\limits_{0\leqslant l\leqslant 2^{J-1}-1} Y_{J-1,l}|}{\xi_{3/4}}\,.
\label{MAD_Definition}
\end{align}
Здесь $Y_{(J-1,1/4)}$ и~$Y_{(J-1,3/4)}$~--- выборочные квантили порядка~$1/4$ и~$3/4$, 
построенные по выборке из половины всех вейв\-лет-ко\-эф\-фи\-ци\-ен\-тов при 
$j\hm=J\hm-1$; $\xi_{3/4}$~--- теоретическая квантиль порядка~$3/4$ 
стандартного нормального распределения ($\xi_{3/4}\hm\approx0,6745$); $\mbox{med}$ 
обозначает выборочную медиану.

Выборочная дисперсия служит самой популярной оценкой величины~$\sigma^2$, и~в~случае 
отсутствия выбросов она наиболее предпочтительна. Однако в~случае, когда 
оценка дисперсии строится по выборке сигнала, естественно ожидать, 
что выборка не будет однородной. Преимущество использования последних 
двух оценок заключается в~их ро\-баст\-ности, т.\,е.\ нечувствительности к~выбросам.

\section{Предельная дисперсия оценки среднеквадратичного риска}

Способ оценивания дисперсии шума влияет на вид предельной дисперсии 
оценки среднеквадратичного риска. Подобный эффект наблюдается и~при 
мягкой пороговой обработке~[4].

\noindent
\textbf{Теорема~2.}\ \textit{Пусть $Kf$ задана на конечном отрезке и~равномерно 
регулярна по Липшицу с~показателем $\gamma\hm>1/4$, а оценка дисперсии 
шума задана соотношением}~\eqref{SampleVarianceDef}. \textit{Тогда}
\begin{equation}
\label{CLT_Operator_SampleVar_Sigma}
\mathsf{P}\left(\frac{\widehat{R}_J(\widehat{\sigma}_S)-R_J(\sigma)}{D_J}<x\right)
\Rightarrow \Phi_{\Upsilon_1}(x),\notag
\end{equation}
\textit{где $\Phi_{\Upsilon_1}(x)$~--- функция распределения нормального 
закона с~нулевым средним и~дисперсией}
$$
\Upsilon_1^2=\fr{1}{2^{4\alpha+1}}+
\fr{2^{4\alpha+1}-1}{2^{4\alpha+1}\left(2^{2\alpha+1}-1\right)^2}\,.
$$

\noindent
Д\,о\,к\,а\,з\,а\,т\,е\,л\,ь\,с\,т\,в\,о\,.\ \ Обозначим
\begin{multline*}
\widehat{U}_J(\sigma)=\sum\limits_{j=0}^{J-1}\sum\limits_{k=0}^{2^j-1}
\beta^2_{j,k}\Bigg[
\left(Y_{j,k}-\widehat{\mu}_{j,k}(\sigma)\right)^2+{}\\
{}+2\sigma^2\fr{\partial}{\partial Y_{j,k}}\widehat{\mu}_{j,k}(\sigma)\Bigg] %\notag
\end{multline*}
и запишем $\widehat{R}_J(\hat{\sigma}_S)-R_J(\sigma)$ в~виде
\begin{multline*}
%\label{Three_Sums}
\widehat{R}_J(\hat{\sigma}_S)-R_J(\sigma)={}\\
{}=\left[\widehat{U}_J(\hat{\sigma}_S)-\widehat{U}_J(\sigma)\right]+
\left[\widehat{R}_J(\sigma)-R_J(\sigma)\right]+{}\\
{}+
\fr{2^{(2\alpha+1)J}-1}{2^{2\alpha+1}-1}(\sigma^2-\hat{\sigma}^2_S)
\equiv S_1+S_2+S_3\,.
\end{multline*}

Повторяя рассуждения из работ~\cite{KS11-1, KS11-2} и~учитывая, что если $\gamma\hm>1/4$, 
то выполнено $2^{J/2}\overline{Y}^2\stackrel{{\sf P}}{\to} 0$ при 
$J\hm\rightarrow\infty$~\cite{KS11-2}, можно показать, что
\begin{equation*}
{\sf P}\left(\fr{S_2+S_3}{D_J}<x\right)\Rightarrow\Phi_{\Upsilon_1}(x)\,.%\notag
\end{equation*}
% на самом деле с~условием Линдеберга чуть по-другому (без ограниченности слагаемых). Но дисперсия равномерно ограничена -- значит выполнено.

Докажем, что $D_J^{-1}S_1\stackrel{{\sf P}}{\to}0$ при $J\hm\rightarrow\infty$. 
Пусть $C_\delta\hm>0$~--- некоторая константа, а $\delta_J\hm=C_\delta J^{1/2}2^{-J/2}$. 
Запишем
\begin{multline*}
S_1=\mathbf{1}\left(\abs{\sigma^2-\hat{\sigma}^2_S}>\delta_J\right)S_1+{}\\
{}+
\mathbf{1}\left(\abs{\sigma^2-\hat{\sigma}^2_S}\leqslant\delta_J\right)
S_1\equiv S'_1+S''_1. %\notag
\end{multline*}
Для произвольного $\varepsilon\hm>0$
\begin{equation*}
{\sf P}\left(S'_1>\varepsilon\right)\leqslant{\sf P}
\left(\abs{\sigma^2-\hat{\sigma}^2_S}>\delta_J\right). %\notag
\end{equation*}
При выполнении условий теоремы, если константа~$C_\delta$ достаточно велика, 
то найдется константа~$\tilde{C}_\delta>0$ такая, что~\cite{KS11-2}
\begin{equation*}
{\sf P}\left(\abs{\sigma^2-\hat{\sigma}^2_S}>\delta_J\right)
\leqslant\tilde{C}_\delta2^{-J/2}. %\notag
\end{equation*}
%% комментарии по поводу этого неравенства и~загрязнения выборки есть в~диссертации
Следовательно, $S'_1\stackrel{P}{\to}0$ при $J\hm\rightarrow\infty$.

Обозначим слагаемые в~сумме~$S''_1$ через~$F_{j,k}(\hat{\sigma}_S)$. Пусть 
$A_j\hm=\sqrt{A\ln 2^j}$, где $0\hm<A\hm<2(\sigma^2\hm-\delta_J)$. Имеем:

\noindent
\begin{multline*}
\hspace*{-9.9pt}\sum\limits_{j=0}^{J-1}\sum\limits_{k=0}^{2^j-1}F_{j,k}\left(\hat{\sigma}_S\right)=
\sum\limits_{j=0}^{J-1}\sum\limits_{k=0}^{2^j-1}
\mathbf{1}(\abs{Y_{j,k}}\leqslant A_j)F_{j,k}(\hat{\sigma}_S)+{}\\
{}+
\sum\limits_{j=0}^{J-1}\sum\limits_{k=0}^{2^j-1}
\mathbf{1}\left(\abs{Y_{j,k}}>A_j\right)F_{j,k}(\hat{\sigma}_S)
\equiv  W_1+W_2. %\notag
\end{multline*}
Рассмотрим $W_1$. Учитывая определения $\widehat{\mu}_{j,k}(\sigma)$, 
$({\partial}/{\partial Y_{j,k}})\widehat{\mu}_{j,k}(\sigma)$ и~$A_j$, 
можно убедиться, что найдут\-ся константы $C_1\hm>0$ и~$\theta\hm>0$ такие, что
\begin{equation*}
\abs{\mathbf{1}\left(\abs{Y_{j,k}}\leqslant A_J\right)
F_{j,k}(\hat{\sigma}_S)}\leqslant C_1 
J^{5/2}2^{(2\alpha-\theta)j-J/2}\;\;\mbox{п.в.} %\notag
\end{equation*}
% поскольку выполнено \mathbf{1}(\abs{\sigma^2-\hat{\sigma}^2_S}\leqslant\delta_J). В логарифме степень: от Y идет 1, от T идет 1, от \delta_J идет 1/2 но для J, а не для j, поэтому берем для всех J^{5/2}. В степени 2: 2\alpha от \beta{j,k}, \theta из-за выбора A, J/2 от \delta_J
Следовательно, $D_J^{-1}W_1\hm\rightarrow 0$ п.в.\ при $J\hm\rightarrow\infty$.

Далее для слагаемых~$W_2$ имеем:
\begin{multline*}
\left\vert \mathbf{1}\left(
\left\vert Y_{j,k}\right\vert
> A_J\right)F_{j,k}
\left(\hat{\sigma}_S\right)\right\vert
\leqslant{}\\
{}\leqslant C_2 J^{3/2}2^{2\alpha j-J/2} 
\mathbf{1}\left( \left\vert Y_{j,k}\right\vert > A_J\right) 
\left\vert Y_{j,k}\right\vert^2\;\;\mbox{п.в.},
%\notag
\end{multline*}
% поскольку выполнено \mathbf{1}(\abs{\sigma^2-\hat{\sigma}^2_S}\leqslant\delta_J). В логарифме от T идет 1, от \delta_J идет 1/2.
где $C_2>0$~--- некоторая константа. Учитывая распределение~$Y_{j,k}$, 
нетрудно убедиться, что
\begin{equation*}
\Expect\frac{1}{D_J} \sum\limits_{j=0}^{J-1}
\sum\limits_{k=0}^{2^j-1} J^{3/2}2^{2\alpha j-J/2} 
\mathbf{1}\left(\abs{Y_{j,k}}> A_j\right)
\abs{Y_{j,k}}^2\to 0
\end{equation*}
при $J\rightarrow\infty$. %\notag
Следовательно, используя неравенство Маркова, получаем, что
\begin{equation*}
D_J^{-1}W_2\stackrel{{\sf P}}{\to}0\;\;\mbox{при}\;J\rightarrow\infty\,. %\notag
\end{equation*}
Таким образом, $D_J^{-1}S_1\stackrel{{\sf P}}{\to}0$ при $J\hm\rightarrow\infty$.

Теорема доказана.

\smallskip

Рассмотрим теперь ситуацию, когда в~качестве оценки~$\sigma$ используется 
величина~$\widehat{\sigma}_{R}$ или~$\widehat{\sigma}_{M}$. 
В~этом случае повышаются требования к~гладкости функции сигнала.

\smallskip

\noindent
\textbf{Теорема~3.}\
\textit{Пусть~$Kf$ задана на конечном отрезке и~равномерно регулярна по 
Липшицу с~показателем $\gamma\hm>1/2$, а оценка дисперсии шума~$\hat{\sigma}$ 
задана соотношением}~\eqref{IQR_Definition} 
\textit{или соотношением}~\eqref{MAD_Definition}. \textit{Тогда}
\begin{equation*}
\label{CLT_Operator_RobVar_Sigma}
\mathsf{P}\left(\fr{\widehat{R}_J(\widehat{\sigma})-R_J(\sigma)}{D_J}<x\right)
\Rightarrow \Phi_{\Upsilon_2}(x)\,, %\notag
\end{equation*}
где $\Phi_{\Upsilon_2}(x)$~--- функция распределения нормального закона 
с~нулевым средним и~дисперсией
\begin{multline*}
\Upsilon_2^2=1+\fr{2^{4\alpha+1}-1}{4(2^{2\alpha+1}-1)^2
\xi_{3/4}^2(\phi(\xi_{3/4}))^2}-{}\\
{}-
\fr{2^{4\alpha+1}-1 }{2^{2\alpha-1}(2^{2\alpha+1}-1)}\,.
\end{multline*}

\noindent
Д\,о\,к\,а\,з\,а\,т\,е\,л\,ь\,с\,т\,в\,о\,.\ \
Как и~в~предыдущей теореме, запишем
$\widehat{R}_J(\hat{\sigma})\hm-R_J(\sigma)\hm=S_1\hm+S_2\hm+S_3.$
Учитывая,\linebreak\vspace*{-12pt}

\pagebreak

\noindent
 что $\gamma\hm>1/2$, и~поступая, как в~работах~\cite{SH18, KS11-2, SH12}, 
с~использованием разложения Бахадура для выборочных квантилей~\cite{S80} и~выборочного 
абсолютного медианного отклонения~\cite{SM09}, можно показать, что
\begin{equation*}
{\sf P}\left(\fr{S_2+S_3}{D_J}<x\right)\Rightarrow\Phi_{\Upsilon_2}(x)\,. %\notag
\end{equation*}
% на самом деле с~условием Линдеберга чуть по-другому (без ограниченности слагаемых). Но дисперсия равномерно ограничена -- значит выполнено.

Используя экспоненциальные неравенства для выборочных квантилей~\cite{S80} 
и~выборочного абсолютного медианного отклонения~\cite{SM09}, получаем, что при 
выполнении условий теоремы найдется такая константа $C_\delta\hm>0$, что при 
$\delta_J\hm=C_\delta J^{1/2}2^{-J/2}$ для некоторой константы~$\widetilde{C}_\delta>0$ 
выполнено:
\begin{align*}
\mathsf{P}\left(\abs{\widehat{\sigma}_{R}-\sigma}>\delta_J\right)
&\leqslant\widetilde{C}_\delta2^{-J/2}\,;
\\
\mathsf{P}\left(\abs{\widehat{\sigma}_{M}-\sigma}>\delta_J\right)
&\leqslant\widetilde{C}_\delta2^{-J/2}\,. %\notag
\end{align*}
%% комментарии по поводу этого неравенства и~загрязнения выборки есть в~диссертации
Далее, повторяя рассуждения предыдущей теоремы, заключаем, что 
$D_J^{-1}S_1\stackrel{{\sf P}}{\to}0$ при $J\hm\rightarrow\infty$.


Теорема доказана.



{\small\frenchspacing
 {%\baselineskip=10.8pt
 \addcontentsline{toc}{section}{References}
 \begin{thebibliography}{99}

\bibitem{HL10}
\Au{Huang H.-C., Lee~T.\,C.\,M.} 
Stabilized thresholding with generalized sure for image denoising~// 
IEEE 17th  Conference (International) on Image Processing
Proceedings.~--- IEEE, 2010. P.~1881--1884.

\bibitem{SH18}
\Au{Shestakov O.\,V.} 
Nonlinear regularization of inverse problems for linear homogeneous transforms 
by the stabilized hard thresholding~// J.~Math. Sci., 2018. Vol.~234. No.\,6. P.~780--785.

\bibitem{KS11-1}
\Au{Кудрявцев А.\,А., Шестаков~О.\,В.} 
Асимптотика оценки риска при вейг\-лет-вейв\-лет разложении наблюдаемого сигнала~// 
T-Comm~--- телекоммуникации и~транспорт, 2011. №\,2. С.~54--57.

\bibitem{KS11-2}
\Au{Кудрявцев А.\,А., Шестаков~О.\,В.} 
Асимптотическое распределение оценки риска пороговой обработки 
вейг\-лет-ко\-эф\-фи\-ци\-ен\-тов сигнала при неизвестном уровне шума~// 
T-Comm~--- телекоммуникации и~транспорт, 2011. №\,5. С.~24--30.

\bibitem{AS98}
\Au{Abramovich F., Silverman~B.\,W.} 
Wavelet decomposition approaches to statistical inverse problems~// 
Biometrika, 1998. Vol.~85. No.\,1. P. 115--129.

\bibitem{Mal99}
\Au{Mallat S.} A~Wavelet tour of signal processing.~--- 
New York, NY, USA: Academic Press, 1999. 857~p.

\bibitem{L97}
\Au{Lee N.} Wavelet-vaguelette decompositions and homogenous equations.~--- 
West Lafayette, IN, USA: Purdue University, 1997.  PhD Thesis. 103~p.

\bibitem{B96}
\Au{Breiman L.} Heuristics of instability and stabilization in model selection~// 
Ann. Stat., 1996. Vol.~24. No.\,6. P.~2350--2383.

\bibitem{J01}
\Au{Jansen M.} Noise reduction by wavelet thresholding.~--- 
Lecture notes in statistics ser.~--- New York, NY, USA: Springer Verlag,
2001. Vol.~161. 196~p.

\bibitem{SH12}
\Au{Шестаков О.\,В.} О~скорости сходимости оценки риска пороговой обработки 
вейв\-лет-ко\-эф\-фи\-ци\-ен\-тов к~нормальному закону при использовании 
робастных оценок дисперсии~// Информатика и~её применения, 2012. Т.~6. Вып.~2. 
С.~122--128.

\bibitem{S80}
\Au{Serfling R.} Approximation theorems of mathematical statistics.~--- 
New York, NY, USA: John Wiley \& Sons, 1980. 371~p.

\bibitem{SM09}
\Au{Serfling R., Mazumder~S.} 
Exponential probability inequality and convergence results for the median 
absolute deviation and its modifications~// Stat. Probabil. Lett., 2009. 
Vol.~79. No.\,16. P.~1767--1773.
 \end{thebibliography}

 }
 }

\end{multicols}

\vspace*{-3pt}

\hfill{\small\textit{Поступила в~редакцию 14.12.18}}

\vspace*{8pt}

%\pagebreak

%\newpage

%\vspace*{-28pt}

\hrule

\vspace*{2pt}

\hrule

%\vspace*{-2pt}

\def\tit{INVERSION OF~HOMOGENEOUS OPERATORS USING~STABILIZED HARD THRESHOLDING 
WITH~UNKNOWN NOISE VARIANCE}

\def\titkol{Inversion of~homogeneous operators using~stabilized hard thresholding 
with~unknown noise variance}

\def\aut{O.\,V.~Shestakov}

\def\autkol{O.\,V.~Shestakov}

\titel{\tit}{\aut}{\autkol}{\titkol}

\vspace*{-11pt}


\noindent
Department of Mathematical Statistics, Faculty of Computational Mathematics and Cybernetics, M.V. Lomonosov Moscow State University, 1-52 Leninskiye Gory, GSP-1, Moscow 119991, Russian Federation
Institute of Informatics Problems, Federal Research Center 
``Computer Science and Control'' of the Russian Academy of Sciences, 44-2~Vavilov Str., 
Moscow 119333, Russian Federation

\def\leftfootline{\small{\textbf{\thepage}
\hfill INFORMATIKA I EE PRIMENENIYA~--- INFORMATICS AND
APPLICATIONS\ \ \ 2019\ \ \ volume~13\ \ \ issue\ 1}
}%
 \def\rightfootline{\small{INFORMATIKA I EE PRIMENENIYA~---
INFORMATICS AND APPLICATIONS\ \ \ 2019\ \ \ volume~13\ \ \ issue\ 1
\hfill \textbf{\thepage}}}

\vspace*{6pt}



\Abste{When inverting linear homogeneous operators, it is necessary to use 
regularization methods, since observed data are usually noisy. For noise suppression, 
threshold processing of  wavelet coefficients of the observed signal function 
is often used. Threshold processing has become a~popular noise suppression tool 
due to its simplicity, computational efficiency, and ability to adapt to functions 
that have different degrees of regularity at different domains. The paper 
discusses the recently proposed stabilized hard thresholding method that eliminates 
the main
drawbacks of soft and hard thresholding methods and studies statistical 
properties of this method. In the data model\linebreak\vspace*{-12pt}}

\Abstend{with an additive Gaussian noise with 
unknown variance, an unbiased estimate of the mean square risk is analyzed and it 
is shown that under certain conditions, this estimate is asymptotically normal and 
the variance of the limit distribution depends on the type of estimate of noise variance.}


\KWE{wavelets; threshold processing; unbiased risk estimate; asymptotic normality;
strong consistency}




\DOI{10.14357/19922264190107}

%\vspace*{-14pt}

\Ack
\noindent
This research was partly supported by the Russian  
Foundation for Basic Research (project No.\,19-07-00352).




%\vspace*{6pt}

  \begin{multicols}{2}

\renewcommand{\bibname}{\protect\rmfamily References}
%\renewcommand{\bibname}{\large\protect\rm References}

{\small\frenchspacing
 {%\baselineskip=10.8pt
 \addcontentsline{toc}{section}{References}
 \begin{thebibliography}{99}
\bibitem{1-sh-1}
\Aue{Huang, H.-C., and T.\,C.\,M.~Lee.} 2010. 
Stabilized thresholding with generalized sure for image denoising. 
\textit{IEEE 17th Conference (International) on Image Processing}. IEEE. 1881--1884.

 

\bibitem{2-sh-1}
\Aue{Shestakov, O.\,V.} 2018. 
Nonlinear regularization of inverse problems for linear homogeneous transforms 
by the stabilized hard thresholding. 
\textit{J.~Math. Sci.} 234(6):780--785.

\bibitem{3-sh-1}
\Aue{Kudryavtsev, A.\,A., and O.\,V.~Shestakov.} 2011. Аsimptotika otsenki riska pri 
veyglet-veyvlet razlozhenii nablyuda\-emo\-go signala [The average risk assessment 
of the wavelet decomposition of the signal].
\textit{T-Comm~--- Telecommunications and Their Application in
Transport Industry} 2:54--57.

\bibitem{4-sh-1}
\Aue{Kudryavtsev, A.\,A., and O.\,V.~Shestakov.} 2011. Аsimptoticheskoe raspredelenie 
otsenki riska porogovoy ob\-ra\-bot\-ki veyglet-koeffitsientov signala pri 
neizvestnom urovne shuma [Asymptotic distribution of the risk estimate of 
the signal vaguelette coefficients thresholding at the unknown noise level]. 
\textit{T-Comm~--- Telecommunications and Their Application in
Transport Industry} 5:24--30.

\bibitem{5-sh-1}
\Aue{Abramovich, F., and B.\,W.~Silverman.} 1998. Wavelet 
decomposition approaches to statistical inverse problems. 
\textit{Biometrika} 85(1):115--129.

\bibitem{6-sh-1}
\Aue{Mallat, S.} 1999. \textit{A~wavelet tour of signal processing.} New York, NY: 
Academic Press. 857 p.

\bibitem{7-sh-1}
\Aue{Lee, N.} 1997. Wavelet-vaguelette decompositions and homogenous equations. 
 West Lafayette, IN: Purdue University. PhD Thesis. 103~p.

\bibitem{8-sh-1}
\Aue{Breiman, L.} 1996. 
Heuristics of instability and stabilization in model selection. 
\textit{Ann. Stat.} 24(6):2350--2383.

\bibitem{9-sh-1}
\Aue{Jansen, M.} 2001. \textit{Noise reduction by wavelet thresholding.} 
Lecture notes in statistics ser.
New York, NY: Springer Verlag.  Vol.~161. 196~p.

\bibitem{10-sh-1}
\Aue{Shestakov, O.\,V.} 2012. O~skorosti skhodimosti otsenki riska porogovoy 
obrabotki veyvlet-koeffitsientov k~nor\-mal'\-no\-mu zakonu pri ispol'zovanii robastnykh 
otsenok dispersii [On the rate of convergence to the normal law of risk estimate for 
wavelet coefficients thresholding when using robust variance estimates]. 
\textit{Informatika i~ee Primeneniya~--- Inform. Appl.}  6(2):122--128.

\bibitem{11-sh-1}
\Aue{Serfling, R.} 1980. \textit{Approximation theorems of mathematical statistics}.
New York, NY: John Wiley \& Sons. 371~p.

\bibitem{12-sh-1}
\Aue{Serfling, R., and S.~Mazumder.} 2009. Exponential probability inequality 
and convergence results for the median absolute deviation and its modifications. 
\textit{Stat. Probabil. Lett.} 79(16):1767--1773.
\end{thebibliography}

 }
 }

\end{multicols}

\vspace*{-6pt}

\hfill{\small\textit{Received December 14, 2018}}

%\pagebreak

%\vspace*{-18pt}  

\Contrl

\noindent
\textbf{Shestakov Oleg V.} (b.\ 1976)~--- 
Doctor of Science in physics and mathematics, professor, Department of 
Mathematical Statistics, Faculty of Computational Mathematics and Cybernetics, 
M.\,V.~Lomonosov Moscow State University, 1-52~Leninskiye Gory, GSP-1, Moscow 119991, 
Russian Federation; senior scientist, Institute of Informatics Problems, 
Federal Research Center ``Computer Science and Control'' 
of the Russian Academy of Sciences, 44-2~Vavilov Str., Moscow 119333, 
Russian Federation; \mbox{oshestakov@cs.msu.su}
\label{end\stat}

\renewcommand{\bibname}{\protect\rm Литература} 
        %7
\def\stat{agalarov}


\def\tit{АЛГОРИТМ ВЫЧИСЛЕНИЯ ЗАГРУЖЕННОСТИ 
ТЕЛЕКОММУНИКАЦИОННОЙ СЕТИ С~ПОВТОРНЫМИ ПЕРЕДАЧАМИ$^*$}
\def\titkol{Алгоритм вычисления загруженности 
телекоммуникационной сети с~повторными передачами} 

\def\autkol{Я.\,М.~Агаларов}
\def\aut{Я.\,М.~Агаларов$^1$}

\titel{\tit}{\aut}{\autkol}{\titkol}

{\renewcommand{\thefootnote}{\fnsymbol{footnote}}\footnotetext[1]
{Работа выполнена при частичной поддержке РФФИ, проекты 08-07-00152-а и
      09-07-12032-офи-м.}}

\renewcommand{\thefootnote}{\arabic{footnote}}
\footnotetext[1]{Институт проблем
информатики Российской академии наук, agglar@yandex.ru}


\end{document}
      \vskip 18pt plus 9pt minus 6pt

      \thispagestyle{headings}

      \begin{multicols}{2}

      \label{st\stat}

    
      \Abst{Рассмотрены модели сети коммутации пакетов c повторными попытками передачи 
пакетов для двух схем распределения буферной памяти: полнодоступной и полного разделения. 
Предложен итерационный метод расчета интенсивностей потоков в сети и вероятностей блокировок 
узлов, где в качестве модели узла используется СМО типа $
      \begin{matrix}
      M \\ \vec{\lambda}
      \end{matrix}
      \left |
      \begin{matrix}
      M \\ \vec{\lambda}
      \end{matrix}
      \right |
      \vec{m} \vert N$. Получено необходимое условие существования решения системы уравнений 
сохранения баланса потоков в установившемся режиме работы сети и доказана монотонная 
сходимость последовательности значений интенсивностей потоков и вероятностей блокировок, 
получаемых предлагаемым методом, к решению указанной системы.}
      
      \KW{слова: сеть коммутации пакетов; буферная память; повторные передачи; вероятность 
блокировки; итерационный метод}

     
\section{Введение}

     Одной из важных проблем, решаемых на этапе проектирования 
телекоммуникационных сетей, является задача предварительного анализа сети 
на предмет возникновения локальных и глобальных перегрузок.
     
     Причинами перегрузок наряду с другими могут быть:
     \begin{itemize}
     \item ограничение объема буферной памяти коммутационного 
оборудования;
     \item блокировка конечных терминалов;
     \item недостаточная производительность вычислительных ресурсов и 
пропускная способность каналов связи.
     \end{itemize}
     
     Ограничение буферной памяти в реальных сетях вызвано не только 
желанием разработчиков снизить себестоимость коммутаторов, но и 
требованиями к параметрам качества обслуживания (среднее время задержки и 
его разброс, вероятность потери пакетов и~др.). В то же время ограничение 
объема буферной памяти узлов является одной из причин роста числа 
повторных передач в сетях с коммутацией пакетов и, как следствие, резкого 
роста нагрузки на отдельных участках сети или сети в целом. Поэтому одной из 
задач предварительного анализа сетей является оценка загруженности узлов и 
каналов связи с учетом ограниченного объема буферной памяти.
     
     Используемые точные методы анализа сетей с коммутацией пакетов (см., 
например,~[1, 2]) разработаны в рамках экспоненциальных СеМО (сетей 
массового обслуживания) со стохастическими маршрутными матрицами и 
неограниченной буферной памятью. Однако предположение о неограниченной 
буферной памяти исключает возможность учета блокировок узлов из-за 
нехватки буферной памяти, которые и являются одной из основных причин 
возникновения повторных передач пакетов в сети.
     
     Большое число работ в последнее время посвящается системам массового 
обслуживания с повторными заявками, одним каналом (обслуживающим 
устройством), ограниченным накопителем и более общими предположениями 
относительно входящих потоков заявок и длительностей обслуживания~[2--7], 
чем при исследовании СеМО. Однако использование этих моделей при расчете 
сетей вызывает очень большие вычислительные трудности.
     
     Из множества приближенных методов расчета сетей с ограниченной 
буферной памятью в узлах следует выделить методы, используемые в теории 
второго порядка для СеМО~[2, 8], и методы, рассматривающие узлы как 
изолированные СМО с пуассоновскими входящими потоками~[9, 10]. Первые 
из них предполагают: 
     \begin{enumerate}[1)]
     \item внешний поток заявок~--- рекуррентный с известными первым и 
вторым моментами интервалов между поступлениями; 
     \item  узел~--- СМО с одним прибором и накопителем;
     \item  время обслуживания в узле~--- независимое с произвольным 
распределением с известными первым и вторым моментами;
     \item движение заявок по сети происходит согласно неразложимой 
стохастической матрице с возможностью случайного ухода из сети в каждом 
узле. Сущность этих методов состоит в том, что они при расчетах используют 
первые и вторые моменты соответствующих распределений интервалов 
поступления и обслуживания заявок. Второй подход отличается тем, что 
потоки, образованные в узлах суперпозицией внешнего потока, повторениями 
по сети не доставленных пакетов и потоками от других узлов,~--- 
пуассоновские потоки, а времена обслуживания~--- экспоненциальные. Общее 
у этих методов то, что они являются итерационными, причем каждая итерация 
выполняется в два этапа: на первом этапе вычисляются характеристики потоков 
в узлах, на втором~--- уточняются другие характеристики сети (вероятности 
блокировок узлов, моменты времени задержки пакетов в узле и~др.). В этих 
методах в качестве моделей линий связи использованы одноканальные СМО. 
     \end{enumerate}
     
     Ниже будут рассмотрены модели сетей коммутации пакетов с повторами 
из источника и из предыдущего узла. Предлагается итерационный метод 
расчета сетей, который реализует второй из упомянутых выше подходов и в 
качестве модели узла использует СМО типа $
      \begin{matrix}
      M \\ \vec{\lambda}
      \end{matrix}
      \left |
      \begin{matrix}
      M \\ \vec{\lambda}
      \end{matrix}
      \right |
      \vec{m} \vert N$. Получено необходимое условие существования решения 
системы уравнений сохранения баланса потоков в установившемся режиме 
работы сети и доказана монотонная сходимость последовательности значений 
интенсивностей потоков и вероятностей блокировок, получаемых 
предлагаемым методом, к решению системы.
     
\section{Общее описание моделей сети}
     
     Рассматривается модель сети с коммутацией пакетов в виде графа, 
состоящего из $U$~вершин и $V$~дуг. Вершины графа отождествляются с 
узлами связи, дуги~--- с линиями связи. Имеется множество источников и 
получателей пакетов, каждый из которых соединен с одним из узлов связи, 
который называется узлом-входом, если соединен с источником, и узлом-
выходом, если соединен с получателем. Передача пакета в сети происходит по 
заданному пути~$l$, соединяющему узел-вход с узлом-выходом. Будем считать 
(без потери общности), что каждый узел сети входит хотя бы в один путь сети и 
множество путей не разбивается на непересекающиеся подмножества. Известна 
интенсивность потока (первичного потока) пакетов, поступающих извне на 
каждый путь~$l$. Узлы сети имеют ограниченную буферную память с заданной 
схемой распределения, линии связи имеют заданное число однородных 
каналов. Поступивший в промежуточный узел пакет принимается в буферную 
память (занимает одно место буферной памяти), если согласно заданной схеме 
распределения ему можно выделить место в буферной памяти (узел не 
блокирован для данного пакета) и он передан без ошибок, иначе он передается 
повторно согласно процедуре повторов (из источника или из предыдущего 
узла), пока не будет успешно передан адресату. Под блокировкой узла (линии) 
понимается такое состояние узла (линии), когда согласно принятой схеме 
распределения памяти поступивший пакет не может быть принять в буфер 
данного узла (линии). Под успешной передачей (попыткой передачи) пакета 
понимается передача (попытка), когда переданный пакет принимается 
последующим узлом в буферную память. При неуспешной попытке передачи 
по линии пакета занятый им буфер освобождается сразу в случае сети с 
повторными попытками из источника и сохраняется за ним в случае повторов 
из предыдущего узла. После успешной передачи пакета занятое им место в 
буферной памяти через заданное время освобождается. Предполагается, что 
пакеты в сети не теряются.
     
     Введем обозначения:
     
     \noindent
     $v$ (или $v_i$, $i = 1$, 2,\ldots)~--- линия связи;
     
     
     \noindent
     $v^+$~--- узел-сток линии~$v$;
     
     \noindent
     $u$ (или $u_i$, $i = 1$, 2,\ldots)~--- узел связи;
     
     \noindent
     $\Omega_u^+$~--- множество исходящих из узла $u$ линий;
     
     \noindent
     $c_v$~--- канальная емкость линии~$v$;
     
     \noindent
     $N_v$~--- емкость буферной памяти, выделенной для линии~$v$;
     
     \noindent
     $N_u$~--- емкость общей буферной памяти узла~$u$;
     
     \noindent
     $L$~--- заданное множество нециклических путей;
     
     \noindent
     $L_v$~--- множество путей, содержащих линию~$v$, ($L_v\subseteq L$);
     
     \noindent
     $l=\{v_1,\ldots ,v_{S_l}\}$~--- путь, содержащий линии $v_1,\ldots 
,v_{S_l}$, где $S_l$~--- число линий в пути~$l$, индексы 1,\ldots , $S_l$ 
показывают порядок следования линий в пути, $v_l$~--- линия, исходящая из 
     узла-входа, $v_{S_l}$~--- линия, входящая в узел-выход;
     
     \noindent
     $u_{S_l+1}$~--- абонентский узел, соединенный с узлом-выходом 
пути~$l$;
     
     \noindent
     $U_u^+$~--- множество различных узлов, следующих после узла~$u$ по 
направлению к адресату в путях, проходящих через узел~$u$;
     
     \noindent
     $V_v^+$~--- множество различных линий, следующих после линии~$v$ 
по направлению к адресату в путях, проходящих по линии~$v$;
     
     \noindent
     $\lambda(l)$~--- интенсивность потока ($l$-потока) пакетов, поступающих 
из источника на узел-вход и требующих передачи на узел-выход, $\lambda(l) 
>0$, $l\in L$;
     
     \noindent
     $\mu_v$~--- интенсивность обслуживания пакета каналом линии~$v$;
     
     \noindent
     $\delta_v$~--- вероятность безошибочной передачи пакета по линии~$v$;
     
     \noindent
     $\Lambda_v$~--- интенсивность потока пакетов, успешно передаваемых 
по линии~$v$;
     
     \noindent
     $\Lambda_v^*$~--- интенсивность суммарного потока пакетов, 
требующих передачи по линии~$v$;
     
     \noindent
     $\Lambda_v^*(l)$~--- интенсивность $l$-потока, поступающего на 
линию~$v$;
     
     \noindent
     $\pi_u$~--- вероятность блокировки узла~$u$;
     $\pi_v$~--- вероятность блокировки узла для пакетов, требующих 
передачи по исходящей из узла линии~$v$.
     
     Во всех рассматриваемых ниже моделях узла коммутации 
предполагается, что потоки $\Lambda_v^*$~--- пуассоновские, а времена 
обслуживания пакетов каналами связи~--- экспоненциальные с параметрами 
$\mu_v$, $v\in V$. Предполагается также, что внешние нагрузки~--- 
реализуемые, т.\,е.\ в стационарном режиме работы сети интенсивности 
первичных входных потоков равны интенсивностям соответствующих 
выходных (покидающих сеть) потоков. Всюду ниже сеть рассматривается в 
стационарном режиме.
     
\section{Сеть с повторами из источника}
     
     В качестве модели коммутационного узла используется СМО с 
ограниченным накопителем (буферной памятью) и несколькими линиями из 
однотипных каналов, в которой сделаны также следующие предположения:
     \begin{enumerate}[1.]
\item Места в буферной памяти распределяются согласно одной из двух 
схем:
\begin{itemize}
\item полнодоступная схема (CS)~--- каждое свободное место хранения 
доступно любой заявке (пакету);
\item схема полного разделения памяти (CP)~--- заявке, требующей передачи 
по линии~$v$ ($v$-заявке), доступны всего~$N_v$ мест, где 
$\sum\limits_{v\in\Omega_u^+} N_v = N$.
\end{itemize}
\item Суммарные потоки первичных и повторных $v$-заявок являются 
независимыми в совокупности пуассоновскими потоками. Для 
обслуживания $v$-заявки требуется одновременно одно место хранения и 
один канал типа~$v$, $v\in\Omega_u^+$.
\item Поступившей в СМО заявке предоставляется место в накопителе, если 
она передана без ошибок и в момент ее поступления в накопителе есть 
доступное свободное место, иначе заявка получает отказ.
\item Принятые в СМО $v$-заявки обслуживаются линией~$v$ в порядке 
поступления.
\item Время занятия канала $v$-заявкой~--- экспоненциально 
распределенная случайная величина с параметром $0<\mu_v<\infty$, $v\in 
\Omega_u^+$, независимая от других случайных событий в системе.
\item Обслуженная заявка освобождает сразу место в накопителе СМО.
\item Заявка, получившая отказ, повторяется через заданное время из 
источника.
\end{enumerate}

     Пусть во всех узлах сети распределение буферной памяти происходит по 
схеме CS. При полнодоступной схеме и повторах из источника в 
установившемся режиме работы сети справедливы следующие соотношения 
для потоков в узлах:
     \begin{align}
     \Lambda_v(l) =\Lambda_v^*(l)\left (1-\pi_v\right )\,,\quad
     \Lambda^*_{v_i}(l) =\Lambda_{v_{i-1}}(l)\,,\quad
     \Lambda_{v_{S_l}} =\lambda(l)\,,\quad l\in L\,,\notag\\
     \Lambda_v=\sum\limits_{l\in L}\Lambda_v(l)\,,\quad 
\Lambda_v^*=\sum\limits_{l\in L}\Lambda_v^*(l)\,,\quad v\in V\,.
     \end{align}
     
     Из~(1) для вычисления параметра $\Lambda_{v_i}(l), $i=1,\ldots ,S_l$, 
$l\inL$, получаем формулу
     \begin{equation}
     \Lambda_{v_i}(l) =\fr{\Lambda_{v_{i+1}}(l)}{(1-
\pi_{u_{i+1}})\delta_{v_{i+1}}} =\fr{\lambda(l)}{(1-
\pi_{u_{S_l+1}})\prod\limits_{}^{S_l} (1-\pi_{u_j})\delta_{v_j}}\,,\quad i=1,\ldots 
,S_l-1\,.
     \end{equation}
     
     Здесь и далее по тексту статьи считается, что при заданных~$\lambda(l)$, 
$l\inL$, величины $\pi_{u_{S_l+1}}(l)$, $l\in L$, заранее вычислены.
     
     Пусть $\overline{k}_u =\{k_v,\ v\in\Omega_u^+\}$~--- состояние буферной 
памяти узла $u\in U$, $k_v$~--- число пакетов в буферной памяти узла, 
передаваемых по линии~$v$, $A_m = 
\{\overline{k}_u:\sum\limits_{v\in\Omega_u^+} k_v=m\}$~--- множество 
различных состояний, при которых в памяти узла заняты ровно $m$~буферов. 
Тогда с учетом введенных выше обозначений и предположений для 
стационарной вероятности блокировки узла можем написать формулу~[2, 11]
\begin{equation}
\pi_u = \fr{1}{G_{N_u}}\sum\limits_{\overline{k}\in A_N} p\left ( 
\overline{k}_u,\overline{\rho}_u^*\right )\,,
\end{equation}
где 
\begin{align}
p(\overline{k}_u,\overline{\rho}_u^*) & = 
\prod\limits_{v\in \Omega_u^+} z_v(\rho_v^*,k_v)\,,\notag\\
Z_v(\rho_v^*,k_v) & = 
\begin{cases}
\fr{\rho_v^{*k_v}}{k_v!} & \mbox{при}\ k_v<c_v\,,\\
\fr{\rho_v^{*k_v}}{c_v ! c_v^{k_v-c_v}} & \mbox{при}\ k_v\geq c_v\,,
\end{cases}\\
G_{N_u} & = \sum\limits_{m=0}^N \sum\limits_{\overline{k}\in A_m} 
p(\overline{k}_u, \overline{\rho}_u^* )\,,\quad \overline{\rho}_u^*=\{\rho_v^*,\ v\in 
\Omega_u^*\}\quad \rho_v^* =\fr{\Lambda_v^*}{\mu_v}\,,\quad v\in\Omega_u^+\,.
     \end{align}
     
     Таким образом, из соотношений~(1)--(5) относительно неизвестных 
величин~$\pi_u$, $u\in U$, получаем систему нелинейных уравнений вида
     \begin{equation*}
     \pi_u = f_u\left ( 
\overline{\lambda},\overline{\mu},\overline{N},\overline{\pi}\right )\,,\quad u\in 
U\,,
     \end{quation*}
     где $\overline{\lambda} =\{\lambda (l),\ l\in\L_u\}$, $\overline{mu} 
=\{\mu_{u^\prime},\ u^\prime\in u\cup U_u^+\}$, $\overline{N}=\{N_{u^\prime},\  
u^\prime \in u\cup U_u^+\}$, $\overline{\pi} = \{\pi_{u^\prime},\ u^\prime \in u\cup 
U_u^+\}$.
     
     Переобозначив $1-\pi+u$ через~$v_u$, выражение в правой части 
равенства для~$p(\overline{k}_u, \overline{\rho}_u^*)$ из (4)~--- через 
$p_{\overline{k}}(\overline{\rho}_u, y_u)$, выражение в правой части равенства 
для~$\pi_u$ из (3)~--- через $1-q_{N_u}(\overline{\rho}_u, y_u)$, где 
$\overline{\rho}_u = (\rho_v, \ v\in\Omega_u^+)$, $\rho_v = \rho_v^* y_u = 
\Lambda_v/\mu_v$, $v\in\Omega_u^+$, получим систему нелинейных уравнений 
относительно неизвестных переменных~$y_u$
     \begin{equation}
     y_u = q_{N_u}\left ( \overline{\rho}_u,y_u\right ),\quad u\in U\,.
     \end{equation}
     
     Отметим, что $\overline{\rho}_u = \{\rho_v, v\in\Omega_u^+\}$ где 
$\rho_v$~--- функция переменных $\overline{y}_u =\{y_{u^\prime},\ u^\prime \in 
U_u^+\}$.
     
     Обозначим набор $\{y_u, u\in U\}$ через~$\overline{y}$. Будем говорить, 
что решение~$\overline{y}$ положительное, если $y_u\in (0,\,1]$ для всех $u\in 
U$.
     
     \medskip
     
     \noindent
     \textbf{Утверждение~1.} \textit{Если} 
     $\overline{y}^\prime = \{u_u^\prime \in (0,\,],\ u\in U\}~--- \textit{решение 
системы уравнений}~(6), \textit{то необходимо выполнение для всех} $u\in U$ 
\textit{условия}
     \begin{equation}
     \fr{\sum\limits_{\overline{k}\in A_{N_u-1}} 
p_{\overline{k}}(\overline{p}_u^\prime , 1)}
     {\sum\limits_{\overline{k}\in A_{N_u}} 
p_{\overline{k}}(\overline{\rho}_u^\prime, 1)} >1\,,
     \label{e7ag}
     \end{equation}
     \textit{где} $\overline{\rho}_u^\prime$~--- \textit{значение 
переменной}~$\overline{\rho}_u$ \textit{при} $\overline{y} 
=\overline{y}^\prime$.
     
     \medskip
     
     \noindent
     Д\,о\,к\,а\,з\,а\,т\,е\,л\,ь\,с\,т\,в\,о\,.\ Пусть $\overline{y}^\prime = 
\{y_u^\prime \in (0,\,1]$, u\in U\}$~--- решение системы~(5). Фиксируем 
произвольный узел~$u$ и положим $u_{u^\prime} = y_u^\prime$ для всех 
$u^\prime\not= u$, $u^\prime\in U$. Отметим, что значение 
переменной~$\overline{\rho}_u$ при заданных значениях~$\lambda(l)$, $l\in L$, 
однозначно определяется переменными~$y_{u^\prime}$, $u^\prime\not= u$, 
$u^\prime\in U$ (см.~(2)). Рассмотрим уравнение
\begin{equation}
Y_u = q_{N_u} (\overline{\rho}_u^\prime, y_u)\,.
\label{e8ag}
\end{equation}
     
     Из работы~[12] (см.\ утверждение~4) следует, что уравнение~(\ref{e8ag}) 
имеет положительное решение тогда и только тогда, когда в узле~$\u$ 
выполняется условие~(\ref{e7ag}), при этом оно будет единственным 
положительным решением. Так как узел $u$~--- произвольный, то получаем, 
что неравенство~(\ref{e7ag}) должно выполняться для всех $u\in U$.
     
     \medskip
     
     \noindent
     \textbf{Следствие.} \textit{Выполнение неравенств} $\mu_v c_v / 
\Lambda_v >1$, $v\in V$, \textit{является необходимым условием 
существования положительного решения системы уравнений}~(\ref{e6ag}).
     
     \medskip
     
     \noindent
     Д\,о\,к\,а\,з\,а\,т\,е\,л\,ь\,с\,т\,в\,о\ непосредственно вытекает из следствия 
утверждения~4 в~[12].
     
     \smallskip
     Пусть задана последовательность $\overline{y}[n] =\{ y_u [n],\ u\in U\}$, 
$n\geq 0, где $y_n[n+1]=q_{N_u}[n+1]=q_{N_u}(\overline{\rho}_u[n],y_u[n])$, 
y_u[0]=1$, $u\in U$, а $\overline{\rho}_u[n]$~--- это~$\overline{\rho}_u$, 
вычисленное при $y_u =1-\pi_u =y_u[n]$. В дальнейшем будем писать 
$\overline{y}[n+1]<\overline{y}[n]$, если для заданного $n\geq 0$ выполняется 
$y_u[n+1]<y_u[n]$ для всех $u\in U$.
     
     \medskip
     
     \noindent
     \textbf{Утверждение 2.} \textit{Для всех} $n\geq 0$ \textit{верно} 
$\overline{y}[n+1] <\overline{y}[n]$.
     \medskip
     
     \noindent
     Д\,о\,к\,а\,з\,а\,т\,е\,л\,ь\,с\,т\,в\,о\,.\ Докажем, что для любых~$u$, 
$u^\prime \in U$, принадлежащих одновременно хотя бы одному пути, 
справедливо неравенство
     \begin{equation}
     \fr{\partial q_{N_u}(\overline{\rho}_u,y_u)}{\partial y_{u^\prime}}>0\,.
     \end{equation}
     
     Взяв производную от $y_u =q_{N_u}(\overline{\rho}_u,y_u)$ как от 
сложной функции, получаем
     \begin{equation}
     \fr{\partial q_{N_u}(\overline{\rho}_u,y_u)}{\partial y_{u^\prime}} = 
\sum\limits_{v\in\Omega_u^+}\fr{\partial q_{N_u}(\overline{\rho}_u,y_u)}{\partial 
\Lambda}\,\fr{\partial \Lambda_v}{\partial y_{u^\prime}}.
     \end{equation}
     
     Введем обозначения:
        .
     Из (3)--(5), взяв производную по~$\Lambda_v$, имеем
     \begin{equation}
     \fr{\partial q_{N_u}(\overline{\rho}_u,y_u)}{\partial\Lambda_v} = 
\fr{1}{\Lambda_v}\,q_{N_u}(\overline{\rho}_u,y_u)\left [ d_{N_u-
1}(\overline{\rho}_u,y_u)-d_{N_u}(\overline{\rho}_u,y_u)\right ]\,.
     \end{equation}
     
     Из (1) и~(2), взяв производную по~$y_{u^\prime}$, получаем
     \begin{equation}
     
      . (12)
     
     Подставив~(11) и~(12) в~(10), имеем
     \begin{equation*}
     \fr{\partial q_{N_u}(\overline{\rho}_u,y_u)}{\partial y_{u^\prime}} = 
\fr{1}{}\,q_{N_u}(\overline{\rho}_u, y_u)\left [d_{N_u}(\overline{\rho}_u,y_u)-
d_{N_u -1} (\overline{\rho}_u,y_u)\right 
]\sum\limits_{v\in\Omega_u^+}\fr{1}{\Lambda_v} \sum\limits_{l:l\in L_v,\, 
\mu^\prime\in l} \Lambda_v(l)\,.
     \end{equation*}
     
     Так как справедливо неравенство $d_{N_u}(\overline{\rho}_u,y_u)-d_{N_u 
-1} (\overline{\rho}_u,y_u) >0$ (см.\ утверждение~1 из~[12]), то из последнего 
равенства следует доказательство неравенства~(9). Из определения 
последовательности $\overline{y}[n]$, $n\geq 0$, и из~(9) следует 
доказательство утверждения~2.
     
     \medskip
     
     \noindent
     \textbf{Утверждение 3.} \textit{Последовательность}~$\overline{y}$, 
$n\geq 0$, \textit{сходится к положительному решению системы}~(6) 
\textit{тогда и только тогда, когда существует положительное решение 
системы}~(6).
     
     \medskip
     
     \noindent
     Д\,о\,к\,а\,з\,а\,т\,е\,л\,ь\,с\,т\,в\,о\,.\ Пусть $\overline{y}^* = \{y_u^*\in 
(0,\,1],\ u\in U\}$~--- решение системы уравнений~(6), $\overline{p}_u^*$~--- 
значение переменной~$\overline{\rho}_u$ при~$\overline{y}^*$. Очевидно, 
$u_u^*<1$, $u\in U$, так как $q_{N_u}(\overline{\rho}_u,y_u)<1$ при любых 
$y_u\in (0,\,1]$, $u\in U$. Пусть для некоторого $n\geq 0$ 
$\overline{y}[n]>\overline{y}^*$ (существование такого~$n$ вытекает из того, 
что $u_n[0] =1$ и $y_u^*<1$, $u\in U$). Тогда, как следует из~(9), для каждого 
узла $u\in U$ $y_u[n+1]=q_{N_u} (\overline{\rho}_u[n],y_u[n]) > q_{N_u}
     (\overline{\rho}_u^*,y_u^*)=y_u^*$, т.\,е.\ последовательность~$u_u[n]$, 
$n\geq 0$, ограничена снизу величиной~$\overline{y}_u^*$. Значит, 
существуют пределы $\lim\limits_{n\rightarrow\infty} y_u[n]=y_u^0\geq 
y_u^*>0$ для всех $u\in U$. Так как $q_{N_u}(\overline{\rho}_u,y_u), \rho_v$, 
$v\in\Omega_u^+$,~--- непрерывные по $y_{u^\prime}$, $u^\prime\in u\cup 
U_u^+$ функции, то можно написать $\lim\limits_{n\rightarrow\infty} q_{N_u} 
(\overline{\rho}_u[n],y_u[n])=q_{N_u}(\overline{\rho}_u^0,y_u^0)=y_u^0$, где 
$\overline{\rho}_u^0$~--- значение переменной~$\overline{\rho}_u$ при 
$y^0_{u^\prime}$, $u^\prime\in U_u^+$, т.\,е.\ $\overline{y}^0 =\{y_u^0\in (0,\,1),\ 
u\in U\}$~--- положительное решение уравнения~(6).
     
     Пусть теперь $\lim\limits_{n\rightarrow\infty} y_u[n] =y_u^*>0$ для всех 
$u\in U$. Тогда, как показано в первой части доказательства утверждения, 
$\overline{y}^0 = \{y_u^0\in (0,\,1),\ u\inU\}$~--- положительное решение 
уравнения~(6). Утверждение~3 доказано.
     
     \medskip
     
     \noindent
     \textbf{Следствие 2.} \textit{Система}~(6) \textit{не имеет 
положительного решения тогда и только тогда, когда} 
$\lim\limits_{n\rightarrow\infty} y_u[n]=y_u^*=0$ \textit{для всех} $u\in U$.
     
     Пусть во всех узлах сети распределение буферной памяти происходит по 
схеме CP. Тогда формула вероятности блокировки узла $u\inU$ для $v$-заявки 
($v\in \Omega_u^*$) записывается в том же виде, что и~(3)--(5), с заменой 
всюду индекса~$u$ на~$v$, за исключением обозначения~$\Omega_u^*$. 
Нетрудно заметить, что в случае этой схемы система уравнений~(6) примет вид
     \begin{equation}
     y_v = q_{N_u}(\rho_v,y_v)\,,\quad v\in V\,,
     \end{equation}
     где $\rho_v$ является функцией~$y_{v^\prime}$, $v^\prime \in V_v^+$, 
которая обладает всеми свойствами системы~(6), использованными при 
доказательстве утверждений~1, 2 и~3 и следствий. Неравенства вида~(6) в 
данном случае преобразуются в $\mu_v c_v/\Lambda_v >1$, $v\in V$.
     
     Заметим также, что все рассуждения, приведенные выше для сети с одной 
только из указанных выше схем распределения буферной памяти, справедливы 
и в смешанном случае, когда в узлах используется любая из этих схем.
     
\section{Сеть с повторами из предыдущего узла}
     
     Рассмотрим сеть с полнодоступной буферной памятью и повторами из 
предыдущего узла. В качестве модели узла используется СМО, отличающаяся 
от СМО, определенной в предыдущем разделе, только пунктами~6 и~7. Вместо 
действий, указанных в этих пунктах, реализуется следующее: выполненная 
     $v$-заявка с заданной вероятностью~$B_{v^+}$ (вероятность блокировки 
последующего узла или ошибки при передаче пакета по линии~$v$) 
повторяется в данном узле через заданное время~$\tau_v$ (тайм-аут) и с 
вероятностью~$1-B_{v^+}$ покидает систему через время~$t_v$ навсегда, 
сразу освободив занятый канал и место в буферной памяти. Для такой модели 
существует более общая формула для вычисления вероятности блокировки 
системы для $v$-заявок (см.~[11, 12]), которая задает зависимость вероятности 
блокировки узла в виде функции от вероятностей блокировок последующих 
узлов~$B_{v^+}$, $v\in \Omega_u^+$, при заданных значениях остальных 
параметров, в частности~$\Lambda_v$, $v\in\Omega_u^+$.
     
     В сети с повторами из предыдущего узла при установившемся режиме 
работы справедливы следующие уравнения баланса потоков:
     \begin{align*}
     \lambda (l) & = \Lambda_v^*(l) (1-\pi_v)\delta_v\,,\quad l\in L_v\,,\\
     \Lambda_v & = \sum\limits_{l\in L_v} \lambda_v(l)\,,\quad 
\Lambda_v^*=\sum\limits_{l\in L_v} \Lambda_v^*(l)\,,\quad v\in V\,.
     \end{align*}
     
     Тогда с учетом введенных выше обозначений и формул~(1)--(7) и~(18) 
из~[12], заменив обозначение~$B_{v^+}$ на~$y_{v^+}$, $v\in\Omega_u^+$, 
можем написать систему нелинейных уравнений
     \begin{align}
     y_u &= q_{N_u}(\overline{\rho}_u,y_u)\ \mbox{при схеме распределения 
CP
     }\,,\\
     y_v &= q_{N_u}(\rho_v,y_v)\ \mbox{ при схеме распределения CS
     },\, \ v\in\Omega_u^+\,\ u\inU\,,
     \end{align}
где компоненты~$\rho_v$ набора $\overline{\rho}_u$~--- функции 
переменных~$y_{v^+}$, $v\in \Omega_u^+$.

     Нетрудно видеть, что системы~(14) и~(15) обладают всеми свойствами 
системы~(6), использованными при доказательстве утверждений~1, 2, 3 и 
следствий, т.\,е.\ для систем~(13) и~(14) также справедливы утверждения~1, 2, 
3 и следствия.
     
\section{Алгоритм расчета}
     
     Для вычисления характеристик потоков в узлах и вероятностей 
блокировок пакетов предлагается следующий алгоритм, описывающий 
изложенную выше итерационную процедуру. Для описания значений, 
вычисляемых на $k$-м шаге алгоритма, к обозначениям соответствующих 
параметров приписывается знак~$[k]$. Введем новые обозначения:
     
     \noindent
     $y_u^v$~--- вероятность отсутствия блокировки узла $u\in U$ для 
пакетов, направляемых на линию $v\in \Omega_u^+$;
     \begin{align*}
     y_u^v & = \begin{cases}
     y_u & \mbox{для}\ v\in\Omega_u^+\ \mbox{при схеме}\ CS\,,\\
     y_v & \mbox{при схеме распределения CP}\,;
     \end{cases}\\
     \overline{\rho}_u^v & = 
     \begin{cases}
     \overline{\rho}_u & \mbox{для}\ v\in\Omega_u^+\ \mbox{при схеме}\ 
CS\,,\\
     \rho_v & \mbox{при схеме распределения CP}\,;
     \end{cases}\\
     q_{N_u}^v(\overline{\rho}_u^v, y_u^v) & = 
     \begin{cases}
     q_{N_u}(\overline{\rho}_u,y_u) & \mbox{для}\ v\in\Omega_u^+\ 
\mbox{при схеме}\ CS\,,\\
     q_{N_v}(\rho_v,y_v)& \mbox{при схеме распределения CP}\,;
     \end{cases}
     \end{align*}
     
     Тогда система уравнений для смешанного варианта сети, аналогичная 
системам~(6), (13)--(15), записывается в виде
     $$
     y_u^v = q_{N_u}^v(\overline{\rho}_u^v,\overline{y}_u^v)\,,\quad u\in 
U\,,\quad v\in\Omega_u^+\,.
     $$

\textbf{Шаг 1.} \textit{Инициализация}. Вычисление начальных значений 
параметров~$\rho_v^*$, $v\in V$: $\Lambda_v[0]=\sum\limits_{l\in L_v} 
\lambda(l)/((1-\pi_{u_{S_l+1}}(l)\prod\limits_{v^\prime\in 
V^+}\delta_{v^\prime}$, $\rho_v^*[0]=\Lambda_v[0]/\mu_v$, $v\in V$, 
$y_u^v[0]=1$, $u\in U$, $v\in\Omega_u^+$.
     
     \textbf{Шаг} $k$ ($k>1$).
     \begin{enumerate}[1.]
\item \textit{Проверка необходимых условий существования решения}. 
Если для некоторой линии $v\in V$ выполняется условие 
$c_v\mu_v/(\Lambda_v[k-1])\leq 1$, то алгоритм заканчивает работу с 
результатом <<система не имеет решения>>. Если в некотором узле~$u$, 
в котором используется полнодоступная схема, условие 
$c_v\mu_v/(\Lambda_v[k-1])> 1$ выполняется для всех $v\in\Omega_u^+$, 
то проверяется условие~(7) заданных $\Lambda_v[k-1]$, $v\in V$, и при 
невыполнении этого условия алгоритм заканчивает работу с результатом 
<<система не имеет решения>>.
     \item \textit{Вычисление вероятностей блокировок}. Используя 
значения параметров $\overline{\rho}_u^v[k-1]$, $y_u^v[k-1]$, $u\in U$, 
$v\in\Omega_u^+$, с помощью соответствующих формул~(3)--(5) или 
формул~(1)--(7) и~(18) из~[12] (в зависимости от типа схемы 
распределения буферной памяти и процедуры повторов передач) 
вычисляется $y_u^v[k]=1-\pi_u[k]$, $u\in U$, $v\in\Omega_u^+$. При этом 
рекомендуется использовать метод свертки Базена (см.~[13]), 
позволяющий производить рекуррентные вычисления (подробно этот 
метод описан также в~[2, 9]).
     \item \textit{Вычисление значений параметров} $\Lambda_v[k]$, $v\in V$:
     \begin{enumerate}[$i$)]
     \item в случае повторов от источника
     \begin{gather*}
     \Lambda_{v_{S_l}}[k]=\lambda(l)\,,\ \Lambda_{v_i}^*(l)[k]=
     \fr{\Lambda_{v_i}(l)[k]}{y^{v_i}_{u_i}[k-1]\delta_{v_i}}\,,
     \Lambda_{v_i-1}(l)[k]=\Lambda_{v_i}^*(l)[k]\,,\ i=1,\ldots ,S_l\,,\ l\in 
L\,,\\
     \Lambda_v^=[k] = \sum\limits_{l\in L_v} \Lambda_v^=(l)[k]\,,\quad 
v\in V\,;
     \end{gather*}
     \item в случае повторов из предыдущего узла
     \begin{equation*}
     \Lambda_v^*[k]=\fr{\Lambda_v[0]}{y_u^v[k-1]\delta_v}\,,\quad 
v\in\Omega_u^+\,,\quad u\in U\,.
     \end{equation*}
     \end{enumerate}
     \item \textit{Проверка условий останова алгоритма}. Если хотя бы для 
одной $v\in V$ для заданного значения точности $\varepsilon >0$ выполняется 
условие
     $$
     \fr{\vert \Lambda_v^*[k]-\Lambda_v^*[k-1]\vert}{\Lambda_v^*[k]} 
>\varepsilon\,,
     $$
     то вычисляются параметры $\overline{\rho}_u^v[k]$, $u\in U$, 
$v\in\Omega_u^+$, и осуществляется переход к шагу~$k$, положив $k$ 
равным~$k+1$, иначе алгоритм завершает работу.
     \end{enumerate}
     
     По завершении алгоритма либо выявляется, что система уравнений не 
имеет положительного решения, либо вычисляются интенсивности потоков, 
поступающих в узлы и на линии сети, и вероятности блокировок узлов для 
пакетов. Далее эти характеристики могут быть использованы для вычисления 
других характеристик сети (средних задержек, среднего числа повторов в узлах, 
узких участков сети и~др.).
     
\section{Примеры расчета}

     В качестве примера рассматривается сеть с тремя узлами, топология 
которой задается графом, показанным на рис.~1. В рассматриваемой сети 
предполагается полнодоступная схема распределения буферной памяти и 
процедура повторных передач из источника. Для вычисления вероятностей 
блокировок узлов и интенсивностей потоков, поступающих на линии связи, 
был использован алгоритм, представленный в разд.~5, и имитационная модель 
сети. В табл.~1 и на рис.~2 приведены результаты вычислений при следующих 
значениях входных параметров: емкости накопителей $N_u = 15$ для всех 
узлов, множество путей
     $$
     L = \{l_1, l_2, l_3, l_4, l_5,l_6\}\,,\  l_1=\{v_1\}\,,\ l_2=\{v_2\}\,,\ 
l_3=\{v_3\}\,,\ l_4=\{v_2,v_3\}\,,\ l_5=\{v_1,v_2\}\,,\ l_6=\{v_3,v_1\}\,,
     $$
     интенсивности первичных потоков $\lambda (l) =2$, 2,5, 2,7, 2,8, 2,9, 3, 
3,1, 3,2, $l \in L$.  Строки~1, 3 в табл.~1 и графики~\textsl{1}, \textsl{3} на 
рис.~2 соответствуют вариантам расчетов с помощью предложенного 
алгоритма, а \textsl{2}, \textsl{4}, \textsl{5}~--- с помощью имитационной 
модели. В вариантах~1 и~2 канальные емкости $c_v = 10$ для всех линий, 
параметр экспоненциального времени обслуживания   для всех линий, 
интервал повтора для всех заявок равен~10, в вариантах~3 и~4 емкости $c_v = 
1$ для всех линий, параметр экспоненциального времени обслуживания $\mu_v 
= 10$ для всех линий, интервал повтора равен~10, в варианте~5 емкости $c_v = 
10$ для всех линий, время обслуживания пакета каналом связи равно~10, 
интервал повтора равен~10. Во всех вариантах первичные потоки~--- 
пуассоновские, $\pi_{u_{S_l+1}}(l) =0$, $\delta_v =1$ $v\in V$, $l\in 
\begin{figure*} %fig1
     \Caption{Граф сети
     \label{f1ag}}
     \end{figure}
     
\begin{table}\small
\begin{center}
\Caption{Зависимость вероятности отсутствия блокировки узла от интенсивности первичных 
потоков
\label{t1ag}}
\vspace*{1ex}

\begin{tabular}{ccccccccc}
\hline
& \multicolumn{8}{$\lambda (l)$\\
\cline{2-9}
    
&2&2.5&2.7&2.8&2.9&3&3.1&3.2\\
\hline
1&0,9967&0,9758&0,9504&0,9272&0,8825&0,0000&0,0000&0\\
2&0,9905&0,9882&0,9257&0,9211&0,6928&0,0000&0,0000&0\\
3&0,9998&0,9964&0,9904&0,9844&0,9746&0,9568&0,9018&0\\
4&1,0000&0,9954&0,9934&0,9873&0,972&0,949&0,8787&0\\
5&0,9986&0,9917&0,9718&0,9677&0,9569&0,8018&0,0000&0\\
     \hline
     \end{tabular}
     \end{center}
     \end{table}
     
     Как показывают результаты, отраженные в табл.~1 и на рис.~2, а также 
другие вычислительные эксперименты, оценки вероятностей блокировок узлов, 
полученные с помощью представленного алгоритма, дают вполне приемлемые 
по точности значения для предварительного анализа сети на реализуемость 
первичных потоков пакетов.
     
     \begin{figure} %fig2
     \Caption{Зависимость вероятности отсутствия блокировки узла от 
интенсивности первичных потоков
     \label{f2ag}}
     \end{figure}
     
     Кроме того, точность результатов, полученных с помощью предлагаемого 
итерационного метода, увеличивается с ростом разветвленности сети и 
увеличением интервала повторов передач пакета. Эксперименты также 
показывают, что, как правило, погрешность, вносимая заменой многоканальной 
линии одноканальной с равной пропускной способностью, больше, чем 
вносимая предположением о пуассоновских входных потоках и 
экспоненциальных временах обслуживания.
     
\section{Заключение}
     
     Проведенные исследования показали, что алгоритм расчета сетей, 
предложенный в данной статье, обладает следующими достоинствами:
     \begin{enumerate}[1.]
     \item Использует в качестве модели сети СеМО, представляющие собой 
совокупность общепринятых СМО типа $
      \begin{matrix}
      M \\ \vec{\lambda}
      \end{matrix}
      \left |
      \begin{matrix}
      M \\ \vec{\lambda}
      \end{matrix}
      \right |
\vec{m} \vert N$ со схемами распределения CS или CP, связанных 
уравнениями баланса потоков в узлах.
\item При реализуемых первичных потоках сходится к положительному 
решению системы уравнений баланса потоков в узлах.
\item При реализуемых первичных потоках вычисляет вероятности 
блокировок узлов и загруженности узлов и каналов связи с приемлемой 
для предварительного анализа сети точностью (относительная 
погрешность вероятности блокировки $\sim 0.1$) .
\item Позволяет определить реализуемость первичных входных потоков.
\item При использовании алгоритма Базена требует для выполнения 
одного шага порядка $\sum\limits_{u\in U} (N_uK_u+N_u^2/2)$ 
арифметических операций, где $K_u$~--- степень узла~$u$.
    \end{enumerate}


{\small\frenchspacing
{%\baselineskip=10.8pt
\addcontentsline{toc}{section}{Литература}
\begin{thebibliography}{99}    
\bibitem{1ag}
\Au{Клейнрок~Л.}
Теория массового обслуживания.~--- М.: Машиностроение, 1979.

\bibitem{2ag}
\Au{Башарин~Г.\,П., Бочаров~П.\,П., Коган~Я.\,А.}
Анализ очередей в вычислительных сетях.~--- М.: Наука, 1989.

\bibitem{3ag}
\Au{Бочаров~П.\,П., Д'Апиче~Ч., Мандзо~Р., Фонг~Н.\,Х.}
Об обслуживании многомерного пуассоновского потока на одном 
приборе с конечным накопителем и повторными заявками~// Проблемы 
передачи информации, 2001. Т.~37. Вып.~4. С.~130--140.

\bibitem{4ag}
\Au{Tsitsiashvili~G.\,Sh., Osipova~M.\,A.}
Construction of queueing networks with stationary product distributions~// 
Proceedings of 5th Workshop (International ) on Retrial Queues.~--- Seoul: 
Korea University, 2004. Р.~111--115.

\bibitem{5ag}
\Au{Моисеева~С.\,П., Морозова~А.\,С., Назаров~А.\,А.}
Исследования СМО с повторным обращением и неограниченным числом 
обслуживающих приборов методом предельной декомпозиции~// 
Вычислительные технологии, 2008. Т.~13. Спец. вып.~5. С.~88--92.

\bibitem{6ag}
\Au{Wuechner~P., Meer~H., Bolc~G., Roszik~J., Sztrik~J.}
Modeling finite-source retrial queueing systems with unreliable heterogeneous 
servers and different service policies using MOSEL~// Proceedings of the 14th 
International conference on analytical and stochastic modeling techniques and 
applications, 2007, Prague, Czech Republic.~--- Sbr.-Dudweiler: Digitaldruck 
Pirrot GmbH, 2007. P.~75--80.

\bibitem{7ag}
\Au{Artalejo~J., G\'{o}mez-Corral~A.}
Retrial queueing systems. A computational approach.~--- Berlin: Springer 
Berlin Heidelberg, 2008.

\bibitem{8ag}
\Au{Бочаров~П.\,П.}
Приближенный метод расчета разомкнутых неэкспоненциальных сетей 
массового обслуживания конечной емкости с потерями или 
блокировками~// Автоматика и телемеханика, 1987. №\,1. C.~55--65.

\bibitem{9ag}
\Au{Вишневский~В.\,М.}
Теоретические основы проектирования компьютерных сетей.~--- М.: 
Техносфера, 2003.

\bibitem{10ag}
\Au{Таранцев~А.\,А.}
Инженерные методы теории массового обслуживания.~--- М.: Наука, 
2007.

\bibitem{11ag}
\Au{Kamoun~F., Kleinrock~L.}
Analysis of shared finite storage in a computer networks node environment 
under general traffic conditions~// IEEE Trans. on Commun., 1980. Vol.~28. 
No.\,7. P.~992--1003.

\bibitem{12ag}
\Au{Агаларов~Я.\,М.}
Приближенный метод вычисления характеристик узла 
телекоммуникационной сети с повторными передачами~// Информатика 
и её применения, 2009. Т.~3. Вып.~2. С.~2--10.

\label{end\stat}


\bibitem{13ag}
\Au{Buzen~J.\,P.}
Computational algorithm for closed queuing networks with exponential 
servers~// Communications ACM, 1973. Vol.~16. No.\,9. P.~527--531.

 \end{thebibliography}
}
}
\end{multicols}  %8
\def\stat{kudr}

\def\tit{ПРИБЛИЖЕННЫЕ МЕТОДЫ РЕШЕНИЯ ЗАДАЧИ ДИАГНОСТИКИ ПЛОСКИМ 
ЗОНДОМ СИЛЬНОИОНИЗОВАННОЙ ПЛАЗМЫ С~УЧЕТОМ КУЛОНОВСКИХ 
СТОЛКНОВЕНИЙ}

\def\titkol{Приближенные методы решения задачи диагностики плоским 
зондом сильноионизованной плазмы} %с~учетом Кулоновских  столкновений}

\def\autkol{И.\,А.~Кудрявцева, А.\,В.~Пантелеев}
\def\aut{И.\,А.~Кудрявцева$^1$, А.\,В.~Пантелеев$^2$}

\titel{\tit}{\aut}{\autkol}{\titkol}

%{\renewcommand{\thefootnote}{\fnsymbol{footnote}}\footnotetext[1]
%{Работа поддержана Российским фондом фундаментальных исследований
%(проекты 11-01-00515а и 11-07-00112а), а также Министерством
%образования и науки РФ в рамках ФЦП <<Научные и
%научно-педагогические кадры инновационной России на 2009--2013~годы>>.}}


\renewcommand{\thefootnote}{\arabic{footnote}}
\footnotetext[1]{Московский авиационный институт, irina.home.mail@mail.ru}
\footnotetext[2]{Московский авиационный институт, avpanteleev@inbox.ru}

\vspace*{-2pt}

\Abst{Сформирована математическая модель, описывающая динамику сильноионизованной 
плазмы с учетом столкновений заряженных частиц вблизи плоского зонда. Модель включает уравнение 
Фоккера--Планка и уравнение Пуассона. Предложено два подхода к решению задачи: на основе метода 
статистических испытаний Мон\-те-Кар\-ло и на основе композиции метода крупных частиц и метода 
расщепления.} 

\vspace*{-2pt}

\KW{телекоммуникационные системы; метод Монте-Карло; метод крупных частиц; метод 
расщепления; зонд; уравнение Фоккера--Планка; уравнение Пуассона} 

\vspace*{-4pt}

 \vskip 8pt plus 9pt minus 6pt

      \thispagestyle{headings}

      \begin{multicols}{2}
      
            \label{st\stat}

\section{Введение}

В настоящее время в области телекоммуникаций все более востребованными становятся 
информационные технологии, основанные на использовании математических моделей и численных 
методов физики плазмы. Поэтому особенно актуальным является решение разнообразных задач анализа 
поведения плазмы, включающих в себя формирование новых моделей и методов их исследования. 
Помимо этого, в разработке телекоммуникационного оборудования эффективно используются 
собственно физические свойства плазмы. В~частности, изготовлена антенна, работа которой основана 
на газовом разряде низкотемпературной плазмы~[1], интенсивно ведутся разработки по созданию и 
усовершенствованию источников бесперебойного питания на основе плазменных элементов~[2, 3]. 
      
      Одним из наиболее перспективных направлений для построения систем оптической 
беспроводной связи является использование лазеров~\cite{4-k, 5-k}. В~этой связи большое внимание 
уделяется использованию плазмы при разработке импульсных сильноточных коммутаторов~\cite{6-k}, 
так как практическое применение подобных разработок требует повышения уровня надежности и 
быстродействия лазерных систем.
      
      Исследования низкотемпературной плазмы также связаны с разработками в области дальней 
космической связи, так как моделирование процессов взаимодействия заряженного тела с верхними 
слоями атмосферы позволяет предлагать способы улучшения существующих систем радиосвязи с 
космическими летательными аппаратами~\cite{7-k}. 
      
      Наряду с этим актуальными также являются задачи диагностики плазмы, поскольку перспективы 
ее использования в области телекоммуникаций после более полного изучения физических свойств 
могут значительно расшириться. 

Для диагностики плазмы применяют зондовые методы исследования~[8--11]. Эти методы относятся к 
классу контактных методов; как следствие, возникает сложность в исследовании пристеночной области 
вблизи зонда, которая характеризуется достаточно сложным распределением потенциала и функциями 
распределения, отличными от максвелловских. 

Данная работа посвящена исследованию переходного режима обтекания заряженного тела плазмой. Для 
переходного режима выполняется следующее условие: длина свободного пробега иона до столкновения 
с нейтральным атомом или другим ионом невелика по сравнению с характерными размерами тела. 
В~этом случае возникает необходимость учета столкновений заряженных частиц с нейтральными 
атомами и кулоновских столкновений. В~работах~\cite{10-k, 11-k} подробно рассмотрена модель с 
учетом столкновений заряженных частиц с нейтральными атомами. В~настоящей статье представлена 
теоретическая модель, описывающая влияния ион-ионных и ион-элек\-т\-рон\-ных столкновений на 
измеряемые характеристики плазмы, что ранее детально не исследовалось.
      
      В~рамках данной работы предлагается модель, описывающая динамику сильноионизованной 
плазмы с учетом кулоновских столкновений. Эта модель учитывает такие процессы взаимодействия, 
как перенос частиц и столкновения между заряженными частицами типа <<ион--ион>> и 
      <<ион--электрон>> под влиянием макроскопического электрического поля. Перечисленные 
процессы описываются самосогласованной системой уравнений, включающей уравнение 
      Фок\-ке\-ра--План\-ка и уравнение Пуассона~[12].
      
      Вычислительная модель задачи строится на основе двух методов: метода статистических 
испытаний Мон\-те-Кар\-ло и композиции метода крупных частиц и метода расщепления. Приведены 
результаты численного моделирования, полученные с использованием вышеперечисленных методов.

\vspace*{-4pt}

\section{Постановка задачи}

\vspace*{-2pt}

Рассматривается следующая физическая постановка зондовой задачи~[11]. В~невозмущенную 
бесконечно протяженную плазму, состоящую из электронов и однозарядных ионов, внесена большая\linebreak 
заряженная до потенциала $\varphi_p$ плоскость. Плоскость, расположенная поперек потока плазмы, 
является идеально поглощающей для электронов. Ионы при ударе о плоскость нейтрализуются. 
Предполагается, что частицы в плазме движутся под действием внешнего электрического поля, 
магнитное поле отсутствует. Концентрации ионов $n_{i\infty}$ и электронов $n_{e\infty}$, а также 
температуры данных час\-тиц~$T_{i\infty}$ 
и~$T_{e\infty}$ в невозмущенной плазме заданы. За начальные 
функции распределения обоих типов час\-тиц принимаются функции распределения Максвелла. 
      
      Требуется с учетом столкновений между заряженными частицами найти напряженность 
самосогласованного электрического поля $\vec{E}(\vec{r},t)$, функции распределения однозарядных 
ионов $f_i(\vec{r}, \vec{v}, t)$ и электронов $f_e(\vec{r}, \vec{v}, t)$, 
а также их моменты (плотности 
токов ионов и электронов  $j_i(\vec{r},t)\hm
=q\int f_i(\vec{r}, \vec{v}, t)\vec{v}\,d\vec{v}$, $j_e(\vec{r},t) 
\hm={\sf e}\int f_e(\vec{r},\vec{v},t)\vec{v}\,d\vec{v}$, где $q=Z_i{\sf e}$, $Z_i=1$~--- заряд иона, ${\sf 
e}$~--- заряд электрона; концентрации ионов и электронов $n_i(\vec{r},t)\hm=\int 
f_i(\vec{r},\vec{v},t)\,d\vec{v}$, $n_e(\vec{r},t)\hm=\int f_e(\vec{r},\vec{v}, t)\,d\vec{v}$). 
Поведение частиц во 
времени~$t$ характеризуется ра\-ди\-ус-век\-то\-ром~$\vec{r}$ и вектором скорости~$\vec{v}$.
      
      Математическая модель, соответствующая данной физической постановке задачи, имеет 
вид~\cite{11-k, 13-k}:

\noindent
      \begin{equation}
      \left.
      \begin{array}{c}
      \fr{\partial f_\alpha (\vec{r},\vec{v},t)}{\partial t}+
      \vec{v}\fr{\partial f_\alpha (\vec{r},\vec{v},t)}{ 
\partial \vec{r}}+
\fr{\vec{F}_\alpha(\vec{r},t)}{m_\alpha}\times{}\\[4pt]
{}\times\fr{\partial f_\alpha(\vec{r},\vec{v},t)}{ \partial 
\vec{v}}=
\left(\fr{\partial f_\alpha(\vec{r},\vec{v},t)}{ \partial t}\right)_{\mathrm{с}}+S_\alpha 
(\vec{r},\vec{v},t)\,;\\[6pt]
      \Delta\varphi(\vec{r},t)=-\fr{{\sf e}}{\varepsilon_0}\left( n_i(\vec{r},t)-n_e(\vec{r},t)\right)\,;\\[6pt]
      \vec{E}(\vec{r},t)=-\nabla \varphi(\vec{r},t)\,.
      \end{array}\!\!
      \right\}\!\!
      \label{e1-k}
      \end{equation}
Здесь первое уравнение~--- уравнение Фок\-ке\-ра--План\-ка для частиц сорта~$\alpha$ ($\alpha=i,e$), 
второе~--- уравнение Пуассона для самосогласованного электрического поля; 
$f_\alpha(\vec{r},\vec{v},t)$~--- функция\linebreak
распределения час\-тиц сорта~$\alpha$; $(\partial 
f_\alpha(\vec{r},\vec{v},t)/\partial t)_{\mathrm{с}}$~--- 
оператор столкновений Фок\-ке\-ра--План\-ка; 
функция~$S_\alpha(\vec{r},\vec{v},t)$ описывает источники или стоки\linebreak
 час\-тиц; 
$\vec{F}_\alpha(\vec{r},t)=q_\alpha\vec{E}(\vec{r},t)$, где $\vec{E}(\vec{r},t)$~--- напряженность 
самосогласованного электрического поля, 
$$
q_\alpha =
\begin{cases}
-{\sf e}\,, & \alpha=e\,,\\
{\sf e}\,, & \alpha=i\,;
\end{cases}
$$
$\varphi(\vec{r},t)$~--- потенциал самосогласованного электрического поля; $n_\alpha(\vec{r},t)$ ($\alpha 
\hm=i,e$)~--- концентрация частиц сорта~$\alpha$; $m_\alpha$~--- масса частицы сорта~$\alpha$; 
$\varepsilon_0$~--- электрическая постоянная. 

Оператор столкновений Фок\-ке\-ра--План\-ка имеет вид~\cite{13-k, 14-k}
\begin{multline*}
\fr{1}{\Gamma_\alpha}\left( \fr{\partial f_\alpha}{\partial t}\right)_{\mathrm{с}} 
=\fr{1}{2}\,\nabla_v\nabla_v:\left(f_\alpha\nabla_v\nabla_vg_\alpha(\vec{r},\vec{v},t)\right)-{}\\
{}-
\nabla_v\cdot\left(f_\alpha\nabla_v h_\alpha\right)\,,
\end{multline*}
где $\nabla_v\nabla_v g_\alpha(\vec{r},\vec{v},t)$~--- ковариантная тензорная производная второго ранга, 
знак двоеточия ($:$) обозначает операцию двойного суммирования:
\begin{gather*}
\Gamma_\alpha=\fr{Z_\alpha^4 {\sf e}^4}{4\pi \varepsilon_0^2 m^2_\alpha}\,\ln D_\alpha\,;
\\
D_\alpha =\fr{12\pi\varepsilon_0 kT_{\alpha\infty}}{Z_\alpha^2 {\sf e}^2}\left( \fr{\varepsilon_0 k 
T_{e\infty}}{n_{e\infty} {\sf e}^2}\right)^{1/2}\,;\\
g_\alpha (\vec{r},\vec{v},t)=\sum\limits_{b=i,e}\left( \fr{Z_b}{Z_\alpha}\right) \int f_b 
(\vec{r},{\vec{v}}^{\,\prime},t)\left\vert \vec{v}-{\vec{v}}^{\,\prime}\right\vert\,d\vec{v}^{\,\prime}\,;\\
h_\alpha (\vec{r},\vec{v},t)=\sum\limits_{b=i,e} \fr{m_\alpha+m_b}{m_b} 
\left(\fr{Z_b}{Z_\alpha}\right)
\int
\fr{f_b(\vec{r},{\vec{v}}^{\,\prime}, t)}{\vert \vec{v}-{\vec{v}}^{\,\prime}\vert}
\,d{\vec{v}}^{\,\prime}\,;\\
Z_\alpha =1\,, \quad \alpha=i,e\,.
\end{gather*}
 
К системе уравнений~(\ref{e1-k}) необходимо добавить начальные и краевые условия:
\begin{equation}
\!\left.
\begin{array}{rrl}
t=0:\ & f_\alpha(\vec{r},\vec{v},0)&=f_\alpha^{\mathrm{maksv}}\,,\enskip \alpha=i,e;\\[9pt]
\vec{r}\in \Omega_p:\ & f_\alpha(\vec{r},\vec{v},t)\big\vert_{\vec{r}\in\Omega_p}&=0\,,\enskip \alpha=i,e\,;\\[9pt]
&\varphi(\vec{r},t)\big\vert_{\vec{r}\in\Omega_p}&=\varphi_p\,;\\[9pt]
\vec{r}\in\Omega_\infty:\ & 
f_\alpha(\vec{r},\vec{v},t)\big\vert_{\vec{r}\in\Omega_\infty}&= %{}\\[9pt]
f_\alpha^{\mathrm{maksv}}\,,\enskip \alpha=i,e\,;\\[9pt]
&\varphi(\vec{r},t)\big\vert_{\vec{r}\in\Omega_\infty}&=0\,,
\end{array}\!\!
\right\}\!\!\!\!
\label{e2-k}
\end{equation}
    где 
    
    \noindent
    \begin{multline*}
    f_\alpha^{\mathrm{maksv}}=n_{\alpha\infty}\left(\fr{m_\alpha}{2k\pi T_{\alpha\infty}}\right)^{3/2}\times{}\\
    {}\times
    \exp\left( -
\fr{m_\alpha}{2kT_{\alpha\infty}}\left\vert\vec{v}-\vec{v}_\infty\right\vert^2\right)\,,
\enskip \alpha=i, e\,;
\end{multline*} 
$\Omega_p$ и $\Omega_\infty$~--- множество радиус-векторов час\-тиц, концы которых принадлежат плоскости зонда и 
границе возмущенной зоны соответственно.

Для решения поставленной задачи введем декартову систему координат таким образом, чтобы 
заряженная плоскость совпала с плоскостью~$0xz$. Тогда положение частицы в пространстве будет 
определяться координатами $x,y,z$, а скорость~--- координатами $v_x, v_y, v_z$. В~силу того что 
плоскость является бесконечно большой в сравнении с характерным размером задачи, функции 
распределения частиц будут зависеть только от переменных $y, v_y, t$.

Поставленную задачу предлагается решать независимо двумя методами. Первый метод основывается на 
методе статистических испытаний Мон\-те-Кар\-ло, второй метод является композицией метода 
расщепления и метода крупных частиц.

\section{Применение метода Монте-Карло}

Запишем самосогласованную систему уравнений~(\ref{e1-k}) и~(\ref{e2-k}) в декартовой системе 
координат с учетом сделанных предположений:
\begin{equation}
\left.
\begin{array}{l}
\fr{\partial f_\alpha}{\partial t}+
v_y\fr{\partial f_\alpha}{\partial y}+\fr{F_y^\alpha}{m_\alpha}\,\fr{\partial 
f_\alpha}{\partial v_y}=\fr{1}{2}\,\fr{\partial^2 }{\partial [v_y]^2}\times{}\\
{}\times \left( 
f_\alpha\fr{\partial^2 g_\alpha  }{\partial [v_y]^2}\right) -
\fr{\partial}{\partial v_y}\left( f_\alpha\fr{\partial h_\alpha}{\partial v_y}\right)\,,
\enskip \alpha=i,e\,;\\[6pt]
    \fr{\partial^2\varphi}{\partial y^2} =-\fr{{\sf e}}{\varepsilon_0}\left(n_i-n_e\right)\,;
    \enskip E_y=-
\fr{\partial\varphi}{\partial y}\,;\\[6pt]
\hspace*{3.1mm}    t=0:\  \hspace*{2.6mm}f_\alpha(y,v_y,0)=f_\alpha^{\mathrm{maksv}}\,,\ \alpha=i,e\,;\\[9pt]
\hspace*{2.9mm} y=0:\ \hspace*{2.8mm}f_\alpha(0,v_y,t)=0\,,\ \alpha=i,e\,;\\[9pt]
\hspace*{24.3mm}\varphi(0,t)=\varphi_p\,;\\[9pt]
y=y_\infty:\ f_\alpha(y_\infty, v_y, t)=f_\alpha^{\mathrm{maksv}}\,,\ \alpha=i,e\,;\\[9pt]
\hspace*{21.5mm}\varphi(y_\infty, t)=0\,.
\end{array}
\right \}
\label{e3-k}
\end{equation}

В полученной системе уравнений~(\ref{e3-k}) перейдем к безразмерным величинам, применив 
соотношение $X=M_X \hat{X}$, где $M_X$~--- масштаб размерной величины~$X$, $\hat{X}$~--- 
безразмерная величина~$X$. В~качестве используемых масштабов были взяты следующие: радиус 
Дебая, скорость теплового движения частиц, концентрация частиц в невозмущенной плазме, потенциал, 
возникающий при разделении зарядов в дебаевской сфере, и производные от них величины.

Система безразмерных уравнений имеет следующий вид:
%\noindent
\begin{equation}
\left.
\begin{array}{l}
\fr{\partial 
\hat{f}_\alpha}{\partial\hat{t}}+A_\alpha\fr{\partial\hat{f}_\alpha}{\partial\hat{y}}+
B_\alpha\hat{E}_y\fr{\partial\hat{f}_\alpha}{\partial \hat{v}_y}={}\\
\!{}=
\fr{\partial^2}{\partial[\hat{v}_y]^2}\left(D_\alpha 
\hat{f}_\alpha\right)-\fr{\partial}{\partial\hat{v}_y}\left(K_\alpha \hat{f}_\alpha\right),\enskip 
\alpha=i,e;\\[9pt]
\fr{\partial^2\hat{\varphi}}{\partial\hat{y}^2}=-\left(\hat{n}_i-\hat{n}_e\right)\,;\enskip \hat{e}_y=-
\fr{\partial\hat\varphi}{\partial\hat{y}}\,;\\[9pt]
\hspace*{3.1mm}\hat{t}=0:\ \hspace*{2.6mm}\hat{f}_\alpha(\hat{y},\hat{v}_y,0)=\hat{f}_\alpha^{\mathrm{maksv}}\,,\enskip \alpha-i,e\,;\\[9pt]
\hspace*{2.9mm}\hat{y}=0:\ \hspace*{2.8mm}\hat{f}_\alpha(0,\hat{v}_y,\hat{t})=0\,,\enskip \alpha=i,e\,;\\[9pt]
\hspace*{24.3mm}\hat\varphi(0,\hat{t})=\hat{\varphi}_p\,;\\[9pt]
\hat{y}=\hat{y}_\infty:\ \hat{f}_\alpha(\hat{y}_\infty, \hat{v}_y, \hat{t})=\hat{f}^{\mathrm{maksv}}_\alpha\,,\enskip 
\alpha=i,e\,;\\[9pt]
\hspace*{21.5mm}\hat\varphi(\hat{y}_\infty,\hat{t})=0\,.
\end{array}
\right\}
\label{e4-k}
\end{equation}
Здесь 

\vspace*{-2pt}

\noindent
\begin{gather*}
A_\alpha=\sqrt{\delta_\alpha }\,\hat{v}_y\,;\enskip 
B_\alpha=\sqrt{\delta_\alpha}\,\fr{z_\alpha}{2\varepsilon_\alpha}\,;\\
\delta_\alpha=\fr{\varepsilon_\alpha}{\mu_\alpha}\,;\enskip 
\varepsilon_\alpha=\fr{T_{\alpha\infty}}{T_{i\infty}}\,;\\
\mu_\alpha=\fr{m_\alpha}{m_i}\,;\enskip 
D_\alpha=A_g^\alpha\fr{\partial^2\hat{g}_\alpha}{\partial  [\hat{v}_y]^2}\,;\\
K_\alpha=A_h^\alpha \fr{\partial \hat{h}_\alpha}{\partial \hat{v}_y}\,,\enskip \alpha=i,e\,,
\end{gather*}
где $A_g^\alpha$ и $A_h^\alpha$~--- коэффициенты, определяемые характерными параметрами 
задачи~\cite{15-k}.

Поиск решения самосогласованной системы уравнений~(\ref{e4-k}) осуществляется по следующей 
схе-\linebreak ме. Вначале находятся значения напряженности\linebreak
 электрического поля по значениям потенциала, 
полученным из граничной задачи для уравнения Пуассона. Далее, используя найденные значения 
напряженности, решается уравнение Фок\-ке\-ра--План\-ка путем перехода к стохастическому 
дифференциальному уравнению (СДУ) Ито:

\noindent
\begin{multline*}
d\Theta_\alpha(\hat{t}) = a_\alpha \left(\hat{t},\Theta_\alpha(\hat{t})\right)+{}\\
{}+\sigma\left(
\hat{t},\Theta_\alpha(\hat{t})\right)\,dW(\hat{t})\,,\quad \alpha=i,e\,,
%\label{e5-k}
\end{multline*}
где 

\noindent
\begin{align*}
\Theta_\alpha(\hat{t})&=\begin{bmatrix}
\hat{y}(\hat{t})\\ \hat{v}_y(\hat{t})
\end{bmatrix}\,;\\
a_\alpha\left(\hat{t},\Theta_\alpha(\hat{t})\right)&=\begin{bmatrix}
-A_\alpha\\ -K_\alpha -B_\alpha \hat{E}_y
\end{bmatrix}\,;\\
\sigma_\alpha\left(\hat{t},\Theta_\alpha(\hat{t})\right)\sigma_\alpha^{\mathrm{T}}\left( 
\hat{t},\Theta_\alpha(\hat{t})\right)&=D_\alpha\,,\enskip \alpha=i,e\,;
\end{align*} 
$W(\hat{t})$~--- стандартный винеровский случайный процесс.
\pagebreak

Для нахождения значений вектора состояния~$\Theta_\alpha(\hat{t})$ применим явную разностную 
схему стохастического метода Эйлера~\cite{16-k}:
\begin{multline*}
\Theta_\alpha^{n+1}=\Theta_\alpha^n +h_\tau a_\alpha \left( \hat{t}_n, \Theta_\alpha^n\right)+\sigma_\alpha 
\left( \hat{t}_n, \Theta_\alpha^n\right)\Delta W_n\,,\\ 
n=0,\ldots , N\,,\ \alpha=i,e\,,
%\label{e6-k}
\end{multline*}
где $\Theta_\alpha^n$, $n=0,\ldots , N$,~--- приближенное значение вектора 
состояния~$\Theta_\alpha(\hat{t})$, $\alpha=i,e$, в момент времени $\hat{t}\hm=\hat{t}_n$, 
$\hat{t}_n\hm=n h_\tau$, $n=0,\ldots , N$; $h_\tau$~--- достаточно малый шаг интегрирования; $\Delta 
W_n$, $n=0,\ldots ,N$,~--- величина приращения винеровского процесса~$W(\hat{t})$ на отрезке $\left[ 
\hat{t}_n,\,\hat{t}_{n+1}\right]$, по определению независимая от~$\Theta_\alpha^0$, 
$\Delta W_0,\ldots , 
\Delta W_{n-1}$: $\Delta W_n\hm=W(\hat{t}_{n-1})\hm-W(\hat{t}_n)$; $\Delta W_n\hm\sim N(0,\,h_\tau)$, 
т.\,е.\ $\Delta W_n$ представляют собой гауссовские случайные величины с нулевыми математическими 
ожиданиями и дисперсиями, равными шагу интегрирования; $\Theta_\alpha^0$~--- значение вектора 
состояния $\Theta_\alpha(\hat{t})$, $\alpha\hm=i,e$, в момент времени $\hat{t}=0$, 
$\Theta_\alpha^0\hm\sim \hat{f}_\alpha^{\mathrm{maksv}}$. 

Частные производные $\partial^2\hat{g}_\alpha/\partial[\hat{v}_y]^2$ и $\partial \hat{h}_\alpha/\partial 
\hat{v}_y$, являющиеся составляющими матрицы $\sigma_\alpha (\hat{t}_n, 
\Theta_\alpha^n)\sigma_\alpha^{\mathrm{T}}(\hat{t}_n,\Theta_\alpha^n)$ и вектора $a_\alpha(\hat{t}_n, 
\Theta_\alpha^n)$ соответственно, аппроксимируются со вторым порядком точности на трехточечном 
шаблоне на основе значений~$\hat{g}_\alpha$ и~$\hat{h}_\alpha$~\cite{17-k}.
      
      В выражения для функций~$\hat{g}_\alpha$ и~$\hat{h}_\alpha$ входят интегралы, которые 
вычисляются методом Мон\-те-Кар\-ло с использованием набора значений скоростной компоненты 
вектора состояния~$\hat{v}_y$, полученных из решения СДУ Ито:
      \begin{equation*}
      \int \hat{f}_\alpha \left\vert \hat{v}_y-
\hat{v}_y^\prime\right\vert\,dv_y^\prime=M\left(\zeta\left(\hat{V}_y\right)\right)\,,
\end{equation*}
где
$$
      \zeta\left(\hat{V}_y\right)=\left\vert \hat{v}_y-\hat{V}_y\right\vert\,,\enskip \hat{V}_y\sim 
\hat{f}_\alpha\,.
  $$
      
      Для вычисления напряженности самосогласованного электрического поля $\hat{E}_y=-
\partial\hat{\varphi}/\partial\hat{y}$, входящей в вектор $a_\alpha(\hat{t}_n, \Theta_\alpha^n)$, необходимо 
аналогично аппроксимировать со вторым порядком точности производную 
$\partial\hat{\varphi}/\partial\hat{y}$ на трехточечном шаблоне с использованием значений 
потенциала~$\hat{\varphi}$~\cite{17-k}. Значения потенциала~$\hat\varphi$ находятся из решения 
уравнения Пуассона. 
      
      Граничную задачу для уравнения Пуассона 
      \begin{align*}
      \fr{\partial^2 \hat\varphi}{\partial \hat{y}^2} & = -\left(\hat{n}_i-\hat{n}_e\right)\,;\\
      \hat{\varphi}\big|_{\hat{y}=0} &=\hat{\varphi}_p\,;\\
      \hat{\varphi}\big|_{\hat{y}_\infty=0} &=0
      \end{align*}
    предлагается решать путем перехода к конечно-разностной системе с последующим ее решением 
методом прогонки~\cite{17-k}:

\noindent
\begin{gather*}
\hat{\varphi}^n_{l-1}+2\hat{\varphi}_l^n+\hat{\varphi}^n_{l+1}=
h_y\hat{\delta}_l^n\,,\enskip l=1,\ldots , 
N_y\,;\\
\hat{\delta}_l^n=-\left( \hat{n}^n_{i,l}-\hat{n}^n_{e,l}\right)\,;\enskip 
\hat{\varphi}_0=\hat{\varphi}_p\,;\enskip \hat{\varphi}_{N_y}=0\,,
\end{gather*}
где $N_y$~--- число шагов по переменной~$\hat{y}$, $h_y$~--- величина шагов разбиения по~$\hat{y}$. 
      
      Концентрации $\hat{n}_\alpha$, $\alpha=i,e$, и плотности токов частиц на зонд~$\hat{f}_\alpha$, 
$\alpha=i,e$, вычисляются согласно описанному выше методу Мон\-те-Карло.

\section{Применение метода расщепления и~метода крупных~частиц}

Решение задачи в данном случае предлагается начать с записи правой части уравнения 
Фок\-ке\-ра--План\-ка в декартовой системе координат в виде:
$$
\mathbf{Q} f_\alpha = \fr{1}{2}\,\fr{\partial^2 f_\alpha}{\partial [v_y]^2}\,\fr{\partial^2 g_\alpha}{\partial 
[v_y]^2}+\fr{\partial f_\alpha}{\partial v_y}\,\fr{\partial C_\alpha}{\partial v_y}+H_\alpha\,,\enskip 
\alpha=i,e\,,
$$  
где 
\begin{align*}
C_\alpha(\vec{r},\vec{v},t)&=
\begin{cases}
\fr{1-\gamma}{Z_i^2}\int\fr{f_e(\vec{r},{\vec{v}}^{\,\prime},t)}{|\vec{v}-{\vec{v}}^{\,\prime} |}\,d{\vec{v}}^{\,\prime}\,, 
&\alpha=i\,;\\[9pt]
\fr{Z_i^2(\gamma-1)}{\gamma}\int \fr{f_i(\vec{r},{\vec{v}}^{\,\prime}, t)}
{|\vec{v}-{\vec{v}}^{\,\prime} 
|}\,d{\vec{v}}^{\,\prime}\,, &\alpha=e\,;
\end{cases} 
\\
H_\alpha&=
\begin{cases}
4\pi \left( \fr{\gamma f_e}{Z_i^2}+f_i\right)f_i\,, & \alpha=i\,;\\[9pt]
4\pi\left(\fr{Z_i^2 f_i}{\gamma}+f_e\right)f_e\,, &\alpha=e\,.
\end{cases}
\end{align*}
Тогда при переходе к безразмерным величинам (см.\ разд.~3) система~(\ref{e1-k}) запишется 
следующим образом:
      \begin{equation}
      \left.
\!\!\begin{array}{l}
      \fr{\partial 
\hat{f}_\alpha}{\partial\hat{t}}+A_\alpha\fr{\partial\hat{f}_\alpha}{\partial\hat{y}}+
B_\alpha  \hat{E}_y
\fr{\partial\hat{f}_\alpha}{\partial\hat{v}_\alpha}=\tilde{\mathbf{Q}}\hat{f}_\alpha\,,\enskip 
\alpha=i,e;\\[9pt]
      \fr{\partial^2\hat{\varphi}}{\partial\hat{y}^2}=-\left( \hat{n}_i-\hat{n}_e\right)\,,\enskip \hat{E}_y=-
\fr{\partial\hat\varphi}{\partial\hat{y}}\,,\\[9pt]
\hspace*{3.1mm}\hat{t}=0:\ \hspace*{2.6mm}\hat{f}_\alpha(\hat{y},\hat{v}_y, 0)=\hat{f}_\alpha^{\mathrm{maksv}}\,,\enskip \alpha=i,e\,,\\[9pt]
\hspace*{2.9mm} \hat{y}=0:\ \hspace*{2.8mm}\hat{f}_\alpha(0,\hat{v}_y,\hat{t})=0\,,\enskip \alpha=i,e\,;\\[9pt]
\hspace*{24.3mm}\hat\varphi(0,\hat{t})=\hat{\varphi}_p\,;\\[9pt]
      \hat{y}=\hat{y}_\infty:\ \hat{f}_\alpha(\hat{y}_\infty, 
\hat{v}_y,\hat{t})=\hat{f}_\alpha^{\mathrm{maksv}}\,,\enskip \alpha=i,e\,;\\[9pt]
\hspace*{21.5mm}\hat{\varphi}(\hat{y}_\infty,\hat{t})=0\,,\\[9pt]
    \end{array}
\right\}\!\!
\label{e7-k}
\end{equation}
где 
\begin{gather*}
\tilde{\mathbf{Q}} \hat{f}_\alpha=D_\alpha\fr{\partial^2\hat{f}_\alpha}{\partial 
[\hat{v}_y]^2}+K_\alpha\fr{\partial\hat{f}_\alpha}{\partial\hat{v}_y}+H_\alpha\,;\\
D_\alpha=A_g^\alpha\fr{\partial^2\hat{g}_\alpha}{\partial [\hat{v}_y]^2}\,;\enskip 
K_\alpha=A_h^\alpha \fr{\partial \hat{h}_\alpha}{\partial\hat{v}_y}\,,\ \alpha=i,e\,.
\end{gather*}

Для решения системы уравнений~(\ref{e7-k}) применяется модификация метода 
расщепления~\cite{17-k}, согласно которой исходная задача разбивается на две вспомогательные. Такое 
разбиение можно осуществить, переписав уравнение Фок\-ке\-ра--План\-ка в следующем виде:
$$
\fr{\partial\hat{f}_\alpha}{\partial\hat{t}} =
\tilde{\mathbf{Q}}_1\hat{f}_\alpha+\tilde{\mathbf{Q}}_2\hat{f}_\alpha\,,
$$
где 
\begin{align*}
\tilde{\mathbf{Q}}_1\hat{f}_\alpha &=-
\left(A_\alpha\fr{\partial\hat{f}_\alpha}{\partial\hat{y}}+
B_\alpha\fr{\partial\hat{f}_\alpha}{\partial\hat{y}}
\right)\,;\\
\tilde{\mathbf{Q}}_2\hat{f}_\alpha 
&=\left(D_\alpha\fr{\partial^2\hat{f}_\alpha}{\partial[\hat{v}_y]^2}+K_\alpha\fr{\partial 
\hat{f}_\alpha}{\partial\hat{v}_y}+H_\alpha\right)\,.
\end{align*}

      Правая часть уравнения Фок\-ке\-ра--План\-ка представляет собой сумму двух операторов, 
первый из которых отвечает за перенос частиц, второй~--- за столкновения заряженных частиц. 
В~результате образуются следующие задачи, которые решаются последовательно:
      \begin{itemize}
\item первая задача:
\begin{align*}
&\fr{\partial w_\alpha(\hat{y},\hat{v}_y,\hat{t})}{\partial\hat{t}} =\mathbf{Q}_1 
w_\alpha(\hat{y},\hat{v}_y,\hat{t})\,,\enskip \alpha=i,e\,;\\[9pt]
&\fr{\partial^2\hat\varphi}{\partial\hat{y}^2}=-\left(\hat{n}_i-\hat{n}_e\right)\,;\enskip
\hat{E}_y=-
\fr{\partial\hat\varphi}{\partial\hat{y}}\,;\\[9pt]
&w_\alpha(\hat{y},\hat{v}_y,\hat{t}^n)=\hat{f}_\alpha(\hat{y},\hat{v}_y,\hat{t}^n)\,,\enskip n=0,\ldots ,N-
1\,;\\[9pt]
&\hspace{2.9mm}\hat{y}=0:\ \hspace*{2.9mm}w_\alpha(0,\hat{v}_y,\hat{t})=0\,,\enskip \alpha=i,e\,;\\[9pt]
&\hspace*{25.1mm}\hat\varphi(0,\hat{t})=\hat{\varphi}_p\,;\\[9pt]
&\hat{y}=\hat{y}_\infty:\ w_\alpha(\hat{y}_\infty, \hat{v}_y, \hat{t})=
\hat{f}_\alpha^{\mathrm{maksv}}\,,\enskip 
\alpha=i,e\,;\\[9pt]
&\hspace*{22.5mm}\hat\varphi(\hat{y}_\infty,\hat{t})=0\,;
\end{align*}
\item вторая задача:
\begin{align*}
\!\!\!\!\!\!\!\fr{\partial s_\alpha(\hat{y},\hat{v}_y,\hat{t})}{\partial \hat{t}} &=\mathbf{Q}_2 
s_\alpha(\hat{y},\hat{v}_y,\hat{t})\,, & \alpha&=i,e\,;\\
\!\!\!\!\!\!\!s_\alpha (\hat{y},\hat{v}_y,\hat{t}^n) &=w_\alpha (\hat{y},\hat{v}_y, \hat{t}^{n+1}),& n&=0,\ldots ,N-
1.
\end{align*}
\end{itemize}

Первая задача представляет собой систему безразмерных уравнений Вла\-со\-ва--Пуас\-со\-на. Для ее 
решения применяется метод крупных частиц~\cite{18-k}. Согласно этому методу решение задачи 
осуществляется путем расщепления на два этапа: на первом этапе не учитываются конвективные члены 
и решение получается обычным интегрированием на неподвижной эйлеровой сетке, а на втором этапе 
рассматривается система, которая описывает перенос частиц в лагранжевой системе координат. Кроме 
того, на первом этапе необходимо решить уравнение Пуассона для получения значений потенциала 
самосогласованного электрического поля. Для этого применяется метод, описанный в разд.~3. 

Вторая задача решается путем перехода к ко\-неч\-но-раз\-ност\-ной сис\-те\-ме. При этом частные 
производные $\partial^2\hat{g}_\alpha/\partial[\hat{v}_y]^2$ и $\partial\hat{h}_\alpha/\partial\hat{v}_y$ 
аппроксимируются со вторым порядком точности с использованием трехточечного шаблона, а 
производная $\partial s_\alpha/\partial\hat{t}$ аппроксимируется на двухточечном шаблоне с первым 
порядком точности~\cite{16-k}. К~полученной системе разностных уравнений предлагается применить 
один из классических методов решения систем линейных уравнений, например метод 
Гаусса~\cite{19-k}.
      
      Решением первой задачи является функция $w_\alpha(\hat{y}, \hat{v}_y, \hat{t}^n)$, 
$n\hm=0,\ldots ,N$, , которая дает начальное условие для второй задачи. Решая вторую задачу, находим 
функцию $s_\alpha(\hat{y},\hat{v}_y,\hat{t}^n)\hm=\hat{f}_\alpha(\hat{y},\hat{v}_y,\hat{t}^n)$, 
$n=1,\ldots ,N$, $\alpha=i,e$, которая определяет решение $\hat{f}_\alpha(\hat{y},\hat{v}_y,\hat{t}^n)$, 
$\alpha=i,e$, исходной системы~(\ref{e7-k}) для рассматриваемых моментов времени $n=1,\ldots ,N$.

Моменты функций распределения $\hat{f}_\alpha$, $\alpha=i,e$, находятся с помощью методов 
численного интегрирования, например метода трапеций~\cite{19-k}.

\section{Результаты численного моделирования}

Для двух описанных выше методов реализованы две отдельные программы в среде {Matlab~7.0}. 
Эти программы позволяют по заданным значениям концентраций и температур частиц $n_{i\infty}$, 
$n_{e\infty}$, $T_{i\infty}$ и~$T_{e\infty}$ в невозмущенной плазме, а также потенциала~$\varphi_p$, 
подаваемого на зонд, изучить эволюцию во времени плотностей тока частиц~$j_i$ и~$j_e$, концентраций 
частиц~$n_i$  и~$n_e$ в произвольной точке пространства в возмущенной зоне, а также динамику 
изменения напряженности~$E_y$ самосогласованного электрического поля во времени и пространстве.

С использованием разработанных программ проведены серии расчетных экспериментов, в которых 
значение концентраций варьировалось в пределах $n_{i\infty} \hm = n_{e\infty}\hm =10^{18}\div 
10^{22}$~м$^{-3}$. Значение температур было выбрано неизменным и равным $T_{i\infty}\hm = 
T_{e\infty}\hm=3000$~K, а значения потенциала, подаваемого на зонд, изменялись в пределах 
$\varphi_p\hm=0\div 2{,}6$~В.

На рис.~1  и~2 приведены графики изменения напряженности самосогласованного электрического
 поля (см.\ рис.~1) и плотности токов ионов (см.\linebreak\vspace*{-12pt}

\pagebreak

\end{multicols}

\begin{figure} %fig1
\vspace*{1pt}
\begin{center}
\mbox{%
\epsfxsize=162.594mm
\epsfbox{kud-1.eps}
}
\end{center}
\vspace*{-9pt}
\Caption{Динамика изменения плотности тока ионов во времени в фиксированной точке возмущенной 
зоны для значений потенциала: \textit{1}~--- $\varphi_p=-6$; 
\textit{2}~--- $\varphi_p=-16$; \textit{3}~--- $\varphi_p=- 30$ 
в случае применения методов Монте-Карло~(\textit{а}) 
и крупных частиц~(\textit{б})}
\end{figure}

\begin{figure} %fig2
\vspace*{1pt}
\begin{center}
\mbox{%
\epsfxsize=162.713mm
\epsfbox{kud-2.eps}
}
\end{center}
\vspace*{-9pt}
\Caption{Динамика изменения напряженности электрического поля во времени в фиксированной точке 
возмущенной зоны для значений потенциала: 
\textit{1}~--- $\varphi_p=-6$; \textit{2}~--- $\varphi_p=-16$; 
\textit{3}~--- $\varphi_p=-30$ в случае применения методов Монте-Карло~(\textit{а}) и
крупных частиц~(\textit{б})
}
\end{figure}

\begin{multicols}{2}

\noindent
 рис.~2) во времени в фиксированной точке пространства 
возмущенной зоны в случае применения обоих разработанных алгоритмов.


На основании полученных результатов можно отметить похожее поведение зависимостей 
напряженности электрического поля и плотности тока от времени в двух рассматриваемых случаях. 
Графики кривых сначала убывают, затем начинают возрастать, выходя в некоторый момент 
времени~$t^\prime$ (момент установления) на стационарные значения. 

Одинаковое поведение 
напряженности и плот\-ности тока можно объяснить из следующих соображений: плотность тока ионов в 
данной области пространства равна произведению концентрации ионов на их направленную скорость и 
на заряд иона. Скорость ионов, в свою очередь, зависит от заряда, массы и напряженности 
электрического поля. 
%\columnbreak

При внесении в плазму отрицательно заряженного зонда возникает электрическое поле, которое 
нарушает квазинейтральность плазмы. Для того чтобы компенсировать действие внешнего 
электрического поля, ионы устремляются к зонду, а электроны~--- от зонда. Это приводит к дисбалансу 
концентраций вблизи зонда и, как следствие, к увеличению разности потенциалов; график 
напряженности электрического поля убывает. Вскоре разделение зарядов компенсирует внешнее 
электрическое поле; график выходит на стационарное значение. 

Также можно отметить, что значения 
напряженности электрического поля и плотности тока частиц на зонд в момент установления для двух 
методов совпадают. 

Момент установления~$t^\prime$ зависит от при\-ме\-ня\-емо\-го метода решения. В~случае метода 
Мон\-те-Кар\-ло $t^\prime=3{,}5\div 4$~ед., а для метода крупных частиц совместно с методом 
расщепления $t^\prime\hm=5\div 5{,}5$~ед. Используя ко\-неч\-но-раз\-ност\-ный метод, можно 
получить динамику изменения функций распределения частиц~$f_\alpha$, $\alpha=i,e$, во времени и 
пространстве. Функции распределения позволяют наглядно представить влияние на картину 
распределения частиц вблизи зонда самой поверхности зонда и электрического поля.

\section{Заключение}
      
      В работе найдено решение задачи диагностики плоским зондом сильноионизованной плазмы с 
учетом столкновений заряженных частиц. Разработана математическая модель исследуемого явления, 
описываемая уравнениями Фок\-ке\-ра--План\-ка и Пуассона. Решение получено двумя методами:\linebreak 
статистическим и ко\-неч\-но-раз\-ност\-ным на основе\linebreak сформированных алгоритмов. Приведены 
резуль-\linebreak таты численного моделирования при различных\linebreak характерных параметрах задачи.
 Из  проведенных 
вычислительных экспериментов вытекает, что искомые величины: напряженность 
электрического поля, плотности токов частиц на зонд, концентрации частиц вблизи зонда~--- как по 
характеру зависимости, так и по числовым значениям совпадают. При применении метода 
      Мон\-те-Кар\-ло момент установления наступает быстрее по сравнению с конечно-разностным 
методом, однако конечно-разностный метод позволяет получить более наглядные результаты.

{\small\frenchspacing
{%\baselineskip=10.8pt
\addcontentsline{toc}{section}{Литература}
\begin{thebibliography}{99}

\bibitem{1-k}
\Au{Alexeff I., Anderson T.}
Experimental and theoretical results with plasma antenna~// IEEE Trans. Plasma Sci., 2006. Vol.~34. 
No.\,2. P.~166--172.

\bibitem{2-k}
\Au{Сысун В.\,И.}
Сильноионизованная низкотемпературная плазма в приборах электронной техники: Методы 
исследования, свойства, применение. Дисс. \ldots д-ра физ.-мат. наук в форме науч. докл.: 
01.04.08.~--- Пет\-ро\-за\-водск, 1996.

\bibitem{3-k}
\Au{Тухас В.\,А.}
Методология создания средств измерений и испытаний на устойчивость к кондуктивным помехам~// 
Мат-лы VI Междунар. симп. по электромагнитной совместимости и 
электромагнитной экологии.~--- СПб., 2005. С.~231--234.

\bibitem{4-k}
\Au{Гудзенко Л.\,И., Яковленко С.\,И.}
Плазменные лазеры.~--- М.: Атомиздат, 1978.  256~с.

\bibitem{5-k}
\Au{Звелто О.}
Принципы лазеров.~--- М.: Мир, 1990.  560~с.

\bibitem{6-k}
\Au{Сысун В.\,И., Хромой Ю.\,Д.}
Расширение канала мощного импульсного разряда в парах ртути~// Электронная техника, 1974. 
Сер.~4. Вып.~10. С.~80--85. 

\bibitem{7-k}
\Au{Винклер Дж.\,Р.}
Искусственные пучки частиц в космической плазме.~--- М.: Мир, 1985.  451~с.

\bibitem{8-k}
\Au{Bernstein I.\,B., Rabinowitz I.\,N.}
Theory of electrostatic probes in low-density plasma~// Phys. Fluids, 1959. Vol.~2. No.\,2. P.~112--121. 

\bibitem{9-k}
\Au{Альперт Я.\,Л., Гуревич А.\,В., Питаевский~Л.\,П.}
Искусственные спутники в разреженной плазме.~--- М.: Наука, 1964.  282~с.

\bibitem{10-k}
\Au{Чан П., Тэлбот Л., Турян~К.}
Электрические зонды в неподвижной и движущейся плазме.~--- М.: Мир, 1978.  202~с.

\bibitem{11-k}
\Au{Алексеев Б.\,В., Котельников В.\,А.}
Зондовый метод диагностики плазмы.~--- М.: Энергоатомиздат, 1989.  240~с.

\bibitem{12-k}
\Au{Пантелеев А.\,В., Кудрявцева И.\,А.}
Формирование математической модели двухкомпонентной плазмы с учетом столкновений 
заряженных частиц в случае плоского зонда~// Теоретические вопросы вычислительной техники и 
программного обеспечения: Межвузовский сб. научн. тр.~--- М.: МИРЭА, 2006. С.~11--21.

\bibitem{13-k}
\Au{Олдер Б.}
Вычислительные методы в физике плазмы.~--- М.: Мир, 1974.  111~с.

\bibitem{14-k}
\Au{Montgomery D.\,C., Tidman D.\,A.}
Plasma kinetic theory.~--- New York, 1964. 

\bibitem{15-k}
\Au{Кудрявцева И.\,А., Пантелеев А.\,В.}
Применение метода Мон\-те-Кар\-ло для анализа поведения двухкомпонентной плазмы с учетом 
столкновений между заряженными частицами~// Теоретические вопросы\linebreak
вычислительной техники и 
программного обеспечения: Межвузовский сб. научн. тр.~--- М.: МИРЭА, 2008. С.~122--128. 

\bibitem{16-k}
\Au{Семенов В.\,В., Пантелеев А.\,В., Руденко~Е.\,А., Бор\-та\-ков\-ский~А.\,С.}
Методы описания, анализа и синтеза нелинейных систем управления.~--- М.: МАИ, 1993.  312~с.

\bibitem{17-k}
\Au{Киреев В.\,И., Пантелеев А.\,В.}
Численные методы в примерах и задачах.~--- М.: Высшая школа, 2006.  480~с.

\bibitem{18-k}
\Au{Белоцерковский О.\,М., Давыдов~Ю.\,М.}
Метод крупных частиц в газовой динамике. Вычислительный эксперимент.~--- М.: Наука, 
Физматгиз, 1982.

\label{end\stat}

\bibitem{19-k}
\Au{Вержбицкий В.\,М.}
Основы численных методов.~--- М.: Высшая школа, 2002.  840~с.
 \end{thebibliography}
}
}


\end{multicols}         %9
\newcommand{\Tsf}{^{\mathsf T}}
\newcommand{\rank}{\mathrm{rank}\,}

\def\stat{logachev}

\def\tit{ПОЛИНОМИАЛЬНЫЕ АЛГОРИТМЫ ВЫЧИСЛЕНИЯ ЛОКАЛЬНЫХ АФФИННОСТЕЙ КВАДРАТИЧНЫХ 
БУЛЕВЫХ ФУНКЦИЙ$^*$}

\def\titkol{Полиномиальные алгоритмы вычисления локальных аффинностей квадратичных 
булевых функций}

\def\aut{О.\,А.~Логачев$^1$, А.\,А.~Сукаев$^2$, С.\,Н.~Федоров$^3$}

\def\autkol{О.\,А.~Логачев, А.\,А.~Сукаев, С.\,Н.~Федоров}

\titel{\tit}{\aut}{\autkol}{\titkol}

\index{Логачев О.\,А.}
\index{Сукаев А.\,А.}
\index{Федоров С.\,Н.}
\index{Logachev O.\,A.}
\index{Sukayev A.\,A.}
\index{Fedorov S.\,N.}


{\renewcommand{\thefootnote}{\fnsymbol{footnote}} \footnotetext[1]
{Работа выполнена при частичной поддержке РФФИ (проект 18-29-03124~мк).}}


\renewcommand{\thefootnote}{\arabic{footnote}}
\footnotetext[1]{Московский государственный университет им.\
М.\,В.~Ломоносова; Институт проб\-лем информатики Федерального исследовательского
центра <<Информатика и~управ\-ле\-ние>> Российской академии наук, \mbox{logol@iisi.msu.ru}}
\footnotetext[2]{Московский государственный университет им.\ 
М.\,В.~Ломоносова, \mbox{asukaev@gmail.com}}
\footnotetext[3]{Московский государственный университет им.\
М.\,В.~Ломоносова, \mbox{s.n.feodorov@yandex.ru}}

\vspace*{-12pt}

 
\Abst{Аффинная нормальная форма позволяет рассматривать произвольную булеву функцию на 
определенных плоскостях (так называемых локальных аффинностях) как аффинную. Данное 
пред\-став\-ле\-ние~--- по сути, аффинная аппроксимация~--- булевых функций может 
помочь в~решении систем нелинейных уравнений над полем из двух элементов. Задача 
решения таких систем (специального вида), среди прочего, используется в~ряде 
методов синтеза и~анализа средств обеспечения информационной безопасности.
В~статье описывается способ нахождения локальных аффинностей для квадратичных 
булевых функций, основанный на теореме Диксона. Тем самым решается задача 
построения аффинных нормальных форм для таких функций. Кроме того, обсуждаются 
вопросы эффективности подобных алгоритмов.
Основная цель данной статьи~--- подготовить базу для готовящейся к~публикации 
работы, предлагающей метод решения систем квадратичных булевых уравнений 
с~помощью <<аппроксимирования>> соответствующих функций их аффинными нормальными 
формами.}


\KW{булева функция; система квадратичных булевых уравнений; 
разбиение векторного пространства; плоскость; локальная аффинность; теорема 
Диксона; аффинная нормальная форма; алгебраический криптоанализ}

\DOI{10.14357/19922264190110}
  
\vspace*{-1pt}


\vskip 10pt plus 9pt minus 6pt

\thispagestyle{headings}

\begin{multicols}{2}

\label{st\stat}


\section{Введение}

\vspace*{-2pt}

Центральная идея алгебраического криптоанализа состоит в~том, чтобы описать 
используемые в~анализируемой криптосхеме преобразования сис\-те\-мой алгебраических 
уравнений (с~некоторой сек\-рет\-ной информацией в~качестве неизвестных) над\linebreak 
конечным полем и~затем решить эту систему.
В~данной статье рассматриваются только булевы системы уравнений, хотя часть 
пред\-став\-лен\-ных здесь результа\-тов может иметь место и~для сис\-тем ал\-геб\-ра\-и\-че\-ских 
уравнений над произвольными конечными полями.

Из теории сложности вычислений известно, что вычислительная задача определения 
совместности систем нелинейных булевых уравнений является NP-пол\-ной~\cite{GJ1982, GT2017}, 
а~вычислительная задача решения систем нелинейных булевых 
уравнений является NP-труд\-ной~\cite{GJ1982,GT2017}.
Однако в~специальных случаях эти задачи могут решаться эффективно (см., 
например,~\cite{GT2017,Smi2000}).

Кроме того, существуют полиномиальные алгоритмы построения по произвольной 
системе уравнений системы с~фиксированной алгебраической 
степенью~\cite[\S\;11.4.2]{Bard2009}, что позволяет, в~частности, ограничиться 
рассмотрением только квадратичных систем уравнений.

Можно выделить несколько основных классов методов, используемых в~криптоанализе 
для решения (или оценки трудоемкости решения) систем полиномиальных булевых 
уравнений:
использование базисов Грёбнера~\cite[section~12.2]{Bard2009}, применение 
программных систем поиска выполняющего набора булевой формулы 
(SAT-solvers)~\cite{BCJ2007}, вероятностные и~тео\-ре\-ти\-ко-ко\-до\-вые 
методы~\cite{LSSYa2015}, а~также  методы линеаризации~\cite[section~12.3]{Bard2009}.
Основная идея методов линеаризации состоит в~применении <<линейных>> методов 
к~нелинейным системам, т.\,е.\ в~построении сис\-тем линейных уравнений, решение 
которых дает возможность найти решение исходной нелинейной системы.

Важным параметром метода линеаризации служит число переменных в~синтезируемых 
линейных системах уравнений. Как правило, речь идет об увеличении (не~более чем 
полиномиальном) количества переменных.
Метод, основанный на рассмотренных в~данной работе идеях, по своей сути, 
осуществляет линеаризацию, но при этом он остав\-ля\-ет число переменных неизменным.

Этот метод решения квадратичных систем булевых уравнений использует локальные 
аффинности уравнений системы и~состоит из двух этапов.
Первый этап (предварительный) содержательно представляет собой описание семейств 
локальных аффинностей уравнений.
Второй этап метода заклю\-ча\-ет\-ся собственно в~решении исходной сис\-те\-мы посредством 
анализа сис\-тем линейных уравнений, полученных с~помощью этих локальных 
аффинностей.

Настоящая работа (в~силу ограниченности\linebreak объема публикации) посвящена 
исследованию первого этапа и,~в~частности, вопросам его эффективности. 
Результаты исследований с~оценкой эф\-фек\-тив\-ности и~описанием параметров второго
\mbox{этапа} предлагаемого метода предполагается опуб\-ли\-ко\-вать в~одном из сле\-ду\-ющих 
выпусков журнала.

\vspace*{-4pt}

\section{Необходимые понятия и~обозначения}

В данной работе булев куб $\{0,1\}^n$ отождествляется с~$n$-мерным векторным 
пространством~$V_n$ над полем из двух элементов~$\mathbb{F}_2$.
Векторы из $V_n$ будет удобнее записывать \textit{строками} длины~$n$. Значок~$\Tsf$ 
используется для операции транспонирования матриц.
Всюду далее~$x$ обозначает вектор $(x_1,x_2,\ldots,x_n)$.

Знак $\oplus$ будет использоваться для записи суммы по модулю~$2$ булевых 
переменных и~операций сложения в~$\mathbb{F}_2$ и~покомпонентного сложения 
в~$V_n$.

Множество всех невырожденных аффинных преобразований (отображений в~себя) 
пространства~$V_n$ обозначается через $\mathrm{GA}(V_n)$. В~матричном 
представлении действие элемента~$\alpha\in\mathrm{GA}(V_n)$ на векторах 
пространства имеет вид $\alpha(x)\hm=xA\oplus b$, где $x$~пробегает~$V_n$; $A$~--- 
невырожденная $(n\times n)$-мат\-ри\-ца над~$\mathbb{F}_2$; $b\hm\in V_n$.

Множество всех булевых функций от $n$~переменных обозначим через
$$
\mathcal{F}_n=\{f\colon V_n\to \mathbb{F}_2\}\,.
$$
Как известно, произвольную булеву функцию~$f$ от переменных $x_1,\ldots,x_n$ 
можно представить (единственным образом) в~виде полинома Жегалкина:
$$
f(x)=\bigoplus_{\varepsilon\in\{0,1\}^n} a_{\varepsilon}x^{\varepsilon}\,,
$$
где %\label{Zhegalkin}
$\varepsilon\hm=(\varepsilon_1,\ldots,\varepsilon_n)$,
$a_{\varepsilon}\hm\in\mathbb{F}_2$ и~$x^{\varepsilon}\hm=x_1^{\varepsilon_1}\cdots x_n^{\varepsilon_n}$ (считаем, 
$x_i^0\hm=1$, $x_i^1\hm=x_i$).
Далее под булевой функцией будет, как правило, подразумеваться ее запись в~виде 
полинома.

Если $\varphi$~--- некоторое преобразование пространства~$V_n$, то его действие на 
функцию~$f\hm\in\mathcal{F}_n$ будем определять и~обозначать так: 
$f^{\varphi}(x)\hm=f(\varphi(x))$.
В~частности, в~случае аффинных преобразований пространства будет рассматриваться 
множество $\mathrm{Orb}_f(\mathrm{GA}(V_n))\hm=\{f^{\varphi}\mid 
\varphi\hm\in\mathrm{GA}(V_n)\}$~--- орбита функции~$f$ относительно действия 
группы~$\mathrm{GA}(V_n)$.
Имея в~виду, что произведение~$\alpha_1\alpha_2$ элементов из $\mathrm{GA}(V_n)$ 
есть композиция $\alpha_1\circ\alpha_2(x)\hm=\alpha_1(\alpha_2(x))$, заметим, что 
действие~$\alpha_1\alpha_2$ на произвольную функцию $f\hm\in\mathcal{F}_n$ 
корректно определять следующим образом:
$$
f^{\alpha_1\alpha_2}(x)=\left(f^{\alpha_1}\right)^{\alpha_2}(x)=f^{\alpha_1}
\left(\alpha_2(x)\right)
=f\left(\alpha_1\alpha_2(x)\right),
$$
поскольку~$\alpha_i$ действуют на булеву функцию преобразованием \textit{ее 
аргумента}.

%Когда мы делаем невырожденную аффинную замену переменных $x'=\alpha(x)=xA\oplus 
%b$, функция~$f(x)$, при подставлении в~нее выражений старых переменных через 
%новые, преобразуется к~виду $f^{\alpha^{-1}}(x')$.


\textit{Алгебраической степенью} булевой функции~$f$ от $n$~переменных называют 
величину

\noindent
$$
\deg f = \max\left\{\sum\limits_{i=1}^n \varepsilon_i\mid a_{\varepsilon}=1\right\}
$$
(суммирование~--- в~$\mathbb{Z}$), т.\,е.\ максимальное число различных 
переменных в~мономах данного представления.

В множестве~$\mathcal{F}_n$ всех булевых функций от~$n$~переменных выделим 
подмножество

\noindent
$$
\mathcal{A}_n=\left\{f\in\mathcal{F}_n\mid \deg f\leqslant 1\right\}.
$$
Составляющие это подмножество функции называются линейными (в~математической 
логике и~кибернетике) или аф\-фин\-но-ли\-ней\-ны\-ми (в~ал\-геб\-ре), однако по сложившейся 
в~криптологии традиции в~данной работе они называются \textit{аффинными}, т.\,е.\ 
понимаются как частный случай аффинного \textit{отоб\-ра\-же\-ния} $n$-мер\-но\-го 
пространства в~одномерное.


Булеву функцию~$f$ c $\deg f\hm\leqslant 2$ будем называть 
\textit{квадратичной}\footnote{В~алгебре такие функции называют 
аффинно-квад\-ра\-тич\-ны\-ми. Квадратичными при этом называют функции, представляемые 
\textit{однородными} полиномами второй степени.}.
По определению квадратичная функция~$f\hm\in\mathcal{F}_n$ (ее полином Жегалкина) 
имеет вид:

\noindent
$$
f(x)= \bigoplus_{1\leqslant i<j\leqslant n} q_{ij}x_ix_j\oplus
\bigoplus_{1\leqslant k\leqslant n} 
l_kx_k \oplus c\,,
$$
где $q_{ij},l_k,c\in\mathbb{F}_2$.

В настоящей работе рассматриваются системы уравнений

\noindent
\begin{equation}
\left.
\begin{array}{c}
        f_1(x_1,\ldots,x_n)=0\,;\\
        f_2(x_1,\ldots,x_n)=0\,;\\
        \vdots\\
        f_m(x_1,\ldots,x_n)=0\\
    \end{array}
    \right\}
    \label{system}
\end{equation}
с квадратичными булевыми функциями~$f_i$, $1\hm\leqslant i\hm\leqslant m$, и~$m\hm>n$.
%Мы предполагаем, что все рассматриваемые нами системы квадратичных уравнений 
%имеют единственное решение.

\pagebreak

В матричном виде квадратичная функция записывается следующим образом:
$$
f(x)=x Q_f x\Tsf\oplus l_f x\Tsf \oplus c\,,
$$
где $Q_f$~--- верхнетреугольная $(n\times n)$-мат\-ри\-ца с~нулевой главной 
диагональю; $l_f\in\mathbb{F}_2^n$; $c\in\mathbb{F}_2$.
Рас\-смат\-ри\-ва\-ют также симметричную матрицу
$$
\tilde{Q}_f=Q_f\oplus Q_f\Tsf\,.
$$
Она определяет билинейную форму
$$
q_f(u,v)=u\tilde{Q}_f v\Tsf=f(u\oplus v)\oplus f(u) \oplus f(v)\oplus c\,,
$$
называемую \textit{ассоциированной с~квадратичной функцией~$f$}.

Булева билинейная форма $q(u,v)$, $u,v\hm\in V_n$, удовле\-тво\-ря\-ющая условиям
$$
q(u,u)=0\,;\qquad q(u,v)=q(v,u)\,,
$$
называется \textit{симплектической}.
Такие билинейные формы находятся во взаимно однозначном соответствии с~булевыми 
симметричными матрицами с~нулевой главной диагональю, называемыми 
\textit{симплектическими матрицами}.

Таким образом, для произвольной квадратичной булевой функции~$f$ 
матрица~$\tilde{Q}_f$~--- сим\-плектическая. Также очевидно, что билинейная\linebreak 
форма~$q_f(u,v)$, ассоциированная с~$f$, является симплектической.

\smallskip

\noindent
\textbf{Предложение~1}\
[6, лемма~3.3.1; 7, \S\;15.2, лемма~3].
\textit{Ранг симплектической матрицы четен.}


\smallskip

\textit{Плоскость}~$\pi$ в~$V_n$~--- это множество вида $v\hm+L$, где~$v$ и~$L$~--- 
соответственно вектор и~подпространство пространства~$V_n$. Другими словами, 
плоскость~--- аффинное подпространство в~$V_n$.
\textit{Размерность плоскости} совпадает с~размерностью соответствующего 
подпространства: $\mathrm{dim}\,\pi=\mathrm{dim}\,L$.
Как известно, любая плоскость является решением некоторой системы линейных 
уравнений, и~на\-обо\-рот: решение произвольной системы линейных уравнений~--- 
плоскость в~соответствующем пространстве.
%То есть, плоскость является линейным многообразием.

Сужение функции $f\hm\in\mathcal{F}_n$ на плоскость~$\pi$ будем обозначать 
через~$f|_{\pi}$. Таким образом, $f|_{\pi}\colon \pi\hm\to\mathbb{F}_2$ 
и~$f|_{\pi}(u)\hm=f(u)$ для всех $u\hm\in\pi$.



\section{Локальная аффинность и~аффинная нормальная форма~булевой~функции}

В этом разделе вводятся понятия, связанные с~представлением произвольной булевой 
функции совокупностью аффинных функций, заданных для определенных плоскостей 
в~векторном пространстве. Более общее изложение этой теории можно найти 
в~работе~\cite{LYaD2007}.

\textit{Локальной аффинностью} функции~$f\hm\in\mathcal{F}_n$ будем называть такую 
плоскость~$\pi$, что $f|_{\pi}$ можно продолжить до аффинной функции, т.\,е.\ 
существует $l\hm\in\mathcal{A}_n$ со свойством $f|_{\pi}\hm=l|_{\pi}$.
Очевидно, для любой булевой функции существует разбиение пространства~$V_n$ на 
ее локальные аффинности.

Возьмем произвольное разбиение $\Pi\hm=\{\pi_1,\ldots,\pi_{\lambda}\}$ 
пространства~$V_n$ на плоскости, являющиеся локальными аффинностями булевой 
функции~$f$ от $n$~переменных.
Будем называть \textit{аффинной нормальной формой} функции~$f$ выражение вида
\begin{equation}
\label{AffNF}
f(x)=\bigoplus_{j=1}^{\lambda}\chi_{\pi_j}(x) l_j(x)\,,
\end{equation}
где для каждого~$j$, $1\hm\leqslant j\hm\leqslant\lambda$, функция~$l_j$ аффинна 
и~$f|_{\pi_j}(x)\hm=l_j|_{\pi_j}(x)$, а~$\chi_{\pi_j}$~--- характеристическая 
функция (индикатор) множества~$\pi_j$.
Функции~$l_j$ из этого выражения для краткости назовем\linebreak
 \textit{аффинными 
аппроксимациями} функции~$f$.
\textit{Длиной аффинной нормальной формы} называется число плоскостей 
в~разбиении~$\Pi$, далее она будет обозначаться через~$\lambda(\Pi)$.

\smallskip

\noindent
\textbf{Замечание~1.}
  Характеристические функции плоскостей известны также под именем 
<<мультиаффинных функций>> \cite{GT2017}, играющих важную роль при описании 
классов эффективно решаемых систем булевых уравнений.


\smallskip

Характеристическая функция плоскости в~пространстве~$V_n$ имеет вполне 
определенный вид. Любая плоскость~$\pi$, как уже отмечалось, может быть задана 
как множество решений системы $d$~линейных уравнений (для некоторого~$d$):
\begin{equation}
\left.
\begin{array}{c}
        h_1(x_1,\ldots,x_n)=0\,;\\
        h_2(x_1,\ldots,x_n)=0\,;\\
        \vdots\\
        h_d(x_1,\ldots,x_n)=0\,,\\
    \end{array}
    \right\}
    \label{chi-system}
\end{equation}
где все $h_i(x)\hm\in\mathcal{A}_n$. Поскольку вектор~$x$ принадлежит 
плоскости~$\pi$ тогда и~только тогда, когда все~$h_i$, $1\hm\leqslant i\hm\leqslant d$, 
обращаются в~нуль на нем, характеристическая функция~$\pi$ выражается следующим 
образом:
$$
\chi_{\pi}(x)=\prod\limits_{i=1}^d (h_i(x)\oplus 1)\,.
$$
Если система линейных уравнений задана в~мат\-рич\-ной форме: $xH\oplus 
(b_1,\ldots,b_d)\hm=0$, $b_i\hm=h_i(0)$, то выражение будет иметь вид:

\noindent
$$
\chi_{\pi}(x)=\prod\limits_{i=1}^d (xH_i\oplus b_i\oplus 1)\,,
$$
где $H_i$~--- столбцы матрицы~$H$.

Как видно из определения, аффинная нормальная форма представляет собой 
в~некотором смыс\-ле ку\-соч\-но-аф\-фин\-ную аппроксимацию булевой функции. На каждой 
локальной аф\-фин\-ности~$\pi_j$ из разбиения $\Pi$ все, кроме одного, слагаемые 
в~выражении~\eqref{AffNF} обращаются в~нуль, и~функция принимает вид 
$f(x)\hm=\chi_{\pi_j}(x)l_j(x)\hm=l_j(x)$ для всех $x\hm\in\pi_j$.

Возможность заменить на плоскости~$\pi_j$ квадратичное уравнение $f(x)\hm=0$ 
линейным уравнением $l_j(x)\hm=0$ вместе с~дописанной к~нему системой~\eqref{chi-system} 
будет использоваться при решении систем полиномиальных уравнений 
в~следующей статье.
Как сказано в~замечании~1, функции~$\chi_{\pi}(x)$, а~также 
и~слагаемые в~аффинной нормальной форме~\eqref{AffNF} являются мультиаффинными 
функциями. Тео\-ре\-ти\-ко-слож\-ност\-ные вопросы, связанные, в~частности, с~решением 
систем мультиаффинных уравнений, а~также оценки числа таких функций 
рассматриваются в~работе~\cite{Gor1995}.

При аффинном преобразовании пространства аффинные нормальные формы сохраняются 
в~том смысле, что выражение, полученное после применения преобразования к~этой 
форме, тоже будет аффинной нормальной формой для некоторой функции.

\smallskip

\noindent
\textbf{Предложение~2.}
\textit{Пусть $\varphi\in\mathrm{GA}(V_n)$ и~$f(x)\hm=
\bigoplus_{j=1}^{\lambda(\Pi)}\chi_{\pi_j}(x) l_j(x)$~--- некоторая 
аффинная нормальная форма функции~$f$.
  Тогда} 
$$
f^{\varphi}(x)=f(\varphi(x))=\bigoplus_{j=1}^{\lambda(\Pi)}\chi_{\pi_j}(\varphi(x)) 
l_j(\varphi(x))$$ 
\textit{есть аффинная нормальная форма функции~$f^{\varphi}$}.


\smallskip

\noindent
Д\,о\,к\,а\,з\,а\,т\,е\,л\,ь\,с\,т\,в\,о\,.\ \ 
Множество $\Pi'\hm=\{\pi'_j\hm=\varphi^{-1}(\pi_j) \mid \pi_j\in\Pi\}$ является 
разбиением пространства~$V_n$ на $\lambda(\Pi)$ плоскостей, поскольку~$\varphi$~--- 
не\-вы\-рож\-ден\-ное аффинное преобразование.
Заметим,\linebreak
 что $\varphi(x)\in\pi_j$ тогда и~только тогда, когда $x\hm\in\varphi^{-1}
(\pi_j)\hm=\pi'_j$.
Поэтому $\chi^{\varphi}_{\pi_j}$~--- характеристическая функция 
плоскости~$\pi'_j$.
Выражение для~$f^{\varphi}$ в~новых обозначениях выглядит следующим образом:
$$
f^{\varphi}(x)=\bigoplus_{j=1}^{\lambda(\Pi')}\chi_{\pi'_j}(x) l_j^{\varphi}(x)\,.
$$
Так как, очевидно, функции $l_j^{\varphi}(x)\hm=l_j(\varphi(x))$ аффинны, полученное 
выражение представляет собой аффинную нормальную форму.


\section{Теорема Диксона и~приведение квадратичных функций к~каноническому 
виду}\label{Dickson}

Благодаря теореме Диксона можно для любой квадратичной булевой функции~$f$ найти 
ее каноническое представление, в~котором она выглядит наиболее просто. Как будет 
видно ниже, это представление~--- элемент из орбиты данной функции 
$\mathrm{Orb}_f(\mathrm{GA}(V_n))$.
Канонический вид квадратичной функции, в~свою очередь, подсказывает прос\-той 
способ построения ее аффинной нормальной \mbox{формы.}

\smallskip

\noindent
\textbf{Теорема~1}\ [10, \S\;199].
\textit{Для любой квад\-ра\-тич\-ной функции~$f\hm\in\mathcal{F}_n$ с~ненулевой 
матрицей~$\tilde{Q}_f$ существует аффинное 
преобразование~$\alpha\hm\in\mathrm{GA}(V_n)$, которое приводит~$f$ к~одному из  
канонических представлений}:
$$f^{\alpha}(x)=x_1x_2\oplus x_3x_4\oplus\cdots\oplus x_{2r-1}x_{2r}\oplus c
$$
\textit{или}
$$f^{\alpha}(x)=x_1x_2\oplus x_3x_4\oplus\cdots\oplus x_{2r-1}x_{2r}\oplus 
x_{2r+1}\,,
$$
где $2r=\rank \tilde{Q}_f$ и~$c\hm\in\mathbb{F}_2$.

\smallskip

Доказательство этого утверждения помимо авторского варианта можно найти также 
в~[6, \S\;3.3; 7, \S\;15.2].

На практике приведение полинома Жегалкина квадратичной булевой функции 
к~каноническому виду можно осуществить следующим способом.

Предположим, не ограничивая общности, что в~полиноме Жегалкина функции~$f$ 
присутствует моном~$x_1x_2$ (иначе с~помощью аффинного преобразования координат 
<<перенумеруем>> переменные).
Представим функцию в~виде:
\begin{multline*}
f(x)=x_1x_2\oplus x_1l_1(x_3,\ldots,x_n) \oplus x_2l_2(x_3,\ldots,x_n) \oplus{}\\
{}\oplus 
q_1(x_3,\ldots,x_n)\,,
\end{multline*}
где $l_1,l_2\in\mathcal{A}_{n-2}$, а~$q_1$~--- некоторая квадратичная функция.
Возьмем отображение~$\varphi_2$ пространства~$V_n$, задаваемое равенством:
\begin{multline*}
\varphi_2(x)=\left(x_1\oplus l_2(x_3,\ldots,x_n),\ x_2\oplus {}\right.\\
\left.{}\oplus
l_1(x_3,\ldots,x_n),\ x_3,\ldots,\ x_n\right)\,,
\end{multline*}
и рассмотрим следующую функцию:
$$
f^{(2)}(x)=x_1x_2\oplus q_2(x_3,\ldots,x_n)\,,
$$
где $q_2=q_1\oplus l_1l_2$~--- квадратичная функция.
Заметим, что $(f^{(2)})^{\varphi_2}\hm=f$.

Затем аналогично предыдущему выделим первые две переменные в~функции 
$q_2(x_3,\ldots,x_n)$.
Здесь берется отображение
\begin{multline*}
\varphi_4(x)=\bigl(x_1,\ x_2,\  x_3\oplus l_4(x_5,\ldots,x_n),\\
x_4\oplus 
l_3(x_5,\ldots,x_n),\ x_5,\ldots,\ x_n\bigr)
\end{multline*}
и функция
$$
f^{(4)}(x)=x_1x_2\oplus  x_3x_4\oplus q_4(x_5,\ldots,x_n)\,,
$$
так что $(f^{(4)})^{\varphi_4}=f^{(2)}$.

Проделываем это до тех пор, пока на некотором шаге не получим аффинную функцию
$$
q_{2r}(x_{2r+1},\ldots,x_n)=\bigoplus_{i=2r+1}^{n}b_ix_i\oplus c
$$
для некоторых $b_i,c\hm\in\mathbb{F}_2$.
Если $b_i\hm=0$ для всех~$i$, $2r\hm+1\hm\leqslant i\hm\leqslant n$, 
то искомый канонический вид 
найден: это функция~$f^{(2r)}$.
Иначе считаем, без ограничения общности, что $b_{2r+1}\hm=1$ и~полагаем
\begin{multline*}
\varphi_{2r+1}(x)=\left(x_1,\ldots,\ x_{2r},\ 
q_{2r}\left(x_{2r+1},\ldots,x_n\right),\right.\\ 
\left.x_{2r+2},\ldots,\ x_n\right)\,.
\end{multline*}
Тогда канонический вид для~$f$~--- это функция
\begin{multline*}
g(x)={}\\
{}=f^{(2r+1)}(x)=x_1x_2\oplus  x_3x_4\oplus \cdots \oplus x_{2r-1}x_{2r} 
\oplus x_{2r+1},
\end{multline*}
причем если положить $\varphi\hm=\varphi_{2r+1}\varphi_{2r}
\varphi_{2r-2}\cdots\varphi_2$, то
$$
g^{\varphi}=\left(\cdots(g^{\varphi_{2r+1}})^{\varphi_{2r}}\cdots\right)^{\varphi_2}=f.$$

%$g^{\varphi}(x)=g\bigl(\varphi_{2r+1}(\ldots\varphi_2(x)\ldots)\bigr)$.
Преобразование~$\varphi$, очевидно, аффинно, невырожденно и~имеет вид:
$$    \hspace*{-33mm}\varphi(x) ={}\hspace*{33mm}
$$
\begin{equation*}
      \begin{split}
    {}=
    x &
    {
      \begin{pmatrix}
\makebox[1.5em]{$1$} &\rule{1.5em}{0pt} & \rule{1.5em}{0pt}   & 
\rule{1.5em}{0pt}   & \rule{1.5em}{0pt}   &  & \rule{1.5em}{0pt}   & 
\rule{1.5em}{0pt}   & \rule{1.5em}{0pt}   &  \rule{1.5em}{0pt}   \\
        0 & 1 &   &   &   &   &   &   &   &   \\
        * & * &\smash[t]{\ddots}&&   &   &   &   & 
\smash[t]{\mbox{\Huge{$0$}}}  &   \\
        * & * & \smash[t]{\ddots}  & 1 &   &   &   &   &   &      \\
        * & * &\smash[t]{\ddots}   & 0 & 1 &   &   &   &   &      \\
        * & * &\smash[t]{\ddots}   & * & * & 1 &   &   &   &   \\
        * & * &\smash[t]{\ddots}   & * & * & \makebox[1.5em]{$b_{2r+2}$}  & 1 &   &  &     \\
        * & * & \smash[t]{\ddots}  & * & * & \makebox[1.5em]{$b_{2r+3}$}  & 0 
&\smash[t]{\ddots}&& \\
        \vdots  &\vdots   & \smash[t]{\ddots}  &\vdots   &\vdots   & \vdots  & \vdots  &\ddots& 
1 &     \\
        * & * &\cdots   & * & * &b_n& 0 &\cdots & 0 & 1\\
      \end{pmatrix}} \oplus \\
   \oplus &
      \;\begin{pmatrix}
      \makebox[1.5em]{$*$}&\makebox[1.5em]{$*$}&\makebox[1.5em]{$\cdots$}&\makebox[1.5
em]{$*$}&\makebox[1.5em]{$*$}&\makebox[1.5em]{$c$}&\makebox[1.5em]{$0$}&\makebox
[1.5em]{$\cdots$}&\makebox[1.5em]{$0$}&\makebox[1.5em]{$0$} \\
      \end{pmatrix}
  \end{split}
\end{equation*}
(здесь знак~$*$ заменяет собой один из элементов~$\mathbb{F}_2$, каждый раз 
свой). Соответственно, преобразование~$\alpha$ из формулировки теоремы Диксона 
является обратным к~$\varphi$.

Как будет показано в~разд.~\ref{canonic-to-ANF}, представление функций~$f_i$ 
в~таком виде, т.\,е.\ нахождение подходящих представителей орбиты 
$\mathrm{Orb}_{f_i}(\mathrm{GA}(V_n))$, позволяет легко выписать аффинные 
нормальные формы для~$f_i$.

\section{Построение аффинной нормальной формы для~квадратичной 
функции}\label{canonic-to-ANF}

Обозначим через $\varphi_i$, $1\hm\leqslant i\hm\leqslant m$, невырожденные аффинные преобразования 
пространства~$V_n$, с~помощью которых функции~$f_i$ приводятся к~каноническому 
виду~$g_i$:
$$
g_i(x)=f_i^{\varphi_i}(x)=x_1x_2\oplus\cdots\oplus x_{2r_i-1}x_{2r_i}\oplus 
b_ix_{2r_i+1}\oplus c_i\,,
$$
где $2r_i=\rank \tilde{Q}_{f_i}$, а $b_i, c_i\hm\in \mathbb{F}_2$.

Очевидно, что если среди первых~$2r_i$ переменных взять все переменные с~четными 
индексами или все с~нечетными и~зафиксировать их значения, то получится 
плоскость, являющаяся локальной аффинностью функции~$g_i$.
Рассмотрим, например, $2^{r_i}$ плоскостей, заданных уравнениями:
$$    \begin{array}{l@{\,}c@{\ }l}
        x_1&=&\delta_1;\\
        x_3&=&\delta_2;\\
        \vdots\\
        x_{2r_i-1}&=&\delta_{r_i},\\
    \end{array}
$$
где $\delta_j\in\mathbb{F}_2$, $1\hm\leqslant j\hm\leqslant r_i$.
Каждую из этих плоскостей обозначим через~$\pi'_{i,\delta}$ со сложным индексом 
$\delta\hm=(\delta_1,\dots,\delta_{r_i})\in\mathbb{F}_2^{r_i}$.
Размерность~$\pi'_{i,\delta}$ равна $n\hm-r_i$, а мощность, соответственно, 
$2^{n-r_i}$.
Нетрудно видеть, что $\Pi'_i\hm=\{\pi'_{i,\delta}\}_{\delta\in\mathbb{F}_2^{r_i}}$ 
является разбиением пространства~$V_n$.

Обозначаемое ниже через~$l'_{i,\delta}$ сужение функции~$g_i$ на каждую из 
плоскостей разбиения~--- аффинно:
\begin{multline*}
l'_{i,\delta}(x)={}\\
{}=g_i|_{\pi'_{i,\delta}}(x)=\delta_1x_2\oplus\delta_2x_4\cdots\oplus\delta_{r_i}x_{2r_i}\oplus b_ix_{2r_i+1}\oplus c_i,
\hspace*{-0.80452pt}
\end{multline*}
а характеристическая функция соответствующей плоскости имеет вид:
$$
\chi_{\pi'_{i,\delta}}(x)=\prod\limits_{k=1}^{r_i}\left(x_{2k-1}\oplus\delta_k\oplus1\right)\,.
$$

С помощью аффинной нормальной формы
$$
g_i(x)=\bigoplus_{\delta\in\mathbb{F}_2^{r_i}} 
\chi_{\pi'_{i,\delta}}(x)l'_{i,\delta}(x)
$$
для канонического представления функции~$f_i$ можно аффинным преобразованием, 
обратным к~$\varphi_i$, получить аффинную нормальную форму для исходной функции:

\vspace*{1pt}

\noindent
$$
f_i(x)=g_i^{\varphi_i^{-1}}(x) = \bigoplus_{\delta\in\mathbb{F}_2^{r_i}} 
\chi_{\pi_{i,\delta}}(x)l_{i,\delta}(x)\,,
$$

\vspace*{-3pt}

\noindent
где $\pi_{i,\delta}\hm=\varphi_i(\pi'_{i,\delta})$ 
и~$l_{i,\delta}(x)\hm=l'_{i,\delta}(\varphi_i^{-1}(x))$.

Разумеется, если алгебраическая степень какой-либо функции~$f_i$ оказалась 
равной~$1$, то искать ничего не нужно: ее полином Жегалкина является ее аффинной 
нормальной формой для тривиального разбиения $\Pi_i\hm=\{V_n\}$.

\smallskip

\noindent
\textbf{Замечание~2}.
    Подобный способ построения аффинной нормальной формы можно использовать 
    и~непосредственно для квадратичной\footnote{Для функций более высоких степеней 
такой подход тоже работает, но описать его строго гораздо сложнее и~полученные 
таким образом локальные аффинности, скорее всего, будут слишком маленькой 
размерности.} функции~$f$ в~ее исходном виде. Нужно просто фиксировать значения 
переменных так, чтобы в~каждом мономе оставалось не более одной свободной 
переменной. Для этого удобнее рассмотреть матрицу~$Q_f$, выбрать в~ней столбец 
или строку с~максимальным числом единиц среди всех столбцов и~строк (пусть это 
будет $k$-я строка) и~зафиксировать~$x_k$. Затем то же проделать, исключив из 
рассмотрения $k$-ю строку и~$k$-й столбец матрицы, и~так далее, пока единицы 
в~матрице не кончатся.
Однако, несмотря на то что здесь имеет место экономия на приведении функции 
к~каноническому виду, такой способ представляется менее эффективным в~следующем 
смысле. Канонический вид квадратичной функции содержит минимальное число мономов 
степени~$2$, поэтому для исходной (неканонической) функции придется фиксировать, 
как правило, большее число переменных. Но с~каждой дополнительно зафиксированной 
переменной размерность локальных аффинностей функции~$f$ уменьшается на~$1$, 
а~их число, соответственно, увеличивается вдвое.

\vspace*{-4pt}


\section{<<Локальные>> системы линейных уравнений}

\vspace*{-2pt}

Идея метода решения систем квадратичных булевых уравнений состоит в~следующем.
Пусть для всех~$f_i$, $1\hm\leqslant i\hm\leqslant m$, 
определены некоторые аффинные нормальные 
формы

\vspace*{1pt}

\noindent
\begin{equation*}
\label{AffNF_ij}
    f_i(x) = \bigoplus_{j=1}^{\lambda(i)} \chi_{\pi_{ij}}(x)l_{ij}(x)\,.
\end{equation*}

\vspace*{-3pt}

\noindent
Исходя из этих аффинных нормальных форм, можно для каждой пары~$i,j$ записать 
эквивалентную уравнению $f_i\hm=0$ на~$\pi_{ij}$ систему линейных уравнений:

\columnbreak

\noindent
\begin{equation*}
    \begin{array}{r@{\ }c@{\ }l}
        l_{ij}(x)&=&0;\\
        h_{ij}^1(x)&=&0;\\
        \vdots\\
        h_{ij}^{d(i,j)}(x)&=&0,\\
    \end{array}
  \label{approx}
\end{equation*}
в которой первое уравнение выражает равенство~$f_i\hm=0$ через аффинную 
аппроксимацию~$l_{ij}(x)$ функции~$f_i(x)$ на плоскости~$\pi_{ij}$, а остальные 
$d(i,j)$ уравнений задают эту плоскость.


\smallskip

\noindent
\textbf{Замечание~3.}\
Если аффинная нормальная форма получена описанным выше способом~--- через 
канонический вид квадратичной функции,~--- то характеристическая функция будет 
иметь вид:

\vspace*{1pt}

\noindent
$$
\chi_{\pi_{i,\delta}}(x)=\prod_{k=1}^{r_i}(\varphi_i^{-1}(x)e_{2k-1}
\Tsf\oplus\delta_k\oplus 1)\,,
$$

\vspace*{-3pt}

\noindent
где $e_{2k-1}$~--- $(2k-1)$-й базисный вектор, т.\,е.\ $\varphi_i^{-1}(x)e_{2k-1}
\Tsf$~--- $(2k-1)$-я компонента вектора~$\varphi_i^{-1}(x)$.
Значит, соответствующую плоскость задают уравнения
 $\{ \varphi_i^{-1}(x)e_{2k-1}\Tsf \oplus \delta_k \hm= 0 
 \mid 1\hm\leqslant k\hm\leqslant r_i \}$.

\smallskip

Таким образом, имеется набор <<локальных>> линейных систем для каждого уравнения 
исходной системы и~для каждой его локальной аффинности.
Метод состоит в~том, чтобы подобрать комбинацию <<локальных>> систем разных 
квадратичных уравнений, в~совокупности дающую решение исходной системы. Если 
решение квадратичной системы единственно (а~это естественное предположение для 
криптоанализа), ровно одна такая комбинация будет иметь решение, и~от того, как 
быстро удастся ее обнаружить, зависит эффективность метода.

\vspace*{-4pt}

\section{О~трудоемкости построения аффинной нормальной формы}

\vspace*{-2pt}

Напомним, что для функций из системы~\eqref{system} $r_i\hm=({1}/{2})\rank 
\tilde{Q}_{f_i}\hm\leqslant {n}/{2}$, $1\hm\leqslant i\hm\leqslant m$,~--- 
параметр, введенный в~разд.~\ref{canonic-to-ANF}.
Алгоритм приведения $m$~функций к~каноническому виду (см.\ разд.~4) 
имеет трудоемкость, оцениваемую выражением $O(\sum\nolimits_{i=1}^m n^2r_i)$, 
а~учитывая 
неравенство $r_i\hm\leqslant {n}/{2}$, имеем~$O(mn^3)$.

При построении аффинных нормальных форм для функции~$f_i$ в~разд.~5 
потребуется порядка $r_i2^{r_i}\hm+ n^3$ операций. 
Значит, для всех~$m$~функций имеем оценку $O(mn^3\hm+\sum\nolimits_{i=1}^m r_i2^{r_i})$.

Таким образом, в~худшем случае, когда все $r_i\hm={n}/{2}$ или даже когда хотя 
бы $r_i\hm=O(n)$ для некоторого~$i$, предложенный алгоритм экспоненциален.
Однако можно рассчитывать, что во встречающихся на практике системах 
квадратичных уравнений параметр~$r_i$ растет (с~увеличением~$n$) медленнее, 
и~тогда можно говорить о полиномиальности алгоритма построения аффинных нормальных 
форм.

В случае, когда система вида~\eqref{system} переопределенная, т.\,е.\ $n\hm\ll m$ 
(переопределенные системы достаточно часто рассматриваются в~задачах 
информатики, теории кодирования и~криптографии), можно рассчитывать на 
существование подсистемы (из~$l$~уравнений с~номерами $i_1,\ldots,i_l$), для 
которой трудоемкость построения аффинных нормальных форм меньше, чем 
экспоненциальная. Например, когда $r_{i_j}\hm=O(\sqrt{n})$, $ 1\hm\leqslant j\hm\leqslant l$, 
оценка со\-от\-вет\-ст\-ву\-ющей трудоемкости для системы~\eqref{system} имеет 
субэкспоненциальный характер.

Рассмотрим в~качестве еще одного примера класс~$\mathcal{K}_m$ систем 
$m$~квадратичных булевых уравнений от $n$~неизвестных вида~\eqref{system}, где 
$m\hm=m(n)$~--- некоторый полином от~$n$ и~где $r_i=
O(\log_2 n)$ для всех~$i$, $1\hm\leqslant i\hm\leqslant m$.


\vspace*{2pt}


\noindent
\textbf{Предложение~3.}
\textit{Для систем~\eqref{system} квадратичных булевых уравнений из 
класса~$\mathcal{K}_m$ существует полиномиальный} (\textit{по~$n$}) \textit{алгоритм построения 
аффинных нормальных форм для функций~$f_i$.}


\smallskip

Для доказательства этого утверждения достаточно рассмотреть предложенный 
в~статье алгоритм построения аффинных нормальных форм для квад\-ра\-тич\-ных булевых 
функций. В~полученной выше оценке  $O(mn^3\hm+\sum\nolimits_{i=1}^m r_i2^{r_i})$
данные в~условии ограничения на~$m$ и~на~$r_i$ дают полиномиальную оценку трудоемкости 
алгоритма.

%Отметим, что если рассматривать систему квадратичных уравнений, описывающую 
%функционирование произвольного фильтрующего генератора, то у всех уравнений 
%системы будет одно и~то же значение $r_i$, определяемое рангом матрицы 
%$\tilde{Q}_{f'}$, где $f'$ "--- ... для фильтрующей функции~$f$. Поэтому

\vspace*{-12pt}

{\small\frenchspacing
 {%\baselineskip=10.8pt
 \addcontentsline{toc}{section}{References}
 \begin{thebibliography}{99}

    \bibitem{GJ1982}
        \Au{Гэри~М., Джонсон~Д.}
        Вычислительные машины и~труднорешаемые задачи~/ Пер. с~англ.~---
        М.: Мир, 1982. 416~с.
        (\Au{Garey~M.\,R., Johnson~D.\,S.} Computers and intractability: 
A~guide to the theory of NP-completeness.~--- San Francisco, CA, USA: W.\,H.~Freeman 
and Co., 1979. 348~p.).

    \bibitem{GT2017}
        \Au{Горшков~С.\,П., Тарасов~А.\,В.}
        Сложность решения сис\-тем булевых уравнений.~---
        М.: Курс, 2017. 192~с.

    \bibitem{Smi2000}
        \Au{Смирнов~В.\,Г.}
        {Некоторые классы эффективно ре\-ша\-емых систем булевых уравнений}~//
        Труды по дискретной математике, 2000. Т.~3. С.~269--282.

    \bibitem{Bard2009}
        \Au{Bard~G.\,V.}
        Algebraic cryptanalysis.~--- Springer, 2009. 389~p.

    \bibitem{BCJ2007}
        \Au{Bard~G., Courtois~N., Jefferson~C.}
        {Efficient methods for conversion and solution of sparse systems of 
        low-degree multivariate polynomials over $\mathrm{GF}(2)$ via SAT-solvers}~//
        Cryptology ePrint Archive. Report 2007/024.
        {\sf http://eprint.iacr.org/2007/024.pdf}.

    \bibitem{LSSYa2015}
        \Au{Логачев~О.\,А., Сальников~А.\,А., Смышляев~С.\,В., 
Ященко~В.\,В.}
        Булевы функции в~теории кодирования и~крип\-то\-ло\-гии.~---
        М.: ЛЕНАНД, 2015. 576~с.

    \bibitem{MWS1979}
        \Au{Мак-Вильямс~Ф.\,Дж., Слоэн~Н.\,Дж.\,А.}
        Теория кодов, исправляющих ошибки~/ Пер. с~англ.~---
        М.: Связь, 1979. 743~с.
        (\Au{MacWilliams~F.\,J., Sloane~N.\,J.\,A.} The theory of 
        error-correcting codes.~--- 
        North-Holland mathematical library ser.~---
        North-Holland Publishing Co., 1977.  774~p.)

    \bibitem{LYaD2007}
        \Au{Logachev~O.\,A., Yashchenko~V.\,V., Denisenko~M.\,P.}
        {Local affinity of Boolean mappings}~//
        Boolean functions in cryptology and information security: Proceedings of the 
NATO Advanced Study Institute.~---
        IOS Press, 2008. P.~148--172.

    \bibitem{Gor1995}
        \Au{Горшков~С.\,П.}
        {Применение теории NP-пол\-ных задач для оценки сложности решения систем 
булевых уравнений}~//
        Обозрение прикладной и~промышленной математики, 1995. Т.~2. Вып.~3. 
С.~325--398.

    \bibitem{Dickson1901}
        \Au{Dickson~L.\,E.}
        Linear groups: With an exposition of the Galois field theory.~---
        Leipzig: B.\,G.\,Teubner, 1901. 322~p.

   % \bibitem{KSh1999}
       % \Au{Kipnis~A., Shamir~A.}
      %  {Cryptanalysis of the HFE public key cryptosystem by relinearization}~//
     %   Advances in cryptology~/
    %    Ed.\ M.\,J.~Wiener.~---
   %     Lectures notes in computer science ser.~---
   %     Springer, 1999. Vol.~1666. P.~19--30.

   % \bibitem{CShPK2000}
  %      \Au{Courtois~N., Klimov~A., Patarin~J., Shamir~A.}
 %       {Efficient algorithms for solving overdefined systems of multivariate 
%polynomial equations}~// Advances in cryptology~/
%Ed.\ B.~Preneel.~---
%         Lectures notes in computer science ser.~--- Springer, 2000. Vol.~1807. 
%P.~392--407.

   % \bibitem{FY1980}
  %      \Au{Fraenkel~A.\,S., Yesha~Y.}
 %       {Complexity of solving algebraic equations}~//
 %       Inform. Process. Lett., 1980. Vol.~10. Iss.~4-5. P.~178--179.

\end{thebibliography} 
 }
 }

\end{multicols}

\vspace*{-3pt}

\hfill{\small\textit{Поступила в~редакцию 11.01.19}}

\vspace*{8pt}

%\pagebreak

%\newpage

%\vspace*{-28pt}

\hrule

\vspace*{2pt}

\hrule

%\vspace*{-2pt}

\def\tit{POLYNOMIAL ALGORITHMS FOR~CONSTRUCTING LOCAL AFFINITIES OF~QUADRATIC BOOLEAN FUNCTIONS}

\def\titkol{Polynomial algorithms for~constructing local affinities of~quadratic Boolean functions}

\def\aut{O.\,A.~Logachev$^{1,2}$, A.\,A.~Sukayev$^1$, and~S.\,N.~Fedorov$^1$}

\def\autkol{O.\,A.~Logachev, A.\,A.~Sukayev, and~S.\,N.~Fedorov}

\titel{\tit}{\aut}{\autkol}{\titkol}

\vspace*{-11pt}


\noindent
$^1$Information Security Institute,  M.\,V.~Lomonosov Moscow State University, 
1~Michurinskiy Prosp., Moscow\linebreak
$\hphantom{^1}$119192, Russian Federation

\noindent
$^2$Institute of Informatics Problems, 
Federal Research Center ``Computer Science and Control'' 
of the Russian\linebreak
$\hphantom{^1}$Academy of Sciences, 44-2~Vavilov Str., Moscow 119333, 
Russian Federation

\def\leftfootline{\small{\textbf{\thepage}
\hfill INFORMATIKA I EE PRIMENENIYA~--- INFORMATICS AND
APPLICATIONS\ \ \ 2019\ \ \ volume~13\ \ \ issue\ 1}
}%
 \def\rightfootline{\small{INFORMATIKA I EE PRIMENENIYA~---
INFORMATICS AND APPLICATIONS\ \ \ 2019\ \ \ volume~13\ \ \ issue\ 1
\hfill \textbf{\thepage}}}

\vspace*{6pt}


\Abste{Due to the affine normal form, one can consider a~Boolean function 
as affine on certain flats in its domain~--- so-called local affinities. 
This Boolean function representation~--- affine approximation~---
could be
useful 
for solving systems of nonlinear equations over two-element field. The problem 
of solving these systems
(of a~special sort) arises, in particular, in some methods 
of the information security tools design and analysis.
The
paper describes an approach to finding local affinities for quadratic Boolean 
functions which is based on Dickson's\linebreak\vspace*{-12pt}}

\Abstend{theorem. By this, one obtains affine 
normal forms for such functions. Besides, the paper concerns the efficiency of 
corresponding algorithms.
This approach can be profitable for constructing efficient methods of solving 
systems of quadratic Boolean equations via ``approximation'' of corresponding 
Boolean functions by their affine normal forms.}

\KWE{Boolean function; system of quadratic Boolean equations; vector 
space partition; flat; local affinity; Dickson's theorem; 
affine normal form (ANF) of Boolean function; algebraic cryptanalysis}






\DOI{10.14357/19922264190110}

\vspace*{-14pt}

\Ack
\noindent
The paper was partly supported by the Russian Foundation for Basic Research 
(project 18-29-03124~mk).





  \begin{multicols}{2}

\renewcommand{\bibname}{\protect\rmfamily References}
%\renewcommand{\bibname}{\large\protect\rm References}

{\small\frenchspacing
 {%\baselineskip=10.8pt
 \addcontentsline{toc}{section}{References}
 \begin{thebibliography}{99}
\bibitem{1-log-1}
\Aue{Garey, M.\,R., and D.\,S.~Johnson.} 1979. \textit{Computers and intractability: 
A~guide to the theory of NP-completeness.} San Francisco, CA: W.\,H.~Freeman and Co. 348~p.
\bibitem{2-log-1}
\Aue{Gorshkov, S.\,P., and A.\,V.~Tarasov.} 2017. \textit{Slozhnost' re\-she\-niya 
sistem bulevykh uravneniy} [Complexity of solving the systems of 
Boolean equations]. Moscow: Kurs. 192~p.
\bibitem{3-log-1}
\Aue{Smirnov, V.\,G.} 2000. Nekotorye klassy effektivno reshaemykh 
sistem bulevykh uravneniy [Some classes of Boolean equation systems 
permitting effective solution]. 
\textit{Trudy po diskretnoy matematike} [Proceedings on Discrete Mathematics] 3:269--282.
\bibitem{4-log-1}
\Aue{Bard, G.\,V.} 2009. \textit{Algebraic cryptanalysis}. Springer. 389~p.
\bibitem{5-log-1}
\Aue{Bard, G., N.~Courtois, and C.~Jefferson.} 2007. 
Efficient methods for conversion and solution of sparse systems of 
low-degree multivariate polynomials over GF(2) via SAT-solvers. 
\textit{Cryptology ePrint Archive}. Report 2007/024. Available at: 
{\sf http://eprint.iacr.org/2007/024.pdf} (accessed August~30, 2018).
\bibitem{6-log-1}
\Aue{Logachev, O.\,A., A.\,A.~Sal'nikov, S.\,V.~Smyshlyaev, and V.\,V.~Yashchenko.} 
2015. \textit{Bulevy funktsii v~teorii kodirovaniya i~kriptologii} 
[Boolean functions in coding theory and cryptology]. Moscow: LENAND. 576~p.
\bibitem{7-log-1}
\Aue{MacWilliams, F.\,J., and N.\,J.\,A.~Sloane.} 1977. 
\textit{The theory of error-correcting codes}. 
North-Holland mathematical library ser.
North-Holland Publishing Co. 774~p.
\bibitem{8-log-1}
\Aue{Logachev, O.\,A., V.\,V.~Yashchenko, and M.\,P.~Denisenko.} 2008. 
Local affinity of Boolean mappings. 
\textit{Boolean functions in cryptology and information security: 
Proceedings of the NATO Advanced Study Institute.} IOS Press. 148--172.
\bibitem{9-log-1}
\Aue{Gorshkov, S.\,P.} 1995. Primenenie teorii NP-polnykh zadach 
dlya otsenki slozhnosti resheniya sistem bulevykh uravneniy 
[Application of the NP-complete problem theory to assessment 
of complexity of solving the systems of Boolean equations]. 
\textit{Obozrenie prikladnoy i~promyshlennoy matematiki} 
[Applied and Industrial Mathematics Review] 2(3):325--398.
\bibitem{10-log-1}
\Aue{Dickson, L.\,E.} 1901. \textit{Linear groups: 
With an exposition of the Galois field theory}. Leipzig: B.\,G.~Teubner. 322~p.
%\bibitem{11-log-1}
%\Aue{Kipnis, A., and A.~Shamir.} 1999. 
%Cryptanalysis of the HFE public key cryptosystem by relinearization. 
%\textit{Advances in cryptology}. Ed. M.\,J.~Wiener.
% Lecture notes in computer science ser.  Springer. 1666:19--30.
%\bibitem{12-log-1}
%\Aue{Courtois, N., A.~Klimov, J.~Patarin, and A.~Shamir.} 2000. 
%Efficient algorithms for solving overdefined systems of multivariate polynomial 
%equations. \textit{Advances in cryptology}. Ed.\ B.~Preneel.
%Lecture notes in computer science ser.  Springer. 1807:392--407.
%\bibitem{13-log-1}
%\Aue{Fraenkel, A.\,S., and Y.~Yesha.} 1980. 
%Complexity of solving algebraic equations. 
%\textit{Inform. Process. Lett.} 10(4-5):178--179.
\end{thebibliography}

 }
 }

\end{multicols}

\vspace*{-6pt}

\hfill{\small\textit{Received January 11, 2019}}

%\pagebreak

%\vspace*{-18pt}

\Contr

\noindent
\textbf{Logachev Oleg A.} (b.\ 1950)~--- 
Candidate of Science (PhD) in physics and mathematics, head of department, 
Information Security Institute, M.\,V.~Lomonosov Moscow State University, 
1~Michurinskiy Prosp., Moscow 119192, Russian Federation; 
senior scientist, Institute of Informatics Problems, 
Federal Research Center ``Computer Science and Control'' 
of the Russian Academy of Sciences, 44-2~Vavilov Str., Moscow 119333, 
Russian Federation; \mbox{logol@iisi.msu.ru }

 



\vspace*{3pt}

\noindent
\textbf{Sukayev Al'bert A.} (b.\ 1994)~--- 
student, Information Security Institute, Moscow State University, 
1~Michurinskiy Prosp., Moscow 119192, Russian Federation; 
\mbox{asukaev@gmail.com}

\vspace*{3pt}

\noindent
\textbf{Fedorov Sergey~N.} (b.\ 1982)~--- 
Candidate of Science (PhD) in physics and mathematics, senior scientist, 
Information Security Institute, M.\,V.~Lomonosov Moscow State University, 
1~Michurinskiy Prosp., Moscow 119192, Russian Federation; 
\mbox{s.n.feodorov@yandex.ru}
\label{end\stat}

\renewcommand{\bibname}{\protect\rm Литература}        %10
\def\stat{gorshenin}

\def\tit{ЗАШУМЛЕНИЕ ДАННЫХ КОНЕЧНЫМИ СМЕСЯМИ НОРМАЛЬНЫХ 
И~ГАММА-РАСПРЕДЕЛЕНИЙ\\ С~ПРИМЕНЕНИЕМ К~ЗАДАЧЕ ОКРУГЛЕНИЯ НАБЛЮДЕНИЙ$^*$}

\def\titkol{Зашумление данных конечными смесями нормальных 
и~гамма-распределений с~применением к~задаче округления} % наблюдений}

\def\aut{А.\,К.~Горшенин$^1$}

\def\autkol{А.\,К.~Горшенин}

\titel{\tit}{\aut}{\autkol}{\titkol}

\index{Горшенин А.\,К.}
\index{Gorshenin A.\,K.}


{\renewcommand{\thefootnote}{\fnsymbol{footnote}} \footnotetext[1]
{Работа выполнена при поддержке РНФ (проект 18-71-00156).}}


\renewcommand{\thefootnote}{\arabic{footnote}}
\footnotetext[1]{Институт проблем информатики Федерального исследовательского центра 
<<Информатика и~управление>> Российской академии наук, \mbox{agorshenin@frccsc.ru}}

\vspace*{-12pt}




\Abst{Во многих реальных задачах проводится статистический анализ данных, 
содержащих дополнительные ошибки измерения, в~том числе в~виде округления, 
что в~ряде ситуаций может приводить к~достаточно существенным искажениям. 
В~настоящей статье для одной из возможных моделей округления получены оценки 
для неизвестного математического ожидания наблюдений в~предположении, что 
исходные данные дополнительно зашумлены с~по\-мощью случайных величин, 
име\-ющих распределения типа конечных смесей нормальных и~гам\-ма-за\-ко\-нов. 
Построены доверительные интервалы для неизвестного математического ожидания 
с~использованием уточненной оценки для дисперсии целой части случайной величины. 
Обсуждается алгоритм определения значения параметра для искусственного шума, 
добавление которого к~исходным данным способствует повышению качества работы 
метода скользящего разделения смесей.}

\KW{зашумленные данные; округленные наблюдения; конечные смеси нормальных 
распределений; конечные смеси гам\-ма-рас\-пре\-де\-ле\-ний; доверительные интервалы;  
метод скользящего разделения смесей}

\DOI{10.14357/19922264180304}
  
\vspace*{-4pt}


\vskip 10pt plus 9pt minus 6pt

\thispagestyle{headings}

\begin{multicols}{2}

\label{st\stat}


\section{Введение}

Во многих реальных задачах данные, являющиеся непрерывными по своей сути, 
регистрируются с~помощью инструментов, вносящих дополнительные ошибки 
измерения, в~том чис\-ле в~виде округления. Таким образом, статистический 
анализ проводится не для исходных, а для преобразованных некоторым 
случайным образом наблюдений, что в~ряде ситуаций может приводить к~достаточно
 существенным искажениям.

Для преодоления данной проблемы развивались различные подходы, в~том числе 
на основе смешанных моделей (см., например, статью~\cite{Wright2003}, в~которой 
различные компоненты  используются для пред\-став\-ле\-ния уровней округления). 
В~работе~\cite{Bai2009} приводятся результаты для моделей авторегрессии и~скользящего 
среднего для округленных данных, а~в~статье~\cite{Zhang2010} эти результаты 
развиваются и~исследуются их асимптотические свойства. 
В~статье~\cite{Zhao2012} исследован метод оценивания па\-ра\-мет\-ров конечных смесей 
вероятностных распределений (в~том чис\-ле, и~многомерных) 
на основе использования EM (expectation-maximization) 
алгоритма~\cite{Korolev2011-i} с~\mbox{целью} получения состоятельных 
и~асимптотически нормальных оценок.

В настоящей статье развиваются результаты для моделей округления, 
описанных в~работах~\cite{Ushakov2015,Ushakov2017a,Ushakov2017b}. 
В~их рамках будут получены оценки для неизве\-ст\-ного математического ожидания 
наблюдений в~предположении, что исходные данные зашумлены с~по\-мощью случайных 
величин, имеющих распределения типа конечных смесей нормальных и~гам\-ма-за\-ко\-нов. 
Это позволяет учесть большее количество случайных факторов, влия\-ющих на величину 
<<дополнительной>> ошибки. Также будут построены доверительные интервалы для 
неизвестного математического ожидания. Выражения для гам\-ма-рас\-пре\-де\-ле\-ний 
получены впервые. Также обсуждается алгоритм определения значения па\-ра\-мет\-ра для 
искусственного шума, добавление которого к~исходным данным способствует 
повышению качества работы метода скользящего разделения смесей~\cite{Gorshenin2016}.

\vspace*{-12pt}

\section{Предположения и~базовые отношения}

Для сокращения формулировок теорем в~сле\-ду\-ющих разделах сделаем ряд 
предположений, на которые будем ссылаться в~дальнейшем. Итак, пусть:
\begin{itemize}
\item[(A)] $X_1,X_2,\ldots$~--- независимые одинаково распределенные 
случайные величины с~неизвестным математическим ожиданием ${\sf E}_X\hm<+\infty$;
\item[(B)] $\varepsilon_1,\varepsilon_2,\ldots$~--- независимые одинаково 
распределенные случайные величины с~математическим ожиданием 
${\sf E}_\varepsilon\hm<+\infty$; %\label{B}
\item[(C)] $X_1,X_2,\ldots$ и~$\varepsilon_1,\varepsilon_2,\ldots$ 
являются независимыми;
\item[(D)] $Y_j=\left[X_j+\varepsilon_j+1/2\right]$ для всех $j\hm=1,2,\ldots$ 
представляют собой округление значения суммы случайных величин $X_j\hm+\varepsilon_j$ 
до ближайшего целого сверху (при этом запись~$[\cdot]$ соответствует целой 
части выражения).
\end{itemize}

В рамках данных предположений в~статье будут рассмотрены вопросы качества 
приближения неизвестного математического ожидания~${\sf E}_X$ для исходных данных 
в~ситуации, когда наблюдения для анализа получены с~аддитивной ошибкой c известными 
распределениями (см.\ предположение~(B)) и~дополнительно округляются до 
ближайшего целого (см.\ предположение~(D)).

Заметим, что в~силу усиленного закона больших чисел справедливы следующие выражения:
\begin{multline}
\fr{1}{n}\sum\limits_{j=1}^n Y_j\xrightarrow[n\to\infty]{\text{п.н.}}
{\sf E}_Y\equiv\mathbb{E}\left[X_1+\varepsilon_1+\fr{1}{2}\right]={}\\
{}=\mathbb{E}\left(X_j+\varepsilon_j+\fr{1}{2}\right)-\mathbb{E}
\left\{X_j+\varepsilon_j+\fr{1}{2}\right\}={}\\
{}={\sf E}_X+{\sf E}_\varepsilon+\fr{1}{2}-\mathbb{E}\left\{X_j+\varepsilon_j+\fr{1}{2}\right\}. 
\label{Law}
\end{multline}

Запись $\{\cdot\}$ в~формуле~\eqref{Law} соответствует дробной 
части выражения, а~п.н.\ обозначает сходимость в~смысле почти наверное.

Для доказательства результатов в~дальнейшем потребуется следующее 
представления для дробной части  абсолютно непрерывной случайной величины~$Z$ 
с~абсолютно  интегрируемой характеристической функцией~$\varphi_Z(t)$
 (см., например, Лемму~4 в~работе~\cite{Ushakov2017b}):
\begin{equation}
\label{Fract}
\mathbb{E}\{Z\}=\fr{1}{2}-\sum\limits_{n=1}^\infty 
\fr{\mathrm{Im}\left (\varphi_Z(2\pi n)\right)}{\pi n}\,.
\end{equation}

Через $\mathrm{Im}\,(\cdot)$ в~формуле~\eqref{Fract} обозначена мнимая часть 
соответствующей функции.

При построении доверительных интервалов в~дальнейшем будет 
использована следующая оценка, справедливая для любой случайной величины~$Z$:
\begin{equation}
\mathbb{D}[Z]\leqslant \left(\sqrt{\mathbb{D} Z}+\fr{1}{2}\right)^2.
\label{Var}
\end{equation}
Она может быть проверена непосредственно с~учетом представления 
$\mathbb{D} [Z]\hm=\mathbb{D}\left(Z\hm-\{Z\}\right)$, неравенства 
Ко\-ши--Бу\-ня\-ков\-ско\-го для ковариации и~соотношения 
 $\mathbb{D}\{Z\}\hm\leqslant 1/4$, справедливого для любой случайной величины~$Z$ 
 (см., например, статью~\cite{Ushakov2017b}). Отметим, что данная оценка 
 является более точной по сравнению с~использованным для аналогичных 
 целей в~работе~\cite{Ushakov2017b} соотношением 
 $\mathbb{D} [Z]\hm\leqslant 2\mathbb{D} Z\hm+1/2$. Действительно,
\begin{equation*}
2\mathbb{D} Z+\fr{1}{2}-\left(\sqrt{\mathbb{D} Z}+\fr{1}{2}\right)^2=
\left(\sqrt{\mathbb{D} Z}-\fr{1}{2}\right)^2\geqslant0\,,
\end{equation*}
причем для всех $\sqrt{\mathbb{D} Z}\hm\neq {1}/{2}$ 
данное неравенство является строгим.

\section{Конечные смеси нормальных законов}

Для случайной величины~$X$, имеющей распределение типа 
конечной смеси нормальных законов~\cite{Korolev2011-i} с~параметрами 
${\bf a}\hm=(a_1,\ldots, a_k)$, $a_j\hm\in \mathbb{R}$, 
$\boldsymbol{\sigma}\hm=(\sigma_1,\ldots, \sigma_k)$, 
$\sigma_j\hm>0$, ${\bf p}\hm=(p_1,\ldots, p_k)$, $p_j\hm\geqslant 0$, 
$\sum\nolimits_{j=1}^{k}p_j\hm=1$, плот\-ность которого задается выражением
\begin{equation}
f_X(x)=\sum\limits_{j=1}^{k}\fr{p_j}{\sigma_j\sqrt{2\pi}}\,e^{-(x-a_j)^2/(2\sigma_j^2)}\,,
\label{FinNormMixt}
\end{equation}
характеристическая функция имеет вид:
\begin{equation}
\varphi_X(t)=\int\limits_{-\infty}^{+\infty}\!\!e^{itx} f_X(x)\, dx = 
\sum\limits_{j=1}^{k}p_j e^{ita_j-\sigma_j^2 t^2/2}.
\label{ChiFinNormMixt}
\end{equation}

Абсолютная интегрируемость  $\varphi_X(t)$ вытекает из свойств 
характеристической функции нормального распределения. 
Заметим, что в~точке $t\hm=2\pi n$ выражение~\eqref{ChiFinNormMixt} принимает 
сле\-ду\-ющий вид:
\begin{equation}
\label{ChiFinNormMixt2npi}
\varphi_X(2\pi n)= \sum\limits_{j=1}^{k}p_j e^{-2\pi^2 \sigma_j^2 n^2}\,.
\end{equation}

Рассмотрим вопрос точности оценивания неизвестного математического ожидания~${\sf E}_X$ 
при до\-бав\-ле\-нии зашумления.

\smallskip

\noindent
\textbf{Теорема~1.}\ 
\textit{Пусть выполнены предположения}~(A)--(D), 
\textit{причем случайные величины~$\varepsilon_j$, $j\hm=1,2,\ldots$, 
имеют распределение типа конечной $k$-ком\-по\-нент\-ной смеси нормальных законов 
вида}~\eqref{FinNormMixt} \textit{с~па\-ра\-мет\-ра\-ми~${\bf a}$, $\boldsymbol{\sigma}$ 
и~${\bf p}$. Тогда}
\begin{equation}
\label{Th1Eq}
\left\lvert {\sf E}_Y-{\sf E}_X\right\rvert \leqslant 
A+\fr{1}{\pi}\left(1+\fr{1}{4\pi^2\sigma^2}\right)e^{-2\pi^2\sigma^2}\,, 
\end{equation}
\textit{где} $A=\max(|a_1|,\ldots,|a_k|)$, $\sigma\hm=\min(\sigma_1,\ldots,\sigma_k)$.

\smallskip


\noindent
Д\,о\,к\,а\,з\,а\,т\,е\,л\,ь\,с\,т\,в\,о\,.\ \
С~учетом пред\-став\-ле\-ний~\eqref{Law},~\eqref{Fract} и~\eqref{ChiFinNormMixt2npi}, 
ограниченности модуля характеристической функции, а~также не\-за\-ви\-си\-мости 
случайных величин~$X_j$ и~$\varepsilon_j$ имеем:
\begin{multline*}
\left\lvert {\sf E}_Y-{\sf E}_X\right\rvert =
\left\lvert {\sf E}_\varepsilon+\fr{1}{2}-\mathbb{E}\left\{X_j+
\varepsilon_j+\fr{1}{2}\right\}\right\rvert={}\\
{}=\left\lvert {\sf E}_\varepsilon+\sum\limits_{n=1}^\infty
\fr{\mathrm{Im} \left(\varphi_{X_j}(2\pi n)\varphi_{\varepsilon_j}(2\pi n)
\varphi_{1/2}(2\pi n)\right)}{\pi n}\right\rvert={}\\
=\left\lvert 
\vphantom{\fr{(-1)^n\sum\nolimits_{j=1}^{k}p_j e^{-2\pi^2 \sigma_j^2 n^2} 
\mathrm{Im} \left(\varphi_{X_j}(2\pi n)\right)}{\pi n}}
{\sf E}_\varepsilon+{}\right.\\
\left.{}+\sum\limits_{n=1}^\infty
\fr{\mathrm{Im} \left(\varphi_{X_j}(2\pi n) 
\sum\nolimits_{j=1}^{k}p_j e^{-2\pi^2 \sigma_j^2 n^2} 
e^{\pi n}\right)}{\pi n}\right\rvert={}\\
{}=\left\lvert 
\vphantom{\fr{(-1)^n\sum\nolimits_{j=1}^{k}p_j e^{-2\pi^2 \sigma_j^2 n^2} 
\mathrm{Im} \left(\varphi_{X_j}(2\pi n)\right)}{\pi n}}
{\sf E}_\varepsilon+{}\right.\\
\left.{}+\sum\limits_{n=1}^\infty
\fr{(-1)^n\sum\nolimits_{j=1}^{k}p_j e^{-2\pi^2 \sigma_j^2 n^2} 
\mathrm{Im} \left(\varphi_{X_j}(2\pi n)\right)}{\pi n}\right\rvert\leqslant{}\\
{}\leqslant \left\lvert {\sf E}_\varepsilon\right\rvert+\left\lvert\
\sum\limits_{j=1}^{k}p_j\sum\limits_{n=1}^\infty 
\fr{1}{\pi n} e^{-2\pi^2 \sigma_j^2 n^2}\right\rvert\leqslant {}\\
\\
{}\leqslant
\max\left(|a_1|,\ldots,|a_k|\right)+{}\\
{}+\sum\limits_{j=1}^{k} 
\fr{p_j}{\pi} \left(\!1+\fr{1}{4\pi^2\sigma_j^2}\!\right)\!e^{-2\pi^2\sigma_j^2}\leqslant{}\\
{}\leqslant
A+\fr{1}\pi\left(1+\fr{1}{4\pi^2\sigma^2}\right)e^{-2\pi^2\sigma^2}\,.
\end{multline*}

Справедливость использованной оценки 
\begin{equation*}
\sum\limits_{n=1}^\infty
\fr{e^{-2\pi^2 \sigma_j^2 n^2}}{n}\leqslant 
\left(1+\fr{1}{4\pi^2\sigma_j^2}\right)e^{-2\pi^2\sigma_j^2}
\end{equation*}
может быть проверена непосредственно (например, см.\ доказательство Теоремы~6 
в~статье~\cite{Ushakov2017b}).~\hfill$\square$

\smallskip

\noindent
\textbf{Замечание~1.}
В~случае, если зашумление производится нормально распределенными случайными 
величинами c нулевыми средними (т.\,е.\ в~формуле~\eqref{Th1Eq} необходимо считать 
$A\hm=0$, $k\hm=1$), то Тео\-ре\-ма~1 совпадает с~результатом, 
полученным в~работе~\cite{Ushakov2017b}.


\smallskip

Рассмотрим вопросы построения доверительного интервала для неизвестного 
математического ожидания~${\sf E}_X$ в~предположении, что случайные величины~$X_j$ не 
содержат ошибок измерения, а~все погрешности учтены исключительно в~за\-шум\-ля\-ющих 
элементах~$\varepsilon_j$.

\smallskip

\noindent
\textbf{Теорема~2.}\ 
\textit{Пусть выполнены предположения}~(A)--(D), 
\textit{причем случайные величины~$\varepsilon_j$, $j\hm=1,2,\ldots$, имеют 
распределение типа конечной $k$-ком\-по\-нент\-ной смеси нормальных законов 
вида}~\eqref{FinNormMixt} \textit{с~параметрами~${\bf a}$, $\boldsymbol{\sigma}$ 
и~${\bf p}$, а~случайные величины} $X_j\stackrel{\text{п.н.}}{=}{\sf E}_X$, $j\hm=1,2,\ldots$ 
\textit{Тогда доверительный интервал для~${\sf E}_X$ при условии $0\hm<\alpha\hm<1$ имеет вид}:
\begin{equation} 
\label{Th2Eq}
\hat{{\sf E}}_X - f({\bf a},\boldsymbol{\sigma},\alpha,n) 
\leqslant {\sf E}_X \leqslant  \hat{{\sf E}}_X + f({\bf a},\boldsymbol{\sigma},\alpha,n),
\end{equation}
\textit{где}

\vspace*{-2pt}

\noindent
\begin{align}
\hat{{\sf E}}_X&=\fr{1}{n} \sum\limits_{j=1}^{n} Y_j\,; \label{Th2hatE}\\
f({\bf a},\boldsymbol{\sigma},\alpha,n)&=
\fr{z_{1-{\alpha}/2}}{\sqrt{n}} \left(\sqrt{A^2+\Sigma^2}+\fr{1}{2}\right) +{}\notag\\
&{}+A+\fr{1}\pi\left(1+\fr{1}{4\pi^2\sigma^2}\right)e^{-2\pi^2\sigma^2}\,;
  \label{Th2f}
\end{align}
\textit{$z_{1-{\alpha}/2}$~--- $\left(1-{\alpha}/2\right)$-кван\-тиль 
стандартного нормального распределения; $A\hm=\max(|a_1|,\ldots,|a_k|)$; 
$\Sigma\hm=\max(\sigma_1,\ldots,\sigma_k)$; $\sigma\hm=\min(\sigma_1,\ldots,\sigma_k)$}. 


\smallskip

\noindent
\noindent
Д\,о\,к\,а\,з\,а\,т\,е\,л\,ь\,с\,т\,в\,о\,.\ \
Из центральной предельной тео\-ре\-мы с~учетом условия~(A) следует, 
что величина~$\hat{{\sf E}}_X$~\eqref{Th2hatE} асимптотически нормальна с~математическим 
ожиданием 
\begin{equation}
{\sf E}_Y\equiv \mathbb{E}\left[{\sf E}_X+\varepsilon_1+\fr{1}{2}\right] \label{EY}
\end{equation}
и дисперсией
\begin{equation}
\fr{1}{n} {\sf D}_Y\equiv \fr{1}{n}\mathbb{D}\left[{\sf E}_X+\varepsilon_1+
\fr{1}{2}\right]. \label{DY}
\end{equation}

Воспользовавшись оценкой~\eqref{Var}, получим:

\vspace*{-2pt}

\noindent
\begin{multline*}
{\sf D}_Y \leqslant  \left(\sqrt{\mathbb{D} \left({\sf E}_X+\varepsilon_1+\fr{1}{2}\right)}+
\fr{1}{2}\right)^2={}\\
{}=
\left(\sqrt{\mathbb{D}\varepsilon_1}+\fr{1}{2}\right)^2= {}\\
{}= \left(\sqrt{\sum\limits_{j=1}^{k}p_j\left(\left(a_j-\sum\limits_{t=1}^{k}
p_t a_t\right)^2+\sigma_j^2\right)}+\fr{1}{2}\right)^2\leqslant {}\\ 
{}\leqslant \left(\sqrt{A^2+\Sigma^2}+\fr{1}{2}\right)^2\,.
\end{multline*}
Тогда доверительный интервал уровня $1\hm-\alpha$ для математического ожидания~${\sf E}_Y$ 
имеет вид:
\begin{equation*}
\mathbb{P}\left(\left\lvert \hat{{\sf E}}_X-{\sf E}_Y\right\rvert \leqslant 
\fr{z_{1-{\alpha}/2}}{\sqrt{n}} 
\left(\sqrt{A^2+\Sigma^2}+\fr{1}{2}\right)\right)\geqslant 1-\alpha\,.
\end{equation*}

\begin{table*}[b]\small
\begin{center}

\begin{tabular}{|c|c|c|c|c|c|c|c|}
\multicolumn{7}{p{100mm}}{Численные решения уравнений~\eqref{f1} и~\eqref{f2} относительно 
параметра~$\sigma$ для некоторых значений~$n$ и~$\alpha$}\\
\multicolumn{7}{c}{\ }\\[-6pt]
\hline
\multicolumn{1}{|c|}{Размер}  & \multicolumn{2}{c|}{Уровень $\alpha=0{,}1$}& 
\multicolumn{2}{c|}{Уровень $\alpha=0{,}05$}& 
\multicolumn{2}{c|}{Уровень $\alpha=0{,}01$}\\
\cline{2-7}
\multicolumn{1}{|c|}{выборки $n$}&$\sigma_1$&$\sigma_2$&$\sigma_1$&$\sigma_2$&$\sigma_1$&$\sigma_2$\\
\hline
$\hphantom{000}100$&$0{,}4302$&$0{,}435$&$0{,}419$&$0{,}425$&$0{,}4002$&$0{,}408$\\
%\hline
$\hphantom{000}200$&$0{,}452$&$0{,}455$ &$0{,}441$&$0{,}445$&$0{,}424$&$0{,}429$\\
%\hline
$\hphantom{00}1000$&$0{,}499$&$0{,}499$ &$0{,}489$&$0{,}489$&$0{,}473$&$0{,}475$\\
%\hline
$\hphantom{0}10000$&$0{,}558$&$0{,}556$ &$0{,}549$&$0{,}547$&$0{,}536$&$0{,}534$\\
%\hline
$100000$&$0{,}611$&$0{,}607$ &$0{,}603$&$0{,}599$&$0{,}591$&$0{,}588$\\
\hline
\end{tabular}
\end{center}
\end{table*}


\noindent
Откуда следует справедливость соотношения~\eqref{Th2Eq} c~уче\-том 
очевидного неравенства

\pagebreak

\noindent
\begin{equation*}
\left\lvert \hat{{\sf E}}_X-{\sf E}_X\right\rvert \leqslant 
\left\lvert \hat{{\sf E}}_X-{\sf E}_Y\right\rvert +\left\lvert {\sf E}_Y-{\sf E}_X\right\rvert 
\end{equation*}
и оценки~\eqref{Th1Eq} из Теоремы~1.~\hfill$\square$

\smallskip

\noindent
\textbf{Замечание~2.}
В~работе~\cite{Gorshenin2016} было продемонстрировано повышение точ\-ности 
работы метода скользящего разделения конечных нормальных смесей за счет 
введения дополнительной компоненты, имеющей нормальное 
распределение $\mathcal{N}(0,\sigma^2)$ с~математическим ожиданием, равным~$0$, 
и~стандартным отклонением~$\sigma$. При этом была отмечена сложность выбора 
параметра~$\sigma$ для сохранения структуры выборки, близкой к~исходной. 
Результат Теоремы~2 может быть использован с~данной целью, если положить $k\hm=1$, 
$a_j\hm=0$ для всех $j\hm=1,2,\ldots$ и~выбирать величину~$\sigma$ как 
минимизирующую длину доверительного интервала~\eqref{Th2Eq}. Для 
этого необходимо найти производную функции $f(0,\sigma,\alpha,n)$~\eqref{Th2f} 
и~численно решить уравнение
\begin{multline}
f_\sigma'(0,\sigma,\alpha,n)\equiv \fr{z_{1-{\alpha}/2}}{\sqrt{n}} - {}\\
{}-
e^{-2\pi^2\sigma^2}\left(4\pi\sigma+\fr{1}{2\pi^3\sigma^3}+
\fr{1}{\pi\sigma}\right)=0
\label{f1}
\end{multline}
относительно неизвестного параметра~$\sigma$ при выбранных значениях величин~$n$ 
и~$\alpha$. В~качестве альтернативы можно использовать вид доверительного интервала 
из статьи~\cite{Ushakov2017b}, полученный с~помощью неравенства $\mathbb{D} [Z]
\hm\leqslant 2\mathbb{D} Z\hm+{1}/{2}$, и~искать решение уравнения вида:
\begin{multline}
\hspace*{-2.90578pt}\fr{2\sigma z_{1-{\alpha}/2}}{\sqrt{n (2\sigma^2+{1}/{2})}} -
 e^{-2\pi^2\sigma^2}\left(4\pi\sigma+\fr{1}{2\pi^3\sigma^3}+
 \fr{1}{\pi\sigma}\right)={}\\
 {}=0\,.\label{f2}
\end{multline}

Примеры найденных значений~$\sigma$ для типичных размеров выборок в~методе 
скользящего разделения смесей (учитываются как возможная ширина окна, 
так и~общее количество наблюдений в~анализируемом ряде) приведены в~таблице 
(использован метод оптимизации \verb"Trust-Region Dogleg" пакета \verb"MATLAB" 
c~настройками по умолчанию), в~которой через~$\sigma_1$ обозначено приближенное  
решение уравнения~\eqref{f1}, a~через $\sigma_2$~--- уравнения~\eqref{f2}.


Проверка практической эффективности данного подхода в~качестве 
критерия выбора параметров зашумляющего распределения для повышения 
точности работы метода скользящего разделения смесей может быть отмечена 
как задача для дальнейших исследований.


\section{Конечные смеси гамма-распределений}

Для случайной величины~$X$, имеющей распределение типа конечной смеси 
гам\-ма-рас\-пре\-де\-ле\-ний с~параметрами ${\bf r}\hm=(r_1,\ldots, r_k)$,
 $r_j\hm>0$, $\boldsymbol{\lambda}\hm=(\lambda_1,\ldots, \lambda_k)$, $\lambda_j\hm>0$, 
 ${\bf p}\hm=(p_1,\ldots, p_k)$, $p_j\hm\geqslant 0$, $\sum\nolimits_{j=1}^{k}p_j\hm=1$, 
 плот\-ность которого задается выражением
\begin{equation}
f_X(x)=\sum\limits_{j=1}^{k}p_j\fr{\lambda_j^{r_j} e^{-\lambda_j x}}
{\Gamma(r_j)}\,x^{r_j-1}\,,
\label{FinGammaMixt}
\end{equation}
характеристическая функция имеет следующий вид:
%характеристическая функция задается следующим выражением:
\begin{equation}
\varphi_X(t)=\!\int\limits_{-\infty}^{+\infty}\!\!\!e^{itx} f_X(x)\, dx = \!
\sum\limits_{j=1}^{k}p_j \left(\!1-\fr{it}{\lambda_j}\right)^{-r_j}\!.\!
\label{ChiFinGammaMixt}
\end{equation}

Отметим, что подобные модели зашумления разумно использовать в~случае, 
если известно, что данные сосредоточены на положительной полуоси, например 
при анализе различных информационных потоков (см., в~част\-ности, 
 работу~\cite{Gorshenin2013}). 

Проверим абсолютную интегрируемость функции $\varphi_X(t)$~\eqref{ChiFinGammaMixt}. 
Имеем:
\begin{multline*}
\int\limits_{-\infty}^{+\infty}\left\lvert\varphi_X(t)\right\rvert\, dt\leqslant 
\sum\limits_{j=1}^{k}p_j \int\limits_{-\infty}^{+\infty}\left\lvert \left(
1-\fr{it}{\lambda_j}\right)^{-r_j}\right\rvert \, dt={}\\
{}=\sum\limits_{j=1}^{k}p_j \int\limits_{-\infty}^{+\infty} \left\lvert\left(
\fr{\lambda_j(\lambda_j+it)}{\lambda_j^2+t^2}\right)^{r_j}\right\rvert\, dt \leqslant{}\\
{}\leqslant\sum\limits_{j=1}^{k}p_j \lambda_j \int\limits_{-\infty}^{+\infty}\left(
1+y^2\right)^{-{r_j}/{2}}\, dy\,.
\end{multline*}

Подынтегральное выражение при $r_j\hm\geqslant 2$ может быть оценено сверху 
функцией $1/({1+y^2})$, при этом соответствующий интеграл равен~$\pi$, что влечет 
абсолютную интегрируемость характеристической функции для конечной смеси 
гам\-ма-рас\-пре\-де\-ле\-ний. Поэтому в~дальнейшем будем предполагать,
 что $r_j\hm\geqslant 2$ для всех возможных значений $j\hm=1,2,\ldots$

Рассмотрим вопрос точ\-ности оценивания неизвестного математического ожидания ${\sf E}_X\hm>0$ 
при добавлении зашумления.

\smallskip

\noindent
\textbf{Теорема~3.}
\textit{Пусть выполнены предположения}~(A)--(D), 
\textit{причем случайные величины~$\varepsilon_j$, $j\hm=1,2,\ldots$, имеют 
распределение типа конечной $k$-ком\-по\-нент\-ной смеси 
гам\-ма-рас\-пре\-де\-ле\-ний вида}~\eqref{FinGammaMixt} 
\textit{с~па\-ра\-мет\-ра\-ми~${\bf r}$, $\boldsymbol{\lambda}$ и~${\bf p}$. Тогда}
\begin{equation}
\label{Th3Eq}
\left\lvert {\sf E}_Y-{\sf E}_X\right\rvert \leqslant \fr{R}{\lambda}+
\fr{\Lambda^{R}}{2^{r}\pi^{r+1}}\left(1+\frac1{r}\right)\,,
\end{equation}
\textit{где} $r=\min(r_1, \ldots,r_k)$; $R\hm=\max(r_1, \ldots,r_k)$; 
$\lambda\hm=\max(\lambda_1, \ldots,\lambda_k)$; 
$\Lambda\hm=\max(\lambda_1, \ldots,\lambda_k)$.

\smallskip

\noindent
Д\,о\,к\,а\,з\,а\,т\,е\,л\,ь\,с\,т\,в\,о\,.\ \
С~учетом пред\-став\-ле\-ний~\eqref{Law} и~\eqref{Fract}, ограниченности 
модуля характеристической функции, перехода от тригонометрической к~показательной 
записи комплексных чисел, а~также независимости случайных величин~$X_j$ 
и~$\varepsilon_j$ \mbox{имеем}:
\begin{multline*}
\left\lvert {\sf E}_Y-{\sf E}_X\right\rvert
\leqslant \left\lvert {\sf E}_\varepsilon\right\rvert+ {}\\
{}+\left\lvert\sum\limits_{n=1}^\infty
\left(
(-1)^n\mathrm{Im} \left(\sum\limits_{j=1}^{k}p_j \varphi_{X_j}(2\pi n)\left(
\vphantom{\fr{2\pi n}{\lambda_j}}
1-{}\right.\right.\right.\right.\\
\left.\left.\left.\left.{}-i\left(\fr{2\pi n}{\lambda_j}\right)\right)^{-r_j}\right)
\Bigg/ ({\pi n})
\vphantom{\sum\limits_{j=1}^{k}}
\right)\right\rvert={}\\
{}=\left\lvert {\sf E}_\varepsilon\right\rvert+ 
\left\lvert\sum\limits_{n=1}^\infty
\left(\!(-1)^n\mathrm{Im} \!\left(\sum\limits_{j=1}^{k}p_j \left(\!
1+\fr{4\pi^2 n^2}{\lambda_j^2}\right)^{- {r_j}/2}\!\times{}\right.\right.\right.\hspace*{-2.8663pt}\\
\left.\left.\left.{}\times \varphi_{X_j}(2\pi n)\,
e^{-ir_j\mathrm{arctan}\,({{t}/{\lambda_j}})}\right)
\Bigg/
({\pi n})
\vphantom{\left(
1+\fr{4\pi^2 n^2}{\lambda_j^2}\right)^{- {r_j}/2}}
\right)\right\rvert\leqslant{}\\
{}\leqslant \left\lvert {\sf E}_\varepsilon\right\rvert+\sum\limits_{j=1}^{k}
p_j\sum\limits_{n=1}^\infty\fr{1}{\pi n}\left(
1+\fr{4\pi^2 n^2}{\lambda_j^2}\right)^{-{r_j}/2}\leqslant{}\\
{}\leqslant  \fr{R}\lambda + \sum\limits_{j=1}^{k}p_j
\sum\limits_{n=1}^\infty\left(\fr{1}{\pi n}\,
\fr{\lambda_j^{r_j}}{(2\pi)^{r_j} n^{r_j}}\right)\leqslant {}
\\
{}\leqslant  \fr{R}{\lambda} + \sum\limits_{j=1}^{k}p_j 
\fr{\lambda_j^{r_j}}{2^{r_j}\pi^{r_j+1}}\left(1+\int\limits_{1}^{\infty}
\fr{1}{ x^{r_j+1}}\,dx\right)
\leqslant{}\\
{}\leqslant \fr{R}{\lambda}+\fr{\Lambda^{R}}{2^{r}\pi^{r+1}}\left(1+\fr{1}{r}\right).
\end{multline*}

При переходе от суммы к~интегралу используется факт убывания функции как переменной~$n$ 
(или~$x$).~\hfill$\square$


\smallskip

\noindent
\textbf{Замечание~3.}\
Теорема~3 описывает соответ\-ст\-ву\-ющий результат для гам\-ма-рас\-пре\-де\-лен\-ных 
за\-шум\-ля\-ющих случайных величин, если положить $k\hm=1$ в~выражении~\eqref{Th3Eq}. 
При этом, очевидно, $r\hm\equiv R$ и~$\lambda\hm\equiv \Lambda$.


\smallskip

Рассмотрим вопросы построения доверительного интервала для неизвестного 
математического ожидания ${\sf E}_X\hm>0$ в~предположении, что случайные величины~$X_j$ 
не содержат ошибок измерения, а все погрешности учтены исключительно в~за\-шум\-ля\-ющих 
элементах~$\varepsilon_j$.

\smallskip

\noindent
\textbf{Теорема~4.}
\textit{Пусть выполнены предположения}~(A)--(D), 
\textit{причем случайные величины~$\varepsilon_j$, $j\hm=1,2,\ldots$, имеют 
распределение типа конечной $k$-ком\-по\-нент\-ной смеси 
гам\-ма-рас\-пре\-де\-ле\-ний вида}~\eqref{FinGammaMixt} 
\textit{с~па\-ра\-мет\-ра\-ми~${\bf r}$, $\boldsymbol{\lambda}$ и~${\bf p}$, 
а~случайные величины} $X_j\stackrel{\text{п.н.}}{=}{\sf E}_X$, $j=1,2,\ldots$ 
\textit{Тогда доверительный интервал для~${\sf E}_X$ при условии $0\hm<\alpha\hm<1$ имеет вид}:
\begin{equation} 
\label{Th4Eq}
\left\lvert {\sf E}_X - \hat{{\sf E}}_X\right\rvert \leqslant  
f({\bf r},\boldsymbol{\lambda},\alpha,n),
\end{equation}
\textit{где}

\vspace*{-9pt}

\noindent
\begin{align}
\hat{{\sf E}}_X&=\fr{1}{n} \sum\limits_{j=1}^{n} Y_j\,; \label{Th4hatE}\\[-4pt]
f({\bf r}, \boldsymbol{\lambda},\alpha,n)&=\fr{z_{1-{\alpha}/2}}{\sqrt{n}} \left(
\sqrt{\fr{R(R+1)}{\lambda^2}-\fr{r^2}{\Lambda^2}}+\fr{1}{2}\right) +{}\notag\\[-1pt]
&\hspace*{20mm}{}+
\fr{R}{\lambda}+\fr{\Lambda^{R}}{2^{r}\pi^{r+1}}\left(1+\fr{1}{r}\right); \notag
\end{align}
\textit{$z_{1-{\alpha}/2}$~--- $\left(1-{\alpha}/2\right)$-кван\-тиль 
стандартного нормального распределения; $r\hm=\min(r_1, \ldots,r_k)$; 
$R\hm=\max(r_1, \ldots,r_k)$; $\lambda\hm=\max(\lambda_1, \ldots,\lambda_k)$; 
$\Lambda\hm=\max(\lambda_1, \ldots,\lambda_k)$}. 

\smallskip

\noindent
Д\,о\,к\,а\,з\,а\,т\,е\,л\,ь\,с\,т\,в\,о\,.\ \
Из центральной предельной теоремы с~учетом условия~(A) 
следует, что величина~$\hat{{\sf E}}_X$~\eqref{Th4hatE} асимптотически нормальна 
с~математическим ожиданием~${\sf E}_Y$~\eqref{EY} и~дисперсией $(1/n){\sf D}_Y$~\eqref{DY}. 
Пользуясь определением и~свойствами гам\-ма-функ\-ции, а~также оценкой~\eqref{Var} 
получим:

\noindent
\begin{multline*}
{\sf D}_Y \leqslant \left(\sqrt{\sum\limits_{j=1}^k p_j
\fr{\lambda_j^{r_j}}{\Gamma(r_j)} \int\limits_{0}^{+\infty} 
e^{\lambda_j x}x^{r_j+1}\, dx}+\fr{1}{2}\right)^2= {}\\[-0.5pt]
= \left(\sqrt{\sum\limits_{j=1}^{k}p_j
\fr{r_j(r_j+1)}{\lambda_j^2}-\left(\sum\limits_{j=1}^{k}p_j
\fr{r_j}{\lambda_j}\right) ^2}+\fr{1}{2}\right)^2\leqslant {}\\[-1.5pt]
{}\leqslant \left(\sqrt{\fr{R(R+1)}{\lambda^2}-\fr{r^2}{\Lambda^2}}+\fr{1}{2}\right)^2\,.
\end{multline*}

Аналогично доказательству Тео\-ре\-мы~2 с~учетом оценки~\eqref{Th3Eq} 
отсюда следует справедливость соотношения~\eqref{Th4Eq}.~\hfill$\square$

\vspace*{-12pt}

\section{Заключение}

Итак, в~работе получены оценки для математического ожидания наблюдений в~предположении 
зашумления конечными смесями нормальных\linebreak (Тео\-ре\-ма~1) 
и~гам\-ма-рас\-пре\-де\-ле\-ний (Тео\-ре\-ма~3). 
%
Построены доверительные интервалы 
для неизвестного математического ожидания в~этих случаях с~использованием 
уточненной оценки~\eqref{Var} 
(Тео\-ре\-мы~2 и~4 соответственно). Отметим, что соответствующие соотношения 
зависят только от <<экстремальных>> значений параметров смесей, но не от числа 
компонент и~весов в~распределении зашумляющих наблюдений. 
%
Замечание~2 
предлагает подход, который  может быть использован для определения неизвестного 
параметра искусственно добавляемого к~исходным данным шума для улучшения качества 
работы метода скользящего разделения смесей.

\smallskip
Автор выражает признательность доктору фи\-зи\-ко-ма\-те\-ма\-ти\-че\-ских наук, 
профессору Виктору Юрьевичу Королеву за идею использования оценки 
дисперсии вида~\eqref{Var} и~другие полезные обсуждения в~рамках 
работы над данной статьей.

\vspace*{-12pt}

{\small\frenchspacing
 {%\baselineskip=10.8pt
 \addcontentsline{toc}{section}{References}
 \begin{thebibliography}{99}
\bibitem{Wright2003} \Au{Wright~D.\,E., Bray~I.} 
A~mixture model for rounded data~// J.~Roy. Stat. Soc.~D 
Sta., 2003. Vol.~52. P.~3--13.

\columnbreak

\bibitem{Bai2009} \Au{Bai~Z., Zheng~S., Zhang~B., Hu~G.} 
Statistical analysis for rounded data~// J.~Stat. Plan.  Infer., 2009. 
Vol.~139. Iss.~8. P.~2526--2542.

\bibitem{Zhang2010} \Au{Zhang~B., Liu~T., Bai~Z.\,D.} 
Analysis of rounded data from dependent sequences~// 
Ann. I.~Stat. Math., 2010. Vol.~62. Iss.~6. P.~1143--1173.

\bibitem{Zhao2012} \Au{Zhao~N., Bai~Z.} 
Analysis of rounded data in mixture normal model~// Stat. Pap., 2012. 
Vol.~53. P.~895--914.

\bibitem{Korolev2011-i} \Au{Королев~В.\,Ю.} 
Ве\-ро\-ят\-но\-ст\-но-ста\-ти\-сти\-че\-ские методы декомпозиции волатильности 
хаотических процессов.~--- М.: Изд-во Моск. ун-та, 2011. 512~с.

\bibitem{Ushakov2015} \Au{Ушаков В.\,Г., Ушаков Н.\,Г.} 
Об усреднении округленных данных~// Информатика и~её применения, 2015. Т.~9. 
Вып.~4. С.~106--109.

\bibitem{Ushakov2017a} \Au{Ушаков~В.\,Г., Ушаков~Н.\,Г.} 
Границы точ\-ности восстановления информации, 
теряемой при округлении результатов наблюдений~// 
Вестник Московского университета. Серия~15: Вычислительная математика и~кибернетика, 
2017. №\,2. С.~26--30.

\bibitem{Ushakov2017b} \Au{Ushakov~N.\,G., Ushakov~V.\,G.} 
Statistical analysis of rounded data: Recovering of information lost due to rounding~// 
J.~Korean Stat. Soc., 2017.  Vol.~46. No.\,3. P.~426--437.

\bibitem{Gorshenin2016} \Au{Gorshenin~A.\,K., Korolev~V.\,Yu.} 
A~noising method for the identification of the stochastic structure of 
information flows~// Comm. Com. Inf. Sc., 2017. 
Vol.~678. P.~279--289.

\bibitem{Gorshenin2013} 
\Au{Gorshenin~A., Korolev~V.} Modelling of statistical
fluctuations of information flows by mixtures of gamma distributions~// 
27th European Conference on Modelling and Simulation Proceedings.~--- 
Dudweiler, Germany: Digitaldruck Pirrot GmbHP, 2013. P.~569--572.
 \end{thebibliography}

 }
 }

\end{multicols}

\vspace*{-6pt}

\hfill{\small\textit{Поступила в~редакцию 03.08.18}}

\vspace*{6pt}

%\newpage

%\vspace*{-24pt}

\hrule

\vspace*{2pt}

\hrule

\vspace*{-2pt}


\def\tit{DATA NOISING BY FINITE NORMAL AND~GAMMA MIXTURES WITH~APPLICATION 
TO~THE~PROBLEM OF~ROUNDED OBSERVATIONS}


\def\titkol{Data noising by finite normal and~gamma mixtures with~application 
to~the~problem of~rounded observations}



\def\aut{A.\,K.~Gorshenin}

\def\autkol{A.\,K.~Gorshenin}

\titel{\tit}{\aut}{\autkol}{\titkol}

\vspace*{-11pt}


\noindent
Institute of Informatics Problems, Federal Research Center ``Computer Science and
Control'' of the Russian Academy of Sciences, 44-2~Vavilov Str., Moscow 119333,
Russian Federation


\def\leftfootline{\small{\textbf{\thepage}
\hfill INFORMATIKA I EE PRIMENENIYA~--- INFORMATICS AND
APPLICATIONS\ \ \ 2018\ \ \ volume~12\ \ \ issue\ 3}
}%
 \def\rightfootline{\small{INFORMATIKA I EE PRIMENENIYA~---
INFORMATICS AND APPLICATIONS\ \ \ 2018\ \ \ volume~12\ \ \ issue\ 3
\hfill \textbf{\thepage}}}

\vspace*{3pt}



\Abste{In many real problems, statistical analysis of data containing additional 
measurement errors, including rounding, is performed, which in some situations can 
lead to sufficiently significant distortions. In this paper, estimates for an 
unknown expectation of observations are obtained for one of the possible 
rounding models under the assumption that the original data are additionally 
noised with random variables having distributions of the type of finite 
mixtures of normal and gamma laws. Confidence intervals for an 
unknown expectation are constructed using the refined estimate for 
the variance of the integer part of the random variable. An algorithm 
for determining the value of the parameter of artificial noise, which 
can be added to the initial data to improve the quality of the 
method of moving separation of mixtures, is discussed.}


\KWE{noisy data; rounded data; finite normal mixtures; finite gamma mixtures; 
confidence intervals; moving separation of mixtures}



\DOI{10.14357/19922264180304}

%\vspace*{-14pt}

\Ack
\noindent
The research was supported by the Russian Science Foundation (project 18-71-00156).



%\vspace*{6pt}

  \begin{multicols}{2}

\renewcommand{\bibname}{\protect\rmfamily References}
%\renewcommand{\bibname}{\large\protect\rm References}

{\small\frenchspacing
 {%\baselineskip=10.8pt
 \addcontentsline{toc}{section}{References}
 \begin{thebibliography}{99}
\bibitem{1-gor-1}
\Aue{Wright,~D.\,E., and I.~Bray.} 2003.
A~mixture model for rounded data.  \textit{J.~Roy. Stat. Soc.~D Sta.} 52:3--13.

\bibitem{2-gor-1}
\Aue{Bai,~Z., S.~Zheng, B.~Zhang, and G.~Hu.} 2009. 
Statistical analysis for rounded data. \textit{J.~Stat. Plan. 
Infer.} 139(8):2526--2542.

\bibitem{3-gor-1}
\Aue{Zhang,~B., T.~Liu, and Z.\,D.~Bai.} 2010. 
Analysis of rounded data from dependent sequences. 
\textit{Ann. I.~Stat. Math.} 62(6):1143--1173.

\bibitem{4-gor-1}
\Aue{Zhao,~N., and Z.~Bai.} 2012. Analysis of rounded data in mixture normal model. 
\textit{Stat. Pap.} 53:895--914.

\bibitem{5-gor-1}
\Aue{Korolev, V.\,Yu.} 2011. 
\textit{Veroyatnostno-statisticheskie metody dekompozitsii volatil'nosti 
khaoticheskikh protsessov} [Probabilistic and statistical methods of 
decomposition of volatility of chaotic processes]. 
Moscow: Moscow University Publishing House. 512~p.

\bibitem{6-gor-1}
\Aue{Ushakov, V.\,G., and N.\,G.~Ushakov.} 
2015. Ob usrednenii okruglennykh dannykh [On averaging of rounded data].
\textit{Informatika i~ee Primeneniya~--- Inform. Appl.} 9(4):106--109.

\bibitem{7-gor-1}
\Aue{Ushakov,~V.\,G., and N.\,G.~Ushakov.} 2017. 
Boundaries of the precision of restoring information lost after rounding
 the results from observations. 
 \textit{Moscow University Computational Math. Cybernetics} 41(2):76--80.

\bibitem{8-gor-1}
\Aue{Ushakov,~N.\,G., and  V.\,G.~Ushakov.} 2017. 
Statistical analysis of rounded data: Recovering of information lost due to rounding. 
\textit{J.~Korean Stat. Soc.} 46(3):426--437.

\bibitem{9-gor-1}
\Aue{Gorshenin,~A.\,K., and V.\,Yu.~Korolev.} 2016. 
A~noising method for the identification of the stochastic structure of information 
flows. \textit{Comm. Com. Inf. Sc.} 678:279--289.

\bibitem{10-gor-1}
\Aue{Gorshenin,~A., and V.~Korolev.} 2013.  Modelling of statistical fluctuations of
information flows by mixtures of gamma distributions. 
\textit{27th European Conference on Modelling and Simulation Proceedings}. 
Dudweiler, Germany: Digitaldruck Pirrot GmbHP. 569--572.

\end{thebibliography}

 }
 }

\end{multicols}

\vspace*{-6pt}

\hfill{\small\textit{Received August 3, 2018}}

%\pagebreak

%\vspace*{-18pt}

\Contrl

\noindent
\textbf{Gorshenin Andrey K.} (b.\ 1986)~--- Candidate of Science (PhD) in physics and
mathematics, associate professor, leading scientist, Institute of Informatics Problems,
Federal Research Center ``Computer Science and Control'' of the Russian Academy of
Sciences, 44-2 Vavilov Str., Moscow 119333, Russian Federation; 
\mbox{agorshenin@frccsc.ru}
\label{end\stat}

\renewcommand{\bibname}{\protect\rm Литература}        %11
\renewcommand{\figurename}{\protect\bf Figure}
\renewcommand{\tablename}{\protect\bf Table}

\def\stat{dulin}


\def\tit{INFORMATION FUSION OF~DOCUMENTS}

\def\titkol{Information fusion of~documents}

\def\autkol{S.\,K.~Dulin, N.\,G.~Dulina, and~P.\,V.~Ermakov}

\def\aut{ S.\,K.~Dulin$^1$, N.\,G.~Dulina$^2$, and~P.\,V.~Ermakov$^3$}

\titel{\tit}{\aut}{\autkol}{\titkol}



\renewcommand{\thefootnote}{\arabic{footnote}}
\footnotetext[1]{Institute of Informatics Problems, Federal Research Center ``Computer Science and Control'' 
of the Russian Academy of Sciences, 44-2~Vavilov Str., Moscow 119333, Russian Federation, 
skdulin@mail.ru}
\footnotetext[2]{A.\,A.~Dorodnicyn Computing Center, Federal Research Center ``Computer Science and 
Control'' of the Russian Academy of Sciences, 40~Vavilov Str., Moscow 119333, Russian Federation, 
ngdulina@mail.ru}
\footnotetext[3]{ TeleRetail GmbH, 30~\mbox{Markenstra{\!\ptb{\ss}}e}, 
D$\ddot{\mbox{u}}$sseldorf 40227,  Germany; petcazay@gmail.com}


\index{Dulin S.\,K.}
\index{Dulina N.\,G.}
\index{Ermakov P.\,V.}
\index{Дулин С.\,К.}
\index{Дулина Н.\,Г.}
\index{Ермаков П.\,В.}


\def\leftfootline{\small{\textbf{\thepage}
\hfill INFORMATIKA I EE PRIMENENIYA~--- INFORMATICS AND
APPLICATIONS\ \ \ 2020\ \ \ volume~14\ \ \ issue\ 1}
}%
 \def\rightfootline{\small{INFORMATIKA I EE PRIMENENIYA~---
INFORMATICS AND APPLICATIONS\ \ \ 2020\ \ \ volume~14\ \ \ issue\ 1
\hfill \textbf{\thepage}}}

%\vspace*{-2pt}



       \Abste{The paper considers the problems associated with the 
creation of an expert base of documents that require prompt 
processing of incoming information and, as a consequence, 
restructuring of the knowledge base. The authors propose procedures 
that reduce the search of the optimal consistent state of 
interrelated documents. An approach to assessing the relationship of 
text documents and informational messages as poorly structured 
objects was developed. The practical implementation of this approach 
is described.}
      
      \KWE{information fusion; controlled data and knowledge consistency; 
knowledge base restructuring}
      
\DOI{10.14357/19922264200117} 
      
      %\vspace*{8pt}
      
      
      \vskip 12pt plus 9pt minus 6pt
      
       \thispagestyle{myheadings}
      
       \begin{multicols}{2}
      
       \label{st\stat}
     
     \section{Introduction}
      
     \noindent
     Combining information of various origins for integrative analysis and 
processing has been called ``Information Fusion''[1], implying that the synthesized 
data carrying information combine type properties of source data and possess 
more information than merely conjunction of information sources considered 
separately. The main difficulty of the synthesis problem is that information sources 
contain heterogeneous data represented by various formats and structures and 
employed in different types of platforms.
     
     The main factors of data heterogeneity and their sources are: various types 
of data, diversity in data origin, various models of database representation, various 
data presentation formats, differentiating in the organization of data storage 
systems, differences in the degree of reliability and accuracy of data, and
variety of  a~degree and form of data structure.
     
     The process of information fusion is a~multilevel process that includes five 
basic stages~\cite{2-d, 3-d, 4-d}:
     \begin{itemize}
\item zero stage~--- the stage of combining sensor signals, designed to obtain 
data indicating semantically clear and interpretable attributes of objects and 
participating in the applications of the research being performed;
\item the first stage is aimed at processing data of the zero stage in order to 
make a decision on the classes of the objects in question and the states of these 
objects;
\item the second stage of Information Fusion, designed to assess the situation, 
including the zero and the first stages. It is used to assess the situational 
interaction of objects considered as a whole;
\item the third stage~--- the stage of evaluation of the interaction ``Impact 
Assessment,'' designed to perform an antagonistic assessment, based on the 
prediction of the situation;
\item the fourth stage~--- the stage of feedbacks, evaluating the possibility of 
using feedbacks in the system in question; and
\item the fifth stage~--- the final stage, the level of man--machine interaction, 
performing correctional actions of the operator for the sake of the system 
control.
\end{itemize}

     Research in the field of Information Fusion mainly focuses on the synthesis 
of data represented by digital images and arrays of data and  
documents~\cite{1-d, 4-d, 5-d}.
     
     Current trends in the development of corporative informational systems 
show that, along with traditional informational resources, the results of intelligent 
activity of experts and analysts become very important for the successful operation 
of large and middle-sized companies. A~unified informational environment of the 
company incorporates these formalized results in an accumulated form such that 
all executives can jointly use this resource in the context of their assignments. The 
role played by the knowledge accumulated in such a~way in the enterprise-wide 
systems allows us to consider this knowledge as very valuable and a~notably 
important resource for a~company, which, together with the traditional resources, 
such as financial, material, human, etc., characterizes the reliability of the 
company. The totality of this knowledge, presented mainly in text form, is the 
intelligent assets of the company, and the competitiveness of the company and its 
adaptability to changing the business environment depends on how efficiently this 
resource is used.
{\looseness=-1

}
     
     An intelligent asset is a~specific resource that requires specialized 
knowledge management systems. These systems enable the search, accumulation, 
and processing of knowledge by experts in solving various analytical problems. 
This tendency in knowledge engineering appeared relatively recently, but interest 
in the development and usage of such systems is permanently growing. This is 
largely due to the significant results achieved by some companies that have 
successfully implemented knowledge management systems into their 
manufacturing activity.
     
     Complex technological solutions designed to support various stages of 
composition and usage of corporative data and knowledge have been embodied in 
the knowledge management systems. At each of these stages, individual problems 
are solved, with the most important of them being associated with tasks related to 
searching, processing documents, and extracting knowledge from them.
     
     Text processing tasks are solved in practically all fields of human activity, 
and the analysis of the current environment is an integral part of practically 
each 
corporative management system securing a timely and adequate reaction to 
changes in the business environment. Actually, operativeness is the basic 
characteristics of monitoring problems, which distinguishes them from the problems 
related to prediction, planning, etc., because the main goal of the monitoring is the 
timely reaction of corresponding management subsystems of the general 
technological scheme of company functioning to changes of internal or external 
factors.
     
     In the general case, the purpose of text processing tasks is to accumulate 
necessary information from different sources, process it analytically, and, on this 
basis, generate corresponding decisions. The character of text processing tasks is 
permanent in the sense that the environment and the parameters of the company 
operation are subject to permanent changes, which requires regular (or periodic) 
sampling of ever changing information.
     
     Text processing tasks can conventionally be divided into two classes: internal 
monitoring and external monitoring.
     
     Internal monitoring is associated mainly with the monitoring of internal 
operation parameters, e.\,g., regular monitoring of the operation of complex installa-
tions, cargo moving, etc. Possible examples are control systems for energy plants, 
freight management, etc. The typical feature of these problems is a relatively 
constant set of parameters used to estimate the state of the process (production, 
physical parameters of an installation, etc.).
     
     In contrast to the internal monitoring, the external monitoring is mainly 
related to the estimation of the state of the environment and external conditions of 
the company operation. As an example, an analysis of consumer demand carried 
out by a commodity-producing company falls into this category. The typical 
feature of these problems is that, first, the parameters to be estimated are poorly 
formalized and, second, the set of these parameters is variable. The latter factor 
requires the restructuring of the analyst knowledge according to the changed 
conditions. All this makes us consider the ``restructurability'' of the expert 
knowledge base as one of the characteristic features of the problems of external 
monitoring.
     
     In the problems of external monitoring, special requirements must be 
imposed on the sources of information used by experts for the localization of 
required knowledge and data. The development of informational technologies 
during recent years has strongly suggested that the Internet is gradually becoming 
the most important source of information in solving analytical problems in 
practically all areas of human activity. Coming up to printed and electronic mass 
media, Internet is often ranked first in operativeness, which makes the Internet the 
most valuable information source in monitoring problems. It is for this reason that, 
in this work, special attention is paid to the solution of monitoring problems 
associated with search and processing of text information in Internet.
     
     \section{Approach to~Provision of~Knowledge Consistency}
     
     \noindent
     In previous works (see~\cite{4-d, 7-d, 6-d}), the authors put forward a procedure 
providing the consistency of the knowledge base dynamically formed by an 
expert, which is based on the analysis of structural interrelations between separate 
components of the knowledge base with subsequent restructuring of it aimed at 
reducing existing inconsistency. In so doing, the basic criterion of structural 
consistency was a concept of polyconsonance of power~$n$~\cite{2-d}.
     
     Consider a knowledge base formed on the basis of search and analysis of 
Internet information. In solving the monitoring problems associated with the 
formation of such a knowledge base, the application of this procedure faces certain 
difficulties resulting from poor formalization and an obscure or ambiguous 
structure of the data (text or multimedia documents). Besides, for the monitoring 
problems considered here, a large number of informational messages directed to the 
expert for analytical processing and replenishment of the knowledge base are 
characteristic. As a result, the amount of resources (especially, time) required for 
the restructuring of a dynamically changing knowledge base is increased 
significantly, which is, perhaps, the main obstacle to the successful practical 
implementation of any procedure of the above type.
     
     One of the major disadvantages of the algorithm proposed in~\cite{4-d} is 
that it is oriented to problems of the search type; that is why, the authors made 
special efforts to reduce the search and thus increase the algorithm efficiency in its 
practical implementation. The results presented below are aimed at the solution of 
the latter problem.
     
     Consider a set of mutually related objects $O = \{o_i\}$ with a similarity 
function~$f$~\cite{3-d} satisfying the condition
     $$
     0\leq f\left( o_i, o_j\right)\leq 1\,.
     $$
     
     Numbers $\alpha$ and~$\beta$ will denote the lower and upper similarity 
thresholds, respectively, satisfying the condition
     $$
     0\leq \alpha\leq \beta\leq 1\,.
     $$
     
    Now, let us introduce the concepts of a negative, positive, and indifferent link 
between two arbitrary elements~$o_i$ and~$o_j$ of the set~$O$. The link is called 
``negative'' if its value does not exceed the lower similarity threshold: $0\leq 
f(o_i,o_j)\leq \alpha$; it is called ``positive'' if the value of the similarity function is 
not less than the upper similarity threshold: $\beta\leq f(o_i,o_j)\leq 1$; and, if 
$\alpha<f<\beta$, it is called ``indifferent'' (zero).
     
     Consider a partition of the given set into a number of nonempty subsets 
$K_1,\ldots , K_n$.
     
     A link between two arbitrary elements~$o_i$ and~$o_j$ of the entire 
set~$O$ is called ``bad'' if one of the following conditions is satisfied:
     \begin{enumerate}[(1)] 
     \item the elements~$o_i$ and~$o_j$ belong to the same subset~$K_x$, and 
the link between them is negative; or
\item the elements~$o_i$ and~$o_j$ belong to different subsets~$K_1$ 
and~$K_2$, and the link between them is positive.
\end{enumerate}

     Using this definition, let us to each object~$o_k$ from the set 
considered   assign the number~$v_k$ of its bad links for a~given partition into subsets. 
Now, let us construct a~vector~$V$ consisting of these values (this vector has 
a~dimension equal to the number of objects in the set) and call it the nodewise 
difference vector (NDV)~\cite{4-d}. The sum of the elements of this vector is 
denoted by $S_{\mathrm{NDV}}$.
     
     Clearly, different partitions of the original set correspond to different NDVs 
and different values of $S_{\mathrm{NDV}}$. According to the algorithm considered, 
the main problem is to find a partition of the given set~$O$ such that the sum 
$S_{\mathrm{NDV}}$ 
takes its minimal value; i.\,e., the total number of bad links tends to zero.
     
     The algorithm~\cite{4-d} developed by the authors consists in 
successive transformations of the set of informational objects on the basis of the 
condition
     $$
     S_{\mathrm{NDV}} > \fr{n(N-n)}{2}
     $$
     where $S_{\mathrm{NDV}}$ is the sum of nodewise differences for the given 
set of~$n$ elements belonging to a pair of consonant subsets of the total 
cardinality~$N$.  If this condition is fulfilled, then the restructuring of the 
considered set results in a decrease of the total sum~$S_{\mathrm{NDV}}$.
     \smallskip
     
     \noindent
     \textbf{Theorem~1.} \textit{Let~$K_1$ and~$K_2$ be two subsets of 
a~given set of mutually related objects~$O$}:
     \begin{align*}
     K_1 &= \left\{ o_i\right\}\,,\ i=1,\ldots, n_1\,;\\
     K_2&= \left\{ o_j\right\}\,, \ j=1,\ldots , n_2\,.
     \end{align*}
     
     \textit{A set containing~$m$~elements from these two subsets satisfies the 
condition of the algorithm if, and only if, the set consisting of all remaining 
elements of these two subsets satisfies the same condition.}
     
     \smallskip
     
     \noindent
     P\,r\,o\,o\,f\,.\ \  First, let us prove the necessity. Let the set of 
objects~$\{o_k\}$, $k = 1,\ldots , m$, satisfy the condition of the algorithm:
     $$
     \sum v_k> \fr{m(n_1+n_2-m)}{2}
     $$
     where $v_k$ are the NDV values for the element with the number~$k$. 
This formula can be transformed to the form:
     $$
     \sum v_k > \fr{\left(n_1+n_2-m\right)
     \left(\left(n_1+n_2\right)-\left(n_1+n_2-m\right)\right)}{2}
     $$
     which means that the set of $n_1+n_2-m$ vectors not belonging to the 
original set also satisfies the condition of the algorithm.
     
     The sufficiency of the condition is proved similarly. The theorem is proved.
     
     \smallskip
     
     \noindent
     \textbf{Corollary.} In order to find a set of objects from two given subsets 
that satisfies the condition of the algorithm, it is sufficient to check the fulfillment 
of this condition only for the subsets consisting of $(n_1+n_2)/2$ objects. In other 
words, only subsets with cardinalities not exceeding half of the sum of the 
cardinalities of the original subsets~$K_1$ and~$K_2$ should be checked.
     \smallskip
     
     \noindent
     P\,r\,o\,o\,f\,.\ \ Indeed, if some set consisting of more than $(n_1+n_2)/2$ 
elements satisfies the condition, then the complement to it also satisfies this 
condition, with the cardinality of the complement being not greater than 
$(n_1+n_2)/2$.

\begin{figure*}[b] %fig1
\vspace*{1pt}
    \begin{center}  
  \mbox{%
 \epsfxsize=160.967mm 
 \epsfbox{dul-1.eps}
 }
\end{center}
\vspace*{-10pt}
\Caption{Determination of vocabulary groups}
\end{figure*}

     
     \smallskip
     
     \noindent
     \textbf{Theorem~2.}\  \textit{Let~$K_1$ and~$K_2$ be two subsets of 
a~given set of mutually related objects~$O$}:
     \begin{align*}
     K_1 &= \left\{o_i\right\}\,,\ i=1,\ldots , n_1\,;\\
     K_2&= \left\{o_j\right\}\,,\ i=1,\ldots , n_2\,.
     \end{align*}
     \textit{Let a set $\{o_k\}$ of $m < (n_1 + n_2)/2$ elements belonging to 
these two subsets satisfy the condition of the algorithm. If a zero NDV element 
corresponds to some element~$o_x$ from this set, then the set of the vectors 
corresponding $O^*=\{o_1, \ldots, o_{x-1}, o_{x+1}, \ldots, o_m\}$ also satisfies the 
condition of the algorithm.}
     
     \smallskip
     
     \noindent
     P\,r\,o\,o\,f\,.\ \ According to the assumption of the theorem, the sum 
$S^*_{\mathrm{NDV}}$ for the set $O^*=\{o_1, \ldots\linebreak
\ldots, o_{x-1}, o_{x+1}, \ldots, o_m\}$ is 
equal to the sum $S_{\mathrm{NDV}}$ of the original set of the elements from the two 
subsets~$K_1$ and~$K_2$:
     $$
S^*_{\mathrm{NDV}} = S_{\mathrm{NDV}}\,.
$$
     
     Denote by~$N$ the total cardinality of the considered subsets: $N = 
n_1+n_2$. Then,
     $$
     (m-1)(N-(m-1)) = m(N-m)+(2m-N-1)\,.
     $$
     
     According to the assumption of the theorem, $m \leq N/2$; hence, $2m-N-1 
< 0$. To complete the proof, let us write the following inequality:
     \begin{multline*}
     S^*_{\mathrm{NDV}} = S_{\mathrm{NDV}} = \sum v_k >\fr{m(N - 
m)}{2} >{}\\
{}> \fr{(m-1)(N - (m-1))}{ 2}
   \end{multline*}
     which means that the set $\{o_1, \ldots, o_{x-1}, o_{x+1}, \ldots, o_m\}$ satisfies 
the condition of the algorithm.
     
     Obviously enough, it follows from this theorem that, in the practical 
implementation of the proposed algorithm, it is sufficient to search for a set of 
elements for the next iteration among those with nonzero NDV values.
{\looseness=1

}
     
\section{Thematic Role of~Similarity}

     \noindent
     The most significant factor affecting the operation of the algorithm 
considered is the similarity function on the basis of which interrelations between 
different elements of a given set are determined. As far as the support of 
monitoring problems is considered, with the texts (in particular, news) and the 
Internet being the elements and the main information source, respectively, the 
construction of the similarity function becomes a fairly difficult problem. Perhaps, 
one of the solutions to this problem could be the use of various methods of 
linguistic analysis to determine the degree of ``likeness'' of two different 
documents, although these methods are not free from some shortcomings 
associated with the hardship of their implementation, adjustment, etc. To 
determine the similarity function in practical applications, the authors have put 
forward another approach. One of the advantages of this new approach is the 
simplicity of implementation and the ``notional transparency.''
     
     The basis of this approach schematically shown in Fig.~1 is the 
determination of vocabulary groups~\cite{7-d}, which denote the sets of keywords 
defined by the expert. The expert assorts the keywords according to 
some criterion, e.\,g., ``thematic meaning:''
     $$
     G_k= \left\{w_i\right\},\enskip i = 1,\ldots ,n_k.
     $$

\begin{figure*}[b] %fig2
\vspace*{1pt}
    \begin{center}  
  \mbox{%
 \epsfxsize=94.043mm 
 \epsfbox{dul-2.eps}
 }
\end{center}
\vspace*{-10pt}
\Caption{A general scheme of operation of iiProcessor system}
\end{figure*}

     Consider an arbitrary element~$o_j$ from a given set~$O$. This object is a 
text document; so, it can be represented as an aggregate of lexical units, i.\,e., 
words. For~$o_j$, let us define its coefficient of correspondence with the dictionary 
group~$G_i$ as the ratio $S(G_i)_j$ of the number of keywords specified in this 
dictionary group and available in the text of the information object itself, to the 
total number of keywords from all dictionary groups, $S(G)_j$ found in this text. 
Then, one can define the factor of correspondence of the object~$o_j$ to the 
vocabulary group~$G_i$ as
     $$
     L^i_j = \fr{S(G_i)_j}{S(G)_j}.
        $$
     
     On the basis of these coefficients, let us define the degree of thematic coupling 
between two arbitrary informational objects as follows:
     \begin{itemize}
     \item[(A)] $f(o_k, o_l) = 1$ if $ S(G)_k = 0$ and  $S(G)_l = 0$;
     \item[(B)] $f(o_k, o_l) = 0$ if  $S(G)_k\not= S(G)_l$ 
     and $S(G)_k S(G)_l\linebreak = 0$; and
     \item[(C)] $f(o_k, o_l) = \max\left( \min\left(L^i_k, L^i_l\right) \right)$, $i = 1, 
\ldots, n$,  for $S(G)_k  S(G)_l\not= 0$
     where $n$ is the number of the vocabulary groups.
     \end{itemize}

     
     Note that the similarity function defined above takes the values on the 
interval from~0 to~1 but lacks associativity, because $0 \leq f(o_i, o_j) \leq 1$. In 
the works devoted to the theoretical grounds of the considered algorithm of 
structural transformations of a set of objects, the associativity of the similarity 
function has not been used; therefore, the fact that the function introduced above is 
not associative does not require any changes in the proposed algorithm. Moreover, 
the lack of associativity here has an additional meaning, which makes it possible to 
treat the function introduced above as a~\textit{thematic} similarity function.
     
     Indeed, if, in the considered text, there are keywords from different 
vocabulary groups, then all the coefficients~$L^i_j$ for this element will be less 
than one. Hence, the value of the similarity function~$f$ will also be less than one, 
and the more the number of the vocabulary groups, the less this value. In practice, 
this could mean that the considered document is of a review nature and, most 
probably, has no distinct ``thematic meaning.''

\begin{figure*} %fig3
\vspace*{1pt}
    \begin{center}  
  \mbox{%
 \epsfxsize=156.872mm 
 \epsfbox{dul-3.eps}
 }
\end{center}
\vspace*{-10pt}
\Caption{Example of use of vocabulary group technique to establish
links between different documents}
\end{figure*}
     
\section{Consistency Controlling Module iiProcessor}

     \noindent
     The authors' technique for providing structural consistency of the knowledge 
base in solving monitoring problems has been implemented in a specialized system 
called an iiProcessor. This system is designed to compose expert knowledge bases 
for social, political, and international sciences. The knowledge bases are 
constructed from the information supplied by various mass media through their 
Internet servers. The main purpose of the system is to accumulate informational 
messages (news) related to the themes of user's interest from various Internet 
sources, to integrate the information into a unified knowledge base, to create links 
between different elements of the knowledge base, and to make subsequent 
restructuring of the knowledge base on the basis of these links, with the result of 
this restructuring being the representation of the body of the information 
accumulated as a logical system of classes. The latter system can be treated as an 
informational model of the problem examined by the expert (for example, the 
social and political situation in a particular region of the world). A~general scheme 
of operation of the system is shown in Fig.~2.



     As a source of information, this system uses the CNN Internet site ({\sf 
http://cnn.com}). Several times a day, this site publishes information covering many 
aspects of social and political life in many countries. In most cases, the 
informational messages are weakly-structured text documents. In order to establish 
links between different documents, the vocabulary group technique described 
above is used (Fig.~3). If various informational messages contain common 
keywords belonging to different vocabulary groups, this technique estimates the 
``likeness'' of the messages. The similarity function classifies these links as 
positive or negative, which makes it possible to construct a~connectivity matrix on 
the set of the informational messages received by the user (see Fig.~3).
     
    


     The mode of ``Keywords'' allows one to get~10 of the most significant key 
words for a~given document with an indication of their weighting factors (Fig.~4).
     
     
     The mode of interrelations (``Correlations'') will allow to get several 
documents that have the greatest interrelations with selected document. This mode 
works only if the loaded document belongs to the current project of the iiProcessor 
system, in which the relationship was evaluated (Fig.~5).
    
     The choice of the CNN server as a source of information is explained by the 
fact that this server is one of the most informationally abundant servers providing 
real-time information. Of course, the choice of the sources of information is 
strongly determined by the character of the problem considered. In this sense, the 
CNN server is not universal. In view of the above considerations, the Restructor 
system is implemented as a~complex of two program modules. The rsn.exe module 
is the basic one. An auxiliary iip.class module executes a real-time search for new 
information in a specified information source in the Internet. With such an 
architecture, this\linebreak\vspace*{-12pt}

{ \begin{center}  %fig4
 \vspace*{-7pt}
     \mbox{%
 \epsfxsize=79mm 
 \epsfbox{dul-4.eps}
 }


\vspace*{4pt}


\noindent
{{\figurename~4}\ \ \small{``Keywords'' mode}}
\end{center}
}

\vspace*{2pt}


{ \begin{center}  %fig5
 \vspace*{-1pt}
    \mbox{%
 \epsfxsize=79mm 
 \epsfbox{dul-5.eps}
 }


\vspace*{4pt}


\noindent
{{\figurename~5}\ \ \small{``Correlation'' mode}}
\end{center}
}

%\vspace*{3pt}



\noindent
 system can be adopted to operation with any informational 
servers in the Internet (and beyond) by replacing only the auxiliary module, 
without changing its kernel where the major mathematical results of the authors' 
approach are implemented.
     
\section{Concluding Remarks}

\noindent
The implementation of the results of Theorems~1 and~2 in the inference engine 
made it possible to considerably reduce the time expenses of the built-in algorithm 
for restructuring the database. The use of the connectivity matrix as the major 
visualization means for the informational objects improved the clearness of the 
representation of the information model of the problem considered by an expert. 
The system has been tested in analyzing the events related to NASA's 
aerospace research.
     
    % \Ack
    % \noindent
    % This work was supported by the Russian Foundation for Basic Research, 
%project No.\,20-07-00329~А.
     
     \renewcommand{\bibname}{\protect\rmfamily References}
     
     
     \vspace*{-9pt}
     
     {\small\frenchspacing
     {\baselineskip=10.45pt
     \begin{thebibliography}{99}
     
     \bibitem{1-d} %1
\Aue{Dasarathy, B.} 2001. Information fusion~--- what, where, why, when, and how? 
\textit{Inform. Fusion} 2(2):75--76.
     
     \bibitem{4-d} %2
\Aue{Dulin, S.\,K.} 1995. The approach to structural consistency of situations' models in 
an active knowledge base. \textit{Workshop of 10th IEEE Symposium 
(International) on Intelligent Control Proceedings}. Monterey, CA: AdRem, Inc. 
253--258.

\bibitem{3-d} %3
\Aue{Duckham, M., and M.~Worboys.} 2007. Automated geographic information 
fusion and ontology alignment. \textit{Spatial data on the Web}. Eds. A.~Belussi, 
B.~Catania, E.~Clementini, and E.~Ferrari.
Berlin: Springer. Ch.~6:109--132. 

\bibitem{2-d} %4
\Aue{Pravia, M.} 2008. Generation of a~fundamental data set for hard/soft information 
fusion. \textit{11th Conference (International) on Information Fusion Proceedings}. 
Cologne: International Society of Information Fusion. 134--145.





\bibitem{5-d} %5
\Aue{Landauer, T.\,K., K.~Kireyev, and C.~Panaccione.} 2011. Word maturity: A~new 
metric for word knowledge. \textit{Sci. Stud. Read.} 15(1):92--108. 

\bibitem{7-d} %6
\Aue{Dulina, N., and O.~Kozhunova.} 2010. Information monitoring system: 
A~problem 
of linguistic resources consistency and verification. \textit{Problems of Cybernetics and 
Informatics: 3rd Conference (International) Proceedings}. Baku.  
56--58.
\bibitem{6-d} %7
\Aue{Dulin, S.\,K., and  N.\,G.~Dulina.} 2018. Ispol'zovanie disseminatsionnykh 
algoritmov dlya formirovaniya nestrukturirovannoy tekstovoy informatsii v~baze 
geodannykh [Using dissemination algorithms for the formation of unstructured textual 
information in the geodatabase]. \textit{Sistemy i~Sredstva Informatiki~--- Systems and 
Means of Informatics} 28(2):42--59.

\end{thebibliography}}}

\end{multicols}

\vspace*{-6pt}

\hfill{\small\textit{Received February 26, 2019}}

\vspace*{-16pt}

\Contr

%\vspace*{-3pt}

\noindent
\textbf{Dulin Sergey K.} (b.\ 1950)~--- Doctor of Science in technology, 
professor, leading scientist, Institute of Informatics Problems, Federal Research 
Center ``Computer Science and Control'' of the Russian Academy of Sciences,  
44-2~Vavilov Str., Moscow 119333, Russian Federation; principal scientist, 
Research \& Design Institute for Information Technology, Signalling and 
Telecommunications on Railway Transport (JSC NIIAS), 27-1~Nizhegorodskaya 
Str., Moscow 109029, Russian Federation; \mbox{skdulin@mail.ru} 

\vspace*{3pt}

\noindent
\textbf{Dulina Natalia G.} (b.\ 1947)~--- Candidate of Science (PhD) in 
technology, leading programmer, A.\,A.~Dorodnicyn Computing Center, Federal 
Research Center ``Computer Science and Control'' of the Russian Academy of 
Sciences, 40~Vavilov Str., Moscow 119333, Russian Federation; 
\mbox{ngdulina@mail.ru}
\vspace*{3pt}

\noindent
\textbf{Ermakov Petr V.} (b.\ 1985)~--- Senior Software Developer, TeleRetail 
GmbH, 30~\mbox{Markenstra{\!\ptb{\ss}}e}, D$\ddot{\mbox{u}}$sseldorf 
40227,  Germany; \mbox{petcazay@gmail.com}

 

%\newpage

\vspace*{8pt}

\hrule

\vspace*{2pt}

\hrule

%\vspace*{-7pt}

%\newpage

%\vspace*{-28pt}

\def\tit{ИНФОРМАЦИОННЫЙ СИНТЕЗ ДОКУМЕНТОВ}

\def\titkol{Информационный синтез документов}

\def\aut{С.\,К.~Дулин$^1$, Н.\,Г.~Дулина$^2$, П.\,В.~Ермаков$^3$}

\def\autkol{С.\,К.~Дулин, Н.\,Г.~Дулина, П.\,В.~Ермаков}

%{\renewcommand{\thefootnote}{\fnsymbol{footnote}} \footnotetext[1]
%{Работа was supported by the Russian Foundation for Basic Research, project No.\,20-07-00329~А.}}



\titel{\tit}{\aut}{\autkol}{\titkol}

\vspace*{-11pt}

\noindent
$^1$Институт проблем информатики Федерального исследовательского центра <<Информатика 
и~управление>>\linebreak
$\hphantom{^1}$Российской академии наук, \mbox{skdulin@mail.ru}

\noindent
$^2$Вычислительный центр им.\ А.\,А.~Дородницына Федерального исследовательского центра 
<<Информатика\linebreak
$\hphantom{^1}$и~управление>> Российской академии наук, \mbox{ngdulina@mail.ru}

\noindent
$^3$TeleRetail GmbH, D$\ddot{\mbox{u}}$sseldorf, Germany

\vspace*{1pt}

\def\leftfootline{\small{\textbf{\thepage}
\hfill ИНФОРМАТИКА И ЕЁ ПРИМЕНЕНИЯ\ \ \ том\ 14\ \ \ выпуск\ 1\ \ \ 2020}
}%
 \def\rightfootline{\small{ИНФОРМАТИКА И ЕЁ ПРИМЕНЕНИЯ\ \ \ том\ 14\ \ \ 
выпуск\ 1\ \ \ 2020
\hfill \textbf{\thepage}}}

\vspace*{-1pt}



\Abst{Рассматриваются проблемы, связанные с созданием экспертной 
базы документов, требующей оперативной обработки поступающей 
информации и, как следствие, реструктуризации базы знаний. 
Предложены процедуры, уменьшающие время поиска оптимального 
согласованного состояния взаимосвязанных документов. Был 
разработан подход к~оценке взаимосвязи текстовых документов 
и~информационных сообщений как плохо структурированных 
объектов. Описана практическая реализация этого подхода.}

\KW{информационный синтез; контролируемая согласованность 
данных и~знаний; реструктуризация базы знаний}


\DOI{10.14357/19922264200117} 

%\vspace*{-3pt}


 \begin{multicols}{2}

\renewcommand{\bibname}{\protect\rmfamily Литература}
%\renewcommand{\bibname}{\large\protect\rm References}

{\small\frenchspacing
{\baselineskip=10.5pt
\begin{thebibliography}{99}
%\vspace*{-3pt} 

\bibitem{1-d-1} %1
\Au{Dasarathy B.} Information fusion~--- what, where, why, when, and how?~// 
Inform. Fusion, 2001. Vol.~2. Iss.~2. P.~75--76.

\bibitem{4-d-1} %2
\Au{Dulin S.\,K.} The approach to structural consistency of situations' models in an 
active knowledge base~// Workshop of 10th IEEE Symposium 
(International) on Intelligent Control Proceedings.~--- Monterey, CA, USA: AdRem, 
Inc., 1995. P.~253--258.

\bibitem{3-d-1} %3
\Au{Duckham M., Worboys~M.} Automated geographic information fusion and 
ontology alignment~// Spatial data on the Web~/ Eds. A.~Belussi, B.~Catania, 
E.~Clementini, E.~Ferrari.~--- Berlin: Springer, 2007. Ch.~6. P.~109--132. 

\bibitem{2-d-1} %4
\Au{Pravia M.} Generation of a fundamental data set for hard/soft information 
fusion~// 11th Conference (International) on Information Fusion.~--- Cologne: 
International Society of Information Fusion, 2008. P.~134--145.




\bibitem{5-d-1} %5
\Au{Landauer T.\,K., Kireyev~K., Panaccione~C.} Word maturity: A~new metric 
for word knowledge~// Sci. Stud. Read., 2011. Vol.~15. Iss.~1. 
P.~92--108. 

\bibitem{7-d-1} %6
\Au{Dulina N., Kozhunova~O.} Information monitoring system: A~problem of 
linguistic resources consistency and verification~// Problems of Cybernetics and 
Informatics: 3rd Conference (International) Proceedings.~--- Baku, 2010.  
P.~56--58.
\bibitem{6-d-1} %7
\Au{Дулин С.\,К., Дулина~Н.\,Г.} Использование диссеминационных 
алгоритмов для формирования неструктурированной текстовой информации 
в базе геоданных~// Системы и средства информатики, 2018. Т.~28. №\,2. 
С.~42--59. 

\end{thebibliography}
} }

\end{multicols}

 \label{end\stat}

 \vspace*{-9pt}

\hfill{\small\textit{Поступила в~редакцию 26.02.2019}}


%\renewcommand{\bibname}{\protect\rm Литература}
\renewcommand{\figurename}{\protect\bf Рис.}
\renewcommand{\tablename}{\protect\bf Таблица}  %12
\def\stat{zatsman}

\def\tit{ТРАНСФОРМАЦИИ ОБЪЕКТОВ ПЕРВОГО И~ВТОРОГО ПОРЯДКА 
В~ЛЕКСИКОГРАФИЧЕСКОЙ ИНФОРМАЦИОННОЙ СИСТЕМЕ$^*$}

\def\titkol{Трансформации объектов первого и~второго порядка 
в~лексикографической информационной системе}

\def\aut{И.\,М.~Зацман$^1$}

\def\autkol{И.\,М.~Зацман}

\titel{\tit}{\aut}{\autkol}{\titkol}

\index{Зацман И.\,М.}
\index{Zatsman I.\,M.}


{\renewcommand{\thefootnote}{\fnsymbol{footnote}} \footnotetext[1]
{Исследование выполнено в~ФИЦ ИУ РАН за счет гранта Российского научного фонда №\,24-18-00155, {\sf 
https://rscf.ru/project/24-18-00155}. Работа выполнялась с~использованием инфраструктуры Центра 
коллективного пользования <<Высокопроизводительные вычисления и~большие данные>> (ЦКП 
<<Информатика>>) ФИЦ ИУ РАН (г.\ Москва).}}


\renewcommand{\thefootnote}{\arabic{footnote}}
\footnotetext[1]{ Федеральный исследовательский центр <<Информатика и~управление>> Российской академии наук, 
\mbox{izatsman@yandex.ru}}

\vspace*{-12pt}


  
  \Abst{Рассматриваются теоретические основания проектирования информационных 
технологий (ИТ) интеграции двуязычных словарей и~параллельных корпусов. Дано описание 
первых результатов создания третьего уровня классификации трансформаций объектов 
предметной области информатики, которую предполагается использовать при создании 
концепции лексикографической информационной системы, обеспечивающей интеграцию. 
Все сущности информатики в~статье разделены на два глобальных класса: объекты и~их 
трансформации. Для каждого такого класса конструируется своя классификация. Ранее были 
описаны два верхних уровня классификации трансформаций объектов предметной области. 
В~данной статье рассматривается третий уровень этой классификации. Основанием для 
построения самого верхнего ее уровня служило деление предметной области информатики 
на среды (ментальная, сенсорно воспринимаемая, цифровая и~ряд других сред), каждая из 
которых по определению включает объекты одной природы. Основанием для построения 
второго уровня классификации трансформаций объектов служила типология знаковых  
сис\-тем А.~Соломоника. Цель статьи состоит в~систематизации трансформаций первого 
и~второго порядка объектов предметной области на третьем уровне этой классификации. 
Основанием для систематизации служит средовая версия иерархии Акоффа.}
  
  \KW{объекты предметной области; трансформации объектов; классификация; данные; 
информация; знание; лексикографическая информационная сис\-тема}

\DOI{10.14357/19922264240211}{VZTGVV}
  
\vspace*{3pt}


\vskip 10pt plus 9pt minus 6pt

\thispagestyle{headings}

\begin{multicols}{2}

\label{st\stat}
  
\section{Введение}

\vspace*{-9pt}

  Возникновение параллельных корпусов, в~которых предложениям 
оригинального текста со\-по\-став\-ле\-ны предложения его перевода, обеспечило 
возможность контрастивного лингвистического\linebreak \mbox{анализа} на принципиально 
новом уровне полноты и~точности, недостижимом в~докорпусную эпоху. 
Пионерскими в~этой области стали работы \mbox{1990-х~гг}. Стига Йоханссона  
с~анг\-ло-нор\-веж\-ским корпусом~[1]. В России параллельные корпусы стали 
формироваться в~начале XXI~века в~рамках Национального корпуса русского 
языка~[2].
  
  Создатели двуязычных словарей используют параллельные корпусы для 
сбора материала и~эмпирической проверки своих гипотез, касающихся 
межъязы\-ко\-вой эквивалентности. Ценность параллельных корпусов 
определяется тем, что в~лингвистике этап сбора исходного материала считается 
наиболее трудоемким и~наименее творческим, а~параллельные корпусы 
позволяют значительно сэкономить время и~силы для творческого этапа 
создания словарей~[3].
 % 
  При этом двуязычные словари, создаваемые на основе исходного материала, 
извлеченного из параллельных корпусов, сейчас формируются без связей с~их 
текстами. Другими словами, онлайновые связи созданных словарей 
с~параллельными корпусами, которые служили источниками исходного 
материала, отсутствуют. 

Параллельные корпусы постоянно пополняются 
новыми текстами, в~предложениях которых можно обнаружить новые значения 
слов и~устойчивых словосочетаний. Однако при этом отсутствуют методы 
и~средства оперативного обновления словарей по корпусным данным. 
В~настоящее время проблема установления связей между двуязычными 
словарями и~параллельными корпусами (далее~--- проблема интеграции) 
находится на стадии поиска концептуальных подходов к~их интеграции на 
уровне значений.
  
  Подход к~решению проблемы интеграции, предлагаемый в~статье, учитывает 
  и~появление новых значений слов и~устойчивых словосочетаний, и~динамику 
смысловых значений, которая обусловлена развитием и~пополнением знания 
лингвистов, фиксирующих эти значения в~результате семантического анализа 
пополняемых корпусных данных. Проведенные эксперименты показали, что 
обнаружение нового лингвистического знания обусловливает и~формирование 
дефиниций новых значений, и~пересмотр уже существующих дефиниций~[4, 5].
  
  Например, в~проведенных экспериментах с~использованием ЦКП 
<<Информатика>> ФИЦ ИУ РАН фиксировалась эволюция значений немецких 
модальных глаголов, исходное состояние значений которых было описано 
в~не\-мец\-ко-рус\-ском словаре. В~экспериментальном массиве текстов как 
потенциальных источниках нового знания 16\,268 предложений содержали 
немецкие модальные глаголы и~в~2041 из них встречался глагол sollen. 
В~начале эксперимента в~словаре были описаны~12~значений этого модального 
глагола. По окончании эксперимента лингвисты обнаружили два новых его 
значения, согласовали их дефиниции и~описали эволюцию дефиниций~[6, 7].
  
  Таким образом, для решения проблемы интеграции требуется фиксировать 
новое знание, обнаруженное лингвистами в~текстовых данных параллельных 
корпусов, отслеживать эволюцию знания, представленного в~виде дефиниций 
значений слов и~устойчивых словосочетаний, и,~соответственно, 
актуализировать электронные двуязычные словари. Предлагаемый 
концептуальный подход к~интеграции, который планируется реализовать 
в~процессе проектирования лексикографической информационной сис\-те\-мы, 
фиксирующей эволюцию лингвистического знания, основан на решении 
следующих задач:\\[-14pt]
  \begin{itemize}
  \item категоризация трех базовых понятий информатики, включенных 
  в~иерархию Акоффа~[8] (данные, информация, знание), на объекты 
проектируемой сис\-те\-мы, которая необходима, чтобы фиксировать 
<<кванты>> нового знания и~отслеживать его эволюцию в~этой сис\-теме;\\[-15pt]
  \item  систематизация трансформаций объектов этой сис\-темы.\\[-14pt]
  \end{itemize}
  
  Цель статьи и~состоит в~решении двух задач: категоризации трех базовых 
понятий информатики на объекты лексикографической информационной  
сис\-те\-мы и~сис\-те\-ма\-ти\-за\-ции трансформаций первого и~второго порядка 
ее объектов.
  
  Трансформациями первого порядка, о которых сказано в~формулировке цели 
статьи, называются взаимные преобразования между двумя объектами  
сис\-те\-мы одной природы. Например, перевод в~сис\-те\-ме текста с~русского 
языка на английский относится к~ним. Трансформациями второго порядка 
и~выше называются взаимные преобразования между двумя и~более объектами 
разной природы. Например, кодирование символов текс\-та компьютерными 
кодами и~их декодирование относятся по определению к~трансформациям 
второго порядка.

%\vspace*{-9pt}
  
\section{Процессы трансформаций в~информатике}

%\vspace*{-3pt}

Процессы трансформаций, рассматриваемые в~статье, относятся к~теоретическому ядру информатики, а~не 
только к~проектированию лексикографической информационной сис\-те\-мы. Например, из трех основных 
подходов к~описанию предметной об\-ласти информатики\footnote{В статье предметная область информатики 
трактуется согласно концепции полиадического компьютинга Пола Розенблума~\cite{9-zac}.} (объектный, 
трансформационный и~синтетический) сис\-те\-ма\-ти\-за\-ция трансформаций ближе всего ко второму 
подходу. Примерами первого подхода, в~рамках которого основное внимание уделяется объектам предметной 
области информатики и~в~меньшей степени отношениям\linebreak между ними, могут служить  
работы~\cite{8-zac, 10-zac, 11-zac}; \mbox{примерами} второго подхода, в~рамках которого основное внимание 
уделяется трансформациям и~в~меньшей степени трансформируемым объектам,~---  
работы~\cite{12-zac, 13-zac}; примерами третьего, синтетического подхода, в~котором уделяется внимание 
и~объектам предметной об\-ласти информатики, и~отношениям между ними, могут служить работы~\cite{14-zac, 
15-zac, 16-zac, 17-zac, 18-zac}.

  Таким образом, для описания трансформаций объектов лексикографической 
информационной\linebreak системы предпочтительнее всего трансформационный 
подход, который упоминается и~в определениях информатики. Например, 
в~2009~г.\ П.~Деннинг и~П.~Розенблум сформулировали суть \mbox{информатики} как 
компьютинга следующим образом: <<$\ldots$информатика~--- это не просто 
алгоритмы и~структуры данных; это преобразования [трансформации] 
представлений>>~\cite{12-zac}. Чуть позже, в~контексте краткого описания 
парадигмы информатики как компьютинга, П.~Деннинг и~П.~Фриман изменили 
эту формулировку на такую: <<Центральный объект внимания в~информатике 
можно определить как информационные процессы~--- \textit{естественные или 
искусственные процессы, преобразующие информацию} (курсив мой~--- 
И.\,З.)>>~\cite{13-zac}. Согласно парадигме, предлагаемой авторами этой 
статьи, на начальном этапе проектирования автоматизированных систем 
базовыми элементами моделей их функционирования служат 
\textit{информационные про\-цессы}.
  
  Однако если 15~лет назад в~формулировке из работы~\cite{13-zac} шла речь 
о~процессах, преобразующих информацию, то в~последние~10~лет в~спектр 
процессов трансформаций все чаще стали включать процессы, преобразующие 
не только информацию, но также и~другие объекты автоматизированных 
систем, в~первую очередь данные и~знания~[19--21]. Например, Виктория 
Стодден, позиционируя науку о~данных как одну из дисциплин информатики, 
говорит, что центральный объект исследований в~науке о~данных~--- это 
<<изучение обобщаемого извлечения знания из данных>>~\cite{21-zac}. 
Увеличение и~чис\-ла объектов, и~спект\-ра процессов их трансформаций 
в~автоматизированных сис\-те\-мах обуслов\-ли\-ва\-ет не\-об\-хо\-ди\-мость 
систематизации и~объектов, и~процессов их трансформаций на начальном этапе 
проектирования сис\-тем.
  
  Для создания концепции лексикографической информационной сис\-те\-мы 
и~проектирования ИТ, обеспечивающих интеграцию 
двуязычных словарей и~параллельных корпусов, сначала выполним 
категоризацию на объекты этой сис\-те\-мы трех базовых понятий информатики 
(данные, информация, знание) в~контексте построения классификаций 
сущностей ее предметной об\-ласти.
  
  Необходимость использования классификаций информатики в~процессе 
создания концепции проиллюстрируем, используя иерархию  
Акоффа~\cite{8-zac}. Он использовал принцип их вертикального размещения 
в~иерархии снизу вверх: данные, информация и~знание. Еще в~ней есть термин 
<<мудрость>>, который в~статье не рассматривается. Такое размещение Акофф 
прокомментировал так: <<Каждое из пе\-ре\-чис\-лен\-ных понятий [кроме данных] 
содержит в~себе нижестоящие$\ldots$>>~\cite{8-zac}.
  
  Этому принципу размещения и~комментарию Акоффа свойственны 
недостатки, проанализированные, в~частности, в~работе~\cite{10-zac}. Главный 
вывод, к~которому пришла Роули после изучения иерархии Акоффа, 
заключается в~следующем: <<$\ldots$информация определяется в~терминах 
данных, знание~--- в~терминах информации$\ldots$ но существует меньше 
консенсуса в~описании трансформаций, которые преобразуют сущности, 
расположенные ниже в~иерархии, в~те, которые находятся над ними, что 
приводит к~их терминологической неопределенности>>~\cite{10-zac}. Причина 
этой неопределенности, скорее всего, в~том, что базовые понятия информатики 
включены в~иерархию Акоффа изолированно от общего контекста 
классификаций сущностей ее предметной об\-ласти.

%\vspace*{-9pt}
  
\section{Классификации сущностей информатики}


%\vspace*{-2pt}

  Все сущности предметной области информатики в~работах~[22, 23] 
разделены на два глобальных класса: ее объекты и~их трансформации. Для 
каждого такого класса была предложена своя классификация. 
В~работе~\cite{22-zac} дано описание классификации объектов предметной 
области информатики, первый уровень которой содержит базовые понятия ее 
предметной области (данные, информация, знания и~др.).  
В~работе~\cite{23-zac} дано описание двух верхних уровней классификации 
трансформаций объектов предметной об\-ласти (см.\ рисунок 
в~работе~\cite{23-zac}). Основанием для построения самого верхнего ее уровня послужило деление 
предметной области информатики на среды\footnote{В~работе~\cite{24-zac} дано описание пяти сред 
предметной области информатики (ментальная; сенсорно воспринимаемая, или информационная; 
цифровая; нейро- и~ДНК-среда), каждая из которых по определению включает объекты одной и~той же 
природы.} и~степень разнообразия природы объектов, вовлеченных в~трансформации:
\begin{itemize}
\item  первый класс верхнего уровня классификации включает 
трансформации объектов в~пределах среды только одной природы 
(трансформации первого порядка);
\item  второй класс включает трансформации объектов, относящихся 
к~двум средам разной природы (трансформации второго порядка);
\item третий и~последующие классы включают трансформации объектов, 
относящихся к~трем и~более средам разной природы (трансформации 
третьего и~более высоких порядков).
\end{itemize}

  В работе~\cite{23-zac} были приведены примеры для трех первых классов 
трансформаций, включая пример трансформаций объектов, относящихся 
к~двум средам разной природы (компьютерное кодирование символов текстов 
с~по\-мощью таб\-лиц Unicode).
  
Основанием для построения второго уровня классификации трансформаций объектов послужила типология 
знаковых сис\-тем А.~Соломоника~\cite[c.~131]{25-zac}: естественные знаковые сис\-те\-мы, образные,  
ес\-тест\-вен\-но-язы\-ко\-в$\acute{\mbox{ы}}$е,  
вер\-баль\-но-не\-сло\-вес\-ные сис\-те\-мы записи\footnote{Под системой записи понимается знаковая 
система, сочетающая вербальные знаки с~несловесными (языки нотной записи, карт, таблиц и~др.).} 
и~формализованные знаковые сис\-те\-мы, включая математические. Введем понятие обобщенного текста~--- 
это текст, который может быть создан в~любой из перечисленных знаковых систем. Тогда обобщенные тексты 
могут быть естественными, образными, ес\-тест\-вен\-но-язы\-ко\-в$\acute{\mbox{ы}}$\-ми,  
вер\-баль\-но-не\-сло\-вес\-ны\-ми и~формализованными. Второй уровень классификации трансформаций 
охватывает не все виды объектов предметной  
об\-ласти информатики, а~только перечисленные~5~видов текс\-тов и~их представления, вовлеченные 
в~процессы трансформаций в~одной или более средах вместе с~данными, знанием и~его концептами.

\begin{figure*}[b] %fig1
\vspace*{6pt}
      \begin{center}
     \mbox{%
\epsfxsize=121.191mm 
\epsfbox{zac-1.eps}
}
\end{center}
\vspace*{-6pt}
\Caption{Средовая версия иерархии Акоффа}
\end{figure*}

\section{Классификация трансформаций: построение~третьего 
уровня}

  Основанием для систематизации трансформаций первого и~второго порядка 
на третьем уровне этой классификации служит иерархия Акоффа~\cite{8-zac}, 
на основе которой и~была создана ее средов$\acute{\mbox{а}}$я версия~[26, 
27]. Для создания средов$\acute{\mbox{о}}$й версии была выполнена 
категоризация трех базовых понятий информатики (данные, информация, 
знания) на объекты лексикографической информационной сис\-те\-мы 
в~процессе создания ее концепции\linebreak (рис.~1).
  


  В отличие от классической иерархии Акоффа, в~ее 
средов$\acute{\mbox{о}}$й версии различаются три вида данных: сенсорно 
воспринимаемые, цифровые и~те данные, которые генерируются 
искусственными нейронными сетями (ИНС) в~системах искусственного интеллекта 
(далее~--- ИИ-дан\-ные). Последний вид данных необходим, например, для 
различения входа и~выхода процесса применения обученной 
ИНС в~цифровой модели генерации знания, описанию которой 
посвящена работа~\cite{27-zac}.
  
  Также предлагается различать два вида информации: сенсорно 
воспринимаемая и~цифровая. Кроме знания в~средов$\acute{\mbox{у}}$ю 
версию добавлены концепты и~ментальные образы сенсорно воспринимаемых 
данных. Последние служат промежуточной сущностью между сенсорно 
воспринимаемыми данными и~генерируемым знанием при описании процессов 
извлечения знания из текстовых данных лексикографической информационной 
системы. Описание объектов средов$\acute{\mbox{о}}$й версии иерархии 
Акоффа (см.\ рис.~1) и~отношений между ними дано в~работах~\cite{26-zac, 28-zac}.
  
  В средов$\acute{\mbox{о}}$й версии число объектов равно восьми. Если 
учитывать направления трансформаций, то между восемью объектами на 
рис.~1 она включает~16 их видов (трансформации на границе между сенсорно 
воспринимаемыми данными и~информацией, обозначенные символом~<<?>>, 
в~статье не рас\-смат\-ри\-ва\-ют\-ся). В~будущем число объектов 
в~средов$\acute{\mbox{о}}$й версии, которая выбрана как основание для 
сис\-те\-ма\-ти\-за\-ции трансформаций первого и~второго порядка, может быть 
увеличено. Для построения классификации трансформаций 
важ\-но не возможное увеличение числа объектов 
и~трансформаций между ними, а то, что их виды в~средов$\acute{\mbox{о}}$й 
версии распределены между трансформациями первого и~второго порядка. Из 
16~видов на рис.~1 шесть относятся к~трансформациям первого порядка, это\linebreak 
виды с~номерами~7, 8, 13--16 (далее~--- типология трансформаций первого 
порядка), а~десять~--- к~трансформациям второго порядка, это виды 
с~\mbox{номерами}~1--6 и~9--12 (далее~--- типология трансформаций второго 
порядка). Разместим обе типологии на третьем уровне классификации (см.\ ее 
схему на рис.~2). Перечислим виды трансформаций первой типологии, вводя 
в~скобках их краткие названия, используемые ниже на рис.~3:
  \begin{description}
  \item[\,] 7~--- членение знания на концепты с~помощью одной или нескольких 
знаковых систем (далее~--- членение знания);
  \item[\,] 8~--- формирование знания на основе концептов (формирование 
знания);
  \item[\,] 13~--- обучение ИНС;
  \end{description}
  
  \vspace*{-6pt}
  
  \pagebreak
  
  \end{multicols}
  
  \begin{figure*} %fig2
\vspace*{1pt}
      \begin{center}
     \mbox{%
\epsfxsize=127.513mm 
\epsfbox{zac-2.eps}
}
\end{center}
\vspace*{-9pt}
\Caption{Схема трех верхних уровней классификации трансформаций объектов (объединены 
по три слоя и~для второго, и~для третьего уровней этой классификации)}
\end{figure*}
  
  \begin{multicols}{2}
  
  \noindent
  \begin{description}
  \item[\,] 14~--- восстановление обучающей информации на основе 
содержания обученной ИНС (обращение ИНС);
  \item[\,] 15~--- использование обученной ИНС (использование ИНС);



  \item[\,] 16~--- восстановление исходных данных, соответствующих 
полученным результатам работы обучен\-ной ИНС (восстановление исходных данных 
по результатам ИНС).
  \end{description}
  
  
  Не все виды трансформаций 13--16 поддерживаются в~конкретных системах 
искусственного интеллекта, но с~теоретической точки зрения все их 
предлагается включить в~первую типологию для полноты спектра видов 
трансформаций.
  
  Перечислим виды трансформаций второй типологии:
  \begin{description}
  \item[\,] 1~--- декодирование цифровых данных в~компьютерных системах 
(декодирование данных);
  \item[\,]  2~--- кодирование сенсорно воспринимаемых данных (кодирование 
данных);
  \item[\,] 3~--- ментальное копирование сенсорно воспринимаемых данных 
(ментальное копирование);
  \item[\,] 4~--- восстановление сенсорно воспринимаемых данных по 
ментальным образам (восстановление по образам);
  \item[\,] 5~--- смысловая интерпретация без деления на концепты ментальных 
образов сенсорно воспринимаемых данных (смысловая интерпретация);
  \item[\,] 6~--- восстановление ментальных образов (восстановление образов);
  \item[\,] 9~--- представление концептов в~виде сенсорно воспринимаемой 
информации, например текс\-та\-ми, формулами, таблицами, рисунками и~т.\,д.\ 
(представление концептов);
  \item[\,] 10~--- понимание смысла сенсорно воспринимаемой информации 
(понимание смысла);
  \item[\,] 11~--- кодирование сенсорно воспринимаемой информации 
(кодирование информации);
\end{description}

\vspace*{-6pt}

\pagebreak

\end{multicols}

\begin{figure*} %fig3
\vspace*{1pt}
      \begin{center}
     \mbox{%
\epsfxsize=163mm 
\epsfbox{zac-3.eps}
}
\end{center}
\vspace*{-9pt}
\Caption{Схема частного случая классификации трансформаций объектов (трансформации 
пронумерованы согласно рис.~1)}
\end{figure*}

\begin{multicols}{2}

\noindent
\begin{description}

  \item[\,] 12~--- декодирование цифровой информации (декодирование 
информации).
  \end{description}
  
  Отметим, что в~существующих ИТ
  и~компьютерных системах наиболее часто используются виды 
трансформаций~13 и~15 типологии первого порядка и~1, 2, 11 и~12 типологии 
второго порядка. На рис.~2 в~первом слое третьего уровня классификации 
показаны типологии первого порядка без указания числа трансформаций в~них 
и~без детализации трансформируемых объектов.
  
  Во втором слое третьего уровня классификации условно (без названий) 
показаны типологии второго порядка. Также на рис.~2 в~третьем слое третьего 
уровня классификации условно (также без названий) показаны типологии 
третьего порядка, которые планируется рассмотреть в~отдельной статье. По 
определению они должны включать трансформации между тремя объектами 
разной природы, но средов$\acute{\mbox{а}}$я версия иерархии Акоффа 
включает трансформации только между двумя объектами разной природы. 
Поэтому потребуется другое основание для их систематизации (ранее были 
рассмотрены отдельные примеры трансформаций третьего 
порядка\footnote{Далеко не всегда трансформации третьего и~более высоких порядков можно 
рассматривать как последовательность трансформаций второго порядка. Примером этого могут 
служить трансформации в~процессе обучения пациента пользованию роботизированной рукой, 
охватывающие личностные концепты пациента, релевантные его намерениям, сигналы активности 
мозга как объекты нейросреды и~компьютерные коды~\cite{29-zac}.}~\cite{29-zac}).

\section{Классификация трансформаций: частный~случай}

  Выше было отмечено, что в~будущем число объектов 
в~средов$\acute{\mbox{о}}$й версии иерархии Акоффа может быть увеличено. 
Это означает, что увеличатся и~чис\-ло объектов, и~чис\-ло трансформаций между 
ними в~классификации трансформаций, так как эта средов$\acute{\mbox{а}}$я 
версия служит по определению основанием для систематизации 
трансформаций первого и~второго порядка. Поэтому на третьем уровне рис.~2 
указаны типологии без детализации объектов и~без указания числа 
трансформаций в~каждой из них. С~одной стороны, при таком подходе 
получаем достаточно общий вид этой классификации, так как она не зависит от 
числа объектов в~том или ином варианте средов$\acute{\mbox{о}}$й версии 
(и~это существенно упрощает рис.~2). С~другой стороны, на третьем уровне 
такой общей классификации подразумевается, но не эксплицируется природа 
трансформируемых объектов и~их возможные сочетания в~трансформациях. 

При проектировании лексикографической информационной системы важно 
эксплицировать природу трансформируемых объектов и~их возможные 
сочетания.
  %
  Поэтому в~парадигму информатики~\cite{30-zac} кроме общей 
классификации трансформаций предлагается включать и~ее частные случаи, 
эксплицирующие природу трансформируемых объектов. 

В~этом разделе 
рассмотрим один частный случай, когда используются только естественные 
знаковые сис\-те\-мы из типологии А.~Соломоника~\cite{25-zac} вместе 
с~данными, знанием и~его концептами. Чис\-ло естественных языков при этом не 
ограничено. И~этот частный случай классификации включает только три 
класса природных трансформаций (первого, второго и~третьего порядка, см.\ 
схему классификации на рис.~3).
  
  Первый и~второй уровни схемы общей классификации (см.\ рис.~2) можно 
объединить в~один уровень в~этом частном случае. Ниже этого уровня 
приведено содержание типологий первого и~второго порядка без содержания 
типологий третьего по\-рядка.




  Наполнение типологий первого и~второго порядка соответствует 
средов$\acute{\mbox{о}}$й версии иерархии Акоффа на рис.~1, содержащей 
6~видов трансформаций типологии первого порядка и~10~видов 
трансформаций типологии второго порядка (на рис.~3 стрелки указывают 
направления трансформаций согласно средов$\acute{\mbox{о}}$й версии на рис.~1).
  
  Таким образом, частный случай классификации содержит для этих двух 
типологий 16~теоретически возможных трансформаций, 6 из которых 
в~настоящее время в~существующих ИТ применяются наиболее часто: виды 
трансформаций~1, 2, 11 и~12 типологии второго порядка реализуются 
с~помощью тех или иных методов ко\-ди\-ро\-ва\-ния/де\-ко\-ди\-ро\-ва\-ния 
(например, с~использованием таблиц Unicode), а~виды трансформаций~13 и~15
 в~типологии первого порядка реализуются полностью с~по\-мощью процессов 
цифровой обработки компьютерами.
  
  Остальные виды трансформаций или применяются намного реже (это 
виды~3, 5, 7, 9 и~10), или находятся в~стадии поиска и~разработки (14 и~16) или 
в~настоящее время носят только теоретический характер, обеспечивая полноту 
первой и~второй типологий (4, 6 и~8). Знаком~<<?>> обозначены те виды 
трансформаций, которые по определению не существуют в~используемой 
парадигме информатики~\cite{30-zac}. Однако возможно, что в~других 
будущих подходах к~построению ее парадигмы эти виды трансформаций будут 
существовать.
  
\section{Заключение}

  На сегодняшний день процесс построения классификаций объектов 
предметной области информатики~\cite{22-zac} и~их  
трансформаций~\cite{23-zac} еще не завершен. Однако первые результаты их 
построения уже используются для создания концепции лексикографической 
информационной сис\-те\-мы, обеспечивающей интеграцию двуязычных 
словарей и~параллельных корпусов.
  
  \bigskip
  
  
  Автор признателен рецензентам за помощь в~улучшении статьи.
  
{\small\frenchspacing
 { %\baselineskip=10.6pt
 %\addcontentsline{toc}{section}{References}
 \begin{thebibliography}{99}
\bibitem{1-zac}
\Au{Aijmer K., Altenberg~B.} Advances in corpus-based contrastive linguistics. Studies in honour 
of Stig Johansson.~--- Amsterdam: John Benjamins, 2013. 295~p.  doi: 10.1075/scl.54.
\bibitem{2-zac}
\Au{Добровольский Д.\,О., Кретов~А.\, А., Шаров~С.\,А.} Корпус параллельных текстов~// 
Научная и~техническая информация. Сер.~2: Информационные процессы и~сис\-те\-мы, 2005. 
№\,6. С.~16--27.
\bibitem{3-zac}
\Au{Добровольский Д.\,О.} Корпус параллельных текстов и~сопоставительная 
лексикология~// Труды Института русского языка им.\ В.\,В.~Виноградова, 2015. №\,6. 
С.~413--449. EDN: VJQBHP.
\bibitem{4-zac}
\Au{Гончаров А.\,А., Зацман~И.\,М., Кружков~М.\,Г.} Эволюция классификаций 
в~надкорпусных базах данных~// Информатика и~её применения, 2020. Т.~14. Вып.~4. 
С.~108--116. doi: 10.14357/19922264200415.  
EDN: \mbox{GKWBZT}.
\bibitem{5-zac}
\Au{Гончаров А.\, А., Зацман И. \,М., Кружков~М.\, Г}. Представление новых 
лексикографических знаний в~динамических классификационных сис\-те\-мах~// 
Информатика и~её применения, 2021. Т.~15. Вып.~1. С.~86--93.  doi: 10.14357/19922264210112. EDN: OPEFXW.
\bibitem{6-zac}
\Au{Zatsman I.} Finding and filling lacunas in linguistic typologies~// 15th Forum (International) 
on Knowledge Asset Dynamics Proceedings.~--- Matera, Italy: Institute of Knowledge Asset 
Management, 2020. P.~780--793.
\bibitem{7-zac}
\Au{Zatsman I.} Three-dimensional encoding of emerging meanings in AI-systems~// 21st 
European Conference on Knowledge Management Proceedings.~--- Reading, U.K.: Academic 
Publishing International Ltd., 2020. P.~878--887.
\bibitem{8-zac}
\Au{Ackoff R.} From data to wisdom~// J.~Applied Systems Analysis, 1989. Vol.~16. No.\,1. P.~3--9.
\bibitem{9-zac}
\Au{Rosenbloom P.\,S.} On computing: The fourth great scientific domain.~--- Cambridge, MA, 
USA: MIT Press, 2013. 307~p.
\bibitem{10-zac}
\Au{Rowley J.} The wisdom hierarchy: Representations of the DIKW hierarchy~// J.~Inf. 
Sci., 2007. Vol.~33. Iss.~2. P.~163--180. doi: 10.1177/0165551506070706.
\bibitem{11-zac} 
\Au{Frick$\acute{\mbox{e}}$~M.\,H.} Data--Information--Knowledge--Wisdom (DIKW) pyramid, 
framework, continuum~// Encyclopedia of big data~/ Eds. L.~Schintler, C.~McNeely.~--- Cham: 
Springer, 2018. 4~p. doi: 10.1007/978-3-319-32001-4\_331-1.
\bibitem{12-zac}
\Au{Denning P., Rosenbloom~P.} Computing: The fourth great domain of science~// Commun. 
ACM, 2009. Vol.~52. Iss.~9. P.~27--29.
\bibitem{13-zac}
\Au{Denning P., Freeman~P.} Computing's paradigm~// Commun.  ACM, 2009. Vol.~52. 
Iss.~12. P.~28--30. doi: 10.1145/ 1610252.1610265.
\bibitem{17-zac} %14
\Au{Farradane J.} Knowledge, information, and information science~// J.~Inf. Sci., 
1980. Vol.~2. Iss.~2. P.~75--80. doi: 10.1177/01655515800020020.

\bibitem{15-zac}
\Au{Шрейдер Ю.\,А.} Информация и~знание~// Сис\-тем\-ная концепция информационных 
процессов.~--- М.: ВНИИСИ, 1988. С.~47--52.
\bibitem{16-zac}
\Au{Ingwersen P.} Information and information science~// Enclyclopaedie of library and 
information science~/ Eds. J.\,D.~McDonald, 
M.~Levine-Clark.~--- New York, NY, USA: Marcel Dekker Inc., 1992. Vol.~56. Sup.~19. 
P.~137--174.

\bibitem{14-zac} %17
Информатика как наука об информации: Информационный, документальный, 
технологический, экономический, социальный и~организационный аспекты~/ Под ред. 
Р.\,С.~Гиляревского.~--- М.: Фаир-Пресс, 2006. 592~с.

\bibitem{18-zac}
\Au{Hjorland B.} Library and information science: practice, theory, and philosophical basis~// 
Inform. Process. Manag., 2000. Vol.~36. Iss.~3. P.~501--531. doi:  
10.1016/S0306-\mbox{4573(99)00038-2}.
\bibitem{19-zac}
Deep shift~--- technology tipping points and societal impact.~--- Geneva: WE Forum, 2015. 44~p. 
{\sf http://www3.weforum.org/docs/WEF\_GAC15\_ Technological\_Tipping\_Points\_report\_2015.pdf}.
\bibitem{20-zac}
\Au{Berman F., Rutenbar~R., Hailpern~B., Christensen~H., Davidson~S., Estrin~D., 
Franklin~M., Martonosi~M., Raghavan~P., Stodden~V., Szalay~A.\,S.} Realizing the potential of 
data science~// Commun.  ACM, 2018. Vol.~61. Iss.~4. P.~67--72. doi: 10.1145/3188721.

\bibitem{21-zac}
\Au{Stodden V.} The data science life cycle: A~disciplined approach to advancing data science as 
a~science~// Commun.  ACM, 2020. Vol.~63. Iss.~7. P.~58--66. doi: 10.1145/ 3360646.


\bibitem{23-zac} %22
\Au{Зацман И.\,М.} Научная парадигма информатики: классификация трансформаций 
объектов предметной об\-ласти~// Системы и~средства информатики, 2023. Т.~33. №\,4. 
С.~126--138. doi: 10.14357/08696527230412. EDN: ZIKUWO.

\bibitem{22-zac} %23
\Au{Зацман И.\,М.} Научная парадигма информатики: классификация объектов предметной  
об\-ласти~// Информатика и~её применения, 2023. Т.~17. Вып.~4. С.~96--103. doi: 
10.14357/19922264230413. EDN: FIUQAT.

\bibitem{24-zac}
\Au{Зацман И.\,М.} О~научной парадигме информатики: верхний уровень классификации 
объектов ее предметной об\-ласти~// Информатика и~её применения, 2022. Т.~16. Вып.~4. 
С.~73--79. doi: 10.14357/ 19922264220411. EDN: XZNKVI.

\bibitem{25-zac}
\Au{Соломоник А.\,Б.} Философия знаковых систем и~язык.~--- М.: ЛКИ, 2011. 408~с.
\bibitem{26-zac}
\Au{Зацман И.\,М.} Трансформация иерархии Акоффа в~научной парадигме информатики~// 
Информатика и~её применения, 2023. Т.~17. Вып.~3. С.~107--113. doi: 
10.14357/19922264230315. EDN: UMVRRV.

\bibitem{27-zac}
\Au{Zatsman I.} Building digital spiral models of knowledge generation~// 19th Forum 
(International) on Knowledge Asset Dynamics Proceedings.~--- Matera, Italy: Arts for Business 
Institute, 2024. P.~2185--2196.
\bibitem{28-zac}
\Au{Zatsman I.} Digital spiral model of knowledge creation and encoding its dynamics~// 18th 
Forum (International) on Knowledge Asset Dynamics Proceedings.~--- Matera, Italy: Arts for 
Business Institute, 2023. P.~581--596.
\bibitem{29-zac}
\Au{Зацман И.\,М.} Интерфейсы третьего порядка в~информатике~// Информатика и~её 
применения, 2019. Т.~13. Вып.~3. С.~82--89. doi: 10.14357/19922264190312. EDN: 
EHRQLF.

\bibitem{30-zac}
\Au{Зацман И.\,М.} Научная парадигма информатики как третьей культуры~//  
На\-уч\-но-тех\-ни\-че\-ская информация. Сер.~1: Организация и~методика информационной 
работы, 2023. №\,11. С.~1--14.

\end{thebibliography}

 }
 }

\end{multicols}

\vspace*{-9pt}

\hfill{\small\textit{Поступила в~редакцию 14.04.24}}

\vspace*{4pt}

%\pagebreak

%\newpage

%\vspace*{-28pt}

\hrule

\vspace*{2pt}

\hrule



\def\tit{OBJECT TRANSFORMATIONS OF~THE~FIRST AND~SECOND ORDER
IN~A~LEXICOGRAPHIC INFORMATION SYSTEM\\[-5pt]}


\def\titkol{Object transformations of~the~first and~second order
in~a~lexicographic information system}


\def\aut{I.\,M.~Zatsman}

\def\autkol{I.\,M.~Zatsman}

\titel{\tit}{\aut}{\autkol}{\titkol}

\vspace*{-13pt}


\noindent
Federal Research Center ``Computer Science and Control'' of the Russian Academy of Sciences, 
44-2~Vavilov Str., Moscow 119133, Russian Federation


\def\leftfootline{\small{\textbf{\thepage}
\hfill INFORMATIKA I EE PRIMENENIYA~--- INFORMATICS AND
APPLICATIONS\ \ \ 2024\ \ \ volume~18\ \ \ issue\ 2}
}%
 \def\rightfootline{\small{INFORMATIKA I EE PRIMENENIYA~---
INFORMATICS AND APPLICATIONS\ \ \ 2024\ \ \ volume~18\ \ \ issue\ 2
\hfill \textbf{\thepage}}}

\vspace*{2pt}



\Abste{The theoretical foundations of the design of information technologies used for 
the integration of bilingual dictionaries and parallel corpora are considered. The 
description of the first outcomes of the creation of the third\linebreak\vspace*{-12pt}}

\Abstend{ level of object 
transformations classification in the subject domain of informatics, which is supposed 
to be used
in creating the lexicographic information system providing integration, is 
given. All the entities of informatics are divided into two global classes: objects and 
their transformations. For each such class, its own classification is constructed. 
Previously, the two upper levels of the object transformation classification in the subject 
domain have been described. The present paper discusses the third level of this classification. The 
basis for the construction of its highest level was the division of the subject domain of 
informatics into media (mental, sensory, digital, and a~number of other media), each 
of which by definition includes objects of the same nature. The Solomonick's 
typology of sign systems served as the basis for constructing the second level of the 
object transformation classification. The aim of the paper is to systematize object 
transformations of the first and second orders at the third level of this classification. 
The basis for systematization is the medium version of the Ackoff's hierarchy.}

\KWE{subject domain objects; object transformations; classification; data; 
information; knowledge; lexicographic information system}


\DOI{10.14357/19922264240211}{VZTGVV}

\vspace*{-12pt}

\Ack

\vspace*{-3pt}


\noindent
The reported study was funded by the Russian Science Foundation, project  
No.\,24-18-00155, {\sf 
https://rscf.ru/project/24-18-00155}. The research was carried out using the infrastructure of the Shared 
Research Facilities ``High Performance Computing and Big Data'' (CKP 
``Informatics'') of FRC CSC RAS (Moscow) .
   


  \begin{multicols}{2}

\renewcommand{\bibname}{\protect\rmfamily References}
%\renewcommand{\bibname}{\large\protect\rm References}

{\small\frenchspacing
 {%\baselineskip=10.8pt
 \addcontentsline{toc}{section}{References}
 \begin{thebibliography}{99} 
\bibitem{1-zac-1}
\Aue{Aijmer, K., and B.~Altenberg.} 2013. \textit{Advances in corpus-based 
contrastive linguistics. Studies in honour of Stig Johansson}. Amsterdam: John 
Benjamins. 295~p. doi: 10.1075/scl.54.
\bibitem{2-zac-1}
\Aue{Dobrovolskiy, D.\,O., A.\,A.~Kretov, and S.\,A.~Sharov.} 2005. Korpus 
parallel'nykh tekstov [Corpus of parallel texts]. \textit{Nauchnaya i~tekhnicheskaya 
informatsiya. Ser. 2. Informatsionnye protsessy i~sistemy} [Scientific and Technical 
Information. Ser.~2: Information Processes and Systems] 6:16--27.
\bibitem{3-zac-1}
\Aue{Dobrovolskiy, D.\,O.} 2015. Korpus parallel'nykh tekstov i~sopostavitel'naya 
leksikologiya [The corpus of parallel texts and contrastive lexicology]. \textit{Trudy 
Instituta russkogo yazyka im. V.\,V.~Vinogradova} [Proceedings of the 
V.\,V.~Vinogradov Russian Language Institute] 6:413--449. EDN: VJQBHP.
\bibitem{4-zac-1}
\Aue{Goncharov, A.\,A., I.\,M.~Zatsman, and M.\,G.~Kruzhkov.} 2020. Evolyutsiya 
klassifikatsiy v~nadkorpusnykh ba\-zakh dannykh [Evolution of classifications in 
supracorpora databases]. \textit{Informatika i~ee Primeneniya~--- Inform. \mbox{Appl.}}  
14(4):108--116. doi: 10.14357/19922264200415.  
EDN: GKWBZT.
\bibitem{5-zac-1}
\Aue{Goncharov, A.\,A., I.\,M.~Zatsman, and M.\,G.~Kruzhkov.} 2021. 
Predstavlenie novykh leksikograficheskikh znaniy v~dinamicheskikh 
klassifikatsionnykh sistemakh [Representation of new lexicographical knowledge in 
dynamic classification systems]. \textit{Informatika i~ee Primeneniya~--- Inform. 
Appl.} 15(1):86--93. doi: 10.14357/19922264210112. EDN: OPEFXW.
\bibitem{6-zac-1}
\Aue{Zatsman, I.} 2020. Finding and filling lacunas in linguistic typologies. 
\textit{15th Forum (International) on Knowledge Asset Dynamics Proceedings}. 
Matera, Italy: Institute of Knowledge Asset Management. 780--793.
\bibitem{7-zac-1}
\Aue{Zatsman, I.} 2020. Three-dimensional encoding of emerging meanings in  
AI-systems. \textit{21st European Conference on Knowledge Management 
Proceedings}. Reading, U.K.: Academic Publishing International Ltd. 878--887.
\bibitem{8-zac-1}
\Aue{Ackoff, R.} 1989. From data to wisdom. \textit{J.~Applied Systems Analysis} 
16(1):3--9.
\bibitem{9-zac-1}
\Aue{Rosenbloom, P.\,S.} 2013. \textit{On computing: The fourth great scientific 
domain}. Cambridge, MA: MIT Press. 307~p.
\bibitem{10-zac-1}
\Aue{Rowley, J.} 2007. The wisdom hierarchy: Representations of the DIKW 
hierarchy. \textit{J.~Inf. Sci.} 33(2):163--180. doi: 10.1177/0165551506070706.
\bibitem{11-zac-1}
\Aue{Frick$\acute{\mbox{e}}$, M.\,H.} 2018.  
Data-Information-Knowledge-Wisdom (DIKW) pyramid, framework, continuum. 
\textit{Encyclopedia of big data}. Eds. L.~Schintler and C.~McNeely. Cham: 
Springer. 4~p. doi: 10.1007/978-3-319-32001- 4\_331-1.
\bibitem{12-zac-1}
\Aue{Denning, P., and P.~Rosenbloom.} 2009. Computing: The fourth great domain 
of science. \textit{Commun. ACM} 52(9):27--29.
\bibitem{13-zac-1}
\Aue{Denning, P., and P.~Freeman.} 2009. Computing's paradigm. \textit{Commun. 
ACM} 52(12):28--30. doi: 10.1145/ 1610252.1610265.

\bibitem{17-zac-1} %14
\Aue{Farradane, J.} 1980. Knowledge, information, and information science. 
\textit{J.~Inf. Sci.} 2(2):75--80. doi: 10.1177/ 01655515800020020.

\bibitem{15-zac-1}
\Aue{Shreyder, Yu.\,A.} 1988. Informatsiya i~znanie [Information and knowledge]. 
\textit{Sistemnaya kontseptsiya in\-for\-ma\-tsi\-on\-nykh protsessov} [System concept of 
information processes]. Moscow: VNIISI. 47--52.
\bibitem{16-zac-1}
\Aue{Ingwersen, P.} 1995. Information and information science. 
\textit{Encyclopedia of library and information science}. Eds. J.\,D.~McDonald and 
M.~Levine-Clark. New York, NY: Marcel Dekker Inc. 56(19):137--174.

\bibitem{14-zac-1} %17
Gilyarevskiy, R.\,S., ed. 2006. \textit{Informatika kak nauka ob informatsii: 
informatsionnyy, dokumental'nyy, tekh\-no\-lo\-gi\-che\-skiy, ekonomicheskiy, sotsial'nyy 
i~organizatsionnyy aspekty} [Informatics as information science: Informational, 
documentary, technological, economic, social, and organizational dimensions]. 
Moscow: FAIR-PRESS. 592~p.

\bibitem{18-zac-1}
\Aue{Hjorland, B.} 2000. Library and information science: Practice, theory, and 
philosophical basis. \textit{Inform. Process. Manag.} 36(3):501--531. doi:  
10.1016/S0306-\mbox{4573(99)00038-2}.
\bibitem{19-zac-1}
Deep shift~--- technology tipping points and societal impact. 2015. \textit{World Economic 
Forum}. Geneva. 44~p. Available at: {\sf 
http://www3.weforum.org/docs/WEF\_ GAC15\_Technological\_Tipping\_Points\_report\_2015.pdf} (accessed May~20, 
2024).
\bibitem{20-zac-1}
\Aue{Berman, F., R.~Rutenbar, B.~Hailpern, H.~Christensen, S.~Davidson, 
D.~Estrin, M.~Franklin, M.~Martonosi, P.~Raghavan, V.~Stodden, and 
A.\,S.~Szalay.} 2018. Realizing the potential of data science. \textit{Commun. ACM} 
61(4):67--72. doi: 10.1145/3188721.
\bibitem{21-zac-1}
\Aue{Stodden, V.} 2020. The data science life cycle: A~disciplined approach to 
advancing data science as a~science. \textit{Commun. ACM} 
 63(7):58--66. doi: 10.1145/3360646.

\bibitem{23-zac-1} %22
\Aue{Zatsman, I.\,M.} 2023. Nauchnaya paradigma informatiki: klassifikatsiya 
transformatsiy ob''ektov predmetnoy oblasti [Scientific paradigm of informatics: 
Transformation classification of domain objects]. \textit{Sistemy i~Sredstva 
Informatiki~--- Systems and Means of Informatics} 33(4):126--138. doi: 
10.14357/08696527230412. EDN: ZIKUWO.

\bibitem{22-zac-1} %23
\Aue{Zatsman, I.\,M.} 2023. Nauchnaya paradigma informatiki: klassifikatsiya 
ob''ektov predmetnoy oblasti [Scientific paradigm of informatics: Classification of 
domain objects]. \textit{Informatika i~ee Primeneniya~--- Inform. Appl.} 
 17(4):96--103. doi: 10.14357/19922264230413. EDN: FIUQAT.
 
\bibitem{24-zac-1}
\Aue{   Zatsman, I.\,M.} 2022. O nauchnoy paradigme informatiki: verkhniy uroven' 
klassifikatsii ob''ektov ee predmetnoy oblasti [On the scientific paradigm of 
informatics: The classification high level of its objects]. \textit{Informatika i~ee 
Primeneniya~--- Inform. Appl.} 16(4):73--79. doi: 10.14357/19922264220411. EDN: 
XZNKVI.
\bibitem{25-zac-1}
\Aue{Solomonick, A.\,B.} 2011. \textit{Filosofiya znakovykh system i~yazyk} 
[Philosophy of sign systems and language]. Moscow: LKI. 408~p.
\bibitem{26-zac-1}
\Aue{Zatsman, I.\,M.} 2023. Transformatsiya ierarkhii Akoffa v~nauchnoy 
paradigme informatiki [Transformation of the Ackoff's hierarchy in the scientific 
paradigm of informatics]. \textit{Informatika i~ee Primeneniya~--- Inform. \mbox{Appl.}} 
17(3):107--113. doi: 10.14357/19922264230315. EDN: UMVRRV.
\bibitem{27-zac-1}
\Aue{Zatsman, I.} 2024. Building digital spiral models of knowledge 
generation. \textit{19th Forum (International) on Knowledge Asset Dynamics 
Proceedings}. Matera, Italy: Arts for Business Institute. 2185--2196.
\bibitem{28-zac-1}
\Aue{Zatsman, I.} 2023. Digital spiral model of knowledge creation and encoding its 
dynamics. \textit{18th Forum (International) on Knowledge Asset Dynamics 
Proceedings}. Matera, Italy: Arts for Business Institute. 581--596.
\bibitem{29-zac-1}
\Aue{Zatsman, I.\,M.} 2019. Interfeysy tret'ego poryadka v~informatike 
 [Third-order interfaces in informatics]. \textit{Informatika i~ee Primeneniya~--- 
Inform. Appl.} 13(3):82--89. doi: 10.14357/19922264190312. EDN: EHRQLF.
\bibitem{30-zac-1}
\Aue{Zatsman, I.} 2023. Scientific paradigm of informatics as a~third culture. 
\textit{Scientific Technical Information Processing} 50(4):246--258. doi: 
10.3103/S0147688223040111. EDN: CKHMYS.

\end{thebibliography}

 }
 }

\end{multicols}

\vspace*{-6pt}

\hfill{\small\textit{Received April 14, 2024}} 


\vspace*{-12pt}


\Contrl

\vspace*{-3pt}

\noindent
\textbf{Zatsman Igor M.} (b.\ 1952)~--- Doctor of Science in technology, head of 
department, Federal Research Center ``Computer Science and Control'' of the 
Russian Academy of Sciences, 44-2~Vavilov Str., Moscow 119333, Russian 
Federation; \mbox{izatsman@yandex.ru}





\label{end\stat}

\renewcommand{\bibname}{\protect\rm Литература}    %13
\def\stat{gorbunova}

\def\tit{РЕСУРСНЫЕ СИСТЕМЫ МАССОВОГО ОБСЛУЖИВАНИЯ КАК~МОДЕЛИ БЕСПРОВОДНЫХ СИСТЕМ 
СВЯЗИ$^*$}

\def\titkol{Ресурсные системы массового обслуживания как~модели беспроводных систем 
связи}

\def\aut{А.\,В.~Горбунова$^1$, В.\,А.~Наумов$^2$, Ю.\,В.~Гайдамака$^3$, К.\,Е.~Самуйлов$^4$}

\def\autkol{А.\,В.~Горбунова, В.\,А.~Наумов, Ю.\,В.~Гайдамака, К.\,Е.~Самуйлов}

\titel{\tit}{\aut}{\autkol}{\titkol}

\index{Горбунова А.\,В.}
\index{Наумов В.\,А.}
\index{Гайдамака Ю.\,В.}
\index{Самуйлов К.\,Е.}
\index{Gorbunova A.\,V.}
\index{Naumov V.\,A.}
\index{Gaidamaka Yu.\,V.}
\index{Samouylov K.\,E.}




{\renewcommand{\thefootnote}{\fnsymbol{footnote}} \footnotetext[1]
{Публикация подготовлена при финансовой поддержке Минобрнауки России 
(проект 2.882.2017/4.6).}}


\renewcommand{\thefootnote}{\arabic{footnote}}
\footnotetext[1]{Российский университета дружбы народов, 
\mbox{gorbunova\_av@rudn.university}}
\footnotetext[2]{Исследовательский институт инноваций, Хельсинки, 
Финляндия, \mbox{valeriy.naumov@pfu.fi}}
\footnotetext[3]{Российский университет дружбы народов; Институт 
проб\-лем информатики Федерального исследовательского центра <<Информатика 
и~управ\-ле\-ние>> Российской академии наук, \mbox{gaydamaka\_yuv@rudn.university}}
\footnotetext[4]{Российский университет дружбы народов; Институт 
проблем информатики Федерального исследовательского центра <<Информатика 
и~управ\-ле\-ние>> Российской академии наук, \mbox{samouylov\_ke@rudn.university}}

\vspace*{-5pt}



\Abst{Представлен обзор ресурсных систем массового обслуживания (СМО), используемых 
для моделирования широкого класса реальных систем, в~которых ресурсы являются 
заведомо ограниченными. Несмотря на объективную важность исследования подобных 
систем, работ, посвященных их анализу, до последнего времени существовало совсем 
немного, что было связано со сложностью построения случайного процесса, 
описывающего их функционирование, и,~соответственно, получения численных 
результатов. Однако за последние годы произошел существенный сдвиг в~изучении 
ресурсных систем, были предложены новые методы их анализа, позволяющие строить 
рекуррентные алгоритмы, пригодные для численных расчетов.
В~этой связи в~обзоре отражена только часть полученных результатов, а~именно:
рассмотрены ресурсные системы без мест для ожидания с~экспоненциальным временем 
обслуживания. Рассмотрены модели беспроводных систем связи, основанные на 
ресурсных СМО (РСМО), выражения для оценки основных 
ве\-ро\-ят\-но\-ст\-но-вре\-мен\-ных характеристик и~алгоритмы их вычисления.}


\KW{ресурсная система массового обслуживания; непрерывный 
ресурс; дискретный ресурс; ограниченный ресурс; рекуррентный алгоритм; 
гетерогенная сеть; стационарное распределение; полумарковский процесс; 
беспроводные системы связи}

\DOI{10.14357/19922264180307}
  
%\vspace*{4pt}


\vskip 10pt plus 9pt minus 6pt

\thispagestyle{headings}

\begin{multicols}{2}

\label{st\stat}

\section{Введение}

В классических СМО приборы и~места ожидания 
играют роль необходимых для обслуживания ресурсов. В~РСМО кроме 
приборов и~мест ожидания заявкам могут потребоваться различные дополнительные 
ресурсы. Это может быть некоторый случайный объем ресурса, занимаемого на время 
ожидания начала обслуживания, либо на время обслуживания, либо на все время 
пребывания заявки в~сис\-те\-ме. Если у~сис\-те\-мы нет достаточного числа свободных 
ресурсов, поступившая заявка теряется. В~дальнейшем будем использовать термин 
<<ресурс>> только для обозначения дополнительного ресурса, отличного от приборов 
или мест ожидания.

Интерес к~РСМО объясняется возможностью их применения для моделирования 
достаточно широкого спектра технических устройств 
и~в~целом ин\-фор\-ма\-ци\-он\-но-вы\-чис\-ли\-тель\-ных систем.
В~частности, если говорить о единственном типе ресурса ограниченного объема, то 
таким образом может моделироваться ограниченность памяти некоторого устройства 
или отдельной системы.
Таким образом, увеличивается реалистичность модели и,~соответственно, ее 
практическая цен\-ность.
%
Если же говорить о~множественных ресурсах, то стоит вспом\-нить услуги 
беспроводных сетей, таких, например, как Long Term Evolution (LTE)~\cite{Andrews}.
 Рост их популярности делает необходимым создание эффективных 
инструментов для оценки телекоммуникационными операторами работы 
радиоинтерфейсов~\cite{Galinina,Samuylov}. В~этих сетях каждая активная сессия 
занимает определенный объем радиоресурсов (например, ширину полосы пропускания 
спектра час\-тот, мощ\-ности передачи радиочастотного усилителя и~др.), которые 
являются заведомо ограниченными и~должны быть распределены при поступлении 
вызова пользователя и~освобождены по завершении сессии~\cite{Naumov_3_2016}.

Стоит отметить, что моделированию беспроводных систем связи с~по\-мощью СМО 
с~множественными ресурсами начиная с~\cite{Gimpelson} посвящено большое чис\-ло 
публикаций. Однако основной акцент в~них делается на анализ различных схем 
распределения ресурсов в~системах c детерминированными требованиями заявок 
к~ресурсам. Обзор этих работ можно найти в~\cite{Kelly,Ross,Basharin}.

Статья организована следующим образом: в~разд.~2 описываются основные типы 
РСМО без мест для ожидания, методы их исследования и~полученные результаты 
в~виде выражений для основных ве\-ро\-ят\-но\-ст\-но-вре\-мен\-н$\acute{\mbox{ы}}$х характеристик 
функционирования указанных сис\-тем. В~разд.~3 представлены подходы, позволяющие 
провести численные расчеты с~по\-мощью полученных соотношений. В~заключении кратко 
подведены итоги работы.

\section{Ресурсные системы массового обслуживания}

Более подробно остановимся на описании общей модели РСМО без мест для ожидания 
(рис.~1).
Система может располагать ограниченным или неограниченным объемом ресурсов как 
одного, так и~нескольких типов. Схему ее функционирования можно описать 
следующим образом:
\begin{enumerate}[(1)]
\item для обслуживания каждой заявки требуется один прибор и~некоторый объем 
ресурса каж\-до\-го типа;
\item поступившая заявка теряется, если в~момент поступления объем 
требуемого ей ресурса превышает объем свободного ресурса этого типа либо все 
приборы заняты;
\item в~момент начала обслуживания заявки суммарный объем занятого ресурса 
каждого типа увеличивается на величину ресурса, выделенного этой заявке;
\item в~момент окончания обслуживания заявки суммарный объем занятого 
ресурса каждого типа уменьшается на величину ресурса, выделенного этой заявке.
\end{enumerate}



В СМО может поступать один или несколько классов заявок, для которых
$A_l(t)$~--- функция распределения времени между поступлениями заявок класса~$l$,
$H_l(t,\mathbf{x})$~--- совместная функция распределения длительности 
обслуживания и~вектора объема ресурсов, необходимых поступившей заявке\linebreak\vspace*{-12pt}

{ \begin{center}  %fig1
 \vspace*{9pt}
  \mbox{%
 \epsfxsize=78.288mm 
 \epsfbox{gor-1.eps}
 }


\vspace*{6pt}


\noindent
{{\figurename~1}\ \ \small{Схема функционирования  РСМО общего вида}}
\end{center}
}

%\vspace*{9pt}

{ \begin{center}  %fig2
 \vspace*{-2pt}
  \mbox{%
 \epsfxsize=61.777mm 
 \epsfbox{gor-2.eps}
 }


\vspace*{9pt}

\noindent
{{\figurename~2}\ \ \small{Схема функционирования простейшей РСМО}}
\end{center}
}

\vspace*{9pt}





\noindent
 класса~$l$, 
$l\hm=\overline{1,L}$.
Для случая, когда случайные величины длительности обслуживания и~вектора объема 
необходимых ресурсов независимы, имеем 
$$
H_l(t,\mathbf{x})=B_l(t)F_l(\mathbf{x})\,,
$$
 где
$B_l(t)$~--- функция распределения времени обслуживания заявки класса~$l$;
$F_l(\mathbf{x})$~--- функция распределения вектора объема ресурсов, тре\-бу\-емых 
заявкам класса~$l$.

Первые статьи, посвященные анализу СМО с~выделением 
каждой поступающей заявке помимо прибора некоторого случайного объема ресурса 
единственного типа появились в~начале \mbox{1970-х~гг.}~\cite{Romm_21_1971,Kac}.
В~част\-ности, в~работе~\cite{Romm_21_1971} рассматривалась бесконечно линейная 
СМО с~пуассоновским входящим потоком 
и~экспоненциальным временем обслуживания (рис.~2). 
Величины требуемых 
ресурсов~--- независимые одинаково распределенные случайные величины с~функцией 
распределения~$F(x)$. В~качестве емкости системы, т.\,е.\ максимально допустимого 
объема ресурсов, выступает величина~$R$.

Система уравнений равновесия (СУР) для случайного процесса, описывающего 
систему, фактически представляет собой обобщение системы Эрланга. В~результате 
решения СУР были получены стационарные вероятности того, что в~сис\-те\-ме 
находится~$k$~заявок:
\begin{equation*}
p_k=\fr{({1}/{k!})({\lambda}/{\mu})^kF^{(k)}(R)}{\sum\nolimits_{i=0}^{\infty}
({1}/{i!})({\lambda}/{\mu})^iF^{(i)}(R)}\,,
\end{equation*}
где $F^{(k)}(x)$ является $k$-крат\-ной сверткой функции распределения~$F(x)$, 
$k=2,3,\ldots$,
$F^{(0)}(x)\hm=1$, $F^{(1)}(x)\hm=F(x)$.
В~условиях описанной модели потеря заявки или отказ в~обслуживании происходят 
только тогда, когда раз\-ность между величиной объема всей сис\-те\-мы и~суммарным 
объемом ресурсов, занятых находящимися в~сис\-те\-ме заявками, меньше, чем величина 
требуемого объема ресурсов у~вновь поступившей заявки. Таким образом, 
вероятность потери заявки равна
\begin{equation*}
B=1-
\fr{\sum\nolimits_{k=0}^{\infty}({1}/{k!})({\lambda}/{\mu})^kF^{(k+1)}(R)}
{\sum\nolimits_{k=0}^{\infty}({1}/{k!})({\lambda}/{\mu})^kF^{(k)}(R)}\,.
\end{equation*}

Для того чтобы более детально ознакомиться с~особенностями построения и~анализа 
РСМО, подробнее остановимся на статье~\cite{Naumov_3_2016}.
Здесь рас\-смат\-ри\-ва\-ет\-ся многолинейная СМО c $N\hm\leq \infty$ приборами. Поступающий 
поток является пуассоновским с~па\-ра\-мет\-ром~$\lambda$, длительности обслуживания 
заявок независимы между собой и~от поступающего потока и~имеют экспоненциальное 
распределение с~параметром~$\mu$. Система располагает ограниченным объемом 
ресурсов~$M$~типов.
Обозначим через~$R_m$ общий объем ресурса типа~$m$, $\mathbf{R}\hm=(R_1,\ldots,R_M)$, 
и~через $\mathbf{r}_j\hm=(r_{j1}, r_{j2},\ldots, r_{jM})$~--- вектор объемов 
ресурсов, необходимых $j$-й поступившей заявке, $j \hm= 1, 2,\ldots$
Будем считать, что случайные векторы~$\mathbf{r}_j$ не зависят от процессов 
поступления и~обслуживания заявок, независимы в~совокупности и~одинаково 
распределены с~функцией распределения $F(\mathbf{x}), \mathbf{x}\hm=(x_1,\ldots,x_M)$.
Состояние такой системы в~момент~$t$ можно описать полумарковским процессом 
$X(t)\hm=\{\xi(t),\boldsymbol{\Gamma}(t)\}$~\cite{Naumov_3_2016}. Здесь~$\xi(t)$~--- 
число заявок в~сис\-те\-ме, а~$\mathbf{\Gamma}(t)\hm=
(\boldsymbol{\gamma}_1(t),\boldsymbol{\gamma}_2(t),\ldots,\boldsymbol{\gamma}_{\xi(t)}
(t))$, где $\boldsymbol{\gamma}_i(t)$~--- вектор объемов 
всех типов ресурсов, занимаемых $i$-й обслуживаемой заявкой. Находящиеся на 
обслуживании заявки перенумеровываются в~порядке убывания остаточного времени 
обслуживания.
Рассмотрим предельное распределение процесса~$X(t):$
\begin{align*}
p_0&=\lim_{t\rightarrow \infty}P\{\xi(t)=0\}\,;
\\
P_k\left(\mathbf{x}_1,\mathbf{x}_2,\ldots,\mathbf{x}_k\right)&=\lim\limits_{t\rightarrow 
\infty} P
\left\{\xi(t)=k;\right.\\
&
 \hspace*{-20mm}\left.\boldsymbol{\gamma}_1(t)\leq 
\mathbf{x}_1,\enskip
\boldsymbol{\gamma}_2(t)\leq 
\mathbf{x}_2,\ldots,\boldsymbol{\gamma}_k(t)\leq \mathbf{x}_k\right\}\,.
\end{align*}
После решения соответствующей СУР получаем
\begin{align*}
%\label{eq:p_0}
p_0&=\left(1+\sum\limits_{i=1}^{N}F^{(k)}(\mathbf{R})\fr{\rho^k}{k!}     \right)^{-1}\,;\\
P_k(\mathbf{x}_1,\mathbf{x}_2,\ldots,\mathbf{x}_k)&=p_0F(\mathbf{x}_1)F(\mathbf{x}_2
)\cdots F(\mathbf{x}_k)\frac{\rho^k}{k!},\\
&\hspace*{-20mm}\mathbf{x}_1,\mathbf{x}_2,\ldots,\mathbf{x}_k \geq \mathbf{0}, \enskip 
\sum\limits_{i=1}^{k}\mathbf{x}_i\leq \mathbf{R}, \enskip 1\leq k \leq N,
\end{align*}
где $\rho=\lambda/\mu$; $F^{(k)}(\mathbf{x})$~--- $k$-крат\-ная свертка 
функции~$F(\mathbf{x})$; $\mathbf{x}_i=(x_{i1},\ldots,x_{iM})$, $i\hm=\overline{1,k}$.
Далее, если обозначить вектор суммарных объемов занятых ресурсов каждого типа 
$\boldsymbol{\delta}(t)\hm=\sum\nolimits_{i=1}^{\xi(t)}\boldsymbol{\gamma}_i(t)$, 
$\boldsymbol{\delta}(t)\hm=(\delta_1(t),\ldots,\delta_M(t))$, стационарное 
распределение~$Q_k(\mathbf{x})$ случайного процесса 
$X(t)\hm=(\xi(t);\boldsymbol{\delta}(t))$ примет вид:
\begin{multline*}
\label{eq:Q}
\hspace*{-6pt}Q_k(\mathbf{x})=\lim_{t\rightarrow \infty} P\{ \xi(t)=k; 
\boldsymbol{\delta}(t)\leq \mathbf{x}\}=p_0F^{(k)}(\mathbf{x}) 
\fr{\rho^{k}}{k!}\,,\\
\mathbf{0}\leq \mathbf{x} \leq \mathbf{R}\,,\enskip  1\leq k \leq N\,.
\end{multline*}

В~\cite{Naumov_6_2015} исследуется РСМО с~единственным типом ограниченного 
ресурса объема~$R$, но уже с~$L$ входящими пуассоновскими потоками 
с~интенсивностями $\lambda_1,\ldots,\lambda_L$ и~с~$N\hm\leq\infty$ приборами.\linebreak 
Длительности обслуживания заявок независимы между собой, от поступающих потоков и~экспоненциально распределены с~параметром~$\mu_l$ для заявок класса~$l$, 
$l\hm=\overline{1,L}$.
Предполагается, что объем %\linebreak 
ресурса, требуемого заявкам класса~$l$, является 
случайной величиной 
с~функцией распределения~$F_l(x)$, не зависящей от процессов поступления 
и~обслуживания заявок. Обслуживающимся заявкам присваивается номер, причем так, 
чтобы заявка с~номером $i$ имела $i$-е по величине остаточное время 
обслуживания. Этот номер следует отличать от порядкового номера заявки. При 
поступлении новой заявки все находящиеся на обслуживании заявки 
перенумеровываются.
Состояние системы в~момент~$t$ описывается полумарковским процессом 
$X(t)\hm=(\xi(t);\boldsymbol{\theta}(t);\boldsymbol{\gamma}(t))$. Здесь, как 
и~прежде, $\xi(t)$~--- число заявок в~сис\-те\-ме; 
$\boldsymbol{\theta}(t)\hm=(\theta_1(t),\theta_2(t),\ldots,\theta_{\xi(t)}(t))$; 
$\boldsymbol{\gamma}(t)\hm=(\gamma_1(t),\gamma_2(t),\ldots,\gamma_{\xi(t)}(t))$, где 
$\theta_i(t)$~--- класс $i$-й обслуживаемой заявки; $\gamma_i(t)$~--- объем 
занимаемого ею ресурса.

Введем стационарное распределение процесса~$X(t)$
\begin{align*}
p_0&=\lim_{t\rightarrow \infty}P\{\xi(t)=0\}\,;
\\
p^k_{l_1,\ldots,l_k}(x_1,\ldots,x_k)&=\lim\limits_{t\rightarrow \infty}P
\left\{\xi(t)=k; \right.\\
&\hspace*{-10mm}\theta_1(t)=l_1,\ldots,\theta_k(t)=l_k; \\
&\left.\gamma_1(t)\leq x_1,\ldots,\gamma_k(t)\leq x_k
\right\}.
\end{align*}
В~результате решения соответствующей СУР получены выражения для стационарных 
вероятностей описанной сис\-те\-мы.
Кроме того, в~\cite{Naumov_6_2015} показано, что стационарные вероятности того, 
что в~сис\-те\-ме находятся
$k_j$ заявок типа~$j$ и~суммарный объем занимаемого ими ресурсов не превосходит~$x_j$, 
$j\hm=\overline{1,L}$, имеют мультипликативный вид:
\begin{equation*}
P_{k_1,\ldots,k_L}(x_1,\ldots,x_k)=p_0
\prod\limits_{j=1}^{L}F_j^{(k_j)}(x_j)\fr{\rho_j^{k_j}}{k_j!}\,.
\end{equation*}

В~\cite{Naumov_10_2015} исследуются показатели эффективности сетей LTE. 
Ресурсная СМО, моделирующая сис\-те\-му, аналогична представленной 
в~\cite{Naumov_6_2015}, но уже с~$M$~типами ограниченных ресурсов, и~потому 
стационарные вероятности
\begin{align*}
p_0&=\lim\limits_{t\rightarrow \infty}P\{\xi(t)=0\}\,;
\\
p^k_{l_1,\ldots,l_k}(\mathbf{x}_1,\ldots,\mathbf{x}_k)&=
\lim\limits_{t\rightarrow \infty}P
\left\{\xi(t)=k;\right.\\
 &\theta_1(t)=l_1,\ldots,\theta_k(t)=l_k;\\
&\left.\boldsymbol{\gamma}_1(t)\leq \mathbf{x}_1,\ldots,\boldsymbol{\gamma}_k(t)\leq 
\mathbf{x}_k
\right\}
\end{align*}
после решения соответствующей СУР примут вид:
\begin{multline*}
\hspace*{70pt}p_0=\left( 
\vphantom{\sum\limits_{r=1}^{N}}
1+{}\right.\\
\left.{}+\sum\limits_{r=1}^{N}\sum\limits_{k_1+\cdots+k_r=r}\hspace*{-3mm}\left(
F_1^{(k_1)}*F_2^{(k_2)}*\cdots *F_r^{(k_r)}
\right)(\mathbf{R})\times{}\right.\\
\left.{}\times \fr{\rho_1^{k_1}}{k_1!}\,\fr{\rho_2^{k_2}}{k_2!}\cdots
\fr{\rho_1^{k_r}}{k_r!} \right)^{-1};
\end{multline*}

\vspace*{-12pt}

\noindent
\begin{multline*}
p^k_{l_1,\ldots,l_k}(\mathbf{x}_1,\ldots,\mathbf{x}_k)={}\\
{}=p_0 
F_{l_1}(\mathbf{x}_1)F_{l_2}(\mathbf{x}_2)\cdots F_{l_k}(\mathbf{x}_k)
\displaystyle\prod\limits_{n=1}^{k}\fr{\lambda_{l_n}}{\sum\nolimits_{i=1}^{n}\mu_{l_i}},\\
1\leq l_1,\ldots,l_k \leq L\,, \enskip 
\mathbf{x}_1,\mathbf{x}_2,\ldots,\mathbf{x}_k\geq \mathbf{0}\,, \\
\displaystyle \sum\limits_{i=1}^{k}\mathbf{x}_i\leq \mathbf{R}, \enskip
1\leq k \leq N,
\end{multline*}
где символ~$*$ означает свертку функции распределения.

В работе~\cite{Naumov_15_2017} моделируется ситуация, когда объем ресурсов, 
запрашиваемых пользователями, может быть не только положительным, но 
и~отрицательным. Запросы на отрицательный объем ресурса увеличивают объем 
доступного ресурса для пользователей, запрашивающих его положительные объемы. 
В~\cite{Naumov_15_2017} предполагается зависимость времени обслуживания 
и~интервалов между поступлениями заявок от числа заявок в~системе. В~результате 
анализа моделей получены формулы для расчета основных 
ве\-ро\-ят\-но\-ст\-но-вре\-мен\-ных характеристик.

В~\cite{ Sopin_12_2017,Sopin_13_2017} для анализа сетей LTE с~динамически 
меняющимися требованиями к~ресурсам рас\-смат\-ри\-ва\-ют\-ся РСМО с~добавлением 
пуассоновского потока сигналов, инициирующего перераспределение ресурсов для 
активных пользователей. Развитием работы~\cite{Sopin_13_2017} 
стали статьи~\cite{Naumov_14_2017, Dohler_2017}.
Были исследованы два сценария перераспределения ресурсов и~сопоставлены 
посредством численного анализа.

В серии работ~\cite{Sopin_13_2017,Sopin_4_2015,Sopin_5_2015,Sopin_7_2016,Sopin_8_2017,Vihrova_
9_2017,Sopin_11_2017,Sopin_17_2018} исследуются РСМО, в~которых объем выделяемых 
заявке ресурсов имеет дискретное распределение, т.\,е.\ для $i$-й поступившей 
в~систему заявки с~вероятностью $p_j\hm=P(r_i\hm=j)$ потребуется ресурс объема~$j$.
Так, в~\cite{Sopin_7_2016} анализируется РСМО с~$L$~входящими пуассоновскими 
потоками и~$M$~типами ресурсов. Получены выражения для стационарных 
вероятностей:
\begin{multline*}
q^k_{k_1,\ldots,k_L}(\mathbf{r}_1,\ldots,\mathbf{r}_L)={}\\
{}=q_0
\sum\limits_{k_1+\cdots+k_l=k}p^{(k_1)}_{1,\mathbf{r}_1}\cdots p^{(k_L)}_{1,\mathbf{r}_L}
\fr{\rho_1^{k_1}}{k_1!}\cdots \fr{\rho_L^{k_L}}{k_L!}\,;
\end{multline*}

\vspace*{-12pt}

\noindent
\begin{multline*}
\hspace*{76pt}q_0={}\\
\!{}=\left(\!
\vphantom{\sum\limits_{k=0}^{N}\sum\limits_{k_1+\cdots+k_L=k}
\sum\limits_{\mathbf{r}_1+\cdots+\mathbf{r}_L \leq 
\mathbf{R}} p^{(k_L)}_{1,\mathbf{r}_L}\fr{\rho_1^{k_1}}{k_1!}\cdots
\fr{\rho_L^{k_L}}{k_L!}}
1+ {}\right. 
\left.\!\!\!\sum\limits_{k=0}^{N}\sum\limits_{k_1+\cdots+k_L=k}
\sum\limits_{\mathbf{r}_1+\cdots+\mathbf{r}_L \leq 
\mathbf{R}} \hspace*{-8pt}p^{(k_L)}_{1,\mathbf{r}_L}\fr{\rho_1^{k_1}}{k_1!}\cdots
\fr{\rho_L^{k_L}}{k_L!}
\!\right)^{\!-1}\!\!\!,\hspace*{-8.1138pt}
\end{multline*}
где $q^k_{k_1,\ldots,k_L}(\mathbf{r}_1,\ldots,\mathbf{r}_L)$~--- это вероятность 
того, что в~системе находятся~$k$~заявок, из которых~$k_1$~--- класса~1, 
$k_2$~--- класса~2 и~т.\,д., а~суммарный объем ресурсов каждого типа, занятых заявками 
класса~1, равен~$\mathbf{r}_1$ и~т.\,д.

В статье~\cite{Vihrova_9_2017} при исследовании той же СМО, что 
и~в~\cite{Sopin_7_2016}, было получено распределение стационарных вероятностей~$q_k(\mathbf{r})$ 
с~объединенным входящим потоком и~средневзвешенным требованием 
$$
p_{\mathbf{r}}= \sum\limits_{l=1}^{L} \fr{\rho_l}{\rho} \,p_{l,\mathbf{r}},$$
 где 
$\rho\hm=\sum\nolimits_{l=1}^{L}\rho_l$:
\begin{equation*}
q_k(\mathbf{r})=q_0\fr{\rho^k}{k!}\,p_{\mathbf{r}}^{(k)}\,, \quad q_0=\left( 
\sum\limits_{k=0}^{N}\sum\limits_{\mathbf{r}=\mathbf{0}}^{\mathbf{R}} p_{\mathbf{r}}^{(k)} 
\right)^{-1},
\end{equation*}
что позволило выразить вероятность блокировки и~среднего объема занятых ресурсов 
в~аналитическом виде.

В случае рассматриваемой СМО, но с~заявками одного класса, 
в~\cite{Sopin_11_2017} представлены выражения для стационарных вероятностей 
состояний числа заявок в~системе и~суммарного объема занятых ресурсов, а~также 
формулы для вероятности блокировки и~среднего объема занятых ресурсов.


\section{Вычисление характеристик ресурсных систем массового обслуживания}

В работе~\cite{Naumov_1_2014} для системы с~одним типом ресурса показано, что 
в~предположении о~гам\-ма-рас\-пре\-де\-ле\-нии необходимого заявкам ресурса плотность 
распределения высвобождаемого заявкой ресурса при заданном числе заявок 
в~системе и~заданном векторе суммарных объемов занятых ресурсов совпадает 
с~бе\-та-рас\-пре\-де\-ле\-ни\-ем, позволяющим легко рассчитывать многократные свертки, к~которым 
приводит необходимость учитывать объемы всех заявок в~системе. В~остальных 
случаях наличие в~формулах большого числа сверток создает значительную 
вычислительную сложность при расчете стационарных характеристик РСМО.
Так, для расчета характеристик СМО из~\cite{Sopin_7_2016} необходимо для каж\-до\-го 
$k\hm\in \{0,\ldots,N\}$, а~также всех наборов векторов $\mathbf{r}\hm \leq \mathbf{R}$ 
хранить в~памяти компьютера значения сверток вероятностей~$p_{\mathbf{r}}$. 
А~при больших значениях~$N$ и~$\mathbf{R}$ вычисление вероятностей блокировок 
системы и~также объемов занятого ресурса по представленным формулам вообще 
нерационально. Поэтому задача получения действенных численных методов является 
крайне важной.
В~работе~\cite{Sopin_8_2017} для модели СМО из~\cite{Sopin_7_2016}, чтобы 
сократить вычисления, был предложен рекуррентный алгоритм вычисления 
нормировочной константы $G(N,\mathbf{R})\hm=q_0^{-1}$, основанный на алгоритме 
Бузена~\cite{Buzen}. Кроме того, на основе разработанного алгоритма были 
получены рекуррентные формулы для вычисления вероятностных характеристик 
сис\-те\-мы: вероятности блокировки сис\-те\-мы, среднего объема дисперсии занятых 
ресурсов.
Если обозначить
\begin{equation*}
G(n,\mathbf{r})\sum\limits_{k=0}^{n}\sum\limits_{\mathbf{j}=\mathbf{0}}^{\mathbf{r}}p_{\mathbf
{j}}^{(k)}\fr{\rho^k}{k!}\,, \enskip 
n\geq 0\,, \enskip \mathbf{r}\geq \mathbf{0}\,,
\end{equation*}
то функция $G(n,\mathbf{r})$ будет удовлетворять следующему рекуррентному 
соотношению:
\begin{multline*}
G(n,\mathbf{r})=G(n-1,\mathbf{r})+{}\\
{}+\fr{\rho}{n!}
\sum\limits_{\mathbf{j}=\mathbf{0}}^{\mathbf{r}}p_{\mathbf{j}}
\left( G(n-1,\mathbf{r}-\mathbf{j})- G(n-2,\mathbf{r}-\mathbf{j})   \right)
\end{multline*}
с начальными условиями
\begin{equation*}
G(0,\mathbf{r})=1,\enskip \mathbf{r}\geq 0\,; \quad
G(1,\mathbf{r})=1+\sum\limits_{\mathbf{j}=\mathbf{0}}^{\mathbf{r}}p_{\mathbf{j}}\,.
\end{equation*}
При анализе РСМО,  описывающих M2M (machine-to-machine) трафик в~сетях LTE, 
аналогичный рекуррент\-ный алгоритм для вычисления нормировочной константы был 
разработан в~\cite{Sopin_11_2017}. Мат\-рич\-ные методы анализа РСМО, применимые при 
моделировании соты сети LTE с~двумя типами трафика, M2M и~H2H (human-to-human), 
предложены в~работах~\cite{Vish_2017,Vish_2016}.

\section{Заключение}

В настоящем обзоре кратко представлены основные разновидности 
РСМО, существующие методы их анализа, выражения для оценки 
основных ве\-ро\-ят\-но\-ст\-но-вре\-мен\-н$\acute{\mbox{ы}}$х 
характеристик и~алгоритмы их вычисления.

{\small\frenchspacing
 {%\baselineskip=10.8pt
 \addcontentsline{toc}{section}{References}
 \begin{thebibliography}{99}
%1
\bibitem{Andrews} %1
\Au{Andrews J.\,G., Buzzi~S., Choi~W., Hanly~S.\,V., Lozano~A., Soong~A.\,C.\,K., 
Zhang~J.\,C.} What will 5G be?~// {IEEE J.~Sel. Area.  
Comm.}, 2014. Vol.~32. No.\,6. P.~1065--1082.



%3
\bibitem{Samuylov} %2
\Au{Buturlin I.\,A., Gaidamaka~Y.\,V., Samuylov~A.\,K.}
Utility function maximization problems for two cross-layer optimization 
algorithms in OFDM wireless networks~// {4th Congress (International) on Ultra 
Modern Telecommunications and Control Systems}, 2012.  P.~63--65.

%2
\bibitem{Galinina} %3
\Au{Galinina O., Andreev~S.\,D., Gerasimenko~M., Kou\-che\-rya\-vy~Y., Himayat~N., 
Yeh~S.\,P., Talwar~S.} Capturing spatial randomness of heterogeneous 
cellular/WLAN deployments with dynamic traffic~// {IEEE J.~Sel. 
Area. Comm.}, 2014. Vol.~32. No.\,6. P.~1083--1099.

%4
\bibitem{Naumov_3_2016}
\Au{Наумов В.\,А., Самуйлов~К.\,Е., Самуйлов~А.\,К.} О~суммарном объеме 
ресурсов, занимаемых обслуживаемыми заявками~// {Автоматика и~телемеханика}, 
2016. №.\,8. С.~125--135.

%5
\bibitem{Gimpelson}
\Au{Gimpelson L.\,A.}
Analysis of mixtures of wide- and narrow-band traffic~// {IEEE T.~Commun.
 Techn.}, 1968. Vol.~13. No.\,3. P.~258--266.

%6
\bibitem{Kelly}
\Au{Kelly F.\,P.}
Loss networks~// {Ann. Appl. Probab.}, 1991. No.\,1. P.~319--378.

%7
\bibitem{Ross}
\Au{Ross K.\,W.}
Multiservice loss models for broadband telecommunication networks.~--- {London: 
Springer-Verlag}, 1995. 343~p.

%8
\bibitem{Basharin}
\Au{Basharin G.\,P., Samouylov~K.\,E., Yarkina~N.\,V., Gudkova~I.\,A.}
A~new stage in mathematical teletraffic theory~// {Automat. Rem. 
Contr.}, 2009. Vol.~70. No.\,12. P.~1954--1964.

%9
\bibitem{Romm_21_1971}
\Au{Ромм Э.\,Л., Скитович~В.\,В.}
Об одном обобщении задачи Эрланга~// {Автоматика и~телемеханика}, 1971. №.\,6. 
С.~164--168.

%10
\bibitem{Kac}
\Au{Кац Б.\,А.}
Об обслуживании сообщений случайной длины~// {Теория массового обслуживания: 
Тр. 3-й Всесоюзн. шко\-лы-со\-ве\-ща\-ния по тео\-рии массового обслуживания}, 1976. 
С.~157--168.

%11
\bibitem{Naumov_6_2015}
\Au{Наумов В.\,А., Самуйлов~А.\,К.}
Модель выделения ресурсов беспроводной сети объемами случайной величины~// 
{Вестник РУДН. Серия: Математика, информатика, физика}, 2015. №\,2. С.~38--45.

%12
\bibitem{Naumov_10_2015}
\Au{Naumov~V., Samouylov~K., Yarkina~N., Sopin~E., Andreev~S., Samuylov~A.}
LTE performance analysis using queuing systems with finite resources and random 
requirements~// {7th Congress on Ultra Modern Telecommunications and Control 
Systems}.~--- IEEE, 2015. P.~100--103.

%13
\bibitem{Naumov_15_2017}
\Au{Naumov V., Samouylov~K.}
Analysis оf multi-resource loss system with state dependent arrival and service 
rates~// {Probab.  Eng. Inform. Sc.}, 2017. 
Vol.~31. No.\,4. P.~413--419.

%14
\bibitem{Sopin_12_2017}
\Au{Samouylov K., Sopin~E., Vikhrova~O.}
Analysis of queueing system with resources and signals~// {Comm.  
Com. Inf. Sc.}, 2017. Vol.~800. P.~358--369.

%15
\bibitem{Sopin_13_2017}
\Au{Sopin E., Vikhrova~O., Samouylov~K.}
LTE network model with signals and random resource requirement~// {9th 
Congress (International) on Ultra Modern Telecommunications and Control Systems 
and Workshops}.~--- IEEE, 2017. P.~101--106.



%17
\bibitem{Dohler_2017}
\Au{Petrov V., Solomitckii~D., Samuylov~A., Lema Maria~A., Gapeyenko~M., 
Moltchanov~D., Andreev~S., Naumov~V., Samouylov~K., Dohler~M., Koucheryavy~Ye.}
Dynamic multi-connectivity performance in ultra-dense urban mmWave deployments~// 
{IEEE J.~Sel. Area. Comm.}, 2017. Vol.~35. No.~9. 
P.~2038--2055.

%16
\bibitem{Naumov_14_2017}
\Au{Наумов В.\,А., Самуйлов~К.\,Е.}
Анализ сетей ресурсных систем массового обслуживания~// {Автоматика 
и~телемеханика}, 2018. №\,5. С.~59--68.

%18
\bibitem{Sopin_4_2015}
\Au{Samouylov K., Sopin~E., Vikhrova~O.}
Analyzing blocking probability in LTE wireless network via queuing system with 
finite amount of resources~// {Comm.  Com. Inf.
Sc.}, 2015. Vol.~564. P.~393--403.

%19
\bibitem{Sopin_5_2015}
\Au{Вихрова О.\,Г., Самуйлов~К.\,Е., Сопин~Э.\,С., Шоргин~С.\,Я.}
К~анализу показателей качества обслуживания в~современных беспроводных сетях~// 
{Информатика и~её применения}, 2015. Т.~9. Вып.~4. С.~48--55.

%20
\bibitem{Sopin_7_2016}
\Au{Sopin~E., Samouylov~K., Vikhrova~O., Kovalchukov~R., Moltchanov~D., 
Samuylov~A.}
Evaluating a case of downlink uplink decoupling using queuing system with random 
requirement~// 
Internet of Things, smart spaces, and
next generation
networks and systems~/
Eds. O.~Galinina, S.\,I.~Balandin, Y.~Koucheryavy.~---
{Lecture notes in computer science ser.}~--- Springer, 2016. Vol.~9870. P.~440--450.

%21
\bibitem{Sopin_8_2017}
\Au{Samouylov K., Sopin~E., Vikhrova~O., Shorgin~S.}
Convolution algorithm for normalization constant evaluation in queuing system 
with random requirements~// {AIP Conf. Proc.}, 2017. Vol.~1863. Art. 
No.\,090004. 4~p.

%22
\bibitem{Vihrova_9_2017}
\Au{Вихрова О.\,Г.}
К~вычислению вероятностных характеристик СМО ограниченной емкости со случайными 
требованиями к~ресурсам~// {Вестник РУДН. Серия: Математика, информатика, 
физика}, 2017. №\,3. С.~203--210.



%24
\bibitem{Sopin_17_2018}
\Au{Sopin E., Samouylov~K.}
On the analysis of the limited resources queuing system under MAP arrivals~// 
{Conference (International)  on Applied Mathematics, Computational Science and 
Systems Engineering}, 2018. Vol.~16. Art. No.\,01008. 4~p.

%23
\bibitem{Sopin_11_2017}
\Au{Sopin E., Gaidamaka~Yu., Markova~E., Vikhrova~O.}
Performance analysis of M2M traffic in LTE network using queuing systems with 
random resource requirements~// {Autom. Control Comp.~S.}, 2018 
(in press).

%25
\bibitem{Naumov_1_2014}
\Au{Наумов В.\,А., Самуйлов~К.\,Е.}
О~моделировании систем массового обслуживания с~множественными ресурсами~// 
{Вестник РУДН. Серия: Математика, информатика, физика}, 2014. №\,3. С.~60--64.

%26
\bibitem{Buzen}
\Au{Buzen J.\,P.}
Computational algorithms for closed queueing networks with exponential servers~// 
{Commun. ACM}, 1973. Vol.~16. P.~527--531.



%28
\bibitem{Vish_2016}
\Au{Вишневский В.\,М., Самуйлов~К.\,Е., Наумов~В.\,А., Яркина~Н.\,В.}
Модель соты LTE с~межмашинным трафиком в~виде мультисервисной системы массового 
обслуживания с~эластичными и~потоковыми заявками и~марковским входящим потоком~// 
{Вестник РУДН. Серия: Математика, информатика, физика}, 2016. №\,4. С.~26--36.

%27
\bibitem{Vish_2017}
\Au{Vishnevsky~V., Samouylov~K., Naumov~V., Krishnamoorty~A., Yarkina~N.}
Multiservice queieing system with map arrivals for modelling LTE cell with H2H 
and M2M communications and M2M aggregation~// {Comm. Com. 
Inf. Sc.}, 2017. Vol.~700. P.~63--74.

 \end{thebibliography}

 }
 }

\end{multicols}

\vspace*{-6pt}

\hfill{\small\textit{Поступила в~редакцию 16.06.18}}

\vspace*{6pt}

%\newpage

%\vspace*{-24pt}

\hrule

\vspace*{2pt}

\hrule

\vspace*{-2pt}


\def\tit{RESOURCE QUEUING SYSTEMS AS~MODELS OF~WIRELESS COMMUNICATION SYSTEMS}


\def\titkol{Resource queuing systems as~models of~wireless communication systems}

\def\aut{A.\,V.~Gorbunova$^1$, V.\,A.~Naumov$^2$, Yu.\,V.~Gaidamaka$^{1,3}$, and~K.\,E.~Samouylov$^{1,3}$}

\def\autkol{A.\,V.~Gorbunova, V.\,A.~Naumov, Yu.\,V.~Gaidamaka, and~K.\,E.~Samouylov}

\titel{\tit}{\aut}{\autkol}{\titkol}

\vspace*{-11pt}


\noindent
$^1$Peoples' Friendship University of Russia 
(RUDN University), 6~Miklukho-Maklaya Str., Moscow 117198, Russian\linebreak
$\hphantom{^1}$Federation

\noindent
$^2$Service Innovation Research Institute, 8A~Annankatu, Helsinki 
00120, Finland

\noindent
$^3$Institute of Informatics Problems, 
Federal Research Center ``Computer Science and Control'' 
of the Russian\linebreak
$\hphantom{^1}$Academy of Sciences, 44-2~Vavilov Str., Moscow 119333, 
Russian Federation


\def\leftfootline{\small{\textbf{\thepage}
\hfill INFORMATIKA I EE PRIMENENIYA~--- INFORMATICS AND
APPLICATIONS\ \ \ 2018\ \ \ volume~12\ \ \ issue\ 3}
}%
 \def\rightfootline{\small{INFORMATIKA I EE PRIMENENIYA~---
INFORMATICS AND APPLICATIONS\ \ \ 2018\ \ \ volume~12\ \ \ issue\ 3
\hfill \textbf{\thepage}}}

\vspace*{3pt}


 
\Abste{The article presents an overview of the resource queuing 
systems used for modeling of a~wide class of real systems with 
admittedly limited resources. Despite the objective importance of studying 
of such systems, there have been very few works devoted to their 
analysis until recently, which was due to the complexity of constructing
a~random process to describe their functioning and, accordingly, of obtaining 
the numerical results. However, in
recent years, there has been 
a~significant shift in the study of the resource systems~--- new 
methods for their analysis have been proposed, which made it possible to 
construct recursive algorithms suitable for the numerical calculations.
In this regard, the current review reflects only a part of the previously 
obtained results, namely, it considers\linebreak\vspace*{-12pt}}

\Abstend{the resource systems without waiting 
space with exponentially distributed service time. The authors consider the models 
of wireless communication systems based on resource queuing systems, expressions 
for estimating the main 
probabilistic, and temporal characteristics and algorithms for their calculation.}

\KWE{resource queueing systems; continuous resource; discrete resource; 
limited resource; recursive algorithm; heterogeneous network; 
stationary distribution; semi-Markov process; wireless communication systems}
 
\DOI{10.14357/19922264180307}

%\vspace*{-14pt}

\Ack
\noindent
The work was partly supported by the Russian Ministry of Education and
Science
(project 2.882.2017/4.6).



%\vspace*{6pt}

  \begin{multicols}{2}

\renewcommand{\bibname}{\protect\rmfamily References}
%\renewcommand{\bibname}{\large\protect\rm References}

{\small\frenchspacing
 {%\baselineskip=10.8pt
 \addcontentsline{toc}{section}{References}
 \begin{thebibliography}{99}
\bibitem{1-gor}
\Aue{Andrews, J.\,G., S.~Buzzi, W.~Choi, S.\,V.~Hanly, A.~Lozano, 
A.\,C.\,K.~Soong, and J.\,C.~Zhang.} 2014. What will 5G be? 
\textit{IEEE J.~Sel. Area. Comm.} 32(6):1065--1082.

\bibitem{3-gor}
\Aue{Buturlin, I.\,A., Y.\,V.~Gaidamaka, and A.\,K.~Samuylov.} 
2012. Utility function maximization problems for two cross-layer optimization 
algorithms in OFDM wireless networks. 
\textit{4th Congress (International) on Ultra Modern Telecommunications and 
Control Systems}. 63--65.

\bibitem{2-gor}
\Aue{Galinina, O.\,S., D.~Andreev, M.~Gerasimenko, Y.~Koucheryavy, N.~Himayat, 
S.\,P.~Yeh, and S.~Talwar.} 2014. Capturing spatial randomness of heterogeneous 
cellular/WLAN deployments with dynamic traffic. 
\textit{IEEE J.~Sel. Area. Comm.} 32(6):1083--1099.
\bibitem{4-gor}
\Aue{Naumov, V.\,A., K.\,E.~Samuilov, and A.\,K.~Samuilov.} 2016. 
On the total 
amount of resources occupied by serviced customers. 
\textit{Automat. Rem. Contr.} 77(8):1419--1427.
\bibitem{5-gor}
\Aue{Gimpelson, L.\,A.} 1968. Analysis of 
mixtures of wide- and narrow-band traffic. 
\textit{IEEE T.~Commun. Techn.} 13(3):258--266.
\bibitem{6-gor}
\Aue{Kelly, F.\,P.} 1991. Loss networks. \textit{Ann. Appl. Probab.} 1:319--378.
\bibitem{7-gor}
\Aue{Ross, K.\,W.} 1995. \textit{Multiservice loss models for broadband telecommunication 
networks}. London: Springer-Verlag. 343~p.
\bibitem{8-gor}
\Aue{Basharin, G.\,P., K.\,E.~Samouylov, N.\,V.~Yarkina, and I.\,A.~Gudkova.} 2009. 
A~new stage in mathematical teletraffic theory. 
\textit{Automat. Rem. Contr.} 70(12):1954--1964.
\bibitem{9-gor}
\Aue{Romm, E.\,L., and V.\,V.~Skitovich.} 1971. Ob odnom obobshchenii zadachi Erlanga 
[On a generalization of the Erlang problem]. 
\textit{Automat. Rem. Contr.} 6:164--168.
\bibitem{10-gor}
\Aue{Kats, B.\,A.} 1976. Ob obsluzhivanii soobshcheniy sluchaynoy dliny 
[On serving messages of random length]. 
\textit{Teoriya massovogo obsluzhivaniya. Tr. 3~Vsesoyuzn. 
shkoly-soveshchaniya po teorii massovogo obsluzhivaniya} 
[Queuing Theory: 3rd All-Union School-Seminar on Queuing Theory Proceedings]. 157--168.
\bibitem{11-gor}
\Aue{Naumov, V.\,A., and A.\,K.~Samuylov.} 
2015. Model' vydeleniya resursov besprovodnoy seti ob''emami sluchaynoy velichiny 
[Queuing system with resource allocation of the random volume]. 
\textit{RUDN J.~Math. 
Information Sci. Phys.} 2:38--45.
\bibitem{12-gor}
\Aue{Naumov, V., K.~Samouylov, N.~Yarkina, E.~Sopin, S.~Andreev, and A.~Samuylov.}
2015. LTE performance analysis using queuing systems with finite resources 
and random requirements. 
\textit{7th Congress on Ultra Modern Telecommunications and Control Systems}. 
IEEE. 100--103.
\bibitem{13-gor}
\Aue{Naumov, V., and K.~Samouylov.} 2017. Analysis оf multi-resource loss 
system with state dependent arrival and service rates. 
\textit{Probab. Eng. Inform. Sc.} 31(4):413--419.
\bibitem{14-gor}
\Aue{Samouylov, K., E.~Sopin, and O.~Vikhrova.} 2017. 
Analysis of queueing system with resources and signals. 
\textit{Comm. Com. Inf. Sc.} 800:358--369.
\bibitem{15-gor}
\Aue{Sopin, E., O.~Vikhrova, and K.~Samouylov.} 2017. 
LTE network model with signals and random resource requirement. 
\textit{9th Congress (International) on Ultra Modern Telecommunications and 
Control Systems and Workshops}. 101--106.

\bibitem{17-gor}
\Aue{Petrov, V., D.~Solomitckii, A.~Samuylov, A.~Maria Lema, M.~Gapeyenko, 
D.~Moltchanov, S.~Andreev, V.~Naumov, K.~Samouylov, M.~Dohler, and Ye.~Koucheryavy}. 
2017. Dynamic multi-connectivity performance in ultra-dense urban 
mmWave deployments. \textit{IEEE J.~Sel. Area. Comm.} 35(9):2038--2055.

\bibitem{16-gor}
\Aue{Naumov, V.\,A., and K.\,E.~Samuilov.} 2018. 
Analysis of networks of the resource queuing systems. 
\textit{Automat. Rem. Contr}. 79(5):822--829.
\bibitem{18-gor}
\Aue{Samouylov, K., E.~Sopin, and O.~Vikhrova.} 2015. 
Analyzing blocking probability in LTE wireless network via queuing system 
with finite amount of resources. 
\textit{Comm. Com. Inf. Sc.} 564:393--403.
\bibitem{19-gor}
\Aue{Vikhrova, O.\,G., K.\,E.~Samouylov, E.\,S.~Sopin, and S.\,Ya.~Shorgin.} 
2015. K~analizu pokazateley kachestva obsluzhivaniya 
v~sovremennykh besprovodnykh setyakh [On performance analysis of modern 
wireless networks]. \textit{Informatika i~ee Primeneniya~---
Inform. Appl.} 9(4):48--55.
\bibitem{20-gor}
\Aue{Sopin, E., K.~Samouylov, O.~Vikhrova, R.~Kovalchukov, D.~Moltchanov, and 
A.~Samuylov.} 2016. Evaluating a~case of downlink uplink decoupling 
using queuing system with random requirement. 
\textit{Internet of Things, smart spaces, and
next generation
networks and systems}.
Eds. O.~Galinina, S.\,I.~Balandin, and Y.~Koucheryavy.
{Lecture notes in computer science ser.} Springer. 9870:440--450.
\bibitem{21-gor}
\Aue{Samouylov, K., E.~Sopin, O.~Vikhrova, and S.~Shorgin.} 
2017. Convolution algorithm for normalization constant evaluation in 
queuing system with random requirements. \textit{AIP Conf. Proc}. 
1863:090004. 4~p.
\bibitem{22-gor}
\Aue{Vikhrova, O.\,G.} 2017. 
K~vychisleniyu veroyatnostnykh kharakteristik SMO ogranichennoy emkosti so 
sluchaynymi trebovaniyami k~resursam [About probability characteristics evaluation 
in queuing system with limited resources and random requirements]. 
\textit{RUDN J.~Math. Information Sci. Phys.} 25(3):203--210.

\bibitem{24-gor}
\Aue{Sopin, E., and K.~Samouylov.} 2018. On the analysis of the limited resources 
queuing system under MAP arrivals. 
\textit{Conference (International) 
on Applied Mathematics, Computational Science and Systems Engineering}. 16:01008. 4~p.

\bibitem{23-gor}
\Aue{Sopin, E., Yu.~Gaidamaka, E.~Markova, and O.~Vikhrova.} 2018 (in press).
 Performance analysis of M2M traffic in LTE network using queuing systems 
 with random resource requirements. 
 \textit{Autom. Control Comp.~S.}
\bibitem{25-gor}
\Aue{Naumov, V.\,A., and K.\,E.~Samuylov.} 2014. 
O~modelirovanii sistem massovogo obsluzhivaniya s~mnozhestvennymi resursami 
[On the modeling of queueing systems with multiple resources]. 
\textit{[RUDN J.~Math. Information Sci. Phys.} 3:60--64.
\bibitem{26-gor}
\Aue{Buzen, J.\,P.} 1973. Computational algorithms for closed queueing 
networks with exponential servers. \textit{Commun. ACM} 16:527--531.

\bibitem{28-gor}
\Aue{Vishnevsky, V.\,M., K.\,E.~Samouylov, V.\,A.~Naumov, and N.\,V.~Yarkina.}
2016. Model' soty LTE s~mezhmashinnym trafikom v~vide mul'tiservisnoy sistemy 
massovogo obsluzhivaniya s~elastichnymi i~potokovymi zayavkami 
i~markovskim vkhodyashchim potokom [Multiservice queuing system with
 elastic and streaming flows and markovian arrival process for modelling 
 LTE cell with M2M traffic]. 
 \textit{RUDN J.~Math. Information Sci. Phys.} 4:26--36.
 
 \bibitem{27-gor}
\Aue{Vishnevsky, V., K.~Samouylov, V.~Naumov, A.~Krishnamoorty, and N.~Yarkina.} 2017.
Multiservice queieing system with map arrivals for modelling LTE cell with H2H 
and M2M communications and M2M aggregation. 
\textit{Comm. Com. Inf. Sc.} 700:63--74.
 
 \end{thebibliography}

 }
 }

\end{multicols}

\vspace*{-6pt}

\hfill{\small\textit{Received June 16, 2018}}

%\pagebreak

%\vspace*{-18pt}

 
\Contr

\noindent
\textbf{Gorbunova Anastasiya V.} (b.\ 1986)~--- 
Candidate of Science (PhD) in physics and mathematics, 
assistant professor, Peoples' Friendship University of Russia 
(RUDN University), 6~Miklukho-Maklaya Str., 
Moscow 117198, Russian Federation; \mbox{gorbunova\_av@rudn.university}

\vspace*{3pt}

\noindent
\textbf{Naumov Valeriy A.} (b.\ 1950)~--- 
Candidate of Science (PhD) in physics and mathematics, 
Research Director, Service Innovation Research Institute, 8A~Annankatu, Helsinki 
00120, Finland; \mbox{valeriy.naumov@pfu.fi}

\vspace*{3pt}

\noindent
\textbf{Gaidamaka Yuliya V.} (b.\ 1971)~--- Doctor of Science in physics and mathematics, 
professor, Peoples' Friendship University of Russia 
(RUDN University), 6~Miklukho-Maklaya Str., 
Moscow 117198, Russian Federation;\linebreak senior scientist, 
Institute of Informatics Problems, 
Federal Research Center\ ``Computer Science and Control''\linebreak 
of the Russian Academy of Sciences, 44-2~Vavilov Str., Moscow 119333, 
Russian Federation; \mbox{gaydamaka\_yuv@rudn.university}

\vspace*{3pt}

\noindent
\textbf{Samuylov Konstantin E.} (b.\ 1955)~--- Doctor of Science in technology, 
professor, Head of Department, Peoples' Friendship University of Russia 
(RUDN University), 6~Miklukho-Maklaya Str., 
Moscow 117198, Russian Federation; senior scientist, 
Institute of Informatics Problems, Federal Research Center 
``Computer Science and Control'' of the Russian Academy of Sciences, 
44-2~Vavilov Str., Moscow 119333, Russian Federation; 
\mbox{samuylov\_ke@rudn.university}

\label{end\stat}

\renewcommand{\bibname}{\protect\rm Литература}        %14
\def\stat{gudkova}

\def\tit{ВЕРОЯТНОСТНАЯ МОДЕЛЬ СОВМЕСТНОГО ИСПОЛЬЗОВАНИЯ РЕСУРСОВ 
БЕСПРОВОДНОЙ СЕТИ С~АДАПТИВНЫМ УПРАВЛЕНИЕМ МОЩНОСТЬЮ$^*$}

\def\titkol{Вероятностная модель совместного использования ресурсов 
беспроводной сети с~адаптивным управлением} % мощностью}

\def\aut{И.\,А.~Гудкова$^1$, С.\,Я.~Шоргин$^2$}

\def\autkol{И.\,А.~Гудкова, С.\,Я.~Шоргин}

\titel{\tit}{\aut}{\autkol}{\titkol}

\index{Гудкова И.\,А.}
\index{Шоргин С.\,Я.}
\index{Gudkova I.\,A.}
\index{Shorgin S.\,Ya.}


{\renewcommand{\thefootnote}{\fnsymbol{footnote}} \footnotetext[1]
{Исследование выполнено при финансовой поддержке Российского научного фонда (проект 16-11-10227).}}


\renewcommand{\thefootnote}{\arabic{footnote}}
\footnotetext[1]{Российский университет дружбы народов;
 Институт проблем информатики Федерального исследовательского 
центра <<Информатика и~управление>> Российской академии наук, \mbox{gudkova\_ia@rudn.university}}

\footnotetext[2]{Институт проблем информатики Федерального исследовательского центра <<Информатика 
и~управление>> Российской академии наук, \mbox{sshorgin@ipiran.ru}}

\vspace*{18pt}



\Abst{Развивающиеся беспроводные сети последующего поколения 
(next generation 
network, NGN)
предполагают новые 
приложения и~услуги как для обычных пользователей, так и~для устройств межмашинного 
взаимодействия (machine-to-machine, M2M). Решение проблемы увеличения требований 
к~пропускной способности сети и~недостаточности спектра радиочастот, в~частности 
в~случае умных городов, может быть достигнуто посредством концепции совместного 
использования радиочастот (licensed shared access, LSA). Авторы предлагают 
математическую модель совместного использования ресурсов с~адаптивным управ\-ле\-ни\-ем 
мощностью. Заложенный в~ней алгоритм позволит избежать интерференции M2M-устройств с~владельцем спектра, в~том числе благодаря тому, что учитывает пространственное 
расположение устройств и~их сессионную активность.}
 
\KW{беспроводная сеть; умный город; межмашинное взаимодействие; совместное 
использование радиочастот; адаптивное управление мощностью; случайный процесс; 
рекуррентный алгоритм; вероятность блокировки; вероятность прерывания обслуживания; 
среднее число устройств}

\DOI{10.14357/19922264170310} 

\vspace*{6pt}


\vskip 10pt plus 9pt minus 6pt

\thispagestyle{headings}

\begin{multicols}{2}

\label{st\stat}

\section{Введение}

  Согласно прогнозам развития сетей по\-сле\-ду\-юще\-го поколения, 
  уже в~2025~г.\ беспроводные сети будут перегружены~[1], что 
повлечет за собой необходимость уточнения и~разработки новых стратегий 
использования спектра радиочастот~[2]. Широкое распространение получают 
автономно функционирующие и~взаимодействующие друг с~другом  
(М2М) недорогие устройства, являющиеся неотъемлемой 
частью <<умных городов>> (smart city). Особенностью M2M-устройств 
является их дистанционное управ\-ле\-ние и~высокая плот\-ность расположения. 
Рост числа M2M-устройств существенно сказывается на использовании спектра 
ра\-дио\-час\-тот ввиду того, что сети изначально разрабатывались для 
взаимодействия между людьми (human-to-human, H2H). 
{\looseness=1

}

Один из вариантов 
решения проб\-ле\-мы~--- это динамическое управление спектром в~рамках 
концепции совместного использования радиочастот 
(LSA)~[3--5]. Доступ к~спектру получают две стороны~--- владелец и~временный 
пользователь~[6, 7]. 

  В статье исследуется один из сценариев применения системы LSA~[8--11], 
где владелец запрашивает спектр радиочастот изредка на непродолжительное 
время. В~остальное же время спектр доступен M2M-устрой\-ст\-вам для 
передачи данных. Статья имеет следующую структуру. 

В~разд.~2 описана 
системная модель совместного использования радиоресурсов с~учетом 
расположения устройств на разном расстоянии от базовой станции~[12--15]. 

В~разд.~3 проводится построение математической модели в~виде двух 
случайных процессов (СП), один из которых фиксирует уровень качества 
канала каж\-до\-го из активных устройств, а~второй, укрупненный,~--- только 
суммарное число устройств. Для СП с~укрупненными со\-сто\-яни\-ями представлен 
рекуррентный алгоритм расчета стационарного распределения вероятностей. 

В~разд.~4 предложены формулы для расчета ключевых показателей 
эффективности системы~--- среднего числа устройств и~вероятностей 
блокировки и~прерывания обслуживания, приведен пример численного анализа.

\section{Системная модель совместного использования ресурсов 
с~разноудаленными от базовой станции устройствами}

Рассмотрим одну соту беспроводной сети радиуса~$R$ с~равномерно 
распределенными по зоне покрытия M2M-устройствами (рис.~1). Устройства 
с~интенсивностью~$\lambda$ переходят в~активное состояние и~передают 
данные в~восходящем канале. Время передачи данных одним устройством 
распределено экспоненциально с~параметром~$\mu$. Каждому устройству 
в~зависимости от дальности расположения от базовой станции (БС) 
присваивается один из пятнадцати уровней качества канала (channel quality 
indicator, CQI)~--- $c\hm=1,\ldots ,15$, причем чем больше~$c$, тем ближе 
устройство к~БС и~выше скорость передачи данных. Объединим устройства 
с~одинаковыми уровнями CQI в~логические группы, тогда скорость передачи 
для всех устройств в~группе будет одинаковая. Далее под расстоянием от 
устройства до БС будем понимать максимально возможное расстояние, на 
котором может быть расположено устройство с~таким же уровнем CQI. Введем 
дополнительное обозначение: $\eta\hm=16-c$; уровень CQI~$c$, 
величина~$\eta$ и~расстояние~$\xi_d(\eta)=RL^{-1}\eta$ от устройства до БС 
являются случайными величинами (СВ). Плотность расстояния от устройства 
до БС $$
f_{\xi_d(\eta)}(d)=\fr{2d}{R^2}\,,
$$
 а~функция распределения (ФР)
$$
F_{\xi_d(\eta)}(d)=\left(\fr{d}{R}\right)^2,\enskip 0\leq d\leq R\,.
$$


 { \begin{center}  %fig1
 \vspace*{5pt}
 \mbox{%
\epsfxsize=72mm %.723mm
\epsfbox{gud-1.eps}
}


\vspace*{4pt}


\noindent
{{\figurename~1}\ \ \small{Пример расположения M2M-устройств в~соте}}
\end{center}
}


\addtocounter{figure}{1}

 \noindent
 Ряд распределения для 
параметра~$\eta$:  
$$
q_l=\fr{2L-2l-1}{L^2}\,,\enskip l=1,\ldots ,L\,.
$$

  
  В качестве примера реализации системы LSA рассмотрим случай 
использования владельцем спектра радиочастот для воздушной телеметрии. 
Предположим, что время, в~течение которого владелец (аэропорт) не 
использует спектр, т.\,е.\ время, когда полоса доступна для M2M-устройств, 
и~время пролета самолета над сотой, т.\,е.\ время, когда полоса недоступна для 
устройств, распределены по экспоненциальному закону с~параметрами~$\alpha$ 
и~$\beta$ соответственно.
  
  Управление радиоресурсами предполагает разделение ресурсов по времени, 
т.\,е.\ деление ширины полосы радиочастот~$\omega$ не происходит, 
а~передача данных осуществляется на постоянной мощности. Если полоса не 
требуется аэропорту, то мощность составляет~$p_1^{\max}$, в~противном 
случае для регулирования интерференции мощность снижается до значения 
$p_0^{\max}\hm<p_1^{\max}$. Такое динамическое изменение мощности 
приводит к~изменению достижимой скорости передачи данных $r\left( 
\xi_{d(\eta)},p_s^{\max}\right)$, $s\hm=0,1$, зависящей также от расстояния 
между устройством и~БС. Согласно формуле Шеннона,
  \begin{multline}
  r\left( \xi_{d(\eta)}, p_s^{\max}\right) =\omega \ln\left( 
1+\fr{Gp_s^{\max}}{((R/L)\eta)^\kappa N_0}\right)\,,\\ s=0,1\,,\enskip 
\eta=1,\ldots ,15\,,
  \label{e1-gud}
  \end{multline}
где $N_0$~---  уровень шума; $G$~--- константа затухания сигнала; $\kappa$~--- 
экспонента затухания сигнала.

  Скорость передачи данных каждым активным M2M-устройством не может 
быть ниже порогового (гарантированного) значения~$r_0$. Если устройству не 
может быть обеспечена скорость~$r_0$, то запрос на передачу данных будет 
заблокирован. Если устройство расположено в~непосредственной близости от 
БС, то скорость передачи согласно формуле Шеннона стремится 
к~бесконечности, поэтому определим минимальное расстояние до БС 
$\xi_d(1)\hm=d_0$, ограничив тем самым максимальную скорость передачи 
данных $r_s^{\max}\hm= r\left( d_0, p_s^{\max}\right)$. Таким образом, если 
$\eta\hm=1$, то достижимая скорость передачи данных $r\left( \xi_{d(\eta)}; 
p_s^{\max}\right)$, если $\eta\hm=2,\ldots ,L$, то она вычисляется по 
формуле~(\ref{e1-gud}). Максимальное число устройств в~соте:
$$
K_s= \left \lfloor  
\fr{r( d_0, p_s^{\max})}{r_0}\right\rfloor\,,\enskip s=0,1
\,.
$$

  Сводный перечень основных обозначений приведен в~табл.~1.

\end{multicols}

  \begin{table*}\small
  \begin{center}
  \Caption{Основные обозначения}
  \vspace*{2ex}
  
  \begin{tabular}{|l|p{340pt}|}
  \hline
  \multicolumn{1}{|c|}{Обозначение}&\multicolumn{1}{c|}{Описание}\\
  \hline
  $R$&Радиус соты, м\\
  $\omega$&Ширина полосы радиочастот, МГц\\
  $L$&Число уровней качества канала CQI \\
  $c$&Уровень CQI (СВ)\\
  $\eta=16-c$&Величина, обратная уровню CQI~$c$  (СВ)\\
  $q_l=\fr{2L-2l-1}{L^2}$&Вероятность того, что уровень CQI равен~$l$\\
  $\alpha^{-1}$&Среднее время доступности полосы, с\\
  $\beta^{-1}$&Среднее время недоступности полосы, с\\
  $k$&Число активных устройств\\
  $s$&Состояние полосы: $s=1$, если полоса доступна; $s=0$, если недоступна\\
  $p_0^{\max}$&Максимальное значение мощности сигнала устройства, если полоса 
недоступна, Вт\\
  $p_1^{\max}$&Максимальное значение мощности сигнала устройства, если полоса 
доступна, Вт\\
  $d_0$&Минимальное расстояние от устройства до БС, м\\
  $r\left( \xi_{d(\eta)}, p_s^{\max}\right)$ &Достижимая скорость передачи 
  для устройства с~уровнем CQI $c\hm=16\hm-\eta$, если полоса находится в~состоянии~$s$ (СВ), бит/с\\
  $r_0^{\max}$&Максимально возможная скорость, если полоса недоступна, бит/с\\
  $r_0$&Гарантированная скорость передачи данных от устройств, бит/с\\
  $r_1^{\max}$&Максимально возможная скорость, если полоса доступна, бит/с\\
  $K_0$&Максимальное число устройств, если полоса недоступна\\
  $K_1$&Максимальное число устройств, если полоса доступна\\
  $\xi_d(\eta)$ &Максимальное расстояние от устройства с~уровнем CQI $c\hm=16\hm-\eta$ 
до БС (СВ), м\\
  $\lambda$&Интенсивность суммарного потока данных от всех устройств в~соте, 1/с\\
  $\mu^{-1}$&Среднее время передачи данных от одного устройства, с\\
  $\rho=\fr{\lambda}{\mu}$&Суммарная предложенная нагрузка от всех устройств в~соте, Эрл\\[-9pt]
&\\
  \hline
  \end{tabular}
  \end{center}
  \end{table*}
  
  \begin{multicols}{2}
  

  
\section{Вероятностная модель и~стационарное распределение 
вероятностей состояний беспроводной сети}

  Перейдем к~построению математической модели. Пусть $\xi(t)$~--- число 
активных M2M-устройств; $\eta_i(t)$~--- значение параметра~$\eta$ для 
устройства~$i$; $\zeta(t)$~--- состояние полосы в~момент времени $t\hm\geq 0$. 
Тогда функционирование соты опишем СП $\left\{ \xi(t),\eta_1(t),\ldots 
,\eta_{\xi(t)},\zeta(t), t\geq0\right\}$ над пространством состояний
  \begin{multline*}
  \mathbf{L}\hm= 
  \left\{ 
    \vphantom{\sum\limits_{i=1}^k}
    (0,s), \left(k,l_1,\ldots ,l_k, s\right), \right.\\ 
  s=0,1,\ l_i=1,\ldots 
,L,\ i=1,\ldots , k,\ k=1,2,\ldots: \\
\left.  \sum\limits_{i=1}^k \fr{r_0}{\omega \ln \left( 1+Gp_s^{\max}/((RL^{-
1}l_i)^\kappa N_0)\right)}\leq 1\right\}\,.
  %\label{e2-gud}
  \end{multline*}

Фрагмент пространства состояний показан на рис.~2.
    

  Перейдем к~СП $\{\xi(t),\zeta(t), t\geq0\}$ с~укрупненными состояниями~--- 
суммарным числом устройств и~состоянием полосы. Пространство состояний 
такого процесса будет иметь вид:
  \begin{equation*}
  \mathbf{L}_1 =\left\{ (k,s):\ k=0,1,\ldots ,K_s,\ s=0,1\right\}\,.
  %\label{e3-gud}
  \end{equation*}
Отметим, что при переходе полосы в~недоступное состояние происходит 
снижение мощности передачи данных с~$p_1^{\max}$ до~$p_0^{\max}$ 
и~прерывание обслуживания $k\hm-K_0$ устройств при условии, что число 
устройств $k\hm>K_0$. При переходе из недоступного в~доступное состояние 
мощность снова повышается. На рис.~3 представлена диаграмма 
интенсивностей переходов данного СП.
    
    

  Обозначим через $P_s(k)$, $s\hm=0,1$, условную ве\-ро\-ят\-ность того, что 
$(k+1)$-е M2M-устройство может быть обслужено при условии, что 
активно~$k$~устройств. Можно показать, что ве\-ро\-ят\-но\-сти~$P_s(k)$ 
вы\-чис\-ля\-ют\-ся по формулам:

\noindent
  \begin{align*}
  P_s(0) &={}\\
  &\hspace*{-8mm}{}=F_{\xi_d(\eta)} \left( \min 
  \left\{ R, \left( \fr{Gp_s^{\max}}{\left( 
e^{r_0/\omega} -1\right) N_0}\right)^{1/\kappa}\right\}\right)\,,\\
&\hspace*{50mm}s=0,1\,;
\end{align*}

\end{multicols}

\begin{figure*} %fig2
 \vspace*{1pt}
\begin{center}
\mbox{%
\epsfxsize=107.234mm
\epsfbox{gud-2.eps}
}
\end{center}
\vspace*{-11pt}
\Caption{Фрагмент диаграммы интенсивностей переходов СП с~детальными состояниями}
%\vspace*{-20pt}
\end{figure*}


\begin{multicols}{2}

\noindent
\begin{align*}
  P_s(k) &= \fr{\Phi\left( (1-m_{k+1,s})/\tau_{k+1,s}\right)}{\Phi\left((1-
m_{ks})/\tau_{ks}\right)}\,,\\
& \hspace*{20mm}k=1,\ldots, K_s,\enskip s=0,1\,,
  \end{align*}
где
\begin{gather*}
\Phi(x) =\fr{1}{\sqrt{2\pi}}\int\limits^x_{-\infty} e^{-t^2/2}\,dt\,;\\
m_{ks} =  kr_0E \left[ \fr{1}{r\left(d,p_s^{\max}\right)}\right]\,;
\end{gather*}

\vspace*{-12pt}

\noindent
\begin{multline*}
\tau^2_{ks}=kr_0^2\left( E\left[ \left( 
\fr{1}{r\left( d,p_s^{\max}\right)}\right)^2\right]- {}\right.\\
\left.{}-
\left( E\left[ \fr{1}{r\left(d, p_s^{\max}\right)}\right]\right)^2\right)\,;
\end{multline*}

\vspace*{-12pt}

\noindent
\begin{multline*}
E\left[ \fr{1}{r\left( d,p_s^{\max}\right)}\right] = \fr{1}{r_s^{\max}}\, 
F_{\xi_d(\eta)} (d_0) +{}\\
{}+
\int\limits_{d_0}^R \fr{1}{\omega \ln (1+Gp_s^{\max}/ 
(x^\kappa N_0))}\, f_{\xi_d(\eta)} (x)\,dx\,;
\end{multline*}

\vspace*{-14pt}

\noindent
\begin{multline*}
E\left[ \left(\fr{1}{r\left( d,p_s^{\max}\right)}\right)^2\right] = \left( \fr{1} 
{r_s^{\max}}\right)^2 F_{\xi_d(\eta)}(d_0) + {}\\
{}+\int\limits^R_{d_0} \fr{1} {\omega^2 
\ln^2 (1+Gp_s^{\max}/(x^\kappa N_0))}\,f_{\xi_d(\eta)} (x)\,dx\,.
\end{multline*}

\begin{figure*} %fig3
     \vspace*{1pt}
\begin{center}
\mbox{%
\epsfxsize=146.037mm
\epsfbox{gud-3.eps}
}
\end{center}
\vspace*{-9pt}
\Caption{Диаграмма интенсивностей переходов СП с~укрупненными состояниями}
\end{figure*}
  
  Случайный процесс $\{ \xi(t),\zeta(t), t\geq0\}$ является марковским, и~для расчета его 
стационарного рас-\linebreak\vspace*{-12pt}

\pagebreak

\noindent
пределения вероятностей~$p(k,s)$, $(k,s)\hm\in 
\mathbf{L}_1$ предлагается следующий рекуррентный алгоритм.
  \begin{enumerate}[1.]
  \item Значения ненормированных вероятностей $q(k,s)$ вычисляются по 
формулам:
  \begin{align*}
  q(0,0)&=1\,;\\
  q(0,1) &=x\,;\\
  q(k,s) & =\delta_{ks}+\gamma_{ks} x\,,\ (k,s)\in \mathbf{L}_1:\ k>0\,,
  \end{align*}
где
$$
x=\fr{(K_1\mu +\alpha)\delta_{K_11}-\lambda P_1(K_1-1)\delta_{K_1-1{,}1}} 
{\lambda P_1(K_1-1)\gamma_{K_1-1,1}-(K_1\mu +\alpha)\gamma_{K_11}}\,.
$$
\item Коэффициенты $\delta_{ks}$ и~$\gamma_{ks}$ вычисляются по 
рекуррентным формулам:
\begin{gather*}
\delta_{00}=1\,,\ \gamma_{00}=0\,;\\
\delta_{01}=0\,,\ \gamma_{01}=1\,;\\
\delta_{10}=\fr{\lambda P_0(0)+\beta}{\mu}\,,\ \gamma_{10}=-
\fr{\alpha}{\mu}\,;\\
\delta_{11}=-\fr{\beta}{\mu}\,,\ \gamma_{11}=\fr{\lambda 
P_1(0)+\alpha}{\mu}\,;
\end{gather*}

\vspace*{-12pt}

\noindent
\begin{multline*}
\delta_{k0}=\fr{\lambda P_0(k-1)+(k-1)\mu+\beta}{k\mu}\,\delta_{k-1,0} - {}\\
{}-
\fr{\lambda P_0(k-2)}{k\mu}\,\delta_{k-2,0}-\fr{\alpha}{k\mu}\,\delta_{k-1,1}\,,\ 
k=2,\ldots ,K_0,\hspace*{-0.26485pt}
\end{multline*}

\vspace*{-12pt}

\begin{multline*}
\gamma_{k0} = \fr{\lambda P_0(k-1)+(k-1)\mu+\beta}{k\mu}\,\gamma_{k-1,0}-{}\\
{}-
\fr{\lambda P_0(k-2)}{k\mu}\,\gamma_{k-2,0} -\fr{\alpha}{k\mu}\,\gamma_{k-
1,1}\,,\ k=2,\ldots,K_0;\hspace*{-1.73058pt}
\end{multline*}

%\vspace*{-12pt}

\noindent
\begin{gather*}
\delta_{k1} = \fr{\lambda P_1(k-1)+(k-1)\mu+\alpha}{k\mu}\,\delta_{k-1,1} - {}\\
{}-
\fr{\lambda P_1(k-2)}{k\mu}\,\delta_{k-2,1} -\fr{\beta}{k\mu}\,\delta_{k-1,0}\,,\\ 
k=2,\ldots ,K_0+1\,;\\
\hspace*{-3mm}\gamma_{k1} =\fr{\lambda P_1(k-1)+(k-1)\mu+\alpha}{k\mu}\,\gamma_{k-1,1} - {}\\
{}-
\fr{\lambda P_1(k-2)}{k\mu}\,\gamma_{k-2,1}- \fr{\beta}{k\mu}\,\gamma_{k-
1,0}\,,\\ 
k=2,\ldots ,K_0+1\,;\\
\delta_{k1} = \fr{\lambda P_1(k-1)+(k-1)\mu+\alpha}{k\mu}\,\delta_{k-1,1}- {}\\
{}-
\fr{\lambda P_1(k-2)}{k\mu}\,\delta_{k-2,1}\,,\   k=K_0+2,\ldots , K_1\,,\\
\gamma_{k1} =\fr{\lambda P_1(k-1)+(k-1)\mu+\alpha}{k\mu}\,\gamma_{k-1,1} - {}\\
{}-
\fr{\lambda P_1(k-2)}{k\mu}\,\gamma_{k-2,1}\,,\ k=K_0+2,\ldots ,K_1\,.
\end{gather*}

\item Значения вероятностей $p(k,s)$ вычисляются по формулам:

\noindent
$$
p(k,s)=\fr{q(k,s)}{\sum\nolimits_{(i,j)\in \mathbf{L}} q(i,j)}\,,\ (k,s)\in 
\mathbf{L}_1\,.
$$
\end{enumerate}

\begin{table*}\small
  \begin{center}
  \Caption{Исходные данные для численного анализа}
  \vspace*{2ex}
  
  \begin{tabular}{|l|c|c|c|}
  \hline
\multicolumn{1}{|c|}{Обозначение}&Случай 1  
(рис.~4)&Случай 2  
(рис.~5)&Случай 3  
(рис.~6)\\
\hline
$R$, м&200--400&200; 400&200; 400\\
$\omega$, МГц&10&10&10\\
$L$&15&15&15\\
$\alpha^{-1}$, мин&20; 30&20; 30&30\\
$\beta^{-1}$, с&20&20&20\\
$p_1^{\max}$, дБ$\cdot$м&23; 42&23--42&23; 42\\
$p_0^{\max}$, дБ$\cdot$м&$p_{\max}/2$&$p_{\max}/2$&$p_{\max}/2$\\
$d_0$, м&$R/15$&$R/15$&$R/15$\\
$r_0$, Мбит/с&1&1&1\\
$\lambda$, 1/с&10&10&2--10\\
$\mu^{-1}$, с&0,1&0,1&0,1\\
$N_0$, дБ$\cdot$м&$-60$&$-60$&$-60$\\
$G$&197,43&197,43&197,43\\
$\kappa$&5&5&5\\
\hline
\end{tabular}
\end{center}
%\vspace*{-6pt}
\end{table*}
\begin{figure*}[b] %fig4
% \vspace*{-6pt}
\begin{center}
\mbox{%
\epsfxsize=162.099mm
\epsfbox{gud-4.eps}
}
\end{center}
\vspace*{-9pt}
\Caption{Показатели эффективности в~зависимости от мощности устройств: 
(\textit{а})~вероятность прерывания обслуживания при 
$R=400$ (\textit{1}~--- $\alpha\hm=1200$; 
\textit{2}~--- $\alpha\hm=1800$); 
(\textit{б})~среднее число активных устройств 
при $\alpha\hm=1800$ (\textit{3}~--- $R\hm=200$; 
\textit{4}~---  $R\hm=400$)}
\end{figure*}

\vspace*{-9pt}

\section{Пример численного анализа и~заключение}

\vspace*{-3pt}

  Зная стационарное распределение вероятностей  $p(k,s)$, 
  $(k,s)\hm\in \mathbf{L}_1$, найдем 
основные показатели эффективности модели~--- вероятность~$B$ блокировки, 
вероятность~$\Pi$ прерывания обслуживания и~среднее число~$\overline{K}$ 
устройств по формулам:

\noindent
  \begin{align*}
  B &= \sum\limits_{k=0}^{K_0-1} \left( 1-P_0(k)\right) p(k,0) 
+{}\notag\\
&\hspace*{30mm}{}+\sum\limits_{k=0}^{K_1-1}\left( 1-P_1(k)\right) p(k,1)\,;
  %\label{e4-gud}
  \\
  \Pi &=  \hspace*{-2mm}\hspace*{-0.72604pt}\sum\limits_{k=K_0+1}^{K_1-1}
  \hspace*{-1mm}
   \fr{\alpha}{\alpha+k\mu+\lambda 
P_1(k)}\, \fr{\begin{pmatrix} k-1\\ k-K_0-1\end{pmatrix}}{ \begin{pmatrix} k\\ k-
K_0\end{pmatrix}}\,p(k,1)+{}\notag\\
&\hspace*{10mm}{}+\fr{\alpha}{\alpha+K_1\mu}\,\fr{\begin{pmatrix} K_1-
1\\ K_1-K_0-1\end{pmatrix}} {\begin{pmatrix} K_1\\ K_1-K_0\end{pmatrix}} 
\,p(K_1,1)\,;
  %\label{e5-gud}
  \\
  \overline{K} &= \sum\limits_{k=0}^{K_0} kp (k,0) +\sum\limits_{k=0}^{K_1} 
kp(k,1)\,.
  %\label{e6-gud}
  \end{align*}
  
  Для проведения численного анализа проанализируем передачу данных  
M2M-устройствами небольшими сессиями, составляющими в~среднем~10~с, 
в~высоком качестве на ско\-рости~1~Мбит/с. 
Рассмотрим небольшой аэропорт, 
в~котором самолеты взлетают раз в~20~(30)~мин, среднее время пролета 
самолета над сотой составляет~20~с. 
Исходные данные представлены в~табл.~2.
  
  
  
  На рис.~4 показана зависимость вероятности прерывания обслуживания 
и~среднего числа активных устройств от их мощности. Вероятность\linebreak 
прерывания обслуживания уменьшается пропорционально увеличению 
мощности, так как при более высокой мощности для устройств достижима 
более высокая скорость. При этом вероятность прерывания ниже для более 
низкой интен\-сив\-ности отключения полосы.
Среднее число устройств 
увеличивается пропорционально радиусу соты
 (см.\linebreak

\end{multicols}

\begin{figure*} %fig5
 \vspace*{1pt}
\begin{center}
\mbox{%
\epsfxsize=161.797mm
\epsfbox{gud-5.eps}
}
\end{center}
\vspace*{-9pt}
\Caption{Показатели эффективности в~зависимости от радиуса соты: 
(\textit{а})~вероятность 
прерывания обслуживания при $W\hm=0{,}2$
(\textit{1}~--- $\alpha\hm=1200$;
\textit{2}~--- $\alpha\hm=1800$); (\textit{б})~среднее число активных 
устройств при $\alpha\hm=1200$ (\textit{3}~--- $W\hm=0{,}2$;
\textit{4}~--- $W=15{,}85$)}
%\end{figure*}
%\begin{figure*} %fig6
 \vspace*{6pt}
\begin{center}
\mbox{%
\epsfxsize=159.48mm
\epsfbox{gud-6.eps}
}
\end{center}
\vspace*{-9pt}
\Caption{Показатели эффективности в~зависимости от интенсивности потока пакетов 
данных от устройств: (\textit{а})~вероятность прерывания обслуживания при 
$R=400$ (\textit{1}~--- $\alpha\hm=1200$; 
\textit{2}~---  $\alpha\hm=1800$); (\textit{б})~среднее 
число активных устройств (\textit{3}~---  $R\hm=400$, $\alpha\hm=1200$;
\textit{4}~--- $R\hm=200$, $\alpha\hm=1800$)}
\vspace*{-30pt}
\end{figure*}

\begin{multicols}{2}

\noindent
 рис.~5). Вероятность 
прерывания оказывается ниже при меньшей интенсивности изъятия полосы 
(см.\ рис.~6). 





  В заключение отметим, что в~статье разработана вероятностная модель 
совместного использования радиочастот, при помощи которой проведен анализ 
показателей эффективности применения политики управления мощ\-ностью 
с~учетом разноудаленных от БС M2M-устройств. 

В~дальнейшем 
предполагается учесть случайную высоту, на которой могут находиться 
устройства.

\vspace*{-12pt}

{\small\frenchspacing
 { %\baselineskip=10pt
 \addcontentsline{toc}{section}{References}
 \begin{thebibliography}{99}
\bibitem{1-gud}
Cisco Visual Networking Index: Global Mobile Data Traffic Forecast Update, 2016--2021 White 
Paper. March~28, 2017. {\sf  
http://www.cisco.com/c/en/us/ solutions/collateral/service-provider/visual-networking-index-vni/mobile-white-paper-c11-520862.html}.
\bibitem{2-gud}
\Au{Andrews J., Buzzi~S., Choi~W., Hanly~S.\,V., Lozano~A., Soong~C.\,K., Zhang~J.\,C.} What 
will 5G be?~// IEEE J.~Sel. Area. Comm., 2014. Vol.~32. P.~1065--1082. 

\bibitem{5-gud} %3
ETSI TR 103 113. Electromagnetic compatibility and Radio spectrum Matters 
(ERM); System Reference document (SRdoc); Mobile broadband services in the  
2\,300~MHz\,--\,2\,400~MHz frequency band under Licensed Shared Access regime. 
v1.1.1. July 2013. 
{\sf 
http:// www.etsi.org/deliver/etsi\_tr/103100\_103199/103113/ 01.01.01\_60/tr\_103113v010101p.pdf}.

\bibitem{3-gud} %4
ETSI TR 103 154. Reconfigurable Radio Systems (RRS); System requirements 
for operation of Mobile Broadband Systems in the 2\,300~MHz\,--\,2\,400~MHz band under Licensed 
Shared Access (LSA). v1.1.1. October 2014.
{\sf 
http://www.etsi.org/deliver/etsi\_TS/103100\_103199/ 103154/01.01.01\_60/ts\_103154v010101p.pdf}.

\bibitem{4-gud} %5
ETSI TR 103 235. Reconfigurable Radio Systems (RRS); System architecture 
and high level procedures for operation of Licensed Shared Access (LSA) in  
the 2\,300~MHz\,--\,2\,400~MHz band. v1.1.1. October 2015. {\sf 
http://www.\linebreak
 etsi.org/deliver/etsi\_ts\%5C103200\_103299\%5C103235
 \%5C01.01.01\_60\%5Cts\_103235v010101p.pdf}.

\bibitem{6-gud} %6
\Au{Buckwitz K., Engelberg J., Rausch~G.} Licensed Shared Access (LSA)~--- regulatory 
background and view of Administrations~// 9th Conference (International) on Cognitive Radio 
Oriented Wireless Networks.~--- IEEE, 2014. P.~413--416.
\bibitem{7-gud}
\Au{Ahokangas P., Matinmikko~M., Yrj$\ddot{\mbox{o}}$l$\ddot{\mbox{a}}$~S., 
Mustonen~M., 
Posti~H., Luttinen~E., Kivim$\ddot{\mbox{a}}$ki~A.} Business models for mobile network 
operators in Licensed Shared Access (LSA)~// IEEE Symposium (International) on Dynamic 
Spectrum Access Networks.~--- IEEE, 2014. P.~263--270.


\bibitem{9-gud} %8
\Au{Borodakiy~V.\,Y., Samouylov~K.\,E., Gudkova~I.\,A., Ostrikova~D.\,Y., Ponomarenko~A.\,A., 
Turlikov~A.\,M., Andreev~S.\,D.} Modeling unreliable LSA operation in 3GPP LTE cellular 
networks~// 6th Congress (International) on Ultra Modern Telecommunications and Control 
Systems and Workshops Proceedings.~--- Piscataway, NJ, USA: IEEE, 2015. 
P.~490--496.

\bibitem{8-gud} %9
\Au{Ponomarenko-Timofeev A., Pyattaev~A., Andreev~S., Kou\-che\-rya\-vy~Ye., Mueck~M., Karls~I}. 
Highly dynamic spectrum management within licensed shared access regulatory framework~// 
IEEE Commun. Mag., 2015. Vol.~54. No.\,3. P.~100--109.

\bibitem{10-gud}
\Au{Gudkova I.\,A., Samouylov~K.\,E., Ostrikova~D.\,Y., Mokrov~E.\,V.,  
Ponomarenko-Timofeev~A.\,A., Andreev~S.\,D., Koucheryavy~Y.\,A.} Service failure and 
interruption probability analysis for Licensed Shared Access regulatory framework~// 7th Congress 
(International) on Ultra Modern Telecommunications and Control Systems and Workshops
 Proceedings.~--- Piscataway, NJ, USA: IEEE Computer Society, 2015. P.~123--131.

\bibitem{11-gud}
\Au{Samouylov K., Gudkova~I., Markova~E., Yarkina~N.} Queuing model with unreliable servers 
for limit power policy within Licensed Shared Access framework~// Internet of things, smart
spaces, and next generation networks and systems~/
Eds. O.~Galinina, S.~Balankin, Y.~Koucheryavy.~---
Lecture notes in computer 
science ser.~--- Springer, 2016. Vol.~9870. P.~404--413.
\bibitem{12-gud}
\Au{Galinina O., Andreev~S.\,D., Gerasimenko~M., Kou\-che\-rya\-vy~Y.\,A., Himayat~N., Yeh S.-P., 
Talwar~S.} Capturing spatial randomness of heterogeneous cellular/WLAN deployments with 
dynamic traffic~// IEEE J.~Sel. Area. Comm., 2014. Vol.~32.  
No.\,6. P.~1083--1099.
\bibitem{13-gud}
\Au{Ahmadian A., Galinina~O., Gudkova~I., Andreev~S., Shorgin~S., Samouylov~K.} On capturing 
spatial diversity of joint M2M/H2H dynamic uplink transmissions in 
3GPP LTE cellular system~// Internet of things, smart
spaces, and next generation networks and systems~/
Eds.\ S.~Balandin, S.~Andreev, Y.~Koucheryavy.~---
Lecture notes in computer science ser.~--- Springer, 2014. Vol.~9247. P.~407--421.
\bibitem{14-gud}
\Au{Samouylov K., Gudkova~I., Markova~E., Dzantiev~I.} On analyzing the blocking probability 
of M2M transmissions for a CQI-based RRM scheme model in 3GPP LTE~// Comm. 
Com. Inf. Sci., 2016. Vol.~638. P.~327--340.
\bibitem{15-gud}
\Au{Gudkova I., Markova~E., Masek~P., Andreev~S., Hosek~J., Yarkina~N., Samouylov~K., 
Koucheryavy~Y.} Modeling the utilization of a multi-tenant band in 3GPP LTE system with 
Licensed Shared Access~// 8th Congress (International) on Ultra Modern Telecommunications and 
Control Systems and Workshops Proceedings.~--- Piscataway, NJ, USA: IEEE, 
2016. P.~179--183.
 \end{thebibliography}

 }
 }

\end{multicols}

\vspace*{-6pt}

\hfill{\small\textit{Поступила в~редакцию 20.04.17}}

\vspace*{6pt}

%\newpage

%\vspace*{-24pt}

\hrule

\vspace*{2pt}

\hrule

\vspace*{-4pt}


\def\tit{PROBABILITY MODEL FOR ANALYZING LICENSED SHARED~ACCESS WITH~ADAPTIVE 
POWER CONTROL IN~A~WIRELESS~NETWORK}

\def\titkol{Probability model for analyzing licensed shared access with adaptive 
power control in a wireless network}

\def\aut{I.\,A.~Gudkova$^{1,2}$ and~S.\,Ya.~Shorgin$^2$}

\def\autkol{I.\,A.~Gudkova and  S.\,Ya.~Shorgin}

\titel{\tit}{\aut}{\autkol}{\titkol}

\vspace*{-9pt}


\noindent
$^1$Peoples' Friendship University of Russia, 6~Miklukho-Maklaya Str., Moscow 117198, Russian Federation

\noindent
$^2$Institute of Informatics Problems, Federal Research Center ``Computer Science and Control'' of the 
Russian\linebreak
$\hphantom{^1}$Academy of Sciences, 44-2~Vavilov Str., Moscow 119333, Russian Federation



\def\leftfootline{\small{\textbf{\thepage}
\hfill INFORMATIKA I EE PRIMENENIYA~--- INFORMATICS AND
APPLICATIONS\ \ \ 2017\ \ \ volume~11\ \ \ issue\ 3}
}%
 \def\rightfootline{\small{INFORMATIKA I EE PRIMENENIYA~---
INFORMATICS AND APPLICATIONS\ \ \ 2017\ \ \ volume~11\ \ \ issue\ 3
\hfill \textbf{\thepage}}}

\vspace*{3pt}



\Abste{Emerging next generation wireless networks involve new applications and services for 
human-to-human and machine-to-machine (M2M) devices. The problem of increasing 
requirements for network capacity and lack of radio spectrum arises. The solution could be found in 
the licensed shared access framework, e.\,g., in the case of smart cities. The authors 
propose a mathematical model of shared access to spectrum with adaptive power control. The 
algorithm makes it possible to avoid the interference between M2M devices and the spectrum 
owner due, in part, to the fact that it takes into account the spatial distribution and session activity of 
devices.}

%\pagebreak

\KWE{wireless network; smart city; machine-to-machine (M2M); licensed shared 
access (LSA); 
adaptive power control; stochastic process; recursive algorithm; 
blocking probability; interruption 
probability; average number of M2M devices}




\DOI{10.14357/19922264170310} 

%\vspace*{-18pt}

%\pagebreak

\Ack
\noindent
This work was financially supported by the Russian Science 
Foundation (grant No.\,16-11-10227).



%\vspace*{3pt}

  \begin{multicols}{2}

\renewcommand{\bibname}{\protect\rmfamily References}
%\renewcommand{\bibname}{\large\protect\rm References}

{\small\frenchspacing
 {\baselineskip=10.282pt
 \addcontentsline{toc}{section}{References}
 \begin{thebibliography}{99}
\bibitem{1-gud-1}
Cisco Visual Networking Index: Global Mobile Data Traffic Forecast Update, 2016--2021 White 
Paper. March~28, 2017. Available at: {\sf 
http://www.cisco.com/c/en/us/ solutions/collateral/service-provider/visual-networking-index-vni/mobile-white-paper-c11-520862.html} (accessed June~26, 2017).
\bibitem{2-gud-1}
\Aue{Andrews, J., S.~Buzzi, W.~Choi, S.\,V.~Hanly, A.~Lozano, C.\,K.~Soong, and 
J.\,C.~Zhang.} 2014. What will 5G be? \textit{IEEE J.~Sel. Area. Comm.}  
32:1065--1082. 

\bibitem{5-gud-1} %3
ETSI TR 103 113. July 2013. Electromagnetic compatibility and Radio spectrum 
Matters (ERM); System 
Reference document (SRdoc); Mobile broadband services in the 2300~MHz\,--\,2400~MHz frequency 
band under Licensed Shared Access regime. Available at: {\sf 
http://www.etsi.org/deliver/etsi\_tr/103100\_103199/ 103113/01.01.01\_60/tr\_103113v010101p.pdf}  
(accessed June 26, 2017).

\bibitem{3-gud-1} %4
ETSI TR 103 154. October 2014. Reconfigurable Radio Systems (RRS); System requirements for operation of 
Mobile Broadband Systems in the 2300~MHz\,--\,2400~MHz band under Licensed Shared Access 
(LSA). v1.1.1.  Available at: {\sf 
http://www.etsi.org/deliver/etsi\_TS/103100\_\linebreak  
103199/103154/01.01.01\_60/ts\_103154v010101p.pdf} (accessed June~26, 2017).
\bibitem{4-gud-1} %5
ETSI TR 103 235. October 2015. Reconfigurable Radio Systems (RRS); System architecture and high level 
procedures for operation of Licensed Shared Access (LSA) in the 
2300~MHz\,--\,2400~MHz band.
v1.1.1. Available at: {\sf 
http://www.etsi.org/deliver/etsi\_ts\%5C103200\_103299
\%5C103235\%5C01.01.01\_60\%5Cts\_103235v010101p. pdf} (accessed June~26, 2017).

\bibitem{6-gud-1}
\Aue{Buckwitz, K., J.~Engelberg, and G.~Rausch.} 2014. Licensed Shared Access (LSA)~--- 
regulatory background and view of Administrations. \textit{9th Conference (International) on 
Cognitive Radio Oriented Wireless Networks}. IEEE. 413--416.
\bibitem{7-gud-1}
\Aue{Ahokangas, P., M. Matinmikko, S.~Yrj$\ddot{\mbox{o}}$l$\ddot{\mbox{a}}$, 
M.~Mustonen, H.~Posti, E.~Luttinen, and A.~Kivim$\ddot{\mbox{a}}$ki.} 2014. Business 
models for mobile network operators in Licensed Shared Access (LSA). \textit{IEEE Symposium 
(International) on Dynamic Spectrum Access Networks}. IEEE. 263--270.

\bibitem{9-gud-1} %8
\Aue{Borodakiy, V.\,Y., K.\,E.~Samouylov, I.\,A.~Gudkova, D.\,Y.~Ostrikova, 
A.\,A.~Ponomarenko, A.\,M.~Turlikov, and S.\,D.~Andreev.} 2014. Modeling unreliable LSA 
operation in 3GPP LTE cellular networks. \textit{6th Congress (International) on Ultra Modern 
Telecommunications and Control Systems and Workshops Proceedings}. Piscataway, NJ: IEEE.  
490--496.

\bibitem{8-gud-1} %9
\Aue{Ponomarenko-Timofeev, A., A.~Pyattaev, S.~Andreev, Ye.~Koucheryavy, M.~Mueck, and 
I.~Karls.} 2015. Highly dynamic spectrum management within licensed shared access regulatory 
framework. \textit{IEEE Commun. Mag.} 54(3):100--109.

\bibitem{10-gud-1}
\Aue{Gudkova, I.\,A., K.\,E.~Samouylov, D.\,Y.~Ostrikova, E.\,V.~Mokrov,  
A.\,A.~Ponomarenko-Timofeev, S.\,D.~Andreev, and Y.\,A.~Koucheryavy}. 2015. Service failure 
and interruption probability analysis for Licensed Shared Access regulatory framework. \textit{7th 
Congress (International) on Ultra Modern Telecommunications and Control Systems and Workshops
Proceedings}. Piscataway,  NJ: IEEE. 123--131.
\bibitem{11-gud-1}
\Aue{Samouylov, K., I.~Gudkova, E.~Markova, and N.~Yarkina}. 2016. Queuing model with 
unreliable servers for limit power policy within Licensed Shared Access framework. 
\textit{Internet of things, smart
spaces, and next generation networks and systems}.
Eds. O.~Galinina, S.~Balankin, Y.~Koucheryavy.
{Lecture 
notes in computer science ser.} Springer. 9870:404--413.
\bibitem{12-gid-1}
\Aue{Galinina, O., S.\,D.~Andreev, M.~Gerasimenko, Y.\,A.~Koucheryavy, N.~Himayat,  
S.-P.~Yeh, and S.~Talwar.} 2014. Capturing spatial randomness of heterogeneous cellular/WLAN 
deployments with dynamic traffic. \textit{IEEE J.~Sel. Area. Comm.}  
32(6):1083--1099.
\bibitem{13-gud-1}
\Aue{Ahmadian, A., O.~Galinina, I.~Gudkova, S.~Andreev, S.~Shorgin, and K.~Samouylov.} 
2014. On capturing spatial diversity of joint M2M/H2H dynamic uplink transmissions in 3GPP 
LTE cellular system. 
\textit{Internet of things, smart
spaces, and next generation networks and systems}.
Eds.\ S.~Balandin, S.~Andreev, Y.~Koucheryavy.
{Lecture notes in computer science ser.} Springer. 9247:407--421.
\bibitem{14-gud-1}
\Aue{Samouylov, K., I.~Gudkova, E.~Markova, and I.~Dzantiev.} 2016. On analyzing the 
blocking probability of M2M transmissions for a CQI-based RRM scheme model in 3GPP LTE. 
\textit{Comm. Com. Inf. Sci.} 638:327--340.
\bibitem{15-gud-1}
\Aue{Gudkova, I., E.~Markova, P.~Masek, S.~Andreev, J.~Hosek, N.~Yarkina, K.~Samouylov, 
and Y.~Koucheryavy.} 2016. Modeling the utilization of a multi-tenant band in 3GPP LTE system 
with Licensed Shared Access. \textit{8th Congress (International) on Ultra Modern 
Telecommunications and Control Systems and Workshops Proceedings}.  Piscataway, NJ: IEEE.  
179--183.
\end{thebibliography}

 }
 }

\end{multicols}

\vspace*{-9pt}



\hfill{\small\textit{Received April 20, 2017}}

\vspace*{-24pt}

\Contr

\noindent
\textbf{Gudkova Irina A.}\ (b.\ 1985)~--- Candidate of Sciences (PhD) in physics and 
mathematics; associate professor, Peoples' Friendship University of Russia,  
6~Miklukho-Maklaya Str., Moscow 117198, Russian Federation; senior scientist, Institute 
of Informatics Problems, Federal Research Center ``Computer Science and Control'' of the 
Russian Academy of Sciences, 44-2~Vavilov Str., Moscow 119333, Russian Federation; 
\mbox{ gudkova\_ia@rudn.university }

%\vspace*{1pt}

\noindent
\textbf{Shorgin Sergey Ya.} (b.\ 1952)~--- Doctor of Science in physics and mathematics, professor; Deputy Director, Federal Research Center 
``Computer Science and Control'' of the Russian Academy of Sciences (FRC CSC RAS); principal scientist, Institute of Informatics Problems, FRC 
CSC RAS; 44-2~Vavilov Str., Moscow 119333, Russian Federation; \mbox{sshorgin@ipiran.ru}

\label{end\stat}


\renewcommand{\bibname}{\protect\rm Литература}   %15
  








%%%%%%%%%%%%%%%%%%%%%%%%%%%%%%%%%%%%%%%%%%%%%%%

%\def\stat{rez}
{%\hrule\par
%\vskip 7pt % 7pt
\raggedleft\Large \bf%\baselineskip=3.2ex
Р\,Е\,Ц\,Е\,Н\,З\,И\,И \vskip 17pt
    \hrule
    \par
\vskip 6pt plus 6pt minus 3pt }

%\thispagestyle{headings} %с верхним колонтитулом
%\thispagestyle{myheadings} %с нижним колонтитулом, но в верхнем РЕЦЕНЗИИ

\def\tit{НОВАЯ КНИГА И.\,Н.~СИНИЦЫНА, А.\,С.~ШАЛАМОВА <<ЛЕКЦИИ ПО ТЕОРИИ 
ИНТЕГРИРОВАННОЙ ЛОГИСТИЧЕСКОЙ ПОДДЕРЖКИ>> (М.: ТОРУС ПРЕСС, 2012. 624~с.)}

%1
\def\aut{Д.ф.-м.н., профессор С.\,Я.~Шоргин}

\def\auf{\ }

\def\leftkol{\ % РЕЦЕНЗИИ
}

\def\rightkol{ \ } 

%\def\leftkol{\ } % ENGLISH ABSTRACTS}

%\def\rightkol{\ } %ENGLISH ABSTRACTS}

%\def\leftkol{РЕЦЕНЗИИ}

%\def\rightkol{РЕЦЕНЗИИ}

\titele{\tit}{\aut}{\auf}{\leftkol}{\rightkol}
\vspace*{-18pt}


     \label{st\stat}

     \begin{multicols}{2}
     {\small
     {\baselineskip=10.1pt
     

      В книге представлено системное изложение теоретических основ одного из новейших 
направлений в \mbox{об\-ласти} экономики послепродажного обслуживания изделий наукоемкой 
продукции (ИНП) длительного пользования~--- интегрированной логистической поддержки
(ИЛП). 
{\looseness=1

}

Приведены также результаты новых работ, выполненных в Институте проблем информатики 
Российской академии наук в рамках научного направления <<Информационные технологии и 
анализ сложных сис\-тем>>.
 {%\looseness=1

}
     
      Излагаемые в книге научные подходы позво\-ляют карди\-наль\-но реформировать 
существующие системы производства и эксплуатации ИНП путем создания и внед\-ре\-ния 
методов рационального и оптимального управ\-ле\-ния процессами расходования 
вре\-мен\-н$\acute{\mbox{ы}}$х, 
мате\-ри\-аль\-ных, трудовых и других ресурсов на всех стадиях жизненного цикла изделий (ЖЦИ) по 
критериям экономической целесообразности и эф\-фек\-тив\-ности.
  {\looseness=1

}
    
      В книге приведен краткий обзор причин возник\-новения и
      развития CALS-методологии как основы 
современных международных стандартов по созданию и функционированию глобальных 
ин\-фор\-ма\-ци\-он\-но-ком\-му\-ни\-ка\-ци\-он\-ных систем, ее ключевых возможностей и эффективности 
результатов ее использования. 
Авторы %\linebreak 
предлагают ряд научных обоснований для разработки 
единой теории проектирования и управления систем ИЛП для полноценного использования 
преимуществ %\linebreak
 суще\-ст\-ву\-ющей методологии, определяют \mbox{общую} структурную схему 
комплексной системы <<ИНП-СППО>> и необходимость разработки для ее описания 
гибридных стохастических моделей.
{%\looseness=1

}

%\columnbreak
      
      Книга состоит из пяти частей, где последовательно излагается материал по каждой из 
следующих тем: <<Интегрированная логистическая поддержка>>, <<Теория гибридных 
стохастических систем и компьютерная поддержка исследований и разработок>>, <<Основы 
математического моделирования, анализа и синтеза систем послепродажного обслуживания>>, 
<<Определение и анализ показателей экспортного потенциала ИНП при проектировании>>, 
<<Задачи управления поддержкой послепродажного обслуживания>>, а также 
<<Моделирование инвестиционных процессов ИЛП в условиях неравновесных финансовых 
рынков>>. 
   
      В конце каждой главы приведены выводы и даны вопросы и задания для 
самоконтроля. В~приложениях содержатся основные определения по программам работ по 
анализу ИЛП, логистическим базам данных и компьютерным решениям, эквивалентной статистической 
линеаризации нелинейных преобразований ИЛП, справочный материал, а также развернутые 
уравнения для вероятностных характеристик.


      \def\leftkol{РЕЦЕНЗИИ}

\def\rightkol{РЕЦЕНЗИИ} 

      
      Книга заинтересует широкий круг специалистов и может быть использована научными 
проектными организациями в сфере промышленного производства ИНП. Большое количество 
иллюстраций, примеров и вопросов, обращенных к читателю, позволяет использовать книгу 
также в качестве учебного пособия для студентов и аспирантов машиностроительных, 
транспортных и~других специальностей, а также для самостоятельного изучения. 
{%\looseness=-1

}

Книга 
представляет несомненный интерес для специалистов и студентов в области прикладной 
математики и информатики.
    

}

}
\end{multicols}

%\newpage

\def\stat{authorsrus}
{%\hrule\par
%\vskip 7pt % 7pt
\raggedleft\Large \bf%\baselineskip=3.2ex
О\,Б\ \ А\,В\,Т\,О\,Р\,А\,Х \vskip 17pt
    \hrule
    \par
\vskip 21pt plus 8pt minus 4pt }


\def\tit{\ }

\def\aut{\ }

\def\auf{\ }

\def\leftkol{\ } % ENGLISH ABSTRACTS}

\def\rightkol{ОБ АВТОРАХ} %ENGLISH ABSTRACTS}

\titele{\tit}{\aut}{\auf}{\leftkol}{\rightkol}
      
            \label{st\stat}



\vspace*{24pt}

\begin{multicols}{2}




\noindent
\textbf{Архипов Олег Петрович} (р.\ 1948)~---
кандидат технических наук, директор Орловского филиала Института проб\-лем информатики
Российской академии наук
%302025, г.Орел, Московское шоссе, д.137

\vspace*{3pt}

\noindent
\textbf{Бирюкова Татьяна Константиновна} (р.\ 1968)~---
кандидат фи\-зи\-ко-ма\-те\-ма\-ти\-че\-ских наук, старший научный сотрудник Института проб\-лем информатики
Российской академии наук

\vspace*{3pt}

\noindent 
\textbf{Бобков  Сергей Геннадьевич} (р.\ 1955)~---
доктор технических наук,  заведующий отделением На\-уч\-но-ис\-сле\-до\-ва\-тель\-ско\-го 
института системных исследований Российской академии наук
%117218, Москва, Нахимовский просп., 36, к.1 

\vspace*{3pt}

\noindent \textbf{Васильев Николай Семенович} (р.\ 1952)~--- доктор 
фи\-зи\-ко-ма\-те\-ма\-ти\-че\-ских наук, профессор, 
МГТУ им.\ Н.\,Э.~Баумана 
%, Москва 105005, 2-я Бауманская ул., д.~5,

\vspace*{3pt}

\noindent
\textbf{Гершкович Максим Михайлович} (р.\ 1968)~---
старший научный сотрудник Института проб\-лем информатики
Российской академии наук

\vspace*{3pt}

\noindent 
\textbf{Дьяченко Юрий Георгиевич} (р.\ 1958)~--- кандидат технических наук, 
старший научный сотрудник Института проб\-лем информатики
Российской академии наук

\vspace*{3pt}

\noindent 
\textbf{Ерошенко Александр Андреевич} (р.\ 1989)~--- аспирант кафедры 
математической статистики факультета вычисли\-тельной математики и кибернетики 
Московского государственного университета им.\ М.\,В.~Ломоносова
%119991, Москва ГСП-1, Ленинские горы, д.\ 1, стр. 52

\vspace*{3pt}
 
\noindent 
\textbf{Захаров Виктор Николаевич} (р.\ 1948)~--- 
доктор технических наук, доцент, ученый секретарь Института проб\-лем информатики
Российской академии наук

\vspace*{3pt}

\noindent
\textbf{Зейфман Александр Израилевич} (р.\ 1954)~---
доктор фи\-зи\-ко-ма\-те\-ма\-ти\-че\-ских наук, профессор, 
заведующий кафедрой Вологодского государственного университета; 
старший научный сотрудник Института проб\-лем информатики
Российской академии наук; главный научный сотрудник ИСЭРТ Российской академии наук

\vspace*{3pt}

\noindent
\textbf{Зыкин Сергей Владимирович} (р.\ 1959)~--- 
доктор технических наук, профессор, заведующий лабораторией Института математики 
им.\ С.\,Л.~Соболева Сибирского отделения Российской академии наук, Новосибирск 
%630090, пр.\ ак.\ Коптюга, 4 

\vspace*{4pt}

\noindent
\textbf{Киреев Владимир Иванович} (р.\ 1938)~---
доктор фи\-зи\-ко-ма\-те\-ма\-ти\-че\-ских наук, профессор Московского 
государственного горного университета
%Адрес: Россия, 119991, г. Москва, Ленинский проспект, д. 6

%\columnbreak

\vspace*{4pt}

\noindent
\textbf{Козеренко Елена Борисовна} (р.\ 1959)~---
кандидат филологических наук, заведующая лабораторией Института проб\-лем информатики
Российской академии наук

\vspace*{4pt}

\noindent
\textbf{Королев Виктор Юрьевич} (р.\ 1954)~--- доктор
фи\-зи\-ко-ма\-те\-ма\-ти\-че\-ских наук, профессор кафедры математической 
статистики факультета вычисли\-тельной математики и кибернетики 
Московского государственного университета; 
ведущий научный сотрудник Института проб\-лем информатики
Российской академии наук

\vspace*{4pt}

\noindent
\textbf{Коротышева Анна Владимировна} (р.\ 1988)~---
старший преподаватель Вологодского государственного университета

\vspace*{4pt}

\noindent 
\textbf{Кун Де Турк} (р.\ 1981)~--- научный сотрудник 
исследовательской группы SMACS факультета телекоммуникаций и обработки информации
Университета Гента, Бельгия
%В-9000 Гент, Бельгия

\vspace*{4pt}

\noindent
\textbf{Лупенцов Олег Сергеевич} (р.\ 1986)~---
аспирант Омского государственного института сервиса
%Омск 644043, ул.\ Певцова 13

\vspace*{4pt}

\noindent
\textbf{Лучко Олег Николаевич} (р.\ 1961)~---
кандидат педагогических наук, профессор, заведующий кафедрой 
Омского государственного института сервиса
%Омск 644043, ул.\ Певцова 13

\vspace*{4pt}

\noindent
\textbf{Малашенко Юрий Евгеньевич} (р.\ 1946)~---
доктор фи\-зи\-ко-ма\-те\-ма\-ти\-че\-ских наук, заведующий сектором 
Вычислительного центра им.\ А.\,А.~Дородницына Российской академии наук
%Адрес: 119333, Москва, ул. Вавилова, 40,

\vspace*{4pt}

\noindent
\textbf{Маньяков Юрий Анатольевич} (р.\ 1984)~---
кандидат технических наук, научный сотрудник Орловского филиала Института проб\-лем информатики
Российской академии наук
%302025, г.Орел, Московское шоссе, д.137

\vspace*{4pt}

\noindent
\textbf{Маренко Валентина Афанасьевна} (р.\ 1951)~---
кандидат технических наук, доцент, старший научный сотрудник 
Института математики им.\ С.\,Л.~Соболева Сибирского отделения Российской академии наук
%Новосибирск 630090, пр. ак. Коптюга, 4 

\vspace*{3pt}

\noindent 
\textbf{Морозов Евсей Викторович} (р.\ 1947)~--- доктор 
фи\-зи\-ко-ма\-те\-ма\-ти\-че\-ских, профессор, ведущий научный сотрудник 
Института прикладных математических исследований Карельского научного центра Российской
академии наук; 
%%185910 Россия, Республика Карелия, г.\ Петрозаводск, ул.\ Пушкинская, 11
профессор Петрозаводского государственного университета, Петрозаводск
%185910 Россия, Республика Карелия, г.\ Петрозаводск, пр.\ Ленина, 33

%\pagebreak

\vspace*{3pt}

\noindent
\textbf{Назарова Ирина Александровна} (р.\ 1966)~---
кандидат фи\-зи\-ко-ма\-те\-ма\-ти\-че\-ских наук, 
научный сотрудник Вычислительного центра им.\ А.\,А.~Дородницына Российской академии наук 
%Адрес: 119333, Москва, ул. Вавилова, 40

\vspace*{3pt}

\noindent
\textbf{Павлов Игорь Валерианович} (р.\ 1945)~--- 
доктор фи\-зи\-ко-ма\-те\-ма\-ти\-че\-ских наук, профессор МГТУ им.\ Н.\,Э.~Баумана 
%Москва 105005, 2-я Бауманская ул., д.~5 

%\pagebreak

\vspace*{3pt}

\noindent 
\textbf{Потахина Любовь Викторовна} (р.\ 1989)~--- аспирантка
Института прикладных математических исследований Карельского научного центра
Российской академии наук; 
%%185910 Россия, Республика Карелия, г.\ Петрозаводск, ул.\ Пушкинская, 11
инженер Петрозаводского государственного университета, Петрозаводск
%185910 Россия, Республика Карелия, г.\ Петрозаводск, пр.\ Ленина, 33

\vspace*{3pt}

\noindent 
\textbf{Рождественский Юрий Владимирович} (р.\ 1952)~--- 
кандидат технических наук, заведующий сектором Института проб\-лем информатики
Российской академии наук

\vspace*{3pt}

\noindent 
\textbf{Синицын Игорь Николаевич} (р.\ 1940)~--- доктор технических наук,
профессор, заслуженный деятель\linebreak\vspace*{-12pt}

\columnbreak

\noindent
 науки РФ, заведующий отделом Института проб\-лем информатики
Российской академии наук

\vspace*{7pt}


\noindent
\textbf{Сиротинин Денис Олегович} (р.\ 1984)~---
кандидат технических наук, научный сотрудник Орловского филиала Института проб\-лем информатики
Российской академии наук
%302025, г.Орел, Московское шоссе, д.137

\vspace*{7pt}

%\columnbreak

\noindent 
\textbf{Соколов  Игорь Анатольевич} (р.\ 1954)~--- академик (действительный член) Российской 
академии наук, доктор технических наук, директор Института проб\-лем информатики
Российской академии наук

\vspace*{7pt}

\noindent
\textbf{Степченков Юрий Афанасьевич} (р.\ 1951)~---
кандидат технических наук, заведующий отделом Института проб\-лем информатики
Российской академии наук

\vspace*{7pt}

\noindent
\textbf{Сурков Алексей Викторович} (р.\ 1978)~--- 
старший научный сотрудник На\-уч\-но-ис\-сле\-до\-ва\-тель\-ско\-го 
института системных исследований Российской академии наук
%117218, Москва, Нахимовский просп., 36, к.1 

\vspace*{7pt}

\noindent 
\textbf{Шестаков Олег Владимирович} (р.\ 1976)~--- доктор 
фи\-зи\-ко-ма\-те\-ма\-ти\-че\-ских, доцент кафедры математической статистики 
факультета вычисли\-тельной математики и кибернетики Московского 
государственного университета им.\ М.\,В.~Ломоносова; 
%119991, Москва ГСП-1, Ленинские горы, д.\ 1, стр. 52
старший научный сотрудник Института проб\-лем информатики
Российской академии наук
%, Москва 119333, ул. Вавилова, д.~44, корп.~2

\vspace*{7pt}

\noindent 
\textbf{Шоргин Сергей Яковлевич} (р.\ 1952.)~--- доктор
фи\-зи\-ко-ма\-те\-ма\-ти\-че\-ских наук, профессор, заместитель директора Института 
проб\-лем информатики Российской академии наук





%%%%%%%%%%%%%%%%%%%%%%%%%%%%%%%%%%%%%%%%%%%%%%%%%%%%%%%%%%%%%%%%%%%%%%%%%%%%%%%




%\def\rightkol{ОБ АВТОРАХ}
%\def\leftkol{ОБ АВТОРАХ}

 \label{end\stat}





%\def\leftfootline{\small{\textbf{\thepage}
%\hfill ИНФОРМАТИКА И ЕЁ ПРИМЕНЕНИЯ\ \ \ том~7\ \ \ выпуск~1\ \ \ 2013}
%}%
% \def\rightfootline{\small{ИНФОРМАТИКА И ЕЁ ПРИМЕНЕНИЯ\ \ \ том~7\ \ \ выпуск~1\ \ \ 2013
%\hfill \textbf{\thepage}}}


%\thispagestyle{myheadings}



\end{multicols}

\newpage  

%\def\stat{cont}
{%\hrule\par
%\vskip 7pt % 7pt
\raggedleft\Large \bf%\baselineskip=3.2ex
А\,В\,Т\,О\,Р\,С\,К\,И\,Й\ \ У\,К\,А\,З\,А\,Т\,Е\,Л\,Ь\ \ З\,А\ \ 2\,0\,0\,7 г. \vskip 17pt
    \hrule
    \par
\vskip 21pt plus 6pt minus 3pt }

\label{st\stat}

\def\tit{\ }

\def\aut{\ }
\def\auf{\ }

\def\leftkol{\ } % ENGLISH ABSTRACTS}

\def\rightkol{\ } %ENGLISH ABSTRACTS}

\titele{\tit}{\aut}{\auf}{\leftkol}{\rightkol}


\contentsline {chapter}{\ }{Выпуск \quad Стр.} 
\contentsline {section}{\textbf{Батракова Д.\,А., Королев В.\,Ю., Шоргин С.\,Я.}\ \ Новый метод вероятностно-ста\-ти\-сти\-че\-ско\-го анализа информационных потоков в\nobreakspace {}телекоммуникационных сетях}{\qquad 1 \qquad 40} 
\contentsline {section}{\textbf{Борисов А.\,В.}\ \ Байесовское оценивание в системах наблюдения с\nobreakspace {}марковскими скачкообразными процессами: игровой подход}{\qquad 2 \qquad 65}
\contentsline {section}{\textbf{Босов А.\,В., Иванов А.\,В.}\ \ Программная инфраструктура информационного Web-пор\-тала}{\qquad 2 \qquad 50}
\contentsline {section}{\textbf{Захаров В.\,Н., Калиниченко Л.\,А., Соколов И.\,А., Ступников С.\,А.}\ \ Конструирование канонических информационных моделей для интегрированных информационных систем}{\qquad 2 \qquad 15}
\contentsline {section}{\textbf{Захаров В.\,Н., Козмидиади В.\,А.}\ \ Средства обеспечения отказоустойчивости при\-ло\-жений}{\qquad 1 \qquad 14} 
\contentsline {section}{\textbf{Иванов А.\,В.}\ \ см. Босов А.\,В.\hfill\hfill\hfill\hfill\hfill\hfill\hfill\hfill\hfill\hfill\hfill\hfill\hfill\hfill\hfill\hfill\hfill\hfill\hfill\hfill\hfill\hfill\hfill\hfill\hfill\hfill\hfill\hfill\hfill\hfill\hfill\hfill\hfill\hfill\hfill}{\ }
\contentsline {section}{\textbf{Ильин В.\,Д., Соколов И.\,А.}\ \ Символьная модель системы знаний информатики в\nobreakspace {}че\-ло\-ве\-ко-автоматной среде}{\qquad 1 \qquad 66} 
\contentsline {section}{\textbf{Калиниченко Л.\,А.}\ \ см. Захаров В.\,Н.\hfill\hfill\hfill\hfill\hfill\hfill\hfill\hfill\hfill\hfill\hfill\hfill\hfill\hfill\hfill\hfill\hfill\hfill\hfill\hfill\hfill\hfill\hfill\hfill\hfill\hfill\hfill\hfill\hfill\hfill\hfill\hfill\hfill\hfill\hfill}{\ }
\contentsline {section}{\textbf{Козеренко Е.\,Б.}\ \ Лингвистическое моделирование для систем машинного перевода и обработки знаний}{\qquad 1 \qquad 54} 
\contentsline {section}{\textbf{Козмидиади В.\,А.}\ \ см. Захаров В.\,Н.\hfill\hfill\hfill\hfill\hfill\hfill\hfill\hfill\hfill\hfill\hfill\hfill\hfill\hfill\hfill\hfill\hfill\hfill\hfill\hfill\hfill\hfill\hfill\hfill\hfill\hfill\hfill\hfill\hfill\hfill\hfill\hfill\hfill\hfill\hfill }{\ } 
\contentsline {section}{\textbf{Королев В.\,Ю.}\ \ см. Батракова Д.\,А.\hfill\hfill\hfill\hfill\hfill\hfill\hfill\hfill\hfill\hfill\hfill\hfill\hfill\hfill\hfill\hfill\hfill\hfill\hfill\hfill\hfill\hfill\hfill\hfill\hfill\hfill\hfill\hfill\hfill\hfill\hfill\hfill\hfill\hfill\hfill}{\ } 
\contentsline {section}{\textbf{Кудрявцев А.\,А., Шоргин С.\,Я.}\ \ Байесовский подход к\nobreakspace {}анализу систем массового обслуживания и\nobreakspace {}показателей надежности}{\qquad 2 \qquad 76}
\contentsline {section}{\textbf{Печинкин А.\,В., Соколов И.\,А., Чаплыгин В.\,В.}\ \ Многолинейная система массового обслуживания с конечным накопителем и ненадежными приборами}{\qquad 1 \qquad 27} 
\contentsline {section}{\textbf{Печинкин А.\,В., Соколов И.\,А., Чаплыгин В.\,В.}\ \ Стационарные характеристики многолинейной\nobreakspace {}системы массового обслуживания с\nobreakspace {}одновременными отказами приборов}{\qquad 2 \qquad 39}
\contentsline {section}{\textbf{Синицын И.\,Н.}\ \ Корреляционные методы построения аналитических информационных моделей флуктуаций полюса Земли по априорным данным}{\qquad 2 \qquad \hphantom{9}2}
\contentsline {section}{\textbf{Синицын И.\,Н.}\ \ Развитие теории фильтров Пугачева для оперативной обработки информации в стохастических системах}{{\qquad 1 \qquad \hphantom{9}3}} 
\contentsline {section}{\textbf{Соколов И.\,А.}\ \ см. Захаров В.\,Н.\hfill\hfill\hfill\hfill\hfill\hfill\hfill\hfill\hfill\hfill\hfill\hfill\hfill\hfill\hfill\hfill\hfill\hfill\hfill\hfill\hfill\hfill\hfill\hfill\hfill\hfill\hfill\hfill\hfill\hfill\hfill\hfill\hfill\hfill\hfill}{\ }
\contentsline {section}{\textbf{Соколов И.\,А.}\ \ см. Ильин В.\,Д.\hfill\hfill\hfill\hfill\hfill\hfill\hfill\hfill\hfill\hfill\hfill\hfill\hfill\hfill\hfill\hfill\hfill\hfill\hfill\hfill\hfill\hfill\hfill\hfill\hfill\hfill\hfill\hfill\hfill\hfill\hfill\hfill\hfill\hfill\hfill}{\ } 
\contentsline {section}{\textbf{Соколов И.\,А.}\ \ см. Печинкин А.\,В.\hfill\hfill\hfill\hfill\hfill\hfill\hfill\hfill\hfill\hfill\hfill\hfill\hfill\hfill\hfill\hfill\hfill\hfill\hfill\hfill\hfill\hfill\hfill\hfill\hfill\hfill\hfill\hfill\hfill\hfill\hfill\hfill\hfill\hfill\hfill}{\ } 
\contentsline {section}{\textbf{Соколов И.\,А.}\ \ см. Печинкин А.\,В.\hfill\hfill\hfill\hfill\hfill\hfill\hfill\hfill\hfill\hfill\hfill\hfill\hfill\hfill\hfill\hfill\hfill\hfill\hfill\hfill\hfill\hfill\hfill\hfill\hfill\hfill\hfill\hfill\hfill\hfill\hfill\hfill\hfill\hfill\hfill}{\ }
\contentsline {section}{\textbf{Ступников С.\,А.}\ \ см. Захаров В.\,Н.\hfill\hfill\hfill\hfill\hfill\hfill\hfill\hfill\hfill\hfill\hfill\hfill\hfill\hfill\hfill\hfill\hfill\hfill\hfill\hfill\hfill\hfill\hfill\hfill\hfill\hfill\hfill\hfill\hfill\hfill\hfill\hfill\hfill\hfill\hfill}{\ }
\contentsline {section}{\textbf{Чаплыгин В.\,В.}\ \ см. Печинкин А.\,В.\hfill\hfill\hfill\hfill\hfill\hfill\hfill\hfill\hfill\hfill\hfill\hfill\hfill\hfill\hfill\hfill\hfill\hfill\hfill\hfill\hfill\hfill\hfill\hfill\hfill\hfill\hfill\hfill\hfill\hfill\hfill\hfill\hfill\hfill\hfill}{\ } 
\contentsline {section}{\textbf{Чаплыгин В.\,В.}\ \ см. Печинкин А.\,В.\hfill\hfill\hfill\hfill\hfill\hfill\hfill\hfill\hfill\hfill\hfill\hfill\hfill\hfill\hfill\hfill\hfill\hfill\hfill\hfill\hfill\hfill\hfill\hfill\hfill\hfill\hfill\hfill\hfill\hfill\hfill\hfill\hfill\hfill\hfill}{\ }
\contentsline {section}{\textbf{Шоргин С.\,Я.}\ \ см. Батракова Д.\,А.\hfill\hfill\hfill\hfill\hfill\hfill\hfill\hfill\hfill\hfill\hfill\hfill\hfill\hfill\hfill\hfill\hfill\hfill\hfill\hfill\hfill\hfill\hfill\hfill\hfill\hfill\hfill\hfill\hfill\hfill\hfill\hfill\hfill\hfill\hfill}{\ } 
\contentsline {section}{\textbf{Шоргин С.\,Я.}\ \ см. Кудрявцев А.\,А.\hfill\hfill\hfill\hfill\hfill\hfill\hfill\hfill\hfill\hfill\hfill\hfill\hfill\hfill\hfill\hfill\hfill\hfill\hfill\hfill\hfill\hfill\hfill\hfill\hfill\hfill\hfill\hfill\hfill\hfill\hfill\hfill\hfill\hfill\hfill}{\ }
%\thispagestyle{myheadings}
\def\leftfootline{\small{\textbf{\thepage}
\hfill ИНФОРМАТИКА И ЕЁ ПРИМЕНЕНИЯ\ \ \ том~1\ \ \ выпуск~2\ \ \ 2007}
}%
 \def\rightfootline{\small{ИНФОРМАТИКА И ЕЁ ПРИМЕНЕНИЯ\ \ \ том~1\ \ \ выпуск~2\ \ \ 2007
 \hfill \textbf{\thepage}}}
 \label{end\stat} 
                     
%\def\stat{cont-e}
{%\hrule\par
%\vskip 7pt % 7pt
\raggedleft\Large \bf%\baselineskip=3.2ex
2\,0\,0\,7\ \ A\,U\,T\,H\,O\,R\ \ I\,N\,D\,E\,X \vskip 17pt
    \hrule
    \par
\vskip 21pt plus 6pt minus 3pt }

\label{st\stat}

\def\tit{\ }

\def\aut{\ }
\def\auf{\ }

\def\leftkol{\ } % ENGLISH ABSTRACTS}

\def\rightkol{\ } %ENGLISH ABSTRACTS}

\titele{\tit}{\aut}{\auf}{\leftkol}{\rightkol}


\contentsline {chapter}{\ }{Issue \quad Page} 
\contentsline {subsection}{\textbf{Batrakova D.\,A., Korolev V.\,Yu., Shorgin S.\,Ya.}\ \ A New Method for the Probabilistic and Statistical Analysis of Information Flows in Telecommunication Networks}{\qquad 1 \qquad 40} 
\contentsline {subsection}{\textbf{Borisov A.\,V.}\ \ Bayesian Estimation in\nobreakspace {}Observation Systems with\nobreakspace {}Markov Jump Processes: Game-Theoretic Approach}{\qquad 2 \qquad 65} 
\contentsline {subsection}{\textbf{Bosov A.\,V., Ivanov A.\,V.}\ \ Linguistic Simulation for Machine Translation and Knowledge Management Systems}{\qquad 2 \qquad 50} 
\contentsline {subsection}{\textbf{Chaplygin V.\,V.} see Pechinkin A.\,V.\hfill\hfill\hfill\hfill\hfill\hfill\hfill\hfill\hfill\hfill\hfill\hfill\hfill\hfill\hfill\hfill\hfill\hfill\hfill\hfill\hfill\hfill\hfill\hfill\hfill\hfill\hfill\hfill\hfill\hfill\hfill\hfill\hfill\hfill\hfill}{\ }
\contentsline {subsection}{\textbf{Chaplygin V.\,V.} see Pechinkin A.\,V.\hfill\hfill\hfill\hfill\hfill\hfill\hfill\hfill\hfill\hfill\hfill\hfill\hfill\hfill\hfill\hfill\hfill\hfill\hfill\hfill\hfill\hfill\hfill\hfill\hfill\hfill\hfill\hfill\hfill\hfill\hfill\hfill\hfill\hfill\hfill}{\ }
\contentsline {subsection}{\textbf{Ilyin V.\,D., Sokolov I.\,A.}\ \ The Symbol Model of Informatics Knowledge System in Human-Automaton Environment}{\qquad 1 \qquad 66} 
\contentsline {subsection}{\textbf{Ivanov A.\,V.} see Bosov A.\,V.\hfill\hfill\hfill\hfill\hfill\hfill\hfill\hfill\hfill\hfill\hfill\hfill\hfill\hfill\hfill\hfill\hfill\hfill\hfill\hfill\hfill\hfill\hfill\hfill\hfill\hfill\hfill\hfill\hfill\hfill\hfill\hfill\hfill\hfill\hfill}{\ }
\contentsline {subsection}{\textbf{Kalinichenko L.\,A.} see Zakharov V.\,N.\hfill\hfill\hfill\hfill\hfill\hfill\hfill\hfill\hfill\hfill\hfill\hfill\hfill\hfill\hfill\hfill\hfill\hfill\hfill\hfill\hfill\hfill\hfill\hfill\hfill\hfill\hfill\hfill\hfill\hfill\hfill\hfill\hfill\hfill\hfill}{\ }
\contentsline {subsection}{\textbf{Korolev V.\,Yu.} see Batrakova D.\,A.\hfill\hfill\hfill\hfill\hfill\hfill\hfill\hfill\hfill\hfill\hfill\hfill\hfill\hfill\hfill\hfill\hfill\hfill\hfill\hfill\hfill\hfill\hfill\hfill\hfill\hfill\hfill\hfill\hfill\hfill\hfill\hfill\hfill\hfill\hfill}{\ }
\contentsline {subsection}{\textbf{Kozerenko E.\,B.}\ \ Linguistic Simulation for Machine Translation and Knowledge Management Systems}{\qquad 1 \qquad 54} 
\contentsline {subsection}{\textbf{Kozmidiady V.\,A.} see Zakharov V.\,N.\hfill\hfill\hfill\hfill\hfill\hfill\hfill\hfill\hfill\hfill\hfill\hfill\hfill\hfill\hfill\hfill\hfill\hfill\hfill\hfill\hfill\hfill\hfill\hfill\hfill\hfill\hfill\hfill\hfill\hfill\hfill\hfill\hfill\hfill\hfill}{\ }
\contentsline {subsection}{\textbf{Kudryavtsev A.\,A., Shorgin S.\,Ya.}\ \ Bayesian Approach to Queueing Systems and Reliability Characteristics}{\qquad 2 \qquad 76} 
\contentsline {subsection}{\textbf{Pechinkin A.\,V., Sokolov I.\,A., Chaplygin V.\,V.}\ \ Multichannel Queuing System with Finite Buffer and Unreliable Servers}{\qquad 1 \qquad 27} 
\contentsline {subsection}{\textbf{Pechinkin A.\,V., Sokolov I.\,A., Chaplygin V.\,V.}\ \ Stationary Characteristics of a Multichannel Queueing System with\nobreakspace {}Simultaneous Refusals of Servers}{\qquad 2 \qquad 39} 
\contentsline {subsection}{\textbf{Shorgin S.\,Ya.} see Batrakova D.\,A.\hfill\hfill\hfill\hfill\hfill\hfill\hfill\hfill\hfill\hfill\hfill\hfill\hfill\hfill\hfill\hfill\hfill\hfill\hfill\hfill\hfill\hfill\hfill\hfill\hfill\hfill\hfill\hfill\hfill\hfill\hfill\hfill\hfill\hfill\hfill}{\ }
\contentsline {subsection}{\textbf{Shorgin S.\,Ya.} see Kudryavtsev A.\,A.\hfill\hfill\hfill\hfill\hfill\hfill\hfill\hfill\hfill\hfill\hfill\hfill\hfill\hfill\hfill\hfill\hfill\hfill\hfill\hfill\hfill\hfill\hfill\hfill\hfill\hfill\hfill\hfill\hfill\hfill\hfill\hfill\hfill\hfill\hfill}{\ }
\contentsline {subsection}{\textbf{Sinitsyn I.\,N.}\ \ Correlational Methods for Analytical Informational Models of the Earth Pole Fluctuations Design Based on a priori Data}{\qquad 2 \qquad \hphantom{9}2}
\contentsline {subsection}{\textbf{Sinitsyn I.\,N.}\ \ Development of Pugachev Filtering for Stochastic Systems}{\qquad 1 \qquad \hphantom{9}3}
\contentsline {subsection}{\textbf{Sokolov I.\,A.} see Ilyin V.\,D.\hfill\hfill\hfill\hfill\hfill\hfill\hfill\hfill\hfill\hfill\hfill\hfill\hfill\hfill\hfill\hfill\hfill\hfill\hfill\hfill\hfill\hfill\hfill\hfill\hfill\hfill\hfill\hfill\hfill\hfill\hfill\hfill\hfill\hfill\hfill}{\ }
\contentsline {subsection}{\textbf{Sokolov I.\,A.} see Pechinkin A.\,V.\hfill\hfill\hfill\hfill\hfill\hfill\hfill\hfill\hfill\hfill\hfill\hfill\hfill\hfill\hfill\hfill\hfill\hfill\hfill\hfill\hfill\hfill\hfill\hfill\hfill\hfill\hfill\hfill\hfill\hfill\hfill\hfill\hfill\hfill\hfill}{\ }
\contentsline {subsection}{\textbf{Sokolov I.\,A.} see Pechinkin A.\,V.\hfill\hfill\hfill\hfill\hfill\hfill\hfill\hfill\hfill\hfill\hfill\hfill\hfill\hfill\hfill\hfill\hfill\hfill\hfill\hfill\hfill\hfill\hfill\hfill\hfill\hfill\hfill\hfill\hfill\hfill\hfill\hfill\hfill\hfill\hfill}{\ }
\contentsline {subsection}{\textbf{Sokolov I.\,A.} see Zakharov V.\,N.\hfill\hfill\hfill\hfill\hfill\hfill\hfill\hfill\hfill\hfill\hfill\hfill\hfill\hfill\hfill\hfill\hfill\hfill\hfill\hfill\hfill\hfill\hfill\hfill\hfill\hfill\hfill\hfill\hfill\hfill\hfill\hfill\hfill\hfill\hfill}{\ }
\contentsline {subsection}{\textbf{Stupnikov S.\,A.} see Zakharov V.\,N.\hfill\hfill\hfill\hfill\hfill\hfill\hfill\hfill\hfill\hfill\hfill\hfill\hfill\hfill\hfill\hfill\hfill\hfill\hfill\hfill\hfill\hfill\hfill\hfill\hfill\hfill\hfill\hfill\hfill\hfill\hfill\hfill\hfill\hfill\hfill}{\ }
\contentsline {subsection}{\textbf{Zakharov V.\,N., Kalinichenko L.\,A., Sokolov I.\,A., Stupnikov S.\,A.}\ \ Development of Canonical Information Models for Integrated Information Systems}{\qquad 2 \qquad 15} 
\contentsline {subsection}{\textbf{Zakharov V.\,N., Kozmidiady V.\,A.}\ \ Means Providing Applications Fault Tolerance}{\qquad 1 \qquad 14} 
\def\leftfootline{\small{\textbf{\thepage}
\hfill ИНФОРМАТИКА И ЕЁ ПРИМЕНЕНИЯ\ \ \ том~1\ \ \ выпуск~2\ \ \ 2007}
}%
 \def\rightfootline{\small{ИНФОРМАТИКА И ЕЁ ПРИМЕНЕНИЯ\ \ \ том~1\ \ \ выпуск~2\ \ \ 2007
 \hfill \textbf{\thepage}}}
 \label{end\stat} 


%\end{document}

%
\def\stat{rekl}
%\label{preobr}

%\def\tit{АКАДЕМИК ПУГАЧЁВ  ВЛАДИМИР СЕМЁНОВИЧ\\
%25.03.1911--25.03.1998}


%   \vspace*{-48pt}
%   \begin{center}\LARGE
%Академик Пугачёв  Владимир Семёнович\\ (25.03.1911--25.03.1998)
%   \end{center}

   %\vspace*{2.5mm}

   \begin{center}

{\prgsh\LARGE
ЮБИЛЕИ}

\end{center}
%\hrule

\vspace*{6pt}


   \vspace*{8mm}

   \thispagestyle{empty}


%\def\stat{emel}


\section*{К 70-летию заместителя директора ИПИ РАН,\\ члена редколлегии журнала
<<Информатика и её применения>>\\ доктора технических наук В.\,И.~Будзко}

\vspace*{18pt}




          \begin{multicols}{2}

%            \label{st\stat}

\begin{center}
\vspace*{1pt}
\mbox{%
\epsfxsize=78mm
\epsfbox{bud-1.eps}
}
\end{center}

\vspace*{12pt}

      14 августа 2014~г.\ исполнилось 70~лет за\-мес\-ти\-те\-лю директора ИПИ РАН по
научной работе доктору технических наук Владимиру Игоревичу Будзко.

      Владимир Игоревич Будзко родился в г.~Москве. Высшее образование получил на факультете
элект\-рон\-но-вы\-чис\-ли\-тель\-ных устройств в Московском
ин\-же\-нер\-но-фи\-зи\-че\-ском институте
(МИФИ), который он окончил в 1968~г., после чего был на\-прав\-лен для прохождения
службы в одну из войс\-ко\-вых частей, где прошел путь от инженера до первого заместителя
командира войсковой части.

      С приходом В.\,И.~Будзко в ИПИ РАН (2001~г.)\ в институте
сформировалось новое научное на\-прав\-ле\-ние теоретических исследований~--- <<Постро\-ение
ин\-фор\-ма\-ци\-он\-но-те\-ле\-ком\-му\-ни\-ка\-ци\-он\-ных\linebreak сис\-тем
высокой до\-ступ\-ности>>. В~рамках этого
направления выполнен широкий круг фундаментальных исследований по поиску подходов и
определению принципов построения средств обеспечения доступности, конфиденциальности
и целостности современных крупномасштабных
ин\-фор\-ма\-ци\-он\-но-те\-ле\-ком\-му\-ни\-ка\-ци\-он\-ных
сис\-тем (ИТС). Разработаны основные сис\-тем\-но-тех\-ни\-че\-ские принципы и базовые
архитектурные решения построения перспективных для условий России ИТС с
централизованной обработкой и хранением информации, сочетающих в себе свойства
высокой доступности, отказо- и катастрофоустойчивости, информационной защищенности.
Определены принципы, методы и математические основы рационального построения и
оптимизации средств восстановления функционирования центров обработки данных (ЦОД)
после возникновения отказов и катастроф, передачи и хранения данных, обеспечения
информационной безопасности при достижении минимальной совокупной стоимости
владения такими системами. Результаты нашли практическое воплощение при реализации
проектов в интересах ряда отечественных государственных и негосударственных
организаций, таких как Банк России (БР), Внешторгбанк, ОАО <<ГМК <<Норильский Никель>>,
<<Газпром>>, Минэкономразвития России, Правительство Москвы, а также ряд силовых
ведомств.

      Под руководством В.\,И.~Будзко начиная с 2001~г.\ выполнен комплекс
      на\-уч\-но-ис\-сле\-до\-ва\-тель\-ских и
      опыт\-но-кон\-ст\-рук\-тор\-ских работ (свыше 100~проектов),
направленных на развитие электронной информационной технологии БР.
Разработаны концепции развития ИТС БР сначала до 2008~г., а затем до 2013~г., которые
были приняты в качестве основы проведения технической политики. За реализацию проекта
<<Катастрофоустойчивая тер\-ри\-то\-ри\-аль\-но-рас\-пре\-де\-лен\-ная
      ин\-фор\-ма\-ци\-он\-но-те\-ле\-ком\-му\-ни\-ка\-ци\-он\-ная сис\-те\-ма централизованной
обработки банковской информации>> В.\,И.~Будзко удостоен Премии Правительства РФ в
области науки и техники за 2010~г.

      В.\,И.~Будзко возглавлял и возглавляет работы по ряду других прикладных проектов,
связанных с созданием, совершенствованием и развитием крупномасштабных ИТС.

      В.\,И.~Будзко~--- генерал-майор, доктор технических наук, член-кор\-рес\-пон\-дент
Академии криптографии РФ, известный ученый в области информатики и применения
информационных технологий при построении территориально распределенных ИТС
различного назначения. Является автором свыше 250~научных работ, опубликованных в
на\-уч\-но-тех\-ни\-че\-ских и специальных изданиях.

    \thispagestyle{empty}

      В.\,И.~Будзко уделяет большое внимание подготовке научных кадров. Под его
руководством защищено 6~диссертаций на соискание ученой степени кандидата
технических наук. Свыше 30~лет он читает лекции в ИКСИ Академии ФСБ, профессор
кафедры НИЯУ МИФИ. Является членом двух диссертационных советов, главным
редактором журнала <<Системы высокой доступности>> и членом редколлегии журнала
<<Информатика и её применения>>.

      \bigskip

      Редакционный совет и Редакционная коллегия журнала <<Информатика и её
применения>> сердечно поздравляют Владимира Игоревича Будзко с 70-ле\-ти\-ем и желают
крепкого здоровья и новых научных достижений.

\end{multicols}

%%Информатика и её применения
%Том 12   Выпуск 1-4   Год 2018

\def\stat{cont}
{%\hrule\par
%\vskip 7pt % 7pt
\raggedleft\Large \bf%\baselineskip=3.2ex
А\,В\,Т\,О\,Р\,С\,К\,И\,Й\ \ У\,К\,А\,З\,А\,Т\,Е\,Л\,Ь\ \ З\,А\ \ 2\,0\,1\,8 г. \vskip 17pt
 \hrule
 \par
\vskip 21pt plus 6pt minus 3pt }

\label{st\stat}

\def\tit{\ }

\def\aut{\ }
\def\auf{\ }

\def\leftkol{\ } % ENGLISH ABSTRACTS}

\def\rightkol{\ } %АВТОРСКИЙ УКАЗАТЕЛЬ ЗА 2018 г.} %ENGLISH ABSTRACTS}

\titele{\tit}{\aut}{\auf}{\leftkol}{\rightkol}
\addcontentsline{toc}{subsection}{\textrm\textbf Авторский указатель за 2018 г.}

\vspace*{-12pt}
\vspace*{-36pt}

\noindent
{\tabcolsep=3pt
\begin{tabular}{p{397pt}cc}
&\textbf{Вып.} & \textbf{Стр.}\\[6pt]
\Avtors{Агаларов~Я.\,М.} Оптимизация объема буферной памяти узла коммутации при схеме\linebreak
\\[-12pt]
\hspace*{23pt}полного разделения памяти&4&25--32\\
\Avtors{Агасандян~Г.\,А.} Континуальный критерий VaR на сценарных рынках&1&31--39\\
\Avtors{Алешин~И.\,С.} О формальной постановке задач поиска сгущений в разреженных булевых\linebreak
\\[-12pt]
\hspace*{23pt}матрицах&1&40--48\\
\Avtors{Арутюнов~Е.\,Н., Кудрявцев~А.\,А., Титова~А.\,И.} Гамма-вейбулловский случай априорных\linebreak
\\[-12pt]
\hspace*{23pt}распределений в~байесовских моделях массового обслуживания&4&92--95\\
\Avtors{Атаева~О.\,М., Серебряков~В.\,А.} Онтология цифровой семантической библиотеки LibMeta&1&\hphantom{1}2--10\\
\Avtors{Басок~Б.\,М., Захаров~В.\,Н., Френкель~С.\,Л.} Использование вероятностной модели вычислений для тестирования одного класса готовых к~использованию программных\linebreak
\\[-12pt]
\hspace*{23pt}компонентов локальных и~сетевых систем&4&44--51\\
\Avtors{Батенков~А.\,А., Маньяков~Ю.\,А., Гасилов~А.\,В., Яковлев~О.\,А.} Математическая модель\linebreak
\\[-12pt]
\hspace*{23pt}оптимальной триангуляции&2&69--74\\
\Avtors{Бахтеев~О.\,Ю.} см.~Огальцов~А.\,В.&&\\
\Avtors{Бахтеев~О.\,Ю.} см.~Смердов~А.\,Н.&&\\
\Avtors{Борисов~А.\,В.} Фильтрация состояний марковских скачкообразных процессов по дискре-\linebreak
\\[-12pt]
\hspace*{23pt}тизованным наблюдениям&3&115--121\\
\Avtors{Босов~А.\,В., Игнатов~А.\,Н., Наумов~А.\,В.} Модель передвижения поездов и маневровых локомотивов на железнодорожной станции в приложении к оценке и анализу\linebreak
\\[-12pt]
\hspace*{23pt}вероятности бокового столкновения&3&107--114\\
\Avtors{Босов~А.\,В., Стефанович~А.\,И.} Управление выходом стохастической дифференциальной системы по квадратичному критерию. I.~Оптимальное решение методом динами-\linebreak
\\[-12pt]
\hspace*{23pt}ческого программирования&3&\hphantom{1}99--106\\
\Avtors{Бунтман~Н.\,В., Гончаров~А.\,А., Зацман~И.\,М., Нуриев~В.\,А.} Количественный анализ\linebreak
\\[-12pt]
\hspace*{23pt}результатов машинного перевода с~использованием надкорпусных баз данных&4&\hphantom{1}96--105\\
\Avtors{Бунтман~Н.\,В.} см.~Нуриев~В.\,А.&&\\
\Avtors{Быковец~Е.\,В., Лаврентьев~В.\,В., Назаров~Л.\,В.} Вероятностная модель влияния книги\linebreak
\\[-12pt]
\hspace*{23pt}заказов на процесс цены&2&29--34\\
\Avtors{Васильева~С.\,Н., Кан~Ю.\,С.} Алгоритм визуализации плоского ядра вероятностной меры&2&60--68\\
\Avtors{Виноградов~Д.\,В.} Учет предварительных оценок скорости порождения сходств спарива-\linebreak
\\[-12pt]
\hspace*{23pt}ющей цепью Маркова&1&49--54\\
\Avtors{Вышинский~Л.\,Л., Флеров~Ю.\,А., Широков~Н.\,И.} Автоматизированная система весового\linebreak
\\[-12pt]
\hspace*{23pt}проектирования самолетов&1&18--30\\
\Avtors{Гайдамака~Ю.\,В.} см.~Горбунова~А.\,В.&&\\
\Avtors{Гайдамака~Ю.\,В.} см.~Самуйлов~К.\,Е.&&\\
\Avtors{Гасилов~А.\,В.} см.~Батенков~А.\,А.,&&\\
\Avtors{Гончаров~А.\,А.} см.~Бунтман~Н.\,В.&&\\
\Avtors{Горбунова~А.\,В., Наумов~В.\,А., Гайдамака~Ю.\,В., Самуйлов~К.\,Е.} Ресурсные системы\linebreak
\\[-12pt]
\hspace*{23pt}массового обслуживания как модели беспроводных систем связи&3&48--55\\
\Avtors{Горшенин~А.\,К.} Зашумление данных конечными смесями нормальных и гамма-рас\-пре-\linebreak
\\[-12pt]
\hspace*{23pt}де\-ле\-ний с применением к задаче округления наблюдений&3&28--34\\
\Avtors{Горшенин~А.\,К.} Развитие сервисов цифровых платформ для преодоления нефинансовых\linebreak
\\[-12pt]
\hspace*{23pt}барьеров&4&106--112\\
\Avtors{Горшенин~А.\,К., Королев~В.\,Ю.} Определение экстремальности объемов осадков на основе\linebreak
\\[-12pt]
\hspace*{23pt}модифицированного метода превышения порогового значения&4&16--24\\
\Avtors{Горшенин~А.\,К.} см.~Королев~В.\,Ю.&&\\
\end{tabular}
}

\pagebreak

\def\leftkol{АВТОРСКИЙ УКАЗАТЕЛЬ ЗА 2018 г.} % ENGLISH ABSTRACTS}

\def\rightkol{АВТОРСКИЙ УКАЗАТЕЛЬ ЗА 2018 г.} %ENGLISH ABSTRACTS}

%\thispagestyle{myheadings}
\def\leftfootline{\small{\textbf{\thepage}
\hfill ИНФОРМАТИКА И ЕЁ ПРИМЕНЕНИЯ\ \ \ том~12\ \ \ выпуск~4\ \ \ 2018}
}%
 \def\rightfootline{\small{ИНФОРМАТИКА И ЕЁ ПРИМЕНЕНИЯ\ \ \ том~12\ \ \ выпуск~4\ \ \ 2018
 \hfill \textbf{\thepage}}}


\noindent
{\tabcolsep=3pt
\begin{tabular}{p{394pt}cc}
&\textbf{Вып.} & \textbf{Стр.}\\[3pt]
\Avtors{Грушо~А.\,А., Грушо~Н.\,А., Забежайло~М.\,И., Смирнов~Д.\,В., Тимонина~Е.\,Е.} Параметриза-\linebreak
\\[-12pt]
\hspace*{23pt}ция в прикладных задачах поиска эмпирических причин&3&62--66\\
\Avtors{Грушо~А.\,А., Грушо~Н.\,А., Левыкин~М.\,В., Тимонина~Е.\,Е.} Методы идентификации захвата хоста в~распределенной информационно-вычислительной сис\-те\-ме, защищенной\linebreak
\\[-12pt]
\hspace*{23pt}с~по\-мощью метаданных&4&39--43\\
\Avtors{Грушо~А.\,А., Забежайло~М.\,И., Зацаринный~А.\,А., Тимонина~Е.\,Е.} О некоторых возможностях управления ресурсами при организации проактивного противодействия\linebreak
\\[-12pt]
\hspace*{23pt}компьютерным атакам&1&62--70\\
\Avtors{Грушо~А.\,А., Тимонина~Е.\,Е., Шоргин~С.\,Я.} Иерархический метод порождения метадан-\linebreak
\\[-12pt]
\hspace*{23pt}ных для управления сетевыми соединениями&2&44--49\\
\Avtors{Грушо~Н.\,А.} см.~Грушо~А.\,А.&&\\
\Avtors{Грушо~Н.\,А.} см.~Грушо~А.\,А.&&\\
\Avtors{Дорофеева~А.\,В.} см.~Королев~В.\,Ю.&&\\
\Avtors{Егоров~А.\,Ю.} см.~Шнурков~П.\,В.&&\\
\Avtors{Егоров~А.\,Ю.} см.~Шнурков~П.\,В.&&\\
\Avtors{Жуков~Д.\,О., Хватова~Т.\,Ю., Лесько~С.\,А., Зальцман~А.\,Д.} Влияние плотности связей на кластеризацию и порог перколяции при распространении информации в~со\-ци\-аль-\linebreak
\\[-12pt]
\hspace*{23pt}ных сетях&2&90--97\\
\Avtors{Забежайло~М.\,И.} см.~Грушо~А.\,А.&&\\
\Avtors{Забежайло~М.\,И.} см.~Грушо~А.\,А.&&\\
\Avtors{Зальцман~А.\,Д.} см.~Жуков~Д.\,О.&&\\
\Avtors{Захаров~В.\,Н.} см.~Басок~Б.\,М.&&\\
\Avtors{Захаров~В.\,Н.} см.~Шанин~И.\,А.&&\\
\Avtors{Зацаринный~А.\,А., Сучков~А.\,П.} Система ситуационного управления как мультисервисная\linebreak
\\[-12pt]
\hspace*{23pt}технология в облачной среде&1&78--88\\
\Avtors{Зацаринный~А.\,А.} см.~Грушо~А.\,А.&&\\
\Avtors{Зацман~И.\,М.} Имплицированные знания: основания и технологии извлечения&3&74--82\\
\Avtors{Зацман~И.\,М.} см.~Бунтман~Н.\,В.&&\\
\Avtors{Зейфман~А.\,И.} см.~Королев~В.\,Ю.&&\\
\Avtors{Зубарев~Д.\,В.} см.~Соченков~И.\,В.&&\\
\Avtors{Игнатов~А.\,Н.} см.~Босов~А.\,В.&&\\
\Avtors{Инькова~О.\,Ю., Кружков~М.\,Г.} Статистический анализ лингвоспецифичности коннек-\linebreak
\\[-12pt]
\hspace*{23pt}торов (на материале параллельных корпусов)&3&83--90\\
\Avtors{Инькова~О.\,Ю.} см.~Нуриев~В.\,А.&&\\
\Avtors{Кан~Ю.\,С.} см.~Васильева~С.\,Н.&&\\
\Avtors{Ковалёв~С.\,П.} Теория категорий как математическая прагматика модельно-ори\-ен\-ти-\linebreak
\\[-12pt]
\hspace*{23pt}ро\-ван\-ной системной инженерии&1&\hphantom{1}95--104\\
\Avtors{Козеренко~Е.\,Б., Кузнецов~К.\,И., Романов~Д.\,А.} Семантическая обработка неструктури-\linebreak
\\[-12pt]
\hspace*{23pt}рованных текстовых данных на основе лингвистического процессора PullEnti&3&91--98\\
\Avtors{Кондранин~Е.\,С., Ушаков~В.\,Г.} Система обслуживания с~относительным приоритетом\linebreak
\\[-12pt]
\hspace*{23pt}и~профилактиками прибора&4&33--38\\
\Avtors{Коновалов~М.\,Г., Разумчик~Р.\,В.} Сравнение двух механизмов активного управления\linebreak
\\[-12pt]
\hspace*{23pt}очередью в~системе $M/D/1/N$&4&\hphantom{1}9--15\\
\Avtors{Коновалов~М.\,Г., Разумчик~Р.\,В.} Управление случайным блужданием с эталонным\linebreak
\\[-12pt]
\hspace*{23pt}стационарным распределением&3&\hphantom{1}2--13\\
\Avtors{Королев~В.\,Ю., Горшенин~А.\,К., Зейфман~А.\,И.} Новые представления обобщенного\linebreak
\\[-12pt]
\hspace*{23pt}распределения Миттаг-Леффлера в~виде смесей и~их приложения&4&75--85\\
\Avtors{Королев~В.\,Ю., Дорофеева~А.\,В.} О~неравномерных оценках точности нормальной аппроксимации для распределений некоторых случайных сумм при ослабленных\linebreak
\\[-12pt]
\hspace*{23pt}моментных условиях&4&86--91\\
\Avtors{Королев~В.\,Ю.} см.~Горшенин~А.\,К.&&\\
\Avtors{Кривенко~М.\,П.}\ Обучаемая классификация данных с учетом анализа главных компонент&3&56--61\\
\Avtors{Кривенко~М.\,П.}\ Реконструкция осей главных компонент&1&71--77\\
\Avtors{Кружков~М.\,Г.} см.~Инькова~О.\,Ю.&&\\
\end{tabular}
}

\pagebreak

\def\leftkol{АВТОРСКИЙ УКАЗАТЕЛЬ ЗА 2018 г.} % ENGLISH ABSTRACTS}

\def\rightkol{АВТОРСКИЙ УКАЗАТЕЛЬ ЗА 2018 г.} %ENGLISH ABSTRACTS}

%\thispagestyle{myheadings}
\def\leftfootline{\small{\textbf{\thepage}
\hfill ИНФОРМАТИКА И ЕЁ ПРИМЕНЕНИЯ\ \ \ том~12\ \ \ выпуск~4\ \ \ 2018}
}%
 \def\rightfootline{\small{ИНФОРМАТИКА И ЕЁ ПРИМЕНЕНИЯ\ \ \ том~12\ \ \ выпуск~4\ \ \ 2018
 \hfill \textbf{\thepage}}}


\noindent
{\tabcolsep=3pt
\begin{tabular}{p{394pt}cc}
&\textbf{Вып.} & \textbf{Стр.}\\[3pt]
\Avtors{Кудрявцев~А.\,А.} Байесовские модели баланса&3&18--27\\
\Avtors{Кудрявцев~А.\,А., Шестаков~О.\,В.} Байесовские модели тестирования больших групп\linebreak
\\[-12pt]
\hspace*{23pt}обслуживающих приборов&1&105--108\\
\Avtors{Кудрявцев~А.\,А., Шестаков О.\,В.} Минимизация ошибок вычисления вейвлет-ко\-эф\-фи-\linebreak
\\[-12pt]
\hspace*{23pt}ци\-ен\-тов при решении обратных задач&2&17--23\\
\Avtors{Кудрявцев~А.\,А.} см.~Арутюнов~Е.\,Н.&&\\
\Avtors{Кузнецов~К.\,И.} см.~Козеренко~Е.\,Б.&&\\
\Avtors{Лаврентьев~В.\,В.} см.~Быковец~Е.\,В.&&\\
\Avtors{Лебедев~А.\,В.} Максимальные ветвящиеся процессы в случайной среде&2&35--43\\
\Avtors{Левыкин~М.\,В.} см.~Грушо~А.\,А.&&\\
\Avtors{Лери~М.\,М., Павлов~Ю.\,Л.} Об устойчивости конфигурационных графов в случайной\linebreak
\\[-12pt]
\hspace*{23pt}среде&2&\hphantom{1}2--10\\
\Avtors{Лесько~С.\,А.} см.~Жуков~Д.\,О.&&\\
\Avtors{Логачев~О.\,А.} Теоретико-информационная характеризация совершенно уравновешен-\linebreak
\\[-12pt]
\hspace*{23pt}ных функций&4&70--74\\
\Avtors{Малашенко~Ю.\,Е., Назарова~И.\,А., Новикова~Н.\,М.} Анализ разрезных повреждений\linebreak
\\[-12pt]
\hspace*{23pt}в~многополюсных сетях&3&35--41\\
\Avtors{Малашенко~Ю.\,Е., Назарова~И.\,А., Новикова~Н.\,М.} Диаграммы уязвимости потоковых\linebreak
\\[-12pt]
\hspace*{23pt}сетевых систем&1&11--17\\
\Avtors{Маньяков~Ю.\,А.} см.~Батенков~А.\,А.&&\\
\Avtors{Мирзабеков~Я.\,М., Шихиев~Ш.\,Б.} Дискретный анализ в синтаксическом анализе&2&\hphantom{1}98--104\\
\Avtors{Мистрюков~А.\,В., Ушаков~В.\,Г.} Достаточные условия эргодичности приоритетных\linebreak
\\[-12pt]
\hspace*{23pt}систем массового обслуживания&2&24--28\\
\Avtors{Назаров~Л.\,В.} см.~Быковец~Е.\,В.&&\\
\Avtors{Назарова~И.\,А.} см.~Малашенко~Ю.\,Е.&&\\
\Avtors{Назарова~И.\,А.} см.~Малашенко~Ю.\,Е.&&\\
\Avtors{Наумов~А.\,В.} см.~Босов~А.\,В.&&\\
\Avtors{Наумов~В.\,А.} см.~Горбунова~А.\,В.&&\\
\Avtors{Наумов~В.\,А.} см.~Сопин~Э.\,С.&&\\
\Avtors{Новикова~Н.\,М.} см.~Малашенко~Ю.\,Е.&&\\
\Avtors{Новикова~Н.\,М.} см.~Малашенко~Ю.\,Е.&&\\
\Avtors{Нуриев~В.\,А., Бунтман~Н.\,В., Инькова~О.\,Ю.} Ошибки и~неточности машинного перевода\linebreak
\\[-12pt]
\hspace*{23pt}русских коннекторов на~французский язык&2&105--113\\
\Avtors{Нуриев~В.\,А.} см.~Бунтман~Н.\,В.&&\\
\Avtors{Огальцов~А.\,В., Бахтеев~О.\,Ю.} Автоматическое извлечение метаданных из научных\linebreak
\\[-12pt]
\hspace*{23pt}PDF-документов&2&75--82\\
\Avtors{Павлов~Ю.\,Л.} см.~Лери~М.\,М.&&\\
\Avtors{Разумчик~Р.\,В.} см.~Коновалов~М.\,Г.&&\\
\Avtors{Разумчик~Р.\,В.} см.~Коновалов~М.\,Г.&&\\
\Avtors{Романов~Д.\,А.} см.~Козеренко~Е.\,Б.&&\\
\Avtors{Самуйлов~К.\,Е., Гайдамака~Ю.\,В., Шоргин~С.\,Я.} Применение моделей случайного\linebreak
\\[-12pt]
\hspace*{23pt}блуждания при моделировании перемещения устройств в~беспроводной сети&4&2--8\\
\Avtors{Самуйлов~К.\,Е.} см.~Горбунова~А.\,В.&&\\
\Avtors{Самуйлов~К.\,Е.} см.~Сопин~Э.\,С.&&\\
\Avtors{Серебряков~В.\,А.} см.~Атаева~О.\,М.&&\\
\Avtors{Синицын~И.\,Н.} Метод интерполяционного аналитического моделирования одномерных\linebreak
\\[-12pt]
\hspace*{23pt}распределений в стохастических системах&1&55--61\\
\Avtors{Смердов~А.\,Н., Бахтеев~О.\,Ю., Стрижов~В.\,В.} Выбор оптимальной модели рекуррентной\linebreak
\\[-12pt]
\hspace*{23pt}сети в~задачах поиска парафраза&4&63--69\\
\Avtors{Смирнов~Д.\,В.} см.~Грушо~А.\,А.&&\\
\Avtors{Сопин~Э.\,С., Наумов~В.\,А., Самуйлов~К.\,Е.} Об инвариантности стационарного распределения системы массового обслуживания с ограниченными ресурсами и~с~ин\-тен-\linebreak
\\[-12pt]
\hspace*{23pt}сив\-ностями поступления и~обслуживания, зависящими от состояния системы&3&42--47\\
\Avtors{Соченков~И.\,В., Зубарев~Д.\,В., Тихомиров~И.\,А.} Эксплоративный патентный поиск&1&89--94\\
\end{tabular}
}

\pagebreak

\def\leftkol{АВТОРСКИЙ УКАЗАТЕЛЬ ЗА 2018 г.} % ENGLISH ABSTRACTS}

\def\rightkol{АВТОРСКИЙ УКАЗАТЕЛЬ ЗА 2018 г.} %ENGLISH ABSTRACTS}

%\thispagestyle{myheadings}
\def\leftfootline{\small{\textbf{\thepage}
\hfill ИНФОРМАТИКА И ЕЁ ПРИМЕНЕНИЯ\ \ \ том~12\ \ \ выпуск~4\ \ \ 2018}
}%
 \def\rightfootline{\small{ИНФОРМАТИКА И ЕЁ ПРИМЕНЕНИЯ\ \ \ том~12\ \ \ выпуск~4\ \ \ 2018
 \hfill \textbf{\thepage}}}


\noindent
{\tabcolsep=3pt
\begin{tabular}{p{394pt}cc}
&\textbf{Вып.} & \textbf{Стр.}\\[3pt]
\Avtors{Стефанович~А.\,И.} см.~Босов~А.\,В.&&\\
\Avtors{Стрижов~В.\,В.} см.~Смердов~А.\,Н.&&\\
\Avtors{Ступников~С.\,А.} см.~Шанин~И.\,А.&&\\
\Avtors{Сурина~А.\,А.} см.~Тырсин~А.\,Н.&&\\
\Avtors{Сучков~А.\,П.} см.~Зацаринный~А.\,А.&&\\
\Avtors{Сюнтюренко~О.\,В.} Финансирование фундаментальных исследований: концептуальный облик системы поддержки принятия решений с использованием методов\linebreak
\\[-12pt]
\hspace*{23pt}наукометрии и анализа данных&1&118--127\\
\Avtors{Тимонина~Е.\,Е.} см.~Грушо~А.\,А.&&\\
\Avtors{Тимонина~Е.\,Е.} см.~Грушо~А.\,А.&&\\
\Avtors{Тимонина~Е.\,Е.} см.~Грушо~А.\,А.&&\\
\Avtors{Тимонина~Е.\,Е.} см.~Грушо~А.\,А.&&\\
\Avtors{Титова~А.\,И.} см.~Арутюнов~Е.\,Н.&&\\
\Avtors{Тихомиров~И.\,А.} см.~Соченков~И.\,В.&&\\
\Avtors{Тырсин~А.\,Н., Сурина~А.\,А.} Модели управления риском в гауссовских стохастических\linebreak
\\[-12pt]
\hspace*{23pt}системах&2&50--59\\
\Avtors{Ушаков~В.\,Г.} см.~Кондранин~Е.\,С.&&\\
\Avtors{Ушаков~В.\,Г.} см.~Мистрюков~А.\,В.&&\\
\Avtors{Флеров~Ю.\,А.} см.~Вышинский~Л.\,Л.&&\\
\Avtors{Френкель~С.\,Л., Ханкин~Д.} Непрерывные обновления маршрута в~SDN с~использованием\linebreak
\\[-12pt]
\hspace*{23pt}проверки соответствия качеству обслуживания&4&52--62\\
\Avtors{Френкель~С.\,Л.} см.~Басок~Б.\,М.&&\\
\Avtors{Ханкин~Д.} см.~Френкель~С.\,Л.&&\\
\Avtors{Хватова~Т.\,Ю.} см.~Жуков~Д.\,О.&&\\
\Avtors{Шанин~И.\,А., Ступников~С.\,А., Захаров~В.\,Н.} Методы и средства обнаружения нештатных\linebreak
\\[-12pt]
\hspace*{23pt}ситуаций, возникающих на элементах жилищно-коммунальной инфраструктуры&3&67--73\\
\Avtors{Шестаков~О.\,В.} Несмещенная оценка риска стабилизированной жесткой пороговой\linebreak
\\[-12pt]
\hspace*{23pt}обработки в модели с долгосрочной зависимостью&2&11--16\\
\Avtors{Шестаков~О.\,В.} Среднеквадратичный риск пороговой обработки при случайном объеме\linebreak
\\[-12pt]
\hspace*{23pt}выборки&3&14--17\\
\Avtors{Шестаков~О.\,В.} см.~Кудрявцев~А.\,А.&&\\
\Avtors{Шестаков~О.\,В.} см.~Кудрявцев~А.\,А.&&\\
\Avtors{Широков~Н.\,И.} см.~Вышинский~Л.\,Л.&&\\
\Avtors{Шихиев~Ш.\,Б.} см.~Мирзабеков~Я.\,М.&&\\
\Avtors{Шнурков~П.\,В., Егоров~А.\,Ю.} Разработка и предварительное исследование стохастической полумарковской модели управления запасом непрерывного продукта при\linebreak
\\[-12pt]
\hspace*{23pt}постоянно происходящем потреблении&1&109--117\\
\Avtors{Шнурков~П.\,В., Егоров~А.\,Ю.} Решение проблемы оптимального управления запасом непрерывного продукта при постоянно происходящем потреблении в стохастической\linebreak
\\[-12pt]
\hspace*{23pt}полумарковской модели&2&83--89\\
\Avtors{Шоргин~С.\,Я.} см.~Грушо~А.\,А.&&\\
\Avtors{Шоргин~С.\,Я.} см.~Самуйлов~К.\,Е.&&\\
\Avtors{Яковлев~О.\,А.} см.~Батенков~А.\,А.&&\\
\end{tabular}
}

%\thispagestyle{myheadings}
\def\leftfootline{\small{\textbf{\thepage}
\hfill ИНФОРМАТИКА И ЕЁ ПРИМЕНЕНИЯ\ \ \ том~12\ \ \ выпуск~4\ \ \ 2018}
}%
 \def\rightfootline{\small{ИНФОРМАТИКА И ЕЁ ПРИМЕНЕНИЯ\ \ \ том~12\ \ \ выпуск~4\ \ \ 2018
 \hfill \textbf{\thepage}}}

 \label{end\stat}

\newpage

%Информатика и её применения
%Том 12   Выпуск 1-4   Год 2018

\def\stat{cont-e}
{%\hrule\par
%\vskip 7pt % 7pt
\raggedleft\Large \bf%\baselineskip=3.2ex
2\,0\,1\,8\ \ A\,U\,T\,H\,O\,R\ \ I\,N\,D\,E\,X \vskip 17pt
 \hrule
 \par
\vskip 21pt plus 6pt minus 3pt }

\label{st\stat}

\def\tit{\ }

\def\aut{\ }
\def\auf{\ }

\def\leftkol{\ } %2018 AUTHOR INDEX} % ENGLISH ABSTRACTS}

\def\rightkol{\ } %2018 AUTHOR INDEX} %ENGLISH ABSTRACTS}

\titele{\tit}{\aut}{\auf}{\leftkol}{\rightkol}
\addcontentsline{toc}{subsection}{\textrm\textbf 2018 Author Index}

\def\leftfootline{\small{\textbf{\thepage}
\hfill INFORMATIKA I EE PRIMENENIYA~--- INFORMATICS AND APPLICATIONS\ \ \ 2018\
\ \ volume~12\ \ \ issue\ 4}
}%
 \def\rightfootline{\small{INFORMATIKA I EE PRIMENENIYA~--- INFORMATICS AND APPLICATIONS\ \ \ 2018\ \ \ volume~12\ \ \ issue\ 4
\hfill \textbf{\thepage}}}

\vspace*{-12pt}
\vspace*{-18pt}

\noindent
{\tabcolsep=3pt
\begin{tabular}{p{396pt}cc}
&\textbf{Issue} & \textbf{Page}\\[6pt]
\Avtors{Agalarov~Yа.\,M.} Optimization of buffer memory size of switching node in mode of full memory\linebreak
\\[-12pt]
\hspace*{23pt}sharing&4&25--32\\
\Avtors{Agasandyan~G.\,A.} Continuous VaR-criterion in scenario markets&1&31--39\\
\Avtors{Aleshin~I.\,S.} On the formalization of tasks searching dense submatrices in boolean sparse\linebreak
\\[-12pt]
\hspace*{23pt}matrices&1&40--48\\
\Avtors{Arutyunov~E.\,N., Kudryavtsev~A.\,A., and~Titova~A.\,I.} Gamma-Weibull \textit{a~priori} distributions\linebreak
\\[-12pt]
\hspace*{23pt}in~Bayesian queuing models&4&92--95\\
\Avtors{Ataeva~O.\,M.} see~Serebryakov~V.\,A.&&\\
\Avtors{Bakhteev~O.\,Y.} see~Ogaltsov~A.\,V.&&\\
\Avtors{Bakhteev~O.\,Y.} see~Smerdov~A.\,N.&&\\
\Avtors{Basok~B.\,M., Zakharov~V.\,N., and~Frenkel~S.\,L.} Using a probabilistic calculation model to test\linebreak
\\[-12pt]
\hspace*{23pt}one class of ready-to-use software components of local and network systems&4&44--51\\
\Avtors{Batenkov~A.\,A., Maniakov Yu.\,A., Gasilov A.\,V., and Yakovlev O.\,A.} Mathematical model\linebreak
\\[-12pt]
\hspace*{23pt}of~optimal triangulation&2&69--74\\
\Avtors{Borisov~A.\,V.} Filtering of Markov jump processes by discretized observations&3&115--121\\
\Avtors{Bosov~A.\,V., Ignatov~A.\,N., and Naumov~A.\,V.} Model of transportation of trains and shunting\linebreak
\\[-12pt]
\hspace*{23pt}locomotives at a railway station for evaluation and analysis of side-collision probability&3&107--114\\
\Avtors{Bosov~A.\,V.\ and Stefanovich~A.\,I.} Stochastic differential system output control by the quadratic\linebreak
\\[-12pt]
\hspace*{23pt}criterion. I.~Dynamic programming optimal solution&3&\hphantom{1}99--106\\
\Avtors{Buntman~N.\,V., Goncharov~A.\,A., Zatsman~I.\,M., and~Nuriev~V.\,A.} Using supracorpora databases\linebreak
\\[-12pt]
\hspace*{23pt}for quantitative analysis of machine translations&4&\hphantom{1}96--105\\
\Avtors{Buntman~N.\,V.} see~Nuriev~V.\,A., &&\\
\Avtors{Bykovets~E.\,V.} see~Nazarov~L.\,V.&&\\
\Avtors{Dorofeeva~A.\,V.} see~Korolev~V.\,Yu.&&\\
\Avtors{Egorov~A.\,Y.} see~Shnurkov~P.\,V.&&\\
\Avtors{Egorov~A.\,Y.} see~Shnurkov~P.\,V.&&\\
\Avtors{Flerov~Yu.\,A.} see~Vyshinsky~L.\,L.&&\\
\Avtors{Frenkel~S.\,L.\ and Khankin~D.} Seamless route updates in software-defined networking via quality\linebreak
\\[-12pt]
\hspace*{23pt}of~service compliance verification &4&52--62\\
\Avtors{Frenkel~S.\,L.} see~Basok~B.\,M.&&\\
\Avtors{Gaidamaka~Yu.\,V.} see~Gorbunova~A.\,V.&&\\
\Avtors{Gaidamaka~Yu.\,V.} see~Samouylov~K.\,E.&&\\
\Avtors{Gasilov A.\,V.} see~Batenkov~A.\,A.&&\\
\Avtors{Goncharov~A.\,A.} see~Buntman~N.\,V.&&\\
\Avtors{Gorbunova~A.\,V., Naumov~V.\,A., Gaidamaka~Yu.\,V., and Samouylov~K.\,E.} Resource queuing\linebreak
\\[-12pt]
\hspace*{23pt}systems as models of wireless communication systems&3&48--55\\
\Avtors{Gorshenin~A.\,K.} Data noising by finite normal and gamma mixtures with application to~the~prob-\linebreak
\\[-12pt]
\hspace*{23pt}lem of rounded observations&3&28--34\\
\Avtors{Gorshenin~A.\,K.} Development of services of digital platforms to overcome nonfinancial barriers&4&106--112\\
\Avtors{Gorshenin~A.\,K.\ and~Korolev~V.\,Yu.} Determining the extremes of precipitation volumes based\linebreak
\\[-12pt]
\hspace*{23pt}on~the~modified ``Peaks over Threshold'' method&4&16--24\\
\Avtors{Gorshenin~A.\,K.} see~Korolev~V.\,Yu.&&\\
\Avtors{Grusho~A.\,A., Grusho~N.\,A., Levykin~M.\,V., and~Timonina~E.\,E.} Methods of identification of host\linebreak
\\[-12pt]
\hspace*{23pt}capture in a distributed information system which is protected on the basis of meta data&4&39--43\\
\Avtors{Grusho~A.\,A., Grusho~N.\,A., Zabezhailo~M.\,I., Smirnov~D.\,V., and Timonina~E.\,E.} Parametrization\linebreak
\\[-12pt]
\hspace*{23pt}in applied problems of search of empirical reasons&3&62--66\\
\end{tabular}
}
\pagebreak

\def\leftfootline{\small{\textbf{\thepage}
\hfill INFORMATIKA I EE PRIMENENIYA~--- INFORMATICS AND APPLICATIONS\ \ \ 2018\
\ \ volume~12\ \ \ issue\ 4}
}%
 \def\rightfootline{\small{INFORMATIKA I EE PRIMENENIYA~---
INFORMATICS AND APPLICATIONS\ \ \ 2018\ \ \ volume~12\ \ \ issue\ 4
\hfill \textbf{\thepage}}}

\def\leftkol{2018 AUTHOR INDEX} % ENGLISH ABSTRACTS}

\def\rightkol{2018 AUTHOR INDEX} %ENGLISH ABSTRACTS}


\noindent
{\tabcolsep=3pt
\begin{tabular}{p{395.48108pt}cc}
&\textbf{Issue} & \textbf{Page}\\[6pt]
\Avtors{Grusho~A.\,A., Timonina~E.\,E., and Shorgin~S.\,Ya.} Hierarchical method of meta data generation\linebreak
\\[-12pt]
\hspace*{23pt}for control of network connections&2&44--49\\
\Avtors{Grusho~A.\,A., Zabezhailo~M.\,I., Zatsarinny~A.\,A., and Timonina~E.\,E.} On some possibilities\linebreak
\\[-12pt]
\hspace*{23pt}of~resource management for organizing active counteraction to computer attacks&1&62--70\\
\Avtors{Grusho~N.\,A.} see~Grusho~A.\,A.&&\\
\Avtors{Grusho~N.\,A.} see~Grusho~A.\,A.&&\\
\Avtors{Ignatov~A.\,N.} see~Bosov~A.\,V.&&\\
\Avtors{Inkova~O.\,Yu.\ and Kruzhkov~M.\,G.} Statistical analysis of language specificity of connectives\linebreak
\\[-12pt]
\hspace*{23pt}based on parallel texts&3&83--90\\
\Avtors{Inkova~O.\,Yu.} see~Nuriev~V.\,A., &&\\
\Avtors{Kan~Yu.\,S.} see~Vasil'eva~S.\,N.&&\\
\Avtors{Khankin~D.} see~Frenkel~S.\,L.&&\\
\Avtors{Khvatova~T.\,Yu.} see~Zhukov~D.\,O.&&\\
\Avtors{Kondranin~E.\,S.\ and~Ushakov~V.\,G.} A~head of the line priority queue with working vacations&4&33--38\\
\Avtors{Konovalov~M.\,G.\ and Razumchik~R.\,V.} Comparison of two active queue management schemes\linebreak
\\[-12pt]
\hspace*{23pt}through the $M/D/1/N$ queue&4&\hphantom{1}9--15\\
\Avtors{Konovalov~M.\,G.\ and Razumchik~R.\,V.} Finding control policy for one discrete-time Markov\linebreak
\\[-12pt]
\hspace*{23pt}chain on [0,1] with a given invariant measure&3&\hphantom{1}2--13\\
\Avtors{Korolev~V.\,Yu.\ and~Dorofeeva~A.\,V.} On nonuniform estimates of accuracy of normal approxima-\linebreak
\\[-12pt]
\hspace*{23pt}tion for distributions of some random sums under relaxed moment conditions&4&86--91\\
\Avtors{Korolev~V.\,Yu., Gorshenin~A.\,K., and~Zeifman~A.\,I. } New mixture representations of~the~general-\linebreak
\\[-12pt]
\hspace*{23pt}ized Mittag-Leffler distribution and their applications&4&75--85\\
\Avtors{Korolev~V.\,Yu.} see~Gorshenin~A.\,K.&&\\
\Avtors{Kovalyov~S.\,P.} Category theory as a mathematical pragmatics of model-based systems engineer-\linebreak
\\[-12pt]
\hspace*{23pt}ing&1&\hphantom{1}95--104\\
\Avtors{Kozerenko~E.\,B., Kuznetsov~K.\,I., and Romanov~D.\,A.} Semantic processing of unstructured\linebreak
\\[-12pt]
\hspace*{23pt}textual data based on the linguistic processor PullEnti&3&91--98\\
\Avtors{Krivenko~M.\,P.} Principal axes reconstruction&1&71--77\\
\Avtors{Krivenko~M.\,P.} Supervised learning classification of data taking into account principal compo-\linebreak
\\[-12pt]
\hspace*{23pt}nent analysis&3&56--61\\
\Avtors{Kruzhkov~M.\,G.} see~Inkova~O.\,Yu.&&\\
\Avtors{Kudryavtsev~A.\,A.} Bayesian balance models&3&18--27\\
\Avtors{Kudryavtsev~A.\,A.\ and Shestakov~O.\,V.} Bayesian models for testing large groups of service devices&1&105--108\\
\Avtors{Kudryavtsev~A.\,A.\ and Shestakov~O.\,V.} Minimization of errors of calculating wavelet coefficients\linebreak
\\[-12pt]
\hspace*{23pt}while solving inverse problems&2&17--23\\
\Avtors{Kudryavtsev~A.\,A.} see~Arutyunov~E.\,N.&&\\
\Avtors{Kuznetsov~K.\,I.} see~Kozerenko~E.\,B.&&\\
\Avtors{Lavrentyev~V.\,V.} see~Nazarov~L.\,V.&&\\
\Avtors{Lebedev~A.\,V.} Maximal branching processes in random environment&2&35--43\\
\Avtors{Leri~M.\,M.\ and Pavlov~Yu.\,L.} On the robustness of configuration graphs in a random environment&2&\hphantom{1}2--10\\
\Avtors{Lesko~S.\,A.} see~Zhukov~D.\,O.&&\\
\Avtors{Levykin~M.\,V.} see~Grusho~A.\,A.&&\\
\Avtors{Logachev~O.\,A.} An information based criterion for perfectly balanced functions&4&70--74\\
\Avtors{Malashenko~Yu.\,E., Nazarova~I.\,A., and Novikova~N.\,M.} Analysis of cutting damages to multipolar\linebreak
\\[-12pt]
\hspace*{23pt}networks&3&35--41\\
\Avtors{Malashenko~Yu.\,E., Nazarova~I.\,A., and Novikova~N.\,M.} Diagrams of the functional vulnerability\linebreak
\\[-12pt]
\hspace*{23pt}of flow network systems&1&11--17\\
\Avtors{Maniakov Yu.\,A.} see~Batenkov~A.\,A.&&\\
\Avtors{Mirzabekov~Ya.\,M.\ and Shihiev~Sh.\,B.} Discrete analysis in parsing&2&\hphantom{1}98--104\\
\Avtors{Mistryukov~A.\,V.\ and Ushakov~V.\,G.} Sufficient ergodicity conditions for priority queues&2&24--28\\
\Avtors{Naumov~A.\,V.} see~Bosov~A.\,V.&&\\
\Avtors{Naumov~V.\,A.} see~Gorbunova~A.\,V.&&\\
\Avtors{Naumov~V.\,A.} see~Sopin~E.\,S.&&\\
\end{tabular}
}
\pagebreak

\def\leftfootline{\small{\textbf{\thepage}
\hfill INFORMATIKA I EE PRIMENENIYA~--- INFORMATICS AND APPLICATIONS\ \ \ 2018\
\ \ volume~12\ \ \ issue\ 4}
}%
 \def\rightfootline{\small{INFORMATIKA I EE PRIMENENIYA~---
INFORMATICS AND APPLICATIONS\ \ \ 2018\ \ \ volume~12\ \ \ issue\ 4
\hfill \textbf{\thepage}}}

\def\leftkol{2018 AUTHOR INDEX} % ENGLISH ABSTRACTS}

\def\rightkol{2018 AUTHOR INDEX} %ENGLISH ABSTRACTS}


\noindent
{\tabcolsep=3pt
\begin{tabular}{p{395.48108pt}cc}
&\textbf{Issue} & \textbf{Page}\\[6pt]
\Avtors{Nazarov~L.\,V., Lavrentyev~V.\,V., and Bykovets~E.\,V.} A~probability model of the influence\linebreak
\\[-12pt]
\hspace*{23pt}of~the~order book on the price process&2&29--34\\
\Avtors{Nazarova~I.\,A.} see~Malashenko~Yu.\,E.&&\\
\Avtors{Nazarova~I.\,A.} see~Malashenko~Yu.\,E.&&\\
\Avtors{Novikova~N.\,M.} see~Malashenko~Yu.\,E.&&\\
\Avtors{Novikova~N.\,M.} see~Malashenko~Yu.\,E.&&\\
\Avtors{Nuriev~V.\,A., Buntman~N.\,V., and Inkova~O.\,Yu.} Machine translation of russian connectives into\linebreak
\\[-12pt]
\hspace*{23pt}french: Errors and quality failures&2&105--113\\
\Avtors{Nuriev~V.\,A.} see~Buntman~N.\,V.&&\\
\Avtors{Ogaltsov~A.\,V.\ and Bakhteev~O.\,Y.} Automatic metadata extraction from scientific PDF documents&2&75--82\\
\Avtors{Pavlov~Yu.\,L.} see~Leri~M.\,M.&&\\
\Avtors{Razumchik~R.\,V.} see~Konovalov~M.\,G.&&\\
\Avtors{Razumchik~R.\,V.} see~Konovalov~M.\,G.&&\\
\Avtors{Romanov~D.\,A.} see~Kozerenko~E.\,B.&&\\
\Avtors{Samouylov~K.\,E., Gaidamaka~Yu.\,V., and~Shorgin~S.\,Ya.} Modeling movement of devices in\linebreak
\\[-12pt]
\hspace*{23pt}a~wireless network by random walk models&4&2--8\\
\Avtors{Samouylov~K.\,E.} see~Gorbunova~A.\,V.&&\\
\Avtors{Samouylov~K.\,Е.} see~Sopin~E.\,S.&&\\
\Avtors{Serebryakov~V.\,A.\ and Ataeva~O.\,M.} Ontology of the digital semantic library LibMeta&1&\hphantom{1}2--10\\
\Avtors{Shanin~I.\,A., Stupnikov~S.\,A., and Zakharov~V.\,N.} Methods and tools for fault detection\linebreak
\\[-12pt]
\hspace*{23pt}on~elements of housing and utility infrastructure&3&67--73\\
\Avtors{Shestakov~O.\,V.} Mean-square thresholding risk with a random sample size&3&14--17\\
\Avtors{Shestakov~O.\,V.} Unbiased risk estimate of stabilized hard thresholding in the model with\linebreak
\\[-12pt]
\hspace*{23pt}a~long-range dependence&2&11--16\\
\Avtors{Shestakov~O.\,V.} see~Kudryavtsev~A.\,A.&&\\
\Avtors{Shestakov~O.\,V.} see~Kudryavtsev~A.\,A.&&\\
\Avtors{Shihiev~Sh.\,B.} see~Mirzabekov~Ya.\,M.&&\\
\Avtors{Shirokov~N.\,I.} see~Vyshinsky~L.\,L.&&\\
\Avtors{Shnurkov~P.\,V.\ and Egorov~A.\,Y.} Development and preliminary study of a~stochastic semi-Markov model of continuous supply of product management under the condition of\linebreak
\\[-12pt]
\hspace*{23pt}constant consumption&1&109--117\\
\Avtors{Shnurkov~P.\,V.\ and Egorov~A.\,Y.} Solution to the problem of optimal control of a~stochastic semi-Markov model of continuous supply of product management under the condition\linebreak
\\[-12pt]
\hspace*{23pt}of~constantly happening consumption&2&83--89\\
\Avtors{Shorgin~S.\,Ya.} see~Grusho~A.\,A.&&\\
\Avtors{Shorgin~S.\,Ya.} see~Samouylov~K.\,E.&&\\
\Avtors{Sinitsyn~I.\,N.} Method of interpolational analytical modeling of processes in stochastic systems&1&55--61\\
\Avtors{Smerdov~A.\,N., Bakhteev~O.\,Y., and~Strijov~V.\,V.} Optimal recurrent neural network model\linebreak
\\[-12pt]
\hspace*{23pt}in~paraphrase detection&4&63--69\\
\Avtors{Smirnov~D.\,V.} see~Grusho~A.\,A.&&\\
\Avtors{Sochenkov~I.\,V., Zubarev~D.\,V., and Tikhomirov~I.\,A.} Exploratory patent search&1&89--94\\
\Avtors{Sopin~E.\,S., Naumov~V.\,A., and Samouylov~K.\,Е.} On the insensitivity of the stationary distribution\linebreak
\\[-12pt]
\hspace*{23pt}of the limited resources queuing system with state-dependent arrival and service rates&3&42--47\\
\Avtors{Stefanovich~A.\,I.} see~Bosov~A.\,V.&&\\
\Avtors{Strijov~V.\,V.} see~Smerdov~A.\,N.&&\\
\Avtors{Stupnikov~S.\,A.} see~Shanin~I.\,A.&&\\
\Avtors{Suchkov~A.\,P.} see~Zatsarinny~A.\,A.&&\\
\Avtors{Surina~A.\,A.} see~Tyrsin~A.\,N.&&\\
\Avtors{Syuntyurenko~O.\,V.} Financing of basic research: Conceptual shape of a system of support\linebreak
\\[-12pt]
\hspace*{23pt}of~decision-making with use of methods of scientometrics and analysis of data&1&118--127\\
\Avtors{Tikhomirov~I.\,A.} see~Sochenkov~I.\,V.&&\\
\Avtors{Timonina~E.\,E.} see~Grusho~A.\,A.&&\\
\Avtors{Timonina~E.\,E.} see~Grusho~A.\,A.&&\\
\end{tabular}
}
\pagebreak

\def\leftfootline{\small{\textbf{\thepage}
\hfill INFORMATIKA I EE PRIMENENIYA~--- INFORMATICS AND APPLICATIONS\ \ \ 2018\
\ \ volume~12\ \ \ issue\ 4}
}%
 \def\rightfootline{\small{INFORMATIKA I EE PRIMENENIYA~---
INFORMATICS AND APPLICATIONS\ \ \ 2018\ \ \ volume~12\ \ \ issue\ 4
\hfill \textbf{\thepage}}}

\def\leftkol{2018 AUTHOR INDEX} % ENGLISH ABSTRACTS}

\def\rightkol{2018 AUTHOR INDEX} %ENGLISH ABSTRACTS}


\noindent
{\tabcolsep=3pt
\begin{tabular}{p{395.48108pt}cc}
&\textbf{Issue} & \textbf{Page}\\[6pt]
\Avtors{Timonina~E.\,E.} see~Grusho~A.\,A.&&\\
\Avtors{Timonina~E.\,E.} see~Grusho~A.\,A.&&\\
\Avtors{Titova~A.\,I.} see~Arutyunov~E.\,N.&&\\
\Avtors{Tyrsin~A.\,N.\ and Surina~A.\,A.} A~model of risk management in Gaussian stochastic systems&2&50--59\\
\Avtors{Ushakov~V.\,G.} see~Kondranin~E.\,S.&&\\
\Avtors{Ushakov~V.\,G.} see~Mistryukov~A.\,V.&&\\
\Avtors{Vasil'eva~S.\,N.\ and Kan~Yu.\,S.} A~visualization algorithm for the plane probability measure kernel&2&60--68\\
\Avtors{Vinogradov~D.\,V.} Influence of preliminary estimates on the speed of search of similarities by\linebreak
\\[-12pt]
\hspace*{23pt}the~coupling Markov chain&1&49--54\\
\Avtors{Vyshinsky~L.\,L., Flerov~Yu.\,A., and Shirokov~N.\,I.} Computer-aided system of aircraft weight\linebreak
\\[-12pt]
\hspace*{23pt}design&1&18--30\\
\Avtors{Yakovlev O.\,A.} see~Batenkov~A.\,A.&&\\
\Avtors{Zabezhailo~M.\,I.} see~Grusho~A.\,A.&&\\
\Avtors{Zabezhailo~M.\,I.} see~Grusho~A.\,A.&&\\
\Avtors{Zakharov~V.\,N.} see~Basok~B.\,M.&&\\
\Avtors{Zakharov~V.\,N.} see~Shanin~I.\,A.&&\\
\Avtors{Zaltsman~A.\,D.} see~Zhukov~D.\,O.&&\\
\Avtors{Zatsarinny~A.\,A.\ and Suchkov~A.\,P.} The situational management system as a multiservice\linebreak
\\[-12pt]
\hspace*{23pt}technology in the cloud&1&78--88\\
\Avtors{Zatsarinny~A.\,A.} see~Grusho~A.\,A.,&&\\
\Avtors{Zatsman~I.\,M.} Implied knowledge: Foundations and technologies of explication&3&74--82\\
\Avtors{Zatsman~I.\,M.} see~Buntman~N.\,V.&&\\
\Avtors{Zeifman~A.\,I.} see~Korolev~V.\,Yu.&&\\
\Avtors{Zhukov~D.\,O., Khvatova~T.\,Yu., Lesko~S.\,A., and Zaltsman~A.\,D.} The influence of the connections' density on clusterization and percolation threshold during information distribution in social\linebreak
\\[-12pt]
\hspace*{23pt}networks&2&90--97\\
\Avtors{Zubarev~D.\,V.} see~Sochenkov~I.\,V.&&\\
\end{tabular}
}

%\thispagestyle{myheadings}
\def\leftfootline{\small{\textbf{\thepage}
\hfill INFORMATIKA I EE PRIMENENIYA~--- INFORMATICS AND APPLICATIONS\ \ \ 2018\
\ \ volume~12\ \ \ issue\ 4}
}%
 \def\rightfootline{\small{INFORMATIKA I EE PRIMENENIYA~---
INFORMATICS AND APPLICATIONS\ \ \ 2018\ \ \ volume~12\ \ \ issue\ 4
\hfill \textbf{\thepage}}}

 \label{end\stat}

\newpage

%   \vspace*{-48pt}

\begin{center}
\vspace*{6pt}
\mbox{%
\epsfxsize=53.502mm
\epsfbox{foto-1.eps}
}
\end{center}

\vspace*{6pt} %Академик


   \begin{center}
\fbox{\Large\textbf{Профессор Игорь Алексеевич Ушаков}}\\[12pt]
\textbf{\large 22.01.1935--27.02.2015}
   \end{center}


   %\vspace*{2.5mm}

   \vspace*{5mm}

   \thispagestyle{empty}

%\

%\vspace*{-12pt}


Редакционный совет и редакционная коллегия журнала <<Информатика и~её применения>> с~глубоким прискорбием извещают, что 27~февраля 2015~г.\ после тяжелой
и~продолжительной болезни скончался Игорь Алексеевич Ушаков~--- доктор технических наук, профессор, член редколлегии журнала <<Информатика и ее применения>>.

Игорь Алексеевич Ушаков окончил Московский авиационный институт, в~1963~г.\ защитил кандидатскую, а~в~1968~г.~--- докторскую диссертацию. С~1958 по 1989~гг.\ работал в~ряде научно-исследовательских организаций СССР, в~том числе руководил отделами в~НИИ АА и~ВЦ АН СССР; с 1969 по 1989 гг. преподавал в~МФТИ (был профессором, а~затем заведующим кафедрой) и~в~МЭИ. С~1989~г.~---- в~США: являлся профессором университета Дж.\ Вашингтона, университета Дж.\ Мэйсона и~Калифорнийского университета, сотрудником компаний MCI, Qualcomm и Hughes.

И.\,А.~Ушаков с момента основания журнала <<Надежность и~контроль качества>> был заместителем ответственного редактора, а~затем на протяжении многих лет членом редколлегии. В~2006~г.\ основал электронный международный журнал ``Reliability: Theory \& Application'', главным редактором которого оставался до конца жизни.

Учебниками и справочниками по теории надежности, написанными И.\,А.~Ушаковым, пользовались и~пользуются несколько поколений ученых и~специалистов в~разных странах мира.

Игорь Алексеевич всегда уделял огромное внимание работе с~молодежью; более~50 его учеников защитили докторские и~кандидатские диссертации.

И.\,А.~Ушаков вел активную научно-про\-све\-ти\-тель\-скую деятельность. В~частности, он был одним из организаторов и~руководителей Московского кабинета качества и~надежности при Политехническом музее (целью этого Кабинета было оказание консультаций работникам промышленных предприятий и~чтение курсов лекций для инженеров, занимающихся проблемой надежности). Находясь в~США, И.\,А.~Ушаков создал международный ин\-тер\-нет-фо\-рум им.\ Б.\,В.~Гнеденко, объединивший около~400~видных специалистов по приложениям теории вероятностей и~математической статистики, преимущественно в~об\-ласти теории надежности и~анализа риска, из десятков стран мира; коллективным членов этого Форума является и~наш журнал. Цели Форума~--- содействие контактам между специалистами из разных стран, организация обмена профессиональными 
новостями и~информацией (новые публикации, предстоящие события и~др.). Также необходимо отметить большое число на\-уч\-но-по\-пу\-ляр\-ных работ, опубликованных И.\,А.~Ушаковым.

И.\,А.~Ушаков обладал большим личным обаянием, имел широкий круг интересов. Все знавшие И.\,А.~Ушакова всегда будут помнить его как замечательного ученого и~прекрасного человека.

\bigskip

Редакционный совет и редакционная коллегия журнала <<Информатика и~её применения>> 
выражают глубокие соболезнования родным и близким покойного, всем, кто его знал и~работал с~ним.


%\def\stat{cont}
{%\hrule\par
%\vskip 7pt % 7pt
\raggedleft\Large \bf%\baselineskip=3.2ex
А\,В\,Т\,О\,Р\,С\,К\,И\,Й\ \ У\,К\,А\,З\,А\,Т\,Е\,Л\,Ь\ \ З\,А\ \ 2\,0\,1\,0 г. \vskip 17pt
    \hrule
    \par
\vskip 21pt plus 6pt minus 3pt }

\label{st\stat}

\def\tit{\ }

\def\aut{\ }
\def\auf{\ }

\def\leftkol{\ } % ENGLISH ABSTRACTS}

\def\rightkol{\ } %АВТОРСКИЙ УКАЗАТЕЛЬ ЗА 2010 г.} %ENGLISH ABSTRACTS}

\titele{\tit}{\aut}{\auf}{\leftkol}{\rightkol}

\vspace*{-12pt}

{\tabcolsep=3pt
\begin{tabular}{p{388pt}rr}
&\textbf{Выпуск} & \textbf{Стр.}\\[6pt]
\hangindent=23pt\noindent\textbf{Арутюнян~А.\,Р.} Моделирование влияния деформаций отпечатков пальцев на 
точность\linebreak
\vspace*{-12pt}\\
\hspace*{23pt}дактилоскопической идентификации$\dotfill$&1&51\\
\hangindent=23pt\noindent\textbf{Архипов~О.\,П., Зыкова~З.\,П.} Интеграция гетерогенной информации о цветных 
пикселях\linebreak
\vspace*{-12pt}\\
\hspace*{23pt}и их цветовосприятии$\dotfill$&4&15\\
\hangindent=23pt\noindent\textbf{Баранов~С.\,И., Френкель~С.\,Л., Захаров~В.\,Н.} Полуформальная верификация 
цифрового устройства с конвейером, основанная на использовании алгоритмических машин\linebreak
\vspace*{-12pt}\\
\hspace*{23pt}состояния$\dotfill$&4&49\\
\textbf{Бекетова~И.\,В.} см.~Каратеев~С.\,Л.&&\\
\textbf{Белоусов~В.\,В.} см.~Синицын~И.\,Н.&&\\
\hangindent=23pt\noindent\textbf{Бенинг~В.\,Е., Королев~Р.\,А.} О предельном поведении мощностей критериев в 
случае\linebreak
\vspace*{-12pt}\\
\hspace*{23pt}распределения Лапласа$\dotfill$&2&63\\
\hangindent=23pt\noindent\textbf{Бенинг~В.\,Е., Сипина~А.\,В.} Асимптотическое разложение для мощности 
критерия,\linebreak
\vspace*{-12pt}\\
\hspace*{23pt}основанного на выборочной медиане, в случае распределения Лапласа$\dotfill$&1&18\\
\textbf{Бондаренко~А.\,В.} см.~Каратеев~С.\,Л.&&\\
\hangindent=23pt\noindent\textbf{Бородина~А.\,В., Морозов~Е.\,В.} Об оценивании асимптотики вероятности 
большого\linebreak
\vspace*{-12pt}\\
\hspace*{23pt}уклонения стационарной регенеративной очереди с одним прибором$\dotfill$&3&29\\
\hangindent=23pt\noindent\textbf{Бунтман~Н.\,В., Минель~Ж.-Л., Ле~Пезан~Д., Зацман~И.\,М.} Типология и 
компьютерное\linebreak
\vspace*{-12pt}\\
\hspace*{23pt}моделирование трудностей перевода$\dotfill$&3&77\\
\textbf{Визильтер~Ю.\,В.} см.~Каратеев~С.\,Л.&&\\
\hangindent=23pt\noindent\textbf{Гавриленко~С.\,В.} Оценки скорости сходимости распределений случайных сумм с 
безгранично делимыми индексами к нормальному закону$\dotfill$&4&81\\
\hangindent=23pt\noindent\textbf{Григорьева~М.\,Е., Шевцова~И.\,Г.} Уточнение неравенства 
Каца--Берри--Эссеена$\dotfill$&2&75\\
\hangindent=23pt\noindent\textbf{Грушо~А.\,А., Грушо~Н.\,А., Тимонина~Е.\,Е.} Поиск конфликтов в политиках 
безопасности: модель случайных графов$\dotfill$&3&38\\
\textbf{Грушо~Н.\,А.} см.~Грушо~А.\,А.&&\\
\hangindent=23pt\noindent\textbf{Гудков~В.\,Ю.} Математические модели изображения отпечатка пальца на основе 
описания линий$\dotfill$&1&58\\
\textbf{Гуртов~А.\,В.} см.~Лукьяненко~А.\,С.&&\\
\textbf{Желтов~С.\,Ю.} см.~Каратеев~С.\,Л.&&\\
\hangindent=23pt\noindent\textbf{Захаров~А.\,А., Серебряков~В.\,А.} Система управления электронной библиотекой 
LibMeta$\dotfill$&4&2\\
\textbf{Захаров~В.\,Н.} см.~Баранов~С.\,И.&&\\
\textbf{Захарова~Т.\,В.} см.~Матвеева~С.\,С.&&\\
\hangindent=23pt\noindent\textbf{Зацаринный~А.\,А., Чупраков~К.\,Г.} Некоторые аспекты выбора технологии для 
постро-\linebreak
\vspace*{-12pt}\\
\hspace*{23pt}ения систем отображения информации ситуационного центра$\dotfill$&3&59\\
\textbf{Зацман~И.\,М.} см.~Бунтман~Н.\,В.&&\\
\hangindent=23pt\noindent\textbf{Зейфман~А.\,И., Коротышева~А.\,В., Сатин~Я.\,А., Шоргин~С.\,Я.} Об 
устойчивости нестаци-\linebreak
\vspace*{-12pt}\\
\hspace*{23pt}онарных систем обслуживания с катастрофами$\dotfill$&3&9\\
\textbf{Зыкова~З.\,П.} см.~Архипов~О.\,П.&&\\
\hangindent=23pt\noindent\textbf{Илюшин~Г.\,Я., Соколов~И.\,А.} Организация управляемого доступа пользователей 
к\linebreak
\vspace*{-12pt}\\
\hspace*{23pt}разнородным ведомственным информационным ресурсам$\dotfill$&1&24\\
\hangindent=23pt\noindent\textbf{Кавагучи~Ю., Ульянов~В.\,В., Фуджикоши~Я.} Приближения для статистик, 
описывающих\linebreak
\vspace*{-12pt}\\
\hspace*{23pt}геометрические свойства данных большой размерности, с оценками 
ошибок$\dotfill$&1&12\\
\hangindent=23pt\noindent\textbf{Каратеев~С.\,Л., Бекетова~И.\,В., Ососков~М.\,В., Князь~В.\,А., 
Визильтер~Ю.\,В., Бондаренко~А.\,В., Желтов~С.\,Ю.} Автоматизированный контроль 
качества цифровых\linebreak
\vspace*{-12pt}\\
\hspace*{23pt}изображений для персональных документов$\dotfill$&1&65\\
\end{tabular}
}

\pagebreak

\def\leftkol{АВТОРСКИЙ УКАЗАТЕЛЬ ЗА 2010 г.} % ENGLISH ABSTRACTS}

\def\rightkol{АВТОРСКИЙ УКАЗАТЕЛЬ ЗА 2010 г.} %ENGLISH ABSTRACTS}

{\tabcolsep=3pt
\begin{tabular}{p{388pt}rr}
&\textbf{Выпуск} & \textbf{Стр.}\\[3pt]
\hangindent=23pt\noindent\textbf{Козеренко~Е.\,Б.} Лингвистические фильтры в статистических моделях машинного\linebreak
\vspace*{-12pt}\\
\hspace*{23pt}перевода$\dotfill$&2&83\\
\hangindent=23pt\noindent\textbf{Козеренко~Е.\,Б., Кузнецов~И.\,П.} Когнитивно-лингвистические представления в 
систе-\linebreak
\vspace*{-12pt}\\
\hspace*{23pt}мах обработки текстов$\dotfill$&3&69\\
\textbf{Князь~В.\,А.} см.~Каратеев~С.\,Л.&&\\
\hangindent=23pt\noindent\textbf{Колесников~А.\,В., Солдатов~С.\,А.} Алгоритм координации для гибридной 
интеллектуальной системы решения сложной задачи оперативно-производственного\linebreak
\vspace*{-12pt}\\
\hspace*{23pt}планирования$\dotfill$&4&61\\
\hangindent=23pt\noindent\textbf{Коновалов~М.\,Г.} О планировании потоков в системах вычислительных 
ресурсов$\dotfill$&2&3\\
\textbf{Конушин~А.\,С.} см.~Конушин~В.\,С.&&\\
\hangindent=23pt\noindent\textbf{Конушин~В.\,С., Кривовязь~Г.\,Р., Конушин~А.\,С.} Алгоритм распознавания людей 
в видео-\linebreak
\vspace*{-12pt}\\
\hspace*{23pt}последовательности по одежде$\dotfill$&1&74\\
\textbf{Корепанов~Э.\, Р.} см.~Синицын~И.\,Н.&&\\
\textbf{Королев~В.\,Ю.} см.~Соколов~И.\,А.&&\\
\textbf{Королев~Р.\,А.} см.~Бенинг~В.\,Е.&&\\
\textbf{Коротышева~А.\,В.} см.~Зейфман~А.\,И.&&\\
\hangindent=23pt\noindent\textbf{Кривенко~М.\,П.} Непараметрическое оценивание элементов байесовского 
клас\-си-\linebreak
\vspace*{-12pt}\\
\hspace*{23pt}фикатора$\dotfill$&2&13\\
\textbf{Кривовязь~Г.\,Р.} см.~Конушин~В.\,С.&&\\
\textbf{Крылов~А.\,С.} см.~Павельева~Е.\,А.&&\\
\hangindent=23pt\noindent\textbf{Крылов~В.\,А.} Моделирование и классификация многоканальных дистанционных\linebreak
\vspace*{-12pt}\\
\hspace*{23pt}изображений с использованием копул$\dotfill$&4&34\\
\hangindent=23pt\noindent\textbf{Крючин~О.\,В.} Разработка параллельных эвристических алгоритмов подбора 
весовых\linebreak
\vspace*{-12pt}\\
\hspace*{23pt}коэффициентов искусственной нейтронной сети$\dotfill$&2&53\\
\hangindent=23pt\noindent\textbf{Кудрявцев~А.\,А., Шоргин~С.\,Я.} Байесовские модели массового обслуживания и 
надеж-\linebreak
\vspace*{-12pt}\\
\hspace*{23pt}ности: характеристики среднего числа заявок в системе $M\vert M \vert 1\vert 
\infty$$\dotfill$&3&16\\
\hangindent=23pt\noindent\textbf{Кузнецов~А.\,А.} Связь между временными и структурно-топологическими 
характери-\linebreak
\vspace*{-12pt}\\
\hspace*{23pt}стиками диаграмм ритма сердца здоровых людей$\dotfill$&4&39\\
\textbf{Кузнецов~И.\,П.} см.~Козеренко~Е.\,Б.&&\\
\textbf{Ле~Пезан~Д.} см.~Бунтман~Н.\,В.&&\\
\hangindent=23pt\noindent\textbf{Лукьяненко~А.\,С., Морозов~Е.\,В., Гуртов~А.\,В.} Анализ сетевого протокола с общей 
функ-\linebreak
\vspace*{-12pt}\\
\hspace*{23pt}цией расширения окна передачи сообщения при конфликтах$\dotfill$&2&46\\
\hangindent=23pt\noindent\textbf{Лямин~О.\,О.} О предельном поведении мощностей критериев в случае обобщенного\linebreak
\vspace*{-12pt}\\
\hspace*{23pt}распределения Лапласа$\dotfill$&3&47\\
\hangindent=23pt\noindent\textbf{Маркин~А.\,В., Шестаков~О.\,В.} Асимптотики оценки риска при пороговой 
обработке\linebreak
\vspace*{-12pt}\\
\hspace*{23pt}вейвлет-вейглет коэффициентов в задаче томографии$\dotfill$&2&36\\
\hangindent=23pt\noindent\textbf{Матвеева~С.\,С., Захарова~Т.\,В.} Сети массового обслуживания с наименьшей 
длиной\linebreak
\vspace*{-12pt}\\
\hspace*{23pt}очереди$\dotfill$&3&22\\
\hangindent=23pt\noindent\textbf{Матюшенко~С.\,И.} Стационарные характеристики двухканальной системы 
обслужива-\linebreak
\vspace*{-12pt}\\
\hspace*{23pt}ния с переупорядочиванием заявок и распределениями фазового типа$\dotfill$&4&68\\
\textbf{Минель~Ж.-Л.} см.~Бунтман~Н.\,В.&&\\
\textbf{Морозов~Е.\,В.} см.~Бородина~А.\,В.&&\\
\textbf{Морозов~Е.\,В.} см.~Лукьяненко~А.\,С.&&\\
\textbf{Ососков~М.\,В.} см.~Каратеев~С.\,Л.&&\\
\hangindent=23pt\noindent\textbf{Павельева~Е.\,А., Крылов~А.\,С.} Поиск и анализ ключевых точек радужной 
оболочки\linebreak
\vspace*{-12pt}\\
\hspace*{23pt}глаза методом преобразования Эрмита$\dotfill$&1&79\\
\textbf{Печинкин~А.\,В.} см.~Френкель~С.\,Л.,&&\\
\hangindent=23pt\noindent\textbf{Протасов~В.\,И.} Составление субъективного портрета с использованием 
эволюционно-\linebreak
\vspace*{-12pt}\\
\hspace*{23pt}го морфинга и квалиметрия метода$\dotfill$&1&83\\
\hangindent=23pt\noindent\textbf{Рудаков~К.\,В., Торшин~И.\,Ю.} Вопросы разрешимости задачи распознавания 
вторичной\linebreak
\vspace*{-12pt}\\
\hspace*{23pt}структуры белка$\dotfill$&2&25\\
\textbf{Сатин~Я.\,А.} см.~Зейфман~А.\,И.&&\\
\hangindent=23pt\noindent\textbf{Сейфуль-Мулюков~Р.\,Б.} Нефть как носитель информации о своем 
происхождении,\linebreak
\vspace*{-12pt}\\
\hspace*{23pt}структуре и эволюции$\dotfill$&1&41\\
\end{tabular}
}

{\tabcolsep=3pt
\begin{tabular}{p{388pt}rr}
&\textbf{Выпуск} & \textbf{Стр.}\\[6pt]
\textbf{Семендяев~Н.\,Н.} см.~Синицын~И.\,Н.&&\\
\textbf{Серебряков~В.\,А.} см.~Захаров~А.\,А.&&\\
\textbf{Синицын~В.\,И.} см.~Синицын~И.\,Н.&&\\
\hangindent=23pt\noindent\textbf{Синицын~И.\,Н., Синицын~В.\,И., Корепанов~Э.\, Р., Белоусов~В.\,В., 
Семендяев~Н.\,Н.} Оперативное построение информационных моделей движения полюса 
Земли\linebreak
\vspace*{-12pt}\\
\hspace*{23pt}методами линейных и линеаризованных фильтров$\dotfill$&1&2\\
\textbf{Сипина~А.\,В.} см.~Бенинг~В.\,Е.&&\\
\hangindent=23pt\noindent\textbf{Соколов~И.\,А.} О работах заслуженного деятеля науки Российской Федерации 
И.\,Н.~Синицына в области информационных технологий и автоматизации (к 70-летию\linebreak
\vspace*{-12pt}\\
\hspace*{23pt}со дня рождения)$\dotfill$&3&84\\
\textbf{Соколов~И.\,А.} см.~Илюшин~Г.\,Я.&&\\
\hangindent=23pt\noindent\textbf{Соколов~И.\,А., Королев~В.\,Ю.} Предисловие$\dotfill$&2&2\\
\textbf{Солдатов~С.\,А.} см.~Колесников~А.\,В.&&\\
\hangindent=23pt\noindent\textbf{Степанов~С.\,Ю.} Использование координатного метода фрагментации 
коммутаторной\linebreak
\vspace*{-12pt}\\
\hspace*{23pt}нейронной сети для сокращения трафика$\dotfill$&2&57\\
\textbf{Тимонина~Е.\,Е.} см.~Грушо~А.\,А.&&\\
\textbf{Торшин~И.\,Ю.} см.~Рудаков~К.\,В.&&\\
\textbf{Ульянов~В.\,В.} см.~Кавагучи~Ю.&&\\
\textbf{Фазекаш~И.} см.~Чупрунов~А.\,Н.&&\\
\textbf{Френкель~С.\,Л.} см.~Баранов~С.\,И.&&\\
\hangindent=23pt\noindent\textbf{Френкель~С.\,Л., Печинкин~А.\,В.} Оценка времени самовосстановления в 
цифровых\linebreak
\vspace*{-12pt}\\
\hspace*{23pt}системах после сбоев, вызываемых переходными помехами$\dotfill$&3&2\\
\textbf{Фуджикоши~Я.} см.~Кавагучи~Ю.&&\\
\hangindent=23pt\noindent\textbf{Цискаридзе~А.\,К.} Математическая модель и метод восстановления позы человека 
по\linebreak
\vspace*{-12pt}\\
\hspace*{23pt}стереопаре силуэтных изображений$\dotfill$&4&27\\
\hangindent=23pt\noindent\textbf{Чупраков~К.\,Г.} К вопросу о размещении коллективных средств отображения в 
ситуа-\linebreak
\vspace*{-12pt}\\
\hspace*{23pt}ционном зале с заданными параметрами$\dotfill$&4&89\\
\textbf{Чупраков~К.\,Г.} см.~Зацаринный~А.\,А.&&\\
\hangindent=23pt\noindent\textbf{Чупрунов~А.\,Н., Фазекаш~И.} Законы повторного логарифма для числа 
безошибочных\linebreak
\vspace*{-12pt}\\
\hspace*{23pt}блоков при помехоустойчивом кодировании$\dotfill$&3&42\\
\textbf{Шевцова~И.\,Г.} см.~Григорьева~М.\,Е.&&\\
\hangindent=23pt\noindent\textbf{Шестаков~О.\,В.} Аппроксимация распределения оценки риска пороговой 
обработки вейвлет-коэффициентов нормальным распределением при использовании 
выбо-\linebreak
\vspace*{-12pt}\\
\hspace*{23pt}рочной дисперсии$\dotfill$&4&73\\
\textbf{Шестаков~О.\,В.} см.~Маркин~А.\,В.&&\\
\textbf{Шоргин~С.\,Я.} см.~Зейфман~А.\,И.&&\\
\textbf{Шоргин~С.\,Я.} см.~Кудрявцев~А.\,А.&&\\
\end{tabular}
}

%\thispagestyle{myheadings}
\def\leftfootline{\small{\textbf{\thepage}
\hfill ИНФОРМАТИКА И ЕЁ ПРИМЕНЕНИЯ\ \ \ том~4\ \ \ выпуск~4\ \ \ 2010}
}%
 \def\rightfootline{\small{ИНФОРМАТИКА И ЕЁ ПРИМЕНЕНИЯ\ \ \ том~4\ \ \ выпуск~4\ \ \ 2010
 \hfill \textbf{\thepage}}}
 \label{end\stat}
%
%Том 10 Выпуск 1-4 Год 2016

\def\stat{cont-e}
{%\hrule\par
%\vskip 7pt % 7pt
\raggedleft\Large \bf%\baselineskip=3.2ex
2\,0\,1\,6\ \ A\,U\,T\,H\,O\,R\ \ I\,N\,D\,E\,X \vskip 17pt
 \hrule
 \par
\vskip 21pt plus 6pt minus 3pt }

\label{st\stat}

\def\tit{\ }

\def\aut{\ }
\def\auf{\ }

\def\leftkol{\ } %2016 AUTHOR INDEX} % ENGLISH ABSTRACTS}

\def\rightkol{\ } %2016 AUTHOR INDEX} %ENGLISH ABSTRACTS}

\titele{\tit}{\aut}{\auf}{\leftkol}{\rightkol}

\def\leftfootline{\small{\textbf{\thepage}
\hfill INFORMATIKA I EE PRIMENENIYA~--- INFORMATICS AND APPLICATIONS\ \ \ 2016\
\ \ volume~10\ \ \ issue\ 4}
}%
 \def\rightfootline{\small{INFORMATIKA I EE PRIMENENIYA~--- INFORMATICS AND APPLICATIONS\ \ \ 2016\ \ \ volume~10\ \ \ issue\ 4
\hfill \textbf{\thepage}}}

\vspace*{-12pt}
\vspace*{-18pt}

{\tabcolsep=2.8pt
\begin{tabular}{p{382pt}cc}
&\textbf{Issue} & \textbf{Page}\\[6pt]
\Avtors{Agalarov~M.\,Ya.} see~Agalarov~Ya.\,M.&&\\
\Avtors{Agalarov~Ya.\,M., Agalarov~M.\,Ya., and
Shorgin~V.\,S.} About the optimal threshold of queue\linebreak
\\[-12pt]
\hspace*{23pt}length in a~particular problem of profit maximization
in the $M/G/1$ queuing system&2&70--79\\
\Avtors{Alexeyevsky~D.\,A.} BioNLP ontology extraction from 
a~restricted language corpus with\linebreak
\\[-12pt]
\hspace*{23pt}context-free grammars&1&119--128\\
\Avtors{Andreev~S.\,D.} see~Gaidamaka~Yu.\,V.&&\\
\Avtors{Andreev~S.\,D.} see~Ometov~A.\,Ya.&&\\
\Avtors{Arkhipov~O.\,P., Arkhipov~P.\,O., and Sidorkin~I.\,I.} The
option to create a~local coordinate\linebreak
\\[-12pt]
\hspace*{23pt}system for synchronization of selected images&3&91--97\\
\Avtors{Arkhipov~P.\,O.} see~Arkhipov~O.\,P.&&\\
\Avtors{Belousov~V.\,V.} see~Shnurkov~P.\,V.&&\\
\Avtors{Belousov~V.\,V.} see~Shnurkov~P.\,V.&&\\
\Avtors{Bening~V.\,E.} Calculation of~the~asymptotic deficiency
of~some statistical procedures based\linebreak
\\[-12pt]
\hspace*{23pt}on~samples with~random sizes&4&34--45\\
\Avtors{Borisov~A.\,V., Bosov~A.\,V., and Miller~G.\,B.} Modeling and
monitoring of VoIP connection&2&\hphantom{1}2--13\\
\Avtors{Bosov~A.\,V.} see~Borisov~A.\,V.&&\\
\Avtors{Briukhov~D.\,O.} see~Stupnikov~S.\,A.&&\\
\Avtors{Callaos~N.\,K.\ and Seyful-Mulyukov~R.\,B.} Complexity and
its information content&1&129--139\\
\Avtors{Chertok~A.\,V., Kadaner~A.\,I., Khazeeva~G.\,T., and
Sokolov~I.\,A.} Regime switching detection\linebreak
\\[-12pt]
\hspace*{23pt}for~the~Levy driven
Ornstein--Uhlenbeck process using CUSUM methods&4&46--56\\
\Avtors{Chichagov~V.\,V.} Asymptotic expansions of mean absolute
error of uniformly minimum variance unbiased and maximum likelihood
estimators on the one-parameter exponential\linebreak
\\[-12pt]
\hspace*{23pt}family model of lattice distributions&3&66--76\\
\Avtors{Danishevsky~V.\,I.} see~Kolesnikov A.\,V.&&\\
\Avtors{Fazliev~A.\,Z.} see~Kalinichenko~L.\,A.&&\\
\Avtors{Fedoseev~A.\,A.} What is behind the concept of ``knowledge in
small packages''&3&105--110\\
\Avtors{Gaidamaka~Yu.\,V., Andreev~S.\,D., Sopin~E.\,S.,
Samouylov~K.\,E., and Shorgin~S.\,Ya.} Interference analysis
of~the~device-to-device communications model with~regard to~a~signal\linebreak
\\[-12pt]
\hspace*{23pt}propagation environment&4&\hphantom{1}2--10\\
\Avtors{Gasilov~A.\,V.} see~Yakovlev~O.\,A.&&\\
\Avtors{Goncharov~A.\,V.\ and Strijov~V.\,V.} Metric time series
classification using weighted dynamic\linebreak
\\[-12pt]
\hspace*{23pt}warping relative to centroids of classes&2&36--47\\
\Avtors{Gordov~E.\,P.} see~Kalinichenko~L.\,A.&&\\
\Avtors{Gorshenin~A.\,K.} Concept of online service for stochastic
modeling of real processes&1&72--81\\
\Avtors{Gorshenin~A.\,K.} see~Shnurkov~P.\,V.&&\\
\Avtors{Gorshenin~A.\,K.} see~Shnurkov~P.\,V.&&\\
\Avtors{Grusho~A.\,A., Grusho~N.\,A., Zabezhailo~M.\,I., and
Timonina~E.\,E.} Integration of statistical and\linebreak
\\[-12pt]
\hspace*{23pt}deterministic methods for
analysis of information security&3&2--8\\
\Avtors{Grusho~A.\,A., Zabezhailo~M.\,I., and Zatsarinny~A.\,A.} On
the advanced procedure to reduce\linebreak
\\[-12pt]
\hspace*{23pt}calculation of Galois closures&4&\hphantom{1}96--104\\
\Avtors{Grusho~N.\,A.} see~Grusho~A.\,A.&&\\
\Avtors{Havanskov~V.\,A.} see~Minin~V.\,A.&&\\
\Avtors{Inkova~O.\,Yu.} see~Zatsman~I.\,M.&&\\
\Avtors{Isachenko~R.\,V.\ and Strijov~V.\,V.} Metric learning in
multiclass time series classification\linebreak
\\[-12pt]
\hspace*{23pt}problem&2&48--57\\
\end{tabular}
}
\pagebreak

\def\leftfootline{\small{\textbf{\thepage}
\hfill INFORMATIKA I EE PRIMENENIYA~--- INFORMATICS AND APPLICATIONS\ \ \ 2016\
\ \ volume~10\ \ \ issue\ 4}
}%
 \def\rightfootline{\small{INFORMATIKA I EE PRIMENENIYA~---
INFORMATICS AND APPLICATIONS\ \ \ 2016\ \ \ volume~10\ \ \ issue\ 4
\hfill \textbf{\thepage}}}

\def\leftkol{2016 AUTHOR INDEX} % ENGLISH ABSTRACTS}

\def\rightkol{2016 AUTHOR INDEX} %ENGLISH ABSTRACTS}


{\tabcolsep=2.83pt
\begin{tabular}{p{382pt}cc}
&\textbf{Issue} & \textbf{Page}\\[6pt]
\Avtors{Kadaner~A.\,I.} see~Chertok~A.\,V.&&\\[.255pt]
\Avtors{Kalinichenko~L.\,A., Volnova~A.\,A., Gordov~E.\,P.,
Kiselyova~N.\,N., Kovaleva~D.\,A., Malkov~O.\,Yu., Okladnikov~I.\,G.,
Podkolodnyy~N.\,L., Pozanenko~A.\,S., Ponomareva~N.\,V.,
Stupnikov~S.\,A.,} \textbf{and Fazliev~A.\,Z.} Data access challenges for data
intensive\linebreak
\\[-12pt]
\hspace*{23pt}research in Russia&1& 2--22\\[.255pt]
\Avtors{Karasikov~M.\,E.\ and Strijov~V.\,V.} Feature-based
time-series classification&4&121--131\\[.255pt]
\Avtors{Khazeeva~G.\,T.} see~Chertok~A.\,V.&&\\[.255pt]
\Avtors{Khokhlov~Yu.\,S.} Multivariate fractional Levy motion and its
applications&2&\hphantom{1}98--106\\[.255pt]
\Avtors{Kirikov~I.\,A., Kolesnikov~A.\,V., Listopad~S.\,V., and
Rumovskaya~S.\,B.} Fine-grained hybrid\linebreak
\\[-12pt]
\hspace*{23pt}intelligent systems. Part 2:
Bidirectional hybridization&1&\hphantom{1}96--105\\[.255pt]
\Avtors{Kirikov~I.\,A., Kolesnikov~A.\,V., Listopad~S.\,V., and
Rumovskaya~S.\,B.} ``Virtual council''~---\linebreak
\\[-12pt]
\hspace*{23pt}source environment
supporting complex diagnostic decision making&3&81--90\\[.255pt]
\Avtors{Kiselyova~N.\,N.} see~Kalinichenko~L.\,A.&&\\[.255pt]
\Avtors{Kolesnikov A.\,V., Listopad~S.\,V., Rumovskaya~S.\,B., and
Danishevsky~V.\,I.} Informal axiomatic\linebreak
\\[-12pt]
\hspace*{23pt}theory of~the~role visual models&4&114--120\\[.255pt]
\Avtors{Kolesnikov~A.\,V.} see~Kirikov~I.\,A.&&\\[.255pt]
\Avtors{Kolesnikov~A.\,V.} see~Kirikov~I.\,A.&&\\[.255pt]
\Avtors{Kolin~K.\,K.} Humanitarian aspects of information
security&3&111--121\\[.255pt]
\Avtors{Konovalov~M.\,G.\ and Razumchik~R.\,V.} Dispatching
to~two parallel nonobservable queues using\linebreak
\\[-12pt]
\hspace*{23pt}only static
information&4&57--67\\[.255pt]
\Avtors{Korchagin~A.\,Yu.} see~Korolev~V.\,Yu.&&\\[.255pt]
\Avtors{Korchagin~A.\,Yu.} see~Korolev~V.\,Yu.&&\\[.255pt]
\Avtors{Korepanov~E.\,R.} see~Sinitsyn~I.\,N.&&\\[.255pt]
\Avtors{Korepanov~E.\,R.} see~Sinitsyn~I.\,N.&&\\[.255pt]
\Avtors{Korolev~V.\,Yu., Korchagin~A.\,Yu., and Zeifman~A.\,I.} The
Poisson theorem for Bernoulli trials\linebreak
\\[-12pt]
\hspace*{23pt}with~a~random probability
of~success and~a~discrete analog of~the~Weibull distribution&4&11--20\\[.255pt]
\Avtors{Korolev~V.\,Yu., Zeifman~A.\,I., and Korchagin~A.\,Yu.}
Asymmetric Linnik distributions as~limit\linebreak
\\[-12pt]
\hspace*{23pt}laws for~random sums
of~independent random variables with~finite variances&4&21--33\\[.255pt]
\Avtors{Koucheryavy~E.\,A.} see~Ometov~A.\,Ya.&&\\[.255pt]
\Avtors{Kovaleva~D.\,A.} see~Kalinichenko~L.\,A.&&\\[.255pt]
\Avtors{Kovalyov~S.\,P.} Metaprogramming to increase
manufacturability of large-scale software-\linebreak
\\[-12pt]
\hspace*{23pt}intensive systems&1&56--66\\[.255pt]
\Avtors{Krivenko~M.\,P.} Significance tests of feature selection for
classification&3&32--40\\[.255pt]
\Avtors{Kruzhkov~M.\,G.} see~Zalizniak~Anna~A.&&\\[.255pt]
\Avtors{Kruzhkov~M.\,G.} see~Zatsman~I.\,M.&&\\[.255pt]
\Avtors{Kudryavtsev~A.\,A.} Bayesian queueing and reliability models:
\textit{A~priori} distributions with\linebreak
\\[-12pt]
\hspace*{23pt}compact support&1&67--71\\[.255pt]
\Avtors{Kudryavtsev~A.\,A.} Characteristics dependent on the balance
coefficient in Bayesian models\linebreak
\\[-12pt]
\hspace*{23pt}with compact support of \textit{a priori}
distributions&3&77--80\\[.255pt]
\Avtors{Kudryavtsev~A.\,A.\ and Palionnaia~S.\,I.} Bayesian recurrent
model of reliability growth:\linebreak
\\[-12pt]
\hspace*{23pt}Parabolic distribution of parameters&2&80--83\\[.255pt]
\Avtors{Kudryavtsev~A.\,A.\ and Titova~A.\,I.} Bayesian queuing
and~reliability models: Degenerate-\linebreak
\\[-12pt]
\hspace*{23pt}Weibull case&4&68--71\\[.255pt]
\Avtors{Leontyev~N.\,D.\ and Ushakov~V.\,G.} Analysis of a queueing
system with autoregressive arrivals\linebreak
\\[-12pt]
\hspace*{23pt}and nonpreemptive priority&3&15--22\\[.255pt]
\Avtors{Listopad~S.\,V.} see~Kirikov~I.\,A.&&\\[.255pt]
\Avtors{Listopad~S.\,V.} see~Kirikov~I.\,A.&&\\[.255pt]
\Avtors{Listopad~S.\,V.} see~Kolesnikov A.\,V.&&\\[.255pt]
\Avtors{Malkov~O.\,Yu.} see~Kalinichenko~L.\,A.&&\\[.255pt]
\Avtors{Markov~A.\,S., Monakhov~M.\,M., and
Ulyanov~V.\,V.} Generalized Cornish--Fisher expansions\linebreak
\\[-12pt]
\hspace*{23pt}for distributions of statistics based on samples
of random size&2&84--91\\[.255pt]
\Avtors{Melnikov~A.\,K.\ and Ronzhin~A.\,F.} Generalized statistical
method of~text analysis based\linebreak
\\[-12pt]
\hspace*{23pt}on~calculation of~probability distributions
of~statistical values&4&89--95\\
\end{tabular}
}
\pagebreak

\def\leftfootline{\small{\textbf{\thepage}
\hfill INFORMATIKA I EE PRIMENENIYA~--- INFORMATICS AND APPLICATIONS\ \ \ 2016\
\ \ volume~10\ \ \ issue\ 4}
}%
 \def\rightfootline{\small{INFORMATIKA I EE PRIMENENIYA~---
INFORMATICS AND APPLICATIONS\ \ \ 2016\ \ \ volume~10\ \ \ issue\ 4
\hfill \textbf{\thepage}}}

\def\leftkol{2016 AUTHOR INDEX} % ENGLISH ABSTRACTS}

\def\rightkol{2016 AUTHOR INDEX} %ENGLISH ABSTRACTS}


{\tabcolsep=3pt
\begin{tabular}{p{381pt}cc}
&\textbf{Issue} & \textbf{Page}\\[6pt]
\Avtors{Meykhanadzhyan~L.\,A.} Stationary characteristics of the finite
capacity queueing system with\linebreak
\\[-12pt]
\hspace*{23pt}inverse service order and generalized
probabilistic priority&2&123--131\\[.23pt]
\Avtors{Miller~G.\,B.} see~Borisov~A.\,V.&&\\[.23pt]
\Avtors{Minin~V.\,A., Zatsman~I.\,M., Havanskov~V.\,A., and
Shubnikov~S.\,K.} Intensity of citation of scientific publications in
inventions on information and computer technologies patented\linebreak
\\[-12pt]
\hspace*{23pt}in Russia by domestic and foreign applicants&2&107--122\\[.23pt]
\Avtors{Monakhov~M.\,M.} see~Markov~A.\,S.&&\\[.23pt]
\Avtors{Naumov~V.\,A.\ and Samouylov~K.\,E.} On relationship
between queuing systems with resources\linebreak
\\[-12pt]
\hspace*{23pt}and Erlang networks&3&\hphantom{1}9--14\\[.23pt]
\Avtors{Okladnikov~I.\,G.} see~Kalinichenko~L.\,A.&&\\[.23pt]
\Avtors{Ometov~A.\,Ya., Andreev~S.\,D., Turlikov~A.\,M., and
Koucheryavy~E.\,A.} Performance analysis of\linebreak
\\[-12pt]
\hspace*{23pt}a wireless data
aggregation system with contention for contemporary sensor
networks&3&23--31\\[.23pt]
\Avtors{Palionnaia~S.\,I.} see~Kudryavtsev~A.\,A.&&\\[.23pt]
\Avtors{Podkolodnyy~N.\,L.} see~Kalinichenko~L.\,A.&&\\[.23pt]
\Avtors{Ponomareva~N.\,V.} see~Kalinichenko~L.\,A.&&\\[.23pt]
\Avtors{Popkova~N.\,A.} see~Zatsman~I.\,M.&&\\[.23pt]
\Avtors{Pozanenko~A.\,S.} see~Kalinichenko~L.\,A.&&\\[.23pt]
\Avtors{Razumchik~R.\,V.} see~Konovalov~M.\,G.&&\\[.23pt]
\Avtors{Ronzhin~A.\,F.} see~Melnikov~A.\,K.&&\\[.23pt]
\Avtors{Rumovskaya~S.\,B.} see~Kirikov~I.\,A.&&\\[.23pt]
\Avtors{Rumovskaya~S.\,B.} see~Kirikov~I.\,A.&&\\[.23pt]
\Avtors{Rumovskaya~S.\,B.} see~Kolesnikov A.\,V.&&\\[.23pt]
\Avtors{Samouylov~K.\,E.} see~Gaidamaka~Yu.\,V.&&\\[.23pt]
\Avtors{Samouylov~K.\,E.} see~Naumov~V.\,A.&&\\[.23pt]
\Avtors{Serebryanskii~S.\,M.} see~Tyrsin~A.\,N.&&\\[.23pt]
\Avtors{Seyful-Mulyukov~R.\,B.} see~Callaos~N.\,K.&&\\[.23pt]
\Avtors{Shestakov~O.\,V.} Statistical properties of the denoising method
based on the stabilized hard\linebreak
\\[-12pt]
\hspace*{23pt}thresholding&2&65--69\\[.23pt]
\Avtors{Shestakov~O.\,V.} The strong law of large numbers for the risk
estimate in the problem of\linebreak
\\[-12pt]
\hspace*{23pt}tomographic image reconstruction from
projections with a correlated noise&3&41--45\\[.23pt]
\Avtors{Shestakov~O.\,V.} see~Zakharova~T.\,V.&&\\[.23pt]
\Avtors{Shnurkov~P.\,V., Gorshenin~A.\,K., and Belousov~V.\,V.}
Analytical solution of~the~optimal control\linebreak
\\[-12pt]
\hspace*{23pt}task of~a~semi-Markov
process with~finite set of~states&4&72--88\\[.23pt]
\Avtors{Shnurkov~P.\,V., Zasypko~V.\,V., Belousov~V.\,V., and
Gorshenin~A.\,K.} Development of the algorithm of numerical solution
of the optimal investment control problem\linebreak
\\[-12pt]
\hspace*{23pt}in the closed dynamical model of three-sector economy&1&82--95\\[.23pt]
\Avtors{Shorgin~S.\,Ya.} see~Gaidamaka~Yu.\,V.&&\\[.23pt]
\Avtors{Shorgin~V.\,S.} see~Agalarov~Ya.\,M.&&\\[.23pt]
\Avtors{Shubnikov~S.\,K.} see~Minin~V.\,A.&&\\[.23pt]
\Avtors{Sidorkin~I.\,I.} see~Arkhipov~O.\,P.&&\\[.23pt]
\Avtors{Sinitsyn~I.\,N.} Analytical modeling of processes in stochastic
systems with complex fractional\linebreak
\\[-12pt]
\hspace*{23pt}order Bessel nonlinearities&3&55--65\\[.23pt]
\Avtors{Sinitsyn~I.\,N.} Orthogonal supoptimal filters for nonlinear
stochastic systems on manifolds&1&34--44\\[.23pt]
\Avtors{Sinitsyn~I.\,N.\ and Korepanov~E.\,R.} Normal Pugachev
conditionally-optimal filters and extra-\linebreak
\\[-12pt]
\hspace*{23pt}polators for state linear stochastic systems&2&14--23\\[.23pt]
\Avtors{Sinitsyn~I.\,N.\ and Sinitsyn~V.\,I.} Analytical modeling of
distributions in stochastic systems on\linebreak
\\[-12pt]
\hspace*{23pt}manifolds based on ellipsoidal approximation&1&45--55\\[.23pt]
\Avtors{Sinitsyn~I.\,N., Sinitsyn~V.\,I., and
Korepanov~E.\,R.} Ellipsoidal suboptimal filters for nonlinear\linebreak
\\[-12pt]
\hspace*{23pt}stochastic systems on manifolds&2&24--35\\[.23pt]
\Avtors{Sinitsyn~V.\,I.} see~Sinitsyn~I.\,N.&&\\[.23pt]
\Avtors{Sinitsyn~V.\,I.} see~Sinitsyn~I.\,N.&&\\[.23pt]
\Avtors{Skvortsov~N.\,A.} see~Stupnikov~S.\,A.&&\\[.23pt]
\Avtors{Sokolov~I.\,A.} see~Chertok~A.\,V.&&\\
\end{tabular}
}
\pagebreak

\def\leftfootline{\small{\textbf{\thepage}
\hfill INFORMATIKA I EE PRIMENENIYA~--- INFORMATICS AND APPLICATIONS\ \ \ 2016\
\ \ volume~10\ \ \ issue\ 4}
}%
 \def\rightfootline{\small{INFORMATIKA I EE PRIMENENIYA~---
INFORMATICS AND APPLICATIONS\ \ \ 2016\ \ \ volume~10\ \ \ issue\ 4
\hfill \textbf{\thepage}}}

\def\leftkol{2016 AUTHOR INDEX} % ENGLISH ABSTRACTS}

\def\rightkol{2016 AUTHOR INDEX} %ENGLISH ABSTRACTS}


{\tabcolsep=3pt
\begin{tabular}{p{382pt}cc}
&\textbf{Issue} & \textbf{Page}\\[6pt]
\Avtors{Sopin~E.\,S.} see~Gaidamaka~Yu.\,V.&&\\
\Avtors{Strijov~V.\,V.} see~Goncharov~A.\,V.&&\\
\Avtors{Strijov~V.\,V.} see~Isachenko~R.\,V.&&\\
\Avtors{Strijov~V.\,V.} see~Karasikov~M.\,E.&&\\
\Avtors{Stupnikov~S.\,A., Briukhov~D.\,O., and Skvortsov~N.\,A.}
Co-lending systemic risk analysis over\linebreak
\\[-12pt]
\hspace*{23pt}heterogeneous data collections&1&23--33\\
\Avtors{Stupnikov~S.\,A.} see~Kalinichenko~L.\,A.&&\\
\Avtors{Suchkov~A.\,P.} see~Zatsarinny~A.\,A.&&\\
\Avtors{Timonina~E.\,E.} see~Grusho~A.\,A.&&\\
\Avtors{Titova~A.\,I.} see~Kudryavtsev~A.\,A.&&\\
\Avtors{Turlikov~A.\,M.} see~Ometov~A.\,Ya.&&\\
\Avtors{Tyrsin~A.\,N.\ and Serebryanskii~S.\,M.} Recognition of
dependences on the basis of inverse\linebreak
\\[-12pt]
\hspace*{23pt}mapping&2&58--64\\
\Avtors{Ulyanov~V.\,V.} see~Markov~A.\,S.&&\\
\Avtors{Ushakov~V.\,G.} Queueing system with working vacations and
hyperexponential input stream&2&92--97\\
\Avtors{Ushakov~V.\,G.} see~Leontyev~N.\,D.&&\\
\Avtors{Volnova~A.\,A.} see~Kalinichenko~L.\,A.&&\\
\Avtors{Yakovlev~O.\,A.\ and Gasilov~A.\,V.} Speeded-up stereo
matching using geodesic support weights&3&\hphantom{1}98--104\\
\Avtors{Zabezhailo~M.\,I.} see~Grusho~A.\,A.&&\\
\Avtors{Zabezhailo~M.\,I.} see~Grusho~A.\,A.&&\\
\Avtors{Zakharova~T.\,V.\ and Shestakov~O.\,V.} Precision analysis of
wavelet processing of aerodynamic\linebreak
\\[-12pt]
\hspace*{23pt}flow patterns&3&46--54\\
\Avtors{Zalizniak~Anna~A.\ and Kruzhkov~M.\,G.} Database
of~Russian impersonal verbal constructions&4&132--141\\
\Avtors{Zasypko~V.\,V.} see~Shnurkov~P.\,V.&&\\
\Avtors{Zatsarinny~A.\,A.\ and Suchkov~A.\,P.} Systems engineering
approaches to~the~establishment of\linebreak
\\[-12pt]
\hspace*{23pt}a~system for~decision support based
on~situational analysis&4&105--113\\
\Avtors{Zatsarinny~A.\,A.} see~Grusho~A.\,A.&&\\
\Avtors{Zatsman~I.\,M., Inkova~O.\,Yu., Kruzhkov~M.\,G., and
Popkova~N.\,A.} Representation of cross-\linebreak
\\[-12pt]
\hspace*{23pt}lingual knowledge about
connectors in supracorpora databases&1&106--118\\
\Avtors{Zatsman~I.\,M.} see~Minin~V.\,A.&&\\
\Avtors{Zeifman~A.\,I.} see~Korolev~V.\,Yu.&&\\
\Avtors{Zeifman~A.\,I.} see~Korolev~V.\,Yu.&&\\
\end{tabular}
}

%\thispagestyle{myheadings}
\def\leftfootline{\small{\textbf{\thepage}
\hfill INFORMATIKA I EE PRIMENENIYA~--- INFORMATICS AND APPLICATIONS\ \ \ 2016\
\ \ volume~10\ \ \ issue\ 4}
}%
 \def\rightfootline{\small{INFORMATIKA I EE PRIMENENIYA~---
INFORMATICS AND APPLICATIONS\ \ \ 2016\ \ \ volume~10\ \ \ issue\ 4
\hfill \textbf{\thepage}}}

 \label{end\stat}

\newpage

%\def\stat{rekl}
%\label{preobr}

%\def\tit{АКАДЕМИК ПУГАЧЁВ  ВЛАДИМИР СЕМЁНОВИЧ\\
%25.03.1911--25.03.1998}


%   \vspace*{-48pt}
%   \begin{center}\LARGE
%Академик Пугачёв  Владимир Семёнович\\ (25.03.1911--25.03.1998)
%   \end{center}
   
   %\vspace*{2.5mm}
   
   \begin{center}

{\prgsh\LARGE
ОБЪЯВЛЕНИЯ О КОНФЕРЕНЦИЯХ}

\end{center}
%\hrule

\vspace*{6pt}

   
   \vspace*{10mm}
   
   \thispagestyle{empty}

\noindent
\begin{tabular}{cc}
%\begin{center}
\multicolumn{1}{c}{\raisebox{-40pt}[0pt][0pt]{\mbox{%
\epsfxsize=33mm
\epsfbox{vspu.eps}
}}}
%\end{center}
&
\tabcolsep=0pt\begin{tabular}{c}
{\prg{\Large\textbf{XII Всероссийское совещание}}}\\[6pt]
{\prg{\Large\textbf{по проблемам управления}}}\\[12pt]
{\prg{\large 16--19 июня 2014~г.}}\\[6pt] 
{\prg{\large Институт проблем управления имени В.\,А.~Трапезникова РАН}}\\[6pt]
{\prg{\large Москва, Россия}}
\end{tabular}
\end{tabular}

\vspace*{60pt}

     
 { %\large    
 XII Всероссийское совещание по проблемам управления (ВСПУ XII), посвященное 75-летию 
Института проблем управления (ИПУ) имени В.\,А.~Трапезникова РАН, проводится 16--19~июня 
2014~г.\ 
в ИПУ РАН (г.~Москва, Россия). ВСПУ XII организуется ИПУ РАН при поддержке РФФИ, Отделения 
энергетики, машиностроения, механики и процессов управления Российской академии наук, 
Российского 
национального комитета по автоматическому управлению, Академии навигации и управ\-ле\-ния 
движением, 
Научного совета РАН по комплексным проблемам управления и автоматизации, Совета по 
мехатронике и робототехнике РАН. Официальный язык Совещания~--- русский.

\vspace*{24pt}
     
     \textbf{Направления работы}
     \begin{enumerate}[1.]
\item Теория систем управления
\item Управление подвижными объектами и навигация
\item Интеллектуальные системы управления
\item Управление в промышленности, транспортом и логистикой
\item Управление системами междисциплинарной природы
\item Средства измерения, вычислений и контроля в управлении
\item Системный анализ и принятие решений в задачах управления
\item Информационные технологии в управлении
\item Проблемы образования в области управления: современное содержание и технологии обучения
\end{enumerate}

\vspace*{24pt}

     Подробная информация о Совещании находится на сайте {\sf http://vspu2014.ipu.ru}. Срок 
окончательной подачи докладов через систему подачи докладов на сайте~--- \textbf{30~ноября} 
2013~г.
}

%\include{rekl-1}

%\end{document}

%   \vspace*{-48pt}

\begin{center}
\vspace*{6pt}
\mbox{%
\epsfxsize=53.502mm
\epsfbox{foto-1.eps}
}
\end{center}

\vspace*{6pt} %Академик


   \begin{center}
\fbox{\Large\textbf{Профессор Игорь Алексеевич Ушаков}}\\[12pt]
\textbf{\large 22.01.1935--27.02.2015}
   \end{center}


   %\vspace*{2.5mm}

   \vspace*{5mm}

   \thispagestyle{empty}

%\

%\vspace*{-12pt}


Редакционный совет и редакционная коллегия журнала <<Информатика и~её применения>> с~глубоким прискорбием извещают, что 27~февраля 2015~г.\ после тяжелой
и~продолжительной болезни скончался Игорь Алексеевич Ушаков~--- доктор технических наук, профессор, член редколлегии журнала <<Информатика и ее применения>>.

Игорь Алексеевич Ушаков окончил Московский авиационный институт, в~1963~г.\ защитил кандидатскую, а~в~1968~г.~--- докторскую диссертацию. С~1958 по 1989~гг.\ работал в~ряде научно-исследовательских организаций СССР, в~том числе руководил отделами в~НИИ АА и~ВЦ АН СССР; с 1969 по 1989 гг. преподавал в~МФТИ (был профессором, а~затем заведующим кафедрой) и~в~МЭИ. С~1989~г.~---- в~США: являлся профессором университета Дж.\ Вашингтона, университета Дж.\ Мэйсона и~Калифорнийского университета, сотрудником компаний MCI, Qualcomm и Hughes.

И.\,А.~Ушаков с момента основания журнала <<Надежность и~контроль качества>> был заместителем ответственного редактора, а~затем на протяжении многих лет членом редколлегии. В~2006~г.\ основал электронный международный журнал ``Reliability: Theory \& Application'', главным редактором которого оставался до конца жизни.

Учебниками и справочниками по теории надежности, написанными И.\,А.~Ушаковым, пользовались и~пользуются несколько поколений ученых и~специалистов в~разных странах мира.

Игорь Алексеевич всегда уделял огромное внимание работе с~молодежью; более~50 его учеников защитили докторские и~кандидатские диссертации.

И.\,А.~Ушаков вел активную научно-про\-све\-ти\-тель\-скую деятельность. В~частности, он был одним из организаторов и~руководителей Московского кабинета качества и~надежности при Политехническом музее (целью этого Кабинета было оказание консультаций работникам промышленных предприятий и~чтение курсов лекций для инженеров, занимающихся проблемой надежности). Находясь в~США, И.\,А.~Ушаков создал международный ин\-тер\-нет-фо\-рум им.\ Б.\,В.~Гнеденко, объединивший около~400~видных специалистов по приложениям теории вероятностей и~математической статистики, преимущественно в~об\-ласти теории надежности и~анализа риска, из десятков стран мира; коллективным членов этого Форума является и~наш журнал. Цели Форума~--- содействие контактам между специалистами из разных стран, организация обмена профессиональными 
новостями и~информацией (новые публикации, предстоящие события и~др.). Также необходимо отметить большое число на\-уч\-но-по\-пу\-ляр\-ных работ, опубликованных И.\,А.~Ушаковым.

И.\,А.~Ушаков обладал большим личным обаянием, имел широкий круг интересов. Все знавшие И.\,А.~Ушакова всегда будут помнить его как замечательного ученого и~прекрасного человека.

\bigskip

Редакционный совет и редакционная коллегия журнала <<Информатика и~её применения>> 
выражают глубокие соболезнования родным и близким покойного, всем, кто его знал и~работал с~ним.



%\end{document}

%\include{IPPM-25}

\def\stat{cont-rus}
{%\hrule\par
%\vskip 7pt % 7pt
\vspace*{-24pt}
\raggedleft\Large \bf%\baselineskip=3.2ex
Правила подготовки рукописей  для публикации в журнале
<<Информатика~и~её~применения>> \vskip 8pt
    \hrule
    \par
\vskip 14pt plus 6pt minus 3pt }

\label{st\stat}

\def\tit{\ }

\def\aut{\ }
\def\auf{\ }

\def\leftkol{\ }
% Правила подготовки рукописей  для публикации в журнале
%<<Информатика и её применения>>

\def\rightkol{\ }
%Правила подготовки рукописей  для публикации в журнале
%<<Информатика и её применения>>}


\titele{\tit}{\aut}{\auf}{\leftkol}{\rightkol}


\vspace*{-60pt}
{ %\small

Журнал <<Информатика и её применения>>
публикует теоретические, обзорные и дискуссионные статьи,
посвященные научным исследованиям и разработкам в области
информатики и ее приложений.

Журнал издается на русском языке. По специальному решению
редколлегии отдельные статьи могут печататься на английском языке.

Тематика журнала охватывает следующие направления:
\begin{itemize}
\item теоретические основы информатики;\\[-15pt]
      \item
математические методы исследования сложных систем и процессов;\\[-15pt]
           \item
информационные системы и сети;\\[-15pt]
                \item
информационные технологии;\\[-15pt]
                     \item
архитектура и программное обеспечение вычислительных комплексов и сетей.\\[-15pt]
\end{itemize}


\noindent
\begin{enumerate}[1.]
\item В журнале печатаются статьи, содержащие результаты, ранее не опубликованные и
не предназначенные к одновременной публикации в других изданиях.

%Публикация не должна нарушать закон об авторских правах.
Публикация предоставленной автором(ами) рукописи не должна нарушать 
положений глав~69, 70 раздела~VII части~IV Гражданского кодекса, 
которые определяют права на результаты интеллектуальной деятельности 
и~средства индивидуализации, в~том числе авторские права, в~РФ.

Ответственность за нарушение авторских прав, в~случае предъявления претензий к~редакции журнала,  
несут авторы статей.



Направляя рукопись в редакцию, авторы сохраняют свои права на данную
рукопись и при этом передают учредителям и редколлегии журнала неисключительные права на
издание статьи на русском языке 
(или на языке статьи, если он отличен от рус\-ско\-го) и~на перевод ее на английский
язык, а~также на
ее распространение в России и за рубежом. 
Каждый автор должен представить в~редакцию подписанный 
с~его стороны <<Лицензионный договор о~передаче неисключительных прав 
на использование произведения>>, текст которого размещен по адресу 
{\sf http://www.ipiran.ru/publications/licence.doc}. 
Этот договор может быть пред\-став\-лен в~бумажном (в~2-х экз.)\ 
или в~электронном виде (отсканированная копия заполненного и~подписанного документа).




Редколлегия вправе запросить у авторов экспертное заключение о возможности
пуб\-ли\-ка\-ции пред\-став\-лен\-ной статьи в открытой печати.\\[-13.5pt]

\item К статье прилагаются данные автора (авторов) (см.\ п.~8). При наличии нескольких
авторов указывается фамилия автора, ответственного за переписку с редакцией.\\[-13.5pt]

\item Редакция журнала осуществляет экспертизу присланных статей в соответствии с
принятой в журнале процедурой рецензирования.

Возвращение рукописи на доработку не означает ее принятия к печати.

Доработанный вариант с ответом на замечания рецензента необходимо прислать в
редакцию.\\[-13.5pt]

\item Решение редколлегии о публикации статьи или ее отклонении сообщается авторам.

Редколлегия может также направить авторам текст рецензии на их статью. Дискуссия по
поводу отклоненных статей не ведется.\\[-13.5pt]

%\pagebreak

\item Редактура статей высылается авторам для просмотра. Замечания к редактуре должны
быть присланы авторами в кратчайшие сроки.\\[-13.5pt]

\item Рукопись предоставляется в электронном виде в форматах MS WORD (.doc или
.docx) или \LaTeX\  (.tex), дополнительно~--- в формате .pdf, на дискете, лазерном диске
или электронной почтой. Предоставление бумажной рукописи необязательно.\\[-13.5pt]

\item При подготовке рукописи в MS Word рекомендуется использовать следующие
настройки.

Параметры страницы:
формат~--- А4; ориентация~--- книжная; поля (см): внутри~--- 2,5, снаружи~--- 1,5,
сверху~--- 2, снизу~--- 2, от края до нижнего колонтитула~--- 1,3.

Основной текст: стиль~--- <<Обычный>>, шрифт~--- Times New Roman, размер~---
14~пунк\-тов, абзацный отступ~--- 0,5~см, 1,5~интервала, выравнивание~--- по ширине.

\pagebreak

\def\leftkol{Правила подготовки рукописей  для публикации в журнале
<<Информатика и её применения>>}

\def\rightkol{Правила подготовки рукописей  для публикации в журнале
<<Информатика и её применения>>}



Рекомендуемый объем рукописи~--- не свыше 10~страниц указанного формата.
При превышении указанного объема редколлегия вправе потребовать от 
автора сокращения объема рукописи.


Сокращения слов, помимо стандартных, не допускаются. Допускается минимальное
количество аббревиатур.


Все страницы рукописи нумеруются.

Шаблоны оформления представлены в интернете:

\noindent
 {\sf
http://www.ipiran.ru/journal/template\_iiep\_ssi\_2024.zip}\\[-14pt]

\item Статья должна содержать следующую информацию на {\bfseries\textit{русском и
английском языках}}:\\[-16pt]

\begin{itemize}
\item название статьи;\\[-15pt]
\item Ф.И.О.\ авторов, на английском можно только имя и фамилию;\\[-15pt]
\item место работы, с указанием почтового адреса организации и электронного адреса каждого
автора;\\[-15pt]
\item сведения об авторах, в соответствии с форматом, образцы которого
представлены на страницах:



\def\leftfootline{\small{\textbf{\thepage}
\hfill ИНФОРМАТИКА И ЕЁ ПРИМЕНЕНИЯ\ \ \ том\ 18\ \ \ выпуск\ 3\ \ \ 2024}
}%
 \def\rightfootline{\small{ИНФОРМАТИКА И ЕЁ ПРИМЕНЕНИЯ\ \ \ том\ 18\ \ \ выпуск\ 3\ \ \ 2024
\hfill \textbf{\thepage}}}



{\sf http://www.ipiran.ru/journal/issues/2013\_07\_01/authors.asp} и

{\sf http://www.ipiran.ru/journal/issues/2013\_07\_01\_eng/authors.asp};
\item аннотация (не менее 100~слов на каждом из языков). Аннотация~--- это краткое
резюме работы, которое может публиковаться отдельно. Она является основным
источником информации в~ин\-фор\-ма\-ци\-он\-ных системах и базах данных. Английская
аннотация должна быть оригинальной, может не быть дословным переводом русского
текста и должна быть написана хорошим английским языком. В~аннотации не должно
быть ссылок на литературу и, по возможности, формул;\\[-15pt]
\item ключевые слова~--- желательно из принятых в мировой
на\-уч\-но-тех\-ни\-че\-ской литературе тематических тезаурусов. Предложения не
могут быть ключевыми словами;\\[-15pt]
\item источники финансирования работы (ссылки на гранты, проекты,
поддерживающие организации и~т.\,п.).
\end{itemize}



%\pagebreak

\item  Требования к спискам литературы.\\[-14pt]

Ссылки на литературу в тексте статьи нумеруются (в квадратных скобках) и
располагаются в каждом из списков литературы в порядке  первых упоминаний. Если источник имеет DOI и/или EDN,
то их необходимо указывать.

Списки литературы представляются в двух вариантах:\\[-14pt]


\noindent
\begin{enumerate}[(1)]
\item \textbf{Список литературы к русскоязычной части}. Русские и английские
работы~---  на языке и в алфавите оригинала;\\[-14.5pt]
\item  \textbf{References}. Русские работы и работы на других языках~--- в латинской
транслитерации с переводом на английский язык; английские работы и работы на других
языках~--- на языке оригинала.
\end{enumerate}

Необходимо для составления списка ``References'' пользоваться размещенной на сайте
{\sf http://www. translit.net/ru/bgn/} бесплатной программой транслитерации русского
 текста в~латиницу. %, при этом в~за\-клад\-ке <<варианты\ldots>> следует выбратьопцию BGN.

Список литературы ``References'' приводится полностью отдельным блоком, повторяя все
позиции из списка литературы к русскоязычной части, независимо от того, имеются или
нет в нем иностранные источники. Если в списке литературы к русскоязычной части есть
ссылки на иностранные публикации, набранные латиницей, они полностью повторяются в
списке ``References''.

Ниже приведены примеры ссылок на различные виды публикаций в списке ``References''.

\def\leftfootline{\small{\textbf{\thepage}
\hfill ИНФОРМАТИКА И ЕЁ ПРИМЕНЕНИЯ\ \ \ том\ 18\ \ \ выпуск\ 3\ \ \ 2024}
}%
 \def\rightfootline{\small{ИНФОРМАТИКА И ЕЁ ПРИМЕНЕНИЯ\ \ \ том\ 18\ \ \ выпуск\ 3\ \ \ 2024
\hfill \textbf{\thepage}}}

{\small

\noindent
\textbf{Описание статьи из журнала:}

\Aue{Zagurenko, A.\,G., V.\,A.~Korotovskikh, A.\,A.~Kolesnikov, A.\,V.~Timonov, and D.\,V.~Kardymon}. 2008.
Tekhniko-ekonomicheskaya optimizatsiya dizayna gidrorazryva plasta [Technical and
economic optimization of the design
of hydraulic fracturing]. \textit{Neftyanoe hozyaystvo} [\textit{Oil Industry}] 11:54--57.

\Aue{Zhang, Z., and D.~Zhu}. 2008. Experimental research on the localized
electrochemical micromachining. \textit{Russ. J.~Electrochem.}  44(8):926--930.
{\sf doi:10.1134/S1023193508080077}.

\noindent
\textbf{Описание статьи из электронного журнала:}

\Aue{Swaminathan, V., E.~Lepkoswka-White, and B.\,P.~Rao}. 1999. Browsers or buyers in cyberspace? An
investigation of electronic factors influencing electronic exchange. \textit{JCMC}
5(2). Available at: {\sf http://www.ascusc.org/jcmc/vol5/issue2/} (accessed April~28, 2011).

\def\leftkol{Правила подготовки рукописей  для публикации в журнале
<<Информатика и её применения>>}

\def\rightkol{Правила подготовки рукописей  для публикации в журнале
<<Информатика и её применения>>}


\noindent
\textbf{Описание статьи из продолжающегося издания (сборника трудов):}

\Aue{Astakhov, M.\,V., and T.\,V.~Tagantsev}. 2006. Eksperimental'noe
issledovanie prochnosti soedineniy ``stal'--kompozit'' [Experimental study of
the strength of joints ``steel--composite'']. \textit{Trudy MGTU
``Matematicheskoe modelirovanie slozhnykh tekh\-ni\-che\-skikh sistem''}
[\textit{Bauman MSTU ``Mathematical Modeling of Complex Technical
Systems'' Proceedings}]. 593:125--130.


\pagebreak



\noindent
\textbf{Описание материалов конференций:}

\Aue{Usmanov, T.\,S., A.\,A.~Gusmanov, I.\,Z.~Mullagalin, R.\,Ju.~Muhametshina, A.\,N.~Chervyakova, and
A.\,V.~Sveshnikov}. 2007. Osobennosti proektirovaniya razrabotki mestorozhdeniy
s primeneniem gidrorazryva
plasta [Features of the design of field development with the use of hydraulic fracturing].
\textit{Trudy 6-go
Mezhdu\-na\-rod\-no\-go Simpoziuma ``Novye resursosberegayushchie tekhnologii nedropol'zovaniya i povysheniya
neftegazootdachi''} [\textit{6th  Symposium (International) ``New Energy Saving Subsoil Technologies and
the Increasing of the Oil and Gas Impact'' Proceedings}]. Moscow. 267--272.



\def\leftfootline{\small{\textbf{\thepage}
\hfill ИНФОРМАТИКА И ЕЁ ПРИМЕНЕНИЯ\ \ \ том\ 18\ \ \ выпуск\ 3\ \ \ 2024}
}%
 \def\rightfootline{\small{ИНФОРМАТИКА И ЕЁ ПРИМЕНЕНИЯ\ \ \ том\ 18\ \ \ выпуск\ 3\ \ \ 2024
\hfill \textbf{\thepage}}}



\noindent
\textbf{Описание книги (монографии, сборники):}



Lindorf, L.\,S., and L.\,G.~Mamikoniants, eds. 1972.
\textit{Ekspluatatsiya turbogeneratorov s neposredstvennym
okhlazhdeniem} [\textit{Operation of turbine generators with direct cooling}].
Moscow: Energy Publs. 352~p.


\Aue{Latyshev, V.\,N.} 2009. \textit{Tribologiya rezaniya. Kn.~1: Friktsionnye protsessy
pri rezanii metallov}
[\textit{Tribology of cutting. Vol.~1: Frictional processes in metal cutting}]. Ivanovo: Ivanovskii
State Univ. 108~p.

\def\leftkol{Правила подготовки рукописей  для публикации в журнале
<<Информатика и её применения>>}

\def\rightkol{Правила подготовки рукописей  для публикации в журнале
<<Информатика и её применения>>}

\noindent
\textbf{Описание переводной книги}
(в списке литературы к русскоязычной части необходимо указать:~/ Пер.\ с англ.~---
после названия книги, а в конце ссылки указать оригинал книги в круглых скобках):
\begin{enumerate}[1.]
\item  В русскоязычной части:

\def\leftfootline{\small{\textbf{\thepage}
\hfill ИНФОРМАТИКА И ЕЁ ПРИМЕНЕНИЯ\ \ \ том\ 18\ \ \ выпуск\ 3\ \ \ 2024}
}%
 \def\rightfootline{\small{ИНФОРМАТИКА И ЕЁ ПРИМЕНЕНИЯ\ \ \ том\ 18\ \ \ выпуск\ 3\ \ \ 2024
\hfill \textbf{\thepage}}}

\Au{Тимошенко С.\,П., Янг Д.\,Х., Уивер~У.}
Колебания в инженерном деле~/ Пер.\ с англ.~--- М.: Машиностроение, 1985. 472~с.
(\Au{Timoshenko~S.\,P., Young~D.\,H., Weaver~W.}
Vibration problems in engineering.~--- 4th ed.~--- New York, NY, USA: Wiley, 1974. 521~p.)\\[-13.5pt]
\item  В англоязычной части:

\Aue{Timoshenko, S.\,P., D.\,H.~Young, and W.~Weaver}.
1974. \textit{Vibration problems in engineering}. 4th ed. New York: 
Wiley. 521~p.
\end{enumerate}

\vspace*{-3pt}


\noindent
\textbf{Описание неопубликованного документа:}


\Aue{Latypov, A.\,R., M.\,M.~Khasanov, and V.\,A.~Baikov}.
2004 (unpubl.). Geologiya i~dobycha (NGT GiD) [Geology and production (NGT GiD)]. Certificate on official registration of the computer program
No.\,2004611198. 

\noindent
\textbf{Описание интернет-ресурса:}


Pravila tsitirovaniya istochnikov [Rules for the citing of sources]. Available at: {\sf
http://www.scribd.com/doc/1034528/} (accessed February~7, 2011).

%\pagebreak

\noindent
\textbf{Описание диссертации или автореферата диссертации:}

\Aue{Semenov, V.\,I.}
2003. Matematicheskoe modelirovanie plazmy v sisteme kompaktnyy tor [Mathematical
modeling of the plasma in the compact torus].  Moscow.  D.Sc.\ Diss. 272~p.

\Aue{Kozhunova, O.\,S.} 2009. Tekhnologiya razrabotki semanticheskogo
slovarya informatsionnogo monitoringa [Technology of development of
semantic dictionary of information monitoring system].  Moscow: IPI RAN. PhD Thesis. 23~p.


\noindent
\textbf{Описание ГОСТа:}

GOST 8.586.5-2005. 2007. Metodika vypolneniya izmereniy. Izmerenie raskhoda i~kolichestva zhidkostey i~gazov
s~pomoshch'yu standartnykh suzhayushchikh ustroystv [Method of measurement.
Measurement of flow rate and volume of liquids and gases by means of orifice devices]. Moscow:
Standardinform  Publs. 10~p.

\noindent
\textbf{Описание патента:}

\Aue{Bolshakov, M.\,V., A.\,V.~Kulakov, A.\,N.~Lavrenov, and M.\,V.~Palkin}.
2006. Sposob orientirovaniya po krenu letatel'nogo
apparata s opti\-che\-skoy golovkoy
samonavedeniya [The way to orient on the roll of aircraft with optical homing head].
Patent RF No.\,2280590.
}

\item Присланные в редакцию материалы авторам не возвращаются.\\[-13.5pt]

\item При отправке файлов по электронной почте просим придерживаться следующих
правил:
\begin{itemize}
\item указывать в поле subject (тема) название журнала и фамилию автора;\\[-13.5pt]
\item указывать в тексте письма название статьи, авторов и~журнал, в~который направляется статья;\\[-13.5pt]
\item использовать attach (присоединение);\\[-13.5pt]
\item в состав электронной версии статьи должны входить: файл, содержащий текст
статьи, и файл(ы), содержащий(е) иллюстрации.\\[-13.5pt]
\end{itemize}

\item Журнал <<Информатика и её применения>> является некоммерческим изданием.
Плата за публикацию не взимается, гонорар авторам не выплачивается.
\end{enumerate}



\def\leftfootline{\small{\textbf{\thepage}
\hfill ИНФОРМАТИКА И ЕЁ ПРИМЕНЕНИЯ\ \ \ том\ 18\ \ \ выпуск\ 3\ \ \ 2024}
}%
 \def\rightfootline{\small{ИНФОРМАТИКА И ЕЁ ПРИМЕНЕНИЯ\ \ \ том\ 18\ \ \ выпуск\ 3\ \ \ 2024
\hfill \textbf{\thepage}}}


\vspace*{-1mm}

\begin{center}

\textbf{Адрес редакции журнала <<Информатика и её применения>>:} \\




Москва 119333, ул.~Вавилова, д.~44, корп.~2, ФИЦ ИУ РАН\\[-10pt]

\

Тел.: +7\,(499)\,135-86-92\ \ Факс:  +7\,(495)\,930-45-05\\[-10pt]

 \

e-mail:   {\sf iiep@frccsc.ru} (Стригина Светлана Николаевна)\\[-10pt]

\

{\sf http://www.ipiran.ru/journal/issues/}
\end{center}
}


\def\leftkol{Правила подготовки рукописей  для публикации в журнале
<<Информатика и её применения>>}

\def\rightkol{Правила подготовки рукописей  для публикации в журнале
<<Информатика и её применения>>}


\def\leftfootline{\small{\textbf{\thepage}
\hfill ИНФОРМАТИКА И ЕЁ ПРИМЕНЕНИЯ\ \ \ том\ 18\ \ \ выпуск\ 3\ \ \ 2024}
}%
 \def\rightfootline{\small{ИНФОРМАТИКА И ЕЁ ПРИМЕНЕНИЯ\ \ \ том\ 18\ \ \ выпуск\ 3\ \ \ 2024
\hfill \textbf{\thepage}}} 
\def\stat{podg-e}
{%\hrule\par
%\vskip 7pt % 7pt
\vspace*{-24pt}
\raggedleft\Large \bf%\baselineskip=3.2ex
Requirements for manuscripts submitted to Journal
``Informatics~and~Applications'' \vskip 8pt
    \hrule
    \par
\vskip 21pt plus 6pt minus 3pt }

\label{st\stat}

\def\tit{\ }

\def\aut{\ }
\def\auf{\ }

\def\leftkol{\ }

\def\rightkol{\ }
%Requirements for manuscripts submitted to Journal
%``Informatics~and~Applications''}

\titele{\tit}{\aut}{\auf}{\leftkol}{\rightkol}

\def\leftfootline{\small{\textbf{\thepage}
\hfill INFORMATIKA I EE PRIMENENIYA~--- INFORMATICS AND APPLICATIONS\ \ \ 2019\
\ \ volume~13\ \ \ issue\ 4}
}%
 \def\rightfootline{\small{INFORMATIKA I EE PRIMENENIYA~--- INFORMATICS AND APPLICATIONS\ \ \ 2019\ \ \ volume~13\ \ \ issue\ 4
\hfill \textbf{\thepage}}}

\vspace*{-60pt}

{\small

\noindent
Journal ``Informatics and Applications'' (Inform.\ Appl.)
publishes theoretical, review, and discussion
articles on the research and development in the
field of informatics and its applications.

The journal is published in Russian.
By a special decision of the editorial
board, some articles can be published in English.


The topics covered include the following areas:
\begin{itemize}
               \item
     theoretical fundamentals of informatics; \\[-14pt]
\item
mathematical methods for studying complex systems and processes; \\[-14pt]
\item
information systems and networks;\\[-14pt]
\item
information technologies; and \\[-14pt]
\item
architecture and software of computational complexes and networks. \\[-14pt]
\end{itemize}

\noindent
\begin{enumerate}[1.]
\item The Journal publishes original articles which have not been published before and are not
intended for simultaneous publication in other editions. An article submitted to the Journal must not violate the
Copyright law. Sending the manuscript to the Editorial Board, the authors retain all rights of the
owners of the manuscript and transfer the nonexclusive rights to publish the article in Russian
(or the language of the article, if not Russian) and its distribution in Russia and abroad to the
Founders and the Editorial Board. Authors should submit a letter to the Editorial Board in the
following form:

{\bfseries\textit{Agreement on the transfer of rights to publish:}}

``\textit{We, the undersigned authors of the manuscript ``\ldots'', pass to the
Founder and the Editorial Board of the Journal ``Informatics and Applications''
the nonexclusive right to publish the manuscript of the article in Russian (or
in English) in both print and electronic versions of the Journal. We affirm
that this publication does not violate the Copyright of other persons or
organizations.}

\textit{Author(s) signature(s): (name(s), address(es), date).}

This agreement should be submitted in paper form or in the form of a scanned copy (signed by
the authors).


%The Editorial Board has the right to request from the authors an official expert conclusion that
%the submitted article has no secret data prohibited for publication. \\[-13.5pt]
\item
A submitted article should be attached with \textbf{the data on the author(s)} (see item~8). If
there are several authors, the contact person should be indicated who is responsible for
correspondence with the Editorial Board and other authors about revisions and final approval
of the proofs.\\[-13.5pt]

\item The Editorial Board of the Journal examines the article according to the established
reviewing procedure. If the authors receive their article for correction after reviewing, it does not
mean that the article is approved for publication. The corrected article should be sent to the
Editorial Board for the subsequent review and approval.\\[-13.5pt]

\item The decision on the article publication or its rejection is communicated to the authors. The
Editorial Board may also send the reviews on the submitted articles to the authors. Any
discussion upon the rejected articles is not possible.\\[-13.5pt]

\item The edited articles will be sent to the authors for proofread. The comments of the authors
to the edited text of the article should be sent to the Editorial Board as soon as possible.\\[-13.5pt]

\item The manuscript of the article should be presented electronically in the MS WORD (.doc or
.docx) or \LaTeX\ (.tex) formats, and additionally in the .pdf format. All documents
 may be sent
by e-mail or provided on a CD or diskette. A~hard copy submission is not necessary.\\[-13.5pt]

\item The recommended typesetting instructions for manuscript.

Pages parameters: format A4, portrait orientation, document margins (cm): left~--- 2.5, right~---
1.5, above~--- 2.0, below~--- 2.0, footer 1.3.

Text: font~---Times New Roman, font size~--- 14, paragraph indent~--- 0.5, line spacing~--- 1.5,
justified alignment.

The recommended manuscript size: not more than 15~pages of the specified format.
If the specified size exceeded, the editorial board is entitled to require the author
to reduce the manuscript.

Use only standard abbreviations. Avoid  abbreviations in the title and
abstract. The full term for which an abbreviation stands should precede
its first use in the text unless it is a standard unit of measurement.

All pages of the manuscript should be numbered.

The templates for the manuscript typesetting are presented on site: {\sf
http://www.ipiran.ru/journal/template.doc}.\\[-13.5pt]


%\def\leftkol{Requirements for manuscripts submitted to Journal
%``Informatics~and~Applications''}

\item The articles should enclose data both in \textbf{Russian and English}:
\begin{itemize}
\item title;\\[-13.5pt]
\item author's name and surname;\\[-13.5pt]
\item affiliation~--- organization, its address with ZIP code, city, country, and
official e-mail address;\\[-13.5pt]
\item data on authors according to the format: (see site)

{\sf http://www.ipiran.ru/journal/issues/2013\_07\_01/authors.asp}  and

{\sf  http://www.ipiran.ru/journal/issues/2013\_07\_01\_eng/authors.asp};\\[-13.5pt]

\pagebreak

\def\leftfootline{\small{\textbf{\thepage}
\hfill INFORMATIKA I EE PRIMENENIYA~--- INFORMATICS AND APPLICATIONS\ \ \ 2019\
\ \ volume~13\ \ \ issue\ 4}
}%
 \def\rightfootline{\small{INFORMATIKA I EE PRIMENENIYA~--- INFORMATICS AND APPLICATIONS\ \ \ 2019\ \ \ volume~13\ \ \ issue\ 4
\hfill \textbf{\thepage}}}


%\def\leftkol{Requirements for manuscripts submitted to Journal
%``Informatics~and~Applications''}

%\def\rightkol{Requirements for manuscripts submitted to Journal
%``Informatics~and~Applications''}



\item abstract (not less than 100 words) both in Russian and in English. Abstract is a short
summary of the article that can be published separately. The abstract is the
main source of information on the article and it could be included in leading information
systems and data bases. The abstract in English has to be an original text and should
not be an exact translation of the Russian one. Good English is required.
In abstracts, avoid references and formulae;\\[-13.5pt]
\item indexing is performed on the basis of keywords. The use of keywords from the
internationally accepted thematic Thesauri is recommended.

%\def\leftkol{Requirements for manuscripts submitted to Journal
%``Informatics~and~Applications''}

%\def\rightkol{Requirements for manuscripts submitted to Journal
%``Informatics~and~Applications''}

Important! Keywords must not be sentences;
\item Acknowledgments.
\end{itemize}

\item References. Russian references have to be presented both in English translation and Latin
transliteration (refer {\sf http://www.translit.net/ru/bgn/}).

Please take into account the following examples of Russian references appearance:

\noindent
\textbf{Article in journal:}

\Aue{Zhang, Z., and D.~Zhu}. 2008. Experimental research on the localized electrochemical
micromachining.
\textit{Rus. J.~Electrochem.}  44(8):926--930. {\sf doi:10.1134/S1023193508080077}.


\noindent
\textbf{Journal article in electronic format:}

\Aue{Swaminathan, V., E.~Lepkoswka-White, and B.\,P.~Rao}. 1999. Browsers or buyers in
cyberspace? An
investigation of electronic factors influencing electronic exchange. \textit{JCMC}
5(2). Available at: {\sf http://www.ascusc.org/jcmc/vol5/issue2/} (accessed April~28, 2011).




\noindent
\textbf{Article from the continuing publication (collection of works, proceedings):}

\Aue{Astakhov, M.\,V., and T.\,V.~Tagantsev}. 2006. Eksperimental'noe
issledovanie prochnosti soedineniy ``stal'--kompozit'' [Experimental study of
the strength of joints ``steel--composite'']. \textit{Trudy MGTU
``Matematicheskoe modelirovanie slozhnykh tekh\-ni\-che\-skikh sistem''}
[\textit{Bauman MSTU ``Mathematical Modeling of Complex Technical
Systems'' Proceedings}]. 593:125--130.

\def\leftfootline{\small{\textbf{\thepage}
\hfill INFORMATIKA I EE PRIMENENIYA~--- INFORMATICS AND APPLICATIONS\ \ \ 2019\
\ \ volume~13\ \ \ issue\ 4}
}%
 \def\rightfootline{\small{INFORMATIKA I EE PRIMENENIYA~--- INFORMATICS AND APPLICATIONS\ \ \ 2019\ \ \ volume~13\ \ \ issue\ 4
\hfill \textbf{\thepage}}}

\def\leftkol{Requirements for manuscripts submitted to Journal
``Informatics~and~Applications''}

\def\rightkol{Requirements for manuscripts submitted to Journal
``Informatics~and~Applications''}

\noindent
\textbf{Conference proceedings:}

\Aue{Usmanov, T.\,S., A.\,A.~Gusmanov, I.\,Z.~Mullagalin, R.\,Ju.~Muhametshina,
A.\,N.~Chervyakova, and
A.\,V.~Sveshnikov}. 2007. Osobennosti proektirovaniya razrabotki mestorozhdeniy
s primeneniem gidrorazryva
plasta [Features of the design of field development with the use of hydraulic fracturing].
\textit{Trudy 6-go
Mezhdu\-na\-rod\-no\-go Simpoziuma ``Novye resursosberegayushchie tekhnologii
nedropol'zovaniya i povysheniya
neftegazootdachi''} [\textit{6th  Symposium (International) ``New Energy Saving Subsoil
Technologies and
the Increasing of the Oil and Gas Impact'' Proceedings}]. Moscow. 267--272.


\noindent
\textbf{Books and other monographs:}




Lindorf, L.\,S., and L.\,G.~Mamikoniants, eds. 1972.
\textit{Ekspluatatsiya turbogeneratorov s neposredstvennym
okhlazhdeniem} [\textit{Operation of turbine generators with direct cooling}].
Moscow: Energy Publs. 352~p.


%\Aue{Latyshev, V.\,N.} 2009. \textit{Tribologiya rezaniya. Kn.~1: Frikcionnye prosessy
%pri rezanii metallov}
%[\textit{Tribology of cutting. Vol.~1: Frictional processes in metal cutting}]. Ivanovo: Ivanovskii
%State Univ. 108~p.


%\noindent
%\textbf{Unpublished material:}

%\Aue{Latypov, A.\,R., M.\,M.~Khasanov, and V.\,A.~Baikov}.
%2004. Geology and production (NGT GiD). Certificate on official registration of the computer
%program
%No.\,2004611198. (In Russian, unpubl.)

%\noindent
%\textbf{Internet-source:}

%APA Style. 2011. Available at: {\sf http://www.apastyle.org/apa-style-help.aspx} (accessed
%February~5, 2011).

%Pravila citirovaniya istochnikov [Rules for the citing of sources]. Available at: {\sf
%http://www.scribd.com/doc/1034528/} (accessed February~7, 2011).


\noindent
\textbf{Dissertation and Thesis:}

%\Aue{Semenov, V.\,I.}
%2003. Matematicheskoe modelirovanie plazmy v sisteme kompaktnyy tor. [Mathematical
%modeling of the plasma in the compact torus]. D.Sc.\ Diss. Moscow. 272~p.

\Aue{Kozhunova, O.\,S.} 2009. Tekhnologiya razrabotki semanticheskogo
slovarya informatsionnogo monitoringa [Technology of development of
semantic dictionary of information monitoring system]. PhD Thesis. Moscow: IPI RAN. 23~p.


\noindent
\textbf{State standards and patents:}

GOST 8.586.5-2005. 2007. Metodika vypolneniya izmereniy. Izmerenie raskhoda i~kolichestva
zhidkostey i gazov 
s~pomoshch'yu standartnykh suzhayushchikh ustroystv [Method of measurement.
Measurement of flow rate and volume of liquids and gases by means of orifice devices]. M.:
Standardinform
Publs. 10~p.

%\noindent
%\textbf{Patent:}

\Aue{Bolshakov, M.\,V., A.\,V.~Kulakov, A.\,N.~Lavrenov, and M.\,V.~Palkin}.
2006. Sposob orientirovaniya po krenu letatel'nogo
apparata s opti\-che\-skoy golovkoy
samonavedeniya [The way to orient on the roll of aircraft with optical homing head].
Patent RF No.\,2280590.

References in Latin transcription are presented in the original language.

References in the text are numbered according to the order of their
first appearance; the number is
placed in square brackets. All items from the reference list should be
cited.\\[-13.5pt]

\item Manuscripts and additional materials are not returned to Authors by the Editorial Board.\\[-13.5pt]

\item Submissions of files by e-mail must include:\\[-13.5pt]
\begin{itemize}
\item   the journal title and author's name in the ``Subject'' field; \\[-13.5pt]
\item   an article and additional materials have to be attached using the ``attach'' function;\\[-13.5pt]
\item   an electronic version of the article should contain the file with the text and a separate file
with figures.\\[-13.5pt]
\end{itemize}

\item ``Informatics and Applications'' journal is not a profit publication. There are no
charges for the authors as well as there are no royalties.\\[-13.5pt]
\end{enumerate}

\def\leftfootline{\small{\textbf{\thepage}
\hfill INFORMATIKA I EE PRIMENENIYA~--- INFORMATICS AND APPLICATIONS\ \ \ 2019\
\ \ volume~13\ \ \ issue\ 4}
}%
 \def\rightfootline{\small{INFORMATIKA I EE PRIMENENIYA~--- INFORMATICS AND APPLICATIONS\ \ \ 2019\ \ \ volume~13\ \ \ issue\ 4
\hfill \textbf{\thepage}}}

\def\leftkol{Requirements for manuscripts submitted to Journal
``Informatics~and~Applications''}

\def\rightkol{Requirements for manuscripts submitted to Journal
``Informatics~and~Applications''}


%\vspace*{5mm}


\begin{center}
\textbf{Editorial Board address:} \\

%ABOUT AUTHORS



FRC CSC RAS, 44, block~2, Vavilov Str., Moscow 119333, Russia\\[-10pt]

\

Ph.: +7\,(499)\,135\,86\,92,\ \ Fax: +7\,(495)\,930\,45\,05\\[-10pt]

\

 e-mail: {\sf rust@ipiran.ru} (to Prof.\ Rustem Seyful-Mulyukov)\\[-10pt]

\

 {\sf http://www.ipiran.ru/english/journal.asp}
\end{center}
 }
%\thispagestyle{myheadings}

\def\leftkol{Requirements for manuscripts submitted to Journal
``Informatics~and~Applications''}

\def\rightkol{Requirements for manuscripts submitted to Journal
``Informatics~and~Applications''}

\def\leftfootline{\small{\textbf{\thepage}
\hfill INFORMATIKA I EE PRIMENENIYA~--- INFORMATICS AND APPLICATIONS\ \ \ 2019\
\ \ volume~13\ \ \ issue\ 4}
}%
 \def\rightfootline{\small{INFORMATIKA I EE PRIMENENIYA~--- INFORMATICS AND APPLICATIONS\ \ \ 2019\ \ \ volume~13\ \ \ issue\ 4
\hfill \textbf{\thepage}}}

 \label{end\stat}

\newpage

%\vspace*{-60pt} {\small
{\baselineskip=9.1pt
\section*{Правила подготовки рукописей статей для публикации в журнале
<<Информатика и её применения>>}

\thispagestyle{empty}

 Журнал <<Информатика и её применения>> публикует
теоретические, обзорные и дискуссионные статьи, посвященные научным
исследованиям и разработкам в области информатики и ее приложений. Журнал
издается на русском языке. По специальному решению редколлегии отдельные статьи,
в виде исключения, могут печататься на английском языке.
Тематика журнала охватывает следующие направления:
\begin{itemize}
\item теоретические основы информатики; %\\[-13.5pt]
\item математические методы исследования сложных систем и процессов; %\\[-13.5pt]
\item информационные системы и сети; %\\[-13.5pt]
\item информационные технологии; %\\[-13.5pt]
\item архитектура и программное
обеспечение вычислительных комплексов и сетей.
\end{itemize}
\begin{enumerate}
\item В журнале печатаются результаты, ранее не
опубликованные и не предназначенные к одновременной публикации в других
изданиях. Публикация не должна нарушать закон об авторских правах. Направляя
свою рукопись в редакцию, авторы автоматически передают учредителям и
редколлегии неисключительные права на издание данной статьи на русском языке и
на ее распространение в России и за рубежом. При этом за авторами сохраняются
все права как собственников данной рукописи. В связи с этим авторами должно
быть представлено в редакцию письмо в следующей форме:
Соглашение о передаче права на публикацию:

\textit{<<Мы, нижеподписавшиеся, авторы рукописи <<$\qquad\qquad$>>, передаем
учредителям и редколлегии журнала <<Информатика и её применения>>
неисключительное право опубликовать данную рукопись статьи на русском языке как
в печатной, так и в электронной версиях журнала. Мы подтверждаем, что данная
публикация не нарушает авторского права других лиц или организаций. Подписи
авторов: (ф.\,и.\,о., дата, адрес)>>.}

Указанное соглашение может быть представлено 
как в бумажном виде, так и в виде отсканированной копии (с подписями авторов).


Редколлегия вправе запросить у авторов экспертное заключение о возможности
опубликования представленной статьи в открытой печати. %\\[-13.5pt]
\item Статья
подписывается всеми авторами. На отдельном листе представляются данные автора
(или всех авторов): фамилия, полные имя и отчество, телефон, факс, e-mail,
почтовый адрес. Если работа выполнена несколькими авторами, указывается фамилия
одного из них, ответственного за переписку с редакцией. %\\[-13.5pt]
\item Редакция журнала
осуществляет самостоятельную экспертизу присланных статей. Возвращение рукописи
на доработку не означает, что статья уже принята к печати. Доработанный вариант
с ответом на замечания рецензента необходимо прислать в редакцию. %\\[-13.5pt]
\item Решение
редакционной коллегии о принятии статьи к печати или ее отклонении сообщается
авторам. Редколлегия не обязуется направлять рецензию авторам отклоненной
статьи. %\\[-13.5pt]
\item Корректура статей высылается авторам для просмотра. Редакция
просит авторов присылать свои замечания в кратчайшие сроки. %\\[-13.5pt]
\item При
подготовке рукописи в MS Word рекомендуется использовать следующие настройки.
Параметры страницы: формат~--- А4; ориентация~--- книжная; поля (см): внутри~---
2,5, снаружи~--- 1,5, сверху~--- 2, снизу~--- 2, от края до нижнего
колонтитула~--- 1,3. Основной текст: стиль~--- <<Обычный>>: шрифт Times New
Roman, размер 14~пунктов, абзацный отступ~--- 0,5~см, 1,5 интервала,
выравнивание~--- по ширине. Рекомендуемый объем рукописи~--- не свыше
25~страниц указанного формата. Ознакомиться с шаблонами, содержащими примеры
оформления, можно по адресу в Интернете:
\textsf{http://www.ipiran.ru/journal/template.doc}.
\item К рукописи, предоставляемой в 2-х
экземплярах, обязательно прилагается электронная версия статьи (как правило, в
форматах MS WORD (.doc) или \LaTeX\ (.tex), а также~--- дополнительно~--- в
формате .pdf) на дискете, лазерном диске или по электронной почте. Сокращения
слов, кроме стандартных, не применяются. Все страницы рукописи должны быть
пронумерованы. %\\[-13.5pt]
\item Статья должна содержать следующую информацию на русском и
английском языках: название, Ф.И.О. авторов, места работы авторов и их
электронные адреса, подробные сведения об авторах, оформленные в соответствии с форматом, 
определяемым файлами {\sf http://www.ipiran.ru/journal/issues/2011\_05\_01/authors.asp} и 
{\sf http://www.ipiran.ru/journal/issues/2011\_01\_eng/authors.asp},
аннотация (не более 100~слов), ключевые слова. Ссылки на
литературу в тексте статьи нумеруются (в квадратных скобках) и располагаются в
порядке их первого упоминания. В~списке литературы не должно быть позиций, на которые нет ссылки в тексте статьи.
Все фамилии авторов, заглавия статей, названия
книг, конференций и~т.\,п.\ даются на языке оригинала, если этот язык
использует кириллический или латинский алфавит. %\\[-13.5pt]
\item Присланные в редакцию материалы авторам не возвращаются.
\item При отправке файлов по электронной
почте просим придерживаться следующих правил:
\begin{itemize}
\item указывать в поле subject (тема) название журнала и фамилию автора; %\\[-13.5pt]
\item использовать attach (присоединение); %\\[-13.5pt]
\item в случае больших объемов информации возможно
использование общеизвестных архиваторов (ZIP, RAR); %\\[-13.5pt]
\item в состав электронной версии статьи должны входить: файл, содержащий текст статьи, и файл(ы),
содержащий(е) иллюстрации. %\\[-13.5pt]
\end{itemize}
\item Журнал <<Информатика и её применения>> является некоммерческим изданием. 
Плата за публикацию с авторов не взимается, гонорар авторам не выплачивается.
\end{enumerate}
\thispagestyle{empty}
\textbf{Адрес редакции:} Москва 119333,
ул.~Вавилова, д.~44, корп.~2, ИПИ РАН\\
\hphantom{\textbf{Адрес редакции:} }Тел.: +7 (499) 135-86-92\ \
Факс:  +7 (495) 930-45-05\ \  E-mail:   rust@ipiran.ru }
}

%\include{ipi-ind}

%\tableofcontents

\end{document}

%\tableofcontents

%\end{document}

%\tableofcontents


\end{document}

\newcommand{\Ack}{\subsection*{\protect\large\bf Acknowledgments}}