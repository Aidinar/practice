\def\stat{agalarov}

\def\tit{ОБ УНИМОДАЛЬНОСТИ ФУНКЦИИ ДОХОДА СИСТЕМЫ МАССОВОГО ОБСЛУЖИВАНИЯ 
ТИПА $G|M|s$\\ С~УПРАВЛЯЕМОЙ ОЧЕРЕДЬЮ}

\def\titkol{Об унимодальности функции дохода 
системы массового обслуживания типа $G|M|s$ с~управляемой очередью}

\def\aut{Я.\,М.~Агаларов$^1$, В.\,Г.~Ушаков$^2$}

\def\autkol{Я.\,М.~Агаларов, В.\,Г.~Ушаков}

\titel{\tit}{\aut}{\autkol}{\titkol}

\index{Агаларов Я.\,М.}
\index{Ушаков В.\,Г.}
\index{Agalarov Ya.\,M.}
\index{Ushakov V.\,G.}


%{\renewcommand{\thefootnote}{\fnsymbol{footnote}} \footnotetext[1]
%{Работа выполнена при частичной поддержке РФФИ (проект 16-07-00677).}}


\renewcommand{\thefootnote}{\arabic{footnote}}
\footnotetext[1]{Институт проб\-лем информатики Федерального исследовательского центра 
<<Информатика и~управ\-ле\-ние>> Российской академии наук, \mbox{agglar@yandex.ru}}
\footnotetext[2]{Факультет вычислительной
математики и~кибернетики Московского государственного университета
имени М.\,В.~Ломоносова; Институт проб\-лем информатики Федерального
исследовательского центра <<Информатика и~управ\-ле\-ние>> Российской
академии наук, \mbox{vgushakov@mail.ru}}

\vspace*{3pt}

 
\Abst{Рассматривается задача максимизации среднего
дохода в~системе массового обслуживания (СМО) типа $G|M|s$ в~единицу
времени на множестве чистых стационарных пороговых стратегий 
с~одной  точкой переключения режима  ограничения доступа. Функция
дохода зависит от следующих параметров, измеряемых в~стоимостных
единицах: плата, получаемая за обслуживание заявок; затраты на
техническое обслуживание прибора; вычет из дохода за задержку
заявок в~очереди; штраф за необслуженные заявки. Доказано, что
функция дохода является унимодальной на множестве рассматриваемых
пороговых стратегий. Предложен алгоритм расчета оптимального
порогового значения и~соответствующего значения максимального
дохода. Приведены результаты вычислительного эксперимента,
иллюстрирующего работу предложенного алгоритма.}

\KW{многоканальная система массового
обслуживания; пороговая стратегия; максимизация дохода}

\DOI{10.14357/19922264190108}
  
%\vspace*{4pt}


\vskip 10pt plus 9pt minus 6pt

\thispagestyle{headings}

\begin{multicols}{2}

\label{st\stat}

\section{Введение}

\vspace*{-3pt}



Для повышения эффективности работы современных вычислительных
сетей используют алгоритмы управления потоками (ограничения
на\-грузки), из которых наиболее часто применяются различные
модификации порогового управления (пороговых стратегий)~[1].
Исследованием и~разработкой пороговых стратегий управления
потоками занимались практически с~начала зарождения вычислительных
сетей с~целью защиты связных и~вычислительных ресурсов от
перегрузок. Одним из основных методов исследования эф\-фек\-тив\-ности
пороговых стратегий  является математическое моделирование 
с~использованием аппарата теории очередей,
которая изучает СМО различного
типа. Большинство работ, в~которых рассмотрены СМО с~пороговой
стратегией управления потоками, посвящено расчету характеристик
системы (средней длины очереди, среднего времени пребывания,
вероятности отклонения заявки, загруженности приборов и~т.\,п.) при
заданной пороговой стратегии. 

В~ряде работ, посвященных данной
тематике, ставится задача оптимизации пороговой стратегии в~смысле
максимизации дохода системы, пред\-став\-лен\-но\-го в~виде стоимостного
функционала (см., например,~[2--7]). Хотя практический интерес 
к~такой постановке задачи в~смысле повышения эффективности сис\-тем,
как  пред\-став\-ля\-ет\-ся, не ниже, чем к~задачам расчета
характеристик систем при фиксированной пороговой стратегии, вопрос
существования эффективных методов и~алгоритмов поиска оптимальных
пороговых стратегий для СМО остается открытым, за исключением
простых СМО ($M|M|1$,  $M|M|n$) и~простых целевых функций
(допустимая средняя задержка заявок в~системе, допустимая
интенсивность потери). Для более сложных СМО (например, $G|M|1,\
M|G|1$) с~более сложными целевыми функциями результаты
исследований ограничиваются математическими постановками задач 
и~эвристическими алгоритмами их решения. 

Из недавно опубликованных 
в~отечественной литературе работ,  посвященных пороговой стратегии
управления очередью с~одной  точкой переключения, отметим~\cite{3-aga}, где
для системы $G|M|n$ сформулирована задача максимизации дохода
сис\-те\-мы на множестве пороговых стратегий с~одним переключением,
фиксированными платой за своевременное обслуживание и~штрафом за
невыполнение этого условия для допущенной в~сис\-те\-му заявки. 
В~работе предложен эвристический алгоритм поиска оптимальной
стратегии и~выдвинута гипотеза о~том, что для сис\-тем $G|G|n$
существует единственное решение указанной задачи в~виде прос\-той
пороговой стратегии. 

Аналогичная задача рас\-смот\-ре\-на в~работе~\cite{4-aga}
для сис\-те\-мы $M|D|1$, где также приведена математическая постановка
задачи и~получена нижняя оценка для оптимального порогового
значения. 

Постановки задач, наиболее близких к~рас\-смат\-ри\-ва\-емой в~настоящей статье, 
приведены в~работах~\cite{5-aga, 6-aga, 7-aga}. В~\cite{5-aga}
решение задачи максимизации\linebreak средне\-го дохода, получаемого сис\-те\-мой
$M|G|1$ в~единицу времени, ищется на множестве смешанных пороговых
стратегий при входной нагрузке меньше единицы. В~работе доказано,
что если решение задачи существует, то оно принадлежит множеству
чистых стратегий. В~работах~\cite{6-aga, 7-aga} аналогичная задача решена
для СМО типа $M|G|1$ и~$G|M|1$ соответственно. 

В~данной статье
рассматривается задача максимизации среднего дохода СМО типа
$G|M|s$ на множестве чистых пороговых стратегий с~одной точкой
переключения режима доступа заявок в~сис\-те\-му. Ниже приведены
результаты исследования, касающегося вопросов существования решения
задачи и~метода поиска оптимальной стратегии.



\section{Постановка задачи}

Рассматривается СМО типа $G|M|s$ с~накопителем бесконечной емкости
и~$s$~приборами обслуживания, в~которую поступает рекуррентный поток
заявок с~функцией  распределения~$H(t)$. Время обслуживания
заявки распределено по экспоненциальному закону с~параметром
$\mu\hm>0$. Поступившая заявка допускается в~накопитель системы
(занимает любое свободное место в~накопителе), если в~момент ее
поступления число заявок в~системе меньше~$k$, $k\hm\geqslant s$~---
некоторое заданное целое число (ниже тривиальный случай $k\hm<s$ не
рассматривается). Такую процедуру доступа заявок в~систему
называют в~литературе пороговой стратегией управ\-ле\-ния доступом, 
и~в~дальнейшем для краткости назовем ее просто стратегией. Обозначим эту стратегию
соответствующим пороговым значением~$k$. Если заявка допущена 
в~накопитель, она занимает любое свободное место в~накопителе 
и~обслуживается на приборе в~порядке поступления. Заявка покидает
систему только при завершении обслуживания, освободив одновременно
прибор и~накопитель, а~на освободившийся прибор поступает
очередная заявка из накопителя (если таковая есть). Система получает
доход, который формируется из следующих составляющих:
\begin{description}
\item $C_0v$~--- плата, получаемая системой, если поступившая заявка
обслужена  системой (принята в~накопитель), $v\hm\geqslant 0$~--- время занятия заявкой прибора
 обслуживания;\\[-14.5pt]
\item
$C_1\geqslant 0$~--- величина штрафа, который платит сис\-те\-ма, если
поступившая заявка отклонена;\\[-14.5pt]
\item
$C_2\geqslant 0$~--- вычет из дохода системы за единицу времени ожидания
заявки в~системе;\\[-14.5pt]
\item
$C_3\geqslant 0$~--- затраты системы в~единицу времени на техническое
обслуживание системы, $C_3\hm<C_0.$
\end{description}

Всюду ниже под доходом системы будем понимать суммарный доход 
с~учетом всех указанных выше со\-став\-ля\-ющих. Отметим, что процесс
обслуживания заявок в~данной сис\-те\-ме описывается цепью Маркова,
где переходы цепи определяются моментами поступления заявок 
и~состояние сис\-те\-мы есть чис\-ло заявок, находящихся в~сис\-те\-ме 
в~момент поступления (см., например,~\cite{8-aga, 9-aga}).

Введем обозначения:
\begin{description}
\item
$\{\pi_i^k,\ 0\leqslant i\leqslant k\}$~--- стационарное распределение
вероятностей цепи при стратегии $k$ ($\pi_i^k$ -- стационарная
вероятность того, что цепь находится в~состоянии $i$);\\[-14.5pt]
\item
$\tilde{g}^k$~--- предельное среднее значение дохода системы в~единицу времени;\\[-14.5pt]
\item
$g^k$~--- предельное среднее значение суммарного дохода системы,
усредненного по числу поступивших заявок;\\[-14.5pt]
\item
$q_i^k$~--- средний доход, получаемый системой в~состоянии~$i$ при
стратегии~$k$, $i\hm\geqslant 0$;\\[-14.5pt]
\item
$\bar{v}=\int\nolimits_0^{\infty}t\,dH(t)$~--- среднее время между
соседними моментами поступления заявок.
\end{description}

Из определения вложенной цепи Маркова следует:
\begin{equation}
\tilde{g}^k=\fr{1}{\bar{v}}\,g^k,
\label{1} 
\end{equation}
где
$$
g^k=\sum\limits_{i=0}^k\pi_i^kq_i^k\,.
$$


Ставится задача нахождения оптимальной стратегии~$k^0$, такой что
\begin{equation*}
 \max\limits_{k\geqslant s}g^k=g^{k^0}\,.
\end{equation*}

\vspace*{-6pt}

\section{Метод решения}

Стационарные вероятности состояний  $\pi_j^k$, $0\hm\leqslant j\hm\leqslant k,$ 
вычисляются по формулам (см.~\cite{8-aga, 9-aga}):

\noindent
\begin{multline}
\label{3}
\pi_j^k=R_j^k\:\pi_k^k,\ j=0,\ldots,k-1\,,\\ \pi_k^k=
\left(1+\sum\limits_{i=0}^{k-1}R_i^k\right)^{-1}\,,
\end{multline}
где
\begin{equation*}
R_k^k=1;\enskip R_{k-1}^k=\fr{1-r_0}{r_0}\,;
\end{equation*}

\vspace*{-12pt}

\noindent
\begin{multline*}
 R_{j-1}^k=\fr{R_j^k(1-r_1)-\sum\limits_{i=j+1}^{k-1}
R_i^kr_{i+1-j}-r_{k-j}}{r_0},\\
 s\leqslant j\leqslant k-1\,;
\end{multline*}

\noindent
\begin{equation*}
R_{s-2}^k=\fr{R_{s-1}^k(1-r_0)-
\sum\limits_{i=s}^{k-1}R_i^kb_{i,s-1}-b_{k,s-1}}{a_{s-2,s-1}}\,;
\end{equation*}
\begin{equation}
 r_{i}=\int\limits_0^{\infty}\fr{(s\mu t)^i}{i!}\,e^{-\mu ts}\,dH(t)\,;
 \label{q3}
\end{equation}

\vspace*{-12pt}

\noindent
\begin{multline*}
R_{j-1}^k={}\\
{}=\fr{R_{j}^k(1-a_{jj})-\sum\limits_{i=j+1}^{s-1}R_i^ka_{ij}-
\sum\limits_{i=s}^{k-1}
R_i^kb_{ij}-b_{kj}}{a_{j-1,j}},\\
1\leqslant j\leqslant s-2\,;
\end{multline*}
\vspace*{-12pt}

\noindent
\begin{multline*}
a_{ij}=\int\limits_0^{\infty}\left(
\begin{array}{c}i+1\\ j\end{array}
\right)
\left(1-e^{-\mu t}\right)^{i+1-j}e^{-\mu tj}\,dH(t),\\
 j\leqslant i+1\leqslant s-1\,;
\end{multline*}
\vspace*{-12pt}

\noindent
\begin{multline*}
b_{ij}=\int\limits_0^{\infty}\left(
\begin{array}{c}s\\
j\end{array}\right)
e^{-\mu tj}\left(\int\limits_0^t\!
\fr{(s\mu y)^{i-s}}{(i-s)!}\left( e^{-\mu y}-{}\right.\right.\\
\left.\left.{}-e^{-\mu t}\right)^{s-j}s\mu
\,dy\!
\vphantom{\int\limits_0^t}
\right)dH(t),\enskip j<s\leqslant i+1\leqslant k\,.
\end{multline*}

Положим
\begin{equation*}
Q_i^k=\sum\limits_{j=0}^i\pi_j^k\,;\quad
A_{k+1}=\fr{1-Q_{s-1}^{k+1}}{1-Q_{s-2}^k}\,;
\end{equation*}

\noindent
\begin{multline}
F(k)=\fr{1}{1-A_{k+1}}\left(\sum\limits_{i=s}^k\pi_i^{k+1}\left(
\vphantom{\sum\limits_{i=s}^k}
\bar{v}-{}\right.\right.\\
\left.{}-\fr{1}{\mu s}\sum\limits_{m=i+2-s}^{\infty}(m-(i+1-s))r_m
\right)\pi_{k+1}^{k+1}
+{}
\\
{}+\pi_{k+1}^{k+1}\left. \left(\bar{v}-\fr{1}{\mu s}
\sum\limits_{m=k+2-s}^{\infty}(m-(k+1-s))r_m \right)\!\right)\,;\\
G(k)=\left(C_0-C_3\right)\bar{v}-C_2F(k)\,,\enskip k\geqslant s\,.
\label{5}
\end{multline}

 Справедливы следующие утверждения.
 
\smallskip

 \noindent
\textbf{Лемма~1.}\ 
\textit{Среднее значение дохода, получаемого сис\-те\-мой при стратегии~$k$ 
в~состоянии~$i,$ равно}
\begin{equation}
\left.
\begin{array}{rl}
\!q_i^k&=\begin{cases}
\fr{C_2}{\mu s}\left(
\vphantom{\sum\limits_{m=i+3-s}^{\infty}}
\fr{1}{2}\sum\limits_{m=1}^{i+2-s}
m(m+2s-2i-3)r_m-{}\right.\\[3pt]
\displaystyle\left.\!{}-\fr{1}{2}(i+1-s)(i+2-s)\!\!\sum\limits_{m=i+3-s}^{\infty}
\!\!\!\!\!r_m\right)+{}&\\[3pt]
\displaystyle\hspace*{1mm}{} +\left(C_0-C_3\right)\bar{v},& 
\hspace*{-33mm}s-1\leqslant i\leqslant k-1;\\[3pt]
\displaystyle\left(C_0-C_3\right)\bar{v},&\hspace*{-27mm} 0\leqslant i\leqslant s-2;
\end{cases}\\[6pt]
\!q_k^k&=q_{k-1}^k-C_0\bar{v}-C_1.
\end{array}\!\!\!\!\!\!\!\!
\right\}\!\!
\label{6}
\end{equation}


\noindent
\textbf{Лемма~2.}\ 
\textit{Справедливы равенства}:
\begin{multline}
\! q_{i+1}^k=q_i^k-\fr{C_2}{\mu s}
\sum\limits_{m=1}^{i+2-s}mr_m-\fr{C_2(i+2-s)}{\mu s}
\!\!\sum\limits_{m=i+3-s}^{\infty} \!\!\!\!\!r_m\,,\\
s-1\leqslant i\leqslant k-2\,.
\label{7}
\end{multline}


\smallskip

\noindent
\textbf{Лемма~3.}\ 
\textit{
Справедливы соотношения}:
\begin{equation}
\left.
\begin{array}{rl}
\!\! \pi_{j+1}^{k+1}&=A_{k+1}\pi_{j}^{k}\,,\enskip s-1\leqslant j\leqslant k\,;\\[6pt]
\!\!g^k-g^{k+1}&=(1-A_{k+1})\left(g^k-G(k)\right),\enskip k\geqslant s\,.
\end{array}\!
\right\}\!
\label{8}
\end{equation}


\noindent
\textbf{Лемма~4.}\ 
\textit{Функция $G(k)$ не возрастает по $k\geqslant s.$}

\smallskip

\noindent
\textbf{Теорема~1.}\ 
\textit{Справедливы следующие утверждения}:
\begin{enumerate}[(1)]
\item \textit{если $\inf\limits_{k\geqslant s}G(k)\hm<\sup\limits_{k\geqslant s}g^k,$ то при любых
значениях параметров $C_i\geqslant 0$, $i\hm=0,1,3$, и~$C_2\hm>0$ существует
оптимальная стратегия~$k_0$ такая, что $s\hm\leqslant k_0\hm<\infty.$ 
В~противном случае $k_0\hm=\infty$};

\item \textit{если $g^s\geqslant G(s),$ то $k^0\hm=s$};

\item \textit{если $C_2=0$ и~$g^s<G(s),$ то $k^0\hm=\infty$};

\item \textit{условие $g^{k^0-1}\hm<g^{k^0}\hm\leqslant g^{k^0+1}$ является необходимым 
и~достаточным для того, чтобы $s\leqslant k^0<\infty$}.
\end{enumerate}

\smallskip

При выполнении условий п.~4 теоремы можно предложить
следующий алгоритм нахождения оптимальной стратегии.
\begin{enumerate}[1.]
\item Положить $k=s.$

\item Вычислить $g^s$ и~$G(s).$

\item Если $C_2=0$ и~$g^s\hm<G(s),$ то положить $k^0\hm=\infty$ и~перейти к~п.~8.

\item Если $g^s\geqslant G(s),$ то положить $k^0=s$ и~перейти к~п.~8.

\item Вычислить $g^{k+1}.$

\item Если выполняется неравенство $g^k\hm<g^{k+1},$ то перейти к~п.~7, 
иначе положить $k^0\hm=k$ и~пе\-рейти к~п.~8.

\item Увеличить~$k$ на единицу и~перейти к~п.~5.

\item Конец алгоритма.
\end{enumerate}

Доказательства лемм~1--4 и~тео\-ре\-мы~1 пред\-став\-ле\-ны в~приложении.

\section{Пример}

В качестве примера применения указанного выше алгоритма нахождения
оптимальной стратегии рассмотрим СМО $H_2|M|s$ с~функцией
распределения входящего потока
$H(t)\hm=\sum\nolimits_{i=1}^2 f_i\left(1-e^{-\lambda_it}\right)$,
$f_1\hm=0{,}4$, $\lambda_1\hm=0{,}8$, $\lambda_2\hm=2.$ Стоимостные параметры
равны: $C_0\hm=20$, $C_1\hm=5$, $C_2\hm=1{,}5$, $C_3\hm=0{,}01.$

На рисунке приведены зависимости функций~$g^k$ и~$G(k)$ от
порогового значения. 


{ \begin{center}  %fig1
 \vspace*{6pt}
 \mbox{%
 \epsfxsize=79.0mm 
 \epsfbox{aga-1.eps}
 }


\end{center}

\vspace*{-1pt}


\noindent
{\small{Зависимость предельного дохода~$g^k$~(\textit{1}) и~функции~$G(k)$~(\textit{2}) от
порогового значения ($\mu\hm=0{,}5$): пустые значки~--- $s\hm=2$;
залитые значки~--- $s\hm=3$}}
}

\vspace*{6pt}



{\small \section*{\raggedleft Приложение}

\noindent
Д\,о\,к\,а\,з\,а\,т\,е\,л\,ь\,с\,т\,в\,о\ леммы~1. 
Фиксируем состояние $s\hm\leqslant i\hm\leqslant
k\hm-1$, и~пусть время нахождения системы в~состоянии~$i$ равно~$v.$
Найдем выражение для суммарного среднего времени ожидания всех
заявок в~очереди в~состоянии~$i,$ т.\,е.\ в~интервале времени~$(0,v].$

Рассмотрим случайные величины вида
$W_l\hm=\sum\nolimits_{j=1}^l\tau_j$, $l\hm\geqslant 1,$ где~$\tau_j$~---
независимые экспоненциально распределенные случайные величины (СВ) с~параметром
$s\hm>0.$ Пусть $B_m$~--- событие вида $(W_m\hm\leqslant v,W_{m+1}\hm>v)$, $m\hm>0$,
$B_0$~--- событие $(W_1\hm>v).$ Заметим, что~$B_m$, $m\hm\geqslant 0$,~---
не\-сов\-мест\-ные события и~в совокупности составляют полную группу
событий. Известно (см., например,~\cite{9-aga}), что сов\-мест\-ное
распределение величин~$W_l$, $l\hm\leqslant m,$ при условии выполнения
события~$B_m$ совпадает с~распределением порядковых статистик из
выборки объема~$m$ из равномерного на~$(0,v]$ распределения, 
а~маргинальным условным распределением СВ ${W_l}/{v}$
является бе\-та-рас\-пре\-де\-ле\-ние с~плот\-ностью
\begin{multline*}
f(x|B_m)={}\\
{}=\begin{cases}
\fr{m!}{(l-1)!(m-l)!}x^{l-1}(1-x)^{m-l}\,, & 0<x<l\,;\\ 
0 & \hspace*{-21pt}\mbox{в~противном~случае.}
\end{cases}
\end{multline*}
Отсюда следует, что $M(W_l|B_m)\hm=({l}/{m+1})v.$

Обозначим:  $\bar{W}_{\mathrm{обсл}/m}$~--- среднее суммарное время
ожидания заявок, завершивших обслуживание или приступивших 
к~обслуживанию в~состоянии~$i$ при условии~$B_m$;
$\bar{W}_{\mathrm{необсл}/m}$~--- среднее суммарное время ожидания
в очереди заявок, не поступивших на обслуживание в~состоянии~$i$
при условии~$B_m.$ Заметим, что при $i\hm<s$ для всех $m\hm>0$ имеют
место равенства
$\bar{W}_{\mathrm{обсл}/m}\hm=\bar{W}_{\mathrm{необсл}/m}\hm=0.$  При $s\hm\leqslant
i\hm\leqslant k-1$, $m\hm\leqslant i\hm-s\hm+1$ получим:
\begin{equation}
 \bar{W}_{\mathrm{обсл}/m}=\fr{mv}{2}\,;\enskip
\bar{W}_{\mathrm{необсл}/m}=(i+1-s-m)v\,.
\label{9}
\end{equation}
Далее, при $m>i+1\hm-s$, $s\hm\leqslant i\hm\leqslant k-1$,
\begin{equation}
 \bar{W}_{\mathrm{обсл}/m}=\fr{(i+1-s)(i-s+2)v}{2(m+1)}\,.
 \label{10}
\end{equation}
При $i=k$ и~$i\hm=k\hm-1$ значения~$\bar{W}_{\mathrm{обсл}/m}$ 
и~$\bar{W}_{\mathrm{необсл}/m}$ совпадают.

Обозначим $q_i^k(v)$ величину суммарного дохода системы 
в~состоянии~$i$ при условии, что время пребывания в~нем равно~$v.$
Из~\eqref{9} и~\eqref{10} при $s\hm\leqslant i\hm\leqslant k-1$ имеем:
\begin{multline*}
q_i^k(v)=C_0v-C_3v-{}\\
{}-\sum\limits_{m=0}^{i-s+1}\!\!e^{-\mu sv}\fr{(\mu sv)^m}{m!}
\left(C_2\left(\bar{W}_{\mathrm{обсл}/m}+\bar{W}_{\mathrm{необсл}/m}\right)\right)-{}\\
{}-
\sum\limits_{m=i+2-s}^{\infty}
e^{-\mu sv}\fr{(\mu sv)^m}{m!}
C_2\bar{W}_{\mathrm{обсл}/m}=C_0v-C_3v+{}\\
{}+C_2
\sum\limits_{m=0}^{i-s+1}\left(\fr{m}{2}-i-1+s\right)e^{-\mu sv}
\fr{(\mu s)^mv^{m+1}}{m!}-{}
\\
{}-C_2\fr{(i+1-s)(i+2-s)}{2}\sum\limits_{m=i+2-s}^{\infty}e^{-\mu sv}
\fr{(\mu s)^mv^{m+1}}{(m+1)!}.
\end{multline*}
Следовательно,
\begin{multline*}
q_i^k=\int\limits_0^{\infty}q_i^k(v)\,dH(v)=
C_0\bar{v}-C_3\bar{v}+{}\\
{}+\fr{C_2}{2\mu s}
\sum\limits_{m=0}^{i-s+1}m(m+1)\int\limits_0^{\infty}
\fr{(\mu sv)^{m+1}}{(m+1)!}e^{-\mu sv}\,dH(v)-{}
\\
{}-\fr{C_2(i+1-s)}{\mu s}
\sum\limits_{m=0}^{i-s+1}(m+1)\int\limits_0^{\infty}\fr{(\mu sv)^{m+1}}
{(m+1)!}e^{-\mu sv}\,dH(v)-{}\\
{}-\fr{C_2(i+1-s)(i+2-s)}{2\mu s}\times{}\\
{}\times\sum\limits_{m=i+2-s}^{\infty}
\int\limits_0^{\infty}\fr{(\mu sv)^{m+1}}{(m+1)!}e^{-\mu sv}\,dH(v)\,.
\end{multline*}
Отсюда и~из~\eqref{q3} следует~\eqref{6} при $s\hm\leqslant i\hm\leqslant k\hm-1.$
Справедливость~\eqref{6} при $i\hm=k$ очевидна.

\smallskip

\noindent
Д\,о\,к\,а\,з\,а\,т\,е\,л\,ь\,с\,т\,в\,о\ леммы~2.\ 
При $s\hm-1\hm\leqslant i\hm\leqslant k\hm-2$ из~\eqref{10} имеем:
\begin{multline*}
q_{i+1}^k-q_i^k=\fr{C_2}{2\mu s}\sum\limits_{m=1}^{i+2-s}\!m(m-1)r_m+{}\\
{}+
\fr{C_2}{2\mu s}(i+2-s)(i+3-s)r_{i+3-s}-\fr{C_2(i+2-s)}{\mu s}
\sum\limits_{m=1}^{i+2-s}mr_m- {}
\\
{}-\fr{C_2}{\mu s}(i+2-s)(i+3-s)r_{i+3-s}-{}\\
{}-\fr{C_2(i+2-s)(i+3-s)}{2\mu s}
\sum\limits_{m=i+4-s}^{\infty}r_m-{}\\
{}-\fr{C_2}{2\mu s}
\sum\limits_{m=1}^{i+2-s}m(m-1)r_m+{}
\\
{}+\fr{C_2(i+1-s)}{\mu s}\sum\limits_{m=1}^{i+2-s}mr_m+{}\\
{}+
\fr{C_2(i+1-s)(i+2-s)}{2\mu s}\sum\limits_{m=i+4-s}^{\infty}\!\!r_m+{}
\\
{}+\fr{C_2}{2\mu s}(i+1-s)(i+2-s)r_{i+3-s}={}\\
{}=-\fr{C_2}{\mu s}
\sum\limits_{m=1}^{i+2-s}mr_m-\fr{C_2(i+2-s)}{\mu s}
\sum\limits_{m=i+3-s}^{\infty}r_m\,.
\end{multline*}

\noindent
Д\,о\,к\,а\,з\,а\,т\,е\,л\,ь\,с\,т\,в\,о\ леммы~3.

Первое соотношение в~лемме~3 следует из~\eqref{3}. Теперь из~\eqref{1} имеем:
\begin{multline*}
g^k-g^{k+1}=\sum\limits_{i=0}^{k}\pi_i^{k}q_i^{k}-
\sum\limits_{i=0}^{k+1}\pi_i^{k+1}q_i^{k+1}={}\\
{}=
\left(C_0-C_3\right)\bar{v}\sum\limits_{i=0}^{s-2}\pi_i^k+
\sum\limits_{i=s-1}^{k}\pi_i^kq_i^k
-\sum\limits_{i=s}^{k+1}\pi_i^{k+1}q_i^{k+1}-{}
\\
{}-\left(C_0-C_3\right)\bar{v}\sum\limits_{i=0}^{s-1}
\pi_i^{k+1}=\left(C_0-C_3\right)\bar{v}\left(Q_{s-2}^k-Q_{s-1}^{k+1}\right)+{}\\
{}+
\sum\limits_{i=s-1}^{k}\pi_i^kq_i^k-A_{k+1}
\sum\limits_{i=s}^{k+1}\pi_{i-1}^kq_i^{k+1}\,.
\end{multline*}

Обозначим $\Delta_{i+1}^{k+1}\hm=q_{i+1}^{k+1}\hm-q_{i}^{k+1}$, $s\hm-1\hm\leqslant
i\hm\leqslant k.$ Заметим, что $q_i^{k+1}\hm=q_i^k$, $0\hm\leqslant i\hm\leqslant k\hm-1,$
$q_{k+1}^{k+1}\hm=q_k^{k+1}\hm-C_1\hm-C_0\hm=q_k^{k}\hm+\Delta_k^{k+1}.$ Имеем:
\begin{multline*}
g^k-g^{k+1}=\left(C_0-C_3\right)\bar{v}\left(
Q_{s-2}^k-Q_{s-1}^{k+1}\right)+
\sum\limits_{i=s-1}^k\pi_i^kq_i^k-{}\\
{}- A_{k+1}
\left(\sum\limits_{i=s-1}^{k-1}\pi_i^k\left(q_i^k+\Delta_{i+1}^{k+1}\right)+
\pi_k^k\left(q_k^k+\Delta_k^{k+1}\right)\right)={}\\
{}=
\fr{Q_{s-1}^{k+1}-Q_{s-2}^k}{1-Q_{s-2}^k}
\Bigg(g^k-\left(C_0-C_3\right)\bar{v}-{}\\
{}-
\fr{1-Q_{s-1}^{k+1}}{Q_{s-1}^{k+1}-Q_{s-2}^k}
\left(\sum\limits_{i=s-1}^{k-1}\pi_i^k\Delta_{i+1}^{k+1}+
\pi_k^k\Delta_k^{k+1}\right)\Bigg)\,.
\end{multline*}
Использовав~\eqref{7}, для $\Delta_{i+1}^{k+1}$ получим
\begin{align*}
\Delta_{i+1}^{k+1}&=-\fr{C_2}{\mu s}\sum\limits_{m=1}^{i+2-s}mr_m-\fr{C_2(i+2-s)}
{\mu s}\sum\limits_{m=i+3-s}^{\infty}\!\!r_m={}\\
&{}=\left(\fr{1}{\mu s}
\sum\limits_{m=i+3-s}^{\infty}(m-i-2+s)r_m-\bar{v}\right) C_2\,,\\
&\hspace*{45mm} s-1\leqslant i\leqslant k-1\,;
\\
 \Delta_k^k&=-C_1-C_0\,;
 \\
\Delta_k^{k+1}&=-C_2\left(\bar{v}-\fr{1}{\mu s}
\sum\limits_{m=i+2-s}^{\infty}(m-k-1+s)r_m\right)\,.
\end{align*}
Отсюда
\begin{multline*}
g^k-g^{k+1}=\fr{Q_{s-1}^{k+1}-Q_{s-2}^k}{1-Q_{s-2}^k}\left(
\vphantom{\sum\limits_{m=0}^{k+1-s}}
g^k-\left(C_0-C_3\right)
\bar{v}+\right.\\
{}+C_2\fr{1-Q_{s-2}^{k+1}}{Q_{s-1}^{k+1}-Q_{s-2}^k}\left(
\vphantom{\sum\limits_{m=0}^{k+1-s}}
\pi_{k+1}^{k+1}\times{}\right.
\\
\left.{}\times\left(\bar{v}- \fr{1}{\mu
s}\sum\limits_{m=k+2-s}^{\infty}(m-k-1+s)r_m\right)+{}\right.\\
\left.\left.{}+
\sum\limits_{i=s}^k
\pi_i^{k+1}\left(\bar{v}-
\fr{1}{\mu s}\sum\limits_{m=k+2-s}^{\infty}\!\!\!(m-i-1+s)r_m\right)
\right)\right).
\end{multline*}
Подставляя в~последнее равенство~\eqref{5}, получаем~\eqref{8}.

\smallskip

\noindent
Д\,о\,к\,а\,з\,а\,т\,е\,л\,ь\,с\,т\,в\,о\ леммы~4.

Положим $f_k=\left(1-A_{k+1}\right)F(k).$ Тогда
\begin{multline*}
f_k=\sum\limits_{i=s}^{k+1}\pi_i^{k+1}\bar{v}-\fr{1}{\mu s}
\sum\limits_{i=s}^{k}\pi_i^{k+1}\!\sum\limits_{m=i+2-s}^{\infty}\!(m-i-1+s)r_m-{}\\
{}-
\fr{1}{\mu s}\pi_{k+1}^{k+1}\sum\limits_{m=i+2-s}^{\infty}(m-k-1+s)r_m={}
\\
{}=\fr{1}{\mu s}\sum\limits_{i=s}^{k}\pi_i^{k+1}\sum\limits_{m=1}^{i+1-s}mr_m+
\fr{1}{\mu s}\pi_{k+1}^{k+1}\sum\limits_{m=1}^{k+1-s}mr_m+{}\\
{}+
\sum\limits_{i=s}^k\pi_i^{k+1} \left(
1-\sum\limits_{m=0}^{i+1-s}r_m\right)\fr{i+1-s}{\mu s}+{}
\\
{}+\pi_{k+1}^{k+1}
\left(1-\sum\limits_{m=0}^{k+1-s}\!r_m\right)\fr{k+1-s}{\mu s}\,.
\end{multline*}
Заметим, что правая часть последнего равенства выражает среднее значение 
длительности времени,
в течение которого в~произвольно взятом состоянии при стратегии $k\hm+1$ все 
приборы заняты.

Отсюда следует, что~$f_k$ возрастает по~$k.$ Докажем, 
что~$A_{k+1}$ возрастает по $k\hm\geqslant s.$ Далее из-за громоздкости
доказательства, полученного для рекуррентного входящего потока,
приводим аналогичное, но гораздо более короткое доказательство для
пуассоновского входящего потока. Пусть~$\lambda$~--- его
интенсивность. Известно, что
\begin{equation*}
\pi_j^k=
\begin{cases}
\pi_0^k \fr{\alpha^j}{j!}, &\enskip 1\leqslant j\leqslant s;\\
\pi_0^k\fr{\alpha^s}{s!}\left(\fr{\alpha}{s}\right)^{s-j},&\enskip
s\leqslant j\leqslant k;
\end{cases}
\end{equation*}
\begin{equation*}
\pi_0^k=\left(1+\sum\limits_{j=1}^s \fr{\alpha^j}{j!}+
\fr{\alpha^s}{s!} \sum\limits_{l=1}^{k-s}\left(\fr{\alpha}{s}\right)^{l}\right)^{-1},\enskip
\alpha=\fr{\lambda}{\mu}\,.
\end{equation*}
Отсюда
\begin{multline}
\fr{1-Q_{s-1}^{k+2}}{1-Q_{s-2}^{k+1}}-
\fr{1-Q_{s-1}^{k+1}}{1-Q_{s-2}^{k}}={}\\
{}=
\fr{\pi_0^{k+2}({\alpha^s}/{s!})
\sum\limits_{i=0}^{k+2-s}\left({\alpha}/{s}\right)^i}
{\pi_0^{k+1}\left({\alpha^{s-1}}/({(s-1)!})+({\alpha^s}/{s!})
\sum\limits_{i=0}^{k+1-s}
\left({\alpha}/{s}\right)^i\right)}-{}\\
{}-
\fr{\pi_0^{k+1}({\alpha^s}/{s!})
\sum\limits_{i=0}^{k+1-s}\left({\alpha}/{s}\right)^i}
{\pi_0^{k}\left({\alpha^{s-1}}/({(s-1)!})+({\alpha^s}/{s!})\sum\limits_{i=0}^{k-s}
\left({\alpha}/{s}\right)^i\right)}\,.
\label{p5}
\end{multline}
Далее
\begin{multline}
\fr{\alpha^s}{s!}\sum\limits_{i=0}^{k+2-s}\left(\fr{\alpha}{s}\right)^i
\left(\fr{\alpha^{s-1}}{(s-1)!}+\fr{\alpha^s}{s!}\sum\limits_{i=0}^{k-s}
\left(\fr{\alpha}{s}\right)^i\right)={}\\
{}=\fr{\alpha^s}{s!}\sum\limits_{i=0}^{k+1-s}\left(\fr{\alpha}{s}\right)^i
\left(\fr{\alpha^{s-1}}{(s-1)!}+\dfrac{\alpha^s}{s!}\sum\limits_{i=0}^{k+1-s}
\left(\fr{\alpha}{s}\right)^i\right).
\label{p6}
\end{multline}
Действительно, разность левой и~правой частей~\eqref{p6} равна
\begin{multline*}
\fr{\alpha^s}{s!}\,\fr{\alpha^{s-1}}{(s-1)!}\left(\sum\limits_{i=0}^{k+2-s}
\left(\fr{\alpha}{s}\right)^i\left(1+\sum\limits_{i=0}^{k-s}
\left(\fr{\alpha}{s}\right)^{i+1}\right)-{}\right.\\
\left.{}-
\sum\limits_{i=0}^{k+1-s}
\left(\fr{\alpha}{s}\right)^i\left(1+\sum\limits_{i=0}^{k+1-s}
\left(\fr{\alpha}{s}\right)^{i+1}\right)\right)={}
\\
{}=\fr{\alpha^s}{s!}\,\fr{\alpha^{s-1}}{(s-1)!}\left(\left(
\fr{\alpha}{s}\right)^{k+2-s}+{}\right.\\
{}+\fr{\alpha}{s}
\left(\left(\fr{\alpha}{s}\right)^{k+1-s}\left(\fr{\alpha}{s}-1\right)
\sum\limits_{i=0}^{k+1-s}
\left(\fr{\alpha}{s}\right)^i-{}\right.{}\\
\left.\left.{}-\left(\fr{\alpha}{s}\right)^{2k+3-2s}\right)\right)=0
\end{multline*}
Следовательно, знак правой части в~\eqref{p5} совпадает со знаком
$\pi_0^{k+2}\pi_0^{k}\hm-\left(\pi_0^{k+1}\right)^2.$ Знак последнего
выражения противоположен знаку разности:
\begin{multline*}
\left(1+\sum\limits_{j=1}^s \fr{\alpha^j}{j!}+
\fr{\alpha^s}{s!} \sum\limits_{l=1}^{k+2-s}\left(\fr{\alpha}{s}\right)^{l}\right)\times{}\\
{}\times
\left(1+\sum\limits_{j=1}^s \fr{\alpha^j}{j!}+
\fr{\alpha^s}{s!} \sum\limits_{l=1}^{k-s}\left(\fr{\alpha}{s}\right)^{l}\right)-{}
\\
{}-\left(1+\sum\limits_{j=1}^s\:\fr{\alpha^j}{j!}+
\fr{\alpha^s}{s!}\sum\limits_{l=1}^{k+1-s}\left(\fr{\alpha}{s}\right)^{l}\right)^2.
\end{multline*}
Эта разность равна:
\begin{multline*}
\fr{\alpha^s}{s!} \left(\fr{\alpha}{s}\right)^{k+1-s}\left(\fr{\alpha}{s}-1\right)
\left(
\vphantom{\sum\limits_{l=1}^{k+1-s}}
1+\sum\limits_{j=1}^s \fr{\alpha^j}{j!}+{}\right.\\
\left.{}+
\fr{\alpha^s}{s!} \sum\limits_{l=1}^{k+1-s}\left(\fr{\alpha}{s}\right)^{l}\right)-
\left(\fr{\alpha^s}{s!}\right)^2\left(\fr{\alpha}{s}\right)^{2k+3-2s}={}
\\
{}=\fr{\alpha^s}{s!}\left(\fr{\alpha}{s}\right)^{k+1-s}
\left(\sum\limits_{j=1}^s\fr{\alpha^j}{s(j-1)!}-
\sum\limits_{j=1}^s\fr{\alpha^j}{j)!}\right).
\end{multline*}
Но
$$
\sum\limits_{j=1}^s\fr{\alpha^j}{s(j-1)!}- \sum\limits_{j=1}^s\fr{\alpha^j}{j!}<0.
$$
Итак, лемма~4 полностью доказана.

\textit{Доказательство теоремы}~1 аналогично доказательству
теоремы~1 в~\cite{7-aga}.

}


{\small\frenchspacing
 {%\baselineskip=10.8pt
 \addcontentsline{toc}{section}{References}
 \begin{thebibliography}{9}
\bibitem{1-aga}
\Au{Welzl M.} Network congestion control: Managing internet traffic.~--- 
New York, NY, USA: Wiley, 2005. 282~p.
\bibitem{2-aga}
\Au{Жерновый Ю.\,В.} Решение задач оптимального синтеза для некоторых 
марковских моделей обслуживания~// Информационные процессы, 2010. 
Т.~10. №\,3. C.~257--274.
\bibitem{3-aga}
\Au{Коновалов М.\,Г.} Об одной задаче оптимального управ\-ле\-ния нагрузкой на сервер~// 
Информатика и~её применения, 2013. Т.~7. Вып.~4. С.~34--43.
\bibitem{4-aga}
\Au{Агаларов Я.\,М.} Пороговая стратегия ограничения доступа к~ресурсам 
в~системе массового обслуживания $M|D|1$ с~функцией
штрафов за несвоевременное обслуживание заявок~// Информатика и~её применения, 2015. 
Т.~9. Вып.~3.  С.~55--64.
\bibitem{5-aga}
\Au{Гришунина Ю.\,Б.} Оптимальное 
управление очередью в~системе 
$M|G|1|\infty$ 
с~возможностью ограничения приема 
заявок~// Автоматика и~телемеханика, 2015. №\,3. 
С.~79--93.
\bibitem{6-aga}
\Au{Агаларов Я.\,М.} Максимизация среднего стационарного дохода системы
массового обслуживания типа
 $M|G|1$~// Информатика и~её применения, 2017. Т.~11. Вып.~2. С.~25--32.
\bibitem{7-aga}
\Au{Агаларов Я.\,М., Шоргин В.\,С.} Об одной задаче максимизации дохода 
сис\-те\-мы
массового обслуживания типа $G|M|1$ с~пороговым управ\-ле\-нием очередью~// Информатика и~её 
применения, 2017. Т.~11. Вып.~4. С.~55--64.

\bibitem{9-aga}
\Au{Karlin S.} A~first course in stochastic processes.~--- 
New York\,--\,London: Academic Press, 1968. 502~p.

\bibitem{8-aga}
\Au{Бочаров П.\,П., Печинкин~А.\,В.} Теория массового обслуживания.~--- М.: РУДН,
1995. 529~с.


 \end{thebibliography}

 }
 }

\end{multicols}

\vspace*{-3pt}

\hfill{\small\textit{Поступила в~редакцию 05.11.18}}

%\vspace*{8pt}

%\pagebreak

\newpage

\vspace*{-29pt}

%\hrule

%\vspace*{2pt}

%\hrule

%\vspace*{-2pt}

\def\tit{ON THE UNIMODALITY OF~THE~INCOME FUNCTION 
OF~A~TYPE $G|M|s$ QUEUEING SYSTEM WITH~CONTROLLED QUEUE}


\def\titkol{On the unimodality of~the~income function 
of~a~type $G|M|s$ queueing system with~controlled queue}

\def\aut{Ya.\,M.~Agalarov$^{1}$ and~V.\,G.~Ushakov$^{1,2}$}

\def\autkol{Ya.\,M.~Agalarov and~V.\,G.~Ushakov}

\titel{\tit}{\aut}{\autkol}{\titkol}

\vspace*{-11pt}


\noindent
$^1$Institute of Informatics Problems, Federal Research Center 
``Computer Science and Control'' of the 
Russian\linebreak
$\hphantom{^1}$Academy of Sciences, 44-2~Vavilov Str., Moscow 119333, Russian Federation

\noindent
$^2$Department of Mathematical Statistics, Faculty of 
Computational Mathematics and Cybernetics, M.\,V.~Lo\-mo-\linebreak
$\hphantom{^1}$no\-sov Moscow State University,
1-52~Leninskye Gory, GSP-1, Moscow 119991,  Russian Federation

\def\leftfootline{\small{\textbf{\thepage}
\hfill INFORMATIKA I EE PRIMENENIYA~--- INFORMATICS AND
APPLICATIONS\ \ \ 2019\ \ \ volume~13\ \ \ issue\ 1}
}%
 \def\rightfootline{\small{INFORMATIKA I EE PRIMENENIYA~---
INFORMATICS AND APPLICATIONS\ \ \ 2019\ \ \ volume~13\ \ \ issue\ 1
\hfill \textbf{\thepage}}}

\vspace*{6pt}


\Abste{The problem of maximizing the average income in a~queuing system of type  
$G|M|s$ on a~set of pure stationary threshold strategies with single point 
switching access restriction mode is considered. The income function depends 
on the following parameters, measured in value units: the fee received 
for servicing requests, the cost of maintenance of the device,
 the deduction of income for the delay applications in the queue, 
 the penalty for unserved applications. It is proved that the income function 
 is unimodal on the set of considered threshold strategies. An algorithm for 
 calculating the optimal threshold value and the corresponding maximum value 
 income is proposed. The results of the computational 
experiment that illustrate the work of the proposed algorithm are given.}

\KWE{multichannel queueing system; threshold management; maximizing income}


\DOI{10.14357/19922264190108}

%\vspace*{-14pt}

%\Ack
%\noindent
%This work was partially supported by the Russian Science Foundation (grant  
%16-07-00677).



%\vspace*{6pt}

  \begin{multicols}{2}

\renewcommand{\bibname}{\protect\rmfamily References}
%\renewcommand{\bibname}{\large\protect\rm References}

{\small\frenchspacing
 {%\baselineskip=10.8pt
 \addcontentsline{toc}{section}{References}
 \begin{thebibliography}{9}
\bibitem{1-aga-1}
\Aue{Welzl, M.} 2005. 
\textit{Network congestion control}. New York, NY: Wiley. 282~p.
\bibitem{2-aga-1}
\Aue{Zhernovyj, Ju.\,V.} 2010. Reshenie zadach optimal'nogo 
sinteza dlya nekotorykh 
markovskikh modeley obsluzhivaniya [Solution of 
optimum synthesis problem for some Markov models of service]. 
\textit{Informatsionnye processy} [Information Processes] 
10(3):257--274.
\bibitem{3-aga-1}
\Aue{Konovalov, M.\,G.} 2013. 
Ob odnoy zadache optimal'nogo 
upravleniya nagruzkoy na server [About one task of overload 
control]. \textit{Informatika i~ee Primeneniya~--- Inform. Appl.} 7(4):34--43.
\bibitem{4-aga-1}
\Aue{Agalarov, Ya.\,M.} 2015. Porogovaya strategiya 
ogranicheniya dostupa k~resursam v~sisteme massovogo obsluzhivaniya 
$M/D/1$ s~funktsiey 
shtrafov za nesvoevremennoe obsluzhivanie zayavok 
[The threshold strategy 
for restricting access in the $M/D/1$ queueing system with penalty 
function for late service]. \textit{Informatika i~ee Primeneniya~--- Inform. Appl.}
9(3):55--64. 
\bibitem{5-aga-1}
\Aue{Grishunina, Y.\,B.} 2015. 
Optimal control of queue in the $M|G|1|\infty$ 
system with possibility of customer admission restriction. 
\textit{Automat. Rem. Contr.} 76(3):433--445.
\bibitem{6-aga-1}
\Aue{Agalarov, Ya.\,M.} 2017. 
Maksimizatsiya srednego sta\-tsi\-o\-nar\-no\-go dokhoda sistemy massovogo obsluzhivaniya tipa 
$M/G/1$
[Maximization of average stationary profit in $M/G/1$ queuing system]. 
\textit{Informatika i~ee Primeneniya~--- Inform. Appl.} 11(2):25--32.
\bibitem{7-aga-1}
\Aue{Agalarov, Ya.\,M., and V.\,S.~Shorgin.} 2017. 
Ob odnoy zadache maksimizatsii dokhoda sistemy massovogo obsluzhivaniya
tipa $G/M/1$ s~porogovym upravleniem oche\-red'yu
[About the problem of profit maximization in $G/M/1$ queuing systems
with threshold control of the queue]. \textit{Informatika i~ee Primeneniya~--- 
Inform. Appl.} 11(4):55--64.
\bibitem{8-aga-1}
\Aue{Karlin, S.} 1968. \textit{A~first course in stochastic processes}. 
New York\,--\,London: Academic Press. 502~p.
\bibitem{9-aga-1}
\Aue{Bocharov, P.\,P., and A.\,V.~Pechinkin.} 1995. 
\textit{Teoriya massovogo obsluzhivaniya} [Queueing theory]. Moscow: RUDN. 529~p.
\end{thebibliography}

 }
 }

\end{multicols}

\vspace*{-6pt}

\hfill{\small\textit{Received November 5, 2018}}

%\pagebreak

%\vspace*{-18pt}

\Contr

\noindent
\textbf{Agalarov Yaver M.} (b.\ 1952)~--- Candidate of Science (PhD) in technology,
associate professor, leading scientist, Institute of Informatics Problems, Federal Research 
Center ``Computer Science and Control'' of the Russian Academy of Sciences, 
44-2~Vavilov Str., Moscow 119333, Russian Federation; \mbox{agglar@yandex.ru}

\vspace*{3pt}

\noindent
\textbf{Ushakov Vladimir G.} (b.\ 1952)~--- Doctor of Science in physics
and mathematics, professor, Department of Mathematical Statistics, Faculty of 
Computational Mathematics and Cybernetics, M.\,V.~Lomonosov Moscow State University,
1-52~Leninskye Gory, GSP-1, Moscow 119991,  Russian Federation;
senior scientist, Institute of Informatics Problems, Federal Research 
Center ``Computer Science and Control'' of the Russian Academy of Sciences, 
44-2~Vavilov Str., Moscow 119333, Russian Federation; \mbox{vgushakov@mail.ru}

\label{end\stat}

\renewcommand{\bibname}{\protect\rm Литература}       