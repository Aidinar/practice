\def\stat{dulin}

\def\tit{СИНТЕЗ ГЕОДАННЫХ В~ПРОСТРАНСТВЕННЫХ ИНФРАСТРУКТУРАХ НА~ОСНОВЕ 
СВЯЗАННЫХ ДАННЫХ$^*$}

\def\titkol{Синтез геоданных в~пространственных инфраструктурах на~основе 
связанных данных}

\def\aut{С.\,К.~Дулин$^1$, Н.\,Г.~Дулина$^2$, О.\,С.~Кожунова$^3$}

\def\autkol{С.\,К.~Дулин, Н.\,Г.~Дулина, О.\,С.~Кожунова}

\titel{\tit}{\aut}{\autkol}{\titkol}

\index{Дулин С.\,К.}
\index{Дулина Н.\,Г.}
\index{Кожунова О.\,С.}
\index{Dulin S.\,K.}
\index{Dulina N.\,G.}
\index{Kozhunova O.\,S.}


{\renewcommand{\thefootnote}{\fnsymbol{footnote}} \footnotetext[1]
{Работа выполнена при частичной поддержке РФФИ (проект 17-20-02153~офи\_м\_РЖД).}}


\renewcommand{\thefootnote}{\arabic{footnote}}
\footnotetext[1]{Научно-исследовательский и~проектно-конструкторский институт информатизации, автоматизации и~связи на 
железнодорожном транспорте (ОАО НИИАС); Институт проб\-лем информатики Федерального 
исследовательского центра <<Информатика и~управ\-ле\-ние>> Российской академии наук, 
\mbox{skdulin@mail.ru}}
\footnotetext[2]{Вычислительный центр им.\ А.\,А.~Дородницына Федерального исследовательского центра <<Информатика 
и~управление>> Российской академии наук, \mbox{ngdulina@mail.ru}}
\footnotetext[3]{Институт проблем информатики Федерального исследовательского центра <<Информатика 
и~управ\-ле\-ние>> Российской академии наук, \mbox{kozhunovka@mail.ru}}

\vspace*{8pt}

 
\Abst{Синтез пространственных данных из различных источников, доступных 
в~Web,~--- главная задача для современных приложений, использующих 
информационный поиск в~сети и~нацеленных на принятие решений на основе 
геоданных. Эта работа посвящена синтезу пространственных данных с~акцентом на 
его приложения в~пространственных инфраструктурах данных (Spatial Data 
Infrastructures~--- SDIs). Возможности интеграции SDIs и~семантического контекста 
обсуждаются при условии согласованного описания и~использования отношений 
характеристик объектов. Предложена классификация и~декомпозиция процессов 
синтеза в~сер\-вис-ори\-ен\-ти\-ро\-ван\-ной структуре для обслуживания широкого 
круга запросов. }

\KW{синтез данных; пространственная инфраструктура данных; связанные данные; 
семантическая сеть}

\DOI{10.14357/19922264190112}
  
%\vspace*{4pt}


\vskip 10pt plus 9pt minus 6pt

\thispagestyle{headings}

\begin{multicols}{2}

\label{st\stat}

\section{Введение}

     
     Быстрое развитие Web от варианта, ориентированного на данные, до 
структур для обслуживания широкого круга запросов наряду с~широким 
распространением мобильных устройств с~определением местоположения 
сильно повлияло на понимание, доступность и~использование геоданных. 
В~результате этого объем геоданных, доступных через сеть, непрерывно 
увеличивается. Для успешного анализа наборов геоданных необходимы методы 
установления связей и~объединения геоданных, полученных из разнообразных 
источников. Становится очевидным, что как только веб-сер\-ви\-сы обеспечат 
синтез пространственной информации из произвольного числа источников 
геоданных, возникнет намного больший потенциал обработки, чем дают 
сегодняшние пространственные инфраструктуры данных (SDIs), которые действуют только как доступные через сеть 
платформы доставки пространственных данных. 

Методики синтеза 
пространственных данных играют важную роль в~создании интегрированного 
представления распределенных источников\linebreak пространственных данных в~сети. 
Поскольку гибкость и~функциональная совместимость~--- клю\-чевые факторы 
в~такой интеграции геоданных,\linebreak использование стандартов~--- неоспоримое 
требование. Поэтому в~дополнение к~геопространственным стандартам, 
установленным открытым геопространственным консорциумом (Open 
Geospacial Consortium~--- OGC), семантические стандарты сети, изданные 
консорциумом Web (World Wide Web Consortium~--- W3C) применительно 
к~связанным геоданным, стали хорошим дополнением для формализации 
и~управления отношениями ха\-рак\-те\-ристик объектов как части процесса их 
синтеза.\linebreak Хотя архитектуры SDIs и~Semantic Web сильно отличаются по линии 
применяемых технологических процессов и~стандартов, они могут дополнять 
друг друга.
     
     В этой работе исследуются подходы, требования и~ограничивающие 
факторы для пространственного синтеза данных на основе веб-сер\-ви\-сов 
с~акцентом на взаимодействие SDI и~стандартов Semantic Web. 

Основными 
целями исследования являются сер\-вис-ори\-ен\-ти\-ро\-ван\-ная декомпозиция 
широкого круга запросов процессов синтеза и~стратегии управ\-ле\-ния 
отношениями характеристик объектов, использующие связанные данные 
в~комбинации с~SDIs.

     \begin{figure*}[b] %fig1
     \vspace*{1pt}
 \begin{center}  
  \mbox{%
 \epsfxsize=144.388mm 
 \epsfbox{dul-1.eps}
 }
\end{center}
\vspace*{-11pt}
     \Caption{Формирование карты на основе слоев}
      \end{figure*}
     
     Классификация и~декомпозиция для обслуживания широкого круга 
запросов пространственных процессов синтеза данных, облегчающих 
выполнение взаимодействующих гибких технологических процессов синтеза, 
рассмотрены в~разд.~2. Определение интеграции SDI и~разработок Semantic 
Web со специфическим акцентом на формализацию и~управление отношениями 
пространственных данных предложено в~разд.~3.
В~разд.~4 обсуждаются классификация и~структура операций низкого уровня процессов
синтеза пространственных данных. Взаимодействие между SDI (интерфейс OWS)
и~компонентами Semantic Web (интерфейс RDF) рас\-смот\-ре\-но в~разд.~5.

\section{Синтез геоданных~--- классификация и~декомпозиция}

     Термин <<синтез данных>> (data fusion) широко используется в~области 
электронной обработки данных и~имеет множество различных определений 
и~классификаций. В~2010~г.\ при исследовании стандартов информационный 
синтез данных был определен как <<акт или процесс объединения или 
соединения данных или информации, касающихся одного или более объектов, 
которые рассматриваются в~явной или неявной структуре знаний, чтобы 
улучшить возможность (или выявить новую возможность) для обнаружения, 
идентификации или характеристик рассматриваемых объектов>>~\cite{1-dul}.
     
     В этой работе синтез пространственных данных рассматривается 
применительно к~множеству источников для извлечения значимой информации, 
касающейся определенного прикладного контекста. Поэтому понимание 
синтеза геоданных, предложенного здесь, включает то, что другие описывают 
как синтез~\cite{2-dul}, интеграцию геоданных~\cite{3-dul} или конкатенацию 
геоданных~\cite{4-dul}.
     
     Многие процессы синтеза могут быть реализованы в~закрытой 
архитектуре при наличии единственного провайдера программного обеспечения и~оборудования. Однако без использования открытых стандартов множество 
геоданных и~сервисов трудно автоматизировать и~масштабировать. 
Стандартизованные данные, приложения и~сервисы обеспечивают 
автоматизированную и~интероперабельную среду синтеза геоданных, 
поддерживая безопасное совместное использование геоданных и~прозрачное 
многократное использование сервисов обработки больших объемов геоданых 
и~непредсказуемых аналитических запросов.
     
     Некоторые элементы предпочтительной открытой структуры 
геоинформационного синтеза на основе геостандартов уже распространяются 
среди пользователей. Например, обслуживание карты в~сети (Web Map 
Service~\cite{5-dul}~--- WMS) обеспечивает синтез карт. Сервис
WMS представляет 
карты как иллюстрированные уровни изображений, полученных из различных 
источников, чтобы географическим оверлеем создавать агрегированные карты, 
удовлетворяющие потребностям пользователей.
     
     Простым примером геоинформационного синтеза можно считать широко 
используемое агрегирование слоев карты (иногда называемых темами) в~единое 
изображение (рис.~1).
     

      
     Очень важным шагом в~исследовании синтеза данных стало введение 
OGC термина \textit{feature}, который не имеет однозначного перевода на 
русский язык, поскольку трактуется в~английском очень широко: особенность, 
характеристика, свойство, признак, функция, феномен, элемент. 
В~терминологической базе OGC ({\sf 
http://www.opengeospatial.org/\linebreak ogc/glossary/f}) этот термин определяется так: 
<<Циф-\linebreak\vspace*{-12pt}

{ \begin{center}  %fig2
 \vspace*{-3pt}
   \mbox{%
 \epsfxsize=77.503mm 
 \epsfbox{dul-2.eps}
 }


\end{center}


\noindent
{{\figurename~2}\ \ \small{Три категории, используемые в~исследовании синтеза геоданных 
(расположены в~соответствии с~увеличением семантического контекста)}}
}

\vspace*{9pt}

\addtocounter{figure}{1}



\noindent
ро\-вое представление объекта реального мира или абстракция реального 
мира. В~качестве одного из его атрибутов оно может иметь пространственные, 
временн$\acute{\mbox{ы}}$е или про\-стран\-ст\-вен\-но-вре\-мен\-н$\acute{\mbox{ы}}$е характеристики. Феномены обычно 
управляются группами в~виде коллекций феноменов. Термины феномен 
и~объект часто используются как синонимы. Термины феномен, набор 
феноменов и~покрытие определяются в~соответствии с~принятыми 
в~OpenGIS>>. В~дальнейшем в~этой работе термин \textit{feature} будет 
обозначаться как феномен.
     
     В исследовании синтеза геоданных принято рассматривать три категории 
(рис.~2): сопоставление наблюдений и~измерений (observation fusion); 
сопоставление феноменов (feature fusion) и~синтез решений (decision 
fusion)~\cite{5-dul}. 
     

\section{Сопоставление наблюдений и~измерений}

     В результате синтеза наблюдений многократные измерения датчиками 
одних и~тех же явлений объединяются в~некое комбинированное наблюдение. 
Процессы сопоставления тем самым формируют комбинацию различных 
измерений датчиков в~виде хорошо описываемого наблюдения, учитывая при 
этом присущие этому процессу неопределенности (например, анализ подписи 
клиента или распознавание его лица в~различных ракурсах). 
     
     Основные требования для сопоставления наблюдений и~измерений 
с~помощью датчика включают~\cite{6-dul}:
     \begin{enumerate}[(1)]
\item формирование системы датчиков, необходимых наблюдений 
и~процессов наблюдения, которые удовлетворяют потребностям 
пользователей;
\item определение возможностей датчиков и~качества измерений; 
\item доступность параметров датчика, которые позволяют автоматически 
определять мес\-то\-на\-хож\-де\-ние наблюдений;
\item определение реального времени наблюдений в~стандартных 
кодировках, включая кодирование неопределенности измерений, 
и~па\-ра\-мет\-ров, необходимых для обработки измерений; 
\item настройка датчиков на проведение наблюдений; 
\item выработка и~публикация предупреждений, которые будут выдаваться 
датчиками или сервисами датчиков; 
\item идентификация и~классификация объекта; 
\item включение сопоставления наблюдений и~измерений обеспечением 
доступа к~механизмам обработки и~необходимой информации об объекте. 
\end{enumerate}

     
     Информация для проведения синтеза может быть получена как из 
наблюдений за датчиками, так и~от людей. Информация как результат 
наблюдения может служить вводом в~процессы сопоставления наблюдений 
и~измерений или может использоваться для идентификации распознаваемых 
объектов, характеристики которых обрабатываются как вводные для процесса 
синтеза. Стандарты для сопоставления наблюдений и~измерений используются 
уже продолжительное время и~доказали свою жизнеспособность.

\section{Сопоставление феноменов и~синтез решений}
     
     Сопоставление феноменов включает обработку наблюдений на более 
высоком уровне семантических свойств феноменов. Это улучшает понимание 
операционной ситуации и~уточняет оценку потенциальных угроз и~влияний, 
позволяя более корректно идентифицировать, классифицировать, связать 
и~объединить объекты, представляющие интерес. Процессы сопоставления 
феноменов включают их обобщение и~агрегирование. Технология 
агрегирования включает полезные опции, позволяя работать с~несовершенными, 
гетерогенными, противоречивыми и~дублированными геоданными. 
     
     Сервис-ориентированная архитектура (service-oriented architecture, 
     SOA)~\cite{7-dul} хорошо 
подходит для поддержки в~распределенных сервисах правил агрегирования 
феноменов. Сопоставление феноменов обеспечивает работу с~более мощными, 
гибкими и~точными информационными ресурсами, чем с~полученными из 
оригинальных источников. 
     
     Синтез решений инициирует процессы, поддерживающие способность 
человека принять решение, обеспечивая среду взаимодействующего сетевого 
обслуживания для оценки ситуации и~поддержки принятия решений, используя 
информацию от различных датчиков и~уже обработанную информацию.
     
     Синтез решений обеспечивает аналитикам среду, где они, используя 
интерфейс отдельного клиента, могут обращаться к~интероперабельным 
ин\-струментальным средствам, чтобы находить, обрабаты\-вать и~использовать 
различные типы данных, полученных от разных датчиков и~баз данных. Синтез 
решений включает использование информации от корпоративных 
геосообществ, которая позволяет оценить ситуацию в~целом и~воспользоваться 
общей операционной ситуацией. Поэтому можно сказать, что информация, 
доступная для осуществления синтеза решений, является коллекцией сведений 
интеллектуального корпоративного капитала. Источниками таких сведений 
выступают люди, документы, оборудование или технические датчики. 
     
     Развитие синтеза решений требует разработок стандартов для 
структурированной информации, таких как методы отображения схемы 
с~соответствующей идентификацией правил отображения, и~увеличения 
акцента на обрабатывание ассоциаций, поскольку идентификация ассоциации 
между объектами лежит в~основе синтеза. Самая эффективная среда для 
реализации указанных категорий синтеза~--- это архитектура с~распределенными 
базами данных и~сервисами, основанными на общем ядре форматов геоданных, 
алгоритмов и~приложений~\cite{8-dul}. 
     
     Синтез решений представляет собой масштабную операцию, 
включающую как геоданные, получаемые от человека с~мобильного устройства, 
так и~геоданные общей обстановки из центра МЧС. Неявная коммуникация~--- 
это сотрудничество с~другими организациями и~ведомствами.
     
     Существующие публикации предлагают разнообразие классификаций для 
синтеза пространственных данных. Эти подходы могут различаться (рис.~3) по 
области применения, уровню автоматизации, частоте операций, уровню 
соответствия, принятой модели данных и~про\-стран\-ст\-вен\-но-вре\-мен\-н$\acute{\mbox{о}}$й 
привязке входных данных~\cite{2-dul}. Соответствующие реализации процессов 
можно различать, основываясь на структурах геоданных ввода и~вывода, 
которые эти процессы поддерживают, на соответствующих прикладных 
стратегиях, на вычислительных показателях и~других качественных мерах.



     Обеспечение процессов синтеза простран\-ст\-венных данных  
в~сервис-ориентированной архитектуре (SOA) для обслуживания 
широкого круга\linebreak запро\-сов, такой как SDI, требует организации и~определения 
соответствующих сервисных интерфейсов. Эти интерфейсы предназначены для 
декомпозиции 
 и~инкапсуляции соответствующих\linebreak наборов взаимодействующих 
функций синтеза, которые будут 
исполь\-зо\-вать\-ся для различных 
технологических процессов синтеза. Такая организация обеспечит интерфейсы 
для поддержки свободного связывания геоданных, многократного 
использования и~компонуемости сервисов. Можно определить три уровня 
детализации:
     \begin{enumerate}[(1)]
\item атомарные операции, которые обеспечивают прикладные 
функциональные возможности системы типа простых арифметических 
операций и~определения географических размеров;
\item операции низкого уровня, которые могут использоваться приложением, 
но все-таки харак\-те\-ризуются как базисные и~поэтому пригодны для 
использования множеством приложений. Они включают, например, 
геометрические преобразования, показывают степень подобия феноменов 
и~операции перемещения феноменов;
\item операции высокого уровня, которые ориентированы на специфику 
определенного приложения, обладая ограниченной гибкостью 
и~возможностью многократного использования, такого как повторяющиеся 
технологические процессы, привязанные к~конкретным исходным данным 
и~целевым показателям приложений.
\end{enumerate}
     
     Таким образом, любая операция может неоднократно использоваться или 
быть композицией операций низшего уровня. Однако на практике эти уровни не 
всегда строго разделяются. Чтобы реализовать синтез пространственных 
данных на основе сервисов гибким и~практичным способом при условии 
неоднократного их применения, можно использовать комплект инструментов 
операций низкого уровня, и~это, возможно, самый перспективный подход. 
Типичный технологический процесс синтеза и~его логическая декомпозиция 
представлены на рис.~4. 
               Этот рисунок служит формальной структурой для процессов синтеза 
низкого уровня, вклю\-ча\-ющей следующие шаги:
\end{multicols}

\begin{figure*} %fig3
 \vspace*{1pt}
 \begin{center}  
  \mbox{%
 \epsfxsize=147.78mm 
 \epsfbox{dul-3.eps}
 }
\end{center}
\vspace*{-13pt}
\Caption{Классификация процесса синтеза пространственных данных}
\vspace*{-3pt}
\end{figure*}

\begin{multicols}{2}


{ \begin{center}  %fig4
 \vspace*{-6pt}
   \mbox{%
 \epsfxsize=78mm 
 \epsfbox{dul-4.eps}
 }


\end{center}

\vspace*{-12pt}


\noindent
{{\figurename~4}\ \ \small{Операции низкого уровня для проведения синтеза пространственных данных}}
}

%\vspace*{9pt}

\addtocounter{figure}{1}


\noindent
     \begin{enumerate}[(1)]
\item поиск и~извлечение геоданных, нацеленных на идентификацию и~сбор 
входных геоданных для синтеза пространственных данных; 
\item улучшение данных производится на основе анализа качественных 
характеристик индивидуальных входных источников, релевантных процессу 
синтеза. Оно включает задачи формирования и~исправления~\cite{9-dul, 10-dul}, 
предварительной обработки признаков и~процессы  
классификации~\cite{11-dul}; 
\item согласование геоданных нацелено на устранение несогласованности 
между вводами, чтобы обеспечить общую синтаксическую, структурную 
и~семантическую основу для поддержки корректного синтеза геоданных. 
Согласование обычно выполняется на уровне набора геоданных и~включает 
обработку позиционной перегруппировки~\cite{4-dul}, синхронизации 
времени, приведения форматов и~координатного  
преобразования~\cite{12-dul} и~генерализации;
\item сопоставление наблюдений и~измерений исследует квантификацию 
размерных отно\-шений между входными источниками.\linebreak Оно может быть 
выполнено в~пред\-став\-ле\-нии феноменов, схеме или на семантическом уровне 
и~типично выражает определенный вид подобия; 
\item сопоставление феноменов использует соотношения измерений для 
определения типов связей между феноменами в~пред\-став\-ле\-нии и~типов 
отношений между характеристиками в~пред\-став\-ле\-нии, схеме или на 
семантическом уровне. Так же как и~соотношение измерений, оно зависит от 
надежности входных источников и~знаний об объекте~\cite{13-dul};
\item синтез решений зависит от предыду\-щих результатов и~определяется 
целевой программой. Стратегии разрешения противоречий могут быть 
применены для устранения любых несогласованностей, 
идентифицированных в~процессе реализации синтеза. Фактически цель 
процесса синтеза может быть достигнута путем использования 
идентифицированных отношений для сопоставления, сравнения, обновления 
или улучшения данных. Этот этап включает передачу феноменов 
и~признаков, которые не присутствуют во всех наборах геоданных 
(например, полученных при взаимных обновлениях), а~также агрегирование 
связанных феноменов, чтобы получить расширенное или улучшенное 
представление феноменов;
\item предоставление геоданных связано с~кодированием, хранением 
и~регистрацией результатов, чтобы облегчить доступ и~использование 
результатов для визуализации или дальнейшего анализа. Идеально, если 
результаты синтеза будут включать информацию о неопределенности 
и~происхождении геоданных.
\end{enumerate}
     
     Шаги, описанные выше, не должны расцениваться как формирование 
строгой последовательности. Отдельные операции могут быть пропущены, 
повторены или объединены иначе. В~частности, шаги~4 (соотношение 
измерений), 5~(со\-по\-став\-ле\-ние феноменов) и~6 (решение) часто 
объединяются в~единый процесс идентификации и~сопоставления феноменов.
     
\section{Синтез пространственных данных на~основе SDI 
и~связанных данных}
     
     Принятие стандарта SDIs привело к~публикации, поиску, доступу 
и~обработке пространственных данных открытым и~стандартизованным\linebreak 
способом~\cite{14-dul}. Semantic Web обеспечивает повсеместный доступ 
к~связанным данным в~Web. Парадигма связанных данных предполагает, что 
URI (Universal Resource Identifier) используется для однозначного определения объекта или понятия, которые 
могут быть соотнесены с~HTTP-ад\-ре\-са\-ми, обеспечивая данные 
в~стандартизованных форматах, связанных с~другими источниками 
данных в~Web.
     
     Ориентируясь на интероперабельный, основанный на сервисах синтез 
пространственных данных, следует учитывать, что все этапы синтеза 
геоданных, описанные в~предыдущем разделе, должны быть согласованы 
с~текущими стандартами SDI и~Semantic Web. Это может быть достигнуто 
с~по\-мощью стандартов OGC (Open Geospatial Consortium) 
для регистрации, кодирования, визуализации 
и~обработки пространственных данных и~использования стандартов W3C 
для 
связанных данных, в~част\-ности в~форматах RDF (Resource Description 
Framework) и~SPARQL (протокол SPARQL и~язык запросов RDF). На рис.~5 
показан пример возможной коммуникации между SDI и~компонентами Semantic 
Web для синтеза пространственных данных в~Web. В~то время как компоненты 
SDI достижимы с~помощью сервисов OWS (OGC Web Services), компоненты 
Semantic Web обеспечивают функциональные возможности, основанные на 
интерфейсах RDF.


  
      
     Метод управления на основе уникальной идентификации в~SDIs 
предопределяет использование связанных данных. Однако управление 
постоянными уникальными идентификаторами и~реализация эффективной 
идентификации ресурса и~стратегии\linebreak\vspace*{-12pt}

{ \begin{center}  %fig5
 \vspace*{12pt}
   \mbox{%
 \epsfxsize=79mm 
 \epsfbox{dul-5.eps}
 }


\end{center}


\noindent
{{\figurename~5}\ \ \small{Взаимодействие между SDI (интерфейс OWS) и~компонентами Semantic 
Web (интерфейс RDF)}}
}

%\vspace*{9pt}

\addtocounter{figure}{1}


\noindent
 управления в~большом масштабе требуют 
значительных усилий по согласованности на концептуальном уровне.
     
\section{Заключение}

     Как показали вышеизложенные исследования, для обеспечения доступа 
     и~использования отношений феноменов из различных распределенных\linebreak 
источников геоданных эти источники данных должны быть полностью 
определены. Схема отношений, которая описывает отношения феноменов, 
адекватные SDI и~согласованные с~принципами связанных данных, должна 
содержать следующие компоненты:
     \begin{itemize}
\item ресурс, соответствующий фундаментальному понятию Semantic Web; 
\item отношение феноменов~--- основной класс, который содержит все 
феномены, участвующие в~отношении, типы связанных отношений 
и~измерения;
\item тип отношения, который квалифицирует типы связей между входными 
феноменами;
\item ресурс феномена, который представляет собой еще один тип феномена, 
связанный отношением. Он указывает на сервис данных SDI, 
обслуживающий тип феномена и~уникально сопоставимый с~ресурсом; 
\item соотношение измерений, обеспечивающее получение информации об 
основных измерениях, которые поддерживаются типами отношений и~сами 
занесены в~отношение, включая информацию о подобии и~секретности. 
\end{itemize}

{\small\frenchspacing
 {%\baselineskip=10.8pt
 \addcontentsline{toc}{section}{References}
 \begin{thebibliography}{99}
\bibitem{1-dul}
OGC Engineering Reports. {\sf http://www.opengeospatial.
org/standards}.
\bibitem{2-dul}
\Au{Ruiz J.\,J., Ariza~F.\,I., Ure$\tilde{\mbox{n}}$a~M.\,A., Bl$\acute{\mbox{a}}$zquez~E.\,B.} Digital map conflation: A~review of the process and a proposal for 
classification~// Int. J.~Geogr. Inf. Sci., 2011. Vol.~25. Iss.~9. P.~1439--1466.
\bibitem{3-dul}
\Au{Schwinn A., Schelp J.} Design patterns for data integration~// J.~Enterprise Information 
Management, 2005. Vol.~18. Iss.~4. P.~471--482.
\bibitem{4-dul}
\Au{L$\acute{\mbox{o}}$pez-V$\acute{\mbox{a}}$zquez~C., Callejo~M.\,A.\,M.} Point- and  
curve-based geometric conflation~// Int. J.~Geogr. Inf. Sci., 2013. Vol.~27. Iss.~1. 
P.~192--207.
\bibitem{5-dul}
OGC Web Services Testbed, Phase~8 (OWS-8) Demonstrations. {\sf 
http://www.opengeospatial.org/pub/www/ ows8/index.html}.
\bibitem{6-dul}
\Au{Botts M., Percivall~G., Carl Reed~C., Davidson~J.} 
OGC Sensor Web Enablement (SWE): Overview and high level 
architecture. OGC White Paper. Document OGC\linebreak 07-165r1, April~2, 2013.
\bibitem{7-dul}
\Au{Pravia M.} Generation of a~fundamental data set for hard/soft information fusion~// 11th 
Conference (International) on Information Fusion.~--- Cologne: International Society of Information 
Fusion, 2008. P.~134--145.
\bibitem{8-dul}
OGC Fusion Standards Study: Phase~2 Engineering Report~/ Ed. G.~Percivall. 
OGC Document 
10-184, December~13, 2010. {\sf http://www.opengeospatial.org/ files/?artifact\_id=41573}.
\bibitem{9-dul}
\Au{Butenuth M., von Goesseln~G., Sester~M.} Integration of heterogeneous geospatial data in a~federated 
database~// ISPRS J.~Photogramm., 2007. Vol.~62. Iss.~5. P.~328--346. 
\bibitem{10-dul}
\Au{Al-Bakri M., Fairbairn~D.} Assessing similarity matching for possible integration of feature 
classifications of geospatial data from official and informal sources~// Int. J.~Geogr. 
Inf. Sci., 2012. Vol.~26. Iss.~8. P.~1437--1456.
\bibitem{11-dul}
\Au{Koukoletsos T., Haklay~M., Ellul~C.} Assessing data completeness of VGI through an 
automated matching procedure for linear data~// T.~GIS, 2012. Vol.~16. Iss.~4. P.~477--498.
\bibitem{12-dul}
\Au{Stankute S., Asche~H.} A~data fusion system for spatial data mining, analysis and 
improvement~// Computational science and its applications~/
Eds. B.~Murgante, O.~Gervasi, S.~Misra, N.~Nedjah, A.\,A.\,C.~Rocha, D.~Taniar, 
B.\,O.~Apduhan.~--- Lecture notes in computer science ser.~---  Springer-Verlag, 2012. 
Vol.~7334. P.~439--449. 
\bibitem{13-dul}
\Au{Cholvy L.} Modelling information evaluation in fusion~//  10th Conference (International ) on 
Information Fusion Proceedings.~--- Quebec, Canada: IEEE, 2007. CD-ROM. P.~1--6. doi: 
10.1109/ICIF.2007.4408060.
\bibitem{14-dul}
\Au{Bernard L., Kanellopoulos~I., Annoni~A., Smits~P.} The European Geoportal~--- one step 
towards the establishment of a~European spatial data infrastructure~// Comput. Environ. 
Urban, 2005. Vol.~29. Iss.~1. P.~15--31.
 \end{thebibliography}

 }
 }

\end{multicols}

\vspace*{-3pt}

\hfill{\small\textit{Поступила в~редакцию 04.07.18}}

%\vspace*{8pt}

%\pagebreak

\newpage

\vspace*{-28pt}

%\hrule

%\vspace*{2pt}

%\hrule

%\vspace*{-2pt}

\def\tit{SYNTHESIS OF~GEODATA IN~SPATIAL INFRASTRUCTURES 
BASED~ON~RELATED DATA}

\def\titkol{Synthesis of~geodata in~spatial infrastructures 
based~on~related data}

\def\aut{S.\,K.~Dulin$^{1,2}$, N.\,G.~Dulina$^3$, and~O.\,S.~Kozhunova$^1$}

\def\autkol{S.\,K.~Dulin , N.\,G.~Dulina, and~O.\,S.~Kozhunova}

\titel{\tit}{\aut}{\autkol}{\titkol}

\vspace*{-11pt}


\noindent
$^1$Institute of Informatics Problems, Federal Research Center ``Computer Science
and Control'' of the Russian\linebreak
$\hphantom{^1}$Academy of Sciences, 44-2~Vavilov Str., Moscow 119333, Russian 
Federation

\noindent
$^2$Research \& Design Institute for Information Technology, Signalling and Telecommunications 
on Railway\linebreak
$\hphantom{^1}$Transport (JSC NIIAS), 27-1~Nizhegorodskaya Str., Moscow 109029, Russian 
Federation

\noindent
$^3$A.\,A.~Dorodnicyn Computing Center, Federal Research Center ``Computer Science
and Control'' of the Russian\linebreak
$\hphantom{^1}$Academy of Sciences, 40~Vavilov Str., Moscow
119333, Russian Federation

\def\leftfootline{\small{\textbf{\thepage}
\hfill INFORMATIKA I EE PRIMENENIYA~--- INFORMATICS AND
APPLICATIONS\ \ \ 2019\ \ \ volume~13\ \ \ issue\ 1}
}%
 \def\rightfootline{\small{INFORMATIKA I EE PRIMENENIYA~---
INFORMATICS AND APPLICATIONS\ \ \ 2019\ \ \ volume~13\ \ \ issue\ 1
\hfill \textbf{\thepage}}}

\vspace*{6pt}



\Abste{Synthesis of spatial data from various sources available on the Web is the main task for 
modern applications that use information retrieval in the network and are aimed at making decisions 
based on geodata. This work is devoted to the synthesis of spatial data with an emphasis on its 
application in Spatial Data Infrastructures (SDIs). The possibilities for 
integrating SDIs and semantic contexts are discussed subject to an agreed description and use of 
object characteristics relations. Classification and decomposition of synthesis processes in 
a~service-oriented structure for servicing a wide range of queries are proposed.}

\KWE{data synthesis; spatial data infrastructure; related data; semantic network}



\DOI{10.14357/19922264190112}

%\vspace*{-14pt}

 \Ack
\noindent
The work was partly supported by the Russian Foundation for Basic Research (project 17-20-
02153~ofi\_m\_ RZhD).



%\vspace*{6pt}

  \begin{multicols}{2}

\renewcommand{\bibname}{\protect\rmfamily References}
%\renewcommand{\bibname}{\large\protect\rm References}

{\small\frenchspacing
 {%\baselineskip=10.8pt
 \addcontentsline{toc}{section}{References}
 \begin{thebibliography}{99}
\bibitem{1-dul-1}
OGC Engineering Reports. 
Available at: {\sf http://www. opengeospatial.org/standards/} (accessed March~19, 2019).
\bibitem{2-dul-1}
\Aue{Ruiz, J.\,J., F.\,J.~Ariza, M.\,A.~Ure$\tilde{\mbox{n}}$a, and E.\,B.~Bl$\acute{\mbox{a}}$zquez.} 2011. Digital map conflation: A~review of the process and 
a~proposal for classification. \textit{Int. J.~Geogr. Inf. Sci.} 25(9):1439--1466.
\bibitem{3-dul-1}
\Aue{Schwinn, A., and J.~Schelp.} 2005. Design patterns for data integration. 
\textit{J.~Enterprise 
Information Management} 18(4):471--482.
\bibitem{4-dul-1}
\Aue{L$\acute{\mbox{o}}$pez-V$\acute{\mbox{a}}$zquez,~C., and M.\,A.\,M.~Callejo.} 2013. 
Point- and curve-based geometric conflation. \textit{Int. J.~Geogr. Inf. Sci.} 
27(1):192--207.
\bibitem{5-dul-1}
OGC Web Services Testbed, Phase~8 (OWS-8) Demonstrations, 
Available at: {\sf http://www.opengeospatial.org/ pub/www/ows8/index.html} (accessed 
March 2018).
\bibitem{6-dul-1}
\Aue{Botts, M., G.~Percivall, C.~Carl Reed, and J.~Davidson.} 
April~2, 2013. OGC Sensor Web Enablement (SWE): Overview and high level 
architecture. OGC White Paper. Document OGC 07-165r1.

\bibitem{7-dul-1}
\Aue{Pravia, M.} 2008. Generation of a~fundamental data set for hard/soft information fusion.
\textit{11th Conference (International) on Information Fusion}. Cologne: International Society of 
Information Fusion. 134--145.
\bibitem{8-dul-1}
Percivall, G., ed. 
December~13, 2010. OGC Fusion Standards Study, Phase~2 Engineering Report. OGC Document  
10-184.  Available at: {\sf 
http://www.opengeospatial. org/files/?artifact\_id=41573} (accessed April 2018).
\bibitem{9-dul-1}
\Aue{Butenuth, M., G.~von Goesseln, and M.~Sester.} 2007. Integration of heterogeneous geospatial data in 
a~federated database. \textit{ISPRS J.~Photogramm.} 62(5):328--346. 
\bibitem{10-dul-1}
\Aue{Al-Bakri, M., and D.~Fairbairn.} 2012. Assessing similarity matching for possible integration 
of feature classifications of geospatial data from official and informal sources. \textit{Int. 
J.~Geogr. Inf. Sci.} 26(8):1437--1456.
\bibitem{11-dul-1}
\Aue{Koukoletsos, T., M.~Haklay, and C.~Ellul.} 2012. Assessing data completeness of VGI 
through an automated matching procedure for linear data. \textit{T.~GIS} 16(4):477--498.
\bibitem{12-dul-1}
\Aue{Stankute, S., and H.~Asche.} 2012. A~data fusion system for spatial data mining, analysis 
and improvement. \textit{Computational science and its applications}. Eds.\ 
B.~Murgante, O.~Gervasi, S.~Misra, N.~Nedjah, A.\,A.\,C.~Rocha, D.~Taniar, and 
B.\,O.~Apduhan.  Lecture notes in computer science ser. Springer-Verlag. 7334:439--449. 
\bibitem{13-dul-1}
\Aue{Cholvy, L.} 2007. Modelling information evaluation in fusion. \textit{10th Conference 
(International) on Information Fusion Proceedings}. Quebec, Canada: IEEE. CD-ROM.  1--6.
doi: 
10.1109/ICIF.2007.4408060.
\bibitem{14-dul-1}
\Aue{Bernard, L., I.~Kanellopoulos, A.~Annoni, and P.~Smits.} 2005. The European  
Geoportal~--- one step towards the establishment of a~European spatial data infrastructure. 
\textit{Comput. Environ. Urban} 29(1):15--31.
\end{thebibliography}

 }
 }

\end{multicols}

\vspace*{-6pt}

\hfill{\small\textit{Received July 4, 2018}}

%\pagebreak

%\vspace*{-18pt}

\Contr

\noindent
\textbf{Dulin Sergey K.} (b.\ 1950)~---
Doctor of Science in technology, professor; leading scientist, Institute of 
Informatics Problems, Federal Research Center ``Computer Science and Control'' 
of the Russian Academy of Sciences, 44-2~Vavilov Str., Moscow 119333, Russian 
Federation; principal scientist, Research \& Design 
Institute for Information Technology, Signalling and Telecommunications on 
Railway Transport (JSC NIIAS), 27-1~Nizhegorodskaya Str., Moscow 109029, Russian 
Federation; \mbox{skdulin@mail.ru}

\vspace*{3pt}


\noindent
\textbf{Dulina Natalia G.} (b.\ 1947)~---
Candidate of Science (PhD) in technology, leading programmer, A.\,A.~Dorodnicyn 
Computing Center, Federal Research Center ``Computer Science and Control'' of 
the Russian Academy of Sciences, 40~Vavilov Str., Moscow 119333, Russian 
Federation; \mbox{ngdulina@mail.ru}
      
\vspace*{3pt}


\noindent
\textbf{Kozhunova Olga S.} (b.\ 1982)~--- Candidate of Science (PhD) in 
technology, Head of Laboratory, Institute of Informatics Problems, Federal 
Research Center ``Computer Science and Control'' of the Russian Academy of 
Sciences, 44-2~Vavilov Str., Moscow 119333, Russian Federation; 
\mbox{kozhunovka@mail.ru}

\label{end\stat}

\renewcommand{\bibname}{\protect\rm Литература}  