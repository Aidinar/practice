\def\stat{sinits}

\def\tit{ИНТЕРПОЛЯЦИОННОЕ АНАЛИТИЧЕСКОЕ 
 МОДЕЛИРОВАНИЕ РАСПРЕДЕЛЕНИЙ В~СЛОЖНЫХ СТОХАСТИЧЕСКИХ СИСТЕМАХ}

\def\titkol{Интерполяционное аналитическое 
 моделирование распределений в~сложных стохастических системах}

\def\aut{И.\,Н.~Синицын$^1$}

\def\autkol{И.\,Н.~Синицын}

\titel{\tit}{\aut}{\autkol}{\titkol}

\index{Синицын И.\,Н.}
\index{Sinitsyn I.\,N.}




%{\renewcommand{\thefootnote}{\fnsymbol{footnote}} \footnotetext[1]
%{Работа поддержана РФФИ (проект 18-07-00274).}}


\renewcommand{\thefootnote}{\arabic{footnote}}
\footnotetext[1]{Институт проблем
информатики Федерального исследовательского центра
<<Информатика и управ\-ле\-ние>> Российской академии наук, \mbox{sinitsin@dol.ru}}

\vspace*{-12pt}


\Abst{Разработаны методы интерполяционного аналитического моделирования (ИАМ) 
в~сложных непрерывных и дискретных скалярных и приводимых к ним векторных
стохастических системах (СтС). 
Выделены типовые классы сложных СтС. Методы основаны на интерполяционном 
решении эволюционного уравнения для одномерной 
характеристической функции (х.ф.). Получены уравнения чувствительности для оценки х.ф.\
 к~параметрам СтС.
В~основу базовых алгоритмов ИАМ положена теорема отсчетов Котельникова. 
Рассмотрены практические вопросы выбора интерполяционных формул 
Котельникова и оценки количества интерполяционных отсчетов.
Для СтС с известной аналитической природой в качестве дальнейшего 
развития следует рассмотреть применение сплайнт-вейв\-лет методов для офлайн 
алгоритмов, а~для онлайн алгоритмов~--- фильтрационных подходов, 
при этом особое внимание следует уделить многомерным распределениям.}

\KW{одномерная плотность вероятности (п.в.);
одномерная характеристическая функция (х.ф.);
стохастическая система (СтС);
стохастический процесс (СтП)}

\DOI{10.14357/19922264190101}
  
\vspace*{-3pt}


\vskip 10pt plus 9pt minus 6pt

\thispagestyle{headings}

\begin{multicols}{2}

\label{st\stat}

\section{Введение}

Среди известных методов аналитического моделирования стохастических процессов 
(СтП) в~дифференциальных СтС
важное место занимают методы, основанные на прямом численном решении эволюционных 
уравнений для $n$-мер\-ной х.ф.~[1]. Мальчиковым в~[2, 3] для гауссовских СтС 
на основе теоремы отсчетов Котельникова~[4] разработан метод ИАМ для одномерной х.ф.

В~[5] дано обобщение~[2, 3] для негауссовских СтС на основе замены негауссовкой 
СтС эквивалентной гауссовской (нормальной) СтС. Дадим обобщение~[5] 
на случай непрерывных и дискретных СтС, не приводимых к гауссовским. Разделы~2 и~3 
посвящены методу ИАМ соответственно для непрерывных и дискретных СтС. 
В~разд.~4 описаны практические способы выбора интерполяционных формул и чисел отсчетов. 
Заключение содержит выводы и некоторые возможные обобщения.

\vspace*{-12pt}

\section{Метод интерполяционного 
аналитического моделирования для~непрерывных стохастических систем}

\vspace*{-3pt}

2.1. Следуя~[5], введем следующие определения и допущения.

\columnbreak

\noindent
\begin{enumerate}[1.]
\item Пусть скалярный СтП  $Y(t) \hm= Y_t$ определяется скалярным 
стохастическим дифференциальным уравнением Ито следующего вида~[1]:

\noindent
\begin{multline}
dY_t = a \left(Y_t, \Theta, t\right) dt + b \left(Y_t, \Theta, t\right)\, 
d W_0 (\Theta, t)+ {}\\
\!{}+\!
    \iii_{R_0^1}\! c(Y_t, \Theta, t, v) P^0 (\Theta, dt, dv),\enskip  
    Y\left(t_0\right)=Y_0\,.\label{e2.1-s}
    \end{multline}
Здесь $W_0 \hm= W_0 (\Theta, t)$~--- скалярный винеровский СтП интенсивности 
$\nu_0\hm=\nu_0 (\Theta, t)$; $\Theta$~--- $n^\Theta$-мер\-ный 
вектор параметров; $P^0(\Theta, {\cal J}, {\cal A})$  пред\-став\-ля\-ет собой 
для любого множества~${\cal A}$ простой пуассоновский СтП, 
причем $P^0 (\Theta, {\cal J},{\cal A})\hm= P(\Theta, {\cal J},{\cal A})
\hm- \mu_P (\Theta, {\cal J},{\cal A})$, где 
$\mu_P (\Theta, {\cal J},{\cal A})$~--- его математическое ожиданием, равное

\noindent
    \begin{multline*}
    \mu_P =\mu_P(\Theta, {\cal J},{\cal A})={\sf M}\, P(\Theta, {\cal J},{\cal A})= {}\\
    {}=
    \iii_{\cal J} \nu_P (\Theta, \tau,{\cal A})\, d\tau\,,
\end{multline*}
а $\nu_P\hm=\nu_P(\Theta, {\cal J},{\cal A})$~--- 
интенсивность соответствующего пуассоновского потока событий; 
${\cal J}\hm= (t_1, t_2]$; интегрирование по~$v$ распространяется на пространство~$R^1$ 
с~выколотым началом координат; $a\hm=a(Y_t, \Theta, t)$, $b\hm=b(Y_t, \Theta, t)$ 
и~$c\hm=c(Y_t, \Theta, t,v)$~--- скалярные нелинейные функции отмеченных аргументов.

Предположим, что при фиксированном~$\Theta$ выполнены условия 
существования и единствен-\linebreak\vspace*{-12pt}
\end{enumerate}

\begin{enumerate}[1.]
\setcounter{enumi}{1}
\item[\ ] ности СтП, определяемого~(\ref{e2.1-s}) 
при соответствующем начальном условии~[1, 6]. Кроме того, функции~$a$,
$b$ и~$c$ и характеристики СтП~$W_0$ и~$P^0$ будем считать дифференцируемыми по~$\Theta$.

В~дальнейшем для белого шума $V\hm=\dot W$ (понимаемого 
в~строгом смысле и являющегося производной по времени от произвольного СтП 
с~независимыми приращениями~$W(t)$) будем использовать уравнение:
\begin{equation}
\dot Y_t =a\left(Y_t, \Theta, t\right)+ 
b\left(Y_t, \Theta, t\right)V\,, \enskip 
V=\dot W\,.\label{e2.2-s}
\end{equation}

\item Существует одномерная плотность $f\hm=f(y;\Theta,t)$, 
а~уравнение для одномерной х.ф.\ $g\hm=g(\lambda;\Theta,t)$  при фиксированном~$\Theta$ 
имеет вид~[1]:
\begin{multline}
    \fr{\prt g (\lambda;\Theta,t)}{\prt t} = \iin \left[ i\lambda a 
    (y,\Theta,t)+ \right.\\
\left.    {}+\chi(\lambda;y,\Theta,t)\right] e^{i\lambda y} f ( y; \Theta,t)\, dy
    \label{e2.3-s}
    \end{multline}
при начальном условии
\begin{equation}
g(\lambda; \Theta,t_0) = g_0 (\lambda;\Theta)\,.
\label{e2.4-s}
\end{equation}
Здесь $f\hm=f(y;\Theta,t)$~--- одномерная плотность вероятности (п.в.), 
связанная с х.ф.~$g(\lambda;\Theta,t)$ следующим Фурье преобразованием:
\begin{equation}
f(y;\Theta,t) = \fr{1}{2\pi} \iin e^{-i\lambda y} g 
(\lambda;\Theta,t)\, d\lambda\,;\label{e2.5-s}
\end{equation}

\vspace*{-12pt}

\noindent
\begin{multline*}
\chi(\lambda;y,\Theta,t) =-\fr{\lambda^2 \nu_0 (\Theta,t)}{2}\, b^2 (y,\Theta,t)+{}\\
{}+ \iii_{R_0'} \exp \left\{\lk i\lambda 
c(y,\Theta,t,v)\rk -1-{}\right.\\
\left.{}-i\la c(y,\Theta,t,v)\right\} \nu_P (\Theta,t,dv)\,.
%\label{e2.6-s}
\end{multline*}

Для стационарных дифференциальных СтС~(\ref{e2.1-s}) 
уравнения для стационарных х.ф.\ $g^*\hm=g^*(\la;\Theta)$  
и~п.в.~$f^*(y;\Theta)$ имеют вид:
\begin{multline}
\!\iin \!\left[ i \lambda a^* (y,\Theta) + \chi( \lambda;y,\Theta)\right]
 e^{i\lambda y} 
f^* (y;\Theta)\, dy ={}\\
{}=0\,;\label{e2.7-s}
\end{multline}

\vspace*{-9pt}

\noindent
\begin{equation*}
f^*(y;\Theta) = \fr{1}{2\pi} \iin e^{i\lambda y} g^* ( \lambda;\Theta)\, dy\,.
\label{e2.8-s}
\end{equation*}
Здесь звездочкой отмечены стационарные значения соответствующих функций.

Для СтС~(\ref{e2.2-s}) уравнения~(\ref{e2.3-s}) и~(\ref{e2.7-s}) имеют сле\-ду\-ющий вид:

\columnbreak

\noindent
    \begin{multline}
    \fr{\prt y ( \lambda;\Theta, t)}{\prt t} = 
    \iin \left[ i\lambda a (y, \Theta, t) +{}\right.\\
    \left. {}+
    \chi( b(y,\Theta, t)\lambda; t)\right] \exp (i\lambda y) f(y,\Theta,t)\,dy\,;
    \label{e2.9-s}
    \end{multline}
    
    \vspace*{-16pt}
    
    \noindent
    \begin{multline*}
    \iin  \lk i\lambda a^* (y, \Theta) + \chi
    \left( b^*(y,\Theta)\lambda; t\right)\rk\times{}\\
    {}\times \exp (i\lambda y) f^*(y,\Theta)\,dy=0\,.
%\label{e2.10-s}
    \end{multline*}

\item Пусть х.ф.\ $g\hm=g(\lambda;\Theta, t)$ для СтС~(\ref{e2.1-s}) 
выражается через значение в фиксированных точках отсчета, применив для этого 
ка\-кую-ни\-будь интерполяционную формулу~[6]:
\begin{equation}
    g(\lambda; \Theta, t) = \sss_{l=-N}^N g_l \left(\lambda_l;\Theta, t\right) 
    \alpha_l (\lambda)+ R_N\,.\label{e2.11-s}
    \end{equation}
Здесь $R_N$~--- остаточный член;  $\lambda_l$~--- 
некоторые фиксированные значения аргумента~$\lambda$ в~точках отсчета х.ф.; 
$\alpha_l(\lambda)$~--- известные функции, удовлетворяющие условию
    $$
    \alpha_l (\lambda_r) =\delta_{lr}\enskip (l,r=0, \pm 1\tr \pm N)\,,
    $$
где $\delta_{lr}$~--- символ Кронекера.

\end{enumerate}

Таким образом, задача сводится к вычислению значений х.ф.\  
в~точках отсчета х.ф. После определения х.ф.~ $g(\lambda;\Theta,t)$ п.в.\
определяется преобразованием Фурье~(\ref{e2.5-s}).

2.2. Рассмотрим метод ИАМ для СтС~(\ref{e2.1-s}).
Подставляя в~(\ref{e2.5-s}) выражение~(\ref{e2.11-s}), найдем:

\noindent
\begin{multline}
f(y;\Theta, t) \approx \sss_{l=-N}^N g_l \left(\lambda_l ; \Theta, t\right) 
    \beta_l (y)\,,
    \\
\beta_l = \fr{1}{2\pi} \iin e^{-i\lambda y} \alpha_l (\lambda)\, d\lambda\,,
%\label{e2.13-s}
\label{e2.12-s}
\end{multline}
где функции $\beta_l(y)$ являются преобразованием \mbox{Фурье} функций~$\alpha_l(\la)$ 
и,~следовательно, известны.
А~значит, искомая п.в.\ выражается через значения х.ф.\ в~точках отсчета. 
Полагая в~(\ref{e2.3-s}) $\la\hm=\la_r$ и~под\-став\-ляя в~него~(\ref{e2.12-s}), получаем:
    \begin{equation}
\fr{\prt g_r}{\prt t} = \dot g_r = \sss_{l=-N}^N \varepsilon_{lr} g_l\,.
\label{e2.14-s}
\end{equation}
Здесь $g_l = g_l (\lambda_r;\Theta, t)$, а~$\varepsilon_{lr}$ равны
\begin{equation}
\varepsilon_{lr}=\varepsilon_{lr}^0 +\varepsilon_{lr}'+\varepsilon_{lr}''\,,
\label{e2.15-s}
\end{equation}
где

\noindent
   $$
   \varepsilon_{lr}^0 = \varepsilon_{lr}^0\left(\lambda_r;\Theta, t\right)= 
   i \lambda_r \iin a(y, \Theta, t) \beta_l (y) e^{i\lambda_r y}\, dy\,;
   $$
   
   \pagebreak
   
\noindent
  \begin{multline*}
  \varepsilon_{lr}' = \varepsilon_{lr}'\left(\lambda_r;\Theta, t\right)={}\\
  {}=
  -\fr{ \lambda_r^2}{2}\,\nu_0 (\Theta,t)  \iin b^2(y, \Theta, t) \beta_l (y) 
  \exp(i\lambda_r y) \,dy\,;
 \end{multline*}
 
 \vspace*{-12pt}
 
 \noindent
\begin{multline}
  \varepsilon_{lr}'' = \varepsilon_{lr}''\left(\lambda_r;\Theta, t\right)=
  \iin \int\limits_{R_0^q} \left\{ \exp \lk i\la_r  c (y,\Theta, t,v)\rk-{}\right.\\
\left.  {}-1-
  i\lambda_r c (y,\Theta, t,v)\right\} \beta_l (y) \exp\lk i\lambda_r y\rk\,
   dvdy.\label{e2.16-s}
   \end{multline}


\noindent
\textbf{Замечание~2.1.}
Уравнение~(\ref{e2.14-s}) линейно, причем интегралы~(\ref{e2.16-s}) 
могут быть вычислены заранее, так как подынтегральные функции известны.

\smallskip

Следуя~[5], выполним с~(\ref{e2.14-s}) следующие преобразования. 
Перейдем от комплексных величин к~действительным, положив
\begin{equation}
\hspace*{-20.5mm}g_r = g_{r1} + i g_{r2}\,;\label{e2.17-s}
\end{equation}
\begin{equation}
\left.
\begin{array}{rl}
 \beta_l &= \beta_{l1} + i\beta_{l2}\,;\\[6pt]
    \beta_{l1}&=\displaystyle \fr{1}{2\pi} \iin \alpha_l (\lambda) 
    \cos \lambda y \,d\lambda\,;\\[6pt]
    \beta_{l2}&= \displaystyle
    \fr{1}{2\pi} \iin \alpha_l (\lambda) \sin \lambda y \,d\lambda\,;
    \end{array}
    \right\}
    \label{e2.18-s}
    \end{equation}
    \begin{equation*}
\hspace*{-18.5mm}\varepsilon_{lr} = \varepsilon_{lr1} + i \varepsilon_{lr2}\,,
\end{equation*}
где

\noindent
\begin{multline*}
\varepsilon_{lr1} =-\lambda_r\iin \left[ \beta_{l1} (y) \sin \lambda_r y +{}\right.\\
\left.{}+ \beta_{l2} (y) 
\cos \lambda_r y\right] a (y, t) \,dy\,;
\end{multline*}

\vspace*{-12pt}

\noindent
\begin{multline}
\varepsilon_{lr2} =-\lambda_r\iin \left[ \beta_{l1} (y) \cos \lambda_r y + {}\right.\\
\left.{}+\beta_{l2} (y)
 \sin \lambda_r y\right] a (y, t)\, dy\,.
 \label{e2.19-s}
 \end{multline}

После подстановки~(\ref{e2.17-s})--(\ref{e2.19-s}) в~(\ref{e2.15-s}) 
и~приравнивая действительные и мнимые части, получим систему линейных 
обыкновенных дифференциальных уравнений для значений х.ф.\ 
в~точках отсчета в~действительной форме:
\begin{equation}
\left.
\begin{array}{rl}
\dot g_{r1} &= G_{r1} = \displaystyle\sss_{l=-N}^N \left( \varepsilon_{lr1} g_{l1} - 
    \varepsilon_{lr2} g_{l2}\right)\,;\\[6pt]
    \dot g_{r2} &= G_{r2} = \displaystyle\sss_{l=-N}^N \left( \varepsilon_{lr2} g_{l1} + 
    \varepsilon_{lr2} g_{l2}\right)\,.
    \end{array}
    \right\}
    \label{e2.20-s}
    \end{equation}
Сократим число уравнений в~(\ref{e2.20-s}), учтя известные свойства х.ф.~[1]:
\begin{equation}
    g(0, \Theta, t) = 1\,; \enskip g(-\lambda; \Theta,t)= 
    \overline{g(\lambda;\Theta t)}\,,\label{e2.21-s}
    \end{equation}
где черта~--- символ комплексного сопряжения. На основании~(\ref{e2.21-s}) находим:
\begin{equation}
\left.
\begin{array}{rlrl}
g_{01} &=1\,;&\quad g_{02} &=0\,;\\[6pt]
 g_{-r,1} &= g_{r1}\,;&\quad 
    g_{-r,2} &= -g_{r2}\,.
    \end{array}
    \right\}
    \label{e2.22-s}
    \end{equation}
Отсюда следует, что~(\ref{e2.20-s}) надо решать лишь для~$g_{r1}$ и $g_{r2}$ при $r\hm>0$.

Наконец, выделим слагаемое, соответствующее $l\hm=0$, далее разобьем полученную 
сумму на две суммы и сменим порядок суммирования на $\overline{1,N}$. 
Тогда придем к следующим окончательным уравнениям:
    \begin{multline*}
    \dot g_{r1} = G_{r1} ={}\\
    {}= \sss_{h=1}^N \lk 
    \left( \varepsilon_{hr1} + \varepsilon_{-hr1}\right) g_{h1} -
    \left(\varepsilon_{hr2} -\varepsilon_{-hr2}\right) g_{h2} \rk\,;
\end{multline*}

\vspace*{-12pt}

\noindent
\begin{multline}
    \dot g_{r2} = G_{r2} = \sss_{h=1}^N \left[ \left( \varepsilon_{hr2} - \varepsilon_{-hr2}
    \right) g_{h1} +{}\right.\\
\left.    {}+\left(\varepsilon_{hr1} +\varepsilon_{-hr1}\right) g_{h2} 
    \right]\enskip 
    \left( r=\overline{1,N}\right)\,.\label{e2.23-s}
    \end{multline}

В случае стационарного распределения уравнения~(\ref{e2.23-s}) принимают вид:
\begin{equation}
G_{r1}^* =0\,;\quad G_{r2}^*=0\,.\label{e2.24-s}
\end{equation}

Выражение~(\ref{e2.5-s}) для п.в.\ после аналогичных преобразований с~х.ф.\
 принимает вид:
 
 \noindent
\begin{multline}
    f(y,\Theta, t) = \beta_{01}(y) + \sss_{h=1}^N \lk \beta_{h1} (y) +
     \beta_{-h1}(y)\rk g_{h1} -{}\\
     {}-\sss_{h=1}^N \lk\beta_{h2} (y) - 
     \beta_{-h2}(y)\rk g_{h2}\,.\label{e2.25-s}
     \end{multline}

В качестве начальных условий для~(\ref{e2.23-s}) следует принять следующие количества:
\begin{equation}
g_{rj} \left(\lambda_{r}; t_0\right) = g_{rj0} \enskip (j=1,2)\,.
\label{e2.26-s}
\end{equation}

Так как х.ф.\ является преобразованием Фурье для п.в., 
то теорема отсчетов Котельникова~[4] полностью определяется последовательностью 
ее значений в дискретном ряде точек. Поэтому, если распределение СтП~$Y_t$ 
сосредоточено на интервале  $2\Delta \hm= 2 \Delta_y$, то согласно~[4] 
$ \alpha_r \hm=\alpha_r(\la)$ и $\beta_r \hm=\beta_r (y)$ можно выбрать в виде:
\begin{equation}
   \alpha_r =\fr{\sin \Delta( \la-\la_r)}{\Delta (\la-\la_r)} \,,\quad 
   \lambda_r =\fr{\pi r}{\Delta}\,; \label{e2.27-s}
   \end{equation}
\begin{equation}
\beta_r = \begin{cases}
  \displaystyle \fr{1}{2\Delta} e^{-i\lambda_r y}& \mbox{ при} \ |y|\leq \Delta\,;\\
   0& \mbox{ при}\ |y|> \Delta\,,
   \end{cases}
   \label{e2.28-s}
   \end{equation}
при этом п.в.\ будет определяться формулой:

\pagebreak

\noindent
\begin{equation*}
f(y,\Theta, t)= \begin{cases}
\displaystyle   \fr{1}{\Delta}\sss_{r=1}^N (g_{r1}\cos \lambda_r y + 
g_{r2} \sin \lambda_r y) 
   &\\
   &\hspace*{-20mm}\mbox{ при } |y|\leq \Delta\,;\\
   0&\hspace*{-20mm}\mbox{ при } |y|> \Delta\,.
   \end{cases}\!\!\!\!
   %\label{e2.29-s}
   \end{equation*}

Чувствительность метода ИАМ к параметрам~$\Theta$ оценивается согласно~[5, 7] 
путем вычисления следующих первых векторных функций чувствительности:
 \begin{equation}
    \nabla^\Theta g_{rj} =\nabla^\Theta g_{rj}(\la_r;\Theta, t) \enskip ( j=1,2)\,,
    \label{e2.30-s}
    \end{equation}
где $\nabla^\Theta$~--- символ вектора частных производных по векторному
 параметру~$\Theta$. В~силу~(\ref{e2.30-s}) 
 соответствующие уравнения для функций чувствительности имеют вид:
\begin{equation}
\dot\nabla^\Theta g_{rj} =\nabla^\Theta G_{rj}\enskip ( j=1,2)\label{e2.31-s}
\end{equation}
при нулевых начальных условиях и
\begin{equation}
\nabla^\Theta G_{rj}^*=0 \enskip ( j=1,2)\label{e2.32-s}
\end{equation}
в стационарном случае.

Таким образом, приходим к следующему обобщению~[5].

\smallskip

\noindent
\textbf{Теорема~1.}\ \textit{Пусть СтС}~(\ref{e2.1-s}) 
\textit{допускает одномерное нестационарное распределение, х.ф.\
 которого удовлетворяет уравнению}~(\ref{e2.3-s}) 
 \textit{при начальном условии}~(\ref{e2.4-s}). 
 \textit{Тогда в основе метода ИАМ х.ф.\ по Котельникову лежат согласно}~(\ref{e2.27-s}) 
 \textit{и}~(\ref{e2.28-s}) \textit{линейные обыкновенные дифференциальные 
 уравнения}~(\ref{e2.23-s}) \textit{при начальных условиях}~(\ref{e2.26-s}). 
 \textit{Для моделирования п.в.\ применяется формула}~(\ref{e2.25-s}). 
 \textit{В~случае, когда имеет место стационарное одномерное распределение, 
 используются уравнения}~(\ref{e2.24-s}). \textit{При этом чувствительность МИАМ 
 оценивается согласно}~(\ref{e2.30-s})--(\ref{e2.32-s}).


\smallskip

2.3. Уравнения МИАМ для СтС~(\ref{e2.2-s}) имеют вид~(\ref{e2.23-s})--(\ref{e2.25-s}), 
если принять
\begin{equation}
\varepsilon_{lr} = \varepsilon_{lr}^0 + \varepsilon_{lr}^1\,,\label{e2.33-s}
\end{equation}
где
\begin{align}
\varepsilon_{lr}^0 &= i \lambda_r \iin a (y, \Theta,t) \beta_l (y) 
    e^{i\la_r y} \,dy\,; \notag %\label{e2.34-s}
    \\
\varepsilon_{lr}^1 &= \iin \chi\left( b(y,\Theta, t)\lambda;\Theta,t\right) \beta_l (y) 
    e^{i\lambda_r y}\, dy\,.\label{e2.35-s}
    \end{align}

В результате имеем следующий алгоритм метода ИАМ.

\smallskip

\noindent
\textbf{Теорема~2.}\ \textit{Пусть СтС}~(\ref{e2.2-s}) 
\textit{допускает одномерное нестационарное распределение, х.ф.\
которого удовлетворяет уравнению}~(\ref{e2.9-s}) \textit{при начальном условии}~(\ref{e2.4-s}).
\textit{Тогда в основе метода ИАМ х.ф.\ по Котельникову согласно}~(\ref{e2.27-s}) 
\textit{и}~(\ref{e2.28-s}) \textit{лежат линейные обыкновенные дифференциальные 
уравнения}~(\ref{e2.23-s})--(\ref{e2.25-s}) 
\textit{при условиях}~(\ref{e2.33-s})--(\ref{e2.35-s}). 
\textit{Для моделирования п.в.\ применяются формулы}~(\ref{e2.22-s}). 
\textit{В~стационарном случае используются 
уравнения}~(\ref{e2.24-s}). \textit{При этом чувствительность метода ИАМ оценивается 
согласно}~(\ref{e2.30-s})--(\ref{e2.32-s}).

\smallskip

2.4. Теперь рассмотрим метод ИАМ для векторных СтС.
Следуя~[2,~3], воспользуемся принципом эквивалентности векторной СтС скалярной СтС. 
В~задачах практики обычно не требуется определения х.ф.\ и~п.в.\
 всех составляющих вектора~$Y_t$,\linebreak а~достаточно определить одномерное распределение 
 некоторых из них. Поэтому будем искать распределения отдельных составляющих. 
 
 Для определения одномерных х.ф.\ и~п.в.\ некоторой $h$-й
  составляющей вектора~$Y_t$
 заменим исход\-ную многомерную СтС эквивалентной одномерной СтС, описываемой 
 стохастическим дифференциальным уравнением Ито (или уравнением 
 с~$\theta$-дифференци\-алом~[1]), выходная переменная~$\tilde Y_{ht}$ 
 которого имеет те же х.ф.\ и~п.в., что и~$h$-я со\-став\-ля\-ющая~$Y_{ht}$.
 

В результате придем к следующему утверж\-дению.

\smallskip

\noindent
\textbf{Теорема~3.}\
\textit{Пусть векторные CтС вида}~(\ref{e2.1-s}) \textit{и}~(\ref{e2.2-s}) 
\textit{приводятся к эквивалентной системе скалярных СтС в~форме Ито. Тогда  
в~основе метода ИАМ лежат уравнения теорем~$1$ и~$2$ для каждой скалярной СтС.}


\smallskip

\noindent
\textbf{Замечание~2.2.}
Выходные переменные~$Y_{ht}$ и~$\tilde Y_{ht}$ в~общем случае~(\ref{e2.2-s})~--- 
различные СтП, поскольку~$ \tilde Y_{ht}$ является марковским СтП,
а~$Y_{ht}$~--- со\-став\-ля\-ющая марковского процесса и,~следовательно, 
сам по себе может и~не быть марковским СтП. Однако одномерные распределения 
у~них одинаковы.

\smallskip

\noindent
\textbf{Замечание~2.3.}
Для квазилинейных СтС с аддитивным шумом замена многомерной системы одномерной 
допустима в случае асимптотической устойчивости линейной детерминированной части. 
Для комп\-лекс\-но-со\-пря\-жен\-ных корней эквивалентная система будет уже двумерной. 
Случаи кратных корней требуют специального рассмотрения.


\section{Метод интерполяционного 
аналитического моделирования для~дискретных стохастических систем}

Следуя~[1], рассмотрим сначала нелинейные дискретные СтС, описываемые 
разностными стохастическими уравнениями вида:

\noindent
\begin{align}
Y_{k+1} &=\omega_k \left(Y_k, \Theta, V_k^d\right) \enskip (k=1,2,\ldots)\,;\label{e3.1-s}
\\
Y_{k+1} &=a_k \left(Y_k, \Theta\right) +b_k \left(Y_k, \Theta\right)V_k^d \notag\\ 
&\hspace*{29mm}(k=1,2,\ldots)\,,\label{e3.2-s}
\end{align}
где $\omega_k$, $a_k$ и~$b_k$~--- скалярные функции. Характеристические функции 
$h_k\hm=h_k (\rho, \Theta)$ векторных случайных величин~$V_k^d$ и~соответствующие им
 п.в.\ $\eta_k \hm= \eta_k(v,\Theta)$ будем считать известными.

Для одномерных п.в.\ и~х.ф., определяемых известными формулами
\begin{align*}
f_k (y) &=\displaystyle\fr{1}{2\pi} \iin 
e^{-i\lambda y} g_k (\la) \,d\lambda\,; \\
g_k(\lambda) &=      {\sf M} \exp \lf i \la Y_k\rf
%\label{e3.3-s}
\end{align*}
применительно к уравнению~(\ref{e3.1-s}) имеют место следующие рекуррентные формулы~[1]:
\begin{multline}
g_{k+1} (\la) = {\sf M} \exp \lf i\la \omega_k (Y_k, \Theta, V_k) \rf = {}\\
{}=
    \iin \iin e^{i\lambda \omega_k (y, v,\Theta)} 
    f_k (y) h_k ( v,\Theta)\, dy dv\,.\label{e3.4-s}
    \end{multline}

Общая рекуррентная формула для одномерного распределения в~(\ref{e3.2-s}) 
немедленно получается из соответствующей формулы~(\ref{e3.4-s})
и~приводит к следующему результату:
 \begin{multline}
 g_{k+1} ={\sf M} \exp \lf i\lambda a_k (Y_k, \Theta) + i\lambda b_k 
    \left(Y_k,\Theta\right) V_k \rf={}\\
{}=\! \iin\iin e^{i\la a_k(y,\Theta) + i\lambda b_k(y,\Theta) v} 
    f_k (y) h_k (v,\Theta)\, dydv={}\\
{}= {\sf M}\lk \exp \lf i \lambda a_k \left(Y_k,\Theta\right) \rf + 
    h_k \left( b_k \left(Y_k,\Theta\right) \lambda
    \right)\rk.\label{e3.5-s}
    \end{multline}

Для СтС~(\ref{e3.2-s}) метод ИАМ приводит к следующим рекуррентным формулам:
 \begin{align}
 g_{r1, k+1} &= G_{r1,k}^d = \!\!\sss_{h=1}^N \!\left[ 
    \left(\varepsilon_{hr1,k}+\varepsilon_{-hr1,k}\right)g_{h1,k}-{}\right.\hspace*{-3.24097pt}\notag\\
&\left.    \hspace*{5mm}{}-
    \left(\varepsilon_{hr2,k}-\varepsilon_{-hr2,k}\right)g_{h2,k}\right]\,;
    \label{e3.6-s}
    \\
g_{r2, k+1} &= G_{r2,k}^d = \!\!\sss_{h=1}^N \!\left[ 
    \left(\varepsilon_{hr2,k}-\varepsilon_{-hr2,k}\right)g_{h1,k}+{}\right.\hspace*{-3.24097pt}\notag\hspace*{-3.24097pt}\\
&\left.    \hspace*{5mm}{}+\left(\varepsilon_{hr1,k}+\varepsilon_{-hr1,k}\right)g_{h2,k}\right]\,;
    \label{e3.7-s}
    \\
    g_{rj} (\lambda_r;k=1) &= g_{rj,1}\enskip (j=1,2)\,;\label{e3.8-s}
    \\
    \varepsilon_{lr,k} &= \varepsilon_{lr,k}^0+\varepsilon_{lr,k}^1\,,\notag
    %\label{e3.9-s}
    \end{align}
    где
\begin{align*}
    \varepsilon_{lr,k}^0&=i\lambda_r \iin a_k 
    \left(y_k,\Theta\right) \beta_l (y_k) \exp (i\la_r y_k) \,dy_k\,;
    %\label{e3.10-s}
\\
    \varepsilon_{lr,k}^1&=\iin \exp\lf \exp\left(i\lambda a_k\right)\rf+
    h_k \left(b_k \lambda\right)\, dy_k\,;
    %\label{e3.11-s}
    \end{align*}
    
   
    \noindent
\begin{align}
    f_k (y;\Theta, k) &=\begin{cases}
    \fr{1}{\Delta} \sss_{r=1}^N g_{r1,k}\cos \lambda_t y+{}\\[3pt]
 \hspace*{3mm}{}+g_{r1,k} 
    \sin \lambda_r y &\hspace*{-10mm}\mbox{ при } |y|\leq\Delta\,;\\[6pt]
    0&\hspace*{-10mm}\mbox{ при } |y|>\Delta\,;
    \end{cases}
    \label{e3.12-s}
    \\[3pt]
 \Delta^\Theta g_{rj,k+1} &=\nabla^\Theta G_{rj,k}^d\,.\label{e3.13-s}
 \end{align}
Здесь аргументы для краткости опущены.

Таким образом, приходим к~сле\-ду\-юще\-му утверж\-де\-нию.

\smallskip

\noindent
\textbf{Теорема~4.}\ \textit{
Пусть дискретная СтС}~(\ref{e3.2-s}) 
\textit{допускает одномерное нестационарное распределение, х.ф.\
которого удовлетворяет разностному уравнению}~(\ref{e3.5-s}) 
\textit{при начальном условии}~(\ref{e3.8-s}). 
\textit{Тогда в основе метода ИАМ х.ф.\ по Котельникову согласно}~(\ref{e2.27-s}) 
\textit{и}~(\ref{e2.28-s}) \textit{лежат рекуррентные формулы}~(\ref{e3.6-s}) 
\textit{и}~(\ref{e3.7-s}) \textit{при начальных условиях}~(\ref{e3.8-s}). 
\textit{Для моделирования п.в.\ применяется формула}~(\ref{e3.12-s}). 
\textit{В~стационарном случае используются}~(\ref{e3.6-s}) 
\textit{и}~(\ref{e3.7-s}), \textit{где $g_{rj,k+1}\hm= g_{rj,k}$ $(j=1,2)$. 
При этом чувствительность МИАМ оценивается согласно}~(\ref{e3.13-s}).

\smallskip


\noindent
\textbf{Замечание~3.1.}
Общий случай СтС~(\ref{e3.1-s}) поддается разработке только при дополнительных 
ограничениях на функции~ $\omega_k$.

\smallskip


\noindent
\textbf{Замечание~3.2.}
До сих пор СтС~(\ref{e3.1-s}) и~(\ref{e3.2-s}) предполагались скалярными. 
В~векторном случае можно непосредственно воспользоваться результатами п.~2.4.

\vspace*{-8pt}

\section{Практические способы выбора интерполяционных формул и~чисел отсчетов}

\vspace*{-2pt}


4.1. В задачах аналитического моделирования СтП в непрерывных и дискретных 
СтС выбор формул интерполяции х.ф.\ в~(\ref{e2.11-s}) 
определяется интервалом значений~$Y_t$, в~котором сосредоточена его~п.в.

Наряду с теоремой Котельникова~[4] для х.ф.\ 
различают следующие интерполяционные функции~[8]:
\begin{enumerate}[(1)]
\item интерполяционная формула Лагранжа;

\item разделенные разности и интерполяционная формула Ньютона;

\item итерационно-ин\-тер\-по\-ля\-ци\-он\-ная формула Эйткена.
\end{enumerate}

Выбор функций  $\alpha_r (\la)$  и~$\beta_l (y)$ посредством этих
интерполяционных формул встречает сле\-ду\-ющие
 трудности: $\lambda_{\max}$ 
(значение аргумента, опре\-де\-ля\-ющее об\-ласть, вне которой х.ф.\
 практически мож\-но
 считать рав\-ной нулю) заранее определить труд\-но.
  Кроме того, формулы для $\beta_r\hm=\beta_r(y)$ при этом оказываются 
 аналитически сложными. %\linebreak\vspace*{-12pt}

\pagebreak

 

Для сложных СтС в случае равностоящих значений аргумента можно 
рекомендовать интерполяционные формулы для х.ф.\
 с~центральными разностями различного порядка, восходящие к Стирлингу, Бесселю, 
 Эверетту и Стефенсону~[8].

4.2. Необходимое приемлемое число~$N$ членов ряда, обеспечивающее 
достаточную точность расчетов~$R_N$ в~(\ref{e2.11-s}) существенно зависит от 
удачного определения интервала~$2\Delta$. Если выбранный 
интервал значительно больше, чем область фактического распределения СтП~$Y_t$, 
то может потребоваться значительное число членов ряда.

В~[2, 3] показано, что для гауссовского шума~$V$ достаточно точные 
результаты получаются уже при $N\hm=3$, а~для большинства негауссовских п.в.~--- 
при выборе $\Delta\hm= (3\ldots5) \sqrt{\tilde D}$, где $\tilde D$~--- 
приближенное значение дисперсии СтП~$Y_t$.

В том случае, когда заранее определить интервал~$\Delta$ нет возможности, то  
чтобы обеспечить в~каж\-дый момент времени разложение п.в.~$Y_t$ на интервале $(3\ldots5) 
\sqrt{\tilde D}$ можно в исходных уравнениях~(\ref{e2.1-s}) и~(\ref{e2.2-s}) 
сделать линейную замену переменных:
\begin{equation*}
Y_t = Z_t \sqrt{c_2} + c_1\,.
%\label{e4.1-s}
\end{equation*}
Здесь $c_1$ и~$c_2$~--- приближенные значения математического ожидания  
$c_1={\sf M}\, Y_t$ и~дисперсии  $c_2 \hm={\sf D}\, Y_t$, 
определяемые приближенно, например\linebreak
 методами параметризации распределений 
(нормальной аппроксимации, моментов, семиинвариантов, ортогональных разложение 
и~др.~[1]). При этом переменная~$Z_t$ имеет близкое к~нулю математическое ожидание 
$c_1\hm=m_z\hm\approx 0$ и~близкую к~единице дисперсию $c_2 \hm=D_z\hm\approx 1$. 
После такой замены уравнение~(\ref{e2.1-s}) 
примет соответственно вид:
\begin{multline}
   \!\! \!\!dZ_t = a^z \left(Z_t,c_y,\Theta, t\right) dt + 
    b^z \left(Z_t, c_y, \Theta, t\right) d W_0 (\Theta, t)\;+{}\\
    {}+
    \int\limits_{R_0^q} c^z\left(Z_t, c_y, \Theta, t, v\right) 
    P^0 (\Theta, dt, dv)\,,\label{e4.2-s}
    \end{multline}
где $c_y \hm= \{c_j\}$~--- вектор параметров ортогонального разложения п.в.\
 СтП~$Y_t$;
 \begin{equation}
 \left.
 \begin{array}{rl}
    a^z &= a^z( Z_t, c_y, \Theta, t) = {}\\[6pt]
    &\hspace*{-5mm}{}=
    \fr{1}{\sqrt{c_2}} a \left( Z_t \sqrt{c_2} + c_1 ,\Theta, t\right) - 
    Z_t \fr{\dot c_2}{2 c_2} - \fr{\dot c_1}{\sqrt{c_2}}\,;
\\[9pt]
    b^z &= b^z\left( Z_t\sqrt{c_2}+c_1,  \Theta, t\right) ={}\\[6pt]
    &\hspace*{19mm}{}= 
\fr{1}{\sqrt{c_2}}\, b \left( Z_t \sqrt{c_2} + c_1 ,\Theta, t\right);
    \\[6pt]
    c^z&= c^z\left( Z_t \sqrt{c_2}+ c_1, \Theta, t, v\right)\,. 
    \end{array}
    \right\}
    \label{e4.3-s}
   \end{equation}
   
  % \vspace*{-12pt}
   
   %\noindent
  % \begin{multline*}
  %  a^z=a^z\left( Z_t, c_y, \Theta, t\right)={}\\
  %  {}= 
  %  \fr{1}{\sqrt{c_2}}\, a \left( Z_t \sqrt{c_2} + c_1 ,\Theta, t\right) - 
  %  Z_t \fr{\dot c_2}{2 c_2} - \fr{\dot c_1}{\sqrt{c_2}}\,,
%\end{multline*}
%\begin{equation}
 %   b^z = b^z\left( Z_t,c_y,  \Theta, t\right) = 
  %  \fr{1}{\sqrt{c_2}}\, b \left( Z_t \sqrt{c_2} + c_1 ,\Theta, t\right)\,.
  %  \label{e4.5-s}
  %  \end{equation}

При применении приближенного метода нормальной аппроксимации (МНА)~[1], 
когда $c^y\hm=\{c_1, c_2\}$ в силу~(\ref{e4.2-s}) и~(\ref{e4.3-s}) 
используются сле\-ду\-ющие уравнения для~$c_1$ и~$c_2$:

\noindent
  \begin{equation}
  \dot c_1 = A^{c_1} (c_1, c_2,\Theta, t);\enskip 
    \dot c_2 = A^{c_2} (c_1, c_2,\Theta, t),\label{e4.6-s}
    \end{equation}
где $A^{c_1}$ и~$A^{c_2}$ известные функции отмеченных функций.

4.3. При вычислении п.в.\ переменной $Z\hm=Z_t$ согласно теоремам~1 и~2 
интервал разложения можно принять равным $3\ldots5$. 
Пользуясь известной формулой вычисления линейной фукнции случайного аргумента~[1], 
получим искомую п.в.:
\begin{equation*}
f(y,\Theta)= \fr{1}{\sqrt{c_2}}\,
     f_z \left( \fr{y-c_1}{\sqrt{c_2}} \right)\,,
     %\label{e4.7-s}
     \end{equation*}
где $f_z$~--- п.в.\ переменной~$Z_t$. При этом~$c_1$ 
и~$c_2$ определяются из~(\ref{e4.6-s}).



%\smallskip

\noindent
\textbf{Замечание~4.1.}
В случае СтС~(\ref{e2.1-s}), когда применяется нелинейная замена переменных  
$z\hm= \varphi
 (Y_t, t)$, в качестве параметров~$c^y$ выбираются коэффициенты 
ортогонального разложения одномерной п.в.; уравнения для~$Z_t$ составляются 
путем применения обобщенной формулы Ито~[1] для СтС~(\ref{e2.1-s}).


%\smallskip

\noindent
\textbf{Замечание~4.2.}
Для дифференциальных СтС описанные практические способы выбора интерполяционных формул 
и~чисел отсчетов сохраняют силу.

\vspace*{-12pt}


\section{Заключение}

\vspace*{-2pt}


\noindent
\begin{enumerate}[5.1.]
\item Для сложных непрерывных и дискретных скалярных и приводимых к ним векторным 
СтС разработаны методы ИАМ. Рассмотрены типовые классы сложных СтС. Разработанные методы 
основаны на интерполяционном решении эволюционного уравнения для одномерной 
х.ф. Получены уравнения для оценки чувствительности х.ф.\ к параметрам СтС.
\item
 В основу базовых алгоритмов ИАМ (теоремы~1--4) положены 
теорема отсчетов Котельникова. Рассмотрены практические вопросы выбора 
интерполяционных формул Котельникова и оценки количества интерполяционных отсчетов.
\item
 Для СтС с известной аналитической природой в качестве 
дальнейшего развития следует рассмотреть применение сплайн-вейв\-лет методов
 для офлайн алгоритмов, а~для онлайн алгоритмов фильтрационных подходов~[1]. 
 При этом особое внимание следует уделить многомерным распределениям.
\end{enumerate}

\vspace*{-12pt}

{\small\frenchspacing
 {%\baselineskip=10.8pt
 \addcontentsline{toc}{section}{References}
 \begin{thebibliography}{9}
 
 \vspace*{-2pt}


\bibitem{1-sin}
\Au{Пугачёв В.\,С., Синицын И.\,Н.}
Теория стохастических систем.~--- М.: Логос, 2000; 2004. 1000~с.
%[Англ. пер. Stochastic Systems. Theory and  Applications. --
%Singapore: World Scientific, 2001. 908~p.].

\bibitem{2-sin}
\Au{Мальчиков С.\,В.}
Приближенный метод определения законов распределения фазовых координат 
нелинейных\linebreak\vspace*{-12pt}

\pagebreak

\noindent
 автоматических систем~// Автоматика и~телемеханика, 1970. №\,5. С.~43--50.

\bibitem{3-sin}
\Au{Казаков И.\,Е., Мальчиков С.\,В.}
Анализ стохастических систем в пространстве состояний.~--- М.: Наука, 1983. 348~с.

\bibitem{4-sin}
\Au{Котельников В.\,А.}
Теория потенциальной помехоустойчивости.~--- М.: Госэнергоиздат, 1956. 412~с.

\bibitem{5-sin}
\Au{Синицын И.\,Н.}
Метод интерполяционного аналитического моделирования одномерных распределений 
в~стохастических системах~// Информатика и её применения, 
2018. Т.~12. Вып.~1. С. 56--62.

\columnbreak

\bibitem{6-sin}
Справочник по теории вероятностей и~математической статистике~/ 
Под ред.\ В.\,С.~Королюка, Н.\,И.~Портенко, А.\,В.~Скорохода, А.\,Ф.~Турбина.~--- 
М.: Наука, 1985. 640~с.



\bibitem{7-sin}
\Au{Евланов Л.\,Г., Константинов~В.\,М.}
Системы со случайными параметрами.~--- М.:
Наука,  1976.  585~с.


\bibitem{8-sin}
\Au{Корн Г., Корн~Т.}
Справочник по  математике (для научных работников и~инженеров)~/
Пер с~англ.~--- 
М.: Наука, 1974. 832~с.
(\Au{Korn~G., Korn~T.} 
Mathematical handbook for scientists and engineers.~--- 
New York\,--\,San Francisco\,--\,Toronto\,--\,London\,--\,Sydney:
McGraw Hill Book Co., 1968. 1147~p.)
 \end{thebibliography}

 }
 }

\end{multicols}

\vspace*{-9pt}

\hfill{\small\textit{Поступила в~редакцию 16.08.18}}

\vspace*{5pt}

%\pagebreak

%\newpage

%\vspace*{-28pt}

\hrule

\vspace*{2pt}

\hrule

%\vspace*{-2pt}

\def\tit{INTERPOLATONAL ANALYTICAL MODELING\\
 IN~COMPLEX STOCHASTIC SYSTEMS}

\def\titkol{Interpolatonal analytical modeling in complex stochastic systems}

\def\aut{I.\,N.~Sinitsyn}

\def\autkol{I.\,N.~Sinitsyn}

\titel{\tit}{\aut}{\autkol}{\titkol}

\vspace*{-11pt}


\noindent
Institute of Informatics Problems, Federal Research Center 
``Computer Science and Control'' of the 
Russian Academy of Sciences, 44-2~Vavilov Str., Moscow 119333, Russian Federation

\def\leftfootline{\small{\textbf{\thepage}
\hfill INFORMATIKA I EE PRIMENENIYA~--- INFORMATICS AND
APPLICATIONS\ \ \ 2019\ \ \ volume~13\ \ \ issue\ 1}
}%
 \def\rightfootline{\small{INFORMATIKA I EE PRIMENENIYA~---
INFORMATICS AND APPLICATIONS\ \ \ 2019\ \ \ volume~13\ \ \ issue\ 1
\hfill \textbf{\thepage}}}

\vspace*{6pt}


\Abste{Methods and algorithms for interpolational analytical modeling 
(IAM) of stochastic processes in complex continuous and discrete stochastic systems 
(StS) which are scalar (or vector StS reducible to scalar) are developed. 
Several types of continuous and discrete StS are considered. The IAM methods 
are based on evolutionary equations numerical interpolation solution 
for one-dimensional characteristic function (c.f.). Special attention is paid to 
c.f.\ sensitivity analysis. In basic IAM algorithms,  
the Kotel'nikov theorem was implemented. Some practical questions concerning 
interpolational formulae and number of intervals for interpolation are 
discussed. In the future, IAM for StS with known analytical structure 
for offline modeling will be based on spline-wavelet methods and for 
online modeling~--- on filtration approaches. 
Special attention should be paid to multidimensional distributions.}


\KWE{one-dimensional characteristic functions (c.f.);
one-dimensional probability density (p.d.);
stochastic processes (StP);
stochastic system (StS)}

\DOI{10.14357/19922264190101}

%\vspace*{-14pt}

%\Ack
%\noindent




\vspace*{-6pt}

  \begin{multicols}{2}

\renewcommand{\bibname}{\protect\rmfamily References}
%\renewcommand{\bibname}{\large\protect\rm References}

{\small\frenchspacing
 {%\baselineskip=10.8pt
 \addcontentsline{toc}{section}{References}
 \begin{thebibliography}{9}

\bibitem{1-s-1}
\Aue{Pugachev, V.\,S., and I.\,N.~Sinitsyn.} 2000, 2004.
Teoriya stokhasticheskikh sistem [Stochastic systems: Theory and  applications]. 
Moscow: Logos. 1000~p.  %[Angl. per. . -- Singapore: World Scientific, 2001].

\bibitem{2-s-1}
\Aue{Mal'chikov, S.\,V.} 1970.
Priblizhennyy metod opredeleniya zakonov raspredeleniya fazovykh koordinat 
nelineynykh avtomaticheskikh sistem [Approximate method for phase coordinates 
analysis in nonlinear control systems]. 
\textit{Automat. Rem. Contr.}
5:43--50.

\bibitem{3-s-1}
\Aue{Kazakov, I.\,E., and S.\,V.~Mal'chikov.} 1983.
\textit{Analiz sto\-kha\-sti\-che\-skikh sistem v~prostranstve sostoyaniy} [Stochastic 
systems analysis in state space]. Moscow: Nauka. 348~p.

\bibitem{4-s-1}
\Aue{Kotel'nikov, V.\,A. } 1956.
\textit{Teoriya potentsial'noy po\-me\-kho\-us\-toy\-chi\-vosti} [Theory of potential noiseproof]. 
Moscow: Gosenergoizdat. 412~p.

\bibitem{5-s-1}
\Aue{Sinitsyn, I.\,N.} 2018. Metod interpolyatsionnogo ana\-li\-ti\-che\-sko\-go 
modelirovaniya odnomernykh raspredeleniy v~stokhasticheskikh sistemakh 
[Method of interpolational analytical modeling of processes in stochastic systems]. 
\textit{Informatika i~ee Primeneniya~--- Inform. Appl.} 12(1):56--62.

\bibitem{6-s-1}
Korolyuk, V.\,S., N.\,I.~Portenko, A.\,V.~Skorokhod, and A.\,F.~Turbin, eds. 1985.
\textit{Spravochnik po teorii veroyatnostey v~matematicheskoy statistike}
[Handbook on probability theory and mathematical statistics]. 
 Moscow: Nauka. 640~p.
 
\bibitem{7-s-1} 
\Aue{Evlanov, L.\,G., and V.\,M.~Konstantinov.} 1976.
\textit{Sistemy so sluchaynymi parametrami} [Systems with random parameters]. Moscow:
Nauka.  585~p.

\bibitem{8-s-1}
\Aue{Korn, G., and T.~Korn.} 1968.
\textit{Mathematical handbook for scientists and engineers}. 
New York\,--\,San Francisco\,--\,Toronto\,--\,London\,--\,Sydney:
McGraw Hill Book Co. 1147~p.
\end{thebibliography}

 }
 }

\end{multicols}

\vspace*{-6pt}

\hfill{\small\textit{Received August 16, 2018}}

%\pagebreak

\vspace*{-24pt}



\Contrl

\noindent
\textbf{Sinitsyn Igor N.} (b.\ 1940)~--- 
Doctor of Science in technology, professor, Honored scientist of RF, 
principal scientist, Institute of Informatics Problems, 
Federal Research Center ``Computer Science and Control'' 
of the Russian Academy of Sciences, 44-2~Vavilov Str., Moscow 119333, 
Russian Federation; \mbox{sinitsin@dol.ru}
\label{end\stat}

\renewcommand{\bibname}{\protect\rm Литература}       