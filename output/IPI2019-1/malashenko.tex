\def \SS{{\frak S}}
\def \SK{{\frak K}}
\def \SL{{\frak L}}



\def\stat{malashenko}

\def\tit{АНАЛИЗ УЯЗВИМОСТИ  МНОГОПОЛЮСНЫХ СЕТЕЙ  ПРИ~СТРУКТУРНЫХ  ПОВРЕЖДЕНИЯХ}

\def\titkol{Анализ уязвимости  многополюсных сетей  при структурных  повреждениях}

\def\aut{Ю.\,Е.~Малашенко$^1$, И.\,А.~Назарова$^2$, Н.\,М.~Новикова$^3$}

\def\autkol{Ю.\,Е.~Малашенко, И.\,А.~Назарова, Н.\,М.~Новикова}

\titel{\tit}{\aut}{\autkol}{\titkol}

\index{Малашенко Ю.\,Е.}
\index{Назарова И.\,А.}
\index{Новикова Н.\,М.}
\index{Malashenko Yu.\,E.}
\index{Nazarova I.\,A.}
\index{Novikova N.\,M.}




%{\renewcommand{\thefootnote}{\fnsymbol{footnote}} \footnotetext[1]
%{Работа поддержана РФФИ (проект 18-07-00274).}}


\renewcommand{\thefootnote}{\arabic{footnote}}
\footnotetext[1]{Федеральный исследовательский центр 
<<Информатика и~управ\-ле\-ние>> Российской академии наук, 
\mbox{malash09@ccas.ru}}
\footnotetext[2]{Федеральный исследовательский центр 
<<Информатика и~управ\-ле\-ние>> Российской академии наук, 
\mbox{irina-nazar@yandex.ru}}
\footnotetext[3]{Федеральный исследовательский центр 
<<Информатика и~управ\-ле\-ние>> Российской академии наук, \mbox{n\_novikova@umail.ru}}

%\vspace*{8pt}



\Abst{Предложен метод получения информативных оценок изменений функциональных   
возможностей  многостоковой   сетевой системы после потенциальной аварии. 
В~рамках  формализма  модели  передачи однопродуктового потока изучается 
множество достижимых век\-то\-ров-ре\-ше\-ний, удовлетворяющих стандартным условиям 
сохранения и~ограничениям на потоки по дугам. Для анализа  изначального состояния 
сис\-те\-мы для каждой стоковой вершины, отдельно и~независимо от остальных, 
вычисляется максимальный поток. Соответствующий минимальный разрез отделяет 
эту стоковую вершину от источника. Все дуги  найден\-но\-го минимального   
разреза  модельно   удаляются  и~в~поврежденной таким образом сети  
оцениваются возможности передачи потоков в~другие стоковые вершины~---   
рассчитываются    пре\-дель\-но-до\-пус\-ти\-мые  
для вершины потоки, которые сравниваются с~их изначальными значениями. Оценки   
ущерба проводятся для различных минимальных разрезов. 
Определяется влияние таких структурных повреждений на  величины потоков 
для  всех стоковых вершин и~строятся агрегированные характеристики 
подверженности вершины влиянию структурных повреждений.}

\KW{структурная уязвимость  сети; подверженность влиянию критических повреждений; 
многополюсная потоковая модель}

\DOI{10.14357/19922264190105}
  
\vspace*{-4pt}


\vskip 10pt plus 9pt minus 6pt

\thispagestyle{headings}

\begin{multicols}{2}

\label{st\stat}


\section{Введение}

В~\cite{MalInf17}  на примере    многопользовательской  сетевой
модели  изучались возможности   передачи всем и~каждому из  пользователей   
запрашиваемых  вели\-чин  потоков при повреждениях  элементов исходной сетевой системы.  
Последовательно определялись   гарантированные оценки   величин  потоков, 
передаваемых в~сети после  аварии.   В~\cite{MalInf181} был предложен метод   
определения  векторов  ущерба  для   пользователей сети   при  выходе из строя 
некоторых  дуг.    Предлагалось  одновременно оценивать как  уязвимость,  
так и~живучесть  сети.  Вычислялись    величины потоков,   которые    
можно гарантированно  доставить пользователям после аварии.

Данное исследование продолжает изучение уязвимости сети, начатое на основе 
общей методологии~\cite{Germ} в~работе~\cite{MalInf183}. Рассматриваются 
предельные функциональные  возможности сети и~их изменения при повреждениях. 
В~начальный момент  для каждой стоковой вершины   вычисляется  предельно 
допустимый поток  и~определяется минимальный разрез~\cite{Yen},  отделяющий  
соответствующую  вершину  от сети. После   удаления из сети всех дуг  найден\-но\-го   
минимального   разреза  вновь   вычисляются   максимальные     потоки для всех    
стоковых вершин и~сравниваются с~их исходными  значениями. Для каждого из 
полученных разрезов (отделяющих одну из стоковых вершин) подсчитываются величины 
ущерба для всех остальных потребителей от подобного повреждения сети и~проводится 
оценка его влияния на данного потребителя.  

По итогам сформированного множества 
оценок для каждого потребителя строятся характеристики его подверженности 
влиянию структурных повреждений рассматриваемого класса  (далее~--- критических).

В работе заложена возможность учета различных критических структурных повреждений и~их влияния на функционирование сети. Полученные характеристики позволяют    
анализировать  уязвимость  стоковых  вершин по отношению  к~удалению      
дуг   минимальных    разрезов содержательно разных  типов.  При этом в~настоящей 
работе (в~отличие от~\cite{Mal186}) предлагается совместно рассматривать 
минимальные разрезы сразу двух типов: по числу входящих в~разрез дуг и~по сумме 
значений их пропускной способности.

Работа построена следующим образом. В~разд.~2 даются основные определения и~вводятся 
обозначения. В~разд.~3 строится набор учи\-ты\-ва\-емых критических структурных повреждений 
сети. В~разд.~4 производится расчет характеристик подвер\-жен\-ности стоковых вершин  
сети их влиянию (на случай возникновения). 
В~заключении работы обсуждается место проведенных исследований в~группе 
пуб\-ли\-ка\-ций по близкой тематике. 

\section{Многополюсная потоковая модель}

Сеть передачи единственного вида продукта в~многополюсной системе 
будем описывать ориентированным графом $\overline {\cal G} \hm= 
\langle \overline{\cal V}, \overline{\cal L} \rangle$ без петель, 
который определяется множествами вершин (узлов) 
$\overline{\cal V} \hm= \left\{v_1,v_2,\ldots,v_ N\right\}$,  
$|\overline{\cal V}| \hm= N$,
и направленных дуг
$\overline {\cal L} \hm=\{l_{ij} \ | \ i \hm\in {\cal N}, \ j \hm\in {\cal N}, 
i \not = j \}$,
соединяющих вершины, где ${\cal N}$~--- множество индексов вершин;
$l_{ij} \hm= (v_i, v_j)$~---  дуга, ведущая из вершины~$v_i$ в~вершину~$v_j$; 
$|\overline{\cal L}| \hm= L$. Здесь и~далее модуль в~применении 
к~множеству показывает его мощ\-ность (чис\-ло элементов множества).

Обозначим через ${\cal V}_\SS$ и~${\cal V}_\SK$  
множества вершин графа $\overline {\cal G}$, 
являющихся соответственно источниками и~стоками для потока, 
который передается по многополюсной сети;
${\cal N}_\SS$ и~${\cal N}_\SK$~---  множества индексов вер\-шин-ис\-точ\-ни\-ков 
и~вер\-шин-сто\-ков:
\begin{multline*}
{\cal V}_\SS = \{v_i | \ i \in {\cal N}_\SS \},\enskip  
|{\cal V}_\SS| = S,\enskip {\cal V}_\SS \subset \overline{{\cal V}}\,,\\  
{\cal N}_\SS \subset {\cal N},\enskip S \geq 1\,;
\end{multline*}

\vspace*{-14pt}

\noindent
\begin{multline*}
{\cal V}_\SK = \{v_i | \ i \in {\cal N}_\SK\},\enskip  
|{\cal V}_\SK | = K,\enskip {\cal V}_\SK \subset \overline{{\cal V}}\,,
\\
{\cal N}_\SK \subset {\cal N},\ K \geq 1\,;
\end{multline*}

\vspace*{-9pt}

\noindent
\begin{equation*}
{\cal V}_\SS \bigcap {\cal V}_\SK = \emptyset\,.
\end{equation*}
Считается, что на дугах графа~$\overline {\cal G}$ заданы веса~--- значения~$d_{ij}$ 
пропускной способности дуг~$l_{ij}$, определяющие максимально допустимую величину 
потока  по дуге,
$ d_{ij} \hm\geq 0 \ \forall l_{ij}\hm\in \overline{\cal L}$.
К~графу $\overline {\cal G}$ добавим:
\begin{description}
\item 

$v_0$~---  фиктивный источник потока бесконечной мощности~--- 
и~фиктивные дуги $(v_0, v_j)$,  $j \in {\cal N}_\SS$, 
соединяющие~$v_0$ с~исходными уз\-ла\-ми-ис\-точ\-никами. 
Для каждой дуги~$l_{0j}$ формально определим верхнее ограничение~$d_{0j}$, 
которое соответствует величине максимального потока из уз\-ла-ис\-точ\-ни\-ка~$v_j$ 
в~сис\-те\-му. Будем называть  дуги $(v_0, v_j)$, $j \hm\in {\cal N}_\SS$,  
ду\-га\-ми-ис\-точ\-ни\-ка\-ми и~обозначим их множество как
$ \hat{{\cal L}}\hm=\{l_{0 j} \ | \  j \in {\cal N}_\SS \}$,  $d_{0j} \hm\geq 0 \ 
\forall l_{0 j} \hm\in \hat{{\cal L}}; $
\item 
$v_{N+1}$~--- единственный фиктивный узел-сток бесконечного объема~--- 
и~фиктивные дуги $(v_i, v_{N+1})$, $i \hm\in {\cal N}_\SK$, соединяющие 
каждый узел-сток с~$v_{N+1}$. Назовем направленные дуги $(v_i, v_{N+1})$, 
$i \hm\in {\cal N}_\SK$,  ду\-га\-ми-сто\-ка\-ми, или стоковыми дугами, 
и~обозначим их множество через
$\tilde{{\cal L}}\hm=\{\ l_{i(N + 1)} \ | \ i \hm \in {\cal N}_\SK\}$, 
$|\tilde{{\cal L}}| \hm=  K.$
Перенумеруем стоковые дуги и~стоковые вершины натуральными числами от~1 до~$K$ 
и~установим взаимно однозначное  соответствие $l_{k} \hm= l _{k (N+1)}$, 
$k \hm= \overline{1, K}$, $k \hm\in {\cal N}_\SK$.
Пусть $y_{k}$~--- величина потока по стоковой дуге~$l_{k}$ из стоковой вершины~$v_k$ 
в~вершину~$v_{N+1}$.
\end{description}

Ориентированный граф, который определяется множествами вершин 
${\cal V} \hm= \overline {\cal V} \bigcup \{v_0, v_{N}\}$ и~дуг 
${\cal L}\hm = \hat{{\cal L}} \bigcup \overline {{\cal L}}$, обозначим 
${\cal G} \hm= \langle {{\cal V}}, {{\cal L}} \rangle $.
Для графа~${\cal G}$\linebreak введем:
$x_{ij}$ --- поток по дуге $l_{ij}, l_{ij}\in {{\cal L}}$, протекающий 
в~соответствии с~ее направлением;
${\cal N}^{-}_j$~--- множество индексов узлов, из которых исходят\linebreak дуги, ведущие 
в~$j$-й узел, ${\cal N}^{-}_j \hm\subset {\cal N} \bigcup \{0\} $;
${\cal N}^{+}_j$~--- множество индексов узлов, в~которые ведут дуги, исходящие 
из $j$-го узла, ${\cal N}^{+}_j \hm\subset {\cal N}$. Вектор потоков
$\mathbf{x}= \langle x_{0j},\ldots, x_{ij},\ldots, x_{iN}\rangle$ по 
дугам $l_{ij}\hm\in {{\cal L}}$, где
$i \in {\cal N} \bigcup \{0\}$, $j \hm\in {\cal N}$, $i \hm\neq j$,
должен удовлетворять
условию сохранения потока в~транзитных узлах
\begin{equation}
 \sum\limits_{i \in {\cal N}^{-}_j}^{}{x_{ij}}= 
 \sum\limits_{i \in {\cal N}^{+}_j}^{}{x_{ji}},  \enskip  
 j \in {\cal N},
 \label{e1-mal}
 \end{equation}
условию сохранения потока в~стоковых узлах
\begin{equation}
 \sum\limits_{i \in {\cal N}^{-}_k}^{}{x_{ik}}= 
 \sum\limits_{i \in {\cal N}^{+}_k}^{}{x_{ki}} + y_k, \ 
 y_k\geq 0, \  k \in {\cal N}_\SK,
 \label{e2-mal}
 \end{equation}
и ограничению на пропускную способность соответствующих дуг
\begin{equation}
0 \le x_{ij} \le d_{ij}, \ l_{ij}\in {{\cal L}}\,. 
\label{e3-mal}
\end{equation}
Вектор $d = \{d_{ij} \ |\  d_{ij} \hm\geq 0, \  l_{ij}\hm\in {\cal L}\}$ 
задает максимально допустимые величины потоков  по дугам. Пусть
$ {\cal X}(d) = \{\mathbf{x} \ | \mbox{ выполняются~(1)--(3)} \}$~--- 
множество допустимых потоков в~сети по всем дугам, кроме стоковых.
Вектор
$ \mathbf{y} \hm= \langle y_{1}, \ldots, y_{k}, \ldots, y_{K}\rangle$, 
$k \hm= \overline{1, K},$
покомпонентно определяет из~(\ref{e2-mal}) 
величину потока, который передается по каждой стоковой дуге 
сети  в~соответствии 
с~некоторым допустимым потоком $\mathbf{x} \hm\in {\cal X}(d)$.
Обозначим через~${{\cal Y}}(d)$ множество всех достижимых 
век\-то\-ров-по\-то\-ков по стоковым дугам $\mathbf{y}$:
$ {{\cal Y}}(d) = \{ \mathbf{y}\ |\ \exists\ \mathbf{x}\hm\in{\cal X}(d):\  
\langle \mathbf{x}, \mathbf{y} \rangle  \mbox{ удовлетворяют (1)--(3)} \}$.  

\section{Критические структурные повреждения }

Структурное повреждение~$w$ определяется  подмножеством дуг сети, 
пропускная способность которых полагается равной нулю,
$w \hm= \{ l_{ij} \ | \ d_{ij} \hm= 0 \}$.  Пусть
$I({w})$~--- список индексов этих дуг (номеров пар вершин начала и~конца дуги).
Структурное повреждение считается \textit{критическим}, если
 при удалении соответствующих дуг максимально возможный поток, 
 хотя бы по одной стоковой дуге, оказывается равным нулю (далее~---
КС-по\-вреж\-де\-ние). Рассмотрим специальные типы КС-по\-вреж\-де\-ний, 
совпадающие с~минимальными разрезами, отделяющими хотя бы одну стоковую вершину 
от источника.

Сначала в~исходном графе сети положим пропускные способности всех дуг 
сети (кроме стоковых) равными единице и~вычислим для некоторой стоковой вершины~$v_a$ 
максимальный поток по исходящей стоковой дуге~$l_a$.

\smallskip

\smallskip

\noindent
\textbf{Задача~1.} Для выделенной стоковой вершины $v_a \hm\in {\cal V}_{\SK}$ найти
\begin{align*}
& \overline y_a^0 = \max\limits_{\mathbf{y}  \in {{\cal Y}(d)}} y_a \\
\mbox{при условиях} \ \  y_{i} = 0  & \ \ \mbox{для всех} \ i \not = a, \  
i \in {\cal N}_{\SK}, \\
d_{ij} = 1 & \ \ \mbox{для всех} \ \ l_{ij}\in {{\cal L}}. 
\end{align*}
Оптимальное решение $\overline y_a^0$~--- максимальный поток по дуге~$l_a$~--- 
численно равен пропускной способности минимального разреза~\cite{Yen}, 
отделяющего вершину~$v_a$. В~данном случае~$\overline y_a^0$ равно числу 
дуг в~минимальном разрезе. Обозначим соответствующее подмножество дуг через~${Q}_a$, 
тогда
$ |{Q}_a | \hm=  \overline y_a^0$.

\smallskip

Введем
$Q(a) = \{ Q_a^1, Q_a^2, \ldots, Q_a^{r(a)}\}$~--- 
множество всех минимальных разрезов для вершины~$v_a$.
Для каждой стоковой вершины~$v_m$, $m \hm\in {\cal N}_{\SK}$, решим задачу~1 
и~найдем все соответствующие множества~$Q(m)$.
При удалении всех дуг некоторого разреза~$Q_m^i$
вершина~$v_m$ оказывается отделена от источника, а~максимальный поток по 
стоковой дуге~$l_m$ становится равным нулю. Все  повреждения~$Q(m)$ объединим в~одно 
множество
$$
{\cal Q} = \bigcup\limits_{ m \in {\cal N}_{\SK}} Q(m)\,.
$$
Минимальные разрезы из ${{\cal Q}}$
назовем КС-по\-вреж\-де\-ни\-ями первого типа.

\smallskip

\smallskip

\noindent
\textbf{Определение~1.} 
Монопольным режимом передачи потока из фиктивного уз\-ла-ис\-точ\-ни\-ка~$v_0$ 
в~фиктивный сток~$v_{N+1}$ по  стоковой дуге~$l_{a}$ будем называть такие 
допустимые векторы потоков $\mathbf{x} \hm\in {\cal X}(d)$, при которых потоки 
по всем остальным стоковым дугам полагаются равными нулю.

\smallskip

В качестве КС-повреждений второго типа рассмотрим минимальные 
разрезы из  решения следующей задачи.

\smallskip

\smallskip

\noindent
\textbf{Задача~2.} Для выделенной стоковой вершины $v_a \hm\in {\cal V}_{\SK}$ найти
$$
 y_a^0 = \max\limits_{\mathbf{y}  \in {{\cal Y}(d)}} y_a 
 $$
$$\mbox{при дополнительных условиях}\ \  y_{i} = 0  $$
$$ \ \ \mbox{для всех} \ i \not = a, \  i \in {\cal N}_{\SK}. 
$$
Оптимальное решение~$y_a^0$~--- максимальный поток,  
достижимый при монопольном режиме пере\-дачи из~$v_0$ по стоковой дуге~$l_a$ 
в~фиктивный\linebreak узел-сток~$v_{N+1}$ (при условии, что потоки по всем остальным 
стоковым дугам полагаются равными нулю), $y_a^0$ будем называть  МРМ-по\-то\-ком. 
МРМ-по\-то\-ку соответствует по крайне мере один минимальный разрез (МРМ-раз\-рез).

Обозначим МРМ-разрез~--- подмножество дуг~$l_{ij}$,  суммарная пропускная способность 
которых равна максимальному потоку~$y_a^0$ и~при удалении которых вершина~$v_a$ 
и~сток~$v_0$ оказываются в~различных связных компонентах графа сети, 
через~$\tilde{H}_a $, тогда
$$
 \sum\limits_{(i, j) \in I(\tilde{H}_a)} d_{ij} = y_a^0\,, 
 $$
где $I(\tilde{H}_a)$~--- список индексов дуг, образующих разрез~$\tilde{H}_a$.
Одновременное удаление дуг из\linebreak МРМ-раз\-ре\-за~$\tilde{H}_a$ отделяет вершину~$v_a$ 
от сети, а~МРМ-по\-ток по стоковой дуге~$l_a$ оказывается нулевым. Введем
$H(a) \hm= \{ H_a^1, H_a^2, \ldots, H_a^{n(a)}\}$~---
множество всех МРМ-раз\-ре\-зов, соответствующих МРМ-по\-то\-ку~$y_a^0$.


Последовательно для каждой стоковой вершины $v_m \hm\in {\cal V}_{\SK}$, 
$m \hm\in {\cal N}_{\SK},$ решим задачу~2 и~сформируем
вектор МРМ-по\-то\-ков
$\mathbf{y^0}\hm= \langle y_{1}^0, y_{2}^0, \ldots, y_{m}^0,\ldots, y_{K}^0\rangle$, 
а~также все множества $H(m) \hm= \{H_m^1, H_m^2, \ldots, H_m^{n(m)}\}$.
Все  повреждения~$H(m)$ объединим в~множество ${{\cal H}}$ КС-по\-вреж\-де\-ний 
второго типа,
$$
{\cal H} = \bigcup\limits_{m \in {\cal N}_{\SK}} H(m)\,.
$$

Рассмотрим множество~${{\cal Q}}\bigcup{{\cal H}}$, 
в~нем могут содержаться разрезы с~одинаковым набором дуг. 
Исклю\-чим из ${{\cal Q}}\bigcup{{\cal H}}$ повторяющиеся разрезы и~сформируем 
множество~${\cal W} \hm= \{w^1, w^2, \ldots, w^{T}\}$, в~котором каждый элемент~$w^i$~--- 
минимальный разрез из ${{\cal Q}}\bigcup{{\cal H}}$, 
однако~$I(w^i)$ отличается от любого~$I(w^j)$, $w^j \hm\in {\cal W}$, 
$i \not = j$. Список дуг (и их индексов), образующих~$w^i$, является уникальным~--- 
не совпадает ни с~одним другим для $w^j \hm\in {\cal W}$. 

\section{Анализ уязвимости и~оценки~повреждений}

Проанализируем изменения МРМ-по\-то\-ков при удалении всех дуг,  
принадлежащих некоторому\linebreak КС-по\-вреж\-де\-нию из множества~${\cal W}$.

\smallskip

\noindent
\textbf{Задача~3.} Для выделенной стоковой вершины $v_a \in {\cal V}_{\SK}$ найти
\begin{align*}
 y_a^0 (w') = & \max\limits_{\mathbf{y}  \in {{\cal Y}(d)}} y_a  \\
\text{при условиях} \  y_{i}(w') = 0  & \ \ 
\mbox{для всех} \ i \not = a, \  i \in {\cal N}_{\SK}, \\
d_{ij} = 0 & \ \ \forall  (i, j) \in I(w'). 
\end{align*}
Оптимальное решение задачи~3~--- значение\linebreak МРМ-по\-то\-ка по стоковой дуге~$l_a$ 
при удалении из сети всех дуг из выбранного КС-по\-вреж\-де\-ния~$w'$.

\smallskip

Последовательно решим задачу~3 для выбранного КС-по\-вреж\-де\-ния~$w'$ 
и~каждой стоковой дуги~$l_k$, $k \hm\in {\cal N}_{\SK}$. Для поврежденной сети %\linebreak 
сформируем вектор МРМ-по\-то\-ков
$\mathbf{y^0}(w')\hm= \langle y_{1}^0(w'), y_{2}^0(w'), \ldots\linebreak \ldots, y_{k}^0(w'),
\ldots, y_{K}^0(w')\rangle$ и~вы\-чис\-лим значения индикатора стоковых дуг,
МРМ-по\-то\-ки по которым равны нулю,
$$
 \forall k \in {\cal N}_{\SK}: \ \  \rho_{k}(w') =
\begin{cases}
1 \,, & \mbox{если} \ y_{k}^0(w') = 0\,;  \\
0\,, & \mbox{если} \ y_{k}^0(w') > 0\,.
\end{cases}
$$
Для КС-повреждения~$w'$ и~найденных значений подсчитаем число
$$
  \nu ^0(w') = \sum\limits_{k = 1}^K \rho_{k}(w')
  $$
  и долю
  $$  
  \rho(w') = \fr{\nu^0(w')}{K}
  $$
стоковых дуг, МРМ-по\-то\-ки по которым равны \mbox{нулю.}

Рассмотрим множество стоковых дуг ${\cal L}^{\pm}(w')$, 
МРМ-по\-ток по которым после КС-по\-вреж\-де\-ния~$w'$ больше нуля, 
${\cal L}^{\pm}(w')\hm= \{ l_k \ | \ k\hm\in {\cal N}_{\SK}, y_{k}^0(w') \hm> 0 \}$. 
В~разд.~3 при решении задачи~2 были получены значения~$y_{k}^0$ для  
МРМ-по\-то\-ков в~исходной не\-повреж\-ден\-ной сети. Определим ущерб  для $k$-й 
стоковой дуги из~${\cal L}^{\pm}(w')$ от КС-по\-вреж\-де\-ния~$w'$ (не важно 
первого или второго типа) как падение МРМ-по\-то\-ка по ней и~дадим оценку 
относительного ущерба за счет снижения МРМ-по\-то\-ка:
$$
 \psi_k(w') = \fr{y_{k}^0 - y_{k}^0(w')}{y_{k}^0}\,, \ 
 y_{k}^0> 0\,, \ k\in {\cal N}_{\SK}\,. 
 $$
Для найденных значений $\psi_k(w')$ определим значение медианы~$\psi(w')$. 
Медиана~$\psi(w')$ делит мно-\linebreak жество~${\cal L}^{\pm}(w')$ на две части~--- 
подмножество~${\cal L}^+(w')$, состоящее из стоковых дуг, для\linebreak которых 
ущерб~$\psi_k(w')$ больше медианы,
${\cal L}^+(w')\hm=  \{ l_k \ | \ \psi_k(w') \hm> \psi(w'), \ l_k\hm\in 
{\cal L}^{\pm}(w') \}$, 
и~под\-мно\-же\-ст\-во~${\cal L}^-(w')$, состоящее из стоковых дуг, 
для которых ущерб не больше медианы, ${\cal L}^-(w')\hm= 
 \{ l_k \ | \ \psi_k(w') \hm \leq \psi(w'), \ l_k\in {\cal L}^{\pm}(w') \}$ 
 (при нечетном числе элементов в~${\cal L}^{\pm}(w')$ 
 в~качестве медианы выбирается среднее значение, а~при четном~--- 
 полусумма двух средних по порядку величин).
Введем индикаторную функцию~$\varphi_k(w')$  и~вычислим ее значения для дуг 
из~${\cal L}^{\pm}(w')$:
$$
  \varphi_k(w') =
\begin{cases}
0 \,, & \mbox{если} \ l_{k} \in {\cal L}^-(w')\,;  \\
1\,, & \mbox{если} \ l_{k} \in {\cal L}^+(w')\,.
\end{cases}
$$

Доля дуг, которые принадлежат подмножеству ${\cal L}^+(w')$, составляет
$$
\delta(w') = \fr{\sum\nolimits_{k = 1}^K\varphi_k(w')}{K} \leq
\fr{1}{2} \left(1 - \rho(w')\right)\,.
$$
Анализ результатов решения задач~2 и~3, полученных для КС-по\-вреж\-де\-ний, 
позволяет получить подробную информацию о последствиях любого\linebreak повреждения сети 
из рассматриваемого класса структурных повреждений. Действительно, пара чисел~--- 
значений параметров $\langle \rho(w'), \psi(w') \rangle$~---
в~агрегированном виде описывают изменения функциональных возможностей 
сетевой системы после КС-по\-вреж\-де\-ния $w'$:
$\rho(w')$~--- доля нулевых МРМ-по\-то\-ков; $\psi(w')$~--- 
медианная величина ущерба для потребителей.
У~некоторых стоковых дуг ущерб оказывается больше значения~$\psi(w')$, 
и~их доля составляет не более
$(1 \hm- \rho(w'))/2$ от их общего числа~$K$. 
При четном числе~$| {\cal L}^{\pm}(w') |$ и~несовпадении средних по 
порядку величин ущерба ($\psi(w')\hm\ne \psi_k(w')$)\linebreak КС-по\-вреж\-де\-ние~$w'$ 
разбивает все множество стоковых дуг, а~также значения соответствующих 
МРМ-по\-то\-ков в~пропорциях (долях)
$\langle \rho(w'), (1 \hm- \rho(w'))/2, (1 \hm- \rho(w'))/2 \rangle$. 
При нечетном числе разбиение аналогично, но надо использовать величины~$\delta(w')$.



МРМ-потоки по стоковым дугам характеризуют возможности сетевой 
системы до и~после по\-вреж\-де\-ния. В~целях получения сравнительных оце\-нок 
уязвимости как подверженности потребителей сети влиянию повреждений 
проанализируем изменения всех стоковых потоков для каждого КС-по\-вреж\-де\-ния 
из множества~${\cal W}$. Для произвольной стоковой вершины определим, 
в~каком подмножестве она оказывается после каждого КС-по\-вреж\-де\-ния. 
Для всех стоковых вершин и~всех повреждений 
из~${\cal W}$ вычислим все значения индикаторных функций~$\rho_k(w^i)$ 
и~$\varphi_k(w^i), k \in {\cal N}_{\SK}, i = \overline{1, T}$. 
Для каждой $v_k, k \in {\cal N}_{\SK}$, подсчитаем долю  тех КС-по\-вреж\-де\-ний 
$w^i \hm\in {\cal W}$, для которых МРМ-по\-ток по стоковой дуге~$l_k$ будет нулевым:
$$
\rho_k({\cal W}) = \sum\limits_{w^i \in {\cal W}}\fr{\rho_k(w^i)}{T}\,.  
$$

Подсчитаем также долю КС-по\-вреж\-де\-ний из~${\cal W}$, при которых ущерб 
потребителя потока в~вершине~$v_k$ (относительные потери МРМ-по\-то\-ка по дуге~$l_k$), 
$k \hm\in {\cal N}_{\SK}$, больше значения соответствующей медианы:
$$
 \varphi_k({\cal W}) = \sum\limits_{w^i \in {\cal W}}\fr{\varphi_k(w^i)}{T}\,. 
 $$

Используем полученные результаты для по\-стро\-ения стандартной диаграммы, 
пред\-став\-ля\-ющей найден\-ные оценки уяз\-ви\-мости по отношению ко всем 
учитываемым в~модели повреждениям на\-глядно в~виде точек на плос\-кости, 
со\-от\-вет\-ст\-ву-\linebreak ющих паре введенных агрегированных характеристик
 для каж\-дой 
стоковой вершины. Это позволит\linebreak ранжировать по\-тре\-би\-те\-лей потока сети по 
подверженности влиянию КС-по\-вреж\-де\-ний заданного класса~(${\cal W}$). 
Ранжирование проводится по двум критериям  $(\rho_k({\cal W}),  \varphi_k({\cal W}))$, 
поэтому на диа\-грам\-ме (см.\ рисунок) используются именно такие ко\-ор\-ди\-наты.



Точки диаграммы с~рисунка, находящиеся на выпуклой оболочке 
или <<в~относительной бли\-зости>> от нее, соответствуют стоковым вершинам, 
наиболее уязвимым (в смысле указанных двух критериев) по отношению 
к~КС-по\-вреж\-де\-ни\-ям первого и~второго типа. При удалении дуг ка\-ко\-го-ли\-бо 
минимального разреза из~${\cal W}$ потоки в~такие стоковые вершины 
в~большем чис\-ле случаев оказываются равными нулю и/или потери МРМ-по\-то\-ка 
оказываются больше медианных значений. Таким образом, найденные стоковые вершины 
(и~стоящие за ними потребители потоков сети) сильнее других подвержены влиянию 
КС-по\-вреж\-де\-ний первого и~второго типа. 

{\begin{center}  %fig
 \vspace*{12pt}
  \mbox{%
 \epsfxsize=78.698mm 
 \epsfbox{mal-1.eps}
 }



\vspace*{6pt}


\noindent
{\small Диаграмма $(\rho_k({\cal W}),  \varphi_k({\cal W}))$}
\end{center}

}



\section{Заключение}

\vspace*{-10pt}

Поиск потребителей сети, наиболее подверженных влиянию ее повреждений, 
осуществляется в~данной работе с~помощью эффективных потоковых алгоритмов. 
Для этого авторы ограничили рассмотрение только такими КС-по\-вреж\-де\-ни\-ями, 
которые представляются базовыми при решении задачи за потенциального противника, 
раз\-ру\-ша\-юще\-го сеть с~целью лишить возможности получения потока хотя бы одного ее 
пользователя. Учет всех комбинаций базовых разрезов и/или не только базовых (т.\,е.\
 минимальных в~том или ином смысле) разрезов существенно усложняет поиск. 
 В~таком случае рекомендуется подход имитационного моделирования, когда наборы дуг 
 сети, образующие разрез, задаются сценарно исследователем (пример подобной 
 программно реализованной системы для конкретного класса многополюсных сетей 
 приведен в~\cite{ Koz17}). Также при наличии дополнительной экспертной 
 информации можно ограничиться лишь некоторыми комбинациями различных разрезов. 
 При этом базовые разрезы все равно должны исследоваться в~полном объеме.

В данной работе авторы ориентируются на анализ влияния повреждений на 
конечных потребителей передаваемого потока. Поэтому после того как разрез 
отобран, в~характеристику уязвимости не включаются ни число дуг, образующих разрез, 
ни их пропускная способность, т.\,е.\ все разрезы 1-го и~2-го типа для проводимого 
анализа равноправны (в~отличие от характеристик уязвимости, введенных 
в~\cite{MalInf181, MalInf183}). Отметим также, что не принимается во внимание 
вес разреза в~следующем смысле.
Для сетевых систем (в~част\-ности, с~гра\-фом-де\-ре\-вом) характерно наличие на 
одном разрезе сети целой <<грозди>> потребителей, для каждого из которых этот разрез~--- 
минимальный в~задаче на максимум потока именно ему. 
В~предложенной статье влияние таких разрезов на каждого потребителя 
рассматривается лишь один раз, хотя в~принципе указанный разрез может 
возникать чаще (при желании <<противника>> повредить любому из соседей 
по сети данного потребителя). Другой подход см.\ в~статье~\cite{Mal186}.

Поскольку рассматриваемые модели являются однопродуктовыми, используются 
потоковые методы анализа уязвимости, разработанные авторами ранее. Тем не 
менее нельзя не упомянуть тео\-ре\-ти\-ко-гра\-фо\-вые способы оценки 
структурной уязвимости, 
которые в~контексте настоящей статьи дают некоторые критерии (см., например, 
в~\cite{Koch}), позволяющие выделить комбинации базовых разрезов для дополнительного 
включения в~множество учитываемых повреждений. Однако при этом необходимо принимать 
во внимание и~мощность\linebreak\vspace*{-12pt}

\pagebreak

\noindent
 повреждения, чтобы не получить тривиальных результатов.

Предлагаемые методы достаточно общие, чтобы применять их и~в~сетях более 
сложной структуры: многопродуктовых, динамических, стохастических. 
В~частности, их можно рекомендовать и~для анализа уязвимости потоковых сетей, 
рассмотренных в~\cite{Kuz}. 

\vspace*{-6pt}

{\small\frenchspacing
 {%\baselineskip=10.8pt
 \addcontentsline{toc}{section}{References}
 \begin{thebibliography}{9}
    
\bibitem{MalInf17} %1
\Au{Малашенко Ю.\,Е., Назарова~И.\,А., Новикова~Н.\,М.} 
Метод анализа функциональной уязвимости потоковых сетевых систем~// 
Информатика и~её применения, 2017. Т.~11. Вып.~4. С.~47--54.

\bibitem{MalInf181}  %2
\Au{Малашенко Ю.\,Е., Назарова~И.\,А., Новикова~Н.\,М.} 
Диаграммы уязвимости потоковых сетевых систем~// Информатика и~её применения, 2018. 
Т.~12. Вып.~1. С.~11--18.

\bibitem{Germ}   %3
\Au{Гермейер Ю.\,Б.} Введение в~теорию исследования операций.~--- 
М.: Наука, 1971. 384~с.

\bibitem{MalInf183}  %4
\Au{Малашенко Ю.\,Е., Назарова~И.\,А., Новикова~Н.\,М.} 
Анализ разрезных повреждений в~многополюсных сетях~// Информатика и~её 
применения, 2018. Т.~12. Вып.~3. С.~35--41.

\bibitem{Yen}  %5
\Au{Йенсен П., Барнес~Д.} Потоковое программирование~/ Пер. с~англ.~--- 
М.: Радио и~связь, 1984. 392~с. (\Au{Jensen~P.\,A., Barnes~J.\,W.} 
Network flow programming.~--- New York, NY, USA: Wiley, 1980. 408~p. )

\bibitem{Mal186}  %6
\Au{Назарова~И.\,А., Малашенко Ю.\,Е.,  Новикова~Н.\,М.} 
Управ\-ле\-ние топ\-лив\-но-энер\-ге\-ти\-че\-ской сис\-те\-мой при крупномасштабных повреждениях.
III.~Критические и~стационарные режимы~// 
Изв. РАН. Тео\-рия и~сис\-те\-мы управ\-ле\-ния, 2018. 
№\,4. С.~107--121.



\bibitem{Koz17} 
\Au{Козлов М.\,В., Малашенко~Ю.\,Е., Назарова~И.\,А. и~др.} 
Управ\-ле\-ние   топ\-лив\-но-энер\-ге\-ти\-че\-ской  сис\-те\-мой  
при  крупномасштабных повреждениях. I.~Сетевая  модель  и~программная реализация~// 
Изв. РАН. Теория и~сис\-те\-мы управления,  2017. №\,6. С.~50--73.

\bibitem{Koch}  
\Au{Кочкаров А.\,А., Салпагаров~М.\,Б., Эльканова~Л.\,М.}
 Дискретная модель структурного разрушения сложных сис\-тем~// Проб\-ле\-мы 
 управ\-ле\-ния, 2007. Вып.~5. С.~21--26.

\bibitem{Kuz}  
\Au{Кузнецов О.\,П., Жилякова~Л.\,Ю.} 
Двусторонние ресурсные сети~--- новая потоковая модель~// Докл. Академии наук, 2010. 
Т.~433. Вып.~5. С.~609--612.
 \end{thebibliography}

 }
 }

\end{multicols}

\vspace*{-3pt}

\hfill{\small\textit{Поступила в~редакцию 08.11.18}}

\vspace*{8pt}

%\pagebreak

%\newpage

%\vspace*{-28pt}

\hrule

\vspace*{2pt}

\hrule

\vspace*{-6pt}

\def\tit{VULNERABILITY ANALYSIS OF~MULTIPOLAR NETWORKS AFTER~STRUCTURAL DAMAGES}

\def\titkol{Vulnerability analysis of~multipolar networks after structural damages}

\def\aut{Yu.\,E.~Malashenko, I.\,A.~Nazarova, and~N.\,M.~Novikova}

\def\autkol{Yu.\,E.~Malashenko, I.\,A.~Nazarova, and~N.\,M.~Novikova}

\titel{\tit}{\aut}{\autkol}{\titkol}

\vspace*{-11pt}


\noindent
%A.\,A.~Dorodnicyn Computing Center, 
Federal Research Center 
``Computer Science and Control'' of the Russian Academy of Sciences, 
40~Vavilov Str., Moscow 119333, Russian Federation

\def\leftfootline{\small{\textbf{\thepage}
\hfill INFORMATIKA I EE PRIMENENIYA~--- INFORMATICS AND
APPLICATIONS\ \ \ 2019\ \ \ volume~13\ \ \ issue\ 1}
}%
 \def\rightfootline{\small{INFORMATIKA I EE PRIMENENIYA~---
INFORMATICS AND APPLICATIONS\ \ \ 2019\ \ \ volume~13\ \ \ issue\ 1
\hfill \textbf{\thepage}}}

\vspace*{6pt}

 

\Abste{A~method  to obtain informative estimates of functionality 
changes in a~multistock network system after possible damages
has been proposed. Within the framework 
of the transmission model of single-commodity flow, the authors study the set 
of achievable flow vectors satisfying the standard conditions of conservation 
and constraints on the flows along arcs. To analyze the initial state of the 
system, for each stock arc, the maximum flow has been calculated separately and 
independently of the flow value across the remaining stock arcs. 
The corresponding minimal cut separates this stock vertex from the source. 
All arcs of the found minimal cut are removed model-wisely and in the network 
which is damaged in this way, the possibilities of transmission flows to other 
stock vertices have been
estimated. The maximum flows for the vertex have been calculated 
and compared with their initial values. Estimates of losses have been compared for 
various minimal cuts. Influences of these structural damages have been constructed 
for all stock vertices.  The vertices' aggregated characteristics of the structural 
damage effect have been calculated.}

\KWE{structural network vulnerability; sensitivity to critical damage; 
multipolar flow model}

\DOI{10.14357/19922264190105}

%\vspace*{-14pt}

%\Ack
%\noindent




\vspace*{-7pt}

  \begin{multicols}{2}

\renewcommand{\bibname}{\protect\rmfamily References}
%\renewcommand{\bibname}{\large\protect\rm References}

{\small\frenchspacing
 {%\baselineskip=10.8pt
 \addcontentsline{toc}{section}{References}
 \begin{thebibliography}{9}
 
% \vspace*{-1pt}

\bibitem{1-mal}
\Aue{Malashenko, Yu.\,E., I.\,A.~Nazarova, and N.\,M.~Novikova.} 
2017. Metod analiza funktsional'noy uyazvimosti potokovykh setevykh system 
[Method of the analysis of the functional vulnerability of flow network systems]. 
\textit{Informatika i~ee Primeneniya~--- Inform. Appl.} 11(4):47--54.
{\looseness=1

}
\bibitem{2-mal}
\Aue{Malashenko, Yu.\,E., I.\,A.~Nazarova, and N.\,M.~Novikova.} 
2018. Diagrammy uyazvimosti potokovykh setevykh sistem [Diagram of the functional 
vulnerability of flow network systems]. 
\textit{Informatika i~ee Primeneniya~--- Inform. Appl.} 12(1):11--18.

\vspace*{-3pt}

\bibitem{4-mal}
\Aue{Germeier, Yu.\,B.} 1971. \textit{Vvedenie v~teoriyu issledovaniya ope\-ra\-tsiy} 
[An introduction to operations research theory]. Moscow: Nauka. 384~p. 
\bibitem{3-mal}
\Aue{Malashenko, Yu.\,E., I.\,A.~Nazarova, and N.\,M.~Novikova.}
2018. Analiz razreznykh povrezhdeniy v~mnogopolyusnykh setyakh 
[Analysis of cutting damages to multipolar networks]. 
\textit{Informatika i~ee Primeneniya~--- Inform. Appl.} 12(3):35--41. 
 
\bibitem{5-mal}
\Aue{Jensen, P.\,A., and J.\,W.~Barnes.} 1980. 
\textit{Network flow programming}. New York, NY: Wiley. 408~p. 
\bibitem{6-mal}
\Aue{Malashenko, Yu.\,E., I.\,A.~Nazarova, and N.\,M.~Novikova.} 
2018. Fuel and energy system control at large-scale damages. 
III.~Emergency and stationary modes. 
\textit{J.~Comput. Sys. Sc. Int.}  57(4):581--593.   
\bibitem{7-mal}
\Aue{Kozlov, M.\,V., Yu.\,E.~Malashenko, I.\,A.~Nazarova, \textit{et al.}} 
2017. Fuel and energy system control at large-scale damages. 
I.~Network model and software implementation.  
\textit{J.~Comput. Sys. Sc. Int.}  56(6):945--968.  
\bibitem{8-mal}
\Aue{Kochkarov, A.\,A., M.\,B.~Salpagarov, and L.\,M.~Elkanova.} 
2007. Diskretnaya model' strukturnogo razrusheniya slozhnykh sistem  
[A~discrete model of compound systems destruction]. 
\textit{Problemy upravleniya} [Control Sciences] 5:21--26. 
\bibitem{9-mal}
\Aue{Kuznetsov,  O.\,P., and L.\,Y.~Zhilyakova.} 2010. 
Bidirectional resource networks: A~new flow model. 
\textit{Dokl. Math.} 82(1):643--646.
\end{thebibliography}

 }
 }

\end{multicols}

\vspace*{-3pt}

\hfill{\small\textit{Received November 8, 2018}}

%\pagebreak

%\vspace*{-18pt}

\Contr

\noindent
\textbf{Malashenko Yuri E.} (b.\ 1946)~--- 
Doctor of Science in physics and mathematics, principal scienist, 
%A.\,A.~Dorodnicyn Computing Center, 
Federal Research Center 
``Computer Science and Control'' of the Russian Academy of Sciences, 40~Vavilov Str., 
Moscow 119333, Russian Federation; \mbox{malash09@ccas.ru}

\vspace*{4pt}
 
\noindent
\textbf{Nazarova Irina A.} (b.\ 1966)~--- 
Candidate of Science (PhD) in physics and mathematics, scientist, 
%A.\,A.~Dorodnicyn Computing Center, 
Federal Research Center 
``Computer Science and Control'' of the Russian Academy of Sciences, 
40~Vavilov Str., Moscow 119333, Russian Federation; \mbox{irina-nazar@yandex.ru}

\vspace*{4pt}
 
\noindent
\textbf{Novikova Natalya M.} (b.\ 1953)~--- 
Doctor of Science in physics and mathematics, professor, leading scientist, 
%A.\,A.~Dorodnicyn Computing Center, 
Federal Research Center 
``Computer Science and Control'' of the Russian Academy of Sciences, 
40~Vavilov Str., Moscow 119333, Russian Federation; \mbox{n\_novikova@umail.ru}


\label{end\stat}

\renewcommand{\bibname}{\protect\rm Литература}       