\def\stat{zatsman}

\def\tit{ЦЕЛЕНАПРАВЛЕННОЕ РАЗВИТИЕ СИСТЕМ ЛИНГВИСТИЧЕСКИХ ЗНАНИЙ:\\ ВЫЯВЛЕНИЕ 
И~ЗАПОЛНЕНИЕ ЛАКУН$^*$}

\def\titkol{Целенаправленное развитие систем лингвистических знаний: выявление 
и~заполнение лакун}

\def\aut{И.\,М.~Зацман$^1$}

\def\autkol{И.\,М.~Зацман}

\titel{\tit}{\aut}{\autkol}{\titkol}

\index{Зацман И.\,М.}
\index{Zatsman I.\,M.}


{\renewcommand{\thefootnote}{\fnsymbol{footnote}} \footnotetext[1]
{Работа выполнена в~Институте проб\-лем информатики ФИЦ ИУ РАН при частичной поддержке РФФИ (проект 
18-07-00192).}}


\renewcommand{\thefootnote}{\arabic{footnote}}
\footnotetext[1]{Институт проб\-лем информатики Федерального исследовательского центра 
<<Информатика и~управ\-ле\-ние>> Российской академии наук, \mbox{izatsman@yandex.ru}}

\footnotetext[2]{В~лингвистике типологии используются, как правило, для 
описания сходства и~различий между языками. В~статье этот термин 
используется в~другом значении: для структурированного описания связей 
исследуемых языковых единиц с~их значениями и~переводными 
эквивалентами.}


\vspace*{2pt}
    
   \Abst{Дано описание процесса целенаправленного развития лингвистических 
типологий$^2$ как форм представления знаний об исследуемых языковых единицах
(ЯЕ). 
Рассматриваемая задача заключается в~обнаружении лакун в~системе современного 
знания (ССЗ)
об исследуемых ЯЕ, что предполагает выбор и~использование некоторого 
эталона, отражающего современный уровень знания в~соответствующей предметной 
области. Процесс обнаружения лакун представляет собой вид лингвистического 
аннотирования текстов, содержащих исследуемые ЯЕ, с~применением 
методов и~средств информатики. В качестве источника новых знаний для заполнения лакун 
используются параллельные тексты, которые включают оригинальные литературные 
произведения и~их переводы. Цель статьи состоит в~описании подхода к~обнаружению 
лакун в~процессе аннотирования параллельных текстов, содержащих исследуемые 
ЯЕ и~их переводы.}
   
   \KW{лингвистическая типология; параллельные тексты; компьютерная лингвистика; 
извлечение новых знаний; информационная технология; целенаправленное развитие 
типологий}

\DOI{10.14357/19922264190113}
  
%\vspace*{4pt}


\vskip 10pt plus 9pt minus 6pt

\thispagestyle{headings}

\begin{multicols}{2}

\label{st\stat}
   
\section{Введение}

\vspace*{-2pt}
    
  Разработка методов и~средств информатики для обнаружения лакун 
  в~ССЗ о~языке в~процессе аннотирования 
параллельных текстов~[1--4]~--- одна из задач проекта по гранту РФФИ, 
который в~настоящее время выполняется в~Институте проблем информатики 
ФИЦ ИУ РАН. Обнаружение лакун является основой для последующего 
целенаправленного пополнения существующих или формирования новых 
лингвистических типологий как форм представления лингвистами 
извлеченного ими знания об исследуемых ЯЕ~[5--8].
  
  Актуальность проблемы обнаружения лакун проявляет себя наиболее 
наглядно в~процессе разработки и~развития систем машинного перевода, когда 
имеющиеся типологии моделей перевода оказываются неполными для тех или 
иных классов ЯЕ и~при этом отсутствуют средства информатики, 
обеспечивающие целенаправленное их пополнение. Более того, для некоторых 
классов ЯЕ такие типологии могут пол\-ностью отсутствовать. 
Тогда речь идет о~не\-об\-хо\-ди\-мости формирования новых лингвистических 
типологий. Например, практически пол\-ностью отсутствует типология русских 
конструкций с~модальным значением как переводных эквивалентов немецких 
модальных глаголов~[9] и~сегодня нет средств информатики, обес\-пе\-чи\-ва\-ющих 
ее целенаправленное формирование. Именно в~подобных ситуациях возникает 
необходимость в~формировании новых и~развитии существующих типологий, 
чем и~обусловлена актуальность создания обеспечивающих их формирование 
методов и~средств информатики.
  
  Таким образом, речь идет не о~представлении в~виде типологий 
конвенциональных лингвистических знаний, эксплицированных 
в~справочниках и~словарях. Предлагается подход к~выявлению лакун в~ССЗ
 о~языке, представленного в~справочниках и~словарях, на 
основе автоматизированной обработки параллельных текстов. Описание 
предлагаемого подхода иллюстрируется примером, в~котором цель 
исследования состоит в~уточнении типологии значений немецких модальных 
глаголов и~их переводных эквивалентов на русском языке. Ее планируется 
пополнять в~процессе обработки не\-мец\-ко-рус\-ских параллельных 
текстов~[9].

  \begin{table*}\small %tabl1
  %\vspace*{-12pt}
  \begin{center}
  \Caption{Примеры объектов интерпретации из трех произведений}
  \vspace*{2ex}
  
  \begin{tabular}{|p{70mm}|p{70mm}|}
  \hline
Sollte jetzt etwa eine Predigt stattfinden?\newline
[Franz Kafka. Der Prozess (1914)]&Неужели сейчас кто-то будет читать проповедь?\newline
[Франц Кафка. Процесс (Р.~Райт-Ковалева, 1965)]\\
\hline
Warum \mbox{mu{\!\ptb{\ss}}te} er diese Demoiselle St$\ddot{\mbox{u}}$wing heiraten und 
den$\ldots$ Laden$\ldots$\newline
[Thomas Mann. Buddenbrooks (1896-1900)]&Зачем ему понадобилось жениться на этой 
мадемуазель Штювинг с~ее$\ldots$ лавкой?\newline
[Томас Манн. Будденброки (Н.~Ман, 1953)]\\
\hline
Er durfte nun eine Weile lang guten Gewissens ruhen.\newline
[Patrick S$\ddot{\mbox{u}}$skind. Das Parfum: Die Geschichte eines M$\ddot{\mbox{o}}$rders 
(1985)]&Теперь он имел право некоторое время отдыхать.\newline
[Патрик Зюскинд. Парфюмер: История одного убийцы (Э.~Венгерова, 1992)]\\
\hline
\end{tabular}
\end{center}
\end{table*}
  
  Процесс пополнения существующих или формирования новых 
лингвистических типологий включает несколько итерационно повторяемых 
стадий, которые в~совокупности образуют одну итерацию целенаправленного 
извлечения знаний и~представления их в~виде новых типологий или новых 
рубрик, включаемых в~существующие типологии. Каждая итерация состоит из 
следующих стадий (их описание является развитием результатов работы~[10]).
  
  \textbf{Стадия~1.} Каждая итерация начинается с~определения одной или 
нескольких ЯЕ, исследуемых на текущей итерации, 
и~поиска в~параллельных текстах тех предложений, которые содержат 
исследуемую ЯЕ, и~соответствующих им переводов. Одно или несколько 
найденных предложений и~их переводы будем называть объектами 
интерпретации. Три их примера из работы~[10] с~тремя разными немецкими 
модальными глаголами, состоящие из одного предложения и~его перевода, 
приведены в~табл.~1.
  

  
  Массив найденных объектов интерпретации может включать тысячи, 
а~иногда и~десятки тысяч предложений и~их переводов~\cite{6-zat, 7-zat}. 
В~таких случаях на первой стадии формируется подмассив объектов 
интерпретации для семантического их анализа именно на этой итерации.
  
  В настоящее время наиболее представительный массив немецко-русских 
параллельных текс\-тов в~электронной форме хранится в~Параллельном 
немецком корпусе (ПНК), который находится в~открытом доступе~[11]. 
Примеры в~табл.~1 скопированы из этого корпуса. Его параллельные тексты 
являются выровненными (оригинальным предложениям поставлены 
в~соответствие их переводы), т.\,е.\ необходимая их фрагментация на объекты 
интерпретации (т.\,е.\ на пары немецких и~русских предложений) уже 
выполнена в~процессе формирования этого корпуса.
  
  \textbf{Стадия~2.} Выполняется лингвистическое аннотирование объектов 
интерпретации с~ис\-сле\-ду\-емы\-ми ЯЕ, найденных в~параллельных текстах на 
стадии~1. Метод аннотирования с~использованием динамических фасетных 
классификаций описан в~работах~\cite{2-zat, 3-zat, 4-zat, 10-zat, 12-zat}. Он 
отличается от традиционных методов аннотирования~\cite{13-zat} тем, что 
допускает возможность незавершенности процесса аннотирования объектов 
интерпретации. В~этом случае анализируются причины незавершенности 
и~в~аннотации объектов интерпретации включаются специальные руб\-ри\-ки-те\-ги, 
которые указывают на ту или иную причину. Например, в~объекте 
интерпретации может встретиться новое значение некоторого немецкого 
модального глагола, которое не отражено в~ССЗ, 
или в~процессе аннотирования не удалось определить значение, 
в~котором употреблен модальный глагол~\cite{10-zat}. Возможна ситуация, 
когда не удается определить смысл всего аннотируемого объекта 
интерпретации.
  
  Для решения задач проекта интерес представляют в~первую очередь те 
рубрики аннотаций, которые в~качестве причины указывают на лакуны в~ССЗ 
о~значениях немецких модальных глаголов. В качестве авторитетного 
описания ССЗ об их значениях лингвистами был выбран не\-мец\-ко-рус\-ский 
словарь~\cite{14-zat}.
  
  \textbf{Стадия~3.} Выполняется семантический анализ\linebreak незавершенных 
аннотаций, помеченных рубриками, которые указывают на причины 
не\-за\-вер\-шен\-ности. Предметом анализа являются те незавершенные аннотации, 
в~которых указано, что не\linebreak удалось определить значение немецкого модального 
глагола или смысл всего объекта интерпретации.
  
  \textbf{Стадия~4.} Выполняется пополнение су\-щест\-ву\-ющей типологии, 
которое основано на процессе доработки незавершенных аннотаций, 
включающих рубрики, которые в~качестве причины указывают на лакуны 
в~ССЗ о~значениях.

  \begin{table*}[b]\small %tabl2
  \begin{center}
  \Caption{Примеры результатов поиска пар предложений по словоформе sollte}
  \vspace*{2ex}
  
  \begin{tabular}{|c|p{70mm}|p{70mm}|}
  \hline
\tabcolsep=0pt\begin{tabular}{c}Номер\\  
пары\end{tabular}&\multicolumn{1}{c|}{Оригинальный 
текст}&\multicolumn{1}{c|}{Перевод}\\
\hline
3433&Sollte jetzt etwa eine Predigt stattfinden?&Неужели сейчас кто-то будет читать 
проповедь?\\
\hline
3435&K sah an der Treppe hinab, die an die S$\ddot{\mbox{a}}$ule sich anschmiegend zur 
Kanzel f$\ddot{\mbox{u}}$hrte und so schmal war, als sollte sie nicht f$\ddot{\mbox{u}}$r 
Menschen, sondern nur zum Schmuck der S$\ddot{\mbox{a}}$ule dienen.&K поглядел на 
лесенку, которая вела на кафедру, лепясь к~самой колонне; она была настолько узкой, что, 
казалось, служила не людям, а~просто украшению колонны.\\
\hline
3439&Sollte wirklich eine Predigt beginnen?&Неужели сейчас начнется проповедь?\\
\hline
3442&Und wenn es schon eine Predigt sein sollte, warum wurde sie nicht von der Orgel 
eingeleitet?&А если уж собираются начинать проповедь, почему перед этим не вступает 
орган?\\
\hline
\end{tabular}
\end{center}
%\end{table*}
%\begin{table*}\small %tabl3
\begin{center}
\Caption{Объединение соседних пар в~один объект интерпретации}
\vspace*{2ex}

\begin{tabular}{|c|p{70mm}|p{73mm}|}
\hline
\tabcolsep=0pt\begin{tabular}{c}Номер\\  
пары\end{tabular}&\multicolumn{1}{c|}{Оригинальный 
текст}&\multicolumn{1}{c|}{Перевод}\\
\hline
3431&Das Ganze war wie zur Qual des Predigers bestimmt, es war unverst$\ddot{\mbox{a}}$ndlich, wozu 
man diese Kanzel ben$\ddot{\mbox{o}}$tigte, da man doch die andere, \mbox{gro{\!\ptb{\ss}}e} und so 
kunstvoll geschm$\ddot{\mbox{u}}$ckte zur Verf$\ddot{\mbox{u}}$gung hatte.
&Казалось, все было задумано нарочно для мучений проповедника, 
и~нельзя было понять, зачем 
нужна эта ка\-фед\-ра, когда можно располагать главной, большой, 
столь искусно разукрашенной.\\
%\hline
3432&K w$\ddot{\mbox{a}}$re auch diese kleine Kanzel \mbox{gewi{\!\ptb{\ss}}} nicht aufgefallen, 
wenn nicht oben eine Lampe befestigt gewesen w$\ddot{\mbox{a}}$re, wie man sie kurz vor einer Predigt 
bereitzustellen pflegt.&
K,~наверно, не заметил бы эту маленькую ка\-фед\-ру, если бы в~ней не горела лампа, какие обычно 
зажигают для проповедника перед проповедью.\\
%\hline
\textbf{3433}&\textbf{Sollte jetzt etwa eine Predigt stattfinden?}&\textbf{Неужели сейчас 
кто-то будет читать проповедь?}\\
%\hline
3434&In der leeren Kirche?&Тут, в~пустом соборе?\\
%\hline
3435&K~sah an der Treppe hinab, die an die S$\ddot{\mbox{a}}$ule 
sich anschmiegend zur Kanzel 
f$\ddot{\mbox{u}}$hrte und so schmal war, als sollte sie nicht 
f$\ddot{\mbox{u}}$r Menschen, sondern 
nur zum Schmuck der S$\ddot{\mbox{a}}$ule dienen.&
K~поглядел на лесенку, которая вела на ка\-фед\-ру, лепясь к~самой колонне;
 она была настолько узкой, 
что, казалось, служила не людям, а~прос\-то украшению колонны.\\
\hline
\end{tabular}
\end{center}
\end{table*}
  
  
  \textbf{Стадия~5.} Вычисляются числовые параметры, характеризующие 
состояние процесса пополнения типологии на момент завершения итерации. 
После этого осуществляется переход к~следующей итерации, начиная с~первой 
стадии, если не достигнута желаемая степень заполнения обнаруженных \mbox{лакун}.
  
  Цель статьи состоит, главным образом, в~описании первых двух из пяти 
перечисленных стадий, включая описание подхода к~обнаружению лакун 
в~процессе аннотирования параллельных текстов, содержащих исследуемые 
ЯЕ, в~интересах пополнения существующей типологии. Кратко 
рас\-смат\-ри\-ва\-ют\-ся последние три стадии. Вопросы формирования новой 
типологии с~нуля в~этой статье не рассматриваются.

\vspace*{-6pt}

\section{Поиск объектов интерпретации в~параллельных текстах}
    
  В интересах решения задач проекта был создан макет базы знаний (МБЗ), 
который обеспечивает выполнение лингвистами пяти перечисленных стадий. 
Для проверки реализуемости предлагаемого подхода к~обнаружению лакун 
в~ССЗ о~немецких модальных глаголах в~МБЗ были загружены выровненные 
тексты ПНК общим объемом~2,6~млн словоупотреблений, включая~1,4~млн 
сло\-во\-упо\-треб\-ле\-ний в~оригинальных текстах на немецком языке и~1,2~млн 
сло\-во\-упо\-треб\-ле\-ний в~их переводах на русский язык. Загруженные 
не\-мец\-ко-рус\-ские параллельные тексты состоят из пар немецких и~русских предложений, 
всего~--- 83\,190~пар (см.\ табл.~1 с~тремя парами). В~среднем одна пара 
содержит 31,25~сло\-во\-упо\-треб\-ле\-ний (16,83 в~левой немецкой части и~14,42 
в~русской). 
  
  Предположим, что сначала для исследования лингвистами был выбран 
модальный глагол sollen и~его значения~\cite{9-zat}. По предварительной 
оценке, для уточнения типологии значений немецких модальных глаголов уже 
загруженный массив не\-мец\-ко-рус\-ских параллельных текстов дает возможность 
лингвистам найти в~нем более~16~тыс.\ пар, включая более~2~тыс.\ пар, 
содержащих словоформы леммы sollen~\cite{10-zat}. Для извлечения объектов 
интерпретации в~МБЗ реализован поиск пар как по словоформам, так и~по 
леммам. Поиск можно вести как по тексту одного произведения, так и~по всем 
загруженным текстам.
  
  Например, если задать поиск пар только в~произведении Ф.~Кафки 
<<Процесс>> по словоформе sollte, то будет найдена~61~пара с~исследуемой 
ЯЕ, четыре из которых приведены в~табл.~2 (включая пару из табл.~1).
  

  По умолчанию каждая пара, найденная по запро\-су лингвиста, является для 
него объектом семанти\-че\-ской интерпретации. Однако лингвист может
переформировать объект интерпретации, объединяя несколько пар. Например, 
если для интерпретации выбрана найденная пара №\,3433, то она может быть 
объединена с~соседними по текс\-ту книги парами в~единый объект 
интерпретации (табл.~3), что обеспечивает тот уровень его локализации 
в~параллельных текс\-тах произведений, который необходим для проведения 
исследования и~извлечения новых знаний. При этом объектом исследования 
является модальный глагол в~паре №\,3433, а~остальные пары образуют только 
контекст, необходимый для формирования аннотации на второй стадии 
итерации. Отметим, что в~паре №\,3435, которая входит и~в поисковую выдачу 
(см.\ табл.~2), и~в~объект интерпретации (см.\ табл.~3), есть тот же самый 
модальный глагол, но это словоупотребление будет рас\-смат\-ри\-вать\-ся 
и~аннотироваться в~рамках уже другого объекта интерпретации и~в~результате 
его семантического анализа будет сформирована другая аннотация.




\section{Аннотирование объектов интерпретации}
    
  Для каждого объекта интерпретации с~исследуемыми ЯЕ, найденного на 
стадии~1 (включая возможное объединение соседних пар), формируется\linebreak 
аннотация. Как уже отмечалось выше, метод анно\-ти\-ро\-ва\-ния, используемый 
в~проекте, отличается тем, что допускает возможность незавершенности\linebreak 
процесса аннотирования. Именно незавершенные аннотации позволяют 
зафиксировать неполноту (лакуны) в~ССЗ о языке и~о модальных глаголах, 
в~частности. Ключевым элементом второй стадии является метод описания 
лакун, рассматриваемый ниже.
  
  В процессе формирования аннотации происходит сопоставление значения 
глагола в~каждом объекте интерпретации со списком значений  
в~не\-мец\-ко-рус\-ском словаре~\cite{14-zat}, который был выбран как 
авторитетное описание ССЗ о~значениях немецких модальных глаголов (как 
исследуемых ЯЕ) и~их переводов на русский язык. 

В~на\-сто\-ящее время 
в~словаре описано~13~значений исследуемого модального глагола sollen, 
которые обозначены в~МБЗ следующими рубриками: sollen-01,\ $\ldots$,  
sollen-13 (ряд значений рассматривается в~работе~\cite{9-zat}). Начальное 
состояние типологии (до аннотирования) включает рубрики только тех 
значений модальных глаголов, которые описаны в~словаре. Основная цель 
лингвистического аннотирования со\-сто\-ит в~пополнении типологии и~словаря 
с~по\-мощью МБЗ.
  
  Если значение глагола в~объекте интерпретации совпадает с~одним из 
этих~13~значений, то в~аннотации этого объекта интерпретации проставляется 
рубрика с~соответствующим номером. Если значение глагола не совпадает ни 
с~одним из этих~13~значений, то в~аннотацию этого объекта интерпретации 
проставляется специальная руб\-ри\-ка sollen-х, которая говорит 
о~незавершенности процесса аннотирования, а~также о~ее причине: отсутствие 
в~типологии рубрики для того значения исследуемой ЯЕ, которое было 
найдено в~объекте интерпретации. 
  
  Таким образом, хранимые в~МБЗ незавершенные аннотации, 
соответствующие им объекты интерпретации и~состояние типологии на каждой 
итера\-ции позволяют зафиксировать текущую неполноту ССЗ о~модальных 
глаголах, а~также причину неполноты (отсутствие найденного нового 
значения), если проставлена специальная рубрика с~литерой~<<х>>. 
  
  Пример незавершенной аннотации приведен в~[10, табл.~3]. 
Последующий семантический анализ незавершенных аннотаций с~одним и~тем 
же тегом sollen-x и~соответствующих им объектов интерпретации может 
привести к~тому, что лингвисты извлекут из них несколько новых значений 
глагола sollen ($=$~не описанных в~словаре и~в~текущем состоянии 
пополняемой типологии). Таким образом, одной и~той же специальной  
руб\-ри\-кой-те\-гом \mbox{sollen-x} могут быть отмечены аннотации с~несколькими 
новыми значениями модального глагола.
  
  В~процессе формирования аннотации возможна такая ситуация, когда 
значение модального глагола в~некотором объекте интерпретации не удается 
определить без проведения дополнительного исследования на сле\-ду\-ющей, 
третьей, стадии, вы\-пол\-ня\-емой после аннотирования. В~этом случае 
в~незавершенной аннотации ставится специальная рубрика sollen-TBD (to be 
defined), которая помечает аннотации и~соответствующие объекты 
интерпретации, требующие проведения дополнительного исследования. 

Аналогичная ситуация возникает, когда в~процессе аннотирования не удается 
определить смысл объекта интерпретации в~целом. В~этом случае ставится 
специальная рубрика unknown.
  
  Предметом анализа на третьей стадии являются те незавершенные 
аннотации, в~которых указано, что не удалось определить значение немецкого 
модального глагола, т.\,е.\ была проставлена рубрика с~аббревиатурой TBD, или 
смысл всего объекта интерпретации, т.\,е.\ была проставлена рубрика unknown.
  
\section{Пополнение типологии}
    
  Для пополнения типологии значений немецких модальных глаголов была 
разработана иерархическая трехуровневая структура (см.\ рисунок). Эта типология 
может обновляться лингвистами в~двух ситуациях:
  \begin{enumerate}[(1)]
    \item описано новое значение модального глагола и~тогда типология 
обновляется на среднем уровне ее структуры;
    \item зафиксирован новый переводной эквивалент на русском языке для 
некоторого значения модального глагола и~тогда типология об\-нов\-ля\-ет\-ся на 
нижнем уровне, где описываются эквиваленты.
    \end{enumerate}
    
  

  На рисунке показаны три уровня типологии. Верхний (постоянный) уровень 
включает немецкие модальные глаголы d$\ddot{\mbox{u}}$rfen, 
k$\ddot{\mbox{o}}$nnen, m$\ddot{\mbox{o}}$gen, lassen, 
m$\ddot{\mbox{u}}$ssen, sollen и~wollen (показаны только первые три глагола). 
Средний уровень включает номера значений немецких модальных глаголов. Число 
значений может увеличиваться в~процессе анализа незавершенных аннотаций. 
Нижний уровень включает для каждого значения глагола те его переводные 
эквиваленты на русском языке, которые описаны лингвистами в~процессе 
аннотирования.


  
  Пополнение типологии, т.\,е.\ выполнение чет\-вер\-той стадии итерации, 
основано на процессе доработки незавершенных аннотаций, имеющих в~МБЗ 
специальную рубрику с~литерой~<<х>>. Для каж\-дой такой аннотации сначала 
создается описание того нового значения модального глагола, которое было 
извлечено из незавершенной аннотации. Например, если было извлечено новое 
значение глагола sollen, а~на среднем уровне текущего состояния типологии 
было описано~13~его значений, то
тогда в~нее добавляется новая рубрика для 
14-го зна-\linebreak\vspace*{-12pt}

{ \begin{center}  %fig1
 \vspace*{12pt}
   \mbox{%
 \epsfxsize=78.409mm 
 \epsfbox{zac-1.eps}
 }


\end{center}


\noindent
{\small{Структура пополняемой типологии немецких модальных глаголов и~их переводных 
эквивалентов на русском языке}}
}

%\vspace*{9pt}

%\addtocounter{figure}{1}

\noindent
 чения на этом уровне. При этом в~незавершенной аннотации 
специальная рубрика sollen-х заменяется на sollen-14. Таким образом, в~случае 
описания на четвертой стадии нового значения незавершенная аннотация 
становится завершенной благодаря появлению в~типологии новой рубрики.
  
  Возможны ситуации, когда лингвисты не вводят в~типологию рубрику 
с~новым номером для обнаруженного нового значения немецкого модального 
глагола, а~меняют дефиницию уже существующей рубрики, т.\,е.\ используют 
существующий номер для нового значения. В~этом случае необходимо 
про\-вес\-ти поиск по рубрике с~этим номером и~отредактировать все найденные 
аннотации (кроме случаев увеличения объема значения со\-от\-вет\-ст\-ву\-ющей
руб\-рики).
  
  На пятой стадии вычисляются сле\-ду\-ющие чис\-ло\-вые па\-ра\-мет\-ры, 
характеризующие со\-сто\-яние процесса пополнения типологии на момент 
завершения итерации:
  \begin{itemize}
  \item число объектов интерпретации, содержащих исследуемые ЯЕ, 
хранящихся в~МБЗ (оно может увеличиться, если в~процессе исследования 
загружаются новые текс\-ты);
  \item число аннотированных объектов интерпретации, в~том чис\-ле 
вклю\-ча\-ющих специальные рубрики;
  \item число новых значений ис\-сле\-ду\-емых ЯЕ, извлеченных из объектов 
интерпретации и~вклю\-чен\-ных в~типологию;
  \item число новых переводных эквивалентов, вклю\-чен\-ных в~типологию.
  \end{itemize}
  
\section{Заключение}
    
  Выполнение лингвистами итерационно повторяемых стадий 
с~использованием МБЗ основано на новом подходе к~уточнению 
лингвистических типологий. Он предполагает определение и~применение 
специальных руб\-рик, фиксирующих причины, по которым процесс 
аннотирования отдельных объектов интерпретации не был завершен. 
  
  Как было показано на рисунке, конструкции русского языка с~модальным 
значением, возникающие в~переводе конструкций немецкого языка 
с~модальными глаголами, группируются по значениям этих глаголов, так как 
цель исследования со\-сто\-ит в~уточнения типологии их значений. Однако 
в~упомянутой ранее работе~\cite{9-zat} одновременно ставится еще одна цель: 
построение типологии русских конструкций с~модальным значением для 
описания соответствующего фрагмента грамматики русского языка. Эта цель 
уже относится только к~русскому языку и~не связана со средствами выражения 
модальности в~немецком языке. Однако именно переводные эквиваленты, 
включаемые в~нижний уровень типологии на рисунке, являются исходной 
информацией для создания новой типологии конструкций с~модальным 
значением для описания соответствующего фрагмента грамматики русского 
языка.
  
  Предлагаемый подход дает возможность не только уточнять уже 
существующие типологии, но и~формировать новые типологии с~нуля, что 
планируется продемонстрировать при продолжении проекта на примере 
построения типологии русских конструкций с~модальным значением 
с~использованием МБЗ. В~существующих методах уточнения лингвистических 
типологий в~процессе аннотирования не допускается никаких изменений 
в~типологии. После завершения работ по аннотированию в~типологию могут 
быть внесены изменения, но в~процессе аннотирования она остается 
неизменной до его окончания~\cite{15-zat}. Такие подходы не позволяют 
формировать типологии с~нуля в~процессе аннотирования, так как для его 
обеспечения уже требуется иметь некоторый начальный вариант типологии. 
  
 {\small\frenchspacing
 {%\baselineskip=10.8pt
 \addcontentsline{toc}{section}{References}
 \begin{thebibliography}{99}
 
 \bibitem{3-zat} %1
\Au{Дурново А.\,А., Зацман~И.\,М., Лощилова~Е.\,Ю.} Кросс\-линг\-ви\-сти\-че\-ская 
база данных для аннотирования ло\-ги\-ко-се\-ман\-ти\-че\-ских отношений 
в~тексте~// Сис\-те\-мы и~средства информатики, 2016. Т.~26. №\,4. С.~124--137.
\bibitem{4-zat} %2
\Au{Зацман И.\,М., Мамонова~О.\,С., Щурова~А.\,Ю.} Обратимость 
и~альтернативность генерализации моделей перево\-да коннекторов 
в~параллельных текстах~// Сис\-те\-мы и~средства информатики, 2017. Т.~27. 
№\,2. С.~125--142.

\bibitem{2-zat} %3
\Au{Зацман И.\,М., Кружков~М.\,Г., Лощилова~Е.\,Ю.} Методы анализа 
частотности моделей перевода коннекторов и~обратимость генерализации 
статистических данных~// Системы и~средства информатики, 2017. Т.~27. №\,4. 
С.~164--176.
\bibitem{1-zat} %4
\Au{Kruzhkov M.\,G.} Approaches to annotation of discourse relations in linguistic 
corpora~// Информатика и~её применения, 2017. Т.~11. Вып.~4. С.~118--125.

\bibitem{5-zat}
\Au{Zatsman I., Buntman~N., Kruzhkov~M., Nuriev~V., Zalizniak Anna~A.} 
Conceptual framework for development of computer technology supporting  
cross-linguistic knowledge discovery~// 15th European Conference on Knowledge 
Management Proceedings.~--- Reading: Academic Publishing International Ltd., 
2014. Vol.~3. P.~1063--1071.
\bibitem{6-zat}
\Au{Zatsman I., Buntman~N.} Outlining goals for discovering new knowledge and 
computerised tracing of emerging meanings~// 16th European Conference on 
Knowledge Management Proceedings.~--- Reading: Academic Publishing 
International Ltd., 2015. P.~851--860.
\bibitem{7-zat}
\Au{Zatsman I., Buntman~N., Coldefy-Faucard~A., Nuriev~V.} WEB knowledge 
base for asynchronous brainstorming~// 17th European Conference on Knowledge 
Management Proceedings.~--- Reading: Academic Publishing International Ltd., 
2016. Vol.~1. P.~976--983.
\bibitem{8-zat}
\Au{Zatsman~I.} Goal-oriented creation of individual knowledge: Model and 
information technology~// 19th European Conference on Knowledge Management 
Proceedings.~--- Reading: Academic Publishing International Ltd., 2018. Vol.~2. 
P.~947--956.
\bibitem{9-zat}
\Au{Добровольский Д.\,О., Зализняк~Анна~А.} Немецкие конструкции 
с~модальными глаголами и~их русские соответствия: проект надкорпусной 
базы данных~// Компьютер\-ная лингвистика и~интеллектуальные технологии: 
По мат-лам Междунар. конф. <<Диалог>>.~--- М.: РГГУ, 2018. Т.~17(24). С.~172--184.
\bibitem{10-zat}
\Au{Зацман И.\,М.} Стадии целенаправленного извлечения знаний, 
имплицированных в~параллельных текстах~// Системы и~средства 
информатики, 2018. Т.~28. №\,3. С.~175--188.
\bibitem{11-zat}
Параллельный немецкий корпус. {\sf http://www. ruscorpora.ru/search-para-de.html}.
\bibitem{12-zat}
\Au{Зализняк Анна~А., Зацман~И.\,М., Инькова~О.\,Ю.} Надкорпусная база 
данных коннекторов: построение сис\-те\-мы терминов~// Информатика и~её 
применения, 2017. Т.~11. Вып.~1. С.~100--106.
\bibitem{13-zat}
Handbook of linguistic annotation~/ Eds. N.~Ide,  J.~Pustejovsky.~--- Dordrecht, 
The Netherlands: Springer Science\;+\;Business Media, 2017. 1468~p.
\bibitem{14-zat}
Немецко-русский словарь: актуальная лексика~/ Под ред. 
Д.\,О.~Добровольского.~--- М.: Лексрус, 2019 (в~печати).
\bibitem{15-zat}
\Au{Zufferey S., Degand~L.} Annotating the meaning of discourse connectives in 
multilingual corpora~// Corpus Linguist. Ling., 2013. Vol.~13. Iss.~2. 
P.~399--423. doi: 10.1515/cllt-2013-0022.
 \end{thebibliography}

 }
 }

\end{multicols}

\vspace*{-3pt}

\hfill{\small\textit{Поступила в~редакцию 15.01.19}}

%\vspace*{8pt}

%\pagebreak

\newpage

\vspace*{-28pt}

%\hrule

%\vspace*{2pt}

%\hrule

%\vspace*{-2pt}

\def\tit{GOAL-ORIENTED DEVELOPMENT OF~LINGUISTIC KNOWLEDGE SYSTEMS:
IDENTIFYING AND~FILLING OF~LACUNAE}

\def\titkol{Goal-oriented development of linguistic knowledge systems:
Identifying and filling of lacunae}

\def\aut{I.\,M.~Zatsman}

\def\autkol{I.\,M.~Zatsman}

\titel{\tit}{\aut}{\autkol}{\titkol}

\vspace*{-11pt}


\noindent
    Institute of Informatics Problems, Federal Research Center ``Computer Science 
and Control'' of the Russian Academy of Sciences, 44-2~Vavilov Str., Moscow 
119333, Russian Federation

\def\leftfootline{\small{\textbf{\thepage}
\hfill INFORMATIKA I EE PRIMENENIYA~--- INFORMATICS AND
APPLICATIONS\ \ \ 2019\ \ \ volume~13\ \ \ issue\ 1}
}%
 \def\rightfootline{\small{INFORMATIKA I EE PRIMENENIYA~---
INFORMATICS AND APPLICATIONS\ \ \ 2019\ \ \ volume~13\ \ \ issue\ 1
\hfill \textbf{\thepage}}}

\vspace*{6pt}


        
    
    \Abste{The description of the process of goal-oriented development of linguistic 
typologies as forms of knowledge representation about linguistic units in question is 
given. The task is to identify and fill the lacunae in the system of modern knowledge 
about the linguistic units. For identifying the lacunae, the selection of a reliable 
authority that reflects the current level of knowledge in the relevant subject area is 
necessary. The process of identifying the lacunae is a~type of annotation of texts 
containing the linguistic units, using methods and means of informatics. As the 
source of new knowledge, parallel texts are used to fill the lacunae. They include 
original literary works and their translations. The objective of the paper is to describe 
the approach to identifying the lacunae in the process of annotating parallel texts 
containing the linguistic units and their translations.}
    
    
    \KWE{linguistic typology; parallel texts; computational linguistics; discovering 
new knowledge; information technology; goal-oriented development of typologies}
    
\DOI{10.14357/19922264190113}

%\vspace*{-14pt}

\Ack
\noindent
The work was fulfilled at the 
Institute of Informatics Problems, Federal Research Center ``Computer Science 
and Control'' of the Russian Academy of Sciences with partial support of the Russian
Foundation for Basic Research (project  18-07-00192).




%\vspace*{6pt}

  \begin{multicols}{2}

\renewcommand{\bibname}{\protect\rmfamily References}
%\renewcommand{\bibname}{\large\protect\rm References}

{\small\frenchspacing
 {%\baselineskip=10.8pt
 \addcontentsline{toc}{section}{References}
 \begin{thebibliography}{99}
 
  \bibitem{3-zat-1} %1
  \Aue{Durnovo, A., I.~Zatsman, and E.~Loshchilova.} 2016.  
Krosslingvisticheskaya baza dannykh dlya annotirovaniya logiko-semanticheskikh 
otnosheniy v~tekste [Cross-linguistic database for annotating logical-semantic 
relations in the text]. \textit{Sistemy i~Sredstva Informatiki~--- Systems and Means of 
Informatics} 26(4):124--137.
  \bibitem{4-zat-1} %2
  \Aue{Zatsman, I., O.~Mamonova, and A.~Shchurova.} 2017. Obratimost' 
i~al'ternativnost' generalizatsii modeley pe\-re\-vo\-da konnektorov v~parallel'nykh 
tekstakh [Reversibility and alternativeness of generalization of connective translation 
models in parallel texts]. \textit{Sistemy i~Sredstva Informatiki~--- Systems and 
Means of Informatics} 27(2):125--142.
 
  \bibitem{2-zat-1} %3
  \Aue{Zatsman, I., M.~Kruzhkov, and E.~Loshchilova.} 2017. Metody analiza 
chastotnosti modeley perevoda konnektorov i~obratimost' generalizatsii 
statisticheskikh dannykh [Methods of frequency analysis of connectives translations 
and reversibility of statistical data generalization]. \textit{Sistemy i~Sredstva 
Informatiki~--- Systems and Means of Informatics} 27(4):164--176.

 \bibitem{1-zat-1} %4
  \Aue{Kruzhkov, M.} 2017. Approaches to annotation of discourse relations in linguistic 
corpora. \textit{Informatika i~ee Primeneniya~--- Inform. Appl.} 11(4):118--125.
 
  \bibitem{5-zat-1}
   \Aue{Zatsman, I., N.~Buntman, M.~Kruzhkov, V.~Nuriev, and Anna 
A.~Zalizniak.} 2014. Conceptual framework for development of computer 
technology supporting cross-linguistic knowledge discovery. \textit{15th European 
Conference on Knowledge Management Proceedings}. Reading: Academic 
Publishing International Limited. 3:1063--1071.
  \bibitem{6-zat-1}
  \Aue{Zatsman, I., and N.~Buntman.} 2015. Outlining goals for discovering new 
knowledge and computerised tracing of emerging meanings discovery. \textit{16th 
European Conference on Knowledge Management Proceedings}. Reading: Academic 
Publishing International Limited. 851--860.
  \bibitem{7-zat-1}
  \Aue{Zatsman, I., N.~Buntman, A.~Coldefy-Faucard, and V.~Nuriev.} 2016.
  WEB knowledge base for asynchronous brainstorming. \textit{17th European 
Conference on Knowledge Management Proceedings}. Reading: Academic 
Publishing International Limited. 1:976--983.
  \bibitem{8-zat-1}
  \Aue{Zatsman, I.} 2018. Goal-oriented creation of individual knowledge: Model 
and information technology. \textit{19th European Conference on Knowledge 
Management Proceedings}. Reading: Academic Publishing International Limited. 
2:947--956.
  \bibitem{9-zat-1}
  \Aue{Dobrovol'skij, D.\,O., and Anna A.~Zalizniak.} 2018. Nemetskiye konstruktsii 
s~modal'nymi glagolami i~ikh rus\-skie sootvetstviya: proekt nadkorpusnoy bazy dannykh 
[German constructions with modal verbs and their Russian correlates: A~supracorpora database 
project]. 
\textit{Computer Linguistics and Intellectual Technologies: Conference 
(International) ``Dialog'' Proceedings}. Moscow: RGGU. 17(24):172--184.
  \bibitem{10-zat-1}
  \Aue{Zatsman, I.} 2018. Stadii tselenapravlennogo izvlecheniya znaniy, 
implitsirovannykh v~parallel'nykh tekstakh [Stages of goal-oriented discovery of 
knowledge implied in parallel texts]. \textit{Sistemy i~Sredstva Informatiki~--- 
Systems and Means of Informatics} 28(3):175--188.
  \bibitem{11-zat-1}
  \textit{Parallel'nyy nemetskiy korpus} [Parallel German corpus]. Available at: {\sf 
http://www.ruscorpora.ru/search-para-de.\linebreak html} (accessed January~18, 2019).
  \bibitem{12-zat-1}
  \Aue{Zaliznyak, Anna~A., I.~Zatsman, and O.~Inkova.} 2017. Nadkorpusnaya 
baza dannykh konnektorov: postroenie sistemy terminov [Supracorpora database on 
connectives: Term system development]. \textit{Informatika i~ee Primeneniya~--- 
Inform. Appl.} 11(1):100--106.
  \bibitem{13-zat-1}
  Ide, N., and J.~Pustejovsky, eds. 2017. \textit{Handbook of linguistic annotation}. 
Dordrecht, The Netherlands: Springer Science\;+\;Business Media. 1468~p.
  \bibitem{14-zat-1}
  Dobrovol'skij, D.\,O., ed. 2019 (in press). \textit{Nemetsko-russkiy slovar': 
aktual'naya leksika} [German-Russian dictionary: Actual vocabulary]. Moscow: 
Leksrus.
  \bibitem{15-zat-1}
  \Aue{Zufferey, S., and L.~Degand.} 2013. Annotating the meaning of discourse 
connectives in multilingual corpora. \textit{Corpus Linguist. Ling.} 
13(2):399--423. doi: 10.1515/cllt-2013-0022.
  \end{thebibliography}

 }
 }

\end{multicols}

\vspace*{-6pt}

\hfill{\small\textit{Received January 15, 2019}}

%\pagebreak

%\vspace*{-18pt}  


\Contrl

\noindent
\textbf{Zatsman Igor M.} (b.\ 1952)~--- Doctor of Science in technology, Head 
of Department, Institute of Informatics Problems, Federal Research Center 
``Computer Science and Control'' of the Russian Academy of Sciences,  
44-2~Vavilov Str., Moscow 119333, Russian Federation; 
\mbox{izatsman@yandex.ru}
\label{end\stat}

\renewcommand{\bibname}{\protect\rm Литература} 
    
    