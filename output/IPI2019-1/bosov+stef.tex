\def\stat{bosov+stef}

\def\tit{УПРАВЛЕНИЕ ВЫХОДОМ СТОХАСТИЧЕСКОЙ ДИФФЕРЕНЦИАЛЬНОЙ СИСТЕМЫ 
ПО~КВАДРАТИЧНОМУ КРИТЕРИЮ. II.~ЧИСЛЕННОЕ РЕШЕНИЕ УРАВНЕНИЙ 
ДИНАМИЧЕСКОГО ПРОГРАММИРОВАНИЯ$^*$}

\def\titkol{Управление выходом стохастической дифференциальной системы 
по квадратичному критерию. II %.~Численное решение уравнений  
%динамического программирования
}

\def\aut{А.\,В.~Босов$^1$, А.\,И.~Стефанович$^2$}

\def\autkol{А.\,В.~Босов, А.\,И.~Стефанович}

\titel{\tit}{\aut}{\autkol}{\titkol}

\index{Босов А.\,В.}
\index{Стефанович А.\,И.}
\index{Bosov A.\,V.}
\index{Stefanovich A.\,I.}


{\renewcommand{\thefootnote}{\fnsymbol{footnote}} \footnotetext[1]
{Работа выполнена при частичной поддержке РФФИ (проект 16-07-00677).}}


\renewcommand{\thefootnote}{\arabic{footnote}}
\footnotetext[1]{Институт проб\-лем информатики Федерального исследовательского центра 
<<Информатика и~управ\-ле\-ние>> Российской академии наук, \mbox{AVBosov@ipiran.ru}}
\footnotetext[2]{Институт проблем информатики Федерального исследовательского центра 
<<Информатика и~управ\-ле\-ние>> Российской академии наук, \mbox{AStefanovich@frccsc.ru}}

\vspace*{-8pt}

 
  

\Abst{Представлена вторая часть исследования задачи оптимального управления 
для диффузионного процесса Ито и~линейного управляемого выхода. 
Оптимальное управление выходом $dz_t\hm= a_t y_t \,dt\hm+b_t z_t \,dt\hm+ 
c_tu_t\,dt\hm+\sigma_t 
\,dw_t$ стохастической дифференциальной системы с~состоянием $dy_t\hm= 
A_t(y_t)\,dt +\Sigma_t (y_t)\, dv_t$ и~квадратичным критерием качества, 
определяемое функцией Беллмана вида $V_t(y,z)\hm= \alpha_t z^2\hm+\beta_t(y) 
z\hm+\gamma_t(y)$, рассчитывается путем приближенного решения сеточными 
методами дифференциальных уравнений для коэффициентов~$\alpha_t$, 
$\beta_t(y)$ и~$\gamma_t(y)$. Подробно рассмотрен модельный пример, 
опирающийся на простую дифференциальную модель для показателя RTT (Round-Trip Time)
сетевого протокола TCP (Transmission Control Protocol). 
Приводятся результаты численного эксперимента, 
позволяющие оценить трудности практической реализации оптимального 
решения и~обозначить задачи дальнейшего исследования.}

\KW{стохастическое дифференциальное уравнение; оптимальное управление; 
динамическое программирование; функция Беллмана; уравнение Риккати; 
линейные уравнения параболического типа}

\DOI{10.14357/19922264190102}
  
\vspace*{-4pt}


\vskip 10pt plus 9pt minus 6pt

\thispagestyle{headings}

\begin{multicols}{2}

\label{st\stat}

\section{Введение}
     
     В работе~\cite{1-b} получены аналитические соотноше\-ния, 
описывающие оптимальное решение в~задаче\linebreak управления линейным выходом 
стохастической дифференциальной системы по квадратичному критерию 
качества. Оптимизируемая динамическая система описывается двумя 
уравнениями: нелинейным стохастическим дифференциальным уравнением 
Ито для состояния и~линейным уравнением для управляемого выхода. 
Квадратичный целевой функционал обеспечил возможность решения задачи 
методом динамического программирования: получены аналитические 
выражения для функции Беллмана и, как следствие, для оптимального 
управления и~для значения функционала качества. Соответствующие 
соотношения содержат, как и~всегда в~подобных задачах, решения 
определенных дифференциальных уравнений (обыкновенных и~в~частных 
производных). Формальная постановка задачи и~основной теоретический 
результат кратко сформулированы в~следующем разделе статьи. В~данной 
работе исследование этой же задачи продолжено, но внимание сосредоточено 
на аспектах практической реализации результата.
     
     С фундаментальной точки зрения в~сравнении с~исходной постановкой 
задачи оптимизации в~стохастической системе поиск решений 
дифференциальных уравнений~--- это удовлетворительный итог 
исследования, достаточный для перехода в~практическую область, т.\,е.\ 
проведения расчетов, модельных как минимум, а то и~с реальными данными. 
На самом деле нюансы численной реализации полученного в~\cite{1-b} 
оптимального управления в~совокупности сами по себе представляют 
исследовательский вызов. В~полном объеме ответить на него пока не 
удается, но начальные результаты, приближенные алгоритмы, модельные 
расчеты получены и~составляют предмет настоящей работы. 

Итоги 
подводятся в~заключении, где в~том числе обозначены перспективы 
дальнейших исследований.

\section{Оптимальное решение задачи управления выходом}

     Следуя~\cite{1-b}, будем предполагать рас\-смат\-ри\-ва\-емые далее 
случайные функции скалярными. Скалярная модель позволяет обойти 
трудности численной реализации, неизбежно возникающие при рос\-те 
размерности. Это принципиально важно для данного этапа исследования, 
поскольку дальнейшие усилия предполагаются в~направлении 
совершенствования предложенных здесь методов приближенного численного 
решения, в~том числе с~учетом возможности роста размерности.
     
     Итак, рассматривается состояние~$y_t$ стохастической 
дифференциальной системы, описываемое нелинейным стохастическим 
дифференциальным уравнением Ито:
     \begin{equation*}
     dy_t=A_t\left( y_t\right) dt+\Sigma_t \left( y_t\right) dv_t\,,\enskip y_0=Y\,,
     %\label{e1-b}
     \end{equation*}
где $v_t$~--- стандартный винеровский процесс; $Y$~--- случайная величина 
с~конечным вторым моментом; функции $A_t$ и~$\Sigma_t$ удовлетворяют 
условиям Ито, обеспечивающим существование единственного 
решения~\cite{2-b}.

     С состоянием этой системы связан выход, который описывается 
функцией~$z_t$, линейно связанной с~$y_t$:
     \begin{equation}
     dz_t=a_t y_t\,dt+b_t z_t \,dt+c_t u_t \,dt+\sigma_t \,dw_t,\enskip z_0=Z,
     \label{e2-b}
     \end{equation}
где $w_t$~--- не зависящий от $v_t$, $Y$ и~$Z$ стандартный винеровский 
процесс; $Z$~--- случайная величина с~конечным вторым моментом; $u_t$~--- 
допустимое управ\-ле\-ние. Функции~$a_t$, $b_t$, $c_t$ и~$\sigma_t$ 
предполагаются ограниченными, процесс управления~--- допустимым 
неупреждающим~\cite{2-b}, что обеспечивает существование решения 
уравнения~(\ref{e2-b}) для любого допустимого управления.
     
     Используется целевой функционал следующего вида:
     \begin{multline*}
     \!J\left( U_0^T\right) =E\!\left\{ \int\limits_0^T\!\big( S_t\left( s_t y_t -g_t z_t-h_t 
u_t\right)^2+G_t z_t^2+{}\right.\\
\left.{}+H_t u_t^2\big)\,dt+
 S_T\left( s_T y_T-g_T z_T\right)^2+G_T z_T^2
      \vphantom{\int\limits_0^T}\right\}\,,\\
     U_0^T=\left\{ u_t, 0\leq t\leq T
     \right\}\,,
     %\label{e3-b}
     \end{multline*}
где $S_t$, $G_t$ и~$H_t$~--- неотрицательные функции.

     Решение задачи поиска $u_t^*$~--- допустимого управ\-ле\-ния, 
доставляющего минимум квадратичному функционалу $J(U_0^T)$, 
составляют следующие соотношения.
     
     Функция Беллмана $V_t(y,z)$ может быть пред\-став\-ле\-на в~виде:
     \begin{equation*}
     V_t(y,z)=\alpha_t z^2+\beta_t(y) z+\gamma_t(y)\,,
     %\label{e4-b}
     \end{equation*}
а оптимальное управление
\begin{multline}
u_t^* =u_t^*(y,z)={}\\[2pt]
{}=-\fr{1}{2}\left( S_th_t^2+H_t\right)^{-1}\left( c_t \left(
2\alpha_t z+\beta_t(y)\right)+{}\right.\\[2pt]
\left.{}+2S_t \left( s_t y-g_t z\right) h_t\right)\,,\enskip
y=y_t\,,\enskip z=z_t\,.
\label{e5-b}
\end{multline}
     
     Коэффициенты $\alpha_t$, $\beta_t(y)$ и~$\gamma_t(y)$ задаются 
следующими дифференциальными уравнениями:
     \begin{multline}
     \fr
{\partial \alpha_t}{\partial t}+
2\alpha_t
     \left( b_t-\left( S_t h_t^2+H_t\right)^{-1} c_t S_t h_t g_t\right)+{}\\[2pt]
     {}+\left( S_t-\left( S_t h_t^2+H_t\right)^{-1} S_t^2 h_t^2\right) g_t^2+{}\\[2pt]
     {}+G_t-
     \left( S_t h_t^2+H_t\right)^{-1} c_t^2 \alpha_t^2=0\,,\\[2pt]
     \alpha_T=S_T g_T^2+G_T\,;
     \label{e6-b}
     \end{multline}

\vspace*{-12pt}

\noindent
\begin{multline}
\fr{\partial \beta_t(y)}{\partial y}+\fr{1}{2}\Sigma_t^2(y)
\fr{\partial^2\beta_t(y)}{\partial y^2} 
+A_t(y)\fr{\partial \beta_t(y)}{\partial y}+{}\\[2pt]
{}+2\alpha_t\left( a_t+\left( S_t h_t^2+H_t\right)^{-1} c_t S_t h_t 
s_t\right)y+{}\\[2pt]
{}+\beta_t(y) \left( b_t-\left( S_t h_t^2+H_t\right)^{-1} c_t S_t h_t 
g_t\right)-{}\\[2pt]
{}-2\left( S_t-\left( S_t h_t^2+H_t\right)^{-1} S_t^2 h_t^2\right) s_t g_t y-{}\\[2pt]
{}-\left( 
S_t h_t^2+H_t\right)^{-1} c_t^2 \alpha_t \beta_t(y)=0\,,\\[2pt]
\beta_T(y)=-2S_T s_T g_T y\,;
\label{e7-b}
\end{multline}

\vspace*{-12pt}

\begin{multline}
\fr{\prt \gamma_t(y)}{\prt  t}+\fr{1}{2}\,\Sigma_t^2(y)\fr{\prt^2 
\gamma_t(y)}{\prt y^2} +\sigma^2_t \alpha_t +A_t(y) \fr{\prt \gamma_t(y)}{\prt 
y}+{}\\[2pt]
{}+ \beta_t(y)\left( a_t+\left( S_t h_t^2+H_t\right)^{-1} c_t S_t h_t s_t\right) 
y+{}\\[2pt]
{}+\left( S_t- \left( S_t h_t^2 +H_t\right)^{-1} S_t^2 h_t^2\right) s_t^2 y^2-{}\\[2pt]
{}-\fr{1}{4}\left( S_t h_t^2 +H_t\right)^{-1} c_t^2 \beta_t^2(y)=0\,,\\[2pt] 
\gamma_T(y)=S_T s_T^2 y^1\,.
\label{e8-b}
\end{multline}
     
     Уравнение~(\ref{e6-b}) является уравнением Риккати и~имеет 
единственное неотрицательное решение для всех $0\hm\leq t\hm\leq T$, так 
как предполагается $S_t h_t^2\hm+ H_t\hm>0$. 
Уравнения~(\ref{e7-b}) 
и~(\ref{e8-b}) представляют собой линейные уравнения в~частных 
производных второго порядка, относятся к~параболическому типу, поскольку 
$\Sigma_t^2(y)\hm>0$. Предполагается, что данные уравнения имеют 
на рассматриваемом интервале $0\hm\leq t\hm\leq T$ хотя бы одно 
ограниченное решение, которое и~определяет оптимальное решение 
рассматриваемой задачи оптимизации.
 

\section{Модельный эксперимент и~приближенное численное 
решение}
     
     Практическая реализация оптимального управ\-ле\-ния~(\ref{e5-b}), 
конечно, возможна только приближенно, рассматривать варианты, когда 
уравнения~(\ref{e6-b}) и~(\ref{e7-b}) могут быть решены аналитически, 
бесперспективно. Приближенное решение очевидным образом получается 
в~результате численного интегрирования уравнений~(\ref{e6-b})  
и~(\ref{e7-b}), а~при не\-об\-хо\-ди\-мости и~(\ref{e8-b}). Для реализации этого 
требуется модельный пример, который позволит, с~одной стороны, 
поэкспериментировать с~численными процедурами, сформировать 
представление об их точности и~ресурсоемкости. С~другой стороны, при 
выборе модельного примера следует ориентироваться на потенциальные 
прикладные области применения представленного результата. В~качестве 
такой возможной области видится область телекоммуникаций~--- 
популярный современный источник приложений для различных 
теоретических задач. Из моделей, исследуемых в~этой области, была выбрана 
простая модель для показателя RTT сетевого протокола 
TCP, предложенная в~\cite{3-b}. Опуская 
детали выбора параметров в~используемой диффузионной модели RTT, 
укажем только итоговое уравнение рассматриваемого модельного примера:
     \begin{multline}
     dy_t= \left( 1-0{,}1y_t\right) dt+0{,}5\sqrt{y_t}\,dv_t,\\ y_0=Y\sim 
N(15,9)\,.
     \label{e9-b}
     \end{multline}
Здесь $N(M,{\sf D})$~--- нормальное распределение со средним~$M$ 
и~дисперсией~${\sf D}$.
     
     Надо отметить, что такое уравнение (с~точ\-ностью до варьируемых 
параметров) изначально известно как модель Кок\-са--Ин\-гер\-со\-на--Рос\-са 
(Cox--Ingersoll--Ross model)~\cite{4-b} и~описывает эволюцию процентных 
ставок. Конечно, для описания реальных данных RTT модель  
типа~(\ref{e9-b}) представляет не более чем начальное приближение, но оно 
удачно уже потому, что дает основу для дальнейших обобщений, уточнений 
модели (см., например,~\cite{5-b}), поэтому удобна и~для целей настоящей 
работы. Свойства процесса~$y_t$ из~(\ref{e9-b}) хорошо изучены, 
в~частности имеется свойство неотрицательности выборочных значений, 
наличие эргодичности, известно предельное распределение и~переходная 
вероятность. Отметим, что начальные условия в~(\ref{e9-b}) выбраны так, 
что $M$ и~${\sf D}$ отличаются от моментных характеристик предельного 
распределения, т.\,е.\ управление ведется в~рамках переходного процесса, 
а~не в~стационарном режиме.
     
     Выход для~(\ref{e9-b}) задается уравнением
          \begin{multline}
     dz_t=0{,}1 y_t\, dt-z_t \,dt+u_t \,dt+dw_t\,,\\ z_0=Z\sim N(9, 9)\,,
     \label{e10-b}
     \end{multline}
    целевой функционал
\begin{multline*}
J\left( U_0^T\right) =E\left\{ \int\limits_0^T \left( \left( y_t-
z_t\right)^2+z_t^2+u_t^2\right) \,dt +{}\right.\\
\left.{}+\left( y_T -z_T\right)^2+z_T^2
     \vphantom{\int\limits_0^T}
     \right\}\,.
%\label{e11-b}
\end{multline*}
     
     Моделировалось $N\hm=1000$ траекторий $y_t$, $z_t^*$ и~$u_t^*$ для 
$T\hm=5$ и~$50$ (через~$z_t^*$ обозначен выход~(\ref{e10-b}), рассчитанный 
для $u_t\hm=u_t^*$). Кроме того, моделировались траектории $z_t^{\mathrm{prog}}$ 
и~$u_t^{\mathrm{prog}}$ для наилучшего программного управления $u_t^{\mathrm{prog}}\hm= E\{ 
u_t^*\}$ и~$z_t^0$ и~$u_t^0$ для неуправляемого выхода~(\ref{e10-b}), т.\,е.\ для 
$u_t\hm=0$. В~каждом случае путем осреднения по пучку траекторий 
оценивались величины~$J(U_0^t)$. Соответственно, в~результате можно 
увидеть не только конечные значения целевой функции для разных 
вариантов управления $J((U^*)_0^T)$, $J((U^{\mathrm{prog}})_0^T)$ и~$J((U^0)_0^T)$, 
но и~их формирование в~динамике.
     
     Для определения оптимального управления чис\-лен\-но решались 
уравнения для~$\alpha_t$ и~$\beta_t(y)$. Сначала было получено чис\-лен\-ное 
решение урав\-не\-ния~(\ref{e6-b}). Это уравнение Риккати, оно имеет 
единственное решение, и~вы\-чис\-ле\-ние этого решения не пред\-став\-ля\-ет труда, 
например неявным методом Эйлера.
     
     Далее решалось уравнение~(\ref{e7-b}). Как уже отмечалось, 
нахождение решений уравнений~(\ref{e7-b}) и~при  
необходимости~(\ref{e8-b}) (это уравнение можно использовать для 
определения качества оптимального управ\-ле\-ния) пред\-став\-ля\-ет 
определенные труд\-но\-сти. Для целей данной работы достаточно использовать 
традиционный подход к~чис\-лен\-но\-му решению~--- метод конечных 
разностей, тем более что варианты этого метода для урав\-не\-ния 
параболического типа дав\-но и~хорошо изучены~\cite{6-b}.
     
     Первое, что требуется сделать,~--- это ограничить область значения 
аргумента~$y$ для формирования численной схемы расчета. С~этой целью 
использовались~$N$~смоделированных методом Эйлера 
траекторий~(\ref{e9-b}): выборочные значения использовались для 
оценивания функций математического ожидания и~дисперсионной, далее 
задавалась  
$3\sigma$-труб\-ка, а~в~расчетах граница значений~$y$ задавалась путем 
добавления 10\% к~границе $3\sigma$-труб\-ки. Рисунок~1 иллюстрирует 
характерные траектории~(\ref{e9-b}), среднее значение процесса 
и~вычисленные границы (закрашенная серым область).

\begin{figure*} %fig1
\vspace*{1pt}
 \begin{center}  
  \mbox{%
 \epsfxsize=163mm 
 \epsfbox{bos-1.eps}
 }
\end{center}
\vspace*{-11pt}
\Caption{Выборочные траектории и~характеризация границы}
\end{figure*}

     Определение границы создает следующую проб\-ле\-му, обычную при 
применении любого сеточного метода, а~именно: отсутствие граничных 
условий для рассматриваемых уравнений. Этих условий нет, и~нет 
физических оснований для их обоснованного выбора. Из-за этого приходится 
выбирать граничные условия довольно волюнтаристски и~надеяться на 
устойчивость решения в~отношении этих граничных условий. 

Выбор 
граничного условия и~анализ его устойчивости выполнялся следующим 
образом. Были взяты два традиционных варианта граничных условий~--- 
в~задаче Дирихле: $\beta_t(y)\hm=0$, т.\,е.\ условие поглощения, и~в~задаче 
Неймана: $\partial \beta_t(y)/\partial y\hm=0$, т.\,е.\ условия отражения, для 
всех граничных точек. Далее уравнение решалось методом конечных 
разностей с~использованием явной и~неявной численных схем сначала для 
одного граничного условия, затем для второго, и~полученные решения 
сравнивались. Устойчивость в~таком случае означает совпадение (близость) 
решений в~заданной области за исключением границы.
     
     Для численного интегрирования в~области $[0,T]\cup \left[\, 
\underline{y}, \overline{y}\right]$ ($\underline{y}$ и~$\overline{y}$~--- 
определенные в~результате моделирования границы) формируется сетка 
$\{t_k, y_i\}$ с~шагами~$\delta_t$ (для интегрирования по переменной~$t$) 
и~$\delta_y$ (для интегрирования по переменной~$y$). С~использованием 
обозначения~$\beta_i^k$ для значения~$\beta_t(y)$ в~узле $(t_k, y_i)$ ее 
производные аппроксимировались следующим образом:
     \begin{gather*}
     \fr{\prt \beta_t(y)}{\prt t}\approx \fr{\beta_i^{k+1}-\beta_i^k}{\delta_t}\,;\quad
     \fr{\prt \beta_t(y)}{\prt y}\approx \fr{\beta^K_{i+1}-\beta^K_{i-
1}}{2\delta_y}\,;\\
     \fr{\prt^2\beta_t(y)}{\prt y^2}\approx \fr{\beta^K_{i+1}-
2\beta_i^K+\beta^K_{i-1}}{\delta_y^2}\,,
     \end{gather*}
где $K=k$ для явной схемы и~$K\hm=k\hm+1$~--- для неявной.
     
     Шаги интегрирования $\delta_t$ и~$\delta_y$ при анализе устойчивости 
выбирались разными, например явная схема обсчитывалась для 
$\delta_t\hm=0{,}0002$ и~$\delta_y\hm=0{,}1$, в~том числе с~учетом того, 
что для нее самой вопрос устойчивости ограничивает вариации шагов. 
Главное, что в~итоге подтвердилась устойчивость результатов численного 
интегрирования $\beta_t(y)$ по отношению к~граничным условиям по 
крайней мере для выбранной модели.
     
     Итоговый расчет выполнен для неявной схемы с~шагами 
$\delta_t\hm=0{,}001$ (в~том числе для приближенного 
вычисления~$\alpha_t$) и~$\delta_y\hm=0{,}01$. Результаты расчетов 
представлены на рис.~2--4, в~частности приведены примеры траекторий для 
выхода и~управлений $z_t^*, u_t^*, z_t^{\mathrm{prog}}, u_t^{\mathrm{prog}}$ и~$z_t^0, u_t^0$.



     Рисунок~4 демонстрирует ожидаемый проигрыш\linebreak неуправляемой 
системы $z_t^0, u_t^0$. Кроме того, обращает на себя внимание высокое 
качество про\-граммного управления и~выхода $z_t^{\mathrm{prog}}, u_t^{\mathrm{prog}}$. 
Объ\-ективных оснований последнее обстоятельство,\linebreak конечно, не имеет 
и~объясняется исключительно выбранными параметрами рассматриваемого 
модельного примера, прежде всего интенсивностями возмущений~--- 
коэффициентов при $v_t$ и~$w_t$, что в~свою очередь сделано для того, чтобы 
на рис.~4 изменения целевого функционала для разных стратегий управ\-ле\-ния 
можно было представить в~одном масштабе.
     
     Наконец, отметим, что существенных техниче\-ских трудностей при 
выполнении расчетов не возникло. Без дополнительных усилий сетка 
размещалась в~оперативной памяти: для интегрирования\linebreak
 с~выбранными 
шагами в~об\-ласти $[0,50]\cup [-10,40]$ было использовано всего лишь~2~ГБ 
памяти, что позволило провести расчеты на обычном персональном 
компьютере. При этом размещение сетки в~оперативной памяти обеспечило 
довольно быст\-рое проведение расчетов. Гораздо больше времен-\linebreak н$\acute{\mbox{ы}}$х ресурсов 
требуется для моделирования пучков траекторий, осред\-не\-ния~--- срав\-не\-ния 
свойств управ\-ле\-ний. Конечно, с~рос\-том размерности ситу-\linebreak\vspace*{-12pt}

\pagebreak

\end{multicols}

\begin{figure*} %fig2
\vspace*{1pt}
 \begin{center}  
  \mbox{%
 \epsfxsize=162.963mm 
 \epsfbox{bos-2.eps}
 }
\end{center}
\vspace*{-11pt}
\Caption{Выборочные траектории $z_t^*$~(\textit{1}), $z_t^{\mathrm{prog}}$~(\textit{2}) 
и~$z_t^0$~(\textit{3})}
%\end{figure*}
%\begin{figure*} %fig3
\vspace*{6pt}
 \begin{center}  
  \mbox{%
 \epsfxsize=163mm 
 \epsfbox{bos-3.eps}
 }
\end{center}
\vspace*{-11pt}
\Caption{Выборочные траектории $u_t^*$~(\textit{1}), $u_t^{\mathrm{prog}}$~(\textit{2}) 
и~$u_t^0$~(\textit{3})}
%\end{figure*}
%\begin{figure*} %fig4
\vspace*{6pt}
 \begin{center}  
  \mbox{%
 \epsfxsize=162.528mm 
 \epsfbox{bos-4.eps}
 }
\end{center}
\vspace*{-11pt}
\Caption{Динамика целевого функционала:
\textit{1}~--- $J((U^*)^t_0)$; \textit{2}~---  $J((U^0)_0^t)$; \textit{3}~--- 
$J((U^{\mathrm{prog}})^t_0)$}
\end{figure*}

\begin{multicols}{2}

\noindent
ация 
принципиально изменится и~затраты на чис\-лен\-ное интегрирование 
дифференциальных уравнений вырастут на порядки.

\vspace*{-9pt}



\section{Заключение}

\vspace*{-2pt}

     Настоящая статья содержит вторую, практическую часть исследования 
задачи оптимизации линейного выхода нелинейной дифференциальной 
системы по квадратичному критерию. В~отношении представленного 
результата следует отметить, что полученное решение представляет 
определенные вычислительные трудности при его приближенной реализации 
путем численного решения дифференциальных уравнений для 
соответствующих коэффициентов. 

Отсутствие физически обоснованных 
граничных условий потребовало значительных усилий для анализа 
численного решения на границе. Кроме того, очевидное обобщение 
постановки на многомерный случай  еще более 
усугубит вычислительные проблемы. Таким образом, затраты вы\-чис\-ли\-тель\-ных ресурсов даже на 
модельные расчеты в~скалярном случае оказались слишком значительными, 
чтобы не учитываться в~теоретической час\-ти решения. По этой причине, 
прежде чем двигаться с~рассмотренной задачей дальше, обобщать постановку 
на многомерный случай, нуж\-но предложить более действенные инструменты 
в~час\-ти чис\-лен\-ной реализации полученных точных решений. Возможности 
для этого имеются, а~соответствующие результаты должны составить 
ближайшие перспективы дальнейших исследований.
     
     Представленная в~данной работе задача оптимизации пока не 
исследовалась детально на предмет практического применения. 
Выполненные модельные расчеты~--- это, скорее, первый шаг в~на\-прав\-ле\-нии 
поиска реальных приложений. При этом основным представляется поиск 
моделей, обладающих практической ценностью и~достаточно исследованных 
в~рамках задач анализа. Источником таких моделей видится область 
инфотелекоммуникаций, активно развиваемая в~последние годы, в~част\-ности 
модели сетевых транспортных протоколов на основе скачкообразных 
марковских процессов~[7--10]. Применение рассмотренной 
оптимизационной задачи к~таким моделям также может стать 
содержательным продолжением представленного в~данной работе 
исследования.

{\small\frenchspacing
 {%\baselineskip=10.8pt
 \addcontentsline{toc}{section}{References}
 \begin{thebibliography}{99}
\bibitem{1-b}
\Au{Босов А.\,В., Стефанович~А.\,И.} Управление выходом 
стохастической дифференциальной системы по квад\-ра\-тич\-но\-му критерию. 
I.~Оптимальное решение методом динамического программирования~// 
Информатика и~её применения, 2018. Т.~12. Вып.~3. С.~99--106. doi: 
10.14357/19922264180314.
\bibitem{2-b}
\Au{Флеминг У., Ришел~Р.} Оптимальное управление 
детерминированными и~стохастическими системами~/ Пер. с~англ.~--- М.: 
Мир, 1978. 316~с. (\Au{Fleming~W.\,H., Rishel~R.\,W.} Deterministic and 
stochastic optimal control.~--- New York, NY, USA: Springer-Verlag, 1975. 
222~p.)
\bibitem{3-b}
\Au{Bohacek S., Rozovskii~B.} A~diffusion model of roundtrip time~// 
Comput. Stat. Data An., 2004. Vol.~45. Iss.~1. P.~25--50. 
doi: 10.1016/S0167-9473(03)00114-2.
\bibitem{4-b}
\Au{Cox J.\,C., Ingersoll~J.\,E., Ross~S.\,A.} A~theory of the term structure of 
interest rates~// Econometrica, 1985. Vol.~53. Iss.~2. P.~385--407. doi: 
10.2307/1911242.
\bibitem{5-b}
\Au{Bohacek S.} A~stochastic model of TCP and fair video transmission~// 
Proc. IEEE INFOCOM, 2003. P.~1134--1144. doi: 
10.1109/INFCOM.2003.1208950.
\bibitem{6-b}
\Au{Саульев В.\,К.} Интегрирование уравнений параболического типа 
методом сеток.~--- М.: Физматлит, 1960. 324~с.
\bibitem{7-b}
\Au{Борисов А.\,В., Миллер~Г.\,Б.}
Анализ и~фильтрация специальных марковских процессов в~дискретном 
времени. II.~Оптимальная фильтрация~// Автоматика и~телемеханика, 
2005. №\,7. С.~112--125. doi: 10.1007/s10513-005-0153-7.
\bibitem{8-b}
\Au{Борисов А.\,В., Миллер~Б.\,М., Семенихин~К.\,В.} Фильт\-ра\-ция 
марковского скачкообразного процесса по наблюдениям 
мультивариантного точечного процесса~// Автоматика и~телемеханика, 
2015. №\,2. С.~34--60. doi: 10.1134/S0005117915020034.
\bibitem{9-b}
\Au{Borisov A., Bosov~A., Miller~G.} Modeling and monitoring of RTP link on 
the receiver side~// Internet of things, smart spaces, and next generation
networks and systems~/ Eds. S.\,I.~Balandin, S.\,D.~Andreev,
Y.~Koucheryavy.~--- Lecture notes in computer science ser.~--- Springer, 2015. Vol.~9247. 
P.~229--241. doi: 10.1007/978-3-319-23126-6\_21.
\bibitem{10-b}
\Au{Борисов А.\,В.} Применение методов оптимальной фильтрации для 
оперативного оценивания состояний сетей массового обслуживания~// 
Автоматика и~телемеханика, 2016. №\,2. С.~115--141. doi: 
10.1134/S0005117916020053.
 \end{thebibliography}

 }
 }

\end{multicols}

\vspace*{-3pt}

\hfill{\small\textit{Поступила в~редакцию 07.06.18}}

\vspace*{8pt}

%\pagebreak

%\newpage

%\vspace*{-28pt}

\hrule

\vspace*{2pt}

\hrule

%\vspace*{-2pt}

\def\tit{STOCHASTIC DIFFERENTIAL SYSTEM OUTPUT CONTROL 
BY~THE~QUADRATIC CRITERION. II.~DYNAMIC PROGRAMMING 
EQUATIONS NUMERICAL SOLUTION}

\def\titkol{Stochastic differential system output control 
by~the~quadratic criterion. II.~Dynamic programming 
equations numerical solution}

\def\aut{A.\,V.~Bosov and~A.\,I.~Stefanovich}

\def\autkol{A.\,V.~Bosov and~A.\,I.~Stefanovich}

\titel{\tit}{\aut}{\autkol}{\titkol}

\vspace*{-11pt}


\noindent
Institute of Informatics Problems, Federal Research Center 
``Computer Science and Control'' of the 
Russian Academy of Sciences, 44-2~Vavilov Str., Moscow 119333, Russian Federation

\def\leftfootline{\small{\textbf{\thepage}
\hfill INFORMATIKA I EE PRIMENENIYA~--- INFORMATICS AND
APPLICATIONS\ \ \ 2019\ \ \ volume~13\ \ \ issue\ 1}
}%
 \def\rightfootline{\small{INFORMATIKA I EE PRIMENENIYA~---
INFORMATICS AND APPLICATIONS\ \ \ 2019\ \ \ volume~13\ \ \ issue\ 1
\hfill \textbf{\thepage}}}

\vspace*{6pt}


\Abste{The second part of the optimal control problem investigation for the Ito 
diffusion process and the controlled linear output is presented. Optimal control for 
output $dz_t= a_t y_t \,dt+b_t z_t \,dt+ c_t u_t\,dt+\sigma_t \,dw_t$ of the stochastic 
differential system $dy_t\hm= A_t(y_t)\,dt +\Sigma_t (y_t) \,dv_t$ and quadratic quality 
criterion defined by Bellman
function having form $V_t(y,z)\hm= \alpha_t 
z^2\hm+\beta_t(y) z\hm+\gamma_t(y)$ is determined numerically by an approximate 
solution to\linebreak\vspace*{-12pt}}

\Abstend{the grid methods of differential equations for the coefficients $\alpha_t$, 
$\beta_t(y)$, and~$\gamma_t(y)$. A~model experiment based on a~simple differential 
presentation for the RTT (Round-Trip Time)
parameter of the TCP (Transmission Control Protocol)
network protocol is considered in detail. 
The results of numerical simulation are given and allow one to assess the difficulties in 
the practical implementation of the optimal solution and define the tasks of further 
research.}

\KWE{stochastic differential equation; optimal control; dynamic programming; 
Bellman function; Riccati equation; linear differential equations of parabolic type}


\DOI{10.14357/19922264190102}

%\vspace*{-14pt}

\Ack
\noindent
This work was partially supported by the Russian Science Foundation (grant  
16-07-00677).



%\vspace*{6pt}

  \begin{multicols}{2}

\renewcommand{\bibname}{\protect\rmfamily References}
%\renewcommand{\bibname}{\large\protect\rm References}

{\small\frenchspacing
 {%\baselineskip=10.8pt
 \addcontentsline{toc}{section}{References}
 \begin{thebibliography}{99}
\bibitem{1-b-1}

\Aue{Bosov, A.\,V., and A.\,I.~Stefanovich.} 2018. Upravlenie vykhodom stokhasticheskoy differentsial'noy
sistemy po kvadratichnomu kriteriyu. I.~Optimal'noe reshenie
metodom dinamicheskogo programmirovaniya
[Stochastic differential system output control
by the quadratic criterion. I.~Dynamic
programming optimal solution]. \textit{Informatika i ee~Primeneniya~--- Inform. Appl.}  
12(3):99--106. doi: 
10.14357/19922264180314. 
\bibitem{2-b-1}
\Aue{Fleming, W.\,H., and R.\,W.~Rishel.} 1975. \textit{Deterministic and 
stochastic optimal control}. New York, NY: Springer-Verlag. 222~p.
\bibitem{3-b-1}
\Aue{Bohacek, S., and B.~Rozovskii.} 2004. A~diffusion model of roundtrip 
time. \textit{Comput. Stat.  Data An}. 45(1):25--50.
doi: 10.1016/S0167-9473(03)00114-2.
\bibitem{4-b-1}
\Aue{Cox, J.\,C., J.\,E.~Ingersoll, and S.\,A.~Ross.} 1985. A~theory of the 
term structure of interest rates. \textit{Econometrica} 53:385--407.
doi: 10.2307/1911242.
\bibitem{5-b-1}
\Aue{Bohacek, S.} 2003. A~stochastic model of TCP and fair video 
transmission. \textit{Proc. IEEE INFOCOM}. 1134--1144.
doi: 10.1109/INFCOM.2003.1208950.
\bibitem{6-b-1}
\Aue{Sauliev, V.\,K.} 1960. \textit{Integrirovanie uravneniy pa\-ra\-bo\-li\-che\-sko\-go 
tipa metodom setok} [Integration of parabolic equations by the grid method]. 
Moscow: Fizmatlit. 324~p.
\bibitem{7-b-1}
\Aue{Borisov, A.\,V., and G.\,B.~Miller.} 2005. Analysis and filtration of 
special discrete-time Markov processes. II.~Optimal filtration. \textit{Autom. 
Rem. Contr.} 66(7):1125--1136.
\bibitem{8-b-1}
\Aue{Borisov, A.\,V., G.\,B.~Miller, and K.\,V.~Semenikhin.} 2015. Filtering 
of the Markov jump process given the observations of multivariate point 
process. \textit{Autom. Rem. Contr.} 76(2):219--240.
\bibitem{9-b-1}
\Aue{Borisov, A., A.~Bosov, and G.~Miller.} 2015. Modeling and monitoring 
of RTP link on the receiver side. 
\textit{Internet of things, smart spaces, and next generation
networks and systems}. Eds. S.\,I.~Balandin, S.\,D.~Andreev,
and Y.~Koucheryavy. Lecture notes in computer science ser. Springer, 
2015. 9247:229--241. doi: 10.1007/978-3-319-23126-6\_21.

\bibitem{10-b-1}
\Aue{Borisov, A.\,V.} 2016. Application of optimal filtering methods for  
on-line of queueing network states. \textit{Autom. Rem. Contr.} 
77(2):277--296.
\end{thebibliography}

 }
 }

\end{multicols}

\vspace*{-6pt}

\hfill{\small\textit{Received June 7, 2018}}

%\pagebreak

%\vspace*{-18pt}
     
     \Contr
     
\noindent
\textbf{Bosov Alexey V.} (b.\ 1969)~--- Doctor of Science in technology, 
principal scientist, Institute of Informatics Problems, Federal Research 
Center ``Computer Science and Control'' of the Russian Academy of Sciences, 
44-2 Vavilov Str., Moscow 119333, Russian Federation; 
\mbox{AVBosov@ipiran.ru}

\vspace*{3pt}

\noindent
\textbf{Stefanovich Alexey I.} (b.\ 1983)~--- principal specialist, Institute of Informatics Problems, 
Federal Research Center ``Computer Science and Control'' of the Russian Academy of Sciences,  
44-2~Vavilov Str., Moscow 119333, Russian Federation; \mbox{AStefanovich@frccsc.ru}
\label{end\stat}

\renewcommand{\bibname}{\protect\rm Литература}       

      