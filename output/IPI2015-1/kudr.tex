\def\stat{kudr}

\def\tit{БАЙЕСОВСКАЯ РЕКУРРЕНТНАЯ МОДЕЛЬ РОСТА НАДЕЖНОСТИ:
БЕТА-РАВНОМЕРНОЕ РАСПРЕДЕЛЕНИЕ ПАРАМЕТРОВ$^*$}

\def\titkol{Байесовская рекуррентная модель роста надежности:
бета-равномерное распределение параметров}

\def\aut{Ю.\,В.~Жаворонкова$^1$,  А.\,А.~Кудрявцев$^2$,  С.\,Я.~Шоргин$^3$}

\def\autkol{Ю.\,В.~Жаворонкова,  А.\,А.~Кудрявцев,  С.\,Я.~Шоргин}

\titel{\tit}{\aut}{\autkol}{\titkol}

{\renewcommand{\thefootnote}{\fnsymbol{footnote}} \footnotetext[1]
{Исследование выполнено при поддержке
Российского научного фонда (проект 14-11-00397).}}


\renewcommand{\thefootnote}{\arabic{footnote}}
\footnotetext[1]{ООО Спутник,
juliana-zh@yandex.ru}
\footnotetext[2]{
Московский государственный университет им.\ М.\,В.~Ломоносова, факультет вычислительной математики и кибернетики,
nubigena@mail.ru}
\footnotetext[3]{Институт проблем информатики Российской академии наук, sshorgin@ipiran.ru}

\Abst{Прогнозирование надежности сложных модифицируемых информационных
систем (СМИС) является в~настоящее время одной из актуальных задач теории массового
обслуживания.
Любая впервые созданная сложная система, предназначенная для
переработки или передачи информационных потоков, как правило, не
обладает требуемой надежностью. Такие системы подвергаются модификациям
в~ходе разработки, опытной эксплуатации и~штатного функционирования.
Целью этих модификаций является увеличение надежности информационных систем.
В~связи с~этим возникает необходимость формализации понятия надежности
модифицируемых информационных систем и~разработки методов и~алгоритмов оценивания
и~прогнозирования различных надежностных характеристик.
Одним из подходов к~определению надежности системы является вычисление
вероятности того, что на сигнал, поданный на вход системы в~определенный
момент времени, система отреагирует корректно.
В~статье рассматривается экспоненциальная рекуррентная модель роста надежности,
в~которой вероятность надежности системы представляется как линейная
комбинация параметров <<дефективности>> и~<<эффективности>> средства,
исправляющего недостатки системы. Предполагается, что исследователь не
имеет точных сведений об исследуемой системе, а~лишь знаком с~характеристиками
класса, из которого берется данная система. В~рамках байесовского подхода
предполагается, что один из показателей <<дефективности>> и~<<эффективности>>
имеет бе\-та-рас\-пре\-де\-ле\-ние, а~другой~--- равномерное распределение.
Вычисляется средняя предельная надежность системы. Приводятся численные
результаты для модельных примеров.}

\KW{модифицируемые информационные системы; теория надежности; байесовский подход;
бета-распределение; равномерное распределение}

\DOI{10.14357/19922264150109}


\vskip 14pt plus 9pt minus 6pt

\thispagestyle{headings}

\begin{multicols}{2}

\label{st\stat}

\section{Постановка задачи}

Задача прогнозирования надежности СМИС была сформулирована в~[1], а~в~дальнейшем более подробно рассмотрена в~[2].
В~статье~[3] дано подробное описание класса моделей, в~рамках которых возникает
необходимость использования байесовского подхода к~анализу роста надежности СМИС,
таких как новая программная система для компьютера, новая
ин\-фор\-ма\-ци\-он\-но-вы\-чис\-ли\-тель\-ная сеть или новая
ад\-ми\-ни\-стра\-тив\-но-ин\-фор\-ма\-ци\-он\-ная система, которые,
как правило, изначально не обладают требуемой надежностью.

Исследуемые СМИС подвергаются периодическим изменениям (модификациям)
с~целью увеличения надежности информационных систем. В~\mbox{статье} рассматривается
описанная в~книге~[2] модель роста надежности, обычно использующаяся, когда
удобно иметь дело непосредственно с~параметром, интерпретируемым как надежность
сис\-темы.
{\looseness=1

}

Рассмотрим произвольную систему, на вход которой подаются некоторые сигналы
(например, команды оператора или внешние воздействия). Реакция системы на
поданные сигналы может быть либо правильной (корректной), либо неправильной
(некорректной).

В~каждый момент времени~$t$ надежность системы можно характеризовать
параметром $p(t)$~--- вероятностью того, что на сигнал, поданный на вход системы
в~момент~$t$, система отреагирует правильно. По смыслу такая характеристика
надежности ближе всего к~традиционно используемому коэффициенту готовности.
В~случайные моменты времени $0\hm=Y_0\hm\le Y_1 \hm\le Y_2 \le\cdots$ система
подвергается (мгновенной) модификации, в~результате чего изменяется параметр
$p(t)$.

Следует обратить внимание на то обстоятельство, что ниже рассматривается
непрерывное время, без привязки напрямую процесса модифицирования системы
к~процессу ее тестирования. Предположим, что траектории процесса $p(t)$
непрерывны справа и~ку\-соч\-но-по\-сто\-ян\-ны, так что $p(t) \hm= p(Y_j)$ при
$Y_j \hm\le t\hm< Y_{j+1}$.

Обозначим $p_j \hm= p(Y_j)$. Рассмотрим поведение~$p_j$
в~зависимости от изменения~$j$. Другими словами, будем изучать изменение
надежности системы в~зависимости от номера модификации.
В~книге~[2] рассматривается, в~частности, следующая рекуррентная модель
рос\-та надежности.  Пусть  $\{(\theta_j, \eta_j)\}$, $j\hm\ge1$,~---
последовательность независимых одинаково распределенных двумерных
случайных векторов таких, что $0 \hm< \eta_1 \hm< 1$; $0 \hm< \theta_1 \hm < 1$
почти наверное.

Задав начальную надежность~$p_0$, рассмотрим модель, определяемую рекуррентным
соотноше\-нием
$$p_{j+1} = \eta_{j+1}p_j + \theta_{j+1}(1-p_j)\,.$$
Эта модель названа дискретной экспоненциальной моделью.
В~такой модели случайные величины~$\eta_j$ (параметры <<дефективности>>)
описывают возможное уменьшение надежности из-за некачественных модификаций,
в~ходе которых вместо исправления существующих дефектов в~систему могут
быть внесены новые, в~то время как величины~$\theta_j$ (параметры <<эффективности>>)
описывают повышение надежности за счет исправления дефектов.

Обозначим $\lambda = 1 \hm- {\sf E}\theta_1$, $\mu  \hm= {\sf E}\eta_1$.
В~[2] доказано, что при условии $\lambda\hm+\mu\hm\neq1$
$$
p=\lim\limits_{j\to\infty}{\sf E} p_j = \fr{\mu}{\lambda+\mu}\,.
$$

Изучение предельного значения средней величины ${\sf E} p_j$ представляет
значительный интерес, поскольку эта величина характеризует асимптотическое
значение надежности системы в~рамках некото\-рой рекуррентной модели,
задаваемой набором $\{(\theta_j, \eta_j)\}$. Из результатов~[2] следует,
что это асимптотическое значение зависит только от средних значений величин
$\{(\theta_j, \eta_j)\}$, $j\hm\ge1$.

В~[3, 4] исследовалась ситуация, при которой рассматривается набор
однотипных сложных модифицируемых объектов (МО), каждый из которых
обслуживается собственной ремонтной бригадой (РБ). Исследователю хотелось
бы определить усредненное значение~$p$ по всем МО. Для решения этой задачи
в~указанной работе предложена так называемая байесовская постановка.
Предполагается, что рассматривается целая группа однотипных МО и~группа
им соответствующих однотипных РБ. Пусть $m\hm=1,2,\ldots$~---
номера этих объектов. Для каждого МО (вместе с~его РБ) существует
собственный набор $\{(\theta_j^m, \eta_j^m)\}$, $j\hm\ge1$, $m\hm\ge1$,
независимых одинаково распределенных при каждом фиксированном~$j$
двумерных случайных векторов таких, что $0\hm < \eta_1^m \hm< 1$;
$0 \hm< \theta_1^m \hm < 1$ почти наверное. Но средние значения величин~$\theta_j^m$,
$\eta_j^m$, $j\hm\ge1$, $m\hm\ge1$, не предполагаются известными; более того, они
не предполагаются даже одинаковыми. Вводится предположение, что величины
$\lambda \hm = 1 \hm-  {\sf E}\theta_j^m$, $\mu \hm= {\sf E}\eta_j^m$ сами
по себе являются случайными, т.\,е.\ на вероятностном пространстве,
в~которое в~качестве элементарных событий входят все рас\-смат\-ри\-ва\-емые
в~рамках данной постановки МО вместе с~их РБ, заданы случайные величины~$\lambda$
и~$\mu$ (которые полагаем независимыми), имеющие смысл $\lambda\hm = 1 \hm-
{\sf E}\theta_j^m$, $\mu \hm= {\sf E}\eta_j^m$, где $m$~--- случайный номер МО.
Принимаемые исследователем за основу распределения величин~$\lambda$ и~$\mu$
будем называть априорными.

Подлежащие вычислению характеристики такой <<рандомизированной>>
группы МО, естественно, являются рандомизацией аналогичных характеристик
<<отдельно взятой>> МО с~учетом априорного распределения параметров~$\lambda$
и~$\mu$, взятого исследователем за основу. Наиболее естественной и~удобной
для изучения характеристикой является усредненное по всем МО значение
предельной вероятности надежности, т.\,е.\
$$
p_{\mathrm{сред}} = {\sf E} p = {\sf E} \fr{\mu}{\lambda+\mu}\,,
$$
где усреднение ведется по совместному распределению случайных
величин $(\lambda,\mu)$.

В рассматриваемой ситуации величины $\eta_j^m$ и~$\theta_j^m$
удовлетворяют ограничениям $0 \hm< \eta_j^m \hm< 1$, $0 \hm< \theta_j^m  \hm< 1$.
Значит, и~значения~$\lambda$ и~$\mu$ величин $1 \hm-
{\sf E}\theta_j^m$ и~${\sf E}\eta_j^m$ соответственно также находятся на отрезке
$[0,1]$. Поэтому в~качестве априорных распределений параметров~$\lambda$
и~$\mu$ следует выбирать только распределения, сосредоточенные на $[0,1]$.

В работах [3, 4] были рассмотрены независимые случайные параметры~$\lambda$
и~$\mu$, имеющие одновременно равномерное или бе\-та-рас\-пре\-де\-ле\-ние
соответственно. В~настоящей статье исследования байесовской рекуррентной
модели роста на\-деж\-ности продолжены для ситуации, когда один из параметров
имеет бе\-та-рас\-пре\-де\-ле\-ние, а~другой~--- равномерное распределение.

\section{Основные результаты}


Введем следующие обозначения. Через $B(m,n)$, $m,n\hm>0$, будем
обозначать бе\-та-функ\-цию. Через $R(a,b)$ и~$\beta(m,n)$ обозначим соответственно
равномерное распределение и~бе\-та-рас\-пре\-де\-ле\-ние. Пусть
$$
(\alpha)_i=\alpha(\alpha+1)\cdots(\alpha+i-1)\,,\enskip
(\alpha)_0=1\,.
$$
Несмотря на то что $(\alpha)_i$ имеет смысл неполного факториала,
нигде далее не требуется, чтобы~$\alpha$ было положительным.
Рассмотрим классическую гипергеометрическую функцию Гаусса
$$
G(\alpha,\beta,\gamma;x)=\sum\limits_{i=0}^\infty\fr{(\alpha)_i(\beta)_i}
{(\gamma)_i \,i!}\,x^i\,.
$$
По аналогии с~9.180.1, 9.180.3 и~9.14 п.\,1 из~[5] введем в~рассмотрение
две обобщенные гипергеометрические функции двух переменных:
\begin{multline}
G^{p,q}_{s,t}(\alpha, \beta_1,\ldots,\beta_p,\beta_1',\ldots,\beta_q';
\gamma,\delta_1,\ldots\\[2pt]
\ldots,\delta_s,\delta_1',\ldots,
\delta_t';x,y)={}\\[2pt]
{}=\sum\limits_{i=0}^\infty\sum\limits_{j=0}^\infty
\fr{(\alpha)_{i+j}(\beta_1)_i\cdots(\beta_p)_i(\beta_1')_j\cdots
(\beta_q')_j}{(\gamma)_{i+j}
(\delta_1)_i\cdots(\delta_s)_i(\delta_1')_j\cdots(\delta_t')_j}\times{}
\\[2pt]
{}\times\fr{x^iy^j}{i!j!}\,;\label{e1-kud}
\end{multline}

\vspace*{-12pt}

\noindent
\begin{multline}
H^{p,q}_{s,t}(\beta_1,\ldots,\beta_p,\beta_1',\ldots,\beta_q';\gamma,\delta_1,
\ldots,\delta_s,\delta_1',\ldots\\[2pt]
\ldots,
\delta_t';x,y)={}\\[2pt]
{}=\sum\limits_{i=0}^\infty\sum\limits_{j=0}^\infty
\fr{(\beta_1)_i\cdots(\beta_p)_i(\beta_1')_j\cdots(\beta_q')_j}{(\gamma)_{i+j}
(\delta_1)_i\cdots(\delta_s)_i(\delta_1')_j\cdots(\delta_t')_j}\times{}\\[2pt]
{}\times
\fr{x^iy^j}{i!j!}\,.\label{e2-kud}
\end{multline}

\smallskip

\noindent
\textbf{Теорема 1.} \textit{Пусть случайные величины~$\lambda$ и~$\mu$
независимы и~имеют соответственно распределения
$R(a,b)$ и~$\beta(m,n)$, где $0\hm\le a\hm<b\hm\le1$, $m,n\hm>0$. Тогда}
\begin{multline}
p_{\mathrm{сред}} = \fr{B(m+1,n)}{(b-a)B(m,n)}\ln\left(\fr{b+1}{a+1}\right)+{}\\
{}+\fr{b(m+1)^{-2}}{(b-a)B(m,n)(b+1)}\times{}\\
{}\times H_{1,0}^{3,2}
\left(\vphantom{\fr{1}{b+1}}
1-n,m+1,m+1,1,1;\right.\\
\left. m+2,m+2;\fr{1}{b+1},1\right)-
\fr{a(m+1)^{-2}}{(b-a)B(m,n)(a+1)}\times{}\\
{}\times H_{1,0}^{3,2}\left(\vphantom{\fr{1}{a+1}}
1-n,m+1,m+1,1,1;\right.\\
\left.m+2,m+2;\fr{1}{a+1},1\right)\,.\label{e3-kud}
\end{multline}


\noindent
Д\,о\,к\,а\,з\,а\,т\,е\,л\,ь\,с\,т\,в\,о\,.\ \
Найдем плотность $f_p(x)$ случайной величины~$p$. Имеем
$$
f_p(x)=\il{a}{b}\fr{y}{(1-x)^2}\,f_\mu\left(\fr{x}{1-x}\,y\right)f_\lambda(y)\,dy\,.
$$

\noindent
Используя замену переменной $z\hm=xy/(1\hm-x)$, по формуле~8.391 из~[5]
для $0\hm<x\hm<1/(b+1)$ имеем

\noindent
\begin{multline*}
f_p(x) = \il{a}{b} \fr{(1-x)^{-2}y}{(b-a)B(m,n)}\times{}\\[2pt]
{}\times
\left(\fr{xy}{1-x}\right)^{m-1} \left(1 - \fr{xy}{1-x}\right)^{n-1} dy ={}\\[2pt]
{} = \fr{(m+1)^{-1}x^{-2}}{(b-a)B(m,n)} \left( \!\left(
\fr{bx}{1-x} \right)^{m+1}\times{}\right.\\[2pt]
\left.{}\times G\left(
1-n,m+1,m+2; \fr{bx}{1-x}\right)  -{}\right.\\[2pt]
\left.{}- \left( \fr{ax}{1-x} \right)^{m+1}
\!  G \left(\!1-n,m+1, m+2; \fr{ax}{1-x}\right)\!  \right)\equiv{}\\[2pt]
  {}\equiv  S_1(x)\,,
 \end{multline*}
а для $1/(b+1)\hm\le x\hm<1/(a+1)$
\begin{multline*}
f_p(x) = \hspace*{-2mm}\il{a}{(1-x)/x} \hspace*{-2mm}\fr{(1-x)^{-2}y}{(b-a)B(m,n)}
\left(\fr{xy}{1-x}\right)^{m-1}\times{}\\[2pt]
{}\times \left(1 - \fr{xy}{1-x}\right)^{n-1} dy =
\fr{x^{-2}}{(b-a)B(m,n)} \times{}\\[2pt]
{}\times \left( B(m+1, n)  -
\fr{1}{m+1}\left( \fr{ax}{1-x} \right)^{m+1} \times{}\right.\\[2pt]
\left.{}\times G
\left(1-n,m+1, m+2; \fr{ax}{1-x}\right)
\vphantom{\left.\fr{xy}{1-x}\right)^{n-1}}
 \right)\equiv S_2(x)\,.
\end{multline*}

Таким образом,
\begin{equation}
{\sf E} p =\! \il{}{}\hspace*{-1mm}xf_p(x)\,dx=\hspace*{-3mm}\il{0}{1/(b+1)}
\hspace*{-3.2mm}xS_{1}(x)\, dx +
\hspace*{-3.2mm}\il{1/(b+1)}{1/(a+1)} \hspace*{-3mm}xS_{2}(x)\, dx.\label{e4-kud}
\end{equation}

Вычислим отдельно первый интеграл из правой части~(\ref{e4-kud}). Имеем
\begin{multline*}
\il{0}{1/(b+1)} \hspace*{-3mm}xS_{1}(x)\, dx = {}\\[2pt]
{}=\hspace*{-3mm}\il{0}{1/(b+1)}
\fr{(m+1)^{-1}x^{-1}}{(b-a)B(m,n)} \left(\! \left(
\fr{bx}{1-x} \right)^{m+1}\times{}\right.\\[2pt]
\left.{}\times G\left(1-n,m+1, m+2; \fr{bx}{1-x}\right)  -
 \left( \fr{ax}{1-x} \right)^{m+1}\!\!\times{} \right.\hspace*{-2.315315pt}\\[2pt]
\left.{} \times G\left(
1-n,m+1,m+2; \fr{ax}{1-x}\right)  \!\right)\, dx\equiv U_1-U_2\,.\hspace*{-0.19241pt}
\end{multline*}

Для первого слагаемого имеем:
\begin{multline*}
U_1 = \fr{b^{m+1}(m+1)^{-1}}{(b-a)B(m,n)} \sum\limits_{i=0}^{\infty}
\fr{(1-n)_i(m+1)_i  b^i}{(m+2)_i i!}\times{}\\
{}\times
\il{0}{1/(b+1)} \fr{x^{m+i}\,dx}{(1-x)^{m+i+1}} =
- \fr{b^{m+1}(m+1)^{-1}}{(b-a)B(m,n)} \times{}\\
{}\times \sum\limits_{i=0}^{\infty}
\fr{(1-n)_i(m+1)_i  b^i}{(m+2)_i i!} \il{0}{1/b}
\fr{\left( (1/b) - z  \right) ^{m+i}}{\left( z - (b+1)/b \right) }\, dz\,.
\end{multline*}

Воспользуемся формулой~3.196.1 из~[5]:
\begin{multline*}
 U_1 = \fr{b^{m+1}(m+1)^{-1}}{(b-a)B(m,n)}
 \sum\limits_{i=0}^{\infty} \fr{(1-n)_i(m+1)_i b^i}{(m+2)_i i!}\,
 \times{}\\[2pt]
 {}\times
 \fr{b}{b+1}\left( \fr{1}{b} \right)^{m+i+1}
 \fr{G(1,1,m+i+2,{1/(b+1)})}{m+i+1} ={}\\[2pt]
{} = \fr{b}{(b-a)B(m,n)} \times{}\\[2pt]
{}\times\sum\limits_{i=0}^{\infty}
\sum\limits_{j=0}^{\infty}
\fr{(1-n)_i(m+i+1)^{-2}j!}{i!(m+i+2)_{j}} \left( \fr{1}{b+1} \right)^{j+1}\,.
\end{multline*}
Аналогично
\begin{multline*}
U_2 = \fr{a^{m+1}}{(b-a)B(m,n)}\times{}\\[2pt]
{}\times
\sum\limits_{i=0}^{\infty} \sum\limits_{j=0}^{\infty}
\fr{(1-n)_i(m+i+1)^{-2}a^ij!}{ i!(m+i+2)_{j}}\times{}\\[2pt]
{}\times \left( \fr{1}{b}
\right)^{m+i}\left( \fr{1}{b+1} \right)^{j+1}\,.
\end{multline*}

Аналогично получаем для второго слагаемого из~(\ref{e4-kud}):
\begin{multline*}
\il{1/(b+1)}{1/(a+1)} xS_{2}(x)\, dx =
\fr{ B(m+1, n)}{(b-a) B(m,n)}\ln\left( \fr{ b+1}{a+1} \right)-{}\\[2pt]
{}- \il{1/(b+1)}{1/(a+1)} \fr{(m+1)^{-1}x^{-1}}{(b-a)B(m,n)} \left(
\fr{ax}{1-x} \right)^{m+1}\times{}\\[2pt]
{}\times G\left(
1-n,m+1,m+2; \fr{ax}{1-x}  \right)\,dx\equiv U_3-U_4\,.
\end{multline*}

Разбив вычитаемое в~последнем выражении на два интеграла, получим

\noindent
\begin{multline*}
U_4= \il{0}{1/(a+1)} \fr{(m+1)^{-1}x^{-1}}{(b-a)B(m,n)} \left(
\fr{ax}{1-x} \right)^{m+1}\times{}\\[2pt]
{}\times G\left(1-n,m+1,m+2; \fr{ax}{1-x}  \right)\,dx-U_2={}\\[2pt]
{} = \fr{a}{(b-a)B(m,n)}\times{}\\[2pt]
{}\times \sum\limits_{i=0}^{\infty}\sum\limits_{j=0}^{\infty}
\fr{(1-n)_i(m+i+1)^{-2}j!}{ i!(m+i+2)_{j}} \left( \fr{1}{a+1}
\right)^{j+1}-U_2\,.\hspace*{-2.3764pt}
\end{multline*}

Подытоживая все вспомогательные выкладки, получаем для~(\ref{e4-kud})
\begin{multline}
{\sf E} p=U_1-U_2+U_3-U_4={}\\[2pt]
{}=
\fr{ B(m+1, n)}{(b-a) B(m,n)}\ln\left(
\fr{ b+1}{a+1} \right)+ \fr{b}{(b-a)B(m,n)}\times{}\\[2pt]
{}\times \sum\limits_{i=0}^{\infty}\sum\limits_{j=0}^{\infty}
\fr{(1-n)_i(m+i+1)^{-2}j!}{i!(m+i+2)_{j}} \left( \fr{1}{b+1} \right)^{j+1}-{}\\
{} -\fr{a}{(b-a)B(m,n)} \times{}\\[2pt]
\hspace*{-3mm}{}\times\sum\limits_{i=0}^{\infty}\sum\limits_{j=0}^{\infty}
\fr{(1-n)_i(m+i+1)^{-2}j!}{ i!(m+i+2)_{j}} \left( \fr{1}{a+1}
\right)^{j+1}.\label{e5-kud}
\end{multline}

Используя элементарные преобразования для~(\ref{e5-kud}), по определению~(\ref{e2-kud})
получаем~(\ref{e3-kud}), что завершает доказательство теоремы.

\smallskip

Теперь рассмотрим симметричный случай для распределений случайных величин~$\lambda$
и~$\mu$.

\medskip

\noindent
\textbf{Теорема 2.} \textit{Пусть случайные величины~$\lambda$ и~$\mu$
независимы и~имеют соответственно распределения $\beta(m,n)$
и~$R(a,b)$, где $0\hm\le a\hm<b\hm\le1$, $m,n\hm>0$. Тогда}
\begin{multline}
p_{\mathrm{сред}} = \fr{B(m + 1, n)}{B(m,n)} \left( 1 + \fr{1}{b-a}
\ln \left( \fr{a+1}{b+1}\right)  \right)+{}\\
{}+\fr{bm^{-1}(m+1)^{-1}}{(b-a)B(m,n)}\times{}\\
{}\times G_{1,0}^{2,1}\left(
m,1-n,m+1,1;m+1,m+2;1,-\fr{1}{b}\right)-{}\\
{}-\fr{am^{-1}(m+1)^{-1}}{(b-a)B(m,n)}\times{}
\\
\!\!\!\!\!{}\times G_{1,0}^{2,1}\left(m,1-n,m+1,1;m+1,m+2;1,
-\fr{1}{a}\!\right).\!\!\!
\label{e6-kud}
\end{multline}

\noindent
Д\,о\,к\,а\,з\,а\,т\,е\,л\,ь\,с\,т\,в\,о\,.\ \
Найдем плотность $f_p(x)$ случайной величины~$p$. Имеем

\pagebreak

\noindent
$$
f_p(x)=\il{0}{1}\fr{y}{(1-x)^2}\,f_\mu\left(\fr{x}{1-x}\,y\right)f_\lambda(y)\,dy\,.
$$
По формуле~8.391 из~[5] для $a/(a + 1) \hm< x \hm< b/(b+1)$ имеем
\begin{multline*}
f_p(x) = \il{a(1 - x)/x}{1} \fr{(1-x)^{-2}}{(b-a)B(m,n)}\, y^{m}
(1 - y)^{n-1}\, dy ={}\\
{}=
 \fr{B(m + 1, n)(1-x)^{-2}}{(b-a) B(m,n)}  -{}\\
{}-  \fr{a^{m+1}(m+1)^{-1}(1-x)^{m-1}}{(b-a) B(m,n)x^{m+1}}\times{}\\
 {}\times
 G \left(1-n,m+1, m+2, \fr{a(1 - x)}{x} \right)\equiv T_1(x)\,,
 \end{multline*}
а для $b/(b + 1)\le x < 1$
\begin{multline*}
\hspace*{-0.46051pt}f_p(x) = \fr{(1-x)^{-2}}{(b - a) B(m,n)} \left(
\il{0}{b(1 - x)/x}\hspace*{-4mm} y^{m} (1 - y)^{n-1} \,dy -{}\right.\\
\left.{}-
\il{0}{a(1 - x)/x} \hspace*{-4mm} y^{m} (1 - y)^{n-1}\, dy   \right)={}\\
{}= \fr{b^{m+1}(m+1)^{-1}(1-x)^{m-1}}{(b - a) B(m,n)x^{m+1}}\times{}\\
{}\times G
\left(1-n,m+1,m+2, \fr{b(1 - x)}{x} \right) -{}\\
{}- \fr{a^{m+1}(m+1)^{-1}(1-x)^{m-1}}{(b - a) B(m,n)x^{m+1}}\times{}\\
{}\times
G\left(1-n,m+1,m+2, \fr{a(1 - x)}{x} \right)\equiv T_2(x)\,.
\end{multline*}

Таким образом,
\begin{equation}
 {\sf E} p = \int\! xf_p(x)\,dx=\hspace*{-3mm}\il{a/(a+1)}{b/(b+1)}\hspace*{-4mm}
 xT_{1}(x)\, dx + \hspace*{-3mm}\il{b/(b+1)}{1} \hspace*{-4mm}xT_{2}(x)\, dx.\!\label{e7-kud}
 \end{equation}

Вычислим отдельно первый интеграл из правой части~(\ref{e7-kud}). Имеем
\begin{multline*}
\il{a/(a+1)}{b/(b+1)} xT_{1}(x) \,dx ={}\\
{}= \fr{B(m + 1, n)}{B(m,n)}
 \left( 1 + \fr{1}{b-a} \ln \left( \fr{a+1}{b+1}\right)  \right) -{}
\end{multline*}

\noindent
\begin{multline*}
{}- \il{a/(a+1)}{b/(b+1)}\fr{a^{m+1}(m+1)^{-1}(1-x)^{m-1}}{(b-a) B(m,n)x^{m}}  \times{}\\
{}\times
G\left(1-n,m+1, m+2, \fr{a(1 - x)}{x} \right)\,dx \equiv V_1-V_2\,.
\end{multline*}
Для вычисления $V_2$ воспользуемся формулой~3.196.1 из~[5]. Имеем
\begin{multline*}
V_2= \fr{a^{m+1}(m+1)^{-1}}{(b - a)B(m,n)} \sum\limits_{i=0}^{\infty}
\fr{(1-n)_{i}(m+1)_{i} a^{i}}{(m+2)_{i} i! }\times{}\\
{}\times
\il{a/(a+1)}{b/(b+1)} \fr{(1-x)^{m+i-1}}{x^{m+i}}\, dx  ={}\\
{}= \fr{a^{m+1}}{(b - a)B(m,n)} \sum\limits_{i=0}^{\infty}
\fr{(1-n)_{i}a^{i}}{(m+i+1)i! }\times{}\\
{}\times
\il{1/(b+1)}{1/(a+1)} \fr{y^{m+i-1}}{(1-y)^{m+i}}\, dy  ={}\\
{}= \fr{a^{m+1}}{(b - a)B(m,n)} \sum\limits_{i=0}^{\infty}
\fr{(1-n)_{i}a^{i}}{(m+i+1) i! }\times{}\\
{}\times
\left( \il{0}{1/(a+1)}
\fr{\left( 1/(a+1)-x\right)^{m+i-1}}
{\left(x+a/(a+1)\right)^{m+i}}\, dx -{}\right.\\
\left.{}-
\il{0}{1/(b+1)}
\fr{\left(1/(b+1)-x\right)^{m+i-1}}
{\left(x+b/(b+1)\right)^{m+i}}\, dx\right)  ={}\\
{}= \fr{a^{m+1}}{(b - a)B(m,n)} \sum\limits_{i=0}^{\infty}
\fr{(1-n)_{i}a^{i}}{(m+i+1) i! }\times{}\\
{}\times\left( \fr{a^{-m-i}}{m+i}\,G\left(1,m+i,m+i+1,-\fr{1}{a}\right)-
\fr{b^{-m-i}}{m+i}\times{}\right.\\
\left.{}\times G\left(1,m+i,m+i+1,-\fr{1}{b}\right)\right)={}\\
{}= \fr{a^{m+1}}{(b - a)B(m,n)} \sum\limits_{i=0}^{\infty}
\fr{(1-n)_{i}a^{i}}{(m+i+1) i! }\times{}\\
{}\times\left(
\sum\limits_{j=0}^\infty \fr{a^{-m-i}(-a)^{-j}}{m+i+j}-
\sum\limits_{j=0}^\infty \fr{b^{-m-i}(-b)^{-j}}{m+i+j}\right)\,.
\end{multline*}


Для второго слагаемого в~(\ref{e7-kud}) аналогично имеем
\begin{multline*}
\il{b/(b+1)}{1}\hspace*{-3mm} xT_{2}(x)\, dx =\hspace*{-3mm}
\il{b/(b+1)}{1} \hspace*{-3mm}\fr{ b^{m+1}(m+1)^{-1} (1-x)^{m-1}}{(b - a) B(m,n)x^{m}}\times{}\\
{}\times
G\left(1-n,m+1,m+2, \fr{b(1 - x)}{x} \right)\,dx -{}
\end{multline*}

\end{multicols}

\begin{table*}[b]\small %tabl1
\begin{center}
\Caption{Частные значения средней надежности ($\lambda\sim R(a,b)$,
$\mu\sim \beta(m,n)$)}
\vspace*{2ex}

%\tabcolsep=5pt
\begin{tabular}{|c|c|c|c|c|c|c|c|c|c|c|}
 \hline
 &\multicolumn{10}{c|}{$m;n$}\\
 \cline{2-11}
 \multicolumn{1}{|c|}{\raisebox{6pt}[0pt][0pt]{$[a,b]$ }}&
  ${1; 7}$ & ${1; 3}$ & ${3; 5}$ &
  ${1; 1}$ & ${6; 10}$ & ${2; 2}$ &
  ${10; 6}$ & ${5; 3}$ & ${3; 1}$ & ${7; 1}$\\
 \hline
$[0,1/4]$&0,47&0,60&0,74&0,75&0,75&0,78&0,84&0,83&0,85&0,88\\
% \hline
$[0,1/2]$&0,34&0,48&0,61&0,63&0,62&0,66&0,73&0,73&0,76&0,79\\
% \hline
$[0,3/4]$&0,28&0,40&0,53&0,56&0,54&0,58&0,65&0,65&0,68&0,72\\
% \hline
$[0,1]$&0,24&0,35&0,47&0,50&0,48&0,52&0,59&0,59&0,62&0,66\\
% \hline
$[1/4,1/2]$&0,22&0,35&0,48&0,52&0,49&0,54&0,62&0,62&0,66&0,70\\
% \hline
$[1/4,3/4]$&0,19&0,30&0,42&0,46&0,43&0,48&0,56&0,55&0,60&0,64\\
% \hline
$[1/4,1]$&0,16&0,27&0,38&0,42&0,38&0,44&0,51&0,51&0,55&0,59\\
% \hline
$[1/2,3/4]$&0,15&0,26&0,36&0,40&0,37&0,42&0,50&0,49&0,54&0,58\\
% \hline
$[1/2,1]$&0,13&0,23&0,32&0,37&0,33&0,38&0,46&0,45&0,49&0,54\\
% \hline
$[3/4,1]$&0,12&0,20&0,29&0,33&0,29&0,35&0,41&0,41&0,45&0,50\\
 \hline
 \end{tabular}
 \end{center}
%\end{table*}
%\begin{table*}\small %tabl2
\begin{center}
\Caption{Частные значения средней надежности ($\lambda\sim R(a,b)$, $\mu\sim \beta(m,n)$)}
\vspace*{2ex}

\tabcolsep=5pt
\begin{tabular}{|c|c|c|c|c|c|c|c|c|c|c|}
 \hline
 &\multicolumn{10}{c|}{$m;n$}\\
 \cline{2-11}
 \multicolumn{1}{|c|}{\raisebox{6pt}[0pt][0pt]{$[a,b]$ }}&
  ${1; 7}$ & ${1,1; 3,3}$ & ${1,2; 2,0}$ &
  ${1,3; 1,3}$ & ${1,4; 2,33}$ & ${1,5; 1,5}$ &
  ${1,6; 0,96}$ & ${1,7; 1,02}$ & ${1,8; 0,6}$ & ${1,9; 0,27}$\\
 \hline
$[0,1/4]$&0,47&0,61&0,70&0,76&0,71&0,77&0,81&0,81&0,85&0,87\\
% \hline
$[0,1/2]$&0,34&0,48&0,57&0,64&0,58&0,65&0,70&0,71&0,75&0,78\\
% \hline
$[0,3/4]$&0,28&0,41&0,50&0,57&0,50&0,57&0,63&0,63&0,67&0,71\\
% \hline
$[0,1]$&0,24&0,36&0,44&0,51&0,45&0,51&0,57&0,57&0,62&0,66\\
% \hline
$[1/4,1/2]$&0,22&0,35&0,45&0,53&0,46&0,53&0,60&0,60&0,65&0,69\\
% \hline
$[1/4,3/4]$&0,19&0,31&0,40&0,47&0,40&0,47&0,54&0,54&0,59&0,63\\
% \hline
$[1/4,1]$&0,16&0,27&0,36&0,43&0,36&0,43&0,49&0,49&0,54&0,59\\
% \hline
$[1/2,3/4]$&0,15&0,26&0,34&0,41&0,35&0,42&0,48&0,48&0,53&0,58\\
% \hline
$[1/2,1]$&0,13&0,23&0,31&0,38&0,31&0,38&0,44&0,44&0,49&0,53\\
% \hline
$[3/4,1]$&0,12&0,20&0,28&0,34&0,28&0,34&0,40&0,40&0,45&0,49\\
 \hline
 \end{tabular}
 \end{center}
 \end{table*}

 \begin{multicols}{2}

\noindent
\begin{multline*}
{}- \il{b/(b+1)}{1} \fr{a^{m+1}(m+1)^{-1} (1-x)^{m-1}}{(b - a) B(m,n)x^{m}}\times{}\\
{}\times
G\left(1-n,m+1,m+2, \fr{a(1 - x)}{x} \right)\,dx \equiv V_3-V_4\,,
\end{multline*}
где
\begin{align*}
V_3&= {}\\
&\hspace*{-5mm}{}=\fr{b}{(b - a)B(m,n)} \sum\limits_{i=0}^{\infty}\sum\limits_{j=0}^\infty
\fr{(1-n)_{i}(-b)^{-j}}{(m+i+j)(m+i+1)i! }\,;\\
V_4&= {}\\
&\hspace*{-5mm}{}=\fr{a^{m+1}b^{-m}}{(b - a)B(m,n)} \sum\limits_{i=0}^{\infty}\sum\limits_{j=0}^\infty
\fr{(1-n)_{i}a^{i}b^{-i}(-b)^{-j}}{(m+i+j)(m+i+1) i!}\,.
\end{align*}

Объединяя полученные результаты, получаем из~(\ref{e7-kud})
\begin{multline*}
{\sf E} p=V_1-V_2+V_3-V_4= {}\\
{}=\fr{B(m + 1, n)}{B(m,n)} \left(
1 + \fr{1}{b-a} \ln \left( \fr{a+1}{b+1}\right)  \right)-{}\\
\!\!{}- \fr{a}{(b - a)B(m,n)} \sum\limits_{i=0}^{\infty}\sum\limits_{j=0}^\infty
\fr{ (1-n)_{i}(-a)^{-j}}{(m+i+j)(m+i+1) i! }+{}
\end{multline*}

\noindent
\begin{multline}
{}+ \fr{b}{(b - a)B(m,n)} \times{}\\
{}\times \sum\limits_{i=0}^{\infty}\sum\limits_{j=0}^\infty
\fr{(1-n)_{i}(-b)^{-j}}{(m+i+j)(m+i+1) i! }\,.\label{e8-kud}
\end{multline}

Используя элементарные преобразования для~(\ref{e8-kud}), по определению~(\ref{e1-kud})
получаем~(\ref{e6-kud}), что завершает доказательство теоремы.

\smallskip

\noindent
\textbf{Замечание 1.} При $a\hm=0$, очевидно, выражение~(\ref{e6-kud}) остается
справедливым, если последнее слагаемое положить равным нулю.

\smallskip

\noindent
\textbf{Замечание 2.} Выражения~(\ref{e3-kud}) и~(\ref{e6-kud}) служат для компактной
записи $p_{\mathrm{сред}}$. Для практического использования
(непосредственного вычисления) имеет смысл представлять
$p_{\mathrm{сред}}$ в~виде рядов типа~(\ref{e5-kud}) и~(\ref{e8-kud}), которые
несложно вычисляются с~любой наперед заданной точностью.

\smallskip

В качестве иллюстрации приведем несколько таблиц со значениями
$p_{\mathrm{сред}}$. Для удобства сравнения с~результатами, опубликованными
в~[3] и~[4],  значения параметров равномерного и~бе\-та-рас\-пре\-де\-ле\-ния
взяты из этих статей. Таблицы~1 и~2 соответствуют распределениям~$\lambda$
и~$\mu$ из теоремы~1, а~табл.~3 и~4 соответствуют теореме~2.

\end{multicols}

\begin{table}\small %tabl3
\begin{center}
\Caption{Частные значения средней надежности ($\lambda\sim \beta(m,n)$, $\mu\sim R(a,b)$)}
\vspace*{2ex}

%\tabcolsep=5pt
\begin{tabular}{|c|c|c|c|c|c|c|c|c|c|c|}
 \hline
 &\multicolumn{10}{c|}{$[a,b]$}\\
 \cline{2-11}
 \multicolumn{1}{|c|}{\raisebox{6pt}[0pt][0pt]{$m;n$ }}&
  $[0,1/4]$ & $[0,1/2]$ & $[0,3/4]$ &
  $[0,1]$ & $[1/4,1/2]$ & $[1/4,3/4]$ &
  $[1/4,1]$ & $[1/2,3/4]$ & $[1/2,1]$ & $[3/4,1]$\\
 \hline
${1; 7}$&0,53&0,79&0,81&0,74&0,72&0,74&0,75&0,71&0,66&0,67\\
% \hline
${1; 3}$&0,40&0,52&0,57&0,65&0,73&0,69&0,75&0,77&0,78&0,75\\
% \hline
${3; 5}$&0,09&0,39&0,72&0,52&0,69&0,73&0,74&0,83&0,85&0,27\\
% \hline
${1; 1}$&0,25&0,35&0,43&0,50&0,51&0,55&0,58&0,61&0,63&0,66\\
% \hline
${6; 10}$&0,25&0,38&0,46&0,54&0,70&0,73&0,74&0,83&0,85&0,86\\
% \hline
${2; 2}$&0,23&0,32&0,41&0,48&0,39&0,47&0,55&0,59&0,62&0,64\\
% \hline
${10; 6}$&0,16&0,27&0,35&0,41&0,55&0,54&0,55&0,65&0,67&0,74\\
% \hline
${5; 3}$&0,17&0,26&0,39&0,41&0,55&0,50&0,48&0,87&0,18&0,87\\
% \hline
${3; 1}$&0,15&0,25&0,32&0,38&0,33&0,41&0,44&0,49&0,51&0,54\\
 %\hline
${7; 1}$&0,13&0,22&0,27&0,34&0,30&0,36&0,38&0,42&0,47&0,49\\
 \hline
 \end{tabular}
 \end{center}
% \end{table*}
%\begin{table*}\small %tabl4
\begin{center}
\Caption{Частные значения средней надежности ($\lambda\sim \beta(m,n)$, $\mu\sim R(a,b)$)}
\vspace*{2ex}

%\tabcolsep=5pt
\begin{tabular}{|c|c|c|c|c|c|c|c|c|c|c|}
 \hline
 &\multicolumn{10}{c|}{$[a,b]$}\\
 \cline{2-11}
 \multicolumn{1}{|c|}{\raisebox{6pt}[0pt][0pt]{$m;n$ }}&
  $[0,1/4]$ & $[0,1/2]$ & $[0,3/4]$ &
  $[0,1]$ & $[1/4,1/2]$ & $[1/4,3/4]$ &
  $[1/4,1]$ & $[1/2,3/4]$ & $[1/2,1]$ & $[3/4,1]$\\
 \hline
${1; 7}$&0,36&0,79&0,81&0,74&0,72&0,74&0,75&0,71&0,69&0,68\\
% \hline
${1,1; 3,3}$&0,53&0,57&0,61&0,60&0,78&0,67&0,66&0,69&0,71&0,69\\
% \hline
${1,2; 2,0}$&0,34&0,50&0,54&0,58&0,62&0,59&0,63&0,68&0,69&0,75\\
% \hline
${1,3; 1,3}$&0,25&0,43&0,48&0,50&0,45&0,53&0,53&0,54&0,60&0,59\\
% \hline
${1,4; 2,33}$&0,37&0,45&0,56&0,50&0,50&0,56&0,59&0,60&0,64&0,63\\
% \hline
${1,5; 1,5}$&0,23&0,35&0,42&0,49&0,46&0,53&0,57&0,58&0,60&0,65\\
% \hline
${1,6; 0,96}$&0,20&0,30&0,36&0,44&0,42&0,46&0,51&0,54&0,59&0,62\\
% \hline
${1,7; 1,02}$&0,15&0,29&0,35&0,41&0,40&0,45&0,49&0,52&0,58&0,59\\
% \hline
${1,8; 0,6}$&0,16&0,28&0,32&0,42&0,34&0,40&0,45&0,45&0,49&0,51\\
% \hline
${1,9; 0,27}$&0,13&0,21&0,28&0,37&0,35&0,39&0,40&0,40&0,46&0,48\\
 \hline
 \end{tabular}
 \end{center}
 \end{table}

\begin{multicols}{2}


\section{Заключение}

Полученные результаты могут применяться, например, для
вычисления других моментов и~построения доверительных интервалов
для характеристики~$p$.
%
В дальнейшем предполагается расширить класс совместных распределений
параметров~$(\lambda, \mu)$, разработать соответствующие расчетные алгоритмы
для вычисления величины $p_{\mathrm{сред}}$ и~провести тестовые расчеты.

\vspace*{-9pt}

{\small\frenchspacing
 {%\baselineskip=10.8pt
 \addcontentsline{toc}{section}{References}
 \begin{thebibliography}{9}
\bibitem{GK}
\Au{Gnedenko B.\,V., Korolev~V.\,Yu.}
Random summation: Limit theorems and applications.~--- Boca Raton, FL:
CRC Press, 1996. 288~p.

\columnbreak

\bibitem{KS}
\Au{Королев В.\,Ю., Соколов И.\,А.}
Основы математической теории надежности модифицируемых систем.~---
М.: ИПИ РАН, 2006. 108~c.

\vspace*{4pt}

\bibitem{KuSoSh}
\Au{Кудрявцев А.\,А., Соколов И.\,А., Шоргин~С.\,Я.}
Байесовская рекуррентная модель роста надежности:
равномерное распределение параметров~//
Информатика и~её применения, 2013. Т.~7. Вып.~2. С.~55--59.

\vspace*{4pt}

\bibitem{ZhaKuSh}
\Au{Жаворонкова Ю.\,В., Кудрявцев А.\,А., Шоргин~С.\,Я.}
Байесовская рекуррентная модель роста надежности:
бе\-та-рас\-пре\-де\-ле\-ние параметров~//
Информатика и~её применения, 2014. Т.~8. Вып.~2. С.~48--54.

\vspace*{4pt}

\bibitem{GR71}
\Au{Градштейн И.\,С., Рыжик И.\,М.}
Таблицы интегралов, сумм, рядов и~произведений.~--- М.: Наука, 1971. 1108~с.
 \end{thebibliography}

 }
 }

\end{multicols}

\vspace*{-3pt}

\hfill{\small\textit{Поступила в редакцию 26.01.15}}

\newpage

%\vspace*{12pt}

%\hrule

%\vspace*{2pt}

%\hrule

\vspace*{-36pt}

\def\tit{BAYESIAN RECURRENT MODEL OF~RELIABILITY GROWTH: BETA-UNIFORM DISTRIBUTION
OF~PARAMETERS}

\def\titkol{Bayesian recurrent model of~reliability growth: Beta-uniform distribution
of~parameters}

\def\aut{Iu.\,V.~Zhavoronkova$^1$, A.\,A.~Kudryavtsev$^2$, and~S.\,Ya.~Shorgin$^3$}

\def\autkol{Iu.\,V.~Zhavoronkova, A.\,A.~Kudryavtsev, and~S.\,Ya.~Shorgin}

\titel{\tit}{\aut}{\autkol}{\titkol}

\vspace*{-9pt}

\noindent
$^1$Sputnik Ltd., 8/2 Prishvina Str., Moscow 127549, Russian Federation


\noindent
$^2$Faculty of Computational
Mathematics and Cybernetics, M.\,V.~Lomonosov Moscow State University,
1-52\linebreak
$\hphantom{^1}$Leninskiye Gory, GSP-1, Moscow 119991, Russian Federation


\noindent
$^3$Institute of Informatics Problems, Russian Academy of Sciences,
44-2 Vavilov Str., Moscow 119333, Russian\linebreak
$\hphantom{^1}$Federation


\def\leftfootline{\small{\textbf{\thepage}
\hfill INFORMATIKA I EE PRIMENENIYA~--- INFORMATICS AND
APPLICATIONS\ \ \ 2015\ \ \ volume~9\ \ \ issue\ 1}
}%
 \def\rightfootline{\small{INFORMATIKA I EE PRIMENENIYA~---
INFORMATICS AND APPLICATIONS\ \ \ 2015\ \ \ volume~9\ \ \ issue\ 1
\hfill \textbf{\thepage}}}

\vspace*{3pt}



\Abste{Forecasting reliability of complex modifiable information systems is one
of the topical problems of the mass service theory nowadays. Any first established
complex system designed for processing or transmission of information flows, as
a rule, does not possess the required reliability. Such systems are subject to
modifications during development, testing, and regular functioning. The purpose
of such modifications is to increase reliability of information systems. In this
connection, there is a necessity to formalize the concept of reliability of
modifiable information systems and to develop methods and algorithms of estimation
and forecasting of various reliability characteristics. One approach to determine
system reliability is to compute the probability that the signal fed to the input
of the system at a given point of time will be reacted to correctly by the system.
The article considers the exponential recurrent growth model of reliability, in
which the probability of system reliability is represented as a linear combination
of ``defectiveness'' and ``efficiency'' parameters of tools correcting the
deficiencies in the system. It is assumed that the researcher does not have
exact information about the system under study and is only familiar with the
characteristics of the class from which this system is taken. In the framework
of the Bayesian approach, it is assumed that one of the indicators of
``defectiveness'' and ``efficiency'' has the beta-distribution and the other
one has the uniform distribution. Average marginal system reliability is
calculated. Numerical results for model examples are obtained.}

\KWE{modifiable information systems; theory of reliability; Bayesian approach;
beta-distribution; uniform distribution}

\DOI{10.14357/19922264150109}

\vspace*{-21pt}

\Ack

\vspace*{-4pt}

\noindent
This work was financially supported by the Russian Science Foundation
(grant No.\,14-11-00397).



%\vspace*{3pt}

  \begin{multicols}{2}

\renewcommand{\bibname}{\protect\rmfamily References}
%\renewcommand{\bibname}{\large\protect\rm References}


{\small\frenchspacing
 {%\baselineskip=10.8pt
 \addcontentsline{toc}{section}{References}
 \begin{thebibliography}{9}

 \vspace*{-4pt}

\bibitem{1-kud-1}
\Aue{Gnedenko, B.\,V., and V.\,Yu.~Korolev}.
1996. \textit{Random summation: Limit theorems and applications}.
Boca Raton, FL: CRC Press, 1996. 288~p.

\bibitem{2-kud-1}
\Aue{Korolev, V.\,Yu., and I.\,A.~Sokolov}. 2006.
\textit{Osnovy ma\-te\-ma\-ti\-che\-skoy teorii nadezhnosti modifitsiruemykh system}
[Fundamentals of mathematical theory of modified systems reliability].
Moscow.: IPI RAN, 2006. 108~p.

\bibitem{3-kud-1}
\Aue{Kudryavtsev, A.\,A., I.\,A.~Sokolov, and S.\,Ya.~Shorgin}.
2013. Bayesovskaya rekurrentnaya model' rosta nadezhnosti:
Ravnomernoe raspredelenie parametrov [Bayesian recurrent model of reliability
growth: Uniform distribution of parameters].
\textit{Informatika i~ee Primeneniya}~--- \textit{Inform. Appl.} 7(2):55--59.

\bibitem{4-kud-1}
\Aue{Zhavoronkova, Iu.\,V., A.\,A.~Kudryavtsev, and S.\,Ya.~Shor\-gin}.
2013. Bayesovskaya rekurrentnaya model' rosta nadezhnosti:
Be\-ta-ras\-pre\-de\-le\-nie parametrov [Bayesian recurrent model
of reliability growth: Beta-distribution of parameters].
\textit{Informatika i~ee Primeneniya}~--- \textit{Inform. Appl.} 8(2):48--54.

\bibitem{5-kud-1}
\Aue{Gradshteyn, I.\,S., and I.\,M.~Ryzhik}. 1971. \textit{Tablitsy integralov,
summ, ryadov i~proizvedeniy} [Tables of integrals, sums, series, and products].
Moscow: Nauka. 1108~p.
\end{thebibliography}

 }
 }

\end{multicols}

\vspace*{-8pt}

\hfill{\small\textit{Received January 26, 2015}}

\vspace*{-24pt}


\Contr

\vspace*{-2pt}

\noindent
\textbf{Zhavoronkova Iuliia V.} (b.\ 1990)~--- software developer,
Sputnik Ltd., 8/2 Prishvina Str., Moscow 127549, Russian Federation; juliana-zh@yandex.ru


%\vspace*{3pt}

\noindent
\textbf{Kudryavtsev Alexey A.} (b.\ 1978)~---
Candidate of Science (PhD) in physics and mathematics, associate professor,
Faculty of Computational Mathematics and
Cybernetics, M.\,V.~Lomonosov Moscow State University,
1-52 Leninskiye Gory, GSP-1, Moscow 119991, Russian Federation; nubigena@mail.ru

%\vspace*{3pt}

\noindent
\textbf{Shorgin Sergey Ya.} (b.\ 1952)~--- Doctor of Science in physics and
mathematics, professor, Deputy Director, Institute of Informatics Problems,
Russian Academy of Sciences, 44-2 Vavilov Str., Moscow 119333,
Russian Federation; sshorgin@ipiran.ru


\label{end\stat}

\renewcommand{\bibname}{\protect\rm Литература}