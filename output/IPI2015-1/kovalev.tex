
\def\stat{kovalev}

\def\tit{ФОРМАЛЬНЫЙ АКСИОМАТИЧЕСКИЙ ПОДХОД
К~АСПЕКТНО-ОРИЕНТИРОВАННОМУ РАСШИРЕНИЮ ТЕХНОЛОГИЙ
ПРОГРАММИРОВАНИЯ$^*$}

\def\titkol{Формальный аксиоматический подход
к~аспектно-ориентированному расширению технологий
программирования}

\def\aut{С.\,П.~Ковалёв$^1$}

\def\autkol{С.\,П.~Ковалёв}

\titel{\tit}{\aut}{\autkol}{\titkol}

{\renewcommand{\thefootnote}{\fnsymbol{footnote}} \footnotetext[1]
{Статья подготовлена при поддержке Российского гуманитарного
научного фонда %РГНФ
(грант 13-03-00384).}}


\renewcommand{\thefootnote}{\arabic{footnote}}
\footnotetext[1]{Институт проблем управления им.\ В.\,А.~Трапезникова Российской академии наук,
kovalyov@nm.ru}

\vspace*{6pt}


\Abst{Исследуется процедура расширения технологий модульной разработки программных
сис\-тем приемами ас\-пект\-но-ори\-ен\-ти\-ро\-ван\-но\-го подхода. Расширение описано как
обогащение формальных моделей программных модулей разметкой их интерфейсов
классами задач, образующими аспектную структуру. Предложен новый подход к разделению
ответственности (separation of concerns) путем естественной модуляризации аспектной
структуры. В~качестве обобщения этого подхода предложена процедура частичной
модуляризации аспектной структуры. Для формализации образующихся конструкций на
общесистемном уровне, не зависящем от частных парадигм программирования,
привлекается теория категорий. Технологиям разработки программ отвечают категории,
объектами которых служат формальные модели программ, а морфизмами~---
технологические операции. Аспектно-ориентированное расширение (АО-расширение)
технологии описано
аксиоматически как преобразование таких категорий~--- функтор, обладающий
сопряженными подходящего вида как справа, так и слева. В~качестве иллюстративного
примера АО-расширения приводится событийный подход к
моделированию систем.}

\KW{аспектно-ориентированное программирование; трассируемость; теория категорий;
формальная технология проектирования; разделение ответственности}


\DOI{10.14357/19922264150105}

\vspace*{6pt}

\vskip 14pt plus 9pt minus 6pt

\thispagestyle{headings}

\begin{multicols}{2}

\label{st\stat}

\section{Введение}

 Современный комплексный подход к созданию программных систем требует
автоматизировать не только основные процессы предметной дея\-тель\-ности, но и
управляющие и обеспечи\-ва\-ющие (а~также обеспечивающие для
обеспечи\-ва\-ющих и~т.\,д.) (см., например,~[1]). В~сложных предметных
областях они обладают значительным разнообразием, глубоко погружаются
в~контекст основной деятельности почти на каждом шаге и~в~то же \mbox{время}
с~трудом совмещаются с~нею на понятийном\linebreak уровне.

Многие задачи
обеспечивающих процессов рассеивают\-ся по системе, не поддаваясь
локализации в рамках программных модулей, авто\-ма\-тизирующих шаги
основных. Использование традицион\-ных <<модульных>> технологий
программирования для таких задач приводит к~значительным затратам труда,
поскольку приходится дуб\-ли\-ро\-вать одни и~те же алгоритмы в~контексте
\mbox{различных} шагов. Примеры можно найти как среди функциональных задач
(ведение информационной модели объекта управления, верификация данных),
так и~среди про\-грамм\-но-тех\-ни\-че\-ских (защита информации, ведение
журналов функционирования системы и~др.).

 Для повышения эффективности программной реализации таких рассеянных
задач в конце\linebreak \mbox{1990-х~гг.}\ была предложена новая парадигма~---
 ас\-пект\-но-ори\-ен\-ти\-ро\-ван\-ное программирование (АОП)~[2].
Рассеянные задачи оформляются в виде аспектов~--- особых программных
единиц, код которых со\-став\-ля\-ет\-ся однократно и~затем автоматически
встав\-ля\-ет\-ся в~код основных единиц в~точках, явно задаваемых внешним
образом. В~результате вставки аспект получает полный доступ к~контексту.
Однако на практике АОП применяется значительно реже, чем модульные
подходы, поскольку на концептуальном уровне неясно, как оптимально
выделять и~соединять аспекты и какие аспекты целесообразно реализовывать
модулями~[3]. Существующие технологии АОП, такие как AspectJ
 (AO-расширение языка Java)~[4],
предлагают лишь частные решения, специфичные для частных парадигм
программирования.

 В связи с этим целью настоящей работы ставится построение универсальной
концептуальной модели расширения <<модульных>> технологий средствами
разработки аспектов~--- основного метода внедрения АОП в практику
программирования. Расширение описано как обогащение модулей разметкой
классами задач, описывающей <<историю>> их разработки, на уровне
интеграционных интерфейсов. Такой подход обусловлен тем, что разметка
позволяет непосредственно проводить трассирование задач~--- операцию,
которая фактически является обратной по отношению к рассеиванию и
вследствие этого лежит в основе АОП~[5]. Классы задач образуют аспектную
структуру модулей, так что ас\-пект\-но-ори\-ен\-ти\-ро\-ван\-ная
(де)композиция проводится согласованно на двух уровнях: модульных основ
и~аспектных структур.

 Чтобы строго сформулировать и верифицировать этот подход, в работе
привлекается теория категорий, поскольку ее средствами можно дать
универсальное аксиоматическое описание процессов создания программных
систем с~позиций общей теории систем~[6]. Результативность
аксиоматического подхода здесь связана с тем, что можно потребовать
соблюдения аксиом при организации (и~тем более при автоматизации) труда
программистов, что обеспечивает применимость и эффективность результатов,
выведенных из аксиом. Отправной точкой служит теоретико-категорная
конструкция формальной технологии проектирования (architecture school)~[7].
Добавление поддержки аспектов в формальную технологию описано в работе
как ее преобразование, порождающее категории помеченных системных
единиц. Эти категории оснащены функторами выделения модульной основы
и~аспектной структуры. Приемы АОП формализуются универсальными
конструкциями в таких категориях, и их свойства строго доказываются путем
вывода из аксиом.

В~качестве источника примеров для иллюстрации
предлагаемого подхода выбрано событийное моделирование программных
систем, поскольку в его понятиях формулируются классические частные
семантические модели АОП (см., например,~[8]). Вообще в литературе
предлагается много подходов к формализации АОП (см., например,~[9]),
однако все они представлены в контексте тех или иных частных формализмов
теоретического программирования (лямб\-да-ис\-чис\-ле\-ние, проверка на
моделях и~др.)\ и~поэтому могут применяться только в рамках частных
парадигм.

 Работа построена следующим образом. В~разд.~2 вводится понятие
формальной технологии. Раздел~3 посвящен преобразованию модульных
технологий в аспект\-но-ори\-ен\-ти\-ро\-ван\-ные. В~разд.~4 и~5 описаны
конструкции полной и частичной модуляризации аспектной структуры
соответственно. В~заключении подводятся итоги исследования.

\section{Теоретико-категорное описание разработки программ}

 \textit{Категория C}~--- это класс абстрактных \textit{объектов} Ob~$C$,
попарно связанных \textit{морфизмами} (аб\-ст\-ракт\-ны\-ми аналогами
отображений)~[10, разд.~1.1]: каждый морфизм $f$ имеет область
$\mathrm{dom}\, f \in \mathrm{Ob}\, C$ и~кообласть
$\mathrm{codom}\, f \hm\in \mathrm{Ob}\, C$.
Соотношения вида $\mathrm{dom}\, f\hm = A$ и~$\mathrm{codom}\, f \hm= B$
наглядно записываются в форме стрелки $f:\ A\hm\to B$, а~множество всех
морфизмов, удовлетворяющих этим соотношениям, обозначается через
$\mathrm{Mor}(A, B)$. Для любой пары морфизмов $f, g$ такой, что
$\mathrm{codom}\, f \hm= \mathrm{dom}\, g$, определена
ком\-по\-зи\-ция-мор\-физм $g\circ f:\ \mathrm{dom}\, f \hm\to \mathrm{codom}\,g$.
Композиция ассоциативна: для любой тройки морфизмов $f, g, h$ если
$\mathrm{codom}\, f \hm= \mathrm{dom}\, g$ и~$\mathrm{codom}\, g \hm= \mathrm{dom}\, h$,
то $h\circ (g \circ f) \hm= (h \circ g) \circ f$. Наконец, любой
объект~$A$ обладает тождественным морфизмом 1$_A:\ A\hm\to A$ таким, что
для любого морфизма $f:\ A\hm\to B$ выполняется соотношение
$f \circ 1_A \hm= 1_B\circ f \hm= f$.

 Для формального аксиоматического описания разработки программных
систем категории применяются начиная с 1970-х~гг.\ (см., например,~[6]).
Здесь объекты отвечают компонентам и системам~--- обычно это формальные
модели программ (ал\-геб\-ра\-и\-че\-ские спецификации, графы, термы
 лямб\-да-ис\-чис\-ле\-ния и~т.\,п.). Морфизмы часто обозначают действия по
интеграции компонентов в системы. Композиция морфизмов отвечает
конструированию многошаговых действий (процессов), а~тож\-де\-ст\-вен\-ные
морфизмы~--- <<ни\-че\-го\-не\-де\-ла\-нию>>. Будем обозначать категорию
такого рода через $c$-DESC. Конфигурации взаимосвязанных компонентов,
из которых собираются системы, задаются $c$-DESC-диа\-грам\-ма\-ми~---
ориентированными графами, вершины которых помечены объектами, а~ребра~---
морфизмами категории $c$-DESC. Актам сборки\linebreak систем отвечают копределы
диаграмм~[10, разд.~3.3]. Поясним конструкцию копредела на нескольких
примерах. В~качестве первого рассмотрим \textit{соединение} компонента~$P$
с~системой~$S$~--- прием сборки, состоящий в добавлении промежуточного
компонента~$G$, называемого <<клеем>> (glue), или связкой (connector)~[7],
который способен интегрироваться как с компонентом, так и~с~системой.
Путем соединения часто строятся системы на базе промежуточного
про\-грамм\-но\-го обес\-пе\-че\-ния
(middleware). Конфигурация соединения имеет вид пары
 $c$-DESC-морфизмов $f:\ P\leftarrow G\to S:\ g$. Ее копредел, на\-зы\-ва\-емый
\textit{кодекартовым квадратом}, задается объ\-ек\-том-вер\-ши\-ной~$V$
и~парой мор\-физ\-мов-ре\-бер $p:\ P\hm\to V\leftarrow S:\ s$ таких, что $p\circ f
\hm= s \circ g$ и~выполняется следующее условие универсальности: для любых
объекта~$T$ и пары морфизмов $u:\ P\hm\to T\leftarrow S:\ v$ если
$u \circ f\hm= v \circ g$, то существует единственный морфизм $w:\ V\hm\to T$,
удовлетворяющий соотношениям $w \circ p \hm= u$ и $w \circ s \hm= v$. Тогда
объект~$V$ действительно отвечает системе, которая собрана из~$S$ и~$P$
путем соединения посредством~$G$ (и~не содержит ничего <<лишнего>>).
Если такой объект~$V$ существует, то он определяется однозначно с
точностью до изоморфизма~--- формального представления несущественного
различия между моделями. (Если же копредела не существует, то делается
вывод, что клей~$G$ не способен соединить компонент~$P$ с~$S$
посредством действий~$f$ и~$g$.)
 %fig1
\vspace*{12pt}
 \begin{center}
 \mbox{%
 \epsfxsize=67.149mm
 \epsfbox{kov-1.eps}
 }
 \end{center}
 \vspace*{12pt}
%\end{figure*}

 Конструкция копредела, как видно из ее названия, является двойственной по
отношению к~конструкции предела, которая была введена первоначально для
нужд приложений теории категорий в~топологии~[10, разд.~3.4].
В~приложениях к~разработке программных систем пределы привлекаются для
формализации процедур разложения систем (декомпозиции), обратных по
отношению к~сборке. Например, в разд.~4 потребуется выделять из\linebreak моделей
части, соответствующие прообразам относительно действия определенных
морфизмов.\linebreak Формально для объекта~$Y$, вложенного в~$Z$ посредством
мономорфизма (категорного аналога инъекции) $m:\ Y\hookrightarrow Z$,
полный прообраз относительно некоторого морфизма $h:\ X\hm\to Z$ строится
при помощи \textit{декартова квадрата}. Это конструкция, двойственная
к~кодекартову квадрату, т.\,е.\ предел диаграммы $m:\ Y\hm\hookrightarrow
Z\leftarrow X:\ h$. Он задается объектом~$W$ и~парой морфизмов $n:\
X\leftarrow W\to Y:\ l$ таких, что $h \circ n \hm= m \circ l$ и выполняется
следующее условие универсальности: для любых объекта~$U$ и~пары
морфизмов $j:\ X \leftarrow U \rightarrow Y:\ k$ если $h \circ j\hm = m \circ k$, то
существует единственный морфизм $t:\ U \to W$, удовлетворяющий
соотношениям $n \circ t \hm= j$ и $l \circ t \hm= k$. Искомый прообраз
объекта~$Y$ выделяется в~$X$ морфизмом~$n$, причем он также является
мономорфизмом.

\columnbreak

 %fig2
\vspace*{-3pt}
 \begin{center}
 \mbox{%
 \epsfxsize=49.166mm
 \epsfbox{kov-2.eps}
 }
 \end{center}
 \vspace*{3pt}


 Другим важным частным случаем (ко)пределов являются регулярные
морфизмы. Рассмотрим произвольную $c$-DESC-диаграмму вида $d, e:\ Q
\rightrightarrows R$, состоящую из двух параллельных морфизмов. Ее копредел,
если он существует, задается морфизмом $r:\ R \to V$ таким, что $r \circ d \hm=
r \circ e$ и если соотношение $x \circ d \hm= x \circ e$ выполнено для
некоторого морфизма $x:\ R \to T$, то существует единственный морфизм
$w:\ V \to T$, удовлетворяющий соотношению $w \circ r \hm= x$. Морфизм~$r$
называется \textit{коуравнителем} пары $d, e$, он является эпиморфизмом
(категорным аналогом сюръекции) и задает некоторую факторизацию
объекта~$R$ (например, если в качестве $c$-DESC\linebreak взять категорию
множеств, то морфизм~$r$ факторизует~$R$ по отношению эквивалентности,
порож\-денному множеством пар $\{(d(z), e(z)) \vert z \hm\in Q\}$).\linebreak Двойственно,
\textit{уравнителем} пары $d, e$ называется морфизм $q:\ W \to Q$ такой, что
$d \circ q \hm= e \circ q$, и~если соотношение $d \circ y \hm= e \circ y$ выполнено
для некоторого морфизма $y:\ U \to Q$, то существует единственный морфизм
$t:\ U \to W$, удовлетворяющий соотношению $q \circ t \hm= y$. Уравнитель
является мономорфизмом и содержательно задает вложение $W$ в $Q$ в
качестве подобъекта (в~категории множеств $W\hm\cong \{z \vert d(z) \hm=
e(z)\} \hm\subseteq Q$). Если некоторый морфизм выступает в качестве
коуравнителя некоторой пары, то он называется \textit{регулярным
эпиморфизмом}~[11, определение~7.71], а если в качестве уравнителя~---
\textit{регулярным мономорфизмом}~[11, определение~7.56].

%fig3
%\vspace*{-3pt}
 \begin{center}
 \mbox{%
 \epsfxsize=53.148mm
 \epsfbox{kov-3.eps}
 }
 \end{center}
% \vspace*{3pt}


 Чтобы описать механизмы формирования конфигураций, формализуется
понятие интеграционного интерфейса~--- <<части>> модели, задающей
правила интеграции других моделей с нею~[7]. \mbox{Например,} у веб-сер\-ви\-са
интерфейсом служит его декларация на языке WSDL
(Web Service Description Language). Формальные модели
интерфейсов образуют категорию, обо\-зна\-ча\-емую через $\mathrm{SIG}$, а~операция
выделения интерфейса у модели программы формализуется как функтор
 $\mathrm{sig}: c\mbox{-}\mathrm{DESC} \to \mathrm{SIG}$, называемый сигнатурным.
 \textit{Функтор}~--- это отображение категорий, переводящее объекты в
объекты, а морфизмы в морфизмы, с сохранением композиции
и~тождественных морфизмов~[10, разд.~1.3]. Поэтому функторами описываются
преобразования моделей программ, совместимые с~интеграцией систем.
Поскольку различные модели могут иметь один и~тот же интерфейс, функтор
$\mathrm{sig}$ не обязан быть инъективным на объектах. Однако $\mathrm{sig}$-обра\-зы двух
различных действий по интеграции одного и того же компонента в одну и ту же
сис\-те\-му должны быть различными: иначе получится, что интерфейсы
недостаточно детально описывают интеграционные возможности компонентов.
Иными словами, функтор $\mathrm{sig}$ должен быть \textit{унивалентным}
(faithful)~[11, определение~3.27(2)], т.\,е.\ инъективным на каждом множестве
$\mathrm{Mor}\,(P, S)$, $P, S \hm\in$\linebreak $\in \mathrm{Ob}\, c$-DESC.

 Кроме того, для каждого интерфейса должна\linebreak существовать хотя бы одна
реализация, поддер\-живающая его интеграционные возможности\linebreak в полном
объеме. Например, WSDL-описание \mbox{любого} веб-сер\-ви\-са можно реализовать
<<заглушками>> (stubs)~--- пустыми процедурами: они могут автоматически
генерироваться CASE-средст\-вами и~поз\-во\-ля\-ют быстро собирать отладочные\linebreak
версии приложений. Формально должен существовать функтор дискретной
реализации $\mathrm{sig}^*:\ \mathrm{SIG} \hm\to c$-DESC такой,
что $\mathrm{sig} \circ \mathrm{sig}^* \hm=
1_{\mathrm{SIG}}$ и~для любых\linebreak $\mathrm{SIG}$-объ\-ек\-та~$I$ и $c$-DESC-объ\-ек\-та~$S$ функтор
$\mathrm{sig}$ сюръективно (следовательно, биективно) отображает множество
$\mathrm{Mor}\,(\mathrm{sig}^*(I), S)$, описыва\-ющее все действия по интеграции
дискретной реализации интерфейса~$I$ в~сис\-те\-му~$S$, на мно\-жество
$\mathrm{Mor}\,(I, \mathrm{sig}(S))$, определяющее интеграционные возможности
интерфейса~$I$. Это означает, что функтор $\mathrm{sig}^*$ сопряжен слева к~$\mathrm{sig}$,
причем единица этого сопряжения тождественна.

 Напомним, что для произвольных категорий $C$, $D$ и функторов $\mathrm{fun}\,:\ C\to
D$, $\mathrm{fad}\,:\ D \hm\to C$ сопряжение $\mathrm{fad}\, \dashv \mathrm{fun}\,$~--- это семейство биекций
$\Phi:\ \mathrm{Mor}\,(\mathrm{fad}\,(S), R) \hm\cong \mathrm{Mor}(S, \mathrm{fun}\,(R))$, $S \hm\in
\mathrm{Ob}\, D$, $R\hm\in \mathrm{Ob}\, C$, естественное в следующем
смысле~[10, разд.~4.1]: для любых $C$-мор\-физ\-мов $f:\ \mathrm{fad}\,(S) \hm\to R$,
$k:\ R \hm\to
Y$ и $D$-мор\-физ\-ма $h:\ X \hm\to S$ выполняется соотношение $\Phi (k \circ f
\circ \mathrm{fad}\,(h)) \hm= \mathrm{fun}\,(k) \circ \Phi (f) \circ h$.
Семейство \mbox{$C$-мор}\-физ\-мов
$\varepsilon_A\hm= \Phi^{-1}(1_{\mathrm{fun}\,(A)}):\ \mathrm{fad}\,(\mathrm{fun}\,(A)) \hm\to A$, $A\hm\in
\mathrm{Ob}\, C$, называется \textit{коединицей} сопряжения: имеем $\Phi (1_R
\circ \varepsilon_R\circ \mathrm{fad}\,(\Phi (f))) \hm= \mathrm{fun}\,(1_R) \circ \Phi (\varepsilon_R) \circ
\Phi (f) \hm= \Phi (f)$, откуда получается <<треугольное тождество>>
сопряжения $\varepsilon_R \circ \mathrm{fad}\,(\Phi (f)) \hm= f$. Это тождество позволяет
вычислить действие сопряжения, если известна его коединица.

%fig4
\vspace*{-6pt}
 \begin{center}
 \mbox{%
 \epsfxsize=44.07mm
 \epsfbox{kov-4.eps}
 }
 \end{center}
 \vspace*{-6pt}


 Естественность сопряжения проявляется в том, что коединица представляет
собой естественное преобразование функтора $\mathrm{fad}\, \circ \mathrm{fun}\,$ в~1$_C$, т.\,е.\ для
любого $C$-мор\-физ\-ма $s:\ A \to B$ выполняется соотношение $\varepsilon_B
\circ \mathrm{fad}\,(\mathrm{fun}\,(s)) \hm= s \circ \varepsilon_A$. Это соотношение является частным
случаем треугольного тождества для морфизма $s \circ \varepsilon_A:\
\mathrm{fad}\,(\mathrm{fun}\,(A)) \hm\to B$, поскольку $\Phi (s \circ \varepsilon_A) \hm= \Phi (s \circ
\varepsilon_A \circ \mathrm{fad}\,(1_{\mathrm{fun}\,(A)})) \hm= \mathrm{fun}\,(s) \circ \Phi (\varepsilon_A) \circ
1_{\mathrm{fun}\,(A)}\hm = \mathrm{fun}\,(s)$.


%fig5
\vspace*{-6pt}
 \begin{center}
 \mbox{%
 \epsfxsize=52.863mm
 \epsfbox{kov-5.eps}
 }
 \end{center}
 \vspace*{-6pt}


 Двойственно, семейство $D$-мор\-физ\-мов $\eta_G \hm= \Phi (1_{\mathrm{fad}\,(G)}):\ G
\hm\to \mathrm{fun}\,(\mathrm{fad}\,(G))$, $G\hm\in \mathrm{Ob}\, D$, образует \textit{единицу}
сопряжения~--- естественное преобразование функтора 1$_D$ в $\mathrm{fun}\, \circ \mathrm{fad}\,$
($\eta_Q \circ t \hm= \mathrm{fun}\,(\mathrm{fad}\,(t)) \circ \eta_P$ для любого $D$-мор\-физ\-ма $t:\
P\hm\to Q$), порождающее второе треугольное тождество $\mathrm{fun}\,(\Phi^{-1}(g))
\circ \eta_S \hm= g$ для любого $D$-мор\-физ\-ма $g:\ S \to \mathrm{fun}\,(R)$.

%fig6
\vspace*{-6pt}
 \begin{center}
 \mbox{%
 \epsfxsize=51.035mm
 \epsfbox{kov-6.eps}
 }
 \end{center}
% \vspace*{-6pt}


 Если единица состоит из тождественных морфизмов, то $\mathrm{fun}\, \circ \mathrm{fad}\, \hm=
1_D$ ввиду естественности, а второе треугольное тождество приобретает вид
$\mathrm{fun}\,(\Phi^{-1}(g)) \hm= g$, откуда $\Phi (f)\hm = \mathrm{fun}\,(f)$ для любого
 $C$-мор\-физ\-ма $f:\ \mathrm{fad}\,(S) \hm\to R$, т.\,е.\ биекция сопряжения действует
так же, как правый сопряженный функтор (это имеет место для функтора
выделения интерфейсов $\mathrm{sig}$). Двойственно, если коединица тождественна, то
$\mathrm{fad}\, \circ \mathrm{fun}\, \hm= 1_C$ и биекция $\Phi^{-1}$ действует как левый
сопряженный функтор.

 Полноценный процесс создания программных систем в дополнение
 к~интеграции включает трансформации (refinements)~--- шаги разработки
индивидуальных компонентов (уточнение требований, реализа\-ция
спецификации на языке программирования и~др.). Трансформации моделей
программ могут быть устроены совершенно иначе, чем действия по
интеграции. Поэтому они описываются морфизмами подходящей категории,
обозначаемой через $r$-DESC, которая в~общем случае отличается от
 $c$-DESC, но обладает таким же классом объектов~[7]. Накладываются
условия естественности выделения интерфейса и трансформации относительно
сборки систем. Получается \textit{формальная технология
проектирования}~[7]~--- сложная категорная конструкция, состоящая из
категорий \mbox{$c$-DESC} и~\mbox{$r$-DESC} вместе с классом $c$-DESC-диа\-грамм,
представляющих конфигурации систем, и~функтором выделения интерфейсов
$\mathrm{sig}:\ c\mbox{-}\mathrm{DESC} \hm\to \mathrm{SIG}$.

 Например, в событийном подходе к проектированию~[12] в роли основной
категории моделей $c$-DESC выступает категория \textbf{Pos} всех частично
упорядоченных множеств и~всех их монотонных отображений.
 \textbf{Pos}-объекты отвечают сценариям поведения программ~---
совокупностям событий, частично упорядоченным
 при\-чин\-но-след\-ст\-вен\-ны\-ми связями. Трансформацией сценария~$X$
в~$Y$, т.\,е.\ морфизмом категории, выступающей в роли $r$-DESC, служит
<<раскрытие>> событий до подсценариев~--- произвольное
антифункциональное тотальное отношение $R\hm\subseteq X \times Y$,
удовлетворяющее условию $\forall\ x, x^\prime \in X$\ $\forall\ y, y^\prime \hm\in
Y$ ($xRy \wedge x^\prime Ry^\prime \wedge x \not=$\linebreak $\not= x^\prime) \Rightarrow (x \leq
x^\prime \hm\Leftrightarrow y \hm\leq y^\prime$). В~качестве функтора $\mathrm{sig}$,
извлекающего интерфейсы из сценариев, выступает функтор $\vert\mbox{--}\vert$:\
$\mathbf{Pos}\to \mathbf{Set}:\ S \mapsto \vert S\vert$, <<забывающий>>
порядок. Левым сопряженным к~нему служит функтор дискретного
упорядочения $\mathrm{id}:\ \mathbf{Set}\hm\to \mathbf{Pos}:\ I \mapsto \langle I,
=\rangle$. Примечательно, что функтор $\mathrm{id}$ в свою очередь имеет левый
сопряженный~--- таковым служит функтор распараллеливания $\mathrm{sconn}:\
\mathbf{Pos}\hm\to \mathbf{Set}$, сопостав\-ля\-ющий каждому частично
упорядоченному множеству множество всех компонент связности его порядка
(взаимно независимых подсценариев). Коединица этого сопряжения
тождественна, поэтому для любых сценария~$S$ и множества~$I$ имеется
биекция $\mathrm{sconn}:\ \mathrm{Mor}\,(S, \mathrm{id}(I)) \hm\cong \mathrm{Mor}\,(\mathrm{sconn}(S), I)$,
т.\,е.\ интерфейсы сценариев в полной мере задают их поведение в~роли как
компонентов, так и~сис\-тем. Такое свойство представляет интерес и для
произвольных технологий.

 \smallskip

 \noindent
 \textbf{Определение 1.} Формальная технология проектирования называется
\textit{структурируемой}, если функтор дискретной реализации интерфейсов
$\mathrm{sig}^*$ имеет\linebreak
 левый сопряженный с тождественной ко\-еди\-ни-\linebreak цей.\hfill$\square$

%\smallskip
\section{Формальные технологии аспектно-ориентированного
проектирования}

 Среди показателей качества программных сис\-тем от рассеивания задач
больше всех страдает трассируемость~--- возможность точно определить, для
решения каких задач в систему включен тот или иной фрагмент~\cite{13-kov}.
В~снижении затрат на трассирование фактически состоит назначение АОП:
главным мотивом его создателей было отсутствие в традиционных языках
программирования конструкций, позволяющих разделять исходный код
программ по классам задач~[2]. Поэтому, как отмечалось во введении,
ас\-пект\-но-ори\-ен\-ти\-ро\-ван\-ный подход в целом рассматривается как
оснащение модулей разметкой, идентифицирующей классы решаемых ими
задач. Универсальная формальная модель АОП основывается на
аксиоматическом описании трассирования, которое строится из конструкций в
формальных технологиях проектирования следующим образом.

 Хорошо известно, что трассируемость легко нарушается при
трансформациях (хрестоматийным примером служит реализация
алгебраической спецификации программы на алгоритмическом языке
программирования). При интеграции, напротив, обычно обеспечивается хотя
бы частичное трассирование (здесь примером служит прямая сумма
индексированного семейства множеств в категории \textbf{Set}, при
построении которой элементы каждой компоненты снабжаются ее индексом~---
<<меткой>>). Ввиду этого для произвольной трансформации~---
$r$-DESC-мор\-физ\-ма $r:\ S \hm\to T$~--- возникает сле\-ду\-ющее необходимое
условие возможности трассировать вдоль нее результат
к~источнику~\cite{12-kov}: \mbox{обращение} ее направления, т.\,е.\
тео\-ре\-ти\-ко-ка\-те\-гор\-ная дуализация должна превращать эту трансформацию в действие по
интеграции~--- в~$c$-DESC-мор\-физм $r^{\mathrm{op}}:\ T \to S$.

 Поскольку в ходе разработки программных сис\-тем трансформация
перемежается со сборкой сис\-тем, необходимо совместно трассировать
трансформации и действия по интеграции. Проще \mbox{всего} провести такое
совместное трассирование, если трасса~$r^{\mathrm{op}}$ обратима
справа~\cite{12-kov}. Действительно, существование $c$-DESC-мор\-физ\-ма
$s:\ S \hm\to T$ такого, что $r^{\mathrm{op}}\circ s \hm= 1_S$, эквивалентно тому,
что для любого $c$-DESC-мор\-физ\-ма $p:\ X \hm\to S$, задающего
интеграцию некоторого компонента~$X$ в систему~$S$, существует действие
по интеграции~$X$ в~$T$, со\-вмес\-ти\-мое с трассированием трансформации~$r$
в том смысле, что композиция трассы~$r^{\mathrm{op}}$ с этим действием
дает~$p$. Таким действием служит $s \circ p$, поскольку $r^{\mathrm{op}}\circ (s
\circ p) \hm= p$.

%fig7
%\vspace*{-6pt}
 \begin{center}
 \mbox{%
 \epsfxsize=50.393mm
 \epsfbox{kov-7.eps}
 }
 \end{center}
% \vspace*{-6pt}


 Заметим, что $s$ является регулярным $c$-DESC-мо\-но\-мор\-физ\-мом~[11,
предложение~7.59(1)], т.\,е.\linebreak задает вложение источника трансформации в
результат в качестве подобъекта. На практике его по\-стро\-е\-ние может быть
весьма трудоемким, но он требует\-ся не всегда, поскольку трассированию вдоль
процессов сборки систем подлежат в первую очередь интеграционные
требования, которые предъявляются к интерфейсам моделей. В~таком случае
достаточно потребовать обратимости справа не для трассы~$r^{\mathrm{op}}$, а
лишь для $\mathrm{SIG}$-мор\-физ\-ма $\mathrm{sig}(r^{\mathrm{op}})$, пред\-став\-ля\-юще\-го действие
трассы на уровне интерфейсов и называемого разметкой~\cite{12-kov}.
Реализация обращения разметки обычно не требует значительных затрат,
поскольку интерфейсы проектируются так, чтобы интегрировать их было
<<проще>>, чем сами модели. Получается частный случай известного подхода
к снижению затрат на трассирование путем ограничения класса трассируемых
требований согласно их значимости (value-based requirements traceability)~[14].

 \smallskip

 \noindent
 \textbf{Определение 2.} $r$-DESC-мор\-физм~$r$ называется
\textit{трассируемой трансформацией}, а двойственный к~не\-му
морфизм~$r^{\mathrm{op}}$~--- \textit{трассой}, если $r^{\mathrm{op}}$
принадлежит категории $c$-DESC и $\mathrm{SIG}$-мор\-физм $\mathrm{sig}(r^{\mathrm{op}})$
является ретракцией (т.\,е.\ имеет правый обратный). $\mathrm{sig}$-об\-раз трассы
называется \textit{разметкой}. \hfill$\square$

 \smallskip

 Например, в технологии событийного моделирования все трансформации
трассируемы, причем класс всех разметок состоит из всех сюръективных
отображений множеств (в силу аксиомы выбора любой
 \textbf{Set}-эпи\-мор\-физм является ретракцией). Действие трансформации
сценариев $t:\ X \hm\to Y$, двойственной к сюръекции $t^{\mathrm{op}}:\
\vert Y\vert \to \vert X\vert$, интуитивно можно трактовать как реализацию классов
задач~--- точек множества~$\vert X\vert$~--- путем раскрытия (expansion) в их
прообразы относительно~$t^{\mathrm{op}}$. По определению трансформация
сценариев разбивает свой результат на подмножества, хорошо упорядоченные
(well-ordered) в том смысле, что для любых $x, y, z, u \hm\in Y$ таких, что
$t^{\mathrm{op}}(x) \hm= t^{\mathrm{op}}(y) \not= t^{\mathrm{op}}(z) \hm=
t^{\mathrm{op}}(u)$, условие $x \hm\leq z$ влечет $y \hm\leq u$.

 Наиболее прямым и экономным способом обеспечения полной
трассируемости является <<запомина\-ние>> трасс трансформаций вместе
с~{моделями} программ, порожденными ими из классов задач~[15]. В~контексте
АОП интерес представляет в~первую очередь влияние трансформаций на
интеграционные возможности моделей, поэтому достаточно присоединить
к~моделям действия трансформаций на уровне интерфейсов, т.\,е.\ разметки.
Интеграция и~трансформация таких обогащенных моделей должна
согласованно выполняться на двух уровнях: модульных основ и~аспектных
структур. Как указано в~[16, разд.~7], в~тео\-рии категорий имеется специальная
конструкция, предназначенная для естественного присоединения действий\linebreak
к~объектам~--- категория запятой (comma category) [10, разд.~2.6]. Рассмотрим
категорию запятой $\mathrm{sig} \downarrow \mathrm{SIG}$. Ее объектами являются все пары вида
$\langle A, l:\ \mathrm{sig}(A) \to L\rangle$, где $A$~---\linebreak $c$-DESC-объ\-ект, $l$~---
$\mathrm{SIG}$-мор\-физм. Морфизмом объекта $\langle A_1, l_1:\ \mathrm{sig}(A_1) \to L_1\rangle$ в
$\langle A_2, l_2: \mathrm{sig}(A_2) \hm\to L_2\rangle$ является любая пара $\langle f:\ A_1
\hm\to A_2,\, b:\ L_1 \hm\to L_2\rangle$ такая,
что $b \circ l_1 \hm= l_2 \circ \mathrm{sig}(f)$.

%fig8
%\vspace*{-6pt}
 \begin{center}
 \mbox{%
 \epsfxsize=60.469mm
 \epsfbox{kov-8.eps}
 }
 \end{center}
% \vspace*{-6pt}


\smallskip

\noindent
\textbf{Определение 3.} \textit{Аспектно-ориентированной моделью}
(АО-мо\-делью) называется любой $(\mathrm{sig} \downarrow \mathrm{SIG})$-объ\-ект
$\langle A, l:\ \mathrm{sig}(A) \hm\to L\rangle$ такой, что $l$ является разметкой.
 $c$-DESC-объ\-ект~$A$ называется (\textit{модульной}) \textit{основой}
АО-модели, $\mathrm{SIG}$-мор\-физм~$l$~--- ее (\textit{аспектной}) \textit{разметкой},
$\mathrm{SIG}$-объ\-ект~$L$~--- ее \textit{аспектной структу}-\linebreak \textit{рой}.\hfill$\square$


 Будем обозначать через AO полную подкатегорию в $\mathrm{sig}\downarrow \mathrm{SIG}$,
класс объектов которой состоит из всех АО-моделей. (Напомним, что полной
подкатегорией в произвольной категории~$C$ называется категория, состоящая
из некоторого класса $C$-объ\-ек\-тов~$X$ и~объединения всех множеств $\mathrm{Mor}\,(A,
B)$, $A, B \hm\in X$, оснащенная унаследованными из $C$ операциями).
Поясним эту конструкцию на примере событийного моделирования.
 У~АО-модели сценария разметка~$l$~--- это сюръективное отображение,
сопоставляющее каждому элементу множества~$A$ (событию) элемент
множества~$L$, который можно рассматривать как обозначение класса задач,
по\-рож\-да\-юще\-го это событие~\cite{12-kov}. В~частности, аспектом естественно
называть любую АО-модель, в которой $L$ состоит из одного элемента (такие
и только такие АО-модели сценариев удовлетворяют приведенному ниже
формальному определению~4)~[17]. Поэтому произвольная АО-модель
сценария~--- это час\-тич\-но-упо\-ря\-до\-чен\-ное мультимножество (pomset),
состоящее из аспектов. Подобный подход к~моделированию сценариев был
предложен еще в~1980-х~гг.~[18], однако природа меток и способы их синтеза
оставались неясными, поскольку они не рассматривались в контексте АОП.

 Категория AO снабжена следующими <<забыва\-ющи\-ми>> функторами,
индуцированными конструкцией категории запятой~\cite{12-kov}:
 \begin{itemize}
 \item
$\mathrm{mod}:\ \mathrm{AO} \to c\mbox{-}\mathrm{DESC}:\ \langle A, l\rangle \mapsto A$, $\langle f, b\rangle
\mapsto f$ (выделение модульной основы);
\item $\mathrm{int} = \mathrm{sig} \circ \mod:\ \mathrm{AO} \to \mathrm{SIG}:\ \langle A, l\rangle \mapsto \mathrm{sig}(A)$,
$\langle f, b\rangle \mapsto \mathrm{sig}(f)$ (выделение исходного интерфейса);
\item $\mathrm{str}:\ \mathrm{AO}\to \mathrm{SIG}:\ \langle A, l\rangle\mapsto \mathrm{codom}\, l$, $\langle f,
b\rangle\mapsto b$ (выделение аспектной структуры).
\end{itemize}

 Функтор $\mod$ примечателен тем, что он позволяет извлекать из АО-модели
модульную основу в форме особого интеграционного интерфейса, т.\,е.\
порождает формальную технологию проектирования, поддерживающую
 ас\-пект\-но-ори\-ен\-ти\-ро\-ван\-ный подход. Действительно, функтор
$\mod$ унивалентен: если $\langle f, b\rangle, \langle f, b^\prime\rangle:\ \langle A,
l\rangle \hm\to \langle B, k\rangle$~--- два произвольных AO-мор\-физ\-ма, то
$b^\prime \circ l \hm= k \circ \mathrm{sig}(f) \hm= b \circ l$, откуда
$b^\prime\hm= b$,
поскольку~$l$~обратим справа. Левым сопряженным к функтору $\mod$
служит функтор дискретной разметки
 $
\mod^*:\ c\mbox{-}\mathrm{DESC} \hm\to \mathrm{AO}:\ A\mapsto \langle A, 1_{\mathrm{sig}(A)}\rangle,\ f \mapsto
\langle f, \mathrm{sig}(f)\rangle\, ,
$
единица этого сопряжения тождественна. Естественным образом
конструируются трансформации и конфигурации АО-моделей~\cite{12-kov}.
В~результате получается AO-\textit{тех\-но\-ло\-гия}~--- формальная
технология проектирования, в которой основной категорией моделей служит
AO, а интерфейсы выделяются функтором $\mathrm{mod}$. Важное прикладное
значение имеет критерий структурируемости этой технологии, который будет
сформулирован и доказан в разд.~5.

 Аспектом (aspect) называется элементарный строительный блок
 ас\-пект\-но-ори\-ен\-ти\-ро\-ван\-ной программы (модели), реализующий
отдельный класс задач. Аспекты способны сохранять свою идентичность в
составе программы, поэтому их аспектная структура не разрушается при
интеграции~[17]. Хорошо сохраняют структуру действия по интеграции,
обладающие на уровне аспектных структур обратимостью \textit{слева}, т.\,е.\
воз\-мож\-ностью идентифицировать аспектную структуру компонента в~со\-ста\-ве
системы путем трассирования\linebreak (см.\ пояснения перед определением~2). Поэтому
AO-мор\-физм $\langle f, b\rangle$ называется \textit{аспектным}, если
$\mathrm{SIG}$-мор\-физм~$b$ обратим слева, и,~в~частности, \textit{изоаспектным},
если $b$ является изоморфизмом. Например, в~АО-тех\-но\-ло\-гии
событийного моделирования аспектными являются все отображения
помеченных сценариев c непустой областью, которые не <<склеивают>>
различные метки.

 \smallskip

 \noindent
 \textbf{Определение 4.} Аспектно-ориентированная модель $A$ называется \textit{аспектом}, если
любой AO-морфизм с областью~$A$ является аспектным.\hfill$\square$

 Сборка программ из аспектов выходит за рамки традиционной компоновки
модулей (linking), поэтому она называется связыванием (weaving). Технологии
АОП предлагают разнообразные инструменты связывания: предкомпиляторы
исходных текстов программ, преобразователи исполняемого байт-ко\-да,
диспетчеры вызова методов. Связывание можно аксиоматически описать
универсальной конструкцией в категории AO~--- копределом диаграммы
соединения специального вида~[17].

\section{Модуляризация аспектов}

 Наиболее явным способом разметки аспектов, составляющих АО-модель,
является их модуляризация, т.\,е.\ оформление в виде отдельных единиц
модульной архитектуры: объектов, таблиц в базе данных и~т.\,д. Такая
модуляризация называется разделением ответственности (separation of concerns)
и входит в число основных целей привлечения АОП~[19].
Модуляризированные аспекты можно собирать в~системы посредством
компоновки, не прибегая к связыванию, что позволяет снизить затраты на
сборку за счет применения широкодоступных <<модульных>> технологий
программирования. В~связи с этим при использовании расширений
традиционных технологий средствами АОП, таких как AspectJ, программы,
скомпонованные из модулей без применения связывания, тривиальным образом
разделяются на модуляризируемые аспекты~[4].

 Модуляризируемые АО-модели выделяются среди прочих тем, что при
интеграции с модулями они ведут себя так же, как модульные единицы.
Возможности интеграции модулей в~АО-мо\-дель определяются ее модульной
основой. В~свою очередь, возможности интеграции АО-мо\-де\-ли в модули
определяются ее аспектной структурой, поскольку при интеграции в~модуль
аспекты, составляющие АО-мо\-дель, выступают в~качестве ее элементарных
единиц. Формально, модуляризируемые модели образуют полную
подкатегорию в~AO (которая будет обозначаться через $m$-AO) такую, что
существует %ас\-пект\-но-ори\-ен\-ти\-ро\-ван\-ное расширение
АО-рас\-ши\-ре\-ние модульной технологии~--- вложение $\mathrm{am}:\
 c\mbox{-}\mathrm{DESC} \mapsto m$-AO, пол\-ностью воспроизводящее
интеграционные возможности модулей в~сле\-ду\-ющем смысле. С~одной
стороны, все способы интеграции модуля $M\hm\in \mathrm{Ob}\,
 c\mbox{-}\mathrm{DESC}$ в~АО-мо\-дель $A\hm\in \mathrm{Ob}\, m\mbox{-}\mathrm{AO}$
задаются множеством морфизмов $\mathrm{Mor}\,(\mathrm{am}(M), A)$, поэтому функтор
выделения модульного интерфейса $\mod$ (точнее, его ограничение на
 $m$-$\mathrm{AO}$, обозначаемое далее через $\mathrm{am}_*$) должен устанавливать
биекцию между множеством $\mathrm{Mor}\,(\mathrm{am}(M),A)$ и множеством
$\mathrm{Mor}\,(M, \mod(A))$. С~другой стороны, все способы интеграции~$A$
в~$M$ задаются множеством морфизмов $\mathrm{Mor}\,(A, \mathrm{am}(M))$, поэтому
должен существовать функтор модуляризации аспектной структуры $\mathrm{am}^*:\
m\mbox{-}\mathrm{AO} \hm\to c\mbox{-}\mathrm{DESC}$, тривиально дей\-ст\-ву\-ющий
на модули ($\mathrm{am}^* \circ \mathrm{am} \hm= 1_{c\mbox{-}\mathrm{DESC}})$
и~уста\-нав\-ли\-ва\-ющий биекцию между
$\mathrm{Mor}\,(A, \mathrm{am}(M))$ и~множеством $\mathrm{Mor}\,(\mathrm{am}^*(A), M)$. При
этом возможность трактовать $c$-DESC-объ\-ект $\mathrm{am}^*(A)$ как <<подъем>>
аспектной структуры АО-мо\-де\-ли~$A$ на модульный уровень
обеспечивается следующим дополнительным требованием естественности:
функтор $\mathrm{sig} \circ \mathrm{am}^*$, выделяющий интеграционный интерфейс из
модуляризированной аспектной структуры, должен совпадать с~ограничением
на $m$-$\mathrm{AO}$ функтора $\mathrm{str}$, определяющего исходную аспектную структуру
на уровне интеграционных интерфейсов.

 Примером расширения, которым обладает любая формальная технология
проектирования, служит изоморфизм между $c$-DESC и полной
подкатегорией в~$\mathrm{AO}$ с~классом объектов
$\{\langle A, 1_{\mathrm{sig}(A)}\rangle \vert A \hm\in
\mathrm{Ob}\, c\mbox{-}\mathrm{DESC}\}$, действующий как функтор $\mathrm{mod}^*$.
Например, в~технологии событийного моделирования он порождает дискретно
размеченные сценарии, которые являются самыми
 <<ас\-пект\-но-не\-ори\-ен\-ти\-ро\-ван\-ны\-ми>>~--- в~них каждый класс
задач помечает только одно событие, так что никакого рассеивания не
происходит. Назовем такое АО-рас\-ши\-ре\-ние тривиальным. Можно
проверить, что действие любого АО-рас\-ши\-ре\-ния (с~точ\-ностью до
естественного изоморфизма), по существу, совпадает с~его действием: модули
(т.\,е.\ $c$-DESC-объ\-ек\-ты) всегда переходят в~АО-мо\-де\-ли, история
получения интерфейсов которых из классов задач путем трансформации
утрачена (тривиальна). Таким образом, АО-рас\-ши\-ре\-ние, по существу,
однозначно определяется своей кообластью~--- классом всех
модуляризируемых АО-мо\-де\-лей. Легко видеть, что конструкция
 АО-рас\-ши\-ре\-ния устойчива относительно ограничения: любая полная
подкатегория в $m$-$\mathrm{AO}$, содержащая класс $\mathrm{am}(\mathrm{Ob}\,
 c\mbox{-}\mathrm{DESC})$, является кообластью АО-рас\-ши\-ре\-ния, действующего
так же, как $\mathrm{am}$. В~связи с~этим интерес представляют расширения, кообласть
которых достаточно <<ве\-лика>>.
{\looseness=1

}

 На языке теории категорий требования к АО-рас\-ши\-ре\-нию компактно
формулируются при помощи конструкции сопряжения функторов.

 \medskip

 \noindent
 \textbf{Определение 5.} Функтор $\mathrm{am}:\ c\mbox{-}\mathrm{DESC} \to m\mbox{-}\mathrm{AO}$,
где $m$-$\mathrm{AO}$~--- некоторая полная подкатегория в~$\mathrm{AO}$, называется
%\textit{ас\-пект\-но-ори\-ен\-ти\-ро\-ван\-ным расширением}
\textit{АО-рас\-ши\-ре\-ни\-ем} формальной технологии, если он обладает
следующими сопряженными функторами:
 \begin{itemize}
\item правый сопряженный $\mathrm{am}_*$ с тождественной единицей, причем
$\mathrm{am}_*(f)\hm = \mod(f)$ для любого $m$-$\mathrm{AO}$-морфизма~$f$;
\item левый сопряженный $\mathrm{am}^*$ с тождественной коединицей, причем
$\mathrm{sig}(\mathrm{am}^*(f)) \hm= \mathrm{str}(f)$ для любого
$m$-$\mathrm{AO}$-морфизма~$f$.
\end{itemize}

Аспектно-ориентированное расширение $\mathrm{am}$ называется:
 \begin{itemize}
\item \textit{тривиальным}, если оно является изоморфизмом;
\item \textit{наибольшим}, если любое АО-рас\-ши\-ре\-ние $\mathrm{am}^\prime:\
c\mbox{-}\mathrm{DESC} \to m\mbox{-}\mathrm{AO}^\prime$ удовлетворяет условию
$\mathrm{Ob}\, m\mbox{-}\mathrm{AO}^\prime \hm\subseteq \mathrm{Ob}\, m\mbox{-}\mathrm{AO}$;
\item \textit{полным}, если для любой АО-мо\-де\-ли существует
$m$-$\mathrm{AO}$-объект, изоморфный ей.\hfill$\square$
\end{itemize}

 При аксиоматическом описании модуляризации аспектов, составляющих
АО-мо\-де\-ли, ключевую роль играет единица сопряжения $\mathrm{am}^* \dashv \mathrm{am}$,
которая будет обозначаться через~$\eta$. По определению для любого
 $m$-$\mathrm{AO}$-объек\-та~$S$ $m$-$\mathrm{AO}$-мор\-физм $\eta_S:\ S \hm\to \mathrm{am}(\mathrm{am}^*(S))$
является прообразом $c$-DESC-мор\-физ\-ма $1_{\mathrm{am}^*(S)}$ при биекции $\mathrm{am}^*:\
\mathrm{Mor}\,(S, \mathrm{am}(\mathrm{am}^*(S))) \hm\cong \mathrm{Mor}\,(\mathrm{am}^*(S), \mathrm{am}^*(S))$,
\mbox{поэтому} $\mathrm{str}(\eta_S) \hm= \mathrm{sig}(\mathrm{am}^*(\eta_S)) \hm= 1_{\mathrm{sig}(\mathrm{am}^*(S))} \hm= 1_{\mathrm{str}(S)}$ ввиду
определения~5, т.\,е.\ морфизм~$\eta_S$ изоаспектен и~его действие\linebreak
нетривиально только на уровне модульной основы. А~поскольку
 $c$-DESC-объ\-ект $\mathrm{am}^*(S)$ представляет на модульном уровне
аспектную структуру АО-мо\-де\-ли~$S$, $c$-DESC-мор\-физм $\mod(\eta_S):\
\mod(S) \hm\to \mathrm{am}^*(S)$ можно рассматривать как канонический способ
интеграции модульной осно-\linebreak вы модели в ее модуляризированную аспектную
структуру. Этот морфизм является регулярным
 \mbox{$c$-DESC}-эпи\-мор\-физ\-мом (см.\ следствие~1.2 ниже), т.\,е.\ задает
факторизацию модульной основы на аспекты. \mbox{Семейство} $\mod(\eta_S)$,
$S\hm\in \mathrm{Ob}\, m\mbox{-}\mathrm{AO}$, образует естественное преобразование
функтора выделения модульной основы $\mathrm{am}_*$ в функтор выделения
аспектной структуры $\mathrm{am}^*$.

 \medskip

 \noindent
 \textbf{Определение 6.} ($\mathrm{am}$-)\textit{модуляризируемой АО-мо\-делью}
называется любой $m$-$\mathrm{AO}$-объ\-ект. ($\mathrm{am}$)-\textit{мо\-ду\-ля\-ри\-за\-цией}
(аспектной структуры) модуляризируемой\linebreak АО-мо\-де\-ли~$S$ называется
 $c$-DESC-мор\-физм $\mathrm{mod}\,(\eta_S)$, где $\eta$~--- единица сопряжения
$\mathrm{am}^* \dashv \mathrm{am}$. Тож\-де\-ст\-вом ($\mathrm{am}$-)\textit{модуляризации}
произвольного\linebreak
\mbox{$m$-$\mathrm{AO}$}-мор\-физ\-ма $m:\ A \hm\to S$ называется равенство
$\mathrm{am}^*(m) \circ
\mathrm{mod}\,(\eta_A) \hm= \mathrm{mod}\,(\eta_S) \circ \mathrm{mod}\,(m)$.\hfill$\square$

 \medskip

 Естественным способом извлечения модуляризированного аспекта из
модульной основы модуляризируемой АО-мо\-де\-ли~$S$ является
трассирование метки аспекта (обозначения классов задач) вдоль модуляризации
аспектной структуры модели. Если $A$~--- извлекаемый аспект, то должна
существовать трассируемая модульная трансформация $r_m:\ \mathrm{am}^*(A) \hm\to
\mathrm{am}^*(S)$ его (модуляризированной) метки в~(модуляризированную) аспектную
структуру модели~$S$. Идентификация метки в~аспектной структуре
производится путем прямого трассирования~--- морфизма, обратного справа
к~трассе~$r_m^{\mathrm{op}}$ (см.\ пояснения перед определением~2). Таким
морфизмом должен служить $\mathrm{am}^*(m)$, если $m$-$\mathrm{AO}$-мор\-физм~$m$ задает
вложение~$A$ в~$S$. Поэтому $\mod(m)$ должен выделять в~$\mod(S)$ полный
прообраз \mbox{$c$-DESC}-объ\-екта $\mathrm{am}^*(A)$ как подобъекта в $\mathrm{am}^*(S)$
относительно $\mod(\eta_S)$. Следовательно, тождество модуляризации для
 $m$-$\mathrm{AO}$-мор\-физ\-ма $m:\ A\hm\to S$ должно задавать декартов квадрат.
Таким образом, подаспекты~--- это вложения аспектов, модуляризация которых
имеет универсальный характер. Благодаря этому, в частности, подаспекты
являются подобъектами в категории~$\mathrm{AO}$. Кроме того, аспекты являются
атомарными единицами разделения ответственности~--- они не\linebreak имеют
собственных подаспектов (см.\ предложение~2 ниже).

%fig9
 \vspace*{6pt}
 \begin{center}
 \mbox{%
 \epsfxsize=59.413mm
 \epsfbox{kov-9.eps}
 }
 \end{center}

% \vspace*{6pt}


 \noindent
 \textbf{Определение 7.} $m$-$\mathrm{AO}$-мор\-физм $m:\ A \hm\to S$ называется
($\mathrm{am}$-)\textit{подаспектом} АО-мо\-де\-ли~$S$, если он удовлетворяет
следующим условиям:

\columnbreak

\noindent
 \begin{enumerate}[($i$)]
\item АО-модель $A$ является аспектом;
\item существует трассируемая трансформация
$r_m:\ \mathrm{am}^*(A) \to \mathrm{am}^*(S)$
такая, что $r_m^{\mathrm{op}}\hm\circ \mathrm{am}^*(m) \hm= 1_{\mathrm{am}^*(A)}$;
\item тождество $\mathrm{am}$-модуляризации для $m$ задает декартов квадрат в
категории $c$-DESC.\hfill$\square$
\end{enumerate}

 \noindent
 \textbf{Предложение 1.} Любой подаспект является аспектным регулярным
$\mathrm{AO}$-мо\-но\-мор\-физ\-мом.

 \smallskip

 \noindent
 Д\,о\,к\,а\,з\,а\,т\,е\,л\,ь\,с\,т\,в\,о\,.\ \ Покажем, что если $m:\ A \hm\to S$~---
подаспект, то тождество единицы $\mathrm{am}(\mathrm{am}^*(m)) \circ \eta_A \hm= \eta_{S} \circ
m$ задает декартов квад\-рат в категории~$\mathrm{AO}$. Действительно, функтор $\mod$
переводит его в тождество модуляризации, которое является декартовым
квадратом по условию~($iii$) определения~7. Функтор $\mathrm{int}$ переводит его в
тож\-де\-ст\-во $\mathrm{str}(m) \circ \mathrm{int}(\eta_A) \hm= \mathrm{int}(\eta_S) \circ \mathrm{int}(m)$, которое
представляет собой $\mathrm{sig}$-об\-раз тождества модуляризации, так что задает
декартов квадрат в~$\mathrm{SIG}$ (ввиду наличия у~функтора $\mathrm{sig}$ левого
сопряженного, он сохраняет все пределы~[11, предложение~18.9]). Наконец,
функтор $\mathrm{str}$ переводит тождество единицы в коммутативный квадрат в~$\mathrm{SIG}$,
две параллельных стороны которого ($\mathrm{str}(\eta_A)$ и~$\mathrm{str}(\eta_S)$) являются
тождественными морфизмами, он также декартов по определению. Используя
эти факты, можно непосредственно проверить, что тождество единицы задает
декартов квадрат. Следовательно, $m$~--- уравнитель пары
 $\mathrm{AO}$-мор\-физ\-мов~$\eta_S$, $(\mathrm{am}(\mathrm{am}^*(m) \circ r_m^{\mathrm{op}}) \circ \eta_S):\ S
\rightrightarrows \mathrm{am}(\mathrm{am}^*(S))$.\hfill$\square$

 \smallskip

 \noindent
 \textbf{Предложение 2.} Следующие утверждения эквивалентны для любого
подаспекта $m:\ A \hm\to S$:
 \begin{enumerate}[($i$)]
\item $S$ является модуляризируемым аспектом;
 \item $m$ является изоморфизмом;
\item $m$ изоаспектен.
\end{enumerate}

\noindent
 Д\,о\,к\,а\,з\,а\,т\,е\,л\,ь\,с\,т\,в\,о\,.\ \

$(i) \Rightarrow (iii)$. Если $S$~--- аспект, то $\mathrm{AO}$-мор\-физм
$t \circ \eta_S:\ S \hm\to \mathrm{am}(\mathrm{am}^*(A))$, где
$t \hm= \mathrm{am}(r_m^{\mathrm{op}})$, является аспектным.
Поскольку~$\eta_S$ изоаспектен, $\mathrm{str}(t)$ обратим слева; следовательно,
$\mathrm{AO}$-мор\-физм~$t$, будучи в~свою очередь обратимым справа,
изоаспектен. Поэтому $\mathrm{SIG}$-мор\-физм $w \hm= \mathrm{str}(\mathrm{am}(\mathrm{am}^*(m)))$,
правый обратный к~$\mathrm{str}(t)$, является изоморфизмом. В~свою очередь,
применяя функтор $\mathrm{str}$ к~тождеству единицы
$\mathrm{am}(\mathrm{am}^*(m)) \circ \eta_A \hm= \eta_S \circ m$,
получаем, что $w \hm= \mathrm{str}(m)$.

 $(iii) \Rightarrow (ii)$. Как отмечалось в доказательстве предложения~1,
 в~$\mathrm{SIG}$ имеется декартов \mbox{квад\-рат}, задаваемый тождеством $\mathrm{str}(m) \circ \mathrm{int}(\eta_A)
\hm= \mathrm{int}(\eta_S) \circ \mathrm{int}(m)$. По предположению $\mathrm{str}(m)$~--- изоморфизм,
поэтому $\mathrm{int}(m)$, будучи параллельным ему ребром декартова квадрата, также
\mbox{является} изоморфизмом, в частности эпиморфизмом. А~поскольку функтор
$\mathrm{int}$, будучи композицией унивалентных функторов $\mod$ и~$\mathrm{sig}$, сам
унивалентен, он отражает эпиморфизмы~[11, предложение~7.44]. В~свою
очередь, если $m$~--- эпиморфизм, то в силу предложения~1 он является
изоморфизмом.

 $(ii)\Rightarrow (i)$. Вытекает из определения~4.\hfill$\square$

 \smallskip

 Естественным (хотя и не единственным) способом модуляризации аспектной
структуры АО-мо\-де\-ли является восстановление трассируемой
трансформации модулей, индуцирующей ее разметку. Действительно, если для
АО-мо\-де\-ли $\langle A, l\rangle$ существует подходящая трассируемая
трансформация $s:\ X \to A$ некоторого $c$-DESC-объ\-ек\-та~$X$ в~$A$,
удовлетворяющая условию $\mathrm{sig}(s^{\mathrm{op}}) \hm= l$, то $X$ можно
рассматривать как модульную единицу, состоящую из всех классов задач
 АО-мо\-де\-ли, а~$s^{\mathrm{op}}$~--- как <<каноническую>> модуляризацию
ее аспектной структуры. Например, у~помеченного сценария трансформация
восстанавливается (очевидным образом) из такой и только такой разметки,
которая разбивает свою область на хорошо упорядоченную совокупность
прообразов точек, причем прообраз любой точки является подаспектом.
Очевидными примерами служат любой аспект (единственным подаспектом
которого ввиду предложения~2 является он сам) и любой непустой дискретно
размеченный сценарий (подаспектом в котором служит любое одноэлементное
подмножество).
Показательным примером является также сценарий, в котором
меткой каждого события выступает содержащая его компонента связности
порядка: множество всех подаспектов такого сценария совпадает с множеством
взаимно независимых подсценариев, так что с формальной точки зрения
распараллеливание (см.\ конструкцию функтора $\mathrm{sconn}$ в конце разд.~2)
оказывается частным случаем модуляризации аспектов.

\section{Частичная модуляризация}

 Для формальных технологий, не обладающих полным
 АО-рас\-ши\-ре\-ни\-ем, можно предложить более слабые подходы к
модуляризации аспектной структуры. В~частности, аспектную разметку
произвольной АО-мо\-де\-ли можно представить на модульном уровне
частичным морфизмом модульной основы. Напомним, что \textit{частичным
морфизмом} объекта~$X$ в~$Y$ называется пара морфизмов с общим
началом, один из которых является мономорфизмом и направлен в~$X$ (он
выделяет <<часть>> объекта~$X$, выступающую областью определения
частичного морфизма), а другой произволен и направлен в~$Y$ (он задает
действие частичного морфизма)~[11, определение~28.1(1)]. При некоторых
технических ограничениях определена композиция частичных морфизмов.
Частичный морфизм можно рас\-смат\-ри\-вать как диаграмму со схемой $*
\leftarrow * \to *$, обозначаемой через~$V$.

 Разметка произвольной АО-мо\-де\-ли порождает частичный
 $c$-DESC-мор\-физм следующим образом. Воспользуемся тем, что
коединица~$\varepsilon$ сопряжения $\mathrm{sig}^* \dashv \mathrm{sig}$ состоит из
мономорфизмов, поскольку $\mathrm{sig}(\varepsilon_A) \hm= 1_{\mathrm{sig}(A)}$, $A\hm\in
\mathrm{Ob}\, c\mbox{-}\mathrm{DESC}$, и любой унивалентный функтор отражает
мономорфизмы~[11, предложение~7.37(2)]. При помощи коединицы можно
выделить в модульной основе АО-мо\-де\-ли дискретную <<часть>> и далее
подействовать на нее дискретной реализацией аспектной разметки.

 \medskip

 \noindent
 \textbf{Определение 8.} \textit{Частичной модуляризацией} произвольной
АО-мо\-де\-ли $\langle A, l:\ \mathrm{sig}(A) \hm\to L\rangle$ называется следующий
частичный $c$-DESC-мор\-физм, действующий из ее модульной основы~$A$
в дискретную реализацию аспектной структуры~$L$:
 $$
 ~~~~~~~~~~~~~~\varepsilon_A:\ A\leftarrow \mathrm{sig}^*(\mathrm{sig}(A)) \to \mathrm{sig}^*(L):\
\mathrm{sig}^*(l).~~~~~~~~~~~~~\square
 $$

 \noindent
 \textbf{Предложение 3.} Отображение, сопоставляющее каж\-дой
 АО-мо\-де\-ли ее частичную модуляризацию, задает полное вложение
категории $\mathrm{AO}$ в категорию диаграмм $c$-DESC$^V$.

 \medskip

 \noindent
 Д\,о\,к\,а\,з\,а\,т\,е\,л\,ь\,с\,т\,в\,о\,.\ \ Выберем произвольно АО-мо\-де\-ли
$S \hm= \langle A, l:\ \mathrm{sig}(A) \to L\rangle$ и $s^\prime\hm = \langle A^\prime, l^\prime:\
\mathrm{sig}(A^\prime) \to L^\prime\rangle$, положим $A^*\hm = \mathrm{sig}^*(\mathrm{sig}(A))$,
$A^{\prime*}\hm = \mathrm{sig}^*(\mathrm{sig}(A^\prime))$. Если $\varepsilon_A \hm=
\varepsilon_{A^\prime}$ и $\mathrm{sig}^*(l) \hm= \mathrm{sig}^*(l^\prime)$, то $A\hm =
\mathrm{codom}\, \varepsilon_A\hm = \mathrm{codom}\,\varepsilon_{A^\prime} \hm=
A^\prime$ и $l \hm= \mathrm{sig}(\mathrm{sig}^*(l)) \hm= \mathrm{sig}(\mathrm{sig}^*(l^\prime)) \hm= l^\prime$.
Далее, любой $\mathrm{AO}$-мор\-физм $\langle f, b\rangle:\ S \hm\to S^\prime$ определяет
семейство $c$-DESC-мор\-физ\-мов $f:\ A \hm\to A^\prime$,
\mbox{$f^*:\ A^* \hm\to A^{\prime *}$,}
$\mathrm{sig}^*(b):\ \mathrm{sig}^*(L) \hm\to \mathrm{sig}^*(L^\prime)$ (здесь $f^* \hm=
\mathrm{sig}^*(\mathrm{sig}(f)))$, которое является естественным преобразованием диаграмм
частичной модуляризации, т.\,е.\ $c$-DESC$^V$-мор\-физ\-мом: по определению
коединицы имеем $f \circ \varepsilon_A \hm= \varepsilon_{A^\prime}\circ f^*$, а
по определению $\mathrm{AO}$-мор\-физ\-ма~--- $\mathrm{sig}^*(b) \circ \mathrm{sig}^*(l) \hm=
\mathrm{sig}^*(l^\prime) \circ f^*$. Ясно, что различные
 AO-мор\-физ\-мы определяют различные естественные преобразования
такого вида.

%fig10
\vspace*{9pt}
 \begin{center}
 \mbox{%
 \epsfxsize=69.856mm
 \epsfbox{kov-10.eps}
 }
 \end{center}


 Таким образом, задано вложение $\mathrm{pmao}:\ \mathrm{AO}\hm\hookrightarrow
 c\mbox{-}\mathrm{DESC}^V$. Чтобы убедиться в его полноте, рассмотрим
произвольное естественное преобразование диаграммы $\mathrm{pmao}(S)$ в
$\mathrm{pmao}(S^\prime)$~--- тройку\linebreak
 $c$-DESC-мор\-физ\-мов $p:\ A\hm\to A^\prime$,
$q:\ A^* \hm\to A^{\prime *}$, $s:\ \mathrm{sig}^*(L)\hm\to \mathrm{sig}^*(L^\prime)$,
удовлетворяющих соотношениям $p\circ\varepsilon_A\hm =
\varepsilon_{A^\prime}\circ q$ и $s \circ \mathrm{sig}^*(l) \hm= \mathrm{sig}^*(l^\prime) \circ q$.
Поскольку $\varepsilon$ состоит из мономорфизмов, из первого равенства
вытекает, что $q \hm= \mathrm{sig}^*(\mathrm{sig}(p))$. С~учетом этого, применяя функтор
$\mathrm{sig}$ ко
второму равенству, получаем, что $\mathrm{sig}(s) \circ l \hm= l^\prime \circ \mathrm{sig}(p)$.
Следовательно, существует $\mathrm{AO}$-мор\-физм $\langle p, \mathrm{sig}(s)\rangle:\ S \hm\to
S^\prime$.\hfill$\square$

 Конечно, частичная модуляризация АО-мо\-де\-ли не отражает
трансформацию, порождающую модульную основу АО-мо\-де\-ли из аспектной
структуры, поскольку действие частичной модуляризации не обязано быть
трассой трансформации. Тем не менее, можно построить <<аппроксимацию>>
модуляризации аспектной структуры, вычисляя копредел диаграммы частичной
модуляризации (если он существует: ребро копредела частичного морфизма,
параллельное его действию, можно рас\-смат\-ри\-вать как универсальное
расширение частичного морфизма на всю свою область). По существу, речь
идет о~соединении (в~смысле разд.~2) модульной основы с дискретной
реализацией аспектной структуры АО-мо\-де\-ли. Оказывается, таким способом
можно вычислить модуляризацию аспектной структуры любой
модуляризируемой АО-мо\-де\-ли. При некотором техническом ограничении
отсюда выводится существование наибольшего АО-рас\-ши\-ре\-ния. Оно
модуляризирует в точности все АО-мо\-де\-ли, аспектную разметку которых
можно <<поднять>> на модульный уровень в виде регулярного эпи\-мор\-физма.
{\looseness=1

}

 Таким образом, переход к обобщенной частичной процедуре модуляризации
аспектной структуры позволяет по-но\-во\-му взглянуть на ее исходную
полную форму и установить ряд ее важных свойств. Отправным пунктом
служит следующий факт: существование копредела у~любой диаграммы
частичной модуляризации является критерием структурируемости
 АО-тех\-но\-ло\-гии~--- необходимого условия существования полного
 АО-рас\-ши\-ре\-ния (см.\ следствие~1.3 ниже). Например, этому критерию
удовлетворяет технология событийного моделирования, поскольку любая
 \textbf{Pos}-диа\-грам\-ма имеет копредел.

 \medskip

 \noindent
 \textbf{Теорема 1.} \textit{Аспектно-ориентированная тех\-но\-ло\-гия
 структурируема тогда
и~только тогда, когда диаграмма частичной модуляризации любой
 АО-мо\-де\-ли имеет копредел.}

 \medskip

 \noindent
 Д\,о\,к\,а\,з\,а\,т\,е\,л\,ь\,с\,т\,в\,о\,.\ \ Рассмотрим произвольные
 $\mathrm{AO}$-диа\-грам\-мы $f:\ A\leftarrow C\to B:\ g$ и $f^\prime:\
A^\prime\hm\leftarrow C^\prime\hm\to B^\prime:\ g^\prime$, обладающие
кодекартовыми квад\-ратами $p\circ f \hm= q \circ g$ и $p^\prime \circ
f^\prime\hm= q^\prime \circ g^\prime$ с~вершинами~$D$ и~$D^\prime$
соответственно. Если $\mathrm{AO}$-мор\-физ\-мы $a:\ A\hm\to A^\prime$, $c:\ C\hm\to
C^\prime$, $b:\ B\hm\to B^\prime$ образуют естественное преобразование этих
диаграмм, то существует единственный $c$-DESC-мор\-физм $d:\ D\hm\to
D^\prime$, удов\-ле\-тво\-ря\-ющий условиям $d\circ p \hm= p^\prime \circ a$ и $d\circ
q \hm= q^\prime \circ b$ (т.\,е.\ делающий куб в~$c$-DESC, составленный из
двух параллельных кодекартовых квадратов и морфизмов $a$, $c$, $b$, $d$,
коммутативным). Обозначим морфизм~$d$ через $\mathrm{colim}(a, c, b)$. Легко
проверить, что для любой подкатегории в~$c$-DESC$^V$, все объекты которой
обладают кодекартовыми квадратами, отображение $\mathrm{colim}$ является функцией
морфизмов функтора, действующего из этой подкатегории в~$c$-DESC,
переводя любой ее объект в~вершину кодекартова квадрата над ним.
В~частности, если все диаграммы из класса $\mathrm{pmao}(\mathrm{Ob}\, \mathrm{AO})$ имеют
копредел, то можно определить функтор $\mathrm{pmod} \hm= \mathrm{colim} \circ \mathrm{pmao}:\
\mathrm{AO}\hm\to c\mbox{-}\mathrm{DESC}$.

%fig11
\vspace*{9pt}
 \begin{center}
 \mbox{%
 \epsfxsize=48.551mm
 \epsfbox{kov-11.eps}
 }
 \end{center}
 \vspace*{9pt}


 Проверим, что функтор $\mathrm{pmod}$ является левым сопряженным к~функтору
$\mathrm{mod}^*$ с~тождественной коединицей, так что согласно определению~1
 АО-тех\-но\-ло\-гия структурируема. Поскольку любая\linebreak АО-мо\-дель из
класса $\mathrm{mod}^*(\mathrm{Ob}\, c\mbox{-}\mathrm{DESC})$ имеет тож\-де\-ст\-вен\-ный морфизм
в~качестве разметки, функтор $\mathrm{colim}$ можно выбрать так, чтобы выполня-\linebreak лось
соотношение $\mathrm{pmod} \circ \mathrm{mod}^* \hm= 1_{c\mbox{-}\mathrm{DESC}}$.\linebreak Покажем, что для любых
$c$-DESC-объ\-ек\-та~$P$, АО-мо\-де\-ли $S \hm= \langle A, l:\ \mathrm{sig}(A) \hm\to
L\rangle$ и~$c$-DESC-мор\-физ\-ма $p:\: \mathrm{pmod}(S) \hm\to P$ существует
единственный $\mathrm{AO}$-мор\-физм $r:\ S \hm\to \mod^*(P)$ такой, что
$\mathrm{pmod}(r)
\hm= p$. Действительно, пусть $u:\ A \hm\to \mathrm{pmod}(S) \leftarrow \mathrm{sig}^*(L):\ v$~---
ребра копредела диаграммы $\mathrm{pmao}(S)$, в~част\-ности $u \circ \varepsilon_A\hm
= v \circ \mathrm{sig}^*(l)$. Отсюда $p\circ u \circ\varepsilon_A\hm = p\circ v \circ \mathrm{sig}^*(l)$,
следовательно, $\mathrm{sig}(p\circ u) \hm= \mathrm{sig}(p\circ v) \circ l$, так что имеется
 $\mathrm{AO}$-мор\-физм $\langle p \circ u, \mathrm{sig}(p \circ v)\rangle:\ S \hm\to
 \mathrm{mod}^*(P)$.
Ввиду предложения~3 под действием функтора $\mathrm{pmod}$ он переходит в~$p$,
поэтому является искомым~$r$ (его единственность вытекает из того, что
1$_P$ и~$\varepsilon_P$~--- мономорфизмы, как видно на диаграмме).

 %fig12
%\vspace*{1pt}
 \begin{center}
 \mbox{%
 \epsfxsize=53.822mm
 \epsfbox{kov-12.eps}
 }
 \end{center}


 Обратно, предположим, что функтор $\mathrm{mod}^*$ обладает левым сопряженным
$\mathrm{mod}^{**}:\ \mathrm{AO}\hm\to c\mbox{-}\mathrm{DESC}$ с~тождественной коединицей.
Обозначим через $\kappa$ единицу этого сопряжения. Для произвольной
 АО-мо\-де\-ли $S\hm = \langle A, l:\ \mathrm{sig}(A) \hm\to L\rangle$ положим $T\hm =
\mathrm{mod}^{**}(S)$, $s \hm= \mod(\kappa_S):\ A \hm\to T$. По определению
 $\mathrm{AO}$-мор\-физ\-ма имеем $\mathrm{str}(\kappa_S) \circ l \hm= 1_{\mathrm{sig}(T)}\circ \mathrm{sig}(s) \hm=
\mathrm{sig}(s)$, поэтому тождество коединицы сопряжения $\mathrm{sig}^* \dashv \mathrm{sig}$ для~$s$
можно записать в виде $s \circ \varepsilon_A\hm = \varepsilon_{T}\circ
\mathrm{sig}^*(\mathrm{sig}(s)) \hm= w\circ \mathrm{sig}^*(l)$,\linebreak
 где $w\hm = \varepsilon_T\circ
\mathrm{sig}^*(\mathrm{str}(\kappa_S))$. Проверим, что оно определяет кодекартов
квад\-рат~---
копредел диаграммы\linebreak $\mathrm{pmao}(S)$. Выберем произвольно пару
 $c$-DESC-мор\-физ\-мов $a:\ A \hm\to X \leftarrow \mathrm{sig}^*(L):\ y$ такую, что
$a\circ \varepsilon_A\hm = y \circ \mathrm{sig}^*(l)$.
Применяя функтор $\mathrm{sig}$ к~этому\linebreak
равенству, получаем $\mathrm{sig}(a) \hm= \mathrm{sig}(y) \circ l$,
так что имеется %\linebreak
 $\mathrm{AO}$-мор\-физм $f \hm= \langle a, \mathrm{sig}(y)\rangle:\ S \hm\to
 \mathrm{mod}^*(X)$.
Тождество единицы для него имеет вид $f \hm= \mathrm{mod}^*(q) \circ \kappa_S$, где
$q\hm = \mathrm{mod}^{**}(f):\ T \hm\to X$. Применяя к~этому тож\-де\-ст\-ву функтор~$\mathrm{mod}$,
получаем, что $a \hm= q \circ s$. Применяя к~нему же функтор $\mathrm{str}$,
получаем,
что $\mathrm{sig}(y) \hm= \mathrm{str}(\mathrm{mod}^*(q)) \circ \mathrm{str}(\kappa_S) \hm= \mathrm{sig}(q \circ w)$, откуда $y
\hm= q \circ w$ ввиду унивалентности функтора $\mathrm{sig}$. Поэтому $q$~--- стрелка
копредела (ее единственность вытекает из определения сопряжения: для
любого $c$-DESC-мор\-физ\-ма~$q^\prime$ соотношения $a\hm= q^\prime
\circ s$\linebreak и~$y \hm= q^\prime \circ w$ влекут $f \hm= \mathrm{mod}^*(q^\prime) \circ
\kappa_S$, откуда\linebreak $q^\prime \hm= q$).\hfill$\square$

%fig13
 \begin{center}
 \mbox{%
 \epsfxsize=76.969mm
 \epsfbox{kov-13.eps}
 }
 \end{center}



 \noindent
 \textbf{Следствие 1.1.} Диаграмма частичной модуляризации любой
модуляризируемой АО-мо\-де\-ли имеет копредел, ребро которого,
параллельное действию частичной модуляризации, задает модуляризацию
аспектной структуры модели.\hfill$\square$

 \smallskip

 \noindent
 \textbf{Следствие 1.2.} Модуляризация аспектной структуры любой
модуляризируемой АО-мо\-де\-ли является регулярным эпиморфизмом.
\hfill$\square$

 \smallskip

 \noindent
 \textbf{Следствие 1.3.} Если некоторая формальная технология
проектирования обладает полным АО-рас\-ши\-ре\-ни\-ем, то
 АО-тех\-но\-ло\-гия над ней структурируема. \hfill$\square$

 \smallskip


 \noindent
 \textbf{Следствие 1.4.} Если категория $c$-DESC подвижна, то
формальная технология проектирования обладает наибольшим
 АО-рас\-ши\-ре\-ни\-ем, причем АО-мо\-дель $\langle A, l\rangle$ содержится
в его кообласти тогда и только тогда, когда существует регулярный
 $c$-DESC-эпи\-мор\-физм с областью~$A$, переходящий в~$l$ под
действием функтора $\mathrm{sig}$.\hfill$\square$

 \smallskip

 Напомним, что категория $c$-DESC, рассматриваемая вместе с функтором
$\mathrm{sig}$, называется подвижной (transportable)~[11, определение~5.1], если для
любых $c$-DESC-объ\-ек\-та~$A$ и $\mathrm{SIG}$-изо\-мор\-физ\-ма~$i$
 с~об\-ластью $\mathrm{sig}(A)$ существует $c$-DESC-изо\-мор\-физм~$j$
 с~об\-ластью~$A$, удовлетворяющий условию $\mathrm{sig}(j) \hm= i$. Например,
категория \textbf{Pos} с~<<забывающим>> функтором $\vert \mbox{--}\vert$:
$\mathbf{Pos}\hm\to\mathbf{Set}$ очевидным образом является подвижной,
поэтому ввиду следствия~1.4 технология событийного моделирования обладает
наибольшим АО-рас\-ши\-ре\-ни\-ем, которое строится следующим образом.

Обозначим через HAOSM полную подкатегорию в категории всех
помеченных сценариев, состоящую из всех объектов, которые допускают
частичное упорядочение своих множеств меток, пре\-вра\-ща\-ющее разметку в
монотонное отображение. Вложение категории \textbf{Pos} в~HAOSM,
задающее дискретную разметку сценария, является
 АО-рас\-ши\-ре\-ни\-ем: правым сопряженным к нему служит функтор,
<<за\-бы\-ва\-ющий>> разметку, а~левым сопряженным~--- функтор, переводящий
произвольный HAOSM-объ\-ект $\langle X, \leq, l:\ X \hm\to L\rangle$
в~частично упорядоченное мно\-же\-ст\-во $\langle L, \preceq\rangle$, где
$\preceq$~--- пересечение всех частичных порядков на~$L$,
пре\-вра\-ща\-ющих~$l$~в~монотонное отоб\-ра\-же\-ние. Ясно, что категория HAOSM
содержит все помеченные сценарии, разметка которых может быть
восстановлена до трансформации модулей (т.\,е.\ непомеченных сценариев), как
описано в~конце разд.~4. Есть в~HAOSM и~помеченные сценарии, не
облада\-ющие этим свойством, например следующий трехэлементный сценарий
с~двумя несравнимыми элементами, размеченный двумя метками: $\bullet
\leftarrow\bullet\rightarrow\circ$. В~свою очередь, линейно упорядоченный
сценарий $\bullet\to\circ\to\bullet$ (<<интерливинг>>) не содержится
в~HAOSM, так что технология событийного моделирования не обладает
полным АО-рас\-ши\-ре\-нием.

\section{Заключение}

 Эффект от применения теории категорий к~фор\-ма\-лизации процессов
создания программных систем обусловлен возможностью явно выразить
основные положения общей теории систем~[6]. \mbox{Поэтому}
 тео\-ре\-ти\-ко-ка\-те\-гор\-ное аксиоматическое описание концепции АОП
как расширения модульного подхода к интеграции систем формально выражает
его суть с общесистемной точки зрения, которая согласно~[20] представляет
собой совокупность принципов обеспечения абстрагируемости, модульности,
удобства сборки систем в условиях рассеивания задач. В~работе удалось строго
сформулировать и доказать путем вывода из аксиом ряд утверждений,
характеризующих ключевые возможности и ограничения АОП как такового,
поскольку его описание не зависит от частных парадигм программирования.

 В результате появилась возможность рационально применять
 ас\-пект\-но-ори\-ен\-ти\-ро\-ван\-ный подход при проектировании сложных
гетерогенных программных систем, эффективно комбинируя его
с~традиционным объектно-ориентированным. Например, для систем
технологического управления был формально смоделирован и затем реализован
программный модуль ведения единого журнала событий, с которым связаны (в
смысле АОП) многочисленные аспекты обработки событий: модули\linebreak изменения
паспорта объекта автоматизации, рас\-чета разнообразных показателей, проверки
соответствия фактического хода автоматизируемых процес\-сов плановому,
оповещения участников процес\-сов и~т.\,д.~[21]. Такое динамическое
связывание аспектов позволяет тонко настраивать порядок выполнения
рассеянных задач непосредственно в~функционирующей системе, без
дублирования их программного кода и вообще без участия программистов.

{\small\frenchspacing
 {%\baselineskip=10.8pt
 \addcontentsline{toc}{section}{References}
 \begin{thebibliography}{99}
\bibitem{1-kov}
\Au{Репин В.\,В., Елиферов В.\,Г.} Процессный подход к~управ\-ле\-нию. Моделирование
биз\-нес-про\-цес\-сов.~--- М.: Манн, Иванов и Фербер, 2013. 544~с.
\bibitem{2-kov}
\Au{Kiczales G., Lamping J., Mendhekar~A., Maeda~C., Lopes~C.\,V., Loingtier~J.-M., Irwin~J.}
Aspect-oriented programming~// 11th Conference (European) on Object-Oriented
Programming Proceedings~/ Eds. M.~Aksit, S.~Matsuoka.~---
Lecture notes in computer science ser.~---
Springer, 1997. Vol.~1241. P.~220--242.
\bibitem{3-kov}
\Au{Steimann F.} The paradoxical success of aspect-oriented programming~// Conference
(International) OOPSLA'06 Proceedings.~--- Portland, 2006. P.~481--497.
\bibitem{4-kov}
\Au{Colyer A., Clement A., Harley~G., Webster~M.} Eclipse AspectJ: Aspect-oriented
programming with AspectJ and the Eclipse AspectJ development tools.~---
Addison-Wesley, 2004. 504~p.
\bibitem{5-kov}
\Au{Rashid A., Chitchyan R.} Aspect-oriented requirements engineering: A roadmap~// 13th
Workshop (International) on Early Aspects Proceedings.~--- Leipzig, 2008. P.~35--41.
\bibitem{6-kov}
\Au{Goguen J.} Categorical foundations for General Systems Theory~// Advances in cybernetics
and systems research.~--- London: Transcripta Books, 1973. P.~121--130.
\bibitem{7-kov}
\Au{Fiadeiro J.\,L., Lopes~A., Wermelinger~M.} A~mathematical semantics for architectural
connectors~// Generic programming~--- advanced lectures~/
Eds.\ R.\,C.~Backhouse, J.~Gibbons.~--- Lecture notes in computer science ser.~---
Springer, 2003. Vol.~2793. P.~190--234.
\bibitem{8-kov}
\Au{Douence R., Fradet~P., S$\ddot{\mbox{u}}$dholt~M}. Trace-based aspects~// Aspect-oriented
software development.~--- Reading: Addison Wesley, 2004. P.~201--218.
\bibitem{9-kov}
\Au{Jagadeesan R., Pitcher~C., Riely~J.} Open bisimulation for aspects~// Conference
(International) AOSD'07 Proceedings.~--- Vancouver, Canada, 2007. P.~107--120.
\bibitem{10-kov}
\Au{Маклейн С.} Категории для работающего математика~/ Пер. с~англ.~---
М.: Физматлит, 2004. 352~с.
(\Au{Mac Lane~S.} {Categories for the working mathematician}.~---
Berlin\,--\,Heidelberg\,--\,New York: Springer, 1978. 317~p.)
\bibitem{11-kov}
\Au{Ad$\acute{\mbox{a}}$mek J., Herrlich~H., Strecker~G.} Abstract and concrete categories.~---
New York:: Wiley and Sons, 1990. 482~p.
\bibitem{12-kov}
\Au{Ковалёв С.\,П.} Формальный подход к~ас\-пект\-но-ори\-ен\-ти\-ро\-ван\-но\-му
моделированию сценариев~// Сиб. журн. индустр. математики, 2010. Т.~13. №\,3. С.~30--42.
\bibitem{13-kov}
\Au{Gotel O., Finkelstein A.} An analysis of the requirements traceability problem~// 1st
Conference (International) on Requirements Engineering Proceedings.~--- Colorado Springs, 1994.
P.~94--101.
\bibitem{14-kov}
\Au{Egyed A., Gr$\ddot{\mbox{u}}$nbacher~P., Heindl~M., Biffl~S.} Value-based requirements
traceability: Lessons learned~// Design requirements engineering:
A~ten-year perspective~/ Eds.
K.~Lyytinen, P.~Loucopoulos, J.~Mylopoulos, B.~Robinson.~---
Lecture notes in business information processing ser.~---
Springer, 2009. Vol.~14.
P.~240--257.
\bibitem{15-kov}
\Au{Aizenbud-Reshef N., Nolan~B., Rubin~J., Shaham-Gafni~Y.} Model traceability~// IBM
Syst.~J., 2006. Vol.~45. No.\,3. P.~515--526.
\bibitem{16-kov}
\Au{Goguen J.} A~categorical manifesto~// Math. Struct. Comp. Sci., 1991. Vol.~1. No.\,1.
P.~49--67.
\bibitem{17-kov}
\Au{Ковалёв С.\,П.} Семантика аспектно-ори\-ен\-ти\-ро\-ван\-но\-го моделирования данных
и~процессов~// Информатика и её применения, 2013. Т.~7. Вып.~3. С.~70--80.
\bibitem{18-kov}
\Au{Pratt V.\,R.} Modeling concurrency with partial orders~// Int. J.~Parallel Prog., 1986.
Vol.~15. No.\,1. P.~33--71.
\bibitem{19-kov}
\Au{Sutton S.\,M., Rouvellou I.} Concern modeling for aspect-oriented software development~//
Aspect-oriented software development.~--- Reading: Addison Wesley, 2004. P.~479--505.
\bibitem{20-kov}
\Au{Rashid A., Moreira A.} Domain models are not aspect free~//
9th Conference (International) on Model Driven Engineering Languages and
Systems Proceedings~/ Eds.\ O.~Nierstrasz, J.~Whittle, D.~Harel, G.~Reggio.~---
Lecture notes in computer science ser.~--- Springer,
2006. Vol.~4199. P.~155--169.
\bibitem{21-kov}
\Au{Андрюшкевич С.\,К., Ковалёв С.\,П.} Динамическое связывание аспектов
в~крупномасштабных системах технологического управления~// Вычисл. технологии, 2011.
Т.~16. №\,6. С.~3--12.

 \end{thebibliography}

 }
 }

\end{multicols}

\vspace*{-3pt}

\hfill{\small\textit{Поступила в редакцию 25.08.14}}

%\newpage

\vspace*{12pt}

\hrule

\vspace*{2pt}

\hrule

%\vspace*{12pt}

\def\tit{FORMAL AXIOMATIC APPROACH TO ASPECT-ORIENTED EXTENSION
OF~PROGRAMMING TECHNOLOGIES}

\def\titkol{Formal axiomatic approach to aspect-oriented extension
of~programming technologies}

\def\aut{S.\,P.~Kovalyov}

\def\autkol{S.\,P.~Kovalyov}

\titel{\tit}{\aut}{\autkol}{\titkol}

\vspace*{-9pt}

 \noindent
Institute of Control Problem, Russian Academy of Sciences, 65 Profsoyuznaya Str., Moscow
117997, Russian Federation


\def\leftfootline{\small{\textbf{\thepage}
\hfill INFORMATIKA I EE PRIMENENIYA~--- INFORMATICS AND
APPLICATIONS\ \ \ 2015\ \ \ volume~9\ \ \ issue\ 1}
}%
 \def\rightfootline{\small{INFORMATIKA I EE PRIMENENIYA~---
INFORMATICS AND APPLICATIONS\ \ \ 2015\ \ \ volume~9\ \ \ issue\ 1
\hfill \textbf{\thepage}}}

\vspace*{3pt}

\Abste{The procedure of extending modular software systems design
technologies by aspect-oriented techniques is considered. The extension
is described as enrichment of formal module models by labeling their
interfaces by concerns they handle which comprise aspect structure.
A~novel approach to separation of concerns based on the natural
modularizing aspect structure is proposed. Partial modularization
of the aspect structure is proposed to generalize this approach.
In order to formalize these constructs at the general systems level
independently of particular programming paradigms, the category theory
is employed. Software engineering technologies are represented as
categories with formal models of programs as objects and technological
operations as morphisms. The aspect-oriented extension of the technology
is axiomatically described as a functor between such categories that has
appropriate right and left adjoints. The event-based approach to system
modeling is employed as an illustrative case of the aspect-oriented extension.}

\KWE{aspect-oriented programming; traceability; category theory;
architecture school; separation of concerns}

\DOI{10.14357/19922264150105}

\Ack
\noindent
The research was financially supported by the Russian Foundation for
Humanities (grant 13-03-00384).



%\vspace*{3pt}

 \begin{multicols}{2}

\renewcommand{\bibname}{\protect\rmfamily References}
%\renewcommand{\bibname}{\large\protect\rm References}



{\small\frenchspacing
 {%\baselineskip=10.8pt
 \addcontentsline{toc}{section}{References}
 \begin{thebibliography}{99}
\bibitem{1-kov-1}
\Aue{Repin, V.\,V., and V.\,G.~Eliferov}. 2013. \textit{Protsessnyy podkhod k~upravleniyu.
Modelirovanie biz\-nes-pro\-tses\-sov} [Process approach to control. Business process
modeling].
Moscow: Mann, Ivanov and Ferber. 544~p.
\bibitem{2-kov-1}
\Aue{Kiczales, G., J.~Lamping, A.~Mendhekar, C.~Maeda, C.\,V.~Lopes, J.-M.~Loingtier, and
J.~Irwin}. 1997. Aspect-oriented programming.
\textit{11th Conference (European) on Object-Oriented
Programming Proceedings}. Eds. M.~Aksit and S.~Matsuoka.
Lecture notes in computer science ser. Springer. 1241:220--242.
\bibitem{3-kov-1}
\Aue{Steimann, F.} 2006. The paradoxical success of aspect-oriented programming.
\textit{Conference (International) \mbox{OOPSLA}'06 Proceedings}. Portland. 481--497.
\bibitem{4-kov-1}
\Aue{Colyer, A., A.~Clement, G.~Harley, and M.~Webster}. 2004. \textit{Eclipse AspectJ:
Aspect-oriented programming with AspectJ and the Eclipse AspectJ development tools}.
Addison-Wesley. 504~p.
\bibitem{5-kov-1}
\Aue{Rashid, A., and R.~Chitchyan}. 2008. Aspect-oriented requirements engineering: A~roadmap.
\textit{13th Workshop (International) on Early Aspects Proceedings}. Leipzig. 35--41.
\bibitem{6-kov-1}
\Aue{Goguen, J.} 1973. Categorical foundations for General Systems Theory. \textit{Advances in
cybernetics and systems research}. London: Transcripta Books. 121--130.
\bibitem{7-kov-1}
\Aue{Fiadeiro, J.\,L., A.~Lopes, and M.~Wermelinger}. 2003. A~mathematical semantics for
architectural connectors. \textit{Generic programming~--- advanced lectures}.
Eds.\ R.\,C.~Backhouse and J.~Gibbons. Lecture notes in computer science ser.
Springer. 2793:190--234.
\bibitem{8-kov-1}
\Aue{Douence, R., P.~Fradet, and M.~S$\ddot{\mbox{u}}$dholt}. 2004. Trace-based aspects.
\textit{Aspect-oriented software development}. Reading: Addison Wesley. 201--218.
\bibitem{9-kov-1}
\Aue{Jagadeesan, R., C.~Pitcher, and J.~Riely}. 2007. Open bisimulation for aspects.
\textit{Conference (International) AOSD'07 Proceedings}. Vancouver, Canada. 107--120.
\bibitem{10-kov-1}
\Aue{Mac Lane, S.} 1978. \textit{Categories for the working mathematician}.
Berlin\,--\,Heidelberg\,--\,New York: Springer. 317~p.
\bibitem{11-kov-1}
\Aue{Ad$\acute{\mbox{a}}$mek, J., H.~Herrlich, and G.~Strecker}. 1990. \textit{Abstract and
concrete categories}. New York: Wiley and Sons. 482~p.
\bibitem{12-kov-1}
\Aue{Kovalyov, S.\,P.} 2010. Formal'nyy podkhod k aspektno-orientirovannomu
modelirovaniyu stsenariev [Formal approach to aspect-oriented scenario
modeling]. \textit{Sibirskiy Zhurnal
Industrial'noy Matematiki} [J.~Appl. Industrial Math.] 13(3):30--42.
\bibitem{13-kov-1}
\Aue{Gotel, O., and A.~Finkelstein}. 1994. An analysis of the requirements traceability problem.
\textit{1st Conference (International) on Requirements Engineering Proceedings}. Colorado
Springs. 94--101.
\bibitem{14-kov-1}
\Aue{Egyed, A., P.~Gr$\ddot{\mbox{u}}$nbacher, M.~Heindl, and S.~Biffl}. 2009. Value-based
requirements traceability: Lessons learned.
\textit{Design requirements engineering:
A~ten-year perspective}. Eds.\ Lyytinen,~K.,  P.~Loucopoulos, J.~Mylopoulos,
and B.~Robinson. Lecture notes in business information processing ser.
Springer 14:240--257.
\bibitem{15-kov-1}
\Aue{Aizenbud-Reshef, N., B.~Nolan, J.~Rubin, and Y.~Shaham-Gafni}. 2006. Model traceability.
\textit{IBM Syst.~J.} 45(3):515--526.
\bibitem{16-kov-1}
\Aue{Goguen, J.} 1991. A categorical manifesto. \textit{Math. Struct. Comp. Sci.}
1(1):49--67.
\bibitem{17-kov-1}
\Aue{Kovalyov, S.\,P.} 2013. Semantika aspektno-ori\-en\-ti\-ro\-van\-nogo
modelirovaniya dannykh i~protsessov [Semantics of aspect-oriented
modeling of data and processes]. \textit{Informatika i~ee
Primeneniya}~--- \textit{Inform. Appl.} 7(3):70--80.
\bibitem{18-kov-1}
\Aue{Pratt, V.\,R.} 1986. Modeling concurrency with partial orders. \textit{Int. J.~Parallel
Prog.} 15(1):33--71.
\bibitem{19-kov-1}
\Aue{Sutton, S.\,M., and I.~Rouvellou}. 2004. Concern modeling for aspect-oriented software
development. \textit{Aspect-oriented software development}. Reading: Addison Wesley. 479--505.
\bibitem{20-kov-1}
\Aue{Rashid, A., and A.~Moreira}. 2006. Domain models are not aspect free.
\textit{9th Conference (International) on Model Driven Engineering Languages and
Systems Proceedings}.  Eds.\ Nierstrasz,~O., J.~Whittle, D.~Harel, and G.~Reggio.
Lecture notes in computer science ser. Springer. 4199:155--169.
\bibitem{21-kov-1}
\Aue{Andryushkevich, S.\,K., and S.\,P.~Kovalyov}. 2011. Dinamicheskoe svyazyvanie aspektov v
krupnomasshtabnykh sistemakh tekhnologicheskogo
upravleniya [Dynamic aspect weaving in large-scale manufacturing control systems].
\textit{Vychislitel'nye Tekhnologii} [J.~Comput.
Technol.]. 16(6):3--12.
\end{thebibliography}

 }
 }

\end{multicols}

\vspace*{-3pt}

\hfill{\small\textit{Received August 25, 2014}}

%\vspace*{-18pt}


 \Contrl

 \noindent
 \textbf{Kovalyov Sergey P.} (b.\ 1972)~--- Doctor of Science in physics and mathematics,
senior scientist, Institute of Control Problem, Russian Academy of Sciences, 65 Profsoyuznaya
Str., Moscow 117997, Russian Federation; kovalyov@nm.ru
\label{end\stat}

\renewcommand{\bibname}{\protect\rm Литература}