\def\stat{spivak}

\def\tit{ОЦЕНКА ПОГРЕШНОСТИ И~ЗНАЧИМОСТИ ИЗМЕРЕНИЙ
ДЛЯ~ЛИНЕЙНЫХ МОДЕЛЕЙ$^*$}

\def\titkol{Оценка погрешности и~значимости измерений для
линейных моделей}

\def\aut{С.\,И.~Спивак$^1$, О.\,Г.~Кантор$^2$, Д.\,С.~Юнусова$^3$,
С.\,И.~Кузнецов$^4$, С.\,В.~Колесов$^5$}

\def\autkol{С.\,И.~Спивак, О.\,Г.~Кантор, Д.\,С.~Юнусова и др.}


\titel{\tit}{\aut}{\autkol}{\titkol}

{\renewcommand{\thefootnote}{\fnsymbol{footnote}} \footnotetext[1]
{Работа выполнена при поддержке РФФИ (проект 13-01-00749).}}


\renewcommand{\thefootnote}{\arabic{footnote}}
\footnotetext[1]{Башкирский государственный университет, semen.spivak@mail.ru}
\footnotetext[2]{Институт социально-экономических исследований Уфимского научного центра Российской
академии наук, o\_kantor@mail.ru}
\footnotetext[3]{Башкирский государственный университет, kazakova\_d\_s@mail.ru}
\footnotetext[4]{Институт органической химии Уфимского научного центра Российской академии наук,
chemorg@anrb.ru}
\footnotetext[5]{Институт органической химии Уфимского научного центра Российской академии наук,
kolesovservic@rambler.ru}

\Abst{Решение задач восстановления линейных зависимостей в тех случаях,
когда точное решение, полученное стандартными методами, не удовлетворяет
объективным требованиям, обусловливает разработку специальных подходов
для их численной реализации. В~статье приводится описание метода
получения приближенных значений параметров линейных зависимостей по
экспериментальным данным, в~основе которого лежит использование
методологии линейного программирования и теории двойственности.
Разработанный метод позволяет не только получать приближенные решения,
обеспечивающие выполнение всех предъявляемых требований к~самой
восстанавливаемой зависимости и~ее параметрам, но и~проводить оценку
погрешности измерений и~их значимости. А~это важно для совершенствования
процедуры построения функциональных зависимостей на стадии планирования
экспериментов в~части уточнения экспериментальных данных или их
исключения из рассмотрения как не удовлетворяющих критериям
достоверности. Приведены результаты апробации предложенного метода для
задач, связанных с исследованиями химических
и~со\-ци\-аль\-но-эко\-но\-ми\-че\-ских систем.}

\KW{задачи восстановления линейных зависимостей; погрешность измерений;
значимость измерений; двойственные оценки}

\DOI{10.14357/19922264150108}

%\vspace*{-6pt}


\vskip 14pt plus 9pt minus 6pt

\thispagestyle{headings}

\begin{multicols}{2}

\label{st\stat}


\section{Введение}

  Рассматривается задача определения пара\-мет\-ров линейных математических
моделей по экспериментальным данным, которые не могут быть рассчитаны
стандартными методами в~силу некоторых\linebreak объективных причин (например,
в~виду ограни\-ченного количества имеющихся измерений или от\-сутствия
информации об их статистических характеристиках).
  Предметом рассмотрения в~данной \mbox{работе} являются системы линейных
алгебраических уравнений вида
  \begin{equation}
  AX=B\,.
  \label{e1-spi}
  \end{equation}

  В системе~(1) $A=(a_{ij})$ и~$B\hm=(b_i)$~--- экспериментальные данные
($i\hm= 1,\ldots, m$, $j\hm= 1,\ldots, n$) , а~$X\hm= (x_1, x_2, \ldots,
x_n)^{\mathrm{T}}$~--- искомые параметры мо\-дели.

  К решению таких задач сводятся многие задачи восстановления линейных
зависимостей по экспериментальным данным, возникающие при
исследованиях в~различных областях научной и~практической деятельности.
Примеры задач подобного рода широко представлены в~рамках таких хорошо
изученных разделов, как определение регрессионных зависимостей и~моделирование
временн$\acute{\mbox{ы}}$х рядов.
%
Основу подавляющего большинства
подходов к~решению задачи~(1) составляют методы математической
статистики, представляющие на сегодняшний день группу подробно изученных
и~хорошо\linebreak зарекомендовавших себя на практике инструментов определения
параметров линейных зависимостей~[1--3].

Существенным
препятствием к~применению статистических методов является {обязательное} наличие
достаточно большого количества экспериментальных данных, что не всегда
достижимо на практике.

При равенстве числа наблюдений и~числа
оцениваемых параметров могут быть использованы и~классические методы
решения квадратных систем линейных уравнений (метод Гаусса, метод
обратной матрицы и~пр.). Однако говорить о~статистической значимости
искомых параметров в~этом случае нельзя, и~основное назначение такого
подхода заключается в~установлении точного вида функциональной связи
между исследуемыми величинами по результатам конкретных наблюдений.

  Применение большинства методов вос\-ста\-нов\-ле\-ния линейных зависимостей
по экспериментальным данным (в~том числе и~перечисленных выше)
сопряжено с~еще одной группой проблем: найденные значения параметров
могут не удовлетворять\linebreak некоторым условиям, вытекающим из их физи\-ческого
смысла (например, применение статисти\-ческих методов или метод решения
линейных \mbox{алгебраических} уравнений не обеспечивают не\-от\-ри\-ца\-тель\-ность
искомых величин).

  В этой связи актуальным является определение приближенного решения
системы~(1) и~оценка величины погрешности измерений, под которой будем
понимать расхождение значений расчетных и~экспериментальных величин
не в~каждом отдельном наблюдении, а~в~целом по всей совокупности наблюдений.
В~свою очередь, это обусловливает необходимость изучения способов
формализации таких задач.

  \section{Описание подхода к~определению погрешности
измерений}

  Без ограничения общности рассуждений будем предполагать, что на
параметры модели~$X$ наложены условия неотрицательности:
  \begin{equation}
  X\geq 0\,.
  \label{e2-spi}
  \end{equation}

  Достаточно часто на значения параметров накладываются ограничения,
выражающие их принадлежность ка\-ко\-му-ли\-бо множеству значений, поэтому
будем считать, что параметры модели также удовлетворяют системе
ограничений
  \begin{equation}
  CX\geq D\,,
  \label{e3-spi}
  \end{equation}
  где $C$~--- это матрица, состоящая из коэффициентов при параметрах
модели в~системе ограничений; $D$~--- концы промежутков значений,
которым принадлежат параметры модели.

  Тогда задача определения приближенного решения системы~(1) с~учетом
ограничений~(2) и~(3) может быть сведена к~задаче линейного
программирования
  \begin{equation}
  \left.
  \begin{array}{c}
  \varepsilon\to \min\,;\\[6pt]
  \vert AX-B\vert \leq \varepsilon\,; \\[6pt]
  CX\geq D\,;\enskip
  X\geq 0\,.
  \end{array}
  \right\}
  \label{e4-spi}
  \end{equation}
  Здесь $A=(a_{ij})$ и~$B\hm=(b_i)$~--- экспериментальные данные ($i\hm=
1,\ldots, m$, $j\hm=1,\ldots, n$); $X\hm= (x_1, x_2, \ldots
x_n)^{\mathrm{T}}$~--- искомые параметры модели; $C\hm= (c_{lj})$~--- это
матрица, состоящая из коэффициентов при параметрах модели в~системе
ограничений ($l\hm= 1,\ldots, k$, $j\hm= 1,\ldots, n$), $D\hm= (d_l)$~---
  век\-тор-стол\-бец, элементы которого~--- концы промежутков значений
параметров модели ($l\hm= 1,\ldots, k$); $\varepsilon$~--- параметр,
характеризующий величину погрешности измерений.

  Отметим, что вместо ограничения $\vert AX-B\vert \hm\leq \varepsilon$ может
использоваться условие вида $\vert AX\hm-B\vert\hm= {E}$, в~котором
элементы матрицы ${E}\hm= (\varepsilon_i)$ ($i\hm= 1,\ldots , m$)
представляют собой параметры, характеризующие величину ошибки в~описании $i$-го эксперимента. Основная сложность такого перехода от
неравенств к~равенствам сопряжена с~ростом числа неизвестных: вместо одной
неизвестной величины~$\varepsilon$ определению подлежат $m$
величин~$\varepsilon_i$. При этом, если ввести обозначение $\varepsilon\hm=
\max\limits_i \varepsilon_i$, легко убедиться, что задача в~постановке~(4) более
предпочтительна не только по причине меньшего числа неизвестных, но
и~в~силу того, что разность между~$AX$ и~$B$, т.\,е.\ левые части
ограничений $\vert AX\hm-B\vert \hm\leq \varepsilon$, позволяют определить
величины~$\varepsilon_i$  ($i\hm= 1,\ldots , m$).

  В результате решения задачи~(4) должны быть определены параметры
модели, удовлетворяющие требуемым ограничениям, и~величина погреш\-ности
измерений.

  Следует отметить, что матрицы~$A$ и~$B$ формируются по результатам
наблюдений, а~потому не могут рассматриваться как абсолютно точные, так
как результаты их измерений неминуемо сопряжены с~некоторыми ошибками.
Такие ошибки приводят к~отклонениям измеряемых значений величин~$A$
и~$B$ от их истинных значений. Причем очевидно, что исследователю эти
отклонения заранее не известны, но, предполагая их наличие, целесообразно
ставить задачу определения параметров модели, удовлетворяющих
ограничениям~(2) и~(3), при условии внесения изменений
в~экспериментальные данные~$A$ и~$B$. Будем предполагать незначительные
отклонения измеряемых значений величин~$A$ и~$B$ от их истинных
значений, что соответствует ситуации, при которой грубые ошибки при
получении экспериментальной информации исключаются. В~данных условиях
целесообразно рассмотреть способы постановки задач определения параметров
модели, обеспечивающих минимально возможные отклонения от
экспериментальных данных, для различных случаев вариации матриц~$A$
и~$B$.

  В случае предполагаемых ошибок в~матрице~$B$, различных для каждого
отдельного измерения $i\hm= 1,\ldots ,m$, задача определения параметров
модели, обеспечивающих минимальное отклонение от экспериментальных
величин $(b_i)$ может быть сведена к~задаче линейного программирования:

\noindent
  \begin{equation}
  \left.
  \begin{array}{c}
  \varepsilon\to \min\,;\\[6pt]
  AX=\Delta B\,;\\[6pt]
  CX\geq D\,;\enskip
  X\geq 0\,;\\[6pt]
  \left\vert \delta_i-1\right\vert \leq \varepsilon\ \ \forall\ i=1,\ldots ,m\,.
  \end{array}
  \right\}
  \label{e5-spi}
  \end{equation}
  Здесь $\varepsilon$~--- это величина погрешности измерений, а~$\Delta$~---
диагональная матрица вида
  $$
  \Delta = \begin{pmatrix}
  \delta_1 & \cdots & 0\\
  \cdots & \cdots &\cdots\\
  0 &\cdots & \delta_m
  \end{pmatrix}\,,
  $$
где $\delta_i$ ($i\hm=1,\ldots ,m$)~--- это неизвестные величины,
отождествляемые с~параметрами, характеризующими ошибки в~измерениях
элементов матрицы~$B$. Величины~$\varepsilon$ и~$\delta_i$ связаны
очевидным соотношением $\varepsilon\hm= \max\limits_i \left\vert \delta_i\hm-
1\right\vert$.

  Решение задачи~(5) обеспечивает определение неотрицательных
параметров~$X$, при которых наибольшая погрешность измерений элементов
матрицы~$B$ минимальна, поскольку каждый элемент этой матрицы
умножается на число, отличающееся от единицы не больше чем
на~$\varepsilon$ в~силу условия $\left\vert \delta_i\hm- 1\right\vert \hm \leq
\varepsilon$ $\forall\ i\hm= 1,\ldots , m$. При такой постановке разницы между
левыми и~правыми частями системы~(1) соответственно равны $(1\hm-
\delta_i)b_i$. Следовательно, справедливы соотношения $\vert AX\hm- B\vert
\hm\leq \varepsilon$, а~это означает, что для системы~(1) максимальная разница
правых и~левых частей по модулю не превысит~$\varepsilon$.

  В случае предполагаемых ошибок в~каждом элементе матрицы~$A$, что
отражает случайный характер ошибок в~каждом измерении экспериментальных
данных $A\hm= (a_{ij})$ и~формально соответствует умножению каждого
элемента матрицы на некоторое число, задача~(1) может быть формализована
в~сле\-ду\-ющем виде:
  \begin{equation}
  \left.
  \begin{array}{c}
  \varepsilon\to \min\,;\\[6pt]
  A^\prime X=B\,;\\[6pt]
  CX\geq D\,;\enskip
  X\geq 0\,;\\[6pt]
  \left\vert \gamma_{ij}-1\right\vert \leq \varepsilon\ \ \ \forall\ i=1,\ldots, m\,,\
\forall\ j=1,\ldots ,n\,.
  \end{array}
  \right\}
  \label{e6-spi}
  \end{equation}
  Здесь матрица $A^\prime$ имеет вид:
  $$
  A^\prime = \begin{pmatrix}
  \gamma_{11} a_{11} &\cdots & \gamma_{1n}a_{1n}\\
  \cdots &\cdots &\cdots\\
  \gamma_{m1}a_{m1} &\cdots & \gamma_{mn} a_{mn}
  \end{pmatrix}\,.
  $$

  Задача~(6) является нелинейной, поскольку в~выражении $A^\prime X\hm=
B$ элементы вектора~$X$~--- неизвестные величины, а~$A^\prime$~---
матрица, которая зависит от $m\times n$ неизвестных величин. Данное
обстоятельство создает существенные проблемы для численной реализации
модели~(6) и~не является предметом рассмотрения данной работы. Вместе с~тем
относительно элементов матрицы~$A$ может быть известна информация,
которая позволит упростить процесс получения решения. Примером тому могут
служить ситуации, при которых либо во всех наблюдениях применительно
к~каж\-дой введенной в~рассмотрение величине совершаются однотипные ошибки,
либо одни и~те же погрешности допускаются в~рамках каж\-до\-го наблюдения и~по отношению ко всем рассматриваемым величинам. Первая ситуация отражает
наличие индивидуальных систематических ошибок при измерении каж\-дой из
наблюдаемых величин, а вторая~--- индивидуальные ошибки каж\-до\-го
отдельного наблюдения. Первая ситуация может возникать, например, ввиду
особенностей приборов, используемых для фиксации значений наблюдаемых
величин, а вторая~--- в~случае зависимости от условий проведения
эксперимента (температурных, временн$\acute{\mbox{ы}}$х и~пр.).

  Первая из перечисленных выше ситуаций отражается в~пропорциональных
изменениях всех элементов каждого из столбцов, а вторая~--- строк. Формально
первая ситуация соответствует умножению матрицы~$A$ на диагональную
матрицу~$\Gamma$ справа ($A^\prime \hm= A\Gamma$, $\Gamma\hm=
(\gamma_{jj})$, $j\hm= 1, \ldots, n$), а вторая~--- слева ($A^\prime \hm= \Gamma
A$, $\Gamma\hm= (\gamma_{ii})$, $i\hm= 1, \ldots ,m$).

  Предполагаемые ошибки в~матрице~$A$, выражающиеся в~пропорциональных изменениях элементов столбцов, никак не отражаются на
погрешности измерений~$\varepsilon$, а~влияют только на параметры~$X$.
Действительно, рассмотрим ограничение задачи~(6):
  $$
  A^\prime X= B\,.
  $$
  Матрица $A^\prime \hm= A\Gamma$; следовательно,
  $$
  A\Gamma X =B\,.
  $$

  Если обозначить $\Gamma X$ через~$X^\prime$, получим систему,
идентичную исходной:
  $$
  AX^\prime =B\,.
  $$

  В случае предполагаемых ошибок в~матрице~$A$, выражающихся в~пропорциональных изменениях строк исходной матрицы~$A$, задача
определения неизвестных параметров эквивалентна задаче~(\ref{e5-spi}).
Покажем это. Соотношения для определения параметров~$X$ на основании
имеющейся информации имеют вид:
  $$
  A^\prime X=B\,.
  $$
  Так как матрица  $A^\prime = \Gamma A$, то
  $$
  \Gamma AX =B\,.
  $$
  Умножим обе части уравнения на $\Gamma^{-1}$:
  $$
  \Gamma^{-1} \Gamma AX = \Gamma^{-1} B\,.
  $$
  С~учетом $\Gamma^{-1}\Gamma ={E}$ получим
  $$
  AX=\Gamma^{-1} B\,.
  $$

  Если $\Gamma^{-1}$ обозначить через~$\Delta$, то последнее соотношение
эквивалентно задаче~(\ref{e5-spi}). Отметим, что обратная
матрица~$\Gamma^{-1}$ всегда существует, поскольку матрица~$\Gamma$~---
это диагональная матрица с~отличными от нуля элементами на главной
диагонали, поэтому и~ее определитель также отличен от нуля. А~такая матрица,
как известно, всегда имеет обратную.

  Основным преимуществом предложенного подхода к~определению
параметров линейных математических моделей~(1) является возможность
учета всех дополнительных требований, предъявляемых к~параметрам~$X$,
уже на стадии формализации модели, что позволяет исключить возможность
получения заведомо неприемлемых результатов.

  \section{Методика определения значимости измерений}

  Важное практическое значение имеет оценка влияния погрешности
экспериментальных данных модели на погрешность
измерений~$\varepsilon$~[4,~5], что позволяет осуществлять анализ
информационной ценности измерений и,~как следствие, выявлять те, которые
следует рассматривать как наиболее недостоверные или значимые и~пр.
Результатами такого анализа могут быть, например, выводы о~необходимости,
при наличии соответствующих возможностей, уточнения некоторых
экспериментальных данных или рекомендации об их исключении из
рассмотрения при непосредственном построении функциональных
зависимостей.

  Известно, что при решении задач линейного программирования для этих
целей используется теория двойственности~[6--8]. Согласно третьей теореме
двойственности компоненты оптимального решения двойственной задачи
равны частным производным целевой функции прямой задачи по
соответствующим параметрам, в~качестве которых выступают свободные
члены системы ограничений исходной задачи. В~силу того, что применительно
к задаче восстановления зависимостей такие па\-ра\-мет\-ры являются величинами,
наблюдаемыми в~ходе эксперимента, предполагая их малые изменения,
с~помощью анализа оптимальных значений двойственных переменных можно
оценить значимость каждого отдельного наблюдения.

  Применительно к~исследуемой в~настоящей работе задаче решение
соответствующих двойственных задач позволяет оценить влияние элементов
матрицы экспериментальных данных~$B$ и~матрицы ограничений на
параметры модели~$D$ в~прямых задачах линейного программирования на
величину минимального значения погрешности измерений~$\varepsilon$. Это
предоставляет возможность выявлять те элементы матриц~$B$ и~$D$, которые
вносят наибольший вклад в~значение погрешности измерений~$\varepsilon$
и~количественно его оценить.

  С~этих позиций для рассмотренных выше задач линейного
программирования целесообразно рассмотреть двойственные к~ним.
Двойственная задача для задачи линейного программирования~(4) имеет вид:
  \begin{equation}
  \left.
  \begin{array}{c}
  \left( B, y^1\right) -\left( B,y^2\right) +\left( D,y^3\right) \to \max\,;\\[6pt]
  A^{\mathrm{T}} y^1 -A^{\mathrm{T}} y^2 +C^{\mathrm{T}} y^3\leq 0\,;\\[3pt]
\displaystyle  \sum\limits_{i=1}^m y_i^1 +\sum\limits_{i=1}^m y_i^2 \leq 1\,;\\[12pt]
  y^1\geq 0\,;\ \ y^2\geq 0\,;\ \ y^3\geq 0\,.
  \end{array}
  \right\}
  \label{e7-spi}
  \end{equation}
  Здесь $y^1=\left( y_i^1\right)$,  $y^2=\left( y_i^2\right)$ ($i\hm= 1,\ldots ,m$)
и~$y^3\hm= \left( y_l^3\right)$ ($l\hm= 1,\ldots ,k$)~--- векторы оптимального
решения двойственной задачи.

  Аналогичным образом может быть выписана двойственная задача для задачи
линейного программирования~(\ref{e5-spi}):
  \begin{equation}
  \left.
  \begin{array}{c}
  \displaystyle -\sum\limits_{i=1}^m y_i^2 +\sum\limits_{i=1}^m y_i^3 +\left(
D,y^4\right)\to\max\,;\\[12pt]
  A^{\mathrm{T}} y^1 +C^{\mathrm{T}}y^4 \leq 0\,;\\[4pt]
  -y_i^1 b_i -y_i^2 +y_i^3 \leq 0\ \ \ \forall\ i=1,\ldots m\,;\\[6pt]
  \displaystyle \sum\limits_{i=1}^m y_i^2 +\sum\limits_{i=1}^m y_i^3\leq
1\,;\\[6pt]
  y^2\geq0\,;\  \ y^3\geq0\,;\ \ y^4\geq0\,.
  \end{array}
  \right\}
  \label{e8-spi}
  \end{equation}
  Здесь $y^1\hm= \left( y_i^1\right)$, $y^2\hm= \left( y_i^2\right)$, $y^3\hm=
\left( y_i^3\right)$ ($i\hm= 1,\ldots ,m$) и~$y^4\hm = \left( y_l^4\right)$ ($l\hm=
1,\ldots , k$)~--- векторы оптимального решения двойственной задачи.

  Заметим, что в~задаче~(4) для оценки степени влияния $i$-го соотношения из
системы неравенств $\vert AX\hm- B\vert \hm \leq \varepsilon$ на значение
погрешности измерений~$\varepsilon$ необходимо рассмотреть
соответствующие компоненты векторов $y^1\hm= \left( y_i^1\right)$ и~$y^2\hm=
\left( y_i^2\right)$, являющихся решением задачи~(\ref{e7-spi}), и~выбрать из
них максимальный. Аналогичную процедуру следует провести с~векторами
$y^2\hm= \left( y_i^2\right)$ и~$y^3\hm= \left( y_i^3\right)$, являющимися
решением задачи~(\ref{e8-spi}), для оценки степени влияния сводного члена
  $i$-го соотношения из системы неравенств $\left\vert \delta_i\hm- 1\right\vert
\hm\leq\varepsilon$ на погрешность измерений~$\varepsilon$
  в~задаче~(\ref{e5-spi}).

  \section{Результаты апробации}

  Ниже представлены результаты определения параметров и~оценки
значимости используемых измерений на примере задачи нахождения
распределения мольных долей фрагментов фуллерена с~различным
количеством заместителей в~макроцепях полимеров, рассмотренной в~работах~[9, 10].
Для нахождения этого распределения составляется сис\-те\-ма
уравнений Бу\-ге\-ра--Лам\-бер\-та, пред\-став\-ля\-ющая собой сис\-те\-му линейных
алгебраических уравнений вида~(1), с~равным числом уравнений и~неизвестных. Элементами квадратной матрицы~$A$ являются молярные
экстинции ядер несвязанного фуллерена и~ядер, ковалентно связанных одной,
двумя и~$n$~связями с~заместителями (заместителями, фрагментами
инициатора, макроцепями) соответственно, а~в~роли~$B$~--- значения
оптических плотностей, измеряемых спектрофотометрически в~ульт\-ра\-фио\-ле\-то\-вой/ви\-ди\-мой
области, для растворов фуллеренсодержащих продуктов (полимеров, смесей
специально химически синтезированных индивидуальных замещенных
фуллеренов). Определению подлежали параметры~$X$, представляющие собой
концентрации содержания фрагментов фуллерена в~макроцепях полимеров. На
основании параметров~$X$ могут быть рассчитаны и~мольные доли
концентраций, для чего необходимо разделить значение каждой концентрации
на сумму всех концентраций.

  Матрицы~$A$ и~$B$ формировались на основании экспериментальных
данных:
\begin{align*}
  A&=\begin{pmatrix}
  54\,000\ & \ 30\,800\ & \ 35\,800\ & \ 28\,500\ & \ 30\,900\\
  30\,900\ & \ 22\,800\ & \ 28\,300\ & \ 27\,900\ & \ 28\,800\\
  19\,600\ & \ 21\,000\ & \ 21\,800\ & \ 18\,500\ &\  16\,050\\
  50\,500\ & \ 24\,370\ & \ 17\,630\ & \ 15\,070\ & \ 11\,800\\
  60\,780\ & \ 24\,150\ & \ 15\,350\ & \ 13\,000\ & \ 11\,700
  \end{pmatrix}\,;\\
   B&=
  \begin{pmatrix}
  2{,}453\\ 2{,}001 \\ 1{,}475 \\1{,}435\\ 1{,}408
  \end{pmatrix}\,.
  \end{align*}

  Решение, полученное методом обратной матрицы, следующее:
 \begin{multline*}
  X=\left( 5{,}067\cdot 10^{-1}\ \ 3{,}554\cdot 10^{-5}\ \
  1{,}211\cdot 10^{-5}\right.\\
\left.-3{,}096\cdot 10^{-6}\ \ 3{,}19\cdot 10^{-5}\right)^{\mathrm{T}}\,.
  \end{multline*}

  Такое решение не имеет смысла, так как не все элементы вектора~$X$
неотрицательны. Именно поэтому определялось приближенное решение
посредством сведения исходной проблемы к~задаче линейного
программирования вида~(4):

\noindent
  \begin{equation}
  \left.
  \begin{array}{c} \varepsilon\to \min\,;\\[6pt]
  \vert AX-B\vert \leq \varepsilon\,;\\[6pt]
  X\geq 0\,.
  \end{array}
  \right\}
  \label{e9-spi}
  \end{equation}

  В данной задаче ограничение $CX\hm \geq D$ отсутствует, поскольку нет
дополнительных ограничений на параметры~$X$, кроме ограничения
неотрицательности.

  Решение задачи~(9):
  \begin{multline*}
  X=\left( 9{,}504\cdot 10^{-7}\ \ 3{,}333\cdot 10^{-5}\ \
  1{,}274\cdot 10^{-5}\right.\\
\left.0\  \ 2{,}964\cdot 10^{-5}\right)^{\mathrm{T}}\,,
  \end{multline*}
  параметр~$\varepsilon$, характеризующий величину погреш\-ности измерений,
в этом случае равен 0{,}002786.

  В предположении существования ошибок в~мат\-ри\-це~$B$ рассматривалась
задача линейного программирования вида~(5), которая в~обозначениях
по\-став\-лен\-ной задачи имеет вид:
  \begin{equation}
  \left.
  \begin{array}{c}
  \varepsilon\to \min\,;\\[6pt]
  AX=\Delta B\,;\\[6pt]
  X\geq 0\,;\\[6pt]
  \left\vert \delta_i-1\right\vert \leq \varepsilon\ \  \forall\ i=1,\ldots, 5\,.
  \end{array}
  \right\}
  \label{e10-spi}
  \end{equation}

  Решение задачи (10):
  \begin{multline*}
  X=\left( 1{,}031\cdot 10^{-6}\ \ 3{,}301\cdot 10^{-5}\ \
  1{,}379\cdot 10^{-5}\right.\\
\left. 0 \ \ 2{,}859\cdot 10^{-5}\right)^{\mathrm{T}}\,.
  \end{multline*}

  Параметры, характеризующие ошибки в~экспериментальных данных
(в~элементах матрицы~$B$):
$\delta_1\hm= 0{,}9986$; $\delta_2\hm=0{,}9986$;
$\delta_3\hm=0{,}9986$; $\delta_4\hm= 1{,}0014$;
$\delta_5\hm=0{,}9986$.
  По\-греш\-ность измерений в~этом случае составила $\varepsilon\hm=
0{,}001396$.

  Заметим, что в~такой постановке, когда каждый элемент матрицы~$B$
умножается на параметр, характеризующий ошибку в~этом элементе матрицы,
погрешность измерений~$\varepsilon$ допускает удобное представление
в~процентах (в~данном случае $\varepsilon\hm=0{,}1396\%$).

  Полученные результаты позволили полагать, что отсутствие физического
смысла в~точном решении рассматриваемой задачи может быть связано с~наличием ошибки в~экспериментальных данных. Для оценки влияния
предполагаемой по\-греш\-ности экспериментальных данных в~матрице~$B$ на
по\-греш\-ность измерений~$\varepsilon$ были выписаны двойственные задачи
к~задачам~(9) и~(10). Двойственная задача к~задаче~(9) имеет вид:

\noindent
  \begin{equation}
  \left.
  \begin{array}{c}
  \left( B,y^1\right) - \left( B, y^2\right) \to \max\,;\\[6pt]
  A^{\mathrm{T}} y^1- A^{\mathrm{T}} y^2\leq 0\,;\\[2pt]
  \displaystyle \sum\limits_{i=1}^5 y_i^1 +\sum\limits_{i=1}^5 y_i^2\leq 1\,;\\[10pt]
  y^1\geq0\,; \quad y^2\geq 0\,.
  \end{array}
  \right\}
  \label{e11-spi}
  \end{equation}
  Здесь $y^1\hm=\left( y_i^1\right)$, $y^2\hm=\left( y_i^2\right)$ ($i\hm=1,\ldots
,5$).

  Решение задачи~(11):

  \noindent
  \begin{align*}
  y^1 &= \left( 0{,}044\ \ 0\ \ 0{,}177 \ \ 0\ \ 0{,}284\right)^{\mathrm{T}}\,;\\
  y^2&= \left( 0 \ \ 0{,}098\ \ 0\ \ 0{,}397\ \ 0\right)^{\mathrm{T}}\,.
  \end{align*}

  Компоненты решения двойственной задачи~(11) показывают, что
наибольший вклад в~значение погрешности измерений~$\varepsilon$
в~задаче~(9) вносит четвер-\linebreak тый элемент матрицы~$B$, так как выражение
$\max\limits_i\max \left\{ y_i^1;\,y_i^2\right\}$ достигает наибольшего значения
(0,397) на четвертой компоненте оптимального решения~$y^2$, а~наименьший
(0,044)~--- первый, так как минимальное значение выражение $\min\limits_i\max
\left\{ y_i^1;\,y_i^2\right\}$ принимает на первой компоненте оптимального
решения~$y^1$.

  Двойственная задача к~задаче~(10) имеет вид:

\noindent
  \begin{equation}
  \left.
  \begin{array}{c}
  \displaystyle -\sum\limits_{i=1}^5 y_i^2 +\sum\limits_{i=1}^5 y_i^3\to
\max\,;\\[12pt]
  A^{\mathrm{T}} y^1\leq0\,;\\[1pt]
  -y_i^1 b_i -y_i^2 +y_i^3\leq 0\ \ \forall\ i=1,\ldots, 5\,;\\[6pt]
  \displaystyle  \sum\limits_{i=1}^5 y_i^2 +\sum\limits_{i=1}^5 y_i^3\leq1\,;\\[12pt]
  y^2\geq0\,;\quad y^3\geq0\,.
  \end{array}
  \right\}
  \label{e12-spi}
  \end{equation}
  Здесь $y^1=\left( y_i^1\right)$, $y^2\hm= \left( y_i^1\right)$, $y^3\hm= \left(
y_i^3\right)$ ($i\hm= 1,\ldots, 5$).

  Решение двойственной задачи~(12):

  \noindent
  \begin{align*}
  y^1 &= \left( 0{,}029\ \ -0{,}064\ \ 0{,}115\ \ -0{,}259\ \
0{,}109\right)^{\mathrm{T}}\,;\\
  y^2 &= \left( 0\ \ 0{,}128\ \ 0{,}371\ \ 0\right)^{\mathrm{T}}\,;\\
  y^3 &= \left( 0{,}07\ \ 0\ \ 0{,}169\ \ 0\ \ 0{,}261\right)^{\mathrm{T}}\,.
  \end{align*}

  Решение двойственной задачи~(12) также показывает, что наибольший вклад
в значение погрешности измерений~$\varepsilon$ в~задаче~(10) определяется
погреш\-ностью четвертого элемента матрицы~$B$, а~наименьший~---
погрешностью первого. Из этого следует, что наименее достоверным следует
полагать измерение величины~$B$ в~четвертом эксперименте.

  Аналогичный подход был применен к~определе\-нию степени влияния
погрешностей наблю\-да\-емых величин на погрешность измерения при решении
задачи моделирования численности населения\linebreak Российской Федерации методом
системной динамики~[10--17]. Общий вид исследованной модели системной
динамики в~терминах разностных уравнений следующий:
  \begin{equation}
  \left.
  \begin{array}{rl}
  \Delta N &= a_1 N^{\alpha_1} D^{\beta_1} I^{\gamma_1} -a_2 N^{\alpha_2}
D^{\beta_2} I^{\gamma_2}\,;\\[6pt]
  \Delta D &= a_3 N^{\alpha_3} D^{\beta_3}I^{\gamma_3} -a_4 N^{\alpha_4}
D^{\beta_4} I^{\gamma_4}\,;\\[6pt]
  \Delta I &= a_5 N^{\alpha_5} D^{\beta_5} I^{\gamma_5} -a_6 N^{\alpha_6}
D^{\beta_6} I^{\gamma_6}\,,
  \end{array}
  \right\}
  \label{e13-spi}
  \end{equation}
где $N$~--- численность населения РФ; $D$~--- душевые доходы за год; $I$~---
индекс потребительских цен. Информационную базу настоящего исследования
составили данные официальной статистической отчетности за период с~1998
по~2010~гг.

  \begin{table*}\small
  \begin{center}
  \Caption{Исходные данные для модели (13)}
  \vspace*{2ex}

  \begin{tabular}{|c|c|c|c|}
  \hline
Год&
\tabcolsep=0pt\begin{tabular}{c}Численность\\ населения РФ $N$,\\ чел. \end{tabular}&
\tabcolsep=0pt\begin{tabular}{c}Душевые доходы $D$,\\ руб./чел. в~год \end{tabular}&
\tabcolsep=0pt\begin{tabular}{c}Индекс\\ потребительских цен $I$,\\ доля ед. \end{tabular}\\
\hline
1998&147\,802\,133&12\,122,4&1,844\\
1999&147\,539\,426&19\,906,8&1,365\\
2000&146\,890\,128&27\,373,2&1,202\\
2001&146\,303\,611&36\,744,0&1,186\\
2002&145\,649\,334&47\,366,4&1,151\\
2003&144\,963\,650&62\,044,8&1,120\\
2004&144\,168\,205&76\,923,6&1,117\\
2005&143\,474\,219&97\,342,8&1,109\\
2006&142\,753\,551&122\,352,0\hphantom{9}&1,090\\
2007&142\,220\,968&151\,232,4\hphantom{9}&1,119\\
2008&142\,008\,800&179\,287,2\hphantom{9}&1,133\\
2009&141\,904\,000&202\,282,8\hphantom{9}&1,088\\
2010&141\,914\,509&226\,572,0\hphantom{9}&1,088\\
\hline
\end{tabular}
\end{center}
\vspace*{-6pt}
\end{table*}

  По результатам специально организованного численного эксперимента,
описанного в~работах~[12, 15--17], на основании данных за~1998--2009~гг.\ была
получена модель, с~достаточно высокой точностью описывающая
экспериментальные данные:
  \begin{equation}
  \left.
\!\! \begin{array}{rl}
\displaystyle \!\!\!\!\!\fr{dN}{dt} &= 8{,}139\cdot 10^{-22} \fr{N^{2{,}05} D^2}{I^2} -
64{,}1\fr{N^{0{,}33} D^{0{,}3}}{I^{0{,}3}};\\[6pt]
\displaystyle \!\!\!\!\!\fr{dD}{dt} &= 560 D^{0{,}35} -9900 I;\\[6pt]
\displaystyle \!\!\!\!\!\fr{dI}{dt} &= 0{,}131 I^{-0{,}4} -0{,}0072 \fr{N^{0{,}092}
D^{0{,}092}}{I^{0{,}092}}.
  \end{array}\!\!
  \right\}\!\!\!\!\!
  \label{e14-spi}
  \end{equation}

  Для непосредственной реализации численного эксперимента в~среде
программирования Delphi был разработан модуль, позволяющий учитывать ряд
вытекающих из смысла решаемой задачи дополнительных условий и~давать
наглядную интерпретацию проводимых расчетов. Концепция модели
базировалась на следующих предположениях:
  \begin{itemize}
\item численность населения влияет как на прирост, так и~на убыль населения
(данное предположение очевидно в~силу того, что чем больше людей, тем чаще
рождаются дети и~тем больше смертных случаев от естественных и~прочих
причин);
\item реальный годовой доход жителей страны влияет на изменение
численности населения как положительно (чем больше реальный доход семьи,
тем проще решиться на рождение ребенка), так и~отрицательно (семьи с~большими доходами, как правило, имеют по 1--2~ребенка);
\item на изменение душевых доходов сами доходы сказываются положительно
(как правило, государство, осуществляющее со\-ци\-аль\-но-на\-прав\-лен\-ную
политику, само индексирует доходы граждан и~не позволяет работодателям
снижать заработную плату);
\item индекс потребительских цен снижает доходы;
\item высокий уровень индекса потребительских цен заставляет население
экономить, что приводит
 к~снижению потребления товаров и~услуг и, как\linebreak\vspace*{-12pt}

\pagebreak



\pagebreak

\noindent
следствие, к~снижению скорости изменения самого индекса потребительских
цен;
\item высокий реальный годовой доход населения обусловливает устойчивый
спрос на товары и~услуги, что обеспечивает благоприятные условия работы для
производителей, а следовательно, замедляет скорость изменения индекса
потребительских цен.
\end{itemize}

  Вместе с~тем, как показывает практика использования статистической
информации, исходные данные для модели~(13) (табл.~1) нельзя рассматривать
как абсолютно точные. Это обусловлено рядом причин объективного
характера: запаздывание сбора и~обработки данных, неточности (а~иногда и~умышленные искажения) предоставляемой информации, аккумулируемой
органами статистики, и~пр. В~этой связи, несмотря на хорошую точность
построенной модели, целесообразной представляется оценка значимости
имеющихся экспериментальных данных. И~одним из возможных способов
реализации этого является описанный в~настоящей работе подход.

  Ниже приводится описание предложенного подхода применительно к~первому уравнению модели~(13). Для этого потребовалось осуществить
линеаризацию данного уравнения, что было осуществлено посредством
использования разложения в~ряд Тейлора с~центром в~точке $\left\{ a_1^0,
a_2^0, \alpha_{1,2}\hm=0, \beta_{1,2}\hm=0, \gamma_{1,2}\hm=0\right\}$:

\vspace*{-3pt}

\noindent
  \begin{multline}
  \Delta N\approx a_1+a_1^0 \ln N\cdot \alpha_1 +a_1^0 \ln D\cdot \beta_1 +a_1^0
\ln I\cdot \gamma_1-{}\\
  {}- a_2 -a_2^0 \ln N\cdot \alpha_2 -a_2^0 \ln D\cdot \beta_2-a_2^0 \ln I\cdot
\gamma_2\,.
  \label{e15-spi}
  \end{multline}



%  \columnbreak

  Были введены следующие обозначения: $x_1\hm= a_1$; $x_2\hm= a_2$;
$x_3\hm=\alpha_1$; $x_4\hm= \alpha_2$; $x_5\hm= \beta_1$; $x_6\hm= \beta_2$;
$x_7\hm= \gamma_1$; $x_8\hm= \gamma_2$.

  На основании исходной информации (см.\ табл.~1) для определения параметров
$\left\{ a_i, \alpha_i, \beta_i, \gamma_i\right\}$, $i\hm=\overline{1,2}$, на
основании соотношений~(\ref{e15-spi}) были сформированы матрицы~$A$
и~$B$:
\begin{multline}
   A=\left(
   \begin{array}{cccccc}
  1 & -1& 0{,}019 & 1205{,}810 & 0{,}09\hphantom{9} & 602{,}720 \\
  1& -1& 0{,}019& 1205{,}696 & 0{,}010& 634{,}514\\
  1& -1& 0{,}019 & 1205{,}413& 0{,}010& 654{,}930\\
  1 &-1& 0{,}019 & 1205{,}157& 0{,}011& 673{,}802\\
  1 &-1& 0{,}019 & 1204{,}869& 0{,}011& 690{,}079\\
  1 &-1& 0{,}019 & 1204{,}567& 0{,}011& 707{,}383\\
  1 &-1& 0{,}019 & 1204{,}214& 0{,}011& 721{,}161\\
  1 &-1& 0{,}019 & 1203{,}905& 0{,}011& 736{,}252\\
  1 &-1& 0{,}019 & 1203{,}582& 0{,}012& 750{,}910\\
  1 &-1& 0{,}019 & 1203{,}342& 0{,}012& 764{,}493\\
  1 &-1& 0{,}019 & 1203{,}247& 0{,}012& 775{,}433\\
  1 &-1& 0{,}019 & 1203{,}199& 0{,}012& 783{,}713
  \end{array}
  \right.\\
  \left.\begin{array}{cc}
   0,000 & 0\\
0{,}000 & {}-19{,}945 \\
0{,}000 & {}-11{,}794 \\
0{,}000 & {}-10{,}935 \\
0{,}000& \hphantom{9}-9{,}014\\
0{,}000&\hphantom{9}-7{,}264\\
0{,}000& \hphantom{9}-7{,}092\\
0{,}000 & \hphantom{9}-6{,}632\\
0{,}000& \hphantom{9}-5{,}524\\
0{,}000 & \hphantom{9}-7{,}207\\
0{,}000& \hphantom{9}-8{,}004\\
0{,}000& \hphantom{9}-5{,}406
   \end{array}
   \right) \,;\quad
  B=\begin{pmatrix}
  -262647{,}1\\
  -649238{,}1\\
  -586457{,}1\\
  -654217{,}1\\
  -685624{,}1\\
  -795385{,}1\\
  -693926{,}1\\
  -720608{,}1\\
  -532523{,}1\\
  -212070{,}1\\
  -104799{,}1\\
  \hphantom{-9}10589{,}9
  \end{pmatrix}\,.
  \label{e16-spi}
  \end{multline}

  \noindent
  (Значения параметров $a_1^0$ и~$a_2^0$ полагались равными 0,001 и~64,1
соответственно.)

\begin{table*}[b]\small %tabl2
\begin{center}
\parbox{362pt}{\Caption{Оценки погрешностей измерений для уравнений модели~(\ref{e13-spi})
и~значимости ограничений на их параметры}


}

\vspace*{2ex}

\begin{tabular}{|c|c|c|c|c|}
\hline
&&&&\\[-10pt]
\multicolumn{1}{|c|}{\raisebox{-6pt}[0pt][0pt]{Параметр}}&Диапазон &
Решение &\multicolumn{2}{c|}{Двойственные
оценки
$y^3$}\\
\cline{4-5}
&вариации&прямой задачи&по левой границе&по правой границе\\
\hline
\multicolumn{5}{|c|}{Уравнение~1 ($\varepsilon\hm=402\,796{,}7$)}\\
\hline
$a_1$&[0, 3] &3,0&---&0\\
$a_2$&[50, 400\,000] &396\,128,4&0&0\\
$\alpha_1$&[1,948, 2,050] &2,05&0&0\\
$\alpha_2$&[0, 5]&0,0&---&0\\
$\beta_1$&[0, 5]&5,0&---&0\\
$\beta_2$&[0, 5]&5,0&---&3,82\\
$\gamma_1$&[$-$5, 0]&0,0&0&---\\
$\gamma_2$&[$-$5, 0]&0,0&0&---\\
\hline
\multicolumn{5}{|c|}{Уравнение 2 ($\varepsilon=394{,}56$)}\\
\hline
$a_3$&[500, 600] &600,0&0&$6{,}6\cdot 10^{-5}$\\
$a_4$&[9000, 11 000] &9000,0& &0\\
$\alpha_3$&[0, 1]&1,0&---&$1{,}4\cdot 10^{-4}$\\
$\alpha_4$&[0, 1]&0,019&---&0\\
$\beta_3$&[0, 1]&0,0&---&0\\
$\beta_4$&[0, 1]&0,05&---&0\\
$\gamma_3$&[0, 1]&0,0&---&0\\
$\gamma_4$&[0, 2]&0,187&---&0\\
\hline
\multicolumn{5}{|c|}{Уравнение 3 ($\varepsilon=0{,}26$)}\\
\hline
$a_5$&[0, 1] &0,0&---&0\\
$a_6$&[0, 1]&0,023&---&0\\
$\alpha_5$&[0, 1]&0,0&---&0\\
$\alpha_6$&[0, 1]&0,0&---&0\\
$\beta_5$&[0, 1]&0,0&---&0\\
$\beta_6$&[0, 1]&0,0&---&0\\
$\gamma_5$&[$-$1, 0]&0,0&0&0\\
$\gamma_6$&[$-$1, 0]&0,0&0&0\\
\hline
\end{tabular}
\end{center}
\end{table*}

  Все требования к~параметрам $\left\{ a_i, \alpha_i, \beta_i, \gamma_i\right\}$,
$i\hm=\overline{1,2}$, были учтены в~виде дополнительных ограничений:

\noindent
  \begin{alignat*}{2}
  0&\leq x_1\leq 3\,; &\quad
  50& \leq x_2\leq 100\,;\\
  1{,}948 &\leq x_3\leq 2{,}050\,; &\quad
  0&\leq x_4\leq 1\,;\\
  0&\leq x_5\leq 5\,; &\quad
  0&\leq x_6\leq 5\,;\\
  -5&\leq x_7\leq 0\,; &\quad
  -5&\leq x_8 \leq 0\,.
  \end{alignat*}

  Таким образом, задача определения приближен\-ного значения параметров
$\left\{ a_i, \alpha_i, \beta_i, \gamma_i\right\}$, $i\hm=\overline{1,2}$, была
формализована в~следующем виде:
  \begin{equation}
  \left.
  \begin{array}{c}
  \varepsilon \to \min\,;\\[6pt]
  \vert AX-B\vert \leq \varepsilon\,;\\[6pt]
  0\leq x_1\leq 3\,;\quad
  50\leq x_2 \leq 400\,000\,;\\[6pt]
  1{,}948\leq x_3\leq 2{,}050\,;\quad
  0\leq x_4\leq 1\,;\\[6pt]
  0\leq x_5\leq 5\,;\quad
  0\leq x_6\leq 5\,;\\[6pt]
  -5\leq x_7 \leq 0\,;\quad
  -5\leq x_8 \leq 0\,.
  \end{array}
  \right\}
  \label{e17-spi}
  \end{equation}
  (Матрицы~$A$ и~$B$ имеют вид~(\ref{e16-spi}).)

  \begin{table*}\small %tabl3
\begin{center}
\Caption{Оценки значимости измерений для уравнений модели~(\ref{e13-spi})}
\vspace*{2ex}

\begin{tabular}{|c|c|c|c|c|c|c|}
\hline
Номер эксперимента &\multicolumn{2}{c|}{Уравнение
1}&\multicolumn{2}{c|}{Уравнение 2}&\multicolumn{2}{c|}{Уравнение 3}\\
\cline{2-7}
&&&&&&\\[-10pt]
(измерения)&$y^1$&$y^2$&$y^1$&$y^2$&\ \ $y^1$\ \ &$y^2$\\
\hline
1&0&0&0,0345&0&0&0,5\\
2&0&0&0&0&0,5&0\\
3&0&0&0&0&0&0\\
4&0&0&0&0&0&0\\
5&0&0&0&0&0&0\\
6&0,5&0&0,1323&0&0&0\\
7&0&0&0&0&0&0\\
8&0&0&0&0&0&0\\
9&0&0&0&0,5&0&0\\
10\hphantom{9}&0&0&0&0&0&0\\
11\hphantom{9}&0&0&0&0&0&0\\
12\hphantom{9}&0&0,5&0,3333&0&0&0\\
\hline
\multicolumn{7}{p{96mm}}{\footnotesize \hspace*{3mm}\textbf{Замечание:}
$y^1$ и~$y^2$~--- верхняя и~нижняя границы,
определяемые соответ\-ст\-ву\-ющим ограничением группы  $\vert AX\hm- B\vert \hm\leq \varepsilon$.}
\end{tabular}
\end{center}
\vspace*{-6pt}
\end{table*}

  Решение прямой задачи~(\ref{e17-spi}):
  $$
  X=\left( 3 \ \ 396\,128{,}4\ \ 2{,}05\ \ 0 \ \ 5\ \ 5\ \ 0\ \ 0\right)^{\mathrm{T}}\,;
  $$
  погрешность измерений $\varepsilon\hm=402\,796{,}7$; решение
двойственной к~ней (см.\ модель~(\ref{e7-spi})):
  \begin{align*}
  y^1 &= \left( 0\  0 \ 0 \ 0 \ 0 \ 0{,}5  \ 0\ 0\ 0\ 0\ 0\ 0\right)^{\mathrm{T}}\,;\\
  y^2 &= \left( 0\ 0\ 0\ 0\ 0\ 0\ 0\ 0\ 0\ 0\ 0\ 0{,}5\right)^{\mathrm{T}}\,;\\
  y^3 &=\left( 0\ 0\ 0\ 0\ 0\ 0\ 0\ 3{,}82\ 0\ 0\right)^{\mathrm{T}}\,.
  \end{align*}

  Аналогичные действия были проведены и~с~остальными уравнениями
системы~(\ref{e13-spi}). В~качестве центров разложения в~рядах Тейлора для
параметров~$a_i$, $i\hm= \overline{3,6}$, были выбраны значения этих же
параметров в~модели~(\ref{e14-spi}). Диапазоны вариации переменных,
необходимые для формирования моделей~(\ref{e17-spi}) по каждому
уравнению системы~(\ref{e13-spi}), приведены в~табл.~2.



  Полученные результаты (см.\ табл.~2) показывают, что на погрешность
измерений~$\varepsilon$ второго и~третьего уравнения системы~(\ref{e13-spi})
заданные диапазоны вариаций параметров каждого из этих уравнений не
оказывают влияния (об этом свидетельствуют нулевые значения двойственных
переменных~$y^3$). Это означает, что величины погрешностей измерений для
данных уравнений в~первую очередь определяются измерениями (табл.~3). Так,
погрешность измерений второго уравнения определяется в~большей степени
  9-м и~12-м измерениями (9-я компонента вектора~$y^2$ и~12 компонента
вектора~$y^1$ равны соответственно 0,5 и~0,333) и~в~меньшей~--- 1-м и~\mbox{6-м}.
На погрешность измерений~$\varepsilon$ третьего уравнения в~равной степени
влияют 1-е и~2-е измерения. Среди параметров первого уравнения есть один
($\beta_2$), верхняя граница значений которого оказывает существенное
влияние на погрешность измерений~$\varepsilon$ (соответствующее значение
двойственной оценки~$y^3$ равно~3,82). Также на значение~$\varepsilon$
в~равной степени оказывают влияние 6-е и~12-е измерения, о~чем
свидетельствуют 6-я и~12-я компоненты векторов~$y^1$ и~$y^2$
соответственно.



  Данная информация может быть использована при планировании дальнейших
экспериментов\linebreak с~моделями~(\ref{e13-spi}) или~(\ref{e14-spi}). При этом должен
соблюдаться следующий принцип: в~каждом после\-дующем эксперименте
информативность новых измере\-ний должна быть не менее значима, что\linebreak должно
отражаться при формировании соответствующих условий. Например, сужение
или увеличение диапазона значений параметра~$\beta_2$ целесообразно
проводить за счет вариации верхней \mbox{границы} интервала его значений.

\vspace*{-9pt}

{\small\frenchspacing
 {%\baselineskip=10.8pt
 \addcontentsline{toc}{section}{References}
 \begin{thebibliography}{99}

 \vspace*{-2pt}

\bibitem{2-spi}
\Au{Кендалл М., Стьюарт А.} Многомерный статистический анализ и~временные ряды.~--- М.: Наука, 1976. 736~с.

\bibitem{1-spi}
\Au{Айвазян С.\,А., Мхитарян В.\,С.} Прикладная статистика и~основы
эконометрики.~--- М.: ЮНИТИ, 1998. 1022~с.

\bibitem{3-spi}
Эконометрика~/ Под ред. И.\,И.~Елисеевой.~--- М.: Финансы и~статистика,
2006. 576~с.
\bibitem{4-spi}
\Au{Марчук Г.\,И.} Методы вычислительной математики.~--- М.: Наука, 1977.
457~с.
\bibitem{5-spi}
\Au{Спивак С.\,И., Тимошенко В.\,И., Слинько~М.\,Г.} Методы построения
кинетических моделей стационарных реакций~// Химическая промышленность,
1979. №\,3. С.~33--36.

\bibitem{8-spi} %6
\Au{Канторович Л.\,В.} Экономический расчет наилучшего использования
ресурсов.~--- М.: Изд-во Академии наук СССР, 1960. 347~с.

\bibitem{6-spi} %7
\Au{Зуховицкий С.\,И., Авдеева Л.\,И.} Линейное и~выпуклое
программирование.~--- М.: Наука, 1967. 460~с.
\bibitem{7-spi} %8
\Au{Канторович Л.\,В., Горстко А.\,Б.} Оптимальные решения в~экономике.~---
М.: Наука, 1972. 231~с.

\bibitem{9-spi}
\Au{Кузнецов С.\,И., Юмагулова Р.\,Х., Медведева~Н.\,А., Хамидуллин~Ф.\,Ф.,
Колесов~С.\,В.} Фуллеренсодержащие полимеры. Уф-спектроскопическое
исследование~// Высокомолекулярные соединения. Сер.~А, 2012. Т.~54. №\,6.
С.~859--864.
\bibitem{10-spi}
\Au{Кузнецов С.\,И., Хамидуллин Ф.\,Ф., Юмагулова~Р.\,Х.,
Медведева~Н.\,А., Лебедев~Ю.\,А., Колесов~С.\,В.} Самоорганизация
функционализированных фуллереном C$_{60}$ макромолекул
полиметилметакрилата и~полистирола~// Высокомолекулярные
соединения. Сер.~А, 2012, Т.~54. №\,10. С.~1527--1531.
\bibitem{13-spi} %11
\Au{Спивак С.\,И., Кантор О.\,Г.} Оценка параметров моделей системной
динамики~// Журнал СВМО, 2011. Т.~13. №\,3. С.~107--113.
\bibitem{17-spi} %12
\Au{Спивак С.\,И., Кантор О.\,Г., Салахов~И.\,Р.} О~программе,
корректирующей систему уравнений~// Журнал СВМО, 2011. Т.~13. №\,4.
С.~87--93.
\bibitem{11-spi} %13
\Au{Спивак С.\,И., Кантор О.\,Г.} Качество моделей математической обработки
наблюдений социально-эко\-но\-ми\-че\-ских систем~// Системы управления и~информационные технологии, 2012. №\,2(48). C.~44--49.

\pagebreak


\bibitem{12-spi} %14
\Au{Спивак С.\,И., Кантор О.\,Г.} Оценка качества спецификации моделей
системной динамики~// Журнал СВМО, 2012. Т.~14. №\,2. С.~34--39.
\bibitem{16-spi} %15
\Au{Спивак С.\,И., Кантор О.\,Г., Салахов И.\,Р.} Вычислительная реализация
оценки управляющих па\-ра\-мет\-ров модели системной динамики~// Вестник
Башкирского университета, 2012. Т.~17. №\,4. С.~1658--1660.
\bibitem{15-spi} %16
\Au{Спивак С.\,И., Кантор О.\,Г., Салахов~И.\,Р.} Алгоритм получения
прогнозируемых параметров социально-эко\-но\-ми\-че\-ских систем~// Системы
управления и~информационные технологии, 2013. №\,4(54). С.~43--45.
\bibitem{14-spi} %17
\Au{Спивак С.\,И., Кантор О.\,Г.} Построение моделей сис\-тем\-ной динамики в~условиях ограниченной экспертной информации~// Информатика и~её
применения, 2014. Т.~8. Вып.~2. С.~112--122.


 \end{thebibliography}

 }
 }

\end{multicols}

\vspace*{-3pt}

\hfill{\small\textit{Поступила в редакцию 04.07.14}}

%\newpage

\vspace*{12pt}

\hrule

\vspace*{2pt}

\hrule

%\vspace*{12pt}

\def\tit{EVALUATION OF MEASUREMENT ACCURACY AND~SIGNIFICANCE
FOR LINEAR MODELS}

\def\titkol{Evaluation of measurement accuracy and~significance
for linear models}

\def\aut{S.\,I.~Spivak$^1$, O.\,G.~Kantor$^2$, D.\,S.~Yunusova$^1$, S.\,I.~Kuznetsov$^3$,
and~S.\,V.~Kolesov$^3$}

\def\autkol{S.\,I.~Spivak, O.\,G.~Kantor, D.\,S.~Yunusova, et al.}

\titel{\tit}{\aut}{\autkol}{\titkol}

\vspace*{-9pt}

\noindent
$^1$Bashkir State University, 32 Validy Str., Ufa 450076, Russian Federation

\noindent
$^2$Institute of Social and Economic Research, Ufa Scientific Center, Russian Academy of
Sciences; 71~Av.~Oktyabrya,\linebreak
$\hphantom{^1}$Ufa 450054, Russian Federation

\noindent
$^3$Institute of Organic Chemistry, Ufa Scientific Center,
 Russian Academy of Sciences, 71~Av.~Oktyabrya, Ufa\linebreak
 $\hphantom{^1}$450054, Russian Federation


\def\leftfootline{\small{\textbf{\thepage}
\hfill INFORMATIKA I EE PRIMENENIYA~--- INFORMATICS AND
APPLICATIONS\ \ \ 2015\ \ \ volume~9\ \ \ issue\ 1}
}%
 \def\rightfootline{\small{INFORMATIKA I EE PRIMENENIYA~---
INFORMATICS AND APPLICATIONS\ \ \ 2015\ \ \ volume~9\ \ \ issue\ 1
\hfill \textbf{\thepage}}}

\vspace*{3pt}

\Abste{Identification of a linear dependency, when exact solution obtained by
standard methods does not meet the objective requirements, determines development
of specific approaches for their numerical realization. A~method to obtain
approximate values of linear models parameters on experimental data, which is
based on the use of the linear programming methodology and the duality theory, is
presented. This method makes it possible to obtain approximate solutions
that fulfill all requirements to the model and its parameters and to evaluate accuracy
and significance of measurements. It is important for improving the procedure of
construction of functional dependencies on the stage of planning experiments if they
do not satisfy the authenticity criteria. The results of testing the proposed method for
problems connected with research of chemical and socioeconomic systems are
given.}

\KWE{problems of linear dependencies recovering; measurement accuracy;
measurement significance; dual estimates}



\DOI{10.14357/19922264150108}

\Ack
\noindent
The research was financially supported by the Russian Foundation for
Basic Research (project 13-01-00749).




%\vspace*{3pt}

  \begin{multicols}{2}

\renewcommand{\bibname}{\protect\rmfamily References}
%\renewcommand{\bibname}{\large\protect\rm References}



{\small\frenchspacing
 {%\baselineskip=10.8pt
 \addcontentsline{toc}{section}{References}
 \begin{thebibliography}{99}
 \bibitem{5-spi-1} %1
\Aue{Kendall, M., and A.~St'yuart}. 1976. \textit{Mnogomernyy statisticheskiy
analiz i~vremennye ryady} [Multivariate statistical analysis and time series].
Moscow: Nauka. 736~p.
\bibitem{1-spi-1} %2
\Aue{Ayvazyan, S.\,A., and V.\,S. Mkhitaryan}. 1998. \textit{Prikladnaya statistika
i~osnovy ekonometriki} [Applied statistics and bases of econometrics]. Moscow:
YUNITI. 1022~p.
\bibitem{17-spi-1} %3
Eliseeva, I.\,I., ed. 2008.
\textit{Ekonometrika} [Econometrics].  Moscow: Finance and
Statistics. 576~p.
\bibitem{8-spi-1} %4
\Aue{Marchuk, G.\,I.} 1977. \textit{Metody vychislitel'noy matematiki} [Methods of
calculus mathematics]. Moscow: Nauka. 457~p.
\bibitem{16-spi-1} %5
\Aue{Spivak, S.\,I., V.\,I. Timoshenko, and M.\,G.~Slin'ko}. 1979. Metody
postroeniya kineticheskikh modeley sta\-tsi\-o\-nar\-nykh reaktsiy [Methods of creation of
kinetic models of stationary reactions].  \textit{Khimicheskaya Promyshlennost'}
[Chemical Industry] 3:33--36.
\bibitem{4-spi-1} %6
\Aue{Kantorovich, L.\,V.} 1960. \textit{Ekonomicheskiy raschet nailuchshego
ispol'zovaniya resursov} [Economic calculation of the best use of resources].
Moscow: Publishing House of Academy of Sciences of the USSR. 347~p.
\bibitem{2-spi-1} %7
\Aue{Zukhovitskiy, S.\,I., and L.\,I.~Avdeeva}. 1967. \textit{Lineynoe i~vypukloe
programmirovanie} [Linear and convex programming]. Moscow: Nauka. 460~p.
\bibitem{3-spi-1} %8
\Aue{Kantorovich, L.\,V., and A.\,B.~Gorstko}. 1972. \textit{Optimal'nye resheniya
v~ekonomike} [Optimal solutions in economics]. Moscow: Nauka. 231~p.
\bibitem{6-spi-1} %9
\Aue{Kuznetsov, S.\,I., R.\,Kh.~Yumagulova, N.\,A.~Medvedeva,
F.\,F.~Khamidullin, and S.\,V.~Kolesov}. 2012. Fullerensoderzhashchie polimery.
UF-spektroskopicheskoe issledovanie [Fullerene polymers. UV spectroscopic
research]. \textit{Vysokomolekulyarnye Soedineniya. Ser.~A} [Macromolecular
Compounds A] 6:859--864.
\bibitem{7-spi-1} %10
\Aue{Kuznetsov, S.\,I., F.\,F. Khamidullin, R.\,Kh.~Yumagulova,
N.\,A.~Medvedeva, Yu.\,A.~Lebedev, and S.\,V.~Kolesov}. 2012.
Samoorganizatsiya funktsionalizirovannykh fullerenom C$_{60}$ makromolekul
polimetilmetakrilata i~polistirola [Self-organization functionalized with fullerenes
C$_{60}$ of macromolecules polymethylmethacrylate and polystyrene].
\textit{Vysokomolekulyarnye Soedineniya. Ser.~A} [Macromolecular Compounds
A] 10:1527--1531.
\bibitem{11-spi-1} %11
\Aue{Spivak, S.\,I., and O.\,G. Kantor}. 2011. Otsenka parametrov modeley
sistemnoy dinamiki [Estimation of parameters of system dynamics models].
\textit{Zh. SVMO} [J.~SVMO] 3:107--113.
\bibitem{15-spi-1} %12
\Aue{Spivak, S.\,I., O.\,G. Kantor, and I.\,R.~Salakhov}. 2011. O~programme,
korrektiruyushchey sistemu uravneniy [About the program correcting system of the
equations]. \textit{Zh. SVMO} [J.~SVMO] 4:87--93.

\bibitem{9-spi-1} %13
\Aue{Spivak, S.\,I., and O.\,G. Kantor}. 2012. Kachestvo modeley matematicheskoy
obrabotki sotsial'no-ekonomicheskikh sistem [Quality of mathematical processing
observations models of socioeconomic systems]. \textit{Sistemy Upravleniya
i~Informatsionnye Tekhnologii} [Management and Information Technology]
2(48):44--49.
\bibitem{10-spi-1} %14
\Aue{Spivak, S.\,I., and O.\,G. Kantor}. 2012. Otsenka kachestva spetsifikatsii
modeley sistemnoy dinamiki [Estimation of quality of specification of system
dynamics models]. \textit{Zh. SVMO} [J.~SVMO] 2:34--39.
\bibitem{14-spi-1} %15
\Aue{Spivak, S.\,I., O.\,G. Kantor, and I.\,R.~Salakhov}. 2012. Vychislitel'naya
realizatsiya otsenki upravlyayushchikh parametrov modeli sistemnoy dinamiki
[Computational realization of estimation of the control parameters of system
dynamics model]. \textit{Vestnik Bashkirskogo Universiteta} [Bashkir University
Bulletin] 4:1658--1660.
\bibitem{13-spi-1} %16
\Aue{Spivak, S.\,I., O.\,G. Kantor, and I.\,R.~Salakhov}. 2013. Algoritm polucheniya
prognoziruemykh parametrov sotsial'no-ekonomicheskikh sistem [Algorithm of
obtaining predicted parameters of social and economic systems]. \textit{Sistemy
Upravleniya i~Informatsionnye Tekhnologii} [Management and Information
Technology] 4(54):43--45.
\bibitem{12-spi-1} %17
\Aue{Spivak, S.\,I., and O.\,G. Kantor}. 2014. Postroenie modeley sistemnoy
dinamiki v usloviyakh ogranichennoy ekspertnoy informatsii [Construction of system
dynamics models in the conditions of limited expert information]. \textit{Informatika
i~ee Primeneniya}~--- \textit{Inform. Appl.} 2:112--122.
\end{thebibliography}

 }
 }

\end{multicols}

\vspace*{-3pt}

\hfill{\small\textit{Received July 4, 2014}}

%\vspace*{-18pt}


     \Contr

     \noindent
     \textbf{Spivak Semen I.} (b.\ 1945)~--- Doctor
     of Science in physics and mathematics, professor, Bashkir
State University, 32 Zaki Validi Str., Ufa 450074, Russian Federation; semen.spivak@mail.ru

\vspace*{3pt}

\noindent
\textbf{Kantor Olga G.} (b.\ 1971)~--- Candidate of Science (PhD) in physics and mathematics, senior
scientist, Institute of Social and Economic Research, Ufa Scientific Centre, Russian Academy of Sciences,
71 October Av., Ufa 450054, Russian Federation; o\_kantor@mail.ru

\vspace*{3pt}

\noindent
\textbf{Yunusova Darya S.} (b.\ 1989)~---  PhD student, Bashkir State University, 32 Zaki Validi Str., Ufa
450074, Russian Federation; kazakova\_d\_s@mail.ru

\vspace*{3pt}

\noindent
\textbf{Kuznetsov Sergey I.} (b.\ 1955)~--- scientist, Institute of Organic Chemistry, Ufa Scientific Center,
Russian Academy of Sciences, 71 October Av., Ufa 450054, Russian Federation; chemorg@anrb.ru

\vspace*{3pt}

\noindent
\textbf{Kolesov Sergey V.} (b.\ 1951)~--- Head of Laboratory, Institute of Organic Chemistry, Ufa
Scientific Center, Russian Academy of Sciences, 71 October Av., Ufa 450054, Russian Federation;
kolesovservic@rambler.ru

\label{end\stat}

\renewcommand{\bibname}{\protect\rm Литература}
