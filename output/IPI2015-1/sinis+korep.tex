
\def\hx{\hat X}

\def\stat{sin+korep}

\def\tit{УСТОЙЧИВЫЕ ЛИНЕЙНЫЕ УСЛОВНО
ОПТИМАЛЬНЫЕ ФИЛЬТРЫ И~ЭКСТРАПОЛЯТОРЫ ДЛЯ СТОХАСТИЧЕСКИХ
СИСТЕМ С~МУЛЬТИПЛИКАТИВНЫМИ ШУМАМИ}

\def\titkol{Устойчивые ЛУОФ и ЛУОЭ %линейные условно оптимальные фильтры и~экстраполяторы
для стохастических
систем с~мультипликативными шумами}

\def\aut{И.\,Н.~Синицын$^1$, Э.\,Р.~Корепанов$^2$}

\def\autkol{И.\,Н.~Синицын, Э.\,Р.~Корепанов}


\titel{\tit}{\aut}{\autkol}{\titkol}

%{\renewcommand{\thefootnote}{\fnsymbol{footnote}} \footnotetext[1]
%{Статья подготовлена при поддержке РГНФ (грант 13-03-00384).}}


\renewcommand{\thefootnote}{\arabic{footnote}}
\footnotetext[1]{Институт проблем
информатики Российской академии наук, sinitsin@dol.ru}
\footnotetext[2]{Институт проблем
информатики Российской академии наук, ekorepanov@ipiran.ru}

\vspace*{-12pt}

\Abst{Статья посвящена теории аналитического синтеза непрерывных
равномерно асимптотически устойчивых условно оптимальных (по
среднеквадратическому критерию) линейных фильтров и экстраполяторов
(ЛУОФ и~ЛУОЭ) для линейных дифференциальных стохастических систем
(СтС) с~линейными мультипликативными шумами. Предполагается, что
наблюдение входит как в~уравнение состояния, так и~в~уравнение
наблюдения. Белые шумы в~уравнениях наблюдения и~состояния
предполагаются заданными априори в~виде производных по времени от
произвольных процессов с~независимыми приращениями.
Доказаны теоремы, лежащие в~основе теории непрерывных устойчивых
ЛУОФ и~ЛУОЭ. Достаточные условия равномерной асимптотической
устойчивости сформулированы в~виде требований положительной
определенности и~равномерной стохастической ограниченности некоторых
матриц, отражающих свойства наблюдаемости и~управляемости. Приведен
иллюстративный пример. Сформулированы некоторые
обобщения.}

\KW{мультипликативный белый шум;
равномерная асимптотическая устойчивость;
стохастическая система (СтС);
точность; уравнение Риккати;
линейный условно оптимальный фильтр и~экстраполятор (ЛУОФ и ЛУОЭ)}

\DOI{10.14357/19922264150106}

\vspace*{-6pt}


\vskip 10pt plus 9pt minus 6pt

\thispagestyle{headings}

\begin{multicols}{2}

\label{st\stat}

\section{Введение}

\vspace*{-2pt}

В настоящее время теория ЛУОФ и~ЛУОЭ Пугачёва для  непрерывных  линейных
СтС с~аддитивными гауссовскими и~негауссовскими шумами
подробно разработана и~реализована в универсальных и~специальных программных
средствах (см., например,~[1--5]).

Для  гауссовских СтС с мультипликативными шумами в уравнениях состояния
и~наблюдения в~[1--3] получены соответствующие матричные уравнения Риккати
для оценки точности ЛУОФ и~ЛУОЭ. При этом их устойчивость для
стохастических процессов (СтП)  в~таких СтС  не рассматривалась.

В настоящей статье на основе известных свойств матричных уравнений
Риккати получены достаточные условия равномерной асимптотической устойчивости
непрерывных ЛУОФ и~ЛУОЭ и~их теория. Сформулированы дальнейшие обобщения.

\vspace*{-6pt}

\section{Равномерно асимптотически устойчивые линейные условно оптимальные фильтры}

\vspace*{-2pt}

Рассмотрим наблюдаемую непрерывную (дифференциальную) СтС
с~мультипликативными и~аддитивными (в~общем случае негауссовскими)
белыми шумами, описываемую следующими стохастическими  дифференциальными
уравнениями Ито~[1--3]:
   \begin{equation}
   \left.
   \begin{array}{rl}
  \hspace*{-1mm}\dot X_t &=aY_t +a_{1}X_t+a_{0} + \left( c_{10} +
    \displaystyle\sss_{r=1}^{n_y} c_{1r} Y_r +  {}\right.\\[16pt]
&\left.\displaystyle \hspace*{10mm}{}+   \sss_{r=1}^{n_x} c_{1,n_y+r} X_r\right) V\,,\enskip X_{t_0} = X_0\,;\\[16pt]
     \hspace*{-1mm}\dot Y_t &= bY_t +b_1X_t+b_0  + \left( c_{20} +
   \displaystyle\sss_{r=1}^{n_y} c_{2r} Y_r +{}\right.\\[16pt]
   &\left.\hspace*{10mm}{}+
\displaystyle    \sss_{r=1}^{n_x} c_{2,n_y+r} X_r\right) V\,,\enskip Y_{t_0} = Y_0\,,
    \end{array}
    \right\}
\label{e2.1-sk}
\end{equation}
где $X_t$ и $Y_t$~--- векторы состояния и~наблюдения
$(\mathrm{dim}\, X_t = n_x$, $\mathrm{dim}\, Y_t \hm= n_y)$;  $V\hm=\dot W$~---
белый шум $(\mathrm{dim}\, V \hm= n_v)$; $W$~---
векторный СтП с~независимыми приращениями вида
\begin{equation}
W(t)= W_0(t) + \iii\limits_{R_0^q} c(\rho) P^0 (t, d\rho)\,.\label{e2.2-sk}
\end{equation}
Здесь $W_0(t)$~--- винеровский СтП интенсивности  $\nu_0(t)$;
$c(\rho)$~--- некоторая векторная функция той же размерности, что
и~$W(t)$, $q$-мер\-но\-го аргумента, а~интеграл при любом  $t\hm\ge t_0$
представляет собой стохастический интеграл по центрированной пуассоновской мере
$P^0 (t, B)$ интенсивности $\nu_P (t, \rho)$, независимой
от винеровского СтП $W_0(t)$ и~име\-ющей независимые значения на
непересекающихся множествах; $B$~---
борелевское множество пространства~$R_0^q$ с~выколотым началом.
При этом интенсивность СтП $W(t)$ определяется по формуле~\cite{1-sk, 3-sk, 4-sk}
\begin{equation*}
\nu =\nu(t) = \nu_0(t) +
\iii_{R_0^q} c(\rho) c (\rho)^{\mathrm{T}}
 \nu_P (\tau,\rho)\, d\rho\,. %\label{e2.3-sk}
\end{equation*}
В~(\ref{e2.1-sk}) $a_0$, $b_0$, $a$, $a_1$, $b$, $b_1$ и
$c_{ij}$ $(i\hm=1,2$, $j\hm=1\tr n_x)$~--- век\-тор\-но-мат\-рич\-ные функции~$t$,
не зависящие от
$X_t\hm=\lk X_1\cdots X_{n_x}\rk^{\mathrm{T}}$ и $Y_t \hm=\lk
Y_1\cdots Y_{n_y}\rk^{\mathrm{T}}$. Следуя~\cite{1-sk, 3-sk, 4-sk},
класс допустимых ЛУОФ зададим линейным уравнением
\begin{multline}
\dot{\hx_t} =a Y_t +a_{1}\hx_t + a_{0} +{}\\
{}+\beta_t \left[ \dot Y_t -
(b Y_t + b_{1}\hx_t + b_{0})\right]\,.\label{e2.4-sk}
\end{multline}
Здесь $\beta_t$ и $\sigma_{ij}$ $(i,j\hm=1,2)$ определяются уравне\-ниями:
\begin{align}
\beta_t &= (R_t b_1^{\mathrm{T}} +\sigma_{12}) \sigma_{22}^{-1}\,;\notag\\ %\label{e2.5-sk}\\
\sigma_{12}&=\left(\! c_{10} +\sss_{r=1}^{n_y+n_x} c_{1r} m_{r}\!\right) \nu
\left( \!c_{20}^{\mathrm{T}} +\sss_{r=1}^{n_y+n_x} c_{2r}^{\mathrm{T}} m_{r}\!\right)+{}\notag\\
&\hspace*{35mm}{}+\sss_{r,s=1}^{n_y+n_x} c_{1r}\nu c_{2s}^{\mathrm{T}}  k_{rs}\,; \notag\\ %\label{e2.6-sk}\\
\sigma_{22}&=\left(\! c_{20} +\sss_{r=1}^{n_y+n_x} c_{2r} m_{r}\!\right)\nu
    \left(\! c_{20}^{\mathrm{T}} +\sss_{r=1}^{n_y+n_x} c_{2r}^{\mathrm{T}} m_{r}\!\right)+{}\notag\\
&\hspace*{35mm}{}+    \sss_{r=1}^{n_y+n_x} c_{2r}\nu c_{2s}^{\mathrm{T}} k_{rs}\,.\label{e2.7-sk}
    \end{align}
Ковариационная матрица  $R_t$ ошибки фильтрации
$\tilde X_t\hm=\hat X_t \hm-X_t$ удовлетворяет матричному уравнению Риккати:
\begin{multline}
\dot R_t = a_1 R_t + R_t a_1^{\mathrm{T}} - {}\\
{}-(R_t b_1^{\mathrm{T}} +\sigma_{12} )
\sigma_{22}^{-1} (b_1 R_t +\sigma_{21}) +\sigma_{11}\,,\label{e2.8-sk}
\end{multline}
где
\begin{multline*}
\hspace*{-9.60526pt}\sigma_{21}= \left( c_{20} +\sss_{r=1}^{n_y+n_x} c_{2r} m_{r}\right)
     \nu  \left( c_{10}^{\mathrm{T}} +\sss_{r=1}^{n_y+n_x} c_{1r}^{\mathrm{T}} m_{r}\right)+{}\\
{}+\sss_{r,s=1}^{n_y+n_x} c_{2r}\nu c_{1s}^{\mathrm{T}} k_{rs}\,;
%\label{e2.9-sk}
\end{multline*}

%\vspace*{-12pt}

\noindent
\begin{multline*}
\hspace*{-9.60526pt}\sigma_{11}= \left( c_{10} +\sss_{r=1}^{n_x+n_y} c_{1r} m_{r}\right) \nu
     \left( c_{10}^{\mathrm{T}} +\sss_{r=1}^{n_x+n_y} c_{1r}^{\mathrm{T}} m_{r}\right)+{}\\
{}+\sss_{r,s=1}^{n_y+n_x} c_{1r}\nu c_{1s}^{\mathrm{T}} k_{rs}\,. %\label{e2.10-sk}
\end{multline*}
При этом вероятностные моменты первого и~второго порядка
$m_t \hm=\lk m_r\rk$, $K_t \hm=\lk k_{rs}\rk$ вектора
$Q_t \hm= \lk X_t Y_t\rk^{\mathrm{T}}$ ($r, s \hm=1\tr n_y+n_x$)
определяются следующими уравнениями:
\begin{equation}
\dot m_t = \bar a m_t+\bar a_0\,,\enskip m_{t_0}= m_0\,;\label{e2.11-sk}
\end{equation}

\vspace*{-14pt}

\noindent
\begin{multline}
\hspace*{-4.38786pt}\dot K_t = \bar a K_t + K_t \bar a^{\mathrm{T}} + c_0\nu c_0^{\mathrm{T}} +\sss_{r=1}^{n_y+n_x}
 (c_0\nu c_r^{\mathrm{T}} + c_r \nu c_0^{\mathrm{T}})m_r+{}\\
{}+ \sss_{r,s=1}^{n_y+n_x} c_r \nu c_s^{\mathrm{T}} (m_rm_s + k_{rs})\,,
\enskip K_{t_0}=K_0\label{e2.12-sk}
\end{multline}


\vspace*{-14pt}

\noindent
\begin{multline*}
\left(\bar a=\begin{bmatrix}
b&b_{1}\\
a&a_{1}\end{bmatrix}\,,\enskip
\bar a_0= \begin{bmatrix}
b_{0}\\
a_{0}\end{bmatrix}\,,\right.\\
\left.c_r = \begin{bmatrix}
c_{2r}\\ a_{1r}
\end{bmatrix}\,,\enskip r=0,1\tr n_y+n_x\right)\,.
\end{multline*}

Таким образом, в~основе теории синтеза  ЛУОФ лежит следующее утверждение.

\smallskip

\noindent
\textbf{Теорема 2.1.}\ \textit{Пусть процессы $X_t$ и~$Y_t$
в~дифференциальной СтС~$(\ref{e2.1-sk})$
обладают конечными вероятностными моментами  второго порядка,
а~матрица~$(\ref{e2.7-sk})$ не вырождена  $(\mathrm{det} \sigma_{22} \hm\ne 0)$.
Тогда  ЛУОФ определяется уравнением~$(\ref{e2.4-sk})$,
а~его точность~--- матричным уравнением Риккати}~(\ref{e2.8-sk}).

\smallskip

\noindent
\textbf{Замечание~2.1.}
Фильтр, определяемый уравнением~(\ref{e2.4-sk}), оптимален
в~классе всех линейных фильт-\linebreak ров, причем ЛУОФ
является оптимальным и~в~более широком классе всех линейных
фильт\-ров.
 В~частном случае линейной СтС~(\ref{e2.1-sk})
 без мультипликативных шумов уравнения~(\ref{e2.4-sk})
 и~(\ref{e2.8-sk}) совпадают с~уравнениями теории оптимальной линейной
\mbox{фильтрации}~\cite{1-sk, 3-sk}. Для винеровского процесса  $W_0(t)$ при
отсутствии мультипликативных шумов ЛУОФ оказывается оптимальным
и~в~классе всех возможных фильтров.

\smallskip

\noindent
\textbf{Замечание~2.2.}
Особенностью ЛУОФ является то обстоятельство, что
$m_r\hm=m_{rt}$, $K_r\hm=K_{rt}$ и~$R\hm=R_t$ могут быть вычислены заранее
во время синтеза фильтра, так как не требуют знания результатов
текущих наблюдений.

Далее, применяя известное матричное неравенство~\cite{6-sk, 7-sk}
к~уравнению Риккати~(\ref{e2.8-sk}), получим  $\forall\, t\hm\ge t_0$

\noindent
\begin{multline*}
0\le R_t (R_0, t_0) \le u_A(t, t_0) R_0 u^{\mathrm{T}} (t, t_0) +{}\\
{}+
\iii_{t_0}^t u_A(t,\tau)  \bar \si_{11} u_A(t,\tau)^{\mathrm{T}} d\tau={}\\
{}=u_A(t, t_0) R_0 u (t, t_0)^{\mathrm{T}} + \mathcal{W}(t_0, t)\,. %\label{e2.13-sk}
\end{multline*}
Здесь введены следующие обозначения: $R_0\hm=R_{t_0}$~--- неотрицательно
определенная  матрица;
$u_A(t, t_0)$~--- фундаментальная матрица однородного уравнения,
полученного из уравнения Риккати~(\ref{e2.8-sk}):
\begin{equation}
\dot Z =A Z\,,
\label{e2.14-sk}
\end{equation}
где
\begin{gather}
 A = a_1 -\beta_t b_1 = a_1 - R_t b_1^{\mathrm{T}} \sigma_{22}^{-1}
b_1 - \sigma_{12} \sigma_{22}^{-1} b_1\,;\notag\\
\mathcal{W} (t_0, t) =\iii_{t_0}^t u_A (t, \tau) \bar \sigma_{11} (\tau)
u_A (t,\tau)^{\mathrm{T}} d\tau\,;\label{e2.15-sk}\\
\bar \sigma_{11} =\sigma_{11}-\sigma_{12} \sigma_{22}^{-1} \sigma_{21}\,.\label{e2.16-sk}
\end{gather}
Наконец, обобщая понятия равномерной наблюдаемости и~управляемости~\cite{6-sk, 7-sk},
введем понятие равномерной ограниченности
\begin{equation}
0<\alpha_1 I_n \le \mathcal{W}(t-t^\prime, t)\le \alpha_2 I_n \enskip
\forall\ t \ge t_0+t^\prime,\label{e2.17-sk}
\end{equation}
где $\alpha_1$, $\alpha_2$, $t^\prime$~--- постоянные; $n\hm=n_y\hm+n_x$; $I_n$~---
единичная  $(n\times n)$-мат\-рица.

Правая часть~(\ref{e2.8-sk}) удовлетворяет условию Липшица, поэтому имеет
место локальное существование и единственность решения~(\ref{e2.8-sk}). Более
того, верхняя граница решения~(\ref{e2.8-sk}) позволяет определить постоянную
Липшица на любом конечном интервале времени. А~это значит, что
уравнение~(\ref{e2.8-sk}) имеет глобальное единственное решение. Методом
детерминированных функций Ляпунова вида $\mathcal{L}= \xi_t^{\mathrm{T}} R_t \xi$
для уравнения~(\ref{e2.14-sk}) аналогично~\cite{6-sk, 7-sk}
устанавливается следующая теорема.

\smallskip

\noindent
\textbf{Теорема 2.2.} \textit{Пусть в~условиях теоремы~$2.1$
система уравнений~$(\ref{e2.1-sk})$
равномерно вполне наблюдаема и~равномерно вполне управляема,
т.\,е.\ матрица~$(\ref{e2.15-sk})$ положительно определена
и~выполнены условия~$(\ref{e2.17-sk})$. Тогда ЛУОФ~$(\ref{e2.4-sk})$
равномерно асимптотически  устойчив, т.\,е.\
тривиальное решение  образуемого из~$(\ref{e2.4-sk})$ при
$\dot Y_t \hm=0$, $Y_t\hm=0$  уравнения~$(\ref{e2.14-sk})$
равномерно асимптотически устойчиво}.

\smallskip

Случай независимых аддитивных и~мультипликативных негауссовских белых шумов~$V_1$
и~$V_2$ соответственно в~уравнениях состояния и~наблюдения получается на
основе уравнений~(\ref{e2.1-sk}), если принять

\noindent
\begin{gather}
V= \lk V_1^{\mathrm{T}} V_2^{\mathrm{T}}\rk^{\mathrm{T}}\,;\enskip \nu = \begin{bmatrix}
    \nu_1&0\\
    0&\nu_2
    \end{bmatrix}\,;\label{e2.18-sk}\\
\hspace*{-16mm}\left( c_{i0} +\sss_{r=1}^{n_y} c_{ir} Y_r +\sss_{s=1}^{n_x}
c_{i, n_y+r} X_r\right) V= {}\notag\\
\hspace{8mm}{}=\left( c_{i0}' +\sss_{r=1}^{n_y} c_{ir}' Y_r +\sss_{s=1}^{n_x}
c_{i, n_y+r}' X_r\right) V_i\,;\label{e2.19-sk}\\
\sigma_{ii} =\left(\! c_{i0}' +\sss_{r=1}^{n_y+n_x}\! c_{ir}' m_r \!\right)
\nu_i \left(\! c_{i0}^{\prime \mathrm{T}} +\sss_{r=1}^{n_y+n_x}\! c_{ir}^{\prime \mathrm{T}} m_r \!\right),\notag\\
\sigma_{ij} =\sigma_{ji}=0\enskip (i\ne j)\,;\notag %\label{e2.20-sk}
\\
\bar \sigma_{11} =\sigma_{11}\,;\label{e2.21-sk}\\
\beta_t = R_t b_1^{\mathrm{T}} \sigma_{22}^{-1}\,.\label{e2.22-sk}
\end{gather}

\noindent
\textbf{Теорема 2.3.} \textit{Пусть процессы  $X_t$ и~$Y_t$ в~системе
уравнений~$(\ref{e2.1-sk})$ при условиях~$(\ref{e2.18-sk})$ и
$(\ref{e2.19-sk})$ обладают конечными вероятностными моментами второго порядка,
а~мат\-ри\-ца~$\sigma_{22}$ не вырождена $(\mathrm{det}\, \sigma_{22}\hm\ne 0)$.
Тогда ЛУОФ определяется уравнением~$(\ref{e2.4-sk})$
при условии~$(\ref{e2.22-sk})$, а~его точность~--- матричным
уравнением~$(\ref{e2.8-sk})$. Для обеспечения равномерной асимптотической
устойчивости ЛУОФ~$(\ref{e2.4-sk})$ достаточно положительной опре\-де\-лен\-ности
матрицы~$(\ref{e2.15-sk})$ при условии~$(\ref{e2.21-sk})$
и~равномерной ограниченности}~(\ref{e2.17-sk}).


\section{Равномерно асимптотически  устойчивые линейные условно
оптимальные экстраполяторы}


Обобщая~\cite{1-sk, 3-sk} на случай белых шумов $V_1$ и~$V_2$ вида~(\ref{e2.2-sk})
и~с~учетом теоремы~2.1, придем к сле\-ду\-ющим утверждениям.

\smallskip

\noindent
\textbf{Теорема 3.1.}
\textit{Пусть процессы $X_t$, $Y_t$ в~уравнениях}
\begin{gather*}
     \dot X_t = a_{1} X_t + a_{0}    + \left( c_{10}
     +\sss_{r=1}^{n_x} c_{1, n_y+r} X_{r}\right)V_1\,,\\
     V_1 = \dot W_1,\enskip X_{t_0}=X_0\,,
     \end{gather*}

     \vspace*{-12pt}

     \noindent
     \begin{multline*}
     \dot Y_t = b Y_t + b_{1} X_t + b_{0} + {}\\
     {}+
     \left( c_{20} +\sss_{r=1}^{n_y} c_{2r} Y_{r}
     +\sss_{r=1}^{n_x} c_{2, n_y+r} X_{r}\right)V_2\,,
          \end{multline*}

          \vspace*{-6pt}

          \noindent
          $$
     V_2 =\dot W_2,\enskip Y_{t_0}=Y_0\,,
$$
\textit{где $W_1=W_1(t)$, $W_2\hm=W_2(t)$~--- независимые процессы
с~независимыми приращениями, обладают конечными вероятностными
моментами второго порядка. Тогда уравнения ЛУОЭ имеют вид:}
\begin{align}
\dot{\hx_t} &=a_{1}(t+\Delta )\hx_t + a_{0} (t+\Delta) +{}\notag\\
&\hspace*{-7mm}{}+\beta_t \lk \dot Y_t -(b Y_t + b_{1} \varepsilon_t^{-1}\hx_t + b_{0} -
b_{1} \varepsilon_t^{-1}h_t) \rk\,; \label{e3.2-sk}\\
\varepsilon_t &= u(t+\Delta ,t)\,,\label{e3.3-sk}
\end{align}
где $u(t,\tau)$~--- \textit{фундаментальная матрица уравнения:}
\begin{align}
\dot u_t &= a_1 (t) u_t\,;\notag %\label{e3.4-sk}
\\
h_t =h(t) &= \iii_t^{t+\Delta} u(t+\Delta,\tau) a_{0} (\tau)\, d\tau\,.\label{e3.5-sk}
\end{align}
\textit{Необходимые для вычисления матриц $\sigma_{22}$  и~$\beta_t$
находятся согласно теореме~$2.3$  для составного вектора
$\lk Y_t^{\mathrm{T}} X_t^{\mathrm{T}} \hx_t^{\mathrm{T}}\rk^{\mathrm{T}}$. Роль матриц~$c_{1r}$, $c_{2r}$ играют матрицы
$[0\, c_{1r}]$ и $[c_{2r}\, 0]$, а~матрица~$\nu$~--- диагональна.
При этом точность экстраполяции определяется путем интегрирования
следующего уравнения:}
\begin{multline*}
\dot R_t = a_{1} (t+\Delta) R_t + R_t a_{1} (t+\Delta)^{\mathrm{T}} -{}\\
{}-\beta_t \Biggl[ \left( c_{20} +\sss_{r=1}^{n_y+n_x} c_{2r} m_r\right) \nu_1
     \left( c_{20}^{\mathrm{T}} +\sss_{r=1}^{n_y+n_x} c_{2r}^{\mathrm{T}} m_r\right)+{}\\
{}+ \sss_{r=1}^{n_y+n_x} c_{2r}\nu_1 c_{2s}^{\mathrm{T}} k_{rs}\Biggr] \beta_t^{\mathrm{T}} + \left[
    \vphantom{\sss_{r=n_y+1}^{n_y+n_x}} c_{10}(t+\Delta) +{}\right.\\
     {}\left.+\sss_{r=n_y+1}^{n_y+n_x} c_{1r} (t+\Delta) m_r(t+\Delta)
     \right] \nu_1 (t+\Delta)\times{}\\
{}\times \lk c_{10} (t+\Delta)^{\mathrm{T}} +\sss_{r=n_y+1}^{n_y+n_x} c_{1r}
(t+\Delta)^{\mathrm{T}} m_r (t+\Delta)\rk+{}\\
{}+ \sss_{s=n_y+1}^{n_y+n_x} c_{1r}(t+\Delta) \nu_1 (t+\Delta) c_{1s}
(t+\Delta)^{\mathrm{T}} k_{rs}={}\\
{}= A_\Delta R_t+ R_t A_\Delta^{\mathrm{T}} + R_t B_\Delta R_t + C_\Delta\,, %\label{e3.6-sk}
\end{multline*}
\textit{причем для обеспечения равномерной асимптотической устойчивости
достаточно выполнения условий}
$$
0< \alpha_1 I_n \le \mathcal{W}_\Delta (t-t^\prime, t) \le \alpha_2 I_n\,, %\eqno(3.7)
$$
\textit{где}
    $$
    \mathcal{W}_\Delta (t-t^\prime, t) = \iii_{t-t^\prime}^t u_\Delta (t, \tau)
    C(\tau) u_\Delta (t,\tau)^{\mathrm{T}} d\tau\,. %\eqno(3.8)
    $$
\textit{Здесь $u_\Delta (t,\tau)$~--- фундаментальная матрица для уравнения
$\dot Z \hm= A_\Delta Z$.}

Из~(\ref{e3.2-sk}) видно, что ЛУОЭ можно представить в~виде последовательного
ЛУОФ, усилителя с~коэффициентом усиления  $\varepsilon_t$~(\ref{e3.3-sk})
и~сумматора, вводящего неслучайное слагаемое~$h_t$~(\ref{e3.5-sk}).
Найденный ЛУОЭ оптимален в~классе всех линейных экстраполяторов~[1--3].

\vspace*{-6pt}

\section{Пример}

\vspace*{-2pt}

Рассмотрим случай, когда  скалярные уравнения~(\ref{e2.1-sk}) и~(\ref{e2.2-sk}) содержат
независимые белые шумы~$V_1$ и~$V_2$:

\noindent
\begin{gather*}
    \dot X_t = a Y_t + a_{1} X_t +( c_{11}X_t+ c_{12}Y_t ) V_1\,,\\
    \dot Y_t = b Y_t + b_{1} X_t +( c_{21}X_t+ c_{22}Y_t ) V_2\,,\\
\nu =\begin{bmatrix}
\nu_{1}&0\\
0&\nu_{2}\end{bmatrix}\,. %\label{e4.2-sk}
\end{gather*}
Обратим внимание на то, что  $c_{1r}$ и~$c_{2r}$ здесь не те, что
в~уравнениях~(\ref{e2.1-sk}). Они представляют собой соответственно первые
и~вторые элементы мат\-риц-строк, на которые умножается вектор $\left[V_1\ V_2\right]^{\mathrm{T}}$
в~(\ref{e2.1-sk}). Для простоты оставляем для них обозначения~$c_{1r}$
и~$c_{2r}$. Тогда имеем:
  \begin{align*}
  \beta_t&= \sigma_{22}^{-1} b_{1}R_t\,; %\label{e4.3-sk}
  \\
\sigma_{22} &= \nu_2 (c_{20} + c_{21} m_1+ c_{22} m_2)^2 + {}\\
&\hspace*{5mm}{}+
\nu_2  (c_{21}^2 k_{11} +  2c_{22} c_{21} k_{21}+ c_{22}^2k_{22})\,. %\label{e4.4-sk}
\end{align*}

Уравнения~(\ref{e2.11-sk}), (\ref{e2.12-sk}) и (\ref{e2.8-sk}),
определяющие  $m_1$, $m_2$, $k_{11}$, $k_{12}$, $k_{21}$, $k_{22}$ и~$R_t$,
имеют  сле\-ду\-ющий вид:
\begin{equation}
\dot m_1 = a_1 m_1 + a m_2\,;\enskip \dot m_2 = b_1 m_1 + b m_2 \,;\label{e4.5-sk}
\end{equation}

\vspace*{-9pt}

\begin{equation}
\left.
\begin{array}{rl}
\dot k_{11} &= 2(a_1 k_{11} + a k_{12}) +{}\\[6pt]
&\hspace*{4mm}{}+\nu_1 (c_{11}m_1 + c_{12} m_2+ c_{10})^2+{}\\[6pt]
 &\hspace*{4mm}{}+\nu_1(c_{11}^2k_{11} +  2c_{12} c_{11} k_{12}+ c_{12}^2 k_{22})\,;\\[9pt]
\dot k_{12} &= (a_1 +b) k_{12} + b_{1} k_{11} +a k_{22}\,; \\[9pt]
\dot k_{22} &= 2(b_1k_{12}+b k_{22}) +{}\\[6pt]
&\hspace*{4mm}{}+\nu_2( c_{21} m_1 + c_{22} m_2 +
c_{20})^2 +{}\\[6pt]
 &\hspace*{4mm}{}+ \nu_2 (c_{21}^2 k_{11} +2 c_{22} c_{21} k_{12} + c_{22}^2 k_{22})\,,
 \end{array}
 \right\}
 \label{e4.6-sk}
 \end{equation}

\vspace*{-12pt}

\noindent
\begin{multline}
\dot R_t = 2 a_{1} R_t -\sigma_{22}^{-1}b_{1}^2 R_t^2 +
\nu_1 (c_{11} m_1+ c_{12} m_2+c_{10} )^2+{}\\
{}+\nu_1(c_{11}^2k_{11} +  2c_{11} c_{12} k_{12}+ c_{12}^2 k_{22})\,.\label{e4.7-sk}
\end{multline}

Согласно~(\ref{e2.16-sk}) $\bar \sigma_{11}$, входящая в~условия
устойчивости ЛУОФ (теорема~3.1), определяется формулой
\begin{multline*}
\bar \sigma_{11} =\sigma_{11} =\nu_1 (c_{11} m_1 + c_{12} m_2 + c_{10} )+{}\\
{}+ \nu_1 (c_{11}^2 k_{11} + 2 c_{11} c_{12} k_{12} + c_{12}^2 k_{22})\,. %\label{e4.8-sk}
\end{multline*}

При $a=a_0\hm=c_{12}\hm=0$, $ a_1\hm=const$ ЛУОЭ пред\-став\-ля\-ет собой
последовательное соединение ЛУОФ и~усилителя с~коэффициентом
$\varepsilon_t \hm=\exp (a_1\Delta)$.

Приравнивая правые части уравнений~(\ref{e4.5-sk})--(\ref{e4.7-sk})
нулю, найдем условия для синтеза стационарных ЛУОФ и~ЛУОЭ с~постоянными
$\beta_t \hm=\beta^*$, $R_t\hm=R^*$, $\varepsilon_t\hm=\varepsilon^*$:
\begin{gather*}
m_1^* =0\,;\enskip m_2^* = 0\,; %\eqno(4.9)$$
\\
2(a_1 k_{11}^* + a k_{12}^*) + {}\hspace*{50mm}\\
\hspace*{13mm}{}+\nu_1^*(c_{11}^2 k_{11}^* + 2 c_{12} c_{11} k_{12}^* + c_{12}^2 k_{22}^*)=
0\,;\\
(a_1 + b) k_{12}^* + b_1 k_{11}^* + a k_{22}^* =0\,;
\end{gather*}

\noindent
\begin{gather*}
2(b_1k_{12}^* + b k_{22}^*) + {}\hspace*{50mm}\\
\hspace*{15mm}{}+\nu_2^* (c_{21}^2 k_{11}^* + 2 c_{22} c_{21} k_{12}^*+ c_{22}^2 k_{22}*)
=0\,; %,\eqno(4.10)$$
\\
   \hspace*{-40mm}2a_1 R^* - {\sigma_{22}^*}^{-1} b_1^2 {R^*}^2 +{}\\
\hspace*{17mm}{}+\nu_1^* (c_{11}^2 k_{11}^* + 2 c_{12} c_{11} k_{12}^* + c_{12}^2 k_{22}^*) =0\,; %,\eqno(4.11)$$
  \\
    \beta^* ={\sigma_{22}^*}^{-1} b_1 R^*\,; %\eqno(4.12)$$
    \\
    \bar \sigma_{11}^* = \nu_1^* ( c_{11}^2 k_{11}* + 2 c_{11} c_{12} k_{12}^* + c_{12}^2 k_{22}^*)\,; %\eqno(4.13)$$
    \\
    \sigma_{22}* = \nu_2^* (c_{21}^2 k_{11}^* + 2 c_{22} c_{21} k_{12}^* + c_{22}^2 k_{22}^*)\,. %\eqno(4.14)$$
    \end{gather*}

Для равномерной асимптотической устойчи\-вости ЛУОФ достаточно отрицательности
коэффициента
    $$
    A= a_1 -\fr{b_1^2 R^*}{\sigma_{22}^*} < 0\,. %\eqno(4.15)
    $$

\section{Заключение}

Получено обобщение известных результатов по\linebreak теории синтеза непрерывных равномерно
асимптотически устойчивых ЛУОФ и~ЛУОЭ для случая наблюдаемых дифференциальных СтС
с~мульти\-пликативными негауссовскими белыми шумами.  Приводится иллюстративный
пример.

Результаты могут быть использованы и~для \mbox{тео\-рии} синтеза дискретных ЛУОФ
и~ЛУОЭ для не\-пре\-рыв\-ных и~дискретных СтС, если
воспользоваться численными
методами приведения\linebreak нелинейных стохастических дифференциальных уравнений
к~разностным на основе обобщенной форму\-лы Ито~\cite{2-sk, 8-sk}.

Практический интерес представляет задача синтеза устойчивых ЛУОФ
и~ЛУОЭ по критериям устойчивости, отличным от критерия равномерной
асимптотической устойчивости, а~также для исследования вопросов
эквивалентности различных широкополосных шумов (в~том числе автокоррелированных).


{\small\frenchspacing
 {%\baselineskip=10.8pt
 \addcontentsline{toc}{section}{References}
 \begin{thebibliography}{9}

\bibitem{1-sk}
\Au{Пугачев В.\,С., Синицын И.\,Н.}
Стохастические дифференциальные системы. Анализ и фильтрация.~---
М.: Наука,  1990. 632~с.
(\Au{Pugachev V.\,S., Sinitsyn~I.\,N.}
Stochastic differential systems. Analysis and filtering.~---
Chichester, New York: Jonh Wiley, 1987. 549~p.)

\bibitem{2-sk}
\Au{Пугачев В.\,С., Синицын И.\,Н.}
Теория стохастических систем.~--- М.: Логос,  2000.  1000~с.
(Stochastic systems. Theory and applications.~--- Singapore: World Scientific,
2001. 908~p.)

\bibitem{3-sk}
\Au{Синицын И.\,Н.} Фильтры Калмана и~Пугачева.~--- 2-е изд.~---
М.: Логос, 2007. 776~с.



\bibitem{5-sk}
\Au{Корепанов Э.\,Р.} Стохастические информационные технологии
на основе фильтров Пугачева~// Информатика и~её применения, 2011. Т.~5.
Вып.~2. С.~36--57.

\bibitem{4-sk}
\Au{Синицын И.\,Н., Синицын В.\,И.}
Лекции по нормальной и эллипсоидальной аппроксимации в~стохастических системах.~---
М.: ТОРУС ПРЕСС, 2013. 476~с.

\bibitem{6-sk}
\Au{Kalman R.} A~new approach to linear filtering and prediction problems~//
J.~Basic Eng. (ASME Trans.), 1960. Vol.~82D. P.~35--45.

\bibitem{7-sk}
\Au{Ройтенберг Я.\,Н.} Автоматическое управление.~---
3-е изд., перераб. и~доп.~--- М.: Наука, 1992. 576~с.

\bibitem{8-sk}
\Au{Синицын И.\,Н.}
Параметрическое статистическое и~аналитическое моделирование распределений
в~нелинейных стохастических системах на многообразиях~// Информатика
и~её применения, 2013. Т.~7. Вып.~2. С.~4--16.
 \end{thebibliography}

 }
 }

\end{multicols}

\vspace*{-3pt}

\hfill{\small\textit{Поступила в редакцию 22.09.14}}

%\newpage

\vspace*{12pt}

\hrule

\vspace*{2pt}

\hrule

%\vspace*{12pt}

\def\tit{STABLE LINEAR CONDITIONALLY OPTIMAL
FILTERS AND~EXTRAPOLATORS FOR~STOCHASTIC SYSTEMS
WITH~MULTIPLICATIVE NOISES}

\def\titkol{Stable linear conditionally optimal
filters and~extrapolators for~stochastic systems
with~multiplicative noises}

\def\aut{I.\,N.~Sinitsyn and E.\,R.~Korepanov}

\def\autkol{I.\,N.~Sinitsyn and E.\,R.~Korepanov}

\titel{\tit}{\aut}{\autkol}{\titkol}

\vspace*{-9pt}

 \noindent
Institute of Informatics Problems, Russian Academy of Sciences,
44-2 Vavilov Str., Moscow 119333, Russian Federation


\def\leftfootline{\small{\textbf{\thepage}
\hfill INFORMATIKA I EE PRIMENENIYA~--- INFORMATICS AND
APPLICATIONS\ \ \ 2015\ \ \ volume~9\ \ \ issue\ 1}
}%
 \def\rightfootline{\small{INFORMATIKA I EE PRIMENENIYA~---
INFORMATICS AND APPLICATIONS\ \ \ 2015\ \ \ volume~9\ \ \ issue\ 1
\hfill \textbf{\thepage}}}

\vspace*{3pt}

\Abste{The applied  theory of analytical synthesis of linear
conditionally optimal filters and extrapolators in linear differential
stochastic systems with white multiplicative non-Gaussian noises
is presented. Efficient criteria of unique asymptotic stability
of conditionally optimal filters and extrapolators are formulated in terms
of special positive definite integral forms and unique boundedness of
controllability and observability matrices. White noises
are assumed to be derivatives of additive and multiplicative non-Gausisan
arbitrary stochastic processes with independent increments.
    An illustrative example is given. Some generalizations are discussed.}


\KWE{accuracy and unique asymptotic stability of filters;
differential stochastic systems;
linear conditionally optimal filters and extrapolators;
multiplicative white noises;
Riccati equation}

\DOI{10.14357/19922264150106}

%\Ack
%\noindent




%\vspace*{3pt}

  \begin{multicols}{2}

\renewcommand{\bibname}{\protect\rmfamily References}
%\renewcommand{\bibname}{\large\protect\rm References}



{\small\frenchspacing
 {%\baselineskip=10.8pt
 \addcontentsline{toc}{section}{References}
 \begin{thebibliography}{9}
\bibitem{1-sk-1}
\Aue{Pugachev, V.\,S., and I.\,N.~Sinitsyn.} 1987.
\textit{Stochastic differential systems. Analysis and filtering}.
Chichester, New York: Jonh Wiley. 549~p.

\bibitem{2-sk-1}
\Aue{Pugachev, V.\,S., and I.\,N.~Sinitsyn } 2001.
\textit{Stochastic systems. Theory and applications}.
Singapore: World Scientific.  908~p.

\bibitem{3-sk-1}
\Aue{Sinitsyn, I.\,N.} 2007. \textit{Fil'try Kalmana i~Pugacheva}
[Kalman and Pugachev filters]. 2-e izd. Moscow: Logos.  776~s.

\bibitem{5-sk-1}
\Aue{Korepanov, E.\,R.} 2011. Stokhasticheskie informatsionnye tekhnologii
na osnove fil'trov Pugacheva [Stochastic informational technologies based on
Pugachev filters]. \textit{Informatika i ee Primeneniya}~---
\textit{Inform Appl}.   5(2):36--57.

\bibitem{4-sk-1}
\Aue{Sinitsyn, I.\,N., and V.\,I.~Sinitsyn} 2013.
Lektsii po normal'noy i~ellipsoidal'noy approksimatsii
v~sto\-kha\-sti\-che\-skikh sistemakh  [Lectures on normal and ellipsoidal
approximation of distributions in stochastic systems]. Moscow:
TORUS PRESS.  476~p.



\bibitem{6-sk-1}
\Aue{Kalman, R.} 1960. A~new approach to linear filtering and prediction problems.
\textit{J.~Basic Eng. (ASME Trans.)} 82D:35--45.

\bibitem{7-sk-1}
\Aue{Roytenberg, Ya.\,N.} 1992. \textit{Avtomaticheskoe upravlenie}
[Automatic control]. 3rd ed. Moscow: Nauka.
576~p.

\bibitem{8-sk-1}
\Aue{Sinitsyn, I.\,N.} 2013. Parametricheskoe statisticheskoe
i~analiticheskoe modelirovanie raspredeleniy v~nelineynykh stokhasticheskikh
sistemakh na mnogoobraziyakh
[Parametric statistical and analytical modeling of distributions in
nonlinear stochastic systems on manifolds].
\textit{Informatika i ee Primeneniya}~--- \textit{Inform Appl.} 7(2):4--16.
\end{thebibliography}

 }
 }

\end{multicols}

\vspace*{-3pt}

\hfill{\small\textit{Received September 22, 2014}}

%\vspace*{-18pt}

\Contr

\noindent
\textbf{Korepanov Eduard R.} (b.\ 1966)~--- Candidate of Science (PhD) in
technology, Head of Laboratory, Institute of Informatics Problems,
Russian Academy of Sciences, 44-2 Vavilov Str., Moscow 119333,
Russian Federation; Ekorepanov@ipiran.ru

\vspace*{3pt}

\noindent
\textbf{Sinitsyn Igor N.} (b.\ 1940)~---
Doctor of Science in technology, professor, Honored scientist of
Russian Federation, Head of Department, Institute of Informatics Problems,
Russian Academy of Sciences, 44-2 Vavilov Str., Moscow 119333,
Russian Federation; sinitsin@dol.ru
\label{end\stat}

\renewcommand{\bibname}{\protect\rm Литература}