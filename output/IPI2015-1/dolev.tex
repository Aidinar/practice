\renewcommand{\figurename}{\protect\bf Figure}
\renewcommand{\tablename}{\protect\bf Table}

\def\stat{dolev}


\def\tit{HEURISTIC CERTIFICATES VIA APPROXIMATIONS}

\def\titkol{Heuristic certificates via approximations}

\def\autkol{Sh.~Dolev  and M.~Kogan-Sadetsky}

\def\aut{Sh.~Dolev$^1$  and M.~Kogan-Sadetsky$^1$}

\titel{\tit}{\aut}{\autkol}{\titkol}

%{\renewcommand{\thefootnote}{\fnsymbol{footnote}}
%\footnotetext[1] {The work of first and second  authors is partially supported by the
%Program of Strategy development of Petrozavodsk State University in
%the framework of the research activity. The third author is a~%postdoctoral fellow with the Research Foundation-Flanders
%(FWO-Vlaanderen).}}

\renewcommand{\thefootnote}{\arabic{footnote}}
\footnotetext[1]{Department of Computer Science, Ben-Gurion
University of the Negev,
Beer-Sheva 84105, Israel}
%\footnotetext[2]{Department of Computer Science, Ben-Gurion
%University of the Negev, Israel, sadetsky@cs.bgu.ac.il}


\vspace*{6pt}

\def\leftfootline{\small{\textbf{\thepage}
\hfill INFORMATIKA I EE PRIMENENIYA~--- INFORMATICS AND APPLICATIONS\ \ \ 2015\ \ \ volume~9\ \ \ issue\ 1}
}%
 \def\rightfootline{\small{INFORMATIKA I EE PRIMENENIYA~--- INFORMATICS AND APPLICATIONS\ \ \ 2015\ \ \ volume~9\ \ \ issue\ 1
\hfill \textbf{\thepage}}}



\Abste{This paper suggests a~new framework in which the quality of a~(not necessarily optimal) heuristic solution is certified by
an approximation algorithm.
Namely, a~result of a~heuristic solution is accompanied by a~scale
obtained from an approximation algorithm. The creation of a~scale is
efficient while getting a~solution from an approximation algorithm
is usually concerned with long calculation relatively to heuristics
approach. On the other hand, a~result obtained by heuristics without
scale might be useless. The criteria for choosing an
approximation scheme for producing a~scale have been investigated. To obtain a~scale in
practice, not only approximations have been examined  by their
{asymptotic} behavior but also relations as a~function of an
input size of a~given problem. For study case only,
heuristic and approximation algorithms for the {SINGLE
KNAPSACK}, {MAX 3-SAT}, and {MAXIMUM BOUNDED THREE-DIMENSIONAL
MATCHING} (MB3DM) NP-hard problems have been examined. The
certificates for the heuristic runs have been obtained
by using fitting approximations.}

\KWE{heuristics; approximation algorithm; optimal
solution; approximation preserving reducibility}

\DOI{10.14357/19922264150103}

%\vspace*{6pt}


\vskip 14pt plus 9pt minus 6pt

      \thispagestyle{myheadings}

      \begin{multicols}{2}

                  \label{st\stat}

\section{Introduction}

\noindent
Many important optimization problems are NP-hard. This rules out the possibility of finding in polynomial
time an optimal solution (for those that are in NP,
unless $P=\mathrm{NP}$). Instead, commonly, artificial
intelligence (AI) methods are used to try to cope with instances of these
problems. Namely, heuristic solutions are used to get a~near to optimal
solution for a~given instance of such problems. However, heuristic
may obtain a~suboptimal solution such as a~local maxima
rather than the absolute maximum. Evaluation of the result obtained
from the heuristic is the complicated task that is addressed in this
work. The present authors suggest using approximation algorithms to obtain a~scale
that may be used to certify heuristic results.

\noindent
\textbf{Heuristic solutions.} Heuristic solutions explore
various states and solutions
for a~given problem in a~short time\footnote[2]{The authors do not
refer to
the $A^*$ heuristic search, in which some optimizations for the exponential
time exhaustive search are introduced for obtaining the
optimal solution at the price of worst case exponential time.}.
There are well-known heuristics such as:
Hill Climbing search that uses local
estimates of the distance to the goal and expands the node with the
smallest estimate; Branch and Bound search that uses the costs of the
already formed partial paths and expands the partial path with the
minimum cost, and {Genetic Algorithms (GA)}.
In particular,  GA  will be explored
as an example for a~heuristic. Genetic algorothms were invented by John
Holland~\cite{H1971} and others, finding their inspiration in the
evolutionary process occurring in nature. A population of
individuals should collectively adapt to conditions of some
environment. In order to face this, the reproduction and survival of
individuals are promoted by the elimination of useless and harmful
features and by gratifying useful behavior.

The dynamics of the natural evolutionary process are complex and
mostly unknown. Nevertheless, the artificial optimization approach
is feasible if in comparison to another approaches, a~good enough
solution can be obtained. One of such different approaches is the
solution that uses approximation algorithms rather than
heuristics.

\noindent
\textbf{Approximation algorithms.} A way to cope with
optimization tasks that do not have a~(known) polynomial solution is
by devising an algorithm that finds a~solution to the problem in a~reasonable time, such that the solution is suboptimal with a~known
ratio or bound from the (unknown) optimal solution. On one hand,
heuristics may perform better than approximations in practice,
namely, obtain a~better solution with the same processing effort, but
on the other hand, heuristics are associated with uncertainty
regarding the quality of
the obtained result.

\noindent
\textbf{The authors' contribution.} The authors suggest a~new
framework in which the quality of a~solution that is based on
heuristics is certified by an approximate algorithm. In this
framework, a~heuristic solution is accompanied by a~scale obtained
from the approximation algorithm. The usefulness of
the framework is demonstrated by presenting criteria for choosing a~fitting
approximation scheme for producing a~scale. The
{SINGLE KNAPSACK}, {MAX 3-SAT}, and {MB3DM} NP-hard problems
are examined.

\noindent
\textbf{Using polynomial time approximation schemes}. In some cases,
it is possible to use
polynomial time approximation schemes (PTAS). Let start by
demonstrating the framework for such a~case, the case of the
SINGLE KNAPSACK problem. To solve instances of SINGLE KNAPSACK,
galib246~\cite{GAlib} library for
genetic algorithms has been used. In
order to certificate the solution obtained from the genetic
algorithm, the approximation algorithm CKPP
(Cardinality Constrained Knapsack Problem)~\cite{KPP99}, has been used. In
the present authors'
experiment~\cite{link-to-the-sofware-cite}, each item has weight and
profit; the restricted total weight was 300~weight units and the
optimization task was to maximize the profit while respecting the
weight bound. The profit obtained by the genetic algorithm was~577~profit
units (with weight of~293~weight units). The total profit
of all the given items was much larger $-$ 1,345 profit units. Thus,
the authors had no clue whether the obtained profit is a~local maxima that is
way beyond the possible maximum profit. {CKPP} approximation
certified the heuristic solution as~0.88 of the possible maximum
profit. Since {CKPP} is {PTAS}, it can be tuned to obtain
a~refined scale using more and more computation time. Using the
performance ratio of~0.75,  the profit of~493~profit
units (with weight of 293~weight units) running in a~few seconds was obtained.

\noindent
\textbf{Using approximation preserving reducibility.}
Obtaining certificates as a~part of the practice in using heuristics
may require the design of many approximations, which is not a~simple
task. Luckily, in some cases, approximation preserving reductions
may be used to obtain an approximation algorithm from variant of a~problem in hand to a~known approximation algorithm. Next,
the way such a~reduction is chosen and used will be demonstrated. The
experimental demonstration for this example obtained a~0.64
certificate for the {GA} result~\cite{link-to-the-sofware-cite}.

\noindent
\textbf{MAX 3-SAT.} An instance of the {MAX 3-SAT}
problem is a~predicate for which each clause is of not more than~3~literals.
A~solution is a~truth assignment for the predicate
variables, such that the number of satisfied clauses is the maximum
possible. One may evaluate a~given assignment for the predicate
variables by the number of clauses it satisfies. Once again,
 galib246 library~\cite{GAlib} was used to solve instances of the {MAX
3-SAT} problem. In their experimental demonstration, the present authors
used a~predicate that consisted of~100~clauses. The heuristic algorithm found
(in a~few seconds) an assignment for the variables that satisfies~58~clauses.
Again, the authors had no clue whether the solution was only~0.58
of the optima.

Some maximization problems have trivial bound on solution size which
is related to the size of the input. This bound gives one a~preliminary Certificate of the heuristic solution. This certificate
maybe good enough if the heuristic solution is at least, say, 0.8
of the input size. This implies that an optimal solution for the
given problem is at least~0.8 of the input size, which is not a~common case. For instance, in the case of the {MAX 3-SAT}
problem, the trivial upper bound is obvious since all predicate's
clauses may be satisfiable by some truth assignment. But it is
possible that any truth assignment can satisfy near to half of
predicate's clauses, and in this case, the trivial upper bound gives
a poor certificate to a~heuristic solution. For some problems, it is
not easy to determine a~trivial upper bound. For example, for the
{SINGLE KNAPSACK} problem, the common case is that only a~small
part of the items can be put into the knapsack.

To demonstrate the way reductions are chosen,
two examples are presented: {MAX 3-SAT} and {MB3DM}.

For the first example,  two reductions
from {MAX 3-SAT} were used: one to {MAX2-SAT} and the second to
{MAX INDEPENDENT SET-B}. In this case, both approximations are not
{PTAS} and, therefore, only a~single scale can be obtained for
a~given instance. It turned out that the scale obtained for the
particular instance, by using the reduction to {MAX 2-SAT},
added no information beyond the trivial bound of 100 clauses, while
the reduction to {MAX INDEPENDENT SET-B} resulted in a~refined
scale of~92~clauses.

Also, two reductions were used from {MB3DM}: one to {
minimization of sum of squared machine loads} and the second to
{STAR-GCA-SIMPLE}. In this case, the reductions once again, both gave
a single scale. It will be shown that according to the all comparable
criteria for choosing reduction $-$ quality of approximation
preserve, quality of approximation algorithm of a~destination
problem, inflation of input, and the complexity of reduction (that
would be defined later in the paper), the reduction to {
MINIMIZATION OF SUM OF SQUARED MACHINE LOADS} is a~better choice for
producing a~scale and a~certificate.
%\end{itemize}

\noindent
\textbf{Choosing approximation preserving reductions.} There
are four main criteria that are suggested using for choosing a~specific
reduction algorithm from a~given problem. The first criterion is the
quality of approximation preserve. This is a~reduction preservation
ratio, which is implied by the parameters of reduction constrains.
The second criterion is the quality of approximation algorithm of
a~destination problem. This means the performance ratio of (best)
known approximate algorithm for a~destination problem. The third
criterion is the inflation of input of a~source problem when it is
translated to input of a~destination problem. The last criterion is
the complexity of reduction functions. This parameter represents
degree of ease of a~reduction usage and also contributes to the
general performance of a~reduction.

\noindent
\textbf{Paper organization.} In section~\ref{s:demo},
the framework will be presented and the way approximation for the
{SINGLE KNAPSACK} problem is used will be demonstarated
to certify a~solution obtained
by a~genetic algorithm. An overview on Genetic Algorithms (the authors'
choice for examining heuristics) appears in section~\ref{s:GA}.
Then, in section~\ref{s:AP}, a~short overview on
approximation algorithms and approximation preserving reductions will
be given including: PTAS, L-reduction, and AP-reduction. Section~\ref{s:ChoosRed} discusses criteria for choosing an approximation
preserving reduction to obtain the best scale. Section~\ref{s:Concluding} concludes the paper.

\vspace*{-6pt}

\section{The Framework and the SINGLE KNAPSACK Problem Example}
\label{s:demo}

\vspace*{-4pt}

\noindent
\textbf{The framework.} First, let define some concepts that
will be used in the paper:
\begin{itemize}
\item \textit{Input}: an instance of hard optimization problem
for which there is no (efficient) exact polynomial algorithm or good
polynomial approximation.
\item \textit{Heuristic certificate}. Given a~solution of some
heuristic algorithm, the quality of this
solution will be determined, i.\,e.,  a~(closest) relation between the
heuristic and the optimal
solutions of a~given problem instance will be found.
\item \textit{Solution scale}. To define a~solution scale,
a~problem solution upper
bound obtained from approximation algorithm is used.
This scale is used to measure the quality of the heuristic solution.
\item $\displaystyle
\mathrm{Certificate}$\linebreak
$=\fr{\mathrm{Heuristic\
solution}}{\mathrm{upper\ bound\ (obtained\ from\
approximations)}}.$
\end{itemize}

The steps defined by the framework are as follows:
\begin{itemize}
\item Choose heuristics, for example, GA
to solve the problem instance.
\item Find (at least) one approximation for the problem.
Sometimes, approximation preserving reductions should be used.
\item Choose, using the criteria that will be defined later in
the
paper, the approximation that leads to the best solution scale.
\item Execute the heuristic and the approximation algorithms
(possibly, in parallel) and compute the certificate according to the
results (possibly, repeatedly, in particular, in the case of
PTAS and {FULLY PTAS}, until the required certificated result is
obtained).
\end{itemize}

\noindent
\textbf{SINGLE KNAPSACK problem example.} Let consider the
{SINGLE KNAPSACK} problem. Given items of different profits and
weights, find the most valuable set of items that fit in a~knapsack
of fixed weight bound. This problem is chosen since it has
PTAS, i.\,e., the algorithm that for each $\epsilon > 0$
produces a~solution that is within~$\epsilon$~factor of being optimal. The
approximation\linebreak\vspace*{-12pt}
%\begin{table*}

{\small %tabl1
\begin{center}
{{\tablename~1}\ \ \small{SINGLE KNAPSACK approximation solution}}\\[6pt]
\tabcolsep=10pt
\begin{tabular}{|c|c|c|c|}
\hline
$\epsilon$ & \tabcolsep=0pt\begin{tabular}{c}Approximation\\ ratio\end{tabular} &
\tabcolsep=0pt\begin{tabular}{c}Solution\\ weight\end{tabular}&
\tabcolsep=0pt\begin{tabular}{c}Solution\\ profit\end{tabular} \\
\hline
0.4 & 2/3 & 293 & 493 \\
\hline
\end{tabular}
\end{center}}

%\vspace*{2pt}

\addtocounter{table}{1}

%\begin{table*}
{\small %tabl2
\begin{center}
{{\tablename~2}\ \ \small{Genetic algorithm solution}}\\[6pt]
\tabcolsep=4pt
\begin{tabular}{|c|c|c|c|c|}
\hline
\tabcolsep=0pt\begin{tabular}{c}Generation\\ id\end{tabular}&
\tabcolsep=0pt\begin{tabular}{c}Solution\\ weight\end{tabular}&
\tabcolsep=0pt\begin{tabular}{c}Solution\\ profit\end{tabular}  &
\tabcolsep=0pt\begin{tabular}{c}Optimal\\ solution\\ upper\\ bound \end{tabular}&
\tabcolsep=0pt\begin{tabular}{c}Heuristic\\ certificate\end{tabular}\\
\hline
12 & 293 & 577 & 740 & 0.78 \\
\hline
\end{tabular}
\end{center}
%\vspace*{-3pt}
}
%\end{table*}

\vspace*{6pt}

\addtocounter{table}{1}


\noindent
 algorithm used is the algorithm CKPP~\cite{KPP99}. Its running time is
$O\left(n^{\lceil1/\varepsilon\rceil-2}\right)$ for $\varepsilon < 1/2$, and
its guaranteed approximation ratio is
$\left({\lceil1/\varepsilon\rceil-1}\right)/{\lceil1/\varepsilon\rceil}$. The
test problem instance consists of 30~items, each with integer profit
and weight, and of a~knapsack with capacity~300. The total profit
of all items is~1,345~profit units and the total weight of all
items is~1,712~weight units. The run was performed with $\varepsilon
= 0.4$ and returns in a~few seconds with the solution presented in
Table~1.




%\end{table*}

The experimental runs of the GA heuristics were
performed by using the galib246 library~\cite{GAlib}, with a~mutation rate $p_{m} = 0.001$, crossover rate $p_{c} = 0.6$, and a~one-point crossover operator~\cite{link-to-the-sofware-cite}. The
initial population consists of a~set of maximal (in respect to
inclusion) feasible solutions that were obtained by the
approximation algorithm during its execution. Therefore, the initial
population set contains a~subpart of an optimal solution assisting
the GA to easily find a~near-to-optimal solution.
Thus, the use of approximation may result in an additional advantage
for the GA. Several runs of the GA
were performed, each of~50~generations and takes a~few seconds; the
best result of these runs is presented in Table~2.



 The quality of the solution that is based on
heuristics by creating a~scale for the problem using the approximate
algorithm solution has been determined. The upper bound of the
solution  can be easily calculated using the approximation algorithm ratio:

\vspace*{-5pt}

\noindent
\begin{multline*}
\mathrm{upper\ bound} =
\fr{\lceil1/\varepsilon\rceil}{\lceil1/\varepsilon\rceil-1}
\\
{}\times \mathrm{approximation\ algorithm\ solution\ profit}.
\end{multline*}

\vspace*{-5pt}

\noindent
 The
performance ratio of the heuristic run is ${\mathrm{heuristic
solution}}/{\mathrm{upper bound}}$, which equals~0.78 in the considered case.
Thus, 0.78~is the heuristic certificate obtained for the heuristic
results.


\vspace*{-8pt}

\section{Heuristic Choice, Genetic Algorithms (based on~\cite{Sipper1996})} \label{s:GA}

\noindent
A~GA is an iterative
search technique which simulates a~population of individuals. The
search space consists of candidate solutions to the given problem,
each one is encoded by a~finite string of symbols called
\textit{gnome}. The GA is especially useful for the
problems with search space that is too large to be exhaustively
explored. Mostly, solutions are represented as binary strings, but
other encodings are also possible, e.\,g., character-based encoding,
real-value encoding, tree representation, etc.

The evolution usually starts from a~population of individuals which
is randomly or heuristically generated. Then, it proceeds by
iterative generations $-$ the fitness of every individual in the
current population is evaluated according to some predefined
objective function, referred to as the \textit{fitness function}.
Multiple individuals are selected from the current population based
on their fitness. One of the simplest selection procedures is
\textit{fitness-proportionate selection}, where individuals are
selected with a~probability proportional to their relative fitness.
This procedure ensures that the expected number of selections of
some individual is proportional to its expediency for the
population. This policy ensures that high-fitness individuals get
higher chance to be promoted and low-fitness individuals are
commonly eliminated.

In order to produce new search points, one cannot use a~selection
only. For this, genetically-inspired operators like
\textit{crossover} and \textit{mutation} were used. Selected individuals are
modified, recombined, and possibly mutated to form a~new population.
Crossover is performed as follows: two individuals called
$\textit{parents}$ are selected and parts of their gnomes are
exchanged between them with probability $p_{\mathrm{cross}}$. This forms two
new individuals, called $\textit{offsprings}$. The simple example of
crossover is when substrings are exchanged after a~randomly selected
crossover point. The mutation is used to maintain genetic diversity
from one generation to the next. It prevents impulsive junction to
local optima rather than to optimal solution. It is performed by
flipping random bits with some usually small probability $p_{\mathrm{mut}}$.
The new population is then used in the next iteration of the
algorithm. Genetic algorithms are stochastic iterative processes
that are not guaranteed to converge. There are a~variety of
termination conditions, e.\,g., a~solution is found that satisfies
minimum criteria, fixed number of generations reached, computational
time limitations, etc. In other words, the process stops when some
acceptable fitness level is reached.

\section{Approximation Algorithms and~Approximation Preserving
Reductions} \label{s:AP}

\noindent
An approximation algorithm for an
optimization problem generates feasible, but not necessarily
optimal, solutions. Since there are no polynomial-time algorithms to
get optimal solutions for {NP}-hard problems (unless $P=\mathrm{NP}$),
usually, nonoptimal solutions are accepted which can be
obtained in polynomial time. Unlike heuristic, the term
\textit{approximation algorithm} implies some proven worst case
bound on the solution quality in measurable time.

The approximation properties of different problems vary a~great
deal. Some problems cannot be approximated even with a~factor of
$n$, for instance, a~problem of finding a~maximum clique in a~graph.
Some other problems can be approximated only with the factor of
$O(\log n)$ like problems related to graph separators, or with a~predefined constant factor like {MAX 3-SAT} problem. Only a~small quantity of problems can be approximated with any arbitrary
constant factor, i.\,e., have {PTAS}. One would prefer an
approximation algorithm to run in polynomial time and have close to~1 \textit{approximation ratio}, i.\,e., the worst-case ratio between
the solution obtained by the approximation algorithm and the optimal
solution. But as the solution given by approximation gets closer to
the optimal solution, the time cost gets closer to exponential.
Therefore, heuristic algorithm may be preferred
to get near-optimal solution with potentially better approximation ratio.

\noindent
\textbf{Polynomial time approximation schemes.} {PTAS} for an optimization problem~$A$ is
a~polynomial-time algorithm which input is the problem instance and
$\epsilon > 0$ and output is the solution that approximates a~given
problem within the factor of~$\epsilon$. The run time of an
algorithm \mbox{PTAS} may depend not just on an input size of a~given
problem, but also on~$\epsilon$.

\smallskip

\noindent
\textbf{Approximation preserving reductions (based on~\cite{TrevisanLuca}).}

\smallskip

\noindent
\textbf{Definition~1}~\cite{TrevisanLuca}.
Given a~class of functions~$F$, an NP-optimization ({NPO})
problem~$A$  that belongs to the class {F-APX} (an abbreviation
of \textit{approximable}) is defined if an r(n)-approximate algorithm~$T$ for~$A$
exists, for some function $r \in F$. In particular, {APX},
$\log\,${APX}, poly{APX}, and exp{APX} will denote the class
{F-APX} with $F$ equal to the set $O(1)$, to the set $O(\log
n)$, to the set $O(n^{\log n})$, and to the set $O(2^{n^{O(\log
n)}})$, respectively.

\smallskip

Given an instance~$I$ of an {NPO} problem,
$|I|$ is used to denote the length of~$I$ and $\mathrm{OPT}\,(I)$ to denote the optimum
value for this instance. For any solution~$S$ to~$I$, the
\textit{objective value} of the solution is denoted by $c(I,S)$.

\smallskip

\noindent
\textbf{Definition 2}~\cite{TrevisanLuca}.
 Given a~solution~$S$ to an instance~$I$ of an
NPO problem, the relative error of~$S$ with respect
to~$I$ is defined as
$$
\varepsilon(I,S):=\max\left\{\fr{c(I,S)}{\mathrm{OPT}\left(I\right)},\,\fr{\mathrm{OPT}\left(I\right)}{c(I,S)}\right\}
$$
The above definition is applied to maximization and
minimization problems as well.

\smallskip

\noindent
\textbf{Definition 3}~\cite{TrevisanLuca}.
An approximation algorithm~$A$ for an
optimization problem~$\Pi$ has performance ratio $R(n)$ if, given an
instance~$I$ of~$\Pi$ with $|I|=n$, the solution $A(I)$ satisfies

\noindent
$$
\max\left\{\fr{c(I,S)}{\mathrm{OPT}\left(I\right)},\,\fr{\mathrm{OPT}\left(I\right)}{c(I,S)}\right\}
\leq R(n)\,.
$$


 A solution of value within a~multiplicative factor~$r$ of
the optimal value is referred to as an {r-approximation}.

A reduction from a~problem~$A$ to a~problem~$B$ specifies some
procedure to solve~$A$ by means of an algorithm solving~$B$. In
content of approximation, the reduction should guarantee that an
approximate solution for~$B$ can be used to obtain an approximate
solution for~$A$. The reduction functions~$f$ and~$g$ are used where~$f$
maps an instance of a~problem~$A$ to an instance of a~problem~$B$
and~$g$ maps an approximate solution of a~problem~$B$ into
a~feasible solution of a~problem~$A$.

\smallskip

\noindent
\textbf{Definition 4}~\cite{TrevisanLuca}. Let~$A$ and~$B$ be two {NPO}
problems, a~reduction template $(f,g)$ between them is a~tuple of polynomial
computable functions such that the following properties hold:
\begin{itemize}
\item for any $x \in I_{A}, f(x) \in I_{B}$;\\[-14pt]
\item for any $x \in I_{A}$, if  $\mathrm{sol}\left(x\right) \neq \emptyset$ then also
$\mathrm{sol}\left(f(x)\right) \neq \emptyset$; and\\[-14pt]
\item for any $y' \in \mathrm{sol}\left(f(x)\right),g(x,y') \in \mathrm{sol}\left(x\right)$.
\end{itemize}

\smallskip

Typically, polynomial time reductions do not preserve the
near-optimality of the solutions. Indeed, all {NP}-complete
problems are equally hard from the viewpoint of obtaining exact
solutions. However, from the viewpoint of obtaining near-optimal
solutions, they exhibit a~rich set of possibilities. An
approximation preserving reduction not only has to map instances of
a problem~$A$ to instances of a~problem~$B$, but it also has to be
able to obtain good solutions for~$A$ from good solutions for~$B$.
Approximation preserving reducibilities are defined by imposing some
relations between the performance ratios of~$y'$ and $g(x,y')$.
Several kinds of reducibilities may be found in literature:
\textit{Strict reducibility, L-reducibility, E-reducibility,
PTAS-reducibility, AP-reducibility}.
These reducibilities are identical with respect to the overall scheme but
differ essentially in the way they preserve approximability: they
range from the \textit{Strict reducibility} in which the error
cannot increase to the \textit{PTAS-reducibility} where there are
basically no restrictions. Below,  only two
reductions will be described in detail:
L-reduction and AP-reduction which are used later in this paper.

\smallskip

\noindent
\textbf{L-reducibility}~\cite{TrevisanLuca}. L-reduction
stands for \textit{linear reduction} because there is a~linear
blow-up in the relative approximation error. The L-reduction
enforces both optimal and approximation solutions of an instance~$I$
of~$A$ to be linearly related, respectively, to optimal and
approximation solutions of the instance~$I'$ of~$B$ to which it is
mapped.

Let $f$ and $g$ be polynomial-time reduction functions from
optimization problem~$A$ to optimization problem~$B$. We say that
$(f,g)$ is an $\textit{L-reduction}$ if there are constants
$\alpha,\beta > 0$ such that for each instance~$x$ of~$A$:
\begin{itemize}
\item the optima of $x$ and $f(x)$, $\mathrm{OPT}\left(x\right)$ and
$\mathrm{OPT}\left(f(x)\right)$, respectively,
satisfy $\mathrm{OPT}\left(f(x)\right) \leq \alpha \mathrm{OPT}\left(x\right)$; and
\item for any solution $y'$ of $f(x)$ with objective value~$c'$,
one can find in a~polynomial time a~solution $y = g(x,y')$ of $x$
with objective value $c$, so that $|\mathrm{OPT}\,(x)-c(y)| \leq
\beta |\mathrm{OPT}\left(f(x)\right)-c'(y')|$.
\end{itemize}

From the definition above, one gets that $E_A(x,g(x,y))\leq \alpha\beta
E_B(f(x),y)$. This inequality implies that if~$A$ is a~minimization
problem and an \mbox{r-approximate} algorithm for~$B$ exists, then
a~$(1+\alpha\beta (r$\linebreak $-1))$-approximate algorithm for~$A$ exists. The
difficulty of L-reducibility is, mainly, due to the fact that it does
not allow the function~\textit{g} to depend on~$\epsilon$: as a~consequence, this function is forced to map optimum solution into
optimum solution. The L-reducibility preserves membership in
PTAS but does not preserve membership in {APX} unless {P}\;=\;{NP} $\bigcap$ co-{NP}. This means that it cannot be
blindly used to obtain the existence of approximation algorithms via
reductions. The constant~$\beta$ will be usually~1.

\smallskip

\noindent
\textbf{AP-reducibility}~\cite{TrevisanLuca}. The
AP-reduction lets the functions~$f$ and~$g$ depend on the expected
performance ratio of the reduction. AP-reducibility preserves
approximation but not optimal solution. There are two different
constraints that are put on the computational time of~$g$ and~$f$:
the computation time should be polynomial for fixed values of the
performance ratio (to preserve membership in PTAS) and the
reduction should be efficient even when poor performance ratios are
required (to preserve membership in $\log$\;{APX} and poly{APX}).
Thus, the computation time should not increase when the performance
ratio decreases.

\smallskip

\noindent
\textbf{Definition 5}~\cite{TrevisanLuca}.
Let~$A$ and~$B$ be two {NPO} problems, with
instances $I_{A}$ and $I_{B}$, respectively. An extended reduction
template $(f,g)$ between them is an ordered
2-tuple of functions such that the following properties hold:
\begin{itemize}
\item for any $x \in I_{A}$ and for any rational $r >
1,f(x,r)$\linebreak $\in I_{B}$ and is computable in time $t_{f}(|x|,r)$.

Moreover, if $\mathrm{sol}\left(x\right) \neq \emptyset$ then also $\mathrm{sol}\left(f(x,r)\right) \neq
\emptyset$;
\item for any $y' \in \mathrm{sol}\left(f(x,r)\right), g(x,y',r) \in \mathrm{sol}\left(x\right)$ is
computable in time $t_{g}(|x|,|y|,r)$;
\item for any fixed $r$, both $t_{f}(\cdot,r)$ and
$t_{g}(\cdot,\cdot,r)$ are bounded by a~polynomial; and
\item for any fixed $n$ and $m$, both $t_{f}(n,\cdot)$ and
$t_{g}(n,m,\cdot)$ are nonincreasing functions.
\end{itemize}

\smallskip

Let~$A$ and~$B$ be two {NPO} problems. $A$~is said to be
\textit{AP-reducible} to $B$, denoted by $A\leq_{AP}B$, if an
extended template $(f,g)$ between $A$ and~$B$ exists and
holds~5~properties. The first four properties are of the extended template
above. They ensure that~$f$ and~$g$ reduction functions are
polynomial in time, when~$f$~maps instances of~$A$ into instances of~$B$
and~$g$~maps solutions for instances of~$B$ into solutions for
instances of~$A$. The fifth property implies the existence of a~positive constant~$\alpha$ such that for any $x \in I_{A}$, for any
rational $r > 1$, and for any $y \in \mathrm{sol}_{B}(f(x,r))$,
$R_{B}(f(x,r),y) \leq r$ implies $R_{A}(x,g(x,y,r)) \leq 1 +
\alpha(r-1)$. The triple $(f,g,\alpha)$ is said to be an
AP-reduction from~$A$ to~$B$.

\section{Choosing Approximation Preserving Reductions}
\label{s:ChoosRed}

\noindent
There are four main criteria that are proposed to
use when choosing a~specific reduction algorithm from a~given
problem. Unlike the traditional criteria,  constants and
not only asymptotic
functions are considered.
\begin{enumerate}[1.]
\item {\it Quality of approximation preserve.} Some of
the reductions preserve approximation by imposing a~linear relation
between the performance ratios of~$y'$ and $g(x,y')$. Others
preserve approximation by putting constrains also on the additive
errors of~$y'$ and $g(x,y')$. The reduction preservation ratio is
implied by the parameters of these constrains.
\item {\it Quality of approximation algorithm of a~destination
problem.} A performance ratio of the (best) known approximate
algorithm for a~destination problem.
\item {\it Inflation of input and the run time complexity of
a~destination problem.} This parameter is implicitly included in
a~quality of approximation preserve parameters. Besides, this
parameter impacts on a~complexity of the
required execution of the approximation algorithm of the destination
problem.
\item {\it Complexity of reduction functions~$f$ and~$g$.} This
parameter represents a~degree of ease of a~reduction use and also
contributes to the general performance of a~reduction.
\end{enumerate}

\subsection{Approximation preserve reductions of~MAX 3-SAT}

\noindent
Let examine two examples of approximation preserving reductions of
{MAX 3-SAT} NPO problem and compare them.

\smallskip

\noindent
\textbf{Example 1}.\
\textbf{MAX 3-SAT\;{\boldmath{$\leq_{L}$}}\;MAX~2-SAT}


\smallskip

\noindent
Let consider an instance of MAX 3-SAT $\varphi = \varphi_1$\linebreak $
\wedge \varphi_2 \wedge\cdots\wedge \varphi_k$ with $k$ clauses, each contains
at most 3~literals. The reduction~$f$~maps $\varphi$ to an instance~$\varphi'$ of MAX 2-SAT, clause by clause, based on the
following roles:
\begin{itemize}
\item if $\varphi_i$ has at most 2~variables, i.\,e., of
the form $(x^{1}_{i})$ or $(x^{1}_{i} \vee x^{2}_{i})$, then
$\varphi'_{i} =
\varphi_i$; and
\item if $\varphi_i$ has 3~variables, i.\,e., of the form
$(x^{1}_{i}\vee x^{2}_{i}\vee x^{3}_{i})$, then $\varphi'_{i} =
(x^{1}_{i}) \wedge (x^{2}_{i}) \wedge (x^{3}_{i}) \wedge (y_{i})
\wedge
  (\bar{x}^{1}_{i} \vee \bar{x}^{2}_{i})$\linebreak $ \wedge (\bar{x}^{1}_{i} \vee
\bar{x}^{3}_{i}) \wedge
  (\bar{x}^{2}_{i} \vee \bar{x}^{3}_{i}) \wedge (x^{1}_{i} \vee \bar{y}_{i})
\wedge (x^{2}_{i} \vee
  \bar{y}_{i}) \wedge (x^{3}_{i} \vee \bar{y}_{i})$, where~$y_i$ is the new
  variable.
  \end{itemize}

Suppose a~truth assignment~$\tau$ for~$\varphi$. If~$\varphi_i$ is
of the first form mentioned above and~$\varphi_i$ is satisfied by~$\tau$,
then, clearly,~$\varphi'_i$ is also satisfied by~$\tau$. If a~clause~$\varphi_i$ is of the second
form and~$\varphi_i$ is
satisfied by~$\tau$, it can be shown that~$\tau$ can be extended to
a~truth assignment~$\tau'$ for~$\varphi'_i$ which satisfies exactly
(and not more) seven clauses of~$\varphi'_i$. Let consider several
cases. In the first case, exactly one literal among $x^{1}_{i}$,
$x^{2}_{i}$, and $x^{3}_{i}$ is set to \textit{true}.
Then, by setting~$y_i$ to \textit{false}, let get an assignment that satisfies exactly
seven of ten clauses in~$\varphi'_i$. In the second case, exactly
two literals among $x^{1}_{i}$, $x^{2}_{i}$, and $x^{3}_{i}$ are set to
\textit{true}. Then, by setting $y_i$ indifferently to \textit{true}
or \textit{false}, one gets an assignment that satisfies exactly seven
of ten clauses in~$\varphi'_i$. In the third case, all literals
$x^{1}_{i}$, $x^{2}_{i}$, and $x^{3}_{i}$ are set to \textit{true}. Then, by
setting~$y_i$ to \textit{false}, one gets an assignment that satisfies
exactly seven of ten clauses in~$\varphi'_i$. Finally, if~$\varphi_i$ is not satisfied by~$\tau$, no truth assignment for~$y_i$ can satisfy more than six clauses of~$\varphi'_i$ while six
are guaranteed by setting~$y_i$ to \textit{false}.

\smallskip

\noindent
\textbf{Lemma 1}~\cite{AP2005}.
\textit{Given a~propositional formula in conjunctive normal form, at least
one half of its clauses can always be satisfied}. (Proof: try some
random assignment. If this does not satisfy half the clauses, then
its bitwise complement will.)


\smallskip

Denote the number of 3-literal clauses of~$\varphi$ by~$m$. This
implies that $\mathrm{opt}\,(\varphi') = 6m + \mathrm{opt}\,(\varphi)$. By using the lemma
above, one gets that $m \leq 2\mathrm{opt}\,(\varphi)$ and $\mathrm{opt}\,(\varphi') = 6m +
\mathrm{opt}\,(\varphi) \leq 13\mathrm{opt}\,(\varphi)$. This means that
$\mathrm{opt}\,(\varphi')
\leq 13\mathrm{opt}\,(\varphi)$ and $\alpha = 13$. Given a~truth assignment
$\tau'$ for~$\varphi'$, let consider its restriction $\tau =
g(\varphi,\tau')$ on the variables of~$\varphi$; for such assignment
$\tau$, one has: $m(\varphi,\tau) \geq m(\varphi',\tau') - 6m$. Then
$\mathrm{opt}\,(\varphi) - m(\varphi,\tau)$\linebreak $ = \mathrm{opt}\,(\varphi') - 6m -
m(\varphi,\tau) \leq \mathrm{opt}\,(\varphi') - m(\varphi',\tau')$. This means
that $\mathrm{opt}\,(\varphi) - m(\varphi,\tau) \leq \mathrm{opt}\,(\varphi') -
m(\varphi',\tau')$ and $\beta = 1$.

\smallskip

\noindent
\textbf{Conclusion 1}.\ According to~\cite{AP2005}, there is an L-reduction
with $\alpha=13$ and $\beta=1$ which implies approximation
algorithm with ratio of $1/(1+\alpha\beta (1/0.955-1)) = 0.62$
for MAX 3-SAT.

\smallskip

Therefore, the values of the four main criteria mentioned
before to analyze the quality of the L-reduction above are:
\begin{enumerate}[(1)]
\item {\it quality of approximation preserve according
to}~\cite{AP2005}: in L-reduction, this criterion is measured by
multiplication of values of the reduction parameters~$\alpha$ and~$\beta$
and equals~13 in this example;
\item {\it quality of approximation algorithm of a~destination
problem according to}~\cite{H1997}: {MAX 2-SAT} can be
approximated with the ratio~0.955;
\item {\it inflation of input and the run time complexity of a~destination problem according
to}~\cite{AP2005}: recall that a~size
of~$\varphi$ is $k$~clauses, where~$m$~of them have 3~variables. A~size
of~$\varphi'$ is, thus, $k - m + 10m = k + 9m$ clauses, since
every clause with 3~variables is expanded to~10~clauses with~2 and~1~variables. The run time complexity
of the used approximation scheme of {MAX 2-SAT} is much more than
linear; and
\item {\it complexity of reduction functions $f$ and~$g$
according to}~\cite{AP2005}: both functions $f$ and~$g$ are linear
in a~size of input.
\end{enumerate}

\noindent
\textbf{Example 2}.\
 \textbf{MAX 3SAT-B {\boldmath{$\leq_{L}$}} MAX INDEPENDENT}

\hspace*{34pt}\textbf{SET-B}

 \smallskip

\noindent
Let consider an instance of MAX 3-SAT $\varphi = \varphi_1$\linebreak $ \wedge
\varphi_2 \wedge\cdots\wedge \varphi_k$ with~$k$~clauses, where there
are at most~$B$ occurrences of each variable, for some constant~$B$.
The reduction~$f$~maps~$\varphi$ to a~graph with vertex degree
bounded by~$B$, for some constant~$B$ (need not be the same as for
MAX 3SAT-B instance) in the following way. Construct a~graph~$G$
with one node for every occurrence of every literal. There is an
edge connecting literal occurrences from the same clause~---
a~triangle for every 3-literal clause and a~single edge for every
2-literal clause. In addition, there is an edge connecting any two
occurrences of complementary literals.

\smallskip

\noindent
\textbf{Claim 1}~\cite{MSAT3Example, P1994}.
If every variable of~$\varphi$
occurs $\leq k$ times in the clauses, then the degree of the graph $G$
is $\leq k + 1$.

\smallskip

Every literal has at most 2 edges to the literals that
appear in its clause, and at most $k-1$ edges to its complementary
instances in other clauses of~$\varphi$. Thus, one gets that a~degree
of the nodes in the graph~$G$ is upper bounded by $k+1$.

\smallskip

\noindent
\textbf{Claim 2}~\cite{MSAT3Example, P1994}.
An independent set of size~$c$
corresponds to a~truth assignment that satisfies at least~$c$~clauses of~$\varphi$.

\smallskip

An independent set cannot select both a~literal and its
complement and can select at most one literal from each clause; so,
a truth assignment can be obtained by setting the variables
according to which nodes were in the independent set. For each node
in the independent set, there is a~satisfied clause. Note that there
may be other satisfied clauses of~$\varphi$.

\smallskip

\noindent
\textbf{Conclusion 2}~\cite{MSAT3Example, P1994}. Let~$c$ be an
independent set of G, and $\tau$ be $g(c)$. Thus,
\begin{multline*}
\mathrm{opt}\,(\varphi) - m(\varphi,\tau)\\
{} =
\mathrm{opt}\,(G) - m(\varphi,\tau) \leq \mathrm{opt}\,(G) - m(G,c)\,.
\end{multline*}
This means that $\beta = 1$.

\columnbreak

\noindent
\textbf{Claim 3}~\cite{MSAT3Example, P1994}. The size of the maximum independent
set in the graph is equal to the maximum number of clauses that can
be satisfied.

\smallskip

For each satisfied clause, there is a~node in the graph~$G$ that can be added to the independent set. This means that
$\mathrm{opt}\,(\varphi) = \mathrm{opt}\,(G)$ and, thus, $\alpha = 1$.

\smallskip

\noindent
\textbf{Conclusion 3}.
According to~\cite{MSAT3Example, P1994},
there is an \mbox{L-reduction} with $\alpha = \beta = 1$, which implies
approximation algorithm with ratio of $\alpha\beta B/6+o(1) =
B/6+o(1)$ or $\alpha\beta O({B}/{\log\log B}) =
O({B}/{\log\log B})$.

\smallskip

Therefore, the values of the four main criteria mentioned
before to analyze the quality of the L-reduction above are:\\[-13pt]
\begin{enumerate}[(1)]
\item {\it quality of approximation preserve according
to}~\cite{MSAT3Example, P1994}: in L-reduction, this criterion
is measured by multiplication of values of the reduction parameters~$\alpha$ and~$\beta$
and equals~1 in this
example;\\[-14pt]
\item {\it quality of approximation algorithm of a~destination
problem according to}~\cite{HR1994}: {MAX INDEPENDENT SET-B} can
be approximated with the ratio ${B}/{6}+o(1)$ or
$O({B}/{\log\log B})$ (depends on a~value of~$B$);\\[-14pt]
\item {\it inflation of input and the run time complexity of
a~destination problem according to}~\cite{MSAT3Example, P1994}:
recall that a~size of~$\varphi$ is~$k$~clauses where~$m$~of them
have 3~literals. Denote the number of distinct literals of~$\varphi$
by~$n$. A size of the graph~$G$ is, thus, $k - m + 3m + n{\lceil k/2
\rceil}^2 = k + 2m + n{\lceil k/2 \rceil}^2$ edges of a~graph~$G$,
where $3m$ stands for number of edges of a~clause with 3~literals,
and $n{\lceil k/2 \rceil}^2$ stands for an upper bound on the number
of edges between literals and their complementary literals. Each
literal appears at most $k$~times (i.\,e., in each clause) and
${\lceil k/2 \rceil}^2$ is the maximal size of full dual graph of~$k$~nodes. A~number of nodes
of graph~$G$ is the total number of
literals, with duplicates, in all clauses of~$\varphi$. The run time
complexity of the used approximation scheme of {MAX INDEPENDENT
SET-B} is near to linear: $N{k}/({j-1})+\min\{k^2N,N\log
N\}+|E|$ where $N$ is number of nodes in the graph~$G$; $E$ is the
number of nodes;  and $j$ is the maximal degree of cliques that should be
removed from graph~$G$ according to the approximation scheme; and
\item {\it complexity of reduction functions~$f$ and~$g$
according to}~\cite{MSAT3Example, P1994}: both functions~$f$
and~$g$ are
linear in a~size of input.
\end{enumerate}

\begin{table*}\small %tabl3
\begin{center}
\Caption{Comparison of Examples~1 and~2}
\label{tab:LReductionsCompare}
\vspace*{2ex}

\tabcolsep=1.1pt
\begin{tabular}{|c|c|c|c|c|c|c|}
\hline
Example& \tabcolsep=0pt\begin{tabular}{c}Quality\\ of approximation\\ preserve\end{tabular} &
 \tabcolsep=0pt\begin{tabular}{c}Approximation ratio\\ of a~destination\\ problem\end{tabular} &
Inflation of input &
 \tabcolsep=0pt\begin{tabular}{c}Run time\\ of a~destination\\ problem\end{tabular} &
 \tabcolsep=0pt\begin{tabular}{c}Complexity\\ of $f$ and $g$\end{tabular}&
 \tabcolsep=0pt\begin{tabular}{c}Approximation ratio\\ of a~source problem\end{tabular} \\
\hline
1 & $\alpha \beta = 13$ & 0.955 & $k + 9m$ &
$\gg$ linear & Linear & 0.509 \\
\hline
&&&&&&\\[-10pt]
2 & $\alpha  \beta = 1$\hphantom{9} & \tabcolsep=0pt\begin{tabular}{c}${B}/{6}+o(1)$,\\
$O({B}/{\log\log B})$\end{tabular} & $k+2m+n{\lceil k/2 \rceil}^2$ &
\tabcolsep=0pt\begin{tabular}{l}$N({k}/({j-1}))$\\
$+{}\min\{k^2N,N\log N\}$\\ $+|E|$, (near to
linear)\end{tabular} &
Linear & \tabcolsep=0pt\begin{tabular}{c}${B}/{6}+o(1)$,\\
$O({B}/{\log\log B})$\end{tabular} \\
\hline
\end{tabular}
\end{center}
%\vspace*{-6pt}
\end{table*}

\noindent
\textbf{Comparison of Examples~1 and~2.} The approximation
ratio of a~source problem depends on the parameter~$k$. Note that
$6/(k+1)>0.509$ leads to $k \leq 10$, and this is the case when one
should prefer {MAX INDEPENDENT SET-B} reduction according to
this criterion. The inflation of input is better for {MAX
2-SAT}, but its run time complexity of destination problem is worse
than of {MAX INDEPENDENT SET-B}. Besides, {MAX 2-SAT}
approximation scheme is cumbersome. So, there are trade-offs in
choosing of reduction in this case. The present authors suggest to make a~choice
according to a~given instance of {MAX 3-SAT} problem.

In the following,  let examine the reductions above with specific
(simplest) approximation algorithms for {MAX 2-SAT} and {MAX
INDEPENDENT SET-B}. It will be shown how to choose a~reduction for getting a~scale given an instance of {MAX 3-SAT}. The comparison
parameters are summarized in Table~3. %~\ref{tab:LReductionsCompare}.




It seems natural to consider first the reduction to {MAX 2-SAT},
since {MAX 2-SAT} and {MAX 3-SAT} are common problems, and
the {MAX 2-SAT} approximation algorithm is easier to
implement~\cite{link-to-the-sofware-cite}. Unfortunately, the reduction to
{MAX 2-SAT} is found useless, and the present authors turned to examine the
reduction to {MAX INDEPENDENT SET-B} which resulted with a~nontrivial scale.

Let use the following input $\varphi = (x_1)\wedge (\bar{x_1})\wedge
\cdots $\linebreak $\cdots\wedge(x_{40})\wedge (\bar{x_{40}})\wedge(x_{41}\vee
x_{42})\wedge(x_{43})\wedge(\bar{x_{41}}\vee
\bar{x_{43}})$\linebreak $\wedge(x_{41}\vee \bar{x_{42}} \vee x_{43})\wedge \cdots
\wedge (x_{53}\vee x_{54})\wedge(x_{55})\wedge(\bar{x_{53}}$\linebreak $\vee
\bar{x_{55}})\wedge(x_{53}\vee \bar{x_{54}} \vee x_{55})$ of {MAX 3-SAT}.
There are~100~clauses in $\varphi$, 85 of them are
one-literal, 10~of them are 2-literal, and~5~of them are 3-literal.
The approximation algorithm for MAX 2-SAT is described
in~\cite{Y1994} and is called \textit{the probabilistic method}. The
approximation algorithm for {MAX INDEPENDENT SET-B} is a~Greedy
algorithm which is described in~\cite{HR1994}. First,  it will be shown that
for the input~$\varphi$, ahead can be determine (i.\,e., without
running the reduction code) that the {MAX 2-SAT} reduction above
does not produce a~good enough scale (in fact, it produces a~trivial
scale that is related to the length of the input).

According to the used approximation algorithm for {MAX 2-SAT},
the translated input~$\varphi^\prime$ is approximated with the ratio
$r=0.37$ of an input size. According to L-reduction properties, this
means that the reduction is $1/(1+\alpha\beta(1/r-1)) =
1/(1+13\cdot1\cdot(1/0.37-1)) = 0.04$-approximative. Assume that the
reduction obtains a~truth assignment $\tau$ that satisfies all~5\
3-literal clauses of~$\varphi$. This means that exactly 7 of~10
clauses of~$\varphi^\prime$ that represent every 3-literal clause
of~$\varphi$, are satisfied. Since there are 5 3-literal clauses in~$\varphi$,
there are $7\cdot5 = 35$ satisfied clauses of total
145~clauses of~$\varphi^\prime$. Since at least~0.37 of~$\varphi^\prime$
clauses should be satisfied, at least~19 2-literal or unit clauses
of~$\varphi^\prime$ are also satisfied. These clauses are not
transformed when~$\varphi$ is translated to~$\varphi^\prime$; so, they
are satisfied also in the reduction solution for~$\varphi$. Thus, one
gets the upper bound of $\min\{|\varphi|,19/0.04\} = \min\{100,475\} =
100$, which adds no information beyond the trivial upper bound
of~$\varphi$ (i.\,e., equals to the input size) and implies the trivial
scale. Otherwise, the reduction obtains a~truth assignment~$\tau$
that does not satisfy all~5 3-literal clauses of~$\varphi$. In this
case, even more 2-literal or unit clauses of~$\varphi$ are satisfied
by~$\tau$ and the reduction once again returns a~trivial upper
bound for~$\varphi$.

The {MAX INDEPENDENT SET-B} reduction~\cite{link-to-the-sofware-cite}
results in a~few seconds in~0.55
of satisfied clauses of~$\varphi$ and its performance ratio is~0.6.
This implies a~non-trivial upper bound of $55/0.6=92$
clauses.

The run of {GA} for {MAX 3-SAT} is performed by using, once
again, galib246 library~\cite{GAlib} with a~mutation rate $p_{m} =
0.001$, crossover rate $p_{c} = 0.6$, and a~one-point crossover
operator~\cite{link-to-the-sofware-cite}. The initial population is
random. In a~few seconds, a~truth assignment was got that satisfies~58
of~100 clauses of~$\varphi$. The solution scale from {MAX
INDEPENDENT SET-B} reduction certifies the {GA} solution as
$58/92=0.64$ of the best possible result.

\vspace*{-6pt}

\subsection{Approximation preserve reductions of~MB3DM}

\vspace*{-2pt}

\noindent
The MB3DM {NPO}
problem is defined as follows. Given are disjointed sets~$A$, $B$,
and~$C$ and a~subset of triples $T \subseteq A \times B \times C$.
Let consider the restricted version of the {MB3DM} problem where
$|A|=|B|=|C|=q$ and each element of these sets appears in one, two,
or three triples of~$T$. The goal is to find a~subset $T' \subseteq T$
of maximum cardinality such that no two triples of~$T'$ agree in any
coordinate. This version of the MB3DM problem has been shown to be
{APX}-hard in~\cite{P1994}. Note that each triple can intersect
at most six other triples, which implies that the maximum matching
consists of at least $|T|/7$ triples. Let examine two examples of
approximation preserving reductions of {MB3DM} and compare them.

\pagebreak

\noindent
\textbf{Example~3.}\ \textbf{MB3DM {\boldmath{$\leq_{L}$}}
MINIMIZATION OF SUM}

\hspace*{34pt}\textbf{OF SQUARED MACHINE
LOADS}


\smallskip

\noindent
The problem of {scheduling unrelated parallel machines} is
defined as follows. Given are a~set of~$n$ independent jobs,
$J_1,\ldots,J_n$ and a~set of~$m$~parallel machines $M_1,\ldots,M_m$. Each
job~$j$ can be allocated to one of the machines in a~subset $M(j)
\subseteq {1,\ldots ,m}$. Each machine can process one job at a~time
and all machines are available at start time. Denote the time a~job~$j$ takes to be proceeded on a~machine~$M_i$ by~$p_{ij}$. In
addition, each job~$j$ has nonnegative weight~$w_j$. The goal is to
arrange the jobs to machines so that the minimal sum of the weighted
completion times is achieved.

Denote the sum of weights of jobs assigned to the machine~$i$ by the
load~$l_i$. Let define {Sum of Squared Machine Loads} as the~$l_2$
norm of the machines load vector
$\overrightarrow{l}=(l_1,\ldots,l_m)$, where $l_2$ is defined by
$(\sum\limits_{i=1}^{m}l_i^2)^{1/2}$. Indeed,~$l_2$ is a~measure of the
quality of a~given assignment.

Let consider an instance~$I$ of the restricted version of MB3DM.
The reduction~$f$~maps~$I$ into an instance of the Sum of Squared
Machine Loads problem in the following way. Let define $3q$ machines,
a machine $M(T_i)$ for each triple~$T_i$ in~$T$, and $3q-|T|$ dummy
machines. Also, let define $5q$ jobs, a~job per each element of~$A$,
$B$, and $C$ and $2q$ dummy jobs. On all machines, each element job
has a~processing time~1 and each dummy job has a~processing time~3.
The element job can be assigned only to some triple machine
$M(T_i)$ and only if the triple assigned with the machine contains
the appropriate element.

Note that for the restricted version of MB3DM, the optimal solution
$\mathrm{Opt}\,(I)$ consists of $q=|A|=|B|=|C|$ triples, since each element of~$A$,
$B$, and $C$ appear in at least one triple of~$T$ and a~feasible
solution consists of disjoint triples, the size of optimal solution
is~$q$. Let estimate a~size of the optimal solution of $f(I)$.
The best case is when all three jobs of each triple~$T_i$ in
$\mathrm{Opt}\,(I)$ are scheduled to the appropriate machine $M(T_i)$. The $2q$
dummy jobs are scheduled to the $2q$ dummy machines, one job per
machine. One gets that every machine has load~3 (three element jobs
with load~1, or one dummy job with load~3). Then the objective value
of this schedule is the sum of squares of machine load and equals to
$3^2 \cdot 3q = 27q$. Thus, $\mathrm{Opt}\,(f(I)) \leq 27q = 27\mathrm{Opt}\,(I)$ which
means that the parameter~$\alpha$ of the L-reduction equals to~27.

Denote by $m_k$, $k=0,\ldots, 3$, a~number of machines in some feasible
solution of $f(I)$ that process exactly~$k$~element jobs. Then the
total number of machines equals $m_0+m_1+m_2+m_3 = 3q$, and
$m_1+2m_2+3m_3 = 3q$ is the total number of element jobs. Note that
according to the reduction definition, the objective value
$c(g(f(I))) = m_3$.

\smallskip

\noindent
\textbf{Lemma 2}~\cite{AERW2002}.
\textit{The objective value $c(s)$ of the feasible solution~$s$ of the
scheduling instance $f(I)$ satisfies $c(s) \geq 29q-2m_3$.}

\smallskip

This lemma yields that $|c(g(s))-\mathrm{Opt}\,(I)| = q$\linebreak $-\;m_3 =
1/2(29q-2m_3-27q) \leq 1/2|c(s)-\mathrm{Opt}\,(f(I))|$ which means that
the~$\beta$ parameter of the L-reduction equals to~1/2.

\smallskip

\noindent
\textbf{Conclusion 4}.
According to~\cite{AERW2002}, there is an L-reduction with
$\alpha=27$ and $\beta=1/2$, which implies approximation algorithm
with ratio of
$1/(1+\alpha\beta(1+\sqrt{2}-1))$\linebreak $=1/(1+27\cdot1/2\cdot\sqrt{2})=0.05$
for {MB3DM}.

\smallskip

Therefore, the values of the four main criteria mentioned
before to analyze the quality of the L-reduction above are:
\begin{enumerate}[(1)]
\item {\it quality of approximation preserve according
to}~\cite{AERW2002}: in L-reduction this criterion is measured by
multiplication of values of the reduction parameters~$\alpha$ and~$\beta$
and equals~13.5 in this example;
\item {\it quality of approximation algorithm of a~destination
problem according to}~\cite{AAGKKV}: the {SUM OF SQUARED MACHINE
LOADS} problem can be approximated with the ratio $1+\sqrt{2}$;
\item {\it inflation of input and the run time complexity of a~destination problem according
to}~\cite{AERW2002}: 3q
machines and 5q jobs are defined  which implies the inflation of input to be 8q.
The run time complexity of the used approximation scheme of {SUM
OF SQUARED MACHINE
LOADS} is quadratic in input size; and
\item {\it complexity of reduction functions $f$ and~$g$
according to}~\cite{AERW2002}: both functions $f$ and~$g$ are linear
in a~size of input.
\end{enumerate}

\smallskip

\noindent
\textbf{Example 4}.
\textbf{MB3DM {\boldmath{$\leq_{AP}$}} STAR-GCA-SIMPLE}

\smallskip

\noindent
Let $A$ and $B$ be two NPO problems. $A$ is said to be
\textit{AP-reducible} to~$B$, denoted by $A\leq_{AP}B$, if an
extended template $(f,g)$ between~$A$ and~$B$ and a~positive
constant~$\alpha$ exist such that for any $x \in I_{A}$, for any
rational $r> 1$, and for any $y \in \mathrm{sol}_{B}(f(x,r))$,
$R_{B}(f(x,r),y) \leq r$ implies $R_{A}(x,g(x,y,r)) \leq 1 +
\alpha(r-1)$. The triple $(f,g,\alpha)$ is said to be an
AP-reduction from $A$ to~$B$~\cite{TrevisanLuca}.

The problem of {STAR-GCA} (General Call Admission control
problem in STAR networks) is defined as follows. A star network is
the undirected graph $G=(V,E)$, when the node $0$ represents a~unique
central node and rest of nodes are connected to it and called outer
nodes. The edge set $E$ consists of the edges $e_i=(i,0)$ for
$i=1,\ldots,n$ according to the network links structure. Each link has
a positive capacity $c(e)$.

A request for a~connection ({call}) is defined by a~tuple
$(u_i,v_i,t_i,d_i,b_i,p_i)$ consisting of a~source node $u_i \in V$,
a destination node $v_i \in V$, a~starting time $t_i$,
a~duration~$d_i$, a~positive bandwidth requirement $b_i$, and a~profit~$p_i$.
A~solution is a~set of accepted calls of those arrived. A feasibility
of a~solution is determined by the sum of bandwidth requirements of
simultaneously active calls using the same edge does not exceed the
capacity of that edge. The goal is to maximize the sum of profits of
the accepted calls.

{STAR-GCA-SIMPLE} is the restriction of the {STAR-GCA}
problem which is defined as follows. Each call~$i$ has one of $t_i
\in {0,1,2}$ starting times, duration $d_i=2$ and unit profit
and the needed bandwidth is also one unit on each call edge. The
capacity of each edge is a~unit which implies that only a~single
path per edge can be active in every given moment. Thus, the
objective function is a~cardinality of accepted calls set.

Let $I$ be an instance of MB3DM problem. The reduction~$f$~maps~$I$
into instance of STAR-GCA-SIMPLE in the following way. For every
element $a_i \in A, b_i \in B$, and $c_i \in C$,  a~link of
this element to the central node~$0$ is defined. For each triple $t_j =
(a_j,b_j,c_j) \in T$, additional three nodes $d_{j,1}$, $d_{j,2}$, and
$d_{j,3}$ are defined and linked to the central node. Also, 5
additional requests are added: $r_1 = (d_{j,1}, d_{j,2})$ with time
interval $[0,2)$; $r_2 = (d_{j,2}, d_{j,3})$ with time interval
$[2,4)$; $r_3 = (d_{j,1}, a_j)$ with time interval $[1,3)$; $r_4 =
(d_{j,2}, b_j)$ with time interval $[1,3)$; and $r_5 = (d_{j,3}, c_j)$
with time interval $[1,3)$.

One has to ensure that no more than three requests for one triple
are accepted. Note that according to the reduction definition above,
either the requests $r_1$ and $r_2$ are accepted since they have
disjoint time intervals or the requests $r_3,r_4$, and $r_5$ are
accepted since they do not share any edge. In any case, the accepted
requests of one triple block the rest of the requests of the same
triple. Let obtain an approximate solution $M_1$ for a~given instance
of MB3DM problem using the following

\smallskip

\noindent
\textbf{Lemma 3}~\cite{AEMSSW2005}. \textit{There is a~greedy
procedure that computes a~1/3-approximation for the bounded maximum
3-dimensional matching problem.}


\smallskip

Then the reduction function~$g$ composes a~solution~$M_2$
using the solution obtained for the instance $f(I)$ of the
STAR-GCA-SIMPLE problem in the following way. If three calls
$r_3$, $r_4$, and $r_5$ of some triple $t \in T$ are accepted, this
triple is added to the solution of the MB3DM problem instance~$I$.
Then, let choose the maximal solution obtained from the reduction and
from the approximation algorithm for MB3DM, i.\,e.,
$|g(f(I))|=\max\{|M_1|,|M_2|\}$. From the lemma above, one gets that
$|M_1| \geq |M^*|/3$ where $M^*$ is the maximum matching for $I$. By
the following

\smallskip

\noindent
\textbf{Lemma 4}~\cite{AEMSSW2005}.
\textit{Let $T \subseteq A \times B \times C$ be an
instance of the maximum three-dimensional matching problem, and let
$(G,R)$ be the corresponding instance of STAR-GCA-SIMPLE defined
above. There is a~feasible solution for $(G,R)$ that accepts
$2|T|+k$ requests iff $T$ has a~matching of size~$k$.}


\smallskip

\noindent
one obtains that $|M_2| \geq |g(f(I))|-2|T|$. Till now,
the first four properties of AP-reduction have been satisfied. Let show
that the fifth property is also holds with $\alpha=43$. According to
the lemma above, an optimal solution for~$f(I)$ consists of
$|M^*|+2|T|$ requests where $M^*$ is the optimal solution of an
instance~$I$. Assume that one has an $r$-approximation algorithm for
$f(I)$ that implies a~solution~$Q$ consists of at least
$(|M^*|+2|T|)/r$ requests. If $r \geq 45/43$, the inequality
$|g(f(I))| \geq |M_1| \geq |M^*|/3$ shows that $g$ computes a~1/3-approximation.
Since $3=1+43(45/43-1)$\linebreak $ \leq 1+43(r-1)$,
$|g(f(I))| \geq {M^*}/({1+43(r-1)})$ and $1+43(r-1) \geq
{M^*}/{|g(f(I))|}$. According to the fifth property of the
AP-reduction, this means that the value of the reduction parameter~$\alpha$
equals~43. Otherwise, $r < 45/43$. From $|Q| \geq
(|M^*|+2|T|)/r$, one  gets
\begin{multline*}
|Q| \geq \fr{2r|T|+\left\vert M^*\right\vert -2(r-1)|T|}{r}\\
{} =
2|T|+ \fr{\left\vert M^*\right\vert-2(r-1)|T|}{r} \\
{}\geq 2|T|+
\fr{\left\vert M^*\right\vert(1-14(r-1))}{r}
\end{multline*}
 where the property $|T| \leq
7|M^*|$ mentioned before in the last inequality was used. As $|g(f(I))| \geq
|M_2| $\linebreak $\geq |Q|-2|T| \geq {(1-14(r-1))|M^*|}/{r}$, one gets that
\begin{multline*}
\fr{\left\vert M^*\right\vert}{|g(f(I))|} \leq \fr{\left\vert M^*\right\vert}
{(1-14(r-1))\left\vert M^*\right\vert/r}\\
{} =
1+\fr{15}{15-14r}(r-1) \leq 1+43(r-1)
\end{multline*}
 where the last inequality
holds for $1<r<45/43$. Again, the fifth property is fulfilled with
$\alpha=43$.

\smallskip

\noindent
\textbf{Conclusion~5}.
According to~\cite{AEMSSW2005}, there is an AP-reduction with
$\alpha=43$, which implies approximation algorithm with ratio of
$1/(1+\alpha(1/1/18-1))$\linebreak $=1/(1+43(18-1))=1/732=0.001$ for {MB3DM}.

\smallskip

\begin{table*}\small %tabl4[htb!]
\begin{center}
\Caption{Comparison of Examples 3 and~4}
\label{tab:APLReductionsCompare}
\vspace*{2ex}

\tabcolsep=1.8pt
\begin{tabular}{|c|c|c|c|c|c|c|}
\hline
Example& \tabcolsep=0pt\begin{tabular}{c}Quality\\ of approximation\\ preserve\end{tabular} &
 \tabcolsep=0pt\begin{tabular}{c}Approximation ratio\\ of a~destination\\ problem\end{tabular} &
Inflation of input &
 \tabcolsep=0pt\begin{tabular}{c}Run time\\ of a~destination\\ problem\end{tabular} &
 \tabcolsep=0pt\begin{tabular}{c}Complexity\\ of $f$ and $g$\end{tabular}&
 \tabcolsep=0pt\begin{tabular}{c}Approximation ratio\\ of a~source problem\end{tabular} \\
\hline
3 & $\alpha \beta = 13.5$ & $1+\sqrt{2}$ & $8q$ & Quadratic in input size & Linear & 0.05 \\
%\hline
4 & $\alpha = 43$ & $1/18$ & $11q\leq\cdots\leq27q$ & Quadratic in input size & Linear & 0.001 \\
\hline
\end{tabular}
\end{center}
\end{table*}


Therefore, the values of the four main criteria mentioned
before to analyze the quality of the AP-reduction above are:
\begin{enumerate}[(1)]
\item {\it quality of approximation preserve according
to}~\cite{AEMSSW2005}: in AP-reduction this criterion is measured by
the value of the reduction parameter~$\alpha$ and
equals~43 in this example;
\item {\it quality of approximation algorithm of a~destination
problem according to}~\cite{AEMSSW2005}: the {STAR-GCA-SIMPLE}
problem can be approximated with the ratio~1/18;
\item {\it inflation of input and the run time complexity of
a~destination problem according to}~\cite{AEMSSW2005}: suppose the
restricted version of {MB3DM} problem as of the Example~3
reduction, where $|A|=|B|=|C|=q$ and $q\leq|T|\leq 3q$. Let add a~vertex for each element of~$A$, $B$, and $C$, and for each triple of~$T$,
let add~3~vertices and~5~requests to~\textit{R}. Then the
inflation of input in this case is $3q+8|T|$. Since $q \leq |T| \leq
3q$, one gets that $11q \leq 3q+8|T| \leq 27q$. The run time complexity
of the used approximation scheme of {STAR-GCA-SIMPLE} is
quadratic in
input size; and
\item {\it complexity of reduction functions~$f$ and~$g$
according to}~\cite{AEMSSW2005}: both functions~$f$ and~$g$ are
linear in a~size of input.
\end{enumerate}

\smallskip

\noindent
\textbf{Comparison of Examples~3 and~4.} The approximation
ratio of the destination problem in Example~3 is tighter than the
approximation ratio of the destination problem in Example~4, namely
$1+\sqrt{2}$ vs.\ $1/18$. Since these examples use different kinds
of reductions,  the quality of approximation
preserve cannot be compared directly but only the approximation ratios obtained for
{MB3DM} problem. According to this criterion, the reduction of
Example~3 is better: 0.05 vs.~0.001. The input inflation for
Example~3 is only slightly smaller, and the execution time of both
destination problems approximation algorithms is quadratic in input
size. Therefore, this criterion does not dominate the choice of a~reduction. The complexity of the reduction functions~$f$ and~$g$ of
both reductions is linear in a~size of input.

According to all comparable criteria, summarized in Table~4,
the reduction of Example~3 is better than the reduction of Example~4.

\vspace*{-6pt}

\section{Concluding Remarks}
\label{s:Concluding}

\vspace*{-4pt}

\noindent
The main goal of this
paper is to define a~new framework in which the quality of a~solution that is based on heuristics is certified by an
approximation algorithm. The framework  was defined and it was
verified that
it is feasible, useful, and essential in order to get a~certificate
for a~heuristic.

It was note that it is possible that the approximation ratio
stated for a~particular approximation algorithm is the
worst case ratio over all inputs, one may investigate
tight approximation ratio for a~given input.
This may further motivate the designer of approximation
schemes to provide a~function from
inputs to approximation ratio rather than
only the worst case single bound. Also, it was found that
the approximation results are useful to define the initial generation
of the GA as an important technique.

The authors hope that the computation of a~scale and a~certificate will become a~standard practice accompanying any
heuristics which, in turn, will assist in core AI tasks such as symbolic planning,
scheduling, and theorem proving.

%\vspace*{-6pt}

\Ack
\noindent
The authors thank Moshe Sipper and Eitan Bachmat for helpful discussions.

\renewcommand{\bibname}{\protect\rmfamily References}

%\vspace*{-6pt}

{\small\frenchspacing
{%\baselineskip=10.8pt
\begin{thebibliography}{99}

\bibitem{H1971} %1
\Aue{Holland, J.} 1971.
Genetic algorithms and the optimal allocation of
trials. \textit{SIAM J.~Comput.} 2:88--105.


\bibitem{GAlib} %2
Wall, M., and MIT. 1994--2005.
GAlib: A~C++ library of genetic algorithm
components. Available at: {\sf
http:// lancet.mit.edu/ga/} (accessed February~10, 2015).

\bibitem{KPP99} %3
\Aue{Kellerer, H., U.~Pferschy, and D.~Pisinger}. 2004.
\textit{Knapsack problems}.
Berlin: Springer. 161--166.

\bibitem{link-to-the-sofware-cite} %4
{\sf http://www.cs.bgu.ac.il/$\sim$sadetsky/Thesis/}
(accessed February~10, 2015).

\bibitem{Sipper1996} %5
\Aue{Sipper, M.}  1996.
A~brief introduction to genetic algorithms.
Available at: {\sf http://www.cs.bgu.ac.il/$\sim$sipper/ ga.html}
(accessed February~10, 2015).

\bibitem{TrevisanLuca} %6
\Aue{Trevisan, L.} 1997.
Reductions and (non-)approximability. Universita
Degli Studi di Roma `La Sapienza,' dotoorato di Rjcerca in
Informatica. IX-97-7:17--35.




\bibitem{AP2005} %7
\Aue{Ausiello, G., and V.~Th.~Paschos}.  2005.
Approximability preserving
reductions. \textit{Cahier Du Lamsade} 227:12.

\bibitem{H1997} %8
\Aue{Hastad, J.} 1997.
Some optimal inapproximability results.
\textit{29th ACM Symposium on Theory of
Computing Proceedings}.  1--10.

\bibitem{MSAT3Example} %9
{\sf http://www.cs.cmu.edu/afs/cs.cmu.edu/academic/cla ss/15451-s00/www/lectures/lect0406post.txt}
(accessed February~10, 2015).

\bibitem{P1994} %10
\Aue{Papadimitriou, C., and M.~Yannakakis}.
1988. Optimization, approximation,
and complexity classes. \textit{20th Annual ACM Symposium on the
Theory of Computing Proceedings.} %ISBN:0-89791-264-0,
229--234.

\bibitem{HR1994} %11
\Aue{Halldorsson, M., and J.~Radhakrishnan}. 1994.
Greed is good: Approximating
independent sets in sparse and\linebreak\vspace*{-12pt}

\pagebreak

\noindent
 bounded-degree graphs. \textit{30th
ACM Symposium on Theory of Computing Proceedings}.
439--448.

\bibitem{Y1994} %12
\Aue{Yannakakis, M.}  1994.
\textit{On the approximation of maximum satisfiability}.
\textit{3rd Annual ACM-SIAM Symposium on Discrete Algorithms Proceedings}.
Orlando, FL, USA.
475--502.


\bibitem{AERW2002} %13
\Aue{Azar, Y., L.~Epstein, Y.~Richter, and G.~Woeginger}.
2004. All-norm approximation algorithms. \textit{J.~Algorithm.} 52(2):120--133.
%ISSN:0196-6774, pp. 120-133, 2004.


\bibitem{AAGKKV} %14
\Aue{Awerbuch,~B.,  Y.~Azar, E.~Grove, M.~Kao, P.~Krishman,
and J.~Vitter}. 1995. Load balancing in the $L_p$ norm.
\textit{IEEE Symposium on Foundations of Computer Science
(FOCS) Proceedings}.


%\bibitem{MM1971} S. \ Matuura, T. \ Mastui,
%``0.935-Approximation Randomized Algorithm for MAX 2SAT and its
%Derandomization,'' Technical Report METR 2001-03, University of
%Tokyo, pp. 10, 2001.

%\bibitem{P1988}
%\Aue{Petrank,  E.} 1994.
%The hardness of approximation: Gap location.
%\textit{Computational Complexity} 4:133--157.


%\bibitem{book}
%\Aue{Vazirani, V.} 2001. \textit{Approximation algorithms}.
%Berllin: Springer-Verlag. 69--72.

\bibitem{AEMSSW2005} %15
\Aue{Adamy,~U., T.~Erlebach, D.~Mitsche, I.~Schurr,
B.~Speckmann, and E.~Welzl}. 2005.
Off-line admission control for advance
reservations in star networks.
\textit{Approximation Online
Algorithms} 3351:211--224.

\end{thebibliography} } }

\end{multicols}

\vspace*{-3pt}

\hfill{\small\textit{Received January 12, 2015}}

\vspace*{-24pt}

\Contr

\noindent
\textbf{Dolev Shlomi} (b.\ 1958)~---
Doctor of Science in computer science, professor, Ben-Gurion University of the Negev,
Beer-Sheva 84105, Israel; dolev@cs.bgu.ac.il

\vspace*{3pt}

\noindent
\textbf{Kogan-Sadetsky Marina} (b.\ 1977)~--- PhD student,
Ben-Gurion University of the Negev,
Beer-Sheva 84105, Israel; sadetsky@cs.bgu.ac.il

\vspace*{10pt}

\hrule

\vspace*{2pt}

\hrule

\vspace*{10pt}

%\newpage


\def\tit{ЭВРИСТИЧЕСКИЕ СЕРТИФИКАТЫ ПОСРЕДСТВОМ ПРИБЛИЖЕНИЙ}

\def\aut{Ш.~Долев$^1$, М.~Коган-Садецкая$^2$}


\def\titkol{Эвристические сертификаты посредством приближений}

\def\autkol{Ш.~Долев$^1$, М.~Коган-Садецкая$^2$}

%{\renewcommand{\thefootnote}{\fnsymbol{footnote}}
%\footnotetext[1]{Работа проводится при финансовой поддержке Программы
%стратегического развития Петрозаводского государственного университета в рамках
%на\-уч\-но-ис\-сле\-до\-ва\-тель\-ской деятельности.}}


\titel{\tit}{\aut}{\autkol}{\titkol}

\vspace*{-12pt}

\noindent
$^1$Факультет компьютерных наук, Университет Бен-Гурион
в~Негеве, Израиль, dolev@cs.bgu.ac.il

\noindent
$^2$Факультет компьютерных наук, Университет Бен-Гурион
в~Негеве, Израиль, sadetsky@cs.bgu.ac.il

\vspace*{6pt}

\def\leftfootline{\small{\textbf{\thepage}
\hfill ИНФОРМАТИКА И ЕЁ ПРИМЕНЕНИЯ\ \ \ том\ 9\ \ \ выпуск\ 1\ \ \ 2015}
}%
 \def\rightfootline{\small{ИНФОРМАТИКА И ЕЁ ПРИМЕНЕНИЯ\ \ \ том\ 9\ \ \ выпуск\ 1\ \ \ 2015
\hfill \textbf{\thepage}}}


\Abst{Предложен новый метод, в~котором качество (необязательно
оптимального) эвристического
решения сертифицировано приближенным алгоритмом, а~именно: результат
эвристического решения
сопровождается шкалой, получаемой из приближенного алгоритма. Создание шкалы
эффективно,
в~то время как  получение решения от алгоритма аппроксимации обычно требует
длительных
расчетов относительно эвристического подхода. С~другой стороны, результаты,
полученные с~помощью эвристики без шкалы, могут быть бесполезными. Исследованы
 критерии
для выбора
схемы аппроксимации для получения шкалы. Чтобы получить шкалу на
практике, приближения рассмотрены не только по их асимптотическому
поведению,
но также изучены соотношения между ними относительно размера ввода для данной
проблемы.
Для практического примера рассмотрены эвристические и приближенные алгоритмы
для задач
{SINGLE KNAPSACK}, {MAX 3-SAT} и~{MAXIMUM BOUNDED THREE-DIMENSIONAL
MATCHING}, которые
являются известными {NP}-слож\-ны\-ми задачами. Получены сертификаты для
эвристических запусков с~использованием подходящих приближений.}


\KW{эвристика; приближенный алгоритм;
оптимальное решение; сводимости сохраняющие приближения}

\DOI{10.14357/19922264150103}

%\vspace*{6pt}


 \begin{multicols}{2}

\renewcommand{\bibname}{\protect\rmfamily Литература}
%\renewcommand{\bibname}{\large\protect\rm References}

{\small\frenchspacing
{%\baselineskip=10.8pt
\begin{thebibliography}{99}

\bibitem{H1971-1} %1
\Au{Holland J.}
Genetic algorithms and the optimal allocation of
trials~// \textit{SIAM J.~Comput.}, 1971. Vol.~2. P.~88--105.

\bibitem{GAlib-1} %2
Wall, M., and MIT. 1994--2005.
GAlib: A~C++ library of genetic algorithm
components. {\sf
http://lancet. mit.edu/ga/}.


\bibitem{KPP99-1} %3
\Au{Kellerer H., Pferschy U., Pisinger ~D.}
{Knapsack problems}.~---
Berlin: Springer, 2004. P.~161--166.

\bibitem{link-to-the-sofware-cite-1} %4
{\sf http://www.cs.bgu.ac.il/$\sim$sadetsky/Thesis/}.

\bibitem{Sipper1996-1} %5
\Au{Sipper M.}  1996.
A~brief introduction to genetic algorithms.
{\sf http://www.cs.bgu.ac.il/$\sim$sipper/ga.html}.

\columnbreak

\bibitem{TrevisanLuca-1} %6
\Au{Trevisan~L.} 1997.
Reductions and (non-)approximability. Universita
Degli Studi di Roma `La Sapienza,' dotoorato di Rjcerca in
Informatica. Vol.~IX-97-7. P.~17--35.

\bibitem{AP2005-1} %7
\Au{Ausiello G., Paschos~V.\,Th.}
Approximability preserving
reductions~// {Cahier Du Lamsade}, 2005. Vol.~227. P.~12.

\bibitem{H1997-1} %8
\Au{Hastad J.}
Some optimal inapproximability results~//
\textit{29th ACM Symposium on Theory of
Computing Proceedings}, 1997.  P.~1--10.

\bibitem{MSAT3Example-1} %9
{\sf http://www.cs.cmu.edu/afs/cs.cmu.edu/academic/cla ss/15451-s00/www/lectures/lect0406post.txt}.


\bibitem{P1994-1} %10
\Au{Papadimitriou C., Yannakakis~M.}
Optimization, approximation,
and complexity classes~// {20th Annual ACM Symposium on the
Theory of Computing Proceedings}, 1988. %ISBN:0-89791-264-0,
P.~229--234.

\bibitem{HR1994-1} %11
\Au{Halldorsson M., Radhakrishnan J.}
Greed is good: Approximating
independent sets in sparse and bounded-degree graphs~// {30th
ACM Symposium on Theory of Computing Proceedings},  1994.
P.~439--448.

\bibitem{Y1994-1} %12
\Au{Yannakakis M.}
{On the approximation of maximum satisfiability}~//
3rd Annual ACM-SIAM Symposium on Discrete Algorithms Proceedings.~---
Orlando, FL, USA,  1994. P.~475--502.

\columnbreak

\bibitem{AERW2002-1} %13
\Au{Azar, Y., L.~Epstein, Y.~Richter, and G.~Woeginger}.
All-norm approximation algorithms~// {J.~Algorithm.}, 2004.
Vol.~52. No.\,2. P.~120--133.
%ISSN:0196-6774, pp. 120-133, 2004.

\bibitem{AAGKKV-1} %14
\Au{Awerbuch~B.,  Azar Y., Grove~E., Kao~M., Krishman~P.,
Vitter~J.} 1995. Load balancing in the $L_p$ norm.
\textit{IEEE Symposium on Foundations of Computer Science
(FOCS) Proceedings}.

\bibitem{AEMSSW2005-1} %15
\Au{Adamy~U., Erlebach T., Mitsche~D., Schurr~I.,
Speckmann~B., Welzl~E.}
Off-line admission control for advance
reservations in star networks~//
{Approximation Online
Algorithms}, 2005. Vol.~3351. P.~211--224.

%\bibitem{MM1971} S. \ Matuura, T. \ Mastui,
%``0.935-Approximation Randomized Algorithm for MAX 2SAT and its
%Derandomization,'' Technical Report METR 2001-03, University of
%Tokyo, pp. 10, 2001.



%\bibitem{P1988-1}
%\Au{Petrank  E.}
%The hardness of approximation: Gap location~//
%{Computational Complexity},  1994. Vol.~4. P.~133--157.


%\bibitem{book-1}
%\Au{Vazirani~V.} {Approximation algorithms}.~---
%Berllin: Springer-Verlag, 2001. P.~69--72.


\end{thebibliography}
} }

\end{multicols}

 \label{end\stat}

% \vspace*{-3pt}

\hfill{\small\textit{Поступила в редакцию 12.01.2015}}
\renewcommand{\bibname}{\protect\rm Литература}
\renewcommand{\figurename}{\protect\bf Рис.}
\renewcommand{\tablename}{\protect\bf Таблица}

