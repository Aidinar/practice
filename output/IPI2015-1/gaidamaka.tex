\def\stat{gaidamaka}

\def\tit{МЕТОД РАСЧЕТА ХАРАКТЕРИСТИК ИНТЕРФЕРЕНЦИИ
ДВУХ ВЗАИМОДЕЙСТВУЮЩИХ УСТРОЙСТВ
В~БЕСПРОВОДНОЙ ГЕТЕРОГЕННОЙ СЕТИ$^*$}

\def\titkol{Метод расчета характеристик интерференции
двух взаимодействующих устройств
в~беспроводной гетерогенной сети}

\def\aut{Ю.\,В.~Гайдамака$^1$, А.\,К.~Самуйлов$^2$}

\def\autkol{Ю.\,В.~Гайдамака, А.\,К.~Самуйлов}

\titel{\tit}{\aut}{\autkol}{\titkol}

{\renewcommand{\thefootnote}{\fnsymbol{footnote}} \footnotetext[1]
{Работа выполнена при финансовой поддержке РФФИ (проекты 14-07-00090
и~15-07-03051).}}


\renewcommand{\thefootnote}{\arabic{footnote}}
\footnotetext[1]{Российский университет дружбы народов, ygaidamaka@sci.pfu.edu.ru}
\footnotetext[2]{Российский университет дружбы народов; Технологический университет г.\ Тампере, Финляндия,
aksamuylov@gmail.com}


\Abst{Одним из показателей качества функционирования современных беспроводных сетей
является отношение сигнала к~сумме интерференции и шума (SINR, Signal to Interference plus
Noise Ratio) в~беспроводных каналах связи. Поскольку значение этой характеристики
существенно зависит от расстояния между интерферирующими устройствами, задача оценки
значения SINR часто сводится к~вычислению длины одной из сторон треугольника,
в~вершинах которого находятся взаимодействующие устройства. В~данной статье решается
задача нахождения математического ожидания и~среднеквадратического отклонения
отношения сигнал/интерференция пары взаимодействующих устройств в достаточно общих
предположениях о~распределении случайных величин (с.в.)\ расстояний между
интерфери\-ру\-ющи\-ми устройствами. Предложенный метод может быть использован
в~качестве основы для анализа интерференции в~гетерогенной сети с~применением различных
беспроводных технологий, включая анализ беспроводных взаимодействий оконечных
устройств, на которые интерференция оказывает наиболее сильное воздействие.}

\KW{беспроводная сеть; LTE; интерференция; SINR; взаимодействие устройств; D2D}

\DOI{10.14357/19922264150102}


\vskip 14pt plus 9pt minus 6pt

\thispagestyle{headings}

\begin{multicols}{2}

\label{st\stat}

\section{Постановка задачи}

  В современных беспроводных сетях, построенных на базе технологии LTE
(Long Term Evolution), оценка интерференции между взаимодействующими
устройствами является одной из основных задач анализа показателей качества
функционирования~[1,~2]. Под интерференцией понимается взаимодействие
сигналов, передаваемых разными\linebreak источниками на одном и~том же канале.
Интерференция вызывает искажение сигнала рас\-смат\-ри\-ва\-емо\-го источника под
воздействием сигнала сторонне\-го источника. В~гетерогенных сетях
беспроводного взаимодействия оконечных устройств D2D
  (device-to-device)~[3], где плот\-ность интерферирующих объектов высока,
интерференция оказывает существенное влияние на принимаемый оконечным
устройством сигнал. При анализе беспроводных взаимодействий устройств
обычно рассматривается несколько источников сигнала (передатчиков),
распределенных на плоскости согласно некоторому закону~[4]. Упрощение
задачи состоит в~том, что, рассмотрев один передатчик и~оценив
характеристики интерференции на соответствующем ему приемном устройстве
(приемнике), можно предположить, что основные показатели будут идентичны
и~для остальных пар <<пе\-ре\-дат\-чик--при\-ем\-ник>>. В~данной статье
решается задача нахождения числовых характеристик отношения
сигнал/ин\-тер\-фе\-рен\-ция пары взаимодействующих устройств.

  Отношение сигнала к сумме интерференции и~шума, SINR,
  является одной из основных характеристик качества канала
  в~беспроводных сетях связи~[5--7]. Отношение сигнала к~сумме интерференции 
и~шума на стороне приемника определяется по следующей формуле:
  \begin{equation}
  \mathrm{SINR} = \fr{S}{\sigma^2 +I}\,,
  \label{e1-gai}
  \end{equation}
где $S$~--- мощность принимаемого сигнала от соответствующего
передатчика; $\sigma^2$~--- мощность шума; $I$~--- мощность принимаемого
сигнала от интерферирующих передатчиков. Согласно линейной модели~[4]
\begin{equation}
S=gl^{-\alpha}\,,
\label{e2-gai}
\end{equation}
где $g$~--- базовая мощность сигнала передатчика, соответствующего
рассматриваемому приемнику; $l$~--- расстояние между передатчиком
и~приемником; $\alpha$~--- коэффициент потерь (path loss exponent),
принимающий значение от~2 (при условии прямой видимости) до~6 (в~худшем
случае). Величина~$I$ в~знаменателе формулы~(1) соответствует суммарной
мощности сигнала от всех интерферирующих передатчиков, где каждое
слагаемое имеет вид~(2). Заметим, что принцип повторного использования
частот (frequency reuse) в~беспроводных сетях связи поколения 4G (4th
Generation) позволяет назначать одну и~ту же единицу ресурса сети (например,
один и~тот же ресурсный блок LTE) нескольким парам взаимодействующих
устройств, если интерференция не превосходит определенного стандартами
уровня.

  Рассмотрим случай, когда несколько принимающих устройств (приемников)
и~одно передающее устройство (передатчик), образующие кластер,
расположены на плоскости внутри круга радиуса~$r_0$, причем передатчик
расположен в центре круга. Такой кластер образуется, например, при
проведении интерактивного занятия преподавателя с учениками, когда можно
предположить, что передатчик располагается в центре круга, а приемники
равномерно распределены внутри круга. Для передачи данных на каждую пару
взаимодействующих устройств внутри кластера планировщиком распределения
радиоресурсов в беспроводной сети 4G назначается по одному ресурсному
блоку LTE, и тогда сигналы взаимодействующих пар не интерферируют друг
с~другом. Но если в соседнем помещении также проходит интерактивное
занятие и там использованы те же ресурсные блоки, то пары из соседних
кластеров, использующие один и тот же ресурсный блок, будут создавать
помехи друг другу. Сведем задачу к анализу взаимодействия двух пар
устройств в двух кластерах, как показано на рис.~1.



  Пару взаимодействующих устройств, для которой будем рассчитывать
показатели качества канала, назовем целевой, а соответствующую ей пару
устройств обозначим TR$_0\hm= \langle \mathrm{Tx}_0, \mathrm{Rx}_0\rangle$.
Остальные пары, которые создают помехи целевой паре 
$\mathrm{TR}_0$,\linebreak\vspace*{-12pt}
\begin{center}  %fig1
\vspace*{8pt}
\mbox{%
 \epsfxsize=77.569mm
 \epsfbox{gai-1.eps}
 }
\end{center}

\noindent
{{\figurename~1}\ \ \small{Схема взаимодействия интерферирующих устройств}}


%\vspace*{9pt}


\addtocounter{figure}{1}


\noindent
 обозначим $\mathrm{TR}_i\hm= \langle
\mathrm{Tx}_i, \mathrm{Rx}_i\rangle$ и~будем называть их интерферирующими. Расстояние
между Rx$_i$ и~Tx$_i$ обозначим $R_i$, а~расстояние между Tx$_0$ и~Tx$_i$
обозначим~$U_i$. Мощность интерферирующего сигнала от пары TR$_i$
является функцией расстояния между приемником Rx$_0$ из целевой пары и
интерферирующим передатчиком~Tx$_i$, которое обозначим~$D_i$. Угол
между прямой, соединяющей целевые передатчик~Tx$_0$ и~приемник~Rx$_0$,
и~прямой, соеди\-ня\-ющей передатчики~Tx$_0$ и~Tx$_i$,
обозначим~$\gamma_i$.

  Рассмотрим систему двух кластеров, показанную на рис.~1. В~условиях
отсутствия шума и~одинаковой базовой мощности~$g$ сигналов обоих
передатчиков искомой характеристикой является отношение
  сигнал/ин\-тер\-фе\-рен\-ция SIR для приемника~Rx$_0$, вычисляемое по
формуле:
  \begin{equation}
\mathrm{SIR}=\left( \fr{D_1}{R_0}\right)^{\alpha}\,.
  \label{e3-gai}
  \end{equation}

  Будем считать, что $R_0$, $U_i$ и~$\gamma_i$ являются
с.в.\ с~заданными функциями распределения. Задача состоит
в~нахождении числовых характеристик с.в.~SIR. Для
решения задачи в следующем разделе статьи предлагается метод нахождения
совместной плотности распределения с.в.~$R_0$ и~$D_i$, что позволяет
вычислять начальные моменты ${\sf E}[\mathrm{SIR}^n]$ с.в.~SIR.

\vspace*{-6pt}

\section{Метод расчета отношения сигнал/интерференция}

%\vspace*{-2pt}

  Как видно из формулы~(3), с.в.~SIR пропорциональна с.в.~$D_1$, которая,
в~свою очередь, зависит от с.в.~$R_0$. В~этом случае для нахождения
характеристик с.в.~SIR необходимо найти совместное распределение
с.в.~$R_0$ и~$D_1$.

  Введем обозначения $\xi_1{:=} R_0$, $\xi_2 {:=} U_1$, $\xi_3 {:=}
\gamma_1$, $\eta_1 {:=} D_1$. Тогда $w_{\xi_1,\xi_2,\xi_3}(x_1,x_2,x_3) {:=}$\linebreak
${=:}\;f_{R_0, U_1, \gamma_1}(x_1,x_2,x_3)$~--- совместная плот\-ность
распределения с.в.~$R_0$, $U_1$ и~$\gamma_1$, а~$W_{\xi_1,\eta_1}(x_1,y_1)
{:=} f_{R_0, D_1}(x_1,y_1)$~--- искомое совместное распределение с.в.~$R_0$
и~$D_1$. По теореме косинусов с.в.~$\eta_1$ является функцией с.в.~$\xi_1$,
$\xi_2$ и~$\xi_3$:
  \begin{equation}
  \eta_1=\sqrt{\xi_1^2+\xi_2^2-2\xi_1\xi_2\cos \xi_3}\,.
  \label{e4-gai}
  \end{equation}

  Следуя~\cite{8-gai, 9-gai}, введя вспомогательную
переменную $\eta_2\hm=\xi_3$, искомое распределение можно найти по
следующей формуле:
  \begin{multline}
W_{\xi_1, \eta_1} (y_1,y_2) ={}\\
{}=\sum\limits_{i=1}^2
\int\limits_{\mathrm{Y}_{3,j}}\!\!\! w_{\xi_1,\xi_2,\xi_3}\left(
y_1,\varphi_i(y_1,y_2,y_3),y_3\right) \times{}\\[-6pt]
{}\times
\left\vert \fr{\partial \varphi_j(y_1,y_2,y_3)}
{\partial y_2}\right\vert\,dy_3\,,
  \label{e5-gai}
  \end{multline}
где $\varphi_j$~--- обратное преобразование правой части формулы~(\ref{e4-gai})
относительно~$\xi_2$:
\begin{align*}
\varphi_1(y_1,y_2,y_3) &= y_1\cos y_3 +\sqrt{y_2^2-y_1^2+y_1^2\cos^2 y_3}\,;\\
\varphi_2(y_1,y_2,y_3) &= y_1\cos y_3 -\sqrt{y_2^2-y_1^2+y_1^2\cos^2 y_3}\,.
\end{align*}

  В формуле~(\ref{e5-gai}) области значений Y$_{3,j}$ переменной~$y_3$ для
$j$-й вет\-ви обратного преобразования определяются системой неравенств:
  \begin{equation}
  \left.
  \begin{array}{c}
  \varphi_j(y_1,y_2,y_3)\geq0\,;\\[6pt]
  y_1\geq 0\,;\\[6pt]
  y_2\geq 0\,;\\[6pt]
  0\leq y_3\leq 2\pi\,.
  \end{array}
  \right\}
  \label{e6-gai}
  \end{equation}

  Решая систему~(\ref{e6-gai}), нетрудно убедиться, что для первой ветви
обратного преобразования  $\mathrm{Y}_{3,1}\hm= \mathrm{Y}_{3,1}^1\cup
\mathrm{Y}_{3,1}^2\cup \mathrm{Y}_{3,1}^3$, где
  \begin{align}
  \hspace*{-2mm}\mathrm{Y}_{3,1}^1 &=\begin{cases}
  0\leq y_2\leq y_1;\\
  0\leq y_3\leq \fr{1}{2}\,\mathrm{arccos}\,\left( \fr{y_1^2-
2y_2^2}{y_1^2}\right);\end{cases}
  \label{e7-1-gai}
\\
\hspace*{-2mm}\mathrm{Y}_{3,1}^2 &= \begin{cases}
  0\leq y_2\leq y_1;\\
  2\pi -\fr{1}{2}\,\mathrm{arccos}\left( \fr{y_1^2-2y_2^2}{y_1^2}\right) \leq
y_3\leq 2\pi;
  \end{cases}\!\!\!\!\!
  \label{e7-2-gai}
  \\
\hspace*{-2mm}\mathrm{Y}_{3,1}^3 &= \begin{cases}
  y_2\geq y_1;\\
  0\leq y_3\leq 2\pi,
  \end{cases}\!\!\!\!\!\!\!\!\!
  \label{e7-3-gai}
  \end{align}
а для второй ветви  $\mathrm{Y}_{3.2}=\mathrm{Y}_{3,2}^1\cup
\mathrm{Y}_{3,2}^2$, где
\begin{align}
\label{e8-1-gai}
\hspace*{-2mm}\mathrm{Y}_{3,2}^1 &= \begin{cases}
0\leq y_2\leq y_1\,;\\
0\leq y_3\leq \fr{1}{2}\,\mathrm{arccos}\left( \fr{y_1^2-2y_2^2}{y_1^2}\right);
\end{cases}
\\
\hspace*{-2mm}\mathrm{Y}_{3,2}^2 &=\begin{cases}
0\leq y_2\leq y_1\,;\\
2\pi -\fr{1}{2}\,\mathrm{arccos} \left( \fr{y_1^2-2y_2^2}{y_1^2}\right) \leq y_3\leq
2\pi.\!\!\!\!\!\!\!\!
\end{cases}
\label{e8-2-gai}
\end{align}

  Таким образом, получена формула для вычисления совместной плотности
с.в.~$R_0$ и~$D_1$:
  \begin{multline}
  W_{\xi_1,\eta_1}(y_1,y_2) ={}\\
  {}=\sum\limits_{i=1}^2 \int\limits_{\mathrm{Y}_{3,i}}
\fr{w_{\xi_1,\xi_2,\xi_3} (y_1,\varphi_i(y_1,y_2,y_3),y_3) y_2} {\sqrt{y_2^2-
y_1^2+y_1^2\cos^2 y_3}}\,dy_3\,,
  \label{e9-gai}
  \end{multline}
где $\mathrm{Y}_{3,j}$ вычисляются по
формулам~(\ref{e7-1-gai})--(\ref{e8-2-gai}).

  В следующем разделе приведен пример численного анализа
с~использованием формул~(\ref{e7-1-gai})--(\ref{e9-gai}).

\section{Пример численного анализа}

  В рассматриваемом примере предложенный выше метод использован для
расчета начальных моментов ${\sf E}[\mathrm{SIR}^n]$ отношения сигнал/интерференция,
которые определяются следующей формулой:
  \begin{multline}
{\sf   E}[\mathrm{SIR}^n] ={}\\
{}=\int\limits_{0\leq y_1\leq r_0} \int\limits_{y_2\geq0} \left(
\fr{y_2}{y_2}\right)^{n\alpha} W_{\xi_1,\eta_1}(y_1,y_2)\,dy_2dy_1\,.
  \label{e10-gai}
  \end{multline}

  Рассматривается случай, когда целевой приемник Rx$_0$ находится внутри
круга единичного \mbox{радиуса} ($r_0\hm=1$), в центре которого расположен
передат\-чик~Tx$_0$, а~интерферирующий передатчик~Tx$_1$~--- в~кольце
вокруг передатчика~Tx$_0$ с~внутренним радиусом~$r_0$ и~внешним
радиусом~$h_0$ (рис.~2).

\begin{center}  %fig2
\vspace*{8pt}
\mbox{%
 \epsfxsize=77.111mm
 \epsfbox{gai-2.eps}
 }


\noindent
{{\figurename~2}\ \ \small{Пример взаимодействия двух устройств}}

\end{center}


\vspace*{9pt}


\addtocounter{figure}{1}


%\noindent


  Тогда с.в.~$R_0$ расстояния от целевого передатчика Tx$_0$ до
соответствующего ему приемника Rx$_0$ и~с.в.~$U_1$ расстояния от целевого
передатчика~Tx$_0$ до интерферирующего передатчика~Tx$_1$ имеют
распределения
  \begin{alignat*}{2}
  f_{R_0}(r) &= 2r\,,&\quad 0&\leq r\leq1\,;\\
  f_{U_1}(u) &= \fr{2u}{h_0^2-1}\,,&\quad 1&\leq u\leq h_0\,.
  \end{alignat*}
Будем считать, что с.в.\ угла~$\gamma_1$ равномерно распределена на отрезке
$[0,\,2\pi]$, а~коэффициент потерь в~формуле~(2) принимает значение
$\alpha\hm=2$. Приняты условные единицы измерения: например, расстояние
между взаимодействующими устройствами может измеряться в~метрах,
а~величина SIR~--- в~децибелах.

  По формулам~(\ref{e7-1-gai})--(\ref{e10-gai}) рассчитано математическое
ожидание отношения сигнал/ин\-тер\-фе\-рен\-ция ${\sf E}[\mathrm{SIR}]$, представленное в
таблице в зависимости от радиуса внешней границы кольца, внутри которого
распределены интерферирующие передатчики. В~таблице также показаны
значения математического ожидания расстояния  ${\sf E}[U_1]$ от целевого
передатчика~Tx$_0$ до интерферирующего передатчика~Tx$_1$.

%  \begin{table*}\small
  \begin{center}
  \begin{tabular}{|c|c|c|}
  \multicolumn{3}{p{48mm}}{Математическое ожидание величины~SIR}\\
  \multicolumn{3}{c}{\ }\\[-5pt]
  \hline
\ \ \ \ $h_0$\ \ \ \ &\ \ \ \ ${\sf E}[U_1]$\ \ \ \ &${\sf E}[\mathrm{SIR}]$\\
\hline
2&1,56&4,84985\\
3&2,17&7,41701\\
4&2,8\hphantom{9}&9,54562\\
5&3,44&11,30286\hphantom{9}\\
\hline
\end{tabular}
\end{center}
%\end{table*}

\begin{center}  %fig3
\vspace*{18pt}
\mbox{%
 \epsfxsize=77.754mm
 \epsfbox{gai-3.eps}
 }
 \end{center}


\noindent
{{\figurename~3}\ \ \small{Числовые характеристики отношения сигнал/ин\-тер\-фе\-рен\-ция: \textit{1}~---
${\sf E}[\mathrm{SIR}]$; \textit{2}~--- $\sigma_{\mathrm{SIR}}$}}

\vspace*{18pt}

  Также были рассчитаны математическое ожидание ${\sf E}[\mathrm{SIR}]$
  и~среднеквадратическое \mbox{отклонение} $\sigma_{\mathrm{SIR}}\hm= \sqrt{{\sf E}[\mathrm{SIR}^2]-
{\sf E}[\mathrm{SIR}]^2}$ отношения сигнал/ин\-тер\-фе\-рен\-ция, показанные на рис.~3 в
зави\-си\-мости от математического ожидания расстояния ${\sf E}[U_1]$ между
целевым передатчиком~Tx$_0$ и~интерферирующим передатчиком~Tx$_1$. Из
таблицы и~графиков видно, что с~ростом расстояния между целевым
и~интерферирующим передатчиком обе \mbox{числовые} характеристики отношения
сиг\-нал/ин\-тер\-фе\-рен\-ция растут, поскольку мощность интерферирующего сигнала
убывает. Вычисления проводились с~использованием встроенных средств
пакета программ Wolfram Mathematica~[10].



\section{Заключение}

  В настоящей статье метод преобразования с.в.\ применен для
анализа основной характеристики качества функционирования беспроводных
сетей, а~именно: отношения сигнал/ин\-тер\-фе\-рен\-ция при заданных
распределениях расстояний между интерферирующими устройствами.
Приведенный пример показывает, что чис\-лен\-ный анализ является достаточно
трудоемким даже в простейших предположениях о~распределении исходных
с.в., а~для оценки характеристик интерференции в~условиях
наличия в~беспроводной сети нескольких источников интерференции требуется
разработка приближенных методов и~имитационных моделей, как это сделано,
например, в~[11]. Задача с несколькими источниками интерференции
в~беспроводных гетерогенных сетях взаимодействующих устройств
представляется особенно актуальной ввиду быст\-ро\-го развития сетей 4G
и~принятия в~ближайшем будущем стандартов для беспроводных сетей 5G~[12].
{\looseness=1

}

  \bigskip

Авторы выражают благодарность проф.\ К.\,Е.~Самуйлову за
плодотворное обсуждение и ценные советы.


{\small\frenchspacing
 {%\baselineskip=10.8pt
 \addcontentsline{toc}{section}{References}
 \begin{thebibliography}{99}
\bibitem{1-gai}
\Au{Гайдамака~Ю.\,В., Ефимушкина~Т.\,В., Самуйлов~А.\,К., Самуйлов~К.\,Е.} Задачи
оптимального планирования межуровневого интерфейса в беспроводных сетях~//
Информатика и~её применения, 2012. Т.~6. Вып.~3. С.~75--81.
\bibitem{2-gai}
\Au{Basharin G.\,P., Gaidamaka Yu.\,V., Samouylov~K.\,E.} Mathematical theory of teletraffic and
its application to the analysis of multiservice communication of next generation networks~//
Autom. Control Comp. Sci., 2013. Vol.~47. No.\,2. P.~62--69.
\bibitem{3-gai}
\Au{Andreev S., Pyattaev A., Johnsson~K., Galinina~O., Koucheryavy~Y.} Cellular traffic
offloading onto network-assisted device-to-device connections~// IEEE Commun. Mag.,
2014. Vol.~52. No.\,4. {\sf http://ieeexplore.ieee.org/\linebreak xpl/tocresult.jsp?isnumber=6807935}.
\bibitem{4-gai}
\Au{Baccelli F., Blaszczyszyn B.} Stochastic geometry and wireless networks. Vol.~I: Theory.~---
Boston: NoW Publs. Inc., 2009. 164~p.


\bibitem{6-gai} %5
\Au{Erturk M.\,C., Mukherjee S., Ishii~H., Arslan~H.} Distributions of transmit power and SINR in
device-to-device networks~// IEEE Commun. Lett., 2013. Vol.~17. No.\,2. {\sf
http://ieeexplore.ieee.org/xpl/tocresult.jsp?isnumber=\linebreak 6472443}.

\bibitem{7-gai} %6
\Au{Kim M., Han Y., Yoon~Y., Chong~Y., Lee~H.} Modeling of adjacent channel interference in
heterogeneous wireless networks~// IEEE Commun. Lett., 2013. Vol.~17. No.\,9. {\sf
http://ieeexplore.ieee.org/xpl/tocresult.jsp?isnumber=\linebreak 6604524}.

\bibitem{5-gai} %7
\Au{Andrews J.\,G., Singh S., Ye~Q., Lin~X., Dhillon~H.\,S.} An overview of load balancing in
hetnets: Old myths and open problems~// IEEE Wirel. Commun., 2014. Vol.~21. No.\,2.
{\sf http://ieeexplore.ieee.org/xpl/tocresult.\linebreak jsp?isnumber=6812279}.


\bibitem{8-gai}
\Au{Левин Б.\,Р.} Теоретические основы статистической радиотехники.~--- 3-е изд.~--- М.:
Радио и связь, 1989. 656~с.
\bibitem{9-gai}
\Au{Mardia K., Jupp P.} Directional statistics.~--- Wiley Press, 1999. 441~p.
\bibitem{10-gai}
Wolfram Mathematica: Программное обеспечение для технических вычислений. {\sf
http://www.wolfram.\linebreak com/mathematica}.
\bibitem{11-gai}
\Au{Гайдамака Ю.\,В., Печинкин А.\,В., Разумчик~Р.\,В., Самуйлов~А.\,К., Самуйлов~К.\,Е.,
Соколов~И.\,А., Сопин~Э.\,С., Шоргин~С.\,Я.} Распределение времени выхода из множества
состояний перегрузки в системе $M\vert M\vert 1\vert \langle L,H\rangle \vert \langle
H,R\rangle$ с~гистерезисным управлением нагрузкой~// Информатика и~её применения,
2013. Т.~7. Вып.~4. С.~20--33.
\bibitem{12-gai}
\Au{Tehrani M., Uysal M., Yanikomeroglu~H.} Device-to-device communication in 5G cellular
networks: Challenges, solutions, and future directions~// IEEE Commun. Mag., 2014.
Vol.~52. No.\,5. {\sf http://ieeexplore. ieee.org/xpl/tocresult.jsp?isnumber=6815882}.
 \end{thebibliography}

 }
 }

\end{multicols}

\vspace*{-3pt}

\hfill{\small\textit{Поступила в редакцию 20.01.15}}

%\newpage

\vspace*{12pt}

\hrule

\vspace*{2pt}

\hrule

%\vspace*{12pt}

\def\tit{METHOD FOR CALCULATING NUMERICAL
CHARACTERISTICS OF~TWO DEVICES INTERFERENCE
FOR~DEVICE-TO-DEVICE COMMUNICATIONS
IN~A~WIRELESS HETEROGENEOUS NETWORK}

\def\titkol{Method for calculating numerical
characteristics of~two devices interference
for~D2D communications
in~a~wireless %heterogeneous
network}

\def\aut{Yu.~Gaidamaka$^1$ and A.~Samuylov$^{1,2}$}

\def\autkol{Yu.~Gaidamaka and A.~Samuylov}

\titel{\tit}{\aut}{\autkol}{\titkol}

\vspace*{-9pt}

 \noindent
$^1$Peoples' Friendship University of Russia,
Applied Probability and Informatics Department,
6~Miklukho-Maklaya\linebreak
$\hphantom{^1}$Str., Moscow 117198, Russian Federation

\noindent
$^2$Tampere University of Technology,
Department of Electronics and Communications Engineering,
10 Korkeak-\linebreak
$\hphantom{^1}$oulunkatu,  Tampere 33720, Finland


\def\leftfootline{\small{\textbf{\thepage}
\hfill INFORMATIKA I EE PRIMENENIYA~--- INFORMATICS AND
APPLICATIONS\ \ \ 2015\ \ \ volume~9\ \ \ issue\ 1}
}%
 \def\rightfootline{\small{INFORMATIKA I EE PRIMENENIYA~---
INFORMATICS AND APPLICATIONS\ \ \ 2015\ \ \ volume~9\ \ \ issue\ 1
\hfill \textbf{\thepage}}}

\vspace*{3pt}


\Abste{In wireless networks, one of the key performance metrics is the signal to noise ratio, SINR. As this metric
highly depends on the distance between the interfering devices, the problem of SINR estimation is often reduced to the
calculation of a triangle's side length, where the vertices represent the interacting devices. This paper addresses the
problem of calculating the numerical characteristics of the signal to interference ratio for a pair of interfering devices
determined by the known distributions of distances between the entities in question. The proposed method can be used
as a basis for analyzing heterogeneous networks, including the analysis of
device-to-device (D2D) communications as one of
the interference-limited cases.}

\KWE{wireless network; LTE; interference; SINR; D2D}




\DOI{10.14357/19922264150102}

\Ack
\noindent
The reported study was partially supported by the Russian Foundation for Basic
Research,  research projects Nos.\,14-07-00090 and
15-07-03051.



%\vspace*{3pt}

  \begin{multicols}{2}

\renewcommand{\bibname}{\protect\rmfamily References}
%\renewcommand{\bibname}{\large\protect\rm References}



{\small\frenchspacing
 {%\baselineskip=10.8pt
 \addcontentsline{toc}{section}{References}
 \begin{thebibliography}{99}
\bibitem{1-gai-1}
\Aue{Gaidamaka, Yu.\,V., T.\,V. Efimushkina, A.\,K.~Samuylov, and K.\,E.~Samouylov}. 2012.
Zadachi optimal'nogo planirovaniya mezhurovnevogo interfeysa v besprovodnykh setyakh
[Cross-layer optimization planning problems in wireless networks]. \textit{Informatika i~ee
Primeneniya}~--- \textit{Inform. Appl.} 6(3):75--81.
\bibitem{2-gai-1}
\Aue{Basharin, G.\,P., Yu.\,V. Gaidamaka, and K.\,E.~Samouylov}. 2013. Mathematical theory of
teletraffic and its application to the analysis of multiservice communication of next generation
networks. \textit{Autom. Control Comp. Sci.} 47 (2):62--69.
\bibitem{3-gai-1}
\Aue{Andreev, S., A. Pyattaev, K.~Johnsson, O.~Galinina, and Y.~Koucheryavy}. 2014. Cellular
traffic offloading onto network-assisted device-to-device connections. \textit{IEEE
Commun. Mag.} 52(4). Available at: {\sf
http://ieeexplore.ieee.\linebreak org/xpl/tocresult.jsp?isnumber=6807935} (accessed January~10, 2015).
\bibitem{4-gai-1}
\Aue{Baccelli, F., and B. Blaszczyszyn.} 2009. \textit{Stochastic geometry and wireless networks}.
Vol.~I: Theory. Boston: NoW Publs. Inc. 164~p.



January~10, 2015). %5
\bibitem{6-gai-1}
\Aue{Erturk, M.\,C., S. Mukherjee, H.~Ishii, and H.~Arslan}. 2013. Distributions of transmit
power and SINR in device-to-device networks. \textit{IEEE Commun. Lett.} 17(2).
Available at: {\sf http://ieeexplore.ieee.org/xpl/tocresult.jsp?isnumber=\linebreak 6472443} (accessed
January~10, 2015).
\bibitem{7-gai-1} %6
\Aue{Kim, M., Y. Han, Y.~Yoon, Y.~Chong, and H.~Lee}. 2013. Modeling of adjacent channel
interference in heterogeneous wireless networks. \textit{IEEE Commun. Lett.} 17(9).
Available at: {\sf http://ieeexplore.ieee.org/\linebreak xpl/tocresult.jsp?isnumber=6604524} (accessed
January~10, 2015).

\bibitem{5-gai-1} %7
\Aue{Andrews, J.\,G., S. Singh, Q.~Ye, X.~Lin, and H.\,S.~Dhillon}. 2014. An overview of load
balancing in hetnets: Old myths and open problems. \textit{IEEE Wirel. Commun.} 21(2).
Available at: {\sf http://ieeexplore.ieee.org/\linebreak xpl/tocresult.jsp?isnumber=6812279} (accessed

\bibitem{8-gai-1}
\Aue{Levin, B.\,R.} 1989. \textit{Teoreticheskie osnovy statisticheskoy radiotekhniki} [Theoretical
basis of statistical radiotechnics]. 3rd ed. Moscow: Radio and Communications. 656~p.
\bibitem{9-gai-1}
\Aue{Mardia, K., and P. Jupp}. 1999. \textit{Directional statistics}. 1st ed. Wiley Press. 441~p.
\bibitem{10-gai-1}
Wolfram mathematica: Software for technical computing. [Free access] Available at: {\sf
http://www.wolfram.\linebreak com/mathematica} (accessed December~1, 2014).
\bibitem{11-gai-1}
\Aue{Gaidamaka, Yu.\,V., A.\,V. Pechinkin, R.\,V.~Razumchik, A.\,K.~Samuylov,
K.\,E.~Samouylov, I.\,A.~Sokolov, E.\,S.~Sopin, and S.\,Ya.~Shorgin}. 2013. Raspredelenie
vremeni vykhoda iz mnozhestva sostoyaniy peregruzki v~sisteme $M\vert M\vert 1\vert \langle
L,H\rangle \vert \langle H,R\rangle$ s~gisterezisnym upravleniem nagruzkoy [The distribution of
the return time from the set of overload states to the set of normal load states in a system $M\vert
M\vert 1\vert \langle L,H\rangle \vert \langle H,R\rangle$ with hysteretic load control].
\textit{Informatika i~ee~Primeneniya}~--- \textit{Inform. Appl.} 7(4):20--33.
\bibitem{12-gai-1}
\Aue{Tehrani, M., M. Uysal, and H.~Yanikomeroglu.} 2014. Device-to-device communication in
5G cellular networks: Challenges, solutions, and future directions.
\textit{IEEE Commun. Mag.} 52(5). Available at: {\sf http://ieeexplore.\linebreak ieee.org/xpl/tocresult.jsp?isnumber=6815882}
(accessed January~10, 2015).
\end{thebibliography}

 }
 }

\end{multicols}

\vspace*{-3pt}

\hfill{\small\textit{Received January 20, 2015}}

%\vspace*{-18pt}


\Contr

\noindent
\textbf{Gaidamaka Yuliya V.} (b.\ 1971)~---
Candidate of Science (PhD) in physics and mathematics, associate
professor, Applied Probability and Informatics Department, Peoples' Friendship University of Russia,
6~Miklukho-Maklaya Str., Moscow 117198, Russian Federation;
ygaidamaka@sci.pfu.edu.ru

\vspace*{3pt}

\noindent
\textbf{Samuylov Andrey K.} (b.\ 1988)~---
PhD student, Peoples' Friendship University of Russia, Moscow 117198, Russian
Federation; researcher, Department of Electronics and Communications Engineering,  Tampere
University of Technology, 10 Korkeakoulunkatu, Tampere 33720, Finland;
aksamuylov@gmail.com

\label{end\stat}

\renewcommand{\bibname}{\protect\rm Литература}