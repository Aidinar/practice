\def\stat{sinitsin}

\def\tit{МОДЕЛИРОВАНИЕ НОРМАЛЬНЫХ ПРОЦЕССОВ В~СТОХАСТИЧЕСКИХ
   СИСТЕМАХ СО~СЛОЖНЫМИ ИРРАЦИОНАЛЬНЫМИ НЕЛИНЕЙНОСТЯМИ$^*$}

\def\titkol{Моделирование нормальных процессов в~стохастических
   системах со~сложными иррациональными нелинейностями}

\def\aut{И.\,Н.~Синицын$^1$, В.\,И.~Синицын$^2$, Э.\,Р.~Корепанов$^3$}

\def\autkol{И.\,Н.~Синицын, В.\,И.~Синицын, Э.\,Р.~Корепанов}


\titel{\tit}{\aut}{\autkol}{\titkol}

{\renewcommand{\thefootnote}{\fnsymbol{footnote}} \footnotetext[1]
{Работа выполнена при поддержке РФФИ (проект 15-07-02244).}}


\renewcommand{\thefootnote}{\arabic{footnote}}
\footnotetext[1]{Институт проблем
информатики Российской академии наук, sinitsin@dol.ru}
\footnotetext[2]{Институт проблем
информатики Российской академии наук, vsinitsin@ipiran.ru}
\footnotetext[3]{Институт проблем
информатики Российской академии наук, ekorepanov@ipiran.ru}


\vspace*{-6pt}

\Abst{Рассматриваются  дифференциальные стохастические системы (СтС),
в~том числе и~на многообразиях, с~винеровскими и~пуассоновскими шумами
и~со сложными иррациональными нелинейностями (СИРН). Такие модели  описывают
поведение многих современных нано- и~квантовооптических  технических средств
информатики. Приводятся уравнения методов нормальной аппроксимации (МНА)
и~статистической линеаризации (МСЛ) для аналитического моделирования нестационарных
и~стационарных нормальных процессов. Рассматриваются методы вычисления типовых
интегралов для детерминированных и~стохастических одно-
и~многомерных  СИРН скалярного и~векторного аргумента. Отмечается
возможность использования цилиндрических функций для аналитического
расчета интегралов. Обсуждается алгоритмическое обеспечение аналитического
и~статистического моделирования. Приводится 7~тестовых примеров для типовых
СИРН. Рассматривается возможность использования МСЛ для нормализации гиббсовских
распределений и~распределений с~инвариантной мерой для СтС  с~СИРН.}

\KW{аналитическое и статистическое моделирование;
гиббсовское распределение; метод нормальной аппроксимации (МНА);
метод статистической линеаризации (МСЛ); распределение с~инвариантной мерой;
сложные иррациональные нелинейности (СИРН);
сложные конечные, дифференциальные и~интегральные нелинейности;
стохастические системы (СтС);
цилиндрические функции}

\vspace*{-6pt}

\DOI{10.14357/19922264150101}


\vskip 10pt plus 9pt minus 6pt

\thispagestyle{headings}

\begin{multicols}{2}

\label{st\stat}


\section{Введение}

Вопросам аналитического и~статистического моделирования нормальных (гауссовских)
стохастических процессов (СтП) в~нелинейных СтС (в~том числе на многообразиях)
на основе МНА и~МСЛ посвящена обширная литература (см.\ обзоры в~[1--5]).
В~[6, 7] дано развитие МНА (МСЛ) для СтС со сложными конечными дифференциальными
и~интегральными нелинейностями.

Рассмотрим обобщение~[6, 7] на случай СтС на многообразиях, содержащих одно-
и~многомерные детерминированные и~стохастические СИРН скалярного
и~векторного аргумента.
Статья состоит из введения, четырех разделов и~заключения. В~разд.~2
приведены исходные дифференциальные стохастические уравнения (понимаемые
в~смысле Ито), а~также уравнения МНА (МСЛ) для СтС с~детерминированными
и~стохастическими СИРН. Алгоритмическое обеспечение  аналитического
и~статистического моделирования рассмотрено в~разд.~3. Типовые интегралы
и~тестовые примеры для типовых СИРН содержатся в~разд.~4. В~разд.~5
обсуждаются вопросы моделирования нормальных гиббсо\-вских и~распределений
с~инвариантной мерой в~гамильтоновых СтС с~линейной диссипацией.

\section{Уравнения нормальной аппроксимации (статистической линеаризации)
для~систем со~сложными иррациональными нелинейностями}

Как известно~\cite{1-sin, 2-sin},  уравнения конечномерных непрерывных
нелинейных систем со стохастическими возмущениями путем расширения вектора
состояния СтС могут быть записаны в~виде сле\-ду\-юще\-го векторного стохастического
дифференциального уравнения Ито:
\begin{multline}
dY_t = a(Y_t, t)\, dt + b (Y_t, t) \,dW_0+ {}\\
{}+\iii_{R_0} c (Y_t, t, v) P^0 (dt, dv)\,,\enskip Y(t_0) = Y_0\,.\label{e2.1-sin}
\end{multline}
Здесь $Y_t$~--- $(p\times 1)$-мер\-ный вектор состояния, $Y_t \hm\in \Delta_y$
($\Delta_y$~--- многообразие состояний);  $a\hm=a(Y_t, t)$ и~$b\hm=b(y_t, t)$~---
известные  $(p\times 1)$-мер\-ная и  $(p\times m)$-мер\-ная функции~$Y_t$ и~$t$;
$W_0\hm= W_0(t)$~--- $(r\times 1)$-мер\-ный винеровский СтП интенсивности
$\nu_0 \hm= \nu_0(t)$; $c(Y_t, t, v)$~--- $(p\times 1)$-мер\-ная функция
$Y_t, t$ и~вспомогательного $(q\times 1)$-мер\-но\-го параметра~$v$;
$\iii\limits_{\Delta} dP^0 (t, A)$~--- центрированная пуассоновская мера,
опре\-де\-ля\-емая следующим образом:
\begin{equation*}
\iii\limits_{\Delta} dP^0 (t, A) = \iii\limits_{\Delta} dP (t,A) =\iii\limits_{\Delta} \nu_P (t,A)\, dt\,.
%\label{e2.2-sin}
\end{equation*}
В~(\ref{e2.1-sin}) принято: $\iii\limits_{\Delta}$~---
число скачков пуассоновского СтП
в~интервале времени  $\Delta \hm= (t_1, t_2]$; $\nu_P (t, A)$~---
интенсивность пуассоновского СтП  $P(t,A)$; $A$~--- некоторое
борелевское множество пространства~$R_0^q$ с~выколотым началом.
Начальное значение~$Y_0$ представляет собой случайную величину
(с.в.), не зависящую от приращений $W_0(t)$ и~$P(t,A)$ на
интервалах времени, следующих за  $t_0, t_0\hm \le t_1\hm\le t_2$, для
любого множества~$A$.

В случае аддитивных нормальных (гауссовских) и~обобщенных пуассоновских
возмущений уравнение~(\ref{e2.1-sin}) имеет вид:
\begin{equation}
\dot Y = a(Y_t,t)+ b_0 (t) V\,, \enskip
V = \dot W\,,\enskip Y(t_0) = Y_0\,.\label{e2.3-sin}
\end{equation}
Здесь $W$~--- СтП с~независимыми приращениями, представляющий собой смесь
нормального и~обобщенного пуассоновского СтП.

Для компонент $\vrp(Y_t, t) \hm= \{a_h, b_{kj}, c_h\}$ функций $a$, $b$ и~$c$,
являющихся СИРН, примем следующие типовые представления:
\begin{align}
\vrp(Y_t, t) &= \sss\limits_{r=1}^p l_{rt}^\vrp \left\vert
Y_{rt}+d_{rt}\right\vert^{\alpha_{r}^\vrp} \rho_r (Y_t)\,;\label{e2.4-sin}\\
\vrp(Y_t, t) &= \sss\limits_{r=1}^p l_{r,ht}^\vrp
\prod\limits_{h=1}^p \left\vert Y_{ht}+d_{ht}\right\vert^{\alpha_{h,r}^\vrp}
\rho_{h,r} (Y_t)\,,\label{e2.5-sin}
\end{align}
где $l_{rt}^\vrp$, $l_{r,ht}^\vrp$, $\alpha_r^\vrp$, $\alpha_{h,r}^\vrp$,
$d_{rt}$, $d_{ht}$ и $\rho_{r}(Y_t)$, $\rho_{h,r}(Y_t)$~---
параметры и~структурные функции СИРН.

Если предположить существование конечных вероятностных моментов
второго порядка для моментов времени~$t_1$ и~$t_2$, то уравнения
МНА примут следующий вид~\cite{2-sin, 3-sin}:
\begin{itemize}
\item  для характеристических функций:
    \begin{align}
       \hspace*{-8mm}g_1^N (\lambda;t) &=\exp \lk i\lambda^{\mathrm{T}} m_t - \fr{1}{2}\,
     \lambda^{\mathrm{T}} K_t \lambda\rk\,;\label{e2.6-sin}\\
   \hspace*{-8mm}g_{t_1, t_2}^N (\lambda_1, \lambda_2;t_1, t_2 ) &=\exp \lk i\bar \lambda^{\mathrm{T}}
    \bar m_2 - \fr{1}{2}\, \bar \lambda^{\mathrm{T}} \bar K_2 \lambda\rk\,,\label{e2.7-sin}
    \end{align}
где
    $$
    \bar \lambda =\lk \lambda_1^{\mathrm{T}}\lambda_2^{\mathrm{T}}\rk^{\mathrm{T}}\,;
    \quad\bar m_2 = \lk m_{t_1}^{\mathrm{T}}\, m_{t_2}^{\mathrm{T}}\rk^{\mathrm{T}}\,;
    $$
    $$
    \bar K_2 \begin{bmatrix}
    K(t_1, t_1)& K(t_1, t_2)\\
    K(t_2, t_1)& K(t_2, t_2)\end{bmatrix};
    $$
\item  для математических ожиданий~$m_t$, ковариационной матрицы~$K_t$
и~матрицы ковариационных функций $K(t_1, t_2)$:
  \begin{gather}
  \dot m_t = a_1 (m_t, K_t, t)\,,\enskip m_0 = m(t_0)\,;\label{e2.8-sin}\\
\dot K_t = a_2 (m_t, K_t, t)\,,\enskip K_0 = K(t_0)\,;\label{e2.9-sin}
\end{gather}
\begin{equation}
\left.
\begin{array}{rl}
\hspace*{-8mm}\fr{\prt K(t_1, t_2)}{\prt t_2 }&= K(t_1, t_2) a_{21} (m_{t_2}, K_{t_2}, t_2)^{\mathrm{T}}\,,
\\[6pt]
\hspace*{-8mm} K(t_1, t_1) &= K_{t_1}\,.
 \end{array}
 \right\}
 \label{e2.10-sin}
\end{equation}
    \end{itemize}
Здесь приняты следующие обозначения:
    \begin{align}
a_1 &= a_1 (m_t, K_t, t) = M_{\Delta_y}^N a (Y_t, t)\,;\label{e2.11-sin}\\
a_2 &= a_2 (m_t, K_t, t) = a_{21} (m_t, K_t, t)+{}\notag\\
&\hspace*{5mm}{}+ a_{21} (m_t, K_t, t)^{\mathrm{T}} +
a_{22}(m_t, K_t, t)\,;\label{e2.12-sin}\\
a_{21} &= a_{21}(m_t, K_t, t)=  M_{\Delta_y}^N a(Y_t, t) Y_{t}^{0T}\,;\label{e2.13-sin}\\
a_{22} &= a_{22}(m_t, K_t, t)= M_{\Delta_y}^N \si (Y_t, t)\,;\label{e2.14-sin}
\\
\sigma (Y_t, t) &= b(Y_t, t) \nu_0(t) b(Y_t, t)^{\mathrm{T}} +{}\notag\\
&\hspace*{5mm}{}+\iii\limits_{R_0^q}
c (Y_t, t, v) c(Y_t, t,v)^{\mathrm{T}} \nu_P (t, dv)\,;\label{e2.15-sin}
\\
m_t &= M_{\Delta_y}^N Y_t\,,\enskip
Y_t^0 = Y_t - m_t\,;\notag\\
K_t &= M_{\Delta_y}^N Y_0^0 Y_t^{0\mathrm{T}}\,;\enskip
K(t_1, t_2) = M_{\Delta_y}^N Y_{t_1}^{0} Y_{t_2}^{0T}\,;\notag
%\label{e2.16-sin}
\end{align}
$M_{\Delta_y}^N$~--- символ вычисления математического ожидания для
нормальных распределений~(\ref{e2.6-sin}) и~(\ref{e2.7-sin}).

Для стационарных СтС нормальные стационарные СтП~--- если они
существуют, то  $m_t\hm =\bar m$, $ K_t\hm =\bar K$, $K(t_1, t_2) \hm=
k(\tau)$ $(\tau \hm= t_1-t_2)$,~--- определяются уравнениями~\cite{2-sin, 3-sin}:
\begin{equation}
a_1 (\bar m, \bar K) =0\,;\quad a_2 (\bar m, \bar K)=0\,;\label{e2.17-sin}
\end{equation}
\begin{equation}
\left.
\begin{array}{rl}
\dot k_\tau (\tau) &= a_{21} (\bar m, \bar K)\bar K^{-1} k(\tau)\,,\\[6pt]
&\hspace*{20mm}k(0) =\bar K \enskip (\forall \tau >0)\,; \\
k(\tau) &= k(-\tau)^{\mathrm{T}} \enskip (\forall\tau <0)\,.
\end{array}
\right\}
\label{e2.18-sin}
\end{equation}
При этом необходимо, чтобы матрица  $a_{21} (\bar m, \bar K)\hm=\bar a_{21}$
была асимптотически устойчивой.

В случае СтС~(\ref{e2.3-sin}) уравнения МНА переходят в~уравнения МСЛ
Казакова~\cite{2-sin, 3-sin}, если принять
\begin{equation}
a(Y_t,t) = a_1 (m_t, K_t) + k_1^a (m_t, K_t) Y_t^0;\label{e2.19-sin}
\end{equation}
\begin{equation}
\left.
\begin{array}{rl}
b(Y_t,t) &= b_0 (t)\,;\\[6pt]
\sigma(Y_t, t)&= b_0(t) \nu(t) b_0(t)^{\mathrm{T}} =
\sigma_0(t)\,;
\end{array}
\right\}
\label{e2.20-sin}
\end{equation}
\begin{equation}
k_1^a (m_t, K_t, t) =\lk \left(\fr{\prt}{\prt m_t} \right)a_0 (m_t, K_t, t)^{\mathrm{T}}\rk^{\mathrm{T}}\,;\label{e2.21-sin}
\end{equation}
\begin{equation}
\dot m_t = a_1 (m_t, K_t, t) \,;\quad m_0 = m(t_0)\,;\label{e2.22-sin}
\end{equation}
\vspace*{-12pt}

\noindent
\begin{multline}
\dot K_t = k_1^a (m_t, K_t, t) K_t + K_t k_1^a (m_t, K_t, t)^{\mathrm{T}} +{}\\
{}+\sigma_0(t)\,,\quad K_0 = K(t_0)\,;
\label{e2.23-sin}
\end{multline}

\vspace*{-12pt}

\noindent
\begin{multline}
\fr{\prt K(t_1, t_2)}{\prt t_2} = K(t_1, t_2) K_{t_2} k_1^a
(m_{t_2}, K_{t_2}, t_2)^{\mathrm{T}}\,,\\
K(t_1, t_2) = K_{t_1}\,.
\label{e2.24-sin}
\end{multline}


Для стационарных СтС~(\ref{e2.3-sin}) при условии асимптотической устойчивости
матрицы $k_1^a (\bar m, \bar K)$ в~основе МСЛ лежат уравнения~(\ref{e2.17-sin})
и~(\ref{e2.18-sin}), записанные в виде:
\begin{gather}
a_1 (\bar m, \bar K) =0\,; \label{e2.25-sin}\\
k_1^a (\bar m, \bar K) \bar K + \bar K k_1^a (\bar m, \bar K)^{\mathrm{T}} +\bar \sigma_0 =0\,;\label{e2.26-sin}
\end{gather}

\noindent
\begin{equation}
\left.
\begin{array}{c}
\!\dot k_\tau (\tau) = k_1^a (\bar m, \bar K)k(\tau)\,,\enskip
k(0) =\bar K \enskip (\forall \tau >0)\,;\\[6pt]
\!k(\tau) = k (-\tau)^{\mathrm{T}} \enskip (\forall \tau <0)\,.
\end{array}\!
\right\}\!
\label{e2.27-sin}
\end{equation}

Уравнения~(\ref{e2.6-sin})--(\ref{e2.10-sin}) лежат в~основе МНА для СтС~(\ref{e2.1-sin}),
а~уравнения~(\ref{e2.19-sin})--(\ref{e2.24-sin})~--- в~основе МСЛ для СтС~(\ref{e2.3-sin}).
Для определения стационарных СтП согласно МНА служат соотношения~(\ref{e2.17-sin}) и~(\ref{e2.18-sin}),
а~МСЛ~--- (\ref{e2.25-sin})--(\ref{e2.27-sin}).

В задачах практики для СтС со стохастическими СИРН  $a$, $b$ и~$c$
являются нормальными случайными функциями состояния~$Y_t$ и~времени:
$$
A=A(Y_t, t)\,;\ B=B(Y_t, t)\,;\  C\hm=C(Y_t, t, v).
$$

При известных
условиях~[1--3] они могут быть представлены в виде соответствующих
канонических разложений по случайным нормальным скалярным с.в.
В~этом случае в представлениях~(\ref{e2.4-sin}) и~(\ref{e2.5-sin}) величины
$l_{rt}^\vrp$ и~$l_{r,ht}^\vrp$ будут независимыми с.в.~$L_{rt}^\vrp$
и~$L_{r,ht}^\vrp$ с~известными математическими ожиданиями
$m_{rt}^{L_\vrp}$ и~$m_{r,ht}^{L_\vrp}$ и~дисперсиями
$D_{rt}^{L_\vrp}$ и~$l_{r,ht}^{L_\vrp}$. Введя расширенный вектор
состояния~$\bar Y_t$, состоящий из~$Y_t$ и~с.в.~$L_{rt}^\vrp$
и~$L_{r,ht}^\vrp$, и~добавив к~уравнениям~(\ref{e2.1-sin}) уравнения $dL_{rt}^\vrp
\hm=0$ и~$dL_{rh,t}^\vrp \hm=0$, придем к~СтС вида~(\ref{e2.1-sin}).

\section{Алгоритмическое обеспечение аналитического и~статистического моделирования}

Как следует из уравнений~(\ref{e2.11-sin})--(\ref{e2.15-sin}),
для МНА необходимо уметь вычислять следующие инте\-гралы:
\begin{multline}
I_0^a = I_0^a (m_t, K_t, t) = a_1 (m_t, K_t, t)={} \\
{}=M_{\Delta_y}^N a(Y_t, t)\,;\label{e3.1-sin}
\end{multline}
\noindent
\begin{align}
I_1^a &= I_1^a (m_t, K_t, t)= a_{21}(m_t, K_t, t)={}\notag\\
&\hspace*{30mm}{}=M_{\Delta_y}^N a(Y_t , t) Y_t^{0\mathrm{T}}\,;\label{e3.2-sin}\\
I_0^{\bar \sigma} &= I_0^{\bar \sigma} (m_t, K_t, t) = a_{22}(m_t, K_t, t) ={}\notag\\
&\hspace*{30mm}{}=M_N \bar \sigma (Y_t, t)\,.\label{e3.3-sin}
\end{align}

Для МСЛ достаточно вычислить интеграл~(\ref{e3.1-sin}), причем интеграл~$I_1^a$
вычисляется по формуле~[2--4]:
\begin{equation*}
k_1^a = k_1^a (m_t, K_t, t)=\lk \left( \fr{\prt}{\prt m_t}\right)
I_0^a (m_t, K_t, t)^{\mathrm{T}}\rk^{\mathrm{T}}\,. %\label{e3.4-sin}
\end{equation*}

В основе аналитического вычисления интегралов~(\ref{e3.1-sin})--(\ref{e3.3-sin})
лежат методы, основанные на свойствах цилиндрических функций~[8--10].
Для чис\-лен\-ных расчетов этих интегралов используются \mbox{методы}~\cite{8-sin}.

 Важно иметь в виду, что уравнения МНА (МСЛ) содержат интегралы
 $I_0^a$, $I_1^a$ и~$I_0^\sigma$ в~виде соответ\-ст\-ву\-ющих коэффициентов.
 Поэтому процедура вы\-числе\-ния интегралов должна быть согласована с~мето\-дом
 численного решения обыкновенных дифференциальных уравнений для~$m_t$, $K_t$
 и~$K(t_1, t_2)$ Эти коэффициенты допускают дифференцирование по~$m_t$ и~$K_t$,
 так как под интегралом стоит сглаживающая нормальная плотность.

В~\cite{5-sin} изложены общие алгоритмы аналитического и статистического
моделирования распределений, в~том числе нормальных в~нелинейных
СтС на многообразиях. Алгоритмы аналитического, статистического
моделирования для СтС с~СИРН, а~также смешанные алгоритмы различной
степени точности относительно шага интегрирования представлены в~\cite{5-sin}.

\section{Тестовые примеры}

\textbf{1.}
Рассмотрим вычисление интегралов~(\ref{e3.1-sin}) и~(\ref{e3.2-sin})
для одномерных СИРН скалярного аргумента
\begin{equation}
\vrp (Y_t, t) =\left\vert Y_t\right\vert^{\alpha-1}\, \mathrm{sign}\, Y_t
\label{e4.1-sin}
\end{equation}
($\alpha$~--- нецелый показатель).

Пользуясь~(\ref{e2.19-sin}), представим~(\ref{e4.1-sin}) в~виде
\begin{equation*}
\vrp(Y_t, t) = \vrp_0 (m_t, D_t, t) + k_1^\vrp(m_t, D_t, t) Y_t^0\,.
%\label{e4.2-sin}
\end{equation*}
Здесь введены следующие обозначения:
\begin{align}
\vrp_0(m_t, D_t, t) &={}\notag\\
&\hspace*{-10mm}{}=\sqrt{\fr{2}{\pi}}\,\Gamma(\alpha)
    D_t^{(\alpha-1)/2} e^{-\xi_t^2/4} D_{-\alpha} (-\xi_t)\,;\label{e4.3-sin}\\
k_1^\vrp (m_t, D_t, t) &= \fr{\prt \vrp_0(m_t, D_t, t)}{\prt m_t}={}\notag\\
&\hspace*{-10mm}{}=
\sqrt{\fr{2}{\pi}}\,\Gamma(\alpha) D_t^{(\alpha/2)-1} e^{-\xi_t^2/4} D_{1-\alpha}
(-\xi_t) \,,\label{e4.4-sin}
\end{align}
где  $\Gamma(\alpha)$~--- гамма-функ\-ция,  $\xi_t \hm= m_t/\sqrt{D_t}$~---
отношение <<сигнал--шум>>; $D_{-\alpha} (\xi_t)$~--- функция параболического
цилиндра~\cite{9-sin}.
При вычислении были учтены следующие соотношения~\cite{9-sin, 8-sin}:
   \begin{multline}
   \iii\limits_0^\infty x^{\alpha-1} e^{-\beta x^2 - \gamma x}\,dx ={}\\
   {}= (2\beta)^{-\alpha/2} \Gamma(\alpha) \exp
   \left(\fr{\gamma^2}{8\beta}\right) D_{-\alpha}
   \left(\fr{\gamma}{\sqrt{2\beta}}\right)\\
   (\mathrm{Re}\,\beta>0\,,\enskip \mathrm{Re}\,\alpha >0)\,;\label{e4.5-sin}
   \end{multline}

   \vspace*{-12pt}

   \noindent
   \begin{multline}
\hspace*{-3mm}\fr{dD_{\nu}(z)}{dz} =
   -\fr{z}{2}\,D_{\nu} (z) - D_{\nu-1} (z) = \fr{z}{2}\, D_{\nu} (z) -
   D_{\nu+1} (z) \\
(\mathrm{Re}\, \beta>0\,,\enskip \mathrm{Re}\,\alpha>0\,,\enskip \nu=-\alpha)\,.
\label{e4.6-sin}
\end{multline}

Соотношения~(\ref{e4.5-sin}) и~(\ref{e4.6-sin}) могут быть использованы
 также для вычисления интегралов~(\ref{e3.3-sin}).

Применяя известное асимптотическое разложение для
функции параболического цилиндра  $D_\nu(z)$~\cite{9-sin, 8-sin}
\begin{multline*}
D_\nu (z) \approx e^{-z^2/4} z^\nu \left[
1- \fr{\nu(\nu-1)}{2z^2} +{}\right.\\
\left.{}+ \fr{\nu(\nu-1) (\nu-2) (\nu-3)}{2\cdot 4 z^2} +\cdots\right]
\\
\left( |z|\gg1\,, \enskip |z|\gg\nu\right)\,,
\label{e4.7-sin}
\end{multline*}
придем к следующим асимптотическим выражениями для
$\vrp_0 (\xi_t, D_t)$ и~$k_1^\vrp (\xi_t, D_t)$:
  \begin{align*}
  \vrp_0 (\xi_t, D_t) &= \sqrt{\fr{2}{\pi}}\, \Gamma(\alpha)
  D_t^{(\alp-1)/2} e^{-\xi_t^2/2} |\xi_t|^{-\alpha}\left[
  \vphantom{\fr{\alpha(\alpha+1)(\alpha+2)(\alpha+3)}{8\xi_t^4}}
  1-{}\right.\notag\\
&\hspace*{-15mm}\left.{}- \fr{\alpha(\alpha+1)}{2\xi_t^2}+
\fr{\alpha(\alpha+1)(\alpha+2)(\alpha+3)}{8\xi_t^4}+\cdots\right]>0\,; \\ %\label{e4.8-sin}\\
k_1^\vrp (\xi_t, D_t) &= \fr{\prt \vrp_0 (\xi_t, D_t)}{\prt \xi_t}\,
\fr{1}{\sqrt{D_t}}\,. %\label{e4.9-sin}
\end{align*}

\textbf{2.}
Для одномерной СИРН
   \begin{equation}
   \vrp (Y_t) =\left\vert Y_t\right\vert^{\alpha-1} \mathbf{1}\, (Y_t)\label{e4.10-sin}
   \end{equation}
имеем
      \begin{align*}
   \vrp_0 (\xi_t, D_t) &= \fr{1}{\sqrt{2\pi}}\, \Gamma (\alpha+1) D_t^{\alpha/2}
   e^{-\xi_t^2/4} D_{-(\alpha+1)} (-\xi_t);\\ %\label{e4.11-sin}\\
k_1^\vrp (\xi_t, D_t) &=\fr{\prt \vrp_0 (\xi_t, D_t)}{\prt \xi_t}\,
\fr{1}{\sqrt{D_t}}. %\label{e4.12-sin}
\end{align*}

\textbf{3.}
Для нелинейной одномерной дифференциальной СтС с~СИРН вида
\begin{equation}
\dot Y_t = a_{0t} - a_t |Y_t|^{\alpha-1} \mathrm{sign}\, Y_t + b_t V
    \label{e4.13-sin}
    \end{equation}
($\alpha$, $a_t$, $b_t$~--- коэффициенты;
$V$~--- нормальный белый шум интенсивности~$\nu_t$) МСЛ приводит
к~сле\-ду\-ющим уравнениями:
\begin{align*}
\dot m_t &= a_{0t} - a_t \vrp_0 (m_t, D_t)\,;\\ %\label{e4.14-sin}\\
\dot D_t &= -2a_t k_1^\vrp (m_t, D_t) + b_t^2 \nu_t\,. %\label{e4.15-sin}
\end{align*}
Здесь $\vrp_0$ и $k_1^\vrp$ определены по~(\ref{e4.3-sin}) и~(\ref{e4.4-sin}).

Для статистической системы~(\ref{e4.13-sin}) при
$a_t \hm=\bar a\hm>0$ имеет место стационарный режим, для которого
$m_t\hm=\bar m$, $D_t \hm= \bar D$ определяются из уравнений
   \begin{equation*}
   \bar a_0 - \bar a \vrp_0 (\bar m, \bar D) =0\,;\quad
   -2\bar a k_1^\vrp (\bar m, \bar D) + \bar b^2 \bar \nu =0\,. %\label{e4.16-sin}
   \end{equation*}

\textbf{4.}
Для одномерной дифференциальной СтС с~СИРН и~параметрическим шумом
\begin{equation*}
\dot Y_t = a_{0t} - a_t Y_t + b_t \left( \left\vert Y_t\right\vert^{\delta-1}
\mathrm{sign}\, Y_t\right) V
%\label{e(4.17-sin}
\end{equation*}
с~учетом результатов примера~1 согласно МНА \mbox{имеем}:
\begin{align*}
\dot m _t &= a_{0t} - a_t m_t\,;\\ %\label{e4.18-sin}\\
\dot D_t &=-2a_t D_t + b_t^2 \nu_t \vrp_0 (m_t, D_t)\,. %\label{e4.19-sin}
\end{align*}
Здесь $\vrp_0$ определяется по~(\ref{e4.3-sin}) при $\delta \hm= 2\alpha \hm-1$.

\textbf{5.}
Для некоторых типовых одномерных нелинейных дифференциальных СтС с~СИРН вида
\begin{equation*}
\dot Y_t =\psi (Y_t) +b_t V\,, %\label{e4.20-sin}
\end{equation*}
допускающих стационарный режим $m_t \hm= \bar m \hm=0$, $D_t \hm= \bar D \hm\ne 0$,
выражения для  $\bar \psi_0 \hm=\bar\psi_0 (0, \bar D)$ имеют соответственно вид:
    \begin{align}
    \psi (Y_t) &= Y_t (Y_t^2 + d^2)^{-1/2};\notag\\
\bar \psi_0 &= \sqrt{\fr{\pi}{\mu}}\, e^{d^2\mu} \lk 1-\tilde\Phi (d\sqrt{\mu})\rk\,,
\label{e4.21-sin}
\end{align}
где
\begin{gather}
\tilde\Phi (x) = \fr{2}{\sqrt{\pi}}
\iii\limits_0^x e^{-t^2}dt\,,\quad
\mu =\fr{1}{2\bar D};\notag
\\
\psi(Y_t) = Y_t^{\alpha-1} \exp (-\gamma Y_t^{-2})\,;\notag\\
\bar \psi_0 = 2 \left(\fr{\gamma}{\beta}\right)^{\alpha/4} K_{\alpha/2}
(2\sqrt{\beta\gamma})\,,
\label{e4.22-sin}
\end{gather}
где $K_z (x)$~--- цилиндрическая функция мнимого аргумента, $\beta \hm=
1/(2\bar D)$;
\begin{gather}
\psi (Y_t) = Y_t^{\mu-1} \sin\gamma Y_t\,;\notag\\
\bar\psi_0 = \gamma e^{-\gamma^2/4\beta} \Gamma
\left(\fr{1+\mu}{2}\right){}_1\! F_1 \left(-\fr{\mu}{2};\fr{3}{2};\fr{\gamma^2}{4\beta}\right)\,,
\label{e4.23-sin}
\end{gather}
где ${ }_1\! F_1$~--- вырожденная гипергеометрическая функция, $\beta =
{1}/{\bar D} >0$, Re $\mu>-1$;
    $$\psi (Y_t) = Y_t^{\mu-1} \cos a Y_t\,;$$
\begin{equation}
\bar\psi_0 = \fr{\Gamma\left({\mu}/{2}\right)}
{\beta^{\mu/2}}{ }_1\! F_1\left(\fr{\mu}{2};\fr{1}{2};-\fr{a^2}{4\beta}\right)
\label{e4.24-sin}
\end{equation}
при
$\beta ={1}/({2\bar D})$, $\mathrm{Re}\, \mu\hm>0$,
$a\hm>0$;
      $$\psi (Y_t) = Y_t^{2\mu-1} \mathrm{sh}\,\gamma Y_t\,;$$

      \vspace*{-12pt}

      \noindent
\begin{multline}
\bar\psi_0 = \Gamma (2\mu) (2\beta)^{-\mu} \exp \left(
\fr{\gamma^2}{8\beta}\right) \left[ D_{-2\mu} \left( -\fr{\gamma}{\sqrt{2}\beta}\right) -
{}\right.\\
\left.{}- D_{-2\mu}  \left( \fr{\gamma}{\sqrt{2}\beta}\right) \right]
\label{e4.25-sin}
\end{multline}
при
$\beta ={1}/({2 \bar D})$, $\mu\hm>-0{,}5$;
   $$\psi (Y_t) = Y_t^{2\mu-1} \mathrm{ch}\,\gamma Y_t\,;$$


\vspace*{-12pt}

   \noindent
   \begin{multline}
\bar\psi_0 = \Gamma (2\mu) (2\beta)^{-\mu} \exp \left(
\fr{\gamma^2}{8\beta}\right) \left[ D_{-2\mu} \left( -\fr{\gamma}
{\sqrt{2}\beta}\right) +{}\right.\\
\left.{}+ D_{-2\mu}  \left( \fr{\gamma}{\sqrt{2}\beta}\right) \right]
\label{e4.26-sin}
\end{multline}
при
$\beta ={1}/({2 \bar{D}})$, $\mu\hm>0$.



\textbf{6.}
Статистическая линеаризация одномерной СИРН двумерного аргумента
$Y_t\hm=\lk Y_{1t} \, Y_{2t}\rk^{\mathrm{T}}$ для независимых нормальных $Y_{it}$
$(i\hm=1,2)$ определяется следующими выражениями:
\begin{equation*}
\vrp (Y_{1t}, Y_{2t}) =\vrp_0 + k_{11}^\vrp Y_1^0 +
k_{12}^\vrp Y_2^0\,,\enskip k_{1i}^\vrp =\fr{\prt \vrp_0}
{\prt m_{it}}\,. %\label{e4.27-sin}
\end{equation*}
Здесь $\vrp_0$, $k_{it}^\vrp$ зависят от~$m_{it}$
и~$D_{it}$. Выражения для~$\vrp_0$ в~случаях~(\ref{e4.1-sin}), (\ref{e4.10-sin}),
(\ref{e4.21-sin})--(\ref{e4.26-sin}) получаются путем перемножения
соответствующих выражений для~$Y_{it}$. Аналогично рассматривается случай СИРН
$n$-мер\-но\-го аргумента $(n\hm>2)$.

\textbf{7.}
Пусть одномерная гауссовская стохастическая СИРН аппроксимирована
согласно~\cite{3-sin} отрезком канонического разложения
\begin{equation*}
A(Y_t) =\sss\limits_{j=1}^{n_j} L_j \left\vert Y_t\right\vert^{\alpha_j -1}
\mathrm{sign}\, Y_t\,, %\label{e4.28-sin}
\end{equation*}
где $L_j$~--- независимые между собой и от $Y_t$ гауссовские с.в.\
с~математическими ожиданиями $m^{L_j}$ и~дисперсиями $D^{L_j}$. Тогда
МСЛ приводит к~сле\-ду\-юще\-му представлению:
\begin{equation*}
A(Y_t) = \vrp_0^A + k_1^A Y_t^0 +\sss\limits_{j=1}^{n_j} k_1^{L_j} L_j^0\,.
%\label{e4.29-sin}
\end{equation*}
Здесь коэффициенты $\vrp_0^A$, $k_1^{L_j}$ и~$k_1^A$ зависят
от $m_t^y$, $m^{L_j}$, $D_t^y$ и~$D^{L_j}$ и~вычисляются по формулам примера~1.


\section{Применение к~моделированию нормальных распределений с~инвариантной мерой}

\vspace*{-2pt}

В~\cite{11-sin} описан ряд многомерных гамильтоновских систем типа Холта,
Фо\-ки\-са--Ла\-гер\-стре\-ма и~др., содержащих СИРН. Для аналитического
моделирования нормальных процессов, аппроксимирующих гиббсовские
и~распределения с~инвариантной мерой, если учесть линейные диссипативные силы
и~нелинейные и~параметрические  стохастические возмущения,  могут быть
непосредственно применимы методы~\cite{1-sin, 2-sin, 12-sin}.

\vspace*{-12pt}

\section{Заключение}

\vspace*{-2pt}

Рассматриваются  дифференциальные СтС, в~том числе и~на многообразиях,
с~винеровскими и~пуассоновскими шумами и~с~СИРН. Такие модели описывают
поведение многих современных на\-но- и~квантовооптических  технических средств
информатики.

Приводятся уравнения МНА
и~МСЛ для аналитического моделирования нестационарных
и~стационарных нормальных процессов.

Рассматриваются методы вычисления типовых интегралов для детерминированных
и~стохастических одно- и~многомерных  СИРН скалярного и~векторного аргумента.
Отмечается возможность использования цилиндрических функций для аналитического
расчета интегралов. Обсуждается алгоритмическое обеспечение аналитического
и~статистического моделирования.

Приводятся 7~тес\-то\-вых примеров для типовых СИРН.

Рассматривается возможность использования МСЛ для нормализации
гиббсовских распределений и~распределений с~инвариантной мерой для СтС  с~СИРН.

Результаты допускают обобщение на случай дискрет\-ных, интегродифференциальных
и~операторных СтС и~СИРН, приводимых к~дифференциальным, в~том числе
с~автокоррелированными шумами. Как отмечалось в~\cite{6-sin, 7-sin},
особый интерес представляет развитие МНА (МСЛ) на случай сложных алгебраических
и~трансцендентных нелинейностей.

\vspace*{-9pt}

{\small\frenchspacing
 {%\baselineskip=10.8pt
 \addcontentsline{toc}{section}{References}
 \begin{thebibliography}{99}

 \vspace*{-2pt}

\bibitem{1-sin}
\Au{Пугачев В.\,С., Синицын И.\,Н.}
Стохастические дифференциальные системы. Анализ и фильтрация.~---
М.: Наука,  1990.  632~с. (\Au{Pugachev V.\,S., Sinitsyn I.\,N.}\linebreak\vspace*{-12pt}
\pagebreak

\noindent
Stochastic differential systems.
Analysis and filtering.~--- Chichester, New York: Jonh Wiley, 1987.
549~p.)

\bibitem{2-sin}
\Au{Пугачев В.\,С., Синицын И.\,Н.}
Теория стохастических систем.~--- М.: Логос, 2000; 2004. 1000~с.
(\Au{Pugachev V.\,S., Sinitsyn I.\,N.}
Stochastic systems. Theory and  applications.~---
Singapore: World Scientific, 2001. 908~p.)

\bibitem{3-sin}
\Au{Синицын И.\,Н.}
Канонические представления случайных функций и их применение в~задачах
компьютерной поддержки научных исследований.~--- М.: ТОРУС
ПРЕСС, 2009. 768~с.


\bibitem{4-sin}
\Au{Синицын И.\,Н.,  Синицын В.\,И.}
Лекции по нормальной и эллипсоидальной аппроксимации распределений
в~стохастических системах.~--- М.: ТОРУС ПРЕСС, 2013. 488~с.

\bibitem{5-sin}
\Au{Синицын И.\,Н.}
Параметрическое статистическое и~аналитическое моделирование распределений
в~нелинейных стохастических системах на многообразиях~//
Информатика и~её применения, 2013. Т.~7. Вып.~2. С.~4--16.
\bibitem{6-sin}
\Au{Синицын И.\,Н., Синицын В.\,И.}
Аналитическое моделирование нормальных процессов в~стохастических системах
со сложными нелинейностями~// Информатика и~её применения, 2014. Т.~8.
Вып.~3. С.~2--4.

\bibitem{7-sin}
\Au{Синицын И.\,Н., Синицын В.\,И., Сергеев~И.\,В., Белоусов~В.\,В., Шоргин~В.\,С.}
Математическое обеспечение аналитического моделирования стохастических систем
со сложными нелинейностями~// Системы и~средства информатики, 2014.
Т.~24. №\,3. С.~4--29.



\bibitem{9-sin}
\Au{Градштейн И.\,С., Рыжик И.\,М.}
Таблицы интегралов, сумм, рядов и~произведений.~--- М.: ГИФМЛ, 1963. 1100~с.

\bibitem{10-sin}
Справочник по специальным функциям~/ Под ред. М.~Абрамовича и~И.~Стигана.~---
М.: Наука, 1979. 832~с.

\bibitem{8-sin}
\Au{Попов Б.\,А., Теслер Г.\,С.}
Вычисление функций на ЭВМ: Справочник.~--- Киев: Наукова Думка, 1984. 599~с.

\bibitem{11-sin}
\Au{Переломов А.\,М.}
Интегрируемые системы классической механики и~ал\-геб\-ры Ли.~---
М.: Наука, 1990. 240~с.

\bibitem{12-sin}
\Au{Синицын И.\,Н.}
Аналитическое моделирование распределений с~инвариантной мерой
в~стохастических системах с~разрывными характеристиками~// Информатика
и~её применения, 2013. Т.~7. Вып.~1. С.~3--11.
 \end{thebibliography}

 }
 }

\end{multicols}

\vspace*{-3pt}

\hfill{\small\textit{Поступила в редакцию 15.01.15}}

%\newpage

\vspace*{8pt}

\hrule

\vspace*{2pt}

\hrule

%\vspace*{12pt}

\def\tit{ANALYTICAL MODELING OF~NORMAL
PROCESSES IN~STOCHASTIC SYSTEMS WITH~COMPLEX IRRATIONAL NONLINEARITIES}

\def\titkol{Analytical modeling of~normal
processes in~stochastic systems with~complex irrational  nonlinearities}

\def\aut{I.\,N.~Sinitsyn, V.\,I.~Sinitsyn, and~E.\,R.~Korepanov}

\def\autkol{I.\,N.~Sinitsyn, V.\,I.~Sinitsyn, and~E.\,R.~Korepanov}

\titel{\tit}{\aut}{\autkol}{\titkol}

\vspace*{-9pt}

 \noindent
Institute of Informatics Problems, Russian Academy of Sciences,
44-2 Vavilov Str., Moscow 119333, Russian Federation


\def\leftfootline{\small{\textbf{\thepage}
\hfill INFORMATIKA I EE PRIMENENIYA~--- INFORMATICS AND
APPLICATIONS\ \ \ 2015\ \ \ volume~9\ \ \ issue\ 1}
}%
 \def\rightfootline{\small{INFORMATIKA I EE PRIMENENIYA~---
INFORMATICS AND APPLICATIONS\ \ \ 2015\ \ \ volume~9\ \ \ issue\ 1
\hfill \textbf{\thepage}}}

\vspace*{3pt}

\Abste{Stochastic systems (including manifolds) with Wiener and
Poisson noises and complex irrational nonlinearities (CIRN) are considered.
Equations and algorithms of analytical modeling based on the normal approximation
method (NAM) and the statistical linearization method (SLM) are given. Typical
integrals and software based on cylindrical functions for computing deterministic
and stochastic CIRN are presented. Seven test examples for typical CIRN are given.
Applications to Gibbs distributions and distributions with invariant measure are
discussed.}


\KWE{analytical and statistical modeling; complex irrational nonlinerarity (CIRN);
normal approximation method (NAM);  statistical linearization method (SLM); test
examples }


\DOI{10.14357/19922264150101}

\vspace*{-9pt}

\Ack
\noindent
The research was supported by the Russian
Foundation for Basic Research (project 15-07-02244).




%\vspace*{3pt}

  \begin{multicols}{2}

\renewcommand{\bibname}{\protect\rmfamily References}
%\renewcommand{\bibname}{\large\protect\rm References}



{\small\frenchspacing
 {%\baselineskip=10.8pt
 \addcontentsline{toc}{section}{References}
 \begin{thebibliography}{99}


\bibitem{1-sin-1}
\Aue{Pugachev, V.\,S., and  I.\,N.~Sinitsyn}.  1987.
\textit{Stochastic differential systems. Analysis and filtering.}
Chichester, New York: Jonh Wiley. 549~p.

\bibitem{2-sin-1}
\Aue{Pugachev, V.\,S., and I.\,N.~Sinitsyn}. 2001.
\textit{Stochastic systems. Theory and  applications.}
Singapore: Worls Scientific. 908~p.

\bibitem{3-sin-1}
\Aue{Sinitsyn, I.\,N.} 2009.
Kanonicheskie predstavleniya sluchaynykh funktsiy i~ikh primenenie
v~zadachakh kom\-p'yu\-ter\-noy podderzhki nauchnykh issledovaniy
[Canonical expansions of random functions and their application to scientific
computer-aided support]. Moscow: TORUS PRESS. 768~p.

\bibitem{4-sin-1}
\Aue{Sinitsyn, I.\,N., and  V.\,I.~Sinitsyn}.  2013.
Lektsii po normal'noy i ellipsoidal'noy approksimatsii raspredeleniy
v~stokhasticheskikh sistemakh [Lectures on normal and ellipsoidal
approximation of distributions in stochastic systems]. Moscow: TORUS PRESS. 488~p.

\bibitem{5-sin-1}
\Aue{Sinitsyn, I.\,N.}  2013.
Parametricheskoe statisticheskoe i~analiticheskoe modelirovanie raspredeleniy
v~nelineynykh stokhasticheskikh sistemakh na mnogo\-ob\-ra\-zi\-yakh
[Parametric statistical and analytical modeling of distributions in
stochastic systems on manifolds].
\textit{Informatika i~ee Primeneniya}~--- \textit{Inform. Appl.} 7(2):4--16.

\bibitem{6-sin-1}
\Aue{Sinitsyn, I.\,N., and V.\,I.~Sinitsyn}. 2014.
 Analiticheskoe modelirovanie normal'nykh protsessov v~sto\-kha\-sti\-che\-skikh
 sistemakh so slozhnymi nelineynostyami [Analytical modeling of normal
  processes in stochastic systems with complex nonlinearities].
  \textit{Informatika i~ee Primeneniya}~--- \textit{Inform. Appl.}  8(3):2--4.

\bibitem{7-sin-1}
\Aue{Sinitsyn, I.\,N., V.\,I.~Sinitsyn, I.\,V.~Sergeev, V.\,V.~Belousov,
and V.\,S.~Shorgin}. 2014.
Matematicheskoe obespechenie analiticheskogo modelirovaniya stokhasticheskikh
sistem so slozhnymi nelineynostyami [Mathematical software for
analytical modeling of stochastic systems with complex nonlinearities].
\textit{Sistemy i~Sredstva Informatiki}~---
\textit{Systems and Means of Informatics}  24(3):4--29.


\bibitem{9-sin-1}
\Aue{Gradshteyn, I.\,S., and I.\,M.~Ryzhik}.  1963.
\textit{Tablitsy integralov, summ, ryadov i~proizvedeniy}
[Tables of integrals, sums, series, and products]. Moscow:  GIFML.  1100~p.

\bibitem{10-sin-1}
Abramovich, M.,  and I.~Stigan, eds. 1979.
\textit{Spravochnik po spetsial'nym funktsiyam} [Handbook on special functions].
Moscow:  Nauka.  832~p.

\bibitem{8-sin-1}
\Aue{Popov, B.\,A., and G.\,S.~Tesler}.  1984.
\textit{Vychislenie funktsiy na EVM. Spravochnik} [Computing of functions].
Kiev: Naukova Dumka.  599~р.

\bibitem{11-sin-1}
\Aue{Perelomov, A.\,M.}  1990.
\textit{Integriruemye sistemy klassicheskoy mekhaniki i~algebry Li}
[Integrable systems of classical mechanics and Li algebra]. Moscow: Nauka.  240~p.

\bibitem{12-sin-1}
\Aue{Sinitsyn, I.\,N.}  2013.
Analiticheskoe modelirovanie raspredeleniy s~invariantnoy meroy
v~stokhasticheskikh sistemakh s~razryvnymi kharakteristikami
[Analytical modeling
of distributions with invariant measure in stochastic systems with
discontinuous characteristics].
\textit{Informatika i~ee Primeneniya}~---
\textit{Inform. Appl.}  7(1):3--11.

\end{thebibliography}

 }
 }

\end{multicols}

\vspace*{-3pt}

\hfill{\small\textit{Received January 15, 2015}}

%\vspace*{-18pt}

\Contr

\noindent
\textbf{Sinitsyn Igor N.} (b.\ 1940)~--- Doctor of Science in technology,
professor, Honored scientist of Russian Federation, Head of Department,
Institute of Informatics Problems, Russian Academy of Sciences,
44-2 Vavilov Str., Moscow 119333,
Russian Federation; sinitsin@dol.ru


\vspace*{3pt}


\noindent
\textbf{Sinitsyn Vladimir I.} (b.\ 1968)~---
Doctor of Science in physics and mathematics, associate professor,
Head of Department, Institute of Informatics Problems,
Russian Academy of Sciences, 44-2 Vavilov Str., Moscow 119333,
Russian Federation;  vsinitsin@ipiran.ru


\vspace*{3pt}

\noindent
\textbf{Korepanov Eduard R.} (b.\ 1966)~--- Candidate of Science (PhD)
in technology, Head of Laboratory, Institute of Informatics Problems,
Russian Academy of Sciences, 44-2 Vavilov Str., Moscow 119333,
Russian Federation;   ekorepanov@ipiran.ru


\label{end\stat}

\renewcommand{\bibname}{\protect\rm Литература}