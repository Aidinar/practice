\def\stat{korchag}

\def\tit{К ОЦЕНКЕ ЭФФЕКТИВНОСТИ УЧЕБНО-ПОЗНАВАТЕЛЬНОЙ ДЕЯТЕЛЬНОСТИ УЧАЩИХСЯ
С~ИСПОЛЬЗОВАНИЕМ ИНФОРМАЦИОННЫХ ТЕХНОЛОГИЙ}

\def\titkol{К оценке эффективности УПД %учебно-познавательной деятельности
учащихся
с~использованием информационных технологий}

\def\aut{О.\,М.~Корчажкина$^1$}

\def\autkol{О.\,М.~Корчажкина}


\titel{\tit}{\aut}{\autkol}{\titkol}

%{\renewcommand{\thefootnote}{\fnsymbol{footnote}} \footnotetext[1]
%{Работа выполнена при поддержке РФФИ (проект 13-01-00749).}}


\renewcommand{\thefootnote}{\arabic{footnote}}
\footnotetext[1]{Институт проблем информатики Российской академии наук, olgakomax@gmail.com}


\Abst{Рассматриваются проблемы измерения эффективности учебно-познавательной
деятель\-ности (УПД) учащихся как показателя соответствия планируемых и достигнутых
образовательных результатов. Этот показатель выражается в~терминах конкретных продуктов
УПД, получаемых в~ходе выполнения мыслительных операций.
Обсуждается вопрос совмещения стиля учения и методов обучения в~условиях интеграции
педагогических и~новых информационных технологий при выполнении заданий различных
типов. Приводится пример вербализации достигнутых результатов
УПД с~использованием мобильных устройств, основанный на таксономии
мыслительных операций Бенджамина Блума.  Установлено, что уровень эффективности
УПД с~использованием ин\-фор\-ма\-ци\-он\-но-ком\-му\-ни\-ка\-ци\-он\-ных технологий
(ИКТ) определяется способностью учителя организовать совместную
работу с~учащимися, ориентированную на развитие форм мыслительной деятельности,
приводящих к~созданию интегрированного персонального познавательного стиля каждого.}

\KW{эффективность обучения; планируемые образовательные результаты; достигнутые
образовательные результаты; мобильные устройства; мыслительные операции;
индивидуальный стиль учения; методы обучения; LOA-тех\-но\-логия}

\DOI{10.14357/19922264150110}

\vspace*{-6pt}


\vskip 14pt plus 9pt minus 6pt

\thispagestyle{headings}

\begin{multicols}{2}

\label{st\stat}



\section{Введение }

     Эффективность обучения как соотношение полезного результата и~затратных
факторов образовательного процесса может выражать различные его стороны, поэтому, когда
встает вопрос об измерении эффективности использования
ИКТ в~образовании, необходимо определиться, какую эффективность следует
рассматривать и~что конкретно понимать под словом <<эффективность>>:
     \begin{itemize}
\item эффективность использования ИКТ (или, более узко, электронных образовательных
ресурсов~--- ЭОР) в~учебном процессе;
\item эффективность учебного процесса с~использованием ИКТ;
\item эффективность УПД учащихся с~использованием ИКТ.
\end{itemize}

   Очевидно, что следует делать акцент не на экономическую эффективность внедрения ИКТ,
т.\,е.\ считать затратами при достижении определенного результата не материальные
вложения государства в~информатизацию образования или затраты конкретного учебного
заведения на приобретение компьютеров, поддержку сетей и~оплату труда педагогов или
технического персонала. За скобками следует оставить также
временн$\acute{\mbox{ы}}$е и~психологические затраты учителя и~учащихся и~рассматривать
эффективность только как \textbf{уровень достижения образовательного результата}.

   Измерение эффективности в~первой трактовке~--- как уровня достижения
   образовательного\linebreak
результа\-та при использовании ИКТ или ЭОР в~учебном процессе~--- должно осуществляться с~привлечением очень многих показателей, например, согласно~[1, с.~298], \textit{ценности
учебного материала}, \textit{мотивации} и~\textit{компетентности} учащихся.

   Среди предложенных авторами~[1] показателей эффективности особые сложности для
измерения вызывает \textit{ценность учебного материала}, которую, основываясь на
положениях теории информации Шеннона и~теории статистических решений, где есть
показатель <<ценность информации>>, можно вслед за самим Клодом Шенноном определить
как <<максимальную пользу, которую данное количество
информации способно принести в~деле уменьшения средних потерь>>.
Однако в~рас\-смат\-ри\-ва\-емом случае под потерями следует,
видимо, подразумевать неусвоенное знание, недополученные навыки и~умения, которые опять
же очень сложно измерить количественно. Кроме того, такой показатель, как <<количество
учебной информации>> в~определении Шеннона, нужно корректно оценить при помощи
соответствующих статистических методов~--- распределения Шеннона, Хартли или Больцмана.
Таким образом, первая трактовка <<эффективность использования ИКТ в~учебном процессе>>
неоднозначна, а сама эффективность в~этом понимании сложна для измерения и, следовательно,
не может быть принята как практический вариант.

   Вторая трактовка эффективности~--- эффективности учебного процесса с~использованием
ИКТ~--- слишком сложная и~многоаспектная категория, и~ее оценивание лежит в~области так
называемых нетривиальных педагогических измерений, что требует привлечения сложного
математического аппарата.

   Третий вариант трактовки эффективности, напрямую связанный с~результатами УПД
учащихся\linebreak с~использованием ИКТ, лежит в~плоскости прак\-тической
деятельности учителя и~может быть \mbox{с~успехом} им реализован, поскольку связан непосредственно с~организацией
учебного процесса, ориентированного на достижение учащимися планируемых
образовательных результатов, декларированных в~Федеральных государственных
стандартах  (далее~--- ФГОС) второго поколения~[2].

\section{Эффективность обучения как~соотношение планируемых
и~достигнутых образовательных результатов}

  В процедуре оценивания результатов УПД учащихся различают три уровня достижения этих
результатов:
  \begin{enumerate}[(1)]
\item планируемый уровень $L_P$~--- тот, что декларируется во ФГОС второго поколения и~находит воплощение в~учебниках и~учеб\-но-ме\-то\-ди\-че\-ских пособиях;
\item реализуемый уровень $L_R$, характеризующий результаты, определяемые учителем
в~зави\-си\-мости от своих профессиональных предпочтений и~условий обучения;
\item достигнутый уровень $L_A$~--- уровень объективных, реальных достижений учащихся.
\end{enumerate}

  В связи с~этим требуется определить, разницу между какими уровнями достижения
образовательных результатов: $\delta_{PR}\hm= (L_P \hm- L_R)/L_P$, $\delta_{RA}\hm = (L_R
\hm- L_A)/L_R$ или $\delta_{PA}\hm = (L_P \hm- L_A)/L_P$~--- необходимо минимизировать,
чтобы можно было говорить об объективном измерении эффективности УПД.

  Очевидно, что наиболее достоверные результаты измерения эффективности УПД дает
показатель~$\delta_{PA}$, поскольку он учитывает приближенность уровня объективных
достижений учащихся к~уровню объективного абсолюта, т.\,е.\ того предела, на который
ориентируют всех участников образовательного процесса нормативные документы
федерального уровня. Именно поэтому при измерении эффективности УПД учащихся
целесообразно говорить об \textit{оценке достижения планируемых образовательных
результатов}.

  Таким образом, система оценки достижения планируемых образовательных результатов,
с~одной стороны, направлена на реализацию требований ФГОС, а~с~другой~--- способствует
объективному измерению эффективности УПД на основе реальных достижений учащихся.

  Основной функцией системы оценки, ориенти\-рующей образовательный процесс на
\mbox{достижение} планируемых результатов, является обеспечение результативной обратной связи,
позволяющей осущест\-влять управление образовательным процессом, особенно в~части
принятия педагогических мер, которые способствовали бы повышению эффективности УПД
учащихся~[3, с.~133].

  В систему оценки достижения планируемых образовательных результатов включаются
сле\-ду\-ющие необходимые компоненты~[3, с.~133]:
  \begin{itemize}
\item формулировка основных направлений и~целей оценочной деятельности;
\item описание объектов и~содержания оценки;
\item задание критериев, описание процедур и~состава инструментария оценивания;
\item описание форм представления результатов, условий и~границ применения системы
оценивания;
\item привлечение разнообразных методов и~форм оценивания, взаимно дополняющих друг
друга.
\end{itemize}

\section{Роль информационно-коммуникационных технологий
в~повышении эффективности учебно-познавательной
деятельности учащихся}

  С расширением процесса информатизации образования и~внедрением в~учебный процесс
новых ИКТ ожидалось повышение его эффектив\-ности. Однако скоро стало очевидным, что
использование ИКТ в~русле традиционного обучения не столь существенно влияет на уровень
обучен\-ности, т.\,е.\ на эффективность обучения, как предполагалось ранее~[4, с.~22, 23], однако
значительно повышает энергоемкость труда учителя за счет необходимости освоения новых
техник и~технологий в~сжатые сроки. Более того, строгие рамки постоянного мониторинга
учебного процесса, в~которые поставлены учителя при освоении новых ИКТ, а~также
повышение интенсивности труда учителя на первых этапах процесса информатизации
образования приводили многих учителей к~резкому неприятию инновационной педагогической
деятельности в~области информатизации образования, а~часть из них~--- даже к~фрустрации
и~профессиональному выгоранию.

  По прошествии последних десяти лет информатизации российского образования, когда
большинство учи\-те\-лей-пред\-мет\-ни\-ков средней школы вольно или невольно с~разной
степенью успешности включились в~этот процесс, по-преж\-не\-му остается без ответа вопрос:
почему же ИКТ не повлияли существенным образом на эффективность обучения?

  Для того чтобы понять это, нужно проследить, к~каким изменениям в~учебном процессе
привели новые ИКТ с~точки зрения технической, со\-ци\-аль\-но-пе\-да\-го\-ги\-че\-ской
  и~пси\-хо\-ло\-го-пе\-да\-го\-ги\-ческой.

  \textit{Техническая сторона} информатизации заклю\-чается в~обновлении как аппаратного,
так и~программного обеспечения сферы образования,\linebreak причем в~последние несколько лет~---
2010--2014~го-\linebreak ды~--- идет процесс так называемой
{\bfseries\textit{электро\-ни\-зации}}~[5,
с.~167], которая характеризуется распространением мобильных электронных устройств\linebreak
различного типа и~назначения, использованием мощных персональных компьютеров,
быст\-ро\-действующих накопителей большой емкости, \mbox{облачных} серверов, новых
информационных и~телекоммуникационных технологий, муль\-ти\-ме\-диа-тех\-но\-ло\-гий
и~виртуальной реальности, появлением 3D-прин\-те\-ров, первого поколения электронных
учебников.

  С \textit{социально-педагогической} точки зрения изменения произошли как в~той роли,
которую стали играть в~учебном процессе учитель и~учащийся, так и~в изменении ожиданий
и~приоритетов учащихся по отношению к~образовательному процессу.

  Если при традиционном обучении учитель является единственным поставщиком готовых
знаний и~контролером их усвоения, то в~ходе внедрения ИКТ в~учебный процесс все
существеннее ощущается крен в~сторону автономии учащихся. При этом превалирующее
влияние приобретают следующие формы обучения:
  \begin{itemize}
\item регулируемое (направляемое) обучение (учи\-тель~--- кон\-суль\-тант, навигатор,
содействующий целенаправленной УПД учащихся по освоению планируемых
компетенций);
\item самообучение (учитель~--- тьютор, модератор, в~обязанности которого входит как
содействие раскрытию потенциальных способностей учащихся, так и~налаживание
контактов между ними для организации совместной работы);
\item саморегулируемое обучение (учитель~--- фасилитатор, работающий в~парадигме
личностно-ори\-ен\-ти\-ро\-ван\-ной педагогики и~спо\-соб\-ст\-ву\-ющий наиболее эффективному
учебному взаимодей\-ствию)~[6, с.~408--424].
\end{itemize}

  Получая б$\acute{\mbox{о}}$льшую автономию, учащиеся самостоятельно начинают
выбирать приоритеты, среди которых в~плане формы обучения основное место занимают:
  \begin{itemize}
\item мобильность~--- желание получать знания не только в~учебной аудитории в~рамках
классно-уроч\-ной системы;
\item планшетизация~--- использование планшетных и~иных мобильных устройств для
обучения;
\item коллаборация и~краудсорсинг~--- сетевое сотрудничество;
\item гейметизация~--- использование интерактивных игровых технологий в~обучении.
\end{itemize}

  Еще в~1990~г.\ в~проекте Концепции информатизации отечественного образования
очередного периода~[7, с.~4] указывалось, что с~\textit{пси\-хо\-ло\-го-пе\-да\-го\-ги\-че\-ской}
точки зрения ИКТ в~образовании способствуют:
  \begin{itemize}
\item раскрытию, сохранению и~развитию индивидуальных способностей обучаемых;
\item формированию у учащихся познавательных способностей, стремления к~самосовершенствованию;
\item обеспечению комплексности изучения явлений действительности, неразрывности
взаимосвязи между естественными и~гуманитарными нау\-ками;
\item постоянному динамическому обновлению содержания, форм и~методов процесса
обучения и~воспитания.
\end{itemize}

  Однако за прошедшую почти четверть века мы так и~не смогли ответить на главный вопрос:
что сделано для того, чтобы ИКТ способствовали раскрытию, сохранению, формированию,
обеспечению, обновлению и~т.п.? Или это должно было произойти автоматически, без
  ка\-ких-ли\-бо усилий со стороны участников образовательного процесса?

  Практика показала, что преимущества, которые предоставляют ИКТ не в~технической,
а,~главным образом, в~пси\-хо\-ло\-го-пе\-да\-го\-ги\-че\-ской сфере, могут привести
к~коренным изменениям в~учебном процессе. Остается только построить учебный процесс
таким образом, чтобы были созданы условия для реализации этих преимуществ. Именно это
обстоятельство позволит впоследствии говорить об эффективности использования ИКТ.

  Итак, вопросы, на которые должен ответить учитель, заинтересованный в~повышении
эффективности своей работы средствами новых ИКТ, можно сформулировать следующим
образом:
  \begin{itemize}
\item С~внедрением ИКТ в~учебный процесс как изменилось мышление учащихся и~в каком
направлении следует развивать их мыслительные способности?
\item С~внедрением ИКТ в~учебный процесс как изменилась УПД учащихся на уроке
и~в~ин\-фор\-ма\-ци\-он\-но-обра\-зо\-ва\-тель\-ной среде (ИОС)~--- какие новые виды
УПД возникают, какие типы заданий предлагать учащимся и~как строить урок в~классе
и~в~ИОС школы?
\end{itemize}

   Ответы на эти вопросы лежат в~плоскости когнитивной психологии, а непременным
условием повышения учебных достижений учащихся, т.\,е.\ эффективности обучения,
специалисты называют проблему совмещения стилей учения учащихся с~методами
обучения~[8, с.~263]. Эта проблема возникает в~связи с~таким, казалось бы, непреложным
фактом, что обучение происходит тем более эффективно, чем более оно соответствует
познавательным стилям учащихся, которые называют еще стилями учения\footnote{Стиль
учения~--- это индивидуальная характеристика личности, психическое образование, которое
является многомерным по своим проявлениям в~различных видах УПД, иерархическим по
устройству, включающему разные уровни стилевого поведения, интегральным по своим
механизмам, являясь продуктом интеграции разных форм индивидуального ментального опыта,
и~гибким по своим адаптационным возможностям, что способствует формированию
интегрированного персонального познавательного стиля~[8, с.~269].}.

   Тогда встает еще один вопрос: а что означает такое соответствие? Означает ли это, что
средства обучения~--- методы обучения, формы предъявления учебного материала,
используемые техники\linebreak и~технологии обучения и~пр.~--- должны подстраиваться под каждого
конкретного учащегося, тем самым формируя его образовательную траекторию? Или же,
наоборот, задача учителя~--- создать такие условия обучения, такую
образовательную  среду,\linebreak
в~которой каждый учащийся, носитель своего персонального познавательного стиля, не только
сможет выбрать свою линию обучения, но и~интеллектуально развиваться, осваивая новые для
себя\linebreak \mbox{способы} познания окружающей действитель\-ности~[8, с.~266, 267]?

   Очевидно, что первый способ не только трудно осуществим, но и~приводит к~закреплению
у~уча\-щегося определенного стиля усвоения учебной\linebreak информа\-ции, ограничивая его
интеллектуальное развитие. Второй же способ, напротив, стимулирует формирование
интегрированного персонального познавательного стиля каждого учащегося и~взаимообогащение стилей учащихся при сотрудничестве, что создает условия для их
дальнейшего интеллектуального воспитания и~развития.

   М.\,А.~Холодная, приводя эти рассуждения, ссылается на целесообразность стилевого
подхода к~обучению, ориентирующегося на внутреннюю дифференциацию как одну из двух
форм индивидуализации образовательного процесса. В~противовес внешней дифференциации,
когда производится отбор детей под определенный тип обучения с~целью создания гомогенных
классов, имеющих однонаправленную специализацию методов обуче\-ния, внутренняя
дифференциация предполагает\linebreak
 <<учет индивидуальных познавательных возможностей каждого
ребенка в~рамках общего для всех гетероген\-ного образовательного пространства~---
вариативного с~точки зрения своего содержания и~видов учебной деятельности
(в~том чис\-ле с~использованием современных педагогических и~информационных технологий)>>~[8,
   с.~265, 266].\linebreak
    И~далее: <<Правильнее говорить не об учете индивидуальных познавательных
стилей детей, а~о~фор-\linebreak мировании у~каждого ребенка персонального познавательного стиля на
основе актуализации\linebreak и~обо\-га\-ще\-ния всех механизмов стилевого поведения>>~[8, с.~271].

   Таким образом, для полноценного интеллектуального развития учащихся, т.\,е.\ для
развития их мыслительных способностей и~повышения эффективности обучения, необходимо
организовать УПД в~такой образовательной среде, которая была бы вариативной за счет
многообразных инструментов, формирующих недостающие механизмы стилевого поведения
учащихся\footnote{Психологи-когнитивисты различают следующие уровни базовых
механизмов стилевого поведения познающей личности: уровень стилей кодирования
информации, основанных на разных модальностях опыта (кинестетической, визуальной,
   сло\-вес\-но-ре\-че\-вой, сен\-сор\-но-эмо\-ци\-о\-наль\-ной); уровень стилей переработки
информации (интеллектуальные, импульсивные, рефлективные, неуспешные); уровень стилей
постановки и~решения проблем (вариации в~наборе приемов решения задачи от адаптивного
к~смыс\-ло\-обра\-зу\-юще\-му); уровень стилей познавательного отношения к~миру (чувственное,
рациональное, сверхчувственное познание и~др.)~[8, с.~270].}.

   Очевидно, что организовать и~обеспечить развитие многообразных познавательных стилей
и~механизмов стилевого поведения учащихся при традиционном обучении весьма трудно,
поскольку\linebreak автономия учащихся ограничена, существует недостаток в~новейших интерактивных
учебных материалах и~средствах обучения, а~учитель и~учащиеся поставлены в~жесткие рамки
клас\-сно-уроч\-ной сис\-те\-мы. Тогда как современная ИОС, предоставляя учащимся разнообразные
мобильные технологические инструменты~--- от приложений для мобильных устройств (см.\
разд.~4) до электронных учебников нового поколения, создает условия как для реального, так
и~для продуктивного виртуального учебного взаимодействия. Эти функции ИОС позволяют
учителю направить УПД учащихся в~русло развития многообразных форм их мыслительной
дея\-тель\-ности, что необходимо приводит к~формированию интегрированного персонального
познавательного стиля каждого учащегося и~в~перспективе~--- к~повышению эффективности
обучения.
  При этом следует учитывать следующие уровни интеграции информационных
и~педагогических технологий при осуществлении учащимися УПД с~использованием ИКТ.

  \bigskip

  \textbf{Уровень 1.} Занятия смешанного типа, когда средствами ИКТ вводится новый
материал, осуществляется его отработка и~контроль усвоения. Учитель использует отдельные
элементы готовых или авторских ЭОР и~неинтерактивные ин\-тер\-нет-ре\-сур\-сы для
визуализации традиционной работы в~классе без привлечения учащихся к~непосредственной
работе с~ЭОР.

\smallskip

  \textbf{Уровень 2.} Занятия смешанного типа, когда средствами ИКТ вводится новый
материал, осуще\-ствляется его отработка и~контроль усвоения. Электронные
образовательные ресурсы и~неинтерактивные
ин\-тер\-нет-ре\-сур\-сы используются как учителем, так и~учащимися для иллюстрации
учебного
материала и~в~виде справочных источников (в~том чис\-ле он\-лайн-сло\-ва\-рей,
предметных
справочников и~энциклопедий, языковых корпусов и~онтологий данных); для работы
в~поисковых сис\-те\-мах, проведения он\-лайн-тес\-ти\-ро\-ва\-ния и~опросов; с~привлечением
личных электронных учебных блокнотов и~заметок при традиционной работе в~классе и/или
дома.

\smallskip

  \textbf{Уровень 3.} Занятия смешанного типа с~использованием электронных конструкторов,
виртуальных сред и/или ин\-тер\-нет-сер\-ви\-сов Web~2.0, позволяющих осуществлять
простейшую визуализацию и~преобразование учебного материала (определение зависимостей,
отношений, построение чертежей, диаграмм, графиков, создание образов, статистическая
обработка данных, оформление в~виде электронных таблиц, ин\-тел\-лект-карт, облака
ключевых слов, интерактивных рабочих листов, электронных каталогов понятий) с~целью его
усвоения при традиционной работе в~классе и/или дома.

\smallskip

  \textbf{Уровень 4.} Занятия смешанного типа, включающие как обязательный компонент
совместную работу учащихся в~учебных сетевых сообществах с~использованием
  ин\-тер\-нет-сер\-ви\-сов Web~2.0 и/или приложений для мобильных устройств (электронная
стена; сервис для создания рабочих групп; пространство для создания заметок и~совместной
работы с~ними в~группе), позволяющих осуществлять простейшие преобразования учебного
материала с~целью достижения коллективного учебного результата.

\smallskip

  \textbf{Уровень 5.} Занятия смешанного типа в~среде
  про\-грам\-мно-методических комплексов
в виде\linebreak  виртуальных предметных сред (лабораторий и~сред, позволяющих осуществлять
алгоритмизацию и~моделирование изучаемых явлений и~процессов по данному предмету
с~использованием встроенных функций системы) при индивидуальной или совместной работе
учащихся в~классе и/или дома, использование мобильных устройств
и~дистанционных многофункциональных приложений 
для совместной работы над проектами, а~также дистанционных технологий, 
в~том числе для
проведения видеоконференций.

\smallskip

  \textbf{Уровень 6.} Комплексные задания 3-, 4- и~5-го уровней в~ИОС учебного заведения,
универсальных рабочих пространствах (в~том числе ИОС электронных учебников)
и~интегральных образовательных платформах Web 2.0.

  \medskip

   Приведенная классификация показывает, что чем выше уровень интеграции педагогических
и~информационных технологий, тем более широкие возможности предоставляются учащимся
для формирования и~развития интегрированного персонального познавательного стиля, что, как
следствие, способствует повышению эффективности УПД с~использованием ИКТ.

\vspace*{-6pt}

\section{Вербализация целей учебно-познавательной деятельности учащихся}

  Выше  за показатель эф\-фек\-тив\-ности обучения был принят уровень соответствия
планируемых и~достигнутых образовательных результатов (минимум показателя
$\delta_{PA}$~--- см.\ разд.~2). Следовательно, эффективность УПД с~использованием ИКТ
должна выражаться, по меньшей мере, в~терминах достигнутых образовательных результатов,
т.\,е.\ ее конкретных продуктов, получаемых в~ходе выполнения мыслительных операций. Это
будет задавать вектор оценочной деятельности учителя, ра\-бо\-та\-юще\-го в~русле
компетентностного и~сис\-тем\-но-де\-я\-тель\-ност\-но\-го подхода.
%
  Поэтому учителю при определе\-нии эффективности УПД учащихся необходимо
ориентироваться на конечные результаты этой деятельности, которые будут понятны ему
и~учащимся. Кроме того, необходима формулировка целей обучения, а~они при
компетентностном и~сис\-тем\-но-де\-я\-тель\-ност\-ном подходе как раз совпадают
с~результатами этой деятельности.

  Как выразить эти цели-результаты таким образом, чтобы можно было не только
однозначно сопоставить то, что было запланировано и~что достигнуто, но и~выработать
адекватные критерии оценки этих достигнутых результатов? Для этого необходима
определенного рода вербализация этих целей, т.\,е.\ формальное представление их
с~по\-мощью языковых средств.

  В качестве удобной последовательности когнитивных педагогических целей, поддающейся
формализации на вербальном уровне, американским психологом методов обучения
Бенджамином Блумом в~1956~г.\ была предложена таксономия (иерархия) этих целей в~виде
перечня мыслительных (когнитивных) операций, или умственных действий, представленная
в~виде пирамиды~[9] (рис.~1), в~основании которой <<лежит>> \textit{знание} как базовый
уровень, а самой высшей по степени сложности и~развития мыслительной операцией является
\textit{оценка}, находящаяся на шестом уровне.


  В приведенной таблице (см.\ с.~\pageref{ttb})
  для каждой мыслительной операции дается набор смысловых
глаголов, соответ\-ст\-вующих различным учебным задачам. Пользуясь этой таблицей, учитель
может соотнести название мыслительной операции с~их содержанием, обеспечить
концентрацию усилий на главных аспектах УПД, наметить
первоочередные  задачи и~перспективы дальнейшей работы, создать\linebreak\vspace*{-12pt}

\begin{center}  %fig1
\vspace*{8pt}
\mbox{%
 \epsfxsize=72.257mm
 \epsfbox{kor-1.eps}
 }



\noindent
{{\figurename~1}\ \ \small{Пирамида Блума}}

 \end{center}

%\vspace*{18pt}
\addtocounter{figure}{1}

\noindent
 возможности для
 разъяснения учащимся ориентиров УПД, сформировать эталоны оценки результатов обучения, обеспечивающие надежность
и~объективность~[11].



  Таким образом, глаголы мыслительных операций, с~одной стороны, помогают учителю в~постановке целей и~задач УПД, а~с~другой~--- позволяют производить описание и~оценку
результатов этой деятельности в~сравнении с~поставленными целями. С~их помощью учитель
выявляет наличие и~характер отклонений от запланированных целей УПД на основе
образовательного мониторинга, определяет их причины~[3, с.~134] и~вносит соответствующие
коррективы.

  В русле развития новых ИКТ в~2000-х~гг.\ появилось так называемое <<Педагогическое
колесо>>, которое позволяет установить соответствие между глаголами мыслительных
операций Блума, видами УПД учащихся и~инструментами мобильных ИКТ~[12] (рис.~2), т.\,е.\
показывает воплощение процесса УПД в~ее продуктивный конкретный результат, достигаемый
с~помощью современных мобильных инструментов. На рис.~2 во внешней части
<<Педагогического колеса>> приведены приложения для iPad Apple, которые являются
инструментами реализации мыслительных операций, собранных в~его центральной части.



  Например, для раздела <<Синтез (Create)>> могут быть использованы следующие
приложения:
  \begin{itemize}
\item \textbf{Aurasma}~--- приложение, позволяющее создавать дополненную реальность;
\item \textbf{Creative Book Builder}~--- приложение, позволяющее создавать, редактировать и~публиковать книги;
\item \textbf{Easy Release}~--- приложение для создания и~редактирования информационных
сообщений;
\item \textbf{Fotobabble}~--- приложение для озвучивания изображений
(создание <<говорящих>> фотографий);
\item \textbf{Garageband}~--- виртуальный самоучитель игры на музыкальных инструментах;
\item \textbf{iMovie}~--- приложение для создания видеороликов;
\item \textbf{Interview Assistant}~--- виртуальный микрофон и~запись звука;
\item \textbf{iTimeLapse Pro}~--- приложение для съемки серии изображений и~их компиляции
в видео;
\item \textbf{Nearpod}~--- интегральная платформа, поз\-во\-ля\-ющая учителю осуществлять
совместную работу с~учащимися и~ее оценку в~реальном времени;
\end{itemize}

\end{multicols}

\begin{table}\small
\label{ttb}
\begin{center}

\tabcolsep=5.8pt
\begin{tabular}{|l|p{60mm}|p{66mm}|}
\multicolumn{3}{p{160mm}}{Таксономия мыслительных операций по Б.~Блуму (1956~г.)$^*$. Строки таблицы
соответствуют уровням <<Пирамиды Блума>>~[10]}\\
\multicolumn{3}{c}{\ }\\[-5pt]
\hline
\multicolumn{1}{|c|}{\tabcolsep=0pt\begin{tabular}{c}Название\\ мыслительной\\ операции\end{tabular}}&
\multicolumn{1}{c|}{\tabcolsep=0pt\begin{tabular}{c}Содержание\\ мыслительной\\ операции\end{tabular}}&
\multicolumn{1}{c|}{\tabcolsep=0pt\begin{tabular}{c}Глагольное выражение\\
 мыслительной\\ операции\end{tabular}}\\
\hline
1. ЗНАНИЕ&Учащийся знает употребляемые термины, конкретные факты, методы и~процедуры,
основные понятия, правила и~принципы
&Упорядочи, определи, продублируй, составь список, соотнеси, запомни, назови, проранжируй,
опознай, отнеси, вспомни, повтори, воспроизведи
\\
\hline
2. ПОНИМАНИЕ&Учащийся понимает правила, факты и~принципы, интерпретирует сло\-ве\-сный
материал, схемы, графики, диаграммы, преобразует словесный материал
в~математические выражения и~наоборот, предположительно оценивает будущие\linebreak
события, последствия, вытекающие из имеющихся
данных&Классифицируй, опиши, обсуди, объясни, вырази, осознай, укажи, расположи, распознай,
сообщи, подтверди, сделай обзор, отбери, отсортируй, расскажи, переведи, проэкстраполируй\\
\hline
3. ПРИМЕНЕНИЕ&Учащийся использует понятия и~принципы в~новых ситуациях, применяет законы и~теории в~конкретных практических ситуациях, демонстрирует правильное применение метода или
процедуры
&Примени, выбери, продемонстрируй, инсценируй, привлеки, проиллюстрируй, проинтерпретируй,
произведи операции, приготовь, выполни, осуществи, отработай, составь план/программу, набросай,
реши, используй\\
\hline
4. АНАЛИЗ&Учащийся выделяет скрытые (неявные) предположения, видит ошибки и~упущения в~логике рассуждений, проводит различия между фактами и~следствиями, оценивает значимость
данных&Проанализируй, оцени, рассчитай, категоризируй, сравни, сопоставь, выскажи критику, составь
диаграмму, различи, распознай, \mbox{найди} отличия, исследуй, проэкспериментируй, подведи итог, проясни,
опробуй\\
\hline
5. СИНТЕЗ&Учащийся пишет небольшое творческое сочинение, предлагает план проведения
эксперимента, использует знания из разных областей, чтобы составить план решения той или иной
проблемы
&Организуй, собери, скомпонуй, сочини, построй, создай, спроектируй, разработай, овладей, организуй,
спланируй, подготовь, предложи, установи, синтезируй, напиши
\\
\hline
6. ОЦЕНКА&Учащийся оценивает логику представления материала в~виде письменного текста,
оценивает соответствие вывода имеющимся данным, оценивает значимость того или иного продукта
деятельности исходя из внутренних или внешних критериев &Оцени, поспорь, осуществи экспертизу,
выбери, сравни, защити, выскажи суждение, взвесь <<за>> и~<<против>>, сделай вывод, спрогнозируй,
проранжируй, выставь оценку, выбери, поддержи, оцени значимость/значение\\
\hline
\multicolumn{3}{p{160mm}}{\footnotesize $^*$ В 1990-х гг.\ группа американских
пси\-хо\-ло\-гов-когни\-ти\-ви\-стов, возглавляемая бывшим учеником Б.~Блума Лорином Андерсоном,
предложила обновленную версию таксономии Блума применительно к~реалиям XXI~в.
В~пирамиде Блума существительные, называющие мыслительные операции, были заменены на
герундий~--- часть речи, описывающую процесс выполнения мыслительных операций
(\textit{knowledge}~--- \textit{remembering};
\textit{comprehension}~--- \textit{understanding};
\textit{application}~---  \textit{applying};
\textit{analysis}~--- \textit{analyzing}), а
\textit{synthesis}~--- \textit{creating} и~\textit{evaluation}~--- \textit{evaluating} поменялись
местами~[10].}
\end{tabular}
\end{center}
\end{table}


\begin{multicols}{2}

\begin{itemize}
\item \textbf{Prezi}~--- приложение для создания презентаций;\\[-10pt]
\item \textbf{ScreenChomp}~--- интерактивная цифровая доска для создания набросков и~заметок;\\[-10pt]
\item \textbf{Toontastic}~--- приложение для создания мультфильмов и~анимированных
изображений;\\[-10pt]
\item \textbf{Voicethread}~--- интерактивное приложение для одновременного выполнения
нескольких операций с~документами: правка, обсуждение, создание схем и~заметок;\\[-10pt]
\item \textbf{Wordpress}~--- система управления содержимым сайта, блога.
\vspace*{-9pt}
\end{itemize}

\columnbreak

\section{Системы оценки эффективности учебно-познавательной деятельности
учащихся как~уровня достижения планируемых образовательных результатов}



  В контексте требований новых ФГОС следует уделить внимание двум основным видам
оценки\linebreak\vspace*{-12pt}


\end{multicols}

  \begin{figure} %fig2
    \vspace*{1pt}
 \begin{center}
 \mbox{%
 \epsfxsize=163mm
 \epsfbox{kor-2.eps}
 }
 \end{center}
 \vspace*{-6pt}
  \Caption{Оригинальная версия <<Педагогического колеса>>~--- современной интерпретации
<<Пирамиды Блума>> с~точки зрения интеграции педагогических и~мобильных ИКТ}
\vspace*{18pt}
   \end{figure}

\begin{multicols}{2}

\noindent
 достижения планируемых образовательных результатов: \textbf{суммирующему
(итоговому) оцениванию и~формирующему (процессуальному) оцениванию}.

  При \textit{суммирующем оценивании} определяется качество усвоения некоторого объема
учебного материа\-ла в~течение определенного временн$\acute{\mbox{о}}$го (четверти,
полугодия, учебного года) или информационного (урока, раздела, модуля, группы модулей,
цик\-ла уроков, курса и~т.\,д.) этапа обуче\-ния, который рассматривается как некоторый итог.

Сум\-ми\-ру\-ющее оценивание имеет целью вы\-став\-ле\-ние частично (промежуточной) или
пол\-ностью итоговой отметки и~производится, как правило, в~числовом выражении.

Суммирующее оценивание
является своего рода объектом презентации передового опыта, отчета учителя и/или учащихся о
проделанной работе \mbox{в~течение} указанного этапа образовательной деятельности (периода
учебного времени или объема учебной информации, подлежащей усвоению). Поэтому
суммирующее оценивание можно рас\-смат\-ри\-вать лишь как простую констатацию факта того,
\mbox{были} или нет достигнуты запланированные образовательные результаты, и~если <<да>>, то
какой уровень был достигнут. Это определяет место сум\-ми\-ру\-юще\-го оценивания как бы вовне
процесса освоения знаний или приобретения компетенций~\cite{10-kor}.

  Значимость \textit{формирующего оценивания} состоит в~том, что оно, будучи комплексной,
интегральной процедурой, включенной в~сам процесс УПД, позволяет получить
<<пошаговые>> данные об уровне развития мыслительных способностей или компетенций
учащихся на сколь угодно мелких промежуточных этапах. Эта процессуальная функция
формирующего оценивания позволяет рассматривать его как необходимую составную часть
эффективного обучения, требующего, однако, дополнительных затрат учебного времени.

  Формирующее оценивание базируется на трех основополагающих принципах
педагогической технологии, получившей название LOA (learning-oriented assessment)~---
<<оценивание, направленное на обучение>>~[13, с.~57, 59--60]:
  \begin{enumerate}[(1)]
\item постановке задач оценивания как учебных задач;
\item привлечению учащихся к~оцениванию работы своих товарищей и~самооценке;
\item осуществлению обратной связи, направленной не на пройденный материал, а на материал,
который предстоит освоить.
\end{enumerate}

  В названии LOA-технологии ключевым является слово <<обучение>>, а~не
<<оценивание>>, поскольку вне процесса обучения преимущества фор\-ми\-ру\-юще\-го оценивания
теряют смысл. Более того, при сбалансированном подходе и~обучение, и~оценивание не просто
движутся в~едином русле достижения эффективных результатов обучения, но эти результаты
являются \textit{запланированными} результатами, поскольку ориентируются на конкретную
учебную цель, поставленную при формулировке учебной задачи, а~именно: подобная цель
является объектом УПД. В~этом смысле формирующее оценивание самым оптимальным
образом ориентирует процесс обучения на достижение запланированных результатов обучения
как его конечной или промежуточной цели.

  \textit{Первый принцип} LOA-тех\-но\-ло\-гии~--- постановка задач оценивания как учебных
задач~--- предполагает совмещение задач оценивания и~задач обучения. Этот принцип состоит в~пошаговом оценивании перспективы успешности/неуспешности решения задачи на
промежуточных, более мелких этапах, т.\,е.\ прогнозирование этой успешности/неуспешности
с~целью корректировки алгоритма ее решения. При этом учащимся с~целью прогнозирования
ситуации поневоле приходится осуществлять действия, связанные с~экстраполяцией учебного
материала, т.\,е.\ с~углублением не в~пройденный учебный материал, а~с~обращением
к~материалу, подлежащему дальнейшему усвоению. Эти шаги способствуют развитию
многообразных форм мыслительной деятельности учащихся.


  \textit{Второй принцип} LOA-тех\-но\-ло\-гии~--- при\-вле\-чение учащихся к~оцениванию
работы своих товарищей и~самооценке~--- способствует не только \mbox{развитию} навыков рефлексии
и самооценки, являющихся важнейшими регулятивными характеристиками как предметных,
так и~метапредметных компетенций. Он приучает их к~осуществлению экспертной оценки в~соответствии с~критериями, разработанными ими самими, к~принятию ответственных решений,
влияющих на конечный результат, к~учебному сотрудничеству. Кроме того, прозрачность
оценки, выносимой в~результате совместного обсуждения, служит залогом понимания
учащимися конечной цели своей УПД.

  \textit{Третий принцип} LOA-тех\-но\-ло\-гии~--- осуществление обратной связи,
направленной не на пройденный материал, а на материал, который предстоит освоить. Обратная
связь сама по себе не побуждает учащихся к~дальнейшему изучению предмета, однако, выступая
как основной способ анализа результатов на отдельных этапах решения учебной задачи с~целью
корректировки путей ее решения, она тем самым нацеливает учащихся на дальнейшее изучение
материала.

  Процедура интеграции оценивания и~обучения в~рамках LOA-тех\-но\-ло\-гии сопоставима
с~одним из принципов формирования операционного стиля мышления, выдвинутых
академиком А.\,П.~Ершовым еще в~1980-х~гг.: планирование структуры целенаправленных
действий в~определенных условиях с~помощью заданного набора средств. Реализация этого
принципа предполагает, что учащийся должен не только представлять себе ситуацию, в~которой
будет осуществляться решение поставленной задачи, но и~уметь анализировать ее, выявляя
имеющиеся средства, доступные резервы и~предполагаемые трудности. Анализ этой ситуации
необходим для выстраивания верной стратегии решения~--- иными словами, создания
адекватной задаче структуры целенаправленных умственных действий (алгоритма),
осуществление которых согласно принятому плану поможет привести к~успешному результату,
что само по себе и~предполагает формирующее оценивание.

  В процессе анализа исходной ситуации учащиеся подбирают ряд более простых целевых
ситуаций, выстраивают их в~определенную иерархию, 
не противоречащую исходной, хотя и~упрощающую ее на некоторых этапах, 
и~тем самым шаг за шагом движутся в~направлении
нужного решения.

  Кроме того, такие пошаговые процедуры сужают поле поиска решения и~тем самым
упрощают его, делая посильным. Таким образом, деление сложных задач на более простые,
элементарные, <<пооперациональные>> задачи, во-пер\-вых, сужает поле поиска вероятного
решения, а~во-вто\-рых, структурирует траекторию поиска, разделяя ее на шаги или этапы, так
что движение осуществляется дозированно, а~на каждом этапе решается некоторая
элементарная задача, приближающая учащегося к~решению исходной более сложной задачи.

  Очень важно подчеркнуть, что деление первоначальной задачи на более мелкие и~простые
дозированные этапы, осуществляемое в~результате анализа исходной ситуации, происходит
именно \mbox{путем} формирующего оценивания, которое на\-прав\-ле\-но на корректировку
стратегической траектории решения задачи (более подробно см.~[14, с.~28--30]). При этом
результаты формирующего оценивания, выраженные в~баллах, демонстрируют не только
правильность решения задачи, но и~рациональность выбранной стратегической траектории
движения к~искомому решению.

  Необходимо отметить, что формирующее оценивание требует одновременной,
<<сиюминутной>> вовлеченности в~процесс обучения и~учителя, и~учащихся, что имеет место
при выполнении заданий, направленных на максимальную кооперацию всех участников
процесса УПД. Одним из форматов урока, способствующего реализации концепции
формирующего оценивания, направленного на достижение планируемых образовательных
результатов при решении конкретных образовательных задач, служит, например, технология
<<перевернутого>> урока. Она предполагает самостоятельную работу учащихся с~электронным
контентом дома, а~основное интерактивное общение ориентирует на выполнение совместных
заданий в~ИОС учебного заведения и~решение задач повышенной труд\-ности в~классе при очном
общении учителя и~уча\-щихся.
{ %\looseness=1

}

  Оценка достижения предметных результатов обучения осуществляется, как правило,
традиционными балльными методами. Однако в~ряде случаев, особенно при текущем или
промежуточном оценивании, которое может вестись в~формате формирующего оценивания,
полученные результаты целесообразно сохранять с~помощью так называемой накопительной
системы оценивания (например, в~форме портфолио) и~затем учитывать при определении
итоговой оценки вкупе с~результатами суммирующего оценивания.

\section{Заключение}

  Уровень эффективности УПД с~использованием ИКТ
определяется способностью учителя организовать совместную работу с~учащимися,
ориентированную на развитие форм мыслительной\linebreak деятель\-ности, приводящих к~созданию
интегрированного персонального познавательного стиля каж\-до\-го. Такую возможность
предоставляют учи\-телю педагогические и~новые информационные техноло\-гии, объединенные
в~целостный дидактический процесс, реализуемый в~ИОС
учебного заведения.

{\small\frenchspacing
 {%\baselineskip=10.8pt
 \addcontentsline{toc}{section}{References}
 \begin{thebibliography}{99}
\bibitem{1-kor}
\Au{Капранов В.\,К., Капранова М.\,Н.} ЭОР от Интернета до учителя~// Информационные
технологии в~образовании XXI~века: Сб. науч. тр. II Всеросс. науч.-практич. конф.~--- М.:
НИЯУ МИФИ, 2012. Т.~2. С.~297--300.
\bibitem{2-kor}
Федеральный государственный образовательный стандарт основного общего образования~/
Минобрнауки РФ.~--- М.: Просвещение, 2011. 48~с. %(Стандарты второго поколения).
\bibitem{3-kor}
\Au{Ривкин Е.\,Ю.} Профессиональная деятельность учителя в~период перехода на ФГОС
основного общего образования: Теория и~технологии.~--- Волгоград: Учитель, 2014. 183~с.
\bibitem{4-kor}
Информационные и~коммуникационные технологии в~образовании~/ Под. ред.
Б.~Дендева.~--- М.: ИИТО ЮНЕСКО, 2013. 320~с.
\bibitem{5-kor}
\Au{Вихрев В.\,В., Христочевская А.\,С., Христочевский~С.\,А.} О~новой концепции
информатизации образования~// Системы и~средства информатики, 2014. Т.~24. №\,4.
С.~162--172.
\bibitem{6-kor}
\Au{Петти Д.} Современное обучение: Практическое руководство~/ Пер. с~англ.
П.~Кириллова.~--- М.: Ломоносовъ, 2010. 624~с. (Прикладная психология).
(\Au{Petti~D.}
Teaching today: A~practical guide.~--- 4th ed.~--- Cheltenhem: Nelson Thornes, 2009. 624~p.)
\bibitem{7-kor}
Концепция информатизации образования~// Информатика и~образование, 1990. №\,1. С.~3--9.
\bibitem{8-kor}
\Au{Холодная М.\,А.} Когнитивные стили. О~природе индивидуального ума.~--- М.: ПЕР СЭ, 2002. 304~с.
\bibitem{9-kor}
\Au{Bloom B.} Developing talent in young people.~--- New York: Ballantine Books, 1985. 558~p.

\bibitem{11-kor}
Bloom's Taxonomy and the Pedagogy Wheel. {\sf
https:// www.gadsdenstate.edu/academics/elearning/pdf/First\linebreak \%20Friday\%20Tech\%20Tip\%20Aug\%202013.pdf}.
\bibitem{10-kor}
\Au{Чайка В.\,М.} Таксономия целей обучения. {\sf
http:// uchebnikionline.com/pedagogika/osnovi\_didaktiki\_-\_chayka\_vm/taksonomiya\_tsiley\_navchannya.htm}.

\bibitem{12-kor}
The Pedagogy Wheel. {\sf http://www.unity.net.au/\linebreak padwheel/padwheelposter.pdf}, {\sf
http://elearningstuff.\linebreak net/wp-content/uploads/2013/06/padagogy-wheel.jpg}.

\pagebreak


\bibitem{13-kor}
\Au{Carless D.}
Learning-oriented assessment: conceptual bases and practical implications~//
Innov. Educ. Teach. Int., 2007. Vol.~44. No.\,1. P.~57--66.
\bibitem{14-kor}
\Au{Корчажкина О.\,М.} Операционный стиль мышления: взгляд четверть века спустя~//
Информатика и~образование, 2010. №\,5. С.~28--36.
 \end{thebibliography}

 }
 }

\end{multicols}

\vspace*{-3pt}

\hfill{\small\textit{Поступила в~редакцию 04.01.15}}

%\newpage

\vspace*{12pt}

\hrule

\vspace*{2pt}

\hrule

%\vspace*{12pt}

\def\tit{ON ACCESS TO THE~EFFICIENCY OF~STUDENTS' COGNITIVE ACTIVITIES
WHILE USING THE~NEW~INFORMATION TECHNOLOGIES}

\def\titkol{On access to the efficiency of students' cognitive activities while using the new information
technologies}

\def\aut{O.\,M.~Korchazhkina}

\def\autkol{O.\,M.~Korchazhkina}

\titel{\tit}{\aut}{\autkol}{\titkol}

\vspace*{-9pt}

\noindent
Institute of Informatics Problems,
Russian Academy of Sciences, 44-2 Vavilov Str., Moscow 119333, Russian Federation


\def\leftfootline{\small{\textbf{\thepage}
\hfill INFORMATIKA I EE PRIMENENIYA~--- INFORMATICS AND
APPLICATIONS\ \ \ 2015\ \ \ volume~9\ \ \ issue\ 1}
}%
 \def\rightfootline{\small{INFORMATIKA I EE PRIMENENIYA~---
INFORMATICS AND APPLICATIONS\ \ \ 2015\ \ \ volume~9\ \ \ issue\ 1
\hfill \textbf{\thepage}}}

\vspace*{3pt}


\Abste{The paper considers a problem of how to measure the efficiency of students' cognitive
activities as the planned outcomes in compliance with the achieved ones, both expressed in terms of
specific products of learning and cognitive activities that are obtained while performing mental tasks.
Combining the style of teaching and learning methods with the use of pedagogical and new
information technologies integrated while performing various types of tasks is discussed. An example
of how to verbalize the results achieved during the learning activities with the use of mobile devices is
given. The way of verbalizing is based on Bloom's taxonomy action verbs. It is found out that the level
of how well students perform cognitive tasks with the use of information and communication
technologies depends on their teacher's ability to collaborate with them while developing all forms of
their mental activity, which leads to building an integrated personal cognitive style for each student.}

\KWE{efficiency of training; planned educational results; achieved educational outcomes; mobile
devices; cognitive/mental tasks; individual style of learning; teaching methods; LOA-technology}

\DOI{10.14357/19922264150110}

%\Ack
%\noindent


%\vspace*{3pt}

  \begin{multicols}{2}

\renewcommand{\bibname}{\protect\rmfamily References}
%\renewcommand{\bibname}{\large\protect\rm References}



{\small\frenchspacing
 {%\baselineskip=10.8pt
 \addcontentsline{toc}{section}{References}
 \begin{thebibliography}{99}

\bibitem{1-kor-1}
\Aue{Kapranov, V.\,K., and M.\,N. Kapranova}. 2012. EOR ot Interneta do uchitelya [Digital
educational resources: From the Internet to the teacher]. \textit{Informatsionnye Tekhnologii
v~Obrazovanii XXI~veka: Sb. nauch. tr. II~Vseross. nauch.-praktich. konf.} [2nd All-Russian
Scientific-Practical Conference ``Information Technologies in Education of the XXI  Century''].
Moscow. 2:297--300.
\bibitem{2-kor-1}
Minobrnauki RF
 [Department of
Education and Science of the Russian Federation]. 2011.
Federal'nyy gosudarstvenny obrazovatel'nyy standart osnovnogo obschego
obrazovaniya [Federal State
Educational Standard of basic general education].   Moscow: Prosveshchenie Publ. House. 48~p.
%(Standarty vtorogo pokoleniya [Standards of the new generation]).
\bibitem{3-kor-1}
\Aue{Rivkin, E.\,Yu.} 2014. \textit{Professional'naya deyatel'nost' uchitelya v period perekhoda na
FGOS osnovnogo obshchego obrazovanyya: Teoriya i~tekhnologii} [The teacher's
professional activity
in transition to the Federal State Educational Standard of basic
general education]. Volgograd: Uchitel'
Publ. House. 183~p.
\bibitem{4-kor-1}
Dendev, B., ed.
2013. \textit{Informatsionnye i kommunikatsionnye tekhnologii v~obrazovanii}
[Information and communication technologies in education].  Moscow:
IITE \mbox{UNESCO}. 320~p.
\bibitem{5-kor-1}
\Aue{Vikhrev, V.\,V., A.\,S.~Christochevskaya, and S.\,A.~Christochevsky}. 2014. O~novoy
kontseptsii informatizatsii obrazovaniya [On a~new conception of informatization of education].
\textit{Sistemy i~Sredstva Informatiki}~--- \textit{Systems and Means of Informatics}
24(4):162--172.
\bibitem{6-kor-1}
\Aue{Petti, D.} 2009. \textit{Teaching today: A~practical guide}. 4th ed. Cheltenhem: Nelson Thornes.
624~p.
\bibitem{7-kor-1}
Kontseptsiya informatizatsii obrazovaniya [The conception of informatization of education].
1990. \textit{Informatika i~Obrazovanie} [Informatics and Education] 1:3--9.
\bibitem{8-kor-1}
\Aue{Kholodnaya, M.\,A.} 2002. \textit{Kognitivnye stili. O~prirode
individual'nogo uma} [Cognitive styles. On the nature of the individual mind]. Moscow:
PERSY Publ. House.\linebreak 304~p.
\bibitem{9-kor-1}
\Aue{Bloom, B.} 1985. \textit{Developing talent in young people.} New York: Ballantine Books. 558~p.

\bibitem{11-kor-1}
Bloom's taxonomy and the Pedagogy Wheel. Available at: {\sf
https://www.gadsdenstate.edu/academics/\linebreak elearning/pdf/First\%20Friday\%20Tech\%20Tip\%20Aug\linebreak \%202013.pdf} (accessed December~1, 2014).

\bibitem{10-kor-1}
\Aue{Chayka, V.\,M.} Taksonomiya tseley obucheniya [Taxonomy of educational objectives].
Available at: {\sf http:// uchebnikionline.com/pedagogika/osnovi\_didaktiki\_-\_chayka\_vm/taksonomiya\_tsiley\_navchannya.htm} (accessed November~3, 2014).

\bibitem{12-kor-1}
Pedagogy Wheel, The. Available at:
{\sf http://www.unity.\linebreak net.au/padwheel/padwheelposter.pdf}; {\sf
http://elearnings tuff.net/wp-content/uploads/2013/06/padagogy-wheel. jpg} (accessed December~11,
2014).
\bibitem{13-kor-1}
\Aue{Carless, D.} 2007. Learning-oriented assessment: conceptual bases and practical implications.
\textit{Innov. Educ.  Teach. Int.} 44(1):57--66.
\bibitem{14-kor-1}
\Aue{Korchazhkina, O.\,M.} 2010. Operatsionnyy stil' myshleniya: Vzglyad chetvert' veka spustya [The
operational style of thinking: A~sight in a quarter of the century]. \textit{Informatika i~Obrazovanie}
[Informatics and Education] 5:28--36.
\end{thebibliography}

 }
 }

\end{multicols}

\vspace*{-3pt}

\hfill{\small\textit{Received January 4, 2015}}

%\vspace*{-18pt}

   \Contrl

   \noindent
   \textbf{Korchazhkina Olga M.} (b.\ 1953)~---
   Candidate of Science (PhD) in technology, senior scientist, Institute of Informatics Problems,
Russian Academy of Sciences, 44-2 Vavilov Str., Moscow 119333, Russian Federation;
olgakomax@gmail.com

\label{end\stat}

\renewcommand{\bibname}{\protect\rm Литература}