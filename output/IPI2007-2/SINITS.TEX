
%------------------ define ---------------------------
%\def\ss{\textstyle}
%\def\kk{\kappa}
 \def\tr{\,,\,\ldots\,,\,}
%\def\prl{\,\parallel}
%\def\prr{\parallel\,}
%\def\paar{\parallel}
%\def\sbs{\subset}
%\def\sps{\supset}
%\def\eps{\varepsilon}
 \def\si{\sigma}
%\def\la{\lambda}
 \def\alp{\alpha}
 \def\w{\omega}
%\def\W{\Omega}
 \def\sssd{\mathop{\sum\limits^2\sum\limits^2}}
 \def\sssn{\mathop{\sum\limits^n\sum\limits^n}}
%\def\liminf{\mathop{\cup\,inf}} \def\limsup{\mathop{\cup\,sup}}
%\def\iint{\int\limits_{-\infty}^{\infty}}
%\def\iii{\int\limits}
%\def\sss{\sum\limits}
 \def\prt{\partial}
 \def\mm{{\rm M}}

%----------------------------------------------------------


\def\stat{sinits}

\def\tit{КОРРЕЛЯЦИОННЫЕ МЕТОДЫ ПОСТРОЕНИЯ АНАЛИТИЧЕСКИХ ИНФОРМАЦИОННЫХ
МОДЕЛЕЙ ФЛУКТУАЦИЙ ПОЛЮСА ЗЕМЛИ ПО АПРИОРНЫМ ДАННЫМ$^*$}
\def\titkol{Корреляционные методы построения аналитических информационных
моделей флуктуаций полюса Земли} % по априорным данным}

\def\autkol{И.\,Н.~Синицын}
\def\aut{И.\,Н.~Синицын$^1$}

\titel{\tit}{\aut}{\autkol}{\titkol}

{\renewcommand{\thefootnote}{\fnsymbol{footnote}}\footnotetext[1]{Работа выполнена при финансовой поддержке РФФИ
(проект №\,07-07-00031) и программы ОИТВС РАН <<Фундаментальные
основы информационных технологий и систем>> (проект 1.5).}
\renewcommand{\thefootnote}{\arabic{footnote}}}

\footnotetext[1]{Институт проблем информатики Российской академии наук, sinitsin@dol.ru}

\index{Синицын И.\,Н.}

\label{st\stat}


\Abst{Методы современной
статистической информатики лежат в основе исследований по
фундаментальной проблеме <<Статистическая динамика вращения Земли>>.
Рассматриваются приближенные и точные корреляционные методы
построения аналитических стохастических моделей с аддитивными и
параметрическими гармоническими и случайными возмущениями для
флуктуаций полюса Земли на интервалах времени 3--5~лет. В основу
построения положены априорные данные по динамической структуре и
стохастическим возмущениям деформируемой Земли. Приводятся примеры
применения методов из состава информационных ресурсов
<<Статистическая динамика вращения Земли>>.}


\KW{флуктуации полюса Земли; информационная аналитическая модель;
априорные данные; спектрально-корреляционные характеристики;
корреляционные методы; стохастические дифференциальные уравнения}

\vskip 24pt plus 9pt minus 6pt

\thispagestyle{headings}

\begin{multicols}{2}

\section{Введение}

Движение Земли относительно центра масс есть сложный многочастотный
нестационарный процесс, который требует всестороннего изучения. Его
исследования представляют интерес в естественно-научном
фундаментальном (астро\-мет\-ри\-ческом), прикладном (геофизическом) и
практическом (навигационном) аспектах. Астрометрические данные
высокоточных измерений (служб\linebreak широт и точного времени)
свидетельствуют о не\-бес\-но-механической природе механизма воз\-буж\-де\-ния
колебаний полюса Земли. По данным измерений международной службы
вращения Земли (МСВЗ), за последние 15--20~лет выделяются
следующие три основные составляющие: 
\begin{enumerate}[(1)]
\item %(1)~
чандлеровское колебание
(свободная нутация), амплитуда которого достигает величин
0,20$^{\prime\prime}$--0,25$^{\prime\prime}$ в год, а период~---
$433 \pm 2$~звездных суток;
{\looseness=1

} 
\item %(2)~
годичное колебание с амплитудой 0,07$^{\prime\prime}$--0,08$^{\prime\prime}$ 
в год и периодом 365,25 звездных суток; 
\item %(3)~
тренд <<среднего>> полюса Земли со средней скоростью 0,005$^{\prime\prime}$ в год. 
\end{enumerate}
Чандлеровская частота колебаний обуслов\-лена\linebreak аддитивным 
лунно-солнечным гравитационным возмущением с комбинационной (годичной и 
чандлеровской) частотой $N_0 \approx 5/6$ цикла в год.\linebreak Собственная 
частота $N$ нутационных колебаний вязко-упругой деформируемой Земли оценивается 
величиной $N=0{,}83 \div 0{,}89$. При этом разность час\-тот $\delta_N =\vert 
N-N_0\vert$ сравнима с коэффициентом диссипации $D$, $\delta_N \sim D$. 
Наиболее чувствительными оказываются возмущения с частотами вблизи 
параметрического резонанса, т.\,е.\ в окрестности значения $2N_0$. Анализ 
частотных характеристик данных измерений \mbox{МСВЗ}, в том числе спектральной 
плотности мощности и потенциала момента гравитационно-приливных сил, 
свидетельствует о наличии возмущений с удвоенной частотой $2N_0$~\mbox{[1--4]}.
{\looseness=-1

}


Различным подходам по построению стохастических моделей движения
полюса Земли посвящены работы~[5--21]. В~\cite{9s, 10s, 13s, 15s} развита
общая тео\-рия распределений флуктуаций движения Земли. Влияние
стохастических параметрических шумов, происходящих от
флуктуационно-дис\-си\-па\-тив\-ных сил в рамках корреляционной %\linebreak 
теории для
построения их динамических структур деформируемой Земли изучено
в~[10--13]. %\cite{8s, 12s, 13s, 15s}. 
В~[16--20] исследован механизм стабилизации %\linebreak
чандлеровских колебаний полюса в условиях параметрических
гармонических диссипативных и гиро\-скопических возмущений на
удвоенной чандлеровской частоте при аддитивных
 и параметрических стохастических возмущениях.

Рассмотрим два важных для практики класса приближенных нелинейных
корреляционных\linebreak
\vspace*{-12pt}
\pagebreak

\noindent
 методов построения аналитических информационных
стохастических моделей:
\begin{itemize}
\item квазилинейные методы, основанные на эквивалентной
статистической линеаризации стохастических нелинейных и
параметрических возмущений;
\item методы параметрического корреляционного построения, основанные на
нелинейных корреляционных уравнениях.
\end{itemize}


При этом, следуя~\cite{22s}, в силу высокой <<добротности>> динамической
структуры деформируемой Земли будем пользоваться <<укороченными>>
уравнениями.


\section{Стохастические дифференциальные уравнения флуктуаций полюса Земли}

Для случая параметрических возмущений, происходящих
 как от флуктуационно-диссипативных сил, так и от гироскопических
 сил, рассмотрим следующие стохастические дифференциальные
 уравнения движения полюса Земли:
\begin{align}
\dot{p} + N_1 q + D_1 p &= M_1\,, \label{e1si} \\
\dot{q} - N_2 p + D_2 q &= M_2\,. \label{e2si}
\end{align}
Здесь $p,q$~--- информационные переменные, доступные измерениям МСВЗ,
представляют собой проекции мгновенной угловой скорости вращения
Земли на связанные оси ($p,q \ll r_*$, $r_*$~--- осевая угловая
скорость вращения Земли, принимаемая постоянной). Если ограничиться
возмущениями на час\-то\-тах $N_0$ и $2 N_0$, отвечающих главным
резонансам, %~\cite{11s}, 
то выражения для коэффициентов гироскопических
($N_1$, $N_2$) и диссипативных ($D_1$, $D_2$) моментов
гравитационно-приливных сил будут иметь вид
{\looseness=1

}
\vspace*{-12pt}
\begin{multline}
N_i = N \left[ 1 + \pi_1^{N_i} \cos \left(N_0 t -\chi_1^{N_i}\right)\right. +{} \\
\left.{} + \pi_2^{N_i} \cos\left( 2 N_0 t -\chi_2^{N_i}\right) + X^{N_i}(t)\right]\,,\label{e3s}
\end{multline}
\begin{multline}
D_i = D \left[ 1 + \pi_1^{D_i} \cos \left(N_0 t -\chi_1^{D_i}\right)\right.  + {}\\
\left. {} + \pi_2^{D_i} \cos\left( 2 N_0 t -\chi_2^{D_i}\right) + X^{D_i}(t)\right]\label{e4s}
\end{multline}
при $i=1$, 2. Моменты $M_1$ и $M_2$ составим из компонент
$M_{10}$ и $M_{20}$, определяющих годичное колебание, переменных
гармонических и случайных составляющих $\tilde M_1$ и $\tilde M_2$,
описываемых выра\-же\-ниями:
{\looseness=1

}
\noindent
\begin{multline}
\tilde M_i = M_1^i \cos \left( N_0 t -\chi_1^{M_i}\right)+{}\\
{} + M_2^i \cos \left( 2 N_0 t - \chi_2^{M_i}\right) + X^{M_i} (t)\,,\label{e5s}
\end{multline}
а также малых возмущающих нелинейных моментов $\Delta M_i =\Delta M_i (p,q,t)$\ $(i=1,2)$.

В (\ref{e3s})--(\ref{e5s}) приняты следующие обозначения:
 $N$~--- частота чандлеровских (собственных) колебаний;
 $N_0$~--- близкая к $N$ комбинационная час\-то\-та, происходящая
вследствие годичной и\linebreak шестилетней составляющих;
 $D$~--- коэффициент регулярных моментов диссипативных сил;
 $\pi_1^{N_i,D_i}$, $\pi_2^{N_i,D_i}$ и $\chi_1^{N_i,D_i,M_i}$,
 $\chi_2^{N_i,D_i,M_i}$ $(i=1,2)$~--- амплитуды и начальные фазы гармонических аддитивных и
параметрических возмущений на частотах $N_0$,\linebreak
 $2 N_0$.

Будем сначала считать, что компоненты широкополосных
возмущений
$X^{N_1} (t) = X^{N_2} (t) = X_3 (t)$, $ X^{D_1}
(t) =X^{D_2} (t) = X_4 (t)$ $(i=1,2)$ представляют собой
нормальные действительные случайные процессы, удовлетворяющие
уравнению формирующего фильтра первого порядка~\cite{23s}:
 \begin{equation}
 \dot X_j =- \alp_j X_j +\si_j \sqrt{2\alp_j}V_j \quad (j=3,4)\,,\label{e6s}
 \end{equation}
где $\alp_j$, $\si_j$~--- параметры фильтра, $V_j$~--- нормальный
белый шум единичной интенсивности. При $t\gg \alp_j^{-1}$
дисперсии и ковариационные функции $X_j$ определяются формулами~\cite{23s}:
 \begin{align*}
 D_j^X &=\si_j^2\,,& K_{ij}^X &=\si_j^2 \rho_{ij}\,,\\
 k_j^X (\tau) &=\si_j^2 e^{-\alp_j \vert \tau\vert}\,,&
 k_{ij}^X &=\si_j^2 \rho_{ij} e^{-\alp_j \vert \tau\vert}\,, %\label{e7s}
 \end{align*}
где $\rho_{ij}$~--- коэффициенты взаимной корреляции $(i,j=3,4)$. Что
касается компонент $X^{M_i}(t)$, то примем их нормальными белыми
шумами $\left[ V_1\, V_2\right]^T$, $ V_1 = X^{M_1} (t)$, $V_2 =
X^{M_2} (t)$ с интенсивностями $\nu_{ij}$ $(i,j=\overline{1,2})$.

В результате уравнения~(\ref{e1si}) и~(\ref{e2si}) примут следу\-ющий вид
\begin{multline*}
\dot p + N \left[ 1 +\pi_1^{N_1} \cos \left(N_0 t -\chi_1^{N_1} \right) +{}\right.\\
\left.{}+\pi_2^{N_1} \cos \left(2N_0 t -\chi_2^{N_1} \right)+ X_3 (t)\right]q +{}\\
{}+ D \left[ 1+\pi_1^{D_1} \cos \left(N_0 t -\chi_1^{D_1} \right)+{}\right.\\
\left.{}+ \pi_2^{D_1} \cos \left(2N_0 t -\chi_2^{D_1} \right)+ X_4(t)\right] p={}\\
 {}= M_{10} +M_1^1 \cos \left(N_0 t -\chi_1^{M_1}\right) +{}\\
 {}+ M_2^1 \cos \left(2 N_0t -\chi_2^{M_1}\right) + V_1(t)+ \Delta M_1 (p,q,t)\,,
\end{multline*}
\begin{multline}
\dot q - N \left[ 1 +\pi_1^{N_2} \cos \left(N_0 t -\chi_1^{N_2} \right) +{}\right.\\
\left.{}+\pi_2^{N_2} \cos \left(2N_0 t -\chi_2^{N_2} \right)+ X_3 (t)\right] p + {}\\
{}+D \left[ 1+\pi_1^{D_2} \cos \left(N_0 t -\chi_1^{D_2} \right)+{}\right.\\
\left.{}+ \pi_2^{D_2} \cos \left(2N_0 t -\chi_2^{D_2} \right)+ X_4(t)\right] q={}\\
 {}= M_{20} + M_1^2 \cos \left(N_0 t -\chi_1^{M_2} \right)+ {}\\
\!\! +M_2^2 \cos \left(2 N_0t -\chi_2^{M_2}\right) + V_2(t) + \Delta M_2 (p,q,t).
 \label{e8s}
\end{multline}

Уравнения~(\ref{e8s}) будем понимать в смысле Стратоновича~\cite{23s}.

Случай, когда все компоненты $X_j = X_j(t)$ $(j=1$,~4) являются
нормальными (гауссовскими) белыми шумами, происходящими от одного
гравитационно-приливного источника, т.\,е.\
\begin{align}
V_1&=\gamma_1 V\,,& V_2&=\gamma_2 V\,,\notag\\
X_3&=\gamma_3 V\,,& X_4&=\gamma_4 V\,,\label{e9s}
\end{align}
(где $\gamma_j$~--- известные коэффициенты; $V$~--- скалярный
нормальный белый шум интенсивности $\nu$), описывается уравнениями~(\ref{e8s}).

Уравнения~(\ref{e8s}) при условии~(\ref{e9s}) будем понимать в смысле Ито~\cite{23s}.
Правила перехода от уравнений в смысле Стратоновича к уравнениям в
смысле Ито даны в~[8--11]. %\cite{8s, 10s, 11s, 23s}.


\section{Квазилинейные корреляционные методы построения аналитических
информационных моделей} %3

Вследствие высокой добротности системы~(\ref{e8s}) и наличия быстрых по
сравнению с периодом чандлеровских колебаний реализаций случайных
широкополосных возмущений $V=\left[ V_1 V_2 V_3 V_4\right]^T$ прямая
компьютерная реализация модели~(\ref{e6s}), (\ref{e8s}) по методу статистического
моделирования~\cite{23s} встречает серьезные вычислительные трудности. 

%z
Для %\linebreak
получения приближенных дифференциальных уравнений коррелиционной
модели флуктуаций колебаний полюса Земли применим метод
корреляционного аналитического моделирования, основанный на
статистической линеаризации~\cite{23s}. С~этой целью перейдем к новым
действительным переменным, положив 
%$Y_1=p$, $Y_2 =q$, $Y_3 = X_3$, $Y_4 = X_4$, 
$$Y_1=p\,,\quad Y_2 =q\,,\quad Y_3 = X_3\,,\quad Y_4 = X_4\,,$$ 
и проведем статистическую линеаризацию параметрических и
нелинейных $\Delta M_i$ возмущений согласно~\cite{23s}:

\noindent
\begin{multline}
Y_iY_j \approx m_im_j + K_{ij} + m_j Y_i^0 + m_i Y_j^0\\
(i,j=\overline{1,4})\,,\label{e10s}
\end{multline}
\begin{multline*}
 \Delta M_i = \Delta_0 M_i (m,K) + \Delta_{1i}M_i (m,K)Y_1^0 +{}\\
{}+\Delta_{2i}M_i(m,K) Y_2^0\,,
\end{multline*}
где $m_i =\mm Y_i$~--- математические ожидания;
$Y_i^0 = Y_i-m_i$~--- центрированные составля-\linebreak ющие;
$K_{ij} = \mm Y_i^0 Y_j^0$~--- дисперсии и ковариации;
$\Delta_0 M_i=\mm [\Delta M_i]$,
$\Delta_{ji}M_i =\prt \Delta M_i/ \prt m_j$ $(i,j=1,2)$.
Тогда~(\ref{e6s}), (\ref{e8s}) при условии~(\ref{e10s})
будут статистически эквивалентны следующей
взаимосвязанной детерминированной нелинейной системе для
математических ожиданий $m_i$:
 $$\dot m = F\,,\ \ \ m = \left[ m_1 m_2 m_3 m_4\right]^T,\ \ \ \
 F= \left[ F_1 F_2 F_3 F_4\right]^T\,,$$
\begin{equation}
m_0 = m(t_0) =\mm \left[ p(t_0) q(t_0)\right]^T\label{e11s}
\end{equation}
и линейных стохастических уравнений для центрированных случайных
составляющих $Y_{i}^0$ $(Y_{i}^0=$\linebreak $= Y_i -m_i)$:
\begin{align}
\dot Y_1^0 &= - NY_2^0 + F_1^0 + V_1, \notag\\[-6pt]
\label{e12s}\\[-6pt]
\dot Y_2^0 &= + NY_1^0 + F_2^0 + V_2\,, \notag\\
\dot Y_j^0 &= - \alp Y_j^0 + \si_j \sqrt{2\alp_i} V_j \quad (j=3,4)\,.\label{e13s}
\end{align}
В (\ref{e11s})--(\ref{e13s}) введены следующие обозначения:
\begin{multline*}
F_1 = F_1(m,K, N_0 t, 2 N_0t) = M_{10}+{}\\
{}+ M_{11} \cos \left( N_0 t -\chi_1^{M_1}\right)
 + M_{12}\cos \left( 2 N_0 t - \chi_2^{M_1}\right) -{}\\
{}- N\left\{ \left[ \!\pi_1^{N_1}
\cos \left( N_0 t -\chi_1^{N_1}\right)\right.\right.+{}\\
\left.\left.{} +\pi_2^{N_1} \cos \left( 2 N_0 t - \chi_2^{N_1}\right)\right]
m_2 + m_2m_3 + k_{23}\right\} -{}\\
{}- D \left\{ \left[ 1 + \pi_1^{D_1}
\cos \left( N_0 t -\chi_1^{D_1}\right)\right.\right.+{}\\
\left.\left.{} + \pi_2^{D_1}\cos \left( 2 N_0 t - \chi_2^{D_1}\right)\right]
 m_1 + m_1m_4 + k_{14}\right\}+{}\\
{} +\Delta_0\mm_1 \,,
 \end{multline*}
 \begin{multline*}
F_2 = F_2(m,K, N_0 t, 2 N_0t)= M_{20}+{}\\
{}+M_{21} \cos \left( N_0 t -\chi_1^{M_2}\right)
 + M_{22}\cos \left( 2 N_0 t - \chi_2^{M_2}\right) +{}\\
{}+ N\left\{ \left[ \pi_1^{N_2}
\cos \left( N_0 t -\chi_1^{N_2}\right)\right.\right.+{}\\
\left.\left.{} +\pi_2^{N_2} \cos \left( 2 N_0 t - \chi_2^{N_2}\right)\right]
 m_1 + m_1m_3 + k_{13}\right\} -{}\\
{}- D \left\{ \left[ 1 + \pi_1^{D_2}
\cos \left( N_0 t -\chi_1^{D_2}\right)\right.\right.+{}\\
\left.\left.{} + \pi_2^{D_2}\cos \left( 2 N_0 t -
 \chi_2^{D_2}\right)\right] k m_2 + m_2m_4 + k_{24}\right\} +{}\\
{}+\Delta_0\mm_2\,,
 \end{multline*}
 \begin{equation}
 F_3 =-\alp_3m_3,\quad F_4 =-\alp_4 m_4\,;
 \label{e14s}
\end{equation}
\pagebreak

\noindent
\begin{multline*}
 F_1^0 = \sum\limits_{i=1}^4 \fr{\partial F_1}{\partial m_i}Y_i^0 ={}\\
 {}=
\left\{- D \left[ 1+\pi_1^{D_1} \cos \left(N_0 t -\chi_1^{D_1}\right)+{}\right.\right.\\
\left.\left.{}+ \pi_2^{D_1} \cos \left(2N_0t -\chi_2^{D_1}\right)+m_4\right]+\Delta_1 M_1\right\} Y_1^0 +{}\\
{}+\left\{- N \left[ \pi_1^{N_1} \cos \left(N_0t -\chi_1^{N_1}\right)+{}\right.\right.\\
\left.\left.{}+ \pi_2^{N_1} \cos \left(2N_0t -\chi_2^{N_1}\right)+ m_3\right] 
+\Delta_{12} M_1 \right\}Y_2^0 -{}\\
{}- Nm_2 Y_3^0 - D m_1 Y_4^0\,,
 \end{multline*}
 \begin{multline}
 F_2^0 = \sum\limits_{i=1}^4 \fr{\partial F_2}{ \partial m_i}Y_i^0 =
\left\{ N \left[\pi_1^{N_2} \cos \left(N_0 t -\chi_1^{N_2}\right)+{}\right.\right.\\
\left.\left.{}+ \pi_2^{N_2} \cos \left(2N_0t -\chi_2^{N_2}\right)+m_3\right] 
+\Delta_{21} M_2 \right\}Y_1^0+{}\\
{}+\left\{ - D \left[1+ \pi_1^{D_2} \cos \left(N_0t -\chi_1^{D_2}\right)+{}\right.\right.\\
\left.\left.{}+ \pi_2^{D_2} \cos \left(2N_0t -\chi_2^{D_2}\right)+ m_4\right]+\Delta_{22}M_1 \right\}Y_2^0 +{}\\
{}+ Nm_1 Y_3^0 - D m_2 Y_4^0\,,
 \label{e15s}
 \end{multline}
где $K= \left[k_{ij}\right]$ $(i,j=\overline{1,4})$~--- ковариационная
матрица~$Y^0_i$.

В силу линейности уравнения~(\ref{e12s}) и (\ref{e13s}) при
условиях~(\ref{e14s}) и (\ref{e15s})
допускают следующую матричную запись:
\begin{equation}
\dot Y^0 = \alp(t,m, K) Y^0 + \beta V\,.\label{e16s}
\end{equation}
Здесь введены обозначения
 \begin{align*}
 Y^0 &=\left[ Y_1^0Y_2^0Y_3^0Y_4^0\right]^T,\\
 V&=\left[ V_1 V_2 V_3 V_4\right]^T\,,%\label{e17s}
\\
\beta &= \mathrm{diag} \left( 1,1, \si_3 \sqrt{2\alp_3} , \si_4\sqrt{2\alp_4}\right)\,,
%\label{e18s}
\end{align*}
\begin{multline*}
 \alp =\alp(t, m,K, N_0t, 2N_0t) ={}\\
{}=\left[ \alp_{ij} (t, m,K)\right]\quad (i,j=\overline{1,4})\,,
\end{multline*}
\begin{multline*}
 \alp_{11}  =\alp_{11} (t,m, N_0t, 2N_0t)= {}\\
{}=- D \left[ 1+\pi_1^{D_1} \cos \left(N_0t  -\chi_1^{D_1}\right)+{}\right.\\
\left.{}+ \pi_2^{D_1} \cos \left(2N_0t -\chi_2^{D_1}\right)+m_4\right]+\Delta_{11}M_1\,,
\end{multline*}
\begin{equation*}
 \langle \alp_{11} \rangle=-D+\langle\Delta_{11}M_1\rangle\,,
\end{equation*}
\begin{multline*}
 \alp_{12} =\alp_{12} (t,m, N_0t, 2N_0t)={}\\
 - N \left[ 1+\pi_1^{N_1} \cos \left(N_0t -\chi_1^{N_1}\right)+{}\right.\\
\left.{} + \pi_2^{N_1} \cos \left(2N_0t -\chi_2^{N_1}\right)+ m_3\right]+\Delta_{12}M_1\,,
\end{multline*}
\begin{equation*}
 \langle\alp_{12}\rangle =-N +\langle\Delta_{12}M_1\rangle\,,
\end{equation*}
\begin{equation*}
 \alp_{13}  =\alp_{13} (t, m, N_0t, 2N_0t) = - N m_2\,,
%\enskip
\end{equation*}
\begin{equation*}
a_{14} (t, m) = - D m_1\,,
\end{equation*}
\begin{multline*}
 \alp_{21} =\alp_{21} (t, m, N_0t, 2N_0t)= {}\\
{}=N \left[1+\pi_1^{N_2} \cos \left(N_0t  -\chi_1^{N_2}\right)+{}\right.\\
\left.{} + \pi_2^{N_2} \cos \left(2N_0t -\chi_2^{N_2}\right)+m_3\right]+\Delta_{21}M_2\,,
\end{multline*}
\begin{equation*}
 \langle\alp_{21}\rangle=N+\langle\Delta_{21}M_2\rangle\,,
\end{equation*}
\begin{multline*}
 \alp_{22} =\alp_{22} (t,m, N_0t, 2N_0t)={}\\
{}= - D \left[1+ \pi_1^{D_2} \cos \left(N_0t -\chi_1^{D_2}\right)+{}\right.\\
\left.{} + \pi_2^{D_2} \cos \left(2N_0t -\chi_2^{D_2}\right)+ m_4\right]+\Delta_{22}M_2\,,
\end{multline*}
\begin{equation*}
 \langle\alp_{22} \rangle=-D+\langle\Delta_{22}M_2\rangle\,,
\end{equation*}
\begin{align}
 \alp_{23} &=\alp_{23} (t, m, N_0t, 2N_0t)= Nm_1\,,\notag\\%[4pt]
 \alp_{24} &=\alp_{24} (t,m, N_0t, 2N_0t) = - D m_2\,,\notag\\%[4pt]
  \alp_{31} &=\alp_{32}=\alp_{33}=0\,, \notag\\%[4pt]
 \alp_{33} &= -\alp_3\,, \notag\\%[4pt]
 \alp_{41} &=\alp_{42}=\alp_{43}=0\,,\notag\\%[4pt]
 \alp_{44} &=-\alp_4\,,\notag\\%[4pt]
 \alp_{25}&=\alp_{31} (t, m, N_0t, 2N_0t)= \alp_{32} (t, m) ={}\notag\\
{}&= \alp_{34} (t, m)=0\,,\notag\\%[4pt]
 \alp_{20} &=\alp_{41} (t, m, N_0t, 2N_0t)= \alp_{42} (t, m)={}\notag\\%[4pt]
{}&= \alp_{43} (t, m)=0\,,\label{e19s}
 \end{align}
где $\langle\ldots \rangle$~--- символ усреднения по времени $2\pi/N_0$.

Обозначим через $\langle\Phi (s, m)\rangle$ эквивалентную
передаточную функцию, в силу~(\ref{e16s}) связывающую~$Y^0$ и $V$ при
усредненных на $2\pi / N_0$ коэффициентах
$\langle \alp (t,m,N_0t, 2N_0 t)\rangle$:
\begin{equation}
Y^0 =\langle \Phi (s, m, N_0t, 2N_0 t)\rangle V\,,\label{e20s}
\end{equation}
 где $\Phi (s,m) = \beta [ \langle \alp(t,m, N_0t, 2N_0 t)\rangle - s I_4 ]^{-1}$,
$ I_4 $~---~единичная $(4\times 4)$-матрица.

В соответствии с формулами теории линейных стохастических систем~\cite{23s}
уравнениям~(\ref{e16s}) отвечают следующие детерминированные
уравнения для дисперсий и ковариаций $K(t) =[K_{ij}]$ $(i,j =$\linebreak
$={\overline{1,4}})$, ковариационных и взаимных ковариационных
функций $K(t_1, t_2)= [ K_{ij} (t_1, t_2)]$ $(i,j = {\overline{1,4}})$:
\begin{align}
\dot{K} (t) &= \alp(t, m, N_0t, 2N_0t) K(t) + {}\notag\\
{}&+K(t) \alp(t, m, N_0t, 2N_0t)^T + \beta\nu \beta^T\,,\notag\\
 K(t_0) &= K_0\,, \label{e21s}
 \end{align}
\begin{multline}
\fr{\partial K(t_1, t_2)}{\partial t_2} = \\
\!{}=K(t_1, t_2) \alp( t_2, m_2, N_0t_2, 2N_0t_2)^T \quad (t_1 < t_2)
\label{e22s}
\end{multline}
(при $t_1 > t_2$ $K(t_1, t_2) = K (t_2, t_1)^T$). Здесь $\nu=
[ \nu_{ij}]$ $(i,j = {\overline{1,4}})$ -- постоянная матрица
интенсивности векторного белого шума $V=[ V_1 V_2 V_3 V_4]^T$.

Вследствие зависимости $F= F(m,K, N_0t, 2N_0t)$ и $\alp
=\alp(m,K,N_0t, 2N_0t)$ от $m$ и $K$ система детерминированных
уравнений~(\ref{e11s}), (\ref{e21s}) и (\ref{e22s}) для математических
ожиданий, дисперсий и ковариаций представляет собой взаимосвязанную замкнутую
систему, определяющую корреляционную дифференциальную модель колебаний
полюса Земли с параметрическими возмущениями.

Таким образом, в основе базового квазилинейного корреляционного
метода построения аналитических информационных стохастических
моделей лежат уравнения~(\ref{e11s}), (\ref{e21s}) и (\ref{e22s}). Суждение об устойчивости
метода можно сделать на основе анализа устойчивости матрицы
$\alp$, определяемой~(\ref{e19s}).

В силу высокой добротности системы~(\ref{e8s}) базовые моментные
уравнения~(\ref{e11s}), (\ref{e21s}) и (\ref{e22s}) могут быть заменены
укороченными первого порядка для медленно меняющихся переменных, описывающих
колебания полюса Земли на частоте~$N$.

Из базовых моментных уравнений с учетом~(\ref{e20s}) вытекают
спект\-раль\-но-кор\-ре\-ля\-ци\-он\-ные версии квазилинейного метода,
обобщающие~\cite{12s, 19s}.\linebreak
 Наконец, если стохастические возмущения в
исходных уравнениях заданы стохастическими каноническими
представлениями (каноническими разложениями и интегральными
каноническими\linebreak представлениями~\cite{23s}), то~(\ref{e11s}) сохраняются, а~(\ref{e16s})
позволяют получить детерминированные уравнения для координатных
функций представлений. Тогда вместо~(\ref{e21s}) и~(\ref{e22s}) будут
использоваться известные конечные соотношения для $K(t)$ и
$K(t_1, t_2)$~\cite{23s}.

В качестве иллюстрации рассмотрим три тес\-товых примера практического
применения квазилинейного метода из состава информационных ресур\-сов
по проблеме <<Статистическая динамика вращения Земли>>.

\section{Пример~1.\ \ Регулярные колебания полюса Земли} %4

%{\sevenrm
 Для изучения регулярных колебаний полюса Земли при $\Delta M_i=0$ на
 основе уравнений для математических ожиданий~(\ref{e11s}) введем нормальные координаты $a$ и $\Delta
 \dot\psi$, имеющие смысл амплитуды и поправки к частоте,
 положив:

 \noindent
\begin{align}
m_1 &= m_{10} +a \cos\psi\,,\quad m_2 = m_{20} +a \sin\psi\,,\notag\\
 \psi&= Nt +\Delta\psi\,, \notag\\ %\label{e23s}\\
 m_{10} &= A_{10} = A_{10}( a,\Delta\psi, Y_3, Y_4, N_0 t, 2 N_0 t)={}\notag\\
&= - N^{-1} \langle F_2\rangle\,,\notag\\
 m_{20} &= A_{20} = A_{20}( a,\Delta\psi, Y_3, Y_4, N_0 t, 2 N_0 t)= {}\notag\\
&= N^{-1} \langle F_1\rangle\,,\label{e24s}
 \end{align}
где $\langle \ldots\rangle$~--- символ усреднения по периоду
чандлеровских колебаний $T =2 \pi N^{-1}$. Тогда в силу~(\ref{e11s}) придем
к следующим моментным уравнениям первого порядка для
амплитудно-частотных характеристик:
\begin{align}
\langle \dot a\rangle &= A = A( a,\Delta\psi, Y_3, Y_4, N_0 t, 2 N_0 t)={}\notag\\
&=\langle F_1\cos \psi\rangle+
 \langle F_2\sin \psi\rangle\,,\label{e25s}\\
\langle \Delta\dot \psi\rangle &= B = B( a,\Delta\psi, Y_3, Y_4, N_0 t, 2 N_0 t)={}\notag\\
&= -a^{-1}\langle F_1\sin \psi\rangle+a^{-1}
 \langle F_2\cos \psi\rangle\,.\label{e26s}
 \end{align}
Функции $A_{10}$, $A_{20}$, $A$ и $B$ в развернутой записи имеют
следующий вид:
\begin{multline}
A_{10} = -\fr{1}{N} \left\{
 M_{20} + N \left[ \langle m_1 m_3\rangle +\langle k_{13}\rangle + {}\right.\right.\\
\left.{}+\fr{1}{2} \pi_1^{N_2}a \cos \left(\Delta \psi +\tilde \chi_1^{N_2}\right)\right]
- D\left[ \langle m_2 m_4\rangle +\langle k_{24}\rangle - {}\right.\\
\left.\left.{}-m_{20}+ \fr{1}{2} \pi_1^{D_2}a \sin \left(\Delta \psi
 +\tilde \chi_1^{D_2}\right)\right]\right\},\label{e27s}
 \end{multline}
 \begin{multline}
A_{20} = \fr{1}{N} \left\{
 M_{10} - N \left[ \langle m_2 m_3\rangle +\langle k_{23}\rangle - {}\right.\right.\\
\left.{}-\fr{1}{2} \pi_1^{N_1}a \sin \left(\Delta \psi
 +\tilde \chi_1^{N_1}\right)\right]-{}\\
{}- D\left[ \langle m_1 m_4\rangle +\langle k_{14}\rangle + {}\right.\\
\left.\left.{}+ m_{10}+ \fr{1}{2} \pi_1^{D_1}a \cos \left(\Delta \psi
 +\tilde \chi_1^{D_1}\right)\right]\right\}\,,\label{e28s}
 \end{multline}
\begin{multline}
\!\!\!A= \fr{M_{11}}{ 2} \cos \left(\Delta \psi+\tilde \chi_1^{M_1}\right)+
\fr{M_{21}}{2} \sin \left(\Delta \psi+\tilde \chi_1^{M_2}\right)+{}\\
{}+\fr{N}{ 2} \left[ -\pi_1^{N_1} m_{20} \cos \left(\Delta \psi+\tilde \chi_1^{N_1}\right)+{}\right.\\
{}+ \pi_1^{N_2} m_{10} \sin \left(\Delta \psi+\tilde \chi_1^{N_2}\right)-{}\\
{}-\fr{\pi_2^{N_1} a}{2} \sin \left(2\Delta \psi+\tilde \chi_2^{N_1}\right)+
\fr{\pi_2^{N_2} a}{2} \sin \left(2\Delta \psi+\tilde \chi_2^{N_2}\right)-{}\\
\left. \vphantom{\pi_2^{N_1}}-2 \langle k_{23}\cos\psi\rangle + 2 \langle k_{13}\sin\psi\rangle\right]-{}\quad\notag
 \end{multline}
\begin{multline}
\quad{}- \fr{D}{2} \left[ 2a +
 \pi_1^{D_1} m_{10} \cos \left(\Delta \psi +\tilde\chi_1^{D_1}\right) -{}\right.\\
{}- \pi_1^{D_2} m_{20} \sin \left(\Delta \psi +\tilde\chi_1^{D_2}\right)+{}\\
{}+ \fr{\pi_2^{D_1} a}{2} \cos \left(2\Delta \psi +\tilde\chi_2^{D_1}\right) -
\fr{\pi_2^{D_2}a}{2} \cos \left(2\Delta \psi+\tilde\chi_2^{D_2}\right) -{}\\
\left. \vphantom{\pi_2^{N_1}}{}-2 \langle k_{14}\cos\psi\rangle-
 2\langle k_{24}\sin\psi\rangle\right]\,,\label{e29s}
 \end{multline}
\begin{multline}
\!\!\!\!B=\!-\fr{M_{11}}{2a} \sin \hspace*{-.4pt}\left(\Delta \psi+\tilde \chi_1^{M_1}\!\right)+
\fr {M_{21}}{2a} \cos \hspace*{-0.4pt}\left(\Delta \psi+\tilde \chi_1^{M_2}\!\right)\!+\\
{}+\fr{N}{2a} \left[ \pi_1^{N_1} m_{20} \sin \left(\Delta \psi+\tilde \chi_1^{N_1}\right)+\right.{}\\
{}+ \pi_1^{N_2} m_{10} \cos \left(\Delta \psi+\tilde \chi_1^{N_2}\right)-{}\\
{}-\fr{\pi_2^{N_1} a}{ 2} \cos \left(2\Delta \psi+\tilde \chi_2^{N_1}\right)+
\fr {\pi_2^{N_2} a}{ 2} \cos \left(2\Delta \psi+\tilde \chi_2^{N_2}\right)+{}\\
\left. \vphantom{\pi_2^{N_1}}{}+2 \langle k_{13}\cos\psi\rangle + 2 \langle k_{23}\sin\psi\rangle\right] -{}\\
{}- \fr{D}{2a} \left[
 - \pi_1^{D_1} m_{10} \sin \left(\Delta \psi +\tilde\chi_1^{D_1}\right) + \right.{}\\
{}+\pi_1^{D_2} m_{20} \cos \left(\Delta \psi +\tilde\chi_1^{D_2}\right)-{}\\
{}- \fr{\pi_2^{D_1} a}{ 2} \sin \left(2\Delta \psi +\tilde\chi_2^{D_1}\right) +
\fr{\pi_2^{D_2}a}{2}
 \sin \left(2\Delta \psi+\tilde\chi_2^{D_2}\right)-{}\\
\left. \vphantom{\pi_2^{N_1}}{}-2 \langle k_{14}\sin\psi\rangle + 2
 \langle k_{24}\cos\psi\rangle \right]\,,\label{e30s}
 \end{multline}
где волной сверху отмечены фазовые соотношения
\begin{align}
\tilde\chi_{1,2}^{D_i,N_i,M_i}&=\chi_{1,2}^{D_i,N_i,M_i}-\delta_N t, \delta_N ={}\notag\\
&= N-N_0 >0,  \quad i=\overline{1,4}\,.
 \label{e31s}
 \end{align}
 %\begin{multicols}{2}

При известных ковариационных характеристиках $k_{ij}$
$(i,j=\overline{1,4})$ уравнения (\ref{e25s}) и (\ref{e26s}) в совокупности с третьим
и четвертым уравнением (\ref{e11s}) представляют собой удобную для
аналитического моделирования детерминированную систему для изучения
медленно меняющихся параметров трендов $m_{10}, m_{20}$, амплитуды
$a$ и поправки к частоте $\Delta \dot\psi$ регулярных колебаний на
частоте, близкой к $N_0$.

Рассмотрим некоторые важные свойства этой модели. Полагая в~(\ref{e27s})--(\ref{e30s}):
\begin{gather}
D\ll N\,, \quad M_{10}\gg D m_{10}\,, \quad M_{20} \gg Dm_{20}\,,\notag\\
 m_{3,4} =0\,, \quad k_{13} = k_{23} = k_{14} = k_{24} =0\,,\label{e32s}
 \end{gather}
получим

%\end{multicols}
%@@@

\begin{multline*}
A_{10} \approx -\fr{1}{N} \left[ M_{20} + 
\fr{N}{ 2} \pi_1^{N_2} a \cos \left(\Delta \psi +\tilde\chi_1^{N_2}\right)-{}\right.\\
\left.{}-\fr{D\pi^{D_2}_1}{2} a \sin \left(\Delta \psi +\tilde\chi_1^{D_2} \right)\right]\,,
\end{multline*}
\begin{multline*}
A_{20} \approx \fr{1}{N} \left[ M_{10} + 
\fr{N\pi_1^{N_1}}{2} a \sin \left(\Delta \psi +\tilde\chi_1^{N_1}\right)-{}\right.\\
\left.{}-\fr{D\pi_1^{D_1}}{2} a \cos \left(\Delta \psi +\tilde\chi_1^{D_1} \right)\right]\,,
%\label{e33s}
\end{multline*}
\begin{multline}
\!\!\!A\approx \fr{M_{11}}{ 2} \cos \left(\Delta\psi +\tilde\chi_1^{M_1}\right) +
\fr{M_{21}}{2} \sin \left(\Delta\psi +\tilde\chi_1^{M_2}\right)+{}\\
{}+\fr{N}{2}\,a\left[ -\pi_2^{N_1} \sin \left(2\Delta\psi +\tilde\chi_2^{N_1}\right)+\right.{}\\
\left.{}+ \pi_2^{N_2} \sin \left(2\Delta\psi +\tilde\chi_2^{N_2}\right)\right]-{}\\
{}- Da\left[ 1+\fr{\pi_2^{D_1}}{ 4} \cos \left(2\Delta\psi +\tilde\chi_2^{D_1}\right)-{}\right.\\
\left.{}-\fr{\pi_2^{D_2}}{4} \cos \left(2\Delta\psi +\tilde\chi_2^{D_2}\right)\right]\,,
\label{e34s}
 \end{multline}
\begin{multline}
\!\!\!\!B\approx\!-\fr{M_{11}}{2a} \sin \hspace*{-0.4pt}\left(\Delta \psi +\tilde\chi_1^{M_1}\!\right)
+\fr{M_{21}}{2a} \cos \hspace*{-0.4pt}\left(\Delta \psi +\tilde\chi_1^{M_2} \!\right)\!+\\
{}+ \fr{N}{4}\left[-\pi_2^{N_1} \cos \left(2\Delta\psi +\tilde\chi_2^{N_1}\right)+{}\right.\\
\left.{}+ \pi_2^{N_2} \cos \left(2\Delta\psi +\tilde\chi_2^{N_2}\right)\right]-{}\\
{}-\fr{D}{4}\left[-\pi_2^{D_1} \sin \left(2\Delta\psi +\tilde\chi_2^{D_1}\right)+ {}\right.\\
\left.{}+\pi_2^{D_2} \sin \left(2\Delta\psi +\tilde\chi_2^{D_2}\right)\right]\,.
\label{e35s}
 \end{multline}
При $N=N_0$, полагая в~(\ref{e25s}) и~(\ref{e26s}) $\dot a =0$, $\Delta \dot \psi
=0$, получим уравнения для определения стационарных значений
$a_*$ и $\Delta \psi_*$:
\begin{multline*}
M_{11} \cos \left(\Delta\psi_* +\chi_1^{M_1}\right)+
 M_{21} \sin \left(\Delta\psi_* +\chi_1^{M_2}\right)+{}\\
{}+ a_* \left\{ N\left[ -\pi_2^{N_1} \sin\left(2\Delta\psi_* +\chi_2^{N_1}\right) +{}\right.\right.\\
{}+\left. \pi_2^{N_2} \sin \left(2\Delta\psi_* +\chi_2^{N_2}\right)\right] -{}\\
{}-D \left[ 1 + \fr{\pi_2^{D_1}}{ 4} \cos \left(2\Delta\psi_* +\chi_2^{D_1}\right)-{}\right.\\
\left.\left.{}-\fr{\pi_2^{D_2}}{ 4}  \cos \left(2\Delta\psi_* +\chi_2^{D_2}\right)\right]\right\}=0\,,
 \end{multline*}
 \begin{multline}
{}-M_{11} \sin\left(\Delta\psi_* +\chi_1^{M_1}\right)+M_{21} \cos \left(\Delta\psi_* +\chi_1^{M_2}\right)+{}\\
{}+\fr{a_*}{ 2} \left\{ N\left[ -\pi_2^{N_1} \cos \left(2\Delta\psi_* +\chi_2^{N_1}\right) +{}\right.\right.\\
\left.{}+ \pi_2^{N_2} \cos \left(2\Delta\psi_* +\chi_2^{N_2}\right)\right]-{}\quad\notag
 \end{multline}
\begin{multline*}
\quad{}-D\left[ -\pi_2^{D_1} \sin \left(2\Delta\psi_* +\chi_2^{D_1}\right)+{}\right.\\
\left. \left.{}+\pi_2^{D_2} \sin \left(2\Delta\psi_* +\chi_2^{D_1}\right)\right] \right\}=0\,.
%\label{e36s}
 \end{multline*}
При этом в силу~(\ref{e24s}) тренды будут определяться уравнениями
\begin{multline*}
 m_{10}= -\fr{M_{20}}{N} -\fr{\pi_1^{N_2}}{2} a \cos \left(\Delta \psi +\chi_1^{N_2}\right) -{}\\
{}-\fr {D}{2N} \pi_1^{D_2} a \sin \left(\Delta \psi +\chi_1^{D_2}\right)\,,
\end{multline*}
\begin{multline*}
 m_{20}= \fr{M_{10}}{N} +\fr{\pi_1^{N_1}}{2} a \sin \left(\Delta \psi +\chi_1^{N_1}\right) -{}\\
{}- \fr{D}{2N} \pi_1^{D_1} a \cos \left(\Delta \psi +\chi_1^{D_1}\right)\,.
%\label{e37s}
\end{multline*}

 %\begin{multicols}{2}

Отсюда можно сделать следующие выводы. 

%z
Во-первых, имеют место тренды
$m_{10}$ и $m_{20}$, вызываемые моментами $M_{10}$ и $M_{20}$, а
также гармоническими параметрическими гироскопическими и
диссипативными моментами на частоте $N_0$. При $N\ne N_0$ тренды
наряду с медленно меняющимися составляющими содержат биения на
частоте $\delta_N = N-N_0$. 

%z
Во-вторых, амплитудно-частотные
характеристики на частоте $N_0$, определяемые~(\ref{e25s}), (\ref{e26s}), (\ref{e34s}) и~(\ref{e35s}),
наряду с членами, пропорциональными аддитивным моментам
$M_{11}$, $M_{21}$, содержат\linebreak компоненты, происходящие от
параметрических гармонических гироскопических и диссипативных
моментов на удвоенной частоте $2N_0$. При $N\ne N_0$
амплитудно-час\-тот\-ные характеристики, кроме медленно меняющихся
составляющих, содержат биения на частоте~$\delta_N$. Так как $N \gg D$,
то гармонические гироскопические параметрические возмущения
оказывают более сильное влияние, чем соответствующие диссипативные.
В силу~(\ref{e29s}) эффективный динамический коэффициент диссипации
определяется из соотношения:
\begin{multline}
D^{\mathrm{Э}} =-\fr{\prt A}{\prt a} = D\left[ 1 + \fr{1}{4} \pi_2^{D_1}
 \cos \left(2\Delta\psi +\tilde\chi_2^{D_1}\right)\right. -{}\\
 {}-\left. \fr{1}{4} \pi_2^{D_2} \cos \left( 2 \Delta \psi + \tilde\chi_2^{D_2}\right)\right]+{}\\
{}+\fr{N}{4}\left[ \pi_2^{N_1} \sin \left(2\Delta\psi +\tilde\chi_2^{N_1}\right) - {}\right.\\
\left.{}-\pi_2^{N_2} \sin \left( 2 \Delta \psi + \tilde\chi_2^{N_2}\right)\right]\,.
\label{e38s}
\end{multline}
При этом $\Delta \psi$ находится интегрированием из~(\ref{e25s}), (\ref{e26s}) с
учетом~(\ref{e29s})--(\ref{e31s}). При $N=N_0$ эффективный коэффициент диссипации
$D^{\mathrm{Э}} $ не содержит биений на частоте $\delta_N$.
Для $\pi_2^{N_1} = \pi_2^{N_2}=0$ из~(\ref{e32s}) приходим к результату~\cite{19s},
объясняющему эффект динамической стабилизации чандлеровских
колебаний вследствие диссипативных возмущений на частоте~$2N_0$. Для
$\pi_2^{N_1}, \pi_2^{N_2}\ne 0$ соответствующий эффект обнаружен в~\cite{10s}.
При $N=N_0$, $ \chi_2^{D_1}= \chi_2^{D_2} = \chi_2^{D}$ и
$\chi_2^{N_1}= \chi_2^{N_2}= \chi_2^N$ из~(\ref{e38s}) имеем следующее
выражение для $D^{\mathrm{ Э}} $:
\begin{multline*}
D^{\mathrm{ Э}}= D \left[ 1+\fr{1}{4} \left( \pi_2^{D_1}
 - \pi_2^{D_2}\right) \cos 2\Delta\psi_*\right] +{}\\
 {}+ \fr{N}{4} \left( \pi_2^{N_1} - \pi_2^{N_2}\right)
 \sin 2\Delta \psi_*\,.
% \label{e39s}
 \end{multline*}
Отсюда следует, что при одинаковых $\pi_2^{D_i N_i}$ динамический
коэффициент диссипации равен статистическому значению
$D^{\mathrm{ Э}} =D$. При разных $\pi_2^{N_i}$
принципиально возможен эффект увеличения диссипации.


\section{Пример~2.\ \ Стохастические колебания полюса Земли
при~независимых аддитивных случайных возмущениях} %5


Уравнения~(\ref{e21s}) при отсутствии случайных параметрических возмущений
($Y_3$, $Y_4=0$) и $\Delta M_i =0$ принимают вид следующей
линейной детерминированной системы уравнений с периодическими
коэффициентами $\alp_{ij}$ $(i,j=1,2)$:
\begin{align}
 \dot k_{11} &=\nu_1 + 2 (\alp_{11} k_{11} +\alp_{12}
 k_{12})\,,\notag\\
 \dot k_{22} &=\nu_2 + 2 (\alp_{21} k_{12} +\alp_{22}
 k_{22})\,,\label{e40s}\\
 \dot k_{12} &=\alp_{21} k_{11}+ (\alp_{11} +\alp_{22})
 k_{12}+\alp_{12} k_{22}\,,\notag
\end{align}
где $\alp_{ij}$ $(i,j=1,2)$ определены~(\ref{e19s}) при $m_3=m_4 =$\linebreak $=0$. Отсюда
видно, что решение уравнений~(\ref{e40s}) приводит к выражениям вида
\begin{multline}
\!\!\!\!k_{ij} = k_{ij}^0 + k_{1,ij}' \cos N_0t + k_{1, ij}'' \sin N_0t+ k_{2, ij}' \cos 2N_0t+{}\\
{}+ k_{2,ij}'' \sin 2N_0t +{\hbox{высшие гармоники}}\,.\label{e41s}
\end{multline}
Здесь $k_{0,ij}$~--- определяют усредненные на интервале времени
$2\pi /N_0$ постоянные значения дисперсий $k_{11}=D_p$, $k_{22} =
D_q$ и ковариации $k_{12} = k_{pq}$; $k_{h, ij}'$ и $k_{h,ij}''$
$(h=1,2\tr\, i,j =1,2)$~-- гармоники $k_{ij}$ на частотах $N_0, 2
N_0 \tr h N_0$.

В этом случае уравнения~(\ref{e22s}) для ковариационных функций имеют вид
при $t_1 < t_2$:


\noindent
\begin{align*}
\fr{\partial K_{11} (t_, t_2) }{\partial t_2}& = \alp_{11 , t_2} K_{11} (t_1, t_2) + \alp_{12, t_2} K_{12} (t_1, t_2)\,,\notag\\
\fr{\partial K_{12} (t_, t_2)}{ \partial t_2} &= \alp_{21 , t_2} K_{11} (t_1, t_2) + \alp_{22, t_2} K_{12} (t_1, t_2)\,,\notag\\
\fr{\partial K_{21} (t_, t_2)} { \partial t_2} &= \alp_{11 , t_2} K_{21} (t_1, t_2) + \alp_{12, t_2} K_{22} (t_1, t_2)\,,\notag\\
\fr{\partial K_{22} (t_, t_2)} {\partial t_2} &= \alp_{21 , t_2} K_{21} (t_1, t_2) + \alp_{22, t_2} K_{22} (t_1, t_2)\,,
%\label{e42s}
\end{align*}
 а при $t> t_2$ $K(t_1, t_2) = K(t_2, t_1)^T$. Здесь введено обозначение
$\alp_{ij, t_2} =\alp_{ij} (t_2, m_2, N_0 t_2, 2N_0 t_2)$.

Таким образом, аддитивные случайные возмущения приводят к
стохастическим колебаниям,\linebreak ковариационные характеристики которых
описываются уравнениями~(\ref{e40s}), а регулярные компоненты~--- уравнениями
для математических\linebreak ожиданий~(\ref{e11s}) при $m_3 = m_4 =0$, $ k_{13}=
k_{23} =$\linebreak$= k_{14} =k_{24}=0$ в формулах (\ref{e27s})--(\ref{e30s}). Из этих формул
следует, что аддитивные случайные возмущения не влияют на регулярные
амплитудно-час\-тот\-ные характеристики и тренды.

В установившемся режиме в силу~(\ref{e20s}) спектральная плотность $s_y
(\w)$ и ковариационная матрица $K= [ k_{ij}]$ $(i, j=1,2)$
определяются формулами спектрально-корреляционной теории
стохастических систем~[23]:
\begin{align*}
s_y(\w) &= \langle \Phi (s, m, N_0 t, 2 N_0 t)\rangle s_v \langle
\Phi (s, m, N_0 t, 2 N_0 t)\rangle^*\,,\\ %\label{e43s}\\
K_y &=\int\limits_{-\infty}^\infty s_y(\w) d\w\,, %\label{e44s}
\end{align*}
где *~--- символ сопряжения; $s_v =\nu (2\pi)^{-1}$ и $\nu =
[ \nu_{ij}]$\ $( i, j=1,2)$~--- спектральная плотность и
интенсивность векторного белого шума $V= [ V_1 V_2 ]^T$
соответственно.
\medskip

%\end{multicols}
%\begin{figure*} %fig1
\vspace*{1pt}
\begin{center}
\mbox{%
\epsfxsize=79.139mm
\epsfbox{sin-1.eps}
}
\end{center}
\vspace*{3pt}
%\Caption{
{\small
\textbf{Рис. 1}\ \ Зависимости дисперсий $k_{11}$, $k_{22}$ и
ковариации $k_{12}$
 от времени для усредненных значений параметров}
%\label{f1s}}
%\end{figure*}
\bigskip
\addtocounter{figure}{1}
%\begin{multicols}{2}

На рис.~1 показаны результаты моделирования зависимости дисперсий
$k_{11}$, $k_{22}$ и ко\-ва\-риа\-ции $k_{12}$ в зависимости от
безразмерного време\-ни $\bar t = Nt$, выполненные на основе
уравнений (\ref{e40s}) при усредненных значениях $\langle\alp_{ij}\rangle$ в~(\ref{e19s}):
 $\langle\alp_{11}\rangle=$\linebreak$=\langle\alp_{22}\rangle=-D$,
 $\langle\alp_{21}\rangle =-\langle\alp_{12}\rangle=N$,
начальных условиях:
 $k_{11} (0) = k_{22} (0) =( (10^{-2})'')^2$, $k_{12}(0) =0$,
 $\nu_1 =\nu_2=\nu = 2\cdot 10^{-5}$ $(''/$год$)^2$ и $D^{-1} =
 10$~лет. Дисперсии $k_{11}$,
 $k_{22}$ и ковариации $k_{12}$ колеблются с частотой $N_0$ и для $t\gg 10$ лет
 асимптотически стремятся к следующим стационарным значениям: $k_{11}^* =
 k_{22}^*= \nu/2D$, $ k_{12}^*=0$. Время установления колебаний
 имеет порядок $D^{-1}=10$~лет.


\section{Пример~3.\ \ Стохастические колебания полюса при~независимых
случайных аддитивных и параметрических возмущениях}


При $\Delta M_i =0$ и $D/N \ll 1$ и независимых возмущениях
$V_j$ для переменных $Y_i$ $(i=\overline{1,4})$ имеем следующие
представления:
\begin{align}
Y_1 &= m_{10} + a \cos\psi + Y_1^0\,,\notag\\
Y_2 &= m_{20} + a \sin\psi + Y_2^0\,,\notag\\
Y_3 &= m_3 + Y_3^0\,,\notag\\
Y_4 &= m_4 + Y_4^0\,,\notag\\
\psi &= Nt +\Delta \psi\,.\label{e45s}
\end{align}
\begin{figure*} %fig2+3
\vspace*{1pt}
\begin{center}
\mbox{%
\epsfxsize=164.416mm
\epsfbox{sin-2.eps}
}
\end{center}
\vspace*{-9pt}
\Caption{Зависимости дисперсий $k_{11}$, $k_{22}$ и ковариации $k_{12}$
от времени для гироскопических гармонических возмущений на
(\textit{а})~чандлеровской частоте ($\pi_1^N=1$, $\pi_2^N=0$) и
\label{f2s}
(\textit{б})~удвоенной чандлеровской частоте ($\pi_1^N=0$, $\pi_2^N=1$)
}
\end{figure*}
Здесь тренды $m_{10}$, $m_{20}$ определяются из~(\ref{e24s}) при условиях~(\ref{e27s})
 и (\ref{e28s}), амплитудно-частотные характеристики $a$, $\Delta \dot \psi$~--- 
 из~(\ref{e25s}) и (\ref{e26s}) при условиях~(\ref{e29s})--(\ref{e31s}), а ковариационные
характеристики $k_{ij}$ $(i,j=\overline{1,4})$~--- из~(\ref{e21s}) и~(\ref{e22s}).
Уравнения~(\ref{e21s}) для дисперсий и ковариаций принимают вид:
\begin{align}
 \dot k_{11} & = \nu_1 + 2( \alp_{11} k_{11} +\alp_{12} k_{12}
 + \alp_{13} k_{13})\,,\notag\\
 \dot k_{12} & = \alp_{21} k_{11} +(\alp_{11}+\alp_{22}) k_{12}
 + \alp_{12} k_{22}  +{}\notag\\
 &\qquad\qquad\qquad\qquad+\alp_{23} k_{13}+\alp_{13} k_{23}\,,\notag\\
 \dot k_{22} & = \nu_2 + 2( \alp_{21} k_{12} +\alp_{22} k_{22}
 + \alp_{23} k_{23})\,,\notag\\
 \dot k_{33} & = 2 \alp_{3} (\si_3^2 -k_{33})\,,\notag\\
 \dot k_{44} &= 2 \alp_{4} (\si_4^2 - k_{44})\,,\notag\\
 \dot k_{34} &=-(\alp_{3}+\alp_4) k_{34}\,,\notag\\
 \dot k_{13} & = ( \alp_{11}-\alp_3) k_{13} +\alp_{12} k_{23}
 + \alp_{13} k_{33}\,,\notag\\
 \dot k_{14} & = ( \alp_{11}-\alp_4) k_{14} +\alp_{12} k_{24}
 + \alp_{13} k_{34}\,,\notag\\
 \dot k_{23} & = \alp_{21} k_{13} +(\alp_{22}-\alp_3) k_{23}
 + \alp_{23} k_{33}\,,\notag\\
 \dot k_{24} & = \alp_{21} k_{14} +(\alp_{22}-\alp_4) k_{24}
 + \alp_{23} k_{34}
 \label{e46s}
\end{align}
при начальных условиях $k_{ij,0}$ $(i,j=\overline{1,4})$. Входящие
в~(\ref{e46s}) коэффициенты $\alp_{ij} =\alp_{ij} (t, m, N_0 t, 2N_0 t)$
определяются~(\ref{e19s}). Решение уравнений~(\ref{e46s}) имеет вид~(\ref{e41s}). Таким
образом, моментные уравнения первого порядка для стохастических
колебаний полюса Земли при независимых случайных аддитивных и
параметрических возмущениях для переменных $m_{10}, m_{20}, a,
\Delta \psi$, $K= [ k_{ij}]$ $(i,j=\overline{1,4})$ в~(\ref{e45s}) имеют
вид~(\ref{e24s})--(\ref{e26s}) и~(\ref{e46s}).

Из этих уравнений следует ряд важных выводов. Во-первых, имеют место
тренды $m_{10}$ и $m_{20}$, вызываемые моментами $M_{10}$ и $M_{20}$,
а также параметрическими гармоническими и случайными
гироскопическими и диссипативными моментами. Особенностью влияния
случайных параметрических возмущений является появление следующих
дополнительных статических составляющих:
\begin{align*}
 m_{10} &= -\langle k_{13}\rangle + \fr{D}{ N} \langle k_{24}\rangle\,,\\
 m_{20} &= -\langle k_{23}\rangle - \fr{D}{ N} \langle k_{14}\rangle\,.
%\label{e47s})
 \end{align*}

Во-вторых, вследствие появления в~(\ref{e29s}) и (\ref{e30s}) дополнительных
составляющих на чаcтоте $N=N_0$ случайные параметрические возмущения
влияют на амплитудно-частотные характеристики, так что:
\begin{align*}
\Delta A &=-\langle k_{23}\cos \psi\rangle+\langle k_{13}\sin \psi\rangle+{}\\
&\qquad\qquad{}+ \fr{D}{Na} \left( \langle k_{24}\sin \psi\rangle +\langle k_{14}\cos \psi\rangle\right)\,,\\
%\end{multline*}
%\begin{multline*}
\Delta B &=\fr{N}{a}\left(-\langle k_{13}\cos \psi\rangle+\langle k_{23}\sin
 \psi\rangle\right)-{}\\
&\qquad\qquad{}- \fr{D}{a} \left( -\langle k_{14}\sin \psi\rangle +\langle k_{24}\cos \psi\rangle\right)\,,
%\label{e48s}
\end{align*}
поскольку выражения $\langle k_{ij}\cos \psi\rangle$, $\langle
k_{ij}\sin \psi\rangle$ $(i=1,2$, $j=3,4)$ в силу~(\ref{e46s}) отличны от нуля.

Наконец, приведем уравнения~(\ref{e22s}) при $D/N \ll $\linebreak $\ll 1$ для
ковариационных функций в развернутом виде:
%\end{multicols}
\begin{align*}
\fr{\partial K_{11} (t_1, t_2)}{ \partial t_2} & = \alp_{11 , t_2} K_{11} (t_1, t_2)+{}\\
 &+ \alp_{12, t_2} K_{12} (t_1, t_2) + \alp_{13, t_2} K_{13} (t_1, t_2)\,,\notag\\[2pt]
\fr{\partial K_{12} (t_1, t_2)}{\partial t_2} &= \alp_{21 , t_2} K_{11} (t_1, t_2)+{}\\
& + \alp_{22, t_2} K_{12} (t_1, t_2) + \alp_{23, t_2} K_{13} (t_1, t_2)\,,\notag\\[2pt]
\fr{\partial K_{22} (t_1, t_2)}{ \partial t_2} &= \alp_{21 , t_2}K_{21} (t_1, t_2)+{}\\
& + \alp_{22, t_2}K_{22} (t_1, t_2) + \alp_{23, t_2} K_{23} (t_1, t_2) \,,\notag\\[2pt]
\fr{\partial K_{21} (t_1, t_2)}{\partial t_2}& = \alp_{11 , t_2} K_{21} (t_1, t_2)+{}\\
& + \alp_{12, t_2} K_{22} (t_1, t_2) + \alp_{13, t_2} K_{23} (t_1, t_2)\,,\notag\\[2pt]
\fr{\partial K_{13} (t_1, t_2)}{\partial t_2}& =- \alp_{3} K_{13} (t_1, t_2)\,,\\
\fr{\partial K_{14} (t_1, t_2)}{ \partial t_2} &=- \alp_{4} K_{14} (t_1,  t_2)\,,\notag\\[2pt]
\fr{\partial K_{23} (t_1, t_2)}{\partial t_2} &=- \alp_{3} K_{23} (t_1, t_2)\,,\\[2pt]
\fr{\partial K_{24} (t_1, t_2)}{\partial t_2} &=- \alp_{4} K_{24} (t_1, t_2)\,,\notag\\[2pt]
\fr{\partial K_{33} (t_1, t_2)}{ \partial t_2}& =- \alp_{3} K_{33} (t_1, t_2)\,,\\[2pt]
\fr{\partial K_{34} (t_1, t_2)}{ \partial t_2} &=- \alp_{43} K_{34} (t_1, t_2)\,,\notag
 \end{align*}
 \begin{align*}
\fr{\partial K_{43} (t_1, t_2)}{ \partial t_2}& =- \alp_{3} K_{43} (t_1, t_2)\,,\\
\fr{\partial K_{44} (t_1, t_2)}{\partial t_2} &=- \alp_{4} K_{44} (t_1, t_2)\,,\notag\\
\fr{\partial K_{31} (t_1, t_2)}{\partial t_2}& = \alp_{11 , t_2} K_{31} (t_1, t_2)+{}\\
& + \alp_{12, t_2} K_{32} (t_1, t_2) + \alp_{13, t_2} K_{33} (t_1, t_2)\,,\notag\\
% \end{align*}
% \begin{align*}
\fr{\partial K_{41} (t_1, t_2)}{\partial t_2} &= \alp_{11 , t_2} K_{41} (t_1, t_2)+{}\\
& + \alp_{12, t_2} K_{42} (t_1, t_2) + \alp_{13, t_2} K_{43} (t_1, t_2)\,,\notag\\
\fr{\partial K_{32} (t_1, t_2)}{\partial t_2} &= \alp_{21 , t_2} K_{31} (t_1, t_2)+{}\\
& + \alp_{22, t_2}K_{32} (t_1, t_2) + \alp_{23, t_2} K_{33} (t_1, t_2)\,,\notag\\
\fr{\partial K_{42} (t_1, t_2)}{\partial t_2}& = \alp_{21 , t_2} K_{41} (t_1, t_2)+{}\\
& + \alp_{22, t_2}K_{42} (t_1, t_2) + \alp_{23, t_2} K_{43} (t_1, t_2)\,.
%\label{e49s}
 \end{align*}
 %\begin{multicols}{2}

На рис.~\ref{f2s} показаны зависимости дисперсий и ковариации,
полученные моделированием уравнений~(\ref{e40s}) при больших синфазных
гироскопических гармонических возмущениях на частотах $N_0 $ и
$2N_0$. Отметим искажение формы колебаний $k_{11}$, $k_{12}$, $k_{22}$.
Влияние гироскопических случайных и гармонических синфазных
возмущений на частотах $N_0$ и $2N_0$ проявляется как в искажении
формы колебаний $k_{11}$, $k_{12}$, $k_{22}$, так и в появлении малых
дополнительных трендов $\sim 10^{-4}$~$''/$год.


\section{Методы параметрического корреляционного построения
аналитических информационных моделей} %7

Уравнения~(\ref{e8s}) (понимаемые в смысле Ито) при условиях~(\ref{e9s}) и
$\Delta M_i =0$ $(i=1,2)$ для переменных $Y= [ Y_1 Y_2]^T$, $
Y_1 = p$, $ Y_2 = q$ допускают следующую векторную запись:
\begin{equation*}
\dot Y = a_0 + a Y + (b_0 + b_1Y_1+ b_2 Y_2) V\,,\label{e50s}
\end{equation*}
где
\begin{align}
a_0&=
\begin{pmatrix}
 a_{01}\\ a_{02}
 \end{pmatrix}\,,
 \quad \langle a_0 \rangle =
 \begin{pmatrix}
 M_{10}\\ M_{20}\end{pmatrix}\,,\label{e51s}
 \end{align}
\begin{multline*}
a_{0i} = M_{i0} + M_1^i \cos \left(N_0 t -\chi_1^{M_1}\right) +{}\\
+ M_2^i \cos \left(N_0 t -\chi_2^{M_2}\right)\quad  (i=1,2)\,,
\end{multline*}
 \begin{equation}
a =\begin{pmatrix}
 a_{11}&a_{12}\\ a_{21}& a_{22}\\
 \end{pmatrix}\,, \quad \langle a \rangle =
 \begin{pmatrix}
 -D&-N\\ N&-D\\
 \end{pmatrix}\,,\label{e52s}
 \end{equation}
\begin{multline*}
a_{11} =-D\left[ 1 + \pi_1^{D_1} \cos \left(N_0t -\chi_1^{D_1}\right)\right. +{}\notag\\
+\left. \pi_2^{D_1} \cos \left(2N_0t -\chi_2^{D_1}\right)\right]\,,\notag
\end{multline*}
\begin{multline*}
a_{12} =-N\left[ 1 + \pi_1^{N_1} \cos \left(N_0t -\chi_1^{N_1}\right)\right. +{}\notag\\
+ \left.\pi_2^{N_1} \cos \left(2N_0t -\chi_2^{N_1}\right)\right]\,,\notag
\end{multline*}
\begin{multline*}
a_{21} =N\left[ 1 + \pi_1^{N_2} \cos \left(N_0t -\chi_1^{N_2}\right) \right.+{}\notag\\
+ \left.\pi_2^{N_2} \cos \left(2N_0t -\chi_2^{N_2}\right)\right]\,,\notag
\end{multline*}
\begin{multline*}
a_{22} =-D\left[ 1 + \pi_1^{D_2} \cos \left(N_0t -\chi_1^{D_2}\right)\right. +{}\notag\\
+ \left.\pi_2^{D_2} \cos \left(2N_0t -\chi_2^{D_2}\right)\right]\,,\notag
\end{multline*}
\begin{align*}
b_0&=\begin{pmatrix}
 b_{01}\\ b_{02}\end{pmatrix}\,,
 \quad b_{01}= \gamma_1,\quad b_{02}=\gamma_2\,,\notag \\ %\label{e53s}
b_h &=\begin{pmatrix}
 b_{h1}\\ b_{h2}\end{pmatrix}
 \quad (h=1,2)\,,\notag\\
b_{11} &= -D\gamma_4\,,\quad b_{12}= N\gamma_3\,,\notag\\
 b_{21} &= -N\gamma_3\,,\quad b_{22}= -D\gamma_4\,.\notag %\label{e54s}
 \end{align*}

Векторному стохастическому дифференциальному уравнению~(\ref{e5s})
отвечают следующие точные нелинейные взаимосвязанные
детерминированные уравнения для моментов первого и второго порядка~\cite{23s}:
\begin{equation}
\dot m = a m + a_0\,,\quad m(t_0) = m_0\,,\label{e55s}
\end{equation}
\begin{multline}
\dot K = a K + K a^T + b_0 \nu b_0^T +
\sum\limits_{h=1}^2 \left( b_h \nu b_0^T + b_0 \nu b_h^T\right) m_h +{}\\
{}+\sssd\limits_{l,h=1} b_h \nu b_l^T (m_h m_l + k_{hl})\,,\quad K(t_0) = K_0\,,\label{e56s}
\end{multline}
 $$\fr{\prt K (t_1, t_2) }{\prt t_2} = K(t_1, t_2) a_{t_2}^T\,,$$
 \begin{align}
K(t_1, t_2) &= K(t_1)& \hbox{ при } t_1&< t_2\,,\notag\\
K(t_1, t_2) &= K(t_2,t_1)^T& \hbox{ при } t_2&< t_1\,.\label{e57s}
\end{align}

 \begin{figure*}[b] %fig4+5
\vspace*{1pt}
\begin{center}
\mbox{%
\epsfxsize=164.412mm
\epsfbox{sin-4.eps}
}
\end{center}
\vspace*{-9pt}
 \Caption{Зависимости дисперсий $k_{11}$, $k_{22}$ и ковариации $k_{12}$
от времени при (\textit{а})~стохастических аддитивных
и параметрических гироскопических возмущениях
 и (\textit{б})~стохастических аддитивных 
 и параметрических диссипативных возмущениях
 \label{f4s}}
 \end{figure*}


Суждение об устойчивости метода в среднем и в среднеквадратичном
можно сделать на основе\linebreak
\pagebreak

\noindent
 анализа устойчивости уравнений в вариациях
для~(\ref{e55s}) и~(\ref{e56s}):
\begin{align*}
\delta \dot m &= a \delta m\,,\\ %\label{e58s}\\
\delta \dot K &= a \delta K +\delta K a^T +\sssn\limits_{l,h=1} b_h \nu b_l^T k_{hl}\,.
%\label{e59s}
\end{align*}

Моментные уравнения~(\ref{e55s})--(\ref{e57s}) лежат в основе базового
параметрического корреляционного метода построения аналитических
информационных моделей флуктуаций полюса Земли. Уравнения~(\ref{e55s})--(\ref{e57s})
вследствие высокой добротности системы позволяют провести
переход к укороченным уравнениям согласно~\cite{22s}. Метод допускает
обобщение на случай, когда возмущения $V_1$, $V_2$, $X_3$, $X_4$ имеют
несколько независимых источников.
{\looseness=1

}
При практическом применении метода, когда исходные уравнения
понимаются в смысле Стратоновича, следует иметь в виду, что в
общем случае $a_0$ и $a$ будут содержать дополнительные члены,
зависящие от интенсивности белых шумов~[9--11]. %\cite{8s, 10s, 11s}.

Из базовых моментных уравнений~(\ref{e55s})--(\ref{e57s}), в случае задания
возмущений стохастическими каноническими представлениями, выводятся
соответствующие уравнения для математических ожиданий и координатных
функций, составляющих основу соответствующей версии метода
параметрического корреляционного построения аналитических моделей.

Приведем тестовый пример использования базового метода из состава
информационных ресурсов по проблеме <<Статистическая динамика
вращения Земли>>.


\section{Пример~4.\ \ Стационарные стохастические параметрические колебания полюса Земли
при~отсутствии аддитивных постоянных и~гармонических возмущений} %8


Полагая в~(\ref{e55s}) и~(\ref{e56s}) $M_{i0} =0$, $ M_j^i =0$\ $(i,j=1,2)$,
$ \nu=\nu^*$, $ \dot m^* =0$, $ \dot K^* =0$, по формулам~(\ref{e51s}) и~(\ref{e52s})
находим $a_0=0$, $ a=\langle a\rangle$. Следовательно,
стационарные значения $m^*=0$, а $K^*=[k_{ij}]$ определяются из
уравнения
\begin{equation*}
\langle a\rangle K^* + K^* \langle a\rangle^T + b_0\nu^* b_0^T +\nu^*
 \sssd\limits_{l,h=1} b_h b_l^T k_{hl}^*\,.
%\label{e60s}
 \end{equation*}


Процесс установления стационарных стохастических колебаний в силу~(\ref{e55s}) и~(\ref{e56s})
описывается следующими уравнениями:
\begin{multline*}
\dot k_{11}  =- 2 D \left( 1 -\fr{\nu D\gamma_4^2}{2}\right) k_{11} + \nu N^2 \gamma_3^2 k_{22}\notag -{}\\
{}- 2 N\left( 1 -\nu D\gamma_3\gamma_4\right) k_{12}+\nu \gamma_1^2\,,\notag
\end{multline*}
\begin{multline*}
\dot k_{12} =N\left( 1 -\nu D\gamma_3\gamma_4\right) k_{11} 
-N\left(1- \nu D \gamma_3 \gamma_4\right)k_{22}\notag -{}\\
{} -2D\left[ 1 -\fr{\nu}{2 D} (N^2\gamma_3^2 - D^2\gamma_4^2)\right] k_{12}\,,\notag
\end{multline*}
\begin{multline}
\dot k_{22}  =\nu N^2 \gamma_3^2 k_{11}- 2 D \left( 1 -\fr{\nu D\gamma_4^2}{2}\right) k_{22} +{}\\
 {}+ 2 N\left( 1 -\nu D\gamma_3\gamma_4\right) k_{12}+\nu \gamma_2^2\,.
\label{e61s}
\end{multline}


Влияние стохастических параметрических возмущений проявляется как
посредством снижения эффективных коэффициентов моментов сил
диссипации $D$ и моментов гироскопических сил $N$, так и в
результате появления дополнительного члена с~$k_{22}$ в первом
 и члена с~$k_{11}$ во втором уравнении~(\ref{e61s}).


На рис.~\ref{f4s} приведены графики установления стационарных
параметрических стохастических колебаний полюса Земли.


\section{Заключение} %9

Разработанные базовые корреляционные методы построения аналитических
информационных моделей флуктуаций полюса Земли по априорным данным
реализованы в виде экспериментального программного обеспечения в
среде MATLAB.

Как показали проведенные вычислительные эксперименты, базовый
квазилинейный метод требует составления $Q_{\mathrm{ КЛМ}} =
(n+2) (n+5)/2$, где $n$~--- число учитываемых возмущений,
определяемых линейным формирующим фильтром первого порядка.
Уравнения~(\ref{e11s}), (\ref{e21s}) и~(\ref{e22s}) нелинейны относительно $m$ и $K$. Метод
допускает обобщение и на негауссовские случайные возмущения.

Базовый метод параметрического корреляционного построения
предполагает составление $Q_{\mathrm{ ПКМ}} = 2\cdot 5/2=5$
уравнений. Уравнения~(\ref{e55s}) и~(\ref{e57s}) линейны относительно $m$ и $K(t_1,
t_2)$, а~(\ref{e56s})~--- относительно $k_{ij}$, но нелинейны относитель\-но~$m_i$.
 Уравнения~(\ref{e55s})--(\ref{e57s}) справедливы только для линейных
стохастических уравнений с параметрическими и аддитивными
возмущениями типа нормального (понимаемых в смысле Ито) белого\linebreak  шума.
Метод допускает обобщение на негауссовские белые шумы. В
практических задачах могут быть использованы комбинированные версии.

Среди направлений дальнейшего развития методов построения
аналитических информационных моделей флуктуаций полюса Земли
следует выделить:
\begin{itemize}
\item быстрые off-line методы построения моделей по апостериорным
данным;
\item
 оперативные on-line методы построения моделей на основе
фильтров Калмана и Пугачева и~др.~\cite{24s, 25s}.
\end{itemize}

\bigskip
Автор выражают благодарность Н.\,Н.~Семендяеву за помощь в выполнении
расчетов.

{\small\frenchspacing
{%\baselineskip=10.8pt
\addcontentsline{toc}{section}{Литература}
\begin{thebibliography}{99}
\bibitem{1s}
\Au{Манк~Н., Макдональд~Г.}
Вращение Земли.~--- М.: Мир, 1964.

\bibitem{2s}
\Au{Мориц~Г., Мюллер~А.}
Вращение Земли: теория и наблюдения.~--- Киев: Наук. думка, 1992.

\bibitem{3s}
IERS Annual Reports, 2000, 2001,2002 (Frankfurt am Mein: BKG. 2001--2003).

\bibitem{4s}
\Au{Акуленко Л.\,Д., Кумакшев~С.\,А., Марков~Ю.\,Г., Рыхлова~Л.\,В.}
Гравитационно-приливной механизм колебаний полюса Земли~// Астрон.
журн., 2005. Т.~82. №\,10. С.~950--960.

\bibitem{5s}
\Au{Марков Ю.\,Г., Синицын~И.\,Н.}
Стохастическая модель движения
полюса деформируемой Земли~// ДАН, 2002. Т.~385. №\,2. С.~186--192.

\bibitem{6s}
\Au{Марков Ю.\,Г., Синицын~И.\,Н.}
Флуктуационно-дис\-си\-па\-тивная
модель движения полюса деформируемой Земли~// ДАН, 2002. Т.~387.
№\,4. С.~482--486.

\bibitem{7s}
\Au{Марков~Ю.\,Г., Синицын~И.\,Н.}
Нелинейные стохастические
корреляционные модели движения полюса деформируемой Земли~// Астрон.
журн., 2003. Т.~80. №\,2. С.~186--192.

\bibitem{9s}
\Au{Марков Ю.\,Г., Синицын~И.\,Н.} Распределение флуктуаций движения полюса Земли // ДАН, 2003. Т.~390. №\,3. С.~343--346.
%Многомерные распределения флуктуаций движения полюса Земли~// ДАН, 2003. Т.~391. №\,2. С.~194--198.

\bibitem{10s}
\Au{Марков Ю.\,Г., Синицын~И.\,Н.}
Многомерные распределения
флуктуаций полюса Земли~// ДАН, 2003. Т.~391. №\,2. С.~194--198.

\bibitem{11s}
\Au{Марков Ю.\,Г., Синицын~И.\,Н.}
Спектрально-кор\-ре\-ля\-ци\-онные модели флуктуаций вращательного
движения Земли~// ДАН, 2003. Т.~393. №\,5. С.~618--623.

\bibitem{8s}
\Au{Марков Ю.\,Г., Синицын~И.\,Н.} Спектрально-корреляционные и кинетические 
модели движения Земли // Астрон. журнал, 2004. Т.~81. №\,2. С.~184--192.
%Влияние параметрических флуктуационно-дисси\-пативных сил на движение полюса Земли~// ДАН, 2003. Т.~390. №\,3. С.~343--346.
{\looseness=1

}
\bibitem{12s}
\Au{Марков Ю.\,Г., Синицын~И.\,Н.}
Влияние параметрических флуктуационно-диссипативных сил на движение
полюса Земли~// ДАН, 2004. Т.~395. №\,1. С.~51--54.
{\looseness=1

}
\bibitem{13s}
\Au{Синицын~И.\,Н.}
Стохастические модели флуктуаций движения Земли в условиях пуассоновских
возмущений~// Системы и средства информатики.
Спец. вып. Геоинформационные технологии.~--- М.: ИПИ РАН, 2004. С.~39--55.

\bibitem{14s}
\Au{Марков Ю.\,Г., Дасаев~Р.\,Р., Перепелкин~В.\,В., Синицын~И.\,Н.,
Синицын~В.\,И.}
Стохастические модели вращения Земли с учетом влияния Луны и планет~//
Космические исследования, 2005. Т.~43. №\,1. С.~54--66.

\bibitem{15s}
\Au{Синицын~И.\,Н.}
Стохастические информационные модели негауссовских флуктуаций движения
полюса Земли~// Системы и средства информатики. Спец. вып.
<<Научно-методические проблемы информатики>>.~---
М.: ИПИ РАН, 2006. С.~157--178.

\bibitem{16s}
\Au{Марков~Ю.\,Г., Синицын~И.\,Н.}
Чандлеровские колебания движения полюса Земли~//
ДАН, 2006. Т.~407. С.~485--488.

\bibitem{17s}
\Au{Марков~Ю.\,Г., Синицын~И.\,Н.}
Флуктуации чандлеровских колебаний полюса Земли~//
ДАН, 2006. Т.~409. №\,1. С.~48--51.

\bibitem{18s}
\Au{Марков Ю.\,Г., Синицын~И.\,Н.}
Чандлеровские колебания полюса Земли при параметрических возмущениях~//
ДАН, 2006. Т.~410. №\,4. С.~1--3.

\bibitem{19s}
\Au{Марков Ю.\,Г., Синицын~И.\,Н.}
Спектрально-кор\-ре\-ля\-ци\-он\-ная
модель флуктуаций чандлеровских колебаний полюса Земли~//
Астрон. журн., 2006. Т.~83. №\,10. С.~950--960.

\bibitem{20s}
\Au{Марков Ю.\,Г., Перепелкин~В.\,В., Синицын~И.\,Н., Корепанов~Э.\,Р.,
Хоанг~Тхо Ши.} Амплитудно-частотный анализ чандлеровских колебаний полюса 
Земли~// Космические исследования, 2007. №\,6.

\bibitem{21s}
\Au{Марков Ю.\,Г., Синицын~И.\,Н.} Стохастическая модель колебаний полюса Земли 
с параметрическими возмущениями~// ДАН, 2007. Т.~417. №\,1. С.~1--4.

\bibitem{22s}
\Au{Синицын~И.\,Н.}
Об укороченных моментных уравнениях статистической динамики движения
полюса Земли~// Системы и средства информатики. Спец. вып.
<<Математические модели в информационных технологиях>>.~---
М.: ИПИ РАН, 2006. С.~24--46.

\bibitem{23s}
\Au{Пугачев В.\,С., Синицын~И.\,Н.}
Теория стохастических систем. 2-е изд.~--- М.: Логос, 2004.

\bibitem{24s}
\Au{Синицын И.\,Н.} Фильтры Калмана и Пугачева. 2-е. изд.~--- М.: Логос, 2007.

\bibitem{25s}
\Au{Синицын И.\,Н.}
Развитие теории фильтров Пугачева для оперативной обработки информации
в стохастических системах~//
Информатика и её применения, 2007. Т.~1. Вып.~1. С.~3--13.
\end{thebibliography}

}
}

\end{multicols}


\label{end\stat}