\documentclass[10pt]{book}
\usepackage[utf8]{inputenc}

\usepackage{latexsym,amssymb,amsmath,indentfirst,shapepar,%fleqn,%
picinpar,shadow,floatflt,enumerate,multicol,ipi}

\usepackage{acad}
\usepackage{courier}
\usepackage{decor}
\usepackage{newton}
\usepackage{pragmatica}
\usepackage{zapfchan}
\usepackage{petrotex}

\input{epsf}

\renewcommand{\bottomfraction}{0.99}
\renewcommand{\topfraction}{0.99}
\renewcommand{\textfraction}{0.01}

%\nofiles

%\includeonly{sinits}
%\includeonly{zahar}
%\includeonly{pech}
%\includeonly{bosov}  %есть
%\includeonly{borisov}
%\includeonly{kush1} %есть
%\includeonly{obchak}
%\includeonly{eng-index}
%\includeonly{cover3}

\setcounter{secnumdepth}{1} %здесь - 3 + chapter = 4


%\usepackage[pdftex]{graphicx}

%\usepackage{oz}

%NEW COMMANDS
\newcommand{\bpre}{\;\mbox{PRE}\;}
\newcommand{\bselect}{\;\mbox{SELECT}\;}
\newcommand{\bwhen}{\;\mbox{WHEN}\;}
\newcommand{\bend}{\;\mbox{END}\;}
\newcommand{\bwhile}{\;\mbox{WHILE}\;}
\newcommand{\bdo}{\;\mbox{DO}\;}
\newcommand{\bif}{\;\mbox{IF}\;}
\newcommand{\bthen}{\;\mbox{THEN}\;}
\newcommand{\belse}{\;\mbox{ELSE}\;}
\newcommand{\belsif}{\;\mbox{ELSIF}\;}
\newcommand{\bany}{\;\mbox{ANY}\;}
\newcommand{\bwhere}{\;\mbox{WHERE}\;}
\newcommand{\bcase}{\;\mbox{CASE}\;}
\newcommand{\bof}{\;\mbox{OF}\;}
\newcommand{\beither}{\;\mbox{EITHER}\;}
\newcommand{\bor}{\;\mbox{OR}\;}
\newcommand{\bvar}{\;\mbox{VAR}\;}
\newcommand{\bin}{\;\mbox{IN}\;}
\newcommand{\bbegin}{\;\mbox{BEGIN}\;}

\newcommand{\bmachine}{\;\mbox{MACHINE}\;}
\newcommand{\brefinement}{\;\mbox{REFINEMENT}\;}
\newcommand{\brefines}{\;\mbox{REFINES}\;}
\newcommand{\bvariables}{\;\mbox{VARIABLES}\;}
\newcommand{\bsets}{\;\mbox{SETS}\;}
\newcommand{\binvariant}{\;\mbox{INVARIANT}\;}
\newcommand{\binitialisation}{\;\mbox{INITIALISATION}\;}
\newcommand{\bconstants}{\;\mbox{CONSTANTS}\;}
\newcommand{\bproperties}{\;\mbox{PROPERTIES}\;}
\newcommand{\bdefinitions}{\;\mbox{DEFINITIONS}\;}
\newcommand{\boperations}{\;\mbox{OPERATIONS}\;}

\newcommand{\bincludes}{\;\mbox{INCLUDES}\;}
\newcommand{\buses}{\;\mbox{USES}\;}
\newcommand{\bsees}{\;\mbox{SEES}\;}

\newcommand{\bdom}{\,\mbox{dom}\,}
\newcommand{\bran}{\,\mbox{ran}\,}
\newcommand{\bseq}{\,\mbox{seq}\,}
\newcommand{\bsize}{\,\mbox{size}\,}
\newcommand{\bmax}{\,\mbox{max}\,}
\newcommand{\bmin}{\,\mbox{min}\,}
\newcommand{\bfirst}{\,\mbox{first}\,}
\newcommand{\blast}{\,\mbox{last}\,}
\newcommand{\bcard}{\,\mbox{card}\,}

\newcommand{\s}{ ${}\hspace{5pt}$ }
\newcommand{\ds}{ ${}\hspace{10pt}$ }
\newcommand{\dds}{ ${}\hspace{15pt}$ }
\newcommand{\ddds}{ ${}\hspace{20pt}$ }
\newcommand{\dddds}{ ${}\hspace{25pt}$ }

\newtheorem{definition}{Определение}
\renewcommand{\thedefinition}{\arabic{definition}.}
%END NEW COMMANDS

%\renewcommand{\baselinestretch}{1.2}

%\pagestyle{myheadings}

\setlength{\textwidth}{167mm}      % 122mm
\setlength{\textheight}{658pt}
%\setlength{\textheight}{635.6pt}
\setlength{\columnsep}{4.5mm}

\setcounter{secnumdepth}{4}

%\addtolength{\headheight}{2pt}
%\addtolength{\headsep}{-2mm}

%\addtolength{\topmargin}{-20mm}  % for printing


\hoffset=-30mm  % From Yap
%\hoffset=-20mm  % From Acrobat

%\voffset=0mm % From Yap
%\voffset=-15mm   % From Acrobat

\addtolength{\evensidemargin}{-9.5mm} % for printing
\addtolength{\oddsidemargin}{9.5mm}  % for printing

%\renewcommand{\thefootnote}{\fnsymbol{footnote}}
%\renewcommand{\thefootnote}{\arabic{footnote}}
\renewcommand{\figurename}{\protect\bf Рис.}
\renewcommand{\tablename}{\protect\bf Таблица}

\newcommand{\Caption}[1]{\caption{\protect\small %\baselineskip=2.5ex
#1}}

\renewcommand{\thefigure}{\arabic{figure}}
\renewcommand{\thetable}{\arabic{table}}
\renewcommand{\theequation}{\arabic{equation}}
\renewcommand{\thesection}{\arabic{section}}

\renewcommand{\contentsname}{СОДЕРЖАНИЕ}
\newcommand{\fr}[2]{\displaystyle\frac{\displaystyle #1\mathstrut}{\displaystyle #2\mathstrut}}

%\renewcommand{\thefootnote}{\fnsymbol{footnote}}
%\newcommand{\g}{\mbox{\textit{g}}}

%\newcommand{\Caption}[1]{\caption{\protect\small\baselineskip=2ex #1}}
\newcounter{razdel}
\setcounter{razdel}{0}


\newcommand{\titel}[4]{%
\

\vspace*{5pt}

\ifodd\therazdel {\raggedright\noindent\Large\textrm\textbf
 \lineskip .75em
  \baselineskip=3.2ex #1 \par}
\vskip 1em {\noindent\large\textrm\textbf #2 \par}
\addcontentsline{toc}{subsection}{{\textrm\textbf #3}\protect\newline #1}
\def\rightheadline{\underline{\noindent\hbox to \textwidth{\hfill\small\textrm{#4}
%\hfill \large\bf\thepage
}}}
\def\leftheadline{\underline{\noindent\parbox{\textwidth}{
%\raggedleft\large\bf\thepage \hfill
\small\textit{#3}\hfill}}}
\def\leftfootline{\small{\textbf{\thepage}
\hfill ИНФОРМАТИКА И ЕЁ ПРИМЕНЕНИЯ\ \ \ том~1\ \ \ выпуск~2\ \ \ 2007}
}%
 \def\rightfootline{\small{ИНФОРМАТИКА И ЕЁ ПРИМЕНЕНИЯ\ \ \ том~1\ \ \ выпуск~2\ \ \ 2007
\hfill \textbf{\thepage}}} \vskip 2em \setcounter{figure}{0}
\setcounter{table}{0} \setcounter{equation}{0} \setcounter{section}{0}
\setcounter{subsection}{0} \setcounter{subsubsection}{0}
\setcounter{footnote}{0} \setcounter{razdel}{0}
%\end{flushleft}
\else {
 \raggedright\noindent\Large\textrm\textbf
 \lineskip .75em
\baselineskip=3.2ex #1 \par} \vskip 1em
%\begin{flushleft}
{\noindent\large\textrm\textbf #2 \par}
\addcontentsline{toc}{subsection}{{\textrm\textbf #3}\protect\newline #1}
\def\rightheadline{\underline{\noindent\hbox to \textwidth{\hfill\small\textrm{#4}
%\hfill \large\bf\thepage
}}}
\def\leftheadline{\underline{\noindent\parbox{\textwidth}{%\raggedleft\large\bf\thepage \hfill
\small\textit{#3}\hfill}}}
\def\leftfootline{\small{\textbf{\thepage}
\hfill ИНФОРМАТИКА И ЕЁ ПРИМЕНЕНИЯ\ \ \ том~1\ \ \ выпуск~2\ \ \ 2007}
}%
 \def\rightfootline{\small{ИНФОРМАТИКА И ЕЁ ПРИМЕНЕНИЯ\ \ \ том~1\ \ \ выпуск~2\ \ \ 2007
\hfill \textbf{\thepage}}} \vskip 2em \setcounter{figure}{0} 
\setcounter{table}{0} \setcounter{equation}{0} \setcounter{section}{0} 
\setcounter{subsection}{0} \setcounter{subsubsection}{0} 
\setcounter{footnote}{0}
%\end{flushleft}
\fi}

\newcommand{\titele}[5]{%
\

%\vspace*{5pt}

\ifodd\therazdel {\raggedright\noindent%\large
\textrm\textbf
 \lineskip .75em
%  \baselineskip=3.2ex 
#1 \par}
\vskip .5em {\noindent\large\textrm\textbf #2 \par} 
\vskip .5em 
 {\noindent\textrm #3 \par} 
\addcontentsline{toc}{subsection}{{\textrm\textbf #1}\protect\newline #2}
\def\rightheadline{\underline{\noindent\hbox to \textwidth{\hfill\small\textrm{#4}
%\hfill \large\bf\thepage
}}}
\def\leftheadline{\underline{\noindent\parbox{\textwidth}{
%\raggedleft\large\bf\thepage \hfill
\small\textrm{#5}\hfill}}}
\def\leftfootline{\small{\textbf{\thepage}
\hfill ИНФОРМАТИКА И ЕЁ ПРИМЕНЕНИЯ\ \ \ том~1\ \ \ выпуск~2\ \ \ 2007}
}%
 \def\rightfootline{\small{ИНФОРМАТИКА И ЕЁ ПРИМЕНЕНИЯ\ \ \ том~1\ \ \ выпуск~2\ \ \ 2007
\hfill \textbf{\thepage}}} \vskip 1em \setcounter{figure}{0} 
\setcounter{table}{0} \setcounter{equation}{0} \setcounter{section}{0} 
\setcounter{subsection}{0} \setcounter{subsubsection}{0} 
\setcounter{footnote}{0} \setcounter{razdel}{0}
%\end{flushleft}
\else {
 \raggedright\noindent%\large
 \textrm\textbf
 \lineskip .75em
%\baselineskip=3.2ex 
#1 \par} \vskip .5em
%\begin{flushleft}
{\noindent\large\textrm\textbf #2 \par} \vskip .5em 
 {\noindent\textrm #3 \par} 
\addcontentsline{toc}{subsection}{{\textrm\textbf #1}\protect\newline #2}
\def\rightheadline{\underline{\noindent\hbox to \textwidth{\hfill\small\textrm{#4}
%\hfill \large\bf\thepage
}}}
\def\leftheadline{\underline{\noindent\parbox{\textwidth}{%\raggedleft\large\bf\thepage \hfill
\small\textrm{#5}\hfill}}}
\def\leftfootline{\small{\textbf{\thepage}
\hfill ИНФОРМАТИКА И ЕЁ ПРИМЕНЕНИЯ\ \ \ том~1\ \ \ выпуск~2\ \ \ 2007}
}%
 \def\rightfootline{\small{ИНФОРМАТИКА И ЕЁ ПРИМЕНЕНИЯ\ \ \ том~1\ \ \ выпуск~2\ \ \ 2007
\hfill \textbf{\thepage}}} \vskip 1em \setcounter{figure}{0} 
\setcounter{table}{0} \setcounter{equation}{0} \setcounter{section}{0} 
\setcounter{subsection}{0} \setcounter{subsubsection}{0} 
\setcounter{footnote}{0}
%\end{flushleft}
\fi}

\def\Abst#1{
\begin{center}\small\nwt
\parbox{150mm}{%\baselineskip=2.5ex
\textbf{Аннотация:}\ \
%\hspace*{\parindent}
#1}
\end{center}}

\def\KW#1{
\begin{center}\small\nwt
\parbox{150mm}{%\baselineskip=2.5ex
\textbf{Ключевые слова:}\ \ #1}
\end{center}}

\def\KWN#1{
%\begin{center}
%\small
%\parbox{150mm}\end{center} 
}


\begin{document}
\Rus

\nwt 
%\ptb

%\renewcommand{\contentsname}{\protect\Large\bf Содержание}

\setcounter{tocdepth}{2}

%\tableofcontents

\renewcommand{\bibname}{\protect\rmfamily Литература}
  \def\Au#1{{\it #1}}

%\newcommand{\No}{№}
  \newcommand{\tg}{\rm  tg}
  \newcommand{\arctg}{\rm  arctg}

\setcounter{page}{1}

\newpage
\addtocounter{razdel}{1}
%\def\razd{РЕГУЛИРУЕМЫЙ ЭЛЕКТРОПРИВОД ДЛЯ ЭЛЕКТРОЭНЕРГЕТИКИ}
%\newpage
%\def\stat{zakh}
\def\tit{СРЕДСТВА ОБЕСПЕЧЕНИЯ ОТКАЗОУСТОЙЧИВОСТИ ПРИЛОЖЕНИЙ}
\def\titkol{Средства обеспечения отказоустойчивости приложений}

\def\aut{В.\,Н.~Захаров$^1$, В.\,А.~Козмидиади$^2$}
\titel{\razd}{\tit}{\aut}{\titkol}


\Abst{Рассмотрены проблемы построения отказоустойчивых серверов, возникающие в связи с недетерминированностью поведения приложений. Предложена формальная модель, описывающая поведение приложения, основными объектами которой являются ресурсы и события. Предложены алгоритмы протоколирования работы приложения на резервном узле кластера, а также восстановления и продолжения его работы при отказе основного узла. При этом для клиентов сбой остается незаметным, за исключением некоторого увеличения времени обслуживания.}

\KW{сервер приложений $\bullet$ прозрачная отказоустойчивость $\diamond$
 процесс $\diamond$ ресурс $\diamond$ событие $\diamond$ контрольная точка
$\bullet$ детерминированность}

\vskip 12pt plus 6pt minus 3pt

\begin{multicols}{2}

\section*{ВВЕДЕНИЕ}

Средства вычислительной техники стали использоваться в областях,
требующих безотказной работы систем в течение многих лет (24 часа
в сутки, 365 дней в году).

\label{st\stat}

\footnotetext{$^1$ФГУП Центральный институт авиационного моторостроения
им. П.И. Баранова, Москва, Россия}
\footnotetext{$^2$ФГУП Центральный институт авиационного моторостроения
им. П.И. Баранова, Москва, Россия}

К таким областям относятся, например, центры хранения и обработки данных  в сетях (системы резервирования билетов, биллинговые,  банковские и т.д.), массированные распределенные вычисления (GRID-вычисления) и другие.

\thispagestyle{headings}

Обычно в подобных системах применяются частные решения, ориентированные в основном на обеспечение надежного хранения данных (например, файловые серверы, использующие для хранения RAID-контроллеры) и корректного их состояния при отказах (серверы баз данных с транзакционным выполнением запросов). Однако большинство приложений не гарантируют, что не произойдет потери части данных при отказе системы. Обычно предполагается, что клиентские средства должны повторять запросы после восстановления серверов, для того, чтобы данные не были потеряны, или что можно сделать возврат по времени на некоторое время назад и повторить работу с этого места. Однако далеко не все клиентские средства и условия применения приложений допускают это.

Отказоустойчивые системы для критически важных приложений, корректно решающие проблемы восстановления после сбоев,   предлагаемые ведущими производителями, как правило, дороги. Кроме того, они включают специфические серверные и клиентские приложения, не совместимые со стандартными приложениями, не обеспечивающими отказоустойчивость. Примером такого подхода к решению проблемы отказоустойчивости  хранения данных являются системы NetApp FAS компании Network Appliance, работающие на базе специализированной операционной системы Data ONTAP [1].

Построение отказоустойчивых систем, использующих серверы со стандартными приложениями, в свете вышесказанного, является актуальной проблемой, вызывающей значительный интерес. Рассмотрение методов достижения прозрачной отказоустойчивости таких систем и является предметом статьи.
\begin{figure*} %fig1
\vspace*{1pt}
\begin{center}
\mbox{%
\epsfxsize=1.6in
\epsfxsize=100mm
\epsfbox{BbR-1.eps}
}
\end{center}
\vspace*{-9pt}
\Caption{Базовый вариант трубы с разными выходными устройствами
(цилиндрическое, расширяющееся и сужающееся сопло)
\label{f1bab}}
\vspace*{-3pt}
\end{figure*}


\section{ОСНОВНЫЕ ПОНЯТИЯ И ПОДХОДЫ}

Под сервером в данной работе понимается вычислительный центр
(отдельный компьютер или кластер) в сети, предоставляющий клиентам
(пользователям, клиентским компьютерам) определенные услуги, разделяя
между ними свои ресурсы. Подобные серверы названы серверами приложений.
Широко распространенным примером сервера такого типа является файловый сервер, обеспечивающий удаленный коллективный доступ к файловой системе. Часто используются вычислительные серверы, предоставляющие клиентам возможность выполнять на них свои программы (например, в центрах коллективного пользования).


Обычно приложение представляет собой программу или группу программ, работающих в операционной среде, создаваемой операционной системой (в другой терминологии - один или несколько взаимодействующих процессов или потоков (threads)), которые реализуют функциональность сервера. Для построения отказоустойчивых серверов приложений широко используется кластерная технология. Следуя [2], кластером, названа разновидность параллельной или распределенной системы, которая:
\begin{itemize}
\item состоит из нескольких компьютеров (узлов кластера), связанных как минимум необходимыми коммуникационными каналами;
\item используется как единый, унифицированный компьютерный ресурс.
\end{itemize}

Прозрачная отказоустойчивость (Transparent Fault Tolerance, TFT) сервера приложений - это такое его поведение при возникновении аппаратных или программных отказов либо отказов в сети, при котором:
\begin{itemize}
\item отказ не вызывает потери или искажения данных, находящихся в базе данных сервера;
\item сервер продолжает нормально функционировать, несмотря на имевшие место отказы.
\end{itemize}

Клиенты сервера "не замечают" произошедших отказов. Единственным\footnote{допустимым
отклонением сервера от нормального поведения с точки зрения клиента является
некоторое увеличение времени обслуживания} (на несколько секунд или десятков секунд).

Обычно приложения, работающие на серверах приложений, не ориентированы на прозрачную отказоустойчивость. Они "заботятся" лишь о собственной целостности (например, состояния файловой системы или базы данных). Восстановление работоспособности сервера приводит к разрыву соединений с клиентами и потере их запросов. Это замечают клиенты - запросы следует повторять, на что клиентские приложения далеко не всегда рассчитаны. В данной работе предполагается, что приложения (прикладные программные средства), выполняемые на сервере, являются стандартными, то есть не имеют специальных средств, обеспечивающих отказоустойчивость.
\begin{figure*}[b] %fig1
\vspace*{1pt}
\begin{center}
\mbox{%
\epsfxsize=1.6in
\epsfxsize=100mm
\epsfbox{BbR-1.eps}
}
\end{center}
\vspace*{-9pt}
\Caption{Базовый вариант трубы с разными выходными устройствами
(цилиндрическое, расширяющееся и сужающееся сопло)
\label{f1bab}}
\vspace*{-3pt}
\end{figure*}

Серьезные исследования в области обеспечения отказоустойчивости серверов были развернуты после создания вычислительных серверов, предназначенных для решения задач, требующих больших вычислительных ресурсов. Решение этих задач выполняется на суперкомпьютерах, обеспечивающих массово-параллельные вычисления и представляющих собой кластеры из сотен и тысяч узлов (процессоров). Однако даже на этих "монстрах" решение может требовать десятков или сотен часов, и одиночный сбой, если не предприняты специальные меры, может привести к необходимости начинать работу сначала. Обычно решение вычислительной задачи в таких случаях осуществляется в модели относительно редко взаимодействующих между собой процессов, выполняемых на разных узлах кластера. Эти взаимодействия нужны для координации работы процессов, в частности, для обмена данными и промежуточными результатами. Взаимодействия опираются на специальный протокол, называемый MPI (Message-Passing Interface) и представляющий собой стандарт "de facto" [3].

Для преодоления последствий сбоя достаточно давно была разработана и широко применяется технология, опирающаяся на механизм контрольных точек (checkpoints) [4-6]. По этой технологии система должна иметь стабильную память, которая не меняется при отказах. Соответствующие программные средства периодически сохраняют информацию о состоянии процессов приложения в стабильной памяти. Все процессы также имеют доступ к устройству стабильной памяти.  В случае отказа или сбоя, записанная в стабильную память информация используется для повторения вычисления с момента, когда была записана эта информация, то есть выполняется откат назад по времени. Данные, сохранение которых позволяет выполнить откат, называются контрольной точкой. В качестве устройства стабильной памяти может использоваться дисковый том, энергонезависимая оперативная память, память другого узла или узлов кластера. В последнем случае узел, которому требуется сохранить информацию, пересылает ее через быстрый канал связи на другой узел. Стабильная память после отказа одного из узлов должна быть доступной узлу, на котором делается повтор.

Однако решение, опирающееся только на контрольные точки, не является прозрачным, поскольку не скрывает от клиентов факт отказа системы и требует от них выполнения определенных действий. Так как при работе процессы обмениваются сообщениями, возможны два варианта решения проблемы. Первый - все процессы выполняют записи контрольных точек одновременно, что затруднительно. Второй вариант, при несоблюдении синхронности, - возврат в каждом процессе к такому скоординированному набору контрольных точек, при котором невозможна противоречивая ситуация. Такая ситуация возникает, когда один процесс вернулся к контрольной точке, после которой он должен получить сообщение от другого процесса, а этот другой процесс вернулся к точке, которая следует за выдачей этого сообщения. Однако при повторе ожидаемое первым процессом сообщение не поступит. В этом случае  возможен эффект домино, в результате процессы оказываются отброшены как угодно далеко назад.

В этом состоит первая проблема, которую необходимо преодолеть.

Если нужно, чтобы последствия отказа узла не были видны клиенту,  это означает:
\begin{itemize}
\item клиент не должен терять и потом восстанавливать соединения с сервером;
\item клиент не должен повторять свои запросы;
\item клиент не должен повторно получать сообщения, которые он уже получил.
\end{itemize}

Вторая проблема, которую надо решать, связана с недетерминированностью поведения сервера приложений. Приведем пример.  Пусть имеется система продажи билетов на самолеты. Два клиента одновременно обратились к системе с запросом билета на один и тот же рейс. Клиентам безразлично, какие места им зарезервирует система. Система выполняет запросы клиентов параллельно, поэтому в какой-то момент между процессами, обрабатывающими эти запросы, может возникнуть конкуренция за ресурс - в данном случае, скажем, рейс. Один из процессов захватывает ресурс первым, резервирует место и освобождает ресурс. Потом второй процесс проделывает то же самое.

Порядок, в котором в этом примере процессы захватили ресурс, зависит от многих факторов и, в конечном счете, случаен. Однако  это не мешает правильному функционированию системы, поскольку клиентам важно одно - получить билеты, причем на разные места. Однако отсутствие детерминизма в поведении приложения приводит к тому, что при повторном выполнении могут быть получены другие результаты: например, клиенту уже сообщено, что ему зарезервировано место №5, а при повторе может получиться, что зарезервировано место №6. Система должна устранить это несоответствие и сделать его невидимым для клиента.
\begin{figure*} %fig1
\vspace*{1pt}
\begin{center}
\mbox{%
\epsfxsize=1.6in
\epsfxsize=100mm
\epsfbox{BbR-1.eps}
}
\end{center}
\vspace*{-9pt}
\Caption{Базовый вариант трубы с разными выходными устройствами
(цилиндрическое, расширяющееся и сужающееся сопло)
\label{f1bab}}
\vspace*{-3pt}
\end{figure*}

Недетерминированность поведения системы это следствие, по крайней мере, двух обстоятельств. Во-первых, это присущая системам с разделением времени неопределенность в порядке выполнения процессов. Во-вторых, это конкуренция процессов за общие ресурсы. Перечислим некоторые причины недетерминированного поведения приложений:
\begin{itemize}
\item синхронизация процессов с помощью семафоров или атомарных операций над операндами в общей памяти процессов;
\item зависимость от порядка получения клиентских запросов;
\item время, затраченное процессом на обработку полученного запроса;
\item генераторы случайных чисел;
\item системное управление процессами и потоками;
\item локальные таймеры;
\item доступ к реальному времени.
\end{itemize}

По различным  причинам время, которое тратится на выполнение вычислительной задачи с одними и теми же исходными данными, не является константой, то есть повторное выполнение может дать другое время. Процессы используют общие ресурсы, обращение к которым требует организации очередности выполнения (сериализации) - первым пришел, первым захватил. И, наконец,  результат работы процесса может зависеть от состояния ресурса, а это состояние может изменить другой процесс, ранее захвативший ресурс. Все это создает значительные трудности при попытках воспроизведения поведения процессов с сохраненной контрольной точки.

Прозрачная отказоустойчивость серверов приложений обычно осуществляется переносом приложения на другой узел кластера, идентичный первому по конфигурации аппаратных средств и операционной среды. Это делается методом, называемым snapshot/restore. На основном узле (оригинале)  периодически фиксируется состояние приложения на этом узле кластера (так называемый снимок или snapshot). После отказа оригинала на резервном узле (копии) делается восстановление (restore), то есть восстанавливается последнее зафиксированное состояние приложения. Операционная среда при этом приводится в состояние, которое соответствует моменту изготовления снимка. После этого узел-копия продолжает работу с зафиксированного места. Сравнение метода  snapshot/restore с другими подходами приведено в [7].

Ниже рассматриваются информационные  технологии, позволяющие решить ряд принципиальных вопросов, связанных с реализацией прозрачной отказоустойчивости серверов приложений. Ими являются:
\begin{itemize}
\item виртуализация операционной среды, в которой работает серверное приложение;
\item отказоустойчивая реализация протокола TCP;
\item создание контрольных точек состояния приложения и файловой системы, которые делаются внешним по отношению к приложению образом;
\item восстановление серверного приложения на основании контрольной точки.
\end{itemize}
\begin{figure*} %fig1
\vspace*{1pt}
\begin{center}
\mbox{%
\epsfxsize=1.6in
\epsfxsize=100mm
\epsfbox{BbR-1.eps}
}
\end{center}
\vspace*{-9pt}
\Caption{Базовый вариант трубы с разными выходными устройствами
(цилиндрическое, расширяющееся и сужающееся сопло)
\label{f1bab}}
\vspace*{-3pt}
\end{figure*}

\section{МОДЕЛЬ ОПИСАНИЯ ПОВЕДЕНИЯ ПРИЛОЖЕНИЯ}

Предлагаемый подход опирается на построение модели вычислений, связанной с использованием понятия времени в многопроцессных приложениях. Впервые подобные проблемы были изучены в классической работе Л. Лампорта [8].

Многопроцессными приложения называются потому, что в них параллельно работают несколько процессов. Процесс ведет себя детерминированно, пока в предписанном кодом порядке выполняет процессорные инструкции. Конечно, его работа может быть прервана практически в любой момент и процессор передан другому процессу или ядру. Поэтому абсолютное время, которое затрачивает процесс на выполнение определенной работы, не  константа, а случайная  величина. То же  относится к относительному времени, то есть времени, которое процесс занимал процессор,  поскольку одни и те же обращения к операционной среде могут вызвать работы разной длительности, а значит потребовать разное время на свое выполнение.

Кэшированность инструкций и данных, а также длина хэш-списков влияют на действительное время пребывания в операционной среде. Утрачивает смысл понятие одновременность действий, поскольку  нельзя установить, выполнили ли два разных процесса какие-либо действия одновременно или одно из них предшествовало другому. Таким образом, с процессом можно связать только его локальное время, которое линейно упорядочивает события,  происходившие в этом процессе.  Глобальное время, линейно упорядочивающее действия во всех процессах, отсутствует. Расстояние (в этом качестве используется время) между действиями оказывается случайной величиной.

Эти соображения важны, поскольку процессы в интересующих нас приложениях взаимодействуют и используют общие ресурсы. Для взаимодействия они используют средства синхронизации, предоставляемые операционной средой - например, наборы семафоров SVR4 (System V Release 4), POSIX-семафоры, бинарные семафоры и другие примитивы взаимного исключения (POSIX- mutual exclusion locks) и т.д. Подобные средства операционной среды, которые позволяют процессам синхронизировать свою деятельность друг с другом или сериализовать обращения к совместно используемым объектам,  будут ниже  называться ресурсами.

С каждым ресурсом связано свое локальное время, линейно упорядочивающее события в жизни ресурса. Например, в случае двоичных семафоров это создание семафора, а также его захват и освобождение процессом. Заметим, что событие - это не намерение процесса (например, захватить бинарный семафор), а сам факт захвата семафора процессом (т.е. успешное выполнение намерения). От изъявления намерения до его осуществления может многое произойти. Например, семафор, который хочет захватить рассматриваемый процесс, принадлежал другому процессу, потом тот процесс его освободил, но семафор был сначала передан операционной средой третьему процессу, который также на него претендовал, и т.д. Поведение рассматриваемого процесса в это время нас не интересует - он ресурсом еще не овладел, а только его захват определяет его дальнейшее поведение. По причинам,  изложенным выше, расстояние между двумя событиями - случайная величина. Однако, события замечательны тем, что они одновременно присутствуют и в локальном времени процесса, и в локальном времени ресурса. Поэтому все, что произошло в истории процесса или/и ресурса до этого события, предшествует ему. Далее  будет считаться, что истории и ресурсов и процессов состоят только из событий, причем между двумя последовательными событиями в жизни процесса последний ведет себя детерминированно.

Это означает, что на  поведении процесса сказывается только его предыдущая история, то есть состояние ресурсов, с которыми он взаимодействовал. Это свойство процессов ниже будет называться локальной детерминированностью. Этим свойством не обладают ресурсы, поскольку - следующее событие в истории ресурса не определяется однозначно по его предыдущей истории. Утверждение, касающееся детерминированного поведения процессов, неявно опирается на предположение,  что учтены все ресурсы, которые могут привести к  недетерминированности процессов.

Таким образом, описанное нами очень неформально время в многопроцессном комплексе представляет собой отношение частичного порядка, введенное на множестве событий. Зная полное состояние комплекса в некоторый момент времени,  нельзя однозначно определить, какое событие в истории ресурса наступит следующим. Можно говорить только о вероятности наступления того или иного события. Недетерминированность поведения есть следствие двух обстоятельств. Во-первых, это неопределенность времени, которое тратит процесс на переход от одного события к другому. Во-вторых, конкуренция процессов за общие ресурсы.

Выполнение приложения, на множестве событий которого введена частичная упорядоченность, можно описать направленным ациклическим графом выполнения. Вершинами этого графа являются события, с каждым  из которых связаны две входящие в него дуги. Одна дуга начинается в событии, которое непосредственно предшествует данному событию в истории процесса, другая - в истории ресурса.

Построение средств обеспечения прозрачной отказоустойчивости приложений опирается на следующее утверждение: для восстановления работы приложения после отказа достаточно располагать:
\begin{itemize}
\item контрольной точкой, которая отражает на некоторый момент времени состояния процессов и других ресурсов, образующих приложение;
\item графом выполнения приложения, который описывает работу приложения, начинающуюся с контрольной точки и заканчивающуюся отказом. Данные, которые нужны для построения графа выполнения, далее называются протоколом.
\end{itemize}
\begin{figure*} %fig1
\vspace*{1pt}
\begin{center}
\mbox{%
\epsfxsize=1.6in
\epsfxsize=100mm
\epsfbox{BbR-1.eps}
}
\end{center}
\vspace*{-9pt}
\Caption{Базовый вариант трубы с разными выходными устройствами
(цилиндрическое, расширяющееся и сужающееся сопло)
\label{f1bab}}
\vspace*{-3pt}
\end{figure*}

Вся эта информация должна находиться в стабильной памяти, не разрушающейся при отказе.

Ниже неформально описан алгоритм восстановления работы приложения после отказа, который опирается на наличие контрольной точки и графа выполнения. Будем считать, что в распоряжении имеются средства, позволяющие остановить процесс в тот момент, когда он намерен совершить некоторую операцию над ресурсом. Заметим, что событие в графе выполнения соответствует не изъявлению намерения, а его удовлетворению, то есть завершению выполнения операции.

Предварительно сделаем следующее:
\begin{itemize}
\item используя контрольную точку, приведем приложение в состояние, соответствующее этой контрольной точке;
\item в графе выполнения пометим все вершины (события) как "не наступившие". У некоторых вершин графа отсутствуют им непосредственно предшествующие; соответствующие события наступили сразу же после создания контрольной точки. Для каждой такой вершины включим в граф дополнительную вершину, ей предшествующую в истории процесса, и отметим эту дополнительную вершину как "наступившую";
\item разрешим процессам приложения выполняться.
\end{itemize}

Пусть некоторый процесс проявляет намерение выполнить операцию над каким-либо ресурсом. Отыщем для этого процесса в его истории последнее наступившее событие. Следующее в его истории событие - это то, которое соответствует требуемой операции. Посмотрим, наступило ли событие в истории ресурса, которое ему предшествует. Если нет, переведем процесс в состояния ожидания, отметив в предшествующем событии, что данный процесс ожидает его наступления. Если да, разрешим процессу выполняться, то есть выполнить операцию над ресурсом.

Пусть некоторый процесс объявляет, что он выполнил операцию над каким-либо ресурсом (это соответствует моменту протоколирования при оригинальном выполнении). Отыщем для этого процесса в его истории последнее наступившее событие и перейдем к следующему событию в его истории. Это опять то событие, которое мы рассматриваем. Отметим его как "наступившее". Если наступления этого события ожидал какой-нибудь процесс, выведем этот процесс из состояния ожидания. Наконец, разрешим процессу, выполнившему операцию, продолжаться дальше.

Когда выясняется, что наступили все события графа выполнения, повторное выполнение считается законченным.

Естественным следствием из сказанного является следующее утверждение: для того, чтобы размер протокола не рос неограниченно, нужно периодически создавать контрольные точки, очищая при этом протокол.

\section{ФОРМАЛЬНОЕ ОПИСАНИЕ МОДЕЛИ ПОВЕДЕНИЯ МНОГОПРОЦЕССНОГО ПРИЛОЖЕНИЯ}
\begin{figure*} %fig1
\vspace*{1pt}
\begin{center}
\mbox{%
\epsfxsize=1.6in
\epsfxsize=100mm
\epsfbox{BbR-1.eps}
}
\end{center}
\vspace*{-9pt}
\Caption{Базовый вариант трубы с разными выходными устройствами
(цилиндрическое, расширяющееся и сужающееся сопло)
\label{f1bab}}
\vspace*{-3pt}
\end{figure*}

Опишем формально поведение приложения, неформальное описание которого было приведено выше. Рассматриваются два типа объектов:
\begin{itemize}
\item ресурсы (r), например, наборы семафоров (POSIX- или SVR4-семафоры), бинарные семафоры (POSIX-mutex's), таймер реального времени, сокеты (sockets), то есть двусторонние виртуальные соединения с внешним миром;
\item процессы (p), например, процессы или потоки (threads) пользователя.
\end{itemize}

\end{multicols}

\label{end\stat}

%\def\stat{batr}

\def\tit{НОВЫЙ МЕТОД ВЕРОЯТНОСТНО-СТАТИСТИЧЕСКОГО\newline
АНАЛИЗА ИНФОРМАЦИОННЫХ ПОТОКОВ
В~ТЕЛЕКОММУНИКАЦИОННЫХ СЕТЯХ$^*$}
\def\titkol{Новый метод вероятностно-статистического
анализа информационных потоков
в~телекоммуникационных сетях}
\def\autkol{Д.\,А.~Батракова, В.\,Ю.~Королев, С.\,Я.~Шоргин}
\def\aut{Д.\,А.~Батракова$^1$, В.\,Ю.~Королев$^2$, С.\,Я.~Шоргин$^3$}

\titel{\tit}{\aut}{\autkol}{\titkol}

{\renewcommand{\thefootnote}{\fnsymbol{footnote}}\footnotetext[1]{Работа 
выполнена при поддержке РФФИ, проекты №№\,04-01-00671, 05-07-90103.} 
\renewcommand{\thefootnote}{\arabic{footnote}}}
 \footnotetext[1]{ИПИ РАН, 
daria.batrakova@gmail.com} \footnotetext[2]{Факультет вычислительной математики 
и кибернетики МГУ им.~М.\,В.~Ломоносова, ИПИ РАН, vkorolev@comtv.ru} 
\footnotetext[3]{ИПИ РАН, sshorgin@ipiran.ru}



\Abst{В данной работе предлагается метод исследования стохастической структуры
хаотических информационных потоков в сложных телекоммуникационных
сетях. Предлагаемый метод основан на стохастической модели
телекоммуникационной сети, в рамках которой она представляется в виде
суперпозиции некоторых простых последовательно-параллельных структур.
Эта модель естественно порождает смеси гамма-распределений для времени
выполнения (обработки) запроса сетью. Параметры получаемой смеси
гамма-распределений характеризуют стохастическую структуру
информационных потоков в сети. Для решения задачи статистического
оценивания параметров смесей экспоненциальных и гамма-распределений
(задачи разделения смесей) используется ЕМ-алгоритм. Чтобы проследить
изменение стохастической структуры информационных потоков во времени,
ЕМ-алгоритм применяется в режиме скользящего окна. Описывается
программный инструментарий для применения полученного решения к
реальным статистическим данным. Приводится интерпретация результатов.}

\KW{телекоммуникационные сети; информационные потоки;
разделение смесей  распределений;
метод скользящего окна;  программа для разделения смесей}

\vskip 24pt plus 9pt minus 6pt

\thispagestyle{headings}

\begin{multicols}{2}


\label{st\stat}

\section{Введение}

Развитие телекоммуникационных сетей, их усложнение поставило перед
инженерами важную прикладную задачу исследования характеристик
информационных потоков, возникающих в этих сетях. Здесь под
информационным потоком мы будем понимать упорядоченное движение
любого вида информации по сети.

Если на заре эры телекоммуникаций, в эпоху первых телефонных линий и
телеграфа эта проблема не была столь насущной, то со временем, при
постепенном охвате мирового пространства сетями возникла необходимость в
построении и исследовании адекватных моделей сетей и процессов,
происходящих в них.

\thispagestyle{headings}


Современные сети~--- это \textit{конвергентные} сети, т.е.\ совокупность крайне
разнородных как по топологии, так и по физической архитектуре сетей, которые
предлагают конечному пользователю самые разнообразные сервисы. Это~--- огромное
виртуальное и физическое пространство, состоящее из миллионов процессоров,
операционных платформ, линий передачи данных и стыковочных узлов.
%
Существует множество классификаций телекоммуникационных сетей по различным
признакам:
\begin{itemize}
\item масштабу (локальные сети~--- LAN, масштаба города~---
MAN, широкого масштаба~--- WAN);
\item топологии, или логической организации (<<звезда>>,
<<кольцо>>, <<шина>>);
\item физической организации (оптические сети, радио);
\item предлагаемым услугам (сотовые сети, для доступа в
Интернет);
\item назначению (военные, гражданские) и~др.
\end{itemize}


Конвергентная сеть входит во все эти классы, причем, как правило,
обладает всеми этими признаками. Поэтому построение модели для ее анализа
является и очень важной, и очень сложной задачей.

Существуют достаточно многочисленные математические методы, ориентированные на
моделирование и анализ телекоммуникационных сетей. В~большинстве своем они
основываются на теории массового обслуживания, то есть разделе теории
вероятностей, посвященном описанию функционирования сложных систем обслуживания
(в том чис\-ле телекоммуникационных сетей и систем) с помощью стохастических
процессов особого вида и анализу таких процессов. Указанная теория является
весьма развитой и широко применяется на практике. Тем не менее, ее применимость
ограничена~--- во-первых, все возрастающей сложностью структур и дисциплин
обслуживания в рас\-смат\-ри\-ва\-емых реальных сетях. Эта сложность во многих
случаях принципиально не может найти адекватного отображения в моделях
массового обслуживания, даже несмотря на постоянно растущую сложность самих
этих моделей. В результате даже модели, допускающие точный математический
анализ, дают возможность расчета всего лишь приближенных значений характеристик
реальных сетей, ибо предположения, принимаемые при построении моделей, во
многих случаях не соответствуют практике. Во-вторых, для описания
телекоммуникационной сети в виде сети массового обслуживания исследователь
должен располагать детальным описанием структуры сети, что далеко не всегда
имеет мес\-то на практике. В-третьих, разработано крайне мало моделей массового
обслуживания, в которых используется в качестве входной информация о
наблюдаемых (статистических) показателях функционирования сети; в то же время,
такая информация очень часто доступна исследователю, и ее использование при
анализе сети весьма целесообразно.

В данной работе предлагается в определенной степени альтернативный подход к
моделированию сложных телекоммуникационных сетей. Строится и исследуется
вероятностная модель сложной телекоммуникационной сети как суперпозиции
достаточно простых структур. При этом практически никакая априорная информация
о структуре исследуемой сети не используется~--- наоборот, в результате
исследования модели исследователь получает приближенное представление об этой
структуре. Характеристики типовых простых структур, составляющих в совокупности
модель сети, и сети в целом при этом подходе описываются
гам\-ма-рас\-пре\-де\-ле\-ни\-я\-ми. Ставится задача оценки параметров модели
на основе статистических данных о функционировании сети, а также предлагается
математическое решение этой задачи. В статье описан также созданный на основе
разработанных математических методов программный инструментарий и приведены
результаты расчетов для реального трафика. {\looseness=-1

}

\section{Математическая модель и~постановка задачи}

\subsection{Логическая модель сети}
 %1.1

Рассмотрим абстрактную \textit{конвергентную телекоммуникационную
сеть}. Это может быть как крупномасштабная транспортная сеть (WAN), сеть
отдельного оператора масштаба города (MAN) с различными сервисами, так и
локальная сеть (LAN).

Любой из этих случаев можно описать как ($p,\,q$)-\textit{сеть}.

\medskip
\textbf{Определение 1.} В теории графов и сетей под ($p,\,q)$-сетью понимается
набор вида $S =$\linebreak $=(G,\,V^\prime ,\,V^{\prime\prime})$, где $G$~---
граф, а $V^\prime$ и $V^{\prime\prime}$~--- выборки из множества $V(G)$ (вершин
графа) длины~$p$ и $q$ соответственно. При этом выборка $V^\prime$
($V^{\prime\prime}$) считается \textit{входной} (\textit{выходной}) выборкой, а
ее $i$-я вершина называется $i$-\textit{м} \textit{входным} (\textit{выходным})
\textit{полюсом} или, иначе, $i$-\textit{м} \textit{входом} (\textit{выходом})
сети~$S$. Вершины, не участвующие во входной и выходной выборках сети,
считаются ее внутренними вершинами~\cite{1bat}.

Сеть $S$ (рис.~\ref{f1bat}) имеет $p$ точек входа~--- точек соединения
с внешней средой (это могут быть точки стыковки разнородных сетей, сетей
различных операторов, физические подключения к интерфейсам
маршрутизаторов и~т.п.). Под \textit{внешней средой} будем понимать другие
сети, которые передают данные в сеть~$S$. Отдельные <<единицы>> данных
(кадры, сообщения, датаграммы, пакеты) поступают на входы сети~$S$,
обрабатываются и подаются на каждый из $q$ выходов, которые могут быть
соединены как с конечными пользователями, так и с другими сетями.
\begin{figure*} %fig1
\vspace*{1pt}
\begin{center}
\mbox{%
\epsfxsize=139.7mm \epsfbox{bat-1.eps}
%\epsfxsize=139.698mm
%\epsfbox{bek-3.eps}
}
\end{center}
\vspace*{-9pt} \Caption{Абстрактная телекоммуникационная сеть \label{f1bat}}
\end{figure*}

Следует отметить, что структура сложных телекоммуникационных сетей обладает
свойством некоторого самоподобия, т.е.\ на каком бы уровне сетевой архитектуры
мы ни рассматривали поведение информационных потоков, мы можем выделить
некоторые базовые структуры, субпотоки, суперпозицией которых мы можем получить
модель конкретной сети, какой бы уровень <<детализации>> сегментов сети мы ни
взяли. Так, например, физические подключения к интерфейсам оптического
кросс-коннекта в этом смысле подобны <<виртуальным>> подключениям к портам TCP
на сервере приложений.

Итак, независимо от уровня сетевой архитектуры мы можем
рассматривать некоторую величину, характеризующую количество каких-либо
ресурсов сети~$S$, занимаемых в процессе передачи и обработки данных.  Это
могут быть ресурсы, относящиеся как к <<объему>> (памяти сетевого
устройства, количеству занятых линий, размеру пакета), так и ко <<времени>>
(времени обслуживания заявки, времени простоя). Эта величина случайна, т.к.\
мы не можем абсолютно точно сказать для сложной телекоммуникационной
сети, какое сообщение на какой из входов поступит и какого размера оно будет.
Таким образом, случайный характер данной величины определяется
случайностью поведения внешней среды.

Пусть $R$~--- это описанная выше случайная величина, $R>0$. Далее, не
ограничивая общности, будем подразумевать под ней время, необходимое для
какой-либо операции сети (обработки <<единицы>> данных), предполагая, что
время обработки прямо зависит от объема сообщения.

\subsection{Вероятностная модель сети} %1.2.

Даже не зная реальной топологии сети, мы можем описать
функционирование некоторых ее участков как процесс выполнения операций
(задач сети) в последовательном  порядке (например, если доступен только
один канал данных) или как процесс одновременного выполнения субопераций
(когда доступно более одного пути выполнения). Это значит, что мы можем
представить функционирование сложной телекоммуникационной сети как
\textit{суперпозицию} таких <<последовательных>> и <<параллельных>>
блоков.

Для построения вероятностной модели распределения~$R$ используется
комбинация асимптотического подхода, основанного на предельных теоремах
теории вероятностей, и принципа максимальной неопределенности (энтропии).

Рассмотрим следующую модель. Предположим, что мы можем разделить
сеть~$S$ на несколько сегментов $S_i$. Пусть $T$~--- случайная величина,
время выполнения операции в отдельно взятом блоке $S_i$ (сегменте сети).

Если операции выполняются \textit{параллельно}, то время, необходимое
для их выполнения~--- это максимальное время, затрачиваемое на какую-либо
субоперацию:
$$
T = \underset{i}{\max}\, T_i\,,
$$
где $T_i$~--- случайные величины для со\-от\-вет\-ст\-ву\-ющих субопераций.
Модель такого выполнения пред\-став\-ле\-на на рис.~\ref{f2bat}.

\begin{figure*} %fig2
\vspace*{1pt}
\begin{center}
\mbox{%
\epsfxsize=117.271mm
\epsfbox{bat-2.eps}
}
\end{center}
\vspace*{-9pt}
\Caption{Параллельное выполнение
\label{f2bat}}
\end{figure*}

Известно, что предельное распределение экстремальных значений для
выборок ~--- это экспоненциальное распределение с плотностью~\cite{2bat}
$$
f(x) =
\begin{cases}
\lambda e^{-\lambda x}\,, & x>0\,,\\
0\,, & x\leq 0\,,
\end{cases}
$$
где $\lambda >0$~--- параметр масштаба.

 Учитывая также энтропийный подход, естественно будет считать
распределение $T$ экспоненциальным, т.к.\ экспоненциальное распределение
обладает наибольшей энтропией среди всех распределений с $x>0$.

Если же операции сети выполняются \textit{последовательно}, то величина
$T$~--- это сумма времен $T_i$, необходимых для выполнения каждой
субоперации:
$$
T = \sum\limits_i T_i\,,
$$
где $T_i$~--- случайные величины для со\-от\-вет\-ст\-ву\-ющих субопераций.
%
Такая модель представлена на рис.~\ref{f3bat}.

\begin{figure*} %fig3
\vspace*{1pt}
\begin{center}
\mbox{%
\epsfxsize=139.592mm
\epsfbox{bat-3.eps}
}
\end{center}
\vspace*{-9pt}
\Caption{Последовательное  выполнение
\label{f3bat}}
\end{figure*}

Это значит, что распределение общей длительности $T$ выполнения
блока~--- это свертка распределений <<элементарных>> времен $T_i$
(экспоненциально распределенных).

Известно, что результатом свертки экспоненциальных распределений
является гамма-распределение, определяемое плотностью
$$
\g_{\lambda , \alpha} (x) =
\begin{cases}
\fr{\lambda_0^{\alpha_0}}{\Gamma (\alpha_0 )}\,x^{\alpha_0-1}
e^{\lambda_0 x}\,, & x>0\,,\\
0,\, & x\leq 0\,,
\end{cases}
$$
где $\alpha >0$~--- параметр формы,  $\lambda >0$  параметр масштаба, а
$\Gamma (z)$~--- гамма-функция Эйлера:
$$
\Gamma (z) = \int\limits_0^\infty x^{z-1} e^{-x}\,dx\,.
$$

\begin{figure*} %fig4
\vspace*{1pt}
\begin{center}
\mbox{%
\epsfxsize=120.831mm
\epsfbox{bat-4.eps}
}
\end{center}
\vspace*{-9pt}
\Caption{Модель пути  обработки сообщения сетью~$S$
\label{f4bat}}
\end{figure*}

Известно~\cite{2bat}, что класс гамма-распределений замкнут над операцией
свертки, поэтому ре\-зуль\-ти\-ру\-ющее распределение случайной величины~$R$
будет также гамма-распределением
$$
\g_{\lambda , \alpha} (x) =
\begin{cases}
\fr{\lambda^{\alpha}}{\Gamma (\alpha )}\,x^{\alpha -1} e^{-\lambda x}\,, &
x>0\,,\\
0,\, & x\leq 0\,.
\end{cases}
$$

В силу случайного характера ввода данных в сеть~$S$ из внешней среды маршрут
передачи данных становится случайным, что представлено на рис.~\ref{f4bat}. Это
означает, что параметры ре\-зуль\-ти\-ру\-юще\-го распределения~$R$ тоже
случайны. Отсюда имеем следующую модель \textit{смеси
гам\-ма-рас\-пре\-де\-ле\-ний}, определяемой плотностью

\begin{equation} %1
p(x) = \iint \g_{\lambda , \alpha}(x)\,dH (\lambda ,\,\alpha )\,,
\end{equation}
где $H(\lambda , \alpha )$~--- смешивающая функция, функция распределения
входных параметров.

Поясним понятие \textit{смеси распределений}.

\medskip
\textbf{Определение~2.} Пусть имеется двух\-па\-ра\-мет\-ри\-че\-ское
семейство $n$-мерных плотностей  распределения
\begin{equation}
F = \{ f_\omega (x;\, \theta (\omega ))\}\,,
\end{equation}
где одномерный (целочисленный или непрерывный) параметр $\omega$ в
качестве нижнего индекса функции $f$ определяет специфику общего вида
каж\-до\-го компонента~--- распределения смеси, а в качестве аргумента при
многомерном, вообще говоря, параметре $\theta$ определяет зависимость
значений хотя бы части компонентов этого параметра от того, в каком именно
составляющем распределении $f_\omega$ он присутствует. Кроме того, пусть
$P = \{P(\omega )\}$~--- \textit{семейство смешивающих функций}
распределения.

Функция плотности распределения
\begin{equation}
f(x) = \int f_\omega (x;\,\theta(\omega ))\,dP (\omega )
\end{equation}
называется $P$-\textit{смесью} (или просто \textit{смесью})
\textit{распределений} семейства~$F$,  интеграл в~(3) понимается в смысле
Лебега--Стильтьеса~\cite{3bat}.

\medskip
\textbf{Определение 3.} Семейство смесей~(3) называется
\textit{идентифицируемым} (\textit{различимым}), если из равенства
$$
\int f_\omega (x;\,\theta(\omega ))\,dP (\omega ) =\int f_\omega
(x,\,\theta(\omega )) dP^*(\omega )
$$
следует, что $P(\omega ) \equiv P^*(\omega )$ для всех $P \in P(\omega
)$~\cite{3bat}.

\subsection{Постановка задачи} %1.3.

Перед нами встает задача \textit{разделения} такой смеси. Вообще говоря,
задача разделения смесей распределений со смешивающими функциями
общего вида является \textit{некорректно поставленной}, т.к.\ она допускает
существование нескольких решений. Поэтому будем искать решение в классе
\textit{конечных идентифицируемых смесей распределений}, где смешивающая
функция дискретна.

Для этого сузим данное выше определение и будем рассматривать в дальнейшем лишь 
случай конечного числа $k$ возможных значений па\-ра\-мет\-ра~$\omega$, что 
соответствует конечному числу скачков смешивающих функций $P(\omega )$.  
Величины этих скачков как раз и будут играть роль \textit{удельных весов} 
(\textit{априорных вероятностей}) $p_j$ компонентов смеси. Более того, в нашем 
случае мы постулируем также однотипность компонентов распределений $f_j$, т.е.\ 
принадлежность всех $f_j$ к одному общему па\-ра\-мет\-ри\-че\-ско\-му 
семейству $\{ f(X;\,\theta )\}$, где $\theta$~--- многомерный, вообще говоря, 
параметр. Так что~(3) в этом случае может быть записано в виде
\begin{equation} %4
p(x) = \sum\limits^k_{j=1} p_j f_j (x;\,\theta_j )\,.
\end{equation}

Переформулируем понятие идентифицируемости (различимости) смесей
специально применительно к такому виду смесей.

\medskip
\textbf{Определение 4.} \textit{Конечная смесь}~(3) называется
\textit{идентифицируемой} (\textit{различимой}), если из равенства
$$
\sum\limits_{j=1}^k p_j f_j (x;\,\theta_j ) = \sum\limits_{l=1}^{k^*} p_l^* f_l
(x;\,\theta_l^* )
$$
следует, что $k=k^*$ и для любого $j$ ($1\leq j \leq k$) найдется такое $l$ 
($1\leq l \leq k^*$), что $p_j = p_l^*$ и $f_j (x;\,\theta_j ) = f_l 
(x;\,\theta_l^* )$~\cite{3bat}.

Решить эту задачу в выборочном варианте~--- значит по выборке
классифицируемых наблюдений
$X_1,\ldots , X_n, $ извлеченной из генеральной совокупности, яв\-ля\-ющей\-ся смесью~(3)
генеральных совокупностей типа~(2) (при заданном общем виде составляющих
смесь функций $f_j (x;\,\theta_j )$), построить статистические оценки для числа
компонентов смеси~$k$, их удельных весов $p_j$ и, главное, для каждого из
компонентов %f_j (x;\,\theta_j )$ анализируемой смеси. Далее будет считать, что
функции $f_j$ однозначно определяются своими параметрами $\theta_j$: $f_j
=f(x;\,\theta_j)$.

Однако не следует ставить знак тождества между задачей разделения смеси
и задачей статистического оценивания параметров в модели~(4) по выборке $
X_1,\ldots , X_n$, поскольку задача разделения сохраняет смысл и
применительно к генеральным совокупностям, т.е.\ в теоретическом
варианте~\cite{3bat}.

Итак, для статистического анализа на основе реальных данных мы
аппроксимируем нашу общую модель~(1) следующей:
$$
p(x) \approx \hat{p}(x) = \sum\limits_{j=1}^k p_j \g_{\lambda_j , \alpha_j}
(x)\,,
$$
где $p_j$~--- дискретные смешивающие параметры, $\g_{\lambda_j , \alpha_j}
(x)$~--- плотности гамма-распределений.

Такая аппроксимация не только позволяет решить поставленную статистическую
задачу, но и полу\-чить наглядную визуализацию результатов. Существуют
достаточно эффективные методики разделения смесей распределений, среди них~---
семейство так называемых \textit{ЕМ-алгоритмов}
(\textit{Expectation-Maximization Algorithms}).

Полученные результаты могут быть достаточно просто интерпретированы. Число
компонентов смеси символизирует число типичных параллельных или
последовательных структур. Значения параметров составляющих смесь
гам\-ма-рас\-пре\-де\-ле\-ний показывают <<степень параллелизма>>
соответствующей структуры: чем ближе параметр формы к~1, тем выше эта
<<степень>>. И наоборот, чем дальше значение параметра формы от~1, тем больше
последовательных операций выполняется в соответствующем блоке.

Веса компонентов характеризуют примерную долю использования
ресурсов для сообщений, соответствующих каждому распределению входных
данных.

Итак, предложенный подход позволяет получить представление о
стохастической структуре телекоммуникационной сети.

\section{ЕМ-алгоритм разделения смесей распределений}

\subsection{Описание алгоритма} %2.1.

Определяемый ниже итерационный алгоритм решения поставленной в
предыдущем разделе задачи относится к процедурам, базирующимся на
\textit{методе максимального правдоподобия}.

Этот алгоритм позволяет находить максимум логарифмической функции
правдоподобия по параметрам $p_1,\,p_2,\ldots ,\,p_k$, $\theta_1 ,\,\theta_2,\ldots ,\,
\theta_k$ при фиксированном $k$ по выборке $X_1, \ldots , X_n$, т.е.\ решение
оптимизационной задачи вида

\begin{equation} \sum\limits_{i=1}^n \ln \left ( \sum\limits_{j=1}^k p_j f_j
(X_i;\,\theta_j )\right ) \rightarrow \underset{p_j,\,\theta_j}{\max}\,.
\end{equation}

Конкретные алгоритмы, построенные по этой схеме, часто называют
\textit{алгоритмами типа ЕМ}, поскольку в каждом из них можно выделить два
этапа, находящихся по отношению друг к другу в последовательности
итерационного взаимодействия: \textit{оценивание} (\textit{Estimation}) и
\textit{максимизация} (\textit{Maximization})~\cite{4bat}.

Введем в рассмотрение так называемые апостериорные вероятности
$\g_{ij}$ принадлежности наблюдения $X_i$ к $j$-му классу:
\begin{equation} %6
\g_{ij} = \fr{p_j f(X_i;\,\theta_j )}{\sum\limits_{l=1}^k p_l f(X_i;\,\theta_l 
)} \ (i=1,\ldots , n;\ j=1,\ldots ,k)\,.\!\!\end{equation} 
Очевидно, что для 
всех $i=1,\ldots ,n$ и $j=1,\ldots ,k$
$$
\g_{ij} \geq 0,\quad \sum_{j=1}^k \g_{ij} =1\,.
$$


Далее обозначим $\Theta = (p_1,\ldots p_k,\,\theta_1,\ldots ,\theta_k )$ и
представим анализируемую логарифмическую функцию правдоподобия
$$
\ln L(\Theta ) = \sum\limits_{i=1}^n \ln \left (\sum\limits_{j=1}^k p_j f_j
(X_i;\,\theta_j )\right )
$$
в виде
\begin{multline}
\ln L (\Theta ) = \sum\limits_{j=1}^k\sum\limits_{i=1}^n \g_{ij} \ln p_j+{}\\
{}+\sum\limits_{j=1}^k\sum\limits_{i=1}^n \g_{ij} f(X_i;\,\theta_j)-
\sum\limits_{j=1}^k\sum\limits_{i=1}^n \g_{ij} \ln \g_{ij}\,.
\end{multline}

Справедливость этого тождества легко проверяется с учетом
$$
\sum\limits_{j=1}^k \g_{ij} =1\,.
$$

Далее идея построения итерационного алгоритма вычисления оценок
$\hat{\Theta} = (\hat{p}_1,\ldots , \hat{p}_k,\
\hat{\theta}_1,\ldots , \hat{\theta}_k)$
для параметров $\Theta = (p_1,\ldots , p_k,\ \theta_1,\ldots , \theta_k)$ состоит в
следующем:
\begin{enumerate}[1.]
\item Выбирается некоторое \textit{начальное приближение}~$\hat{\Theta}^0$.
\item \textbf{E-step:} вычисляются по формулам~(6) начальные приближения
$\g_{ij}^0$ для апостериорных вероятностей $\g_{ij}$~--- \textit{этап
оценивания}.
\item \textbf{M-step:} затем, возвращаясь к~(7), при вычисленных значениях
$\g^0_{ij}$ следует определить значения $\hat{\Theta}^1$ из условия
максимизации отдельно каждого из первых двух слагаемых правой
части~(7), поскольку первое слагаемое
$$
\sum_{j=1}^k \sum_{i=1}^n \g_{ij} \ln p_j
$$
зависит только от параметров $p_j$, а второе слагаемое
$$
\sum_{j=1}^k \sum_{i=1}^n \g_{ij} f(X_i;\,\theta_j )
$$
зависит только от параметров $\theta_j$~--- \textit{этап максимизации}.
\item Проверяется \textit{условие останова}:
$$
\parallel \Theta^{(t)} - \Theta^{t-1}\parallel <\varepsilon\,,
$$
где $t$~--- номер итерации, а
$\parallel\bullet\parallel$~--- евклидова норма, для некоторого $\varepsilon
>0$.
\end{enumerate}

Очевидно, решение оптимизационной задачи
$$
\sum\limits_{j=1}^k\sum\limits_{i=1}^n \g_{ij}^{(t)}\ln p_j \rightarrow
\underset{p_j}{\max}
$$
дается выражением (с учетом $\sum_{j=1}^k p_j =1$):
$$
p_{ij}^{(t+1)} =\fr{1}{n}\,\sum\limits_{i=1}^n \g_{ij}^{(t)}\,,
$$
где $t$~--- номер итерации, $t = 0$, 1, 2,\,\ldots

Решение оптимизационной задачи
$$
\sum\limits_{j=1}^k \sum\limits_{i=1}^n \g_{ij}^{(t)} f(X_i;\,\theta_j )
\rightarrow \underset{\theta_j}{\max}
$$
получить намного проще решения задачи~(5): выражение для $\theta_j$
записывается с учетом знания конкретного вида функций
$f(X,\,\theta)$~\cite{3bat}.

\subsection{О сходимости алгоритма} %2.2.

В работе М.\,И.~Шлезингера~\cite{5bat}, где эта схема (позднее названная
ЕМ-схемой) впервые предложена, установлены и основные свойства
реа\-ли\-зу\-ющих ее алгоритмов. В частности, было доказано, что при достаточно
широких предположениях \textit{предельные точки} всякой последовательности,
порожденной итерациями ЕМ-алгоритма, являются стационарными точками
оптимизируемой логарифмической функции правдоподобия $\ln L(\Theta )$ и что
найдется неподвижная точка алгоритма, к которой будет сходиться каждая из таких
последовательностей. Если дополнительно потребовать положительной
определенности информационной мат\-ри\-цы Фишера для $\ln L(\Theta )$ при
истинных зна\-че\-ни\-ях па\-ра\-мет\-ра $\Theta$, то можно показать, что
асимптотически по $n\rightarrow\infty$ (т.е.\ при больших выборках) существует
единственное сходящееся (по веро\-ят\-но\-сти) решение $\hat{\Theta}(n)$
уравнений метода максимального правдоподобия и, кроме того, существует в
пространстве параметров $\Theta$ норма, в которой последовательность
$\Theta^{(t)}(n)$, порожденная ЕМ-ал\-го\-рит\-мом, сходится к $\hat{\Theta}
(n)$, если только начальная аппроксимация $\hat{\Theta}^0$ не была слишком
далека от~$\hat{\Theta} (n)$. {%\looseness=1

}

Таким образом, результаты исследования свойств ЕМ-алгоритмов метода
максимального правдоподобия разделения смеси и их практическое
использование показали, что они являются достаточно работоспособными (при
известном чис\-ле компонентов смеси) даже при большом чис\-ле $k$ компонентов и
при высоких размерностях анализируемого признака~$X$~\cite{3bat}.

\subsection{Уравнения для смеси экспоненциальных распределений}
%2.3.

Применим описанный выше алгоритм к разделению смеси
экспоненциальных распределений:
$$
p(x) = \sum\limits_{j=1}^k p_j \lambda_j e^{-\lambda_j x}\,.
$$
Получаем следующие итерационные уравнения:
\begin{align*}
\g_{ij}^{(t+1)} & = \fr{p_j^{(t)}\lambda_j^{(t)}e^{-
\lambda_j^{(t)}X_i}}{\sum\limits_{l=1}^k p_l^{(t)}\lambda_l^{(t)}
e^{-\lambda_l^{(t)}X_i}}\,,\\
p_j^{(t+1)} & = \fr{1}{n}\,\sum\limits_{i=1}^n \g_{ij}^{(t)}\,.
\end{align*}

Чтобы найти  оценки $\lambda_j$, подсчитаем первую производную функции
$$\sum_{j=1}^k\sum_{i=1}^n \g_{ij}^{(t)} \ln (\lambda_j e^{-\lambda_j X_i}):$$
\vspace*{-8pt}
\begin{multline*}
\left ( \sum\limits_{j=1}^k \sum\limits_{i=1}^n
\g_{ij}^{(t)}\ln \left ( \lambda_j
e^{-\lambda_j X_i} \right ) \right )^\prime \lambda_j =\\[-3pt]
{}= \left (
\sum\limits_{j=1}^k\sum\limits_{i=1}^n \g_{ij}^{(t)}\ln (\lambda_j -\lambda_j X_i )
\right )^\prime \lambda_j =\\[-3pt]
{}= \sum\limits_{i=1}^n \g_{ij}^{(t)}\left (
\fr{1}{\lambda_j} - X_i \right )\,.
\end{multline*}

Разрешая уравнение
$$
\sum\limits_{i=1}^n \g_{ij}^{(t)}\left ( \fr{1}{\lambda_j} -X_i\right ) =0
$$
относительно $\lambda_j$, получаем следующее итерационное уравнение:
$$
\lambda_j^{(t+1)} = \fr{\sum\limits_{i=1}^n
\g_{ij}^{(t)}}{\sum\limits_{i=1}^n \g_{ij}^{(t)} X_i}\,.
$$

\subsection{Уравнения для смеси гамма-распределений } %2.4.

Применим теперь ЕМ-алгоритм к смеси гам\-ма-рас\-пре\-де\-ле\-ний вида
$$
p(x) = \sum\limits_{j=1}^k p_j \fr{\alpha_j^{\alpha_j} x^{\alpha_j -
1}}{\lambda_j^{\alpha_j} \Gamma (\alpha_j )}\,e^{-(\alpha_j / \lambda_j)x}\,.
$$

Такая параметризация удобна для нахождения
оценок~$\alpha_j$~\cite{6bat}.

Аналогичным способом выписываются итерационные уравнения:
\begin{align*}
\g_{ij}^{(t+1)} & =   \fr{p_j^{(t)}\fr{(\alpha_j^{\alpha_j} )^{(t)}
x^{\alpha_j - 1}}{(\lambda_j^{\alpha_j} )^{(t)}\Gamma (\alpha_j)}\,
e^{-(\alpha_j /\gamma_j)^{(t)}x}}{\sum\limits_{l=1}^k
p_l^{(t)}\fr{(\alpha_l^{\alpha_l})^{(t)} x^{\alpha_l -
1}}{(\lambda_l^{\alpha_l})^{(t)}\Gamma (\alpha_l )}\,
e^{-(\alpha_l /\lambda_l)^{(t)} x}}\,,\\
p_j^{(t+1)} & = \fr{1}{n}\,\sum\limits_{i=1}^n \g_{ij}^{(t)}\,.
\end{align*}

Далее найдем оценки $\lambda_j$ для данного случая, приравнивая
производную
\begin{equation} %8
\sum\limits_{j=1}^k \sum\limits_{i=1}^n \g_{ij}^{(t)} \ln \left (
\fr{\alpha_j^{\alpha_j} x^{\alpha_j -1}}{\lambda_j^{\alpha_j}\Gamma
(\alpha_j)}\,e^{-(\alpha_j /\lambda_j) x}\right )
\end{equation}
по $\lambda_j$ к нулю и разрешая относительно~$\lambda_j$ уравнение:
$$
\sum\limits_{i=1}^n \g_{ij}^{(t+1)}\left ( \fr{\alpha_j^{(t)}}{\lambda_j^{(t)}}
- \fr{\alpha_j^{(t)}X_i}{\left ( \lambda_j^{(t)}\right )^2}\right ) =0 \,.
$$
Получаем
$$
\lambda_j^{(t+1)} = \fr{\sum\limits_{i=1}^n \g_{ij}^{(t)}
X_i}{\sum\limits_{i=1}^n \g_{ij}^{(t)}}\,.
$$

Для того чтобы получить итерационные уравнения для $\alpha_j$, найдем
первую производную~(8):
\begin{multline*}
\left ( \sum\limits_{j=1}^k\sum\limits_{i=1}^n \g_{ij}^{(t)}\ln \left (
\fr{\alpha_j^{\alpha_j} x^{\alpha_j -1}}{\lambda_j^{\alpha_j}\Gamma (\alpha_j
)}\,e^{-(\alpha_j /\lambda_j ) x} \right ) \right )^\prime \alpha_j ={}\\[-3pt]
{}=\left ( \sum\limits_{j=1}^k\sum\limits_{i=1}^n \g_{ij}^{(t)}\ln \left (
\fr{\alpha_j^{\alpha_j}}{\lambda_j^{\alpha_j}}\right ) - \ln \Gamma (\alpha_j )+{} \right.\\[-3pt]
{}+\left.
(\alpha_j -1 )\ln X_i - \fr{\alpha_j}{\lambda_j}\,X_i \right )^\prime \alpha_j =\\[-3pt]
{}=\sum\limits_{i=1}^n \g_{ij}^{(t)} \left ( \ln \alpha_j +1-\ln \lambda_j -
\fr{\Gamma^\prime (\alpha_j )}{\Gamma (\alpha_j)}\right.+\\[-3pt]
{}+\left. \ln X_i - \fr{X_i}{\lambda_j}\right )\,;
\end{multline*}
\begin{multline*}
\sum\limits_{i=1}^n \g_{ij}^{(t)} \left(  \ln \alpha_j +1 -\ln \lambda_j -{}\right. \\[-3pt]
\left. {}-\fr{\Gamma^\prime (\alpha_j )}{\Gamma (\alpha_j )}+\ln X_i 
-\fr{X_i}{\lambda_j} \right) =0\,;
\end{multline*}
\begin{multline}
\fr{\Gamma^\prime (\alpha_j )}{\Gamma (\alpha_j )} ={}\\[-3pt]
{}= \fr{\sum\limits_{i=1}^n \g_{ij}^{(t)} \left ( \ln \alpha_j +1-\ln\lambda_j 
+\ln X_i -\fr{X_i}{\lambda_j} \right )}{\sum\limits_{i=1}^n \g_{ij}^{(t)}}\,.
\end{multline}
%
Здесь $\Gamma^\prime (\alpha_j ) / \Gamma (\alpha_j )$~--- это
\textit{логарифмическая производная гамма-функции}. Для нее существует так
называемое \textit{разложение Абрамовитца}--\textit{Стигана}~\cite{4bat}:
$$
\fr{\Gamma^\prime (\alpha ) }{ \Gamma (\alpha )} = \mathrm{log}\,\alpha -
\fr{1}{2\alpha }-\fr{1}{12\alpha^2 }+\ldots
$$

Подставим первые три члена разложения в~(9) и разрешим это уравнение
относительно~$\alpha_j$:
$$
\alpha_{ij}^{(t+1)} = \fr{\sum\limits_{i=1}^n
\g_{ij}^{(t+1)}}{2\sum\limits_{i=1}^n \g_{ij}^{(t +1)}\left ( \fr{X_i}{\lambda_j^{(t)}} -
\ln \fr{X_i}{\lambda_j^{(t)}} -1\right )}\,.
$$
В итоге получаем итерационные уравнения для ~$\alpha_j$.

\section{Описание программного обеспечения (программа~ЕМ)}

\subsection{Назначение программы} %3.1.

Разработанная авторами статьи программа ЕМ предназначена для решения задачи
разделения смесей экспоненциальных и гамма-распределений, поставленной в
разд.~2, с использованием ЕМ-ал\-го\-рит\-ма и формул, описанных в разд.~3.

\subsection{Инструменты разработки} %3.2.

Для создания программы была использована среда разработки Microsoft
Visual Studio .NET 2005 и объектно-ориентированный язык C\#. Для
визуализации результатов была использована свободно распространяемая
графическая библиотека ZedGraph~\cite{7bat}.


\subsection{Возможности  программы} %3.3.

\noindent
\begin{itemize}
\item Загрузка выборочных данных из текстового файла
\item Оценивание по выборке параметров смеси экспоненциальных
распределений
\item Оценивание по выборке параметров смеси гамма-распределений
\item Отслеживание изменений параметров смесей распределений во
времени в режиме <<скользящего окна>>
\item Построение гистограммы по выборке
\end{itemize}

\subsection{Входные и выходные данные. Функционирование
программы} %3.4.

В качестве \textit{входных данных} программа ЕМ получает:
\begin{itemize}
\item выборочные данные из текстового файла;
\item число компонентов смеси;
\item размер <<скользящего окна>>;
\item размер класса гистограммы.
\end{itemize}

На \textit{выходе} мы получаем:
\begin{itemize}
\item точечные оценки параметров смеси экспоненциальных
распределений;
\item точечные оценки параметров смеси гамма-распределений;
\item графическое изображение результирующей смеси распределения;
\item графическое изображение компонентов каж\-дой смеси;
\item графическое изображение того, как меняются параметры смесей
распределений с течением времени в режиме <<скользящего окна>>;
\item гистограмма, построенная по выборке;
\item значение статистического теста.
\end{itemize}

Выборочные данные загружаются из текстового файла в память программы и подаются
на вход двум независимо работающим реализациям ЕМ-алгоритма~--- для
идентификации смеси экспоненциальных распределений и для идентификации смеси
гамма-распределений. Результатом их работы являются наборы значений оцениваемых
параметров модели, предложенной в разд.~2. Кроме того, результирующие
распределения визуализируются в виде графиков. В программе можно запустить
режим <<скользящего окна>>, который для всех подвыборок заданного
размера с помощью ЕМ-алгоритма оценивает параметры смесей распределений этих
подвыборок. Все действия программы документируются в окне информации.

\section{Описание тестовых расчетов}

С использованием разработанной программы были проведены тестовые
расчеты на выборочных данных реального сетевого трафика.

На вход программы EM были поданы выборки трафика:
\begin{enumerate}[I]
\item Между лабораторией Lawrence Berkeley (Berkeley, California) и
внешним миром размера примерно 7000~\cite{8bat}~--- \textit{выборка~1}.
\item
Сети радиодоступа ЗАО <<Синтерра>> размера примерно 1000~\cite{9bat}~---
 \textit{выборка~2}.
\end{enumerate}

\subsection{Выборка 1 ``Berkeley''} %5.1.

При числе компонентов смеси~5 и случайном начальном приближении
были получены результаты, представленные в табл.~\ref{t1bat}.


Данные результаты иллюстрирует рис.~\ref{f5bat}.

Гистограмма  на рис.~\ref{f6bat} показывает статистическую значимость
полученных результатов.

Данная выборка обладает той особенностью, что она собиралась в течение
достаточно длительного времени и в ней агрегирован самый разнородный
трафик. Поэтому в ней присутствует не только большое количество
<<коротких>> сообщений (что обычно для выборок из телетрафика), но и
некоторый массив сообщений средней длины, а также определенный
<<выброс>> больших сообщений. Это свидетельствует о \textit{пиковости}
телетрафика на довольно больших промежутках времени.

Как мы видим, ЕМ-алгоритм удачно справился с задачей идентификации
смеси.

\subsection{Выборка~2 ``Synterra''} %5.2.

Результаты применения ЕМ-алгоритма к выборке ``Synterra''
представлены в табл.~\ref{t2bat}.
\begin{table*}\small
\begin{minipage}[t]{76mm}
\begin{center}
\Caption{Результаты применения ЕМ-алго\-рит\-ма к выборке~1 ``Berkeley'' 
\label{t1bat}} \vspace*{2ex}

\tabcolsep=8.7pt
\begin{tabular}{|c|c|c|}
\hline
№&Начальное приближение&Результат\\
\hline
\multicolumn{3}{|c|}{$P$}\\
\hline
0&0,2&0,1896\\
1&0,2&0,1858\\
2&0,2&0,1830\\
3&0,2&0,2259\\
4&0,2&0,2154\\
\hline
\multicolumn{3}{|c|}{$\alpha$}\\
\hline
0&2,7028&10,9783\hphantom{9}\\
1&3,6273&5,8621 \\
2&5,7598&2,7092\\
3&0,2315&1,0235\\
4&0,9110&0,4772\\
\hline
\multicolumn{3}{|c|}{$\lambda$}\\
\hline
0&85,2066&137,1714  \\
1&23,9592&136,7349\\
2&63,8425&132,6482\\
3&\hphantom{9}1,8026&116,7317\\
4&98,3882&102,5278\\
\hline
\end{tabular}
\end{center}
\end{minipage}\hfill
\begin{minipage}[t]{76mm}
%\end{table*}
%\begin{table*}\small
\begin{center}
\Caption{Результаты применения ЕМ-алго\-рит\-ма к выборке~2 ``Synterra'' 
\label{t2bat}} \vspace*{2ex}

\tabcolsep=8.7pt
\begin{tabular}{|c|c|c|}
\hline
№&Начальное приближение&Результат\\
\hline
\multicolumn{3}{|c|}{$P$}\\
\hline
0&0,2&$0{,}3815\hphantom{{}\cdot 10^{-9}}$\\
1&0,2&$0{,}3594\hphantom{{}\cdot 10^{-9}}$\\
2&0,2&$0{,}2589\hphantom{{}\cdot 10^{-9}}$\\
3&0,2&$0{,}4401\cdot 10^{-9}$\\
4&0,2&$0{,}0\hphantom{{}\cdot 10^{-9}999}$\\
\hline
\multicolumn{3}{|c|}{$\alpha$}\\
\hline
0&6,0804&1,5833\\
1&3,1838&0,8554\\
2&1,4886&0,4557\\
3&4,6407&0,2278\\
4&3,7843&0,1139\\
\hline
\multicolumn{3}{|c|}{$\lambda$}\\
\hline
0&17,3387&15,8682\\
1&47,8294&16,9150\\
2&54,1984&19,2866\\
3&\hphantom{1}8,6254&19,2866\\
4&\hphantom{1}5,7252&19,2866\\
\hline
\end{tabular}
\end{center}
\end{minipage}
\end{table*}


Данные результаты иллюстрируют рис.~\ref{f7bat}.


Эти результаты также отражают действительную картину, как показано на
рис.~\ref{f8bat}.


Этот трафик был снят с базовой станции <<Лукойл-Юго-Запад>> сети
широкополосного радиодоступа ЗАО <<Синтерра>>. Сеть радиодоступа
является реализацией так называемой <<последней мили>>, переносящей два
разных вида трафика: данные (Ethernet пакеты) и голос (IP-телефония, VoIP).
Поэтому здесь присутствуют в качестве основной массы короткие, но
интенсивные сообщения (пакеты SIP и голосовые фреймы), а также длинные
сообщения, содержащие данные.

Как мы видим, программная реализация ЕМ-ал\-го\-рит\-ма успешно справилась с
задачей разделения смесей распределений для этих двух выборок, что делает
данную программу удобным инструментом построения стохастической картины
конкретной сети. По полученным данным, используя метод интерпретации,
предложенный в разд.~2, можно получить представление о количестве
последовательных и параллельных структур вероятностной модели сети.

\subsection{Режим <<скользящего окна>>} %5.3.

Результаты для выборки
``Berkeley'' в режиме <<скользящего окна>>  представлены
на рис.~\ref{f9bat}.


Данные графики показывают изменение параметров распределений подвыборок выборки 
``Berkeley''. Видно, что параметры распределений подвыборок не остаются 
неизменными во времени, наоборот, они имеют внешне случайный характер. На 
рис.~\ref{f9bat},\,\textit{в} видна даже своеобразная пульсация первой 
компоненты.
%
На основании расчетов можно сделать вывод о том, что пиковость трафика
обусловливается как формой, так и интенсивностью сообщений.

\section{Заключение}

В данной работе исследована вероятностная модель  информационных потоков,
возникающих в сложных телекоммуникационных конвергентных сетях, построенная с
помощью асимптотического и энтропийного подходов. Эта модель предполагает, что
функционирование сложной телекоммуникационной сети можно представить в виде
суперпозиции довольно простых стохастических структур~--- последовательных и
параллельных, которые по\-рож\-да\-ют смеси гамма-распределений для случайной
величины времени обработки и передачи сообщений в сети. Предложена простая
интерпретация параметров данной модели.
\begin{figure*} %fig5
\vspace*{1pt}
\begin{center}
\mbox{%
\epsfxsize=130mm %145.109mm 
\epsfbox{bat-5.eps} }
\end{center}
\vspace*{-13pt} \Caption{Компоненты смеси начального приближения~(\textit{а}) и 
результата~(\textit{б}) для выборки~1 ``Berkeley'' \label{f5bat}}
%\end{figure*}
%\begin{figure*} %fig6
\vspace*{12pt}
\begin{center}
\mbox{%
\epsfxsize=130mm %148.256mm 
\epsfbox{bat-7.eps} }
\end{center}
\vspace*{-13pt} \Caption{График смеси распределений~(\textit{1}) и гистограмма 
для выборки~1 ``Berkeley''~(\textit{2}) \label{f6bat}}
\end{figure*}



\begin{figure*} %fig7
\vspace*{1pt}
\begin{center}
\mbox{%
\epsfxsize=130mm %144.283mm 
\epsfbox{bat-8.eps} }
\end{center}
\vspace*{-16pt} \Caption{Компоненты смеси начального приближения~(\textit{а}) и 
результата~(\textit{б}) для выборки~2 ``Synterra'' \label{f7bat}}
%\end{figure*}
%\begin{figure*} %fig8
\vspace*{12pt}
\begin{center}
\mbox{%
\epsfxsize=130mm %148.256mm 
\epsfbox{bat-10.eps} }
\end{center}
\vspace*{-11pt} \Caption{График смеси распределений~(\textit{1}) и гистограмма
для выборки~2 ``Synterra''~(\textit{2}) \label{f8bat}}
\end{figure*}

\begin{figure*} %fig9
\vspace*{1pt}
\begin{center}
\mbox{%
\epsfxsize=119.041mm
\epsfbox{bat-11.eps} }
\end{center}
\vspace*{-9pt} \Caption{Изменение  смешивающих параметров~(\textit{а}), 
параметров формы~(\textit{б}) и параметров масштаба~(\textit{в}) во времени для 
выборки~1 ``Berkeley'' \label{f9bat}}
\end{figure*}

Для решения вытекающей из модели задачи предложен итерационный алгоритм,
базирующийся на методе максимального правдоподобия~--- ЕМ-ал\-го\-ритм, для
которого получены формулы для конкретного вида смесей~--- экспоненциальных и
гамма-распределений.
%
Кроме того, разработан программный инструментарий для оценки параметров 
предложенной модели на выборках из реальных трафиковых данных. Проведены 
исследования, которые подтвердили предположения вероятностной модели. 


Получение информации о стохастической структуре
телекоммуникационных сетей и наличие программных инструментов для
выявления более или менее стабильных структур позволит понять причины
возникновения неожиданных больших нагрузок, предотвратить такие нагрузки,
а также поможет в будущем в проектировании надежных, оптимальных по
стоимости и уровню сервиса телекоммуникационных сетей нового поколения.

%\vspace*{-15pt} 
{\small\frenchspacing
{%\baselineskip=10.8pt
\addcontentsline{toc}{section}{Литература}
\begin{thebibliography}{9}
\bibitem{1bat}
Teletraffic Engeneering Handbook. International Telecommunication Union, 
Geneva, 2005 {\sf http://www.itu.int}. \vspace*{5pt} 
\bibitem{2bat}
\Au{Севастьянов~Б.\,А.} Курс теории вероятностей и математической статистики. 
М., 2004. \vspace*{5pt} 
\bibitem{3bat}
\Au{Айвазян~C.\,А., Бухштабер~В.\,М., Енюков~И.\,С, Мешалкин~Л.\,Д.} Прикладная 
статистика. Классификация и снижение размерности~// Финансы и статистика. М., 
1989. \vspace*{5pt} 
\bibitem{4bat}
\Au{Bilmes~J.\,A.} A gentle tutorial of the EM algorithm and its application to 
parameter estimation for Gaussian mixture and hidden Markov models. Berkeley, 
CA, USA: International Computer Science Institute,  1998. \vspace*{5pt} 
\bibitem{5bat}
\Au{Шлезингер~М.\,И.} О самопроизвольном различении образов~// Шлезингер~М.\,И. 
Читающие. автоматы. Киев: Наукова думка, 1965. С.~38--45. \vspace*{5pt} 
\bibitem{6bat}
\Au{Hsiao~I.-T., Rangarajan~A., Gindi~G.}. Joint-MAP 
reconstruction/segmentation for transmission tomography using mixture-models as 
priors. Yale University, 1998. \vspace*{5pt} 
\bibitem{7bat}
{\sf http://zedgraph.org}. \vspace*{4pt} 
\bibitem{8bat}
{\sf http://ita.ee.lbl.gov/html/contrib/LBL-PKT.html}. \vspace*{5pt} 
\bibitem{9bat}
{\sf http://www.synterra.ru}.
\end{thebibliography}

} } \label{end\stat}
\end{multicols}


%\addtocounter{razdel}{1}
%\def\razd{НЕРЕГУЛИРУЕМЫЙ ЭЛЕКТРОПРИВОД ДЛЯ ЭЛЕКТРОЭНЕРГЕТИКИ}

\setcounter{page}{2}

%   { %\Large  
   { %\baselineskip=16.6pt
   
   \vspace*{-48pt}
   \begin{center}\LARGE
   \textit{Предисловие}
   \end{center}
   
   %\vspace*{2.5mm}
   
   \vspace*{25mm}
   
   \thispagestyle{empty}
   
   { %\small 

    
Вниманию читателей журнала <<Информатика и её применения>> предлагается 
очередной тематический выпуск <<Вероятностно-статистические методы и 
задачи информатики и информационных технологий>>. Предыдущие тематические 
выпуски журнала по данному направлению вышли в 2008~г.\ (т.~2, вып.~2), 
в 2009~г.\ (т.~3, вып.~3) и в 2010~г.\ (т.~4, вып.~2). 

Статьи, собранные в данном журнале, посвящены разработке новых вероятностно-статистических 
методов, ориентированных на применение к решению конкретных задач информатики и информационных 
технологий, а также~--- в ряде случаев~--- и других прикладных задач. Проблематика, охватываемая 
публикуемыми работами, развивается в рамках научного сотрудничества между Институтом проблем 
информатики Российской академии наук (ИПИ РАН) и Факультетом вычислительной математики и 
кибернетики Московского государственного университета им.\ М.\,В.~Ломоносова в ходе работ 
над совместными научными проектами (в том числе в рамках функционирования 
Научно-образовательного центра <<Вероятностно-статистические методы анализа рисков>>). 
Многие из авторов статей, включенных в данный номер журнала, являются активными участниками 
традиционного международного семинара по проблемам устойчивости стохастических моделей, 
руководимого В.\,М.~Золотаревым и В.\,Ю.~Королевым; регулярные сессии этого семинара 
проводятся под эгидой МГУ и ИПИ РАН (в 2011~г.\ указанный семинар проводится в октябре 
в Калининградской области РФ). 

Наряду с представителями ИПИ РАН и МГУ в число авторов данного выпуска журнала входят 
ученые из Научно-исследовательского института системных исследований РАН, Института 
проблем технологии микроэлектроники и особочистых материалов РАН, Института 
прикладных математических исследований Карельского НЦ РАН, Московского 
авиационного института, Вологодского государственного педагогического университета, 
НИИММ им.\ Н.\,Г.~Чеботарева, Казанского государственного университета, Дебреценского 
университета (Венгрия).

Несколько статей выпуска посвящено разработке и применению стохастических методов и 
информационных технологий для решения различных прикладных задач. В~работе В.\,Г.~Ушакова 
и О.\,В.~Шестакова рассмотрена задача определения вероятностных характеристик случайных 
функций по распределениям интегральных преобразований, возникающих в задачах эмиссионной 
томографии. В~статье Д.\,О.~Яковенко и М.\,А.~Целищева рассмотрены некоторые вопросы 
математической теории риска и предложен новый подход к диверсификации инвестиционных 
портфелей. Работа И.\,А.~Кудрявцевой и А.\,В.~Пантелеева посвящена построению и 
исследованию математической модели, описывающей динамику сильноионизованной плазмы. 
В~статье П.\,П.~Кольцова изучается качество работы ряда алгоритмов сегментации изображений. 
Статья А.\,Н.~Чупрунова и И.~Фазекаша посвящена вероятностному анализу числа без\-оши\-бочных 
блоков при помехоустойчивом кодировании; получены усиленные законы больших чисел для указанных 
величин.

В данном выпуске традиционно присутствует тематика, весьма активно разрабатываемая в течение 
многих лет специалистами ИПИ РАН и МГУ,~--- методы моделирования и управления для 
информационно-телекоммуникационных и вычислительных систем, в частности методы 
теории массового обслуживания. В~статье А.\,И.~Зейфмана с соавторами рассматриваются 
модели обслуживания, описываемые марковскими цепями с непрерывным временем в случае 
наличия катастроф. В~работе М.\,М.~Лери и И.\,А.~Чеплюковой рассматриваются случайные 
графы Интернет-типа, т.\,е.\ графы, степени вершин которых имеют степенные распределения; 
такие задачи находят применение при исследовании глобальных сетей передачи данных. 
Работа Р.\,В.~Разумчика посвящена исследованию систем массового обслуживания специального 
вида~--- с отрицательными заявками и хранением вытесненных заявок.

Ряд статей посвящен развитию перспективных теоретических 
вероятностно-статистических методов, которые находят широкое применение в различных 
задачах информатики и информационных технологий. В~работе В.\,Е.~Бенинга, А.\,К.~Горшенина 
и В.\,Ю.~Королева рассмотрена задача статистической проверки гипотез о числе компонент 
смеси вероятностных распределений, приводится конструкция асимптотически наиболее мощного 
критерия. Результаты этой работы найдут применение в ряде прикладных задач, использующих 
математическую модель смеси вероятностных распределений (в информатике, моделировании 
финансовых рынков, физике турбулентной плазмы и~т.\,д.). В~статье В.\,Ю.~Королева, 
И.\,Г.~Шевцовой и С.\,Я.~Шоргина строится новая, улучшенная оценка точности нормальной 
аппроксимации для пуассоновских случайных сумм; как известно, указанные случайные суммы 
широко используются в качестве моделей многих реальных объектов, в том числе в информатике, 
физике и других прикладных областях. Работа В.\,Г.~Ушакова и Н.\,Г.~Ушакова посвящена 
исследованию ядерной оценки плотности распределения; эти результаты могут применяться, 
в част\-ности, при анализе трафика в телекоммуникационных системах. Серьезные приложения 
в статистике могут получить результаты работы О.\,В.~Шестакова, в которой доказаны оценки 
скорости сходимости распределения выборочного абсолютного медианного отклонения к нормальному 
закону. 

\smallskip

Редакционная коллегия журнала выражает надежду, что данный тематический  выпуск 
будет интересен специалистам в области теории вероятностей и математической статистики 
и их применения к решению задач информатики и информационных технологий.
     
     %\vfill 
     \vspace*{20mm}
     \noindent
     Заместитель главного редактора журнала <<Информатика и её 
применения>>,\\
     директор ИПИ РАН, академик  \hfill
     \textit{И.\,А.~Соколов}\\
     
     \noindent
     Редактор-составитель тематического выпуска,\\
     профессор кафедры математической статистики факультета\\
      вычислительной математики и кибернетики МГУ им.\ М.\,В.~Ломоносова,\\
     ведущий научный сотрудник ИПИ РАН,\\ 
доктор физико-математических наук \hfill
      \textit{В.\,Ю.~Королев}
     
     } }
     }

\def\stat{sinits}

\def\tit{АНАЛИТИЧЕСКОЕ МОДЕЛИРОВАНИЕ
НОРМАЛЬНЫХ ПРОЦЕССОВ В~СТОХАСТИЧЕСКИХ СИСТЕМАХ СО~СЛОЖНЫМИ~НЕЛИНЕЙНОСТЯМИ}

\def\titkol{Аналитическое моделирование
нормальных процессов в~стохастических системах со~сложными нелинейностями}

\def\aut{И.\,Н.~Синицын$^1$, В.\,И.~Синицын$^2$}

\def\autkol{И.\,Н.~Синицын, В.\,И.~Синицын}

\titel{\tit}{\aut}{\autkol}{\titkol}

\renewcommand{\thefootnote}{\arabic{footnote}}
\footnotetext[1]{Институт проблем
информатики Российской академии наук, sinitsin@dol.ru}
\footnotetext[2]{Институт проблем
информатики Российской академии наук, vsinitsin@ipiran.ru}


\Abst{Рассматриваются конечномерные дифференциальные стохастические системы
(ДСтС) и эредитарные (интегродифференциальные) стохастические системы  (ЭСтС)
с винеровскими и пуассоновскими шумами, приводимые к ДСтС со сложными конечными,
дифференциальными и интегральными нелинейностями. Такие модели функционирования
описывают поведение многих современных нано- и кван\-то\-во-оп\-ти\-че\-ских
технических средств информатики. Приводятся уравнения методов нормальной
аппроксимации (МНА) и статистической линеаризации (МСЛ) для аналитического
моделирования нестационарных и стационарных нормальных (гауссовских) процессов
в нелинейных ДСтС и  нелинейных ЭСтС путем аппроксимации эредитарных ядер
линейными операторными уравнениями для дифференцируемых нелинейностей и
сингулярными ядрами для недифференцируемых нелинейностей. Рассматриваются
методы вычисления типовых интегралов МНА (МСЛ) для сложных (многомерных и
векторного аргумента) конечных и дифференциальных нелинейностей. Особое
внимание уделяется иррациональным и дробно-рациональным нелинейностям
скалярного аргумента. Приводятся примеры вычисления интегралов. Подробно
рассматриваются вопросы вычисления типовых интегралов МНА (МСЛ) для сложных
интегральных нелинейностей.}

\KW{аналитическое моделирование;
дифференциальные стохастические системы с винеровскими и пуассоновскими шумами (ДСтС);
метод нормальной аппроксимации (МНА);
метод статистической линеаризации (МСЛ);
сложные иррациональные нелинейности;
сложные конечные, дифференциальные и интегральные нелинейности;
эредитарные стохастические системы (ЭСтС), приводимые к дифференциальным}

\DOI{10.14357/19922264140302}

\vspace*{9pt}

\vskip 16pt plus 9pt minus 6pt

\thispagestyle{headings}

\begin{multicols}{2}

\label{st\stat}


\section{Введение}


Моделями функционирования многих современных технических сис\-тем информатики
служат стохастические системы (СтС), описываемые дифференциальными, интегральными
и интегродифференциальными уравнениями со сложными дроб\-но-ра\-ци\-о\-наль\-ны\-ми,
иррациональными и интегральными нелинейностями. В~[1] дано систематическое
изложение МНА и МСЛ для ДСтС и ЭСтС, приводимых к дифференциальным.

Обобщая~[2--7], рассмотрим развитие МНА и МСЛ для аналитического моделирования
нормальных стохастических процессов (СтП) на случай СтС со сложными конечными,
дифференциальными и интегральными нелинейностями.

Как показано в~\cite{4-sin}, альтернативным подходом к аналитическому моделированию
СтП в ДСтС и ЭСтС служит подход, основанный на дискретизации стохастических
дифференциальных уравнений на основе использования обобщенной формы Ито и
кратных стохастических интегралов от винеровских и пуассоновских СтП с
последующим применением дискретных версий МНА (МСЛ).

Статья состоит из введения, пяти разделов и заключения.

В~разд.~2 и~3
приводятся уравнения МНА и МСЛ для аналитического моделирования одно- и
двумерных распределений стационарных и нестационарных СтП в ДСтС и ЭСтС,
приводимых к ДСтС.

Типовые интегралы МНА и МСЛ рассматриваются в разд.~4.

Особенности аналитического моделирования в ДСтС со сложными конечными и
дифференциальными нелинейностями обсуждаются в разд.~5.

Раздел~6
посвящен аналитическому моделированию СтП в ДСтС со сложными интегральными
нелинейностями.

Приводятся примеры.


\section{Уравнения методов нормальной~аппроксимации и~статистической
линеаризации для~дифференциальных стохастических систем}

Как известно~\cite{2-sin, 3-sin},  уравнения конечномерных непрерывных нелинейных сис\-тем
со стохастическими возмущениями путем расширения вектора состояния ДСтС
могут быть записаны в виде следующего векторного стохастического
дифференциального уравнения Ито:
    \begin{multline}
    dY_t = a(Y_t, t)\, dt + b (Y_t, t) \,dW_0+{}\\
    {}+ \iii_{R_0} c (Y_t, t, v) P^0
    (dt, dv)\,,\enskip Y(t_0) = Y_0\,.\label{e2.1-sin}
    \end{multline}
Здесь $a=a(Y_t, t)$ и $b\hm=b(y_t, t)$~--- известные
$(p\times 1)$-мер\-ная и  $(p\times m)$-мер\-ная функции~$Y_t$ и~$t$;
$W_0\hm= W_0(t)$~--- $r$-мер\-ный винеровский СтП интенсивности
$\nu_0 \hm= \nu_0(t)$; $c(Y_t, t, v)$~--- $(p\times 1)$-мер\-ная функция  $Y_t, t$
и вспомогательного $(q\times 1)$-мер\-но\-го па\-ра\-мет\-ра~$v$;
$\iii_{\Delta} dP^0 (t, A)$~--- центрированная пуассоновская мера,
определяемая
\begin{equation*}
\iii_{\Delta} dP^0 (t, A) = \iii_{\Delta} dP (t,A) =
\iii_{\Delta} \nu_P (t,A)\, dt\,. %\label{e2.2-sin}
\end{equation*}
В~(\ref{e2.1-sin}) принято: $\iii_{\Delta}$~-- число скачков пуассоновского
СтП в интервале времени  $\Delta \hm= (t_1, t_2]$; $\nu_P (t, A)$~---
интенсивность пуассоновского СтП  $P(t,A)$; $A$~--- некоторое борелевское
множество пространства  $R_0^q$ с выколотым началом.
Начальное значение~$Y_0$ представляет собой случайную величину, не зависящую
от приращений СтП  $W_0(t)$ и $P(t,A)$ на интервалах времени, следующих
за~$t_0$, $t_0 \hm\le t_1\hm\le t_2$ для любого множества~$A$.

В случае аддитивных нормальных (гауссовских) и обобщенных
пуассоновских возмущений уравнение~(\ref{e2.1-sin}) имеет вид:
\begin{equation}
\dot Y_t = a(Y_t,t)+ b_0 (t) V\,, \enskip
V = \dot W\,,\enskip Y(t_0) = Y_0\,.\label{e2.3-sin}
\end{equation}
Здесь $W$~--- СтП с независимыми приращениями, представляющий собой
смесь нормального и обобщенного пуассоновского СтП.

Если предположить существование конечных вероятностных
моментов второго порядка для моментов времени~$t_1$ и~$t_2$, то уравнения
МНА примут следующий вид~\cite{2-sin, 3-sin}:
\begin{itemize}
\item  для характеристических функций
    \begin{equation}
    g_1^N (\la;t) =\exp \lk i\la^{\mathrm{T}} m_t - \fr{1}{2}\, \la^{\mathrm{T}} K_t \la\rk\,;\label{e2.4-sin}
    \end{equation}
\begin{equation}
\hspace*{-7.5mm}g_{t_1, t_2}^N (\la_1, \la_2;t_1, t_2 ) =\exp \lk i\bar \la^{\mathrm{T}} \bar m_2 -
\fr{1}{2}\, \bar \la^{\mathrm{T}} \bar K_2 \la\rk\,,\!\!\label{e2.5-sin}
\end{equation}
где
    \begin{gather*}
    \bar \la =\lk \la_1^{\mathrm{T}}\la_2^{\mathrm{T}}\rk^{\mathrm{T}}\,; \quad
        \bar m_2 = \lk m_{t_1}^{\mathrm{T}} m_{t_2}^{\mathrm{T}}\rk^{\mathrm{T}}\,;\\
        \bar K_2= \begin{bmatrix}
    K(t_1, t_1)& K(t_1, t_2)\\
    K(t_2, t_1)& K(t_2, t_2)
    \end{bmatrix}\,;
    \end{gather*}

\item для математических ожиданий  $m_t$, ковариационной матрицы~$K_t$ и
матрицы ковариационных функций $K(t_1, t_2)$:
    \begin{equation}
    \dot m_t = a_1 (m_t, K_t, t)\,,\enskip m_0 = m(t_0)\,;\label{e2.6-sin}
    \end{equation}
\begin{equation}
\dot K_t = a_2 (m_t, K_t, t)\,,\enskip K_0 = K(t_0)\,;\label{e2.7-sin}
\end{equation}

\vspace*{-12pt}

\noindent
\begin{multline}
\fr{\prt K(t_1, t_2)}{\prt t_2 }= K(t_1, t_2) a_{21} (m_{t_2}, K_{t_2}, t_2)^{\mathrm{T}}\,;\\
K(t_1, t_1) = K_{t_1}\,.
\label{e2.8-sin}
\end{multline}
    \end{itemize}
Здесь приняты следующие обозначения:
\begin{equation}
a_1 = a_1 (m_t, K_t, t) = M_N a (Y_t, t)\,;\label{e2.9-sin}
\end{equation}

\vspace*{-12pt}

\noindent
\begin{multline}
a_2 = a_2 (m_t, K_t, t) = a_{21} (m_t, K_t, t)+{}\\
{}+ a_{21} (m_t, K_t, t)^{\mathrm{T}} +
a_{22}(m_t, K_t, t)\,;\label{e2.10-sin}
\end{multline}

\vspace*{-12pt}

\noindent

\begin{equation}
a_{21} = a_{21}(m_t, K_t, t)=  M_N a(Y_t, t) Y_{t}^{0\mathrm{T}}\,;\label{e2.11-sin}
\end{equation}
\begin{equation*}
a_{22} = a_{22}(m_t, K_t, t)= M_N \sigma (Y_t, t)\,;
%\label{e2.12-sin}
\end{equation*}

\vspace*{-12pt}

\noindent
\begin{multline*}
\sigma (Y_t, t) = b(Y_t, t) \nu_0(t) b(Y_t, t)^{\mathrm{T}} +{}\\
{}+
\iii_{R_0^q} c (Y_t, t, v) c(Y_t, t,v)^{\mathrm{T}}
\nu_P (t, dv)\,; %\label{e2.13-sin}
\end{multline*}

\vspace*{-12pt}

\begin{gather*}
m_t = MY_t\,,\quad Y_t^0 = Y_t - m_t\,,\\
K_t = M_N Y_0^0 Y_t^{0\mathrm{T}}\,,\quad K(t_1, t_2) =
M_N Y_{t_1}^0 Y_{t_2}^0\,; %\label{e2.14-sin}
\end{gather*}
$M_N$~--- символ вычисления математического ожидания для нормальных
распределений~(\ref{e2.4-sin}) и~(\ref{e2.5-sin}).

Для стационарных ДСтС нормальные стационарные СтП~--- если они существуют,
то  $m_t \hm=\bar m$, $ K_t \hm=\bar K$, $K(t_1, t_2) \hm= k(\tau)$
$(\tau \hm= t_1\hm-t_2)$,~--- определяются уравнениями~\cite{2-sin, 3-sin}:
   \begin{equation}
   a_1 (\bar m, \bar K) =0\,;\enskip a_2 (\bar m, \bar K)=0\,;\label{e2.15-sin}
   \end{equation}
   \begin{equation}
   \left.
   \hspace*{-2.8mm}\begin{array}{l}
  \dot k_\tau (\tau) = a_{21} (\bar m, \bar K)\bar K^{-1} k(\tau)\,;\\[9pt]
  k(0) =\bar K \enskip (\forall \tau >0)\,, \
  k(\tau) = k(-\tau)^{\mathrm{T}} \enskip
  (\forall\tau <0)\,.
  \end{array}\!\!
  \right\}\!\!
  \label{e2.16-sin}
  \end{equation}
При этом необходимо, чтобы матрица  $a_{21} (\bar m, \bar K)\hm=\bar a_{21}$
была бы асимптотически устойчивой.

Для ДСтС~(\ref{e2.3-sin}) уравнения МНА переходят в уравнения МСЛ
Казакова~\cite{2-sin, 3-sin}, если принять
\begin{equation}
a(Y_t,t) = a_1 (m_t, K_t) + k_1^a (m_t, K_t) Y_t^0\,;\label{e2.17-sin}
\end{equation}
\begin{equation}\left.
\begin{array}{rl}
b(Y_t,t) &= b_0 (t)\,;\\[9pt]
    \si(Y_t, t)&= b_0(t) \nu(t) b_0(t)^{\mathrm{T}} = \si_0(t)\,,
    \end{array}
    \right\}\label{e2.18-sin}
    \end{equation}
    \begin{equation}
k_1^a (m_t, K_t, t) =\lk \left(\fr{\prt}{\prt m_t} \right)
    a_0 (m_t, K_t, t)^{\mathrm{T}}\rk^{\mathrm{T}}\,;\label{e2.19-sin}
    \end{equation}
    \begin{equation}
\dot m_t = a_1 (m_t, K_t, t) \,,\enskip m_0 = m(t_0)\,,\label{e2.20-sin}
\end{equation}

\vspace*{-12pt}

\noindent
\begin{multline}
\dot K_t = k_1^a (m_t, K_t, t) K_t + K_t k_1^a (m_t, K_t, t)^{\mathrm{T}}
    +\si_0(t)\,;\\
    K_0 = K(t_0)\,;
    \label{e2.21-sin}
    \end{multline}

    \vspace*{-12pt}

    \noindent
\begin{multline}
\fr{\prt K(t_1, t_2)}{\prt t_2} =
    K(t_1, t_2) k_{t_2} k_1^a (m_{t_2}, K_{t_2}, t_2)^{\mathrm{T}}\,;\\
    K(t_1, t_2) = K_{t_1}\,.
    \label{e2.22-sin}
\end{multline}

Для стационарных ДСтС~(\ref{e2.3-sin})
при условии асимптотической устойчивости матрицы $k_1^a (\bar m, \bar K)$
в основе МСЛ лежат уравнения~(\ref{e2.15-sin}), записанные в виде:
    \begin{gather}
    a_1 (\bar m, \bar K) =0\,; \label{e2.23-sin}\\
k_1^a (\bar m, \bar K) \bar K + \bar K k_1^a
(\bar m, \bar K)^{\mathrm{T}} +\bar \si_0 =0\,;\label{e2.24-sin}
\end{gather}

\vspace*{-12pt}

\noindent
\begin{multline}
k_\tau (\tau) = k_1^a (\bar m, \bar K)k(\tau)\,,\enskip
k(0) =\bar K \enskip (\forall \tau >0)\,,\\
k(\tau) = k (-\tau)^{\mathrm{T}} \enskip (\forall \tau <0)\,.
\label{e2.25-sin}
\end{multline}

Уравнения~(\ref{e2.4-sin})--(\ref{e2.8-sin})
лежат в основе МНА для ДСтС~(\ref{e2.1-sin}), а уравнения~(\ref{e2.17-sin})--(\ref{e2.22-sin})~---
в основе МСЛ для ДСтС~(\ref{e2.3-sin}). Для определения стационарных СтП
согласно МНА служат соотношения~(\ref{e2.15-sin}) и~(\ref{e2.16-sin}),
а МСЛ~--- (\ref{e2.17-sin})--(\ref{e2.25-sin}).

\section{Уравнения методов нормальной~аппроксимации и~статистической линеаризации
для~эредитарных стохастических систем, приводимых к~дифференциальным}

Рассмотрим ЭСтС, описываемую интегродифференциальным уравнением Ито
следующего вида~\cite{7-sin}:

\noindent
\begin{multline}
dX_t = \lk a(X_t, t) +\iii_{t_0}^t a_1 (X(\tau) ,\tau, t)\,d\tau\rk dt+{}\\
{}+\lk b(X_t, t) +\iii_{t_0}^t b_1 (X(\tau) ,\tau, t)\,d\tau\rk dW_0+{}\\
\hspace*{-1.5mm}{}+\!\!\iii_{R_0^q}\!\!\lk c(X_t, t,v) +\!\iii_{t_0}^t\! c_1 (X(\tau) ,\tau, t,v)\,d\tau\!\rk\! dP^0 (t, dv)
\!\!\!\!\label{e3.1-sin}
\end{multline}
с начальным условием  $X(t_0) = X_0$. В~(\ref{e3.1-sin})
сохранены обозначения разд.~2.

Функции $a=a(X_t, t)$, $a_1\hm = a_1(X (\tau),\tau, t)$,
$b\hm=b(X_t, t)$, $b_1\hm = b_1(X (\tau),\tau, t)$,
$c\hm=c(X_t,t,v)$ и $c_1\hm = c_1(X (\tau),\tau, t,v)$ имеют
соответственно размерности $p\times 1$, $p\times 1$, $p\times r$,
$p\times r$, $p\times 1$ и $p\times 1$ и допускают представления следующего вида:
\begin{equation}
\left.
\begin{array}{rl}
a_1&=A(t,\tau) \vrp (X(\tau) , \tau)\,;\\[9pt]
b_1&=B(t,\tau) \psi (X(\tau) ,  \tau)\,;\\[9pt]
c_1&=C(t,\tau) \chi (X(\tau) ,  \tau, v)\,.
\end{array}
\right\}
\label{e3.2-sin}
\end{equation}
Здесь эредитарные ядра $A\hm=A(t,\tau)\hm=\lk A_{ij}(t,\tau)\rk$
$(i,j\hm=\overline{1,p})$,
$B\hm=B(t,\tau)=\lk B_{i l}(t,\tau)\rk$ $(i\hm=\overline{1,p}$;
$l\hm=\overline{1,r})$ и $C\hm=C(t,\tau)=\lk C_{ij}(t,\tau)\rk$
$(i,j\hm=\overline{1,p})$ имеют соответственно размерности
$p\times p$, $p\times r$ и $p\times p$. Они удовлетворяют следующим условиям
физической реализуемости и асимптотического затухания:
\begin{multline}
A_{ij}(t,\tau)=0;\enskip B_{i l}(t,\tau)=0;\\[1pt]
C_{ij}(t,\tau)=0\enskip \forall \tau >t;\label{e3.3-sin}
\end{multline}

\vspace*{-12pt}

\begin{equation}
\left.
\hspace*{-3mm}\begin{array}{c}
\displaystyle\iin\! \lv A_{ij} (t,\tau) \rv d\tau <\infty\,;\
\displaystyle\iin\! \lv B_{i l} (t,\tau) \rv d\tau <\infty \,;\\[9pt]
\displaystyle\iin \!\lv C_{ij} (t,\tau) \rv d\tau <\infty\,.
\end{array}\!
\right\}\!
\label{e3.4-sin}
\end{equation}

В дальнейшем ограничимся случаем, когда эредитарные ядра удовлетворяют
линейным операторным уравнениям~\cite{6-sin, 5-sin, 7-sin}.

Нелинейные в общем случае функции $\vrp\hm=\vrp(X(\tau),\tau)$,
$\psi \hm=\psi(X(\tau), \tau)$, $\chi \hm=\chi (X(\tau),  \tau, v)$
отражают нелинейные свойства ЭСтС, зависят от  $X(\tau)$ и имеют размерности
$p\times 1$, $p\times p$, $p\times 1$ соответственно.

Важный класс  эредитарных ядер представляют собой
сингулярные (вырожденные) ядра, когда имеют место представления:
\begin{equation}
\left.
\hspace*{-3mm}\begin{array}{rl}
A_{ij} (t,\tau) &= A_{ij}^+(t) A_{ij}^-(\tau)\,;\\[9pt]
B_{i l} (t,\tau)& = B_{il}^+(t) B_{il}^-(\tau)\,;\\[9pt]
C_{ij} (t,\tau) &= C_{ij}^+ ( t) C_{ij}^- (\tau)\
(i,l= \overline{1,p}, j=\overline{1,r}).
\end{array}\!
\right\}\!\!
\label{e3.5-sin}
\end{equation}

В~\cite{6-sin, 5-sin, 7-sin} показано, что для дифференцируемых нелинейных
функций~$\vrp$, $\psi$, $\chi$ путем расширения вектора состояния за счет
инструментальных переменных, аппроксимируемых линейными операторными уравнениями,
определяющими эредитарные ядра в ЭСтС, (\ref{e3.1-sin})--(\ref{e3.4-sin})
приводятся к ДСтС вида~(\ref{e2.1-sin}) или~(\ref{e2.3-sin}).
В~случае недифференцируемых нелинейных функций~$\vrp$, $\psi$, $\chi$
ЭСтС~(\ref{e3.1-sin})--(\ref{e3.4-sin}) приводятся к~(\ref{e2.1-sin}) или~(\ref{e2.3-sin})
на основе аппроксимации вырожденными (сингулярными) ядрами~\cite{6-sin, 5-sin, 7-sin}.

Таким образом, после приведения ЭСтС~(\ref{e3.1-sin}) к ДСтС~(\ref{e2.1-sin})
или~(\ref{e2.3-sin}) можно воспользоваться уравнениями МНА и МСЛ разд.~2.

\section{Типовые интегралы методов нормальной аппроксимации и~статистической
линеаризации}

Как следует из уравнений~(\ref{e2.9-sin})--(\ref{e2.11-sin}),
для МНА необходимо уметь вычислять следующие интегралы:
\begin{multline}
I_0^a = I_0^a (m_t, K_t, t) = a_1 (m_t, K_t, t)={}\\
{}= M_N a(Y_t, t)\,;
\label{e4.1-sin}
\end{multline}

\vspace*{-12pt}

\noindent
\begin{multline}
I_1^a = I_1^a (m_t, K_t, t)= a_{21}(m_t, K_t, t)= {}\\
{}=M_N a(Y_t , t) Y_t^{0\mathrm{T}}\,;\label{e4.2-sin}
\end{multline}

\vspace*{-12pt}

\noindent
\begin{multline}
I_0^{\bar \si} = I_0^{\bar \si} (m_t, K_t, t) = a_{22}(m_t, K_t, t) ={}\\
{}= M_N \bar \si (Y_t, t)\,.\label{e4.3-sin}
\end{multline}
Для МСЛ достаточно вычислить интеграл~(\ref{e4.1-sin}),
причем интеграл~$I_1^a$ вычисляется по формуле~\cite{2-sin, 3-sin, 4-sin}:
\begin{equation*}
k_1^a = k_1^a (m_t, K_t, t)=\lk \left( \fr{\prt}{\prt m_t}\right)
I_0^a (m_t, K_t, t)^{\mathrm{T}}\rk^{\mathrm{T}}. %\label{e4.4-sin}
\end{equation*}

\medskip

\noindent
\textbf{Пример 1.} В~[1] для типовых степенных, тригоно\-мет\-ри\-че\-ских,
показательных и ку\-соч\-но-по\-сто\-ян\-ных нелинейностей $Z_t \hm=\vrp (Y_t, t)$
скалярного и векторного аргумента приведены формулы для интегралов
$I_0^\vrp \hm= I_0^\vrp (m_t^y, K_t^y, t)$, а также
$k_1^\vrp \hm= k_1^\vrp (m_t^y, K_t^y, t)$.

\medskip

\noindent
\textbf{Замечание.}
 Важно иметь в виду, что уравнения МНА (МСЛ) содержат интегралы
 $I_0^a$, $I_1^a$, $I_0^\si$ в виде соответствующих коэффициентов.
 Поэтому процедура вычисления интегралов должна быть согласована с
 методом численного решения обыкновенных дифференциальных уравнений для
 $m_t$, $K_t$ и $K(t_1, t_2)$. Эти коэффициенты допускают дифференцирование
 по~$m_t$ и~$K_t$, так как под интегралом стоит сглаживающая нормальная плотность.

\section{Сложные конечные и~дифференциальные нелинейности}

Важный класс сложных конечных нелинейностей (многомерных и векторного аргумента)
представляют собой сложные функции вида:
    \begin{equation*}
    \xi =\vrp (X_t, Y_t, t)\,,\enskip X_t =\psi (Y_t, t)\,. %\label{e5.1-sin}
    \end{equation*}
В~этом случае вычисление интегралов (см.\ разд.~4) проводится по совокупности
переменных  $\lk X_t^{\mathrm{T}} Y_t^{\mathrm{T}}\rk^{\mathrm{T}}$.
К таким нелинейностям, например, относятся дроб\-но-ра\-ци\-о\-наль\-ные,
иррациональные  нелинейности, выражаемые специальными функциями, многозначные
нелинейности, зависящие от СтП~$X_t$ и его производных~$\dot X_t$,  $\ddot X_t$
и~др.

\medskip

\noindent
\textbf{Пример 2.}
Рассмотрим вычисление интегралов~(\ref{e4.1-sin}) и~(\ref{e4.2-sin})
для сложных одномерных иррациональных нелинейностей скалярного аргумента
\begin{equation}
\vrp (Y_t, t) =\lv Y_t\rrv^{\alpha-1}\, \mathrm{sgn}\, Y_t
\label{e5.2-sin}
\end{equation}
($\alpha$~--- нецелый показатель).

Пользуясь~(\ref{e2.16-sin}) и~(\ref{e2.19-sin}), представим~(\ref{e5.2-sin}) в виде
\begin{equation*}
\vrp(Y_t, t) = \vrp_0 (m_t, D_t, t) + k_1^\vrp(m_t, D_t, t) Y_t^0. %\label{e5.3-sin}
\end{equation*}
Здесь введены следующие обозначения:
\begin{gather*}
\vrp_0(m_t, D_t, t) =\Gamma(\alpha) D_t^{1/2} e^{-\xi^2/4} D_{-\alpha} (\xi)\,;%\label{e5.4-sin}
\\
k_1^a (m_t, D_t, t) =\fr {\prt \vrp_0(m_t, D_t, t)}{\prt m_t}\,,%\label{e5.5-sin}
\end{gather*}
где  $\Gamma(\alpha)$~--- гамма-функция,  $\xi \hm= m_t/\sqrt{D_t}$~---
отношение <<сиг\-нал--шум>>; $D_{-\alpha} (\xi)$~---
функция параболического цилиндра~\cite{9-sin}.
При вычислении были учтены следующие соотношения~\cite{9-sin, 8-sin}:
\begin{multline}
\iii_0^\infty x^{\alpha-1} e^{-\beta x^2 - \gamma x} \,dx ={}\\
{}=
(2\beta)^{-\alpha/2} \Gamma(\alpha) \exp \left(\fr{\gamma^2}{8\beta}\right)
D_{-\alpha} \left(\fr{\gamma}{\sqrt{2\beta}}\right)\,;\label{e5.6-sin}
\end{multline}

\vspace*{-12pt}

\noindent
\begin{multline}
\fr{dD_\rho(\xi)}{d\xi} =
   -\fr{\xi}{2}\, D_\rho (\xi) -\rho D_{\rho-1} (\xi) =
   \fr{\xi}{2}\, D_\rho (\xi) -{}\\
   {}- D_{\rho+1} (\xi) \enskip
   (\mathrm{Re}\, \beta>0\,,\enskip \mathrm{Re}\,\alpha>0\,,\enskip
   \rho=-\alpha)\,.\label{e5.7-sin}
   \end{multline}

Соотношения~(\ref{e5.6-sin}) и~(\ref{e5.7-sin})
могут быть использованы также для вычисления интегралов~(\ref{e4.3-sin}).

\medskip

\noindent
\textbf{Замечание.}
Для вычисления интегралов $I_0^a$, $I_1^a$ и $I_0^{\bar \si}$
применительно к типовым иррациональным нелинейностям вида
    $\lv Y_t\rrv^{\alp-1} e^{\delta Y_t}$, $\lv Y_t\rrv^{\alp-1}  \cos \w Y_t$,
    $\lv Y_t\rrv^{\alp-1}  \sin \w Y_t$
и более общим нелинейностям \mbox{вида}
    \begin{equation*}
    \vrp (Y_t, t) =\Phi^\vrp \left( \lv Y_t\rrv^{\alpha-1}, t\right) %\label{e5.8-sin}
    \end{equation*}
можно рекомендовать известные численные методы вычисления функций на ЭВМ~\cite{8-sin}.

\smallskip

\noindent
\textbf{Пример 3.}
Для нелинейной дроб\-но-ра\-ци\-о\-наль\-ной функции

\noindent
\begin{equation*}
\vrp (Y_t, t) = \fr{a}{(b+Y_t)^2} %\label{e5.9-sin}
\end{equation*}
имеем

\vspace*{-3pt}

\noindent
\begin{gather*}
\vrp_0 (m_t, D_t, t) =a b^{-2} \lk 1+ \chi (m_t, D_t, t)\rk\,; %\label{e5.10-sin}
\\
k_1^\vrp (m_t, D_t, t) =  a b^{-2}\fr{\prt \chi (m_t, D_t, t)}{\prt m_t}\,. %\label{e5.11-sin}
\end{gather*}
Здесь

\vspace*{-3pt}

\noindent
\begin{multline*}
\chi (m_t, D_t, t) ={}\\
{}=\sss_{n=1}^\infty \sss_{l=0}^{E(n/2)}
\fr{(-1)^n (n+1) n!}{(n-2l)! (2l)!}\, b^{-n} m_t^n \left( \fr{D_t}{ 2 m_t^2}
\right)^l, %\label{e5.12-sin}
\end{multline*}
где  $E(n/2)$~--- целая часть~$n/2$; $a\hm=a(t)$; $b\hm= b(t)$.

\vspace*{-6pt}

\section{Сложные интегральные нелинейности}

\vspace*{-2pt}

Пусть сначала векторно-матричная нелинейность имеет эредитарный характер, т.\,е.\
\begin{equation}
\underline{\vrp} (Y_t, t) =\iii_{t_0}^t A(t,\tau) \vrp (Y(\tau), \tau) \,d\tau\,.
\label{e6.1-sin}
\end{equation}
Тогда, как показано в~\cite{6-sin, 5-sin, 7-sin}, следует соответст\-ву\-ющие
интегродифференциальные соотношения путем введения  инструментальных
переменных привести к дифференциальным соотношениям.  Для
дифференцируемых функций~$\vrp$ и асимптотически устойчивых ядер
$A(t,\tau)$ зависимость~(\ref{e3.5-sin}) имеет следующий дифференциальный вид:
\begin{equation*}
F^A (t, D) \underline{\vrp} (Y_t, t) = H^A (t, D) \vrp (Y_t, t)\,. %\label{e6.2-sin}
\end{equation*}
Здесь $F^A (t, D)$ и  $H^A (t, D)$~--- линейные дифференциальные операторы $(D\hm= d/dt)$.

Для недифференцируемых функций~$\vrp$ и асимптотически устойчивых
сингулярных ядер~(\ref{e3.5-sin}) используются соотношения:
\begin{equation*}
\underline{\vrp} (Y_t, t) = A^+ Z\,,\enskip
\dot Z = A^- \vrp\,,\enskip
Z(t_0)=0\,. %\label{e6.3-sin}
\end{equation*}

Многочисленные примеры аналитического моделирования ЭСтС можно найти
в~[1--3, 5, 7, 10, 11].

Как отмечалось в~\cite{3-sin}, часто наряду с интегральными
нелинейностями~(\ref{e6.1-sin}) рассматривают нелинейности вида:

\columnbreak

\noindent
\begin{equation*}
Z_s =\sss_{\rho=1}^R \mathcal{A}_\rho \vrp_\rho (Y_{t_1}\tr Y_{t_r})\,, %\label{e6.2-sin}
\end{equation*}
где $\mathcal{A}_1 \tr \mathcal{A}_R$~--- произвольные линейные операторы,
действующие над функциями~$r$ переменных  $t_1\tr t_r$; $\vrp_\rho
\hm=\vrp_\rho (Y_{t_1} \tr Y_{t_r})$~--- линейные функции отмеченных
переменных. Такие нелинейности называются приводимыми к линейным.
Важным частным случаем~(\ref{e6.1-sin}) являются интегральные нелинейности вида:

\noindent
\begin{gather}
Z_s =\iii_T \vrp (Y_t, t, s)\, dt\,; \notag%\label{e6.3-sin}
\\
Z_s =\!\iii_T \!\cdots\!\iii_T\! \vrp (Y_{t_1}\tr Y_{t_r}; t_1\tr t_r, s)\,dt_1
\ldots dt_r,\notag %\label{e6.4-sin}
\end{gather}
В этом случае используется МСЛ по совокупности переменных  $Y_{t_1} \tr Y_{t_r}$.

\vspace*{-9pt}

\section{Заключение}

\vspace*{-2pt}

Разработаны методы и алгоритмы МНА и МСЛ для ДСтС и ЭСтС,
приводимых к ДСтС со сложными конечными, дроб\-но-ра\-ци\-о\-наль\-ны\-ми,
иррациональными, а также дифференциальными и интегральными нелинейностями.
Приведены примеры.

Результаты допускают обобщение на случай ДСтС и ЭСтС со
стохастическими нелинейностями, заданными каноническими разложениями и
интегральными каноническими  представлениями~\cite{1-sin, 3-sin, 11-sin}.

\vspace*{-9pt}

{\small\frenchspacing
 {%\baselineskip=10.8pt
 \addcontentsline{toc}{section}{References}
 \begin{thebibliography}{99}

 \vspace*{-2pt}

\bibitem{1-sin}
\Au{Синицын И.\,Н.,  Синицын~В.\,И.}
Лекции по нормальной и эллипсоидальной аппроксимации распределений в
стохастических сис\-те\-мах.~--- М.: ТОРУС ПРЕСС, 2013. 488~с.

\bibitem{6-sin} %2
\Au{Синицын И.\,Н. }
Stochastic hereditary control systems~// Проблемы управления и
теории информации, 1986. Т.~15. №\,4. С.~287--298.

\bibitem{2-sin} %3
\Au{Пугачев В.\,С., Синицын~И.\,Н.}
Стохастические дифференциальные сис\-те\-мы. Анализ и фильтрация.~--- М.:
Наука,  1990.  632~с. [Англ. пер.
 Stochastic differential systems.
Analysis and filtering.~--- Chichester, New York: Jonh Wiley, 1987.
549~p.].

\bibitem{5-sin} %4
\Au{Синицын И.\,Н. }
Конечномерные распределения процессов в стохастических интегральных
и интегродифференциальных системах~// Preprints of the 2nd IFAC
Symposium on Stochastic Control.~--- Vilnius: Pergamon Press,
1987.  Vol.~1. P.~144--153.

\bibitem{3-sin} %5
\Au{Пугачев В.\,С., Синицын~И.\,Н.}
Теория стохастических систем.~--- М.: Логос, 2000; 2004. 1000~с.
[Англ. пер.\linebreak\vspace*{-12pt}

\pagebreak

\noindent Stochastic systems. Theory and  applications.~---
Singapore: World Scientific, 2001. 908~p.].

\bibitem{4-sin} %6
\Au{Синицын И.\,Н.}
Параметрическое статистическое и аналитическое моделирование распределений
в нелинейных стохастических сис\-те\-мах на многообразиях~//
Информатика и её применения, 2013. Т.~7. Вып.~2. С.~4--16.

\bibitem{7-sin} %7
\Au{Синицын И.\,Н. }
Анализ и моделирование распределений в эредитарных стохастических
сис\-те\-мах~// Информатика и её применения, 2014. Т.~8. Вып.~1.\linebreak
С.~2--11.



\bibitem{9-sin} %8
\Au{Градштейн И.\,С., Рыжик~И.\,М.}
Таблицы интегралов, сумм, рядов и произведений.~--- М.: ГИФМЛ, 1963. 1100~с.

\bibitem{8-sin} %9
\Au{Попов Б.\,А., Теслер~Г.\,С. }
Вычисление функций на ЭВМ: Справочник.~--- Киев: Наукова Думка, 1984. 599~с.


\bibitem{11-sin} %10
\Au{Синицын И.\,Н.}
Канонические представления случайных функций и их применение в
задачах компьютерной поддержки научных исследований.~--- М.: ТОРУС
ПРЕСС, 2009. 768~с.

\bibitem{10-sin} %11
\Au{Синицын И.\,Н., Синицын~В.\,И., Корепанов~Э.\,Р., Белоусов~В.\,В.,
Сергеев~И.\,В., Басилашвили~Д.\,А.}
Опыт моделирования эредитарных стохастических сис\-тем~//
Кибернетика и высокие технологии XXI века: Сб. докл.  XIII Междунар.
науч.-технич. конф.~--- Воронеж: Саквоее, 2012. Т.~2. C.~346--357.

 \end{thebibliography}

 }
 }

\end{multicols}

\vspace*{-9pt}

\hfill{\small\textit{Поступила в редакцию 05.05.14}}

%\newpage

\vspace*{12pt}

\hrule

\vspace*{2pt}

\hrule

\vspace*{12pt}

\def\tit{ANALYTICAL MODELING OF NORMAL PROCESSES
 IN~STOCHASTIC SYSTEMS WITH~COMPLEX NONLINEARITIES}

\def\titkol{Analytical modeling of normal processes
 in~stochastic systems with~complex nonlinearities}

\def\aut{I.\,N.~Sinitsyn and V.\,I.~Sinitsyn}

\def\autkol{I.\,N.~Sinitsyn and V.\,I.~Sinitsyn}

\titel{\tit}{\aut}{\autkol}{\titkol}

\vspace*{-9pt}

\noindent
Institute of Informatics Problems, Russian Academy of Sciences,
44-2 Vavilov Str., Moscow 119333, Russian Federation


\def\leftfootline{\small{\textbf{\thepage}
\hfill INFORMATIKA I EE PRIMENENIYA~--- INFORMATICS AND
APPLICATIONS\ \ \ 2014\ \ \ volume~8\ \ \ issue\ 3}
}%
 \def\rightfootline{\small{INFORMATIKA I EE PRIMENENIYA~---
INFORMATICS AND APPLICATIONS\ \ \ 2014\ \ \ volume~8\ \ \ issue\ 3
\hfill \textbf{\thepage}}}

\vspace*{6pt}

\Abste{Differential stochastic systems (DStS) with Wiener and Poisson
noises and complex finite, differential, and  integral nonlinearities and
hereditary StS reducible to DStS are considered. Equations and algorithms
of analytical modeling based on the normal approximation method (NAM) and the
statistical linearization method (SLM) are given. The case of complex
continuous and discontinuous nonlinearities of scalar and vector arguments
is considered. Special attention is paid to NAM (SLM) typical integrals
for finite rational and irrational nonlinear and integral scalar and vector
nonlinear functions. The general case of integral nonlinearities reducible to
linear is considered. Test examples are given.}

\KWE{analytical modeling;
complex finite differential and integral nonlinearities;
complex irrational nonlinerarites
differential stochastic system with Wiener and Poisson noises;
method of normal approximation;
method of statistical linearization;
hereditary stochastic systems reducible to differential}

\DOI{10.14357/19922264140302}

  \begin{multicols}{2}

\renewcommand{\bibname}{\protect\rmfamily References}
%\renewcommand{\bibname}{\large\protect\rm References}

{\small\frenchspacing
 {%\baselineskip=10.8pt
 \addcontentsline{toc}{section}{References}
 \begin{thebibliography}{99}



\bibitem{1-sin-1}
\Aue{Sinitsyn, I.\,N., and  V.\,I.~Sinitsyn}.  2013.
Lektsii po normal'noy i ellipsoidal'noy approksimatsii raspredeleniy
v stokhasticheskikh sistemakh [Lectures on normal and ellipsoidal
approximation of distributions in stochastic systems].
Moscow: TORUS PRESS. 488~p.

\bibitem{6-sin-1} %2
\Aue{Sinitsyn, I.\,N.}  1986.
{Stochastic hereditary control systems}.
\textit{Problems Control Inform. Theory} 15(4):287--298.

\bibitem{2-sin-1} %3
\Aue{Pugachev, V.\,S., and  I.\,N.~Sinitsyn}.  1987.
\textit{Stochastic differential systems. Analysis and filtering.}
Chichester, New York: Jonh Wiley. 549~p.

\bibitem{5-sin-1} %4
\Aue{Sinitsyn, I.\,N.}  1987.
Konechnomernye raspredeleniya protsessov v stokhasticheskikh integral'nykh
i in\-teg\-ro\-dif\-fe\-ren\-tsial'nykh sistemakh [Finite dimensional distributions
of processes in stochastic integral and integrodifferential systems].
\textit{2nd  Symposium (International) IFAC on Stochastic Control
Preprints}. Vilnius: Pergamon Press. 1:144--153.

\bibitem{3-sin-1} %5
\Aue{Pugachev, V.\,S., and I.\,N.~Sinitsyn}. 2001.
\textit{Stochastic systems. Theory and  applications}.
Singapore: World Scientific. 908~p.

\bibitem{4-sin-1} %6
\Aue{Sinitsyn, I.\,N.}  2013.
Parametricheskoe statisticheskoe i analiticheskoe modelirovanie
raspredeleniy v nelineynykh stokhasticheskikh sistemakh na mnogoobraziyakh
[Parametric statistical and analytical modeling of distributions in
stochastic systems on manifolds].
\textit{Informatika i ee Primeneniya}~--- \textit{Inform. Appl.} 7(2):4--16.


\bibitem{7-sin-1} %7
\Aue{Sinitsyn, I.\,N.}  2014.
Analiz i modelirovanie raspredeleniy v ereditarnykh stokhasticheskikh sistemakh
[Analysis and modeling of distributions in hereditary stochastic systems].
\textit{Informatika i ee Primeneniya}~--- \textit{Inform. Appl.} 8(1):2--11.

\bibitem{9-sin-1} %8
\Aue{Gradshteyn, I.\,S., and I.\,M.~Ryzhik}.  1963.
\textit{Tablitsy integralov, summ, ryadov i proizvedeniy}
[Tables of integrals, sums, series, and products]. Moscow:  GIFML.   1100~p.

\pagebreak

\bibitem{8-sin-1} %9
\Aue{Popov, B.\,A., and G.\,S.~Tesler}.  1984.
\textit{Vychislenie funktsiy na EVM}. Spravochnik [Computing of functions].
Kiev: Naukova Dumka.  599~p.


\bibitem{11-sin-1} %10
\Au{Sinitsyn, I.\,N.} 2009.
\textit{Kanonicheskie predstavleniya sluchaynykh funktsiy i ikh primenenie v
zadachakh komp'yuternoy podderzhki nauchnykh issledovaniy}
[Canonical expansions of random functions and its application to
scientific computer-aided support]. Moscow: TORUS PRESS. 768~p.

\bibitem{10-sin-1} %11
\Aue{Sinitsyn, I.\,N., V.\,I.~Sinitsyn, E.\,R.~Korepanov,
V.\,V.~Belousov, I.\,V.~Sergeev, and D.\,A.~Basilashvili}.
2012. Opyt modelirovaniya ereditarnykh stokhasticheskikh sistem
[Experience of modeling in hereditary stochastic systems].
\textit{Kibernetika i Vysokie Tekhnologii XXI~Veka:
Sbornik dokladov  XIII Mezhdunar. nauch.-tekhnich. konf.}
[Cybernatics ans High Technologies of the XXI Century: Materials of
XIII  Scientific and Technological Conference (International)].
Voronezh: Sakvoee. 2:346--357.

\end{thebibliography}

 }
 }

\end{multicols}

\vspace*{-6pt}

\hfill{\small\textit{Received May 05, 2014}}

\vspace*{-18pt}

\Contr

\noindent
\textbf{Sinitsyn Igor N.} (b.\ 1940)~---
Doctor of Science in technology, professor, Honored scientist of RF, Head of Department, Institute of
Informatics Problems, Russian Academy of Sciences,
44-2 Vavilov Str., Moscow 119333, Russian
Federation; sinitsin@dol.ru

\vspace*{3pt}

\noindent
\textbf{Sinitsyn Vladimir I.} (b.\ 1968)~--- Doctor of Science in physics
and mathematics, associate professor, Head of Department, Institute of
Information Problems, Russian Academy of Sciences,
44-2 Vavilov Str., Moscow 119333, Russian Federation; VSinitsin@ipiran.ru




\label{end\stat}

\renewcommand{\bibname}{\protect\rm Литература} %1

\include{zahar}  %2

\include{pech}  %3

\def\stat{bosov+stef}

\def\tit{УПРАВЛЕНИЕ ВЫХОДОМ СТОХАСТИЧЕСКОЙ ДИФФЕРЕНЦИАЛЬНОЙ СИСТЕМЫ 
ПО~КВАДРАТИЧНОМУ КРИТЕРИЮ. I.~ОПТИМАЛЬНОЕ РЕШЕНИЕ МЕТОДОМ 
ДИНАМИЧЕСКОГО ПРОГРАММИРОВАНИЯ$^*$}

\def\titkol{Управление выходом стохастической дифференциальной системы 
по~квадратичному критерию. I}
%.~Оптимальное решение методом 
%динамического программирования}

\def\aut{А.\,В.~Босов$^1$, А.\,И.~Стефанович$^2$}

\def\autkol{А.\,В.~Босов, А.\,И.~Стефанович}

\titel{\tit}{\aut}{\autkol}{\titkol}

\index{Босов А.\,В.}
\index{Стефанович А.\,И.}
\index{Bosov A.\,V.}
\index{Stefanovich A.\,I.}




{\renewcommand{\thefootnote}{\fnsymbol{footnote}} \footnotetext[1]
{Работа выполнена при частичной поддержке РФФИ (проект 16-07-00677).}}


\renewcommand{\thefootnote}{\arabic{footnote}}
\footnotetext[1]{Институт проблем информатики Федерального исследовательского центра <<Информатика 
и~управление>> Российской академии наук, \mbox{AVBosov@ipiran.ru}}
\footnotetext[2]{Институт проблем информатики Федерального исследовательского центра <<Информатика 
и~управление>> Российской академии наук, \mbox{AStefanovich@frccsc.ru}}

%\vspace*{8pt}



  
  \Abst{Решается задача оптимального управления для диффузионного процесса 
Ито и~линейного управ\-ля\-емо\-го выхода. Рассматриваемая постановка близка 
к~классической ли\-ней\-но-квад\-ра\-тич\-ной гауссовской задаче управления 
(linear-quadratic Gaussian (LQG) control). Отличия состоят в~том, что состояние описывается нелинейным 
дифференциальным уравнение Ито $dy_t\hm= A_t(y_t) \,dt\hm+ \Sigma_t(y_t)\,dv_t$ 
и~не зависит от управ\-ле\-ния~$u_t$, оптимизации подлежит управ\-ля\-емый 
линейный выход $dz_t\hm= a_t y_t\,dt\hm+ b_t z_t \,dt\hm+ c_t u_t \,dt\hm+ \sigma_t\, 
dw_t$. Дополнительные обобщения внесены в~квад\-ра\-тич\-ный критерий качества 
с~целью воз\-мож\-ности постановки таких задач, как отслеживание выходом 
состояния или управ\-ле\-ни\-ем~--- линейной комбинации состояния и~выхода. Для 
решения используется метод динамического программирования. Функцию 
Беллмана позволяет найти предположение о~ее структуре вида $V_t(y,z)\hm= 
\alpha_t z^2\hm+ \beta_t(y)z \hm+\gamma_t(y)$. Решение дают три 
дифференциальных уравнения для коэффициентов~$\alpha_t$, $\beta_t(y)$ 
и~$\gamma_t(y)$. Эти уравнения со\-став\-ля\-ют оптимальное решение 
рас\-смат\-ри\-ва\-емой задачи.}
  
  \KW{стохастическое дифференциальное уравнение; оптимальное управ\-ле\-ние; 
динамическое программирование; функция Беллмана; уравнение Риккати; 
линейные уравнения параболического типа}

\DOI{10.14357/19922264180314}
  
%\vspace*{4pt}


\vskip 10pt plus 9pt minus 6pt

\thispagestyle{headings}

\begin{multicols}{2}

\label{st\stat}

\section{Введение}

     Ключевые результаты в~области оптимизации стохастических 
динамических систем, со\-став\-ля\-ющие классическую теорию управления, 
получены более~40~лет назад (такова работа~[1] в~отношении задачи 
управ\-ле\-ния ли\-ней\-но-гаус\-сов\-ски\-ми стохастическими сис\-те\-ма\-ми по 
квад\-ра\-тич\-но\-му критерию). К~классической тео\-рии следует относить 
линейные модели стохастических сис\-тем и~квадратичный критерий качества. 
Это исходный базис, на котором основано множество успешно 
исследованных и~решенных задач стохастического управ\-ле\-ния 
и~оптимизации. 

Дальнейшее развитие~--- это новые модели и~критерии, но 
прежде всего это новые методы: от тео\-рии линейных регуляторов, метода 
динамического программирования и~принципа максимума к~адаптивному 
и~минимаксному подходу, импульсному управ\-ле\-нию и~т.\,д. Множество 
инноваций как в~час\-ти моделей, так и~в~час\-ти математического аппарата, 
имевших мес\-то в~по\-сле\-ду\-ющие годы, существенно обогатили тео\-рию 
управ\-ле\-ния. Но и~до настоящего времени линейные модели и~квадратичный 
критерий, несмотря на всю справедливую критику в~отношении их 
аде\-кват\-ности и~гиб\-кости, сохраняют исследовательский интерес и~находят 
современные области приложения.
     
     Не претендуя на сколь\-ко-ни\-будь полное обосно\-ва\-ние последнего 
тезиса, приведем несколько примеров, показавшихся наиболее ин\-те\-рес\-ными. 

Так, в~[2] решается ли\-ней\-но-квад\-ра\-тич\-ная за\-да\-ча в~игровой 
постановке с~запаздыванием. В~близ\-кой по модели работе~[3] задача 
управ\-ле\-ния ставится в~терминах $H_\infty$-ро\-баст\-ности. Точнее \mbox{называть} 
эту тематику $H_2/H_\infty$-управ\-ле\-ни\-ем, и~работ по этой теме очень 
много. Аккуратности ради следует уточнить, что под линейными 
понимаются модели с~мультипликативными по состоянию воз\-му\-ще\-ниями. 

Совсем другой класс моделей, особо популярных в~по\-след\-ние годы, 
составляют скачкообразные процессы. Например, линейные уравнения 
в~сочетании с~пуассоновскими скачками в~[4] используются в~моделях, 
описывающих различные показатели функционирования сетевых протоколов 
передачи данных транспортного уровня. Телекоммуникации представляют 
в~последние годы самый популярный прикладной материал для 
исследований, работ по этой проб\-ле\-ма\-ти\-ке множество, математические 
техники привлекаются самые разные и~самые современные, но и~линейным 
моделям место находится. Еще один любопытный пример исследования 
скачкообразного процесса и~оптимизации на основе квад\-ра\-тич\-но\-го критерия 
можно найти в~[5] применительно к~задаче инвестирования на финансовом 
рынке. Наконец, упомянем еще работу~[6], подводящую итог исследований 
в~отношении классической детерминированной  
ли\-ней\-но-квад\-ра\-тич\-ной задачи с~использованием техники матричных 
неравенств.
     
     В данной работе также эксплуатируются привлекательные свойства 
линейных моделей и~квад\-ра\-тич\-но\-го критерия, причем в~стохастической 
постановке. На\-прав\-ле\-ни\-ем для обобщения \mbox{выбрана} модель динамики 
сис\-те\-мы: основные усилия на\-прав\-ле\-ны на то, чтобы сделать ее нелинейной. 
Кроме того, пред\-став\-лен\-ная постановка может рас\-смат\-ри\-вать\-ся и~как 
обобщение ранее решенной задачи в~дискретном времени~[7, 8] на время 
непрерывное. В~упомянутых работах помимо собственно модельной 
постановки важна еще и~привлекаемая прикладная об\-ласть~--- 
функционирование сложных программных сис\-тем. Результатов, 
ориентированных непосредственно на такие приложения, к~настоящему 
времени пренебрежимо мало, поэтому~[7, 8]~--- это еще и~прикладное 
обоснование рас\-смат\-ри\-ва\-емой далее задачи.
     
     Оптимизируемая динамическая сис\-те\-ма описывается двумя 
уравнениями. Состояние задается нелинейным стохастическим 
дифференциальным уравнением Ито, не содержащим управ\-ля\-емой 
переменной. Возмущение здесь описывается стандартным винеровским 
процессом, накладываются простые условия существования 
и~един\-ст\-вен\-ности решения. Поскольку состояние не управ\-ля\-ет\-ся, то уместно 
его интерпретировать как слож\-ное внешнее возмущение. Вторая 
переменная~--- управ\-ля\-емый выход~--- задается линейным стохастическим 
дифференциальным уравнением. Цель оптимизации выхода формируется 
квадратичным критерием общего вида. Формальная постановка задачи 
приведена в~сле\-ду\-ющем разделе.
     
     Для решения задачи используется метод динамического 
программирования, решается уравнение Беллмана~[9]. Соответственно, 
в~результате получаются аналитические выражения и~для оптимального 
управ\-ле\-ния, и~для значения функционала качества. Технически 
традиционный, стандартный подход к~задаче обременен, пожалуй, 
единственной проблемой~--- поиском верного пред\-став\-ле\-ния структуры 
функции Беллмана. Справиться с~этой проблемой в~большей степени удается 
за счет результата, полученного при решении дискретного по времени 
аналога рассматриваемой постановки~\cite{8-bos}. Конечные соотношения 
для оптимального решения, как и~во всех подобных задачах, включая 
классическую ли\-ней\-но-квад\-ра\-тич\-ную, содержат решения 
определенных дифференциальных уравнений (обыкновенных и~в~частных 
производных). Вывод этих уравнений и~со\-став\-ля\-ет содержание первой час\-ти 
данной работы. Во второй части будет обсуждаться их приближенное 
чис\-лен\-ное решение и~компьютерные эксперименты.
     
     Кратко обозначим основные положения, при\-вле\-ка\-емые далее 
к~решению задачи, следуя в~основном обозначениям 
и~терминологии~\cite{9-bos}, а~именно: будем рассматривать задачу 
оптимального управления в~стохастической динамической сис\-те\-ме по полной 
информации, применяя метод динамического программирования. В~качестве 
целевого функционала, опре\-де\-ля\-юще\-го качество управ\-ле\-ния $U_0^T\hm= \{ 
u_t,\ 0\leq t\leq T\}$, выступает
     \begin{equation}
     J\left(U_0^T\right)={\sf E}\left\{ \int\limits_0^T L_t \left(x_t, u_t\right)\,dt+ 
l\left(x_T\right)\right\}\,.
     \label{e1-bos}
     \end{equation}
Здесь ${\sf E}\{\cdot\}$~--- оператор математического ожидания; $x_t$~--- 
случайный процесс, описываемый стохастическим дифференциальным 
уравнением Ито
     \begin{equation}
     dx_t=m_t\left( x_t, u_t\right) dt+ \sigma_t\left( x_t\right)dW_t\,,\enskip 
x_0=X\,,
     \label{e2-bos}
     \end{equation}
где $W_t$~--- стандартный винеровский процесс подходящей раз\-мер\-ности; 
$X$~--- случайный вектор.

     $U_0^T$ будем выбирать из класса допустимых неупреждающих (по 
отношению к~$W_t$) управлений~\cite{9-bos}. Соответственно, 
относительно функций сноса и~диффузии~$m_t$ и~$\sigma_t$  
в~(\ref{e2-bos}) будем предполагать выполненными ка\-кие-ли\-бо условия 
существования сильного решения для заданного до\-пус\-ти\-мо\-го управ\-ле\-ния. 
Например, для управ\-ле\-ния с~обратной связью $u_t\hm= u_t(x_t)$ будем 
считать, что $m_t(x,u_t(x))$ и~$\sigma_t(x)$ удовлетворяют условию 
линейного рос\-та и~локальному условию Липшица по~$x$ равномерно 
по~$t$ (т.\,е.\ условиям Ито).
     
     Для поиска оптимального управления, минимизирующего $J(U_0^T)$, 
рас\-смат\-ри\-ва\-ет\-ся функция Беллмана
     \begin{equation}
     V_t(x)=\left.\mathop{\mathrm{inf}}\limits_{U_t^T} {\sf E} \left\{ \int\limits_t^T 
L_t \left( x_t, u_t\right)\,dt+l\left( x_T\right) \right\vert \mathcal{F}_t^x\right\}\,,
     \label{e3-bos}
     \end{equation}
где $\mathcal{F}_t^x$~--- $\sigma$-ал\-геб\-ра, по\-рож\-ден\-ная~$x_\tau$, 
$0\hm\leq \tau\hm\leq t$, ${\sf E}\{\cdot\vert \mathcal{F}\}$~--- оператор условного 
математического ожидания относительно~$\mathcal{F}$. Соответственно, 
в~качестве достаточного условия оп\-ти\-маль\-ности воспользуемся уравнением 
динамического программирования
\begin{multline}
\fr{\partial V_t(x)}{\partial t} +\fr{1}{2}\sum\limits^n_{i,j=1} \sigma^2_{t_{ij}}
\fr{\partial^2 V_t(x)}{\partial x_i \partial x_j}+{}\\
{}+\min\limits_u\left[  
\sum\limits^n_{i=1} m_{t_i} \fr{\partial V_t(x)}{\partial x_i} + L_t(x,u)\right] 
=0\,,\\
V_T(x)=l(x)\,,
\label{e4-bos}
\end{multline}
где $m_{t_i}$~--- $i$-й элемент век\-тор-функ\-ции~$m_t(x,u)$; 
$\sigma^2_{t_{ij}} \hm= \sum\nolimits^m_{k=1} 
\sigma_{t_{ik}}\sigma_{t_{ki}}$, $\sigma_{t_{ij}}$~--- $i$-й по строке, $j$-й 
по столб\-цу элемент мат\-рич\-ной функции~$\sigma_t(x)$; $n$ и~$m$~--- 
размерности~$x_t$ и~$W_t$ соответственно.

     Традиционно в~рамках применения метода динамического 
программирования будем предполагать, что функции~$L_t$, $l$, $m_t$ 
и~$\sigma_t$ обеспечивают существование хотя бы одного решения 
уравнения~(\ref{e4-bos}), а~следовательно, и~оптимального 
управления~$u_t^*$, $0\hm\leq t\hm\leq T$, до\-став\-ля\-юще\-го минимум 
целевому функционалу~(\ref{e1-bos}). Задача оптимизации далее получается 
путем указания конкретных выражений для~$L_t$, $l$, $m_t$ и~$\sigma_t$.

\section{Постановка задачи управления выходом}

     Рассматриваемые далее случайные функции будут предполагаться 
скалярными. Такое упрощение позволит разгрузить выкладки и~итоговые 
выражения от не самых существенных деталей.
     
     Рассмотрим стохастическую дифференциальную сис\-те\-му, со\-сто\-яние 
которой представляет диффузи\-он\-ный процесс~$y_t$, описываемый 
нелинейным стохастическим дифференциальным уравнением Ито
     \begin{equation}
     dy_t=A_t\left( y_t\right) dt +\Sigma_t \left( y_t\right) dv_t\,,\enskip 
y_0=Y\,,
     \label{e5-bos}
     \end{equation}
где $v_t$~--- стандартный (одномерный) винеровский процесс; $Y$~--- 
случайная величина с~конечным вторым моментом; функции~$A_t$ 
и~$\Sigma_t$ удовлетворяют условиям Ито:
\begin{equation*}
\left\vert A_t(y)\right\vert +\left\vert \Sigma_t(y)\right\vert \leq C(1+\vert y\vert )\ 
\mbox{для\ всех } 0\leq t\leq T\,;
\end{equation*}

\vspace*{-12pt}

\noindent
\begin{multline*}
\hspace*{-2.10051pt}\left\vert A_t\left(y_1\right) -A_t \left( y_2\right) \right\vert +\left\vert 
\Sigma_t\left( y_1\right) -\Sigma_t \left(y_2\right)\right\vert \leq
C\left\vert y_1-y_2\right\vert\\
 \mbox{для\ всех\ } 0\leq t\leq T\ \mbox{и } 
y_1,y_2\in \mathbb{R}^1\,,
\end{multline*}
обеспечивающим существование единственного сильного (потраекторного) 
решения уравнения.
     
     Будем считать, что~$y_t$ описывает состояние некоторой 
динамической системы. Соответственно, поведение этой сис\-те\-мы опишем 
выходом, линейно связанным с~со\-сто\-янием:
     \begin{equation}
     dz_t=a_t y_t \,dt+ b_t z_t \,dt+ c_t u_t \,dt+\sigma_t \,dw_t\,,\enskip
     z_0=Z\,.
     \label{e6-bos}
     \end{equation}
Здесь $w_t$~--- не зависящий от~$v_t$, $Y$ и~$Z$ стандартный (одномерный) 
винеровский процесс; $Z$~--- случайная величина с~конечным вторым 
моментом; $u_t$~--- допустимое неупреждающее управ\-ле\-ние, качество 
которого определяется целевым функционалом следующего вида:
\begin{multline}
\!\hspace*{-3.98538pt}J\left( U_0^T\right) ={\sf E}\left\{ \int\limits_0^T \!\left( S_t\left( s_ty_t-g_t z_t -h_t 
u_t\right)^2 +G_t z_t^2+{}\right.\right.\\
\left.\left.{}+ H_t u_t^2
\vphantom{S_t\left( s_ty_t-g_t z_t -h_t 
u_t\right)^2}
\right) dt+S_T\left( s_T y_T -g_T 
z_T\right)^2+G_T z_T^2
\vphantom{\int\limits_0^T}\right\}\,,
\label{e7-bos}
\end{multline}
где $S_t$, $G_t$ и~$H_t$~--- неотрицательные функции\linebreak
$0\hm\leq t\hm\leq T$. 
Такой критерий отражает физический смысл задачи распределения ресурсов 
со\-глас\-но аналогичной~(\ref{e5-bos})--(\ref{e7-bos}) задаче для дис\-крет\-но\-го 
времени, рас\-смот\-рен\-ной в~\cite{7-bos}. В~част\-ности,  
функци\-онал~(\ref{e7-bos}) поз\-во\-ля\-ет ставить задачи отслеживания
 выходом 
со\-сто\-яния сис\-те\-мы, используя сла\-га\-емое $(y_t\hm- z_t)^2$, или 
управлением~--- линейной комбинации со\-сто\-яния и~выхода, сла\-га\-емое типа\linebreak 
$(y_t\hm+ z_t\hm- u_t)^2$. Поскольку задача формулируется 
в~предположении наличия пол\-ной информации о~со\-сто\-янии~$y_t$ 
и~выходе~$z_t$ (соответствующую $\sigma$-ал\-геб\-ру 
обозначим~$\mathcal{F}_t^{y,z}$), то допустимое управ\-ле\-ние ищется 
в~классе~$\mathcal{F}_t^{y,z}$-из\-ме\-ри\-мых неупреждающих функций 
(и,~как будет показано далее, оказывается управ\-ле\-ни\-ем с~обратной связью).

     Функции~$a_t$, $b_t$, $c_t$ и~$\sigma_t$ будем предполагать 
ограниченными: $\vert a_t\vert \hm+ \vert b_t\vert \hm+\vert c_t\vert \hm+ \vert 
\sigma_t \vert \hm\leq C$ для всех $0\hm\leq t\hm\leq T$, процесс  
управления~--- допустимым не\-упреж\-да\-ющим~\cite{9-bos}, обеспечивая, 
таким образом, существование сильного решения урав\-не\-ния~(\ref{e6-bos}) 
для любого допустимого управ\-ления.
     
     Задачу составляет поиск~$u_t^*$~--- допустимого управ\-ле\-ния, 
доставляющего минимум квад\-ра\-тич\-но\-му функционалу~$J(U_0^T)$.
      
     Поставленная задача очевидным образом формулируется в~терминах 
введенных выше в~(\ref{e1-bos})--(\ref{e3-bos}) обозначений, а~именно: 
     требуется обозначить
     \begin{gather*}
      x_t=\begin{pmatrix}
     y_t\\ z_t\end{pmatrix};\quad  m_t(x_t, u_t)=\begin{pmatrix}
     A_t(y_t)\\ a_t y_t +b_t z_t +c_t u_t\end{pmatrix};\\
     \sigma_t(x_t)= \begin{pmatrix}
     \Sigma_t(y_t)& 0\\
     0& \sigma_t\end{pmatrix};\quad W_t=\begin{pmatrix}
     v_t \\ w_t\end{pmatrix}
     %     \label{e8-bos}
     \end{gather*}
для записи уравнения со\-сто\-яния типа~(\ref{e2-bos}) и
\begin{align*}
L_t(x,u)&= L_t(y,z,u) ={}\\
&\hspace*{3mm}{}=S_t\left( s_t y-g_t z -h_t u\right)^2 +G_t z^2 +H_t  u^2\,;\\
l(x)&= l(y,z) =S_T \left( S_T y-g_T z\right)^2 +G_T z^2
%\label{e9-bos}
\end{align*}
для записи целевого функционала в~виде~(\ref{e1-bos}).

     Функция Беллмана~(\ref{e3-bos}) принимает вид 
     $V_t(x)\hm= V_t(y,z)$. Для записи со\-от\-вет\-ст\-ву\-юще\-го~(\ref{e4-bos}) 
уравнения Беллмана для~$V_t(y,z)$ заметим, что
     $$
     \left( \sigma^2_{t_{ij}}\right)_{i,j=1,2}= \begin{pmatrix}
     \Sigma_t^2(y) & 0\\
     0 & \sigma_t^2\end{pmatrix}\,.
     $$
     
     С~учетом перечисленных обозначений урав\-не\-ние динамического 
программирования~(\ref{e4-bos}) принимает вид:
     \begin{multline}
     \fr{\partial V_t(y,z)}{\partial t} +\fr{1}{2}\left( \Sigma_t^2(y) \fr{\partial^2 
V_t(y,z)} {\partial y^2}+\sigma_t^2\fr{\partial^2 V_t(y,z)} {\partial 
z^2}\right)+{}\\
    {}+\min\limits_u\! \left[ A_t(y) \fr{\partial V_t(y,z)}{\partial y}+\left( a_t 
y+b_t z+c_t u\right) \fr{\partial V_t(y,z)}{\partial z} +{}\right.\hspace*{-3pt}\\
\left.{}+ S_t\left( s_t y-g_t z-h_t 
u\right)^2+G_t z^2+H_t u^2
     \vphantom{\fr{\partial V_t(y,z)}{\partial y}}\right] =0\,,\\
     V_T(y,z)=S_T\left( s_T y-g_T z\right)^2+G_T z^2\,.
     \label{e10-bos}
     \end{multline}
     Это и~есть то самое уравнение, которое требуется решить: 
существование решения данного урав\-не\-ния суть достаточное условие 
оптимальности; оптимальное управ\-ле\-ние при этом~--- точ\-ка минимума 
со\-от\-вет\-ст\-ву\-юще\-го сла\-га\-емого.
     
\section{Динамическое программирование и~оптимальное 
управление}

     В рассматриваемой постановке линейность\linebreak выхода и~квадратичность 
критерия дают те же преимущества, что и~в~классической  
ли\-ней\-но-квад\-ра\-тич\-ной задаче управ\-ле\-ния~\cite{1-bos}, а~именно: 
позволяют сразу определить вид оптимального управ\-ле\-ния и~фактические 
условия его существования. Действительно, со\-хра\-няя в~(\ref{e10-bos}) под 
знаком $\min\nolimits_u$ только члены, зависящие от~$u$, получаем
     \begin{multline*}
     \fr{\partial V_t(y,z)}{\partial t} +\fr{1}{2}\left( \Sigma_t^2(y) \fr{\partial^2 
V_t(y,z)} {\partial y^2}+\sigma_t^2\fr{\partial^2 V_t(y,z)} {\partial 
z^2}\right)+{}\\
     {}+A_t(y)\fr{\partial V_t(y,z)}{\partial y}+\left( a_t y+b_t z\right) 
\fr{\partial V_t(y,z)}{\partial z}+{}\\
{}+S_t\left( s_t y-g_t z\right)^2 +G_t z^2+{}
\end{multline*}

\noindent
\begin{multline*}
     {}+\min\limits_u \left[ \left( c_t \fr{\partial V_t(y,z)}{\partial z}-2S_t \left( 
s_t y-g_t z\right) h_t\right)u +{}\right.\\
\left.{}+\left( S_t h_t^2+H_t\right) u^2
\vphantom{\fr{\partial V_t(y,z)}{\partial z}}
\right]=0\,,
     %\label{e11-bos}
     \end{multline*}
откуда в~предположении $S_t h_t^2\hm+ H_t\hm>0$ следует, что существует 
оптимальное управ\-ле\-ние, которое определяется равенством
\begin{multline}
u_t^* = u_t^*(y,z)=-\fr{1}{2}\left( S_t h_t^2 +H_t\right)^{-1} \left( c_t 
\fr{\partial V_t(y,z)}{\partial z}-{}\right.\\
\left.{}-2S_t\left( s_t y-g_t z\right) h_t
\vphantom{\fr{\partial V_t(y,z)}{\partial z}}
\right)
\label{e12-bos}
\end{multline}
и доставляет минимум соответствующему сла\-га\-емо\-му в~урав\-не\-нии Беллмана, 
равный
$-\left( S_t h_t^2\hm+\right.$\linebreak
$\left.{}+H_t\right)^{-1} \left( c_t 
{\partial V_t(y,z)}/{\partial 
z}\hm-2S_t\left( s_t y \hm-g_t z\right) h_t \right)^2/4.
$ 
     
     Отметим, что, как и~в~классической ли\-ней\-но-квад\-ра\-тич\-ной 
задаче, управ\-ле\-ние из класса до\-пус\-ти\-мых не\-упреж\-да\-ющих получилось 
управ\-ле\-ни\-ем с~обратной связью.
     
     Таким образом, функция Беллмана описывается сле\-ду\-ющим 
дифференциальным уравнением:
     \begin{multline}
     \fr{\partial V_t(y,z)}{\partial t} +\fr{1}{2}\left( \Sigma_t^2(y) \fr{\partial^2 
V_t(y,z)} {\partial y^2}+\sigma_t^2\fr{\partial^2 V_t(y,z)} {\partial 
z^2}\right)+{}\\
     {}+ A_t(y) \fr{\partial V_t(y,z)}{\partial y}+\left( a_t y+b_t z\right) 
\fr{\partial V_t(y,z)}{\partial z}+{}\\
{}+ S_t \left( s_t y- g_t z\right)^2 +G_t z^2-
 \fr{1}{4}\left( S_t h_t^2+H_t\right)^{-1}\times{}\\
 {}\times \left( c_t \fr{\partial V_t(y,z)} 
{\partial z}-2S_t\left( s_t y -g_t z\right) h_t \right)^2=0\,.
     \label{e13-bos}
     \end{multline}
     
     Возводя в~квадрат по\-след\-нее сла\-га\-емое в~(\ref{e13-bos}), перепишем 
его в~виде:
     \begin{multline}
     \fr{\partial V_t(y,z)}{\partial t} +\fr{1}{2}\left( \Sigma_t^2(y) \fr{\partial^2 
V_t(y,z)} {\partial y^2}+\sigma_t^2\fr{\partial^2 V_t(y,z)} {\partial 
z^2}\!\right)+{}\\
{}+A_t(y) \fr{\partial V_t(y,z)}{\partial y}
+ \left( 
\vphantom{\left( S_t h_t^2 +H_t\right)^{-1}}
a_t y+b_t z+{}\right.\\
\left.{}+\left( S_t h_t^2 +H_t\right)^{-1}
 c_t S_t \left( s_t y-g_t z\right) h_t
\right) 
     \fr{\partial V_t(y,z)}{\partial z}+{}\\
     {}+\left( S_t-\left( S_t h_t^2 +H_t\right)^{-1} S_t^2 h_t^2\right)\left( s_t y -
g_t z\right)^2+{}\\
     \!\!{}+
     G_t z^2 -\fr{1}{4}\left( S_t h_t^2+H_t\right)^{-1}\! c_t^2
     \left(\fr{\partial V_t(y,z)}{\partial z}\right)^{\!2}=0\,.\!\!
     \label{e14-bos}
     \end{multline}
     
     Рассматривая полученное уравнение, заметим, что его решение может 
быть пред\-став\-ле\-но в~виде:
   \begin{equation}
     V_t(y,z)= \alpha_t z^2+\beta_t(y) z +\gamma_t(y)\,,
     \label{e15-bos}
     \end{equation}
т.\,е.\ будем искать решение при дополнительном предположении 
о~квад\-ра\-тич\-ности функции Белл\-ма\-на по переменной~$z$, и~сведем, таким 
образом, поиск оптимального решения к~уравнениям относительно функций 
$\alpha_t$, $\beta_t(y)$ и~$\gamma_t(y)$. Отметим сразу, что явный вид 
функции~$\gamma_t(y)$ для реализации оптимального управ\-ле\-ния не 
требуется, однако далее будет предложен вариант вы\-чис\-ле\-ния и~этой 
функции, что пред\-став\-ля\-ет\-ся небесполезным, поскольку позволит выполнять 
расчет минимума целевого функционала. Источником для 
предложения~(\ref{e15-bos}) является уже упоминавшаяся аналогичная 
задача для случая дис\-крет\-но\-го времени~\cite{7-bos, 8-bos}. В~той задаче 
выражение для функции Беллмана получается формально без 
дополнительных усилий. При этом форма~(\ref{e15-bos}) обнаруживается 
как свойство оптимального решения. В~рассматриваемом случае 
непрерывного времени~(\ref{e15-bos}) постулируется, а~пра\-виль\-ность 
постулата под\-тверж\-да\-ет\-ся далее ре\-зуль\-ти\-ру\-ющи\-ми уравнениями 
для~$\alpha_t$, $\beta_t(y)$ и~$\gamma_t(y)$ Кроме того, данное 
предположение пред\-став\-ля\-ет\-ся вы\-те\-ка\-ющим из линейной структуры задачи 
в~отношении переменной~$z$, в~част\-ности, тем фактом, что такой вид 
функции Беллмана обеспечивает линейность оптимального 
управ\-ле\-ния~(\ref{e12-bos}) по~$z$.

     Граничное условие при выбранном предположении~(\ref{e15-bos}) 
принимает вид:

\noindent
     \begin{multline*}
     V_T(y,z)= S_T\left( s_T y- g_T z\right)^2+G_T z^2 ={}\\[-0.5pt]
     {}=\alpha_T z^2 
+\beta_T(y) z +\gamma_T(y)\,,
    \end{multline*}
т.\,е.

\noindent
\begin{align*}
\alpha_T&= S_T g_T^2 +G_T\,;\\[-0.5pt]
\beta_T(y)&=-2S_T s_T g_T y\,;\\[-0.5pt]
\gamma_T(y)&=S_T s_T^2 y^2\,.
%\label{e16-bos}
\end{align*}
          При этом само оптимальное управ\-ле\-ние, определенное 
выражением~(\ref{e12-bos}), оказывается управ\-ле\-ни\-ем с~обратной связью 
по~$y_t$ и~$z_t$:

\noindent
     \begin{multline}
     u_t^*=u_t^*(y,z) ={}\\[-0.5pt]
     {}=
     -\fr{1}{2}\left( S_t h_t^2 +H_t\right)^{-1}
     \left( c_t \left( 2\alpha_t z +\beta_t(y)\right) +{}\right.\\[-0.5pt]
    \left. {}+2S_t\left( s_t y-g_t z\right) 
h_t\right)\,.
     \label{e17-bos}
     \end{multline}
          Подставляем $V_t(y,z)\hm= \alpha_t z^2 \hm+ \beta_t(y) 
z\hm+\gamma_t(y)$ в~(\ref{e14-bos}):

\noindent
     \begin{multline*}
     \fr{\partial \alpha_t}{\partial t}\, z^2 +
     \fr{\partial \beta_t(y)}{\partial t}\,z +
     \fr{\partial \gamma_t(y)}{\partial t}+{}\\[-0.5pt]
     {}+\fr{1}{2}\left( \Sigma_t^2(y) \left( 
\fr{\partial^2\beta_t(y)}{\partial y^2}\,z +\fr{\partial^2 \gamma_t(y)}{\partial 
y^2}\right) +2\sigma_t^2\alpha_t\right)+{}\\[-0.5pt]
 {}+A_t(y)\left(\fr{\partial \beta_t(y)}{\partial y}\,z + \fr{\partial 
\gamma_t(y)}{\partial y}\right) +{}\\[-0.5pt]
\hspace*{-0.22987pt}{}+\left( a_t y+b_t z+\left( S_t h_t^2 +H_t\right)^{-1} c_t S_t \left( s_t y-
g_t z\right) h_t\right)\times{}
\end{multline*}

\noindent
\begin{multline*}
         {}\times \left( 2\alpha_t z+\beta_t(y)\right)+{}\\
     {}+\left( S_t-\left( S_t h_t^2 +H_t\right)^{-1} S_t^2 h_t^2\right)\left( s_t y-
g_t z\right)^2+{}\\
     {}+ G_t z^2 -\fr{1}{4}\left( S_t h_t^2 +H_t\right)^{-1} c_t^2 \left( 
2\alpha_t z+\beta_t(y)\right)^2=0\,.
     \end{multline*}
          Далее выделяем слагаемые при~$z^2$, $z$ и~$z^0$
          
          \noindent
     \begin{multline*}
     \fr{\partial \alpha_t}{\partial t}\, z^2 +\fr{\partial \beta_t(y)}{\partial t}\,z +
     \fr{\partial \gamma_t(y)}{\partial 
t}+\fr{1}{2}\,\Sigma_t^2(y)\fr{\partial^2\beta_t(y)}{\partial y^2}\,z+ {}\\
{}+
\fr{1}{2}\,\Sigma_t^2(y)\fr{\partial^2\gamma_t(y)}{\partial 
y^2}+\sigma_t^2\alpha_t+A_t(y)\fr{\partial \beta_t(y)}{\partial y}\,z +{}\\
{}+A_t(y) \fr{\partial 
\gamma_t(y)}{\partial y}+{}\\
{}+ 2\alpha_t \left( b_t -\left( S_t h_t^2+H_t\right)^{-1} c_t 
S_t h_t g_t \right)z^2+{}\\
     {}+
     \left( 2\alpha_t\left( \alpha_t+\left( S_t h_t^2+H_t\right)^{-1} c_t S_t h_t 
s_t\right)y +{}\right.\\
\left.{}+\beta_t(y) \left( b_t-\left( S_t h_t^2+H_t\right)^{-1} c_t S_t h_t 
g_t\right) \right) z+{}\\
     {}+\beta_t(y)\left( a_t +\left( S_t h_t^2+H_t\right)^{-1} c_t S_t h_t s_t\right) 
y+{}\\
{}+ \left( S_t -\left( S_t h_t^2+H_t\right)^{-1} S_t^2 h_t^2\right) g_t^2 z^2-{}\\
     {}- 2\left( S_t -\left( S_t h_t^2+H_t\right)^{-1} S_t^2 h_t^2\right) s_t g_t yz 
+{}\\
{}+
     \left( S_t-\left( S_t h_t^2+H_t\right)^{-1} S_t^2 h_t^2\right) s_t^2 y^2+{}\\
     {}+G_t z^2 -\left( S_t h_t^2 +H_t\right)^{-1} c_t^2 \alpha_t^2 z^2 -{}\\
     {}-\left( 
S_t h_t^2+H_t\right)^{-1} c_t^2 \alpha_t \beta_t(y) z-{}\\
{}-
\fr{1}{4}\left( S_t h_t^2+H_t\right)^{-1}  c_t^2 \beta_t^2(y)=0\,,
     \end{multline*}
группируем их и~получаем сле\-ду\-ющие уравнения:
\begin{itemize}
\item  для~$\alpha_t$:

\noindent
\begin{multline}
\fr{\partial\alpha_t}{\partial t}+2\alpha_t\left( b_t-\left( S_t h_t^2+H_t\right)^{-1} c_t 
S_t h_t g_t\right)+{}\\
{}+ \left( S_t- \left( S_t h_t^2+H_t\right)^{-1} S_t^2 h_t^2\right) 
g_t^2+G_t-{}\\
\hspace*{-8mm}{}-\left( S_t h_t^2+H_t\right)^{-1} c_t^2 \alpha_t^2 =0\,,\enskip \alpha_T=S_T 
g_t^2+G_T\,;\!\!
\label{e18-bos}
\end{multline}
\item для $\beta_t$:

\noindent
\begin{multline}
\fr{\partial\beta_t(y)}{\partial 
t}+\fr{1}{2}\,\Sigma_t^2(y)\fr{\partial^2\beta_t(y)}{\partial y^2} 
+A_t(y)\fr{\partial \beta_t(y)}{\partial y}+{}\\
{}+ 2\alpha_t\left( a_t +\left( S_t h_t^2+H_t\right)^{-1} c_t S_t h_t s_t\right) y+{}\\
{}+
\beta_t(y)\left( b_t -\left( S_t h_t^2 +H_t\right)^{-1} c_t S_t h_t g_t\right)-{}\\
{}-2\left( S_t-\left( S_t h_t^2+H_t\right)^{-1} S_t^2 h_t^2\right) s_t g_t y-{}
\\
{}-
\left( S_t h_t^2+H_t\right)^{-1} c_t^2 \alpha_t \beta_t(y)=0\,,\\
\beta_T(y)=-2S_T s_T g_T y\,;
\label{e19-bos}
\end{multline}
\item  для $\gamma_t$:
\begin{multline}
\hspace*{-0.8pt}\fr{\partial \gamma_t(y)}{\partial t}+\fr{1}{2}\,\Sigma_t^2(y)
\fr{\partial^2 \gamma_t(y)}{\partial y^2} +\sigma_t^2 \alpha_t +A_t(y)
\fr{\partial \gamma_t(y)}{\partial y}+{}\\
{}+ \beta_t(y)\left( a_t +\left( S_t h_t^2+H_t\right)^{-1} c_t S_t h_t s_t\right) y+{}\\
{}+
\left( S_t-\left( S_t h_t^2+H_t\right)^{-1} S_t^2 h_t^2\right)  s_t^2 y^2-{}\\
{}-\fr{1}{4}\left( S_t h_t^2+H_t\right)^{-1} c_t^2 \beta_t^2(y) =0\,,\\
\gamma_T(y)=S_T s_T^2 y^2\,.
\label{e20-bos}
\end{multline}
\end{itemize}
     
     Уравнение~(\ref{e18-bos}), легко заметить, является уравнением 
Риккати, которое в~силу сформулированного выше условия   
имеет единственное неотрицательное решение для всех $0\hm\leq t\hm\leq T$. 
Этот факт требует дополнительного комментария. Для получения 
уравнения~(\ref{e18-bos}) рас\-смот\-рим исходную задачу при дополнительных 
условиях $a_t\hm=0$ и~$s_t\hm=0$ для всех $0\hm\leq t\hm\leq T$. Нетрудно 
видеть, что эти условия рассматриваемую по\-ста\-нов\-ку сводят фактически 
к~классической ли\-ней\-но-квад\-ра\-тич\-ной задаче. Имеющуюся 
в~рассматриваемой формулировке чуть более общую форму целевой 
функции (принципиального значения это обобщение, конечно, не имеет) 
сведем к~классической еще одним предположением: $S_t\hm=0$ для всех 
$0\hm\leq t\hm\leq T$. Теперь уравнение для~$\alpha_t$ принимает хорошо 
известный вид:
     \begin{equation}
     \fr{\partial \alpha_t}{\partial t}+2\alpha_t b_t +G_t- H_t^{-1} c_t^2 
\alpha_t^2=0\,,\enskip \alpha_T=G_T\,.
     \label{e21-bos}
     \end{equation}

     В таком случае, как известно~\cite{10-bos}, существует единственное 
оптимальное управление~--- линейное с~обратной связью по выходу~$z_t$, 
с~коэффициентом усиления, опи\-сы\-ва\-емым уравнением  
Риккати~(\ref{e21-bos}). Именно этот результат дают  
уравнения~(\ref{e18-bos})--(\ref{e20-bos}) и~описываемая ими функция 
Беллмана~(\ref{e15-bos}), так как из $a_t\hm=0$ и~$s_t\hm=0$ немедленно 
следует, что $\beta_t(y)\hm=0$, откуда, в~свою очередь, с~учетом 
не\-за\-ви\-си\-мости решения от~$y_t$ следует, что $\gamma_t(y)\hm=\gamma_t$, 
т.\,е.\ не зависит от~$y$ и~задается уравнением: 
     $$
     \fr{\partial \gamma_t(y)}{\partial t} +\sigma^2_t \alpha_t=0\,,\enskip 
\gamma_T=0\,.
     $$ 
     Оптимальное управ\-ле\-ние при этом 
     $$
     u_t^*= -H_t^{-1} c_t \alpha_t z_t\,,
     $$
      т.\,е.\ все полностью совпадает с~известным классическим решением.
     
     С уравнениями~(\ref{e19-bos}) и~(\ref{e20-bos}) ситуация, естественно, 
обстоит сложнее. Это линейные уравнения второго порядка параболического 
типа, поскольку\linebreak
 $\Sigma_t^2(y)\hm>0$. Фактически отсутствуют 
конструктивные условия, гарантирующие существование их\linebreak
 решений 
(требовать, чтобы все фигурирующие в~уравнениях коэффициенты были 
представлены аналитическими функциями на всем пространстве значений, 
вряд ли целесообразно), поэтому далее будем предполагать, что данные 
уравнения имеют на рас\-смат\-ри\-ва\-емом интервале $0\hm\leq t\hm\leq T$ хотя 
бы одно ограниченное решение и~именно эти условия будем рас\-смат\-ри\-вать 
как достаточные условия существования оптимального решения 
рассматриваемой задачи.
     
     Таким образом, доказана следующая тео\-рема.
     
     \smallskip
     
     \noindent
     \textbf{Теорема.}\ \textit{Пусть для диффузионного 
процесса}~(\ref{e5-bos}) \textit{выполнены условия Ито, для 
     процесса}~(\ref{e6-bos})~--- \textit{ограничены коэффициенты, 
уравнения}~(\ref{e18-bos})--(\ref{e20-bos}) \textit{имеют ограниченные 
решения для $0\hm\leq t\hm\leq T$. Тогда минимум  
функционалу}~(\ref{e7-bos}) \textit{доставляет оптимальное 
управ\-ле\-ние}~(\ref{e17-bos}), \textit{где} $y\hm= y_t$; $z\hm=z_t$.
     
\section{Заключение}

     Рассмотренная задача оптимизации в~целом близка и~по модели, и~по 
критерию к~классической ли\-ней\-но-квад\-ра\-тич\-ной постановке. 
Принципиальным отличием является нелинейная модель для описания 
со\-сто\-яния динамической сис\-те\-мы, в~которой отсутствует управ\-ля\-ющее 
воздействие.\linebreak
 Такую модель наряду с~традиционной интер\-пре\-тацией  
<<со\-сто\-яние--вы\-ход>> мож\-но понимать как\linebreak модель неконтролируемого 
слож\-но\-го внешнего воздействия. Небольшое дополнительное отличие дает 
предложенная форма квад\-ра\-тич\-но\-го критерия, поз\-во\-ля\-ющая, в~част\-ности, 
ставить такие задачи, как отслеживание выходом или управ\-ле\-ни\-ем со\-сто\-яния 
сис\-те\-мы или ее выхода.
     
     Поскольку обсуждать возможности точного решения уравнений, 
определяющих оптимальное управ\-ле\-ние, не имеет смыс\-ла, наиболее 
актуальной далее является задача их приближенного чис\-лен\-но\-го решения 
и~анализа воз\-мож\-ности практической реализации. Этому посвящена вторая 
часть данной работы, пла\-ни\-ру\-емая к~выходу в~ближайшее время.

{\small\frenchspacing
 {%\baselineskip=10.8pt
 \addcontentsline{toc}{section}{References}
 \begin{thebibliography}{99}
\bibitem{1-bos}
\Au{Athans M.} Editorial on the LQG problem~// IEEE~T. Automat. Contr., 1971. Vol.~16. 
No.\,6. P.~528--552. doi: 10.1109/TAC.1971.1099845.
\bibitem{2-bos}
\Au{Wu Z.} Forward-backward stochastic differential equations, linear quadratic stochastic 
optimal control and nonzero sum differential games~// J.~Syst. Sci. Complex., 2005. Vol.~18. 
No.\,2. P.~179--192.
\bibitem{3-bos}
\Au{Chen B.\,S., Zhang~W.} Stochastic H2/H1 control with state-dependent noise~// IEEE 
T.~Automat. Contr., 2004. Vol.~49. No.\,1. P.~45--56. doi: 10.1109/TAC.2003.821400.
\bibitem{4-bos}
\Au{Bohacek S.} A~stochastic model of TCP and fair video transmission~// IEEE 
INFOCOM, 2003. Vol.~2. P.~1134--1144. doi: 10.1109/INFCOM.2003.1208950.
\bibitem{5-bos}
\Au{Домбровский В.\,В., Объедко~Т.\,Ю.} Управление с~прогнозированием системами 
с~марковскими скачками при ограничениях и~применение к~оптимизации 
инвестиционного портфеля~// Автомат. телемех., 2011. №\,5. С.~96--112. doi: 
10.1134/S0005117911050079.
\bibitem{6-bos}
\Au{Баландин Д.\,В., Коган~М.\,М.} Оптимальное линейно-квад\-ра\-тич\-ное управление: от 
матричных уравнений к~линейным матричным неравенствам~// Автомат. телемех., 2011. 
№\,11. С.~60--69. doi: 10.1134/ S0005117911110038.
\bibitem{7-bos}
\Au{Босов А.\,В.} Обобщенная задача распределения ресурсов программной системы~// 
Информатика и~её применения, 2014. Т.~8. Вып.~2. С.~39--47. doi: 
10.14357/19922264140204.
\bibitem{8-bos}
\Au{Босов А.\,В.} Управление линейным выходом дискретной стохастической системы по 
квадратичному критерию~// Изв. РАН. Теория и~системы управления, 2016. №\,3.  
С.~19--35. doi: 10.1134/S1064230716030060.
\bibitem{9-bos}
\Au{Флеминг У., Ришел~Р.} Оптимальное управление детерминированными 
и~стохастическими системами~/ Пер. с~англ.~--- М.: Мир, 1978. 316~с. 
(\Au{Fleming~W.\,H., Rishel~R.\,W.} Deterministic and stochastic optimal control.~--- New 
York, NY, USA: Springer-Verlag, 1975. 222~p.)
\bibitem{10-bos}
\Au{Девис М.\,Х.\,А.} Линейное оценивание и~стохастическое управление~/ Пер. с~англ.~--- 
М.: Наука, 1984. 206~с. (\Au{Davis~M.\,H.\,A.} Linear estimation and stochastic control.~--- 
London: Chapman and Hall, 1977. 224~p.)

 \end{thebibliography}

 }
 }

\end{multicols}

\vspace*{-6pt}

\hfill{\small\textit{Поступила в~редакцию 30.03.18}}

\vspace*{4pt}

%\newpage

%\vspace*{-24pt}

\hrule

\vspace*{2pt}

\hrule

\vspace*{-2pt}


\def\tit{STOCHASTIC DIFFERENTIAL SYSTEM OUTPUT CONTROL 
BY~THE~QUADRATIC CRITERION.~I.~DYNAMIC\\ PROGRAMMING 
OPTIMAL SOLUTION}


\def\titkol{Stochastic differential system output control 
by~the~quadratic criterion. I.~Dynamic programming 
optimal solution}

\def\aut{A.\,V.~Bosov and~A.\,I.~Stefanovich}

\def\autkol{A.\,V.~Bosov and~A.\,I.~Stefanovich}

\titel{\tit}{\aut}{\autkol}{\titkol}

\vspace*{-11pt}


\noindent
Institute of Informatics Problems, Federal Research Center ``Computer Science 
and Control'' of the Russian Academy of Sciences, 44-2~Vavilov Str., Moscow 
119333, Russian Federation


\def\leftfootline{\small{\textbf{\thepage}
\hfill INFORMATIKA I EE PRIMENENIYA~--- INFORMATICS AND
APPLICATIONS\ \ \ 2018\ \ \ volume~12\ \ \ issue\ 3}
}%
 \def\rightfootline{\small{INFORMATIKA I EE PRIMENENIYA~---
INFORMATICS AND APPLICATIONS\ \ \ 2018\ \ \ volume~12\ \ \ issue\ 3
\hfill \textbf{\thepage}}}

\vspace*{3pt}



\Abste{The problem of optimal control for the Ito diffusion 
process and a~controlled linear output is solved. The considered 
statement is close to the classical linear-quadratic Gaussian 
control  (LQG control) problem. Differences consist in the fact 
that the state is described by the nonlinear differential Ito equation  $dy_y = A_t(y_t) 
\,dt+\Sigma_t(y_t)\,dv_t$ and does not depend on the control~$u_t$, 
optimization subject is controlled linear output 
 $dz_t=a_ty_t\,dt +b_tz_t\,dt +c_t u_t\,dt +\sigma_t \,dw_t$. 
Additional generalizations are included in the quadratic 
quality criterion for the purpose of statement such problems 
as state tracking by output or a linear combination of state 
and output tracking by control. The method of dynamic programming 
is used for the solution. 
The assumption about Bellman function in the form  $V_t(y,z)= \alpha_t 
z^2+\beta_t(y) z+\gamma_t(y)$ allows one to find it. 
Three differential equations for the coefficients $\alpha_t$,  $\beta_t(y)$,
and $\gamma_t(y)$ give the solution. 
These equations constitute the optimal solution of the problem under consideration.}

\KWE{stochastic differential equation; optimal control; dynamic programming; 
Bellman function; Riccati equation; linear differential equations of parabolic type}


\DOI{10.14357/19922264180314}

\vspace*{-12pt}

\Ack
\noindent
This work was partially supported by the Russian Science Foundation (grant  
16-07-00677).



%\vspace*{6pt}

  \begin{multicols}{2}

\renewcommand{\bibname}{\protect\rmfamily References}
%\renewcommand{\bibname}{\large\protect\rm References}

{\small\frenchspacing
 {%\baselineskip=10.8pt
 \addcontentsline{toc}{section}{References}
 \begin{thebibliography}{99}
\bibitem{1-bos-1}
\Aue{Athans, M.} 1971. Editorial on the LQG problem. \textit{IEEE~T. 
Automat. Contr.} 16(6):528--552. doi: 10.1109/ TAC.1971.1099845.
\bibitem{2-bos-1}
\Aue{Wu, Z.} 2005. Forward-backward stochastic differential equations, linear 
quadratic stochastic optimal control and\linebreak\vspace*{-12pt}

\columnbreak

\noindent
 nonzero sum differential games. 
\textit{J.~Syst. Sci. Complex.} 18(2):179--192.
\bibitem{3-bos-1}
\Aue{Chen, B.\,S. and W.~Zhang.} 2004. Stochastic H2/H1 control with  
state-dependent noise. \textit{IEEE~T. Automat. Contr.} 49(1):45--56.
doi: 10.1109/TAC.2003.821400.
\bibitem{4-bos-1}
\Aue{Bohacek, S.} 2003. A~stochastic model of TCP and fair video 
transmission. \textit{IEEE INFOCOM}. 2:1134--1144.
doi: 10.1109/INFCOM.2003.1208950.
\bibitem{5-bos-1}
\Aue{Dombrovskii, V.\,V., and T.\,Yu.~Ob''edko.} 2011. Predictive control of 
systems with Markovian jumps under constraints and its application to the 
investment portfolio optimization. \textit{Automat. Rem. Contr.}  
72(5):989--1003.
\bibitem{6-bos-1}
\Aue{Balandin, D.\,V., and M.\,M.~Kogan.} 2011. Optimal linear-quadratic 
control: From matrix equations to linear matrix inequalities. \textit{Automat. 
Rem. Contr.} 72(11):2276--2284.
\bibitem{7-bos-1}
\Aue{Bosov, A.\,V.} 2014. Obobshchennaya zadacha raspredeleniya resursov 
programmnoy sistemy [The generalized problem of software system resources 
distribution]. \textit{Informatika i~ee Primeneniya~--- Inform. Appl.}  
8(2):39--47. doi: 
10.14357/19922264140204.
\bibitem{8-bos-1}
\Aue{Bosov, A.\,V.} 2016. Discrete stochastic system linear output control 
with respect to a quadratic criterion. \textit{J.~Comput. Syst. Sc. 
Int.} 55(3):349--364.
\bibitem{9-bos-1}
\Aue{Fleming, W.\,H., and R.\,W.~Rishel.} 1975. \textit{Deterministic and 
stochastic optimal control.} New York, NY: Springer-Verlag. 222~p.
\bibitem{10-bos-1}
\Aue{Davis, M.\,H.\,A.} 1977. \textit{Linear estimation and stochastic 
control.} London: Chapman and Hall. 224~p.
\end{thebibliography}

 }
 }

\end{multicols}

\vspace*{-6pt}

\hfill{\small\textit{Received March 30, 2018}}

%\pagebreak

%\vspace*{-18pt}
     
     \Contr
     
       \noindent
       \textbf{Bosov Alexey V.} (b.\ 1969)~--- Doctor of Science in technology, 
principal scientist, Institute of Informatics Problems, Federal Research 
Center ``Computer Science and Control'' of the Russian Academy of Sciences, 
44-2~Vavilov Str., Moscow 119333, Russian Federation; 
\mbox{AVBosov@ipiran.ru}
       
       \vspace*{3pt}
       
       \noindent
       \textbf{Stefanovich Alexey I.} (b.\ 1983)~--- principal specialist, 
Institute of Informatics Problems, Federal Research Center ``Computer Science 
and Control'' of the Russian Academy of Sciences, 44-2~Vavilov Str., Moscow 
119333, Russian Federation; \mbox{AStefanovich@frccsc.ru}
\label{end\stat}

\renewcommand{\bibname}{\protect\rm Литература}       

        %4 есть

%\newcommand {\ff}{{\mathcal F}}
\newcommand {\ebd}{\triangleq}
\newcommand{\me}[2]{\mathbf{E}_{ #1 }\left\{ \mathop{#2} \right\} }



\def\stat{borisov}

\def\tit{ФИЛЬТРАЦИЯ СОСТОЯНИЙ МАРКОВСКИХ СКАЧКООБРАЗНЫХ ПРОЦЕССОВ 
ПО~ДИСКРЕТИЗОВАННЫМ НАБЛЮДЕНИЯМ$^*$}

\def\titkol{Фильтрация состояний марковских скачкообразных процессов 
по~дискретизованным наблюдениям}

\def\aut{А.\,В.~Борисов$^1$}

\def\autkol{А.\,В.~Борисов}

\titel{\tit}{\aut}{\autkol}{\titkol}

\index{Борисов А.\,В.}
\index{Borisov A.\,A.}




{\renewcommand{\thefootnote}{\fnsymbol{footnote}} \footnotetext[1]
{Работа выполнена при частичной поддержке РФФИ (проект 16-07-00677).}}


\renewcommand{\thefootnote}{\arabic{footnote}}
\footnotetext[1]{Институт проблем информатики Федерального исследовательского центра <<Информатика 
и~управление>> Российской академии наук,
\mbox{aborisov@frccsc.ru}}

%\vspace*{8pt}



\Abst{Статья посвящена решению задачи оптимальной 
фильтрации состояний однородного марковского скачкообразного процесса (МСП). 
Наблюдения представляют собой приращения случайных процессов~--- интегральных 
преобразований состояний, зашумленные винеровскими процессами, интенсивность 
которых также зависит от оцениваемого состояния. Оптимальная оценка в~моменты 
получения нового наблюдения вычисляется как функция предыдущей оценки и~новых 
наблюдений, а~между моментами наблюдений~--- простейшим прогнозом в~силу системы 
уравнений Колмогорова. Рекуррентная формула пересчета ресурсозатратна, так как 
содержит  интегралы~--- мас\-штаб\-но-сдви\-го\-вые смеси многомерных гауссиан, 
где в~качестве смешивающих выступают распределения времени пребывания 
состояния в~каждом из возможных значений. Предложены более простые аппроксимации, 
основанные на предположении об ограниченности числа скачков состояния за время между 
наблюдениями. Получены универсальные локальная и~глобальная характеристики точности 
аппроксимаций, зависящие от па\-ра\-мет\-ров оцениваемого процесса, величины 
временн$\acute{\mbox{о}}$го шага  между наблюдениями и~максимального числа учитываемых скачков.}

\KW{марковский скачкообразный процесс; оптимальная фильтрация; мультипликативные 
шумы в~наблюдениях; стохастическое дифференциальное уравнение; численная аппроксимация}

\DOI{10.14357/19922264180316}
  
%\vspace*{4pt}


\vskip 10pt plus 9pt minus 6pt

\thispagestyle{headings}

\begin{multicols}{2}

\label{st\stat}



 \section{Введение}
 
 Фильтр Вонэма~\cite{Won_65}~--- один из редких удачных случаев, когда 
 оценка оптимальной фильтрации состо\-яния стохастической системы наблюдения 
 выражается в~виде решения некоторой замк\-ну\-той\linebreak конечномерной сис\-те\-мы 
 стохастических дифференциальных уравнений. 
 
 Алгоритм данного фильт\-ра 
 позволяет вычислить оценку фильт\-ра\-ции со\-сто\-яния \textit{марковского скачкообразного 
 процесса} с~\mbox{конечным} множеством состояний по наблюдениям в~присутствии 
 аддитивных винеровских шумов. Теоретически оптимальная оценка со\-сто\-яния~--- 
 его условное распределение в~текущий момент времени~--- 
 обладает очевидными свойствами неотрицательности и~нормировки. 
 При чис\-лен\-ной реализации данного фильтра классическим методом 
 Эй\-ле\-ра--Ма\-ру\-ямы~\cite{KP_92} данные свойства могут не сохраняться и~процедура 
 вы\-чис\-ле\-ния становится неустойчивой.  В~связи с~этим обстоятельством разрабатывались 
 другие алгоритмы чис\-лен\-но\-го решения уравнения фильтра Вонэма, обладающие 
 требуемыми свойствами устойчивости (см.~\cite{YZL_04, PR_10} и~библиографию в~них). 
 В~час\-ти этих работ доказана лишь слабая сходимость пред\-ла\-га\-емых аппроксимационных 
 схем к~оценке фильт\-ра Вонэма, в~то время как ка\-кая-ли\-бо 
 характеризация точ\-ности этих приближений отсутствует.
 
 В~\cite{B_18} было представлено распространение фильт\-ра Вонэма на случай 
 наблюдений с~мультипликативными шумами. При этом уравнение обобщенного 
 фильт\-ра содержит в~правой части квадратическую характеристику шумов в~наблюдениях. 
 Данный процесс на практике никогда не наблюдается непосредственно, а~является лишь 
 некоторым нелинейным интегральным преобразованием наблюдений. Очевидно, что 
 имеющиеся в~настоящий момент времени алгоритмы приближенного вычисления оценки 
 фильтрации Вонэма для данной системы не подходят. 
 
 Целью предлагаемой работы является ис\-поль\-зование результатов оптимальной 
 фильтрации со\-стояний сис\-тем с~дискретным временем для аппроксимации решения 
 аналогичной задачи для\linebreak стохастических дифференциальных сис\-тем. 
 
 Статья организована следующим образом. Раздел~2 содержит формальную постановку 
 задачи фильт\-ра\-ции со\-сто\-яний однородного МСП с~конечным множеством со\-сто\-яний 
 по наблюдениям, полученным путем временн$\acute{\mbox{о}}$й дискретизации процессов с~непрерывным 
 временем~--- интегральных преобразований со\-сто\-яния сис\-те\-мы в~присутствии 
 мультипликативных винеровских шумов.\linebreak
  В~разд.~3 пред\-став\-ле\-но решение поставленной 
 задачи фильт\-ра\-ции: пересчет оценок со\-сто\-яний в~момент получения новых 
 дискретизованных наблюдений выполняется в~соответствии с~некоторыми\linebreak 
 рекуррентными интегральными соотношениями, в~то время как между 
 моментами наблюдений оценка корректируется в~соответствии с~прогнозом в~силу 
 сис\-те\-мы уравнений Колмогорова. Вы\-чис\-ли\-тель\-ная слож\-ность 
 упомянутых выше интегральных\linebreak 
 соотношений связана с~тем, что в~расчет принимается воз\-мож\-ность того, что между 
 моментами наблюдений оцениваемое со\-сто\-яние может совершить произвольное чис\-ло 
 скачков. В~разд.~4 пред\-став\-лен более простой алгоритм приближенного вы\-чис\-ле\-ния 
 оценки фильт\-ра\-ции, основанный на ограничении возможного числа учитываемых скачков 
 МСП. Доказана тео\-ре\-ма, опре\-де\-ля\-ющая как\linebreak
  локальную (одношаговую), так и~глобальную 
 (многошаговую) характеристики точ\-ности предложенного при\-бли\-же\-ния~--- 
 $\ell_1$-нор\-мы ошибки аппроксимации. Полученные характеристики являются\linebreak 
 универсальными, т.\,е.\ не асимптотическими по шагу дискретизации, и~зависят от характеристик 
 самого МСП, %\linebreak
  шага временн$\acute{\mbox{о}}$й дискретизации и~чис\-ла
  скачков со\-сто\-яния, учи\-ты\-ва\-емых 
 на шаге. Об\-суж\-де\-ние результатов и~заключительные комментарии пред\-став\-ле\-ны 
 в~разд.~5.
 
 \section{Постановка задачи фильтрации}
 
 На полном вероятностном пространстве с~фильт\-ра\-цией 
 $(\Omega,\mathcal{F},\mathcal{P},\{\mathcal{F}_{t}\}_{t \geqslant 0})$ рассматривается система наблюдений
\begin{equation}
 \left.
 \begin{array}{rl}
 \displaystyle X_t &=X_0 +  \displaystyle
 \int\limits_0^t \Lambda^{\top}X_{s}\,ds + \mu_s\,;  \\[6pt]
 \displaystyle Y_k &=  \displaystyle\int\limits_{t_{k-1}}^{t_k}fX_s\,ds+
 \int\limits_{t_{k-1}}^{t_k} 
 \sum\limits_{n=1}^NX_s^ng_n \,dW_s, \\[6pt]
 &\hspace*{10mm}\{t_k\}_{k \geqslant 0}: \; 0 = t_0 < t_1 < t_2\cdots,
 \end{array}
 \right\}
 \label{eq:obsys_1}
 \end{equation}
 где
  \begin{itemize}
  \item
  $X_t \ebd \mathrm{col}\left(X_t^1,\ldots,X_t^N\right) \hm\in \mathbb{S}^N$~--- 
  ненаблюда\-емое состояние системы, являющееся однородным МСП с~конечным 
  множеством состояний $ \mathbb{S}^N \ebd$\linebreak $\ebd \{e_1,\ldots,e_N\}$ ($\mathbb{S}^N$~--- 
  множество единичных векторов евклидова пространства~$\mathbb{R}^N$), 
  матрицей интенсивностей переходов~$\Lambda$ и~начальным распределением~$\pi$;
  \item
  $\mu_t \ebd \mathrm{col}\left(
  \mu_t^1,\ldots,\mu_t^N\right)\hm\in \mathbb{R}^N$~--- 
  ${\mathcal{F}}_t$-со\-гла\-со\-ван\-ный мартингал;
  \item
  $\{Y_k\}_{k \in \mathbb{N}}:\;  Y_k \ebd \mathrm{col}\left(Y_k^1,\ldots,Y_k^M\right) 
  \hm\in \mathbb{R}^M$~--- последовательность дискретизованных наблюдений, 
  доступных в~известные неслучайные  моменты времени~$\{t_k\}_{k \in \mathbb{N}}$,
в~которых $W_t \ebd$\linebreak $\ebd \mathrm{col}\left(W_t^1,\ldots,W_t^M\right) \hm\in \mathbb{R}^M$
 является ${\mathcal{F}}_t$-со\-гла\-со\-ван\-ным стандартным винеровским процессом, 
 определяющим шумы в~наблюдениях,\linebreak  $f$~--- $(M \times N)$-мер\-ная 
 мат\-ри\-ца плана наблюдений, а~набор мат\-риц~$\{g_n\}_{n=\overline{1,N}}$ 
 характеризует интенсивности шумов в~зависимости от текущего состояния~$X_t$.
  \end{itemize}
  
  Введем также в~рассмотрение неубывающие семейства $\sigma$-ал\-гебр 
  $\mathcal{O}_k \ebd \sigma\{ Y_{\ell}: \; 1 \hm\leqslant \ell \hm\leqslant k\}$ 
  и~$\mathcal{O}_t \ebd  \mathcal{O}_{k(t)}$, где 
  $k(t) \ebd \sum\nolimits_{j \in \mathbb{N}}\mathbf{I}(t-t_{j})$; 
  $\mathcal{O}_0 \ebd \{\varnothing,\; \Omega\}$.
  
   \textit{Задача оптимальной фильтрации состояния~$X$ по наблюдениям~$Y$} 
   заключается в~нахождении \textit{условного математического ожидания} (УМО)
  \begin{equation*}
  \widehat{X}_t \ebd {\sf E}\left\{X_t|\mathcal{O}_{t} \right\}\,.
 % \label{eq:fest_1}
  \end{equation*}
  
  Относительно системы~(\ref{eq:obsys_1})  сделаны следующие предположения:
   \begin{itemize}
 \item[(а)]
 ${\mathcal{F}}_t \equiv {\mathcal{F}}_{t}^X \bigvee 
 {\mathcal{F}}_{t}^W $ для любого $t \hm\geqslant 0$;
 \item[(б)]
 шумы в~наблюдениях равномерно невырожденные, т.\,е.\
  $g_ng_n^{\top} \hm\geqslant \alpha I \hm> 0$ для всех $n\hm=\overline{1,N}$ 
  и~некоторого $\alpha\hm>0$.
% \item
 % Верно неравенство
  %\begin{equation}
  %\min_{1\leqslant k \leqslant N}|\lambda_{kk}| > 0.
  %\label{eq:ineq_0}
  % \end{equation}
 %\item
 %Для любого $t \geqslant 0$ все компоненты вектора $p_t \ebd \me{}{X_t}$ строго %положительны. 
 \end{itemize} 

 \section{Уравнения оптимального фильтра} 
 
 Для получения уравнений оптимального фильт\-ра воспользуемся подходом, 
 применяемым для решения аналогичной задачи в~стохастических сис\-те\-мах 
 наблюдения с~дискретным временем~\cite{BSh_85}. 
 Воспользу\-ем\-ся методом математической индукции. 
 
 При $r=0$ 
 \begin{equation}
 \widehat{X}_{t_0}={\sf E}\{X_0|\mathcal{O}_0\}={\sf E}\{X_0\}=\pi\,.
 \label{eq:in_cond}
 \end{equation} 
 
 Пусть для некоторого $ r \hm\geqslant 0$ известна оценка оптимальной 
 фильтрации~$\widehat{X}_{t_r} \hm= {\sf E}{X_{t_r} |\mathcal{O}_r}$. 
 Определим оценку оптимальной фильтрации~$\widehat{X}_{t} $ для $t\hm \in (t_r,t_{r+1}]$. 
 
 Для произвольного момента $t \hm\in (t_r,t_{r+1})$ в~силу мартингального 
 разложения МСП~$X_t$ и~свойств УМО верна следующая цепочка равенств:
 \begin{multline*}
 \widehat{X}_{t} = {\sf E}\left\{X_t | \mathcal{O}_r\right\}={}\\
 {}=
 {\sf E}\left\{{\cal P}^{\top}(t_r,t)X_{t_r}+
 \int\limits_{t_r}^t{\cal P}^{\top}(t_r,s)\,dM_s\big\vert \mathcal{O}_r\right\} = {}
\end{multline*}

\noindent
   \begin{multline}
 \hspace*{-11.66pt}{}=\mathcal{P}^{\top}(t_r,t)\widehat{X}_{t_r} + {\sf E}\hspace*{-2pt}
 \left\{{\sf E}\hspace*{-2pt}\left\{\int\limits_{t_r}^t\hspace*{-2pt}\mathcal{P}^{\top}(t_r,s)\,dM_s |
 {\mathcal{F}}_{t_r}\right\}\!\big\vert 
 \mathcal{O}_r\!\right\} ={}\hspace*{-4.24124pt}\\
 {}=
  \mathcal{P}^{\top}(t_r,t)\widehat{X}_{t_r}\,,
 \label{eq:bw_obs}
 \end{multline}
 где $\mathcal{P}(s,t)$ $(s \hm\leqslant t)$~--- матрица переходной ве\-ро\-ят\-ности МСП 
 на промежутке $[s,t]$, являющаяся решением сис\-те\-мы дифференциальных 
 уравнений Колмогорова
 \begin{equation*}
 \mathcal{P}'_t(s,t) = \mathcal{P}(s,t) \Lambda, \enskip t > s, \enskip \mathcal{P}(s,s) = I.
 \end{equation*}
 В случае однородного МСП $\mathcal{P}(s,t) \hm= e^{(t-s)\Lambda}$.
 
 Далее необходимо определить совместное распределение $(X_{t_{r+1}},Y_{r+1})$ 
 относительно~$ \mathcal{O}_r$. Из модели наблюдений следует, что 
 распределение~$Y_{r+1}$ относительно 
 $\sigma$-ал\-геб\-ры~$\mathcal{F}^X_{t_{r+1}} \vee \mathcal{O}_r$~---
 гауссовское с~параметрами 
 \begin{align*}
{\sf E}\left\{Y_{r+1}|{\mathcal{F}}^X_{t_{r+1}}\right\}& = f \tau_{r+1}\,; \\[6pt]
 \mathrm{cov} \left(Y_{r+1},Y_{r+1}|{\mathcal{F}}^X_{t_{r+1}}\right) &= 
 \displaystyle\sum\limits_{n=1}^N \tau_{r+1}^n g_ng_n^{\top}\,,
% \label{eq:occup_1}
 \end{align*}
 где $\tau_{r+1} \hm= \tau_{r+1}(X(\omega))=
 \mathrm{col}\left(\tau_{r+1}^1,\ldots,\tau_{r+1}^N\right) \ebd$\linebreak
 $\ebd 
 \int\nolimits_{t_r}^{t_{r+1}}X_s\,ds$~--- случайный вектор, $n$-я 
 компонента которого равна времени пребывания процесса~$X$ в~со\-сто\-янии~$e_n$ 
 на  интервале времени $[t_r, t_{r+1}]$. 
 Обозначим через $\mathcal{D}_{r+1} \ebd \{u=\mathrm{col}\,(u^1,\ldots,u^N):\; 
 u_m \hm\geqslant 0,\; \sum\nolimits_{m=1}^Mu_m\hm= t_{r+1}-t_r\}$ $(M-1)$-мер\-ный 
 симплекс в~пространстве~$\mathbb{R}^M$, являющийся носителем распределения 
 вектора~$\tau_{r+1}$. Пусть $\rho^{k,\ell}_{r+1}(du)$~--- 
 распределение вектора $\tau_{r+1} X_{t_{r+1}}^{\ell}$ при условии $X_{t_r}\hm=e_k$, 
 т.\,е.\ 
 для любого $\mathcal{A} \hm\in \mathcal{B}(\mathbb{R}^M)$ верно тождество:
\begin{multline*}
 \mathbf{P}\left\{\omega: \; X_{t_{r+1}}(\omega)=e_{\ell},\right.\\
 \left. 
 \tau_{r+1}(X(\omega)) \in \mathcal{A}\;|\;X_{t_r}=e_k\right\} \equiv
   \rho^{k,\ell}_{r+1}(\mathcal{A})\,.
\end{multline*}
 
Обозначим через
\begin{multline*}
 \mathcal{N}(y,m,K) \ebd (2\pi)^{-M/2} \mathrm{ det}^{-1/2} K\times{}\\
 {}\times\exp
 \left\{ -\fr{1}{2}\left(y-m\right)^{\top}K^{-1}(y-m)\right\}
\end{multline*}
 $M$-мер\-ную плот\-ность гауссовского распределения с~математическим 
 ожиданием~$m$ и~ковариационной матрицей~$K$.
 
 Из марковского свойства  $\{X_{t_{r}},Y_{r})\}_{r \geqslant 0}$ 
 относительно~${\mathcal{F}}_{t_{r}}$~\cite{ZhSh_95} и~теоремы Фубини следует, что 
 для любого  множества $\mathcal{A} \hm\in \mathcal{B}(\mathbb{R}^M)$ 
 верна следующая цепочка равенств:
 \begin{multline*}
 {\sf E}\left\{X_{t_{r+1}}\mathbf{I}_{\mathcal{A}}
 \left(Y_{r+1}\right)\big|\mathcal{O}_r\right\}={}\\
 {}=
{\sf E}\left\{{\sf E}\left\{X_{t_{r+1}}\mathbf{I}_{\mathcal{A}}
\left(Y_{r+1}\right)\big|
\mathcal{F}^X_{t_{r+1}} \vee \mathcal{O}_r\right\}
 \big|\mathcal{O}_r\right\} = {}
\end{multline*}

\noindent
\begin{multline*}
 %{}=
% {\sf E}\left\{{\sf E}\left\{X_{t_{r+1}}\mathbf{I}_{\mathcal{A}}
% \left(Y_{r+1}\right)\vert X_{t_r}\right\}
% \vert\mathcal{O}_r\right\} = {}\\
% {}=
%{\sf E}\left\{\sum\limits_{k=1}^N {\sf E}\left\{X_{t_{r+1}}\mathbf{I}_{\mathcal{A}}
%\left(Y_{r+1}\right)  \big| X_{t_r}=e_k\right\}X_{t_r}^k
% \big|\mathcal{O}_r\right\} = {}\\ 
% {}=
% \sum\limits_{k=1}^N{\sf E}
% \left\{X_{t_{r+1}}\mathbf{I}_{\mathcal{A}}\left(Y_{r+1}\right)\bigl| X_{t_r}=e_k\right\} 
% \widehat{X}_{t_r}^k ={}\\
% {}=\!
% \sum\limits_{k=1}^N{\sf E}
% \left\{{\sf E}\left\{X_{t_{r+1}}\mathbf{I}_{\mathcal{A}}
% \left(Y_{r+1}\right)\!\bigl| {\mathcal{F}}_{t_{r+1}}\right\}\!\bigl| 
% X_{t_r}\!=e_k\right\} \widehat{X}_{t_r}^k ={}\\
% {}=
% \sum\limits_{k=1}^N {\sf E}\left\{
% \vphantom{\int\limits_A\left(\sum\limits_{p=1}^N\right)}
% X_{t_{r+1}} \times{}\right.\\
% {}\times\int\limits_{\mathcal{A}}  
% \mathcal{N}\left(y,f \tau_{r+1}(X),\sum\limits_{p=1}^N \tau_{r+1}^p(X) g_pg_p^{\top}\right)dy
% \Biggl| X_{t_r}={}\\
%\left. {}=e_k
% \vphantom{\int\limits_A\left(\sum\limits_{p=1}^N\right)}
%\right\} \widehat{X}_{t_r}^k = 
% \sum\limits_{k=1}^N \int\limits_{\mathcal{A}}{\sf E}\left\{ 
% \vphantom{\sum\limits_{p=1}^N}
% X_{t_{r+1}} \times{}\right.\\
% {}\times\mathcal{N}\left(y,f \tau_{r+1}(X),\sum\limits_{p=1}^N \tau_{r+1}^p(X) 
% g_p g_p^{\top}\right)
% \Biggl| X_{t_r}={}\\
%\left. {}=e_k
%\vphantom{\sum\limits^N_{p=1}}
%\right\} \widehat{X}_{t_r}^k\, dy
 %={}\\
 {}=
 \sum\limits_{\ell=1}^N e_{\ell} \int\limits_{\mathcal{A}} 
 \left[ \sum\limits_{k=1}^N 
 \int\limits_{\mathcal{D}_{r+1}} 
 \mathcal{N}\left(y,f u,\sum_{p=1}^N u^p g_pg_p^{\top}\right)\times{}\right.\\
\left. {}\times
 \rho^{k,\ell}_{r+1}(du)\widehat{X}_{t_r}^k
 \vphantom{\int\limits_A\sum\limits_{p=1}^N}
 \right] 
 dy,
 \end{multline*}
 из чего следует, что интегранд в~квадратных скобках в~последнем выражении 
 определяет искомое совместное распределение $(X_{t_{r+1}},Y_{r+1})$ 
 относительно~$ \mathcal{O}_r$. Оценка~$\widehat{X}_{t_{r+1}}$ покомпонентно 
 определяется~\cite{BSh_85} с~помощью обобщенного варианта формулы Байеса:
 \begin{multline}
 \widehat{X}_{t_{r+1}}^j = {}\\
 \hspace*{-1mm}{}=
 \fr{\int\nolimits_{\mathcal{D}_{r+1}}\hspace*{-6mm} 
 \mathcal{N}\left(Y_{r+1},f u,\sum\nolimits_{p=1}^N \hspace*{-2mm}
 u^p g_pg_p^{\top}\!\right)\hspace*{-1mm}
 \sum\nolimits_{k=1}^N \hspace*{-2mm}
 \widehat{X}_{t_r}^k
 \rho^{k,j}_{r+1}(du)
 }
 { \int\nolimits_{\mathcal{D}_{r+1}} \hspace*{-6mm}
 \mathcal{N}\left(Y_{r+1},f v,\sum\nolimits_{q=1}^N \hspace*{-2mm}
 v^q g_qg_q^{\top}\!\right)\hspace*{-1mm}
 \sum\nolimits_{i,\ell=1}^N \hspace*{-2mm}
 \widehat{X}_{t_r}^i
 \rho^{i,\ell}_{r+1}(dv)
  },  \\ 
  j = \overline{1,N}\,.
 \label{eq:filt_1}
 \end{multline}
 Таким образом, доказана следующая
 
 %\smallskip
 
 \noindent
 \textbf{Лемма~1.}
\textit{Если для системы наблюдения}~(\ref{eq:obsys_1}) 
\textit{верны условия~(а) и~(б), то оценка~$\widehat{X}_t$ оптимальной фильтрации 
определяется формулой}~(\ref{eq:in_cond}) 
\textit{при $t\hm=0$, рекуррентным соотношением}~(\ref{eq:filt_1})~---
\textit{в~моменты~$t_{r+1}$ получения наблюдений~$Y_{r+1}$ 
и~формулой}~(\ref{eq:bw_obs})~--- 
\textit{в~промежутках времени между моментами получения наблюдений}.


\smallskip
 

 
 Несмотря на компактную запись~(\ref{eq:filt_1}), их прямая численная реализация 
 ресурсозатратна. Во-пер\-вых, в~(\ref{eq:filt_1}) требуется вычислять 
 распределения мас\-штаб\-но-сдви\-го\-вых смесей многомерных нормальных 
 распределений, что является трудоемкой\linebreak процедурой. Во-вто\-рых, 
 распределения~$\rho^{k,j}_{r+1}$ вре-\linebreak мени пребывания представляют собой 
 сумму\linebreak бесконечного ряда, слагаемые которого вычис\-ляются с~помощью 
 некоторой рекуррентной про\-це\-дуры~\cite{S_00}. В-третьих, 
 распределения~$\rho^{k,j}_{r+1}$ не являются абсолютно непрерывными 
 относительно меры Ле\-бега.
 { %\looseness=1
 
 }
 
 Следующий раздел посвящен численной аппроксимации~(\ref{eq:filt_1}) и~исследованию 
 ее точностных характеристик.
 
 \section{Приближенное вычисление оценки фильтрации}
 
 Без ограничения общности будем считать, что сетка~$\{t_r\}_{r \geqslant 0}$ 
 является равномерной с~шагом~$\Delta$, т.\,е.\ $t_r \hm= r \Delta$ 
 и~$\mathcal{D}_r \hm\equiv \mathcal{D}$.
 Обозначим через~$N_{r+1}$ об-\linebreak\vspace*{-12pt}
 
 \pagebreak
 
 \noindent
 щее число скачков процесса~$X_t$, имевших место 
 на промежутке $(t_r,t_{r+1}]$. Тогда из формулы полной вероятности следует, 
 что~(\ref{eq:filt_1}) представима в~виде:
 \begin{multline}
 \widehat{X}_{t_{r+1}}^j =  \left(
 \int\limits_{\mathcal{D}} 
 \mathcal{N}\left(Y_{r+1},f u,\sum\limits_{p=1}^N u^p g_pg_p^{\top}\right)\times{}\right.\\
\left. {}\times
 \sum\limits_{h=0}^{\infty}\sum\limits_{k=1}^N \widehat{X}_{t_r}^k
 \rho^{k,j,h}_{r+1}(du)
 \right)\Bigg/ \\
 \left(
 \vphantom{\sum\limits_{m=0}^{\infty}
 \sum\limits_{i,\ell=1}^N \widehat{X}_{t_r}^i
 \rho^{i,\ell,m}_{r+1}(dv)}
 \int\limits_{\mathcal{D}} 
 \mathcal{N}\left(Y_{r+1},f v,\sum\limits_{q=1}^N v^q g_qg_q^{\top}\right)\times{}\right.\\
\left.{}\times \sum\limits_{m=0}^{\infty}
 \sum\limits_{i,\ell=1}^N \widehat{X}_{t_r}^i
 \rho^{i,\ell,m}_{r+1}(dv)
 \right)
  \,, \enskip j = \overline{1,N}\,,
  \label{eq:filt_1_1}
 \end{multline}
 где 
 $ \rho^{k,j,h}_{r+1}(du)$~--- распределение вектора 
 $\tau_{r+1}X_{t_{r+1}}^{j}\mathbf{I}_{\{h\}}(N_{r+1})$ при 
 условии $X_{t_r}\hm=e_k$, т.\,е.\ 
 для любого $\mathcal{A} \hm\in \mathcal{B}(\mathbb{R}^M)$ верно тождество
\begin{multline*}
 \mathbf{P}\left\{\omega: \; X_{t_{r+1}}(\omega)=e_{j}, \; N_{r+1} = h,\right.\\ 
\left. \tau_{r+1}(X(\omega)) \in \mathcal{A}\;|\;X_{t_r}=e_k\right\} \equiv
  \rho^{k,j,h}_{r+1}(\mathcal{A}).
\end{multline*}
В качестве аппроксимации оценок можно использовать  
 $\overline{X}_{t_{r+1}}^n \ebd 
 \mathrm{col}\,(\overline{X}_{t_{r+1}}^{n,1},\ldots,\overline{X}_{t_{r+1}}^{n,N})$, 
 полученные из~(\ref{eq:filt_1_1}) путем урезания сумм ряда в~числителе и~знаменателе:
 
 \noindent
 \begin{multline}
 \overline{X}_{t_{r+1}}^{n,j} = 
 \left(
 \int\limits_{\mathcal{D}} 
 \mathcal{N}\left(Y_{r+1},f u,\sum\limits_{p=1}^N u^p g_pg_p^{\top}\right)\times{}\right.\\[-1pt]
\left.{}\times \sum\limits_{h=0}^{n}\sum\limits_{k=1}^N \overline{X}_{t_r}^k
 \rho^{k,j,h}_{r+1}(du)
 \right)\Bigg/ \\[-1pt]
 \left(
 \int\limits_{\mathcal{D}} 
 \mathcal{N}\left(Y_{r+1},f v,\sum\limits_{q=1}^N v^q g_qg_q^{\top}\right)\times{}\right.\\[-1pt]
\left. {}\times
 \sum\limits_{m=0}^{n}
 \sum\limits_{i,\ell=1}^N \overline{X}_{t_r}^i
 \rho^{i,\ell,m}_{r+1}(dv)
  \right)\,, \enskip
   j = \overline{1,N}.
  \label{eq:filt_2}
 \end{multline}
 Ниже по формуле полной вероятности получены интегралы из~(\ref{eq:filt_2}) для 
 $h\hm=0,1,2$:
 
\vspace*{-3pt}

 \noindent
  \begin{multline*}
 \int\limits_{\mathcal{D}}  \mathcal{N}
 \left(Y_{r+1},f u,\sum\limits_{p=1}^N u^p g_pg_p^{\top}\right) 
 \rho^{k,j,0}_{r+1}(du) = {}\\[-1pt]
 {}=
 \delta_{kj}\mathcal{N}\left(Y_{r+1},\Delta f^j,\Delta g_jg_j^{\top}\right)
 e^{\lambda_{jj}\Delta};
 %\label{eq:h0}
\\[-1pt]
 \int\limits_{\mathcal{D}}  \mathcal{N}\left(
 Y_{r+1},f u,\sum\limits_{p=1}^N u^p g_pg_p^{\top}\right) 
 \rho^{k,j,1}_{r+1}(du) ={} 
 \end{multline*}
 
 \noindent
 \begin{multline}
 \hspace*{-6.7pt}{}=\left(1-\delta_{kj}\right)\lambda_{kj}e^{\lambda_{jj}\Delta}
\! \int\limits_0^{\Delta}\!
 e^{(\lambda_{kk}-\lambda_{jj})u^k}
 \mathcal{N}\left(Y_{r+1},u^kf^k +{}\right.\hspace*{-0.28818pt}\\[-1pt]
\hspace*{-3mm}\left. {}+ \left(\Delta - u^k\right)f^j, u^k g_kg_k^{\top}+
 \left(\Delta-u^k\right)g_jg_j^{\top}\right)\,du^k;
 \label{eq:h1}
 \end{multline}
 
 \vspace*{-12pt}
 
 \noindent
 \begin{multline}
 \int\limits_D \mathcal{N}\left( 
Y_{r+1},f u,\sum\limits_{p=1}^N u^p g_pg_p^{\top}\right)du ={}\\[-1pt]
{}=
\sum\limits_{\substack{{\ell:\ell \neq k,}\\ {\ell \neq j}}}
 \lambda_{k\ell}\lambda_{\ell j} e^{\lambda_{jj}\Delta}\times {}\\[-1pt] 
 {}\times
 \int\limits_0^{\Delta} \int\limits_0^{\Delta-u^k} \!
e^{(\lambda_{kk}-\lambda_{\ell\ell})u^k+(\lambda_{\ell\ell}-
 \lambda_{jj})u^{\ell}}\times{} \\[-1pt] 
{}  \times
 \mathcal{N}\left(Y_{r+1},u^k f^k+u^{\ell}f^{\ell}+\left(
 \Delta-u^k-u^{\ell} \right)f^j,\right.\\[-1pt]
 \hspace*{-1mm}\left.
 u^k g_kg_k^{\top}+u^{\ell}g_{\ell}g_{\ell}^{\top}+\left(
 \Delta-u^k-u^{\ell} \right)
 g_jg_j^{\top}
 \right) du^{\ell}du^{k}, \!\!
  \label{eq:h2}
 \end{multline} 
 
\vspace*{-2pt}
 
 \noindent
  где  $\delta_{ij}$~--- символ Кронекера. Интегралы для $h\hm>2$ также могут 
  быть получены в~явном виде, однако их сложность резко возрастает.
 

   Так как система~(\ref{eq:obsys_1}) является автономной, то в~качестве локальной 
   характеристики бли\-зости~$\{\overline{X}_{t_r}\}$ 
   к~$\{\widehat{X}_{t_r}\}$ может быть выбрана величина
   
\noindent
 \begin{multline*}
 \overline{\sigma}(\pi) \ebd {\sf E}\left\{
 \|\widehat{X}_{t_{1}}(\pi, Y_{1}) - \overline{X}_{t_{1}}
 \left(\pi,Y_{1}\right)\|_{1}\right\} = {}\\
 {}=
 \sum\limits_{j=1}^N{\sf E}
 \left\{\left\vert \widehat{X}^j_{t_{1}}\left(\pi, Y_{1}\right) - \overline{X}^{n,j}_{t_{1}}
 \left(\pi,Y_{1}\right)\right\vert\right\}.
 %\label{eq:prec_1}
 \end{multline*}
 При этом начальное распределение $\pi \hm\in \mathcal{D}_1 \ebd $\linebreak $\ebd
 \{\mathrm{col}\,(\pi^1,\ldots,\pi^N):\;\pi^j > 0$, 
 $\sum\nolimits_{j=1}^N\pi^j\hm=1\}$ является начальным условием применения 
 одного шага рекурсии~(\ref{eq:filt_1}) или~(\ref{eq:filt_2}) для вычисления 
 оценки~$\widehat{X}_{t_{1}}$
   или~$\overline{X}_{t_{1}}$ соответственно. Фактически, 
 характеристика~$\overline{\sigma}(\pi)$ определяет, насколько сильно 
 рекурсивные схемы~(\ref{eq:filt_1}) и~(\ref{eq:filt_2}) разойдутся за 
 один шаг, стартуя из общей точки~$\pi$.
 
 Рекуррентные схемы~(\ref{eq:filt_1}) и~(\ref{eq:filt_2}), примененные~$r$~раз, 
 позволяют вычислить оценки~$\widehat{X}_{t_r}$ и~$\overline{X}_{t_r}$ 
 в~точке~$t_r$. В~качестве характеристики точности глобальной аппроксимации в~этом 
 случае естественно рассмотреть величину
 
 \vspace*{-2pt}
 
 \noindent
 \begin{equation*}
 \overline{\Sigma}_{t_r}(\pi) \ebd {\sf E}
 \left\{\|\widehat{X}_{t_{r}} - \overline{X}_{t_{r}}\|_{1}\right\} = 
 \!\sum\limits_{j=1}^N\!{\sf E}
 \left\{\left\vert \widehat{X}^j_{t_{r}} - 
 \overline{X}^{n,j}_{t_{r}}\right\vert \right\}.
% \label{eq:prec_2}
 \end{equation*}
 
 Следующее утверждение определяет оценки локальной и~глобальной 
 точности схемы аппроксимации~(\ref{eq:filt_2}).
 
 %\smallskip
 
 \noindent
 \textbf{Теорема~1.}\
\textit{Выполняются неравенства} 

%\vspace*{-2pt}

\noindent
 \begin{equation}
 \sup_{\pi \in \mathcal{D}_1} \overline{\sigma}(\pi) 
 \leqslant 2 \fr{(\overline{\lambda}\Delta)^{n+1}}{(n+1)!}\,;
 \label{eq:prec_loc}
\end{equation}

\noindent
\begin{align}
  \sup\limits_{\pi \in \mathcal{D}_1} \overline{\Sigma}_{t_r}(\pi)
   &\leqslant 2r \fr{(\overline{\lambda}\Delta)^{n+1}}{(n+1)!} +{}\notag\\[-0.5pt]
   &\hspace*{-20mm}{}+
  r(r-1)\left(
  \fr{(\overline{\lambda}\Delta)^{n+1}}{(n+1)!}
  \right)^2
  \left(
  1-\fr{(\overline{\lambda}\Delta)^{n+1}}{(n+1)!}
  \right)^{r-2},
 \label{eq:prec_glob}
 \end{align}
 
 \vspace*{-2pt}
 
 \noindent
 \textit{где} $\overline{\lambda} \ebd \max_{1 \leqslant j \leqslant N}|\lambda_{jj}|$.


%\smallskip

 Доказательство теоремы~1 приведено в~приложении.
 
 Данное утверждение представляет полезные оценки точности. Во-пер\-вых, 
 они являются равномерными по начальному распределению $\pi \hm\in \mathcal{D}_1$. 
 Во-вто\-рых, оценки носят универсальный, а~не асимптотический характер. Это 
 существенно в~практических задачах оценивания по дискретизованным 
 наблюдениям с~физическими или алгоритмическими ограничениями на шаг 
 по времени. Например, в~случае наблюдаемого процесса восстановления в~силу 
 центральной предельной теоремы для процессов восстановления~\cite{B_80} его
  приращения можно рассматривать как гауссовские случайные величины. 
  Однако данная аппроксимация обладает удовлетворительной точностью 
  только в~случае, когда шаг дискретизации по времени достаточно большой. 
 %
 В-третьих, неравенство~(\ref{eq:prec_glob}) позволяет получить порядок 
 аппроксимации при $\Delta \hm\to 0$. Зафиксируем момент времени $t\hm=T$ и~рассмотрим 
 характеристику $\sup\nolimits_{\pi \in \mathcal{D}_1} 
 \overline{\Sigma}_{T}(\pi)$ при $r\hm={T}/{\Delta}$ и~$\Delta \hm\to 0$. 
 Как только~$\Delta$ становится настолько мало, что 
 $\max\left({(\overline{\lambda}\Delta)^{n+1}}/{(n+1)!}, 
 \Delta ({T\lambda^{n+1}}/{(n+1)!})\right)\hm< 1$, из~(\ref{eq:prec_glob}) 
 следует неравенство
  %\begin{equation}
  $\sup\nolimits_{\pi \in \mathcal{D}_1} \overline{\Sigma}_{T}(\pi) 
  \hm\leqslant  ({3\overline{\lambda}^{n+1}}/{(n+1)!}) T\Delta^n.$
 %\label{eq:prec_asympt}
 %\end{equation}
 Это значит, что с~ростом времени~$T$ 
 ошибка аппроксимации копится пропорционально~$T$ и~при этом порядок точности 
 по~$\Delta$ равен~$n$.
 
 %\vspace*{-7pt}
 
  \section{Заключение}
  
  \vspace*{-4pt}
 
  В работе решена задача оценивания состояния однородного МСП по 
  дискретизованным наблюдениям. Получено аналитическое решение и~его 
  чис\-лен\-ные аппроксимации. Локальные и~глобальные показатели точ\-ности этих 
  приближений в~статье так\-же пред\-став\-ле\-ны. Примечательно, что  част\-ный случай 
  аппроксимаций~(\ref{eq:filt_2}) при $n\hm=0$ и~$\Lambda\hm=0$ был ранее 
  пред\-став\-лен в~\cite{B_17_1,B_17_2} для решения задачи байесовской классификации 
  случайного вектора по непрерывным наблюдениям с~мультипликативными шумами. 
 % 
Алгоритм оптимальной фильт\-ра\-ции и~его субоптимальные версии могут 
рас\-смат\-ри\-вать\-ся в~качестве основы чис\-лен\-ной реализации обобщения фильт\-ра 
Вонэма для сис\-тем с~мультипликативными шумами в~наблюдениях. 
Однако для их непосредственного использования необходимо решить 
следующие проб\-ле\-мы. Во-пер\-вых, в~(\ref{eq:h1}) и~(\ref{eq:h2}) присутствуют
 многомерные интегралы. Следует выяснить, какую результирующую погрешность 
 будут вносить ошибки их вы\-чис\-ле\-ния. Во-вто\-рых, представляется интересным 
 определить характеристики точ\-ности оптимальной фильт\-ра\-ции по дискретизованным 
 наблюдениям по отношению к~оптимальной фильт\-ра\-ции по непрерывным наблюдениям: 
 каков порядок точ\-ности по шагу временной дискретизации~$\Delta$? Для случая 
 вы\-чис\-ле\-ния классического фильт\-ра Вонэма с~по\-мощью алгоритма Эй\-ле\-ра--Ма\-ру\-ямы 
 подобный результат известен: порядок глобальной ошибки равен~${1}/{2}$. 
 Перечисленные задачи являются предметом дальнейших исследований.
 
 
  \vspace*{-10pt}
 
{\small
\subsection*{\raggedleft Приложение} 

\vspace*{-2pt}


\noindent
Д\,о\,к\,а\,з\,а\,т\,е\,л\,ь\,с\,т\,в\,о\ \ теоремы~1.\ \ Введем следующие 
обозначения для случайных величин и~мат\-риц, составленных из них:
\begin{align*}
\xi^{ji}(\ell)&\ebd 
\sum\limits_{h=0}^n \int\limits_{\mathcal{D}} 
 \mathcal{N}\left(Y_{\ell},f u,\sum\limits_{p=1}^N u^p g_pg_p^{\top}\right)
 \rho^{j,i,h}_{1}(du)\,; \\
  \theta^{ji}(\ell)&\ebd 
\sum\limits_{h=n+1}^{\infty} \int\limits_{\mathcal{D}} 
 \mathcal{N}\left(Y_{\ell},f u,\sum\limits_{p=1}^N u^p g_pg_p^{\top}\right)
 \rho^{j,i,h}_{1}(du)\,;
\\
 \xi(\ell)&\ebd \|\xi^{ji}(\ell)\|_{j,i=\overline{1,N}}\,,\quad 
 \Xi(r) \ebd \xi(r) \xi(r-1)\cdots \xi(1)\,;
 \\
 \theta(\ell)&\ebd \|\theta^{ji}(\ell)\|_{j,i=\overline{1,N}}\,, \quad 
 \Theta(r) \ebd \theta(r) \theta(r-1)\cdots \theta(1)\,.
%\label{eq:not_1}
\end{align*}
 
 Рекуррентные формулы~(\ref{eq:filt_1}) и~(\ref{eq:filt_2}) можно записать в~явной 
 форме
 
 
\noindent
\begin{align*}
 \widehat{X}_{t_r}& = \left( \mathbf{1}\left(\Xi(r) + 
 \Theta(r)\right)\pi\right)^{-1} \left(\Xi(r) + \Theta(r)\right)\pi\,;
\\
 \overline{X}_{t_r} &= \left( \mathbf{1}\Xi(r)\pi\right)^{-1} \Xi(r) \pi,
\end{align*}

\vspace*{-2pt}

\noindent
где $\mathbf{1} \ebd (1,\ldots,1)$~--- век\-тор-стро\-ка 
подходящей раз\-мер\-ности, составленная из единиц.

%Далее для краткости записи зависимость от~$r$ в~обозначениях~$\Xi(r)$ 
%и~$\Theta(r)$ будет опущена. 
Верна следующая цепочка неравенств:

 \vspace*{-3pt}

\noindent
\begin{multline}
\overline{\Sigma}_{t_r}(\pi)=%
%\me{}{\left\| 
%\widehat{X}_{t_r}(\pi, Y_1,\ldots,Y_r) - \overline{X}_{t_r}(\pi, Y_1,\ldots,Y_r)
%\right\|_1} =\\=
{\sf E}\left\{\left\| 
\fr{1}{\mathbf{1}\left(\Xi(r) + \Theta(r)\right)\pi} \left(\Xi(r) +{}\right.\right.\right.\\[-1pt]
\left.\left.\left.{}+ \Theta(r)\right)\pi
- \fr{1}{\mathbf{1}\Xi(r)\pi}\,\Xi(r) \pi
\right\|_1\right\} ={} \\[-1pt]
{}=
{\sf E}\left\{\fr{1}{\mathbf{1}\left(\Xi(r) + \Theta(r)\right)\pi \mathbf{1}\Xi(r)\pi}
\left\|
 \mathbf{1}\Xi(r) \pi \Theta(r)\pi -{}\right.\right.\\[-1pt]
\left.\left. {}- \mathbf{1}\Theta(r)\pi \Xi(r) \pi
 \right\|_1
 \vphantom{\fr{1}{\mathbf{1}\left(\Xi(r) + \Theta(r)\right)\pi \mathbf{1}\Xi(r)\pi}}
\right\} \leqslant {}\\[-1pt]
{}\leqslant 
{\sf E}\left\{\fr{1}{\mathbf{1}\left(\Xi(r) + \Theta(r)\right)\pi \mathbf{1}\Xi(r)\pi}
\left(
\mathbf{1}\Xi(r)\pi \| \Theta(r)\pi \|_1 +{}\right.\right.\\[-1pt]
\left.\left.{}+ \mathbf{1}\Theta(r)\pi 
\|
\Xi(r) \pi
\|_1
\right)
 \vphantom{\fr{1}{\mathbf{1}\left(\Xi(r) + \Theta(r)\right)\pi \mathbf{1}\Xi(r)\pi}}
\right\} ={}\\[-1pt]
{}=
2\,{\sf E}\left\{\fr{1}{\mathbf{1}\left(\Xi(r) + \Theta(r)\right)\pi}\mathbf{1}\Theta(r)\pi 
\right\}.
\label{eq:ineq_1}
\end{multline}

 
 \noindent
 Рассмотрим случайные события $a_{\ell} \ebd \{\omega \in \Omega: 
 N_{\ell}(\omega) \hm\leqslant n\}$, $\ell \hm= \overline{1,r}$, и~$A_r \ebd \{
 \omega\hm \in \Omega: \max_{1 \leqslant {\ell} \leqslant r}N_{\ell}(\omega) 
 \hm\leqslant n
 \}\hm=\prod\nolimits_{\ell=1}^r a_{\ell}$ и~оценку 
 $
 \widetilde{X}_{t_r}(\pi, Y_1,\ldots,Y_r)\ebd$\linebreak $\ebd
 {\sf E}\left\{X_{t_r}(\omega)\mathbf{I}_{A_r}(\omega)|\mathcal{O}_r\right\}.
 $
 Используя введенные выше обозначе\-ния и~абстрактный вариант формулы Байеса, 
 получаем, что
 
 \noindent
\begin{align}
\widetilde{X}_{t_r}& = \fr{1}{{\mathbf{1}\left(\Xi(r) + 
 \Theta(r)\right)\pi}}\,\Xi(r)\pi\,;\notag
 \\
\widehat{X}_{t_r} - \widetilde{X}_{t_r} &=
{\sf E}\left\{X_{t_r}(\omega)\mathbf{I}_{\overline{A}_r}(\omega)|\mathcal{O}_r\right\} ={}\notag\\[-1pt]
&\hspace*{17mm}{}= 
\fr{1}{\mathbf{1}\left(\Xi(r) + \Theta(r)\right)\pi}\Theta(r)\pi\,. 
\label{eq:eq_2}
 \end{align}
 Из (\ref{eq:ineq_1}) и~(\ref{eq:eq_2}) для $r\hm=1$ следует, что
 
 \vspace*{-4pt}
 
 \noindent
 \begin{multline}
 \overline{\sigma}(\pi) \leqslant 2\,{\sf E}
 \left\{\|{\sf E}\left\{X_{t_1}(\omega)\mathbf{I}_{\overline{a}_1}(\omega)|\mathcal{O}_1
 \right\}\|_1
 \right\} ={}\\[-1.5pt]
 {}=
 2\,{\sf E}\left\{\sum\limits_{n=1}^N {\sf E}
 \left\{X^n_{t_1}(\omega)\mathbf{I}_{\overline{a}_1}
 (\omega)|\mathcal{O}_1\right\}\right\} ={} \\[-2pt] 
 {}=
  2\,{\sf E}\left\{{\sf E}\left\{\mathbf{I}_{\overline{a}_1}(\omega)|\mathcal{O}_1
  \right\}\right\} =
   2 \mathbf{P}\left\{\overline{a}_1(\omega)\right\}.
\label{eq:ineq_3}
\end{multline}

 \vspace*{-2pt}
 
 \noindent
 Процесс $N^X_t$ общего числа скачков состояния~$X_t$ является считающим, и~его
  квадратическая характеристика равна 
  
\vspace*{-2pt}
  
  \noindent
 $$
 \langle N^X, N^X\rangle_t = - \int\limits_0^t \sum\limits_{n=1}^N \lambda_{nn} X_s^n\,ds\,,
 $$
 поэтому искомая вероятность ограничена сверху:
 $$ 
 \mathbf{P}\left\{\overline{a}_1(\omega)\right\} \leqslant 
 e^{-\overline{\lambda}\Delta}\sum\limits_{k=n+1}^{\infty} 
 \fr{(\overline{\lambda}\Delta)^{k}}{k!} <
 \fr{(\overline{\lambda}\Delta)^{n+1}}{(n+1)!}.
 $$
 
  \vspace*{-2pt}
  
  \noindent
 Из последнего неравенства и~(\ref{eq:ineq_3}) следует, что  для любого 
 начального распределения~$\pi$ выполняется неравенство $\overline{\sigma}(\pi)  
 \hm< 2({(\overline{\lambda}\Delta)^{n+1}}/{(n+1)!})$, т.\,е.\ 
 локальная оценка~(\ref{eq:prec_loc}) верна.
 
 С помощью марковского свойства пары $(X_t, N^X_t)$ и~последнего 
 неравенства можно оценить сверху вероятность 
 $\mathbf{P}\left\{\overline{A}_r(\omega)\right\}$:
 
  \vspace*{-2pt}
 
 \noindent
 \begin{multline*}
 \mathbf{P}\left\{\overline{A}_r(\omega)\right\} \leqslant 1 - \left(
 1- \fr{(\overline{\lambda}\Delta)^{n+1}}{(n+1)!}
 \right)^r \leqslant r \fr{(\overline{\lambda}\Delta)^{n+1}}{(n+1)!} + {}\\[-1pt]
 {}+\left|
 \sum\limits_{k=2}^r C_r^k \left(-\fr{(\overline{\lambda}\Delta)^{n+1}}{(n+1)!}
 \right)^k
 \right| \leqslant
 r \fr{(\overline{\lambda}\Delta)^{n+1}}{(n+1)!} +{}\\[-1pt]
 {}+\fr{r(r-1)}{2}
 \left(
 \fr{(\overline{\lambda}\Delta)^{n+1}}{(n+1)!}
 \right)^2
 \left(
 1-\fr{(\overline{\lambda}\Delta)^{n+1}}{(n+1)!}
 \right)^{r-2},
 \end{multline*} 
 из чего следует истинность глобальной оценки~(\ref{eq:prec_glob}).
Теорема~1 доказана.

}

%\vspace*{-12pt}

{\small\frenchspacing
 {%\baselineskip=10.8pt
 \addcontentsline{toc}{section}{References}
 \begin{thebibliography}{99}

\bibitem{Won_65}
\Au{Wonham W.} 
Some applications of stochastic differential equations to optimal
  nonlinear filtering~//
SIAM~J.~Control, 1965. Vol.~2. P.~347--369. 

\bibitem{KP_92}
\Au{Kloeden P., Platen E.} Numerical solution of stochastic
differential equations.~--- Berlin: Springer, 1992.~636~p.

\bibitem{YZL_04}
\Au{Yin G., Zhang Q., Liu Y.} 
Discrete-time approximation of Wonham filters~//
J.~Control Theory Applications, 2004. Iss.~2. P.~1--10.

\bibitem{PR_10}
\Au{Platen E., Rendek R.}
Quasi-exact approximation of hidden Markov chain filters~//
Communicat.~Stoch.~Analys., 2010. Vol.~4. Iss.~1. P.~129--142.

\bibitem{B_18}
\Au{Борисов А.} Фильтрация Вонэма по наблюдениям с~мультипликативными шумами~// 
Автоматика и~телемеханика, 2018.
№~1. C.~52--65. 
 
  \bibitem{BSh_85} %6
\Au{Бертсекас Д., Шрив С.} Стохастическое оптимальное управление. 
Случай дискретного времени~/ Пер. с~англ.~--- М.: Наука, 1985.~280~c.
(\Au{Betsekas~D.\,P., Shreve~S.\,E.} Stochastic optimal control:
The discrete-time case.~--- Orlando, FL, USA:
Academic Press Inc., 1978. 323~p.)

  \bibitem{ZhSh_95} %7
\Au{Жакод Ж., Ширяев А.} Предельные теоремы для случайных процессов,~I.~/
Пер. с~англ.~--- 
М.: Физматлит, 1995.~544~c.
(\Au{Jacod~J., Shiryaev~A.} Limit theorems for stochastic processes.~---
Berlin: Springer, 2003. 664~p.)

\bibitem{S_00}
\Au{Sericola B.} Occupation times in Markov processes~//
Commun. Stat. Stochastic Models, 2000. Vol.~16. Iss.~5. P.~479--510. 

  \bibitem{B_80}
\Au{Боровков А.} Асимптотические методы в~тео\-рии массового обслуживания.~--- 
М.: Физматлит, 1995.~384~c.

  \bibitem{B_17_1}
\Au{Борисов А.} Классификация по непрерывным наблюдениям с~мультипликативными шумами.~I. 
Формулы байесовской оценки~// Информатика и~её применения, 2017. Т.~11. Вып.~1. C.~11--19.
doi: 10.14357/19922264170102.

  \bibitem{B_17_2}
\Au{Борисов А.} Классификация по непрерывным наблюдениям с~мультипликативными 
шумами.~II. Алгоритм численной реализации оценки~// Информатика и~её 
применения, 2017. Т.~11. Вып.~2. C.~33--41.
doi: 10.14357/19922264170204.

 \end{thebibliography}

 }
 }

\end{multicols}

\vspace*{-4pt}

\hfill{\small\textit{Поступила в~редакцию 10.07.18}}

\vspace*{6pt}

%\pagebreak

%\newpage

%\vspace*{-28pt}

\hrule

\vspace*{2pt}

\hrule

%\vspace*{-2pt}

\def\tit{FILTERING OF~MARKOV JUMP PROCESSES\\ BY~DISCRETIZED OBSERVATIONS}

\def\titkol{Filtering of Markov jump processes by discretized observations}

\def\aut{A.\,V.~Borisov}

\def\autkol{A.\,V.~Borisov}

\titel{\tit}{\aut}{\autkol}{\titkol}

\vspace*{-11pt}


\noindent
Institute of Informatics Problems, Federal Research Center ``Computer Science 
and Control'' of the Russian Academy of Sciences, 44-2~Vavilov Str., Moscow 
119333, Russian Federation


\def\leftfootline{\small{\textbf{\thepage}
\hfill INFORMATIKA I EE PRIMENENIYA~--- INFORMATICS AND
APPLICATIONS\ \ \ 2018\ \ \ volume~12\ \ \ issue\ 3}
}%
 \def\rightfootline{\small{INFORMATIKA I EE PRIMENENIYA~---
INFORMATICS AND APPLICATIONS\ \ \ 2018\ \ \ volume~12\ \ \ issue\ 3
\hfill \textbf{\thepage}}}

\vspace*{6pt}



\Abste{The article is devoted to a~solution of the optimal filtering problem 
of a~homogenous Markov
jump process state. The available observations represent 
time increments of the integral transformations of the Markov\linebreak\vspace*{-12pt}}

\Abstend{state corrupted by 
Wiener processes. The noise intensity is also state-dependent. At the instant of 
the consecutive
observation obtaining, the optimal estimate is calculated recursively 
as a~function of previous estimate and the new observation, meanwhile between 
observations the filtering estimate is a simple forecast by virtue of the Kolmogorov 
differential system. The recursion is rather expensive because of  need to calculate 
the integrals, which are the location-scale mixtures of Gaussians. The mixing 
distributions represent the occupation of the state in each of possible values 
during the mid-observation intervals. The paper contains numerically cheaper 
approximations, based on the restriction of the state transitions number between 
the observations. Both the local and global characteristics of approximation 
accuracy are obtained as functions of the dynamics parameters, mid-observation 
interval length, and upper bound of transitions number.}

\KWE{Markov jump process; optimal filtering; multiplicative observation noises; 
stochastic differential equation; numerical approximation}




\DOI{10.14357/19922264180316}

%\vspace*{-14pt}

\Ack
\noindent
The work was supported in part by the Russian Foundation
for Basic Research (Project No.\,16-07-00677).



%\vspace*{6pt}

  \begin{multicols}{2}

\renewcommand{\bibname}{\protect\rmfamily References}
%\renewcommand{\bibname}{\large\protect\rm References}

{\small\frenchspacing
 {%\baselineskip=10.8pt
 \addcontentsline{toc}{section}{References}
 \begin{thebibliography}{99}
\bibitem{Won_65-1}
\Aue{Wonham, W.} 1965.
Some applications of stochastic differential equations to optimal
  nonlinear filtering.
\textit{SIAM~J.~Control} 2:347--369. 

\bibitem{KP_92-1}
\Aue{Kloeden,~P., and E.~Platen.} 1992. \textit{Numerical solution of stochastic
differential equations.} Berlin: Springer. 636~p.

\bibitem{YZL_04-1}
\Aue{Yin,~G., Q.~Zhang, and Y.~Liu.} 2004.
Discrete-time approximation of Wonham filters.
\textit{J.~Control Theory Applications} 2:1--10.

\bibitem{PR_10-1}
\Aue{Platen, E., and R.~Rendek.} 2010.
Quasi-exact approximation of hidden Markov chain filters.
\textit{Communicat. Stoch. Analys.} 4(1):129--142.

\bibitem{B_18-1}
\Aue{Borisov, A.} 2018. Wonham filtering by observations
with multiplicative noises. \textit{Automat.~Rem.~Contr.} 79(1):39--50.  
doi: 10.1134/ S0005117918010046.
 
  \bibitem{BSh_85-1}
\Aue{Bertsekas, D., and S.~Shreve.} 1996.
\textit{Stochastic optimal control: The discrete-time case}.
Nashua, NH: Athena Scientific. 330~p.
  
  \bibitem{ZhSh_95-1}
  \Aue{Jacod,~J., and A.~Shiryaev.} 2003.
\textit{Limit theorems for stochastic processes.}
Berlin: Springer. 664~p.

\bibitem{S_00-1}
\Aue{Sericola, B.}
2000. Occupation times in Markov processes.
\textit{Commun. Stat.} 16(5):479--510. 

  \bibitem{B_80-1}
\Aue{Borovkov, A.} 1984.
 \textit{Asymptotic methods in queueing theory}. 
 Hoboken, NJ: Wiley-Blackwell.~304~p.

  \bibitem{B_17_1-1}
  \Aue{Borisov, A.} 2017. 
  Klassifikatsiya po ne\-pre\-ryv\-nym nablyu\-de\-miyam s~mul'tiplikativnymi shumami. I. 
  Formuly bayesov\-skoy otsenki [Classification by continuous-time observations
in multiplicative noise. I.~Formulae for Bayesian 
estimate]. \textit{Informatika i~ee Primeneniya~--- Inform.~Appl.}
11(1):11--19. doi: 10.14357/19922264170102.

  \bibitem{B_17_2-1}
\Aue{Borisov, A.} 2017. Klassifikatsiya po nepreryvnym nablyudemiyam 
s~mul'tiplikativnymi summami. II.~Formuly bayesovskoy otsenki 
[Classification by continuous-time observations
in multiplicative noise. II.~Numerical algorithm].
\textit{Informatika i~ee Primeneniya~--- Inform.~Appl.}
11(2):33--41. doi: 10.14357/19922264170204.

\end{thebibliography}

 }
 }

\end{multicols}

\vspace*{-6pt}

\hfill{\small\textit{Received July 10, 2018}}

%\pagebreak

%\vspace*{-18pt}

\Contrl

\noindent
\textbf{Borisov Andrey V.} (b.\ 1965)~--- 
Doctor of Science in physics and mathematics, principal scientist, Institute of
Informatics Problems, Federal Research Center ``Computer Science and Control''
 of the Russian Academy of
Sciences, 44-2 Vavilov Str., Moscow 119333, Russian Federation; 
\mbox{aborisov@frccsc.ru}
\label{end\stat}

\renewcommand{\bibname}{\protect\rm Литература}         %5

\include{kush1} %6 есть

\include{obchak}

\def\stat{cont}
{%\hrule\par
%\vskip 7pt % 7pt
\raggedleft\Large \bf%\baselineskip=3.2ex
А\,В\,Т\,О\,Р\,С\,К\,И\,Й\ \ У\,К\,А\,З\,А\,Т\,Е\,Л\,Ь\ \ З\,А\ \ 2\,0\,0\,7 г. \vskip 17pt
    \hrule
    \par
\vskip 21pt plus 6pt minus 3pt }

\label{st\stat}

\def\tit{\ }

\def\aut{\ }
\def\auf{\ }

\def\leftkol{\ } % ENGLISH ABSTRACTS}

\def\rightkol{\ } %ENGLISH ABSTRACTS}

\titele{\tit}{\aut}{\auf}{\leftkol}{\rightkol}


\contentsline {chapter}{\ }{Выпуск \quad Стр.} 
\contentsline {section}{\textbf{Батракова Д.\,А., Королев В.\,Ю., Шоргин С.\,Я.}\ \ Новый метод вероятностно-ста\-ти\-сти\-че\-ско\-го анализа информационных потоков в\nobreakspace {}телекоммуникационных сетях}{\qquad 1 \qquad 40} 
\contentsline {section}{\textbf{Борисов А.\,В.}\ \ Байесовское оценивание в системах наблюдения с\nobreakspace {}марковскими скачкообразными процессами: игровой подход}{\qquad 2 \qquad 65}
\contentsline {section}{\textbf{Босов А.\,В., Иванов А.\,В.}\ \ Программная инфраструктура информационного Web-пор\-тала}{\qquad 2 \qquad 50}
\contentsline {section}{\textbf{Захаров В.\,Н., Калиниченко Л.\,А., Соколов И.\,А., Ступников С.\,А.}\ \ Конструирование канонических информационных моделей для интегрированных информационных систем}{\qquad 2 \qquad 15}
\contentsline {section}{\textbf{Захаров В.\,Н., Козмидиади В.\,А.}\ \ Средства обеспечения отказоустойчивости при\-ло\-жений}{\qquad 1 \qquad 14} 
\contentsline {section}{\textbf{Иванов А.\,В.}\ \ см. Босов А.\,В.\hfill\hfill\hfill\hfill\hfill\hfill\hfill\hfill\hfill\hfill\hfill\hfill\hfill\hfill\hfill\hfill\hfill\hfill\hfill\hfill\hfill\hfill\hfill\hfill\hfill\hfill\hfill\hfill\hfill\hfill\hfill\hfill\hfill\hfill\hfill}{\ }
\contentsline {section}{\textbf{Ильин В.\,Д., Соколов И.\,А.}\ \ Символьная модель системы знаний информатики в\nobreakspace {}че\-ло\-ве\-ко-автоматной среде}{\qquad 1 \qquad 66} 
\contentsline {section}{\textbf{Калиниченко Л.\,А.}\ \ см. Захаров В.\,Н.\hfill\hfill\hfill\hfill\hfill\hfill\hfill\hfill\hfill\hfill\hfill\hfill\hfill\hfill\hfill\hfill\hfill\hfill\hfill\hfill\hfill\hfill\hfill\hfill\hfill\hfill\hfill\hfill\hfill\hfill\hfill\hfill\hfill\hfill\hfill}{\ }
\contentsline {section}{\textbf{Козеренко Е.\,Б.}\ \ Лингвистическое моделирование для систем машинного перевода и обработки знаний}{\qquad 1 \qquad 54} 
\contentsline {section}{\textbf{Козмидиади В.\,А.}\ \ см. Захаров В.\,Н.\hfill\hfill\hfill\hfill\hfill\hfill\hfill\hfill\hfill\hfill\hfill\hfill\hfill\hfill\hfill\hfill\hfill\hfill\hfill\hfill\hfill\hfill\hfill\hfill\hfill\hfill\hfill\hfill\hfill\hfill\hfill\hfill\hfill\hfill\hfill }{\ } 
\contentsline {section}{\textbf{Королев В.\,Ю.}\ \ см. Батракова Д.\,А.\hfill\hfill\hfill\hfill\hfill\hfill\hfill\hfill\hfill\hfill\hfill\hfill\hfill\hfill\hfill\hfill\hfill\hfill\hfill\hfill\hfill\hfill\hfill\hfill\hfill\hfill\hfill\hfill\hfill\hfill\hfill\hfill\hfill\hfill\hfill}{\ } 
\contentsline {section}{\textbf{Кудрявцев А.\,А., Шоргин С.\,Я.}\ \ Байесовский подход к\nobreakspace {}анализу систем массового обслуживания и\nobreakspace {}показателей надежности}{\qquad 2 \qquad 76}
\contentsline {section}{\textbf{Печинкин А.\,В., Соколов И.\,А., Чаплыгин В.\,В.}\ \ Многолинейная система массового обслуживания с конечным накопителем и ненадежными приборами}{\qquad 1 \qquad 27} 
\contentsline {section}{\textbf{Печинкин А.\,В., Соколов И.\,А., Чаплыгин В.\,В.}\ \ Стационарные характеристики многолинейной\nobreakspace {}системы массового обслуживания с\nobreakspace {}одновременными отказами приборов}{\qquad 2 \qquad 39}
\contentsline {section}{\textbf{Синицын И.\,Н.}\ \ Корреляционные методы построения аналитических информационных моделей флуктуаций полюса Земли по априорным данным}{\qquad 2 \qquad \hphantom{9}2}
\contentsline {section}{\textbf{Синицын И.\,Н.}\ \ Развитие теории фильтров Пугачева для оперативной обработки информации в стохастических системах}{{\qquad 1 \qquad \hphantom{9}3}} 
\contentsline {section}{\textbf{Соколов И.\,А.}\ \ см. Захаров В.\,Н.\hfill\hfill\hfill\hfill\hfill\hfill\hfill\hfill\hfill\hfill\hfill\hfill\hfill\hfill\hfill\hfill\hfill\hfill\hfill\hfill\hfill\hfill\hfill\hfill\hfill\hfill\hfill\hfill\hfill\hfill\hfill\hfill\hfill\hfill\hfill}{\ }
\contentsline {section}{\textbf{Соколов И.\,А.}\ \ см. Ильин В.\,Д.\hfill\hfill\hfill\hfill\hfill\hfill\hfill\hfill\hfill\hfill\hfill\hfill\hfill\hfill\hfill\hfill\hfill\hfill\hfill\hfill\hfill\hfill\hfill\hfill\hfill\hfill\hfill\hfill\hfill\hfill\hfill\hfill\hfill\hfill\hfill}{\ } 
\contentsline {section}{\textbf{Соколов И.\,А.}\ \ см. Печинкин А.\,В.\hfill\hfill\hfill\hfill\hfill\hfill\hfill\hfill\hfill\hfill\hfill\hfill\hfill\hfill\hfill\hfill\hfill\hfill\hfill\hfill\hfill\hfill\hfill\hfill\hfill\hfill\hfill\hfill\hfill\hfill\hfill\hfill\hfill\hfill\hfill}{\ } 
\contentsline {section}{\textbf{Соколов И.\,А.}\ \ см. Печинкин А.\,В.\hfill\hfill\hfill\hfill\hfill\hfill\hfill\hfill\hfill\hfill\hfill\hfill\hfill\hfill\hfill\hfill\hfill\hfill\hfill\hfill\hfill\hfill\hfill\hfill\hfill\hfill\hfill\hfill\hfill\hfill\hfill\hfill\hfill\hfill\hfill}{\ }
\contentsline {section}{\textbf{Ступников С.\,А.}\ \ см. Захаров В.\,Н.\hfill\hfill\hfill\hfill\hfill\hfill\hfill\hfill\hfill\hfill\hfill\hfill\hfill\hfill\hfill\hfill\hfill\hfill\hfill\hfill\hfill\hfill\hfill\hfill\hfill\hfill\hfill\hfill\hfill\hfill\hfill\hfill\hfill\hfill\hfill}{\ }
\contentsline {section}{\textbf{Чаплыгин В.\,В.}\ \ см. Печинкин А.\,В.\hfill\hfill\hfill\hfill\hfill\hfill\hfill\hfill\hfill\hfill\hfill\hfill\hfill\hfill\hfill\hfill\hfill\hfill\hfill\hfill\hfill\hfill\hfill\hfill\hfill\hfill\hfill\hfill\hfill\hfill\hfill\hfill\hfill\hfill\hfill}{\ } 
\contentsline {section}{\textbf{Чаплыгин В.\,В.}\ \ см. Печинкин А.\,В.\hfill\hfill\hfill\hfill\hfill\hfill\hfill\hfill\hfill\hfill\hfill\hfill\hfill\hfill\hfill\hfill\hfill\hfill\hfill\hfill\hfill\hfill\hfill\hfill\hfill\hfill\hfill\hfill\hfill\hfill\hfill\hfill\hfill\hfill\hfill}{\ }
\contentsline {section}{\textbf{Шоргин С.\,Я.}\ \ см. Батракова Д.\,А.\hfill\hfill\hfill\hfill\hfill\hfill\hfill\hfill\hfill\hfill\hfill\hfill\hfill\hfill\hfill\hfill\hfill\hfill\hfill\hfill\hfill\hfill\hfill\hfill\hfill\hfill\hfill\hfill\hfill\hfill\hfill\hfill\hfill\hfill\hfill}{\ } 
\contentsline {section}{\textbf{Шоргин С.\,Я.}\ \ см. Кудрявцев А.\,А.\hfill\hfill\hfill\hfill\hfill\hfill\hfill\hfill\hfill\hfill\hfill\hfill\hfill\hfill\hfill\hfill\hfill\hfill\hfill\hfill\hfill\hfill\hfill\hfill\hfill\hfill\hfill\hfill\hfill\hfill\hfill\hfill\hfill\hfill\hfill}{\ }
%\thispagestyle{myheadings}
\def\leftfootline{\small{\textbf{\thepage}
\hfill ИНФОРМАТИКА И ЕЁ ПРИМЕНЕНИЯ\ \ \ том~1\ \ \ выпуск~2\ \ \ 2007}
}%
 \def\rightfootline{\small{ИНФОРМАТИКА И ЕЁ ПРИМЕНЕНИЯ\ \ \ том~1\ \ \ выпуск~2\ \ \ 2007
 \hfill \textbf{\thepage}}}
 \label{end\stat}

\def\stat{cont-e}
{%\hrule\par
%\vskip 7pt % 7pt
\raggedleft\Large \bf%\baselineskip=3.2ex
2\,0\,0\,7\ \ A\,U\,T\,H\,O\,R\ \ I\,N\,D\,E\,X \vskip 17pt
    \hrule
    \par
\vskip 21pt plus 6pt minus 3pt }

\label{st\stat}

\def\tit{\ }

\def\aut{\ }
\def\auf{\ }

\def\leftkol{\ } % ENGLISH ABSTRACTS}

\def\rightkol{\ } %ENGLISH ABSTRACTS}

\titele{\tit}{\aut}{\auf}{\leftkol}{\rightkol}


\contentsline {chapter}{\ }{Issue \quad Page} 
\contentsline {subsection}{\textbf{Batrakova D.\,A., Korolev V.\,Yu., Shorgin S.\,Ya.}\ \ A New Method for the Probabilistic and Statistical Analysis of Information Flows in Telecommunication Networks}{\qquad 1 \qquad 40} 
\contentsline {subsection}{\textbf{Borisov A.\,V.}\ \ Bayesian Estimation in\nobreakspace {}Observation Systems with\nobreakspace {}Markov Jump Processes: Game-Theoretic Approach}{\qquad 2 \qquad 65} 
\contentsline {subsection}{\textbf{Bosov A.\,V., Ivanov A.\,V.}\ \ Linguistic Simulation for Machine Translation and Knowledge Management Systems}{\qquad 2 \qquad 50} 
\contentsline {subsection}{\textbf{Chaplygin V.\,V.} see Pechinkin A.\,V.\hfill\hfill\hfill\hfill\hfill\hfill\hfill\hfill\hfill\hfill\hfill\hfill\hfill\hfill\hfill\hfill\hfill\hfill\hfill\hfill\hfill\hfill\hfill\hfill\hfill\hfill\hfill\hfill\hfill\hfill\hfill\hfill\hfill\hfill\hfill}{\ }
\contentsline {subsection}{\textbf{Chaplygin V.\,V.} see Pechinkin A.\,V.\hfill\hfill\hfill\hfill\hfill\hfill\hfill\hfill\hfill\hfill\hfill\hfill\hfill\hfill\hfill\hfill\hfill\hfill\hfill\hfill\hfill\hfill\hfill\hfill\hfill\hfill\hfill\hfill\hfill\hfill\hfill\hfill\hfill\hfill\hfill}{\ }
\contentsline {subsection}{\textbf{Ilyin V.\,D., Sokolov I.\,A.}\ \ The Symbol Model of Informatics Knowledge System in Human-Automaton Environment}{\qquad 1 \qquad 66} 
\contentsline {subsection}{\textbf{Ivanov A.\,V.} see Bosov A.\,V.\hfill\hfill\hfill\hfill\hfill\hfill\hfill\hfill\hfill\hfill\hfill\hfill\hfill\hfill\hfill\hfill\hfill\hfill\hfill\hfill\hfill\hfill\hfill\hfill\hfill\hfill\hfill\hfill\hfill\hfill\hfill\hfill\hfill\hfill\hfill}{\ }
\contentsline {subsection}{\textbf{Kalinichenko L.\,A.} see Zakharov V.\,N.\hfill\hfill\hfill\hfill\hfill\hfill\hfill\hfill\hfill\hfill\hfill\hfill\hfill\hfill\hfill\hfill\hfill\hfill\hfill\hfill\hfill\hfill\hfill\hfill\hfill\hfill\hfill\hfill\hfill\hfill\hfill\hfill\hfill\hfill\hfill}{\ }
\contentsline {subsection}{\textbf{Korolev V.\,Yu.} see Batrakova D.\,A.\hfill\hfill\hfill\hfill\hfill\hfill\hfill\hfill\hfill\hfill\hfill\hfill\hfill\hfill\hfill\hfill\hfill\hfill\hfill\hfill\hfill\hfill\hfill\hfill\hfill\hfill\hfill\hfill\hfill\hfill\hfill\hfill\hfill\hfill\hfill}{\ }
\contentsline {subsection}{\textbf{Kozerenko E.\,B.}\ \ Linguistic Simulation for Machine Translation and Knowledge Management Systems}{\qquad 1 \qquad 54} 
\contentsline {subsection}{\textbf{Kozmidiady V.\,A.} see Zakharov V.\,N.\hfill\hfill\hfill\hfill\hfill\hfill\hfill\hfill\hfill\hfill\hfill\hfill\hfill\hfill\hfill\hfill\hfill\hfill\hfill\hfill\hfill\hfill\hfill\hfill\hfill\hfill\hfill\hfill\hfill\hfill\hfill\hfill\hfill\hfill\hfill}{\ }
\contentsline {subsection}{\textbf{Kudryavtsev A.\,A., Shorgin S.\,Ya.}\ \ Bayesian Approach to Queueing Systems and Reliability Characteristics}{\qquad 2 \qquad 76} 
\contentsline {subsection}{\textbf{Pechinkin A.\,V., Sokolov I.\,A., Chaplygin V.\,V.}\ \ Multichannel Queuing System with Finite Buffer and Unreliable Servers}{\qquad 1 \qquad 27} 
\contentsline {subsection}{\textbf{Pechinkin A.\,V., Sokolov I.\,A., Chaplygin V.\,V.}\ \ Stationary Characteristics of a Multichannel Queueing System with\nobreakspace {}Simultaneous Refusals of Servers}{\qquad 2 \qquad 39} 
\contentsline {subsection}{\textbf{Shorgin S.\,Ya.} see Batrakova D.\,A.\hfill\hfill\hfill\hfill\hfill\hfill\hfill\hfill\hfill\hfill\hfill\hfill\hfill\hfill\hfill\hfill\hfill\hfill\hfill\hfill\hfill\hfill\hfill\hfill\hfill\hfill\hfill\hfill\hfill\hfill\hfill\hfill\hfill\hfill\hfill}{\ }
\contentsline {subsection}{\textbf{Shorgin S.\,Ya.} see Kudryavtsev A.\,A.\hfill\hfill\hfill\hfill\hfill\hfill\hfill\hfill\hfill\hfill\hfill\hfill\hfill\hfill\hfill\hfill\hfill\hfill\hfill\hfill\hfill\hfill\hfill\hfill\hfill\hfill\hfill\hfill\hfill\hfill\hfill\hfill\hfill\hfill\hfill}{\ }
\contentsline {subsection}{\textbf{Sinitsyn I.\,N.}\ \ Correlational Methods for Analytical Informational Models of the Earth Pole Fluctuations Design Based on a priori Data}{\qquad 2 \qquad \hphantom{9}2}
\contentsline {subsection}{\textbf{Sinitsyn I.\,N.}\ \ Development of Pugachev Filtering for Stochastic Systems}{\qquad 1 \qquad \hphantom{9}3}
\contentsline {subsection}{\textbf{Sokolov I.\,A.} see Ilyin V.\,D.\hfill\hfill\hfill\hfill\hfill\hfill\hfill\hfill\hfill\hfill\hfill\hfill\hfill\hfill\hfill\hfill\hfill\hfill\hfill\hfill\hfill\hfill\hfill\hfill\hfill\hfill\hfill\hfill\hfill\hfill\hfill\hfill\hfill\hfill\hfill}{\ }
\contentsline {subsection}{\textbf{Sokolov I.\,A.} see Pechinkin A.\,V.\hfill\hfill\hfill\hfill\hfill\hfill\hfill\hfill\hfill\hfill\hfill\hfill\hfill\hfill\hfill\hfill\hfill\hfill\hfill\hfill\hfill\hfill\hfill\hfill\hfill\hfill\hfill\hfill\hfill\hfill\hfill\hfill\hfill\hfill\hfill}{\ }
\contentsline {subsection}{\textbf{Sokolov I.\,A.} see Pechinkin A.\,V.\hfill\hfill\hfill\hfill\hfill\hfill\hfill\hfill\hfill\hfill\hfill\hfill\hfill\hfill\hfill\hfill\hfill\hfill\hfill\hfill\hfill\hfill\hfill\hfill\hfill\hfill\hfill\hfill\hfill\hfill\hfill\hfill\hfill\hfill\hfill}{\ }
\contentsline {subsection}{\textbf{Sokolov I.\,A.} see Zakharov V.\,N.\hfill\hfill\hfill\hfill\hfill\hfill\hfill\hfill\hfill\hfill\hfill\hfill\hfill\hfill\hfill\hfill\hfill\hfill\hfill\hfill\hfill\hfill\hfill\hfill\hfill\hfill\hfill\hfill\hfill\hfill\hfill\hfill\hfill\hfill\hfill}{\ }
\contentsline {subsection}{\textbf{Stupnikov S.\,A.} see Zakharov V.\,N.\hfill\hfill\hfill\hfill\hfill\hfill\hfill\hfill\hfill\hfill\hfill\hfill\hfill\hfill\hfill\hfill\hfill\hfill\hfill\hfill\hfill\hfill\hfill\hfill\hfill\hfill\hfill\hfill\hfill\hfill\hfill\hfill\hfill\hfill\hfill}{\ }
\contentsline {subsection}{\textbf{Zakharov V.\,N., Kalinichenko L.\,A., Sokolov I.\,A., Stupnikov S.\,A.}\ \ Development of Canonical Information Models for Integrated Information Systems}{\qquad 2 \qquad 15} 
\contentsline {subsection}{\textbf{Zakharov V.\,N., Kozmidiady V.\,A.}\ \ Means Providing Applications Fault Tolerance}{\qquad 1 \qquad 14} 
\def\leftfootline{\small{\textbf{\thepage}
\hfill ИНФОРМАТИКА И ЕЁ ПРИМЕНЕНИЯ\ \ \ том~1\ \ \ выпуск~2\ \ \ 2007}
}%
 \def\rightfootline{\small{ИНФОРМАТИКА И ЕЁ ПРИМЕНЕНИЯ\ \ \ том~1\ \ \ выпуск~2\ \ \ 2007
 \hfill \textbf{\thepage}}}
 \label{end\stat}

\include{cover3}

%\tableofcontents


\end{document}


\setcounter{secnumdepth}{1} %здесь - 3 + chapter = 4