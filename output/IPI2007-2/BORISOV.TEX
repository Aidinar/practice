
\newcommand {\diag}{\mathop{\mathrm{ diag}}}
\newcommand {\col}{\mathop{\mathrm{col}}}
\newcommand {\Aarg}{\mathop{\mathrm {Arg}}}
\newcommand {\vv}{\mathop{\mathrm {vec}}}
\newcommand {\conv}{\mathop{\mathrm {conv}}}
%\renewcommand {\tr}{\mathop{\mathrm {tr}}}
\newcommand {\ppp}{{\mathcal P}}
\newcommand {\jj}{{\mathbf J}}
\newcommand {\UU}{{\mathrm U}}
\newcommand {\V}{{\mathrm V}}
\newcommand {\UR}{{\mathcal U}}
\newcommand {\VR}{{\mathcal V}}
\newcommand {\sind}{^{(s)}}
\newcommand {\st}{^{(\prime)}}
\newcommand {\nin}{\bar{\in}}
\newcommand {\cp}{\mathop{\bf P'}}
\newcommand {\cpp}{\mathop{\bf P''}}
\newcommand{\pp}[1]{\mathbf{P}\left\{ #1 \right\}}
\newcommand{\me}[2]{\mathbf{E}_{ #1 }\left\{ \mathop{#2} \right\} }
\newcommand{\mme}[2]{\widetilde{\mathbf{E}}_{ #1 }\left\{ \mathop{#2} \right\} }
\newcommand{\bi}{{\mathbf I}}
%\newcommand {\e}{\varepsilon}
%\newcommand {\g}{\gamma}
\newcommand {\et}{\tilde{\varepsilon}}
%\newcommand {\fr}[2]{\frac{\ds #1}{\ds #2}}
\newcommand {\ebd}{\stackrel{\mathrm {def}}{=}}


\def\stat{borisov}

\def\tit{БАЙЕСОВСКОЕ ОЦЕНИВАНИЕ В СИСТЕМАХ НАБЛЮДЕНИЯ
С~МАРКОВСКИМИ СКАЧКООБРАЗНЫМИ ПРОЦЕССАМИ: ИГРОВОЙ
ПОДХОД$^*$}

\def\titkol{Байесовское оценивание в системах наблюдения с
марковскими скачкообразными процессами: игровой подход}
\def\autkol{А.\,В.~Борисов}
\def\aut{А.\,В.~Борисов$^1$}

\titel{\tit}{\aut}{\autkol}{\titkol}

{\renewcommand{\thefootnote}{\fnsymbol{footnote}}\footnotetext[1]{Работа
выполнена при частичной финансовой поддержке
РФФИ (грант \mbox{№\,05-01-00508-a}) и программы ОИТВС РАН
<<Фундаментальные алгоритмы информационных технологий>>~(проект 1.5).}
\renewcommand{\thefootnote}{\arabic{footnote}}}


\footnotetext[1]{Институт проблем информатики Российской академии наук, ABorisov@ipiran.ru}

\index{Борисов А.\,В.}

%\vspace*{-6pt}

\Abst{В статье рассматривается задача совместного
оценивания состояния и параметров в сис\-те\-мах наблюдения с
марковскими скачкообразными процессами с конечным числом состояний
по имеющимся непрерывным и считающим наблюдениям. Уравнения динамики
и наблюдений зависят от случайного конечномерного параметра,
имеющего неизвестное распределение с заданным носителем. В качестве
критериев качества выступают условные математические ожидания
некоторых квадратичных функций оценок. Доказано утверждение о
существовании седловой точки в поставленной минимаксной задаче.
Получена характеризация наихудшего распределения и минимаксной
оценки как решения более простой двойственной задачи. Практическая
применимость представленных результатов проиллюстрирована на примере
решения задачи оперативного оценивания состояния TCP-соединения,
функционирующего в условиях неопределенности.}

\KW{фильтр Вонэма; минимаксное оценивание;
обобщенный квадратичный критерий; уравнение Закаи}

\vskip 21pt plus 6pt minus 6pt

\thispagestyle{headings}

\begin{multicols}{2}


\label{st\stat}

 \section{Введение}

 Фильтр Калмана--Бьюси~\cite{kalmanbucy} и фильтр Вонэма~\cite{wonham} являются
 самыми известными и широко применяемыми на практике СК-оптимальными конечномерными оценками
 фильтрации состояний стохастических дифференциальных систем. Известно, что в большом
 числе прикладных задач
 оценивания имеет место априорная неопределенность пара\-мет\-ров системы наблюдения, что
 породило не только множество модификаций указанных алгоритмов
 фильтрации, но и привело к возникновению целого спектра направлений
 теории оценивания в стохастических системах, и, в частности,
 мини\-макс\-ного подхода. При этом большее внимание было уделено фильтру
 Калмана--Бьюси, и основные результаты опираются на
 наличие у рассматриваемой задачи оценивания в той или иной мере следующих ключевых
 свойств: <<линейная система наблюдения~-- гауссовость~-- допустимость линейных
 оценок>>. Возможность гауссовских шумов в линейной системе наблюдения
 обеспечивала оптимальность линейных
 оценок. В случае же отсутствия гауссо\-вости чаще всего класс
 допустимых оценок принудительно ограничивался классом линейных оценок.
 Не ставя целью написание всеобъемлющего обзора по
 фильтрации Калмана--Бьюси в минимаксной постановке, отметим лишь ряд
 работ, постановка задач и результаты которых близки к данной статье.


 В~\cite{m_m} исследовалась задача минимаксной фильтрации в линейных
 системах наблюдения по безусловным квадратичному и
 обобщенному квадратичному критериям. При этом параметрическая
 неопределенность присутствовала как в интенсивностях шумов, так и в
 матрицах динамики и плана системы наблюдения. Решение
 соответствующей игровой задачи было найдено в классе смешанных
 стратегий, так как в чистых стратегиях оно в общем\linebreak случае не
 существует. Это означает, что совокупности неопределенных
 параметров была навязана\linebreak случайная природа и седловая точка
 критерия определялась парой <<наихудшее распределение~-- наилучший оцениватель для наихудшего
 распределения>>. При этом наилучшая оценка являлась взвешенным
 значением некоторого набора оценок фильтрации Калмана. В~\cite{an}
 данные результаты были обобщены и на случай непараметрической
 неопределенности распределений шумов в системе наблюдения.

 В работе~\cite{k_k} в линейной гауссовской системе наблюдения
 присутствовала параметрическая неопределенность шумов в уравнениях
 состояния и наблюдения, а в качестве критерия оптимальности
 выступал условный СК-критерий. Вид критерия, наличие свойств линейности и
 гауссовости самой системы наблюдения, а также отсутствие
 неопределенности в матрицах динамики и плана системы наблюдения~---
 предпосылки того, что игровая задача имеет решение в чистых
 стратегиях, а наилучшей оценкой является линейная оценка фильтрации
 Калмана, построенная для наихудших пара\-мет\-ров шумов.

 В упомянутых выше работах по минимаксному оцениванию рассматривались
 системы наблюдения с дискретным временем. В~\cite{p_m, s_l_p}
 исследовались задачи фильтрации состояния
 стохастических дифференциальных систем наблюдения с
 неопределенностями в интенсивностях шумов, а в качестве критериев
 оптимальности выступали, соответственно, безусловные интегральный и мгновенный
 СК-критерии. Линейность оптимальных фильтров в
 данном случае была обусловлена допустимостью гауссовского
 распределения шумов в системе.

 Задача фильтрации Вонэма, заключающаяся в СК-оп\-ти\-маль\-ном оценивании
 состояния марковского скачкообразного процесса с конечным чис\-лом
 состояний (МСПКЧС) по косвенным непре\-рыв\-ным и считающим наблюдениям, об\-ла\-дает %\linebreak
 \mbox{рядом} ключевых отличий от задачи фильтрации Калмана, затрудняющих решение
 со\-от\-вет\-ст\-ву\-ющей минимаксной задачи.
 Во-первых, несмотря на то что исследуемая система наблюдения
 формально допускает запись в виде линейной дифференциальной стохастической
 системы~\cite{elliott2}, она не является гауссовской из-за негауссовости мартингала
 в представлении оцениваемого марковского процесса.
 Во-вторых,
 характеристика данного мартингала связана с матрицей интенсивностей
 переходов марковского процесса. При наличии неопределенности этой
 матрицы данное свойство автоматически ведет к
 неопределенности характеристики мартингала.
 В-третьих,
 оценки фильтрации Вонэма являются принципиально нелинейными и представляют собой
 условное распределение оцениваемого процесса относительно имеющихся
 наблюдений. Для исследуемой системы наблюдения %\linebreak 
 можно, конечно, решать задачу минимаксной
 линейной фильтрации. Однако полученные оценки уже не будут
 обладать свойством вероятностного распределения: условие нормировки
 сохранится, в то время как неотрицательность не будет иметь места.


 Целью данной работы является корректная постанов\-ка и решение задачи
 совместного мини\-макс\-ного оценивания марковских процессов с конеч\-ным
 числом состояний и идентификации неопределенного параметра системы наблюдения, присутствующего
 как в уравнении динамики, так и в наблюдениях.

 Статья построена следующим образом. Раздел~2 содержит описание исследуемой
 системы наблюдения. Ее ключевой особенностью является то, что тройка
 <<матрица интенсивностей переходов~--\linebreak
 матрица плана~-- интенсивность считающих наблюдений>> является функцией ненаблюдаемого\linebreak
 случайного параметра, имеющего известное распределение. 
 %В разделе 
% Также 
Приведено решение задачи~со\-вмест\-ного байесовского оценивания состояния процесса и указанного
 параметра, являющееся основой для решения задач минимаксного оценивания.
{\looseness=-1

}

 Раздел~3 посвящен детальной постановке задачи\linebreak
 минимаксного оценивания. Ранее подобная постановка
 с точки зрения вида показателя качества и класса допустимых оценок
 встречалась в работах автора~\cite{bor_2007_2, bor_2007} и имеет ряд особенностей по сравнению с
 традиционными задачами минимаксного оценивания.
 Во-первых, в качестве критериев выступают {\bfseries\textit{условные}}
 математические ожидания некоторых квадратичных функций оценок
 относительно имеющихся наблюдений. Во-вторых, допустимые оценки
 могут быть {\bfseries\textit{нелинейными}}. В-третьих, класс неопределенности
 включает в себя все распределения параметра, сосредоточенные на известном подмножестве конечномерного
 евклидова пространства. Приведены доводы в пользу практической применимости
 поставленных задач оценивания.
 {\looseness=-1
 
 }

 Раздел~4 содержит решение поставленной задачи оценивания. В нем
 приведена теорема о существовании седловой точки соответствующей
 минимаксной оптимизационной задачи, даны формулы, определяющие
 оптимальную оценку, а также указаны некоторые свойства наихудших
 распределений.

 В разделе~5 приводится содержательный прикладной пример,
 иллюстрирующий практическую применимость предлагаемых минимаксных
 оценок. Рассмотрена задача мониторинга состояния TCP-соединения,
 функционирующего в соответствии с классической моделью Джилберта~\cite{gilb},
 в условиях априорной неопределенности характеристик
 канала.

 В заключительном разделе~6 обсуждаются полученные результаты, а
 также приводятся возможные направления дальнейших исследований в
 данной области.
\vspace*{-3pt}

 \section{Байесовское оценивание в~системах наблюдения с~марковскими скачкообразными процессами}
\vspace*{-3pt}

 На конечном отрезке времени $[0,T]$ рассматривается система наблюдения
 
 \noindent
\begin{equation}
 \left.
 \begin{array}{rl}
 \displaystyle \theta_t\!\!\!&=\;\theta_0+\int\limits_0^t \Lambda^* \theta_{s-}\,ds +
 M^{\theta}_t\,; \\
 \displaystyle Q_t\!\!\!&=\;\int\limits_0^t \mu \theta_{s-}\,ds +
 M^{Q}_t\,; \\
 \displaystyle U_t\!\!\!&=\;\int\limits_0^t A \theta_{s-}\,ds + \varepsilon W_t\,.
 \end{array}
 \right\}
 \label{eq:markobsys}
\end{equation}
 Здесь
 $\theta_t \in S_n$~--- ненаблюдаемый однородный МСПКЧС с
 множеством состояний;
 $S_n=$\linebreak $=\{e_1,\ldots,e_n\}$ ($e_k$ обозначает $k$-й единичный вектор в евклидовом пространстве~$\mathbb{R}^n$),
 с известным начальным распределением~$p_0$ и матрицей пе\-реход\-ных вероятностей
 $\Lambda \in \mathbb{R}^{n \times n}$;
 $Q_t \in \mathbb{Z}_+$~--- счи\-та\-ющий процесс наблюдений,
 интенсивность кото\-ро\-го $\mu\theta$ зависит от состояния
$\theta$ ($\mu \in \mathbb{R}^{1 \times n}$
 представляет собой век\-тор-стро\-ку возможных интенсивностей);
 $U_t \in \mathbb{R}^{m \times 1}$~--- процесс непрерывных наблюдений ($A \in \mathbb{R}^{m \times n}$
 является матрицей плана наблюдений);
 $W_t \in \mathbb{R}^{m \times 1}$~--- винеровский процесс, пред\-став\-ля\-ющий ошибки непрерывных
 наблюдений;
 $\varepsilon \varepsilon^* > 0$~--- невырожденная интенсивность ошибок непрерывных наблюдений.

 Первые два уравнения в~(\ref{eq:markobsys})
 являются мартингальным разложением для $\theta$ и $Q$~\cite{elliott2}:
 процессы $M^{\theta}_t$ и $M^{Q}_t$ являются
$\mathcal{F}_t^{\theta,Q}$-согласованными
 центрированными квадратично-интегрируемыми мартингалами с
 квадратическими характеристиками
\begin{align*}
\langle M^{\theta}, M^{\theta}\rangle _t &= \displaystyle \int\limits_0^t
(\diag(\Lambda^*\theta_{s-})- \Lambda^*\diag(\theta_{s-})-{}\\
&\hspace*{80pt}-\diag(\theta_{s-})\Lambda)\,ds\,,\\
\langle M^{Q}, M^{Q}\rangle _t &= \displaystyle \int\limits_0^t
\mu\theta_{s-}\,ds\,.\hfill\
\end{align*}
 В дальнейшем в работе предполагается, что мат\-ри\-ца интенсивностей переходов $\Lambda=\Lambda(\gamma(\omega))$,
 мат\-ри\-ца плана непрерывных наблюдений
 $A=A(\gamma(\omega))$ и вектор интенсивностей считающих наблюдений $\mu=\mu(\gamma(\omega))$
 являются известными функциями ненаблюдаемого случайного векторного параметра $\gamma(\omega) \in \mathbb{R}^{k \times 1}$.

 Для описания вероятностного пространства с фильтрацией используются
 следующие обозначения:
\begin{description}
\item $\mathcal{F}_t \ebd \sigma \{
\gamma(\omega),\theta_s(\omega), Q_s(\omega), W_s(\omega),\; 0
\leqslant s \leqslant t \}$~--- <<универсальный>>
 естественный поток $\sigma$-подал\-гебр, порожденный $\theta$, $Q$, $W$
и $\gamma$; \\
\item $\mathcal{F} \ebd \bigvee_{t\in[0,T]}\mathcal{F}_t$;\\
\item $\mathcal{O}_t \ebd \sigma \{ Q_s(\omega), U_s(\omega), \;
0 \leqslant s \leqslant t\}$~--- естественный поток
$\sigma$-подалгебр, порожденный наблюдениями $Q$ и $U$.
\end{description}

 Предполагается, что вероятностная мера $\mathbf{P}_F$, определенная на измеримом пространстве $(\Omega, \mathcal{F})$,
 известна и удовлетворяет следующим условиям:
 \begin{enumerate}[I]
 \item \label{con_1}
 $\mathbf{P}_F\{\omega: \gamma(\omega) \in
\mathcal{L}\}=F(\mathcal{L})$, где $F(\cdot)$~--- известное
априорное распределение параметра $\gamma$, причем существует и
известно множество $\mathcal{C} \subseteq \mathbb{R}^{k \times 1}$
такое, что $\gamma(\omega) \in \mathcal{C} \subseteq \mathbb{R}^{k
\times 1}$ для любого $\omega \in \Omega$;
 \item \label{con_2}
 $A=A(q): \mathcal{C} \to R^{n \times m}$, $\Lambda=\Lambda(q): \mathcal{C} \to R^{n \times
 n}$ и $\mu=\mu(q): \mathcal{C} \to R^{1 \times n}$~--- известные
 ограниченные функции; при этом $\Lambda(q)$ и $\mu(q)$ удов\-ле\-тво\-ря\-ют обычным условиям интенсивностей:
 $\lambda_{ij}(q)\geqslant 0$, если $i \neq j$, $\sum_{j=1}^n\lambda_{ij}(q) \equiv
 0$ и $\mu_i(q)\geqslant 0$ для $i=1,2,\ldots,n$;
 \item \label{con_3}
 Начальное условие $\theta_0(\omega)$, случайный вектор $\gamma(\omega)$
 и шум $W_t(\omega)$ в наблюдениях независимы в совокупности;
 \item \label{con_4}
 $M^Q \perp \!\!\!\perp M^{\theta}$ \cite{l_sh}, т.\,е.\
 $M^{\theta}_0M^Q_0=0$ почти наверное (п.н.), и для любого марковского
 относительно $\{\mathcal{F}_t\}$ момента $\tau$ выполняется равенство
 $\me{F}{M^{\theta}_{\tau}M^Q_{\tau}}=0 $;
 \item \label{con_5}
 Верно равенство
\begin{multline*}
\displaystyle \mathbf{E}_F\left\{\exp\left[
 \int\limits_0^t A^*(\gamma)(\varepsilon\varepsilon^*)^{-1}\,dU_s+{}\right. \right.\\
 {}+
 \frac{1}{2}\int\limits_0^t
 A^*(\gamma)(\varepsilon\varepsilon^*)^{-1}A(\gamma)\,ds
 + {}\\
 {}+ \int\limits_0^t \ln (\mu(\gamma)\theta_{s-})\,dQ_s -{}\\
 \left.\left.{}- \int\limits_0^t (\mu(\gamma)\theta_{s-}-1)\,ds
 \right]\right\}\!=\!1\,.
\end{multline*}
 \end{enumerate}
 Индекс $F$ в обозначениях $\mathbf{P}_F$ и $\mathbf{E}_F$ используется для
 указания зависимости вероятностной меры и соответствующего математического
 ожидания от априорного распределения $F$ параметра $\gamma(\omega)$.

 {\bfseries\textit{Задача байесовского оценивания}} вектора $z_t \ebd$\linebreak
$\ebd \col(\theta_t, \gamma)$ заключается в нахождении
 $\widehat{z}^F_t \ebd $\linebreak $\ebd \me{F}{z_t | \mathcal{O}_t}$.

 Как известно, оценка $\widehat{z}^F_t$ является оптимальной как в смысле \emph{безусловного СК-критерия}
\begin{equation*}
 \displaystyle \widehat{z}^F_t \in \Aarg \min_{\overline{z}_t \in
 \mathcal{M}_t}\me{F}{\|z_t-\overline{z}_t\|^2}\,,
 %\label{eq:crit_1}
\end{equation*}
 так и его \emph{условной версии}
 \begin{equation}
 \displaystyle \widehat{z}^F_t \in \Aarg \min_{\overline{z}_t \in
 \mathcal{M}_t}\me{F}{\|z_t-\overline{z}_t\|^2|\mathcal{O}_t}\,.
 \label{eq:crit_2}
\end{equation}
 При этом множество допустимых оценок $\mathcal{M}_t$ вектора
 $\overline{z}_t$ включает в себя все $\mathcal{O}_t$-измеримые
 функции, а
 оптимальность $\widehat{z}^F_t$ в смысле условного критерия (\ref{eq:crit_2}) означает,
 что неравенство
$$\me{F}{\|z_t-\widehat{z}^F_t\|^2|\mathcal{O}_t} \leqslant
\me{F}{\|z_t-\overline{z}_t\|^2|\mathcal{O}_t}$$
 выполняется $\mathbf{P}_F$-п.н.\ для любой другой оценки $\overline{z}_t \in
 \mathcal{M}_t$.

 Сформулированная выше проблема оценивания может быть очевидным
 образом трансформирована в задачу оптимальной нелинейной
 фильтрации. Мартингальные представления процессов $\theta$ и $Q$
 остаются справедливыми при замене неслучайных матрицы $\Lambda$ и вектора $\mu$
 их случайными аналогами $\Lambda(\omega)$ и $\mu(\omega)$, и
 формулы для квадратических характеристик $\langle
 M^{\theta}, M^{\theta}\rangle _t$ и $\langle
 M^{N}, M^{N}\rangle _t$ также формально не изменяются.

 Рассмотрим новую систему наблюдения относительного расширенного вектора
состояния $z_t \ebd $\linebreak $\ebd\col(\theta_t,\gamma_t)$
 \begin{equation}
 \left.
 \begin{array}{rl}
\tabcolsep=0pt
 \displaystyle \theta_t\!\!\!&=\;\theta_0+\int\limits_0^t \Lambda^*(\gamma_{s-}) \theta_{s-}\,ds +
 M^{\theta}_t\,; \\
 \gamma_t\!\!\!&=\;\gamma\,; \\
 \displaystyle Q_t\!\!\!&=\;\int\limits_0^t \mu(\gamma_{s-}) \theta_{s-}\,ds +
 M^{N}_t\,;\\
 \displaystyle U_t\!\!\!&=\;\int\limits_0^t A(\gamma_{s-}) \theta_{s-}\,ds + \varepsilon
 W_t
 \end{array}
 \right\}
 \label{eq:markobsys_1}
\end{equation}
 и определим условные вероятности
\begin{align*}
\widehat{P}^F_i(\mathcal{L},t) &\ebd
 \mathbf{P}_F\{\gamma_t \in \mathcal{L},
 \theta_t =e_i\;|\; \mathcal{U}_t\}\,,\quad
 i =\overline{1,n}\,,\\
\widehat{P}^F(\mathcal{L},t)& \ebd \col
 (\widehat{P}^F_1(\mathcal{L},t),\ldots,\widehat{P}^F_n(\mathcal{L},t))\,.
\end{align*}

\paragraph*{Предложение 1.}Если условия~\ref{con_1}--\ref{con_5} выполнены для системы
 наблюдения~(\ref{eq:markobsys_1}), то условное распределение $\widehat{P}^F(\mathcal{L},t)$
 определяется следующим образом:
\begin{align*}
 \widehat{P}^F(\mathcal{L},t)&=K\displaystyle \int\limits_{\mathcal{L} }
 \widetilde{\theta}_t(q)F(dq)\,. %\label{eq:condis}
\end{align*}
Здесь
$$K=\left(\int\limits_{\mathcal{C}}
 \mathbf{1}\widetilde{\theta}_t(q)F(dq)\right)^{-1},$$
% где
$\widetilde{\theta}_t(q)$~--- ненормированное условное распределение $\theta_t$ по наблюдениям $\mathcal{O}_t$,
 вычисленное с помощью алгоритма фильтрации Вонэма в предположении,
 что $\gamma(\omega)= q$:
 
 \noindent
 \begin{multline}
 \widetilde{\theta}_t(q)=p_0+
 \int\limits_0^t\Lambda^*(q)\widetilde{\theta}_{s-}(q)\,ds +{}\\
{} +
 \int\limits_0^t
 \diag (\widetilde{\theta}_{s-}(q)) A^*(q)(\varepsilon \varepsilon^*)^{-1}\,dU_s+ {}\\
{}+
 \int\limits_0^t
 [\diag \mu(q)-I_{n \times n}]\widetilde{\theta}_{s-}(q)\,(dQ_s-ds)\,,
 \label{eq:wonham}
\end{multline}
 где
$I_{n \times n}$~--- единичная матрица размера $n \times n$,
а $\mathbf{1}$~--- вектор-строка соответствующей размерности,
составленная из единиц.

Байесовская оценка $\widehat{z}^F_t$ вектора $z_t$
 определяется формулами:
 \begin{align*}
 \widehat{\theta}^F_t &=K \displaystyle \int\limits_{\mathcal{C}}
 \widetilde{\theta}_t(q)F(dq)\,,\\ %\notag\\[-9pt]
% & \label{eq:bayes}\\[-9pt]
 \widehat{\gamma}^F_t & =K \int\limits_{\mathcal{C}}
 q\mathbf{1}\widetilde{\theta}_t(q)F(dq)\,. %\notag
 \end{align*}

Доказательство предложения~1 аналогично приведенному
в~\cite{bor_2007} для случая непрерывных наблюдений.

Представленное решение задачи байесовского оценивания является
основой для решения соответствующей задачи в условиях
неопределенности, о которой пойдет речь в следующем разделе.

 \section{Постановка задачи минимаксного оценивания} %3

 Рассмотрим систему наблюдения~(\ref{eq:markobsys_1}), для которой
 выполнены условия~\ref{con_2}--\ref{con_5}, а условие~\ref{con_1}
заменено на \ref{con_1}$^\prime$:
 \begin{enumerate}[I$^\prime$]
 \item \label{con_1p}
%1')
Распределение $F(\mathcal{L}) = \mathbf{P}_F\{\omega \in \Omega
\!:\! \gamma(\omega) \in \mathcal{L}\}$ априори \emph{не известно};
при этом $\gamma(\omega) \in \mathcal{C} \subseteq $\linebreak $\subseteq\mathbb{R}^{k
\times 1}$ для любого $\omega \in \Omega$ и $\mathcal{C}$~---
известный выпуклый компакт.
 \end{enumerate}

 В качестве множества неопределенности $\mathbb{F}$ выступает набор всех распределений $F$
 па\-ра\-мет\-ра $\gamma(\omega)$, удовлетворяющих указанным свойствам.
 В отличие от постановки задачи байесовского
оценивания вектора $z_t$, рассмотренной в предыдущем разделе, из-за
данной неопределен\-ности имеется не одно, а целое семейство
вероятност\-ных пространств с фильтрацией
$\mathcal{P}_{\mathbb{F}}= $\linebreak $=\{(\Omega,\mathcal{F},
\mathbf{P}_F,\{\mathcal{F}_t\}_{t \in [0,T]})\}_{F \in \mathbb{F}}$,
параметризованное распределением $F \in \mathbb{F}$.

Пусть $g_t \ebd g(z_t)$~--- сигнальный процесс, подлежащий оцениванию. Данный 
процесс является известной функцией обобщенного состояния 
сис\-те\-мы~(\ref{eq:markobsys_1}): $g(z): S_n \times \mathcal{C} \to \mathbb{R}^{l \times 1}$,
причем 

\noindent
$$ \sup_{F \in
\mathbb{F}}\me{F}{\|g_t\|^2} < \infty\;.$$

 Для дальнейшей корректной постановки задачи минимаксного оценивания
 необходимо однозначным образом выбрать версию условного
 математического ожидания, для чего дополнительно потребуются
 следующие обозначения:
\begin{description}
\item $\mathbf{C}^m[0,t]$~--- пространство непрерывных функций $f: [0,t]
 \to \mathbb{R}^{m \times 1}$;
\item
 $\mathbf{B}[0,t]$~--- множество кусочно-постоянных неубывающих
 функций, начинающихся из 0, со скачками, равными +1,
 причем число скачков на конечных интервалах конечно;
\item
 $\mathcal{B}_t$~--- $\sigma$-алгебра цилиндрических множеств на
$\mathbf{C}^m[0,t] \times \mathbf{B}[0,t]$, пополненная множествами
$\mathbf{P}_F$-меры нуль для всех $F \in \mathbb{F}$;
\item
 $\mathbf{W}^m[0,t]$~--- подмножество пространства
$\mathbf{C}^m[0,t]$ функций $\varphi$, равномерно непрерывных по
Гёльдеру с любым показателем $\alpha \in (0,\frac{1}{2})$, т.\,е.\
функций, для которых неравенство
 \begin{equation}
 \sup_{0 \leqslant s_1 < s_2 \leqslant
 t}\|\varphi(s_2)-\varphi(s_1)\|^{\alpha} < K_{\varphi}|s_2-s_1|^{\alpha}
 \label{levy}
 \end{equation}
 выполняется для некоторой конечной константы $K_{\varphi}=K(\varphi)$ и
любого $\alpha \in (0,\frac{1}{2})$;
\item
 $O^t \ebd (U^t,Q^t)$ --- траектории непрерывных и считающих
 наблюдений, полученные на отрезке времени $[0,t]$.
\end{description}

Согласно~\cite{doob} в стохастическом базисе с фильт\-рацией $(\Omega,
\mathcal{F},\mathbf{P}_F,\{\mathcal{F}_t\}_{t\in[0,T]})$ (т.\,е.\ при
фиксированном распределении $F \in \mathbb{F}$) условное
матема\-ти\-ческое ожидание $\mathcal{F}_t$-измеримой интегрируемой
случайной величины $\eta(\omega)$ определяется только
\mbox{$\mathbf{P}_F$-п.н.}:
 \[
 \widehat{\eta}^F\ebd
 \me{F}{\eta(\omega)|\mathcal{O}_t}=\eta^F(O^t)\quad
 \mathbf{P}_F\mbox{\rm -п.н.}
 \]
 для некоторой $\mathcal{B}_t$-измеримой функции
$\widehat{\eta}^F(\cdot):\;\mathbf{C}^m[0,t] \times \mathbf{B}[0,t] \to \mathbb{R}$.
 Это означает, что для любого $F \in \mathbb{F}$
 существует подмножество $\mathbf{P}_F$-меры нуль, на котором условное
 ожидание может определяться произвольным образом. Если
 игнорировать указанную неоднозначность, то можно \mbox{прийти} к тому, что
 для любой траектории наблюдений $O^t\in \mathbf{C}^m[0,t] \times
\mathbf{B}[0,t]$ найдется такое
 распределение $F \in \mathbb{F}$, что
 $\me{F}{\eta(\omega)|\mathcal{O}_t}$ на данной траектории определяется неоднозначно, что
 делает последующую задачу оптимизации некорректной. Чтобы избежать
 такой ситуации, необходимо единым образом выбрать версию условного
 математического ожидания для всех возможных мер $\mathbf{P}_F$.

 Выберем подмножество $\mathcal{X}[0,t] \subseteq
 \mathbf{C}^m[0,t] \times $\linebreak
 $\times \mathbf{B}[0,t]$ пространства допустимых траекторий наблюдений $O^t$
такое, что $\mathbf{P}_F\{O^t \in \mathcal{X}[0,t]\}\equiv 1$ для
любого $F \in \mathbb{F}$.

 Зафиксируем некоторое распределение $F \in \mathbb{F}$ и рассмотрим
 соответствующее вероятностное пространство с фильтрацией
 $(\Omega, \mathcal{F},\mathbf{P}_F,\{\mathcal{F}_t\}_{t\in[0,T]})$.
 Введем процесс
 \begin{multline*}
 \Phi_t=\exp\left[
 \int\limits_0^t A^*(\gamma)(\varepsilon\varepsilon^*)^{-1}\,dU_s+{}\right.\\
{}+ \fr{1}{2}\int\limits_0^t
 A^*(\gamma)(\varepsilon\varepsilon^*)^{-1}A(\gamma)\,ds +
 \int\limits_0^t \ln (\mu(\gamma)\theta_{s-})\,dQ_s -{}\\
\left. {}- \int\limits_0^t (\mu(\gamma)\theta_{s-}-1)\,ds
 \right]
\end{multline*}
и новую вероятностную меру $\widetilde{\mathbf{P}}_F$ такую, что
 \[
 \frac{d \widetilde{\mathbf{P}}_F}{d \mathbf{P}_F}=(\Phi_T)^{-1}\,.
 \]
 По теореме Гирсанова \cite{wong,elliott1}
 $\overline{U}_t \ebd (\varepsilon\varepsilon^*)^{-1/2}U_t$
 является $\mathcal{O}_t$-согласованным стандартным винеровским
 процессом в $(\Omega,
\mathcal{F},\widetilde{\mathbf{P}}_F,\{\mathcal{O}_t\}_{t\in[0,T]})$,
а $Q_t$~--- $\mathcal{O}_t$-согласованным стандартным пуассоновским.\linebreak
Помимо этого известно (см., например, \cite{l_sh_0}), что почти все
траектории винеровского процесса являются непрерывными по Гёльдеру с
показателем \mbox{$\alpha \in (0,\frac{1}{2})$}, т.\,е.\ удовлетворяют
условию~(\ref{levy}), а траектории стандартного пуассоновского процесса,
рассмотренные на отрезке $[0,t]$, почти наверное принадлежат
множеству $\mathbf{B}[0,t]$. При этом существенным является то, что
процесс $\Phi_t$, определяющий производную Радона--Никодима\linebreak
$ {d \widetilde{\mathbf{P}}_F}/{d \mathbf{P}_F}$,
не зависит от конкретного вида распределения $F \in \mathbb{F}$
 и $\widetilde{\mathbf{P}}_F \sim \mathbf{P}_F$ для любого $F \in \mathbb{F}$.
 Из всех этих фактов следует, что можно выбрать
$\mathcal{X}[0,t] = \displaystyle \mathbf{W}^m[0,t] \times \mathbf{B}[0,t]$.

Определив таким образом множество <<хороших>> траекторий
наблюдения, необходимо исключить неоднозначность определения
условного математического ожидания в случае <<аномальных>>
наблюдений, выбрав одну его версию, общую для всех распределений $F
\in \mathbb{F}$ (и для всех функций, для которых это математическое
ожидание существует), следующим образом:
 $
 \me{F}{\eta(\omega)|\mathcal{O}_t} \equiv 0$ для любых $ F \in
 \mathbb{F}$, если $\omega \in \Omega:\; O^t(\omega) \notin
 \mathcal{X}[0,t].
 $
 В дальнейшем изложении, когда речь пойдет об условном математическом
 ожидании относительно имеющихся наблюдений, всегда будет выбираться
 именно эта его версия.

В качестве множества допустимых оценивателей $\mathbb{G}_t$
выбираются все $\mathcal{B}_t$-измеримые функции
$\overline{g}_t=\overline{g}_t(u,v): \mathbf{C}^m[0,t] \times
\mathbf{B}[0,t] \to \mathbb{R}^{l \times 1}$ такие, что 
$\sup_{F \in \mathbb{F}}\me{F}{\|\overline{g}(O^t)\|^2} < \infty$ и
$\overline{g}(u,v) \equiv 0$ для любого $(u,v) \notin
\mathcal{X}[0,t]$.

 Пусть $a: \mathcal{C} \to \mathbb{R}^{r \times 1}$~--- некоторая известная ограниченная
 функция. Задана некоторая фик\-си\-рованная вспомогательная оценка
 $\overline{a}_t=\overline{a}_t(O^t)$\linebreak вектора
 $a(\gamma(\omega))$, т.\,е.\
 $\overline{a}_t: \mathbf{C}^m[0,t] \times
\mathbf{B}[0,t] \to$\linebreak $\to \mathbb{R}^r$ --- $\mathcal{B}_t$-из\-ме\-ри\-мая
функция такая, что $\sup_{F \in \mathbb{F},\ t \in [0,T]}\me{F}
{\|\overline{a}_t(O^t)\|^2} < \infty$.

Рассматривается семейство целевых функций $J_{O^t}$,
 параметризованное траекторией наблюдений~$O^t$:
 \begin{multline*}
 J_{O^t}(F,\overline{g}_t)\ebd \mathbf{E}_F \left\{ \| g_t-\overline{g}_t(O^t)\|^2-{}\right.\\
\left.{}- \| a(\gamma)-\overline{a}_t(O^t)\|^2\big\vert \mathcal{O}_t \right \}\,.
 \label{eq:loss}
 \end{multline*}

 {\bfseries\textit{Задача минимаксного апостериорного оценивания сигнального
 процесса $g_t$}} заключается в нахождении такой оценки
 $\widehat{\mathbf{g}}_t$, что
\begin{equation}
 \widehat{\mathbf{g}}_t \in \Aarg \min_{\overline{g}_t \in
 \mathbb{G}_t} \sup_{F \in \mathbb{F}}J_{O^t}(F,\overline{g}_t)
 \label{eq:problem}
\end{equation}
 для $\mathbf{P}_F$-почти всех траекторий $O^t$ одновременно для
 всех распределений $F \in \mathbb{F}$.

Предложенная задача оценивания является не совсем обычной и
нуждается в трактовке.

Во-первых, целевая функция под знаком супремума содержит
\emph{условное} математическое ожидание. Этот выбор объясняется тем,
что любая прикладная задача оценивания рассматривается в конечном
счете <<в привязке>> к некоторой реализации наблюдений
(выборке, траектории и~пр.) В случае априорной неопределенности в
системе наблюдения использование безусловного минимаксного критерия
подразумевает нахождение <<наихудших>> параметров данной
системы безотносительно полученных наблюдений и дальнейшее
построение оптимальных оценок для системы с такими параметрами. В
отличие от этого, использование предлагаемого условного минимаксного
критерия предполагает поиск <<наихудших>> параметров
применительно не только к самой системе, но и к реализовавшимся
наблюдениям.

Во-вторых, оцениваемый сигнальный процесс в задаче является
некоторой совместной функцией неопределенных параметров системы
наблюдения $\gamma$, постоянных во времени, а также текущего
(изменяющегося) состояния системы $\theta_t$. Такой вид оцениваемого
процесса позволяет решать одновременно задачу минимаксной
идентификации параметров системы наблюдения и фильтрации ее параметров
применительно к реализовавшейся траектории наблюдений $O^t$.

В-третьих, включение в $J_{O^t}(F,\overline{g}_t)$ вычитаемого,
зависящего от некоторой опорной оценки $\overline{a}_t$ параметра
$a(\gamma)$ связано со следующим обстоятельством.
 С практической точки зрения наделение не\-опре\-де\-лен\-но\-го параметра
 $\gamma$ свойством слу\-чай\-но\-сти не всегда оправдано.
 Действительно, в большинстве прикладных задач он является по своей природе точно не известным,
 но \emph{детерминированным}. Таким образом, приписывание ему случайной природы часто
 является вынужденной мерой, так как в предположении
 детерминированности $\gamma$ соот\-вет\-ст\-ву\-ющий
 критерий может не иметь седловой точки на множестве допустимых значений аргументов (см., например, \cite{m_m}
 для задач ми\-ни\-макс\-но\-го оценивания в линейных системах с дискретным
 временем). Выбор предлагаемого обобщенного квадратичного критерия качества связан именно с
 необ\-хо\-ди\-мостью <<компенсировать>> за\-час\-тую искусственную
 рандомизацию неизвестного па\-ра\-мет\-ра. Дело в том, что во многих
 прикладных задачах оценивания помимо множества неопределенности для
 параметра доступна различная дополнительная априорная и/или
 статистическая информация: оценки некоторых моментов этих
 параметров, их опорные значения и~т.\,д.
 Например, дополнительная априорная информация о распределении~$F$ вида
 $\me{F}{\gamma(\omega)}=\overline{a}$ может быть интерпретирована
 указанным образом и включена в критерий. Естественно, что данные такого рода могут сыграть
 значительную роль при оценивании, и их учет закладывается именно в обобщенном квад\-ра\-тич\-ном
 критерии.\footnote{Доктора А.\,И.~Матасов и Д.\,В.~Болотин в частной беседе с автором обращали внимание
 на то, что предлагаемый критерий оценивания близок по своей природе к популярному
 $H_{\infty}$-критерию, который <<штрафует природу>> за шумы высокой мощности в системе. Новый же критерий оценивания
штрафует распределения $F$ за большие отклонения случайных
 величин $a(\gamma)$ от их опорных значений
 $\overline{a}_t(O^t)$.}


 \section{Решение задачи минимаксного оценивания} %4

 Сигнальный процесс, подлежащий оцениванию, может быть представлен в виде
 \begin{equation*}
 g_t=\mathfrak{g}(\gamma_t)\theta_t\,,
% \label{eq:est_pr1}
 \end{equation*}
 где матрица
$\mathfrak{g}(\gamma_t)\ebd \|g_j(e_i,\gamma_t)\|_{\substack{{i=\overline{1,n}}\\ {j=\overline{1,l}}}}$.
 В случае, если распределение $F$ известно, СК-оп\-ти\-маль\-ная оценка $\widehat{g}^F_t$
 данного процесса определяется %по теореме~1
в соответствии с Предложением~1

\noindent
 \begin{equation}
 \widehat{g}^F_t=
\left( \int\limits_{\mathcal{C}}\mathbf{1}\widetilde{\theta}_t(q)F(dv)
\right)^{-1}
\int\limits_{\mathcal{C}}\mathfrak{g}(q)\widetilde{\theta}_t(q)F(dq)\,,
 \label{eq:opt_est_2}
 \end{equation}
 где $\widetilde{\theta}_t(q)$~--- ненормированное распределение
 состояния $\theta_t$, определяемое уравнением фильтра
Вонэма~(\ref{eq:wonham}) в предположении $\gamma(\omega)=q$.
 При этом ненормированное математическое ожидание $\widetilde{g}^F_t \ebd \me{F}{\Phi_t
 g_t|\mathcal{O}_t}$ вычисляется по формуле
 \begin{equation*}
 \widetilde{g}^F_t=
\int\limits_{\mathcal{C}}\mathfrak{g}(q)\widetilde{\theta}_t(q)F(dq)\,.
% \label{eq:opt_est_2_1}
 \end{equation*}
 Необходимо отметить, что
 \begin{equation*}
 \widehat{g}^q_t \ebd \frac{\mathfrak{g}(q)\widetilde{\theta}_t(q)}
{\mathbf{1}\widetilde{\theta}_t(q)}
% \label{eq:opt_est_3}
 \end{equation*}
 является условным математическим ожиданием оцениваемого процесса $g_t$ в
 случае, когда распределение $F$ известно и сосредоточено в точке $q \in
 \mathcal{C}$,~а
 \begin{equation*}
 \widetilde{g}^q_t \ebd
 \mathfrak{g}(q)\widetilde{\theta}_t(q)
% \label{eq:opt_est_4}
 \end{equation*}
 является соответствующим ненормированным ожиданием. Очевидно, что имеет место сле\-ду\-ющая связь между
 $\widetilde{g}^F$ и $\widetilde{g}^q$:
 \begin{equation*}
 \widetilde{g}^F_t=
\int\limits_{\mathcal{C}}\widetilde{g}^q_tF(dq)\,.
% \label{eq:opt_est_5}
 \end{equation*}
 Прежде чем сформулировать основное утверждение о решении задачи
 минимаксного оценивания, введем в рассмотрение следующий векторный случайный процесс
 $v_t=v(z_t) \in \mathbb{R}^{(l+r+3)\times 1}$:
 \begin{equation*}
 v_t\ebd \col
 (1,\|a(\gamma)\|^2,a(\gamma),\|g_t\|^2,g_t)
% \label{eq:opt_est_6}
 \end{equation*}
 и соответствующие ему ненормированные условные ожидания
 $\widetilde{v}_t^q(q,O^t)$, вычисленные по одноточечным распределениям,
 сосредоточенным в точках $q \in \mathcal{C}$.

 \paragraph*{Предложение 2.}Пусть для системы наблюдения~(\ref{eq:markobsys_1})
 выполнены предположения \ref{con_1}$^\prime$, \ref{con_2}--\ref{con_5}, тогда верны 
 следующие утверждения:
 \begin{itemize}
 \item[(а)] если существует решение двойственной задачи (зависимость оценок от наблюдений $O^t$ в формуле ниже опущена)
 \begin{multline}
 \displaystyle \widehat{\mathrm{F}} \in \Aarg \max_{F \in \mathbb{F}}
 \left\{\left(\widehat{\|g_t\|^2}^F-\|\widehat{g_t}^F\|^2\right) -{}\right.\\
 {}- \left(\widehat{\|a_t\|^2}^F -\|\widehat{a}_t^F\|^2\right)-{}\\
 \left. {}-  \me{F} {\|a(\gamma)-\widehat{a}_t^F\|^2} \right\}\,,
 \label{eq:saddle_2}
\end{multline}
 то целевая функция $J_{O^t}(F,\overline{g}_t)$ при
 $\mathbf{P}_F$-почти всех $O^t$ ($F \in \mathbb{F}$ --- любое допустимое распределение
 из множества неопределенности) имеет седловую точку
 $(\widehat{\mathrm{F}},\widehat{\mathbf{g}}_t)$ на множестве $\mathbb{F} \times
 \mathbb{G}$: наихудшее распределение $\widehat{\mathrm{F}}$
 является решением~(\ref{eq:saddle_2}), а
 $ \widehat{\mathbf{g}}_t=\widehat{\mathbf{g}}_t(O^t)
 = \widehat{g}^{\widehat{\mathrm{F}}}_t(O^t)
% \label{eq:saddle_3}
 $
---~байесовская оценка, вычисленная с помощью формул~(\ref{eq:wonham})
 и~(\ref{eq:opt_est_2}) для наихудшего распределения~$\widehat{\mathrm{F}}$;
 при этом оценка $\widehat{\mathbf{g}}_t$ является решением
 задачи~(\ref{eq:problem}) минимаксного апостериорного оценивания сигнального
 процесса $g_t$;
 \item[(б)]
 если оценка $\widetilde{v}_t^q(q,O^t)$ является непрерывной функцией аргумента
 $q$ при фиксированной
 траектории наблюдений $O^t$, то решение двойственной задачи~(\ref{eq:saddle_2})
 гарантированно существует; при этом также существует вариант наихудшего распределения
 $\widehat{\mathrm{F}}$, сосредоточенного не более чем в $l+r+4$ точках
 множества $\mathcal{C}$.
 \end{itemize}


Доказательство предложения~2 аналогично
доказательству соответствующей теоремы о минимаксном оценивании в
скрытых марковских моделях~\cite{bor_2007_2}.
 \begin{description}
\item[Замечание 1.] Множество $\mathbb{F}$ допустимых распределений, рассмотренное в
 предложении~2, может быть заменено на любое
 подмножество, замкнутое в слабой топологии.
 При этом истинность утверждения~(а) предложения сохранится, в
 то время как выполнение~(б) не гарантировано.
\item[Замечание 2.]
 В утверждении~(б) предложения~2 говорится о существовании
 дискретного варианта наихудшего распределения. Вообще говоря,
 наихудшее распределение в данной задаче не единственно и не
 обязательно дискретно. Характеризация всего множества наихудших
 распределений является чрезвычайно трудной зада\-чей, выходящей за
 рамки предмета исследований данной статьи.
 \begin{figure*}%[p] fig1+2
\vspace*{1pt}
\begin{center}
\mbox{%
\epsfxsize=123.911mm
\epsfbox{bor-1.eps}
}
\end{center}
\vspace*{-9pt}
 \Caption{Геометрическая иллюстрация минимаксного фильтра Вонэма:
(\textit{а})~случай №\,1 взаимного расположения множества возможных оценок
 $\widehat{\theta}^{\widehat{\mathrm{F}}}_t(O^t)$
 и наихудшей оценки $\mathbf{p}$;
(\textit{б})~случай №\,2 взаимного расположения  множества возможных оценок
 $\widehat{\theta}^{\widehat{\mathrm{F}}}_t(O^t)$
 и наихудшей оценки $\mathbf{p}$
 \label{f1bor}}
 \vspace*{-3pt}
\end{figure*}

\item[Замечание 3.]
 Из вида двойственной задачи~(\ref{eq:saddle_2}) следует, что
 выбор наихудшего распределения зависит от вида оцениваемого
 сигнала, т.\,е.\ от функции $g(z_t)$: для разных оцениваемых сигналов
 оно будет разным, и в общем случае <<равномерно наихудших>>
 распределений, видимо, не существует.
\item[Замечание 4.]
 В частном случае задачи оценивания~(\ref{eq:problem}), когда
 $g_t=\theta_t$, $a(\gamma)\equiv 0$ и $\overline{a}(O^t)\equiv 0$ (\emph{минимаксный фильтр
 Вонэма}),
 решение задачи минимаксного оценивания имеет простую геометрическую интерпретацию.
 \end{description}

 Рассмотрим случайный вектор $X$, имеющий дискретное распределение
 на множестве $S_n$ со средним $p \ebd \me{p}{X}$, и определим, какое
 распределение $X$ максимизирует критерий
 $$
 \me{p}{\|X-\me{p}{X}\|^2}=\mathrm{tr} \{ \diag (p) -pp^*\}=1-\|p\|^2\,,
 $$
 т.\,е.\ обладает <<наибольшим разбросом>> вокруг своего математического
 ожидания. Оказывается, что таким свойством обладает дискретное
 равномерное распределение на $S_n$:
 $\mathbf{p}=\col(1/n,\ldots,1/n)$ и при этом
 $$
\max_{p \in
 \Pi}\me{p}{\|X-\mathbf{E}_p\{X\}\|^2}=1-\fr{1}{n}\,,
 $$
 где $\Pi \ebd\{x \in \mathbb{R}^n,\; x_i \geqslant 0,\;i=1,\ldots,n,\;
 \mathbf{1}x=1\}$~--- <<вероятностный симплекс>>, т.\,е.\ множество
 \emph{всех} возможных распределений $p$.
 Если множеством допустимых средних значений $p$ является некоторое
 подмножество $\mathcal{Q}$ <<вероятностного>> симплекса $\Pi$, то наибольшим
 разбросом обладает распределение со средним значением, ближайшим к
 равномерному,~т.\,е.\
 \begin{multline*}
 \displaystyle \Aarg\max_{p \in \mathcal{Q}}\mathrm{E}_{p}
\left \{ \| X-\mathrm{E}_{p}\{X\}\|^2\right \}={}\\
{}= \Aarg\min_{p \in \mathcal{Q}}\|p-\mathbf{p}\|^2=\Aarg\min_{p \in \mathcal{Q}}\|p\|^2\,.
 \end{multline*}
 Пусть $\mathcal{Q}=\widehat{\Theta}^{\mathbb{F}}_t(O^t)$ --- множество всех
 допустимых условных математических ожиданий $\widehat{\theta}^F_t(O^t)$, вычисленных с
 помощью~(\ref{eq:wonham}) и (\ref{eq:opt_est_2}) по фиксированной траектории
 наблюдений $O^t$. 
  Если $\mathbf{p} \in  \mathcal{Q}$, то наихудшее
 распределение $\widehat{\mathrm{F}}$ таково, что оно приводит условное
 математическое ожидание $\widehat{\theta}^{\widehat{\mathrm{F}}}_t(O^t)$
 в точку $\mathbf{p}$ (равномерное распределение) в момент времени~$t$
 (рис.~\ref{f1bor},\,\textit{а}).
%
 Если $\mathbf{p} \notin \mathcal{Q}$, то наихудшее распределение
 $\widehat{\mathrm{F}}$ таково, что оно приводит условное математическое
 ожидание в точку из $\mathcal{Q}$, ближайшую к равномерному
 распределению~$\mathbf{p}$ (рис.~\ref{f1bor},\,\textit{б}).


 \section{Иллюстративный пример: мониторинг состояния TCP-соединения
 в~условиях неопределенности}

 Рассмотрим модель Джилберта функционирования соединения по протоколу
 TCP~\cite{gilb}. Предполагается, что состояние данного соединения,
 недоступное прямому наблюдению, описывается \mbox{МСПКЧС} $\theta_t$ с двумя
 состояниями: <<хорошим>> ($\theta_t=e_1$) и <<плохим>> ($\theta_t=e_2$).
 Соответствующая система наблюдения, являющаяся частным случаем~(\ref{eq:markobsys}),
 имеет вид
 \begin{align*}
 \theta_t&=\theta_0+\int\limits_0^t
 \begin{bmatrix}
 -\lambda_1 & \lambda_1 \\
 \lambda_2 & -\lambda_2
 \end{bmatrix}^*\theta_{s-}\,ds+ M^{\theta}_t\,,\\
 Q_t &=
 \int\limits_0^t
 \begin{bmatrix}
 \mu_1 & \mu_2
 \end{bmatrix}
 \theta_{s-}\,ds+ M^{Q}_t\,,\\
 U_t &=
 \int\limits_0^t
 \begin{bmatrix}
 A_1 & A_2
 \end{bmatrix}
 \theta_{s-}\,ds+ \varepsilon W_t\,.
 %\label{eq:markobsys_2}
 \end{align*}

\begin{figure*} %fig3
\vspace*{1pt}
\begin{center}
\mbox{%
\epsfxsize=123.019mm
\epsfbox{bor-3.eps}
}
\end{center}
\vspace*{-9pt} \Caption{Наблюдения: сплошная черная линия --- масштабированные
 приращения непрерывных наблюдений $\Delta U_t/\Delta t$;
черные точки --- моменты потерь пакетов $Q_t$ \label{pic:pic1}}
\end{figure*}

 Матрица интенсивностей переходов $\Lambda$ процесса $\theta_t$ априори
 неизвестна, однако обычно известны диапазоны возможных значений ее элементов:
 $\lambda_i \in [\underline{\lambda}_i,
 \overline{\lambda}_i], \; i=1,2$.

 Далее, в протоколе TCP доступным наблюдению является поток потерь
 пакетов $Q_t$, относительно которого предполагается, что он
 является считающим с интенсивностью $\mu \theta$, зависящей от текущего состояния соединения $\theta_t$.
 Очевидно, $\mu_2 > \mu_1$, т.\,е.\ интенсивность потерь пакетов в
<<плохом>> состоянии выше, чем в <<хорошем>>. Истинные значения
 интенсивностей также точно не известны, но даны возможные диапазоны их
 изменений: $\mu_i \in [\underline{\mu}_i, \overline{\mu}_i]$, $ i=1,2$.

 Непрерывные наблюдения $U_t$ представляют собой интегральные
 исторические данные о времени подтверждения доставки пакетов (RTT):
 $A_1$ и~$A_2$ являются неизвестными средними значениями RTT, соответственно,
 в <<хорошем>> и <<плохом>> состояниях ($A_2 > A_1$), а винеровский процесс
 $\varepsilon W_t$ описывает флуктуации RTT. Вектор $A$ является
 неопределенным так же, как $\Lambda$ и $\mu$, и его множество
 неопределенности вновь задано: $A_i \in [\underline{A}_i, \overline{A}_i]$,
 $ i=1,2$.

 Задача состоит в оперативном оценивании состояния
 $\theta_t$ TCP-соединения в условиях априорной неопределенности как
 в уравнении состояния, так и в наблюдениях. Предлагаемую прикладную
 задачу можно рассматривать как частный случай исследованной задачи
 минимаксной фильтрации~(\ref{eq:problem}) (минимаксной фильтрации Вонэма).

 Численное моделирование проводилось со следующи\-ми значениями параметров:
 $\lambda_1=1{,}9$, $\lambda_2=5{,}1$, $A_1=29$, $A_2=51$, $\mu_1=0{,}5$,
 $\mu_2=60$,\linebreak 
  $\varepsilon=0{,}5$.

 Неопределенность была задана следующими границами:
 $\underline{\lambda}_1=0{,}5$, $\overline{\lambda}_1=2$,
 $\underline{\lambda}_2=5$, $\overline{\lambda}_2=25$,
 $\underline{A}_1=20$, $\overline{A}_1=30$,
 $\underline{A}_2=50$, $\overline{A}_2=120$,
 $\underline{\mu}_1=0{,}1$, $\overline{\mu}_1=1{,}1$,
 $\underline{\mu}_2=50$, $\overline{\mu}_2=100$.

 На рис.~\ref{pic:pic1} приведены масштабированные приращения непрерывных
 наблюдений ${\Delta U_t}/{\Delta t}$ ($\Delta t=$\linebreak $=0,01$)
 и моменты потерь пакетов $Q_t$.


 На рис.~\ref{f4bor} показано истинное значение индикатора
<<хорошего>> состояния $\theta_t^*e_1$ в сравнении с оценкой
фильтра Вонэма, вычисленной при известных значениях тройки
$(\Lambda,\mu,A)$, и предлагаемая минимаксная оценка.


На рис.~\ref{pic:pic3} сравниваются индикатор $\theta_t^*e_1$ с предлагаемой 
минимаксной оценкой и оценкой фильтрации Вонэма, в которой неопределенные 
$(\Lambda,\mu,A)$ заменены на их опорные значения~--- центры соответствующих 
множеств неопределенности: $\lambda_i^{gv}=$\linebreak 
$={(\underline{\lambda}_i+\overline{\lambda}_i)}/{2}$, 
$\mu_i^{gv}={(\underline{\mu}_i+\overline{\mu}_i)}/{2}$, 
$A_i^{gv}={(\underline{A}_i+\overline{A}_i)}/{2}$, $i=1,2$.


 Ясно, что качество оценивания, предоставляемое минимаксным фильтром
 и фильтром Вонэма, вычисленным по опорным значениям, не может
 быть корректно сопоставлено с качеством фильтра Вонэма, вычисленного при
 известных точных значениях $(\Lambda,\mu,A)$, из-за разной априорной
 информации, используемой при оценивании. Тем не менее для некоторой
 оценки $\nu$ состояния $\theta$ рассмотрим среднюю $\mathcal{L}_2$
 норму ее ошибки:
$$\fr{1}{T}\int\limits_0^T\|\theta_t-\nu_t\|^2dt\;.$$
 Будучи вычисленным для $T=10$ в этом примере, данный показатель качества
 оказался равен 0,00438 для фильтра Вонэма, вычисленного при известных
 $(\Lambda,\mu,A)$, 0,1753 для минимаксного фильтра, и 0,46246 для фильтра
 Вонэма, вычисленного по опорным значениям. Дело в том, что неопределенность
 параметров, соответствующих <<плохому>> состоянию, шире, и выбранные опорные
 значения оказались далеки от истинных значений. В~этом заключается причина
 неудовлетворительной идентифицируемости <<плохого>> состояния фильтром
 Во\-нэ\-ма, вычисленным по опорным значениям (см., например, интервалы
 (0,68, 1,63) и (1,9, 2,32) на рис.~\ref{pic:pic3}). На этих интервалах
 указанный фильтр Вонэма демонстрирует осциллирующий характер. Это означает,
 что для надежной идентификации <<плохого>> состояния фильтром Вонэма,
 вычисленным по опорным значениям, недостаточно наблюдений, а требуется более
 точная априорная информация. В этой же ситуации предлагаемый минимаксный
 фильтр оказывается более робастным, предлагая <<нейтральную>> равномерную
 оценку $\widehat{\theta}_t=\col(0{,}5,\; 0{,}5)$.

\end{multicols}

\begin{figure*} %fig4
\vspace*{1pt}
\begin{center}
\mbox{%
\epsfxsize=117.835mm
\epsfbox{bor-4.eps}
}
\end{center}
\vspace*{-9pt}
\Caption{Индикатор <<хорошего>> состояния ({серая жирная
линия}), оценка фильтра Вонэма при известных $(\Lambda,\mu , A)$
({черная сплошная линия}) и минимаксная оценка
({пунктирная линия})
\label{f4bor}}
\end{figure*}

\begin{figure*} %fig5
\vspace*{1pt}
\begin{center}
\mbox{%
\epsfxsize=117.835mm
\epsfbox{bor-5.eps}
}
\end{center}
\vspace*{-9pt}
\Caption{Индикатор <<хорошего>> состояния ({серая жирная линия}), оценка
 фильтра Вонэма при опорных значениях $(\Lambda,\mu,A)$ ({черная сплошная
 линия}) и минимаксная оценка ({пунктирная линия})} \label{pic:pic3}
\end{figure*}

\begin{multicols}{2}

 \section{Заключение} %6

 В статье представлены следующие результаты.
\begin{enumerate}[1.]
\item
Задача оценивания в системе наблюдения с МСПКЧС, содержащей
неопределенность в уравнении состояния и наблюдениях, сформулирована
в игровой (минимаксной) постановке.
\item
Предложено утверждение, определяющее решение указанной минимаксной
задачи оценивания: предъявлены условия существования седловой точки,
представлена двойственная задача оптимизации, определяющая наихудшее
распределение, а также найдены некоторые характеристики этих
распределений и соответствующих минимаксных оценивателей.
\item
Применимость предлагаемых минимаксных оценок проиллюстрирована на
примере оперативного оценивания состояния TCP-соединения в условиях
априорной неопределенности по историческим данным RTT и потоку
потерь пакетов.
\end{enumerate}

 В то же время, полученные результаты определяют задачи, требующие решения.
 Во-первых, практическое использование предлагаемых минимаксных оценок
 требует разработки эффективных численных схем реализации данных оценок.
 Во-вторых, актуальным является поиск легко проверяемых условий, гарантирующих
 существование\linebreak
  решения двойственной задачи. В-третьих, рас\-смот\-рение задач
 управления в соответствующей минимаксной постановке также выглядит
 весьма перспективным.

{\small\frenchspacing
{%\baselineskip=10.8pt
\addcontentsline{toc}{section}{Литература}
\begin{thebibliography}{99}
 \bibitem{kalmanbucy}
 \Au{Kalman~R.\,E., Bucy~R.\,S.}
 New results in
 linear filtering and prediction problems~// Trans. ASME, 1961.
 Ser.~D. Vol.~83. P.~{95--111}.

 \bibitem{wonham}
 \Au{Wonham~W.\,N.}
 Some applications of stochastic differential equations to optimal nonlinear filtering~//
 SIAM J.\ of Control, 1965. No.\,2. P.~347--369.

\bibitem{m_m}
\Au{Martin~C.\,J., Mintz~M.}
Robust filtering and
prediction for linear systems with uncertain dynamics:
 A game-theoretic approach~//
 IEEE Trans.\ Autom.\ Contr., 1983. Vol.~9. P.~\mbox{888--896}.


 \bibitem{an}
 \Au{Ананьев Б.\,И.}
 Минимаксная линейная фильтрация многошаговых процессов с
 неопределенными распределениями возмущений~//
 АиТ, 1993. №\,10. С.~\mbox{131--139}.

 \bibitem{k_k}
 \Au{Кац~И.\,Я., Куржанский~А.\,Б.}
 Минимаксная
 многошаговая фильтрация в статистически неопределенных ситуациях~//
 АиТ, 1978. №\,11. С.~\mbox{79--87}.


 \bibitem{p_m}
 \Au{Панков~А.\,Р., Миллер~Г.\,Б.}
 Фильтрация случайного процесса в статистически
 неопределенной линейной стохастической дифференциальной системе~//
 АиТ, 2005. №\,1. C.~\mbox{59--71}.

 \bibitem{s_l_p}
 \Au{Siemenikhin~K.\,V., Lebedev~M.\,V., Platonov~E.\,N.}
 Kalman filtering by
minimax criterion with uncertain noise intensity functions~// Proc.
Joint 44th IEEE Conf. on Decision and Control and European Control
Conf. (CDC-ECC'2005).~--- Seville, 2005. P.~\mbox{1929--1934}.

\bibitem{elliott2}
\Au{Elliott R.\,J., Aggoun~L., Moore~J.\,B.}
Hidden Markov
models: Estimation and control.~--- Berlin: Springer-Verlag, 1995.

 \bibitem{bor_2007_2}
 \Au{Борисов~А.\,В.}
 Минимаксное апостериорное оценивание в скрытых марковских моделях~//
 АиТ, 2007. №\,11. C.~\mbox{31--45}.

 \bibitem{bor_2007}
\Au{Борисов~А.\,В.}
 Минимаксное апостериорное оценивание марковских процессов с конечным
 числом состояний~// АиТ, 2008 (в печати).

\bibitem{gilb}
\Au{Gilbert~E.\,M.}
Capacity of a burst-noise channel~//
Bell Syst. Tech. J., 1960. №\,5. P.~\mbox{1253--1265}.

 \bibitem{l_sh}
 \Au{Липцер~Р.\,Ш., Ширяев~А.\,Н.}
 Теория мартингалов.~--- М.: Наука, 1986.

 \bibitem{doob}
 \Au{Дуб Дж.\,Л.}
 Вероятностные процессы.~--- М.: ИЛ, 1956.

 \bibitem{wong}
 \Au{Wong~E., Hajek~B.}
 Stochastic processes in engineering systems.~--- New York: Springer, 1985.

 \bibitem{elliott1}
 \Au{Эллиотт~Р.}
 Стохастический анализ и его приложения.~--- М.: Мир, 1986.

 \bibitem{l_sh_0}
 \Au{Липцер Р.\,Ш., Ширяев~А.\,Н.}
 Статистика случайных процессов.~--- М.: Наука, 1974.
\end{thebibliography}

}
}

\end{multicols}


\label{end\stat}