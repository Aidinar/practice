\def\stat{abstr}
{%\hrule\par
%\vskip 7pt % 7pt
\raggedleft\Large \bf%\baselineskip=3.2ex
A\,B\,S\,T\,R\,A\,C\,T\,S \vskip 17pt
    \hrule
    \par
\vskip 21pt plus 6pt minus 3pt }

\def\tit{CORRELATIONAL METHODS FOR ANALYTICAL INFORMATIONAL
MODELS OF~THE EARTH POLE FLUCTUATIONS DESIGN BASED ON A PRIORI DATA}

%1
\def\aut{I.\,N.~Sinitsyn}
\def\auf{IPI RAN, sinitsin@dol.ru}

\def\leftkol{\ } % ENGLISH ABSTRACTS}

\def\rightkol{\ } %ENGLISH ABSTRACTS}

\titele{\tit}{\aut}{\auf}{\leftkol}{\rightkol}


\noindent Modern statistical informatics methods are the basis of
investigation for fundamental problem ``Statistical Dynamics of the
Earth Pole Rotation.''
Approximate and exact correlational methods for 3--5-year analytical
informational models  with additive and parametric harmonic and stochastic
disturbances of the Earth Pole fluctuations design are developed. Examples
from informational resources ``Statistical Dynamics for Earth Pole Rotation''
are presented.
\label{st\stat}

 \KWN{Earth Pole fluctuations; analytical informational model;
a priori data; correlational characteristics; correlational methods; stochastic 
differential equations}

% \thispagestyle{headings}

\vskip 18pt plus 6pt minus 3pt

%\vfil

%2
\def\tit{DEVELOPMENT OF CANONICAL INFORMATION MODELS
FOR INTEGRATED INFORMATION SYSTEMS}
\def\aut{V.\,N.~Zakharov$^1$, L.\,A.~Kalinichenko$^2$,
I.\,A.~Sokolov$^3$, and S.\,A.~Stupnikov$^4$}

\def\auf{$^1$IPI RAN, vzakharov@ipiran.ru\\[1pt]
$^2$IPI RAN, leonidk@synth.ipi.ac.ru\\[1pt]
$^3$IPI RAN, isokolov@ipiran.ru\\[1pt]
$^4$IPI RAN, ssa@ipi.ac.ru}


%\def\leftkol{ENGLISH ABSTRACTS}

%\def\rightkol{ENGLISH ABSTRACTS}

\titele{\tit}{\aut}{\auf}{\leftkol}{\rightkol}

\noindent  The problem of unification of heterogeneous models of information 
resources representation including databases, ontologies, services, and 
processes for development of distributed information systems using 
heterogeneous information resources is considered. Special attention is drawn 
to the verifiable methods of  information model mapping and synthesis of 
extensible canonical information models. An architecture of Information Model 
Unifier as well as an example of mapping of specific information model into 
canonical one are presented. A comparison of proposed methods with known 
approaches is provided.


\KWN{information model; information model semantics; model mapping; model 
mapping commutativity; canonical model; canonical model synthesis; canonical 
model extension; refinement; model refinement verification; information model 
unifier; metacompilation}

%\vfil
 \vskip 18pt plus 6pt minus 3pt
% \vskip 24pt plus 9pt minus 6pt

%3
\def\tit{STATIONARY CHARACTERISTICS OF A MULTICHANNEL QUEUEING SYSTEM
WITH~SIMULTANEOUS REFUSALS OF SERVERS}

\def\aut{A.\,V.~Pechinkin$^1$, I.\,A.~Sokolov$^2$, and~V.\,V.~Chaplygin$^3$}
\def\auf{$^1$IPI RAN, apechinkin@ipiran.ru\\[1pt]
$^2$IPI RAN, isokolov@ipiran.ru\\[1pt]
$^3$IPI RAN, vchaplygin@ipiran.ru}


%\def\leftkol{ENGLISH ABSTRACTS}

%\def\rightkol{ENGLISH ABSTRACTS}

\titele{\tit}{\aut}{\auf}{\leftkol}{\rightkol}

\noindent The multichannel queueing system with a semi-Markovian input flow, a 
phase type distribution of the servicing and simultaneously unreliable servers failure 
is considered. Mathematical relations allowing to calculate the 
main stationary characteristics of system performance are found for different versions
of failure and restoration process.

\KWN{queueing system; unreliable servers}
\pagebreak

%\vful

% \vskip 24pt plus 9pt minus 6pt
%\vskip 6pt plus 3pt minus 3pt
%\vspace*{12pt}

%4
\def\tit{LINGUISTIC SIMULATION FOR MACHINE
TRANSLATION AND KNOWLEDGE MANAGEMENT SYSTEMS}

\def\aut{A.\,V.~Bosov$^1$ and A.\,V.~Ivanov$^2$}
\def\auf{$^1$IPI RAN, AVBosov@ipiran.ru\\[1pt]
$^2$IPI RAN, AIvanov@ipiran.ru}
\def\leftkol{ENGLISH ABSTRACTS}

\def\rightkol{ENGLISH ABSTRACTS}

\titele{\tit}{\aut}{\auf}{\leftkol}{\rightkol}

\noindent
Portal solution that has been implemented in the context of RAS informatization
program is considered. The elaborated architecture of informational portal is
described, base requirements for software solution
are discussed. The choices of informational technologies are proven.

\KWN{Internet standards; multitier architecture; web-portal;
adapter; .Net technology}


% \vskip 24pt plus 9pt minus 6pt
\vskip 18pt plus 6pt minus 3pt

%5
\def\tit{BAYESIAN ESTIMATION IN~OBSERVATION SYSTEMS WITH~MARKOV JUMP
PROCESSES: GAME-THEORETIC APPROACH}

\def\aut{A.\,V. Borisov}

\def\auf{IPI RAN, ABorisov@ipiran.ru}

\def\leftkol{ENGLISH ABSTRACTS}

\def\rightkol{ENGLISH ABSTRACTS}

\titele{\tit}{\aut}{\auf}{\leftkol}{\rightkol}

\noindent
The paper is devoted to the mutual estimation/identification problem
for the finite-state Markov jump processes given both the diffusion and
counting observations. The dynamic and observation equations depend on the
random parameter with uncertain distribution having a known support set.
An objective is a conditional expectation of some quadratic function 
of filtering estimate. 
The paper contains an assertion concerning saddle-point
existence in the stated minimax problem. The least favorable distribution
and the minimax estimate are characterized in terms of the dual optimization
problem. Practical applicability of the obtained results is demonstrated
by the illustrating example of TCP link status monitoring under uncertain
channel characteristics.

\KWN{Wonham filter; minimax estimation; generalized quadratic criterion; Zakai 
equation}

%\vfil
\vskip 18pt plus 6pt minus 3pt


%6
\def\tit{BAYESIAN APPROACH TO QUEUEING SYSTEMS AND RELIABILITY CHARACTERISTICS}

\def\aut{A.\,A.~Kudryavtsev$^1$ and S.\,Ya.~Shorgin$^2$}

\def\auf{$^1$M.\,V.~Lomonosov Moscow State University,
%Факультет вычислительной математики и кибернетики МГУ им. М.\,В.~Ломоносова,
nubigena@hotmail.com\\[1pt]
$^2$IPI RAN, sshorgin@ipiran.ru}

%\def\leftkol{ENGLISH ABSTRACTS}

%\def\rightkol{ENGLISH ABSTRACTS}

\titele{\tit}{\aut}{\auf}{\leftkol}{\rightkol}

\noindent Bayesian approach to the statement of certain problems of the queueing 
theory and the reliability theory is considered.  The method provides the 
randomization of system characteristics with regard to a priori distributions 
of input parameters. This approach could be used to calculate average values 
of performance and reliability characteristics for the large groups of systems or 
devices. In the paper, the review of preceding results and new results 
concerning a case of Erlang a priori distribution is presented.

\KWN{Bayesian approach; queueing systems; reliability;  mixed distributions;
modeling}
%байесовский подход; системы массового обслуживания; надежность; смешанные
%распределения; моделирование}

 \label{end\stat}
 %\pagebreak