\def\stat{bosov}
\def\tit{ПРОГРАММНАЯ ИНФРАСТРУКТУРА ИНФОРМАЦИОННОГО WEB-ПОРТАЛА}
\def\titkol{Программная инфраструктура
информационного web-портала}

\def\aut{А.\,В.~Босов$^1$, А.\,В.~Иванов$^2$}
\def\autkol{А.\,В.~Босов, А.\,В.~Иванов}

\titel{\tit}{\aut}{\autkol}{\titkol}

%\renewcommand{\thefootnote}{\arabic{footnote}}
\footnotetext[1]{Институт проблем информатики Российской академии наук, AVBosov@ipiran.ru}
\footnotetext[2]{Институт проблем информатики Российской академии наук, AIvanov@ipiran.ru}

\Abst{В работе представлено портальное решение, реализованное в рамках программы
информатизации РАН. Описана разработанная архитектура информационного портала,
обсуждаются основные требования к программному решению, обосновывается выбор
базовых технологий.}

\KW{стандарты Интернет; многозвенная архитектура; web-портал; адаптер;
технология .NET}

\vskip 24pt plus 9pt minus 6pt

\thispagestyle{headings}

\begin{multicols}{2}


\label{st\stat}

\section{Введение}

Сравнительно недавно проблематике Интер\-нет-порталов уделялось повышенное
внимание практически всеми участниками ИТ-сообщества, что связывалось, прежде всего,
с грандиозными ожиданиями от применения новой технологии. На сегодняшний момент
можно констатировать, что тенденции развития портальных технологий свидетельствуют
о замедлении процесса интенсивного роста. Первоначальный ажиотаж фактически сошел
на нет, сократилось число участников рынка портальных решений~\cite{1bos}, количество
инновационных решений в новых версиях продуктов ведущих производителей невелико.

 Основной причиной неоправданных ожиданий, по-видимому, можно считать
трудности, с которыми сталкиваются пользователи <<коробочных>> решений,
пытающиеся применять их в своих целях, высокие требования к уровню квалификации
обслуживающего и эксплуатирующего персонала и~т.\,д. Однако вместе с тем изменился
и подход к использованию портальных решений~--- теперь они рассматриваются как
часть более крупных систем (например, систем управления предприятием), а не как
самостоятельный продукт. При этом портал становится тем, ради чего и создавался,~---
инструментом для решения вполне определенной задачи создания единой <<точки
доступа>> к различным данным и сервисам в рамках унифицированного web-интерфейса.
Таким образом, для потребителей становится осознанной реальная ценность портала как
инструмента web-интеграции в широком смыс\-ле. Об этой новой тенденции
свидетельствует и рост числа внедрений порталов~\cite{2bos}.

 Однако нельзя признать, что технология порталов достигла высшей точки своего
развития. Представляется, что отмеченное замедление связано с технологическими
трудностями, решение которых могло бы повысить эффективность использования
порталов. Так, практически невозможно, не прибегая к программированию, интегрировать
в портале информационные системы предприятий, с большими сложностями сопряжено
управление неструктурированной и слабоструктурированной информацией, недостаточно
развиты средства интеграции web-сайтов. Не всегда эти проблемы может решить
потребитель портального решения, следствием чего являются обращение к
производителю для консультаций или разработки дополнительных модулей и
дополнительные расходы на эксплуатацию.

 Таким образом, необходимость поиска новых решений в технологии порталов
очевидна. Именно попытка избежать проблем, присущих известным портальным
решениям, и обусловила постановку задачи создания собственного решения в рамках
программы информатизации Российской академии наук, действующей с
2002~г.~\cite{3bos}. За время выполнения портального проекта сложилась
представительная кооперация, получены существенные результаты, части которых и
посвящена данная работа.

 \begin{figure*} %fig1
 \vspace*{1pt}
\begin{center}
\mbox{%
\epsfxsize=158.529mm
\epsfbox{bos-1.eps}
}
\end{center}
\vspace*{-9pt}
 \Caption{Многозвенная архитектура портала
 \label{f1bos}}
 \end{figure*}

 Созданное в РАН портальное решение изна\-чаль\-но проектировалось с целью
интеграции мно\-же\-ст\-ва источников слабоструктурированной ин\-фор\-ма\-ции, 
прежде всего научной тематики. Посколь\-ку существующие классификации порталов 
не позволяли однозначно отнести разработанную web-систему к одному из классов, 
для ее обозначения
использован термин <<информационный web-портал>>~[3--5], %\cite{3bos}--\cite{5bos},
под которым понимается web-сис\-тема:
 \begin{itemize}
\item обеспечивающая возможность предоставления информации по большому
набору взаимосвязанных тематик всем заинтересованным в ней пользователям;
\item обладающая способностью предоставлять интерактивные услуги как
пользователям, так и другим web-системам;
\item способная обрабатывать разнородную информацию, поступающую из
различных информационных источников;
\item имеющая средства интеграции с информационными источниками,
позволяющие объединять схемы данных, системы защиты и поиска;
\item способная работать как со структурированными, так и с
неструктурированными и/или слабоструктурированными данными.
\end{itemize}


 Далее описывается архитектура и особенности программной инфраструктуры
информационного web-портала.

\section{Постановка задачи и~общая схема решения}

 В процессе работы над проектом информационного web-портала сформировался
следующий список основных требований, которым должно удов\-ле\-тво\-рять
создаваемое решение:
 \begin{itemize}
\item разделение оформления и информационного содержания формируемых
web-страниц;
\item возможность повторного использования программного кода, формирующего
контент на web-страницах;
\item возможность формирования содержания страниц как сочетания статической (редко
меняющейся) и динамической (генерируемой при обращении к службам или
web-сервисам) информации;
\item возможность унификации работы с различными информационными источниками
(файлами, базами данных, web-сервисами, прикладными информационными системами);
\item гибкость в управлении конфигурацией web-системы;
\item возможность расширения функциональности системы за счет подключения
новых информационных источников, компонентов и модулей, не затрагивая уже
существующих и не меняя механизмов работы с данными;
\item обеспечение управления информационным порталом с помощью web-интерфейса;
\item комплексное обеспечение безопасности web-системы;
\item поддержка современных стандартов пред\-став\-ле\-ния и обработки данных
(XML, SOAP, RSS и~пр.).
\end{itemize}

 Из перечисленных требований наиболее <<тяжелым>>, несомненно, является
возможность эффективного взаимодействия с множеством разнородных источников
информации. Для достижения максимальной гибкости при поддержке взаимодействия с
источниками целесообразно выделение отдельного уровня программной инфраструктуры.
Кроме того, любая web-система как система с тонким клиентом требует выделения
отдельного уровня для серверной бизнес-логики портала. С учетом этого для
информационного web-портала была выбрана многозвенная клиент-серверная
архитектура, означающая, что в программной инфраструктуре явно выделяются два
промежуточных уровня: бизнес-логики и интеграции данных. На уровне бизнес-логики
выполняются алгоритмы работы с данными. Уровень интеграции данных обеспечивает
единый метод доступа к данным, находящимся в различных хранилищах данных либо
доступным через вызовы web-сервисов (рис.~\ref{f1bos}).

\section{Реализация требований к~порталу}

\subsection{Разделение оформления и~информационного содержания
web-страниц}

 Под оформлением web-страниц понимается не только стиль текстовых и
графических элементов сформированной HTML-страницы, но и общие для нескольких
страниц элементы пользовательского интерфейса (например, меню, заголовки и~т.\,п.), а
также компоновка элементов страницы (<<модульная сетка>>). В такой постановке задача
разделения оформления и информационного содержания может быть решена несколькими
способами:
 \begin{itemize}
\item с помощью фреймов HTML, путем разделения элементов пользовательского
интерфейса и информационного содержания страницы и размещения их в разных
фреймах;
\item с помощью клиентского или серверного XSLT-преобразования,
заключающегося в применении к информационному содержанию,
представленному в формате XML, набора правил, определяющих порядок
формирования HTML-представления;
\item с помощью шаблонов, содержащих общий для всех web-страниц код и
фиксированные позиции для размещения элементов пользовательского интерфейса
и информационного содержания страницы.
\end{itemize}

 Перечисленные способы принципиально отличаются как по требованиям,
предъявляемым к программной инфраструктуре, так и по условиям применения и
решаемым задачам. Так, использование фреймов значительно менее удобно в пользовании
по сравнению с другими способами и не позволяет получать удовлетворительные
результаты с точки зрения безопасности, а XSLT-преобразование достаточно ресурсоемко
и допускает представление входных данных только в формате XML. Перед
информационным web-порталом стояла задача обеспечения возможности формирования
HTML-страниц с использованием данных из независимых информационных блоков,
получаемых из различных источников данных. По этим причинам для минимизации
требований к источникам данных~--- поставщикам контента портала, снижения
требований к ресурсам, упрощения управления дизайном в качестве средства разделения
оформления и информационного содержания были использованы \textbf{шаблоны}.


 Сами по себе шаблоны не несут информационного наполнения, они являются лишь
контейнером для блоков информации и элементов пользовательского интерфейса и
позволяют избежать дублирования общей информации. С помощью шаблонов могут быть
реализованы различные варианты представления одних и тех же данных, например
страница может иметь стандартный вид и вид, предназначенный для печати.
Использование шаблонов в\linebreak 
насто\-я\-щее время является общепринятым подходом в
большинстве web-сис\-тем. К~со\-жа\-ле\-нию, производители web-технологий
(PHP, JSP, ASP и
ASP.NET) вплоть до последнего времени не предлагали стандартизованных решений для
реализации шаблонов. Все доступные системы (FastTemplate, Jakarta Turbine, ASP
Templates) были разработаны либо в рамках движения Open Source, либо сторонними
фирмами. В связи с этим подходы к реализации и использованию шаблонов в различных
web-сис\-те\-мах существенно отличаются. Реализация %\linebreak
шаб\-ло\-нов может сводиться к простой
подстановке текста, генерации содержимого с по\-мощью различных языков сценариев
либо построению шаб\-ло\-нов на основе компонентного подхода.

 Следует отметить, что системы для построения сайтов на основе шаблонов,
получившие название ``web framework'', дополнительно включают прикладные
программные интерфейсы, разделение HTML и программного кода и поддержку
архитектуры MVC.

 Для информационного web-портала был выбран компонентный подход с
использованием програм\-мных визуальных компонентов. Это обуслов\-ле\-но удобством
формирования шаблона из ком\-по\-нен\-тов (не прибегая к написанию про\-грам\-мно\-го кода),
возможностью программно управлять дизайном страниц, наличием инструментальных
средств для визу\-аль\-ной разработки шаблонов и как результат~--- большей гибкостью
решения. В отличие от шаб\-ло\-нов на основе подстановки текста шаблоны на основе
компонентного подхода могут про\-грам\-мно конфигурироваться, их проще сделать
персонализируемыми, а их программный код существенно проще, чем в случае шаблонов,
формируемых с помощью сценариев. Шаблоны могут вообще не содержать программного
кода, если логика работы страницы может быть обеспечена средствами инфраструктуры
портала.

\subsection{Повторное использование программного кода} %3.2

 При формировании HTML-страницы для каж\-дого информационного блока или
элемента пользовательского интерфейса сервером портала выполняется некоторый
программный код, связанный с соответствующим шаблоном и ге\-не\-ри\-ру\-ющий
 HTML-представление. Поскольку один и тот же код должен выполняться при
формировании разных страниц и может использоваться в разных шаблонах, возникает
потребность выделить его в самостоятельный программный модуль. Кроме того,
необходим механизм для добавления такого программного кода к выбранной странице
или шаблону. Решить эту задачу позволяет использование \textbf{визуальных
компонентов}.

 Визуальные компоненты представляют собой программные модули, формирующие
фрагменты HTML-представления из блоков информации, полученных в результате
выполнения запросов к информационным источникам и службам портала. Визуальные
компоненты, по сути, являются частью используемой среды разработки, так как
наследуются от определенных классов и имеют поддержку определенных интерфейсов.
Визуальные компоненты легко размещаются на страницах и шаблонах, их можно
настраивать с помощью стандартных механизмов, можно организовывать обработку
событий, генерируемых компонентами в ответ на действия пользователя. В программную
инфраструктуру портала включен ряд стандартных компонентов для использования на
страницах, другие компоненты могут быть созданы сторонними разработчиками на
основе требований, принятых в портале, и интерфейсов, предоставляемых
инфраструктурой портала.

 Идеология визуальных компонентов используется во многих существующих
программных продуктах, например Microsoft Digital Dashboard (DDB). Однако DDB~---
это технология преимущественно нижнего уровня для связывания рабочего места
пользователя с нужными информационными и вычислительными ресурсами. Она требует
разработки дополнительных компонентов, чтобы можно было решать поисковые задачи
или задачи управления логической структурой страниц. Кроме того, DDB ограничивает
клиента использованием строго определенных программных продуктов и
необходимостью установки дополнительных клиентских компонентов.

 Следует отметить, что реализация визуальных компонентов в информационном
web-портале опирается на работу с сервером интеграции~--- элементом программной
инфраструктуры, описанным далее. Здесь же важно отметить, что использование сервера
интеграции позволяет разработчикам визуальных компонентов абстрагироваться от
деталей работы как с данными, так и с поставляющими их источниками. При этом все
возможности программной инфраструктуры портала остаются доступными.

 Инфраструктура портала содержит целый ряд технических решений, позволяющих
создавать новые компоненты: это и среда разработки, и наличие прикладного
программного интерфейса, позволяющего реализовать разграничение доступа,
управ\-ле\-ние оформлением, привязку к источникам %\linebreak 
\mbox{данных,} а также
взаимодействие визуальных компонентов на странице.

\subsection{Сочетание статической и~динамической информации} %3.3

 На страницах портала должна отображаться как статическая информация,
получаемая, например, из файлов или баз данных, так и динамическая, формируемая как
результат запроса пользователя к~базе данных, web-сервису или прикладной
ин\-формационной системе, взаимодействующей с порталом. При этом на одной странице
должна отображаться информация из нескольких информационных источников.

 Эта задача решена в информационном web-портале двумя способами: с помощью
визуальных компонентов и \textbf{технологии связывания данных} (databind).
Визуальные компоненты позволяют формировать на страницах меню, списки
новостей, выполнять поисковые запросы. Однако, поскольку набор визуальных
компонентов для шаблона фиксирован, возникает потребность в дополнительном
механизме для получения данных и их вывода на странице. Для этого создан
механизм связывания данных.

 Связывание данных~--- это механизм для выполнения запросов к информационным
источникам и помещения их результатов в тексте web-страницы с помощью сценария,
располагаемого на самой странице. В момент получения текста страницы из
информационного источника производится анализ содержащихся в нем сценариев
связывания данных, выполнение этих сценариев и подстановка результатов запросов
вместо текста сценариев.

 Для реализации данного механизма в состав программной инфраструктуры портала
введен интерпретатор сценариев, реализованный в составе специального визуального
компонента.

\subsection{Унификация работы с~различными информационными
источниками} %3.4

 Для решения задачи поддержки взаимодействия портала с различными
информационными источниками (как традиционными базами данных, так и
 web-сер\-ви\-са\-ми, прикладными информационными системами и~т.\,п.)
потребовалась разработка механизма интеграции данных. В имеющейся реализации
информационного web-портала этот механизм включает три составляющих:
адаптеры~\cite{6bos}, механизм командных запросов и сервер ин\-те\-гра\-ции~\cite{7bos}.
{\looseness=1

}

 Источники информации в портале представлены посредством традиционного
механизма \textbf{адап\-те\-ров}. Адаптер в портале реализует связку
 <<ме\-нед\-жер--ре\-сурс>>, в которой менеджер ресурса, реализуемый как отдельный
программный модуль, играет роль представителя информационного источника в портале
и отвечает за взаимодействие программной инфраструктуры портала с ресурсом
(конкретным источником информации или службой). Использование адаптеров
обеспечивает унификацию доступа к информационным источникам и службам,
возможность работы с разнородными источниками информации и гибкого подключения
новых источников, позволяет снижать нагрузку на источник информации за счет
кэширования данных уже выполненных запросов, а также реализовывать разграничение
доступа на уровне отдельных источников информации.

 При использовании этого подхода адаптеры должны разрабатываться для всех
вновь подключаемых к порталу ресурсов. Для упрощения их создания предпринят ряд
шагов, в частности отказ от жесткой стандартизации функциональности адап\-те\-ров,
наличие нескольких уровней интеграции информационных ресурсов, использование
распространенных форматов представления данных.

 \textbf{Механизм командных запросов} в инфраструктуре портала используется для
работы с данными в информационных источниках. Для чтения, создания, модификации и
удаления данных используются командные запросы, которые представляют собой списки
команд, представленные в формате XML. Команды связываются с каждым адаптером и в
унифицированном виде задают действия, выполняемые для доступа к данным
соответствующего информационного источника. Команды могут содержать параметры,
что необходимо, к примеру, при доступе к базам данных и другим хра\-ни\-лищам.
{\looseness=1

}

 Использование такого подхода более удобно, чем фиксация функциональности
адаптеров источников данных в стандартизированных и, следовательно, неизменяемых
интерфейсах портала, и в то же время не приводит к необходимости реа\-ли\-за\-ции прямого
доступа к конкретным базам данных и службам, а также обработки данных
не\-по\-сред\-ст\-вен\-но в шаблонах страниц или визуальных ком\-по\-нен\-тах. Поддержка
механизма командных \mbox{запросов} реализуется \textbf{сервером интеграции}, яв\-ля\-ющим\-ся
\mbox{частью} программной инфраструктуры портала. Этот элемент инфраструктуры
обеспечивает поддержку единой в рамках портала схемы данных, получение запросов на
доступ к данным и выполнение соответствующих запросу команд адап\-те\-ра\-ми, сбор и
консолидацию результатов выполнения %\linebreak
 команд.
{%\looseness=1

}

 Имеющаяся на данный момент реализация информационного web-портала уже
содержит целый ряд реализованных адаптеров, позволяющих подключать к порталу как
стандартные информационные источники (базы данных, web-сервисы), так и специальные
системы, к которым в основном относятся прикладные и интегрированные решения,
созданные в организациях РАН. Наиболее ярким и многофункциональным из
реализованных примеров является компонент, обеспечивающий подключение к порталу и
эффективное взаимодействие с ним Интегрированной системы информационных ресурсов
РАН~\cite{8bos}.

\subsection{Управление конфигурацией web-системы} %3.5

 Конфигурационная информация web-системы включает множество параметров
различных типов, в том числе и достаточно сложно структурированных, доступ к которым
из программного кода должен быть достаточно быстр и удобен. В то же время
конфигурационная информация должна быть доступна администратору системы, в связи с
чем желательно представление этой информации в человекочитаемой форме. Эти
требования несколько противоречат друг другу, поэтому в качестве компро\-мис\-са было
выбрано представление конфигурационной информации на языке XML. Практически все
используемые настройки информационного web-портала представлены в этом формате.

 Основным элементом конфигурационной информации для портала является
\textbf{логическая структура страниц}. При отображении web-стра\-ниц
используется ассоциированный с каждой страницей запрос, содержащий команды на
получение информации от информационных источников и служб портала. Страницы,
отображаемые средствами информационного web-портала, регистрируются в едином
конфигурационном XML-фай\-ле, стандартизированном в рамках портала структуры.
Конфигурационный файл обеспечивает хранение таких ключевых параметров
страницы, как URL, права доступа к странице, ее положение в главном меню,
используемый для формирования HTML-стра\-ни\-цы шаблон, командный запрос,
используемый для получения информационного содержания страницы, описание
страницы для карты сайта портала %\linebreak
 и~др.


 В конфигурационном файле обеспечена возможность использовать один набор
параметров не только для одной страницы, но и для группы страниц. При этом
переменная часть URL страницы преобразуется в параметр запроса, передаваемого
шаблону. Такая возможность используется при доступе к базам данных и позволяет
идентифицировать каждый объект, хранящийся в базе данных, собственным URL.

\subsection{Расширение функциональности системы} %3.6

 Наличие средств расширения принципиально важно для информационного
 web-портала. Очевидно, что интеграция данных не может решить всех задач при
подключении нового информационного источника. Например, для навигации по ресурсам
информационного источника может потребоваться формирование специального раздела
меню внут\-ри существующей в портале логической структуры страниц на основе
предоставляемых источником данных. Для отображения ресурсов может потребоваться
дополнительный шаблон и~т.\,д. В связи с этим в программной инфраструктуре портала
изначально были заложены механизмы расширения функциональности.

 В настоящий момент функциональность портала может быть расширена с
помощью \textbf{внешних модулей}, являющихся, по сути, частью адаптеров, с помощью
которых к порталу подключаются источники информации. Эти модули позволяют
динамически построить раздел меню на основе полученных из информационного
источника данных и подключить дополнительные шаблоны для отоб\-ра\-же\-ния данных
источника. Модули должны поддерживать определенные в программной инфраструктуре
интерфейсы, чтобы взаимодействовать с порталом.

\subsection{Управление порталом с~помощью web-интерфейса} %3.7

 Как и большинство подобных систем, информационный web-портал оснащен
 web-интерфейсом администрирования~--- в настоящее время наличие такого
интерфейса обязательно для размещения web-систем на необслуживаемых серверах.

 Построение пользовательского и административного web-интерфейсов имеет ряд
принципиальных отличий от построения интерфейса прикладных программ. Наиболее
существенное отличие связано с использованием протокола HTTP, взаимодействие по
которому не является непрерывным (disconnected) и не имеет состояния сеанса (stateless).
Web-запрос не несет в себе историю предыдущих запросов.

В связи с этим традиционный
для клиентских прикладных программ интерфейс, который отражает предысторию
действий пользователя и обеспечивает согласованную обработку данных при %\linebreak
исполь\-зо\-ва\-нии нескольких окон приложения, не может быть реализован средствами
 web-тех\-но\-ло\-гий напрямую. Однако существует несколько мето\-дов, позволяющих в
той или иной мере преодолеть указанную проблему. Это механизм сессий, файлы
``cookie'', использование скрытых полей форм и передача состояния через параметры
web-за\-проса.
{\looseness=1

}

 \textbf{Сессии пользователей} обеспечивают сохранение состояния сеанса в
создаваемом на web-сервере объекте. Идентификатор данного объекта передается через
заголовок web-запроса или как параметр строки запроса. Эта технология не ограничивает
объем сохраняемых данных сеанса и работоспособна при отсутствии или запрещении в
web-браузерах поддержки файлов ``cookie''. В то же время, данная технология
затрудняет масштабирование web-системы, так как сессия пользователя привязана к
одному web-серверу и необходимы дополнительные действия для координации
одновременной работы группы web-серверов. Кроме того, эта технология поддерживается
не всеми web-серверами и требует дополнительных ресурсов web-сервера.

 \textbf{Файлы ``cookie''} сами по себе могут исполь\-зоваться для сохранения
состояния сеанса. Их использование не создает трудностей для масштабирования системы
и не требует специальной поддержки со стороны web-сервера. Однако объем информации,
который может быть помещен в заголовках web-запросов, ограничен, а необходимость
передачи заголовков в web-запросе увеличивает общее время передачи информации.
Кроме того, передача файлов ``cookie'' может быть запрещена, отсутствовать в
клиентских web-браузерах, либо блокироваться брандмауэрами. Тем не менее практика их
использования общепринята и чаще всего является оптимальным решением.

 Состояние сеанса может сохраняться в \textbf{скрытых полях} HTML-форм. Такой
вариант применим, только если используется отправка формы на сервер (submit). При
переходе по ссылке или пере\-на\-прав\-ле\-нии запроса на сервере данные полей формы
теряются. Однако данный механизм удобен для пошагового заполнения форм и
организации серверной обработки событий. Например, технология Microsoft
ASP.NET~\cite{9bos} содержит реализацию такого механизма, известную как состояние
отображения (ViewState).

 Передача состояния сеанса через \textbf{параметры строки запроса} не
предъявляет специальных требований к web-браузерам и web-серверам и возможна
практически в любом случае. Однако объем информации, передаваемой таким образом,
минимален из-за ограничения максимальной длины строки запроса (не более
255~символов в большинстве web-браузеров). Кроме того, для передачи данных требуется
перекодировка, а сами данные, вследствие открытости для пользователя, легко могут быть
изменены.

 Технология ASP.NET, которая была использована при реализации обсуждаемого
решения, предоставляет необходимые инструменты и классы для использования всех
перечисленных методов. Таким образом, в информационном web-портале используется
сочетание нескольких методов для сохранения состояния сеанса, выбираемых исходя из
простоты и эффективности решения конкретных задач. Именно в зависимости от
требований к взаимодействию отдельных элементов интерфейса используются файлы
``cookie'', параметры запроса и скрытые поля форм.

 Другая проблема построения интерфейса связана с организацией взаимодействия
его элементов и, в частности, с использованием нескольких окон для модификации
данных. В общем случае окна web-браузера являются независимыми, и если два окна
используются для модификации одних и тех же данных, изменения, производимые в
одном из окон, не отражаются в другом. В настоящий момент единственным способом
организации динамического взаимодействия элементов интерфейса является
использование сценариев. К сожалению, уровень поддержки языков сценариев в
 web-браузерах различных производителей сильно отличается.
 Недостаточным
является и уровень соответствия стандартам. Производители web-браузеров делают
акцент на поддержке собственных, несовместимых расширений языков сценариев и тем
самым ставят разработчиков сценариев в зависимость от своих продуктов. Это усложняет
разработку из-за необходимости создания разных версий сценариев. Но, если число
пользователей ограничено, что имеет место в случае административного интерфейса,
проще ориентироваться на использование одного web-браузера. Так, административный
интерфейс информационного web-портала ориентирован на использование web-браузера
Internet Explorer.

 Сочетание перечисленных методов сохранения состояния сеанса и динамического
взаимодействия элементов интерфейса позволило сделать административный интерфейс
достаточно функциональным и удобным и обеспечить удаленное управление
информационным web-порталом.

\subsection{Обеспечение безопасности web-системы} %3.8

\vspace*{-2pt}

 Наиболее критичная проблема построения web-систем~--- проблема безопасности.
В отличие от локально выполняемой на компьютере программы все данные,
передаваемые между сервером и клиентом, потенциально уязвимы. Кроме того, сами
данные могут представлять собой большую опасность при отсутствии надлежащего
контроля. Язык разметки HTML предоставляет множество способов создания активного
содержимого, которое может быть выполнено без участия или с минимальным участием
пользователя в web-браузере. Отображение данных, вводимых пользователем и
вклю\-ча\-ющих активное содержимое, может причинить различный вред, начиная с
повреждения web-страниц и заканчивая кражей данных сеанса других пользователей,
в~том числе данных аутентификации. При этом использование защищенных протоколов типа
HTTPS не является препятствием для активного содержимого~\cite{10bos}.

 В связи с этим при построении системы безопасности портала решались две
задачи~\cite{11bos}: реализация собственно разграничения доступа к информационным
ресурсам и защита web-интерфейса.

 К вопросам организации системы безопасности относится и такой инструмент
интеграции информационных систем, как технология однократной регистрации SSO
(Single Sign On). Технология подразумевает создание такой инфраструктуры портала, при
которой данные аутентификации пользователя, полученные после однократной
регистрации, могут использоваться всеми подсистемами.

 При построении системы разграничения до\-сту\-па было решено отказаться от
создания единой сис\-те\-мы авторизации, так как практическая реа\-ли\-зация
такого решения означает вмешательство в сис\-те\-мы безопасности всех
интегрируемых средствами портала информационных источников. Кроме того, сложность
управления подобной централизованной службой оказалась бы слишком ве\-ли\-ка.
В~свя\-зи с этим программная ин\-фра\-струк\-тура портала ограничена единым
сервисом аутентификации и механизмом авторизации, предо\-став\-ля\-ющим базовые
инструменты для управления ролевой моделью разграничения доступа, позволяющим
передать решение задач авторизации на конкретные подсистемы и информационные
источники.

 Для обеспечения безопасности web-интерфейса в информационном web-портале
используется несколь\-ко методов~\cite{11bos}. В первую очередь, это использование
защищенного протокола HTTPS. Протокол резко снижает вероятность раскрытия и
подмены данных, передаваемых между сервером и клиентом.
Во-вторых, это
использование циф\-ро\-вой подписи данных состояния сеанса, а также ограничение срока
годности этих данных. В-треть\-их, сочетание фильтров активного содержимого и
 HTML-кодирования данных, получаемых от пользователя. Цифровая подпись
данных сеанса формируется с использованием алгоритмов RSA и SHA1. Фильтры
активного содержимого, построенные на основе регулярных выражений, обеспечивают
обнаружение и удаление опасных входных данных. HTML-кодирование предотвращает
попадание любого HTML-кода в неизменном виде в выходные данные.

\subsection{Использование современных стандартов} %3.9

 Отображение HTML-страниц в Информационном web-портале осуществляется в
стандарте \textit{HTML~4.01 Transitional}, в ближайших версиях будет обеспечена
поддержка стандарта \textit{XHTML~1.0 Transitional}. Для стилевого оформления
используется стандарт \textit{CSS~1.0}.

 В качестве стандарта для обмена управляющей информацией и обработки данных
используется язык \textit{XML}. В частности, для представления командных запросов,
хранения параметров конфигурации и при передаче данных адаптеров используются
специальные подмножества языка XML, стандартизированные в рамках портала.

 В настоящее время использование XML стало общепринятым, что объясняется
удобством этого языка для автоматизированной обработки. Немаловажную роль играет и
наличие большого количества продуктов, поддерживающих подготовку и обработку
представленных в XML данных.

 Помимо XML в информационном web-портале используется формат
\textit{RSS~2.0} для экспорта новостной информации. Выполнение запросов и обмен
информацией с web-сервисами портала производится в формате \textit{SOAP}, описание
web-сервисов портала выполнено в стандарте \textit{WSDL}.

\section{Архитектура Информационного web-портала} %4

 Все перечисленные выше положения нашли свое отражение в существующем
программном решении. На основе этих концепций была разработана архитектура
решения (рис.~\ref{f2bos}). На рисунке выделены основные подсистемы
информационного web-портала и составляющие их компоненты. Ключевыми для
портала являются сервис аутентификации, система представления и управления
контентом, сервер интеграции и файловое хранилище.


 \textbf{Систему аутентификации} образуют следующие компоненты~\cite{11bos}:
 \begin{itemize}
\item сервис аутентификации;
\item интерфейс системы аутентификации;
\item администратор пользователей.
\end{itemize}

 \textbf{Систему представления и управления контентом} образуют следующие
компоненты~\cite{5bos}:
 \begin{itemize}
\item модуль работы с конфигурационной информацией;
\item редактор конфигурационной информации;
\item шаблоны представления;
\item визуальные компоненты;
\item шаблоны редактирования;
\item подсистемы администраторов статей, но\-во\-стей, новостных дайджестов;
\item подсистема форума;
\item конвертор RSS.
\end{itemize}


 \textbf{Файловое хранилище} образуют следующие компоненты:
 \begin{itemize}
\item интерфейс файлового хранилища;
\item администратор файлового хранилища.
\end{itemize}

 Помимо компонентов подсистем в состав архитектуры портала входят следующие
компоненты:
 \begin{itemize}
\item web-сервер;
\item модуль виртуализации путей;
\item сервер интеграции;
\item сервер командных запросов;
\item сервер баз данных;
\item служба поиска;
\item файловый сервер;
\item подключенные к порталу внешние информационные источники;
\item подключенные к порталу внешние информационные службы/сервисы.
 \end{itemize}

 Ключевой компонент портала~--- \textbf{web-сервер}, обеспечивающий прием
запросов, генерируемых web-браузерами пользователей, передачу запросов компонентам
портала и возврат клиентам HTML-представления результатов запросов,
сформированного компонентами портала. В качестве web-сервера в рассматриваемом
решении используется Microsoft Internet Information Services. Системы
представления и управления контентом и файло-\linebreak
\vspace*{-12pt}
\pagebreak
\end{multicols}

 \begin{figure} %fig2
\vspace*{1pt}
\begin{center}
\mbox{%
\epsfxsize=125.584mm
\epsfbox{bos-2.eps}
}
\end{center}
%\vspace*{-9pt}
 \Caption{Архитектура портала
 \label{f2bos}}
 \end{figure}

\begin{multicols}{2}

\noindent
вого хранилища являются приложениями ASP.NET,
поэтому для обеспечения их работы на web-сервере установлена среда выполнения .NET
Framework.

Кроме того, функциональность web-сер\-ве\-ра расширена \textbf{модулем
виртуализации путей}. Этот модуль необходим для преобразования виртуальных URI
(Uniform Resource Identifier) страниц web-сер\-ве\-ра в физические пути к шаблонам
пред\-став\-ле\-ния, которые фактически формируют HTML-пред\-став\-ле\-ние запрошенных
страниц. Данное преобразование обозначается термином ``URL rewriting''.
Преобразование путей позволяет делать пути страниц более наглядными и
запоминающимися, упрощает их ввод с клавиатуры, облегчает индексирование портала
поисковыми системами. Преобразование путей осуществляется модулем виртуализации
на основе конфигурационной информации, хранящейся в файле формата XML. Этот файл
описывает соответствие URI и шаблонов представления. Модуль выполнен как ISAPI
(Internet Server Application Program Interface) фильтр на языке C++.

 \textbf{Шаблоны представления} предназначены для формирования
 HTML-пред\-став\-ле\-ния требуемой страницы. Web-стра\-ни\-ца собирается из блоков
(заголовка, меню, блоков новостей и контента и~т.\,п.). HTML-пред\-став\-ле\-ние каждого
такого блока формируется соответствующим визуальным ком\-по\-нен\-том. В~шаблоне
осуществляется конкатенация полу\-чен\-ных фрагментов HTML-кода. Шаблон %\linebreak
представления служит контейнером для набора %\linebreak
 визуальных и невизуальных компонентов
интерфейса. Кроме того, шаблон содержит программный код, осуществляющий
получение необходимых па\-ра\-мет\-ров компонентов от модуля работы с конфигурационной
информацией. Каждый шаблон пред\-став\-ле\-ния выполнен как обычная ASPX-стра\-ни\-ца.

 \textbf{Визуальные компоненты} портала обеспечивают визуализацию блоков
информации на web-стра\-ни\-це. Компоненты осуществляют получение информационного
содержания из различных источников (данных файла конфигурации, хранилища портала,
внешних информационных источников), а затем формируют на его основе фрагмент
HTML-кода, используемый затем при сборке страницы. Взаимодействие с этими
источниками строится с помощью \textbf{сервера интеграции}~\cite{7bos}. Для каждой
отображаемой страницы определен набор параметров, обеспечивающих получение
необходимых данных от сервера интеграции. \textbf{Модуль работы с конфигурационной
информацией} осуществляет определение этого набора параметров на основе адреса
страницы, после чего сервер интеграции возвращает набор данных с помощью
компонента ``dataset''.

 Для создания и модификации конфигурационной информации предназначен
компонент <<\textbf{Редактор конфигурационной информации}>>. Этот компонент
отвечает за создание и модификацию страниц сайта портала, формирование логической
структуры страниц (меню) сайта, назначение шаб\-ло\-нов отображения страницам. Редактор
отоб\-ра\-жа\-ет структуру сайта, описываемую конфигурационным файлом портала
Content\_Structure.xml в наглядной форме, и позволяет редактировать ее, выполняя
необходимые изменения в конфигурационном файле. Редактор представляет собой
 ASPX-страницу, обеспечивающую выполнение сле\-ду\-ющих действий:
 \begin{itemize}
\item создание\,/\,удаление\,/\,изменение раздела, под\-раздела;
\item изменение позиции раздела в логической структуре отображения;
\item добавление\,/\,удаление\,/\,изменение страницы раз\-дела;
\item изменение позиции страницы в логической структуре отображения;
\item редактирование данных о разделах и страницах для выбранного языка (русский,
английский) представления;
\item вызов шаблона редактирования для страницы раздела;
\item вызов редактора командного запроса;
\item управление разграничением доступа к стра\-нице.
\end{itemize}
Информация о разделах и страницах, как и весь остальной контент сайта портала, может
быть представлена на различных языках. Список поддерживаемых языков и
зарегистрированных шаблонов не ограничен и хранится в файле конфигурации редактора,
являющемся XML-документом.

 С редактором конфигурационной информации связаны \textbf{шаблоны
редактирования}. Они используются для наглядного визуального формирования
необходимых наборов параметров для шаблонов представления. Например, шаблоны
редактирования обеспечивают пользователю интерфейс для выбора статей, разделов
новостей, режимов отображения визуальных компонентов и других параметров,
используемых при настройке страниц. Каждому шаблону представления соответствует
шаблон редактирования.

 \textbf{Сервис аутентификации} отвечает за идентификацию пользователя,
выполняющего текущий запрос к web-странице. Данный компонент выполняет проверку
наличия у пользователя действительного билета безопасности~\cite{11bos}. Билет
безопасности выдается пользователю при регистрации~--- на основании введенных им
имени и пароля~--- и затем передается в каждом последующем запросе с помощью файлов
``cookie''. В билете содержатся сведения о назначенных пользователю ролях и членстве
в группах. Срок действия билета ограничен. На основе информации, получаемой от
данного модуля, модуль конфигурации предоставляет или запрещает доступ пользователя
к запрошенному ресурсу.

 Регистрация новых пользователей и назначение им прав осуществляются
компонентом <<\textbf{Администратор пользователей}>>. Этот компонент представляет
собой интерфейс к базе данных, в которой хранятся учетные данные пользователей и
информация о присвоенных им ролях и членстве в группах. Роль пользователя определяет
его права на доступ к той или иной функциональности. Роли могут назначаться как путем
прямого присвоения, так и путем включения пользователей в группы. С~по\-мощью
компонента <<Администратор пользователей>> производится заведение новых
пользователей, модификация их учетных данных, создание и модификация ролей и групп.

 Для создания и модификации информации в собственном хранилище портала (там
могут быть размещены такие типы контента, как статьи, новости, документы) в
хранилище портала используются компоненты <<\textbf{Администратор статей}>>,
<<\textbf{Администратор новостей}>> и <<\textbf{Администратор дайджестов}>>. Все
эти компоненты представляют собой интерфейсы к хранилищу портала и предназначены
для создания и модификации соответствующих видов документов, организации этих
документов в виде набора папок и управления правами доступа. Необходимость
использования нескольких интерфейсов обусловлена разной организацией работы с
новостями и статьями, соображениями удобства работы и разграничения доступа.

 Новости, подготовленные с помощью компонента <<Администратор новостей>>,
могут быть экспортированы в формате RSS. Для этого в составе системы представления и
управления содержанием предназначен компонент <<\textbf{Конвертер RSS}>>.

 В составе программной инфраструктуры име\-ет\-ся также компонент
<<\textbf{Форум}>>, служащий для постро\-ения системы обмена сообщениями по
стандартной схеме форумов. Этот компонент обеспечивает управление структурой
разделов форума, содержит пользовательские интерфейсы для просмотра тем и
сообщений форума, создания новых тем и сообщений, администрирования.

 Взаимодействие компонентов программной инфраструктуры с хранилищем
портала, внешними информационными источниками и службами осуществляется через
\textbf{сервер интеграции}. Он реализует промежуточный уровень взаимодействия с
данными, обеспечивая единый метод работы с данными, находящимися в различных
информационных источниках и службах, имеющих различную организацию, структуру
данных (схему) и интерфейс доступа.

 Хранилище данных портала построено на основе \textbf{сервера баз данных}, в
качестве которого исполь\-зуется Microsoft SQL Server. Взаимодействие с сервером баз
данных осуществляется посредством \textbf{сервера командных запросов}~\cite{7bos},
который реализует промежуточный уровень взаимодействия с данными, находящимися в
реляционной базе данных.

 Компонент <<\textbf{Интерфейс файлового хранилища}>> предоставляет
пользовательский интерфейс, обеспечивающий просмотр папок, поиск и загрузку файлов,
помещенных в файловое хранилище. Метаинформация (размер, тип, описание, источник
файла и~пр.) размещается в базе данных. Для размещения новых файлов, задания их
атрибутов и управления структурой папок используется компонент
<<\textbf{Администратор файлового хранилища}>>.

 \textbf{Служба поиска}~--- один из важнейших компонентов портала, реализующий
функции полнотекстового и атрибутного поиска информации~\cite{12bos}. Каждый
информационный компонент или служба, будучи зарегистрированными в сервере
ин\-те\-гра\-ции, декларируют свою поисковую функциональность. Эта декларация включает
перечень %\linebreak
 по\-исковых команд, список атрибутов, по которым возможен атрибутный поиск,
и описание возвращаемых результатов поиска. Служба поиска, по\-лучая эти метаданные от
сервера интеграции, обеспечивает интеграцию поисковых возможностей компонентов и
служб в подсистему полнотекстового поиска. Эта подсистема имеет единый
пользовательский интерфейс и доступна с любых страниц сайта портала через
стандартный визуальный компонент полнотекстового поиска. Для атрибутного поиска
служба, руководствуясь полученными метаданными, обеспечивает формирование
поискового пользовательского интерфейса. Формирование интерфейса осуществляется
путем добавления к стандартному поисковому шаблону необходимых пользовательских
элементов управления в соответствии с названиями, типами и именами полей атрибутов, а
также стандартного поля результатов поиска, представляющего собой таблицу, колонки
которой соответствуют возвращаемым полям результата поиска.

 Служба поиска также управляет асинхронной передачей запросов адаптерам,
сбором, консолидацией и ранжированием результатов поиска и, наконец,
предоставлением их пользователю.

\section{Алгоритмы обработки пользовательского запроса} %5

 Каждому обращению пользовательского браузера к web-серверу портала с
запросом на получение страницы соответствует определенная последовательность
действий, выполняемая про\-грам\-мной инфраструктурой портала для формирования
запрашиваемой в web-запросе информации.

 Сразу после приема HTTP-запроса web-сер\-ве\-ром портала он передается на
обработку модулю виртуализации путей, который преобразует URI запроса в физический
путь к файлу. Преобразование осуществляется на основе данных конфигурации,
результатом является физический путь к шаб\-ло\-ну запрошенной страницы. После этого
шаб\-лон выполняется web-сервером, обрабатываясь в контексте портального приложения
ASP.NET. Преж\-де чем управление будет передано коду шаблона, оно передается модулям
сервиса аутентификации и конфигурации.

 Сервис аутентификации идентифицирует пользователя, в данный момент
обращающегося к web-сер\-ве\-ру, используя билет безопасности: выполняется проверка
наличия в запросе пользователя билета безопасности, проверка правильности этого %\linebreak
билета, выдача билета безопасности при вводе пользователем имени и пароля на странице
ре\-ги\-ст\-рации.
{\looseness=1

}

 В модуле конфигурации определяется и сохраняется для дальнейшего
использования набор параметров, которые должны быть переданы коду шаб\-ло\-на для
обработки текущего запроса. Этот набор определяется в соответствии с полученным в
запросе URL. Далее модуль конфигурации обрабатывает запрос авторизации ASP.NET
приложения. К~каж\-дой web-стра\-ни\-це доступ может быть открыт либо всем
пользователям, либо только аутентифицированным пользователям, либо пользователям,
вла\-деющим определенными ролями. Если уровень %\linebreak 
полномочий пользователя
недостаточен, доступ к странице блокируется. Администратор портала может управлять
декларативной авторизацией на уровне страниц путем модификации со\-от\-вет\-ст\-ву\-ющих
параметров конфигурации.

 Следует отметить, что авторизация в портале реализована также и в ряде других
подсистем, например администраторах статей и новостей, форуме, файловом хранилище.
Обеспечение разграничения доступа к ресурсам в данном случае возлагается на
конкретные подсистемы и информационные компоненты, централизованной же является
только служба аутентификации. Подробнее организация и функционирование системы
безопасности портала рассматриваются в~\cite{11bos}.

 После выполнения кода обработки событий в модулях авторизации и
конфигурации управление передается непосредственно коду шаблона. Шаб\-лон может
содержать любой код, использующий любые услуги, предоставляемые .NET и ASP.NET
(такие как работа с базами данных, файлами, вызов системных функций и~т.\,п.). Однако
типовой сценарий выполнения кода шаблона, отображающего некоторую информацию в
соответствии с запросом пользователя, выглядит следующим образом:
 \begin{itemize}
\item получение данных из различных источников (адаптеров) в соответствии с
запросом пользователя (параметрами HTTP-запроса, значениями полей формы,
текущим языком и~т.\,п.);
\item преобразование полученной информации в формат HTML.
\end{itemize}

 Типовой сценарий выполнения кода шаблона, модифицирующего данные в
соответствии с запросом пользователя, выглядит так:
 \begin{itemize}
\item модификация информации в соответствующих источниках, при этом
исходными данными обычно являются значения полей формы;
\item формирование HTML-страницы с результатом операции или переадресация
на другую стра\-ницу.
\end{itemize}

 При решении задач получения и модификации данных в обоих типовых сценариях
используется сервер интеграции, который принимает командный запрос,
содержащий набор команд, по\-лу\-ча\-ющих или модифицирующих необходимые данные,
и набор параметров, необходимых для выпол\-не\-ния команд. Также серверу
интеграции передается компонент ``dataset'' (набор данных). При выполнении
сервером интеграции команды чтения этот компонент заполняется данными,
полученными в результате запроса. При выполнении команды модификации данные,
находящиеся в компоненте ``dataset'', переносятся в соответствующий
информационный компонент. Сервер интеграции, анализируя командный запрос,
определяет необходимые для его выполнения адаптеры и направляет в них
отдельные команды, используя информацию, предо\-став\-лен\-ную при
регистрации информационного источника. Затем, в соответствии со схемой данных
источника, сервер интеграции производит заполнение компонента ``dataset''.

\section{Структура программных модулей реализованного
решения} %6

 Большинство компонентов текущей реализации информационного web-портала
выполнены с использованием технологии Microsoft .NET в среде разработки
Microsoft Visual Studio. Структура основных программных модулей показана
на рис.~\ref{f3bos}.

 Программную инфраструктуру портала обра\-зуют:
 \begin{itemize}
\item программные компоненты, отвечающие за задачи аутентификации, проверки
прав доступа, доступ к адаптерам портала, поддержание логической структуры
отображения информации;
\item инструментальные средства портала, обеспечивающие решение
административных задач и задач редактирования данных;
\item средства разработки, обеспечивающие создание новых компонентов и
ресурсов, используемых в портале.
\end{itemize}

 В качестве ключевых модулей, образующих основу инфраструктуры, можно
выделить следующие:
 \begin{itemize}
\item Interfaces~--- модуль, описывающий внутренние интерфейсы портала;
\end{itemize}

\pagebreak
\end{multicols}


 \begin{figure} %fig3
\vspace*{1pt}
\begin{center}
\mbox{%
\epsfxsize=127.994mm
\epsfbox{bos-3.eps}
}
\end{center}
%\vspace*{-9pt}
 \Caption{Основные программные модули портала
 \label{f3bos}}
 \end{figure}

\begin{multicols}{2}

\noindent
\begin{itemize}
\item BaseAuth~--- модуль, отвечающий за проверку прав доступа к страницам;
\item PortalConfig~--- модуль, обеспечивающий работу с файлом
конфигурации портала, опи\-сы\-ва\-ющим логическую структуру отображения;
\item XSUnitingServer~--- модуль, обеспечивающий взаимодействие со службой
командных запросов.
\end{itemize}

 В модуле Interfaces описаны интерфейсы, обеспечивающие взаимодействие
визуальных компонентов между собой и работу с логической структурой отображения
портала, описанной в файле конфигурации.

 Модуль BaseAuth предоставляет ряд функций, позволяющих определить:
аутентифицирован ли пользователь, инициировавший HTTP(HTTPS)-запрос, каковы его
полномочия, проверить наличие у него необходимых полномочий и в случае их
отсутствия запретить доступ к странице и потребовать авторизации.

 Для работы с логической структурой отображения портала и определения
необходимых атрибутов страниц (таких как текст командного запроса, название пункта
меню и~т.\,д.) по их URL предназначен модуль PortalConfig. Модуль позволяет избежать
прямой работы с данными файла конфигурации и предоставляет ряд функций для поиска
и навигации в структуре страниц. Кроме того, модуль обеспечивает хранение различных
настроек пользовательского интерфейса, например текущего языка и параметров
персонализации. Хранение настроек в течение сеанса работы пользователя реализовано с
помощью механизма ``cookie''.

 Модуль XSUnitingServer представляет собой интерфейс сервера интеграции. С
помощью этого модуля шаблоны и службы портала могут выполнить командные запросы
и использовать сервисные функции для подстановки параметров запросов (в том числе
встроенных, например текущего языка интерфейса).

 Набор визуальных компонентов, обес\-пе\-чи\-ва\-ющих отображение информации и
элементов пользовательского интерфейса, содержится в библиотеках PortalContentLib и
PortalAdvancedControls. %\linebreak
 Следу\-ет отметить, что эти модули включают только специфичные
для портала компоненты: различные меню, компонент переключения языков интерфейса,
стандартные заголовки и подвалы страниц. Для решения большинства других задач
достаточно использования стандартных компонентов, имеющихся в среде разработки
Microsoft Visual Studio .NET. Набор стандартных компонентов достаточно велик и
включает такие элементы, как текстовые метки, поля ввода, списки, поля со списками,
флажки, таблицы, изображения, кнопки и~др. Визуальные компоненты могут быть
выполнены и в виде отдельных программных файлов, подключаемых на этапе дизайна
страницы.

 Параметры отображения любого компонента могут настраиваться на этапе дизайна
(с помощью страницы свойств) и в процессе выполнения страницы (в программном коде
шаблона). Если для дизайна шаблона не используется среда Visual Studio .NET, параметры
отображения могут быть указаны непосредственно в коде ASPX-страницы, так как
каждый используемый в шаблоне компонент описывается с помощью расширенных тэгов
языка HTML. Компоненты поддерживают использование CSS классов, поэтому возможно
управление дизайном страниц и традиционным способом~--- через таблицы стилей.

 Для использования программных компонентов в стандартных шаблонах портала
необходимо, чтобы компоненты были разработаны в среде Microsoft Visual Studio .NET с
использованием одного из поддерживаемых платформой .NET языков (C\#, VB, J\# или
C++). Сами шаблоны страниц в портале также построены на базе ASPX-страниц. Это не
является обязательным требованием, однако при использовании других технологий
удобство использования программной инфраструктуры портала будет снижено.

 В общем случае ASPX-страница шаблона содержит:
 \begin{itemize}
\item элементы оформления, созданные с помощью HTML и CSS;
\item набор визуальных компонентов;
\item компонент ``dataset'', связанный со всеми компонентами, использующими
портальную технологию связывания данных;
\item программный код, обеспечивающий логику работы шаблона.
\end{itemize}

 Наряду с HTML-представлением, хранящимся в виде файла с расширением ASPX,
шаблон содержит программный код, хранящийся в виде файла с расширением CS, ссылка
на который указывается в соответствующем ASPX-файле. Программный код шаблона
получает управление при поступлении очередного web-запроса, в случае если содержимое
запрошенной страницы не было сгенерировано ранее и помещено в кэш среды
выполнения .NET (.NET framework).

 Шаблоны редактирования устроены аналогично шаблонам представления. Отличия
заключаются только в использовании визуальных компонентов, обеспечивающих не
только отображение, но и модификацию информации, и наличии программного кода,
выполняющего формирование корректного командного запроса, сохраняемого в файле
конфигурации, и помещение модифицированной информации в соответствующее
хранилище.

 Поскольку шаблоны редактирования вызываются по фиксированным URL, то для
определения текущей редактируемой страницы им передается параметр, содержащий URL
страницы в логической структуре отображения. С помощью модуля PortalConfig в коде
шаблона можно получить значения всех необходимых параметров для редактируемой
страницы.

 Как уже отмечалось выше, шаблоны редактирования~--- не единственный
инструмент редактирования содержания. Каждая служба (прикладная сис\-те\-ма) может
иметь собственный пользовательский интерфейс для управления хранилищем
инфор\-ма\-ции. В этом случае пользовательский интерфейс для управления хранилищем
представляет собой одну или несколько независимых ASPX-стра\-ниц, код которых
обеспечивает выполнение всех необходимых действий по управлению содержанием
хранилища информации. Эти страницы могут не регистрироваться в файле конфигурации,
однако при этом проверка прав доступа должна выполняться непосредственно в коде
страниц с помощью функций модуля BaseAuth. Тем не менее им в полной мере доступны
все службы и функции программной инфраструктуры портала.

\section{Заключение} %7

 Информационный web-портал, представленный в данной статье, является
результатом совместного труда коллектива авторов и выполнен, в том числе,
и в интересах РАН. Описанное решение на данный момент внедрено, имеет статус
основного источника информации о РАН в сети Интернет и доступно по адресу
{\sf www.ras.ru}. Реализация данного проекта замечена и привлекает внимание
не только использующих его результаты академических организаций, но и
производителей аналогичного программного обеспечения~\cite{13bos, 14bos}.

 Основными особенностями архитектуры и практической реализации программного
решения, описанного в данной статье, являются:
 \begin{enumerate}[(1)]
\item использование современных методов построения крупных web-систем, таких как
многозвенная архитектура, виртуализация путей, разделение оформления и
содержания, поддержка работы как со структурированными, так и
неструктурированными данными, наличие web-интерфейса администрирования,
предоставление данных как для просмотра пользователями, так и для обработки
web-системами;
\item соответствие современным требованиям в час\-ти представления данных,
безопасности, принципа единой регистрации;
\item применение современных технологий, таких как Microsoft .NET, web-сервисы, и
стандартов (SQL, XML, SOAP, WSDL, RSS);
\item полная независимость и самостоятельность решения;
\item использование компонентного подхода как при отображении информации, так и при
работе с информационными источниками;
\item механизм унификации работы с разнородными информационными
источниками на основе технологии командных запросов;
\item наличие методологии и развитого инструментария, позволяющих расширять
возможности как системы представления портала, так и средств интеграции.
\end{enumerate}

{\small\frenchspacing
{%\baselineskip=10.8pt
\addcontentsline{toc}{section}{Литература}
\begin{thebibliography}{99}
\bibitem{1bos}
\Au{Коули С.} Конец рынка порталов~// Computerworld, %. Изд-во <<Открытые системы>>,
2005. №\,32. {\sf http://www.osp.ru/cw/2005/32/ 036\_1.htm}.

\bibitem{2bos}
Порталы сегодня~// Computerworld, %. Изд-во <<Открытые системы>>,
2004. №\,36. {\sf http: //www.osp.ru/cw/2004/36/000\_39.htm}.

\bibitem{3bos}
\Au{Соколов И.\,А., Босов~А.\,В., Бездушный~А.\,Н.}
О Информационном Web-портале Российской академии наук~// Системы и средства
информатики, %. М.: Наука,
2003. Вып.~13. С.~119--138.

\bibitem{4bos}
\Au{Босов А.\,В., Серебряков В.\,А.}
Информационный веб-портал Российской академии наук~// Материалы конференции
<<Государство в XXI~веке>>, Москва, 10~ап\-реля 2003~г. {\sf
http://www.microsoft.com/rus/events/ gov2003/schedule.asp}.

\bibitem{5bos}
\Au{Босов А.\,В., Иванов~А.\,В.}
 О реализации системы управления содержанием информационного Web-пор\-та\-ла~//
Информационные технологии и вы\-чис\-ли\-тель\-ные сис\-те\-мы, %. М.: ИМВС РАН,
2004. №\,4. С.~85--103.

\bibitem{6bos}
\Au{Босов А.\,В., Чавтараев Р.\,Б.}
Управление информационными компонентами Web-портала РАН~// Сис\-те\-мы и
средства информатики, %. М.: Наука,
2003. Вып.~13. С.~156--171.

\bibitem{7bos}
\Au{Босов А.\,В., Полухин А.\,Н.}
Технология доступа и интеграции данных в информационном web-портале~//
Сис\-те\-мы и средства информатики, %. М.: Наука,
2006. Вып.~16. С.~355--373.

\bibitem{8bos}
\Au{Bezdushnyi A.\,N., Zhizhchenko A.\,B., Kulagin~M.\,V., Serebryakov~V.\,A.}
 Integrated information resource system of the Russian Academy of Sciences and a technology
for developing digital libraries~// Programming and Computer Software, 2000. Vol.~26. No.\,4.
P.~177--185.

\bibitem{9bos}
\Au{Рихтер Дж.}
Программирование на платформе Microsoft .NET Framework. Мастер класс~/
Пер. с англ. 3-е изд.~--- М.: Издательско-торговый дом <<Русская редакция>>;
СПб.: Питер, 2005.

\bibitem{10bos}
\Au{Howard~M., LeBlank~D.}
Writing secure code. 2nd ed. Microsoft Press, 2003.

\bibitem{11bos}
\Au{Босов А.\,В., Полухин~А.\,Н.}
О реализации сервиса аутентификации web-портала~// Информационные технологии
и вычислительные системы, %. М.: ИМВС РАН,
2005. №\,3. С.~50--60.

\bibitem{12bos}
\Au{Босов А.\,В., Чавтараев~Р.\,Б.}
Организация поиска в Информационном web-портале~// Системы и средства
информатики. Специальный выпуск <<Научно-ме\-то\-до\-ло\-ги\-че\-ские
проблемы информатики>>, %. М.: Наука,
2006. С.~438--460.

\bibitem{13bos}
Россия. Веб-портал РАН обеспечивает эффективный доступ научного сообщества к
актуальной информации~// Информационный бюллетень Microsoft, 2004. Вып.~26.
Ноябрь.

\bibitem{14bos}
Российская академия наук реализует систему управ\-ле\-ния содержанием своего
информационного веб-портала на основе платформы Microsoft .Net. Примеры внедрения.
{\sf http://www.microsoft.com/Rus/\linebreak Casestudies/CaseStudy.aspx?id=318}.
\end{thebibliography}

}
}

\end{multicols}

\label{end\stat}