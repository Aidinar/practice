\def\Bar{\hat}

\def\stat{pechin}

\def\tit{СТАЦИОНАРНЫЕ ХАРАКТЕРИСТИКИ
МНОГОЛИНЕЙНОЙ~СИСТЕМЫ МАССОВОГО ОБСЛУЖИВАНИЯ
С~ОДНОВРЕМЕННЫМИ ОТКАЗАМИ ПРИБОРОВ$^*$
}
\def\titkol{Стационарные характеристики многолинейной СМО
%системы массового обслуживания
с одновременными отказами приборов}

\def\aut{А.\,В.~Печинкин$^1$, И.\ А.~Соколов$^2$, В.\,В.~Чаплыгин$^3$}

\def\autkol{А.\,В.~Печинкин, И.\ А.~Соколов, В.\,В.~Чаплыгин}

\titel{\tit}{\aut}{\autkol}{\titkol}

{\renewcommand{\thefootnote}{\fnsymbol{footnote}}\footnotetext[1]
{Работа выполнена при поддержке РФФИ
(гранты №№~05-07-90103, 06-07-89056).}

\renewcommand{\thefootnote}{\arabic{footnote}}}
\footnotetext[1]{Институт проблем информатики Российской академии наук, apechinkin@ipiran.ru}
\footnotetext[2]{Институт проблем информатики Российской академии наук, isokolov@ipiran.ru}
\footnotetext[3]{Институт проблем информатики Российской академии наук, vchaplygin@ipiran.ru}

\index{Печинкин А.\,В.}
\index{Соколов И.\,А.}
\index{Чаплыгин В.\,В.}

\Abst{Рассматривается многолинейная система массового обслуживания
с полумарковским потоком заявок, обслуживанием фазового типа
и ненадежными приборами,
отказывающими и восстанавливающимися одновременно.
Получены математические соотношения для расчета основных
стационарных показателей функционирования системы при некоторых
вариантах процесса отказов--восстановлений приборов.}

\KW{система массового обслуживания; ненадежные приборы}

\label{st\stat}

\vskip 18pt plus 6pt minus 6pt

\thispagestyle{headings}

\begin{multicols}{2}

\section{Описание системы}

Статья посвящена анализу многолиненейных систем массового
обслуживания (СМО) с ненадежными приборами и
является логическим продол\-жением~[1, 2], в которых для некоторых
типов многолинейных СМО с независимыми отказами ненадежных приборов
найдены выражения для расчета показателей функционирования.
Опираясь на результаты, полученые в~[3] для системы $SM/MSP/n/r$,
далее будут выведены математические соотношения, позволяющие вычислять
основные стационарные характеристики многолинейных систем с полумарковским
входящим потоком заявок, обслуживанием фазового типа и различными вариантами
процесса одновременных отказов--восстановлений ненадежных приборов.

Рассмотрим многолинейную СМО $SM/PH/n/r$ с полумарковским входящим
потоком заявок, распределением фазового типа времени обслуживания
каждой заявки, накопителем конечной или бесконечной емкости и
ненадежными приборами.

Полумарковский входящий поток заявок определяется полумарковским процессом с 
конечным множеством состояний $\{1,2,\ldots,I\}$, $1\le I<\infty$, поведение 
которого описывается матрицей $A(x)=$\linebreak 
$=(A_{ij}(x))_{i,j=\overline{1,I}}$, где $A_{ij}(x)$~--- вероятность того, что 
полумарковский процесс за время меньше $x$ пе\-рейдет из состояния $i$ сразу в 
состояние $j$,  $i,j=\overline{1,I}$.

Введем обозначения:\\[-16pt]
\begin{description}
\item $A=A(\infty)$~--- матрица переходных ве\-роятностей вложенной цепи
Маркова полумарковского процесса;
\item $\vec\pi_a$~--- вектор-строка стационарных вероятностей вложенной
цепи Маркова;
\item $\vec1$~--- вектор-столбец из единиц,
\item $\vec0$~--- нулевой вектор-столбец,
\item $0$~--- нулевая матрица (наряду с обычным чис\-лом~0),
\item $E$~--- единичная матрица, размерность и порядок которых
определяются либо из контекста, либо нижним индексом,
например $\vec1_k$, $0_{m,l}$;
\item $\overline a =
\vec\pi_a \int\limits_0^\infty x\, dA(x)\, \vec1
< \infty
$~--- среднее время между поступлениями заявок в стационарном
режиме функционирования системы.
\end{description}

Более подробное описание полумарковского входящего потока,
а также некоторые естественные дополнительные предположения
относительно параметров, которые будут предполагаться
выполненными, приведены в~[1, 2].

Каждый из $n$ имеющихся в системе приборов может находиться
либо в исправном, либо в неисправном состоянии.
Состояние прибора будем считать исправным, если на приборе находится
заявка и прибор занят ее обслуживанием или если прибор свободен,
готов принять заявку и немедленно начать ее обслуживание.
Состояние прибора будем считать неисправным, если на приборе находится
заявка, но прибор ее не обслуживает, или если прибор свободен,
но не может немедленно начать обслуживание заявки, если таковая на него
поступит.
Если при отказе прибора (переходе прибора из исправного состояния в
неисправное) на нем находится заявка, то она остается на приборе до
момента восстановления (перехода прибора из неисправного состояния в
исправное) и затем, в зависимости от варианта функционирования
сис\-те\-мы, либо дообслуживается, либо обслуживается заново.

Будем называть прибор занятым, если на нем находится заявка, и
свободным в противном случае.
Если в некоторый момент времени на обслуживание поступает очередная
заявка, но все приборы заняты, то эта заявка попадает в накопитель,
становясь в очередь на обслуживание.
Заявки из очереди на обслуживание выбираются в порядке их поступления
в накопитель.

В случае накопителя конечной емкости $r$ заявка, поступающая в систему,
в которой уже имеется $n+r$ заявок ($n$ на приборах и $r$ в очереди)
теряется.

Распределение фазового типа (PH-рас\-пре\-де\-ле\-ние) времени обслуживания
заявки с числом фаз обслуживания $J$, $1\le J <\infty$, описывается матрицей 
$H=(h_{ij})_{i,j=\overline{1,J}}$ и вектор-строкой $\vec 
h=(h_i)_{i=\overline{1,J}}$ (см.\ также~[1, 2]). Здесь $h_{ij}$, 
$i,j=\overline{1,J}$,\ \ $j\ne i$,~--- интенсивность изменения фазы 
обслуживания с \mbox{$i$-й}~на $j$-ю (без окончания обслуживания заявки); 
$h_i$, $i=\overline{1,J}$,~--- вероятность того, что обслуживание заявки 
начинается с фазы $i$.

Положим также $\vec h^*= -H\vec1$.
Координата $h_i^*$ вектора $\vec h^*$ представляет собой
интенсивность окончания обслуживания заявки при фазе обслуживания $i$.
Напомним, что функцию распределения фазового типа времени обслуживания
заявки можно записать в виде
$$
H(x) = 1-\vec h\, e^{Hx} \vec1\,.
$$

{\bf Цель настоящей работы} заключается в нахождении программно
реализуемых математических соотношений для расчета основных
стационарных харак\-теристик описанной выше СМО $SM/PH/n/r$\ \
($r\le\infty$) с ненадежными приборами.
При этом, в отли\-чие от~[1, 2], будем считать, что отказы приборов
происходят одновременно.

Будут рассмотрены {\bf следующие варианты функционирования системы}
$SM/PH/n/r$ с ненадежными приборами:
\begin{itemize}
\item в разделе~3 заявки обслуживаются заново, свободные и занятые приборы
одновременно отказывают и восстанавливаются;
\item
в разделе~4 заявки дообслуживаются, свободные и занятые приборы
одновременно отказывают и восстанавливаются;
\item
в разделе~5 заявки обслуживаются заново, занятые приборы одновременно
отказывают и восстанавливаются, свободные приборы находятся только в
исправном состоянии;
\item
в разделе~6 заявки обслуживаются заново, отказывают и
восстанавливаются с разной интенсивностью одновременно свободные и
одновременно занятые приборы;
\item
в разделе~7 заявки дообслуживаются, занятые приборы одновременно
отказывают и восстанавливаются, свободные приборы находятся только
в исправном состоянии;
\item
в разделе~8 заявки дообслуживаются, отказывают и восстанавливаются
с разной ин\-тен\-сив\-ностью одновременно свободные и одновременно занятые
приборы.
\end{itemize}

Во всех вариантах заявки могут поступать на все приборы:
сначала они ищут свободный исправный прибор, затем, если такого нет,
но есть свободный неисправный прибор, становятся на него, и, наконец,
если все приборы заняты, становятся в очередь (или даже теряются
в случае переполнения системы с конечным накопителем).

\section{Базовая модель}    %2

Дальнейшие исследования в части вычисления стационарных
распределений вероятностей состояний будут опираться на
базовую модель СМО.
Базовая модель, а также формулы для вычисления ее основных
стационарных характеристик приведены для случая бесконечного
накопителя в~[1], а для случая конечного накопителя --- в~[2].

Ограничимся кратким описанием базовой модели.

Модель представляет собой многолинейную СМО с накопителем бесконечной
емкости, надежными приборами, полумарковским входящим потоком
(процессом генерации) заявок, описанным в предыдущем пункте, и марковским
процессом обслуживания заявок, определяемым следующим образом.

Поскольку основная сложность дальнейшего анализа СМО с
ненадежными приборами будет состоять в том, чтобы построить для
этой СМО марковский процесс обслуживания, приводящий к
эквивалентной (с точки зрения вероятностей состояний) базовой
модели, приведем подробное описание такого процесса.

Если в системе находится $k$, $k\ge 0$, заявок (далее
будем говорить также, что процесс обслуживания находится на
слое $k$), то процесс обслуживания может находиться в одном
из $l_k$, $l_k<\infty$, состояний (фаз обслуживания).
Далее, если в некоторый момент в системе находится $k$, $k \ge 1$,
заявок и фаза обслуживания равна $i$, $i=\overline{1,l_k}$, то
за <<малое>> вре\-мя~$\Delta$ с вероятностью
$\lambda^{(k)}_{ij}\Delta + o(\Delta)$ фаза изменится на~$j$,
$j=\overline{1,l_k}$, $j\ne i$, и все заявки будут продолжать
обслуживаться,
а с вероятностью $n^{(k)}_{ij}\Delta + o(\Delta)$ фаза изменится
на $j$, $j=\overline{1,l_{k-1}}$, но обслуживание одной из
заявок закончится и она покинет систему. Матрицы из элементов
$\lambda^{(k)}_{ij}$ и $n^{(k)}_{ij}$ будем обозначать через
$\Lambda_k$ и $N_k$, $k \ge 1$.
Если же в системе отсутствуют заявки, то за <<малое>>\ время $\Delta$
с вероятностью $\lambda^{(0)}_{ij}\Delta + o(\Delta)$ фаза изменится
с $i$ на $j$, $i,j=\overline{1,l_0}$, $j\ne i$, естественно,
без окончания обслуживания заявки.
Матрицу из элементов $\lambda^{(0)}_{ij}$ будем обозначать через~$\Lambda_0$.

Кроме того, будем предполагать, что $l_k=l$ при
$k \ge n$, матрицы $\Lambda_k = \Lambda$ совпадают при
$k \ge n$, а матрицы $N_k = N$ совпадают при
$k \ge n+1$.

Введем также матрицу $\Lambda^* = \Lambda+N$, причем матрица
$\Lambda^*$ предполагается неразложимой, а мат\-ри\-ца $N$~---
ненулевой. Более того, будем предполагать, что при исходных
параметрах рассматриваемой СМО введенная далее вложенная цепь
Маркова будет неприводимой.

Наконец, при $k=\overline{0,n-1}$ будем предполагать, что если в
момент поступления очередной заявки в системе имеется $k$ других
заявок и фаза обслуживания равна $i$, $i=\overline{1,l_k}$, то
после поступления новой заявки фаза обслуживания с вероятностью
$\omega^{(k)}_{ij}$ изменится на $j$, $j=\overline{1,l_{k+1}}$.
Соответственно, матрицу из элементов $\omega^{(k)}_{ij}$ будем
обозначать через $\Omega_k$.

Введем стационарную интенсивность обслуживания $\mu$ при
бесконечной очереди следующим образом:
$$
\label{mu_general}
\mu = \vec \pi_s N \vec1\,,
$$
где через $\vec\pi_s$ обозначена вектор-строка стационарных
вероятностей марковского процесса  с инфинитезимальной матрицей
$\Lambda^*$.
В случае бесконечного накопителя будем предполагать, что $\rho<1$,
а число~$\rho$, определяемое выражением
$\rho = (\overline a \mu)^{-1}$, назовем нагрузкой на систему.

В~[1, 2] для базовой модели СМО через исходные параметры находятся
(для случаев бесконечного и конечного накопителей) инфинитезимальная
мат\-ри\-ца и матрица из вероятностей переходов за время $t$ марковского
процесса обслуживания заявок.
Затем вычисляется матрица
переходных вероятностей вложенной цепи Маркова, порожденной моментами
поступления заявок в систему.
Наконец, определяются вектор-строки, координатами
которых являются стационарные вероятности по вложенной цепи Маркова и
по времени того, что в системе находятся $k$ заявок, а фазы процессов
генерации и обслуживания заявок равны $i$ и $j$.
Кроме того, там же вычисляются стационарное распределение $W(x)$
времени ожидания начала обслуживания заявки (в случае конечного
накопителя предполагается, что заявка, не принятая в систему из-за
переполнения, имеет нулевое время ожидания начала обслуживания) и
стационарная вероятность $\pi$ потери заявки.

В настоящей работе, так же, как и в~[1, 2], вмес\-то линейной
нумерации состояний процесса обслуживания будем использовать
мультииндексную нумерацию, при которой номер состояния определяется
мультииндексом $(i_1,\ldots,i_r)$ или объединением таких
мультииндексов, где каждому числу $k$ заявок в системе соответствует
одно или несколько значений $r$, а $i_1,\ldots,i_r=\overline{1,J}$
(напомним, что $J$~--- число фаз PH-распределения времени
обслуживания заявки), причем в случае $k\ge n$ подмножества
мультииндексов совпадают.
Перейти от мультииндексной нумерации состояний процесса обслуживания
к линейной в системе можно различными способами.
В частности, рациональный способ такого перехода (на примере СМО
$SM/PH/n/\infty$ с надежными приборами) предложен в~[3].

\section{Одновременные отказы всех приборов
и~обслуживание заявок заново}                                %3

Наиболее просто к базовой СМО приводится система $SM/PH/n/r$
с ненадежными приборами, функционирование процесса
отказов--вос\-ста\-нов\-лений которой описывается следующим
образом:
\begin{itemize}
\item все (занятые и свободные) приборы отказывают одновременно
с интенсивностью $\alpha$ и также одновременно
восстанавливаются с интенсивностью $\beta$;
    \item поступающая в систему заявка, заставшая все приборы в неисправном
состоянии, становится на свободный неисправный прибор, если
таковой имеется, в противном случае попадает в накопитель
(или вообще теряется, если в системе конечный накопитель
и все приборы и места ожидания заняты);
\item заявка, обслуживание которой было прервано из-за отказа прибора,
после восстановления прибора обслуживается заново.
\end{itemize}

Определим множество состояний на слое $k$:
\begin{align*}
{\cal X}_0 &= \{(1), (0)\}\,;\\
{\cal X}_k &= \{(i_1,\ldots,i_k), (0)\}\,,\quad k=\overline{1,n-1}\,;\\
{\cal X}_k &= \{(i_1,\ldots,i_n), (0)\}\,,\quad \ \ k \ge n\,.
\end{align*}
Здесь состояние~$(1)$ означает (для слоя 0~--- заявки в системе
отсутствуют), что все приборы исправны, состояние $(0)$
(для всех слоев)~--- все приборы неисправны, значение $l$-й
координаты $i_l$, $l=\overline{1,k}$ или $l=\overline{1,n}$,
мультииндекса $(i_1,\ldots,i_k)$ или $(i_1,\ldots,i_n)$ показывает
фазу обслуживания заявки на $l$-м приборе (далее будем считать,
что нумерация приборов определяется моментами поступления находящихся
на них заявок).

Теперь для того чтобы найти стационарные вероятности состояний,
достаточно воспользоваться базовой моделью, в которой для марковского
процесса обслуживания нужно положить
\begin{align*}
\Omega_k &=
\begin{pmatrix}
E_{J^k} \otimes \vec h   &  \vec0_{J^k}    \\
\vec0_{J^{k+1}}^T        &  1              \\
\end{pmatrix}\,, \quad  k=\overline{0,n-1}\,;\\
N_k&=
\begin{pmatrix}
\tilde N_k           &  \vec0_{J^k}  \\
\vec0_{J^{k-1}}^T    &  0            \\
\end{pmatrix}\,,
\quad k=\overline{1,n}\,;\\
N&=
\begin{pmatrix}
\tilde N     &  \vec0_{J^n}  \\
\vec0_{J^n}        &  0      \\
\end{pmatrix}\,;\\
\Lambda_k &=
\begin{pmatrix}
\tilde \Lambda_k - \alpha E_{J^k}   &    \alpha \vec1_{J^k}  \\
\beta \vec \omega_k                 &    - \beta             \\
\end{pmatrix}\,,
\quad k=\overline{0,n-1}\,;\\
\Lambda &=
\begin{pmatrix}
\tilde \Lambda - \alpha E_{J^n}   &    \alpha \vec1_{J^n}  \\
\beta \vec \omega                 &    - \beta             \\
\end{pmatrix}\,,
\end{align*}
%%%%%%%%%%%%%%%%%%%%%%%%%%%%%%%
где
\begin{align}
\tilde N_k  & = \vec h^* \otimes E_J \otimes \ldots \otimes E_J +
E_J \otimes \vec h^* \otimes \ldots \otimes E_J +{}\notag\\
&\ \ \ \ {}+ \ldots + E_J \otimes E_J \otimes \ldots \otimes \vec h^*\,,
\quad k=\overline{1,n}\,;\label{3.1} \\       %(3.1)
%%%%%%%%%%%%%%%%%%%%%%
\tilde N & = (\vec h^* \otimes E_J \otimes \ldots \otimes E_J +
E_J \otimes \vec h^* \otimes \ldots \otimes E_J +{}\notag\\
&\ \ \ \ \ \ \ \ \ \ \ \ \ \ \ {}+ \ldots + E_J \otimes E_J \otimes \ldots \otimes \vec h^* )
 \otimes \vec h\,; \label{3.2}  \\                               %(3.2)
%%%%%%%%%%%%%%%%%%%%%%
\tilde \Lambda_k & = H \otimes E_J \otimes \ldots \otimes E_J +
E_J \otimes H \otimes \ldots \otimes E_J +{}\notag\\
&{}+ \ldots + E_J \otimes E_J \otimes \ldots \otimes H\,,
\quad k=\overline{0,n-1}\,;  \label{3.3}          \\            %(3.3)
%%%%%%%%%%%%%%%%%%%%%%%%
\tilde \Lambda & = H \otimes E_J \otimes \ldots \otimes E_J +
E_J \otimes H \otimes \ldots \otimes E_J +{}\notag\\
&\ \ \ \ \ \ \ \ \ \ \ \ \ \ \ \ \ \ \ \ \ \ \ \ \ \ {}+ \ldots + E_J \otimes E_J \otimes \ldots
\otimes H\,; \label{3.4} \\                                   %(3.4)
%%%%%%%%%%%%%%%%%%%%%%%%
\vec \omega_k & = \vec h \otimes \vec h \otimes \ldots \otimes \vec h\,,
\quad k=\overline{0,n-1}\,; \label{3.5} \\                      %(3.5)
%%%%%%%%%%%%%%%%%%%%%%%
\vec \omega & = \vec h \otimes \vec h \otimes \ldots
\otimes \vec h\,, \label{3.6}                               %(3.6)
\end{align}
%%%%%%%%%%%%%%%%%%%%%
причем в формулах~(\ref{3.1}) и~(\ref{3.3}) $k$~слагаемых,
в формулах~(\ref{3.2}) и~(\ref{3.4}) $n$~слагаемых,
в формуле~(\ref{3.5}) $k$~сомножителей и в формуле~(\ref{3.6})
$n$~сомножителей.

Подставляя найденные значения матриц в базовую модель и
производя расчеты, находим стационарные вероятности по
вложенной цепи Маркова и по времени того, что в системе
находятся $k$ заявок, а фазы процессов генерации и
обслуживания заявок равны $i$ и $j$, и стационарное
распределение времени ожидания начала обслуживания заявки.

Наконец, обратимся к вычислению стационарного распределения
времени пребывания заявки в системе.
Для этого введем новые PH-рас\-пре\-де\-ле\-ния $G_1(x)$ и $G_2(x)$
с $J+1$ фазами обслуживания, одной и той же матрицей
$$
G=
\begin{pmatrix}
H-\alpha E       &  \alpha \vec1  \\
\beta \vec h     &  - \beta       \\
\end{pmatrix}
$$
и векторами $ \vec g_1 = (0,\ldots, 0,1)$ и
$ \vec g_2 = (h_1,\ldots, h_J,0)$, соответственно.
Кроме того, обозначим через $\hat p_0$ сумму всех координат
всех векторов $\vec p^{\,*}_{k}$, $k=\overline{1,n}$, соответствующих
состояниям, при которых приборы находятся в неисправном состоянии,
а через~$\hat p_1$~--- сумму всех остальных координат этих векторов.
{\looseness=1

}

Заметим теперь, что если в момент поступления заявки в систему
все приборы заняты, то эта заявка сначала ждет момента
поступления на прибор, а затем сразу же начинает обслуживаться
(попадая на одну из первых $J$ фаз своего обслуживания).
Если же в момент поступления заявки в систему имеется хотя бы
один свободный прибор, то заявка занимает один из них и
либо сразу же начинает обслуживаться на одной из $J$ фаз
(приборы находятся в исправном состоянии),
либо (приборы неисправны) ожидает начала обслуживания на этом
приборе (будем считать в этом случае, что заявка находится на
$(J+1)$-й фазе).


С учетом сделанных замечаний стационарное распределение времени
пребывания заявки в сис\-те\-ме можно записать в виде
%%%%%%%%%%%%%%%%%%%%%%
\begin{multline}
V(x) =
\hat{p}_0 G_1 (x)
+
\hat{p}_1 G_2 (x) +{}\\
{}+ \int\limits_{0+}^{x}
G_2(x-y)\,dW(y)\,,
\quad x>0\,,
\label{V_x_1}
\end{multline}
в случае бесконечного накопителя и
%%%%%%%%%%%%%%%%%%%%%%%%%%%
\begin{multline}
\label{Vx2}
V(x) = \fr{1}{1-\pi} \Bigg( \hat{p}_0 G_1 (x) +
\hat{p}_1 G_2 (x) +{}\\
{}+ \int\limits_{0+}^{x} G_2(x-y)\, dW(y) \Bigg)\,,
\quad x>0\,,
\end{multline}
в случае конечного накопителя,
где $W(x)$~--- функция стационарного распределения времени ожидания
начала обслуживания заявки, а $\pi$~--- стационарная вероятность
потери заявки.
Значение $0{+}$ нижнего предела интегралов означает, что скачок
$W(0{+})-W(0)$ функции $W(x)$ в интеграле не учитывается.

\section{Одновременные отказы всех приборов
и~дообслуживание заявок}    %  4

Следующая система, которая будет здесь рассмотрена, отличается от
разобранной в предыдущем разделе только тем, что
заявка, обслуживание которой было прервано из-за отказа прибора,
после восстановления прибора дообслуживается.
Собственно говоря, и ее исследование практически ничем не
отличается от исследования предыдущей СМО.

Для приведения рассматриваемой СМО к базовой модели множество
состояний на слое~$k$ зададим следующим образом:
\begin{align*}
{\cal X}_0 &= \{(j)\}\,;\\
{\cal X}_k  &= \{(i_1,\ldots,i_k;j)\}\,,\
\ k=\overline{1,n-1}\,;\\
{\cal X}_k &= \{(i_1,\ldots,i_n;j)\}\,,
\ \ k \ge n\,,
\end{align*}
где $i_l$~--- значение $l$-й координаты
$i_l$, $l=\overline{1,k}$ или $l=\overline{1,n}$,
мультииндекса $(i_1,\ldots,i_k)$ или $(i_1,\ldots,i_n)$ показывает
фазу обслуживания заявки на $l$-м приборе,
а $j=1$, если приборы исправны, и $j=0$, если приборы неисправны.

Определим матрицы марковского процесса обслуживания
базовой модели:
\begin{align*}
\Omega_k &=
\begin{pmatrix}
E_{J^k} \otimes \vec h   &  0_{J^k,J^{k+1}}         \\
0_{J^k,J^{k+1}}          &  E_{J^k} \otimes \vec h  \\
\end{pmatrix}\,,\\
N_k &=
\begin{pmatrix}
\tilde N_k       &  0_{J^k,J^{k-1}}  \\
0_{J^k,J^{k-1}}  &  0_{J^k,J^{k-1}}  \\
\end{pmatrix}\,,
\quad k=\overline{1,n}\,;\\
N &=
\begin{pmatrix}
\tilde N         &  0_{J^n,J^n}  \\
0_{n^J,n^J}      &  0_{J^n,J^n}  \\
\end{pmatrix}\,;\\
\Lambda_k &=
\begin{pmatrix}
\tilde \Lambda_k - \alpha E_{J^k}    &    \alpha E_{J^k}   \\
\beta E_{J^k}                        &    - \beta E_{J^k}  \\
\end{pmatrix}\,,
\quad k=\overline{0,n-1}\,;\\
\Lambda &=
\begin{pmatrix}
\tilde \Lambda - \alpha E_{J^n}   &    \alpha E_{J^n}  \\
\beta E_{J^n}                     &    - \beta E_{J^n} \\
\end{pmatrix}\,,
\end{align*}
где матрицы $\tilde N_k$, $\tilde N$, $\tilde \Lambda_k$ и
$\tilde \Lambda$ определены в предыдущем разделе
формулами~(\ref{3.1})--(\ref{3.6}).

Теперь, как и в предыдущем разделе, из базовой модели
находятся стационарные вероятности по вложенной цепи Маркова
и по времени того, что в системе имеется $k$ заявок, а фазы
процессов генерации и обслуживания заявок равны $i$ и $j$, и
стационарное распределение времени ожидания начала обслуживания
заявки.

Стационарное распределение времени пребывания заявки в системе
вы\-чис\-ля\-ет\-ся по формулам~(\ref{V_x_1}) и~(\ref{Vx2}),
где вероятности~$\hat p_0$ и $\hat p_1$ имеют прежний смысл,
а PH-распределения $G_1(x)$ и $G_2(x)$ с $2J$ фазами
определяются матрицей
$$
G=
\begin{pmatrix}
H-\alpha E       &  \alpha E       \\
\beta E          &  - \beta E      \\
\end{pmatrix}
$$
и векторами $\vec g_1 = (0,\ldots,0,h_1,\ldots, h_J)$ и
$\vec g_2 = $\linebreak $=(h_1,\ldots, h_J,0,\ldots,0)$.

\section{Одновременные отказы занятых приборов и~обслуживание заявок
заново, свободные приборы находятся только в исправном состоянии}         % 5

Особенностями функционирования модели, рассмотренной в этом
разделе, являются:
\begin{itemize}
\item занятые приборы отказывают одновременно с интенсивностью $\alpha$
и одновременно восстанавливаются с интенсивностью $\beta$,
свободные приборы находятся только в исправном состоянии;%\\[-15pt]
\item
заявка, поступающая в систему, начинает обслуживаться на
свободном приборе, если таковой имеется, в противном случае
попадает в накопитель (для системы с конечным накопителем
может также потеряться);%\\[-15pt]
\item
если занятые приборы восстанавливаются и до момента окончания их
восстановления в систему поступают еще заявки, которые начинают
обслуживаться на свободных приборах, то приборы с этими
вновь поступившими заявками также могут отказать с интенсивностью
$\alpha$, присоединившись к ранее отказавшим приборам.
При этом восстановление происходит одновременно для всех приборов
с интенсивностью~$\beta$ вне зависимости от того, в какой именно
момент они отказали;%\\[-15pt]
\item
процесс восстановления неисправных приборов не влияет на
процесс обслуживания исправными приборами;%\\[-15pt]
\item
заявки, обслуживание которых было прервано отказами приборов,
после восстановления приборов обслуживаются заново.
\end{itemize}

Заметим, что рассматриваемая в этом разделе модель с отказами
функционирует таким образом, что, помимо нескольких исправных
занятых приборов, могут восстанавливаться еще несколько приборов
с заявками. Такая ситуация принципиально невозможна для моделей,
рассмотренных выше.

Назовем слоем $k$, $k\ge 0$, множество всех состояний процесса
обслуживания, в которых общее число заявок в системе равно $k$,
причем состояние слоя $k$, при котором нет исправных приборов с
заявками, будем обозначать через $(0)$.

Слой $k$ при $k=\overline{0,n-1}$ представляет собой множество

\noindent
\begin{equation*}
{\cal X}_k
=
%\bigcup\limits_{m=0}^{k}
\{(i_1,\ldots,i_m),\ \ m=\overline{0,k}\}\,.
\end{equation*}
Состояние $(i_1,\ldots,i_m)$ означает, что первый прибор обслуживает
заявку на фазе $i_1$, $\ldots$, $m$-й прибор~--- %\linebreak
%\vspace*{-12pt} 
%\pagebreak
%
%\noindent
на фазе $i_m$, а
$(k-m)$ приборов с заявками восстанавливаются.

Слой $k$ при $k \ge n$ имеет вид
\begin{equation*}
{\cal X}_k
=
%\bigcup\limits_{m=0}^{n}
\{(i_1,\ldots,i_m),\ \ m=\overline{0,n}\}\,.
\end{equation*}
Состояние $(i_1,\ldots,i_m)$ означает, что первый прибор обслуживает
заявку на фазе $i_1$, $\ldots,$ $m$-й прибор~--- на фазе $i_m$, остальные
$(n-m)$ приборов с заявками восстанавливаются
и еще $(k-n)$ заявок находятся в накопителе.

Далее условимся обозначать через
$(i_1,\ldots,i^{(l)},\ldots,i_{m-1})$, $l=\overline{1,m}$,
состояние, при котором занято обслуживанием $m$ приборов (число
свободных приборов и число заявок в очереди определяются слоем),
причем
первый прибор находится на фазе $i_1$,
второй~--- на фазе $i_2$,
$\ldots,$
$(l-1)$-й~--- на фазе $i_{l-1}$,
$l$-й~--- на фазе $i$,
$(l+1)$-й~--- на фазе $i_{l}$,
$\ldots,$
$m$-й~--- на фазе $i_{m-1}$.

Предполагая, что новые заявки в систему не поступают, перечислим
все возможные переходы между состояниями при обслуживании
заявок с указанием интенсивностей переходов.

Начнем с переходов из-за изменения фазы обслуживания заявки:
\begin{itemize}
\item
из состояния
$(i_1,\ldots,i^{(l)},\ldots,i_{m-1})$,
$m=\overline{1,k}$,
слоя $k$, $k=\overline{1,n-1}$, возможен переход в состояние
$(i_1,\ldots,j^{(l)},\ldots,i_{m-1})$,
$j\ne i$,
того же слоя с интенсивностью $h_{ij}$ при изменении фазы
обслуживания заявки на $l$-м приборе с $i$-й на~$j$-ю;
\item
из состояния
$(i_1,\ldots,i^{(l)},\ldots,i_{m-1})$,
$m=\overline{1,n}$,
слоя $k$, $k\ge n$, возможен переход в состояние
$(i_1,\ldots,j^{(l)},\ldots,i_{m-1})$,
$j\ne i$,
того же слоя с интенсивностью $h_{ij}$ при изменении фазы
обслуживания заявки на $l$-м приборе с $i$-й на $j$-ю.
\end{itemize}

Следующий тип переходов~--- при отказах исправных приборов с
заявками:
\begin{itemize}
\item
из состояния
$(i_1,\ldots,i_{m})$,
$m=\overline{1,k}$,
слоя $k$, $k=\overline{1,n-1}$, возможен переход в состояние
$(0)$ того же слоя с интенсивностью $\alpha$ при отказе
исправных занятых приборов;
\item
из состояния
$(i_1,\ldots,i_{m})$,
$m=\overline{1,n}$,
слоя $k$, $k\ge n$, возможен переход в состояние $(0)$
того же слоя с интенсивностью $\alpha$ при отказе исправных
занятых приборов.
\end{itemize}

Аналогично определяются переходы при восстановлении приборов:
\begin{itemize}
\item
из состояния
$(i_1,\ldots,i_{m})$,
$m=\overline{0,k-1}$,
слоя~$k$, $k=\overline{1,n-1}$, возможен переход в состояние
$(i_1,\ldots,i_{m},j_{1},\ldots,j_{k-m})$
того же слоя с интенсивностью $\beta h_{j_1}\cdots h_{j_{k-m}}$
при восстановлении приборов;
\item
 из состояния
$(i_1,\ldots,i_{m})$,
$m=\overline{0,n-1}$,
слоя~$k$, $k\ge n$, возможен переход в состояние
$(i_1,\ldots,i_{m},j_{1},\ldots,j_{n-m})$
того же слоя с интенсивностью $\beta h_{j_1}\cdots h_{j_{n-m}}$
при восстановлении приборов.
\end{itemize}

Следующие переходы происходят при окончании обслуживания заявки:
\begin{itemize}
\item из состояния
$(i_1,\ldots,i^{(l)},\ldots,i_{m-1})$,
$m=\overline{1,k}$,
$l=\overline{1,m}$,
слоя $k$, $k=\overline{1,n}$, возможен переход в состояние
$(i_1,\ldots,i_{m-1})$ слоя $(k-1)$ с интенсивностью $h^*_i$
при окончании обслуживания заявки на $l$-м приборе при
$i$-й фазе;
\item
из состояния
$(i_1,\ldots,i^{(l)},\ldots,i_{m-1})$,
$m=\overline{1,n}$,
$l=\overline{1,m}$,
слоя $k$, $k>n$, возможен переход в состояние
$(i_1,\ldots,i_{m-1},j)$ слоя $(k-1)$ с
интенсивностью $h^*_i h_j$ при окончании обслуживания заявки
на $l$-м приборе при $i$-й фазе и поступлении на него новой
заявки из очереди на $j$-ю фазу.
\end{itemize}

Осталось определить вероятности изменения состояний процесса
обслуживания в момент поступления в систему заявки, т.\,е.\
элементы мат\-риц~$\Omega_l$,\ \ $l=\overline{0,n-1}$:
\begin{itemize}
\item из состояния
$(i_1,\ldots,i_{m})$, $m=\overline{0,k}$,
слоя $k$, $k=$\linebreak $=\overline{0,n-1}$,
 процесс обслуживания при поступлении заявки переходит в состояние
$(i_1,\ldots,i_{m},i)$ слоя $k+1$
с вероятностью $h_i$ того, что поступающая в систему заявка начнет
обслуживаться с фазы $i$.
\end{itemize}

Теперь, сформировав матрицы $\Lambda_k$, $N_k$ и $\Omega_k$
базовой модели и воспользовавшись формулами для ее расчета,
можно найти стационарные вероятности по вложенной цепи Маркова
и по времени того, что в системе имеется $k$ заявок, а фазы
процессов генерации и обслуживания заявок равны $i$ и $j$, и
стационарное распределение времени ожидания начала обслуживания
заявки.

Стационарное распределение $V(x)$ времени пребывания заявки
в системе определяется соотношением
\begin{equation}
\label{V_x_2}
V(x) =
\int\limits_0^x
W(x-y)\, dG(y)
\end{equation}
в случае бесконечного накопителя и
%%%%%%%%%%%%%%%%%%%%%%%%%%%
\begin{multline}
\label{Vx3}
V(x) =
\fr{1}{1-\pi}
\Bigg(
(\hat p_0 + \hat p_1) G(x)+{}\\
{}+
\int\limits_{0+}^{x}
G(x-y)\, dW(y)
\Bigg)\,,
\quad x>0\,,
\end{multline}
в случае конечного накопителя,
где
вероятности $\hat p_0$ и $\hat p_1$ те же самые, что и в
разделе~3,
а $G(x)$~--- функция распределения фазового типа с матрицей
$$
G=
\begin{pmatrix}
H-\alpha E       &  \alpha \vec1  \\
\beta \vec h     &  - \beta        \\
\end{pmatrix}
$$
и вектором
$\vec g = (h_1,\ldots, h_J,0)$.

\section{Независимые отказы свободных и занятых приборов
и~обслуживание заявок заново, заявки поступают на~все приборы}       % 6

Следующая СМО имеет такие особенности:
\begin{itemize}
\item все занятые приборы отказывают одновременно с интенсивностью $\alpha$
и одновременно восстанавливаются с интенсивностью $\beta$;
\item
все свободные приборы отказывают одновременно с интенсивностью
$\alpha^*$ и одновременно восстанавливаются с интенсивностью $\beta^*$;
\item
 одновременные отказы и восстановления занятых приборов и
одновременные отказы и восстановления свободных приборов
происходят независимо друг от друга;
\item
если в системе имеются исправные свободные приборы, то поступающая
заявка становится на один из таких приборов.
Если есть только неисправные свободные приборы, то заявка становится
на неисправный свободный прибор, причем такой прибор с заявкой
восстанавливается уже одновременно со всеми неисправными занятыми
приборами с интенсивностью $\beta$.
Наконец, если вообще нет
свободных приборов, заявка попадает в накопитель (или даже теряется);
\item
заявки, обслуживание которых было прервано отказами приборов,
после восстановления приборов обслуживаются заново.
\end{itemize}

Определим слои, на которых может находиться процесс обслуживания,
следующим образом.

Слой $k$ при $k=\overline{0,n-1}$ представляет собой множество
\begin{equation*}
{\cal X}_k
=
\{(i_1,\ldots,i_m;j)\,,\ \ m=\overline{0,k}\,,\ \   j=\overline{0,n-k}\}\,.
\end{equation*}
Состояние $(i_1,\ldots,i_m;j)$ означает, что $m$ приборов обслуживают
заявки на фазах $i_1,\ldots, i_m$, $(k-m)$ приборов с заявками
восстанавливаются, $j$ свободных приборов исправны и $(n-k-j)$ свободных
приборов восстанавливаются.
Состояние слоя $k$, при котором $k$ приборов с заявками
восстанавливаются и $j$ свободных приборов исправны, будем обозначать
через $(0;j)$.

Слой $k$ при $k \ge n$ имеет вид
\begin{equation*}
{\cal X}_k
=
\{(i_1,\ldots,i_m)\,,\ \ m=\overline{0,n}\}\,.
\end{equation*}
Состояние $(i_1,\ldots,i_m)$ означает, что $m$ приборов исправны и
обслуживают заявки на фазах $i_1,\ldots,i_m$, $(n-m)$ приборов с
заявками восстанавливаются и еще $(k-n)$ заявок находятся в накопителе.
Состояние слоя $k$, при котором $n$ приборов с заявками
восстанавливаются, будем обозначать через $(0)$.

Далее условимся обозначать через
$(i_1,\ldots,i^{(l)},\ldots,i_{m-1})$, $l=\overline{1,m}$,
состояние слоя~$k$, $k \ge n$, при котором  все приборы исправны,
причем первый прибор находится на фазе $i_1$,
второй~--- на фазе $i_2$,
$\ldots,$
$(l-1)$-й~--- на фазе $i_{l-1}$,
$l$-й~--- на фазе $i$,
$(l+1)$-й~--- на фазе $i_{l}$,
$\ldots,$
$m$-й~--- на фазе $i_{m-1}$.
Аналогичным образом определяется состояние
$(i_1,\ldots,i^{(l)},\ldots,i_{m-1};j)$
слоя $k$, $k=\overline{0,n-1}$.

Перечислим все возможные переходы между состояниями при обслуживании
заявок (предполагается, что новые заявки в систему не поступают)
с указанием интенсивностей переходов.

Начнем с переходов из-за изменения фазы обслуживания заявки:
\begin{itemize}
\item из состояния
$(i_1,\ldots,i^{(l)},\ldots,i_{m-1};d)$,\
$m=\overline{0,k}$,\  $d=\overline{0,n-k}$,
слоя $k$, $k=\overline{1,n-1}$, возможен переход в состояние
$(i_1,\ldots,j^{(l)},\ldots,i_{m-1};d)$, $j\ne i$,
того же слоя с интенсивностью $h_{ij}$ при изменении фазы
обслуживания заявки на $l$-м приборе с $i$-й на $j$-ю;
\item
 из состояния
$(i_1,\ldots,i^{(l)},\ldots,i_{m-1})$,\ $m=\overline{0,n}$,
слоя $k$,\ $k\ge n$, возможен переход в состояние
$(i_1,\ldots,j^{(l)},\ldots,i_{m-1})$,\ \ $j\ne i$,
того же слоя с интенсивностью $h_{ij}$ при изменении фазы
обслуживания заявки на $l$-м приборе с $i$-й на $j$-ю.
\end{itemize}

Следующий тип переходов связан с отказами приборов с заявками:
\begin{itemize}
\item из состояния
$(i_1,\ldots,i_{m};d)$,\  $m=\overline{1,k}$,\  $d=$\linebreak $=\overline{0,n-k}$,
слоя $k$,\  $k=\overline{1,n-1}$, возможен переход в состояние
$(0;d)$
того же слоя с интен\-сив\-ностью $\alpha$ при отказе всех приборов;
\item
из состояния
$(i_1,\ldots,i_{m})$,\  $m=\overline{1,n}$,
слоя $k$,\  $k\ge n$, возможен переход в состояние
$(0)$
того же слоя с интенсивностью $\alpha$ при отказе всех приборов.
\end{itemize}

Аналогично определяются переходы из-за отказов приборов без заявок:
\begin{itemize}
\item из состояния
$(i_1,\ldots,i_{m};d)$,\  $m=\overline{1,k}$,\  $d=$\linebreak $=\overline{1,n-k}$,
слоя $k$,\  $k=\overline{1,n-1}$, возможен переход в состояние
$ (i_1,\ldots,i_{m};0)$
того же слоя с интенсивностью $\alpha^*$ при отказе всех приборов.
\end{itemize}

Переходы при восстановлении приборов с заявками определяются
следующим образом:
\begin{itemize}
\item
из состояния
$(i_1,\ldots,i_{m};d)$,\  $m=\overline{1,k}$,\  $d=$\linebreak $=\overline{0,n-k}$,
слоя $k$,\ $k=\overline{1,n-1}$, возможен переход в состояние
$(i_1,\ldots,i_{k};d)$
того же слоя с интенсивностью $\beta h_{i_{m+1}}\cdots h_{i_k}$
при вос\-ста\-нов\-ле\-нии всех приборов;
\item
из состояния
$(i_1,\ldots,i_{m})$,\  $m=\overline{1,n}$,
слоя $k$,\  $k\ge n$, возможен переход в состояние
$(i_1,\ldots,i_{n})$
того же слоя с интенсивностью $\beta h_{i_{m+1}}\cdots h_{i_n}$
при восстановлении всех приборов.
\end{itemize}

Переходы при восстановлении приборов без заявок определяются так:
\begin{itemize}
\item
из состояния
$(i_1,\ldots,i_{m};d)$, \ $m=\overline{0,k}$,\  $d=$\linebreak $=\overline{0,n-k}$,
слоя $k$,\ $k=\overline{1,n-1}$, возможен переход в состояние
$(i_1,\ldots,i_{m};n-k)$
того же слоя с интенсивностью $\beta^*$ при восстановлении всех
приборов.
\end{itemize}

Следующие переходы происходят при окончании обслуживания заявок:
\begin{itemize}
\item
из состояния
$(i_1,\ldots,i^{(l)},\ldots,i_{m-1};d)$,\
$m=\overline{1,k}$, \  $d=\overline{0,n-k}$,
слоя $k$,\ $k=\overline{1,n}$, возможен переход в состояние
$(i_1,\ldots,i_{m-1};d+1)$
слоя $(k-1)$ с интенсивностью $h^*_i $ при окончании обслуживания
заявки на $l$-м приборе при $i$-й фазе;
\item
из состояния
$(i_1,\ldots,i^{(l)},\ldots,i_{m-1})$, \ $m=\overline{1,n}$,
слоя $k$,\ $k>n$, возможен переход в состояние
$(i_1,\ldots,i_{m-1},j)$
слоя $(k-1)$ с интенсивностью $h^*_i h_j$ при окончании обслуживания
заявки на $l$-м приборе при $i$-й фазе и поступлении на него
новой заявки из очереди на $j$-ю фазу.
\end{itemize}

Определим ненулевые элементы матриц
$\Omega_l$,\  $l=\overline{0,n-1}$:
\begin{itemize}
\item
из состояния
$(i_1,\ldots,i_{m};d)$,\  $m=\overline{1,k}$,  \ $d=$\linebreak $=\overline{1,n-k}$,
слоя $k$,\ $k=\overline{0,n-1}$,  при поступлении заявки
процесс обслуживания заявок переходит в состояние
$(i_1,\ldots,i_{m},i;d-1)$, $m=\overline{1,n}$, слоя $k+1$
с вероятностью $h_i$ того, что поступающая в систему заявка
начнет обслуживаться с фазы~$i$.
\end{itemize}

Найдя таким образом матрицы
$\Lambda_k$,\  $k=\overline{0,n-1}$,\  $\Lambda$,\
$N_k$,\  $k=\overline{1,n}$, \ $N$ и
$\Omega_k$,\  $k=\overline{0,n-1}$, базовой модели, можно
воспользоваться результатами для базовой системы для вычисления
стационарных вероятностей по вложенной цепи Маркова
и по времени того, что в системе имеется $k$ заявок, а фазы
процессов генерации и обслуживания заявок равны $i$ и $j$, и
стационарного распределения времени ожидания начала
обслуживания заявки.

Стационарное распределение времени пребывания заявки в системе
можно вычислить по формулам~(\ref{V_x_1}) и (\ref{Vx2}), где
вероятности $\hat p_1$ и $\hat p_2$ и распределения фазового
типа $G_1(x)$ и $G_2(x)$ находятся так же, как и в разделе~3.

\section{Одновременные отказы занятых приборов и~дообслуживание
заявок, свободные приборы находятся только в~исправном состоянии}         %  7

Рассматриваемый в этом разделе вариант системы характеризуется
следующими особенностями:
\begin{itemize}
\item
занятые приборы отказывают одновременно с интенсивностью $\alpha$
и затем одновременно восстанавливаются с интенсивностью $\beta$,
свободные приборы находятся только в исправном состоянии;
\item
заявка, поступающая в систему, начинает обслуживаться на свободном
приборе, если таковой имеется, или же попадает в накопитель (или
теряется);
\item
если занятые приборы восстанавливаются и до момента окончания
восстановления в систему поступают заявки, которые начинают
обслуживаться на свободных приборах, то приборы с этими поступившими
заявками также могут (одновременно) отказать с интенсивностью $\alpha$,
присоединившись к ранее отказавшим приборам.
При этом восстановление
происходит одновременно для всех приборов с ин\-тен\-сив\-ностью $\beta$
вне зависимости от того, в какой именно момент они отказали;
\item
процесс восстановления неисправных приборов не влияет на процесс
обслуживания исправных приборов;
\item
заявки, обслуживание которых было прервано отказами приборов,
после восстановления приборов дообслуживаются.
\end{itemize}

Назовем слоем $k$, $k\ge 0$, множество всех состояний процесса
обслуживания, в которых общее число заявок в системе равно $k$.

Слой $k$ при $k=\overline{0,n-1}$ представляет собой множество
\begin{equation*}
{\cal X}_k
=
%\bigcup\limits_{m=0}^{k}
\{(i_1,\ldots,i_m;j_1,\ldots,j_{k-m}),\ \ m=\overline{0,k}\}\,.
\end{equation*}
Состояние $(i_1,\ldots,i_m;j_1,\ldots,j_{k-m})$ означает, что первый
прибор обслуживает заявку на фазе $i_1$, $\ldots,$ $m$-й прибор~---
на фазе $i_m$, а $(k-m)$ приборов с заявками восстанавливаются,
прервав обслуживание заявок на фазах $j_1,\ldots,j_{k-m}$,
соответственно.

Слой $k$ при $k \ge n$ имеет вид
\begin{equation*}
{\cal X}_k
=
%\bigcup\limits_{m=0}^{n}
\{(i_1,\ldots,i_m;j_1,\ldots,j_{n-m})\,, \ m=\overline{0,n}\}\,.
\end{equation*}
Состояние $(i_1,\ldots,i_m;j_1,\ldots,j_{n-m})$ означает, что
первый прибор обслуживает заявку на фазе $i_1$, $\ldots,$
$m$-й прибор~--- на фазе $i_m$, остальные $(n-m)$ приборов с
заявками восстанавливаются, прервав обслу\-жи\-ва\-ние заявок на фазах
$j_1,\ldots,j_{n-m}$, соответственно, и еще $(k-n)$ заявок находятся
в накопителе.

Далее условимся обозначать через
$(i_1,\ldots,i^{(l)},\ldots,i_{m-1}; j_1,\ldots,j_{r})$,\
$l=\overline{1,m}$,
состояние, при котором занято обслуживанием $m$ приборов,
причем
первый прибор находится на фазе~$i_1$,
второй~--- на фазе $i_2$,
$\ldots,$
$(l-1)$-й~--- на фазе~$i_{l-1}$,
$l$-й~--- на фазе $i$,
$(l+1)$-й~--- на фазе $i_{l}$,
$\ldots,$
$m$-й~--- на фазе $i_{m-1}$,
и еще $r$ приборов с заявками восстанавливаются, прервав обслуживание
заявок на фазах $j_1,\ldots,j_{r}$.
Состояние слоя $k$, при котором все приборы с заявками восстанавливаются,
будем обозначать через $(0;j_1,\ldots,j_{k})$ при $k=\overline{0,n-1}$
и через $(0;j_1,\ldots,j_{n})$ при $k\ge n$.
Состояние слоя $k$, при котором все приборы с заявками исправны,
будем обозначать через $(i_1,\ldots,i_{k};0)$ при $k=\overline{0,n-1}$
и через $(i_1,\ldots,i_{n};0)$ при $k\ge n$.

Предполагая, что новые заявки в систему не поступают, перечислим все
возможные переходы между состояниями при обслуживании заявок с
указанием интенсивностей переходов.

Начнем с переходов из-за изменения фазы обслуживания заявок:
\begin{itemize}
\item
из состояния
$(i_1,\ldots,i^{(l)}\!,\ldots,i_{m-1};j_1,\ldots,j_{k-m})$, %\hspace*{-1.3pt}\newline
$m=\overline{1,k}$,
слоя $k$, \ $k=\overline{1,n-1}$, возможен переход в состояние
$(i_1,\ldots,j^{(l)},\ldots,i_{m-1};j_1,\ldots,j_{k-m})$,\  $j\ne i$,
того же слоя с интенсивностью $h_{ij}$ при изменении фазы обслуживания
заявки на $l$-м приборе с $i$-й на $j$-ю;
\item
из состояния
$(i_1,\ldots,i^{(l)}\!,\ldots,i_{m-1};j_1,\ldots,j_{n-m})$, %\hspace*{-1.75pt}\newline
$m=\overline{1,n}$,
слоя $k$, \ $k\ge n$, возможен переход в состояние
$(i_1,\ldots,j^{(l)},\ldots,i_{m-1};j_1,\ldots,j_{n-m})$,\  $j\ne i$,
того же слоя с интенсивностью $h_{ij}$ при изменении фазы обслуживания
заявки на $l$-м приборе с $i$-й на $j$-ю.
\end{itemize}

Следующий тип переходов образуется при отказах исправных
приборов с заявками:
\begin{itemize}
\item
из состояния
$(i_1,\ldots,i_{m};j_1,\ldots,j_{k-m})$,\  $m=\overline{1,k}$,
слоя $k$,\ $k=\overline{1,n-1}$, возможен переход в состояние
$(0;i_1,\ldots,i_{m},j_1,\ldots,j_{k-m})$ того же слоя с
интенсивностью $\alpha$ при отказе исправных занятых приборов;
\item
из состояния
$(i_1,\ldots,i_{m};j_1,\ldots,j_{n-m})$, \ $m=$\linebreak $=\overline{1,n}$,
слоя $k$, \ $k\ge n$, возможен переход в состояние
$(0;i_1,\ldots,i_{m},j_1,\ldots,j_{n-m})$ того же слоя с
интенсивностью $\alpha$ при отказе исправных занятых приборов.
\end{itemize}

Аналогично определяются переходы при восстановлении приборов:
\begin{itemize}
\item
из состояния
$(i_1,\ldots,i_{m};j_{1},\ldots,j_{k-m})$,\  $m=$\linebreak $=\overline{0,k-1}$,
слоя $k$,\ $k=\overline{1,n-1}$, возможен переход в состояние
$(i_1,\ldots,i_{m},j_{1},\ldots,j_{k-m};0)$
того же слоя с интенсивностью $\beta $ при восстановлении приборов;
\item
из состояния
$(i_1,\ldots,i_{m};j_{1},\ldots,j_{n-m})$,\ $m=$\linebreak $=\overline{0,n-1}$,
слоя $k$,\  $k\ge n$, возможен переход в состояние
$(i_1,\ldots,i_{m},j_{1},\ldots,j_{n-m};0)$
того же слоя с интенсивностью $\beta $ при восстановлении приборов.
\end{itemize}

Следующие переходы происходят при окончании обслуживания заявок:
\begin{itemize}
\item
из состояния
$(i_1,\ldots,i^{(l)}\!,\ldots,i_{m-1};j_{1},\ldots,j_{k-m})$,\
$m=\overline{1,k}$,
слоя $k$,\ $k=\overline{1,n}$, возможен переход в состояние
$(i_1,\ldots,i_{m-1};j_{1},\ldots,j_{k-m})$ слоя $(k-1)$ с
интенсивностью $h^*_i$ при окончании обслуживания заявки на $l$-м
приборе при $i$-й фазе;
\item
из состояния
$(i_1,\ldots,i^{(l)}\!,\ldots,i_{m-1};j_{1},\ldots,j_{n-m})$,\
$m=\overline{1,n}$,
слоя $k$,\ $k>n$, возможен переход в состояние
$(i_1,\ldots,i_{m-1},j;j_{1},\ldots,j_{n-m})$
слоя $(k-1)$ с интенсивностью $h^*_i h_{j}$ при окончании обслуживания
заявки на $l$-м приборе при $i$-й фазе и поступлении на него
новой заявки из очереди на $j$-ю фазу.
\end{itemize}

Осталось определить вероятности изменения состояний процесса
обслуживания в моменты поступления в систему заявок, т.\,е.\ элементы
мат\-риц~$\Omega_l$, \ $l=\overline{0,n-1}$:
\begin{itemize}
\item
из состояния
$(i_1,\ldots,i_{m};j_{1},\ldots,j_{k-m})$,\  $m=\overline{0,k}$,
слоя $k$, \ $k=\overline{0,n-1}$,
при поступлении заявки процесс обслуживания переходит в состояние
$(i_1,\ldots,i_{m},i;j_{1},\ldots,j_{k-m})$
слоя $k+1$ с ве\-ро\-ят\-ностью $h_i$ того, что поступающая в систему
заявка начнет обслуживаться с фазы $i$.
\end{itemize}

Сформировав матрицы $\Lambda_k$, $N_k$ и $\Omega_k$, можно, как и
прежде, воспользоваться формулами для расчета характеристик базовой
модели с отказами.

Стационарное распределение времени пребывания заявки в системе
вычисляется по формулам~(\ref{V_x_2}) и~(\ref{Vx3}), в которых
параметры функции распределения фазового типа $G(x)$ имеют вид
$$
G=
\begin{pmatrix}
H-\alpha E     &    \alpha E   \\
\beta E        &    -\beta E   \\
\end{pmatrix}\,,\ \
\vec{g}_2 =
(h_1,\ldots,h_J,0,\ldots,0)\,.
$$

\section{Независимые одновременные отказы занятых и~свободных
приборов и дообслуживание заявок, заявки поступают на~все~приборы}    %  8

Наконец, последняя рассматриваемая здесь сис\-те\-ма имеет следующие
особенности:
\begin{itemize}
\item
все занятые приборы отказывают одновременно с интенсивностью $\alpha$
и одновременно восстанавливаются с интенсивностью $\beta$;
\item
все свободные приборы отказывают одновременно с интенсивностью
$\alpha^*$ и одновременно восстанавливаются с интенсивностью $\beta^*$;
\item
 одновременные отказы и восстановления занятых приборов и
одновременные отказы и восстановления свободных приборов
происходят независимо друг от друга;
\item
если в системе имеются исправные свободные приборы, поступающая заявка
становится на один из таких приборов, иначе (если нет исправных свободных
приборов, но имеются неисправные свободные приборы)
становится на неисправный свободный прибор, который восстанавливается
уже с интенсивностью $\beta$;
\item
заявки, обслуживание которых было прервано отказами приборов,
после восстановления приборов дообслуживаются.
\end{itemize}

Определим слои, на которых может находиться процесс обслуживания,
следующим образом.

Слой $k$ при $k=\overline{0,n-1}$ представляет собой множество
\begin{multline*}
{\cal X}_k
=
\{(i_1,\ldots,i_m;j_1,\ldots,j_{k-m};j)\,,\\
m=\overline{0,k},\ \ j=\overline{0,n-k}\}\,.
\end{multline*}
Состояние $(i_1,\ldots,i_m;j_1,\ldots,j_{k-m};j)$ означает,
что $m$ приборов обслуживают заявки на фазах $i_1,\ldots, i_m$
и $(k-m)$ приборов с заявками прервали обслуживание на фазах
$j_1,\ldots,j_{k-m}$ и восстанавливаются, $j$~свободных приборов
исправны, $(n-k-j)$ свободных приборов восстанавливаются.
Состояние слоя~$k$, при котором $k$ приборов с заявками
восстанавливаются, прервав обслуживание на фазах $j_1,\ldots,j_{k}$,
и $j$ свободных приборов исправны, будем обозначать через
$(0;j_1,\ldots,j_k;j)$,
а состояние слоя $k$, при котором $k$ исправных приборов обслуживают
заявки на фазах $j_1,\ldots,j_{k}$ и $j$ свободных приборов исправны,
будем обозначать через $(j_1,\ldots,j_k;0;j)$.

Слой $k$ при $k \ge n$ имеет вид
\begin{equation*}
{\cal X}_k
=
\{(i_1,\ldots,i_m;j_1,\ldots,j_{n-m}),\ \ m=\overline{0,n}\}.
\end{equation*}
Состояние $(i_1,\ldots,i_m; j_1,\ldots,j_{n-m})$ означает, что $m$
приборов исправны и обслуживают заявки на фазах $i_1,\ldots,i_m$,
$(n-m)$ приборов с заявками прервали обслуживание заявок на фазах
$j_1,\ldots,j_{n-m}$ и восстанавливаются и еще $(k-n)$ заявок
находятся в накопителе.
Состояние слоя $k$, при котором $n$ приборов с заявками
восстанавливаются, прервав обслуживание на фазах $j_1,\ldots,j_{n}$,
будем обозначать через $(0;j_1,\ldots,j_{n})$,
а состояние слоя $k$, при котором все приборы исправны и обслуживают
заявки на фазах $j_1,\ldots,j_{k}$, будем обозначать через
$(j_1,\ldots,j_n;0)$.

Далее условимся при $k \ge n$ обозначать через
$(i_1,\ldots,i^{(l)},\ldots,i_{m-1};j_1,\ldots,j_{n-m})$,
$l=\overline{1,m}$,
состояние, при котором $m$ приборов исправны, причем первый прибор
находится на фазе $i_1$, второй~--- на фазе $i_2$,
$\ldots,$
$(l-1)$-й~--- на фазе $i_{l-1}$,
$l$-й~--- на фазе~$i$,
$(l+1)$-й~--- на фазе $i_{l}$,
$\ldots,$
$m$-й~--- на фазе $i_{m-1}$
и еще $(n-m)$ приборов восстанавливаются, прервав обслуживание на
фазах $j_1,\ldots,j_{k-m}$.
Аналогичным образом при $k=\overline{0,n-1}$ определяется состояние
$(i_1,\ldots,i^{(l)},\ldots,i_{m-1};j_1,\ldots,j_{s};j)$.

Перечислим все возможные переходы между состояниями при обслуживании
заявок (предполагается, что новые заявки в систему не поступают)
с указанием интенсивностей переходов.

Начнем с переходов из-за изменения фазы обслуживания заявки:
\begin{itemize}
\item
из состояния
$(i_1,\ldots,i^{(l)},\ldots,i_{m-1};j_1,\ldots,j_{k-m};$ $d)$,\
$l=\overline{1,m}$, \ $d=\overline{0,n-k}$,
слоя $k$,\ $k=$\linebreak $=\overline{1,n-1}$, возможен переход в состояние
$(i_1,\ldots,j^{(l)},\ldots,i_{m-1};j_1,\ldots,j_{k-m};d)$,\
$j\ne i$,
того же слоя с интенсивностью $h_{ij}$ при изменении фазы
обслуживания заявки на $l$-м приборе с $i$-й на $j$-ю;
\item
из состояния
$(i_1,\ldots,i^{(l)}\!,\ldots,i_{m-1};j_1,\ldots,j_{n-m})$,\
$l=\overline{1,m}$,
слоя $k$,\ $k\ge n$, возможен переход в состояние
$(i_1,\ldots,j^{(l)},\ldots,i_{m-1};j_1,\ldots,j_{n-m})$,\
$j\ne i$,
того же слоя с интенсивностью $h_{ij}$ при изменении фазы
обслуживания заявки на $l$-м приборе с $i$-й на $j$-ю.
\end{itemize}

Следующий тип переходов происходит из-за отказов приборов с
заявками:
\begin{itemize}
\item
из состояния
$(i_1,\ldots,i_{m};j_1,\ldots,j_{k-m};d)$, \  $d=$\linebreak $=\overline{0,n-k}$,
слоя $k$,\  $k=\overline{1,n-1}$, возможен переход в состояние
$(0;i_1,\ldots,i_{m},j_1,\ldots,j_{k-m};d)$
того же слоя с интенсивностью $\alpha$ при отказе исправных
приборов с заявками;
\item
 из состояния
$(i_1,\ldots,i_{m};j_1,\ldots,j_{n-m})$
слоя $k$,\ $k\ge n$, возможен переход в состояние
$(0;i_1,\ldots,i_{m},j_1,\ldots,j_{n-m})$ того же слоя с
интенсивностью $\alpha$ при отказе исправных приборов с заявками.
\end{itemize}

Аналогично определяются переходы из-за отказа приборов без заявок:
\begin{itemize}
\item
из состояния
$(i_1,\ldots,i_{m};j_1,\ldots,j_{k-m};d)$, \ $d=$\linebreak $=\overline{1,n-k}$,
слоя $k$,\ $k=\overline{1,n-1}$, возможен переход в состояние
$(i_1,\ldots,i_{m};j_1,\ldots,j_{k-m};0)$
того же слоя с интенсивностью $\alpha^*$ при отказе исправных
свободных приборов.
\end{itemize}

Переходы при восстановлении приборов с заявками определяются
следующим образом:
\begin{itemize}
\item
из состояния
$(i_1,\ldots,i_{m};j_1,\ldots,j_{k-m};d)$, \  $d=$\linebreak $=\overline{0,n-k}$,
слоя $k$, \ $k=\overline{1,n-1}$, возможен переход в состояние
$(i_1,\ldots,i_{k},j_1,\ldots,j_{k-m};0;d)$ того же слоя с
интенсивностью $\beta$ при вос\-ста\-нов\-ле\-нии всех неисправных
приборов с за\-яв\-ками;
\item
из состояния
$(i_1,\ldots,i_{m};j_1,\ldots,j_{n-m})$
слоя $k$,\ $k\ge n$, возможен переход в состояние
$(i_1,\ldots,i_{n},j_1,\ldots,j_{n-m};0)$
того же слоя с интенсивностью $\beta$ при восстановлении всех
неисправных приборов с заявками.
\end{itemize}

Переходы при восстановлении приборов без заявок определяются
следующим образом:
\begin{itemize}
\item
из состояния
$(i_1,\ldots,i_{m};j_1,\ldots,j_{k-m};d)$, \  $d=$\linebreak $=\overline{0,n-k-1}$,
слоя $k$,\ $k=\overline{1,n-1}$, возможен переход в состояние
$(i_1,\ldots,i_{m};j_1,\ldots,j_{k-m};$\linebreak $n-k)$
того же слоя с интенсивностью $\beta^*$ при восстановлении всех
неисправных приборов без заявок.
\end{itemize}

Следующие переходы происходят при окончании обслуживания заявки:
\begin{itemize}
\item
из состояния
$(i_1,\ldots,i^{(l)},\ldots,i_{m-1};j_1,\ldots,j_{k-m};$ $d)$,\
$d=\overline{0,n-k}$, \
$l=\overline{1,m}$,
слоя $k$, \ $k=\overline{1,n}$, возможен переход в состояние
$(i_1,\ldots,i_{m-1};j_1,\ldots,j_{k-m};d+1)$
слоя $(k-1)$ с интенсивностью $h^*_i $ при окончании обслуживания
заявки на $l$-м приборе при $i$-й фазе;
\item
из состояния
$(i_1,\ldots,i^{(l)}\!,\ldots,i_{m-1};j_1,\ldots,j_{n-m})$,\
$l=\overline{1,m}$,
слоя $k$, \ $k>n$, возможен переход в состояние
$(i_1,\ldots,i_{m-1},j;j_1,\ldots,j_{n-m})$
слоя $(k-1)$ с интенсивностью $h^*_i h_j$ при окончании обслуживания
заявки на $l$-м приборе при $i$-й фазе и поступлении на него
новой заявки из очереди на $j$-ю фазу.
\end{itemize}

Определим ненулевые элементы матриц
$\Omega_l$,\  $l=\overline{0,n-1}$:
\begin{itemize}
\item
из состояния
$(i_1,\ldots,i_{m};j_1,\ldots,j_{k-m};d)$,\ $d=$\linebreak $=\overline{1,n-k}$,
слоя $k$,\ $k=\overline{0,n-1}$,
при поступлении заявки процесс обслуживания заявок переходит в состояние
$(i_1,\ldots,i_{m},i;j_1,\ldots,j_{k-m};d-1)$ слоя $k+1$
с вероятностью $h_i$ того, что поступающая в систему заявка начнет
обслуживаться с фазы~$i$;
\item
из состояния
$(i_1,\ldots,i_{m};j_1,\ldots,j_{k-m};0)$
слоя $k$, \ $k=\overline{0,n-1}$,
при поступлении заявки процесс обслуживания заявок переходит в состояние
$(i_1,\ldots,i_{m};j_1,\ldots,j_{k-m},j;0)$ слоя $k+1$
с вероятностью $h_j$ того, что поступающая в систему заявка (после
восстановления неисправных приборов с заявками) начнет обслуживаться
с фазы $j$.
\end{itemize}

Найдя таким образом матрицы $\Lambda_k$,\  $k=\overline{0,n-1}$,\
$\Lambda$,\  $N_k$,\  $k=\overline{1,n}$,\  $N$ и
$\Omega_k$, \ $k=\overline{0,n-1}$, снова
можно воспользоваться результатами раздела~2 для отыскания стационарных
характеристик системы.

Стационарное распределение $V(x)$ времени пребывания заявки в системе
вычисляется точно так же, как и в разделе~4.

\section{Заключение}

Получены математические соотношения для расчета
стационарных характеристик СМО с полумарковским входящим потоком,
обслуживанием фазового типа и ненадежными приборами.
Рассмотрены различные варианты функционирования СМО с одновременно
отказывающими приборами при экспоненциальном процессе
отказов--вос\-ста\-нов\-ле\-ний.
Для некоторых типов СМО с ненадежными приборами на основе полученных
в настоящей работе результатов были написаны программные модули расчета
основных показателей функционирования.

{\small\frenchspacing
{%\baselineskip=10.8pt
\addcontentsline{toc}{section}{Литература}
\begin{thebibliography}{9}
\bibitem{1p}
\Au{Печинкин~А.\,В., Соколов~И.\,А., Чаплыгин~В.\,В.}
Многолинейные системы массового обслуживания с неза\-ви\-си\-мы\-ми
отказами и восстановлениями приборов~//
Системы и средства информатики.
Спец.\ выпуск <<Математическое и алгоритмическое обеспечение
информационно-телекоммуникационных сис\-тем>>.~---
М.: ИПИ РАН, 2006. С.~99--123.

\bibitem{2p}
\Au{Печинкин А.\,В., Соколов~И.\,А., Чаплыгин~В.\,В.}
Многолинейная система массового обслуживания с конечным
накопителем и ненадежными приборами~//
Информатика и её применения, 2007. Т.~1. Вып.~1. С.~27--39.

\bibitem{3p}
{\it Печинкин~А.\,В., Чаплыгин~В.\,В.}
Стационарные характеристики системы массового обслуживания $SM/MSP/n/r$~//
Автоматика и телемеханика, 2004. №\,9. С.~85--100.
\end{thebibliography}

}
}

\end{multicols}


\label{end\stat}