\def\stat{bosov}

\def\tit{УПРАВЛЕНИЕ ЛИНЕЙНЫМ ВЫХОДОМ МАРКОВСКОЙ ЦЕПИ 
ПО~КВАДРАТИЧНОМУ КРИТЕРИЮ. СЛУЧАЙ ПОЛНОЙ ИНФОРМАЦИИ$^*$}

\def\titkol{Управление линейным выходом марковской цепи 
по~квадратичному критерию. Случай полной информации}

\def\aut{А.\,В.~Босов$^1$}

\def\autkol{А.\,В.~Босов}

\titel{\tit}{\aut}{\autkol}{\titkol}

\index{Босов А.\,В.}
\index{Bosov A.\,V.}


{\renewcommand{\thefootnote}{\fnsymbol{footnote}} \footnotetext[1]
{Работа выполнялась с~использованием инфраструктуры Центра коллективного пользования 
<<Высокопроизводительные вы\-чис\-ле\-ния и~большие данные>> (ЦКП <<Информатика>> ФИЦ ИУ 
РАН, Москва).}}


\renewcommand{\thefootnote}{\arabic{footnote}}
\footnotetext[1]{Федеральный исследовательский центр <<Информатика и~управление>> Российской 
академии наук, \mbox{ABosov@frccsc.ru}}

\vspace*{-6pt}


\Abst{Решена задача оптимального управления линейным выходом 
стохастической дифференциальной системы формируемым аддитивным 
скачкообразно изменяющимся входным воздействием. Цель оптимизации 
задается квадратичным критерием специального вида, позволяющим 
формализовать задачи слежения за скачкообразно изменяющейся целью 
и~стабилизации системы около направлений, определяемых входом. Задача 
решается в~предположении наличия полной информации, т.\,е.\ известного 
состояния входной марковской цепи. Такая постановка дополняет ранее 
полученное решение задачи с~неполной информацией, в~которой имеет место 
разделение задач управления и~оценивания, обеспечиваемого оптимальным в~этом 
случае фильтром Вонэма. Поэтому полученный в~статье результат, помимо того 
что имеет самостоятельное значение, дает еще и~эталонное решение для анализа 
качества управления в~условиях косвенных наблюдений. Решение 
рассматриваемой задачи, как и~в~постановке с~неполной информацией, 
обеспечивается непосредственным применением метода динамического 
программирования. Уравнение Беллмана уточняется для заданной модели  
входа~--- используется мартингальное представление цепи и~ограниченная 
единичными координатными векторами область значений. Проведен численный 
эксперимент, результаты которого иллюстрируют работоспособность полученных 
алгоритмов управления в~обеих постановках: с~полной информацией 
и~косвенными наблюдениями.}

\KW{марковский скачкообразный процесс; линейная стохастическая 
дифференциальная система; оптимальное управление; квадратичный критерий; 
динамическое программирование}

\DOI{10.14357/19922264220203}
  
%\vspace*{-3pt}


\vskip 10pt plus 9pt minus 6pt

\thispagestyle{headings}

\begin{multicols}{2}

\label{st\stat}

\section{Введение}

     В статье представлено решение задачи, постановка которой 
предложена в~[1] формально даже в~более сложном случае. 
В~оптимизируемой системе определен вектор состояния, описываемый 
цепью Маркова, и~выходной вектор, описываемый линейным 
дифференциальным уравнением Ито с~винеровским процессом. Выход 
интерпретируется как косвенные наблюдения за состоянием марковской 
цепи, которая входит в~наблюдения аддитивно. Решение задачи в~[1] 
получено путем разделения задач управления и~оценивания, а оптимальный 
для такой системы наблюдения фильтр Вонэма~[2] тоже описывается 
обычным уравнением Ито. Из-за этого эквивалентная задача для состояния, 
заданного уравнением фильтрации, линейно преобразованных наблюдений 
и~неизменившегося при этом квадратичного критерия решается 
классическими методами оптимального управления. Деталям этого решения и~подробному исследованию задачи в~случае наблюдаемого входа, 
описываемого традиционным диффузионным процессом, посвящена 
работа~[3], а важное дополнение для случая зависимых возмущений 
в~уравнениях входа и~выхода получено в~[4].
     
     Задача, точно такая же, как в~[1], но в~условиях полной информации, 
т.\,е.\ в~предположении, что состояние входной цепи Маркова известно 
и~отсутствует необходимость его оценивания по косвенным наблюдениям, 
ранее не рассматривалась. Ее решение~--- основная цель данной статьи. 
Интересно заметить: хотя и~считается, что классическими методами 
оптимального управления~[5, 6] хорошо решаются именно задачи с~полной 
информацией, рассматриваемая модель привлекала ранее внимание только 
в~постановке с~ненаблюдаемым входом. Так, в~[7, 8] рассматриваемая 
модель оптимизировалась по упрощенному критерию, в~котором 
учитывались только слагаемые, штра\-фу\-ющие за величину управ\-ле\-ния, в~[9] 
решение получено для классического квад\-ра\-тич\-но\-го критерия и~даже более 
общей модели входа. В~связи с~этими результатами~[1] можно считать 
обобщением на случай квадратичного критерия специального вида, 
позволяющего решать задачи типа слежения за целью, в~роли которой 
выступает ненаблюдаемый вход. Основу для решения в~любой из 
перечисленных работ со\-став\-ля\-ют уравнения оптимальной фильт\-ра\-ции на 
основе обновляющих процессов~[10], фильтр Вонэма в~случае цепи~--- 
частный случай. Отсутствие внимания к~формально более прос\-то\-му случаю 
с~полной информацией, воз\-мо\-жно, объясняется интерпретацией входного 
процесса как сложного возмущения, с~воздействием которого требуется 
бороться. Критерий более общего вида~\cite{1-bos, 4-bos} больше 
ориентирован на постановки с~конструктивной интерпретацией входа как 
воздействия, на которое надо реагировать, изменять поведение выхода. 
Такой взгляд превращает возмущение в~со\-сто\-яние, из-за этого становится 
целесообразным рассматривать задачу в~случае полной информации. 
В~данной статье возможность такого подхода иллюстрируется при\-клад\-ным 
примером, приводятся результаты расчетов, подтверждающие при\-ме\-ни\-мость 
полученных алгоритмов управления и~позволяющие сравнить их качество 
в~постановках с~полной и~неполной информацией о~со\-сто\-янии.
{\looseness=1

}
     
\section{Постановка задачи управления выходом наблюдаемой 
цепи}

     На каноническом вероятностном пространстве $(\Omega, \mathcal{F}, 
\mathcal{P}, \mathcal{F}_t)$, $t\hm\in [0,T]$, рассмотрим линейную 
дифференциальную стохастическую систему
     \begin{equation}
     dz_t= a_t y_t \,dt+ b_t z_t \,dt+ c_t u_t \,dt+ \sigma_t\, dw_t\,,\ 
     z_0=Z\,,
     \label{e1-bos}
     \end{equation}
с управляемым выходом $z_t\hm\in \mathbb{R}^{n_z}$, винеровским 
процессом $w_t\hm\in \mathbb{R}^{n_w}$ и~аддитивным входом~$y_t$, 
представляющим собой марковскую цепь с~множеством состояний $\{e_1, 
\ldots , e_{n_y}\}$, сформированным из единичных координатных векторов 
в~евклидовом пространстве~$\mathbb{R}^{n_y}$, известным начальным 
состоянием $y_0\hm= {Y}$ с~распределением~$\pi_0$ и~матрицей 
интенсивностей переходов~$\Lambda_t$. Остальные величины 
в~уравнении~(1): случайный вектор $Z\hm\in \mathbb{R}^{n_z}$~--- 
гауссовская случайная величина с~известными моментами; $w_t$, $y_t$, $Y$ 
и~$Z$ независимы в~совокупности; заданные матричные функции $a_t\hm\in 
\mathbb{R}^{n_z\times n_y}$, $b_t\hm\in \mathbb{R}^{n_z\times n_z}$, 
$c_t\hm\in \mathbb{R}^{n_z\times n_u}$ и~$\sigma_t\hm\in 
\mathbb{R}^{n_z\times n_w}$ ограничены.
     
     Вектор управлений $u_t\hm\in \mathbb{R}^{n_u}$ выбирается из класса 
допустимых управлений~--- случайных процессов второго порядка, 
формируемых по закону управления с~обратной связью~\cite{6-bos}, т.\.е.\ 
в~виде $u_t\hm= U_t(y_t, z_t)$. Качество закона управления $U_t\hm= 
U_t(y,z)$, $y\hm\in \mathbb{R}^{n_y}$, $z\hm\in \mathbb{R}^{n_z}$, 
определяется функционалом вида

\noindent
     \begin{multline}
     J(U_0^T) =\mathbb{E}\left\{ 
     \int\limits_0^T \left\| P_t y_t +Q_t z_t +R_t u_t\right\|^2_{S_t} dt +{}\right.\\
\left.     {}+\left\| 
P_T y_T+ Q_T z_T\right\|^2_{S_T}\right\}\,,
     \label{e2-bos}
     \end{multline}
где $U_0^T=\{ U_t(y,z),\ 0\leq t\leq T\}$; $P_t\hm\in \mathbb{R}^{n_J\times 
n_y}$, $Q_t\hm\in \mathbb{R}^{n_J\times n_z}$, $R_t\hm\in 
\mathbb{R}^{n_J\times n_u}$, $S_t\hm\in \mathbb{R}^{n_J\times n_J}$,  
$S_t\hm\geq 0$, $S_t\hm= S_t^\prime$, $0\hm\leq t\hm\leq T$,~--- заданные 
ограниченные матричные функции; $\| x\|_S^2\hm= x^\prime S x$~--- весовая 
функция для симметричной неотрицательно определенной матрицы~$S$, 
единичной матрице $S\hm=\mathbf{1}$ соответствует евклидова норма $\| 
x\|^2_{\mathbf{1}}\hm= \vert x\vert^2$, $x^\prime$~--- транспонированная 
матрица~$x$. Для существования решений выписанных далее уравнений 
предполагается, что все функции $\Lambda_t$, $a_t$, $b_t$, $c_t$, $\sigma_t$, 
$P_t$, $Q_t$, $R_t$ и~$S_t$ ку\-соч\-но-не\-пре\-рыв\-ны. Кроме того, 
предполагаются выполненными условия невырожденности $R^\prime_t S_t 
R_t \hm>0$ и~$\sigma_t \sigma_t^\prime \hm>0$.
     
     Обозначим через $u_t^*$ оптимальное управление, доставляющее 
минимум квадратичному функционалу $J(U_0^T)$:
     \begin{equation}
 \left( U^*\right)_0^T\! =\!\left\{ U_t^*(y,z),\ 0\leq t\leq T\right\} \in \argmin 
J\!\left( U_0^T\right)\!,\!
     \label{e3-bos}
     \end{equation} 
т.\,е.\ $u_t^*=U_t^*(y_t, z_t^*)$, где $z_t^*$~--- решение~(\ref{e1-bos}) при 
$u_t\hm= u_t^*$~--- является $\mathcal{F}_t^{y,z}$-из\-ме\-ри\-мым процессом, 
$\sigma$-ал\-геб\-ра $\mathcal{F}_t^{y,z}$ порождена наблюдаемыми величинами 
$\{y_\tau, z_\tau,\ 0\hm\leq \tau\hm\leq t\}$, $\mathcal{F}_t^{y,z}\hm\subseteq 
\mathcal{F}_t\hm\subseteq \mathcal{F}$.

\section{Основное утверждение}

     Вначале заметим, что в~[1] получено решение для варианта 
рассматриваемой задачи в~постановке с~неполной информацией, т.\,е.\ 
в~предположении, что целевой функционал~(2) оптимизируется для 
управлений вида $u_t\hm= U_t(z_t)$. Со\-от\-вет\-ст\-ву\-ющее оптимальное 
управление~--- $u_t^{**}\hm= U_t^{**}(z_t^{**})$, где $z_t^{**}$~--- 
решение~(\ref{e1-bos}) при $u_t\hm= u_t^{**}$~--- является  
$\mathcal{F}_t^z$-из\-ме\-ри\-мым процессом, $\sigma$-ал\-геб\-ра~$\mathcal{F}_t^z$ 
порождена наблюдаемыми величинами $\{z_\tau, \ 0\hm\leq\tau\hm\leq t\}$, 
$\mathcal{F}_t^z\hm\subseteq \mathcal{F}_t^{y,z}\hm\subseteq \mathcal{F}_t\hm\subseteq 
\mathcal{F}$. Это решение при выполнении перечисленных в~постановке задачи 
условий существует и~определяется следующими соотношениями:
     \begin{multline*}
     u_t^{**}= -\fr{1}{2} \left( R^\prime_t S_t R_t\right)^{-1} \left( c_t^\prime 
\left( 2\alpha_t z_t^{**}+\beta_t \hat{y}_t\right) +{}\right.\\
\left.{}+2R_t^\prime S_t\left( P_t 
\hat{y}_t +Q_t z_t^{**}\right)\right)\,,
%\label{e4-bos}
\end{multline*}
где $\hat{y}_t=\mathbb{E}\left\{ y_t\vert \mathcal{F}_t^z\right\}$~--- оценка 
фильтра Вонэма
\begin{multline}
d\hat{y}_t= \Lambda_t^\prime \hat{y}_t\,dt +\left(\mathrm{diag}\,(\hat{y}_t)-
\hat{y}_t \hat{y}_t^\prime \right) a_t^\prime \left(\sigma_t 
\sigma_t^\prime\right)^{-1} \left( dz_t -{}\right.\\
\left.{}-a_t \hat{y}_t \,dt -b_t z_t\, dt -c_t u_t \,dt\right)\,,\enskip
\hat{y}_0=\mathbb{E}\{Y\}\,,
\label{e5-bos}
\end{multline}
а функции $\alpha_t$ и~$\beta_t$~--- решения уравнений
\begin{multline}
\fr{d\alpha_t}{dt}-\left( M_t^{\alpha} \alpha_t +\alpha_t^\prime \left( 
M_t^\alpha\right)^\prime \right) +N_t^\alpha -{}\\
{}- \alpha_t^\prime c_t \left( R_t^\prime S_t R_t\right)^{-1} c_t^\prime 
\alpha_t=0\,,\enskip \alpha_T= Q_T^\prime S_T Q_T\,;
\label{e6-bos}
\end{multline}

\noindent
\begin{equation}
\fr{d\beta_t}{dt}+ \beta_t \Lambda_t^\prime +M_t^\beta -N_t^\beta 
\beta_t=0\,,\enskip 
\beta_T=2Q_T^\prime S_T P_T\,.
\label{e7-bos}
\end{equation}
Здесь обозначено:
\begin{align*}
M_t^\alpha &= Q^\prime_t S_t R_t \left( R_t^\prime S_t R_t\right)^{-1} c_t^\prime\,;\\
N_t^\alpha &= Q^\prime_t \left( S_t -S_t R_t\left( R_t^\prime S_t R_t\right)^{-
1}R_t^\prime S_t\right) Q_t\,;\\
M_t^\beta &= 2\left(\left( a_t^\prime -P_t^\prime S_t R_t \left( R_t^\prime S_t 
R_t\right)^{-1} c_t^\prime\right) \alpha_t +{}\right.\\
&\hspace*{8mm}\left.{}+P_t^\prime \left( S_t -S_t R_t \left( 
R_t^\prime S_t R_t\right)^{-1} R_t^\prime S_t\right) Q_t\right)\,;\\
N_t^\beta &= Q_t^\prime S_t R_t \left( R_t^\prime S_t R_t\right)^{-1} c_t^\prime 
+\alpha_t c_t \left( R_t^\prime S_t R_t\right)^{-1} c_t^\prime\,.
\end{align*}
          Это решение получено путем разделения задач управления 
и~фильтрации, в~результате чего исходная постановка с~ненаблюдаемым 
состоянием~$y_t$ и~наблюдениями~$z_t$ заменяется на эквивалентную 
задачу с~полной информацией для известного состояния~$\hat{y}_t$ 
и~выхода~$z_t$. При этом искомый закон управления~$U_t^{**}(z)$ 
принимает вид $U_t^{**}(y,z)$, в~котором переменная $y\hm\in 
\mathbb{R}^{n_y}$ соответствует состоянию~$\hat{y}_t$, переменная 
$z\hm\in \mathbb{R}^{n_z}$~--- как и~было, выходу~$z_t$ и~формально 
$y\hm= y(z)$.
     
     Теперь можно сформулировать основной результат для 
рассматриваемой постановки с~полной информацией.
     
     \smallskip
     
     \noindent
     \textbf{Теорема.}\ \textit{Оптимальное решение $U_t^*$ задачи 
оптимизации}~(\ref{e3-bos}) \textit{для целевого функционала $J(U_0^T)$, 
заданного в}~(\ref{e2-bos}), \textit{определяется следующим 
соотношением}:
     \begin{multline}
     \!U_t^*= U_t^*(y,z) =-\fr{1}{2} \left( R_t^\prime S_t R_t\right)^{-1}\! \left( 
c_t^\prime \left( 2\alpha_t z+\beta_t y\right) +{}\right.\\
\left.{}+2R_t^\prime S_t \left( P_t y+Q_t 
z\right)\right),
     \label{e8-bos}
     \end{multline}
\textit{где функции $\alpha_t$ и~$\beta_t$~--- решения}~(\ref{e6-bos}) и~(\ref{e7-bos}).

\smallskip

     Таким образом, решение задачи с~полной информацией совпадает 
с~решением задачи в~случае косвенных наблюдений, представленном 
в~разделенном виде: $U_t^*(y,z)\hm= U_t^{**}(y,z)$, траектории 
оптимального управления $u_t^*\hm= U_t^{**}(y_t, z_t^*)$, т.\,е.\ 
получаются подстановкой в~$U_t^{**}(y,z)$ величин $y\hm= y_t$, 
траектории~$u_t^{**}$ получаются так же, но подстановкой $y\hm= 
\hat{y}_t$. Такое свойство решения в~\cite{9-bos} названо сильным 
разделением.
     
\smallskip

\noindent
Д\,о\,к\,а\,з\,а\,т\,е\,л\,ь\,с\,т\,в\,о\,.\ \ Вывод~(\ref{e8-bos}) в~целом 
обеспечивается выкладками, проделанными в~\cite{3-bos} для решения 
аналогичной задачи для такого же целевого функционала $J(U_0^T)$, 
выхода~$z_t$, заданного~(\ref{e1-bos}), и~входного воздействия~$y_t$, 
описываемого диффузионным уравнением с~винеровским процессом 
$V_t\hm\in \mathbb{R}^{n_V}$, не зависящим от~$w_t$ , $Y$ и~$Z$:
\begin{equation}
dy_t = \Phi_t(y_t) \,dt+ \Sigma_t(y_t) \,dV_t\,,\enskip y_0=Y\,.
\label{e9-bos}
\end{equation}

     
     В рассматриваемом случае, как и~в~\cite{3-bos}, для решения 
применим метод динамического программирования. Для функции Беллмана
     \begin{multline*}
     V_t= V_t(y,z) ={}\\
     {}= \mathop{\mathrm{inf}}\limits_{U_t^T} \mathbb{E}\left\{ 
\int\limits_t^T \left\| P_s y_s +Q_s z_s +R_s u_s\right\|^2_{S_s} ds +{}\right.\\
\left.{}+
     \left\| P_T y_T +Q_T z_T \right\|^2_{S_T}
     \vphantom{\int\limits_t^T}\right\}
     \end{multline*}
воспользуемся уравнением в~следующем виде (см.~\cite[гл.~6, \S\,2]{6-bos}):
$$
\fr{\partial V_t}{\partial t} +\min\limits_u \left\{ \mathcal{A}_t^u \{V_t\}+\left\| P_t y 
+Q_t z + R_t u\right\|^2_{S_t}\right\} =0\,,
$$
где оператор $\mathcal{A}_t^u$ определяет  
дифференциал~$V_t$ (см.~\cite[гл.~5, \S\,5]{6-bos}, а~именно:
$$
\fr{d}{dt}\,\mathbb{E}\left\{ V_t\vert \mathcal{F}_t^{y,z}\right\} =\mathbb{E}\left\{
\fr{\partial V_t}{\partial t} +\mathcal{A}_t^u\{V_t\}\vert \mathcal{F}_t^{y,z}\right\}\,.
$$
     
     Выражение для $\mathcal{A}_t^u$, когда~$y_t$ задано  
уравнением~(\ref{e9-bos}), дает формула Ито для диффузионного процесса. 
В~этом случае уравнение Беллмана принимает вид:
     \begin{multline}
     \fr{\partial V_t}{\partial t} +\fr{1}{2}\left( \mathrm{tr}\left\{ 
\Sigma_t^\prime \fr{\partial^2 V_t}{\partial z^2}\,\Sigma_t\right\} 
+\mathrm{tr}\left\{ \sigma_t^\prime \fr{\partial^2 V_t}{\partial 
z^2}\,\sigma_t\right\}\right)+{}\\
     {}+ \min\limits_u \left\{ \Phi_t^\prime \fr{\partial V_t}{\partial y} +\left( 
a_t y +b_t z+ c_t u\right)^\prime \fr{\partial V_t}{\partial z}+{}\right.\\
\left.{}+
    \left\| P_t y+ Q_t z +R_t u\right\|^2_{S_t}
    \vphantom{\fr{\partial V_t}{\partial y}}
    \right\}=0\,.
     \label{e10-bos}
\end{multline}
     
     В рассматриваемом случае марковской цепи~$y_t$  
вместо~(\ref{e9-bos}) воспользуемся мартингальным  
пред\-став\-ле\-ни\-ем~\cite{2-bos}:
     \begin{equation}
     dy_t= \Lambda_t^\prime y_t \,dt +d\Lambda_t^y\,,\enskip y_0=Y\,,
     \label{e11-bos}
     \end{equation}
где $\Lambda_t^y$~--- $\mathcal{F}_t$-со\-гла\-со\-ван\-ный мартингал  
с~квад\-ра\-тич\-ной характеристикой 
\begin{multline*}
\left\langle \Lambda^y, \Lambda^y\right\rangle_t ={}\\
{}= \int\limits_0^t \left( 
\mathrm{diag}\left(\Lambda^\prime_s y_s\right) -
\Lambda^\prime_s\mathrm{diag}\left( y_s\right) -\mathrm{diag}\left( 
y_s\right)\Lambda_s\right) ds\,.
\end{multline*}
 
     
     Соответственно, выражение для $\mathcal{A}_t^u$ 
и~скачкообразного~$y_t$, заданного уравнением~(\ref{e11-bos}), дает 
формула Ито для семимартингалов~\cite{11-bos} и~уравнение Беллмана 
записывается в~виде:
     \begin{multline}
     \fr{\partial V_t}{\partial t}+\fr{1}{2}\!\left( \mathrm{tr}\!\left\{ \fr{\partial^2 
V_t}{\partial y^2}\left\langle \Lambda^y, \Lambda^y\right\rangle_t\right\} 
+\mathrm{tr} \left\{ \sigma_t^\prime \fr{\partial^2 V_t}{\partial 
z^2}\,\sigma_t\right\}\right)+{}\\[3pt]
     {}+
     \min\limits_u \left\{ y^\prime \Lambda_t \fr{\partial V_t}{\partial y} +\left( 
a_t y+b_t z+c_t u\right)^\prime \fr{\partial V_t}{\partial z} +{}\right.\hspace*{-2.20961pt}\\[3pt]
\left.{}+\left\| P_t y+Q_t 
z+R_t u\right\|^2_{S_t}
\vphantom{\fr{\partial V_t}{\partial y}}
\right\}=0\,.
     \label{e12-bos}
     \end{multline}
     
     Решить~(\ref{e12-bos}), как и~(\ref{e10-bos}), позволяют 
предположения о~специальной форме функции Беллмана. Во-пер\-вых,  
и~(\ref{e12-bos}), и~(\ref{e10-bos}) позволяют предположить квадратичный 
вид~$V_t$ по переменной~$z$:
     \begin{equation}
     V_t= z^\prime \alpha_t z + z^\prime B_t(y) +\Gamma_t(y)\,.
     \label{e13-bos}
     \end{equation}
     
     Следующее предположение сделаем для коэффициента $B_t(y)$. Для 
диффузионного процесса~$y_t$, заданного уравнением~(\ref{e9-bos}), 
дополнительно предположим, что имеет место случай линейного сноса, т.\,е.\ 
$\Phi_t(y)\hm= \varphi_t y$. Для скачкообразного процесса~$y_t$, заданного 
уравнением~(\ref{e11-bos}), это условие выполнено изначально, можно 
считать, что $\Phi_t(y)\hm= \Lambda_t^\prime y$. В~этом частном 
случае~(\ref{e13-bos}) уточняется дополнительным предположением 
линейности $B_t(y)\hm= \beta_t y$, т.\,е.
     \begin{equation}
     V_t= z^\prime \alpha_t z+ z^\prime \beta_t y+ \Gamma_t(y).
     \label{e14-bos}
\end{equation}
     
     Окончательное предположение о виде~$V_t$ можно сделать только для 
скачкообразного процесса~$y_t$, заданного уравнением~(\ref{e11-bos}), 
а~именно: поскольку цепь~$y_t$ принимает значения единичных векторов 
$\{e_1, \ldots , e_{n_y}\}$, то в~функции Беллмана~(\ref{e14-bos}) можно 
задать линейным третий коэффициент, т.\,е.\  $\Gamma_t(y)\hm= \gamma_t  y+\delta_t$ и~окончательно
     \begin{equation}
     V_t= z^\prime\alpha_t z+ z^\prime\beta_t y+ \gamma_t y +\delta_t\,.
     \label{e15-bos}
     \end{equation}
     
     Действительно, в~общем виде~$V_t$ можно записать как $V_t(y,z)\hm= 
(V_t(e_1,z), \ldots , V_t(e_{n_y}, z))y$, что с~учетом  
зависимости~(\ref{e14-bos}) от~$z$ как раз и~дает~(\ref{e15-bos}).
     
     Таким образом, уравнение Беллмана для рассматриваемой задачи 
принимает вид:
     \begin{multline}
     \fr{\partial V_t}{\partial t} +\fr{1}{2}\mathrm{tr}
     \left\{ \sigma_t^\prime \fr{\partial^2 V_t}{\partial 
z^2}\,\sigma_t\right\}+{}\\[2pt]
     {}+
     \min\limits_u\left\{ y^\prime \Lambda_t \fr{\partial V_t}{\partial y} +\left( 
a_t y+b_t z+c_t u\right)^\prime \fr{\partial V_t}{\partial z} +{}\right.\\[2pt]
\left.{}+\left\| P_t y+Q_t z+R_t 
u\right\|^2_{S_t}
\vphantom{\fr{\partial V_t}{\partial y}}\right\}=0
     \label{e16-bos}
\end{multline}
и решается с~начальным условием
\begin{equation*}
V_T= \left\| P_T y+Q_T z\right\|^2_{S_T}\,,
%\label{e17-bos}
\end{equation*}
т.\,е.\ $\alpha_T=Q_T^\prime S_T Q_T$, $\beta_T\hm= 2Q_T^\prime S_T P_T$, 
$\gamma_T y\hm= y^\prime P_T^\prime S_T P_T y$ и~$\delta_T\hm=0$.
     
     Последний этап доказательства:  подстановку в~(\ref{e16-bos}) 
решения в~виде~(\ref{e15-bos})~--- можно выполнить как непосредственно, 
так и~сославшись на уже проделанные выкладки в~\cite{3-bos}. Учитывая 
сходство уравнений~(\ref{e16-bos}) и~(\ref{e10-bos}) и~сходство их 
решений~(\ref{e15-bos}) и~(\ref{e14-bos}), доказываемый результат, т.\,е.\ 
соотношение для оптимального управления~(\ref{e8-bos}) 
с~параметрами~(\ref{e6-bos}) и~(\ref{e7-bos}), из решения~(\ref{e16-bos}) 
можно получить как частный случай. Это соображение завершает 
доказательство.

\smallskip
     
     В дополнение к~доказательству заметим, что \mbox{общее} решение  
из~\cite{3-bos} помимо уравнений для коэффициентов~$\alpha_t$ и~$\beta_t$ 
включает и~уравнение для коэффициента $\Gamma_t(y)$ в~(\ref{e16-bos}). 
Соответственно, в~рассматриваемом случае вместо $\Gamma_t(y)$ 
записываются уравнения для~$\gamma_t$ и~$\delta_t$:
     \begin{equation}
     \left.
     \begin{array}{rlrl}
     \hspace*{-3mm}\fr{\partial \gamma_t}{\partial t} +y^\prime \Lambda_t 
\Lambda_t^\prime+M_t^\gamma y&=0\,,& \gamma_T y&= y^\prime 
P_T^\prime S_T P_T y;\\[6pt]
\fr{\partial \delta_t}{\partial t} +\mathrm{tr}\left\{ \sigma_t^\prime \alpha_t 
\sigma_t\right\}&=0\,, & \delta_T&=0\,,
\end{array}\!
\right\}\!\!
 \label{e18-bos}
 \end{equation}
 где
 \begin{multline*}
     M_t^\gamma y = N_t^\gamma(y) ={}\\
    {}=y^\prime \beta_t^\prime \left( a_t-
c_t\left( R_t^\prime S_t R_t\right)^{-1} R_t^\prime S_t P_t\right) y+{}\\
    {}+y^\prime P_t^\prime \left( S_t-S_t R_t\left( R_t^\prime S_t R_t\right)^{-
1} R_t^\prime S_t\right) P_t y-{}\\
{}-
     \fr{1}{4} \,y^\prime \beta_t^\prime c_t\left( R_t^\prime S_t R_t\right)^{-1} 
c_t^\prime \beta_t y\,.
          \end{multline*}
     

     Особого практического значения уравнения~(\ref{e18-bos}) не имеют, 
но позволяют еще раз прокомментировать использованное в~доказательстве 
свойство линейности функции Беллмана. Это предположение означает, 
в~частности, что в~(\ref{e18-bos}) должна иметь место линейность 
$N_t^\gamma(y)$, несмотря на наличие множителей~$y^\prime y$. Это 
условие обеспечивается пред\-став\-ле\-нием 
$$
M_t^\gamma y = \left(  N_t^\gamma\left(e_1\right),\ldots , N_t^\gamma\left(e_{n_y}\right)\right) y\,.
$$ 

Аналогично 
равенство $\gamma_T y\hm= y^\prime P_T^\prime S_T P_T y$, записанное 
в~виде 
$$
\gamma_T y = \left( e^\prime_1 P_T^\prime S_T P_T e_1,\ldots , 
e^\prime_{n_y} P_T^\prime S_T P_T e_{n_y}\right) y\,,
$$
 дает начальное условие 
для вычисления~$\gamma_t$. Отметим, что такое свойство типично для 
задач, ис\-поль\-зу\-ющих модель марковской цепи  
с~об\-ластью значений $\{e_1, \ldots , e_{n_y}\}$.

\section{Модельный пример}

     Представляемый численный пример использует модель простой 
механической системы, в~качестве которой может выступать 
воздействующее устройство типа силового привода. Состояние такого 
устройства определяется его положением и~ско\-ростью. Значение ско\-рости 
формируется силовым воздействием, аддитивно зависящим от текущего 
положения~$x_t$, текущей ско\-рости~$v_t$, контролируемого управ\-ля\-юще\-го 
воздействия~$u_t$ и~неконтролируемого входа~$y_t$. При отсутствии 
воздействия~$y_t$ физический смысл привода состоит в~стабилизации 
положения и~скорости около нуля. Включение в~модель~$y_t$ предполагает 
изменение цели управления на стабилизацию положения около 
изменяющегося значения, формируемого воздействием~$y_t$. Если бы~$y_t$ 
описывалось линейным дифференциальным уравнением с~винеровским 
процессом, то такую задачу можно было бы формализовать классической  
ли\-ней\-но-квад\-ра\-ти\-че\-ской постановкой, похожий по существу пример 
есть в~\cite{12-bos}. В~предлагаемой модели~$y_t$~--- марковская цепь, 
и~привод должен следовать за одним из состояний этой цепи. Данное 
описание реализует модель вида
     \begin{equation}
     \left.
     \begin{array}{rl}
 dx_t&=v_t \,dt\,,\enskip t\in (0,T]\,,\\[6pt]
 dv_t&= ax_t \,dt+ bv_t \,dt+ cy_t\, dt+{}\\
 &\hspace*{20mm}{}+ hu_t \,dt+\displaystyle \sqrt{g}\,dw_t\,,
     \end{array}
\!     \right\}\!\!
     \label{e19-bos}
     \end{equation}
     %
     Здесь предполагается, что цепь~$y_t$ имеет три состояния $\{e_1, e_2, 
e_3\}$, постоянную матрицу интенсивностей 
$$
\Lambda_t= \Lambda= 
\begin{pmatrix}
     -0{,}5 & 0{,}5 & 0\\
     0{,}5 & -1 & 0{,}5\\
     0& 0{,}5 & -0{,}5
     \end{pmatrix}
     $$ 
     и~начальное распределение $\pi_0\hm= (1,\, 0,\, 0)^\prime$, т.\,е.\ 
$y_0\hm= Y\hm= e_1$, а~целевой функционал
     \begin{equation}
     J\left( U_0^T\right) =\mathbb{E}\left\{ \int\limits_0^T \left( \left\vert Cy_t-
x_t\right\vert^2 +R\left\vert u_t\right\vert^2\right)dt\right\}.
     \label{e20-bos}
\end{equation}
     
     Начальные значения $x_0$ и~$v_0$~--- независимые гауссовские 
случайные величины с~нулевым средним и~дисперсиями~$\sigma_x^2$ 
и~$\sigma_v^2$ соответственно. Остальные параметры  
модели~(\ref{e19-bos})~--- это числа~$a$, $b$, $h$, $g$, $R$ и~$T$ и~строки 
$c\hm= (c_1, c_2, c_3)$ и~$C\hm= (C_1, C_2, C_3)$. Полагая $C_i\hm= -c_i/a$, 
получаем цель управления, состоящую в~стабилизации положения~$x_t$ 
около величины~$C_i$, если цепь~$y_t$ находится в~состоянии~$e_i$.
     
     Будем рассматривать случай, когда система~(\ref{e19-bos}) устойчива, 
что обеспечивается условиями $b\hm< 0$ и~$b^2\hm+ 4a \hm<0$. Для расчета 
выбран следующий набор
 значений параметров: $a\hm= -1$; $b\hm= -0{,}1$; 
$h\hm= 10$; $g\hm= 1{,}0$; $R\hm= 0{,}01$; $T\hm= 20$; $\sigma_x^2\hm=1$; 
$\sigma_v^2\hm=1$;  $c_1\hm=1$; $c_2\hm=4$; $c_3\hm=8$.
     
Интегрирование системы~(\ref{e19-bos}) выполнено методом Эйлера 
с~шагом 0,001, неявный метод Эйлера\linebreak с~таким же шагом использовался для 
при\-бли\-жен\-но\-го интегрирования~(\ref{e6-bos}) и~(\ref{e7-bos}) при 
вы\-чис\-ле\-нии коэффициентов~$\alpha_t$ и~$\beta_t$. Формальное нарушение 
мо\-делью~(\ref{e19-bos}) условия не\-вы\-рож\-ден\-ности $\sigma_t \sigma_t^\prime 
\hm>0$\linebreak не мешает вычислить оценку фильт\-ра Вонэма, поскольку для данной 
модели очевидно, что $\mathcal{F}_t^{x,v}\hm= \mathcal{F}_t^v$, поэтому оценка 
$$
\hat{y}_t= \mathbb{E}\left\{ y_t\vert \mathcal{F}_t^z\right\}= \mathbb{E}\left\{ y_t\vert \mathcal{F}_t^v\right\}
$$ 
вычисляется как решение 
уравнения~(\ref{e5-bos}) с~использованием только наблюдения~$v_t$ также 
методом Эйлера с~шагом~0,001.
     
     Моделировался пучок из  траекторий цепи~$y_t$, управлений $u_t^*$ 
и~$u_t^{**}$ и~соответствующих траекторий системы $x_t^*$, $x_t^{**}$ 
и~$v_t^*$, $v_t^{**}$. Рассчитаны также траектории неуправляемой 
системы $x_t^0$ и~$v_t^0$, т.\,е.\ решения~(\ref{e19-bos}) для $u_t\hm= 
u_t^0\hm=0$. Для анализа качества управлений с~учетом динамики 
вычислено значение целевого функционала 
     $$
     J\left(U_0^t\right)= \mathbb{E}\left\{ \int\limits_0^t \left( \left\vert C y_s 
- x_s\right\vert^2 \hm+ R\left\vert u_s\right\vert^2\right)ds\right\},
$$
 т.\,е.\ 
функции~$J((U^*)_0^t)$, $J((U^{**})_0^t)$ и~$J((U^0)_0^t)$ для 
соответствующих управлений.


     
     Результаты расчетов проиллюстрированы на рис.~1, где показан 
пример траекторий положений $x_t^*$, $x_t^{**}$ и~$x_t^0$, фор\-ми\-ру\-емых 
управлениями~$u_t^*$, $u_t^{**}$ и~$u_t^0$, на рис.~2 и~3, где пред\-став\-ле\-ны 
со\-от\-вет\-ст\-ву\-ющие тра\-ек\-то\-рии скоростей и~управ\-ле\-ний, и~на 
рис.~4 с~графиками роста целевого функционала.
     
    

     
     Результирующие значения целевого функционала~(\ref{e20-bos}): 
$J((U^*)_0^{20})\hm= 22{,}6$; $J((U^{**})_0^{20})\hm= 37{,}2$; $J((U^0)_0^{20})\hm= 409{,}5$.

{ \begin{center}  %fig1
 \vspace*{12pt}
   \mbox{%
\epsfxsize=78mm
\epsfbox{bos-1.eps}
}

\end{center}

\noindent
{{\figurename~1}\ \ \small{
Примеры траекторий положений: \textit{1}~--- $x_t^*$; \textit{2}~--- $x_t^{**}$; 
\textit{3}~--- $x_t^0$; \textit{4}~--- цель $C y_t$
}}}

%\vspace*{6pt}



{ \begin{center}  %fig2
 \vspace*{3pt}
    \mbox{%
\epsfxsize=78mm
\epsfbox{bos-2.eps}
}

\end{center}

%\vspace*{-3pt}

\noindent
{{\figurename~2}\ \ \small{
Примеры траекторий скоростей: \textit{1}~--- $v_t^*$; \textit{2}~--- $v_t^{**}$; 
\textit{3}~--- $v_t^0$; \textit{4}~--- цель $C y_t$
}}}

\vspace*{9pt}

{ \begin{center}  %fig3
 \vspace*{3pt}
   \mbox{%
\epsfxsize=78mm
\epsfbox{bos-3.eps}
}

\end{center}


%\vspace*{-3pt}

\noindent
{{\figurename~3}\ \ \small{
Примеры траекторий управлений: 
\textit{1}~--- $u_t^*$; \textit{2}~--- $u_t^{**}$; \textit{3}~--- $u_t^0$; \textit{4}~--- цель $C 
y_t$
}}}

\vspace*{9pt}



{ \begin{center}  %fig4
 \vspace*{-3pt}
     \mbox{%
\epsfxsize=78.952mm
\epsfbox{bos-4.eps}
}

\end{center}

%\vspace*{-3pt}

\noindent
{{\figurename~4}\ \ \small{
Динамика целевой функции: \textit{1}~--- $J((U^*)_0^t)$; \textit{2}~--- 
$J((U^{**})_0^t)$; \textit{3}~--- $J((U^0)_0^t)$
}}}

\vspace*{-6pt}



\section{Заключение}

     Главный теоретический результат статьи~--- это полученное 
выражение~(\ref{e8-bos}) для оптимального управ\-ле\-ния $U_t^*(y,z)$ 
в~постановке с~полной информацией. Совпадение решения с~рассмотренным 
в~\cite{1-bos} случаем с~косвенными наблюдениями ожидалось, но требовало 
формального подтверждения. Практический результат~--- это выполненный 
расчет, т.\,е.\ реализация алгоритмов управления $U_t^*(y,z)$ 
и~$U_t^{**}(z)$, подтвердившая их практическую применимость 
и~эффективность на подходящем прикладном примере. При этом работа 
с~примером имела одну известную  
труд\-ность, связанную с~фильтром Вонэма, численная реализация которого 
требует особого внимания в~части устойчивости. В~связи с~решаемой 
задачей стабилизации привода предполагается исследовать этот вопрос 
дополнительно.

%\vspace*{-6pt}
     
{\small\frenchspacing
 {\baselineskip=12pt
 %\addcontentsline{toc}{section}{References}
 \begin{thebibliography}{99}
\bibitem{1-bos}
\Au{Босов А.\,В.} Управление линейным выходом марковской цепи по квадратичному 
критерию~// Информатика и~её применения, 2021. Т.~15. Вып.~2. С.~3--11.
\bibitem{2-bos}
\Au{Elliott R.\,J., Aggoun~L., Moore~J.\,B.} Hidden Markov models: Estimation and control.~--- 
New York, NY, USA: Springer-Verlag, 1995. 382~p.
\bibitem{3-bos}
\Au{Босов А.\,В.} Задача управления линейным выходом нелинейной неуправляемой 
стохастической дифференциальной системы по квадратичному критерию~// Известия 
РАН. Теория и~системы управления, 2021. №\,5. С.~52--73.
\bibitem{4-bos}
\Au{Босов А.\,В.} О~некоторых частных случаях в~задаче управления выходом 
стохастической дифференциальной системы по квадратичному критерию~// Информатика 
и~её применения, 2021. Т.~15. Вып.~1. С.~11--17.

\bibitem{6-bos}
\Au{Флеминг У., Ришел Р.} Оптимальное управление детерминированными 
и~стохастическими системами~/ Пер. с~англ.~--- М.: Мир, 1978. 316~с. 
(\Au{Fleming~W.\,H., Rishel~R.\,W.} Deterministic and stochastic optimal control.~--- New York, 
NY, USA:  Springer-Verlag, 1975. 222~p.)
{\looseness=1

}

\bibitem{5-bos}
\Au{Bertsekas D.\,P.} Dynamic programming and optimal control.~--- 4th ed.~--- Cambridge, 
MA, USA: Athena Scientific, 2017. Vol.~I. 576~p.

\bibitem{7-bos}
\Au{Bene{\!\ptb{\!\v{s}}}~V.} Quadratic approximation by linear systems controlled from partial 
observations~// Stochastic analysis~/ Eds. E.~Mayer-Wolf, E.~Merzbach, A.~Shwartz.~--- 
Boston, MA, USA: Academic Press, 1991. P.~39--50.
\bibitem{8-bos}
\Au{Helmes K., Rishel~R.} The solution of a partially observed stochastic optimal control 
problem in terms of predicted miss~// IEEE T. Automat. Contr., 1992. Vol.~37. No.\,9.  
P.~1462--1464.
\bibitem{9-bos}
\Au{Rishel R.} A~strong separation principle for stochastic control systems driven by a~hidden 
Markov model~// SIAM J.~Control Optim., 1994. Vol.~32. No.\,4.  
P.~1008--1020.
{\looseness=1

}

\bibitem{10-bos}
\Au{Липцер Р.\,Ш., Ширяев А.\,Н.} Статистика случайных процессов (нелинейная 
фильтрация и~смежные вопросы).~--- М.: Наука, 1974. 696~c.
\bibitem{11-bos}
\Au{Липцер Р.\,Ш., Ширяев А.\,Н.} Теория мартингалов.~--- М.: Наука, 1986, 512~с.
\bibitem{12-bos}
\Au{Athans M., Falb~P.\,L.} Optimal control: An introduction to the theory and its 
applications.~--- New York, NY, USA: Dover Publications, 2007. 879~p.
\end{thebibliography}

 }
 }
 

\end{multicols}

\vspace*{-3pt}

\hfill{\small\textit{Поступила в~редакцию 09.12.21}}

\vspace*{8pt}

%\pagebreak

%\newpage

%\vspace*{-28pt}

\hrule

\vspace*{2pt}

\hrule

%\vspace*{-2pt}

\def\tit{LINEAR OUTPUT CONTROL OF~MARKOV CHAIN BY~SQUARE~CRITERION. 
СOMPLETE INFORMATION CASE}


\def\titkol{Linear output control of~Markov chain by~square criterion. 
Сomplete information case}


\def\aut{A.\,V.~Bosov}

\def\autkol{A.\,V.~Bosov}

\titel{\tit}{\aut}{\autkol}{\titkol}

\vspace*{-8pt}


\noindent
Federal Research Center ``Computer Science and Control'' of the Russian Academy of Sciences, 
44-2~Vavilov Str., Moscow 119333, Russian Federation

\def\leftfootline{\small{\textbf{\thepage}
\hfill INFORMATIKA I EE PRIMENENIYA~--- INFORMATICS AND
APPLICATIONS\ \ \ 2022\ \ \ volume~16\ \ \ issue\ 2}
}%
 \def\rightfootline{\small{INFORMATIKA I EE PRIMENENIYA~---
INFORMATICS AND APPLICATIONS\ \ \ 2022\ \ \ volume~16\ \ \ issue\ 2
\hfill \textbf{\thepage}}}

\vspace*{3pt}

      
      
      
      
      
      \Abste{The problem of optimal control of the linear output of 
      a~stochastic differential system, formed by an additive jumping input, was solved. 
      The goal of optimization is set by a quadratic criterion of a~special type which allows 
      one to formalize the tasks of tracking an abruptly changing target and stabilizing the system 
      near the directions determined by the input. The problem is solved under the assumption that 
      there is complete information, i.\,e., the known state of the input Markov chain.
       This statement complements the previously obtained solution of the problem with incomplete 
       information, in which the control and estimation problems are separated, provided by the optimal 
       in this case Wonham filter. The result obtained in the article, in addition to its 
       independent significance, also provides a~reference solution for analyzing the quality of 
       control under conditions of indirect observations. The solution of the problem under consideration, 
       as in the statement with incomplete information, is provided by the direct application of the 
       dynamic programming method. The Bellman's equation is refined for a~given input model~--- 
       a~martingale representation of the chain and a~range of values limited by unit coordinate 
       vectors are used. A~numerical experiment was carried out, the results of which illustrate 
       the efficiency of the obtained 
      control algorithms in both settings, with complete information and indirect observations.}
      
      \KWE{Markov jump process; linear stochastic differential system; 
      optimal control; quadratic criterion; dynamic programming}
      
      
     
      
\DOI{10.14357/19922264220203}

%\vspace*{-16pt}

 \Ack
      \noindent
      The research was carried out using the infrastructure of the Shared Research Facilities ``High 
Performance Computing and Big Data'' (CKP ``Informatics'') of FRC CSC RAS (Moscow).




%\vspace*{4pt}

  \begin{multicols}{2}

\renewcommand{\bibname}{\protect\rmfamily References}
%\renewcommand{\bibname}{\large\protect\rm References}

{\small\frenchspacing
 {%\baselineskip=10.8pt
 \addcontentsline{toc}{section}{References}
 \begin{thebibliography}{99}
      \bibitem{1-bos-1}
      \Aue{Bosov, A.\,V.} 2021. Upravlenie lineynym vykhodom mar\-kov\-skoy tsepi po 
kvadratichnomu kriteriyu [Linear output control of Markov chains by the quadratic criterion]. 
\textit{Informatika i~ee Primeneniya~--- Inform. Appl.} 15(2):3--11.
      \bibitem{2-bos-1}
      \Aue{Elliott, R.\,J., L.~Aggoun, and J.\,B.~Moore.} 1995. \textit{Hidden Markov models: 
Estimation and control.} New York, NY: Springer-Verlag. 382~p.
      \bibitem{3-bos-1}
      \Aue{Bosov, A.\,V.} 2021. The problem of controlling the linear output of a nonlinear 
uncontrollable stochastic differential system by the square criterion. \textit{J.~Comput. Sys. 
Sc. Int.}  60(5):719--739.
      \bibitem{4-bos-1}
      \Aue{Bosov, A.\,V.} 2021. O~nekotorykh chastnykh sluchayakh v~zadache upravleniya 
vykhodom stokhasticheskoy differentsial'noy sistemy po kvadratichnomu kriteriyu [On some special 
cases in the problem of stochastic differential system output control by the quadratic criterion]. 
\textit{Informatika i~ee Primeneniya~--- Inform. Appl.} 15(1):11--17.
    
      \bibitem{6-bos-1}
      \Aue{Fleming, W.\,H., and R.\,W.~Rishel.} 1975. \textit{Deterministic and stochastic optimal 
control}. New York, NY: Springer-Verlag. 222~p.

  \bibitem{5-bos-1}
      \Aue{Bertsekas, D.\,P.} 2017. \textit{Dynamic programming and optimal control}. Cambridge, MA: 
Athena Scientific. Vol.~1. 576~p.

      \bibitem{7-bos-1}
      \Aue{Bene{\!\ptb{\v{s}}}, V.} 1991. Quadratic approximation by linear systems controlled from 
partial observations. \textit{Stochastic analysis}. Eds. E.~Mayer-Wolf, E.~Merzbach, and A.~Shwartz. 
39--50.
      \bibitem{8-bos-1}
      \Aue{Helmes, K., and R.~Rishel.} 1992. The solution of a~partially observed stochastic optimal 
control problem in terms of predicted miss. \textit{IEEE T. Automat. Contr.} 37(9):1462--1464.
      \bibitem{9-bos-1}
      \Aue{Rishel, R.} 1994. A~strong separation principle for stochastic control systems driven by 
a~hidden Markov model. \textit{SIAM J.~Control Optim.} 32(4):1008--1020.
      \bibitem{10-bos-1}
      \Aue{Liptser, R.\,S., and A.\,N.~Shiryaev.} 2001. \textit{Statistics of random processes II. 
Applications}. Berlin: Springer-Verlag. 402~p.
      \bibitem{11-bos-1}
      \Aue{Liptser, R., and A.~Shiryaev.} 1989. \textit{Theory of martingales}. Dortrecht: Springer. 
792~p.
      \bibitem{12-bos-1}
      \Aue{Athans, M., and P.\,L.~Falb.} 2007. \textit{Optimal control: An introduction to the theory 
and its applications}. New York, NY: Dover Publications. 879~p.
 \end{thebibliography}

 }
 }

\end{multicols}

\vspace*{-6pt}

\hfill{\small\textit{Received December 9, 2021}}     


      
      \Contrl
      
      \noindent
      \textbf{Bosov Alexey V.} (b.\ 1969)~--- Doctor of Science in technology, principal scientist, 
Institute of Informatics Problems, Federal Research Center ``Computer Science and Control'' of the 
Russian Academy of Sciences, 44-2~Vavilov Str., Moscow 119333, Russian Federation; 
\mbox{AVBosov@ipiran.ru}
      
  

\label{end\stat}

\renewcommand{\bibname}{\protect\rm Литература}    
      