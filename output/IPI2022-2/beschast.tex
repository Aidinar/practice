\def\stat{beschastnyy}

\def\tit{АНАЛИЗ ПЛОТНОСТИ БАЗОВЫХ СТАНЦИЙ 5G~NR ДЛЯ~ПРЕДОСТАВЛЕНИЯ УСЛУГ 
ВИРТУАЛЬНОЙ И~ДОПОЛНЕННОЙ РЕАЛЬНОСТИ$^*$}

\def\titkol{Анализ плотности базовых станций 5G NR для предоставления услуг 
виртуальной и~дополненной реальности}

\def\aut{В.\,А.~Бесчастный$^1$, Д.\,Ю.~Острикова$^2$, С.\,Я.~Шоргин$^3$, 
Д.\,А.~Молчанов$^4$, Ю.\,В.~Гайдамака$^5$}

\def\autkol{В.\,А.~Бесчастный, Д.\,Ю.~Острикова, С.\,Я.~Шоргин и~др.}
%$^3$,  Д.\,А.~Молчанов$^4$, Ю.\,В.~Гайдамака$^5$}

\titel{\tit}{\aut}{\autkol}{\titkol}

\index{Бесчастный В.\,А.}
\index{Острикова Д.\,Ю.}
\index{Шоргин С.\,Я.}
\index{Молчанов Д.\,А.}
\index{Гайдамака Ю.\,В.}
\index{Beschastnyi V.\,A.}
\index{Ostrikova D.\,Yu.}
\index{Shorgin S.\,Ya.}
\index{Moltchanov D.\,A.}
\index{Gaidamaka Yu.\,V.}


{\renewcommand{\thefootnote}{\fnsymbol{footnote}} \footnotetext[1]
{Публикация выполнена при финансовой поддержке РНФ (проект 22-29-00694).}}


\renewcommand{\thefootnote}{\arabic{footnote}}
\footnotetext[1]{Российский университет дружбы народов, beschastnyy-va@rudn.ru}
\footnotetext[2]{Российский университет дружбы народов, ostrikova-dyu@rudn.ru}
\footnotetext[3]{Федеральный исследовательский центр <<Информатика и~управление>> Российской академии наук, 
\mbox{sshorgin@ipiran.ru}}
\footnotetext[4]{  Университет Тампере, Финляндия, dmitri.moltchanov@tuni.fi}
\footnotetext[5]{Российский университет дружбы народов; Федеральный исследовательский центр <<Информатика 
и~управление>> Российской академии наук, \mbox{gaydamaka-yuv@rudn.ru}}

\vspace*{-6pt}

  
  

  

  \Abst{Технология пятого поколения <<новое радио>> (5G New Radio, 5G NR), 
работающая в~диапазоне частот миллиметрового диапазона (mmWave), разработана для 
поддержки ресурсоемких приложений, требующих чрезвычайно высоких скоростей на 
уровне радиоинтерфейса. В~сис\-те\-мах NR использование антенных решеток, 
формирующих особые узкие диаграммы направленности излучения, позволяет избегать 
высоких потерь и~помех при передаче сигнала, но в~то же время сокращает площадь 
покрытия отдельно взятого луча, а~следовательно, и~число многоадресных пользователей, 
которые могут быть обслужены с~его помощью. В~результате требуются эффективные 
алгоритмы доставки данных для поддержки таких услуг как в~традиционных сетях 5G NR, так и~в~сетях
 на базе беспилотных летательных аппаратов (БПЛА). В~работе рассматривается 
передача потоковых данных для услуг виртуальной ре\-аль\-ности с~использованием технологии 
масштабируемого видеокодирования, которая использует возможности многоадресной 
передачи для предоставления базового слоя с~низким качеством разрешения получаемого 
контента и~одноадресной передачи для предоставления дополнительных слоев 
с~повышенным качеством. С~использованием аппарата стохастической геометрии и~теории 
массового обслуживания разработан метод, позволяющий оценить минимальную плотность 
развертывания базовых станций (БС) mmWave NR для обеспечения заданной 
производительности многослойных услуг с~многоадресной передачей в~зависимости от их 
различных требований и~структуры, а также от плотности расположения абонентских 
терминалов.}
   
  \KW{5G; <<новое радио>>; миллиметровый диапазон; виртуальная реальность; 
многоадресные соединения; масштабируемое видеокодирование; кластеризация}

\DOI{10.14357/19922264220213}
  
\vspace*{-3pt}


\vskip 10pt plus 9pt minus 6pt

\thispagestyle{headings}

\begin{multicols}{2}

\label{st\stat}
  
\section{Введение}
  
  В настоящее время консорциум 3GPP уже завершил основные этапы 
стандартизации технологии NR в~версиих~15 и~~16~[1]. Такие сис\-те\-мы, 
ра\-бо\-та\-ющие как в~микроволновом, так и~в~миллиметровом диапазонах, 
обещают обеспечить чрезвычайно высокие скорости передачи данных на 
уровне радиоинтерфейса~[2]. На текущий момент внимание как 3GPP, так 
и~исследовательского сообщества, сосредоточено на вопросе предоставления 
дополнительных услуг поверх нового уникального интерфейса радиодоступа 
с~возможностью многоадресной передачи~[3]. 
  
  В работе рассматривается многослойная услуга виртуальной реальности 
(Virtual Reality, VR), где базовый слой, передача которого предполагается 
многоадресной, обеспечивает самое низкое качество воспроизведения видео, 
а~каждый новый слой содержит дополнительные данные для повышения качества 
воспроизведения по технологии масштабируемого видеокодирования. По 
запросу нового пользователя VR-услу\-га должна быть предостав\-ле\-на с~базовым 
уровнем качества, а затем качество восприятия (Quality of Experience, QoE) 
можно улучшить, добавив дополнительные слои~[4]. Такая возможность 
зависит от полосы пропускания и~зоны покрытия. Дополнительный слой может 
содержать альтернативный контент или данные для улучшения качества 
текущего воспроизведения, его передача моделируются с~помощью 
одноадресной сессии с~собственным фиксированным требованием к~ресурсу. 
  
  Большинство проведенных на данный момент исследований для 
многослойных мно\-го\-ад\-рес\-ных/од\-но\-ад\-рес\-ных услуг сосредоточены 
на оптимизации уже развернутой системы для заданных параметров качества 
обслуживания~[5--7]. Однако для сетевых операторов не менее важен вопрос 
оценки требуемой плотности БС NR для заданной стохастической нагрузки 
трафика в~заданной об\-ласти. Помимо наземных сис\-тем плот\-ность 
расположения точек доступа имеет еще большее значение для сис\-тем, 
использующих БПЛА~[8]. В~этом случае БПЛА могут выступать как в~качестве 
потребителя услуги, так и~в~качестве поставщика, предоставляя услугу 
некоторой группе пользователей. Именно на решение этих проб\-лем направлена 
данная работа, где с~использованием аппарата стохастической 
геометрии и~тео\-рии массового обслуживания разработана математическая модель для 
расчета доли многослойных многоадресных VR-сес\-сий, которые могут быть 
обслужены в~зависимости от параметров точки доступа и~нагрузки на систему.

\vspace*{-6pt}

\section{Системная модель}

\vspace*{-2pt}
  
  Рассматривается зона покрытия БС NR в~виде сектора радиуса~$R$, который 
рассчитывается в~соответствии с~моделью распространения сигнала 
в~миллиметровом диапазоне~[9]. Сота обслуживается тремя антеннами, каждая 
из которых покрывает сектор с~центральным углом~120$^\circ$. Высота БС 
фиксирована и~равна~$h_A$, пользовательское устройство (ПУ) находится на 
высоте~$h_U$. Система функционирует на рабочей частоте~$f_c$, при этом 
каждая БС имеет в~своем распоряжении~$B$~ресурсных блоков. Предположим 
наличие в~соте потенциальных блокаторов сигнала~--- людей, распределенных 
случайным образом в~соответствии с~пуассоновским точечным процессом 
(ПТП) в~$\mathrm{Re}^2$ с~плот\-ностью~$\lambda_B$, блокирующих своим телом пути 
распространения  сигнала между БС и~ПУ. Блокаторы моделируются как 
цилиндры радиусом~$r_B$ и~высотой $h_B\hm > h_U$. 
  
  В работе рассматривается VR-услу\-га с~четырьмя уровнями качества 
(слоями): одним базовым и~тремя дополнительными. Обозначим требуемую 
скорость передачи данных базового слоя~$d_M$. Этот слой предоставляется 
всем пользователям услуги. В~то же время все пользователи пытаются 
повысить качество услуги и~получить дополнительные слои данных 
с~требованиями $d_{U,l}$, $l\hm=1,2,3$. Предполагается, что процесс 
поступления пользовательских запросов является пуассоновским  
с~па\-ра\-мет\-ром~$\Lambda$, а~длительности VR-сес\-сий имеют 
экспоненциальное распределение с~па\-ра\-мет\-ром~$\mu$.
 
 При поступлении запроса на предоставление услуги от ПУ из зоны, не 
покрытой многоадресной сессией, организуется новая многоадресная сессия, 
и~пользователь всегда получает базовый слой данных. Если же ПУ находится 
в~зоне действия уже установленной многоадресной сессии, оно присоединяется 
к~ней, не требуя дополнительных ресурсов. Далее, если на БС достаточно 
ресурсов для загрузки дополнительных слоев, инициируется их передача.

\vspace*{-6pt}

\section{Математическая модель}

\vspace*{-4pt}
  
  В данном разделе приводится аналитический метод группировки 
пользователей многоадресной сессии, после чего формулируется задача 
доставки дополнительных слоев VR-услу\-ги в~виде системы массового 
обслуживания. Посредством итеративного увеличения радиуса 
покрытия БС такой подход позволяет рассчитывать необходимую плот\-ность 
развертывания.
  
  Формирование групп многоадресных сессий основано на принципе выбора 
максимальной ширины по уровню половинной мощности~$\alpha$, 
позволяющей установить соединение с~ПУ, находящимся на границе 
обслуживаемой соты. Для определения значения~$\alpha$ сначала необходимо 
найти подходящее усиление на антенне БС, которое впоследствии может быть 
скорректировано в~меньшую сторону с~учетом доступных конфигураций 
антенной решетки:
  \begin{equation}
  G_A= \fr{S(R)[N_0+M_T]}{P_A G_U R^{-\zeta_T} e^{-KR} p_B(R)}\,.
  \label{e1-bes}
  \end{equation}
  
  Получив значение~$G_A$, можно рассчитать число многоадресных групп, 
необходимое для покрытия всей обслуживаемой зоны 
$$
N= \left\lceil \fr{\Theta}{G_A}\right\rceil,
$$
 где $\Theta$~--- ширина дуги сектора антенны. Для того 
чтобы воспользоваться моделью Хеллмана~[10], необходимо перевести ширину 
угла~$\alpha$ в~длину дуги образуемого им сегмента:
$$
\xi= \fr{G_A\pi R}{180}\,.
$$ 
Теперь, согласно~[11], длины пробелов между парами соседних теней имеют 
экспоненциальное распределение с~параметром $\lambda K^2/(2R)$. Это 
позволяет выразить вероятность того, что пробел имеет ширину от~$k$ 
до~$k\hm+1$ сегментов, в~виде
  \begin{equation}
  q_k= e^{-\lambda K^2 k\xi/(2R)} - e^{-\lambda K^2(k+1)\xi/(2R)}\,,
  \label{e2-bes}
  \end{equation}
а~потому не требует покрытия лучами. В~конечном итоге это позволяет 
оценить среднее число лучей многоадресных сессий, попадающих в~пробелы, 
как
\begin{equation}
\Delta = \fr{\pi \lambda K^2}{3\left(e^{\lambda B(R-
Q)+\lambda}\right)}\sum\limits^N_{k=1} kq_k\,.
\label{e3-bes}
\end{equation}
  
  Таким образом, среднее число многоадресных сессий в~соте можно найти как 
$N\hm- \Delta$. В~то же время требование отдельной многоадресной сессии 
зависит от того, все ли ПУ группы находятся в~условиях прямой видимости 
или есть хотя бы одно заблокированное ПУ, которое вынуждает всю группу 
снижать схему модуляции и~кодирования~[12]. Чтобы рассчитать средний 
объем ресурса, тре\-бу\-емый для обслуживания многоадресной сессии, необходимо найти 
вероятность вхождения некоторого чис\-ла ПУ~$u$ в~группу, которая имеет 
пуассоновское распределение, по свойству ПТП:
  \begin{equation}
  q_u= \fr{e^{-\lambda_n}\lambda_n^u}{u!}\,,
  \label{e4-bes}
  \end{equation}
где $\lambda_n= \lambda\pi (R^2\hm- Q^2)\alpha/360$.

  Следовательно, средний объем требуемого ресурса можно рассчитать как
  \begin{multline}
  b_M= \sum\limits^\infty_{u=1} q_u\left[ \prod\limits^u_{i=1} \left(1-
p_{B,i}\right) b_{M,L} +{}\right.\\
\left.{}+\left( 1-\prod\limits^u_{i=1} \left(1-p_{B,i}\right)\right) 
b_{M,B}\right],
  \label{e5-bes}
  \end{multline}
где $b_{M,L}$ и~$b_{M,B}$~--- требования к~объему ресурса в~условиях 
прямой видимости и~в~состоянии блокировки соответственно;  $p_{B,i}$~--- 
вероятность блокировки ПУ, которое является $i$-м соседом для БС в~ПТП.

  Как показано в~[13], вероятность блокировки прямой видимости зависит как 
от расстояния между ПУ и~БС, которое имеет распределение расстояния до  
$i$-го ближайшего соседа в~ПТП с~плот\-ностью
  \begin{multline}
  f_i(x)=\fr{2(\pi\lambda)^i}{(i-1)!}\,x^{2i-1} e^{-\pi\lambda x^2}\,,\\
   x>0\,,\enskip  i=1,2,\ldots,
  \label{e5-1.bes}
  \end{multline}
так и~от интенсивности блокаторов, пересекающих зону блокировки
\begin{equation}
\mu_{B,i}= \int\limits_0^\infty\! f_i(x) \fr{(x[h_B -h_U] +r_B [h_T-h_U])} 
{(2r_B\lambda_B v)^{-1} (h_T- h_U)}\,dx\,,\!
\label{e6-bes}
\end{equation}
где $h_T$, $h_U$ и~$h_B$~--- высоты БС, ПУ и~блокаторов соответственно; 
$r_B$ и~$v$~--- радиус и~скорость блокаторов. Это позволяет найти 
вероятность блокировки ПУ в~виде
\begin{equation}
p_{B,i}= \fr{\mu_{B,i}}{\mu_{B,i}+v/(2r_B)}\,,
\label{e7-bes}
\end{equation}
где $2r_B/v$~--- среднее время прохождения блокатором зоны блокировки под 
прямым углом к~ее длинной стороне.
  
  Для оценки среднего объема ресурсов, необходимых для доставки 
дополнительных слоев видео с~по\-мощью одноадресных сессий, 
рассматриваются ресурсы, позволяющие загружать каждый слой некоторой 
доле пользователей. Процесс загрузки дополнительных слоев моделируется 
в~виде СМО~[14] 
$$
\begin{matrix} M \\ \lambda\end{matrix} \left\vert \begin{matrix} M 
\\ \mu\end{matrix}\right\vert C,
$$ 
где $\lambda$~--- интенсивность 
поступления запросов на загрузку слоя видео; $\mu^{-1}$~--- средняя 
длительность загружаемого фрагмента видео, которая имеет экспоненциальное 
распределение;  $C$~--- чис\-ло активных одноадресных сессий, которые 
необходимо поддерживать для выполнения определенных требований по 
вероятности успешной загрузки. Для расчета вероятности сброса сессии 
$E_C(\rho)$ можно воспользоваться первой формулой Эрланга
  \begin{equation*}
  E_C(\rho)=\fr{\rho^C/C!}{\sum\nolimits^C_{m=0} (\rho^m/m!)}\,,\enskip 
0\leq\rho < \infty\,,
 % \label{e8-bes}
  \end{equation*}
где $\rho=\lambda/\mu$. 
  
  Теперь можно оценить средний объем тре\-бу\-емых ресурсов для загрузки 
дополнительного $l$-го слоя как
  \begin{equation*}
  E_l[U]=C_l \left[ p_{B,i} b_{U_l,B} +\left( 1-p_{B,i}\right) b_{U_l,L}\right],
 % \label{e9-bes}
  \end{equation*}
где $b_{U_l,B}$ и~$b_{U_l,L}$~--- требования отдельно взятого ПУ для 
загрузки $l$-го слоя в~условиях заблокированной и~незаблокированной прямой 
видимости. 
  
  Зададим вектор $\mathbf{p}_U$ с~элементами $p_l$~---  вероятностями 
успешной загрузки $l$-го слоя видео, $l\hm=1,\ldots , L$. Тогда в~общем виде 
схема нахождения минимальной требуемой плот\-ности развертывания БС 
выглядит следующим образом: 
\begin{itemize}
\item[(а)] для максимально большого допустимого 
радиуса соты вы\-чис\-лить объем ресурса для доставки базового слоя, используя 
выражения~(1)--(\ref{e7-bes}); 
\item[(б)] для каждого дополнительного слоя найти 
такое минимальное~$C_l$, при котором будет выполняться условие по 
ве\-ро\-ят\-ности успешной загрузки~$p_l$; 
\item[(в)] сложить рассчитанные требования на 
базовый и~дополнительные слои и~сравнить с~объемом ресурсов на БС; 
\item[(г)] если сумма требований меньше доступного ресурса, то повторить расчет для 
меньшего радиуса соты, иначе принять предпоследнее значение радиуса за 
минимально до\-пус\-ти\-мое.
\end{itemize}

\section{Численный анализ}
  
  В данном разделе проводится численный анализ влияния системных 
параметров, представленных в~таблице, на минимальную допустимую 
плотность развертывания БС NR.
  
  \begin{table*}[b]\small
  \begin{center}
  \begin{tabular}{|c|l|c|}
  \multicolumn{3}{c}{Системные параметры}\\
  \multicolumn{3}{c}{\ }\\[-6pt]
  \hline
Обозначение&\multicolumn{1}{c|}{Описание}&Значения по умолчанию\\
\hline
$f_C$ &Рабочая частота&73 ГГц\\
$B$&Число доступных ресурсных блоков&264\\
$r_B$&Радиус блокатора&0,4 м\\
$h_B$&Высота блокатора&1,7 м\\
$h_U$&Высота ПУ&1,5 м\\
$h_T$&Высота БС&4 м\\
$v$&Скорость блокатора&1 м/с\\
$P_T$&Излучаемая мощность на БС&2 Вт\\
$N_U$&Число конфигураций антенны ПУ&$8\times8$\\
$\lambda_B$&Плотность блокаторов&0,3 ед./м$^2$\\
$N_0$&Шум&$-$84~дБ\\
$\zeta_T$&Коэффициент затухания&2,1\\
$\lambda$&Интенсивность запросов на видеосессии от ПУ&1/3600 сессий/с\\
$\mu^{-1}$&Средняя длительность видеосессии&15 с\\
$d_M$&Требуемая скорость для загрузки базового слоя&7,78 Мбит/с\\
$\mathbf{d}_U$&Требуемые скорости для загрузки дополнительных слоев&19{,}78; 
25{,}81; 31{,}96~Мбит/с\\
\hline
\end{tabular}
\end{center}
%\end{table*}
\setcounter{table}{0}
\renewcommand{\tablename}{\protect\bf Рис.}
%\begin{figure*} %fig1
\vspace*{7pt}
  \begin{center}  
    \mbox{%
\epsfxsize=163mm
\epsfbox{bes-1.eps}
}
\end{center}
\vspace*{-9pt}
\Caption{Оптимальная плотность развертывания NR БС в~зависимости от плотности 
блокаторов~(\textit{а}) и~пользовательских устройств~(\textit{б}): \textit{1}~--- 
$\mathbf{p}_U\hm= (0{,}9; 0{,}8; 0{,}7)$;  \textit{2}~--- $(0{,}75; 0{,}5; 0{,}25)$; \textit{3}~--- 
$\mathbf{p}_U\hm= (0{,}5; 0{,}3; 0{,}1)$}
  %\end{figure*}
  \end{table*}
  
  \renewcommand{\figurename}{\protect\bf Рис.}
\renewcommand{\tablename}{\protect\bf Таблица}
  
  
  В данной работе рассматриваются три профиля качества обслуживания, 
характеризующися вектором~$\mathbf{p}_U$: строгий $(0{,}9; 0{,}8; 0{,}7)$, 
средний $(0{,}75; 0{,}5; 0{,}25)$ и~мягкий $(0{,}5; 0{,}3; 0{,}1)$.
  
  Одним из ключевых параметров, оказывающих влияние на 
производительность системы, является плот\-ность блокаторов. 

На 
рис.~1,\,\textit{a} представлен график зависимости оптимальной плотности 
развертывавния БС от~$\lambda_B$, на котором можно заметить, что мягкий 
и~средний профили не сильно подвержены влиянию блокаторов, в~отличие от 
пользователей со строгим профилем, для которого при $\lambda_B\hm= 
1{,}0$~ед./м$^2$ требуется 110~БС на~1~км$^2$. Для менее строгих профилей 
требуемая плотность развертывания при этом практически вдвое меньше.
  

  На рис.~1,\,\textit{б} изображена зависимость оптимальной плотности 
развертывавния БС от плотности ПУ. Очевидно, что чем более строгий профиль 
у~пользователей, тем большую нагрузку они создают на сеть в~целом и~тем 
большая плотность развертывания требуется для обеспечения эффективного 
покрытия. Здесь стоит отметить, что рассматриваемая зависимость для всех 
профилей имеет практически линейный характер.

\setcounter{figure}{1}
\begin{figure*} %fig2
\vspace*{1pt}
  \begin{center}  
    \mbox{%
\epsfxsize=163mm
\epsfbox{bes-2.eps}
}
\end{center}
\vspace*{-9pt}
\Caption{Оптимальная плотность развертывания БС NR для разных режимов передачи 
базового слоя~(\textit{а}) 
(\textit{1}~---  $\mathbf{p}_U\hm= (0{,}9; 0{,}8; 0{,}7)$;  \textit{2}~--- 
$(0{,}75; 0{,}5; 0{,}25)$;  \textit{3}~--- $\mathbf{p}_U\hm= 
(0{,}5; 0{,}3; 0{,}1)$; пустые значки~--- многоадресные сессии; залитые значки~--- одноадресные сессии)
    и~отношения долей одноадресных и~многоадресных сессий~(\textit{б}) (\textit{1}~--- 
$\mathbf{p}_U\hm= (0{,}9; 0{,}8; 0{,}7)$;  \textit{2}~--- $(0{,}75; 0{,}5; 0{,}25)$; \textit{3}~--- 
$\mathbf{p}_U\hm= (0{,}5; 0{,}3; 0{,}1)$)}
  \end{figure*}
  
  На рис.~2,\,\textit{а} показа зависимость плотности развертывания от все той 
же плотности ПУ, однако здесь присутствуют две схемы доставки базового 
слоя: с~по\-мощью многоадресных сессий и~с помощью одноадресных. 
Сравнение схем показывает, что использование многоадресных сессий при 
низких плотностях ПУ малоэффективно, но для плотных сетей (при $\sigma\hm 
\geq 0{,}4$~ед./м$^2$) они позволяют добиться существенного выигрыша за 
счет переиспользования ресурсов для базового слоя.
  
  Для рис.~2,\,\textit{б} введен дополнительный коэффициент~$L_R$, который 
обозначает отношение числа одноадресных сессий к~многоадресным. 
Естественным образом с~ростом числа одноадресных сессий возрастает 
необходимая плотность БС. Однако в~то же время возрастает и~разрыв между 
значениями для разных профилей, что объясняется резким повышением 
требований к~дополнительным слоям видео.
  

\vspace*{-6pt}

\section{Заключение}

\vspace*{-4pt}
  
  В данной работе предложен метод оценки производительности NR-систем 
при предоставлении услуги масштабируемого VR-видео посредством 
одноадресных и~многоадресных сессий. Численный анализ показал, что при 
низкой плотности пользователей использование многоадресных сессий 
малоэффективно, а наибольший позитивный эффект от их применения 
наблюдается тогда, когда требования к~ресурсам для доставки базового слоя 
начинают превосходить требования для дополнительных слоев видео. Также 
показано, что параметры качества обслуживания наряду с~плотностью 
пользователей имеют значительное влияние на необходимую плотность 
развертывания БС. При этом наиболее сильное влияние 
плотности блокаторов наблюдается в~случае наиболее высоких требований 
к~качеству обслуживания. В~целом, в~за\-ви\-си\-мости от различных системных 
па\-ра\-мет\-ров, плот\-ность раз\-вер\-ты\-ва\-ния варь\-и\-ру\-ет\-ся от~20 до~250~БС на~1~км$^2$. 
  
  Цель дальнейших исследований~--- расширение предложенной модели для 
сценария с~использованием БПЛА в~качестве подвижных точек доступа, что 
позволит исследовать проблему кластеризации роев.
  
{\small\frenchspacing
 {%\baselineskip=10.8pt
 %\addcontentsline{toc}{section}{References}
 \begin{thebibliography}{99}
\bibitem{1-bes}
\Au{Holma H., Toskala~A., Nakamura~T.} 5G technology: 3GPP New Radio.~--- New York, NY, 
USA:  John Wiley \& Sons, 2020. 536~p.
\bibitem{2-bes}
\Au{Lin X., Li~J., Baldemair~R., \textit{et al.}} 5G New Radio: Unveiling the essentials of the next 
generation wireless access technology~// IEEE Communications Standards Magazine, 2019. 
Vol.~3. Iss.~3. P.~30--37. doi: 10.1109/ \mbox{MCOMSTD}.001.1800036.
\bibitem{3-bes}
\Au{Le T.\,K., Salim~U., Kaltenberger~F.} An overview of physical layer design for ultra-reliable 
low-latency communications in 3GPP Releases 15, 16, and 17~// IEEE Access, 2020. Vol.~9. 
P.~433--444. doi: 10.1109/\mbox{ACCESS}. 2020.3046773.
\bibitem{4-bes}
\Au{Karembai A.\,K., Thompson~J., Seeling~P.} Towards prediction of immersive virtual reality 
image quality of experience and quality of service~// Future Internet, 2018. Vol.~10. Iss.~7. 
Art.~63. 12~p. doi: 10.3390/fi10070063.
\bibitem{5-bes}
\Au{Nasrabadi A.\,T., Mahzari~A., Beshay~J.\,D., Prakash~R.} Adaptive 360-degree video 
streaming using layered video coding~//  Virtual Reality Conference Proceedings.~--- 
Pis-\linebreak\vspace*{-12pt}
 
 \pagebreak
 
 \noindent
 cataway, NJ, USA: IEEE, 2017. P.~347--348. doi: 10.1109/ VR.2017.7892319.

\bibitem{7-bes} %6
\Au{Park J., Hwang J., Wei~H.} Cross-layer optimization for VR Video Multicast Systems~// 
 Global Communications Conference Proceedings.~--- Piscataway, NJ, USA: 
IEEE, 2018. P.~206--212. doi: 10.1109/GLOCOM.2018.8647389.

\bibitem{6-bes} %7
\Au{Long K., Cui~Y., Ye~C., Liu~Z.} Optimal transmission of multi-quality tiled 360 VR video by 
exploiting multicast opportunities~// Global Communications Conference 
 Proceedings.~--- Piscataway, NJ, USA: IEEE, 2019. Art.~9014280. 6~p. doi: 
10.1109/\linebreak GLOBECOM38437.2019.9014280.
\bibitem{8-bes}
\Au{Tang N., Tang H., Li~B. Yuan~X.} Joint maneuver and beamwidth optimization for  
UAV-enabled multicasting~// IEEE Access, 2019. Vol.~7. P.~149503--149514. doi: 
10.1109/ACCESS.2019.2947031.
\bibitem{9-bes}
3GPP Technical Specification 38.211: Physical channels and modulation (Release~16), 2021. 
{\sf https://www. 3gpp.org/ftp/Specs/archive/38\_series/38.211/38211-g50.zip}.
\bibitem{10-bes}
\Au{Хеллман О.} Введение в~теорию оптимального поиска~/ Пер. с~англ. 
Е.\,М.~Столяровой.~--- М.: Наука, 1985. 246~с.
\bibitem{11-bes}
\Au{Gapeyenko M., Samuylov~A., Gerasimenko~M., Moltchanov~D., Singh~S., Akdeniz~M.\,R., 
Aryafar~E., Himayat~N., Andreev~S., Koucheryavy~Y.} On the temporal effects of mobile blockers 
in urban millimeter-wave cellular scenarios~// IEEE T. Veh. Technol., 2017. 
Vol.~66. No.\,11. P.~10124--10138. doi: 10.1109/TVT.2017.2754543.
\bibitem{12-bes}
\Au{Samuylov A., Beschastnyi~V., Moltchanov~D., Ostrikova~D., Gaidamaka~Y., Shorgin~V.} 
Modeling coexistence of unicast and multicast communications in 5G New Radio systems~// 
30th Annual  Symposium (International) on Personal, Indoor and Mobile Radio 
Communications  Proceedings.~--- Piscataway, NJ, USA: IEEE, 2019. Art.~8904350. 
6~p. doi: 10.1109/PIMRC.2019.8904350.
\bibitem{13-bes}
\Au{Begishev V., Moltchanov D., Sopin~E., Samuylov~A., Andreev~S., Koucheryavy~Y., 
Samouylov~K.} Quantifying the impact of guard capacity on session continuity in 3GPP new radio 
systems~// IEEE T. Veh. Technol., 2019. Vol.~68. No.\,12. P.~12345--12359. 
doi: 10.1109/TVT.2019.2948702.
\bibitem{14-bes}
\Au{Горбунова А.\,В., Наумов~В.\,А., Гайдамака~Ю.\,В., Самуйлов~К.\,Е.} Ресурсные системы массового обслуживания как модели беспроводных систем 
связи~// Информатика и~её применения, 2018. 
Т.~12. Вып.~3. С.~48--55. doi: 10.14357/19922264180307.
 \end{thebibliography}

 }
 }

\end{multicols}

\vspace*{-6pt}

\hfill{\small\textit{Поступила в~редакцию 30.01.22}}

\vspace*{8pt}

%\pagebreak

%\newpage

%\vspace*{-28pt}

\hrule

\vspace*{2pt}

\hrule

%\vspace*{-2pt}

\def\tit{DENSITY ANALYSIS OF~mmWave NR DEPLOYMENTS FOR~DELIVERING SCALABLE 
AR/VR VIDEO SERVICES}


\def\titkol{Density analysis of~mmWave NR deployments for~delivering scalable 
AR/VR video services}


\def\aut{V.\,A.~Beschastnyi$^1$, D.\,Yu.~Ostrikova$^1$, S.\,Ya.~Shorgin$^2$, D.\,A.~Moltchanov$^3$,\\ 
and~Yu.\,V.~Gaidamaka$^{1,2}$}

\def\autkol{V.\,A.~Beschastnyi, D.\,Yu.~Ostrikova, S.\,Ya.~Shorgin, et al.}
%D.\,A.~Moltchanov$^3$,  and~Yu.\,V.~Gaidamaka$^{1,2}$}

\titel{\tit}{\aut}{\autkol}{\titkol}

\vspace*{-10pt}


 \noindent
  $^1$Peoples' Friendship University of Russia (RUDN University), 6~Miklukho-Maklaya Str., 
Moscow 117198, Russian\linebreak
$\hphantom{^1}$Federation
  
  \noindent
  $^2$Federal Research Center ``Computer Science and Control'' of the Russian Academy of 
Sciences, 44-2~Vavilov\linebreak
$\hphantom{^1}$Str., Moscow 119133, Russian Federation
  
  \noindent
  $^3$Tampere University, 7~Korkeakoulunkatu, Tampere 33720, Finland

\def\leftfootline{\small{\textbf{\thepage}
\hfill INFORMATIKA I EE PRIMENENIYA~--- INFORMATICS AND
APPLICATIONS\ \ \ 2022\ \ \ volume~16\ \ \ issue\ 2}
}%
 \def\rightfootline{\small{INFORMATIKA I EE PRIMENENIYA~---
INFORMATICS AND APPLICATIONS\ \ \ 2022\ \ \ volume~16\ \ \ issue\ 2
\hfill \textbf{\thepage}}}

\vspace*{6pt}  
  
  
  
  
  \Abste{The 5G New Radio (NR) technology operating in millimeter-wave 
  (mmWave) frequency band is designed to support bandwidth-greedy applications requiring 
  extraordinary rates at the access interface. In NR systems, the use of antenna arrays 
  that form directional radiation patterns allows to avoid high propagation losses 
  and interference but at the same time reduces the coverage area of a~single beam and, 
  hence, the number of multicast users that can be served by the beam. As a~result, 
  efficient algorithms are required to support such services in both terrestrial systems 
  and drone-assisted systems that utilize unmanned aerial vehicles as access points.
The present authors consider the streaming data delivery of virtual reality 
   services using scalable video coding technology which utilizes multicast capabilities 
   for baseline layer and unicast transmissions for delivering an enhanced experience. 
   By utilizing the tools of stochastic geometry and queuing theory, the authors develop 
   a~simple method allowing one to estimate the deployment density of mmWave NR base stations  
   to provide a~given performance of multilayer multicast services depending on their various 
   requirements and structure as well as on the density of users.}
  
  \KWE{5G; New Radio; mmWave; multi-layer VR; multicasting; scalable video coding; clustering}
  

  
\DOI{10.14357/19922264220213}

\vspace*{-8pt}

  \Ack
  \noindent
  The publication has been funded by the Russian Science Foundation, project 22-29-00694. 

%\vspace*{4pt}

  \begin{multicols}{2}

\renewcommand{\bibname}{\protect\rmfamily References}
%\renewcommand{\bibname}{\large\protect\rm References}

{\small\frenchspacing
 {%\baselineskip=10.8pt
 \addcontentsline{toc}{section}{References}
 \begin{thebibliography}{99}
\bibitem{1-bes-1}
  \Aue{Holma, H., A.~Toskala, and T.~Nakamura.} 2020. \textit{5G technology: 3GPP New 
Radio}. New York, NY:  John Wiley \& Sons. 536~p.
\bibitem{2-bes-1}
  \Aue{Lin, X., J.~Li, R.~Baldemair, %J.\,F.\,T.~Cheng, S.~Parkvall, D.\,C.~Larsson, 
%H.~Koorapaty, M.~Frenne, S.~Falahati, A.~Grovlen, 
\textit{et al.}} 2019. 5G New Radio: Unveiling 
the essentials of the next generation wireless access technology. \textit{IEEE Communications 
Standards Magazine} 3(3):30--37. doi: 10.1109/MCOMSTD.001.1800036.
\bibitem{3-bes-1}
  \Aue{Le, T.\,K., U.~Salim, and F.~Kaltenberger.} 2020. An overview of physical layer design 
for ultra-reliable low-latency communications in 3GPP Releases~15, 16, and 17. \textit{IEEE 
Access} 9:433--444. doi: 10.1109/ACCESS.2020.3046773.
\bibitem{4-bes-1}
  \Aue{Karembai, A.\,K., J.~Thompson, and P.~Seeling.} 2018. Towards prediction of immersive 
virtual reality image quality of experience and quality of service. \textit{Future Internet} 10(7):63. 
12~p. doi: 10.3390/fi10070063.
\bibitem{5-bes-1}
  \Aue{Nasrabadi, A.\,T., A.~Mahzari, J.\,D.~Beshay, and R.~Prakash.} 2017. Adaptive  
360-degree video streaming using layered video coding. \textit{Virtual Reality Conference 
Proceedings}. Piscataway, NJ:  IEEE. 347--348. doi: 10.1109/VR.2017.7892319.

\bibitem{7-bes-1} %6
  \Aue{Park, J., J. Hwang, and H.~Wei.} 2018. Cross-layer optimization for VR video multicast 
systems. \textit{Global Communications Conference Proceedings}.  Piscataway, NJ: IEEE. 206--212. doi: 
10.1109/GLOCOM.2018.8647389.

\bibitem{6-bes-1} %7
  \Aue{Long, K., Y.~Cui, C.~Ye, and Z.~Liu.} 2019. Optimal transmission of multi-quality tiled 
360~VR video by exploiting multicast opportunities. \textit{Global Communications Conference 
Proceedings}. Piscataway, NJ: IEEE. Art.~9014280. 6~p. doi: 10.1109/\linebreak GLOBECOM38437.2019.9014280.
\bibitem{8-bes-1}
  \Aue{Tang N., H.~Tang, B.~Li, and X.~Yuan.} 2019. Joint maneuver and beamwidth 
optimization for UAV-enabled multicasting. \textit{IEEE Access} 7:149503--149514. doi: 
10.1109/\linebreak ACCESS.2019.2947031.
\bibitem{9-bes-1}
  3GPP Technical Specification 38.211: Physical channels and modulation. 2021. Available at: {\sf 
https://www.3gpp. org/ftp/Specs/archive/38\_series/38.211/38211-g90.zip} (accessed April~20, 
2022).
\bibitem{10-bes-1}
  \Aue{Hellman, O.} 1985. \textit{Vvedenie v~teoriyu optimal'nogo poiska} [Introduction to the 
optimal search theory]. Moscow: Nauka. 246~p.
\bibitem{11-bes-1}
  \Aue{Gapeyenko, M., A.~Samuylov, M.~Gerasimenko, D.~Mol\-tcha\-nov, S.~Singh,  
M.\,R.~Akdeniz, E.~Aryafar, N.~Hi\-ma\-yat, S.~Andreev, and Y.~Koucheryavy.} 2017. On the 
temporal effects of mobile blockers in urban millimeter-wave cellular scenarios. \textit{IEEE 
T.~Veh. Technol.} 66(11):10124--10138. doi: 10.1109/TVT.2017.2754543.
\bibitem{12-bes-1}
  \Aue{Samuylov, A., V.~Beschastnyi, D.~Moltchanov, D.~Ostrikova, Y.~Gaidamaka, and 
V.~Shorgin.} 2019. Modeling coexistence of unicast and multicast communications in 5G New 
Radio systems. \textit{30th Annual  Symposium (International) on Personal, Indoor and Mobile 
Radio Communications Proceedings}. Piscataway, NJ:  IEEE. Art.~8904350. 6~p. doi: 10.1109/PIMRC.2019.8904350.
\bibitem{13-bes-1}
  \Aue{Begishev, V., D.~Moltchanov, E.~Sopin, A.~Samuylov, S.~Andreev, Y.~Koucheryavy, and 
K.~Samouylov.} 2019. Quantifying the impact of guard capacity on session continuity in 3GPP new 
radio systems. \textit{IEEE T. Veh. Technol.} 68(12):12345--12359. doi: 
10.1109/TVT.2019.2948702.
\bibitem{14-bes-1}
  \Aue{Gorbunova, A.\,V., V.\,A.~Naumov, Y.\,V.~Gaidamaka, and K.\,E.~Samouylov.} 2018. 
Resursnye sistemy massovogo obsluzhivaniya kak modeli besprovodnykh sistem svyazi [Resource 
queuing systems as models of wireless communication systems]. \textit{Informatika i~ee 
Primeneniya~--- Inform. Appl.} 12(3):48--55. doi: 10.14357/19922264180307. 

\end{thebibliography}

 }
 }

\end{multicols}

\vspace*{-6pt}

\hfill{\small\textit{Received January 30, 2022}}


  
  \Contr
  
  \noindent
  \textbf{Beschastnyi Vitalii A.} (b.\ 1992)~--- Candidate of Science (PhD) in physics and mathematics, 
assistant professor, Department of Applied Probability and Informatics, Peoples' Friendship 
University of Russia (RUDN University), 6~Miklukho-Maklaya Str., Moscow 117198, Russian 
Federation; \mbox{beschastnyy-va@rudn.ru}

\vspace*{3pt}
  
    \noindent
  \textbf{Ostrikova Daria Yu.} (b.\ 1988)~--- Candidate of Science (PhD) in physics and mathematics, 
associate professor, Department of Applied Probability and Informatics, Peoples' Friendship 
University of Russia (RUDN University), 6~Miklukho-Maklaya Str., Moscow 117198, Russian 
Federation; \mbox{ostrikova-dyu@rudn.ru}

\vspace*{3pt}
  
    \noindent
  \textbf{Shorgin Sergey Ya.} (b.\ 1952)~--- Doctor of Science in physics and mathematics, professor, 
principal scientist, Institute of Informatics Problems, Federal Research Center ``Computer Science 
and Control'' of the Russian Academy of Sciences, 44-2~Vavilov Str., Moscow 119133, Russian 
Federation; \mbox{sshorgin@ipiran.ru}

\vspace*{3pt}
  
   \noindent
  \textbf{Moltchanov Dmitri A.} (b.\ 1978)~--- Doctor of Science in technology, associate professor, 
Department of Electronics and Communications Engineering, Tampere University,
 7~Korkeakoulunkatu, Tampere 33720, Finland; \mbox{dmitri.moltchanov@tuni.fi}
 
 \vspace*{3pt}
  
  
    \noindent
  \textbf{Gaidamaka Yuliya V.} (b.\ 1971)~--- Doctor of Science in physics and mathematics, professor, 
Department of Applied Probability and Informatics, Peoples' Friendship University of Russia 
(RUDN University), 6~Miklukho-Maklaya Str., Moscow 117198, Russian Federation; senior 
scientist, Institute of Informatics Problems, Federal Research Center ``Computer Science and 
Control'' of the Russian Academy of Sciences, 44-2~Vavilov Str., Moscow 119333, Russian 
Federation; \mbox{gaydamaka-yuv@rudn.ru}
  

\label{end\stat}

\renewcommand{\bibname}{\protect\rm Литература} 
  