\def\stat{zatsman}

\def\tit{СРЕДОВЫЕ МОДЕЛИ ИЗВЛЕЧЕНИЯ ИЗ ТЕКСТА\\ НОВЫХ ТЕРМИНОВ И ИНДИКАТОРОВ 
НАСТРОЕНИЙ$^*$}

\def\titkol{Средовые модели извлечения из текста новых терминов и~индикаторов 
настроений}

\def\aut{И.\,М.~Зацман$^1$, О.\,В.~Золотарев$^2$, А.\,Х.~Хакимова$^3$}

\def\autkol{И.\,М.~Зацман, О.\,В.~Золотарев, А.\,Х.~Хакимова}

\titel{\tit}{\aut}{\autkol}{\titkol}

\index{Зацман И.\,М.}
\index{Золотарев О.\,В.}
\index{Хакимова А.\,Х.}
\index{Zatsman I.\,M.}
\index{Zolotarev O.\,V.}
\index{Khakimova A.\,K.}


{\renewcommand{\thefootnote}{\fnsymbol{footnote}} \footnotetext[1]
{Исследование выполнено с~использованием ЦКП <<Информатика>> ФИЦ ИУ РАН при финансовой поддержке 
РФФИ (проект 20-04-60185).}}


\renewcommand{\thefootnote}{\arabic{footnote}}
\footnotetext[1]{Федеральный исследовательский центр <<Информатика и~управление>> Российской академии наук, 
\mbox{izatsman@yandex.ru}}
\footnotetext[2]{Институт информационных систем и~инженерно-компьютерных технологий Российского нового университета,\\ 
  \mbox{ol-zolot@yandex.ru}}
\footnotetext[3]{Институт информационных систем и~инженерно-компьютерных технологий Российского нового 
университета,\\ \mbox{aida\_khatif@mail.ru}}

%\vspace*{-6pt}


  
  \Abst{Рассматриваются модели информатики, названные средовыми, так как они создаются 
  в~соответствии с~парадигмой деления предметной области информатики на среды различной 
природы. Основная идея создания средовых моделей состоит в~явном 
использовании трех и~более сред, границ между ними, а~также распределении этапов 
информационных технологий (ИТ) по средам и~границам при моделировании ИТ. 
Сопоставляются две средовые модели, ориентированные на проектирование ИТ 
и~автоматизированных систем (АС), обес\-пе\-чи\-ва\-ющих извлечение индикаторов настроений и~новых 
терминов сочетанием этапов программного и~экспертного анализа текстов. Первая модель, 
предназначенная для описания процессов извлечения только новых терминов, получила 
название  
ин\-фор\-ма\-ци\-он\-но-тех\-но\-ло\-ги\-че\-ски ориентированной модели (модель ITO). Вторая 
модель описывает процессы извлечения индикаторов настроений из текс\-та сообщений 
пользователей социальных сетей и,~одновременно, новых терминов, если они встретились 
в~этом текс\-те. Она получила название модель ITO-Sent. Основная цель статьи со\-сто\-ит 
в~сопоставительном описании этих двух моделей.}
  
\KW{индикаторы настроений; извлечение новых терминов; модель ИТО; анализ текс\-тов; 
модель ITO-Sent; проектирование информационных технологий}

\DOI{10.14357/19922264220208}
  
%\vspace*{-3pt}


\vskip 10pt plus 9pt minus 6pt

\thispagestyle{headings}

\begin{multicols}{2}

\label{st\stat}

\section{Введение}

  Сопоставляемые модели ITO и~ITO-Sent применяются в~проекте РФФИ  
№\,20-04-60185 при проектировании ИТ, используемой 
для решения двух задач: (1)~извлечения новых терминов из текстов\linebreak научных 
документов и~(2)~поиска индикаторов настроений (=\;тер\-ми\-нов-мар\-ке\-ров) 
в~текстах сообщений пользователей социальных сетей, об\-суж\-да\-ющих пандемию 
нового коронавируса. При \mbox{решении} первой задачи эта технология позволяет 
обнаружить новые термины, представляющие научное знание и~его эволюцию во 
времени. При решении второй задачи она помогает экспертам сопоставить 
понимание пользователями социальных сетей вопросов, относящихся к~пандемии, 
с~эво\-лю\-ци\-о\-ни\-ру\-ющим научным знанием, найти индикаторы настроений 
и~определить отношение пользователей к~обсуждаемым вопросам. Отметим, что 
появление новых терминов обуслов\-ле\-но, как правило, изменением научного 
знания о~пандемии.
  
  Анализ настроений и~определение их индикаторов (sentiment analysis)~--- это 
одно из направлений исследований в~области обработки естественного языка. Под 
индикаторами настроений обычно подразумевают эмоционально\footnote[4]{ Психолог 
Пол Экман выделяет шесть основных эмоций: радость (счастье), печаль, страх, гнев, удивление 
и~отвращение~[1, 2].} окрашенные слова и~выражения, в~том чис\-ле оценки, высказанные 
автором в~тексте (в~проекте РФФИ это вопросы, относящиеся к~пандемии нового 
коронавируса; индикаторы настроений отражают, например, степень 
психологической на\-пря\-жен\-ности пользователей социальных сетей, которые их 
обсуждают).
  
  Анализ текста для извлечения индикаторов настроений может быть проведен на 
уровне леммы, нескольких эмоционально окрашенных лексем (связанных друг 
с~другом семантическими\linebreak отношениями), предложения и~всего документа~[3]. 
В~процессе извлечения индикаторов настроений, как правило, создается 
тональный словарь тер\-ми\-нов-мар\-ке\-ров (sentiments dictionary), которые могут 
быть эмоционально окрашенными словами или символами, обозначающими 
эмоции~[4]. В~своей простейшей форме тональный словарь представляет собой 
список тер\-ми\-нов-мар\-ке\-ров с~присвоенными значениями показателя 
настроения (положительное, нейтральное, отрицательное).

\begin{figure*}[b] %fig1
\vspace*{1pt}
  \begin{center}  
    \mbox{%
\epsfxsize=128mm
\epsfbox{zac-1.eps}
}
\end{center}
\vspace*{-9pt}
\Caption{Модель ITO извлечения новых терминов из текстов и их социализации}
\end{figure*}
  
  Описание процесса создания модели ITO как тео\-ре\-ти\-че\-ской основы 
проектирования ИТ извлечения нового знания из 
текстов дано в~работах~[5--7]. Ее вариант, адаптированный для одновременного 
извлечения новых терминов и~индикаторов настроений, названный ITO-Sent, 
рас\-смот\-рен в~работе~[8]. Цель настоящей статьи со\-сто\-ит в~сопоставительном 
описании этих моделей.
  
\section{Сопоставление моделей}

  Модель ITO первоначально была создана в~результате увеличения с~двух до 
трех числа сред (по сравнению с~моделью генерации знания  
(=\;мо\-дель SECI)~[9--11], но в~которой среды не эксплицированы) за счет 
добавления циф\-ро\-вой среды для решения двух задач~\cite{5-zac, 6-zac, 12-zac}:
  \begin{enumerate}[(1)]
\item извлечения из текс\-тов новых значений ис\-сле\-ду\-емых языковых единиц;
\item извлечения из текс\-тов новых терминов для создания и~обновления базы 
медицинских знаний о~терминологических портретах заболеваний.
\end{enumerate}

  При создании модели ITO информатика позиционируется как четвертая отрасль 
научного знания при делении всей его совокупности на четыре  
отрасли~\cite{13-zac, 14-zac}. Применение такого позиционирования дало 
возможность использовать парадигму средового деления предметной об\-ласти 
информатики (ПОИ), соотнести процессы генерации знания и~информационные 
трансформации со средами ПОИ и~границами между ними~[15--17]. Основная 
идея создания этой модели со\-сто\-ит в~явном использовании трех сред ПОИ 
различной природы (ментальной, информационной и~циф\-ро\-вой).
  
  В задаче извлечения новых терминов из текстов документов базы данных 
PubMed и~тер\-ми\-нов-мар\-ке\-ров из сообщений социальной сети Twitter 
описание этих трех сред выглядит следующим образом~\cite{8-zac}:
  \begin{enumerate}[(1)]
\item ментальная среда содержит концепты терминов как понятий 
человеческого знания в~умах (сознании) экспертов, которые формируются 
в~рамках той или иной вербальной знаковой сис\-те\-мы (см.\ обуслов\-лен\-ность 
концептов ис\-поль\-зу\-емы\-ми знаковыми сис\-те\-ма\-ми в~описаниях к~рис.~2 
в~работе~\cite{17-zac} и~рис.~1 и~2  
в~работе~\cite{18-zac});
\item информационная среда включает в~себя перцептивные (=\;сенсорно 
воспринимаемые) формы пред\-став\-ле\-ния концептов знания человека, образующие текс\-ты, 
таблицы, диаграммы, формулы и~пр.;
\item цифровая среда охватывает коды текс\-тов документов и~сообщений 
в~компьютерах, телекоммуникационных сетях, базах данных\linebreak и~пр.
  \end{enumerate}
  
  \begin{figure*} %fig2
\vspace*{1pt}
  \begin{center}  
    \mbox{%
\epsfxsize=128mm
\epsfbox{zac-2.eps}
}
\end{center}
\vspace*{-12pt}
\Caption{Модель ITO-Sent (числовая нумерация процессов используется в~разд.~4)}
\vspace*{-3pt}
\end{figure*}

  Модель ITO включает эти три среды и~соотносит процессы извлечения новых 
терминов из текстов с~тремя средами и~границами между ними (рис.~1). При 
выполнении этих процессов для того, чтобы найти новые термины, сначала 
программно формируется массив документов как потенциальных источников 
новых терминов (ПИНТ), что обозначено стрелкой с~греческой буквой~$\alpha$ на 
рис.~1 и~2. При его формировании используются словари уже известных терминов 
из базы терминологических знаний (БТЗ), что обозначено стрелкой с~греческой 
буквой~$\beta$ (словари описаны в~разд.~3). В~отобранных документах 
программно отмечаются контексты потенциально новых терминов (ПНТ), 
которые затем визуализируются. Каждый отмеченный контекст служит входом 
операции концептуализации (стрелка~А), при выполнении которой эксперт 
сравнивает этот контекст с~найденными дефинициями существующих терминов, 
извлечение которых из словарей обозначено стрелкой с~греческой 
бук\-вой~$\gamma$.
  



  Если термин признается экспертом новым, то его аннотация 
  с~\textit{личностной дефиницией} вводится в~БТЗ, включающую словари известных 
терминов и~аннотации новых терминов с~их дефинициями, сформированными 
экспертами. В~противном случае обработка этого контекста ПНТ в~модели ITO 
прекращается (но она продолжается в~модели ITO-Sent, см.\ рис.~2).
  
  Отметим, что если бы в~проекте использовалась модель SECI, то новые 
термины с~коллективными (=\;со\-гла\-со\-ван\-ны\-ми в~коллективе экспертов) 
дефинициями служили бы входом процесса синтеза, а~не вводились бы в~БТЗ. 
Базы знаний в~модели SECI отсутствуют~\cite{8-zac, 9-zac, 10-zac, 11-zac}. 
В~качестве источника личностного знания модель SECI указывает коллективное 
знание, не определяя его первоисточники. Модель ITO содержит базу данных 
первоисточников в~циф\-ро\-вой среде, в~которой и~осуществляется поиск ПИНТ 
(см.\ разд.~4).
  
  Для того чтобы сформировать коллективную аннотацию нового термина 
согласно модели ITO, из БТЗ извлекается личностная аннотация эксперта и~она 
визуализируется, что обозначено стрелками~B и~C. Затем, в~процессе синтеза, эта 
аннотация объединяется (обозначено 
стрелками~D и~E) с~контекстом нового термина (КНТ), из которого этот термин был извлечен в~процессе 
концептуализации. Его дефиниции согласовываются в~коллективе экспертов 
и~ими формируется коллективная аннотация, которая вводится в~БТЗ. 
  
  Отметим, что процесс концептуализации выполняется для \textit{всех} 
контекстов ПНТ, что условно обозначено \textit{одинарным контуром} стрелки~A 
на входе этого процесса. Однако на вход процесса синтеза поступают только те 
контексты, которые \textit{действительно оказались источниками новых 
терминов (т.\,е.\ КНТ)}, что условно обозначено \textit{двойным контуром} стрелки~E на 
входе процесса синтеза. Двойной контур стрелок~B--E на рис.~1 и~2 подчеркивает 
одновременное использование контекстов и~соответствующих личностных 
аннотаций новых терминов в~операции синтеза. Новизна термина становится 
ясной только после завершения концептуализации его кон\-текс\-та и~сопоставления 
его смыс\-ла в~этом кон\-текс\-те с~дефинициями словарей уже известных терминов. 
После операции синтеза процессы социализации (согласования смыс\-ло\-во\-го 
содержания нового термина) и~экстернализации полностью соответствуют модели 
SECI~\cite{9-zac, 10-zac, 11-zac}, но есть одно существенное отличие после 
выполнения этих двух процессов: согласованная аннотация нового термина 
вводится в~БТЗ.
  
  Между моделями ITO и~SECI существуют и~другие отличия~\cite{5-zac}.  
Во-пер\-вых, модель ITO различает два вида представления явных знаний: 
словами и~компьютерными кодами. Во-вто\-рых, модель ITO включает четыре 
дополнительных процесса: визуализацию, оцифровку, концептуализацию 
и~аннотирование, которых нет в~модели SECI. В-треть\-их, модель ITO включает 
стандарт для определения новизны терминов (на рис.~1 и~2 это словари БТЗ). 
И,~наконец, модель ITO является теоретической основой разработки 
ИТ для целенаправленного извлечения нового знания из 
текстов. И эта модель уже используется для проектирования таких 
технологий~\cite{6-zac, 7-zac, 12-zac, 19-zac}, но без извлечения  
тер\-ми\-нов-мар\-ке\-ров эмоций.
  



  В проекте РФФИ №\,20-04-60185 используется терминологическое сопоставление 
научной системы знаний о~пандемии нового коронавируса и~знаний о~ней 
пользователей социальных сетей, а~также выполняется извлечение 
 тер\-ми\-нов-мар\-ке\-ров в~целях оценки динамики психологического состояния 
пользователей социальных сетей. Следовательно, из текстов необходимо 
извлекать не только новые термины, но также и~термины-маркеры. Поэтому 
в~рамках этого проекта в~модель ITO добавляется тональный словарь  
тер\-ми\-нов-мар\-ке\-ров (словарь \mbox{T-M}). Кроме того, в~модель добавляется 
процесс аннотирования новых тер\-ми\-нов-мар\-ке\-ров в~информационной среде 
(см. процесс~6 на рис.~2).
  
  В модели ITO после процесса концептуализации обработка ПНТ с~его 
контекстом прекращается, если термин не признается новым (см.\ рис.~1). В~модели 
ITO-Sent его обработка продолжается: если эксперт находит в~контексте тер\-мин-мар\-кер эмоций (процесс его поиска не показан на рис.~2), то он проверяется на 
новизну по словарю Т-М, который является частью БТЗ. В~случае его новизны он 
аннотируется вместе с~контекстом (см.\ вертикальную стрелку <<да>> на входе 
процесса~6; связь процессов проверки и~аннотирования с~БТЗ не показана), 
аннотация оцифровывается и~вместе с~дефиницией термина-маркера добавляется 
в~тональный словарь Т-М. Такая же обработка согласно модели ITO-Sent 
выполняется и~для контекстов новых терминов (см. горизонтальную стрелку 
<<да>>).
  
\section{Словари базы терминологических знаний}

  Согласно модели ITO-Sent, БТЗ содержит словари известных терминов для 
экспертной оценки новизны тех ПНТ, которые извлекаются из текстов. БТЗ может 
включать как сами словари, так и~ссылки на них. При проектировании технологии 
извлечения новых англоязычных терминов из текстов документов базы данных 
PubMed в~качестве эталона новизны применялись два словаря  
Merriam-Webster~\cite{20-zac, 21-zac}. Для оценки новизны терминов 
в~русскоязычных документах и~твитах применяется словарь Ожегова~[22], 
который насчитывает более 100\,000~слов. Словарь Merriam-Webster~\cite{20-zac} 
применяется для оценки новизны слов в~англоязычных твитах. Если термин 
отсутствует в~со\-от\-вет\-ст\-ву\-ющем словаре, то он считается новым. В~про\-ек\-ти\-ру\-емой 
ИТ в~БТЗ включены ссылки на вышеперечисленные 
словари, содержащие определения терминов. Для доступа к~определениям 
терминов используются именно эти ссылки~\cite{20-zac, 21-zac, 22-zac}.
  
  Например, слов <<антипрививочник>> и~<<антипрививочница>> нет в~словаре 
Ожегова. Поэтому им присваивается статус <<новый термин>>. В~то же время 
термин <<anti-vaxxer>> появился в~английском языке более 20~лет назад и~для 
него уже есть дефиниция (<<лицо, которое выступает против использования 
вакцин или нормативных актов, предписывающих 
вакцинацию>>\footnote{В~оригинале <<a person who opposes the use of vaccines or regulations 
mandating vaccination>>~\cite{23-zac}.}~\cite{23-zac}), поэтому он не считается новым. 
Отметим, что дефиниции медицинского словаря Merriam-Webster~\cite{21-zac} 
используются в~проекте для оценки новизны научной лексики по медицине.
  
  В проекте РФФИ значение показателя настроений присваивается каждому 
термину-маркеру на основе тернарной шкалы полярностей, используемой 
в~информационном ресурсе Sentiment140 при аннотировании твитов  
с~тер\-ми\-на\-ми-мар\-ке\-ра\-ми~\cite{24-zac}. Этот ресурс содержит аннотации 
1\,600\,000~твитов, извлеченных с~помощью Twitter API. Каждая аннотация 
содержит одно из трех значений показателя настроений по тернарной шкале 
полярностей (0\;=\;от\-ри\-ца\-тель\-ный; 2\;=\;нейт\-раль\-ный;  
4\;=\;по\-ло\-жи\-тель\-ный), а~также идентификатор твита, его текст и~дату 
создания аннотации. Отметим, что при формировании ресурса Sentiment140 
применялся новый подход к~автоматической классификации настроений 
в~сообщениях с~применением алгоритмов машинного обучения~\cite{25-zac}.
  
  В рамках проекта РФФИ был создан тональный словарь терминов-маркеров по 
социально-пси\-хо\-ло\-ги\-че\-ской тематике с~указанием для каждого из них одного, 
двух или трех значений показателя настроений. В~процессе аннотирования 
тер\-ми\-нов-мар\-ке\-ров (см.\ рис.~2) им присваиваются значения показателей с~учетом их 
контекстов на основе тернарной шкалы полярностей. В~разных контекстах 
одному и~тому же тер\-ми\-ну-мар\-ке\-ру могут быть присвоены разные значения 
показателя настроений. Поэтому в~тональном словаре термин-маркер может иметь 
более одного значения показателя. Для каждого значения всех показателей 
в~словаре эмоционально окрашенной лексики также указана его 
частотность~\cite{26-zac}, которая определена по экспериментальному массиву, 
содержащему 12\,193~твита.
  
\section{Информационная технология}

  На основе модели ITO-Sent в~рамках проекта РФФИ разрабатывается 
ИТ, этапы которой соответствуют 17~процессам, 
распределенным по трем средам и~границам между ними. Они обозначены 
прямоугольниками с~числами от 1 до 14 и~стрелками~$\alpha$, $\beta$ и~$\gamma$ 
на рис.~2. Последние три процесса обозначены стрелками, чтобы упростить эти 
рисунки. Ниже кратко описаны технологические этапы, соответствующие 
17~процессам модели, представленной на рис.~2.
  
  Выполнение поискового запроса в~цифровой среде сообщений Twitter 
(документов PubMed), сформированного экспертом, а~также про\-грам\-мный выбор 
из них терминов, сравнение выбранных терминов со словарями традиционных 
терминов, составление списка ПНТ и~определение их контекстов в~электронных  
до\-ку\-мен\-тах/со\-об\-ще\-ни\-ях, позиционируемых как ПИНТ, на рис.~1 и~2 
обозначены стрелкой~$\alpha$, а~обращение к~словарям традиционных 
терминов~--- стрелкой~$\beta$.
  
  Визуализация электронных ПИНТ с~отмеченными контекстами ПНТ обозначена 
на рис.~2 прямоугольником с~цифрой~1 (далее в~скобках указаны 
идентификаторы прямоугольников и~стрелок на рис.~2), а его передача на этап 
концептуализации~(3), выполняемой экспертом с~использованием дефиниций 
словарей ($\gamma$), обозначена стрелкой~А на рис.~1 и~2. Если термин 
в~контексте некоторого ПИНТ позиционируется экспертом как новый, то он 
аннотируется~(4), его аннотация оцифровывается~(5) и~она вводится в~БТЗ. Если 
в~этом же контексте есть новый термин-маркер, то он аннотируется экспертом~(6), 
эта личностная аннотация оцифровывается~(7) и~вводится в~БТЗ. В~противном 
случае обработка этого контекста прекращается. Если термин в~этом же контексте 
позиционируется экспертом как синоним известного термина, но в~контексте 
существует новый тер\-мин-мар\-кер, то он аннотируется~(6), его аннотация 
оцифровывается~(7) и~вводится в~БТЗ. В~противном случае обработка этого 
контекста прекращается.
  
  Электронные личностные аннотации извлекаются из БТЗ~(B). Аннотации 
визуализируются~(8), объединяются~(9) с~контекстом аннотированного термина 
(C, D, E), анализируются и~затем интерпретируются экспертами~(10) в~целях 
формирования согласованного понимания термина в~процессе его 
социализации~(11). Если экспертам это удается (противоположный случай не 
показан на рис.~1 и~2), то термину присваивается статус согласованного. При 
этом границы его контекста могут быть изменены в~процессе 
экстернализации~(12) согласованного концепта. Суть процесса экстернализации 
заключается в~формировании экспертами коллективной дефиниции этого термина 
на основе его контекста (на рис.~2 связь согласованного термина с~его контекстом 
не показана). Затем эксперты формируют коллективную аннотацию термина~(13), 
включающую его дефиницию и~отношения с~другими терминами. Аннотация 
оцифровывается~(14) и~вводится в~БТЗ.
  
\section{Заключение}

  Модель SECI используется в~сфере экономики уже более 30~лет~\cite{9-zac} 
для описания уже состоявшихся процессов генерации знания, протекающих в~двух 
средах, ментальной и~информационной, в~част\-ности для описания процесса 
<<мозгового штурма>>. Добавление цифровой среды и~потенциальных 
источников нового знания в~ходе оцифровки модели SECI обусловило начало 
формирования нового теоретического направления в~информатике: 
\textit{средовые модели информационных технологий и~систем}. В~самом 
названии этого направления обозначено его прикладное назначение. Например, 
это проектирование ИТ и~АС, обеспечивающих целенаправленную генерацию нового знания 
в~процессе концептуализации потенциальных источников нового знания~\cite{5-zac, 
6-zac, 7-zac, 12-zac, 19-zac, 27-zac, 28-zac}. В~2012~г.\ средовые модели 
применялись при проектировании индикаторов мониторинга и~оценке на\-уч\-но-ис\-сле\-до\-ва\-тель\-ских программ~\cite{29-zac, 30-zac}.
  
  Первой моделью, созданной для решения задач целенаправленной генерации 
нового знания, стала средовая модель ITO и~ее частные случаи. На ее основе уже 
создана и~используется ИТ для извлечения из текстов новых значений немецких 
модальных глаголов~\cite{5-zac, 7-zac, 12-zac, 19-zac, 27-zac, 28-zac}. В~стадии 
проектирования находятся еще две ИТ. Первая ИТ предназначена для извлечения 
из текстов новых терминов для создания и~обновления базы медицинских знаний 
о терминологических профилях заболеваний~\cite{6-zac}. Для проектирования 
второй ИТ, кратко описанной в~разделе~4, был создан новый вариант средовой 
модели ITO, получившей название <<модель ITO-Sent>>.
  
  Основное отличие между моделями ITO и~ITO-Sent (см.\ рис.~1 и~2) состоит в~том, 
что в~последнюю добавлен процесс аннотирования экспертами контекстов новых 
терминов-маркеров и~включение их личностных аннотаций в~словарь тональных 
слов. На рис.~2 не показано их согласование в~коллективе экспертов, но оно 
может быть выполнено по аналогии с~формированием коллективных аннотаций 
новых терминов (см.\ процессы~9--14).
  
  В заключение отметим, что в~соответствии с~парадигмой деления предметной 
области информатики на среды различной природы~\cite{15-zac, 16-zac, 17-zac} 
основная идея создания моделей ITO и~ITO-Sent, других средовых моделей ИТ 
и~АС заключается в~явном использовании трех и~более сред, границ между ними, 
а~также в~распределении технологических этапов по средам и~границам при 
моделировании ИТ и~информационных процессов в~АС.
  
{\small\frenchspacing
 {%\baselineskip=10.8pt
 %\addcontentsline{toc}{section}{References}
 \begin{thebibliography}{99}
\bibitem{1-zac}
\Au{Ekman P.} Are there basic emotions?~// Psychol. Rev., 1992. Vol.~99. No.\,3.  
P.~550--553.
\bibitem{2-zac}
\Au{Ekman P.} Basic emotions~// Handbook of cognition and emotion~/ Eds. T.~Dalgleish, 
M.~Power.~--- Chichester, U.K.: John Wiley and Sons Ltd., 1999. P.~45--60. 
\bibitem{3-zac}
\Au{Cruz F.\,L., Troyano~J.\,A., Pontes~B., Ortega~F.\,J.} Building layered, multilingual sentiment 
lexicons at synset and lemma levels~// Expert Syst. Appl., 2014. Vol.~41. No.\,13. 
P.~5984--5994.
\bibitem{4-zac}
\Au{Thelwall M., Buckley~K., Paltoglou~G.} Sentiment in Twitter events~// J.~Am. Soc.  
Inf. Sci. Tec., 2011. Vol.~62. No.\,2. P.~406--418.
\bibitem{5-zac}
\Au{Zatsman I.} A~model of goal-oriented knowledge discovery based on human-computer 
symbiosis~// 16th Forum (International) on Knowledge Asset Dynamics Proceedings.~--- Matera, Italy: 
Arts for Business Institute, 2021. P.~297--312.
\bibitem{6-zac}
\Au{Zatsman I., Khakimova~A.} New knowledge discovery for creating terminological profiles of 
diseases~// 22nd European Conference on Knowledge Management Proceedings.~--- Reading, U.K.: 
Academic Publishing International Ltd., 2021. P.~837--846.
\bibitem{7-zac}
\Au{Зацман И.\,М.} Компьютерная и~экономическая модели генерации нового знания: 
сопоставительный анализ~// Системы и~средства информатики, 2021. Т.~31. №\,4. С.~84--96.
\bibitem{8-zac}
\Au{Зацман И.\,М.} Модель процесса извлечения новых терминов и~тональных слов из 
текстов~// Системы и~средства информатики, 2022. Т.~32. №\,2. С.~115--127.
\bibitem{9-zac}
\Au{Nonaka I.} The knowledge-creating company~// Harvard Bus. Rev., 1991. Vol.~69. No.\,6.  
P.~96--104.
\bibitem{10-zac}
\Au{Nonaka I.} A~dynamic theory of organizational knowledge creation~// Organ. Sci., 1994. Vol.~5. 
No.\,1. P.~14--37.
\bibitem{11-zac}
\Au{Нонака И., Такеучи~Х.} Компания~--- создатель знания~/ Пер. c англ.~--- М.: Олимп-бизнес, 
2003. 384~с. (\Au{Nonaka~I., Takeuchi~H.} The knowledge-creating company.~--- Oxford, NY, 
USA: Oxford University Press, 1995. 284~p.)
\bibitem{12-zac}
\Au{Зацман И.\,М.} Проб\-лем\-но-ори\-ен\-ти\-ро\-ван\-ная актуализация словарных статей 
двуязычных словарей и~медицинской терминологии: сопоставительный анализ~// Информатика 
и~её применения, 2021. Т.~15. Вып.~1. С.~94--101.
\bibitem{13-zac}
\Au{Denning~P., Rosenbloom~P.} Computing: The fourth great domain of science~// Commun. 
ACM, 2009. Vol.~52. No.\,9. P.~27--29.
\bibitem{14-zac}
\Au{Rosenbloom P.\,S.} On computing: The fourth great scientific domain.~--- Cambridge, MA, USA: 
MIT Press, 2013. 308~p.
\bibitem{15-zac}
\Au{Зацман И.\,М.}
 Таблица интерфейсов информатики как ин\-фор\-ма\-ци\-он\-но-компью\-тер\-ной науки~//
  Научно-тех\-ни\-че\-ская информация. Серия~1: Организация и~методика информационной работы, 2014.
№\,11. С.~1--15.
  

\bibitem{16-zac}
\Au{Зацман И.\,М.} Интерфейсы третьего порядка в~информатике~// Информатика и~её 
применения, 2019. Т.~13. Вып.~3. С.~82--89.
\bibitem{17-zac}
\Au{Зацман И.\,М.} Кодирование концептов в~цифровой среде~// Информатика и~её применения, 
2019. Т.~13. Вып.~4. С.~97--106.
\bibitem{18-zac}
\Au{Zatsman I.} Three-dimensional encoding of emerging meanings in AI-systems~// 21st European 
Conference on Knowledge Management Proceedings.~--- Reading, U.K.: Academic Publishing 
International Ltd., 2020. P.~878--887.
\bibitem{19-zac}
\Au{Зацман И.\,М.} Проб\-лем\-но-ори\-ен\-ти\-ро\-ван\-ная верификация полноты темпоральных 
онтологий и~заполнение понятийных лакун~// Информатика и~её применения, 2020. Т.~14. 
Вып.~3. С.~119--128.
\bibitem{20-zac}
The dictionary by Merriam-Webster. {\sf https://www.\linebreak  merriam-webster.com/browse/dictionary/a}.
\bibitem{21-zac}
Merriam-Webster medical dictionary. {\sf https://www.\linebreak merriam-webster.com/medical}.
\bibitem{22-zac}
Toлкoвый cлoвapь pyccкoгo языкa Oжeгoвa C.\,И. {\sf https://ozhegov.textologia.ru}.
\bibitem{23-zac}
Anti-vaxxer~// The dictionary by Merriam-Webster. {\sf  
https://www.merriam-webster.com/browse/dictionary/\linebreak anti-vaxxer}.
\bibitem{24-zac}
Sentiment140 dataset with 1.6~million tweets. Sentiment analysis with tweets.  {\sf 
https://www.kaggle.com/\linebreak kazanova/sentiment140}.
\bibitem{25-zac}
\Au{Go A., Bhayani~R., Huang~L.} Twitter sentiment classification using distant supervision. {\sf 
https://www.\linebreak researchgate.net/publication/228523135\_Twitter\_\linebreak sentiment\_classification\_using\_distant\_supervision.}
\bibitem{26-zac}
Словарь эмоционально окрашенной лексики. {\sf http://bigwer.ru/zol2/index.html}.
\bibitem{27-zac}
\Au{Зацман И.\,М.} Стадии целенаправленного извлечения знаний, имплицированных 
в~параллельных текстах~// Системы и~средства информатики, 2018. Т.~28. №\,3. С.~175--188.

\columnbreak

\bibitem{28-zac}
\Au{Зацман И.\,М.} Формы представления нового знания, извлеченного из текстов~// 
Информатика и~её применения, 2021. Т.~15. Вып.~3. С.~83--90.
\bibitem{29-zac}
\Au{Zatsman I.} Tracing emerging meanings by computer: Semiotic framework~// 13th European 
Conference on Knowledge Management Proceedings.~--- Reading, U.K.: Academic Publishing 
International Ltd., 2012. Vol.~2. P.~1298--1307.
\bibitem{30-zac}
\Au{Zatsman I.} Denotatum-based models of knowledge creation for monitoring and evaluating R\&D 
program implementation~// 11th IEEE Conference (International) on Cognitive Informatics and 
Cognitive Computing Proceedings.~--- Los Alamitos, CA, USA: IEEE Computer Society Press, 2012. 
P.~27--34.
\end{thebibliography}

 }
 }

\end{multicols}

\vspace*{-8pt}

\hfill{\small\textit{Поступила в~редакцию 12.04.22}}

\vspace*{8pt}

%\pagebreak

%\newpage

%\vspace*{-28pt}

\hrule

\vspace*{2pt}

\hrule

%\vspace*{-2pt}

\def\tit{MEDIUM MODELS FOR~DISCOVERING NOVEL TERMS AND~SENTIMENTS FROM~TEXTS}


\def\titkol{Medium models for~discovering novel terms and~sentiments from~texts}


\def\aut{I.\,M.~Zatsman$^1$, O.\,V.~Zolotarev$^2$, and~A.\,K.~Khakimova$^2$}

\def\autkol{I.\,M.~Zatsman, O.\,V.~Zolotarev, and~A.\,K.~Khakimova}

\titel{\tit}{\aut}{\autkol}{\titkol}

\vspace*{-8pt}


\noindent
$^1$Federal Research Center ``Computer Science and Control'' of the Russian Academy of Sciences,  
44-2~Vavilov\linebreak
$\hphantom{^1}$Str., Moscow 119333, Russian Federation

\noindent
$^2$Russian New University, 22~Radio Str., Moscow 105005, Russian Federation

\def\leftfootline{\small{\textbf{\thepage}
\hfill INFORMATIKA I EE PRIMENENIYA~--- INFORMATICS AND
APPLICATIONS\ \ \ 2022\ \ \ volume~16\ \ \ issue\ 2}
}%
 \def\rightfootline{\small{INFORMATIKA I EE PRIMENENIYA~---
INFORMATICS AND APPLICATIONS\ \ \ 2022\ \ \ volume~16\ \ \ issue\ 2
\hfill \textbf{\thepage}}}

\vspace*{3pt} 




\Abste{The models of informatics (=\;computer and information science), called medium models, are 
considered. They are created according to the paradigm of dividing the subject domain of informatics 
into media of different nature, for example, mental, information, and digital ones. The principal idea of 
creating medium models is the usage of three or more media, the boundaries between them, as well as 
the distribution of information technology (IT) stages by media and boundaries during IT modeling. 
Two medium models are compared which are focused on the design of IT and computerised systems 
that provide the discovery of sentiments and new terms through human--computer symbiosis during 
analysis of texts. The first model, designed to describe the processes of discovering new terms, was 
called the information technology-oriented  (ITO) model. The second model describes the 
processes of discovering sentiments in messages of social network users and, at the same time, new 
terms if they are found in these messages. It was named the ITO-Sent model. The principal aim of the 
paper is to compare these two models.}

\KWE{sentiments; discovering new terms; ITO model; text analysis; ITO-Sent model; information 
technology design}



\DOI{10.14357/19922264220208}

\vspace*{-18pt}

\Ack

\vspace*{-4pt}


\noindent
The research was carried out using the infrastructure of the shared research facilities CKP 
``Informatics'' of FRC CSC RAS. The reported study was funded by RFBR, project number  
20-04-60185.


%\vspace*{4pt}

  \begin{multicols}{2}

\renewcommand{\bibname}{\protect\rmfamily References}
%\renewcommand{\bibname}{\large\protect\rm References}

{\small\frenchspacing
 {%\baselineskip=10.8pt
 \addcontentsline{toc}{section}{References}
 \begin{thebibliography}{99}
 
 \vspace*{-4pt}
 
\bibitem{1-zac-1}
\Aue{Ekman, P.} 1992. Are there basic emotions? \textit{Psychol. Rev.} 99(3):550--553.
\bibitem{2-zac-1}
\Aue{Ekman, P.} 1999. Basic emotions. \textit{Handbook of cognition and emotion}. Eds. 
T.~Dalgleish and M.~Power. Chichester, U.K.: John Wiley and Sons Ltd. 45--60.
\bibitem{3-zac-1}
\Aue{Cruz, F.\,L., J.\,A.~Troyano, B.~Pontes, and F.\,J.~Ortega.} 2014. Building layered, multilingual 
sentiment lexicons at synset and lemma levels. \textit{Expert Syst. Appl.}  
41(13):5984--5994.
\bibitem{4-zac-1}
\Aue{Thelwall, M., K.~Buckley, and G.~Paltoglou.} 2011. Sentiment in Twitter events. \textit{J.~Am. 
Soc. Inf. Sci. Tec.} 62(2):406--418.
\bibitem{5-zac-1}
\Aue{Zatsman, I.} 2021. A~model of goal-oriented knowledge discovery based on human--computer 
symbiosis. \textit{16th Forum (International) on Knowledge Asset Dynamics Proceedings}. Matera, Italy: 
Arts for Business Institute. 297--312.
\bibitem{6-zac-1}
\Aue{Zatsman, I., and A.~Khakimova.} 2021. New knowledge discovery for creating terminological 
profiles of diseases. \textit{22nd European Conference on Knowledge Management Proceedings}. 
Reading, U.K.: Academic Publishing International Ltd. 837--846.
\bibitem{7-zac-1}
\Aue{Zatsman, I.} 2021. Komp'yuternaya i~ekonomicheskaya modeli generatsii novogo znaniya: 
sopostavitel'nyy ana\-liz [Computer and economic models of new knowledge generation: 
A~comparative analysis]. \textit{Sistemy i~Sredstva Informatiki~--- Systems and Means of 
Informatics} 31(4):84--96.
\bibitem{8-zac-1}
\Aue{Zatsman, I.} 2022. Model' protsessa izvlecheniya novykh terminov i~tonal'nykh slov iz tekstov 
[A~model of discovering novel terms and sentiments in texts]. \textit{Sistemy i~Sredstva  
Informatiki~--- Systems and Means of Informatics} 32(2):115--127.
\bibitem{9-zac-1}
\Aue{Nonaka, I.} 1991. The knowledge-creating company. \textit{Harvard Bus. Rev.} 69(6):96--104.
\bibitem{10-zac-1}
\Aue{Nonaka, I.} 1994. A~dynamic theory of organizational knowledge creation. \textit{Organ. Sci.} 
5(1):14--37.
\bibitem{11-zac-1}
\Aue{Nonaka, I., and H.~Takeuchi.} 1995. \textit{The knowledge-creating company}. Oxford, NY: 
Oxford University Press. 284~p.
\bibitem{12-zac-1}
\Aue{Zatsman, I.} 2021. Problemno-orientirovannaya ak\-tua\-li\-za\-tsiya slovarnykh statey dvuyazychnykh 
slovarey i~me\-di\-tsin\-skoy terminologii: sopostavitel'nyy analiz [Problem-oriented updating of dictionary 
entries of bilingual \mbox{dictionaries} and medical terminology: Comparative analysis]. \textit{Informatika 
i~ee Primeneniya~--- Inform. Appl.} 15(1):94--101.
\bibitem{13-zac-1}
\Aue{Denning, P., and P.~Rosenbloom.} 2009. Computing: The fourth great domain of science. 
\textit{Commun. ACM} 52(9):27--29.
\bibitem{14-zac-1}
\Aue{Rosenbloom, P.\,S.} 2013. \textit{On computing: The fourth great scientific domain}. Cambridge, 
MA: MIT Press. 308~p.
\bibitem{15-zac-1}
\Aue{Zatsman, I.} 2014. A~table of interfaces of informatics as computer and information science. 
\textit{Sci. Tech. Inf. Proc.} 41(4):233--246.
\bibitem{16-zac-1}
\Aue{Zatsman, I.\,M.} 2019. Interfeysy tret'ego poryadka v informatike [Third-order interfaces in 
informatics]. \textit{Informatika i~ee Primeneniya~--- Inform. Appl.} 13(3):82--89.
\bibitem{17-zac-1}
\Aue{Zatsman, I.\,M.} 2019. Kodirovanie kontseptov v tsifrovoy srede [Digital encoding of concepts]. 
\textit{Informatika i~ee Primeneniya~--- Inform. Appl.} 13(4):97--106.
\bibitem{18-zac-1}
\Aue{Zatsman, I.} 2020. Three-dimensional encoding of emerging meanings in AI-systems. \textit{21st 
European Conference on Knowledge Management Proceedings}. Reading, U.K.: Academic Publishing 
International Ltd. 878--887.
\bibitem{19-zac-1}
\Aue{Zatsman, I.} 2020. Problemno-orientirovannaya ve\-ri\-fi\-ka\-tsiya polnoty temporal'nykh ontologiy 
i~zapolnenie po\-nya\-tiy\-nykh lakun [Problem-oriented verifying the completeness of temporal ontologies 
and filling conceptual lacunas]. \textit{Informatika i~ee Primeneniya~--- Inform. Appl.}  
14(3):119--128.
\bibitem{20-zac-1}
The dictionary by Merriam-Webster. Available at: {\sf  
https://\linebreak www.merriam-webster.com/browse/dictionary/a} (accessed April~25, 2022).
\bibitem{21-zac-1}
Merriam-Webster medical dictionary. Available at: {\sf https://www.merriam-webster.com/medical} 
(accessed April~25, 2022). 
\bibitem{22-zac-1}
\textit{Tolkovyy slovar' russkogo yazyka Ozhegova S.\,I.} [Ozhegov's explanatory Russian 
dictionary]. Available at: {\sf https://ozhegov.textologia.ru} (accessed April~25, 2022).
\bibitem{23-zac-1}
Anti-vaxxer. Available at: {\sf https://www.merriam-webster.\linebreak com/dictionary/anti-vaxxer} (accessed 
April~25, 2022).
\bibitem{24-zac-1}
Sentiment140 dataset with 1.6~million tweets.  Sentiment analysis with tweets. Available at: {\sf 
https://\linebreak www.kaggle.com/kazanova/sentiment140} (accessed April~25, 2022).
\bibitem{25-zac-1}
Twitter sentiment classification using distant supervision. Available at: {\sf 
https://www.researchgate.net/\linebreak publication/228523135\_Twitter\_sentiment\_\linebreak classification\_using\_distant\_supervision} (accessed 
April~25, 2022).
\bibitem{26-zac-1}
Slovar' emotsional'no okrashennoy leksiki [Sentiments dictionary]. Available at: {\sf 
http://bigwer.ru/zol2/\linebreak index.html} (accessed April~25, 2022).
\bibitem{27-zac-1}
\Aue{Zatsman, I.} 2018. Stadii tselenapravlennogo izvlecheniya znaniy, implitsirovannykh 
v~parallel'nykh tekstakh [Stages of goal-oriented discovery of knowledge implied in parallel texts]. 
\textit{Sistemy i~Sredstva Informatiki~--- Systems and Means of Informatics} 28(3):175--188.
\bibitem{28-zac-1}
\Aue{Zatsman, I.} 2021. Formy predstavleniya novogo znaniya, izvlechennogo iz tekstov [Forms 
representing new knowledge discovered in texts]. \textit{Informatika i~ee Primeneniya~--- Inform. 
Appl.} 15(3):83--90.
\bibitem{29-zac-1}
\Aue{Zatsman, I.} 2012. Tracing emerging meanings by computer: Semiotic framework. \textit{13th 
Conference (European) on Knowledge Management Proceedings}. Reading, U.K.: Academic Publishing International Ltd. 
2:1298--1307.
\bibitem{30-zac-1}
\Aue{Zatsman, I.} 2012. Denotatum-based models of knowledge creation for monitoring and 
evaluating R\&D program implementation. \textit{11th IEEE Conference (International) on Cognitive 
Informatics and Cognitive Computing Proceedings}. Los Alamitos, CA: IEEE Computer Society Press. 
27--34.
\end{thebibliography}

 }
 }

\end{multicols}

\vspace*{-9pt}

\hfill{\small\textit{Received April 12, 2022}}


\vspace*{-22pt}

\Contr

\vspace*{-5pt}

\noindent
\textbf{Zatsman Igor M.} (b.\ 1952)~--- Doctor of Science in technology, head of department, 
Institute of Informatics Problems, Federal Research Center ``Computer Science and Control'' of the 
Russian Academy of Sciences, 44-2~Vavilov Str., Moscow 119333, Russian Federation; 
\mbox{izatsman@yandex.ru}


\vspace*{2pt}

\noindent
\textbf{Zolotarev Oleg V.} (b.\ 1959)~--- Candidate of Science (PhD) in technology, head of department, 
Russian New University, 22~Radio Str., Moscow 105005, Russian Federation;  
\mbox{ol-zolot@yandex.ru}

\vspace*{2pt}

\noindent
\textbf{Khakimova Aida Kh.} (b.\ 1963)~--- Candidate of Science (PhD) in biology, leading scientist, 
Russian New University, 22~Radio Str., Moscow 105005, Russian Federation; 
\mbox{aida\_khatif@mail.ru}

 

\label{end\stat}

\renewcommand{\bibname}{\protect\rm Литература}    