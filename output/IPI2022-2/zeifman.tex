\def\stat{zeifman}

\def\tit{О~МОНОТОННОСТИ НЕКОТОРЫХ КЛАССОВ МАРКОВСКИХ ЦЕПЕЙ}

\def\titkol{О~монотонности некоторых классов марковских цепей}

\def\aut{Я.\,А.~Сатин$^1$, А.\,Л.~Крюкова$^2$,  В.\,С.~Ошушкова$^3$, А.\,И.~Зейфман$^4$}

\def\autkol{Я.\,А.~Сатин, А.\,Л.~Крюкова,  В.\,С.~Ошушкова, А.\,И.~Зейфман}

\titel{\tit}{\aut}{\autkol}{\titkol}

\index{Сатин Я.\,А.}
\index{Крюкова А.\,Л.}
\index{Ошушкова В.\,С.}
\index{Зейфман А.\,И.}
\index{Satin Y.\,A.}
\index{Kryukova A.\,L.}
\index{Oshushkova V.\,S.}
\index{Zeifman A.\,I.}


%{\renewcommand{\thefootnote}{\fnsymbol{footnote}} \footnotetext[1]
%{Работа выполнялась с~использованием инфраструктуры Центра коллективного пользования 
%<<Высокопроизводительные вычисления и~большие данные>> (ЦКП <<Информатика>> ФИЦ ИУ 
%РАН, Москва).}}


\renewcommand{\thefootnote}{\arabic{footnote}}
\footnotetext[1]{Вологодский государственный университет, yacovi@mail.ru}
\footnotetext[2]{Вологодский государственный университет, 
kryukovaforstudents@gmail.com}
\footnotetext[3]{ООО Инновейтив пипл, oshushonok@yandex.ru}
\footnotetext[4]{Вологодский государственный университет;
Федеральный исследовательский центр
<<Информатика и~управление>> Российской академии наук;  Вологодский
научный центр Российской академии наук;
 Московский центр фундаментальной и~прикладной математики, \mbox{a\_zeifman@mail.ru}}

\vspace*{-3pt}
  



\Abst{Вводится отношение частичного порядка для марковских 
цепей, исследуются условия монотонности для некоторых классов марковских 
процессов с~непрерывным временем. Сформулированы соответствующие теоремы о 
монотонности, позволяющие сравнивать между собой две (неоднородные, вообще 
говоря) марковские цепи с~одинаковыми начальными условиями и~разными матрицами 
интенсивностей. Рассмотрены классы процессов, для которых заведомо выполнены 
условия, гарантирующие монотонность. Известный интерес с~точки зрения приложений 
представляют марковские цепи с~интервальными интенсивностями. Полученные 
в~работе условия монотонности дают возможность некоторым образом продвинуться 
в~изучении неоднородных марковских цепей с~интервальными интенсивностями, 
а~именно:  в~настоящей работе рассмотрена в~качестве примера система обслуживания  
$M_t/M_t/S/S$ с~интервальными коэффициентами. Полученные результаты подтверждены 
вычислительным экспериментом и~проиллюстрированы  соответствующими графиками 
основных вероятностных характеристик системы.}

\KW{монотонность марковских процессов; нестационарная система 
массового обслуживания;
марковские цепи с~интервальными интенсивностями; предельные характеристики}

\DOI{10.14357/19922264220204}
  
%\vspace*{-3pt}


\vskip 10pt plus 9pt minus 6pt

\thispagestyle{headings}

\begin{multicols}{2}

\label{st\stat}


\section{Введение}

Основное внимание в~настоящей статье уделяется исследованию условий монотонности 
для некоторых классов марковских процессов с~непрерывным временем. Имеется ряд 
работ, так или иначе связанных со сравнением и~монотонностью  марковских цепей 
(см., например,~[1--3]). Как правило, отношение порядка при этом задается  так 
же, как в~\cite{Doorn}, это определение и~используется далее.

Итак, пусть ${\bf p}\left(t\right)\hm =
\left(p_0\left(t\right), p_1\left(t\right), \ldots \right)^{\mathrm{T}}$~--- вектор
распределения вероятностей в~момент времени~$t$, если $p_i\left(t\right)\hm\geq 0$ 
и~$\sum p_i\left(t\right)\hm= 1$. Зададим на множестве таких векторов отношение 
частичного порядка следующим образом: ${\bf p^{\,1}}\left(t\right) \hm\succ{\bf 
p^{\,2}}\left(t\right)$, если и~только если для каждого $i = 1, 2, \ldots$ 
выполнено следующее неравенство:
        \begin{equation*}
   % \label{eq_0.1}
        \sum\limits_{j  \geq i} p^1_j\left(t\right)\geq \sum\limits_{ j  \geq  i} 
p^2_j\left(t\right).
    \end{equation*}
    
Пусть теперь $X_{1}(t)$ и~$X_{2}(t)$~--- марковские цепи.
Тогда  будем говорить, что $X_1(t) \hm\succ X_2(t) $, если для каждого $t\hm\geq 0$ 
сохраняется соответствующее неравенство между векторами распределения 
вероятностей ${\bf p^1}\left(t\right) \hm\succ{\bf p^2}\left(t\right)$ при одних 
и~тех же начальных векторах вероятностей состояний.

В следующем разделе проводится сравнение двух  марковских цепей в~смысле их 
монотонности, затем  рассмотрены классы процессов, для которых выполнены 
вышеупомянутые условия. Известный интерес с~точки зрения приложений приобретают 
марковские цепи с~интервальными интенсивностями (см., например,~\cite{Chitraganti, Xie}),  
так что далее рас\-смот\-ре\-на в~качестве примера система 
обслуживания $M_t/M_t/S/S$ с~интервальными коэффициентами.

\section{Монотонность }

Пусть $X(t)$ и~$X^*(t)$~--- вообще говоря, неоднородные  марковские цепи 
с~непрерывным временем, с~матрицами интенсивностей~$Q(t)$ и~$Q^*(t)$ 
соответственно. Обозначим через $A(t) \hm= Q^{\mathrm{T}}(t)$  и~$A^*(t) \hm= \left(
 Q^*\right)^{\mathrm{T}}(t)$ транспонированные матрицы  интенсивностей для этих цепей. Для 
описания рассматриваемых марковских цепей будем использовать прямые системы 
Колмогорова
\begin{align}
\label{eq_1}
\fr{dp}{dt}&=A(t)p(t);
\\
\label{eq_2}
\fr{dp^*}{dt}&=A^*(t)p^*(t),
 \end{align}
\noindent предполагая выполненными стандартные условия на интенсивности как 
функции от времени (т.\,е.\ локальная интегрируемость на $[0,\infty)$ всех 
$a_{ij}(t)$ и~ограниченность почти при всех $t \ge 0$ соответствующих 
операторных функций из~$l_1$ в~себя (см., например,~\cite{Zeifman})).

Потребуем дополнительно, чтобы  основные мат\-ри\-цы этих сис\-тем были связаны между 
собой равенством
\begin{equation*}
%\label{eq_3}
A^*(t) = A(t)+C(t)
\end{equation*}
и нулевой столбец матрицы $C\left(t\right)$ состоял только из нулей. 
Заметим также, что, как и~в~матрицах~$A\left( t\right)$ и~$A^*\left( t\right)$, 
в~матрице~$C\left(t\right)$ суммы элементов по столбцам равны нулю.

Выполним теперь следующее преобразование обеих систем, состоящее в~исключении 
нулевого состояния. Для этого положим
$$
p_0\left( t\right) =1-\sum\limits_{i\geq 1}p_i\left( t\right).
$$
Тогда системы~(\ref{eq_1}) и~(\ref{eq_2}) примут вид:
\begin{align}
\label{eq_4}
\frac{dz}{dt}&=B(t)z(t)+f(t);
\\
\label{eq_5}
\frac{dz^*}{dt}&=B^*(t)z^*(t)+f^*(t),
\end{align}

\noindent 
где вектор $f(t)$ и~вектор~$f^*(t)$ состоят из элементов нулевых 
столбцов мат\-риц~$A\left( t\right)$ и~$A^*\left( t\right)$ соответственно, 
а~значит, совпадают.


Теперь нетрудно видеть, что матрица~$C^*\left( t\right)$ получается исключением  
из матрицы~$C\left( t\right)$ нулевой строки и~нулевого столбца:

\vspace*{2pt}

\noindent
\begin{equation*}
%\label{eq_6}
B^*(t)=B(t)+C^*(t),
\end{equation*}

\vspace*{-2pt}

\noindent 
и~равенство~(\ref{eq_5}) можно записать в~виде

\vspace*{2pt}

\noindent
\begin{equation}
\label{eq_7}
\fr{dz^*}{dt}=B(t)z^*(t)+ C^*(t)z^*(t)+f^*(t).
\end{equation}

\vspace*{-2pt}

\noindent 
Рассматривая разность~(\ref{eq_7}) и~(\ref{eq_4}), получим сле\-ду\-ющее 
равенство:

\vspace*{2pt}

\noindent
\begin{equation}
\label{eq_8}
\fr{d(z^*(t)-z(t))}{dt}=B(t)(z^*(t)-z(t))+ C^*(t)z^*(t).
\end{equation}

\vspace*{-2pt}

Пусть положительные числа $d_{1}, d_{2}, d_{3}, \ldots$ являются элементами 
соответствующих строк верхнетреугольной матрицы вида

\vspace*{2pt}

\noindent
\begin{equation*}
%\label{eq_9}
D= \left(
\begin{array}{cccc}
d_{1}   & d_{1}  & d_{1}   &   \cdots  \\
0       & d_{2}  & d_{2}   &   \cdots  \\
0       & 0      & d_{3}   &   \cdots  \\
\vdots  & \vdots & \vdots  &   \ddots  \\
\end{array}
\right). 
\end{equation*}

\vspace*{-2pt}

\noindent 
Соответственно, обратная матрица представляется таким образом:

\columnbreak

\noindent
\begin{equation*}
%\label{eq_10}
D^{-1}= \left(
\begin{array}{cccc}
\fr{1}{d_{1}}   & \displaystyle -\fr{1}{d_{2}}  & 0                  &   \cdots  \\
0                 & \displaystyle  \fr{1}{d_{2}}  &\displaystyle  -\fr{1}{d_{3}}   &   \cdots  \\
0                 &  0                &  \displaystyle \fr{1}{d_{3}}   &   \cdots  \\
\vdots  & \vdots & \vdots  &   \ddots  \\
\end{array}
\right). 
\end{equation*}

\noindent Рассмотрим теперь произведение~$DBD^{-1}$

\vspace*{-2pt}

\noindent
\begin{multline*}
%\label{eq_11}
\left(
\begin{array}{cc}
\displaystyle \sum\limits_{i=1}^{\infty} b_{i1}   & \displaystyle
\fr{d_1}{d_{2}} \left( \sum\limits_{i=1}^{\infty} 
b_{i2} - \sum\limits_{i=1}^{\infty} b_{i1} \right)  \\
\displaystyle\fr{d_2}{d_{1}}  \sum\limits_{i=2}^{\infty} b_{i1}                &  
\displaystyle\sum\limits_{i=2}^{\infty} b_{i2} - \sum\limits_{i=2}^{\infty} b_{i1}\\
\displaystyle\fr{d_3}{d_{1}}  \sum\limits_{i=3}^{\infty} b_{i1}               &  
\displaystyle\fr{d_3}{d_{2}} \left( \sum\limits_{i=3}^{\infty} b_{i2} - \sum\limits_{i=3}^{\infty} 
b_{i1} \right)\\
\vdots  & \vdots                
\end{array}
\right. \\
\left.
\begin{array}{cc}
 \displaystyle\fr{d_1}{d_{3}} \left( 
\sum\limits_{i=1}^{\infty} b_{i3} - \sum\limits_{i=1}^{\infty} b_{i2}\right)                 &   
\cdots  \\
\displaystyle\fr{d_2}{d_{3}} \left( \sum\limits_{i=2}^{\infty} b_{i3} - \sum\limits_{i=2}^{\infty} 
b_{i2}\right)   &   \cdots  \\
\displaystyle\sum\limits_{i=3}^{\infty} b_{i3} - \sum\limits_{i=3}^{\infty} b_{i2}   &   \cdots  \\
 \vdots  &   \ddots  \\
\end{array}
\right)
\end{multline*}


\vspace*{-2pt}

\noindent 
и~потребуем, чтобы полученная матрица была существенно 
неотрицательной.


Умножим равенство~(\ref{eq_8}) на матрицу~$D$ слева:
\begin{equation*}
%\label{eq_12}
\fr{dD\left(z^*(t)-z(t)\right)}{dt} = DB(z^*(t)-z(t)) + DC^*z^*(t),
\end{equation*}
преобразуем к~виду

\vspace*{-3pt}

\noindent
\begin{multline*}
%\label{eq_13}
\fr{dD\left(z^*(t)-z(t)\right)}{dt} = {}\\
{}=DBD^{-1} D (z^*(t)-z(t))+ DC^*z^*(t).
\end{multline*}

\vspace*{-2pt}

Получаем

\vspace*{-3pt}

\noindent
\begin{multline*}
D\left(z^*(t)-z(t)\right)=U(t,0)D\left(z^*(0)-z(0)\right)+{}\\
{}+
\int\limits_0^t U(t,s) DC^*(s)z^*(s)\,ds\,.
\end{multline*}

\vspace*{-3pt}

Заметим, что $U(t,s)\hm\geq 0$ при всех $t\hm>s\hm>0$ в~силу существенной 
неотрицательности матрицы~$DBD^{-1}$.


Если для матриц~$A(t)$ и~$A^*(t)$ выполнены условия:
\begin{itemize}
  \item для всех $i$ имеет место равенство $a_{i0}(t)\hm=a^*_{i0}(t)$;
    \item при $i>j$ ($j\hm\neq 0$)  выполняется неравенство $a^*_{ij}(t)\hm> 
a_{ij}(t)$;
    \item при $i<j$, напротив, $a^*_{ij}(t)\hm<a_{ij}(t)$,
\end{itemize}
то получаем  $DC^*(s)\hm\ge 0 $, а~значит, $Dz^*(t)\hm\ge Dz(t)$.
Отсюда вытекает следующее утверждение.

\smallskip

\noindent
\textbf{Теорема~1.}\ \textit{Пусть  $X_1(t)$ и~$X_2(t)$~--- марковские цепи и~при каждом $t 
\hm\ge 0$ выполнены условия: при $i\hm>j$ ($j\hm\neq 0$) $a^*_{ij}(t)\hm> a_{ij}(t)$, при 
$i\hm<j$ $a^*_{ij}(t)\hm<a_{ij}(t)$ и, кроме того,  $a_{i0}(t)\hm=a^*_{i0}(t)$ при  всех~$i$  
(т.\,е.\ одинаковы интенсивности выходы из нулевого состояния). Тогда  
$X_2(t) \hm\succ X_1(t)$.
}

\smallskip

Аналогично получается

\smallskip

\noindent
\textbf{Теорема~2.}\ \textit{Пусть $X_1(t)$ и~$X_2(t)$~--- марковские цепи, для 
интенсивностей которых  при всех $t\hm \ge 0$ справедливы неравенства: при $i\hm>j$ 
$a^*_{ij}(t)\hm< a_{ij}(t)$, а при $i\hm<j$ $a^*_{ij}(t)\hm>a_{ij}(t)$ и, кроме того,   
$a_{i0}(t)\hm=a^*_{i0}(t)$ для всех~$i$. Тогда  $X_1(t) \hm\succ X_2(t)$.
}

\smallskip

Отметим еще, что из неравенства $X_2(t)\hm \succ X_1(t)$ вытекает аналогичное 
неравенство для математических ожиданий соответствующих цепей при одинаковых 
начальных распределениях вероятностей состояний.


\section{Классы монотонных марковских цепей}


Перечислим некоторые классы марковских цепей, для которых выполнены доказанные 
выше тео\-ре\-мы~1 и~2.

\smallskip

\noindent{\rm I.}  \textit{Неоднородный процесс рождения и~гибели.} В~этом 
случае
$a_{ij}(t)\hm=0$ для любого $t \hm\ge 0$, если $|i\hm-j|\hm>1$. Здесь
$a_{i,i+1}(t)\hm=\mu_{i+1}(t)$ и~$a_{i+1,i}(t)\hm=\lambda_{i}(t)$~--- интенсивности 
рождения и~гибели соответственно.

1. Если размерность $S=\infty$, т.\,е.\ матрица интенсивностей имеет вид
\begin{multline*}
%\label{eq_2.1}
A\left( t\right) ={}\\
{}=\!\left(\!
\begin{array}{ccccc}
-\lambda_0                 & \mu_1                                                   
& 0                                 & 0                                                 
& \cdots \\
 \lambda_0                 & -\left( \lambda_1 +\mu_1\right)                         
& \mu_2                             & 0                                                 
& \cdots \\
 0                         & \lambda_1                                               
& -\left( \lambda_2 +\mu_2\right)   & \mu_3                                             
& \cdots \\
 0                         & 0                                                       
& \lambda_2                         & -\left( \lambda_3 +\mu_3\right)                   
& \cdots \\
\vdots                     & \vdots                                                  
& \vdots                            & \vdots                                            
& \ddots \\
\end{array}\!
\right)\!,\hspace*{-3.02283pt}
\end{multline*}


\noindent
 то имеем
\begin{multline*}
C^*\left( t\right) =
 \left(
\begin{array}{c}
  \left( \lambda_1 +\mu_1\right)-\left( \lambda_1^* +\mu_1^*\right)                                         \\
  \lambda_1^*-\lambda_1                                                      \\
  0                                                                                                   \\
\vdots                                                                      
\end{array}
\right.\\[3pt]
\left.
\begin{array}{ccc}
 \mu_2^*- \mu_2 &  0                                                                 & \cdots \\
 \left( \lambda_2 +\mu_2\right)-\left( \lambda_2^* +\mu_2^*\right)&  \mu_3^* - \mu_3                              & \cdots \\
 \lambda_2^*-\lambda_2&  \left( \lambda_3 +\mu_3\right)-\left( \lambda_3^* +\mu_3^*\right) & \cdots \\
\vdots &\vdots                                        & \ddots 
\end{array}
\right)
\end{multline*}
и


\columnbreak
%\vspace*{-8pt}

\noindent
\begin{multline*}
DC^*\left( t\right) ={}\\[2pt]
{}={ %\small 
\left(
\begin{array}{cccc}
  d_1\left( \mu_1-\mu_1^*\right)                                & 0                                               
& 0                                                            & \cdots \\
  d_2\left(\lambda_1^*-\lambda_1 \right)                        & 
d_2\left( \mu_2-\mu_2^*\right)             & 0                                                            
& \cdots \\
  0                                                                  & 
d_3\left(\lambda_2^*-\lambda_2\right)      & d_3\left( \mu_3-
\mu_3^*\right)                          & \cdots \\
\vdots                                                               & \vdots                                          
& \vdots                                                       & \ddots \\
\end{array}
\right)},
\end{multline*}

\vspace*{-6pt}

\noindent
т.\,е.\ получаем, что знак~$DC^*(s)$ определяется именно описанными отношениями 
между интенсивностями матриц~$A(t)$ и~$A^*(t)$. Тем самым иллюстрируется 
выполнение теорем~1 и~2.

\vspace*{2pt}

2. В случае конечного пространства ($S \hm< \infty$) матрица интенсивностей имеет 
вид

\vspace*{-6pt}

\noindent
\begin{multline*}
\!\!\!A(t)\! =\!\!
\left(\!\!
\begin{array}{ccccc}
-\lambda_0                 & \mu_1     & 0                                 & 0     & \cdots   \\
 \lambda_0                 & -\left( \lambda_1 +\mu_1\right)                         
& \mu_2                             & 0                                                 
& \cdots     \\
 0                         & \lambda_1                                               
& -\left( \lambda_2 +\mu_2\right)   & \mu_3                                             
& \cdots     \\
 0                         & 0                                                       
& \lambda_2                         & -\left( \lambda_3 +\mu_3\right)                   
& \cdots  \\
\vdots                     & \vdots                                                  
& \vdots                            & \vdots                                            
& \ddots  \\
0                          & 0                                                       
& 0                                 & \cdots                                            
& \cdots  
\end{array}\!
\right.\hspace*{-1.5842pt}\\
\left.
\begin{array}{cc}
 \cdots  & 0      \\
 \cdots  & 0      \\
 \cdots  & 0      \\
  \cdots  & 0      \\
 \ddots  & \mu_S  \\
 \cdots  & -\mu_S
\end{array}\!
\right),
\end{multline*}
%}

\vspace*{-6pt}

\noindent 
а для  $DC^*$  получаем выражение

\vspace*{-6pt}

\noindent
\begin{multline*}
%\label{eq_2.6.1}
\left(
\begin{array}{cccc}
d_1\left( \mu_1-\mu_1^*\right)                                & 0                                               
& 0                                       &\cdots\\
d_2\left(\lambda_1^*-\lambda_1 \right)                        & 
d_2\left( \mu_2-\mu_2^*\right)             & 0 &\cdots                                      
\\
0                                                                  & 
d_3\left(\lambda_2^*-\lambda_2\right)      & d_3\left( \mu_3-
\mu_3^*\right)     & \cdots \\
\vdots                                                             & \vdots                                          
& \vdots                                  & \ddots \\
0                                                                  & 0                                               
& 0                                       & \cdots 
\end{array}
\right.\\
\left.
\begin{array}{cc}
\cdots & 0\\
\cdots & 0\\
 \cdots & 0\\
 \ddots & 0\\
 \cdots & d_S\left(\mu_S-
\mu_S^*\right)
\end{array}
\right).
\end{multline*}

\vspace*{-6pt}


\noindent{\rm II.} \textit{Неоднородные цепи, в~которых интенсивность 
<<рож\-де\-ния>> группы из~$k$ элементов не зависит от размера популяции  в~этот 
момент.} При этом $a_{ij}(t)\hm=0$ для любого $t \hm\ge 0$, если $i\hm<j\hm-1$, 
и~$a_{i+k,i}(t)\hm=a_k(t)$ для $k\hm\ge 1$~--- интенсивность рождения 
группы из $k$ элементов, а~$a_{i,i+1}(t)\hm=\mu_{i+1}(t)$~--- интенсивность 
гибели одного элемента при условии, что их в~популяции $i\hm+1$;
кроме того, пусть для каждого~$i$ выполнено неравенство $a_{i} \hm\geq a_{i+1}$.



\smallskip

3. Если размерность $S\hm=\infty$, то
\begin{equation*}
%\label{f_1.1.7}
A\left( t\right) =  \left(
\begin{array}{ccccccc}
a_{00}(t)  & \mu_1(t)   & 0          & 0          &\cdots   \\
a_1(t)     & a_{11}(t)  & \mu_2(t)   & 0          &\cdots   \\
a_2(t)     & a_1(t)     & a_{22}(t)  & \mu_3(t)   & \cdots  \\
\vdots     & \vdots     & \vdots     & \vdots     & \ddots  \\
\end{array}
\right).
\end{equation*}

\noindent 
В этом случае произведение~$DC^*$ можно записать в~виде
\begin{equation*}
%\label{eq_2.6.2}
{%\small 
\left(
\begin{array}{cccc}
d_1\left( \mu_1-\mu_1^*\right)                                & 0                                               
& 0                                       & \cdots \\
0                                                                  & 
d_2\left( \mu_2-\mu_2^*\right)             & 0                                       
& \cdots \\
0                                                                  & 0                                               
& d_3\left( \mu_3-\mu_3^*\right)     & \cdots \\
\vdots                                                             & \vdots                                          
& \vdots                                  & \ddots
 \end{array}
\right)}
\end{equation*}

\vspace*{2pt}

\noindent 
или, что то же самое, через диагональные элементы:
\begin{equation*}
%\label{eq_2.6.3}
{ %\small 
\left(
\begin{array}{cccc}
d_1\left( a_{11}^*- a_{11}\right)                             & 0                                               
& 0                                           & \cdots \\
0                                                                  & 
d_2\left( a_{22}^*- a_{22}\right)          & 0                                           
& \cdots \\
0                                                                  & 0                                               
& d_3\left( a_{33}^*- a_{33}\right)      & \cdots \\
\vdots                                                             & \vdots                                          
& \vdots                                      & \ddots
 \end{array}
\right).}
\end{equation*}

\vspace*{2pt}

4. В случае конечного пространства ($S \hm< \infty$)
\begin{multline*}
A\left( t\right) =
 \left(
\begin{array}{ccccc}
a_{00}(t)  & \mu_1(t)   & 0          & 0          &\cdots \\
a_1(t)     & a_{11}(t)  & \mu_2(t)   & 0          &\cdots  \\
a_2(t)     & a_1(t)     & a_{22}(t)  & \mu_3(t)   & \cdots \\
\vdots     & \vdots     & \vdots     & \vdots     & \ddots  \\
a_{S-1}(t) & a_{S-2}(t) & a_{S-3}(t) & a_{S-4}(t) & \cdots \\
a_{S}(t)   & a_{S-1}(t) & a_{S-2}(t) & a_{S-3}(t) & \cdots 
\end{array}
\right. \\[3pt]
 \left.
\begin{array}{ccc}
\cdots  & 0              & 0 \\
\cdots  & 0              & 0 \\
 \cdots & 0              & 0 \\
 \ddots & \vdots         & \vdots \\
\cdots & a_{S-1,S-1}(t) &\mu_{S-1}(t) \\
 \cdots & a_1(t)         & a_{SS}(t)
\end{array}
\right)\!.
%\label{f_1.1.7.1}
\end{multline*}


\noindent 
В этом случае произведение~$DC^*$ можно записать в~виде
\begin{multline*}
%\label{eq_2.6.4}
\left(
\begin{array}{cccc}
d_1\left( \mu_1-\mu_1^*\right)                     & 0                                            
& 0                                       & \cdots \\
0                                                       & d_2\left( \mu_2-
\mu_2^*\right)          & 0                                       & \cdots \\
0                                                       & 0                                            
& d_3\left( \mu_3-\mu_3^*\right)     & \cdots \\
\vdots                                                  & \vdots                                       
& \vdots                                  & \ddots \\
 0                                                      & 0                                            
& 0                                       & \cdots 
\end{array}
\right.\\[3pt]
\left.
\begin{array}{cc}
 \cdots & 0\\
 \cdots & 0\\
\cdots & 0\\
 \ddots & \vdots\\
 \cdots & d_{S-1}\left( \mu_{S-1}-\mu_{S-1}^*\right)
\end{array}
\right).
\end{multline*}


\noindent{\rm III.}  \textit{Неоднородные цепи, в~которых интенсивность гибели 
группы из $k$ элементов не зависит от размера популяции в~этот момент.} При 
этом   $a_{ij}(t)\hm=0$ для любого $t\hm \ge 0$, если $i\hm>j\hm+1$, 
и~$a_{i,i+k}(t)\hm=b_k(t)$ для $k\hm\ge 1$~--- интенсивность гибели 
группы из $k$ элементов, а~$a_{i+1,i}(t)\hm=\lambda_{i}(t)$~--- 
интенсивность рождения одного элемента при условии, что их в~популяции $i\hm+1$;
  кроме того, пусть для каждого~$i$ выполнено неравенство $b_{i}\hm \geq b_{i+1}$.



\smallskip

5. Если размерность $S\hm=\infty$, то
\begin{equation*}
%\label{f_1.1.8}
A\left( t\right) =  \left(
\begin{array}{ccccc}
a_{00}(t)     & b_1(t)       & b_2(t)    & b_3(t) &\cdots   \\
\lambda_1(t)  & a_{11}(t)    & b_1(t)    & b_2(t) &\cdots   \\
0             & \lambda_2(t) & a_{22}(t) & b_1(t) &\cdots   \\
\vdots        & \vdots       & \vdots    & \vdots &\ddots   \\

\end{array}
\right).
\end{equation*}

\noindent 
Выполнив необходимые преобразования, получаем произведение~$DC^*$:
\begin{multline*}
%\label{eq_2.6.5}
\left(
\begin{array}{cc}
d_1\left( b_1-b_1^*\right)                    & d_1\left( b_2-
b_2^*\right)                 \\
d_2\left(\lambda_2^*-\lambda_2 \right)      & d_2\left( \left( b_1-
b_1^*\right)+\left( b_2-b_2^*\right)\right)  \\
0                                                  & d_3\left(\lambda_3^*-
\lambda_3\right)      \\
\vdots                                             & \vdots                                          
\end{array}
\right.\\[3pt]
\left.
\begin{array}{cc}
 d_1\left( b_3-b_3^*\right)         & \cdots \\
d_2\left( \left( b_2-
b_2^*\right)+\left( b_3-b_3^*\right)\right)  & \cdots \\
 d_3\left( \displaystyle\sum\limits_{i=1}^{3} b_{i} - \sum\limits_{i=1}^{3} 
b_{i}^*\right)     & \cdots \\
 \vdots                                  & \ddots
\end{array}
\right).
\end{multline*}


6. В случае конечного пространства ($S \hm< \infty$)
\begin{multline*}
A\left( t\right) ={}\\
\!= \!\!\left(\!
\begin{array}{ccccccc}
a_{00}(t)     & b_1(t)       & b_2(t)    & b_3(t) &\cdots  & b_{S-1}(t)       & 
b_{S}(t) \\
\lambda_1(t)  & a_{11}(t)    & b_1(t)    & b_2(t) &\cdots  & b_{S-2}(t)       & 
b_{S-1}(t) \\
0             & \lambda_2(t) & a_{22}(t) & b_1(t) & \cdots & b_{S-3}(t)       & 
b_{S-2}(t) \\
\vdots        & \vdots       & \vdots    & \vdots & \ddots & \vdots           & 
\vdots \\
0             & 0            & 0         & 0      & \cdots & a_{S-1S-1}(t)    & 
b_1(t) \\
0             & 0            & 0         & 0      & \cdots & \lambda_{S-1}(t) & 
a_{SS}(t)
\end{array}\!\!
\right)\!\!.\hspace*{-8.98357pt}
%\label{f_1.1.8.1}
\end{multline*}


\noindent 
Выполнив необходимые преобразования, получаем произведение~$DC^*$:
\begin{multline*}
%\label{eq_2.6.6}
\left(
\begin{array}{cc}
d_1\left( b_1-b_1^*\right)                    & d_1\left( b_2-b_2^*\right)                  \\
d_2\left(\lambda_2^*-\lambda_2 \right)  &d_2((b_1-b_1^*)+(b_2 -b_2^*))   \\
0                                                  & d_3\left(\lambda_3^*-\lambda_3\right)       \\
\vdots                                             & \vdots                                           \\
0             & 0       
\end{array}
\right.\\[3pt]
\left.
\begin{array}{ccc}
 d_1\left( b_3-b_3^*\right)         & \cdots &d_1\left( b_{S}-b_{S}^*\right) \\
 d_2\left( \left( b_2-b_2^*\right)+\left( b_3-b_3^*\right)\right)  & \cdots &d_2 \displaystyle\sum\limits_{i=S-1}^{S} \left( b_{i} -  b_{i}^*\right) \\
 d_3 \displaystyle \sum\limits_{i=1}^{3} \left(b_{i} - b_{i}^*\right)    & \cdots & d_3\displaystyle\sum\limits_{i=S-2}^{S} \left( b_{i} -  b_{i}^*\right)\\
 \vdots                                  & \ddots                                 & \vdots \\
 0            & \cdots & d_S\displaystyle\sum\limits_{i=1}^{S}  \left(  b_{i} - b_{i}^*\right)
\end{array}
\right).
\end{multline*}


\noindent{\rm IV.}  \textit{Неоднородные цепи, в~которых интенсивность переходов 
не зависит от размера популяции в~этот момент}, т.\,е.\ интенсивность рождения 
группы из $k$ частиц $a_{i+k,i}(t)\hm=a_k(t)$, а
    интенсивность гибели группы из $k$ элементов  $a_{i,i+k}(t)\hm=b_k(t)$  для 
$k\hm\ge 1$;
  кроме того, пусть для каждого~$i$ выполнены неравенства $a_{i} \hm\geq a_{i+1}$ 
  и~$b_{i} \hm\geq b_{i+1}$.

7. Если размерность $S=\infty$, то
\begin{equation*}
%\label{f_1.1.10}
A\left( t\right) = 
 \left(
\begin{array}{ccccc}
a_{00}(t)  & b_1(t)     & b_2(t)     & b_3(t)     &\cdots  \\
a_1(t)     & a_{11}(t)  & b_1(t)     & b_2(t)     &\cdots  \\
a_2(t)     & a_1(t)     & a_{22}(t)  & b_1(t)     & \cdots  \\
\vdots     & \vdots     & \vdots     & \vdots     & \ddots  
\end{array}
\right)\!.
\end{equation*}

\noindent 
Матрица $DC^{*}$ имеет вид:
\begin{equation*}
%\label{eq_2.6.7}
 \left(
\begin{array}{cccc}
d_1\left( a_{11}^*- a_{11}\right)                             & 
d_1\left( b_2-b_2^*\right)                 & d_1\left( b_3-
b_3^*\right)             & \cdots \\
0                                                                  & 
d_2\left( a_{22}^*- a_{22}\right)          & d_2\left( b_2-
b_2^*\right)             & \cdots \\
0                                                                  & 0                                               
& d_3\left( a_{33}^*- a_{33}\right)      & \cdots \\
\vdots                                                             & \vdots                                          
& \vdots                                      & \ddots
 \end{array}
\right).
\end{equation*}

\noindent Можно переписать эту матрицу следующим образом:
\begin{equation*}
%\label{eq_2.6.9}
 \left(
\begin{array}{cccc}
d_1\left( b_1-b_1^*\right)                          & d_1\left( b_2-
b_2^*\right)                 & d_1\left( b_3-b_3^*\right)             & 
\cdots \\
0                                                        & 
d_2\displaystyle\sum\limits_{i=1}^{2} \left(  b_{i} - b_{i}^*\right)          & d_2\left( 
b_2-b_2^*\right)             & \cdots \\
0                                                        & 0                                               
& d_3\displaystyle \sum\limits_{i=1}^{3} \left(  b_{i} - b_{i}^*\right)      & \cdots \\
\vdots                                                   & \vdots                                          
& \vdots                                      & \ddots
 \end{array}
\right).
\end{equation*}

8. В~случае конечного пространства ($S \hm< \infty$) матрица~$A\left( t\right)$ 
имеет вид:
\begin{multline*}
%\label{f_1.1.11}
\left(
\begin{array}{ccccc}
a_{00}(t)  & b_1(t)     & b_2(t)     & b_3(t)     &\cdots   \\
a_1(t)     & a_{11}(t)  & b_1(t)     & b_2(t)     &\cdots  \\
a_2(t)     & a_1(t)     & a_{22}(t)  & b_1(t)     & \cdots \\
\vdots     & \vdots     & \vdots     & \vdots     & \ddots\\
a_{S-1}(t) & a_{S-2}(t) & a_{S-3}(t) & a_{S-4}(t) & \cdots\\
a_{S}(t)   & a_{S-1}(t) & a_{S-2}(t) & a_{S-3}(t) & \cdots
\end{array}
\right.\\
\left.
\begin{array}{ccc}
\cdots  & b_{S-1}(t)     & b_{S}(t) \\
\cdots  & b_{S-2}(t)     & b_{S-1}(t) \\
 \cdots & b_{S-3}(t)     & b_{S-2}(t) \\
 \ddots & \vdots         & \vdots \\
\cdots & a_{S-1,S-1}(t) & b_1(t) \\
 \cdots & a_1(t)         & a_{SS}(t)
\end{array}
\right).
\end{multline*}

\noindent 
Матрица $DC^{*}$ имеет вид:
\begin{multline*}
%\label{eq_2.6.8}
\left(
\begin{array}{cccc}
d_1\left( a_{11}^*- a_{11}\right)              & d_1\left( b_2-
b_2^*\right)                 & d_1\left( b_3-b_3^*\right)             & 
\cdots \\
0                                                   & d_2\left( a_{22}^*- 
a_{22}\right)          & d_2\left( b_2-b_2^*\right)             & \cdots \\
0                                                   & 0                                               
& d_3\left( a_{33}^*- a_{33}\right)      & \cdots \\
\vdots                                              & \vdots                                          
& \vdots                                      & \ddots \\
0             & 0            & 0            & \cdots 
 \end{array}
\right.\\
\left.
\begin{array}{cc}
\cdots & d_1\left( b_S-b_S^*\right)\\
\cdots & 
d_2\left( b_{S-1}-b_{S-1}^*\right)\\
\cdots & d_3\left( b_{S-2}-
b_{S-2}^*\right)\\
 \ddots & \vdots\\
 \cdots & d_S\left( a_{SS}^*- 
a_{SS}\right)
 \end{array}
\right)\!.
\end{multline*}

\noindent 
Как и~в предыдущем случае, можем переписать последнюю матрицу:
\begin{multline*}
%\label{eq_2.6.10}
\left(
\begin{array}{cccc}
d_1\left( b_1-b_1^*\right)     & d_1\cdot\left( b_2-b_2^*\right)                 
& d_1\left( b_3-b_3^*\right)             & \cdots \\
0                                   & d_2\displaystyle\sum\limits_{i=1}^{2} \left(  b_{i} - 
b_{i}^*\right)          & d_2\left( b_2-b_2^*\right)             & \cdots\\
0                                   & 0                                               
& d_3\displaystyle\sum\limits_{i=1}^{3} \left(  b_{i} - b_{i}^*\right)      & \cdots \\
\vdots                              & \vdots                                          
& \vdots                                      & \ddots \\
0                                   & 0                                               
& 0            & \cdots
 \end{array}
\right.\\
\left.
\begin{array}{cc}
\cdots & d_1\left( b_S-b_S^*\right)\\
\cdots & d_2\left( b_{S-1}-b_{S-1}^*\right)\\
 \cdots & d_3\left( b_{S-2}-b_{S-2}^*\right)\\
\ddots & \vdots\\
\cdots & d_S\displaystyle\sum\limits_{i=1}^{S} \left(  b_{i} - b_{i}^*\right)
 \end{array}
\right).
\end{multline*}

Подробное описание классов~I--IV и~преобразований их матриц можно найти в~\cite{Zeifman}.

\begin{figure*}[b] %fig1
\vspace*{8pt}
  \begin{center}  
    \mbox{%
\epsfxsize=163mm
\epsfbox{zei-1.eps}
}
\end{center}
\vspace*{-12pt}
\Caption{Среднее число требований~(\textit{а}) и~вероятность отсутствия требований~(\textit{б})
 в~системе обслуживания при $X(0)\hm = 0$: \textit{1}~--- $(\lambda, \overline{\mu})$; \textit{2}~--- $(\lambda, \underline{\mu})$;
 \textit{3}~--- $(\lambda, \mu)$}
%\end{figure*}
%\begin{figure*} %fig2
\vspace*{3pt}
  \begin{center}  
    \mbox{%
\epsfxsize=163mm
\epsfbox{zei-3.eps}
}
\end{center}
\vspace*{-12pt}
\Caption{Среднее число требований~(\textit{а}) и~вероятность отсутствия требований~(\textit{б})
 в~системе обслуживания при $X(0)\hm = 0$: \textit{1}~--- $(\overline{\lambda},\mu)$; \textit{2}~--- $(\underline{\lambda},\mu)$;
 \textit{3}~--- $(\lambda,\mu)$}
\end{figure*}

\section{Марковские цепи с~интервальными интенсивностями}


Полученные ранее результаты можно применить к~исследованию как скорости 
сходимости, так и~среднего числа требований в~нестационарных марковских системах 
обслуживания.

В качестве примера такого исследования рассмотрим марковскую цепь~$X(t)$, 
опи\-сы\-ва\-ющую чис\-ло требований в~системе обслуживания  $M_t/M_t/S/S$ (см., 
например,~\cite{Doorn2009}). Это процесс рож\-де\-ния и~гибели с~конечным чис\-лом 
со\-сто\-яний и~транспонированной мат\-ри\-цей интенсивностей вида

\noindent
\begin{multline*}
A(t)=\left(
\begin{array}{ccc}
-\lambda(t)  & \mu(t)              & 0  \\
 \lambda(t)  &-(\lambda(t)+\mu(t)) & 2\mu(t)  \\
  0          &\lambda(t)           & -\left(\lambda(t)+2\mu(t)\right) \\
    0          & 0                   & \lambda(t)         \\
 \ddots      & \ddots              & \ddots   \\
 \cdots      & \cdots              & \cdots  \\
\end{array}
\right.\\
\left.
\begin{array}{ccc}
 0                                   & \cdots & \cdots  \\
 0                                  & \cdots  & \cdots \\
3\mu(t)                            & \cdots   & \cdots\\
 -\left(\lambda(t)+3\mu(t)\right) & \cdots  & \cdots\\
\ddots                             & \cdots  & S\mu(t)\\
\ddots                             & \lambda(t) & -S\mu(t)\\
\end{array}
\right).
\end{multline*}

\vspace*{-3pt}

Допустим, что интенсивность поступления запросов $\lambda(t)$ остается 
неизменной,  а~интенсивность обработки запросов $\mu(t)$  варьируется 
в~интервале $\left[\,\underline\mu(t); \overline\mu(t)\right]$. Тогда имеет место 
следующая теорема.

\smallskip

\noindent
\textbf{Теорема~3.}\ \textit{Пусть рассматриваются два марковских процесса:  $\{X_1(t), \ 
t\hm\geq 0 \}$ с~интенсивностями $\lambda(t)$ и~$\underline\mu(t)$ и~$\{X_2(t), \ 
t\hm\geq 0 \}$ с~интенсивностями~$\lambda(t)$ и~$\overline\mu(t)$. Тогда $X_1(t) 
\hm\succ X_2(t)$. При этом в~случае одинаковых начальных условий для среднего числа 
требований в~системе справедливо неравенство $E(X_1(t)) \hm\ge  E(X_2(t))$.
}

\medskip


Отметим, что если пространство состояний марковской цепи конечно, то можно 
оставлять неизменным не <<нулевой>>, а~последний столбец~$A(t)$, т.\,е.\ 
фиксировать можно не интенсивность поступления требований, а~интенсивность 
обслуживания. Если теперь~$\mu(t)$ остается неизменной, а~интенсивность 
поступления требований варьируется в~интервале $\left[\underline\lambda(t); 
\overline\lambda(t)\right]$, то имеет место следующая тео\-рема.

\medskip

\noindent
\textbf{Теорема~4.}\ \textit{Пусть рассматриваются два марковских процесса: $\{X_1(t), \ 
t\hm\geq 0 \}$ с~интенсивностями $\overline\lambda(t)$ и~$\mu(t)$ и~$\{X_2(t), \ 
t\hm\geq 0 \}$ с~интенсивностями $\underline\lambda(t)$ и~$\mu(t)$. Тогда $X_1(t) 
\hm\succ X_2(t)$}.

\smallskip

Рассмотрим теперь конкретные модели $M_t/M_t/S/S$. Положим $S\hm=100$.

На рис.~1 показаны среднее число требований в~системе и~вероятность пустой 
очереди в~ситуации, когда $\lambda(t)\hm= 1{,}5 \hm+ \sin(2\pi t)$, $\mu(t)\hm= 1{,}5\hm + 
\cos(2\pi t)$. Соответственно, выбираем $\underline\mu(t)\hm= 1{,}3\hm + \cos(2\pi t)$ 
и~$\overline\mu(t)\hm= 1{,}7 \hm + \cos(2\pi t)$.






На рис.~2 показаны среднее число требований в~системе и~вероятность пустой 
очереди в~ситуации, когда $\lambda(t)\hm= 1{,}5\hm + \sin(2\pi t)$, $\mu(t)\hm= 1{,}5 \hm+ 
\cos(2\pi t)$. Соответственно, выбираем $\underline\lambda(t)\hm= 1{,}3 \hm+ \cos(2\pi 
t)$,  $\overline\lambda(t)\hm= 1{,}7 \hm + \cos(2\pi t)$.









{\small\frenchspacing
 {\baselineskip=10.5pt
 %\addcontentsline{toc}{section}{References}
 \begin{thebibliography}{9}



\bibitem{Keilson} %1
\Au{Keilson J.,  Kester A.} Monotone matrices and monotone Markov processes~// 
Stoch. Proc. Appl., 1977. Vol.~5. Iss.~3. P.~231--241. 
doi: 10.1016/0304-4149(77)90033-3.

\bibitem{Doorn} %2
\Au{Van Doorn E.\,A.} Preliminaries~// Stochastic monotonicity and queueing 
applications of birth-death processes.~--- Lecture notes in statistics ser.~--- 
New York, NY, USA: Springer, 1981.  Vol.~4. P.~1--10. doi: 10.1007/978-1-4612-5883-4\_1.

\bibitem{Gaudio} %3
\Au{Gaudio J., Amin~S., Jaillet~P.} Exponential convergence rates for 
stochastically ordered Markov processes under perturbation~// Syst. Control 
Lett., 2019. Vol.~133. Art.~104515. 7~p. doi: 10.1016/j.sysconle.2019.104515.


\bibitem{Chitraganti} %4
\Au{Chitraganti S., Aberkane~S.,  Aubrun~C.} Mean square stability of non-homogeneous 
Markov jump linear systems using interval analysis~// European 
Control Conference Proceedings.~--- Piscataway, NJ, USA: IEEE, 2013. P.~3724--3729. 
doi: 10.23919/ECC.2013.6669298.

\bibitem{Xie} %5
\Au{Xie F., Wu B., Hu~Y.,  \textit{et al.}} A~generalized Markov chain model based on 
generalized interval probability~// Sci. China Technol. Sc., 2013. Vol.~56. 
P.~2132--2136. doi: 10.23919/ECC.2013.6669298.


\bibitem{Zeifman}
\Au{Zeifman A., Satin~Ya., Kryukova~A.,  Razumchik~R., Kiseleva~K.,  
Shilova~G.} On the three methods for bounding the rate of convergence for some 
continuous-time Markov chains~// Int. J.~Appl. Math. Comp., 2020. 
Vol.~30. Iss.~2. P.~251--266. doi: 10.34768/amcs-2020-0020.

\bibitem{Doorn2009}
\Au{Van Doorn E.\,A.,  Zeifman~A.\,I.}  On the speed of convergence to 
stationarity of the Erlang loss system~// Queueing Syst., 2009. Vol.~63. 
Iss.~1. P.~241--252. doi: 10.1007/s11134-009-9134-9.

\end{thebibliography}

 }
 }
 

\end{multicols}

\vspace*{-12pt}

\hfill{\small\textit{Поступила в~редакцию 30.03.22}}

\vspace*{6pt}

%\pagebreak

%\newpage

%\vspace*{-28pt}

\hrule

\vspace*{2pt}

\hrule

%\vspace*{-2pt}

\def\tit{ON MONOTONICITY OF~SOME CLASSES OF~MARKOV CHAINS}


\def\titkol{On monotonicity of~some classes of~Markov chains}


\def\aut{Y.\,A.~Satin$^1$, A.\,L.~Kryukova$^1$, V.\,S.~Oshushkova$^2$, and~A.\,I.~Zeifman$^{1,3,4,5}$}

\def\autkol{Y.\,A.~Satin, A.\,L.~Kryukova, V.\,S.~Oshushkova, and~A.\,I.~Zeifman}

\titel{\tit}{\aut}{\autkol}{\titkol}

\vspace*{-15pt}


\noindent
$^1$Department of Applied Mathematics, Vologda State University, 
15~Lenin Str., Vologda 160000, Russian Federation

\noindent
$^2$Innovative People Ltd., 26-28 Leninskaya Sloboda Str., Moscow 115280, Russian Federation


\noindent
$^3$Federal Research Center ``Computer Science and Control'' of the Russian Academy of Sciences, 
44-2~Vavilov\linebreak
$\hphantom{^1}$Str., Moscow 119133, Russian Federation

\noindent
$^4$Vologda Research Center of the Russian Academy of Sciences, 
56A~Gorky Str., Vologda 160014, Russian\linebreak
$\hphantom{^1}$Federation

\noindent
$^5$Moscow Center for Fundamental and Applied Mathematics, 
M.\,V.~Lomonosov Moscow State University,\linebreak
$\hphantom{^1}$1~Leninskie Gory, GSP-1, Moscow 119991, Russian Federation

\def\leftfootline{\small{\textbf{\thepage}
\hfill INFORMATIKA I EE PRIMENENIYA~--- INFORMATICS AND
APPLICATIONS\ \ \ 2022\ \ \ volume~16\ \ \ issue\ 2}
}%
 \def\rightfootline{\small{INFORMATIKA I EE PRIMENENIYA~---
INFORMATICS AND APPLICATIONS\ \ \ 2022\ \ \ volume~16\ \ \ issue\ 2
\hfill \textbf{\thepage}}}

\vspace*{3pt}



\Abste{The authors define a~relation of partial order for Markov chains and study 
conditions of monotonicity for some classes of continuous-time Markov processes. 
The corresponding theorems of monotonicity are formulated. The authors describe 
in detail the classes of processes which satisfy conditions of monotonicity. There are 
a~lot of applications of Markov chains with interval intensities that is why the authors consider it. 
The monotonicity conditions obtained in this paper make it possible to advance in some way in the study 
of Markov processes with interval intensities. Namely, in the present paper, the authors consider as 
an example a~queuing system $M_t/M_t/S/S$ with interval coefficients. The results obtained are 
confirmed by a~computational experiment and illustrated by the corresponding graphs 
of the limiting characteristics.}

\KWE{monotonicity of Markov processes; nonstationary queuing system; Markov chains with interval 
intensities; limit characteristics}

\DOI{10.14357/19922264220204}

%\vspace*{-16pt}

%\Ack
%\noindent
    




\vspace*{-8pt}

  \begin{multicols}{2}

\renewcommand{\bibname}{\protect\rmfamily References}
%\renewcommand{\bibname}{\large\protect\rm References}

{\small\frenchspacing
 {\baselineskip=10.5pt
 \addcontentsline{toc}{section}{References}
 \begin{thebibliography}{9}
 
 \vspace*{-3pt}

\bibitem{2-zei-1} %1
\Aue{Keilson, J., and A.~Kester.}
 1977. Monotone matrices and monotone Markov processes. 
 \textit{Stoch. Proc. Appl.} 5(3):231--241. doi: 10.1016/0304-4149(77)90033-3.
\bibitem{3-zei-1} %2
\Aue{Van Doorn, E.\,A.}
 1981. Preliminaries. \textit{Stochastic monotonicity and queueing applications of birth-death processes}. 
 New York, NY: Springer. 1--10. doi: 10.1007/978-1-4612-5883-4\_1.
 
 \bibitem{1-zei-1} %3
\Aue{Gaudio, J., S.~Amin, and P.~Jaillet.}
 2019. Exponential convergence rates for stochastically ordered Markov processes under 
 perturbation. \textit{Syst. Control Lett.} 133:104515. 7~p. doi: 10.1016/j.sysconle.2019.104515.
 
\bibitem{4-zei-1}
\Aue{Chitraganti, S., S.~Aberkane, and C.~Aubrun.}
 2013. Mean square stability of non-homogeneous Markov jump linear systems using interval analysis. 
 \textit{European Control Conference Proceedings}. 3724--3729. doi: 10.23919/ ECC.2013.6669298.
\bibitem{5-zei-1}
\Aue{Xie, F., B.~Wu, Y.~Hu, \textit{et al.}}
 2013. A~generalized Markov chain model based on generalized interval probability. 
 \textit{Sci. China Technol. Sc.} 56:2132--2136. doi: 10.1007/s11431-013-5285-3.
\bibitem{6-zei-1}
\Aue{Zeifman, A., Ya.~Satin, A.~Kryukova, R.~Razumchik, K.~Kiseleva, and G.~Shilova.}
 2020. On the three methods for bounding the rate of convergence for some continuous-time Markov chains. 
 \textit{Int. J.~Appl. Math. Comp. Sci.} 30(2):251--266. doi: 10.34768/amcs-2020-0020.
\bibitem{7-zei-1}
\Aue{Van Doorn, E.\,A., and A.\,I.~Zeifman.}  2009. On the speed of convergence to stationarity 
of the Erlang loss system. \textit{Queueing Syst.} 63(1):241--252. doi: 10.1007/s11134-009-9134-9.

\end{thebibliography}

 }
 }

\end{multicols}

\vspace*{-6pt}

\hfill{\small\textit{Received March 30, 2022}}   

\Contr

\noindent
\textbf{Satin Yacov A.} (b.\ 1978)~---  Candidate of Science (PhD) in physics and mathematics, 
associate professor,  Department of Applied Mathematics, Vologda State University, 15~Lenin Str., 
Vologda 160000, Russian Federation; \mbox{yacovi@mail.ru}

\vspace*{3pt}
 
\noindent
\textbf{Kryukova Anastasia L.} (b.\ 1980)~--- 
Candidate of Science (PhD) in physics and mathematics, associate professor, Department 
of Applied Mathematics, Vologda State University, 15~Lenin Str., Vologda 160000, Russian Federation; 
\mbox{kryukovaforstudents@gmail.com}

\vspace*{3pt}
 
\noindent
\textbf{Oshushkova Viktorya S.} (b.\ 1996)~--- programmer, Innovative People Ltd., 26-28~Leninskaya Sloboda 
Str., Moscow 115280, Russian Federation; \mbox{oshushonok@yandex.ru}

\vspace*{3pt}
 
\noindent
\textbf{Zeifman Alexander I.} (b.\ 1954)~--- Doctor of Science in physics and mathematics, professor, 
head of department, Vologda State University, 15~Lenin Str., Vologda 160000, Russian Federation; 
senior scientist, Institute of Informatics Problems, Federal Research Center 
``Computer Science and Control'' of the Russian Academy of Sciences, 44-2~Vavilov Str., 
Moscow 119133, Russian Federation; 
principal scientist, Vologda Research Center of the Russian Academy of Sciences, 56A~Gorky Str., 
Vologda 160014, Russian Federation; leading scientist, 
Moscow Center for Fundamental and Applied Mathematics, M.\,V.~Lomonosov Moscow State University, 1~Leninskie Gory, GSP-1, Moscow 119991, Russian Federation
\mbox{a\_zeifman@mail.ru}




\label{end\stat}

\renewcommand{\bibname}{\protect\rm Литература}   