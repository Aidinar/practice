\def\stat{konovalov-razumchik}

\def\tit{СИНТЕЗ УПРАВЛЕНИЯ ДВУМЕРНЫМ СЛУЧАЙНЫМ 
БЛУЖДАНИЕМ С~ЭТАЛОННЫМ СТАЦИОНАРНЫМ 
РАСПРЕДЕЛЕНИЕМ$^*$}

\def\titkol{Синтез управления двумерным случайным 
блужданием с~эталонным стационарным 
распределением}

\def\aut{М.\,Г.~Коновалов$^1$, Р.\,В.~Разумчик$^2$}

\def\autkol{М.\,Г.~Коновалов, Р.\,В.~Разумчик}

\titel{\tit}{\aut}{\autkol}{\titkol}

\index{Коновалов М.\,Г.}
\index{Разумчик Р.\,В.}
\index{Konovalov M.\,G.}
\index{Razumchik R.\,V.}


{\renewcommand{\thefootnote}{\fnsymbol{footnote}} \footnotetext[1]
{Исследование выполнено с~использованием ЦКП <<Информатика>> ФИЦ ИУ РАН при частичной финансовой поддержке РФФИ (проект 20-07-00804).}}


\renewcommand{\thefootnote}{\arabic{footnote}}
\footnotetext[1]{Федеральный исследовательский центр <<Информатика и~управление>> Российской академии наук, 
\mbox{mkonovalov@ipiran.ru}}
\footnotetext[2]{Федеральный исследовательский центр <<Информатика и~управление>> Российской академии наук, 
\mbox{rrazumchik@ipiran.ru}}

\vspace*{-8pt}




   \Abst{Описан конструктивный метод решения новой <<обратной>> задачи управ\-ле\-ния 
случайным блуж\-да\-ни\-ем (\mbox{цепью} Маркова) с~непрерывным ограниченным и~замкнутым двумерным 
множеством со\-сто\-яний, вероятности перехода которой зависят от текущего значения цепи. Цель 
управ\-ле\-ния~--- отыска\-ние такой стратегии, при которой стационарное распределение цепи наиболее 
при\-бли\-же\-но в~определенном смысле к~эталонному. Реализация метода осуществляется с~помощью 
квазиградиентного алгоритма определения оптимальных значений па\-ра\-мет\-ров стратегии, 
основанного на оценках част\-ных производных целевой функции по наблюдениям за фазовой 
траекторией. Приведены чис\-лен\-ные результаты работы алгоритма в~примерах с~прос\-тей\-ши\-ми 
эталонными плотностями. Внедрение полученных результатов в~подвижных объектах (например, 
беспилотных летательных аппаратах) позволяет наделить их спо\-соб\-ностью сто\-ха\-сти\-че\-ско\-го 
автономного движения.}
    
   \KW{управление марковской цепью; непрерывное множество состояний; квазиградиентные 
алгоритмы; беспилотные летательные аппараты}

\DOI{10.14357/19922264220214}
  
%\vspace*{-3pt}


\vskip 10pt plus 9pt minus 6pt

\thispagestyle{headings}

\begin{multicols}{2}

\label{st\stat}
  
  \section{Введение}
  
Статья посвящена изложению избранных результатов численных 
экспериментов с~моделью одного особого и~хорошо известного класса 
управ\-ля\-емых случайных блуж\-да\-ний по ограниченному замкнутому 
множеству. Одномерный случай описан, например, в~[1, 
разд.~2], двумерный случай~--- в~[2]. 

Пусть $n\hm=0, 1, 2, 
\ldots$ и~пусть $x_n\hm\in [0,1]$~--- положение блуж\-да\-юще\-го объекта 
в~момент~$n$ на отрезке $[0,1]$. В~этом положении выбирается 
на\-прав\-ле\-ние движения, т.\,е.\ один из двух доступных отрезков $[0,x_n]$ или 
$[x_n,1]$ с~вероятностями соответственно $s(x_n)$ и~$1\hm- s(x_n)$. Затем, 
если был вы\-бран отрезок $[0,x_n]$, объект переходит в~точку $f(x_n)\hm\in 
[0,x_n]$, иначе в~точку $g(x_n)\hm\in [x_n,1]$. Функции $f(x_n)$ и~$g(x_n)$ 
могут быть как детерминированными, так и~рандомизированными. Не\-смот\-ря 
на прос\-то\-ту, эта модель находит применение в~ряде важ\-ных приложений 
(см., например, [1, разд.~1]). 
  
  С теоретической точки зрения основные вопросы, которые поднимаются 
  в~научной литературе про марковские цепи, подобные $\{x_n, \ n\geq 0\}$, 
касаются условий существования стационарных распределений (и~их 
един\-ст\-вен\-ности), на\-хож\-де\-ния чис\-лен\-ных алгоритмов их расчета или 
замкнутых формул. Судя по пуб\-ли\-ка\-ци\-ям в~на\-уч\-ной периодической печати, 
они изучены если не ис\-чер\-пы\-ва\-юще, то достаточно хорошо. 

Значительно 
меньше информации по <<об\-рат\-ным>> задачам управ\-ле\-ния или, 
 по-дру\-го\-му, задачам синтеза. Пример ее типичной формулировки можно 
дать на приведенном выше примере: найти $s(x_n)$, $f(x_n)$ и~$g(x_n)$, 
которые приводят к~заданному стационарному распределению цепи $\{x_n,\ 
n\geq 0\}$. 
  
  Внимание к~<<обратным>> задачам привлекается проб\-ле\-ма\-ми, с~которыми 
приходится сталкиваться в~на\-сто\-ящее время на практике при применении, 
в~част\-ности, автономных (беспилотных, безэкипажных) летательных аппаратов 
(далее~--- БЛА). Один из вариантов полетного задания БЛА~--- 
пат\-ру\-ли\-ро\-ва\-ние территории\footnote[3]{Всюду в~статье, если речь идет об <<обратных>> 
задачах, подразумевается, что территория является закрытой.}, т.\,е.\ такое движение, при 
котором каж\-дая точ\-ка территории оказывается в~итоге под наблюдением 
(см., например,~[3, 4]). Алгоритмы патрулирования находят применение во 
многих приложениях робототехники, из которых отметим: задачи 
продолжительного картирования, когда необходимо посещать все точ\-ки 
мест\-ности, чтобы обнаруживать ее изменения~[5]; задачи обследования 
территории (обнаружения несанкционированного  
до\-сту\-па/раз\-ме\-ще\-ния), в~которых важ\-но не ос\-тав\-лять 
необследованными какие-либо ее участки~[6, 7]; задачи содействия 
спасательным операциям~[8]. 
  
  В контексте приложений робототехники словосочетание <<патрулирование 
территории>> (наблюдение, исследование, разведка и~т.\,п.)\ в~на\-уч\-ной 
и~технической литературе трактуется по-раз\-но\-му (см.\ обсуждение, 
например, в~[9, разд.~1]). Здесь всюду под ним будет подразумеваться 
движение одного или нескольких БЛА, обес\-пе\-чи\-ва\-ющее посещение ими 
время от времени каж\-дой точ\-ки территории. В~за\-ру\-беж\-ной литературе для 
со\-от\-вет\-ст\-ву\-ющих задач есть устоявшееся на\-зва\-ние~--- Repeated Coverage 
Problems, и~к~на\-сто\-яще\-му времени для них пред\-ло\-же\-но большое чис\-ло 
решений с~привлечением методов все\-воз\-мож\-ных научных дисциплин (см., 
например, обзор в~[9, разд.~2]). Работы~[10--19] поз\-во\-ля\-ют со\-ста\-вить 
пред\-став\-ле\-ние об основных научных сообществах, за\-ни\-ма\-ющих\-ся этой 
проб\-ле\-ма\-ти\-кой\footnote{Отдельно отметим, что в~некоторых возникающих здесь задачах 
больших успехов удается достичь на основе методов идентификации и~синтеза систем 
управ\-ле\-ния~\cite{26-kraz}.}. 

Если не вдаваться в~(порой очень сложные 
и~со\-став\-ля\-ющие основную цен\-ность решений) подробности, наиболее 
упо\-тре\-би\-тель\-ные подходы к~организации периодического покрытия 
территории \mbox{одним} или группой БЛА могут быть отнесены к~одному из двух 
типов. В~одних движения БЛА час\-тич\-но или пол\-ностью спланированы 
заранее (см., например,~\cite{3-kraz, 20-kraz}). Другие подходы отличает 
не\-пред\-ска\-зу\-емость движения БЛА: решение о~на\-прав\-ле\-нии движения 
выбирается случайным образом и/или исходя из текущей  
си\-ту\-а\-ции/по\-ло\-же\-ния (см., например,~\cite{21-kraz, 22-kraz, 23-kraz}). 
К~по\-след\-не\-му типу относится и~подход, основанный на решениях 
<<обратных>> задач. Судя по пуб\-ли\-ка\-ци\-ям в~открытой периодической 
печати, он пока не получил большого распространения. Связано это, главным 
образом, с~тем, что эф\-фек\-тив\-ность раз\-ра\-ба\-ты\-ва\-емых на его основе 
алгоритмов невозможно контролировать так, как это принято в~мировой 
практике. 
%
Действительно, уже по приведенному выше примеру мож\-но 
понять, что при поиске решений <<обратной>> задачи не требуется 
учитывать ограничения, например на среднее время воз\-вра\-ще\-ния (в~точ\-ку), 
сред\-нюю длину пройденного пути за один период патрулирования, 
наибольшее время воз\-вра\-ще\-ния по всем точ\-кам территории и~т.\,п. Но, 
несмотря на это обстоятельство, решения <<обрат\-ных>> задач обладают 
важными достоинствами, когда на первом плане стоят такие понятия, как 
на\-деж\-ность и~устой\-чи\-вость\footnote{Например, к~выходу из строя одного или 
нескольких БЛА или каналов связи.}. По\-лу\-ча\-ющи\-еся алгоритмы оказываются 
пол\-ностью автономными\footnote{В частности, не требуют наличия обратной связи. 
Отметим, что проблема автономности БЛА многогранна и~не ограничивается теми аспектами, 
которые обсуждаются в~данной \mbox{статье}; некоторое представление 
о~ней позволяет составить введение к~\mbox{статье}~\cite{27-kraz}.} и,~при наличии со\-от\-вет\-ст\-ву\-ющих технических 
возможностей\footnote{То есть при наличии в~БЛА средств сбора и~обработки информации 
с~использованием технического зрения и~эхолокации.}, адап\-тив\-ными. 
  
  В следующем разделе дается описание модели\footnote{ Эта модель не нова, 
  в~тех или иных терминах она уже встречалась в~мировой литературе (см., 
например,~\cite[section~5]{24-kraz}).} случайного блуж\-да\-ния по двумерному 
ограниченному замкнутому множеству, которая и~пред\-став\-ля\-ет собой 
пред\-мет внимания данной \mbox{статьи}. Легко видеть, что, фиксируя ее па\-ра\-мет\-ры 
(т.\,е.\ стратегию управ\-ле\-ния) по произволу, из нее можно получать для БЛА 
различные сто\-ха\-сти\-че\-ские автономные алгоритмы 
движения\footnote{Поскольку в~рассматриваемой постановке БЛА отождествляются 
с~точками на плос\-кости, то известная проб\-ле\-ма обеспечения отсутствия столкновений (коллизий) 
(см., 
 например,~\cite{25-kraz}) не требует внимания. Но при этом необходимо предполагать, что при 
реализации предлагаемого решения на практике в~каждый БЛА должен быть встроен локальный 
механизм по разрешению коллизий. Интуитивно понятно, что, поскольку в~результате коллизии 
новые положения БЛА непредсказуемы, их наличие не мешает достижению той цели, которая 
рас\-смат\-ри\-ва\-ет\-ся в~статье.}. Однако остается совершенно неясным то, как 
осуществлять выбор па\-ра\-мет\-ров (т.\,е.\ решать <<обратную>> задачу) таким 
образом, чтобы БЛА время от времени посещал каж\-дую точ\-ку территории. 
Для внесения яс\-ности в~этот вопрос предполагается, что в~полетном задании 
БЛА должна быть указана целевая относительная час\-то\-та (далее~--- эталонная 
плот\-ность) посещения любого ее участ\-ка. Тогда искомой стратегией следует 
считать ту, что минимизирует (ка\-кое-ли\-бо) отклонение плот\-ности 
стационарного распределения случайного блуж\-да\-ния от эталонной 
плот\-ности. Если ограничиться множеством стратегий, параметризованных 
конечномерными наборами чис\-ло\-вых па\-ра\-мет\-ров, то ис\-ко\-мую стратегию 
можно най\-ти приближенно. 

В~разд.~3 крат\-ко описывается обоб\-ща\-ющий 
результаты~[1] подход к~решению, который, по сути, заключается 
в~использовании градиентного метода минимизации функции нескольких 
переменных. Тео\-ре\-ти\-че\-ский анализ схо\-ди\-мости со\-от\-вет\-ст\-ву\-юще\-го алгоритма 
требует отдельной статьи и~здесь не рас\-смат\-ри\-ва\-ет\-ся.

 В~разд.~4 приводятся 
результаты вы\-чис\-ли\-тель\-ных экспериментов с~несколькими эталонными 
плотностями. 

За\-клю\-чи\-тель\-ный раз\-дел по\-свя\-щен краткому об\-суж\-де\-нию 
результатов и~перспектив дальнейшей раз\-ра\-бот\-ки темы. 
  
  \section{Постановка задачи}
  
  Рассматривается управ\-ля\-емая случайная по\-сле\-до\-ва\-тель\-ность, элементы 
которой принимают значения (со\-сто\-яния) из множества 
$$
X= \left\{x= 
(x_1, x_2): 0\hm\leq x_i\hm\leq 1,\ i\hm=1,2\right\}.
$$
 Множество управ\-ле\-ния 
содержит 4~элемента:
$$
U= \left\{ u^{(1)}, u^{(2)}, u^{(3)}, 
u^{(4)}\right\}.
$$
 Сама по\-сле\-до\-ва\-тель\-ность обозначается $x(n)\hm= \left( 
x_1(n), x_2(n)\right)$, $n\hm=0,1,2,\ldots$, а~по\-сле\-до\-ва\-тель\-ность 
управ\-ле\-ний~--- $u(n)$, $n\hm=1,2,\ldots$ Со\-сто\-яние $x(0)$ в~начальный 
момент произвольно. Переход из со\-сто\-яния $x(n\hm-1)$ в~со\-сто\-яние~$x(n)$ 
зависит от управ\-ле\-ния~$u(n)$ и~подчиняется сле\-ду\-ющим правилам. 
  
  Пусть $\theta_n$, $n\hm=1,2,\ldots,$~--- по\-сле\-до\-ва\-тель\-ность независимых 
случайных величин, распределенных равномерно на отрезке $[0,\Theta]$, 
$0\hm< \Theta\leq 1$. Тогда
  \begin{itemize}
  \item если $u(n)\hm=u^{(1)}$ (сдвиг по оси~$X_1$ в~положительном 
на\-прав\-ле\-нии), то $x_1(n)\hm=x_1(n\hm-1)\hm+\theta(1\hm- x_1(n-1))$; 
$x_2(n)\hm= x_2(n\hm-1)$;
  
  \item
  если $u(n)\hm= u^{(2)}$ (сдвиг по оси~$X_1$ в~отрицательном 
на\-прав\-ле\-нии), то $x_1(n)\hm= x_1(n\hm-1) \hm- \theta x_1(n\hm-1)$; 
$x_2(n)\hm= x_2(n\hm-1)$;
  
  \item
  если $u(n)\hm= u^{(3)}$ (сдвиг по оси~$X_2$ в~положительном 
на\-прав\-ле\-нии), то $x_1(n) \hm= x_1(n\hm-1)$; $x_2(n)\hm= x_2(n\hm-1) 
\hm+\theta (1\hm-x_2(n\hm-1))$;
  
  \item
  если $u(n)=u^{(4)}$ (сдвиг по оси~$X_1$ в~отрицательном на\-прав\-ле\-нии), 
то $x_1(n)\hm= x_1(n\hm-1)$; $x_2(n)\hm= x_2(n\hm-1)\hm- \theta x_1(n\hm-
1)$. 
  \end{itemize}
  
  Зададим множество~$S$ четырехкомпонентных функций вида $s\hm= \left( 
s^{(1)}, \ldots , s^{(4)}\right): X\hm\to [0,1]$, причем каждая компонента 
определена на всем множестве~$X$ и~для всех~$x$ выполняется 
соотношение
  \begin{equation}
  \sum\limits^4_{k=1} s^{(k)}(x)=1\,.
  \label{e1-kraz}
  \end{equation}
  
  Функцию $s\hm\in S$ мож\-но использовать для выбора управ\-ле\-ния из 
множества~$U$: если процесс находится в~со\-сто\-янии~$x$, то 
управ\-ле\-ние~$u^{(k)}$ выбирается с~ве\-ро\-ят\-ностью~$s^{(k)}(x)$. 
  
  Если в~процессе управ\-ле\-ния применяется одна и~та же функция~$s$, то 
будем говорить об использовании одноименной стратегии~$s$. Получаем 
множество (однородных марковских) стратегий, за которым оставим так\-же 
обозначение~$S$. 
  
  При фиксированной стратегии $s\hm\in S$ процесс $x(n)$ пред\-став\-ля\-ет 
собой марковскую цепь. Легко понять, что переходные плотности~$q^{(k)}$ 
при переходе из со\-сто\-яния~$x$ и~при условии, что выбрано 
управ\-ле\-ние~$u^k$, имеют сле\-ду\-ющий вид: 
  \begin{align*}
  q^{(1)} (x,y) &= \begin{cases}
  \Theta^{-1}(1-x_1)^{-1}\,, & \!\!\mbox{если }\\
  & \hspace*{-58pt}0\leq x_1\leq y_1\leq x_i(1+\Theta)  
+\Theta\,,\\
& \hspace*{12mm} 0\leq y_2\leq 1\,;\\
  0 & \!\!\!\!\!\mbox{в\ остальных\ случаях};
  \end{cases}
  \\
  q^{(2)}(x,y) &= \begin{cases}
  (\Theta x_1)^{-1}\,, & \!\!\mbox{если } x_1(1-\Theta)\leq y_1\leq x_1\,,\\
  & \hspace*{22mm}0\leq  y_2\leq 1\,;\\
  0 & \!\!\mbox{в\ остальных\ случаях};
  \end{cases}\\
  q^{(3)}(x,y) &= \begin{cases}
  \Theta^{-1} (1-x_2)^{-1}\,, & \!\!\mbox{если } 0\leq y_1\leq 1\,,\\
  & \hspace*{-16mm}0\leq x_2\leq  y_2\leq x_i(1+\Theta)+\Theta\,;\\
  0 & \!\!\!\!\!\mbox{в\ остальных\ случаях};
  \end{cases}
  \\
  q^{(4)}(x,y) &= \begin{cases}
  (\Theta x_2)^{-1}\,, & \!\!\mbox{если } 0\leq y_1\leq 1\,,\\
  & \hspace*{7mm}x_2(1-\Theta) \leq  y_2\leq x_2\,;\\
  0 & \!\!\mbox{в\ остальных\ случаях};
  \end{cases}
  \end{align*}
  
  Безусловная переходная плотность $p(x,y)$ вероятности перехода из 
состояния~$x$ в~$y$ равна 
$$
p(x,y)= \sum\limits^4_{k=1} s^{(k)}(x) 
q^{(k)}(x,y),
$$
 а~переходная плот\-ность за~$n$~шагов задается 
соотношениями
 \begin{align*}
 p^1(x,y) &=p(x,y);\\
p^{n+1}(x,y)&=\!\int\limits_X\!\! p^n(x,z) p(z,y)\,dz\,,\enskip
 n>1\,.
\end{align*}
Заметим, что переходные вероятности $p(x,y)$ и~$p^{n+1}(x,y)$ зависят от 
выбора стратегии. 
  
  Разобьем множество $X$ со\-сто\-яний цепи на не\-пе\-ре\-се\-ка\-ющи\-еся 
подмножества: 
$$
X= \mathop{\bigcup}\limits^M_{m=1} X_m;\enskip 
X_l\mathop{\bigcap} X_m =\emptyset,\enskip l\not= m\,,
$$
 и~обозначим через~$S_M$ множество  
ку\-соч\-но-по\-сто\-ян\-ных функций вида 
$$
s(x)= \sum\limits^M_{m=1} 
s_m(x)\mathbf{1}_m(x),
$$
 где $s_m\hm= \left( s_m^{(1)}, \ldots , 
s_m^{(4)}\right)$, $0\hm\leq s_m^k \hm\leq 1$; $\mathbf{1}_m$~--- 
индикаторная функция множества~$X_m$. Применение неизменной 
функции $s\hm\in S_M$ в~течение всего процесса управ\-ле\-ния означает, что 
в~любом со\-сто\-янии $x\hm\in X_m$ с~ве\-ро\-ят\-ностью~$s_m^{(k)}$ выбирается 
управ\-ле\-ние~$u^{(k)}$. Таким образом, можно говорить о~множестве 
(однородных марковских) стратегий~$S_M$. Очевидно, $S_M\hm\subset S$. 
  
  Предположим, что для любой стратегии $s\hm\in S_M$ переходная 
плот\-ность имеет стационарную плот\-ность~$\pi$, так что выполняется 
равенство 
$$
\pi(y)= \int\limits_X \pi(x) p(x,y)\,dx\,.
$$
 Пусть задана некоторая 
<<эталонная>> плот\-ность~$\hat{\pi}$. Цель управ\-ле\-ния заключается 
в~нахождении такой стратегии $s\hm\in S_M$, которой соответствует 
стационарное распределение~$\pi$, <<близкое>> 
к~распределению~$\hat{\pi}$. В~качестве <<меры бли\-зости>> плотностей 
выберем величину 
$$
w= \int\limits_X (\pi(x)\hm- \hat{\pi}(x))^2\,dx\,. 
$$
Таким образом, требуется отыскать стратегию~$s$ из заданного 
па\-ра\-мет\-ри\-зо\-ван\-но\-го множества, которая минимизирует функцию~$w$. 
  
  \section{Метод решения}
  
  С учетом соотношения~(1) из вектора~$s$   мож\-но удалить одну из компонент, 
например компоненту~$s^{(4)}$. Таким образом, плот\-ность~$\pi$ 
и~функция~$w$ зависят от $3M$ независимых переменных и~речь идет 
фактически о~минимизации функции
  \begin{multline*}
  \tilde{w}\left( s_1^{(1)}, s_1^{(2)}, s_1^{(3)}; \ldots ; s_M^{(1)}, s_M^{(2)}, 
s_M^{(3)}\right)={}\\
  {}= w\left( s_1^{(1)}, s_1^{(2)}, s_1^{(3)}, 1-s_1^{(1)}-s_1^{(2)}-s_1^{(3)}; 
\ldots \right.\\
\left.\ldots , s_M^{(1)}, s_M^{(2)}, s_M^{(3)}, 1-s_M^{(1)}-s_M^{(2)}- 
s_M^{(3)}\right)
  \end{multline*}
на множестве
\begin{multline*}
\tilde{S}_M= \left\{ s=\left( \left( s_1^1,s_1^2,s_1^3\right), \ldots , \left( s_M^1, 
s_M^2, s_M^3\right)\right):\right.\\ 
\left.s_m^1+s_m^2+s_m^3\leq 1,\ 0\leq s_m^l \leq 
1\right\}.
\end{multline*}
  
  Пусть процесс $x(n)$ управ\-ля\-ет\-ся согласно некоторой стратегии 
$\sigma\hm= \left( \sigma(1), \ldots , \sigma(n), \ldots \right)$ сле\-ду\-юще\-го вида. 
Элемент $\sigma(n)$ по\-сле\-до\-ва\-тель\-ности~$\sigma$ является функцией со 
значениями в~множестве~$\tilde{S}_M$ и~\mbox{пред\-став\-ля\-ет} собой правило 
выбора управ\-ле\-ния $u(n)$ в~момент~$n$. Это означает, что если 
$\sigma(n)\hm= s\hm\in \tilde{S}_M$ и~$x(n)\hm\in X_m$, то вероятность 
события $\{ u(n)\hm= u^{(l)}\}$ равна~$s_m^{(k)}(x(n))$ для $l\hm= 1,2,3$, 
а~вероятность события $\{ u(n)\hm= u^{(4)}\}$ равна $1\hm- 
\sum\nolimits^3_{k=1} s_m^{(l)}(x(n))$. Функции $\sigma(n)$ зависят, вообще 
говоря, от всей предыс\-то\-рии, и,~следовательно, стратегия~$\sigma$ не 
является ни марковской, ни однородной. Она пред\-став\-ля\-ет собой алгоритм, 
который осуществляет такую трансформацию элементов~$\sigma(n)$, что 
с~увеличением~$n$ их значения приближаются к~точ\-ке минимума 
функции~$\tilde{w}$ на множестве~$S_M$. Механизм трансформации 
основан на методе проекции градиента, и~ключевое значение имеет формула 
для част\-ных производных функции~$\tilde{w}$. 
  
  Пусть процесс $x(n)$ управ\-ля\-ет\-ся согласно стратегии $s\hm\in 
\tilde{S}_M$ и~пусть~$m$ и~$l$~--- фиксированные натуральные чис\-ла, 
$1\hm\leq m\hm\leq M$, $1\hm\leq l\hm\leq 3$. Имеет место равенство:
  \begin{multline}
  \fr{\partial \tilde{w}}{\partial s_m^l} =2\sum\limits^\infty_{n=0} 
\int\limits_{\,X_m} \pi(x)\int\limits_X \left( q^{(l)}(x,y) -g^{(4)} (x,y)\right)\times{}\\
{}\times \int\limits_X 
p^n(y,z)\left( \pi(z)- \hat{\pi}(z)\right) dzdydx\,.
  \label{e2-kraz}
  \end{multline}

Аналог формулы~(\ref{e2-kraz}) для одномерного управ\-ля\-емо\-го случайного 
блуждания приведен в~[1]. Стоит отметить, что формула~(\ref{e2-kraz}) очень 
<<похожа>> на формулу для част\-ных производных предельного сред\-не\-го 
дохода в~задаче об управ\-ле\-нии марковской цепью с~непрерывным 
множеством со\-сто\-яний~\cite{28-kraz};\linebreak та, в~свою очередь, аналогична 
пред\-став\-ле\-нию градиента целевой функции в~той же задаче, но со счетным 
множеством со\-сто\-яний. В~то же время рас\-смат\-ри\-ва\-емая задача отличается от 
классической\linebreak схемы марковского процесса принятия решений (хотя бы по 
причине отсутствия одношагового дохода). Тем не менее вывод 
формулы~(\ref{e2-kraz}), который здесь не приводится, во многом повторяет 
аналогичное доказательство для классической задачи со счет\-ным 
множеством со\-сто\-яний~\cite{29-kraz}. 

\begin{figure*}[b] %fig1
\vspace*{3pt}
  \begin{center}  
    \mbox{%
\epsfxsize=163mm
\epsfbox{kon-1.eps}
}
\end{center}
\vspace*{-9pt}
\Caption{Равномерная в~полосе эталонная плот\-ность~(\textit{а}), ее оценка~(\textit{б}), 
мгновенное положение~100 еще не обучив\-ших\-ся БЛА~(\textit{в}) и~мгновенное 
положение~100~обучив\-ших\-ся БЛА~(\textit{г})}
%\end{figure*}
%\begin{figure*} %fig2
\vspace*{18pt}
  \begin{center}  
    \mbox{%
\epsfxsize=163mm
\epsfbox{kon-2.eps}
}
\end{center}
\vspace*{-9pt}
\Caption{Неравномерная в~полосе эталонная плот\-ность~(\textit{а}), ее оценка~(\textit{б}), 
мгновенное положение 100 еще не обучив\-ших\-ся БЛА~(\textit{в}) и~мгновенное положение 
100~обучив\-ши\-хся БЛА~(\textit{г}). Более насыщенный цвет соответствует большему 
значению плот\-ности}
\end{figure*}
  
  
  Прямые вычисления по формуле~(\ref{e2-kraz}) невозможны, поскольку, 
несмотря на относительно простой вид переходной плот\-ности, аналитическое 
пред\-став\-ле\-ние для стационарной плот\-ности~$\pi$ и,~соответственно, для 
функции~$\tilde{w}$ и~ее част\-ных производных неизвестно. Приходится 
поэтому строить оценки част\-ных производных по наблюдениям за 
траекторией управ\-ля\-емой по\-сле\-до\-ва\-тель\-ности. Для этого используется 
сле\-ду\-ющая интерпретация формулы~(\ref{e2-kraz}). Рас\-смот\-рим 
процесс~$x(n)$, для которого начальное со\-сто\-яние есть $x(0)\hm= x\hm\in 
X_m$, первое управ\-ле\-ние~--- $u(1)\hm= u^{(k)}$, $k\hm=1,\ldots , 4$, а~затем 
неизменно применяются правила стратегии~$s$. Обозначим через 
$\mathsf{M}_m^{(k)}(x)[\cdot ]$ математическое ожидание, по\-рож\-да\-емое 
таким процессом, и~положим $\gamma(x)\hm=2(\pi (x)\hm- \hat{\pi}(x))$. 
Тогда, как легко видеть,
  \begin{multline}
  \mathsf{M}_m^{(k)}(x) [\gamma(x(n))] ={}\\
  {}=\int\limits_X q^{(k)} (x,y) 
\int\limits_X p^n(y,z) \gamma(z)\,dzdy\,,
  \label{e3-kraz}
  \end{multline}
а формула~(\ref{e2-kraz}) приобретает вид:
$$
\fr{\partial\tilde{w}}{\partial s_m^l} =\sum\limits^\infty_{n=0} \left( 
g_m^{(l)}(n)- g_m^{(4)}(n)\right)\,,
$$
где 
$$
g_m^{(k)}(n)= \int\limits_{X_m} \pi(x) \mathsf{M}_m^{(k)} (x) \left[ 
\gamma(x(n))\right] dx\,,\enskip k=1,\ldots , 4\,.
$$
 Заметим, что если интерпретировать величину $\gamma(x(n))$ как 
<<одношаговый доход>>, то величина $g_m^{(k)}(n)$ будет означать 
усредненный по предельному распределению средний доход 
спустя $n$~\mbox{так\-тов} после того, как в~момент на\-хож\-де\-ния во множестве 
со\-сто\-яний~$X_m$ было применено управ\-ле\-ние~$u^{(k)}$.
{\looseness=1

} 


  Оценки величин $g_m^{(k)}(n)$ строятся с~помощью техники скользящих 
средних (оцен\-ки с~забыванием). Такой подход позволяет одновременно 
оценивать предельное распределение, со\-от\-вет\-ст\-ву\-ющее фиксированной 
точ\-ке $s\hm\in S_M$, и~осуществлять поиск на множестве~$S_M$. 
  
  \section{Численные примеры}
  
  \vspace*{-6pt}
  
  Предположим, что эталонная плотность $\hat{\pi}$ равномерна в~полосе 
$0{,}4\hm\leq x\hm+y \hm\leq 0{,}8$ (рис.~1,\,\textit{а}), т.\,е.\ 
$\hat{\pi}(x,y)\hm= 4{,}165 I_{\{ 0{,}4\leq x+y\leq 0{,}8\}}$. Стационарная 
плот\-ность~$\pi$, полученная в~результате применения предложенного 
в~предыду\-щем разделе метода, изоб\-ра\-же\-на на рис.~1,\,\textit{б}. 



  Наблюдаемое на рис.~1,\,\textit{а} и~1,\,\textit{б} совпадение плотностей 
иллюстрирует достижение по\-став\-лен\-ной цели управ\-ле\-ния. Допуская 
некоторую воль\-ность речи, можно сказать, что такова типичная картина при 
любой (хорошей) эталонной плот\-ности. Например, оценка~$\pi$ 
неравномерной в~полосе эталонной плот\-ности $\hat{\pi}(x,y)\hm= 13{,}709 
(0{,}25x \hm- y \hm+ 0{,}25)^+$ по предложенному методу изоб\-ра\-же\-на на 
рис.~2,\,\textit{б}. 
  

 
  Полноценный трехмерный образ оценки~$\pi$ для нормальной эталонной 
плот\-ности $\hat{\pi}(x,y) \hm= 4{,}41 e^{-(x-0{,}4)^2/0{,}16 - (y-0{,}6)^2/0{,}04}$ 
дан на рис.~3,\,\textit{б}. Вместе с~линиями уровня на 
рис.~3,\,\textit{в} и~3,\,\textit{г} он дает воз\-мож\-ность визуально оценить 
степень бли\-зости по\-верх\-ности~$\pi$ к~эталонной. 
  
  
  Как стационарным плотностям на рис.~1,\,\textit{б} и~2,\,\textit{б}, так 
и~любой плот\-ности~$\pi$ соответствуют вполне\linebreak\vspace*{-12pt}

\pagebreak

\end{multicols}

 \begin{figure*} %fig3
  \vspace*{1pt}
  \begin{center}  
    \mbox{%
\epsfxsize=159.939mm
\epsfbox{kon-3.eps}
}
\end{center}
\vspace*{-9pt}
  \Caption{Нормальная эталонная плот\-ность~(\textit{а}), ее оценка~(\textit{б}) и~линии 
уровня~(\textit{в}, \textit{г})}
  \end{figure*}

\begin{multicols}{2}

\noindent
 определенные стратегии 
$s\hm= \left( s^{(1)}, \ldots s^{(4)}\right)$. Вернемся к~содержательной 
постановке задачи, в~соответствии с~которой значение цепи в~каж\-дый 
момент времени~--- это координаты БЛА. Реализовав в~нем на аппаратном 
уровне либо предложенное в~разд.~3 решение, либо уже <<готовую>> 
стратегию~$s$, его мож\-но отправить в~автономный полет (по правилам, 
изложенным в~начале разд.~2) на заданной заранее высоте. Тогда его 
сто\-ха\-сти\-че\-ское поведение будет <<следовать>> плот\-ности~$\pi$. На 
рис.~1,\,\textit{в} и~1,\,\textit{г} (аналогично рис.~2,\,\textit{в}, 2,\,\textit{г}) мож\-но видеть 
сделанные в~случайные моменты времени мгновенные снимки 
соответственно 100~единиц еще не обучив\-ших\-ся и~уже обучив\-ших\-ся БЛА на 
фоне эталонных плотностей. 

\vspace*{-6pt}
  
  \section{Заключение}
  
  \vspace*{-2pt}
  
  Рассмотренная в~статье задача относится к~теории управ\-ле\-ния 
многомерными марковскими цепями с~непрерывным множеством состояний. 
Предложенное решение представляет собой градиентный алгоритм 
коррекции па\-ра\-мет\-ров стратегии, причем оценки производных целевой 
функции строятся по результатам наблюдений. Его внедрение в~БЛА дает 
пол\-ностью автономный аппарат, со\-вер\-ша\-ющий выглядящие случайными 
движения, но в~итоге приводящие к~заранее обозначенной (см.\ разд.~1) 
цели. 
  
  Выявленная в~\cite{1-kraz} и~под\-тверж\-ден\-ная пред\-став\-лен\-ны\-ми здесь 
результатами эф\-фек\-тив\-ность градиентного подхода к~созданию автономных 
алгоритмов движения дает основание продолжать \mbox{исследо\-ва\-ния} в~этом 
направлении. В~тео\-ре\-ти\-че\-ском плане, безусловно, важен вопрос схо\-ди\-мости 
алгоритмов, а~так\-же вопрос обоснования метода на произвольные 
распределения без предположения о~существовании плотностей. 
В~практическом же отношении интерес представляет разработка такой 
модели случайного блуж\-да\-ния, которая учитывает возможное наличие 
каналов связи между БЛА, и~со\-от\-вет\-ст\-ву\-юще\-го алгоритма оцен\-ки 
па\-ра\-мет\-ров об\-щей стратегии. 

\vspace*{-6pt}
  
{\small\frenchspacing
 {%\baselineskip=10.8pt
 %\addcontentsline{toc}{section}{References}
 \begin{thebibliography}{99}
 
 \vspace*{-2pt}
 
 
\bibitem{1-kraz}
\Au{Коновалов М.\,Г., Разумчик~Р.\,В.} Управление случайным блужданием с~эталонным 
стационарным распределением~// Информатика и~её применения, 2018. Т.~12. Вып.~3. 
С.~2--13. 
\bibitem{2-kraz}
\Au{Коновалов М.\,Г., Коновалова~И.\,Н., Разумчик~Р.\,В.} Управ\-ле\-ние двумерной 
марковской \mbox{цепью} с~непрерывным ограниченным множеством со\-сто\-яний, приводящее 
к~заданному стационарному распределению~//  
Ин\-фор\-ма\-ци\-он\-но-те\-ле\-ком\-му\-ни\-ка\-ци\-он\-ные технологии и~математическое 
моделирование высокотехнологичных систем.~--- М.: РУДН, 2020. С.~276--279. 
\bibitem{3-kraz}
\Au{Elmaliach Y., Noa~A., Kaminka~G.} Multi-robot area patrol under frequency constraints~// 
Ann. Math. Artif. Intel., 2009. Vol.~57. No.\,3-4. P.~293--320. 
\bibitem{4-kraz}
\Au{Elor Y., Bruckstein~A.} Autonomous multi-agent cycle based patrolling~//  
 Swarm intelligence~/ Eds. M.~Dorigo, M.~Birattari, G.\,A.~Di Caro, \textit{et al.}~--- Lecture notes in 
computer science ser.~--- Springer, 2010. Vol.~6234. P.~119--130. 
\bibitem{5-kraz}
\Au{LaValle S., Hinrichsen~J.} Visibility-based pursuit-evasion: The case of curved 
environments~// IEEE T. Robotic. Autom., 2001. Vol.~17. No.\,2. P.~196--202. 
\bibitem{6-kraz}
\Au{Gerkey B.\,P., Thrun~S., Gordon~G.} Visibility-based pursuitevasion with limited field of 
view~// Int. J.~Robot. Res., 2006. Vol.~25. No.\,4. P.~299--315. 
\bibitem{7-kraz}
\Au{Fazli P., Davoodi~A., Pasquier~P., Mackworth~A.\,K.} Complete and robust cooperative 
robot area coverage with limited range~// Conference (International) on 
Intelligent Robots and Systems Proceedings.~--- Piscataway, NJ, USA: IEEE, 2010. P.~5577--5582. 
\bibitem{8-kraz}
\Au{Jennings J., Whelan~G., Evans~W.} Cooperative search and rescue with a team of mobile 
robots~// 8th Conference (International) on Advanced Robotics Proceedings.~--- Piscataway, NJ, 
USA: IEEE, 1997. P.~193--200.
\bibitem{9-kraz}
\Au{Fazli P., Davoodi~A., Mackworth~A.\,K.} Multi-robot repeated area coverage: Performance 
optimization under various visual ranges~// 9th Conference on Computer and Robot Vision 
Proceedings.~--- Piscataway, NJ, USA: IEEE, 2012. P.~298--305. 
\bibitem{10-kraz}
\Au{Carlsson S., Nilsson~B.\,J., Ntafos~S.\,C.} Optimum guard covers and m-watchmen routes 
for restricted polygons~// Int. J.~Comput. Geom. Ap., 1993. Vol.~3. No.\,1. 
P.~85--105. 

\bibitem{13-kraz} %11
\Au{Toth P., Vigo~D.} The vehicle routing problem.~--- Philadelphia, PA, USA: Society for 
Industrial Mathematics, 2002. 358~p. 

\bibitem{11-kraz} %12
\Au{Machado A., Ramalho~G., Zucker~J.\,D., Drogoul~A.} Multiagent patrolling: An empirical 
analysis of alternative architectures~// Multi-agent-based 
simulation~II~/ Eds. J.~Sim$\tilde{\mbox{a}}$o Sichman, F.~Bousquet, P.~Davidsson.~--- 
Lecture notes in computer science ser.~--- Springer, 2003. Vol.~2158. P.~155--170. 

\bibitem{12-kraz} %13
\Au{Gasparri A., Krishnamachari~B., Sukhatme~G.} A~framework for multi-robot node 
coverage in sensor networks~// Ann. Math. Artif. Intel., 2008. Vol.~52. 
No.\,2. P.~281--305. 



\bibitem{14-kraz} %14
\Au{Каляев И.\,А., Гайдук~А.\,Р., Капустян~С.\,Г.} Модели и~алгоритмы коллективного 
управ\-ле\-ния в~группах роботов.~--- М.: Физматлит, 2009. 280~с. 





\bibitem{17-kraz} %15
\Au{Birk A., Wiggerich~B., Bulow~H., Pfingsthorn~M., Schwertfeger~S.} Safety, security, and 
rescue missions with an unmanned aerial vehicle (UAV)~// J.~Intell. Robot. Syst., 
2011. Vol.~64. No.\,1. P.~57--76. 

\bibitem{19-kraz} %16
\Au{Дивеев А.\,И., Шмалько~Е.\,Ю., Рындин~Д.\,А.} Решение задачи оптимального 
управ\-ле\-ния группой роботов эволюционными алгоритмами~// Информационные 
и~математические технологии в~науке и~управ\-ле\-нии, 2017. Вып.~3(7). С.~109--126. 

\bibitem{16-kraz} %17
\Au{Пшихопов В.\,Х., Медведев~М.\,Ю.} Групповое управ\-ле\-ние движением мобильных 
роботов в~неопределенной среде с~использованием неустойчивых режимов~// Труды 
\mbox{СПИИРАН}, 2018. №\,60. С.~39--63. 

\bibitem{18-kraz} %18
\Au{Arbanas B., Ivanovic~A., Car~M., Orsag~M., Petrovic~T., Bogdan~S.} Decentralized 
planning and control for UAV--UGV cooperative teams~// Auton. Robot., 2018. Vol.~42. 
No.\,8. P.~1601--1618. 

\bibitem{15-kraz} %19
\Au{Сенотов В.\,Д., Алисейчик~А.\,П., Павловский~Е.\,В., Подопросветов~А.\,В., 
Орлов~И.\,А.} Алгоритмы стайного децентрализованного управ\-ле\-ния движением группы 
роботов с~дифференциальным приводом~// Препринты ИПМ им.\ М.\,В.~Келдыша, 2020. 
Вып.~123. 39~с. 

\bibitem{26-kraz} %20
\Au{Кибзун А.\,И., Синицын~И.\,Н.} Современные проб\-ле\-мы тео\-рии оптимизации 
стохастических сис\-тем~// Автоматика и~телемеханика, 2020. Вып.~11. С.~3--10. 

\bibitem{20-kraz} %21
\Au{Portugal D., Rocha~R.} MSP algorithm: Multirobot patrolling based on territory allocation 
using balanced graph partitioning~//  Symposium on Applied Computing Proceedings.~--- 
New York, NY, USA: ACM, 2010. P.~1271--1276.



\bibitem{22-kraz} %22
\Au{Reif J., Wang~H.} Social potential fields: A~distributed behavioral control for autonomous 
robots~// Robot. Auton. Syst., 1999. Vol.~27. No.\,3. P.~171--194. 
\bibitem{23-kraz} %23
\Au{Chu H.\,N., Glad~A., Simonin~O., Sempe~F., Drogoul~A., Charpillet~F.} Swarm 
approaches for the patrolling problem, information propagation vs. pheromone evaporation~// 
19th  Conference (International) on Tools with Artificial Intelligence Proceedings.~--- Los 
Alamitos, CA, USA: IEEE, 2007. Vol.~1. P.~442--449. 

\bibitem{21-kraz} %24
\Au{Shvets E.\,A., Nikolaev~D.\,P.} Complex approach to long-term multi-agent mapping in low 
dynamic environments~// Proc. SPIE,  2015. Vol.~9875. Art. 98752A. 10~p. doi: 10.1117/12.2228708.

\bibitem{27-kraz} %25
\Au{Миллер Б.\,М., Степанян~К.\,В., Попов~А.\,К., Миллер~А.\,Б.} Навигация БПЛА на 
основе последовательностей изоб\-ра\-же\-ний, ре\-гист\-ри\-ру\-емых бортовой видеокамерой~// 
Автоматика и~телемеханика, 2017. Вып.~12. С.~141--154. 

\bibitem{24-kraz} %26
\Au{Ramli M.\,A., Leng~G.} The stationary probability density of a class of bounded Markov 
processes~// Adv. Appl. Probab., 2010. Vol.~42. P.~986--993. 
\bibitem{25-kraz} %27
\Au{Dotsenko A., Diveev~A., Cevallos~J.\,P.\,C.} Collision avoidance at swarm regrouping 
using modified network operator method with various number of arguments~// 14th Conference 
on Industrial Electronics and Applications Proceedings.~--- Piscataway, NJ, USA: IEEE, 2019. 
P.~768--773. 



\bibitem{28-kraz}
\Au{Коновалов М.\,Г., Разумчик~Р.\,В.} Диспетчеризация в~сис\-те\-ме с~параллельным 
обслуживанием с~по-\linebreak\vspace*{-12pt}

\pagebreak

\noindent
мощью распределенного градиентного управ\-ле\-ния марковской 
цепью~// Информатика и~её применения, 2021. Т.~15. Вып.~3. С.~41--50. 
\bibitem{29-kraz}
\Au{Коновалов М.\,Г.} Методы адаптивной обработки информации и~их приложения.~--- 
М.: ИПИ РАН, 2007. 212~с. 

\end{thebibliography}

 }
 }

\end{multicols}

\vspace*{-8pt}

\hfill{\small\textit{Поступила в~редакцию 17.04.22}}

\vspace*{8pt}

%\pagebreak

%\newpage

%\vspace*{-28pt}

\hrule

\vspace*{2pt}

\hrule

%\vspace*{-2pt}

\def\tit{CONTROLLING A~BOUNDED TWO-DIMENSIONAL MARKOV CHAIN 
WITH~A~GIVEN INVARIANT MEASURE}


\def\titkol{Controlling a~bounded two-dimensional Markov chain 
with~a~given invariant measure}


\def\aut{M.\,G.~Konovalov and R.\,V.~Razumchik}

\def\autkol{M.\,G.~Konovalov and R.\,V.~Razumchik}

\titel{\tit}{\aut}{\autkol}{\titkol}

\vspace*{-8pt}


\noindent
Federal Research Center ``Computer Science and Control'' of the Russian Academy 
of Sciences, 44-2~Vavilov Str., Moscow 119333, Russian Federation

\def\leftfootline{\small{\textbf{\thepage}
\hfill INFORMATIKA I EE PRIMENENIYA~--- INFORMATICS AND
APPLICATIONS\ \ \ 2022\ \ \ volume~16\ \ \ issue\ 2}
}%
 \def\rightfootline{\small{INFORMATIKA I EE PRIMENENIYA~---
INFORMATICS AND APPLICATIONS\ \ \ 2022\ \ \ volume~16\ \ \ issue\ 2
\hfill \textbf{\thepage}}}

\vspace*{3pt} 


\Abste{Consideration is given to the two-dimensional discrete-time Markov chain (random 
walk) with the bounded continuous state space (rectangle). Upon each transition, depending on 
its current position and if not on the boundary, the chain moves in one of four possible 
directions (north, south, east, or west). Having selected a~direction, the length of the jump within 
the admissible interval is determined by the random variable. Assuming that some (reference) 
distribution on the state space is given, one seeks to solve the inverse control problem, i.\,e., to 
find such a~control strategy (probabilities of choosing either direction) which brings the 
stationary distribution of the chain close (in a~certain sense) to the reference distribution. The 
solution based on the policy gradient method is proposed. Illustrative examples are provided.} 

\KWE{Markov chain control; continuous state space; policy gradient; unmanned air vehicles}




\DOI{10.14357/19922264220214}

\vspace*{-14pt}

\Ack

\vspace*{-2pt}


\noindent
The research was carried out using the infrastructure of the shared research facilities CKP 
``Informatics'' of FRC CSC RAS. The research was partially supported by the Russian Foundation for Basic Research (project No.\,20-07-00804).


%\vspace*{4pt}

  \begin{multicols}{2}

\renewcommand{\bibname}{\protect\rmfamily References}
%\renewcommand{\bibname}{\large\protect\rm References}

{\small\frenchspacing
 {%\baselineskip=10.8pt
 \addcontentsline{toc}{section}{References}
 \begin{thebibliography}{99}
\bibitem{1-kraz-1}
\Aue{Konovalov, M., and R.~Razumchik.} 2018. Upravlenie sluchaynym bluzhdaniem 
s~etalonnym statsionarnym raspredeleniem [Finding control policy for one discrete-time 
Markov chain on $[0,1]$ with a given invariant measure]. \textit{Informatika i~ee 
Primeneniya~--- Inform. \mbox{Appl.}} 12(3):2--13. 
\bibitem{2-kraz-1}
\Aue{Konovalov, M.\,G., I.\,N.~Konovalova, and R.\,V.~Razumchik.} 2020. Upravlenie 
dvumernoy markovskoy tsep'yu s~ne\-pre\-ryv\-nym ogranichennym mnozhestvom so\-sto\-yaniy, 
pri\-vo\-dya\-shchee k~za\-dan\-no\-mu sta\-tsi\-o\-nar\-no\-mu ras\-pre\-de\-le\-niyu [Control of a two-dimensional 
Markov chain with a~continuous bounded set of states leading to a~given stationary distribution]. 
\textit{Informatsionno-telekommunikatsionnye tekhnologii i~matematicheskoe modelirovanie 
vysokotekhnologichnykh sistem} [Information and telecommunication technologies and 
mathematical modeling of high-tech systems]. Moscow: RUDN University Publs. 276--279.
\bibitem{3-kraz-1}
\Aue{Elmaliach, Y., A.~Noa, and G.~Kaminka.} 2009. Multi-robot area patrol under frequency 
constraints. \textit{Ann. Math. Artif. Intel.} 57(3-4):293--320. 
\bibitem{4-kraz-1}
\Aue{Elor, Y., and A.~Bruckstein.} 2010. Autonomous multi-agent cycle based patrolling. 
\textit{Swarm intelligence}. Eds. M.~Do\-ri\-go, M.~Birattari, G.\,A.~Di Caro, \textit{et al}.
 Lecture notes in computer science ser. Springer.  6234:119--130. 
\bibitem{5-kraz-1}
\Aue{LaValle, S., and J.~Hinrichsen.} 2001. Visibility-based pursuit-evasion: The case of 
curved environments. \textit{IEEE T. Robotic. Autom.} 17(2):196--202. 
\bibitem{6-kraz-1}
\Aue{Gerkey, B.\,P., S.~Thrun, and G.~Gordon.} 2006. Visibility-based pursuit-evasion with 
limited field of view. \textit{Int. J.~Robot. Res.} 25(4):299--315. 
\bibitem{7-kraz-1}
\Aue{Fazli, P., A.~Davoodi, P.~Pasquier, and A.\,K.~Mackworth.} 2010. Complete and robust 
cooperative robot area coverage with limited range. \textit{Conference (International) on 
Intelligent Robots and Systems Proceedings}. Piscataway, NJ: IEEE. 5577--5582. 
\bibitem{8-kraz-1}
\Aue{Jennings, J., G.~Whelan, and W.~Evans.} 1997. Cooperative search and rescue with 
a~team of mobile robots. \textit{8th Conference (International) on Advanced Robotics 
Proceedings}. Piscataway, NJ: IEEE. 193--200.
\bibitem{9-kraz-1}
\Aue{Fazli, P., A.~Davoodi, and A.\,K.~Mackworth.} 2012. Multi-robot repeated area 
coverage: Performance optimization under various visual ranges. \textit{9th Conference on 
Computer and Robot Vision Proceedings}. Piscataway, NJ: IEEE. 298--305. 
\bibitem{10-kraz-1}
\Aue{Carlsson, S., B.\,J.~Nilsson, and S.\,C.~Ntafos.} 1993 Optimum guard covers and  
m-watchmen routes for restricted polygons. \textit{Int. J.~Comput. Geom.  
Ap.} 3(1):85--105. 

\bibitem{13-kraz-1} %11
\Aue{Toth, P., and D.~Vigo.} 2002. \textit{The vehicle routing problem}. Philadelphia, PA: 
Society for Industrial Mathematics. 358~p. 

\bibitem{11-kraz-1} %12
\Aue{Machado, A., G.~Ramalho, J.\,D.~Zucker, and A.~Drogoul.} 2003. Multiagent patrolling: 
An empirical analysis of alternative architectures. \textit{Multi-agent-based simulation~II}. Eds. 
J.~Sim$\tilde{\mbox{a}}$o Sichman, F.~Bousquet, and P.~Davidsson. Lecture notes in 
computer science ser. Springer. 2158:155--170. 
\bibitem{12-kraz-1} %13
\Aue{Gasparri, A., B.~Krishnamachari, and G.~Sukhatme.} 2008. A~framework for multi-robot 
node coverage in sensor networks. \textit{Ann. Math. Artif. Intel.} 
52(2):281--305. 

\bibitem{14-kraz-1} %14
\Aue{Kalyaev, I.\,A., A.\,R.~Gayduk, and S.\,G.~Kapustyan.} 2009. \textit{Modeli i~algoritmy 
kollektivnogo upravleniya v gruppakh robotov} [Models and algorithms of collective control in 
groups of robots]. Moscow: Fizmatlit. 280~p. 





\bibitem{17-kraz-1} %15
\Aue{Birk, A., B.~Wiggerich, H.~Bulow, M.~Pfingsthorn, and S.~Schwertfeger}. 2011. Safety, 
security, and rescue missions with an unmanned aerial vehicle (UAV). \textit{J.~Intell.  
Robot. Syst.} 64(1):57--76. 
\bibitem{19-kraz-1} %16
\Aue{Diveev, A.\,I., E.\,Yu.~Shmalko, and D.\,A.~Ryndin.} 2017. Reshenie zadachi 
optimal'nogo upravleniya gruppoy robotov evolyutsionnymi algoritmami [Solution of optimal control problem for group of robots by
evolutionary algorithms]. \textit{Informatsionnye 
i~matematicheskie tekhnologii v~nauke i~upravlenii} [Information and Mathematical 
Technologies in Science and Management] 3(7):109--126. 

\bibitem{16-kraz-1}  %17
\Aue{Pshikhopov, V.\,Kh., and M.\,Yu.~Medvedev.} 2018. Gruppovoe upravlenie 
dvizheniem mobil'nykh robotov v~ne\-opre\-de\-len\-noy srede s~ispol'zovaniem neustoychivykh 
rezhimov [Group control of autonomous robots motion in uncertain environment via unstable modes]. 
\textit{Trudy \mbox{SPIIRAN}} [SPIIRAS Proceedings] 60:39--63. 

\bibitem{18-kraz-1} %18
\Aue{Arbanas, B., A.~Ivanovic, M.~Car, M.~Orsag, T.~Petrovic, and S.~Bogdan.} 2018. 
Decentralized planning and control for UAV--UGV cooperative teams. \textit{Auton. 
Robot.} 42(8):1601--1618. 
\bibitem{15-kraz-1} %19
\Aue{Senotov, V.\,D., A.\,P.~Aliseychik, E.\,V.~Pavlovsky, A.\,V.~Podoprosvetov, and 
I.\,A.~Orlov.} 2020. Algoritmy staynogo detsentralizovannogo upravleniya dvizheniem gruppy 
robotov s~differentsial'nym privodom [Algorithms for swarm decentralized motion control of 
group of robots with a differential drive]. \textit{Keldysh Institute Preprints} %[KIAM Preprint] 
123. 39~p. 

\bibitem{26-kraz-1} %20
\Aue{Kibzun, A.\,I., and I.\,N.~Sinitsyn.} 2020. Sovremennye problemy teorii optimizatsii 
stokhasticheskikh sistem [Modern problems of optimization theory stochastic systems]. 
\textit{Automat. Rem. Contr.} 11:3--10.
\bibitem{20-kraz-1} %21
\Aue{Portugal, D., and R.~Rocha.} 2010. MSP algorithm: Multi-robot patrolling based on 
territory allocation using balanced graph partitioning. \textit{Symposium on Applied Computing 
Proceedings}. New York, NY: ACM. 1271--1276.

\bibitem{22-kraz-1}
\Aue{Reif, J., and H.~Wang.} 1999. Social potential fields: A~distributed behavioral control for 
autonomous robots. \textit{Robot. Auton. Syst.} 27(3):171--194.
\bibitem{23-kraz-1}
\Aue{Chu, H.\,N., A.~Glad, O.~Simonin, F.~Sempe, A.~Drogoul, and F.~Charpillet.} 2007. 
Swarm approaches for the patrolling problem, information propagation vs.\ pheromone 
evaporation.  \textit{19th Conference (International) on Tools with Artificial Intelligence 
Proceedings}. Alamitos, CA: IEEE. 1:442--449. 
\bibitem{21-kraz-1} %24
\Aue{Shvets, E.\,A., and D.\,P.~Nikolaev.} 2015. Complex approach to long-term multi-agent 
mapping in low dynamic environments. \textit{Proc. SPIE} 9875:98752A. 10~p. doi: 10.1117/12.2228708.
\bibitem{27-kraz-1} %25
\Aue{Miller, B.\,M., K.\,V.~Stepanyan, A.\,K.~Popov, and A.\,B.~Miller.} 2017. UAV 
navigation based on videosequences captured by the onboard video camera. \textit{Automat. 
Rem. Contr.} 17:2211--2221. 
\bibitem{24-kraz-1} %26
\Aue{Ramli, M.\,A., and G.~Leng.} 2010. The stationary probability density of a class of 
bounded Markov processes. \textit{Adv. Appl. Probab.} 42:986--993. 
\bibitem{25-kraz-1} %27
\Aue{Dotsenko, A., A.~Diveev, and J.\,P.\,C.~Cevallos.} 2019. Collision avoidance at swarm 
regrouping using modified network operator method with various number of arguments. 
\textit{14th Conference on Industrial Electronics and Applications Proceedings}.  Piscataway, NJ: IEEE.  
768--773. 


\bibitem{28-kraz-1}
\Aue{Konovalov, M., and R.~Razumchik.} 2021. Dis\-pet\-che\-ri\-za\-tsiya v~sisteme s~parallel'nym 
obsluzhivaniem s~po\-moshch'yu ras\-pre\-de\-len\-no\-go gra\-di\-ent\-no\-go up\-rav\-le\-niya mar\-kov\-skoy 
tsep'yu [Routing jobs to heterogeneous parallel queues using distributed policy gradient 
algorithm]. \textit{Informatika i~ee Primeneniya~--- Inform. Appl.} 15(3):41--50. 
\bibitem{29-kraz-1}
\Aue{Konovalov, M.\,G.} 2007. \textit{Metody adaptivnoy obrabotki informatsii i~ikh 
prilozheniya} [Methods of adaptive information processing and their applications]. Moscow: IPI 
RAN. 212~p. 

\end{thebibliography}

 }
 }

\end{multicols}

\vspace*{-6pt}

\hfill{\small\textit{Received April 17, 2022}}
  

\Contr

\noindent
\textbf{Konovalov Mikhail G.} (b.\ 1950)~--- Doctor of Science in technology, principal 
scientist, Institute of Informatics Problems, Federal Research Center ``Computer Science and 
Control'' of the Russian Academy of Sciences, 44-2~Vavilov Str., Moscow 119333, Russian 
Federation; \mbox{mkonovalov@ipiran.ru }

\vspace*{3pt}

\noindent
\textbf{Razumchik Rostislav V.} (b. 1984)~--- Candidate of Science (PhD) in physics and 
mathematics, leading scientist, Institute of Informatics Problems, Federal Research Center 
``Computer Science and Control'' of the Russian Academy of Sciences, 44-2~Vavilov Str., 
Moscow 119333, Russian Federation; \mbox{rrazumchik@ipiran.ru}



\label{end\stat}

\renewcommand{\bibname}{\protect\rm Литература}    