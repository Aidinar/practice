\def\stat{grusho}

\def\tit{АНАЛИЗ ЦЕПОЧЕК ПРИЧИННО-СЛЕДСТВЕННЫХ СВЯЗЕЙ}

\def\titkol{Анализ цепочек причинно-следственных связей}

\def\aut{А.\,А.~Грушо$^1$, Н.\,А.~Грушо$^2$, М.\,И.~Забежайло$^3$, 
А.\,А.~Зацаринный$^4$, Е.\,Е.~Тимонина$^5$, С.\,Я.~Шоргин$^6$}

\def\autkol{А.\,А.~Грушо, Н.\,А.~Грушо, М.\,И.~Забежайло и~др.} 
%А.\,А.~Зацаринный$^4$, Е.\,Е.~Тимонина$^5$, С.\,Я.~Шоргин$^6$}

\titel{\tit}{\aut}{\autkol}{\titkol}

\index{Грушо А.\,А.}
\index{Грушо Н.\,А.}
\index{Забежайло М.\,И.}
\index{Зацаринный А.\,А.}
\index{Тимонина Е.\,Е.}
\index{Шоргин С.\,Я.}
\index{Grusho A.\,A.}
\index{Grusho N.\,A.}
\index{Zabezhailo M.\,I.}
\index{Zatsarinny A.\,A.}
\index{Timonina E.\,E.}
\index{Shorgin S.\,Ya.}


%{\renewcommand{\thefootnote}{\fnsymbol{footnote}} \footnotetext[1]
%{Работа выполнена при поддержке Министерства науки и~высшего образования Российской Федерации (проект 
%075-15-2020-799).}}


\renewcommand{\thefootnote}{\arabic{footnote}}
\footnotetext[1]{Федеральный исследовательский центр <<Информатика и~управление>> Российской академии наук, 
\mbox{grusho@yandex.ru}}
\footnotetext[2]{Федеральный исследовательский центр <<Информатика и~управление>> Российской академии наук, 
\mbox{info@itake.ru}}
\footnotetext[3]{Федеральный исследовательский центр <<Информатика и~управление>> Российской академии наук, 
\mbox{m.zabezhailo@yandex.ru}}
\footnotetext[4]{Федеральный исследовательский центр <<Информатика и~управление>> Российской академии наук, 
\mbox{AZatsarinny@ipiran.ru}}
\footnotetext[5]{Федеральный исследовательский центр <<Информатика и~управление>> Российской академии наук, 
\mbox{eltimon@yandex.ru}}
\footnotetext[6]{Федеральный исследовательский центр <<Информатика и~управление>> Российской академии наук, 
\mbox{sshorgin@ipiran.ru}}

%\vspace*{10pt}


  
  
  \Abst{Работа посвящена анализу возможностей использования причинно-следственных 
связей для контроля выполнения информационных технологий  (ИТ) в~распределенных 
информационных системах (РИС). Для проведения этого анализа построен простой пример фрагмента 
некоторой фиксированной ИТ, в~котором разобраны ситуации, когда 
контроль может опираться на при\-чин\-но-след\-ст\-вен\-ные связи между действиями 
в~ИТ и~когда предполагаемые при\-чин\-но-след\-ст\-вен\-ные связи просто 
не существуют.
  Показана ограниченность возможностей использования при\-чин\-но-след\-ст\-вен\-ных 
связей для повышения доверия к~результатам сложных компьютерных вычислений. Эта 
ограниченность основана на том, что для получения причинно-следственных связей в~ожидании 
определенного следствия часто невозможно контролировать связи характеристик, обязательных 
для того, чтобы искомое следствие вытекало из действий, которые предполагаются его 
причиной.
  Для использования при\-чин\-но-след\-ст\-вен\-ных связей в~ИТ
необходимо дополнять действия ИТ действиями по контролю связей 
между характеристиками. Без этого условия правильно построенная последовательность 
действий ИТ становится необходимым, но не достаточным условием 
ожидаемого следствия.}
  
  \KW{информационная безопасность; причинно-следственные связи; мониторинг 
информационных технологий}

\DOI{10.14357/19922264220209}
  
\vspace*{12pt}


\vskip 10pt plus 9pt minus 6pt

\thispagestyle{headings}

\begin{multicols}{2}

\label{st\stat}
   
  \section{Введение }
  
  В связи с~распространением использования искусственного интеллекта 
в~процессе принятия решений~[1, 2] становится актуальной проблема доверия 
к~результатам компьютерных вычислений.\linebreak Исследованиям при\-чин\-но-след\-ст\-вен\-ных 
связей для получения и~обоснования выводов компьютерных 
вычислений посвящено много работ.
{\looseness=1

}

 Следует отметить работу~[3], в~которой 
предложен способ объяснимости на основе агентов, задающих простые вопросы 
и~способных анализировать ответы на эти вопросы. Работа выполнена на основе 
каузальной модели Halpern--Pearl~[4--7]. Большинство результатов в~модели, 
разработанной в~работе~\cite{7-gr}, направлено на требующие объяснений 
медицинские приложения, в~которых использованы вероятностные инструменты. 
Вместе с~тем в~обзоре~[8] указаны приложения, связанные с~улучшением методов 
машинного обучения, образованием, об\-ластью принятия решений в~политике.
{\looseness=1

}
  
  Данная работа посвящена анализу возможностей использования  
при\-чин\-но-след\-ствен\-ных связей в~контроле выполнения ИТ в~РИС. Для 
проведения этого анализа построен простой пример фрагмента некоторой 
фиксированной ИТ, в~котором разобраны ситуации, когда контроль может 
опираться на при\-чин\-но-след\-ст\-вен\-ные связи между действиями в~ИТ 
и~когда предполагаемые при\-чин\-но-след\-ст\-вен\-ные связи просто не 
существуют. 


Показано, что анализ при\-чин\-но-след\-ст\-вен\-ных связей служит 
тонким инструментом, использование которого для подтверждения доверия 
и~объяснимости результатов компьютерных вычислений является ограниченным. 
Это утверждение основано на том, что для наличия  
при\-чин\-но-след\-ст\-вен\-ных связей необходимо выполнение ограничений на 
связи между характеристиками различных действий, образующих 
предполагаемые причины, и~предполагаемых представляющих интерес следствий. 
Такие связи обязательно должны выполняться между элементами различных 
множеств, возможно очень больших. Отсюда возникают алгоритмические 
и~вычислительные проб\-лемы.

  
  \section{Модель причинно-следственных связей в~задачах 
мониторинга рабочих процессов в~распределенных информационных системах}
  
  Мониторинг функционирования РИС и,~в~частности, безопасности РИС~--- 
один из главных компонентов обеспечения информационной безопасности. 
Данные мониторинга представляют собой последовательности объектов 
определенного вида. Вместе с~тем работа РИС описывается последовательностью 
действий с~данными~[9]. Действие состоит из 3~составляющих:
  \begin{enumerate}[(1)]
\item инициация действия;
\item реализация действия; 
\item инициация других действий после выполнения действия. 
\end{enumerate}
  
  Определим понятие параметра в~системе. 
  
  Пусть $A$~--- переменная величина 
с~об\-ластью допустимых значений $D(A)$. Элементы множества $D(A)$ будем 
называть характеристиками, или значениями, параметра~$A$. Тогда 
действие~$X$ можно описать как минимум тремя параметрами $A_1(X)$, $A_2(X)$ 
и~$A_3(X)$ с~соответствующими множествами значений этих параметров 
(характеристиками) $D_1(X)$, $D_2(X)$ и~$D_3(X)$. В~множество $D_1(X)$ входят 
идентификаторы действия~$X$ и~действий, которые воздействуют 
и~инициируют~$X$. В~множество $D_2(X)$ входят правила действия~$X$, 
описания формы действия~$X$ и~ограничения, применимые к~данным, 
вызывающим действие~$X$. В~множество $D_3(X)$ входят пути и~средства 
воздействия на другие действия.
  
  Объектом $O(X)$ назовем множество выбранных в~соответствующих областях 
переменных характеристик действия~$X$. Будем считать, что по любой 
характеристике~$x$ в~объекте $O(X)$ можно восстановить имя параметра из 
множеств $A_1(X)$ или $A_3(X)$, который содержит эту характеристику. 
Данные представляют собой последовательность объектов
  \begin{equation}
  O_1, O_2, \ldots , O_N\,.
  \label{e1-gr}
  \end{equation}
  
  Повторяющиеся конфигурации действий (схемы действий), как правило, 
связаны с~при\-чин\-но-след\-ст\-вен\-ны\-ми связями, которые существуют в~ИТ, 
исполняемых в~РИС. Поэтому методы анализа при\-чин\-но-след\-ст\-вен\-ных 
связей должны использоваться в~анализе мониторинга РИС, особенно когда такой 
анализ надо проводить быстро и~даже приближенно.
  
  Общее определение цепочки при\-чин\-но-след\-ст\-вен\-ных связей вытекает из 
работы~\cite{6-gr}.
  
  Действия $X_1, X_2, \ldots , X_k$ являются непосредственной причиной 
следствия, т.\,е.\ набора действий $Y_1, Y_2, \ldots , Y_s$, если существует хотя 
бы один набор $x_1, x_2, \ldots , x_k$ значений выходных характеристик действий 
такой, при котором однозначно определены выходные значения $y_1, y_2, \ldots , 
y_s$ действий $Y_1, Y_2, \ldots , Y_s$, т.\,е.\ всегда, когда в~результатах 
действий $X_1, X_2, \ldots , X_k$ появляется набор значений $x_1, x_2, \ldots , 
x_k$  в~непосредственных результатах этого набора действий, появляется набор 
значений $y_1, y_2, \ldots , y_s$ действий $Y_1, Y_2, \ldots , Y_s$.
  
  Фрагментом причинно-след\-ст\-вен\-ных связей длины~2 будем называть 
наборы действий $X_1, X_2, \ldots , X_k$, $Y_1, Y_2, \ldots , Y_s$ и~$B$, для 
которых действия $X_1, X_2, \ldots , X_k$ являются непосредственной причиной 
следствия~--- набора действий $Y_1, Y_2, \ldots , Y_s$, а~действия $Y_1, Y_2, 
\ldots , Y_s$ являются непосредственной причиной действия~$B$. Пусть об\-ласть 
значений $D(B)$ действия~$B$ принимает два значения: 1 и~0.
  
  Более естественно было бы определять причину, если вместо квантора 
существования использовать квантор всеобщности, т.\,е.\ для всех наборов $x_1, 
x_2, \ldots , x_k$ всегда порождаются наборы $y_1, y_2, \ldots , y_s$. Тогда 
цепочки при\-чин\-но-след\-ст\-вен\-ных связей определялись бы суперпозицией 
отоб\-ра\-жений. 
  
  Далее рассмотрим обоснование того, что проб\-ле\-ма цепочек при\-чин\-но-след\-ст\-вен\-ных 
  связей не может рассматриваться так упрощенно.
  
  \vspace*{-4pt}
  
\section{Проблемы описания причинно-следственных связей}

   Рассмотрим следующий пример. Пусть $U_1$, $U_2$, $U_3$ и~$U_4$~--- 
пользователи, участвующие в~некоторой фиксированной ИТ; ФХ~--- файловое 
хранилище. Содержание фрагмента ИТ состоит в~следующих действиях: 
  \begin{align*}
  X_1 & = \{\mbox{пользователь}\ \ldots\  \mbox{готовит\ файл}\ \ldots\\
  &\hspace*{29.5mm}\ldots\ \mbox{для\ пользователя}\ \ldots \};\\ 
  X_2 & = \{\mbox{пользователь}\ \ldots\ \mbox{кладет\ файл}\ \ldots\ 
\mbox{в~ФХ}\};\\ 
  X_3& = \{\mbox{пользователь}\ \ldots\ \mbox{берет\ файл}\ \ldots\  
\mbox{из~ФХ}\}.
  \end{align*} 
  
  \vspace*{-2pt}
  
  Каждое действие может быть выполнено правильно и~неправильно. Правила 
определения правильности действий заложены в~соотношениях входных 
и~выходных значений параметров. Правильная последовательность действий 
в~рассматриваемом фрагменте ИТ сле\-ду\-ющая: 

\noindent
  \begin{align*}
  &\{U_1\ \mbox{готовит\ файл}\  Z_1\ \mbox{для\ пользователя}\ U_3\};\\
    &\{U_2\ \mbox{готовит\ файл}\ Z_2\ \mbox{для\ пользователя}\ U_4\};
    \end{align*}
    
    \noindent
      \begin{align*}
  &\{U_1\ \mbox{кладет\ файл}\ Z_1\  \mbox{в~ФХ}\};\\
  &\{U_2\ \mbox{кладет\ файл}\ Z_2\ \mbox{в~ФХ}\};\\
  &\{U_3\ \mbox{берет\ файл}\ Z_1\ \mbox{из~ФХ}\};\\
  &\{U_4\ \mbox{берет\ файл}\ Z_2\ \mbox{из~ФХ}\}.
  \end{align*}
  
   Безошибочными считаются действия: $X_2$~--- положить файл в~ФХ;  
$X_3$~--- взять файл из ФХ; $X_4$~--- подготовить сообщение~<<1>> 
или~<<0>> от пользователя, взявшего файл; $X_5$~--- использовать 
\mbox{сообщения}~<<1>> или~<<0>> для принятия решения. $U_1$, $U_2$, $U_3$ 
и~$U_4$~--- неизменяемые характеристики действий; $Z_1$, $Z_2$ и~$Z$~--- 
изменяемые характеристики ИТ. Объекты, наблюдаемые при выполнении ИТ:
  \begin{align*}
  O_1 &= \{U_1\    \mbox{получил\ доступ\ к~ФХ},\\
  &\hspace*{37mm}\mbox{положил\ файл}\ Z_1\  
\mbox{в~ФХ}\};\\
  O_2& = \{U_2\   \mbox{получил\ доступ\ к\ ФХ,}\\
  &\hspace*{37mm}\mbox{положил\ файл}\ Z_2\ 
\mbox{в~ФХ}\};\\
  O_3& = \{U_3\ \mbox{получил\ доступ\ к~ФХ},\\
  &\hspace*{37mm}\mbox{взял\ файл}\ Z_1\  \mbox{из~ФХ}\};\\
  O_4& = \{U_4\ \mbox{получил\ доступ\ к~ФХ},\\
  &\hspace*{37mm}\mbox{взял\ файл}\ Z_2\ 
\mbox{в~ФХ}\};\\
  O_5& = \{U_1\ \mbox{получил\ доступ\ к~ФХ},\\
  &\hspace*{37mm}\mbox{положил\ файл}\ Z\ 
\mbox{в~ФХ}\};\\
  O_6& = \{U_3\ \mbox{получил\ доступ\ к~ФХ},\\
  &\hspace*{37mm}\mbox{взял\ файл}\ Z_2\ \mbox{из~ФХ}\};\\
  O_7& = \{U_4\ \mbox{получил\ доступ\ к~ФХ},\\
  &\hspace*{37mm}\mbox{взял\ файл}\ Z_1\ 
\mbox{из~ФХ}\}.
  \end{align*}
  
  Результаты действий $X_4$ и~$X_5$ будут рассмотрены позже. Рассмотрим  
при\-чин\-но-след\-ст\-вен\-ные связи на первых шагах. Действия на первом 
шаге~--- подготовка документов $Z_1$, $Z_2$ и~$Z$. Правила определения 
правильности действия~$X_1$ состоят в~привязке имени пользователя к~названию 
файла. Действия~$X_2$ не имеют правил контроля правильности, так как 
пользователь кладет в~ФХ тот файл, который был привязан к~его имени на первом 
шаге. Действие~$X_3$ имеет правила контроля, которые состоят в~том, что 
пользователь берет из ФХ файл, который имеет нужное название. Действия 
пользователей отражаются в~событиях. Пользователи могут совершать ошибки. 
Например, пользователь~$U_1$ неправильно привязал подготовленный файл 
и~в~ФХ положил файл с~именем~$Z$. Тогда пользователь~$U_3$ не находит 
файл с~именем~$Z_1$. По истечении контрольного времени~$U_3$ с~помощью 
действия~$X_4$ формирует сообщение~<<0>> и~посылает его действию~$X_5$. 
При этом пользователь~$U_2$ все сделал правильно и~пользователь~$U_4$, 
получив файл~$Z_2$, сформировал сообщение~<<1>> и~отослал его 
действию~$X_5$. Получив сообщения~<<0>> и~<<1>>, действие~$X_5$ 
формирует сигнал тревоги. Правильное выполнение ИТ соответствует 
сообщениям~<<1>> и~<<1>>. Если пользователи~$U_1$ и~$U_2$ перепутали 
имена файлов, то формально на действие~$X_5$ поступают сигналы~<<1>> 
и~<<1>>, но в~реализации технологии произошла ошибка. Поэтому правила 
контроля при выполнении действия~$X_1$ имеют большое значение. 
  
  Теперь проведем анализ построенного примера с~позиций при\-чин\-но-след\-ст\-вен\-ных связей. 
  Рас\-смот\-рен\-ные цепочки действий соответствуют при\-чин\-но-след\-ст\-вен\-ным 
связям только при правильных связях между характеристиками 
проводимых действий. При нарушении связей проделанные действия не 
становятся причиной следствия (выполнения ИТ), причем эти связи должны 
выполняться по всей цепочке действий вплоть до проявления следствия 
(выполнение ИТ). Любая частичная цепочка не позволяет сделать вывод 
о~существовании причинно-следственных связей. Отметим, что разрывы 
в~последовательности действий (взаимодействие через ФХ) сохраняют 
требование, что вся цепочка действий должна удовлетворять связям между 
характеристиками. Требования существования правил контроля правильности 
соотношений между характеристиками делают трудоемким использование  
при\-чин\-но-след\-ст\-вен\-ных связей для анализа и~прогнозирования  
вы\-пол\-ни\-мости~ИТ.
  
  Пример также показывает, что набор характеристик, которые могут 
удовлетворять связям, определяющим причину для искомого следствия, за счет 
действий пользователей, участвующих в~ИТ, могут потерять свойство 
становиться причиной этого свойства.
  
  Можно рассматривать сложные комплексные действия вместо рассмотренных 
  в~примере. Задача построения связей характеристик и~правил контроля выполнения 
этих связей в~таком случае становится сложной. Но можно использовать 
вероятностный подход, основанный на оценках совместных распределений 
характеристик пред\-шест\-ву\-ющих и~по\-сле\-ду\-ющих действий. В~применении 
к~построенному примеру вероятность связи~$U_1$ и~$Z_1$, а~также связи 
между~$Z_1$ и~$U_3$ больше, чем вероятности других конфигураций. 
Статистические оценки этих вероятностей на правильных прецедентах~[10] 
позволят получить оценку вероятности следствия, т.\,е.\ прогнозировать 
вероятность правильного выполнения ИТ~[11]. Эта оценка служит оценкой 
доверия к~результатам выполнения ИТ.
  
  Построенные фрагменты при\-чин\-но-след\-ст\-вен\-ных связей могут иметь 
самостоятельную ценность для контроля правильности выполнения особо важных 
фрагментов ИТ при допустимой сложности выявления связей между 
характеристиками действий, при которых существуют  
при\-чин\-но-след\-ст\-вен\-ные связи. Как было показано в~примере, по 
результатам выполнения фрагмента ИТ ставится сенсор, который позволяет 
с~помощью полученных сообщений~<<1>> и~<<0>> построить оценку 
правильности выполнения этого фрагмента. Здесь речь идет именно об оценке, 
так как в~примере было показано, что сенсор подтверждает правильность 
выполнения ИТ, но на самом деле возможна ошибка, поэтому при оценке~<<1>> 
и~<<1>> необходимо, двигаясь в~направлении от следствия к~причине, проверять 
выполнение всех правил и~соотношений между характеристиками. Этот алгоритм 
усложняет, но в~то же время улучшает оценку контроля. 
  
  Заметим, что все~<<1>> могут ложно возникнуть только при перестановках 
данных, управ\-ля\-ющих действиями, поэтому доверие к~фрагменту ИТ можно 
повышать, контролируя только отдельные цепочки обработки данных. Ошибка 
в~отдельной цепочке при всех~<<1>> суммарного контроля означает 
неправильное направление потока данных при перемешивании цепочек.
  
  Построенные методы использования при\-чин\-но-след\-ст\-вен\-ных связей 
дают необходимые, но не достаточные условия контроля правильности 
вычислений с~их по\-мощью. При этом даже в~рас\-смот\-рен\-ном выше простейшем 
примере показано, что следствие может участвовать в~анализе причинных связей 
между характеристиками. Это, в~свою очередь, затрудняет использование  
при\-чин\-но-след\-ст\-вен\-ных связей для прогнозов следствий.
  
  
  \section{Контроль информационных технологий по~последовательностям сопровождающих 
действия объектов }
  
  Напомним, что данные представляют собой последовательность объектов~(1). 
При этом наборы па\-ра\-мет\-ров, определяющих характеристики объектов, могут 
быть разными у разных объектов.
  
  Множество характеристик любого наблюдаемого объекта~$O(X)$ делится на 
две непересекающиеся части~$B$ и~$C$. Множество~$B$ объединяет 
характеристики, которые не зависят от характеристик других объектов. Эти 
характеристики служат своего рода идентификаторами объекта. Множество 
характеристик~$C$ зависит от предыдущих объектов в~том смыс\-ле, что  
ка\-кие-то характеристики предыдущих объектов определяют значения 
характеристик из~$C$ в~$O(X)$ или участвуют в~инициации последующих 
действий. Это значит, что множество характеристик, которые определяют 
значения характеристик из~$C$, становятся причиной появления множества~$C$~[12--14] и~для них выполнены ограничительные связи между 
характеристиками. Таким образом, можно считать, что некоторые объекты 
образуют при\-чин\-но-след\-ст\-вен\-ные связи друг с~другом. По этим связям 
и~известным описаниям ИТ восстанавливаются фрагменты ИТ, определенные 
выше. В~данном случае при\-чин\-но-след\-ст\-вен\-ные связи существуют 
и~работают, так как после проведенных действий не последовало сигнала тревоги 
или ИТ не остановлена при выполнении ка\-ко\-го-то действия, поэтому при 
контроле исполнения ИТ не нужно детально проверять выполнение связей 
характеристик проведенных действий. Тогда анализ объектов~(1) позволяет не 
только частично контролировать исполнение ИТ, но даже восстанавливать 
последовательность действий при выполнении неизвестной ИТ или 
возникновении сомнений в~отсутствии вредоносного кода. Идея контроля ИТ по 
последовательности~(1) с~по\-мощью при\-чин\-но-след\-ст\-вен\-ных связей 
рассматривалась ранее. В~\mbox{статье}~\cite{15-gr} исследована проблема контроля 
безопас\-ности ИТ на основе последовательности данных 
компьютерного аудита при известной схеме действий в~ИТ.
   
  Подобные задачи возникают также в~поиске первопричин неявных сбоев 
и~аномалий в~компьютерных сис\-те\-мах~[15, 16]. Приближенные методы  
при\-чин\-но-след\-ст\-вен\-но\-го анализа (агрегация действий) позволяют решать 
задачи поиска первопричины сбоя, или аномалии, и~выявлять такие аномалии 
с~точностью до блоков допустимо малых размеров. Такой подход позволяет 
быстро восстанавливать рабочие процессы за счет замены блока в~сложной 
технической системе без детального анализа причины сбоя.

\vspace*{-4pt}
  
  \section{Заключение }
  
  В работе показана ограниченность возможностей использования при\-чин\-но-след\-ст\-вен\-ных 
  связей для повышения доверия к~результатам сложных 
компьютерных вычислений. Эта ограниченность основана на том, что для 
получения при\-чин\-но-след\-ст\-вен\-ных связей в~ожидании определенного 
следствия часто невозможно контролировать связи характеристик, обязательных 
для того, чтобы искомое следствие получалось из действий, которые 
предполагаются причиной данного следствия, т.\,е.\ совокупность действий, 
которые должны породить искомое следствие, должны быть дополнены 
действиями контроля связей всех характеристик. Без этого соответствие  
<<при\-чи\-на--след\-ст\-вие>> мо-\linebreak\vspace*{-12pt}

\pagebreak


\noindent
жет в~некоторых случаях существовать, 
а~в~некоторых отсутствовать, причем правильно построенная априори 
последовательность характеристик может перестать становиться причиной 
ожидаемого свойства за счет действий пользователей, не на\-ру\-ша\-ющих со\-став 
характеристик.
  
  Если цепочка данных показывает, что искомое следствие получено, то 
исследование последовательности характеристик действий может позволить 
установить связи между \mbox{характеристиками}, чтобы существовали  
при\-чин\-но-след\-ст\-вен\-ные связи. Для этого можно использовать машинное 
обуче\-ние.
  
  Таким образом, для использования при\-чин\-но-след\-ст\-вен\-ных связей в~ИТ 
необходимо дополнять действия ИТ действиями по контролю связей между 
характеристиками. Без этого условия правильно построенная последовательность 
действий ИТ становится необходимым, но не достаточным условием ожидаемого 
следствия. 
  
{\small\frenchspacing
 {%\baselineskip=10.8pt
 %\addcontentsline{toc}{section}{References}
 \begin{thebibliography}{99}

\bibitem{2-gr} %1
 Workshop on Explainable Artificial Intelligence Proceedings, 2017.
 {\sf  
http://home.earthlink.net/dwaha/research/\linebreak meetings/ijcai17-xai/}.

\bibitem{1-gr} %2
DARPA sets up fast track for third wave AI~// Pakistan Defence, July~26, 2018.

 
\bibitem{3-gr}
\Au{Verma P., Srivastava~S.} Learning causal models of autonomous agents using interventios~// Workshop on Generalization in Planning Proceedings,
 2021. \mbox{arXiv}:\linebreak 2108.09586 [cs.AI]. {\sf 
https://pulkitverma.net/assets/ pdf/vs\_genplan21/vs\_genplan21.pdf}.
\bibitem{4-gr}
\Au{Halpern J.\,Y., Pearl~J.} Causes~//
{Brit. J.~Philos. Sci.} Vol.~56. No.\,4. P.~843--887.

\bibitem{7-gr} %5
\Au{Pearl J.} Causal inference~// Workshop on Causality  Proceedings: Objectives and Assessment at NIPS~/ 
Eds.\ I.~Guyon, D.~Janzing, B.~Sch$\ddot{\mbox{o}}$lkopf.~--- 
Proceedings of machine learning research ser.~--- Whistler, Canada, 2010. Vol.~6. P.~39--58. 

\bibitem{6-gr} %6
\Au{Pearl J.} The mathematics of causal inference~// \textit{Joint Statistical Meetings 
Proceedings}.~--- ASA, 2013. P.~2515--2529.

\bibitem{5-gr} %7
\Au{Halpern J.\,Y.} A~modification of the Halpern--Pearl definition of causality~// 4th 
Conference (International) on Artificial Intelligence Proceedings.~--- AAAI, 2015. P.~3022--3033. 


\bibitem{8-gr}
\Au{Yao L., Chu~Z., Li~S., Li~Y., Gao~J., Zhang~A.} A~survey on causal inference~// ACM T. Knowl. 
Discov.~D., 2021. Vol.~15. Iss.~5. Article 74. 46~p. doi: 10.1145/3444944.
\bibitem{9-gr}
\Au{Agrawal R., Gunopulos~D., Leymann~F.} Mining process models from workflow logs~// 
Advances in database technology~/ Eds. H.-J.~Schek, G.~Alonso, F.~Saltor, I.~Ramos.~--- Lecture 
notes in computer science ser.~--- Berlin, Heidelberg: Springer-Verlag, 1998. Vol.~1377. P.~467--483. 
doi: 10.1007/BFb0101003.
\bibitem{10-gr}
\Au{Грушо А.\,А., Грушо Н.\,А., Забежайло~М.\,И., Смирнов~Д.\,В., Тимонина~Е.\,Е., 
Шоргин~С.\,Я.} 
Статистика и~кластеры в~поисках аномальных вкраплений в~условиях больших данных~// 
Информатика и~её применения, 2021. Т.~15. Вып.~4. С.~79--86.
\bibitem{11-gr}
\Au{Grusho A., Grusho~N., Zabezhailo~M., Timonina~E.} Evaluation of trust in computer-computed 
results~// Comm. Com. Inf. Sc., 2022. Vol.~1552. P.~420--432.

\bibitem{14-gr} %12
\Au{Grusho A., Grusho~N., Zabezhailo~M., Zatsarinny~A., Timonina~E.} Information security of 
SDN on the basis of metadata~// Computer network security~/ Eds. J.~Rak, J.~Bay, I.\,V.~Kotenko, 
\textit{et al.}~--- Lecture notes in computer science ser.~--- Springer, 2017. Vol.~10446. P.~339--347. 

\bibitem{12-gr} %13
\Au{Грушо Н.\,А., Грушо~А.\,А., Забежайло~М.\,И., Тимонина~Е.\,Е.} Методы нахождения 
причин сбоев в~информационных технологиях с~помощью метаданных~// Информатика и~её 
применения, 2020. Т.~14. Вып.~2. С.~33--39. doi: 10.14357/19922264200205.
\bibitem{13-gr} %14
\Au{Грушо А.\,А., Тимонина Е.\,Е., Грушо~Н.\,А., Терехина~И.\,Ю.} Выявление аномалий  
с~по\-мощью метаданных~// Информатика и~её применения, 2020. Т.~14. Вып.~3. С.~76--80. doi: 10.14357/19922264200311.

\bibitem{15-gr}
\Au{Grusho A., Grusho~N., Zabezhailo~M., Timonina~E., Senchilo~V.} Metadata for root cause 
analysis~// Communications ECMS, 2021. Vol.~35. Iss.~1. P.~267--271. doi: 10.7148/ 2021-0267.
\bibitem{16-gr}
\Au{Грушо А.\,А., Грушо~Н.\,А., Забежайло~М.\,И., Тимонина~Е.\,Е.} Использование 
противоречий в~данных для поиска неявных сбоев в~компьютерных сис\-те\-мах~//  
Проб\-ле\-мы информационной безопас\-ности. Компьютерные системы, 2021. №\,3(47).  
С.~63--71. 
\end{thebibliography}

 }
 }

\end{multicols}

\vspace*{-8pt}

\hfill{\small\textit{Поступила в~редакцию 17.04.22}}

\vspace*{8pt}

%\pagebreak

%\newpage

%\vspace*{-28pt}

\hrule

\vspace*{2pt}

\hrule

%\vspace*{-2pt}

\def\tit{CAUSE-AND-EFFECT CHAIN ANALYSIS}


\def\titkol{Cause-and-effect chain analysis}


\def\aut{A.\,A.~Grusho, N.\,A.~Grusho, M.\,I.~Zabezhailo, A.\,A.~Zatsarinny, E.\,E.~Timonina,\\ 
and~S.\,Ya.~Shorgin}

\def\autkol{A.\,A.~Grusho, N.\,A.~Grusho, M.\,I.~Zabezhailo, et al.}
%A.\,A.~Zatsarinny, E.\,E.~Timonina$^1$,  and~S.\,Ya.~Shorgin}

\titel{\tit}{\aut}{\autkol}{\titkol}

\vspace*{-15pt}


\noindent
Federal Research Center ``Computer Science and Control'' of the Russian Academy 
of Sciences, 44-2~Vavilov Str., Moscow 119333, Russian Federation


\def\leftfootline{\small{\textbf{\thepage}
\hfill INFORMATIKA I EE PRIMENENIYA~--- INFORMATICS AND
APPLICATIONS\ \ \ 2022\ \ \ volume~16\ \ \ issue\ 2}
}%
 \def\rightfootline{\small{INFORMATIKA I EE PRIMENENIYA~---
INFORMATICS AND APPLICATIONS\ \ \ 2022\ \ \ volume~16\ \ \ issue\ 2
\hfill \textbf{\thepage}}}

\vspace*{3pt} 



\Abste{The paper is devoted to the analysis of the possibilities of using cause-and-effect 
relationships in the control of the realization of information technologies in distributed information 
systems. To carry out this analysis,
the simple example of a~fragment of some fixed information 
technology has been built which examines situations\linebreak\vspace*{-12pt}}

\Abstend{where control can rely on cause-and-effect 
relationships between actions in information technologies and when alleged cause-and-effect 
relationships simply do not exist. The paper shows the limitations of using cause-and-effect 
relationships to increase confidence in the results of complex computer calculations. This limitation is 
based on the fact that in order to get causal relationships in waiting of the certain result, it is often 
impossible to control the relationships of characteristics that are mandatory for the desired 
consequence to be obtained from actions that are assumed to be its cause. To use cause-and-effect 
relationships in information technology, it is necessary to supplement the actions of information 
technology with actions to control the relationships between characteristics. Without this condition, the 
properly constructed sequence of information technology actions is a necessary but insufficient 
condition of the expected consequence.}

\KWE{information security; cause-and-effect relations; monitoring of information technologies}

\DOI{10.14357/19922264220209}

%\vspace*{-16pt}

%\Ack
%\noindent




%\vspace*{4pt}

  \begin{multicols}{2}

\renewcommand{\bibname}{\protect\rmfamily References}
%\renewcommand{\bibname}{\large\protect\rm References}

{\small\frenchspacing
 {%\baselineskip=10.8pt
 \addcontentsline{toc}{section}{References}
 \begin{thebibliography}{99}
 
 \bibitem{2-gr-1} %1
Workshop on Explainable Artificial Intelligence Proceedings. 2017. Available at: {\sf  
http://home.earthlink.net/\linebreak dwaha/research/meetings/ijcai17-xai/} (accessed April~28, 2022).

\bibitem{1-gr-1} %2
DARPA sets up fast track for third wave AI. July~26, 2018. Available at: {\sf 
https://defence.pk/pdf/threads/darpa-sets-up-fast-track-for-third-wave-ai.569563/} (accessed April~28, 
2022).

\bibitem{3-gr-1}
\Aue{Verma, P., and S.~Srivastava.} 2021. Learning causal models of autonomous agents using 
interventions. \textit{Workshop on Generalization in Planning Proceedings}. Available at: {\sf 
https://pulkitverma.net/assets/pdf/vs\_genplan21/vs\_\linebreak genplan21.pdf} (accessed April~28, 2022).
\bibitem{4-gr-1}
\Aue{Halpern, J.\,Y., and J.~Pearl.} 2005. Causes and explanations: A~structural-model approach. 
Part~I: Causes. \textit{Brit. J.~Philos. Sci.} 56(4):843--887.

\bibitem{7-gr-1} %5
\Aue{Pearl, J.} 2010. Causal inference. \textit{Workshop on Causality  Proceedings: Objectives and Assessment at NIPS}. 
Eds.\ I.~Guyon, D.~Janzing, and B.~Sch$\ddot{\mbox{o}}$lkopf. 
Proceedings of machine learning research ser. Whistler, Canada. 6:39--58. 

\bibitem{6-gr-1}
\Aue{Pearl, J.} 2013. The mathematics of causal inference. \textit{Joint Statistical Meetings 
Proceedings}. ASA. 2515--2529.

\bibitem{5-gr-1} %7
\Aue{Halpern, J.\,Y.} 2015. A~modification of the Halpern--Pearl definition of causality. \textit{24th 
Conference (International) on Artificial Intelligence Proceedings}. AAAI. 3022--3033. 


\bibitem{8-gr-1}
\Aue{Yao, L., Z.~Chu, S.~Li, Y.~Li, J.~Gao, and A.~Zhang.} 2021. A~survey on causal inference. 
\textit{ACM T. Knowl. Discov.~D.} 15(5):74. 46~p. doi: 10.1145/3444944.
\bibitem{9-gr-1}
\Aue{Agrawal, R., D.~Gunopulos, and F.~Leymann.} 1998. Mining process models from workflow 
logs. \textit{Advances in database technology}. Eds. H.\,J.~Schek, G.~Alonso, F.~Saltor, and 
I.~Ramos. Lecture notes in computer science ser. Berlin, Heidelberg: Springer-Verlag. 1337:467--483.
doi: 10.1007/BFb0101003.
\bibitem{10-gr-1}
\Aue{Grusho, A.\,A., N.\,A.~Grusho, M.\,I.~Zabezhailo, D.\,V.~Smirnov, E.\,E.~Timonina and 
S.\,Ya.~Shorgin.} 2021. Statistika i~klastery v~poiskakh anomal'nykh vkrapleniy v~usloviyakh 
bol'shikh dannykh [Statistics and clusters for detection of anomalous insertions in Big Data 
environment]. \textit{Informatika i~ee Primeneniya~--- Inform. Appl.} 15(4):79--86.
\bibitem{11-gr-1}
\Aue{Grusho, A., N.~Grusho, M.~Zabezhailo, and E.~Timonina.} 2022. Evaluation of trust in 
computer-computed results.  \textit{Comm. Com. Inf. Sc.} 1552:420--432.

\bibitem{14-gr-1} %12
\Aue{Grusho, A., N.~Grusho, M.~Zabezhailo, A.~Zatsarinny, and E.~Timonina.} 2017. Information 
security of SDN on the basis of metadata. \textit{Computer network security}. Eds. J.~Rak, J.~Bay, 
I.~V. Kotenko, \textit{et al.} Lecture notes in computer science ser. Springer. 10446:339--347.

\bibitem{12-gr-1} %13
\Aue{Grusho, N.\,A., A.\,A.~Grusho, M.\,I.~Zabezhailo, and E.\,E.~Timonina.} 2020. Metody 
nakhozhdeniya prichin sboev v~informatsionnykh tekhnologiyakh s~pomoshch'yu metadannykh 
[Methods of finding the causes of information technology failures by means of metadata]. 
\textit{Informatika i~ee Primeneniya~--- Inform. Appl.} 14(2):33--39. doi: 10.14357/19922264200205.
\bibitem{13-gr-1} %14
\Aue{Grusho, A.\,A., E.\,E.~Timonina, N.\,A.~Grusho, and I.\,Yu.~Teryokhina}. 2020. Vyyavlenie 
anomaliy s pomoshch'yu metadannykh [Identifying anomalies using metadata]. \textit{Informatika 
i~ee Primeneniya~--- Inform. Appl.} 14(3):76--80. doi: 10.14357/19922264200311.

\bibitem{15-gr-1}
\Aue{Grusho, A., N.~Grusho, M.~Zabezhailo, E.~Timonina, and V.~Senchilo}. 2021. Metadata for 
root cause analysis. \textit{Communications ECMS} 35(1):267--271. doi: 10.7148/ 2021-0267.
\bibitem{16-gr-1}
\Aue{Grusho, A.\,A., N.\,A.~Grusho, M.\,I.~Zabezhailo, and E.\,E.~Timonina.} 2021. Use of 
contradictions in data for finding implicit failures in computer systems. \textit{Autom. Control  
Comp.~S.} 55(8):1115--1120.
\end{thebibliography}

 }
 }

\end{multicols}

\vspace*{-7pt}

\hfill{\small\textit{Received April 17, 2022}}

\vspace*{-22pt}

\Contr

\vspace*{-3pt}

\noindent
\textbf{Grusho Alexander A.} (b.\ 1946)~--- Doctor of Science in physics and mathematics, professor, 
principal scientist, Institute of Informatics Problems, Federal Research Center ``Computer Science and 
Control'' of the Russian Academy of Sciences, 44-2~Vavilov Str., Moscow 119333, Russian 
Federation; \mbox{grusho@yandex.ru}

\vspace*{3pt}

\noindent
\textbf{Grusho Nikolai A.} (b.\ 1982)~--- Candidate of Science (PhD) in physics and mathematics, 
senior scientist, Institute of Informatics Problems, Federal Research Center ``Computer Science and 
Control'' of the Russian Academy of Sciences, 44-2~Vavilov Str., Moscow 119133, Russian 
Federation; \mbox{info@itake.ru}

\pagebreak

\noindent
\textbf{Zabezhailo Michael I.} (b.\ 1956)~--- Doctor of Science in physics and mathematics, principal 
scientist, A.\,A.~Dorodnicyn Computing Center, Federal Research Center ``Computer Science and 
Control'' of the Russian Academy of Sciences, 40~Vavilov Str., Moscow 119333, Russian Federation; 
\mbox{m.zabezhailo@yandex.ru}

\vspace*{3pt}

\noindent
\textbf{Timonina Elena E.} (b.\ 1952)~--- Doctor of Science in technology, professor, leading scientist, 
Institute of Informatics Problems, Federal Research Center ``Computer Science and Control'' of the 
Russian Academy of Sciences, 44-2~Vavilov Str., Moscow 119133, Russian Federation; 
\mbox{eltimon@yandex.ru}

\vspace*{3pt}

\noindent
\textbf{Zatsarinny Alexander A.} (b.\ 1951)~--- Doctor of Science in technology, professor, principal 
scientist, Federal Research Center ``Computer Science and Control'' of the Russian Academy of 
Sciences; 44-2~Vavilov Str., Moscow 119333, Russian Federation; \mbox{AZatsarinny@ipiran.ru}

\vspace*{3pt}

\noindent
\textbf{Shorgin Sergey Ya.} (b.\ 1952)~--- Doctor of Science in physics and mathematics, professor, 
principal scientist, Institute of Informatics Problems, Federal Research Center ``Computer Science and 
Control'' of the Russian Academy of Sciences, 44-2~Vavilov Str., Moscow 119133, Russian 
Federation; \mbox{sshorgin@ipiran.ru}

   

\label{end\stat}

\renewcommand{\bibname}{\protect\rm Литература}    