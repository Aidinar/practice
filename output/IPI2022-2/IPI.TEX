\documentclass[10pt]{book}
\usepackage[utf8]{inputenc}

\usepackage{latexsym,amssymb,amsfonts,amsmath,amsxtra,dsfont,
indentfirst,shapepar,%fleqn,%
picinpar,shadow,floatflt,enumerate,multicol,colortbl,moreverb,cite,ipi}

\usepackage{rotating}
\usepackage{mathrsfs}
\usepackage[noend]{algorithmic}
\usepackage{ulem}
\usepackage{graphicx}
%\usepackage{algorithm2e}
\usepackage[linesnumbered,boxed,ruled]{algorithm2e}
%\usepackage{xypic}
\usepackage{oldgerm}
\usepackage{epic}
\usepackage{eepic}

\SetAlgorithmName{Algorithm}{алгоритм}{Список алгоритмов}

%из Дюковой

\newcommand{\algKeyword}[1]{{\bf #1}}
\newcommand{\Proc}[1]{\text{\tt #1}}
\def\CALL{\algKeyword{call}~}

\newenvironment{AlgProcedure}[1]
{
\small
\medskip
%    \hrule
\medskip
\algKeyword{PROCEDURE} #1
\begin{algorithmic}[1]}
{\end{algorithmic}
%    \hrule
\bigskip
}

\def\CALL{\algKeyword{call}~}

%конец для Дюковой

%\RequirePackage[ruled]{algorithm}


\input{epsf}

%\nofiles

%\includeonly{avtor}    %pdf
%\includeonly{podgot-rus-site,podgot-eng-site}  
%\includeonly{podgot-rus,podgot-eng}  
%\includeonly{ipi-ind} 
%\includeonly{index-15i}
%\includeonly{toc-rus, toc-en}
%\includeonly{toc-rus}
%\includeonly{toc-en} 
%\includeonly{popravka}



%\includeonly{agasand}                %1pdf+авт+
%\includeonly{kravtsova}              %2pdf+авт+
%\includeonly{bosov}                  %3pdf+авт+
%\includeonly{zeifman}                %4pdf+авт+
%\includeonly{torshin}                %5pdf+авт+
%\includeonly{shestakov}              %6pdf+авт+
%\includeonly{inkova}                 %7pdf+авт+
%\includeonly{zatsman}                %8pdf+авт+
%\includeonly{grusho}                 %9pdf+авт+
%\includeonly{shnurkov}    %10pdf+авт+
%\includeonly{sinitsin}    %11pdf+
%\includeonly{rumovskaya}             %12pdf+авт+
%\includeonly{beschast}    %13pdf
%\includeonly{konovalov}   %14pdf

%%%%\includeonly{nekrolog-new}



%\includeonly{rekl}




\usepackage{acad}
%\usepackage{courier}
\usepackage{decor}
\usepackage{newton}
\usepackage{pragmatica}
\usepackage{zapfchan}
\usepackage{petrotex}
\usepackage{bm}                     % полужирные греческие буквы
\usepackage{upgreek}                % прямые греческие буквы \upalpha
\usepackage{eufrak}
\usepackage{verbatim}

\renewcommand{\bottomfraction}{0.99}
\renewcommand{\topfraction}{0.99}
\renewcommand{\textfraction}{0.01}

\setcounter{secnumdepth}{1} %здесь - 3 + chapter = 4

\arraycolsep=1.5pt

%\usepackage[pdftex]{graphicx}

%\usepackage{oz}

%NEW COMMANDS


\renewcommand*{\hm}[1]{#1\nobreak\discretionary{}%
            {\hbox{$\mathsurround=0pt #1$}}{}} %% Дублирует знаки операций
                               %при переносе в формуле (перед знаком, который
                               %надо продублировать ставится команда \hm)

%\newcommand{\endproof}{\hfill$\Box$}
\renewcommand{\r}{\mathbb{R}}
%\newcommand{\I}{{\rm I\hspace{-0.7mm}I}}
%\newcommand{\Ikl}{{\tt{1}}\hspace*{-1.44mm}\mathtt{1}}
\newcommand{\Ik}{\mbox{{\small \tt {1}}\hspace{-1.3mm}{\tt 1}}}
\newcommand{\argmin}{\mathop{\mathrm{arg}\,\mathrm{min}}}
\newcommand{\argmax}{\mathop{\mathrm{arg}\,\mathrm{max}}}
%\newcommand{\capr}{\mathop{\cap\,}}
%\newcommand{\cupr}{\mathop{\cup\,}}
%\def\argmin{\mathop{arg\,min}}

\def\vrp{\varphi}
\def\prt{\partial}
\def\mm{{\sf M}}
\def\modnop#1{\mathop{#1}\limits_{n}}
\def\eam{\mathbin{{\mathop{=}\limits^{\mathrm{def}}}}}
\def\dey#1#2{#1 (#2)}
\def\deyc#1#2{#1 \cdot  #2}
\def\ra#1{\;\mathop{\to}\limits^{#1}\;}
\def\raz#1{\;\mathop{\longrightarrow}\limits^{\!\!\!#1}\;}
\def\ral#1{\;\mathop{\longrightarrow}\limits^{#1}\;}

\newcommand{\Nor}{\mathcal{N}}
\newcommand{\T}{\mathbb{T}}
\newcommand{\Z}{\mathbb{Z}}



\newcommand{\il}[2]{\int\limits_{#1}^{#2}}%интеграл с пределами #1 и #2

\def\sm2{\mathop {\sum\limits^{n^\Theta}\sum\limits^{n^\Theta}}}
\def\sss{\sum\limits}
\def\tr{,\,\ldots\,,\,}
\def\rk{\right]}
\def\lk{\left[}
\def\rf{\right\}}
\def\lf{\left\{}
\def\lv{\,\left\vert}
\def\rv{\right\vert\,}
\def\iii{\int\limits}
\def\iin{\int\limits_{-\infty}^\infty}
\def\rrv{\right\vert}


\def\ee{{\cal E}}
\def\ww{{\cal W}}
\def\yy{{\cal Y}}
\def\vv{{\cal V}}

\newcommand{\R}{\mathbb R}
\newcommand{\E}{\mathbb E}
\newcommand{\N}{\mathbb N}

\renewcommand{\P}{\mathbb{P}}

\newcommand{\h}{{\bf H}}
\newcommand{\p}{{\sf P}}  % вероятность

\newcommand{\e}{{\sf E}}  % мат. ожидание
\newcommand{\D}{{\sf D}}  % дисперсия
\newcommand{\eps}{\varepsilon}
\newcommand{\vp}{{\mathbf p}}
\newcommand{\vz}{{\mathbf z}}
\newcommand{\vx}{{\mathbf x}}
\newcommand{\vf}{{\mathbf f}}
\newcommand{\F}{{\mathcal F}}
\def\ap{{\mathrm{ЭР}}}
\newcommand{\ud}{\Delta_n} %uniform ditance
\newcommand{\nud}{\Delta_n(x)}
%\renewcommand{\Re}{\mathrm{Re}\,}

\newcommand{\abs}[1]{\left\vert#1\right\vert}

\newcommand{\norm}[1]{\left\Vert#1\right\Vert}
\def\da{(\Delta_t,A)}

\newcommand{\corr}{\mathrm{corr}}

\newcommand{\cov}{\mathrm{cov}}
\newcommand{\Expect}{\mathbb{E}}

\def\w{\omega}
\def\W{\Omega}

\def\inh{\int\limits_{nh}^{(n+1)h}}

\def\sumin{\sum_{i=1}^N}


\def\bxt{(Y,t)}
\def\xt{(y,t)}

\def\ovth{{\fr{\tau-nh}{h}}}
\def\ov{\overline}
\def\tm{\tilde m}
\def\tl{\tilde\lambda}
\def\tB{\widetilde B}
\def\tb{\tilde b}
\def\ld{\ldots}
\def\cd{\cdots}


\DeclareMathOperator{\sign}{sign}

%\newcommand{\gr}{{\geqslant}}


\newcommand{\g}{\mbox{\textit{g}}}

\renewcommand{\la}{\lambda}
\newcommand{\si}{\sigma}
\newcommand{\alp}{\alpha}

\newcommand{\pto}{\stackrel{P}{\longrightarrow}} % сходимость по веpоятности

\newcommand{\eqd}{\stackrel{\mathrm{d}}{=}} % равенство по pаспpеделению
\newcommand{\eqdelta}{\stackrel{\triangle}{=}} % равенство по pаспpеделению

\def\be#1{\begin{equation}\label{#1}}
\def\ee{\end{equation}}
\def\re#1{(\ref{#1})}

\def\bn{\begin{enumerate}}
\def\en{\end{enumerate}}
\def\bi{\begin{itemize}}
\def\ei{\end{itemize}}
%\def\i{\item}

%\newcommand{\kp}{\kappa}
%\def\Q{{\cal Q}} \def\H{{\cal H}}
%\newcommand{\bet}{\beta_{2+\delta}}


%\newtheorem{definition}{Определение}
%\renewcommand{\thedefinition}{\arabic{definition}.}
%END NEW COMMANDS

%\renewcommand{\baselinestretch}{1.2}

%\pagestyle{myheadings}

\setlength{\textwidth}{167mm}      % 122mm
\setlength{\textheight}{658pt}
%\setlength{\textheight}{635.6pt}
\setlength{\columnsep}{4.5mm}

\setcounter{secnumdepth}{4}

%\addtolength{\headheight}{2pt}
%\addtolength{\headsep}{-2mm}

\addtolength{\topmargin}{-7mm}  % for printing


%\hoffset=-30mm  % From Yap
\hoffset=-23mm  % From Acrobat

%\voffset=0mm % From Yap
\voffset=-5mm   % From Acrobat

%\addtolength{\evensidemargin}{-2.5mm} % for printing
%\addtolength{\oddsidemargin}{2.5mm}  % for printing

\addtolength{\evensidemargin}{-12mm} % for printing
\addtolength{\oddsidemargin}{8mm}  % for printing

%\renewcommand{\thefootnote}{\fnsymbol{footnote}}
%\renewcommand{\thefootnote}{\arabic{footnote}}
\renewcommand{\figurename}{\protect\bf Рис.}
\renewcommand{\tablename}{\protect\bf Таблица}

\newcommand{\Caption}[1]{\caption{\protect\small %\baselineskip=2.5ex
#1}}

\renewcommand{\thefigure}{\arabic{figure}}
\renewcommand{\thetable}{\arabic{table}}
\renewcommand{\theequation}{\arabic{equation}}
\renewcommand{\thesection}{\arabic{section}}

\renewcommand{\contentsname}{СОДЕРЖАНИЕ}
\newcommand{\fr}[2]{\displaystyle\frac{\displaystyle #1\mathstrut}{\displaystyle #2\mathstrut}}

%\renewcommand{\thefootnote}{\fnsymbol{footnote}}
%\newcommand{\g}{\mbox{\textit{g}}}

%\newcommand{\Caption}[1]{\caption{\protect\small\baselineskip=2ex #1}}
\newcounter{razdel}
\setcounter{razdel}{0}

\def\god{2022}
\def\tom{16}
\def\vyp{2}


\newcommand{\titel}[4]{%
\

\vspace*{5pt}

\ifodd\therazdel {\raggedright\noindent\Large\textrm\textbf
 \lineskip .75em
  \baselineskip=3.2ex #1 \par}
\vskip 1em {\noindent\large\textrm\textbf #2 \par}
\addcontentsline{toc}{subsection}{{\textrm\textbf #1}\protect\newline #2}
\def\rightheadline{\underline{\noindent\hbox to \textwidth{\hfill\small\textrm{#4}
%\hfill \large\bf\thepage
}}}
\def\leftheadline{\underline{\noindent\parbox{\textwidth}{
%\raggedleft\large\bf\thepage \hfill
\small\textit{#3}\hfill}}}
\def\leftfootline{\small{\textbf{\thepage}
\hfill ИНФОРМАТИКА И ЕЁ ПРИМЕНЕНИЯ\ \ \ том~\tom\ \ \ выпуск~\vyp\ \ \ \god}
}%
 \def\rightfootline{\small{ИНФОРМАТИКА И ЕЁ ПРИМЕНЕНИЯ\ \ \ том~\tom\ \ \ выпуск~\vyp\ \ \ \god
\hfill \textbf{\thepage}}}
\vskip 2em \setcounter{figure}{0}
\setcounter{table}{0}
\setcounter{equation}{0}
\setcounter{section}{0}
\setcounter{subsection}{0}
\setcounter{subsubsection}{0}
\setcounter{footnote}{0}
\setcounter{razdel}{0}
%\end{flushleft}
\else {
 \raggedright\noindent\Large\textrm\textbf
 \lineskip .75em
\baselineskip=3.2ex #1 \par} \vskip 1em
%\begin{flushleft}
{\noindent\large\textrm\textbf #2 \par}
\addcontentsline{toc}{subsection}{{\textrm\textbf #1}\protect\newline #2}
\def\rightheadline{\underline{\noindent\hbox to \textwidth{\hfill\small\textrm{#4}
%\hfill \large\bf\thepage
}}}
\def\leftheadline{\underline{\noindent\parbox{\textwidth}{%\raggedleft\large\bf\thepage \hfill
\small\textit{#3}\hfill}}}
\def\leftfootline{\small{\textbf{\thepage}
\hfill ИНФОРМАТИКА И ЕЁ ПРИМЕНЕНИЯ\ \ \ том~\tom\ \ \ выпуск~\vyp\ \ \ \god}
}%
 \def\rightfootline{\small{ИНФОРМАТИКА И ЕЁ ПРИМЕНЕНИЯ\ \ \ том~16\ \ \ выпуск~\vyp\ \ \ 2022
\hfill \textbf{\thepage}}} \vskip 2em \setcounter{figure}{0}
\setcounter{table}{0} \setcounter{equation}{0} \setcounter{section}{0}
\setcounter{subsection}{0} \setcounter{subsubsection}{0}
\setcounter{footnote}{0}
%\end{flushleft}
\fi}

\newcommand{\titelr}[2]{%
\

\vspace*{5pt}

\ifodd\therazdel {\raggedright\noindent%\Large\textrm\textbf
 \lineskip .75em
  \baselineskip=3.2ex #1 \par}
\vskip 1em {\noindent\normalsize\textrm\textbf #2 \par}
\else {
 \raggedright\noindent\Large\textrm\textbf
 \lineskip .75em
\baselineskip=3.2ex #1 \par} \vskip 1em
%\begin{flushleft}
{\noindent\large\textrm\textbf #2 \par
%\noindent\normalsize\textrm\textbf #2 \par
} \fi}

\newcommand{\titele}[5]{%
\

%\vspace*{5pt}

\ifodd\therazdel {\raggedright\noindent\large
\textrm\textbf
 \lineskip .75em
%  \baselineskip=3.2ex
#1 \par}
\vskip .5em {\noindent\large\textrm\textbf #2 \par}
\vskip .5em
 {\noindent\textrm #3 \par}
\addcontentsline{toc}{subsection}{{\textrm\textbf #1}\protect\newline #2}
\def\rightheadline{\underline{\noindent\hbox to \textwidth{\hfill\small\textrm{#4}
%\hfill \large\bf\thepage
}}}
\def\leftheadline{\underline{\noindent\parbox{\textwidth}{
%\raggedleft\large\bf\thepage \hfill
\small\textrm{#5}\hfill}}}
\def\leftfootline{\small{\textbf{\thepage}
\hfill ИНФОРМАТИКА И ЕЁ ПРИМЕНЕНИЯ\ \ \ том~16\ \ \ выпуск~2\ \ \ 2022}
}%
 \def\rightfootline{\small{ИНФОРМАТИКА И ЕЁ ПРИМЕНЕНИЯ\ \ \ том~16\ \ \ выпуск~2\ \ \ 2022
\hfill \textbf{\thepage}}} \vskip 1em \setcounter{figure}{0}
\setcounter{table}{0} \setcounter{equation}{0} \setcounter{section}{0}
\setcounter{subsection}{0} \setcounter{subsubsection}{0}
\setcounter{footnote}{0} \setcounter{razdel}{0}
%\end{flushleft}
\else {
 \raggedright\noindent\large
 \textrm\textbf
 \lineskip .75em
%\baselineskip=3.2ex
#1 \par} \vskip .5em
%\begin{flushleft}
{\noindent\large\textrm\textbf #2 \par} \vskip .5em
 {\noindent\textrm #3 \par}
\addcontentsline{toc}{subsection}{{\textrm\textbf #1}\protect\newline #2}
\def\rightheadline{\underline{\noindent\hbox to \textwidth{\hfill\small\textrm{#4}
%\hfill \large\bf\thepage
}}}
\def\leftheadline{\underline{\noindent\parbox{\textwidth}{%\raggedleft\large\bf\thepage \hfill
\small\textrm{#5}\hfill}}}
\def\leftfootline{\small{\textbf{\thepage}
\hfill ИНФОРМАТИКА И ЕЁ ПРИМЕНЕНИЯ\ \ \ том~16\ \ \ выпуск~2\ \ \ 2022}
}%
 \def\rightfootline{\small{ИНФОРМАТИКА И ЕЁ ПРИМЕНЕНИЯ\ \ \ том~16\ \ \ выпуск~2\ \ \ 2022
\hfill \textbf{\thepage}}} \vskip 1em \setcounter{figure}{0}
\setcounter{table}{0} \setcounter{equation}{0} \setcounter{section}{0}
\setcounter{subsection}{0} \setcounter{subsubsection}{0}
\setcounter{footnote}{0}
%\end{flushleft}
\fi}

\def\Abst#1{
\begin{center}\small\nwt
\parbox{150mm}{%\baselineskip=2.5ex
\textbf{Аннотация:}\ \
%\hspace*{\parindent}
#1}
\end{center}}
\def\Abste#1{
\begin{center}\small\nwt
\parbox{150mm}{%\baselineskip=2.5ex
\textbf{Abstract:}\ \
%\hspace*{\parindent}
#1}
\end{center}}

\def\DOI#1{
\begin{center}\small\nwt
\parbox{150mm}{%\baselineskip=2.5ex
\textbf{DOI:}\ \
%\hspace*{\parindent}
#1}
\end{center}}

\def\Abstend#1{
\begin{center}\small\nwt
\parbox{150mm}{%\baselineskip=2.5ex
%\hspace*{\parindent}
#1}
\end{center}}


\def\KW#1{
\begin{center}\small\nwt
\parbox{150mm}{%\baselineskip=2.5ex
\textbf{Ключевые слова:}\ \ #1}
\end{center}}

\def\KWE#1{
\begin{center}\small\nwt
\parbox{150mm}{%\baselineskip=2.5ex
\textbf{Keywords:}\ \ #1}
\end{center}}


\def\KWN#1{
%\begin{center}
%\small
%\parbox{150mm}\end{center}
}

\newcommand{\Avtors}[1]{%\smallskip
%\vspace*{.5pt}
\hangindent=23pt\noindent
%\nwt
{\bfseries#1}\
}


\renewcommand{\thesubsection}{\thesection.\arabic{subsection}\hspace*{-5pt}}
\renewcommand{\thesubsubsection}{\thesubsection\hspace*{5pt}.\arabic{subsubsection}\hspace*{-3pt}}

\newcommand{\Ack}{\section*{\protect\rmfamily Acknowledgments}\noindent}
\newcommand{\Contr}{\section*{\protect\rmfamily Contributors}\noindent}
\newcommand{\Contrl}{\section*{\protect\rmfamily Contributor}\noindent}

\makeindex


\begin{document}
\Rus

\nwt
%\ptb


%\renewcommand{\contentsname}{\protect\Large\bf Содержание}

\setcounter{tocdepth}{2}

%\tableofcontents

\renewcommand{\bibname}{\protect\rmfamily Литература}
  \def\Au#1{{\it #1}}
    \def\Aue#1{{#1}}

%\newcommand{\No}{№}
  \newcommand{\tg}{\,\mathrm{tg}\,}
    \newcommand{\ctg}{\,\mathrm{ctg}\,}
  \newcommand{\arctg}{\,\mathrm{arctg}\,}

\def\forallb{\mathop{\forall}}
\def\cupb{\mathop{\cup}}
\def\existsb{\mathop{\exists}}


\newpage
\addtocounter{razdel}{1}
%\def\razd{РЕГУЛИРУЕМЫЙ ЭЛЕКТРОПРИВОД ДЛЯ ЭЛЕКТРОЭНЕРГЕТИКИ}


\setcounter{page}{2}

%   { %\Large  
   { %\baselineskip=16.6pt
   
   \vspace*{-48pt}
   \begin{center}\LARGE
   \textit{Предисловие}
   \end{center}
   
   %\vspace*{2.5mm}
   
   \vspace*{25mm}
   
   \thispagestyle{empty}
   
   { %\small 

    
Вниманию читателей журнала <<Информатика и её применения>> предлагается 
очередной тематический выпуск <<Вероятностно-статистические методы и 
задачи информатики и информационных технологий>>. Предыдущие тематические 
выпуски журнала по данному направлению вышли в 2008~г.\ (т.~2, вып.~2), 
в 2009~г.\ (т.~3, вып.~3) и в 2010~г.\ (т.~4, вып.~2). 

Статьи, собранные в данном журнале, посвящены разработке новых вероятностно-статистических 
методов, ориентированных на применение к решению конкретных задач информатики и информационных 
технологий, а также~--- в ряде случаев~--- и других прикладных задач. Проблематика, охватываемая 
публикуемыми работами, развивается в рамках научного сотрудничества между Институтом проблем 
информатики Российской академии наук (ИПИ РАН) и Факультетом вычислительной математики и 
кибернетики Московского государственного университета им.\ М.\,В.~Ломоносова в ходе работ 
над совместными научными проектами (в том числе в рамках функционирования 
Научно-образовательного центра <<Вероятностно-статистические методы анализа рисков>>). 
Многие из авторов статей, включенных в данный номер журнала, являются активными участниками 
традиционного международного семинара по проблемам устойчивости стохастических моделей, 
руководимого В.\,М.~Золотаревым и В.\,Ю.~Королевым; регулярные сессии этого семинара 
проводятся под эгидой МГУ и ИПИ РАН (в 2011~г.\ указанный семинар проводится в октябре 
в Калининградской области РФ). 

Наряду с представителями ИПИ РАН и МГУ в число авторов данного выпуска журнала входят 
ученые из Научно-исследовательского института системных исследований РАН, Института 
проблем технологии микроэлектроники и особочистых материалов РАН, Института 
прикладных математических исследований Карельского НЦ РАН, Московского 
авиационного института, Вологодского государственного педагогического университета, 
НИИММ им.\ Н.\,Г.~Чеботарева, Казанского государственного университета, Дебреценского 
университета (Венгрия).

Несколько статей выпуска посвящено разработке и применению стохастических методов и 
информационных технологий для решения различных прикладных задач. В~работе В.\,Г.~Ушакова 
и О.\,В.~Шестакова рассмотрена задача определения вероятностных характеристик случайных 
функций по распределениям интегральных преобразований, возникающих в задачах эмиссионной 
томографии. В~статье Д.\,О.~Яковенко и М.\,А.~Целищева рассмотрены некоторые вопросы 
математической теории риска и предложен новый подход к диверсификации инвестиционных 
портфелей. Работа И.\,А.~Кудрявцевой и А.\,В.~Пантелеева посвящена построению и 
исследованию математической модели, описывающей динамику сильноионизованной плазмы. 
В~статье П.\,П.~Кольцова изучается качество работы ряда алгоритмов сегментации изображений. 
Статья А.\,Н.~Чупрунова и И.~Фазекаша посвящена вероятностному анализу числа без\-оши\-бочных 
блоков при помехоустойчивом кодировании; получены усиленные законы больших чисел для указанных 
величин.

В данном выпуске традиционно присутствует тематика, весьма активно разрабатываемая в течение 
многих лет специалистами ИПИ РАН и МГУ,~--- методы моделирования и управления для 
информационно-телекоммуникационных и вычислительных систем, в частности методы 
теории массового обслуживания. В~статье А.\,И.~Зейфмана с соавторами рассматриваются 
модели обслуживания, описываемые марковскими цепями с непрерывным временем в случае 
наличия катастроф. В~работе М.\,М.~Лери и И.\,А.~Чеплюковой рассматриваются случайные 
графы Интернет-типа, т.\,е.\ графы, степени вершин которых имеют степенные распределения; 
такие задачи находят применение при исследовании глобальных сетей передачи данных. 
Работа Р.\,В.~Разумчика посвящена исследованию систем массового обслуживания специального 
вида~--- с отрицательными заявками и хранением вытесненных заявок.

Ряд статей посвящен развитию перспективных теоретических 
вероятностно-статистических методов, которые находят широкое применение в различных 
задачах информатики и информационных технологий. В~работе В.\,Е.~Бенинга, А.\,К.~Горшенина 
и В.\,Ю.~Королева рассмотрена задача статистической проверки гипотез о числе компонент 
смеси вероятностных распределений, приводится конструкция асимптотически наиболее мощного 
критерия. Результаты этой работы найдут применение в ряде прикладных задач, использующих 
математическую модель смеси вероятностных распределений (в информатике, моделировании 
финансовых рынков, физике турбулентной плазмы и~т.\,д.). В~статье В.\,Ю.~Королева, 
И.\,Г.~Шевцовой и С.\,Я.~Шоргина строится новая, улучшенная оценка точности нормальной 
аппроксимации для пуассоновских случайных сумм; как известно, указанные случайные суммы 
широко используются в качестве моделей многих реальных объектов, в том числе в информатике, 
физике и других прикладных областях. Работа В.\,Г.~Ушакова и Н.\,Г.~Ушакова посвящена 
исследованию ядерной оценки плотности распределения; эти результаты могут применяться, 
в част\-ности, при анализе трафика в телекоммуникационных системах. Серьезные приложения 
в статистике могут получить результаты работы О.\,В.~Шестакова, в которой доказаны оценки 
скорости сходимости распределения выборочного абсолютного медианного отклонения к нормальному 
закону. 

\smallskip

Редакционная коллегия журнала выражает надежду, что данный тематический  выпуск 
будет интересен специалистам в области теории вероятностей и математической статистики 
и их применения к решению задач информатики и информационных технологий.
     
     %\vfill 
     \vspace*{20mm}
     \noindent
     Заместитель главного редактора журнала <<Информатика и её 
применения>>,\\
     директор ИПИ РАН, академик  \hfill
     \textit{И.\,А.~Соколов}\\
     
     \noindent
     Редактор-составитель тематического выпуска,\\
     профессор кафедры математической статистики факультета\\
      вычислительной математики и кибернетики МГУ им.\ М.\,В.~Ломоносова,\\
     ведущий научный сотрудник ИПИ РАН,\\ 
доктор физико-математических наук \hfill
      \textit{В.\,Ю.~Королев}
     
     } }
     }


   

\def\stat{agasandyan}

\def\tit{МНОГОМЕРНЫЕ БАТТЕРФЛЯИ\\ В~ЗАДАЧАХ ОПТИМИЗАЦИИ ПО CC-VaR}

\def\titkol{Многомерные баттерфляи в~задачах оптимизации 
по~CC-VaR}

\def\aut{Г.\,А.~Агасандян$^1$}

\def\autkol{Г.\,А.~Агасандян}

\titel{\tit}{\aut}{\autkol}{\titkol}

\index{Агасандян Г.\,А.}
\index{Agasandyan G.\,A.}


%{\renewcommand{\thefootnote}{\fnsymbol{footnote}} \footnotetext[1]
%{Работа выполнена при поддержке Министерства науки и~высшего образования
%Российской федерации, грант №\,075-15-2020-799.}}


\renewcommand{\thefootnote}{\arabic{footnote}}
\footnotetext[1]{Федеральный исследовательский центр <<Информатика и~управление>> Российской 
академии наук, \mbox{agasand17@yandex.ru}}

\vspace*{-6pt}
 
  
  \Abst{Работа продолжает исследование технических проблем, связанных с~применением  
континуального критерия VaR (CC-VaR) на многомерных рынках опционов. 
В~предположении, что на рынке сценарными баттерфляями непосредственно не торгуют, 
разрабатывается методика получения их реп\-ли\-ка\-ции из многомерных $\alpha$-оп\-ци\-онов~--- 
многомерного обобщения обычных одномерных опционов, таких как коллы и~путы. Работа 
служит непосредственным расширением предложенного в~предыдущей работе автора 
способа, позволяющего конструировать индикаторы базиса на многомерном сценарном 
рынке комбинациями многомерных бинарных опционов. Методика основывается на 
теоремах паритета для одномерного рынка традиционных опционов и~пригодна для рынков 
произвольной размерности, но ее фактическая реализация проводится для двумерных 
рынков. Приводятся конструкции базисов из $\alpha$-оп\-ци\-онов~--- как однотипных, так 
и~смешанных естественных с~выделенным цент\-ром рынка. Теоретические пред\-став\-ле\-ния 
оптимальных портфелей в~этих базисах иллюстрируются на примере конкретного 
двумерного рынка.}
   
  \KW{базовые активы; многомерный рынок; функция рисковых предпочтений инвестора; 
континуальный критерий VaR (CC-VaR); стоимостная и~прогнозная плотности; опционы 
колл и~пут; $\alpha$-оп\-ци\-оны; сценарные баттерфляи; базисы; центр рынка; портфели 
баттерфляев}

 \DOI{10.14357/19922264230114} 
  
\vspace*{-2pt}


\vskip 10pt plus 9pt minus 6pt

\thispagestyle{headings}

\begin{multicols}{2}

\label{st\stat}
   
  \section{Введение}
  
  Проблемы применения на рынках опционов введенного автором 
континуального критерия VaR (CC-VaR) рассматриваются в~[1--5]. Настоящую 
работу можно рассматривать как продолжение исследования~[6], в~котором 
предлагались варианты репликации индикаторов базиса на многомерном 
сценарном рынке комбинациями так называемых $\zeta$-оп\-ци\-онов 
(многомерных бинарных опционов).\linebreak Здесь подобная задача решается для более 
сложных инструментов~--- многомерных аналогов одномерных базисных 
баттерфляев, которые реплицируются комбинациями так называемых
  $\alpha$-оп\-ци\-онов~--- многомерных аналогов традиционных \mbox{опционов} типа 
колл и~пут. 
  
  Основания для такого рассмотрения и~его проб\-ле\-мы, связанные 
с~применением континуального критерия VaR (CC-VaR), приведены в~[6], там 
же вводятся многие обозначения, которые используются и~здесь. 
Теоретической моделью при построении $\alpha$-рын\-ка служит также 
многомерный $\delta$-ры\-нок~\cite{5-aga, 6-aga}. 
  
  В работе для многомерных рынков опционов решаются те же проблемы 
технического характера, что и~для $\zeta$-рын\-ков~--- рынков  многомерных 
бинарных опционов. Но на этот раз в~отношении своего инструментария они 
в~большей мере напоминают проб\-ле\-мы традиционных рынков опционов, на 
которых в~отсутствие баттерфляев в~качестве объектов непосредственной 
торговли предлагается получать их в~виде комбинаций коллов и~путов. 
  
  \section{Теоретический $\alpha$-рынок и~его~свойства}
  
  Вновь рассматривается многомерный $\delta$-ры\-нок (однопериодный, 
теоретический и~идеальный)\linebreak с~$n$ ($>1$) базовыми активами, векторы цен 
которых в~конце периода $\bm{x}\hm= (x_1, x_2, \ldots, x_n)$, $x_l\hm\in {\sf X}_l \hm\subset 
\mathfrak{R}$, $l\hm\in N\hm=\{1, \ldots , n\}$, образуют $n$-мер\-ное множество 
${\mathsf X}\hm=\prod_{l\in N} {\mathsf X}_l$. На~${\mathsf X}$ заданы 
\textit{прогнозная} $p(\bm{x})$ и~\textit{стоимостная} $c(\bm{x})$ плотности, 
по\-рож\-да\-ющие вероятностные меры~${\mathsf P}\{\cdot\}$ и~${\mathsf 
C}\{\cdot\}$. 
  
  Платежная функция произвольного инструмента~$\bm{I}$ обозначается 
$\pi(\bm{x}; \bm{I})$, его рыночная сто\-и\-мость и~средний, с~точки зрения 
инвестора, доход, рас\-счи\-тан\-ные по плотностям $c(\bm{x})$ и~$p(\bm{x})$ 
соответственно, определяются соотношениями: 
  $$
  \vert \bm{I}\vert =\int\limits_{\mathsf X} \pi (\bm{x};\bm{I}) 
c(\bm{x})\,d\bm{x}\,;\enskip
  \left\| \bm{I}\right\| =\int\limits_{\mathsf X} \pi(\bm{x};\bm{I}) 
p(\bm{x})\,d\bm{x}\,.
  $$
  
  Базис рынка составляют $\delta$-ин\-стру\-мен\-ты $\bm{D}(\bm{s})$, 
$\bm{s}\hm\in {\mathsf X}$, с~обобщенной $n$-мер\-ной $\delta$-функ\-ци\-ей 
относительно~$\bm{s}$ в~качестве платежной: 

\noindent
  \begin{multline}
    \bm{D}(\bm{s}) =\prod\limits_{l\in N} \bm{D}_l(s_l)\,,\\
  \pi(\bm{x};\bm{D}(\bm{s}))=\delta(\bm{x}-\bm{s})=\prod\limits_{l\in N} 
\delta(x_l-s_l).
  \label{1-aga}
  \end{multline}
  
  Инструмент $\bm{G}$ с~произвольной измеримой платежной 
функцией~$g(\bm{x})$ и~его стоимость имеют вид: 
  \begin{align*}
  \bm{G}&= \int\limits_{\mathsf X} g(\bm{s}) \bm{D}(\bm{s})\,d\bm{s}\,;\\
  \vert \bm{G}\vert &=\int\limits_{\mathsf X} g(\bm{s}) \vert \bm{D}(\bm{s}) \vert 
\,d\bm{s}= \int\limits_{\mathsf X} g(\bm{s}) c(\bm{s}) \,d\bm{s}\,.
  \end{align*}
  
  Наряду с~<<полноправными>> $n$-мер\-ны\-ми инструментами на рынке 
присутствуют и~их $k$-мер\-ные версии, у~которых $n\hm- k$ координатных 
базовых активов пред\-став\-ле\-ны в~форме одномерных единичных без\-рис\-ко\-вых 
инструментов. 
  
  Для индикаторов $\bm{H}\{M\}$, $M\hm\subset {\mathsf X}$, без\-рис\-ко\-во\-го 
актива $\bm{U}\hm=\bm{H}\{ {\mathsf X}\}$ и~их цен 
  \begin{gather*}
  \bm{H}\{ M\}=\int\limits_M \bm{D}(\bm{s})\, d\bm{s}\,;\enskip
   \vert \bm{H}\{M\}\vert =\int\limits_M c(\bm{s})\,d\bm{s}\,;\\
   \vert \bm{U}\vert ={\mathsf C}\{ {\mathsf X}\}= \int\limits_{\mathsf X} c(\bm{s}) 
\,d\bm{s}=\fr{1}{r}\,,
   \end{gather*}
где $r$~--- приравниваемый единице безрисковый доход за период. 
  
  На \textit{одномерном} рынке опционы пут~$\bm{P}_s$ и~колл~$\bm{C}_s$ 
со страйком~$s$ задаются своими платежными функциями: 
  \begin{equation}
  \left.
  \begin{array}{rl}
  \pi(x;\bm{P}_s)&=\max (0,s-x);\\[6pt]
  \pi(x;{\bm C}_x)&=\max (0,x-s),\ x,s \in {\mathsf X}\subset \mathfrak{R}\,.
  \end{array}
  \right\}
  \label{e2-aga}
  \end{equation}
  %
  Для них выполняется формула паритета ($\bm{X}$~--- вектор базовых 
активов)
  $$
  \bm{C}_s-\bm{P}_s= \bm{X}-s\bm{U}\,.
  $$
  
  Нормированными спрэдами быка с~парой страйков $s\hm-h$, $s \hm\in 
{\mathsf X}$ ($h \hm>0$)  и~медведя с~парой страйков~$s$, $s\hm+h \hm\in 
{\mathsf X}$ служат комбинации опционов соответственно 
  \begin{equation}
  \left.
  \begin{array}{rl}
 \!\!\!\! \bm{S}_{s;h}^{\mathrm{bull}} &= \fr{1}{h}\left( \bm{C}_{s-h}-\bm{C}_s\right) =
\bm{U}+\fr{1}{h}\left( \bm{P}_{s-h}-\bm{P}_s\right)\,;\\[6pt]
 \!\! \!\! \bm{S}_{s;h}^{\mathrm{bear}} &=\fr{1}{h}\left( \bm{P}_{s+h}-\bm{P}_s\right) 
=\bm{U}+\fr{1}{h} \left( \bm{C}_{s+h}-\bm{C}_s\right)
  \end{array}\!
  \right\}\!\!
  \label{e3-aga}
  \end{equation}
с платежными функциями 
\begin{equation}
\left.
\begin{array}{rl}
 \!\!\!\!\pi\left( x;\bm{S}_{s;h}^{\mathrm{bull}}\right) &= \min \left(\! 1,\fr{1}{h}\max (0,x-(s-h))\!\right);\\[9pt]
\! \!\!\!\pi\left( x;\bm{S}_{s;h}^{\mathrm{bear}}\right) &= \min \left(\! 1,\fr{1}{h}\max (0, (s+h)-x)\!\right).
\end{array}\!
\right\}\!
\label{e4-aga}
\end{equation}
  
  Нормированные симметричные баттерфляи с~тройкой страйков $s\hm-h$, $s$, 
$s\hm+h \hm\in {\mathsf X}$ образуются комбинациями 
  \begin{multline}
  \bm{B}_{s;h} = \fr{1}{h}\left( \bm{C}_{s-h} -2\bm{C}_s +\bm{C}_{s+h}\right) 
={}\\
  {}= \fr{1}{h} \left( \bm{P}_{s-h} -2\bm{P}_s+\bm{P}_{s+h}\right) 
={}\\
{}=\bm{U}+\fr{1}{h}\left( \bm{P}_{s-h} -\bm{P}_s -
\bm{C}_s+\bm{C}_{s+h}\right)
  \label{e5-aga}
  \end{multline}
с платежными функциями 
\begin{equation}
\pi \left( x;\bm{B}_{s;h}\right) =\fr{1}{h}\max ( 0, h-\vert x-s\vert).
\label{e6-aga}
\end{equation}
  
  В комбинациях~(\ref{e3-aga}) и~(\ref{e5-aga}) безрисковый 
инструмент~$\bm{U}$ выполняет функцию маржевого инструмента 
и~применяется инвестором в~соответствии с~требованиями рынка. Формально 
верно еще одно пред\-став\-ле\-ние:
  \begin{equation*}
  \bm{B}_{s;h}=\fr{1}{h}\left( \bm{C}_{s-h} -\bm{C}_s -\bm{P}_s 
+\bm{P}_{s+h}\right)\,,
 % \label{e7-aga}
  \end{equation*}
но оно не является \textit{естественным} (страйки коллов в~комбинации ниже 
страйков путов) и~потому далее не используется. 
  
  Можно было бы рассматривать и~не создающие принципиальных трудностей 
несимметричные баттерфляи (с~неравными по длине сценариями 
и~неравномерной линейкой страйков), но они, как правило, не применяются на 
рынках и~к~тому же сильно загромождали бы изложение. 
  
  С целью алгоритмической автоматизации дальнейших построений для 
одномерных опционов $\bm{P}_s$ и~$\bm{C}_s$ вводятся также обозначения 
$\bm{O}_s^-$ (и~$\bm{O}_{0;s}$) и~$\bm{O}_s^+$ (и~$\bm{O}_{1;s})$, которые 
могут обрастать дополнительными индексами координат $l\hm\in N$: 
  \begin{equation}
  \bm{O}_{0;s}\equiv \bm{O}_s^- \equiv \bm{P}_s\,;\enskip
  \bm{O}_{1;s} \equiv \bm{O}_s^+\equiv \bm{C}_s\,.
  \label{e8-aga}
  \end{equation}
  
  \textit{Многомерным} обобщением одномерных опционов $\bm{P}_s$ 
и~$\bm{C}_s$ служат $n$-мер\-ные $\alpha$-\textit{оп\-ци\-оны} 
$\bm{A}_{\bm{\alpha};\bm{s}}$ векторного типа~$\bm{\alpha}$ и~с~векторным 
страйком $\bm{s}\hm \in \mathfrak{R}^n$, задаваемые вместе с~платежными 
функциями соотношениями 
  \begin{multline}
  {A}_{\alpha;s}=\prod\limits_{i\in N} \bm{O}_{i\beta_i;s_i}\,,\\
  \pi\left(\bm{x}, A_{\alpha;s}\right)=
  \prod_{l\in N}\pi \left(x_l; \bm{O}_{l\beta_l;s_l}\right),\ \bm{x}\in 
\mathfrak{R}^n\,,\\[6pt]
  \pi\left( x_l; \bm{O}_{l\beta_l;s_l}\right) =\omega_{l\beta_l;s_l}(x_l)={}\\[6pt]
  \hspace*{10mm}{}=\max  \left(0,\alpha_l (x_l-s_l)\right), \enskip l\in N\,.
    \label{e9-aga}
  \end{multline}
  
  Как и~в~[6], вектор~$\bm{\alpha}$ с~компонентами $\alpha_l\hm= \pm1$, $l\hm\in N$, 
в~индексах инструментов (или просто~$\pm$) определяет векторный тип  
$\alpha$-оп\-ци\-онов~$\bm{A}_{\bm{\alpha};\bm{s}}$. Вектор $\bm{\beta}\hm= 
(\bm{\alpha}\hm+1)/2$, дублирующий~$\bm{\alpha}$, вводится для удобства по 
техническим причинам и~принимает для каждого $l\hm\in N$ два значения: 
$$
\beta_l= \begin{cases}
0 &\mbox{для\ пута;}\\ 
1 & \mbox{для\ колла.}
\end{cases}
$$ 
  
  Для каждого векторного страйка~$\bm{s}$ на $n$-мер\-ном рынке могут 
котироваться $2^n$ типов $\alpha$-оп\-ци\-онов. Рынок $n$-мер\-ных  
$\alpha$-оп\-ци\-онов с~их $k$-мер\-ны\-ми версиями, $k\hm< n$, называется  
$n$-мер\-ным $\alpha$-\textit{рын\-ком}. 
  
  \section{Двумерный дискретный $\alpha$-рынок}
   
  В основе дискретного $\alpha$-рын\-ка лежит \textit{сценарный} рынок~--- 
сценарная дискретизация двумерного тео\-ре\-ти\-че\-ско\-го $\delta$-рын\-ка. Как 
и~для бинарного рынка, используется в~большей мере адаптированная 
к~двумерному случаю очевидная сис\-те\-ма обозначений, но учитывается 
и~специфика требований $\alpha$-рынка. 
  
  Цены двух базовых активов \textit{теоретического} двумерного  
$\delta$-рын\-ка обозначаются~$x$ и~$y$, страйки опционов~--- 
соответственно~$s$ и~$t$, $x, s \hm\in {\mathsf X} \hm=[a_1,b_1)\hm\subset 
\mathfrak{R}$, $y,t \hm\in {\mathsf Y}\hm=[a_2,b_2) \hm\subset \mathfrak{R}$. 
Дискретизация осуществляется равномерным разбиением множества~${\mathsf 
X}$ на~$v_1$ интервалов (сценариев), ${\mathsf Y}$~--- на $v_2$ интервалов. 
Одномерные сценарии на~${\mathsf X}$ и~${\mathsf Y}$ даются формулами: 
  \begin{multline}
  S_i= \left[ x_{i-1},x_i\right),\ x_i=a_1+ih_1,\ h_1=\fr{b_1-a_1}{v_1},\\
   i\in  \bm{I},\ x_0=a_1; \label{e10-aga}
   \end{multline}
   
   \vspace*{-12pt}
   
   \noindent
   \begin{multline}
  T_j= \left [ y_{j-1},y_j\right),\ y_j= a_2+jh_2,\ h_2=\fr{b_2-a_2}{v_2},\\ j\in 
\bm{J},\ y_0=a_2\,,
  \label{e11-aga}
\end{multline}
где $\bm{I}=\{1,2, \ldots, v_1\}$, $\bm{J}\hm=\{1,2,\ldots , v_2\}$, а~номер 
сценария совпадает с~индексом его правой границы. Двумерными сценариями 
служат прямые произведения всех пар $S_i\times T_j$, $i\hm\in \bm{I}$, $j\hm\in 
\bm{J}$. 
  
  На сценарном рынке базис образуют индикаторы сценариев 
$\bm{D}_{ij}\hm=\bm{H}\{S_i\times T_j\}$, но для $\alpha$-рын\-ка при той же структуре 
сценариев уместнее использовать иной базис~--- из баттерфляев~$\bm{B}_{ij}$, 
задаваемых с~учетом определения~(\ref{e5-aga}), но для специально 
подобранных страйков. Страйками~$s_i$ и~$t_j$ одномерных опционов 
и~упомянутых баттерфляев служат середины сценариев~(\ref{e10-aga}) 
и~(\ref{e11-aga}): 
  $$
  s_i = \fr{x_{i-1} + x_i}{2}\,,\  i\in \bm{I}; \enskip   t_j = \fr{y_{j-1} + y_j}{2},\   j\in 
\bm{J}\,. 
  $$
  %
  При этом параметр~$h$ для баттерфляев~(\ref{e5-aga}), равный длине 
сценариев, определяется в~(\ref{e10-aga}) и~\ref{e11-aga}). Для удобства 
записи формул также доопределяются параметры $s_0 \hm= a_1$, $s_{v_1+1}\hm = 
b_1$, $t_0 \hm= a_2$, $t_{v_2+1}\hm = b_2$, но они страйками не являются. 
  
  Портфель с~вектором~$\bm{g}$ весов базисных баттерфляев в~двумерном 
случае приобретает вид: 
  \begin{equation}
  \bm{G}= \sum\limits_{i\in \bm{I}, j\in \bm{J}} g_{ij} \bm{B}_{ij}\,.
  \label{e12-aga}
  \end{equation}
  
  Двумерным обобщением обычных опционов служат инструменты, 
характеризуемые парой страйков $(s_i,t_j)$, или просто $(i,j)$, 
с~дополнительным указанием типа (лучше в~терминах $\bm{\beta}\hm=(\beta_1, 
\beta_2))$: 
\begin{multline*}
\bm{A}_{\beta_1\beta_2;ij} 
=\bm{O}_{\beta_1;1,i}\bm{O}_{\beta_2;2,j}=\bm{O}_{\beta_1;i\cdot} 
\bm{O}_{\beta_2;\cdot j}\,,\\ 
i\in \bm{I}\,,\ j\in \bm{J}\,.
\end{multline*}
  %
  Также рассматриваются и~их одномерные версии, обозначаемые~$\bm{A}_{i\cdot}$ 
и~$\bm{A}_{\cdot j}$ с~маркером <<точка>> в~позиции, отведенной координате 
безрискового актива. 
  
  Для представления произвольного инструмента~$\bm{G}$~(\ref{e12-aga}) 
в~базисе из $\alpha$-оп\-ци\-онов (избыточным в~сравнении с~базисом из 
баттерфляев) все нормированные баттерфляи в~(\ref{e12-aga}) следует 
реплицировать в~терминах $\alpha$-оп\-ци\-онов. 
  
  В соответствии с~(\ref{e9-aga}) для конструирования репликаций следует 
перемножать одномерные представления сценарных баттерфляев, выбирая 
подходящие сомножители из~(\ref{e3-aga}) и~(\ref{e5-aga}). Для одномерного 
рынка с~$v$~сценариями $i\hm\in \bm{I}$ справедливы такие репликации 
баттерфляев коллами и~путами:
  \begin{multline}
  \bm{B}_i={}\\
  \!\!\!\!{}=\begin{cases}
  \bm{U}-\fr{\bm{O}_1^+ -\bm{O}_2^+}{h}=\fr{\bm{O}_2^- -\bm{O}_1^-}{h}\,, & i=1;\\
  \fr{\bm{O}_{i-1}^+-2\bm{O}_i^+ +\bm{O}_{i+1}^+}{h}={}&\\
  \hspace*{7mm}{}= \fr{\bm{O}^-_{i-1}-2\bm{O}_i^- +\bm{O}^-_{i+1}}{h}={}&\\
  \hspace*{12mm}{}= \bm{U}-\fr{\bm{O}_i^- -\bm{O}^-_{i-1}}{h} -{}&\\
  \hspace*{15mm}{}- \fr{\bm{O}_i^+  - \bm{O}^+_{i+1}}{h}\,, &\hspace*{-7mm} 1<i<v\,;\\
  \fr{\bm{O}^+_{v-1} -\bm{O}_v^+}{h}=\bm{U}-\fr{\bm{O}_v^- -\bm{O}^-_{v-
1}}{h}\,, & i=v\,.
  \end{cases}\!
  \label{e13-aga}
  \end{multline}
  
  Базисные инструменты для $i\hm=\overline{1,v}$ являются спрэдами, но их для 
удобства также называем баттерфляями (\textit{усеченными}). 
Инструмент~$\bm{U}$ в~выписанных соотношениях, как в~(\ref{e3-aga}) 
и~(\ref{e5-aga}),  выполняет функцию маржевого инструмента. 
  
  Подобно сценарным базисам для $\zeta$-рынка~\cite{6-aga} построение 
двумерных базисов из $\alpha$-оп\-ци\-онов проводится на основе одномерных 
базисов, но их элементы на этот раз выбираются из~(\ref{e13-aga}). Строятся 
три варианта репликации базисов: два однотипных (один в~путах, другой 
в~коллах) и~третий~--- смешанный естественный. Если в~базисе $v$~сценариев, 
а~центральный страйк~$i_c$, то 
  \begin{itemize}
\item однотипный базис при $\alpha\hm=-1$ (в~путах):
\begin{equation}
\left.
\begin{array}{rl}
\bm{B}_1^- &=\fr{\bm{O}_2^*-\bm{O}_1^-}{h}\,;\\[6pt]
  \bm{B}_i^- &= \fr{\bm{O}^-_{i-1} -2\bm{O}_i^- +\bm{O}^-_{i+1}}{h}\,;\\[6pt]
   \bm{B}^-_v &=\bm{U}- \fr{\bm{O}_v^- -\bm{O}^-_{v-1}}{h}\,;
   \end{array}
   \right\}
\label{e14-aga}
\end{equation}
\item однотипный базис при $\alpha\hm=+1$ (в~коллах):
\begin{equation}
\left.
\begin{array}{rl}
\bm{B}_1^+ &\equiv \bm{U} - \fr{\bm{O}_1^+ -\bm{O}_2^+}{h}\,;\\[6pt]
  \bm{B}_i^+ &= \fr{\bm{O}^+_{i-1} -2\bm{O}_i^+ +\bm{O}^+_{i+1}}{h}\,;\\[6pt]
    \bm{B}_v^+ &\equiv \fr{\bm{O}^+_{v-1} -\bm{O}_v^+}{h}\,;
    \end{array}
    \right\}
\label{e15-aga}
\end{equation}
\item смешанный естественный базис:
\begin{equation}
\left.
\begin{array}{l}
\!\!\!\bm{B}_1^m\equiv \fr{\bm{O}_2^- -  \bm{O}_1^-}{h}\,;\\[6pt]
  \!\!\!\bm{B}_i^m \equiv \fr{\bm{O}^-_{i-1} -2\bm{O}_i^- +\bm{O}^-_{i+1}}{h}\,,\ 0< i< i_c;\\[6pt]
\!\!\!\bm{B}^m_{i_c} \equiv  \bm{U}-\fr{\bm{O}^-_{i_c-1} -\bm{O}^-_{i_c} -
\bm{O}^+_{i_c} +\bm{O}^+_{i_c+1}}{h}\,;\\[6pt]
\!\!\!\bm{B}_i^m\equiv \fr{\bm{O}^+_{i-1} -2\bm{O}_i^++\bm{O}^+_{i+1}}{h_i}\,,\ i_c<i<v\,;\\[6pt]
 \!\!\!\bm{B}_v^m\equiv \fr{\bm{O}^+_{v-1} -\bm{O}_v^+}{h}\,.
\end{array}\!
\right\}\!
\label{e16-aga}
\end{equation}
  \end{itemize}
  
  \section{Формирование базисов и~платежных функций 
портфелей $\alpha$-опционов}
  
  На основе соотношений~(\ref{e14-aga})--(\ref{e16-aga}) введенные 
  в~многомерном случае произвольной размерности конструкции здесь 
переписываются для двумерного  
$\alpha$-рын\-ка в~однотипных и~смешанных вариантах. 
  
  Поскольку каждый двумерный базисный баттерфляй определяется как 
произведение двух одномерных (что соответствует перемножению платежных 
функций), его репликации двумерными\linebreak $\alpha$-оп\-ци\-она\-ми находятся 
перемножением пары подходящих представлений  
из~(\ref{e14-aga})--(\ref{e16-aga}). 
  
  \textit{Однотипная} репликация сценарных баттерфляев проводится 
  $\alpha$-оп\-ци\-она\-ми единого типа $\bm{\alpha}\hm=\{\alpha_1, \alpha_2\}$. Он фиксируется 
заранее, и~потому используются более простые соотношения~(\ref{e14-aga}) 
и~(\ref{e15-aga}), а~обозначение типа опциона опускается. 
  
  Каждое перемножение сумм одномерных опционов в~(\ref{e14-aga}) 
или~(\ref{e15-aga}) дает сумму парных произведений этих опционов, которые 
затем следует замещать согласно~(\ref{e9-aga}) эквивалентными двумерными 
$\alpha$-оп\-ци\-она\-ми по правилам 
  \begin{multline}
  1\to \bm{U}\,,\enskip \bm{O}_{1,i}\bm{O}_{2,j}\to \bm{A}_{ij}\,,\\ 
\bm{O}_{1,i}\bm{U}_2\to \bm{A}_{i\cdot}\,,\enskip \bm{U}_1 \bm{O}_{2,j}\to 
\bm{A}_{\cdot j}\,.
  \label{e17-aga}
  \end{multline}
  
  \textit{Смешанная} репликация осуществляется аналогично, но указание типа в~обозначениях необходимо, и~потому правила трансформации приобретают 
вид: 
  \begin{multline}
  1\to \bm{U}\,,\ \bm{O}_{1,i}^{\alpha_1} \bm{O}_{2,j}^{\alpha_2} \to 
\bm{A}_{ij}^{\bm {\alpha}}=\bm{A}_{\beta_1,\beta_2;ij},\\
\bm{O}_{1,i}^{\alpha_1}\to \bm{A}_{\beta_1;i,\cdot} \left( = 
\bm{A}^{\alpha_1}_{i\cdot}\right)\,,\ \bm{O}^{\alpha_2}_{2,j} \to 
\bm{A}_{\beta_2;\cdot,j}\left( =\bm{A}^{\alpha_2}_{\cdot j}\right),\\
 \beta_1,\beta_2\in  \{0,1\}\,,
\label{e18-aga}
\end{multline}
  
  На двумерном $\alpha$-рын\-ке в~соответствии с~чис\-лом возможных 
векторов~$\bm{\alpha}$ насчитываются четыре варианта однотипных базисов 
и~один смешанный (естественный с~заданным центром рынка). 
  
  Для каждого варианта с~\textit{однотипным} базисом и~оптимальным 
портфелем фиксируется тип~$\bm{\alpha}$, и~он становится типом всех  
$\alpha$-оп\-ци\-онов варианта. В~двумерном случае таких типов четыре: $\{-1, -
1\}$; $\{-1,+1\}$; $\{+1,-1\}$; $\{+1,+1\}$. Последовательным применением 
правил~(\ref{e17-aga}) ко всем страйкам для каж\-до\-го значения векторного 
параметра~$\bm{\alpha}$ находятся искомые четыре базиса. В~однотипном случае 
для каждой компоненты рынка наличествуют три качественно различных 
представления по варианту страйка~--- двум крайним и~общему внутреннему, 
и~потому их $3^2\hm=9$. В~качестве примера приводится базис для 
$\bm{\alpha}\hm=\{-1,+1\}$, т.\,е.\ в~терминах $\alpha$-оп\-ци\-онов~$\bm{A}_{01}$ 
(остальные три \textit{однотипных} базиса выписываются сходным образом), 
при этом в~списке принимается $0\hm<i \hm< v_1$, $0 \hm< j \hm< v_2$: 
  \begin{align*}
  \bm{B}_{1,1}&=\fr{\bm{A}_{1,1}- \bm{A}_{1,2} -
\bm{A}_{2,1}+\bm{A}_{2,2}}{h_1h_2}+{}\\
&\hspace*{35mm}{}+ \fr{-\bm{A}_{1,\cdot} +\bm{A}_{2,\cdot}}{h_1}\,;\\
  \bm{B}_{1,j} &= \fr{-\bm{A}_{1,j-1} +2\bm{A}_{1,j} -\bm{A}_{1,j+1}}{h_1h_2}+{}\\
  &  \hspace*{15mm} {}+
\fr{\bm{A}_{2,j-1} -2\bm{A}_{2,j} +\bm{A}_{2,j+1}}{h_1h_2}\,;\\
   \bm{B}_{1,v_2}&= \fr{-\bm{A}_{1,v_2-1} +\bm{A}_{1,v_2} +\bm{A}_{2,v_2-1}- \bm{A}_{2,v_2}}{h_1h_2}\,;
  \end{align*}
  
\noindent
  \begin{align*}
   \bm{B}_{i,1}&= \fr{ -\bm{A}_{i-1,1} +\bm{A}_{i-1,2} +2\bm{A}_{i,1}}{h_1h_2}+{}\\
  &\hspace*{15mm}{}+   \fr{ -2\bm{A}_{i,2} -\bm{A}_{i+1,1} +\bm{A}_{i+1,2}}{h_1h_2}+{}\\
  &\hspace*{25mm}{}+ \fr{ \bm{A}_{i-1,\cdot } - 2\bm{A}_{i,\cdot} +\bm{A}_{i+1,\cdot}}{h_1}\,;\\
  \bm{B}_{i,j} &=\fr{\bm{A}_{i-1,j-1} -2\bm{A}_{i-1,j} +\bm{A}_{i-1,j+1}}{h_1h_2}+{}\\
  &\hspace*{11mm}{}+ \fr{-2\bm{A}_{i,j-1} +4\bm{A}_{i,j} -2\bm{A}_{i,j+1}}{h_1h_2} +{}\\
 &\hspace*{13mm}{}+  \fr{\bm{A}_{i+1,j-1} -2\bm{A}_{i+1,j}+\bm{A}_{i+1,j+1}}{h_1h_2}\,;\\
  \bm{B}_{i,v_2}&= \fr{\bm{A}_{i-1,v_2-1} -\bm{A}_{i-1,v_2} -2\bm{A}_{i,v_2-1}}{h_1h_2} +{}\\
 & \hspace*{16mm}{}+\fr{2\bm{A}_{i,v_2} +\bm{A}_{i+1,v_2-1} -\bm{A}_{i+1,v_2}}{h_1h_2}\,;\\
  \bm{B}_{v_1,1} &=\bm{U} +\fr{ -\bm{A}_{\cdot,1} +\bm{A}_{\cdot, 2}}{h_2}+ {}\\
 & \hspace*{2mm}{}+\fr{-\bm{A}_{v_1-1,1} +\bm{A}_{v_1-1,2} +\bm{A}_{v_1,1} -\bm{A}_{v_1,2}}{h_1h_2} +{}\\
&\hspace*{37mm}{}+\fr{\bm{A}_{v_1-1,\cdot}- \bm{A}_{v_1,\cdot}}{h_1}\,;\\
  \bm{B}_{v_1,j}&= \fr{\bm{A}_{\cdot,j-1}- 2\bm{A}_{\cdot,j} +\bm{A}_{\cdot,j+1}}{h_2} +{}\\
&\hspace*{2mm}{}+\fr{\bm{A}_{v_1-1,j-1}-2\bm{A}_{v_1-1,j}+\bm{A}_{v_1-1,j+1}}{h_1h_2}+{}\\
&\hspace*{15mm}{}+\fr{ -\bm{A}_{v_1,j-1}+2\bm{A}_{v_1,j}-\bm{A}_{v_1,j+1}}{h_1h_1}\,;\\
  \bm{B}_{v_1,v_2}&= \fr{\bm{A}_{\cdot,v_2-1}- \bm{A}_{\cdot,v_2}}{h_2} 
+{}\\
&\hspace*{-7mm}{}+\fr{\bm{A}_{v_1-1, v_2-1}- \bm{A}_{v_1-1,v_2} - \bm{A}_{v_1,v_2-1} 
+\bm{A}_{v_1,v_2}}{h_1h_2}\,.
  \end{align*}
    Здесь в~индексах опционов маркер <<точка>> отмечает координату 
безрискового актива, а~под~$\bm{A}_{i,\cdot}$ и~$\bm{A}_{\cdot,j}$, как уже обсуждалось 
выше, понимаются двумерные инструменты $\bm{A}_i\times \bm{U}_2$ 
и~$\bm{U}_1\times \bm{A}_j$ соответственно. 
  
  \textit{Смешанный} базис состоит из $5^2\hm=25$ качественно различных 
вариантов представления базисных инструментов, поскольку для каждой 
компоненты рынка вариантов страйка пять: два крайних, один центральный 
и~два внутренних, ниже и~выше центра. Их перечень получается применением 
правил~(\ref{e16-aga}). Приводим лишь часть базиса, связанную с~первым по 
отношению к~центру рынка квадрантом, т.\,е.\ для $1 \hm\leq i \hm\leq i_c$, $1 
\hm\leq j \hm\leq j_c$ (прочие части образуются аналогично):
  \begin{align*}
  \bm{B}_{1,1}&=\fr{\bm{A}_{00;1,1} - \bm{A}_{00;1,2} - \bm{A}_{00;2,1} + 
\bm{A}_{00;2,2}}{h_1h_2}\,; 
\end{align*}

  \noindent
  \begin{align*}
  \bm{B}_{1,j}&=\fr{-\bm{A}_{00;1,j-1} + 2\bm{A}_{00;1,j} - \bm{A}_{00;1,j+1}}{h_1h_2} + {}\\
&\hspace*{-3mm}{}+
\fr{\bm{A}_{00;2,j-1} - 2\bm{A}_{00;2,j} + \bm{A}_{00;2,j+1}}{h_1h_2}\,,\enskip 0<j<j_c\,; \\
  \bm{B}_{1,j_c}&=\fr{- \bm{A}_{00;1,j_c-1} + \bm{A}_{00;1,j_c} + \bm{A}_{00;2,j_c - 1}}{h_1h_2} +{}\\
  &\hspace*{-3mm}{}+  
  \fr{- \bm{A}_{00;2,j_c} + \bm{A}_{01;1,j_c} - \bm{A}_{01;1,j_c+1} - \bm{A}_{01;2,j_c} }{h_1h_2}+{}\\
&\hspace*{21mm}{}+ \fr{\bm{A}_{01;2,j_c+1}}{h_1h_2} + \fr{- \bm{A}_{0;1,\cdot} + \bm{A}_{0;2,\cdot}}{h_1}\,; \\
  \bm{B}_{i,1}&=\fr{-\bm{A}_{00;i-1,1} + \bm{A}_{00;i-1,2} + 2\bm{A}_{00;i,1}}{h_1h_2}+{}\\
  &\hspace*{-4mm}{}+ \fr{ -  2\bm{A}_{00;i,2} - \bm{A}_{00;i+1,1} + \bm{A}_{00;i+1,2}}{h_1h_2}\,,\enskip   0<i<i_c\,; \\
  \bm{B}_{i,j}&= \fr{\bm{A}_{00;i-1,j-1} - 2\bm{A}_{00;i-1,j} + \bm{A}_{00;i-1,j+1}}{h_1h_2}+{}\\
  &\hspace*{2mm}{}+ \fr{- 2\bm{A}_{00;i,j-1} + 4\bm{A}_{00;i,j} - 2\bm{A}_{00;i,j+1}}{h_1h_2} +{}\\
&\hspace*{3mm}{}+\fr{\bm{A}_{00;i+1,j-1} - 2\bm{A}_{00;i+1,j} + \bm{A}_{00;i+1,j+1}}{h_1h_2}\,,\\
    & \hspace*{35mm}0<i<i_c,\enskip  0<j<j_c\,; \\
  \bm{B}_{i,j_c}&=\fr{\bm{A}_{00;i-1,j_c-1} - \bm{A}_{00;i-1,j_c} - 2\bm{A}_{00;i,j_c-1}}{h_1h_2} +{} \\
 &\hspace*{1mm}{}+\fr{2\bm{A}_{00;i,j_c} + \bm{A}_{00;i+1,j_c-1} - \bm{A}_{00;i+1,j_c}}{h_1h_2}+{}\\
 &\hspace*{2mm}{}+ \fr{ -\bm{A}_{01;i-1,j_c} + \bm{A}_{01;i-1,j_c+1} + 2\bm{A}_{01;i,j_c}}{h_1h_2}+{}\\
& {}+\fr{ - 2\bm{A}_{01;i,j_c+1} - \bm{A}_{01;i+1,j_c} + \bm{A}_{01;i+1,j_c+1}}{h_1h_2} +{}\\
&\hspace*{4mm}{}+ \fr{\bm{A}_{0;i-1,\cdot} - 2\bm{A}_{0;i,\cdot} + \bm{A}_{0;i+1,\cdot}}{h_1},\enskip    0<i<i_c\,; \\
  \bm{B}_{i_c,1}&=\fr{-\bm{A}_{00;i_c-1,1} + \bm{A}_{00;i_c-1,2} + \bm{A}_{00;i_c,1}}{h_1h_2}+{}\\
  &\hspace*{5mm}{}+ \fr{ -\bm{A}_{00; i_c,2} + \bm{A}_{10;i_c,1} - \bm{A}_{10;i_c,2}}{h_1 h_2}+{}\\
  &\hspace*{-2mm}{}+\fr{ - \bm{A}_{10;i_c+1,1} +\bm{A}_{10;i_c+1,2}}{h_1h_2} +   \fr{- \bm{A}_{0;\cdot,1} + \bm{A}_{0;\cdot,2}}{h_2}\,; \\
   \bm{B}_{i_c,j}&=\fr{\bm{A}_{00;i_c-1,j-1} - 2\bm{A}_{00;i_c-1,j} + \bm{A}_{00;i_c-1,j+1}}{h_1h_2} + {}\\
   &\hspace*{2mm}{}+   \fr{-\bm{A}_{00;i_c,j-1} + 2\bm{A}_{00;i_c,j} - \bm{A}_{00;i_c,j+1}}{h_1h_2}+{}\\
   &\hspace*{3mm}{}+ \fr{ - \bm{A}_{10;i_c,j-1} + 2\bm{A}_{10;i_c,j} - \bm{A}_{10;i_c,j+1}}{h_1h_2} +{}\\
  &\hspace*{-1mm} {}+\fr{ \bm{A}_{10;i_c+1,j-1} - 2\bm{A}_{10;i_c+1,j} + \bm{A}_{10;i_c+1,j+1}}{h_1h_2} +{}\\
&\hspace*{1mm}{}+ \fr{\bm{A}_{0;\cdot,j-1} - 2\bm{A}_{0;\cdot,j} + \bm{A}_{0;\cdot,j+1}}{h_2}\,,\enskip    0<j<j_c\,; 
 \end{align*}
  
 \noindent
  \begin{align*}
  \bm{B}_{i_c,j_c}& =1 + 
  \fr{\bm{A}_{00;i_c-1,j_c-1} - \bm{A}_{00;i_c-1,j_c}}{h_1h_2}+{}\\[2pt]
  &{}+\fr{ - \bm{A}_{00;i_c,j_c-1} + 
\bm{A}_{00;i_c,j_c} - \bm{A}_{01;i_c-1,j_c}}{h_1h_2} +{}\\[2pt]
&{}+ \fr{\bm{A}_{01;i_c-1,j_c+1} + \bm{A}_{01;i_c,j_c} - \bm{A}_{01;i_c,j_c+1}}{h_1h_2}+{}\\[2pt]
&{}+ \fr{ - \bm{A}_{10;i_c,j_c-1} + \bm{A}_{10;i_c,j_c} + \bm{A}_{10;i_c+1,j_c-1}}{h_1h_2}+{}\\[2pt]
&{}+ \fr{ -\bm{A}_{10;i_c+1,j_c} + \bm{A}_{11;i_c,j_c} - \bm{A}_{11;i_c,j_c+1}}{h_1h_2}+{}\\[2pt]
&{}+\fr{ - \bm{A}_{11;i_c+1,j_c} + \bm{A}_{11;i_c+1,j_c+1}}{h_1h_2} + {}\\[2pt]
&{}+\fr{\bm{A}_{0;i_c-1,\cdot} - \bm{A}_{0;i_c, \cdot} - \bm{A}_{1;i_c,\cdot} + \bm{A}_{1;i_c+1,\cdot}}{h_1} + {}\\[2pt]
&\hspace*{3mm}{}+\fr{\bm{A}_{0;\cdot,j_c-1} - \bm{A}_{0;\cdot,j_c} - \bm{A}_{1;\cdot,j_c} + \bm{A}_{1;\cdot,j_c+1}}{h_2}\,. 
  \end{align*}
  %
  В этом списке присутствуют обозначения инструментов~$\bm{A}$ 
с~четырьмя и~тремя индексами. В~первой группе пара индексов до точки 
с~запятой означает тип двумерного $\alpha$-оп\-ци\-она~(\ref{e18-aga}), а~после 
нее~--- его страйк. Во второй группе представлены одномерные версии 
двумерных $\alpha$-оп\-ци\-онов. Индекс до точки с~запятой означает тип опциона, 
числовой индекс после нее~--- его страйк, а~позиция маркера <<точка>> 
показывает координату безрискового актива. 
  
  Сценарные баттерфляи, полученные из $\alpha$-оп\-ци\-онов, позволяют 
произвольный инструмент на рынке представить в~виде портфеля  
$\alpha$-оп\-ци\-онов. Для нахождения его доходов следует воспользоваться 
соотношениями~(\ref{e2-aga}) с~учетом переопределения~(\ref{e8-aga}). Так, 
в~однотипном случае платежная функция портфеля находится в~соответствии 
с~(\ref{e17-aga}) по правилам: 
  \begin{multline}
  \bm{U}\to 1\,,\enskip \bm{A}_{ij}\to \omega_{1;i}(x)\omega_{2;j}(y)\,,\\[2pt] 
\bm{A}_{i\cdot}\to \omega_{1;i}(x)\,,\enskip 
 \bm{A}_{\cdot j}\to \omega_{2;j}(y)\,;
  \label{e19-aga}
  \end{multline}
  
  \vspace*{-12pt}
  
  \noindent
  \begin{multline*}
  \omega_{\beta_1;i\cdot}(x)=\max \left( 0,\alpha_1(x-s_i)\right)\,,\
  i\in \bm{I};\\[2pt]
   \omega_{2;i}(y)=\max \left( 0,\alpha_2(y-t_j)\right), \enskip j\in \bm{J}\,.
 % \label{e20-aga}
  \end{multline*}
  
  Аналогично в~соответствии с~(\ref{e18-aga}) записываются в~смешанном 
случае правила формирования платежных функций: 
  \begin{multline*}
  \bm{U}\to 1\,,\enskip \bm{A}_{\beta;ij}\to \omega_{\beta_1,1;i}(x)\omega_{\beta_2,2;j}(y),\\[2pt]
  \bm{A}_{\beta_1;i,\cdot}\to \omega_{\beta_1,1;i}(x),\enskip
  \bm{A}_{\beta_2;\cdot,j}\to \omega_{\beta_2,2;j}(y)\,;
  \end{multline*}
  
  \vspace*{-12pt}
  
  \noindent
  \begin{multline}
  \omega_{\beta_1,1;i}(x)=\max \left( 0,\alpha_1(x-s_i)\right),\enskip i\in \bm{I},\\[2pt]
  \omega_{\beta_2,2;j}(y)=\max \left( 0,\alpha_2(y-t_j)\right),\enskip j\in\bm{J}.
  \label{e21-aga}
  \end{multline}
  
  \section{Иллюстративный пример}
  
  \vspace*{-1pt}
  
  Для построения мер ${\mathsf C}\{\cdot\}$ и~${\mathsf P}\{\cdot\}$ и~их 
сравнительного анализа данные в~примере заимствуются из~[6]. Так, 
принимается ${\mathsf X}\hm=[0,1)$, ${\mathsf Y}\hm=[0,1)$, а~для ${\mathsf 
F}_{\mathsf {CX}}(x)$ и~${\mathsf F}_{\mathsf{CY}}(y)$ выбираются  
бе\-та-рас\-пре\-де\-ле\-ния с~па\-ра\-мет\-ра\-ми $\{3/2,2\}$ и~$\{3/2,3\}$ 
соответственно, для ${\mathsf F}_{\mathsf PX}(x)$ и~${\mathsf F}_{\mathsf PY}(y)$~--- $\{2,3\}$ и~$\{2,4\}$:

\vspace*{-2pt}

\noindent
  \begin{align*}
  {\mathsf F}_{\mathsf {CX}} (x) &= \fr{x^{3/2}(5-3x)}{2}\,;\\  
  {\mathsf F}_{\mathsf {CY}}(y)&= \fr{y^{3/2}(35-42y+15y^2)}{8}\,;\\
  {\mathsf F}_{\mathsf {PX}}(x) &= x^2\left(6-8x+3x^2\right);\\
  {\mathsf F}_{\mathsf {PY}}(y)&= y^2\left( 10-20y+15y^2-4y^3\right).
  \end{align*}
  %
  
  \vspace*{-2pt}
  
  \noindent
  Из них совместные функции распределения для обеих мер строятся как
  
  \vspace*{-5pt}
  
  \noindent 
  \begin{multline}
  {\mathsf F}(x,y)={}\\
\hspace*{-3mm}  {}= {\mathsf F}_{\mathsf X}(x) {\mathsf F}_{\mathsf Y}(y) \left( 
1+3\kappa \left( 1-{\mathsf F}_{\mathsf X}(x)\right) \left(1-{\mathsf F}_{\mathsf 
Y}(y)\right)\right).
  \label{e22-aga}
  \end{multline}
  
\vspace*{-1pt}
 
  Искомые двумерные функции распределения ${\mathsf F}_{\mathsf C}(x,y)$ 
и~${\mathsf F}_{\mathsf P} (x,y)$ определяются подстановкой в~(\ref{e22-aga}) 
в~качестве параметра, отвечающего за корреляционную связь компонент, 
соответственно $\kappa_c\hm=0$ и~$\kappa_p\hm=0{,}2$. Из них простым 
смешанным дифференцированием по обеим переменным находятся плотности 
$c(x,y)$ и~$p(x,y)$, но ввиду громоздкости записей они здесь не приводятся. 
  
  Двумерная дискретизация множества ${\mathsf X}\times \mathsf{Y}$ 
в~примере проводится также при $v_1\hm=6$ и~$v_2\hm=5$, а~центральным 
выбирается страйк $i_c\hm=3$, $j_c\hm=3$. 
  
  В отличие от $\zeta$-рын\-ков~[6], для которых в~целях применения 
дискретного алгоритма находились стоимости сценарных индикаторов, для  
\mbox{$\alpha$-рын}\-ков следует вычислять стоимости сценарных баттерфляев. И~потому 
для адекватного сравнения относительных доходов естественно вычислять и~их 
средние доходы. И~те и~другие, а~это векторы $\bm{c}^B$ и~$\bm{p}^B$, 
определяются интегрированием платежных функций~(\ref{e4-aga})  
и~(\ref{e6-aga}) с~плотностями $c(x,y)$ и~$p(x,y)$ соответственно. 
  
  Применением к~этим векторам дискретного алгоритма оптимизации~[6], 
основанного на процедуре Ней\-ма\-на--Пир\-со\-на~\cite{7-aga}, определяется 
вектор весов базисных баттерфляев для оптимального двумерного портфеля. 
При этом в~качестве функции рисковых предпочтений выбирается 
$\varphi(\varepsilon)\hm=\varepsilon^2$, $\varepsilon\hm\in [0,1]$.

\begin{figure*} %fig1
\vspace*{1pt}
\begin{minipage}[t]{80mm}
\begin{center}
   \mbox{%
\epsfxsize=77.328mm
\epsfbox{aga-1.eps}
}
\end{center}
\vspace*{-9pt}
\Caption{Доходы оптимального опционного портфеля при дискретизации $6\times5$}
\end{minipage}
%\end{figure*}
\hfill
%\begin{figure*} %fig2
\vspace*{1pt}
\begin{minipage}[t]{80mm}
\begin{center}
   \mbox{%
\epsfxsize=77.328mm
\epsfbox{aga-2.eps}
}
\end{center}
\vspace*{-9pt}
\Caption{Доходы сценарного портфеля при дискретизации $40\times40$}
\end{minipage}
\end{figure*}

  
  В результате  получается вектор весов 
  
  \vspace*{-4pt}
  
  \noindent
 \begin{multline*}
  \bm{g}=\{0{,}118; 0{,}159; 0{,}0113; 0{,}000219; 0{,}000008; 0{,}414;\\
   1{,}0; 0{,}228; 0{,}0151; 0{,}000989; 0{,}0739; 0{,}788; 0{,}602;\\
      0{,}0873; 0{,}00175; 0{,}0069; 0{,}309; 0{,}495; 0{,}176; 0{,}00191;
\end{multline*}

%\vspace*{-3pt}

%\pagebreak

\noindent
 \begin{multline*}
   \hspace*{-4.8848pt}0{,}000907; 0{,}0254; 0{,}0405; 0{,}0291; 0{,}00159; 0{,}0000058;\\
   0{,}0000848; 0{,}00151; 0{,}00112;  0{,}000009\}. 
  \end{multline*}
  
  \vspace*{-2pt}

  
  Он порождает оптимальный портфель~(\ref{e12-aga}) с~инвестиционной 
суммой, средним доходом и~средней доходностью соответственно 
  $A\hm=0{,}28642$,  $R\hm=0{,}364418$  и~$y\hm=0{,}272317$. 
      График его платежной функции изображен на рис.~1. Для сравнения на 
рис.~2 приведен аналогичный график для сценарного рынка при дискретизации 
$40\times40$.
  

  По понятным причинам графики платежных функций на рис.~1 и~2 
демонстрируют большее взаимное сходство, чем аналогичная пара графиков 
из~[6]. Но и~различие между собой графиков на рис.~1 и~2 пред\-став\-ля\-ет\-ся 
естественным. 
  
  Остается определить оптимальные портфели в~терминах $\alpha$-оп\-ци\-онов 
рас\-смат\-ри\-ва\-емых типов. Для нахождения каждого такого пред\-став\-ле\-ния 
в~формулу~(\ref{e12-aga}) следует для всех пар $(i,j)$ под\-став\-лять вмес\-то 
индикаторов~$\bm{B}_{ij}$ со\-от\-вет\-ст\-ву\-ющие им пред\-став\-ле\-ния  
в~$\alpha$-оп\-ци\-онах. В~результате после упрощений получаются четыре 
однотипных портфеля и~один смешанный. Так, оптимальный 
\textit{однотипный} портфель для $\bm{\alpha}\hm= \{-1, +1\}$, т.\,е.\ образованный 
путами для первого актива и~коллами~--- для второго: 

\vspace*{-2pt}

\noindent
 \begin{multline*}
  \bm{G}_{01}=0{,}000006 \bm{U} + 16{,}369\bm{A}_{11} - 35{,}107\bm{A}_{12} + {}\\
  {}+12{,}681\bm{A}_{13} + 5{,}641\bm{A}_{14} + 0{,}416\bm{A}_{15} + 1{,}774\bm{A}_{1\cdot} - {}\\
{}-12{,}549\bm{A}_{21} + 48{,}875\bm{A}_{22} - 39{,}318\bm{A}_{23} + 1{,}263\bm{A}_{24} + {}\\
{}+ 1{,}729\bm{A}_{25} - 3{,}812\bm{A}_{2\cdot} - 16{,}162\bm{A}_{31} + 9{,}717\bm{A}_{32} + {}\\
{}+21{,}364\bm{A}_{33} - 15{,}431\bm{A}_{34} + 0{,}512\bm{A}_{35} + 1{,}636\bm{A}_{3\cdot} + {}\\
{}+ 4{,}007\bm{A}_{41} - 20{,}268\bm{A}_{42} + 19{,}612\bm{A}_{43} + 3{,}707\bm{A}_{44} - {}\\
{}-7{,}058\bm{A}_{45} + 0{,}366\bm{A}_{4\cdot} + 7{,}603\bm{A}_{51} - 2{,}899\bm{A}_{52} - {}\\
{}- 13{,}593\bm{A}_{53} + 5{,}279\bm{A}_{54} + 3{,}609\bm{A}_{55} + 0,031\bm{A}_{5\cdot} + {}\\
{}+ 0{,}731\bm{A}_{61} - 0{,}318\bm{A}_{62} - 0{,}745\bm{A}_{63} - 0{,}458\bm{A}_{64} + {}
\end{multline*}

\noindent
\begin{multline*}
{}+0{,}791\bm{A}_{65} + 0{,}005\bm{A}_{6\cdot} + 0{,}0004\bm{A}_{\cdot1} + 0{,}007\bm{A}_{\cdot 2} -{}\\
{}- 0{,}009\bm{A}_{\cdot 3} - 0{,}004\bm{A}_{\cdot 4} + 0{,}006\bm{A}_{\cdot 5}.
\end{multline*}

\vspace*{-2pt}


  
  Платежные функции всех однотипных портфелей получаются по 
правилам~(\ref{e19-aga}). Проведенные расчеты под\-тверж\-да\-ют вер\-ность 
алгоритма. Все они, несмотря на внешнее различие их пред\-став\-ле\-ний, на 
идеальном рынке должны иметь единую платежную функцию с~графиком, 
пред\-став\-лен\-ным на рис.~1. 
  
  Оптимальный \textit{смешанный портфель} строится вновь по 
формуле~(\ref{e12-aga}), но в~смешанном базисе\linebreak естественного происхождения с~выделенным 
цент\-раль\-ным страйком~$(3, 3)$. В~первом квад\-ран\-те\linebreak 
(относительно цент\-ра рынка) используются $\alpha$-оп\-ци\-оны~$\bm{A}_{11}$, во 
втором~--- $\bm{A}_{01}$, в~треть\-ем~--- $\bm{A}_{00}$, в~\mbox{чет\-вер\-том}~--- 
$\bm{A}_{10}$. 
  
  Вычисления с~применением~(\ref{e12-aga}) дают оптимальный 
\textit{смешанный} портфель:

\vspace*{-2pt}
 

\noindent
\begin{multline*}
  \bm{G}_m=0{,}602\bm{U} + 16{,}369\bm{A}_{00;11} - 35{,}107\bm{A}_{00;12} + {}\\
  {}+
18{,}737\bm{A}_{00;13} - 12{,}549\bm{A}_{00;21} + 48{,}876\bm{A}_{00;22} - {}\\
{}-
36{,}326\bm{A}_{00;23} - 3{,}82\bm{A}_{00;31} - 13{,}769\bm{A}_{00;32} +{}\\
{}+ 17{,}589\bm{A}_{00;33} - 
6{,}056\bm{A}_{01;13} + 5{,}641\bm{A}_{01;14} +{}\\
{}+ 0{,}416\bm{A}_{01;15} - 2{,}992\bm{A}_{01;23} + 
1{,}263\bm{A}_{01;24} + {}\\
{}+1{,}729\bm{A}_{01;25} + 9{,}049\bm{A}_{01;33} - 6{,}903\bm{A}_{01;34} - {}\\
{}-2{,}145\bm{A}_{01;35} - 12{,}342\bm{A}_{10;31} + 23{,}486\bm{A}_{10;32} - {}\\
{}-11{,}144\bm{A}_{10;33} + 4{,}007\bm{A}_{10;41} - 20{,}268\bm{A}_{10;42} + {}\\
{}+16{,}261\bm{A}_{10;43} + 7{,}603\bm{A}_{10;51} - 2{,}899\bm{A}_{10;52} -{}\\
{}-4{,}704\bm{A}_{10;53} +  0{,}731\bm{A}_{10;61} - 0{,}318\bm{A}_{10;62} - {}\\
{}-0{,}413\bm{A}_{10;63} + 5{,}871\bm{A}_{11;33} - 8{,}528\bm{A}_{11;34} +{}\\
{}+ 2{,}657\bm{A}_{11;35} + 3{,}351\bm{A}_{11;43} + 3{,}707\bm{A}_{11;44} - {}\\
{}-7{,}058\bm{A}_{11;45} - 8{,}889\bm{A}_{11;53} + 5{,}279\bm{A}_{11;54} +{}\\
{}+ 3{,}609\bm{A}_{11;55} - 0{,}333\bm{A}_{11;63} - 0{,}458\bm{A}_{11;64} +{}
\end{multline*}

\noindent
\begin{multline*}
{}+ 0{,}791\bm{A}_{11;65} + 1{,}3\bm{A}_{0;1\cdot} + 0{,}943\bm{A}_{0;2\cdot} - 2{,}243\bm{A}_{0;3\cdot} +{}\\
{}+ 3{,}568\bm{A}_{0;\cdot 1} - 4{,}497\bm{A}_{0;\cdot 2} + 0{,}929\bm{A}_{0;\cdot 3} - 0{,}642\bm{A}_{1;3\cdot} - {}\\
{}-
2{,}085\bm{A}_{1;4\cdot} + 2{,}492\bm{A}_{1;5\cdot} + 0{,}234\bm{A}_{1;6\cdot} - 
2{,}573\bm{A}_{1;\cdot 3} +{}\\
{}+ 2{,}145\bm{A}_{1;\cdot 4} + 0{,}428\bm{A}_{1;\cdot 5}. 
  \end{multline*}
  
  \vspace*{-2pt}
  
\noindent
  В этом портфеле 56~инструментов; среди них один безрисковый актив, 
42~двумерных $\alpha$-оп\-ци\-она~$\bm{A}_{00}$, $\bm{A}_{01}$, $\bm{A}_{10}$ 
и~$\bm{A}_{11}$ и~13~одномерных версий~$\bm{A}_0, \bm{A}_1$ в~количествах 
9, 9, 12, 12 и~6, 7 соответственно. 
  
  Двумерные $\alpha$-оп\-ци\-оны снабжены четырьмя индексами; первые два из 
них (до точки с~запятой) показывают тип опциона по каждой координате 
в~терминах~$\beta$, другие два индекса~--- номера страй\-ков. 
  
  Одномерные версии снабжены двумя индексами и~маркером <<точка>>. 
Один индекс до точки с~запятой указывает тип опциона, индекс после нее~---  
номер страй\-ка, а~маркер~--- координату под\-ра\-зу\-ме\-ва\-емо\-го безрискового 
актива.
  
  Платежная функция смешанного портфеля строится по  
правилам~(\ref{e21-aga}). Ее графиком служит все тот же график на рис.~1. 
  
  \section{Заключение }
  
  В работе решена задача алгоритмического на\-хож\-де\-ния пред\-став\-ле\-ний 
многомерных баттерфляев при сценарной дискретизации рынка и~\mbox{по\-стро\-ения} 
из них базиса в~терминах $\alpha$-оп\-ци\-онов~--- многомерного обобщения 
традиционных опционов колл и~пут. На конкретном примере двумерного рынка 
продемонстрирована работа этого алгоритма и~ее результат. Подобные расчеты 
могут быть без принципиальных трудностей реализованы и~для рынков 
большей размерности. Поскольку бат\-тер\-фляи для одномерного рынка 
образуются из трех страй\-ков, а~индикаторы~--- из двух, то их многомерные 
реп\-ли\-ка\-ции получаются еще более громоздкими. Фактические расчеты, 
проведенные для $n\hm=4$, показали, что уже объем перечня базисных 
инструментов однотипного базиса, например в~терминах~$\bm{A}_{0000}$, 
аналогичного~$\bm{A}_{00}$ из разд.~4, был соразмерен всему объему 
на\-сто\-ящей работы, а~потому и~на многомерных $\alpha$-рын\-ках лучше торговать 
непосредственно бат\-тер\-фля\-ями, а~не $\alpha$-оп\-ци\-онами. 
  
{\small\frenchspacing
 {%\baselineskip=10.8pt
 %\addcontentsline{toc}{section}{References}
 \begin{thebibliography}{9}
  \bibitem{1-aga}
  \Au{Агасандян~Г.\,А.} Применение континуального критерия VaR на 
финансовых рынках.~--- М.: ВЦ РАН, 2011. 299~с. 
  \bibitem{2-aga}
  \Au{Агасандян~Г.\,А.} Континуальный критерий VaR и~оптимальный 
портфель инвестора~// Управ\-ле\-ние большими сис\-те\-ма\-ми, 2018. Вып.~73. 
С.~6--26.
  \bibitem{3-aga}
  \Au{Агасандян~Г.\,А.} Континуальный критерий VaR на сценарных рынках~// 
Информатика и~её применения, 2018. Т.~12. Вып.~1. С.~32--40. 
  \bibitem{4-aga}
  \Au{Агасандян~Г.\,А.} Вычисление показателей оптимальных по CC-VaR 
портфелей на рынках опционов~// Информатика и~её применения, 2019. Т.~13. 
Вып.~3. С.~75--84. 
  \bibitem{5-aga}
  \Au{Агасандян~Г.\,А.} Многомерные рынки опционов и~оптимизация по  
CC-VaR~// Управ\-ле\-ние большими сис\-те\-ма\-ми, 2020. Вып.~88. С.~5--25.
  \bibitem{6-aga}
  \Au{Агасандян~Г.\,А.} Многомерные бинарные рынки и~CC-VaR~// 
Информатика и~её применения, 2022. Т.~16. Вып.~2. С.~2--10.
  \bibitem{7-aga}
  \Au{Крамер~Г.} Математические методы статистики~/ Пер. с~англ.~--- М.: 
Мир, 1975. 750~с. (\Au{Cramer~H.} Mathematical methods of statistics.~--- 
Princeton, NJ, USA: Princeton University Press, 1946. 575~p.)
\end{thebibliography}

 }
 }

\end{multicols}

\vspace*{-6pt}

\hfill{\small\textit{Поступила в~редакцию 09.03.22}}

\vspace*{8pt}

%\pagebreak

%\newpage

%\vspace*{-28pt}

\hrule

\vspace*{2pt}

\hrule

%\vspace*{-2pt}

\def\tit{MULTIDIMENSIONAL BUTTERFLIES IN~PROBLEMS 
OF~OPTIMIZATION ON CC-VaR}


\def\titkol{Multidimensional butterflies in~problems 
of~optimization on CC-VaR}


\def\aut{G.\,A.~Agasandyan}

\def\autkol{G.\,A.~Agasandyan}

\titel{\tit}{\aut}{\autkol}{\titkol}

\vspace*{-8pt}


\noindent
Federal Research Center ``Computer Science and Control'' of the Russian Academy 
of Sciences, 44-2~Vavilov Str., Moscow 119333, Russian Federation


\def\leftfootline{\small{\textbf{\thepage}
\hfill INFORMATIKA I EE PRIMENENIYA~--- INFORMATICS AND
APPLICATIONS\ \ \ 2023\ \ \ volume~17\ \ \ issue\ 1}
}%
 \def\rightfootline{\small{INFORMATIKA I EE PRIMENENIYA~---
INFORMATICS AND APPLICATIONS\ \ \ 2023\ \ \ volume~17\ \ \ issue\ 1
\hfill \textbf{\thepage}}}

\vspace*{3pt} 
  
  
  


  
  \Abste{The work continues studying problems of using continuous VaR-criterion (CC-VaR) in 
financial markets. Again some technical problems are concerned. However, they emerge this time 
not in multidimensional relatively simple binary markets but in multidimensional markets that are 
an extension of one-dimensional traditional
markets of options such as calls and puts. In assumption 
that scenario butterflies are not traded in markets directly, a~method of receiving their replication 
from multidimensional options, i.\,e., $\alpha$-options, is developed. It is based on options parity 
theorems and can be applied to markets of arbitrary dimension, but actual realization is conducted
for two-dimensional markets. The bases constructions in terms of $\alpha$-options both one-type and 
natural mixed with\linebreak\vspace*{-12pt}}

\Abstend{selected market center are produced. Theoretical representations of optimal 
portfolios in these bases accompanied with the payoffs diagram are illustrated by the distinctive 
example of a two-dimensional market.}
  
  \KWE{underliers; multidimensional market; investor's risk preferences function; continuous 
VaR-criterion; cost and forecast densities; scenario indicators; bases; binary options; one-type 
portfolio; market center; mixed portfolio}
  
 \DOI{10.14357/19922264230114} 

%\vspace*{-16pt}

%\Ack
%\noindent

  

%\vspace*{4pt}

  \begin{multicols}{2}

\renewcommand{\bibname}{\protect\rmfamily References}
%\renewcommand{\bibname}{\large\protect\rm References}

{\small\frenchspacing
 {%\baselineskip=10.8pt
 \addcontentsline{toc}{section}{References}
 \begin{thebibliography}{9} 
  \bibitem{1-aga-1}
  \Aue{Agasandyan, G.\,A.} 2011. \textit{Pri\-me\-ne\-nie kon\-ti\-nu\-al'\-no\-go kri\-te\-riya VaR 
na fi\-nan\-so\-vykh ryn\-kakh} [Application of continuous VaR-criterion in financial 
markets]. Moscow: CCRAS. 299~p.
  \bibitem{2-aga-1}
  \Aue{Agasandyan, G.\,A.} 2018. Kon\-ti\-nu\-al'\-nyy kri\-te\-riy VaR i~op\-ti\-mal'\-nyy 
port\-fel' in\-ves\-to\-ra [Continuous VaR-criterion and investor's optimal portfolio]. 
\textit{Upravlenie bol'shimi sistemami} [Large-Scale Systems Control] 73:6--26.
  \bibitem{3-aga-1}
  \Aue{Agasandyan, G.\,A.} 2018. Kon\-ti\-nu\-al'\-nyy kri\-te\-riy VaR na stse\-nar\-nykh 
ryn\-kakh [Continuous VaR-criterion in scenario markets]. \textit{Informatika i~ee 
Primeneniya~--- Inform. Appl.} 12(1):32--40.
  \bibitem{4-aga-1}
  \Aue{Agasandyan, G.\,A.} 2019. Vy\-chis\-le\-nie po\-ka\-za\-te\-ley op\-ti\-mal'\-nykh po  
CC-VaR port\-fe\-ley na ryn\-kakh op\-tsi\-o\-nov [Performance estimations for  
optimal-on-CC-VaR portfolios in option markets]. \textit{Informatika i~ee 
Primeneniya~--- Inform. Appl.} 13(3):75--84.
  \bibitem{5-aga-1}
  \Aue{Agasandyan, G.\,A.} 2020. Mno\-go\-mer\-nye ryn\-ki op\-tsi\-o\-nov i~op\-ti\-mi\-za\-tsiya 
po CC-VaR [Multidimensional option markets and optimization on CC-VaR]. 
\textit{Upravlenie bol'shimi sistemami} [Large-Scale Systems Control] 88:5--25.
  \bibitem{6-aga-1}
  \Aue{Agasandyan, G.\,A.} 2022. Mno\-go\-mer\-nye bi\-nar\-nye ryn\-ki i~CC-VaR 
[Multidimensional binary markets and CC-VaR]. \textit{Informatika i~ee 
Primeneniya~--- Inform. Appl.} 16(2):2--10.
  \bibitem{7-aga-1}
  \Aue{Cramer, H.} 1946. \textit{Mathematical methods of statistics}. Princeton, 
NJ: Princeton University Press. 575~p.

\end{thebibliography}

 }
 }

\end{multicols}

\vspace*{-6pt}

\hfill{\small\textit{Received March 9, 2022}}

  
  \Contrl
  
  \noindent
  \textbf{Agasandyan Gennady A.} (b.\ 1941)~--- Doctor of Science in physics and 
mathematics, leading scientist, A.\,A.~Dorodnicyn Computing Center, Federal 
Research Center ``Computer Science and Control'' of the Russian Academy of 
Sciences, 40~Vavilov Str., Moscow 119333, Russian Federation; 
\mbox{agasand17@yandex.ru}
  

   
\label{end\stat}

\renewcommand{\bibname}{\protect\rm Литература} 
      %1
\def\stat{kravtsova}

\def\tit{ИСПОЛЬЗОВАНИЕ КРИТЕРИЕВ СТАЦИОНАРНОСТИ ДЛЯ~НАСТРОЙКИ МОДЕЛЕЙ ПРИ~ПРОГНОЗИРОВАНИИ ВРЕМЕННЫХ РЯДОВ}

\def\titkol{Использование критериев стационарности для настройки моделей при~прогнозировании временных рядов}

\def\aut{О.\,А.~Кравцова$^1$}

\def\autkol{О.\,А.~Кравцова}

\titel{\tit}{\aut}{\autkol}{\titkol}

\index{Кравцова О.\,А.}
\index{Kravtsova O.\,A.}


%{\renewcommand{\thefootnote}{\fnsymbol{footnote}} \footnotetext[1]
%{Работа выполнена при поддержке Министерства науки и~высшего образования Российской Федерации (проект 
%075-15-2020-799).}}


\renewcommand{\thefootnote}{\arabic{footnote}}
\footnotetext[1]{Московский государственный университет имени М.\,В.~Ломоносова, \mbox{phd3984@gse.cs.msu.ru}}

\vspace*{-6pt}

     

    
\Abst{Рассматривается возможность использования информации о стационарности 
остатков для совершенствования процедуры прогнозирования нестационарных временных 
рядов. При традиционном подходе данная процедура используется только в~формате 
подтверждения или отвержения гипотезы о~нестационарности остатков. В~настоящей статье 
критерии стационарности предлагается использовать при настройке гиперпараметров для построения 
моделей прогнозирования. Методика основана на концепции коинтеграции Грейнд\-же\-ра для поиска 
статистически значимой связи между временными рядами. Для уменьшения ошибки прогноза моделей в~качестве 
функции потерь используется p-зна\-че\-ние тес\-тов на ста\-ци\-о\-нар\-ность. В~качестве данных для проверки 
использовались экономические и~сгенерированные временные ряды. Проведенные эксперименты показали, 
что нередко такой подход оказывается более эффективным по сравнению с~традиционными способами 
настройки моделей.}
    
\KW{временные ряды; стационарность; деревья решений; регрессионный анализ}

\DOI{10.14357/19922264220202}
  
%\vspace*{-4pt}


\vskip 10pt plus 9pt minus 6pt

\thispagestyle{headings}

\begin{multicols}{2}

\label{st\stat}
    

\section{Введение}

%\vspace*{-5pt}

Прогнозирование временн$\acute{\mbox{ы}}$х рядов является одной из самых исследуемых задач. 
В~общем виде задача прогнозирования состоит в~выборе такого алгоритма, 
который обеспечивает максимальное качество прогноза относительно выбранной функции потерь. 
Оптимальной настройкой для модели считается та, которая позволяет получить минимум для заданного 
функционала качества~\cite{1-kr}.

Если остатки модели нестационарны, то модель имеет неодинаковую точность прогноза в~разные 
периоды времени, т.\,е.\ нуждается в~корректировке. 
В~связи с~этим получил распространение подход, основанный на концепции коинтеграции Грейнд\-же\-ра~\cite{2-kr}, 
в~котором исследование распределения остатков по времени является составляющей оценки качества 
регрессионных моделей. Данная концепция широко используется в~финансовом секторе, например в~парном 
трейдинге для выявления ложных регрессионных связей~\cite{3-kr}.

Для прогнозирования нестационарных вре\-мен\-н$\acute{\mbox{ы}}$х рядов, которые весьма часто пред\-став\-ля\-ют основную 
проб\-ле\-му при прогнозировании экономических вре\-мен\-н$\acute{\mbox{ы}}$х рядов, используются такие\linebreak модели,
 как \mbox{ARIMA} (Autoregressive Integrated\linebreak Moving Average)~\cite{4-kr, 5-kr} и~деревья решений~\cite{6-kr, 7-kr}. 
 Указанные модели обладают ограниченной прогностической спо\-соб\-ностью при использовании на 
 нестационарных временных рядах. Для увеличения прогностической способности используются 
 различные гибридные модели и~ан\-самб\-ле\-вые методы. Модель ARIMA используют в~качестве 
 дополнения к~нейронной сети~\cite{8-kr, 9-kr, 10-kr} и~вместе с~вейвлетами~\cite{11-kr, 12-kr} 
 деревья решений объединяют в~ан\-самб\-ле\-вый метод~\cite{13-kr, 14-kr}.


В настоящей статье для стандартных моделей ARIMA и~ансамблей деревьев решений  
функция\linebreak потерь дополняется информацией о~<<степени>> статистически значимой связи между
 прогнозом и~реальными значениями (p-зна\-че\-ние тес\-тов на ста\-ци\-о\-нар\-ность). 
 Проверка предложенного метода \mbox{проведена} на экономических рядах и~сгенерированных псевдовыборках. 
 В~част\-ности, анализ и~прогнозирование паттернов поведения акций~--- одна из самых исследуемых задач 
 в~сфере фондового рынка. Особый интерес представляет прогнозирование рядов для краткосрочной
  перспективы, так как долгосрочная перспектива, как правило, показывает довольно низкие
   доходности. Большинство инвесторов теряет свой капитал в~краткосрочном периоде прогнозирования, 
   поэтому методы анализа цен акций требуют улучшения~\cite{15-kr}.


\vspace*{-6pt}


\section{Стационарность в~функции~потерь}

 Среди этапов построения модели ARIMA можно выделить 
 сле\-ду\-ющие: идентификация модели, оценка ее параметров и~диагностическая проверка~\cite{16-kr}. 
 Наиболее часто используемый в~про\-грам\-мных пакетах подход к~выбору оптимальных гиперпараметров 
 основан на критерии Акаике (AIC) из~\cite{17-kr}.

Для построения деревьев использовался алгоритм CART (Classification and Regression Trees)~\cite{18-kr}. 
Небольшие модификации могут вызывать сильные изменения в~решении дерева, поэтому необходимо 
тщательно подготавливать данные и~следить за процедурой настройки гиперпараметров~\cite{19-kr}. 
Для тестирования использовалась библиотека Scikit-learn. 
В~качестве способа подбора гиперпараметров был выбран так называемый поиск по сетке (Grid search)~\cite{18-kr}.

Данная работа основана на утверждении, что\linebreak если остатки стационарны, то модель является хорошо 
специфицированной, т.\,е.\ если модель и~совершает ошибки, то эти ошибки происходят 
по большей части из-за шумовой компоненты в~данных, которая не поддается прогнозированию с~по\-мощью 
математических методов.

В качестве первого теста на стационарность был выбран классический тест единичного корня~--- 
тест Ди\-ки--Фул\-ле\-ра~\cite{21-kr}, где использование единичного корня с~дрейфом и~детерминированного 
временн$\acute{\mbox{о}}$го тренда является опциональным:
\begin{equation*}
\Delta y_i = \alpha_0 + \alpha_1  t+\delta y_{i-1}+\varepsilon_i.
\end{equation*}


Для подсчета p-значения используется t-рас\-пре\-де\-ле\-ние (распределение Стьюдента) со ста\-ти\-сти\-кой~$\mathrm{DF}$:
\begin{equation*}
F(x) = \fr{1}{2}+x\Gamma\left(\fr{v+1}{2}\right),\enskip
 \mathrm{DF} = \fr{\bar{\alpha}} {\mathrm{s.\,e.}(\bar{\alpha})}\,,
\end{equation*}
$X \in \mathbf{R} \thicksim \mathrm{St}\left(v\right)$, $v>0$; $\Gamma$~--- гам\-ма-функ\-ция Эйлера; 
$\mathrm{s.\,e.}$ (standard error)~--- 
стандартная ошибка; $\bar{\alpha}$~--- оценка параметра~$\alpha$. 
Тогда функцию потерь можно пред\-ста\-вить с~по\-мощью p-зна\-че\-ния и~итоговая задача 
оптимизации будет выглядеть сле\-ду\-ющим образом:
\begin{equation}
\label{eqloss}
    F \left(\fr{\bar{\alpha}} {\mathrm{s.\,e.}(\bar{\alpha})}\right) \rightarrow \min\,.
\end{equation}

Экономические ряды характеризуются структурными сдвигами, т.\,е.\ 
возникают качественные изменения связей. Такой сдвиг практически невозможно предсказать,
 поэтому тест Ди\-ки--Фул\-ле\-ра покажет отсутствие стационарности даже в~случае правильной 
 спецификации. Тест Зи\-во\-та--Энд\-рю\-са позволяет обработать единичное изменение среднего и~единичное 
 изменение наклона тренда~\cite{22-kr}:
 
 \noindent
\begin{multline*}
\Delta y_i =c + \alpha y_{t-1}+\beta t  +\gamma \mathrm{DU}_t+ \theta \mathrm{DT}_t+{}\\
{} +\sum\limits_{j=1}^{k}d_j\Delta y_{t-j} + \varepsilon_t.
\end{multline*}

Функция потерь для данного теста аналогична случаю теста Ди\-ки--Фул\-ле\-ра в~формуле~(\ref{eqloss}). 
Если в~обуча\-ющих данных наблюдается структурное изменение, т.\,е.\ 
меняется тренд или среднее для одного участка выборки сильно отличается от 
среднего другого участка, будем использовать тест Зи\-во\-та--Энд\-рю\-са, 
в~противном случае используем тест Ди\-ки--Фул\-лера.

\vspace*{-2pt}

\section{Алгоритмы улучшения моделей прогнозирования}

В качестве данных для подтверждения гипотезы о релевантности использования остатков 
в~функции потерь для моделей прогнозирования были использованы следующие: курс рубля, цена на нефть, 
индекс Мосбиржи, цена на золото~\cite{23-kr}, сгенерированные ряды. Динамика этих данных напрямую 
отражается на издержках во всех отраслях экономики~\cite{24-kr}, поэтому прогнозирование данных рядов 
является актуальной задачей. Для рядов использовались средние показатели за неделю. 
Под сгенерированными рядами понимаются ряды, полученные с~по\-мощью метода бутстрепа из 
исходных (в~итоге получаем~50~различных рядов). Для каждого такого ряда строился прогноз с~по\-мощью 
модели ARIMA. В~качестве регрессоров для деревьев решений использовались лаговые значения 
и~--- для деревьев решений~--- остатки от оценок ARIMA. Все прогнозы строились на периоды от~4 до~12~недель. 
Хорошим прогнозом будем считать модель с~прогнозом с~наименьшей ошибкой, высокой корреляцией 
между спрогнозированными значениями и~действительными, стационарными остатками.

В общем виде рассмотрим алгоритм для улучшения модели ARIMA с~помощью использования
 p-зна\-че\-ния тес\-тов на ста\-цио\-нар\-ность.
\ %%<-- этот пробел для того, чтобы первый элемент перечня был

%% на следующей строке, а не в~подбор к~заголовку окружения

\vspace*{2pt}

\textbf{Алгоритм 1} (ARIMA) %\label{alg:1}
    
        \noindent
        \begin{enumerate}[1.]

       
        \item
        Используем метод поиска по сетке для различных гиперпараметров 
        модели прогнозирования для разных размеров обучающей и~тестовой выборки.
        
        \item
        Для каждой комбинации считаем величину среднеквадратичной ошибки (MSE), 
        p-зна\-че\-ние теста на стационарность для остатков модели (разница 
        между оцененными значениями обучающей выборки и~значениями обучающей выборки) и~критерий AIC.
        
        
        \item
        Отсортируем комбинации по метрике MSE для фиксированной длины тестовой выборки.
        \item
        
        Посчитаем разность между показателями \mbox{p-зна}\-че\-ния для первой настройки и~второй настройки, 
        аналогично посчитаем разность для критерия AIC.
        
        
        \item
        Если разность для p-зна\-че\-ния оказалась положительной и~эта разница превышает~0,001, 
        то переходим к~сле\-ду\-юще\-му шагу. Иначе выбираем настройку с~минимальным AIC.
        
        \item
        
        Повторяем процесс сравнения для второй и~трет\-ьей настройки и~т.\,д., пока не выполнятся оба условия.
        \end{enumerate}

\noindent
Таким образом, для обучающей выборки считаем MSE, p-зна\-че\-ние и~AIC, а~для тестовой~--- только MSE.
Для деревьев решений нет необходимости в~оценке сложности модели, однако, чтобы 
лучше контролировать процесс обучения, рассмотрим корреляции. Показатель корреляции 
Пирсона для фиксированной длины тестовой выборки высчитаем сле\-ду\-ющим образом.

\vspace*{2pt}

\textbf{Алгоритм 2} (Корреляции) % \label{alg:11}
    
        \noindent
        \begin{enumerate}[1.]
        
        \item
        Из исходного ряда выберем фиксированный размер обучающей выборки для 
        фиксированной даты последнего наблюдения.

        
        \item
        Из обучающей выборки исключим последние наблюдения размером с~тестовую выборку, 
        на этих значениях будем вычислять коэффициент корреляции. Назовем их валидационной 
        выборкой. Объединим оставшиеся значения из обучающей выборки с~предшествующими ей 
        значениями из исходной выборки
        
        \item
        Для разных размеров обучающей выборки (т.\,е.\ различных подвыборок из <<пред\-шест\-ву\-ющие 
        значения исходной вы\-бор\-ки\;+\;обуча\-ющая вы\-бор\-ка\;$-$\;раз\-мер тес\-то\-вой выборки>>) 
        используем поиск по сетке гиперпараметров, построим прогнозы для валидационной выборки. 
        Посчитаем корреляцию прогноза валидационной выборки с~истинными значениями
        
        \item
        Для каждого набора гиперпараметров для изначальной фиксированной даты 
        последнего наблюдения из п.~1 и~изначально заданной длины тестовой выборки сохраним 
        значения корреляций.
        

    \end{enumerate}


Теперь рассмотрим общий алгоритм для де\-ревь\-ев решений.

\vspace*{2pt}

\textbf{Алгоритм 3} (Деревья решений)

    \noindent
    \begin{enumerate}[1.]
    

        \item Используем метод поиска по сетке для различных гиперпараметров модели 
        прогнозирования для разных размеров обучающей и~тес\-то\-вой выборки.

        
        \item

        Для каждой комбинации считаем величину MSE и~p-зна\-че\-ние 
        теста на ста\-ци\-о\-нар\-ность для остатков модели.

        
        
        \item

        Отсортируем комбинации по метрике MSE для фиксированной длины тес\-то\-вой выборки.

        \item

        Посчитаем разность между показателями \mbox{p-зна}\-че\-ния для первой настройки и~второй настройки.

        
        \item

        Если разность для p-зна\-че\-ния оказалась положительной и~эта разница превышает~0,001, 
        то переходим к~следующему шагу. Иначе выбираем настройку с~наибольшим показателем корреляции. 
        Показатель корреляции выбирается из сохраненных значений из алгоритма~2 для выбранной длины 
        обуча\-ющей и~тес\-то\-вой выборок.

        
        \item

        Повторяем процесс сравнения для второй и~треть\-ей настройки и~т.\,д., пока не выполнятся оба условия.

        
        
    \end{enumerate}




\section{Вычислительный эксперимент}

 Вначале рассмотрим возможность использования величины p-зна\-че\-ния тес\-тов 
 на стационарность и~среднеквадратичной ошибки для отбора размера тренировочной 
 выборки при прогнозировании с~по\-мощью алгоритма построения модели ARIMA на основе 
 критерия Акаике из~\cite{17-kr}.

В качестве примера на рис.~1 продемонстрирована величина ошибки прогноза (MSE) 
 цены на нефть для разных промежутков ряда, где размер обуча\-ющей выборки подбирался 
 на основании критерия стационарности остатков~--- p\_val и~на критерии величины среднеквадратичной ошибки~--- 
 mse.  Чем меньше оцененные значения отличаются от действительных, 
 тем лучше получится результат прогноза. График показывает, что использование 
 стационарности остатков в~качестве критерия отбора размера обуча\-ющей выборки в~ряде случаев 
 уже позволяет уменьшать ошибку прогноза.
 


На рис.~2 для выбранного оптимального размера обучающей выборки показана 
разница между величиной ошибки алгоритма, основанного на популярном подходе к~настройке ARIMA из~\cite{17-kr}, 
mse(auto\_arima),  и~на критерии стационарности для различных рядов, mse(p\_val). 
Итоговая разница никогда не становится меньше нуля, что говорит о~том, что  метод 
как минимум не хуже, а~в~ряде случае и~лучше, чем классический подход подбора.



Для деревьев решений рассмотрим анализ одного конкретного нестационарного ряда фиксированной длины на рис.~3 
(в~данном случае ряд индекса\linebreak\vspace*{-12pt}

\pagebreak

\end{multicols}


\begin{figure*} %fig1
\vspace*{1pt}
  \begin{center}  
    \mbox{%
\epsfxsize=99.747mm
\epsfbox{kra-1.eps}
}

\end{center}
\vspace*{-9pt}
\Caption{Величина ошибки прогноза (MSE) цены на нефть для двух критериев 
    отбора размера обучающей выборки: \textit{1}~--- критерий стационарности остатков~--- p\_val; \textit{2}~--- 
    критерий величины среднеквадратичной ошибки~--- mse}
    \label{pinki}
\end{figure*}

\begin{figure*} %fig2
\vspace*{9pt}
  \begin{center}  
    \mbox{%
\epsfxsize=102.865mm
\epsfbox{kra-2.eps}
}

\end{center}
\vspace*{-9pt}
\Caption{Разница между величиной ошибки прогноза алгоритма, основанного на 
    критерии Акаике~(\textit{1}) и~на критерии стационарности~(\textit{2})}
    \label{pic2}
\end{figure*}

\begin{multicols}{2}

\noindent
 Мосбиржи). На этом рисунке p\_val~--- 
отбор на основе стационарности остатков, min mse train~--- на основе величины ошибки mse, min mse test~--- 
минимально возможная ошибка при лучшем подборе гиперпараметров для заданного размера тренировочной выборки. 
На некоторых участках графики p\_val~(\textit{1}) и~min mse train~(\textit{2}) пересекаются. 

Каждый из графиков показывает, 
насколько минимальной может быть ошибка прогноза на тестовой выборке для конкретного временн$\acute{\mbox{о}}$го 
ряда при фиксированной длине прогноза для разных размеров обучающей выборки, т.\,е.\ 
перебираем различные комбинации гиперпараметров для разных размеров обуча\-ющей выборки и~выбираем те, 
которые на тестовой выборке продемонстрировали наименьшую ошибку прогноза. Таким образом, задача~--- 
приблизиться к~минимуму графика  min mse train.



Однако, в~отличие от ARIMA, использование \mbox{p-зна}\-че\-ния для деревьев не всегда позволяет 
улучшить результат. Использование такого алгоритма релевантно только в~том случае, если прогноз 
дела-\linebreak\vspace*{-12pt}

\pagebreak

\end{multicols}

\begin{figure*} %fig3
\vspace*{1pt}
  \begin{center}  
    \mbox{%
\epsfxsize=100.047mm
\epsfbox{kra-3.eps}
}

\end{center}
\vspace*{-9pt}
\Caption{Подбор размера обучающей выборки деревьев решений для одного из рядов: \textit{1}~--- p\_val; \textit{2}~--- min mse train; \textit{3}~--- min mse test}
    \label{pic3}
\end{figure*}

\begin{figure*} %fig4
\vspace*{9pt}
  \begin{center}  
    \mbox{%
\epsfxsize=103.065mm
\epsfbox{kra-4.eps}
}

\end{center}
\vspace*{-9pt}
\Caption{Разница между величиной ошибки прогнозов алгоритмов, основанных на 
    величине ошибки MSE~(\textit{1}) и~на критерии стационарности~(\textit{2})}
    \label{pic4}
\end{figure*}

\begin{multicols}{2}

\noindent
ется
 на период от~8 до~12~недель. На рис.~4 пред\-став\-ле\-ны только результаты прогноза для~8 и~12~недель.
  Как и~в~случае с~\mbox{ARIMA}, график позволяет говорить о~том, 
что метод с~учетом стационарности остатков как минимум не хуже  (разность между величиной 
ошибки традиционного подхода и~метода с~учетом стационарности никогда не меньше нуля).



Основываясь на эмпирических результатах исследования, 
мож\-но сделать вывод, что использование информации о стационарности остатков
 представляется релевантным для уменьшения ошибки прогноза.

Однако данный метод обладает рядом минусов. Во-пер\-вых, это высокая вычислительная слож\-ность, 
так как необходимо рассматривать множество различных комбинаций гиперпараметров. Во-вто\-рых, 
для деревьев решений результат работы алгоритма нестабилен для разных размеров периода 
прогнозирования. В~связи с~этим следует довести метод до более компактного итеративного процесса, 
который бы обеспечивал статистически значимое улучшение прогноза моделей.


\section{Заключение}

Для выбора корректной настройки гиперпараметров моделей прогнозирования 
существует большое разнообразие подходов. В~данной статье рассматривались наиболее популярные 
подходы\linebreak к~построению прогнозов нестационарных вре\-мен\-н$\acute{\mbox{ы}}$х рядов с~по\-мощью \mbox{ARIMA} и~деревьев решений. 
В~качестве подхода к~настройке гиперпараметров был выбран метод из~\cite{17-kr} 
для ARIMA и~алгоритм поиска по сетке для метода CART построения деревьев решений. 
В~статье проверяется возможность использования p-зна\-че\-ния тестов на стационарность в~функции 
потерь для улучшения прогнозов данных алгоритмов.

Наибольшую эффективность тесты на стационарность для остатков модели продемонстрировали 
для модели ARIMA, для деревьев решений использование p-зна\-че\-ния нерелевантно при слишком 
малом или слишком большом периоде прогнозирования.

{\small\frenchspacing
 {%\baselineskip=10.8pt
 %\addcontentsline{toc}{section}{References}
 \begin{thebibliography}{99}

\bibitem{1-kr}
\Au{Khmelnitskaya A.\,B.}
Social welfare functions for different subgroup utility scales~// 
Constructing and applying objective functions~/ Eds. A.~Tangian, J.~Gruber.~--- 
Lecture notes in economics and mathematical systems ser.~---
 Berlin, Heidelberg: Springer-Verlag, 2002. 
Vol.~510. P.~515--530.  doi: 10.1007/978-3-642-56038-5.


\bibitem{2-kr}
\Au{Engle R.\,F., Granger~C.\,W.\,J.} 
Co-integration and error correction: Representation, estimation, and testing~//  
Econometrica, 1987. Vol.~55. No.\,2. P.~251--276.  doi: 10.2307/ 1913236.


\bibitem{3-kr}
\Au{Hatemi-J A.}
Tests for cointegration with two unknown regime shifts with an application to financial market integration~//  
Empir. Econ., 2008. Vol.~35. P.~497--505. doi: 10.1007/s00181-007-0175-9.

\bibitem{5-kr} %4
\Au{Ariyo A.\,A., Adewumi~A.\,O., Ayo~C.\,K.}
Stock price prediction using the ARIMA model~// 16th 
 Conference (International) on Computer Modeling and Simulation Proceedings.~--- Piscataway, NJ, USA: IEEE, 2014. P.~106--112. 
 doi: 10.1109/UKSim.2014.67.

\bibitem{4-kr} %5
\Au{Li T., Zhong~J., Huang~Z.}
Potential dependence of financial cycles between emerging and developed countries: Based on ARIMA-GARCH 
copula model~// Emerg. Mark. Financ. Tr., 2019. Vol.~56. No.\,6. P.~1--14. doi: 10.1080/ 1540496X.2019.1611559.


\bibitem{7-kr} %6
\Au{Gepp A., Kumar~K., Bhattacharya~S.}
Business failure prediction using decision trees~//
J.~Forecasting, 2010. Vol.~29. No.\,6. P.~536--555. doi: 10.1002/for.1153.

\bibitem{6-kr} %7
\Au{Aggarwal C.\,C.}
Mining time series data~// Data mining.~--- Springer, 2015. P.~457--491. 
doi: 10.1007/978-3-319-14142-8.


\bibitem{8-kr}
\Au{Namini S.\,S., Tavakoli~N., Namin~A.\,S.}
A~comparison of ARIMA and LSTM in forecasting time series~//  17th Conference 
(International) on Machine Learning and Applications Proceedings.~--- Piscataway, NJ, USA: IEEE, 2018. 
P.~536--555. doi: 10.1109/ICMLA.2018.00227.


\bibitem{9-kr}
\Au{Temur A.\,S., Akgun~M., Temur~G.}
Predicting housing sales in Turkey using ARIMA, LSTM and hybrid models~//  J.~Bus. Econ. 
Manag., 2019. Vol.~20. No.\,5. P.~920--938.
doi: 10.3846/jbem.2019.10190.

\bibitem{10-kr}
\Au{Wang Z., Lou~Y.}
 Hydrological time series forecast model based on wavelet de-noising and ARIMA-LSTM~//  
 3rd Information Technology, Networking, Electronic and Automation Control Conference Proceedings.~--- 
 Piscataway, NJ, USA: IEEE, 2019. P.~1697--1701. doi: 10.1109/\linebreak ITNEC.2019.8729441.




\bibitem{12-kr} %11
\Au{Ye T.}
Stock forecasting method based on wavelet analysis and ARIMA-SVR model~// 3rd Conference 
(International) on Information Management Proceedings.~--- Piscataway, NJ, USA: IEEE, 2017. P.~102--106. 
doi: 10.1109/\mbox{INFOMAN}.2017.7950355.

\bibitem{11-kr} %12
\Au{Singh S., Parmar K.\,S., Kumar~J., Makkhan~S.\,J.\,S.}
Development of new hybrid model of discrete wavelet decomposition and autoregressive integrated 
moving average\linebreak (\mbox{ARIMA}) models in application to one month forecast the casualties cases of COVID-19~// 
Chaos Soliton. Fract., 2020. Vol.~135. Art.~109866.  doi: 10.1016/ j.chaos.2020.109866.

\bibitem{14-kr} %13
\Au{Zhou F., Zhang Q., Sornette~D., Jiang~L.}
Cascading logistic regression onto gradient boosted decision trees for forecasting and trading stock 
indices~//  Appl. Soft Comput., 2019. Vol.~84. No.\,2. Art.~105747. 13~p. doi: 10.1016/ j.asoc.2019.105747.

\bibitem{13-kr} %14
\Au{Ribeiro M.\,H.\,D.\,M., Coelho~L.}
Ensemble approach based on bagging, boosting and stacking for short-term prediction in 
agribusiness time series~// Appl. Soft Comput., 2020. Vol.~86. Art.~105837. 17~p.
 doi: 10.1016/j. asoc.2019.105837.




\bibitem{15-kr}
\Au{Elder A.}
The new trading for a living: Psychology, discipline, trading tools and systems, risk control, 
trade management.~--- 
Wiley Trading ser.~--- New York, NY, USA: John Wiley \& Sons Ltd., 2014. 304~p.

\bibitem{16-kr}
\Au{Tabachnick B.\,G., Fidell L.\,S.}
 Using multivariate statistics.~--- 6th ed.~--- London: Pearson Education, 2013. 983~p.


\bibitem{17-kr}
\Au{Hyndman R., Khandakar Y.}
Automatic time series forecasting: The forecast package for~R~// J.~Stat. Softw., 2008. Vol.~27. 
No.\,3. P.~1--22. doi: 10.18637/jss.v027.i03.

\bibitem{18-kr}
\Au{Pedregosa F., Varoquaux~G., Gramfort~A.,  \textit{et al.}} Scikit-learn: Machine learning in Python~// J.~Mach. Learn. Res., 
2011. Vol.~12. P.~2825--2830. %Decision Trees.
%{\sf https://scikit-learn.org/stable/modules/tree.html.}

\bibitem{19-kr}
\Au{Singh S., Gupta P.}
Comparative study Id3, Cart and C4.5 decision tree algorithm: A~survey~// Int. J.~Advanced Information 
Science Technology, 2014. Vol.~3. No.\,7. P.~47--52. doi: 10.15693/ijaist/2014.v3i7.


%\bibitem{20-kr}
%\Au{Pedregosa F., \textit{et al.}} Scikit-learn: Machine learning in Python~// J.~Machine 
%Learning Research, 2011. Vol.~12. P.~2825-2830. GridSearchCV.
%{\sf https://scikit-learn.org/\linebreak stable/modules/generated/sklearn.model\_selection.GridSearchCV.html}

\bibitem{21-kr} %20
\Au{Dickey D.\,A., Fuller W.\,A.}
Distribution of the estimators for autoregressive time series with a unit root~// 
J.~Am. Stat. Assoc., 1979. Vol.~74. No.\,366. P.~427--431. doi: 10.2307/ 2286348.

\bibitem{22-kr}
\Au{Zivot E., Andrews~D.}
Further evidence on the great crash, the oil-price shock, and the unit-root hypothesis~// J.~Bus. 
Econ. Stat., 2002. Vol.~10. No.\,3. P.~251--270. doi: 10.2307/ 1391541.

\bibitem{23-kr}
Финам. Данные от 01.01.2005 до 20.11.2020. {\sf https:// www.finam.ru}.

\bibitem{24-kr}
\Au{Литвинова Я.\,С.} Цена на нефть как ключевой фактор воздействия на российскую валюту~// 
Проб\-ле\-мы и~перспективы экономики и~управ\-ле\-ния: Мат-лы I~Междунар. на\-учн. конф.~--- 
СПб.: Реноме, 2012. С.~19--21. {\sf  https://moluch.ru/conf/econ/ archive/15/2170}.
\end{thebibliography}

 }
 }

\end{multicols}

\vspace*{-8pt}

\hfill{\small\textit{Поступила в~редакцию 26.10.21}}

\vspace*{8pt}

%\pagebreak

%\newpage

%\vspace*{-28pt}

\hrule

\vspace*{2pt}

\hrule

%\vspace*{-2pt}

\def\tit{MODEL SETTING USING STATIONARITY CRITERIA\\ FOR~TIME SERIES FORECASTING}


\def\titkol{Model setting using stationarity criteria for time series forecasting}


\def\aut{O.\,A.~Kravtsova}

\def\autkol{O.\,A.~Kravtsova}

\titel{\tit}{\aut}{\autkol}{\titkol}

\vspace*{-8pt}


\noindent
M.\,V.~Lomonosov Moscow State University, 1-52~Leninskie Gory, GSP-1, Moscow 119991, Russian Federation

\def\leftfootline{\small{\textbf{\thepage}
\hfill INFORMATIKA I EE PRIMENENIYA~--- INFORMATICS AND
APPLICATIONS\ \ \ 2022\ \ \ volume~16\ \ \ issue\ 2}
}%
 \def\rightfootline{\small{INFORMATIKA I EE PRIMENENIYA~---
INFORMATICS AND APPLICATIONS\ \ \ 2022\ \ \ volume~16\ \ \ issue\ 2
\hfill \textbf{\thepage}}}

\vspace*{3pt} 







\Abste{The article discusses the possibility of using the information on the stationarity 
of residuals to improve the procedure of forecasting nonstationary time series.
 In the traditional approach, this procedure is used only to confirm or reject 
 the hypothesis of nonstationarity of residuals. In this article, the stationarity 
 test is used for fine-tuning of hyperparameters of the forecasting models. The technique 
 is based on the Granger cointegration approach property to find a~statistically 
 significant relationship between time series. The author used the p-value of stationarity 
 tests as a~loss function. Economic and generated time series were used as data for verification. 
 The experiments have shown that this approach is often more effective in comparison with the traditional 
 methods of tuning models.}

\KWE{time series; stationarity; decision trees; regression analysis}

\DOI{10.14357/19922264220202}

%\vspace*{-16pt}

%\Ack
%\noindent




%\vspace*{4pt}

  \begin{multicols}{2}

\renewcommand{\bibname}{\protect\rmfamily References}
%\renewcommand{\bibname}{\large\protect\rm References}

{\small\frenchspacing
 {%\baselineskip=10.8pt
 \addcontentsline{toc}{section}{References}
 \begin{thebibliography}{99}
\bibitem{1-kr-1}
\Aue{Khmelnitskaya, A.\,B.} 2002. Social welfare 
functions for different subgroup utility scales. \textit{Constructing and applying objective functions}. 
Eds. A.~Tangian, and J.~Gruber. Lecture notes in economics and mathematical systems ser.
Berlin, Heidelberg: Springer-Verlag. 510:515--530. 
doi: 10.1007/978-3-642-56038-5.
\bibitem{2-kr-1}
\Aue{Engle, R.\,F., and C.\,W.\,J.~Granger.}
 1987. Co-integration and error correction: Representation, estimation, and testing. 
 \textit{Econometrica} 55(2):251--276. doi: 10.2307/ 1913236.
\bibitem{3-kr-1}
\Aue{Hatemi-J, A.} 2008. Tests for cointegration with two unknown regime shifts with an 
application to financial market integration. \textit{Empir. Econ.} 35:497--505. 
doi: 10.1007/ s00181-007-0175-9.

\bibitem{5-kr-1} %4
\Aue{Ariyo, A.\,A., A.\,O.~Adewumi, and C.\,K.~Ayo.} 2014. 
Stock price prediction using the ARIMA model. \textit{16th  Conference
(International) on Computer Modeling and Simulation Proceedings}. Piscataway, NJ: IEEE. 106--112. 
doi: 10.1109/UKSim.2014.67.

\bibitem{4-kr-1} %5
\Aue{Li, T., J.~Zhong, and Z.~Huang.} 2019. Potential dependence
 of financial cycles between emerging and developed countries: Based on ARIMA-GARCH copula model.
 \textit{Emerg. Mark. Financ. Tr.} 56(6):1--14. 
 doi: 10.1080/ 1540496X.2019.1611559.


\bibitem{7-kr-1} %6
\Aue{Gepp, A., K.~Kumar, and S.~Bhattacharya.} 2010. Business failure prediction using decision trees. 
\textit{J.~Forecasting} 29(6):536--555. doi: 10.1002/for.1153.

\bibitem{6-kr-1} %7
\Aue{Aggarwal, C.\,C.} 2015. Mining time series data. \textit{Data mining}. Springer. 457--491.

\bibitem{8-kr-1}
\Aue{Namini, S.\,S., N.~Tavakoli, and A.\,S.~Namin.} 2018. A~comparison of ARIMA and LSTM 
in forecasting time series. 
\textit{17th Conference (International) on Machine Learning and Applications Proceedings}. 
Piscataway, NJ: IEEE. 536--555. doi: 10.1109/ICMLA.2018.00227.
\bibitem{9-kr-1}
\Aue{Temur, A.\,S., M.~Akgun, and G.~Temur.} 2019. Predicting housing sales in Turkey using ARIMA, 
LSTM and hybrid models. \textit{J.~Bus. Econ. Manag.} 20(5):920--938. doi: 10.3846/jbem.2019.10190.
\bibitem{10-kr-1}
\Aue{Wang, Z., and Y.~Lou.} 2019. Hydrological time series forecast model based on wavelet de-noising 
and ARIMA-LSTM. \textit{3rd Information Technology, Networking, Electronic and Automation Control 
Conference Proceedings}. Piscataway, NJ: IEEE. 1697--1701. doi: 10.1109/\linebreak ITNEC.2019.8729441.


\bibitem{12-kr-1} %11
\Aue{Ye, T.} 2017. Stock forecasting method based on wavelet analysis and ARIMA-SVR model. 
\textit{3rd Conference (International)  on Information Management Proceedings}.
Piscataway, NJ: IEEE. 102--106. 
doi: 10.1109/ \mbox{INFOMAN}.2017.7950355.

\bibitem{11-kr-1} %12
\Aue{Singh, S., K.\,S.~Parmar, J.~Kumar, and S.\,J.\,S.~Makkhan.}
 2020. Development of new hybrid model of discrete wavelet decomposition and autoregressive 
 integrated moving average (ARIMA) models in application to one month forecast the casualties cases 
 of COVID-19. \textit{Chaos Soliton. Fract.} 135(1):109866. doi: 10.1016/j.chaos.2020.109866. 8~p.
 

\bibitem{14-kr-1} %13
\Aue{Zhou, F., Q.~Zhang, D.~Sornette, and L.~Jiang.}
 2019. Cascading logistic regression onto gradient boosted decision trees for forecasting and 
 trading stock indices. \textit{\mbox{Appl}. Soft Comput.} 84(2):105747. 13~p. 
 doi: 10.1016/j.asoc. 2019.105747. 
 
 \bibitem{13-kr-1} %14
\Aue{Ribeiro, M.\,H.\,D.\,M., and L.~Coelho.}
 2020. Ensemble approach based on bagging, boosting and stacking for short-term prediction in
  agribusiness time series. \textit{Appl. Soft Comput.} 86:105837. 17~p. 
  doi: 10.1016/j.asoc.2019.105837. 
  
\bibitem{15-kr-1}
\Aue{Elder, A.} 2014. \textit{The new trading for a~living: Psychology, discipline, trading tools and systems, 
risk control, trade management}. 
Wiley Trading ser.
New York, NY: John Wiley \& Sons Ltd. 304~p.
\bibitem{16-kr-1}
\Aue{Tabachnick, B.\,G., and L.\,S.~Fidell.}
2013. \textit{Using multivariate statistics}. 6th ed. London: Pearson Education. 983~p.
\bibitem{17-kr-1}
\Aue{Hyndman, R., and Y.~Khandakar.} 2008. Automatic time series forecasting: The forecast package for 
R. \textit{J.~Stat. Softw.} 27(3):1--22. doi: 10.18637/jss.v027.i03.
\bibitem{18-kr-1}
\Aue{Pedregosa, F., G.~Varoquaux, A.~Gramfort, \textit{et al.}}
 2011. Scikit-learn: Machine learning in Python. Decision Trees. 
 \textit{J.~Mach. Learn. Res.} 12:2825--2830.
\bibitem{19-kr-1}
\Aue{Singh, S., and P.~Gupta.}
 2014. Comparative study Id3, cart and C4.5 decision tree algorithm: A~survey. 
 \textit{Int. J.~Advanced Information Science Technology} 3(7):47--52. doi: 10.15693/ijaist/2014.v3i7.
%\bibitem{20-kr-1}
%\Aue{Pedregosa, F., \textit{et al.}} 2011. Scikit-learn: Machine learning in Python. 
%GridSearchCV. \textit{J.~Machine Learning Research} 12:2825--2830.
\bibitem{21-kr-1}
\Aue{Dickey, D.\,A., and W.\,A.~Fuller.}
 1979. Distribution of the estimators for autoregressive time series with a~unit root. 
 \textit{J.~Am. Stat. Assoc.} 74(366):427--431. doi: 10.2307/ 2286348.
\bibitem{22-kr-1}
\Aue{Zivot, E., and D.~Andrews.} 2002. Further evidence on the great crash, the oil-price shock,
 and the unit-root hypothesis. \textit{J.~Bus. Econ. Stat.} 10(3):251--270. 
 doi: 10.2307/ 1391541.
\bibitem{23-kr-1}
Finam. Dannye ot 01.01.2005 do 20.11.2020 [Data from 01.01.2005 to 20.11.2020]. Available at: 
{\sf https://www.\linebreak finam.ru/} (accessed April~6, 2022). 
\bibitem{24-kr-1}
\Aue{Litvinova, Ya.\,S.} 2012. Tsena na neft' kak klyuchevoy faktor vozdeystviya na rossiyskuyu valyutu 
[Oil price as a~key factor affecting the Russian currency]. 
\textit{Mat-ly I Mezhdunar. nauchn. konf. ``Problemy i~perspektivy ekonomiki i~upravleniya''}
 [I Scientific Conference (International) ``Problems and Prospects of Economics and Management'' Proceedings]. 
 St.\ Petersburg: Renome. 19--21. Available at: 
 {\sf  https://moluch.ru/conf/econ/archive/15/ 2170} (accessed June~3, 2022).
 \end{thebibliography}

 }
 }

\end{multicols}

\vspace*{-6pt}

\hfill{\small\textit{Received October 26, 2021}}

\Contrl

\noindent
\textbf{Kravtsova Olga A.} (b.\ 1996)~--- 
PhD student, Faculty of Computational Mathematics and Cybernetics, M.\,V.~Lomonosov Moscow State University, 
1-52~Leninskie Gory, GSP-1, Moscow 119991, Russian Federation; \mbox{phd3984@gse.cs.msu.ru}



   

\label{end\stat}

\renewcommand{\bibname}{\protect\rm Литература}     %2
\def\stat{bosov+stef}

\def\tit{УПРАВЛЕНИЕ ВЫХОДОМ СТОХАСТИЧЕСКОЙ ДИФФЕРЕНЦИАЛЬНОЙ СИСТЕМЫ 
ПО~КВАДРАТИЧНОМУ КРИТЕРИЮ. I.~ОПТИМАЛЬНОЕ РЕШЕНИЕ МЕТОДОМ 
ДИНАМИЧЕСКОГО ПРОГРАММИРОВАНИЯ$^*$}

\def\titkol{Управление выходом стохастической дифференциальной системы 
по~квадратичному критерию. I}
%.~Оптимальное решение методом 
%динамического программирования}

\def\aut{А.\,В.~Босов$^1$, А.\,И.~Стефанович$^2$}

\def\autkol{А.\,В.~Босов, А.\,И.~Стефанович}

\titel{\tit}{\aut}{\autkol}{\titkol}

\index{Босов А.\,В.}
\index{Стефанович А.\,И.}
\index{Bosov A.\,V.}
\index{Stefanovich A.\,I.}




{\renewcommand{\thefootnote}{\fnsymbol{footnote}} \footnotetext[1]
{Работа выполнена при частичной поддержке РФФИ (проект 16-07-00677).}}


\renewcommand{\thefootnote}{\arabic{footnote}}
\footnotetext[1]{Институт проблем информатики Федерального исследовательского центра <<Информатика 
и~управление>> Российской академии наук, \mbox{AVBosov@ipiran.ru}}
\footnotetext[2]{Институт проблем информатики Федерального исследовательского центра <<Информатика 
и~управление>> Российской академии наук, \mbox{AStefanovich@frccsc.ru}}

%\vspace*{8pt}



  
  \Abst{Решается задача оптимального управления для диффузионного процесса 
Ито и~линейного управ\-ля\-емо\-го выхода. Рассматриваемая постановка близка 
к~классической ли\-ней\-но-квад\-ра\-тич\-ной гауссовской задаче управления 
(linear-quadratic Gaussian (LQG) control). Отличия состоят в~том, что состояние описывается нелинейным 
дифференциальным уравнение Ито $dy_t\hm= A_t(y_t) \,dt\hm+ \Sigma_t(y_t)\,dv_t$ 
и~не зависит от управ\-ле\-ния~$u_t$, оптимизации подлежит управ\-ля\-емый 
линейный выход $dz_t\hm= a_t y_t\,dt\hm+ b_t z_t \,dt\hm+ c_t u_t \,dt\hm+ \sigma_t\, 
dw_t$. Дополнительные обобщения внесены в~квад\-ра\-тич\-ный критерий качества 
с~целью воз\-мож\-ности постановки таких задач, как отслеживание выходом 
состояния или управ\-ле\-ни\-ем~--- линейной комбинации состояния и~выхода. Для 
решения используется метод динамического программирования. Функцию 
Беллмана позволяет найти предположение о~ее структуре вида $V_t(y,z)\hm= 
\alpha_t z^2\hm+ \beta_t(y)z \hm+\gamma_t(y)$. Решение дают три 
дифференциальных уравнения для коэффициентов~$\alpha_t$, $\beta_t(y)$ 
и~$\gamma_t(y)$. Эти уравнения со\-став\-ля\-ют оптимальное решение 
рас\-смат\-ри\-ва\-емой задачи.}
  
  \KW{стохастическое дифференциальное уравнение; оптимальное управ\-ле\-ние; 
динамическое программирование; функция Беллмана; уравнение Риккати; 
линейные уравнения параболического типа}

\DOI{10.14357/19922264180314}
  
%\vspace*{4pt}


\vskip 10pt plus 9pt minus 6pt

\thispagestyle{headings}

\begin{multicols}{2}

\label{st\stat}

\section{Введение}

     Ключевые результаты в~области оптимизации стохастических 
динамических систем, со\-став\-ля\-ющие классическую теорию управления, 
получены более~40~лет назад (такова работа~[1] в~отношении задачи 
управ\-ле\-ния ли\-ней\-но-гаус\-сов\-ски\-ми стохастическими сис\-те\-ма\-ми по 
квад\-ра\-тич\-но\-му критерию). К~классической тео\-рии следует относить 
линейные модели стохастических сис\-тем и~квадратичный критерий качества. 
Это исходный базис, на котором основано множество успешно 
исследованных и~решенных задач стохастического управ\-ле\-ния 
и~оптимизации. 

Дальнейшее развитие~--- это новые модели и~критерии, но 
прежде всего это новые методы: от тео\-рии линейных регуляторов, метода 
динамического программирования и~принципа максимума к~адаптивному 
и~минимаксному подходу, импульсному управ\-ле\-нию и~т.\,д. Множество 
инноваций как в~час\-ти моделей, так и~в~час\-ти математического аппарата, 
имевших мес\-то в~по\-сле\-ду\-ющие годы, существенно обогатили тео\-рию 
управ\-ле\-ния. Но и~до настоящего времени линейные модели и~квадратичный 
критерий, несмотря на всю справедливую критику в~отношении их 
аде\-кват\-ности и~гиб\-кости, сохраняют исследовательский интерес и~находят 
современные области приложения.
     
     Не претендуя на сколь\-ко-ни\-будь полное обосно\-ва\-ние последнего 
тезиса, приведем несколько примеров, показавшихся наиболее ин\-те\-рес\-ными. 

Так, в~[2] решается ли\-ней\-но-квад\-ра\-тич\-ная за\-да\-ча в~игровой 
постановке с~запаздыванием. В~близ\-кой по модели работе~[3] задача 
управ\-ле\-ния ставится в~терминах $H_\infty$-ро\-баст\-ности. Точнее \mbox{называть} 
эту тематику $H_2/H_\infty$-управ\-ле\-ни\-ем, и~работ по этой теме очень 
много. Аккуратности ради следует уточнить, что под линейными 
понимаются модели с~мультипликативными по состоянию воз\-му\-ще\-ниями. 

Совсем другой класс моделей, особо популярных в~по\-след\-ние годы, 
составляют скачкообразные процессы. Например, линейные уравнения 
в~сочетании с~пуассоновскими скачками в~[4] используются в~моделях, 
описывающих различные показатели функционирования сетевых протоколов 
передачи данных транспортного уровня. Телекоммуникации представляют 
в~последние годы самый популярный прикладной материал для 
исследований, работ по этой проб\-ле\-ма\-ти\-ке множество, математические 
техники привлекаются самые разные и~самые современные, но и~линейным 
моделям место находится. Еще один любопытный пример исследования 
скачкообразного процесса и~оптимизации на основе квад\-ра\-тич\-но\-го критерия 
можно найти в~[5] применительно к~задаче инвестирования на финансовом 
рынке. Наконец, упомянем еще работу~[6], подводящую итог исследований 
в~отношении классической детерминированной  
ли\-ней\-но-квад\-ра\-тич\-ной задачи с~использованием техники матричных 
неравенств.
     
     В данной работе также эксплуатируются привлекательные свойства 
линейных моделей и~квад\-ра\-тич\-но\-го критерия, причем в~стохастической 
постановке. На\-прав\-ле\-ни\-ем для обобщения \mbox{выбрана} модель динамики 
сис\-те\-мы: основные усилия на\-прав\-ле\-ны на то, чтобы сделать ее нелинейной. 
Кроме того, пред\-став\-лен\-ная постановка может рас\-смат\-ри\-вать\-ся и~как 
обобщение ранее решенной задачи в~дискретном времени~[7, 8] на время 
непрерывное. В~упомянутых работах помимо собственно модельной 
постановки важна еще и~привлекаемая прикладная об\-ласть~--- 
функционирование сложных программных сис\-тем. Результатов, 
ориентированных непосредственно на такие приложения, к~настоящему 
времени пренебрежимо мало, поэтому~[7, 8]~--- это еще и~прикладное 
обоснование рас\-смат\-ри\-ва\-емой далее задачи.
     
     Оптимизируемая динамическая сис\-те\-ма описывается двумя 
уравнениями. Состояние задается нелинейным стохастическим 
дифференциальным уравнением Ито, не содержащим управ\-ля\-емой 
переменной. Возмущение здесь описывается стандартным винеровским 
процессом, накладываются простые условия существования 
и~един\-ст\-вен\-ности решения. Поскольку состояние не управ\-ля\-ет\-ся, то уместно 
его интерпретировать как слож\-ное внешнее возмущение. Вторая 
переменная~--- управ\-ля\-емый выход~--- задается линейным стохастическим 
дифференциальным уравнением. Цель оптимизации выхода формируется 
квадратичным критерием общего вида. Формальная постановка задачи 
приведена в~сле\-ду\-ющем разделе.
     
     Для решения задачи используется метод динамического 
программирования, решается уравнение Беллмана~[9]. Соответственно, 
в~результате получаются аналитические выражения и~для оптимального 
управ\-ле\-ния, и~для значения функционала качества. Технически 
традиционный, стандартный подход к~задаче обременен, пожалуй, 
единственной проблемой~--- поиском верного пред\-став\-ле\-ния структуры 
функции Беллмана. Справиться с~этой проблемой в~большей степени удается 
за счет результата, полученного при решении дискретного по времени 
аналога рассматриваемой постановки~\cite{8-bos}. Конечные соотношения 
для оптимального решения, как и~во всех подобных задачах, включая 
классическую ли\-ней\-но-квад\-ра\-тич\-ную, содержат решения 
определенных дифференциальных уравнений (обыкновенных и~в~частных 
производных). Вывод этих уравнений и~со\-став\-ля\-ет содержание первой час\-ти 
данной работы. Во второй части будет обсуждаться их приближенное 
чис\-лен\-ное решение и~компьютерные эксперименты.
     
     Кратко обозначим основные положения, при\-вле\-ка\-емые далее 
к~решению задачи, следуя в~основном обозначениям 
и~терминологии~\cite{9-bos}, а~именно: будем рассматривать задачу 
оптимального управления в~стохастической динамической сис\-те\-ме по полной 
информации, применяя метод динамического программирования. В~качестве 
целевого функционала, опре\-де\-ля\-юще\-го качество управ\-ле\-ния $U_0^T\hm= \{ 
u_t,\ 0\leq t\leq T\}$, выступает
     \begin{equation}
     J\left(U_0^T\right)={\sf E}\left\{ \int\limits_0^T L_t \left(x_t, u_t\right)\,dt+ 
l\left(x_T\right)\right\}\,.
     \label{e1-bos}
     \end{equation}
Здесь ${\sf E}\{\cdot\}$~--- оператор математического ожидания; $x_t$~--- 
случайный процесс, описываемый стохастическим дифференциальным 
уравнением Ито
     \begin{equation}
     dx_t=m_t\left( x_t, u_t\right) dt+ \sigma_t\left( x_t\right)dW_t\,,\enskip 
x_0=X\,,
     \label{e2-bos}
     \end{equation}
где $W_t$~--- стандартный винеровский процесс подходящей раз\-мер\-ности; 
$X$~--- случайный вектор.

     $U_0^T$ будем выбирать из класса допустимых неупреждающих (по 
отношению к~$W_t$) управлений~\cite{9-bos}. Соответственно, 
относительно функций сноса и~диффузии~$m_t$ и~$\sigma_t$  
в~(\ref{e2-bos}) будем предполагать выполненными ка\-кие-ли\-бо условия 
существования сильного решения для заданного до\-пус\-ти\-мо\-го управ\-ле\-ния. 
Например, для управ\-ле\-ния с~обратной связью $u_t\hm= u_t(x_t)$ будем 
считать, что $m_t(x,u_t(x))$ и~$\sigma_t(x)$ удовлетворяют условию 
линейного рос\-та и~локальному условию Липшица по~$x$ равномерно 
по~$t$ (т.\,е.\ условиям Ито).
     
     Для поиска оптимального управления, минимизирующего $J(U_0^T)$, 
рас\-смат\-ри\-ва\-ет\-ся функция Беллмана
     \begin{equation}
     V_t(x)=\left.\mathop{\mathrm{inf}}\limits_{U_t^T} {\sf E} \left\{ \int\limits_t^T 
L_t \left( x_t, u_t\right)\,dt+l\left( x_T\right) \right\vert \mathcal{F}_t^x\right\}\,,
     \label{e3-bos}
     \end{equation}
где $\mathcal{F}_t^x$~--- $\sigma$-ал\-геб\-ра, по\-рож\-ден\-ная~$x_\tau$, 
$0\hm\leq \tau\hm\leq t$, ${\sf E}\{\cdot\vert \mathcal{F}\}$~--- оператор условного 
математического ожидания относительно~$\mathcal{F}$. Соответственно, 
в~качестве достаточного условия оп\-ти\-маль\-ности воспользуемся уравнением 
динамического программирования
\begin{multline}
\fr{\partial V_t(x)}{\partial t} +\fr{1}{2}\sum\limits^n_{i,j=1} \sigma^2_{t_{ij}}
\fr{\partial^2 V_t(x)}{\partial x_i \partial x_j}+{}\\
{}+\min\limits_u\left[  
\sum\limits^n_{i=1} m_{t_i} \fr{\partial V_t(x)}{\partial x_i} + L_t(x,u)\right] 
=0\,,\\
V_T(x)=l(x)\,,
\label{e4-bos}
\end{multline}
где $m_{t_i}$~--- $i$-й элемент век\-тор-функ\-ции~$m_t(x,u)$; 
$\sigma^2_{t_{ij}} \hm= \sum\nolimits^m_{k=1} 
\sigma_{t_{ik}}\sigma_{t_{ki}}$, $\sigma_{t_{ij}}$~--- $i$-й по строке, $j$-й 
по столб\-цу элемент мат\-рич\-ной функции~$\sigma_t(x)$; $n$ и~$m$~--- 
размерности~$x_t$ и~$W_t$ соответственно.

     Традиционно в~рамках применения метода динамического 
программирования будем предполагать, что функции~$L_t$, $l$, $m_t$ 
и~$\sigma_t$ обеспечивают существование хотя бы одного решения 
уравнения~(\ref{e4-bos}), а~следовательно, и~оптимального 
управления~$u_t^*$, $0\hm\leq t\hm\leq T$, до\-став\-ля\-юще\-го минимум 
целевому функционалу~(\ref{e1-bos}). Задача оптимизации далее получается 
путем указания конкретных выражений для~$L_t$, $l$, $m_t$ и~$\sigma_t$.

\section{Постановка задачи управления выходом}

     Рассматриваемые далее случайные функции будут предполагаться 
скалярными. Такое упрощение позволит разгрузить выкладки и~итоговые 
выражения от не самых существенных деталей.
     
     Рассмотрим стохастическую дифференциальную сис\-те\-му, со\-сто\-яние 
которой представляет диффузи\-он\-ный процесс~$y_t$, описываемый 
нелинейным стохастическим дифференциальным уравнением Ито
     \begin{equation}
     dy_t=A_t\left( y_t\right) dt +\Sigma_t \left( y_t\right) dv_t\,,\enskip 
y_0=Y\,,
     \label{e5-bos}
     \end{equation}
где $v_t$~--- стандартный (одномерный) винеровский процесс; $Y$~--- 
случайная величина с~конечным вторым моментом; функции~$A_t$ 
и~$\Sigma_t$ удовлетворяют условиям Ито:
\begin{equation*}
\left\vert A_t(y)\right\vert +\left\vert \Sigma_t(y)\right\vert \leq C(1+\vert y\vert )\ 
\mbox{для\ всех } 0\leq t\leq T\,;
\end{equation*}

\vspace*{-12pt}

\noindent
\begin{multline*}
\hspace*{-2.10051pt}\left\vert A_t\left(y_1\right) -A_t \left( y_2\right) \right\vert +\left\vert 
\Sigma_t\left( y_1\right) -\Sigma_t \left(y_2\right)\right\vert \leq
C\left\vert y_1-y_2\right\vert\\
 \mbox{для\ всех\ } 0\leq t\leq T\ \mbox{и } 
y_1,y_2\in \mathbb{R}^1\,,
\end{multline*}
обеспечивающим существование единственного сильного (потраекторного) 
решения уравнения.
     
     Будем считать, что~$y_t$ описывает состояние некоторой 
динамической системы. Соответственно, поведение этой сис\-те\-мы опишем 
выходом, линейно связанным с~со\-сто\-янием:
     \begin{equation}
     dz_t=a_t y_t \,dt+ b_t z_t \,dt+ c_t u_t \,dt+\sigma_t \,dw_t\,,\enskip
     z_0=Z\,.
     \label{e6-bos}
     \end{equation}
Здесь $w_t$~--- не зависящий от~$v_t$, $Y$ и~$Z$ стандартный (одномерный) 
винеровский процесс; $Z$~--- случайная величина с~конечным вторым 
моментом; $u_t$~--- допустимое неупреждающее управ\-ле\-ние, качество 
которого определяется целевым функционалом следующего вида:
\begin{multline}
\!\hspace*{-3.98538pt}J\left( U_0^T\right) ={\sf E}\left\{ \int\limits_0^T \!\left( S_t\left( s_ty_t-g_t z_t -h_t 
u_t\right)^2 +G_t z_t^2+{}\right.\right.\\
\left.\left.{}+ H_t u_t^2
\vphantom{S_t\left( s_ty_t-g_t z_t -h_t 
u_t\right)^2}
\right) dt+S_T\left( s_T y_T -g_T 
z_T\right)^2+G_T z_T^2
\vphantom{\int\limits_0^T}\right\}\,,
\label{e7-bos}
\end{multline}
где $S_t$, $G_t$ и~$H_t$~--- неотрицательные функции\linebreak
$0\hm\leq t\hm\leq T$. 
Такой критерий отражает физический смысл задачи распределения ресурсов 
со\-глас\-но аналогичной~(\ref{e5-bos})--(\ref{e7-bos}) задаче для дис\-крет\-но\-го 
времени, рас\-смот\-рен\-ной в~\cite{7-bos}. В~част\-ности,  
функци\-онал~(\ref{e7-bos}) поз\-во\-ля\-ет ставить задачи отслеживания
 выходом 
со\-сто\-яния сис\-те\-мы, используя сла\-га\-емое $(y_t\hm- z_t)^2$, или 
управлением~--- линейной комбинации со\-сто\-яния и~выхода, сла\-га\-емое типа\linebreak 
$(y_t\hm+ z_t\hm- u_t)^2$. Поскольку задача формулируется 
в~предположении наличия пол\-ной информации о~со\-сто\-янии~$y_t$ 
и~выходе~$z_t$ (соответствующую $\sigma$-ал\-геб\-ру 
обозначим~$\mathcal{F}_t^{y,z}$), то допустимое управ\-ле\-ние ищется 
в~классе~$\mathcal{F}_t^{y,z}$-из\-ме\-ри\-мых неупреждающих функций 
(и,~как будет показано далее, оказывается управ\-ле\-ни\-ем с~обратной связью).

     Функции~$a_t$, $b_t$, $c_t$ и~$\sigma_t$ будем предполагать 
ограниченными: $\vert a_t\vert \hm+ \vert b_t\vert \hm+\vert c_t\vert \hm+ \vert 
\sigma_t \vert \hm\leq C$ для всех $0\hm\leq t\hm\leq T$, процесс  
управления~--- допустимым не\-упреж\-да\-ющим~\cite{9-bos}, обеспечивая, 
таким образом, существование сильного решения урав\-не\-ния~(\ref{e6-bos}) 
для любого допустимого управ\-ления.
     
     Задачу составляет поиск~$u_t^*$~--- допустимого управ\-ле\-ния, 
доставляющего минимум квад\-ра\-тич\-но\-му функционалу~$J(U_0^T)$.
      
     Поставленная задача очевидным образом формулируется в~терминах 
введенных выше в~(\ref{e1-bos})--(\ref{e3-bos}) обозначений, а~именно: 
     требуется обозначить
     \begin{gather*}
      x_t=\begin{pmatrix}
     y_t\\ z_t\end{pmatrix};\quad  m_t(x_t, u_t)=\begin{pmatrix}
     A_t(y_t)\\ a_t y_t +b_t z_t +c_t u_t\end{pmatrix};\\
     \sigma_t(x_t)= \begin{pmatrix}
     \Sigma_t(y_t)& 0\\
     0& \sigma_t\end{pmatrix};\quad W_t=\begin{pmatrix}
     v_t \\ w_t\end{pmatrix}
     %     \label{e8-bos}
     \end{gather*}
для записи уравнения со\-сто\-яния типа~(\ref{e2-bos}) и
\begin{align*}
L_t(x,u)&= L_t(y,z,u) ={}\\
&\hspace*{3mm}{}=S_t\left( s_t y-g_t z -h_t u\right)^2 +G_t z^2 +H_t  u^2\,;\\
l(x)&= l(y,z) =S_T \left( S_T y-g_T z\right)^2 +G_T z^2
%\label{e9-bos}
\end{align*}
для записи целевого функционала в~виде~(\ref{e1-bos}).

     Функция Беллмана~(\ref{e3-bos}) принимает вид 
     $V_t(x)\hm= V_t(y,z)$. Для записи со\-от\-вет\-ст\-ву\-юще\-го~(\ref{e4-bos}) 
уравнения Беллмана для~$V_t(y,z)$ заметим, что
     $$
     \left( \sigma^2_{t_{ij}}\right)_{i,j=1,2}= \begin{pmatrix}
     \Sigma_t^2(y) & 0\\
     0 & \sigma_t^2\end{pmatrix}\,.
     $$
     
     С~учетом перечисленных обозначений урав\-не\-ние динамического 
программирования~(\ref{e4-bos}) принимает вид:
     \begin{multline}
     \fr{\partial V_t(y,z)}{\partial t} +\fr{1}{2}\left( \Sigma_t^2(y) \fr{\partial^2 
V_t(y,z)} {\partial y^2}+\sigma_t^2\fr{\partial^2 V_t(y,z)} {\partial 
z^2}\right)+{}\\
    {}+\min\limits_u\! \left[ A_t(y) \fr{\partial V_t(y,z)}{\partial y}+\left( a_t 
y+b_t z+c_t u\right) \fr{\partial V_t(y,z)}{\partial z} +{}\right.\hspace*{-3pt}\\
\left.{}+ S_t\left( s_t y-g_t z-h_t 
u\right)^2+G_t z^2+H_t u^2
     \vphantom{\fr{\partial V_t(y,z)}{\partial y}}\right] =0\,,\\
     V_T(y,z)=S_T\left( s_T y-g_T z\right)^2+G_T z^2\,.
     \label{e10-bos}
     \end{multline}
     Это и~есть то самое уравнение, которое требуется решить: 
существование решения данного урав\-не\-ния суть достаточное условие 
оптимальности; оптимальное управ\-ле\-ние при этом~--- точ\-ка минимума 
со\-от\-вет\-ст\-ву\-юще\-го сла\-га\-емого.
     
\section{Динамическое программирование и~оптимальное 
управление}

     В рассматриваемой постановке линейность\linebreak выхода и~квадратичность 
критерия дают те же преимущества, что и~в~классической  
ли\-ней\-но-квад\-ра\-тич\-ной задаче управ\-ле\-ния~\cite{1-bos}, а~именно: 
позволяют сразу определить вид оптимального управ\-ле\-ния и~фактические 
условия его существования. Действительно, со\-хра\-няя в~(\ref{e10-bos}) под 
знаком $\min\nolimits_u$ только члены, зависящие от~$u$, получаем
     \begin{multline*}
     \fr{\partial V_t(y,z)}{\partial t} +\fr{1}{2}\left( \Sigma_t^2(y) \fr{\partial^2 
V_t(y,z)} {\partial y^2}+\sigma_t^2\fr{\partial^2 V_t(y,z)} {\partial 
z^2}\right)+{}\\
     {}+A_t(y)\fr{\partial V_t(y,z)}{\partial y}+\left( a_t y+b_t z\right) 
\fr{\partial V_t(y,z)}{\partial z}+{}\\
{}+S_t\left( s_t y-g_t z\right)^2 +G_t z^2+{}
\end{multline*}

\noindent
\begin{multline*}
     {}+\min\limits_u \left[ \left( c_t \fr{\partial V_t(y,z)}{\partial z}-2S_t \left( 
s_t y-g_t z\right) h_t\right)u +{}\right.\\
\left.{}+\left( S_t h_t^2+H_t\right) u^2
\vphantom{\fr{\partial V_t(y,z)}{\partial z}}
\right]=0\,,
     %\label{e11-bos}
     \end{multline*}
откуда в~предположении $S_t h_t^2\hm+ H_t\hm>0$ следует, что существует 
оптимальное управ\-ле\-ние, которое определяется равенством
\begin{multline}
u_t^* = u_t^*(y,z)=-\fr{1}{2}\left( S_t h_t^2 +H_t\right)^{-1} \left( c_t 
\fr{\partial V_t(y,z)}{\partial z}-{}\right.\\
\left.{}-2S_t\left( s_t y-g_t z\right) h_t
\vphantom{\fr{\partial V_t(y,z)}{\partial z}}
\right)
\label{e12-bos}
\end{multline}
и доставляет минимум соответствующему сла\-га\-емо\-му в~урав\-не\-нии Беллмана, 
равный
$-\left( S_t h_t^2\hm+\right.$\linebreak
$\left.{}+H_t\right)^{-1} \left( c_t 
{\partial V_t(y,z)}/{\partial 
z}\hm-2S_t\left( s_t y \hm-g_t z\right) h_t \right)^2/4.
$ 
     
     Отметим, что, как и~в~классической ли\-ней\-но-квад\-ра\-тич\-ной 
задаче, управ\-ле\-ние из класса до\-пус\-ти\-мых не\-упреж\-да\-ющих получилось 
управ\-ле\-ни\-ем с~обратной связью.
     
     Таким образом, функция Беллмана описывается сле\-ду\-ющим 
дифференциальным уравнением:
     \begin{multline}
     \fr{\partial V_t(y,z)}{\partial t} +\fr{1}{2}\left( \Sigma_t^2(y) \fr{\partial^2 
V_t(y,z)} {\partial y^2}+\sigma_t^2\fr{\partial^2 V_t(y,z)} {\partial 
z^2}\right)+{}\\
     {}+ A_t(y) \fr{\partial V_t(y,z)}{\partial y}+\left( a_t y+b_t z\right) 
\fr{\partial V_t(y,z)}{\partial z}+{}\\
{}+ S_t \left( s_t y- g_t z\right)^2 +G_t z^2-
 \fr{1}{4}\left( S_t h_t^2+H_t\right)^{-1}\times{}\\
 {}\times \left( c_t \fr{\partial V_t(y,z)} 
{\partial z}-2S_t\left( s_t y -g_t z\right) h_t \right)^2=0\,.
     \label{e13-bos}
     \end{multline}
     
     Возводя в~квадрат по\-след\-нее сла\-га\-емое в~(\ref{e13-bos}), перепишем 
его в~виде:
     \begin{multline}
     \fr{\partial V_t(y,z)}{\partial t} +\fr{1}{2}\left( \Sigma_t^2(y) \fr{\partial^2 
V_t(y,z)} {\partial y^2}+\sigma_t^2\fr{\partial^2 V_t(y,z)} {\partial 
z^2}\!\right)+{}\\
{}+A_t(y) \fr{\partial V_t(y,z)}{\partial y}
+ \left( 
\vphantom{\left( S_t h_t^2 +H_t\right)^{-1}}
a_t y+b_t z+{}\right.\\
\left.{}+\left( S_t h_t^2 +H_t\right)^{-1}
 c_t S_t \left( s_t y-g_t z\right) h_t
\right) 
     \fr{\partial V_t(y,z)}{\partial z}+{}\\
     {}+\left( S_t-\left( S_t h_t^2 +H_t\right)^{-1} S_t^2 h_t^2\right)\left( s_t y -
g_t z\right)^2+{}\\
     \!\!{}+
     G_t z^2 -\fr{1}{4}\left( S_t h_t^2+H_t\right)^{-1}\! c_t^2
     \left(\fr{\partial V_t(y,z)}{\partial z}\right)^{\!2}=0\,.\!\!
     \label{e14-bos}
     \end{multline}
     
     Рассматривая полученное уравнение, заметим, что его решение может 
быть пред\-став\-ле\-но в~виде:
   \begin{equation}
     V_t(y,z)= \alpha_t z^2+\beta_t(y) z +\gamma_t(y)\,,
     \label{e15-bos}
     \end{equation}
т.\,е.\ будем искать решение при дополнительном предположении 
о~квад\-ра\-тич\-ности функции Белл\-ма\-на по переменной~$z$, и~сведем, таким 
образом, поиск оптимального решения к~уравнениям относительно функций 
$\alpha_t$, $\beta_t(y)$ и~$\gamma_t(y)$. Отметим сразу, что явный вид 
функции~$\gamma_t(y)$ для реализации оптимального управ\-ле\-ния не 
требуется, однако далее будет предложен вариант вы\-чис\-ле\-ния и~этой 
функции, что пред\-став\-ля\-ет\-ся небесполезным, поскольку позволит выполнять 
расчет минимума целевого функционала. Источником для 
предложения~(\ref{e15-bos}) является уже упоминавшаяся аналогичная 
задача для случая дис\-крет\-но\-го времени~\cite{7-bos, 8-bos}. В~той задаче 
выражение для функции Беллмана получается формально без 
дополнительных усилий. При этом форма~(\ref{e15-bos}) обнаруживается 
как свойство оптимального решения. В~рассматриваемом случае 
непрерывного времени~(\ref{e15-bos}) постулируется, а~пра\-виль\-ность 
постулата под\-тверж\-да\-ет\-ся далее ре\-зуль\-ти\-ру\-ющи\-ми уравнениями 
для~$\alpha_t$, $\beta_t(y)$ и~$\gamma_t(y)$ Кроме того, данное 
предположение пред\-став\-ля\-ет\-ся вы\-те\-ка\-ющим из линейной структуры задачи 
в~отношении переменной~$z$, в~част\-ности, тем фактом, что такой вид 
функции Беллмана обеспечивает линейность оптимального 
управ\-ле\-ния~(\ref{e12-bos}) по~$z$.

     Граничное условие при выбранном предположении~(\ref{e15-bos}) 
принимает вид:

\noindent
     \begin{multline*}
     V_T(y,z)= S_T\left( s_T y- g_T z\right)^2+G_T z^2 ={}\\[-0.5pt]
     {}=\alpha_T z^2 
+\beta_T(y) z +\gamma_T(y)\,,
    \end{multline*}
т.\,е.

\noindent
\begin{align*}
\alpha_T&= S_T g_T^2 +G_T\,;\\[-0.5pt]
\beta_T(y)&=-2S_T s_T g_T y\,;\\[-0.5pt]
\gamma_T(y)&=S_T s_T^2 y^2\,.
%\label{e16-bos}
\end{align*}
          При этом само оптимальное управ\-ле\-ние, определенное 
выражением~(\ref{e12-bos}), оказывается управ\-ле\-ни\-ем с~обратной связью 
по~$y_t$ и~$z_t$:

\noindent
     \begin{multline}
     u_t^*=u_t^*(y,z) ={}\\[-0.5pt]
     {}=
     -\fr{1}{2}\left( S_t h_t^2 +H_t\right)^{-1}
     \left( c_t \left( 2\alpha_t z +\beta_t(y)\right) +{}\right.\\[-0.5pt]
    \left. {}+2S_t\left( s_t y-g_t z\right) 
h_t\right)\,.
     \label{e17-bos}
     \end{multline}
          Подставляем $V_t(y,z)\hm= \alpha_t z^2 \hm+ \beta_t(y) 
z\hm+\gamma_t(y)$ в~(\ref{e14-bos}):

\noindent
     \begin{multline*}
     \fr{\partial \alpha_t}{\partial t}\, z^2 +
     \fr{\partial \beta_t(y)}{\partial t}\,z +
     \fr{\partial \gamma_t(y)}{\partial t}+{}\\[-0.5pt]
     {}+\fr{1}{2}\left( \Sigma_t^2(y) \left( 
\fr{\partial^2\beta_t(y)}{\partial y^2}\,z +\fr{\partial^2 \gamma_t(y)}{\partial 
y^2}\right) +2\sigma_t^2\alpha_t\right)+{}\\[-0.5pt]
 {}+A_t(y)\left(\fr{\partial \beta_t(y)}{\partial y}\,z + \fr{\partial 
\gamma_t(y)}{\partial y}\right) +{}\\[-0.5pt]
\hspace*{-0.22987pt}{}+\left( a_t y+b_t z+\left( S_t h_t^2 +H_t\right)^{-1} c_t S_t \left( s_t y-
g_t z\right) h_t\right)\times{}
\end{multline*}

\noindent
\begin{multline*}
         {}\times \left( 2\alpha_t z+\beta_t(y)\right)+{}\\
     {}+\left( S_t-\left( S_t h_t^2 +H_t\right)^{-1} S_t^2 h_t^2\right)\left( s_t y-
g_t z\right)^2+{}\\
     {}+ G_t z^2 -\fr{1}{4}\left( S_t h_t^2 +H_t\right)^{-1} c_t^2 \left( 
2\alpha_t z+\beta_t(y)\right)^2=0\,.
     \end{multline*}
          Далее выделяем слагаемые при~$z^2$, $z$ и~$z^0$
          
          \noindent
     \begin{multline*}
     \fr{\partial \alpha_t}{\partial t}\, z^2 +\fr{\partial \beta_t(y)}{\partial t}\,z +
     \fr{\partial \gamma_t(y)}{\partial 
t}+\fr{1}{2}\,\Sigma_t^2(y)\fr{\partial^2\beta_t(y)}{\partial y^2}\,z+ {}\\
{}+
\fr{1}{2}\,\Sigma_t^2(y)\fr{\partial^2\gamma_t(y)}{\partial 
y^2}+\sigma_t^2\alpha_t+A_t(y)\fr{\partial \beta_t(y)}{\partial y}\,z +{}\\
{}+A_t(y) \fr{\partial 
\gamma_t(y)}{\partial y}+{}\\
{}+ 2\alpha_t \left( b_t -\left( S_t h_t^2+H_t\right)^{-1} c_t 
S_t h_t g_t \right)z^2+{}\\
     {}+
     \left( 2\alpha_t\left( \alpha_t+\left( S_t h_t^2+H_t\right)^{-1} c_t S_t h_t 
s_t\right)y +{}\right.\\
\left.{}+\beta_t(y) \left( b_t-\left( S_t h_t^2+H_t\right)^{-1} c_t S_t h_t 
g_t\right) \right) z+{}\\
     {}+\beta_t(y)\left( a_t +\left( S_t h_t^2+H_t\right)^{-1} c_t S_t h_t s_t\right) 
y+{}\\
{}+ \left( S_t -\left( S_t h_t^2+H_t\right)^{-1} S_t^2 h_t^2\right) g_t^2 z^2-{}\\
     {}- 2\left( S_t -\left( S_t h_t^2+H_t\right)^{-1} S_t^2 h_t^2\right) s_t g_t yz 
+{}\\
{}+
     \left( S_t-\left( S_t h_t^2+H_t\right)^{-1} S_t^2 h_t^2\right) s_t^2 y^2+{}\\
     {}+G_t z^2 -\left( S_t h_t^2 +H_t\right)^{-1} c_t^2 \alpha_t^2 z^2 -{}\\
     {}-\left( 
S_t h_t^2+H_t\right)^{-1} c_t^2 \alpha_t \beta_t(y) z-{}\\
{}-
\fr{1}{4}\left( S_t h_t^2+H_t\right)^{-1}  c_t^2 \beta_t^2(y)=0\,,
     \end{multline*}
группируем их и~получаем сле\-ду\-ющие уравнения:
\begin{itemize}
\item  для~$\alpha_t$:

\noindent
\begin{multline}
\fr{\partial\alpha_t}{\partial t}+2\alpha_t\left( b_t-\left( S_t h_t^2+H_t\right)^{-1} c_t 
S_t h_t g_t\right)+{}\\
{}+ \left( S_t- \left( S_t h_t^2+H_t\right)^{-1} S_t^2 h_t^2\right) 
g_t^2+G_t-{}\\
\hspace*{-8mm}{}-\left( S_t h_t^2+H_t\right)^{-1} c_t^2 \alpha_t^2 =0\,,\enskip \alpha_T=S_T 
g_t^2+G_T\,;\!\!
\label{e18-bos}
\end{multline}
\item для $\beta_t$:

\noindent
\begin{multline}
\fr{\partial\beta_t(y)}{\partial 
t}+\fr{1}{2}\,\Sigma_t^2(y)\fr{\partial^2\beta_t(y)}{\partial y^2} 
+A_t(y)\fr{\partial \beta_t(y)}{\partial y}+{}\\
{}+ 2\alpha_t\left( a_t +\left( S_t h_t^2+H_t\right)^{-1} c_t S_t h_t s_t\right) y+{}\\
{}+
\beta_t(y)\left( b_t -\left( S_t h_t^2 +H_t\right)^{-1} c_t S_t h_t g_t\right)-{}\\
{}-2\left( S_t-\left( S_t h_t^2+H_t\right)^{-1} S_t^2 h_t^2\right) s_t g_t y-{}
\\
{}-
\left( S_t h_t^2+H_t\right)^{-1} c_t^2 \alpha_t \beta_t(y)=0\,,\\
\beta_T(y)=-2S_T s_T g_T y\,;
\label{e19-bos}
\end{multline}
\item  для $\gamma_t$:
\begin{multline}
\hspace*{-0.8pt}\fr{\partial \gamma_t(y)}{\partial t}+\fr{1}{2}\,\Sigma_t^2(y)
\fr{\partial^2 \gamma_t(y)}{\partial y^2} +\sigma_t^2 \alpha_t +A_t(y)
\fr{\partial \gamma_t(y)}{\partial y}+{}\\
{}+ \beta_t(y)\left( a_t +\left( S_t h_t^2+H_t\right)^{-1} c_t S_t h_t s_t\right) y+{}\\
{}+
\left( S_t-\left( S_t h_t^2+H_t\right)^{-1} S_t^2 h_t^2\right)  s_t^2 y^2-{}\\
{}-\fr{1}{4}\left( S_t h_t^2+H_t\right)^{-1} c_t^2 \beta_t^2(y) =0\,,\\
\gamma_T(y)=S_T s_T^2 y^2\,.
\label{e20-bos}
\end{multline}
\end{itemize}
     
     Уравнение~(\ref{e18-bos}), легко заметить, является уравнением 
Риккати, которое в~силу сформулированного выше условия   
имеет единственное неотрицательное решение для всех $0\hm\leq t\hm\leq T$. 
Этот факт требует дополнительного комментария. Для получения 
уравнения~(\ref{e18-bos}) рас\-смот\-рим исходную задачу при дополнительных 
условиях $a_t\hm=0$ и~$s_t\hm=0$ для всех $0\hm\leq t\hm\leq T$. Нетрудно 
видеть, что эти условия рассматриваемую по\-ста\-нов\-ку сводят фактически 
к~классической ли\-ней\-но-квад\-ра\-тич\-ной задаче. Имеющуюся 
в~рассматриваемой формулировке чуть более общую форму целевой 
функции (принципиального значения это обобщение, конечно, не имеет) 
сведем к~классической еще одним предположением: $S_t\hm=0$ для всех 
$0\hm\leq t\hm\leq T$. Теперь уравнение для~$\alpha_t$ принимает хорошо 
известный вид:
     \begin{equation}
     \fr{\partial \alpha_t}{\partial t}+2\alpha_t b_t +G_t- H_t^{-1} c_t^2 
\alpha_t^2=0\,,\enskip \alpha_T=G_T\,.
     \label{e21-bos}
     \end{equation}

     В таком случае, как известно~\cite{10-bos}, существует единственное 
оптимальное управление~--- линейное с~обратной связью по выходу~$z_t$, 
с~коэффициентом усиления, опи\-сы\-ва\-емым уравнением  
Риккати~(\ref{e21-bos}). Именно этот результат дают  
уравнения~(\ref{e18-bos})--(\ref{e20-bos}) и~описываемая ими функция 
Беллмана~(\ref{e15-bos}), так как из $a_t\hm=0$ и~$s_t\hm=0$ немедленно 
следует, что $\beta_t(y)\hm=0$, откуда, в~свою очередь, с~учетом 
не\-за\-ви\-си\-мости решения от~$y_t$ следует, что $\gamma_t(y)\hm=\gamma_t$, 
т.\,е.\ не зависит от~$y$ и~задается уравнением: 
     $$
     \fr{\partial \gamma_t(y)}{\partial t} +\sigma^2_t \alpha_t=0\,,\enskip 
\gamma_T=0\,.
     $$ 
     Оптимальное управ\-ле\-ние при этом 
     $$
     u_t^*= -H_t^{-1} c_t \alpha_t z_t\,,
     $$
      т.\,е.\ все полностью совпадает с~известным классическим решением.
     
     С уравнениями~(\ref{e19-bos}) и~(\ref{e20-bos}) ситуация, естественно, 
обстоит сложнее. Это линейные уравнения второго порядка параболического 
типа, поскольку\linebreak
 $\Sigma_t^2(y)\hm>0$. Фактически отсутствуют 
конструктивные условия, гарантирующие существование их\linebreak
 решений 
(требовать, чтобы все фигурирующие в~уравнениях коэффициенты были 
представлены аналитическими функциями на всем пространстве значений, 
вряд ли целесообразно), поэтому далее будем предполагать, что данные 
уравнения имеют на рас\-смат\-ри\-ва\-емом интервале $0\hm\leq t\hm\leq T$ хотя 
бы одно ограниченное решение и~именно эти условия будем рас\-смат\-ри\-вать 
как достаточные условия существования оптимального решения 
рассматриваемой задачи.
     
     Таким образом, доказана следующая тео\-рема.
     
     \smallskip
     
     \noindent
     \textbf{Теорема.}\ \textit{Пусть для диффузионного 
процесса}~(\ref{e5-bos}) \textit{выполнены условия Ито, для 
     процесса}~(\ref{e6-bos})~--- \textit{ограничены коэффициенты, 
уравнения}~(\ref{e18-bos})--(\ref{e20-bos}) \textit{имеют ограниченные 
решения для $0\hm\leq t\hm\leq T$. Тогда минимум  
функционалу}~(\ref{e7-bos}) \textit{доставляет оптимальное 
управ\-ле\-ние}~(\ref{e17-bos}), \textit{где} $y\hm= y_t$; $z\hm=z_t$.
     
\section{Заключение}

     Рассмотренная задача оптимизации в~целом близка и~по модели, и~по 
критерию к~классической ли\-ней\-но-квад\-ра\-тич\-ной постановке. 
Принципиальным отличием является нелинейная модель для описания 
со\-сто\-яния динамической сис\-те\-мы, в~которой отсутствует управ\-ля\-ющее 
воздействие.\linebreak
 Такую модель наряду с~традиционной интер\-пре\-тацией  
<<со\-сто\-яние--вы\-ход>> мож\-но понимать как\linebreak модель неконтролируемого 
слож\-но\-го внешнего воздействия. Небольшое дополнительное отличие дает 
предложенная форма квад\-ра\-тич\-но\-го критерия, поз\-во\-ля\-ющая, в~част\-ности, 
ставить такие задачи, как отслеживание выходом или управ\-ле\-ни\-ем со\-сто\-яния 
сис\-те\-мы или ее выхода.
     
     Поскольку обсуждать возможности точного решения уравнений, 
определяющих оптимальное управ\-ле\-ние, не имеет смыс\-ла, наиболее 
актуальной далее является задача их приближенного чис\-лен\-но\-го решения 
и~анализа воз\-мож\-ности практической реализации. Этому посвящена вторая 
часть данной работы, пла\-ни\-ру\-емая к~выходу в~ближайшее время.

{\small\frenchspacing
 {%\baselineskip=10.8pt
 \addcontentsline{toc}{section}{References}
 \begin{thebibliography}{99}
\bibitem{1-bos}
\Au{Athans M.} Editorial on the LQG problem~// IEEE~T. Automat. Contr., 1971. Vol.~16. 
No.\,6. P.~528--552. doi: 10.1109/TAC.1971.1099845.
\bibitem{2-bos}
\Au{Wu Z.} Forward-backward stochastic differential equations, linear quadratic stochastic 
optimal control and nonzero sum differential games~// J.~Syst. Sci. Complex., 2005. Vol.~18. 
No.\,2. P.~179--192.
\bibitem{3-bos}
\Au{Chen B.\,S., Zhang~W.} Stochastic H2/H1 control with state-dependent noise~// IEEE 
T.~Automat. Contr., 2004. Vol.~49. No.\,1. P.~45--56. doi: 10.1109/TAC.2003.821400.
\bibitem{4-bos}
\Au{Bohacek S.} A~stochastic model of TCP and fair video transmission~// IEEE 
INFOCOM, 2003. Vol.~2. P.~1134--1144. doi: 10.1109/INFCOM.2003.1208950.
\bibitem{5-bos}
\Au{Домбровский В.\,В., Объедко~Т.\,Ю.} Управление с~прогнозированием системами 
с~марковскими скачками при ограничениях и~применение к~оптимизации 
инвестиционного портфеля~// Автомат. телемех., 2011. №\,5. С.~96--112. doi: 
10.1134/S0005117911050079.
\bibitem{6-bos}
\Au{Баландин Д.\,В., Коган~М.\,М.} Оптимальное линейно-квад\-ра\-тич\-ное управление: от 
матричных уравнений к~линейным матричным неравенствам~// Автомат. телемех., 2011. 
№\,11. С.~60--69. doi: 10.1134/ S0005117911110038.
\bibitem{7-bos}
\Au{Босов А.\,В.} Обобщенная задача распределения ресурсов программной системы~// 
Информатика и~её применения, 2014. Т.~8. Вып.~2. С.~39--47. doi: 
10.14357/19922264140204.
\bibitem{8-bos}
\Au{Босов А.\,В.} Управление линейным выходом дискретной стохастической системы по 
квадратичному критерию~// Изв. РАН. Теория и~системы управления, 2016. №\,3.  
С.~19--35. doi: 10.1134/S1064230716030060.
\bibitem{9-bos}
\Au{Флеминг У., Ришел~Р.} Оптимальное управление детерминированными 
и~стохастическими системами~/ Пер. с~англ.~--- М.: Мир, 1978. 316~с. 
(\Au{Fleming~W.\,H., Rishel~R.\,W.} Deterministic and stochastic optimal control.~--- New 
York, NY, USA: Springer-Verlag, 1975. 222~p.)
\bibitem{10-bos}
\Au{Девис М.\,Х.\,А.} Линейное оценивание и~стохастическое управление~/ Пер. с~англ.~--- 
М.: Наука, 1984. 206~с. (\Au{Davis~M.\,H.\,A.} Linear estimation and stochastic control.~--- 
London: Chapman and Hall, 1977. 224~p.)

 \end{thebibliography}

 }
 }

\end{multicols}

\vspace*{-6pt}

\hfill{\small\textit{Поступила в~редакцию 30.03.18}}

\vspace*{4pt}

%\newpage

%\vspace*{-24pt}

\hrule

\vspace*{2pt}

\hrule

\vspace*{-2pt}


\def\tit{STOCHASTIC DIFFERENTIAL SYSTEM OUTPUT CONTROL 
BY~THE~QUADRATIC CRITERION.~I.~DYNAMIC\\ PROGRAMMING 
OPTIMAL SOLUTION}


\def\titkol{Stochastic differential system output control 
by~the~quadratic criterion. I.~Dynamic programming 
optimal solution}

\def\aut{A.\,V.~Bosov and~A.\,I.~Stefanovich}

\def\autkol{A.\,V.~Bosov and~A.\,I.~Stefanovich}

\titel{\tit}{\aut}{\autkol}{\titkol}

\vspace*{-11pt}


\noindent
Institute of Informatics Problems, Federal Research Center ``Computer Science 
and Control'' of the Russian Academy of Sciences, 44-2~Vavilov Str., Moscow 
119333, Russian Federation


\def\leftfootline{\small{\textbf{\thepage}
\hfill INFORMATIKA I EE PRIMENENIYA~--- INFORMATICS AND
APPLICATIONS\ \ \ 2018\ \ \ volume~12\ \ \ issue\ 3}
}%
 \def\rightfootline{\small{INFORMATIKA I EE PRIMENENIYA~---
INFORMATICS AND APPLICATIONS\ \ \ 2018\ \ \ volume~12\ \ \ issue\ 3
\hfill \textbf{\thepage}}}

\vspace*{3pt}



\Abste{The problem of optimal control for the Ito diffusion 
process and a~controlled linear output is solved. The considered 
statement is close to the classical linear-quadratic Gaussian 
control  (LQG control) problem. Differences consist in the fact 
that the state is described by the nonlinear differential Ito equation  $dy_y = A_t(y_t) 
\,dt+\Sigma_t(y_t)\,dv_t$ and does not depend on the control~$u_t$, 
optimization subject is controlled linear output 
 $dz_t=a_ty_t\,dt +b_tz_t\,dt +c_t u_t\,dt +\sigma_t \,dw_t$. 
Additional generalizations are included in the quadratic 
quality criterion for the purpose of statement such problems 
as state tracking by output or a linear combination of state 
and output tracking by control. The method of dynamic programming 
is used for the solution. 
The assumption about Bellman function in the form  $V_t(y,z)= \alpha_t 
z^2+\beta_t(y) z+\gamma_t(y)$ allows one to find it. 
Three differential equations for the coefficients $\alpha_t$,  $\beta_t(y)$,
and $\gamma_t(y)$ give the solution. 
These equations constitute the optimal solution of the problem under consideration.}

\KWE{stochastic differential equation; optimal control; dynamic programming; 
Bellman function; Riccati equation; linear differential equations of parabolic type}


\DOI{10.14357/19922264180314}

\vspace*{-12pt}

\Ack
\noindent
This work was partially supported by the Russian Science Foundation (grant  
16-07-00677).



%\vspace*{6pt}

  \begin{multicols}{2}

\renewcommand{\bibname}{\protect\rmfamily References}
%\renewcommand{\bibname}{\large\protect\rm References}

{\small\frenchspacing
 {%\baselineskip=10.8pt
 \addcontentsline{toc}{section}{References}
 \begin{thebibliography}{99}
\bibitem{1-bos-1}
\Aue{Athans, M.} 1971. Editorial on the LQG problem. \textit{IEEE~T. 
Automat. Contr.} 16(6):528--552. doi: 10.1109/ TAC.1971.1099845.
\bibitem{2-bos-1}
\Aue{Wu, Z.} 2005. Forward-backward stochastic differential equations, linear 
quadratic stochastic optimal control and\linebreak\vspace*{-12pt}

\columnbreak

\noindent
 nonzero sum differential games. 
\textit{J.~Syst. Sci. Complex.} 18(2):179--192.
\bibitem{3-bos-1}
\Aue{Chen, B.\,S. and W.~Zhang.} 2004. Stochastic H2/H1 control with  
state-dependent noise. \textit{IEEE~T. Automat. Contr.} 49(1):45--56.
doi: 10.1109/TAC.2003.821400.
\bibitem{4-bos-1}
\Aue{Bohacek, S.} 2003. A~stochastic model of TCP and fair video 
transmission. \textit{IEEE INFOCOM}. 2:1134--1144.
doi: 10.1109/INFCOM.2003.1208950.
\bibitem{5-bos-1}
\Aue{Dombrovskii, V.\,V., and T.\,Yu.~Ob''edko.} 2011. Predictive control of 
systems with Markovian jumps under constraints and its application to the 
investment portfolio optimization. \textit{Automat. Rem. Contr.}  
72(5):989--1003.
\bibitem{6-bos-1}
\Aue{Balandin, D.\,V., and M.\,M.~Kogan.} 2011. Optimal linear-quadratic 
control: From matrix equations to linear matrix inequalities. \textit{Automat. 
Rem. Contr.} 72(11):2276--2284.
\bibitem{7-bos-1}
\Aue{Bosov, A.\,V.} 2014. Obobshchennaya zadacha raspredeleniya resursov 
programmnoy sistemy [The generalized problem of software system resources 
distribution]. \textit{Informatika i~ee Primeneniya~--- Inform. Appl.}  
8(2):39--47. doi: 
10.14357/19922264140204.
\bibitem{8-bos-1}
\Aue{Bosov, A.\,V.} 2016. Discrete stochastic system linear output control 
with respect to a quadratic criterion. \textit{J.~Comput. Syst. Sc. 
Int.} 55(3):349--364.
\bibitem{9-bos-1}
\Aue{Fleming, W.\,H., and R.\,W.~Rishel.} 1975. \textit{Deterministic and 
stochastic optimal control.} New York, NY: Springer-Verlag. 222~p.
\bibitem{10-bos-1}
\Aue{Davis, M.\,H.\,A.} 1977. \textit{Linear estimation and stochastic 
control.} London: Chapman and Hall. 224~p.
\end{thebibliography}

 }
 }

\end{multicols}

\vspace*{-6pt}

\hfill{\small\textit{Received March 30, 2018}}

%\pagebreak

%\vspace*{-18pt}
     
     \Contr
     
       \noindent
       \textbf{Bosov Alexey V.} (b.\ 1969)~--- Doctor of Science in technology, 
principal scientist, Institute of Informatics Problems, Federal Research 
Center ``Computer Science and Control'' of the Russian Academy of Sciences, 
44-2~Vavilov Str., Moscow 119333, Russian Federation; 
\mbox{AVBosov@ipiran.ru}
       
       \vspace*{3pt}
       
       \noindent
       \textbf{Stefanovich Alexey I.} (b.\ 1983)~--- principal specialist, 
Institute of Informatics Problems, Federal Research Center ``Computer Science 
and Control'' of the Russian Academy of Sciences, 44-2~Vavilov Str., Moscow 
119333, Russian Federation; \mbox{AStefanovich@frccsc.ru}
\label{end\stat}

\renewcommand{\bibname}{\protect\rm Литература}       

           %3
%\newcommand{\A}{{\mathbf A}}
%\newcommand{\B}{{\mathbf B}}
%\newcommand{\la}{{\lambda}}
%\newcommand{\be}{\begin{equation}}
%\newcommand{\ee}{\end{equation}}
%\newcommand{\ber}{\begin{eqnarray}}
%\newcommand{\eer}{\end{eqnarray}}

%\newcommand{\nin}{\noindent}
%\newcommand{\non}{\nonumber}
%\newcommand{\half}{\frac{1}{2}}
%\newcommand{\quarter}{\frac{1}{4}}

\def\stat{zeifman}

\def\tit{ОБ ОДНОМ КЛАССЕ МАРКОВСКИХ СИСТЕМ ОБСЛУЖИВАНИЯ$^*$}

\def\titkol{Об одном классе марковских систем обслуживания}

\def\autkol{Я.\,А.~Сатин, А.\,И.~Зейфман, А.\,В.~Коротышева, С.\,Я.~Шоргин}
\def\aut{Я.\,А.~Сатин$^1$, А.\,И.~Зейфман$^2$, А.\,В.~Коротышева$^3$, С.\,Я.~Шоргин$^4$}

\titel{\tit}{\aut}{\autkol}{\titkol}

{\renewcommand{\thefootnote}{\fnsymbol{footnote}}\footnotetext[1]
{Исследование поддержано РФФИ, гранты 11-07-00112-а и 11-01-12026-офи-м.}}


\renewcommand{\thefootnote}{\arabic{footnote}}
\footnotetext[1]{Вологодский государственный педагогический
университет, yacovi@mail.ru}
\footnotetext[2]{Вологодский государственный педагогический университет;  
Институт проблем информатики Российской академии наук; 
Институт социально-экономического развития территорий Российской академии наук,  a\_zeifman@mail.ru}
\footnotetext[3]{Вологодский государственный педагогический
университет,  a\_korotysheva@mail.ru}
\footnotetext[4]{Институт проблем информатики Российской академии наук, SShorgin@ipiran.ru}


\Abst{Рассматриваются модели обслуживания, описываемые конечными марковскими 
цепями с непрерывным временем. При этом предполагается,  что интенсивности 
поступления и обслуживания требований не зависят от числа требований в сис\-те\-ме. 
Получены оценки скорости сходимости и устойчивости различных характеристик таких сис\-тем.}

\KW{нестационарные марковские системы
обслуживания; скорость сходимости; устойчивость; оценки}

 \vskip 14pt plus 9pt minus 6pt

      \thispagestyle{headings}

      \begin{multicols}{2}
      
            \label{st\stat}

\section{Введение}

Классы систем массового обслуживания, описываемых процессами
рождения и гибели (стационарными и нестационарными, с катастрофами)
изучались начиная с 1970-х~гг.\ многими авторами
(см., например,~[1--6]). С~помощью методов,
разработанных одним из авторов настоящей \mbox{статьи}\linebreak (подробное изложение
этих методов приведено в~[7--9]), для таких сис\-тем
удалось получить точные оценки скорости сходимости и устойчивости.

Оказывается, этот же подход можно применить и к существенно более 
общему классу систем обслуживания.

Рассмотрим систему массового обслуживания, число требований в которой 
описывается нестационарной марковской цепью с непрерывным временем и 
конечным пространством состояний, причем требования могут поступать и 
обслуживаться группами.

Пусть $X=X(t)$, $t\geq 0$,~--- число требований в системе обслуживания ($0 \hm\le X(t) \hm\le r$).

Обозначим через 
\begin{gather*}
p_{ij}(s,t)=\mathrm{Pr}\left\{ X(t)=j\left| X(s)=i\right.
\right\}\,,\\
i,j \ge 0\,,\ 0\leq s\leq t\,,
\end{gather*}
переходные вероятности
процесса $X\hm=X(t)$, а через  $p_i(t)\hm=\mathrm{Pr}\left\{ X(t) \hm=i \right\}$~---
его вероятности состояний.

Будем предполагать, что интенсивности поступления и обслуживания $k$ требований в 
момент~$t$ в сис\-те\-ме об\-слу\-жи\-ва\-ния ($\lambda_{k}(t)$ и  $\mu_{k}(t)$ соответственно)  
не зависят от числа требований, находящихся в системе в момент~$t$, являются локально 
интегрируемыми на $[0,\infty)$ функциями времени~$t$ и, кроме того, 
$\lambda_{k+1}(t) \hm\le \lambda_{k}(t)$ и  $\mu_{k+1}(t) \hm\le \mu_{k}(t)$ при всех~$k$ 
и почти при всех $t \hm\ge 0$.

Тогда для описания вероятностной динамики процесса получаем прямую систему Колмогорова в виде
\begin{equation} 
\fr{d\vp}{dt}=A(t)\vp(t)\,,
\label{ur_1}
\end{equation}
 где
 {\footnotesize
\begin{multline*}
A(t)={}\\
{}=
\begin{pmatrix}
a_{00}(t) & \mu_1(t)  & \mu_2(t)   & \mu_3(t)  & \mu_4(t) & \cdots & \mu_r(t) \\
\la_1(t)   & a_{11}(t)  & \mu_1(t)  & \mu_2(t)   & \mu_3(t)  & \cdots & \mu_{r-1}(t) \\
\la_2(t)  & \la_1(t)    & a_{22}(t)& \mu_1(t)  & \mu2(t)    &  \cdots & \mu_{r-2}(t) \\
\cdots&\cdots&\cdots&\cdots&\cdots&\cdots&\cdots \\
\la_r(t) & \la_{r-1}(t) & \la_{r-2}(t) & \cdots & \la_2(t)  & \la_1 (t)   &  a_{rr}(t)
\end{pmatrix}\,,
\end{multline*}}
причем  
$$
a_{ii}(t)=-\sum\limits_{k=1}^{i}\mu_k(t) - \sum\limits_{k=1}^{r-i} \la_{r-k}(t)\,.
$$

Далее будем обозначать через $\|\bullet\|$  $l_1$-нор\-му, т.\,е.\ 
$\|{\vx}\|\hm=\sum|x_i|$, а $\|B\| \hm= \max\limits_j \sum\limits_i |b_{ij}|$, 
если $B \hm= (b_{ij})_{i,j=0}^{r}$.
%
Тогда, в частности, имеем 
$$
\|A(t)\| \le 2\sum\limits_{k=1}^{r}(\la_{k}(t)+ \mu_k(t))
$$ 
при  всех $t \hm\ge 0$.

Через 
$$
E(t,k) = E\left\{X(t)\left|X(0)\hm=k\right.\right\}
$$ 
будем далее обозначать математическое ожидание процесса (среднее число требований) в момент~$t$ 
при условии, что в нулевой момент времени он находится в состоянии~$k$, 
а через $E_{\bf p}(t)$ обозначим математическое ожидание процесса в момент~$t$ 
при начальном распределении вероятностей состояний $\mathbf{p}(0) \hm= \mathbf{p}$.

\section{Оценки скорости сходимости}

Рассмотрим вспомогательную последовательность положительных чисел $\{d_i\}$, $i\hm=1, \dots,r$.

Положим
\begin{equation*}
d=\min\limits_{1 \le i \le r} d_i\,; \enskip 
G=\sum\limits_{i=1}^r d_i\,; \enskip W=\min\limits_k \fr{d_k}{k}\,.
%\label{2.01}
\end{equation*}

Рассмотрим величины
\begin{multline*}
\alpha_i(t)= -a_{ii}(t)+\la_{r-i+1}(t)-\sum\limits_{k=1}^{i-1}(\mu_{i-k}(t)-{}\\
{}-
\mu_i(t))\fr{d_k}{d_i}-\sum\limits_{k=1}^{r-i}(\la_k(t)-\la_{i+r-1}(t))\fr{d_{k+i}}{d_i}\,,
%\label{2.02}
\end{multline*}

\noindent
\begin{equation*}
\alpha(t)=\min\limits_{1 \le i \le r}\alpha_i(t)\,.
%\label{2.03}
\end{equation*}

\smallskip

\noindent
\textbf{Теорема~1.} \textit{Пусть существует последовательность положительных 
чисел  $\{d_j\}$ такая, что}
\begin{equation}
\int\limits_0^{\infty} \alpha(t)\, dt = + \infty\,.
\label{2.031}
\end{equation}
\textit{Тогда $X(t)$ слабо эргодичен, при
любых начальных условиях} $\mathbf{p}^*(s)$, $\mathbf{p}^{**}(s)$ 
\textit{и любых $s$, $t$, $0\le s\le t$, справедлива оценка
\begin{equation} 
\label{2.04}
\|\vp^*(t)-\vp^{**}(t)\| \le \fr{8G}{d}\,e^{-\int\limits_s^t {\alpha(u)\,du}}\,.
\end{equation}
Кроме того,  $X(t)$ имеет предельное среднее $\phi(t)$ и при любых~$k$ и $t \hm\ge 0$ справедливо неравенство}:
\begin{equation}
\label{2.05}
|E(t,k)-\phi(t)|\le \fr{4G}{W}\,e^{-\int\limits_0^t {\alpha(u)\,du}}\,.
\end{equation}


\smallskip


\noindent
Д\,о\,к\,а\,з\,а\,т\,е\,л\,ь\,с\,т\,в\,о\,.\

Пользуясь предложенным в предыдущих работах способом, 
выразим 
$$
p_0=1-\sum\limits_{1\le i \le r}{p_i}\,.
$$

Тогда получим неоднородное уравнение:
\begin{equation} 
\label{ur_per}
\fr{d\vz}{dt}= B(t)\vz(t)+\vf(t)\,, 
%\label{2.06}
\end{equation}
\noindent
где $\vf(t)=\left(\la_1, \la_2,\cdots,\la_r \right)^{\mathrm{T}}$;

\end{multicols}


\hrule

\vspace*{6pt}

\begin{equation*}
B = \left(
\begin{array}{cccccccc}
a_{11}- \la_1   & \mu_1 - \la_1   & \mu_2 - \la_1   & \mu_3 -\la_1   & \cdots& \cdots & \mu_{r-1}- \la_1  \\
\la_1 -\la_2    & a_{22} -\la_2  & \mu_1-\la_2   & \mu_2 -\la_2     & \cdots&  \cdots & \mu_{r-2} -\la_2 \\
\la_2 -\la_3    & \la_1 -\la_3   & a_{33} -\la_3  & \mu_1-\la_2   & \cdots&  \cdots & \mu_{r-3} -\la_3 \\
\cdots&\cdots&\cdots&\cdots&\cdots&\cdots&\cdots \\
\la_{r-1} -\la_r  &\la_{r-2} -\la_r & \cdots & \cdots & \la_2 -\la_r   & \la_1 -\la_r     &  a_{rr} -\la_r
\end{array}
\right)\,.
%\label{2.07}
\end{equation*}

Рассмотрим треугольную матрицу
\begin{equation*}
D=\begin{pmatrix}
d_1   & d_1 & d_1 & \cdots & d_1 \\
0   & d_2  & d_2  &   \cdots & d_2 \\
\cdots&\cdots&\cdots&\cdots&\cdots \\
0  & 0 & \cdots & 0 &  d_r
\end{pmatrix}
%\label{2.08}
\end{equation*}
и соответствующую норму $\|{\bf z}\|_{D}\hm=\|D {\bf z}\|_1$.

Тогда имеем:
\begin{equation*}
 D BD^{-1}=\left(
\begin{array}{ccccccc}
a_{11}-\la_r  &  (\mu_1-\mu_2) \fr{d_1}{d_2}  & (\mu_2-\mu_3)\fr{d_1}{d_3}  & \cdots &  (\mu_{r-1}-\mu_r)\fr{d_1}{d_r} \\
(\la_1-\la_r) \fr{d_2}{d_1} &  a_{22}-\la_{r-1}  &(\mu_1-\mu_3)\fr{d_2}{d_3}  & \cdots &  (\mu_{r-2}-\mu_r)\fr{d_2}{d_r} \\
(\la_2-\la_r) \fr{d_3}{d_1} &  (\la_1-\la_{r-1})\fr{d_3}{d_2}   &a_{33}-\la_{r-2}   & \cdots &  (\mu_{r-3}-\mu_r)
\fr{d_3}{d_r}  \\
\cdots&\cdots&\cdots&\cdots&\cdots \\
(\la_{r-1} -\la_r) \fr{d_r}{d_1} & (\la_{r-2} -\la_{r-1}) \fr{d_r}{d_2}  & (\la_{r-3} -\la_{r-2}) \fr{d_r}{d_3}  & \cdots & a_{rr}-\la_1 \\
\end{array}
\right)\,.
%\label{2.09}
\end{equation*}


\begin{multicols}{2}


Далее, оценивая логарифмическую норму оператора~$B(t)$ (см., например, 
подробное рассмотрение в~[8--10]), получаем
\begin{multline*}
\gamma \left(B(t)\right)_{1D} = \gamma \left(DB(t)D^{-1}\right)_{1}={}\\
{}=
\max \left(\vphantom{\sum\limits_{k=1}^{i-1}}
a_{ii}(t) - \la_{r-i+1}(t) + \sum\limits_{k=1}^{i-1}\left(\mu_{i-k}(t)-{}\right.\right.\\
\left.\left.{}-\mu_i(t)\right)
\fr{d_k}{d_i} +
\sum\limits_{k=1}^{r-i}(\la_k(t)-\la_{i+r-1}(t))\fr{d_{k+i}}{d_i}\right) ={}\\
{}=
 - \min \alpha_i(t) = - \alpha(t)\,.
% \label{2.10}
\end{multline*}
Тогда\\[-7.9pt]
\begin{equation*}
\|\vz^*(t)-\vz^{**}(t)\|_{1D}\le  e^{-\int\limits_s^t {\alpha(u)du}}\|\vz^*(s)-\vz^{**}(s)\|_{1D}
%\label{2.11}
\end{equation*}
для всех $0 \le s \le t$ и любых начальных условий $\vz^*(s)$, $\vz^{**}(s)$.

Теперь, учитывая оценки для сравнения норм (см., например,~\cite{z08b}), получаем:
\begin{multline*}
\|\vp^*(t)-\vp^{**}(t)\| \le 2\|\vz^*(t)-\vz^{**}(t)\| \le{}\\
{}\le  \fr{4}{d}\|\vz^*(t)-\vz^{**}(t)\|_{1D}\le{} \\
{} \le \fr{4}{d}\,e^{-\int\limits_s^t {\alpha(u)\,du}}\|\vz^*(s)-\vz^{**}(s)\|_{1D} 
\le{}\\
{}\le
 \fr{4G}{d}\,e^{-\int\limits_s^t {\alpha(u)\,du}}\|\vz^*(s)-\vz^{**}(s)\| \le{} \\
{} \le  \fr{4G}{d}\,e^{-\int\limits_s^t {\alpha(u)\,du}}\|\vp^*(s)-\vp^{**}(s)\| \le 
\fr{8G}{d}\,e^{-\int\limits_s^t {\alpha(u)\,du}} 
%\label{2.11-a}
\end{multline*}
для любых начальных условий ${\bf p^*}(s)$, ${\bf p^{**}}(s)$ и любых $s,t$, $0\hm\le s\hm\le t$.

Из слабой эргодичности процесса с конечным пространством состояний 
вытекает существование предельного среднего, начальные условия для которого можно 
в общем случае выбрать произвольно.
Для оценки средних воспользуемся неравенством, приведенным в параграфе~2.3 из~\cite{z08b}:
\begin{multline*}
\|{\bf z}\|_{1D} = d_0 \left|\sum\limits_{i=1}^{\infty} p_i \right|
+ d_1 \left|\sum\limits_{i=2}^{\infty} p_i \right| + \dots \ge{}\\
{}\ge 
 W \sum\limits_{k \ge 1} k \left|\sum\limits_{i \ge k} p_i\right| \ge \fr{W}{2}
\sum\limits_{k \ge 1} k \left|p_k\right|\,.  
%\label{2.12}
\end{multline*}
Получаем теперь
\begin{multline*}
|E(t,k)-\phi(t)|\le \fr{2}{W}\,\|\vp^*(t)-\vp^{**}(t)\|_{1D}\le {} \\
{}\le\fr{2}{W}\,e^{-\int\limits_0^t {\alpha(u)\,du}}\|{\bf e}_k -
\vp^{**}(0)\|_{1D} \le \frac{4G}{W}e^{-\int\limits_0^t
{\alpha(u)\,du}}\,,
%\label{2.13}
\end{multline*}
что и требовалось доказать.
\columnbreak

%\smallskip

\noindent
\textbf{Замечание~1.} {Положим в условиях теоремы~1 
$$
\beta(t)=\max\limits_{1 \le i \le r}\alpha_i(t)\,.
$$ 
Тогда, пользуясь внедиагональной неотрицательностью матрицы $DB(t)D^{-1}$ 
с помощью методики, описанной в~\cite{z08b, z95b}, получаем справедливость неравенства

\noindent
\begin{equation*} 
%\label{2.14}
\|\vp^*(t)-\vp^{**}(t)\| \ge \fr{d}{8G}\,e^{-\int\limits_s^t {\beta(u)\,du}}
\end{equation*}
при любых $s$, $t$, $0\le s\le t$ и уже не при любых начальных условиях~${\bf p^*}(s)$, 
${\bf p^{**}}(s)$, а таких, что  $D\left({\bf p^*}(s) \hm-{\bf p^{**}}(s)\right) \hm\ge 0.$ 
Следовательно, оценки тео\-ре\-мы~1 будут заведомо иметь точный по времени порядок, если удастся 
выбрать вспомогательную последовательность $\{d_i\}$ так, что $\alpha(t)\hm=\beta(t)$, т.\,е.\ 
все $\alpha_i(t)$ одинаковы (не зависят от индекса~$i$)}.



\smallskip

Введем теперь в рассмотрение величины

\vspace*{-1pt}

\noindent
\begin{multline*}
\zeta_i(t)= -a_{ii}(t)+\la_{r-i+1}(t)+{}\\
{}+\sum\limits_{k=1}^{i-1}\left(\mu_{i-k}(t)-
\mu_i(t)\right) \fr{d_k}{d_i}+{}\\
{}+\sum\limits_{k=1}^{r-i}\left(\la_k(t)-\la_{i+r-1}(t)\right)\fr{d_{k+i}}{d_i}\,;
%\label{2.0211}
\end{multline*}
\begin{equation*}
\chi(t)=\max\limits_{1 \le i \le r}\zeta_i(t)\,.
%\label{2.0311}
\end{equation*}

\noindent
\textbf{Замечание 2.} {В условиях теоремы~1 при любых начальных условиях 
${\bf p^*}(s)$, ${\bf p^{**}}(s)$ и любых $s,t$,  $0\le s\le t$, 
справедлива следующая двухсторонняя оценка скорости сходимости:

\vspace*{-1pt}

\noindent
\begin{multline*} 
%\label{2.041}
\!\!\!\fr{d}{4G}\,e^{-\int\limits_s^t {\chi(u)\,du}}\|\vp^*(s)-\vp^{**}(s)\| \le
 \|\vp^*(t)-\vp^{**}(t)\| \le {}\\
 {}\le\fr{4G}{d}\,e^{-\int\limits_s^t {\alpha(u)\,du}}\|\vp^*(s)-\vp^{**}(s)\|.
\end{multline*}
Таким образом, можно оценить и сверху и снизу время  вхождения 
сис\-те\-мы обслуживания в предельный режим. Более подробно о получении 
нижних оценок см., например, в~\cite{z95b, gz05}.}

\smallskip

Рассмотрим два частных случая теоремы.

\smallskip

\noindent
\textbf{Следствие 1}. \textit{Пусть при выполнении остальных условий теоремы~1 
вместо}~(\ref{2.031}) \textit{выполняется условие $\alpha(t) \hm\ge \alpha \hm> 0$ 
почти при всех $t \hm\ge 0$. Тогда вместо}~(\ref{2.04}) \textit{и}~(\ref{2.05}) 
\textit{справедливы оценки}:

\vspace*{-1pt}

\noindent
\begin{align*} 
%\label{2.15}
\|\vp^*(t)-\vp^{**}(t)\| &\le \fr{8G}{d}\,e^{-\alpha \left(t-s\right)}\,;
\\
%\label{2.16}
|E(t,k)-\phi(t)|&\le \fr{4G}{W}\,e^{- \alpha t}\,.
\end{align*}

\pagebreak

%\smallskip

Положим 
\begin{gather*}
M_0=\max\limits_{|t-s|\le 1}\int\limits_s^t \alpha(u)\,du;\\
\alpha^* = \int\limits_0^1 \alpha(t)\, dt\,; \quad
M=e^{M_0+\alpha^*}\,.
\end{gather*}
С учетом неравенства 
$$
e^{-\int\limits_s^t {\alpha(u)\,du}} \hm\le M e^{-\alpha^* (t-s)}
$$ 
получаем следующее утверждение.

\smallskip

\noindent
\textbf{Следствие~2.} \textit{Пусть все $\lambda_k(t)$ и $\mu_k(t)$ 1-пе\-ри\-одич\-ны,  
а при выполнении остальных условий теоремы~1 вместо}~(\ref{2.031}) 
\textit{выполняется условие  $\alpha^* \hm> 0$.  Тогда предельный режим (скажем, $\vp^*(t)$) 
и соответствующее ему предельное среднее $\phi^*(t)$ можно выбрать 
1-пе\-ри\-оди\-че\-ски\-ми, а вместо}~(\ref{2.04}) \textit{и}~(\ref{2.05}) 
\textit{справедливы оценки}:
\begin{equation*} 
%\label{2.17}
\|\vp(t) - \vp^*(t)\| \le \fr{8GM}{d}\,e^{-\alpha^*t}
\end{equation*}
\textit{и, кроме того,}
\begin{equation*}
|E(t,k)-\phi^*(t)|\le \fr{4GM}{W}\,e^{-\alpha^*t}
%\label{2.18}
\end{equation*}
\textit{при любом $k$ и $t \ge 0$}.



\section{Устойчивость}

Рассмотрим также <<возмущенный>> процесс обслуживания $\bar{X}\hm=\bar{X}(t)$, $t\hm\geq 0$, 
в котором интенсивности поступления и обслуживания требований также не зависят от чис\-ла 
требований в системе, обозначая его соответствующие характеристики теми же буквами с 
чертой сверху. Для прос\-то\-ты записи оценок будем предполагать, что возмущения 
<<равномерно малы>>, т.\,е.\ выполняется неравенство $\| A(t)-\bar{A}(t)\| \hm\le \varepsilon$. 
Первые результаты для нестационарных цепей с непрерывным временем получены в~\cite{z85}, 
а детальное рассмотрение для более общего случая неравномерных оценок можно без труда 
провести так же, как это сделано в~\cite{z98, ae}. Для получения требуемых равномерных 
оценок устойчивости необходима экспоненциальная эргодичность соответствующего процесса, 
т.\,е.\ существование положительных констант $N$, $a$ таких, что  для правой части~(\ref{2.04}) 
справедливо неравенство:
\begin{equation}
e^{-\int\limits_s^t {\alpha(u)\,du}} \le Ne^{-a\left(t-s\right)}\,.
\label{3.01}
\end{equation}
Оценка~(\ref{3.01}) заведомо имеет место, в частности, если выполнены условия одного из следствий 
предыду\-ще\-го параграфа.

\smallskip

\noindent
\textbf{Теорема~2.}
\textit{Пусть выполнены условия теоремы~1 и}~(\ref{3.01}). \textit{Тогда при
 любых начальных условиях ${\bf p}(s)$ и ${\bar{\bf p}}(s)$ для процессов~$X(t)$ 
 и $\bar{X}(t)$ соответственно справедливы следующие оценки устойчивости:}
\begin{align*} 
%\label{3.02}
\limsup_{t \to \infty}  \|{\bf p}(t)- \bar{\bf p}(t)\| &\le
\fr{\varepsilon(1+\ln(4GN/d))}{a}\,;
\\
% \label{3.03}
\limsup\limits_{t \to \infty}   |E_{\bf p}(t)- \bar{E}_{\bar{\bf p}(t)}|&\le 
\fr{r \varepsilon(1+\ln(4GN/d))}{a}\,.
\end{align*}


\smallskip

\noindent
Д\,о\,к\,а\,з\,а\,т\,е\,л\,ь\,с\,т\,в\,о\ основано на подходе, 
введенном для стационарных процессов в~\cite{mit03} и описанном для нестационарной 
ситуации в~\cite{z11}.
Если  при любых начальных условиях для исходного процесса справедлива оценка
\begin{equation*} 
%\label{3.04}
\|\vp(t) - \vp^*(t)\| \le ce^{-b\left(t-s\right)}\,,
\end{equation*}
то, полагая
\begin{multline*}
\beta (t, s)=\sup\limits_{ \| {\bf v} \| =1, \sum {v_i}=0}
{\|V(t,s){\bf v}(t,s)\|} ={}\\
{}= \fr{1}{2} \max_{i,j} \sum\limits_k {|p_{ik}(t,
s)-p_{jk}(t, s)|}\,, 
\end{multline*}
где $V(t, s)$~--- матрица Коши
уравнения~(\ref{ur_1}), получаем в итоге следующее неравенство:
\begin{equation*}
\|{\bf p}(t)-\bar{\bf p}(t)\| \le{}
\begin{cases}
\|{\bf p}(s)-{\bf \bar{p}}(s)\|+ (t-s)\varepsilon \,, &\\
&\hspace*{-35mm} 0<t< b^{-1} \ln \left(\fr{c}{2}\right)\,; \\
b^{-1}\left(\ln \fr{c}{2} +1-\fr{c}{2}\,e^{-b(t-s)}\right)\varepsilon +{}&\\
{}+
\fr{c}{2}\,e^{-b(t-s)} \|{\bf p}(s)-{\bf \bar{p}}(s)\|\,, &\\
&\hspace*{-30mm}t\ge b^{-1}\ln \left(\fr{c}{2}\right)
\end{cases}
%\label{3.05}
\end{equation*}
для любых начальных условий ${\bf p}(s)$ и $\bar{\bf p}(s)$.
Из неравенств~(\ref{2.04}) и~(\ref{3.01}) вытекает, что $b=a$, $c={8GN}/{d}$.  
Устремив $t \hm\to \infty$ и взяв $s\hm=0$, получаем требуемые оценки.


\smallskip

\noindent
\textbf{Замечание~3.} 
В полученную оценку устойчивости для математического ожидания процесса 
в качестве множителя входит размерность~$r$, поэтому иногда лучший результат 
удается получить при помощи другого подхода, описанного в работе~\cite{z11}.

\smallskip

Положим 
$$
S=\max\limits_{{1 \le i, j \le r}} \fr{d_i}{d_j}\,,
$$ 
и пусть числа $K, L$ таковы, что 

\noindent
$$
d_1\la_1(t) + (d_1+d_2)\la_2(t) + \dots + 
\left(\sum\limits_{1 \le i \le r}d_i\right) \la_r(t) \le K\,,
$$ 
а 

\noindent
\begin{multline*}
d_1(\la_1(t)-\bar{\la}_1(t)) + (d_1+d_2)(\la_2(t)-\bar{\la}_2(t)) + \dots\\
\dots + 
\left(\sum\limits_{1 \le i \le r}d_i\right) (\la_r(t)-\bar{\la}_r(t)) \le 
L\varepsilon
\end{multline*} 
почти при всех $t \ge 0.$

\smallskip

\noindent
\textbf{Теорема~3.}
\textit{Пусть  выполнены условия теоремы~2 и, кроме того, при всех~$k$ 
и почти всех $t \hm\ge 0$ $\la_k(t) \hm< \infty$. Тогда при любых начальных условиях 
${\bf p}(s)$ и ${\bar{\bf p}}(s)$ для процессов $X(t)$ и $\bar{X}(t)$ 
соответственно справедливо неравенство}

\noindent
\begin{equation*}
\limsup\limits_{t \to \infty}   |E_{\bf p}(t)- \bar{E}_{\bar{\bf p}(t)}|\le 
\fr{ N\varepsilon\left(L a+ 2KNS\right)}{W a \left(a-2\varepsilon S\right)}\,.
\end{equation*}


\smallskip

\noindent
Д\,о\,к\,а\,з\,а\,т\,е\,л\,ь\,с\,т\,в\,о.\
 Перепишем исходную систему~(\ref{ur_per}) для невозмущенного процесса в следующем виде:
 \noindent
 
\begin{equation*}
\fr{d\vp}{dt}=\bar{B}(t)\vp(t) + {\bf f}(t)+\left(B(t)-\bar{B}(t)\right)\vp(t)\,.
%\label{eq112-n}
\end{equation*}
Тогда

\noindent
\begin{multline*}
\vp(t)=\bar{U}(t,0)\vp(0)+\int\limits_0^t \bar{U}(t,\tau){\bf{f}}(\tau) \, d\tau+{}\\
{}+\int\limits_0^t \bar{U}(t,\tau) \left(B(\tau)-\bar{B}(\tau)\right)\vp(\tau)\, d\tau\,;
\end{multline*}

\vspace*{-9pt}

\begin{equation*}
\hspace*{-15mm}\bar{\vp}(t)=\bar{U}(t,0)\bar{\vp}(0)+\int\limits_0^t \bar{U}(t,\tau){\bf{f}}(\tau) \, d\tau,
\end{equation*}
где $U(t,s)$~--- матрица Коши для уравнения~(\ref{ur_per}).
В любой норме при одинаковых начальных условиях получаем следующую оценку:
%\noindent
\begin{multline}
 \label{3000}
\!\!\!\!\!\!\left\|\vp(t)-\bar{\vp}(t)\right\|\le \!\!\int\limits_0^t \!\!\|\bar{U}(t,\tau)\|
\left(\| B(\tau)-\bar{B}(\tau)\| \|\vp(\tau)\| +\right.\\
\left.{}+ \| \vf(\tau)-\bar{\vf}(\tau)\|\right)\,d\tau\,.\!
\end{multline}
Имеем почти при всех $t \ge 0$:
\begin{equation*}
\|B(t)-\bar{B}(t)\|_{1D}=\|D(B(t)-\bar{B}(t))D^{-1}\| \le 2S\varepsilon\,;
%\label{3002}
\end{equation*}
%
%\vspace*{-14pt}
%
%\noindent
\begin{multline*}
\|{\bf f}(t)\|_{1D} \le d_1\la_1(t) + (d_1+d_2)\la_2(t) + \dots + {}\\
{}+
\left(\sum\limits_{1 \le i \le r}d_i\right) \la_r(t) \le K\,, 
\quad \|\vf(\tau)-\bar{\vf}(\tau)\|_{1D} \le L\varepsilon\,.
%\label{3002-a}
\end{multline*}
А тогда
\begin{multline*}
\gamma(\bar{B}(t))_{1D} \le \gamma(DB(t)D^{-1})+\|B(t)-\bar{B}(t)\|_{1D} \le  {}\\
{}\le -
\alpha(t)+2S \varepsilon \,.
% \label{3003}
\end{multline*}

Оценим теперь
\begin{multline*} 
%\label{8402}
\!\|{\bf p}(t)\|_{1D} \le
\|U(t){\bf p}(0) \|_{1D} +
 \int\limits_0^t \!\!\| U(t,\tau){\bf f}(\tau)\, d\tau \|_{1D} \le {}\\
 {}\le
 N e^{-a t} \| \vp(0)\|_{1D}  + \fr{K N}{a}.
\end{multline*}

 Тогда с учетом~(\ref{3000}) получаем:
\begin{multline*} 
%\label{3004}
\left\|\vp(t)-\bar{\vp}(t)\right\|_{1D}\le N\int\limits_0^t e^{-(a - 2\varepsilon S)(t-\tau)}\times{}\\
{}\times
\left(2S\varepsilon (N e^{-a \tau} \| \vp(0)\|_{1D}  + \fr{K N}{a}) +  L\varepsilon \right)\, d\tau  \le {} \\
{}\le  o(1)+\fr{ N\varepsilon(L+{2KNS}/{a})}{a-2\varepsilon S}\,. 
\end{multline*}

\vspace*{-9pt}

\section{Примеры}

\noindent
\textbf{Пример 1.}

Рассмотрим исходный процесс обслуживания с интенсивностями 
$\la_1(t)\hm=\la_2(t)\hm=\la_3(t)\hm=\la(t) \hm= 3\hm+\sin{2\pi t}$, 
$\mu_1(t)\hm=\mu_2(t)\hm=  \mu(t) \hm= 2\hm+\cos{2\pi t}$, 
$\la_4(t)=\ldots=\la_r(t)\hm=\mu_3(t)=\ldots=\mu_r(t)\hm=0$. Выберем последовательность  
$d_k\hm=h^k$, где $0{,}82 \hm< h \hm<1$. Тогда имеем
$$
d=h^r\,; \quad G \le \fr{h}{1-h}\,; \quad W=\fr{h^r}{r}\,.
$$

Будем предполагать, что возмущенный процесс имеет такую же структуру 
мат\-ри\-цы интенсивностей, причем $|\la(t)\hm-\bar{\la}(t)| \hm\le \varepsilon$ 
и  $|\mu(t)\hm-\bar{\mu}(t)| \hm\le \varepsilon$ почти при всех $t \hm\ge 0$. 
Отметим кстати, что при этом $\| A(t)\hm-\bar{A}(t)\| \hm\le 10 \varepsilon$ почти при 
всех $t \hm\ge 0$. Рассмотрим дальнейшие оценки:
$$
S=\fr{1}{h^2}\,; \ K=4 \left(3h+2h^2+h^3\right)\,; \ L=3h+2h^2+h^3\,;
$$
$$
\alpha(t) \ge \la(t)\left(3 - h - h^2 -h^3\right)-\mu(t)\left(\fr{1}{h^2}+\fr{1}{h}-2\right)\,;
$$
$$
\alpha^*= 3\left(3 - h - h^2 -h^3\right)-2\left(\fr{1}{h^2}+\fr{1}{h}-2\right)\,;
$$


\noindent
\begin{multline*}
M_0 \le \int\limits_0^1 |\alpha(t)|\, dt \le 4\left(3 - h - h^2 -h^3\right)+{}\\
{}+
3\left(\fr{1}{h^2}+\fr{1}{h}-2\right)\,;
\end{multline*}

\vspace*{-9pt}

\noindent
$$
M=e^{\alpha^*+M_0}\,.
$$

Если, например, взять 
$h\hm=0{,}9$, то $\alpha^*\hm=0{,}992$, $M_0\hm=3{,}281$, $M\hm=71{,}737$.

Тогда получаем следующие оценки.

По следствию~2
\begin{align*}
 \|{\bf p}(t)- {\bf p^{*}}(t)\| &\le \fr{8Me^{-\alpha^*t}}{h^{r-1}(1-h)}\,;\\
|E_{\bf p}(t)-\phi^*(t)| &\le  \fr{4Mre^{-\alpha^*t}}{h^{r-1}(1-h)}\,.
\end{align*}

По теореме~2 ($N=M$, $a=\alpha^*$) с использованием оценок следствия~2
\begin{align*}
\limsup\limits_{t \to \infty} \|{\bf p}(t)- \bar{\bf p}(t)\| &\le{} \notag\\
&\hspace*{-15mm}{}\le \fr{\varepsilon(1+\ln({4M}/({h^{r-1}(1-h)})))}{\alpha^*}\,;\\
\limsup\limits_{t \to \infty}   |E_{\bf p}(t)- \bar{E}_{\bar{\bf p}(t)}| &\le \notag\\
&\hspace*{-15mm}{}\le\fr{r\varepsilon(1+\ln(4M/(h^{r-1}(1-h))))}{\alpha^*}\,.
\end{align*}

По теореме~3 с использованием оценок следствия~2
\begin{multline*}
\limsup\limits_{t \to \infty}   |E_{\bf p}(t)- \bar{E}_{\bar{\bf p}(t)}| \le {}\\
{}\le
\fr{rM\varepsilon(3h+2h^2+h^3)(\alpha^* h^2+8M)}{h^r\alpha^*(\alpha^* h^2-2\varepsilon)}\,.
\end{multline*}

\noindent

\textbf{Пример 2.}

Рассмотрим процесс с интенсивностями 
$\la_1(t)\hm=\la_2(t)\hm=\ldots=\la_r(t) \hm= \la(t) \hm= 3\hm+\sin{2\pi t}$; 
$\mu_1(t)\hm=\mu_2(t)\hm= \mu(t) \hm= 2+\cos{2\pi t}$;
$\mu_3(t)=\ldots=\mu_r(t)=0$.

Будем предполагать, что возмущенный процесс имеет такую же структуру 
мат\-ри\-цы интен\-сив\-ностей, причем $|\la(t)-\bar{\la}(t)| \hm\le \varepsilon$ и  
$|\mu(t)-\bar{\mu}(t)| \hm\le \varepsilon$ почти при всех $t \hm\ge 0$. 
При этом будем иметь $\| A(t)\hm-\bar{A}(t)\| \hm\le 2r \varepsilon$ почти при всех $t \hm\ge 0$.

Выберем последовательность $d_k\hm=1$. Тогда  
\begin{gather*}
d=1\,; \enskip G=r\,; \enskip W=\fr{1}{r}\,; \enskip S=1\,; \\
K=\fr{4r(1+r)}{2}\,; \quad L=\fr{r(1+r)}{2}\,;
\\
\alpha(t)=\la(t)\,; \ \alpha=2\,; \ \alpha^*=3\,; M_0 \le 4\,; \ M \le  e^{7}\,.
\end{gather*}

И получаем следующие оценки.

\columnbreak

По следствию~1
\begin{align*}
 \|{\bf p^*}(t)- {\bf p^{**}}(t)\| &\le 8re^{-2t}\,;\\
|E_{\bf p}(t)- \phi(t)|&\le  4r^2 e^{-2t}\,.
\end{align*}

По следствию~2
\begin{align*}
\|{\bf p}(t)- {\bf p^{*}}(t)\| &\le 8re^{7-3t}\,;
\\[6pt]
|E_{\bf p}(t)- \phi^*(t)| &\le 4r^2 e^{7-3t}\,.
\end{align*}

По теореме~2 ($N=1$, $a=\alpha$) с учетом оценок следствия~1
\begin{align*}
\limsup\limits_{t \to \infty} \|{\bf p}(t)- \bar{\bf p}(t)\| &\le 
\fr{\varepsilon(1+\ln{4r})}{2}\,;
\\[6pt]
\limsup\limits_{t \to \infty}   |E_{\bf p}(t)- \bar{E}_{\bar{\bf p}(t)}|
&\le \fr{r\varepsilon(1+\ln{4r})}{2}\,.
\end{align*}

По теореме~2 ($N=M$, $a=\alpha^*$) с учетом оценок следствия~2
\begin{align*}
\limsup\limits_{t \to \infty} \|{\bf p}(t)- \bar{\bf p}(t)\| &\le 
\fr{\varepsilon(8+\ln{4r})}{3}\,;
\\
\limsup\limits_{t \to \infty}   \left|E_{\bf p}(t)- \bar{E}_{\bar{\bf p}(t)}\right| &\le 
\fr{r\varepsilon(8 + \ln{4r})}{3}\,.
\end{align*}

По теореме~3 с учетом оценок следствия~1
\begin{equation*}
\limsup\limits_{t \to \infty}   \left|E_{\bf p}(t)- \bar{E}_{\bar{\bf p}(t)}\right| \le 
\fr{5 \varepsilon r^2 (1+r)}{4(1- \varepsilon)}\,.
\end{equation*}

По теореме~3 с учетом оценок следствия~2
\begin{equation*}
\limsup\limits_{t \to \infty}   \left|E_{\bf p}(t)- \bar{E}_{\bar{\bf p}(t)}\right| \le 
\fr{\varepsilon e^{7} r^2 (1+r) (3+8e^{7})}{6(3-2\varepsilon)}\,.
\end{equation*}

{\small\frenchspacing
{%\baselineskip=10.8pt
\addcontentsline{toc}{section}{Литература}
\begin{thebibliography}{99}

 \bibitem{b} %1
\Au{Баруча-Рид~А.\,Т.} Элементы теории марковских процессов и их
приложения.~--- М.: Наука, 1969.

\bibitem{gm}  %2
\Au{Гнеденко~Б.\,В., Макаров~И.\,П.} Свойства решений задачи с потерями
в случае периодических интенсивностей~// Дифф. уравнения, 1971.
Вып.~7. С.~1696--1698.

\bibitem{g1}   %3
\Au{Gnedenko~D.\,B.} On a generalization of Erlang formulae~// 
Zastosow. Mat., 1971. Vol.~12. P.~239--242.

\bibitem{S}  %4
\Au{Саати~Т.\,Л.} Элементы теории массового обслуживания
 и ее приложения.~--- М.: Сов. радио, 1971.

\bibitem{g}  %5
\Au{Gnedenko~B., Soloviev~A.} On the conditions of the
existence of final probabilities for a Markov process~// Math.
Operations. Stat., 1973. P.~379--390.

\bibitem{gk} %6
\Au{Гнеденко~Б.\,В., Коваленко~И.\,Н.} Введение в теорию массового
обслуживания.~--- М.: Наука, 1987.
\pagebreak

\bibitem{gz00}   %7
\Au{Granovsky~B.\,L., Zeifman~A.\,I.}  The N-limit of spectral gap of 
a class of birth-death Markov chains~//
 Appl. Stoch. Models Business Ind., 2000. Vol.~16. P.~235--248.

\bibitem{z08b}  %8
\Au{Зейфман~А.\,И., Бенинг~В.\,Е., Соколов~И.\,А.} 
Марковские цепи и модели с непрерывным временем.~--- М.: Элекс-КМ, 2008.

\bibitem{dzp} %9
\Au{Van Doorn~E.\,A., Zeifman~A.\,I., Panfilova~T.\,L.}  
Bounds and asymptotics for the rate of convergence of birth-death processes~//  
Th. Prob. Appl., 2010. Vol.~54. P.~97--113.

\bibitem{z95b}   %10
\Au{Zeifman~A.\,I.} Upper and lower bounds on the rate of
convergence for nonhomogeneous birth and death processes~//  Stoch.
Proc. Appl., 1995. Vol.~59. P.~157--173.

\bibitem{gz05}  %11
\Au{Granovsky~B.\,L., Zeifman~A.\,I.} On the lower bound of the spectrum
 of some mean-field models~// Theory Prob. Appl., 2005. Vol.~49. P.~148--155.
 
\bibitem{z85}  %12
\Au{Zeifman~A.\,I.} Stability for contionuous-time
nonhomogeneous Markov chains~// Lect. Notes Math.,  1985. Vol.~1155.
P.~401--414.

\bibitem{z98} %13
\Au{Zeifman~A.} Stability of birth and death processes~// 
J.~Math. Sci., 1998. Vol.~91. P.~3023--3031.

\bibitem{ae} %14
\Au{Андреев~Д., Елесин~М., Кузнецов~А., Крылов~Е., Зейфман~А.}
Эргодичность и устойчивость нестационарных систем обслуживания~//
Теория вероятностей и математическая статистика, 2003. Т.~68.
С.~1--11.

\bibitem{mit03} %15
\Au{Mitrophanov~A.\,Yu.} Stability and exponential convergence of continuous-time 
Markov chains~//  J. Appl. Prob., 2003. Vol.~40. P.~970--979.

\label{end\stat} 

\bibitem{z11} %16
\Au{Зейфман~А.\,И., Коротышева~А.\,В., Панфилова~Т.\,Л., Шоргин~С.\,Я.} 
Оценки устойчивости  для некоторых систем обслуживания с катастрофами~//  
Информатика и её применения, 2011. Т.~5. Вып.~3. С.~27--33.
 \end{thebibliography}
}
}


\end{multicols}          %4
\def\stat{torshin}

\def\tit{О ПОРОЖДЕНИИ СИНТЕТИЧЕСКИХ ПРИЗНАКОВ НА~ОСНОВЕ~ОПОРНЫХ ЦЕПЕЙ 
И~ПРОИЗВОЛЬНЫХ МЕТРИК В~РАМКАХ~ТОПОЛОГИЧЕСКОГО ПОДХОДА 
К~АНАЛИЗУ ДАННЫХ.\\ ЧАСТЬ~2.~ЭКСПЕРИМЕНТАЛЬНАЯ АПРОБАЦИЯ\\ НА~ЗАДАЧАХ ФАРМАКОИНФОРМАТИКИ$^*$}

\def\titkol{О порождении синтетических признаков на основе опорных цепей 
и~произвольных метрик} % в~рамках топологического подхода  к~анализу данных. Часть~2. Экспериментальная апробация на  задачах фармакоинформатики}

\def\aut{И.\,Ю.~Торшин$^1$}

\def\autkol{И.\,Ю.~Торшин}

\titel{\tit}{\aut}{\autkol}{\titkol}

\index{Торшин И.\,Ю.}
\index{Torshin I.\,Yu.}


{\renewcommand{\thefootnote}{\fnsymbol{footnote}} \footnotetext[1]
{Работа выполнена при поддержке гранта РНФ (проект №\,23-21-00154) с~использованием инфраструктуры 
Центра коллективного пользования <<Высокопроизводительные вычисления и~большие данные>> (ЦКП 
<<Информатика>>) ФИЦ ИУ РАН (г.~Москва).}}


\renewcommand{\thefootnote}{\arabic{footnote}}
\footnotetext[1]{Федеральный исследовательский центр <<Информатика и~управление>> Российской академии наук, 
\mbox{tiy135@yahoo.com}}

\vspace*{-12pt}


\Abst{Рассмотрение прецедентных отношений между признаками и~таргетной переменной в~виде наборов элементов булевой решетки указывает на возможность порождения 
синтетических признаков с~использованием метрических функций расстояния. 
Сформулированы подходы к~(1)~оценке релевантности (<<информативности>>) метрик 
по отношению к~решаемым задачам, (2)~порождению и~(3)~отбору синтетических 
признаков, более информативных, чем исходные признаковые описания. Представленные 
результаты топологического анализа 2400~выборок данных  
<<мо\-ле\-ку\-ла--свойство>> из ProteomicsDB позволили получить достаточно 
эффективные алгоритмы прогнозирования свойств молекул (ранговая корреляция  
в~кросс-ва\-ли\-да\-ции~--- $0{,}90\pm0{,}23$). На данной выборке задач установлены 
метрики, которые наиболее часто порождают информативные синтетические признаки: 
максимальное уклонение Колмогорова, <<косое>> расстояние, метрики Lp, Реньи, фон 
Мизеса. Для решения изученного комплекса задач показано преимущество полиномных 
корректоров по сравнению с~нейросетевыми и~с~корректорами типа <<случайный 
лес>>.}

\KW{топологический анализ данных; теория решеток; алгебраический подход 
Ю.\,И.~Жу\-рав\-лё\-ва; фармакоинформатика}

\DOI{10.14357/19922264240207}{OTXCUD}
  
\vspace*{-1pt}


\vskip 10pt plus 9pt minus 6pt

\thispagestyle{headings}

\begin{multicols}{2}

\label{st\stat}

\section{Введение}

     В первой части работы~[1] принимается, что задано регулярное 
множество прецедентов 
$$
\mathbf{Q}\hm= \{\mathrm{D}(x_i)\vert x_i\in 
\mathbf{X}\}
$$ 
на решетке $L(T(\mathbf{X}))$, по\-рож\-ден\-ное на основе 
множества исходных описаний объектов $\mathbf{X}\hm= \{ x_1, \ldots , 
x_{N_0}\}$. Для индивидуального объекта\linebreak $x_i\hm\in \mathbf{X}$ 
прецедентному соотношению между значениями признаками 
$\Gamma_k(x_i)$ и~\mbox{$t$-й} таргетной переменной соответствует множество пар 
$\{(\{\Gamma_k^{-1}(\Gamma_k(x_i)),\linebreak k\hm=\overline{1, ,n}\}, \Gamma_t^{-1}(\Gamma_t(x_i))), i\hm=\overline{1,N_0},\
 k\hm=\overline{1,n},\linebreak t\hm=\overline{n+1, n+l}\}$, где $l$~--- 
число таргетных переменных. В~рамках топологической теории 
распознавания прецедентное соотношение между множествами $\{ 
\Gamma_k^{-1}(\Gamma_k(x_i))\}$ и~$\Gamma_t^{-1}(\Gamma_t(x_i))$ 
моделируется как со\-от\-вет\-ст\-ву\-ющие массивы расстояний, по\-рож\-да\-емые той 
или иной мет\-ри\-кой~$\rho_m$: $L(T(\mathbf{X}))^2\hm\to [0\ldots 1]$, 
$m\hm= \overline{1, m_0}$. В~[1] предложены способы <<встра\-и\-ва\-ния>> 
в~формализм полуэмирических рас\-сто\-яний на множествах $a\hm\in 
L(T(\mathbf{X}))$, векторах $\vec{v}_\alpha [a] \hm= ( v_{\alpha_1}[a], 
v_{\alpha_2}[a], \ldots , v_{\alpha_i}[a],\ldots)$ и~функциях 
$\hat{\phi}(x)\bm{\Gamma}_t(u)$. 
     
     Здесь для практического приложения формализма сформулированы 
подходы к~исследованию свойств~$\rho_m$, способы оценки релевантности 
функций~$\rho_m$ по отношению к~решаемым задачам, способы 
порождения и~отбора синтетических признаков, основанных на~$\rho_m$. 
Представлены результаты экспериментальной апробации на задачах 
фармакоинформатики.
     
\section{Об исследовании свойств функций расстояния~$\rho_m$}

    Рабочая гипотеза настоящего исследования со\-сто\-ит в~том, что для 
порождения более <<информативных>> признаков могут использоваться 
полуэмпирические функционалы расстояния на \mbox{множествах}, векторах, 
функциях~[2]. Метрические свойства ис\-поль\-зу\-емых функций 
расстояния~$\rho_m$ могут исследоваться аналитически или комбинаторно 
с~использованием аксиом метрики~[3]. Для анализа свойств этих 
функционалов в~топологической теории распознавания вводится следующее 
понятие.

\smallskip

\noindent
\textbf{Определение~1.} Обобщенной оценочной функцией расстояния 
будем называть конструкцию вида 
$$
\rho(a,b) = f(g ( v[a\vee b]) - g(v[a\wedge b])),
$$
 в~которой~$f$ и~$g$~--- функции, монотонные на 
соответствующих участках действительной оси; $v:\ L\hm\to R^+$~--- 
изотонная оценка, для которой выполнено условие оценки (\textbf{уО}: $\forall_L 
a,b: v[a]\hm+v[b]\hm= v[a \wedge b]\hm+ v[a\vee b]$) и~изотонности 
(\textbf{уИ}:  $\forall_L a,b: a\supseteq b \hm\Rightarrow v[a]\hm\geq v[b]$). 

\smallskip

\noindent
\textbf{Теорема~1.} \textit{Функция расстояния~$\rho$ считается 
обобщенной оценочной функцией расстояния тогда и~только тогда, когда 
$\rho(a,b)\hm= \rho(a\vee b, a\wedge b)$, а~термы от $a$ и~$b$ в~формуле для 
$\rho(a,b)$ представляют собой композицию монотонной функции 
и~изотонной оценки}. 

\smallskip

Необходимость следует из  $a\vee b\hm= (a\vee b)\vee (a\wedge b)$ и~$a\wedge b \hm= (a\vee b) \wedge (a\wedge b)$  при 
подстановке $a\vee b$ и~$a\wedge b$ вместо $a$ и~$b$ в~определение~1. 
Эквивалентность $\rho(a,b)$ и~$\rho(a\vee b, a\wedge b)$ указывает на то, что 
в~выражение для вычисления~$\rho$ входят тер\-мы-функ\-ци\-о\-на\-лы, 
содержащие выражения $a\vee b$ и~$a\wedge b$, взаимозаменяемые с~$a$ 
и~$b$, т.\,е.\ термы вида $g^\prime (a\vee b)$ и~$g^\prime(a\wedge b)$. По 
условию теоремы эти термы включают монотонную функцию от изотонной 
оценки, т.\,е.~$g^\prime$ монотонна. Так как $\rho$~--- функция расстояния, 
то $g^\prime$-тер\-мы не могут входить в~выражение для~$\rho$ в~виде 
произведения, суммы, отношения, степени или суммы, а~только в~виде 
разности, т.\,е.\
$$
\rho(a,b) = f\left(g^\prime(a\vee b) \hm- g^\prime (a\wedge b)\right),
$$ 
из чего следует достаточность. Теорема доказана.

\smallskip

\noindent
\textbf{Следствие~1.} Для обобщенной оценочной~$\rho$ 
\begin{multline*}
\forall \ell \subseteq L(T(\mathbf{X})): \Delta_{\vee\wedge}(\ell)\equiv 0,\\ 
\Delta_{\vee\wedge}(\ell)=  \sum\limits_{a,b\in \ell} \vert\rho(a,b)- 
\rho(a\vee b, a\wedge b)\vert \fr{2}{\vert\ell\vert/(\vert\ell\vert -1)}\,.
\end{multline*}

\smallskip

\noindent
\textbf{Следствие~2.} Выберем <<опорное>> множество $a\hm\in 
L(T(\mathbf{X}))$ и~обобщенную оценочную~$\rho$. При $f(x)\hm= g(x)\hm= x$ 
$v_{a,\rho}[b]\hm= \rho(a,b)\hm= \rho(a\vee b, a\wedge b)$~--- изотонная 
оценка. 

Следует из того, что любая линейная комбинация изотонных оценок~--- 
изотонная оценка при условии положительной определенности (теорема~2 
в~[4]). Также проверяется прямой подстановкой $v_{a,\rho}[b]$ в~уО и~уИ. 

\smallskip

\noindent
\textbf{Следствие~3.} Расстояния Фре\-ше--Ни\-ко\-ди\-ма, Амана,  
Рэн\-да/Ще\-ка\-нов\-ско\-го, Со\-ка\-ла--Сни\-са (варианты~1, 2 и~3),  
Рас\-се\-ла--Рао, Род\-же\-ра--Та\-ни\-мо\-то, Фейта, Тверского и~Юле 
могут служить обобщенными оценочными функциями расстояния. 

\smallskip

\noindent
\textbf{Следствие~4.} Расстояния Симпсона, Бра\-у\-на--Блан\-ке, 
Андерберга и~Говера-2  не входят в~число обобщенных оценочных функций 
расстояния.

\smallskip

     Теорема~1 со следствиями предоставляет аналитический 
и~комбинаторный инструментарий для исследования свойств 
полуэмпирических функций расстояния. Если заданная~$\rho$ служит 
обобщенной оценочной функцией расстояния, то могут быть получены 
соответствующие аналитические выражения для функций~$f$ и~$g$. 
Например, расстояние Со\-ка\-ла--Сни\-са-2
$$
\rho(a,b) = 1- \fr{\vert a\cap 
b\vert }{\vert a\cup b\vert + \vert a\Delta b\vert}
$$ 
выступает 
обобщенным оценочным расстоянием с~$f(x)\hm= (e^x\hm-1)/(0{,}5e^x\hm-1)$ и~$g(x)\hm=\ln (x)$. При невозможности аналитической проверки 
свойства~$\rho$ как обобщенной оценочной могут быть изучены на 
подмножествах~$\ell$ решетки $L(T(\mathbf{X}))$ посредством вычисления 
значений функционала $\Delta_{\vee\wedge}(\ell)$ (следствие~1). 

\section{О способах оценки релевантности метрик~$\rho_m$ по~отношению к~задаче клас\-сификации/прогнозирования}

     Биекция между множеством прецедентов~$\mathbf{Q}$ и~множеством 
исходных описаний объектов~$\mathbf{X}$, существующая при выполнении 
условия регулярности по Журавлёву ($\forall \mathrm{x}\hm\in \mathbf{X}, 
\mathrm{x}\hm= D^{-1}(D(\mathrm{x}))$, гарантирует однозначность 
соответствия описаний~$x_i$ и~$q_i$. Это делает возможным рассматривать 
прецедентные соотношения, заданные на~$\mathbf{Q}$, в~терминах 
множеств $\{ \Gamma_k^{-1}(\Gamma_k(x_i))\}$ и~$\Gamma_t^{-1}( 
\Gamma_t(x_i))$ с~использованием расстояний~$\rho_m$ на подмножествах 
множества~$\mathbf{X}$~[1].
     
     Пусть таргетный класс объектов $\mathbf{c}_{\bm{\alpha}}$ задан 
посредством $\alpha$-го значения $t$-й переменной $\lambda_{t\alpha}\hm\in 
\mathrm{I}_t$, $t\hm= \overline{n+1,  n+l}$, как $\mathbf{c}_{{\bm 
\alpha}} \hm= \Gamma_t^{-1}(\lambda_{t\alpha})$. В~случае числовой 
переменной за $\mathbf{c}_{\bm{\alpha}}$ может приниматься каждый из 
элементов $u(\lambda_{t\alpha})$ цепи~$A_t$. Так как 
$L(T(\mathbf{X}))$ булева, то дополнение множества 
$\mathbf{c}_{\bm{\alpha}}$, $\overline{\mathbf{c}}_{\bm{\alpha}} \hm= 
\mathbf{X}\backslash \Gamma_t^{-1}(\lambda_{t\alpha})$, определено 
однозначно. Таким образом, выделение класса $\mathbf{c}_{\bm{\alpha}}$ 
порождает задачу классификации $\mathbf{c}_{\bm{\alpha}}/ 
\overline{\mathbf{c}}_{\bm{\alpha}}$. Любая задача числового 
прогнозирования может быть сведена к~последовательности корректно 
решаемых задач $\mathbf{c}_{\bm{\alpha}}/ 
\overline{\mathbf{c}}_{\bm{\alpha}}$~\cite{5-tor}.
     
     Пусть задано подмножество признаков~$p \hm\subseteq [1\ldots n]$ 
     и~элемент решетки $c\in L(T(\mathbf{X}))$. Определим функцию 
$$
\bm{\rho}_{\mathbf{mc}} (x_i, c, {p}) \hm= \{ \rho_m(c, \Gamma_k^{-1}(\Gamma_k (x_i)),\ k\hm\in {p})\}.
$$
 При заданных~$\rho_m$, $p$, 
$\mathbf{c}_{\bm{\alpha}}$ и~$\overline{\mathbf{c}}_{\bm{\alpha}}$ 
для~$x_i$ вычислимы множества расстояний $\bm{\rho}_{\mathbf{mc}}(x_i, 
\mathbf{c}_{\bm{\alpha}}, {p})$ и~$\bm{\rho}_{\mathbf{mc}}(x_i, 
\overline{\mathbf{c}}_{\bm{\alpha}}, {p})$. Обозначим 
\begin{align*}
\bm{\rho}_{\mathbf{m}\bm{\alpha}}(x_i) &=  \bm{\rho}_{\mathbf{mc}} 
(x_i, \mathbf{c}_{\bm{\alpha}}, [1\ldots n]); \\
\bm{\rho}_{\mathbf{m}\overline{\bm{\alpha}}} (x_i) &= 
\bm{\rho}_{\mathbf{mc}}(x_i, \overline{\mathbf{c}}_{\bm{\alpha}} , [1\ldots n]).
\end{align*}
 Для $x_i\hm\in \mathbf{X}$ 
определено множество 
\begin{multline*}
\bm{\rho}_{\mathbf{m}}(x_i,{p})=\left \{ \rho_{mk_1k_2}(x_i, {p}) = {}\right.\\
{}\rho_m\left(\Gamma^{-1}_{k_1}\left(\Gamma_{k_1}(x_i), \Gamma^{-1}_{k_2}\left(\Gamma_{k_2}(x_i)\right)\right)\right),\\
\left. k_1, k_2\hm \in {p},\  k_1\not= k_2\right\},\ \bm{\rho}_{\mathbf{m}}(x_i)=  \bm{\rho}_{\mathbf{m}}(x_i, [1\ldots n]).
\end{multline*}
     
     На основе $\bm{\rho}_{\mathbf{m}{\bm{\alpha}}}(x_i)$ 
и~$\bm{\rho}_{\mathbf{m}\overline{\bm{\alpha}}}(x_i)$ вводятся оценки 
релевантности~$\rho_m$. По отношению к~задаче $\mathbf{c}_{\bm{\alpha}}/ 
\overline{\mathbf{c}}_{\bm{\alpha}}$ более релевантна или 
<<информативна>> такая мет\-ри\-ка~$\rho_m$, которая для всех $x\hm\in 
\mathbf{c}_{\bm{\alpha}}$ минимизирует расстояния в~списке 
$\bm{\rho}_{\mathbf{m}{\bm{\alpha}}}(x)$ и~максимизирует расстояния 
в~списке $\bm{\rho}_{\mathbf{m}\overline{\bm{\alpha}}}(x)$ (т.\,е.\ 
<<приближает>> объекты к~их классам). Выделены два взаимосвязанных 
направления дальнейших исследований: 
\begin{enumerate}[(1)]
\item нахождение подмножеств $p$ 
признаков, <<более информативных>> для~$\rho_m$;  
\item на\-строй\-ка/вы\-бор~$\rho_m$ при фиксированном~$p$.
\end{enumerate}
     
     Для $c^\prime\hm\in L(T(\mathbf{X}))$ определим 
$\vartheta_{\mathbf{mc}}$, операцию слияния списков 
$\bm{\rho}_{\mathbf{mc}}$:
$$
\vartheta_{\mathbf{mc}}(c^\prime, c, 
{p})\hm= \bigcup\limits_{y\in c^\prime} \bm{\rho}_{\mathbf{mc}} (y,c, 
{p}).
$$
 Обозначим 
 $$
 \vartheta_{\mathbf{m}\bm{\alpha}}(\mathbf{c}, 
{p}) \!=\! \vartheta_{\mathbf{mc}}(\mathbf{c}, 
\mathbf{c}_{\bm{\alpha}}, {p});\ 
\vartheta_{\mathbf{m}\bm{\alpha}}(\mathbf{c},{p})\!=\! 
\vartheta_{\mathrm{mc}}(\mathbf{c}, \overline{\mathbf{c}}_{\bm \alpha}, 
{p}),
$$
 вычислим множества $\vartheta_{\mathbf{m}{\bm \alpha}} 
(\mathbf{c}_{\bm \alpha}, {p})$ и~$\vartheta_{\mathbf{m}{\bm \alpha}} 
(\overline{\mathbf{c}}_{\bm \alpha},{p})$ и~сформируем 
эмпирические функции распределения (э.ф.р.)\ $\hat{\phi}(x) 
\vartheta_{\mathbf{m}{\bm \alpha}} (\mathbf{c}_{\bm \alpha}, {p})$ 
и~$\hat{\phi}(x) \vartheta_{\mathbf{m}{\bm \alpha}} 
(\overline{\mathbf{c}}_{\bm \alpha}, {p})$. На пространстве 
однородных монотонно возрастающих функций 
\begin{multline*}
\mathbf{M}^+_{0\ldots1} ={}\\
{}= 
\{f: [0\ldots 1]\hm\to [0\ldots 1],\ x\geq y\hm\Rightarrow f(x)\geq f(y)\}
\end{multline*}
введем 
функционал расстояния $d_f$: $\mathbf{M}^+_{0..1}\hm\to [0\ldots 1]$ 
(максимальное уклонение Колмогорова $D(f(x), g(x))\hm= \mathrm{sup}_x 
\vert f(x)\hm- g(x)\vert$, метрики фон Мизеса, Реньи и~др.). Выбор~$d_f$ 
делает возможной постановку ряда задач топологического анализа данных:
     \begin{enumerate}[(1)]
\item количественные оценки релевантности~$\rho_m$ как 
$d_f(\hat{\phi}(x)\vartheta_{\mathbf{m}{\bm \alpha}}(\mathbf{c}_{\bm \alpha}, 
{p}), \hat{\phi}(x)\vartheta_{\mathbf{m}{\bm 
\alpha}}(\overline{\mathbf{c}}_{\bm \alpha}, {p}))$ для 
разных~$\mathbf{c}_{\bm \alpha}$, $\lambda_{t\alpha} \hm\in \mathrm{I}_t$, 
$\alpha \hm= \overline{1, \vert \mathrm{I}_t\vert}$;
\item задачи оптимизации для увеличения разделения классов 
$\mathbf{c}_{\bm \alpha}/\overline{\mathbf{c}}_{\bm \alpha}$ 
($\argmax_{\rho_m,{p}} d_f(\hat{\phi}\vartheta_{\mathbf{m}{\bm \alpha}}(\overline{\mathbf{c}}_{\bm \alpha},{p}), 
\hat{\phi}\vartheta_{\mathbf{m}{\bm \alpha}}(\mathbf{c}_{\bm \alpha}, {p}))$,
$\argmax_{\rho_m,{p}} d_f(\hat{\phi}\vartheta_{\mathbf{m}\overline{\bm{\alpha}}}, (\overline{\mathbf{c}}_{\bm \alpha}, {p}), 
\hat{\phi}\vartheta_{\mathbf{m}\overline{\bm \alpha}}
(\mathbf{c}_{\bm \alpha}, {p}))$  и~др.);
\item определение $\rho_q$-мет\-рик на пространстве объектов~[2, с.~184--199] 
(например, в~виде $d_f (\hat{\phi}\bm{\rho}_{\mathbf{m}{\bm \alpha}}(x, 
{p}), \hat{\phi}\bm{\rho}_{\mathbf{m}{\bm \alpha}}(y, {p})), 
d_f (\hat{\phi}\bm{\rho}_{\mathbf{m}}(x,{p})$, 
$\hat{\phi}\bm{\rho}_{\mathbf{m}}(y, {p}))$); 
\item оценка близости метрик~$\rho_q$ к~метрике разреза по классам 
$\mathbf{c}_{\bm{\alpha}}/ \overline{\mathbf{c}}_{\bm{\alpha}}$; 
\item формулировка критериев раз\-ре\-ши\-мости/ре\-гу\-ляр\-ности задачи 
$\mathbf{c}_{\bm{\alpha}}/ \overline{\mathbf{c}}_{\bm{\alpha}}$~[6]; 
\item оценки компактности классов $\mathbf{c}_{\bm{\alpha}}$  
и~$\overline{\mathbf{c}}_{\bm{\alpha}}$~[3]. 
\end{enumerate}

\section{О способах порождения и~отбора синтетических 
признаков на~основании функций расстояния}

     Множества $\bm{\rho}_{\mathbf{m}{\bm \alpha}}(x_i,{p})$, 
$\bm{\rho}_{\mathbf{m}{\overline{\bm \alpha}}}(x_i, {p})$ 
и~$\bm{\rho}_{\mathbf{m}}(x_i)$ и~отдельные $\rho_m(\mathbf{c}_{\bm 
\alpha}, \Gamma_k^{-1}(\Gamma_k(x_i))$ используются для формирования 
синтетических числовых признаков $\Gamma_{k^\prime}(x_i)$ 
объекта~$x_i$, $k^\prime\hm= \overline{n+ l+1, n+l+n_S}$. 
Значение синтетического признака~$\Gamma_{k^\prime}(x_i)$ зависит от 
выбора~$\rho_m$, классов $\mathbf{c}_{\bm{\alpha}}$ 
и~$\overline{\mathbf{c}}_{\bm{\alpha}}$  и~от способа его вы\-чис\-ле\-ния: 
\begin{enumerate}[(1)]
\item $\rho_m(\mathbf{c}_{\bm \alpha}, \Gamma_k^{-1}(\Gamma_k(x_i))$; 
\item $\rho_m(\overline{\mathbf{c}}_{\bm \alpha}, \Gamma_k^{-1}(\Gamma_k(x_i))$; 
\item $\rho_m(\mathbf{c}_{\bm \alpha}, \ldots ) \hm- \rho_m(\overline{\mathbf{c}}_{\bm \alpha}, \ldots)$;
\item $1\hm- \rho_m(\mathbf{c}_{\bm \alpha}, \ldots)$;
\item значения э.ф.р.\ 
$\hat{\phi}(x)\bm{\rho}_{\mathbf{m}{\bm \alpha}}(x_i,{p})$ при 
разных~$x$ (например, соответствующих процентилям 
$\hat{\phi}\bm{\rho}_{\mathbf{m}{\bm \alpha}}(x_i,{p})$); 
\item значения $\hat{\phi}(x)\bm{\rho}_{\mathbf{m}\overline{\bm{\alpha}}} 
(x_i, {p})$ при разных~$x$;
\item $\hat{\phi}(x\hm+ \Delta x) 
\bm{\rho}_{\mathbf{m}{\bm \alpha}}(x_i,p) \hm- 
\hat{\phi}(x)\bm{\rho}_{\mathbf{m}{\bm \alpha}} (x_i, {p})$ 
и~$\hat{\phi}(x\hm+ \Delta x) \bm{\rho}_{\mathbf{m}{\overline{\bm \alpha}}} 
(x_i,{p}) \hm- \hat{\phi}(x)\bm{\rho}_{\mathbf{m}\overline{\bm 
\alpha}} (x_i, {p})$, где $\Delta x$~--- шаг.
\end{enumerate}
     
     Кроме того, $\mathbf{c}_{\bm{\alpha}}$ может определяться как 
$\Gamma_t^{-1}(\lambda_{t\alpha})$ или как $u(\lambda_{t\alpha})$; если 
$\mathbf{c}_{\bm \alpha} \hm= \Gamma_t^{-1}(\lambda_{t\alpha})$, то 
$\overline{\mathbf{c}}_{\bm{\alpha}}$ может быть равно $\Gamma^{-1}_t 
(\lambda_{t\alpha+1})$; классы $\mathbf{c}_{\bm{\alpha}}/ 
\overline{\mathbf{c}}_{\bm{\alpha}}$  
$t$-й переменной могут определяться с~использованием раз\-би\-ений на 
различные процентили (которые определяются как подвыборка значений 
$\lambda_{t\alpha} \hm\in \mathrm{I}_t$) и~т.\,д. 
     
     Таким образом, предлагаемые схемы порождают значительное число 
синтетических признаков $\Gamma_{k^\prime}(x_i)$ ($10n$ и~более при $n$ 
исходных признаках $\Gamma_k$), что делает необходимым введение 
процедур отбора признаков. Таргетная переменная $\Gamma_t(x_i)$~--- 
чис\-ло\-вая, и~по\-рож\-да\-емые признаки $\Gamma_{k^\prime}(x_i)$~--- также 
чис\-ло\-вые. Для данного случая в~прикладной математике имеется несколько 
различных подходов к~оценке взаимосвязи $\Gamma_t(x_i)$ 
и~$\Gamma_{k^\prime}(x_i)$: корреляционные оценки (для линейных 
закономерностей), полиномная аппроксимация с~оценкой качества (для 
нелинейных закономерностей) и~методы теории  
ве\-ро\-ят\-но\-стей\,/\,ма\-те\-ма\-ти\-че\-ской статистики, не зависящие от 
вида закономерности (в~том числе на основе <<взаимной 
информации>>~[7]).
{\looseness=1

}
     
     Наиболее фундаментальным представляется тес\-ти\-ро\-ва\-ние взаимосвязи 
двух переменных на осно\-ве <<нулевой гипотезы>> об их независимости. 
Пусть заданы пары тестируемых значений, $(x_i, y_i)$,\linebreak $i\hm= \overline{1,\mathbf{n}_{(\mathrm{x,y})}}$, э.ф.р.~$F_{xy}(x,y)$ характеризует 
совместное распределение~$x$ и~$y$, а~э.ф.р.~$F_{{x}}(x)$ 
и~$F_{{y}}(y)$~--- индивидуальные распределения переменных. 
Эмпирическая функция распределения нулевой \mbox{гипотезы} (независимость~$x$ и~$y$) определяется как 
$F_{{x}}(x)F_{{y}}(y)$. 
     
     Для оценки отличий между $F_{{xy}}(x,y)$\linebreak 
и~$F_{{x}}(x) F_{{y}}(y)$ необходимо ввести расстояние 
меж-\linebreak ду такими функциями (так называемую <<статисти-\linebreak ку>>) и~оценить 
достоверность различий посред\-ст\-вом \mbox{того} или иного статистического\linebreak \mbox{тес\-та}. 
В~качестве расстояния можно использовать функции~$d_f$, адап\-ти\-ро\-ван\-ные 
для 2-мер\-но\-го случая (например, макси\-маль\-ное уклонение 
     $D(\mathrm{F}_{{xy}}(x,y), \mathrm{F}_{{x}}(x) 
\mathrm{F}_{{y}}(y)) \hm= \max ( \vert 
\mathrm{F}_{{xy}}(x_i,y_i) \hm- \mathrm{F}_{{x}}(x_i) 
\mathrm{F}_{{y}}(y_i)\vert )$) и~статистический тест  
Кол\-мо\-го\-ро\-ва--Смир\-но\-ва 
$P_{\mathrm{КС}}$ $(D 
(\mathrm{F}_{{xy}}(x,y), \mathrm{F}_{{x}}(x) 
\mathrm{F}_{{y}}(y)), n_{(x,y)})$. Тогда $1\hm- 
P_{\mathrm{КС}}$ характеризует <<информативность>>~$x$ 
относительно~$y$. 
     
     Более универсальным подходом к~оценке достоверности различий 
между $\mathrm{F}_{{xy}}(x,y)$ и~$\mathrm{F}_{{x}}(x) 
\mathrm{F}_{{y}}(y)$ считается прямое вычисление выбранной 
статистики~$d_f$ на множествах пар значений $(x_i, y_i)$, полученных 
датчиком случайных чисел. 
     
     Пусть \textit{оператор $\hat{\zeta}$, семплирующий} 
множество~$\mathbf{X}$, формирует набор семплов 
$$
\hat{\zeta}\mathbf{X}\hm= \{a_1, a_2, \ldots , a_k, \ldots , 
a_{\vert\hat{\zeta}X\vert}\vert a_k\hm\subset \mathbf{X}\},
$$
 а~процедура 
random~--- датчик случайных чисел (в~диапазоне $[0\ldots 1]$). Для каждого 
семпла~$a_k$ принимается, что ${n}_{({x,y})} \hm= \vert 
a_k\vert$, и~вычисляется множество значений~$d_f$ для случайных 
выборок, 

\noindent
\begin{multline*}
\mathrm{rnd}\,(\hat{\zeta}\mathbf{X}, d_f)= \left\{ 
\vphantom{i=\overline{1,\left\vert \hat{\zeta} X\right\vert }}
d_f\left(
\vphantom{\overline{1, \vert a_i\vert }}
\mathrm{F}_{{xy}}(x_{ij}, y_{ij}), 
\mathrm{F}_{{x}}(x_{ij}) \mathrm{F}_{{y}}(y_{ij}),\right.\right.\\
\left.\left. x_{ij}, 
y_{ij}= \mathrm{random},\  j=\overline{1, \vert a_i\vert }\right),\ i=\overline{1,\left\vert \hat{\zeta} X\right\vert }\right\}.
\end{multline*}

 Для $a\hm\in \hat{\zeta} \mathbf{X}$ значение 
${P}(d_f, \hat{\zeta}\mathbf{X}, a, k^\prime, t)\hm= 1\hm-
\hat{\phi}(d_f(\mathrm{F}_{k^\prime t}(\Gamma_{k^\prime}(z), \Gamma_t(z)), F_{k^\prime}(\Gamma_{k^\prime}(z)) 
\mathrm{F}_t(\Gamma_t(z)))\vert z\hm\in a) \mathrm{rnd}\,(\hat{\zeta}\mathbf{X}, d_f)$~--- статистическая достоверность 
<<зависимости>> $\Gamma_t(z)$ и~$\Gamma_{k^\prime}(z)$ по 
статистике~$d_f$ на семпле~$a$, а~$1\hm- P(d_f, 
\hat{\zeta}\mathbf{X}, a, k^\prime, t)$ количественно оценивает зависимость.
    

При заданном способе оценки зависимости $1\hm- P(d_f, 
\hat{\zeta}\mathbf{X}, a, k^\prime, t)$ задача отбора информативных 
признаков решается посредством так называемого\linebreak  
В-ал\-го\-рит\-ма, исходно разработанного для построения оптимальных 
словарей финальных ин\-фор\-маций (чему и~соответствует литера~<<В>>)~[8]. 
\mbox{Данный} алгоритм, основанный на критерии раз\-ре\-ши\-мости по Журавлёву, 
позволяет выбирать множества финальных информаций на основе 
максимального час\-тич\-но\-го покрытия при минимуме\linebreak элементов покрытия. 
Замена мощности пересечения множеств на $1\hm- P(d_f, 
\hat{\zeta}\mathbf{X}, a, k^\prime, t)$ приведет к~тому, что  
В-ал\-го\-ритм будет выбирать минимум признаков с~максимальной 
<<информативностью>>\linebreak (наиболее информативные признаки, см.\ 
теоремы~1, 7  и~8 работы~[8]).

    Таким образом, в~рамках развиваемого формализма синтез более 
информативных синтетических~$\Gamma_{k^\prime}(x_i)$ осуществляется 
в~5~стадий: 
\begin{enumerate}[(1)]
\item определяется набор исходных (как правило, 
<<низкоинформативных>>) признаков~$\Gamma_k(x_i)$ и~таргетная 
переменная~$\Gamma_t(x_i)$;
\item вводится набор метрик~$\rho_m$, 
оценивается их релевантность $d_f(\hat{\phi}(x)\vartheta_{\mathbf{m}{\bm 
\alpha}}(\mathbf{c}_{\bm \alpha},{p})$,\linebreak 
$\hat{\phi}(x)\vartheta_{\mathbf{m}{\bm \alpha}}(\overline{\mathbf{c}}_{\bm 
\alpha}, {p}))$ для каждого класса~$\mathbf{c}_{\bm \alpha}$ 
значений $t$-й переменной и~отбираются наиболее релевантные~$\rho_m$; 
\item посредством каждой из отобранных~$\rho_m$ по\-рож\-да\-ют\-ся 
синтетические признаки~$\Gamma_{k^\prime}(x_i)$;
\item посредством 
вычислений $1\hm- P(d_f, \hat{\zeta}\mathbf{X}, a, k^\prime, t)$  
и~В-ал\-го\-рит\-ма отбирается минимальное чис\-ло признаков максимальной 
<<ин\-фор\-ма\-тив\-ности>>;
\item применяется алгоритм прогнозирования 
таргетной переменной (корректор по Жу\-рав\-лё\-ву--Ру\-да\-кову). 
\end{enumerate}

\begin{table*}\small
\begin{center}
\begin{tabular}{|l|c|c|}
\multicolumn{3}{p{140mm}}{Ранговые корреляции между экспериментальными 
и~расчетными значениями $EC_{50}$ и~других величин хемокиномного анализа: $r$~--- 
коэффициент ранговой корреляции на обучении; $r_c$~---  на контроле. Усреднение~$r$ 
и~$r_c$ проводилось по 2400~выборкам хемокиномных данных}\\
\multicolumn{3}{c}{\ }\\[-6pt]
\hline
\multicolumn{1}{|c|}{{Эксперимент}}&$r$&$r_c$\\
\hline
{\boldmath $f_{\theta_k}$}\textbf{-алгоритмы, корректор~--- нейросеть}&\boldmath{$0{,}88\pm 
0{,}15$}&\boldmath{$0{,}86\pm0{,}20$}\\
Синтетические $\Gamma_{k^\prime}(x_i)$, корректор~--- нейросеть (2~слоя)&$0{,}45\pm 
0{,}22$&$0{,}22\pm 0{,}21$\\
Синтетические $\Gamma_{k^\prime}(x_i)$, корректор~--- нейросеть 
(10~слоев)&$0{,}52\pm 0{,}25$&$0{,}21\pm 0{,}20$\\
Синтетические $\Gamma_{k^\prime}(x_i)$, корректор~--- <<случайный лес>>, 
вариант~1&$0{,}98\pm 0{,}15$&$0{,}67\pm 0{,}31$\\
Синтетические $\Gamma_{k^\prime}(x_i)$, корректор~--- <<случайный лес>>, 
вариант~2&$0{,}99\pm 0{,}14$&$0{,}71\pm 0{,}35$\\
\textbf{Синтетические {\boldmath $\Gamma_{k^\prime}(x_i)$}, полиномные корректоры, 
вариант~1}&\boldmath{$0{,}93\pm 0{,}11$}&\boldmath{$0{,}90\pm 0{,}23$}\\
\textbf{Синтетические {\boldmath $\Gamma_{k^\prime}(x_i)$}, полиномные корректоры, 
вариант~2}&\boldmath{$0{,}95\pm0{,}08$}&\boldmath{$0{,}86\pm 0{,}27$}\\
\hline
\end{tabular}
\end{center}
\end{table*}

\section{Экспериментальная апробация }

    Формализм апробирован на комплексе задач\linebreak фармакоинформатики: 
получение количественных оценок ингибирования киназ протеома 
перспективными лекарствами (хемокиномный анализ)~[9]. Использованы 
2400~выборок данных <<\mbox{мо\-ле\-ку\-ла}--свой\-ст\-во>> из ProteomicsDB; 
свойства молекул включили константы $EC_{50}$ и~активности для 
концентраций~$(E_j(C_i))$.

     Исходные признаки $\Gamma_k(x_i)$ определялись как булевы 
инварианты над множествами $\chi$-це\-пей и~$\chi$-уз\-лов 
хемографов~$x_i$, как и~в~[9]. Таргетная $\Gamma_t(x_i)$ определялась как 
числовое значение прогнозируемого свойства. В~качестве~$\rho_m$ 
использовались функции расстояния на множествах, векторах и~э.ф.р.\ (всего 
65~функций из справочника~[2]). Классы~$\mathbf{c}_{\bm{\alpha}}$ 
определялись как квартили значений~$\Gamma_t$. Векторы элементов 
$L(T(\mathbf{X}))$ формировались из оценок $v^+_\alpha$, $v^-_\alpha$ 
и~$d_\alpha$~\cite{4-tor} для каждого~$\mathbf{c}_{\bm{\alpha}}$. 
Релевантность~$\rho_m$ по $d_f(\hat{\phi}(x),\vartheta_{\mathbf{m}{\bm 
\alpha}}(\mathbf{c}_{\bm{\alpha}},{p}), 
\hat{\phi}(x)\vartheta_{\mathbf{m}{\bm \alpha}} 
(\overline{\mathbf{c}}_{\bm{\alpha}}, {p}))$ оценивалась для 
каждого~$\mathbf{c}_{\bm{\alpha}}$, $d_f$~--- максимальное уклонение. 
Синтетические признаки~$\Gamma_{k^\prime}(x_i)$ по\-рож\-да\-лись всеми 
перечисленными выше способами; их отбор проводился В-ал\-го\-рит\-мом 
с~использованием $1\hm- {P}(d_f, \hat{\zeta}\mathbf{X}, a, 
     k^\prime, t)$. 
     
     В качестве корректоров использовались нейронные сети с~несколькими 
слоями (от~2 до~10) с~функцией активации softmax, полиномы различных 
конструкций (более 20~формул, в~том числе квазиполиномные модели 
с~элементарными функциями) и~<<случайные леса>> решающих деревьев. 
Оператор семплирования~$\hat{\zeta}$ был реализован как десятикратная  
кросс-ва\-ли\-да\-ция с~делением каждой выборки объектов на 80\% 
(обучение) и~20\% (конт\-роль). Результаты экспериментов суммированы 
в~таблице.
     

     
     Наилучший результат применения нового <<топологического>> 
формализма с~полиномным корректором ($r_c\hm=0{,}90\hm\pm0{,}23$) 
немного превзошел наилучший результат применения \mbox{метода} опорных 
функций (композиций вида $f_{\theta_k} \hm= g(f_1(\sum \omega_k^j x_k), 
\ldots\linebreak \ldots , f_l(\sum \omega_k^j x_k))$, см.~[9]), для которого 
$r_c\hm=0{,}86\hm\pm0{,}20$. Полиномными формулами, наиболее часто 
показывавшими наилучший результат, оказались полиномы 1-й или 2-й 
степеней с~произведениями переменных первой степени, полиномы 5-й 
степени, квазиполиномы 5-й степени с~сигмоидами и~Фурье-по\-ли\-но\-мы  
3-й степени.
     
     Нейросетевые корректоры всех использованных конфигураций 
отличались крайне низкими показателями ($r\hm=0{,}45\hm\pm0{,}22$, 
$r_c\hm=0{,}22\hm\pm0{,}21$), а~<<случайный лес>> приводил 
к~существенному переобучению (см.\ таб\-ли\-цу). При этом в~290 
из~2400~выборок данных (12\%) <<случайный лес>> приводил к~улучшению 
результатов по сравнению с~наилучшими полиномными корректорами, 
а~в~1670 из 2400~выборок данных (70\%)~--- к~ухудшению.
     
     
     Анализ синтетических признаков $\Gamma_{k^\prime}(x_i)$, 
вошедших в~наилучшие полиномные модели, показал, что среди более 
информативных (по оценке $1\hm- P(d_f, \hat{\zeta}\mathbf{X},  
a, k^\prime, t)$) признаков чаще всего встречались признаки, порождаемые 
с~использованием э.ф.р.\ на основе опорных цепей (теорема~1 в~1-й части 
работы~[1]), среди наименее информативных~--- исходные признаки 
$\Gamma_k(x_i)$ и~признаки на основе отдельных расстояний 
$\rho_m(\mathbf{c}_{\bm{\alpha}} , \Gamma_k^{-1}(\Gamma_k(x_i))$. 
Функциями~$\rho_m$, наиболее часто порождающими информативные 
$\Gamma_{k^\prime}(x_i)$ на пространстве э.ф.р., оказались максимальное 
уклонение Колмогорова, <<косое>> расстояние, метрики $\mathrm{Lp}$, 
Реньи, $\chi2$, фон Мизеса, инженерная~\cite{2-tor}. В~среднем по всем 
выборкам данных эти~7~разновидностей~$\rho_m$ порождали более 50\% 
самых информативных признаков~$\Gamma_{k^\prime}(x_i)$, отобранных  
В-ал\-го\-рит\-мом.

\vspace*{-6pt}

\section{Заключение}

\vspace*{-2pt}

    Предлагаемый подход к~порождению информативных синтетических 
признаков подразумевает последовательные трансформации описаний 
объекта:\\[-13pt]
\begin{enumerate}[(1)]
\item исходное множество значений признаков;\\[-13.5pt]
\item множество 
соответствующих элементов решетки;\\[-13.5pt] 
\item ~множество расстояний 
(измеряемых посредством~$\rho_m$) между элементами решетки, 
соответствующими классам и~признакам;\\[-13.5pt]
\item множество э.ф.р.\ расстояний, 
измеренных заданными~$\rho_m$;\\[-13.5pt] 
\item множество синтетических признаков 
объ-\linebreak екта.
\end{enumerate}

\noindent
 Использование многочисленных метрик на стадии порождения 
признаков позволяет рассматривать развиваемый формализм как вариант 
развития идеологии АВО (алгоритмы вычисления \mbox{оценок}) научной школы 
Ю.\,И.~Журавлёва. Экспериментальная апробация предлагаемого подхода на 
2400~однородных задачах фармакоинформатики позволила повысить 
аккуратность и~обобщающую способность алгоритмов. 


{\small\frenchspacing
 {\baselineskip=10.6pt
 %\addcontentsline{toc}{section}{References}
 \begin{thebibliography}{99}
  
  \bibitem{1-tor}
\Au{Торшин И.\,Ю.} О~порождении синтетических признаков на основе опорных цепей 
и~произвольных метрик в~рамках топологического подхода к~анализу данных. Часть~1. 
Включение в~формализм эмпирических функций расстояния~// Информатика и~её 
применения, 2024. Т.~18. Вып.~1. С.~71--77. doi: 10.14357/19922264240110. EDN: 
RIVOXR.
  \bibitem{2-tor}
  \Au{Деза Е.\,И., Деза~М.\,М.} Энциклопедический словарь расстояний~/ Пер. с~англ.~--- М.: Наука, 
2008. 444~с. (\Au{Deza~E.\,I., Deza~M.\,M.} {Dictionary of distances}.~--- North-Holland: 
Elsevier, 2006. 412~p. doi: 10.1016/B978-0-444-52087-6.X5000-8.)
  \bibitem{3-tor}
  \Au{Torshin I.\,Y., Rudakov~K.\,V.} Combinatorial analysis of the solvability properties of 
the problems of recognition and completeness of algorithmic models. Part~2: Metric approach 
within the framework of the theory of classification of feature values~// Pattern Recognition Image 
Analysis, 2017. Vol.~27. No.\,2. P.~184--199. doi: 10.1134/S1054661817020110.
  \bibitem{4-tor}
\Au{Торшин И.\,Ю.} О~формировании множеств прецедентов на основе таблиц 
разнородных признаковых описаний методами топологической теории анализа данных~// 
Информатика и~её применения, 2023. Т.~17. Вып.~3. С.~2--7. doi: 
10.14357/19922264230301. EDN: AQEUYO.
  \bibitem{5-tor}
  \Au{Torshin I.\,Yu., Rudakov~K.\,V.} On the procedures of generation of numerical features 
over partitions of sets of objects in the problem of predicting numerical target variables~// 
Pattern Recognition Image Analysis, 2019. Vol.~29. No.\,4. P.~654--667. doi: 
10.1134/S1054661819040175. 
  \bibitem{6-tor}
  \Au{Torshin I.\,Y., Rudakov~K.\,V.} Combinatorial analysis of the solvability properties of 
the problems of recognition and completeness of algorithmic models. Part~1: Factorization 
approach~// Pattern Recognition Image Analysis, 2017. Vol.~27. No.\,1. P.~16--28. doi: 
10.1134/S1054661817010151.
  \bibitem{7-tor}
  \Au{Sosa-Cabrera G., G$\acute{\mbox{o}}$mez-Guerrero~S.,  
Garc$\acute{\iota}$a-Torres~M., Schaerer~C.\,E.} Feature selection: A~perspective on inter-attribute 
cooperation~// Int. J. Data Science Analytics, 2024. Vol.~17. P.~139--151. doi:  
10.1007/s41060-023-00439-z.
  \bibitem{8-tor}
  \Au{Torshin I.\,Y.} Optimal dictionaries of the final information on the basis of the solvability 
criterion and their applications in bioinformatics~// Pattern Recognition Image Analysis, 2013. 
Vol.~23. No.\,2. P.~319--327. doi: 10.1134/S1054661813020156.
  \bibitem{9-tor}
\Au{Торшин И.\,Ю.} О~задачах оптимизации, воз\-ни\-ка\-ющих при применении 
топологического анализа данных к~поиску алгоритмов прогнозирования 
с~фиксированными корректорами~// Информатика и~её применения, 2023. Т.~17. Вып.~2. 
С.~2--10. doi: 10.14357/19922264230201. EDN: IGSPEW.

\end{thebibliography}

 }
 }

\end{multicols}

\vspace*{-8pt}

\hfill{\small\textit{Поступила в~редакцию 09.04.24}}

\vspace*{6pt}

%\pagebreak

%\newpage

%\vspace*{-28pt}

\hrule

\vspace*{2pt}

\hrule



\def\tit{ON THE GENERATION OF~SYNTHETIC FEATURES BASED~ON~SUPPORT~CHAINS 
AND~ARBITRARY METRICS\\ WITHIN THE~FRAMEWORK OF~A~TOPOLOGICAL 
APPROACH\\ TO~DATA ANALYSIS. PART~2. EXPERIMENTAL TESTING 
ON~PHARMACOINFORMATICS PROBLEMS}


\def\titkol{On the generation of~synthetic features based on~support chains 
and~arbitrary metrics} % within the~framework of~a~topological  approach to~data analysis. Part~2. Experimental testing  on~pharmacoinformatics problems}


\def\aut{I.\,Yu.~Torshin}

\def\autkol{I.\,Yu.~Torshin}

\titel{\tit}{\aut}{\autkol}{\titkol}

\vspace*{-15pt}


\noindent
Federal Research Center ``Computer Science and Control'' of the Russian Academy of 
Sciences, 44-2~Vavilov Str., Moscow 119333, Russian Federation

\def\leftfootline{\small{\textbf{\thepage}
\hfill INFORMATIKA I EE PRIMENENIYA~--- INFORMATICS AND
APPLICATIONS\ \ \ 2024\ \ \ volume~18\ \ \ issue\ 2}
}%
 \def\rightfootline{\small{INFORMATIKA I EE PRIMENENIYA~---
INFORMATICS AND APPLICATIONS\ \ \ 2024\ \ \ volume~18\ \ \ issue\ 2
\hfill \textbf{\thepage}}}

\vspace*{3pt}
  
  


\Abste{Consideration of precedent relationships between features and a target variable in the 
form of sets of Boolean lattice elements indicates the possibility of generating synthetic features 
using metric distance functions. Approaches to ($i$)~assessing the relevance (``informativeness'') 
of metrics in relation to the problems being solved; ($ii$)~generating; and ($iii$)~selecting synthetic 
features that are more informative than the original feature descriptions are formulated. The 
results of topological analysis of~2400~samples of ``molecule--property'' data
from 
ProteomicsDB made it possible to obtain fairly effective algorithms for 
predicting the properties of molecules (rank correlation in cross-validation is~$0.90\pm 0.23$). 
Using this sample of problems, metrics have been established\linebreak\vspace*{-12pt}}

\Abstend{that most often generate 
informative synthetic features: maximum Kolmogorov deviation, ``oblique'' distance, and Lp, Renyi, 
and von Mises metrics. To solve the studied set of problems, the advantage of polynomial 
correctors compared to neural network and random forest correctors is shown.}

\KWE{topological data analysis; lattice theory; algebraic approach of Yu.\,I.~Zhuravlev; 
pharmacoinformatics}




\DOI{10.14357/19922264240207}{OTXCUD}

%\vspace*{-12pt}

\Ack

\vspace*{-3pt}


\noindent
The research was funded by the Russian Science Foundation, project No.\,23-21-00154. The 
research was carried out using the infrastructure of the Shared Research Facilities ``High 
Performance Computing and Big Data'' (CKP ``Informatics'') of FRC CSC RAS (Moscow).
 


  \begin{multicols}{2}

\renewcommand{\bibname}{\protect\rmfamily References}
%\renewcommand{\bibname}{\large\protect\rm References}

{\small\frenchspacing
 {%\baselineskip=10.8pt
 \addcontentsline{toc}{section}{References}
 \begin{thebibliography}{9} 
 
 %\vspace*{-3pt}
  \bibitem{1-tor-1}
\Aue{Torshin, I.\,Yu.} 2024. O~porozhdenii sinteticheskikh priznakov na osno\-ve opor\-nykh 
tsepey i~proizvol'nykh metrik v~ram\-kakh topologicheskogo podkhoda k~analizu dannykh. 
Chast'~1. Vklyuchenie v~formalizm empiricheskikh funktsiy rasstoyaniya [On the generation 
of synthetic features based on support chains and arbitrary metrics within a~topological approach 
to data analysis. Part~1. Inclusion of empirical distance functions into the formalism]. 
\textit{Informatika i~ee Primeneniya~--- Inform Appl.} 18(1):71--77. doi: 
10.14357/19922264240110. EDN: RIVOXR.
  \bibitem{2-tor-1}
\Aue{Deza, E.\,I., and M.\,M.~Deza.} 2006. \textit{Dictionary of distances}. North-Holland: 
Elsevier. 412~p. doi: 10.1016/B978-0-444-52087-6.X5000-8.
  \bibitem{3-tor-1}
\Aue{Torshin, I.\,Yu., and K.\,V.~Rudakov.} 2017. Combinatorial analysis of the solvability 
properties of the problems of recognition and completeness of algorithmic models. Part~2: 
Metric approach within the framework of the theory of classification of feature values. 
\textit{Pattern Recognition Image Analysis} 27(2):184--199. doi: 10.1134/S1054661817020110.
  \bibitem{4-tor-1}
\Aue{Torshin, I.\,Yu.} 2023. O~formirovanii mnozhestv pretsedentov na osnove tablits 
raznorodnykh priznakovykh opisaniy metodami topologicheskoy teorii analiza dannykh [On the 
formation of sets of precedents based on tables of heterogeneous feature descriptions by methods 
of topological theory of data analysis]. \textit{Informatika i~ee Primeneniya~--- Inform Appl.} 
17(3):2--7. doi: 10.14357/19922264230301. EDN: AQEUYO.
  \bibitem{5-tor-1}
\Aue{Torshin, I.\,Yu., and K.\,V.~Rudakov.} 2019. On the procedures of generation of 
numerical features over partitions of sets of objects in the problem of predicting numerical target 
variables. \textit{Pattern Recognition Image Analysis} 29(4):654--667. doi: 
10.1134/S1054661819040175.
  \bibitem{6-tor-1}
\Aue{Torshin, I.\,Y., and K.\,V.~Rudakov.} 2017. Combinatorial analysis of the solvability of 
the problems of recognition, completeness of algorithmic models. Part~1: Factorization 
approach. \textit{Pattern Recognition Image Analysis} 27(1):16--28. doi: 
10.1134/S1054661817010151.
  \bibitem{7-tor-1}
\Aue{Sosa-Cabrera, G., S.~Gуmez-Guerrero, \mbox{M.~Garc$\acute{\!\mbox{{\ptb{\i}}}}$a}-Torres, 
and C.\,E.~Schaerer.} 2024. Feature selection: A~perspective on inter-attribute cooperation. \textit{Int. J. 
Data Science Analytics} 17:139--151. doi: 10.1007/s41060-023-00439-z.
  \bibitem{8-tor-1}
\Aue{Torshin, I.\,Y.} 2013. Optimal dictionaries of the final information on the basis of the 
solvability criterion and their applications in bioinformatics. \textit{Pattern Recognition Image 
Analysis}  23(2):319--327. doi: 10.1134/ S1054661813020156.
  \bibitem{9-tor-1}
\Aue{Torshin, I.\,Yu.} 2023. O~zadachakh optimizatsii, voznikayushchikh pri primenenii 
topologicheskogo analiza dannykh k~poisku algoritmov prognozirovaniya s~fiksirovannymi 
korrektorami [On optimization problems arising from the application of topological data analysis 
to the search for forecasting algorithms with fixed correctors]. \textit{Informatika i~ee 
Primeneniya~--- Inform Appl.} 17(2):2--10. doi: 10.14357/19922264230201. EDN: IGSPEW.

\end{thebibliography}

 }
 }

\end{multicols}

\vspace*{-6pt}

\hfill{\small\textit{Received April 9, 2024}} 

\vspace*{-12pt}


\Contrl

\vspace*{-3pt}

\noindent
\textbf{Torshin Ivan Y.} (b.\ 1972)~--- Candidate of Science (PhD) in physics and mathematics, 
Candidate of Science (PhD) in chemistry, leading scientist, Federal Research Center ``Computer 
Science and Control'' of the Russian Academy of Sciences, 44-2~Vavilov Str, Moscow 119333, 
Russian Federation; \mbox{tiy135@yahoo.com}
  
  



\label{end\stat}

\renewcommand{\bibname}{\protect\rm Литература}    %5
\def\stat{shestakov+vor}

\def\tit{АСИМПТОТИЧЕСКАЯ НОРМАЛЬНОСТЬ И~СИЛЬНАЯ СОСТОЯТЕЛЬНОСТЬ ОЦЕНКИ РИСКА ПРИ~ИСПОЛЬЗОВАНИИ FDR-ПОРОГА В УСЛОВИЯХ СЛАБОЙ ЗАВИСИМОСТИ}

\def\titkol{Асимптотическая нормальность и~сильная состоятельность оценки риска при~использовании FDR-порога} % в~условиях слабой зависимости}

\def\aut{М.\,О.~Воронцов$^1$, О.\,В.~Шестаков$^2$}

\def\autkol{М.\,О.~Воронцов, О.\,В.~Шестаков}

\titel{\tit}{\aut}{\autkol}{\titkol}

\index{Воронцов М.\,О.}
\index{Шестаков О.\,В.}
\index{Vorontsov M.\,O.}
\index{Shestakov O.\,V.}


%{\renewcommand{\thefootnote}{\fnsymbol{footnote}} \footnotetext[1]
%{Работа 
%выполнена при поддержке Программы развития МГУ, проект №\,23-Ш03-03. При анализе 
%данных использовалась инфраструктура Центра коллективного пользования 
%<<Высокопроизводительные вычисления и~большие данные>> 
%(ЦКП <<Информатика>>) ФИЦ ИУ РАН (г.~Москва)}}


\renewcommand{\thefootnote}{\arabic{footnote}}
\footnotetext[1]{Московский государственный университет 
имени~М.\,В.~Ломоносова, факультет вычислительной математики и~кибернетики;  
Московский центр фундаментальной и~прикладной математики, \mbox{m.vtsov@mail.ru}}
\footnotetext[2]{Московский государственный университет 
имени М.\,В.~Ломоносова, факультет вычислительной математики и~кибернетики; 
Федеральный исследовательский центр <<Информатика и~управление>> Российской 
академии наук; Московский центр фундаментальной и~прикладной математики, 
\mbox{oshestakov@cs.msu.ru}}


\vspace*{-12pt}





\Abst{Рассматривается подход к~решению задачи удаления шума в~большом массиве 
разреженных данных, основанный на методе контроля средней доли ложных отклонений 
гипотез (False Discovery Rate, FDR). Данный подход эквивалентен процедурам 
пороговой обработки, обнуляющим компоненты массива, значения которых не 
превосходят некоторого заданного порога.  Наблюдения в~модели считаются слабо 
зависимыми. Для контроля степени зависимости используются ограничения на 
коэффициент сильного перемешивания и~максимальный коэффициент корреляции. 
В~качестве меры эффективности рассматриваемого подхода используется 
среднеквадратичный риск. Вычислить значение риска можно только на тестовых 
данных, поэтому в~работе рассматривается его статистическая оценка и~исследуются 
ее свойства. Показана асимптотическая нормальность и~сильная состоятельность 
оценки риска при использовании FDR-по\-ро\-га в~условиях слабой зависимости в~данных.}

\KW{пороговая обработка; множественная проверка гипотез; 
оценка риска}

\DOI{10.14357/19922264240309}{ZOQVTO}
  
%\vspace*{-6pt}


\vskip 10pt plus 9pt minus 6pt

\thispagestyle{headings}

\begin{multicols}{2}

\label{st\stat}



\section{Введение}

Во многих прикладных областях возникает задача обработки больших массивов 
зашумленных данных. Примерами служат задачи обработки изоб\-ра\-же\-ний с~высоким 
разрешением~\cite{FDRImage}, задачи множественной проверки гипотез, возникающие 
в~\mbox{исследованиях} в~об\-ласти генетики~\cite{MultipleTesting}, и~другие проб\-ле\-мы. 
В~связи с~этим рас\-смот\-рим модель
$$
x_i = \mu_i + z_i, \enskip i=\overline{1,n}\,,
$$
где $\mu_i\in\mathbb{R}$~--- <<полезные>> данные; $z_i \sim N(0,\sigma^2)$~--- 
шум. Задача заключается в~нахождении оценки неизвестного вектора $\mu \hm= 
(\mu_1,\ldots,\mu_n)$ как функции вектора $x \hm= (x_1,\ldots,x_n)$ и~может 
рассматриваться как задача множественной проверки гипотез о~равенстве нулю 
компонент вектора~$\mu$~\cite{AdaptingFDR}. При этом обычно предполагается, что 
вектор~$\mu$ имеет в~определенном смысле <<разреженную>> структуру, т.\,е.\ для 
<<полезных>> данных используется <<экономное>> представление.



В работе~\cite{AdaptingFDR} для решения рассматриваемой задачи в~условиях 
независимости компонент вектора~$x$ и~разреженности вектора~$\mu$ была 
предложена процедура построения оценки~$\hat{\mu}_F$ вектора~$\mu$, основанная 
на методе контроля средней доли ложных отклонений (FDR) 
гипотез при помощи алгоритма Бен\-жа\-ми\-ни--Хох\-бер\-га,
и~было проведено исследование асимптотики ее среднеквадратичного риска. 
В~работах~\cite{ZasShe17,Mathematics2020} была показана состоятельность 
и~асимптотическая нормальность оценки риска данной процедуры. Аналогичные 
результаты для других методов построения~$\hat{\mu}_F$ получены в~работах~\cite{Shestakov2021-1,Shestakov2021-2,Shestakov2022}.

В то же время в~определенных приложениях, например  при анализе полученных 
в~результате использования ДНК-мик\-ро\-чи\-пов данных~\cite{ResultsOnFDRUnderDependence}, исследовании геофизических процессов 
и~анализе помех\linebreak в~телекоммуникационных каналах, условие незави\-си\-мости компонент 
вектора $x$ может не выполняться. Ранее в~работах~\cite{VorontsovShestakov2023,Vorontsov2024} была \mbox{исследована} асимп\-то\-ти\-ка 
среднеквадратичного риска оценки~$\hat{\mu}_F$ \mbox{в~случае}, когда~$\mu$ принадлежит 
одному из классов разреженности
$$
l_0[\eta] = \left\{\mu\,:\, ||\mu||_0 \leq \eta n\right\}, \enskip \eta \in 
(0,1),
$$

\vspace*{-12pt}

\noindent
\begin{multline*}
m_p[\eta] \equiv{}\\
{}\equiv \left\{\mu \in \mathbb{R}^n : |\mu|_{(k)} \leq \eta n^{1/p} 
k^{-1/p},\ k=\overline{1,n}\right\}, \\
 p\in(0, 2),
\end{multline*}
а компоненты вектора~$x$ слабо зависимы~--- имеют достаточно быстро убывающий 
коэффициент сильного перемешивания~\cite{Bosq}

\noindent
\begin{multline*}
\alpha(k) = \sup\limits_{1\leq m\leq n}\alpha\left(\sigma(x_i, i\leq m), 
\sigma(x_i, i\geq m+k)\right), \\ 
k=\overline{1,n-1}\,,
\end{multline*}
где символом $\sigma(x_i, i\in I)$ обозначена сиг\-ма-ал\-геб\-ра, порожденная 
множеством случайных величин $\{x_i, i \hm\in I\}$, а~мера  $\alpha(\cdot, \cdot)$ 
близости двух сиг\-ма-ал\-гебр определяется как
$$
\alpha(\mathcal{B},\mathcal{C}) = \sup\limits_{B\in\mathcal{B}, 
C\in\mathcal{C}} \left|\p(BC)-\p(B)\p(C)\right|.
$$

В настоящей работе показана асимптотическая нормальность и~сильная 
состоятельность оценки риска при применении FDR-про\-це\-ду\-ры в~случае, когда 
компоненты вектора~$x$ слабо зависимы, а~$\mu$ принадлежит одному из классов 
раз\-ре\-жен\-ности: 
$l_0[\eta]$ или $m_p[\eta]$.


\section{Обработка вектора данных с~помощью FDR-процедуры}

Широким классом методов построения оценки~$\hat{\mu}$ стала пороговая обработка 
вектора~$x$ с~некоторым порогом~$T$. Различают жесткую пороговую обработку, при 
которой полагается
\begin{equation*}
\left(\hat{\mu}\right)_i  = p_H(x_i,T) \equiv
 \begin{cases}
   x_i, & |x_i| > T\,;\\
   0, & |x_i| \leq T\,,
 \end{cases}
\end{equation*}
и мягкую пороговую обработку, для которой
\begin{equation*}
(\hat{\mu})_i  = p_S(x_i,T) \equiv
 \begin{cases}
   x_i-T, & \hphantom{\vert\vert}x_i > T;\\
   x_i+T, & \hphantom{\vert\vert}x_i <- T;\\
   0, & |x_i| \leq T.
 \end{cases}
\end{equation*}
Среднеквадратичный риск подобных процедур определяется как
\begin{equation}
\label{riskDef}
R(T) = {\mathsf E} ||\hat{\mu}-\mu||^2 = \sum\limits_{i=1}^n {\mathsf E} \left((\hat{\mu})_i-
\mu_i\right)^2.
\end{equation}
Обозначим через~$T_m$ наилучшее значение порога:
$$
T_m : \, R(T_m) = \min\limits_{T} R(T).
$$

Предложенная в~\cite{AdaptingFDR} процедура заключается в~жесткой пороговой 
обработке компонент вектора~$x$ с~порогом $\hat{t}_F \hm= \hat{t}_F(x)$, и~ее 
результат~--- оценка $\hat{\mu}_F$ вектора~$\mu$ с~компонентами $(\hat{\mu}_F)_i  
\hm= p_H(x_i,\hat{t}_F)$, где
\begin{multline*}
\hat{t}_F = \sigma z\left(\fr{q \hat{k}_F}{2n}\right), \enskip
\hat{k}_F = \max 
\left\{k \, :\, |x|_{(k)} \geq t_k \right\}, \\
 t_k = \sigma z\left(\fr{q  k}{2n}\right);
\end{multline*}
$z(\alpha)$ --- квантиль уровня $(1\hm-\alpha)$ стандартного нормального 
распределения; $|x|_{(k)}$~--- $k$-й элемент вектора, получаемого в~результате 
упорядочения вектора~$|x|$ по невозрастанию:
$$
|x|_{(1)} \geq |x|_{(2)} \geq \cdots \geq |x|_{(n)};
$$
$q\in(0;1)$~--- управ\-ля\-ющий параметр FDR-ме\-то\-да.
Далее полагается, что $q\hm\equiv q_n$ зависит от~$n$. В~\cite{AdaptingFDR} 
показано, что эта процедура эквивалентна множественной проверке гипотез 
о~равенстве нулю компонент наблюдаемого вектора. Также показано, что с~помощью 
метода штрафных функций данную процедуру можно свести к~другим видам пороговой 
обработки, в~част\-ности к~мягкой пороговой обработке.

В работах~\cite{VorontsovShestakov2023, Vorontsov2024} была исследована 
асимптотика среднеквадратичного риска~$R(\hat{t}_F)$ описанной процедуры 
в~случае, когда компоненты вектора $x$ слабо зависимы, а $\mu$ принадлежит классу 
разреженности~$\Theta_n$, где~$\Theta_n$ есть~$l_0[\eta_n]$ или~$m_p[\eta_n]$. 
Было показано, что~$R(\hat{t}_F)$ асимптотически отличается от минимаксного 
риска
$\inf\nolimits_{\hat{\mu}\hm=\hat{\mu}(x)} \sup\nolimits_{\mu\in \Theta_n} {\mathsf E} 
||\hat{\mu}-\mu||^2$
на множитель не более чем логарифмического по\-рядка.

Отметим, что в~выражении для среднеквадратичного риска~(\ref{riskDef}) 
присутствуют неизвестные величины~$\mu_i$, а~потому вычислить~$R(T_m)$ и~$T_m$ 
не представляется возможным. На практике можно пользоваться, например, следующей 
оценкой среднеквадратичного риска~\cite{Mallat}:
$$
\hat{R}(T) = \sum\limits_{i=1}^n F[x_i, T],
$$
где  
\begin{multline*}
F[x_i, T] = {}\\[3pt]
{}=\!\begin{cases}
\left(x_i^2-\sigma^2\right) \Ik(|x_i|\leq T) + \sigma^2 \Ik\left(|x_i|>T\right) &\\[3pt]
&\hspace*{-53mm}\mbox{для\ жесткой\ пороговой\ обработки};\\[3pt]
\left(x_i^2-\sigma^2\right) \Ik\left(|x_i|\leq T\right) + (\sigma^2+T^2) 
\Ik \left(|x_i|>T\right) \hspace*{-11.21576pt}&\\[3pt]
&\hspace*{-51mm}\mbox{для\ мягкой\ пороговой\ обработки}.
\end{cases}\hspace*{-7.17859pt}
\end{multline*}


\noindent
\textbf{Замечание}.\ При пороговой обработке иногда также используется так 
называемый универсальный порог $T_U\hm = \sigma \sqrt{2\ln n}$, предложенный 
в~работе~\cite{spatialAdaptation}. Исследования в~\cite{AdaptingSURE, ExactRisk} 
показали, что порог~$T_U$ в~определенном смысле максимальный, и~рас\-смат\-ри\-вать 
пороги выше него не имеет смысла. Более того, нетрудно показать, что $t_k \hm< T_U$ 
для всех~$k$ и~всех достаточно больших~$n$, в~связи с~чем всюду далее полагаем, 
что порог~$\hat{t}_F$ выбирается на отрезке $[0; T_U]$.

\section{Вспомогательные утверждения}

Кроме коэффициента сильного перемешивания~$\alpha(\cdot)$ также понадобится 
следующее понятие~\cite{Bosq}.

\smallskip

\noindent
\textbf{Определение.} %\label{defRho}
Максимальным коэффициентом корреляции~$\rho(\cdot)$ компонент вектора~$x$ 
называется
\begin{multline*}
\rho (k) \equiv \rho_n (k) = {}\\
{}=\sup\limits_{1\leq m\leq n}\rho\left(\sigma(x_i, 
i\leq m), \sigma(x_i, i\geq m+k)\right), \\
 k=\overline{1,n-1}\,,
\end{multline*}
где мера $\rho(\cdot, \cdot)$ близости двух сиг\-ма-ал\-гебр определяется как
$$
\rho(\mathcal{B},\mathcal{C}) = \sup\limits_{\substack{\xi 
\in\mathcal{L}^2(\mathcal{B}) \\
 \eta \in\mathcal{L}^2(\mathcal{C})}} 
\left|\mathrm{corr}\,(\xi, \eta)\right|.
$$


Введем обозначения:
$$
T_1 = \sqrt{2\ln \eta_n^{-p}};  \,\gamma_n = \fr{1}{\ln\ln n}; \, \kappa_n 
= \fr{n \eta_n^p T_1^{-p}}{1 - q_n - \gamma_n}; 
$$
$$ 
\kappa_n^0 = \fr{[n \eta_n]}{1 - q_n - \gamma_n} ;\, \rho^\star (k) = 
\sup\limits_{n\geq k+1} \rho(k), k \in \mathbb{N} ;
$$
$$
t_{\kappa_n} = \sigma z\left(\fr{q_n \kappa_n }{2n}\right) , \,\, t_{\kappa_n^0} 
= \sigma z\left(\fr{q_n \kappa_n^0 }{2n}\right).
$$


Следующие два утверждения показывают, что случайный порог~$\hat{t}_F$ в~случае 
$\mu\hm\in m_p[\eta_n]$ (соответственно $\mu\hm\in l_0[\eta_n]$) с~большой 
вероятностью будет не меньше~$t_{\kappa_n}$ (соответственно~$ t_{\kappa_n^0}$). 
Их  доказательства приведены в~работах~\cite{VorontsovShestakov2023, Vorontsov2024}.

\smallskip

\noindent
%\begin{lem}\label{lem5}
\textbf{Лемма~1.}\ \textit{Пусть $n^{-\delta_1} \hm\leq \eta_n^p \hm\leq n^{-\delta_2}$, 
$0\hm<\delta_2\hm<\delta_1<1$, $\mathrm{lim\,inf} q_n \ln n \hm\geq C \hm> 0$, 
$m\hm\in[1;n/2]\cap\mathbb{N}$, а $\alpha(\cdot)$~--- коэффициент сильного 
перемешивания компонент вектора~$x$. Для некоторого $N\hm\in\mathbb{N}$ при $n \hm\geq 
N$ справедливо}
\begin{multline*}
\hspace*{-3pt}\sup\limits_{\mu\in m_p[\eta_n]} \p \left(\hat{k}_F \geq \kappa_n \right) \leq 
4 n \exp\left\{-\fr{m}{256n}  \kappa_n q_n \gamma_n^2    \right\}+{}\\
{}+ 22\left(1+\fr{8n}{\kappa_n q_n \gamma_n}\right)^{1/2} n m 
\alpha\left(\left[\fr{n}{2m}\right]\right).
\end{multline*}



\smallskip

\noindent
\textbf{Лемма 2.}\ 
%\label{lem1}
\textit{Пусть $\eta_n \hm\leq b\hm<1$, $m\in[1;n/2]\cap\mathbb{N}$, а~$\alpha(\cdot)$~--- 
коэффициент сильного перемешивания компонент вектора~$x$. Для некоторого 
$N\hm\in\mathbb{N}$ при $n \hm\geq N$ справедливо}
\begin{multline*}
\sup\limits_{\mu\in l_0[\eta_n]} \p \left(\hat{k}_F \geq \kappa_n^0 \right) 
\leq{}\\
{}\leq 4 n \exp\left\{-\fr{(1-b)m}{64n}\,  \kappa_n^0 q_n \gamma_n^2    
\right\}+{}\\
{}+ 22\left(1+\fr{4n}{(1-b)\kappa_n^0 q_n \gamma_n}\right)^{1/2} n m 
\alpha\left(\left[\fr{n}{2m}\right]\right).
\end{multline*}

Следующие два утверждения доказаны в~\cite{Bosq} и~представляют собой аналоги 
неравенств Хеффдинга и~Бернштейна для слабо зависимых случайных величин.


\smallskip

\noindent
\textbf{Лемма 3.}\
\textit{Пусть для набора действительных случайных величин $X_1, \ldots, X_n$ 
с~коэффициентом сильного перемешивания $\alpha(\cdot)$ выполняется ${\mathsf E} X_i \hm=0$, 
$|X_i|\hm\leq b$, $i\hm=\overline{1,n}$. Тогда для любого целого числа $m\hm\in[1; n/2]$ 
и~любого $\eps\hm>0$ справедливо}
\begin{multline*}
\p\left(\left|\sum\limits_{i=1}^n X_i\right| > n\eps \right) \leq 4 
\exp\left\{-\fr{\eps^2 m}{8 b^2}\right\}+ {}\\
{}+
22\left(1+\fr{4b}{\eps}\right)^{1/2} m\, 
\alpha\left(\left[\fr{n}{2m}\right]\right).
\end{multline*}


\smallskip

\noindent
\textbf{Лемма 4.}\
\textit{Пусть для набора действительных случайных величин $X_1, \ldots, X_k$ 
с~коэффициентом сильного перемешивания $\alpha(\cdot)$ выполняется ${\mathsf E} X_i \hm=0$, 
$|X_i|\hm\leq b$, $i\hm=\overline{1,k}$. Тогда для любого целого числа $m\hm\in[1; k/2]$ 
и~любого $\eps\hm>0$ справедливо}
\begin{multline*}
\p\left(\left|\sum\limits_{i=1}^k X_i\right| > \eps \right) \leq 4 
\exp\left\{-\fr{\eps^2 m}{8 v^2 k^2}\right\}+{}\\
{}+ 22\left(1+\fr{4bk}{\eps}\right)^{1/2} m\, 
\alpha\left(\left[\fr{k}{2m}\right]\right),
\end{multline*}
\textit{где $p = k/(2m)$}:
\begin{multline*}
v^2 =
 \fr{b \eps}{2k} + {}\\
 {}+\fr{2}{p^2} \,  \max\limits_{ j\in[0,\,2m-1]} 
{\mathsf E} \big( ([jp]+1-jp)X_{[jp]+1} + X_{[jp]+2}+{}\\
{}+ \cdots +  X_{[(j+1)p]} + ((j+1)p-[(j+1)p])X_{[(j+1)p+1]}\big)^2.
\end{multline*}

\noindent
\textbf{Замечание.}
Если существует такое число $S \hm> 0$, что сразу для всех $i\hm\in[1;k]$  выполняется 
${\mathsf E} X_i^2 \hm\leq S^2$, то в~качестве~$v^2$ можно взять
$$
v^2 = \fr{b \eps}{2k} + 8 S^2.
$$


Д\,о\,к\,а\,з\,а\,т\,е\,л\,ь\,с\,т\,в\,о\ \ сле\-ду\-юще\-го утверж\-де\-ния приведено в~работе~\cite{AdaptingFDR}.

\smallskip

\noindent
\textbf{Лемма 5.}\ 
\textit{Для $y\leq 0{,}01$ справедливы представления}
\begin{multline}
\label{lem1eq1}
z^2(y) = 2 \ln y^{-1} - \ln \ln y^{-1} - r_2(y), \\
 r_2(y) \in [1{,}8; 3];
\end{multline}

\noindent
\begin{equation}
\label{lem1eq2}
z(y) = \sqrt{2 \ln y^{-1}} - r_1(y), \, \, r_1(y) \in [0; 1{,}5].
\end{equation}


\section{Асимптотическая нормальность оценки риска при~применении FDR-процедуры в~условиях слабой зависимости}

Перейдем к~описанию достаточных условий для асимптотической нормальности оценки 
риска $\hat{R}(\hat{t}_F)$ в~случае $\mu \hm\in m_p[\eta_n]$.

\smallskip

\noindent
\textbf{Теорема~1.}\
\textit{Пусть $\mu \hm\in m_p[\eta_n],$ $\eta_n^p \hm\in[n^{-\delta_1}; n^{-\delta_2}],$ $1/2 \hm< 
\delta_2 \hm< \delta_1<1;$ имеются такие константы $c_1, c_2>0$, что для 
коэффициента сильного перемешивания $\alpha(\cdot)$ компонент вектора $x$ 
справедливо  $\alpha(k) \hm\leq c_1 k^{-1-(5/2)\delta_1/(1-\delta_1)-c_2},$ 
$k\hm=\overline{1,n-1};$ $q_n \hm< c_3 \hm< 1;$ $\mathrm{lim\,inf} q_n \ln n \hm= c_4 \hm> 0;$ и,~кроме того, 
для максимального коэффициента корреляции $\rho(\cdot)$ компонент вектора~$x$ 
справедливо}
$$
\sum\limits_{k = 1}^{\infty} \sup\limits_{n\geq k+1} \rho(k) \equiv 
\sum\limits_{k = 1}^{\infty}  \rho^\star (k) = c_5 < \infty. 
$$
\textit{Тогда при $n \to \infty$}
$$
\fr{\hat{R}(\hat{t}_F) - R(T_m)}{C_\rho \sqrt{2n}} \Rightarrow N(0, 1),
$$
\textit{где}
$$
C_\rho = \sigma^2\sqrt{1 +  \lim\limits_{n\to\infty} \fr{1}{n} \sum\limits_{j\neq i} \mathrm{corr}^2 (x_i, x_j)}.
$$

\noindent
Д\,о\,к\,а\,з\,а\,т\,е\,л\,ь\,с\,т\,в\,о\  \
 приводится для метода мягкой пороговой обработки; в~случае жесткой пороговой 
обработки доказательство аналогично. Обозначим
$$
U(T) = \hat{R}(T) -  \hat{R}(T_m) = \sum \limits_{i=1}^n H_i(T, T_m),
$$
где
$$
H_i(T, T_m) = F[x_i, T] - F[x_i, T_m].
$$
Имеем

\vspace*{-3pt}

\noindent
\begin{multline}
\label{D00}
\hat{R}(\hat{t}_F) - R(T_m) + \hat{R}(T_m) - \hat{R}(T_m) ={}\\
{}= \hat{R}(T_m) - 
R(T_m) + U(\hat{t}_F).
\end{multline}
Покажем, что
\begin{equation}
\label{D0}
\fr{\hat{R}(T_m) - R(T_m)}{C_\rho\sqrt{2n}} \Rightarrow N(0, 1).
\end{equation}


Повторяя рассуждения из~\cite{KuShe2016_1,KuShe2016_2,Jansen}, можно показать, 
что $T_m \hm\geq t_{\kappa_n}$. Учитывая также $T_m\hm \leq T_U$, имеем 
$$
C \sqrt{\ln n} \leq T_m \leq C^\prime \sqrt{\ln n}
$$ 
для некоторых положительных констант $C$ и~$C^\prime$.

\columnbreak

В случае мягкой пороговой обработки $\hat{R}(T_m)$ представляет собой 
несмещенную оценку~$R(T_m)$, а~при жесткой пороговой обработке и~выполнении 
условий теоремы смещение стремится к~нулю при делении на $\sqrt{n}$~\cite{Mallat}.

Для дисперсии числителя~(\ref{D0}) имеем:
\begin{multline*}
{\mathsf D} \left(\hat{R}(T_m) - R(T_m)\right) = \sum\limits_{i=1}^n {\mathsf D} F[x_i, T_m] + {}\\
{}+
\sum\limits_{i=1}^n\sum\limits_{\substack{j=1 \\  j\neq i}}^n \mathrm{cov}\left( F[x_i, T_m], F[x_j, 
T_m] \right).
\end{multline*}

Поскольку $\mu \in m_p[\eta_n]$,
\begin{equation}
\left.
\begin{array}{l}
 \displaystyle\sum\limits_{i: |\mu_i| > 1/T_1} {\mathsf D} F[x_i, T_m]  \leq{}\\
 \hspace*{15mm}{}\leq  4\left(\sigma^2 + T_m^2\right)^2 n \eta_n^p 
T_1^p = o(n);
\\[6pt]
\displaystyle \sum\limits_{\substack{{i,j: \max\{|\mu_i|, |\mu_j|\} > 1/T_1,}\\{j\neq i}}}  \hspace*{-12mm}\mathrm{cov}\,(F[x_i, 
T_m],F[x_j, T_m])  \leq{}\\
\hspace*{10mm}{}\leq 16\left(\sigma^2 + T_m^2\right)^2 n \eta_n^p T_1^p c_5 = o(n). 
\end{array}
\right\}    
\label{D2}
\end{equation}
Далее, учитывая что ${\mathsf D} x_i^2 \hm= 2\sigma^4 \hm+ 4\sigma^2 \mu_i^2$, нетрудно 
убедиться, что
\begin{multline}
\label{D3}
\sum\limits_{i: |\mu_i| \leq 1/T_1}\hspace*{-4mm} {\mathsf D} F[x_i, T_m] ={}\\
{}= \sum\limits_{i: |\mu_i| \leq 1/T_1} \hspace*{-4mm} {\mathsf D} 
x_i^2 + o(n) = 2\sigma^4 n + o(n).
\end{multline}


Введем обозначение 
$$
D_n = \left\{(i,j) : \max\left\{|\mu_i|, |\mu_j|\right\}  \leq \fr{1}{T_1}\,, \enskip j\hm\neq i\right\}.
$$
 Для суммы ковариаций аналогично~(\ref{D3}) получим
\begin{multline*}
\sum\limits_{(i,j)\in D_n} \hspace*{-2mm}\mathrm{cov}\left( F[x_i, T_m], F[x_j, T_m] \right) = {}\\
{}=
\sum\limits_{(i,j)\in D_n} \hspace*{-2mm}\mathrm{cov}\left( x_i^2, x_j^2 \right) + o(n).
\end{multline*}
Воспользуемся тождеством~\cite{Eroshenko}
$$
\mathrm{cov}\left (x_i^2, x_j^2\right) = 4 {\mathsf E} x_i {\mathsf E} x_j \mathrm{cov}\left(x_i, x_j\right) + 2 \mathrm{cov}^2 \left(x_i, x_j\right)
$$
для вектора $(x_i, x_j)$, имеющего двумерное нормальное распределение. Заметим, 
что
\begin{gather*}
 \sum\limits_{(i,j)\in D_n} 4 | {\mathsf E} x_i {\mathsf E} x_j \mathrm{cov}\left(x_i, x_j\right)| \leq 8 T_1^{-2} 
\sigma^2 n c_5 = o(n);
\\
\sum\limits_{(i,j)\in D_n} 2 \mathrm{cov}^2 (x_i, x_j)  = 2\sigma^4 \sum\limits_{(i,j)\in D_n} 
\mathrm{corr}^2 (x_i, x_j). 
\end{gather*}
Более того, поскольку  %< 4 \sigma^2 n c_5.$$
\begin{equation*}
\sum\limits_{\substack{{i,j: \max\{|\mu_i|, |\mu_j|\} > 1/T_1} \\ {j\neq i}}}
\hspace*{-10mm}\mathrm{corr}^2 (x_i, x_j)  
\leq  4 n \eta_n^p T_1^p c_5 =  o(n),
\end{equation*}
имеем
\begin{multline*}
\sum\limits_{(i,j)\in D_n} \mathrm{corr}^2 (x_i, x_j) ={}\\
{}= \sum\limits_{j\neq i} \mathrm{corr}^2 (x_i, x_j) 
+o(n)= c_6 n + o(n),
\end{multline*}
где
$$
c_6 = \lim\limits_{n\to\infty} \fr{1}{n} \sum\limits_{j\neq i} \mathrm{corr}^2 (x_i, x_j) 
\leq 2 c_5.
$$
Полагая $C_\rho \hm= \sigma^2\sqrt{1 + c_6}$, получим, наконец,
\begin{equation}
\label{D1}
{\mathsf D} \left(\hat{R}(T_m) - R(T_m)\right)  =  2 n C_\rho^2 + o(n).
\end{equation}
Заметим, что из~(\ref{D2}), (\ref{D3}) и~(\ref{D1}) следует, что
\begin{equation}
\label{D5}
\sup\limits_{n} \fr{\sum\nolimits_{i=1}^n {\mathsf D} F[x_i, T_m]}{V_n^2} < \infty\,,
\end{equation}
где 
$$
V_n^2 = {\mathsf D} \sum\limits_{i=1}^n \left(F[x_i, T_m] \hm- {\mathsf E} F[x_i, T_m]\right).
$$
Кроме того, поскольку $F[x_i, T_m]$ по модулю ограничены величиной $\sigma^2 \hm+ 
T_m^2$, выполнено условие Линдеберга: для любого $\eps\hm>0$ при $n \hm\to \infty$
\begin{multline}
\label{D6}
\!\!\!\fr{1}{V_n^2}\sum\limits_{i=1}^n {\mathsf E} \left( \!\left( F\left[x_i, T_m\right]\! -\! {\mathsf E} F\left[x_i, T_m\right]\right)^2 
\Ik \left(\vert F\left[x_i, T_m\right] -{}\right.\right.\hspace*{-2.69505pt}\\
\left.\left.{}- {\mathsf E} F\left[x_i, T_m\right]\vert >\eps V_n\right)\!
\vphantom{\left( F\left[x_i, T_m\right]\! -\! {\mathsf E} F\left[x_i, T_m\right]\right)^2}
\right) 
\to  0\,.
\end{multline}
Из~(\ref{D1})--(\ref{D6}), очевидного неравенства
$$ 
\lim\limits_{k\to\infty} \sup\limits_{n\geq k+1}\rho(k) \equiv 
\lim\limits_{k\to\infty} \rho^\star (k)  < 1
$$
 и~центральной предельной теоремы для сильно перемешанных случайных величин~\cite{Peligrad} следует~(\ref{D0}).

Перейдем к~доказательству того, что $U(\hat{t}_F) \, n^{-1/2} \overset{\, \p \, }{\to} 0$.
Всюду далее, не ограничивая общности, полагаем $\sigma=1$. 
Введем обозначения:

\noindent
\begin{align*}
S_1(T) &= \sum\limits_{i: |\mu_i| > 1/T_1} H_i(T, T_m); \\
S_2(T) &= \sum\limits_{i: |\mu_i| \leq 1/T_1} H_i(T, T_m); 
\\
N_1(a, b) &= \sum\limits_{i: |\mu_i| > 1/T_1} \Ik (a<|x_i|\leq b); \\ 
N_2(a, b) &= \sum\limits_{i: |\mu_i| \leq 1/T_1} \Ik (a<|x_i|\leq b);
\end{align*}

\noindent
\begin{align*}
Z_l(T) &= S_l(T) - {\mathsf E} S_l(T),\enskip l = 1,2\,; \\  
d_n &= \fr{T_U -  t_{\kappa_n}}{n};\\
T_j^{\prime} &= t_{\kappa_n}+j d_n,\enskip j = \overline{0,n-1}\,.
\end{align*} 

\vspace*{-3pt}

\noindent
Для произвольного $\eps>0$

\vspace*{-3pt}

\noindent
\begin{multline}
\p \left( \fr{|U(\hat{t}_F)|}{\sqrt{n}}> 4\eps \right) \leq 
\p\left(\hat{t}_F \leq t_{\kappa_n}\right) + {}\\
{}+\p \left(\fr{\sup\nolimits_{T\in 
[t_{\kappa_n}, T_U]} |U(T)|}{\sqrt{n}}>4\eps \right)\leq  {}\\
{}\leq \p\left(\hat{t}_F \leq t_{\kappa_n}\right) + \p\left(\fr{\sup\nolimits_{T\in 
[t_{\kappa_n}, T_U]} |{\mathsf E} U(T)|}{\sqrt{n}}>\eps\right)+{}\\
{}+ \p \left(\sup\limits_{T\in [t_{\kappa_n}, T_U]} |Z_1(T)| > 
\eps\sqrt{n}\right) +{}\\
{}+ \p \left(\sup\limits_{j \in [0, n-1]} |Z_2(T_j^{\prime})| > 
\eps\sqrt{n}\right) +{}\\
{}+ \p \left(\sup\limits_{\substack{j \in [0, n-1] \\
 T\in [T_j^{\prime},T_j^{\prime}+d_n]}} |Z_2(T)-Z_2(T_j^{\prime})| > \eps\sqrt{n}\right).
\label{M1}
\end{multline}
Заметим, что $\gamma_n\hm > \ln^{-1} n$, $\kappa_n\hm > n \eta_n^p \ln ^{-1} n \hm\geq 
n^{1-\delta_1} \ln ^{-1} n$ и~$q_n\hm > c_4 \ln ^{-1} n /2$ для всех достаточно 
больших~$n$.
Для первого слагаемого в~(\ref{M1}) по лемме~1 с~$m \hm= n^{\delta_1} \ln 
^7 n$ для  больших~$n$ имеем

\vspace*{-3pt}

\noindent
\begin{multline}
\label{M1next}
\p\left(\hat{t}_F \leq t_{\kappa_n}\right)  = \p \left(\hat{k}_F \geq \kappa_n 
\right) \leq 4 n e^{-\ln^2 n} + {}\\
{}+n^{1+(3/2)\,\delta_1} \ln^9 n \, 
\alpha\left(\left[\fr{n^{1-\delta_1}}{\ln^{7} n}\right]\right) = o(1)
\end{multline}
при $n\to\infty$. 
Для оценки второго слагаемого в~(\ref{M1}) заметим, что при $T \hm\in 
[t_{\kappa_n}, T_U]$ справедливо
\begin{equation}
\label{M2}
{\mathsf E} H_i(T, T_m) \leq T_U^2 + 1.
\end{equation}
Если же кроме $T \hm\in [t_{\kappa_n}, T_U]$ также выполнено $|\mu_i| \hm\leq T_1^{-1}$, то

\vspace*{-6pt}

\noindent
\begin{multline*}
|{\mathsf E} H_i (T, T_m)| \leq 2 T_U^2 \, \p \left(|x_i| > t_{\kappa_n}\right) \leq {}\\
{}\leq2 
T_U^2 \, \p \left(|x_i-\mu_i| > t_{\kappa_n}-T_1^{-1}\right) \leq{}\\
{}\leq 2 T_U^2  \exp\left\{ -\fr{1}{2} \left(t_{\kappa_n} - T_1^{-
1}\right)^2 \right\}  \leq{}\\
{}\leq
 4 (\ln n)  \exp\left\{ -\fr{1}{2} 
\left(z\left(\fr{q_n\kappa_n}{2n}\right)\right)^2 + t_{\kappa_n} T_1^{-
1}\right\},
\end{multline*}

\vspace*{-2pt}

\noindent
где использовано неравенство 

\noindent
$$
2(1-\Phi(x))\hm \leq \fr{e^{-x^2/2}}{x}
$$

\pagebreak


\noindent
 для $x\hm\geq 0$ 
($\Phi(x)$~--- функция распределения $N(0,1)$). Рас\-смот\-рим выражение 
в~экспоненте. Второе слагаемое не превышает $1\hm+o(1)$ при $n\hm\to\infty$, поскольку 
$t_{\kappa_n} \hm\leq T_1 (1+o(1))$ при $\sigma\hm=1$, что нетрудно получить из 
определения~$t_{\kappa_n}$, пред\-став\-ле\-ния~(\ref{lem1eq2}) и~ограничения на~$q_n$ 
из формулировки тео\-ре\-мы. Для первого слагаемого, используя пред\-став\-ле\-ние~(\ref{lem1eq1}) 
и~ограничения, наложенные на~$q_n$, при больших~$n$ получим
\begin{multline*}
-\fr{1}{2}\left(z\left(\fr{q_n \kappa_n}{2n}\right)\right)^2 \leq - \ln 
\fr{2n (1-q_n-\gamma_n)}{q_n n \eta_n^p T_1^{-p}} + {}\\
{}+\fr{1}{2} \ln 
\left((1+o(1)) \ln \eta_n^{-p}\right) + \fr{3}{2} \leq{}\\
{}\leq \ln \fr{c_3}{1-c_3} + \ln \eta_n^p + \ln T_1^{-p} + \ln T_1 + 
\fr{3}{2}+ o(1).
\end{multline*}
Из приведенных соотношений следует, что с~некоторой константой $c_7 = c_7(c_3, 
p, \delta_1, \delta_2, c_4)$
\begin{equation}\label{M3}
\sup\limits_{\substack{i: |\mu_i| \leq 1/T_1 \\ T\in [t_{\kappa_n}, T_U]}} |{\mathsf E} 
H_i (T, T_m)|  \leq c_7 (\ln n)^{(3-p)/2}\eta_n^p.
\end{equation}
Из (\ref{M2}) и~(\ref{M3}) с~учетом $\delta_2 \hm> 1/2$ следует
\begin{multline*}
\sup\limits_{T\in [t_{\kappa_n}, T_U]} |{\mathsf E} U(T)| \leq{}\\
{}\leq 
 n\eta_n^p T_1^p 
(T_U^2+1) + c_7 (\ln n)^{(3-p)/2} n \eta_n^p = o(\sqrt{n})
\end{multline*}
при $n\to\infty$, а следовательно, для любого $\eps\hm>0$ второе слагаемое в~(\ref{M1}) обращается в~ноль для всех достаточно больших~$n$.

Далее, поскольку при $T \hm\leq T_U$ и~$\sigma\hm=1$
$$
|H_i(T, T_m) - {\mathsf E} H_i(T, T_m)| \leq 2 (T_U^2 +2), \enskip i=\overline{1, n}\,,
$$
а число слагаемых в~$Z_1(T)$ не превосходит $n\eta_n^p T_1^p$, имеем
$$
\sup\limits_{T\in [t_{\kappa_n}, T_U]} |Z_1(T)|  \leq 2 n\eta_n^p T_1^p (T_U^2 
+2) = o(\sqrt{n})
$$
при $n\to\infty$, а следовательно, для любого $\eps\hm>0$ и~третье слагаемое в~(\ref{M1}) обращается в~ноль для всех достаточно больших~$n$.

Перейдем к~оценке четвертого слагаемого в~(\ref{M1}). Аналогично~(\ref{M3}) 
можно получить:
\begin{multline}
\label{M10}
\!\!\sup\limits_{\substack{i: |\mu_i| \leq 1/T_1 \\ T\in [t_{\kappa_n}, T_U]}} \!{\mathsf D} 
H_i (T, T_m)  \leq \!\sup\limits_{\substack{i: |\mu_i| \leq 1/T_1 \\ T\in 
[t_{\kappa_n}, T_U]}} \!{\mathsf E} \left(H_i (T, T_m)\right)^2  \leq{}\\
{}\leq 2 c_7 (\ln n)^{(5-p)/2} \eta_n^p.
\end{multline}
По лемме~4 с~$m \hm= \sqrt{n} (\ln n)^3$ и~$k \hm= n-[n\eta_n^p T_1^p]$ 
для четвертого слагаемого в~(\ref{M1}) имеем:

\noindent
\begin{multline}
\p \left(\sup\limits_{j \in [0, n-1]} |Z_2(T_j^\prime)| > \eps\sqrt{n}\right) 
\leq {}\\
{}\leq \sum\limits_{j \in [0, n-1]} \hspace*{-3mm}\p \left( |Z_2(T_j^\prime)| > \varepsilon\sqrt{n}\right)\leq{}\\
{}\leq 4 n \exp \left\{ - \fr{\eps^2 n^{3/2} (\ln n)^3}{n-[n\eta_n^p T_1^p]}\!\Bigg/\! \big( 8 (T_U^2+2)\eps\sqrt{n} +{}\right.\\
\left.{}+ 128 c_7 (\ln n)^{(5-p)/2} \eta_n^p  (n-
[n\eta_n^p T_1^p])\big) 
\vphantom{ \fr{\eps^2 n^{3/2} (\ln n)^3}{n-[n\eta_n^p T_1^p]}}
\right\} +{}\\
{}
+ 22 \left(1+\fr{8(T_U^2+2) (n-[n\eta_n^p T_1^p])}{\eps 
\sqrt{n}}\right)^{1/2}\times{}\\
{}\times n^{3/2} (\ln n)^3 \alpha\left(\left[\fr{n-[n\eta_n^p 
T_1^p]}{2 (\ln n)^3 \sqrt{n}}\right]\right).
\label{M5}
\end{multline}
Используя ограничения $n^{-\delta_1}\hm\leq \eta_n^p \leq n^{-\delta_2}$ 
и~$1/2\hm<\delta_2\hm<\delta_1\hm<1$, из~(\ref{M5}) получим для любого $\eps\hm>0$
$$
\p \left(\sup\limits_{j \in [0, n-1]} |Z_2(T_j^\prime)| > \eps\sqrt{n}\right) 
\to 0
$$
при $n \to \infty$.

Рассмотрим, наконец, пятое слагаемое в~(\ref{M1})). Заметим, что при $0\hm< a \hm< b$ 
справедливо
$$
|Z_2(b)-Z_2(a)| \leq 2 |N_2(a,b)-{\mathsf E} N_2(a,b)| + n (b^2-a^2).
$$
Полагая $a = T_j^\prime$, $b \hm= T \hm\in [T_j^\prime, T_j^\prime+d_n]$ для 
произвольного $j \hm\in [0, n-1]$ и~учитывая, что
$$
(T^2 - (T_j^\prime )^2) = (T - T_j^\prime)(T+ T_j^\prime ) \leq  2 d_n T_U < 2 
T_U^2 n^{-1}; 
$$

\vspace*{-12pt}

\noindent
\begin{multline*}
\p\left(T_j^\prime < |x_i| \leq T \right) \leq \p\left(T_j^\prime < |x_i| \leq 
T_j^\prime+d_n\right) <{}\\
{}< d_n < T_U n^{-1}, 
\end{multline*}
получим  оценку
$$
|Z_2(T)-Z_2(T_j^\prime)| \leq 2 N_2(T_j^\prime, T) +  3 T_U^2 .
$$
Далее, поскольку $N_2 (T_j^\prime, T) \hm\leq N_2 (T_j^\prime, T_j^\prime+d_n)$ и~${\mathsf E} N_2 (T_j^\prime, T_j^\prime+d_n) \hm< T_U^2$,
имеем
\begin{multline*}
\sup\limits_{T \in [T_j^\prime, T_j^\prime+d_n]} |Z_2(T)-Z_2(T_j^\prime)| \leq {}\\
{}\leq
2 \left|N_2 (T_j^\prime, T_j^\prime+d_n) - {\mathsf E} N_2 (T_j^\prime, 
T_j^\prime+d_n)\right| +  5 T_U^2 .
\end{multline*}
Аналогично~(\ref{M3}) показывается, что
\begin{multline}
\label{M11}
\sup\limits_{\substack{i : |\mu_i| \leq 1/T_1 \\ j \in [0, n-1]}} {\mathsf D} \Ik 
(T_j^\prime < |x_i| \leq T_j^\prime + d_n) <{}\\
{}< c_7 (\ln n)^{(1-p)/2} \eta_n^p.
\end{multline}
Пусть $n > N(\eps)$ настолько, что 
$$
\fr{\eps\sqrt{n} - 5 T_U^2}{2} > \fr{\eps \sqrt{n} }{4}\,.
$$
%
 Тогда для пятого слагаемого в~(\ref{M1}) по лемме~4 с~$m \hm= 
\sqrt{n} (\ln n)^2$ и~$k \hm= n\hm-[n\eta_n^p T_1^p]$ имеем
\begin{multline}
\p \left(\sup\limits_{\substack{j \in [0, n-1] \\ T\in 
[T_j^{\prime},T_j^{\prime}+d_n]}} |Z_2(T)-Z_2(T_j^{\prime})| > 
\eps\sqrt{n}\right) \leq{}\\
{}\leq  \sum\limits_{j \in [0, n-1]} \p \left(  \left|N_2 (T_j^\prime, 
T_j^\prime+d_n) -{}\right.\right.\\
\left.\left.{}- {\mathsf E} N_2 (T_j^\prime, T_j^\prime+d_n)\right| > \fr{\eps\sqrt{n}}{4} 
\right) \leq{}\\
{}\leq  4n \exp \left\{ -  \fr{\eps^2 n^{3/2} (\ln n)^2}{(n-[n\eta_n^p T_1^p])^{-1}}\Bigg/ 
\big( 16 \eps \sqrt{n} +{}\right.\\
\left.{}+ 64 c_7 (\ln n)^{(1-p)/2} \eta_n^p (n-[n\eta_n^p 
T_1^p]) \big) 
\vphantom{\fr{\eps^2 n^{3/2} (\ln n)^2}{(n-[n\eta_n^p T_1^p])^{-1}}}
\right\} +{}\\
{}+ 22 \left(1+\fr{16 (n-[n\eta_n^p T_1^p])}{\eps \sqrt{n}}\right)^{1/2}\times{}\\
{}\times 
n^{3/2} (\ln n)^2 \alpha\left(\left[\fr{n-[n\eta_n^p T_1^p]}{2 (\ln n)^2 
\sqrt{n}}\right]\right).
\label{M6}
\end{multline}
Используя ограничения $n^{-\delta_1}\hm\leq \eta_n^p\hm \leq n^{-\delta_2}$ 
и~$1/2\hm<\delta_2\hm<\delta_1<1$, из~(\ref{M6}) получим для любого $\eps\hm>0$
$$
\p \left(\sup\limits_{\substack{j \in [0, n-1] \\ T\in 
[T_j^{\prime},T_j^{\prime}+d_n]}} |Z_2(T)-Z_2(T_j^{\prime})| > 
\eps\sqrt{n}\right) \to 0
$$
при $n \to \infty$.

Таким образом, показано, что для любого $\eps>0$ все слагаемые в~(\ref{M1}) 
стремятся к~нулю при $n\to\infty$. Следовательно,
$$
\fr{|U(\hat{t}_F)|}{\sqrt{n}}  \overset{\, \p \, }{\to} 0 \,,
$$
что вместе с~(\ref{D0}) завершает доказательство тео\-ремы.~\hfill$\square$

\smallskip

Следующая теорема дает достаточные условия для асимптотической нормальности 
оценки риска $\hat{R}(\hat{t}_F)$ в~случае $\mu \hm\in l_0[\eta_n]$.

\smallskip

\noindent
\textbf{Теорема 2.}\ 
\textit{Пусть $\mu \hm\in l_0[\eta_n]$, $\eta_n\hm\in[n^{-\delta_1}, n^{-\delta_2}]$, $1/2\hm < 
\delta_2\hm < \delta_1\hm<1;$ имеются такие константы $c_1, c_2\hm>0$, что для 
коэффициента сильного перемешивания $\alpha(\cdot)$ компонент вектора~$x$ 
справедливо} 
\begin{gather*}
\alpha(k) \leq c_1 k^{-1-(5/2)\delta_1/(1\hm-\delta_1)\hm-c_2},\enskip 
k=\overline{1,n-1};\\
 q_n < c_3 < 1;\enskip \mathrm{lim\,inf} q_n \ln n = c_4 > 0;
\end{gather*}
\textit{для максимального коэффициента корреляции~$\rho(\cdot)$ компонент вектора~$x$ 
справедливо}
$$
\sum\limits_{k = 1}^{\infty} \sup\limits_{n\geq k+1} \rho(k) \equiv 
\sum\limits_{k = 1}^{\infty}  \rho^\star (k) = c_5 < \infty. 
$$
\textit{Тогда при $n \to \infty$}
$$
\fr{\hat{R}(\hat{t}_F) - R(T_m)}{C_\rho \sqrt{2n}} \Rightarrow N(0, 1),
$$
\textit{где}
$$
C_\rho = \sigma^2\sqrt{1 +   \lim\limits_{n\to\infty} \fr{1}{n} 
\sum\limits_{j\neq i} \mathrm{corr}^2 (x_i, x_j)}\,.
$$

\noindent
Д\,о\,к\,а\,з\,а\,т\,е\,л\,ь\,с\,т\,в\,о\  проводится аналогично доказательству теоремы~1. 
Переменная~$D_n$ теперь определяется как $D_n \hm= \{(i,j) : 
|\mu_i|\hm=|\mu_j|=0$, $j\hm\neq i\}$. Условия вида $|\mu_i|\hm<T_1^{-1}$ (вида 
$|\mu_i|\hm\geq T_1^{-1}$) заменяются условиями  $\mu_i\hm=0$ (соответственно 
$|\mu_i|\hm>0$).
Поскольку $\mu \hm\in l_0[\eta_n]$, количество~$i$ таких, что $|\mu_i|\hm>0$ 
(а~значит, и~число слагаемых в~$Z_1(T)$), не превышает~$[n \eta_n]$.

Для оценки первого слагаемого в~(\ref{M1}) используется лемма~2, 
в~которой можно взять, например, $b\hm=1/2$, а~для~$\kappa_n^0$ использовать оценку 
$\kappa_n^0 \hm> n \eta_n$. Формулы (\ref{M3}),  (\ref{M10}) и~(\ref{M11}) 
принимают вид соответственно
\begin{align*}
\sup\limits_{\substack{i: \mu_i =0 \\ T\in [t_{\kappa_n^0}, T_U]}} |{\mathsf E} H_i (T, 
T_m)| & \leq c_8 (\ln n)^{3/2} \eta_n ;
\\
\sup\limits_{\substack{i: \mu_i =0 \\ T\in [t_{\kappa_n^0}, T_U]}} {\mathsf D} H_i (T, 
T_m)  & \leq 2 c_8 (\ln n)^{5/2} \eta_n;
\\
\sup\limits_{\substack{i : \mu_i =0 \\ j \in [0, n-1]}} {\mathsf D} \Ik (T_j^\prime < 
|x_i| \leq T_j^\prime + d_n) &< c_8 (\ln n)^{1/2} \eta_n,
\end{align*}
где $c_8 = c_8(c_3,\delta_1, \delta_2, c_4)$. В~остальном доказательство 
аналогично.~\hfill$\square$

\section{Сильная состоятельность оценки риска при~применении FDR-процедуры 
в~условиях слабой зависимости}

Следующая теорема дает достаточные условия для сильной состоятельности оценки 
риска $\hat{R}(\hat{t}_F)$ в~случаях $\mu \hm\in m_p[\eta_n]$ и~$\mu \hm\in 
l_0[\eta_n]$.

\smallskip

\noindent
\textbf{Теорема 3.}
\textit{Пусть $\mu\hm \in m_p[\eta_n]$, $\eta_n^p\hm\in[n^{-\delta_1}, n^{-\delta_2}]$ либо 
$\mu \hm\in l_0[\eta_n]$, $\eta_n\hm\in[n^{-\delta_1}, n^{-\delta_2}]$; $0 \hm< \delta_2 
\hm< \delta_1<1$; имеются такие константы $c_1, c_2\hm>0$, что для коэффициента 
сильного перемешивания $\alpha(\cdot)$ компонент вектора~$x$ справедливо}  
$\alpha(k) \hm\leq c_1 k^{-2-(7/2)\delta_1/(1\hm-\delta_1)\hm-c_2}$, $k\hm=\overline{1,n-1}$; 
$q_n \hm< c_3 \hm< 1$; $\mathrm{lim\,inf} q_n \ln n \hm= c_4 \hm> 0$. \textit{Тогда при} $n \hm\to \infty$
$$
\fr{\hat{R}(\hat{t}_F) - R(T_m)}{n} \rightarrow 0 \, \, \,\textit{п.~в.}
$$


\noindent
Д\,о\,к\,а\,з\,а\,т\,е\,л\,ь\,с\,т\,в\,о\,.  Воспользуемся представлением~(\ref{D00}).

Покажем, что $(\hat{R}(T_m)-R(T_m))n^{-1}\hm \to 0$ п.~в.\ при $n\hm\to\infty$. 
При мягкой пороговой обработке ${\mathsf E} \hat{R}(T_m) \hm= R(T_m)$, а~при жесткой 
пороговой обработке
\begin{multline*}
\fr{\hat{R}(T_m)-R(T_m)}{n} = {}\\
{}=\fr{\hat{R}(T_m)-{\mathsf E} \hat{R}(T_m)}{n} 
+\fr{{\mathsf E}\hat{R}(T_m)-R(T_m)}{n}\,,
\end{multline*}
где второе слагаемое стремится к~нулю при $n\to\infty$ \cite{Mallat}. 
Следовательно, достаточно показать, что $(\hat{R}(T_m)\hm-{\mathsf E}\hat{R}(T_m))n^{-1} \hm\to 0$ п.~в.

Полагая в~лемме~3 $X_i \hm= F[x_i, T_m] \hm- {\mathsf E} F[x_i, T_m]$, $b \hm= 
2(\sigma^2\hm+T_m^2)$ и~$m \hm= n^{1/4}$ и~учитывая ограничения на $\alpha(\cdot)$ из 
условия, нетрудно убедиться, что для всех~$n$
$$
\p \left(\left| \fr{\hat{R}(T_m)-{\mathsf E} \hat{R}(T_m)}{n}\right| >\eps \right) 
\leq \fr{c_5}{n^{1+c_6}}\,, 
$$
где константы $c_5$, $c_6$ положительны. Отсюда
$$
\sum\limits_{n=1}^{\infty}\p \left(\left|\fr{\hat{R}(T_m)-{\mathsf E} 
\hat{R}(T_m)}{n}\right| >\eps \right) < \infty,
$$
и по теореме~1.3.4 из~\cite{Serfling2002} 
$$
\left(\hat{R}(T_m)-{\mathsf E}\hat{R}(T_m)\right)n^{-1} \to 0~\mbox{п.~в.}
$$



Покажем теперь, что  $U(\hat{t}_F) \, n^{-1}\hm \to 0$ п.~в. Доказательство 
проведено для $\mu \hm\in m_p[\eta_n]$, в~случае $\mu\hm \in l_0[\eta_n]$ 
доказательство аналогично.
Аналогично формуле~(\ref{M1}), для произвольного $\eps\hm>0$ в~терминах тео\-ре\-мы~1 имеем
\begin{multline*}
\p \left( \fr{|U(\hat{t}_F)|}{n}> 4\eps \right) \leq \p\left(\hat{t}_F 
\leq t_{\kappa_n}\right) +{}\\
{}+ \p\left(\fr{\sup\nolimits_{T\in [t_{\kappa_n}, T_U]} |{\mathsf E} 
U(T)|}{n}>\eps\right)+{}\\
{}+ \p \left(\sup\limits_{T\in [t_{\kappa_n}, T_U]} |Z_1(T)| > \eps n\right) +{}
\end{multline*}

\noindent
\begin{multline}
{}+ \p  \left(\sup\limits_{j \in [0, n-1]} |Z_2(T_j^{\prime})| > \eps n\right) +{}\\
{}+ \p \left(\sup\limits_{\substack{j \in [0, n-1] \\ T\in 
[T_j^{\prime},T_j^{\prime}+d_n]}} |Z_2(T)-Z_2(T_j^{\prime})| > \eps n\right).
\label{M1SC}
\end{multline}
Применяя рассуждения, аналогичные приведенным в~доказательстве теоремы~1, можно показать, что
$$
\sup\limits_{T\in [t_{\kappa_n}, T_U]} |{\mathsf E} U(T)| = o(n); \enskip
\sup\limits_{T\in [t_{\kappa_n}, T_U]} |Z_1(T)|  = o(n),
$$
откуда следует, что второе и~третье слагаемые в~(\ref{M1SC}) обращаются в~ноль 
для всех достаточно больших~$n$.

Для некоторых положительных констант  $c_7$ и~$c_8$ первое, четвертое и~пятое 
слагаемые  в~(\ref{M1SC}) не превышают $c_7 n^{-1-c_8}$ для всех достаточно 
боль\-ших~$n$, что можно показать с~помощью ограничения на $\alpha(\cdot)$ из 
условия и~рассуждений, аналогичных приведенным при выводе соответственно формул~(\ref{M1next}), (\ref{M5}) и~(\ref{M6}), с~тем отличием, что при применении 
леммы~4 полагается $m \hm= (\ln n)^3$.

Из доказанного следует, что
$$
\sum\limits_{n=1}^{\infty}\p \left( \fr{|U(\hat{t}_F)|}{n}> 4\eps \right) 
< \infty,
$$
и по теореме~1.3.4 из~\cite{Serfling2002} $U(\hat{t}_F) \, n^{-1} \to 0$ п.~в., 
что завершает доказательство теоремы.~\hfill$\square$



{\small\frenchspacing
 {\baselineskip=11.5pt
 %\addcontentsline{toc}{section}{References}
 \begin{thebibliography}{99}
\bibitem{FDRImage}
\Au{Krylov V.\,A., Moser~G., Serpico~S.\,B., Zerubia~J.}
False discovery rate approach to unsupervised image change detection~// IEEE 
T. Image Process., 2016. Vol.~25. No.\,10. P.~4704--4718. doi: 10.1109/TIP.2016.2593340.

\bibitem{MultipleTesting} %2
\Au{Menyhart~O., Weltz~B., Gyorffy~B.}
MultipleTesting.com: A~tool for life science researchers for multiple hypothesis 
testing correction~// PLoS One, 2021. Vol.~16. No.\,6. Art.~0245824. doi: 10.1371/journal.pone.0245824.

\bibitem{AdaptingFDR} %3
\Au{Abramovich~F., Benjamini~Y., Donoho~D., Johnstone~I.}
Adapting to unknown sparsity by controlling the false discovery rate~// Ann. Stat., 2006. Vol.~34. No.\,2. P.~584--653.
doi: 10.1214/009053606000000074.

\bibitem{ZasShe17} %4
\Au{Заспа~А.\,Ю., Шестаков~О.\,В.}
Состоятельность оценки риска при множественной проверке гипотез с~FDR-по\-ро\-гом~// 
Вестник ТвГУ. Сер. Прикладная математика, 2017. Вып.~1. С.~5--16.
doi: 10.26456/vtpmk119. EDN: YFYJXT.

\bibitem{Mathematics2020} %5
\Au{Palionnaya~S.\,I., Shestakov~O.\,V.}
Asymptotic properties of MSE estimate for the false discovery rate controlling 
procedures in multiple hypothesis testing // Mathematics, 2020. Vol.~8. No.~11. 
Art.~1913. 11~p. doi: 10.3390/ math8111913.

\bibitem{Shestakov2021-1} %6
\Au{Шестаков~О.\,В.}
Анализ несмещенной оценки среднеквадратичного риска метода блочной пороговой 
обработки~// Информатика и~её применения, 2021. Т.~15. Вып.~2. С.~30--35.
doi: 10.14357/19922264210205. EDN: DSQQAU.

\bibitem{Shestakov2021-2} %7
\Au{Шестаков~О.\,В.}
Пороговые функции в~методах подавления шума, основанных на вейв\-лет-раз\-ло\-же\-нии 
сигнала~// Информатика и~её применения, 2021. Т.~15. Вып.~3. С.~51--56.
doi: 10.14357/19922264210307. EDN: WSEAYG.

\bibitem{Shestakov2022} %8
\Au{Шестаков~О.\,В.}
Несмещенная оценка риска пороговой обработки с~двумя пороговыми значениями~// 
Информатика и~её применения, 2022. Т.~16. Вып.~4. С.~14--19.
doi: 10.14357/19922264220403. EDN: \mbox{DZBVLC}.

\bibitem{ResultsOnFDRUnderDependence} %9
\Au{Farcomeni~A.}
Some results on the control of the false discovery rate under dependence~// 
Scand. J. Stat., 2007. Vol.~34. No.\,2. P.~275--297.
doi: 10.1111/j.1467-9469.2006.00530.x.

\bibitem{VorontsovShestakov2023} %10
\Au{Воронцов~М.\,О., Шестаков~О.\,В.}
Среднеквадратичный риск FDR-про\-це\-ду\-ры в~условиях слабой за\-ви\-си\-мости~// 
Информатика и~её применения, 2023. Т.~17. Вып.~2. С.~34--40.
doi: 10.14357/19922264230205. EDN: AVJZDX.

\bibitem{Vorontsov2024} %11
\Au{Воронцов~М.\,О.}
Анализ среднеквадратичного риска при использовании методов множественной 
проверки гипотез для выбора параметров пороговой обработки в~условиях слабой 
зависимости~// Вестник Московского университета. Сер. 15: Вычислительная 
математика и~кибернетика, 2024. №\,2. С.~18--24.

\bibitem{Bosq} %12
\Au{Bosq~D.}
Nonparametric statistics for stochastic processes: Estimation and prediction.~--- 
Lecture notes in statistics ser.~--- New York, NY, USA: Springer, 1996. Vol.~110. 
188~p.

\bibitem{Mallat} %13
\Au{Mallat~S.}
A wavelet tour of signal processing.~--- New York, NY, USA: Academic Press, 1999. 
857~p.

\bibitem{spatialAdaptation} %14
\Au{Donoho~D., Johnstone~I.}
Ideal spatial adaptation via wavelet shrinkage~// Biometrika, 1994. Vol.~81. 
No.\,3. P.~425--455. doi: 10.1093/biomet/81.3.425.

\bibitem{AdaptingSURE} %15
\Au{Donoho D., Johnstone I.\,M.}
Adapting to unknown smoothness via wavelet shrinkage~// J.~Amer. Stat. Assoc., 
1995. Vol.~90. P.~1200--1224.

\bibitem{ExactRisk} %16
\Au{Marron J.\,S., Adak~S., Johnstone~I.\,M., Neumann~M.\,H., Patil~P.}
Exact risk analysis of wavelet regression~// J.~Comput. Graph. Stat., 1998. 
Vol.~7. P.~278--309. doi: 10.1080/ 10618600.1998.10474777.

\bibitem{Jansen} %17
\Au{Jansen~M.}
Noise reduction by wavelet thresholding.~-- Lecture notes in statistics ser.~--- 
New York, NY, USA: Springer, 2001. Vol.~161. 217~p.

\bibitem{KuShe2016_1} %18
\Au{Кудрявцев~А.\,А., Шестаков~О.\,В.}
Асимптотическое поведение порога, минимизирующего усредненную\linebreak вероятность ошибки 
вычисления вейв\-лет-ко\-эф\-фи\-ци\-ен\-тов~// Докл. Акад. наук, 2016. Т.~468. №\,5. 
С.~487--491.

\bibitem{KuShe2016_2} %19
\Au{Кудрявцев~А.\,А., Шестаков~О.\,В.}
Асимптотически оптимальная пороговая обработка вейв\-лет-ко\-эф\-фи\-ци\-ен\-тов в~моделях с~негауссовым распределением шума~// Докл. Акад. наук, 2016. Т.~471. №\,1. 
С.~11--15.



\bibitem{Eroshenko} %20
\Au{Ерошенко~А.\,А.}
Статистические свойства оценок сигналов и~изображений при пороговой обработке 
коэффициентов в~вейв\-лет-раз\-ло\-же\-ни\-ях: Дис.\ \ldots\ канд. физ.-мат. наук.~--- 
М.: МГУ, 2015. 82~с.

\bibitem{Peligrad} %21
\Au{Peligrad~M.}
On the asymptotic normality of sequences of weak dependent random variables~// 
J. Theor. Probab., 1996. Vol.~9. No.\,3. P.~703--715. doi: 10.1007/BF02214083.

\bibitem{Serfling2002} %22
\Au{Serfling~R.\,J.}
Approximation theorems of mathematical statistics.~--- New York, NY, USA: John Wiley \&~Sons, Inc., 2002. 371~p.

\end{thebibliography}

 }
 }

\end{multicols}

\vspace*{-6pt}

\hfill{\small\textit{Поступила в~редакцию 21.05.24}}

\vspace*{8pt}

%\pagebreak

%\newpage

%\vspace*{-28pt}

\hrule

\vspace*{2pt}

\hrule



\def\tit{ASYMPTOTIC NORMALITY AND STRONG CONSISTENCY\\ OF~RISK ESTIMATE WHEN USING THE~FDR THRESHOLD\\ UNDER WEAK DEPENDENCE CONDITION}


\def\titkol{Asymptotic normality and strong consistency of~risk estimate when using the~FDR threshold under weak dependence condition}


\def\aut{M.\,O.~Vorontsov$^{1,2}$ and~O.\,V.~Shestakov$^{1,2,3}$}

\def\autkol{M.\,O.~Vorontsov and~O.\,V.~Shestakov}

\titel{\tit}{\aut}{\autkol}{\titkol}

\vspace*{-13pt}


\noindent
$^{1}$Department of Mathematical Statistics, Faculty of Computational Mathematics and Cybernetics,
 M.\,V.~Lo\-mo-\linebreak
 $\hphantom{^1}$nosov Moscow State University, 1-52~Leninskie Gory, GSP-1, Moscow 119991, Russian Federation

\noindent
$^{2}$Moscow Center for Fundamental and Applied Mathematics, M.\,V.~Lomonosov Moscow State University,\linebreak
$\hphantom{^1}$1~Leninskie Gory, GSP-1, Moscow 119991, Russian Federation

\noindent
$^{3}$Federal Research Center ``Computer Science and Control'' of the Russian Academy of Sciences, 44-2~Vavilov\linebreak
$\hphantom{^1}$Str., Moscow 119333, Russian Federation


\def\leftfootline{\small{\textbf{\thepage}
\hfill INFORMATIKA I EE PRIMENENIYA~--- INFORMATICS AND
APPLICATIONS\ \ \ 2024\ \ \ volume~18\ \ \ issue\ 3}
}%
 \def\rightfootline{\small{INFORMATIKA I EE PRIMENENIYA~---
INFORMATICS AND APPLICATIONS\ \ \ 2024\ \ \ volume~18\ \ \ issue\ 3
\hfill \textbf{\thepage}}}

\vspace*{2pt}






\Abste{An approach to solving the problem of noise removal in a large array of sparse data is considered
 based on the method of controlling the average proportion of false hypothesis rejections (False Discovery Rate, FDR). 
 This approach is equivalent to threshold processing procedures that remove array components whose values do not exceed 
 some specified threshold. The observations in the model are considered weakly dependent. To control the\linebreak\vspace*{-12pt}}
 
 \Abstend{degree of dependence, 
 restrictions on the strong mixing coefficient and the maximum correlation coefficient are used. The mean-square risk is 
 used as a measure of the effectiveness of the considered approach. It is possible to calculate the risk value only on the test data;
  therefore, its statistical estimate is considered in the work and its properties are investigated. The asymptotic normality and
   strong consistency of the risk estimate are proved when using the FDR threshold under conditions of weak dependence in the data.}

\KWE{thresholding; multiple hypothesis testing; risk estimate}

\DOI{10.14357/19922264240309}{ZOQVTO}

%\vspace*{-12pt}


    
   %   \Ack

%\vspace*{-3pt}
%\noindent



  \begin{multicols}{2}

\renewcommand{\bibname}{\protect\rmfamily References}
%\renewcommand{\bibname}{\large\protect\rm References}

{\small\frenchspacing
 {\baselineskip=10.8pt
 \addcontentsline{toc}{section}{References}
 \begin{thebibliography}{99} 

%1
\bibitem{FDRImage-1}
\Aue{Krylov, V.\,A., G.~Moser, S.\,B.~Serpico, and J.~Zerubia.} 2016. 
False discovery rate approach to unsupervised image change detection. 
\textit{IEEE T. Image Process.} 25(10):4704--4718. doi: 10.1109/TIP.2016.2593340.

%2
\bibitem{MultipleTesting-1}
\Aue{Menyhart, O., B.~Weltz, and B.~Gyorffy.} 2021. 
MultipleTesting.com: A~tool for life science researchers for multiple hypothesis testing correction. 
\textit{PLoS One} 16(6):0245824. 
doi: 10.1371/journal.pone.0245824.

%3
\bibitem{AdaptingFDR-1}
\Aue{Abramovich, F., Y.~Benjamini, D.~Donoho, and I.\,M.~Johnstone.} 2006. 
Adapting to unknown sparsity by controlling the false discovery rate. 
\textit{Ann. Stat.} 34(2):584--653. 
doi: 10.1214/009053606000000074.


%4
\bibitem{ZasShe17-1}
\Aue{Zaspa, A.\,Yu., and O.\,V.~Shestakov.} 2017.
Sostoyatel'nost' otsenki riska pri mnozhestvennoy proverke gipotez s~FDR-porogom
 [Consistency of the risk estimate of the multiple hypothesis testing with the FDR threshold]. 
\textit{Vestnik TvGU. Ser.: Prikladnaya matematika} [Herald of Tver State University. Ser. Applied Mathematics] 1:5--16.
doi: 10.26456/vtpmk119. EDN: YFYJXT.

%5
\bibitem{Mathematics2020-1}
\Aue{Palionnaya, S.\,I., and O.\,V.~Shestakov.} 2020. 
Asymptotic properties of MSE estimate for the false discovery rate controlling procedures in multiple hypothesis testing. 
\textit{Mathematics} 8(11):1913. 11~p.
doi: 10.3390/math8111913.

%6
\bibitem{Shestakov2021-1-1}
\Aue{Shestakov, O.\,V.} 2021.
Analiz nesmeshchennoy otsenki srednekvadratichnogo riska metoda blochnoy po\-ro\-go\-voy obrabotki 
[Analysis of the unbiased mean-square risk estimate of the block thresholding method]. 
\textit{Informatika i~ee Primeneniya~--- Inform. Appl.} 15(2):30--35.
doi: 10.14357/19922264210205. EDN: DSQQAU.

%7
\bibitem{Shestakov2021-2-1}
\Aue{Shestakov, O.\,V.} 2021.
Porogovye funktsii v~metodakh podavleniya shuma, osnovannykh na veyvlet-razlozhenii signala 
[Thresholding functions in the noise suppression methods based on the wavelet expansion of the signal]. 
\textit{Informatika i~ee Primeneniya~--- Inform. Appl.} 15(3):51--56.
doi: 10.14357/19922264210307. EDN: WSEAYG.

%8
\bibitem{Shestakov2022-1}
\Aue{Shestakov, O.\,V.} 2022.
Nesmeshchennaya otsenka riska porogovoy obrabotki s dvumya porogovymi znacheniyami [Unbiased thresholding risk estimate with two threshold values]. 
\textit{Informatika i~ee Primeneniya~--- Inform. Appl.} 16(4):14--19.
doi: 10.14357/19922264220403. EDN: DZBVLC.

%9
\bibitem{ResultsOnFDRUnderDependence-1}
\Aue{Farcomeni, A.} 2007. Some results on the control of the false discovery rate under dependence. 
\textit{Scand. J. Stat.} 34(2):275--297. 
doi: 10.1111/j.1467-9469.2006.00530.x.

%10
\bibitem{VorontsovShestakov2023-1}
\Aue{Vorontsov, M.\,O., and O.\,V.~Shestakov.} 2023.
Sred\-ne\-kvad\-ra\-tich\-nyy risk FDR-protsedury v~usloviyakh slaboy za\-vi\-si\-mosti [Mean-square risk of the FDR procedure under weak dependence]. 
\textit{Informatika i~ee Primeneniya~--- Inform. Appl.} 17(2):34--40.
doi: 10.14357/19922264230205. EDN: AVJZDX.

%11
\bibitem{Vorontsov2024-1}
\Aue{Vorontsov, M.\,O.} 2024. 
RMS risk analysis when using multiple hypothesis testing select parameters of thresholding under conditions of weak dependence. 
\textit{Moscow University Computational Mathematics Cybernetics} 48:91--97. 
doi: 10.3103/S027864192470002X.

%12
\bibitem{Bosq-1}
\Aue{Bosq, D.} 1996. 
\textit{Nonparametric statistics for stochastic processes: Estimation and prediction}. 
Lecture notes in statistics ser. New York, NY: Springer Verlag. Vol.~110. 188~p.

%13
\bibitem{Mallat-1}
\Aue{Mallat, S.} 1999. 
\textit{A wavelet tour of signal processing}. New York, NY: Academic Press. 857~p.

%14
\bibitem{spatialAdaptation-1}
\Aue{Donoho, D., and I.\,M.~Johnstone.} 1994. 
Ideal spatial adaptation via wavelet shrinkage. 
\textit{Biometrika} 81(3):425--455. doi: 10.1093/biomet/81.3.425.

%15
\bibitem{AdaptingSURE-1}
\Aue{Donoho, D., and I.\,M.~Johnstone.} 1995. 
Adapting to unknown smoothness via wavelet shrinkage. 
\textit{J. Am. Stat. Assoc.} 90(432):1200--1224. doi: 10.1080/01621459. 1995.10476626.

%16
\bibitem{ExactRisk-1}
\Aue{Marron, J.\,S., S.~Adak, I.\,M.~Johnstone, M.\,H.~Neumann, and P.~Patil.} 1998. 
Exact risk analysis of wavelet regression. 
\textit{J.~Comput. Graph. Stat.} 7(3):278-309. doi: 10.1080/ 10618600.1998.10474777.

%17
\bibitem{Jansen-1}
\Aue{Jansen, M.} 2001. 
\textit{Noise reduction by wavelet thresholding}. Lecture notes in statistics ser. New York, NY: Springer Verlag. Vol.~161. 217~p.

%18
\bibitem{KuShe2016_1-1}
\Aue{Kudryavtsev, A.\,A., and O.\,V.~Shestakov.} 2016. 
Asymptotic behavior of the threshold minimizing the average probability of error in calculation of wavelet coefficients. 
\textit{Dokl. Math.} 93(3):295--299.
doi: 10.1134/S1064562416030212. EDN: WUMUEV. 

%19
\bibitem{KuShe2016_2-1}
\Aue{Kudryavtsev, A.\,A., and O.\,V.~Shestakov.} 2016. 
Asymptotically optimal wavelet thresholding in the models with non-Gaussian noise distributions. 
\textit{Dokl. Math.} 94(3):615--619.
doi: 10.1134/S1064562416060028. EDN: YUYVUP.




%20
\bibitem{Eroshenko-1}
\Aue{Eroshenko, A.\,A.} 2015. Statisticheskie svoystva otsenok signalov i~izobrazheniy pri porogovoy obrabotke ko\-ef\-fi\-tsi\-en\-tov 
v~veyvlet-razlozheniyakh 
[Statistical properties of signal and image estimates under thresholding of coefficients in wavelet decompositions]. Moscow: MSU. PhD Diss. 82~p.

%21
\bibitem{Peligrad-1}
\Aue{Peligrad, M.} 1996. 
On the asymptotic normality of sequences of weak dependent random variables. 
\textit{J. Theor. Probab.} 9(3):703--715. doi: 10.1007/BF02214083.

%22
\bibitem{Serfling2002-1}
\Aue{Serfling, R.\,J.} 2002. 
\textit{Approximation theorems of mathematical statistics}. New York, NY: John Wiley \&~Sons. 371~p.
\end{thebibliography}

 }
 }

\end{multicols}

\vspace*{-6pt}

\hfill{\small\textit{Received May 21, 2024}} 

%\vspace*{-18pt}

\Contr

\vspace*{-3pt}


\noindent
\textbf{Vorontsov Mikhail O.} (b.\ 1996)~--- PhD student, Department of Mathematical Statistics, 
Faculty of Computational Mathematics and Cybernetics, M.\,V.~Lomonosov Moscow State University, 1-52~Leninskie Gory, GSP-1, Moscow 119991, Russian Federation;  
mathematician, Moscow Center for Fundamental and Applied Mathematics, M.\,V.~Lomonosov Moscow State University, 1~Leninskie Gory, GSP-1, Moscow 119991, Russian Federation;
\mbox{m.vtsov@mail.ru}

\vspace*{6pt}

\noindent
\textbf{Shestakov Oleg V.} (b.\ 1976)~--- Doctor of Science in physics and mathematics, professor, Department of Mathematical Statistics,
 Faculty of Computational Mathematics and Cybernetics, M.\,V.~Lomonosov Moscow State University, 1-52~Leninskie Gory, GSP-1, Moscow 119991, Russian Federation; 
 senior scientist, Federal Research Center ``Computer Science and Control'' of the Russian Academy of Sciences, 44-2~Vavilov Str., Moscow 119333, 
 Russian Federation; leading scientist, Moscow Center for Fundamental and Applied Mathematics, M.\,V.~Lomonosov Moscow State University, 
 1~Leninskie Gory, GSP-1, Moscow 119991, Russian Federation; \mbox{oshestakov@cs.msu.su}


\label{end\stat}

\renewcommand{\bibname}{\protect\rm Литература}  %6
\def\stat{inkova}

\def\tit{СТЕПЕНЬ СЕМАНТИЧЕСКОЙ БЛИЗОСТИ ДИСКУРСИВНЫХ ОТНОШЕНИЙ: МЕТОДЫ И~ИНСТРУМЕНТЫ РАСЧЕТА$^*$}

\def\titkol{Степень семантической близости дискурсивных отношений: методы и~инструменты расчета}

\def\aut{О.\,Ю.~Инькова$^1$, М.\,Г.~Кружков$^2$}

\def\autkol{О.\,Ю.~Инькова, М.\,Г.~Кружков}

\titel{\tit}{\aut}{\autkol}{\titkol}

\index{Инькова О.\,Ю.}
\index{Кружков М.\,Г.}
\index{Inkova O.\,Yu.}
\index{Kruzhkov M.\,G.}


{\renewcommand{\thefootnote}{\fnsymbol{footnote}} \footnotetext[1]
{Работа выполнена в~Федеральном исследовательском центре <<Информатика и~управление>> Российской 
академии наук с~использованием ЦКП <<Информатика>> ФИЦ ИУ РАН.}}


\renewcommand{\thefootnote}{\arabic{footnote}}
\footnotetext[1]{Федеральный исследовательский центр <<Информатика и~управление>> Российской академии наук; 
Женевский университет, \mbox{olyainkova@yandex.ru}}
\footnotetext[2]{Федеральный исследовательский центр <<Информатика и~управление>> Российской 
академии наук, \mbox{magnit75@yandex.ru}}

%\vspace*{-14pt}


  
  \Abst{Рассматриваются методы оценки семантической близости дискурсивных 
отношений. Авторы предлагают несколько подходов к~решению этой проблемы с~применением двух информационных ресурсов: коллекции сформированных авторами 
структурированных определений ло\-ги\-ко-се\-ман\-ти\-че\-ских отношений (ЛСО) 
и~Надкорпусной базы данных коннекторов (НБДК), включающей в~себя аннотации переводных 
соответствий текстовых фрагментов с~маркерами ЛСО на русском, французском 
и~итальянском языках. Показано, что при оценке семантической близости ЛСО высокий 
приоритет будут иметь такие факторы, как принадлежность различительных признаков ЛСО к~одному семейству в~структурированных определениях отношений, соответствия между 
показателями различных ЛСО в~оригинальных и~переводных текстах, а также случаи, когда 
различные ЛСО выражаются одинаковыми показателями в~разных контекстах. Менее значим 
фактор сочетаемости различных ЛСО в~рамках одного и~того же контекста. Предполагается, 
что на основе сформулированных методов станет возможным более точно определить, какие 
различительные признаки ЛСО имеют наибольший вес при определении их семантической  
бли\-зости.}
  
  \KW{надкорпусная база данных; логико-семантические отношения; коннекторы; 
аннотирование; фасетная классификация}

  \DOI{10.14357/19922264230412}{FXTSPZ}
  
%\vspace*{-1pt}


\vskip 10pt plus 9pt minus 6pt

\thispagestyle{headings}

\begin{multicols}{2}

\label{st\stat}
  
\section{Степень семантической близости дискурсивных 
отношений}

%\vspace*{-4pt}

  Проблемы классификации дискурсивных отношений, обеспечивающих 
связность текста, занимают лингвистов и~специалистов по автоматической 
обработке текста не один десяток лет: первые исследования начались  
в~1970-х~гг.~[1, 2]. Были предложены их многочисленные классификации (ср.\ 
наиболее известные~[3--7]), однако никто, насколько известно авторам, не 
пытался определить степень семантической близости (ССБ) дискурсивных 
отношений. Это связано прежде всего с~тем, что классификации имеют, за 
редким исключением~\cite{7-in, 8-in, 9-in}, форму списка, и~этот вопрос просто 
не ставился. Однако его решение полезно не только для анализа текста, в~том 
числе автоматического, но и~для когнитивных наук и~переводоведения, 
поскольку позволяет выявить общие закономерности человеческого мышления.
  
  Кроме того, сами дискурсивные отношения определены во многом неточно 
или тавтологично\footnote[3]{См., например, определение отношения альтернативы 
(disjunction) в~теории риторической структуры: (а)~элемент пред\-став\-ля\-ет собой (не 
обязательно исключающую) альтернативу другому; (б)~слу\-ша\-ющий/чи\-та\-тель 
распознает, что связанные элементы альтернативны (см.\ {\sf http://www.sfu.ca/rst}).}, схожие 
или идентичные отношения носят даже в~англоязычных классификациях разные 
названия, а одинаковые названия описывают разную языковую реальность. 
Например, в~теории сегментированного представления дискурса (Segmented 
Discourse Representation Theory, SDRT~[10]) отношение contrast включает как 
отношения <<вопреки ожидаемому>>, так и~уступительные отношения. 
В~классификации Пенсильванского аннотированного корпуса им 
соответствуют два отношения (opposition и~contra-expectation)~\cite{7-in}, 
а~в~теории риторической структуры~--- contrast и~concession~[11] (подробнее 
см.~\cite[с.~37]{9-in}). 

\begin{table*}[b]\small %tabl1
\vspace*{-10pt}
\begin{center}
\Caption{Структурированные определения уступительных ЛСО и~ЛСО <<вопреки 
ожидаемому>>}
\vspace*{2ex}

\tabcolsep=3pt
\begin{tabular}{|l|p{40mm}|p{38mm}|p{57mm}|}
\hline
\multicolumn{1}{|c|}{\textbf{ЛСО}} & \multicolumn{1}{c|}{\tabcolsep=0pt\begin{tabular}{c}\textbf{Базовая семантическая}\\ \textbf{операция}\end{tabular}}&
\multicolumn{1}{c|}{\textbf{Уровень}} &
\multicolumn{1}{c|}{ \tabcolsep=0pt\begin{tabular}{c}\textbf{Дополнительные}\\ \textbf{характеристики}\end{tabular}}\\
\hline
&&&\\[-20pt]
\multicolumn{1}{|l|}{\raisebox{-26pt}[0pt][0pt]{\textbf{Уступительные}}}& 
%\begin{itemize}
\multicolumn{1}{l|}{\raisebox{-26pt}[0pt][0pt]{\ \ \ \  --\ \ операция импликации}}
%\end{itemize} 
& 
%\begin{itemize}
\multicolumn{1}{l|}{\raisebox{-26pt}[0pt][0pt]{\tabcolsep=0pt\begin{tabular}{l}\ \ \ \ --\ \ пропозициональный\\
\hphantom{\ \ \ \ --\ \ }уровень\end{tabular}}}
%\end{itemize}
&
\begin{itemize}
\item $p$ и~$q$~--- положения вещей;\vspace*{-3pt}
\item как правило, если имеет место $q$, то не имеет места~$p$\vspace*{-8pt}
   \end{itemize}
\\
\hline
&&&\\[-20pt]
\multicolumn{1}{|l|}{\raisebox{-48pt}[0pt][0pt]{\tabcolsep=0pt\begin{tabular}{l}\textbf{<<Вопреки}\\ \textbf{ожидаемому>>}\end{tabular} }}& 
%\begin{itemize}
\multicolumn{1}{l|}{\raisebox{-48pt}[0pt][0pt]{\tabcolsep=0pt\begin{tabular}{l}\ \ \ \  --\ \ операция сравнения,\\
 \hphantom{\ \ \ \ --\ \ }уста\-нав\-ли\-ва\-ющая не-\\
 \hphantom{\ \ \ \ --\ \ }сходство $p$ и~$q$\end{tabular}}}
%\end{itemize} 
&
%\begin{itemize}
\multicolumn{1}{l|}{\raisebox{-48pt}[0pt][0pt]{\tabcolsep=0pt\begin{tabular}{l}\ \ \ \  --\ \ пропозициональный\\ 
 \hphantom{\ \ \ \ --\ \ }уровень\end{tabular}}}
%\end{itemize} 
&
 \begin{itemize}
 \item $q$ имеет большую аргументативную\newline силу, чем~$p$;\vspace*{-3pt}
  \item положение вещей $p$ служит аргументом в~пользу ожи\-да\-емо\-го вывода~$r$;\vspace*{-3pt}
  \item положение вещей $q$ служит аргументом в~пользу ожи\-да\-емо\-го вывода не-$r$\vspace*{-8pt}
  \end{itemize}\\
\hline
\end{tabular}
\end{center}
\end{table*}
  
  В~этой связи были сделаны попытки сравнить\linebreak существующие 
классификации, чтобы понять, насколько соотносимы выделяемые в~них 
дискурсивные отношения~[12--14]. В~[14] для этого применяется 
набор различительных признаков. Этих\linebreak признаков, однако, недостаточно, чтобы 
сформулировать уникальное определение отношения, и~некоторые из них 
имеют одинаковый набор признаков. Это касается, например, четырех 
отношений (narration, precondition, background и~parallel) в~SDRT~\cite[с.~38]{14-in}. 
  
  В~работе~[15] были заложены основы для разработки структурированных 
определений дискурсивных, или в~терминологии автора  
ло\-ги\-ко-се\-ман\-ти\-че\-ских, отношений на основе применяемой 
в~НБДК классификации. Каждое 
ЛСО может быть описано набором различительных признаков (см.\ примеры 
в~\cite{16-in} и~\cite{17-in}). Некоторые признаки оказываются общими для 
нескольких ЛСО, другие~--- индивидуальны, т.\,е.\ свойственны только данному 
ЛСО. На момент написания статьи в~НБДК были описаны 26~ЛСО 
с~использованием~52~различительных признаков. Это позволяет дать каждому 
ЛСО уникальное определение (см.\ примеры в~разд.~2), а~также определить 
ССБ ЛСО. 

\vspace*{-6pt}
  
\section{Критерии, лежащие в~основе определения степени 
семантической близости логико-семантических отношений}

\vspace*{-3pt}

  В~предыдущей работе авторов~[17] показано, что не все различительные 
признаки имеют одинаковый вес при определении семантической близости 
ЛСО и~что, предположительно, наибольшее значение имеет принадлежность 
общих признаков к~одному семейству. 
  

  
  В~основе уступительных ЛСО и~ЛСО <<вопреки ожидаемому>> лежат 
разные базовые операции: импликация~--- для первого и~сравнение, 
уста\-нав\-ли\-ва\-ющее несходство $p$ и~$q$,~--- для второго (табл.~1). Это 
значит, что эти два ЛСО находятся в~разных семантических группах. Оба ЛСО 
при этом установлены на пропозициональном уровне, т.\,е.\ непосредственно 
между положениями дел $p$ и~$q$, которые они связывают, и~оба используют 
отрицательный коррелят одного из положений вещей. Иначе говоря, признаки 
<<как правило, если имеет место~$q$, то не имеет места $p$>> и~<<положение 
вещей~$q$ служит аргументом в~пользу ожидаемого вывода не-$r$>> 
принадлежат к~одному семейству. В~примере~(1) с~ЛСО <<вопреки 
ожидаемому>>: \textit{Ему [$\ldots$] очень неприятно было сталкиваться с~народом,} {\bfseries\textit{но}} \textit{он шел именно туда, где виднелось больше 
народу}. [Ф.\,М.~Достоевский. Преступление и~наказание], положение вещей 
$p$\;=\;<<ему очень неприятно было сталкиваться с~народом>> ориентирует в~пользу вывода $r$\;=\;<<он не должен был бы идти к~народу>>. Этот вывод 
опровергается непосредственно в~$q$ (=\;не-$r$)\;=\;<<он шел именно туда, где 
виднелось больше народу>>. Семантический механизм, лежащий в~основе 
уступительных отношений (их прототипическим показателем может считаться 
союз \textit{хотя}), совпадает с~этим семантическим механизмом, но 
в~зеркальном отражении: 
  \begin{gather*}
p\ \mbox{\textit{хотя}}\  q (q \to  \mbox{не-}p)\\
p \to r\ \mbox{но}\  q\ (q = \mbox{не-}r),\ \mbox{т.\,е.}\ p \to \mbox{не-}q\ 
\mbox{\textit{но}}\ q.
\end{gather*}
  %
  Отсюда необходимость при замене \textit{хотя} на \textit{но} и~наоборот 
изменить порядок следования фрагментов текста: \textit{Ему неприятно было 
сталкиваться с~народом}, {\bfseries\textit{но}} \textit{он шел туда, где виднелось 
больше народу} (ЛСО <<вопреки ожидаемому>>); \textit{Он шел туда, где 
виднелось больше народу}, {\bfseries\textit{хотя}} \textit{ему неприятно было 
сталкиваться с~народом} (ЛСО уступки)~\cite{18-in}. Это позволяет говорить 
о~семантической близости двух ЛСО и,~например, в~классификации~\cite{7-in} 
они объединены в~одну группу concession.

\begin{table*}[b]\small %tabl2
\vspace*{-6pt}
\begin{center}
\Caption{Логико-семантические отношения, соответствующие ЛСО <<вопреки ожидаемому>> в~оригинальных и~переводных текстах }
\vspace*{2ex}

\tabcolsep=4.3pt
\begin{tabular}{|c|l|c|c|c|c|c|c|}
\hline
\textbf{ЛСО1}&\multicolumn{1}{c|}{\textbf{ЛСО2}}&\textbf{1}\;+\;\textbf{2}&\textbf{1}&
\textbf{2}&\textbf{1}\;$\to$\;\textbf{2}&\textbf{2}\;$\to$\;\textbf{1}&\textbf{Сумма}\\
\hline
<<вопреки ожидаемому>>&уступительные&237\hphantom{9}&2140&853&11,07\%\hphantom{9}&27,78\%\hphantom{9}&38,86\%\hphantom{9}\\
<<вопреки ожидаемому>>&одновременность&139\hphantom{9}&2140&1268\hphantom{9}&6,50\%&10,96\%\hphantom{9}&17,46\%\hphantom{9}\\
<<вопреки ожидаемому>>&соединительные&149\hphantom{9}&2140&2088\hphantom{9}&6,96\%&7,14\%&14,10\%\hphantom{9}\\
<<вопреки ожидаемому>>&сопоставительные&78&2140&807&3,64\%&9,67\%&13,31\%\hphantom{9}\\
<<вопреки ожидаемому>>&пропозициональное 
сопутствование&39&2140&378&1,82\%&10,32\%\hphantom{9}&12,14\%\hphantom{9}\\
<<вопреки ожидаемому>>&исключение из 
рассмотрения&\hphantom{9}8&2140&\hphantom{9}90&0,37\%&8,89\%&9,26\%\\
<<вопреки ожидаемому>>&иллокутивное 
сопутствование&17&2140&471&0,79\%&3,61\%&4,40\%\\
<<вопреки ожидаемому>>&интенсиональная 
генерализация&\hphantom{9}8&2140&248&0,37\%&3,23\%&3,60\%\\
<<вопреки ожидаемому>>&замещение&\hphantom{9}7&2140&294&0,33\%&2,38\%&2,71\%\\
<<вопреки ожидаемому>>&пропозициональная 
коррекция&\hphantom{9}4&2140&165&0,19\%&2,42\%&2,61\%\\
<<вопреки ожидаемому>>&условные&12&2140&1075\hphantom{9}&0,56\%&1,12\%&1,68\%\\
<<вопреки ожидаемому>>&спецификация&11&2140&1608\hphantom{9}&0,51\%&0,68\%&1,20\%\\
<<вопреки ожидаемому>>&исключение&\hphantom{9}5&2140&615&0,23\%&0,81\%&1,05\%\\
<<вопреки ожидаемому>>&отрицательная 
альтернатива&\hphantom{9}2&2140&271&0,09\%&0,74\%&0,83\%\\
<<вопреки ожидаемому>>&оговорка&\hphantom{9}1&2140&150&0,05\%&0,67\%&0,71\%\\
<<вопреки ожидаемому>>&экстенсиональная 
генерализация&\hphantom{9}2&2140&588&0,09\%&0,34\%&0,43\%\\
<<вопреки ожидаемому>>&переформулирование&\hphantom{9}2&2140&1183\hphantom{9}&0,09\%&0,17\%&0,26\%\\
<<вопреки ожидаемому>>&пропозициональная 
альтернатива&\hphantom{9}1&2140&1238\hphantom{9}&0,05\%&0,08\%&0,13\%\\
\hline
\multicolumn{8}{p{163mm}}{\footnotesize \hspace*{3mm}Расшифровка названий столбцов: 
1\;+\;2~--- число переводных аннотаций, в~которых ЛСО1 в~тексте на одном языке 
соответствует ЛСО2 в~тексте на другом языке; 1~--- число аннотаций, в~которых в~любом из 
текстов проставлено ЛСО1; 2~--- число аннотаций, в~которых в~любом из текстов 
проставлено ЛСО2; 1\;$\to$\;2~--- процент соответствия для ЛСО1 с~ЛСО2; 2\;$\to$\;1~--- 
процент соответствия для ЛСО2 с~ЛСО1; сумма~--- сумма двух предыдущих показателей.}
\end{tabular}
\end{center}
\end{table*}

  
  
  Кроме того, сформулирована гипотеза, согласно которой при определении 
ССБ ЛСО могут учитываться также другие 
факторы:
\begin{enumerate}[(1)] 
\item соответствия ЛСО в~оригинальных и~переводных текстах; 
\item случаи, когда разные ЛСО выражаются одним и~тем же показателем; 
\item сочетаемость показателей ЛСО в~одном фрагменте текста.
\end{enumerate}
 В~НБДК для 
ЛСО, имеющих структурированные определения, были получены 
количественные данные по этим трем критериям.

  
  
\subsection{Соответствие логико-семантических отношений в~оригинальных и~переводных текстах}

  Соответствие ЛСО в~оригинальных и~переводных текстах означает, что 
некоторому ЛСО в~тексте оригинала, точнее, его показателю, соответствует 
показатель иного ЛСО в~тексте перевода. Так, если для перевода на 
французский язык коннектора \textit{но} в~примере~(1) был выбран коннектор 
\textit{mais}, также выражающий ЛСО <<вопреки ожидаемому>>: (2)~\textit{Il 
lui $\acute{\mbox{e}}$tait d$\acute{\mbox{e}}$sagr$\acute{\mbox{e}}$able, 
tr$\grave{\mbox{e}}$s d$\acute{\mbox{e}}$sagr$\acute{\mbox{e}}$able, de 
rencontrer du monde} {\bfseries\textit{mais}} \textit{il allait justement 
l$\grave{\mbox{a}}$ o$\grave{\mbox{u}}$ l'on en voyait le plus} [перевод 
$\acute{\mbox{E}}$lisabeth Guertik], то в~примере~(3) тот же коннектор 
переведен \textit{bien que}~--- показателем уступительных ЛСО: 
\textit{С~такой поправкой смысл телеграммы становился ясен,} 
{\bfseries\textit{но}}\textit{, конечно, трагичен}.~--- \textit{Ainsi 
corrig$\acute{\mbox{e}}$, le t$\acute{\mbox{e}}$l$\acute{\mbox{e}}$gramme 
prenait un sens parfaitement clair,} {\bfseries\textit{bien que}} \textit{tragique, 
naturellement}. [М.~Булгаков. Мастер и~Маргарита, перевод Claude Ligny].
  
  Количественные данные по ЛСО, соответствующим ЛСО <<вопреки 
ожидаемому>> в~оригинальных и~переводных текстах на русском, французском и~итальянском языках, приведены в~табл.~2.
  
  
  Для ЛСО <<вопреки ожидаемому>> в~НБДК сформирована 2141~двуязычная 
аннотация. В~237~случаях ему соответствует уступительное ЛСО. Это 
подтверждает важность критерия принадлежности \mbox{различительных} признаков к~одному семейству. 

Схожую картину можно наблюдать для других отношений 
(табл.~3): для сопоставительных и~соединительных ЛСО (основаны на 
общей базовой операции и~имеют общий различительный признак 
<<сходство~$p$ и~$q$ относительно некоторого ``общего\linebreak знаменателя''>>); для 
ЛСО оговорки и~пропозициональной альтернативы (они имеют общий 
различительный признак~--- <<$p$ и~$q$~--- положения вещей, име\-ющие 
статус гипотезы>>); для ЛСО \mbox{одновременности} и~со\-по\-став\-ле\-ния (их 
различительные при\-зна\-ки <<T$p$ включает в~себя T$q$>> и~<<$p$ и~$q$ 
актуальны для говорящего в~момент речи T$d$>> принадлежат к~семейству 
признаков <<Единство временного интервала>>); для ЛСО одновременности 
и~пропозиционального сопутствования (об\-щий признак <<T$p$ включает 
в~себя T$q$>>). 
  
\begin{table*}\small %tabl3
\begin{center}
\Caption{Соответствия других ЛСО }
\vspace*{2ex}

\begin{tabular}{|l|l|c|c|c|c|c|c|}
\hline
\multicolumn{1}{|c|}{\textbf{ЛСО1}}&\multicolumn{1}{c|}{\textbf{ЛСО2}}&\textbf{1}\;+\;\textbf{2}&\textbf{1}&\textbf{2}&\textbf{1}\;
$\to$\;\textbf{2}&\textbf{2}\;$\to$\;\textbf{1}&\textbf{Сумма}\\
\hline
соединительные&сопоставительные&272\hphantom{9}&2088&807&13,03\%&33,71\%&46,73\%\\
оговорка&пропозициональная альтернатива&40&\hphantom{9}150&1238\hphantom{9}&26,67\%&\hphantom{9}3,23\%&29,90\%\\
одновременность&сопоставление&180\hphantom{9}&1268&807&14,20\%&22,30\%&36,50\%\\
одновременность &пропозициональное 
сопутствование&43&1268&378&\hphantom{9}3,39\%&11,38\%&14,77\%\\
\hline
\end{tabular}
\end{center}
\vspace*{-4pt}
\end{table*}

\begin{table*}[b]\small %tabl4
\vspace*{-12pt}
\begin{center}
\Caption{Количественные данные по ЛСО, выражаемым одним показателем}
\vspace*{2ex}

\begin{tabular}{|c|l|l|c|}
\hline 
\textbf{Язык}&\multicolumn{1}{c|}{\textbf{Коннектор}}&\multicolumn{1}{c|}{\textbf{ЛСО}}&\textbf{Количество аннотаций}\\
\hline
\multicolumn{1}{|c|}{\raisebox{-11pt}[0pt][0pt]{RU}}&\multicolumn{1}{l|}{\raisebox{-11pt}[0pt][0pt]{а то}}&отрицательная альтернатива&125\hphantom{9}\\
&&пропозициональная альтернатива&12\\
&&исключение из рассмотрения&\hphantom{9}6\\
\hline
\multicolumn{1}{|c|}{\raisebox{-6pt}[0pt][0pt]{RU}}&\multicolumn{1}{l|}{\raisebox{-6pt}[0pt][0pt]{если$\|$то}}&условные&183\hphantom{9}\\
&&сопоставительные&13\\
\hline
\multicolumn{1}{|c|}{\raisebox{-6pt}[0pt][0pt]{RU}}&\multicolumn{1}{l|}{\raisebox{-6pt}[0pt][0pt]{когда}}&одновременность&13\\
&&условные&\hphantom{9}1\\
\hline
\multicolumn{1}{|c|}{\raisebox{-6pt}[0pt][0pt]{RU}}&\multicolumn{1}{l|}{\raisebox{-6pt}[0pt][0pt]{когда$\|$то}}&одновременность&38\\
&&условные&\hphantom{9}6\\
\hline
\multicolumn{1}{|c|}{\raisebox{-11pt}[0pt][0pt]{RU}}
&\multicolumn{1}{l|}{\raisebox{-11pt}[0pt][0pt]{между тем}}
&одновременность&126\hphantom{9}\\
&&<<вопреки ожидаемому>>&53\\
&&сопоставительные&11\\
\hline
\multicolumn{1}{|c|}{\raisebox{-6pt}[0pt][0pt]{RU}}&\multicolumn{1}{l|}{\raisebox{-6pt}[0pt][0pt]{между тем как}}&сопоставительные&29\\
&&одновременность&\hphantom{9}6\\
\hline
\multicolumn{1}{|c|}{\raisebox{-18pt}[0pt][0pt]{RU}}
&\multicolumn{1}{l|}{\raisebox{-18pt}[0pt][0pt]{разве}}
&оговорка&20\\
&&исключение&\hphantom{9}5\\
&&исключение из рассмотрения&\hphantom{9}4\\
&&условные&\hphantom{9}2\\
\hline
\multicolumn{1}{|c|}{\raisebox{-6pt}[0pt][0pt]{FR}}&\multicolumn{1}{l|}{\raisebox{-6pt}[0pt][0pt]{cependant}}&<<вопреки ожидаемому>>&100\hphantom{9}\\
&&одновременность&27\\
\hline
\multicolumn{1}{|c|}{\raisebox{-6pt}[0pt][0pt]{FR}}&\multicolumn{1}{l|}{\raisebox{-6pt}[0pt][0pt]{en m$\hat{\mbox{e}}$me temps}}&одновременность&29\\
&&сопоставительные&\hphantom{9}1\\
\hline
\multicolumn{1}{|c|}{\raisebox{-6pt}[0pt][0pt]{FR}}&\multicolumn{1}{l|}{\raisebox{-6pt}[0pt][0pt]{quand}}&одновременность&197\hphantom{9}\\
&&условные&10\\
\hline
\end{tabular}
\end{center}
\end{table*}

  
  Напротив, ЛСО, соответствующие ЛСО <<вопреки ожидаемому>> 
и~представленные менее чем в~1\% аннотаций (см.\ табл.~2), не имеют 
различительных признаков, принадлежащих к~одному семейству, и~выбор их 
показателей для перевода показателя ЛСО <<вопреки ожидаемому>> может 
быть квалифицирован как авторский и~контекстуальный.
  
\subsection{Разные логико-семантические отношения выражаются одним~и~тем~же~показателем}

  Известно, что коннекторы в~значительной своей части относятся 
к~многозначным языковым единицам, т.\,е.\ могут служить показателями более 
чем одного ЛСО. Так, для русского союза \textit{и} принято выделять пять 
значений: сочинительное, временного следования, добавления,  
ре\-зуль\-та\-тив\-но-след\-ст\-вен\-ное и~несоответствия; для союза 
\textit{когда}~--- два: одновременности и~условия; у~союза \textit{но} 
выделяются собственно противительное  
и~про\-ти\-ви\-тель\-но-усту\-пи\-тель\-ное значения, а~у~\textit{хотя}~--- 
уступительное и~усту\-пи\-тель\-но-про\-ти\-ви\-тель\-ное и~т.\,д.~[19--21]. Это 
отражают и~данные НБДК, причем с~указанием на частотность того или иного 
значения коннектора в~сформированных аннотациях. 

В~табл.~4 приведены 
выборочно данные для многозначных коннекторов русского и~французского 
языков.
  

  
  Приведенные данные подтверждают прежде всего положения теории 
грамматикализации, согласно которым семантическая эволюция языковых 
единиц имеет определенные закономерности.\linebreak Так, было показано, что на основе 
значения одновременности может развиваться семантика сопоставления и~противопоставления, а~также импликации~\cite{22-in}. Это хорошо видно на 
примере \mbox{коннекторов} \textit{когда}, \textit{между тем}, а~также французских 
\textit{cependant} `в~то же время, однако', \textit{en m$\hat{\mbox{e}}$me temps} 
`в~то же время' и~\textit{quand} `когда' (см.\ табл.~4). С~другой стороны, эти 
данные подтверждают гипотезу авторов о~том, что набор ЛСО, которые может 
маркировать один показатель, не случаен, а~включает семантически близкие 
ЛСО. Так, коннектор \textit{разве} зафиксирован в~НБДК как показатель ЛСО 
оговорки, исключения, исключения из рассмотрения и~условия. Эти ЛСО имеют 
общие различительные признаки. Ло\-ги\-ко-се\-ман\-ти\-че\-ские отношения оговорки и~условия~--- два признака: 
базовая операция импликации и~признаки из семейства гипотетичность; ЛСО 
условия и~исключения устанавливаются на пропозициональном уровне, а~ЛСО 
оговорки и~исключения из рас\-смот\-ре\-ния~--- на уров\-не вы\-ска\-зы\-ва\-ния; ЛСО 
оговорки, исключения и~исключения из рас\-смот\-ре\-ния обладают общими 
признаками на уровне семейства признаков (семантика исключения), а~ЛСО 
исключения и~исключения из рас\-смот\-ре\-ния осно\-ва\-ны на общей базовой 
операции (соотнесение элемента и~множества).
  
  Таким образом, данный критерий может быть полезен при определении CСБ 
ЛСО и~иметь достаточно высокий приоритет.
  
\subsection{Сочетаемость логико-семантических отношений в~рамках одного фрагмента текста}

  Третий критерий, который можно учитывать при определении ССБ ЛСО,~--- 
сочетаемость ЛСО, точнее их показателей. Здесь, однако, возникает ряд 
сложностей, связанных с~тем, что возможность сочетаемости показателей 
зависит в~первую очередь от морфологической природы показателя ЛСО. Как 
известно, коннекторы относятся к~разнообразным морфологическим классам: 
сочинительные со\-юзы (\textit{и}, \textit{а}, \textit{но}); подчинительные союзы 
(\textit{хотя}, \textit{потому что}, \textit{как}), так называемые 
<<конкретизаторы со\-юзов>>, перешедшие в~класс коннекторов, как правило, из 
наречных выражений (\textit{в~то же время}, \textit{однако}, \textit{впрочем}); 
предлоги (\textit{кроме}, \textit{после}). Союзы, например, как сочинительные, 
так и~подчинительные, не могут сочетаться между собой в~рамках единого 
фрагмента текста, и, наоборот, наибольшей легкостью в~сочетании именно с~союзами обладают <<конкретизаторы>> (\textit{но однако}, \textit{но впрочем}, 
\textit{а~между тем}, \textit{или например}, \textit{и~в~частности}). Если для 
показателей некоторых ЛСО можно выявить закономерности, то другие менее 
избирательны в~своих сочетаниях. Так, показатель ЛСО спецификации 
\textit{например} сочетается со всеми сочинительными союзами, а~показатель 
ЛСО <<вопреки ожидаемому>> \textit{впрочем} только с~союзами~\textit{а} 
и~\textit{но}, т.\,е.\ показателями близких ему (\textit{а}) или тех же (\textit{но}) 
ЛСО. Можно также учитывать двухместные реализации коннекторов, т.\,е.\ 
такие, где компоненты коннектора находятся в~каждом из соединяемых 
фрагментов текста, например \textit{хотя$\ldots$\ но}: \textit{Хотя он меня 
очень уговаривал, но я~не согласился}. Но такие сочетания возможны не для 
всех ЛСО и~сужают круг возможностей для получения адекватных 
количественных данных.
 
  В~связи с~вышесказанным при подсчете ССБ ЛСО этот критерий может 
использоваться лишь как дополнительный.
  
\section{Заключение}

  Из четырех рассмотренных критериев определения ССБ ЛСО: 
(1)~принадлежности различительных признаков ЛСО к~одному семейству, 
(2)~соответствия ЛСО в~оригинальных и~переводных \mbox{текс\-тах}, (3)~возможности 
одного показателя выражать разные ЛСО и~(4)~сочетаемости показателей ЛСО 
в~одном фрагменте текста~--- первые три могут иметь достаточно высокий 
приоритет. Четвертый признак обладает, напротив, наименьшим весом при 
определении ССБ ЛСО. 
  
  Степень детальности разметки, а следовательно, и~определений ЛСО не 
позволяет пока объяснить некоторые явления. Например, семантическую 
близость ЛСО условия и~одновременности, который подтверждается как их 
соответствиями в~оригинальных и~переводных текстах, так и~воз\-мож\-ностью 
выражаться одним показателем (\textit{когда}). Их общий признак <<T$p$ 
включает в~себя T$q$>> не входит в~определение условных ЛСО, так как 
соотношение временн$\acute{\mbox{ы}}$х планов положений вещей~$p$ и~$q$ может быть 
самым различным в~условном периоде. С~другой стороны, при ЛСО 
одновременности различным может быть их семантическое соотношение 
(семантическая независимость, противопоставленность, причина, следствие 
и~т.\,д.). Перевод показателя ЛСО одновременности показателем условных 
ЛСО наблюдается только при одновременной реализации положений 
вещей~$p$ и~$q$ и~при возможности установить между ними отношение 
импликации. Семантическая близость данных двух ЛСО может быть, 
следовательно, установлена на более низком иерархическом уровне, а~именно: 
при определении частных случаев его реализации. В~НБДК такая возможность 
предусмотрена, что позволит в~дальнейшем более детально описывать каждое 
ЛСО и~его виды, а~значит, более точно определить ССБ ЛСО.
{\looseness=1

}
  
{\small\frenchspacing
 {\baselineskip=10.6pt
 %\addcontentsline{toc}{section}{References}
 \begin{thebibliography}{99}
\bibitem{1-in}
\Au{Hobbs J.\,R.} A~computational approach to discourse analysis.~--- 
New York, NY, USA: Department of Computer Science, City College, City University of New 
York, 1976.  Research Report 76-2. P.~28--38.
\bibitem{2-in}
\Au{Hobbs J.\,R.} Why is discourse coherent?~--- Menlo Park, CA, 
USA: SRI International, 1978. SRI Technical Note 176. 44~p.
\bibitem{3-in}
\Au{Halliday M.\,A.\,K., Hasan~R.}  Cohesion in English.~--- London: Longman, 1976. 374~p.


\bibitem{5-in} %4
\Au{Mann W.\,C., Thompson~S.\,A.} Rhetorical structure theory: Towards a functional theory of 
text organization~// Text, 1988. Vol.~8. No.\,3. P.~243--281. doi: 10.1515/text.\linebreak  1.1988.8.3.243.

\bibitem{6-in} %5
\Au{Asher N.} Reference to abstract objects in discourse.~--- Dordrecht: Kluwer, 1993. 455~p.

\bibitem{4-in} %6
\Au{Halliday M.\,A.\,K.} An introduction to functional grammar.~--- 2nd ed.~--- London: 
Edward Arnold, 1994. 434~p.

\bibitem{7-in} %7
PDTB Research Group. The Penn Discourse Treebank 2.0 annotation manual.~--- Philadelphia, PA, USA: Institute for Research in Cognitive Science, University 
of Pennsylvania, 2007.  Technical Report 
IRCS-08-01. 104~p. {\sf https://www.cis.upenn.edu/$\sim$elenimi/\linebreak pdtb-manual.pdf}.
\bibitem{8-in}
\Au{Breindl E., Volodina~A., \mbox{Wa{\!\ptb{\!\ss}}\,ner}~U.\,H.} Handbuch der deutschen 
Konnektoren~2: Semantik der deutschen Satzverkn$\ddot{\mbox{u}}$pfer.~--- Berlin: Walter de Gruyter, 2014. 
1327~p.
\bibitem{9-in}
\Au{Инькова О.\,Ю.} Логико-се\-ман\-ти\-че\-ские отношения: проблемы 
классификации~// Связность текста: мереологические ло\-ги\-ко-се\-ман\-ти\-че\-ские 
отношения.~--- М.: ЯСК, 2019. С.~11--98.
\bibitem{10-in}
\Au{Asher N., Lascarides~A.} Logics of conversation.~--- Cambridge: Cambridge University 
Press, 2003. 526~p.
\bibitem{11-in}
\Au{Carlson L., Marcu D.} Discourse tagging reference manual.~--- Marina del Rey, CA, USA: Information Sciences Institute, University of Southern 
California, 2001.  Technical Report ISI-TR-545. 87~p.



\bibitem{13-in} %12
\Au{Chiarcos Ch.} Towards interoperable discourse annotation: Discourse features in the 
Ontologies of Linguistic Annotation~// 9th Conference (International) on Language Resources 
and Evaluation Proceedings~/ Eds.\ N.~Calzolari, K.~Choukri, T.~Declerck, \textit{et al.}~--- Reykjavik, Iceland: European Language Resources Association 
(ELRA), 2014. P.~4569--4577.

\bibitem{12-in} %13
\Au{Benamara F., Taboada~M.} Mapping different rhetorical relation annotations: A~proposal~// 
4th Joint Conference on Lexical and Computational Semantics  Proceedings~/ Eds.\ M.~Palmer, G.~Boleda, P.~Rosso.~--- Denver, CO, USA: 
Association for Computational Linguistics, 2015. Р.~147--152. doi: 10.18653/v1/S15-1016.

\bibitem{14-in}
\Au{Sanders T., Demberg~V., Hoek~J., Scholman~M., Asr~F.\,T., Zufferey~S., Evers-Vermeul~J.} 
Unifying dimensions in coherence relations: How various annotation frameworks are related~// 
Corpus Linguist. Ling., 2018. Vol.~17. No.\,1. P.~1--71. doi:  
10.1515/cllt-2016-0078.
\bibitem{15-in}
\Au{Инькова О.\,Ю.} Определения дискурсивных отношений: опыт Надкорпусной базы 
данных коннекторов~// Компьютерная лингвистика и~интеллектуальные технологии: По 
мат-лам ежегодной \mbox{Междунар.} конф. <<Диалог>>.~--- М.: РГГУ, 2021. Вып.~20(27). 
С.~328--338.
\bibitem{16-in}
\Au{Инькова О.\,Ю., Кружков М.\,Г.} Структурированные определения дискурсивных 
отношений в~Надкорпусной базе данных коннекторов~// Информатика и~её применения, 
2021. Т.~15. Вып.~4. С.~27--32. doi: 10.14357/19922264210404. EDN: EZJXVI.

\bibitem{17-in}
\Au{Инькова О.\,Ю., Кружков М.\,Г.} Критерии определения семантической близости 
дискурсивных отношений~// Информатика и~её применения, 2023. Т.~17. Вып.~3.  
С.~100--106. doi: 10.14357/19922264230314. EDN: UJZJZI.

\bibitem{18-in}
\Au{Инькова О.\,Ю., Нуриев В.\,А.} Насколько лингвоспецифичен союз \textit{хотя}?~// 
Компьютерная лингвистика и~интеллектуальные технологии: По мат-лам ежегодной 
Междунар. конф. <<Диалог>>.~--- М.: РГГУ, 2018. Вып.~17(24). С.~254--266.

\bibitem{20-in} %19
Словарь современного русского литературного языка: в~17~т.~/ Под ред. 
В.\,И.~Чернышева.~--- М., Л.: Изд-во Академии наук СССР~/ Наука, 1950--1965.

\bibitem{19-in} %20
Русская грамматика~/ Под ред. Н.\,Ю.~Шведовой.~--- М.: Наука, 1980.   Т.~2.
714~с.

\bibitem{21-in}
Словарь русского языка: в~4~т.~/ Под ред. А.\,П.~Ев\-гень\-евой.~--- М.: Русский язык, 
 1981--1984. 
\bibitem{22-in}
\Au{Heine B., Kuteva T.} World lexicon of grammaticalization.~--- Cambridge: Cambridge 
University Press, 2002. 387~p.
\end{thebibliography}

 }
 }

\end{multicols}

\vspace*{-10pt}

\hfill{\small\textit{Поступила в~редакцию 15.10.23}}

\vspace*{8pt}

%\pagebreak

%\newpage

%\vspace*{-28pt}

\hrule

\vspace*{2pt}

\hrule



\def\tit{EVALUATING THE DEGREE OF~DISCOURSE RELATIONS SEMANTIC AFFINITY: 
METHODS AND~INSTRUMENTS}


\def\titkol{Evaluating the degree of~discourse relations semantic affinity: 
Methods and instruments}


\def\aut{O.\,Yu.~Inkova$^{1,2}$ and~M.\,G.~Kruzhkov$^1$}

\def\autkol{O.\,Yu.~Inkova and~M.\,G.~Kruzhkov}

\titel{\tit}{\aut}{\autkol}{\titkol}

\vspace*{-14pt}


\noindent
$^1$Federal Research Center ``Computer Science and Control'' of the Russian Academy of Sciences, 
44-2~Vavilov\linebreak
$\hphantom{^1}$Str., Moscow 119333, Russian Federation

\noindent
$^2$University of Geneva, 22 Bd des Philosophes, CH-1205 Geneva 4, Switzerland


\def\leftfootline{\small{\textbf{\thepage}
\hfill INFORMATIKA I EE PRIMENENIYA~--- INFORMATICS AND
APPLICATIONS\ \ \ 2023\ \ \ volume~17\ \ \ issue\ 4}
}%
 \def\rightfootline{\small{INFORMATIKA I EE PRIMENENIYA~---
INFORMATICS AND APPLICATIONS\ \ \ 2023\ \ \ volume~17\ \ \ issue\ 4
\hfill \textbf{\thepage}}}

\vspace*{3pt}




\Abste{The methods for evaluating semantic affinity of discourse relations are examined. The 
authors propose several approaches to this problem using two information resources: 
a~collection of structured definitions of logical-semantic relations (LSRs) formed by the authors
and the Supracorpora 
Database of Connectives incorporating\linebreak\vspace*{-12pt}}

\Abstend{corpus-based annotations of translation correspondences 
that include text fragments with LSR markers in Russian,
French, and Italian. It is demonstrated that when it comes to 
assessing the semantic affinity of LSRs, the following factors will be of a~higher priority: affiliation of 
distinctive features of LSRs with the same family in the structured definitions of relations; correspondences 
between markers of different LSRs in the source and target texts; and cases when different LSRs are 
regularly expressed by the same markers in different contexts. Of a~lesser importance is the factor of 
compatibility of different LSRs within the same context. It is assumed that based on the proposed 
methods, it will become possible to specify more precisely which distinguishing features of LSRs 
have the greatest impact on their potential semantic affinity.}

\KWE{supracorpora database; logical-semantic relations; connectives; annotation; faceted 
classification}


  \DOI{10.14357/19922264230412}{FXTSPZ}

\vspace*{-16pt}

\Ack

\vspace*{-3pt}

\noindent
The research was carried out using the infrastructure of the Shared Research Facilities ``High 
Performance Computing and Big Data'' (CKP ``Informatics'') of FRC CSC RAS (Moscow).


\vspace*{6pt}

  \begin{multicols}{2}

\renewcommand{\bibname}{\protect\rmfamily References}
%\renewcommand{\bibname}{\large\protect\rm References}

{\small\frenchspacing
 {%\baselineskip=10.8pt
 \addcontentsline{toc}{section}{References}
 \begin{thebibliography}{99}
\bibitem{1-in-1}
\Aue{Hobbs, J.\,R.} 1976. A~computational approach to discourse analyses. New York, NY: 
Department of Computer Science, City College, City University of New York. Research Report  
76-2. 28--38.
\bibitem{2-in-1}
\Aue{Hobbs, J.\,R.} 1978. Why is discourse coherent? Menlo Park, CA: SRI International. SRI 
Technical Note 176. 44~p.
\bibitem{3-in-1}
\Aue{Halliday, M.\,A.\,K., and R.~Hasan.} 1976. \textit{Cohesion in English}. London: Longman. 
374~p.


\bibitem{5-in-1} %4
\Aue{Mann, W.\,C., and S.\,A.~Thompson.} 1988. Rhetorical structure theory: Towards 
a~functional theory of text organization. \textit{Text} 8(3):243--281. doi: 
10.1515/text.1.1988.8.3.243.
\bibitem{6-in-1} %5
\Aue{Asher, N.} 1993. \textit{Reference to abstract objects in discourse}. Dordrecht: Kluwer. 
455~p.
\bibitem{4-in-1} %6
\Aue{Halliday, M.\,A.\,K.} 1994. \textit{An introduction to functional grammar}. 2nd ed. London: 
Edward Arnold. 434~p.

\bibitem{7-in-1}
PDTB Research Group. 2007. The Penn Discourse Treebank 2.0 annotation manual. Philadelphia, 
PA: Institute for Research in Cognitive Science, University of Pennsylvania. Technical Report 
IRCS-08-01. 104~p. Available at: {\sf https://www.cis.upenn.edu/$\sim$elenimi/pdtb-manual.pdf} 
(accessed November~28, 2023).
\bibitem{8-in-1}
\Aue{Breindl, E., A.~Volodina, and U.\,H.~Wa{\!\ptb{\!\ss}}ner.} 2014. \textit{Handbuch der 
deutschen Konnektoren~2: Semantik der deutschen Satzverkn$\ddot{\mbox{u}}$pfer}. Berlin: Walter de Gruyter. 
1327~p.
\bibitem{9-in-1}
\Aue{Inkova, O.\,Yu.} 2019. Logiko-semanticheskie otnosheniya: problemy klassifikatsii  
[Logical-semantic relations: Classification problems]. \textit{Svyaznost' teksta: mereologicheskie 
logiko-semanticheskie otnosheniya} [Text coherence: Mereological logical semantic relations]. 
Moscow: LRC Publishing House. 11--98.
\bibitem{10-in-1}
\Aue{Asher, N., and A.~Lascarides.} 2003. \textit{Logics of conversation}. Cambridge: Cambridge 
University Press. 526~p.
\bibitem{11-in-1}
\Aue{Carlson, L., and D.~Marcu.} 2001. Discourse tagging reference manual.  Marina del Rey, CA: Information Sciences Institute, University of Southern 
California. Technical Report 
ISI-TR-545.  87~p. Available at: {\sf https://www.isi.edu/~marcu/discourse/tagging-ref-manual.pdf} 
(accessed November~28, 2023).

\bibitem{13-in-1} %12
\Aue{Chiarcos, Ch.} 2014. Towards interoperable discourse annotation: Discourse features in the 
Ontologies of Linguistic Annotation. \textit{9th Conference (International) on\linebreak Language Resources 
and Evaluation Proceedings}. Eds. N.~Calzolari, K.~Choukri, T.~Declerck, \textit{et al.} Reykjavik, Iceland: 
European Language Resources Association. 4569--4577.
{ %\looseness=1

}

\bibitem{12-in-1} %13
\Aue{Benamara, F., and M.~Taboada.} 2015. Mapping different rhetorical relation annotations: 
A~proposal. \textit{4th Joint Conference on Lexical and Computational Semantics}. Eds. 
M.~Palmer, G.~Boleda, and P.~Rosso. Denver, CO, USA: Association for Computational 
Linguistics. 147--152. doi: 10.18653/v1/S15-1016.

\bibitem{14-in-1}
\Aue{Sanders, T., V.~Demberg, J.~Hoek, M.~Scholman, F.\,T.~Asr, S.~Zufferey, and  
J.~Evers-Vermeul.} 2018. Unifying dimensions in coherence relations: How various annotation 
frameworks are related. \textit{Corpus Linguist. Ling.} 17(1):1--71. doi: 10.1515/cllt-2016-0078.
\bibitem{15-in-1}
\Aue{Inkova, O.\,Yu.} 2021. Opredeleniya diskursivnykh otnosheniy: opyt Nadkorpusnoy bazy 
dannykh konnektorov [Definition of discursive relations: The experience of the supracorpora 
database of connectors]. \textit{Komp'yuternaya lingvistika i~intellektual'nye Tekhnologii: Po 
mat-lam ezhegodnoy Mezhdunar.  konf. ``Dialog''} [Computational Linguistics 
and Intellectual Technologies: Papers from the Annual Conference (International) ``Dialogue'']. 
Moscow: RGGU. 20(27):328--338.
\bibitem{16-in-1}
\Aue{Inkova, O.\,Yu., and M.\,G.~Kruzhkov.} 2021. Strukturirovannye opredeleniya 
diskursivnykh otnosheniy v~Nadkorpusnoy baze dannykh konnektorov [Structured definitions of 
discourse relations in the Supracorpora Database of Connectives]. \textit{Informatika i~ee 
Primeneniya~--- Inform. Appl.} 15(4):27--32. doi: 10.14357/ 19922264210404. EDN: EZJXVI.
\bibitem{17-in-1}
\Aue{Inkova, O.\,Yu., and M.\,G.~Kruzhkov.} 2023. Kriterii opredeleniya semanticheskoy blizosti 
diskursivnykh otnosheniy [Evaluation criteria for discourse relations semantic affinity]. 
\textit{Informatika i~ee Primeneniya~--- Inform. Appl.} 17(3):100--106. doi: 
10.14357/19922264230314. EDN: UJZJZI.

\pagebreak


\bibitem{18-in-1}
\Aue{Inkova, O.\,Yu., and V.\,A.~Nuriev.} 2018. Naskol'ko lingvospetsifichen soyuz \textit{khotya}? [To 
what extent is the conjunction \textit{khotya} language-specific?]. \textit{Komp'yuternaya lingvistika 
i~intellektual'nye tekhnologii: Po mat-lam ezhegodnoy Mezhdunar. konf. ``Dialog''} 
[Computational Linguistics and Intellectual Technologies: Papers from the Annual Conference 
(International) ``Dialogue'']. Moscow: RGGU. 17(24):254--266. 

\bibitem{20-in-1} %19
Chernyshev, V.\,I., ed. 1950--1965. \textit{Slovar' sovremennogo russkogo literaturnogo yazyka} 
[Dictionary of modern Russian literary language]. In 17~vols. Moscow, Leningrad: USSR Academy 
of Sciences Publishing House/Nauka.

\bibitem{19-in-1} %20
Shvedova, N.\,Yu., ed. 1980. \textit{Russkaya grammatika} [Russian grammar]. Moscow: Nauka. Vol.~2. 714~p.

\bibitem{21-in-1} %21
Evgen'eva, A.\,P., ed. 1981--1984. \textit{Slovar' russkogo yazyka} [Dictionary of the Russian 
language].  Moscow: Russkiy yazyk. 4~vols.


\bibitem{22-in-1}
\Aue{Heine, B., and T.~Kuteva.} 2002. \textit{World lexicon of grammaticalization}. Cambridge: 
Cambridge University Press. 387~p.

\end{thebibliography}

 }
 }

\end{multicols}

\vspace*{-6pt}

\hfill{\small\textit{Received October 5, 2023}} 

%\vspace*{-18pt}

\Contr

\vspace*{-4pt}

\noindent
\textbf{Inkova Olga Yu.} (b.\ 1965)~--- Doctor of Science in philology, senior scientist, Federal 
Research Center ``Computer Science and Control'' of the Russian Academy of Sciences,  
44-2~Vavilov Str., Moscow 119333, Russian Federation; faculty member, University of Geneva, 
22~Bd des Philosophes, CH-1205 Geneva~4, Switzerland; \mbox{olyainkova@yandex.ru}

\vspace*{3pt}

\noindent
\textbf{Kruzhkov Mikhail G.} (b.\ 1975)~--- senior scientist, Federal Research Center ``Computer 
Science and Control'' of the Russian Academy of Sciences, 44-2~Vavilov Str., Moscow 119333, 
Russian Federation; \mbox{magnit75@yandex.ru}


\label{end\stat}

\renewcommand{\bibname}{\protect\rm Литература}     %7
\def\stat{zatsman}

\def\tit{ТРАНСФОРМАЦИИ ОБЪЕКТОВ ПЕРВОГО И~ВТОРОГО ПОРЯДКА 
В~ЛЕКСИКОГРАФИЧЕСКОЙ ИНФОРМАЦИОННОЙ СИСТЕМЕ$^*$}

\def\titkol{Трансформации объектов первого и~второго порядка 
в~лексикографической информационной системе}

\def\aut{И.\,М.~Зацман$^1$}

\def\autkol{И.\,М.~Зацман}

\titel{\tit}{\aut}{\autkol}{\titkol}

\index{Зацман И.\,М.}
\index{Zatsman I.\,M.}


{\renewcommand{\thefootnote}{\fnsymbol{footnote}} \footnotetext[1]
{Исследование выполнено в~ФИЦ ИУ РАН за счет гранта Российского научного фонда №\,24-18-00155, {\sf 
https://rscf.ru/project/24-18-00155}. Работа выполнялась с~использованием инфраструктуры Центра 
коллективного пользования <<Высокопроизводительные вычисления и~большие данные>> (ЦКП 
<<Информатика>>) ФИЦ ИУ РАН (г.\ Москва).}}


\renewcommand{\thefootnote}{\arabic{footnote}}
\footnotetext[1]{ Федеральный исследовательский центр <<Информатика и~управление>> Российской академии наук, 
\mbox{izatsman@yandex.ru}}

\vspace*{-12pt}


  
  \Abst{Рассматриваются теоретические основания проектирования информационных 
технологий (ИТ) интеграции двуязычных словарей и~параллельных корпусов. Дано описание 
первых результатов создания третьего уровня классификации трансформаций объектов 
предметной области информатики, которую предполагается использовать при создании 
концепции лексикографической информационной системы, обеспечивающей интеграцию. 
Все сущности информатики в~статье разделены на два глобальных класса: объекты и~их 
трансформации. Для каждого такого класса конструируется своя классификация. Ранее были 
описаны два верхних уровня классификации трансформаций объектов предметной области. 
В~данной статье рассматривается третий уровень этой классификации. Основанием для 
построения самого верхнего ее уровня служило деление предметной области информатики 
на среды (ментальная, сенсорно воспринимаемая, цифровая и~ряд других сред), каждая из 
которых по определению включает объекты одной природы. Основанием для построения 
второго уровня классификации трансформаций объектов служила типология знаковых  
сис\-тем А.~Соломоника. Цель статьи состоит в~систематизации трансформаций первого 
и~второго порядка объектов предметной области на третьем уровне этой классификации. 
Основанием для систематизации служит средовая версия иерархии Акоффа.}
  
  \KW{объекты предметной области; трансформации объектов; классификация; данные; 
информация; знание; лексикографическая информационная сис\-тема}

\DOI{10.14357/19922264240211}{VZTGVV}
  
\vspace*{3pt}


\vskip 10pt plus 9pt minus 6pt

\thispagestyle{headings}

\begin{multicols}{2}

\label{st\stat}
  
\section{Введение}

\vspace*{-9pt}

  Возникновение параллельных корпусов, в~которых предложениям 
оригинального текста со\-по\-став\-ле\-ны предложения его перевода, обеспечило 
возможность контрастивного лингвистического\linebreak \mbox{анализа} на принципиально 
новом уровне полноты и~точности, недостижимом в~докорпусную эпоху. 
Пионерскими в~этой области стали работы \mbox{1990-х~гг}. Стига Йоханссона  
с~анг\-ло-нор\-веж\-ским корпусом~[1]. В России параллельные корпусы стали 
формироваться в~начале XXI~века в~рамках Национального корпуса русского 
языка~[2].
  
  Создатели двуязычных словарей используют параллельные корпусы для 
сбора материала и~эмпирической проверки своих гипотез, касающихся 
межъязы\-ко\-вой эквивалентности. Ценность параллельных корпусов 
определяется тем, что в~лингвистике этап сбора исходного материала считается 
наиболее трудоемким и~наименее творческим, а~параллельные корпусы 
позволяют значительно сэкономить время и~силы для творческого этапа 
создания словарей~[3].
 % 
  При этом двуязычные словари, создаваемые на основе исходного материала, 
извлеченного из параллельных корпусов, сейчас формируются без связей с~их 
текстами. Другими словами, онлайновые связи созданных словарей 
с~параллельными корпусами, которые служили источниками исходного 
материала, отсутствуют. 

Параллельные корпусы постоянно пополняются 
новыми текстами, в~предложениях которых можно обнаружить новые значения 
слов и~устойчивых словосочетаний. Однако при этом отсутствуют методы 
и~средства оперативного обновления словарей по корпусным данным. 
В~настоящее время проблема установления связей между двуязычными 
словарями и~параллельными корпусами (далее~--- проблема интеграции) 
находится на стадии поиска концептуальных подходов к~их интеграции на 
уровне значений.
  
  Подход к~решению проблемы интеграции, предлагаемый в~статье, учитывает 
  и~появление новых значений слов и~устойчивых словосочетаний, и~динамику 
смысловых значений, которая обусловлена развитием и~пополнением знания 
лингвистов, фиксирующих эти значения в~результате семантического анализа 
пополняемых корпусных данных. Проведенные эксперименты показали, что 
обнаружение нового лингвистического знания обусловливает и~формирование 
дефиниций новых значений, и~пересмотр уже существующих дефиниций~[4, 5].
  
  Например, в~проведенных экспериментах с~использованием ЦКП 
<<Информатика>> ФИЦ ИУ РАН фиксировалась эволюция значений немецких 
модальных глаголов, исходное состояние значений которых было описано 
в~не\-мец\-ко-рус\-ском словаре. В~экспериментальном массиве текстов как 
потенциальных источниках нового знания 16\,268 предложений содержали 
немецкие модальные глаголы и~в~2041 из них встречался глагол sollen. 
В~начале эксперимента в~словаре были описаны~12~значений этого модального 
глагола. По окончании эксперимента лингвисты обнаружили два новых его 
значения, согласовали их дефиниции и~описали эволюцию дефиниций~[6, 7].
  
  Таким образом, для решения проблемы интеграции требуется фиксировать 
новое знание, обнаруженное лингвистами в~текстовых данных параллельных 
корпусов, отслеживать эволюцию знания, представленного в~виде дефиниций 
значений слов и~устойчивых словосочетаний, и,~соответственно, 
актуализировать электронные двуязычные словари. Предлагаемый 
концептуальный подход к~интеграции, который планируется реализовать 
в~процессе проектирования лексикографической информационной сис\-те\-мы, 
фиксирующей эволюцию лингвистического знания, основан на решении 
следующих задач:\\[-14pt]
  \begin{itemize}
  \item категоризация трех базовых понятий информатики, включенных 
  в~иерархию Акоффа~[8] (данные, информация, знание), на объекты 
проектируемой сис\-те\-мы, которая необходима, чтобы фиксировать 
<<кванты>> нового знания и~отслеживать его эволюцию в~этой сис\-теме;\\[-15pt]
  \item  систематизация трансформаций объектов этой сис\-темы.\\[-14pt]
  \end{itemize}
  
  Цель статьи и~состоит в~решении двух задач: категоризации трех базовых 
понятий информатики на объекты лексикографической информационной  
сис\-те\-мы и~сис\-те\-ма\-ти\-за\-ции трансформаций первого и~второго порядка 
ее объектов.
  
  Трансформациями первого порядка, о которых сказано в~формулировке цели 
статьи, называются взаимные преобразования между двумя объектами  
сис\-те\-мы одной природы. Например, перевод в~сис\-те\-ме текста с~русского 
языка на английский относится к~ним. Трансформациями второго порядка 
и~выше называются взаимные преобразования между двумя и~более объектами 
разной природы. Например, кодирование символов текс\-та компьютерными 
кодами и~их декодирование относятся по определению к~трансформациям 
второго порядка.

%\vspace*{-9pt}
  
\section{Процессы трансформаций в~информатике}

%\vspace*{-3pt}

Процессы трансформаций, рассматриваемые в~статье, относятся к~теоретическому ядру информатики, а~не 
только к~проектированию лексикографической информационной сис\-те\-мы. Например, из трех основных 
подходов к~описанию предметной об\-ласти информатики\footnote{В статье предметная область информатики 
трактуется согласно концепции полиадического компьютинга Пола Розенблума~\cite{9-zac}.} (объектный, 
трансформационный и~синтетический) сис\-те\-ма\-ти\-за\-ция трансформаций ближе всего ко второму 
подходу. Примерами первого подхода, в~рамках которого основное внимание уделяется объектам предметной 
области информатики и~в~меньшей степени отношениям\linebreak между ними, могут служить  
работы~\cite{8-zac, 10-zac, 11-zac}; \mbox{примерами} второго подхода, в~рамках которого основное внимание 
уделяется трансформациям и~в~меньшей степени трансформируемым объектам,~---  
работы~\cite{12-zac, 13-zac}; примерами третьего, синтетического подхода, в~котором уделяется внимание 
и~объектам предметной об\-ласти информатики, и~отношениям между ними, могут служить работы~\cite{14-zac, 
15-zac, 16-zac, 17-zac, 18-zac}.

  Таким образом, для описания трансформаций объектов лексикографической 
информационной\linebreak системы предпочтительнее всего трансформационный 
подход, который упоминается и~в определениях информатики. Например, 
в~2009~г.\ П.~Деннинг и~П.~Розенблум сформулировали суть \mbox{информатики} как 
компьютинга следующим образом: <<$\ldots$информатика~--- это не просто 
алгоритмы и~структуры данных; это преобразования [трансформации] 
представлений>>~\cite{12-zac}. Чуть позже, в~контексте краткого описания 
парадигмы информатики как компьютинга, П.~Деннинг и~П.~Фриман изменили 
эту формулировку на такую: <<Центральный объект внимания в~информатике 
можно определить как информационные процессы~--- \textit{естественные или 
искусственные процессы, преобразующие информацию} (курсив мой~--- 
И.\,З.)>>~\cite{13-zac}. Согласно парадигме, предлагаемой авторами этой 
статьи, на начальном этапе проектирования автоматизированных систем 
базовыми элементами моделей их функционирования служат 
\textit{информационные про\-цессы}.
  
  Однако если 15~лет назад в~формулировке из работы~\cite{13-zac} шла речь 
о~процессах, преобразующих информацию, то в~последние~10~лет в~спектр 
процессов трансформаций все чаще стали включать процессы, преобразующие 
не только информацию, но также и~другие объекты автоматизированных 
систем, в~первую очередь данные и~знания~[19--21]. Например, Виктория 
Стодден, позиционируя науку о~данных как одну из дисциплин информатики, 
говорит, что центральный объект исследований в~науке о~данных~--- это 
<<изучение обобщаемого извлечения знания из данных>>~\cite{21-zac}. 
Увеличение и~чис\-ла объектов, и~спект\-ра процессов их трансформаций 
в~автоматизированных сис\-те\-мах обуслов\-ли\-ва\-ет не\-об\-хо\-ди\-мость 
систематизации и~объектов, и~процессов их трансформаций на начальном этапе 
проектирования сис\-тем.
  
  Для создания концепции лексикографической информационной сис\-те\-мы 
и~проектирования ИТ, обеспечивающих интеграцию 
двуязычных словарей и~параллельных корпусов, сначала выполним 
категоризацию на объекты этой сис\-те\-мы трех базовых понятий информатики 
(данные, информация, знание) в~контексте построения классификаций 
сущностей ее предметной об\-ласти.
  
  Необходимость использования классификаций информатики в~процессе 
создания концепции проиллюстрируем, используя иерархию  
Акоффа~\cite{8-zac}. Он использовал принцип их вертикального размещения 
в~иерархии снизу вверх: данные, информация и~знание. Еще в~ней есть термин 
<<мудрость>>, который в~статье не рассматривается. Такое размещение Акофф 
прокомментировал так: <<Каждое из пе\-ре\-чис\-лен\-ных понятий [кроме данных] 
содержит в~себе нижестоящие$\ldots$>>~\cite{8-zac}.
  
  Этому принципу размещения и~комментарию Акоффа свойственны 
недостатки, проанализированные, в~частности, в~работе~\cite{10-zac}. Главный 
вывод, к~которому пришла Роули после изучения иерархии Акоффа, 
заключается в~следующем: <<$\ldots$информация определяется в~терминах 
данных, знание~--- в~терминах информации$\ldots$ но существует меньше 
консенсуса в~описании трансформаций, которые преобразуют сущности, 
расположенные ниже в~иерархии, в~те, которые находятся над ними, что 
приводит к~их терминологической неопределенности>>~\cite{10-zac}. Причина 
этой неопределенности, скорее всего, в~том, что базовые понятия информатики 
включены в~иерархию Акоффа изолированно от общего контекста 
классификаций сущностей ее предметной об\-ласти.

%\vspace*{-9pt}
  
\section{Классификации сущностей информатики}


%\vspace*{-2pt}

  Все сущности предметной области информатики в~работах~[22, 23] 
разделены на два глобальных класса: ее объекты и~их трансформации. Для 
каждого такого класса была предложена своя классификация. 
В~работе~\cite{22-zac} дано описание классификации объектов предметной 
области информатики, первый уровень которой содержит базовые понятия ее 
предметной области (данные, информация, знания и~др.).  
В~работе~\cite{23-zac} дано описание двух верхних уровней классификации 
трансформаций объектов предметной об\-ласти (см.\ рисунок 
в~работе~\cite{23-zac}). Основанием для построения самого верхнего ее уровня послужило деление 
предметной области информатики на среды\footnote{В~работе~\cite{24-zac} дано описание пяти сред 
предметной области информатики (ментальная; сенсорно воспринимаемая, или информационная; 
цифровая; нейро- и~ДНК-среда), каждая из которых по определению включает объекты одной и~той же 
природы.} и~степень разнообразия природы объектов, вовлеченных в~трансформации:
\begin{itemize}
\item  первый класс верхнего уровня классификации включает 
трансформации объектов в~пределах среды только одной природы 
(трансформации первого порядка);
\item  второй класс включает трансформации объектов, относящихся 
к~двум средам разной природы (трансформации второго порядка);
\item третий и~последующие классы включают трансформации объектов, 
относящихся к~трем и~более средам разной природы (трансформации 
третьего и~более высоких порядков).
\end{itemize}

  В работе~\cite{23-zac} были приведены примеры для трех первых классов 
трансформаций, включая пример трансформаций объектов, относящихся 
к~двум средам разной природы (компьютерное кодирование символов текстов 
с~по\-мощью таб\-лиц Unicode).
  
Основанием для построения второго уровня классификации трансформаций объектов послужила типология 
знаковых сис\-тем А.~Соломоника~\cite[c.~131]{25-zac}: естественные знаковые сис\-те\-мы, образные,  
ес\-тест\-вен\-но-язы\-ко\-в$\acute{\mbox{ы}}$е,  
вер\-баль\-но-не\-сло\-вес\-ные сис\-те\-мы записи\footnote{Под системой записи понимается знаковая 
система, сочетающая вербальные знаки с~несловесными (языки нотной записи, карт, таблиц и~др.).} 
и~формализованные знаковые сис\-те\-мы, включая математические. Введем понятие обобщенного текста~--- 
это текст, который может быть создан в~любой из перечисленных знаковых систем. Тогда обобщенные тексты 
могут быть естественными, образными, ес\-тест\-вен\-но-язы\-ко\-в$\acute{\mbox{ы}}$\-ми,  
вер\-баль\-но-не\-сло\-вес\-ны\-ми и~формализованными. Второй уровень классификации трансформаций 
охватывает не все виды объектов предметной  
об\-ласти информатики, а~только перечисленные~5~видов текс\-тов и~их представления, вовлеченные 
в~процессы трансформаций в~одной или более средах вместе с~данными, знанием и~его концептами.

\begin{figure*}[b] %fig1
\vspace*{6pt}
      \begin{center}
     \mbox{%
\epsfxsize=121.191mm 
\epsfbox{zac-1.eps}
}
\end{center}
\vspace*{-6pt}
\Caption{Средовая версия иерархии Акоффа}
\end{figure*}

\section{Классификация трансформаций: построение~третьего 
уровня}

  Основанием для систематизации трансформаций первого и~второго порядка 
на третьем уровне этой классификации служит иерархия Акоффа~\cite{8-zac}, 
на основе которой и~была создана ее средов$\acute{\mbox{а}}$я версия~[26, 
27]. Для создания средов$\acute{\mbox{о}}$й версии была выполнена 
категоризация трех базовых понятий информатики (данные, информация, 
знания) на объекты лексикографической информационной сис\-те\-мы 
в~процессе создания ее концепции\linebreak (рис.~1).
  


  В отличие от классической иерархии Акоффа, в~ее 
средов$\acute{\mbox{о}}$й версии различаются три вида данных: сенсорно 
воспринимаемые, цифровые и~те данные, которые генерируются 
искусственными нейронными сетями (ИНС) в~системах искусственного интеллекта 
(далее~--- ИИ-дан\-ные). Последний вид данных необходим, например, для 
различения входа и~выхода процесса применения обученной 
ИНС в~цифровой модели генерации знания, описанию которой 
посвящена работа~\cite{27-zac}.
  
  Также предлагается различать два вида информации: сенсорно 
воспринимаемая и~цифровая. Кроме знания в~средов$\acute{\mbox{у}}$ю 
версию добавлены концепты и~ментальные образы сенсорно воспринимаемых 
данных. Последние служат промежуточной сущностью между сенсорно 
воспринимаемыми данными и~генерируемым знанием при описании процессов 
извлечения знания из текстовых данных лексикографической информационной 
системы. Описание объектов средов$\acute{\mbox{о}}$й версии иерархии 
Акоффа (см.\ рис.~1) и~отношений между ними дано в~работах~\cite{26-zac, 28-zac}.
  
  В средов$\acute{\mbox{о}}$й версии число объектов равно восьми. Если 
учитывать направления трансформаций, то между восемью объектами на 
рис.~1 она включает~16 их видов (трансформации на границе между сенсорно 
воспринимаемыми данными и~информацией, обозначенные символом~<<?>>, 
в~статье не рас\-смат\-ри\-ва\-ют\-ся). В~будущем число объектов 
в~средов$\acute{\mbox{о}}$й версии, которая выбрана как основание для 
сис\-те\-ма\-ти\-за\-ции трансформаций первого и~второго порядка, может быть 
увеличено. Для построения классификации трансформаций 
важ\-но не возможное увеличение числа объектов 
и~трансформаций между ними, а то, что их виды в~средов$\acute{\mbox{о}}$й 
версии распределены между трансформациями первого и~второго порядка. Из 
16~видов на рис.~1 шесть относятся к~трансформациям первого порядка, это\linebreak 
виды с~номерами~7, 8, 13--16 (далее~--- типология трансформаций первого 
порядка), а~десять~--- к~трансформациям второго порядка, это виды 
с~\mbox{номерами}~1--6 и~9--12 (далее~--- типология трансформаций второго 
порядка). Разместим обе типологии на третьем уровне классификации (см.\ ее 
схему на рис.~2). Перечислим виды трансформаций первой типологии, вводя 
в~скобках их краткие названия, используемые ниже на рис.~3:
  \begin{description}
  \item[\,] 7~--- членение знания на концепты с~помощью одной или нескольких 
знаковых систем (далее~--- членение знания);
  \item[\,] 8~--- формирование знания на основе концептов (формирование 
знания);
  \item[\,] 13~--- обучение ИНС;
  \end{description}
  
  \vspace*{-6pt}
  
  \pagebreak
  
  \end{multicols}
  
  \begin{figure*} %fig2
\vspace*{1pt}
      \begin{center}
     \mbox{%
\epsfxsize=127.513mm 
\epsfbox{zac-2.eps}
}
\end{center}
\vspace*{-9pt}
\Caption{Схема трех верхних уровней классификации трансформаций объектов (объединены 
по три слоя и~для второго, и~для третьего уровней этой классификации)}
\end{figure*}
  
  \begin{multicols}{2}
  
  \noindent
  \begin{description}
  \item[\,] 14~--- восстановление обучающей информации на основе 
содержания обученной ИНС (обращение ИНС);
  \item[\,] 15~--- использование обученной ИНС (использование ИНС);



  \item[\,] 16~--- восстановление исходных данных, соответствующих 
полученным результатам работы обучен\-ной ИНС (восстановление исходных данных 
по результатам ИНС).
  \end{description}
  
  
  Не все виды трансформаций 13--16 поддерживаются в~конкретных системах 
искусственного интеллекта, но с~теоретической точки зрения все их 
предлагается включить в~первую типологию для полноты спектра видов 
трансформаций.
  
  Перечислим виды трансформаций второй типологии:
  \begin{description}
  \item[\,] 1~--- декодирование цифровых данных в~компьютерных системах 
(декодирование данных);
  \item[\,]  2~--- кодирование сенсорно воспринимаемых данных (кодирование 
данных);
  \item[\,] 3~--- ментальное копирование сенсорно воспринимаемых данных 
(ментальное копирование);
  \item[\,] 4~--- восстановление сенсорно воспринимаемых данных по 
ментальным образам (восстановление по образам);
  \item[\,] 5~--- смысловая интерпретация без деления на концепты ментальных 
образов сенсорно воспринимаемых данных (смысловая интерпретация);
  \item[\,] 6~--- восстановление ментальных образов (восстановление образов);
  \item[\,] 9~--- представление концептов в~виде сенсорно воспринимаемой 
информации, например текс\-та\-ми, формулами, таблицами, рисунками и~т.\,д.\ 
(представление концептов);
  \item[\,] 10~--- понимание смысла сенсорно воспринимаемой информации 
(понимание смысла);
  \item[\,] 11~--- кодирование сенсорно воспринимаемой информации 
(кодирование информации);
\end{description}

\vspace*{-6pt}

\pagebreak

\end{multicols}

\begin{figure*} %fig3
\vspace*{1pt}
      \begin{center}
     \mbox{%
\epsfxsize=163mm 
\epsfbox{zac-3.eps}
}
\end{center}
\vspace*{-9pt}
\Caption{Схема частного случая классификации трансформаций объектов (трансформации 
пронумерованы согласно рис.~1)}
\end{figure*}

\begin{multicols}{2}

\noindent
\begin{description}

  \item[\,] 12~--- декодирование цифровой информации (декодирование 
информации).
  \end{description}
  
  Отметим, что в~существующих ИТ
  и~компьютерных системах наиболее часто используются виды 
трансформаций~13 и~15 типологии первого порядка и~1, 2, 11 и~12 типологии 
второго порядка. На рис.~2 в~первом слое третьего уровня классификации 
показаны типологии первого порядка без указания числа трансформаций в~них 
и~без детализации трансформируемых объектов.
  
  Во втором слое третьего уровня классификации условно (без названий) 
показаны типологии второго порядка. Также на рис.~2 в~третьем слое третьего 
уровня классификации условно (также без названий) показаны типологии 
третьего порядка, которые планируется рассмотреть в~отдельной статье. По 
определению они должны включать трансформации между тремя объектами 
разной природы, но средов$\acute{\mbox{а}}$я версия иерархии Акоффа 
включает трансформации только между двумя объектами разной природы. 
Поэтому потребуется другое основание для их систематизации (ранее были 
рассмотрены отдельные примеры трансформаций третьего 
порядка\footnote{Далеко не всегда трансформации третьего и~более высоких порядков можно 
рассматривать как последовательность трансформаций второго порядка. Примером этого могут 
служить трансформации в~процессе обучения пациента пользованию роботизированной рукой, 
охватывающие личностные концепты пациента, релевантные его намерениям, сигналы активности 
мозга как объекты нейросреды и~компьютерные коды~\cite{29-zac}.}~\cite{29-zac}).

\section{Классификация трансформаций: частный~случай}

  Выше было отмечено, что в~будущем число объектов 
в~средов$\acute{\mbox{о}}$й версии иерархии Акоффа может быть увеличено. 
Это означает, что увеличатся и~чис\-ло объектов, и~чис\-ло трансформаций между 
ними в~классификации трансформаций, так как эта средов$\acute{\mbox{а}}$я 
версия служит по определению основанием для систематизации 
трансформаций первого и~второго порядка. Поэтому на третьем уровне рис.~2 
указаны типологии без детализации объектов и~без указания числа 
трансформаций в~каждой из них. С~одной стороны, при таком подходе 
получаем достаточно общий вид этой классификации, так как она не зависит от 
числа объектов в~том или ином варианте средов$\acute{\mbox{о}}$й версии 
(и~это существенно упрощает рис.~2). С~другой стороны, на третьем уровне 
такой общей классификации подразумевается, но не эксплицируется природа 
трансформируемых объектов и~их возможные сочетания в~трансформациях. 

При проектировании лексикографической информационной системы важно 
эксплицировать природу трансформируемых объектов и~их возможные 
сочетания.
  %
  Поэтому в~парадигму информатики~\cite{30-zac} кроме общей 
классификации трансформаций предлагается включать и~ее частные случаи, 
эксплицирующие природу трансформируемых объектов. 

В~этом разделе 
рассмотрим один частный случай, когда используются только естественные 
знаковые сис\-те\-мы из типологии А.~Соломоника~\cite{25-zac} вместе 
с~данными, знанием и~его концептами. Чис\-ло естественных языков при этом не 
ограничено. И~этот частный случай классификации включает только три 
класса природных трансформаций (первого, второго и~третьего порядка, см.\ 
схему классификации на рис.~3).
  
  Первый и~второй уровни схемы общей классификации (см.\ рис.~2) можно 
объединить в~один уровень в~этом частном случае. Ниже этого уровня 
приведено содержание типологий первого и~второго порядка без содержания 
типологий третьего по\-рядка.




  Наполнение типологий первого и~второго порядка соответствует 
средов$\acute{\mbox{о}}$й версии иерархии Акоффа на рис.~1, содержащей 
6~видов трансформаций типологии первого порядка и~10~видов 
трансформаций типологии второго порядка (на рис.~3 стрелки указывают 
направления трансформаций согласно средов$\acute{\mbox{о}}$й версии на рис.~1).
  
  Таким образом, частный случай классификации содержит для этих двух 
типологий 16~теоретически возможных трансформаций, 6 из которых 
в~настоящее время в~существующих ИТ применяются наиболее часто: виды 
трансформаций~1, 2, 11 и~12 типологии второго порядка реализуются 
с~помощью тех или иных методов ко\-ди\-ро\-ва\-ния/де\-ко\-ди\-ро\-ва\-ния 
(например, с~использованием таблиц Unicode), а~виды трансформаций~13 и~15
 в~типологии первого порядка реализуются полностью с~по\-мощью процессов 
цифровой обработки компьютерами.
  
  Остальные виды трансформаций или применяются намного реже (это 
виды~3, 5, 7, 9 и~10), или находятся в~стадии поиска и~разработки (14 и~16) или 
в~настоящее время носят только теоретический характер, обеспечивая полноту 
первой и~второй типологий (4, 6 и~8). Знаком~<<?>> обозначены те виды 
трансформаций, которые по определению не существуют в~используемой 
парадигме информатики~\cite{30-zac}. Однако возможно, что в~других 
будущих подходах к~построению ее парадигмы эти виды трансформаций будут 
существовать.
  
\section{Заключение}

  На сегодняшний день процесс построения классификаций объектов 
предметной области информатики~\cite{22-zac} и~их  
трансформаций~\cite{23-zac} еще не завершен. Однако первые результаты их 
построения уже используются для создания концепции лексикографической 
информационной сис\-те\-мы, обеспечивающей интеграцию двуязычных 
словарей и~параллельных корпусов.
  
  \bigskip
  
  
  Автор признателен рецензентам за помощь в~улучшении статьи.
  
{\small\frenchspacing
 { %\baselineskip=10.6pt
 %\addcontentsline{toc}{section}{References}
 \begin{thebibliography}{99}
\bibitem{1-zac}
\Au{Aijmer K., Altenberg~B.} Advances in corpus-based contrastive linguistics. Studies in honour 
of Stig Johansson.~--- Amsterdam: John Benjamins, 2013. 295~p.  doi: 10.1075/scl.54.
\bibitem{2-zac}
\Au{Добровольский Д.\,О., Кретов~А.\, А., Шаров~С.\,А.} Корпус параллельных текстов~// 
Научная и~техническая информация. Сер.~2: Информационные процессы и~сис\-те\-мы, 2005. 
№\,6. С.~16--27.
\bibitem{3-zac}
\Au{Добровольский Д.\,О.} Корпус параллельных текстов и~сопоставительная 
лексикология~// Труды Института русского языка им.\ В.\,В.~Виноградова, 2015. №\,6. 
С.~413--449. EDN: VJQBHP.
\bibitem{4-zac}
\Au{Гончаров А.\,А., Зацман~И.\,М., Кружков~М.\,Г.} Эволюция классификаций 
в~надкорпусных базах данных~// Информатика и~её применения, 2020. Т.~14. Вып.~4. 
С.~108--116. doi: 10.14357/19922264200415.  
EDN: \mbox{GKWBZT}.
\bibitem{5-zac}
\Au{Гончаров А.\, А., Зацман И. \,М., Кружков~М.\, Г}. Представление новых 
лексикографических знаний в~динамических классификационных сис\-те\-мах~// 
Информатика и~её применения, 2021. Т.~15. Вып.~1. С.~86--93.  doi: 10.14357/19922264210112. EDN: OPEFXW.
\bibitem{6-zac}
\Au{Zatsman I.} Finding and filling lacunas in linguistic typologies~// 15th Forum (International) 
on Knowledge Asset Dynamics Proceedings.~--- Matera, Italy: Institute of Knowledge Asset 
Management, 2020. P.~780--793.
\bibitem{7-zac}
\Au{Zatsman I.} Three-dimensional encoding of emerging meanings in AI-systems~// 21st 
European Conference on Knowledge Management Proceedings.~--- Reading, U.K.: Academic 
Publishing International Ltd., 2020. P.~878--887.
\bibitem{8-zac}
\Au{Ackoff R.} From data to wisdom~// J.~Applied Systems Analysis, 1989. Vol.~16. No.\,1. P.~3--9.
\bibitem{9-zac}
\Au{Rosenbloom P.\,S.} On computing: The fourth great scientific domain.~--- Cambridge, MA, 
USA: MIT Press, 2013. 307~p.
\bibitem{10-zac}
\Au{Rowley J.} The wisdom hierarchy: Representations of the DIKW hierarchy~// J.~Inf. 
Sci., 2007. Vol.~33. Iss.~2. P.~163--180. doi: 10.1177/0165551506070706.
\bibitem{11-zac} 
\Au{Frick$\acute{\mbox{e}}$~M.\,H.} Data--Information--Knowledge--Wisdom (DIKW) pyramid, 
framework, continuum~// Encyclopedia of big data~/ Eds. L.~Schintler, C.~McNeely.~--- Cham: 
Springer, 2018. 4~p. doi: 10.1007/978-3-319-32001-4\_331-1.
\bibitem{12-zac}
\Au{Denning P., Rosenbloom~P.} Computing: The fourth great domain of science~// Commun. 
ACM, 2009. Vol.~52. Iss.~9. P.~27--29.
\bibitem{13-zac}
\Au{Denning P., Freeman~P.} Computing's paradigm~// Commun.  ACM, 2009. Vol.~52. 
Iss.~12. P.~28--30. doi: 10.1145/ 1610252.1610265.
\bibitem{17-zac} %14
\Au{Farradane J.} Knowledge, information, and information science~// J.~Inf. Sci., 
1980. Vol.~2. Iss.~2. P.~75--80. doi: 10.1177/01655515800020020.

\bibitem{15-zac}
\Au{Шрейдер Ю.\,А.} Информация и~знание~// Сис\-тем\-ная концепция информационных 
процессов.~--- М.: ВНИИСИ, 1988. С.~47--52.
\bibitem{16-zac}
\Au{Ingwersen P.} Information and information science~// Enclyclopaedie of library and 
information science~/ Eds. J.\,D.~McDonald, 
M.~Levine-Clark.~--- New York, NY, USA: Marcel Dekker Inc., 1992. Vol.~56. Sup.~19. 
P.~137--174.

\bibitem{14-zac} %17
Информатика как наука об информации: Информационный, документальный, 
технологический, экономический, социальный и~организационный аспекты~/ Под ред. 
Р.\,С.~Гиляревского.~--- М.: Фаир-Пресс, 2006. 592~с.

\bibitem{18-zac}
\Au{Hjorland B.} Library and information science: practice, theory, and philosophical basis~// 
Inform. Process. Manag., 2000. Vol.~36. Iss.~3. P.~501--531. doi:  
10.1016/S0306-\mbox{4573(99)00038-2}.
\bibitem{19-zac}
Deep shift~--- technology tipping points and societal impact.~--- Geneva: WE Forum, 2015. 44~p. 
{\sf http://www3.weforum.org/docs/WEF\_GAC15\_ Technological\_Tipping\_Points\_report\_2015.pdf}.
\bibitem{20-zac}
\Au{Berman F., Rutenbar~R., Hailpern~B., Christensen~H., Davidson~S., Estrin~D., 
Franklin~M., Martonosi~M., Raghavan~P., Stodden~V., Szalay~A.\,S.} Realizing the potential of 
data science~// Commun.  ACM, 2018. Vol.~61. Iss.~4. P.~67--72. doi: 10.1145/3188721.

\bibitem{21-zac}
\Au{Stodden V.} The data science life cycle: A~disciplined approach to advancing data science as 
a~science~// Commun.  ACM, 2020. Vol.~63. Iss.~7. P.~58--66. doi: 10.1145/ 3360646.


\bibitem{23-zac} %22
\Au{Зацман И.\,М.} Научная парадигма информатики: классификация трансформаций 
объектов предметной об\-ласти~// Системы и~средства информатики, 2023. Т.~33. №\,4. 
С.~126--138. doi: 10.14357/08696527230412. EDN: ZIKUWO.

\bibitem{22-zac} %23
\Au{Зацман И.\,М.} Научная парадигма информатики: классификация объектов предметной  
об\-ласти~// Информатика и~её применения, 2023. Т.~17. Вып.~4. С.~96--103. doi: 
10.14357/19922264230413. EDN: FIUQAT.

\bibitem{24-zac}
\Au{Зацман И.\,М.} О~научной парадигме информатики: верхний уровень классификации 
объектов ее предметной об\-ласти~// Информатика и~её применения, 2022. Т.~16. Вып.~4. 
С.~73--79. doi: 10.14357/ 19922264220411. EDN: XZNKVI.

\bibitem{25-zac}
\Au{Соломоник А.\,Б.} Философия знаковых систем и~язык.~--- М.: ЛКИ, 2011. 408~с.
\bibitem{26-zac}
\Au{Зацман И.\,М.} Трансформация иерархии Акоффа в~научной парадигме информатики~// 
Информатика и~её применения, 2023. Т.~17. Вып.~3. С.~107--113. doi: 
10.14357/19922264230315. EDN: UMVRRV.

\bibitem{27-zac}
\Au{Zatsman I.} Building digital spiral models of knowledge generation~// 19th Forum 
(International) on Knowledge Asset Dynamics Proceedings.~--- Matera, Italy: Arts for Business 
Institute, 2024. P.~2185--2196.
\bibitem{28-zac}
\Au{Zatsman I.} Digital spiral model of knowledge creation and encoding its dynamics~// 18th 
Forum (International) on Knowledge Asset Dynamics Proceedings.~--- Matera, Italy: Arts for 
Business Institute, 2023. P.~581--596.
\bibitem{29-zac}
\Au{Зацман И.\,М.} Интерфейсы третьего порядка в~информатике~// Информатика и~её 
применения, 2019. Т.~13. Вып.~3. С.~82--89. doi: 10.14357/19922264190312. EDN: 
EHRQLF.

\bibitem{30-zac}
\Au{Зацман И.\,М.} Научная парадигма информатики как третьей культуры~//  
На\-уч\-но-тех\-ни\-че\-ская информация. Сер.~1: Организация и~методика информационной 
работы, 2023. №\,11. С.~1--14.

\end{thebibliography}

 }
 }

\end{multicols}

\vspace*{-9pt}

\hfill{\small\textit{Поступила в~редакцию 14.04.24}}

\vspace*{4pt}

%\pagebreak

%\newpage

%\vspace*{-28pt}

\hrule

\vspace*{2pt}

\hrule



\def\tit{OBJECT TRANSFORMATIONS OF~THE~FIRST AND~SECOND ORDER
IN~A~LEXICOGRAPHIC INFORMATION SYSTEM\\[-5pt]}


\def\titkol{Object transformations of~the~first and~second order
in~a~lexicographic information system}


\def\aut{I.\,M.~Zatsman}

\def\autkol{I.\,M.~Zatsman}

\titel{\tit}{\aut}{\autkol}{\titkol}

\vspace*{-13pt}


\noindent
Federal Research Center ``Computer Science and Control'' of the Russian Academy of Sciences, 
44-2~Vavilov Str., Moscow 119133, Russian Federation


\def\leftfootline{\small{\textbf{\thepage}
\hfill INFORMATIKA I EE PRIMENENIYA~--- INFORMATICS AND
APPLICATIONS\ \ \ 2024\ \ \ volume~18\ \ \ issue\ 2}
}%
 \def\rightfootline{\small{INFORMATIKA I EE PRIMENENIYA~---
INFORMATICS AND APPLICATIONS\ \ \ 2024\ \ \ volume~18\ \ \ issue\ 2
\hfill \textbf{\thepage}}}

\vspace*{2pt}



\Abste{The theoretical foundations of the design of information technologies used for 
the integration of bilingual dictionaries and parallel corpora are considered. The 
description of the first outcomes of the creation of the third\linebreak\vspace*{-12pt}}

\Abstend{ level of object 
transformations classification in the subject domain of informatics, which is supposed 
to be used
in creating the lexicographic information system providing integration, is 
given. All the entities of informatics are divided into two global classes: objects and 
their transformations. For each such class, its own classification is constructed. 
Previously, the two upper levels of the object transformation classification in the subject 
domain have been described. The present paper discusses the third level of this classification. The 
basis for the construction of its highest level was the division of the subject domain of 
informatics into media (mental, sensory, digital, and a~number of other media), each 
of which by definition includes objects of the same nature. The Solomonick's 
typology of sign systems served as the basis for constructing the second level of the 
object transformation classification. The aim of the paper is to systematize object 
transformations of the first and second orders at the third level of this classification. 
The basis for systematization is the medium version of the Ackoff's hierarchy.}

\KWE{subject domain objects; object transformations; classification; data; 
information; knowledge; lexicographic information system}


\DOI{10.14357/19922264240211}{VZTGVV}

\vspace*{-12pt}

\Ack

\vspace*{-3pt}


\noindent
The reported study was funded by the Russian Science Foundation, project  
No.\,24-18-00155, {\sf 
https://rscf.ru/project/24-18-00155}. The research was carried out using the infrastructure of the Shared 
Research Facilities ``High Performance Computing and Big Data'' (CKP 
``Informatics'') of FRC CSC RAS (Moscow) .
   


  \begin{multicols}{2}

\renewcommand{\bibname}{\protect\rmfamily References}
%\renewcommand{\bibname}{\large\protect\rm References}

{\small\frenchspacing
 {%\baselineskip=10.8pt
 \addcontentsline{toc}{section}{References}
 \begin{thebibliography}{99} 
\bibitem{1-zac-1}
\Aue{Aijmer, K., and B.~Altenberg.} 2013. \textit{Advances in corpus-based 
contrastive linguistics. Studies in honour of Stig Johansson}. Amsterdam: John 
Benjamins. 295~p. doi: 10.1075/scl.54.
\bibitem{2-zac-1}
\Aue{Dobrovolskiy, D.\,O., A.\,A.~Kretov, and S.\,A.~Sharov.} 2005. Korpus 
parallel'nykh tekstov [Corpus of parallel texts]. \textit{Nauchnaya i~tekhnicheskaya 
informatsiya. Ser. 2. Informatsionnye protsessy i~sistemy} [Scientific and Technical 
Information. Ser.~2: Information Processes and Systems] 6:16--27.
\bibitem{3-zac-1}
\Aue{Dobrovolskiy, D.\,O.} 2015. Korpus parallel'nykh tekstov i~sopostavitel'naya 
leksikologiya [The corpus of parallel texts and contrastive lexicology]. \textit{Trudy 
Instituta russkogo yazyka im. V.\,V.~Vinogradova} [Proceedings of the 
V.\,V.~Vinogradov Russian Language Institute] 6:413--449. EDN: VJQBHP.
\bibitem{4-zac-1}
\Aue{Goncharov, A.\,A., I.\,M.~Zatsman, and M.\,G.~Kruzhkov.} 2020. Evolyutsiya 
klassifikatsiy v~nadkorpusnykh ba\-zakh dannykh [Evolution of classifications in 
supracorpora databases]. \textit{Informatika i~ee Primeneniya~--- Inform. \mbox{Appl.}}  
14(4):108--116. doi: 10.14357/19922264200415.  
EDN: GKWBZT.
\bibitem{5-zac-1}
\Aue{Goncharov, A.\,A., I.\,M.~Zatsman, and M.\,G.~Kruzhkov.} 2021. 
Predstavlenie novykh leksikograficheskikh znaniy v~dinamicheskikh 
klassifikatsionnykh sistemakh [Representation of new lexicographical knowledge in 
dynamic classification systems]. \textit{Informatika i~ee Primeneniya~--- Inform. 
Appl.} 15(1):86--93. doi: 10.14357/19922264210112. EDN: OPEFXW.
\bibitem{6-zac-1}
\Aue{Zatsman, I.} 2020. Finding and filling lacunas in linguistic typologies. 
\textit{15th Forum (International) on Knowledge Asset Dynamics Proceedings}. 
Matera, Italy: Institute of Knowledge Asset Management. 780--793.
\bibitem{7-zac-1}
\Aue{Zatsman, I.} 2020. Three-dimensional encoding of emerging meanings in  
AI-systems. \textit{21st European Conference on Knowledge Management 
Proceedings}. Reading, U.K.: Academic Publishing International Ltd. 878--887.
\bibitem{8-zac-1}
\Aue{Ackoff, R.} 1989. From data to wisdom. \textit{J.~Applied Systems Analysis} 
16(1):3--9.
\bibitem{9-zac-1}
\Aue{Rosenbloom, P.\,S.} 2013. \textit{On computing: The fourth great scientific 
domain}. Cambridge, MA: MIT Press. 307~p.
\bibitem{10-zac-1}
\Aue{Rowley, J.} 2007. The wisdom hierarchy: Representations of the DIKW 
hierarchy. \textit{J.~Inf. Sci.} 33(2):163--180. doi: 10.1177/0165551506070706.
\bibitem{11-zac-1}
\Aue{Frick$\acute{\mbox{e}}$, M.\,H.} 2018.  
Data-Information-Knowledge-Wisdom (DIKW) pyramid, framework, continuum. 
\textit{Encyclopedia of big data}. Eds. L.~Schintler and C.~McNeely. Cham: 
Springer. 4~p. doi: 10.1007/978-3-319-32001- 4\_331-1.
\bibitem{12-zac-1}
\Aue{Denning, P., and P.~Rosenbloom.} 2009. Computing: The fourth great domain 
of science. \textit{Commun. ACM} 52(9):27--29.
\bibitem{13-zac-1}
\Aue{Denning, P., and P.~Freeman.} 2009. Computing's paradigm. \textit{Commun. 
ACM} 52(12):28--30. doi: 10.1145/ 1610252.1610265.

\bibitem{17-zac-1} %14
\Aue{Farradane, J.} 1980. Knowledge, information, and information science. 
\textit{J.~Inf. Sci.} 2(2):75--80. doi: 10.1177/ 01655515800020020.

\bibitem{15-zac-1}
\Aue{Shreyder, Yu.\,A.} 1988. Informatsiya i~znanie [Information and knowledge]. 
\textit{Sistemnaya kontseptsiya in\-for\-ma\-tsi\-on\-nykh protsessov} [System concept of 
information processes]. Moscow: VNIISI. 47--52.
\bibitem{16-zac-1}
\Aue{Ingwersen, P.} 1995. Information and information science. 
\textit{Encyclopedia of library and information science}. Eds. J.\,D.~McDonald and 
M.~Levine-Clark. New York, NY: Marcel Dekker Inc. 56(19):137--174.

\bibitem{14-zac-1} %17
Gilyarevskiy, R.\,S., ed. 2006. \textit{Informatika kak nauka ob informatsii: 
informatsionnyy, dokumental'nyy, tekh\-no\-lo\-gi\-che\-skiy, ekonomicheskiy, sotsial'nyy 
i~organizatsionnyy aspekty} [Informatics as information science: Informational, 
documentary, technological, economic, social, and organizational dimensions]. 
Moscow: FAIR-PRESS. 592~p.

\bibitem{18-zac-1}
\Aue{Hjorland, B.} 2000. Library and information science: Practice, theory, and 
philosophical basis. \textit{Inform. Process. Manag.} 36(3):501--531. doi:  
10.1016/S0306-\mbox{4573(99)00038-2}.
\bibitem{19-zac-1}
Deep shift~--- technology tipping points and societal impact. 2015. \textit{World Economic 
Forum}. Geneva. 44~p. Available at: {\sf 
http://www3.weforum.org/docs/WEF\_ GAC15\_Technological\_Tipping\_Points\_report\_2015.pdf} (accessed May~20, 
2024).
\bibitem{20-zac-1}
\Aue{Berman, F., R.~Rutenbar, B.~Hailpern, H.~Christensen, S.~Davidson, 
D.~Estrin, M.~Franklin, M.~Martonosi, P.~Raghavan, V.~Stodden, and 
A.\,S.~Szalay.} 2018. Realizing the potential of data science. \textit{Commun. ACM} 
61(4):67--72. doi: 10.1145/3188721.
\bibitem{21-zac-1}
\Aue{Stodden, V.} 2020. The data science life cycle: A~disciplined approach to 
advancing data science as a~science. \textit{Commun. ACM} 
 63(7):58--66. doi: 10.1145/3360646.

\bibitem{23-zac-1} %22
\Aue{Zatsman, I.\,M.} 2023. Nauchnaya paradigma informatiki: klassifikatsiya 
transformatsiy ob''ektov predmetnoy oblasti [Scientific paradigm of informatics: 
Transformation classification of domain objects]. \textit{Sistemy i~Sredstva 
Informatiki~--- Systems and Means of Informatics} 33(4):126--138. doi: 
10.14357/08696527230412. EDN: ZIKUWO.

\bibitem{22-zac-1} %23
\Aue{Zatsman, I.\,M.} 2023. Nauchnaya paradigma informatiki: klassifikatsiya 
ob''ektov predmetnoy oblasti [Scientific paradigm of informatics: Classification of 
domain objects]. \textit{Informatika i~ee Primeneniya~--- Inform. Appl.} 
 17(4):96--103. doi: 10.14357/19922264230413. EDN: FIUQAT.
 
\bibitem{24-zac-1}
\Aue{   Zatsman, I.\,M.} 2022. O nauchnoy paradigme informatiki: verkhniy uroven' 
klassifikatsii ob''ektov ee predmetnoy oblasti [On the scientific paradigm of 
informatics: The classification high level of its objects]. \textit{Informatika i~ee 
Primeneniya~--- Inform. Appl.} 16(4):73--79. doi: 10.14357/19922264220411. EDN: 
XZNKVI.
\bibitem{25-zac-1}
\Aue{Solomonick, A.\,B.} 2011. \textit{Filosofiya znakovykh system i~yazyk} 
[Philosophy of sign systems and language]. Moscow: LKI. 408~p.
\bibitem{26-zac-1}
\Aue{Zatsman, I.\,M.} 2023. Transformatsiya ierarkhii Akoffa v~nauchnoy 
paradigme informatiki [Transformation of the Ackoff's hierarchy in the scientific 
paradigm of informatics]. \textit{Informatika i~ee Primeneniya~--- Inform. \mbox{Appl.}} 
17(3):107--113. doi: 10.14357/19922264230315. EDN: UMVRRV.
\bibitem{27-zac-1}
\Aue{Zatsman, I.} 2024. Building digital spiral models of knowledge 
generation. \textit{19th Forum (International) on Knowledge Asset Dynamics 
Proceedings}. Matera, Italy: Arts for Business Institute. 2185--2196.
\bibitem{28-zac-1}
\Aue{Zatsman, I.} 2023. Digital spiral model of knowledge creation and encoding its 
dynamics. \textit{18th Forum (International) on Knowledge Asset Dynamics 
Proceedings}. Matera, Italy: Arts for Business Institute. 581--596.
\bibitem{29-zac-1}
\Aue{Zatsman, I.\,M.} 2019. Interfeysy tret'ego poryadka v~informatike 
 [Third-order interfaces in informatics]. \textit{Informatika i~ee Primeneniya~--- 
Inform. Appl.} 13(3):82--89. doi: 10.14357/19922264190312. EDN: EHRQLF.
\bibitem{30-zac-1}
\Aue{Zatsman, I.} 2023. Scientific paradigm of informatics as a~third culture. 
\textit{Scientific Technical Information Processing} 50(4):246--258. doi: 
10.3103/S0147688223040111. EDN: CKHMYS.

\end{thebibliography}

 }
 }

\end{multicols}

\vspace*{-6pt}

\hfill{\small\textit{Received April 14, 2024}} 


\vspace*{-12pt}


\Contrl

\vspace*{-3pt}

\noindent
\textbf{Zatsman Igor M.} (b.\ 1952)~--- Doctor of Science in technology, head of 
department, Federal Research Center ``Computer Science and Control'' of the 
Russian Academy of Sciences, 44-2~Vavilov Str., Moscow 119333, Russian 
Federation; \mbox{izatsman@yandex.ru}





\label{end\stat}

\renewcommand{\bibname}{\protect\rm Литература}    %8
\def\stat{grusho}

\def\tit{АРХИТЕКТУРНЫЕ РЕШЕНИЯ В~ЗАДАЧЕ ВЫЯВЛЕНИЯ МОШЕННИЧЕСТВА ПРИ~АНАЛИЗЕ 
ИНФОРМАЦИОННЫХ ПОТОКОВ В~ЦИФРОВОЙ ЭКОНОМИКЕ$^*$}

\def\titkol{Архитектурные решения в~задаче выявления мошенничества при~анализе 
информационных потоков в
%~цифровой 
экономике}

\def\aut{А.\,А.~Грушо$^1$, М.\,И.~Забежайло$^2$, Н.\,А.~Грушо$^3$, 
Е.\,Е.~Тимонина$^4$}

\def\autkol{А.\,А.~Грушо, М.\,И.~Забежайло, Н.\,А.~Грушо, 
Е.\,Е.~Тимонина}

\titel{\tit}{\aut}{\autkol}{\titkol}

\index{Грушо А.\,А.}
\index{Забежайло М.\,И.}
\index{Грушо Н.\,А.}
\index{Тимонина Е.\,Е.}
\index{Grusho A.\,A.}
\index{Zabezhailo M.\,I.}
\index{Grusho N.\,A.}
\index{Timonina E.\,E.}


{\renewcommand{\thefootnote}{\fnsymbol{footnote}} \footnotetext[1]
{Работа частично поддержана РФФИ (проекты 18-29-03081 и~18-07-00274).}}


\renewcommand{\thefootnote}{\arabic{footnote}}
\footnotetext[1]{Институт проблем информатики Федерального исследовательского центра <<Информатика и~управление>> 
Российской академии наук, grusho@yandex.ru}
\footnotetext[2]{Институт проблем информатики Федерального исследовательского центра <<Информатика и~управление>> 
Российской академии наук, m.zabezhailo@yandex.ru}
\footnotetext[3]{Институт проблем информатики Федерального исследовательского центра <<Информатика и~управление>> 
Российской академии наук, info@itake.ru}
\footnotetext[4]{Институт проблем информатики Федерального исследовательского центра <<Информатика и~управление>> 
Российской академии наук, eltimon@yandex.ru}

\vspace*{-12pt}
   

 
  
  \Abst{Cформулирован подход к~исследованию некоторых видов мошенничества в~цифровой 
экономике с~использованием причинно-следственных связей. Во всех видах рассматриваемых 
мошенничеств должно наблюдаться несоответствие между целями финансовых транзакций 
и~реальной стоимостью достижения этих целей. Данные о транзакциях можно собирать, 
наблюдая информационные потоки, в~которых отражаются эти транзакции. Архитектура сбора 
данных и~их анализа может быть организована с~помощью распределенных реестров 
с~централизованным консенсусом, что позволяет создать аналог электронной бухгалтерской 
книги, фиксирующей финансово-экономическую деятельность субъектов цифровой экономики в~регионе. 
  Рассматриваемые методы выявления мошенничества основаны на противоречиях 
между действиями, описанными в~транзакциях, и~информацией, содержащейся в~планах, 
стандартах, прецедентах и~др. Рассмотрен метод, основанный на некоторой упрощенной схеме 
реализации абстрактного проекта. Для выявления противоречий необходимо проводить анализ 
от следствия к~причине, т.\,е.\ искать аномалии в~информации, описывающей порождение 
наблюдаемых следствий. 
  Показано, как в~реализации проекта можно выделять простые <<необходимые условия>> 
нарушения при\-чин\-но-след\-ст\-вен\-ных связей, т.\,е.\ множество <<необходимых условий>>, 
нарушение которых свидетельствует о наличии мошенничества. Это множество <<необходимых 
условий>> можно назвать метаданными для контроля проекта на выявление мошенничества.} 
 
 
  \KW{цифровая экономика; информационные потоки; при\-чин\-но-след\-ст\-вен\-ные связи; 
выявление мошеннических схем} 

\DOI{10.14357/19922264190204}
  
\vspace*{-4pt}


\vskip 10pt plus 9pt minus 6pt

\thispagestyle{headings}

\begin{multicols}{2}

\label{st\stat}

\section{Введение}

\vspace*{3pt}

  В работе сформулирован подход к~исследованию некоторых видов 
мошенничества в~цифровой экономике с~использованием  
при\-чин\-но-след\-ст\-вен\-ных связей. Рассматриваются три вида мошенничества, 
а именно:
  \begin{enumerate}[(1)]
\item отмыв денег; 
\item обман при выполнении договорных обязательств при реализации 
технических проектов (строительные проекты и~др.); 
\item незаконный вывод денег. 
\end{enumerate}

  Названные виды мошенничества могут быть сведены к~решению одного типа 
задач. Для отмывания денег источник должен заключать фиктивные контракты, 
в~соответствии с~которыми будут переводиться средства за заведомо ненужную 
работу и~материалы. 
  
  Мошенничество, связанное с~невыполнением договорных обязательств, связано 
со снижением качества услуг, качества и~количества закупаемых 
материалов, выполнением работ с~ненадлежащим качеством. 
  
  Вывод денег связан с~переводом средств фир\-мам-од\-но\-днев\-кам, которые 
заведомо не могут выполнить обязательства по контрактам, за которые им 
переводятся средства. 
  
  Таким образом, во всех трех видах рассматриваемых мошенничеств должно 
наблюдаться несоответствие между целями финансовых транзакций и~реальной 
стоимостью достижения этих целей. Данные о транзакциях можно собирать, 
наблюдая информационные потоки, в~которых отражаются эти транзакции. 
  
  Однако для наблюдения таких информационных потоков необходимо создавать 
архитектуру\linebreak телекоммуникационной системы, позволяющей перехватывать 
и~собирать данные о всех транзакциях. Например, такая архитектура может быть 
организована с~помощью распределенных реестров с~централизованным 
консенсусом, т.\,е.\ все информационные потоки, сформированные в~цифровой 
экономике и~несущие информацию о транзакциях, проходят через некоторый 
центральный узел, запоминающий их в~форме распределенного реестра. Такие 
реестры могут дублироваться в~аналогичных центрах различных регионов, что 
позволяет создать аналог электронной бухгалтерской книги, фиксирующей 
фи\-нан\-со\-во-эко\-но\-ми\-че\-скую деятельность субъектов цифровой экономики. Такой 
подход предложено реализовать на базе системы ситуационных центров, что 
отражено в~работах~[1, 2].
  
  Собранная из информационных потоков информация о~транзакциях, т.\,е.\ 
о~контрактах, договорах, платежах, отчетах, закупленных материалах, 
характеристиках исполнителей работ и~др., собирается в~базе данных в~указанном 
центре. Согласно теории интеллектуальных сис\-тем~[3], эту базу данных можно 
называть базой фактов (БФ). Базу фактов можно представить как бинарную мат\-ри\-цу, 
строки которой описывают характеристики, входящие в~транзакции, а столбцы 
нумеруются характеристиками. Строки матрицы будем называть 
\textit{объектами}~[4, 5]. 
  
  Рассматриваемые в~работе методы выявления мошенничества будут основаны 
на противоречиях между действиями, описанными в~транзакциях, и~информацией, 
содержащейся в~планах, стандартах, прецедентах и~др. Для нахождения 
противоречий в~архитектуре центра предусмотрена другая база данных~--- база 
знаний (БЗ)~\cite{3-gr, 6-gr}, которая устроена так же, как БФ. 
  
  Информация в~БЗ собирается на основе положительного опыта или расчетов. 
Используя БЗ, можно выводить факты нарушения при\-чин\-но-след\-ст\-вен\-ных 
связей. Нарушения при\-чин\-но-след\-ст\-вен\-ных связей будем называть 
\textit{аномалиями}. 
  
  Для упрощения дальнейшее изложение будет вестись в~рамках поиска 
противоречий при выполнении некоторого абстрактного проекта. Выявление 
аномалий будет происходить на основе фактов из БФ с~помощью знаний из БЗ 
методами искусственного интеллекта и~интеллектуального анализа 
данных~\cite{6-gr}. 

\vspace*{-10pt}
  
  \section{Модели}
  
  \vspace*{-3pt}
  
  Наиболее сложная из рассмотренных выше задач~--- выявление противоречий, 
т.\,е.\ использование БЗ для получения новых знаний и~выявление аномалий из 
полученных фактов. 
  
  Все способы выявления противоречий основаны на определении 
  причинно-следственных связей. При этом противоречия в~параметрах транзакций по 
отношению к~требуемым в~БЗ составляют сущность аномалий. 
  
   Далее будет рассмотрен метод, основанный на некоторой упрощенной схеме 
реализации абстрактного проекта. 
  
  Каждый проект имеет цель: например, цель представляет собой построение 
некоторой системы. Воспользуемся структурным подходом, который позволяет 
строить проект на основе разбиения системы на подсистемы и~определения 
взаимодействий подсистем~\cite{7-gr}. При этом каждая подсистема также 
представима структурной моделью. 
  
  Как сама система, так и~каждая ее подсистема имеют свой функционал 
и~спецификацию, па\-ра\-мет\-ры настройки и~домены параметров настройки. Кроме 
этих характеристик существует множество характеристик, связанных 
с~<<жизненным циклом>> создания системы. Сюда входят работы, ресурсы, 
сроки выполнения работ по созданию подсистем и~самой системы, стоимости 
компонентов и~материалов, стоимости работ, схемы поставок, договорные 
обязательства и~др. Все характеристики связаны между собой, поэтому можно 
говорить о стоимости и~времени изготовления структурных компонентов системы. 
  
  Одной из важнейших характеристик является смета (система смет для 
подсистем). Смета сопоставляет каждому компоненту системы стоимость его 
изготовления и~настройки. 
  
  Схема построения системы может быть пред\-став\-ле\-на диаграммой, 
изображенной на рис.~1. 

{ \begin{center}  %fig1
 \vspace*{9pt}
   \mbox{%
 \epsfxsize=79mm 
 \epsfbox{gru-1.eps}
 }


\vspace*{9pt}


\noindent
{{\figurename~1}\ \ \small{Диаграмма достижения цели}}
\end{center}
}

\vspace*{9pt}

\addtocounter{figure}{1}
  
  


  Представленная на рис.~1 диаграмма позволяет описать основные классы 
возможных противоречий при достижении цели. Противоречия возникают, когда 
данные БФ не соответствуют требуемым характеристикам. 
  
  
  \section{Потенциальные классы аномалий при~достижении цели}
  
  Выделим четыре потенциальных класса противоречий, которые показывают, 
каким образом нужно искать эти противоречия.
  
 
  Противоречие цели и~проекта (рис.~2) возникает при отсутствии обоснования 
или в~случае логического противоречия между возможностями проектируемого 
функционала и~целью системы. Отметим, что в~проект входят сроки, перечень 
работ, материалы, настройки, которые описываются соответствующими 
параметрами и~допустимыми значениями этих параметров. Проект формируется 
на основе БЗ и~расчетов, исходя из информации, полученной по аналогии 
с~другими проектами и~решениями, которые считаются апробированными. 
  
  Отметим, что цель порождает проект и~в этом смысле является причиной 
проекта. Однако для анализа противоречий необходимо двигаться по штриховой 
стрелке диаграммы (см.\ рис.~2) от проекта к~цели. В~самом деле, любой компонент 
проекта направлен на теоретическое достижение цели. Цель~--- сложный объект, 
поэтому в~проекте могут возникнуть характеристики, противоречащие хотя бы 
некоторым характеристикам цели. Это делает проект противоречивым, но вывод 
об этом может быть сделан только на уровне описания цели. 
  

  Противоречия между проектом и~его реализацией, исключая настройки 
(рис.~3), могут возникать, например, при закупке исполнителем материалов более 
низкого качества по более низким ценам, при попытках достижения требуемых 
сроков работы за счет снижения качества выполнения работ, за счет нахождения 
<<объективных>> причин для увеличения сроков работы и,~следовательно, 
увеличения цены реализации проекта. 


  Для выявления указанных противоречий необходимо двигаться по диаграмме 
(см.\ рис.~3) в~обратную сторону в~соответствии со~штриховыми стрелками. 
Действительно, выявить противоречия между характеристиками закупленных 
материалов и~требуемыми по проекту можно только при обращении к~проекту 
и~его спецификациям. Манипуляции со сроками работы также можно выявить 
только при обращении к~соответствующим расчетам в~проекте. Задержки в~сроках 
работы, связанные с~поставками материалов, можно определить только на 
предыдущем этапе диаграммы (см.\ рис.~3) в~описании проекта. 


  


  Противоречия между реализацией проекта и~его настройкой (рис.~4) возникает, 
когда не удается добиться требуемых значений параметров функционала, не 
удается обеспечить необходимый уровень\linebreak\vspace*{-12pt}

{ \begin{center}  %fig2
 \vspace*{-6pt}
   \mbox{%
 \epsfxsize=16mm 
 \epsfbox{gru-2.eps}
 }


\vspace*{6pt}


\noindent
{{\figurename~2}\ \ \small{Противоречия цели и~проекта}}
\end{center}
}

%\vspace*{9pt}

\addtocounter{figure}{1}

{ \begin{center}  %fig3
 \vspace*{6pt}
    \mbox{%
 \epsfxsize=79mm 
 \epsfbox{gru-3.eps}
 }


\end{center}

\vspace*{-2pt}


\noindent
{{\figurename~3}\ \ \small{Противоречия проекта и~его реализации (без настройки)}}
}

\vspace*{6pt}

\addtocounter{figure}{1}

{ \begin{center}  %fig4
 \vspace*{1pt}
   \mbox{%
 \epsfxsize=54.5mm 
 \epsfbox{gru-4.eps}
 }


\end{center}


\noindent
{{\figurename~4}\ \ \small{Противоречия реализации проекта и~его на\-стройки}}
}

%\vspace*{9pt}

\addtocounter{figure}{1}

{ \begin{center}  %fig5
 \vspace*{5pt}
    \mbox{%
 \epsfxsize=79mm 
 \epsfbox{gru-5.eps}
 }


\end{center}



\noindent
{{\figurename~5}\ \ \small{Противоречия цели и~достигнутой реализации проекта}}
}

\vspace*{6pt}

\addtocounter{figure}{1}

\noindent
 качества реализации проекта. Для 
определения противоречия в~настройках надо опять же двигаться по диаграмме 
(см.\ рис.~4) в~обратную сторону по штриховым стрелкам, так как для выявления 
характеристик результатов работы, которые не дают возможности реализации 
определенного функционала, необходимо иметь информацию о результатах этой 
работы. 


  



  Противоречие между целью и~достигнутой реализацией проекта (рис.~5) 
возникает, когда реализованная система не позволяет достичь цели. В~этом случае 
опять противоречие нужно искать, двигаясь от цели к~реальному достигнутому 
функционалу по штриховой стрелке (см.\ рис.~5).
  
  Суммируя положения, изложенные в~данном разделе, приходим к~выводу, что 
для выявления противоречий необходимо проводить анализ от следствия 
к~причине, т.\,е.\ искать аномалии в~информации, описывающей порождение 
наблюдаемых следствий. 
  
  
  \section{Связь противоречий и~причин}
  
  Прежде чем построить связь между причинами и~противоречиями, кратко 
опишем простейшую модель связи этих понятий. Причины и~противоречия будут 
сформулированы для представления компонентов системы как объектов, 
обладающих наборами известных характеристик~\cite{4-gr, 5-gr}. 
  
  Пусть $U\hm=\{\alpha, \beta, \ldots\}$~--- совокупность характеристик 
(пространство характеристик). Согласно~\cite{4-gr} \textit{объектом}~$O$ 
называется любое подмножество характеристик $O\hm\subseteq U$. Рассмотрим 
последовательность объектов, возможно в~различных пространствах 
характеристик. 
  
  \smallskip
  
  \noindent
  \textbf{Определение~1.}\ Объект~$P$ с~числом характеристик, большим или 
равным~2, является \textit{причиной} объекта (\textit{свойства})~$B$ в~цепочке 
наблюдаемых объектов тогда и~только тогда, когда выполнены следующие 
условия:
  \begin{enumerate}[(1)]
\item для каждого объекта~$C$, если $P\hm\subseteq C$, то $C\mapsto B$, где 
$C\mapsto B$ означает, что объект~$B$ присутствует в~объекте, следующем за 
объектом~$C$;
\item объект~$P$ является минимальным объектом, удовлетворяющим 
условию~1, а~именно: $\forall \alpha\hm\in P$ объект~$P\backslash \{\alpha\}$ 
не является причиной, т.\,е.\ $\exists C:\ \alpha\not\in C$, $P\backslash 
\{\alpha\}\hm\subseteq C$ и~$C\not\mapsto B$, где $C\not\mapsto B$ означает, 
что~$B$ не может содержаться в~объекте, следующем за объектом~$C$. 
\end{enumerate}

  Приведенное определение причины является упрощением причин, 
возникающих в~реальном мире. Например, реальные причины могут возникать\linebreak 
как совокупность характеристик из разных пространств. Одно следствие может 
порождаться разными причинами или возникать из внешних\linebreak и~ненаблюдаемых 
характеристик. Однако пред\-став\-лен\-ная далее формализация позволяет доступно 
изложить при\-чин\-но-след\-ст\-вен\-ные истоки противоречий, которые 
инициируют в~дальнейшем глубокое исследование рассматриваемых процессов.
  
  Будем считать, что для любого интересующего нас свойства~$B$ существует 
причина. Тогда справедлива следующая теорема.
  
  \smallskip
  
  \noindent
  \textbf{Теорема~1.}\ \textit{Для любого свойства~$B$ существует 
единственная причина}. 
  
  \smallskip
  
  \noindent
  Д\,о\,к\,а\,з\,а\,т\,е\,л\,ь\,с\,т\,в\,о\,.\ \ Доказательство будем вести от противного, 
т.\,е.\ предположим, что существуют две причины свойства~$B$: $P$ 
и~$P^\prime$, $P\hm\not= P^\prime$. Тогда существует $\alpha\hm\in U$, которое 
удовлетворяет одному из двух условий:
  \begin{itemize}
\item[(а)] $\alpha\in P$, $\alpha\notin P^\prime$;
\item[(б)] $\alpha\notin P$, $\alpha \in P^\prime$.
\end{itemize}

  Пусть выполняется условие~(б). Тогда $P^\prime\backslash \{\alpha\}$ не 
является причиной по условию~2 определения~1, т.\,е.\ $\exists C$ такое, что 
$\alpha\notin C$, $P^\prime\backslash \{\alpha\}\hm\subseteq C$ и~$C\not\mapsto B$. 
Но если~$B$ произошло и~$P$ его причина, то $C\mapsto B$, что противоречит 
предположению. Теорема~1 доказана.
  
  \smallskip
  
  \noindent
  \textbf{Лемма.} \textit{Если $P$~--- причина появления свойства~$B$, то 
объект~$B$ определяет существование свойства~$P$ в~объекте, 
предшествующем~$B$. }
  
  \smallskip
  
  \noindent
  Д\,о\,к\,а\,з\,а\,т\,е\,л\,ь\,с\,т\,в\,о\,.\ \ Из предположения, что у~каж\-до\-го 
свойства~$B$ есть причина, и~условия, что~$P$ является причиной~$B$, следует, 
что при появлении в~данных свойства~$B$ объект~$C$, предшествующий 
появлению~$B$, содержит как часть объект~$P$. Это следует из теоремы~1 
и~определения причины. 
  
  Докажем принцип <<необходимого условия>>, который, несмотря на простоту 
доказательства, будет играть в~дальнейшем существенную роль.
  
  \smallskip
  
  \noindent
  \textbf{Теорема~2.} \textit{Если~$P$~--- причина появления свойства~$B$ 
и~$A\hm\subseteq P$, то объект~$B$ определяет наличие свойства~$A$ 
в~объекте, предшествующем~$B$}. 
  
  \smallskip
  
  \noindent
  Д\,о\,к\,а\,з\,а\,т\,е\,л\,ь\,с\,т\,в\,о\,.\ \ Пусть в~данных имеется объект~$B$ 
и~$P\mapsto B$, тогда в~силу существования и~единственности причины~$B$ 
в~данных должен существовать объект~$C$, предшествующий~$B$ 
и~содержащий причину~$P$. Поскольку $A\hm\subseteq P$ и~$B$ содержит 
причину~$P$, то $B\mapsto A$. С~учетом леммы теорема~2 доказана.
  
  \smallskip
  
  Пусть даны пространства $U_1, U_2,\ldots$ и~имеется последовательность 
данных (процесс выполнения этапов проекта в~соответствии с~рис.~1) $A, B, 
\ldots$, где каждый объект является подмножеством некоторого 
пространства~$U_i$, $i\hm=1,\ldots$ Тогда в~объекте~$A$ присутствует 
причина~$P$ появления интересующего нас свойства~$C$ в~объекте~$B$. Пусть 
$P\hm\subseteq A$, тогда по теореме~2 $\forall \alpha\hm\in P$:  
$C\mapsto \{\alpha\}$, т.\,е.\ из появления~$C$ следует появление 
характеристики~$\alpha$ в~предшествующем объекте. Это необходимое условие 
того, что~$C$ удовлетворяет причинно-следственным связям развития процесса 
выполнения проекта. Если для~$C$ нет характеристики~$\alpha$, которую можно 
отнести к~причине~$C$, то можно считать, что нарушена  
при\-чин\-но-след\-ст\-вен\-ная связь и~$C$~--- аномальный объект. 
  
  \smallskip
  
  \noindent
  \textbf{Пример.} Если объект~$C$ состоит в~получении суммы~$a$ 
фирмой~$K$, то согласно теореме~2 в~пред\-шест\-ву\-ющем объекте должна 
существовать причина перевода суммы~$a$ на фирму~$K$. Если эта причина 
в~проекте отсутствует, то это можно считать признаком мошеннической схемы. 
Все проекты по предположению собираются из <<кубиков>>, содержащихся в~БЗ. 
Тогда можно сравнить цену объекта~$C$, породившего получение суммы~$a$, 
и~сумму, присутствующую в~смете проекта. Если разница велика, то это либо 
ошибка проекта, либо признак мошеннической схемы.
  
  \section{Поиск противоречий на~основе~принципа <<необходимых~условий>>}
   
  Как было показано в~разд.~3, нахождение противоречий соответствуют 
движению от следствия к~причине. Для каждого объекта в~наблюдаемых данных 
выявление причин его появления является трудоемкой задачей. Кроме того, при 
реализации контроля соблюдения при\-чин\-но-след\-ст\-вен\-ных связей на 
большом множестве участников экономической деятельности задача анализа 
причин становится трудоемкой. Поэтому процедуру контроля необходимо разбить 
на два этапа, где первый этап состоит в~анализе простых <<необходимых 
условий>> проявления мошенничества, когда используется хотя бы одна 
известная характеристика причины. Второй этап (в~режиме офлайн) состоит 
в~выявлении причин, позволяющих провести анализ источников мошеннических 
схем. 
  
  Один из подходов к~выбору <<необходимых условий>> состоит в~построении 
множества подцелей исходной цели проекта (структурный метод построения 
проекта~\cite{7-gr}). Каждая подцель описывается диаграммой на рис.~1, 
и~реализации подцелей должны образовывать полный функционал цели. Это 
является необходимым, но не достаточным условием достижения цели, так как 
при таком подходе отсутствует компонент согласования всех подцелей в~единую 
систему. Однако такой подход значительно упрощает анализ выполнения проекта 
на предмет поиска мошенничества. Если признаки мошенничества будут 
обнаружены в~реализации хотя бы одной из подцелей, то это значит, что 
мошенничество присутствует в~реализации всего проекта. 
  
  Аналогично в~реализации каждого этапа в~любой из подцелей можно выделять 
простые <<необходимые условия>> нарушения при\-чин\-но-след\-ст\-венн\-ых 
связей. 
  
  Таким образом, получается множество <<необходимых условий>>, нарушение 
которых свидетельствует о наличии мошенничества. Это множество 
<<необходимых условий>> можно назвать метаданными~[8, 9] для контроля 
проекта на выявление мошенничества. 
  
  
  \section{Заключение }
  
  В поиске противоречий необходимо от транзакций, соответствующих 
следствиям при\-чин\-но-след\-ст\-вен\-ных связей, переходить к~анализу причин 
наблюдаемых следствий. Это сложная задача, которая связана с~описанием причин 
определенных свойств. 
  
  В работе представлена модель, позволяющая строить множество необходимых 
условий соответствия наблюдаемого следствия вызвавшей его причине. Этот 
подход делает поиск противоречий вполне вычислимой задачей, но не гарантирует 
успех. 
  
  {\small\frenchspacing
 {%\baselineskip=10.8pt
 \addcontentsline{toc}{section}{References}
 \begin{thebibliography}{9}
\bibitem{1-gr}
\Au{Грушо А.\,А., Зацаринный~А.\,А., Тимонина~Е.\,Е.} Блокчейны цифровой экономики на базе 
системы ситуационных центров и~централизованного консенсуса~// Радиолокация, навигация, 
связь: Мат-лы XXV Междунар. научн.-технич. конф.~---
Воронеж: Издательский дом ВГУ, 2019. Т.~6. С.~183--191. 
\bibitem{2-gr}
\Au{Grusho A., Zatsarinny~A., Timonina~E.} A~system approach to information security in 
distributed ledgers on the situational centers platform.~---
Lecture notes in computer science ser.~--- Springer, 2019 
(in press).
\bibitem{3-gr}
\Au{Финн В.\,К.} Искусственный интеллект: Методология, применения, философия.~--- М.: 
Красанд, 2011. 448~с.

\bibitem{5-gr} %4
\Au{Аншаков~О.\,М., Фабрикантова~Е.\,Ф.} ДСМ-ме\-тод автоматического порождения 
гипотез: Логические и~эпистемологические основания.~--- М.: Либроком, 2009. 432~с.

\bibitem{4-gr} %5
\Au{Poelmans J., Elzinga~P., Viaene~S., Dedene~G.} Formal concept analysis in knowledge 
discovery: A~survey~// Conceptual structures: From information to intelligence~/ Eds.\ M.~Croitoru, 
S.~Ferr$\acute{\mbox{e}}$, and D.~Lukose.~--- Lecture notes in computer science 
ser.~--- Berlin--Heidelberg: Springer, 2010. Vol.~6208.  P.~139--153.

\bibitem{6-gr}
\Au{Панкратова~Е.\,С., Финн~В.\,К.} Автоматическое по\-рож\-де\-ние гипотез в~интеллектуальных 
системах.~--- М.: Либроком, 2009. 528~с. 
\bibitem{7-gr}
\Au{Денисов А.\,А., Колесников~Д.\,Н.} Теория больших систем управления.~--- Л.: Энергоиздат, 1982. 488~с.

\bibitem{9-gr}
\Au{Грушо А.\,А., Грушо Н.\,А., Забежайло~М.\,И., Смирнов~Д.\,В., Тимонина~Е.\,Е.} 
Параметризация в~прикладных задачах поиска эмпирических причин~// Информатика и~её 
применения, 2018. Т.~12. Вып.~3. С.~62--66.

\bibitem{8-gr}
\Au{Грушо А.\,А., Грушо Н.\,А., Левыкин~М.\,В., Тимонина~Е.\,Е.} Методы идентификации 
захвата хоста в~распределенной ин\-фор\-ма\-ци\-он\-но-вы\-чис\-ли\-тель\-ной сис\-те\-ме, 
защищенной с~помощью метаданных~// Информатика и~её применения, 2018. Т.~12. Вып.~4. 
С.~41--45.

 \end{thebibliography}

 }
 }

\end{multicols}

\vspace*{-3pt}

\hfill{\small\textit{Поступила в~редакцию 03.04.19}}

%\vspace*{8pt}

%\pagebreak

\newpage

\vspace*{-28pt}

%\hrule

%\vspace*{2pt}

%\hrule

%\vspace*{-2pt}

\def\tit{ARCHITECTURAL DECISIONS IN~THE~PROBLEM 
OF~IDENTIFICATION OF~FRAUD IN~THE~ANALYSIS 
OF~INFORMATION FLOWS IN~DIGITAL ECONOMY\\[-5pt]}


\def\titkol{Architectural decisions in~the~problem 
of~identification of~fraud in~the~analysis 
of~information flows in~digital economy}

\def\aut{A.\,A.~Grusho, M.\,I.~Zabezhailo, N.\,A.~Grusho, and~E.\,E.~Timonina}

\def\autkol{A.\,A.~Grusho, M.\,I.~Zabezhailo, N.\,A.~Grusho, and~E.\,E.~Timonina}

\titel{\tit}{\aut}{\autkol}{\titkol}

\vspace*{-13pt}


 \noindent
   Institute of Informatics Problems, Federal Research Center ``Computer Sciences and 
Control'' of the Russian Academy of Sciences; 44-2~Vavilov Str., Moscow 119133, 
Russian Federation

\def\leftfootline{\small{\textbf{\thepage}
\hfill INFORMATIKA I EE PRIMENENIYA~--- INFORMATICS AND
APPLICATIONS\ \ \ 2019\ \ \ volume~13\ \ \ issue\ 2}
}%
 \def\rightfootline{\small{INFORMATIKA I EE PRIMENENIYA~---
INFORMATICS AND APPLICATIONS\ \ \ 2019\ \ \ volume~13\ \ \ issue\ 2
\hfill \textbf{\thepage}}}

\vspace*{3pt}


   
     
   \Abste{An approach to a~research of some types of fraud in digital economy with the usage of relationships of 
cause and effect is formulated. In all types of the considered frauds, the discrepancy between the 
purposes of financial transactions and actual cost of achievement of these purposes
has to be observed. Data on 
transactions can be collected by observing information flows in which these transactions are reflected. 
The architecture of data collection and their analysis can be organized by means of the distributed 
ledgers with the centralized consensus that allows creating an analog of the electronic account book 
fixing financial and economic activity of subjects of digital economy in the region. 
   The methods of fraud identification considered are based on the contradictions 
between actions described in transactions and information, which is contained in plans, standards, 
precedents, etc. 
   The method based on a~simplified scheme of implementation of the abstract project is considered. 
For identification of contradictions, it is necessary to carry out the analysis from the effect to the cause, 
i.\,e., to look for anomalies in information describing the generation of the observed effects. 
   It is shown how in implementation of the project it is possible to allocate simple ``necessary 
conditions'' of violation of cause and effect relationships, i.\,e., a~set of ``necessary conditions'' 
violation of which demonstrates fraud existence. It is possible to call this set of "necessary conditions" 
by metadata for control of the project for fraud identification.} 
   
   \KWE{digital economy; information flows; relationships of reason and effect; detection of 
fraudulent schemes}
   
  

 \DOI{10.14357/19922264190204}

\vspace*{-20pt}

 \Ack
   \noindent
   The work was partially supported by the Russian Foundation for Basic Research (projects  
18-29-03081 and 18-07-00274).



%\vspace*{6pt}

  \begin{multicols}{2}

\renewcommand{\bibname}{\protect\rmfamily References}
%\renewcommand{\bibname}{\large\protect\rm References}

{\small\frenchspacing
 {\baselineskip=10.5pt
 \addcontentsline{toc}{section}{References}
 \begin{thebibliography}{9}
\bibitem{1-gr-1}
\Aue{Grusho, A.\,A., A.\,A.~Zatsarinny, and E.\,E.~Timonina.} 2019. Blokcheyny tsifrovoy ekonomiki 
na baze sistemy situatsionnykh tsentrov i~tsentralizovannogo konsensusa [Blockchains of digital 
economy on the basis of the system of the situational centres and the centralized consensus]. 
\textit{25th Scientific and Technical Conference (International) ``Radar-Location, Navigation, 
Communication'' Proceedings}. Voronezh: VSU Publs. 6:183--191.
\bibitem{2-gr-1}
\Aue{Grusho, A., A.~Zatsarinny, and E.~Timonina.} 2019 (in press). 
A~system approach to information security 
in distributed ledgers on the situational centers platform. 
Lecture notes in computer science ser. Springer.
\bibitem{3-gr-1}
\Aue{Finn, V.\,K.} 2011. \textit{Iskusstvennyy intellekt: Metodologiya, primeneniya, filosofiya} 
[Artificial intelligence: Methodology, applications, philosophy]. Moscow: KRASAND. 448~p.

\bibitem{5-gr-1}
\Aue{Anshakov, O.\,M., and E.\,F.~Fabrikantova}. 2009. \textit{DSM-metod avtomaticheskogo porozhdeniya gipotez: Logicheskie 
i~epistemologicheskie osnovaniya} [JSM-method of automatic hypothesis generation: Logical and 
epistemological]. Moscow: KD LIBROKOM. 432~p.
\bibitem{4-gr-1} %5
\Aue{Poelmans, J., P.~Elzinga, S.~Viaene, and G.~Dedene.} 2010. Formal concept analysis in 
knowledge discovery: A~survey. \textit{Conceptual structures: From information to intelligence}. 
Eds.\ M.~Croitoru, S.~Ferr$\acute{\mbox{e}}$, and D.~Lukose. Lecture notes in 
computer science ser. Berlin--Heidelberg: Springer. 6208:139--153.

\bibitem{6-gr-1}
\Aue{Pankratov, E.\,S., and V.\,K.~Finn}. 
2009. \textit{Avtomaticheskoe porozhdenie gipotez v~intellektual'nykh 
sistemakh} [Automatic hypotheses generation in intelligent systems]. Moscow: KD 
\mbox{LIBROKOM}.  528~p. 
\bibitem{7-gr-1}
\Aue{Denisov, A.\,A., and D.\,N.~Kolesnikov.} 1982. \textit{Teoriya bol'shikh 
sistem upravleniya} [Theory of big control systems]. Leningrad: Energoizdat. 488~p.

\bibitem{9-gr-1}
\Aue{Grusho, A.\,A., N.\,A.~Grusho, M.\,I.~Zabezhailo, D.\,V.~Smirnov, and 
E.\,E.~Timonina.} 2018. 
Parametrizatsiya v~prikladnykh zadachakh poiska empiricheskikh prichin 
[Parametrization in applied 
problems of search of the empirical reasons]. 
\textit{Informatika i~ee Primeneniya~--- 
Inform. Appl.} 12(3):62--66.

\bibitem{8-gr-1}
\Aue{Grusho, A.\,A., N.\,A.~Grusho, M.\,V.~Levykin, and E.\,E.~Timonina.} 2018. Metody 
identifikatsii zakhvata khosta v~raspredelennoy informatsionno-vychislitel'noy sisteme, 
zashchishchennoy s~pomoshch'yu metadannykh [Methods of identification of host capture 
in the  distributed information system which is protected on the base of meta data].
\textit{Informatika i~ee 
Primeneniya~--- Inform. Appl.} 12(4):41--45.
{ %\looseness=1

}

\end{thebibliography}

 }
 }

\end{multicols}

\vspace*{-12pt}

\hfill{\small\textit{Received April 3, 2019}}

%\pagebreak

%\vspace*{-18pt}

\Contr

\noindent
\textbf{Grusho Alexander A.} (b.\ 1946)~--- Doctor of Science in physics and 
mathematics, professor, principal scientist, Institute of Informatics Problems, 
Federal Research Center ``Computer Sciences and Control'' of the Russian 
Academy of Sciences; 44-2~Vavilov Str., Moscow 119133, Russian Federation; 
\mbox{grusho@yandex.ru} 

\vspace*{3pt}

\noindent
\textbf{Zabezhailo Michael I.} (b.\ 1956)~--- Doctor of Science in physics and 
mathematics, principal scientist, Institute of Informatics Problems, Federal Research 
Center ``Computer Sciences and Control'' of the Russian Academy of Sciences;  
44-2~Vavilov Str., Moscow 119133, Russian Federation; 
\mbox{m.zabezhailo@yandex.ru} 

\vspace*{3pt}


\noindent
\textbf{Grusho Nikolai A.} (b.\ 1982)~--- Candidate of Science (PhD) in physics 
and mathematics, senior scientist, Institute of Informatics Problems, Federal 
Research Center ``Computer Sciences and Control'' of the Russian Academy of 
Sciences; 44-2~Vavilov Str., Moscow 119133, Russian Federation; 
\mbox{info@itake.ru} 

\vspace*{3pt}


\noindent
\textbf{Timonina Elena E.} (b.\ 1952)~--- Doctor of Science in technology, 
professor, leading scientist, Institute of Informatics Problems, Federal Research 
Center ``Computer Sciences and Control'' of the Russian Academy of Sciences;  
44-2~Vavilov Str., Moscow 119133, Russian Federation; 
\mbox{eltimon@yandex.ru} 

\label{end\stat}

\renewcommand{\bibname}{\protect\rm Литература}      %9

\def\stat{shnurkov}

\def\tit{АНАЛИТИЧЕСКОЕ РЕШЕНИЕ ЗАДАЧИ ОПТИМАЛЬНОГО УПРАВЛЕНИЯ ПОЛУМАРКОВСКИМ ПРОЦЕССОМ\\ 
С~КОНЕЧНЫМ МНОЖЕСТВОМ СОСТОЯНИЙ$^*$}

\def\titkol{Аналитическое решение задачи оптимального управления полумарковским 
процессом} %с~конечным множеством состояний}

\def\aut{П.\,В.~Шнурков$^1$, А.\,К.~Горшенин$^2$, В.\,В.~Белоусов$^3$}

\def\autkol{П.\,В.~Шнурков, А.\,К.~Горшенин, В.\,В.~Белоусов}

\titel{\tit}{\aut}{\autkol}{\titkol}

\index{Шнурков П.\,В.}
\index{Горшенин А.\,К.}
\index{Белоусов В.\,В.}
\index{Shnurkov P.\,V.}
\index{Gorshenin A.\,K.}
\index{Belousov V.\,V.}


{\renewcommand{\thefootnote}{\fnsymbol{footnote}} \footnotetext[1]
{Работа выполнена при частичной поддержке РФФИ (проект 15-07-05316).}}


\renewcommand{\thefootnote}{\arabic{footnote}}
\footnotetext[1]{Национальный исследовательский университет <<Высшая школа экономики>>, 
\mbox{pshnurkov@hse.ru}}
\footnotetext[2]{Институт проблем информатики Федерального исследовательского центра <<Информатика 
и~управ\-ле\-ние>> Российской академии наук, \mbox{agorshenin@frccsc.ru}}
\footnotetext[3]{Институт проблем информатики Федерального исследовательского центра <<Информатика 
и~управление>> Российской академии наук, \mbox{vbelousov@ipiran.ru}}

%\vspace*{-6pt}

\Abst{Настоящее исследование посвящено теоретическому обоснованию нового метода 
нахождения оптимальной стратегии управления полумарковским процессом с~конечным 
множеством состояний. Рассматриваются марковские рандомизированные стратегии 
управления, определяемые конечным набором вероятностных мер, соответствующих 
каждому состоянию. Характеристикой качества управления служит стационарный 
стоимостной показатель. Данный показатель представляет собой дроб\-но-ли\-ней\-ный 
интегральный функционал от набора вероятностных мер, задающих стратегию управления. 
Для этого функционала известны явные аналитические представления подынтегральных 
функций числителя и~знаменателя. Дальнейшие результаты основываются на новой 
усиленной и~обобщенной форме теоремы об экстремуме дроб\-но-ли\-ней\-но\-го интегрального 
функционала. Доказывается, что проблемы существования оптимальной стратегии управления 
полумарковским процессом и~ее нахождения сводятся к~задаче численного исследования 
на глобальный экстремум заданной функции от конечного числа вещественных переменных.}

\KW{оптимальное управление полумарковским процессом; стационарный стоимостной 
показатель качества управления; дроб\-но-ли\-ней\-ный интегральный функционал}

\DOI{10.14357/19922264160408} 

\vspace*{9pt}


\vskip 10pt plus 9pt minus 6pt

\thispagestyle{headings}

\begin{multicols}{2}

\label{st\stat}

\section{Введение}

Теория оптимального управления марковскими и~полумарковскими случайными 
процессами интенсивно развивается с~начала 1960-х~гг. Еще в~первых 
основополагающих исследованиях рассматривались не только проблемы существования 
оптимальных стратегий управления, но и~способы нахождения этих стратегий. 

Для решения таких проблем, имеющих алгоритмическое содержание, использовались 
открытые незадолго до этого мощные методы прикладной математики: линейное 
программирование и~динамическое программирование. Отметим, прежде всего, 
классическую работу Р.~Ховарда~\cite{1}, в~которой метод динамического 
программирования был применен для решения проблемы оптимального управления 
марковским процессом с~непрерывным временем. В~дальнейшем В.\,В.~Рыков~\cite{2} 
доказал, что для аналогичной модели управления марковским процессом с~учетом 
переоценки оптимальной стратегией также является стационарная.

Важную роль в~развитии теории управления случайными процессами сыграла работа 
В.~Джевелла~\cite{3}, в~которой были впервые рассмотрены полумарковские модели 
управления для вариантов с~переоценкой и~без переоценки. Данная работа была 
переведена на русский язык и~послужила основой для многих последующих работ 
отечественных и~зарубежных специалистов. В~частности, Б.~Фокс показал~\cite{4}, 
что оптимальной стратегией управления полумарковским процессом в~варианте без 
переоценки является стационарная; аналогичные результаты были получены Э.~Денардо 
и~для варианта с~переоценкой~\cite{5}.

Среди последующих исследований алгоритмической направленности отметим работы 
Р.~Ховарда~\cite{6}, Б.~Фокса~\cite{4}, а также С.~Осаки и~Х.~Майна~\cite{7}. 
В~этих работах для нахождения оптимальных стратегий управления полумарковскими 
процессами использовался метод линейного программирования.

В 1970~г.\ была опубликована фундаментальная монография Х.~Майна и~С.~Осаки~\cite{8}, 
переведенная на русский язык в~1977~г., в~которой были систе\-ма\-ти\-зи\-ро\-ва\-ны и~изложены 
основные результаты по теории оптимального управления марковскими и~полумарковскими 
случайными процессами. Фактически данная книга стала итогом исследований по проблемам 
стохастического управления\linebreak
 за~10~лет. Отметим, что в~этой монографии рас\-смат\-ри\-ва\-лись 
марковские и~полумарковские модели управления с~конечными множествами состояний 
и~допустимых решений, принимаемых \mbox{в~каждом} состоянии. Были получены принципиальные 
тео\-ре\-ти\-че\-ские результаты, заключающиеся в~том, что оптимальные стратегии управ\-ле\-ния 
для основных видов рас\-смат\-ри\-ва\-емых моделей с~переоценкой и~без переоценки являются 
детерминированными и~стационарными. Были разработаны и~обоснованы процедуры нахождения 
оптимальных стратегий управления. В~частности, для модели управления полумарковским 
процессом без переоценки, когда множество со\-сто\-яний образует один эргодический класс, 
а~показатель качества управления пред\-став\-ля\-ет собой стационарный средний удельный 
доход (см.~[8, гл.~5, п.~5.5]), процедура поиска оптимальной рандомизированной 
стратегии осуществлялась методом линейного программирования. Обратим особое внимание 
на данный результат, поскольку аналогичная модель управления полумарковским 
процессом будет рассмотрена в~настоящей работе.

Принципиальную роль в~развитии теории стохастического управления сыграла 
монография И.\,И.~Гихмана и~А.\,В.~Скорохода~\cite{9}. В~этой книге были впервые 
систематически изложены основы теории оптимального управления случайными процессами 
с~дискретным и~непрерывным временем, включая теорию управления процессами, которые 
описываются стохастическими дифференциальными уравнениями. Отдельно были рас\-смот\-ре\-ны 
проблемы управления марковскими процессами с~дискретным временем и~скачкообразными 
марковскими процессами с~непрерывным временем. Роли множеств состояний и~допустимых 
управ\-ле\-ний играли пространства весьма общей структуры. Для широких классов функционалов 
качества управ\-ле\-ния (так называемых эволюционных функционалов в~марковских моделях 
с~дискретным временем и~интегральных функционалов накопления в~марковских моделях 
с~непрерывным временем) были доказаны теоремы о~существовании и~формах пред\-став\-ле\-ния 
оптимальных стратегий управ\-ле\-ния. Было установлено, что для однородных марковских 
моделей оптимальные стратегии управ\-ле\-ния существуют, являются стационарными 
и~детерминированными. Иначе говоря, такие стратегии задаются детерминированными 
функциями, аргументом которых является со\-сто\-яние сис\-те\-мы в~момент принятия решения, 
и~не зависящими от самого момента принятия решения. Что же касается важного вопроса 
о~формах представления этих функций, то их можно охарактеризовать следующим образом. 
Были найдены функциональные уравнения, осложненные условием экстремума, которым 
удовле\-тво\-ря\-ют упомянутые функции. По существу эти соотношения пред\-став\-ля\-ют собой 
уравнения Беллмана для соответствующих динамических стохастических моделей.

Особо отметим, что в~монографии~\cite{9} не рас\-смат\-ри\-ва\-лись проблемы управления 
полумарковскими процессами. Однако дальнейшее развитие общей теории управления 
такими процессами шло по пути, идейно намеченному в~указанной книге.

В последующие годы развитие теории управ\-ле\-ния полумарковскими процессами 
осуществля-\linebreak лось по направлению усложнения моделей и~обобщения исходных предположений. 
Например,\linebreak в~работах~\cite{10, 11} рассмотрены управляемые по\-лумарковские процессы при 
весьма общих предположениях относительно характера пространств состояний и~управлений. 
Проблемы управления исследовались по отношению к~различным видам целевых показателей, 
обобщающих упомянутый выше стационарный показатель средней удельной прибыли. В~этих 
работах доказывается, что оптимальная стратегия управления по отношению к~каж\-до\-му из 
показателей существует и~является одной и~той же стационарной детерминированной 
стратегией, определяемой некоторой функцией, заданной на множестве со\-сто\-яний процесса. 
Об этой функции известно лишь то, что она удовлетворяет некоторому интегральному 
уравнению, которое по содержанию пред\-став\-ля\-ет собой уравнение Бел\-лма\-на для 
соответствующей задачи управ\-ления.

Среди исследований, предшествовавших настоящему, отметим работу 
В.\,А.~Каштанова~[12, гл. 13]. В этом разделе коллективной монографии~\cite{12} 
автором была рассмотрена проблема оптимального управления полумарковским 
процессом с~конечным множеством состояний и~множеством возможных решений, 
которое представляет собой произвольный интервал множества вещественных чисел. 
Модель относится к~виду моделей без переоценки, показателем качества управления 
служит стационарное значение среднего удельного дохода, определяемое аналогично 
классическим работам~\cite{3, 8}. Рандомизированное управление в~каждом состоянии 
определяется в~соответствии с~вероятностным распределением, совокупность которых 
задает\linebreak
 стратегию управления. В.\,А.~Каш\-та\-но\-вым было\linebreak сформулировано утверждение о том, 
что стацио\-нарное значение среднего удельного дохода представляет собой 
дроб\-но-ли\-ней\-ный интегральный функционал от набора вероятностных распределений, 
образующих стратегию управления. При этом\linebreak ранее~[12, гл.~10; 13] было уста\-нов\-ле\-но, 
что дроб\-но-ли\-ней\-ный функционал достигает экстремума на вырожденных распределениях. 
Отсюда естест-\linebreak венно следует, что оптимальная стратегия управ\-ле-ния является 
детерминированной и~должна\linebreak определяться точкой экстремума функции, представляющей 
собой отношение подынтегральных функций чис\-ли\-те\-ля и~знаменателя данного 
дроб\-но-ли\-ней\-но\-го функционала. Однако в~\cite{12} не были получены явные 
представления для указан-\linebreak ных функций. Кроме того, приведенный в~гл.~10 
монографии~\cite{12} вариант теоремы об экстремуме дроб\-но-ли\-ней\-но\-го 
интегрального функционала требовал проверки выполнения условия существования 
этого экстремума. Такие условия указаны не были. В~связи с~этими обстоятельствами 
использовать полученные в~\cite{12} результаты для доказательства существования 
оптимальной детерминированной стратегии управ\-ле\-ния полумарковским процессом и~для 
строгого обоснования способа нахождения такой стратегии оказалось невозможным.

Настоящее исследование посвящено теоретическому обоснованию нового метода 
нахождения\linebreak оптимальной стратегии управления полумарковским процессом с~конечным 
множеством со\-сто\-яний. Рассматриваются марковские рандомизи\-рованные стратегии 
управления, определяемые конеч\-ным набором вероятностных мер, соответствующих 
каждому состоянию. Показателем качества управления служит уже упоминавшийся 
классический  показатель~--- стационарное значение средней удельной прибыли. 
Доказано, что этот показатель представляет собой дроб\-но-ли\-ней\-ный интегральный 
функционал от набора вероятностных мер, задающих стратегию управления. При этом, 
в~отличие от~\cite{12}, получены явные аналитические представления для подынтегральных 
функций числителя и~знаменателя этого дроб\-но-ли\-ней\-но\-го\linebreak
 функционала. Дальнейшие 
результаты основываются на новой усиленной и~обобщенной форме\linebreak
 теоремы об экстремуме 
дроб\-но-ли\-ней\-но\-го интегрального функционала, впервые опубликованной 
в~работе П.\,В.~Шнуркова~\cite{14}. Согласно\linebreak
 утверж\-де\-нию этой теоремы, если 
существует глобальный экстремум так называемой основной функции дроб\-но-ли\-ней\-но\-го 
функционала, которая пред\-став\-ля\-ет собой отношение подынтегральных функций чис\-ли\-те\-ля 
и~знаменателя, то существует безусловный экстремум самого дроб\-но-ли\-ней\-но\-го 
функционала, который достигается на наборе вырожденных вероятностных распределений, 
сосредоточенных в~точке глобального экстремума. В~этом случае оптимальная стратегия 
управ\-ле\-ния существует, является стационарной и~детерминированной и~определяется точкой, 
в~которой основная\linebreak функция достигает глобального экстремума. Таким\linebreak образом, проблемы 
существования оптимальной стратегии управ\-ле\-ния полумарковским процессом и~ее 
нахождения сводятся к~задаче чис\-лен\-но\-го исследования на глобальный экстремум 
заданной функции от конечного чис\-ла вещественных переменных.

\section{Общее описание модели управления полумарковским случайным процессом}

Построим модель управления полумарковским случайным процессом, следуя общему 
подходу, принятому в~классических работах~\cite{3, 8}. Пусть $\xi(t)$~--- 
случайный полумарковский процесс с~конечным множеством состояний
$X\hm=\{1,2,\ldots, N\}$, $N\hm< \infty$. Обозначим через~$t_n$, $n=0,1,2,\ldots$, 
$t_0\hm=0$, случайные моменты изменения состояний данного процесса, 
$\theta_n\hm=t_{n+1}-t_n$, $n\hm=0,1,2,\ldots$, $\xi_n\hm=\xi(t_n)\hm=\xi(t_n+0)$, 
$n\hm=0,1,2,\ldots$ (предполагается, что траектории процесса~$\xi(t)$ 
непрерывны справа). Случайная последовательность~$\{\xi_n\}$
образует цепь Маркова, вложенную в~полумарковский процесс~$\xi(t)$.
Зададим набор измеримых пространств\linebreak $(U_1, \mathscr{B}_1), 
(U_2, \mathscr{B}_2), \ldots, (U_N, \mathscr{B}_N)$, где $U_i$~--- 
множество возможных допустимых управ\-ле\-ний, $\mathscr{B}_i$~--- $\sigma$-ал\-геб\-ра 
подмножеств множества~$U_i$, вклю\-ча\-ющая в~себя все одноточечные подмножества\linebreak  
множества~$U_i$, т.\,е.\ если $u_i \hm\in U_i$, то $\{u_i\} \hm\in \mathscr{B}_i$, 
$i\hm=1,2,\ldots, N$.
Пусть $\Gamma_i$~--- некоторое множество всевозможных вероятностных мер $\Psi_i 
\hm \in \Gamma_i$, заданных на $\sigma$-ал\-геб\-ре~$\mathscr{B}_i$, $i\hm=1,2,\ldots,N$.

Поскольку идейное содержание и~свойства вероятностных мер существенно используются 
в~данной работе, укажем на некоторые фундаментальные издания, в~которых 
изложена соответствующая тео\-рия. Понятие и~основные свойства вероятностной 
меры определены и~подробно проанализированы в~книге А.\,Н.~Ширяева~\cite[гл.~II]{15}. 
Глубокое изложение основ теории вероятностных мер имеется также в~книге 
А.\,А.~Боровкова~\cite{16}. Заметим попутно, что в~книге~\cite{16} имеются разделы, 
посвященные изложению основ теории полумарковских и~регенерирующих случайных процессов. 
Из зарубежных изданий отметим фундаментальную работу П.~Хеннекена и~А.~Тортра~\cite{17}, 
основная часть которой посвящена изложению математических основ теории вероятностей.

Введем специальное понятие вырожденной вероятностной меры, которое будет часто 
использоваться в~дальнейшем. Пусть $(U_0, \mathscr{B}_0)$~--- некоторое измеримое 
пространство, $\mathscr{B}_0$~--- $\sigma$-ал\-геб\-ра подмножеств множества~$U_0$, 
включающая в~себя все одноточечные подмножества этого множества.

\medskip

\noindent
\textbf{Определение 1.}\ Вероятностная мера~$\Psi^*$, заданная 
на $\sigma$-ал\-геб\-рe~$\mathscr{B}_0$, называется вырожденной, если существует 
такой элемент $u^* \hm\in U_0$, для которого выполняются условия $\Psi^*(\{u^*\})\hm=
1$, $\Psi^*(U_0 \setminus \{u^*\})\hm=0$, где $\{u^*\}=u^*$~--- 
множество, состоящее из единственной точки $u^* \hm\in U_0$. Соответствующая 
точка $u^* \hm\in U_0$ будет называться точкой сосредоточения вырожденной 
вероятностной меры~$\Psi^*$.
Таким образом, всякая вырожденная вероятностная мера~$\Psi^*$ определяется 
своей точкой сосредоточения~$u^*$. В~дальнейшем будем использовать 
обозначение~$\Psi_{u^*}^{*}$, имея в~виду, что вырожденная вероятностная мера~$\Psi^*$ 
сосредоточена в~точке~$u^*$.
Отметим также, что вырожденная вероятностная мера~$\Psi_{u^*}^{*}$ соответствует 
детерминированной величине, которая принимает фиксированное значение $u\hm=u^*$ 
с~вероятностью, равной единице.

\medskip

Обозначим через $\Gamma_0$ множество всех  вероятностных мер, заданных 
на измеримом пространстве ($U_0, \mathscr{B}_0$), а через~$\Gamma_0^*$~--- 
множество всех вырожденных вероятностных мер, заданных на этом пространстве, 
$\Gamma_0^*\hm\in \Gamma_0$. Аналогичные обозначения будут использоваться 
и~в~дальнейшем. Заметим, что множество~$\Gamma_0^*$ находится во взаимно
 однозначном соответствии с~множеством точек сосредоточения вырожденных 
 вероятностных мер, т.\,е.\ с~множеством~$U_0$.

Пусть $\Gamma_i^{*}$~--- множество всех вырожденных мер, заданных на 
$\sigma$-ал\-геб\-ре~$\mathscr{B}_i$, $\Gamma_i^{*}\hm\subset \Gamma_i$.
Произвольная вероятностная мера~$\Psi_i$ описывает случайную величину, 
принимающую значения в~$U_i$, а вырожденная мера~$\Psi_i^*$, сосредоточенная 
в~точке~$u_i^*$, соответствует детерминированной величине $u_i^*\hm\in U_i$.
Предполагается, что соответствующие конструкции определены на всех измеримых 
пространствах управлений $(U_1, \mathscr{B}_1), (U_2, \mathscr{B}_2), \ldots, 
(U_N,\mathscr{B}_N)$.

Предположим, что управления случайным полумарковским процессом~$\xi(t)$ 
осуществляются в~моменты времени~$t_n,$ $n\hm=0,1,2,\ldots,$
непосредственно после изменения состояния процесса. Если\linebreak 
$\xi_n\hm=\xi(t_n)\hm=i \hm\in X$, то значение управления представляет 
собой случайную величину~$u_n$, принимающую значения в~множестве допустимых 
управ\-ле\-ний~$U_i$ и~описываемую вероятностной мерой (распределе\-ни\-ем 
вероятностей) $\Psi_i \hm\in \Gamma_i$.
Будем предполагать, что при фиксированном условии $\xi_n\hm=\xi(t_n)=i$ 
управ\-ле\-ние определяется независимо от прошлого поведения процесса~$\xi(t)$ 
и~вероятностная мера~$\Psi_i$,
описывающая стохастическое управление~$u_n$, зависит только от состояния $i\hm\in X$.
Тогда выбор управ\-ле\-ний в~моменты изменения состояний $\{t_n, n\hm=0,1,2,\ldots \}$ 
описывается набором вероятностных мер (распределений вероятностей) 
$(\Psi_1, \Psi_2,\ldots, \Psi_N)$, 
$\Psi_i \hm\in \Gamma_i$, $i\hm=1,2,\ldots,N$.
Назовем любой такой набор стратегией управ\-ле\-ния полумарковским процессом~$\xi(t)$. 
По своим свойствам такая стратегия является марковской, однородной 
и~рандомизированной.

Следуя классической монографии П.~Халмоша~\cite[гл.~VII]{18}, 
рассмотрим декартово произведение пространств $U\hm=U_1\times U_2\times \cdots\times U_N$ 
и~соответствующих $\sigma$-ал\-гебр $\mathscr{B}\hm=\mathscr{B}_1\times \mathscr{B}_2
\times \cdots \times\mathscr{B}_N$. Обозначим через $\Psi\hm=\Psi_1\times \Psi_2\times \cdots
\times \Psi_N$ вероятностную меру на~$(U,\mathscr{B})$, определяемую как 
произведение мер $\Psi_1,\Psi_2, \ldots , \Psi_N$, где $\Psi_i \hm\in \Gamma_i$, 
$i\hm=1,2,\ldots,N$. Обозначим также через~$\Gamma$ множество вероятностных мер~$\Psi$, 
заданных на~$(U,\mathscr{B})$, которые пред\-став\-ля\-ют собой произведение 
мер $\Psi_1,\Psi_2, \ldots , \Psi_N$, где $\Psi_i \hm\in \Gamma_i$, $i\hm=1,2,\ldots,N$.
Множество~$\Gamma$ можно отож\-де\-ст\-вить с~множеством всех стратегий управ\-ле\-ния 
полумарковским процессом~$\xi(t)$.

Проблема оптимального управления полумар\-ковским процессом~$\xi(t)$ будет в~дальнейшем 
сформулирована в~виде задачи безусловного экстремума некоторого функционала 
$I(\Psi)\hm=I(\Psi_1,\Psi_2, \ldots , \Psi_N)$, заданного на множестве 
допустимых стратегий управления. Содержание показателя качества управления~$I(\Psi)$, 
аналитическое представление для него, а~также описание множества допустимых 
стратегий управления будут приведены в~последующих разделах данной работы.

Для получения дальнейших результатов потребуются различные вероятностные 
характеристики управляемого полумарковского процесса~$\xi(t)$. Как известно из
 общей теории полумарковских процессов~\cite{19, 20}, 
 основной вероятностной характеристикой такого процесса является так называемая 
 полумарковская функция. Определим эту функцию для процесса с~управлением 
 (см.~\cite[гл.~5]{8}):
\begin{multline}
Q_{ij}(t,u)=
{\sf P}\left(\xi_{n+1}=j,\theta_n<t \mid \xi_n=i, u_n=u\right)\,,\\
t\in [0,\infty)\,,\ u\in U_i\,;\ i,j\in X=\{1,2,\ldots,N\}\,. \label{e1}
\end{multline}
Используя полумарковские функции, можно получить вероятности перехода 
управляемой цепи Маркова~$\{\xi_n\}$:
\begin{multline}
p_{ij}(u)={\sf P}\left(\xi_{n+1}=j \mid \xi_n=i, u_n=u\right)= {}\\
{}=
\lim\limits_{t\rightarrow\infty}Q_{ij}(t,u)\,,\enskip
u\in U_i\,;\enskip i,j\in X\,, 
\label{e2}
\end{multline}
а также функции распределения длительностей пребывания полумарковского 
процесса~$\xi(t)$ в~соответствующих состояниях:

\noindent
\begin{multline}
H_{i}(t,u)={\sf P}\left(\theta_n<t \mid \xi_n=i, u_n=u\right)={}\\
{}=
\sum\limits_{j\in X}Q_{ij}(t,u)\,,\enskip
t\in [0,\infty)\,,\  u\in U_i\,; \  i\in X\,. 
\label{e3}
\end{multline}

Обозначим через
\begin{multline}
T_{i}(u)=\mathbf{E}\left[\theta_n \mid \xi_n=i, u_n=u\right]={}\\
{}=
\int\limits_0^{\infty}\left[1-H_i(t,u)\right]\,dt\,,\enskip
u\in U_i\,,\ i\in X\,, 
\label{e4}
\end{multline}
математические ожидания длительностей пребывания полумарковского процесса~$\xi(t)$ 
в~каждом из состояний.

Введенные выше характеристики~(1)--(4) определены для случая, когда 
в~момент изменения состояния~$t_n$ процесс оказывается в~состоянии~$i$ 
и~принимается решение $u\hm\in U_i$. При заданной стратегии управления 
$\Psi\hm=\left(\Psi_1,\Psi_2, \ldots , \Psi_N\right)$ можно записать 
соответствующие вероятностные характеристики без условия на управление, а~именно:
\begin{multline*}
Q_{ij}(t)={\sf P}\left(\xi_{n+1}=j,\theta_n<t \mid \xi_n=i\right)={}\\
{}=
\int\limits_{U_i}Q_{ij}(t,u) \,d\Psi_i(u)\,,\enskip 
t\in [0,\infty)\,,\ i,j\in X\,; %\label{e5}
\end{multline*}

\vspace*{-12pt}

\noindent
\begin{multline}
p_{ij}={\sf P}\left(\xi_{n+1}=j \mid \xi_n=i\right)=
\int\limits_{U_i}p_{ij}(u)\, d\Psi_i(u)\,,\\  
i,j\in X\,; 
\label{e6}
\end{multline}

\vspace*{-9pt}

\noindent
\begin{equation}
T_{i}=\mathbf{E}\left[\theta_n \mid \xi_n=i\right]=
\int\limits_{U_i}T_{i}(u)\,d\Psi_i(u)\,,\enskip i\in X\,. 
\label{e7}
\end{equation}
В дальнейшем будем предполагать, что для рас\-смат\-ри\-ва\-емой 
полумарковской модели заданы вероятностные характеристики 
$p_{ij}(u)$, $u\hm\in U_i$, $i,j\hm\in X$, и~$T_i(u)$, $u\hm\in U_i$, $i\hm\in X$, 
определяемые соотношениями~(\ref{e2}) и~(\ref{e4}). 
Для фиксированной стратегии управления $\Psi\hm=(\Psi_1, \Psi_2,\ldots, \Psi_N)$ 
соответствующие вероятностные характеристики~$p_{ij}$ и~ $T_i$, $i,j\hm\in X,$ 
определены равенствами~(\ref{e6}) и~(\ref{e7}) без условий на управление.

\section{Стационарный стоимостной показатель качества управления}

Определим некоторый стоимостной аддитивный функционал, связанный 
с~рассматриваемым полумарковским процессом~$\xi(t)$. По содержанию этот функционал 
представляет собой случайный\linebreak доход или прибыль, накопленную за период времени $[0,t]$. 
Определения такого функционала приведены в~основополагающих работах~[3; 8, гл.~5].\linebreak 
Обозначим через $\widetilde{v}(t)$, $t\hm\geq 0$, значение этого аддитивного 
функционала в~момент времени~$t$; $\widetilde{v}_n\hm=\widetilde{v}(t_n\hm+0)$~--- 
соответствующее значение непосредственно после очередного момента изменения\linebreak 
состояния~$t_n$, $n\hm=0,1,2,\ldots$; $\widetilde{v}_0\hm=v_0$~--- 
заданное начальное значение в~момент $t\hm=0$. Рассмотрим величину
\begin{multline}
d_i(u)=\mathbf{E}\left[\widetilde{v}_{n+1}-\widetilde{v}_n \mid \xi_n=i\,, 
u_n=u\right]\,,\\
u\in U_i\,, \enskip i\in X\,, \label{e8}
\end{multline}
представляющую собой математическое ожидание приращения стоимостного 
аддитивного функционала за период времени между последовательными 
изменениями состояния полумарковского процесса~$\xi(t)$. Тогда соответствующее 
математическое ожидание, вычисляемое без условия на решение, 
принимаемое в~момент времени~$t_n$, представляется в~виде:
\begin{equation*}
d_i=\mathbf{E}\left[\widetilde{v}_{n+1}-\widetilde{v}_n \mid \xi_n=i\right]=
\!\int\limits_{U_i}\!d_i(u)\,d\Psi_i(u)\,,\ i\in X \,. %\label{e9}
\end{equation*}

Предположим, что для заданной стратегии управ\-ле\-ния 
$\Psi\hm=(\Psi_1,\Psi_2,\ldots,\Psi_N)$ вложенная цепь Маркова~$\{\xi_n\}$ 
имеет ровно один класс возвратных положительных состояний (по терминологии, 
принятой в~\cite{8}, такое множество состояний называется эргодическим классом). 
Как известно~\cite[гл.~VIII]{15}, данное условие является необходимым 
и~достаточным для существования единственного\linebreak стационарного распределения. 
Обозначим это стационарное распределение цепи Маркова~$\{\xi_n\}$ через 
$\pi\hm=(\pi_1, \pi_2,\ldots, \pi_N)$. Заметим, что данное\linebreak распределение зависит  
от стратегии управления $\Psi\hm=(\Psi_1,\Psi_2,\ldots,\Psi_N)$. При указанном 
условии имеет место следующий результат, называемый эргодической теоремой 
для аддитивного стоимостного функционала:
\begin{equation}
I=\lim\limits_{t\rightarrow\infty}\fr{\mathbf{E}\widetilde{v}(t)}{t}=
\fr{\sum\nolimits_{i=1}^N d_i\pi_i}{\sum\nolimits_{i=1}^N T_i\pi_i}\,. 
\label{e10}
\end{equation}

Соотношение~(\ref{e10}) доказано в~работе~\cite[гл.~5]{8}. Заметим, что аналогичные 
результаты имеют мес\-то для гораздо более общих полумарковских моделей~\cite{10, 11}.

По своему прикладному содержанию величина, определяемая соотношением~(\ref{e10}), 
представляет собой
среднюю удельную прибыль, связанную с~эволюцией системы в~стационарном
режиме. Кроме того, величина~$I$ представляет собой функционал от
набора вероятностных распределений~$\Psi_{i}$, $i\hm\in\lbrace 1,\ldots
,N\rbrace $, определяющих стратегию управле-\linebreak\vspace*{-12pt}

\pagebreak

\noindent
ния системой. 
В~дальнейшем будем рассматривать стационарный стоимостной функционал 
$I\hm=I(\Psi_{1},\Psi_{2},\ldots , \Psi_{N})$ как
показатель качества управ\-ле\-ния системой и~построенным полумарковским
процессом~$\xi (t)$.

\section{Представление стационарного показателя в~форме
дробно-линейного интегрального функционала}

В данном разделе будет приведено утверждение об аналитическом
представлении стационарного стоимостного функционала~(\ref{e10}), 
служащего критерием качества управления в~рассматриваемой задаче управления 
полумарковским процессом.

\smallskip

\noindent
\textbf{Теорема 1.} \textit{Стационарный стоимостной показатель, 
определяемый равенством}~(\ref{e10}), \textit{представляет собой дроб\-но-ли\-ней\-ный
функционал от вероятностных распределений~$\Psi_{i}(u_{i})$,
$i\hm\in\{1,\dots,N\}$. Данный функционал задается
аналитически следующей формулой:}
\begin{multline}
I=I(\Psi_{1},\ldots, \Psi_{N})={}\\
\hspace*{-2mm}{}=\!
\fr{\int\nolimits_{U_1}\!{\cdots\! 
\int\nolimits_{U_N}\!{A(u_{1},\ldots ,u_{N})d\Psi_{1}(u_{1})\cdots
\,d\Psi_{N}(u_{N})}}}{\int\nolimits_{U_1}{\!\cdots\! \int\nolimits_{U_N}\!{B(u_{1},\ldots ,u_{N})
\,d\Psi_{1}(u_{1})\ldots
d\Psi_{N}(u_{N})}}},\!\!\! \label{e11}
\end{multline}
\textit{где подынтегральные функции числителя и~знаменателя выражаются
соотношениями}:
\begin{align}
A(u_{1},\ldots
,u_{N})&={}\notag\\
&\hspace*{-20mm}{}=\sum\limits_{i=1}^{N}{d_{i}(u_{i})}{\widehat{D}}^{(i)}(u_{1}, \ldots
,u_{i-1},u_{i+1}, \ldots , u_{N})\,;  \label{e12}\\
 B(u_{1},\ldots
,u_{N})&={}\notag\\
&\hspace*{-20mm}{}=\sum\limits_{i=1}^{N}{T_{i}(u_{i})}{\widehat{D}}^{(i)}(u_{1}, \ldots
,u_{i-1},u_{i+1}, \ldots , u_{N})\,.  \label{e13}
\end{align}
\textit{В свою очередь, функции} ${\widehat{D}}^{(i)}(u_{1}, \ldots
,u_{i-1},u_{i+1}, \ldots$\linebreak $\ldots , u_{N})$, $i\hm\in\{1,\dots,N\}$, 
\textit{входящие в~правые части формул}~(\ref{e12}) и~(\ref{e13}), 
\textit{определяются следующим образом:}

\noindent
\begin{multline}
{\widehat{D}}^{(i)}(u_{1}, \ldots ,u_{i-1},u_{i+1}, \ldots , u_{N})={}
\\
{}=(-1)^{N+i+2}\sum\limits_{\alpha ^{(N),i}}{(-1)}^{\delta (\alpha
^{(N),i}) }{\widehat{D}}_{0}^{(i)}\left(\alpha ^{(N),i},u_{1}, \ldots\right.\\
\left.\ldots , u_{i-1},u_{i+1}, \ldots , u_{N}\right)\,. \label{e14}
\end{multline}
\textit{Здесь} $\alpha ^{(N),i}=(\alpha _{1}, \ldots , \alpha _{i-1},\alpha_{i+1}, \ldots , 
\alpha _{N})$~\textit{--- произвольная
перестановка чисел }$(1, \ldots , i-1, i+1, \ldots , N)$;
$\delta
(\alpha ^{(N),i})$~\textit{--- число инверсий в~перестановке} 
$\alpha ^{(N),i}$;

\noindent
\begin{multline}
{\widehat{D}}_{0}^{(i)}(\alpha ^{(N),i},u_{1}, \ldots ,u_{i-1},u_{i+1},
\ldots , u_{N})={}\\
{} ={\widetilde{p}}_{1,\alpha _{1}}\left(u_{1}\right)\cdots {\widetilde{p}}_{i-1,\alpha
_{i-1}}\left(u_{i-1}\right){\widetilde{p}}_{i+1,\alpha _{i+1}}\left(u_{i+1}\right)\cdots\\
\cdots
{\widetilde{p}}_{N,\alpha _{N}}\left(u_{N}\right)\,, 
\label{e15}
\end{multline}
где
\begin{multline}
 {\widetilde{p}}_{k,\alpha _{k}}(u_{k})=
\begin{cases}
p_{kk}(u_{k})-1,\  & \alpha _{k}=k\,; \\
p_{k,\alpha _{k}}(u_{k}),\  & \alpha _{k}\ne k, \\
\end{cases}\\
 k=1, \ldots , i-1, i+1, \ldots ,N\,. \label{e16}
 \end{multline}
\textit{Функции $p_{ij}(u_i)$, $T_{i}(u_{i})$ и~$d_{i}(u_{i})$,
$u_i\hm\in U_i$, $i,j\hm\in \{1,2,\ldots,N\}$, 
входящие в~соотношения}~(\ref{e12})--(\ref{e16}), 
\textit{определяются равенствами}~(\ref{e2}), (\ref{e4}) \textit{и}~(\ref{e8}) \textit{соответственно.}

\smallskip

\noindent
Д\,о\,к\,а\,з\,а\,т\,е\,л\,ь\,с\,т\,в\,о\ теоремы~1 
в~весьма сжатой форме приведено в~работе~\cite{21}. Читателю, интересующемуся 
более подробным обоснованием данного результата, порекомендуем обратиться к~тексту 
кандидатской диссертации А.\,В.~Иванова~\cite[гл.~3]{22}.

\smallskip

Итак, теорема~1 позволяет получить явное аналитическое представление 
для стационарного стоимостного показателя вида~(\ref{e10}) в~форме 
дроб\-но-ли\-ней\-но\-го интегрального функционала от набора\linebreak вероятностных мер 
$\Psi\hm=(\Psi_{1},\Psi_{2},\ldots , \Psi_{N})$, за\-да\-ющих стратегию управления 
полумарковским процессом~$\xi(t)$. При этом подынтегральные функции числителя 
и~знаменателя задаются формулами~(\ref{e12}), (\ref{e13}) 
и~вспомогательными равенствами~(\ref{e14})--(\ref{e16}). Таким образом, функция
\begin{equation}
C\left(u_1, u_2,\ldots, u_N\right)=\fr{A(u_1, u_2,\ldots, u_N)}{B(u_1, u_2,\ldots, u_N)}\,,
\label{e17}
\end{equation}
которая в~дальнейшем будет называться основной функцией дроб\-но-ли\-ней\-но\-го 
интегрального функционала~(\ref{e11}) и~которая будет играть важную роль 
в~дальнейшем исследовании, также явно определяется формулами~(\ref{e17}), 
(\ref{e12}), (\ref{e13}).

\section{Формальная постановка оптимизационной задачи 
и~условия существования оптимальной стратегии управления полумарковским процессом}

Будем рассматривать проблему управления полумарковским процессом~$\xi(t)$ в~форме 
экстремальной задачи
\begin{multline}
I(\Psi)=I\left(\Psi_1, \Psi_2,\ldots,\Psi_N\right)\rightarrow \mathrm{extr}\,,
\\
\Psi=\left(\Psi_1, \Psi_2,\ldots,\Psi_N\right)\in\Gamma\,. \label{e18}
\end{multline}
При этом показатель качества управления~$I(\Psi)$ представляет собой 
дроб\-но-ли\-ней\-ный интегральный функционал вида~(\ref{e11}).

Для решения экстремальной задачи~(\ref{e18}) воспользуемся некоторым утверждением 
об экстремуме дроб\-но-ли\-ней\-но\-го интегрального функционала. Прежде 
чем сформулировать данное утверждение, отметим, что в~теории оптимизации 
хорошо известны задачи, в~которых целевая функция представляет собой 
отношение двух линейных отображений, а имеющиеся ограничения также линейны. 
Такой раздел называется дроб\-но-ли\-ней\-ным программированием. Основные
 теоретические результаты данного направления изложены в~работе~\cite{23},
  там же приведена подробная библиография. В~дальнейшем потребуется некоторый 
  специальный результат о безусловном экстремуме дроб\-но-ли\-ней\-но\-го 
  интегрального функционала вида~(\ref{e11}), который был впервые сформулирован 
  в~работе~\cite{14}. Заметим, что для использования этого результата необходимо, 
  чтобы выполнялись некоторые предварительные условия, которые в~данном случае 
  можно сформулировать следующим образом:
\begin{enumerate}[1.]
\item Интегральные выражения
\begin{align*}
I_1(\Psi)&=I_1\left(\Psi_1,\Psi_2,\ldots,\Psi_N\right)={}&\\
&\hspace*{-13mm}{}=\int\limits_{U_1}\!\cdots\!
\int\limits_{U_N}\!\!A\left(u_1,\ldots ,u_N\right)\,
d\Psi_1\left(u_1\right) %d\Psi_2\left(u_2\right)
\cdots
 d\Psi_N\left(u_N\right)\,;
\\
I_2(\Psi)&=I_2\left(\Psi_1,\Psi_2,\ldots,\Psi_N\right)={}&\\
&\hspace*{-13mm}{}=\int\limits_{U_1}\!\cdots\!\int\limits_{U_N}\!\!
B\left(u_1,\ldots,u_N\right)\,
d\Psi_1\left(u_1\right)% d\Psi_2\left(u_2\right)\cdots\\
\cdots d\Psi_N\left(u_N\right)
\end{align*}
определены для всех стратегий управления $\Psi\hm=(\Psi_1, \ldots,\Psi_N)
\hm\in \Gamma$.

\item Функционал $I_2(\Psi)=I_2(\Psi_1, \ldots,\Psi_N)\hm\neq 0$ 
для всех $\Psi\hm=(\Psi_1, \ldots,\Psi_N)\hm\in \Gamma$.

\item Множество $\Gamma$ включает в~себя множество всех вырожденных 
вероятностных мер: $\Gamma^* \hm\subset \Gamma$.
\end{enumerate}

Сделаем несколько важных замечаний по поводу введенных предварительных условий.

\smallskip

\noindent
\textbf{Замечание~1.}\ Из условия~2 следует, что функция $B(u_1, u_2,\ldots, u_N)$ 
не может принимать значения разных знаков. С~учетом условия~3 
получаем, что указанная функция должна обладать \mbox{свойством} строгой 
знакопостоянности на всем множестве~$U$. С~другой стороны, если выполняется 
условие строгой знакопостоянности функции $B(u_1, u_2,\ldots, u_N), 
(u_1, u_2,\ldots, u_N)\hm\in U$, то условие~2 выполняется автоматически.

\smallskip

\noindent
\textbf{Замечание~2.}\ Если рассматривать в~качестве целевого функционала 
$I(\Psi_1, \Psi_2,\ldots,\Psi_N)$ экстремальной задачи~(\ref{e18}) 
стационарный стоимостной пока\-затель~(\ref{e10}), то функция $B(u_1,u_2,\ldots,u_N)$ 
имеет\linebreak следующее теоретическое содержание. Данная функция представляет собой условное 
математическое ожидание длительности периода времени между соседними моментами 
изменения со\-сто\-яния полумарковского процесса~$\xi(t)$ при условии, что стратегия 
его управ\-ле\-ния является детерминированной и~задается набором значений аргументов 
$(u_1,u_2,\ldots,u_N)$. Тогда условие строгой положительности функции 
$B(u_1,u_2,\ldots,u_N)$ при всех $(u_1,u_2,\ldots,u_N)\hm\in U$ является естественным 
и~фактически означает, что при любой заданной детерминированной стратегии 
управ\-ле\-ния процесс~$\xi(t)$ не имеет мгновенных со\-сто\-яний, длительность пребывания 
в~которых равна нулю.

\smallskip

\noindent
\textbf{Замечание~3.}\ Сделаем некоторые замечания, связан\-ные с~подынтегральной 
функцией числителя дроб\-но-ли\-ней\-но\-го интегрального функционала~(\ref{e11}). 
Как и~ранее, будем рассматривать в~качестве целевого функционала $I(\Psi_1, \Psi_2,\ldots,\Psi_N)$\linebreak 
экстремальной задачи~(\ref{e18}) стационарный стоимостной показатель~(\ref{e10}). 
Тогда для любого фиксированного набора значений аргументов $(u_1,u_2,\ldots,u_N)\hm\in U$ 
значение функции $A(u_1,u_2,\ldots\linebreak \ldots,u_N)$ представляет собой условное математическое
 ожидание приращения рассматриваемого стоимостного функционала, 
 происшедшее за время пребывания полумарковского процесса~$\xi(t)$ в~некотором 
 фиксированном  состоянии при условии, что стратегия управления является 
 детерминированной и~задается указанным набором $(u_1,u_2,\ldots,u_N)\hm\in U$. 
 Отметим, что в~теореме об экстремуме дроб\-но-ли\-ней\-но\-го интегрального 
 функционала, доказанной в~работе~\cite[гл.~10]{12}, 
 на подынтегральную функцию числителя накладываются условия ограниченности на 
 всем множестве значений аргумента. Для многих математических моделей и~связанных 
 с~ними задач оптимального управления такое условие является излишне ограничительным. 
 В~качестве примера можно привести модели оптимального управления запасом непрерывного 
 продукта, рассмотренные в~работах~\cite{27, 28}. 
 В~настоящем исследовании на функцию $A(u_1,u_2,\ldots,u_N)$ накладывается только 
 условие интегрируемости по любому заданному набору вероятностных мер 
 $\Psi\hm=(\Psi_1, \Psi_2,\ldots,\Psi_N)$, образующему стратегию управления 
 полумарковским процессом~$\xi(t)$ (условие~1 системы предварительных условий).

\smallskip

\noindent
\textbf{Замечание~4.} Условия~1--3 являются необходимыми для корректной 
постановки задачи безусловного экстремума дроб\-но-ли\-ней\-но\-го интегрального 
функционала. Если этот функционал служит показателем качества в~задаче оптимального 
управления случайным процессом, то необходимо добавить к~этим условиям дополнительное, 
связанное с~некоторой регулярностью самого управляемого процесса, а~именно: некоторый 
содержательный показатель, связанный с~поведением этого процесса, должен существовать 
и~быть представимым в~виде дроб\-но-ли\-ней\-но\-го интегрального функционала. 
Если потребовать, чтобы выполнялось эргодическое соотношение~(\ref{e10}), 
то можно использовать\linebreak теорему~1 и~сформулировать задачу оптимального управ\-ле\-ния 
в~виде~(\ref{e18}) для дроб\-но-ли\-ней\-но\-го\linebreak интегрального функционала~(\ref{e11}). 
Таким образом, необходимо ввести условие, обеспечивающее существование единственного 
стационарного распределения вложенной цепи Маркова и~выполнение\linebreak соотношения~(\ref{e10}). 
По аналогии с~[8, гл.~5] сформулируем это дополнительное условие в~следующем виде:
\begin{enumerate}
\setcounter{enumi}{3}
\item Для любой рассматриваемой стратегии управ\-ле\-ния $\Psi\hm=
(\Psi_1, \Psi_2,\ldots,\Psi_N)\hm\in \Gamma$ вложенная цепь Маркова 
полумарковского процесса $\xi(t)$ имеет ровно один класс возвратных 
положительных состояний.
\end{enumerate}

Теперь определим понятие допустимой стратегии управления полумарковским процессом 
с~конечным множеством состояний.

\smallskip

\noindent
\textbf{Определение~2.}\ Назовем стратегию управления 
$\Psi\hm=(\Psi_1, \Psi_2,\ldots,\Psi_N)$ 
допустимой в~данной задаче, если она удовлетворяет условиям~1--4.


\smallskip

\noindent
\textbf{Замечание~5.}\ Как следует из замечания~1, если потребовать, 
чтобы функция $B(u_1, u_2,\ldots,u_N)$ являлась строго знакопостоянной при 
всех $(u_1, u_2,\ldots,u_N)\hm\in U$, то можно считать допустимыми стратегии 
$(\Psi_1, \Psi_2,\ldots,\Psi_N)$, удовлетворяющие условиям~1, 3, 4. С~учетом замечания~2 
о~естественном характере условия строгой знакопостоянности функции $B(u_1,u_2,\ldots,u_N)$ 
при всех значениях аргументов $(u_1, u_2,\ldots,u_N)\hm\in U$ будем требовать 
выполнения этого условия в~формулировке приводимой в~дальнейшем основной 
теоремы об оптимальной стратегии управления полумарковским процессом.

\smallskip

\noindent
\textbf{Замечание~6.}\ Ниже будет сформулирована и~доказана основная 
теорема об оптимальной стра\-тегии управления полумарковским процессом с~конеч\-ным 
множеством состояний. Будем формулировать эту теорему по отношению к~экстремальной 
задаче~(\ref{e18}), в~которой целевой функционал $I(\Psi_1, \Psi_2,\ldots,\Psi_N)$ 
имеет вид дроб\-но-ли\-ней\-но\-го интегрального функционала. 
Это обстоятельство связано с~тем, что целевой функционал в~задаче 
оптимального управления необязательно должен иметь характер стационарного 
стоимостного показателя вида~(\ref{e10}). В~частности, еще в~1983~г.\ П.\,В.~Шнурковым 
было установлено~\cite{24}, что ряд показателей, связанных 
с~временем пребывания управляемого полумарковского процесса в~заданном конечном 
подмножестве состояний, имеет структуру дроб\-но-ли\-ней\-но\-го интегрального 
функционала от набора вероятностных мер, определяющих стратегию управления. 
Таким образом, рассматриваемая задача управления имеет более общий характер, 
чем задача, в~которой целевой функционал представляет собой стационарный 
стоимостной показатель вида~(\ref{e10}).






\smallskip

\noindent
\textbf{Замечание~7.}\ Если рассматривать задачу оптимального управления 
полумарковским процессом, в~кото\-рой целевой функционал не совпадает 
со стационарным стоимостным показателем~(\ref{e10}), то возможно, что могут 
потребоваться другие дополнительные условия, обеспечивающие существование этого 
показателя и~его представление в~форме~(\ref{e11}). В~связи с~этим в~формулировке 
основной теоремы будем использовать термин допустимые стратегии в~широком смысле, 
имея в~виду выполнение всех необходимых условий для каждого рассмат\-ри\-ва\-емо\-го 
показателя качества управления.

\smallskip


\noindent
\textbf{Замечание 8.} Множество допустимых стратегий может 
не совпадать с~множеством всех возможных стратегий управления. 
В~частности, допустимые стратегии могут состоять только из дискретных вероятностных 
мер $\Psi_1, \Psi_2,\ldots,\Psi_N$, т.\,е.\ таких, которые сосредоточены на дискретных 
множествах точек пространств $U_1, U_2,\ldots,U_N$.

\section{Теоретическое решение задачи оптимального управления}

Перейдем к~формулировке и~доказательству тео\-ре\-мы об 
оптимальной стратегии управ\-ле\-ния полумарковским процессом с~конечным 
множеством состояний.

\smallskip

\noindent
\textbf{Теорема~2.} \textit{Рассмотрим проблему оптимального управ\-ле\-ния 
полумарковским процессом~$\xi(t)$ в~виде экстремальной задачи}~(\ref{e18}), 
\textit{определенной на множестве допустимых стратегий $\Gamma$, 
для дроб\-но-ли\-ней\-но\-го 
функционала}~(\ref{e11}). \textit{Пусть функция $B(u_1,u_2,\ldots,u_N)$, 
входящая в~определение функционала}~(\ref{e11}),
\textit{является строго знакопостоянной (строго положительной или строго отрицательной) 
при всех значениях аргументов $(u_1,u_2,\ldots,u_N)\hm\in U$.
Тогда справедливы сле\-ду\-ющие утверждения}:
\begin{enumerate}[1.]
\item \textit{Если функция} $C(u_1,u_2,\ldots,u_N)\hm=A(u_1,u_2,\ldots$\linebreak
$\ldots,u_N)/{B(u_1,u_2,\ldots,u_N)}$ 
\textit{ограничена сверху или снизу и~достигает глобального экст\-ре\-му\-ма на множестве
$U\hm=U_1\times U_2\times \cdots \times U_N$ (максимума или минимума), 
то оптимальная стратегия управления полумарковским процессом~$\xi(t)$ существует, 
является детерминированной и~определяется
вырожденной вероятностной мерой $\Psi^*\hm\in \Gamma^*$, сосредоточенной в~точке, 
в~которой достига\-ет соответствующего экстремума функция $C(u_1,u_2,\ldots,u_N)$,
и~при этом выполняются соотношения}:
\begin{multline}  %{\substack{{i=\overline{1,n}}\\ {j=\overline{1,l}}}}
\max\limits_{\Psi \in \Gamma} I(\Psi)=
\max\limits_{\substack{{\Psi_i \in \Gamma_i\,,}\\ 
{i=\overline{1,N}}}}
I\left(\Psi_1,\Psi_2,\ldots,\Psi_N\right)={}\\
{}=
\max\limits_{\substack{{\Psi_i^* \in \Gamma_i^*,}\\ 
{i=\overline{1,N}}}}
 I\left(\Psi_1^*,\Psi_2^*,\ldots,\Psi_N^*\right)={}\\
{}=\max\limits_{(u_1,u_2,\ldots,u_N)\in U}\fr{A(u_1,u_2,\ldots,u_N)}
{B(u_1,u_2,\ldots,u_N)}\,; \label{e19}
\end{multline}

\vspace*{-12pt}

\noindent
\begin{multline*}
\min\limits_{\Psi \in \Gamma} I(\Psi)=
\min\limits_{\substack{{\Psi_i \in \Gamma_i\,,}\\ 
{i=\overline{1,N}}}} I\left(\Psi_1,\Psi_2,\ldots,\Psi_N\right)={}\\
{}=
\min\limits_{\substack{{\Psi_i^* \in \Gamma_i^*,}\\ 
{i=\overline{1,N}}}}
I\left(\Psi_1^*,\Psi_2^*,\ldots,\Psi_N^*\right)={}\\
{}=\min\limits_{(u_1,u_2,\ldots,u_N)\in U}\fr{A(u_1,u_2,\ldots,u_N)}
{B(u_1,u_2,\ldots,u_N)}\,. %\label{e20}
\end{multline*}
\item \textit{Если функция $C(u_1,u_2,\ldots,u_N)\hm=
{A(u_1,u_2,\ldots,u_N)}/{B(u_1,u_2,\ldots,u_N)}$ ограничена сверху или снизу, 
но не достигает глобального экстремума на множестве $U\hm=U_1\times U_2\times\cdots
\times U_N$,
то для любого $\varepsilon\hm > 0$ можно выбрать $\varepsilon$-оп\-ти\-маль\-ную 
детерминированную стратегию управления полумарковским процессом~$\xi(t)$, 
которая определяется вырожденной
вероятностной мерой $\Psi^{*(+)}(\varepsilon)\hm\in \Gamma^*$ или вырожденной
вероятностной мерой $\Psi^{*(-)}(\varepsilon)\hm\in \Gamma^*$, в~зависимости от 
вида экстремума (максимума или минимума) в~задаче}~(\ref{e18}). 
\textit{При этом вероятностная мера $\Psi^{*(+)}(\varepsilon)\hm\in \Gamma^*$ может быть 
сосредоточена в~любой точке $\left(u_1^{(+)}(\varepsilon),u_2^{(+)}(\varepsilon),\ldots,
u_N^{(+)}(\varepsilon)\right)$, удовлетворяющей соотношению}:
\begin{multline}
\sup\limits_{(u_1,u_2,\ldots,u_N) \in U}
\fr{A(u_1,u_2,\ldots,u_N)}{B(u_1,u_2,\ldots,u_N)}-\varepsilon <{}\\
{}<
\fr{A\left(u_1^{(+)}(\varepsilon),u_2^{(+)}(\varepsilon),\ldots,u_N^{(+)}
(\varepsilon)\right)}
{B\left(u_1^{(+)}(\varepsilon),u_2^{(+)}(\varepsilon),\ldots,u_N^{(+)}
(\varepsilon)\right)}<{}\\
{}<\sup\limits_{(u_1,u_2,\ldots,u_N) \in U}
\fr{A(u_1,u_2,\ldots,u_N)}{B(u_1,u_2,\ldots,u_N)}<\infty\,, 
\label{e21}
\end{multline}
\textit{если функция $C(u_1,u_2,\ldots,u_N)$ ограничена сверху 
и~экстремальная задача}~(\ref{e18}) 
\textit{представляет собой задачу на максимум. Аналогично вероятностная мера 
$\Psi^{*(-)}(\varepsilon)\hm\in \Gamma^*$ может быть сосредоточена в~любой точке 
$\left(u_1^{(-)}(\varepsilon),u_2^{(-)}(\varepsilon),\ldots,u_N^{(-)}(\varepsilon)
\right)$, удовлетворяющей соотношению}:

\noindent
\begin{multline*}
-\infty<\inf\limits_{(u_1,u_2,\ldots,u_N) \in U}\fr{A(u_1,u_2,\ldots,u_N)}
{B(u_1,u_2,\ldots,u_N)} <{}\\
{}<
\fr{A\left(u_1^{(-)}(\varepsilon),u_2^{(-)}
(\varepsilon),\ldots,u_N^{(-)}(\varepsilon)\right)}
{B\left(u_1^{(-)}(\varepsilon),u_2^{(-)}(\varepsilon),\ldots,
u_N^{(-)}(\varepsilon)\right)}<{}\\
{}<\inf\limits_{(u_1,u_2,\ldots,u_N) \in U}
\fr{A(u_1,u_2,\ldots,u_N)}{B(u_1,u_2,\ldots,u_N)}+\varepsilon\,, 
%\label{e22}
\end{multline*}
\textit{если функция $C(u_1,u_2,\ldots,u_N)$ ограничена снизу и~экстремальная 
задача}~(\ref{e18})  \textit{представляет собой задачу на минимум}.
\item \textit{Если функция $C(u_1,u_2,\ldots,u_N)\hm=
{A(u_1,u_2,\ldots,u_N)}/{B(u_1,u_2,\ldots,u_N)}$ не ограничена сверху 
или снизу, то оптимальной стратегии управления в~смысле
соответствующей экстремальной задачи не существует. 
При этом найдется такая последовательность вырожденных вероятностных
мер~$\Psi^{*(+)}(n)$, сосредоточенных в~точках 
$\left(u_1^{(+)}(n),u_2^{(+)}(n),\ldots,u_N^{(+)}(n)\right)$, $n\hm=1,2,\dots $, 
для которых выполняется соотношение}:
\begin{multline*}
I\left(\Psi^*(n)\right)={}\\
{}=
I\left(\Psi_1^{*(+)}(n),\Psi_2^{*(+)}(n),\ldots,\Psi_N^{*(+)}(n)\right)={}\\
{}=\fr{A\left(u_1^{(+)}(n),u_2^{(+)}(n),\ldots,u_N^{(+)}(n)\right)}
{B\left(u_1^{(+)}(n),u_2^{(+)}(n),\ldots,u_N^{(+)}(n)\right)}\to 
\infty\\
\mbox{при}\ n\to\infty\,, 
%\label{e23}
\end{multline*}
\textit{если функция $C(u_1,u_2,\ldots,u_N)$ не ограничена сверху. 
Аналогично найдется такая последовательность вырожденных вероятностных
мер~$\Psi^{*(-)}(n)$, сосредоточенных в~точках 
$\left(u_1^{(-)}(n),u_2^{(-)}(n),\ldots,u_N^{(-)}(n)\right)$, 
$n\hm=1,2,\dots $, для которых выполняется соотношение}:
\begin{multline*}
I\left(\Psi^{*(-)}(n)\right)={}\\
{}= I
\left(\Psi_1^{*(-)}(n),\Psi_2^{*(-)}(n),\ldots,\Psi_N^{*(-)}(n)\right)={}\\
{}=\fr{A\left(u_1^{(-)}(n),u_2^{(-)}(n),\ldots,u_N^{(-)}(n)\right)}
{B\left(u_1^{(-)}(n),u_2^{(-)}(n),\ldots,u_N^{(-)}(n)\right)}\to 
-\infty\\
\mbox{при}~~n\to\infty\,,  
%\label{e24}
\end{multline*}
\textit{если функция $C(u_1,u_2,\ldots,u_N)$ не ограничена \mbox{снизу}}.
\end{enumerate}
\textit{При этом сформулированные утверждения каждого пункта теоремы~$2$ 
могут выполняться как по отдельности, для одного из двух
видов экстремума, так и~совместно, для обоих видов экстремума.}

\smallskip

Прежде чем непосредственно доказывать теорему~2, докажем некоторые 
вспомогательные утверждения.

\smallskip

\noindent
\textbf{Лемма~1.}\ 
\textit{Рассмотрим дроб\-но-ли\-ней\-ный интегральный функционал 
$I(\Psi_1, \Psi_2,\ldots, \Psi_N)$ вида}~(\ref{e11}), 
\textit{заданный на некотором множестве наборов вероятностных мер 
$\Psi\hm=(\Psi_1, \Psi_2,\ldots, \Psi_N)\hm \in \Gamma$. Предположим, что на 
множестве~$\Gamma$ выполняется условие~$1$ из набора предварительных условий 
и~функция $B(u_1, u_2,\ldots, u_N)$  обладает свойством строгой знакопостоянности 
при всех $(u_1, u_2,\ldots, u_N) \hm\in U$. Тогда справедливы следующие утверждения}:
\begin{enumerate}[1.]
\item \textit{Если основная функция 
$C(u_1, u_2,\ldots, u_N)\hm={A(u_1, u_2,\ldots, u_N)}/{B(u_1, u_2,\ldots, u_N)}$ 
ограничена сверху, т.\,е.\ выполняется условие}
\begin{multline}
C\left(u_1, u_2,\ldots, u_N\right)=
\fr{A(u_1, u_2,\ldots, u_N)}{B(u_1, u_2,\ldots, u_N)}\leq {}\\
{}\leq
c_0^{(+)}<\infty \,, \enskip \left(u_1, u_2,\ldots, u_N\right) \in U\,, \label{e25}
\end{multline}
\textit{то имеет место неравенство}:
\begin{equation}
I\left(\Psi_1, \Psi_2,\ldots, \Psi_N\right)\leq c_0^{(+)} 
\label{e26}
\end{equation}
\textit{для всех} $(\Psi_1, \Psi_2,\ldots, \Psi_N) \in \Gamma$.
\item \textit{Если основная функция 
$C(u_1, u_2,\ldots, u_N)\hm={A(u_1, u_2,\ldots, u_N)}/{B(u_1, u_2,\ldots, u_N)}$ 
ограничена снизу, т.\,е.\ выполняется условие}
\begin{multline*}
C\left(u_1, u_2,\ldots, u_N\right)=\fr{A(u_1, u_2,\ldots, u_N)}{B(u_1, u_2,\ldots, 
u_N)}\geq{}\\
{}\geq c_0^{(-)}>-\infty \,, 
\left(u_1, u_2,\ldots, u_N\right) \in U\,, 
%\label{e27}
\end{multline*}
\textit{то имеет место неравенство}:
\begin{equation*}
I\left(\Psi_1, \Psi_2,\ldots, \Psi_N\right)\geq c_0^{(-)} 
%\label{e28}
\end{equation*}
\textit{для всех} $(\Psi_1, \Psi_2,\ldots, \Psi_N) \hm\in \Gamma$.
\end{enumerate}

\noindent
Д\,о\,к\,а\,з\,а\,т\,е\,л\,ь\,с\,т\,в\,о\ \ леммы~1.\ 
Докажем первое утверждение леммы. Предположим сначала, 
что функция $B(u_1, u_2,\ldots,  u_N)$ строго положительна:
\begin{equation}
B\left(u_1, u_2,\ldots, u_N\right)>0\,,\enskip
\left(u_1, u_2,\ldots, u_N\right)\in U\,. \label{e29}
\end{equation}
Заметим, что в~таком случае по свойству интеграла~\cite[гл.~V]{18}
\begin{multline}
\hspace*{-2mm}\int\limits_{U_1}\!\!\cdots\! \!\int\limits_{U_N}\!\!B(u_1, \ldots,u_N) \,
d\Psi_1(u_1)%d\Psi_2(u_2)\cdots\\
\cdots d\Psi_N(u_N)>0 \!\!\!\!\label{e30}
\end{multline}
для любого фиксированного набора $\Psi\hm=(\Psi_1, \ldots, \Psi_N)\hm\in \Gamma$.
Из неравенства~(\ref{e25}) с~уче\-том~(\ref{e29}) получаем:
\begin{multline}
\hspace*{-4mm}A\left(u_1,\ldots, u_N\right)\leq{}\\
\hspace*{-4mm}{}\leq c_0^{(+)} B\left(u_1, \ldots, u_N\right)\,, 
\left(u_1, \ldots, u_N\right)\in U\,. \label{e31}
\end{multline}
В свою очередь, из неравенства~(\ref{e31}) и~свойств интеграла следует:
\begin{multline}
\int\limits_{U_1}\!\!\cdots\! \!\int\limits_{U_N}\!\!A(u_1,\ldots, u_N) \,
d\Psi_1\left(u_1\right)%d\Psi_2\left(u_2\right)\cdots\\
\cdots d\Psi_N\left(u_N\right)\leq\\
\hspace*{-24pt}\leq 
c_0^{(+)}\!\!\int\limits_{U_1}\!\!\cdots\!\! \int\limits_{U_N}\!\!\!B\!\left(u_1,\ldots, u_N\right)
 d\Psi_1\!\left(u_1\right)\!%d\Psi_2\left(u_2\right)\cdots\\
 \cdots d\Psi_N\!\left(u_N\right)\!\! 
 \label{e32}
\end{multline}
для любого фиксированного набора $\Psi\hm=(\Psi_1, \ldots, \Psi_N)\hm\in \Gamma$. 
Но тогда из~(\ref{e32}) с~учетом~(\ref{e30}) получаем:
\begin{multline}
I(\Psi_1, \ldots, \Psi_N)={}\\
{}=
\fr{\int\nolimits_{U_1}\!\cdots\! \int\nolimits_{U_N}\!\!A\left(u_1, \ldots, u_N\right)\,
 d\Psi_1(u_1)\cdots d\Psi_N(u_N)}{
\int\nolimits_{U_1}\!\cdots\! \int\nolimits_{U_N}\!\!B\left(u_1, \ldots, u_N\right)\,
 d\Psi_1(u_1)
 \cdots d\Psi_N(u_N)}\leq{}\\
 {}\leq c_0^{(+)} 
 \label{e33}
\end{multline}
для любого фиксированного набора $(\Psi_1, \ldots\linebreak\ldots, \Psi_N)\hm\in \Gamma$.

Предположим теперь, что функция $B(u_1,\ldots, u_N)$ строго отрицательна:
\begin{equation}
B(u_1,\ldots, u_N)<0 \quad \left(u_1, \ldots, u_N\right)\in U\,. 
\label{e34}
\end{equation}
Тогда
\begin{multline}
\hspace*{-6pt}\int\limits_{U_1}\!\!\cdots\!\! \int\limits_{U_N}\!\!B\!\left(u_1,\ldots, u_N\right)\!
 d\Psi_1(u_1) \cdots d\Psi_N(u_N)<0 \!\!\!
 \label{e35}
\end{multline}
для любого фиксированного набора $(\Psi_1, \ldots\linebreak \ldots, \Psi_N)\hm\in \Gamma$.

Как и~ранее, будем исходить из неравенства~(\ref{e25}). 
При выполнении условий~(\ref{e34}) и~(\ref{e35}) характер неравенств~(\ref{e31}) 
и~(\ref{e32}) меняется на противоположный, но характер неравенства~(\ref{e33}) 
остается неизменным. Таким образом, для любой функции 
$B(u_1, u_2,\ldots, u_N)$, обладающей свойством строгой знакопостоянности, 
из условия~(\ref{e25}) следует выполнение неравенства~(\ref{e33}), 
которое совпадает с~(\ref{e26}). Первое утверждение леммы~1 доказано. 
Второе утверждение доказывается аналогично. Лемма~1 доказана.

\smallskip

\noindent
\textbf{Лемма 2.} \textit{Рассмотрим дроб\-но-ли\-ней\-ный интегральный функционал 
$I(\Psi_1, \Psi_2,\ldots, \Psi_N)$ вида}~(\ref{e11}), 
\textit{заданный на некотором множестве наборов вероятностных мер 
$\Psi\hm=(\Psi_1, \Psi_2,\ldots, \Psi_N)\hm\in \Gamma$. Предпо\-ложим, что на 
множестве~$\Gamma$ выполняется условие~$1$ из набора предварительных условий 
и~функция $B(u_1, u_2,\ldots, u_N)$ обладает свойством строгой знакопостоянности 
при всех $(u_1, u_2,\ldots, u_N)\hm\in U$. Тогда справедливы следующие утверждения}:
\begin{enumerate}[1.]
\item \textit{Если основная функция $C(u_1, u_2,\ldots, u_N)\hm=
{A(u_1, u_2,\ldots, u_N)}/{B(u_1, u_2,\ldots, u_N)}$ ограничена сверху, 
но не достигает своего максимального 
значения, то имеет место неравенство}:
\begin{multline}
I\left(\Psi_1, \Psi_2,\ldots, \Psi_N\right)<{}\\
{}< \sup\limits_{(u_1, u_2,\ldots, u_N)\in U}
 C\left(u_1, u_2,\ldots, u_N\right)<\infty \label{e36}
\end{multline}
\textit{для всех} $(\Psi_1, \Psi_2,\ldots, \Psi_N)\in \Gamma$.
\item \textit{Если основная функция $C(u_1, u_2,\ldots, u_N)\hm=
{A(u_1, u_2,\ldots, u_N)}/{B(u_1, u_2,\ldots, u_N)}$ ограничена снизу, 
но не достигает своего минимального значения, то имеет место неравенство}:
\begin{multline*}
I\left(\Psi_1, \Psi_2,\ldots, \Psi_N\right)>{}\\
{}> \inf\limits_{(u_1, u_2,\ldots, u_N)\in U} 
C\left(u_1, u_2,\ldots, u_N\right)>-\infty 
%\label{e37}
\end{multline*}
\textit{для всех} $(\Psi_1, \Psi_2,\ldots, \Psi_N)\hm\in \Gamma$.
\end{enumerate}

\noindent
Д\,о\,к\,а\,з\,а\,т\,е\,л\,ь\,с\,т\,в\,о\ \ леммы~2. 
Докажем первое утверждение леммы. Поскольку множество значений 
основной функции $C(u_1, u_2,\ldots, u_N)$ ограничено сверху, оно имеет конечную 
верхнюю грань:
$$
\exists \sup\limits_{(u_1, u_2,\ldots, u_N)\in U} 
C\left(u_1, u_2,\ldots, u_N\right)<\infty
$$
(см.~\cite[гл.~1, \S3, п.~3.4, теорема~1]{25}).

По условию функция $C(u_1, u_2,\ldots, u_N)$ не достигает своего максимального 
значения. Следовательно, выполняется неравенство:
\begin{multline}
C(u_1, u_2,\ldots, u_N)=\fr{A(u_1, u_2,\ldots, u_N)}{B(u_1, u_2,\ldots, u_N)}<{}\\
{}< 
\sup\limits_{(u_1, u_2,\ldots, u_N)\in U} C(u_1, u_2,\ldots, u_N)<\infty\,, 
\\
\left(u_1, u_2,\ldots, u_N\right)\in U\,.
\label{e38}
\end{multline}
Взяв за основу строгое неравенство~(\ref{e38}), проведем рассуждения, аналогичные тем, 
которые были проведены в~лемме~1 по отношению к~неравенству~(\ref{e25}). 
В~результате получим строгое неравенство~(\ref{e36}).

Второе утверждение леммы~2 доказывается аналогично. Лемма~2 доказана.

\noindent
Д\,о\,к\,а\,з\,а\,т\,е\,л\,ь\,с\,т\,в\,о\ 
\ теоремы~2.
Начнем с~доказательства утверждения~1. Предположим сначала, что основная 
функция $C(u_1, u_2,\ldots, u_N)={A(u_1, u_2,\ldots, u_N)}/{B(u_1, u_2,\ldots, u_N)}$ 
ограничена сверху и~достигает глобального максимума на множестве~$U$ 
в~некоторой точке $u^{(+)}\hm=\left(u^{(+)}_1,u^{(+)}_2,\ldots,u^{(+)}_N\right)\hm\in U$,
а~именно:
\begin{multline*}
\max\limits_{(u_1, u_2,\ldots, u_N)\in U} C\left(u_1, u_2,\ldots, u_N\right) = {}\\
{}=
C\left(u^{(+)}_1,u^{(+)}_2,\ldots,u^{(+)}_N\right)<\infty\,.
\end{multline*}
Тогда выполняется соотношение:
\begin{multline}
C(u_1, u_2,\ldots, u_N)=\fr{A(u_1, u_2,\ldots, u_N)}{B(u_1, u_2,\ldots, u_N)}
\leq{}\\
{}\leq C\left(u^{(+)}_1,u^{(+)}_2,\ldots,u^{(+)}_N\right)<\infty\,, 
\\
\left(u_1, u_2,\ldots, u_N\right)\in U\,.
\label{e39}
\end{multline}
Условия леммы~1 выполнены, и~можно воспользоваться ее утверждениями. 
Согласно первому из них, если выполняется неравенство~(\ref{e39}), 
то имеет место соотношение:
\begin{equation*}
I(\Psi_1, \Psi_2,\ldots, \Psi_N)\leq 
C\left(u^{(+)}_1,u^{(+)}_2,\ldots,u^{(+)}_N\right)<\infty 
%\label{e40}
\end{equation*}
для всех стратегий управления $\Psi\hm=(\Psi_1, \Psi_2,\ldots\linebreak
\ldots, \Psi_N)\hm\in \Gamma$.

Таким образом, множество значений дроб\-но-ли\-ней\-но\-го интегрального 
функционала $I(\Psi_1, \Psi_2,\ldots, \Psi_N)$ ограничено сверху при всех 
$\Psi\hm=(\Psi_1, \Psi_2,\ldots, \Psi_N)\hm\in \Gamma$. Тогда существует верхняя 
грань этого множества и~выполняется неравенство:
\begin{multline}
\sup\limits_{(\Psi_1, \Psi_2,\ldots, \Psi_N)\in \Gamma} 
I\left(\Psi_1, \Psi_2,\ldots, \Psi_N\right)\leq {}\\
{}\leq
C\left(u^{(+)}_1,u^{(+)}_2,\ldots,u^{(+)}_N\right). \label{e41}
\end{multline}
Рассмотрим детерминированную стратегию управ\-ле\-ния 
$\Psi^{*(+)}\hm=\left(\Psi_1^{*(+)}, \Psi_2^{*(+)},\ldots, \Psi_N^{*(+)}\right)$, 
в~которой каждая вероятностная мера~$\Psi_i^{*(+)}$ является вы\-рож\-ден\-ной 
и~сосредоточена в~точке $u_i^{(+)}$, $i\hm=\overline{1, N}$.
По свойству интеграла
\begin{multline}
I\left(\Psi_1^{*(+)}, \Psi_2^{*(+)},\ldots ,\Psi_N^{*(+)}\right)={}\\
{}=
C\left(u^{(+)}_1,u^{(+)}_2,\ldots,u^{(+)}_N\right). \label{e42}
\end{multline}
Из соотношений~(\ref{e41}) и~(\ref{e42}) получаем:
\begin{multline}
\sup\limits_{(\Psi_1, \Psi_2,\ldots, \Psi_N)\in \Gamma} 
I\left(\Psi_1, \Psi_2,\ldots, \Psi_N\right)\leq{}\\
{}\leq
 I\left(\Psi_1^{*(+)}, 
\Psi_2^{*(+)},\ldots, \Psi_N^{*(+)}\right). \label{e43}
\end{multline}
Заметим дополнительно, что выполняются отношения принадлежности:
\begin{equation}
\Psi^{*(+)}=\left(\Psi_1^{*(+)}, \Psi_2^{*(+)},\ldots, \Psi_N^{*(+)}\right) 
\in \Gamma^* \subset \Gamma\,. \label{e44}
\end{equation}
Из~(\ref{e44}) и~свойства верхней грани следует:
\begin{multline}
\sup\limits_{\left(\Psi_1^{*}, \Psi_2^{*},\ldots, \Psi_N^{*}\right) \in \Gamma^*} 
I\left(\Psi_1^{*}, \Psi_2^{*},\ldots, \Psi_N^{*}\right)\leq {}\\
{}\leq
\sup\limits_{\left(\Psi_1, \Psi_2,\ldots, \Psi_N\right) 
\in \Gamma} I\left(\Psi_1, \Psi_2,\ldots, \Psi_N\right)\,. 
\label{e45}
\end{multline}
Объединяя~(\ref{e42}), (\ref{e43}) и~(\ref{e45}), получаем соотношение:
\begin{multline}
\sup\limits_{\left(\Psi_1^{*}, \Psi_2^{*},\ldots, \Psi_N^{*}\right) 
\in \Gamma^*} I\left(\Psi_1^{*}, \Psi_2^{*},\ldots, 
\Psi_N^{*}\right)\leq{}\\
{}\leq \sup\limits_{\left(\Psi_1, \Psi_2,\ldots, \Psi_N\right) 
\in \Gamma} I\left(\Psi_1, \Psi_2,\ldots, \Psi_N\right)\leq{}\\
{}\leq I\left(\Psi_1^{*(+)}, \Psi_2^{*(+)},\ldots, \Psi_N^{*(+)}\right)={}\\
{}=
\fr{A\left(u^{(+)}_1,u^{(+)}_2,\ldots,u^{(+)}_N\right)}{B\left(u^{(+)}_1,u^{(+)}_2,
\ldots,u^{(+)}_N\right)}\,.
 \label{e46}
\end{multline}
Из соотношения~(\ref{e46}) с~учетом~(\ref{e44}) получаем, что максимум 
функционала $I(\Psi_1, \Psi_2,\ldots, \Psi_N)$ на множестве допустимых стратегий 
$\Psi\hm=(\Psi_1, \Psi_2,\ldots, \Psi_N)\hm\in \Gamma$ существует и~достигается 
на детерминированной стратегии $\left(\Psi_1^{*(+)}, \Psi_2^{*(+)},\ldots, 
\Psi_N^{*(+)}\right)$.

Кроме того, выполняются соотношения~(\ref{e19}). Таким образом, утверждение~1 
в~случае, когда основная функция $C(u_1, u_2,\ldots, u_N)$ достигает глобального 
максимума, доказано. Соответствующее утверждение в~случае, когда основная функция 
$C(u_1, u_2,\ldots, u_N)$ достигает глобального минимума, доказывается аналогично. 
При этом используется второе утверждение леммы~1.

\smallskip

Перейдем к~доказательству второго утверждения теоремы~2. Предположим, что основная 
функция $C(u_1, u_2,\ldots, u_N)\hm=A(u_1, u_2,\ldots$\linebreak
$\ldots, u_N)/{B(u_1, u_2,\ldots, u_N)}$ 
ограничена сверху, но не достигает глобального максимума на множестве 
$U \hm= U_1 \times U_2 \times \cdots \times U_N$. Тогда множество значений 
основной функции имеет конечную верхнюю грань:

\noindent
\begin{multline*}
C\left(u_1, u_2,\ldots, u_N\right)=\fr{A(u_1, u_2,\ldots, u_N)}
{B(u_1, u_2,\ldots, u_N)}<{}\\
{}<
\sup\limits_{(u_1, u_2,\ldots, u_N)\in U} \fr{A(u_1, u_2,\ldots, u_N)}
{B(u_1, u_2,\ldots, u_N)}<\infty\,, 
\\
\left(u_1, u_2,\ldots, u_N\right)\in U\,.
%\label{e47}
\end{multline*}
По определению верхней грани для любого фиксированного $\varepsilon \hm>0$ 
существует точка $(u_1^{(+)}(\varepsilon), u_2^{(+)}(\varepsilon),\ldots, 
u_N^{(+)}(\varepsilon))$ такая, что выполняется двойное неравенство~(\ref{e21}) 
(см.~\cite[гл.~1, \S\,3, п.~3.4]{25}). Иначе говоря, значение основной функции 
в~указанной точке лежит в~левой \mbox{$\varepsilon$-окрест}\-ности верхней грани. 
Рассмотрим детерминированную стратегию управления 
$\Psi^{*(+)}(\varepsilon)\hm=\!\left(\Psi_1^{*(+)}(\varepsilon), 
\Psi_2^{*(+)}(\varepsilon),\ldots, \Psi_N^{*(+)}(\varepsilon)\!\right)$, компонентами\linebreak 
которой являются вырожденные вероятностные меры $\Psi_1^{*(+)}(\varepsilon), 
\Psi_2^{*(+)}(\varepsilon),\ldots, \Psi_N^{*(+)}(\varepsilon)$, причем вырожденная 
мера~$\Psi_i^{*(+)}(\varepsilon)$ сосредоточена в~точке~$u_i^{(+)}(\varepsilon)$,
$i\hm=1,2,\ldots,N$.

По свойству интеграла
\begin{multline}
I\left(\Psi_1^{*(+)}(\varepsilon), \Psi_2^{*(+)}(\varepsilon),\ldots,
 \Psi_N^{*(+)}(\varepsilon)\right)={}\\
 {}=
 C\left(u_1^{(+)}(\varepsilon), u_2^{(+)}(\varepsilon),\ldots, 
 u_N^{(+)}(\varepsilon)\right)\,. 
 \label{e48}
\end{multline}
Из соотношения~(\ref{e48}) с~учетом указанного свойства основной функции получаем:
\begin{multline}
\sup\limits_{(u_1, u_2,\ldots, u_N)\in U} \fr{A(u_1, u_2,\ldots, u_N)}
{B(u_1, u_2,\ldots, u_N)}-\varepsilon<{}\\
{}< I\left(\Psi_1^{*(+)}(\varepsilon), 
\Psi_2^{*(+)}(\varepsilon),\ldots, \Psi_N^{*(+)}(\varepsilon)\right)<{}
\\
{}< \sup\limits_{(u_1, u_2,\ldots, u_N)\in U} \fr{A(u_1, u_2,\ldots, u_N)}
{B(u_1, u_2,\ldots, u_N)}<\infty\,. 
\label{e49}
\end{multline}
Заметим также, что в~рассматриваемом случае выполнены условия леммы~2. 
Воспользуемся первым утверждением этой леммы, а~именно соотношением~(\ref{e36}):
\begin{multline}
I(\Psi_1, \Psi_2,\ldots, \Psi_N)< {}\\
{}<\sup\limits_{(u_1, u_2,\ldots, u_N)
\in U} \fr{A(u_1, u_2,\ldots, u_N)}{B(u_1, u_2,\ldots, u_N)}<\infty 
\label{e50}
\end{multline}
для всех $(\Psi_1, \Psi_2,\ldots, \Psi_N)\in\Gamma$.

Из соотношений~(\ref{e49}) и~(\ref{e50}) следует, что детерминированная стратегия 
$\Psi^{*(+)}(\varepsilon)\hm=\left(\Psi_1^{*(+)}(\varepsilon), \Psi_2^{*(+)}(\varepsilon),
\ldots, \Psi_N^{*(+)}(\varepsilon)\right)$, опре\-де\-ля\-емая набором вырожденных 
вероятностных мер, сосредоточенных в~соответствующих точках 
$\left(u_1^{(+)}(\varepsilon), u_2^{(+)}(\varepsilon),\ldots, 
u_N^{(+)}(\varepsilon)\right)$, является $\varepsilon$-оп\-ти\-маль\-ной. 
Вторая часть утверждения~2 теоремы~2, связанная со свойствами нижней грани, 
доказывается аналогично.

Докажем третье утверждение теоремы~2. Предположим, что множество значений 
основной функции $C(u_1, u_2,\ldots, u_N)\hm=
A(u_1, u_2,\ldots$\linebreak $\ldots, u_N)/{B(u_1, u_2,\ldots, u_N)}$
не является ограниченным сверху на множестве $U\hm=U_1\times U_2 \times \cdots $\linebreak
$\cdots \times U_N$.
Тогда существует последовательность\linebreak точек $\left(u_1^{(+)}(n), u_2^{(+)}(n),
\ldots,u_N^{(+)}(n)\right)\hm\in U$, $n\hm=1,2,\ldots$, для которой
\begin{multline}
C\left(u_1^{(+)}(n), u_2^{(+)}(n),\ldots,u_N^{(+)}(n)\right)={}\\
{}=
\fr{A\left(u_1^{(+)}(n), u_2^{(+)}(n),\ldots,u_N^{(+)}(n)\right)}
{B\left(u_1^{(+)}(n), u_2^{(+)}(n),\ldots,u_N^{(+)}(n)\right)}
\longrightarrow \infty \,,\\
n\rightarrow \infty\,.
\label{e51}
\end{multline}
Зафиксируем некоторую последовательность точек $\left(u_1^{(+)}(n), u_2^{(+)}(n),
\ldots,u_N^{(+)}(n)\right)\hm\in U$, $n\hm=1,2,\ldots$, обладающих указанным свойством, 
и~рассмотрим последовательность детерминированных  стратегий управления 
$\Psi^{*(+)}(n)\hm=\left(\Psi_1^{*(+)}(n), \Psi_2^{*(+)}(n),\ldots, 
\Psi_N^{*(+)}(n)\right)$, $n\hm=1,2,\ldots$, определяемых набором вырожденных 
вероятностных мер, сосредоточенных в~соответствующих точках 
$\left(u_1^{(+)}(n), u_2^{(+)}(n),\ldots,u_N^{(+)}(n)\right)$, $n\hm=1,2,\ldots$ 
По свойству интеграла для любого фиксированного значения $n=1,2,\ldots$ 
выполняется равенство:
\begin{multline}
I \left(\Psi^{*(+)}(n)\right)={}\\
{}=I\left(\Psi_1^{*(+)}(n), \Psi_2^{*(+)}(n),\ldots,
 \Psi_N^{*(+)}(n)\right)={}\\
{}=\fr{A\left(u_1^{(+)}(n), u_2^{(+)}(n),\ldots,u_N^{(+)}(n)\right)}
{B\left(u_1^{(+)}(n), u_2^{(+)}(n),\ldots,u_N^{(+)}(n)\right)}\,. 
\label{e52}
\end{multline}
Из соотношений~(\ref{e51}) и~(\ref{e52}) следует, что
\begin{multline}
I\left(\Psi^{*(+)}(n)\right)={}\\
{}=I\left(\Psi_1^{*(+)}(n), \Psi_2^{*(+)}(n),\ldots, 
\Psi_N^{*(+)}(n)\right)\longrightarrow\infty\,,\\ 
n \rightarrow\infty\,.
 \label{e53}
\end{multline}
Соотношение~(\ref{e53}) означает, что множество значе\-ний дроб\-но-ли\-ней\-но\-го 
интегрального функциона\-ла $I(\Psi_1, \Psi_2,\ldots, \Psi_N)$ вида~(\ref{e11}) 
не ограничено сверху\linebreak на множестве наборов вырожденных вероятностных мер 
$\left(\Psi_1^{*(+)}(n), \Psi_2^{*(+)}(n),\ldots, \Psi_N^{*(+)}(n)\right)\hm\in\Gamma^*$, 
а~следовательно, и~на более широком\linebreak множестве наборов вероятностных 
мер $(\Psi_1, \Psi_2,\ldots$\linebreak $\ldots, \Psi_N)\hm\in\Gamma$. В~таком случае решения экстремальной 
задачи~(\ref{e18}) в~форме задачи на максимум не существует. Соответствующее утвержде\-ние 
для варианта, когда множество значений основной функции $C(u_1, u_2,\ldots,u_N)
\hm=A(u_1, u_2,\ldots$\linebreak $\ldots,u_N)/{B(u_1, u_2,\ldots,u_N)}$ 
не является ограниченным снизу, доказывается аналогично. Третье утверж\-де\-ние теоремы~2 
доказано. Тем самым тео\-ре\-ма~2 доказана полностью.

\smallskip

Применим теорему~2 для решения поставленной задачи оптимального управления. 
Из утверждения этой теоремы следует, что для доказательства су-\linebreak ществования 
оптимального управ\-ле\-ния и~его нахождения необходимо исследовать на 
глобальный экстремум основную функцию дроб\-но-ли\-ней\-но\-го интегрального 
функционала $C(u_1,u_2,\ldots,u_N)$, определяемую формулой~(\ref{e17}) с~учетом 
равенств~(\ref{e12})--(\ref{e16}). В~некоторых случаях, например когда основной 
процесс~$\xi(t)$ является регенерирующим, а~стоимостные характеристики 
модели задаются линейными функциями, такое исследование можно провести 
аналитически. Однако для подавляющего большинства полумарковских моделей 
для этого необходимо использовать численные методы.

\section{Заключение}

В заключительной части работы приведем \mbox{краткое} описание теоретической 
основы метода решения задачи оптимального управления полумарковским 
процессом с~конечным множеством состояний.

\begin{enumerate}[1.]
\item Исходная проблема оптимального управления формулируется в~виде 
экстремальной задачи~(\ref{e18}). Целевым показателем качества управ\-ле\-ния в~данной задаче 
служит величина~(\ref{e10}), которая имеет характер средней удельной прибыли.
\item Доказывается, что стационарный показатель~(\ref{e10}) представим в~виде 
дроб\-но-ли\-ней\-но\-го интегрального функционала~(\ref{e11}), для которого явно 
определяются подынтегральные функции числителя и~знаменателя, а~следовательно, 
и~основная функция данного функционала.
\item Используется теорема об экстремуме дроб\-но-ли\-ней\-но\-го интегрального 
функционала. На основании утверждений этой теоремы уста\-нав\-ли\-ва\-ет\-ся, что 
исходная задача оптимального управления сводится к~исследованию на глобальный 
экстремум основной функции этого функционала, для которой получено явное 
аналитическое представление.
\end{enumerate}

Заметим, что такое исследование задач оптимального управления 
стохастическими системами фактически уже было проведено в~ряде работ П.\,В.~Шнуркова 
и~его соавторов. В~частности, в~работе~\cite{26} была рассмотрена модель 
управления для обрывающегося процесса восстановления, описывающего функционирование 
некоторой технической системы. Задача управления решалась для различных показателей 
эффективности и~надежности этой системы, имеющих структуру дроб\-но-ли\-ней\-но\-го 
интегрального функционала.

В работах~\cite{27, 28} рассматривались модели регенерирующих процессов 
для исследования сис\-тем управления запасами. Различные показатели качества 
управления были представлены в~форме дроб\-но-ли\-ней\-ных интегральных функционалов. 
Основные функции этих функционалов были найде\-ны в~явной форме и~исследовались 
на глобальный экстремум. В~работах~\cite{21,29} рассматривалась достаточно 
сложная полумарковская модель с~конечным множеством состояний, описывающая 
сис\-те\-му управления запасом непрерывного продукта. Показатели качества управления в~этой 
модели также имели структуру дроб\-но-ли\-ней\-ных интегральных функционалов, 
для основных функций которых были найдены явные аналитические представления. 
Упомянем также работы~\cite{30, 31}, в~которых была исследована полумарковская 
модель с~дис\-крет\-но-не\-пре\-рыв\-ным фазовым пространством. Показатели 
качества управления в~этой  модели были найдены в~явной форме как функции от 
двух непрерывных параметров управления.

Фактически во всех упомянутых работах уже был использован метод решения задачи 
оптимального управления регенерирующим или полумарковским случайным процессом, 
основанный на исследовании экстремальных свойств основной функции соответствующего 
дроб\-но-ли\-ней\-но\-го интегрального функционала. Из соображений, изложенных 
во\linebreak введении, следует, что в~период написания и~пуб\-ли\-кации этих работ данный метод 
не имел стро\-гого обоснования. Однако после публикации\linebreak работы~\cite{14} и~настоящего 
исследования можно утверж\-дать, что полученные в~них результаты полностью теоретически 
обоснованы.

Таким образом, изложенный выше метод решения проблемы оптимального управления 
полумарковскими процессами с~конечными множествами состояний может быть успешно 
реализован для многих задач, рассматриваемых в~различных областях прикладной 
теории вероятностей.

Практическая реализация численной процедуры поиска оптимального решения на примере\linebreak 
полумарковской модели управления запасом непрерывного продукта (подробнее 
см.~\cite{21, 29}), ба\-зи\-ру\-юща\-яся на изложенных выше результатах (в~частности, 
теореме~1), была осуществлена А.\,К.~Горшениным и~соавторами 
в~статье~\cite{Gorshenin2015}. Коротко опишем наиболее важные аспекты этой работы.

Для решения поставленной задачи опти\-мального управления была создана 
специальная программа \verb"Inventory" на встроенном языке программирования 
пакета \verb"MATLAB", ее возможности\linebreak кратко представле\-ны в~упомянутой ранее 
\mbox{статье}~\cite{Gorshenin2015}. В~программе \verb"Inventory" реализованы функции 
для оценивания через заданные исходные параметры вероятностных и~стоимостных 
характеристик модели, которые в~дальнейшем используются для поиска значений 
основной функции дроб\-но-ли\-ней\-но\-го функционала~(\ref{e17}). Точка глобального 
экстремума этой функции и~определяет оптимальное управление.

В качестве начальных данных необходимо задание следующих параметров:
\begin{itemize}
\item спрос и~вместимость склада;
\item разбиение множества значений объема запаса;
\item вероятностные характеристики, описывающие модель пополнения запаса;
\item условные математические ожидания длительностей задержек пополнения запаса;
\item функции для характеризации затрат и~доходов.
\end{itemize}

По итогам работы программы \verb"Inventory" ряд вспомогательных функций 
представляется в~аналитической форме (в частности, с~использованием аппарата 
символьных вычислений  \verb"Symbolic Toolbox"\linebreak пакета \verb"MATLAB"), выводится 
точка глобального экстремума функции нескольких вещественных переменных~(\ref{e17}), 
найденная с~помощью применения численных и~при\-бли\-жен\-но-ана\-ли\-ти\-че\-ских\linebreak 
аппроксимаций. 
Также формируются графики оценок значений ве\-ро\-ят\-ност\-но-сто\-и\-мост\-ных 
характеристик 
и~основной функции дроб\-но-ли\-ней\-но\-го функционала~(\ref{e17}), либо трехмерных 
сечений в~случае наличия более трех параметров управления (переменных).

Функциональность пакета \verb"Inventory" может быть расширена для практической 
реализации метода решения задачи поиска оптимального управ\-ле\-ния полумарковскими 
процессами с~конечными множествами состояний, рассмотренного в~данной статье.


 {\small\frenchspacing
 {%\baselineskip=10.8pt
 \addcontentsline{toc}{section}{References}
 \begin{thebibliography}{99}
 \bibitem{1}
\Au{Ховард Р.} Динамическое программирование и~марковские процессы~/ 
Пер. с~англ.~--- М.: Сов. радио, 1964. 189~с.
(\Au{Howard~R.\,A.} Dynamic programming and Markov processes.~--- 
Cambridge, MA, USA: MIT Press, 1960. 136~p.)
\bibitem{2} 
\Au{Рыков В.\,В.} Управляемые марковские процессы с~конечными пространствами 
состояний и~управлений~// Теория вероятностей и~ее применения, 1966. Т.~11. 
Вып.~2. С.~343--351.
\bibitem{3} 
\Au{Джевелл В.} Управляемые полумарковские процессы~// Кибернетич. сборник.~--- 
М.: Мир, 1967. Вып.~4. С.~97--134.
%{\em Jewell W.\,S.} Markov-renewal programming~// Operations Research, 1963. Vol.~11. P.~938--971.
\bibitem{4} 
\Au{Fox B.} Markov renewal programming by linear fractional programming~// 
SIAM J.~Appl. Math., 1966. Vol.~14. P.~1418--1432.
\bibitem{5} 
\Au{Denardo E.\,V.} Contraction mappings in the theory underlying dinamic programming~// 
SIAM Rev., 1967. Vol.~9. P.~165--177.

\bibitem{6} 
\Au{Howard R.\,A.} Research in semi-Markovian decision structures~// 
J.~Oper. Res. Soc. Japan, 1963. Vol.~6. P.~163--199.
\bibitem{7} 
\Au{Osaki S., Mine H.} Linear programming algorithms for Markovian decision processes~//
 J.~Math. Anal. Appl., 1968. Vol.~22. P.~356--381.
\bibitem{8} 
\Au{Майн Х., Осаки С.} Марковские процессы принятия решений~/ Пер. с~англ.~--- 
М.: Наука, 1977. 176~с.
(\Au{Mine~H., Osaki~S.} 
Markovian decision processes.~--- New York, NY, USA: 
American Elsevier Publishing Co., 1970. 142~p.)
\bibitem{9} 
\Au{Гихман И.\,И., Скороход А.\,В.} Управляемые случайные процессы.~--- 
Киев: Наукова думка, 1977. 251~с.
\bibitem{10} 
\Au{Luque-Vasquez F., Herndndez-Lerma~О.} Semi-Markov control models with average costs~// 
Appl. Math., 1999. Vol.~26. No.\,3. P.~315--331.
\bibitem{11} 
\Au{Vega-Amaya O., Luque-Vasquez~F.} Sample-path average cost optimality for 
semi-Markov control processes on Borel spaces: Unbounded costs and mean holding times~// 
Appl. Math., 2000. Vol.~27. No.\,3. P.~343--367.
\bibitem{12} 
Вопросы математической теории надежности~/ Под ред. Б.\,В. Гнеденко.~--- 
М.: Радио и~связь, 1983. 376~с.
\bibitem{13} 
\Au{Барзилович Е.\,Ю., Каштанов~В.\,А.} Некоторые математические вопросы теории 
обслуживания сложных систем.~---  М.: Сов. радио, 1971. 272~с.
\bibitem{14} 
\Au{Шнурков П.\,В.} О~решении проблемы безусловного экстремума для 
дроб\-но-ли\-ней\-но\-го интегрального функционала на множестве вероятностных мер~// 
Докл. РАН. Сер. Математика, 2016. Т.~470. №\,4. C.~387--392.
\bibitem{15} 
\Au{Ширяев А.\,Н.}  Вероятность.~--- М.:~МЦНМО, 2011. Кн.~1. 552~с. Кн.~2. 968~с.
\bibitem{16} 
\Au{Боровков А.\,А.} Теория вероятностей.~--- М.: Либроком, 2009. 656~c.
\bibitem{17} 
\Au{Хеннекен П.\,Л., Тортра А.} Теория вероятностей 
и~некоторые ее приложения.~--- М.: Наука, 1974. 472~c.
\bibitem{18} 
\Au{Халмош П.} Теория меры~/ Пер. с~англ.~--- М.: ИЛ, 1953. 282~c.
(\Au{Halmos~P.} Measure theory.~--- Litton Educational Publishing, Inc. 1950. 304~p.)
\bibitem{19} 
\Au{Королюк В.\,С., Турбин~А.\,Ф.} Полумарковские процессы и~их приложения.~--- 
Киев:~Наукова думка, 1976. 184~с.
\bibitem{20} 
\Au{Janssen J., Manca R.} Applied semi-Markov processes.~--- New York,
NY, USA: Springer, 2006. 309~p.
\bibitem{21} 
\Au{Шнурков П.\,В., Иванов~А.\,В.} Анализ дискретной полумарковской модели
 управления запасом непрерывного продукта при периодическом прекращении потребления~// 
 Дискретная математика, 2014. Т.~26. Вып.~1. С.~143--154.
\bibitem{22} 
\Au{Иванов~А.\,В.} Анализ дискретной полумарковской модели
 управления запасом непрерывного продукта при периодическом прекращении 
 потребления.~--- М.: НИУ ВШЭ, 2014.  Дисс.\ \ldots\ канд. физ.-мат. наук. 120~с.
\bibitem{23}  %23
\Au{Bajalinov~E.\,B.} Linear-fractional programming. 
Theory, methods, applications and software.~--- 
Boston/\linebreak Dordrecht/London: Kluwer Academic Publs., 2003. 423~p.

\bibitem{27} %27
\Au{Шнурков П.\,В., Мельников~Р.\,В.} Оптимальное управление запасом 
непрерывного продукта в~модели регенерации~// Обозрение прикладной 
и~промышленной математики, 2006. Т.~13. Вып.~3. С.~434--452.
\bibitem{28} 
\Au{Шнурков П.\,В., Мельников~Р.\,В.} 
Исследование проб\-ле\-мы управления запасом непрерывного продукта при детерминированной 
задержке поставки~// Автоматика и~телемеханика, 2008. Т.~10. С.~93--113.


\bibitem{24}  %26
\Au{Шнурков П.\,В.} Методы исследования задач оптимального обслуживания 
в~математической теории надежности.~--- 
М.: МИЭМ, 1983.  Дисс.\ \ldots\ канд. физ.-мат. наук.

 \bibitem{25}  %25
\Au{Кудрявцев Л.\,Д.} Курс математического анализа. Т.~1.~--- 
М.: Дрофа, 2006. 704~с.

\bibitem{26} %24
\Au{Шнурков П.\,В.} Оптимальное обслуживание на периоде 
до первого отказа системы~// Применение аналитических методов в~вероятностных
 задачах.~--- Киев: Институт математики АН УССР, 1986. С.~121--129.

\bibitem{29} 
\Au{Шнурков П.\,В., Иванов~А.\,В.} Исследование задачи оптимизации в~дискретной 
полумарковской модели управления непрерывным запасом~// Вестник МГТУ им.\ 
Н.\,Э. Баумана. Сер.\ Естественные науки, 2013. Т.~3. Вып.~50. С.~62--87.
\bibitem{30} 
\Au{Shnourkoff P.\,V.} The two-element system with one 
restoring device optimum maintenance~// Stoch. Anal. Appl., 1997. 
Vol.~15. No.\,5. P.~823--837.
\bibitem{31} 
\Au{Shnourkoff P.\,V.} The two-element system optimum maintenance tills the first fail~// 
Stoch. Anal. Appl., 2001. Vol.~19. No.\,6. P.~1005--1024.
\bibitem{Gorshenin2015} 
\Au{Gorshenin~A.\,K., Belousov~V.\,V., Shnourkoff~P.\,V.,
Ivanov~A.\,V.} Numerical research of the optimal control problem in the semi-Markov 
inventory model~// AIP Conference Proceedings, 2015. Vol.~1648. {250007}. 4~p.
%\bibitem{33} {\em Горшенин А.\,К., Белоусов В.\,В., Шнурков П.\,В.} 2016. Система управления запасами на основе стохастических полумарковских моделей. Свидетельство о государственной регистрации программы для ЭВМ \textnumero 2016619021.
 \end{thebibliography}

 }
 }

\end{multicols}

\vspace*{-6pt}

\hfill{\small\textit{Поступила в~редакцию 15.07.16}}

%\vspace*{8pt}

\newpage

\vspace*{-24pt}

%\hrule

%\vspace*{2pt}

%\hrule

%\vspace*{8pt}


\def\tit{ANALYTICAL SOLUTION OF~THE~OPTIMAL CONTROL TASK OF~A~SEMI-MARKOV 
PROCESS WITH~FINITE SET OF~STATES}

\def\titkol{Analytical solution of~the~optimal control task of~a~semi-Markov 
process with~finite set of~states}

\def\aut{P.\,V.~Shnurkov$^{1}$, A.\,K.~Gorshenin$^{2}$, and~V.\,V.~Belousov$^{2}$}

\def\autkol{P.\,V.~Shnurkov, A.\,K.~Gorshenin, and~V.\,V.~Belousov}

\titel{\tit}{\aut}{\autkol}{\titkol}

\vspace*{-9pt}


    
\noindent
$^1$National Research University Higher School of Economics, 34~Tallinskaya Str., 
Moscow, 123458, Russian\linebreak
$\hphantom{^9}$Federation

\noindent
$^2$Institute of Informatics Problems, Federal Research Center 
``Computer Science and Control'' of the Russian\linebreak
$\hphantom{^9}$Academy of Sciences, 44-2~Vavilova Str., 
Moscow 119333, Russian Federation



\def\leftfootline{\small{\textbf{\thepage}
\hfill INFORMATIKA I EE PRIMENENIYA~--- INFORMATICS AND
APPLICATIONS\ \ \ 2016\ \ \ volume~10\ \ \ issue\ 4}
}%
 \def\rightfootline{\small{INFORMATIKA I EE PRIMENENIYA~---
INFORMATICS AND APPLICATIONS\ \ \ 2016\ \ \ volume~10\ \ \ issue\ 4
\hfill \textbf{\thepage}}}

\vspace*{3pt}


\Abste{The theoretical verification of the new method of finding 
the optimal strategy of control of a~semi-Markov process with finite set of states is 
presented. The paper considers Markov randomized strategies of control, determined by 
a~finite collection of probability measures, corresponding to each state. The quality 
characteristic is the stationary cost index. This index is a~linear-fractional integral 
functional, depending on collection of probability measures, giving the strategy of control. 
Explicit analytical forms of integrands of numerator and denominator of this 
linear-fractional integral functional are known. The basis of consequent results is 
the new generalized and strengthened form of the theorem about an extremum of 
a~linear-fractional integral functional. It is proved that problems of existence 
of an optimal control strategy of a~semi-Markov process and finding this strategy 
can be reduced to the task of numerical analysis of global extremum for 
the given function, depending on finite number of real arguments.}

\KWE{optimal control of a~semi-Markov process; stationary cost index of quality control; 
linear-fractional integral functional}




\DOI{10.14357/19922264160408} 

\vspace*{-16pt}

\Ack
\noindent
The research was partially supported by the Russian Foundation 
for Basic Research (project 15-07-05316).



%\vspace*{3pt}

  \begin{multicols}{2}

\renewcommand{\bibname}{\protect\rmfamily References}
%\renewcommand{\bibname}{\large\protect\rm References}

{\small\frenchspacing
 {%\baselineskip=10.8pt
 \addcontentsline{toc}{section}{References}
 \begin{thebibliography}{99}
\bibitem{1-1}
\Aue{Howard,~R.\,A.} 1960. \textit{Dynamic programming and Markov processes}. 
Cambridge, MA: MIT Press. 136~p.
\bibitem{2-1}
\Aue{Rykov,~V.\,V.} 1966. Upravlyaemye markovskie protsessy 
s~konechnymi prostranstvami sostoyaniy i~upravleniy 
[Controlled Markov processes with finite spaces of states and controls ]. 
\textit{Teoriya veroyatnostey i~ee primeneniya} 
[Theory of Probability and Its Applications] 11(2):343--351.
\bibitem{3-1}
\Aue{Jewell,~W.\,S.} 1963. Markov-renewal programming. 
\textit{Oper. Res.} 11:938--971.
\bibitem{4-1}
\Aue{Fox,~B.} 1966. Markov renewal programming by linear fractional programming. 
\textit{SIAM J.~Appl. Math.} 14:1418--1432.
\bibitem{5-1}
\Aue{Denardo, E.\,V.} 1967. Contraction mappings in the theory underlying dinamic 
programming. \textit{SIAM Rev.} 9:165--177.
\bibitem{6-1}
\Aue{Howard,~R.\,A.} 1963. Research in semi-Markovian decision structures. 
\textit{J.~Oper. Res. Soc. Japan} 6:163--199.
\bibitem{7-1}
\Aue{Osaki,~S., and H.~Mine.} 1968. Linear programming algorithms 
for Markovian decision processes. \textit{J.~Math. Anal. Appl.} 22:356--381.
\bibitem{8-1}
\Aue{Mine,~H., and S.~Osaki.} 1970. 
\textit{Markovian decision processes}. New York, NY: Elsevier. 142~p.
\bibitem{9-1}
\Aue{Gikhman,~I.\,I., and A.\,V.~Skorokhod.} 1977. 
\textit{Upravlyaemye sluchaynye protsessy} 
[Controlled random processes]. Kiev: Naukova Dumka. 251~p.
\bibitem{10-1}
\Aue{Luque-Vasquez,~F., and О.~Herndndez-Lerma.} 1999. 
Semi-Markov control models with average costs. \textit{Appl. Math.} 26(3):315--331.
\bibitem{11-1}
\Aue{Vega-Amaya,~O., and  F.~Luque-Vasquez.} 2000.  
Sample-path average cost optimality for semi-Markov control processes on Borel spaces: 
Unbounded costs and mean holding times. \textit{Appl. Math.} 27(3):343--367.
\bibitem{12-1}
Gnedenko,~B.~V., ed. 1983. 
\textit{Voprosy matematicheskoy teorii nadezhnosti} 
[Problems of the mathematical theory of reliability].  Moscow: Radio i~svyaz'. 376~p.
\bibitem{13-1}
\Aue{Barzilovich,~E.\,Yu., and V.\,A.~Kashtanov.} 1971. 
\textit{Nekotorye matematicheskie voprosy teorii obsluzhivaniya slozhnykh sistem}  
[Some mathematical questions in theory of complex systems maintenance]. 
Moscow: Sovetskoe radio. 272~p.
\bibitem{14-1}
\Aue{Shnurkov,~P.\,V.} 2016. Solution of the unconditional extremum problem for 
a~linear-fractional 
integral functional on a~set of probability measures. 
\textit{Dokl. Math.} 94(2):550--554.
\bibitem{15-1} %14
\Aue{Shiryaev,~A.\,N.} 2016. 
\textit{Probability-1}. Graduate texts in mathematics ser.
New York, NY: Springer. Vol.~95. 503~p.;
2017. \textit{Probability-2.} Vol.~900. 500~p.
\bibitem{16-1}
\Aue{Borovkov,~А.\,А.} 2009. 
\textit{Teoriya veroyatnostey} [Probability theory]. Moscow: Librokom. 656~p.
\bibitem{17-1}
\Aue{Khenneken,~P.\,L., and A.~Tortra.} 1974. 
\textit{Teoriya veroyatnostey i~nekotorye ee prilozheniya} 
[Probability theory and some of its applications]. Moscow: Nauka. 472~p.
\bibitem{18-1}
\Aue{Halmos,~P.} 1950. \textit{Measure theory}. Litton Educational Publishing. 304~p.
\bibitem{19-1}
\Aue{Korolyuk, V.\,S., and A.\,F.~Turbin.} 1976. 
\textit{Polumarkovskie protsessy i~ikh prilozheniya} 
[Semi-Markov processes and their applications]. Kiev: Naukova Dumka. 184~p.
\bibitem{20-1}
\Aue{Janssen,~J., and R.~Manca.} 2006. 
\textit{Applied semi-Markov processes}. New York, NY: Springer. 309~p.
\bibitem{21-1}
\Aue{Shnurkov,~P.\,V, and A.\,V~Ivanov.} 2015. Analysis of a~discrete semi-Markov model of continuous inventory 
control with periodic interruptions of consumption. 
\textit{Discrete Math. \mbox{Appl}.} 25(1):59--67.
\bibitem{22-1} %21
\Aue{Ivanov,~A.\,V.} 2014. Analiz diskretnoy polumarkovskoy modeli upravleniya 
zapasom nepreryvnogo produkta pri periodicheskom prekrashchenii potrebleniya 
[Analysis of a~discrete semi-Markov control model of continuous product inventory 
in a~periodic cessation of consumption].  
Moscow: Natsional'nyy Issledovatel'skiy Universitet ``Vysshaya Shkola Ekonomiki.'' 
PhD Thesis. 120~p.
\bibitem{23-1} %22
\Aue{Bajalinov,~E.\,B.} 2003. 
\textit{Linear-fractional programming. Theory, methods, applications and software}. 
Boston/\linebreak Dordrecht/London: Kluwer Academic Publs. 423~p.
\bibitem{26-1} %24
\Aue{Shnurkov,~P.\,V., and R.\,V.~Mel'nikov.} 2006. Optimal'noe upravlenie 
zapasom nepreryvnogo produkta v modeli regeneratsii [Optimal control of 
a~continuous product inventory in the regeneration model]. 
\textit{Obozrenie prikladnoy i~promyshlennoy matematiki} [Rev. Appl. Ind. Math.]
13(3):434--452.

\bibitem{25-1} %25
\Aue{Shnurkov,~P.\,V., and R.\,V.~Mel'nikov.} 2008. 
Analysis of the problem of continuous-product inventory control under deterministic 
lead time. \textit{Automat. Rem. Contr.} 69(10):1734--1751.

\columnbreak

\bibitem{24-1} %26
\Aue{Shnurkov,~P.\,V.} 1983. Metody issledovaniya zadach optimal'nogo obsluzhivaniya 
v~matematicheskoy teorii nadezhnosti [Research methods of optimal service problems 
in the mathematical theory of reliability].  
Moscow: Moskovskiy Institut Elektronnogo Mashinostroeniya.  PhD Thesis. 


\bibitem{27-1} %27
\Aue{Kudryavtsev,~L.\,D.} 2006. 
\textit{Kurs matematicheskogo analiza} 
[A~course of mathematical analysis]. Vol.~1. Moscow: Drofa. 704~p.

\bibitem{28-1}
\Aue{Shnurkov,~P.\,V.} 1986. Optimal'noe obsluzhivanie na periode do 
pervogo otkaza sistemy [The optimum service period until the first system failure]. 
\textit{Primenenie analiticheskikh metodov v~veroyatnostnykh zadachakh} 
[The application of analytical methods in probabilistic tasks]. Kiev:
Institute of Mathematics of the Academy of Sciences of the USSR. 121--129.

\bibitem{29-1}
\Aue{Shnurkov,~P.\,V., and A.\,V.~Ivanov.} 2013. Issledovanie zadachi optimizatsii 
v~diskretnoy polumarkovskoy modeli upravleniya nepreryvnym zapasom 
[Study of the optimization problem in discrete semi-Markov model of continuous 
inventory control]. \textit{Vestnik MGTU im.\ N.\,E.~Baumana. Ser. 
Estestvennye nauki} [Vestnik of MSTU named after N.\,E.~Bauman. Ser. Natural sciences] 
3(50):62--87.
\bibitem{30-1}
\Aue{Shnourkoff,~P.\,V.} 1997. The two-element system with one restoring device 
optimum maintenance.  \textit{Stoch. Anal. Appl.} 15(5):823--837.
\bibitem{31-1}
\Aue{Shnourkoff,~P.\,V.} 2001. The two-element system optimum maintenance tills 
the first fail. \textit{Stoch. Anal. Appl.} 19(6):1005--1024.
\bibitem{32-1}
\Aue{Gorshenin,~A.\,K., V.\,V.~Belousov, P.\,V.~Shnourkoff, and A.\,V.~Ivanov.}
2015. Numerical research of the optimal control problem in the semi-Markov 
inventory model. \textit{AIP Conference Proceedings} 1648:250007.
\end{thebibliography}

 }
 }

\end{multicols}

\vspace*{-3pt}

\hfill{\small\textit{Received July 15, 2016}}

\Contr

\noindent
\textbf{Shnurkov Peter V.} (b.\ 1953)~---
 Candidate of Science (PhD) in physics and mathematics, 
 associate professor, National Research University Higher School of Economics, 
 34~Tallinskaya Str., Moscow 123458, Russian Federation; \mbox{pshnurkov@hse.ru} 
 
 \vspace*{3pt}
 
 \noindent
\textbf{Gorshenin Andrey K.}  (b.\ 1986)~---
Candidate of Science (PhD) in physics and mathematics, leading scientist, 
Institute of Informatics Problems, Federal Research Center ``Computer Science 
and Control'' of the Russian Academy of Sciences, 44-2~Vavilov Str., Moscow 119333, 
Russian Federation; associate professor, Federal State Budget Educational 
Institution of Higher Education ``Moscow Technological University,'' 
78~Vernadskogo Ave., Moscow 119454, Russian Federation;
\mbox{agorshenin@frccsc.ru}

\vspace*{3pt}

\noindent
\textbf{Belousov Vasiliy V.} (b.\ 1977)~---
Candidate of Science (PhD) in technology, senior scientist, Institute of 
Informatics Problems, Federal Research Center ``Computer Science and Control'' 
of the Russian Academy of Sciences, 44-2~Vavilov Str., Moscow 119333, Russian 
Federation; \mbox{VBelousov@ipiran.ru}
\label{end\stat}


\renewcommand{\bibname}{\protect\rm Литература}   %10
\def\stat{sinits}

\def\tit{АНАЛИТИЧЕСКОЕ МОДЕЛИРОВАНИЕ  РАСПРЕДЕЛЕНИЙ С~ИНВАРИАНТНОЙ
МЕРОЙ В~СТОХАСТИЧЕСКИХ СИСТЕМАХ С~РАЗРЫВНЫМИ ХАРАКТЕРИСТИКАМИ$^*$}

\def\titkol{Аналитическое моделирование  распределений с~инвариантной
мерой в~стохастических системах} % с~разрывными характеристиками}

\def\autkol{И.\,Н.~Синицын}

\def\aut{И.\,Н.~Синицын$^1$}

\titel{\tit}{\aut}{\autkol}{\titkol}

{\renewcommand{\thefootnote}{\fnsymbol{footnote}}\footnotetext[1]
{Работа выполнена при финансовой поддержке РФФИ
(проект №\,13-07-00036) и программой <<Интеллектуальные информационные 
технологии, системный анализ и автоматизация>> (проект~1.7).}}

\renewcommand{\thefootnote}{\arabic{footnote}}
\footnotetext[1]{Институт проблем информатики Российской академии наук, sinitsin@dol.ru}



\Abst{На базе методов нормальной аппроксимации и статистической линеаризации разработаны 
точные и приближенные алгоритмы аналитического моделирования плотностей стохастических 
режимов с инвариантной мерой в гауссовых и негауссовых стохастических системах (СтС)
с разрывными 
характеристиками. Рассмотрены особенности моделирования в СтС с 
пуассоновскими шумами. На тестовых примерах показана достаточная для многих приложений 
точность алгоритмов.}

\KW{автокоррелированная помеха; аналитическое моделирование;
интегродифференциальные уравнения Пугачёва; метод нормальной аппроксимации;
метод статистической линеаризации; нелинейная гауссовская и негауссовская стохастическая система в смысле Ито;
пуассоновская стохастическая сис\-те\-ма; распределение с инвариантной мерой;
стохастический режим}

\vskip 14pt plus 9pt minus 6pt

      \thispagestyle{headings}

      \begin{multicols}{2}

            \label{st\stat}



\section{Введение}

Следуя [1, 2], будем рассматривать нестационарный стохастических режим $Z\hm=Z(t)$ 
в нелинейной дифференциальной СтС, понимаемой в смысле Ито:
    \begin{equation}
    \dot Z = a(Z,t) + b (Z,t) V\,, \enskip Z(t_0) = Z_0\,.
    \label{e1.1-sin}
    \end{equation}
Здесь $Z$~--- $k$-мер\-ный вектор состояния СтС, $Z\hm\in \Delta$ ($\Delta$~--- 
многообразие состояний); $a\hm=a(Z,t)$ и $b\hm= b(Z,t)$~--- детерминированные  
$(k\times 1)$- и $(k\times m)$-мер\-ные  функции  отмеченных аргументов; 
$V\hm=V(t)$~--- $m$-мер\-ный вектор негауссовских (в общем случае) белых шумов 
с нулевыми математическими ожиданиями и представляющий собой среднеквадратичную 
(с.к.)\ производную процесса с независимыми приращениями  $W\hm=W(t)$, 
$V\hm=\dot W$. Обозначим через $\chi\hm=\chi(\mu;t)$ логарифмическую производную 
одномерной характеристической функции $h_1\hm=h_1(\mu;t)$ процесса $W\hm=W(t)$, определяемую формулой
    \begin{equation}
    \chi(\mu;t)=\fr{\prt \ln h_1 (\mu;t)}{\prt t}=
    \fr{1}{h_1(\mu;t)}\,\fr{\prt h_1(\mu;t)}{\prt t}\,.
    \label{e1.2-sin}
    \end{equation}

Начальное состояние $Z_0$ будем считать случайной величиной (СВ), не зависящей 
от $W(t)$ для $t\hm>t_0$. Предположим, что стохастический режим $Z(t)$ является 
сильным решением~(\ref{e1.1-sin}), а функции $a,b$ и $\chi$ удовлетворяют известным 
условиям существования и единственности~[1, 2].

Пусть существуют одно- и $n$-мерные плот\-ности\linebreak $f_1\hm=f_1(z;t)$ и 
$f_n\hm= f_n(z_1\tr z_n; t_1 \tr t_n)$ и характеристические функции $g_1\hm=g_1(\la;t)$ и 
$g_n\hm=g_n(\la_1\tr \la_n; t_1\tr t_n)$ $(n\hm\ge 2)$, удовлетворяющие 
интегродифференциальным уравнениям\linebreak Пугачева~[1, 2]:
    \begin{multline}
    \fr{\prt f_1(z;t)}{\prt t}+\fr{\prt^{\mathrm{T}}}{\prt z}\lk a(z,t)f_1(z;t)\rk = 
\fr{1}{(2 \pi)^k} \times{}\\
{}\times \iin\iin \chi(b(\xi,t)^{\mathrm{T}}\la;t) e^{i\la^{\mathrm{T}}(\xi-z)} f_1(z;t) \,
    d\xi d\la\,;
    \label{e1.3-sin}
    \end{multline}
   \begin{equation}
    f_1(z;t_0)=f_0(z)\,;\label{e1.4-sin}
    \end{equation}
    
    \vspace*{-12pt}
    
    \begin{multline*}
\fr{\prt f_n(z_1\tr z_n;t_1\tr t_n)}{\prt t_n}+{}\\
{}+\fr{\prt^{\mathrm{T}}}{\prt z_n}\left[
a(z_n, t_n) f_n (z_1\tr z_n; t_1\tr t_n)\right]={}\\
{}= \fr{1}{(2\pi)^{kn}} \iin\iin \chi(b(\xi_n, t_n)^{\mathrm{T}} \la_n;t_n) \times{}\\
{}\times \exp\lf i \sss_{l=1}^n \la_l^{\mathrm{T}} (\xi_l-z_l)\rf \times{}\\
{}\times
f_n (\xi_1\tr \xi_n; t_1\tr t_n)\,d\xi_1\cdots d\xi_n d\la_1\cdots d\la_n\,;
%\label{e1.5-sin}
\end{multline*}

\vspace*{-12pt}

\begin{multline*}
f_n(z_1\tr z_{n-1},z_n;t_1\tr t_{n-1},t_{n})={}\\
{}= f_{n-1} (z_1\tr z_{n-1};t_1\tr t_{n-1})\delta (z_n - z_{n-1})\,;
%\label{e1.6-sin}
\end{multline*}
       
        
\noindent        
\begin{multline}
\fr{\prt g_1 (\la;t)}{\prt t} -{}\\
{}-\fr{1}{(2\pi)^k} \iin \iin i\la^{\mathrm{T}} a (z,t) 
e^{i(\la^{\mathrm{T}} -\mu^{\mathrm{T}})z} g_1 (\mu;t)\, d\mu dz={}\\
{}=\fr{1}{(2\pi)^k} \iin \iin \chi(b(z,t)^{\mathrm{T}} \la^{\mathrm{T}};t) 
e^{i(\la^{\mathrm{T}} -\mu^{\mathrm{T}})z} \times{}\\
{}\times
g_1 (\mu;t)\, d\mu dz\,;
\label{e1.7-sin}
\end{multline}
\begin{equation}
g_1(\la;t_0) = g_0(\la)\,\,; \label{e1.8-sin}
\end{equation}

\vspace*{-12pt}

\begin{multline*}
\fr{\prt g_n (\la_1\tr \la_n; t_1\tr t_n)}{\prt t_n} -{}\\
{}-
\fr{1}{(2\pi)^{kn}} \iin \cdots \iin i\la^{\mathrm{T}} a (z_n,t_n) \times{}\\
{}\times \exp \lk i \sss\limits_{k=1}^n (\la_k^{\mathrm{T}} - \mu_k^{\mathrm{T}}) z_k\rk \times{}\\
{}\times g_n 
(\mu_1\tr \mu_n; t_1\tr t_n)\, d\mu_1 \cdots d \mu_n dz_1\cdots dz_n={}\\
{}= \fr{1}{(2\pi)^{kn}} \iin\cdots \iin \chi (b(z_n;t)^{\mathrm{T}} \la_n;t_n)\times{}\\
{}\times\exp \lk i \sss_{k=1}^n (\la_k^{\mathrm{T}} - \mu_k^{\mathrm{T}}) z_k\rk \times{}\\
{}\times g_n 
(\mu_1\tr \mu_n; t_1\tr t_n) \,d\mu_1 \cdots d \mu_n dz_1\cdots dz_n;
%\label{e1.9-sin}
\end{multline*}

\vspace*{-12pt}

\noindent
\begin{multline*}
g_n (\la_1\tr \la_n; t_1\tr t_{n-1},t_{n-1})= {}\\
{}=
g_{n-1} (\la_1\tr \la_{n-2},\la_{n-1}+\la_n; t_1\tr t_{n-1})\,, %\label{e1.10-sin}
\end{multline*}
 $$
        t_1\le t_2 \le \cdots \le t_n,\enskip n=2,3,\ldots
        $$

При этом одно- и $n$-мер\-ные плотности и характеристические функции связаны 
между собой известными соотношениями:
\begin{equation*}
f_1(z;t) = \fr{1}{(2\pi)^{k}} \iin e^{-i\mu^{\mathrm{T}} z} g_1(\mu;t) d\mu\,; %\label{e1.11-sin}
    \end{equation*}
      \begin{equation*}
   g_1(\la;t) = \iin e^{i\la^{\mathrm{T}} z} f_1(z;t)\, dz\,; %\label{e1.12-sin}
   \end{equation*}
   
   \vspace*{-12pt}

\noindent
\begin{multline}
f_n( z_1\tr z_n; t_1\tr t_n) ={}\\
{}=
\fr{1}{(2\pi)^{kn}} 
\iin\cdots \iin \exp \lf - i \sss_{l=1}^n \la_l^{\mathrm{T}} z_l\rf \times{}\\
{}\times g_n (\la_1\tr \la_n; t_1\tr t_n)\, d\la_1\cdots d\la_n\,;\label{e1.13-sin}
\end{multline}


\vspace*{-12pt}

\noindent
\begin{multline*}
g_n (\la_1\tr \la_n; t_1\tr t_n) ={}\\
{}=\iin\cdots \iin \exp\lf i \sss_{l=1}^n \la_l^{\mathrm{T}} z_l\rf \times{}\\
{}\times f_n (z_1\tr z_n; t_1\tr t_n)\, dz_1\cdots dz_n\,. %\label{e1.14-sin}
\end{multline*}

Для нахождения одномерных плотностей $f_1(z,t) \hm= f_1^* (z)$ и характеристических функций 
$g_1(\la;t) \hm= g_1^* (\la)$ стохастических режимов в стационарных СтС~(\ref{e1.1-sin}) при
    \begin{equation}
    a(z,t) = a^*(z)\,;\ b(z,t)=b^*(z)\,;\ \chi(\mu;t)= \chi^*(\mu)
    \label{e1.15-sin}
    \end{equation}
следует в~(\ref{e1.3-sin}) и~(\ref{e1.7-sin}) положить 
$\prt f_1/\prt t \hm= 0$ и $\prt g_1/ \prt t \hm=0$. В~результате получим соответственно
\begin{multline*}
\fr{\prt^{\mathrm{T}}}{\prt z}\lk a^* (z) f_1^* (z)\rk = {}\\
{}=
\fr{1}{(2\pi)^k} \iin \iin \chi^* (b^*(\xi)^{\mathrm{T}} \la) e^{i\la^{\mathrm{T}}(\xi-z)} f_1^* (\xi)\, d\xi d\la\,;
%\label{e1.16-sin}
\end{multline*}

\vspace*{-12pt}

\noindent
\begin{multline*}
-\fr{1}{(2\pi)^k} \iin  \iin i\la^{\mathrm{T}} a^*(z) e^{i(\la^{\mathrm{T}}-\mu^{\mathrm{T}})z} g_1^*(\mu)\, d\mu dz={}\\
{}=\fr{1}{(2\pi)^k} \iin  \iin \chi^*(b^*(z)^{\mathrm{T}}\la) e^{i(\la^{\mathrm{T}}-\mu^{\mathrm{T}})z} g_1^*(\mu)\, d\mu dz.
%\label{e1.17-sin}
\end{multline*}
Поставим задачу разработки точных и приближенных  алгоритмов
аналитического моделирования распределений (плотностей и
характеристических функций) стохастических режимов  $Z\hm=Z(t)$ в
нелинейных гауссовских и негауссовских СтС~(\ref{e1.1-sin})  с разрывными
характеристиками $a\hm=a(z,t)$ и $b\hm=b(z,t)$, обладающих свойством
сохранения инвариантной меры, т.\,е.\ удовлетворяющих уравнениям~(\ref{e1.3-sin})
и~(\ref{e1.7-sin}) при $\chi\hm=0$.

Условия сохранения инвариантной меры можно представить в следующем развернутом виде:
\begin{equation}
\left.
\begin{array}{c}
\displaystyle\fr{\prt f_1 (z;t)}{\prt t} + A_a f_1 (z;t) =0\,;\\[9pt] 
\hspace*{-4.5mm}\displaystyle A_a f_1(z;t) = 
    \fr{\prt^{\mathrm{T}}}{\prt z} \lk a(z,t) f_1(z;t)\rk = \mathrm{div}\, \pi(z;t)\,;
    \end{array}
    \right\}
    \label{e1.18-sin}
    \end{equation}
\begin{equation}
\left.
\begin{array}{c}
A_a^* f_1^*(z) =0\,;\\[9pt]
\displaystyle A_a^* f_1^* (z) = \fr{\prt^{\mathrm{T}}}{ \prt z} \lk a^* 
(z) f_1^* (z)\rk =\mathrm{div}\, \pi^* (z)\,;
\end{array}
\right\}
\label{e1.19-sin}
\end{equation}
$$
\fr{\prt g_1 (\la;t)}{\prt t} - B_a g_1(\la;t) =0\,;
$$

\vspace*{-14pt}

\noindent
\begin{multline}
B_a g_1(\la;t) ={}\\[2pt]
{}=\fr{1}{(2\pi)^k} \iin\iin i\la^{\mathrm{T}} a(z,t) e^{i(\la^{\mathrm{T}}-\mu^{\mathrm{T}})z}
 g_1(\mu;t)\, d\mu dz={}\\[2pt]
{}= \iin i\la^{\mathrm{T}} a(z,t) e^{i\la^{\mathrm{T}}z} f_1(z;t)\, dz={}\\[2pt]
{}= \iin e^{i\la^{\mathrm{T}} z} i\la^{\mathrm{T}} \pi(z;t)\, dz\,;
\label{e1.20-sin}
\end{multline}

\vspace*{-9pt}

\noindent
\begin{equation}
\left.
\begin{array}{c}
\hspace*{-45mm}B_a^* g_1^* (\la)=0\,;\\[12pt]
\hspace*{-48mm}B_a^* g_1^* (\la) = {}\\[10pt]
\hspace*{-3mm}{}=\fr{1}{(2\pi)^k} \iin\! i\la^{\mathrm{T}} a^* (z) e^{i(\la^{\mathrm{T}} -\mu^{\mathrm{T}})z} g_1^* (\mu)\, d\mu dz={}\\[10pt]
{}=\displaystyle\iin\! i\la^{\mathrm{T}} a^*(z) e^{i\la^{\mathrm{T}}z} f_1^* (z)\, dz = {}\\[10pt]
\displaystyle{}=
\iin e^{i\la^{\mathrm{T}} z} i\la^{\mathrm{T}} \pi^* (z)\, dz\,.
\end{array}
\right\}
\label{e1.21-sin}
\end{equation}
Для гауссовских (нормальных) СтС с гладкими характери\-стиками точные и приближенные 
методы  и алгоритмы аналитического моделирования рассмотрены в~[1--15]. 

Особое внимание 
уделим приближенным методам, основанным на методах нор\-маль\-ной аппроксимации и статистической 
линеаризации. Подробно рассмотрим их применение к пуассоновским СтС.



\section{Точные методы и~алгоритмы аналитического моделирования распределений 
с~инвариантной мерой}

Пусть функция~$a$ в СтС~(\ref{e1.1-sin}) допускает пред\-став\-ле\-ние
\begin{equation}
a= a(z,t) = a_1(z,t) +a_2 (z,t) \label{e2.1-sin}
\end{equation}
такое, что функция  $f_1\hm=f_1(z;t)$ является плот\-ностью инвариантной меры 
невозмущенной шумами системы, описываемой векторным обыкновенным дифференциальным 
уравнением вида
   \begin{equation}
   \dot z = a_1 (z,t)\,,\label{e2.2-sin}
   \end{equation}
т.\,е.\ удовлетворяет условию~(\ref{e1.18-sin}):
\begin{equation}
\fr{\prt f_1 (z;t)}{\prt t}+ \fr{\prt^{\mathrm{T}}}{\prt z} \lk a_1 (z,t) f_1(z;t)\rk =0\,.
\label{e2.3-sin}
\end{equation}

Для гладких функций $a_1\hm=a_1(z,t)$ вопросы существования и основные свойства 
интегральных 
 инвариантов изучены в~\cite{16-sin, 17-sin}. При этом в~(\ref{e2.1-sin}) 
функция $a_2 \hm= a_2(z,t)$ определяется путем решения следующего интегродифференциального 
уравнения:
\begin{multline}
\fr{\prt^{\mathrm{T}}}{\prt z}\lk a_2 (z,t) f_1(z;t) \rk 
=
\fr{1}{(2\pi)^k}\times{}\\
{}\times \iin\iin \chi(b(\xi,t)^{\mathrm{T}} \la;t) 
e^{i\la^{\mathrm{T}}(\xi-z)} f_1(\xi;t)\, d\xi d\la\,.\label{e2.4-sin}
\end{multline}
В общем случае нахождение функций $a_1$ и~$a_2$ в~(\ref{e2.1-sin})~--- такая же
трудная задача, как решение уравнений~(\ref{e1.3-sin}) и~(\ref{e1.4-sin}).

Для стационарных СтС, когда выполнены условия~(\ref{e1.15-sin}), 
уравнения~(\ref{e2.1-sin})--(\ref{e2.4-sin}) имеют вид:
\begin{align}
a(z)&= a_1(z) + a_2(z)\,;\label{e2.5-sin}
\\
\dot z &= a_1(z)\,,\label{e2.6-sin}
\\
\fr{\prt^{\mathrm{T}}}{\prt z}\lk a_2^*(z) f_1^*(z)\rk &= {}\notag\\
&\hspace*{-28mm}{}=
\fr{1}{(2\pi)^k} \!\!\iin \iin\!\! \chi^* (b^*(\xi)^{\mathrm{T}} \la) 
e^{i\la^{\mathrm{T}}(\xi-z)} f_1^*(\xi)\, d\xi d\la\,.\!\!\!\label{e2.7-sin}
\end{align}
В этом случае можно выбирать невозмущенную сис\-те\-му~(\ref{e2.6-sin}) так, чтобы
она имела первые интегралы.

В терминах характеристических функций соотношения~(\ref{e2.3-sin}), (\ref{e2.4-sin})
и~(\ref{e2.7-sin}) могут быть записаны следующим образом:

\noindent
\begin{equation}
\fr{\prt g_1 (\la;t)}{\prt t} - B_{a_1} g_1(\la;t) =0\,;\label{e2.8-sin}
\end{equation}
\begin{equation*}
B_{a_1}^* g_1^*(\la) =0\,. %\label{e2.9-sin}
\end{equation*}
Для составляющих $a_2(z,t)$ и $a_2^*(z)$ имеют место уравнения
\begin{multline}
B_{a_2} g_1(\la;t) 
= \fr{1}{(2\pi)^k} \times{}\\
\hspace*{-2.5mm}{}\times\iin\iin \!\chi(b(z,t)^{\mathrm{T}} \la;t) 
e^{i(\la^{\mathrm{T}}-\mu^{\mathrm{T}})z} g_1(\mu;t) \,d\mu dz;\label{e2.10-sin}
\end{multline}

\vspace*{-16pt}

\noindent
\begin{multline}
B_{a_2}^* g_1^*(\la) 
= \fr{1}{(2\pi)^k} \times{}\\
{}\times\iin\iin 
\chi^*(b^*(z)^{\mathrm{T}} \la) e^{i(\la^{\mathrm{T}}-\mu^{\mathrm{T}})z} g_1^*(\mu)\, d\mu dz\,.
\label{e2.11-sin}
\end{multline}

Отсюда вытекают конструктивные точные алгоритмы аналитического
моделирования распределений с инвариантной мерой. В~их основе лежат
следующие теоремы.

%\pagebreak

\medskip

\noindent
\textbf{Теорема~2.1.} \textit{Функция $f_1\hm=f_1(z;t)$ будет решением}~(\ref{e1.3-sin})
\textit{и}~(\ref{e1.4-sin}) \textit{тогда и только тогда, когда $a\hm=a(z,t)$ допускает
представление}~(\ref{e2.1-sin}) \textit{такое, что $f_1\hm=f_1(z;t)$ является плотностью
инвариантной меры обыкновенного дифференциального уравнения}~(\ref{e2.2-sin}),
\textit{т.\,е.\ удовле\-тво\-ря\-ет условию}~(\ref{e2.3-sin}). \textit{При этом со\-став\-ля\-ющая $a_2$
определяется из решения интегродифференциального уравнения}~(\ref{e2.4-sin}).

\medskip

\noindent
\textbf{Теорема~2.2.} \textit{Функция $f_1^*\hm=f_1^*(z)$ будет решением}~(\ref{e1.3-sin}) 
\textit{тогда и только тогда, когда $a^*\hm=a^*(z)$ допускает
представление}~(\ref{e2.5-sin}) \textit{такое, что $f_1^*\hm=f_1^*(z)$ является плотностью
инвариантной меры}~(\ref{e2.6-sin}). \textit{При этом составляющая $a_2^{*}$
определяется из решения  уравнения}~(\ref{e2.7-sin}).

\medskip

\noindent
\textbf{Теорема~2.3.} \textit{Функция $g_1\hm=g_1(\la;t)$ будет ре\-ше\-нием}~(\ref{e1.7-sin}), 
(\ref{e1.8-sin}) \textit{тогда и только тогда, когда недиф\-фе\-ренцируемая функция
$a\hm=a(z,t)$  допускает пред\-став\-ление}~(\ref{e2.1-sin}) \textit{такое, что
$g_1\hm=g_1(\la;t)$ является ха\-рак\-теристической функцией инвариантной
меры \mbox{уравнения}}~(\ref{e2.2-sin}), \textit{т.\,е.\ удовлетворяет условию}~(\ref{e2.8-sin}). 
\textit{При этом составляющая $a_2$ определяется из уравнения}~(\ref{e2.10-sin}).

\medskip

\noindent
\textbf{Теорема 2.4.} \textit{Функция $g_1^*\hm=g_1^*(\la)$  будет решением}~(\ref{e1.13-sin}) 
\textit{тогда и только тогда, когда недифференцируемая функция $a^*\hm=a^*(z)$  
допускает представление}~(\ref{e2.5-sin}) \textit{такое, что $g_1^*$ является  
характеристической функцией инвариантной меры}~(\ref{e2.2-sin}). 
\textit{При этом $a_2^*$ определяется из решения}~(\ref{e2.11-sin}).

\smallskip

Теоремы~2.1--2.4 легко обобщаются на случай многомерных распределений с инвариантной мерой.

\section{Приближенные методы и~алгоритмы аналитического моделирования распределений 
с~инвариантной мерой, основанные на~нормальной аппроксимации и статистической линеаризации}

Пусть нелинейная СтС~(\ref{e1.1-sin}) допускает применение метода нормальной аппроксимации 
(МНА)~[1, 2]. Тогда одно- и двумерные нормальные плот\-ности $f_1^{\mathrm{МНА}}$,
 $f_2^{\mathrm{МНА}}$ и характеристические функции  $g_1^{\mathrm{МНА}}$,  
 $g_2^{\mathrm{МНА}}$, а также вектор математического ожидания $m_t = M^{\mathrm{МНА}} Z(t)$, 
 ковариационная мат\-ри\-ца $K_t \hm= M^{\mathrm{МНА}} Z^{0\mathrm{T}} Z^0 (t)$ 
 $(Z^0 (t) \hm= Z(t) \hm- m_t)$ и матрица ковариационных функций 
 $K(t_1, t_2) \hm= M^{\mathrm{МНА}} Z^{0\mathrm{T}} (t_1) Z^0 (t_2)$ $(t_1\hm< t_2)$ определяются 
 следующими уравнениями:
    \begin{multline}
    f_1^{\mathrm{МНА}} = f_1^{\mathrm{МНА}} (z;t, m_t, K_t) =
    \lk (2\pi)^k |K_t|\rk^{-1/2}\times{}\\
    {}\times \exp \lf -  \fr{1}{ 2} 
    \left(z^{\mathrm{T}} - m_t^{\mathrm{T}}\right) K_t^{-1}(z-m_t)\rf\,;\label{e3.1-sin}
    \end{multline}
    
    \vspace*{-12pt}
    
    \noindent
\begin{multline}
f_2^{\mathrm{МНА}} ={}\\
= f_2^{\mathrm{МНА}} (z_1, z_2;t_1, t_2, m_{t_1}, m_{t_2}, K_{t_1}, K_{t_2}, K(t_1, t_2))=\\
{}=\lk (2\pi)^k |\bar K_2|\rk^{-1/2}\times{}\\
\hspace*{-2mm}{}\times \exp \lf - 
([z_1^{\mathrm{T}} z_2^{\mathrm{T}}] - \bar m_2^{\mathrm{T}}) 
\bar K_2^{-1}([z_1^{\mathrm{T}} z_2^{\mathrm{T}}]^{\mathrm{T}}-\bar m_2)\rf;
\!\!\label{e3.2-sin}
\end{multline}
\begin{equation}
g_1^{\mathrm{МНА}} (\la;t)=
\exp\lf i\la^{\mathrm{T}} m- \fr{1}{2}\,\la^{\mathrm{T}} K_t \la\rf\,;\label{e3.3-sin}
\end{equation}

\vspace*{-12pt}

\noindent
\begin{multline}
g_2^{\mathrm{МНА}} (\la_1, \la_2; t_1,t_2) ={}\\
{}= \exp \lf i \bar \la^{\mathrm{T}} \bar m_2 - 
    \fr{1}{2} \,\bar \la^{\mathrm{T}} \bar K_2 \bar \la\rf\,;\label{e3.4-sin}
    \end{multline}
$$
    \bar \la =\lk \la_1^{\mathrm{T}} \la_2^{\mathrm{T}}\rk^{\mathrm{T}}\,;\enskip 
    \bar m_2 =\lk m_{t_1}^{\mathrm{T}} m_{t_2}^{\mathrm{T}}\rk^{\mathrm{T}}\,;
    $$
    $$
    \bar K_2 =\begin{bmatrix}
        K(t_1, t_1)&K(t_1, t_2)\\[3pt]
        K(t_2, t_1)& K(t_2, t_2)
        \end{bmatrix}\,;
        $$
  \begin{multline}
  \dot m_t = a_1 (t, m_t, K_t) ={}\\
  {}=\iin a(z,t) f_1^{\mathrm{МНА}} (z; t, m_t, K_t) \,dz\,;
  \label{e3.5-sin}
  \end{multline}

\vspace*{-12pt}

\noindent
\begin{multline}
\dot K_t = a_2(t, m_t, K_t) = a_{21} + a_{12}+a_{22}={}\\
{}=\left[ \iin a(z,t) (z^{\mathrm{T}}-m_t^{\mathrm{T}}) + (z-m_t) a^{\mathrm{T}} (z,t) +{}\right.\\
\left.{}+ \sigma (z,t)
\vphantom{\iin}\right] f_1^{\mathrm{МНА}} (z;t, m_t, K_t)\, dz\,;
\label{e3.6-sin}
\end{multline}

\vspace*{-12pt}

\noindent
\begin{multline}
\fr{\prt K(t_1, t_2)}{\prt t_2} ={}\\
{}= a_3 (t_1, t_2, m_{t_1},m_{t_2}, K_{t_1}, K_{t_2}, K(t_1,t_2))={}\\
{}=\lk (2\pi)^{2k} |\bar K_2|\rk^{-1/2}\times{}\\
{} \times\iin\iin (z_1-m_{t_1}) a(z_2, t_2)
\exp\left\{ - ([z_1^{\mathrm{T}} z_2^{\mathrm{T}}]-\bar m_2^{\mathrm{T}})\times{}\right.\\
\left.{}\times\bar K_2^{-1} 
([z_1^{\mathrm{T}} z_2^{\mathrm{T}}]-\bar m_2)\right\} dz_1 dz_2\,.
\label{e3.7-sin}
\end{multline}
Здесь введены следующие обозначения:
\begin{equation}
\left.
\begin{array}{c}
z_1=z_{t_1}\,;\enskip  z_2=z_{t_2}\,;\enskip \bar m_2 =\lk m_{t_1}^{\mathrm{T}} m_{t_2}^{\mathrm{T}}\rk^{\mathrm{T}}\,;\\[9pt]
\displaystyle \bar K_2 =\begin{bmatrix}
        K(t_1,t_1)&K(t_1, t_2)\\[3pt]
        K(t_2, t_1)& K(t_2, t_2)
        \end{bmatrix}\,,
        \end{array}
        \right\}
        \label{e3.8-sin}
        \end{equation}
\begin{equation}
\sigma(z,t) = b(z,t) \nu(t) b(z,t)^{\mathrm{T}}\,,\label{e3.9-sin}
\end{equation}
где $\nu=\nu(t)$~--- интенсивность негауссовского белого шума $V\hm=V(t)$.

Для стационарных СтС  при $\dot m^* \hm=0$, $\dot K^* \hm=0$, 
$K(t_1, t_2)\hm= k(\tau)$ $(\tau\hm=t_1-t_2)$  соотношения~(\ref{e3.5-sin})--(\ref{e3.9-sin}) 
принимают вид:
\begin{equation}
a_1^* (m^*, K^*) =0\,;\label{e3.10-sin}
\end{equation}
\begin{equation}
    a_2^*(m^*, K^*) =0\,;\label{e3.11-sin}
    \end{equation}
    \begin{equation}
    \fr{dk(\tau) }{d\tau} = a_{11}^{\mathrm{МНА}} (m^*, K^*) k(\tau)\,;\label{e3.12-sin}
    \end{equation}
$$
k(\tau) = k(-\tau^{\mathrm{T}})\,;\enskip k(0)=K\,.
$$
Из уравнения~(\ref{e3.12-sin}) следует, что алгоритм МНА будет устойчивым, если матрица 
$a_{11}^{\mathrm{МНА}} (m_t, K_t, t)$ будет асимптотически устойчива.

Для $m$ и $K$ уравнения метода статистической линеаризации (МСЛ) в 
нелинейных СтС  при аддитивных шумах, когда $b(z,t) \hm= b_0(t)$, $b^*(z)\hm=b_0^*$ 
получаются из~(\ref{e3.5-sin})--(\ref{e3.7-sin}) и (\ref{e3.10-sin})--(\ref{e3.12-sin}) 
как частный случай.

Условия наличия нормального распределения с инвариантной мерой~(\ref{e1.18-sin}) 
и~(\ref{e1.19-sin}), если заменить $a(z,t)$ статистически
линеаризованным выраже\-нием
\begin{equation*}
    a(Z,t)\approx a_{10}^{\mathrm{МНА}} (t, m_t, K_t) + a_{11}^{\mathrm{МНА}} (t, m_t, K_t) 
    (Z-m_t)\,, %\label{e3.13-sin}
    \end{equation*}
где
\begin{equation*}
a_{10}^{\mathrm{МНА}} =a_{10}^{\mathrm{МНА}} (t, m_t, K_t)\equiv a_1\,; %\label{e3.14-sin}
\end{equation*}
    
    
   
    \noindent
    \begin{multline*}
    a_{11}^{\mathrm{МНА}}=a_{11}^{\mathrm{МНА}} (t, m_t, K_t) = {}\\
    {}=\lk \iin a(z,t) (z^{\mathrm{T}}-m_t^{\mathrm{T}}) 
        f_1^{\mathrm{МНА}} (z; t , m_t, K_t)\, dz\rk\times{}\\
        {}\times K_t^{-1} 
=\left(\fr{\prt}{\prt m_t} a_1^{\mathrm{T}}\right)^{\mathrm{T}}\,, %\label{e3.15-sin}
\end{multline*}
приводят к следующим соотношениям:
        \begin{multline}
\fr{\prt f_1^{\mathrm{МНА}} (z; t, m_t, K_t)}{\prt t} +\fr{\prt^{\mathrm{T}}}{ \prt z} 
\left\{ \left[ a_{10}^{\mathrm{МНА}} (t, m_t, K_t) 
+{}\right.\right.\\
\left.{}+ a_{11}^{\mathrm{МНА}} (t, m_t, K_t) (z-m_t) \vphantom{a_{10}^{\mathrm{МНА}}}
\right]\times{}\\
\left.{}\times 
     f_1^{\mathrm{МНА}} ( z; t , m_t, K_t)\right\} =0\,;
     \label{e3.16-sin}
     \end{multline}
     
     
     \noindent
\begin{multline}
\hspace*{-9.81628pt}\fr{\prt^{\mathrm{T}}}{\prt z} \left\{ \left[ a_{10}^{*{\mathrm{МНА}}}(m^*, K^*) + 
 a_{11}^{*{\mathrm{МНА}}}(m^*, K^*) (z-m^*)\right] \times{}\right.\\
\left.{}\times f_1^{*{\mathrm{МНА}}}(z; m^*, K^*)\right\} =0\,,\label{e3.17-sin}
 \end{multline}
где
\begin{multline*}
f_1^{*{\mathrm{МНА}}} (z; m^*, K^*) = \lk (2\pi)^k |K^*|\rk^{-1/2}\times{}\\
{}\times \exp \lf -
    \fr{1}{2} (z^{\mathrm{T}}-m^{*\mathrm{T}})(K^*)^{-1} (z-m^*)\rf\,.
    \end{multline*}

Аналогично в развернутом виде выписываются условия~(\ref{e1.20-sin}) и~(\ref{e1.21-sin}):
\begin{multline}
\fr{\prt g_1^{\mathrm{МНА}} (\la;t)}{\prt t} -\iin i\la^{\mathrm{T}} \left[ a_{10}^{\mathrm{МНА}} 
    (m_t, K_t, t) +{}\right.\\[2pt]
\left.    {}+ a_{11}^{\mathrm{МНА}} (m_t, K_t, t) (z- m_t) \right]\times{}\\[2pt]
{}\times e^{i\la^{\mathrm{T}} z} f_1^{\mathrm{МНА}} (z; m_t, K_t, t)\, dz=0\,;\label{e3.18-sin}
\end{multline}


\noindent
\begin{multline}
\iin i\la^{\mathrm{T}} \left[ a_{10}^{*{\mathrm{МНА}} } (m^*, K^*) 
+{}\right.\\[2pt]
\left.{}+a_{11}^{*{\mathrm{МНА}} } 
    (m^*, K^*) (z-m^*)\right]\times{}\\[2pt]
    {}\times
     e^{i\la^{\mathrm{T}}z} f_1^{*{\mathrm{МНА}} } (z; m^*, K^*)\, dz =0\,.
    \label{e3.19-sin}
    \end{multline}

Отсюда вытекают следующие теоремы.

\bigskip

\noindent
\textbf{Теорема~3.1.}\ \textit{Если существуют одно- и двумерные  плотности
стохастического режима, а  матрица $a_{11}^{\mathrm{МНА}}$ коэффициентов
статистической (нормальной) линеаризации асимптотически устойчива,
то приближенный алгоритм аналитического моделирования МНА
нестационарных стохастических режимов в СтС}~(\ref{e1.1-sin}) \textit{с инвариантной
мерой определяется выражениями}~(\ref{e3.1-sin})--(\ref{e3.7-sin}) и~(\ref{e3.16-sin}).

\bigskip

\noindent
\textbf{Теорема 3.2.}\ \textit{Если существуют стационарные одно- и
двумерные плотности стохастического режима, а матрица
$a_{11}^{*{\mathrm{МНА}}}$  коэффициентов статистической (нормальной)
линеаризации асимптотически устойчива, то приближенный алгоритм
аналитического моделирования стационарных стохастических режимов с
инвариантной мерой в стационарной СтС}~(\ref{e1.1-sin}) \textit{определяется 
выражениями}~(\ref{e3.10-sin})--(\ref{e3.12-sin}) и~(\ref{e3.17-sin}).

\bigskip

Как известно~[1, 2], одно- и двумерные нормальные распределения
определяют и все  $n$-мер\-ные распределения $(n\hm\ge 3)$, поэтому МНА и
МСЛ дают приближенные алгоритмы для любых многомерных плотностей
стохастических режимов, если они существуют. Аналогично
формулируются теоремы~3.3 и~3.4 на основе условий~(\ref{e3.18-sin}) и~(\ref{e3.19-sin}).


\section{О других приближенных методах и~алгоритмах аналитического моделирования 
распределений с~инвариантной мерой}

\vspace*{-2pt}

 Обобщением МНА являются различные
приближенные методы, основанные на параметризации распределений~[1, 2].
Аппроксимируя одномерную характеристическую функцию $g_1 (\la;t)$
и соответствующую плотность $f_1 (z,t)$ известными функциями
 $g_1^* (\la;\theta)$, $f_1^* (z;\theta)$,  зависящими от
конечномерного векторного параметра~$\theta$, можно свести задачу
приближенного определения одномерного распределения к выводу из
уравнения для характеристических функций обыкновенных
дифференциальных уравнений, определяющих~$\theta$ как функцию
времени. Это относится и к остальным многомерным распределениям.
При аппроксимации многомерных распределений целесообразно выбирать
последовательности функций $\{ f_n^* (z_1,\ldots,z_n;\theta_n)\}$ и 
$\{g_n^* (\la_1\tr \la_n;\theta_n)\}$, каждая пара
которых находилась бы в такой  зависимости от векторного параметра~$\theta_n$, 
чтобы при любом~$n$ множество параметров, образующих
вектор~$\theta_n$, включало в качестве подмножества множество
параметров, образующих вектор~$\theta_{n-1}$. Тогда при
аппроксимации $n$-мер\-но\-го распределения придется определять только
те координаты вектора~$\theta_n$, которые не были определены ранее
при аппроксимации функций $f_1, g_1\tr f_{n-1}$, $g_{n-1}$.

В зависимости от того, что представляют собой параметры, от
которых зависят функции $f_n^* (z_1\tr z_n;\theta_n)$ и 
$g_n^* (\la_1\tr \la_n;\theta_n)$, аппрок-\linebreak симирующие неизвестные
многомерные плотности $f_n (z_1,  \ldots,z_n; t_1 \tr t_n)$ и
характеристические функции $g_n (\la_1\tr \la_n; t_1,\ldots,t_n)$,
используются различные приближенные методы решения
 уравнений при условиях~(9)--(12), определяющих\linebreak многомерные
распределения вектора состояния сис\-те\-мы~$X_t$, в частности методы
моментов (ММ), семиинвариантов (МСИ), ортогональных разложений
(МОР), квазимоментов (МКМ) и~др.~[1, 2].

\vspace*{-6pt}


\section{Обобщение на~случай стохастических систем с~автокоррелированными шумами}

\vspace*{-2pt}

Пусть  СтС описывается нелинейным, в общем случае векторным дифференциальным 
стохастическим уравнением Ито~\cite{1-sin, 2-sin, 15-sin, 18-sin}

\noindent
\begin{equation}
\left.
\begin{array}{c}
    \dot Z = a(Z,t) + b_U(Z,t) U\,;\\[6pt] 
\displaystyle    \sss_{i=0}^l \alpha_i U^{(i)} =
\displaystyle\sss_{j=0}^h \beta_j V^{(j)}\enskip (h<l)\,.
\end{array}
\right\}
    \label{e5.1-sin}
    \end{equation}
    Здесь $U=U(t)$~--- векторная помеха размерности  $m\times 1$; $V\hm=V(t)$~--- 
    негауссовский белый шум с нулевым математическим ожиданием и известной функцией  
    $\chi\hm=\chi(\mu;t)$. В~таком случае в за\-ви\-си\-мости от степени <<гладкости>> 
    стохастического режима $Z\hm=Z(t)$ и помехи $U\hm=U(t)$ уравнения~(\ref{e5.1-sin})  
    путем расширения вектора состояния согласно~[1, 2] приводятся к виду~(\ref{e1.1-sin}) 
    для расширенного вектора состояния~$\bar Z$. Тогда, но уже для расширенного вектора 
    состояния СтС, при решении уравнений~(\ref{e5.1-sin}) могут быть использованы точные 
    (разд.~2) и приближенные (разд.~3) методы и алгоритмы аналитического моделирования 
    нестационарных и стационарных распределений с инвариантной мерой.

\section{Особенности аналитического моделирования распределений с~инвариантной мерой 
в~пуассоновских стохастических системах}

Рассмотрим СтС~(\ref{e1.1-sin}) при $b(z,t) \hm=I_m$ для обобщенного пуассоновского 
белого шума  $V^{\mathrm{OP}}\hm=  V^{\mathrm{OP}}(t)$, когда функция~(\ref{e1.2-sin}) 
определяется формулой
\begin{equation*}
\chi^{\mathrm{OP}} (\mu;t) =\lk g_c^{\mathrm{OP}} (\mu) -
1\rk \nu^{\mathrm{OP}} (t)\,, %\label{e6.1-sin}
\end{equation*}
где $g_c^{\mathrm{OP}} \hm=g_c^{\mathrm{OP}} (\mu)$~--- характеристическая 
функция скачков; $\nu^{\mathrm{OP}} \hm= \nu^{\mathrm{OP}} (t)$~--- 
интенсивность пуассоновского белого шума 
$V^{\mathrm{OP}}\hm=V^{\mathrm{OP}} (t)$. Обозначим через $f_c^{\mathrm{OP}} \hm=
 f_c^{\mathrm{OP}} (z)$ плотность скачков обобщенного пуассоновского процесса. 
 Тогда~(\ref{e1.3-sin}) будет представлять собой известное уравнение Фел\-ле\-ра--Кол\-мо\-го\-ро\-ва
\begin{multline}
\fr{\prt f_1(z;t)}{\prt t} + \fr{\prt^{\mathrm{T}}}{\prt z} 
    \lk a(z,t) f_1(z;t)\rk ={}\\
    \hspace*{-3mm}{}= \nu^{\mathrm{OP}} (t) \lk \iin f_c^{\mathrm{OP}} (z-\xi) f_1 (\xi;t)\, d\xi - f_1(z;t)\rk
    \label{e6.2-sin}
    \end{multline}
с начальным условием~(\ref{e1.4-sin}). В~случае простого пуассоновского белого шума 
с единичными скачками $g_c (\mu) \hm= e^{i\mu}$.

Для  стационарной пуассоновской СтС~(\ref{e1.1-sin}) уравнение~(\ref{e6.2-sin}) имеет следующий вид:
\begin{multline}
\fr{\prt^{\mathrm{T}}}{\prt z} \lk a^* (z) f_1^* (z)\rk = {}\\
{}=
\nu^{\mathrm{OP} *} \lk \iin f_c^{\mathrm{OP}} (z-\xi) f_1^* (\xi)\, d\xi- 
f_1^* (z)\rk\,.\label{e6.3-sin}
\end{multline}

Пользуясь уравнениями~(\ref{e6.2-sin}), (\ref{e6.3-sin})  
и результатами разд.~1 и~2, нетрудно сформулировать следующие утверждения.

\medskip

\noindent
\textbf{Теорема 6.1.}\ \textit{Функция $f_1 \hm= f_1(z;t)$ будет
нестационарным решением}~(\ref{e6.2-sin}), (\ref{e1.4-sin}) \textit{тогда и только тогда, 
когда $a$ допускает представление}~(\ref{e2.1-sin}) \textit{такое, что $f_1$ является плот\-ностью
инвариантной меры обыкновенного дифференциального уравнения}~(\ref{e2.2-sin}),
\textit{т.\,е.\ удовле\-тво\-ря\-ет условию}~(\ref{e2.3-sin}), \textit{а составляющая $a_2$ определяется
из решения следующего уравнения}:
\begin{multline*}
    \fr{\prt^{\mathrm{T}}}{\prt z} \lk a_2 (z,t) f_1 (z;t)\rk =
     \fr{1}{(2\pi)^k}\times{}\\
     {}\times \iin\iin \chi^{\mathrm{OP}} 
    \left(b(\xi,t)^{\mathrm{T}} \la;t\right) e^{i\la^{\mathrm{T}}(\xi-z)} f_1(\xi,t)\,d\xi d\la\,.
%    \label{e6.4-sin}
    \end{multline*}

%\smallskip

\noindent
\textbf{Теорема 6.2.}\ \textit{Функция $f_1^* \hm= f_1^* (z)$ будет стационарным 
решением}~(\ref{e6.3-sin}) \textit{тогда и только тогда, когда $a_2^*$ допускает 
представление}~(\ref{e2.5-sin}) \textit{такое, что  $f_1^*$ является плот\-ностью 
инвариантной меры}~(\ref{e2.6-sin}), \textit{а составляющая $a_2^{*}$ определяется 
из решения следующего уравнения}:
\begin{multline*}
\fr{\prt^{\mathrm{T}} }{\prt z} \lk a_2^{*} (z) f_1^* (z)\rk ={}\\
{}=
    \fr{1}{(2\pi)^k} \iin\iin \chi^{\mathrm{OP} *} (b(\xi)^{\mathrm{T}} \la) 
    e^{i\la^{\mathrm{T}}(\xi-z)} f_1^*(\xi)\,d\xi d\la\,.
%    \label{e6.5-sin}
    \end{multline*}

При использовании МНА и МСЛ для пуассоновских СтС непосредственно применяются теоремы~3.1--3.4, 
причем в формулу~(\ref{e3.9-sin}) для  
$\sigma(z,t)$ входит интенсивность 
$\nu^{\mathrm{OP}} (t)$ обобщенного пуассоновского белого шума.

\section{Тестовые примеры}

\noindent
\textbf{Пример~1}. Рассмотрим осциллятор Дуффинга в обобщенной пуассоновской 
стохастической среде:
\begin{equation}
\ddot X +\w^2 X -\mu X^3 =-\delta^{\mathrm{OP}} \dot X + V^{\mathrm{OP}} (t)\,.\label{e7.1-sin}
\end{equation}
Уравнения МСЛ для~(\ref{e7.1-sin}) имеют следующий вид:
\begin{equation}
\dot m_X = m_{\dot X}\,;\enskip 
\dot m_{\dot X} =- \w_{\mathrm{э}}^2 m_X -\delta^{\mathrm{OP}} m_{\dot X}\,;
\label{e7.2-sin}
\end{equation}
    \begin{equation}
    \left.
    \begin{array}{rl}
    \dot D_{X} &= 2 K_{X\dot X}\,;\\[6pt] 
    \dot D_{\dot X} &=\nu^{\mathrm{OP}} - 2 (\w_{1 \mathrm{э}}^2 K_{X\dot X} + 
    \delta^{\mathrm{OP}} D_{\dot X})\,;\\[6pt]
\dot K_{X\dot X} &= D_{\dot X} -\w_{1 \mathrm{э}}^2 D_X - 
\delta^{\mathrm{OP}} K_{X\dot X}\,.
\end{array}
\right\}
 \label{e7.3-sin}
\end{equation}
Здесь кубическая функция $X^3$ была заменена на статистически линеаризованную при 
гауссовом распределении с дисперсией  $D_X$ согласно~[1, 2]:
\begin{equation*}
X^3 \approx k_0 (m_X, D_X) m_X + k_1 (m_X, D_X) X^0\,,\label{e7.4-sin}
\end{equation*}
где
\begin{align*}
k_0 (m_X, D_X) &= m_X^2 + 3 D_X\,;\\ 
k_1 (m_X, D_X) &= 3 (m_X^2 + D_X)\,;\\
%\label{e7.5-sin}
\w_{\mathrm{э}}^2 &=\w^2 \lk 1- \fr{\mu (m_X^2 + 3D_X)}{\w^2}\rk\,;\\
\w_{1 \mathrm{э}}^2 &=\w^2 \lk 1-  \fr{3\mu (m_X^2 + D_X)}{\w^2}\rk \enskip 
(\w_{\mathrm{э}}>\w_{1 \mathrm{э}})\,.
\end{align*}
%\label{e7.6-sin}
Из~(\ref{e7.2-sin}) и~(\ref{e7.3-sin}) в стационарном режиме имеем:
\begin{gather*}
m_X^* =0\,;\enskip 
m_{\dot X}^* =0\,;\enskip 
K_{X\dot X}^* =0\,;\\
D_{\dot X}^* =\vartheta\,;\enskip 
\vartheta =  \fr{\nu^{\mathrm{OP}}}{ 2\delta^{\mathrm{OP}}}\,,
\end{gather*}
%\label{e7.7-sin}
а $D_X^*$ определяется из уравнения:
    \begin{equation*}
    \w_{1 \mathrm{э}}^2 (D_X^*) D_X^* =\vartheta\,. %\label{e7.8-sin}
    \end{equation*}
Условие наличия стационарного распределения с инвариантной мерой~(\ref{e3.17-sin}) 
требует консерватизма линеаризованной левой части~(\ref{e7.1-sin}). 
Процесс установления стационарных стохастических колебаний происходит 
в два этапа: сначала устанавливается $D_{\dot X}^*$, а затем $D_X^*$.

Интересно отметить, что уравнения МСЛ~(\ref{e7.2-sin}) и~(\ref{e7.3-sin}) сохраняют свой
вид и для любого белого шума интенсивности  $\nu(t)$,
представляющего собой с.к., производную от произвольного процесса с
независимыми приращениями~$W(t)$. Для гауссовского белого шума
$\nu\hm=\nu^G$ соответствующие результаты получены в~\cite{1-sin, 2-sin, 15-sin}. Как
показали вычислительные эксперименты для значений~$\mu$, отвечающих
стохастическим колебаниям, точность составляет около 10\%~\cite{15-sin}.

\medskip

\noindent
\textbf{Пример~2}.\  Для осциллятора Дуффинга в автокоррелированной  пуассоновской среде, когда
\begin{equation*}
\ddot X+ \w^2 X -\mu X^3 =-\delta^{\mathrm{OP}} \dot X + U\,;\enskip 
\dot U +\gamma U =V^{\mathrm{OP}} (t)\,, %\label{e7.9-sin}
\end{equation*}
уравнения МСЛ для  $Z\hm= [X\dot X U]^{\mathrm{T}}$ имеют вид~(\ref{e3.5-sin}) и~(\ref{e3.6-sin}) при
    \begin{gather*}
   a_1 = \begin{bmatrix}
        m_{\dot X}\\
        -\w_{ \mathrm{э}}^2 m_X-\delta^{\mathrm{OP}} m_{\dot X}\\
        -m_U\end{bmatrix}\,;\\
    \alpha=  \begin{bmatrix}
            0&1&0\\
            -\w_{1 \mathrm{э}}^2&-\delta^{\mathrm{OP}}&0\\
            0&0&-\gamma\end{bmatrix}\,;\enskip
    \beta= \begin{bmatrix}
        0&0&0\\
        0&0&0\\
        0&0&1\end{bmatrix}\,;
%        \label{e7.10-sin}
\\
a_2 =\alpha K_t+ K_t \alpha^{\mathrm{T}} +\beta \nu^{\mathrm{OP}} \beta^{\mathrm{T}}\,.
        \end{gather*}
Здесь $\nu^{\mathrm{OP}} =\nu^{\mathrm{OP}}(t)$~--- интенсивность белого шума 
$V^{\mathrm{OP}}(t)$. 
Отсюда аналитическим мо\-де\-ли\-ро\-ванием определяются стационарные
режимы, а также режимы их установления. Так же, как в\linebreak случае
автокоррелированных гауссовских белых шумов~\cite{1-sin, 2-sin, 15-sin}, точность МСЛ
за счет <<профильтрованности>> помех значительно повышается и
достигает 2\%--4\%. Результат справедлив и для произвольных
негауссовских белых шумов.

\medskip

\noindent
\textbf{Пример 3}.\  Для релейного осциллятора в гауссовской стохастической среде
\begin{equation}
\ddot X + \w^2 {\mathrm{sgn}} X = -\delta^G \dot X + V^G + U_0\label{e7.11-sin}
\end{equation}
плотность распределения стационарного режима стохастических колебаний при $U_0\hm=0$ 
определяется формулой Гиббса~[1, 2]:
\begin{equation}
f^* (x,\dot x) = c \exp \lf - 
    \fr{H(x,\dot x)}{\vartheta^G}\rf\,,\enskip \vartheta^G = 
    \fr{\nu^G}{ 2\delta^G}\,.\label{e7.12-sin}
    \end{equation}
Здесь через
\begin{equation*}
H(x,\dot x) = \fr{\dot x^2}{2} +\Pi(x)\,,\enskip \Pi (x) =\w^2 |x|\,, %\label{e7.13-sin}
\end{equation*}
обозначена полная энергия осциллятора.

Для~(\ref{e7.11-sin}) при  $U_0\hm\ne 0$, если заменить релейную характеристику 
статистически линеаризованной, согласно~[1, 2]
\begin{equation*}
\mathrm{sgn}\, X = k_0 (m_X, D_X) m_X + k_1 (m_X, D_X) (X^0 - m_X)\,; %\label{e7.14-sin}
\end{equation*}
    $$
    k_0(m_X, D_X) =\fr{2}{ m_X} \Phi \left( \fr{m_X}{\sqrt{D_X}}\right)\,;
    $$
    $$ 
    k_1 (m_X,D_X) = \fr{1}{\sqrt{D_X}} \sqrt{\fr{2}{\pi}}\, \exp \left( -\fr{m_X^2}{2D_X}\right)\,;
    $$
\begin{equation}
\Phi (\tau) = \fr{1}{2\pi} \int\limits_0^\tau e^{-t^2/2} dt\,.\label{e7.15-sin}
\end{equation}
Тогда уравнения МСЛ будут иметь вид:
\begin{equation}
\left.
\begin{array}{rl}
\dot m_X &= m_{\dot X}\,;\\[9pt]
\dot m_X &= U_0 - \w^2 k_0 (m_X, D_X) m_X -\delta m_{\dot X}\,;
\end{array}
\right\}
\label{e7.16-sin}
\end{equation}
    \begin{equation}
\left.
\hspace*{-3.5mm}\begin{array}{c}
    \dot D_X = 2 K_{X\dot X}\,;
\\
    \dot D_{\dot X} = \nu^G - 2\lk \delta D_{\dot X} + \w^2 k_1(m_X,D_X) K_{X\dot X}\rk\,;\\[9pt]
    \dot K_{X\dot X} = D_{\dot X} - \w^2 k_1 (m_X, D_X) D_X - \delta K_{X\dot X}\,,
    \end{array}
    \right\}\!\!
    \label{e7.17-sin}
    \end{equation}
где $\delta \hm= \delta^G$, $\nu\hm=\nu^G$.
Отсюда для стационарных стохастических колебаний имеем связанную систему уравнений:
\begin{equation}
m_{\dot X}^* =0\,;\enskip \w^2 k_0 (m_X^*, D_X^*) = U_0\,;\label{e7.18-sin}
\end{equation}
\begin{equation}
\left.
\begin{array}{c}
K_{X\dot X}^* =0\,;\enskip 
D_X^* =\vartheta=\displaystyle \fr{\nu}{ 2\delta}\,;\\[9pt]
k_1(m_X^*, D_X^*) D_X^* =\rho= \displaystyle \fr{\vartheta}{\w^2} =\fr{\nu}{ 2\delta \w^2}\,.
\end{array}
\right\}
\label{e7.19-sin}
\end{equation}

При $U_0 =0$ из~(\ref{e7.15-sin}), (\ref{e7.18-sin}) и~(\ref{e7.19-sin}) находим:
\begin{equation*}
m_X^* =0\,;\enskip 
m_{\dot X}^* =0\,; \enskip 
D_{\dot X}^* =\vartheta\,;\enskip 
D_X^* =  \fr{\pi}{2}\,\rho^2\,. %\label{e7.20-sin}
\end{equation*}
Отсюда видно, что стационарная дисперсия скорости совпадает с точным
решением~(\ref{e7.12-sin}). Стационарная дисперсия координаты, найденная
согласно МСЛ, отличается от следующего точного решения, полученного
согласно~(\ref{e7.12-sin}). При $\rho\hm \le 1$ относительная ошибка составляет
10\%. Стационарные колебания по~$X$ и $\dot X$ не коррелированы.

Уравнения~(\ref{e7.16-sin}) и~(\ref{e7.17-sin}) показывают, что процесс установления 
режима стохастических колебаний происходит в две стадии: сначала устанавливается 
стационарное распределение по ско\-рости~$\dot X$, а затем по координате~$X$.

\medskip

\noindent
\textbf{Пример 4}.  В~условиях примера~3, но для пуассоновской среды, когда
    \begin{equation*}
    \ddot X +\w^2 {\mathrm{sgn}} X =-\delta^{\mathrm{OP}} \dot X + 
    V^{\mathrm{OP}} + U_0\,,
%    \label{e7.21-sin}
    \end{equation*}
уравнения МСЛ имеют вид~(\ref{e7.16-sin}), (\ref{e7.17-sin}), если принять 
$\delta\hm= \delta^{\mathrm{OP}}$, $ \nu\hm=\nu^{\mathrm{OP}}$, 
$\vartheta\hm=\vartheta^{\mathrm{OP}}\hm=\nu^{\mathrm{OP}}/(2\delta^{\mathrm{OP}})$, 
$\rho \hm=\vartheta^{\mathrm{OP}}/\w^2$. Точного аналитического уравнения 
Фел\-ле\-ра--Кол\-мо\-го\-ро\-ва не обнаружено.

Другие тестовые примеры можно найти в~[10, 12--14].

\section{Заключение}

Дано обобщение точных и приближенных (основанных на параметризации распределений)\linebreak 
методов и алгоритмов теории распределений с инвари\-антной мерой на случай нелинейных 
дифференциальных гауссовых и негауссовых стохастических систем с гладкими и разрывными 
характеристиками.

Особое внимание уделено пуассоновским стохастическим системам с разрывными характеристиками.

На тестовых примерах показана достаточная точность для практических приложений в стохастической 
информатике.

{\small\frenchspacing
{%\baselineskip=10.8pt
\addcontentsline{toc}{section}{Литература}
\begin{thebibliography}{99}
\bibitem{1-sin}
\Au{Пугачёв В.\,С., Синицын И.\,Н.} Стохастические дифференциальные системы. 
Анализ и фильтрация.~--- 2-е изд., доп.~--- М.: Наука, 1990.

\bibitem{2-sin}
\Au{Пугачёв В.\,С., Синицын И.\,Н.} Теория стохастических систем.~--- 2-е изд.~--- М.: Логос,  2004.

\bibitem{3-sin}
\Au{Moshchuk N.\,K., Sinitsyn I.\,N.} On stationary distributions in nonlinear 
stochastic differential systems: Preprint.~--- Coventry, UK: 
University of Warwick, Mathematics Institute, 1989. 15~p.

\bibitem{4-sin}
\Au{Moshchuk N.\,K., Sinitsyn I.\,N.} On stochastic nonholonomic systems: Preprint.~--- 
Coventry, UK: University of Warwick, Mathematics Institute, 1989. 32~p.

\bibitem{5-sin}
\Au{Мощук Н.\,К., Синицын И.\,Н.} О~стохастических неголономных системах~// 
Прикладная механика и математика, 1990. Т.~54. Вып.~2. С.~213--223.

\bibitem{6-sin}
\Au{Moshchuk N.\,K., Sinitsyn I.\,N.} On stationary distributions in 
nonlinear stochastic differential systems~// Quart. J. Mech. Appl. Math., 1991. Vol.~44.  
Pt.~4.  P.~571--579.

\bibitem{7-sin}
\Au{Мощук Н.\,К., Синицын И.\,Н.} О~стационарных и приводимых к стационарным 
режимах в нормальных стохастических системах~// 
Прикладная механика и математика, 1991. Т.~55. Вып.~6. С.~895--903.

\bibitem{8-sin}
\Au{Мощук Н.\,К., Синицын И.\,Н.} Распределения с инвариантной мерой в механических 
стохастических нормальных сис\-те\-мах~// Докл. АН СССР, 1992. Т.~322. №\,4. С.~662--667.

\bibitem{9-sin}
\Au{Синицын И.\,Н.} Конечномерные распределения с инвариантной мерой в стохастических 
механических сис\-те\-мах~// Докл. РАН, 1993. Т.~328. №\,3. С.~308--310.

\bibitem{13-sin} %10
\Au{Soize C.} The Fokker--Plank equation for stochastic dynamical systems 
and its explicit steady state solutions.~--- Singapore: World Scientific,  1994.

\bibitem{10-sin} %11
\Au{Синицын И.\,Н.} Конечномерные распределения с инвариантной мерой в 
стохастических нелинейных дифференциальных системах.~--- М.: Диалог--МГУ, 1997. С.~129--140.

\bibitem{11-sin} %12
\Au{Синицын И.\,Н., Корепанов Э.\,Р., Белоусов~В.\,В.} 
Точные методы расчета стационарных режимов с инвариантной мерой в стохастических 
сис\-те\-мах управ\-ле\-ния~// Кибернетика и технологии XXI~ве\-ка: Тр.\ II Междунар. 
науч.-техн. конф. C\&T'2002.~--- Воронеж: Саквое, 2002. С.~124--131.

\bibitem{12-sin} %13
\Au{Синицын И.\,Н., Корепанов Э.\,Р., Белоусов~В.\,В.} 
Точные аналитические методы в статистической динамике нелинейных 
ин\-фор\-ма\-ци\-он\-но-управ\-ля\-ющих сис\-тем~// Сис\-те\-мы и средства информатики. 
Спец. вып. Математическое и алгоритмическое обеспечение 
ин\-фор\-ма\-ци\-он\-но-те\-ле\-ком\-му\-ни\-ка\-ци\-он\-ных сис\-тем.~--- М.: Наука, 2002. С.~112--121.

\bibitem{14-sin}
\Au{Синицын И.\,Н.} Развитие методов аналитического моделирования распределений с 
инвариантной мерой в стохастических сис\-те\-мах~// Современные проб\-ле\-мы 
прикладной математики, информатики и автоматизации: Тр. Междунар. науч.-техн. семинара.~--- 
Севастополь, 2012. С.~24--35.

\bibitem{15-sin}
\Au{Синицын И.\,Н.} Аналитическое моделирование распределений с инвариантной мерой 
в стохастических сис\-те\-мах с автокоррелированными шумами~// 
Информатика и её применения, 2012. Т.~6. Вып.~4. С.~4--8.

\bibitem{16-sin}
\Au{Немыцкий В.\,В., Степанов В.\,В.} Качественная теория дифференциальных уравнений.~--- 
М.--Л.: Гостехиздат, 1949.


\bibitem{17-sin}
\Au{Козлов В.\,В.} О~существовании интегрального инварианта гладких динамических систем~// 
ПММ, 1987. №\,1. С.~538--545.

\label{end\stat}

\bibitem{18-sin}
\Au{Синицын И.\,Н.} Фильтры Калмана и Пугачёва.~--- 2-е изд.~--- М.: Логос, 2007.
\end{thebibliography}
}
}

\end{multicols}  %11
\def\stat{rumovskaya}

\def\tit{МЕТОД ВИЗУАЛИЗАЦИИ СНИЖЕНИЯ ИНТЕНСИВНОСТИ И~РАЗРЕШЕНИЯ 
КОНФЛИКТОВ В~ГИБРИДНЫХ ИНТЕЛЛЕКТУАЛЬНЫХ МНОГОАГЕНТНЫХ 
СИСТЕМАХ}

\def\titkol{Метод визуализации снижения интенсивности и~разрешения 
конфликтов в~ГиИМАС}

\def\aut{С.\,Б.~Румовская$^1$, И.\,А.~Кириков$^2$}

\def\autkol{С.\,Б.~Румовская, И.\,А.~Кириков}

\titel{\tit}{\aut}{\autkol}{\titkol}

\index{Румовская С.\,Б.}
\index{Кириков И.\,А.}
\index{Rumovskaya S.\,B.}
\index{Kirikov I.\,A.}


%{\renewcommand{\thefootnote}{\fnsymbol{footnote}} \footnotetext[1]
%{Работа выполнена при поддержке Министерства науки и~высшего образования Российской Федерации (проект 
%075-15-2020-799).}}


\renewcommand{\thefootnote}{\arabic{footnote}}
\footnotetext[1]{Калининградский филиал Федерального исследовательского центра <<Информатика и~управ\-ле\-ние>> 
Российской академии наук, \mbox{sophiyabr@gmail.com}}
\footnotetext[2]{Калининградский филиал Федерального исследовательского центра <<Информатика и~управ\-ле\-ние>> 
Российской академии наук, \mbox{baltbipiran@mail.ru}}

%\vspace*{-6pt}

  
  \Abst{Многие практические проблемы диктуют необходимость коллективного решения, 
обеспечивающего плюрализм мнений, интеграцию частных точек зрения и снижение числа 
ошибок. Моделирование работы таких коллективов специалистов гибридными 
интеллектуальными многоагентными системами (\mbox{ГиИМАС}), учитывая 
особенности их групповой динамики, позволит повысить качество и~эффективность 
решения, а~также всесторонне рассмотреть проблему и~процесс ее преодоления, в~том числе 
с~по\-мощью визуализации конфликтов и~процессов управления ими, предоставляя новую 
информацию по разрешению конфликтов и~в~системе, и~в~реальном коллективе 
специалистов. Работа посвящена разработке метода визуализации процессов разрешения 
конфликтов в~рамках \mbox{ГиИМАС} с~проб\-лем\-но- и~про\-цес\-сно-ори\-ен\-ти\-ро\-ван\-ны\-ми конструктивными конфликтами.}
  
  \KW{коллектив специалистов; конфликт агентов; визуализация разрешения конфликта}
  
  \DOI{10.14357/19922264220212}
  
\vspace*{-3pt}


\vskip 10pt plus 9pt minus 6pt

\thispagestyle{headings}

\begin{multicols}{2}

\label{st\stat}
  
\section{Введение}

  В~[1--6] предложены \mbox{ГиИМАС}, 
которые релевантны групповой динамике коллектива специалистов~[7--9], 
решающего проблему, и моделируют проб\-лем\-но-
и~про\-цес\-сно-ори\-ен\-ти\-ро\-ван\-ные конфликты~\cite{1-kir}. Это конструктивные 
инструментальные конфликты~[10], актуализирующие плюрализм мнений 
относительно проблемы и способствующие поиску оптимальных способов ее 
решения. Такие конфликты идентифицируются~\cite{2-kir}, 
интенсифицируются~\cite{3-kir} и разрешаются~\cite{4-kir} в рамках 
\mbox{ГиИМАС}, повышая их 
релевантность работе реальных малых коллективов специалистов. Также 
в~\cite{5-kir, 6-kir} разработаны методы визуализации возникающих между 
агентами конфликтов и процесса их интенсификации в~\mbox{ГиИМАС}, что 
повышает прозрачность работы системы для пользователя. В~\cite{4-kir} 
описан один из методов управления конфликтами~--- разрешение 
конструктивных инструментальных конфликтов, включающий такие стратегии 
разрешения противоречий (СРП)~\cite{11-kir}, как переговоры~--- обмен 
знаниями и информацией о целях между агентами для достижения соглашения; 
делегирование~--- привлечение третьей стороны (агента) с более развитой базой 
знаний и возможностями, но не способной напрямую взаимодействовать 
с~другими агентами; голосование~--- агенты голосуют по всем предварительно 
предложенным ими решениям; самомодификация~--- агент при возникновении 
конфликта вместо взаимодействия с целью выработки соглашения меняет свое 
поведение; игнорирование~--- отказ от разрешения конфликта ввиду его низкой 
интенсивности.
  
  Работ, содержащих визуализацию конкретных конфликтов, найдено было 
мало~\cite{5-kir}, и все они отоб\-ра\-жа\-ют динамику деструктивных 
макроконфликтов~\cite{12-kir} без деталей взаимодействия участников, 
а~работы с~визуализацией динамики конфликта в~малых группах специалистов, 
решающих проб\-ле\-му (в~том чис\-ле снижения интенсивности и~разрешения 
конфликта), отсутствуют. 
  
  Цель настоящей работы~--- разработка метода визуализации процесса 
снижения интенсивности и~разрешения конфликтов на базе предложенного 
метода их идентификации~\cite{2-kir}, функции управления~\cite{3-kir} 
и~метода разрешения~\cite{4-kir} в~рамках представленной в~\cite{1-kir} 
модели \mbox{ГиИМАС}, что сделает\linebreak возникшие противоречия контрастными, 
предо\-став\-ляя детальную визуализацию разрешения конфликта, явно 
отображающую снижение интенсивности конфликта между каждой парой 
\mbox{конфликтующих} агентов~--- тип конфликта, напряженность между 
участниками, их изменение и~применяемую стратегию разрешения конфликта 
или ее отсутствие. 

\section{Разрешение конфликтов между~агентами как часть 
функции~агента-фасилитатора <<управление конфликтом>>}

  В~\cite{3-kir} задана функция агента-фа\-си\-ли\-та\-то\-ра (АФ)  
<<управ\-ле\-ние конфликтом>>, в рамках которой вы\-чис\-ля\-ет\-ся среднее 
арифметическое показателей взаимозависимости целей 
агентов~$\mathrm{gd}^{\mathrm{himas}}$, затем запускается 
функция идентификации конфликтов, анализирующая решения, предложенные 
агентами-специалистами (АС), и формирующая матрицу конфликтов 
$\mathbf{CNF}$ между парами агентов, элемент которой представляет собой 
кортеж~((3) из~\cite{3-kir}):
  \begin{multline}
  \mathrm{cnf}_{ij\,\mathrm{cnft}}={}\\
  {}=\left\langle \mathrm{ag}_i, \mathrm{ag}_j, 
\mathrm{cnfin}, \mathrm{cnft}, \mathrm{ACT}_i^{\mathrm{agcr}}, 
\mathrm{ACT}_j^{\mathrm{agcr}}\right\rangle\,.
\label{e1-kir}
  \end{multline}
Здесь $\mathrm{ag}_i$ и~$\mathrm{ag}_j$~--- это аген\-ты-субъ\-ек\-ты конфликта,  
$i,j\hm\in \mathbf{N}$, $i,j\hm\in [1,n]$, $i\not=j$; $\mathrm{cnfin}\hm\in [0,1]$~--- 
напряженность конфликта; $\mathrm{cnft}$~--- <<тип конфликта>>, \mbox{определяется} на 
множестве $\mathrm{CNFT}\hm = \left\{ \mathrm{cnft}_{\mathrm{prb}}\right.\hm =$\;<<проб\-лем\-но-ори\-ен\-ти\-ро\-ван\-ный>>, 
$\mathrm{cnft}_{\mathrm{prc}}$\;=\;<<про\-цес\-сно-ори\-ен\-ти\-ро\-ван\-ный>>$\left.\right\}$; 
$\mathrm{ACT}_i^{\mathrm{agcr}}$ и~$\mathrm{ACT}_j^{\mathrm{agcr}}$~--- множества допустимых действий агентов 
$\mathrm{ag}_i$ и~$\mathrm{ag}_j$ соответственно по разрешению противоречий,  
$\mathrm{ACT}_i^{\mathrm{agcr}}\hm\subseteq \mathrm{ACT}_i^{\mathrm{ag}}$, 
$\mathrm{ACT}_j^{\mathrm{agcr}}\hm\subseteq \mathrm{ACT}_j^{\mathrm{ag}}$, 
$\mathrm{ACT}_i^{\mathrm{agcr}}, \mathrm{ACT}_j^{\mathrm{agcr}} \hm\subseteq 
\mathrm{ACT}^{\mathrm{agcr}}$, 
  причем $\mathrm{ACT}_i^{\mathrm{ag}}$ и~$\mathrm{ACT}_j^{\mathrm{ag}}$~--- множества действий агентов 
$\mathrm{ag}_i$ и~$\mathrm{ag}_j$ соответственно, 
а~$\mathrm{ACT}^{\mathrm{agcr}}$~--- упорядоченное по отношению предпочтения 
$\overset{\mathrm{prf}}{\prec}$ множество допустимых стратегий АС по 
разрешению противоречий, включающее стратегии переговоров, 
делегирования, голосования, самомодификации и игнорирования 
соответственно:
\begin{multline*}
\mathrm{ACT}^{\mathrm{agcr}} ={}\\
{}=\left( \!\left\{ 
\mathrm{act}^{\mathrm{agcr}}_{\mathrm{ig}},  
\mathrm{act}^{\mathrm{agcr}}_{\mathrm{sm}}, 
\mathrm{act}^{\mathrm{agcr}}_{\mathrm{vot}}, 
\mathrm{act}^{\mathrm{agcr}}_{\mathrm{del}}, 
\mathrm{act}^{\mathrm{agcr}}_{\mathrm{neg}}\right\},\overset{\mathrm{prf}} 
{\prec}\right).
\end{multline*}
  
  После формирования матрицы $\mathbf{CNF}$ вычисляется общий 
показатель напряженности конфликта в \mbox{ГиИМАС} 
$\mathrm{cnf}^{\mathrm{himas}}$~\cite{3-kir}. На этом же этапе, если 
пользователь установил перед началом работы \mbox{ГиИМАС} 
необходимость визуализации работы коллектива агентов, запускается метод 
визуализации конфликта (МВК)~\cite{5-kir} и отображаются 
$\mathrm{gd}^{\mathrm{himas}}$ и~$\mathrm{cnf}^{\mathrm{himas}}$ 
с~пороговыми значениями (по умолчанию~0 и~0,5 соответственно). Затем 
в~зависимости от значений $\mathrm{gd}^{\mathrm{himas}}$ 
и~$\mathrm{cnf}^{\mathrm{himas}}$ выполняется функция <<стимуляция 
конфликтов>> или <<разрешение конфликтов>>, а~также визуализация этих 
процессов (при необходимости). Если в результате выполнения одной из этих 
функций активируется признак завершения работы \mbox{ГиИМАС}, то 
инициализируется процедура окончания работы системы.
  
  Последовательность шагов функции <<управление конфликтом>> 
(ПШФУК) АФ представлена в~\cite{3-kir}. Визуализация стимуляции 
конфликта между агентами как модификация \mbox{ПШФУК}, дополненная 
запуском МВК~\cite{5-kir} на базе матрицы $\mathbf{CNF}$, а~также 
визуализацией $\mathrm{gd}^{\mathrm{himas}}$ 
и~$\mathrm{cnf}^{\mathrm{himas}}$ с их пороговыми значениями, описана 
в~\cite{6-kir}. Алгоритм снижения интенсивности и разрешения конфликтов 
(АСИРК) в~\mbox{ГиИМАС} предлагается в~\cite{4-kir}. Метод визуализации 
разрешения конфликтов (МВРК) по всем парам конфликтующих агентов 
работает параллельно АСИРК. Рассмотрим подробнее предлагаемый метод 
МВРК в~\mbox{ГиИМАС}.
  
\section{Метод визуализации снижения интенсивности 
и~разрешения конфликтов}

  Если пользователь установил перед запуском работы \mbox{ГиИМАС} флаг 
<<необходимости визуализации динамики возможного конфликта>>, то 
в~рамках работы системы запускается не функция <<управление 
конфликтом>>, а~ее модификация, включающая визуализацию и запуск 
МВК~\cite{6-kir}. По аналогии с~отоб\-ра\-же\-ни\-ем вероятности перехода 
конфликта с~одного уровня на другой в~работе~\cite{13-kir}, разрешенные 
конфликты и~стратегия разрешения противоречий, выбранная для применения 
между парой агентов на очередном шаге работы алгоритма АСИРК, 
визуализируются с помощью матрицы. В~связи с~этим для последующей 
работы МВРК перед запуском модификации ПШФУК необходимо 
сформировать матрицу~$\mathbf{V}_{m\times m}$ размерности~$m$ (равна 
мощности множества коллектива агентов), по диагонали которой стоят~0 
($v_{ii}\hm=0$), а~на остальных позициях~---~1 ($\forall\,i\not=j, i,j\hm\in [1,m]$, 
$v_{ij}\hm=1$), и~установить: $k\hm=1$; $\mathbf{V}_{m\times m}$. Рассмотрим последовательность шагов МВРК. 
  
  \textbf{Первый шаг} реализуется после запуска функции <<разрешение 
конфликтов>> на паре агентов $\mathrm{ag}_i$, $\mathrm{ag}_j$ (субъектов 
конфликта). Его исполнение связано с верхней границей размерности малого 
коллектива специалистов относительно успешного руководства группой~--- 
соответствует <<магическому числу>> Дж.~Миллера ($7 \hm\pm 2$), так как 
при численности свыше~10~человек возрастают число подгрупп и вероятность 
противостояния лицу, при\-ни\-ма\-юще\-му решения, осложняется координация. 
Однако, чтобы учесть все возможные варианты и улучшить восприятие 
пользователем визуализации смены страте-\linebreak гий разрешения конфликтов между 
агентами и~наличие возможных подгрупп, проверяем условие\linebreak <<$m\hm> 
10$>>: если принимает значение <<истина>>, то запускаем функцию 
<<выделение подгрупп конфликтующих агентов>> (ФВПКА), иначе переходим 
ко\linebreak второму шагу. Для того чтобы выделить возможные подгруппы  
не\-конф\-лик\-ту\-ющих/сла\-бо\-конф\-лик\-ту\-ющих между собой агентов, 
воспользуемся \mbox{алгоритмом} поиска сообществ IS$^2$~\cite{14-kir, 15-kir}, 
приведенном в~обзоре~\cite{16-kir} и~применяющемся к~взвешенным 
неориентированным графам. IS$^2$ учитывает возможность принадлежности 
вершины нескольким сообществам и комбинирует алгоритмы 
последовательного обхода (Iterative Scan, IS) и удаления по рангу (Rank 
Removal, RaRe). Если перед началом работы ГиИМАС пользователь установил 
флаг <<выявление подгрупп>>, то ФВПКА запускается при любом~$m$ 
и~включает в себя следующие шаги: 
  \begin{itemize}
\item формирование матрицы $\mathbf{PPK}$ на базе матриц $\mathbf{CP}$ 
и~$\mathbf{CPR}$, полученных из $\mathbf{CNF}$ в процессе работы МВК 
($\mathrm{cp}_{ij}$ описывает величину напряженности  
проб\-лем\-но-ори\-ен\-ти\-ро\-ван\-но\-го конфликта между агентами, 
а~$\mathrm{cpr}_{ij}$~--- про\-цес\-сно-ори\-ен\-ти\-ро\-ван\-но\-го конфликта):
$$
\mathrm{ppk}_{ij}= \begin{cases}
0\,, &\hspace*{-30mm}\mbox{если } i=j\\
& \hspace*{-38mm}\mbox{или } \left(0{,}5\left( \mathrm{cp}^2_{ij}+\mathrm{cpr}^2_{ij}\right)\right)^{0{,}5} \!>\! \mathrm{cnfin}^{\mathrm{htr}}\,;\\
10\,000, &\hspace*{-25mm}\mbox{если } \mathrm{cpr}_{ij}=\mathrm{cp}_{ij}=0\,;\\
\left( 0{,}5\left( 
\mathrm{cp}^2_{ij}+\mathrm{cpr}_{ij}^{2}\right)\right)^{-0{,}5} & \hspace*{-1mm}\mbox{в\ 
противном}\\
&\hspace*{-1mm}\mbox{случае,}
\end{cases}
$$
где $\mathrm{cnfin}^{\mathrm{htr}}$~--- верхний порог интенсивности 
конфликта (по умолчанию~0,5). 
  
  Чем выше напряженность конфликта между агентами, тем меньше вес ребра. 
Если вес равен нулю, то ребро отсутствует, в частности если между агентами 
имеет место сильный конфликт (напряженность выше порогового значения);
\item запуск алгоритма RaRe на матрице $\mathbf{PPK}$: 
\begin{enumerate}[(1)]
\item подсчет рангов 
всех вершин (возможны разные подходы~\cite{7-kir, 8-kir}, возьмем за меру 
степень вершины);
\item удаление всех высокоранговых вершин с~\mbox{целью} 
получения ядер (размерность по умолчанию~---~2, можно корректировать) 
будущих сообществ; 
\item последовательное добавление каждой удаленной 
вершины к ядрам~--- если добавление приводит к~увеличению весовой 
функции~\cite{7-kir} ($W \hm= W(C)/(W(C) \hm+ W_{\mathrm{out}}(C))$, где 
$W(C)$~--- сумма весов ребер внутри сообщества~$C$; 
$W_{\mathrm{out}}(C)$~--- сумма весов ребер вне сообщества~$C$), то оставляем 
вершину в сообществе. Вершина может добавляться к~нескольким ядрам, 
образуя пересекающиеся сообщества;
\end{enumerate}
\item запуск алгоритма IS для уточнения результата, полученного от RaRe: 
выбирается произвольная вершина $\mathrm{ppk}_i$ в~качестве начального сообщества, 
к~которой на каждом шаге добавляются другие вершины графа до тех пор, 
пока улучшается значение весовой функции~$W$. Однако добавляемые 
вершины выбираются не из всего графа, а~только из сообщества, полученного 
с~по\-мощью RaRe и~содержащего вершину $\mathrm{ppk}_i$, а~также из соседних 
сообществ;
\item упорядочение строк и столбцов матрицы $\mathbf{V}_{m\times m}$ 
в~соответствии с~полученным разбиением~$C$ коллектива агентов на 
сообщества.
\end{itemize}
  
  \textbf{Второй шаг}~--- отображение последней сохраненной визуализации 
(уклад\-ки графа) конфликта и~под ней отображение матрицы 
$\mathbf{V}_{m\times m}$ как таб\-ли\-цы  (рис.~1). 
  
  На рис.~1 $i$-е строка и столбец таб\-ли\-цы, отоб\-ра\-жа\-ющей мат\-ри\-цу, 
подписаны значением $\mathrm{id}_i^{\mathrm{ag}}$. На примере коллектива 
агентов, решающего задачу диагностики рака поджелудочной железы, 
$\mathrm{id}_{\mathrm{id}}^{\mathrm{ag}}=$\;$\{$<<АХ>>, <<АОНЛ>>, 
<<АЛПР-Т>>, <<АСУЗИ>>, <<АВЛД>>, <<АСЛД>>$\}$. По диагонали 
отображения матрицы располагаются черные квадраты (соответствуют 
$v_{ii}\hm=0$ в~$\mathbf{V}_{m\times m}$), так как сам с собой агент не 
конфликтует и эта область не интересна, а~остальные элементы 
матрицы~$\mathbf{V}_{m\times m}$ отображаются белыми квадратами 
($v_{ij}\hm=1$, $i\not= j$).
  
  На графе конфликта (см.\ рис.~1) толщиной и~цветом линии (от свет\-ло-се\-ро\-го 
до черного) отображается величина среднего квадратического напряженностей 
конфликтов между агентами (сплошной линией, если превалирует  
проб\-лем\-но-ори\-ен\-ти\-ро\-ван\-ный конфликт; штриховой~--- если  
про\-цес\-сно-ори\-ен\-ти\-ро\-ван\-ный). Каждая вершина подписана 
идентификатором соответствующего агента.
   Слева от графа отображены $\mathrm{gd}^{\mathrm{himas}}$ 
   и~$\mathrm{cnf}^{\mathrm{himas}}$~\cite{6-kir}~--- их пороги, значения, цвет 
   и~символы (вычисляются в~начале управ\-ле\-ния конфликтом~\cite{6-kir}). Для 
$\mathrm{gd}^{\mathrm{himas}}$: \raisebox{-1pt}[0pt][0pt]{\mbox{%
     \epsfxsize=3.8mm 
    \epsfbox{rum-t-1.eps}
     }}  темно-серого цвета, если выше нуля, и \raisebox{-
1pt}[0pt][0pt]{\mbox{%
     \epsfxsize=3.8mm 
    \epsfbox{rum-t-2.eps}
     }} свет\-ло-се\-ро\-го цвета, если ниже или равно нулю. Для 
$\mathrm{cnf}^{\mathrm{himas}}$: \raisebox{-1pt}[0pt][0pt]{\mbox{%
     \epsfxsize=3.8mm 
    \epsfbox{rum-t-3.eps}
     }}  тем\-но-се\-ро\-го цвета, если меньше порогового значения 
(определяется в ходе тестирования системы, по умолчанию равно~0,5), 
\raisebox{-1pt}[0pt][0pt]{\mbox{%
     \epsfxsize=3.8mm 
    \epsfbox{rum-t-4.eps}
     }}  свет\-ло-се\-ро\-го цвета, если равно порогу, и \raisebox{-
1pt}[0pt][0pt]{\mbox{%
     \epsfxsize=3.8mm 
    \epsfbox{rum-t-5.eps}
     }}, если больше порогового значения. Под 
$\mathrm{gd}^{\mathrm{himas}}$ и~$\mathrm{cnf}^{\mathrm{himas}}$ 
расположена неактивная пустая иконка стимуляции  
конфликта~\cite{3-kir, 6-kir}.

\end{multicols}

\begin{figure*} %fig1
\vspace*{1pt}
  \begin{center}  
    \mbox{%
\epsfxsize=133.521mm
\epsfbox{rum-1.eps}
}
\end{center}
\vspace*{-9pt}
\Caption{Визуализация разрешения конфликта между агентами: АХ~--- хирург; АОНЛ~--- 
онколог по нехирургическому лечению; АЛПР-Т~--- лицо, принимающее решение 
(терапевт); АСУЗИ~--- специалист по ультразвуковому исследованию; АВЛД~--- врач 
лабораторной диагностики; АСЛД~--- специалист по лучевой диагностике}
\end{figure*} 

\begin{multicols}{2}
  
  \textbf{Третий шаг.} После того как АСИРК~\cite{4-kir} запросит у конфликтующих 
агентов $\mathrm{ag}_i$ и~$\mathrm{ag}_j$ множества реализуемых ими СРП 
$\mathrm{ACT}_i^{\mathrm{agcr}}$ и $\mathrm{ACT}_j^{\mathrm{agcr}}$, 
сформирует на их базе упорядоченное множество (список) 
$\mathrm{ACT}_{ijc}^{\mathrm{agcr}}$ СРП между данной парой агентов 
и~выберет стратегию по правилу из~\cite{4-kir}, элементу~$v_{ij}$ 
матрицы~$\mathbf{V}_{m\times m}$, который соответствует паре 
конфликтующих агентов $\mathrm{ag}_i$ и~$\mathrm{ag}_j$, присваивается 
значение, соответствующее ситуации:
  \begin{itemize}
\item если $\mathrm{ACT}_{ijc}^{\mathrm{agcr}}=\varnothing$, т.\,е.\ конфликт не может быть разрешен и функция 
завершает свою ра-\linebreak боту, то $v_{ij}\hm=2$, а~в~соответствующей ячейке\linebreak 
таблицы белый квадрат заменяется на икон-\linebreak ку~<< \raisebox{-1pt}[0pt][0pt]{\mbox{%
 \epsfxsize=3.9mm 
  \epsfbox{rum-t-7.eps}
   }}>>; 
   
\item если $\mathrm{ACT}_{ijc}^{\mathrm{agcr}}\not= \varnothing$, то 
в~зависимости от запущенной СРП на паре агентов: 
$$
v_{ij}=\begin{cases}
3 (\mbox{<<}\raisebox{-1pt}[0pt][0pt]
{\mbox{%
   \epsfxsize=3.9mm 
  \epsfbox{rum-t-8.eps}
   }}\mbox{>>}) & \mbox{--- переговоры};\\
   4 (\mbox{<<}\raisebox{-1pt}[0pt][0pt]{\mbox{%
   \epsfxsize=3.9mm 
  \epsfbox{rum-t-9.eps}
   }}\mbox{>>}) & \mbox{--- делегирование};\\
    5 (\mbox{<<}\raisebox{-1pt}[0pt][0pt]{\mbox{%
   \epsfxsize=3.9mm 
  \epsfbox{rum-t-10.eps}
   }}\mbox{>>}) & \mbox{--- голосование}; \\
   6 (\mbox{<<}\raisebox{-1pt}[0pt][0pt]{\mbox{%
   \epsfxsize=3.9mm 
  \epsfbox{rum-t-11.eps}
   }}\mbox{>>}) & \mbox{--- самомодификация}; \\
   7 (\mbox{<<}\raisebox{-1pt}[0pt][0pt]{\mbox{%
   \epsfxsize=3.3mm 
  \epsfbox{rum-t-12.eps}
   }}\mbox{>>}) & \mbox{--- игнорирование}. 
   \end{cases}
   $$
   \end{itemize}
  
  Таким образом, множество
  $$
  \mathrm{ve}= \{0, 1, 2, 3, 4, 5, 6, 7\}
  $$ 
  биективно отображается 
на множество 
$$
\mathrm{sve}=\left\{ \raisebox{-1pt}[0pt][0pt]{\mbox{%
     \epsfxsize=3.3mm 
    \epsfbox{rum-t-13.eps}
     }},  \raisebox{-1pt}[0pt][0pt]{\mbox{%
        \epsfxsize=3.3mm 
       \epsfbox{rum-t-14.eps}
        }}, \raisebox{-1pt}[0pt][0pt]{\mbox{%
      \epsfxsize=3.9mm 
     \epsfbox{rum-t-7.eps}
      }}, \raisebox{-1pt}[0pt][0pt]{\mbox{%
      \epsfxsize=3.9mm 
     \epsfbox{rum-t-8.eps}
      }}, \raisebox{-1pt}[0pt][0pt]{\mbox{%
      \epsfxsize=3.9mm 
     \epsfbox{rum-t-9.eps}
      }} , \raisebox{-1pt}[0pt][0pt]{\mbox{%
     \epsfxsize=3.9mm 
    \epsfbox{rum-t-10.eps}
     }}, \raisebox{-1pt}[0pt][0pt]{\mbox{%
      \epsfxsize=3.9mm 
     \epsfbox{rum-t-11.eps}
      }}, \raisebox{-1pt}[0pt][0pt]{\mbox{%
      \epsfxsize=3.3mm 
     \epsfbox{rum-t-12.eps}
      }}\right\}.
      $$
      
       \begin{figure*}[b] %fig2
  \vspace*{2pt}
  \begin{center}  
    \mbox{%
\epsfxsize=133.973mm
\epsfbox{rum-2.eps}
}
\end{center}
\vspace*{-9pt}
  \Caption{Визуализация промежуточного этапа разрешения конфликта между агентами}
   \end{figure*}
  
 \textbf{Четвертый шаг}~--- сохранить визуализацию, полученную на $k$-м цикле 
управления конфликтом: 
  \begin{itemize}
\item сохранить в $k$-й элемент $\mathbf{VRK}_{1\times K}$ 
матрицу~$\mathbf{V}_{m\times m}$ ($\mathrm{vrk}_{1k}\hm= \mathbf{V}_{m\times m}$); 
\item сохранить укладку графа конфликтующих агентов, полученную 
в~результате работы МВК, как $k$-й элемент $\mathbf{VK}_{1\times K}$; 
\item сохранить значение $\mathrm{gd}^{\mathrm{himas}}$ 
и~$\mathrm{cnf}^{\mathrm{himas}}$ как очередной $k$-й элемент множеств 
$\mathrm{VGD}$ и~$\mathrm{VCNF}$ соответственно и установить $k\hm = 
k\hm+1$. 
\end{itemize}
  
  Матрицы $\mathbf{VRK}_{1\times K}$ и~$\mathbf{VK}_{1\times K}$, 
а~также множества $\mathrm{VGD}$ и~$\mathrm{VCNF}$ позволят 
пользователю при необходимости просмотреть весь визуальный ряд\linebreak\vspace*{-10pt}

\columnbreak

\noindent
 конфликта 
между агентами, смену напряженности и СРП в динамике или пошагово.
  
  Если выбрана стратегия <<игнорирование>>, то конфликт считается 
разрешенным, АСИРК и~\mbox{МВРК} завершают работу, иначе АФ ожидает  
со\-об\-ще\-ний-ре\-ше\-ний от АС, которые они выработают после применения 
соответствующей стратегии. Получив такие сообщения, АФ вновь 
идентифицирует конфликт между парой агентов~\cite{4-kir} согласно АСИРК 
и~запускает МВРК.
  
  Пример визуализации промежуточного этапа разрешения конфликта 
представлен на рис.~2. В~сравнении с рис.~1 видно, что конфликты между 
агентами АВЛД и АСЛД разрешены и~на данном этапе для агентов АВЛД 
и~АСУЗИ выбрана стратегия переговоров.
  
 

\section{Заключение}
  
  Моделирование и визуализация процессов разрешения конфликтов агентов 
избавляет пользователя от необходимости ручного анализа и~выбора\linebreak 
альтернативы из предлагаемого множества вариантов, тем самым повышая 
эффективность работы \mbox{ГиИМАС}. В~работе предложен новый метод 
визуализации разрешения конфликтов, ба\-зи\-ру\-ющий\-ся на алгоритме 
АСИРК~\cite{4-kir}, методе визуализации конфликта~\cite{5-kir} и алгоритме 
поиска сообществ IS$^2$~\cite{14-kir, 15-kir}. Метод визуализации разрешения конфликтов в~\mbox{ГиИМАС} 
интегрирует пиктографическую, визуальную (графы и таблицы) и численную 
информации, детально отображая процессы снижения интенсивности 
и~разрешения конфликта агентов. Данный метод предоставляет возможность 
отследить изменение напряженности между агентами, используемые стратегии 
разрешения конфликтов, а также наличие подгрупп в коллективе агентов.
  
{\small\frenchspacing
 {%\baselineskip=10.8pt
 %\addcontentsline{toc}{section}{References}
 \begin{thebibliography}{99}
\bibitem{1-kir}
\Au{Листопад С.\,В., Кириков~И.\,А.} Моделирование конфликтов агентов в гибридных 
интеллектуальных многоагентных сис\-те\-мах~// Сис\-те\-мы и средства информатики, 2019. 
Т.~29. №\,3. С.~139--148. doi: 10.14357/ 08696527190312.
\bibitem{2-kir}
\Au{Листопад С.\,В., Кириков~И.\,А.} Метод идентификации конфликтов агентов 
в~гибридных интеллектуальных многоагентных сис\-те\-мах~// Сис\-те\-мы и средства 
информатики, 2020. Т.~30. №\,1. С.~56--65. doi: 10.14357/ 08696527200105.

\bibitem{5-kir} %3
\Au{Румовская С.\,Б., Кириков~И.\,А.} Метод визуального представления конфликтов 
в~гибридных интеллектуальных многоагентных сис\-те\-мах~// Информатика и её 
применения, 2020. Т.~14. Вып.~4. С.~77--82. doi: 10.14357/19922264200411.

\bibitem{3-kir} %4
\Au{Листопад С.\,В., Кириков~И.\,А.} Стимуляция конфликтов агентов в гибридных 
интеллектуальных многоагентных сис\-те\-мах~// Сис\-те\-мы и средства информатики, 2021. 
Т.~31. №\,2. С.~47--58. doi: 10.14357/ 08696527210205.


\bibitem{6-kir} %5
\Au{Румовская С.\,Б., Кириков~И.\,А.} Метод визуализации стимуляции конфликтов в 
гибридных интеллектуальных многоагентных сис\-те\-мах~// Информатика и~её применения, 
2021. Т.~15. Вып.~3. С.~75--82. doi: 10.14357/19922264210310.

\bibitem{4-kir} %6
\Au{Листопад С.\,В., Кириков~И.\,А.} Разрешение конфликтов в гибридных 
интеллектуальных многоагентных сис\-те\-мах~// Информатика и её применения, 2022. 
Т.~16. Вып.~1. С.~54--60.

\bibitem{9-kir}  %7
\Au{Shaw M.} Group dynamics: The psychology of small group behavior.~--- New York, NY, 
USA: McGraw-Hill, 1981. 531~p.

\bibitem{7-kir} %8
\Au{Андреева Г.\,М.} Социальная психология.~--- М.: Аспект-пресс, 2009. 393~с.
\bibitem{8-kir} %9
\Au{Brown R., Pehrson~S.} Group processes: Dynamics with and between groups.~--- 3rd ed.~--- 
Oxford: Wiley-Blackwell, 2019. 344~p. doi: 10.1002/9781118719244.

\bibitem{10-kir}
\Au{Емельянов С.\,М.} Конфликтология.~--- 4-е изд.~--- М.: Юрайт, 2018. 322~с.
\bibitem{11-kir}
\Au{Behfar K., Peterson~R., Mannix~E., Trochim~W.} The critical role of conflict resolution in 
teams: A~close look at the links between conflict type, conflict management strategies, and team 
outcomes~// J.~Appl. Psychol., 2008. Vol.~93. No.\,1. P.~170--188. doi:  
10.1037/0021-9010.93.1.170.
\bibitem{12-kir}
\Au{Анцупов А.\,Я., Баклановский~С.\,В.} Конфликтология в~схемах и~комментариях.~--- 2-е изд.~--- СПб.: Питер, 2009. 304~с.
\bibitem{13-kir}
\Au{Cusack J.\,J., Bradfer-Lawrence~T., Baynham-Herd~Z., \textit{et al.}} Measuring the intensity 
of conflicts in conservation~// Conserv. Lett., 2021. Vol.~14. Iss.~3. Art. e12783. 11~p. doi: 
10.1111/conl.12783.
\bibitem{14-kir}
\Au{Baumes J., Goldberg~M.\,K., Krishnamoorthy~M.\,S., Magdon-Ismail~M., Preston~N.} 
Finding communities by clustering a graph into overlapping subgraphs~// Conference (International)
 on Applied Computing Proceedings.~--- IADIS Press, 2005. Vol.~1. P.~97--104. 
\bibitem{15-kir}
\Au{Baumes J., Goldberg~M., Magdon-Ismail~M.} Efficient identification of overlapping 
communities~// Intelligence and security informatics~/ Eds. P.\,B.~Kantor, G.~Muresan, F.\,S.~Roberts, \textit{et al.}~--- 
Lecture notes in computer science ser.~--- Berlin--Heidelberg: Springer, 2005. 
Vol.~3495. P.~27--36. doi: 10.1007/11427995\_3. 
\bibitem{16-kir}
\Au{Вирцева Н.\,С., Вишняков~И.\,Э., Иванов~И.\,П.} Способы выделения сообществ 
с~определенными типами отношений в графах на основе биллинговой информации~// 
Вестник МГТУ им.\ Н.\,Э.~Баумана. Сер. Приборостроение, 2021. Вып.~2(135). С.~4--22. doi: 
10.18698/0236-3933-2021-2-4-22. 
\end{thebibliography}

 }
 }

\end{multicols}

\vspace*{-3pt}

\hfill{\small\textit{Поступила в~редакцию 25.03.22}}

%\vspace*{8pt}

\newpage


\vspace*{-28pt}

%\hrule

%\vspace*{2pt}

%\hrule

%\vspace*{-2pt}

\def\tit{VISUAL REPRESENTATION OF~THE~DECREASE IN CONFLICT INTENSITY 
AND~ITS~RESOLUTION IN~HYBRID INTELLIGENT MULTIAGENT SYSTEMS}


\def\titkol{Visual representation of~the~decrease in conflict intensity 
and~its~resolution in~hybrid intelligent multiagent systems}


\def\aut{S.\,B.~Rumovskaya and~I.\,A.~Kirikov}

\def\autkol{S.\,B.~Rumovskaya and~I.\,A.~Kirikov}

\titel{\tit}{\aut}{\autkol}{\titkol}

\vspace*{-8pt}


\noindent
   Kaliningrad Branch of the Federal Research Center ``Computer Science and Control'' of the 
Russian Academy of Sciences, 5~Gostinaya Str., Kaliningrad 236000, Russian Federation


\def\leftfootline{\small{\textbf{\thepage}
\hfill INFORMATIKA I EE PRIMENENIYA~--- INFORMATICS AND
APPLICATIONS\ \ \ 2022\ \ \ volume~16\ \ \ issue\ 2}
}%
 \def\rightfootline{\small{INFORMATIKA I EE PRIMENENIYA~---
INFORMATICS AND APPLICATIONS\ \ \ 2022\ \ \ volume~16\ \ \ issue\ 2
\hfill \textbf{\thepage}}}

\vspace*{3pt} 
   
   
      
   
   \Abste{Many practical problems require a~collective solution ensuring pluralism of opinions, 
integration of private points of view, and reduction of errors. The authors propose to model the work 
of such groups of specialists with hybrid intelligent multiagent systems considering the peculiarities 
of their group dynamics. Such approach would provide improving the quality and efficiency of the 
solution as well as comprehensive consideration of the problem and the process of its overcoming 
including visualization of conflicts and processes of their management. The latter would provide a 
new information on conflict resolution both in the system and in the real group of specialists. The 
work is devoted to the development of a method for visualization of conflict resolution processes 
within the framework of hybrid intelligent multiagent systems with problem-oriented and process-oriented constructive conflicts.}
   
   \KWE{collective of specialists; conflict; visualization of the conflict resolution}
   
   
\DOI{10.14357/19922264220212}

%\vspace*{-16pt}

%\Ack
%\noindent




%\vspace*{4pt}

  \begin{multicols}{2}

\renewcommand{\bibname}{\protect\rmfamily References}
%\renewcommand{\bibname}{\large\protect\rm References}

{\small\frenchspacing
 {%\baselineskip=10.8pt
 \addcontentsline{toc}{section}{References}
 \begin{thebibliography}{99}
\bibitem{1-kir-1}
   \Aue{Listopad, S.\,V., and I.\,A.~Kirikov.} 2019. Modelirovanie konfliktov agentov 
v~gibridnykh intellektual'nykh mnogoagentnykh sistemakh [Modeling of agent conflicts in hybrid 
intelligent multiagent systems]. \textit{Sistemy i~Sredstva Informatiki~--- Systems and Means of 
Informatics} 29(3):139--148. doi: 10.14357/08696527190312.
\bibitem{2-kir-1}
   \Aue{Listopad, S.\,V., and I.\,A.~Kirikov.} 2020. Metod identifikatsii konfliktov agentov 
v~gibridnykh intellektual'nykh mnogoagentnykh sistemakh [Agent conflict identification method in 
hybrid intelligent multiagent systems]. \textit{Sistemy i~Sredstva Informatiki~--- Systems and 
Means of Informatics} 30(1):56--65. doi: 10.14357/08696527200105.

\bibitem{5-kir-1} %3
   \Aue{Rumovskaya, S.\,B., and I.\,A.~Kirikov.} 2020. Metod vi\-zu\-al'\-no\-go predstavleniya 
konfliktov v~gibridnykh intellektual'nykh mnogoagentnykh sistemakh [Conflict visual 
representation method in collective decision-making within hybrid intelligent multiagent systems]. 
\textit{Informatika i~ee Primeneniya~--- Inform. Appl.} 14(4):77--82. doi: 
10.14357/19922264200411. 

\bibitem{3-kir-1} %4
   \Aue{Listopad, S.\,V., and I.\,A.~Kirikov.} 2021. Stimulyatsiya konfliktov agentov 
v~gibridnykh intellektual'nykh mnogoagentnykh sistemakh [Stimulation of agent conflicts in hybrid 
intelligent multiagent systems]. \textit{Sistemy i~Sredstva Informatiki~--- Systems and Means of 
Informatics} 31(2):47--58. doi: 10.14357/08696527210205.

\bibitem{6-kir-1} %5
   \Aue{Rumovskaya, S.\,B., and I.\,A.~Kirikov.} 2021. Metod vi\-zu\-a\-li\-za\-tsii stimulyatsii konfliktov 
v~gibridnykh intellektual'nykh mnogoagentnykh sistemakh [Visual representation method for the 
conflict stimulation in hybrid\linebreak intelligent multiagent systems]. \textit{Informatika i~ee 
Primeneniya~--- Inform. Appl.} 15(3):75--82. doi: 10.14357/ 19922264210310. 
\bibitem{4-kir-1} %6
   \Aue{Listopad, S.\,V., and I.\,A.~Kirikov.} 2022. Razreshenie konfliktov v~gibridnykh 
intellektual'nykh mnogoagentnykh sistemakh [Resolving conflicts in hybrid intelligent multi-agent 
systems]. \textit{Informatika i~ee Primeneniya~--- Inform. Appl.} 16(1):54--60.

\bibitem{9-kir-1} %7
   \Aue{Shaw, M.} 1981. \textit{Group dynamics: The psychology of small group behavior}. New 
York, NY: McGraw-Hill. 531~p.
\bibitem{7-kir-1} %8
   \Aue{Andreeva, G.\,M.} 2009. \textit{Sotsial'naya psikhologiya} [Social psychology]. Moscow: 
Aspect-press. 393~p.
\bibitem{8-kir-1} %9
   \Aue{Brown, R., and S.~Pehrson.} 2019. \textit{Group processes: Dynamics with and between 
groups}. 3rd ed. Oxford: Wiley-Blackwell. 344~p.  doi: 10.1002/9781118719244.

\bibitem{10-kir-1}
   \Aue{Emel'yanov, S.\,M.} 2018. \textit{Konfliktologiya} [Conflictology]. Moscow: Yurayt. 
322~p. 
\bibitem{11-kir-1}
   \Aue{Behfar, K., R.~Peterson, E.~Mannix,  and W.~Trochim.} 2008. The critical role of conflict 
resolution in teams: A~close look at the links between conflict type, conflict management 
strategies, and team outcomes. \textit{J.~Appl. Psychol}. 93(1):170--88. doi:  
10.1037/0021-9010.93.1.170.
\bibitem{12-kir-1}
   \Aue{Antsupov, A.\,Ya., and S.\,V.~Baklanovskiy.} 2009. \textit{Konfliktologiya v~skhemakh 
i~kommentariyakh} [Conflictology in schemes and comments]. St.\ Petersburg: Piter. 304~p.
\bibitem{13-kir-1}
   \Aue{Cusack, J.\,J., T.~Bradfer-Lawrence, Z.~Baynham-Herd, \textit{et al.}} 2021. Measuring 
the intensity of conflicts in conservation. \textit{Conserv. Lett.} 14(3):e12783. 11~p. doi: 
10.1111/ conl.12783.
\bibitem{14-kir-1}
   \Aue{Baumes, J., M.\,K.~Goldberg, M.\,S.~Krishnamoorthy, \textit{et al.}} 2005. Finding 
communities by clustering a graph into overlapping sub-graphs. \textit{Conference (International) 
on Applied Computing Proceedings}. IADIS. 1:97--104. 
   \bibitem{15-kir-1}
   \Aue{Baumes, J., M.~Goldberg, and M.~Magdon-Ismail.} 2005. Efficient identification of 
overlapping communities. \textit{Conference (International) on Intelligence and Security 
Informatics Proceedings}. Eds. P.\,B.~Kantor, G.~Muresan, F.\,S.~Roberts, \textit{et al}. 
Lecture notes in computer science ser. Berlin--Heidelberg: Springer, 2005. 
3495:27--36. doi: 10.1007/11427995\_3. 
   \bibitem{16-kir-1}
   \Aue{Virtseva, N.\,S., I.\,E.~Vishnyakov, and I.\,P.~Ivanov.} 2021. Sposoby vydeleniya 
soobshchestv s~opredelennymi tipami otnosheniy v~grafakh na osnove billingovoy informatsii 
[Methods of detecting communities with certain relationship types in graphs using billing 
information]. \textit{Vestnik MGTU im. N.\,E.~Baumana. Ser. Priborostroyeniye} [Herald of the
Bauman Moscow State Technical University. Ser. Instrument Engineering] 2(135):4--22. doi:  
10.18698/ 0236-3933-2021-2-4-22.
   \end{thebibliography}

 }
 }

\end{multicols}

\vspace*{-6pt}

\hfill{\small\textit{Received March 25, 2022}}
   
   \Contr
   
   \noindent
   \textbf{Rumovskaya Sophiya B.} (b.\ 1985)~--- Candidate of Science (PhD) in technology, 
scientist, Kaliningrad Branch of the Federal Research Center ``Computer Science and Control'' of 
the Russian Academy of Sciences, 5~Gostinaya Str., Kaliningrad 236000, Russian Federation; 
\mbox{sophiyabr@gmail.com}
   
   \vspace*{3pt}
   
   \noindent
   \textbf{Kirikov Igor A.} (b.\ 1955)~--- Candidate of Science (PhD) in technology, director, 
Kaliningrad Branch of the Federal Research Center ``Computer Science and Control'' of the Russian 
Academy of Sciences, 5~Gostinaya Str., Kaliningrad 236000, Russian Federation; 
\mbox{baltbipiran@mail.ru}

\label{end\stat}

\renewcommand{\bibname}{\protect\rm Литература}        %12 
\def\stat{beschastnyy}

\def\tit{АНАЛИЗ ПЛОТНОСТИ БАЗОВЫХ СТАНЦИЙ 5G~NR ДЛЯ~ПРЕДОСТАВЛЕНИЯ УСЛУГ 
ВИРТУАЛЬНОЙ И~ДОПОЛНЕННОЙ РЕАЛЬНОСТИ$^*$}

\def\titkol{Анализ плотности базовых станций 5G NR для предоставления услуг 
виртуальной и~дополненной реальности}

\def\aut{В.\,А.~Бесчастный$^1$, Д.\,Ю.~Острикова$^2$, С.\,Я.~Шоргин$^3$, 
Д.\,А.~Молчанов$^4$, Ю.\,В.~Гайдамака$^5$}

\def\autkol{В.\,А.~Бесчастный, Д.\,Ю.~Острикова, С.\,Я.~Шоргин и~др.}
%$^3$,  Д.\,А.~Молчанов$^4$, Ю.\,В.~Гайдамака$^5$}

\titel{\tit}{\aut}{\autkol}{\titkol}

\index{Бесчастный В.\,А.}
\index{Острикова Д.\,Ю.}
\index{Шоргин С.\,Я.}
\index{Молчанов Д.\,А.}
\index{Гайдамака Ю.\,В.}
\index{Beschastnyi V.\,A.}
\index{Ostrikova D.\,Yu.}
\index{Shorgin S.\,Ya.}
\index{Moltchanov D.\,A.}
\index{Gaidamaka Yu.\,V.}


{\renewcommand{\thefootnote}{\fnsymbol{footnote}} \footnotetext[1]
{Публикация выполнена при финансовой поддержке РНФ (проект 22-29-00694).}}


\renewcommand{\thefootnote}{\arabic{footnote}}
\footnotetext[1]{Российский университет дружбы народов, beschastnyy-va@rudn.ru}
\footnotetext[2]{Российский университет дружбы народов, ostrikova-dyu@rudn.ru}
\footnotetext[3]{Федеральный исследовательский центр <<Информатика и~управление>> Российской академии наук, 
\mbox{sshorgin@ipiran.ru}}
\footnotetext[4]{  Университет Тампере, Финляндия, dmitri.moltchanov@tuni.fi}
\footnotetext[5]{Российский университет дружбы народов; Федеральный исследовательский центр <<Информатика 
и~управление>> Российской академии наук, \mbox{gaydamaka-yuv@rudn.ru}}

\vspace*{-6pt}

  
  

  

  \Abst{Технология пятого поколения <<новое радио>> (5G New Radio, 5G NR), 
работающая в~диапазоне частот миллиметрового диапазона (mmWave), разработана для 
поддержки ресурсоемких приложений, требующих чрезвычайно высоких скоростей на 
уровне радиоинтерфейса. В~сис\-те\-мах NR использование антенных решеток, 
формирующих особые узкие диаграммы направленности излучения, позволяет избегать 
высоких потерь и~помех при передаче сигнала, но в~то же время сокращает площадь 
покрытия отдельно взятого луча, а~следовательно, и~число многоадресных пользователей, 
которые могут быть обслужены с~его помощью. В~результате требуются эффективные 
алгоритмы доставки данных для поддержки таких услуг как в~традиционных сетях 5G NR, так и~в~сетях
 на базе беспилотных летательных аппаратов (БПЛА). В~работе рассматривается 
передача потоковых данных для услуг виртуальной ре\-аль\-ности с~использованием технологии 
масштабируемого видеокодирования, которая использует возможности многоадресной 
передачи для предоставления базового слоя с~низким качеством разрешения получаемого 
контента и~одноадресной передачи для предоставления дополнительных слоев 
с~повышенным качеством. С~использованием аппарата стохастической геометрии и~теории 
массового обслуживания разработан метод, позволяющий оценить минимальную плотность 
развертывания базовых станций (БС) mmWave NR для обеспечения заданной 
производительности многослойных услуг с~многоадресной передачей в~зависимости от их 
различных требований и~структуры, а также от плотности расположения абонентских 
терминалов.}
   
  \KW{5G; <<новое радио>>; миллиметровый диапазон; виртуальная реальность; 
многоадресные соединения; масштабируемое видеокодирование; кластеризация}

\DOI{10.14357/19922264220213}
  
\vspace*{-3pt}


\vskip 10pt plus 9pt minus 6pt

\thispagestyle{headings}

\begin{multicols}{2}

\label{st\stat}
  
\section{Введение}
  
  В настоящее время консорциум 3GPP уже завершил основные этапы 
стандартизации технологии NR в~версиих~15 и~~16~[1]. Такие сис\-те\-мы, 
ра\-бо\-та\-ющие как в~микроволновом, так и~в~миллиметровом диапазонах, 
обещают обеспечить чрезвычайно высокие скорости передачи данных на 
уровне радиоинтерфейса~[2]. На текущий момент внимание как 3GPP, так 
и~исследовательского сообщества, сосредоточено на вопросе предоставления 
дополнительных услуг поверх нового уникального интерфейса радиодоступа 
с~возможностью многоадресной передачи~[3]. 
  
  В работе рассматривается многослойная услуга виртуальной реальности 
(Virtual Reality, VR), где базовый слой, передача которого предполагается 
многоадресной, обеспечивает самое низкое качество воспроизведения видео, 
а~каждый новый слой содержит дополнительные данные для повышения качества 
воспроизведения по технологии масштабируемого видеокодирования. По 
запросу нового пользователя VR-услу\-га должна быть предостав\-ле\-на с~базовым 
уровнем качества, а затем качество восприятия (Quality of Experience, QoE) 
можно улучшить, добавив дополнительные слои~[4]. Такая возможность 
зависит от полосы пропускания и~зоны покрытия. Дополнительный слой может 
содержать альтернативный контент или данные для улучшения качества 
текущего воспроизведения, его передача моделируются с~помощью 
одноадресной сессии с~собственным фиксированным требованием к~ресурсу. 
  
  Большинство проведенных на данный момент исследований для 
многослойных мно\-го\-ад\-рес\-ных/од\-но\-ад\-рес\-ных услуг сосредоточены 
на оптимизации уже развернутой системы для заданных параметров качества 
обслуживания~[5--7]. Однако для сетевых операторов не менее важен вопрос 
оценки требуемой плотности БС NR для заданной стохастической нагрузки 
трафика в~заданной об\-ласти. Помимо наземных сис\-тем плот\-ность 
расположения точек доступа имеет еще большее значение для сис\-тем, 
использующих БПЛА~[8]. В~этом случае БПЛА могут выступать как в~качестве 
потребителя услуги, так и~в~качестве поставщика, предоставляя услугу 
некоторой группе пользователей. Именно на решение этих проб\-лем направлена 
данная работа, где с~использованием аппарата стохастической 
геометрии и~тео\-рии массового обслуживания разработана математическая модель для 
расчета доли многослойных многоадресных VR-сес\-сий, которые могут быть 
обслужены в~зависимости от параметров точки доступа и~нагрузки на систему.

\vspace*{-6pt}

\section{Системная модель}

\vspace*{-2pt}
  
  Рассматривается зона покрытия БС NR в~виде сектора радиуса~$R$, который 
рассчитывается в~соответствии с~моделью распространения сигнала 
в~миллиметровом диапазоне~[9]. Сота обслуживается тремя антеннами, каждая 
из которых покрывает сектор с~центральным углом~120$^\circ$. Высота БС 
фиксирована и~равна~$h_A$, пользовательское устройство (ПУ) находится на 
высоте~$h_U$. Система функционирует на рабочей частоте~$f_c$, при этом 
каждая БС имеет в~своем распоряжении~$B$~ресурсных блоков. Предположим 
наличие в~соте потенциальных блокаторов сигнала~--- людей, распределенных 
случайным образом в~соответствии с~пуассоновским точечным процессом 
(ПТП) в~$\mathrm{Re}^2$ с~плот\-ностью~$\lambda_B$, блокирующих своим телом пути 
распространения  сигнала между БС и~ПУ. Блокаторы моделируются как 
цилиндры радиусом~$r_B$ и~высотой $h_B\hm > h_U$. 
  
  В работе рассматривается VR-услу\-га с~четырьмя уровнями качества 
(слоями): одним базовым и~тремя дополнительными. Обозначим требуемую 
скорость передачи данных базового слоя~$d_M$. Этот слой предоставляется 
всем пользователям услуги. В~то же время все пользователи пытаются 
повысить качество услуги и~получить дополнительные слои данных 
с~требованиями $d_{U,l}$, $l\hm=1,2,3$. Предполагается, что процесс 
поступления пользовательских запросов является пуассоновским  
с~па\-ра\-мет\-ром~$\Lambda$, а~длительности VR-сес\-сий имеют 
экспоненциальное распределение с~па\-ра\-мет\-ром~$\mu$.
 
 При поступлении запроса на предоставление услуги от ПУ из зоны, не 
покрытой многоадресной сессией, организуется новая многоадресная сессия, 
и~пользователь всегда получает базовый слой данных. Если же ПУ находится 
в~зоне действия уже установленной многоадресной сессии, оно присоединяется 
к~ней, не требуя дополнительных ресурсов. Далее, если на БС достаточно 
ресурсов для загрузки дополнительных слоев, инициируется их передача.

\vspace*{-6pt}

\section{Математическая модель}

\vspace*{-4pt}
  
  В данном разделе приводится аналитический метод группировки 
пользователей многоадресной сессии, после чего формулируется задача 
доставки дополнительных слоев VR-услу\-ги в~виде системы массового 
обслуживания. Посредством итеративного увеличения радиуса 
покрытия БС такой подход позволяет рассчитывать необходимую плот\-ность 
развертывания.
  
  Формирование групп многоадресных сессий основано на принципе выбора 
максимальной ширины по уровню половинной мощности~$\alpha$, 
позволяющей установить соединение с~ПУ, находящимся на границе 
обслуживаемой соты. Для определения значения~$\alpha$ сначала необходимо 
найти подходящее усиление на антенне БС, которое впоследствии может быть 
скорректировано в~меньшую сторону с~учетом доступных конфигураций 
антенной решетки:
  \begin{equation}
  G_A= \fr{S(R)[N_0+M_T]}{P_A G_U R^{-\zeta_T} e^{-KR} p_B(R)}\,.
  \label{e1-bes}
  \end{equation}
  
  Получив значение~$G_A$, можно рассчитать число многоадресных групп, 
необходимое для покрытия всей обслуживаемой зоны 
$$
N= \left\lceil \fr{\Theta}{G_A}\right\rceil,
$$
 где $\Theta$~--- ширина дуги сектора антенны. Для того 
чтобы воспользоваться моделью Хеллмана~[10], необходимо перевести ширину 
угла~$\alpha$ в~длину дуги образуемого им сегмента:
$$
\xi= \fr{G_A\pi R}{180}\,.
$$ 
Теперь, согласно~[11], длины пробелов между парами соседних теней имеют 
экспоненциальное распределение с~параметром $\lambda K^2/(2R)$. Это 
позволяет выразить вероятность того, что пробел имеет ширину от~$k$ 
до~$k\hm+1$ сегментов, в~виде
  \begin{equation}
  q_k= e^{-\lambda K^2 k\xi/(2R)} - e^{-\lambda K^2(k+1)\xi/(2R)}\,,
  \label{e2-bes}
  \end{equation}
а~потому не требует покрытия лучами. В~конечном итоге это позволяет 
оценить среднее число лучей многоадресных сессий, попадающих в~пробелы, 
как
\begin{equation}
\Delta = \fr{\pi \lambda K^2}{3\left(e^{\lambda B(R-
Q)+\lambda}\right)}\sum\limits^N_{k=1} kq_k\,.
\label{e3-bes}
\end{equation}
  
  Таким образом, среднее число многоадресных сессий в~соте можно найти как 
$N\hm- \Delta$. В~то же время требование отдельной многоадресной сессии 
зависит от того, все ли ПУ группы находятся в~условиях прямой видимости 
или есть хотя бы одно заблокированное ПУ, которое вынуждает всю группу 
снижать схему модуляции и~кодирования~[12]. Чтобы рассчитать средний 
объем ресурса, тре\-бу\-емый для обслуживания многоадресной сессии, необходимо найти 
вероятность вхождения некоторого чис\-ла ПУ~$u$ в~группу, которая имеет 
пуассоновское распределение, по свойству ПТП:
  \begin{equation}
  q_u= \fr{e^{-\lambda_n}\lambda_n^u}{u!}\,,
  \label{e4-bes}
  \end{equation}
где $\lambda_n= \lambda\pi (R^2\hm- Q^2)\alpha/360$.

  Следовательно, средний объем требуемого ресурса можно рассчитать как
  \begin{multline}
  b_M= \sum\limits^\infty_{u=1} q_u\left[ \prod\limits^u_{i=1} \left(1-
p_{B,i}\right) b_{M,L} +{}\right.\\
\left.{}+\left( 1-\prod\limits^u_{i=1} \left(1-p_{B,i}\right)\right) 
b_{M,B}\right],
  \label{e5-bes}
  \end{multline}
где $b_{M,L}$ и~$b_{M,B}$~--- требования к~объему ресурса в~условиях 
прямой видимости и~в~состоянии блокировки соответственно;  $p_{B,i}$~--- 
вероятность блокировки ПУ, которое является $i$-м соседом для БС в~ПТП.

  Как показано в~[13], вероятность блокировки прямой видимости зависит как 
от расстояния между ПУ и~БС, которое имеет распределение расстояния до  
$i$-го ближайшего соседа в~ПТП с~плот\-ностью
  \begin{multline}
  f_i(x)=\fr{2(\pi\lambda)^i}{(i-1)!}\,x^{2i-1} e^{-\pi\lambda x^2}\,,\\
   x>0\,,\enskip  i=1,2,\ldots,
  \label{e5-1.bes}
  \end{multline}
так и~от интенсивности блокаторов, пересекающих зону блокировки
\begin{equation}
\mu_{B,i}= \int\limits_0^\infty\! f_i(x) \fr{(x[h_B -h_U] +r_B [h_T-h_U])} 
{(2r_B\lambda_B v)^{-1} (h_T- h_U)}\,dx\,,\!
\label{e6-bes}
\end{equation}
где $h_T$, $h_U$ и~$h_B$~--- высоты БС, ПУ и~блокаторов соответственно; 
$r_B$ и~$v$~--- радиус и~скорость блокаторов. Это позволяет найти 
вероятность блокировки ПУ в~виде
\begin{equation}
p_{B,i}= \fr{\mu_{B,i}}{\mu_{B,i}+v/(2r_B)}\,,
\label{e7-bes}
\end{equation}
где $2r_B/v$~--- среднее время прохождения блокатором зоны блокировки под 
прямым углом к~ее длинной стороне.
  
  Для оценки среднего объема ресурсов, необходимых для доставки 
дополнительных слоев видео с~по\-мощью одноадресных сессий, 
рассматриваются ресурсы, позволяющие загружать каждый слой некоторой 
доле пользователей. Процесс загрузки дополнительных слоев моделируется 
в~виде СМО~[14] 
$$
\begin{matrix} M \\ \lambda\end{matrix} \left\vert \begin{matrix} M 
\\ \mu\end{matrix}\right\vert C,
$$ 
где $\lambda$~--- интенсивность 
поступления запросов на загрузку слоя видео; $\mu^{-1}$~--- средняя 
длительность загружаемого фрагмента видео, которая имеет экспоненциальное 
распределение;  $C$~--- чис\-ло активных одноадресных сессий, которые 
необходимо поддерживать для выполнения определенных требований по 
вероятности успешной загрузки. Для расчета вероятности сброса сессии 
$E_C(\rho)$ можно воспользоваться первой формулой Эрланга
  \begin{equation*}
  E_C(\rho)=\fr{\rho^C/C!}{\sum\nolimits^C_{m=0} (\rho^m/m!)}\,,\enskip 
0\leq\rho < \infty\,,
 % \label{e8-bes}
  \end{equation*}
где $\rho=\lambda/\mu$. 
  
  Теперь можно оценить средний объем тре\-бу\-емых ресурсов для загрузки 
дополнительного $l$-го слоя как
  \begin{equation*}
  E_l[U]=C_l \left[ p_{B,i} b_{U_l,B} +\left( 1-p_{B,i}\right) b_{U_l,L}\right],
 % \label{e9-bes}
  \end{equation*}
где $b_{U_l,B}$ и~$b_{U_l,L}$~--- требования отдельно взятого ПУ для 
загрузки $l$-го слоя в~условиях заблокированной и~незаблокированной прямой 
видимости. 
  
  Зададим вектор $\mathbf{p}_U$ с~элементами $p_l$~---  вероятностями 
успешной загрузки $l$-го слоя видео, $l\hm=1,\ldots , L$. Тогда в~общем виде 
схема нахождения минимальной требуемой плот\-ности развертывания БС 
выглядит следующим образом: 
\begin{itemize}
\item[(а)] для максимально большого допустимого 
радиуса соты вы\-чис\-лить объем ресурса для доставки базового слоя, используя 
выражения~(1)--(\ref{e7-bes}); 
\item[(б)] для каждого дополнительного слоя найти 
такое минимальное~$C_l$, при котором будет выполняться условие по 
ве\-ро\-ят\-ности успешной загрузки~$p_l$; 
\item[(в)] сложить рассчитанные требования на 
базовый и~дополнительные слои и~сравнить с~объемом ресурсов на БС; 
\item[(г)] если сумма требований меньше доступного ресурса, то повторить расчет для 
меньшего радиуса соты, иначе принять предпоследнее значение радиуса за 
минимально до\-пус\-ти\-мое.
\end{itemize}

\section{Численный анализ}
  
  В данном разделе проводится численный анализ влияния системных 
параметров, представленных в~таблице, на минимальную допустимую 
плотность развертывания БС NR.
  
  \begin{table*}[b]\small
  \begin{center}
  \begin{tabular}{|c|l|c|}
  \multicolumn{3}{c}{Системные параметры}\\
  \multicolumn{3}{c}{\ }\\[-6pt]
  \hline
Обозначение&\multicolumn{1}{c|}{Описание}&Значения по умолчанию\\
\hline
$f_C$ &Рабочая частота&73 ГГц\\
$B$&Число доступных ресурсных блоков&264\\
$r_B$&Радиус блокатора&0,4 м\\
$h_B$&Высота блокатора&1,7 м\\
$h_U$&Высота ПУ&1,5 м\\
$h_T$&Высота БС&4 м\\
$v$&Скорость блокатора&1 м/с\\
$P_T$&Излучаемая мощность на БС&2 Вт\\
$N_U$&Число конфигураций антенны ПУ&$8\times8$\\
$\lambda_B$&Плотность блокаторов&0,3 ед./м$^2$\\
$N_0$&Шум&$-$84~дБ\\
$\zeta_T$&Коэффициент затухания&2,1\\
$\lambda$&Интенсивность запросов на видеосессии от ПУ&1/3600 сессий/с\\
$\mu^{-1}$&Средняя длительность видеосессии&15 с\\
$d_M$&Требуемая скорость для загрузки базового слоя&7,78 Мбит/с\\
$\mathbf{d}_U$&Требуемые скорости для загрузки дополнительных слоев&19{,}78; 
25{,}81; 31{,}96~Мбит/с\\
\hline
\end{tabular}
\end{center}
%\end{table*}
\setcounter{table}{0}
\renewcommand{\tablename}{\protect\bf Рис.}
%\begin{figure*} %fig1
\vspace*{7pt}
  \begin{center}  
    \mbox{%
\epsfxsize=163mm
\epsfbox{bes-1.eps}
}
\end{center}
\vspace*{-9pt}
\Caption{Оптимальная плотность развертывания NR БС в~зависимости от плотности 
блокаторов~(\textit{а}) и~пользовательских устройств~(\textit{б}): \textit{1}~--- 
$\mathbf{p}_U\hm= (0{,}9; 0{,}8; 0{,}7)$;  \textit{2}~--- $(0{,}75; 0{,}5; 0{,}25)$; \textit{3}~--- 
$\mathbf{p}_U\hm= (0{,}5; 0{,}3; 0{,}1)$}
  %\end{figure*}
  \end{table*}
  
  \renewcommand{\figurename}{\protect\bf Рис.}
\renewcommand{\tablename}{\protect\bf Таблица}
  
  
  В данной работе рассматриваются три профиля качества обслуживания, 
характеризующися вектором~$\mathbf{p}_U$: строгий $(0{,}9; 0{,}8; 0{,}7)$, 
средний $(0{,}75; 0{,}5; 0{,}25)$ и~мягкий $(0{,}5; 0{,}3; 0{,}1)$.
  
  Одним из ключевых параметров, оказывающих влияние на 
производительность системы, является плот\-ность блокаторов. 

На 
рис.~1,\,\textit{a} представлен график зависимости оптимальной плотности 
развертывавния БС от~$\lambda_B$, на котором можно заметить, что мягкий 
и~средний профили не сильно подвержены влиянию блокаторов, в~отличие от 
пользователей со строгим профилем, для которого при $\lambda_B\hm= 
1{,}0$~ед./м$^2$ требуется 110~БС на~1~км$^2$. Для менее строгих профилей 
требуемая плотность развертывания при этом практически вдвое меньше.
  

  На рис.~1,\,\textit{б} изображена зависимость оптимальной плотности 
развертывавния БС от плотности ПУ. Очевидно, что чем более строгий профиль 
у~пользователей, тем большую нагрузку они создают на сеть в~целом и~тем 
большая плотность развертывания требуется для обеспечения эффективного 
покрытия. Здесь стоит отметить, что рассматриваемая зависимость для всех 
профилей имеет практически линейный характер.

\setcounter{figure}{1}
\begin{figure*} %fig2
\vspace*{1pt}
  \begin{center}  
    \mbox{%
\epsfxsize=163mm
\epsfbox{bes-2.eps}
}
\end{center}
\vspace*{-9pt}
\Caption{Оптимальная плотность развертывания БС NR для разных режимов передачи 
базового слоя~(\textit{а}) 
(\textit{1}~---  $\mathbf{p}_U\hm= (0{,}9; 0{,}8; 0{,}7)$;  \textit{2}~--- 
$(0{,}75; 0{,}5; 0{,}25)$;  \textit{3}~--- $\mathbf{p}_U\hm= 
(0{,}5; 0{,}3; 0{,}1)$; пустые значки~--- многоадресные сессии; залитые значки~--- одноадресные сессии)
    и~отношения долей одноадресных и~многоадресных сессий~(\textit{б}) (\textit{1}~--- 
$\mathbf{p}_U\hm= (0{,}9; 0{,}8; 0{,}7)$;  \textit{2}~--- $(0{,}75; 0{,}5; 0{,}25)$; \textit{3}~--- 
$\mathbf{p}_U\hm= (0{,}5; 0{,}3; 0{,}1)$)}
  \end{figure*}
  
  На рис.~2,\,\textit{а} показа зависимость плотности развертывания от все той 
же плотности ПУ, однако здесь присутствуют две схемы доставки базового 
слоя: с~по\-мощью многоадресных сессий и~с помощью одноадресных. 
Сравнение схем показывает, что использование многоадресных сессий при 
низких плотностях ПУ малоэффективно, но для плотных сетей (при $\sigma\hm 
\geq 0{,}4$~ед./м$^2$) они позволяют добиться существенного выигрыша за 
счет переиспользования ресурсов для базового слоя.
  
  Для рис.~2,\,\textit{б} введен дополнительный коэффициент~$L_R$, который 
обозначает отношение числа одноадресных сессий к~многоадресным. 
Естественным образом с~ростом числа одноадресных сессий возрастает 
необходимая плотность БС. Однако в~то же время возрастает и~разрыв между 
значениями для разных профилей, что объясняется резким повышением 
требований к~дополнительным слоям видео.
  

\vspace*{-6pt}

\section{Заключение}

\vspace*{-4pt}
  
  В данной работе предложен метод оценки производительности NR-систем 
при предоставлении услуги масштабируемого VR-видео посредством 
одноадресных и~многоадресных сессий. Численный анализ показал, что при 
низкой плотности пользователей использование многоадресных сессий 
малоэффективно, а наибольший позитивный эффект от их применения 
наблюдается тогда, когда требования к~ресурсам для доставки базового слоя 
начинают превосходить требования для дополнительных слоев видео. Также 
показано, что параметры качества обслуживания наряду с~плотностью 
пользователей имеют значительное влияние на необходимую плотность 
развертывания БС. При этом наиболее сильное влияние 
плотности блокаторов наблюдается в~случае наиболее высоких требований 
к~качеству обслуживания. В~целом, в~за\-ви\-си\-мости от различных системных 
па\-ра\-мет\-ров, плот\-ность раз\-вер\-ты\-ва\-ния варь\-и\-ру\-ет\-ся от~20 до~250~БС на~1~км$^2$. 
  
  Цель дальнейших исследований~--- расширение предложенной модели для 
сценария с~использованием БПЛА в~качестве подвижных точек доступа, что 
позволит исследовать проблему кластеризации роев.
  
{\small\frenchspacing
 {%\baselineskip=10.8pt
 %\addcontentsline{toc}{section}{References}
 \begin{thebibliography}{99}
\bibitem{1-bes}
\Au{Holma H., Toskala~A., Nakamura~T.} 5G technology: 3GPP New Radio.~--- New York, NY, 
USA:  John Wiley \& Sons, 2020. 536~p.
\bibitem{2-bes}
\Au{Lin X., Li~J., Baldemair~R., \textit{et al.}} 5G New Radio: Unveiling the essentials of the next 
generation wireless access technology~// IEEE Communications Standards Magazine, 2019. 
Vol.~3. Iss.~3. P.~30--37. doi: 10.1109/ \mbox{MCOMSTD}.001.1800036.
\bibitem{3-bes}
\Au{Le T.\,K., Salim~U., Kaltenberger~F.} An overview of physical layer design for ultra-reliable 
low-latency communications in 3GPP Releases 15, 16, and 17~// IEEE Access, 2020. Vol.~9. 
P.~433--444. doi: 10.1109/\mbox{ACCESS}. 2020.3046773.
\bibitem{4-bes}
\Au{Karembai A.\,K., Thompson~J., Seeling~P.} Towards prediction of immersive virtual reality 
image quality of experience and quality of service~// Future Internet, 2018. Vol.~10. Iss.~7. 
Art.~63. 12~p. doi: 10.3390/fi10070063.
\bibitem{5-bes}
\Au{Nasrabadi A.\,T., Mahzari~A., Beshay~J.\,D., Prakash~R.} Adaptive 360-degree video 
streaming using layered video coding~//  Virtual Reality Conference Proceedings.~--- 
Pis-\linebreak\vspace*{-12pt}
 
 \pagebreak
 
 \noindent
 cataway, NJ, USA: IEEE, 2017. P.~347--348. doi: 10.1109/ VR.2017.7892319.

\bibitem{7-bes} %6
\Au{Park J., Hwang J., Wei~H.} Cross-layer optimization for VR Video Multicast Systems~// 
 Global Communications Conference Proceedings.~--- Piscataway, NJ, USA: 
IEEE, 2018. P.~206--212. doi: 10.1109/GLOCOM.2018.8647389.

\bibitem{6-bes} %7
\Au{Long K., Cui~Y., Ye~C., Liu~Z.} Optimal transmission of multi-quality tiled 360 VR video by 
exploiting multicast opportunities~// Global Communications Conference 
 Proceedings.~--- Piscataway, NJ, USA: IEEE, 2019. Art.~9014280. 6~p. doi: 
10.1109/\linebreak GLOBECOM38437.2019.9014280.
\bibitem{8-bes}
\Au{Tang N., Tang H., Li~B. Yuan~X.} Joint maneuver and beamwidth optimization for  
UAV-enabled multicasting~// IEEE Access, 2019. Vol.~7. P.~149503--149514. doi: 
10.1109/ACCESS.2019.2947031.
\bibitem{9-bes}
3GPP Technical Specification 38.211: Physical channels and modulation (Release~16), 2021. 
{\sf https://www. 3gpp.org/ftp/Specs/archive/38\_series/38.211/38211-g50.zip}.
\bibitem{10-bes}
\Au{Хеллман О.} Введение в~теорию оптимального поиска~/ Пер. с~англ. 
Е.\,М.~Столяровой.~--- М.: Наука, 1985. 246~с.
\bibitem{11-bes}
\Au{Gapeyenko M., Samuylov~A., Gerasimenko~M., Moltchanov~D., Singh~S., Akdeniz~M.\,R., 
Aryafar~E., Himayat~N., Andreev~S., Koucheryavy~Y.} On the temporal effects of mobile blockers 
in urban millimeter-wave cellular scenarios~// IEEE T. Veh. Technol., 2017. 
Vol.~66. No.\,11. P.~10124--10138. doi: 10.1109/TVT.2017.2754543.
\bibitem{12-bes}
\Au{Samuylov A., Beschastnyi~V., Moltchanov~D., Ostrikova~D., Gaidamaka~Y., Shorgin~V.} 
Modeling coexistence of unicast and multicast communications in 5G New Radio systems~// 
30th Annual  Symposium (International) on Personal, Indoor and Mobile Radio 
Communications  Proceedings.~--- Piscataway, NJ, USA: IEEE, 2019. Art.~8904350. 
6~p. doi: 10.1109/PIMRC.2019.8904350.
\bibitem{13-bes}
\Au{Begishev V., Moltchanov D., Sopin~E., Samuylov~A., Andreev~S., Koucheryavy~Y., 
Samouylov~K.} Quantifying the impact of guard capacity on session continuity in 3GPP new radio 
systems~// IEEE T. Veh. Technol., 2019. Vol.~68. No.\,12. P.~12345--12359. 
doi: 10.1109/TVT.2019.2948702.
\bibitem{14-bes}
\Au{Горбунова А.\,В., Наумов~В.\,А., Гайдамака~Ю.\,В., Самуйлов~К.\,Е.} Ресурсные системы массового обслуживания как модели беспроводных систем 
связи~// Информатика и~её применения, 2018. 
Т.~12. Вып.~3. С.~48--55. doi: 10.14357/19922264180307.
 \end{thebibliography}

 }
 }

\end{multicols}

\vspace*{-6pt}

\hfill{\small\textit{Поступила в~редакцию 30.01.22}}

\vspace*{8pt}

%\pagebreak

%\newpage

%\vspace*{-28pt}

\hrule

\vspace*{2pt}

\hrule

%\vspace*{-2pt}

\def\tit{DENSITY ANALYSIS OF~mmWave NR DEPLOYMENTS FOR~DELIVERING SCALABLE 
AR/VR VIDEO SERVICES}


\def\titkol{Density analysis of~mmWave NR deployments for~delivering scalable 
AR/VR video services}


\def\aut{V.\,A.~Beschastnyi$^1$, D.\,Yu.~Ostrikova$^1$, S.\,Ya.~Shorgin$^2$, D.\,A.~Moltchanov$^3$,\\ 
and~Yu.\,V.~Gaidamaka$^{1,2}$}

\def\autkol{V.\,A.~Beschastnyi, D.\,Yu.~Ostrikova, S.\,Ya.~Shorgin, et al.}
%D.\,A.~Moltchanov$^3$,  and~Yu.\,V.~Gaidamaka$^{1,2}$}

\titel{\tit}{\aut}{\autkol}{\titkol}

\vspace*{-10pt}


 \noindent
  $^1$Peoples' Friendship University of Russia (RUDN University), 6~Miklukho-Maklaya Str., 
Moscow 117198, Russian\linebreak
$\hphantom{^1}$Federation
  
  \noindent
  $^2$Federal Research Center ``Computer Science and Control'' of the Russian Academy of 
Sciences, 44-2~Vavilov\linebreak
$\hphantom{^1}$Str., Moscow 119133, Russian Federation
  
  \noindent
  $^3$Tampere University, 7~Korkeakoulunkatu, Tampere 33720, Finland

\def\leftfootline{\small{\textbf{\thepage}
\hfill INFORMATIKA I EE PRIMENENIYA~--- INFORMATICS AND
APPLICATIONS\ \ \ 2022\ \ \ volume~16\ \ \ issue\ 2}
}%
 \def\rightfootline{\small{INFORMATIKA I EE PRIMENENIYA~---
INFORMATICS AND APPLICATIONS\ \ \ 2022\ \ \ volume~16\ \ \ issue\ 2
\hfill \textbf{\thepage}}}

\vspace*{6pt}  
  
  
  
  
  \Abste{The 5G New Radio (NR) technology operating in millimeter-wave 
  (mmWave) frequency band is designed to support bandwidth-greedy applications requiring 
  extraordinary rates at the access interface. In NR systems, the use of antenna arrays 
  that form directional radiation patterns allows to avoid high propagation losses 
  and interference but at the same time reduces the coverage area of a~single beam and, 
  hence, the number of multicast users that can be served by the beam. As a~result, 
  efficient algorithms are required to support such services in both terrestrial systems 
  and drone-assisted systems that utilize unmanned aerial vehicles as access points.
The present authors consider the streaming data delivery of virtual reality 
   services using scalable video coding technology which utilizes multicast capabilities 
   for baseline layer and unicast transmissions for delivering an enhanced experience. 
   By utilizing the tools of stochastic geometry and queuing theory, the authors develop 
   a~simple method allowing one to estimate the deployment density of mmWave NR base stations  
   to provide a~given performance of multilayer multicast services depending on their various 
   requirements and structure as well as on the density of users.}
  
  \KWE{5G; New Radio; mmWave; multi-layer VR; multicasting; scalable video coding; clustering}
  

  
\DOI{10.14357/19922264220213}

\vspace*{-8pt}

  \Ack
  \noindent
  The publication has been funded by the Russian Science Foundation, project 22-29-00694. 

%\vspace*{4pt}

  \begin{multicols}{2}

\renewcommand{\bibname}{\protect\rmfamily References}
%\renewcommand{\bibname}{\large\protect\rm References}

{\small\frenchspacing
 {%\baselineskip=10.8pt
 \addcontentsline{toc}{section}{References}
 \begin{thebibliography}{99}
\bibitem{1-bes-1}
  \Aue{Holma, H., A.~Toskala, and T.~Nakamura.} 2020. \textit{5G technology: 3GPP New 
Radio}. New York, NY:  John Wiley \& Sons. 536~p.
\bibitem{2-bes-1}
  \Aue{Lin, X., J.~Li, R.~Baldemair, %J.\,F.\,T.~Cheng, S.~Parkvall, D.\,C.~Larsson, 
%H.~Koorapaty, M.~Frenne, S.~Falahati, A.~Grovlen, 
\textit{et al.}} 2019. 5G New Radio: Unveiling 
the essentials of the next generation wireless access technology. \textit{IEEE Communications 
Standards Magazine} 3(3):30--37. doi: 10.1109/MCOMSTD.001.1800036.
\bibitem{3-bes-1}
  \Aue{Le, T.\,K., U.~Salim, and F.~Kaltenberger.} 2020. An overview of physical layer design 
for ultra-reliable low-latency communications in 3GPP Releases~15, 16, and 17. \textit{IEEE 
Access} 9:433--444. doi: 10.1109/ACCESS.2020.3046773.
\bibitem{4-bes-1}
  \Aue{Karembai, A.\,K., J.~Thompson, and P.~Seeling.} 2018. Towards prediction of immersive 
virtual reality image quality of experience and quality of service. \textit{Future Internet} 10(7):63. 
12~p. doi: 10.3390/fi10070063.
\bibitem{5-bes-1}
  \Aue{Nasrabadi, A.\,T., A.~Mahzari, J.\,D.~Beshay, and R.~Prakash.} 2017. Adaptive  
360-degree video streaming using layered video coding. \textit{Virtual Reality Conference 
Proceedings}. Piscataway, NJ:  IEEE. 347--348. doi: 10.1109/VR.2017.7892319.

\bibitem{7-bes-1} %6
  \Aue{Park, J., J. Hwang, and H.~Wei.} 2018. Cross-layer optimization for VR video multicast 
systems. \textit{Global Communications Conference Proceedings}.  Piscataway, NJ: IEEE. 206--212. doi: 
10.1109/GLOCOM.2018.8647389.

\bibitem{6-bes-1} %7
  \Aue{Long, K., Y.~Cui, C.~Ye, and Z.~Liu.} 2019. Optimal transmission of multi-quality tiled 
360~VR video by exploiting multicast opportunities. \textit{Global Communications Conference 
Proceedings}. Piscataway, NJ: IEEE. Art.~9014280. 6~p. doi: 10.1109/\linebreak GLOBECOM38437.2019.9014280.
\bibitem{8-bes-1}
  \Aue{Tang N., H.~Tang, B.~Li, and X.~Yuan.} 2019. Joint maneuver and beamwidth 
optimization for UAV-enabled multicasting. \textit{IEEE Access} 7:149503--149514. doi: 
10.1109/\linebreak ACCESS.2019.2947031.
\bibitem{9-bes-1}
  3GPP Technical Specification 38.211: Physical channels and modulation. 2021. Available at: {\sf 
https://www.3gpp. org/ftp/Specs/archive/38\_series/38.211/38211-g90.zip} (accessed April~20, 
2022).
\bibitem{10-bes-1}
  \Aue{Hellman, O.} 1985. \textit{Vvedenie v~teoriyu optimal'nogo poiska} [Introduction to the 
optimal search theory]. Moscow: Nauka. 246~p.
\bibitem{11-bes-1}
  \Aue{Gapeyenko, M., A.~Samuylov, M.~Gerasimenko, D.~Mol\-tcha\-nov, S.~Singh,  
M.\,R.~Akdeniz, E.~Aryafar, N.~Hi\-ma\-yat, S.~Andreev, and Y.~Koucheryavy.} 2017. On the 
temporal effects of mobile blockers in urban millimeter-wave cellular scenarios. \textit{IEEE 
T.~Veh. Technol.} 66(11):10124--10138. doi: 10.1109/TVT.2017.2754543.
\bibitem{12-bes-1}
  \Aue{Samuylov, A., V.~Beschastnyi, D.~Moltchanov, D.~Ostrikova, Y.~Gaidamaka, and 
V.~Shorgin.} 2019. Modeling coexistence of unicast and multicast communications in 5G New 
Radio systems. \textit{30th Annual  Symposium (International) on Personal, Indoor and Mobile 
Radio Communications Proceedings}. Piscataway, NJ:  IEEE. Art.~8904350. 6~p. doi: 10.1109/PIMRC.2019.8904350.
\bibitem{13-bes-1}
  \Aue{Begishev, V., D.~Moltchanov, E.~Sopin, A.~Samuylov, S.~Andreev, Y.~Koucheryavy, and 
K.~Samouylov.} 2019. Quantifying the impact of guard capacity on session continuity in 3GPP new 
radio systems. \textit{IEEE T. Veh. Technol.} 68(12):12345--12359. doi: 
10.1109/TVT.2019.2948702.
\bibitem{14-bes-1}
  \Aue{Gorbunova, A.\,V., V.\,A.~Naumov, Y.\,V.~Gaidamaka, and K.\,E.~Samouylov.} 2018. 
Resursnye sistemy massovogo obsluzhivaniya kak modeli besprovodnykh sistem svyazi [Resource 
queuing systems as models of wireless communication systems]. \textit{Informatika i~ee 
Primeneniya~--- Inform. Appl.} 12(3):48--55. doi: 10.14357/19922264180307. 

\end{thebibliography}

 }
 }

\end{multicols}

\vspace*{-6pt}

\hfill{\small\textit{Received January 30, 2022}}


  
  \Contr
  
  \noindent
  \textbf{Beschastnyi Vitalii A.} (b.\ 1992)~--- Candidate of Science (PhD) in physics and mathematics, 
assistant professor, Department of Applied Probability and Informatics, Peoples' Friendship 
University of Russia (RUDN University), 6~Miklukho-Maklaya Str., Moscow 117198, Russian 
Federation; \mbox{beschastnyy-va@rudn.ru}

\vspace*{3pt}
  
    \noindent
  \textbf{Ostrikova Daria Yu.} (b.\ 1988)~--- Candidate of Science (PhD) in physics and mathematics, 
associate professor, Department of Applied Probability and Informatics, Peoples' Friendship 
University of Russia (RUDN University), 6~Miklukho-Maklaya Str., Moscow 117198, Russian 
Federation; \mbox{ostrikova-dyu@rudn.ru}

\vspace*{3pt}
  
    \noindent
  \textbf{Shorgin Sergey Ya.} (b.\ 1952)~--- Doctor of Science in physics and mathematics, professor, 
principal scientist, Institute of Informatics Problems, Federal Research Center ``Computer Science 
and Control'' of the Russian Academy of Sciences, 44-2~Vavilov Str., Moscow 119133, Russian 
Federation; \mbox{sshorgin@ipiran.ru}

\vspace*{3pt}
  
   \noindent
  \textbf{Moltchanov Dmitri A.} (b.\ 1978)~--- Doctor of Science in technology, associate professor, 
Department of Electronics and Communications Engineering, Tampere University,
 7~Korkeakoulunkatu, Tampere 33720, Finland; \mbox{dmitri.moltchanov@tuni.fi}
 
 \vspace*{3pt}
  
  
    \noindent
  \textbf{Gaidamaka Yuliya V.} (b.\ 1971)~--- Doctor of Science in physics and mathematics, professor, 
Department of Applied Probability and Informatics, Peoples' Friendship University of Russia 
(RUDN University), 6~Miklukho-Maklaya Str., Moscow 117198, Russian Federation; senior 
scientist, Institute of Informatics Problems, Federal Research Center ``Computer Science and 
Control'' of the Russian Academy of Sciences, 44-2~Vavilov Str., Moscow 119333, Russian 
Federation; \mbox{gaydamaka-yuv@rudn.ru}
  

\label{end\stat}

\renewcommand{\bibname}{\protect\rm Литература} 
    %13

\def\stat{konovalov}

\def\tit{ОБ АДАПТИВНЫХ СТРАТЕГИЯХ И~УСЛОВИЯХ~ИХ~СУЩЕСТВОВАНИЯ$^*$}

\def\titkol{Об адаптивных стратегиях и~условиях их 
существования}

\def\autkol{М.\,Г.~Коновалов}

\def\aut{М.\,Г.~Коновалов$^1$}

\titel{\tit}{\aut}{\autkol}{\titkol}

{\renewcommand{\thefootnote}{\fnsymbol{footnote}}\footnotetext[1]
{Работа выполнена при поддержке РФФИ, грант № 11-07-00112.}}

\renewcommand{\thefootnote}{\arabic{footnote}}
\footnotetext[1]{Институт проблем информатики Российской академии наук, mkonovalov@ipiran.ru}



\Abst{Рассматривается задача оптимального управления в отсутствие априорной 
информации об управляемом объекте. Решением задачи является построение адаптивных 
стратегий на основе наблюдений, доступных в процессе управления. Изучаются 
некоторые условия адаптивной управляемости объекта. В~качестве математической 
модели используются управляемые случайные последовательности.}

\KW{управляемые случайные последовательности; адаптивные стратегии; условия 
существования}

\vskip 14pt plus 9pt minus 6pt

      \thispagestyle{headings}

      \begin{multicols}{2}

            \label{st\stat}


\section{Введение}

  Тема статьи относится к области адаптивных методов обработки информации с целью 
принятия оптимальных решений. Потребность в адаптивном\linebreak
подходе возникает в задачах 
с большой информационной неопределенностью, что наиболее характерно для 
телекоммуникационных систем, автоматизированных производственных процессов, 
робототех\-ни\-ки и других сфер, неразрывно связанных с компьютерной обработкой 
информации. Понятие неопределенности многозначно и связано с отсутствием априорных 
сведений, недетерминированностью, а также с неполнотой наблюдений. 
К~перечисленным факторам в нарастающей степени добавляется <<избыточность>> 
информации, которая порождается чрезмерно прогрессирующими объемами 
передаваемой и хранимой информации и обусловлена экспоненциальным ростом 
пропускной способности телекоммуникационных сетей, а также емкостей носителей 
информации.
  
  Идея адаптации (приспособления, самоорганизации), заимствованная из 
биологического мира, начала активно эксплуатироваться в науке примерно с середины 
прошлого века. Кратко, она заключается в том, чтобы, целенаправленно взаимодействуя с 
окружающей средой, отбирать и использовать поступающую информацию, необходимую 
для принятия оптимальных решений с точки зрения поставленной цели.
  
  Данная статья посвящена теоретическим аспектам адаптации. В~качестве исходного 
пред\-став\-ле\-ния использована схема, которая опирается на пред\-став\-ление о паре 
  <<объект--субъект>>, взаимодействующей в дискретном времени путем 
попеременного обмена сигналами. При этом субъект воздействует на объект с помощью 
управлений, получая в ответ сигналы, называемые наблюдениями. Действия субъекта 
преследуют цель, выраженную в наличии определенных свойств у траектории 
наблюдений.
  
  Основная отличительная особенность заключается в предположении, что действия 
субъекта происходят при недостаточной информации об объекте. В~качестве 
математической модели объекта взята конструкция управляемой случайной 
последовательности. В~терминах этого аппарата легко очерчиваются четыре аспекта 
информационной неопределенности:
  \begin{enumerate}[(1)]
\item недетерминированность понимается как стохастичность;
\item недостаток информации об объекте трактуется как неполное знание вероятностного 
распределения, задающего процесс;
  \item неполнота наблюдений означает, что состояния процесса наблюдаются лишь 
частично;
  \item недостаток знаний выражается в неумении \mbox{найти} или рассчитать ту или иную 
характеристику, связанную со случайной последовательностью, даже при наличии 
априорной информации о распределении процесса и полной его наблюдаемости.
  \end{enumerate}
  
  Субъект ассоциируется с алгоритмом, согласно которому выбираются управления, 
регулирующие траекторию случайной последовательности. Такой алгоритм принято 
называть стратегией управ\-ле\-ния. Задача заключается в том, чтобы выбрать стратегию, 
достигающую цели в ситуации, когда информация субъекта об объекте ограничена. 
По-дру\-го\-му можно сказать, что речь идет о построении стратегии, достигающей цели (в 
данном случае~--- максимизации предельного среднего дохода) для любого процесса из 
некоторого заданного класса объектов. Такие стратегии называют адаптивными по 
отношению к заданному классу объектов~[1].
  
  В разд.~2 даются формальные определения объекта, цели и адаптивной стратегии 
управления.
  
  В разд.~3 анализируются условия существования адаптивной стратегии. В~качестве 
необходимых условий обсуждаются два требования, которые, как представляется, должны 
выполняться из интуитивных соображений.
  
  Первое из необходимых условий связано с принципиальной особенностью адаптивных 
стратегий, которые, прежде чем выйти на <<оптимальный режим>>, должны затратить 
некоторое время на <<обуче\-ние>>. (На самом деле в рассматриваемой постановке процесс 
обучения для адаптивных стратегий длится даже неограниченно долго.) Естественно 
предположить, что подобные стратегии могут реализоваться, только если в процессе 
обучения не будут совершены <<непоправимые ошибки>>. Это соображение 
раскрывается на примерах и получает формальное описание.
  
  Второе необходимое условие является менее очевидным. Оно связано с гипотезой о 
том, что адаптивная стратегия управления классом случайных последовательностей 
существует лишь тогда, когда для данного класса возможно построение так называемой 
адаптивной стратегии перебора. Это выражается в том, что существует и заранее известно 
некоторое счетное множество вариантов поведения, среди которого для данного класса 
обязательно найдется оптимальный или близкий к нему вариант. Данное соображение 
также иллюстрировано примерами и приведена теорема о критерии существования 
адаптивной стратегии для определенного класса объектов.
  
  Подход, использованный в статье, а также полученные результаты являются 
продолжением направления, представленного в работе~[2].
  
\section{Постановка задачи адаптивного управления}
  
  Пусть  время $t$ пробегает значения 0, 1, \ldots\ и пусть заданы измеримые 
пространства $(X,\mathbf{X})$, $(Y,\mathbf{Y})$, $(Z,\mathbf{Z})$ (соответственно 
пространства \textit{состояний}, \textit{управлений} и \textit{наблюдений}).
  
  Общая траектория процесса упорядочена в виде последовательности $x_0, y_1, 
z_1,x_1,\ldots$\linebreak $\ldots , x_{t-1},y_t,z_t,x_t,\ldots$ Предыстория процесса до момента~$t$ 
включительно обозначается как

\noindent
  \begin{gather*}
 \! x^t=x_0^t=(x_0,\ldots , x_{t-1});\ \ \ y^t=y_1^t=(y_1, \ldots , y_{t-1});\\
  z^t=z_1^t=(z_1,  \ldots , z_{t-1})\,.
  \end{gather*}
  
  Траектории процесса определяются последовательностями условных вероятностных 
распределений~$\mu$, $\nu$ и~$\sigma$.
  
  Последовательность $\mu\hm=(\mu_0,\mu_1,\ldots ,\mu_t, \ldots)$ задает механизм 
смены состояний. В~этой последовательности $\mu_0$~--- вероятностное распределение 
на $(X,\mathbf{X})$; $\mu_t=\mu_t(A\vert x^{t-1},y^t)$, $t\hm>0$~---  условная 
(переходная) вероятность, которая при любых наборах $(x^{t-1},y^t)$ является 
вероятностной мерой на $(X,\mathbf{X})$ и при любом $A\hm\in X$ является измеримой 
функцией относительно $x^{t-1},y^t$.
  
  Последовательность $\nu\hm=(\nu_1, \ldots , \nu_t, \ldots)$ задает механизм появления 
наблюдений. В~этой последовательности каждый элемент $\nu_t\hm=\nu_t(C\vert x^{t-1}, 
y^t)$, $t\hm>0$, представляет собой условное распределение, которое при любом условии 
является вероятностной мерой на $(Z,\mathbf{Z})$ и для любого $C\hm\in Z$ является 
измеримой функцией относительно переменных, стоящих в условии. Пара $o\hm= 
(\mu,\nu)$ называется объектом.
  
  Последовательность $\sigma\hm= (\sigma_1, \ldots , \sigma_t. \ldots)$ называется 
(допустимой) \textit{стратегией} и определяет выбор управлений. В~этой 
последовательности:
%\smallskip
   $\sigma_1\hm=\sigma_1(\cdot)$~--- вероятностная мера на $(Y,\mathbf{Y})$; 
      $\sigma_{t+1}\hm=\sigma_{t+1}(B\vert y^t,z^t)$, $t\hm>0$,~--- условная вероятность, 
которая при любых $y^t,z^t$ является вероятностной мерой на $(Y,\mathbf{Y})$ и при 
любом $B\hm\in Y$ является измеримой функцией относительно $y^t,z^t$. Элементы 
последовательности~$\sigma$ называются (допустимыми) \textit{правилами}.

%\smallskip
  
  Введем обозначение для прямых произведений множеств:
  $$
  \Omega_0=X\,;\enskip \Omega_t=X^{t+1}\times Y^t\times Z^t\,,\enskip t>0\,,
  $$
а также для наименьших $\sigma$-ал\-гебр, порожденных соответствующими 
$\sigma$-ал\-геб\-рами:
$$
\mathbf{F}_0=\mathbf{X}\,;\enskip \mathbf{F}_t=\mathbf{X}\otimes \mathbf{Y}\otimes 
\mathbf{Z}\otimes \mathbf{X}\otimes \cdots \otimes \mathbf{Y}\otimes \mathbf{Z}\otimes 
\mathbf{X}
$$
($\mathbf{X}$ повторяется $t+1$ раз, $\mathbf{Y}$ и $\mathbf{Z}$~--- $t$ раз, $t\hm>0$).
  
  Положим
  
  \vspace*{3pt}
  
  \noindent
  $$
  \Omega =\prod\limits_{t\geq 0}\Omega_t\,;\enskip 
\mathbf{F}=\mathop{\otimes}\limits_{t\geq0}\mathbf{F}_t\,.
  $$ 
  
  Согласно общей теории~\cite{3-kon} последовательности $o\hm=(\mu,\nu)$ и~$\sigma$ 
порождают на пространстве $(\Omega, \mathbf{F})$ вероятностную меру $\mathbf{P}\hm= 
\mathbf{P}_{o,\sigma}\hm=\mathbf{P}_{\mu,\nu,\sigma}$, которая согласована с 
элементами этих последовательностей следующим образом. Случайные 
последова\-тель\-ности

\vspace*{-3pt}

\noindent
  \begin{gather*}
  x_t=x_t(\omega)\,;\enskip  
  y_{t+1}=y_{t+1}(\omega)\,;\\
  z_{t+1}= z_{t+1}(\omega)\,,\enskip  \omega\in \Omega\,,\  t\geq 0\,,
  \end{gather*}
удовлетворяют соотношениям:

\pagebreak

\noindent
$$
\mathbf{P}(x_0(\omega)\in A_0)=\int\limits_{A_0} \mu_0(dx_0)\,;
$$

\vspace*{-12pt}

\noindent
\begin{multline*}
\mathbf{P}\left(x_0(\omega)\in A_0\,,\  y_1(\omega)\in B_1\,,\ 
z_1(\omega)\in C_1, \ldots \right.\\[1pt]
\left.{}\ldots\,,
y_t(\omega)\in B_t\,,\  z_t(\omega)\in C_t\,,\  x_t(\omega)\in A_t\right)={}\\[1pt]
{}=\int\limits_{A_0}\mu_0(dx_0)\int\limits_{B_1}\sigma_1(dy_1)\int\limits_{C_1}\nu_1(dz_1
\vert x_0, y_1)\cdots{}\\[1pt]
{}\cdots
\int\limits_{B_t}\sigma_t\left(dy_t\vert y^{t-1},z^{t-1}\right) 
\int\limits_{C_t} \nu_t\left( dz_t\vert x^{t-
1},y^t\right) \times{}\\[1pt]
{}\times
\int\limits_{A_t} \mu_t\left( dx_t\vert x^{t-1},y^t\right)
\end{multline*}
для любых $A_t\in X$, $B_{t+1}\hm\in Y$, $C_{t+1}\hm\in Z$, $t\hm\geq 0$.
  
  По определению стратегии, ее правила зависят от предыдущих управлений и 
наблюдений, но не от предыдущих состояний. Это соответствует предположению о том, 
что состояния объекта не наблюдаемы в ходе процесса управления. В~частных случаях 
объект $o\hm=(\mu,\nu)$ может, конечно, описывать полностью наблюдаемый процесс. 
Например, если все множества $X_t$ содержат один и тот же единственный элемент. 
Другой простой пример~--- когда наблюдения тождественны состояниям. Однако на 
самом деле, как показывает лемма~1, с формальной точки зрения рассмотрение объекта с 
<<ненаблюдаемой>> компонентой всегда можно заменить изучением полностью 
наблюдаемого процесса.
  
  \medskip
  
  \noindent
  \textbf{Лемма 1.} \textit{Для любого объекта $o\hm=(\mu,\nu)$ условная вероятность 
$\mathbf{P}\left(dz_t\vert y^t,z^{t-1}\right)$ не зависит от стратегии~$\sigma$ при любых 
$t\hm>0$.}
  
  \medskip
  
  \noindent
  Д\,о\,к\,а\,з\,а\,т\,е\,л\,ь\,с\,т\,в\,о\,.\ Согласно отмеченной выше согласованности 
условных распределений $\mu,\nu,o$ и порождаемой ими меры~\textbf{P} имеем 
соотношения:
  \begin{multline*}
  I_1=\mathbf{P}\left(
  y_1(\omega)\in B_1,\ z_1(\omega)\in C_1\right) ={}\\[1pt]
  {}=
  \mathbf{P}\left( x_0(\omega)\in X_0\,,\ y_1(\omega)\in B_1\,,\ z_1(\omega)\in 
C_1\right)={}\\[1pt]
  {}=\int\limits_{X_0} \int\limits_{B_1} \int\limits_{C_1} \mu_0\left(dx_0\right) 
\sigma_1\left(dy_1\right) \nu_1\left(dz_1\vert x_0,y_1\right)={}\\[1pt]
  {}= \int\limits_{B_1}\int\limits_{C_1}\sigma_1\left(dy_1\right) \int\limits_{X_0}\mu_0\left( 
dx_0\right) \nu_1\left( dz_1\vert x_0,y_1\right)\,,
  \end{multline*}
справедливые при любых $B_1\hm\in Y$ и $C_1\in Z$. Кроме того, по определению 
условной вероятности
$$
I_1=\int\limits_{B_1}\int\limits_{C_1}\sigma_1\left(dy_1\right) \mathbf{P}\left(dz_1\vert 
y_1\right)\,.
$$
  
  Сравнивая оба выражения для~$I_1$, получаем, что
  $$
  \mathbf{P}\left( dz_1\vert 
y_1\right)=\int\limits_{X_0}\mu_0\left(dx_0\right)\nu_1\left(dz_1\vert x_0, y_1\right)\,,
  $$
т.\,е.\ утверждение леммы справедливо для $t\hm=1$. Пусть оно верно для $n\hm=1, 2, 
\ldots , t\hm-1$. Для любых $B_1\hm\in Y$, $C_1\hm\in Z$, \ldots , $B_{t-1}\hm\in Y$, 
$C_t\hm\in Z$ имеем:

\noindent
\begin{multline*}
I_t=\mathbf{P}\left( y_1(\omega)\in B_1\,,\ z_1(\omega)\in C_1, \ldots{}\right.\\[1pt]
\left.{}\ldots , y_t(\omega)\in B_t\,,\ 
z_t(\omega) \in C_t\right)={}\\[1pt]
{}=
\mathbf{P}\left( x_0(\omega)\in X\,,\ y_1(\omega)\in B_1\,,\ z_1(\omega)\in C_1\,, \ldots\right.\\[1pt]
\left.{}\ldots , x_{t-
1}(\omega)\in X\,,\ y_t(\omega)\in B_t\,,\ z_t(\omega)\in C_t\right)={}\\[1pt]
{}=
\int\limits_X \int\limits_{B_1} \int\limits_{C_1}\ldots \\[1pt]
\ldots\int\limits_X \int\limits_{B_t} 
\int\limits_{C_t} \mu_0\left( dx_0\right) \sigma_1\left( dy_1\right) \nu_1\left( dz_1\vert 
x_0,y_1\right)\cdots{}\\[1pt]
\cdots \mu_{t-1}\left( dx_{t-1}\vert x^{t-2} y^{t-1}\right) \sigma_t \left( dy_t\vert y^{t-
1},z^{t-1}\right)\times{}\\[1pt]
{}\times \nu_t\left( dz_t\vert x^{t-1},y^t\right)={}\\[1pt]
{}=\int\limits_{B_1} \sigma_1\left( dy_1\right) \int\limits_{C_1} \int\limits_{B_2} 
\sigma_2\left( dy_2\vert z_1\right)\cdots\\[1pt]
\cdots \int\limits_{C_{t-1}}\int\limits_{B_t} \sigma_t \left( 
dy_t\vert y^{t-1},z^{t-1}\right)\times{}\\[1pt]
{}\times \int\limits_X \mu_0\left(dx_0\right) \nu_1\left( dz_1\vert 
x_o,y_1\right)\cdots{}\\[1pt]
{}\cdots \int\limits_{X_{t-1}}\mu_{t-1}\left( dx_{t-1}\vert x^{t-2} y^{t-1}\right) \nu_t \left( 
dz_t\vert x^{t-1}, y^t\right)={}\\[1pt]
{}=\int\limits_{B_1} \sigma_1\left( dy_1\right) \int\limits_{C_1} 
\int\limits_{B_2}\sigma_2\left( dy_2\vert z_1\right)\cdots\\[1pt]
\cdots \int\limits_{C_{t-1}} 
\int\limits_{B_t} \sigma_t\left( dy_t\vert y^{t-1},z^{t-1}\right) \int\limits_{C_t} 
\mathbf{P}\left( dz_1\vert y_1\right)\ldots{}\\[1pt]
{}\cdots \mathbf{P}\left( dz_{t-1}\vert y^{t-1},z^{t-2}\right) \mathbf{P}\left( dz_t\vert y^t, 
z^{t-1}\right)\,.
\end{multline*}
Отсюда получаем, что

\noindent
  \begin{multline*}
\hspace*{-6.95218pt}\mathbf{P}\left( dz_1\vert y_1\right)\cdots \mathbf{P}\left( dz_{t-1}\vert y^{t-1},z^{t-
2}\right) \mathbf{P}\left( dz_t\vert y^t,z^{t-1}\right)={}\\[1pt]
  {}=\int\limits_X \mu_0\left( dx_0\right) \nu_1\left( dz_1\vert x_o,y_1\right)\cdots \\[1pt]
  \cdots
\int\limits_X \mu_{t-1}\left( dx_{t-1}\vert x^{t-2}y^{t-1}\right) \nu_t\left( dz_t\vert x^{t-
1},y^t\right)\,.
  \end{multline*}
  
  Следовательно, по предположению индукции $\mathbf{P}\left( dz_t\vert y^t,z^{t-
1}\right)$ не зависит от~$\sigma$.
  
  Таким образом, не уменьшая общности, можно ограничиться (что и будет сделано в 
оставшейся части текста) рассмотрением полностью наблюда-\linebreak\vspace*{-12pt}

\pagebreak

\noindent
емых объектов $o\hm=\mu$, 
управляемых (допустимыми) стратегиями~$\sigma$ c правилами вида
  $$
  \sigma_1=\sigma_1\left(\cdot\right)\,;\enskip \sigma_{t+1}=\sigma_{t+1}\left( \cdot \vert 
y^t,x^t\right)\,,\enskip t>0\,.
  $$
(Множество всех таких стратегий при заданных пространствах состояний и управлений 
далее обозначается через~$\Sigma$.) В~этом случае вероятностная мера 
$\mathbf{P}\hm=\mathbf{P}_{\mu,\sigma}$ определена на пространстве $(\Omega, 
\mathbf{F})$, в котором $\Omega\hm=\prod\limits_{t\geq0} X^{t+1}\times Y^t$, 
$\mathbf{F}\mathop{\otimes}\limits_{t\geq0} \mathbf{F}_t$, где $\mathbf{F}_0\hm=\mathbf{X}$; 
$\mathbf{F}_t=\mathbf{X}\otimes \mathbf{Y}\otimes \mathbf{X}\otimes \cdots \otimes 
\mathbf{Y}\otimes \mathbf{X}$ и согласована с последовательностями~$\mu$ и~$\sigma$. 
Через $\mathbf{F}_t$ обозначена $\sigma$-ал\-геб\-ра, порожденная предысторией 
$(x^t,y^t)$ до момента~$t$ включительно.
  
  В то же время необходимо заметить, что предположение о наличии 
<<двухступенчатой>> структуры у объектов (со\-сто\-яние--наблю\-де\-ние) может 
принести пользу при их изучении. Так происходит, например, в теории частично 
наблюдаемых управляемых марковских процессов.
  
  Предположим далее, что на наблюдаемой части траектории процесса задан 
одношаговый доход (в момент~$t$), и будем считать, что этот доход имеет вид 
$g_t\hm=g(x_t)$, где $g:\ X\rightarrow (0,\,1)\subset \mathbb{R}$~--- измеримая числовая 
функция со значениями из интервала (0,\,1).
  
  Обозначим через $v_{t,s}\hm=s^{-1}\sum\limits_{n=1}^s g_{t+n}$ среднее 
арифметическое доходов на промежутке от $t+1$ до $t\hm+s$ ($t\hm\geq0$, $s\hm\geq 1$).
  
  Если объект~$\mu$ управляется согласно стратегии~$\sigma$, то число
  $$
  w_t(\mu,\sigma) =\sup \left\{ c:\ \mathbf{P}_{\mu,\sigma} \left( 
\lim\limits_{\overline{s\rightarrow\infty}} v_{t,s}>c\right) =1\right\}
  $$
характеризует получаемый при этом гарантированный предельный средний доход 
начиная с момента $t=1$. Поскольку $\lim\limits_{\overline{s\rightarrow\infty}} v_{t,s}$ не 
зависит от~$t$, то $w_0(\mu,\sigma)\hm=w_1(\mu,\sigma)\hm=w_2(\mu,\sigma)\hm=\cdots$. 
Величина $w(\mu,\sigma)\hm=w_0(\mu,\sigma)$ играет в дальнейшем роль целевой 
функции и называется просто \textit{доходом} (при управлении объектом~$\mu$ с 
помощью стратегии~$\sigma$).
  
  Из определения дохода следует, что для любого $t>0$ выполняется условие
  $$
  \mathbf{P}_{\mu,\sigma}\left( \lim\limits_{\overline{s\rightarrow\infty}} v_{t,s}\geq 
w(\mu,\sigma)\vert \mathbf{F}_{t-1}\right)=1
  $$
почти наверное.
  Столь общее определение дохода, без предположений об эргодичности, оказывается 
полезным в теоретических рассмотрениях, однако на практике все же среднее 
арифметическое ведет себя более или менее регулярным образом. Поэтому введем 
следующее определение.
{ %\looseness=1

}
  
  Стратегия~$\sigma$ называется \textit{эргодической} по отношению к классу~$M$, 
если для любого объекта $\mu\hm\in M$ и любого $\varepsilon\hm>0$ выполняется 
условие $\sum\limits_{s=1}^\infty a_s\hm<\infty$, где $a_s\hm= 
a_s(\mu,\sigma,\varepsilon)\hm=\sup\limits_{t\geq0} \mathbf{P}_{\mu,\sigma}\left( \left\vert 
v_{t,s}-w(\mu,\sigma)\right\vert >\varepsilon\vert \mathbf{F}_t\right)$. Обозначим еще
  $$
  W=W(\mu) =\sup\limits_\sigma w(\mu,\sigma)\,,
  $$
где точная верхняя грань берется по всем допустимым стратегиям. Стратегия~$\sigma$ 
называется $\varepsilon$-\textit{оп\-ти\-маль\-ной}, если выполняется неравенство
$$
w(\mu,\sigma)\geq W-\varepsilon\,,\enskip \varepsilon\geq 0\,.
$$
  
  Далее объекты будут объединяться в множества объектов (классы объектов). При этом 
без дополнительных оговорок всюду предполагается, что
  \begin{itemize}
  \item все объекты из класса имеют одинаковые пространства состояний, управлений (и 
наблюдений);
  \item в качестве множества допустимых стратегий берется определенное выше 
множество~$\Sigma$;
  \item функция одношаговых доходов~$g$ одна и та же для всех объектов.
  \end{itemize}
  
  Пусть $M$~--- класс объектов. Стратегия~$\sigma$ является равномерно 
  $\varepsilon$-оп\-ти\-маль\-ной относительно этого класса, если последнее неравенство 
выполняется для всех $\mu\hm\in M$. Такую стратегию будем называть также 
  $\varepsilon$-\textit{адап\-тив\-ной} по отношению к классу~$M$. Класс объектов, для 
которого существует $\varepsilon$-адап\-тив\-ная стратегия, называется 
  $\varepsilon$-\textit{адап\-тив\-но управ\-ля\-емым}. (Если $\varepsilon\hm=0$, то 
приставка <<$\varepsilon$->> в этих определениях опускается.)
  
  Основная задача адаптивного управления заключается в построении адаптивных 
стратегий для различных классов объектов. 

К~настоящему вре\-ме\-ни получено много 
решений для многочисленных вариантов этой задачи. Подобные результаты являются 
фактически достаточными условиями адаптивной управ\-ля\-емости. Ниже, однако, будет 
уделено внимание также необходимым условиям существования адаптивных стратегий. 
Подчеркнем, что рассматриваемая постановка задачи предполагает, по сути, наличие 
лишь минимальной априорной информации об объекте управления~--- необходимо знать 
множество управлений~$Y$.

\section{Некоторые условия адаптивной управляемости}

  Пусть $\mu\in M$~--- фиксированный объект, а $\sigma\hm\in \Sigma$~--- 
фиксированная стратегия из некоторой среды. Набор, состоящий из первых $t$ правил 
стратегии~$\sigma$, будем обозначать через $\sigma^t\hm=(\sigma_1, \ldots , \sigma_t)$. 
Таким образом, $\sigma\hm=(\sigma^t, \sigma_{t+1},\sigma_{t+2}, \ldots)$. Положим
  $$
  w_t^*(\mu,\sigma) =w_t^*(\mu,\sigma^t)=\sup\limits_{\sigma_{t+1},\sigma_{t+2}, \ldots} 
w_t(\mu,\sigma)\,,
  $$
где верхняя грань берется по всем допустимым правилам начиная с момента $t\hm+1$. В 
этих обозначениях $w_0^*(\mu,\sigma) \hm=W(\mu)$. Ясно, что $W(\mu)\hm\geq 
w_1^*(\mu,\sigma)\hm\geq w_2^*(\mu,\sigma)\geq \cdots$
  
  Стратегию~$\sigma$ назовем $\varepsilon$-\textit{по\-вреж\-да\-ющей} для 
объекта~$\mu$, если
  $$
  \inf\left\{ t:\ w_t^*(\mu,\sigma)<W(\mu)-\varepsilon\right\} <\infty\,,\enskip \varepsilon>0\,.
  $$
  
  Пример~1 показывает, что существуют объекты, для которых каждая стратегия~--- 
$\varepsilon$-по\-вреж\-да\-ющая (с разными значениями~$\varepsilon$).
  
  \medskip
  
  \noindent
  \textbf{Пример~1.} Множество~$X$ состояний объекта~$\mu$ образовано точками с 
неотрицательными целочисленными координатами на плоскости, $X\hm=\{ (i,j), 
i\hm\geq0,\ j\hm\geq0\}$. Множество управлений $Y\hm=\{1;2\}$. Начальное состояние 
$x_0=(0,\,0)$. Детерминированные переходы между состояниями заданы следующим 
образом ($t\hm>0$, $i\hm\geq0$):
  \begin{align*}
  \mu_t\left( x_t=(i+1{,}0)\vert x_{t-1}=(i,0),y_t=1\right)&=1\,;\\
  \mu_t\left( x_t=(i,j+1)\vert x_{t-1}=(i,j),y_t=1\right)&=1\,,\ j>0\,;\\
  \mu_t\left( x_t=(i,j+1)\vert x_{t-1}=(i,j),y_t=2\right)&=1\,, j\geq 0\,.
  \end{align*}
  
  Одношаговые доходы определены как $g(i,0)\hm=0$, $g(i,j)\hm=1-2^{-i}$ для $i\geq 0$, 
$j\hm>0$.
  
  Стратегия, состоящая из бесконечного повторения управления~1, приносит доход~0. 
Стратегия, в которой управление~2 первый раз применяется (детерминировано) в 
момент~$t$, приносит доход $1\hm-2^{t-1}$, что меньше максимально возможного 
на~$2^{t-1}$. Рандомизация правил и их зависимость от предыстории не вносит 
принципиальных изменений~--- каждая стратегия остается 
  $\varepsilon$-по\-вреж\-да\-ющей относительно предельно наибольшего, но 
недостижимого значения~1.
  
  В примере~2 оптимальная стратегия для любого объекта из класса является 
повреждающей для остальных объектов.
  
  \medskip
  
  \noindent
  \textbf{Пример~2.} Пусть $X\hm= \{0, 1, 2, \ldots\}\cup \{a,b\}$; $Y\hm=\{0;\,1\}$; 
$g(a)\hm=1$; $g(b)\hm=g(i)\hm=0$, $i\hm\geq0$. Зададим счетное множество объектов 
$M\hm=\{\mu^{(k)},\ k\hm=0, 1, \ldots\}$. Пусть для всех~$k$:
  \begin{align*}
  \mu^{(k)}(x_0=0)&=1\,;\\
  \mu^{(k)}(x_{t+1}=i+1\vert x_t=i, y_t=0)&=1\,,\enskip i\geq0\,;\\
     \mu^{(k)}(x_{t+1}=a\vert x_t=k,y_t=1)&=1\,;\\
     \mu^{(k)}(x_{t+1}=b\vert x_t=i,y_t=1) &=1\,,\enskip i\not=k\,;\\
     \mu^{(k)}(x_{t+1}=a\vert x_t=a,y_t=j)&={}\\
&\hspace*{-45mm}{}=\mu^{(k)}(x_{t+1}=b\vert 
x_t=b,y_t=j)=1\,,\enskip j=0\vee 1\,.
     \end{align*}
  
  Таким образом, состояния $a$ и $b$~--- погло\-ща\-ющие, причем в состояние~$a$, 
приносящее максимальный доход, объект~$\mu^{(k)}$ может попасть, только если 
применить управление~1, находясь в со\-сто\-янии~$k$. Первые (существенные) правила 
оптимальной стратегии для объекта~$\mu^{(k)}$ требуют применения управления~0 до 
достижения состояния~$k$, а затем применения в этом состоянии управления~1. Однако 
такая стратегия является повреждающей для всех остальных объектов. Следовательно, для 
класса~$M$ не существует равномерно оптимальной стра\-тегии.
{\looseness=1

}
  
  Пусть $M$~--- класс объектов. Обозначим через $\Sigma_\varepsilon(\mu)$ множество 
$\varepsilon$-по\-вреж\-да\-ющих стратегий для объекта~$\mu$, $\mu\hm\in M$. Положим 
$\Sigma_\varepsilon(M)\bigcap\limits_{\mu\in M}\left( \Sigma\backslash 
\Sigma_\varepsilon(\mu)\right)$.
  
  \medskip
  
  \noindent
  \textbf{Лемма~2.} \textit{Для того чтобы существовала $\varepsilon$-адап\-тив\-ная 
стратегия, необходимо, чтобы $\Sigma_\varepsilon(M)\not=\emptyset$.}
  \medskip
  
  \noindent
  Д\,о\,к\,а\,з\,а\,т\,е\,л\,ь\,с\,т\,в\,о\,.\ Если $\Sigma_\varepsilon\not= \emptyset$, то любая 
допустимая стратегия хотя бы для одного из объектов является 
  $\varepsilon$-по\-вреж\-да\-ющей и, следовательно, не является 
  $\varepsilon$-оп\-ти\-маль\-ной, а потому не может быть равномерно 
  $\varepsilon$-оп\-ти\-маль\-ной по отношению к классу~$M$.
  
  В примере~3, несмотря на наличие по\-вреж\-да\-ющих стратегий, адаптивная стратегия 
существует.
  
  \medskip
  
  \noindent
  \textbf{Пример~3.} Пусть $X\hm=Y\hm=\{1, \ldots , K\}$ и пусть задана 
детерминированная функция~$f:\ X\hm\rightarrow X$, которая представляет собой 
циклическую подстановку на множестве~$X$,  т.\,е.\ $f(i)\not= f(j)$, если $i\not= j$; 
$i,j\hm=1, \ldots , K$. Рассмотрим следующий неоднородный во времени 
детерминированный объект. Положим
  \begin{align*}
  \mu_0(x_0=1)&=1\,;\\
  \mu_t(x_t=f(k)\vert x^{t-1},y^t) &= I_{\{y_t=k\}}\,,\ 0<k\,,\ t\leq K\,;\\
  \mu_t(x_t=f(k)\vert x^{t-1},y^t) &=I_{\{y_{K+1}=k}\,,\\
  & \hspace*{10mm}0<k\leq K\,,\enskip t>K
  \end{align*}
($I_A$~--- индикатор события~$A$).
  
  Одношаговые доходы определим как $g(i)\hm=i$, $i\hm\in X$.
  
  Так определенный объект обозначим через~$\mu^f$. Ясно, что для этого объекта 
траектория управ\-ля\-емо\-го процесса, начиная с момента $K+1$, и, следовательно, доход 
зависят исключительно от управ\-ле\-ния, примененного в момент $K+1$. Доход будет 
максимален (и равен~$K$) тогда и только тогда, когда $y_{K+1}\hm= k^\prime \hm= 
k^*(f)\hm=\argmax\limits_{1\leq k\leq K} f(k)$.
  
  Пусть $M=\{\mu^f\}$~--- совокупность всех объектов данного вида (которая содержит 
$K!$ элементов). Очевидно, для класса~$M$ существует равномерно оптимальная 
стратегия, доставляющая доход, равный~$K$. Например, достаточно вначале в моменты 
$t\hm=1, \ldots , K$ по одному разу применить каждое из управлений, а затем в момент 
$K+1$ применить управление~$k^*$, которое будет выявлено путем наблюдения за 
полученными одношаговыми доходами. Таким образом, на первых тактах необходимо совершить 
<<обучение>>~--- выявить управление, приносящее наибольший одношаговый доход. 
В~то же время существуют и повреждающие стратегии. Например, стратегия, в которой 
первые $K$ правил заключаются в применении управления~1. Правило~$\sigma_{K+1}$ 
такой стратегии может быть построено только в виде зависимости от управления~1 и от 
значения $f(1)$, поэтому при любом его определении найдется объект~$\mu^f$, для 
которого в момент $K+1$ будет с положительной вероятностью предписано применение 
неоптимального управления, и, следовательно, доход будет меньше~$K$.
  
  В примере 3 <<обучение>> оказалось возможным только благодаря знанию структуры 
процессов. Если бы заранее не было известно, что необходимо на первых тактах по разу 
<<испробовать>> все управ\-ле\-ния, то легко можно было пропустить период, когда 
возможно обучение, и совершить тем самым <<непоправимую ошибку>>. Следовательно, 
для того чтобы конструктивно построить равномерно оптимальную стратегию, 
необходима дополнительная информация. Это противоречит избранному принципу 
постановки задачи~--- минимальности априорной информации об объекте. 

Введем более 
жесткое определение адаптивной стратегии, которое, в част\-ности, устраняет указанное 
несоответствие.
  
  Пусть $M$~--- некоторый класс объектов. Эргодическая стратегия~$\sigma$ (ее 
определение дано в конце разд.~2) называется \textit{устойчивой} по отношению к 
классу~$M$, если для любого объекта $\mu\hm\in M$ стратегия~$\tilde{\sigma}$, 
полученная из стратегии~$\sigma$ путем произвольной (допустимой) замены конечного 
числа правил, (1)~имеет одинаковый со стратегией доход 
$w(\mu,\sigma)\hm=w(\mu,\tilde{\sigma})$ и (2)~является эргодической по отношению к 
классу~$M$.

%\columnbreak
  
  Адаптивная стратегия для класса~$M$ называется \textit{строго адаптивной}, если она 
устойчивая по отношению к этому классу.
  
  \medskip
  
  \noindent
  \textbf{Пример~4.} Легко показать, что строго адаптивными являются 
многочисленные адаптивные стратегии для класса управляемых конечных связных 
марковских цепей~[1, 2].
  
  Рассмотрим еще один мотив, выдвигаемый в качестве необходимого условия 
адаптивной управ\-ля\-емости.
  
  \medskip
  
  \noindent
  \textbf{Пример~5.} Пусть класс объектов состоит из функций вещественного 
аргумента~$u$ вида $\mu^y\hm=\mu^y(u)\hm=I_{\{u=y\}}$, $y\hm\in [0,\,1]$. (В~терминах 
управляемых случайных последовательностей: $X\hm= \{0;1]\}$, $Y\hm=[0,1]$; 
$\mu_t(x_t\vert x^{t-1},y^t)\hm=x_t I_{\{y_t=y\}}+ (1-x_t)I_{\{y_t=y\}}$; $g(x)\hm=x$, 
$x\hm\in X$.) Интуитивно представляется очевидным, что невозможно найти максимум 
такой функции за счетное число шагов, если не знать значение, в котором она обращается 
в единицу. В~то же время формально для каждого объекта~$\mu^y$ существует 
оптимальная стратегия. Например, можно постоянно повторять управление~$y$. Однако 
не существует стратегии, равномерно оптимальной по отношению к классу 
$M\hm=\{\mu^y\}$. В~такой стратегии для каждого $y\hm\in [0,\,1]$ необходимо должно 
было бы выполняться следующее условие: $\sigma_t(y_t=y\vert \cdot)>0$ хотя бы для 
одного значения~$t$. Но это невозможно, поскольку для фиксированного значения~$t$ 
данное неравенство может быть выполнено лишь для счетного множества значений~$y$, а 
$t$ также пробегает счетное множество значений. Счетное объединение счетных 
множеств само счетно, поэтому необходимое неравенство не может быть выполнено для 
всех точек на отрезке [0,\,1].
  
  Аналогичные рассуждения показывают, что в данном примере не существует счетного 
множества стратегий, обладающего тем свойством, что для любого объекта найдется 
$\varepsilon$-оп\-ти\-маль\-ная стратегия из этого множества.
  
  Конечное или счетное множество стратегий $\Sigma\hm=\{\sigma(1),\sigma(2), \ldots \}$ 
назовем \textit{базовым} по отношению к классу объектов $M\hm\in \mathcal{M}$, если:
  \begin{enumerate}[(1)]
  \item для любого объекта из $M$ и любого $\varepsilon\hm>0$ существует оптимальная 
стратегия из множества~$\Sigma$;
  \item любая стратегия $\sigma(i)$ является устойчивой по отношению к классу~$M$.
  \end{enumerate}
  
  \smallskip
  
  \noindent
  \textbf{Теорема.} \textit{Строго адаптивная стратегия для класса объектов~$M$ 
существует тогда и только тогда, когда для этого класса существует базовое 
множество стратегий~$\Sigma$.}


%\hfill {\large Приложение~1}

\bigskip

%\pagebreak

\noindent
Д\,о\,к\,а\,з\,а\,т\,е\,л\,ь\,с\,т\,в\,о\ \ теоремы.

Необходимость условий в данном случае является тривиальной, поскольку строго 
адаптивная стратегия, если она существует, образует базовое множество 
стратегий~$\Sigma$, состоящее из одного элемента.
  
  Докажем достаточность. Определим с по\-мощью стратегий из~$\Sigma$ новую 
стратегию $a$ следующим образом. Обозначим
  $$
  \theta_{t,n}=\mathrm{Int}\left(\left( 1-v_{t,n}\right)^{-n}\right)\,,
  $$
где $\mathrm{Int}\left(a\right)$ означает целую часть числа~$a$, и зададим 
последовательность марковских моментов $\tau\hm=\{\tau_n\}$ с помощью рекуррентных 
соотношений

\pagebreak

\noindent
$$
\tau_0=0\,,\enskip \tau_n=\tau_{n-1}+n+\theta_n\,,
$$
где $\theta_n\hm=\theta_{\tau_{n-1},n}$. Соответствующие $\sigma$-ал\-геб\-ры обозначим 
$\mathbf{F}_{(n)}\hm=\mathbf{F}_{\tau_{n-1}}$.
  
  Будем считать, что на пространстве $(\Omega,\mathbf{F})$ задана последовательность 
случайных величин $\beta\hm=\{\beta_n\}$, независимых 
относительно~$\mathbf{F}_{(n)}$. Каждая случайная величина имеет одно и то же 
невырожденное распределение $\{b_i\}$ на множестве номеров стратегий из~$\Sigma$.
  
  Определим правила стратегии $a\hm=a(\Sigma,\beta)$ формулой
  $$
  a_t=\sum\limits_{n=1}^\infty \sigma_t(\beta_n) I_{\{\tau_{n-1}<t\leq \tau_n\}}\,,
  $$
где $\sigma_t(\beta_n)$~--- правило стратегии $\sigma(i)\hm\in\Sigma$ в момент~$t$, если 
$\beta_n\hm=i$.
  
  Наглядно работа стратегии~$a$ выглядит следующим образом. Процесс управления 
разбивается на этапы. Этап с номером $n$ начинается в момент $\tau_{n-1}+1$ и 
оканчивается в момент~$\tau_n;\tau_0\hm=0$. В~момент, предшествующий началу 
очередного этапа, определяется номер стратегии в множестве~$\Sigma$, из которой будут 
взяты правила для применения на данном этапе. Этот номер равен значению случайной 
величины~$\beta_n$. Продолжительность $n$-го этапа равна $n\hm+\theta_n$ и зависит, 
следовательно, от номера этапа и от оценки качества применяемой стратегии, полученной 
в течение первых $n$ тактов этапа. Стратегия~$a$ называется стратегией перебора~[2]. 
Таким образом, последовательность~$\beta$ определяет на каждом этапе выбор стратегии 
из множества~$\Sigma$, правила из которой применяются на этом этапе.
  
  Пусть задан объект $\mu\hm\in M$ и пусть $W\hm=W(\mu)$~--- точная верхняя грань 
доходов для этого объекта, взятая по всем допустимым стратегиям, и пусть %также
  \begin{alignat*}{2}
  W_i&=w(\mu,\sigma(i))\,; &\enskip v_n^{(1)}&=v_{\tau_{n-1},n}\,;\\
  v_n^{(2)}&=v_{\tau_{n-1},n+\theta_n}\,; &\enskip \Delta_n&=\tau_n-\tau_{n-1}=n+\theta_n\,.
  \end{alignat*}
  
  Для произвольного $\varepsilon>0$ определим множества
  $$
  A_n^{(k)}(\varepsilon)=\left\{ v_n^{(k)}\geq W-\varepsilon\right\}\,,
  $$
обозначая их дополнения $\overline{A_n^{(k)}(\varepsilon)}$, $k=1, 2$.
  
  Обозначим
  \begin{align*}
  s_n^{(1)} &= \sum\limits_{l=1}^n I_{A_l^{(1)}(\varepsilon)\cap 
{A_l^{(2)}(2\varepsilon)}} \Delta_l\,;\\
  s_n^{(2)} &= \sum\limits_{l=1}^n I_{A_l^{(1)}\cap 
\overline{A_l^{(2)}(2\varepsilon)}}\Delta_l\,;\\
  s_n^{(3)} &= \sum\limits_{l=1}^n I_{\overline{A_l^{(1)}(\varepsilon)}}\Delta_l\,,
  \end{align*}
так что $\tau_n\hm=\sum\limits_{l=1}^n \Delta_l\hm= s_n^{(1)}\hm+ s_n^{(2)}\hm+ 
s_n^{(3)}$.

\columnbreak

  
  С~помощью введенных обозначений запишем оценку для усредненного дохода к 
моменту~$\tau_n$:
  \begin{multline}
  w_n=\fr{1}{\tau_n}\sum\limits_{t=1}^{\tau_n} g_t=\fr{\sum\limits_{l=1}^n 
v_l^{(2)}\Delta_l} {\sum\limits_{l=1}^n \Delta_l}\geq{}\\
{}\geq (W-2\varepsilon) \fr{s_n^{(1)}} 
{s_n^{(1)}+s_n^{(2)}+s_n^{(3)}}\,.
  \label{e1-kon}
  \end{multline}
  
  Для оценки суммы $s_n^{(1)}$ запишем неравенство
  $$
  s_n^{(1)}\geq \Delta_{v_n}\,,
  $$
в котором обозначено
$$
v_n=\max\left\{ l:\ l\leq n,\ A_l^{(1)}(\varepsilon)\cap A_l^{(2)}(2\varepsilon)\right\}\,.
$$
  
  Оценим вероятность события $B_n\hm=\{v_n\hm\leq n-\ln n\}$, для которого выполняется 
включение
  $$
  B_n\subset \bigcap\limits_{n-\ln n<l\leq n} 
  \overline{A_l^{(1)}(\varepsilon)}\cap \overline{A_l^{(2)}(2\varepsilon)}\,.
  $$
  
  Согласно определениям эргодической стратегии, базового множества стратегий и 
семейства случайных величин~$\beta$ имеем:
  \begin{multline*}
  \mathbf{P}_{a} \left( \overline{A_l^{(1)}(\varepsilon)}\cup\overline{A_l^{(2)} 
(2\varepsilon)}\,\Big\vert \mathbf{F}_{(l)}\right)\leq{}\\
  {}\leq
  \sum\limits_{\substack{{i\in \mathcal{I};}\\ {W_i\leq W-\varepsilon/2}}}\!\!\!\!
   \mathbf{P}_{a}\left(\beta_l=i\vert 
\mathbf{F}_{(l)}\right)+{}\\
{}+  %\substack{{i=\overline{1,n}}\\ {j=\overline{1,l}}}
\sum\limits_{\substack{{i\in \mathcal{I};}\\ {W_i\leq W-\varepsilon/2}}}\!\!\!\!
\mathbf{P}_{a}\left( \overline{A_l^{(1)}(\varepsilon)}, \ \beta_l=i
\vert \mathbf{F}_{(l)}\right)\leq{}\\
  {}\leq \sum\limits_{\substack{{i\in \mathcal{I};}\\ {W_i\leq W-\varepsilon/2}}}\!\!\!\!
  \mathrm{P}_{a}(\beta_l=i)+{}\\
{}+\sum\limits_{\substack{{i\in \mathcal{I};}\\ {W_i> W-
\varepsilon/2}}}
\!\!\!\!\mathbf{P}_{a}\left( v_l^{(1)}\leq W_i-\fr{\varepsilon}{2}, \beta_l=i\vert 
\mathbf{F}_{(l)}\right) \leq{}\\
  {}\leq \sum\limits_{\substack{{i\in \mathcal{I};}\\ {W_i\leq W-\varepsilon/2}}}\!\!\!\!
   b_i+a_l\left( 
\fr{\varepsilon}{2}\right) \leq q<1
  \end{multline*}
при всех достаточно больших~$l$. Отсюда следует, что для всех достаточно больших 
значений~$n$ выполняется неравенство
$$
\mathbf{P}_a(B_n)\leq q^{n-\ln n}\,.
$$
  
  Следовательно, согласно лемме Бо\-ре\-ля--Кан\-тел\-ли
  \begin{equation}
  \mathbf{P}_{a}\left( \overline{\lim\limits_{n\rightarrow\infty}} B_n\right)=0\,.
  \label{e2-kon}
  \end{equation}
  
  Это означает, что
  $$
  s_n^{(1)}\geq \Delta_{v_n}\geq (1-W-\varepsilon)^{-n+\ln n}\,.
  $$
  
  Оценим сумму $s_n^{(2)}$. Обозначив 
$C_n\hm=A_n^{(1)}(\varepsilon)\cap$\linebreak 
$\cap\overline{A_n^{(2)}(2\varepsilon)}$ и $W_{(n)}\hm=\sum\limits_{i\in 
I} W_i I_{\{\beta_n=i\}}$, получим:
  \begin{multline*}
  \mathrm{P}_{a}\left(C_n\vert \mathrm{ F}_{(n)}\right)=
  \mathrm{P}_{a|} \left( C_n, W_{(n)}<W-\fr{3\varepsilon}{2}\vert \mathrm{
  F}_{(n)}\right) +{}\\
  {}+ \mathrm{P}_{a}\left( 
  C_n, W_{(n)}\geq W-\fr{3\varepsilon}{2}\vert \mathrm{
  F}_{(n)}\right)\leq{}\\
  {}\leq \mathrm{P}_{a}\left( v_n^{(1)}>W-\varepsilon,\, W_{(n)}<W-\fr{3\varepsilon}{2}\vert \mathrm{
  F}_{(n)}\right)+{}\\
  {}+
  \mathrm{P}_{a} \left( v_n^{(2)}\leq W-2\varepsilon,\, W_{(n)}\geq W-
\fr{3\varepsilon}{2}\vert \mathrm{
  F}_{(n)}\right)\leq{}\\
  {}\leq \sum\limits_{i\in \mathcal{I}; W_i\leq W- \varepsilon/2} \mathrm{P}_{a}\left(
  v_{\tau_n,n}>W_i+\fr{\varepsilon}{2},\, \beta_l=i\vert\mathrm{F}_{(n)}\right)+{}\\
  {}+\sum\limits_{\substack{{i\in \mathcal{I};}\\ {W_i> W- 3\varepsilon/2}}}\!\!\!\!
   \mathbf{P}_{a} \left( 
v_{\tau_n,n+\theta_n}\leq W_i-\fr{\varepsilon}{2},\,\beta_l=i\vert\mathbf{F}_{(n)}\right)\leq {}\\
{}\leq
a_n\left( \fr{\varepsilon}{2}\right)\,.
  \end{multline*}
  
  Из определения базового множества стратегий следует, что
  $$
  \sum\limits_{n=1}^\infty \mathbf{P}_{a} (C_n)<\infty\,,
  $$
поэтому согласно лемме Бо\-ре\-ля--Кан\-тел\-ли полу\-чаем:
\begin{equation}
\mathbf{P}_{a}\left( \overline{\lim\limits_{n\rightarrow\infty}} C_n\right) =0\,.
\label{e3-kon}
\end{equation}
  
  Отсюда следует, что
  $$
  \sup\limits_n s_n^{(2)}\leq c<\infty\,.
  $$
  
  Для суммы $s_n^{(3)}$ имеем следующую оценку:
  $$
  s_n^{(3)}\geq \sum\limits_{l=1}^n \left(n+(1-W+\varepsilon)^{-l}\right)< n^2+n(1-
W+\varepsilon)^{-n}.
  $$
  
  Подставляя оценки, полученные для сумм $s_n^{(k)}$, в неравенство~(\ref{e1-kon}), 
получаем:
  \begin{multline*}
  w_n\geq (W-\varepsilon) \left( 1+\fr{s_n^{(2)}+s_n^{(3)}}{s_n^{(1)}}\right)^{-1}\geq 
{}\\
  {}\geq (W-\varepsilon)\left( 1+\fr{c+n^2+n(1-W+\varepsilon)^{-n}}{(1-W-\varepsilon/2)^{-
n+\ln n}}\right)^{-1}\geq{}\\
{}\geq W-3\varepsilon
  \end{multline*}
для всех достаточно больших значений~$n$. Отсюда
\begin{equation}
\lim\limits_{\overline{n\rightarrow\infty}} w_n\geq W\,.
\label{e4-kon}
\end{equation}
  
  Рассмотрим далее множество
  $$
  \Omega^\prime =\left\{ \lim\limits_{n\rightarrow\infty} w_n =W\right\}\cap 
\overline{B}\cap\overline{C}\,,
  $$
где $\overline{B}$ и $\overline{C}$ означают соответственно дополнения к множествам 
$B\hm= \overline{\lim\limits_{n\rightarrow\infty}} B_n$ и $C\hm= 
\overline{\lim\limits_{n\rightarrow\infty}} C_n$.
  
  Согласно формулам~(\ref{e2-kon})--(\ref{e4-kon})
  $$
  \mathbf{P}_{a}\left(\Omega^\prime\right) =1\,.
  $$
  
  Определим следующие события:
  
  \noindent
  \begin{align*}
  D_{n,t}^{(1)} &= \left\{ \tau_{n-1}<t\leq \tau_{n-1}+n\right\} \cap \Omega^\prime\,;\\
  D_{n,t}^{(2)} &= \left\{\tau_{n-1}+n<t\leq \tau_n\right\}\cap \Omega^\prime\,;\\
  D_{n,t}^{(3)} &= \left\{ \tau_{n-1}<t\leq \tau_n\right\} \cap \Omega^\prime\,.
  \end{align*}
  
  На множестве $D_{n,t}^{(1)}$ усредненный доход $v_t\hm=v_{0,t}\hm=
  t^{-1}\sum\limits_{s=1}^t g_s$ оценивается с помощью формулы~(\ref{e1-kon}) как
  
    \noindent
  $$
  v_t\geq \fr{\tau_{n-1} w_n}{\tau_{n-1}+n+\theta_n}\geq W-\varepsilon_n^{(1)}\,,
  $$
где $\varepsilon_n^{(1)}\hm\rightarrow0$ при $n\hm\rightarrow\infty$.
  
  Пусть событие $D_{n,t}^{(2)}$ имеет место. Тогда $\theta_n\geq (1\hm- 
W\hm+\varepsilon)^{-n}$. Кроме того, из определения событий $B_n$, $B$, 
$D_{n,t}^{(2)}$ следует, что для всех достаточно больших значений~$n$ выполняется 
неравенство $v_n\hm> n-\ln n$. Следовательно, на множестве~$D_n^{(2)}$ справедлива 
оценка

  \noindent
  $$
  v_t\geq \fr{\tau_{n-1} w_n}{\tau_{n-1}+n+\theta_n}\geq W-\varepsilon_n^{(2)}\,,
  $$
где $\varepsilon_n^{(2)}\hm\rightarrow0$ при $n\hm\rightarrow\infty$.
  
  Из определения событий $C_n$, $C$, $D_{n,t}^{(3)}$ вытекает, что
  
    \noindent
  $$
  D_{n,t}^{(3)} \subset \left\{ \min\limits_{n<m\leq n+\theta_n} v_{n,m}\geq W-
2\varepsilon\right\}\,,
  $$
поэтому на множестве $D_n^{(3)}$ справедливы неравенства:

  \noindent
\begin{multline*}
\!\!v_t\geq \fr{\tau_{n-1} w_n}{t}+\left(1- \fr{\tau_{n-1}}{t}\right) \left( 1-\tau_n\right)^{-1} 
\!\!\sum\limits_{s=\tau_{n-1}+1}^t \!\!\!\!g_s\geq{}\\
{}\geq \fr{\tau_{n-1} w_n}{t}+\left( 1-\fr{\tau_{n-1}}{t}\right)\left( W-2\varepsilon\right) \geq 
W-2\varepsilon -\varepsilon_n^{(3)},
\end{multline*}
где $\varepsilon_n^{(3)}\rightarrow0$ при $n\hm\rightarrow\infty$.

\pagebreak
  
  Таким образом, на множестве
  $$
  D_{n,t}=\bigcup\limits_{k=1}^3 D_{n,t}^{(k)} = \left\{ \tau_{n-1}<t\leq \tau_n\right\} \cap 
\Omega^\prime
  $$
имеет место оценка $v_n\hm\geq W-\varepsilon-\varepsilon_n$, где 
$\varepsilon_n\hm\rightarrow 0$ при $n\hm\rightarrow\infty$. Достаточность утверждения 
теоремы следует из соотношений $\Omega\hm= \bigcup\limits_{n=1}^\infty \left\{ \tau_{n-
1}\hm<t\hm\leq \tau_n\right\}$ и $\lim\limits_{t\rightarrow\infty} I_{D_{n,t}}\hm=0$.

\section{Заключение}

  Адаптивные стратегии, позволяющие достигать цели в условиях информационной 
неопреде\-лен\-ности, основываясь на <<обучении>> в процессе взаимодействия с объектом, 
находят все более широкое практическое применение. 

В~этой работе было уделено 
внимание теоретическим аспектам адаптивного подхода. Сформулированы определения 
адаптивных стратегий и приведена формальная постановка задачи адаптивного 
управления. Сформулированы и доказаны некоторые утверждения о необходимых 
условиях и достаточных условиях адап\-тив\-ной управляемости. 

Продолжение исследований 
в данном на\-прав\-ле\-нии позволит найти ответы на принципиальные вопросы, в каких 
ситуациях можно рассчитывать на <<приспособление к неизвестной среде>> и сколь 
универсальными могут быть <<обучающиеся>> алгоритмы.



{\small\frenchspacing
{%\baselineskip=10.8pt
\addcontentsline{toc}{section}{Литература}
\begin{thebibliography}{9}


  \bibitem{1-kon}
  \Au{Sragovich~V.\,G.}
  Mathematical theory of adaptive control.~--- Singapore: World Scientific, 2006.
  \bibitem{2-kon}
  \Au{Коновалов~М.\,Г.}
  Методы адаптивной обработки информации и их приложения.~--- М.: ИПИ РАН, 2007.
  
  \label{end\stat}
  
  \bibitem{3-kon}
  \Au{Неве~Ж.}
  Математические основы теории вероятностей.~--- М.: Мир, 1969.
\end{thebibliography}
}
}


\end{multicols} %14

%%%%%%%%%%%%%%%%%%%%%%%%%%%%%%%%%%%%%%%%

%\def\stat{rez}
{%\hrule\par
%\vskip 7pt % 7pt
\raggedleft\Large \bf%\baselineskip=3.2ex
Р\,Е\,Ц\,Е\,Н\,З\,И\,И \vskip 17pt
    \hrule
    \par
\vskip 6pt plus 6pt minus 3pt }

%\thispagestyle{headings} %с верхним колонтитулом
%\thispagestyle{myheadings} %с нижним колонтитулом, но в верхнем РЕЦЕНЗИИ

\def\tit{НОВАЯ КНИГА И.\,Н.~СИНИЦЫНА, А.\,С.~ШАЛАМОВА <<ЛЕКЦИИ ПО ТЕОРИИ 
ИНТЕГРИРОВАННОЙ ЛОГИСТИЧЕСКОЙ ПОДДЕРЖКИ>> (М.: ТОРУС ПРЕСС, 2012. 624~с.)}

%1
\def\aut{Д.ф.-м.н., профессор С.\,Я.~Шоргин}

\def\auf{\ }

\def\leftkol{\ % РЕЦЕНЗИИ
}

\def\rightkol{ \ } 

%\def\leftkol{\ } % ENGLISH ABSTRACTS}

%\def\rightkol{\ } %ENGLISH ABSTRACTS}

%\def\leftkol{РЕЦЕНЗИИ}

%\def\rightkol{РЕЦЕНЗИИ}

\titele{\tit}{\aut}{\auf}{\leftkol}{\rightkol}
\vspace*{-18pt}


     \label{st\stat}

     \begin{multicols}{2}
     {\small
     {\baselineskip=10.1pt
     

      В книге представлено системное изложение теоретических основ одного из новейших 
направлений в \mbox{об\-ласти} экономики послепродажного обслуживания изделий наукоемкой 
продукции (ИНП) длительного пользования~--- интегрированной логистической поддержки
(ИЛП). 
{\looseness=1

}

Приведены также результаты новых работ, выполненных в Институте проблем информатики 
Российской академии наук в рамках научного направления <<Информационные технологии и 
анализ сложных сис\-тем>>.
 {%\looseness=1

}
     
      Излагаемые в книге научные подходы позво\-ляют карди\-наль\-но реформировать 
существующие системы производства и эксплуатации ИНП путем создания и внед\-ре\-ния 
методов рационального и оптимального управ\-ле\-ния процессами расходования 
вре\-мен\-н$\acute{\mbox{ы}}$х, 
мате\-ри\-аль\-ных, трудовых и других ресурсов на всех стадиях жизненного цикла изделий (ЖЦИ) по 
критериям экономической целесообразности и эф\-фек\-тив\-ности.
  {\looseness=1

}
    
      В книге приведен краткий обзор причин возник\-новения и
      развития CALS-методологии как основы 
современных международных стандартов по созданию и функционированию глобальных 
ин\-фор\-ма\-ци\-он\-но-ком\-му\-ни\-ка\-ци\-он\-ных систем, ее ключевых возможностей и эффективности 
результатов ее использования. 
Авторы %\linebreak 
предлагают ряд научных обоснований для разработки 
единой теории проектирования и управления систем ИЛП для полноценного использования 
преимуществ %\linebreak
 суще\-ст\-ву\-ющей методологии, определяют \mbox{общую} структурную схему 
комплексной системы <<ИНП-СППО>> и необходимость разработки для ее описания 
гибридных стохастических моделей.
{%\looseness=1

}

%\columnbreak
      
      Книга состоит из пяти частей, где последовательно излагается материал по каждой из 
следующих тем: <<Интегрированная логистическая поддержка>>, <<Теория гибридных 
стохастических систем и компьютерная поддержка исследований и разработок>>, <<Основы 
математического моделирования, анализа и синтеза систем послепродажного обслуживания>>, 
<<Определение и анализ показателей экспортного потенциала ИНП при проектировании>>, 
<<Задачи управления поддержкой послепродажного обслуживания>>, а также 
<<Моделирование инвестиционных процессов ИЛП в условиях неравновесных финансовых 
рынков>>. 
   
      В конце каждой главы приведены выводы и даны вопросы и задания для 
самоконтроля. В~приложениях содержатся основные определения по программам работ по 
анализу ИЛП, логистическим базам данных и компьютерным решениям, эквивалентной статистической 
линеаризации нелинейных преобразований ИЛП, справочный материал, а также развернутые 
уравнения для вероятностных характеристик.


      \def\leftkol{РЕЦЕНЗИИ}

\def\rightkol{РЕЦЕНЗИИ} 

      
      Книга заинтересует широкий круг специалистов и может быть использована научными 
проектными организациями в сфере промышленного производства ИНП. Большое количество 
иллюстраций, примеров и вопросов, обращенных к читателю, позволяет использовать книгу 
также в качестве учебного пособия для студентов и аспирантов машиностроительных, 
транспортных и~других специальностей, а также для самостоятельного изучения. 
{%\looseness=-1

}

Книга 
представляет несомненный интерес для специалистов и студентов в области прикладной 
математики и информатики.
    

}

}
\end{multicols}

%\newpage

%\def\stat{popravka}



\def\tit{ПОПРАВКА К СТАТЬЕ О.\,В.~ШЕСТАКОВА 
<<ПОРОГОВЫЕ ФУНКЦИИ В~МЕТОДАХ ПОДАВЛЕНИЯ ШУМА, ОСНОВАННЫХ~НА~ВЕЙВЛЕТ-РАЗЛОЖЕНИИ СИГНАЛА>>\\
(Информатика и её применения, 2021. Т.\ 15.  Вып.\,3. C.\ 51--56)}



\def\titkol{Поправка к статье О.\,В.~Шестакова\\
<<Пороговые функции в~методах подавления шума, основанных
на~вейвлет-разложении сигнала>>}



  \def\aut{\ }

  \def\autkol{\ } 

\titel{\tit}{\aut}{\autkol}{\titkol}

\def\leftfootline{\small{\textbf{\thepage}
\hfill INFORMATIKA I EE PRIMENENIYA~--- INFORMATICS AND
APPLICATIONS\ \ \ 2021\ \ \ volume~15\ \ \ issue\ 4}
}%
 \def\rightfootline{\small{INFORMATIKA I EE PRIMENENIYA~---
INFORMATICS AND APPLICATIONS\ \ \ 2021\ \ \ volume~15\ \ \ issue\ 4
\hfill \textbf{\thepage}}}


 \label{st\stat}

 \thispagestyle{headings}
 
 \vspace*{-24pt}  

\noindent
{\textbf{DOI:} 10.14357/19922264210307}

\vspace*{20pt}

\def\leftfootline{\small{\textbf{\thepage}
\hfill INFORMATIKA I EE PRIMENENIYA~--- INFORMATICS AND
APPLICATIONS\ \ \ 2021\ \ \ volume~15\ \ \ issue\ 4}
}%
 \def\rightfootline{\small{INFORMATIKA I EE PRIMENENIYA~---
INFORMATICS AND APPLICATIONS\ \ \ 2021\ \ \ volume~15\ \ \ issue\ 4
\hfill \textbf{\thepage}}}


%%%%%%%%%

\medskip

\noindent
С.~55, вместо 

\bigskip

\noindent
{\large ANALYSIS OF THE UNBIASED MEAN-SQUARE RISK ESTIMATE\\[6pt]
 OF~THE~BLOCK THRESHOLDING METHOD}

 



\bigskip

\noindent
должно быть

\bigskip

\noindent
{\large THRESHOLDING FUNCTIONS IN~THE~NOISE SUPPRESSION METHODS\\[6pt] 
BASED ON~THE~WAVELET EXPANSION OF~THE~SIGNAL}

 



 
\vskip 10pt plus 9pt minus 6pt

 \thispagestyle{headings}
 
 %\vspace*{-22pt}
  

\label{end\stat}

\renewcommand{\bibname}{\protect\rm Литература} 


\vspace*{8pt}

\hrule

\vspace*{2pt}

\hrule 

\vspace*{12pt}


\def\stat{popravka-1}



\def\tit{ПОПРАВКА К СТАТЬЕ А.\,А.~КУДРЯВЦЕВА, О.\,В.~ШЕСТАКОВА, С.\,Я.~ШОРГИНА
<<МЕТОД ОЦЕНИВАНИЯ ПАРАМЕТРОВ ИЗГИБА, ФОРМЫ И~МАСШТАБА
ГАММА-ЭКСПОНЕНЦИАЛЬНОГО РАСПРЕДЕЛЕНИЯ>>\\
(Информатика и её применения, 2021. Т.\ 15.  Вып.\,3. C.\ 57--62)}



\def\titkol{Поправка к статье А.\,А.~Кудрявцева, О.\,В.~Шестакова, С.\,Я.~Шоргина
<<Метод оценивания параметров изгиба, формы и масштаба
гамма-экспоненциального распределения>>}



  \def\aut{\ }

  \def\autkol{\ } 

\titel{\tit}{\aut}{\autkol}{\titkol}


 \label{st\stat}

 \thispagestyle{headings}
 
 \vspace*{-24pt}  

\noindent
{\textbf{DOI:} 10.14357/19922264210308}

\vspace*{20pt}




%%%%%%%%%

\medskip

\noindent
С.~61, вместо 

\bigskip

\noindent
{\large PROBABILISTIC CHARACTERISTICS OF~BALANCE INDEX
OF~FACTORS\\[6pt] 
WITH~GENERALIZED GAMMA DISTRIBUTION}



 



\bigskip

\noindent
должно быть

\bigskip

\noindent
{\large A METHOD FOR~ESTIMATING BENT, SHAPE AND~SCALE PARAMETERS\\[6pt] 
OF~THE~GAMMA-EXPONENTIAL DISTRIBUTION} 



 



 
\vskip 10pt plus 9pt minus 6pt

 \thispagestyle{headings}
 
 %\vspace*{-22pt}
  

\label{end\stat}

\renewcommand{\bibname}{\protect\rm Литература}  
%\include{popravka-1}

\def\stat{authorsrus}
{%\hrule\par
%\vskip 7pt % 7pt
\raggedleft\Large \bf%\baselineskip=3.2ex
О\,Б\ \ А\,В\,Т\,О\,Р\,А\,Х \vskip 17pt
    \hrule
    \par
\vskip 21pt plus 8pt minus 4pt }


\def\tit{\ }

\def\aut{\ }

\def\auf{\ }

\def\leftkol{\ } % ENGLISH ABSTRACTS}

\def\rightkol{ОБ АВТОРАХ} %ENGLISH ABSTRACTS}

\titele{\tit}{\aut}{\auf}{\leftkol}{\rightkol}
      
            \label{st\stat}



\vspace*{24pt}

\begin{multicols}{2}




\noindent
\textbf{Архипов Олег Петрович} (р.\ 1948)~---
кандидат технических наук, директор Орловского филиала Института проб\-лем информатики
Российской академии наук
%302025, г.Орел, Московское шоссе, д.137

\vspace*{3pt}

\noindent
\textbf{Бирюкова Татьяна Константиновна} (р.\ 1968)~---
кандидат фи\-зи\-ко-ма\-те\-ма\-ти\-че\-ских наук, старший научный сотрудник Института проб\-лем информатики
Российской академии наук

\vspace*{3pt}

\noindent 
\textbf{Бобков  Сергей Геннадьевич} (р.\ 1955)~---
доктор технических наук,  заведующий отделением На\-уч\-но-ис\-сле\-до\-ва\-тель\-ско\-го 
института системных исследований Российской академии наук
%117218, Москва, Нахимовский просп., 36, к.1 

\vspace*{3pt}

\noindent \textbf{Васильев Николай Семенович} (р.\ 1952)~--- доктор 
фи\-зи\-ко-ма\-те\-ма\-ти\-че\-ских наук, профессор, 
МГТУ им.\ Н.\,Э.~Баумана 
%, Москва 105005, 2-я Бауманская ул., д.~5,

\vspace*{3pt}

\noindent
\textbf{Гершкович Максим Михайлович} (р.\ 1968)~---
старший научный сотрудник Института проб\-лем информатики
Российской академии наук

\vspace*{3pt}

\noindent 
\textbf{Дьяченко Юрий Георгиевич} (р.\ 1958)~--- кандидат технических наук, 
старший научный сотрудник Института проб\-лем информатики
Российской академии наук

\vspace*{3pt}

\noindent 
\textbf{Ерошенко Александр Андреевич} (р.\ 1989)~--- аспирант кафедры 
математической статистики факультета вычисли\-тельной математики и кибернетики 
Московского государственного университета им.\ М.\,В.~Ломоносова
%119991, Москва ГСП-1, Ленинские горы, д.\ 1, стр. 52

\vspace*{3pt}
 
\noindent 
\textbf{Захаров Виктор Николаевич} (р.\ 1948)~--- 
доктор технических наук, доцент, ученый секретарь Института проб\-лем информатики
Российской академии наук

\vspace*{3pt}

\noindent
\textbf{Зейфман Александр Израилевич} (р.\ 1954)~---
доктор фи\-зи\-ко-ма\-те\-ма\-ти\-че\-ских наук, профессор, 
заведующий кафедрой Вологодского государственного университета; 
старший научный сотрудник Института проб\-лем информатики
Российской академии наук; главный научный сотрудник ИСЭРТ Российской академии наук

\vspace*{3pt}

\noindent
\textbf{Зыкин Сергей Владимирович} (р.\ 1959)~--- 
доктор технических наук, профессор, заведующий лабораторией Института математики 
им.\ С.\,Л.~Соболева Сибирского отделения Российской академии наук, Новосибирск 
%630090, пр.\ ак.\ Коптюга, 4 

\vspace*{4pt}

\noindent
\textbf{Киреев Владимир Иванович} (р.\ 1938)~---
доктор фи\-зи\-ко-ма\-те\-ма\-ти\-че\-ских наук, профессор Московского 
государственного горного университета
%Адрес: Россия, 119991, г. Москва, Ленинский проспект, д. 6

%\columnbreak

\vspace*{4pt}

\noindent
\textbf{Козеренко Елена Борисовна} (р.\ 1959)~---
кандидат филологических наук, заведующая лабораторией Института проб\-лем информатики
Российской академии наук

\vspace*{4pt}

\noindent
\textbf{Королев Виктор Юрьевич} (р.\ 1954)~--- доктор
фи\-зи\-ко-ма\-те\-ма\-ти\-че\-ских наук, профессор кафедры математической 
статистики факультета вычисли\-тельной математики и кибернетики 
Московского государственного университета; 
ведущий научный сотрудник Института проб\-лем информатики
Российской академии наук

\vspace*{4pt}

\noindent
\textbf{Коротышева Анна Владимировна} (р.\ 1988)~---
старший преподаватель Вологодского государственного университета

\vspace*{4pt}

\noindent 
\textbf{Кун Де Турк} (р.\ 1981)~--- научный сотрудник 
исследовательской группы SMACS факультета телекоммуникаций и обработки информации
Университета Гента, Бельгия
%В-9000 Гент, Бельгия

\vspace*{4pt}

\noindent
\textbf{Лупенцов Олег Сергеевич} (р.\ 1986)~---
аспирант Омского государственного института сервиса
%Омск 644043, ул.\ Певцова 13

\vspace*{4pt}

\noindent
\textbf{Лучко Олег Николаевич} (р.\ 1961)~---
кандидат педагогических наук, профессор, заведующий кафедрой 
Омского государственного института сервиса
%Омск 644043, ул.\ Певцова 13

\vspace*{4pt}

\noindent
\textbf{Малашенко Юрий Евгеньевич} (р.\ 1946)~---
доктор фи\-зи\-ко-ма\-те\-ма\-ти\-че\-ских наук, заведующий сектором 
Вычислительного центра им.\ А.\,А.~Дородницына Российской академии наук
%Адрес: 119333, Москва, ул. Вавилова, 40,

\vspace*{4pt}

\noindent
\textbf{Маньяков Юрий Анатольевич} (р.\ 1984)~---
кандидат технических наук, научный сотрудник Орловского филиала Института проб\-лем информатики
Российской академии наук
%302025, г.Орел, Московское шоссе, д.137

\vspace*{4pt}

\noindent
\textbf{Маренко Валентина Афанасьевна} (р.\ 1951)~---
кандидат технических наук, доцент, старший научный сотрудник 
Института математики им.\ С.\,Л.~Соболева Сибирского отделения Российской академии наук
%Новосибирск 630090, пр. ак. Коптюга, 4 

\vspace*{3pt}

\noindent 
\textbf{Морозов Евсей Викторович} (р.\ 1947)~--- доктор 
фи\-зи\-ко-ма\-те\-ма\-ти\-че\-ских, профессор, ведущий научный сотрудник 
Института прикладных математических исследований Карельского научного центра Российской
академии наук; 
%%185910 Россия, Республика Карелия, г.\ Петрозаводск, ул.\ Пушкинская, 11
профессор Петрозаводского государственного университета, Петрозаводск
%185910 Россия, Республика Карелия, г.\ Петрозаводск, пр.\ Ленина, 33

%\pagebreak

\vspace*{3pt}

\noindent
\textbf{Назарова Ирина Александровна} (р.\ 1966)~---
кандидат фи\-зи\-ко-ма\-те\-ма\-ти\-че\-ских наук, 
научный сотрудник Вычислительного центра им.\ А.\,А.~Дородницына Российской академии наук 
%Адрес: 119333, Москва, ул. Вавилова, 40

\vspace*{3pt}

\noindent
\textbf{Павлов Игорь Валерианович} (р.\ 1945)~--- 
доктор фи\-зи\-ко-ма\-те\-ма\-ти\-че\-ских наук, профессор МГТУ им.\ Н.\,Э.~Баумана 
%Москва 105005, 2-я Бауманская ул., д.~5 

%\pagebreak

\vspace*{3pt}

\noindent 
\textbf{Потахина Любовь Викторовна} (р.\ 1989)~--- аспирантка
Института прикладных математических исследований Карельского научного центра
Российской академии наук; 
%%185910 Россия, Республика Карелия, г.\ Петрозаводск, ул.\ Пушкинская, 11
инженер Петрозаводского государственного университета, Петрозаводск
%185910 Россия, Республика Карелия, г.\ Петрозаводск, пр.\ Ленина, 33

\vspace*{3pt}

\noindent 
\textbf{Рождественский Юрий Владимирович} (р.\ 1952)~--- 
кандидат технических наук, заведующий сектором Института проб\-лем информатики
Российской академии наук

\vspace*{3pt}

\noindent 
\textbf{Синицын Игорь Николаевич} (р.\ 1940)~--- доктор технических наук,
профессор, заслуженный деятель\linebreak\vspace*{-12pt}

\columnbreak

\noindent
 науки РФ, заведующий отделом Института проб\-лем информатики
Российской академии наук

\vspace*{7pt}


\noindent
\textbf{Сиротинин Денис Олегович} (р.\ 1984)~---
кандидат технических наук, научный сотрудник Орловского филиала Института проб\-лем информатики
Российской академии наук
%302025, г.Орел, Московское шоссе, д.137

\vspace*{7pt}

%\columnbreak

\noindent 
\textbf{Соколов  Игорь Анатольевич} (р.\ 1954)~--- академик (действительный член) Российской 
академии наук, доктор технических наук, директор Института проб\-лем информатики
Российской академии наук

\vspace*{7pt}

\noindent
\textbf{Степченков Юрий Афанасьевич} (р.\ 1951)~---
кандидат технических наук, заведующий отделом Института проб\-лем информатики
Российской академии наук

\vspace*{7pt}

\noindent
\textbf{Сурков Алексей Викторович} (р.\ 1978)~--- 
старший научный сотрудник На\-уч\-но-ис\-сле\-до\-ва\-тель\-ско\-го 
института системных исследований Российской академии наук
%117218, Москва, Нахимовский просп., 36, к.1 

\vspace*{7pt}

\noindent 
\textbf{Шестаков Олег Владимирович} (р.\ 1976)~--- доктор 
фи\-зи\-ко-ма\-те\-ма\-ти\-че\-ских, доцент кафедры математической статистики 
факультета вычисли\-тельной математики и кибернетики Московского 
государственного университета им.\ М.\,В.~Ломоносова; 
%119991, Москва ГСП-1, Ленинские горы, д.\ 1, стр. 52
старший научный сотрудник Института проб\-лем информатики
Российской академии наук
%, Москва 119333, ул. Вавилова, д.~44, корп.~2

\vspace*{7pt}

\noindent 
\textbf{Шоргин Сергей Яковлевич} (р.\ 1952.)~--- доктор
фи\-зи\-ко-ма\-те\-ма\-ти\-че\-ских наук, профессор, заместитель директора Института 
проб\-лем информатики Российской академии наук





%%%%%%%%%%%%%%%%%%%%%%%%%%%%%%%%%%%%%%%%%%%%%%%%%%%%%%%%%%%%%%%%%%%%%%%%%%%%%%%




%\def\rightkol{ОБ АВТОРАХ}
%\def\leftkol{ОБ АВТОРАХ}

 \label{end\stat}





%\def\leftfootline{\small{\textbf{\thepage}
%\hfill ИНФОРМАТИКА И ЕЁ ПРИМЕНЕНИЯ\ \ \ том~7\ \ \ выпуск~1\ \ \ 2013}
%}%
% \def\rightfootline{\small{ИНФОРМАТИКА И ЕЁ ПРИМЕНЕНИЯ\ \ \ том~7\ \ \ выпуск~1\ \ \ 2013
%\hfill \textbf{\thepage}}}


%\thispagestyle{myheadings}



\end{multicols}

\newpage  

%\def\stat{cont}
{%\hrule\par
%\vskip 7pt % 7pt
\raggedleft\Large \bf%\baselineskip=3.2ex
А\,В\,Т\,О\,Р\,С\,К\,И\,Й\ \ У\,К\,А\,З\,А\,Т\,Е\,Л\,Ь\ \ З\,А\ \ 2\,0\,0\,7 г. \vskip 17pt
    \hrule
    \par
\vskip 21pt plus 6pt minus 3pt }

\label{st\stat}

\def\tit{\ }

\def\aut{\ }
\def\auf{\ }

\def\leftkol{\ } % ENGLISH ABSTRACTS}

\def\rightkol{\ } %ENGLISH ABSTRACTS}

\titele{\tit}{\aut}{\auf}{\leftkol}{\rightkol}


\contentsline {chapter}{\ }{Выпуск \quad Стр.} 
\contentsline {section}{\textbf{Батракова Д.\,А., Королев В.\,Ю., Шоргин С.\,Я.}\ \ Новый метод вероятностно-ста\-ти\-сти\-че\-ско\-го анализа информационных потоков в\nobreakspace {}телекоммуникационных сетях}{\qquad 1 \qquad 40} 
\contentsline {section}{\textbf{Борисов А.\,В.}\ \ Байесовское оценивание в системах наблюдения с\nobreakspace {}марковскими скачкообразными процессами: игровой подход}{\qquad 2 \qquad 65}
\contentsline {section}{\textbf{Босов А.\,В., Иванов А.\,В.}\ \ Программная инфраструктура информационного Web-пор\-тала}{\qquad 2 \qquad 50}
\contentsline {section}{\textbf{Захаров В.\,Н., Калиниченко Л.\,А., Соколов И.\,А., Ступников С.\,А.}\ \ Конструирование канонических информационных моделей для интегрированных информационных систем}{\qquad 2 \qquad 15}
\contentsline {section}{\textbf{Захаров В.\,Н., Козмидиади В.\,А.}\ \ Средства обеспечения отказоустойчивости при\-ло\-жений}{\qquad 1 \qquad 14} 
\contentsline {section}{\textbf{Иванов А.\,В.}\ \ см. Босов А.\,В.\hfill\hfill\hfill\hfill\hfill\hfill\hfill\hfill\hfill\hfill\hfill\hfill\hfill\hfill\hfill\hfill\hfill\hfill\hfill\hfill\hfill\hfill\hfill\hfill\hfill\hfill\hfill\hfill\hfill\hfill\hfill\hfill\hfill\hfill\hfill}{\ }
\contentsline {section}{\textbf{Ильин В.\,Д., Соколов И.\,А.}\ \ Символьная модель системы знаний информатики в\nobreakspace {}че\-ло\-ве\-ко-автоматной среде}{\qquad 1 \qquad 66} 
\contentsline {section}{\textbf{Калиниченко Л.\,А.}\ \ см. Захаров В.\,Н.\hfill\hfill\hfill\hfill\hfill\hfill\hfill\hfill\hfill\hfill\hfill\hfill\hfill\hfill\hfill\hfill\hfill\hfill\hfill\hfill\hfill\hfill\hfill\hfill\hfill\hfill\hfill\hfill\hfill\hfill\hfill\hfill\hfill\hfill\hfill}{\ }
\contentsline {section}{\textbf{Козеренко Е.\,Б.}\ \ Лингвистическое моделирование для систем машинного перевода и обработки знаний}{\qquad 1 \qquad 54} 
\contentsline {section}{\textbf{Козмидиади В.\,А.}\ \ см. Захаров В.\,Н.\hfill\hfill\hfill\hfill\hfill\hfill\hfill\hfill\hfill\hfill\hfill\hfill\hfill\hfill\hfill\hfill\hfill\hfill\hfill\hfill\hfill\hfill\hfill\hfill\hfill\hfill\hfill\hfill\hfill\hfill\hfill\hfill\hfill\hfill\hfill }{\ } 
\contentsline {section}{\textbf{Королев В.\,Ю.}\ \ см. Батракова Д.\,А.\hfill\hfill\hfill\hfill\hfill\hfill\hfill\hfill\hfill\hfill\hfill\hfill\hfill\hfill\hfill\hfill\hfill\hfill\hfill\hfill\hfill\hfill\hfill\hfill\hfill\hfill\hfill\hfill\hfill\hfill\hfill\hfill\hfill\hfill\hfill}{\ } 
\contentsline {section}{\textbf{Кудрявцев А.\,А., Шоргин С.\,Я.}\ \ Байесовский подход к\nobreakspace {}анализу систем массового обслуживания и\nobreakspace {}показателей надежности}{\qquad 2 \qquad 76}
\contentsline {section}{\textbf{Печинкин А.\,В., Соколов И.\,А., Чаплыгин В.\,В.}\ \ Многолинейная система массового обслуживания с конечным накопителем и ненадежными приборами}{\qquad 1 \qquad 27} 
\contentsline {section}{\textbf{Печинкин А.\,В., Соколов И.\,А., Чаплыгин В.\,В.}\ \ Стационарные характеристики многолинейной\nobreakspace {}системы массового обслуживания с\nobreakspace {}одновременными отказами приборов}{\qquad 2 \qquad 39}
\contentsline {section}{\textbf{Синицын И.\,Н.}\ \ Корреляционные методы построения аналитических информационных моделей флуктуаций полюса Земли по априорным данным}{\qquad 2 \qquad \hphantom{9}2}
\contentsline {section}{\textbf{Синицын И.\,Н.}\ \ Развитие теории фильтров Пугачева для оперативной обработки информации в стохастических системах}{{\qquad 1 \qquad \hphantom{9}3}} 
\contentsline {section}{\textbf{Соколов И.\,А.}\ \ см. Захаров В.\,Н.\hfill\hfill\hfill\hfill\hfill\hfill\hfill\hfill\hfill\hfill\hfill\hfill\hfill\hfill\hfill\hfill\hfill\hfill\hfill\hfill\hfill\hfill\hfill\hfill\hfill\hfill\hfill\hfill\hfill\hfill\hfill\hfill\hfill\hfill\hfill}{\ }
\contentsline {section}{\textbf{Соколов И.\,А.}\ \ см. Ильин В.\,Д.\hfill\hfill\hfill\hfill\hfill\hfill\hfill\hfill\hfill\hfill\hfill\hfill\hfill\hfill\hfill\hfill\hfill\hfill\hfill\hfill\hfill\hfill\hfill\hfill\hfill\hfill\hfill\hfill\hfill\hfill\hfill\hfill\hfill\hfill\hfill}{\ } 
\contentsline {section}{\textbf{Соколов И.\,А.}\ \ см. Печинкин А.\,В.\hfill\hfill\hfill\hfill\hfill\hfill\hfill\hfill\hfill\hfill\hfill\hfill\hfill\hfill\hfill\hfill\hfill\hfill\hfill\hfill\hfill\hfill\hfill\hfill\hfill\hfill\hfill\hfill\hfill\hfill\hfill\hfill\hfill\hfill\hfill}{\ } 
\contentsline {section}{\textbf{Соколов И.\,А.}\ \ см. Печинкин А.\,В.\hfill\hfill\hfill\hfill\hfill\hfill\hfill\hfill\hfill\hfill\hfill\hfill\hfill\hfill\hfill\hfill\hfill\hfill\hfill\hfill\hfill\hfill\hfill\hfill\hfill\hfill\hfill\hfill\hfill\hfill\hfill\hfill\hfill\hfill\hfill}{\ }
\contentsline {section}{\textbf{Ступников С.\,А.}\ \ см. Захаров В.\,Н.\hfill\hfill\hfill\hfill\hfill\hfill\hfill\hfill\hfill\hfill\hfill\hfill\hfill\hfill\hfill\hfill\hfill\hfill\hfill\hfill\hfill\hfill\hfill\hfill\hfill\hfill\hfill\hfill\hfill\hfill\hfill\hfill\hfill\hfill\hfill}{\ }
\contentsline {section}{\textbf{Чаплыгин В.\,В.}\ \ см. Печинкин А.\,В.\hfill\hfill\hfill\hfill\hfill\hfill\hfill\hfill\hfill\hfill\hfill\hfill\hfill\hfill\hfill\hfill\hfill\hfill\hfill\hfill\hfill\hfill\hfill\hfill\hfill\hfill\hfill\hfill\hfill\hfill\hfill\hfill\hfill\hfill\hfill}{\ } 
\contentsline {section}{\textbf{Чаплыгин В.\,В.}\ \ см. Печинкин А.\,В.\hfill\hfill\hfill\hfill\hfill\hfill\hfill\hfill\hfill\hfill\hfill\hfill\hfill\hfill\hfill\hfill\hfill\hfill\hfill\hfill\hfill\hfill\hfill\hfill\hfill\hfill\hfill\hfill\hfill\hfill\hfill\hfill\hfill\hfill\hfill}{\ }
\contentsline {section}{\textbf{Шоргин С.\,Я.}\ \ см. Батракова Д.\,А.\hfill\hfill\hfill\hfill\hfill\hfill\hfill\hfill\hfill\hfill\hfill\hfill\hfill\hfill\hfill\hfill\hfill\hfill\hfill\hfill\hfill\hfill\hfill\hfill\hfill\hfill\hfill\hfill\hfill\hfill\hfill\hfill\hfill\hfill\hfill}{\ } 
\contentsline {section}{\textbf{Шоргин С.\,Я.}\ \ см. Кудрявцев А.\,А.\hfill\hfill\hfill\hfill\hfill\hfill\hfill\hfill\hfill\hfill\hfill\hfill\hfill\hfill\hfill\hfill\hfill\hfill\hfill\hfill\hfill\hfill\hfill\hfill\hfill\hfill\hfill\hfill\hfill\hfill\hfill\hfill\hfill\hfill\hfill}{\ }
%\thispagestyle{myheadings}
\def\leftfootline{\small{\textbf{\thepage}
\hfill ИНФОРМАТИКА И ЕЁ ПРИМЕНЕНИЯ\ \ \ том~1\ \ \ выпуск~2\ \ \ 2007}
}%
 \def\rightfootline{\small{ИНФОРМАТИКА И ЕЁ ПРИМЕНЕНИЯ\ \ \ том~1\ \ \ выпуск~2\ \ \ 2007
 \hfill \textbf{\thepage}}}
 \label{end\stat} 
                     
%\def\stat{cont-e}
{%\hrule\par
%\vskip 7pt % 7pt
\raggedleft\Large \bf%\baselineskip=3.2ex
2\,0\,0\,7\ \ A\,U\,T\,H\,O\,R\ \ I\,N\,D\,E\,X \vskip 17pt
    \hrule
    \par
\vskip 21pt plus 6pt minus 3pt }

\label{st\stat}

\def\tit{\ }

\def\aut{\ }
\def\auf{\ }

\def\leftkol{\ } % ENGLISH ABSTRACTS}

\def\rightkol{\ } %ENGLISH ABSTRACTS}

\titele{\tit}{\aut}{\auf}{\leftkol}{\rightkol}


\contentsline {chapter}{\ }{Issue \quad Page} 
\contentsline {subsection}{\textbf{Batrakova D.\,A., Korolev V.\,Yu., Shorgin S.\,Ya.}\ \ A New Method for the Probabilistic and Statistical Analysis of Information Flows in Telecommunication Networks}{\qquad 1 \qquad 40} 
\contentsline {subsection}{\textbf{Borisov A.\,V.}\ \ Bayesian Estimation in\nobreakspace {}Observation Systems with\nobreakspace {}Markov Jump Processes: Game-Theoretic Approach}{\qquad 2 \qquad 65} 
\contentsline {subsection}{\textbf{Bosov A.\,V., Ivanov A.\,V.}\ \ Linguistic Simulation for Machine Translation and Knowledge Management Systems}{\qquad 2 \qquad 50} 
\contentsline {subsection}{\textbf{Chaplygin V.\,V.} see Pechinkin A.\,V.\hfill\hfill\hfill\hfill\hfill\hfill\hfill\hfill\hfill\hfill\hfill\hfill\hfill\hfill\hfill\hfill\hfill\hfill\hfill\hfill\hfill\hfill\hfill\hfill\hfill\hfill\hfill\hfill\hfill\hfill\hfill\hfill\hfill\hfill\hfill}{\ }
\contentsline {subsection}{\textbf{Chaplygin V.\,V.} see Pechinkin A.\,V.\hfill\hfill\hfill\hfill\hfill\hfill\hfill\hfill\hfill\hfill\hfill\hfill\hfill\hfill\hfill\hfill\hfill\hfill\hfill\hfill\hfill\hfill\hfill\hfill\hfill\hfill\hfill\hfill\hfill\hfill\hfill\hfill\hfill\hfill\hfill}{\ }
\contentsline {subsection}{\textbf{Ilyin V.\,D., Sokolov I.\,A.}\ \ The Symbol Model of Informatics Knowledge System in Human-Automaton Environment}{\qquad 1 \qquad 66} 
\contentsline {subsection}{\textbf{Ivanov A.\,V.} see Bosov A.\,V.\hfill\hfill\hfill\hfill\hfill\hfill\hfill\hfill\hfill\hfill\hfill\hfill\hfill\hfill\hfill\hfill\hfill\hfill\hfill\hfill\hfill\hfill\hfill\hfill\hfill\hfill\hfill\hfill\hfill\hfill\hfill\hfill\hfill\hfill\hfill}{\ }
\contentsline {subsection}{\textbf{Kalinichenko L.\,A.} see Zakharov V.\,N.\hfill\hfill\hfill\hfill\hfill\hfill\hfill\hfill\hfill\hfill\hfill\hfill\hfill\hfill\hfill\hfill\hfill\hfill\hfill\hfill\hfill\hfill\hfill\hfill\hfill\hfill\hfill\hfill\hfill\hfill\hfill\hfill\hfill\hfill\hfill}{\ }
\contentsline {subsection}{\textbf{Korolev V.\,Yu.} see Batrakova D.\,A.\hfill\hfill\hfill\hfill\hfill\hfill\hfill\hfill\hfill\hfill\hfill\hfill\hfill\hfill\hfill\hfill\hfill\hfill\hfill\hfill\hfill\hfill\hfill\hfill\hfill\hfill\hfill\hfill\hfill\hfill\hfill\hfill\hfill\hfill\hfill}{\ }
\contentsline {subsection}{\textbf{Kozerenko E.\,B.}\ \ Linguistic Simulation for Machine Translation and Knowledge Management Systems}{\qquad 1 \qquad 54} 
\contentsline {subsection}{\textbf{Kozmidiady V.\,A.} see Zakharov V.\,N.\hfill\hfill\hfill\hfill\hfill\hfill\hfill\hfill\hfill\hfill\hfill\hfill\hfill\hfill\hfill\hfill\hfill\hfill\hfill\hfill\hfill\hfill\hfill\hfill\hfill\hfill\hfill\hfill\hfill\hfill\hfill\hfill\hfill\hfill\hfill}{\ }
\contentsline {subsection}{\textbf{Kudryavtsev A.\,A., Shorgin S.\,Ya.}\ \ Bayesian Approach to Queueing Systems and Reliability Characteristics}{\qquad 2 \qquad 76} 
\contentsline {subsection}{\textbf{Pechinkin A.\,V., Sokolov I.\,A., Chaplygin V.\,V.}\ \ Multichannel Queuing System with Finite Buffer and Unreliable Servers}{\qquad 1 \qquad 27} 
\contentsline {subsection}{\textbf{Pechinkin A.\,V., Sokolov I.\,A., Chaplygin V.\,V.}\ \ Stationary Characteristics of a Multichannel Queueing System with\nobreakspace {}Simultaneous Refusals of Servers}{\qquad 2 \qquad 39} 
\contentsline {subsection}{\textbf{Shorgin S.\,Ya.} see Batrakova D.\,A.\hfill\hfill\hfill\hfill\hfill\hfill\hfill\hfill\hfill\hfill\hfill\hfill\hfill\hfill\hfill\hfill\hfill\hfill\hfill\hfill\hfill\hfill\hfill\hfill\hfill\hfill\hfill\hfill\hfill\hfill\hfill\hfill\hfill\hfill\hfill}{\ }
\contentsline {subsection}{\textbf{Shorgin S.\,Ya.} see Kudryavtsev A.\,A.\hfill\hfill\hfill\hfill\hfill\hfill\hfill\hfill\hfill\hfill\hfill\hfill\hfill\hfill\hfill\hfill\hfill\hfill\hfill\hfill\hfill\hfill\hfill\hfill\hfill\hfill\hfill\hfill\hfill\hfill\hfill\hfill\hfill\hfill\hfill}{\ }
\contentsline {subsection}{\textbf{Sinitsyn I.\,N.}\ \ Correlational Methods for Analytical Informational Models of the Earth Pole Fluctuations Design Based on a priori Data}{\qquad 2 \qquad \hphantom{9}2}
\contentsline {subsection}{\textbf{Sinitsyn I.\,N.}\ \ Development of Pugachev Filtering for Stochastic Systems}{\qquad 1 \qquad \hphantom{9}3}
\contentsline {subsection}{\textbf{Sokolov I.\,A.} see Ilyin V.\,D.\hfill\hfill\hfill\hfill\hfill\hfill\hfill\hfill\hfill\hfill\hfill\hfill\hfill\hfill\hfill\hfill\hfill\hfill\hfill\hfill\hfill\hfill\hfill\hfill\hfill\hfill\hfill\hfill\hfill\hfill\hfill\hfill\hfill\hfill\hfill}{\ }
\contentsline {subsection}{\textbf{Sokolov I.\,A.} see Pechinkin A.\,V.\hfill\hfill\hfill\hfill\hfill\hfill\hfill\hfill\hfill\hfill\hfill\hfill\hfill\hfill\hfill\hfill\hfill\hfill\hfill\hfill\hfill\hfill\hfill\hfill\hfill\hfill\hfill\hfill\hfill\hfill\hfill\hfill\hfill\hfill\hfill}{\ }
\contentsline {subsection}{\textbf{Sokolov I.\,A.} see Pechinkin A.\,V.\hfill\hfill\hfill\hfill\hfill\hfill\hfill\hfill\hfill\hfill\hfill\hfill\hfill\hfill\hfill\hfill\hfill\hfill\hfill\hfill\hfill\hfill\hfill\hfill\hfill\hfill\hfill\hfill\hfill\hfill\hfill\hfill\hfill\hfill\hfill}{\ }
\contentsline {subsection}{\textbf{Sokolov I.\,A.} see Zakharov V.\,N.\hfill\hfill\hfill\hfill\hfill\hfill\hfill\hfill\hfill\hfill\hfill\hfill\hfill\hfill\hfill\hfill\hfill\hfill\hfill\hfill\hfill\hfill\hfill\hfill\hfill\hfill\hfill\hfill\hfill\hfill\hfill\hfill\hfill\hfill\hfill}{\ }
\contentsline {subsection}{\textbf{Stupnikov S.\,A.} see Zakharov V.\,N.\hfill\hfill\hfill\hfill\hfill\hfill\hfill\hfill\hfill\hfill\hfill\hfill\hfill\hfill\hfill\hfill\hfill\hfill\hfill\hfill\hfill\hfill\hfill\hfill\hfill\hfill\hfill\hfill\hfill\hfill\hfill\hfill\hfill\hfill\hfill}{\ }
\contentsline {subsection}{\textbf{Zakharov V.\,N., Kalinichenko L.\,A., Sokolov I.\,A., Stupnikov S.\,A.}\ \ Development of Canonical Information Models for Integrated Information Systems}{\qquad 2 \qquad 15} 
\contentsline {subsection}{\textbf{Zakharov V.\,N., Kozmidiady V.\,A.}\ \ Means Providing Applications Fault Tolerance}{\qquad 1 \qquad 14} 
\def\leftfootline{\small{\textbf{\thepage}
\hfill ИНФОРМАТИКА И ЕЁ ПРИМЕНЕНИЯ\ \ \ том~1\ \ \ выпуск~2\ \ \ 2007}
}%
 \def\rightfootline{\small{ИНФОРМАТИКА И ЕЁ ПРИМЕНЕНИЯ\ \ \ том~1\ \ \ выпуск~2\ \ \ 2007
 \hfill \textbf{\thepage}}}
 \label{end\stat} 


%\end{document}

%
\def\stat{rekl}
%\label{preobr}

%\def\tit{АКАДЕМИК ПУГАЧЁВ  ВЛАДИМИР СЕМЁНОВИЧ\\
%25.03.1911--25.03.1998}


%   \vspace*{-48pt}
%   \begin{center}\LARGE
%Академик Пугачёв  Владимир Семёнович\\ (25.03.1911--25.03.1998)
%   \end{center}

   %\vspace*{2.5mm}

   \begin{center}

{\prgsh\LARGE
ЮБИЛЕИ}

\end{center}
%\hrule

\vspace*{6pt}


   \vspace*{8mm}

   \thispagestyle{empty}


%\def\stat{emel}


\section*{К 70-летию заместителя директора ИПИ РАН,\\ члена редколлегии журнала
<<Информатика и её применения>>\\ доктора технических наук В.\,И.~Будзко}

\vspace*{18pt}




          \begin{multicols}{2}

%            \label{st\stat}

\begin{center}
\vspace*{1pt}
\mbox{%
\epsfxsize=78mm
\epsfbox{bud-1.eps}
}
\end{center}

\vspace*{12pt}

      14 августа 2014~г.\ исполнилось 70~лет за\-мес\-ти\-те\-лю директора ИПИ РАН по
научной работе доктору технических наук Владимиру Игоревичу Будзко.

      Владимир Игоревич Будзко родился в г.~Москве. Высшее образование получил на факультете
элект\-рон\-но-вы\-чис\-ли\-тель\-ных устройств в Московском
ин\-же\-нер\-но-фи\-зи\-че\-ском институте
(МИФИ), который он окончил в 1968~г., после чего был на\-прав\-лен для прохождения
службы в одну из войс\-ко\-вых частей, где прошел путь от инженера до первого заместителя
командира войсковой части.

      С приходом В.\,И.~Будзко в ИПИ РАН (2001~г.)\ в институте
сформировалось новое научное на\-прав\-ле\-ние теоретических исследований~--- <<Постро\-ение
ин\-фор\-ма\-ци\-он\-но-те\-ле\-ком\-му\-ни\-ка\-ци\-он\-ных\linebreak сис\-тем
высокой до\-ступ\-ности>>. В~рамках этого
направления выполнен широкий круг фундаментальных исследований по поиску подходов и
определению принципов построения средств обеспечения доступности, конфиденциальности
и целостности современных крупномасштабных
ин\-фор\-ма\-ци\-он\-но-те\-ле\-ком\-му\-ни\-ка\-ци\-он\-ных
сис\-тем (ИТС). Разработаны основные сис\-тем\-но-тех\-ни\-че\-ские принципы и базовые
архитектурные решения построения перспективных для условий России ИТС с
централизованной обработкой и хранением информации, сочетающих в себе свойства
высокой доступности, отказо- и катастрофоустойчивости, информационной защищенности.
Определены принципы, методы и математические основы рационального построения и
оптимизации средств восстановления функционирования центров обработки данных (ЦОД)
после возникновения отказов и катастроф, передачи и хранения данных, обеспечения
информационной безопасности при достижении минимальной совокупной стоимости
владения такими системами. Результаты нашли практическое воплощение при реализации
проектов в интересах ряда отечественных государственных и негосударственных
организаций, таких как Банк России (БР), Внешторгбанк, ОАО <<ГМК <<Норильский Никель>>,
<<Газпром>>, Минэкономразвития России, Правительство Москвы, а также ряд силовых
ведомств.

      Под руководством В.\,И.~Будзко начиная с 2001~г.\ выполнен комплекс
      на\-уч\-но-ис\-сле\-до\-ва\-тель\-ских и
      опыт\-но-кон\-ст\-рук\-тор\-ских работ (свыше 100~проектов),
направленных на развитие электронной информационной технологии БР.
Разработаны концепции развития ИТС БР сначала до 2008~г., а затем до 2013~г., которые
были приняты в качестве основы проведения технической политики. За реализацию проекта
<<Катастрофоустойчивая тер\-ри\-то\-ри\-аль\-но-рас\-пре\-де\-лен\-ная
      ин\-фор\-ма\-ци\-он\-но-те\-ле\-ком\-му\-ни\-ка\-ци\-он\-ная сис\-те\-ма централизованной
обработки банковской информации>> В.\,И.~Будзко удостоен Премии Правительства РФ в
области науки и техники за 2010~г.

      В.\,И.~Будзко возглавлял и возглавляет работы по ряду других прикладных проектов,
связанных с созданием, совершенствованием и развитием крупномасштабных ИТС.

      В.\,И.~Будзко~--- генерал-майор, доктор технических наук, член-кор\-рес\-пон\-дент
Академии криптографии РФ, известный ученый в области информатики и применения
информационных технологий при построении территориально распределенных ИТС
различного назначения. Является автором свыше 250~научных работ, опубликованных в
на\-уч\-но-тех\-ни\-че\-ских и специальных изданиях.

    \thispagestyle{empty}

      В.\,И.~Будзко уделяет большое внимание подготовке научных кадров. Под его
руководством защищено 6~диссертаций на соискание ученой степени кандидата
технических наук. Свыше 30~лет он читает лекции в ИКСИ Академии ФСБ, профессор
кафедры НИЯУ МИФИ. Является членом двух диссертационных советов, главным
редактором журнала <<Системы высокой доступности>> и членом редколлегии журнала
<<Информатика и её применения>>.

      \bigskip

      Редакционный совет и Редакционная коллегия журнала <<Информатика и её
применения>> сердечно поздравляют Владимира Игоревича Будзко с 70-ле\-ти\-ем и желают
крепкого здоровья и новых научных достижений.

\end{multicols}

%%Информатика и её применения
%Том 14 Выпуск 1-4 Год 2020

\def\stat{cont}
{%\hrule\par
%\vskip 7pt % 7pt
\raggedleft\Large \bf%\baselineskip=3.2ex
А\,В\,Т\,О\,Р\,С\,К\,И\,Й\ \ У\,К\,А\,З\,А\,Т\,Е\,Л\,Ь\ \ З\,А\ \ 2\,0\,2\,0 г. \vskip 17pt
 \hrule
 \par
\vskip 21pt plus 6pt minus 3pt }

\label{st\stat}

\def\tit{\ }

\def\aut{\ }
\def\auf{\ }

\def\leftkol{\ } % ENGLISH ABSTRACTS}

\def\rightkol{\ } %АВТОРСКИЙ УКАЗАТЕЛЬ ЗА 2020 г.} %ENGLISH ABSTRACTS}

\titele{\tit}{\aut}{\auf}{\leftkol}{\rightkol}
\addcontentsline{toc}{subsection}{\textrm\textbf Авторский указатель за 2020 г.}

\vspace*{-24pt}

\noindent
{\tabcolsep=3pt
\begin{tabular}{p{397pt}cc}
&\textbf{Вып.} & \textbf{Стр.}\\[6pt]
\Avtors{Абгарян~К.\,К., Гаврилов~Е.\,С.} Интеграционная платформа для многомасштабного моде-\linebreak
\\[-12pt]
\hspace*{23pt}лирования нейроморфных систем&2&104--110\\
\Avtors{Абгарян~К.\,К., Колбин~И.\,С.} Применение многомасштабного подхода и методов анализа\linebreak
\\[-12pt]
\hspace*{23pt}данных для моделирования теплопроводности в слоистых структурах&4&91--99\\
\Avtors{Агаларов~Я.\,М.} Оптимизация емкости основного накопителя в системе массового\linebreak
\\[-12pt]
\hspace*{23pt}обслуживания типа $G/M/1/K$ с дополнительным накопителем&2&72--79\\
\Avtors{Агасандян~Г.\,А.} Вычислительные аспекты применения CC-VaR на совокупности рынков&3&62--70\\
\Avtors{Агеев~К.\,А., Сопин~Э.\,С., Яркина~Н.\,В., Самуйлов~К.\,Е., Шоргин~С.\,Я.} Анализ механизмов\linebreak
\\[-12pt]
\hspace*{23pt}нарезки сети с учетом гарантий для различных типов трафика&3&\hphantom{1}94--100\\
\Avtors{Адамова~К.\,А.} см.\ Шнурков~П.\,В.&&\\
\Avtors{Базилевский~М.\,П.} Многофакторные модели полносвязной линейной регрессии без\linebreak
\\[-12pt]
\hspace*{23pt}ограничений на соотношения дисперсий ошибок переменных&2&92--97\\
\Avtors{Бахтеев~О.\,Ю.} см.\ Грабовой~А.\,В.&&\\
\Avtors{Беленков~В.\,Г.} см.\ Будзко~В.\,И.&&\\
\Avtors{Бетелин~В.\,Б., Кушниренко~А.\,Г., Леонов~А.\,Г.} Основные понятия программирования\linebreak
\\[-12pt]
\hspace*{23pt}в изложении для дошкольников&3&55--61\\
\Avtors{Бетелин~В.\,Б., Кушниренко~А.\,Г., Семенов~А.\,Л., Сопрунов~С.\,Ф.} О цифровой грамотности\linebreak
\\[-12pt]
\hspace*{23pt}и средах ее формирования&4&100--107\\
\Avtors{Борисов~А.\,В.} Численные схемы фильтрации марковских скачкообразных процессов по\linebreak
\\[-12pt]
\hspace*{23pt}дискретизованным наблюдениям II: случай аддитивных шумов&1&17--23\\
\Avtors{Борисов~А.\,В.} Численные схемы фильтрации марковских скачкообразных процессов по\linebreak
\\[-12pt]
\hspace*{23pt}дискретизованным наблюдениям III: случай мультипликативных шумов&2&10--18\\
\Avtors{Босов~А.\,В.} Управление выходом стохастической дифференциальной системы по квад-\linebreak
\\[-12pt]
\hspace*{23pt}ратичному критерию. V. Случай неполной информации о состоянии&2&19--25\\
\Avtors{Босов~А.\,В., Мартюшова~Я.\,Г., Наумов~А.\,В., Сапунова~А.\,П.} Байесовский подход к~по\-стро\-ению индивидуальной траектории пользователя в~системе дистанционного\linebreak
\\[-12pt]
\hspace*{23pt}обучения&3&86--93\\
\Avtors{Босов~А.\,В., Стефанович~А.\,И.} Управление выходом стохастической дифференциальной\linebreak
\\[-12pt]
\hspace*{23pt}системы по квадратичному критерию. IV. Альтернативное численное решение&1&24--30\\
\Avtors{Брюхов~Д.\,О., Ступников~С.\,А., Ковалёв~Д.\,Ю., Шанин~И.\,А.} Нейрофизиология как\linebreak
\\[-12pt]
\hspace*{23pt}предметная область для решения задач с интенсивным использованием данных&1&40--47\\
\Avtors{Будзко~В.\,И., Ядринцев~В.\,В., Соченков~И.\,В., Королёв~В.\,И., Беленков~В.\,Г.} Об одном подходе
 к формированию в условиях высокой неопределенности марке-\linebreak
\\[-12pt]
\hspace*{23pt}ров конфиденциальности в системах интенсивного использования данных&4&69--76\\
\Avtors{Вайсер~К.\,О.} см.\ Потанин~М.\,С.&&\\
\Avtors{Вохминцев~А.\,В., Мельников~А.\,В., Пачганов~C.\,А.} Метод навигации и составления карты в трехмерном пространстве на основе комбинированного решения вариационной\linebreak
\\[-12pt]
\hspace*{23pt}подзадачи точка--точка ICP для аффинных преобразований&1&101--112\\
\Avtors{Гаврилов~Е.\,С.} см.\ Абгарян~К.\,К.&&\\
\Avtors{Гайдамака~Ю.\,В.} см.\  Москалева~Ф.\,А.&&\\
\Avtors{Голембиовский~Д.\,Ю.} см.\ Данилишин~А.\,Р.&&\\
\Avtors{Голембиовский~Д.\,Ю.} см.\ Данилишин~А.\,Р.&&\\
\Avtors{Гончаров~А.\,А., Зацман~И.\,М., Кружков~М.\,Г.} Эволюция классификаций в надкорпусных\linebreak
\\[-12pt]
\hspace*{23pt}базах данных&4&108--116\\
\Avtors{Гончаров~А.\,В., Стрижов~В.\,В.} Выравнивание декартовых произведений упорядоченных\linebreak
\\[-12pt]
\hspace*{23pt}множеств&1&31--39\\
\end{tabular}
}

\pagebreak

\def\leftkol{АВТОРСКИЙ УКАЗАТЕЛЬ ЗА 2020 г.} % ENGLISH ABSTRACTS}

\def\rightkol{АВТОРСКИЙ УКАЗАТЕЛЬ ЗА 2020 г.} %ENGLISH ABSTRACTS}

%\thispagestyle{myheadings}
\def\leftfootline{\small{\textbf{\thepage}
\hfill ИНФОРМАТИКА И ЕЁ ПРИМЕНЕНИЯ\ \ \ том~14\ \ \ выпуск~4\ \ \ 2020}
}%
 \def\rightfootline{\small{ИНФОРМАТИКА И ЕЁ ПРИМЕНЕНИЯ\ \ \ том~14\ \ \ выпуск~4\ \ \ 2020
 \hfill \textbf{\thepage}}}


\noindent
{\tabcolsep=3pt
\begin{tabular}{p{394pt}cc}
&\textbf{Вып.} & \textbf{Стр.}\\[3pt]
\Avtors{Горшенин~А.\,К., Королев~В.\,Ю.} Аппроксимация распределений размеров частиц лунного\linebreak
\\[-12pt]
\hspace*{23pt}реголита на основе метода статистической симуляции выборок&2&50--57\\
\Avtors{Горшенин~А.\,К., Королев~В.\,Ю., Щербинина~А.\,А.} Статистическое оценивание распределений случайных коэффициентов стохастического дифференциального уравнения\linebreak
\\[-12pt]
\hspace*{23pt}Ланжевена&3&\hphantom{1}3--12\\
\Avtors{Горшенин~А.\,К., Кузьмин~В.\,Ю.} Анализ конфигураций LSTM-сетей для построения\linebreak
\\[-12pt]
\hspace*{23pt}среднесрочных векторных прогнозов&1&10--16\\
\Avtors{Грабовой~А.\,В., Бахтеев~О.\,Ю., Стрижов~В.\,В.} Введение отношения порядка на множестве\linebreak
\\[-12pt]
\hspace*{23pt}параметров аппроксимирующих моделей&2&58--65\\
\Avtors{Грушо~А.\,А., Забежайло~М.\,И., Смирнов~Д.\,В., Тимонина~Е.\,Е.} О вероятностных оценках\linebreak
\\[-12pt]
\hspace*{23pt}достоверности эмпирических выводов&4&3--8\\
\Avtors{Грушо~А.\,А., Забежайло~М.\,И., Смирнов~Д.\,В., Тимонина~Е.\,Е., Шоргин~С.\,Я.} Методы\linebreak
\\[-12pt]
\hspace*{23pt}математической статистики в задаче поиска инсайдера&3&71--75\\
\Avtors{Грушо~А.\,А., Забежайло~М.\,И., Тимонина~Е.\,Е.} О каузальной репрезентативности обуча-\linebreak
\\[-12pt]
\hspace*{23pt}ющих выборок прецедентов в задачах диагностического типа&1&80--86\\
\Avtors{Грушо~А.\,А., Тимонина~Е.\,Е., Грушо~Н.\,А., Терехина~И.\,Ю.} Выявление аномалий с по-\linebreak
\\[-12pt]
\hspace*{23pt}мощью метаданных&3&76--80\\
\Avtors{Грушо~А.\,А.} см.\ Грушо~Н.\,А.&&\\
\Avtors{Грушо~Н.\,А., Грушо~А.\,А., Забежайло~М.\,И., Тимонина~Е.\,Е.} Методы нахождения причин\linebreak
\\[-12pt]
\hspace*{23pt}сбоев в информационных технологиях  с помощью метаданных&2&33--39\\
\Avtors{Грушо~Н.\,А.} см.\ Грушо~А.\,А.&&\\
\Avtors{Данилишин~А.\,Р., Голембиовский~Д.\,Ю.} Оценка стоимости опционов на основе моделей\linebreak
\\[-12pt]
\hspace*{23pt}ARIMA--GARCH с ошибками, распределенными по закону $S_u$ Джонсона&4&83--90\\
\Avtors{Данилишин~А.\,Р., Голембиовский~Д.\,Ю.} Риск-нейтральная динамика для модели ARIMA-\linebreak
\\[-12pt]
\hspace*{23pt}GARCH с ошибками, распределенными по закону $S_U$ Джонсона&1&48--55\\
\Avtors{Диментов~А.\,В.} см.\ Краснов~Ф.\,В.&&\\
\Avtors{Донской~В.\,И.} Извлечение оптимизационных моделей из данных&3&109--118\\
\Avtors{Дубнов~Ю.\,А.} см.\ Попков~Ю.\,С.&&\\
\Avtors{Дулин~С.\,К., Дулина~Н.\,Г., Ермаков~П.\,В.} Информационный синтез документов&1&128--135\\
\Avtors{Дулина~Н.\,Г.} см.\ Дулин~С.\,К.&&\\
\Avtors{Дьяченко~Ю.\,Г.} см.\ Соколов~И.\,А.&&\\
\Avtors{Ермаков~П.\,В.} см.\ Дулин~С.\,К.&&\\
\Avtors{Ефросинин~Д.\,В.} см.\ Харин~П.\,А.&&\\
\Avtors{Жолобов~В.\,А.} см.\ Потанин~М.\,С.&&\\
\Avtors{Забежайло~М.\,И.} см.\ Грушо~А.\,А.&&\\
\Avtors{Забежайло~М.\,И.} см.\ Грушо~А.\,А.&&\\
\Avtors{Забежайло~М.\,И.} см.\ Грушо~А.\,А.&&\\
\Avtors{Забежайло~М.\,И.} см.\ Грушо~Н.\,А.&&\\
\Avtors{Захаров В. Н.} см.\ Френкель С. Л.&&\\
\Avtors{Зацман~И.\,М.} Проблемно-ориентированная верификация полноты темпоральных\linebreak
\\[-12pt]
\hspace*{23pt}онтологий и заполнение понятийных лакун&3&119--128\\
\Avtors{Зацман~И.\,М.} см.\ Гончаров~А.\,А.&&\\
\Avtors{Зацман~И.\,М.} см.\ Нуриев~В.\,А.&&\\
\Avtors{Зейфман~А.\,И.} см.\ Сатин~Я.\,А.&&\\
\Avtors{Кириков~И.\,А.} см.\ Румовская~С.\,Б.&&\\
\Avtors{Кирилюк~И.\,Л., Сенько~О.\,В.} Выбор моделей оптимальной сложности методами Монте-Карло (на примере моделей производственных функций регионов Российской\linebreak
\\[-12pt]
\hspace*{23pt}Федерации)&2&111--118\\
\Avtors{Ковалёв~Д.\,Ю.} см.\ Брюхов~Д.\,О.&&\\
\Avtors{Козеренко~Е.\,Б., Михеев~М.\,Ю., Сомин~Н.\,В., Эрлих~Л.\,И., Кузнецов~К.\,И.} Аналити\-че\-ская
текс\-тология в системах интеллектуальной обработки неструктурированных\linebreak
\\[-12pt]
\hspace*{23pt}данных&1&113--120\\
\Avtors{Колбин~И.\,С.} см.\ Абгарян~К.\,К.&&\\
\end{tabular}
}

\pagebreak

\def\leftkol{АВТОРСКИЙ УКАЗАТЕЛЬ ЗА 2020 г.} % ENGLISH ABSTRACTS}

\def\rightkol{АВТОРСКИЙ УКАЗАТЕЛЬ ЗА 2020 г.} %ENGLISH ABSTRACTS}

%\thispagestyle{myheadings}
\def\leftfootline{\small{\textbf{\thepage}
\hfill ИНФОРМАТИКА И ЕЁ ПРИМЕНЕНИЯ\ \ \ том~14\ \ \ выпуск~4\ \ \ 2020}
}%
 \def\rightfootline{\small{ИНФОРМАТИКА И ЕЁ ПРИМЕНЕНИЯ\ \ \ том~14\ \ \ выпуск~4\ \ \ 2020
 \hfill \textbf{\thepage}}}


\noindent
{\tabcolsep=3pt
\begin{tabular}{p{394pt}cc}
&\textbf{Вып.} & \textbf{Стр.}\\[3pt]
\Avtors{Королев~В.\,Ю.} О распределении отношения суммы элементов выборки, превосходящих\linebreak
\\[-12pt]
\hspace*{23pt}некоторый порог, к сумме всех элементов выборки.~I&3&35--43\\
\Avtors{Королев~В.\,Ю.} О распределении отношения суммы элементов выборки, превосходящих\linebreak
\\[-12pt]
\hspace*{23pt}некоторый порог, к сумме всех элементов выборки.~II&4&33--36\\
\Avtors{Королев~В.\,Ю.} см.\ Горшенин~А.\,К&&\\
\Avtors{Королев~В.\,Ю.} см.\ Горшенин~А.\,К.&&\\
\Avtors{Королёв~В.\,И.} см.\ Будзко~В.\,И.&&\\
\Avtors{Костина~А.\,А., Мирин~А.\,Ю., Молдовян~Д.\,Н., Фахрутдинов~Р.\,Ш.} Метод задания конечных некоммутативных ассоциативных алгебр произвольной четной размерности\linebreak
\\[-12pt]
\hspace*{23pt}для построения постквантовых криптосхем&1&\hphantom{1}94--100\\
\Avtors{Кочеткова~И.\,А.} см.\ Харин~П.\,А.&&\\
\Avtors{Краснов~Ф.\,В., Диментов~А.\,В., Шварцман~М.\,Е.} Использование тематических моделей\linebreak
\\[-12pt]
\hspace*{23pt}для парного сравнения  коллекций научных статей&3&129--135\\
\Avtors{Кривенко~М.\,П.} Последовательный анализ серий данных на основе многомерных ре-\linebreak
\\[-12pt]
\hspace*{23pt}фе\-рен\-с\-ных регионов&2&86--91\\
\Avtors{Кружков~М.\,Г.} см.\ Гончаров~А.\,А.&&\\
\Avtors{Кудрявцев~А.\,А., Шестаков~О.\,В.} Метод логарифмических моментов для оценивания\linebreak
\\[-12pt]
\hspace*{23pt}параметров гамма-экспоненциального распределения&3&49--54\\
\Avtors{Кузнецов~К.\,И.} см.\ Козеренко~Е.\,Б.&&\\
\Avtors{Кузьмин~В.\,Ю.} см.\ Горшенин~А.\,К.&&\\
\Avtors{Кушниренко~А.\,Г.} см.\ Бетелин~В.\,Б.&&\\
\Avtors{Кушниренко~А.\,Г.} см.\ Бетелин~В.\,Б.&&\\
\Avtors{Леонов~А.\,Г.} см.\ Бетелин~В.\,Б.&&\\
\Avtors{Макеева~Е.\,Д.} см.\ Харин~П.\,А.&&\\
\Avtors{Малашенко~Ю.\,Е., Назарова~И.\,А.} Аппроксимация множества достижимых потоков\linebreak
\\[-12pt]
\hspace*{23pt}многопользовательской сети&3&81--85\\
\Avtors{Мартюшова~Я.\,Г.} см.\ Босов~А.\,В.&&\\
\Avtors{Матюшенко~С.\,И., Разумчик~Р.\,В.} Стационарные характеристики системы Geo$/G/1/\infty $\linebreak
\\[-12pt]
\hspace*{23pt}с неординарным входящим потоком, управляющим размером очереди&4&25--32\\
\Avtors{Мейханаджян~Л.\,А., Разумчик~Р.\,В.} Стационарные характеристики системы $M/G/2/\infty$ с одним частным случаем дисциплины инверсионного порядка обслуживания\linebreak
\\[-12pt]
\hspace*{23pt}с обобщенным  вероятностным приоритетом&2&66--71\\
\Avtors{Мельников~А.\,В.} см.\ Вохминцев~А.\,В.&&\\
\Avtors{Мельников~С.\,Ю., Самуйлов~К.\,Е.} Статистические свойства двоичных неавтономных\linebreak
\\[-12pt]
\hspace*{23pt}регистров сдвига  с внутренним суммированием&2&80--85\\
\Avtors{Милованова~Т.\,А., Разумчик~Р.\,В.} Однолинейная система массового обслуживания с инверсионным порядком обслуживания с вероятностным приоритетом, групповым\linebreak
\\[-12pt]
\hspace*{23pt}пуассоновским потоком и фоновыми заявками&3&26--34\\
\Avtors{Мирин~А.\,Ю.} см.\ Костина~А.\,А.&&\\
\Avtors{Михеев~М.\,Ю.} см.\ Козеренко~Е.\,Б.&&\\
\Avtors{Молдовян~Д.\,Н.} см.\ Костина~А.\,А.&&\\
\Avtors{Москалева~Ф.\,А., Гайдамака~Ю.\,В., Шоргин~В.\,С.} Влияние параметров изоляции на\linebreak
\\[-12pt]
\hspace*{23pt}разделение ресурсов при нарезке сети&4&\hphantom{1}9--16\\
\Avtors{Назарова~И.\,А.} см.\ Малашенко~Ю.\,Е.&&\\
\Avtors{Наумов~А.\,В.} см.\ Босов~А.\,В.&&\\
\Avtors{Наумов~В.\,А., Самуйлов~К.\,Е.} О марковских и рациональных потоках случайных со-\linebreak
\\[-12pt]
\hspace*{23pt}бытий.~I&3&13--19\\
\Avtors{Наумов~В.\,А., Самуйлов~К.\,Е.} О марковских и рациональных потоках случайных со-\linebreak
\\[-12pt]
\hspace*{23pt}бытий.~II&4&37--46\\
\Avtors{Новиков~Д.\,А.} см.\ Шнурков~П.\,В.&&\\
\Avtors{Нуриев~В.\,А., Зацман~И.\,М.} Редуцирование спектра моделей перевода в надкорпусных\linebreak
\\[-12pt]
\hspace*{23pt}базах данных&2&119--126\\
\Avtors{Пачганов~C.\,А.} см.\ Вохминцев~А.\,В.&&\\
\end{tabular}
}

\pagebreak

\def\leftkol{АВТОРСКИЙ УКАЗАТЕЛЬ ЗА 2020 г.} % ENGLISH ABSTRACTS}

\def\rightkol{АВТОРСКИЙ УКАЗАТЕЛЬ ЗА 2020 г.} %ENGLISH ABSTRACTS}

%\thispagestyle{myheadings}
\def\leftfootline{\small{\textbf{\thepage}
\hfill ИНФОРМАТИКА И ЕЁ ПРИМЕНЕНИЯ\ \ \ том~14\ \ \ выпуск~4\ \ \ 2020}
}%
 \def\rightfootline{\small{ИНФОРМАТИКА И ЕЁ ПРИМЕНЕНИЯ\ \ \ том~14\ \ \ выпуск~4\ \ \ 2020
 \hfill \textbf{\thepage}}}


\noindent
{\tabcolsep=3pt
\begin{tabular}{p{394pt}cc}
&\textbf{Вып.} & \textbf{Стр.}\\[3pt]
\Avtors{Попков~А.\,Ю.} см.\ Попков~Ю.\,С.&&\\
\Avtors{Попков~Ю.\,С., Попков~А.\,Ю., Дубнов~Ю.\,А.} Методы детерминированных и рандомизи-\linebreak
\\[-12pt]
\hspace*{23pt}рованных энтропийных проекций для редукции размерности матрицы данных&4&47--54\\
\Avtors{Попов~Г.\,А., Симаворян~С.\,Ж., Симонян~А.\,Р., Улитина~Е.\,И.} Моделирование процесса мониторинга систем информационной безопасности на основе систем массового\linebreak
\\[-12pt]
\hspace*{23pt}обслуживания&1&71--79\\
\Avtors{Попов~М.\,В., Посыпкин~М.\,А.} Аппроксимация множества решений систем нелинейных\linebreak
\\[-12pt]
\hspace*{23pt}неравенств с использованием графических ускорителей&3&20--25\\
\Avtors{Посыпкин~М.\,А.} см.\ Попов~М.\,В.&&\\
\Avtors{Потанин~М.\,С., Вайсер~К.\,О., Жолобов~В.\,А., Стрижов~В.\,В.} Оптимизация структуры\linebreak
\\[-12pt]
\hspace*{23pt}сетей глубокого обучения&4&55--62\\
\Avtors{Разумчик~Р.\,В.} см.\ Матюшенко~С.\,И.&&\\
\Avtors{Разумчик~Р.\,В.} см.\ Мейханаджян~Л.\,А.&&\\
\Avtors{Разумчик~Р.\,В.} см.\ Милованова~Т.\,А.&&\\
\Avtors{Рождественский~Ю.\,В.} см.\ Соколов~И.\,А.&&\\
\Avtors{Румовская~С.\,Б., Кириков~И.\,А.} Метод визуального представления конфликтов в гибрид-\linebreak
\\[-12pt]
\hspace*{23pt}ных интеллектуальных многоагентных системах&4&77--82\\
\Avtors{Самуйлов~К.\,Е.} см.\ Агеев~К.\,А.&&\\
\Avtors{Самуйлов~К.\,Е.} см.\ Мельников~С.\,Ю.&&\\
\Avtors{Самуйлов~К.\,Е.} см.\ Наумов~В.\,А.&&\\
\Avtors{Самуйлов~К.\,Е.} см.\ Наумов~В.\,А.&&\\
\Avtors{Сапунова~А.\,П.} см.\ Босов~А.\,В.&&\\
\Avtors{Сатин~Я.\,А., Зейфман~А.\,И., Шилова~Г.\,Н.} О подходах к построению предельных режимов\linebreak
\\[-12pt]
\hspace*{23pt}для некоторых моделей массового обслуживания&2&3--9\\
\Avtors{Севастьянов~Л.\,А., Щетинин~Е.\,Ю.} О методах повышения точности многоклассовой\linebreak
\\[-12pt]
\hspace*{23pt}классификации на несбалансированных данных&1&63--70\\
\Avtors{Семенов~А.\,Л.} см.\ Бетелин~В.\,Б.&&\\
\Avtors{Сенько~О.\,В.} см.\ Кирилюк~И.\,Л.&&\\
\Avtors{Серебрянский~С.\,М., Тырсин~А.\,Н.} Повышение точности решения обратных задач за\linebreak
\\[-12pt]
\hspace*{23pt}счет уточнения граничных условий&1&56--62\\
\Avtors{Симаворян~С.\,Ж.} см.\ Попов~Г.\,А.&&\\
\Avtors{Симонян~А.\,Р.} см.\ Попов~Г.\,А.&&\\
\Avtors{Смирнов~Д.\,В.} см.\ Грушо~А.\,А.&&\\
\Avtors{Смирнов~Д.\,В.} см.\ Грушо~А.\,А.&&\\
\Avtors{Соколов~И.\,А., Степченков~Ю.\,А., Дьяченко~Ю.\,Г., Рождественский~Ю.\,В.} Повышение\linebreak
\\[-12pt]
\hspace*{23pt}сбоеустойчивости самосинхронных схем&4&63--68\\
\Avtors{Сомин~Н.\,В.} см.\ Козеренко~Е.\,Б.&&\\
\Avtors{Сопин~Э.\,С.} см.\ Агеев~К.\,А.&&\\
\Avtors{Сопрунов~С.\,Ф.} см.\ Бетелин~В.\,Б.&&\\
\Avtors{Соченков~И.\,В.} см.\ Будзко~В.\,И.&&\\
\Avtors{Степченков~Ю.\,А.} см.\ Соколов~И.\,А.&&\\
\Avtors{Стефанович~А.\,И.} см.\ Босов~А.\,В.&&\\
\Avtors{Стрижов~В.\,В.} см.\ Гончаров~А.\,В.&&\\
\Avtors{Стрижов~В.\,В.} см.\ Грабовой~А.\,В.&&\\
\Avtors{Стрижов~В.\,В.} см.\ Потанин~М.\,С.&&\\
\Avtors{Ступников~С.\,А.} см.\ Брюхов~Д.\,О.&&\\
\Avtors{Терехина~И.\,Ю.} см.\ Грушо~А.\,А.&&\\
\Avtors{Тимонина~Е.\,Е.} см.\  Грушо~А.\,А.&&\\
\Avtors{Тимонина~Е.\,Е.} см.\ Грушо~А.\,А.&&\\
\Avtors{Тимонина~Е.\,Е.} см.\ Грушо~А.\,А.&&\\
\Avtors{Тимонина~Е.\,Е.} см.\ Грушо~А.\,А.&&\\
\Avtors{Тимонина~Е.\,Е.} см.\ Грушо~Н.\,А.&&\\
\Avtors{Тырсин~А.\,Н.} см.\ Серебрянский~С.\,М.&&\\
\Avtors{Улитина~Е.\,И.} см.\ Попов~Г.\,А.&&\\
\end{tabular}
}

\pagebreak

\def\leftkol{АВТОРСКИЙ УКАЗАТЕЛЬ ЗА 2020 г.} % ENGLISH ABSTRACTS}

\def\rightkol{АВТОРСКИЙ УКАЗАТЕЛЬ ЗА 2020 г.} %ENGLISH ABSTRACTS}

%\thispagestyle{myheadings}
\def\leftfootline{\small{\textbf{\thepage}
\hfill ИНФОРМАТИКА И ЕЁ ПРИМЕНЕНИЯ\ \ \ том~14\ \ \ выпуск~4\ \ \ 2020}
}%
 \def\rightfootline{\small{ИНФОРМАТИКА И ЕЁ ПРИМЕНЕНИЯ\ \ \ том~14\ \ \ выпуск~4\ \ \ 2020
 \hfill \textbf{\thepage}}}


\noindent
{\tabcolsep=3pt
\begin{tabular}{p{394pt}cc}
&\textbf{Вып.} & \textbf{Стр.}\\[3pt]
\Avtors{Фахрутдинов~Р.\,Ш.} см.\ Костина~А.\,А.&&\\
\Avtors{Френкель С. Л., Захаров В. Н.} Совместная оценка предсказуемости данных и качества\linebreak
\\[-12pt]
\hspace*{23pt}предикторов&2&40--49\\
\Avtors{Харин~П.\,А., Макеева~Е.\,Д., Кочеткова~И.\,А., Ефросинин~Д.\,В., Шоргин~С.\,Я.} 
Система массового обслуживания с орбитами для анализа совместного обслуживания трафика 
с малыми задержками URLLC и~широкополосного доступа eMBB в~беспроводных\linebreak
\\[-12pt]
\hspace*{23pt}сетях пятого поколения&4&17--24\\
\Avtors{Хусаинов~А.\,А.} Производительность ограниченного конвейера&1&87--93\\
\Avtors{Шанин~И.\,А.} см.\ Брюхов~Д.\,О.&&\\
\Avtors{Шварцман~М.\,Е.} см.\ Краснов~Ф.\,В.&&\\
\Avtors{Шестаков~О.\,В.} Асимптотика оценки среднеквадратичного риска в задаче обращения\linebreak
\\[-12pt]
\hspace*{23pt}преобразования Радона по проекциям, регистрируемым на случайной сетке&2&26--32\\
\Avtors{Шестаков~О.\,В.} Асимптотическая регулярность вейвлет-методов обращения линейных однородных операторов по наблюдениям, регистрируемым в случайные моменты\linebreak
\\[-12pt]
\hspace*{23pt}времени&1&3--9\\
\Avtors{Шестаков~О.\,В.} О статистических свойствах оценки риска в задаче обращения преобра-\linebreak
\\[-12pt]
\hspace*{23pt}зования Радона при случайном объеме проекционных данных&3&44--48\\
\Avtors{Шестаков~О.\,В.} см.\ Кудрявцев~А.\,А.&&\\
\Avtors{Шилова~Г.\,Н.} см.\ Сатин~Я.\,А.&&\\
\Avtors{Шихиев~Ф.\,Ш.} см.\ Шихиев~Ш.\,Б.&&\\
\Avtors{Шихиев~Ш.\,Б., Шихиев~Ф.\,Ш.} Инкапсуляция семантических представлений в элементы\linebreak
\\[-12pt]
\hspace*{23pt}грамматики&1&121--127\\
\Avtors{Шнурков~П.\,В., Адамова~К.\,А.} Решение задачи безусловного экстремума для дробно-\linebreak
\\[-12pt]
\hspace*{23pt}линейного интегрального функционала, зависящего от параметра&2&\hphantom{1}98--103\\
\Avtors{Шнурков~П.\,В., Новиков~Д.\,А.} О концепции стохастической модели с управлением в~моменты выхода процесса на границу заданного подмножества множества\linebreak
\\[-12pt]
\hspace*{23pt}состояний&3&101--108\\
\Avtors{Шоргин~В.\,С.} см.\ Москалева~Ф.\,А.&&\\
\Avtors{Шоргин~С.\,Я.} см.\ Агеев~К.\,А.&&\\
\Avtors{Шоргин~С.\,Я.} см.\ Грушо~А.\,А.&&\\
\Avtors{Шоргин~С.\,Я.} см.\ Харин~П.\,А.&&\\
\Avtors{Щербинина~А.\,А.} см.\ Горшенин~А.\,К.&&\\
\Avtors{Щетинин~Е.\,Ю.} см.\ Севастьянов~Л.\,А.&&\\
\Avtors{Эрлих~Л.\,И.} см.\ Козеренко~Е.\,Б.&&\\
\Avtors{Ядринцев~В.\,В.} см.\ Будзко~В.\,И.&&\\
\Avtors{Яркина~Н.\,В.} см.\ Агеев~К.\,А.&&\\
\end{tabular}
}

%\thispagestyle{myheadings}
\def\leftfootline{\small{\textbf{\thepage}
\hfill ИНФОРМАТИКА И ЕЁ ПРИМЕНЕНИЯ\ \ \ том~14\ \ \ выпуск~4\ \ \ 2020}
}%
 \def\rightfootline{\small{ИНФОРМАТИКА И ЕЁ ПРИМЕНЕНИЯ\ \ \ том~14\ \ \ выпуск~4\ \ \ 2020
 \hfill \textbf{\thepage}}}

 \label{end\stat}

\newpage

\def\stat{cont-e}
{%\hrule\par
%\vskip 7pt % 7pt
\raggedleft\Large \bf%\baselineskip=3.2ex
2\,0\,2\,0\ \ A\,U\,T\,H\,O\,R\ \ I\,N\,D\,E\,X \vskip 17pt
 \hrule
 \par
\vskip 21pt plus 6pt minus 3pt }

\label{st\stat}

\def\tit{\ }

\def\aut{\ }
\def\auf{\ }

\def\leftkol{\ } %2020 AUTHOR INDEX} % ENGLISH ABSTRACTS}

\def\rightkol{\ } %2020 AUTHOR INDEX} %ENGLISH ABSTRACTS}

\titele{\tit}{\aut}{\auf}{\leftkol}{\rightkol}
\addcontentsline{toc}{subsection}{\textrm\textbf 2020 Author Index}

\def\leftfootline{\small{\textbf{\thepage}
\hfill INFORMATIKA I EE PRIMENENIYA~--- INFORMATICS AND APPLICATIONS\ \ \ 2020\
\ \ volume~14\ \ \ issue\ 4}
}%
 \def\rightfootline{\small{INFORMATIKA I EE PRIMENENIYA~--- INFORMATICS AND APPLICATIONS\ \ \ 2020\ \ \ volume~14\ \ \ issue\ 4
\hfill \textbf{\thepage}}}

\vspace*{-24pt}

\noindent
{\tabcolsep=3pt
\begin{tabular}{p{395.89pt}cc}
&\textbf{Issue} & \textbf{Page}\\[6pt]
\Avtors{Abgaryan~K.\,K. and Gavrilov~E.\,S.} Integration platform for multiscale modeling of neuromorphic\linebreak
\\[-12pt]
\hspace*{23pt}systems&2&104--110\\
\Avtors{Abgaryan~K.\,K. and Kolbin~I.\,S.} Application of multiscale approach and data sciences for\linebreak
\\[-12pt]
\hspace*{23pt}modeling thermal conductivity in layered structures&4&91--99\\
\Avtors{Adamova~K.\,A.} see Shnurkov~~P.\,V.&&\\
\Avtors{Agalarov~Ya.\,M.} Optimization of the capacity of the main storage in $G/M/1/K$ queueing system\linebreak
\\[-12pt]
\hspace*{23pt}with an additional storage device&2&72--79\\
\Avtors{Agasandyan~G.\,A.} Computational aspects of optimization on CC-VaR in a complex of markets&3&62--70\\
\Avtors{Ageev~K.\,A., Sopin~E.\,S., Yarkina~N.\,V., Samouylov~K.\,E., and Shorgin~S.\,Ya.} Analysis of the\linebreak
\\[-12pt]
\hspace*{23pt}network slicing mechanisms with guaranteed allocated resources for various traffic types&3&\hphantom{1}94--100\\
\Avtors{Bakhteev~O.\,Yu.} see Grabovoy~A.\,V.&&\\
\Avtors{Bazilevskiy~M.\,P.} Multifactor fully connected linear regression models without constraints to the\linebreak
\\[-12pt]
\hspace*{23pt}ratios of variables errors variances&2&92--97\\
\Avtors{Belenkov~V.\,G.} see Budzko~V.\,I.&&\\
\Avtors{Betelin~V.\,B., Kushnirenko~A.\,G., and Leonov~A.\,G.} Basic concepts of programming expounded\linebreak
\\[-12pt]
\hspace*{23pt}for preschoolers&3&55--61\\
\Avtors{Betelin~V.\,B., Kushnirenko~A.\,G., Semenov~A.\,L., and Soprunov~S.\,F.} About digital literacy and\linebreak
\\[-12pt]
\hspace*{23pt}environments for its development&4&100--107\\
\Avtors{Borisov~A.\,V.} Numerical schemes of Markov jump process filtering given discretized observa-\linebreak
\\[-12pt]
\hspace*{23pt}tions~II: Additive noise case&1&17--23\\
\Avtors{Borisov~A.\,V.} Numerical schemes of Markov jump process filtering given discretized observa-\linebreak
\\[-12pt]
\hspace*{23pt}tions III: Multiplicative noises case&2&10--18\\
\Avtors{Bosov~A.\,V.} Stochastic differential system output control by the quadratic criterion. V. Case of\linebreak
\\[-12pt]
\hspace*{23pt}incomplete state information&2&19--28\\
\Avtors{Bosov~A.\,V., Martyushova~Ya.\,G., Naumov~A.\,V., and Sapunova~A.\,P.} Bayesian approach to the\linebreak
\\[-12pt]
\hspace*{23pt}construction of an individual user trajectory in the system of distance learning&3&86--93\\
\Avtors{Bosov~A.\,V. and Stefanovich~A.\,I.} Stochastic differential system output control by the quadratic\linebreak
\\[-12pt]
\hspace*{23pt}criterion. IV. Alternative numerical decision&1&24--30\\
\Avtors{Briukhov~D.\,O., Stupnikov~S.\,A., Kovalev~D.\,Yu., and Shanin~I.\,A.} Neurophysiology as a subject\linebreak
\\[-12pt]
\hspace*{23pt}domain for~data intensive problem solving&1&40--47\\
\Avtors{Budzko~V.\,I., Yadrintsev~V.\,V., Sochenkov~I.\,V., Korolev~V.\,I., and Belenkov~V.\,G.} Extraction of confidentiality markers from texts under conditions of high uncertainty in systems with\linebreak
\\[-12pt]
\hspace*{23pt}data intensive usage&4&69--76\\
\Avtors{Danilishin~A.\,R. and Golembiovsky~D.\,Yu.} Estimating the fair value of options based on\linebreak
\\[-12pt]
\hspace*{23pt}ARIMA--GARCH models with errors distributed according to the Johnson's $S_u$ law&4&83--90\\
\Avtors{Danilishin~A.\,R. and Golembiovsky~D.\,Yu.} Risk-neutral dynamics for the ARIMA-GARCH\linebreak
\\[-12pt]
\hspace*{23pt}random process with errors distributed according to the Johnson's $S_u$ law&1&48--55\\
\Avtors{Diachenko~Yu.\,G.} see Sokolov~I.\,A.&&\\
\Avtors{Dimentov~A.\,V.} see Krasnov~F.\,V.&&\\
\Avtors{Donskoy~V.\,I.} Optimization models extraction from data&3&109--118\\
\Avtors{Dubnov~Y.\,A.} see Popkov~Y.\,S.&&\\
\Avtors{Dulin~S.\,K., Dulina~N.\,G., and Ermakov~P.\,V.} Information fusion of documents&1&128--135\\
\Avtors{Dulina~N.\,G.} see Dulin~S.\,K.&&\\
\Avtors{Efrosinin~D.\,V.} see Kharin~P.\,A.&&\\
\Avtors{Ehrlich~L.\,I.} see Kozerenko~E.\,B.&&\\
\Avtors{Ermakov~P.\,V.} see Dulin~S.\,K.&&\\
\end{tabular}
}
\pagebreak

\def\leftfootline{\small{\textbf{\thepage}
\hfill INFORMATIKA I EE PRIMENENIYA~--- INFORMATICS AND APPLICATIONS\ \ \ 2020\
\ \ volume~14\ \ \ issue\ 4}
}%
 \def\rightfootline{\small{INFORMATIKA I EE PRIMENENIYA~---
INFORMATICS AND APPLICATIONS\ \ \ 2020\ \ \ volume~14\ \ \ issue\ 4
\hfill \textbf{\thepage}}}

\def\leftkol{2020 AUTHOR INDEX} % ENGLISH ABSTRACTS}

\def\rightkol{2020 AUTHOR INDEX} %ENGLISH ABSTRACTS}


\noindent
{\tabcolsep=3pt
\begin{tabular}{p{395.48108pt}cc}
&\textbf{Issue} & \textbf{Page}\\[6pt]
\Avtors{Fahrutdinov~R.\,Sh.} see Kostina~A.\,A.&&\\
\Avtors{Frenkel~S.\,L. and Zakharov~V.\,N.} Joint assessment of data predictability and quality pre-\linebreak
\\[-12pt]
\hspace*{23pt}dictors&2&40--49\\
\Avtors{Gaidamaka~Yu.\,V.} see Moskaleva~F.\,A.&&\\
\Avtors{Gavrilov~E.\,S.} see Abgaryan~K.\,K.&&\\
\Avtors{Golembiovsky~D.\,Yu.} see Danilishin~A.\,R.&&\\
\Avtors{Golembiovsky~D.\,Yu.} see Danilishin~A.\,R.&&\\
\Avtors{Goncharov~A.\,V. and Strijov~V.\,V.} Alignment of ordered set Cartesian product&1&31--39\\
\Avtors{Goncharov~A.\,A., Zatsman~I.\,M., and Kruzhkov~M.\,G.} Evolution of classifications in supracorpora\linebreak
\\[-12pt]
\hspace*{23pt}databases&4&108--116\\
\Avtors{Gorshenin~A.\,K. and Korolev~V.\,Yu.} Approximation of particle size distributions of lunar regolith\linebreak
\\[-12pt]
\hspace*{23pt}based on the resampling&2&50--57\\
\Avtors{Gorshenin~A.\,K., Korolev~V.\,Yu., and Shcherbinina~A.\,A.} Statistical estimation of distributions\linebreak
\\[-12pt]
\hspace*{23pt}of random coefficients in the Langevin stochastic differential equation&3&\hphantom{1}3--12\\
\Avtors{Gorshenin~A.\,K. and Kuzmin~V.\,Yu.} Analysis of configurations of LSTM networks for medium-\linebreak
\\[-12pt]
\hspace*{23pt}term vector forecasting&1&10--16\\
\Avtors{Grabovoy~A.\,V., Bakhteev~O.\,Yu., and Strijov~V.\,V.} Ordering the set of neural network parameters&2&58--65\\
\Avtors{Grusho~A.\,A., Timonina~E.\,E., Grusho~N.\,A., and Teryokhina~I.\,Yu.} Identifying anomalies using\linebreak
\\[-12pt]
\hspace*{23pt}metadata&3&76--80\\
\Avtors{Grusho~A.\,A., Zabezhailo~M.\,I., Smirnov~D.\,V., and Timonina~E.\,E.} On probabilistic estimates of\linebreak
\\[-12pt]
\hspace*{23pt}the validity of empirical conclusions&4&3--8\\
\Avtors{Grusho~A.\,A., Zabezhailo~M.\,I., and Timonina~E.\,E.} On causal representativeness of training\linebreak
\\[-12pt]
\hspace*{23pt}samples of precedents in diagnostic type tasks&1&80--86\\
\Avtors{Grusho~A.\,A.} see Grusho~N.\,A.&&\\
\Avtors{Grusho~N.\,A., Grusho~A.\,A., Zabezhailo~M.\,I., and Timonina~E.\,E.} Methods of finding the causes\linebreak
\\[-12pt]
\hspace*{23pt}of information technology failures by means of metadata&2&33--39\\
\Avtors{Grusho~N.\,A., Zabezhailo~M.\,I., Smirnov~D.\,V., Timonina~E.\,E., and Shorgin~S.\,Ya.} Mathematical\linebreak
\\[-12pt]
\hspace*{23pt}statistics in the task of identifying hostile insiders&3&71--75\\
\Avtors{Grusho~N.\,A.} see Grusho~A.\,A.&&\\
\Avtors{Kharin~P.\,A., Makeeva~E.\,D., Kochetkova~I.\,A., Efrosinin~D.\,V., and Shorgin~S.\,Ya.} Retrial\linebreak
\\[-12pt]
\hspace*{23pt}queuing model for analyzing joint URLLC and eMBB transmission in 5G networks&4&17--24\\
\Avtors{Khusainov~A.\,A.} Performance of the bounded pipeline&1&87--93\\
\Avtors{Kirikov~I.\,A.} see Rumovskaya~S.\,B.&&\\
\Avtors{Kirilyuk~I.\,L. and Sen'ko~O.\,V.} Selection of optimal complexity models by methods of nonparametric statistics (on the example of production function model of regions of the Russian\linebreak
\\[-12pt]
\hspace*{23pt}Federation)&2&111--118\\
\Avtors{Kochetkova~I.\,A.} see Kharin~P.\,A.&&\\
\Avtors{Kolbin~I.\,S.} see Abgaryan~K.\,K.&&\\
\Avtors{Korolev~V.\,I.} see Budzko~V.\,I.&&\\
\Avtors{Korolev~V.\,Yu.} On the distribution of the ratio of the sum of sample elements exceeding\linebreak
\\[-12pt]
\hspace*{23pt}a threshold to the total sum of sample elements.~I&3&35--43\\
\Avtors{Korolev~V.\,Yu.} On the distribution of the ratio of the sum of sample elements exceeding\linebreak
\\[-12pt]
\hspace*{23pt}a threshold to the total sum of sample elements.~II&4&33--36\\
\Avtors{Korolev~V.\,Yu.} see Gorshenin~A.\,K.&&\\
\Avtors{Korolev~V.\,Yu.} see Gorshenin~A.\,K.&&\\
\Avtors{Kostina~A.\,A., Mirin~A.\,Yu., Moldovyan~D.\,N., and Fahrutdinov~R.\,Sh.} Method for defining finite noncommutative associative algebras of arbitrary even dimension for development of the\linebreak
\\[-12pt]
\hspace*{23pt}postquantum cryptoschemes&1&\hphantom{1}94--100\\
\Avtors{Kovalev~D.\,Yu.} see Briukhov~D.\,O.&&\\
\Avtors{Kozerenko~E.\,B., Mikheev~M.\,Y., Somin~N.\,V., Ehrlich~L.\,I., and Kuznetsov~K.\,I.} Analytical\linebreak
\\[-12pt]
\hspace*{23pt}textology in intelligent processing systems for unstructured data&1&113--120\\
\Avtors{Krasnov~F.\,V., Dimentov~A.\,V., and Shvartsman~M.\,E.} Using topic models for pairwise comparison\linebreak
\\[-12pt]
\hspace*{23pt}of collections of scientific papers&3&129--135\\
\end{tabular}
}
\pagebreak

\def\leftfootline{\small{\textbf{\thepage}
\hfill INFORMATIKA I EE PRIMENENIYA~--- INFORMATICS AND APPLICATIONS\ \ \ 2020\
\ \ volume~14\ \ \ issue\ 4}
}%
 \def\rightfootline{\small{INFORMATIKA I EE PRIMENENIYA~---
INFORMATICS AND APPLICATIONS\ \ \ 2020\ \ \ volume~14\ \ \ issue\ 4
\hfill \textbf{\thepage}}}

\def\leftkol{2020 AUTHOR INDEX} % ENGLISH ABSTRACTS}

\def\rightkol{2020 AUTHOR INDEX} %ENGLISH ABSTRACTS}


\noindent
{\tabcolsep=3pt
\begin{tabular}{p{395.48108pt}cc}
&\textbf{Issue} & \textbf{Page}\\[6pt]
\Avtors{Krivenko~M.\,P.} Sequential analysis of serial measurements based on multivariate reference\linebreak
\\[-12pt]
\hspace*{23pt}regions&2&86--91\\
\Avtors{Kruzhkov~M.\,G.} see Goncharov~A.\,A.&&\\
\Avtors{Kudryavtsev~A.\,A. and Shestakov~O.\,V.} Method of logarithmic moments for estimating the\linebreak
\\[-12pt]
\hspace*{23pt}gamma-exponential distribution parameters&3&49--54\\
\Avtors{Kushnirenko~A.\,G.} see Betelin~V.\,B.&&\\
\Avtors{Kushnirenko~A.\,G.} see Betelin~V.\,B.&&\\
\Avtors{Kuzmin~V.\,Yu.} see Gorshenin~A.\,K.&&\\
\Avtors{Kuznetsov~K.\,I.} see Kozerenko~E.\,B.&&\\
\Avtors{Leonov~A.\,G.} see Betelin~V.\,B.&&\\
\Avtors{Makeeva~E.\,D.} see Kharin~P.\,A.&&\\
\Avtors{Malashenko~Yu.\,E. and Nazarova~I.\,A.} Approximation of the multiuser network feasible\linebreak
\\[-12pt]
\hspace*{23pt}flows set&3&81--85\\
\Avtors{Martyushova~Ya.\,G.} see Bosov~A.\,V.&&\\
\Avtors{Matyushenko~S.\,I. and Razumchik~R.\,V.} Stationary characteristics of discrete-time Geo$/G/1/\infty$\linebreak
\\[-12pt]
\hspace*{23pt}queue with batch arrivals and one queue skipping policy&4&25--32\\
\Avtors{Melnikov~A.\,V.} see Vokhmintcev~A.\,V.&&\\
\Avtors{Melnikov~S.\,Yu. and Samouylov~K.\,E.} Statistical properties of binary nonautonomous shift\linebreak
\\[-12pt]
\hspace*{23pt}registers with internal xor&2&80--85\\
\Avtors{Meykhanadzhyan~L.\,A. and Razumchik~R.\,V.} Stationary characteristics of $M/G/2/\infty$ queue\linebreak
\\[-12pt]
\hspace*{23pt}with identical servers, LIFO service, and resampling policy&2&66--71\\
\Avtors{Mikheev~M.\,Y.} see Kozerenko~E.\,B.&&\\
\Avtors{Milovanova~T.\,A. and Razumchik~R.\,V.} A single-server queueing system with LIFO service,\linebreak
\\[-12pt]
\hspace*{23pt}probabilistic priority, batch Poisson arrivals, and background customers&3&26--34\\
\Avtors{Mirin~A.\,Yu.} see Kostina~A.\,A.&&\\
\Avtors{Moldovyan~D.\,N.} see Kostina~A.\,A.&&\\
\Avtors{Moskaleva~F.\,A., Gaidamaka~Yu.\,V., and Shorgin~V.\,S.} Impact of the isolation parameters on\linebreak
\\[-12pt]
\hspace*{23pt}resource allocation in the network slicing model&4&\hphantom{1}9--16\\
\Avtors{Naumov~A.\,V.} see Bosov~A.\,V.&&\\
\Avtors{Naumov~V.\,A. and Samouylov~К.\,Е.} On Markovian and rational arrival processes.~I&3&13--19\\
\Avtors{Naumov~V.\,A. and Samouylov~K.\,E.} On Markovian and rational arrival processes.~II&4&37--46\\
\Avtors{Nazarova~I.\,A.} see Malashenko~Yu.\,E.&&\\
\Avtors{Novikov~D.\,A.} see Shnurkov~P.\,V.&&\\
\Avtors{Nuriev~V.\,A. and Zatsman~I.\,M.} Reducing the spectrum of translation models in supracorpora\linebreak
\\[-12pt]
\hspace*{23pt}databases&2&119--126\\
\Avtors{Pachganov~S.\,A.} see Vokhmintcev~A.\,V.&&\\
\Avtors{Popkov~A.\,Y.} see Popkov~Y.\,S.&&\\
\Avtors{Popkov~Y.\,S., Popkov~A.\,Y., and Dubnov~Y.\,A.} Deterministic and randomized methods of entropy\linebreak
\\[-12pt]
\hspace*{23pt}projection for dimensionality reduction problems&4&47--54\\
\Avtors{Popov~G.\,A., Simavoryan~S.\,Zh., Simonyan~A.\,R., and Ulitina~E.\,I.} Modeling of monitoring of\linebreak
\\[-12pt]
\hspace*{23pt}information security process on the basis of queuing systems&1&71--79\\
\Avtors{Popov~M.\,V. and Posypkin~M.\,A.} Approximation of the set of solutions of systems of nonlinear\linebreak
\\[-12pt]
\hspace*{23pt}inequalities using graphic accelerators&3&20--25\\
\Avtors{Posypkin~M.\,A.} see Popov~M.\,V.&&\\
\Avtors{Potanin~M.\,S., Vayser~K.\,O., Zholobov~V.\,A., and Strijov~V.\,V.} Deep learning neural network\linebreak
\\[-12pt]
\hspace*{23pt}structure optimization&4&55--62\\
\Avtors{Razumchik~R.\,V.} see Matyushenko~S.\,I.&&\\
\Avtors{Razumchik~R.\,V.} see Meykhanadzhyan~L.\,A.&&\\
\Avtors{Razumchik~R.\,V.} see Milovanova~T.\,A.&&\\
\Avtors{Rogdestvenski~Yu.\,V.} see Sokolov~I.\,A.&&\\
\Avtors{Rumovskaya~S.\,B. and Kirikov~I.\,A.} Conflict visual representation method in hybrid intelligent\linebreak
\\[-12pt]
\hspace*{23pt}multiagent systems&4&77--82\\
\Avtors{Samouylov~K.\,E.} see Ageev~K.\,A.&&\\
\end{tabular}
}
\pagebreak

\def\leftfootline{\small{\textbf{\thepage}
\hfill INFORMATIKA I EE PRIMENENIYA~--- INFORMATICS AND APPLICATIONS\ \ \ 2020\
\ \ volume~14\ \ \ issue\ 4}
}%
 \def\rightfootline{\small{INFORMATIKA I EE PRIMENENIYA~---
INFORMATICS AND APPLICATIONS\ \ \ 2020\ \ \ volume~14\ \ \ issue\ 4
\hfill \textbf{\thepage}}}

\def\leftkol{2020 AUTHOR INDEX} % ENGLISH ABSTRACTS}

\def\rightkol{2020 AUTHOR INDEX} %ENGLISH ABSTRACTS}


\noindent
{\tabcolsep=3pt
\begin{tabular}{p{395.48108pt}cc}
&\textbf{Issue} & \textbf{Page}\\[6pt]
\Avtors{Samouylov~K.\,E.} see Melnikov~S.\,Yu.&&\\
\Avtors{Samouylov~K.\,E.} see Naumov~V.\,A.&&\\
\Avtors{Samouylov~K.\,Е.} see Naumov~V.\,A.&&\\
\Avtors{Sapunova~A.\,P.} see Bosov~A.\,V.&&\\
\Avtors{Satin~Ya.\,A., Zeifman~A.\,I., and Shilova~G.\,N.} On approaches to constructing limiting regimes\linebreak
\\[-12pt]
\hspace*{23pt}for some queuing models&2&3--9\\
\Avtors{Semenov~A.\,L.} see Betelin~V.\,B.&&\\
\Avtors{Sen'ko~O.\,V.} see Kirilyuk~I.\,L.&&\\
\Avtors{Serebryanskii~S.\,M. and Tyrsin~A.\,N.} Improvement of the accuracy of solution of tasks for the\linebreak
\\[-12pt]
\hspace*{23pt}account of the construction of boundary conditions&1&56--62\\
\Avtors{Sevastianov~L.\,A. and Shchetinin~E.\,Yu.} On methods for improving the accuracy of multiclass\linebreak
\\[-12pt]
\hspace*{23pt}classification on imbalanced data&1&63--70\\
\Avtors{Shanin~I.\,A.} see Briukhov~D.\,O.&&\\
\Avtors{Shcherbinina~A.\,A.} see Gorshenin~A.\,K.&&\\
\Avtors{Shchetinin~E.\,Yu.} see Sevastianov~L.\,A.&&\\
\Avtors{Shestakov~O.\,V.} Asymptotic regularity of the wavelet methods of inverting linear homogeneous\linebreak
\\[-12pt]
\hspace*{23pt}operators from observations recorded at random times&1&3--9\\
\Avtors{Shestakov~O.\,V.} Asymptotics of the mean-square risk estimate in the problem of inverting the\linebreak
\\[-12pt]
\hspace*{23pt}Radon transform from projections registered on a random grid&2&29--32\\
\Avtors{Shestakov~O.\,V.} On the statistical properties of risk estimate in the problem of inverting the\linebreak
\\[-12pt]
\hspace*{23pt}Radon transform with a random volume of projection data&3&44--48\\
\Avtors{Shestakov~O.\,V.} see Kudryavtsev~A.\,A.&&\\
\Avtors{Shihiev~F.\,Sh.} see Shihiev~Sh.\,B.&&\\
\Avtors{Shihiev~Sh.\,B. and Shihiev~F.\,Sh.} Incapsulation of semantic representations into elements of\linebreak
\\[-12pt]
\hspace*{23pt}a grammar&1&121--127\\
\Avtors{Shilova~G.\,N.} see Satin~Ya.\,A.&&\\
\Avtors{Shnurkov~~P.\,V. and Adamova~K.\,A.} Solution of the unconditional extremal problem for a~linear-\linebreak
\\[-12pt]
\hspace*{23pt}fractional integral functional dependent on the parameter&2&\hphantom{1}98--103\\
\Avtors{Shnurkov~P.\,V. and Novikov~D.\,A.} On the concept of a stochastic model with control at the\linebreak
\\[-12pt]
\hspace*{23pt}moments of the process at the border of a presented subset of multiple states&3&101--108\\
\Avtors{Shorgin~S.\,Ya.} see Ageev~K.\,A.&&\\
\Avtors{Shorgin~S.\,Ya.} see Grusho~N.\,A.&&\\
\Avtors{Shorgin~S.\,Ya.} see Kharin~P.\,A.&&\\
\Avtors{Shorgin~V.\,S.} see Moskaleva~F.\,A.&&\\
\Avtors{Shvartsman~M.\,E.} see Krasnov~F.\,V.&&\\
\Avtors{Simavoryan~S.\,Zh.} see Popov~G.\,A.&&\\
\Avtors{Simonyan~A.\,R.} see Popov~G.\,A.&&\\
\Avtors{Smirnov~D.\,V.} see Grusho~A.\,A.&&\\
\Avtors{Smirnov~D.\,V.} see Grusho~N.\,A.&&\\
\Avtors{Sochenkov~I.\,V.} see Budzko~V.\,I.&&\\
\Avtors{Sokolov~I.\,A., Stepchenkov~Yu.\,A., Diachenko~Yu.\,G., and Rogdestvenski~Yu.\,V.} Improvement of\linebreak
\\[-12pt]
\hspace*{23pt}self-timed circuit soft error tolerance&4&63--68\\
\Avtors{Somin~N.\,V.} see Kozerenko~E.\,B.&&\\
\Avtors{Sopin~E.\,S.} see Ageev~K.\,A.&&\\
\Avtors{Soprunov~S.\,F.} see Betelin~V.\,B.&&\\
\Avtors{Stefanovich~A.\,I.} see Bosov~A.\,V.&&\\
\Avtors{Stepchenkov~Yu.\,A.} see Sokolov~I.\,A.&&\\
\Avtors{Strijov~V.\,V.} see Goncharov~A.\,V.&&\\
\Avtors{Strijov~V.\,V.} see Grabovoy~A.\,V.&&\\
\Avtors{Strijov~V.\,V.} see Potanin~M.\,S.&&\\
\Avtors{Stupnikov~S.\,A.} see Briukhov~D.\,O.&&\\
\Avtors{Teryokhina~I.\,Yu.} see Grusho~A.\,A.&&\\
\Avtors{Timonina~E.\,E.} see Grusho~A.\,A.&&\\
\end{tabular}
}
\pagebreak

\def\leftfootline{\small{\textbf{\thepage}
\hfill INFORMATIKA I EE PRIMENENIYA~--- INFORMATICS AND APPLICATIONS\ \ \ 2020\
\ \ volume~14\ \ \ issue\ 4}
}%
 \def\rightfootline{\small{INFORMATIKA I EE PRIMENENIYA~---
INFORMATICS AND APPLICATIONS\ \ \ 2020\ \ \ volume~14\ \ \ issue\ 4
\hfill \textbf{\thepage}}}

\def\leftkol{2020 AUTHOR INDEX} % ENGLISH ABSTRACTS}

\def\rightkol{2020 AUTHOR INDEX} %ENGLISH ABSTRACTS}


\noindent
{\tabcolsep=3pt
\begin{tabular}{p{395.48108pt}cc}
&\textbf{Issue} & \textbf{Page}\\[6pt]
\Avtors{Timonina~E.\,E.} see Grusho~A.\,A.&&\\
\Avtors{Timonina~E.\,E.} see Grusho~A.\,A.&&\\
\Avtors{Timonina~E.\,E.} see Grusho~N.\,A.&&\\
\Avtors{Timonina~E.\,E.} see Grusho~N.\,A.&&\\
\Avtors{Tyrsin~A.\,N.} see Serebryanskii~S.\,M.&&\\
\Avtors{Ulitina~E.\,I.} see Popov~G.\,A.&&\\
\Avtors{Vayser~K.\,O.} see Potanin~M.\,S.&&\\
\Avtors{Vokhmintcev~A.\,V., Melnikov~A.\,V., and Pachganov~S.\,A.} Simultaneous localization and mapping method in  three-dimensional space based on the combined solution of the  point--point\linebreak
\\[-12pt]
\hspace*{23pt}variation problem ICP for an affine transformation&1&101--112\\
\Avtors{Yadrintsev~V.\,V.} see Budzko~V.\,I.&&\\
\Avtors{Yarkina~N.\,V.} see Ageev~K.\,A.&&\\
\Avtors{Zabezhailo~M.\,I.} see Grusho~A.\,A.&&\\
\Avtors{Zabezhailo~M.\,I.} see Grusho~A.\,A.&&\\
\Avtors{Zabezhailo~M.\,I.} see Grusho~N.\,A.&&\\
\Avtors{Zabezhailo~M.\,I.} see Grusho~N.\,A.&&\\
\Avtors{Zakharov~V.\,N.} see Frenkel~S.\,L.&&\\
\Avtors{Zatsman~I.\,M.} Problem-oriented verifying the completeness  of~temporal ontologies and\linebreak
\\[-12pt]
\hspace*{23pt}filling~conceptual lacunas&3&119--128\\
\Avtors{Zatsman~I.\,M.} see Goncharov~A.\,A.&&\\
\Avtors{Zatsman~I.\,M.} see Nuriev~V.\,A.&&\\
\Avtors{Zeifman~A.\,I.} see Satin~Ya.\,A.&&\\
\Avtors{Zholobov~V.\,A.} see Potanin~M.\,S.&&\\
\end{tabular}
}

%\thispagestyle{myheadings}
\def\leftfootline{\small{\textbf{\thepage}
\hfill INFORMATIKA I EE PRIMENENIYA~--- INFORMATICS AND APPLICATIONS\ \ \ 2020\
\ \ volume~14\ \ \ issue\ 4}
}%
 \def\rightfootline{\small{INFORMATIKA I EE PRIMENENIYA~---
INFORMATICS AND APPLICATIONS\ \ \ 2020\ \ \ volume~14\ \ \ issue\ 4
\hfill \textbf{\thepage}}}

 \label{end\stat}

\newpage


%\linebreak
%\\[-12pt]
%\hspace*{23pt}
%%Информатика и её применения
%Том 15 Выпуск 1-4 Год 2021

\def\stat{cont}
{%\hrule\par
%\vskip 7pt % 7pt
\raggedleft\Large \bf%\baselineskip=3.2ex
А\,В\,Т\,О\,Р\,С\,К\,И\,Й\ \ У\,К\,А\,З\,А\,Т\,Е\,Л\,Ь\ \ З\,А\ \ 2\,0\,2\,1 г. \vskip 17pt
 \hrule
 \par
\vskip 21pt plus 6pt minus 3pt }

\label{st\stat}

\def\tit{\ }

\def\aut{\ }
\def\auf{\ }

\def\leftkol{\ } % ENGLISH ABSTRACTS}

\def\rightkol{\ } %АВТОРСКИЙ УКАЗАТЕЛЬ ЗА 2021 г.} %ENGLISH ABSTRACTS}

\titele{\tit}{\aut}{\auf}{\leftkol}{\rightkol}
\addcontentsline{toc}{subsection}{\textrm\textbf Авторский указатель за 2021 г.}

\vspace*{-24pt}

\noindent
{\tabcolsep=3pt
\begin{tabular}{p{397pt}cc}
&\textbf{Вып.} & \textbf{Стр.}\\[6pt]
\Avtors{Абгарян~К.\,К., Гаврилов~Е.\,С.} Распределенная информационная система для расчета\linebreak
\\[-12pt]
\hspace*{23pt}структурных свойств композиционных материалов&4&50--58\\
\Avtors{Агаларов~Я.\,М.} Оптимальное пороговое управление доступом в системе $M/M/s$ с не-\linebreak
\\[-12pt]
\hspace*{23pt}однородными приборами и общим накопителем&1&57--64\\
\Avtors{Андрианова~Е.\,Г.} см.\ Сигов~А.\,С.&&\\
\Avtors{Арутюнов~Е.\,Н., Кудрявцев~А.\,А., Недоливко~Ю.\,Н.} Вероятностные характеристики\linebreak
\\[-12pt]
\hspace*{23pt}индекса баланса факторов, имеющих обобщенные гамма-распределения&1&65--71\\
\Avtors{Базилевский~М.\,П.} Метод выпрямления искаженных из-за мультиколлинеарности\linebreak
\\[-12pt]
\hspace*{23pt}коэффициентов в регрессионных моделях&2&60--65\\
\Avtors{Бахтеев~О.\,Ю.} см.\ Гребенькова~О.\,С.&&\\
\Avtors{Бахтеев~О.\,Ю.} см.\ Кузнецова~Р.\,В.&&\\
\Avtors{Борисов~А.\,В., Казанчян~Д.\,Х.} Фильтрация состояний марковских скачкообразных про-\linebreak
\\[-12pt]
\hspace*{23pt}цессов по комплексным наблюдениям I: точное решение задачи&2&12--19\\
\Avtors{Борисов~А.\,В., Казанчян~Д.\,Х.} Фильтрация состояний марковских скачкообразных про-\linebreak
\\[-12pt]
\hspace*{23pt}цессов по комплексным наблюдениям II: численный алгоритм&3&\hphantom{1}9--15\\
\Avtors{Босов~А.\,В.} О некоторых частных случаях в задаче управления выходом стохастической\linebreak
\\[-12pt]
\hspace*{23pt}дифференциальной системы по квадратичному критерию&1&11--17\\
\Avtors{Босов~А.\,В.} Управление линейным выходом марковской цепи по квадратичному кри-\linebreak
\\[-12pt]
\hspace*{23pt}терию&2&\hphantom{1}3--11\\
\Avtors{Босов~А.\,В., Жуков~Д.\,В.} Экспертная система для мониторинга и прогнозирования\linebreak
\\[-12pt]
\hspace*{23pt}процессов распределения ресурсов&3&29--40\\
\Avtors{Босов~А.\,В., Игнатов~А.\,Н., Наумов~А.\,В.} Алгоритмы приближенного решения задачи\linebreak
\\[-12pt]
\hspace*{23pt}назначения <<технологического окна>> на участках железнодорожной сети&4&\hphantom{1}3--11\\
\Avtors{Брюхов~Д.\,О., Ступников~С.\,А., Ковалёв~Д.\,Ю., Шанин~И.\,А.} Архитектура распределенного\linebreak
\\[-12pt]
\hspace*{23pt}решения задач анализа данных в области нейрофизиологии&1&78--85\\
\Avtors{Власкина~А.\,С.} см.\ Кочеткова~И.\,А.&&\\
\Avtors{Ву~Н.\,Н.} см.\ Кочеткова~И.\,А.&&\\
\Avtors{Вышинский~Л.\,Л., Флёров~Ю.\,А.} Информационная модель весового облика летательных\linebreak
\\[-12pt]
\hspace*{23pt}аппаратов&1&50--56\\
\Avtors{Вышинский~Л.\,Л., Флёров~Ю.\,А.} Теоретические основы формирования весового облика\linebreak
\\[-12pt]
\hspace*{23pt}самолета&4&\hphantom{1}93--102\\
\Avtors{Гаврилов~Е.\,С.} см.\ Абгарян~К.\,К.&&\\
\Avtors{Гончаренко~М.\,Б., Захарова~Т.\,В.} Некоторые свойства смесей нормальных распределений\linebreak
\\[-12pt]
\hspace*{23pt}и~их приложения к задачам магнитоэнцефалографии&2&44--51\\
\Avtors{Гончаров~А.\,А., Зацман~И.\,М.} Принципы структуризации статей в электронных словарях&2&89--95\\
\Avtors{Гончаров~А.\,А., Зацман~И.\,М., Кружков~М.\,Г.} Представление новых лексикографических\linebreak
\\[-12pt]
\hspace*{23pt}знаний в динамических классификационных системах&1&86--93\\
\Avtors{Гончаров~А.\,А., Зацман~И.\,М., Кружков~М.\,Г., Лощилова~Е.\,Ю.} Отражение эволюции\linebreak
\\[-12pt]
\hspace*{23pt}лексикографических знаний в~динамических классификационных системах&4&41--49\\
\Avtors{Гончаров~А.\,А., Инькова~О.\,Ю.} Извлечение знаний о средствах выражения логико-\linebreak
\\[-12pt]
\hspace*{23pt}се\-ман\-ти\-че\-ских отношений при помощи надкорпусной базы данных&2&\hphantom{1}96--103\\
\Avtors{Горшенин~А.\,К., Кузьмин~В.\,Ю.} Метод повышения точности нейросетевых прогнозов с использованием смешанных вероятностных моделей и его реализация в виде\linebreak
\\[-12pt]
\hspace*{23pt}цифрового сервиса&3&63--74\\
\Avtors{Гребенькова~О.\,С., Бахтеев~О.\,Ю., Стрижов~В.\,В.} Вариационная оптимизация модели\linebreak
\\[-12pt]
\hspace*{23pt}глубокого обучения с контролем сложности&1&42--49\\
\end{tabular}
}

\pagebreak

\def\leftkol{АВТОРСКИЙ УКАЗАТЕЛЬ ЗА 2021 г.} % ENGLISH ABSTRACTS}

\def\rightkol{АВТОРСКИЙ УКАЗАТЕЛЬ ЗА 2021 г.} %ENGLISH ABSTRACTS}

%\thispagestyle{myheadings}
\def\leftfootline{\small{\textbf{\thepage}
\hfill ИНФОРМАТИКА И ЕЁ ПРИМЕНЕНИЯ\ \ \ том~15\ \ \ выпуск~4\ \ \ 2021}
}%
 \def\rightfootline{\small{ИНФОРМАТИКА И ЕЁ ПРИМЕНЕНИЯ\ \ \ том~15\ \ \ выпуск~4\ \ \ 2021
 \hfill \textbf{\thepage}}}


\noindent
{\tabcolsep=3pt
\begin{tabular}{p{394pt}cc}
&\textbf{Вып.} & \textbf{Стр.}\\[3pt]
\Avtors{Гринченко~С.\,Н.} Антропогенная <<третья>> природа: относительно автономный статус\linebreak
\\[-12pt]
\hspace*{23pt}ее искусственных интеллектуальных субъектов&4&110--114\\
\Avtors{Гринченко~С.\,Н.} О системной иерархии искусственного интеллекта&1&111--115\\
\Avtors{Грушо~А.\,А., Грушо~Н.\,А., Забежайло~М.\,И., Смирнов~Д.\,В., Тимонина~Е.\,Е., Шоргин~С.\,Я.} Статистика и кластеры в~поисках аномальных вкраплений в~условиях больших\linebreak
\\[-12pt]
\hspace*{23pt}данных&4&79--86\\
\Avtors{Грушо~А.\,А., Грушо~Н.\,А., Забежайло~М.\,И., Тимонина~Е.\,Е.} Удаленный мониторинг\linebreak
\\[-12pt]
\hspace*{23pt}рабочих процессов&3&2--8\\
\Avtors{Грушо~А.\,А., Забежайло~М.\,И., Смирнов~Д.\,В., Тимонина~Е.\,Е.} Интеллектуальный анализ\linebreak
\\[-12pt]
\hspace*{23pt}пополняемых коллекций Big Data в режиме процессно-реального времени&2&36--40\\
\Avtors{Грушо~Н.\,А.} см.\ Грушо~А.\,А.&&\\
\Avtors{Грушо~Н.\,А.} см.\ Грушо~А.\,А.&&\\
\Avtors{Дараселия~А.\,В., Сопин~Э.\,С., Молчанов~Д.\,А., Самуйлов~К.\,Е.} Анализ стратегии разгрузки\linebreak
\\[-12pt]
\hspace*{23pt}базовых станций 5G NR с помощью технологии NR-U&3&98--111\\
\Avtors{Дорофеева~А.\,В.} см.\ Королев~В.\,Ю.&&\\
\Avtors{Дьяченко~Ю.\,Г.} см.\ Соколов~И.\,А.&&\\
\Avtors{Дюкова~Е.\,В., Масляков~Г.\,О.} О выборе частичных порядков на множествах значений\linebreak
\\[-12pt]
\hspace*{23pt}признаков в~задаче классификации&4&72--78\\
\Avtors{Егорова~А.\,Ю.} см.\ Нуриев~В.\,А.&&\\
\Avtors{Жуков~Д.\,В.} см.\ Босов~А.\,В.&&\\
\Avtors{Жуков~Д.\,О., Хватова~Т.\,Ю., Зальцман~А.\,Д.} Моделирование стохастической динамики изменения состояний узлов и~перколяционных переходов в~социальных сетях\linebreak
\\[-12pt]
\hspace*{23pt}с~учетом самоорганизации и наличия памяти&1&102--110\\
\Avtors{Забежайло~М.\,И.} см.\ Грушо~А.\,А.&&\\
\Avtors{Забежайло~М.\,И.} см.\ Грушо~А.\,А.&&\\
\Avtors{Забежайло~М.\,И.} см.\ Грушо~А.\,А.&&\\
\Avtors{Зальцман~А.\,Д.} см.\ Жуков~Д.\,О.&&\\
\Avtors{Захарова~Т.\,В.} см.\ Гончаренко~М.\,Б.&&\\
\Avtors{Зацман~И.\,М.} Концепция создания ВОЗ-центра компетенций по пандемиям и~эпиде-\linebreak
\\[-12pt]
\hspace*{23pt}миям: ключевые понятия и~их терминологический анализ&4&103--109\\
\Avtors{Зацман~И.\,М.} Проблемно-ориентированная актуализация словарных статей двуязыч-\linebreak
\\[-12pt]
\hspace*{23pt}ных словарей и медицинской терминологии: сопоставительный анализ&1&\hphantom{1}94--101\\
\Avtors{Зацман~И.\,М.} Формы представления нового знания, извлеченного из текстов&3&83--90\\
\Avtors{Зацман~И.\,М.} см.\ Гончаров~А.\,А.&&\\
\Avtors{Зацман~И.\,М.} см.\ Гончаров~А.\,А.&&\\
\Avtors{Зацман~И.\,М.} см.\ Гончаров~А.\,А.&&\\
\Avtors{Зейфман~А.\,И., Сатин~Я.\,А., Ковалёв~И.\,А.} Об одной нестационарной модели обслужи-\linebreak
\\[-12pt]
\hspace*{23pt}вания с катастрофами и тяжелыми хвостами&2&20--25\\
\Avtors{Игнатов~А.\,Н.} см.\ Босов~А.\,В.&&\\
\Avtors{Инькова~О.\,Ю., Кружков~М.\,Г.} Структурированные определения дискурсивных отно-\linebreak
\\[-12pt]
\hspace*{23pt}шений в~надкорпусной базе данных коннекторов&4&27--32\\
\Avtors{Инькова~О.\,Ю.} см.\ Гончаров~А.\,А.&&\\
\Avtors{Истратов~Л.\,А.} см.\ Сигов~А.\,С.&&\\
\Avtors{Казанчян~Д.\,Х.} см.\ Борисов~А.\,В.&&\\
\Avtors{Казанчян~Д.\,Х.} см.\ Борисов~А.\,В.&&\\
\Avtors{Каменских~А.\,Н.} см.\ Соколов~И.\,А.&&\\
\Avtors{Кириков~И.\,А., Листопад~С.\,В.} Согласование целей агентов сплоченных гибридных\linebreak
\\[-12pt]
\hspace*{23pt}интеллектуальных многоагентных систем&2&66--71\\
\Avtors{Кириков~И.\,А.} см.\ Румовская~С.\,Б.&&\\
\Avtors{Ковалёв~Д.\,Ю.}см.\ Брюхов~Д.\,О.&&\\
\Avtors{Ковалёв~И.\,А.} см.\ Зейфман~А.\,И.&&\\
\Avtors{Ковалёв~С.\,П.} Методы теории категорий в цифровом проектировании гетерогенных\linebreak
\\[-12pt]
\hspace*{23pt}киберфизических систем&1&23--29\\
\end{tabular}
}

\pagebreak

\def\leftkol{АВТОРСКИЙ УКАЗАТЕЛЬ ЗА 2021 г.} % ENGLISH ABSTRACTS}

\def\rightkol{АВТОРСКИЙ УКАЗАТЕЛЬ ЗА 2021 г.} %ENGLISH ABSTRACTS}

%\thispagestyle{myheadings}
\def\leftfootline{\small{\textbf{\thepage}
\hfill ИНФОРМАТИКА И ЕЁ ПРИМЕНЕНИЯ\ \ \ том~15\ \ \ выпуск~4\ \ \ 2021}
}%
 \def\rightfootline{\small{ИНФОРМАТИКА И ЕЁ ПРИМЕНЕНИЯ\ \ \ том~15\ \ \ выпуск~4\ \ \ 2021
 \hfill \textbf{\thepage}}}


\noindent
{\tabcolsep=3pt
\begin{tabular}{p{394pt}cc}
&\textbf{Вып.} & \textbf{Стр.}\\[3pt]
\Avtors{Коновалов~М.\,Г., Разумчик~Р.\,В.} Диспетчеризация в системе с параллельным обслужи-\linebreak
\\[-12pt]
\hspace*{23pt}ванием с помощью распределенного градиентного управления марковской цепью&3&41--50\\
\Avtors{Королев~В.\,Ю., Дорофеева~А.\,В.} О точности нормальной аппроксимации при отсутствии\linebreak
\\[-12pt]
\hspace*{23pt}нормальной сходимости&1&116--121\\
\Avtors{Кочеткова~И.\,А., Власкина~А.\,С., Ву~Н.\,Н., Шоргин~В.\,С.} Система массового обслуживания с управляемым по сигналам перераспределением приборов для анализа\linebreak
\\[-12pt]
\hspace*{23pt}нарезки ресурсов сети 5G&3&91--97\\
\Avtors{Кочеткова~И.\,А., Кущазли~А.\,И., Харин~П.\,А., Шоргин~С.\,Я.} Модель схемы приоритетного доступа 
трафика URLLC и~eMBB в~сети пятого поколения в~виде ресурсной\linebreak
\\[-12pt]
\hspace*{23pt}системы массового обслуживания&4&87--92\\
\Avtors{Кривенко~М.\,П.} Мягкие вычисления в задачах медицинской диагностики&2&52--59\\
\Avtors{Кружков~М.\,Г.} см.\ Гончаров~А.\,А.&&\\
\Avtors{Кружков~М.\,Г.} см.\ Гончаров~А.\,А.&&\\
\Avtors{Кружков~М.\,Г.} см.\ Инькова~О.\,Ю.&&\\
\Avtors{Кудрявцев~А.\,А., Шестаков~О.\,В.} Минимаксные оценки функции потерь, основанной на интегральных вероятностях ошибок при пороговой обработке вейвлет-\linebreak
\\[-12pt]
\hspace*{23pt}коэффициентов&4&12--19\\
\Avtors{Кудрявцев~А.\,А., Шестаков~О.\,В., Шоргин~С.\,Я.} Метод оценивания параметров изгиба,\linebreak
\\[-12pt]
\hspace*{23pt}формы и масштаба гамма-экспоненциального распределения&3&57--62\\
\Avtors{Кудрявцев~А.\,А.} см.\ Арутюнов~Е.\,Н.&&\\
\Avtors{Кузнецова~Р.\,В., Бахтеев~О.\,Ю., Чехович~Ю.\,В.} Методы обнаружения переводных\linebreak
\\[-12pt]
\hspace*{23pt}заимствований в больших текстовых коллекциях&1&30--41\\
\Avtors{Кузьмин~В.\,Ю.} см.\ Горшенин~А.\,К.&&\\
\Avtors{Кущазли~А.\,И.} см.\ Кочеткова~И.\,А.&&\\
\Avtors{Липатьев~А.\,А.} Неасимптотический анализ статистики Бартлетта--Нанда--Пилая для\linebreak
\\[-12pt]
\hspace*{23pt}данных большой размерности&1&72--77\\
\Avtors{Листопад~С.\,В.} см.\ Кириков~И.\,А.&&\\
\Avtors{Лощилова~Е.\,Ю.} см.\ Гончаров~А.\,А&&\\
\Avtors{Малашенко~Ю.\,Е.} Максимальные межузловые потоки при предельной загрузке много-\linebreak
\\[-12pt]
\hspace*{23pt}пользовательской сети&3&24--28\\
\Avtors{Малашенко~Ю.\,Е., Назарова~И.\,А.} Анализ распределения предельных нагрузок в~мно-\linebreak
\\[-12pt]
\hspace*{23pt}гопользовательской сети&4&20--26\\
\Avtors{Масляков~Г.\,О.} см.\ Дюкова~Е.\,В.&&\\
\Avtors{Молчанов~Д.\,А.} см.\ Дараселия~А.\,В.&&\\
\Avtors{Монахов~М.\,М.} Разложения Чебышёва--Эджворта для распределений обобщенных\linebreak
\\[-12pt]
\hspace*{23pt}статистик типа Хотеллинга, построенных по выборкам случайного размера&2&72--81\\
\Avtors{Назарова~И.\,А.} см.\ Малашенко~Ю.\,Е.&&\\
\Avtors{Наумов~А.\,В.} см.\ Босов~А.\,В.&&\\
\Avtors{Недоливко~Ю.\,Н.} см.\ Арутюнов~Е.\,Н.&&\\
\Avtors{Нуриев~В.\,А., Егорова~А.\,Ю.} Методы оценки качества машинного перевода: современное\linebreak
\\[-12pt]
\hspace*{23pt}состояние&2&104--111\\
\Avtors{Павлов~Ю.\,Л.} Связность конфигурационных графов в моделях сложных сетей&1&18--22\\
\Avtors{Разумчик~Р.\,В.} см.\ Коновалов~М.\,Г.&&\\
\Avtors{Рождественский~Ю.\,В.} см.\ Соколов~И.\,А.&&\\
\Avtors{Румовская~С.\,Б., Кириков~И.\,А.} Метод визуализации стимуляции конфликтов в гибрид-\linebreak
\\[-12pt]
\hspace*{23pt}ных интеллектуальных многоагентных системах&3&75--82\\
\Avtors{Самуйлов~К.\,Е.} см.\ Дараселия~А.\,В.&&\\
\Avtors{Сатин~Я.\,А.} см.\ Зейфман~А.\,И.&&\\
\Avtors{Севастьянов~Л.\,А.} см.\ Щетинин~Е.\,Ю.&&\\
\Avtors{Сигов~А.\,С., Андрианова~Е.\,Г., Истратов~Л.\,А.} Стохастическая динамика самоорганизу-\linebreak
\\[-12pt]
\hspace*{23pt}ющихся социальных систем с памятью (электоральные процессы)&2&112--121\\
\Avtors{Синицын~И.\,Н.} Нормальные субоптимальные фильтры для дифференциальных стоха-\linebreak
\\[-12pt]
\hspace*{23pt}стических систем, не разрешенных относительно производных&1&\hphantom{1}3--10\\
\end{tabular}
}

\pagebreak

\def\leftkol{АВТОРСКИЙ УКАЗАТЕЛЬ ЗА 2021 г.} % ENGLISH ABSTRACTS}

\def\rightkol{АВТОРСКИЙ УКАЗАТЕЛЬ ЗА 2021 г.} %ENGLISH ABSTRACTS}

%\thispagestyle{myheadings}
\def\leftfootline{\small{\textbf{\thepage}
\hfill ИНФОРМАТИКА И ЕЁ ПРИМЕНЕНИЯ\ \ \ том~15\ \ \ выпуск~4\ \ \ 2021}
}%
 \def\rightfootline{\small{ИНФОРМАТИКА И ЕЁ ПРИМЕНЕНИЯ\ \ \ том~15\ \ \ выпуск~4\ \ \ 2021
 \hfill \textbf{\thepage}}}


\noindent
{\tabcolsep=3pt
\begin{tabular}{p{394pt}cc}
&\textbf{Вып.} & \textbf{Стр.}\\[3pt]
\Avtors{Смирнов~Д.\,В.} см.\ Грушо~А.\,А.&&\\
\Avtors{Смирнов~Д.\,В.} см.\ Грушо~А.\,А.&&\\
\Avtors{Соколов~И.\,А., Степченков~Ю.\,А., Дьяченко~Ю.\,Г., Рождественский~Ю.\,В., Каменских~А.\,Н.}\linebreak
\\[-12pt]
\hspace*{23pt}Базис реализации сбоеустойчивых электронных схем&4&65--71\\
\Avtors{Сопин~Э.\,С.} см.\ Дараселия~А.\,В.&&\\
\Avtors{Степченков~Ю.\,А.} см.\ Соколов~И.\,А.&&\\
\Avtors{Стрижов~В.\,В.} см.\ Гребенькова~О.\,С.&&\\
\Avtors{Ступников~С.\,А.}см.\ Брюхов~Д.\,О.&&\\
\Avtors{Сушко~Д.\,В.} Алгоритмы сжатия данных массивов силовых кривых I: кодирование\linebreak
\\[-12pt]
\hspace*{23pt}ошибок предсказания&2&82--88\\
\Avtors{Сушко~Д.\,В.} Алгоритмы сжатия данных массивов силовых кривых II: кодирование\linebreak
\\[-12pt]
\hspace*{23pt}компонент вейвлет-преобразования&3&16--23\\
\Avtors{Тимонина~Е.\,Е.} см.\ Грушо~А.\,А.&&\\
\Avtors{Тимонина~Е.\,Е.} см.\ Грушо~А.\,А.&&\\
\Avtors{Тимонина~Е.\,Е.} см.\ Грушо~А.\,А.&&\\
\Avtors{Ушаков~В.\,Г., Ушаков~Н.\,Г.} Многомерные распределения выходящих потоков в системе\linebreak
\\[-12pt]
\hspace*{23pt}с абсолютным приоритетом&2&26--29\\
\Avtors{Ушаков~Н.\,Г.} см.\ Ушаков~В.\,Г.&&\\
\Avtors{Флёров~Ю.\,А.} см.\ Вышинский~Л.\,Л.&&\\
\Avtors{Флёров~Ю.\,А.} см.\ Вышинский~Л.\,Л.&&\\
\Avtors{Харин~П.\,А.} см.\ Кочеткова~И.\,А.&&\\
\Avtors{Хватова~Т.\,Ю.} см.\ Жуков~Д.\,О.&&\\
\Avtors{Чехович~Ю.\,В.} см.\ Кузнецова~Р.\,В.&&\\
\Avtors{Шанин~И.\,А.}см.\ Брюхов~Д.\,О.&&\\
\Avtors{Шестаков~О.\,В.} Анализ несмещенной оценки среднеквадратичного риска метода блоч-\linebreak
\\[-12pt]
\hspace*{23pt}ной пороговой обработки&2&30--35\\
\Avtors{Шестаков~О.\,В.} Пороговые функции в методах подавления шума, основанных на\linebreak
\\[-12pt]
\hspace*{23pt}вейвлет-разложении сигнала&3&51--56\\
\Avtors{Шестаков~О.\,В.} см.\ Кудрявцев~А.\,А.&&\\
\Avtors{Шестаков~О.\,В.} см.\ Кудрявцев~А.\,А.&&\\
\Avtors{Шнурков~П.\,В.} Создание стохастической динамической односекторной экономической модели с~дискретным временем и~анализ соответствующей задачи оптимального\linebreak
\\[-12pt]
\hspace*{23pt}управления&4&33--40\\
\Avtors{Шоргин~В.\,С.} см.\ Кочеткова~И.\,А.&&\\
\Avtors{Шоргин~С.\,Я.} см.\ Грушо~А.\,А.&&\\
\Avtors{Шоргин~С.\,Я.} см.\ Кочеткова~И.\,А.&&\\
\Avtors{Шоргин~С.\,Я.} см.\ Кудрявцев~А.\,А.&&\\
\Avtors{Щетинин~Е.\,Ю., Севастьянов~Л.\,А.} О методах переноса глубокого обучения в~задачах\linebreak
\\[-12pt]
\hspace*{23pt}классификации биомедицинских изображений&4&59--64\\
\end{tabular}
}

%\thispagestyle{myheadings}
\def\leftfootline{\small{\textbf{\thepage}
\hfill ИНФОРМАТИКА И ЕЁ ПРИМЕНЕНИЯ\ \ \ том~15\ \ \ выпуск~4\ \ \ 2021}
}%
 \def\rightfootline{\small{ИНФОРМАТИКА И ЕЁ ПРИМЕНЕНИЯ\ \ \ том~15\ \ \ выпуск~4\ \ \ 2021
 \hfill \textbf{\thepage}}}

 \label{end\stat}

\newpage

\def\stat{cont-e}
{%\hrule\par
%\vskip 7pt % 7pt
\raggedleft\Large \bf%\baselineskip=3.2ex
2\,0\,2\,1\ \ A\,U\,T\,H\,O\,R\ \ I\,N\,D\,E\,X \vskip 17pt
 \hrule
 \par
\vskip 21pt plus 6pt minus 3pt }

\label{st\stat}

\def\tit{\ }

\def\aut{\ }
\def\auf{\ }

\def\leftkol{\ } %2021 AUTHOR INDEX} % ENGLISH ABSTRACTS}

\def\rightkol{\ } %2021 AUTHOR INDEX} %ENGLISH ABSTRACTS}

\titele{\tit}{\aut}{\auf}{\leftkol}{\rightkol}
\addcontentsline{toc}{subsection}{\textrm\textbf 2021 Author Index}

\def\leftfootline{\small{\textbf{\thepage}
\hfill INFORMATIKA I EE PRIMENENIYA~--- INFORMATICS AND APPLICATIONS\ \ \ 2021\
\ \ volume~15\ \ \ issue\ 4}
}%
 \def\rightfootline{\small{INFORMATIKA I EE PRIMENENIYA~--- INFORMATICS AND APPLICATIONS\ \ \ 2021\ \ \ volume~15\ \ \ issue\ 4
\hfill \textbf{\thepage}}}

\vspace*{-24pt}

\noindent
{\tabcolsep=3pt
\begin{tabular}{p{395.89pt}cc}
&\textbf{Issue} & \textbf{Page}\\[6pt]
\Avtors{Abgaryan~K.\,K. and Gavrilov~E.\,S.} Distributed information system for calculating the structural\linebreak
\\[-12pt]
\hspace*{23pt}properties of composite materials&4&50--58\\
\Avtors{Agalarov~Ya.\,M.} Optimal threshold-based admission control in the $M/M/s$ system with hetero-\linebreak
\\[-12pt]
\hspace*{23pt}geneous servers and a common queue&1&57--64\\
\Avtors{Andrianova~E.\,G.} see Sigov~A.\,S.&&\\
\Avtors{Arutyunov~E.\,N., Kudryavtsev~A.\,A., and~Nedolivko~Iu.\,N.} Probabilistic characteristics of balance\linebreak
\\[-12pt]
\hspace*{23pt}index of factors with generalized gamma distribution&1&65--71\\
\Avtors{Bakhteev~O.\,Yu.} see Grebenkova~O.\,S.&&\\
\Avtors{Bakhteev~O.\,Yu.} see Kuznetsova~R.\,V.&&\\
\Avtors{Bazilevskiy~M.\,P.} Method of straightening distorted due to multicollinearity coefficients in\linebreak
\\[-12pt]
\hspace*{23pt}regression models&2&60--65\\
\Avtors{Borisov~A.\,V.\ and~Kazanchyan~D.\,Kh.} Filtering of Markov jump processes given composite\linebreak
\\[-12pt]
\hspace*{23pt}ob\-ser\-va\-tions I: Exact solution&2&12--19\\
\Avtors{Borisov~A.\,V.\ and~Kazanchyan~D.\,Kh.} Filtering of Markov jump processes given composite\linebreak
\\[-12pt]
\hspace*{23pt}observations II: Numerical algorithm&3&\hphantom{1}9--15\\
\Avtors{Bosov~A.\,V.} Linear output control of Markov chains by the quadratic criterion&2&\hphantom{1}3--11\\
\Avtors{Bosov~A.\,V.} On some special cases in the problem of stochastic differential system output control\linebreak
\\[-12pt]
\hspace*{23pt}by the quadratic criterion&1&11--17\\
\Avtors{Bosov~A.\,V., Ignatov~A.\,N., and Naumov~A.\,V.} Algorithms for an approximate solution of the\linebreak
\\[-12pt]
\hspace*{23pt}track possession problem on the railway network segment&4&\hphantom{1}3--11\\
\Avtors{Bosov~A.\,V.\ and~Zhukov~D.\,V.} Expert system for monitoring and forecasting of resource allocation\linebreak
\\[-12pt]
\hspace*{23pt}processes&3&29--40\\
\Avtors{Briukhov~D.\,O., Stupnikov~S.\,A., Kovalev~D.\,Yu., and~Shanin~I.\,A.} An architecture for distributed\linebreak
\\[-12pt]
\hspace*{23pt}data analysis problem solving in neurophysiology&1&78--85\\
\Avtors{Chekhovich~Yu.\,V.} see Kuznetsova~R.\,V.&&\\
\Avtors{Daraseliya~А.\,V., Sopin~E.\,S., Moltchanov~D.\,А., and~Samouylov~K.\,E.} Analysis of 5G NR base\linebreak
\\[-12pt]
\hspace*{23pt}stations offloading by means of NR-U technology&3&98--111\\
\Avtors{Diachenko~Yu.\,G.} see Sokolov~I.\,A.&&\\
\Avtors{Djukova~E.\,V. and Masliakov~G.\,O.} On the choice of partial orders on feature values sets in the\linebreak
\\[-12pt]
\hspace*{23pt}supervised classification problem&4&72--78\\
\Avtors{Dorofeeva~A.\,V.} see Korolev~V.\,Yu.&&\\
\Avtors{Egorova~A.\,Yu.} see Nuriev~V.\,A.&&\\
\Avtors{Flerov~Yu.\,A.} see Vyshinsky~L.\,L.&&\\
\Avtors{Flerov~Yu.\,A.} see Vyshinsky~L.\,L.&&\\
\Avtors{Gavrilov~E.\,S.} see Abgaryan~K.\,K.&&\\
\Avtors{Goncharenko~M.\,B.\ and~Zakharova~T.\,V.} Some properties of Gaussian mixtures and applications\linebreak
\\[-12pt]
\hspace*{23pt}to magnetoencephalography problems&2&44--51\\
\Avtors{Goncharov~A.\,A.\ and~Inkova~O.\,Yu.} Extracting knowledge about means of expression of\linebreak
\\[-12pt]
\hspace*{23pt}logical-semantic relations from the supracorpora database&2&\hphantom{1}96--103\\
\Avtors{Goncharov~A.\,A.\ and~Zatsman~I.\,M.} Structuring principles of electronic dictionary's entries&2&89--95\\
\Avtors{Goncharov~A.\,A., Zatsman~I.\,M., and~Kruzhkov~M.\,G.} Representation of new lexicographical\linebreak
\\[-12pt]
\hspace*{23pt}knowledge in dynamic classification systems&1&86--93\\
\Avtors{Goncharov~A.\,A., Zatsman~I.\,M., Kruzhkov~M.\,G., and Loshchilova~E.\,Yu.} Capturing evolution\linebreak
\\[-12pt]
\hspace*{23pt}of lexicographic knowledge in dynamic classification systems&4&41--49\\

\Avtors{Gorshenin~A.\,K.\ and~Kuzmin~V.\,Yu.} Method for improving accuracy of neural network forecasts\linebreak
\\[-12pt]
\hspace*{23pt}based on probability mixture models and its implementation as a digital service&3&63--74\\
\end{tabular}
}
\pagebreak

\def\leftfootline{\small{\textbf{\thepage}
\hfill INFORMATIKA I EE PRIMENENIYA~--- INFORMATICS AND APPLICATIONS\ \ \ 2021\
\ \ volume~15\ \ \ issue\ 4}
}%
 \def\rightfootline{\small{INFORMATIKA I EE PRIMENENIYA~---
INFORMATICS AND APPLICATIONS\ \ \ 2021\ \ \ volume~15\ \ \ issue\ 4
\hfill \textbf{\thepage}}}

\def\leftkol{2021 AUTHOR INDEX} % ENGLISH ABSTRACTS}

\def\rightkol{2021 AUTHOR INDEX} %ENGLISH ABSTRACTS}


\noindent
{\tabcolsep=3pt
\begin{tabular}{p{395.5pt}cc}
&\textbf{Issue} & \textbf{Page}\\[6pt]
\Avtors{Grebenkova~O.\,S., Bakhteev~O.\,Yu., and~Strijov~V.\,V.} Variational deep learning model optimi-\linebreak
\\[-12pt]
\hspace*{23pt}zation with complexity control&1&42--49\\[-.15pt]
\Avtors{Grinchenko~S.\,N.} Anthropogenic ``third'' nature: The relatively autonomous status of its artificial\linebreak
\\[-12pt]
\hspace*{23pt}intellectual subjects&4&110--114\\[-.15pt]
\Avtors{Grinchenko~S.\,N.} On the system hierarchy of artificial intelligence&1&111--115\\[-.15pt]
\Avtors{Grusho~A.\,A., Grusho~N.\,A., Zabezhailo~M.\,I., Smirnov~D.\,V., Timonina~E.\,E., and Shorgin~S.\,Ya.}\linebreak
\\[-12pt]
\hspace*{23pt}Statistics and clusters for detection of anomalous insertions in Big Data
en\-vi\-ron\-ment&4&79--86\\[-.15pt]
\Avtors{Grusho~A.\,A., Grusho~N.\,A., Zabezhailo~M.\,I., and~Timonina~E.\,E.} Remote monitoring of\linebreak
\\[-12pt]
\hspace*{23pt}workflows&3&2--8\\[-.15pt]
\Avtors{Grusho~A.\,A., Zabezhailo~M.\,I., Smirnov~D.\,V., and~Timonina~E.\,E.} Intelligent analysis of Big\linebreak
\\[-12pt]
\hspace*{23pt}Data extendible collections under the limits of process-real time&2&36--43\\[-.15pt]
\Avtors{Grusho~N.\,A.} see Grusho~A.\,A.&&\\[-.15pt]
\Avtors{Grusho~N.\,A.} see Grusho~A.\,A.&&\\[-.15pt]
\Avtors{Ignatov~A.\,N.} see Bosov~A.\,V.&&\\[-.15pt]
\Avtors{Inkova~O.\,Yu.~and Kruzhkov~M.\,G.} Structured definitions of discourse relations in the\linebreak
\\[-12pt]
\hspace*{23pt}Supracorpora Database of Connectives&4&27--32\\[-.15pt]
\Avtors{Inkova~O.\,Yu.} see Goncharov~A.\,A.&&\\[-.15pt]
\Avtors{Istratov~L.\,A.} see Sigov~A.\,S.&&\\[-.15pt]
\Avtors{Kamenskih~A.\,N.} see Sokolov~I.\,A.&&\\[-.15pt]
\Avtors{Kazanchyan~D.\,Kh.} see Borisov~A.\,V.&&\\[-.15pt]
\Avtors{Kazanchyan~D.\,Kh.} see Borisov~A.\,V.&&\\[-.15pt]
\Avtors{Kharin~P.\,A.} see Kochetkova~I.\,A.&&\\[-.15pt]
\Avtors{Khvatova~T.\,Yu.} see Zhukov~D.\,O.&&\\[-.15pt]
\Avtors{Kirikov~I.\,A.\ and~Listopad~S.\,V.} Coordination of agents' goals in cohesive hybrid intelligent\linebreak
\\[-12pt]
\hspace*{23pt}multiagent systems&2&66--71\\[-.15pt]
\Avtors{Kirikov~I.\,A.} see Rumovskaya~S.\,B.&&\\[-.15pt]
\Avtors{Kochetkova~I.\,A., Kushchazli~A.\,I., Kharin~P.\,A., and Shorgin~S.\,Ya.} Model for analyzing priority 
admission control of URLLC and eMBB communications in 5G networks as a~resource\linebreak
\\[-12pt]
\hspace*{23pt}queuing system&4&87--92\\[-.15pt]
\Avtors{Kochetkova~I.\,A., Vlaskina~A.\,S., Vu~N.\,N., and~Shorgin~V.\,S.} Queuing system with signals for\linebreak
\\[-12pt]
\hspace*{23pt}dynamic resource allocation for analyzing network slicing in 5G networks&3&91--97\\[-.15pt]
\Avtors{Konovalov~M.\,G.\ and~Razumchik~R.\,V.} Routing jobs to heterogeneous parallel queues using\linebreak
\\[-12pt]
\hspace*{23pt}distributed policy grandient algorithm&3&41--50\\[-.15pt]
\Avtors{Korolev~V.\,Yu.\ and~Dorofeeva~A.\,V.} On the accuracy of the normal approximation under the\linebreak
\\[-12pt]
\hspace*{23pt}violation of the normal convergence&1&116--121\\[-.15pt]
\Avtors{Kovalev~D.\,Yu.} see Briukhov~D.\,O.&&\\[-.15pt]
\Avtors{Kovalev~I.\,A.} see Zeifman~A.\,I.&&\\[-.15pt]
\Avtors{Kovalyov~S.\,P.} Methods of the category theory in digital design of heterogeneous cyber-physical\linebreak
\\[-12pt]
\hspace*{23pt}systems&1&23--29\\[-.15pt]
\Avtors{Krivenko~M.\,P.} Soft computing in problems of medical diagnostics&2&52--59\\[-.15pt]
\Avtors{Kruzhkov~M.\,G.} see Goncharov~A.\,A&&\\[-.15pt]
\Avtors{Kruzhkov~M.\,G.} see Goncharov~A.\,A.&&\\[-.15pt]
\Avtors{Kruzhkov~M.\,G.} see Inkova~O.\,Yu.&&\\[-.15pt]
\Avtors{Kudryavtsev~A.\,A. and Shestakov~O.\,V.} Minimax estimates of the loss function based on integral\linebreak
\\[-12pt]
\hspace*{23pt}error probabilities during threshold processing of wavelet coefficients&4&12--19\\[-.15pt]
\Avtors{Kudryavtsev~A.\,A., Shestakov~O.\,V., and~Shorgin~S.\,Ya.} A~method for estimating bent, shape and\linebreak
\\[-12pt]
\hspace*{23pt}scale parameters of the gamma-exponential distribution&3&57--62\\[-.15pt]
\Avtors{Kudryavtsev~A.\,A.} see Arutyunov~E.\,N.&&\\[-.15pt]
\Avtors{Kushchazli~A.\,I.} see Kochetkova~I.\,A.&&\\[-.15pt]
\Avtors{Kuzmin~V.\,Yu.} see Gorshenin~A.\,K.&&\\[-.15pt]
\Avtors{Kuznetsova~R.\,V., Bakhteev~O.\,Yu., and~Chekhovich~Yu.\,V.} Methods of cross-lingual text reuse\linebreak
\\[-12pt]
\hspace*{23pt}detection in large textual collections&1&30--41\\[-.15pt]
\Avtors{Lipatiev~A.\,A.} Nonasymptotic analysis of Bartlett--Nanda--Pillai statistic for high-dimensional\linebreak
\\[-12pt]
\hspace*{23pt}data&1&72--77\\
\end{tabular}
}
\pagebreak

\def\leftfootline{\small{\textbf{\thepage}
\hfill INFORMATIKA I EE PRIMENENIYA~--- INFORMATICS AND APPLICATIONS\ \ \ 2021\
\ \ volume~15\ \ \ issue\ 4}
}%
 \def\rightfootline{\small{INFORMATIKA I EE PRIMENENIYA~---
INFORMATICS AND APPLICATIONS\ \ \ 2021\ \ \ volume~15\ \ \ issue\ 4
\hfill \textbf{\thepage}}}

\def\leftkol{2021 AUTHOR INDEX} % ENGLISH ABSTRACTS}

\def\rightkol{2021 AUTHOR INDEX} %ENGLISH ABSTRACTS}


\noindent
{\tabcolsep=3pt
\begin{tabular}{p{395.5pt}cc}
&\textbf{Issue} & \textbf{Page}\\[6pt]
\Avtors{Listopad~S.\,V.} see Kirikov~I.\,A.&&\\
\Avtors{Loshchilova~E.\,Yu.} see Goncharov~A.\,A.&&\\
\Avtors{Malashenko~Yu.\,E.} Maximum internode flows at peak load of a multiuser network&3&24--28\\
\Avtors{Malashenko~Yu.\,E. and Nazarova~I.\,A.} Analysis of peak load distribution in the multiuser\linebreak
\\[-12pt]
\hspace*{23pt}network&4&20--26\\
\Avtors{Masliakov~G.\,O.} see Djukova~E.\,V.&&\\
\Avtors{Moltchanov~D.\,А.} see Daraseliya~А.\,V.&&\\
\Avtors{Monakhov~M.\,M.} Chebyshev--Edgeworth expansions for distributions of generalised Hotelling-\linebreak
\\[-12pt]
\hspace*{23pt}type statistics based on random size samples&2&72--81\\
\Avtors{Naumov~A.\,V.} see Bosov~A.\,V.&&\\
\Avtors{Nazarova~I.\,A.} see Malashenko~Yu.\,E.&&\\
\Avtors{Nedolivko~Iu.\,N.} see Arutyunov~E.\,N.&&\\
\Avtors{Nuriev~V.\,A.\ and~Egorova~A.\,Yu.} Methods of quality estimation for machine translation:\linebreak
\\[-12pt]
\hspace*{23pt}State-of-the-art&2&104--111\\
\Avtors{Pavlov~Yu.\,L.} Connectivity of configuration graphs in complex network models&1&18--22\\
\Avtors{Razumchik~R.\,V.} see Konovalov~M.\,G.&&\\
\Avtors{Rogdestvenski~Yu.\,V.} see Sokolov~I.\,A.&&\\
\Avtors{Rumovskaya~S.\,B.\ and~Kirikov~I.\,A.} Visual representation method for the conflict stimulation in\linebreak
\\[-12pt]
\hspace*{23pt}hybrid intelligent multiagent systems&3&75--82\\
\Avtors{Samouylov~K.\,E.} see Daraseliya~А.\,V.&&\\
\Avtors{Satin~Ya.\,A.} see Zeifman~A.\,I.&&\\
\Avtors{Sevastianov~L.\,A.} see Shchetinin~E.\,Yu.&&\\
\Avtors{Shanin~I.\,A.} see Briukhov~D.\,O.&&\\
\Avtors{Shchetinin~E.\,Yu. and Sevastianov~L.\,A.} On transfer learning methods in biomedical images\linebreak
\\[-12pt]
\hspace*{23pt}classification tasks&4&59--64\\
\Avtors{Shestakov~O.\,V.} Analysis of the unbiased mean-square risk estimate of the block thresholding\linebreak
\\[-12pt]
\hspace*{23pt}method&2&30--35\\
\Avtors{Shestakov~O.\,V.} Thresholding functions in the noise suppression methods based on the wavelet\linebreak
\\[-12pt]
\hspace*{23pt}expansion of the signal&3&51--56\\
\Avtors{Shestakov~O.\,V.} see Kudryavtsev~A.\,A.&&\\
\Avtors{Shestakov~O.\,V.} see Kudryavtsev~A.\,A.&&\\
\Avtors{Shnurkov~P.\,V.} Creation of a stochastic dynamic one-sector economic model with discrete time\linebreak
\\[-12pt]
\hspace*{23pt}and analysis of the corresponding optimal control problem&4&33--40\\
\Avtors{Shorgin~S.\,Ya.} see Grusho~A.\,A.&&\\
\Avtors{Shorgin~S.\,Ya.} see Kochetkova~I.\,A.&&\\
\Avtors{Shorgin~S.\,Ya.} see Kudryavtsev~A.\,A. &&\\
\Avtors{Shorgin~V.\,S.} see Kochetkova~I.\,A.&&\\
\Avtors{Sigov~A.\,S., Andrianova~E.\,G., and~Istratov~L.\,A.} Stochastic dynamics of self-organizing social\linebreak
\\[-12pt]
\hspace*{23pt}systems with memory (electoral processes)&2&112--121\\
\Avtors{Sinitsyn~I.\,N.} Normal suboptimal filtering for differential stochastic systems with unsolved\linebreak
\\[-12pt]
\hspace*{23pt}derivatives&1&\hphantom{1}3--10\\
\Avtors{Smirnov~D.\,V.} see Grusho~A.\,A.&&\\
\Avtors{Smirnov~D.\,V.} see Grusho~A.\,A.&&\\
\Avtors{Sokolov~I.\,A., Stepchenkov~Yu.\,A., Diachenko~Yu.\,G., Rogdestvenski~Yu.\,V., and Kamenskih~A.\,N.}\linebreak
\\[-12pt]
\hspace*{23pt}The electronic component base of failure resilience digital circuits&4&65--71\\
\Avtors{Sopin~E.\,S.} see Daraseliya~А.\,V.&&\\
\Avtors{Stepchenkov~Yu.\,A.} see Sokolov~I.\,A.&&\\
\Avtors{Strijov~V.\,V.} see Grebenkova~O.\,S.&&\\
\Avtors{Stupnikov~S.\,A.} see Briukhov~D.\,O.&&\\
\Avtors{Sushko~D.\,V.} Compression algorithms for force volume data I: Coding of prediction errors&2&82--88\\
\Avtors{Sushko~D.\,V.} Compression algorithms for force volume data II: Coding of wavelet transform\linebreak
\\[-12pt]
\hspace*{23pt}components&3&16--23\\
\Avtors{Timonina~E.\,E.} see Grusho~A.\,A.&&\\
\end{tabular}
}
\pagebreak

\def\leftfootline{\small{\textbf{\thepage}
\hfill INFORMATIKA I EE PRIMENENIYA~--- INFORMATICS AND APPLICATIONS\ \ \ 2021\
\ \ volume~15\ \ \ issue\ 4}
}%
 \def\rightfootline{\small{INFORMATIKA I EE PRIMENENIYA~---
INFORMATICS AND APPLICATIONS\ \ \ 2021\ \ \ volume~15\ \ \ issue\ 4
\hfill \textbf{\thepage}}}

\def\leftkol{2021 AUTHOR INDEX} % ENGLISH ABSTRACTS}

\def\rightkol{2021 AUTHOR INDEX} %ENGLISH ABSTRACTS}


\noindent
{\tabcolsep=3pt
\begin{tabular}{p{395.5pt}cc}
&\textbf{Issue} & \textbf{Page}\\[6pt]
\Avtors{Timonina~E.\,E.} see Grusho~A.\,A.&&\\
\Avtors{Timonina~E.\,E.} see Grusho~A.\,A.&&\\
\Avtors{Ushakov~N.\,G.} see Ushakov~V.\,G.&&\\
\Avtors{Ushakov~V.\,G.\ and~Ushakov~N.\,G.} The multivariate distributions of output streams in a queueing\linebreak
\\[-12pt]
\hspace*{23pt}system with preemptive repeat priority&2&26--29\\
\Avtors{Vlaskina~A.\,S.} see Kochetkova~I.\,A.&&\\
\Avtors{Vu~N.\,N.} see Kochetkova~I.\,A.&&\\
\Avtors{Vyshinsky~L.\,L.\ and~Flerov~Yu.\,A.} Information model of aircraft weight profile&1&50--56\\
\Avtors{Vyshinsky~L.\,L. and Flerov~Yu.\,A.} Theoretical foundation of formation of aircraft weight\linebreak
\\[-12pt]
\hspace*{23pt}appearance&4&\hphantom{1}93--102\\
\Avtors{Zabezhailo~M.\,I.} see Grusho~A.\,A.&&\\
\Avtors{Zabezhailo~M.\,I.} see Grusho~A.\,A.&&\\
\Avtors{Zabezhailo~M.\,I.} see Grusho~A.\,A.&&\\
\Avtors{Zakharova~T.\,V.} see Goncharenko~M.\,B.&&\\
\Avtors{Zaltcman~A.\,D.} see Zhukov~D.\,O.&&\\
\Avtors{Zatsman~I.\,M.} Forms representing new knowledge discovered in texts&3&83--90\\
\Avtors{Zatsman~I.\,M.} Problem-oriented updating of dictionary entries of bilingual dictionaries and\linebreak
\\[-12pt]
\hspace*{23pt}medical terminology: Comparative analysis&1&\hphantom{1}94--101\\
\Avtors{Zatsman~I.\,M.} The conception of creating WHO Hub for Pandemic and Epidemic Intelligence:\linebreak
\\[-12pt]
\hspace*{23pt}Keywords and their terminological analysis&4&103--109\\
\Avtors{Zatsman~I.\,M.} see Goncharov~A.\,A.&&\\
\Avtors{Zatsman~I.\,M.} see Goncharov~A.\,A.&&\\
\Avtors{Zatsman~I.\,M.} see Goncharov~A.\,A.&&\\
\Avtors{Zeifman~A.\,I., Satin~Ya.\,A., and~Kovalev~I.\,A.} On one nonstationary service model with\linebreak
\\[-12pt]
\hspace*{23pt}catastrophes and heavy tails&2&20--25\\
\Avtors{Zhukov~D.\,O., Khvatova~T.\,Yu., and~Zaltcman~A.\,D.} Modeling of the stochastic dynamics of changes in node states and percolation transitions in social networks with self-organization\linebreak
\\[-12pt]
\hspace*{23pt}and memory&1&102--110\\
\Avtors{Zhukov~D.\,V.} see Bosov~A.\,V.&&\\
\end{tabular}
}

%\thispagestyle{myheadings}
\def\leftfootline{\small{\textbf{\thepage}
\hfill INFORMATIKA I EE PRIMENENIYA~--- INFORMATICS AND APPLICATIONS\ \ \ 2021\
\ \ volume~15\ \ \ issue\ 4}
}%
 \def\rightfootline{\small{INFORMATIKA I EE PRIMENENIYA~---
INFORMATICS AND APPLICATIONS\ \ \ 2021\ \ \ volume~15\ \ \ issue\ 4
\hfill \textbf{\thepage}}}

 \label{end\stat}

\newpage


%\linebreak
%\\[-12pt]
%\hspace*{23pt}

%
   \vspace*{-46pt}

\begin{center}
\vspace*{4pt}
\mbox{%

\epsfxsize=55mm %112.705
\epsfbox{zhur-2.eps}
}
%\end{center}

\vspace*{10pt} 


%   \begin{center}
\fbox{\large\textbf{Академик Юрий Иванович Журавлёв}}\\[10pt]
\textbf{\large 14.01.1935--14.01.2022}
   \end{center}


   %\vspace*{2.5mm}

   \vspace*{5mm}

   \thispagestyle{empty}

%\

%\vspace*{-12pt}
       


В январе этого года ушел из жизни главный научный сотрудник Федерального исследовательского 
центра <<Информатика и управление>> РАН, председатель Редакционного совета журнала 
<<Информатика и~её применения>> академик Юрий Иванович Журавлёв. В~его лице мировая 
наука потеряла одного из своих ярчайших представителей~--- выдающегося ученого-исследователя 
и~талантливого ученого-организатора.

Юрий Иванович родился в Воронеже в 1935~г.\ в семье ученого и врача. Среднее образование 
получил в школе №\,6 г.~Фрунзе (ныне Бишкек) Киргизской ССР. В~1952~г.\ поступил на 
ме\-ха\-ни\-ко-ма\-те\-ма\-ти\-че\-ский факультет МГУ им.\ М.\,В.~Ломоносова. В~1957~г.\ Юрий Иванович 
защищает диплом и продолжает обучение в аспирантуре Московского университета на кафедре 
вычислительной математики (возглавляемой тогда академиком С.\,Л.~Соболевым). После 
успешной защиты кандидатской диссертации (к.ф.-м.н., 1959 г., научный руководитель~--- 
А.\,А.~Ляпунов, оппоненты~--- чл.-корр.\ А.\,А.~Марков, к.ф.-м.н.\ О.\,Б.~Лупанов) и~до 
окончательного переезда в Москву в 1969~г.\ работал в Институте математики Сибирского 
отделения АН СССР, занимая в нем последовательно должности младшего научного сотрудника, 
заведующего отделом, заведующего отделением, заместителя директора по научной работе. 
В~этот период (1954--1966~гг.)\ им был опубликован цикл работ по решению задач алгебры и 
математической логики, причем полученные результаты применялись для создания эффективных 
программ для ЭВМ, конструирования схем и сетей для обработки информации. Наиболее значимый 
результат этого периода научной работы~--- обоснование нового направления исследований, 
общей теории локальных алгоритмов. В~ней были окончательно объединены топологические 
принципы и теория алгоритмов. Эта теория и легла в основу докторской диссертации Юрия 
Ивановича (д.ф.-м.н., 1965~г.)\ по еще тогда новой научной специальности <<Математическая 
кибернетика>>. Оппонировали ему как специалисты по кибернетике~--- академик 
В.\,М.~Глушков, член-корреспондент А.\,А.~Ляпунов и О.\,Б.~Лупанов, так и про\-фес\-сор-ал\-геб\-раист А.\,Д.~Тайманов. 

В 1969~г.\ Юрий Иванович переезжает в Москву и возглавляет в Вычислительном центре АН 
СССР лабораторию проблем распознавания. Впоследствии он~--- заместитель директора по 
научной работе. Научные интересы этого периода связаны с проблемами классификации или 
распознавания образов. В~1976--1978~гг.\ Юрий Иванович публикует цикл работ по ставшему 
вскоре знаменитым алгебраическому подходу к проблеме синтеза корректных алгоритмов. Эти 
работы определили современное состояние всей проблематики распознавания и многих смежных 
областей прикладной математики и информатики. В~своих основополагающих работах Юрий 
Иванович показал, что можно в явном виде строить экстремальные по качеству алгоритмы для 
решения очень широких классов плохо формализованных задач. 
{\looseness=-1

}





Научные заслуги Юрия Ивановича получили широкое признание. В~1966~г.\ он совместно с 
О.\,Б.~Лупановым и чле\-ном-кор\-рес\-пон\-ден\-том АН СССР С.\,В.~Яблонским были удостоены 
звания лауреата Ленинской премии в~об\-ласти науки и техники. В~1984~г.\ Юрий Иванович 
был избран членом-корреспондентом АН СССР (по специальности <<Информатика>>), 
а~в~1992~г.~--- академиком РАН (по той же специальности).\linebreak\vspace*{-12pt}

\pagebreak

\

\vspace*{-46pt}

\noindent
\begin{floatingfigure}{48mm}
\begin{center}
%\vspace*{6pt}
\mbox{%

\epsfxsize=46mm %112.705
\epsfbox{zhur-3.eps}
}
\end{center}
\vspace*{6pt}
\end{floatingfigure}

 \thispagestyle{empty}

\noindent
В~1986~г.\ за цикл прикладных 
работ ему и ряду его учеников была при\-суж\-де\-на премия Совета Министров СССР. Он являлся 
членом иностранных академий наук, председателем секции <<Прикладная математика
 и~информатика>> Отделения математических наук РАН, председателем экспертного совета ВАК 
России по управ\-ле\-нию и информатике, заслуженным профессором нескольких университетов, 
председателем Российской ассоциации <<Распознавание образов и обработка изображений>>, 
членом исполкома Международной ассоциации IAPR (распознавание образов и обработка 
изображений). Был награжден 8-ю орденами и медалями СССР и России.

Юрий Иванович проводил большую научно-литературную работу, являясь, в том числе, главным 
редактором международных научных журналов и членом редколлегий ряда рецензируемых 
научных журналов. 


Параллельно с активной научной деятельностью Юрий Иванович вел и преподавательскую 
работу. С~1961 по~1969~гг.~--- в Новосибирском государственном университете на кафедре 
алгебры и математической логики, которую возглавлял в то время академик А.\,И.~Мальцев. 
С~1970~г., будучи уже профессором (1967~г.),~--- в Московском физико-техническом институте 
на кафедре академика Н.\,Н.~Моисеева. В~1997~г.\ по предложению ректора МГУ им.\ 
М.\,В.~Ломоносова академика В.\,А.~Садовничего Юрий Иванович организовал на факультете 
Вычислительной математики и кибернетики новую кафедру <<Математические методы 
прогнозирования>>, которой и руководил до конца жизни. В~2008~г.\ ему была присуждена 
премия Совета Министров РФ в области образования. С~1965~г.\ Юрий Иванович периодически 
читал курсы лекций за рубежом, в университетах США, Франции, Финляндии, Швеции, Австрии, 
Польши, Болгарии, ГДР и других стран. Эта работа в существенной степени обеспечила широкое 
международное признание советской и российской науки в области дискретной математики и~распознавания образов. 

%\begin{floatingfigure}{60mm}
\begin{figure}[b]
\begin{center}
\vspace*{-6pt}
\mbox{%

\epsfxsize=112mm %90mm %112.705
\epsfbox{zhur-1.eps}
}
\end{center}
\end{figure}
%\end{floatingfigure}

Понимая важность вопроса воспитания подрастающего поколения для развития науки в стране, 
Юрий Иванович вскоре после защиты первой диссертации включился в работу по подготовке 
научных кадров. Им создана большая научная школа: под руководством Юрия Ивановича 
защищены более 100~кандидатских диссертаций по всевозможным разделам естествознания 
(математике, информатике, медицине, технике, экономике, геологии), не один десяток докторов 
наук. Он воспитал академиков и членов-корреспондентов РАН и академий государств СНГ. 
С~большим вниманием и участием Юрий Иванович относился к развитию научных школ страны 
в~об\-ласти обработки изображений, распознавания образов и компьютерной оптики. 

Для всех коллег и учеников Юрия Ивановича он останется примером замечательного человека, 
та\-лант\-ли\-во\-го педагога и выдающегося, преданного служению науке ученого. 


%\def\stat{cont}
{%\hrule\par
%\vskip 7pt % 7pt
\raggedleft\Large \bf%\baselineskip=3.2ex
А\,В\,Т\,О\,Р\,С\,К\,И\,Й\ \ У\,К\,А\,З\,А\,Т\,Е\,Л\,Ь\ \ З\,А\ \ 2\,0\,1\,0 г. \vskip 17pt
    \hrule
    \par
\vskip 21pt plus 6pt minus 3pt }

\label{st\stat}

\def\tit{\ }

\def\aut{\ }
\def\auf{\ }

\def\leftkol{\ } % ENGLISH ABSTRACTS}

\def\rightkol{\ } %АВТОРСКИЙ УКАЗАТЕЛЬ ЗА 2010 г.} %ENGLISH ABSTRACTS}

\titele{\tit}{\aut}{\auf}{\leftkol}{\rightkol}

\vspace*{-12pt}

{\tabcolsep=3pt
\begin{tabular}{p{388pt}rr}
&\textbf{Выпуск} & \textbf{Стр.}\\[6pt]
\hangindent=23pt\noindent\textbf{Арутюнян~А.\,Р.} Моделирование влияния деформаций отпечатков пальцев на 
точность\linebreak
\vspace*{-12pt}\\
\hspace*{23pt}дактилоскопической идентификации$\dotfill$&1&51\\
\hangindent=23pt\noindent\textbf{Архипов~О.\,П., Зыкова~З.\,П.} Интеграция гетерогенной информации о цветных 
пикселях\linebreak
\vspace*{-12pt}\\
\hspace*{23pt}и их цветовосприятии$\dotfill$&4&15\\
\hangindent=23pt\noindent\textbf{Баранов~С.\,И., Френкель~С.\,Л., Захаров~В.\,Н.} Полуформальная верификация 
цифрового устройства с конвейером, основанная на использовании алгоритмических машин\linebreak
\vspace*{-12pt}\\
\hspace*{23pt}состояния$\dotfill$&4&49\\
\textbf{Бекетова~И.\,В.} см.~Каратеев~С.\,Л.&&\\
\textbf{Белоусов~В.\,В.} см.~Синицын~И.\,Н.&&\\
\hangindent=23pt\noindent\textbf{Бенинг~В.\,Е., Королев~Р.\,А.} О предельном поведении мощностей критериев в 
случае\linebreak
\vspace*{-12pt}\\
\hspace*{23pt}распределения Лапласа$\dotfill$&2&63\\
\hangindent=23pt\noindent\textbf{Бенинг~В.\,Е., Сипина~А.\,В.} Асимптотическое разложение для мощности 
критерия,\linebreak
\vspace*{-12pt}\\
\hspace*{23pt}основанного на выборочной медиане, в случае распределения Лапласа$\dotfill$&1&18\\
\textbf{Бондаренко~А.\,В.} см.~Каратеев~С.\,Л.&&\\
\hangindent=23pt\noindent\textbf{Бородина~А.\,В., Морозов~Е.\,В.} Об оценивании асимптотики вероятности 
большого\linebreak
\vspace*{-12pt}\\
\hspace*{23pt}уклонения стационарной регенеративной очереди с одним прибором$\dotfill$&3&29\\
\hangindent=23pt\noindent\textbf{Бунтман~Н.\,В., Минель~Ж.-Л., Ле~Пезан~Д., Зацман~И.\,М.} Типология и 
компьютерное\linebreak
\vspace*{-12pt}\\
\hspace*{23pt}моделирование трудностей перевода$\dotfill$&3&77\\
\textbf{Визильтер~Ю.\,В.} см.~Каратеев~С.\,Л.&&\\
\hangindent=23pt\noindent\textbf{Гавриленко~С.\,В.} Оценки скорости сходимости распределений случайных сумм с 
безгранично делимыми индексами к нормальному закону$\dotfill$&4&81\\
\hangindent=23pt\noindent\textbf{Григорьева~М.\,Е., Шевцова~И.\,Г.} Уточнение неравенства 
Каца--Берри--Эссеена$\dotfill$&2&75\\
\hangindent=23pt\noindent\textbf{Грушо~А.\,А., Грушо~Н.\,А., Тимонина~Е.\,Е.} Поиск конфликтов в политиках 
безопасности: модель случайных графов$\dotfill$&3&38\\
\textbf{Грушо~Н.\,А.} см.~Грушо~А.\,А.&&\\
\hangindent=23pt\noindent\textbf{Гудков~В.\,Ю.} Математические модели изображения отпечатка пальца на основе 
описания линий$\dotfill$&1&58\\
\textbf{Гуртов~А.\,В.} см.~Лукьяненко~А.\,С.&&\\
\textbf{Желтов~С.\,Ю.} см.~Каратеев~С.\,Л.&&\\
\hangindent=23pt\noindent\textbf{Захаров~А.\,А., Серебряков~В.\,А.} Система управления электронной библиотекой 
LibMeta$\dotfill$&4&2\\
\textbf{Захаров~В.\,Н.} см.~Баранов~С.\,И.&&\\
\textbf{Захарова~Т.\,В.} см.~Матвеева~С.\,С.&&\\
\hangindent=23pt\noindent\textbf{Зацаринный~А.\,А., Чупраков~К.\,Г.} Некоторые аспекты выбора технологии для 
постро-\linebreak
\vspace*{-12pt}\\
\hspace*{23pt}ения систем отображения информации ситуационного центра$\dotfill$&3&59\\
\textbf{Зацман~И.\,М.} см.~Бунтман~Н.\,В.&&\\
\hangindent=23pt\noindent\textbf{Зейфман~А.\,И., Коротышева~А.\,В., Сатин~Я.\,А., Шоргин~С.\,Я.} Об 
устойчивости нестаци-\linebreak
\vspace*{-12pt}\\
\hspace*{23pt}онарных систем обслуживания с катастрофами$\dotfill$&3&9\\
\textbf{Зыкова~З.\,П.} см.~Архипов~О.\,П.&&\\
\hangindent=23pt\noindent\textbf{Илюшин~Г.\,Я., Соколов~И.\,А.} Организация управляемого доступа пользователей 
к\linebreak
\vspace*{-12pt}\\
\hspace*{23pt}разнородным ведомственным информационным ресурсам$\dotfill$&1&24\\
\hangindent=23pt\noindent\textbf{Кавагучи~Ю., Ульянов~В.\,В., Фуджикоши~Я.} Приближения для статистик, 
описывающих\linebreak
\vspace*{-12pt}\\
\hspace*{23pt}геометрические свойства данных большой размерности, с оценками 
ошибок$\dotfill$&1&12\\
\hangindent=23pt\noindent\textbf{Каратеев~С.\,Л., Бекетова~И.\,В., Ососков~М.\,В., Князь~В.\,А., 
Визильтер~Ю.\,В., Бондаренко~А.\,В., Желтов~С.\,Ю.} Автоматизированный контроль 
качества цифровых\linebreak
\vspace*{-12pt}\\
\hspace*{23pt}изображений для персональных документов$\dotfill$&1&65\\
\end{tabular}
}

\pagebreak

\def\leftkol{АВТОРСКИЙ УКАЗАТЕЛЬ ЗА 2010 г.} % ENGLISH ABSTRACTS}

\def\rightkol{АВТОРСКИЙ УКАЗАТЕЛЬ ЗА 2010 г.} %ENGLISH ABSTRACTS}

{\tabcolsep=3pt
\begin{tabular}{p{388pt}rr}
&\textbf{Выпуск} & \textbf{Стр.}\\[3pt]
\hangindent=23pt\noindent\textbf{Козеренко~Е.\,Б.} Лингвистические фильтры в статистических моделях машинного\linebreak
\vspace*{-12pt}\\
\hspace*{23pt}перевода$\dotfill$&2&83\\
\hangindent=23pt\noindent\textbf{Козеренко~Е.\,Б., Кузнецов~И.\,П.} Когнитивно-лингвистические представления в 
систе-\linebreak
\vspace*{-12pt}\\
\hspace*{23pt}мах обработки текстов$\dotfill$&3&69\\
\textbf{Князь~В.\,А.} см.~Каратеев~С.\,Л.&&\\
\hangindent=23pt\noindent\textbf{Колесников~А.\,В., Солдатов~С.\,А.} Алгоритм координации для гибридной 
интеллектуальной системы решения сложной задачи оперативно-производственного\linebreak
\vspace*{-12pt}\\
\hspace*{23pt}планирования$\dotfill$&4&61\\
\hangindent=23pt\noindent\textbf{Коновалов~М.\,Г.} О планировании потоков в системах вычислительных 
ресурсов$\dotfill$&2&3\\
\textbf{Конушин~А.\,С.} см.~Конушин~В.\,С.&&\\
\hangindent=23pt\noindent\textbf{Конушин~В.\,С., Кривовязь~Г.\,Р., Конушин~А.\,С.} Алгоритм распознавания людей 
в видео-\linebreak
\vspace*{-12pt}\\
\hspace*{23pt}последовательности по одежде$\dotfill$&1&74\\
\textbf{Корепанов~Э.\, Р.} см.~Синицын~И.\,Н.&&\\
\textbf{Королев~В.\,Ю.} см.~Соколов~И.\,А.&&\\
\textbf{Королев~Р.\,А.} см.~Бенинг~В.\,Е.&&\\
\textbf{Коротышева~А.\,В.} см.~Зейфман~А.\,И.&&\\
\hangindent=23pt\noindent\textbf{Кривенко~М.\,П.} Непараметрическое оценивание элементов байесовского 
клас\-си-\linebreak
\vspace*{-12pt}\\
\hspace*{23pt}фикатора$\dotfill$&2&13\\
\textbf{Кривовязь~Г.\,Р.} см.~Конушин~В.\,С.&&\\
\textbf{Крылов~А.\,С.} см.~Павельева~Е.\,А.&&\\
\hangindent=23pt\noindent\textbf{Крылов~В.\,А.} Моделирование и классификация многоканальных дистанционных\linebreak
\vspace*{-12pt}\\
\hspace*{23pt}изображений с использованием копул$\dotfill$&4&34\\
\hangindent=23pt\noindent\textbf{Крючин~О.\,В.} Разработка параллельных эвристических алгоритмов подбора 
весовых\linebreak
\vspace*{-12pt}\\
\hspace*{23pt}коэффициентов искусственной нейтронной сети$\dotfill$&2&53\\
\hangindent=23pt\noindent\textbf{Кудрявцев~А.\,А., Шоргин~С.\,Я.} Байесовские модели массового обслуживания и 
надеж-\linebreak
\vspace*{-12pt}\\
\hspace*{23pt}ности: характеристики среднего числа заявок в системе $M\vert M \vert 1\vert 
\infty$$\dotfill$&3&16\\
\hangindent=23pt\noindent\textbf{Кузнецов~А.\,А.} Связь между временными и структурно-топологическими 
характери-\linebreak
\vspace*{-12pt}\\
\hspace*{23pt}стиками диаграмм ритма сердца здоровых людей$\dotfill$&4&39\\
\textbf{Кузнецов~И.\,П.} см.~Козеренко~Е.\,Б.&&\\
\textbf{Ле~Пезан~Д.} см.~Бунтман~Н.\,В.&&\\
\hangindent=23pt\noindent\textbf{Лукьяненко~А.\,С., Морозов~Е.\,В., Гуртов~А.\,В.} Анализ сетевого протокола с общей 
функ-\linebreak
\vspace*{-12pt}\\
\hspace*{23pt}цией расширения окна передачи сообщения при конфликтах$\dotfill$&2&46\\
\hangindent=23pt\noindent\textbf{Лямин~О.\,О.} О предельном поведении мощностей критериев в случае обобщенного\linebreak
\vspace*{-12pt}\\
\hspace*{23pt}распределения Лапласа$\dotfill$&3&47\\
\hangindent=23pt\noindent\textbf{Маркин~А.\,В., Шестаков~О.\,В.} Асимптотики оценки риска при пороговой 
обработке\linebreak
\vspace*{-12pt}\\
\hspace*{23pt}вейвлет-вейглет коэффициентов в задаче томографии$\dotfill$&2&36\\
\hangindent=23pt\noindent\textbf{Матвеева~С.\,С., Захарова~Т.\,В.} Сети массового обслуживания с наименьшей 
длиной\linebreak
\vspace*{-12pt}\\
\hspace*{23pt}очереди$\dotfill$&3&22\\
\hangindent=23pt\noindent\textbf{Матюшенко~С.\,И.} Стационарные характеристики двухканальной системы 
обслужива-\linebreak
\vspace*{-12pt}\\
\hspace*{23pt}ния с переупорядочиванием заявок и распределениями фазового типа$\dotfill$&4&68\\
\textbf{Минель~Ж.-Л.} см.~Бунтман~Н.\,В.&&\\
\textbf{Морозов~Е.\,В.} см.~Бородина~А.\,В.&&\\
\textbf{Морозов~Е.\,В.} см.~Лукьяненко~А.\,С.&&\\
\textbf{Ососков~М.\,В.} см.~Каратеев~С.\,Л.&&\\
\hangindent=23pt\noindent\textbf{Павельева~Е.\,А., Крылов~А.\,С.} Поиск и анализ ключевых точек радужной 
оболочки\linebreak
\vspace*{-12pt}\\
\hspace*{23pt}глаза методом преобразования Эрмита$\dotfill$&1&79\\
\textbf{Печинкин~А.\,В.} см.~Френкель~С.\,Л.,&&\\
\hangindent=23pt\noindent\textbf{Протасов~В.\,И.} Составление субъективного портрета с использованием 
эволюционно-\linebreak
\vspace*{-12pt}\\
\hspace*{23pt}го морфинга и квалиметрия метода$\dotfill$&1&83\\
\hangindent=23pt\noindent\textbf{Рудаков~К.\,В., Торшин~И.\,Ю.} Вопросы разрешимости задачи распознавания 
вторичной\linebreak
\vspace*{-12pt}\\
\hspace*{23pt}структуры белка$\dotfill$&2&25\\
\textbf{Сатин~Я.\,А.} см.~Зейфман~А.\,И.&&\\
\hangindent=23pt\noindent\textbf{Сейфуль-Мулюков~Р.\,Б.} Нефть как носитель информации о своем 
происхождении,\linebreak
\vspace*{-12pt}\\
\hspace*{23pt}структуре и эволюции$\dotfill$&1&41\\
\end{tabular}
}

{\tabcolsep=3pt
\begin{tabular}{p{388pt}rr}
&\textbf{Выпуск} & \textbf{Стр.}\\[6pt]
\textbf{Семендяев~Н.\,Н.} см.~Синицын~И.\,Н.&&\\
\textbf{Серебряков~В.\,А.} см.~Захаров~А.\,А.&&\\
\textbf{Синицын~В.\,И.} см.~Синицын~И.\,Н.&&\\
\hangindent=23pt\noindent\textbf{Синицын~И.\,Н., Синицын~В.\,И., Корепанов~Э.\, Р., Белоусов~В.\,В., 
Семендяев~Н.\,Н.} Оперативное построение информационных моделей движения полюса 
Земли\linebreak
\vspace*{-12pt}\\
\hspace*{23pt}методами линейных и линеаризованных фильтров$\dotfill$&1&2\\
\textbf{Сипина~А.\,В.} см.~Бенинг~В.\,Е.&&\\
\hangindent=23pt\noindent\textbf{Соколов~И.\,А.} О работах заслуженного деятеля науки Российской Федерации 
И.\,Н.~Синицына в области информационных технологий и автоматизации (к 70-летию\linebreak
\vspace*{-12pt}\\
\hspace*{23pt}со дня рождения)$\dotfill$&3&84\\
\textbf{Соколов~И.\,А.} см.~Илюшин~Г.\,Я.&&\\
\hangindent=23pt\noindent\textbf{Соколов~И.\,А., Королев~В.\,Ю.} Предисловие$\dotfill$&2&2\\
\textbf{Солдатов~С.\,А.} см.~Колесников~А.\,В.&&\\
\hangindent=23pt\noindent\textbf{Степанов~С.\,Ю.} Использование координатного метода фрагментации 
коммутаторной\linebreak
\vspace*{-12pt}\\
\hspace*{23pt}нейронной сети для сокращения трафика$\dotfill$&2&57\\
\textbf{Тимонина~Е.\,Е.} см.~Грушо~А.\,А.&&\\
\textbf{Торшин~И.\,Ю.} см.~Рудаков~К.\,В.&&\\
\textbf{Ульянов~В.\,В.} см.~Кавагучи~Ю.&&\\
\textbf{Фазекаш~И.} см.~Чупрунов~А.\,Н.&&\\
\textbf{Френкель~С.\,Л.} см.~Баранов~С.\,И.&&\\
\hangindent=23pt\noindent\textbf{Френкель~С.\,Л., Печинкин~А.\,В.} Оценка времени самовосстановления в 
цифровых\linebreak
\vspace*{-12pt}\\
\hspace*{23pt}системах после сбоев, вызываемых переходными помехами$\dotfill$&3&2\\
\textbf{Фуджикоши~Я.} см.~Кавагучи~Ю.&&\\
\hangindent=23pt\noindent\textbf{Цискаридзе~А.\,К.} Математическая модель и метод восстановления позы человека 
по\linebreak
\vspace*{-12pt}\\
\hspace*{23pt}стереопаре силуэтных изображений$\dotfill$&4&27\\
\hangindent=23pt\noindent\textbf{Чупраков~К.\,Г.} К вопросу о размещении коллективных средств отображения в 
ситуа-\linebreak
\vspace*{-12pt}\\
\hspace*{23pt}ционном зале с заданными параметрами$\dotfill$&4&89\\
\textbf{Чупраков~К.\,Г.} см.~Зацаринный~А.\,А.&&\\
\hangindent=23pt\noindent\textbf{Чупрунов~А.\,Н., Фазекаш~И.} Законы повторного логарифма для числа 
безошибочных\linebreak
\vspace*{-12pt}\\
\hspace*{23pt}блоков при помехоустойчивом кодировании$\dotfill$&3&42\\
\textbf{Шевцова~И.\,Г.} см.~Григорьева~М.\,Е.&&\\
\hangindent=23pt\noindent\textbf{Шестаков~О.\,В.} Аппроксимация распределения оценки риска пороговой 
обработки вейвлет-коэффициентов нормальным распределением при использовании 
выбо-\linebreak
\vspace*{-12pt}\\
\hspace*{23pt}рочной дисперсии$\dotfill$&4&73\\
\textbf{Шестаков~О.\,В.} см.~Маркин~А.\,В.&&\\
\textbf{Шоргин~С.\,Я.} см.~Зейфман~А.\,И.&&\\
\textbf{Шоргин~С.\,Я.} см.~Кудрявцев~А.\,А.&&\\
\end{tabular}
}

%\thispagestyle{myheadings}
\def\leftfootline{\small{\textbf{\thepage}
\hfill ИНФОРМАТИКА И ЕЁ ПРИМЕНЕНИЯ\ \ \ том~4\ \ \ выпуск~4\ \ \ 2010}
}%
 \def\rightfootline{\small{ИНФОРМАТИКА И ЕЁ ПРИМЕНЕНИЯ\ \ \ том~4\ \ \ выпуск~4\ \ \ 2010
 \hfill \textbf{\thepage}}}
 \label{end\stat}
%
%Том 10 Выпуск 1-4 Год 2016

\def\stat{cont-e}
{%\hrule\par
%\vskip 7pt % 7pt
\raggedleft\Large \bf%\baselineskip=3.2ex
2\,0\,1\,6\ \ A\,U\,T\,H\,O\,R\ \ I\,N\,D\,E\,X \vskip 17pt
 \hrule
 \par
\vskip 21pt plus 6pt minus 3pt }

\label{st\stat}

\def\tit{\ }

\def\aut{\ }
\def\auf{\ }

\def\leftkol{\ } %2016 AUTHOR INDEX} % ENGLISH ABSTRACTS}

\def\rightkol{\ } %2016 AUTHOR INDEX} %ENGLISH ABSTRACTS}

\titele{\tit}{\aut}{\auf}{\leftkol}{\rightkol}

\def\leftfootline{\small{\textbf{\thepage}
\hfill INFORMATIKA I EE PRIMENENIYA~--- INFORMATICS AND APPLICATIONS\ \ \ 2016\
\ \ volume~10\ \ \ issue\ 4}
}%
 \def\rightfootline{\small{INFORMATIKA I EE PRIMENENIYA~--- INFORMATICS AND APPLICATIONS\ \ \ 2016\ \ \ volume~10\ \ \ issue\ 4
\hfill \textbf{\thepage}}}

\vspace*{-12pt}
\vspace*{-18pt}

{\tabcolsep=2.8pt
\begin{tabular}{p{382pt}cc}
&\textbf{Issue} & \textbf{Page}\\[6pt]
\Avtors{Agalarov~M.\,Ya.} see~Agalarov~Ya.\,M.&&\\
\Avtors{Agalarov~Ya.\,M., Agalarov~M.\,Ya., and
Shorgin~V.\,S.} About the optimal threshold of queue\linebreak
\\[-12pt]
\hspace*{23pt}length in a~particular problem of profit maximization
in the $M/G/1$ queuing system&2&70--79\\
\Avtors{Alexeyevsky~D.\,A.} BioNLP ontology extraction from 
a~restricted language corpus with\linebreak
\\[-12pt]
\hspace*{23pt}context-free grammars&1&119--128\\
\Avtors{Andreev~S.\,D.} see~Gaidamaka~Yu.\,V.&&\\
\Avtors{Andreev~S.\,D.} see~Ometov~A.\,Ya.&&\\
\Avtors{Arkhipov~O.\,P., Arkhipov~P.\,O., and Sidorkin~I.\,I.} The
option to create a~local coordinate\linebreak
\\[-12pt]
\hspace*{23pt}system for synchronization of selected images&3&91--97\\
\Avtors{Arkhipov~P.\,O.} see~Arkhipov~O.\,P.&&\\
\Avtors{Belousov~V.\,V.} see~Shnurkov~P.\,V.&&\\
\Avtors{Belousov~V.\,V.} see~Shnurkov~P.\,V.&&\\
\Avtors{Bening~V.\,E.} Calculation of~the~asymptotic deficiency
of~some statistical procedures based\linebreak
\\[-12pt]
\hspace*{23pt}on~samples with~random sizes&4&34--45\\
\Avtors{Borisov~A.\,V., Bosov~A.\,V., and Miller~G.\,B.} Modeling and
monitoring of VoIP connection&2&\hphantom{1}2--13\\
\Avtors{Bosov~A.\,V.} see~Borisov~A.\,V.&&\\
\Avtors{Briukhov~D.\,O.} see~Stupnikov~S.\,A.&&\\
\Avtors{Callaos~N.\,K.\ and Seyful-Mulyukov~R.\,B.} Complexity and
its information content&1&129--139\\
\Avtors{Chertok~A.\,V., Kadaner~A.\,I., Khazeeva~G.\,T., and
Sokolov~I.\,A.} Regime switching detection\linebreak
\\[-12pt]
\hspace*{23pt}for~the~Levy driven
Ornstein--Uhlenbeck process using CUSUM methods&4&46--56\\
\Avtors{Chichagov~V.\,V.} Asymptotic expansions of mean absolute
error of uniformly minimum variance unbiased and maximum likelihood
estimators on the one-parameter exponential\linebreak
\\[-12pt]
\hspace*{23pt}family model of lattice distributions&3&66--76\\
\Avtors{Danishevsky~V.\,I.} see~Kolesnikov A.\,V.&&\\
\Avtors{Fazliev~A.\,Z.} see~Kalinichenko~L.\,A.&&\\
\Avtors{Fedoseev~A.\,A.} What is behind the concept of ``knowledge in
small packages''&3&105--110\\
\Avtors{Gaidamaka~Yu.\,V., Andreev~S.\,D., Sopin~E.\,S.,
Samouylov~K.\,E., and Shorgin~S.\,Ya.} Interference analysis
of~the~device-to-device communications model with~regard to~a~signal\linebreak
\\[-12pt]
\hspace*{23pt}propagation environment&4&\hphantom{1}2--10\\
\Avtors{Gasilov~A.\,V.} see~Yakovlev~O.\,A.&&\\
\Avtors{Goncharov~A.\,V.\ and Strijov~V.\,V.} Metric time series
classification using weighted dynamic\linebreak
\\[-12pt]
\hspace*{23pt}warping relative to centroids of classes&2&36--47\\
\Avtors{Gordov~E.\,P.} see~Kalinichenko~L.\,A.&&\\
\Avtors{Gorshenin~A.\,K.} Concept of online service for stochastic
modeling of real processes&1&72--81\\
\Avtors{Gorshenin~A.\,K.} see~Shnurkov~P.\,V.&&\\
\Avtors{Gorshenin~A.\,K.} see~Shnurkov~P.\,V.&&\\
\Avtors{Grusho~A.\,A., Grusho~N.\,A., Zabezhailo~M.\,I., and
Timonina~E.\,E.} Integration of statistical and\linebreak
\\[-12pt]
\hspace*{23pt}deterministic methods for
analysis of information security&3&2--8\\
\Avtors{Grusho~A.\,A., Zabezhailo~M.\,I., and Zatsarinny~A.\,A.} On
the advanced procedure to reduce\linebreak
\\[-12pt]
\hspace*{23pt}calculation of Galois closures&4&\hphantom{1}96--104\\
\Avtors{Grusho~N.\,A.} see~Grusho~A.\,A.&&\\
\Avtors{Havanskov~V.\,A.} see~Minin~V.\,A.&&\\
\Avtors{Inkova~O.\,Yu.} see~Zatsman~I.\,M.&&\\
\Avtors{Isachenko~R.\,V.\ and Strijov~V.\,V.} Metric learning in
multiclass time series classification\linebreak
\\[-12pt]
\hspace*{23pt}problem&2&48--57\\
\end{tabular}
}
\pagebreak

\def\leftfootline{\small{\textbf{\thepage}
\hfill INFORMATIKA I EE PRIMENENIYA~--- INFORMATICS AND APPLICATIONS\ \ \ 2016\
\ \ volume~10\ \ \ issue\ 4}
}%
 \def\rightfootline{\small{INFORMATIKA I EE PRIMENENIYA~---
INFORMATICS AND APPLICATIONS\ \ \ 2016\ \ \ volume~10\ \ \ issue\ 4
\hfill \textbf{\thepage}}}

\def\leftkol{2016 AUTHOR INDEX} % ENGLISH ABSTRACTS}

\def\rightkol{2016 AUTHOR INDEX} %ENGLISH ABSTRACTS}


{\tabcolsep=2.83pt
\begin{tabular}{p{382pt}cc}
&\textbf{Issue} & \textbf{Page}\\[6pt]
\Avtors{Kadaner~A.\,I.} see~Chertok~A.\,V.&&\\[.255pt]
\Avtors{Kalinichenko~L.\,A., Volnova~A.\,A., Gordov~E.\,P.,
Kiselyova~N.\,N., Kovaleva~D.\,A., Malkov~O.\,Yu., Okladnikov~I.\,G.,
Podkolodnyy~N.\,L., Pozanenko~A.\,S., Ponomareva~N.\,V.,
Stupnikov~S.\,A.,} \textbf{and Fazliev~A.\,Z.} Data access challenges for data
intensive\linebreak
\\[-12pt]
\hspace*{23pt}research in Russia&1& 2--22\\[.255pt]
\Avtors{Karasikov~M.\,E.\ and Strijov~V.\,V.} Feature-based
time-series classification&4&121--131\\[.255pt]
\Avtors{Khazeeva~G.\,T.} see~Chertok~A.\,V.&&\\[.255pt]
\Avtors{Khokhlov~Yu.\,S.} Multivariate fractional Levy motion and its
applications&2&\hphantom{1}98--106\\[.255pt]
\Avtors{Kirikov~I.\,A., Kolesnikov~A.\,V., Listopad~S.\,V., and
Rumovskaya~S.\,B.} Fine-grained hybrid\linebreak
\\[-12pt]
\hspace*{23pt}intelligent systems. Part 2:
Bidirectional hybridization&1&\hphantom{1}96--105\\[.255pt]
\Avtors{Kirikov~I.\,A., Kolesnikov~A.\,V., Listopad~S.\,V., and
Rumovskaya~S.\,B.} ``Virtual council''~---\linebreak
\\[-12pt]
\hspace*{23pt}source environment
supporting complex diagnostic decision making&3&81--90\\[.255pt]
\Avtors{Kiselyova~N.\,N.} see~Kalinichenko~L.\,A.&&\\[.255pt]
\Avtors{Kolesnikov A.\,V., Listopad~S.\,V., Rumovskaya~S.\,B., and
Danishevsky~V.\,I.} Informal axiomatic\linebreak
\\[-12pt]
\hspace*{23pt}theory of~the~role visual models&4&114--120\\[.255pt]
\Avtors{Kolesnikov~A.\,V.} see~Kirikov~I.\,A.&&\\[.255pt]
\Avtors{Kolesnikov~A.\,V.} see~Kirikov~I.\,A.&&\\[.255pt]
\Avtors{Kolin~K.\,K.} Humanitarian aspects of information
security&3&111--121\\[.255pt]
\Avtors{Konovalov~M.\,G.\ and Razumchik~R.\,V.} Dispatching
to~two parallel nonobservable queues using\linebreak
\\[-12pt]
\hspace*{23pt}only static
information&4&57--67\\[.255pt]
\Avtors{Korchagin~A.\,Yu.} see~Korolev~V.\,Yu.&&\\[.255pt]
\Avtors{Korchagin~A.\,Yu.} see~Korolev~V.\,Yu.&&\\[.255pt]
\Avtors{Korepanov~E.\,R.} see~Sinitsyn~I.\,N.&&\\[.255pt]
\Avtors{Korepanov~E.\,R.} see~Sinitsyn~I.\,N.&&\\[.255pt]
\Avtors{Korolev~V.\,Yu., Korchagin~A.\,Yu., and Zeifman~A.\,I.} The
Poisson theorem for Bernoulli trials\linebreak
\\[-12pt]
\hspace*{23pt}with~a~random probability
of~success and~a~discrete analog of~the~Weibull distribution&4&11--20\\[.255pt]
\Avtors{Korolev~V.\,Yu., Zeifman~A.\,I., and Korchagin~A.\,Yu.}
Asymmetric Linnik distributions as~limit\linebreak
\\[-12pt]
\hspace*{23pt}laws for~random sums
of~independent random variables with~finite variances&4&21--33\\[.255pt]
\Avtors{Koucheryavy~E.\,A.} see~Ometov~A.\,Ya.&&\\[.255pt]
\Avtors{Kovaleva~D.\,A.} see~Kalinichenko~L.\,A.&&\\[.255pt]
\Avtors{Kovalyov~S.\,P.} Metaprogramming to increase
manufacturability of large-scale software-\linebreak
\\[-12pt]
\hspace*{23pt}intensive systems&1&56--66\\[.255pt]
\Avtors{Krivenko~M.\,P.} Significance tests of feature selection for
classification&3&32--40\\[.255pt]
\Avtors{Kruzhkov~M.\,G.} see~Zalizniak~Anna~A.&&\\[.255pt]
\Avtors{Kruzhkov~M.\,G.} see~Zatsman~I.\,M.&&\\[.255pt]
\Avtors{Kudryavtsev~A.\,A.} Bayesian queueing and reliability models:
\textit{A~priori} distributions with\linebreak
\\[-12pt]
\hspace*{23pt}compact support&1&67--71\\[.255pt]
\Avtors{Kudryavtsev~A.\,A.} Characteristics dependent on the balance
coefficient in Bayesian models\linebreak
\\[-12pt]
\hspace*{23pt}with compact support of \textit{a priori}
distributions&3&77--80\\[.255pt]
\Avtors{Kudryavtsev~A.\,A.\ and Palionnaia~S.\,I.} Bayesian recurrent
model of reliability growth:\linebreak
\\[-12pt]
\hspace*{23pt}Parabolic distribution of parameters&2&80--83\\[.255pt]
\Avtors{Kudryavtsev~A.\,A.\ and Titova~A.\,I.} Bayesian queuing
and~reliability models: Degenerate-\linebreak
\\[-12pt]
\hspace*{23pt}Weibull case&4&68--71\\[.255pt]
\Avtors{Leontyev~N.\,D.\ and Ushakov~V.\,G.} Analysis of a queueing
system with autoregressive arrivals\linebreak
\\[-12pt]
\hspace*{23pt}and nonpreemptive priority&3&15--22\\[.255pt]
\Avtors{Listopad~S.\,V.} see~Kirikov~I.\,A.&&\\[.255pt]
\Avtors{Listopad~S.\,V.} see~Kirikov~I.\,A.&&\\[.255pt]
\Avtors{Listopad~S.\,V.} see~Kolesnikov A.\,V.&&\\[.255pt]
\Avtors{Malkov~O.\,Yu.} see~Kalinichenko~L.\,A.&&\\[.255pt]
\Avtors{Markov~A.\,S., Monakhov~M.\,M., and
Ulyanov~V.\,V.} Generalized Cornish--Fisher expansions\linebreak
\\[-12pt]
\hspace*{23pt}for distributions of statistics based on samples
of random size&2&84--91\\[.255pt]
\Avtors{Melnikov~A.\,K.\ and Ronzhin~A.\,F.} Generalized statistical
method of~text analysis based\linebreak
\\[-12pt]
\hspace*{23pt}on~calculation of~probability distributions
of~statistical values&4&89--95\\
\end{tabular}
}
\pagebreak

\def\leftfootline{\small{\textbf{\thepage}
\hfill INFORMATIKA I EE PRIMENENIYA~--- INFORMATICS AND APPLICATIONS\ \ \ 2016\
\ \ volume~10\ \ \ issue\ 4}
}%
 \def\rightfootline{\small{INFORMATIKA I EE PRIMENENIYA~---
INFORMATICS AND APPLICATIONS\ \ \ 2016\ \ \ volume~10\ \ \ issue\ 4
\hfill \textbf{\thepage}}}

\def\leftkol{2016 AUTHOR INDEX} % ENGLISH ABSTRACTS}

\def\rightkol{2016 AUTHOR INDEX} %ENGLISH ABSTRACTS}


{\tabcolsep=3pt
\begin{tabular}{p{381pt}cc}
&\textbf{Issue} & \textbf{Page}\\[6pt]
\Avtors{Meykhanadzhyan~L.\,A.} Stationary characteristics of the finite
capacity queueing system with\linebreak
\\[-12pt]
\hspace*{23pt}inverse service order and generalized
probabilistic priority&2&123--131\\[.23pt]
\Avtors{Miller~G.\,B.} see~Borisov~A.\,V.&&\\[.23pt]
\Avtors{Minin~V.\,A., Zatsman~I.\,M., Havanskov~V.\,A., and
Shubnikov~S.\,K.} Intensity of citation of scientific publications in
inventions on information and computer technologies patented\linebreak
\\[-12pt]
\hspace*{23pt}in Russia by domestic and foreign applicants&2&107--122\\[.23pt]
\Avtors{Monakhov~M.\,M.} see~Markov~A.\,S.&&\\[.23pt]
\Avtors{Naumov~V.\,A.\ and Samouylov~K.\,E.} On relationship
between queuing systems with resources\linebreak
\\[-12pt]
\hspace*{23pt}and Erlang networks&3&\hphantom{1}9--14\\[.23pt]
\Avtors{Okladnikov~I.\,G.} see~Kalinichenko~L.\,A.&&\\[.23pt]
\Avtors{Ometov~A.\,Ya., Andreev~S.\,D., Turlikov~A.\,M., and
Koucheryavy~E.\,A.} Performance analysis of\linebreak
\\[-12pt]
\hspace*{23pt}a wireless data
aggregation system with contention for contemporary sensor
networks&3&23--31\\[.23pt]
\Avtors{Palionnaia~S.\,I.} see~Kudryavtsev~A.\,A.&&\\[.23pt]
\Avtors{Podkolodnyy~N.\,L.} see~Kalinichenko~L.\,A.&&\\[.23pt]
\Avtors{Ponomareva~N.\,V.} see~Kalinichenko~L.\,A.&&\\[.23pt]
\Avtors{Popkova~N.\,A.} see~Zatsman~I.\,M.&&\\[.23pt]
\Avtors{Pozanenko~A.\,S.} see~Kalinichenko~L.\,A.&&\\[.23pt]
\Avtors{Razumchik~R.\,V.} see~Konovalov~M.\,G.&&\\[.23pt]
\Avtors{Ronzhin~A.\,F.} see~Melnikov~A.\,K.&&\\[.23pt]
\Avtors{Rumovskaya~S.\,B.} see~Kirikov~I.\,A.&&\\[.23pt]
\Avtors{Rumovskaya~S.\,B.} see~Kirikov~I.\,A.&&\\[.23pt]
\Avtors{Rumovskaya~S.\,B.} see~Kolesnikov A.\,V.&&\\[.23pt]
\Avtors{Samouylov~K.\,E.} see~Gaidamaka~Yu.\,V.&&\\[.23pt]
\Avtors{Samouylov~K.\,E.} see~Naumov~V.\,A.&&\\[.23pt]
\Avtors{Serebryanskii~S.\,M.} see~Tyrsin~A.\,N.&&\\[.23pt]
\Avtors{Seyful-Mulyukov~R.\,B.} see~Callaos~N.\,K.&&\\[.23pt]
\Avtors{Shestakov~O.\,V.} Statistical properties of the denoising method
based on the stabilized hard\linebreak
\\[-12pt]
\hspace*{23pt}thresholding&2&65--69\\[.23pt]
\Avtors{Shestakov~O.\,V.} The strong law of large numbers for the risk
estimate in the problem of\linebreak
\\[-12pt]
\hspace*{23pt}tomographic image reconstruction from
projections with a correlated noise&3&41--45\\[.23pt]
\Avtors{Shestakov~O.\,V.} see~Zakharova~T.\,V.&&\\[.23pt]
\Avtors{Shnurkov~P.\,V., Gorshenin~A.\,K., and Belousov~V.\,V.}
Analytical solution of~the~optimal control\linebreak
\\[-12pt]
\hspace*{23pt}task of~a~semi-Markov
process with~finite set of~states&4&72--88\\[.23pt]
\Avtors{Shnurkov~P.\,V., Zasypko~V.\,V., Belousov~V.\,V., and
Gorshenin~A.\,K.} Development of the algorithm of numerical solution
of the optimal investment control problem\linebreak
\\[-12pt]
\hspace*{23pt}in the closed dynamical model of three-sector economy&1&82--95\\[.23pt]
\Avtors{Shorgin~S.\,Ya.} see~Gaidamaka~Yu.\,V.&&\\[.23pt]
\Avtors{Shorgin~V.\,S.} see~Agalarov~Ya.\,M.&&\\[.23pt]
\Avtors{Shubnikov~S.\,K.} see~Minin~V.\,A.&&\\[.23pt]
\Avtors{Sidorkin~I.\,I.} see~Arkhipov~O.\,P.&&\\[.23pt]
\Avtors{Sinitsyn~I.\,N.} Analytical modeling of processes in stochastic
systems with complex fractional\linebreak
\\[-12pt]
\hspace*{23pt}order Bessel nonlinearities&3&55--65\\[.23pt]
\Avtors{Sinitsyn~I.\,N.} Orthogonal supoptimal filters for nonlinear
stochastic systems on manifolds&1&34--44\\[.23pt]
\Avtors{Sinitsyn~I.\,N.\ and Korepanov~E.\,R.} Normal Pugachev
conditionally-optimal filters and extra-\linebreak
\\[-12pt]
\hspace*{23pt}polators for state linear stochastic systems&2&14--23\\[.23pt]
\Avtors{Sinitsyn~I.\,N.\ and Sinitsyn~V.\,I.} Analytical modeling of
distributions in stochastic systems on\linebreak
\\[-12pt]
\hspace*{23pt}manifolds based on ellipsoidal approximation&1&45--55\\[.23pt]
\Avtors{Sinitsyn~I.\,N., Sinitsyn~V.\,I., and
Korepanov~E.\,R.} Ellipsoidal suboptimal filters for nonlinear\linebreak
\\[-12pt]
\hspace*{23pt}stochastic systems on manifolds&2&24--35\\[.23pt]
\Avtors{Sinitsyn~V.\,I.} see~Sinitsyn~I.\,N.&&\\[.23pt]
\Avtors{Sinitsyn~V.\,I.} see~Sinitsyn~I.\,N.&&\\[.23pt]
\Avtors{Skvortsov~N.\,A.} see~Stupnikov~S.\,A.&&\\[.23pt]
\Avtors{Sokolov~I.\,A.} see~Chertok~A.\,V.&&\\
\end{tabular}
}
\pagebreak

\def\leftfootline{\small{\textbf{\thepage}
\hfill INFORMATIKA I EE PRIMENENIYA~--- INFORMATICS AND APPLICATIONS\ \ \ 2016\
\ \ volume~10\ \ \ issue\ 4}
}%
 \def\rightfootline{\small{INFORMATIKA I EE PRIMENENIYA~---
INFORMATICS AND APPLICATIONS\ \ \ 2016\ \ \ volume~10\ \ \ issue\ 4
\hfill \textbf{\thepage}}}

\def\leftkol{2016 AUTHOR INDEX} % ENGLISH ABSTRACTS}

\def\rightkol{2016 AUTHOR INDEX} %ENGLISH ABSTRACTS}


{\tabcolsep=3pt
\begin{tabular}{p{382pt}cc}
&\textbf{Issue} & \textbf{Page}\\[6pt]
\Avtors{Sopin~E.\,S.} see~Gaidamaka~Yu.\,V.&&\\
\Avtors{Strijov~V.\,V.} see~Goncharov~A.\,V.&&\\
\Avtors{Strijov~V.\,V.} see~Isachenko~R.\,V.&&\\
\Avtors{Strijov~V.\,V.} see~Karasikov~M.\,E.&&\\
\Avtors{Stupnikov~S.\,A., Briukhov~D.\,O., and Skvortsov~N.\,A.}
Co-lending systemic risk analysis over\linebreak
\\[-12pt]
\hspace*{23pt}heterogeneous data collections&1&23--33\\
\Avtors{Stupnikov~S.\,A.} see~Kalinichenko~L.\,A.&&\\
\Avtors{Suchkov~A.\,P.} see~Zatsarinny~A.\,A.&&\\
\Avtors{Timonina~E.\,E.} see~Grusho~A.\,A.&&\\
\Avtors{Titova~A.\,I.} see~Kudryavtsev~A.\,A.&&\\
\Avtors{Turlikov~A.\,M.} see~Ometov~A.\,Ya.&&\\
\Avtors{Tyrsin~A.\,N.\ and Serebryanskii~S.\,M.} Recognition of
dependences on the basis of inverse\linebreak
\\[-12pt]
\hspace*{23pt}mapping&2&58--64\\
\Avtors{Ulyanov~V.\,V.} see~Markov~A.\,S.&&\\
\Avtors{Ushakov~V.\,G.} Queueing system with working vacations and
hyperexponential input stream&2&92--97\\
\Avtors{Ushakov~V.\,G.} see~Leontyev~N.\,D.&&\\
\Avtors{Volnova~A.\,A.} see~Kalinichenko~L.\,A.&&\\
\Avtors{Yakovlev~O.\,A.\ and Gasilov~A.\,V.} Speeded-up stereo
matching using geodesic support weights&3&\hphantom{1}98--104\\
\Avtors{Zabezhailo~M.\,I.} see~Grusho~A.\,A.&&\\
\Avtors{Zabezhailo~M.\,I.} see~Grusho~A.\,A.&&\\
\Avtors{Zakharova~T.\,V.\ and Shestakov~O.\,V.} Precision analysis of
wavelet processing of aerodynamic\linebreak
\\[-12pt]
\hspace*{23pt}flow patterns&3&46--54\\
\Avtors{Zalizniak~Anna~A.\ and Kruzhkov~M.\,G.} Database
of~Russian impersonal verbal constructions&4&132--141\\
\Avtors{Zasypko~V.\,V.} see~Shnurkov~P.\,V.&&\\
\Avtors{Zatsarinny~A.\,A.\ and Suchkov~A.\,P.} Systems engineering
approaches to~the~establishment of\linebreak
\\[-12pt]
\hspace*{23pt}a~system for~decision support based
on~situational analysis&4&105--113\\
\Avtors{Zatsarinny~A.\,A.} see~Grusho~A.\,A.&&\\
\Avtors{Zatsman~I.\,M., Inkova~O.\,Yu., Kruzhkov~M.\,G., and
Popkova~N.\,A.} Representation of cross-\linebreak
\\[-12pt]
\hspace*{23pt}lingual knowledge about
connectors in supracorpora databases&1&106--118\\
\Avtors{Zatsman~I.\,M.} see~Minin~V.\,A.&&\\
\Avtors{Zeifman~A.\,I.} see~Korolev~V.\,Yu.&&\\
\Avtors{Zeifman~A.\,I.} see~Korolev~V.\,Yu.&&\\
\end{tabular}
}

%\thispagestyle{myheadings}
\def\leftfootline{\small{\textbf{\thepage}
\hfill INFORMATIKA I EE PRIMENENIYA~--- INFORMATICS AND APPLICATIONS\ \ \ 2016\
\ \ volume~10\ \ \ issue\ 4}
}%
 \def\rightfootline{\small{INFORMATIKA I EE PRIMENENIYA~---
INFORMATICS AND APPLICATIONS\ \ \ 2016\ \ \ volume~10\ \ \ issue\ 4
\hfill \textbf{\thepage}}}

 \label{end\stat}

\newpage

%\def\stat{rekl}
%\label{preobr}

%\def\tit{АКАДЕМИК ПУГАЧЁВ  ВЛАДИМИР СЕМЁНОВИЧ\\
%25.03.1911--25.03.1998}


%   \vspace*{-48pt}
%   \begin{center}\LARGE
%Академик Пугачёв  Владимир Семёнович\\ (25.03.1911--25.03.1998)
%   \end{center}
   
   %\vspace*{2.5mm}
   
   \begin{center}

{\prgsh\LARGE
ОБЪЯВЛЕНИЯ О КОНФЕРЕНЦИЯХ}

\end{center}
%\hrule

\vspace*{6pt}

   
   \vspace*{10mm}
   
   \thispagestyle{empty}

\noindent
\begin{tabular}{cc}
%\begin{center}
\multicolumn{1}{c}{\raisebox{-40pt}[0pt][0pt]{\mbox{%
\epsfxsize=33mm
\epsfbox{vspu.eps}
}}}
%\end{center}
&
\tabcolsep=0pt\begin{tabular}{c}
{\prg{\Large\textbf{XII Всероссийское совещание}}}\\[6pt]
{\prg{\Large\textbf{по проблемам управления}}}\\[12pt]
{\prg{\large 16--19 июня 2014~г.}}\\[6pt] 
{\prg{\large Институт проблем управления имени В.\,А.~Трапезникова РАН}}\\[6pt]
{\prg{\large Москва, Россия}}
\end{tabular}
\end{tabular}

\vspace*{60pt}

     
 { %\large    
 XII Всероссийское совещание по проблемам управления (ВСПУ XII), посвященное 75-летию 
Института проблем управления (ИПУ) имени В.\,А.~Трапезникова РАН, проводится 16--19~июня 
2014~г.\ 
в ИПУ РАН (г.~Москва, Россия). ВСПУ XII организуется ИПУ РАН при поддержке РФФИ, Отделения 
энергетики, машиностроения, механики и процессов управления Российской академии наук, 
Российского 
национального комитета по автоматическому управлению, Академии навигации и управ\-ле\-ния 
движением, 
Научного совета РАН по комплексным проблемам управления и автоматизации, Совета по 
мехатронике и робототехнике РАН. Официальный язык Совещания~--- русский.

\vspace*{24pt}
     
     \textbf{Направления работы}
     \begin{enumerate}[1.]
\item Теория систем управления
\item Управление подвижными объектами и навигация
\item Интеллектуальные системы управления
\item Управление в промышленности, транспортом и логистикой
\item Управление системами междисциплинарной природы
\item Средства измерения, вычислений и контроля в управлении
\item Системный анализ и принятие решений в задачах управления
\item Информационные технологии в управлении
\item Проблемы образования в области управления: современное содержание и технологии обучения
\end{enumerate}

\vspace*{24pt}

     Подробная информация о Совещании находится на сайте {\sf http://vspu2014.ipu.ru}. Срок 
окончательной подачи докладов через систему подачи докладов на сайте~--- \textbf{30~ноября} 
2013~г.
}

%\include{rekl-1}

%\end{document}

%\include{nekrolog-rb}


%\end{document}

%\include{IPPM-25}

\def\stat{cont-rus}
{%\hrule\par
%\vskip 7pt % 7pt
\vspace*{-24pt}
\raggedleft\Large \bf%\baselineskip=3.2ex
Правила подготовки рукописей  для публикации в журнале
<<Информатика~и~её~применения>> \vskip 8pt
    \hrule
    \par
\vskip 14pt plus 6pt minus 3pt }

\label{st\stat}

\def\tit{\ }

\def\aut{\ }
\def\auf{\ }

\def\leftkol{\ }
% Правила подготовки рукописей  для публикации в журнале
%<<Информатика и её применения>>

\def\rightkol{\ }
%Правила подготовки рукописей  для публикации в журнале
%<<Информатика и её применения>>}


\titele{\tit}{\aut}{\auf}{\leftkol}{\rightkol}


\vspace*{-60pt}
{ %\small

Журнал <<Информатика и её применения>>
публикует теоретические, обзорные и дискуссионные статьи,
посвященные научным исследованиям и разработкам в области
информатики и ее приложений.

Журнал издается на русском языке. По специальному решению
редколлегии отдельные статьи могут печататься на английском языке.

Тематика журнала охватывает следующие направления:
\begin{itemize}
\item теоретические основы информатики;\\[-15pt]
      \item
математические методы исследования сложных систем и процессов;\\[-15pt]
           \item
информационные системы и сети;\\[-15pt]
                \item
информационные технологии;\\[-15pt]
                     \item
архитектура и программное обеспечение вычислительных комплексов и сетей.\\[-15pt]
\end{itemize}


\noindent
\begin{enumerate}[1.]
\item В журнале печатаются статьи, содержащие результаты, ранее не опубликованные и
не предназначенные к одновременной публикации в других изданиях.

%Публикация не должна нарушать закон об авторских правах.
Публикация предоставленной автором(ами) рукописи не должна нарушать 
положений глав~69, 70 раздела~VII части~IV Гражданского кодекса, 
которые определяют права на результаты интеллектуальной деятельности 
и~средства индивидуализации, в~том числе авторские права, в~РФ.

Ответственность за нарушение авторских прав, в~случае предъявления претензий к~редакции журнала,  
несут авторы статей.



Направляя рукопись в редакцию, авторы сохраняют свои права на данную
рукопись и при этом передают учредителям и редколлегии журнала неисключительные права на
издание статьи на русском языке 
(или на языке статьи, если он отличен от рус\-ско\-го) и~на перевод ее на английский
язык, а~также на
ее распространение в России и за рубежом. 
Каждый автор должен представить в~редакцию подписанный 
с~его стороны <<Лицензионный договор о~передаче неисключительных прав 
на использование произведения>>, текст которого размещен по адресу 
{\sf http://www.ipiran.ru/publications/licence.doc}. 
Этот договор может быть пред\-став\-лен в~бумажном (в~2-х экз.)\ 
или в~электронном виде (отсканированная копия заполненного и~подписанного документа).




Редколлегия вправе запросить у авторов экспертное заключение о возможности
пуб\-ли\-ка\-ции пред\-став\-лен\-ной статьи в открытой печати.\\[-13.5pt]

\item К статье прилагаются данные автора (авторов) (см.\ п.~8). При наличии нескольких
авторов указывается фамилия автора, ответственного за переписку с редакцией.\\[-13.5pt]

\item Редакция журнала осуществляет экспертизу присланных статей в соответствии с
принятой в журнале процедурой рецензирования.

Возвращение рукописи на доработку не означает ее принятия к печати.

Доработанный вариант с ответом на замечания рецензента необходимо прислать в
редакцию.\\[-13.5pt]

\item Решение редколлегии о публикации статьи или ее отклонении сообщается авторам.

Редколлегия может также направить авторам текст рецензии на их статью. Дискуссия по
поводу отклоненных статей не ведется.\\[-13.5pt]

%\pagebreak

\item Редактура статей высылается авторам для просмотра. Замечания к редактуре должны
быть присланы авторами в кратчайшие сроки.\\[-13.5pt]

\item Рукопись предоставляется в электронном виде в форматах MS WORD (.doc или
.docx) или \LaTeX\  (.tex), дополнительно~--- в формате .pdf, на дискете, лазерном диске
или электронной почтой. Предоставление бумажной рукописи необязательно.\\[-13.5pt]

\item При подготовке рукописи в MS Word рекомендуется использовать следующие
настройки.

Параметры страницы:
формат~--- А4; ориентация~--- книжная; поля (см): внутри~--- 2,5, снаружи~--- 1,5,
сверху~--- 2, снизу~--- 2, от края до нижнего колонтитула~--- 1,3.

Основной текст: стиль~--- <<Обычный>>, шрифт~--- Times New Roman, размер~---
14~пунк\-тов, абзацный отступ~--- 0,5~см, 1,5~интервала, выравнивание~--- по ширине.

\pagebreak

\def\leftkol{Правила подготовки рукописей  для публикации в журнале
<<Информатика и её применения>>}

\def\rightkol{Правила подготовки рукописей  для публикации в журнале
<<Информатика и её применения>>}



Рекомендуемый объем рукописи~--- не свыше 10~страниц указанного формата.
При превышении указанного объема редколлегия вправе потребовать от 
автора сокращения объема рукописи.


Сокращения слов, помимо стандартных, не допускаются. Допускается минимальное
количество аббревиатур.


Все страницы рукописи нумеруются.

Шаблоны оформления представлены в интернете:

\noindent
 {\sf
http://www.ipiran.ru/journal/template\_iiep\_ssi\_2024.zip}\\[-14pt]

\item Статья должна содержать следующую информацию на {\bfseries\textit{русском и
английском языках}}:\\[-16pt]

\begin{itemize}
\item название статьи;\\[-15pt]
\item Ф.И.О.\ авторов, на английском можно только имя и фамилию;\\[-15pt]
\item место работы, с указанием почтового адреса организации и электронного адреса каждого
автора;\\[-15pt]
\item сведения об авторах, в соответствии с форматом, образцы которого
представлены на страницах:



\def\leftfootline{\small{\textbf{\thepage}
\hfill ИНФОРМАТИКА И ЕЁ ПРИМЕНЕНИЯ\ \ \ том\ 18\ \ \ выпуск\ 3\ \ \ 2024}
}%
 \def\rightfootline{\small{ИНФОРМАТИКА И ЕЁ ПРИМЕНЕНИЯ\ \ \ том\ 18\ \ \ выпуск\ 3\ \ \ 2024
\hfill \textbf{\thepage}}}



{\sf http://www.ipiran.ru/journal/issues/2013\_07\_01/authors.asp} и

{\sf http://www.ipiran.ru/journal/issues/2013\_07\_01\_eng/authors.asp};
\item аннотация (не менее 100~слов на каждом из языков). Аннотация~--- это краткое
резюме работы, которое может публиковаться отдельно. Она является основным
источником информации в~ин\-фор\-ма\-ци\-он\-ных системах и базах данных. Английская
аннотация должна быть оригинальной, может не быть дословным переводом русского
текста и должна быть написана хорошим английским языком. В~аннотации не должно
быть ссылок на литературу и, по возможности, формул;\\[-15pt]
\item ключевые слова~--- желательно из принятых в мировой
на\-уч\-но-тех\-ни\-че\-ской литературе тематических тезаурусов. Предложения не
могут быть ключевыми словами;\\[-15pt]
\item источники финансирования работы (ссылки на гранты, проекты,
поддерживающие организации и~т.\,п.).
\end{itemize}



%\pagebreak

\item  Требования к спискам литературы.\\[-14pt]

Ссылки на литературу в тексте статьи нумеруются (в квадратных скобках) и
располагаются в каждом из списков литературы в порядке  первых упоминаний. Если источник имеет DOI и/или EDN,
то их необходимо указывать.

Списки литературы представляются в двух вариантах:\\[-14pt]


\noindent
\begin{enumerate}[(1)]
\item \textbf{Список литературы к русскоязычной части}. Русские и английские
работы~---  на языке и в алфавите оригинала;\\[-14.5pt]
\item  \textbf{References}. Русские работы и работы на других языках~--- в латинской
транслитерации с переводом на английский язык; английские работы и работы на других
языках~--- на языке оригинала.
\end{enumerate}

Необходимо для составления списка ``References'' пользоваться размещенной на сайте
{\sf http://www. translit.net/ru/bgn/} бесплатной программой транслитерации русского
 текста в~латиницу. %, при этом в~за\-клад\-ке <<варианты\ldots>> следует выбратьопцию BGN.

Список литературы ``References'' приводится полностью отдельным блоком, повторяя все
позиции из списка литературы к русскоязычной части, независимо от того, имеются или
нет в нем иностранные источники. Если в списке литературы к русскоязычной части есть
ссылки на иностранные публикации, набранные латиницей, они полностью повторяются в
списке ``References''.

Ниже приведены примеры ссылок на различные виды публикаций в списке ``References''.

\def\leftfootline{\small{\textbf{\thepage}
\hfill ИНФОРМАТИКА И ЕЁ ПРИМЕНЕНИЯ\ \ \ том\ 18\ \ \ выпуск\ 3\ \ \ 2024}
}%
 \def\rightfootline{\small{ИНФОРМАТИКА И ЕЁ ПРИМЕНЕНИЯ\ \ \ том\ 18\ \ \ выпуск\ 3\ \ \ 2024
\hfill \textbf{\thepage}}}

{\small

\noindent
\textbf{Описание статьи из журнала:}

\Aue{Zagurenko, A.\,G., V.\,A.~Korotovskikh, A.\,A.~Kolesnikov, A.\,V.~Timonov, and D.\,V.~Kardymon}. 2008.
Tekhniko-ekonomicheskaya optimizatsiya dizayna gidrorazryva plasta [Technical and
economic optimization of the design
of hydraulic fracturing]. \textit{Neftyanoe hozyaystvo} [\textit{Oil Industry}] 11:54--57.

\Aue{Zhang, Z., and D.~Zhu}. 2008. Experimental research on the localized
electrochemical micromachining. \textit{Russ. J.~Electrochem.}  44(8):926--930.
{\sf doi:10.1134/S1023193508080077}.

\noindent
\textbf{Описание статьи из электронного журнала:}

\Aue{Swaminathan, V., E.~Lepkoswka-White, and B.\,P.~Rao}. 1999. Browsers or buyers in cyberspace? An
investigation of electronic factors influencing electronic exchange. \textit{JCMC}
5(2). Available at: {\sf http://www.ascusc.org/jcmc/vol5/issue2/} (accessed April~28, 2011).

\def\leftkol{Правила подготовки рукописей  для публикации в журнале
<<Информатика и её применения>>}

\def\rightkol{Правила подготовки рукописей  для публикации в журнале
<<Информатика и её применения>>}


\noindent
\textbf{Описание статьи из продолжающегося издания (сборника трудов):}

\Aue{Astakhov, M.\,V., and T.\,V.~Tagantsev}. 2006. Eksperimental'noe
issledovanie prochnosti soedineniy ``stal'--kompozit'' [Experimental study of
the strength of joints ``steel--composite'']. \textit{Trudy MGTU
``Matematicheskoe modelirovanie slozhnykh tekh\-ni\-che\-skikh sistem''}
[\textit{Bauman MSTU ``Mathematical Modeling of Complex Technical
Systems'' Proceedings}]. 593:125--130.


\pagebreak



\noindent
\textbf{Описание материалов конференций:}

\Aue{Usmanov, T.\,S., A.\,A.~Gusmanov, I.\,Z.~Mullagalin, R.\,Ju.~Muhametshina, A.\,N.~Chervyakova, and
A.\,V.~Sveshnikov}. 2007. Osobennosti proektirovaniya razrabotki mestorozhdeniy
s primeneniem gidrorazryva
plasta [Features of the design of field development with the use of hydraulic fracturing].
\textit{Trudy 6-go
Mezhdu\-na\-rod\-no\-go Simpoziuma ``Novye resursosberegayushchie tekhnologii nedropol'zovaniya i povysheniya
neftegazootdachi''} [\textit{6th  Symposium (International) ``New Energy Saving Subsoil Technologies and
the Increasing of the Oil and Gas Impact'' Proceedings}]. Moscow. 267--272.



\def\leftfootline{\small{\textbf{\thepage}
\hfill ИНФОРМАТИКА И ЕЁ ПРИМЕНЕНИЯ\ \ \ том\ 18\ \ \ выпуск\ 3\ \ \ 2024}
}%
 \def\rightfootline{\small{ИНФОРМАТИКА И ЕЁ ПРИМЕНЕНИЯ\ \ \ том\ 18\ \ \ выпуск\ 3\ \ \ 2024
\hfill \textbf{\thepage}}}



\noindent
\textbf{Описание книги (монографии, сборники):}



Lindorf, L.\,S., and L.\,G.~Mamikoniants, eds. 1972.
\textit{Ekspluatatsiya turbogeneratorov s neposredstvennym
okhlazhdeniem} [\textit{Operation of turbine generators with direct cooling}].
Moscow: Energy Publs. 352~p.


\Aue{Latyshev, V.\,N.} 2009. \textit{Tribologiya rezaniya. Kn.~1: Friktsionnye protsessy
pri rezanii metallov}
[\textit{Tribology of cutting. Vol.~1: Frictional processes in metal cutting}]. Ivanovo: Ivanovskii
State Univ. 108~p.

\def\leftkol{Правила подготовки рукописей  для публикации в журнале
<<Информатика и её применения>>}

\def\rightkol{Правила подготовки рукописей  для публикации в журнале
<<Информатика и её применения>>}

\noindent
\textbf{Описание переводной книги}
(в списке литературы к русскоязычной части необходимо указать:~/ Пер.\ с англ.~---
после названия книги, а в конце ссылки указать оригинал книги в круглых скобках):
\begin{enumerate}[1.]
\item  В русскоязычной части:

\def\leftfootline{\small{\textbf{\thepage}
\hfill ИНФОРМАТИКА И ЕЁ ПРИМЕНЕНИЯ\ \ \ том\ 18\ \ \ выпуск\ 3\ \ \ 2024}
}%
 \def\rightfootline{\small{ИНФОРМАТИКА И ЕЁ ПРИМЕНЕНИЯ\ \ \ том\ 18\ \ \ выпуск\ 3\ \ \ 2024
\hfill \textbf{\thepage}}}

\Au{Тимошенко С.\,П., Янг Д.\,Х., Уивер~У.}
Колебания в инженерном деле~/ Пер.\ с англ.~--- М.: Машиностроение, 1985. 472~с.
(\Au{Timoshenko~S.\,P., Young~D.\,H., Weaver~W.}
Vibration problems in engineering.~--- 4th ed.~--- New York, NY, USA: Wiley, 1974. 521~p.)\\[-13.5pt]
\item  В англоязычной части:

\Aue{Timoshenko, S.\,P., D.\,H.~Young, and W.~Weaver}.
1974. \textit{Vibration problems in engineering}. 4th ed. New York: 
Wiley. 521~p.
\end{enumerate}

\vspace*{-3pt}


\noindent
\textbf{Описание неопубликованного документа:}


\Aue{Latypov, A.\,R., M.\,M.~Khasanov, and V.\,A.~Baikov}.
2004 (unpubl.). Geologiya i~dobycha (NGT GiD) [Geology and production (NGT GiD)]. Certificate on official registration of the computer program
No.\,2004611198. 

\noindent
\textbf{Описание интернет-ресурса:}


Pravila tsitirovaniya istochnikov [Rules for the citing of sources]. Available at: {\sf
http://www.scribd.com/doc/1034528/} (accessed February~7, 2011).

%\pagebreak

\noindent
\textbf{Описание диссертации или автореферата диссертации:}

\Aue{Semenov, V.\,I.}
2003. Matematicheskoe modelirovanie plazmy v sisteme kompaktnyy tor [Mathematical
modeling of the plasma in the compact torus].  Moscow.  D.Sc.\ Diss. 272~p.

\Aue{Kozhunova, O.\,S.} 2009. Tekhnologiya razrabotki semanticheskogo
slovarya informatsionnogo monitoringa [Technology of development of
semantic dictionary of information monitoring system].  Moscow: IPI RAN. PhD Thesis. 23~p.


\noindent
\textbf{Описание ГОСТа:}

GOST 8.586.5-2005. 2007. Metodika vypolneniya izmereniy. Izmerenie raskhoda i~kolichestva zhidkostey i~gazov
s~pomoshch'yu standartnykh suzhayushchikh ustroystv [Method of measurement.
Measurement of flow rate and volume of liquids and gases by means of orifice devices]. Moscow:
Standardinform  Publs. 10~p.

\noindent
\textbf{Описание патента:}

\Aue{Bolshakov, M.\,V., A.\,V.~Kulakov, A.\,N.~Lavrenov, and M.\,V.~Palkin}.
2006. Sposob orientirovaniya po krenu letatel'nogo
apparata s opti\-che\-skoy golovkoy
samonavedeniya [The way to orient on the roll of aircraft with optical homing head].
Patent RF No.\,2280590.
}

\item Присланные в редакцию материалы авторам не возвращаются.\\[-13.5pt]

\item При отправке файлов по электронной почте просим придерживаться следующих
правил:
\begin{itemize}
\item указывать в поле subject (тема) название журнала и фамилию автора;\\[-13.5pt]
\item указывать в тексте письма название статьи, авторов и~журнал, в~который направляется статья;\\[-13.5pt]
\item использовать attach (присоединение);\\[-13.5pt]
\item в состав электронной версии статьи должны входить: файл, содержащий текст
статьи, и файл(ы), содержащий(е) иллюстрации.\\[-13.5pt]
\end{itemize}

\item Журнал <<Информатика и её применения>> является некоммерческим изданием.
Плата за публикацию не взимается, гонорар авторам не выплачивается.
\end{enumerate}



\def\leftfootline{\small{\textbf{\thepage}
\hfill ИНФОРМАТИКА И ЕЁ ПРИМЕНЕНИЯ\ \ \ том\ 18\ \ \ выпуск\ 3\ \ \ 2024}
}%
 \def\rightfootline{\small{ИНФОРМАТИКА И ЕЁ ПРИМЕНЕНИЯ\ \ \ том\ 18\ \ \ выпуск\ 3\ \ \ 2024
\hfill \textbf{\thepage}}}


\vspace*{-1mm}

\begin{center}

\textbf{Адрес редакции журнала <<Информатика и её применения>>:} \\




Москва 119333, ул.~Вавилова, д.~44, корп.~2, ФИЦ ИУ РАН\\[-10pt]

\

Тел.: +7\,(499)\,135-86-92\ \ Факс:  +7\,(495)\,930-45-05\\[-10pt]

 \

e-mail:   {\sf iiep@frccsc.ru} (Стригина Светлана Николаевна)\\[-10pt]

\

{\sf http://www.ipiran.ru/journal/issues/}
\end{center}
}


\def\leftkol{Правила подготовки рукописей  для публикации в журнале
<<Информатика и её применения>>}

\def\rightkol{Правила подготовки рукописей  для публикации в журнале
<<Информатика и её применения>>}


\def\leftfootline{\small{\textbf{\thepage}
\hfill ИНФОРМАТИКА И ЕЁ ПРИМЕНЕНИЯ\ \ \ том\ 18\ \ \ выпуск\ 3\ \ \ 2024}
}%
 \def\rightfootline{\small{ИНФОРМАТИКА И ЕЁ ПРИМЕНЕНИЯ\ \ \ том\ 18\ \ \ выпуск\ 3\ \ \ 2024
\hfill \textbf{\thepage}}} 
\def\stat{podg-e}
{%\hrule\par
%\vskip 7pt % 7pt
\vspace*{-24pt}
\raggedleft\Large \bf%\baselineskip=3.2ex
Requirements for manuscripts submitted to Journal
``Informatics~and~Applications'' \vskip 8pt
    \hrule
    \par
\vskip 21pt plus 6pt minus 3pt }

\label{st\stat}

\def\tit{\ }

\def\aut{\ }
\def\auf{\ }

\def\leftkol{\ }

\def\rightkol{\ }
%Requirements for manuscripts submitted to Journal
%``Informatics~and~Applications''}

\titele{\tit}{\aut}{\auf}{\leftkol}{\rightkol}

\def\leftfootline{\small{\textbf{\thepage}
\hfill INFORMATIKA I EE PRIMENENIYA~--- INFORMATICS AND APPLICATIONS\ \ \ 2019\
\ \ volume~13\ \ \ issue\ 4}
}%
 \def\rightfootline{\small{INFORMATIKA I EE PRIMENENIYA~--- INFORMATICS AND APPLICATIONS\ \ \ 2019\ \ \ volume~13\ \ \ issue\ 4
\hfill \textbf{\thepage}}}

\vspace*{-60pt}

{\small

\noindent
Journal ``Informatics and Applications'' (Inform.\ Appl.)
publishes theoretical, review, and discussion
articles on the research and development in the
field of informatics and its applications.

The journal is published in Russian.
By a special decision of the editorial
board, some articles can be published in English.


The topics covered include the following areas:
\begin{itemize}
               \item
     theoretical fundamentals of informatics; \\[-14pt]
\item
mathematical methods for studying complex systems and processes; \\[-14pt]
\item
information systems and networks;\\[-14pt]
\item
information technologies; and \\[-14pt]
\item
architecture and software of computational complexes and networks. \\[-14pt]
\end{itemize}

\noindent
\begin{enumerate}[1.]
\item The Journal publishes original articles which have not been published before and are not
intended for simultaneous publication in other editions. An article submitted to the Journal must not violate the
Copyright law. Sending the manuscript to the Editorial Board, the authors retain all rights of the
owners of the manuscript and transfer the nonexclusive rights to publish the article in Russian
(or the language of the article, if not Russian) and its distribution in Russia and abroad to the
Founders and the Editorial Board. Authors should submit a letter to the Editorial Board in the
following form:

{\bfseries\textit{Agreement on the transfer of rights to publish:}}

``\textit{We, the undersigned authors of the manuscript ``\ldots'', pass to the
Founder and the Editorial Board of the Journal ``Informatics and Applications''
the nonexclusive right to publish the manuscript of the article in Russian (or
in English) in both print and electronic versions of the Journal. We affirm
that this publication does not violate the Copyright of other persons or
organizations.}

\textit{Author(s) signature(s): (name(s), address(es), date).}

This agreement should be submitted in paper form or in the form of a scanned copy (signed by
the authors).


%The Editorial Board has the right to request from the authors an official expert conclusion that
%the submitted article has no secret data prohibited for publication. \\[-13.5pt]
\item
A submitted article should be attached with \textbf{the data on the author(s)} (see item~8). If
there are several authors, the contact person should be indicated who is responsible for
correspondence with the Editorial Board and other authors about revisions and final approval
of the proofs.\\[-13.5pt]

\item The Editorial Board of the Journal examines the article according to the established
reviewing procedure. If the authors receive their article for correction after reviewing, it does not
mean that the article is approved for publication. The corrected article should be sent to the
Editorial Board for the subsequent review and approval.\\[-13.5pt]

\item The decision on the article publication or its rejection is communicated to the authors. The
Editorial Board may also send the reviews on the submitted articles to the authors. Any
discussion upon the rejected articles is not possible.\\[-13.5pt]

\item The edited articles will be sent to the authors for proofread. The comments of the authors
to the edited text of the article should be sent to the Editorial Board as soon as possible.\\[-13.5pt]

\item The manuscript of the article should be presented electronically in the MS WORD (.doc or
.docx) or \LaTeX\ (.tex) formats, and additionally in the .pdf format. All documents
 may be sent
by e-mail or provided on a CD or diskette. A~hard copy submission is not necessary.\\[-13.5pt]

\item The recommended typesetting instructions for manuscript.

Pages parameters: format A4, portrait orientation, document margins (cm): left~--- 2.5, right~---
1.5, above~--- 2.0, below~--- 2.0, footer 1.3.

Text: font~---Times New Roman, font size~--- 14, paragraph indent~--- 0.5, line spacing~--- 1.5,
justified alignment.

The recommended manuscript size: not more than 15~pages of the specified format.
If the specified size exceeded, the editorial board is entitled to require the author
to reduce the manuscript.

Use only standard abbreviations. Avoid  abbreviations in the title and
abstract. The full term for which an abbreviation stands should precede
its first use in the text unless it is a standard unit of measurement.

All pages of the manuscript should be numbered.

The templates for the manuscript typesetting are presented on site: {\sf
http://www.ipiran.ru/journal/template.doc}.\\[-13.5pt]


%\def\leftkol{Requirements for manuscripts submitted to Journal
%``Informatics~and~Applications''}

\item The articles should enclose data both in \textbf{Russian and English}:
\begin{itemize}
\item title;\\[-13.5pt]
\item author's name and surname;\\[-13.5pt]
\item affiliation~--- organization, its address with ZIP code, city, country, and
official e-mail address;\\[-13.5pt]
\item data on authors according to the format: (see site)

{\sf http://www.ipiran.ru/journal/issues/2013\_07\_01/authors.asp}  and

{\sf  http://www.ipiran.ru/journal/issues/2013\_07\_01\_eng/authors.asp};\\[-13.5pt]

\pagebreak

\def\leftfootline{\small{\textbf{\thepage}
\hfill INFORMATIKA I EE PRIMENENIYA~--- INFORMATICS AND APPLICATIONS\ \ \ 2019\
\ \ volume~13\ \ \ issue\ 4}
}%
 \def\rightfootline{\small{INFORMATIKA I EE PRIMENENIYA~--- INFORMATICS AND APPLICATIONS\ \ \ 2019\ \ \ volume~13\ \ \ issue\ 4
\hfill \textbf{\thepage}}}


%\def\leftkol{Requirements for manuscripts submitted to Journal
%``Informatics~and~Applications''}

%\def\rightkol{Requirements for manuscripts submitted to Journal
%``Informatics~and~Applications''}



\item abstract (not less than 100 words) both in Russian and in English. Abstract is a short
summary of the article that can be published separately. The abstract is the
main source of information on the article and it could be included in leading information
systems and data bases. The abstract in English has to be an original text and should
not be an exact translation of the Russian one. Good English is required.
In abstracts, avoid references and formulae;\\[-13.5pt]
\item indexing is performed on the basis of keywords. The use of keywords from the
internationally accepted thematic Thesauri is recommended.

%\def\leftkol{Requirements for manuscripts submitted to Journal
%``Informatics~and~Applications''}

%\def\rightkol{Requirements for manuscripts submitted to Journal
%``Informatics~and~Applications''}

Important! Keywords must not be sentences;
\item Acknowledgments.
\end{itemize}

\item References. Russian references have to be presented both in English translation and Latin
transliteration (refer {\sf http://www.translit.net/ru/bgn/}).

Please take into account the following examples of Russian references appearance:

\noindent
\textbf{Article in journal:}

\Aue{Zhang, Z., and D.~Zhu}. 2008. Experimental research on the localized electrochemical
micromachining.
\textit{Rus. J.~Electrochem.}  44(8):926--930. {\sf doi:10.1134/S1023193508080077}.


\noindent
\textbf{Journal article in electronic format:}

\Aue{Swaminathan, V., E.~Lepkoswka-White, and B.\,P.~Rao}. 1999. Browsers or buyers in
cyberspace? An
investigation of electronic factors influencing electronic exchange. \textit{JCMC}
5(2). Available at: {\sf http://www.ascusc.org/jcmc/vol5/issue2/} (accessed April~28, 2011).




\noindent
\textbf{Article from the continuing publication (collection of works, proceedings):}

\Aue{Astakhov, M.\,V., and T.\,V.~Tagantsev}. 2006. Eksperimental'noe
issledovanie prochnosti soedineniy ``stal'--kompozit'' [Experimental study of
the strength of joints ``steel--composite'']. \textit{Trudy MGTU
``Matematicheskoe modelirovanie slozhnykh tekh\-ni\-che\-skikh sistem''}
[\textit{Bauman MSTU ``Mathematical Modeling of Complex Technical
Systems'' Proceedings}]. 593:125--130.

\def\leftfootline{\small{\textbf{\thepage}
\hfill INFORMATIKA I EE PRIMENENIYA~--- INFORMATICS AND APPLICATIONS\ \ \ 2019\
\ \ volume~13\ \ \ issue\ 4}
}%
 \def\rightfootline{\small{INFORMATIKA I EE PRIMENENIYA~--- INFORMATICS AND APPLICATIONS\ \ \ 2019\ \ \ volume~13\ \ \ issue\ 4
\hfill \textbf{\thepage}}}

\def\leftkol{Requirements for manuscripts submitted to Journal
``Informatics~and~Applications''}

\def\rightkol{Requirements for manuscripts submitted to Journal
``Informatics~and~Applications''}

\noindent
\textbf{Conference proceedings:}

\Aue{Usmanov, T.\,S., A.\,A.~Gusmanov, I.\,Z.~Mullagalin, R.\,Ju.~Muhametshina,
A.\,N.~Chervyakova, and
A.\,V.~Sveshnikov}. 2007. Osobennosti proektirovaniya razrabotki mestorozhdeniy
s primeneniem gidrorazryva
plasta [Features of the design of field development with the use of hydraulic fracturing].
\textit{Trudy 6-go
Mezhdu\-na\-rod\-no\-go Simpoziuma ``Novye resursosberegayushchie tekhnologii
nedropol'zovaniya i povysheniya
neftegazootdachi''} [\textit{6th  Symposium (International) ``New Energy Saving Subsoil
Technologies and
the Increasing of the Oil and Gas Impact'' Proceedings}]. Moscow. 267--272.


\noindent
\textbf{Books and other monographs:}




Lindorf, L.\,S., and L.\,G.~Mamikoniants, eds. 1972.
\textit{Ekspluatatsiya turbogeneratorov s neposredstvennym
okhlazhdeniem} [\textit{Operation of turbine generators with direct cooling}].
Moscow: Energy Publs. 352~p.


%\Aue{Latyshev, V.\,N.} 2009. \textit{Tribologiya rezaniya. Kn.~1: Frikcionnye prosessy
%pri rezanii metallov}
%[\textit{Tribology of cutting. Vol.~1: Frictional processes in metal cutting}]. Ivanovo: Ivanovskii
%State Univ. 108~p.


%\noindent
%\textbf{Unpublished material:}

%\Aue{Latypov, A.\,R., M.\,M.~Khasanov, and V.\,A.~Baikov}.
%2004. Geology and production (NGT GiD). Certificate on official registration of the computer
%program
%No.\,2004611198. (In Russian, unpubl.)

%\noindent
%\textbf{Internet-source:}

%APA Style. 2011. Available at: {\sf http://www.apastyle.org/apa-style-help.aspx} (accessed
%February~5, 2011).

%Pravila citirovaniya istochnikov [Rules for the citing of sources]. Available at: {\sf
%http://www.scribd.com/doc/1034528/} (accessed February~7, 2011).


\noindent
\textbf{Dissertation and Thesis:}

%\Aue{Semenov, V.\,I.}
%2003. Matematicheskoe modelirovanie plazmy v sisteme kompaktnyy tor. [Mathematical
%modeling of the plasma in the compact torus]. D.Sc.\ Diss. Moscow. 272~p.

\Aue{Kozhunova, O.\,S.} 2009. Tekhnologiya razrabotki semanticheskogo
slovarya informatsionnogo monitoringa [Technology of development of
semantic dictionary of information monitoring system]. PhD Thesis. Moscow: IPI RAN. 23~p.


\noindent
\textbf{State standards and patents:}

GOST 8.586.5-2005. 2007. Metodika vypolneniya izmereniy. Izmerenie raskhoda i~kolichestva
zhidkostey i gazov 
s~pomoshch'yu standartnykh suzhayushchikh ustroystv [Method of measurement.
Measurement of flow rate and volume of liquids and gases by means of orifice devices]. M.:
Standardinform
Publs. 10~p.

%\noindent
%\textbf{Patent:}

\Aue{Bolshakov, M.\,V., A.\,V.~Kulakov, A.\,N.~Lavrenov, and M.\,V.~Palkin}.
2006. Sposob orientirovaniya po krenu letatel'nogo
apparata s opti\-che\-skoy golovkoy
samonavedeniya [The way to orient on the roll of aircraft with optical homing head].
Patent RF No.\,2280590.

References in Latin transcription are presented in the original language.

References in the text are numbered according to the order of their
first appearance; the number is
placed in square brackets. All items from the reference list should be
cited.\\[-13.5pt]

\item Manuscripts and additional materials are not returned to Authors by the Editorial Board.\\[-13.5pt]

\item Submissions of files by e-mail must include:\\[-13.5pt]
\begin{itemize}
\item   the journal title and author's name in the ``Subject'' field; \\[-13.5pt]
\item   an article and additional materials have to be attached using the ``attach'' function;\\[-13.5pt]
\item   an electronic version of the article should contain the file with the text and a separate file
with figures.\\[-13.5pt]
\end{itemize}

\item ``Informatics and Applications'' journal is not a profit publication. There are no
charges for the authors as well as there are no royalties.\\[-13.5pt]
\end{enumerate}

\def\leftfootline{\small{\textbf{\thepage}
\hfill INFORMATIKA I EE PRIMENENIYA~--- INFORMATICS AND APPLICATIONS\ \ \ 2019\
\ \ volume~13\ \ \ issue\ 4}
}%
 \def\rightfootline{\small{INFORMATIKA I EE PRIMENENIYA~--- INFORMATICS AND APPLICATIONS\ \ \ 2019\ \ \ volume~13\ \ \ issue\ 4
\hfill \textbf{\thepage}}}

\def\leftkol{Requirements for manuscripts submitted to Journal
``Informatics~and~Applications''}

\def\rightkol{Requirements for manuscripts submitted to Journal
``Informatics~and~Applications''}


%\vspace*{5mm}


\begin{center}
\textbf{Editorial Board address:} \\

%ABOUT AUTHORS



FRC CSC RAS, 44, block~2, Vavilov Str., Moscow 119333, Russia\\[-10pt]

\

Ph.: +7\,(499)\,135\,86\,92,\ \ Fax: +7\,(495)\,930\,45\,05\\[-10pt]

\

 e-mail: {\sf rust@ipiran.ru} (to Prof.\ Rustem Seyful-Mulyukov)\\[-10pt]

\

 {\sf http://www.ipiran.ru/english/journal.asp}
\end{center}
 }
%\thispagestyle{myheadings}

\def\leftkol{Requirements for manuscripts submitted to Journal
``Informatics~and~Applications''}

\def\rightkol{Requirements for manuscripts submitted to Journal
``Informatics~and~Applications''}

\def\leftfootline{\small{\textbf{\thepage}
\hfill INFORMATIKA I EE PRIMENENIYA~--- INFORMATICS AND APPLICATIONS\ \ \ 2019\
\ \ volume~13\ \ \ issue\ 4}
}%
 \def\rightfootline{\small{INFORMATIKA I EE PRIMENENIYA~--- INFORMATICS AND APPLICATIONS\ \ \ 2019\ \ \ volume~13\ \ \ issue\ 4
\hfill \textbf{\thepage}}}

 \label{end\stat}

\newpage

%\vspace*{-60pt} {\small
{\baselineskip=9.1pt
\section*{Правила подготовки рукописей статей для публикации в журнале
<<Информатика и её применения>>}

\thispagestyle{empty}

 Журнал <<Информатика и её применения>> публикует
теоретические, обзорные и дискуссионные статьи, посвященные научным
исследованиям и разработкам в области информатики и ее приложений. Журнал
издается на русском языке. По специальному решению редколлегии отдельные статьи,
в виде исключения, могут печататься на английском языке.
Тематика журнала охватывает следующие направления:
\begin{itemize}
\item теоретические основы информатики; %\\[-13.5pt]
\item математические методы исследования сложных систем и процессов; %\\[-13.5pt]
\item информационные системы и сети; %\\[-13.5pt]
\item информационные технологии; %\\[-13.5pt]
\item архитектура и программное
обеспечение вычислительных комплексов и сетей.
\end{itemize}
\begin{enumerate}
\item В журнале печатаются результаты, ранее не
опубликованные и не предназначенные к одновременной публикации в других
изданиях. Публикация не должна нарушать закон об авторских правах. Направляя
свою рукопись в редакцию, авторы автоматически передают учредителям и
редколлегии неисключительные права на издание данной статьи на русском языке и
на ее распространение в России и за рубежом. При этом за авторами сохраняются
все права как собственников данной рукописи. В связи с этим авторами должно
быть представлено в редакцию письмо в следующей форме:
Соглашение о передаче права на публикацию:

\textit{<<Мы, нижеподписавшиеся, авторы рукописи <<$\qquad\qquad$>>, передаем
учредителям и редколлегии журнала <<Информатика и её применения>>
неисключительное право опубликовать данную рукопись статьи на русском языке как
в печатной, так и в электронной версиях журнала. Мы подтверждаем, что данная
публикация не нарушает авторского права других лиц или организаций. Подписи
авторов: (ф.\,и.\,о., дата, адрес)>>.}

Указанное соглашение может быть представлено 
как в бумажном виде, так и в виде отсканированной копии (с подписями авторов).


Редколлегия вправе запросить у авторов экспертное заключение о возможности
опубликования представленной статьи в открытой печати. %\\[-13.5pt]
\item Статья
подписывается всеми авторами. На отдельном листе представляются данные автора
(или всех авторов): фамилия, полные имя и отчество, телефон, факс, e-mail,
почтовый адрес. Если работа выполнена несколькими авторами, указывается фамилия
одного из них, ответственного за переписку с редакцией. %\\[-13.5pt]
\item Редакция журнала
осуществляет самостоятельную экспертизу присланных статей. Возвращение рукописи
на доработку не означает, что статья уже принята к печати. Доработанный вариант
с ответом на замечания рецензента необходимо прислать в редакцию. %\\[-13.5pt]
\item Решение
редакционной коллегии о принятии статьи к печати или ее отклонении сообщается
авторам. Редколлегия не обязуется направлять рецензию авторам отклоненной
статьи. %\\[-13.5pt]
\item Корректура статей высылается авторам для просмотра. Редакция
просит авторов присылать свои замечания в кратчайшие сроки. %\\[-13.5pt]
\item При
подготовке рукописи в MS Word рекомендуется использовать следующие настройки.
Параметры страницы: формат~--- А4; ориентация~--- книжная; поля (см): внутри~---
2,5, снаружи~--- 1,5, сверху~--- 2, снизу~--- 2, от края до нижнего
колонтитула~--- 1,3. Основной текст: стиль~--- <<Обычный>>: шрифт Times New
Roman, размер 14~пунктов, абзацный отступ~--- 0,5~см, 1,5 интервала,
выравнивание~--- по ширине. Рекомендуемый объем рукописи~--- не свыше
25~страниц указанного формата. Ознакомиться с шаблонами, содержащими примеры
оформления, можно по адресу в Интернете:
\textsf{http://www.ipiran.ru/journal/template.doc}.
\item К рукописи, предоставляемой в 2-х
экземплярах, обязательно прилагается электронная версия статьи (как правило, в
форматах MS WORD (.doc) или \LaTeX\ (.tex), а также~--- дополнительно~--- в
формате .pdf) на дискете, лазерном диске или по электронной почте. Сокращения
слов, кроме стандартных, не применяются. Все страницы рукописи должны быть
пронумерованы. %\\[-13.5pt]
\item Статья должна содержать следующую информацию на русском и
английском языках: название, Ф.И.О. авторов, места работы авторов и их
электронные адреса, подробные сведения об авторах, оформленные в соответствии с форматом, 
определяемым файлами {\sf http://www.ipiran.ru/journal/issues/2011\_05\_01/authors.asp} и 
{\sf http://www.ipiran.ru/journal/issues/2011\_01\_eng/authors.asp},
аннотация (не более 100~слов), ключевые слова. Ссылки на
литературу в тексте статьи нумеруются (в квадратных скобках) и располагаются в
порядке их первого упоминания. В~списке литературы не должно быть позиций, на которые нет ссылки в тексте статьи.
Все фамилии авторов, заглавия статей, названия
книг, конференций и~т.\,п.\ даются на языке оригинала, если этот язык
использует кириллический или латинский алфавит. %\\[-13.5pt]
\item Присланные в редакцию материалы авторам не возвращаются.
\item При отправке файлов по электронной
почте просим придерживаться следующих правил:
\begin{itemize}
\item указывать в поле subject (тема) название журнала и фамилию автора; %\\[-13.5pt]
\item использовать attach (присоединение); %\\[-13.5pt]
\item в случае больших объемов информации возможно
использование общеизвестных архиваторов (ZIP, RAR); %\\[-13.5pt]
\item в состав электронной версии статьи должны входить: файл, содержащий текст статьи, и файл(ы),
содержащий(е) иллюстрации. %\\[-13.5pt]
\end{itemize}
\item Журнал <<Информатика и её применения>> является некоммерческим изданием. 
Плата за публикацию с авторов не взимается, гонорар авторам не выплачивается.
\end{enumerate}
\thispagestyle{empty}
\textbf{Адрес редакции:} Москва 119333,
ул.~Вавилова, д.~44, корп.~2, ИПИ РАН\\
\hphantom{\textbf{Адрес редакции:} }Тел.: +7 (499) 135-86-92\ \
Факс:  +7 (495) 930-45-05\ \  E-mail:   rust@ipiran.ru }
}

%\include{ipi-ind}

%\tableofcontents

\end{document}

%\tableofcontents

%\end{document}

%\tableofcontents


\end{document}

\newcommand{\Ack}{\subsection*{\protect\large\bf Acknowledgments}}

\vphantom*{\int\limits_0^T}

{ \begin{center}  %fig1
 \vspace*{3pt}
    \mbox{%
 \epsfxsize=79mm 
 \epsfbox{gru-1.eps}
 }

\end{center}

\noindent
{{\figurename~1}\ \ \small{
Временные зависимости данные 
}}}

\vspace*{6pt}

\addtocounter{figure}{1}

$\acute{\mbox{о}}$

\linebreak