\def\stat{rumovskaya}

\def\tit{МЕТОД ВИЗУАЛИЗАЦИИ СНИЖЕНИЯ ИНТЕНСИВНОСТИ И~РАЗРЕШЕНИЯ 
КОНФЛИКТОВ В~ГИБРИДНЫХ ИНТЕЛЛЕКТУАЛЬНЫХ МНОГОАГЕНТНЫХ 
СИСТЕМАХ}

\def\titkol{Метод визуализации снижения интенсивности и~разрешения 
конфликтов в~ГиИМАС}

\def\aut{С.\,Б.~Румовская$^1$, И.\,А.~Кириков$^2$}

\def\autkol{С.\,Б.~Румовская, И.\,А.~Кириков}

\titel{\tit}{\aut}{\autkol}{\titkol}

\index{Румовская С.\,Б.}
\index{Кириков И.\,А.}
\index{Rumovskaya S.\,B.}
\index{Kirikov I.\,A.}


%{\renewcommand{\thefootnote}{\fnsymbol{footnote}} \footnotetext[1]
%{Работа выполнена при поддержке Министерства науки и~высшего образования Российской Федерации (проект 
%075-15-2020-799).}}


\renewcommand{\thefootnote}{\arabic{footnote}}
\footnotetext[1]{Калининградский филиал Федерального исследовательского центра <<Информатика и~управ\-ле\-ние>> 
Российской академии наук, \mbox{sophiyabr@gmail.com}}
\footnotetext[2]{Калининградский филиал Федерального исследовательского центра <<Информатика и~управ\-ле\-ние>> 
Российской академии наук, \mbox{baltbipiran@mail.ru}}

%\vspace*{-6pt}

  
  \Abst{Многие практические проблемы диктуют необходимость коллективного решения, 
обеспечивающего плюрализм мнений, интеграцию частных точек зрения и снижение числа 
ошибок. Моделирование работы таких коллективов специалистов гибридными 
интеллектуальными многоагентными системами (\mbox{ГиИМАС}), учитывая 
особенности их групповой динамики, позволит повысить качество и~эффективность 
решения, а~также всесторонне рассмотреть проблему и~процесс ее преодоления, в~том числе 
с~по\-мощью визуализации конфликтов и~процессов управления ими, предоставляя новую 
информацию по разрешению конфликтов и~в~системе, и~в~реальном коллективе 
специалистов. Работа посвящена разработке метода визуализации процессов разрешения 
конфликтов в~рамках \mbox{ГиИМАС} с~проб\-лем\-но- и~про\-цес\-сно-ори\-ен\-ти\-ро\-ван\-ны\-ми конструктивными конфликтами.}
  
  \KW{коллектив специалистов; конфликт агентов; визуализация разрешения конфликта}
  
  \DOI{10.14357/19922264220212}
  
\vspace*{-3pt}


\vskip 10pt plus 9pt minus 6pt

\thispagestyle{headings}

\begin{multicols}{2}

\label{st\stat}
  
\section{Введение}

  В~[1--6] предложены \mbox{ГиИМАС}, 
которые релевантны групповой динамике коллектива специалистов~[7--9], 
решающего проблему, и моделируют проб\-лем\-но-
и~про\-цес\-сно-ори\-ен\-ти\-ро\-ван\-ные конфликты~\cite{1-kir}. Это конструктивные 
инструментальные конфликты~[10], актуализирующие плюрализм мнений 
относительно проблемы и способствующие поиску оптимальных способов ее 
решения. Такие конфликты идентифицируются~\cite{2-kir}, 
интенсифицируются~\cite{3-kir} и разрешаются~\cite{4-kir} в рамках 
\mbox{ГиИМАС}, повышая их 
релевантность работе реальных малых коллективов специалистов. Также 
в~\cite{5-kir, 6-kir} разработаны методы визуализации возникающих между 
агентами конфликтов и процесса их интенсификации в~\mbox{ГиИМАС}, что 
повышает прозрачность работы системы для пользователя. В~\cite{4-kir} 
описан один из методов управления конфликтами~--- разрешение 
конструктивных инструментальных конфликтов, включающий такие стратегии 
разрешения противоречий (СРП)~\cite{11-kir}, как переговоры~--- обмен 
знаниями и информацией о целях между агентами для достижения соглашения; 
делегирование~--- привлечение третьей стороны (агента) с более развитой базой 
знаний и возможностями, но не способной напрямую взаимодействовать 
с~другими агентами; голосование~--- агенты голосуют по всем предварительно 
предложенным ими решениям; самомодификация~--- агент при возникновении 
конфликта вместо взаимодействия с целью выработки соглашения меняет свое 
поведение; игнорирование~--- отказ от разрешения конфликта ввиду его низкой 
интенсивности.
  
  Работ, содержащих визуализацию конкретных конфликтов, найдено было 
мало~\cite{5-kir}, и все они отоб\-ра\-жа\-ют динамику деструктивных 
макроконфликтов~\cite{12-kir} без деталей взаимодействия участников, 
а~работы с~визуализацией динамики конфликта в~малых группах специалистов, 
решающих проб\-ле\-му (в~том чис\-ле снижения интенсивности и~разрешения 
конфликта), отсутствуют. 
  
  Цель настоящей работы~--- разработка метода визуализации процесса 
снижения интенсивности и~разрешения конфликтов на базе предложенного 
метода их идентификации~\cite{2-kir}, функции управления~\cite{3-kir} 
и~метода разрешения~\cite{4-kir} в~рамках представленной в~\cite{1-kir} 
модели \mbox{ГиИМАС}, что сделает\linebreak возникшие противоречия контрастными, 
предо\-став\-ляя детальную визуализацию разрешения конфликта, явно 
отображающую снижение интенсивности конфликта между каждой парой 
\mbox{конфликтующих} агентов~--- тип конфликта, напряженность между 
участниками, их изменение и~применяемую стратегию разрешения конфликта 
или ее отсутствие. 

\section{Разрешение конфликтов между~агентами как часть 
функции~агента-фасилитатора <<управление конфликтом>>}

  В~\cite{3-kir} задана функция агента-фа\-си\-ли\-та\-то\-ра (АФ)  
<<управ\-ле\-ние конфликтом>>, в рамках которой вы\-чис\-ля\-ет\-ся среднее 
арифметическое показателей взаимозависимости целей 
агентов~$\mathrm{gd}^{\mathrm{himas}}$, затем запускается 
функция идентификации конфликтов, анализирующая решения, предложенные 
агентами-специалистами (АС), и формирующая матрицу конфликтов 
$\mathbf{CNF}$ между парами агентов, элемент которой представляет собой 
кортеж~((3) из~\cite{3-kir}):
  \begin{multline}
  \mathrm{cnf}_{ij\,\mathrm{cnft}}={}\\
  {}=\left\langle \mathrm{ag}_i, \mathrm{ag}_j, 
\mathrm{cnfin}, \mathrm{cnft}, \mathrm{ACT}_i^{\mathrm{agcr}}, 
\mathrm{ACT}_j^{\mathrm{agcr}}\right\rangle\,.
\label{e1-kir}
  \end{multline}
Здесь $\mathrm{ag}_i$ и~$\mathrm{ag}_j$~--- это аген\-ты-субъ\-ек\-ты конфликта,  
$i,j\hm\in \mathbf{N}$, $i,j\hm\in [1,n]$, $i\not=j$; $\mathrm{cnfin}\hm\in [0,1]$~--- 
напряженность конфликта; $\mathrm{cnft}$~--- <<тип конфликта>>, \mbox{определяется} на 
множестве $\mathrm{CNFT}\hm = \left\{ \mathrm{cnft}_{\mathrm{prb}}\right.\hm =$\;<<проб\-лем\-но-ори\-ен\-ти\-ро\-ван\-ный>>, 
$\mathrm{cnft}_{\mathrm{prc}}$\;=\;<<про\-цес\-сно-ори\-ен\-ти\-ро\-ван\-ный>>$\left.\right\}$; 
$\mathrm{ACT}_i^{\mathrm{agcr}}$ и~$\mathrm{ACT}_j^{\mathrm{agcr}}$~--- множества допустимых действий агентов 
$\mathrm{ag}_i$ и~$\mathrm{ag}_j$ соответственно по разрешению противоречий,  
$\mathrm{ACT}_i^{\mathrm{agcr}}\hm\subseteq \mathrm{ACT}_i^{\mathrm{ag}}$, 
$\mathrm{ACT}_j^{\mathrm{agcr}}\hm\subseteq \mathrm{ACT}_j^{\mathrm{ag}}$, 
$\mathrm{ACT}_i^{\mathrm{agcr}}, \mathrm{ACT}_j^{\mathrm{agcr}} \hm\subseteq 
\mathrm{ACT}^{\mathrm{agcr}}$, 
  причем $\mathrm{ACT}_i^{\mathrm{ag}}$ и~$\mathrm{ACT}_j^{\mathrm{ag}}$~--- множества действий агентов 
$\mathrm{ag}_i$ и~$\mathrm{ag}_j$ соответственно, 
а~$\mathrm{ACT}^{\mathrm{agcr}}$~--- упорядоченное по отношению предпочтения 
$\overset{\mathrm{prf}}{\prec}$ множество допустимых стратегий АС по 
разрешению противоречий, включающее стратегии переговоров, 
делегирования, голосования, самомодификации и игнорирования 
соответственно:
\begin{multline*}
\mathrm{ACT}^{\mathrm{agcr}} ={}\\
{}=\left( \!\left\{ 
\mathrm{act}^{\mathrm{agcr}}_{\mathrm{ig}},  
\mathrm{act}^{\mathrm{agcr}}_{\mathrm{sm}}, 
\mathrm{act}^{\mathrm{agcr}}_{\mathrm{vot}}, 
\mathrm{act}^{\mathrm{agcr}}_{\mathrm{del}}, 
\mathrm{act}^{\mathrm{agcr}}_{\mathrm{neg}}\right\},\overset{\mathrm{prf}} 
{\prec}\right).
\end{multline*}
  
  После формирования матрицы $\mathbf{CNF}$ вычисляется общий 
показатель напряженности конфликта в \mbox{ГиИМАС} 
$\mathrm{cnf}^{\mathrm{himas}}$~\cite{3-kir}. На этом же этапе, если 
пользователь установил перед началом работы \mbox{ГиИМАС} 
необходимость визуализации работы коллектива агентов, запускается метод 
визуализации конфликта (МВК)~\cite{5-kir} и отображаются 
$\mathrm{gd}^{\mathrm{himas}}$ и~$\mathrm{cnf}^{\mathrm{himas}}$ 
с~пороговыми значениями (по умолчанию~0 и~0,5 соответственно). Затем 
в~зависимости от значений $\mathrm{gd}^{\mathrm{himas}}$ 
и~$\mathrm{cnf}^{\mathrm{himas}}$ выполняется функция <<стимуляция 
конфликтов>> или <<разрешение конфликтов>>, а~также визуализация этих 
процессов (при необходимости). Если в результате выполнения одной из этих 
функций активируется признак завершения работы \mbox{ГиИМАС}, то 
инициализируется процедура окончания работы системы.
  
  Последовательность шагов функции <<управление конфликтом>> 
(ПШФУК) АФ представлена в~\cite{3-kir}. Визуализация стимуляции 
конфликта между агентами как модификация \mbox{ПШФУК}, дополненная 
запуском МВК~\cite{5-kir} на базе матрицы $\mathbf{CNF}$, а~также 
визуализацией $\mathrm{gd}^{\mathrm{himas}}$ 
и~$\mathrm{cnf}^{\mathrm{himas}}$ с их пороговыми значениями, описана 
в~\cite{6-kir}. Алгоритм снижения интенсивности и разрешения конфликтов 
(АСИРК) в~\mbox{ГиИМАС} предлагается в~\cite{4-kir}. Метод визуализации 
разрешения конфликтов (МВРК) по всем парам конфликтующих агентов 
работает параллельно АСИРК. Рассмотрим подробнее предлагаемый метод 
МВРК в~\mbox{ГиИМАС}.
  
\section{Метод визуализации снижения интенсивности 
и~разрешения конфликтов}

  Если пользователь установил перед запуском работы \mbox{ГиИМАС} флаг 
<<необходимости визуализации динамики возможного конфликта>>, то 
в~рамках работы системы запускается не функция <<управление 
конфликтом>>, а~ее модификация, включающая визуализацию и запуск 
МВК~\cite{6-kir}. По аналогии с~отоб\-ра\-же\-ни\-ем вероятности перехода 
конфликта с~одного уровня на другой в~работе~\cite{13-kir}, разрешенные 
конфликты и~стратегия разрешения противоречий, выбранная для применения 
между парой агентов на очередном шаге работы алгоритма АСИРК, 
визуализируются с помощью матрицы. В~связи с~этим для последующей 
работы МВРК перед запуском модификации ПШФУК необходимо 
сформировать матрицу~$\mathbf{V}_{m\times m}$ размерности~$m$ (равна 
мощности множества коллектива агентов), по диагонали которой стоят~0 
($v_{ii}\hm=0$), а~на остальных позициях~---~1 ($\forall\,i\not=j, i,j\hm\in [1,m]$, 
$v_{ij}\hm=1$), и~установить: $k\hm=1$; $\mathbf{V}_{m\times m}$. Рассмотрим последовательность шагов МВРК. 
  
  \textbf{Первый шаг} реализуется после запуска функции <<разрешение 
конфликтов>> на паре агентов $\mathrm{ag}_i$, $\mathrm{ag}_j$ (субъектов 
конфликта). Его исполнение связано с верхней границей размерности малого 
коллектива специалистов относительно успешного руководства группой~--- 
соответствует <<магическому числу>> Дж.~Миллера ($7 \hm\pm 2$), так как 
при численности свыше~10~человек возрастают число подгрупп и вероятность 
противостояния лицу, при\-ни\-ма\-юще\-му решения, осложняется координация. 
Однако, чтобы учесть все возможные варианты и улучшить восприятие 
пользователем визуализации смены страте-\linebreak гий разрешения конфликтов между 
агентами и~наличие возможных подгрупп, проверяем условие\linebreak <<$m\hm> 
10$>>: если принимает значение <<истина>>, то запускаем функцию 
<<выделение подгрупп конфликтующих агентов>> (ФВПКА), иначе переходим 
ко\linebreak второму шагу. Для того чтобы выделить возможные подгруппы  
не\-конф\-лик\-ту\-ющих/сла\-бо\-конф\-лик\-ту\-ющих между собой агентов, 
воспользуемся \mbox{алгоритмом} поиска сообществ IS$^2$~\cite{14-kir, 15-kir}, 
приведенном в~обзоре~\cite{16-kir} и~применяющемся к~взвешенным 
неориентированным графам. IS$^2$ учитывает возможность принадлежности 
вершины нескольким сообществам и комбинирует алгоритмы 
последовательного обхода (Iterative Scan, IS) и удаления по рангу (Rank 
Removal, RaRe). Если перед началом работы ГиИМАС пользователь установил 
флаг <<выявление подгрупп>>, то ФВПКА запускается при любом~$m$ 
и~включает в себя следующие шаги: 
  \begin{itemize}
\item формирование матрицы $\mathbf{PPK}$ на базе матриц $\mathbf{CP}$ 
и~$\mathbf{CPR}$, полученных из $\mathbf{CNF}$ в процессе работы МВК 
($\mathrm{cp}_{ij}$ описывает величину напряженности  
проб\-лем\-но-ори\-ен\-ти\-ро\-ван\-но\-го конфликта между агентами, 
а~$\mathrm{cpr}_{ij}$~--- про\-цес\-сно-ори\-ен\-ти\-ро\-ван\-но\-го конфликта):
$$
\mathrm{ppk}_{ij}= \begin{cases}
0\,, &\hspace*{-30mm}\mbox{если } i=j\\
& \hspace*{-38mm}\mbox{или } \left(0{,}5\left( \mathrm{cp}^2_{ij}+\mathrm{cpr}^2_{ij}\right)\right)^{0{,}5} \!>\! \mathrm{cnfin}^{\mathrm{htr}}\,;\\
10\,000, &\hspace*{-25mm}\mbox{если } \mathrm{cpr}_{ij}=\mathrm{cp}_{ij}=0\,;\\
\left( 0{,}5\left( 
\mathrm{cp}^2_{ij}+\mathrm{cpr}_{ij}^{2}\right)\right)^{-0{,}5} & \hspace*{-1mm}\mbox{в\ 
противном}\\
&\hspace*{-1mm}\mbox{случае,}
\end{cases}
$$
где $\mathrm{cnfin}^{\mathrm{htr}}$~--- верхний порог интенсивности 
конфликта (по умолчанию~0,5). 
  
  Чем выше напряженность конфликта между агентами, тем меньше вес ребра. 
Если вес равен нулю, то ребро отсутствует, в частности если между агентами 
имеет место сильный конфликт (напряженность выше порогового значения);
\item запуск алгоритма RaRe на матрице $\mathbf{PPK}$: 
\begin{enumerate}[(1)]
\item подсчет рангов 
всех вершин (возможны разные подходы~\cite{7-kir, 8-kir}, возьмем за меру 
степень вершины);
\item удаление всех высокоранговых вершин с~\mbox{целью} 
получения ядер (размерность по умолчанию~---~2, можно корректировать) 
будущих сообществ; 
\item последовательное добавление каждой удаленной 
вершины к ядрам~--- если добавление приводит к~увеличению весовой 
функции~\cite{7-kir} ($W \hm= W(C)/(W(C) \hm+ W_{\mathrm{out}}(C))$, где 
$W(C)$~--- сумма весов ребер внутри сообщества~$C$; 
$W_{\mathrm{out}}(C)$~--- сумма весов ребер вне сообщества~$C$), то оставляем 
вершину в сообществе. Вершина может добавляться к~нескольким ядрам, 
образуя пересекающиеся сообщества;
\end{enumerate}
\item запуск алгоритма IS для уточнения результата, полученного от RaRe: 
выбирается произвольная вершина $\mathrm{ppk}_i$ в~качестве начального сообщества, 
к~которой на каждом шаге добавляются другие вершины графа до тех пор, 
пока улучшается значение весовой функции~$W$. Однако добавляемые 
вершины выбираются не из всего графа, а~только из сообщества, полученного 
с~по\-мощью RaRe и~содержащего вершину $\mathrm{ppk}_i$, а~также из соседних 
сообществ;
\item упорядочение строк и столбцов матрицы $\mathbf{V}_{m\times m}$ 
в~соответствии с~полученным разбиением~$C$ коллектива агентов на 
сообщества.
\end{itemize}
  
  \textbf{Второй шаг}~--- отображение последней сохраненной визуализации 
(уклад\-ки графа) конфликта и~под ней отображение матрицы 
$\mathbf{V}_{m\times m}$ как таб\-ли\-цы  (рис.~1). 
  
  На рис.~1 $i$-е строка и столбец таб\-ли\-цы, отоб\-ра\-жа\-ющей мат\-ри\-цу, 
подписаны значением $\mathrm{id}_i^{\mathrm{ag}}$. На примере коллектива 
агентов, решающего задачу диагностики рака поджелудочной железы, 
$\mathrm{id}_{\mathrm{id}}^{\mathrm{ag}}=$\;$\{$<<АХ>>, <<АОНЛ>>, 
<<АЛПР-Т>>, <<АСУЗИ>>, <<АВЛД>>, <<АСЛД>>$\}$. По диагонали 
отображения матрицы располагаются черные квадраты (соответствуют 
$v_{ii}\hm=0$ в~$\mathbf{V}_{m\times m}$), так как сам с собой агент не 
конфликтует и эта область не интересна, а~остальные элементы 
матрицы~$\mathbf{V}_{m\times m}$ отображаются белыми квадратами 
($v_{ij}\hm=1$, $i\not= j$).
  
  На графе конфликта (см.\ рис.~1) толщиной и~цветом линии (от свет\-ло-се\-ро\-го 
до черного) отображается величина среднего квадратического напряженностей 
конфликтов между агентами (сплошной линией, если превалирует  
проб\-лем\-но-ори\-ен\-ти\-ро\-ван\-ный конфликт; штриховой~--- если  
про\-цес\-сно-ори\-ен\-ти\-ро\-ван\-ный). Каждая вершина подписана 
идентификатором соответствующего агента.
   Слева от графа отображены $\mathrm{gd}^{\mathrm{himas}}$ 
   и~$\mathrm{cnf}^{\mathrm{himas}}$~\cite{6-kir}~--- их пороги, значения, цвет 
   и~символы (вычисляются в~начале управ\-ле\-ния конфликтом~\cite{6-kir}). Для 
$\mathrm{gd}^{\mathrm{himas}}$: \raisebox{-1pt}[0pt][0pt]{\mbox{%
     \epsfxsize=3.8mm 
    \epsfbox{rum-t-1.eps}
     }}  темно-серого цвета, если выше нуля, и \raisebox{-
1pt}[0pt][0pt]{\mbox{%
     \epsfxsize=3.8mm 
    \epsfbox{rum-t-2.eps}
     }} свет\-ло-се\-ро\-го цвета, если ниже или равно нулю. Для 
$\mathrm{cnf}^{\mathrm{himas}}$: \raisebox{-1pt}[0pt][0pt]{\mbox{%
     \epsfxsize=3.8mm 
    \epsfbox{rum-t-3.eps}
     }}  тем\-но-се\-ро\-го цвета, если меньше порогового значения 
(определяется в ходе тестирования системы, по умолчанию равно~0,5), 
\raisebox{-1pt}[0pt][0pt]{\mbox{%
     \epsfxsize=3.8mm 
    \epsfbox{rum-t-4.eps}
     }}  свет\-ло-се\-ро\-го цвета, если равно порогу, и \raisebox{-
1pt}[0pt][0pt]{\mbox{%
     \epsfxsize=3.8mm 
    \epsfbox{rum-t-5.eps}
     }}, если больше порогового значения. Под 
$\mathrm{gd}^{\mathrm{himas}}$ и~$\mathrm{cnf}^{\mathrm{himas}}$ 
расположена неактивная пустая иконка стимуляции  
конфликта~\cite{3-kir, 6-kir}.

\end{multicols}

\begin{figure*} %fig1
\vspace*{1pt}
  \begin{center}  
    \mbox{%
\epsfxsize=133.521mm
\epsfbox{rum-1.eps}
}
\end{center}
\vspace*{-9pt}
\Caption{Визуализация разрешения конфликта между агентами: АХ~--- хирург; АОНЛ~--- 
онколог по нехирургическому лечению; АЛПР-Т~--- лицо, принимающее решение 
(терапевт); АСУЗИ~--- специалист по ультразвуковому исследованию; АВЛД~--- врач 
лабораторной диагностики; АСЛД~--- специалист по лучевой диагностике}
\end{figure*} 

\begin{multicols}{2}
  
  \textbf{Третий шаг.} После того как АСИРК~\cite{4-kir} запросит у конфликтующих 
агентов $\mathrm{ag}_i$ и~$\mathrm{ag}_j$ множества реализуемых ими СРП 
$\mathrm{ACT}_i^{\mathrm{agcr}}$ и $\mathrm{ACT}_j^{\mathrm{agcr}}$, 
сформирует на их базе упорядоченное множество (список) 
$\mathrm{ACT}_{ijc}^{\mathrm{agcr}}$ СРП между данной парой агентов 
и~выберет стратегию по правилу из~\cite{4-kir}, элементу~$v_{ij}$ 
матрицы~$\mathbf{V}_{m\times m}$, который соответствует паре 
конфликтующих агентов $\mathrm{ag}_i$ и~$\mathrm{ag}_j$, присваивается 
значение, соответствующее ситуации:
  \begin{itemize}
\item если $\mathrm{ACT}_{ijc}^{\mathrm{agcr}}=\varnothing$, т.\,е.\ конфликт не может быть разрешен и функция 
завершает свою ра-\linebreak боту, то $v_{ij}\hm=2$, а~в~соответствующей ячейке\linebreak 
таблицы белый квадрат заменяется на икон-\linebreak ку~<< \raisebox{-1pt}[0pt][0pt]{\mbox{%
 \epsfxsize=3.9mm 
  \epsfbox{rum-t-7.eps}
   }}>>; 
   
\item если $\mathrm{ACT}_{ijc}^{\mathrm{agcr}}\not= \varnothing$, то 
в~зависимости от запущенной СРП на паре агентов: 
$$
v_{ij}=\begin{cases}
3 (\mbox{<<}\raisebox{-1pt}[0pt][0pt]
{\mbox{%
   \epsfxsize=3.9mm 
  \epsfbox{rum-t-8.eps}
   }}\mbox{>>}) & \mbox{--- переговоры};\\
   4 (\mbox{<<}\raisebox{-1pt}[0pt][0pt]{\mbox{%
   \epsfxsize=3.9mm 
  \epsfbox{rum-t-9.eps}
   }}\mbox{>>}) & \mbox{--- делегирование};\\
    5 (\mbox{<<}\raisebox{-1pt}[0pt][0pt]{\mbox{%
   \epsfxsize=3.9mm 
  \epsfbox{rum-t-10.eps}
   }}\mbox{>>}) & \mbox{--- голосование}; \\
   6 (\mbox{<<}\raisebox{-1pt}[0pt][0pt]{\mbox{%
   \epsfxsize=3.9mm 
  \epsfbox{rum-t-11.eps}
   }}\mbox{>>}) & \mbox{--- самомодификация}; \\
   7 (\mbox{<<}\raisebox{-1pt}[0pt][0pt]{\mbox{%
   \epsfxsize=3.3mm 
  \epsfbox{rum-t-12.eps}
   }}\mbox{>>}) & \mbox{--- игнорирование}. 
   \end{cases}
   $$
   \end{itemize}
  
  Таким образом, множество
  $$
  \mathrm{ve}= \{0, 1, 2, 3, 4, 5, 6, 7\}
  $$ 
  биективно отображается 
на множество 
$$
\mathrm{sve}=\left\{ \raisebox{-1pt}[0pt][0pt]{\mbox{%
     \epsfxsize=3.3mm 
    \epsfbox{rum-t-13.eps}
     }},  \raisebox{-1pt}[0pt][0pt]{\mbox{%
        \epsfxsize=3.3mm 
       \epsfbox{rum-t-14.eps}
        }}, \raisebox{-1pt}[0pt][0pt]{\mbox{%
      \epsfxsize=3.9mm 
     \epsfbox{rum-t-7.eps}
      }}, \raisebox{-1pt}[0pt][0pt]{\mbox{%
      \epsfxsize=3.9mm 
     \epsfbox{rum-t-8.eps}
      }}, \raisebox{-1pt}[0pt][0pt]{\mbox{%
      \epsfxsize=3.9mm 
     \epsfbox{rum-t-9.eps}
      }} , \raisebox{-1pt}[0pt][0pt]{\mbox{%
     \epsfxsize=3.9mm 
    \epsfbox{rum-t-10.eps}
     }}, \raisebox{-1pt}[0pt][0pt]{\mbox{%
      \epsfxsize=3.9mm 
     \epsfbox{rum-t-11.eps}
      }}, \raisebox{-1pt}[0pt][0pt]{\mbox{%
      \epsfxsize=3.3mm 
     \epsfbox{rum-t-12.eps}
      }}\right\}.
      $$
      
       \begin{figure*}[b] %fig2
  \vspace*{2pt}
  \begin{center}  
    \mbox{%
\epsfxsize=133.973mm
\epsfbox{rum-2.eps}
}
\end{center}
\vspace*{-9pt}
  \Caption{Визуализация промежуточного этапа разрешения конфликта между агентами}
   \end{figure*}
  
 \textbf{Четвертый шаг}~--- сохранить визуализацию, полученную на $k$-м цикле 
управления конфликтом: 
  \begin{itemize}
\item сохранить в $k$-й элемент $\mathbf{VRK}_{1\times K}$ 
матрицу~$\mathbf{V}_{m\times m}$ ($\mathrm{vrk}_{1k}\hm= \mathbf{V}_{m\times m}$); 
\item сохранить укладку графа конфликтующих агентов, полученную 
в~результате работы МВК, как $k$-й элемент $\mathbf{VK}_{1\times K}$; 
\item сохранить значение $\mathrm{gd}^{\mathrm{himas}}$ 
и~$\mathrm{cnf}^{\mathrm{himas}}$ как очередной $k$-й элемент множеств 
$\mathrm{VGD}$ и~$\mathrm{VCNF}$ соответственно и установить $k\hm = 
k\hm+1$. 
\end{itemize}
  
  Матрицы $\mathbf{VRK}_{1\times K}$ и~$\mathbf{VK}_{1\times K}$, 
а~также множества $\mathrm{VGD}$ и~$\mathrm{VCNF}$ позволят 
пользователю при необходимости просмотреть весь визуальный ряд\linebreak\vspace*{-10pt}

\columnbreak

\noindent
 конфликта 
между агентами, смену напряженности и СРП в динамике или пошагово.
  
  Если выбрана стратегия <<игнорирование>>, то конфликт считается 
разрешенным, АСИРК и~\mbox{МВРК} завершают работу, иначе АФ ожидает  
со\-об\-ще\-ний-ре\-ше\-ний от АС, которые они выработают после применения 
соответствующей стратегии. Получив такие сообщения, АФ вновь 
идентифицирует конфликт между парой агентов~\cite{4-kir} согласно АСИРК 
и~запускает МВРК.
  
  Пример визуализации промежуточного этапа разрешения конфликта 
представлен на рис.~2. В~сравнении с рис.~1 видно, что конфликты между 
агентами АВЛД и АСЛД разрешены и~на данном этапе для агентов АВЛД 
и~АСУЗИ выбрана стратегия переговоров.
  
 

\section{Заключение}
  
  Моделирование и визуализация процессов разрешения конфликтов агентов 
избавляет пользователя от необходимости ручного анализа и~выбора\linebreak 
альтернативы из предлагаемого множества вариантов, тем самым повышая 
эффективность работы \mbox{ГиИМАС}. В~работе предложен новый метод 
визуализации разрешения конфликтов, ба\-зи\-ру\-ющий\-ся на алгоритме 
АСИРК~\cite{4-kir}, методе визуализации конфликта~\cite{5-kir} и алгоритме 
поиска сообществ IS$^2$~\cite{14-kir, 15-kir}. Метод визуализации разрешения конфликтов в~\mbox{ГиИМАС} 
интегрирует пиктографическую, визуальную (графы и таблицы) и численную 
информации, детально отображая процессы снижения интенсивности 
и~разрешения конфликта агентов. Данный метод предоставляет возможность 
отследить изменение напряженности между агентами, используемые стратегии 
разрешения конфликтов, а также наличие подгрупп в коллективе агентов.
  
{\small\frenchspacing
 {%\baselineskip=10.8pt
 %\addcontentsline{toc}{section}{References}
 \begin{thebibliography}{99}
\bibitem{1-kir}
\Au{Листопад С.\,В., Кириков~И.\,А.} Моделирование конфликтов агентов в гибридных 
интеллектуальных многоагентных сис\-те\-мах~// Сис\-те\-мы и средства информатики, 2019. 
Т.~29. №\,3. С.~139--148. doi: 10.14357/ 08696527190312.
\bibitem{2-kir}
\Au{Листопад С.\,В., Кириков~И.\,А.} Метод идентификации конфликтов агентов 
в~гибридных интеллектуальных многоагентных сис\-те\-мах~// Сис\-те\-мы и средства 
информатики, 2020. Т.~30. №\,1. С.~56--65. doi: 10.14357/ 08696527200105.

\bibitem{5-kir} %3
\Au{Румовская С.\,Б., Кириков~И.\,А.} Метод визуального представления конфликтов 
в~гибридных интеллектуальных многоагентных сис\-те\-мах~// Информатика и её 
применения, 2020. Т.~14. Вып.~4. С.~77--82. doi: 10.14357/19922264200411.

\bibitem{3-kir} %4
\Au{Листопад С.\,В., Кириков~И.\,А.} Стимуляция конфликтов агентов в гибридных 
интеллектуальных многоагентных сис\-те\-мах~// Сис\-те\-мы и средства информатики, 2021. 
Т.~31. №\,2. С.~47--58. doi: 10.14357/ 08696527210205.


\bibitem{6-kir} %5
\Au{Румовская С.\,Б., Кириков~И.\,А.} Метод визуализации стимуляции конфликтов в 
гибридных интеллектуальных многоагентных сис\-те\-мах~// Информатика и~её применения, 
2021. Т.~15. Вып.~3. С.~75--82. doi: 10.14357/19922264210310.

\bibitem{4-kir} %6
\Au{Листопад С.\,В., Кириков~И.\,А.} Разрешение конфликтов в гибридных 
интеллектуальных многоагентных сис\-те\-мах~// Информатика и её применения, 2022. 
Т.~16. Вып.~1. С.~54--60.

\bibitem{9-kir}  %7
\Au{Shaw M.} Group dynamics: The psychology of small group behavior.~--- New York, NY, 
USA: McGraw-Hill, 1981. 531~p.

\bibitem{7-kir} %8
\Au{Андреева Г.\,М.} Социальная психология.~--- М.: Аспект-пресс, 2009. 393~с.
\bibitem{8-kir} %9
\Au{Brown R., Pehrson~S.} Group processes: Dynamics with and between groups.~--- 3rd ed.~--- 
Oxford: Wiley-Blackwell, 2019. 344~p. doi: 10.1002/9781118719244.

\bibitem{10-kir}
\Au{Емельянов С.\,М.} Конфликтология.~--- 4-е изд.~--- М.: Юрайт, 2018. 322~с.
\bibitem{11-kir}
\Au{Behfar K., Peterson~R., Mannix~E., Trochim~W.} The critical role of conflict resolution in 
teams: A~close look at the links between conflict type, conflict management strategies, and team 
outcomes~// J.~Appl. Psychol., 2008. Vol.~93. No.\,1. P.~170--188. doi:  
10.1037/0021-9010.93.1.170.
\bibitem{12-kir}
\Au{Анцупов А.\,Я., Баклановский~С.\,В.} Конфликтология в~схемах и~комментариях.~--- 2-е изд.~--- СПб.: Питер, 2009. 304~с.
\bibitem{13-kir}
\Au{Cusack J.\,J., Bradfer-Lawrence~T., Baynham-Herd~Z., \textit{et al.}} Measuring the intensity 
of conflicts in conservation~// Conserv. Lett., 2021. Vol.~14. Iss.~3. Art. e12783. 11~p. doi: 
10.1111/conl.12783.
\bibitem{14-kir}
\Au{Baumes J., Goldberg~M.\,K., Krishnamoorthy~M.\,S., Magdon-Ismail~M., Preston~N.} 
Finding communities by clustering a graph into overlapping subgraphs~// Conference (International)
 on Applied Computing Proceedings.~--- IADIS Press, 2005. Vol.~1. P.~97--104. 
\bibitem{15-kir}
\Au{Baumes J., Goldberg~M., Magdon-Ismail~M.} Efficient identification of overlapping 
communities~// Intelligence and security informatics~/ Eds. P.\,B.~Kantor, G.~Muresan, F.\,S.~Roberts, \textit{et al.}~--- 
Lecture notes in computer science ser.~--- Berlin--Heidelberg: Springer, 2005. 
Vol.~3495. P.~27--36. doi: 10.1007/11427995\_3. 
\bibitem{16-kir}
\Au{Вирцева Н.\,С., Вишняков~И.\,Э., Иванов~И.\,П.} Способы выделения сообществ 
с~определенными типами отношений в графах на основе биллинговой информации~// 
Вестник МГТУ им.\ Н.\,Э.~Баумана. Сер. Приборостроение, 2021. Вып.~2(135). С.~4--22. doi: 
10.18698/0236-3933-2021-2-4-22. 
\end{thebibliography}

 }
 }

\end{multicols}

\vspace*{-3pt}

\hfill{\small\textit{Поступила в~редакцию 25.03.22}}

%\vspace*{8pt}

\newpage


\vspace*{-28pt}

%\hrule

%\vspace*{2pt}

%\hrule

%\vspace*{-2pt}

\def\tit{VISUAL REPRESENTATION OF~THE~DECREASE IN CONFLICT INTENSITY 
AND~ITS~RESOLUTION IN~HYBRID INTELLIGENT MULTIAGENT SYSTEMS}


\def\titkol{Visual representation of~the~decrease in conflict intensity 
and~its~resolution in~hybrid intelligent multiagent systems}


\def\aut{S.\,B.~Rumovskaya and~I.\,A.~Kirikov}

\def\autkol{S.\,B.~Rumovskaya and~I.\,A.~Kirikov}

\titel{\tit}{\aut}{\autkol}{\titkol}

\vspace*{-8pt}


\noindent
   Kaliningrad Branch of the Federal Research Center ``Computer Science and Control'' of the 
Russian Academy of Sciences, 5~Gostinaya Str., Kaliningrad 236000, Russian Federation


\def\leftfootline{\small{\textbf{\thepage}
\hfill INFORMATIKA I EE PRIMENENIYA~--- INFORMATICS AND
APPLICATIONS\ \ \ 2022\ \ \ volume~16\ \ \ issue\ 2}
}%
 \def\rightfootline{\small{INFORMATIKA I EE PRIMENENIYA~---
INFORMATICS AND APPLICATIONS\ \ \ 2022\ \ \ volume~16\ \ \ issue\ 2
\hfill \textbf{\thepage}}}

\vspace*{3pt} 
   
   
      
   
   \Abste{Many practical problems require a~collective solution ensuring pluralism of opinions, 
integration of private points of view, and reduction of errors. The authors propose to model the work 
of such groups of specialists with hybrid intelligent multiagent systems considering the peculiarities 
of their group dynamics. Such approach would provide improving the quality and efficiency of the 
solution as well as comprehensive consideration of the problem and the process of its overcoming 
including visualization of conflicts and processes of their management. The latter would provide a 
new information on conflict resolution both in the system and in the real group of specialists. The 
work is devoted to the development of a method for visualization of conflict resolution processes 
within the framework of hybrid intelligent multiagent systems with problem-oriented and process-oriented constructive conflicts.}
   
   \KWE{collective of specialists; conflict; visualization of the conflict resolution}
   
   
\DOI{10.14357/19922264220212}

%\vspace*{-16pt}

%\Ack
%\noindent




%\vspace*{4pt}

  \begin{multicols}{2}

\renewcommand{\bibname}{\protect\rmfamily References}
%\renewcommand{\bibname}{\large\protect\rm References}

{\small\frenchspacing
 {%\baselineskip=10.8pt
 \addcontentsline{toc}{section}{References}
 \begin{thebibliography}{99}
\bibitem{1-kir-1}
   \Aue{Listopad, S.\,V., and I.\,A.~Kirikov.} 2019. Modelirovanie konfliktov agentov 
v~gibridnykh intellektual'nykh mnogoagentnykh sistemakh [Modeling of agent conflicts in hybrid 
intelligent multiagent systems]. \textit{Sistemy i~Sredstva Informatiki~--- Systems and Means of 
Informatics} 29(3):139--148. doi: 10.14357/08696527190312.
\bibitem{2-kir-1}
   \Aue{Listopad, S.\,V., and I.\,A.~Kirikov.} 2020. Metod identifikatsii konfliktov agentov 
v~gibridnykh intellektual'nykh mnogoagentnykh sistemakh [Agent conflict identification method in 
hybrid intelligent multiagent systems]. \textit{Sistemy i~Sredstva Informatiki~--- Systems and 
Means of Informatics} 30(1):56--65. doi: 10.14357/08696527200105.

\bibitem{5-kir-1} %3
   \Aue{Rumovskaya, S.\,B., and I.\,A.~Kirikov.} 2020. Metod vi\-zu\-al'\-no\-go predstavleniya 
konfliktov v~gibridnykh intellektual'nykh mnogoagentnykh sistemakh [Conflict visual 
representation method in collective decision-making within hybrid intelligent multiagent systems]. 
\textit{Informatika i~ee Primeneniya~--- Inform. Appl.} 14(4):77--82. doi: 
10.14357/19922264200411. 

\bibitem{3-kir-1} %4
   \Aue{Listopad, S.\,V., and I.\,A.~Kirikov.} 2021. Stimulyatsiya konfliktov agentov 
v~gibridnykh intellektual'nykh mnogoagentnykh sistemakh [Stimulation of agent conflicts in hybrid 
intelligent multiagent systems]. \textit{Sistemy i~Sredstva Informatiki~--- Systems and Means of 
Informatics} 31(2):47--58. doi: 10.14357/08696527210205.

\bibitem{6-kir-1} %5
   \Aue{Rumovskaya, S.\,B., and I.\,A.~Kirikov.} 2021. Metod vi\-zu\-a\-li\-za\-tsii stimulyatsii konfliktov 
v~gibridnykh intellektual'nykh mnogoagentnykh sistemakh [Visual representation method for the 
conflict stimulation in hybrid\linebreak intelligent multiagent systems]. \textit{Informatika i~ee 
Primeneniya~--- Inform. Appl.} 15(3):75--82. doi: 10.14357/ 19922264210310. 
\bibitem{4-kir-1} %6
   \Aue{Listopad, S.\,V., and I.\,A.~Kirikov.} 2022. Razreshenie konfliktov v~gibridnykh 
intellektual'nykh mnogoagentnykh sistemakh [Resolving conflicts in hybrid intelligent multi-agent 
systems]. \textit{Informatika i~ee Primeneniya~--- Inform. Appl.} 16(1):54--60.

\bibitem{9-kir-1} %7
   \Aue{Shaw, M.} 1981. \textit{Group dynamics: The psychology of small group behavior}. New 
York, NY: McGraw-Hill. 531~p.
\bibitem{7-kir-1} %8
   \Aue{Andreeva, G.\,M.} 2009. \textit{Sotsial'naya psikhologiya} [Social psychology]. Moscow: 
Aspect-press. 393~p.
\bibitem{8-kir-1} %9
   \Aue{Brown, R., and S.~Pehrson.} 2019. \textit{Group processes: Dynamics with and between 
groups}. 3rd ed. Oxford: Wiley-Blackwell. 344~p.  doi: 10.1002/9781118719244.

\bibitem{10-kir-1}
   \Aue{Emel'yanov, S.\,M.} 2018. \textit{Konfliktologiya} [Conflictology]. Moscow: Yurayt. 
322~p. 
\bibitem{11-kir-1}
   \Aue{Behfar, K., R.~Peterson, E.~Mannix,  and W.~Trochim.} 2008. The critical role of conflict 
resolution in teams: A~close look at the links between conflict type, conflict management 
strategies, and team outcomes. \textit{J.~Appl. Psychol}. 93(1):170--88. doi:  
10.1037/0021-9010.93.1.170.
\bibitem{12-kir-1}
   \Aue{Antsupov, A.\,Ya., and S.\,V.~Baklanovskiy.} 2009. \textit{Konfliktologiya v~skhemakh 
i~kommentariyakh} [Conflictology in schemes and comments]. St.\ Petersburg: Piter. 304~p.
\bibitem{13-kir-1}
   \Aue{Cusack, J.\,J., T.~Bradfer-Lawrence, Z.~Baynham-Herd, \textit{et al.}} 2021. Measuring 
the intensity of conflicts in conservation. \textit{Conserv. Lett.} 14(3):e12783. 11~p. doi: 
10.1111/ conl.12783.
\bibitem{14-kir-1}
   \Aue{Baumes, J., M.\,K.~Goldberg, M.\,S.~Krishnamoorthy, \textit{et al.}} 2005. Finding 
communities by clustering a graph into overlapping sub-graphs. \textit{Conference (International) 
on Applied Computing Proceedings}. IADIS. 1:97--104. 
   \bibitem{15-kir-1}
   \Aue{Baumes, J., M.~Goldberg, and M.~Magdon-Ismail.} 2005. Efficient identification of 
overlapping communities. \textit{Conference (International) on Intelligence and Security 
Informatics Proceedings}. Eds. P.\,B.~Kantor, G.~Muresan, F.\,S.~Roberts, \textit{et al}. 
Lecture notes in computer science ser. Berlin--Heidelberg: Springer, 2005. 
3495:27--36. doi: 10.1007/11427995\_3. 
   \bibitem{16-kir-1}
   \Aue{Virtseva, N.\,S., I.\,E.~Vishnyakov, and I.\,P.~Ivanov.} 2021. Sposoby vydeleniya 
soobshchestv s~opredelennymi tipami otnosheniy v~grafakh na osnove billingovoy informatsii 
[Methods of detecting communities with certain relationship types in graphs using billing 
information]. \textit{Vestnik MGTU im. N.\,E.~Baumana. Ser. Priborostroyeniye} [Herald of the
Bauman Moscow State Technical University. Ser. Instrument Engineering] 2(135):4--22. doi:  
10.18698/ 0236-3933-2021-2-4-22.
   \end{thebibliography}

 }
 }

\end{multicols}

\vspace*{-6pt}

\hfill{\small\textit{Received March 25, 2022}}
   
   \Contr
   
   \noindent
   \textbf{Rumovskaya Sophiya B.} (b.\ 1985)~--- Candidate of Science (PhD) in technology, 
scientist, Kaliningrad Branch of the Federal Research Center ``Computer Science and Control'' of 
the Russian Academy of Sciences, 5~Gostinaya Str., Kaliningrad 236000, Russian Federation; 
\mbox{sophiyabr@gmail.com}
   
   \vspace*{3pt}
   
   \noindent
   \textbf{Kirikov Igor A.} (b.\ 1955)~--- Candidate of Science (PhD) in technology, director, 
Kaliningrad Branch of the Federal Research Center ``Computer Science and Control'' of the Russian 
Academy of Sciences, 5~Gostinaya Str., Kaliningrad 236000, Russian Federation; 
\mbox{baltbipiran@mail.ru}

\label{end\stat}

\renewcommand{\bibname}{\protect\rm Литература}       