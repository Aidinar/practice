\def\stat{agasand}

\def\tit{МНОГОМЕРНЫЕ БИНАРНЫЕ РЫНКИ И~CC-VaR}

\def\titkol{Многомерные бинарные рынки и~CC-VaR}

\def\aut{Г.\,А.~Агасандян$^1$}

\def\autkol{Г.\,А.~Агасандян}

\titel{\tit}{\aut}{\autkol}{\titkol}

\index{Агасандян Г.\,А.}
\index{Agasandyan G.\,A.}


%{\renewcommand{\thefootnote}{\fnsymbol{footnote}} \footnotetext[1]
%{Работа выполнена при поддержке Министерства науки и~высшего образования Российской Федерации (проект 
%075-15-2020-799).}}


\renewcommand{\thefootnote}{\arabic{footnote}}
\footnotetext[1]{Федеральный исследовательский центр <<Информатика и~управление>> Российской 
академии наук, \mbox{agasand17@yandex.ru}}

%\vspace*{-6pt}

 
  
  \Abst{Работа продолжает изучение проблем использования континуального критерия VaR 
(CC-VaR) на финансовых рынках. Речь идет о~некоторых технических проблемах, 
возникающих на многомерных опционных рынках~--- рынках, порожденных несколькими 
стохастически связанными между собой базовыми активами. Рассматривается 
многомерное расширение бинарных рынков~--- упрощенного варианта рынков обычных 
опционов, таких как коллы и~путы. В~дискретном по инструментарию варианте рынка они 
также служат простейшим усложнением сценарного рынка. В~предположении, что 
сценарными индикаторами на рынке в~полной мере непосредственно не торгуют, 
предлагается методика получения репликации из бинарных инструментов. Она основывается 
на теоремах паритета для одномерного бинарного рынка и~подробно излагается для 
двумерного. Приводятся конструкции базисов из бинарных инструментов как однотипных, 
так и~естественных смешанных с~некоторым назначенным центром рынка.  
Тео\-ре\-ти\-че\-ские конструкции оптимальных портфелей в~этих базисах иллюстрируются 
на примере конкретного двумерного рынка.}
   
  \KW{базовые активы; многомерный рынок; функция рисковых предпочтений инвестора; 
континуальный критерий VaR (CC-VaR); стоимостная и~прогнозная плотности; сценарные 
индикаторы; базисы; бинарные опционы; однотипный портфель; центр рынка; смешанный 
портфель}

\DOI{10.14357/19922264220201}
  
%\vspace*{-3pt}


\vskip 10pt plus 9pt minus 6pt

\thispagestyle{headings}

\begin{multicols}{2}

\label{st\stat}
  
  \section{Введение}
  
  Проблемы применения на рынках опционов введенного автором 
континуального критерия VaR (CC-VaR) рассматриваются в~ряде работ~[1--5]. 
Этот критерий служит средством, позволяющим ин\-вес\-то\-ру наиболее адекватно 
отражать свои рисковые предпочтения. Для его полноценного использования 
необходимо также, чтобы на рынке были представлены финансовые 
инструменты в~большом разнообразии. На сегодняшний день к~таковым можно 
отнести, пожалуй, лишь рынки опционов с~одним базовым активом 
и~подробной линейкой страйков. Для них теоретической моделью служит 
одномерный $\delta$-ры\-нок~\cite{2-ag, 3-ag, 4-ag}. 
  
  Проблема с~определением многомерного рынка в~контексте применения  
CC-VaR требует особого подхода. Таких рынков пока нет, а механического 
объединения нескольких рынков с~разными базовыми активами явно 
недостаточно~[6, 7]. Следует вводить опционы на совокупности активов, 
причем специальным образом. Как оказывается, именно одномерный 
теоретический $\delta$-ры\-нок со своим специфическим характером без труда 
обобщается на многомерный случай, а его приближенные версии вполне могут 
быть реализованы. 
  
  В работе решаются проблемы технического характера, которые более 
свойственны многомерным рынкам, но их зародыши возникают на 
одномерных. Суть в~том, что, например, на (одномерных) рынках 
традиционных опционов непосредственно баттерфляями не торгуют, а торгуют 
коллами и~путами, но для применения CC-VaR большее значение имеют 
именно баттерфляи, что следует учитывать в~многомерном случае. В~работе  
проб\-ле\-ма решается для бинарного рынка~--- упрощенной модели рынка 
обычных опционов. 
  
  \section{Теоретический $\zeta$-рынок и~его~свойства}
  
  Многомерный $\delta$-рынок~--- это рынок, в~\mbox{основе} которого лежат $n$ 
($> 1$) базовых активов ($n$~--- раз\-мер\-ность рынка). Как и~ранее, рынок 
рас\-смат\-ри\-ва\-ет\-ся однопериодным, тео\-ре\-ти\-че\-ским и~идеальным~\cite{4-ag, 5-ag}. 
Векторы цен базовых активов в~конце периода $\boldsymbol{x}\hm= (x_1, x_2, 
\ldots , x_n)$, $x_l \hm\in {\sf X}_l \hm\subset \mathfrak{R}$, $l \hm\in N\hm= \{1, 
\ldots , n\}$, образуют $n$-мер\-ное множество ${\sf X}\hm= \prod\nolimits_{l\in 
N}{\sf X}_l$. Заданы две неотрицательные функции~--- \textit{прогнозная} 
и~\textit{стоимостная} плотности, соответственно $p(\boldsymbol{x})$ 
и~$c(\boldsymbol{x})$, $\boldsymbol{x}\hm\in {\sf X}$, по\-рож\-да\-ющие меры ${\sf P}\{\cdot\}$ 
и~${\sf C}\{\cdot\}$, первая из которых вероятностная. 
  
  Важно отметить, что на введенном $n$-мер\-ном рынке \textit{не торгуют} 
инструментами иной раз\-мер\-ности. Даже если в~описании конкретного 
инструмента фигурируют лишь~$k$ ($< n$) базовых активов, то непременно 
подразумевается, что в~нем присутствуют и~остальные базовые активы в~форме 
безрисковых единичных инструментов, хотя они для удобства в~записи 
инструментов могут опускаться. Их называем $k$-\textit{мер\-ны\-ми версиями} 
$n$-мер\-ных инструментов. 
  
  Платежная функция инструмента~$\boldsymbol{I}$ обозначается 
$\pi(\boldsymbol{x}; \boldsymbol{I})$, $\vert\boldsymbol{I}\vert$~--- его рыночная стоимость, 
рассчитанная по плот\-ности~$c(\boldsymbol{x})$, а~$\|\boldsymbol{I}\|$~--- средний с~точки 
зрения инвестора доход, рассчитанный по плот\-ности $p(\boldsymbol{x})$. Имеют место 
соотношения:
  $$
  \vert\boldsymbol{I} \vert=  \int\limits_{{\sf X}}\pi (\boldsymbol{x};\boldsymbol{I}) 
c(\boldsymbol{x})\,d\boldsymbol{x}\,;\enskip \| \boldsymbol{I}\| =\int\limits_{{\sf X}}\pi(\boldsymbol{x};\boldsymbol{I}) 
p(\boldsymbol{x})\,d\boldsymbol{x}\,.
  $$
  
  На рынке обращаются $\delta$-ин\-стру\-мен\-ты $\boldsymbol{D}(\boldsymbol{s})$ 
с~(обобщенными) $n$-мер\-ны\-ми $\delta$-функ\-ци\-ями 
относительно~$\boldsymbol{s}$ в~качестве платежных функций: 
  $$
  \pi(\boldsymbol{x};\boldsymbol{D}(\boldsymbol{s}))=\delta(\boldsymbol{x}-\boldsymbol{s})\,;\enskip \vert 
\boldsymbol{D}(\boldsymbol{s})\vert=c(\boldsymbol{s})\,, \ s\in{\sf X}\,.
  $$
  
  Фактически $\boldsymbol{D}(\boldsymbol{s})$~--- это произведение $n$ одномерных  
$\delta$-ин\-стру\-мен\-тов, а его платежная функция~--- произведение $n$ 
одномерных $\delta$-функ\-ций: 
  \begin{equation}
  \left.
  \begin{array}{rl}
  \boldsymbol{D}(\boldsymbol{s})&=\prod\limits_{l\in N} \boldsymbol{D}_l\left(s_l\right)\,;\\[8pt]
\pi(\boldsymbol{x};\boldsymbol{D}(\boldsymbol{s}))&=\prod_{l\in N} \delta \left( x_l-s_l\right).
\end{array}
\right\}
  \label{e1-ag}
  \end{equation}
  
  Рыночный инструмент $\boldsymbol{G}$ с~произвольной измеримой платежной 
функцией $g(\boldsymbol{x})$ и~его стоимость представляются соответственно в~виде:
 \begin{align*}
  \boldsymbol{G}&=\int\limits_{\sf X} g(\boldsymbol{s}) \boldsymbol{D}(\boldsymbol{s})\,d\boldsymbol{s};\\
  \vert \boldsymbol{G}\vert &= \int\limits_{\sf X} g(\boldsymbol{s})\vert \boldsymbol{D}(\boldsymbol{s})\vert 
d\boldsymbol{s}=\int\limits_{\sf X} g(\boldsymbol{s}) c(\boldsymbol{s})\,d\boldsymbol{s}\,.
  \end{align*}
  
  Так, рассматриваются инструменты $\boldsymbol{H}\{M\}$, $M \hm\subset {\sf X}$, 
называемые \textit{индикаторами} (множеств~$M$), и,~в~частности, 
$\boldsymbol{U}\hm= \boldsymbol{H}\{{\ sf X}\}$, при этом 
  \begin{equation}
  \left.
  \begin{array}{rl}
  \boldsymbol{H}\{M\} &=\displaystyle \int\limits_M \boldsymbol{D}(\boldsymbol{s})\,d\boldsymbol{s}\,;\\[13pt]
  \vert \boldsymbol{H}\{M\}\vert &= \displaystyle \int\limits_M c(\boldsymbol{s})\,d\boldsymbol{s}\,;\\[13pt]
  \vert \boldsymbol{U}\vert={\sf C}\{\mathsf{X}\} &=\displaystyle \int\limits_{\mathsf X}  c
  (\boldsymbol{s})\,d\boldsymbol{s}=\displaystyle\fr{1}{r}\,,
  \end{array}
  \right\}
  \label{e2-ag}
  \end{equation}
где $r$ имеет смысл безрискового дохода за период. Без ущерба для общности 
принимается $r\hm\equiv  1$. В~таком случае функция $c(\boldsymbol{x})$ приобретает 
свойства плот\-ности вероятности и~ее можно интерпретировать как 
порождаемую рынком \textit{стоимостную} плот\-ность. В~противовес ей 
плотность вероятности $p(\boldsymbol{x})$ можно считать \textit{справедливой} с~точки 
зрения ин\-вес\-то\-ра ценой $\delta$-ин\-стру\-мента. 
  
  Частным случаем~(\ref{e2-ag}) являются играющие в~дальнейшем изложении 
важную роль многомерные бинарные опционы, которые вводятся по некоторой 
аналогии с~традиционными опционами колл и~пут. 
  
  На \textit{одномерном} рынке \textit{бинарным коллом} $\boldsymbol{U}_s^+$ 
и~\textit{бинарным путом}~$\boldsymbol{U}_s^-$ со страйком~$s$ называются 
инструменты с~платежными функциями соответственно 
  \begin{multline}
  \pi\left( x;\boldsymbol{U}_s^+\right) =\chi_{x\geq s}(x);\
  \pi\left( x; \boldsymbol{U}_s^-\right) =\chi_{x<s}(x),\\
   x,s\in{\sf X}\subset \mathfrak{R}\,,
  \label{e3-ag}
  \end{multline}
где $\chi_M(x)$, $M\subset {\sf X}$,~--- характеристическая функция 
(индикатор) множества~$M$. Справедливы очевидные соотношения, которые 
можно рассматривать как теоремы паритета одномерного рынка: 
\begin{equation}
\left.
\begin{array}{rl}
\boldsymbol{U}_s^+ +\boldsymbol{U}_s^- &=\boldsymbol{U}\,;\\[6pt]
\boldsymbol{H}\{M\}&=\boldsymbol{U}_a^+-\boldsymbol{U}_b^+ = \boldsymbol{U}_b^- - \boldsymbol{U}_a^-\,,
\end{array}
\right\}
\label{e4-ag}
%\label{e5-ag}
\end{equation}
где
$M=[a,b)$, $a,b\in {\sf X}.$

  \textit{Многомерным} обобщением одномерных бинарных 
опционов~$\boldsymbol{U}_s^+$ и~$\boldsymbol{U}_s^-$ служат $n$-мер\-ные 
инструменты~$\boldsymbol{Z}_{\boldsymbol{\alpha};\boldsymbol{s}}$ векторного типа~$\boldsymbol{\alpha}$ 
и~с~векторным страйком $\boldsymbol{s}\hm\in \mathfrak{R}^n$, называемые  
$\zeta$-\textit{оп\-ци\-о\-на\-ми} и~задаваемые по аналогии с~(\ref{e1-ag}) 
соотношениями: 
  \begin{equation}
  \left.
  \begin{array}{rl}
 \boldsymbol{Z}_{\boldsymbol{\alpha};\boldsymbol{s}}&=\displaystyle\prod\limits_{i\in N} 
\boldsymbol{O}_{i\beta_i;s_i};\\[6pt]
  \pi\left( \boldsymbol{x};\boldsymbol{Z}_{\boldsymbol{\alpha};\boldsymbol{s}}\right) &=\displaystyle\prod\limits_{i\in N} 
\pi\left( x_l;\boldsymbol{O}_{l\beta_l;s_l}\right), \
 \boldsymbol{x}\in \mathfrak{R}^n\,.
 \end{array}
 \right\}
  \label{e6-ag}
  \end{equation}
Здесь вводятся новые обозначения, которые час\-тич\-но дублируют прежние, но 
более удобны для целей алгоритмической автоматизации дальнейших 
построений. Так, для компонентных инструментов
\begin{equation}
\boldsymbol{O}_{0;s}\equiv \boldsymbol{U}_s^-\,;\enskip \boldsymbol{O}_{1;s}\equiv \boldsymbol{U}_s^+\,.
\label{e7-ag}
\end{equation}
  
  Вектор $\boldsymbol{\alpha}$, $\alpha_l \hm= \pm1$ (или просто~$\pm$, как для 
бинарных коллов и~путов в~индексах инструментов), $l \hm\in N$, определяет 
векторный тип $\zeta$-оп\-ци\-о\-нов~$\boldsymbol{Z}$, $\boldsymbol{s}\hm\in 
\mathfrak{R}^n$~--- векторный страйк. Вектор 
$\boldsymbol{\beta}\hm= (\boldsymbol{\alpha} \hm+ 1)/2$, дублирующий~$\boldsymbol{\alpha}$, 
вводится по техническим причинам и~принимает два значения: 
$$
\beta_l = \begin{cases}
0 & \mbox{для\ пута};\\
 1 & \mbox{для\ колла}.
 \end{cases}
 $$
 По необходимости в~(\ref{e7-ag}) 
подобно~(\ref{e6-ag}) в~индексы может добавляться параметр $l\hm\in N$. 
  
  Очевидно, для каждого векторного страйка на \mbox{$n$-мер}\-ном рынке могут 
котироваться $2^n$ типов \mbox{$\zeta$-оп}\-ци\-о\-нов. Рынок \mbox{$n$-мер}\-ных  
$\zeta$-оп\-ци\-о\-нов с~их \mbox{$k$-мер}\-ны\-ми версиями, $k\hm< n$, называется  
$n$-мер\-ным \mbox{$\zeta$-рын}\-ком. 
  
  \section{Двумерный дискретный рынок}
  
  Рассматривается сценарная дискретизация двумерного теоретического  
$\delta$-рын\-ка. Все конструкции, введенные для многомерного случая, здесь 
сохраняют свою силу, но для удобства используется в~большей мере 
адаптированная к~двумерному случаю легко воспринимаемая система 
обозначений. 
  
  Цены двух базовых активов теоретического двумерного $\delta$-рын\-ка 
обозначаются~$x$ и~$y$, страйки опционов~--- соответственно~$s$ и~$t$, где
$x, s \hm\in {\sf X}\hm = [a_1, b_1) \hm\subset\mathfrak{R}$; $y, t \hm\in{\sf 
Y}\hm = [a_2, b_2) \hm\subset\mathfrak{R}$. Сценарный рынок строится 
равномерным разбиением множества~${\sf X}$ на~$v_1$ интервалов 
(сценариев), а~${\sf Y}$~--- на $v_2$ интервалов. Одномерные сценарии 
на~${\sf X}$ и~${\sf Y}$ даются фор\-му\-лами: 
  \begin{align*}
S_i&=\left[ s_{i-1},s_i\right)\,,\ s_i=a_1+ih_1\,,\ h_1=\fr{b_1-a_1}{v_1}\,,\\
  & \hspace*{52mm}i\in I\,,\  s_0=a_1\,;\\
T_j&=\left[ t_{j-1}, t_j\right)\,,\ t_j=a_2+jh_2\,,\ h_2= \fr{b_2-a_2}{v_2}\,,\\
  & \hspace*{52mm}j\in J\,,\  t_0=a_2\,.
  \end{align*}
  %
  Здесь $I = \{1,\ldots ,v_1\}$, $J \hm= \{1,\ldots ,v_2\}$, а~номер сценария 
совпадает с~индексом его правой границы. Двумерными сценариями будут 
прямые произведения всех пар $S_i\times T_j$, $i \hm\in I$, $j \hm\in J$, 
образующие равные между собой прямоугольники. 
  
  На сценарном рынке базис образуют инструментальные индикаторы 
сценариев 
$$
\boldsymbol{D}_{ij}\equiv \boldsymbol{H}\left[S_i\times T_j\right],
$$
 а~произвольный портфель 
с~вектором весов~$\boldsymbol{g}$ этих инструментов для двумерного случая 
приобретает вид: 
  \begin{equation}
  \boldsymbol{G}= \sum\limits_{i\in I, j\in J} g_{ij}\boldsymbol{D}_{ij}.
  \label{e8-ag}
  \end{equation}
  
  Двумерным обобщением бинарных опционов служат инструменты, 
определяемые векторными страйками $(s_i, t_j)$, или просто $(i, j)$, а~именно: 
  \begin{multline*}
  \boldsymbol{Z}_{\beta_1\beta_2;ij}= 
\boldsymbol{O}_{\beta_1;1,i}\boldsymbol{O}_{\beta_2;2,j}=\boldsymbol{O}_{\beta_1;i\cdot}
  \boldsymbol{O}_{\beta_2;\cdot j}\,,\\
   i\in I\,,\ j\in J\,.
\end{multline*}  
  
  Дискретный двумерный $\zeta$-ры\-нок образуют инструменты $\boldsymbol{Z}_{ij}$, 
$i\hm\in I$, $j \hm\in J$, с~возможной фиксацией типа 
(в~терминах~$\boldsymbol{\alpha}$ или~$\boldsymbol{\beta}$). На нем 
торгуются и~их одномерные версии~$\boldsymbol{Z}_{i\cdot}$  и~$\boldsymbol{Z}_{\cdot j}$ 
с~маркером <<точка>> для координаты безрискового актива. 
  
  Далее предполагается, что на рынке сценарными индикаторами 
непосредственно не торгуют, и~потому для использования 
представления~(\ref{e8-ag}) приходится реплицировать их из  
$\zeta$-оп\-ци\-о\-нов. Ситуация заимствована с~рынков обычных опционов, 
где торгуют, как правило, коллами и~путами, но не баттерфляями. 
  
  Для одномерных компонент рынка с~$v$ сценариями справедливы формулы:
   \begin{equation}
  \hspace*{-3mm}\boldsymbol{D}_i= \begin{cases}
  \boldsymbol{U}-\boldsymbol{U}_1^{+} =\boldsymbol{U}_1^{-}\,, &\hspace*{-7mm}i=1\,;\\
  \boldsymbol{U}^+_{i-1}-\boldsymbol{U}_i^+ = \boldsymbol{U}_i^- - \boldsymbol{U}^-_{i-1} ={}&\\
  &\hspace*{-40mm}{}= \boldsymbol{U} -  \boldsymbol{U}^-_{i-1} - \boldsymbol{U}_i^+\,,\ 1<i<v\,;\\
  \boldsymbol{U}^+_{v-1} = \boldsymbol{U}- \boldsymbol{U}^-_{v-1}\,, & \hspace*{-7mm}i=v\,.
  \end{cases}
  \label{e9-ag}
  \end{equation}
  
  \vspace*{-6pt}
  
\noindent
  Эти соотношения вытекают из теорем паритета для одномерных бинарных 
опционов~(\ref{e4-ag}). Согласно этим соотношениям для 
внутренних страйков возможны три варианта репликации~--- два однотипных 
и~один \textit{естественный} смешанный. В~соответствии с~(\ref{e6-ag}) для 
конструирования репликации необходимо перемножать одномерные 
представления индикаторов сценариев. 
  
  Базис $\zeta$-рынка может быть однотипным (из путов или коллов) 
и~смешанным (одновременно из тех и~других). Притом что однотипных 
базисов всего два, число возможных смешанных базисов даже при небольших 
масштабах рынка весьма значительно. Но именно смешанные базисы 
представляют интерес, поскольку, как правило, не вся линейка (по страйкам) 
опционов котируется на рынке. Во всяком случае, на традиционных 
одномерных рынках опционы со страйками, значимо ниже текущей цены 
базового актива для колла и~значимо выше~--- для пута, весьма слабо 
представлены на рынках. 
  
  В связи с~этим построение многомерного смешанного базиса и~смешанного 
портфеля проводится по упрощенной схеме в~предположении наличия так 
называемого центра рынка (центрального страйка). На реальном рынке центр, 
как правило, не единствен и~располагается вблизи вектора текущих цен 
базовых активов. В~тео\-ре\-ти\-че\-ской схеме сценарные индикаторы будут 
строиться из бинарных путов со страйками ниже центра и~из бинарных 
коллов~--- выше центра. Такие представления портфелей называем 
\textit{естественными}. 
  
  Согласно~(\ref{e9-ag}), для каждого внутреннего, потенциально 
центрального, страйка с~номером~$m$ принципиально возможны две 
качественно раз\-ли\-ча\-ющи\-еся естественная смешанная и~однотипная\linebreak 
комбинации бинарных инструментов, реплицирующие индикатор сценария ($ 1<m<v$): 
  \begin{equation}
  \left.
  \begin{array}{l}
(a)~~~~~~~~\boldsymbol{D}_m=\boldsymbol{U}-\boldsymbol{U}^-_{m-1} - \boldsymbol{U}_m^+\,; \\[6pt]
(b)~~~~~~~~\boldsymbol{D}_m = \boldsymbol{U}_m^- - \boldsymbol{U}^-_{m-1}.
\end{array}
\right\}
   \label{e10-ag}
  \end{equation}
  
  \vspace*{-3pt}
  
  Комбинация $\boldsymbol{U}_{m-1}^+ \hm- \boldsymbol{U}_m^+$, также представленная 
в~(\ref{e9-ag}), неприемлема, поскольку в~ней страйк первого колла находится 
ниже центра. 

\pagebreak
  
  Для репликации базиса с~комбинацией~($a$) необходимо присутствие на 
рынке всех бинарных путов $\boldsymbol{U}_i^-$ для $i\hm<m$ и~всех бинарных 
коллов~$\boldsymbol{U}_i^+$ для $i\hm\geq  m$. Случай~($b$) отличается от~($a$) 
лишь тем, что на рынке дополнительно присутствует опцион~$\boldsymbol{U}^+_{m-
1}$, т.\,е.\ котируются оба опциона: $\boldsymbol{U}_{m-1}^-$ и~$\boldsymbol{U}^+_{m-1}$. 
  
  Для большей универсализации алгоритма оптимизации в~работе 
используется более сложное смешанное представление~($a$) в~(\ref{e10-ag}).
  
  Построение возможных двумерных сценарных базисов из  
$\zeta$-оп\-ци\-о\-нов, как и~вообще многомерных, проводится на основе 
правил построения для одномерного рынка. Для него с~учетом~(\ref{e9-ag}), 
(\ref{e10-ag}) возможны три варианта репликации сценарных базисов~--- два 
однотипных (один из бинарных путов, другой из бинарных коллов) и~третий 
смешанный (из тех и~других с~центром в~страйке~$m$). 
  
  Базис в~бинарных путах (тип $\alpha \hm= -1$, или $\beta\hm = 0$) для~$v$ 
сценариев имеет вид:

\vspace*{-6pt}

\noindent
  \begin{multline}
  \boldsymbol{D}_1=\boldsymbol{O}_{0;1}\,; \ \boldsymbol{D}_i= - \boldsymbol{O}_{0;i-1} +\boldsymbol{O}_{0;i}\,, \ 
0<i< v\,; \\
 \boldsymbol{D}_v= \boldsymbol{U}-\boldsymbol{O}_{0;v-1}\,;
  \label{e11-ag}
  \end{multline}
  
\vspace*{-3pt}

  \noindent
в бинарных коллах ($\alpha\hm = +1$, или $\beta\hm = 1$):

\vspace*{-6pt}

\noindent
\begin{multline}
\boldsymbol{D}_1 = \boldsymbol{U} -\boldsymbol{O}_{1;1}\,;\  \boldsymbol{D}_i=\boldsymbol{O}_{1;i-1} -
\boldsymbol{O}_{1;i}\,,\ 0<i<v;\\
 \boldsymbol{D}_v=\boldsymbol{O}_{1;v-1}\,;
\label{e12-ag}
\end{multline}

\vspace*{-3pt}

\noindent
смешанный:

\vspace*{-6pt}

\noindent
\begin{multline}
\boldsymbol{D}_1=\boldsymbol{O}_{0;1}\,;\ \boldsymbol{D}_i=-\boldsymbol{O}_{0;i-1}+\boldsymbol{O}_{0;i}\,,\enskip  0<i<m\,;  \\
  \boldsymbol{D}_m=\boldsymbol{U}-\boldsymbol{O}_{0;m-1}-\boldsymbol{O}_{1;i};
\\
\!\!\boldsymbol{D}_i= \boldsymbol{O}_{1;i-1} -\boldsymbol{O}_{1;i}\,, \ m<i<v;\ \boldsymbol{D}_v=\boldsymbol{O}_{1;v-1}\,.
\!\!\label{e13-ag}
\end{multline}

\vspace*{-3pt}
  
  Поскольку каждый двумерный сценарный индикатор есть произведение двух 
одномерных (согласно перемножению платежных функций), его представление 
в~терминах двумерных $\zeta$-оп\-ци\-о\-нов получается перемножением пары 
подходящих базисных инструментов из~(\ref{e11-ag})--(\ref{e13-ag}). 

%\vspace*{-3pt}
  
  \section{Формирование базисов двумерного $\zeta$-рынка}
  
  Введенные в~многомерном случае произвольной размерности конструкции 
здесь переписываются для двумерного $\zeta$-рын\-ка. При построении базисов 
и~портфелей рассматриваются два варианта~--- однотипный и~смешанный.
  
  
  \textit{Однотипная} репликация сценарных индикаторов проводится 
в~терминах  $\zeta$-оп\-ци\-о\-нов единого типа~$\boldsymbol{\alpha}$ 
(и~$\boldsymbol{\beta}\hm= \{\beta_1, \beta_2\}$), фиксируемого заранее, 
и~потому применяются более простые варианты~(\ref{e11-ag}) и~(\ref{e12-ag}) 
без обозначения типа. 
  
  Каждое перемножение таких одномерных представлений приводит к~сумме 
парных произведений одномерных опционов, которые затем замещаются\linebreak\vspace*{-12pt}

\columnbreak

\noindent
 эквивалентными по платежным функциям двумерными  
$\zeta$-оп\-ци\-о\-на\-ми по правилам 
  \begin{equation}
\boldsymbol{O}_{1,i}\boldsymbol{O}_{2,j}\!\to\! \boldsymbol{Z}_{ij}\,;\enskip \boldsymbol{O}_{1,i}\boldsymbol{U}_{2}\!\to\! 
\boldsymbol{Z}_{i\cdot}\,;\enskip \boldsymbol{U}_1\boldsymbol{O}_{2,j}\!\to\! \boldsymbol{Z}_{\cdot j}.\!
  \label{e14-ag}
  \end{equation}
  
  После решения задачи оптимизации~\cite{2-ag, 5-ag} дискретным 
алгоритмом, основанным на процедуре Ней\-ма\-на--Пир\-со\-на~\cite{8-ag}, 
и~нахождения вектора весов оптимального портфеля базисных  
$\zeta$-оп\-ци\-о\-нов находится его платежная функция. Для этого 
используются соотношения~(\ref{e3-ag}) с~учетом  
переопределений~(\ref{e7-ag}), и~потому сам поиск проводится по правилам: 
  \begin{equation}
  \left.
  \begin{array}{l}
  \!\!\boldsymbol{Z}_{\boldsymbol{\beta};ij}\to \upsilon_{\beta_1;i\cdot}(x)\upsilon_{\beta_2;\cdot 
j}(y)\,,\enskip  \boldsymbol{Z}_{i\cdot}\to \upsilon_{\beta_1;i\cdot}(x)\,,\\[6pt]
     \hspace*{27mm}\boldsymbol{Z}_{\cdot j}\to 
\upsilon_{\beta_2;\cdot j}(y)\,,\enskip \boldsymbol{U}\to 1\,;\\[6pt]
    \!\!\upsilon_{\beta_i\cdot; i\cdot}(x)={}\\[6pt]
    \!\!{}= \left\{ \chi_{x-s_i<0}(x), \beta_i=0; \chi_{x-s_i\geq0}(x), \beta_i=1\right\}\,,\\[6pt]
\hspace*{56mm} i\in \boldsymbol{I}\,;\\[6pt]
    \!\!\upsilon_{\cdot\beta_j;\cdot j}(y)={}\\[6pt]
    \!\!{}= \left\{ 
  \chi_{y-t_j<0}(y), \beta_j=0; \chi_{y-t_j\geq 0} (y), \beta_j=1\right\}, \\[6pt] 
   \hspace*{56mm}j\in  \boldsymbol{J}\,.
   \end{array}
   \right\}
     \label{e15-ag}
   \end{equation}
  
  При \textit{смешанной} репликации каждое слагаемое также является 
произведением одномерных опционов и~замещается двумерным опционом. При 
этом указание типа в~индексах существенно и~применяются правила 
трансформации

\vspace*{-6pt}

\noindent
  \begin{multline}
  \boldsymbol{O}_{\beta_1;i\cdot} \boldsymbol{O}_{\beta_2;\cdot j}\to 
\boldsymbol{Z}_{\beta_1\beta_2;ij}\,;\enskip
  \boldsymbol{O}_{\beta_1;i\cdot} \to \boldsymbol{Z}_{\beta_1;i\cdot}\,;\\ 
  \boldsymbol{O}_{\beta_2;\cdot j}\to \boldsymbol{Z}_{\beta_2;\cdot j}\,;\enskip 1\to \boldsymbol{U}\,.
  \label{e16-ag}
  \end{multline}
  
  \vspace*{-3pt}
  
  Аналогично переписываются правила формирования платежных функций 
для отдельных опционов и~портфеля в~целом (с~учетом~(\ref{e15-ag})): 

\vspace*{-6pt}

\noindent
  \begin{multline}
  \boldsymbol{U}\to 1\,;\enskip 
\boldsymbol{Z}_{\beta_1\beta_2;ij}\to\upsilon_{\beta_1;i\cdot}(x)\upsilon_{\beta_2;\cdot 
j}(y)\,;\\
  \boldsymbol{Z}_{\beta_1;i\cdot}\to \upsilon_{\beta_1;i\cdot}(x)\,;\enskip
   \boldsymbol{Z}_{\beta_2;\cdot  j}\to \upsilon_{\beta_2;\cdot j}(y).
  \label{e17-ag}
  \end{multline}
  
  \vspace*{-3pt}
  
  На двумерном $\zeta$-рын\-ке в~соответствии с~чис\-лом возможных 
векторов~$\boldsymbol{\alpha}$ на\-счи\-ты\-ва\-ют\-ся четыре варианта однотипных 
базисов и~один смешанный (естественный с~заданным цент\-ром рынка). 
  
  Для каждого варианта с~\textit{однотипным} базисом и~оптимальным 
портфелем фиксируется тип~$\boldsymbol{\alpha}$, и~он приписывается всем 
$\zeta$-оп\-ци\-о\-нам варианта. В~двумерном случае таких вариантов четыре: 
$\{-1,-1\}$, $\{-1,+1\}$, $\{+1,-1\}$, $\{+1,+1\}$. Последовательным 
применением правил~(\ref{e14-ag}) ко всем страйкам для каж\-до\-го значения 
векторного параметра~$\boldsymbol{\alpha}$ находятся искомые четыре базиса. 
Поскольку в~однотипном случае для каж\-дой компоненты рын\-ка качественно 
различаются три варианта реплика-\linebreak\vspace*{-12pt}

\pagebreak

\noindent
ции~--- один общий для всех внут\-рен\-них 
страйков и~еще два для крайних, то для двумерного случая вариантов 
$3^2\hm = 9$. Во всех четырех выписываемых ниже спис\-ках принимается 
$0 \hm< i \hm< v_1$ и~$0 \hm< j \hm< v_2$. 
  
  \textit{Однотипные базисы}:
  \begin{itemize}
  \item
  с $\boldsymbol{\alpha} = \{-1,-1\}$
  \begin{align*}
  \boldsymbol{D}_{1,1}& = \boldsymbol{Z}_{1,1}; \\
  \boldsymbol{D}_{1,j}& = -\boldsymbol{Z}_{1,j-1} + \boldsymbol{Z}_{1,j}; \\
  \boldsymbol{D}_{1,v_2} &= \boldsymbol{Z}_{1,\cdot} - \boldsymbol{Z}_{1,v_2-1}; \\
  \boldsymbol{D}_{i,1}& = -\boldsymbol{Z}_{i-1,1} + \boldsymbol{Z}_{i,1}; \\
  \boldsymbol{D}_{i,j}& = \boldsymbol{Z}_{i-1,j-1} - \boldsymbol{Z}_{i-1,j} - \boldsymbol{Z}_{i,j-1} + \boldsymbol{Z}_{i,j}; \\
  \boldsymbol{D}_{i,v_2} & = -\boldsymbol{Z}_{i-1,\cdot} + \boldsymbol{Z}_{i-1,v_2-1} + \boldsymbol{Z}_{i,\cdot} - 
\boldsymbol{Z}_{i,v_2-1}; \\
  \boldsymbol{D}_{v_1,1}& = \boldsymbol{Z}_{\cdot,1} - \boldsymbol{Z}_{v_1-1,1}; \\
  \boldsymbol{D}_{v_1,j} & = -\boldsymbol{Z}_{\cdot,j-1} + \boldsymbol{Z}_{\cdot,j} + \boldsymbol{Z}_{v_1-1,j-1} - 
\boldsymbol{Z}_{v_1-1,j}; \\
  \boldsymbol{D}_{v_1,v_2} & = \boldsymbol{U} - \boldsymbol{Z}_{\cdot,v_2-1} - \boldsymbol{Z}_{v_1-1,\cdot} + \boldsymbol{Z}_{v_1-
1,v_2-1}; 
  \end{align*}
  
 \item  с $\boldsymbol{\alpha} = \{-1,+1\}$ 
  \begin{align*}
  \boldsymbol{D}_{1,1}& = -\boldsymbol{Z}_{1,1} + \boldsymbol{Z}_{1,\cdot}; \\
  \boldsymbol{D}_{1,j}& = \boldsymbol{Z}_{1,j-1} - \boldsymbol{Z}_{1,j}; \\
  \boldsymbol{D}_{1,v_2} & = \boldsymbol{Z}_{1,v_2-1}; \\
  \boldsymbol{D}_{i,1} & = \boldsymbol{Z}_{i-1,1} - \boldsymbol{Z}_{i-1,\cdot} - \boldsymbol{Z}_{i,1} + \boldsymbol{Z}_{i,\cdot}; \\
  \boldsymbol{D}_{i,j} & = -\boldsymbol{Z}_{i-1,j-1} + \boldsymbol{Z}_{i-1,j} + \boldsymbol{Z}_{i,j-1} - \boldsymbol{Z}_{i,j}; \\
  \boldsymbol{D}_{i,v_2} & = -\boldsymbol{Z}_{i-1,v_2-1} + \boldsymbol{Z}_{i,v_2-1}; \\
  \boldsymbol{D}_{v_1,1} & = \boldsymbol{U} - \boldsymbol{Z}_{\cdot,1} + \boldsymbol{Z}_{v_1-1,1} - \boldsymbol{Z}_{v_1-1,\cdot};\\ 
  \boldsymbol{D}_{v_1,j} & = \boldsymbol{Z}_{\cdot,j-1} - \boldsymbol{Z}_{\cdot,j} - \boldsymbol{Z}_{v_1-1,j-1} + 
\boldsymbol{Z}_{v_1-1,j}; \\
  \boldsymbol{D}_{v_1,v_2} & = \boldsymbol{Z}_{\cdot,v_2-1} - \boldsymbol{Z}_{v_1-1,v_2-1}; 
  \end{align*}
  
\item с $\boldsymbol{\alpha} = \{+1,-1\}$ 
  \begin{align*}
  \boldsymbol{D}_{1,1} & = -\boldsymbol{Z}_{1,1} + \boldsymbol{Z}_{\cdot,1};\\
  \boldsymbol{D}_{1,j} & = \boldsymbol{Z}_{1,j-1} - \boldsymbol{Z}_{1,j} - \boldsymbol{Z}_{\cdot,j-1} + 
\boldsymbol{Z}_{\cdot,j};\\
  \boldsymbol{D}_{1,v_2} & = \boldsymbol{U}- \boldsymbol{Z}_{1,\cdot} + \boldsymbol{Z}_{1,v_2-1} - \boldsymbol{Z}_{\cdot,v_2-1};\\
  \boldsymbol{D}_{i,1} & = \boldsymbol{Z}_{i-1,1} - \boldsymbol{Z}_{i,1};\\
  \boldsymbol{D}_{i,j} & = -\boldsymbol{Z}_{i-1,j-1} + \boldsymbol{Z}_{i-1,j} + \boldsymbol{Z}_{i,j-1} - \boldsymbol{Z}_{i,j};\\
  \boldsymbol{D}_{i,v_2} & = \boldsymbol{Z}_{i-1,\cdot} - \boldsymbol{Z}_{i-1,v_2-1} - \boldsymbol{Z}_{i,\cdot} + 
\boldsymbol{Z}_{i,v_2-1};\\
  \boldsymbol{D}_{v_1,1} & = \boldsymbol{Z}_{v_1-1,1};\\
  \boldsymbol{D}_{v_1,j} & = -\boldsymbol{Z}_{v_1-1,j-1} + \boldsymbol{Z}_{v_1-1,j};\\
  \boldsymbol{D}_{v_1,v_2} & = \boldsymbol{Z}_{v_1-1,\cdot} - \boldsymbol{Z}_{v_1-1,v_2-1};
  \end{align*}
  
\item  с $\boldsymbol{\alpha} = \{+1,+1\}$
  \begin{align*}
  \boldsymbol{D}_{1,1} & = \boldsymbol{U} + \boldsymbol{Z}_{1,1} - \boldsymbol{Z}_{1,\cdot} - \boldsymbol{Z}_{\cdot,1}; \\
  \boldsymbol{D}_{1,j}& = -\boldsymbol{Z}_{1,j-1} + \boldsymbol{Z}_{1,j} + \boldsymbol{Z}_{\cdot,j-1 } - 
\boldsymbol{Z}_{\cdot,j}; \\
  \boldsymbol{D}_{1,v_2} & = -\boldsymbol{Z}_{1,v_2-1} + \boldsymbol{Z}_{\cdot,v_2-1}; \\
  \boldsymbol{D}_{i,1} & = -\boldsymbol{Z}_{i-1,1} + \boldsymbol{Z}_{i-1,\cdot} + \boldsymbol{Z}_{i,1} - \boldsymbol{Z}_{i,\cdot};\\
     \boldsymbol{D}_{i,j} & = \boldsymbol{Z}_{i-1,j-1} - \boldsymbol{Z}_{i-1,j} - \boldsymbol{Z}_{i,j-1} + \boldsymbol{Z}_{i,j}; \\
       \boldsymbol{D}_{i,v_2} & = \boldsymbol{Z}_{i-1,v_2-1} - \boldsymbol{Z}_{i,v_2-1};\\
         \boldsymbol{D}_{v_1,1}& = -\boldsymbol{Z}_{v_1-1,1} + \boldsymbol{Z}_{v_1-1,\cdot};
  \end{align*}
  
  \noindent
    \begin{align*}
  \boldsymbol{D}_{v_1,j} & = \boldsymbol{Z}_{v_1-1,j-1} - \boldsymbol{Z}_{v_1-1,j};\\
  \boldsymbol{D}_{v_1,v_2}& = \boldsymbol{Z}_{v_1-1,v_2-1}. 
  \end{align*}
  \end{itemize}
  
  \vspace*{-8pt}
  
  Напомним, что маркер <<точка>> в~индексах инструментов указывает на 
координату безрискового актива, а~под~$\boldsymbol{Z}_{i,\cdot}$ и~$\boldsymbol{Z}_{\cdot,j}$ 
понимаются двумерные инструменты $\boldsymbol{Z}_i\times \boldsymbol{U}_2$ 
и~$\boldsymbol{U}_1\times\boldsymbol{Z}_j$ соответственно. 
  
  \textit{Смешанный базис} состоит из $5^2\hm = 25$ качественно различных 
вариантов представлений базисных инструментов, поскольку для каждой 
компоненты рынка вариантов страйков пять: два крайних, один центральный 
и~два внутренних, ниже и~выше центра. Их перечень получается применением 
правил~(\ref{e16-ag}), центральные страйки обозначаются~$i_c$ и~$j_c$, а~тип 
опционов дается в~терминах~$\boldsymbol{\beta}$:

\vspace*{-8pt}

\noindent
  \begin{align*}
  \boldsymbol{D}_{1,1} &= \boldsymbol{Z}_{00;1,1}; \\
  \boldsymbol{D}_{1,j} &= -\boldsymbol{Z}_{00;1,j-1} + \boldsymbol{Z}_{00;1,j},\ 0 < j < j_c; \\
  \boldsymbol{D}_{1,j_c} &= \boldsymbol{Z}_{0;1,\cdot} - \boldsymbol{Z}_{00;1,j_c-1} - \boldsymbol{Z}_{01;1,j_c}; \\
  \boldsymbol{D}_{1,j} &= \boldsymbol{Z}_{01;1,j-1} - \boldsymbol{Z}_{01;1,j},\  j_c < j < v_2;\\ 
  \boldsymbol{D}_{1,v_2} &= \boldsymbol{Z}_{01;1,v_2-1}; \\
  \boldsymbol{D}_{i,1} &= -\boldsymbol{Z}_{00;i-1,1} + \boldsymbol{Z}_{00;i,1},\ 0 < i < i_c; \\
  \boldsymbol{D}_{i,j} &= \boldsymbol{Z}_{00;i-1,j-1} - \boldsymbol{Z}_{00;i-1,j} - \boldsymbol{Z}_{00;i,j-1} +{}\\
  &\hspace*{10mm}{}+ \boldsymbol{Z}_{00;i,j},\enskip  0 < i < i_c,\ 0 < j < j_c;\\
  \boldsymbol{D}_{i,j_c} &= -\boldsymbol{Z}_{0;i-1,\cdot} + \boldsymbol{Z}_{0;i,\cdot} + \boldsymbol{Z}_{00;i-1,j_c-1} - {}\\
&\hspace*{-5mm}{}-\boldsymbol{Z}_{00;i,j_c-1} + \boldsymbol{Z}_{01;i-1,j_c} - \boldsymbol{Z}_{01;i,j_c},\ 0 < i < i_c; \\
  \boldsymbol{D}_{i,j} &= -\boldsymbol{Z}_{01;i-1,j-1} + \boldsymbol{Z}_{01;i-1,j} + \boldsymbol{Z}_{01;i,j-1} - {}\\
&\hspace*{10mm} {}-
\boldsymbol{Z}_{01;i,j},\ 0 < i < i_c,\  j_c < j < v_2; \\
  \boldsymbol{D}_{i,v_2} &= -\boldsymbol{Z}_{01;i-1,v_2-1} + \boldsymbol{Z}_{01;i,v_2-1},\ 0 < i < i_c; \\
  \boldsymbol{D}_{i_c,1} &= \boldsymbol{Z}_{0;\cdot,1} - \boldsymbol{Z}_{00;i_c-1,1} - \boldsymbol{Z}_{10;i_c,1}; \\
  \boldsymbol{D}_{i_c,j} &= -\boldsymbol{Z}_{0;\cdot,j-1} + \boldsymbol{Z}_{0;\cdot,j} + \boldsymbol{Z}_{00;i_c-1,j-1} - {}\\
&\hspace*{-5mm}  {}-
\boldsymbol{Z}_{00;i_c-1,j} + \boldsymbol{Z}_{10;i_c,j-1} - \boldsymbol{Z}_{10;i_c,j},\ 0 < j < j_c; \\
  \boldsymbol{D}_{i_c,j_c} &= \boldsymbol{U} - \boldsymbol{Z}_{0;i_c-1,\cdot} - \boldsymbol{Z}_{0;\cdot,j_c-1} - 
\boldsymbol{Z}_{1;i_c,\cdot} -{}\\
&{}- \boldsymbol{Z}_{1;\cdot,j_c} + \boldsymbol{Z}_{00;i_c-1,j_c-1} + \boldsymbol{Z}_{01;i_c-1,j_c} + {}\\
&\hspace*{20mm}{}+
\boldsymbol{Z}_{10;i_c,j_c-1} + \boldsymbol{Z}_{11;i_c,j_c}; \\
  \boldsymbol{D}_{i_c,j} &= \boldsymbol{Z}_{1;\cdot,j-1} - \boldsymbol{Z}_{1;\cdot,j} - \boldsymbol{Z}_{01;i_c-1,j-1} + {}\\
&  \hspace*{-2mm}
{}+
\boldsymbol{Z}_{01;i_c-1,j} - \boldsymbol{Z}_{11;i_c,j-1} + \boldsymbol{Z}_{11;i_c,j}, \ %\\
%&\hspace*{45mm}  
j_c < j < v_2; \\
  \boldsymbol{D}_{i_c,v_2} &= \boldsymbol{Z}_{1;\cdot,v_2-1} - \boldsymbol{Z}_{01;i_c-1,v_2-1} - \boldsymbol{Z}_{11;i_c,v_2-1}; \\
  \boldsymbol{D}_{i,1} &= \boldsymbol{Z}_{10;i-1,1} - \boldsymbol{Z}_{10;i,1},\ i_c < i < v_1; \\
  \boldsymbol{D}_{i,j} &= -\boldsymbol{Z}_{10;i-1,j-1} + \boldsymbol{Z}_{10;i-1,j} + \boldsymbol{Z}_{10;i,j-1} -{}\\
&  \hspace*{10mm}{}- 
\boldsymbol{Z}_{10;i,j},\ i_c < i < v_1,\ 0 < j < j_c; \\
  \boldsymbol{D}_{i,j_c} &= \boldsymbol{Z}_{1;i-1,\cdot} - \boldsymbol{Z}_{1;i,\cdot} - \boldsymbol{Z}_{10;i-1,j_c-1} + {}\\
  & \hspace*{-2mm}
  {}+
\boldsymbol{Z}_{10;i,j_c-1} - \boldsymbol{Z}_{11;i-1,j_c} + \boldsymbol{Z}_{11;i,j_c}, \ 
%&\hspace*{45mm}
 i_c < i < v_1; \\
  \boldsymbol{D}_{i,j} &= \boldsymbol{Z}_{11;i-1,j-1} - \boldsymbol{Z}_{11;i-1,j} - \boldsymbol{Z}_{11;i,j-1} +{}\\
& \hspace*{10mm} {}+ \boldsymbol{Z}_{11;i,j},\  i_c < i < v_1,\  j_c < j < v_2; \\
\boldsymbol{D}_{i,v_2} &= \boldsymbol{Z}_{11;i-1,v_2-1} - \boldsymbol{Z}_{11;i,v_2-1},\ i_c < i < v_1; 
  \end{align*}
  
  \vspace*{-16pt}
  
 \noindent
    \begin{align*}
    \boldsymbol{D}_{v_1,1} &= \boldsymbol{Z}_{10;v_1-1,1}; \\
  \boldsymbol{D}_{v_1,j} & = -\boldsymbol{Z}_{10;v_1-1,j-1} + \boldsymbol{Z}_{10;v_1-1,j},\ 0 < j < j_c; 
  \end{align*}
  
  \vspace*{-12pt}
    
    \pagebreak
    
    
  \noindent
    \begin{align*}
          \boldsymbol{D}_{v_1,j_c} &= \boldsymbol{Z}_{1;v_1-1,\cdot} - \boldsymbol{Z}_{10;v_1-1,j_c-1} - \boldsymbol{Z}_{11;v_1-1,j_c}; \\
  \boldsymbol{D}_{v_1,j} &= \boldsymbol{Z}_{11;v_1-1,j-1} - \boldsymbol{Z}_{11;v_1-1,j}, \  j_c < j < v_2; \\
  \boldsymbol{D}_{v_1,v_2}& = \boldsymbol{Z}_{11;v_1-1,v_2-1}. 
  \end{align*}
  
  
  \vspace*{-3pt}
  
  \noindent
  В этом списке присутствуют обозначения инструментов~$\boldsymbol{Z}$ 
с~четырьмя и~тремя индексами. В~первой группе пара индексов до точки 
с~запятой означает тип двумерного $\zeta$-оп\-ци\-о\-на~(\ref{e16-ag}), а~после 
нее~--- его страйк. Во второй группе даются одномерные версии двумерных 
$\zeta$-оп\-ци\-о\-нов. Индекс до точки с~запятой означает тип опциона, индекс 
после нее~--- его страйк. 
  
  Тем самым получены пять вариантов репликаций сценарного базиса, 
выражающих все сценарные индикаторы в~терминах $\zeta$-оп\-ци\-о\-нов, 
ко\-ти\-ру\-ющих\-ся на $\zeta$-рын\-ке. По ним можно находить\linebreak и~\mbox{цены} базисных 
индикаторов и~оптимального портфеля в~целом. 
  
  Для проверки правильности расчетов использовались несколько тестов. Один 
аналитически вычисляет сумму базисных инструментов: она должна быть 
равной двумерному единичному безрисковому активу. Другой выборочно 
проверяет равенство для каждой пары цен $(x, y)$ портфельных доходов по 
всем пяти вариантам репликаций между собой. Наконец, то же совпадение 
устанавливается по графикам доходов просто визуально. 

%\vspace*{-9pt}
  
  \section{Иллюстративный пример}
  
%  \vspace*{-2pt}
  
  Для построения мер ${\sf C}\{\cdot\}$ и~${\sf P}\{\cdot \}$ в~примере 
предлагается удобное средство аналитического задания двумерных функций 
распределения на основе одномерных ${\sf F}_{\sf X}(x)$ и~${\sf F}_{\sf Y}(y)$: 

\vspace*{-6pt}

\noindent
  \begin{multline}
  {\sf F}(x,y) = {}\\
  \!\!\! {}= {\sf F}_{\sf X}(x){\sf F}_{\sf Y}(y)\left( 1+3r\left(1-{\sf F}_{\sf 
X}(x)\right)\left( 1-{\sf F}_{\sf Y}(y)\right)\right),
  \label{e18-ag}
  \end{multline}
  
  \vspace*{-3pt}
  
  \noindent
где параметр~$r$ отвечает за корреляционную связь компонент. В~качестве одномерных стоимостных ${\sf F}_{\sf CX}(x)$, ${\sf F}_{\sf CY}(y)$ 
и~прогнозных ${\sf F}_{\sf PX}(x)$, ${\sf F}_{\sf PY}(y)$ функций распределения 
задаются бе\-та-рас\-пре\-де\-ле\-ния с~небольшими полуцелыми значениями 
параметров на множествах ${\sf X}\hm=  [0, 1]$, ${\sf Y}\hm = [0, 1]$. 
  
  Так, для ${\sf F}_{\sf CX}(x)$ и~${\sf F}_{\sf CY}(y)$ выбираются па\-ра\-мет\-ры 
$(3/2, 2)$ и~$(3/2, 3)$ соответственно, для ${\sf F}_{\sf PX}(x)$ и~${\sf F}_{\sf 
PY}(y)$~--- $(2, 3)$ и~$(2, 4)$:

\noindent
  \begin{align*}
  {\sf F}_{\sf CX}(x)&=\fr{x^{3/2}(5-3x)}{2}\,;\\
   {\sf F}_{\sf CY} (y)&= 
\fr{y^{3/2}\left(35-42y+15y^2\right)}{8}\,;
\\
  {\sf F}_{\sf PX}(x)&= x^2\left(6-8x+3x^2\right)\,,\\
   {\sf F}_{\sf PY}(y)&=y^2\left(10-20y+15y^2-4y^3\right).
  \end{align*}
  
  Подобное сочетание функций настраивает инвестора на так называемую 
<<продажу во\-ла\-тиль\-ности>>. Принимается еще $r_c\hm = 0$ 
и~$r_p \hm= 0{,}2$, и~тогда искомые двумерные функции распределения ${\sf F}_{\sf C}(x,y)$ 
и~${\sf F}_{\sf P}(x, y)$ определяются по формуле~(\ref{e18-ag}), 
а~дифференцированием можно находить и~плотности. Однако их записи ввиду 
громоздкости здесь опускаются. Более того, вероятности сценариев 
вычисляются непосредственно из функций распределения: вероятность паре 
цен базовых активов оказаться в~прямоугольнике  
$(x_{i-1}, x_i]\times(y_{j-1}, y_j]$ равна второй разности (дискретному аналогу 
смешанной производной)

\vspace*{-6pt}

\noindent
  \begin{multline}
  {\sf F}\left( x_i, y_j\right) -{\sf F}\left( x_{i-1},y_j\right) -{\sf F}\left(  
x_i,y_{j-1}\right) +{}\\
{}+{\sf F}\left( x_{i-1},y_{i-1}\right).
  \label{e19-ag}
  \end{multline}
  
  \vspace*{-3pt}
  
  Двумерная дискретизация множества ${\sf X}\times {\sf Y}$ в~примере 
проводится при $v_1 \hm= 6$ и~$v_2\hm = 5$, центральным выбирается страйк 
$i_c \hm= 3$, $j_c \hm= 3$. Согласно~(\ref{e19-ag}) находятся матрицы 
стоимостей индикаторов сценариев и~их средних доходов (и~эквивалентные им 
записи в~форме векторов). 
  
  В примере в~качестве функции рисковых предпочтений инвестора 
выбирается функция (повышенного риска) 
$\phi(\varepsilon) \hm= \varepsilon^2$, $\varepsilon\hm\in [0, 1]$. На основании 
такой информации дискретный алгоритм оптимизации~\cite{3-ag, 5-ag} дает 
лексикографически упорядоченный вектор весов матрицы индикаторов 
сценарного базиса 

\vspace*{-6pt}

\noindent
 \begin{multline*}
  \boldsymbol{g} = \{0{,}118; 0{,}159; 0{,}00445; 0{,}0000989; 0{,}000008;\\ 
  0{,}414; 1{,}0; 0{,}228; 0{,}0151; 0{,}0000189; 0{,}0581; 0{,}788;\\
  0{,}602; 0{,}0873; 0{,}00175; 0{,}0113; 0{,}309; 0{,}495; 0{,}176;\\
  0{,}00191; 0{,}000989; 0{,}0254; 0{,}0739; 0{,}0291; 0{,}00118;\\
  0{,}0000058; 0{,}000261; 0{,}00159; 0{,}00112;  0{,}000009\}.
\end{multline*}

\vspace*{-3pt}

\noindent 
  Он порождает оптимальный сценарный портфель~(\ref{e8-ag}) 
с~инвестиционной суммой, средним доходом и~средней доходностью 
соответственно: 
  $$
  A = 0{,}291162;\enskip R = 0{,}376015;\enskip y = 0{,}291428. 
  $$
  %
  График его платежной функции дается на рис.~1. 
    Для сравнения на рис.~2 изображен график платежной функции 
оптимального портфеля на том же $\delta$-рын\-ке (при тех же стоимостной 
и~прогнозной плотностях), но при дискретизации $40\times40$, результаты для 
которой уже весьма близки к~теоретическим для континуальной модели 
$\delta$-рынка.

\begin{figure*} %fig2
\vspace*{1pt}
\begin{minipage}[t]{80mm}
  \begin{center}  
    \mbox{%
\epsfxsize=72.063mm
\epsfbox{aga-1.eps}
}
\end{center}
\vspace*{-11pt}
\Caption{Доходы оптимального сценарного портфеля при дискретизации $6\times5$}
\end{minipage}
%\end{figure*}
\hfill
%\begin{figure*} %fig1
\vspace*{1pt}
\begin{minipage}[t]{80mm}
  \begin{center}  
    \mbox{%
\epsfxsize=72.063mm
\epsfbox{aga-2.eps}
}
\end{center}
\vspace*{-11pt}
\Caption{Доходы оптимального сценарного портфеля при дискретизации $40\times40$}
\end{minipage}
\vspace*{12pt}
\end{figure*}
  
  Очевидно различие двух графиков вследствие значительного различия 
в~уровнях дискретизации модели сценарного рынка. Тем не менее в~функции 
на рис.~2 без труда угадывается сглаженная версия функции на рис.~1. 

%\pagebreak
  
  Однако цель этим еще не достигается, поскольку требуется определить 
оптимальные портфели не в~сценарных индикаторах, а~в~$\zeta$-оп\-ци\-о\-нах 
рас\-смат\-ри\-ва\-емых типов. Для нахождения таких пред\-став\-ле\-ний 
в~формулу~(\ref{e8-ag}) следует для каждой пары $(i, j)$ под\-став\-лять вместо 
индикатора~$\boldsymbol{D}_{ij}$ со\-от\-вет\-ст\-ву\-ющее ему представление 
в~$\zeta$-оп\-ци\-о\-нах с~по\-сле\-ду\-ющим упрощением получаемой суммы.

 В~результате 
получаются четыре однотипных портфеля и~один смешанный. 
  
  \textit{Оптимальные однотипные портфели}: 
\begin{itemize}
\item   
для $\boldsymbol{\alpha}\hm= \{-1,-1\}$

  \vspace*{-6pt}

\noindent
\begin{multline*}
  \hspace*{-21.2pt}\boldsymbol{G} = 0{,}000009 \boldsymbol{U} + 0{,}546 \boldsymbol{Z}_{1,1} - 0{,}618 \boldsymbol{Z}_{1,2} - 
0{,}209 \boldsymbol{Z}_{1,3} -{}\\
  \hspace*{-11pt}{}- 0{,}015 \boldsymbol{Z}_{1,4} - 0{,}0000109 \boldsymbol{Z}_{1,\cdot} + 
0{,}143 \boldsymbol{Z}_{2,1} + 0{,}586 \boldsymbol{Z}_{2,2} -{}\\
  \hspace*{-11pt}{}- 0{,}302 \boldsymbol{Z}_{2,3} -  0{,}0705 \boldsymbol{Z}_{2,4} - 0{,}00173 \boldsymbol{Z}_{2,\cdot} - 0{,}432 \boldsymbol{Z}_{3,1} + {}\\
  \hspace*{-11pt}{}+ 0{,}371 \boldsymbol{Z}_{3,2} + 0{,}196 \boldsymbol{Z}_{3,3} - 0{,}0886 \boldsymbol{Z}_{3,4} - 0{,}000155 \boldsymbol{Z}_{3,\cdot} - {}\\
  \hspace*{-11pt}{} - 0{,}274 \boldsymbol{Z}_{4,1} - 0{,}137 \boldsymbol{Z}_{4,2} + 0{,}274 \boldsymbol{Z}_{4,3} + 0{,}146 \boldsymbol{Z}_{4,4} +{}\\
  \hspace*{-11pt}{}+ 0{,}000721 \boldsymbol{Z}_{4,\cdot} - 0{,}0241 \boldsymbol{Z}_{5,1} - 0{,}0472 \boldsymbol{Z}_{5,2} + {}\\
  \hspace*{-11pt}{}+0{,}0444 \boldsymbol{Z}_{5,3} + 0{,}0268 \boldsymbol{Z}_{5,4} + 0{,}00118 \boldsymbol{Z}_{5,\cdot} -{}\\
  \hspace*{-11pt}{}- 0{,}000255 \boldsymbol{Z}_{\cdot,1} - 0{,}00133 \boldsymbol{Z}_{\cdot,2} + 0{,}000469 \boldsymbol{Z}_{\cdot,3} +{}\\
  \hspace*{-11pt}{}+ 0{,}00111\boldsymbol{Z}_{\cdot,4}; 
  \end{multline*}
    
%    \vspace*{-6pt}
    
  \item   для $\boldsymbol{\alpha} = \{-1,+1\}$
  
  \vspace*{-6pt}
  
  \noindent
\begin{multline*}
\hspace*{-11pt}\boldsymbol{G} ={}\\
\hspace*{-11pt}{}= 0{,}0000058 \boldsymbol{U} - 0{,}546 \boldsymbol{Z}_{1,1} + 0{,}618 \boldsymbol{Z}_{1,2} + 0{,}209 \boldsymbol{Z}_{1,3} +{}\\
\hspace*{-11pt}{}+ 0{,}015 \boldsymbol{Z}_{1,4} - 0{,}296 \boldsymbol{Z}_{1,\cdot} - 0{,}143 \boldsymbol{Z}_{2,1} - 0{,}586 \boldsymbol{Z}_{2,2} +{}\\
\hspace*{-11pt}{}+ 0{,}302 \boldsymbol{Z}_{2,3} +  0{,}0705 \boldsymbol{Z}_{2,4} + 0{,}356 \boldsymbol{Z}_{2,\cdot} + 0{,}432 \boldsymbol{Z}_{3,1} - {}\\
\hspace*{-11pt}{}-0{,}371 \boldsymbol{Z}_{3,2} - 0{,}196 \boldsymbol{Z}_{3,3} + 0{,}0886 \boldsymbol{Z}_{3,4} + 0{,}0468 \boldsymbol{Z}_{3,\cdot} + {}\\
\hspace*{-11pt}{}+0{,}274 \boldsymbol{Z}_{4,1} + 0{,}137 \boldsymbol{Z}_{4,2} - 0{,}274 \boldsymbol{Z}_{4,3} - 0{,}146 \boldsymbol{Z}_{4,4} +{}
\\
\hspace*{-19.50235pt}{}+ 0{,}0103 \boldsymbol{Z}_{4,\cdot} +  0{,}0241 \boldsymbol{Z}_{5,1} + 0{,}0472 \boldsymbol{Z}_{5,2} - 0{,}0444 \boldsymbol{Z}_{5,3} - {}\\
\hspace*{-11pt}{}- 0{,}0268 \boldsymbol{Z}_{5,4} + 0{,}000984 \boldsymbol{Z}_{5,\cdot} + 0{,}000255 \boldsymbol{Z}_{\cdot,1} +{}\\
\hspace*{-11pt}{}+ 0{,}00133 \boldsymbol{Z}_{\cdot,2} - 0{,}000469 \boldsymbol{Z}_{\cdot,3} - 0{,}00111 \boldsymbol{Z}_{\cdot,4}; 
 \end{multline*}

   \vspace*{-10pt}

\columnbreak
 

   
 \item 
  для $\boldsymbol{\alpha} = \{+1,-1\}$
  \vspace*{-6pt}

\noindent
\begin{multline*}
 \hspace*{-21.1084pt}\boldsymbol{G} = 0{,}000008 \boldsymbol{U} - 0{,}546\boldsymbol{Z}_{1,1} + 0{,}618 \boldsymbol{Z}_{1,2} + 
0{,}209 \boldsymbol{Z}_{1,3} +{}\\
\hspace*{-13pt}{}+ 0{,}015 \boldsymbol{Z}_{1,4} + 0{,}0000109 \boldsymbol{Z}_{1,\cdot} - 0{,}143 \boldsymbol{Z}_{2,1} - 0{,}586 \boldsymbol{Z}_{2,2} + {}\\
\hspace*{-13pt}{}+0{,}302 \boldsymbol{Z}_{2,3} + 0{,}0705 \boldsymbol{Z}_{2,4} + 0{,}00173 \boldsymbol{Z}_{2,\cdot} + 0{,}432 \boldsymbol{Z}_{3,1} - {}\\
\hspace*{-13pt}{}-0{,}371 \boldsymbol{Z}_{3,2} - 0{,}196 \boldsymbol{Z}_{3,3} + 0{,}0886 \boldsymbol{Z}_{3,4} + 0{,}000155 \boldsymbol{Z}_{3,\cdot} +{}\\
\hspace*{-13pt}{}+ 0{,}274 \boldsymbol{Z}_{4,1} + 0{,}137 \boldsymbol{Z}_{4,2} - 0{,}274 \boldsymbol{Z}_{4,3} - 0{,}146 \boldsymbol{Z}_{4,4} -{}\\
\hspace*{-13pt}{}- 0{,}000721 \boldsymbol{Z}_{4,\cdot} + 0{,}0241 \boldsymbol{Z}_{5,1} + 0{,}0472 \boldsymbol{Z}_{5,2} -{}\\
\hspace*{-13pt}{}- 0{,}0444 \boldsymbol{Z}_{5,3} - 0{,}0268 \boldsymbol{Z}_{5,4} - 0{,}00118 \boldsymbol{Z}_{5,\cdot} - 0{,}0407 \boldsymbol{Z}_{\cdot,1} + {}\\
\hspace*{-13pt}{}+ 0{,}154 \boldsymbol{Z}_{\cdot,2} + 0{,}00436 \boldsymbol{Z}_{\cdot,3} + 0{,}0000909\boldsymbol{Z}_{\cdot,4}; 
 \end{multline*}
 
   \vspace*{-6pt}
   
 \item 
  для $\boldsymbol{\alpha} = \{+1,+1\}$
  \vspace*{-3pt}

\noindent
 \begin{multline*}
  \hspace*{-13pt}\boldsymbol{G} = 0{,}118\boldsymbol{U} + 0{,}546 \boldsymbol{Z}_{1,1} - 0{,}618 \boldsymbol{Z}_{1,2} - 
0{,}209 \boldsymbol{Z}_{1,3} - {}\\
  \hspace*{-13pt}{}- 0{,}015 \boldsymbol{Z}_{1,4} + 0{,}296 \boldsymbol{Z}_{1,\cdot} + 0{,}143 \boldsymbol{Z}_{2,1} + 0{,}586 \boldsymbol{Z}_{2,2} -{}\\
  \hspace*{-13pt}{}- 0{,}302 \boldsymbol{Z}_{2,3} - 0{,}0705 \boldsymbol{Z}_{2,4} - 0{,}356 \boldsymbol{Z}_{2,\cdot} - 0{,}432 \boldsymbol{Z}_{3,1} + {}\\
  \hspace*{-13pt}{}+ 0{,}371 \boldsymbol{Z}_{3,2} + 0{,}196 \boldsymbol{Z}_{3,3} - 0{,}0886 \boldsymbol{Z}_{3,4} - 0{,}0468 \boldsymbol{Z}_{3,\cdot} -{}\\
  \hspace*{-13pt}{}- 0{,}274 \boldsymbol{Z}_{4,1} - 0{,}137 \boldsymbol{Z}_{4,2} + 0{,}274 \boldsymbol{Z}_{4,3} + 0{,}146 \boldsymbol{Z}_{4,4} -{}\\
  \hspace*{-13pt}{}- 0{,}0103\boldsymbol{Z}_{4,\cdot} - 0{,}0241 \boldsymbol{Z}_{5,1} - 0{,}0472 \boldsymbol{Z}_{5,2} + 0{,}0444 \boldsymbol{Z}_{5,3} + {}\\
  \hspace*{-13pt}{}+0{,}0268 \boldsymbol{Z}_{5,4} - 0{,}000984 \boldsymbol{Z}_{5,\cdot} + 0{,}0407 \boldsymbol{Z}_{\cdot,1} - 
0{,}154 \boldsymbol{Z}_{\cdot,2} - {}\\
  \hspace*{-13pt}{}- 0{,}00436\boldsymbol{Z}_{\cdot,3} - 0{,}0000909 \boldsymbol{Z}_{\cdot,4}. 
\end{multline*}
\end{itemize}

\vspace*{-3pt}
  
  В этом перечне первый из четырех портфелей образован бинарными путами 
для каждого актива, второй~--- бинарными путами для первого актива 
и~бинарными коллами~--- для второго, третий~--- бинарными коллами для 
первого актива и~бинарными путами~--- для второго, четвертый~--- бинарными 
коллами для каждого актива. Индексы показывают номера страйков.


  
  Платежные функции всех однотипных портфелей получаются по 
правилам~(\ref{e15-ag}). Проведенные
расчеты лишь подтверждают верность 
алгоритма, поскольку все они, несмотря на внешнее различие их записей, 
имеют на идеальном рынке единую платежную функцию с~тем же графиком на 
рис.~1. 
  
  \textit{Оптимальный смешанный портфель} строится вновь по 
формуле~(\ref{e8-ag}), но в~смешанном базисе естественного происхождения 
с~выделенным центральным страйком~$(3, 3)$. Центр рынка образует свои 
четыре квадранта, в~каждом из которых в~портфеле используются однотипные 
двумерные $\zeta$-оп\-ци\-о\-ны. В~первом квадранте для каждого актива 
используются бинарные коллы, во втором~--- бинарные путы для первого 
актива и~бинарные коллы~--- для второго, в~третьем~--- бинарные путы для 
каждого актива, в~четвертом~--- бинарные коллы для первого актива 
и~бинарные путы~--- для второго. 
  
  Вычисления с~применением~(\ref{e8-ag}) дают оптимальный 
\textit{смешанный} портфель 
\begin{multline*}
  \boldsymbol{G} = 0{,}602 \boldsymbol{U} - 0{,}224 \boldsymbol{Z}_{0;1\cdot} - 0{,}374\boldsymbol{Z}_{0;2\cdot} - 
  0{,}73 \boldsymbol{Z}_{0;\cdot 1} + {}\\
{}+0{,}186 \boldsymbol{Z}_{0;\cdot 2} - 0{,}107 \boldsymbol{Z}_{1;3\cdot} - 0{,}421 \boldsymbol{Z}_{1;4\cdot} 
- 0{,}0723 \boldsymbol{Z}_{1;5\cdot} -{}\\
{}- 0{,}515\boldsymbol{Z}_{1;\cdot 3} - 0{,}0856\boldsymbol{ Z}_{1;\cdot 4} + 0{,}546 \boldsymbol{Z}_{00;1,1} -{}\\
{}- 0{,}618 \boldsymbol{Z}_{00;1,2} + 0{,}143 \boldsymbol{Z}_{00;2,1} + 0{,}586 \boldsymbol{Z}_{00;2,2} +{}\\
{}+ 0{,}209 \boldsymbol{Z}_{01;1,3} + 0{,}015 \boldsymbol{Z}_{01;1,4} + 0{,}302 \boldsymbol{Z}_{01;2,3} + {}\\
{}+0{,}0705 \boldsymbol{Z}_{01;2,4} + 0{,}432 \boldsymbol{Z}_{10;3,1} - 0{,}371\boldsymbol{Z}_{10;3,2} +{}\\
{}+ 0{,}274 \boldsymbol{Z}_{10;4,1} + 0{,}137\boldsymbol{Z}_{10;4,2} + 0{,}0241 \boldsymbol{Z}_{10;5,1} + {}\\
{}+0{,}0472\boldsymbol{Z}_{10;5,2} + 0{,}196 \boldsymbol{Z}_{11;3,3} - 0{,}0886 \boldsymbol{Z}_{11;3,4} +{}\\
{}+ 0{,}274 \boldsymbol{Z}_{11;4,3} + 0{,}146 \boldsymbol{Z}_{11;4,4} + 0{,}0444 \boldsymbol{Z}_{11;5,3} +{}\\
{}+ 0{,}0268 \boldsymbol{Z}_{11;5,4}. 
\end{multline*}
  Здесь первые девять инструментов вида $\boldsymbol{Z}_{\beta;i,\cdot}$ 
или~$\boldsymbol{Z}_{\beta;\cdot,j}$ отвечают одномерным версиям  
$\zeta$-оп\-ци\-о\-нов и~снабжены двумя индексами и~маркером <<точка>>. 
Индекс $\beta\hm=0$ означает пут, $\beta\hm = 1$~--- колл, индексы~$i$ и~$j$ 
показывают номера страйков. Остальные восемнадцать инструментов~--- 
двумерные $\zeta$-оп\-ци\-о\-ны с~двумя индексами до точки с~запятой, 
показывающими их тип, и~двумя индексами после нее с~номерами страйков. 
  
  Платежная функция смешанного портфеля строится по  
правилам~(\ref{e17-ag}). Ее описывает все тот же график на рис.~1. 
  
  \section{Заключение }
  
  В работе решена задача алгоритмического нахождения представлений 
индикаторов многомерного сценарного базиса в~терминах  
$\zeta$-оп\-ци\-о\-нов~--- многомерного обобщения бинарных опционов. На 
конкретном примере двумерного рынка продемонстрирована работа этого 
алгоритма и~ее результат. Подобные расчеты могут быть реализованы без 
принципиальных трудностей и~для рынков большей размерности. Имеет смысл 
решить такую же задачу для более сложных рынков $\alpha$-оп\-ци\-о\-нов~--- 
многомерного обобщения обычных коллов и~путов. 
  
  Тем не менее <<проклятие размерности>> не отменяется. Вычислительная 
сложность задачи быстро возрастает с~ростом размерности и~степени сценарной 
детализации. Число качественно различающихся базисных инструментов 
растет как $3^n$ для однотипных случаев и~$5^n$~--- для смешанных, а также 
растет и~общее число страйков $\prod_{i\in N} v_i$. Представляется, что при 
наличии адекватного спроса на использование CC-VaR на многомерных рынках 
лучше торговать непосредственно не $\zeta$-оп\-ци\-о\-на\-ми, а~индикаторами 
многомерных сценариев. Все это лишь подчеркивает целесообразность 
проведенного исследования. 
  
{\small\frenchspacing
 {%\baselineskip=10.8pt
 %\addcontentsline{toc}{section}{References}
 \begin{thebibliography}{9}
  \bibitem{1-ag}
  \Au{Agasandian G.\,A.} Optimal behavior of an investor in option market~//  Joint Conference 
(International) on Neural Networks Proceedings.~--- Piscataway, NJ, USA: IEEE, 2002. Vol.~2. 
P.~1859--1864. 
  \bibitem{2-ag}
  \Au{Агасандян Г\, А.} Применение континуального критерия VaR на финансовых 
рынках.~--- М.: ВЦ РАН, 2011. 299~с. 
  \bibitem{3-ag}
  \Au{Агасандян Г.\,А.} Континуальный критерий VaR на многомерных рынках 
опционов.~--- М.: ВЦ РАН, 2015. 297~с. 
  \bibitem{4-ag}
  \Au{Агасандян Г.\,А.} Континуальный критерий VaR и~оптимальный портфель 
инвестора~// Управление большими системами, 2018. 
Вып.~73. С.~6--26.
  \bibitem{5-ag}
  \Au{Агасандян Г.\,А.} Континуальный критерий VaR на сценарных рынках~// 
Информатика и~её применения, 2018. Т.~12. Вып.~1. С.~32--40. 
 
  \bibitem{7-ag}
  \Au{Brigo D., Mercuri~F., Rapisarda~F., Scotti~R.} Approximated moment matching 
dynamics for basket-options pricing~// Quant. Financ., 2004. Vol.~4. Iss.~1. P.~1--16.

 \bibitem{6-ag}
  \Au{Alexander C., Venkatramanan~A.} Analytic approximations for multi-asset option 
pricing~// Math. Financ., 2012. Vol.~22. Iss.~4. P.~667--689.
 
  \bibitem{8-ag}
  \Au{Крамер Г.} Математические методы статистики~/ Пер. с~англ.~--- М.: Мир, 1975. 
750~с. (\Au{Cramer~H.} Mathematical methods of statistics.~--- Princeton, NJ, USA: Princeton 
University Press, 1946. 575~p.)
\end{thebibliography}

 }
 }

\end{multicols}

\vspace*{-8pt}

\hfill{\small\textit{Поступила в~редакцию 24.03.21}}

%\vspace*{8pt}

%\pagebreak

\newpage

\vspace*{-28pt}

%\hrule

%\vspace*{2pt}

%\hrule

%\vspace*{-2pt}

\def\tit{MULTIDIMENSIONAL BINARY MARKETS AND~CC-VaR}


\def\titkol{Multidimensional binary markets and~CC-VaR}


\def\aut{G.\,A.~Agasandyan}

\def\autkol{G.\,A.~Agasandyan}

\titel{\tit}{\aut}{\autkol}{\titkol}

\vspace*{-8pt}


\noindent
Federal Research Center ``Computer Science and Control'' of the Russian Academy 
of Sciences, 44-2~Vavilov Str., Moscow 119333, Russian Federation

\def\leftfootline{\small{\textbf{\thepage}
\hfill INFORMATIKA I EE PRIMENENIYA~--- INFORMATICS AND
APPLICATIONS\ \ \ 2022\ \ \ volume~16\ \ \ issue\ 2}
}%
 \def\rightfootline{\small{INFORMATIKA I EE PRIMENENIYA~---
INFORMATICS AND APPLICATIONS\ \ \ 2022\ \ \ volume~16\ \ \ issue\ 2
\hfill \textbf{\thepage}}}

\vspace*{3pt} 

  
  \Abste{The work further investigates the problems of using the continuous VaR-criterion 
  (CC VaR) in financial markets. It deals with some technical problems arising in multidimensional
   markets~--- markets generated by several stochastically connected 
   underliers. The multidimensional extension of binary markets,
    a~simplified markets variant of traditional options such as calls and puts, is considered. 
   They are also the simplest extension of scenario markets in discrete-on-instruments markets. 
   Based on the supposition that scenario indicators are not fully traded in the market directly, 
   an approach to replicating such indicators by binary instruments is suggested. 
   This approach is based on the parity theorems for one-dimensional markets. 
   It is formed for multidimensional markets and is described in details for two-dimensional markets. 
   The constructions of bases for both single-type versions and the natural mixed version with 
   a~fixed market center are given. The theoretical constructions with optimal portfolios
    representations in these bases 
  are illustrated on the example of a~specific two-dimensional market.} 
  
  \KWE{underliers; multidimensional market; investor's risk preferences function; 
  continuous VaR-criterion; cost and forecast densities; scenario indicators; bases; binary options; 
  one-type portfolio; market center; mixed portfolio}
 
\DOI{10.14357/19922264220201}

%\vspace*{-16pt}

%\Ack
%\noindent




%\vspace*{4pt}

  \begin{multicols}{2}

\renewcommand{\bibname}{\protect\rmfamily References}
%\renewcommand{\bibname}{\large\protect\rm References}

{\small\frenchspacing
 {%\baselineskip=10.8pt
 \addcontentsline{toc}{section}{References}
 \begin{thebibliography}{9}
  \bibitem{1-ag-1}
  \Aue{Agasandian, G.\,A.} 2002. Optimal behavior of an investor in option market. 
\textit{Joint Conference (International) on Neural Networks Proceedings}. Piscataway, 
NJ: IEEE. 2:1859--1864. 
  \bibitem{2-ag-1}
  \Aue{Agasandyan, G.\,A.} 2011. \textit{Primenenie kontinual'nogo kriteriya VaR 
na finansovykh rynkakh} [Application of continuous VaR-criterion in financial 
markets]. Moscow: CC RAS. 299~p.
  \bibitem{3-ag-1}
  \Aue{Agasandyan, G.\,A.} 2015. \textit{Kontinual'nyy kriteriy VaR na 
mnogomernykh rynkakh optsionov} [Continuous VaR-criterion in multidimensional 
option markets]. Moscow: CC RAS. 297~p. 
  \bibitem{4-ag-1}
  \Aue{Agasandyan, G.\,A.} 2018. Kontinual'nyy kriteriy VaR i~optimal'nyy 
portfel' investora [Continuous VaR-criterion and investor's optimal portfolio]. 
\textit{Upravleniye bol'shimi sistemami} [Large-Scale Systems Control] 73:6--26. 
  \bibitem{5-ag-1}
  \Aue{Agasandyan, G.\,A.} 2018. Kontinual'nyy kriteriy VaR na stsenarnykh 
rynkakh [Continuous VaR-criterion in scenario markets]. \textit{Informatika i~ee 
Primeneniya~--- Inform. Appl.} 12(1):32--40. 
  
  \bibitem{7-ag-1}
  \Aue{Brigo, D., F.~Mercurio, F.~Rapisarda, and R.~Scotti.} 2004. Approximated 
moment matching dynamics for basket-options pricing. \textit{Quant. Financ.} 
4(1):1--16. 

\bibitem{6-ag-1}
  \Aue{Alexander, C., and A.~Venkatramanan.} 2012. Analytic approximations for  
multi-asset option pricing. \textit{Math. Financ.} 22(4):667--689.

  \bibitem{8-ag-1}
  \Aue{Cramer, H.} 1946. \textit{Mathematical methods of statistics}. Princeton, 
NJ: Princeton University Press. 575~p.
\end{thebibliography}

 }
 }

\end{multicols}

\vspace*{-6pt}

\hfill{\small\textit{Received March 24, 2021}}
  
  \Contrl
  
  \noindent
  \textbf{Agasandyan Gennady A.} (b.\ 1941)~--- Doctor of Science in physics and 
mathematics, leading scientist, A.\,A.~Dorodnicyn Computing Center, Federal 
Research Center ``Computer Science and Control''of the Russian Academy of 
Sciences, 40~Vavilov Str., Moscow 119333, Russian Federation; 
\mbox{agasand17@yandex.ru}
  

   

\label{end\stat}

\renewcommand{\bibname}{\protect\rm Литература}    