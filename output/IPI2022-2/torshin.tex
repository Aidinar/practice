\def\stat{torshin}

\def\tit{О ПРИМЕНЕНИИ ТОПОЛОГИЧЕСКОГО ПОДХОДА К~АНАЛИЗУ ПЛОХО ФОРМАЛИЗУЕМЫХ 
ЗАДАЧ ДЛЯ ПОСТРОЕНИЯ АЛГОРИТМОВ ВИРТУАЛЬНОГО СКРИНИНГА  
КВАНТОВО-МЕХАНИЧЕСКИХ СВОЙСТВ\\ ОРГАНИЧЕСКИХ МОЛЕКУЛ II: СОПОСТАВЛЕНИЕ 
ФОРМАЛИЗМА С~КОНСТРУКТАМИ КВАНТОВОЙ МЕХАНИКИ И~ЭКСПЕРИМЕНТАЛЬНАЯ 
АПРОБАЦИЯ\\ ПРЕДЛОЖЕННЫХ АЛГОРИТМОВ$^*$}

\def\titkol{О применении топологического подхода к~анализу плохо формализуемых 
задач для построения алгоритмов II} % виртуального скрининга  
%квантово-механических свойств органических молекул II: Сопоставление 
%формализма с~конструктами квантовой механики и~экспериментальная 
%апробация предложенных алгоритмов}

\def\aut{И.\,Ю.~Торшин$^1$}

\def\autkol{И.\,Ю.~Торшин}

\titel{\tit}{\aut}{\autkol}{\titkol}

\index{Торшин И.\,Ю.}
\index{Torshin I.\,Yu.}


{\renewcommand{\thefootnote}{\fnsymbol{footnote}} \footnotetext[1]
{Работа выполнена при поддержке РФФИ (проекты 19-07-00356, 18-07-00944 и~20-07-00537).}}


\renewcommand{\thefootnote}{\arabic{footnote}}
\footnotetext[1]{Федеральный исследовательский центр <<Информатика и~управление>> Российской академии наук, 
\mbox{tiy135@yahoo.com}}

\vspace*{-10pt}



\Abst{Показаны соответствия между описаниями молекул в~рамках теории хемографов, 
внутренними координатами молекул и~$\psi$-функ\-ци\-ями. Полученные результаты 
сопоставимы: (1)~с~решениями одноэлектронного уравнения Шредингера (УШ) на фрагментах 
молекул с~учетом перекрывания фрагментов; (2)~c~аддитивными схемами расчета электронной 
плотности в~тео\-рии функционала электронной плотности; (3)~с~учетом интегралов 
перекрывания в~тео\-рии молекулярных орбиталей (МО). Апробация алгоритмов на выборке из 
134~тыс.\ органических молекул показала ранговые корреляции порядка 0,75 (95\%, 
достоверный интервал 0,67--0,85) между результатами расчетов по предлагаемым 
алгоритмам и~значениями исследованных кван\-то\-во-ме\-ха\-ни\-че\-ских (КМ) показателей молекул. 
Cкорость вычислений по предлагаемым алгоритмам на несколько порядков превышает 
скорость КМ вычислений, что важно для проведения 
скринингов молекул.}

\KW{алгебраический подход; хемоинформатика; размеченные графы; комбинаторный 
анализ разрешимости}

\DOI{10.14357/19922264220205}
  
%\vspace*{-3pt}


\vskip 10pt plus 9pt minus 6pt

\thispagestyle{headings}

\begin{multicols}{2}

\label{st\stat}

\section{Введение}

     В первой части статьи~[1] было показано, что при определении алфавита 
меток~$Y$ для произвольного хемогр$\acute{\mbox{а}}$фа ${X}\hm\in \mathbf{X}$ может 
быть вычислено множество $\tilde{\mathrm{Y}}({X})\hm\subset \tilde{\mathrm{Y}}$ 
всех $\chi$-це\-пей в~$X$, включающее все множества $\tilde{\mathrm{Y}}^m$ 
$\chi$-це\-пей длины~$m$ и~множество 
$\hat{\mathrm{Y}}({X})\hm\subset \hat{\mathrm{Y}}$ всех  
$\chi$-уз\-лов~X. На основании множеств $\tilde{\mathrm{Y}}({X})$ 
и~$\hat{\mathrm{Y}}({X})$ строятся соответствующие $\chi$-ин\-ва\-ри\-ан\-ты и~их 
кортежи посредством \textit{оператора вхождения множества подграфов} 
$\boldsymbol{\pi}\hm= \{ \bm{\pi}_1,\bm{\pi}_2, \ldots , \bm{\pi}_n\}\hm\subset 
\bm{\Gamma}$ в~хемограф: 
$$
\hat{\beta}[X]\bm{\pi}= \left(\vert \bm{\pi}\cap \bm{\Pi}(X)\vert \hm> 0\right).
$$
 Обозначая через 
 $$
 \hat{\bm{\beta}}\bm{\pi}= \left\{ \hat{\beta}\bm{\pi}_1, \hat{\beta}\bm{\pi}_2, \ldots , \hat{\beta}\bm{\pi}_n\right\}
 $$ 
 
 \vspace*{-2pt}
 
 \noindent
результат последовательного применения~$\hat{\beta}$ к~$\bm{\pi}$, а~через 
$\hat{\iota}[i]\bm{\pi}(\mathrm{X})$~--- $i$-й элемент  
кор\-теж-ин\-ва\-ри\-ан\-та~$\hat{\iota}\bm{\pi}$, получаем условие 
\textit{хе\-мо\-мет\-ри\-че\-ско\-го анализа}:

\noindent
     \begin{equation}
     \argmin\limits_{\{\omega_k\}} \sum\limits_{m=1}^{\vert \mathbf{X}\vert} 
\left\vert S \!\left( \sum\limits^n_{k=1} \omega_k s \left( 
\hat{\iota}[k]\hat{\bm{\beta}}[X_m]\bm{\pi}\right)\right)- T_m\right\vert,\!
     \label{e1-tor}
     \end{equation}
     
     \vspace*{-2pt}
     
     \noindent
где $T_m$~--- значения прогнозируемой величины для молекул, 
соответствующих хемографам в~обуча\-ющей выборке~$\mathbf{X}$; $S, s: 
\mathrm{R}\hm\to \mathrm{R}^+$~--- <<сгла\-жи\-ва\-ющие>> функции. При 
$S\hm\equiv 1$ и~$s\hm\equiv 1$ условие~(\ref{e1-tor}) соответствует определению 
взвешенной метрики Хэм\-минга
\begin{equation}
\rho_q(X_1,X_2) =\fr{1}{n}\sum\limits^n_{k=1}\omega_k \hat{\iota} [k] 
\hat{\bm{\beta}}[X_1] {\pi} \otimes \hat{\iota}[k]\hat{\bm{\beta}} 
[X_2]\bm{\pi}.\!
\label{e2-tor}
\end{equation}

\vspace*{-2pt}
    
В выражениях~(1) и~(2) настраиваемыми па\-ра\-мет\-ра\-ми являются 
веса~$\omega_k$, а~множество подграфов~$\bm{\pi}$ и~функции~$S$ и~$s$ в~(1) 
задаются исследователем. Множество~$\bm{\pi}$ определяется на \mbox{основании} 
\textit{операторов построения прообраза} $\chi$-це\-пи~$\alpha$, 
$\hat{\mu}_c^{-1}\alpha$, и~\textit{по\-стро\-ения прообраза $\chi$-уз\-ла}~$\kappa$, 
$\hat{\mu}_\kappa^{-1}\kappa$, так что
для~$\bm{\alpha}\hm\subset \tilde{\mathrm{Y}}$ определено 
$\hat{\bm{\mu}}_{\mathbf{c}}^{-1}\bm{\alpha} \hm= \{\hat{\mu}_c^{-1}\alpha, 
\alpha\hm\in \bm{\alpha}\}$, а для $\bm{\kappa}\hm\subset \hat{\mathrm{Y}}$~--- 
$\hat{\bm{\mu}}^{-1}_{\mathbf{\kappa}}\bm{\kappa}\hm= \{\hat{\mu}_\kappa^{-1}\kappa, 
\kappa\hm\in \bm{\kappa}\}$. Тогда~$\bm{\pi}$ опреде-\linebreak\vspace*{-12pt}

\pagebreak

\noindent
 ляется как 
$\hat{\bm{\mu}}_{\mathbf{c}}^1\bm{\alpha}$, 
$\hat{\bm{\mu}}_{\bm{\kappa}}^1\bm{\kappa}$, 
$\hat{\bm{\mu}}^1_{\mathbf{c}}\bm{\alpha}\cup 
\hat{\bm{\mu}}_{\bm{\kappa}}^1\bm{\kappa}$ с~использованием множеств 
$\chi$-це\-пей фиксированной длины $m (\tilde{\mathrm{Y}}^m)$ и~т.\,д. 

Далее 
рассмотрены соответствия между результатами применения топологического 
анализа к~хемографам и~некоторыми математическими конструктами квантовой 
механики.

\section{Основные постулаты квантовой~механики}

    Аксиоматику, лежащую в~основе математических конструкций КМ, 
удобно представлять в~виде четырех постулатов~\cite{2-tor}, анализ которых 
позволяет рассмотреть соответствия между квантовой механикой и~предлагаемым формализмом.

\smallskip

\noindent
\textbf{Постулат~1.} \textit{Состояние квантовой системы из $N$ 
микрочастиц (электронов и~ядер) полностью определяется волновой функцией 
от ра\-ди\-ус-век\-то\-ров час\-тиц и~времени $($пси-функ\-ци\-ей 
$\psi(\mathbf{x},t): \mathrm{R}^{3N+1}\hm\to \mathrm{R})$, $\mathbf{x}\hm\in 
\mathrm{R}^{3N}$~--- \textit{вектор координат в~конфигурационном 
пространстве, $t$~--- время}. Квадрат $\psi$-функ\-ции отражает плот\-ность 
вероятности состояния, заданного координатами час\-тиц. В~стационарном 
состоянии $\psi$-функ\-ция определяется как} $\psi(\mathbf{x}): 
\mathrm{R}^{3N}\hm\to \mathrm{R}$.

\smallskip

\noindent
\textbf{Постулат~2.} \textit{Наблюдаемая физическая величина~$A$ 
представима в~виде линейного оператора$\hat{A}$, так что среднее 
значение~$A$ вычисляется как 
$$
\bar{A}=\int\limits_{\mathrm{R}^{3N}} \psi^* \hat{\mathrm{A}} \psi\,d\mathbf{x}\,,
$$ где $\psi^*$  комплексно сопряжена~$\psi$}.

\smallskip

\noindent
\textbf{Постулат~3.} \textit{Изменение волновой функции во времени 
определяется УШ}. Временн$\acute{\mbox{а}}$я 
форма УШ для гамильтонана $\hat{\mathrm{H}}$ записывается как
\begin{equation*}
\hat{\mathrm{H}}\psi= i \hbar\fr{\partial \psi}{\partial t}\,,\enskip 
\hat{\mathrm{H}}=\hat{\mathrm{T}}+{V}\,.
%\label{e3-tor}
\end{equation*}
Здесь $\hat{\mathrm{T}}=\sum \hat{\mathrm{T}}_i$~--- терм кинетической энергии, 
где 
$$
\hat{\mathrm{T}}_i= \fr{\hbar^2}{2m_i}\,\Delta_i,\
\Delta_i= \fr{\partial}{\partial x_i^2}+ \fr{\partial}{\partial y_i^2}+ \fr{\partial}{\partial z_i^2}
$$
 ($m_i$~--- масса  
$i$-й частицы);  $V(\mathbf{x},t): \mathrm{R}^{3N+1}\hm\to \mathrm{R}$~--- 
терм потенциальной энергии, где $\mathbf{x}\hm\in \mathrm{R}^{3N}$~--- 
конкатенация векторов координат ядер $\mathbf{R}\hm= \left( 
\vec{\mathrm{R}}_\alpha\right)$ и~электронов $\mathbf{r}\hm= \left( 
\vec{\mathrm{r}}_i\right)$, $\vec{\mathrm{r}}_i\hm= \left(x_i, y_i, z_i\right)$.

\smallskip

\noindent
\textbf{Постулат~4.} \textit{Электроны квантовой системы неразличимы}.

\smallskip

     В гамильтоновых системах выполнено условие сохранения энергии, так 
что $V(\mathbf{x}): \mathrm{R}^{3N}\hm\to \mathrm{R}$ 
и~$\psi(\mathbf{x},t)\hm= \psi(\mathbf{x})\bm{\chi}(t)$, что соответствует 
стационарной форме УШ:
     \begin{equation*}
     \hat{\mathrm{H}}\psi(\mathbf{r},\mathbf{R})=E\psi(\mathbf{r},\mathbf{R})
\,,
    % \label{e3.1-tor}
     \end{equation*}
где $E$~--- полная энергия системы. После введения так называемого 
адиабатического приближения (Бор\-на--Оп\-пен\-гей\-ме\-ра: кинетическая 
энергия ядер пренебрежимо мала), гамильтониан УШ записывается как 
<<электронный гамильтониан>>:
\begin{multline}
\hat{\mathrm{H}}_e=\fr{1}{2} \sum\limits_i \fr{\hbar^2}{2m_e} 
\,\Delta_i(\mathbf{r})+{}\\
{}+V_{ee}(\mathbf{r})+V_{en}(\mathbf{r},\mathbf{R})+ 
V_{nn}(\mathbf{R})\,,
\label{e3.2-tor}
\end{multline}
где $i$-суммирование проводится по электронам ($m_e$); 
$V_{ee}(\mathbf{r})\hm= (1/2)\sum\nolimits_i \sum\nolimits_{j\not= i} 1/d_{ij}$, 
$d_{ij}\hm= \| \vec{\mathbf{r}}_i\hm- \vec{\mathbf{r}}_j\|$~--- терм  
внут\-рен\-ней энергии межэлектронного взаимодействия; $V_{en}\hm=
 -\sum\nolimits_{i,\alpha} Z_{\alpha}/\mathbf{R}_{\alpha i}$,\linebreak 
$\mathbf{R}_{\alpha i} \hm= \|\vec{\mathbf{R}}_\alpha \hm- \vec{\mathbf{r}}_i\|$~--- 
терм элект\-рон\-но-ядер\-но\-го, а~$V_{nn}\hm= \sum\nolimits_{\alpha,\beta} 
Z_\alpha Z_\beta/\mathbf{R}_{\alpha\beta}$, $\mathbf{R}_{\alpha\beta}\hm= \| 
\vec{\mathbf{R}}_\alpha\hm- \vec{\mathbf{R}}_\beta \|$~--- терм межъядерного 
взаимодействия ($i,j$-суммирование проводится по электронам,  
а~$\alpha,\beta$-суммирование~--- по ядрам). Вычисление КМ-показателей 
молекул, рассматриваемых в~настоящей статье, основано на решениях 
уравнения~(\ref{e3.2-tor}).

     
     После нумерации частиц в~молекулярной сис\-те\-ме ${C}\hm= 
(\mathbf{r}_j) \hm\subset \mathrm{R}^3$, $j\hm= 1,\ldots , N$, оператор 
$\hat{\mathrm{P}}: 2^{\mathrm{R}^3}\hm\to \mathrm{R}^{3N}$ определяется 
как конкатенация $\hat{\mathrm{P}}\,{C}\hm= \left( \vec{\mathrm{r}}_1, 
\ldots , \vec{\mathrm{r}}_j, \ldots , \vec{\mathrm{r}}_N\right)$, так что 
$\psi(\hat{\mathrm{P}}\,{C})$ зависит от декартовых координат. По 
построению, для~$\hat{\mathrm{P}}$ всегда имеется обратный 
оператор~$\hat{\mathrm{P}}^{-1}$, ${C}\hm\equiv \hat{\mathrm{P}}^{-
1}\hat{\mathrm{P}}\mathrm{C}$. Декартовы координаты ${C}\hm= 
(\mathbf{r}_j)$ позволяют найти внутренние координаты 
$\mathbf{M}({C})\hm= (d_{ij}({C}))$ посредством такого 
$\hat{\mathrm{E}}: 2^{\mathrm{R}^3}\hm\to \mathrm{R}^{{C}_N^2}$, 
$\mathbf{M}({C})\hm= \hat{\mathrm{E}}{C}$, что $d_{ij}\hm= 
\| \vec{\mathrm{r}}_i-\vec{\mathrm{r}}_j\|$. Существует  
$\hat{\mathrm{E}}^{-1}$, так что $\mathbf{M}({C})$ позволяет 
найти~${C}$  с~точ\-ностью до афинного преобразования. Мат\-ри\-ца 
инцидентности хемографа есть факторизация~$\mathbf{M}({C})$ на 
основании правил теории химической  
связи~\cite{2-tor, 3-tor}.

\section{Интерпретация с~точки зрения теории химической 
связи}

    Совокупность $\chi$-инвариантов, в~которые вовлечена данная вершина 
$\chi$-гра\-фа, соответствующая одному из атомов молекулы, описывает 
некоторый локальный контекст данного атома в~молекуле. В~тео\-рии 
химической связи (композит классических и~КМ-пред\-став\-ле\-ний)  
гео\-мет\-рия локального окружения атома описывается на основании 
гибридных состояний атомов, которые могут быть использованы для 
порождения алфавита~$Y$ и~словарей~$\tilde{\mathrm{Y}}$ 
и~$\hat{\mathrm{Y}}$~\cite{3-tor}. Присутствие в~(1) весов $\omega_k$ 
соответствует суммированию свойства молекулы ($k$-я переменная) по  
$\chi$-фраг\-мен\-там (каждый из которых характеризуется определенным 
вкладом в~это свойство), так что~(1) подразумевает две гипотезы: 
(1)~аддитивность и~(2)~фиксированность.

\smallskip

\noindent
\textbf{Определение~1.} Гипотеза аддитивности: свойство всей молекулы 
представимо как сумма вкладов $\chi$-ин\-ва\-ри\-ан\-тов, каждый из которых 
соответствует определенному $\chi$-фраг\-мен\-ту молекулы.

\smallskip

\noindent
\textbf{Определение~2.} Гипотеза фиксированности вклада: каждый  
$\chi$-ин\-ва\-ри\-ант вносит одинаковый вклад в~исследуемое свойство во все 
молекулы, содержащие соответствующие $\chi$-фраг\-менты.


\section{Интерпретация в~терминах одноэлектронной 
модели}
    
Адиабатическое приближение~(\ref{e3.2-tor}) подразумевает, что терм 
$V_{nn}(\mathbf{R})$ фиксирован для заданной конфигурации 
ядер~$\mathbf{R}$, так что~$\mathbf{R}$ отражает параметры\linebreak стационарного 
УШ. Одноэлектронное приближение упрощает~(\ref{e3.2-tor}) за счет 
пересмотра~$V_{ee}$, которое аппроксимируется одноэлектронными 
операторами. В~рамках одноэлектронного (\mbox{Харт\-ри--Фо\-ка}) приближения 
$\psi(\mathbf{r},\mathbf{R})$ ищется в~виде 
$$
\psi(\mathbf{r},\mathbf{R})= \prod\limits_i \psi_1\left(\vec{\mathrm{r}}_i, \mathbf{R}\right),
$$
 где каждая из 
$\psi_1(\vec{\mathrm{r}}_i,\mathbf{R})$~--- решение задачи движения одного 
электрона в~поле всех ядер, так называемого одноэлектронного УШ:
\begin{equation}
\left.
\begin{array}{c}
\hat{\mathrm{h}}_i\psi_{1,k}\left(\vec{\mathrm{r}}_i,\mathbf{R}\right) = e_{ik} 
\psi_{1,k}\left( \vec{\mathrm{r}}_i,\mathbf{R}\right)\,;\\[6pt]
\hat{\mathrm{h}}_i= \displaystyle -\fr{\hbar^2}{2m_e}\,\Delta_i -\sum\limits_\alpha 
\fr{Z_\alpha}{\mathrm{R}_{\alpha t}}+V_i(i)\,;\\[6pt]
\hat{\mathrm{H}}_e= \sum\limits_i \hat{\mathrm{h}}_i,\enskip 
{E}_k=\sum\limits_i e_{ik}\,,
\end{array}
\right\}
\label{e3.3-tor}
\end{equation}
где $V_i(i)\hm= (1/2)\sum\nolimits_{i\not= j} 1/d_{ij}$~--- терм потенциальной 
энергии, а~$\psi_{1,k}$ ортонормированы~\cite{2-tor}. В~соответствии 
с~постулатом~4 $\psi_{1,k}$ представляются в~виде определителя, 
составленного из одноэлектронных линейно независимых функций.
    
Таким образом, в~одноэлектронном приближении энергия системы 
представляется как сумма собственных значений 
фокианов~$\hat{\mathrm{h}}_i$. Хотя в~$\hat{\mathrm{h}}_i$ входят 
координаты всех ядер~$\mathbf{R}$, с~увеличением размера молекулы на 
движение электрона будут влиять только ближайшие ядра, что делает 
адекватным представление молекулы в~виде набора локальных структурных 
фрагментов. Последнее соответствует суммированию свойств молекулы по  
$\chi$-фраг\-мен\-там в~задаче~(1), т.\,е.\ $\chi$-фраг\-мен\-там в~предбазе 
$\mathbf{U}(\mathbf{X})$ сопоставлен определенный набор одноэлектронных 
волновых функций~$\psi_{1,k}$, что отвечает гипотезе аддитивности. 
Выражение~(1) усиливает ограничения на~$\psi_{1,k}$: принимается, что 
любая~$\psi_{1,k}$ одинакова в~контексте структуры различных молекул.

\section{Интерпретация в~теории молекулярных орбиталей}
    
В теории МО рассматриваются только орбитали 
валентных электронов молекулы. Молекулярные орбитали пред\-став\-ля\-ют\-ся линейными 
комбинациями (ЛК) \mbox{атомных} орбиталей (АО), что соответствует 
аппроксимации решения~(\ref{e3.3-tor}) водородоподобными  
$\psi$-функ\-ци\-ями при больших (ангстремы) и~малых (доли ангстрема) 
расстояниях~\cite{2-tor}. Пусть произвольная МО~$\Psi$ представима как ЛК 
$\Psi\hm= \sum\nolimits_i c_i \psi_i$. Подставляя эту ЛК в~выражение 
в~постулате~2 и~вводя условие нормировки~$\Psi$ для попарно 
неортогональных~$\psi_i$, получаем
\begin{equation*}
\bar{A}=\fr{\sum\nolimits_i \sum\nolimits_j c_i c_j 
{A}_{ij}} {\sum\nolimits_i \sum\nolimits_j c_i c_j S_{ij}}\,,
%\label{e4-tor}
\end{equation*}
где ${A}_{ij}=\int\nolimits_{\mathrm{R}^{3N}} \psi_i \hat{\mathrm{A}} 
\psi_j\,d\mathbf{x}$ и~${S}_{ij}\hm= \int\nolimits_{\mathrm{R}^{3N}} 
\psi_i \psi_j\,d\mathbf{x}$~--- так называемые интегралы перекрывания, 
характеризующие межэлектронные взаимодействия~$\psi_i$. Если каждая из 
функций~$\psi_i$ в~ЛК представляет ту или иную АО $i$-го атома молекулы, то 
матрица инцидентности ($m_{ij}(X)$) может рассматриваться как факторизация 
(${S}_{ij}$), где б$\acute{\mbox{о}}$льшим~${S}_{ij}$ 
соответствуют б$\acute{\mbox{о}}$льшие веса ребер~$m_{ij}$. 

     
     Пусть имеется достаточно большое множество 
хемографов~$\mathbf{X}$. Рассмотрим два фрагмента одной молекулы 
из~$\mathbf{X}$, каждый из которых соответствует определенному 
$\chi$-ин\-ва\-ри\-ан\-ту. Например, пусть $\alpha,\beta\hm\in \tilde{Y}$, так что 
определены $\hat{\mu}_c^{-1}\alpha$ и~$\hat{\mu}_c^{-1}\beta$ (случай  
с~$\chi$-уз\-ла\-ми рассматривается аналогично). Определим три 
подмножества~$\mathbf{X}$: 
     \begin{align*}
AB&= \left\{ {X}\hm\in 
\mathbf{X}\vert \hat{\mu}_c^{-1}\alpha \overline{\in}{X}, 
\hat{\mu}_c^{-1}\beta \overline{\in}{X}\right\};\\
A&= \left\{ 
{X}\hm\in \mathbf{X}\vert \hat{\mu}_c^{-1}\alpha 
\overline{\in}{X}\right\}\backslash AB;\\ 
B&= \left\{ 
{X}\hm\in \mathbf{X}\vert 
     \hat{\mu}_c^{-1}\beta\overline{\in}{X}\right\}\backslash AB\,. 
     \end{align*}
Для определенности пусть $\vert \hat{\mu}_c^{-1}\alpha\vert \hm=1$ и~$\vert  
\hat{\mu}_c^{-1}\beta\vert \hm=1$ (случаи с~$\vert \hat{\mu}_c^{-1}\alpha\vert 
\hm>1$ и~$\vert \hat{\mu}_c^{-1}\beta\vert \hm>1$ принципиально не 
отличаются).
     
     В соответствии с~определением~2 будем считать, что все  
$\chi$-фраг\-мен\-ты $\hat{\mu}_c^{-1}\alpha$ описываются одной и~той же 
$\psi$-функ\-ци\-ей $\psi_{{A}}$, а~все $\chi$-фраг\-мен\-ты 
$\hat{\mu}_c^{-1}\beta$~--- $\psi_{{B}}$. Взаимодействие 
между~$\psi_{{A}}$ и~$\psi_{{B}}$ зависит от расстояния 
$d_{{AB}}$ между ними в~каждой молекуле из $AB$: чем 
дальше расположены $\chi$-фраг\-мен\-ты, тем 
меньше~${S}_{{AB}}$ и~тем более адекватно 
описание~$\psi_{{AB}}$ как ЛК $\psi_{{A}}$ 
и~$\psi_{{B}}$. Технически~$d_{{AB}}$ между  
$\chi$-фраг\-мен\-та\-ми в~составе одной молекулы может оцениваться 
различными способами  
(сред\-нее/ми\-ни\-маль\-ное расстояние между атомами, длина наикратчайшего 
пути и~т.\,д.).     С~точки зрения теории МО подтверждением физического 
смысла гипотезы аддитивности является сле\-ду\-ющая тео\-рема.


\smallskip

\noindent
\textbf{Теорема~1} (об аддитивной коррекции взаимодействий). \textit{Пусть 
оценка взаимодействия между \mbox{$\chi$-фраг}\-мен\-та\-ми в~произвольном 
хемографе монотонно убывает при увеличении расстояния между  
$\chi$-фраг\-мен\-та\-ми и~не зависит ни от типов $\chi$-фраг\-мен\-тов, ни 
от расположения  
$\chi$-фраг\-мен\-тов в~контексте хемографа, ни от выбранного способа 
измерения расстояния между фрагментами. Тогда вклады любых двух  
$\chi$-фраг\-мен\-тов можно считать независимыми, а~поправку на 
взаимодействие между парой $\chi$-фраг\-мен\-тов учитывать как вклад 
третьего $\chi$-фраг\-мен\-та, образующего с~парой  
$\chi$-фраг\-мен\-тов связный подграф}.

\smallskip


\noindent
    Д\,о\,к\,а\,з\,а\,т\,е\,л\,ь\,с\,т\,в\,о\,.\ \ Вне зависимости от процедуры 
порождения  
$\chi$-фраг\-мен\-тов и~способа измерения расстояния между  
$\chi$-фраг\-мен\-та\-ми, если для двух фрагментов молекулы не имеется 
третьего фрагмента, соединяющего их, то вклады обоих $\chi$-фраг\-мен\-тов 
в~общее свойство молекулы можно считать независимыми и~перекрыванием 
таких удаленных орбиталей можно пренебречь 
(${S}_{AB}\hm=0$). Пусть два  
$\chi$-фраг\-мен\-та $\hat{\mu}_c^{-1}\alpha$ и~$\hat{\mu}_c^{-1}\beta$, $\vert 
\hat{\mu}_c^{-1}\alpha\vert\hm=1$ и~$\vert \hat{\mu}_c^{-1}\beta\vert\hm=1$, 
вносят вклады~$\omega_\alpha$ и~$\omega_\beta$ в~свойство произвольной 
молекулы, а~$\omega_{\alpha\beta}$~--- поправка на 
взаимодействие~$\hat{\mu}_c^{-1}\alpha$ и~$\hat{\mu}_c^{-1}\beta$ (случаю 
$\vert \hat{\mu}_c^{-1}\alpha\vert\hm>1$ соответствует $\omega_\alpha\vert 
\hat{\mu}_c^{-1}\alpha\vert$).
     
     Пусть поправка на взаимодействие $\alpha$ и~$\beta$~--- монотонно 
убывающая~$f^{-}$, $\omega_{\alpha\beta}\hm= f^-(d_{\alpha\beta})$, так что 
вклад $\alpha$ и~$\beta$ равен $\omega_\alpha\hm+ \omega_\beta\hm+ f^-
(d_{\alpha\beta})$. Для $m$ $\chi$-фраг\-мен\-тов хемографа~$\mathrm{X}$ 
вычислим расстояния $\{d_{ij}\}$ от $i$-го  
$\chi$-фраг\-мен\-та~$\mu_i$ до всех остальных $\chi$-фраг\-мен\-тов~$\mu_j$, 
$i\not= j$, так что для фрагментов с~одинаковыми распределениями $\{d_{ij}\}$ 
суммарная поправка равна $s_i\hm= \sum\nolimits_{j\not= i} f^{-}(d_{ij})$.
     
     Распределения расстояний $\{d_{ij}\}$ зависят от центральности 
расположения фрагмента. Центром графа будем считать $i$-е вершины 
с~минимальными значениями <<центральности>> $c_i\hm= 
\sum\nolimits_{j=1,m} d_{ij}$. Для фрагментов с~одинаковой 
центральностью~$c_i$ поправки~$s_i$ равны. Для двух фрагментов с~разной 
центральностью пусть $i_1$~--- более периферийный, а~$i_2$~--- более 
центральный, так что $c_{i_1}\geq c_{i_2}$ и~$s_{i_1}\hm\leq s_{i_2}$. Если 
имеется третий фрагмент~$i_3$ из $\bm{\Pi}({X})$, соединяющий 
$\chi$-фраг\-мен\-ты~$\mu_{i_1}$ и~$\mu_{i_2}$, то $c_{i_1}\hm\geq 
c_{i_3}\hm\geq c_{i_2}$ и~$s_{i_1}\hm\leq s_{i_3}\hm\leq s_{i_2}$, так что 
вклад всех трех фрагментов равен $\omega_{i_1}\hm+ \omega_{i_2} \hm+ 
\omega_{i_3}+ f^-(d_{i_1i_2})\hm+ f^-(d_{i_1i_3})\hm+ f^-(d_{i_2i_3})$. Однако 
по условию теоремы взаимодействие между $\chi$-фраг\-мен\-та\-ми не зависит 
от расположения $\chi$-фраг\-мен\-тов в~контексте хемографа, поэтому 
$s_{i_1}\hm= s_{i_3}\hm= s_{i_2}\hm= s$. Тогда сумма 
поправок~$\omega_{i_1i_2}$ по всем парам $\chi$-фраг\-мен\-тов~$\mu_{i_1}$ 
и~$\mu_{i_2}$ равна $\sum\nolimits_{i=1,m} s_i$, т.\,е.\ является произведением 
числа $\chi$-фраг\-мен\-тов хемографа~$m$ на константу~$s$. Соответственно, 
вклад~$\omega_{i_3}$ третьего фрагмента~$\mu_{i_3}$ можно рассматривать 
как поправку на взаимодействие между двумя фрагментами~$\mu_{i_1}$ 
и~$\mu_{i_2}$. Тео\-ре\-ма доказана.
     
     \vspace*{2pt}
     
     
\noindent
\textbf{Следствие~1.} Условию тео\-ре\-мы соответствуют наборы  
$\chi$-фраг\-мен\-тов, полученные пол\-ным перебором $\chi$-под\-гра\-фов 
($\chi$-це\-пей или $\chi$-уз\-лов).

     \vspace*{2pt}
     
     
\noindent
\textbf{Следствие~2.} Вычисление свойства всей молекулы осуществимо 
суммированием по $\chi$-фраг\-мен\-там. 

     \vspace*{2pt}
     
     
\noindent
\textbf{Следствие~3.} При учете взаимодействий $\chi$-фраг\-мен\-тов по 
условию теоремы в~вычисляемое свойство молекулы входит терм~$ms$, 
равный произведению числа $\chi$-фраг\-мен\-тов хемографа~$m$ на 
константу~$s$. 
     
          \vspace*{2pt}
     
     
\noindent
\textbf{Следствие~4.} При выполнении условия теоремы свойство~$W$ 
молекулы~$X$ рассчитывается по аддитивной схеме 

%\vspace*{2pt}

\noindent
$$
W=\sum\limits_{i=1}^m \omega_i+ ms\,,
$$

\vspace*{-2pt}

\noindent
 т.\,е.\ суммированием по $m$  
$\chi$-фраг\-мен\-там, $\mu_i\hm\in \bm{\Pi}(X)$. Из этого следует, что 
соответствующие $\psi$-функ\-ции~$\psi_i$ взаимно ортогональны (т.\,е.\ 
интегралы их перекрывания равны нулю).
    
\smallskip

Теорема~1 показывает, что при определенных условиях, накладываемых на 
процедуры по\-стро\-ения множеств хемоинвариантов $\tilde{\mathrm{Y}}({X})$ 
и~$\hat{\mathrm{Y}}({X})$, даже %\linebreak 
простейшая аддитивная схема 
расчета свойств молекулы, соответствующая постановке задачи 
хемореактивного анализа в~простейшей форме~(1),\linebreak 
позволяет учитывать 
взаимодействия между \mbox{$\chi$-фраг}\-мен\-та\-ми, т.\,е.\ интегралы 
перекрывания~${S}_{AB}$. Более того, добавление 
в~множества $\tilde{\mathrm{Y}}({X})/\hat{\mathrm{Y}}({X})$ 
\mbox{$\chi$-ин}\-ва\-ри\-ан\-тов, соответствующих <<третьим>> \mbox{$\chi$-фраг}\-мен\-там 
(т.\,е.\ тем, что образуют связный подграф для произвольной пары  
\mbox{$\chi$-фраг}\-мен\-тов), позволяет предполагать ортогональность  
\mbox{$\psi$-функ}\-ций, соответствующих \mbox{$\chi$-ин}\-ва\-ри\-ан\-там в~таких  
$\tilde{\mathrm{Y}}({X})/\hat{\mathrm{Y}}({X})$ 
(следствие~4).
     
     Экспериментальная верификация всего комплекса гипотез в~условии  
тео\-ре\-мы~1 заключается: (1)~в оценке корреляции между значениями КМ 
параметров молекул и~результатами \mbox{расчетов} по предлагаемой в~следствии~4 
аддитивной схеме; (2)~в~оценке расстояний между соответствующими  
\mbox{$\chi$-фраг}\-мен\-та\-ми в~множестве~$AB$ на основе эмпирической функции 
распределения (э.\,ф.\,р.)\ рас\-сто\-яний $\{d_{AB}\}$ между  
$\chi$-фраг\-мен\-та\-ми $\hat{\mu}^{-1}_c\alpha$ и~$\hat{\mu}_c^{-1}\beta$ 
с~по\-сле\-ду\-ющим анализом $\rho$-спект\-ров и~других свойств возникающих 
при этом  $\rho_L$-кон\-фи\-гу\-ра\-ций~\cite{1-tor}. 
     
    В целом, с~точки зрения тео\-рии МО подразумевается делокализация 
электронов вокруг \mbox{$\chi$-фраг}\-мен\-тов, соответствующих  
$\chi$-це\-пям/$\chi$-уз\-лам. Обобществление электрона в~случае $\chi$-уз\-ла 
вполне \mbox{представимо}, так как последний представляет ближайшее ковалентное 
окружение произвольного атома. 

\section{Интерпретация с~точки зрения теории функционала 
электронной плотности}
    
Центральная идея теории функционала электронной плотности (density 
functional theory, DFT) заключается в~переформулировке постулатов~1--4 
в~формах, включающих электронную плот\-ность сис\-те\-мы 
$$
\rho\left(\vec{\mathrm{r}}\right)= N\int\limits_{\mathrm{R}^{3N}} \vert 
\Psi(\mathbf{x})\vert^2d\mathbf{x}\,.
$$ 
Последняя имеет четкий физический смысл 
и~может быть экспериментально оценена. Вводится такое преобразование~F, 
что 
$$
\Psi(\mathbf{x})= \mathrm{F}\left(\rho(\vec{\mathrm{r}})\right); \quad
\bar{{A}}= \left\langle \mathrm{F}\left(\rho(\vec{\mathrm{r}})\right)\vert 
\hat{\mathrm{A}}\vert \mathrm{F}\left(\rho(\vec{\mathrm{r}})\right)\right\rangle,
$$
 а полная 
энергия системы в~приближении  
Бор\-на--Оп\-пен\-гей\-ме\-ра равна 
\begin{align*}
\mathrm{E}\left(\rho(\vec{\mathrm{r}})\right)&= 
\mathrm{F}_{\mathrm{НК}}\left(\rho(\vec{\mathrm{r}})\right)+ 
V_{en}(\mathrm{F}\left(\rho(\vec{\mathrm{r}})\right);\\ 
\mathrm{F}_{\mathrm{НК}}\left(\rho(\vec{\mathrm{r}})\right)&= \hat{\mathrm{T}} 
\mathrm{F}\left(\rho(\vec{\mathrm{r}})\right)+ 
V_{ee}\left(\mathrm{F}\left(\rho(\vec{\mathrm{r}})\right)\right).
\end{align*}
 Процедуры вычисления 
$\rho(\vec{\mathrm{r}})$ естественным образом допускают аддитивные модели. 
Соответственно, представление молекулы как набора $\chi$-фраг\-мен\-тов~--- 
это способ аддитивного, пофрагментного описания карты электронной 
плотности молекулы. 

\section{О применении алгоритмов для~скрининга молекул}

    Очевидно, что точность вычислений по аддитивной схеме~(1), вследствие 
весьма сильных предположений об аддитивности и~постоянстве вклада  
$\chi$-ин\-ва\-ри\-ан\-тов (определения~1 и~2), вряд ли когда-нибудь приблизится к~результатам, по\-лу\-ча\-емым 
в~рамках разработанных ранее вы\-чис\-ли\-тель\-ных 
схем КМ. Тем не менее ряд особенностей корреляционного облака точек 
\begin{multline*}
{O}(\mathbf{X})= \left\{\left({W}_m({X}), 
{T}_m({X})\right),\right.\\
\left. {X}\in \mathbf{X}\,,\enskip m\hm= 1, \ldots , \vert 
\mathbf{X}\vert\right\},
\end{multline*}
 где ${W}_m({X})$ вычисляется для всех 
${X}\hm\in\mathbf{X}$ (например, в~соответствии с~решением 
задачи~(1) или с~использованием более сложных схем алгебраического 
подхода), позволяет оценить практическую применимость получаемых 
результатов. 
     
     Во-первых, кросс-ва\-ли\-да\-ци\-он\-ные оценки коэффициента 
корреляции и~других функционалов оценки адекватности моделей на 
${O}(\mathbf{X})$ (стандартное отклонение, коэффициент 
детерминации, комбинаторные и~прочие функционалы робастного линейного 
сглаживания, различные статистические функционалы и~др.)\ позволяют 
разносторонне оценить качество оценочных расчетов величин~$T$ посредством 
модели~$W$. Во-вто\-рых, может быть оценена релевантность характеристик 
облака точек ${O}(\mathbf{X})$ для решения соответствующих задач 
классификации и~высокопроизводительного скрининга молекул \textit{in silico}. 

Сле\-ду\-ющая тео\-ре\-ма может быть полезна для планирования 
вычислительных экспериментов и~анализа полученных данных.

\smallskip

\noindent
\textbf{Теорема~2} (о скрининге). \textit{Точность классификации хемографов 
из~$\mathbf{X}$   по интересующим процентилям значений пропорциональна степени 
покрытия корреляционного облака точек ${O}(\mathbf{X}) \hm= \{ 
({W}_m, {T}_m)\}$ ячейками главной диагонали координатной 
сетки, образованной соответствующими процентилями значений}~$T$ \textit{и}~$W$. 

\smallskip
    
\noindent
Д\,о\,к\,а\,з\,а\,т\,е\,л\,ь\,с\,т\,в\,о\ проводится посредством рассмотрения в~решетке 
${L}({T}(\mathbf{X}))$ цепей ${A}(\mathbf{X})$ 
и~${A}^\prime(\mathbf{X})$, соответствующих величине~$T$ 
и~модели~$W$. Рас\-смат\-ри\-ва\-ют\-ся э.\,ф.\,р.\ 
$\mathrm{cdf}({A}(\mathbf{X}))$ 
и~$\mathrm{cdf}({A}^\prime(\mathbf{X}))$, взаимно однозначное 
соответствие между процентилями 
$\Pi(p,\mathrm{cdf}({A}(\mathbf{X})))$ 
и~$\Pi(p,\mathrm{cdf}({A}^\prime(\mathbf{X})))$, так что на основе 
${O}(\mathbf{X})$ вычисляются проценты ошибок классификации 1-го 
и~2-го типа. 

\smallskip

\noindent
\textbf{Следствие~1.} Величина коэффициента корреляции 
$r({O}(\mathbf{X}))$~--- непрямая характеристика аккуратности 
классификации по процентилям значений. 

\smallskip

\noindent
\textbf{Следствие~2.} Разность коэффициентов корреляции на обучении 
и~контроле косвенно характеризует переобученность алгоритма 
классификации.

\begin{table*}\small
\begin{center}
\begin{tabular}{|l|l|l|c|c|c|}
\multicolumn{6}{p{133mm}}{Результаты кросс-валидационного тестирования разработанных 
скрининговых алгоритмов для 15~КМ показателей молекул; $r(c)$~--- среднее значение 
рангового коэффициента корреляции на контроле, $SD(c)$~--- стандартное отклонение 
в~прогнозировании ранга КМ-показателя на контроле}\\
\multicolumn{6}{c}{\ }\\[-6pt]
\hline
\multicolumn{1}{|c|}{Константа}&\multicolumn{1}{|c} 
{КМ-показатель}&\multicolumn{1}{c|}{Единицы}&$r$&$r(c)$&$SD(c)$\\
\hline
$A$&Вращательная константа $A$&ГГц&0,77&0,73&0,18\\
$B$&Вращательная константа $B$&ГГц&0,74&0,73&0,19\\
$C$&Вращательная константа $C$&ГГц&0,72&0,71&0,20\\
$M$&Дипольный момент&Дебай&0,72&0,72&0,20\\
$\alpha$&Изотропная поляризуемость&Бор$^3$&0,69&0,67&0,21\\
HOMO&Энергия высшей занятой МО&Хартри&0,82&0,79&0,17\\
LUMO&Энергия низшей вакантной МО&Хартри&0,85&0,83&0,15\\
Gap&Зазор LUMO-HOMO&Хартри&0,86&0,83&0,15\\
$r_2$&Электронный пространственный экстент&Бор$^2$&0,67&0,67&0,21\\
$\mathrm{ZPVE}$&Вибрационная $E$ нулевого уровня&Хартри&0,85&0,85&0,15\\
$U_0$&Внутренняя энергия (0 K)&Хартри&0,69&0,67&0,21\\
$U$&Внутренняя энергия (298,15 K)&Хартри&0,69&0,67&0,21\\
$H$&Энтальпия (298,15 K)&Хартри&0,69&0,67&0,21\\
$G$&Свободная энергия (298,15 K)&Хартри&0,69&0,67&0,21\\
$C_v$&Теплоемкость (298,15 K)&кал/(М$\cdot$K)&0,75&0,75&0,19\\
\hline
\end{tabular}
\end{center}
\vspace*{-4pt}
\end{table*}

\section{Результаты экспериментальной апробации}
    
Тестирование моделей порождения информативных числовых признаков 
хемографов, основанных на решении задач типа~(1) и~соответствующих 
алгоритмов прогнозирования числовых переменных~\cite{4-tor}, было 
проведено на выборке из 134\,000 стабильных органических молекул 
с~максимум девятью тяжелыми атомами C, O, N и~F (далее~---  
134K)~\cite{5-tor}. Исходные описания хемографов в~134K представлены в~виде 
матриц ${M}(X)$, отражающих кратности химических связей. Было использовано 
множество меток~$Y$, включающее элементы декартова произведения 
химического типа элемента на заряд и~допустимые гибридизационные 
состояния атомов~\cite{6-tor}. Над~$Y$ строились инварианты из семейств 
$\hat{\iota}\hat{\bm{\beta}}[\mathbf{X}]\hat{\bm{\mu}}_{\mathbf{c}}^{-
1}\tilde{\mathrm{Y}}^n$ ($n\hm=1, \ldots , 7$), 
$\hat{\iota}\hat{\bm{\beta}}[\mathbf{X}]\hat{\bm{\mu}}_{\bm {\kappa}}^{-
1}\hat{Y}(k)$ ($k\hm=3,\ldots , 7$) 
и~$\hat{\iota}\hat{\bm{\beta}}[\mathbf{X}](\hat{\bm{\mu}}^{-1}_{\mathbf{c}} 
\tilde{\mathrm{Y}}^n \cup \hat{\bm{\mu}}^{-1}_{\bm{\kappa}} \hat{Y}(k))$. 
Результаты тестирования регулярности по Журавлеву~\cite{6-tor} позволили 
найти оптимальные значения~$k$ и~$n$ ($k \hm= 4$; $n \hm= 5$). Как 
и~в~работе~\cite{4-tor}, прогнозирование числовых величин проводилось 
алгоритмически с~\textit{линейным распознающим оператором} на основании  
кор\-теж-ин\-ва\-ри\-ан\-та $\hat{\iota}\bm{\iota}_{\mathbf{e}}\hm= \hat{\iota} 
\hat{\bm{\beta}}[\mathbf{X}](\hat{\bm{\mu}}^{-1}_{\mathbf{c}} 
\tilde{\mathrm{Y}}^5 \cup \hat{\bm{\mu}}^{-1}_{\bm{\kappa}} \hat{\mathrm{Y}}(4))$ 
и~корректором из 6~операций, настраиваемыми мультистартовой 
стохастической оптимизацией. 

    Результаты серии кросс-ва\-ли\-да\-ци\-он\-ных экспериментов 
(10~разбиений 134K в~соотношении <<слу\-чай--конт\-роль>> $6:1$) показали 
наилучшие результаты: (1)~при учете эффектов атомов водорода; 
(2)~при использовании линейного решающего правила и~единичных 
<<сглаживающих>> функций ${S,s}: \mathrm{R}\hm\to \mathrm{R}^+$; 
(3)~при использовании коррекции на число $\chi$-фраг\-мен\-тов хемографа (см.\ 
теорему~1). Результаты вычислительных экспериментов суммированы  
в~таб\-лице. 



     
     Несмотря на отличия в~аккуратности скрининговых оценок различных 
КМ-свойств молекул (см.\ таб\-ли\-цу), алгоритмы для вычисления всех 
15~свойств показали приемлемую обобщающую\linebreak способность. Последняя 
может быть косвенно охарактеризована, например, различиями между 
значениями коэффициентов корреляции~$r$ и~$r(c)$,\linebreak полученных 
соответственно на обучении и~контроле (тео\-ре\-ма~2), которые составили 
в~среднем всего~0,016 (95\%, достоверный интервал 0,003--0,041). Одним из 
лучших скрининговых алгоритмов, разработанных в~рамках топологической 
теории хемографов, оказался алгоритм вычислений ширины щели 
 LUMO-HOMO (Lowest Unoccupied and Highest Occupied Molecular Orbitals): $r \hm= 0{,}86$ на обучении и~$r(c)\hm = 0{,}83$ на контроле 
при стандартном отклонении 0,14--0,17 (см.\ рисунок). Несмотря на заметную 
<<размытость>> корреляционного облака $O(\mathbf{X})$ на рисунке, при 
скрининге молекул по LUMO-HOMO квартиль наибольших значений позволяет 
выделить 77\% соединений с~наибольшими значениями данного свойства 
молекул (тео\-ре\-ма~2).
     
     \begin{figure*} %fig1
     \vspace*{1pt}
  \begin{center}  
    \mbox{%
\epsfxsize=102.258mm
\epsfbox{tor-1.eps}
}


%\vspace*{-9pt}

     {\small Пример ранговой корреляции для ширины щели LUMO-HOMO}
     \end{center}
    % \vspace*{-6pt}
      \end{figure*}
     
     
     При экспертном анализе ошибок прогнозирования было установлено, что 
разделение всей выборки 134K на две подгруппы молекул: по\-ли\-цик\-ли\-че\-ские, 
ароматические и~алифатические \mbox{соединения} ($n \hm= 26\,765$) и~все 
остальные~--- с~раздельным обуче\-ни\-ем на каждой из подгрупп позволило 
улучшить кросс-ва\-ли\-да\-ци\-он\-ную корреляцию для зазора LUMO-HOMO 
от $r(c) \hm= 0{,}83$ до~0,88 при снижении стандартного отклонения от~0,17 
до~0,11. 
     
     Анализ весов $\omega_i$ и~значений $\varphi_\iota 
(\hat{\iota}\bm{\chi},i,\mathrm{Pr})$ (см.\ тео\-ре\-му~2 в~[1])  
$\chi$-ин\-ва\-ри\-ан\-тов позволяет выявить хемоинварианты, которые вносят 
наибольшие абсолютные вклады. Например, в~увеличение щели LUMO-HOMO 
наибольший вклад вносили хемоинварианты, содержащие \textit{атомы 
углерода с~выраженным стерическим напряжением}, \textit{алифатические 
цепи}, а~в~сужение~---  
$\pi$-\textit{сис\-те\-мы}, что полностью соответствует основам теории 
химической связи.
    
При условии проведения предварительной подготовки (конвертирование 
матрицы ин\-ци\-дент\-ности каждого хемографа~X в~множество хемоинвариантов 
$ \hat{\bm{\beta}}[{X}](\hat{\bm{\mu}}^{-1}_{\mathbf{c}} 
\tilde{\mathrm{Y}}^5({X})\cup 
\hat{\bm{\mu}}^{-1}_{\bm{\kappa}}\hat{Y}(4,{X}))\hm\subset \bm{\iota}_{\mathbf{e}}$,\linebreak 
настройка вектора па\-ра\-мет\-ров $\theta(\mathrm{Pr})$) скорость вычислений 
алгоритма $\hat{\mathrm{A}}(\theta(\mathrm{Pr}))$ становится на несколько 
порядков выше, чем высокоточные \mbox{КМ-рас}\-че\-ты. Таким образом, 
разработанные алгоритмы приемлемы для проведения виртуальных скринингов 
КМ-свойств молекул. 

\vspace*{-6pt}

\section{Заключение}

\vspace*{-2pt}
    
Предлагаемые процедуры скринингового моделирования КМ-свойств молекул 
находятся в~русле,\linebreak важном для решения задач теоретической и~практической 
химии. В~теоретической химии крайне важна разработка моделей, приемлемых 
для всех классов соединений и~поз\-во\-ля\-ющих уста\-нав\-ли\-вать 
\mbox{полуколичественные} взаимосвязи между  
элект\-рон\-но-про\-стран\-ст\-вен\-ным строением молекул и~их свойствами. 
Такие модели должны обеспечивать выделение структурных признаков, 
определяющих свойства молекул, возможные реакции молекул, прогнозировать 
эффекты модификации структуры молекулы. Необходимость таких моделей 
очевидна хотя бы потому, что число возможных органических молекул 
измеряется сотнями миллиардов и~получить данные точных количественных 
КМ-рас\-че\-тов для каждой из таких молекул не представляется возможным. 
     
     Предлагаемые в~настоящей работе <<топологические>> модели отличает 
хорошая фи\-зи\-ко-хи\-ми\-че\-ская ин\-тер\-пре\-ти\-ру\-емость (в~том чис\-ле 
в~терминах кван\-то\-вой тео\-рии) и~высокая ско\-рость \mbox{вычислений}. Эти 
особенности разработанных моделей обеспечивают возможность их 
применения для решения широкого круга задач, таких как оценка 
КМ свойств метаболитов и~олигопептидов, по\-иск/ди\-зайн 
молекул с~заданными наборами КМ-свойств в~рамках решения задач 
материаловедения, дизайн новых лекарств, репозиционирование уже известных 
лекарств и~др.

\vspace*{-6pt}

{\small\frenchspacing
 {%\baselineskip=10.8pt
 %\addcontentsline{toc}{section}{References}
 \begin{thebibliography}{9}
 
 \vspace*{-2pt}
 
  \bibitem{1-tor}
 \Au{Торшин И.\,Ю.}  О~применении топологического подхода к~анализу плохо формализуемых задач для 
построения алгоритмов виртуального скрининга кван\-то\-во-ме\-ха\-ни\-че\-ских свойств 
органических молекул~I: Основы проблемно ориентированной теории~// Информатика и~её 
применения, 2022. Т.~16. Вып.~1. С.~39--45.
  \bibitem{2-tor}
  \Au{Степанов Н.\,Ф.} Квантовая механика и~квантовая химия.~--- М.: Мир, 2001. 519~с.
  \bibitem{3-tor}
  \Au{Torshin I.\,Yu., Rudakov~K.\,V.} On the application of the combinatorial theory of 
solvability to the analysis of chemographs. Part~1: Fundamentals of modern chemical bonding 
theory and the concept of the chemograph~// Pattern Recognition Image Analysis, 2014. Vol.~24. 
No.\,1. P.~11--23.
  \bibitem{4-tor}
  \Au{Torshin I.\,Y., Rudakov~K.\,V.} On the procedures of generation of numerical features over 
partitions of sets of objects in the problem of predicting numerical target variables~// Pattern 
Recognition Image Analysis, 2019. Vol.~29. No.\,4. P.~654--667. doi: 
10.1134/S1054661819040175.
  \bibitem{5-tor}
  \Au{Ramakrishnan R., Dral~P., Rupp~M.} Quantum chemistry structures and properties of 
134~kilo molecules~// Sci-\linebreak\vspace*{-11pt} 

\columnbreak

\noindent
entific Data, 2014. Vol.~1. No.\,1. Art. 140022.  7~p. doi: 10.1038/sdata.2014.22.
  \bibitem{6-tor}
  \Au{Torshin I.\,Yu., Rudakov~K.\,V.} On the application of the combinatorial theory of 
solvability to the analysis of chemographs. Part~2. Local completeness of invariants of chemographs 
in view of the combinatorial theory of solvability~// Pattern Recognition Image Analysis, 2014. 
Vol.~24. No.\,2. P.~196--208.
\end{thebibliography}

 }
 }

\end{multicols}

\vspace*{-8pt}

\hfill{\small\textit{Поступила в~редакцию 05.04.21}}

\vspace*{8pt}

%\pagebreak

%\newpage

%\vspace*{-28pt}

\hrule

\vspace*{2pt}

\hrule

%\vspace*{-2pt}

\def\tit{ON THE APPLICATION OF~A~TOPOLOGICAL APPROACH TO~ANALYSIS OF~POORLY 
FORMALIZED PROBLEMS FOR~CONSTRUCTING ALGORITHMS FOR~VIRTUAL 
SCREENING OF~QUANTUM-MECHANICAL PROPERTIES OF~ORGANIC 
MOLECULES~II: COMPARISON OF~FORMALISM WITH~CONSTRUCTIONS 
OF~QUANTUM MECHANICS AND~EXPERIMENTAL APPROBATION 
OF~THE~PROPOSED ALGORITHMS}


\def\titkol{On the application of~a~topological approach to analysis of~poorly 
formalized problems for~constructing algorithms II}
%for~virtual 
%screening of~quantum-mechanical properties of~organic 
%molecules~II: Comparison of~formalism with~constructions 
%of~quantum mechanics and~experimental approbation 
%of~the~proposed algorithms}


\def\aut{I.\,Yu.~Torshin}

\def\autkol{I.\,Yu.~Torshin}

\titel{\tit}{\aut}{\autkol}{\titkol}

\vspace*{-8pt}


\noindent
Federal Research Center ``Computer Science and Control'' of the Russian Academy 
of Sciences, 44-2~Vavilov Str., Moscow 119333, Russian Federation

\def\leftfootline{\small{\textbf{\thepage}
\hfill INFORMATIKA I EE PRIMENENIYA~--- INFORMATICS AND
APPLICATIONS\ \ \ 2022\ \ \ volume~16\ \ \ issue\ 2}
}%
 \def\rightfootline{\small{INFORMATIKA I EE PRIMENENIYA~---
INFORMATICS AND APPLICATIONS\ \ \ 2022\ \ \ volume~16\ \ \ issue\ 2
\hfill \textbf{\thepage}}}

\vspace*{3pt} 
  



\Abste{Correspondences between descriptions of molecules in the framework of the theory 
of chemographs, internal coordinates of molecules, and $\psi$-functions are shown. 
The results obtained are comparable: 
($i$)~with the solutions of the one-electron Schr$\ddot{\mbox{o}}$dinger 
equation on fragments of molecules with allowance for the overlap of fragments; 
($ii$)~with additive schemes for calculating electron density in the electron density functional theory; and 
($iii$)~with allowance for overlap integrals in the theory of molecular orbitals. 
Approbation of the algorithms on a~sample of 134~thousand organic molecules showed rank correlations
 of the order of~0.75 (95\%, reliable interval 0.67--0.85) 
 between the results of calculations using the proposed algorithms and the values of the 
 investigated quantum mechanical properties of molecules. The calculation speed of the 
 proposed algorithms is several orders of magnitude higher than the speed of quantum mechanical 
calculations which is important for screening the molecules.}

\KWE{algebraic approach; chemoinformatics; labeled graphs; combinatorial solvability analysis}



\DOI{10.14357/19922264220205}

\vspace*{-12pt}

\Ack
\noindent
This work was supported in part by RFBR grants 19-07-00356, 18-07-00944, and 20-07-00537.

%\vspace*{-4pt}

  \begin{multicols}{2}

\renewcommand{\bibname}{\protect\rmfamily References}
%\renewcommand{\bibname}{\large\protect\rm References}

{\small\frenchspacing
 {%\baselineskip=10.8pt
 \addcontentsline{toc}{section}{References}
 \begin{thebibliography}{9}
  \bibitem{1-tor-1}
\Aue{Torshin, I.\,Yu.} 2022. O~primenenii topologicheskogo podkhoda k~analizu plokho 
formalizuemykh zadach dlya postroeniya algoritmov virtual'nogo skrininga  
kvantovo-mekhanicheskikh svoystv organicheskikh molekul I: Osno\-vy problemno 
orientirovannoy teorii [On the application of a~topological approach to analysis of poorly 
formalized problems for constructing algorithms for virtual screening of quantum-mechanical 
properties of organic molecules I: The basics of the problem-oriented theory]. \textit{Informatika 
i~ee Primeneniya~--- Inform Appl.} 15(1):39--45.
  \bibitem{2-tor-1}
\Aue{Stepanov, N.\,F.} 2001. \textit{Kvantovaya mekhanika i~kvantovaya khimiya} [Quantum 
mechanics and quantum chemistry]. Moscow: Mir. 519~p.
  \bibitem{3-tor-1}
\Aue{Torshin, I.\,Yu., and K.\,V.~Rudakov.} 2014. On the application of the combinatorial theory 
of solvability to the analysis of chemographs. Part~1: Fundamentals of modern chemical bonding 
theory and the concept of the chemograph. \textit{Pattern Recognition Image Analysis}  
24(1):11--23.
  \bibitem{4-tor-1}
\Aue{Torshin, I.\,Yu., and K.\,V.~Rudakov.} 2019. On the procedures of generation of numerical 
features over partitions of sets of objects in the problem of predicting numerical target variables. 
\textit{Pattern Recognition Image Analysis} 29(4):654--667. doi: 10.1134/S1054661819040175.
  \bibitem{5-tor-1}
\Aue{Ramakrishnan, R., P.~Dral, and M.~Rupp.} 2014. Quantum chemistry structures and 
properties of 134 kilo molecules. \textit{Scientific Data} 1(1):140022. 7~p. doi: 10.1038/ sdata.2014.22.
  \bibitem{6-tor-1}
\Aue{Torshin, I.\,Yu., and K.\,V.~Rudakov.} 2014. On the application of the combinatorial theory 
of solvability to the analysis of chemographs. Part~2: Local completeness of invariants of 
chemographs in view of the combinatorial theory of solvability. \textit{Pattern Recognition Image 
Analysis} 24(2):196--208.
\end{thebibliography}

 }
 }

\end{multicols}

\vspace*{-6pt}

\hfill{\small\textit{Received April 5, 2021}}

  \Contrl
  
  \noindent
  \textbf{Torshin Ivan Y.} (b.\ 1972)~--- Candidate of Science (PhD) in physics and mathematics, 
Candidate of Science (PhD) in chemistry, senior scientist, A.\,A.~Dorodnicyn Computing Center, 
Federal Research Center ``Computer Science and Control'' of the Russian Academy of Sciences, 
40~Vavilov Str., Moscow 119333, Russian Federation; \mbox{tiy135@yahoo.com}
  


\label{end\stat}

\renewcommand{\bibname}{\protect\rm Литература}    