\def\stat{inkova}

\def\tit{ПРИНЦИПЫ ОПИСАНИЯ ПОКАЗАТЕЛЕЙ ЛОГИКО-СЕМАНТИЧЕСКИХ ОТНОШЕНИЙ 
И~ИХ ИЕРАРХИИ$^*$}

\def\titkol{Принципы описания показателей логико-семантических отношений 
и~их иерархии}

\def\aut{А.\,А.~Дурново$^1$, О.\,Ю.~Инькова$^2$, Н.\,А.~Попкова$^3$}

\def\autkol{А.\,А.~Дурново, О.\,Ю.~Инькова, Н.\,А.~Попкова}

\titel{\tit}{\aut}{\autkol}{\titkol}

\index{Дурново А.\,А.}
\index{Инькова О.\,Ю.}
\index{Попкова Н.\,А.}
\index{Durnovo A.\,A.}
\index{Inkova O.\,Yu.}
\index{Popkova N.\,A.}


{\renewcommand{\thefootnote}{\fnsymbol{footnote}} \footnotetext[1]
{Работа выполнялась с~использованием инфраструктуры Центра коллективного пользования 
<<Высокопроизводительные вычисления и~большие данные>> (ЦКП <<Информатика>>) ФИЦ ИУ РАН 
(г.~Москва) в~рамках швей\-цар\-ско-рос\-сий\-ско\-го проекта <<Методология аннотирования в~надкорпусной базе 
данных коннекторов>> по гранту Швейцарского государственного секретариата по образованию, исследованиям 
и~инновациям.}}


\renewcommand{\thefootnote}{\arabic{footnote}}
\footnotetext[1]{Федеральный исследовательский центр <<Информатика и~управление>> Российской академии наук, 
\mbox{duralex49@mail.ru}}
\footnotetext[2]{Федеральный исследовательский центр <<Информатика и~управление>> 
Российской академии наук; Женевский университет, \mbox{olyainkova@yandex.ru}}
\footnotetext[3]{Федеральный исследовательский центр <<Информатика и~управление>> Российской академии наук, 
\mbox{natasha\_\_popkova@mail.ru}}

\vspace*{-12pt}

    
  
  \Abst{Рассматриваются возможности аннотирования в~корпусах с~дискурсивной разметкой. 
Показано, что корпуса, созданные на основе теории риторической структуры (ТРС), содержат 
только аннотации отношений связности текста, или риторических отношений (РО). Корпус 
Пенсильванского университета PDTB аннотирует, напротив, показатели отношений, как 
и~Надкорпусная база данных коннекторов (НБДК). Показано, что корпус RST Signaling Corpus (RST-SC), 
также созданный на основе ТРС, хотя и~аннотирует показатели РО, но не может совместить 
разметку РО и~их показателей в~форме единой аннотации. Эту задачу решают корпус GUM 
и~Надкорпусная база данных иерархии (НБДИ) ло\-ги\-ко-се\-ман\-ти\-че\-ских отношений (ЛСО). Последняя 
имеет ряд преимуществ: возможность поиска, получения статистики, а~также формирования 
двуязычных аннотаций. Это позволяет выявить как универсальные явления в~дискурсивной 
организации текста, так и~явления, специфичные для того или иного исследуемого языка.}
  
  \KW{надкорпусная база данных; аннотирование корпусов текстов; дискурсивные 
отношения; коннектор}

\DOI{10.14357/19922264220207}
  
\vspace*{-2pt}


\vskip 10pt plus 9pt minus 6pt

\thispagestyle{headings}

\begin{multicols}{2}

\label{st\stat}
  
\section{Введение}

\vspace*{-1pt}

    Языковые единицы, сигнализирующие, что между высказываниями, 
составляющими текст, есть некоторое дискурсивное отношение, играют важную 
роль в~обеспечении связности текс\-та при решении широкого спектра задач 
компьютерной лингвистики. Круг этих единиц широк: он включает коннекторы 
(например, сочинительные и~подчинительные союзы \textit{а}, \textit{хотя}, 
\textit{потому что}), а~также целый ряд других языковых единиц и~явлений 
(лексические средства, синтаксические конструкции, временн$\acute{\mbox{ы}}$е формы глаголов и~т.\,д.), 
которые сигнализируют о~наличии дискурсивных отношений между 
фрагментами текста (подробнее см.~[1]). Так, в~примере~(1) отношение цели 
выражено инфинитивной конструкцией~[2, с.~7]:


\noindent
(1)~[\textit{To encourage} more competition among exporting countries,] [the U.\ S.\ is 
proposing that export subsidies, including tax incentives for exporters, be phased out in 
five years.] 

\smallskip
    
Такие показатели, или сигналы, дискурсивных отношений помогают пишущему 
выразить эти отношения, а~чи\-та\-юще\-му~--- их распознать, тем самым об\-лег\-чая 
интерпретацию текста, а~так\-же его компьютерную обработку, например при 
решении задач машинного перевода. 

    В существующих до недавнего времени корпусах с~дискурсивной разметкой, 
как правило, аннотировались либо сами показатели (см., например,\linebreak 
Пенсильванский дискурсивно аннотированный корпус~--- PDTB~[3]), либо, 
напротив, дискурсивные отношения, свя\-зы\-ва\-ющие высказывания (корпуса, 
основанные на ТРС~[4--6]). Цель данного 
исследования~--- сравнить на концептуальном уров\-не два новых ресурса, 
поз\-во\-ля\-ющих совмещать оба вида разметки: корпус GUM, в~основе 
аннотирования которого лежит ТРС, и~НБДИ ЛСО, разработанную 
в~ИПИ РАН. Начнем с~краткого обзора существующих решений, затем опишем 
принципы аннотирования в~НБДИ ЛСО и~сравним ее возможности 
с~возможностями, пред\-остав\-ля\-емы\-ми корпусом GUM.

\begin{figure*} %fig1
\vspace*{1pt}
  \begin{center}  
    \mbox{%
\epsfxsize=161.735mm
\epsfbox{ink-1.eps}
}

\end{center}
\vspace*{-6pt}
\Caption{Фрагмент графа из Ru-RSTreebank~[4].
Номера соответствуют 
ЭДЕ, горизонтальные линии~--- ЭДЕ, вертикальная чер\-та~--- группировки ЭДЕ, 
а~так\-же отношения между ними, указывающие либо на отношение равноправия 
(наклонная черта), либо зависимости (направленная дуга от зависимой ЭДЕ, 
сателлита, к~главной, яд\-ру)}
\vspace*{3pt}
\end{figure*}

\section{Корпуса, основанные на~теории риторической структуры}

    Наиболее распространенной теорией, лежащей в~основе анализа 
    и~аннотирования текс\-тов, является ТРС~[7]. Созданные 
на ее основе корпуса не располагали до последнего времени программным 
обеспечением, позволяющим размечать одновременно <<риторические>> 
(в~данной терминологии) отношения, или отношения связности текста, и~их 
показатели. Это обусловлено прежде всего тем, что, согласно ТРС, текст 
с~дискурсивной точки зрения пред\-став\-ля\-ет собой иерархическую древовидную 
структуру (граф) элементарных дискурсивных единиц (ЭДЕ), которыми являются 
клаузы, и~аннотации должны учитывать слож\-ность этих древовидных структур. 

В~идеале единый граф должен охватывать текст в~целом, но на практике это 
возможно лишь для небольших по объему текс\-тов, в~большинстве случаев 
новостных газетных статей или блогов. Наиболее распространенный материал для 
английского языка~--- выборка статей из The Wall Street Journal, использующийся, 
как будет показано ниже, в~различных аннотированных корпусах. Графы для 
текстов большей величины строятся в~пределах абзацев, что 
может даже специально оговариваться в~инструкциях по аннотированию, 
например в~[8]. 
{\looseness=1

}

На рис.~1 приведен фраг\-мент графа из русского корпуса  
Ru-RSTreebank, соз\-дан\-но\-го на основе ТРС и~который отражает наиболее широкое 
представление о~РО и~их показателях. 




Первое, что следует отметить,~--- это отсутствие между ЭДЕ~1, 2 и~3  
ка\-ко\-го-ли\-бо РО, несмотря на постулат ТРС о том, что все ЭДЕ обязательно должны быть связаны между собой 
некоторым РО~[7, с.~248]. Заметим, что и~сле\-ду\-ющие блоки ЭДЕ (9--15, 16--25, 26--37, $\ldots$), а~также 
заключительная ЭДЕ 94\footnote{В силу большого объема графа для всего текста нет возможности привести его 
полностью. Он доступен по адресу {\sf https:// rstreebank.ru/text/blogs\_1}.} находятся на одном иерархическом 
уровне и~не связаны друг с~другом никаким РО. Это подтверждает необоснованность утверждения ТРС, что любой 
текст пред\-став\-ля\-ет собой единое дерево <<риторической>> за\-ви\-си\-мости. 
    
Выделяются в~самостоятельную ЭДЕ фотографии (ЭДЕ~8 IMG на рис.~1), 
которые, по мнению разработчиков Ru-RSTreebank, всегда связаны с~предыдущим 
текстом отношением Elaboration (детализация\footnote{Русские соответствия 
английских названий РО приводятся по~[9, с.~438--440].}), наименее чет\-ко 
определенным РО в~ТРС~[10]. 




    Граф отражает иерархические РО между блоками ЭДЕ~3--5 и~6--7. Эти блоки 
связаны между собой РО Joint (конъюнкция). Каждый из них содержит ЭДЕ, 
связанные причинными РО (Cause на рис.~1), расположенными соответственно 
уровнем ниже. Однако отношения, свя\-зы\-ва\-ющие ЭДЕ~6 и~7, вряд ли могут 
считаться риторическими, так как связывают ска\-зу\-емое и~синтаксически 
зависимое от него прямое дополнение. Иначе говоря, речь идет не 
о~семантической, а~о~чис\-то синтаксической за\-ви\-си\-мости.

\vspace*{-6pt}

\section{Корпуса, аннотирующие показатели дискурсивных отношений: 
PDTB и~НБДК}

    Корпус PDTB был соз\-дан для разметки именно показателей отношений~[11, 
12], по\-ни\-ма\-емых, однако, $\acute{\mbox{у}}$же, чем в~ТРС: это только 
отношения, потенциально вы\-ра\-жа\-емые коннекторами или схожими с~ними по 
функциям языковыми единицами, так называемыми альтернативными 
лексикализациями дискурсивных отношений (Alternative Lexicalization,~[13]), 
которые грамматики соответствующих языков не относят к~чис\-лу сочинительных 
или подчинительных союзов и~их аналогов; ср., например \textit{after then} 
в~примере~(2)~[14, с.~22]:

\noindent
(2)~And she further stunned her listeners by revealing her secret garden design method: 
Commissioning a~friend to spend <<five or six thousand dollars$\ldots$ on books that 
I~ultimately cut up>>. AltLex [\textit{After that}], the layout had been easy.


%\smallskip

    Понимание дискурсивного отношения в~PDTB близ\-ко к~тому, что принято 
называть ЛСО~[10, 15].\linebreak Для разметки ЛСО, 
которые могут как иметь эксплицитный показатель, так и~не иметь его (так 
на\-зы\-ва\-емые имплицитные ЛСО), используются текс\-ты уже упоминавшегося The 
Wall Street Journal\linebreak
 общим объемом чуть больше 1~млн сло\-во\-упо\-треб\-ле\-ний. 
Аннотация не имеет форму графа и~не обязательно должна охватывать весь текст. 
В~ресурсе PDTB границы фрагментов текс\-та, непосредственно связанные 
коннектором, выделяются жел\-тым (первый фрагмент) и~синим (второй, т.\,е.\ 
вводимый коннектором) цветом, а~коннектор~--- красным. Фрагменты текста, не 
задействованные в~установлении ЛСО, выделяются розовым (относятся к~первому 
фрагменту) и~фиолетовым (относятся ко второму). На рис.~2 им соответствуют 
различные оттенки серого.

    


На верхнем уровне иерархии (см.\ рис.~2,\,\textit{б}) находится ЛСО причины, 
выражаемое \textit{with}. Первый фрагмент текс\-та включает в~себя ЛСО 
(одновременности), оформляемое \textit{once} (см.\ рис.~2,\,\textit{а}). В~графе оно 
будет находиться уров\-нем ниже. Каждое из ЛСО визуализируется отдельно. При 
разметке имплицитных ЛСО используется операция под\-ста\-нов\-ки коннектора, 
который уместен по смыс\-лу (рис.~3). В~окне, по\-ка\-зы\-ва\-ющем ЛСО 
(см.\ рис.~3,\,\textit{б}) в~аннотируемом текс\-те, имплицитные ЛСО конъюнкции 
предлагается эксплицировать при помощи \textit{in addition}.




  Как и~в PDTB, разметка текста в~НБДК, разработанной в~ИПИ РАН, исходит из показателя ЛСО, а~разметка всего 
текста не обязательна. Это особенно важ\-но, поскольку НБДК использует текс\-ты, 
преимущественно художественные, из параллельных французского 
и~итальянского подкорпусов Национального корпуса русского языка 
(НКРЯ)~[17], которые имеют большие размеры. B~отличие от PDTВ, аннотация 
в~НБДК является двуязычной, т.\,е.\ аннотируется фрагмент текс\-та оригинала 
и~его перевод. Аннотация не фиксирует границ свя\-зы\-ва\-емых коннектором 
фрагментов текс\-та и~непосредственно не отражает иерархию ЛСО. Ей 
соответствуют в~аннотации признаки $\langle$SubCNT$\rangle$ (встроенный 
коннектор) и~$\langle$SuperCNT$\rangle$ (встраивающий) 
коннектор\footnote{Подробнее об архитектуре НБДК, ее функциональных 
возможностях и~ис\-поль\-зу\-емых в~ней мет\-ках см.~[18, 19].}. 
  
\begin{table*}\small %tabl1
\begin{center}
\Caption{Аннотация из НБДК для коннектора \textit{то есть}}
\vspace*{2ex}

\begin{tabular}{|p{40mm}|p{34mm}|p{40mm}|p{34mm}|}
\hline
\multicolumn{1}{|c|}{\tabcolsep=0pt\begin{tabular}{c}Контекст коннектора\\ в~оригинале\end{tabular}}&
\multicolumn{1}{c|}{\tabcolsep=0pt\begin{tabular}{c}Коннектор в~оригинале\\ и~его признаки\end{tabular}}&
\multicolumn{1}{c|}{\tabcolsep=0pt\begin{tabular}{c}Контекст коннектора\\ 
в~переводе\end{tabular}}&\multicolumn{1}{c|}{\tabcolsep=0pt\begin{tabular}{c}Коннектор в~переводе\\ и~его признаки\end{tabular}}\\
\hline
\textit{Если} он хотел жить  
по-сво\-ему, \textbf{то есть} лежать молча, дремать \textit{или} ходить по комнате, Алексеева 
как будто не было тут&
\textbf{то есть}\newline
<переформулирование$\rangle$\newline
$\langle$без предикации$\rangle$\newline
$\langle$начальная$\rangle$\newline
$\langle$p CNT q$\rangle$\newline
$\langle$CNT$\rangle$\newline
$\langle$SubCNT$\rangle$\newline
$\langle$SuperCNT$\rangle$&
S'il voulait continuer de vivre $\grave{\mbox{a}}$ sa mani$\grave{\mbox{e}}$re, 
 \textbf{c'est-$\grave{\mbox{a}}$-dire} rester couch$\acute{\mbox{e}}$ en silence, somnoler 
\textit{ou} marcher de long en large dans la chambre, c'$\acute{\mbox{e}}$tait comme si 
Alex$\acute{\mbox{e}}$ev n'$\acute{\mbox{e}}$tait pas l$\grave{\mbox{a}}$&
\textbf{c'est-$\grave{\mbox{a}}$-dire}\newline
$\langle$переформулирование$\rangle$\newline
$\langle$без предикации$\rangle$\newline
$\langle$начальная$\rangle$\newline
$\langle$p CNT q$\rangle$\newline
$\langle$CNT$\rangle$\newline
$\langle$SuperCNT$\rangle$\newline
$\langle$SubCNT$\rangle$\\
\hline
\end{tabular}
\end{center}
\vspace*{-3pt}
\end{table*}
    
Аннотация сформирована для коннектора \textit{то есть} (табл.~1), для которого 
проставлены (2-я колонка) обе метки. По отношению к~коннектору \textit{если} он 
является встроенным ($\langle$SubCNT$\rangle$), так как находится в~первом из 
связанных им фраг\-мен\-тах текс\-та. Но во фрагменте текс\-та, вводимом \textit{то 
есть}, находится коннектор \textit{или}, по отношению к~которому\linebreak он является 
встраивающим ($\langle$SuperCNT$\rangle$). В~графе на\, верх\-нем\, уровне будет 
находиться \textit{если},\,  уровнем %\linebreak\vspace*{-12pt}

%\pagebreak


\end{multicols}

\begin{figure*}[h] %fig2
\vspace*{-6pt}
  \begin{center}  
    \mbox{%
\epsfxsize=163mm
\epsfbox{ink-2.eps}
}

\end{center}
\vspace*{-13pt}
\Caption{Пример аннотации с~несколькими ЛСО из PDTB~[16]}
%\end{figure*}
%\begin{figure*} %fig3
\vspace*{4pt}
  \begin{center}  
    \mbox{%
\epsfxsize=163mm
\epsfbox{ink-3.eps}
}

\end{center}
\vspace*{-13pt}
\Caption{Пример аннотации с~имплицитными ЛСО~[16]}
\vspace*{-12pt}
\end{figure*}

\begin{multicols}{2}


\noindent
  ниже~--- \textit{то есть}, а~на еще более 
низ\-ком уровне~--- \textit{или}.
 
%\vspace*{-4pt}


\section{Корпуса, совмещающие разметку дискурсивных отношений 
и~их показателей: RST-SC, GUM и~НБДИ ЛСО}

\vspace*{-2pt}

    Изначально, как было показано, корпуса, созданные на основе ТРС, не 
предусматривали аннотирование показателей РО. Эту задачу решает корпус RST-SC. В~качестве исходного материала \mbox{берутся} 
существующие в~корпусе RST Discourse Treebank (RST-DT) графы, к~которым 
добавляются аннотации для показателей, благодаря которым каждое РО может 
быть идентифицировано. В~чис\-ло показателей помимо коннекторов включаются 
показатели разнообразной природы: лексические, морфологические (временн$\acute{\mbox{ы}}$е 
формы), семантические (синонимия, антонимия и~др.), синтаксические 
(различные виды придаточных и~др.), графические (знаки препинания и~др.)\ 
и~т.\,д.~[20]. 

    Аннотации связаны не непосредственно с~фрагментом текста, а с~отношением, 
прикрепленным к~узлу дерева. Для одного и~того же РО возможно аннотирование 
нескольких показателей, соответствующих разным словам, а~некоторые сигналы 
не соответствуют словам в~текс\-те (например, раз\-биение на абзацы, знаки 
пунктуации и~др.). Так, в~<<John is tall. Mary is short>> РО контраста выражено 
двумя видами показателей: лексическими (антонимичными прилагательными 
\textit{tall} и~\textit{schort}) и~синтаксическим (синтаксический параллелизм 
фрагментов текста)~[20, с.~153--154]. 

Результат аннотирования показателей РО 
представлен в~виде таб\-ли\-цы (которая может занимать\linebreak\vspace*{-12pt}




{ \begin{center}  %fig4
 \vspace*{-3pt}
   \mbox{%
\epsfxsize=64.133mm
\epsfbox{ink-4.eps}
}

\end{center}

\noindent
{{\figurename~4}\ \ \small{
Аннотация РО и~их показателей в~rstWeb~[22, с.~57] 
}}}

\vspace*{6pt}

\noindent
 несколько страниц), 
отражающей наличие одного или нескольких показателей, при\-над\-леж\-ность 
показателя к~классу (лексический, морфологический
 и~т.\,д.), сам показатель и~его 
функцию в~обеспечении связ\-ности текста.



    Основной вклад другого ресурса rstWeb~[21] заключается в~предостав\-ле\-нии 
нового типа аннотации в~рамках RST. Несмотря на то что новый инструмент 
аннотирования разработан на основе существующего интерфейса RST, он 
устраняет существенный пробел в~аннотировании, который не смог устранить 
RST-SC: он позволяет связать разметку РО и~их показателей в~едином формате 
аннотации (рис.~4). 


    В отличие от аннотации на рис.~1, где размечены только РО, в~аннотации на 
рис.~4 размечаются и~показатели (выделены серым на рис.~4, желтым~--- 
в~ресурсе). Для этого используется кноп\-ка~S, расположенная рядом с~РО 
и~поз\-во\-ля\-ющая выбрать соответствующий показатель из списка\footnote[1]{Кнопка 
<<X>> служит для очистки родительского узла, <<Т>>~--- для добавления 
фрагмента текста, <<$\wedge$>>~--- для создания РО. Риторические отношения редактируются при 
помощи операции drag-and-drop.}. Аннотации до\-ступ\-ны в~находящемся 
в~свободном доступе корпусе GUM, соз\-дан\-ном в~Джорджтаунском университете 
(Вашингтон)~[23].

    Надкорпусная база данных иерархии ЛСО была создана в~Институте проблем информатики РАН для 
решения двух задач аннотирования, которые\linebreak\vspace*{-12pt}

\pagebreak

\end{multicols}

\setcounter{figure}{4}
\begin{figure*} %fig5
\vspace*{1pt}
  \begin{center}  
    \mbox{%
\epsfxsize=149.61mm
\epsfbox{ink-5.eps}
}

\end{center}
\vspace*{-9pt}
\Caption{Аннотация из НБДИ ЛСО}
\end{figure*}

\begin{table*}\small %tabl2
\begin{center}
\Caption{Возможности GUM и~НБДИ ЛСО}   %\multicolumn{1}{|c|}{\raisebox{-6pt}[0pt][0pt]{
\vspace*{2ex}

\tabcolsep=3pt
\begin{tabular}{|l|p{14mm}|p{14mm}|p{33mm}|c|p{15mm}|p{35mm}|p{14mm}|}
\hline 
\multicolumn{1}{|c|}{\raisebox{-18pt}[0pt][0pt]{\tabcolsep=0pt\begin{tabular}{c}База\\ 
данных\end{tabular}}}& \multicolumn{3}{c|}{Аннотирование} & 
\multicolumn{1}{c|}{\raisebox{-18pt}[0pt][0pt]{\tabcolsep=0pt\begin{tabular}{c}Число\\ языков\end{tabular}}} &
\multicolumn{1}{c|}{\raisebox{-18pt}[0pt][0pt]{Поиск}} & \multicolumn{1}{c|}{\raisebox{-18pt}[0pt][0pt]{\tabcolsep=0pt\begin{tabular}{c}Хранение\\ информации\end{tabular}}} & 
\multicolumn{1}{c|}{\raisebox{-18pt}[0pt][0pt]{\tabcolsep=0pt\begin{tabular}{c}Стати-\\стика\end{tabular}}}\\
\cline{2-4} 
& \multicolumn{1}{c|}{\tabcolsep=0pt\begin{tabular}{c}Отно-\\ шения\end{tabular}} & \multicolumn{1}{c|}{\tabcolsep=0pt\begin{tabular}{c}Показа-\\тели\end{tabular}} & 
\multicolumn{1}{c|}{\tabcolsep=0pt\begin{tabular}{c}Видоизменение\\ исходного\\ текста\end{tabular}} &&&&\\
\hline
{\raisebox{-6pt}[0pt][0pt]{\tabcolsep=0pt\begin{tabular}{l}Корпус\\ GUM\end{tabular}}}&Весь спектр отношений связ\-ности 
(РО) & Весь спектр языковых и~неязыковых единиц &\multicolumn{1}{c|}{Нет} & \multicolumn{1}{c|}{Один} & \multicolumn{1}{c|}{Нет} & Хранение аннотаций и~истории ее 
состояния (изменения, удаление, восстановление) на сервере, до\-ступ\-ном 
зарегистрированным пользователям, а~так\-же воз\-мож\-ность установки программного 
обеспечения на персональный компьютер& \multicolumn{1}{c|}{Нет}\\
\hline
{\raisebox{-6pt}[0pt][0pt]{\tabcolsep=0pt\begin{tabular}{l}НБДИ\\ ЛСО\end{tabular}}}&ЛСО, потенциально выражаемые коннекторами& Кон\-нек\-то\-ры&Видоизменять 
ан\-но\-ти\-ру\-емые кон\-текс\-ты (в~част\-ности, соз\-да\-вать кон\-текс\-ты с~пропущенными 
фрагментами), а~так\-же работать с~пус\-ты\-ми кон\-текс\-та\-ми (содержащими только 
коннектор) &  \multicolumn{1}{c|}{Два} & Поиск по ЛСО, показателю, приз\-на\-кам кон\-текс\-та  
упо\-треб\-ле\-ния по\-ка\-за\-теля &Хранение аннотаций и~истории ее со\-сто\-яния 
(изменения, удаление, восстановление) на сервере, до\-ступ\-ном 
зарегистрированным пользователям& Ста\-ти\-сти\-ка по любому из параметров поиска, а~также SQL-за\-просы\\
\hline
\end{tabular}
\end{center}
\vspace*{-3pt}
\end{table*}


\begin{multicols}{2}

\noindent
 недоступны в~ранее создан\-ной 
НБДК: отражение иерархии ЛСО, обес\-пе\-чи\-ва\-ющих связ\-ность текс\-та, 
и~визуализация границ фраг\-мен\-тов текс\-та, связанных ЛСО\footnote[1]{Подробнее 
об архитектуре НБДИ ЛСО см.~[24].} (рис.~5). Верхний уровень содержит 
контекст употребления коннектора в~том виде, в~каком он зафиксирован в~НБДК. 
Однако лишь часть его непосредственно участвует в~ЛСО условия 
\textit{если}$\|$\textit{то}, что отражено в~узлах, содержащих компоненты данного 
коннектора. Во фрагменте текс\-та, вводимого первой частью коннектора, находится 
ЛСО сопоставления, вводимое~\textit{а}, а~его левый контекст содержит два  
ЛСО~--- со\-по\-став\-ле\-ния и~времени, оформ\-ля\-емых неоднословным коннектором 
\textit{а~потом}. Таким образом, ЛСО, выражаемое \textit{а~потом}, 
<<встраивается>> в~ЛСО, вы\-ра\-жа\-емое~\textit{а}, которое, в~свою очередь, 
<<встраивается>> в~ЛСО, вы\-ра\-жа\-емое \textit{если}$\|$\textit{то}. Эту иерархию 
ЛСО и~отражает приведенный на рис.~5 граф.



    Хотя аннотирование в~корпусе GUM и~в НБДИ ЛСО ставят перед собой схожие 
задачи: аннотирование как самого отношения, так и~его показателя, эти два 
ресурса имеют существенные различия в~предостав\-ля\-емых пользователю 
воз\-мож\-но\-стях (табл.~2).

\vspace*{-6pt}


\section{Заключение}

\vspace*{-2pt}

  Лингвистический ресурс НБДИ ЛСО не имеет отечественных и~зарубежных 
аналогов по своим функциональным возможностям. Помимо поиска и~получения 
статистических данных он поз\-во\-ля\-ет исследователям проводить контрастивные 
исследования, со\-по\-став\-ляя текс\-ты на двух языках. Это дает воз\-мож\-ность, с~одной 
стороны, вы\-яв\-лять универсальные закономерности в~дискурсивной организации 
текста, а~с~другой~--- лингвоспецифичные черты, ха\-рак\-те\-ри\-зу\-ющие тот или 
иной язык, особенно в~зоне функционирования показателей отношений связ\-ности.

\vspace*{-6pt}
  
{\small\frenchspacing
 { %\baselineskip=10.6pt
 %\addcontentsline{toc}{section}{References}
 \begin{thebibliography}{99}
 
\vspace*{-2pt}
 
\bibitem{1-in}
\Au{Гончаров А.\,А., Инькова~О.\,Ю.} Извлечение знаний о~средствах выражения 
ло\-ги\-ко-се\-ман\-ти\-че\-ских отношений при помощи Надкорпусной базы данных~// Информатика и~её 
применения, 2021. Т.~15. Вып.~2. С.~96--103.
\bibitem{2-in}
\Au{Das D., Taboada~M.} RST Signalling Corpus Annotation Manual. 2014. {\sf 
https://www.sfu.ca/$\sim$mtaboada/ docs/publications/RST\_Signalling\_Corpus\_Annotation\_\linebreak Manual.pdf}.
\bibitem{3-in}
Penn Discourse Treebank Project (PDTB). {\sf https://www. seas.upenn.edu/$\sim$pdtb}.
\bibitem{4-in}
Ru-RSTreebank: Русскоязычный дискурсивный корпус. {\sf https://rstreebank.ru}.
\bibitem{5-in}
\Au{Carlson L., Marcu~D.} Discourse Tagging Reference Manual: Technical Report ISI-TR-545.~--- 
Marina del Rey, CA, USA: The University of Southern California, 2001. 87~p.
\bibitem{6-in}
\Au{Carlson L., Marcu~D., Okurowski~M.\,E.} Building a~discourse-tagged corpus in the framework 
of rhetorical structure theory~// Current directions in discourse and dialogue~/ Eds. J.~van Kuppevelt, 
R.~Smith.~--- Dordrecht: Kluwer Academic Publs., 2003. P.~85--109.
\bibitem{7-in}
\Au{Mann W., Thompson~S.} Rhetorical structure theory: Towards a functional theory of text 
organization~// Text, 1988. Vol.~8. Iss.~3. P.~243--281. doi: 10.1515/ text.1.1988.8.3.243.
\bibitem{8-in}
Руководство по разметке текстов (на основе тео\-рии риторических структур).~--- Ru-RSTreebank, 
2019.   
{\sf https://docs.google.com/document/d/1wd-sgGyIo5AQ q2IPj6jWa\_QmU0fUohXj48qsfVDgcBs/edit\#heading= h.gjdgxs}. 
\bibitem{9-in}
Рассказы о сновидениях: Корпусное исследование устного русского дискурса~/ Под ред. 
А.\,А.~Кибрика, В.\,И.~Подлесской.~--- М.: Языки славянских культур, 2009. 736~с.
\bibitem{10-in}
\Au{Инькова О.\,Ю.} Логико-се\-ман\-ти\-че\-ские отношения: проб\-ле\-мы клас\-си\-фи\-ка\-ции~// 
%\Au{Инькова~О., Манзотти~Э.} 
Связность текста: Мереологические  
ло\-ги\-ко-се\-ман\-ти\-че\-ские отношения.~--- М.: Языки славянских культур, 2019. С.~11--98.
\bibitem{11-in}
\Au{Prasad R., Miltsakaki~E., Dinesh~N., Lee~A., Joshi~A., Webber~B.\,L.} The Penn Discourse 
Treebank~1.0 Annotation Manual.~--- Philadelphia, PA, USA: 
Institute for Research in Cognitive Science, University of Pennsylvania, 2006. Technical Report No.\,IRCS-06-01.
\bibitem{12-in}
\Au{Prasad R., Webber~B., Lee~A., Joshi~A.} The Penn Discourse Treebank~3.0 Annotation 
Manual.~--- Philadelphia, PA, USA: Linguistic Data Consortium, University of Pennsylvania, 2019. 
81~p. %doi: 10.35111/qebf-gk47.
{\sf 
https://catalog.ldc.upenn.edu/ docs/LDC2019T05/PDTB3-Annotation-Manual.pdf}.

\bibitem{13-in}
\Au{Prasad R., Joshi~A., Webber~B.} Realization of discourse relations by other means: Alternative 
lexicalizations~// 23rd Conference (International) on Computational Linguistics: Posters Volume.~--- 
Beijing, 2010. P.~1023--1031.
{\sf 
https://www.aclweb.org/anthology/C10- 2118.pdf}.
\bibitem{14-in}
\Au{Prasad R., Miltsakaki~E., Dinesh~N., Lee~A., Joshi~A.} The Penn Discourse Treebank~2.0 
Annotation Manual.~--- Philadelphia, PA, USA: Institute for 
Research in Cognitive Science, University of Pennsylvania, 2008. Technical Report No.\,IRCS-08-01.
{\sf https://repository.upenn.edu/ cgi/viewcontent.cgi?article=1203\&context=ircs\_\linebreak reports}.

\bibitem{15-in}
\Au{Инькова-Манзотти О.\,Ю.} Коннекторы противопоставления во французском и~русском 
языках: сопоставительное исследование.~--- М.: МГУ-Информэлектро, 2001. 432~с.
\bibitem{16-in}
PennDiscourse Treebank Project (PDTB). {\sf https://www. seas.upenn.edu/$\sim$pdtb}.
\bibitem{17-in}
Национальный корпус русского языка (НКРЯ). {\sf https://ruscorpora.ru/new}.
\bibitem{18-in}
\Au{Inkova O., Popkova~N.} Statistical data as information source for linguistic analysis of Russian 
connectors~// Информатика и~её применения, 2017. Т.~11. Вып.~3. С.~123--131.
\bibitem{19-in}
\Au{Инькова О.\,Ю.} Лингвоспецифичность коннекторов: методы и~параметры описания~// 
Семантика коннекторов: контрастивное исследование~/ Под ред. О.\,Ю.~Иньковой.~--- М.: 
ТОРУС ПРЕСС, 2018. С.~5--23.

\pagebreak

\bibitem{20-in}
\Au{Das D., Taboada M.} RST Signalling Corpus: A~corpus of signals of coherence relations~// Lang. 
Resour.  Eval., 2018. Vol.~52. P.~149--184.
\bibitem{21-in}
\Au{Zeldes A.} rstWeb~--- a~browser-based annotation interface for rhetorical structure theory and 
discourse relations~//  NAACL--HLT Proceedings.~--- San Diego, CA, USA: Association for 
Computational Linguistics, 2016. P.~1--5.
\bibitem{22-in}
\Au{Gessler~L., Liu~J., Zeldes~A.} A~Discourse Signal Annotation System for RST Trees~//  
Discourse Relation Parsing and Treebanking Proceedings.~--- Minneapolis, MN, USA: Association for Computational 
Linguistics, 2019. P.~56--61. doi: 10.18653/v1/W19-2708. 
\bibitem{23-in}
GUM: The Georgetown University Multilayer Corpus. {\sf 
https://corpling.uis.georgetown.edu/gum/annotations.\linebreak html}.
\bibitem{24-in}
\Au{Дурново А.\,А., Инькова~О.\,Ю., Попкова~Н.\,А.} Архитектура базы данных иерархии 
ло\-ги\-ко-се\-ман\-ти\-че\-ских отношений~// Сис\-те\-мы и~средства информатики, 2022. Т.~32. №\,1. 
С.~114--125.

\end{thebibliography}

 }
 }

\end{multicols}

\vspace*{-10pt}

\hfill{\small\textit{Поступила в~редакцию 07.04.21}}

\vspace*{6pt}

%\pagebreak

%\newpage

%\vspace*{-28pt}

\hrule

\vspace*{2pt}

\hrule

%\vspace*{-2pt}

\def\tit{PRINCIPLES OF DESCRIBING MARKERS\\
OF~LOGICAL-SEMANTIC 
RELATIONS AND~THEIR HIERARCHY}


\def\titkol{Principles of describing markers of~logical-semantic 
relations and~their hierarchy}


\def\aut{A.\,A.~Durnovo$^1$, O.\,Yu.~Inkova$^{1,2}$, and~N.\,A.~Popkova$^1$}

\def\autkol{A.\,A.~Durnovo, O.\,Yu.~Inkova, and~N.\,A.~Popkova}

\titel{\tit}{\aut}{\autkol}{\titkol}

\vspace*{-8pt}



\noindent
$^1$Federal Research Center ``Computer Science and Control'' of the Russian Academy 
of Sciences, 44-2~Vavilov\linebreak
$\hphantom{^1}$Str., Moscow 119333, Russian Federation

\noindent
$^2$University of Geneva, 22 Bd des Philosophes, CH-1205 Geneva~4, Switzerland

\def\leftfootline{\small{\textbf{\thepage}
\hfill INFORMATIKA I EE PRIMENENIYA~--- INFORMATICS AND
APPLICATIONS\ \ \ 2022\ \ \ volume~16\ \ \ issue\ 2}
}%
 \def\rightfootline{\small{INFORMATIKA I EE PRIMENENIYA~---
INFORMATICS AND APPLICATIONS\ \ \ 2022\ \ \ volume~16\ \ \ issue\ 2
\hfill \textbf{\thepage}}}

\vspace*{3pt} 



\Abste{The article deals with annotation strategies in corpora with 
discourse markup. It is shown that Rhetorical Structure Theory (RST)-based corpora 
only contain annotations of coherence relations, or rhetorical relations (RR). 
In contrast, the Penn Discourse Treebank (PDTB) of the University of Pennsylvania 
annotates relations markers, as does the Supracorpora Database of Connectives. 
The RST Signaling Corpus (RST-SC), also based on RST, has been shown to annotate 
RR markers, but cannot combine the markup of RRs and their markers in 
a~single annotation. This problem is solved by the GUM corpus and the Supracorpora Database 
of Hierarchy of Logical-Semantic Relations. The latter has a few advantages: the ability to 
search, to obtain statistics, and to form bilingual annotations. This makes it possible 
to identify both universal 
phenomena in the discursive organization of the text and language-specific phenomena.}

\KWE{supracorpora database; corpus of texts' annotation; discourse relations; connective}



\DOI{10.14357/19922264220207}

\vspace*{-21pt}

\Ack

\vspace*{-4pt}

\noindent
The research was carried out using the infrastructure of the Shared Research Facilities 
``High Performance Computing and Big Data'' (CKP ``Informatics'') of FRC CSC RAS 
(Moscow) and supported by the Switzerland--Russia Research Preparation Grant 
``Annotation methodology in a~supracoprora database of connectives'' of the State 
Secretariat for Education, Research, and Innovation.



\vspace*{-2pt}

  \begin{multicols}{2}

\renewcommand{\bibname}{\protect\rmfamily References}
%\renewcommand{\bibname}{\large\protect\rm References}

{\small\frenchspacing
 {\baselineskip=10.6pt
 \addcontentsline{toc}{section}{References}
 

 \begin{thebibliography}{99}
 
  \vspace*{-2pt}
  
\bibitem{1-in-1}
\Aue{Goncharov, A.\,A., and O.\,Yu.~Inkova.} 2021. Izvlechenie znaniy o~sredstvakh 
vyrazheniya logiko-semanticheskikh otnosheniy pri pomoshchi Nadkorpusnoy bazy 
dannykh [Extracting knowledge about means of expression of logical-semantic relations 
from the Supracorpora database]. \textit{Informatika i~ee Primeneniya~--- Inform. 
Appl.} (15)2:96--103.
\bibitem{2-in-1}
\Aue{Das, D., and M.~Taboada.} 2014. RST Signalling Corpus annotation manual. 
Available at: {\sf https://www.sfu.ca/ $\sim$mtaboada/docs/publications/RST\_Signalling\_\linebreak Corpus\_Annotation\_Manual.pdf} 
(accessed April~22, 2022).
\bibitem{3-in-1}
Penn Discourse Treebank Project (PDTB). Available at: {\sf  
https://www.seas.upenn.edu/$\sim$pdtb/} (accessed April~22, 2022).
\bibitem{4-in-1}
Ru-RSTreebank: Russkoyazychnyy diskursivnyy korpus [Ru-RSTreebank: Russian 
discourse corpus]. Available at: {\sf https://rstreebank.ru/} (accessed April~22, 2022).
\bibitem{5-in-1}
\Aue{Carlson, L., and D.~Marcu.} 2001. \textit{Discourse tagging reference manual}. 
87~p. Available at: {\sf ftp://128.9.176.20/ isipubs/tr-545.pdf} (accessed April~22, 
2022).
\bibitem{6-in-1}
\Aue{Carlson, L., D.~Marcu, and M.\,E.~Okurowski.} 2003. Building a discourse-tagged 
corpus in the framework of rhetorical structure theory. \textit{Current directions in 
discourse and dialogue}. Eds J.~van Kuppevelt and R.~Smith. Dordrecht: Kluwer 
Academic Publs. 85--109.
\bibitem{7-in-1}
\Aue{Mann, W.\,C., and S.\,A.~Thompson.} 1988. Rhetorical structure theory: Towards 
a functional theory of text organization. \textit{Text} 8(3):243--281. doi: 10.1515/text.1.1988.8.3.243.
\bibitem{8-in-1}
Ru-RSTreebank. 2019.
Rukovodstvo po razmetke tekstov (na osnove teorii ritoricheskikh struktur) [Text 
Markup Guide (based on the theory of rhetorical structures)]. Available at: {\sf 
https://docs.google.com/\linebreak document/d/1wd-sgGyIo5AQq2IPj6jWa\_QmU0fUohXj\linebreak 48qsfVDgcBs/edit\#heading=h.gjdgxs} (accessed April~22, 2022).
\bibitem{9-in-1}
Kibrik, A.\,A., and V.\,M.~Podlesskaya, eds. 2009. \textit{Rasskazy o~snovideniyakh: 
Korpusnoe issledovanie ustnogo russkogo diskursa} [Night dream stories. A~corpus 
study of spoken Russian discourse]. Moscow: LRC Publishing House. 736~p.
\bibitem{10-in-1}
\Aue{Inkova, O.\,Yu.} 2019. Logiko-semanticheskie otnosheniya: Problemy 
klassifikatsii [Logical-semantic relations: Classification problems]. \textit{Svyaznost' 
teksta: mereologicheskie logiko-semanticheskie otnosheniya} [Text coherence: 
Mereological logical semantic relations]. Moscow: LRC Publishing House. 11--98.
\bibitem{11-in-1}
\Aue{Prasad, R., E.~Miltsakaki, N.~Dinesh, A.~Lee, A.~Joshi, and B.\,L.~Webber.} 
2006. The Penn Discourse Treebank~1.0 Annotation Manual. 
Philadelphia, PA: Institute for Research in Cognitive Science, University of Pennsylvania. Technical Report No.\,IRCS-06-01. 
Available at: {\sf 
https://repository.upenn.edu/ircs\_reports/3/} (accessed April~22, 2022).
\bibitem{12-in-1}
\Aue{Prasad, R., B.~Webber, A.~Lee, and A.~Joshi.} 2019. The Penn Discourse 
Treebank~3.0 Annotation Manual. Philadelphia, PA: Linguistic Data Consortium, University of Pennsylvania. 
81~p.  Available at: {\sf 
https://catalog.\linebreak ldc.upenn.edu/docs/LDC2019T05/PDTB3-Annotation-Manual.pdf} 
(accessed April~22, 2022).
\bibitem{13-in-1}
\Aue{Prasad, R., A.~Joshi, and B.~Webber.} 2010. Realization of discourse relations by 
other means: Alternative lexicalizations. \textit{23rd Conference (International) on 
Computational Linguistics Proceedings: Posters Volume}. Beijing. 1023--1031. Available at: {\sf 
https://www.aclweb. org/anthology/C10-2118.pdf} (accessed April~22, 2022).
\bibitem{14-in-1}
\Aue{Prasad, R., E.~Miltsakaki, N.~Dinesh, A.~Lee, and A.~Joshi.}
 2008. The Penn Discourse Treebank~2.0 Annotation Manual. 
Philadelphia, PA: Institute for Research in Cognitive Science, University of 
Pennsylvania. Technical Report IRCS-08-01. Available at: 
{\sf https://repository.upenn. edu/cgi/viewcontent.cgi?article=1203\&context=ircs\_\linebreak reports}. 
(accessed June~9, 2022). 
\bibitem{15-in-1}
\Aue{Inkova-Manzotti, O.\,Yu.} 2001. \textit{Konnektory protivopostavleniya vo 
frantsuzskom i~russkom yazykakh. Sopostavitel'noe issledovanie} [Connectives of 
opposition in French and Russian. A~comparative study]. Moscow: Informelektro. 
432~p.
\bibitem{16-in-1}
Penn Discourse Treebank Project (PDTB). Available at: {\sf  
https:// www.seas.upenn.edu/$\sim$pdtb/} (accessed April~22, 2022).
\bibitem{17-in-1}
Natsional'nyy korpus russkogo yazyka (NKRYa) [Russian National Corpus (RNC)]. Available at: {\sf 
http://www.ruscorpora.ru} (accessed April~22, 2022).
\bibitem{18-in-1}
\Aue{Inkova, O., and N.~Popkova.} 2017. Statistical data as information source for 
linguistic analysis of Russian connectors. \textit{Informatika i~ee Primeneniya~--- 
Inform. Appl.} 11(3):123--131.
\bibitem{19-in-1}
\Aue{Inkova, O.\,Yu.} 2018. Lingvospetsifichnost' konnektorov: metody i~parametry 
opisaniya [Language specificity of connectives methods and parameters of description]. 
\textit{Semantika konnektorov: kontrastivnoe issledovanie} [Semantics of connectives: 
A~contrastive study]. Ed.\ O.\,Yu.~Inkova. Moscow: TORUS PRESS. 5--23.
\bibitem{20-in-1}
\Aue{Das, D., and M.~Taboada.} 2018. RST signalling corpus: A~corpus of signals of 
coherence relations. \textit{Lang. Resour.  Eval.} 52:149--184.
\bibitem{21-in-1}
\Aue{Zeldes, A.} 2016. rstWeb~--- a~browser-based annotation interface for rhetorical 
structure theory and discourse relations. \textit{NAACL-HLT Proceedings}. San Diego, CA: 
Association for Computational Linguistics. 1--5.
\bibitem{22-in-1}
\Aue{Gessler, L., J.~Liu, and A.~Zeldes.} 2019. A discourse signal annotation system 
for RST trees. \textit{Discourse Relation Parsing and Treebanking Proceedings}. 
Minneapolis, MN: Association for Computational Linguistics. 56--61.
doi: 10.18653/v1/W19-2708. 
\bibitem{23-in-1}
GUM: The Georgetown University Multilayer Corpus. Available at: {\sf 
https://corpling.uis.georgetown. edu/gum/annotations.html} (accessed April~22, 2022).
\bibitem{24-in-1}
\Aue{Durnovo, A.\,A., O.\,Yu.~Inkova, and N.\,A.~Popkova.} 2022. Arkhitektura bazy 
dannykh iyerarkhii logiko-semanticheskikh otnosheniy [Database of hierarchies of 
logical-semantic relations: Architecture]. \textit{Sis\-te\-my i~Sred\-st\-va Informatiki~--- 
Systems and Means of Informatics} (32)1:114--125.

\end{thebibliography}

 }
 }
 

\end{multicols}

\vspace*{-8pt}

\hfill{\small\textit{Received April 7, 2021}}

\vspace*{-20pt}

\Contr

\vspace*{-6pt}


\noindent
\textbf{Durnovo Aleksandr A.} (b.\ 1949)~--- leading programmer, Institute of 
Informatics Problems, Federal Research Center ``Computer Science and Control'' of the 
Russian Academy of Sciences, 44-2~Vavilov Str., Moscow 119333, Russian 
Federation; \mbox{duralex49@mail.ru}

\vspace*{1pt}

\noindent
\textbf{Inkova Olga Yu.} (b.\ 1965)~--- Doctor of Science in philology, senior scientist, 
Institute of Informatics Problems, Federal Research Center ``Computer Science and 
Control'' of the Russian Academy of Sciences, 44-2~Vavilov Str., Moscow 119333, 
Russian Federation; faculty member, University of Geneva, 22 Bd des Philosophes, 
CH-1205 Geneva~4, Switzerland; \mbox{olyainkova@yandex.ru}

\vspace*{1pt}

\noindent
\textbf{Popkova Nataliia A.} (b.\ 1992)~--- junior scientist, Institute of Informatics 
Problems, Federal Research Center ``Computer Science and Control'' of the Russian 
Academy of Sciences, 44-2~Vavilov Str., Moscow 119333, Russian Federation; 
\mbox{natasha\_\_popkova@mail.ru}



\label{end\stat}

\renewcommand{\bibname}{\protect\rm Литература}    