\def\stat{shnurkov}

\def\tit{ОБ АНАЛИТИЧЕСКОЙ СТРУКТУРЕ НЕКОТОРЫХ ВИДОВ ЦЕЛЕВЫХ ФУНКЦИОНАЛОВ, 
СВЯЗАННЫХ С~ЗАДАЧАМИ УПРАВЛЕНИЯ ПОЛУМАРКОВСКИМИ СЛУЧАЙНЫМИ ПРОЦЕССАМИ}

\def\titkol{Об аналитической структуре некоторых видов целевых функционалов, 
связанных с~задачами управления} % полумарковскими случайными процессами}

\def\aut{П.\,В.~Шнурков$^1$}

\def\autkol{П.\,В.~Шнурков}

\titel{\tit}{\aut}{\autkol}{\titkol}

\index{Шнурков П.\,В.}
\index{Shnurkov P.\,V.}


%{\renewcommand{\thefootnote}{\fnsymbol{footnote}} \footnotetext[1]
%{Работа выполнена при поддержке Министерства науки и~высшего образования Российской Федерации (проект 
%075-15-2020-799).}}


\renewcommand{\thefootnote}{\arabic{footnote}}
\footnotetext[1]{Национальный исследовательский университет 
<<Высшая школа экономики>>, \mbox{pshnurkov@hse.ru}}

\vspace*{-10pt}




\Abst{Проведено исследование аналитической структуры трех видов функционалов 
от управляемого полумарковского процесса с~конечным множеством состояний. 
Доказано, что все эти математические объекты представимы в~виде дроб\-но-ли\-ней\-но\-го 
интегрального функционала, заданного на конечном наборе вероятностных мер, 
определяющих стратегию управления соответствующим полумарковским процессом. При 
этом для каждого из этих функционалов получены явные представления для 
подынтегральных функций числителя и~знаменателя через исходные вероятностные 
характеристики управляемого полумарковского процесса. Данный результат позволяет 
свести проблему оптимального управ\-ле\-ния полумарковским процессом с~конкретным 
целевым функционалом к~задаче исследования на глобальный экстремум заданной 
функции конечного числа переменных.}

\KW{стохастические модели управления; оптимальное 
управление полумарковскими процессами; дроб\-но-ли\-ней\-ный интегральный функционал; 
основная функция дроб\-но-ли\-ней\-но\-го интегрального функционала}

\DOI{10.14357/19922264220210}
  
\vspace*{-5pt}


\vskip 10pt plus 9pt minus 6pt

\thispagestyle{headings}

\begin{multicols}{2}

\label{st\stat}

\section{Введение}

\vspace*{-2pt}

Идея о дробно-ли\-ней\-ной структуре целевого функционала в~задаче оптимального 
управления полумарковским случайным процессом была, по-видимому, впервые 
реализована в~классической работе Х.~Майна и~С.~Осаки~\cite{1}. В~этой работе 
рассматривались различные модели управления марковскими и~полумарковскими 
случайными процессами с~конечными множествами состояний и~допустимых решений. 
Для модели управ\-ле\-ния полумарковским процессом без переоценки, когда множество 
состояний образует один эргодический класс, а~показатель эффективности 
представляет собой стационарный средний удельный доход~[1, разд.~5.5], 
полученное аналитическое выражение для указанного показателя имеет вид отношения\linebreak 
двух линейных функций от дискретного вероятностного распределения, определяющего 
принимаемое решение. Со\-от\-вет\-ст\-ву\-ющая \mbox{экстремальная} задача на множестве 
дискретных вероятностных распределений была решена авторами методом линейного 
программирования. Было установлено, что оптимальная стратегия управления 
является детерминированной. Результатов для более общих моделей управления 
в~данной работе получено не было, однако сама идея оказалась весьма плодотворной. 
Среди основополагающих исследований в~об\-ласти тео\-рии управ\-ле\-ния полумарковскими 
случайными процессами следует также отметить работу~\cite{2}, в~которой были 
формально определены понятия аддитивного функционала дохода, связанного 
с~полумарковским процессом, и~получены эргодические теоремы для этого функционала. 
Это позволило поставить задачу управления по отношению к~стационарному 
показателю эф\-фек\-тив\-ности, который по своему экономическому содержанию 
пред\-став\-ля\-ет собой средний удельный доход в~рассматриваемой стохастической 
модели. Такой подход к~задаче оптимального управления многократно использовался 
в~дальнейшем вплоть до современных исследований~[3, 4], в~которых 
рассматривались полумарковские модели при весьма общих предположениях на 
пространства состояний и~управ\-ле\-ний.

Важный результат, связанный со структурой стационарного стоимостного показателя 
эффективности управления, представлен В.\,А.~Каштановым в~гл.~13 коллективной 
монографии~\cite{5}. Автором рассмотрена проблема оптимального управления 
полумарковским процессом с~конечным множеством состояний и~множеством возможных 
решений, которое представляет собой произвольный интервал множества вещественных 
чисел. Модель относится к~виду моделей без переоценки, показателем качества 
управления служит стационарное значение среднего удельного дохода, опре\-де\-ля\-емое 
аналогично [1, 2]. Как и~в указанных работах, рандомизированное управление в~каж\-дом 
со\-сто\-янии определяется в~соответствии с~вероятностным распределением, 
а~их совокупность задает стратегию управ\-ле\-ния. Автором сформулировано утверж\-де\-ние 
о~том, что стационарное значение среднего удельного дохода пред\-став\-ля\-ет собой 
дроб\-но-ли\-ней\-ный интегральный функционал от набора вероятностных распределений, 
образующих стратегию управления. При этом ранее~[5, гл.~10; 6] было установлено, 
что дроб\-но-ли\-ней\-ный функционал достигает экстремума на вы\-рож\-ден\-ных 
распределениях. Отсюда следует, что оптимальная стратегия управ\-ле\-ния является 
детерминированной и~должна определяться точкой экстремума функции, 
представляющей собой отношение подынтегральных функций числителя и~знаменателя 
данного дроб\-но-ли\-ней\-но\-го функционала. Однако в~\cite{5} не были получены явные 
представления для указанных функций. Кроме того, приведенный в~гл.~10 
монографии~\cite{5} вариант теоремы об экстремуме дроб\-но-ли\-ней\-но\-го интегрального 
функционала требовал проверки выполнения условия существования этого экстремума. 
Такие условия не были уста\-нов\-ле\-ны. В~связи с~этими обстоятельствами использовать 
полученные в~\cite{5} результаты для доказательства существования оптимальной 
детерминированной стратегии управ\-ле\-ния полумарковским процессом и~для строгого 
обоснования способа нахождения такой стратегии оказалось невозможным. Результаты 
работы~[5], связанные со структурой стационарного стоимостного показателя 
эффективности управления, были обобщены и~усилены в~[7]. Для со\-от\-вет\-ст\-ву\-ющей 
модели управ\-ле\-ния полумарковским процессом с~конечным множеством со\-сто\-яний 
и~произвольным пространством допустимых решений было доказано утверждение 
о~дроб\-но-ли\-ней\-ной структуре стационарного стоимостного показателя эффективности 
управления. При этом были найдены явные аналитические формулы для 
подын\-тег\-раль\-ных функций интегральных выражений в~числителе и~знаменателе этого 
дроб\-но-ли\-ней\-но\-го функционала.

В настоящей работе получены принципиально новые результаты о~дроб\-но-ли\-ней\-ной 
интегральной структуре иных видов функционалов, возникающих в~задачах управ\-ле\-ния 
полумарковскими процессами. Эти функционалы связаны со случайной длительностью 
пребывания процесса в~заданном конечном подмножестве состояний. По отношению 
к~возможным приложениям такую случайную величину можно интерпретировать как время 
до первого отказа сис\-те\-мы, или время до первого достижения некоторого 
<<неприемлемого>> состояния. Математические ожидания таких случайных величин можно 
рассматривать как разумные и~естественные показатели эффективности управления 
соответствующим случайным процессом.


\section{Об аналитическом представлении классического стационарного 
стоимостного функционала}

Начнем изложение с~исследования классического варианта стационарного 
стоимостного показателя эффективности управления, идея которого восходит 
к~работам~[1, 2]. Рас\-смот\-рим управ\-ля\-емый полумарковский процесс $\xi(t)$ 
с~конечным множеством со\-сто\-яний $X\hm=\{1,2,\ldots ,N\}$, где $N\hm>0$~--- заданное целое 
положительное чис\-ло. Процесс~$\xi(t)$ управ\-ля\-ет\-ся в~моменты~$t_n$, 
$n\hm=0,1,2,\ldots$, $t_0\hm=0$, в~которые происходят последовательные изменения состояний 
(скачки). Управление процессом представляет собой случайную величину~$u_n$, 
принимающую значения из некоторого множества возможных управлений~$U$. Обычно 
под $U$ понимается множество вещественных чисел или некоторое его подмножество, 
на котором задана стандартная $\sigma$-ал\-геб\-ра борелевских множеств. Зададим на 
множестве~$U$ набор вероятностных мер (распределений) 
$\Psi_1(\cdot),\Psi_2(\cdot),\ldots,\Psi_N(\cdot)$, которые будут определять 
принимаемые решения об управ\-ле\-нии. Если выполняется условие, состоящее в~том, 
что\linebreak $\xi_n\hm=\xi(t_n+0)\hm=i$, то управ\-ле\-ние~$u_n$ определяется как случайная 
величина, принимающая значения из $U$ и~имеющая распределение~$\Psi_i(u)$, 
$i\hm=1,2,\ldots,N$. Заданный набор вероятностных \mbox{распределений} 
$\Psi_1(\cdot),\Psi_2(\cdot),\dots,\Psi_N(\cdot)$ образует стратегию управ\-ле\-ния 
полумарковским процессом~$\xi(t)$. Вероятностные распределения, входящие в~него, 
часто называются управ\-ля\-ющи\-ми. Последовательность $\{\xi_n\hm=\xi(t_n+0), 
n\hm=0, 1, \ldots\}$ образует управ\-ля\-емую цепь Маркова, вложенную в~данный 
полумарковский процесс~$\xi(t)$. Обозначим через $\theta_n\hm=t_{n+1}\hm-t_n$ 
случайную длительность интервала времени $[t_n, t_{n+1}]$ между 
последовательными изменениями состояний процесса.

Введем основные вероятностные характеристики управляемого полумарковского 
процесса~$\xi(t)$:
\begin{itemize}

\item условные вероятности перехода вложенной цепи Маркова при условии, что 
управ\-ле\-ние принимает фиксированное значение:
\begin{multline}
p_{ij}(u)=\mathsf{P}\left(\xi_{n+1}=j \mid \xi_n=i,\ u_n=u\right),\\
 i,j \in X,\ u\in U; 
\label{e1-sh}
\end{multline}
\item 
полумарковские функции управляемого полумарковского процесса~$\xi(t)$:
\begin{multline}
Q_{ij}(u)=\mathsf{P}\left(\xi_{n+1}=j, \theta_n<t \mid \xi_n=i, u_n=u\right),\\
 i,j \in X\,,\enskip u \in  U\,,\enskip t \geq 0\,;
 \label{e2-sh}
\end{multline}
\item
 условные математические ожидания длительностей пребывания полумарковского 
процесса в~его состояниях при условии, что управ\-ле\-ние принимает фиксированное 
значение:
\begin{multline}
\hspace*{-6pt}m_i(u)=\mathsf{E}\left[\theta_n \mid \xi_n=i, u_n=u\right]={}\\
{}=\sum\limits_{j \in X} 
\int\limits_0^{\infty} td_t Q_{ij}(t,u),\enskip
 i \in X\,,\enskip u \in U\,.\!\!
  \label{e3-sh}
\end{multline}
\end{itemize}

Рассмотрим также со\-от\-вет\-ст\-ву\-ющие вероятностные характеристики без условия на 
управ\-ле\-ние, т.\,е.\ усредненные по соответствующим управ\-ля\-ющим вероятностным 
распределениям:
\begin{align}
 \!\!p_{ij}&=\mathsf{P}\left(\xi_{n+1}=j|\xi_{n}=i\right)={}\notag\\
& \hspace*{20mm}{}=\int\limits_U p_{ij}(u) \,d\Psi_{i}(u),\enskip 
i,j\in X; \label{e4-sh}\\
\!\!Q_{ij}(t)&=\mathsf{P}\left(\xi_{n+1}=j,\theta_{n}<t|\xi_{n}=i\right)={}\notag\\
&\hspace*{6mm}{}=\int\limits_U  Q_{ij}(t,u)\, d\Psi_{i}(u),\enskip i,j\in X,\ t\geq0;\label{e5-sh}\\
\!\!m_{i}&=\mathsf{E}\left[ \theta_{n} |\xi_{n}=i\right] = \int\limits_U m_{i}(u) \,d\Psi_{i}(u),\enskip 
i\subset X. \!\label{e6-sh}
\end{align}

Подынтегральные функции в~правых частях соотношений~(\ref{e4-sh})--(\ref{e6-sh}), формально 
и~содержательно определяемые равенствами~(\ref{e1-sh})--(\ref{e3-sh}) соответственно, предполагаются 
заданными.

Предположим, следуя работе~\cite{1}, что при стратегии управления, определяемой 
набором управляющих вероятностных распределений $ \Psi_{1}(u), 
\Psi_{2}(u),\ldots ,\Psi_{N}(u)$, цепь Маркова~$\{\xi_{n}\}$, вложенная 
в~полумарковский процесс~$\xi(t)$, имеет ровно один класс возвратных положительных 
состояний. Как известно, тогда у этой цепи существует единственное стационарное 
распределение 
$$
\Pi=\left(\pi_{1},\pi_{2},\ldots ,\pi_{N}\right).
$$

Следуя классической традиции решения задач стохастического управ\-ле\-ния~[1, 2], 
рассмотрим стационарный стоимостный показатель эф\-фек\-тив\-ности управ\-ле\-ния вида
\begin{equation}
I=\fr{\sum\nolimits_{j=1}^{N}c_j\pi_j}{\sum\nolimits_{j=1}^{N}m_j\pi_j}\,.
\label{e7-sh}
\end{equation}
Стоимостные характеристики~$c_j$, входящие в~формулу~(\ref{e7-sh}), определяются некоторым 
стоимостным аддитивным функционалом~$\eta(t)$, связанным с~исходным 
полумарковским процессом~$\xi(t)$. Конструктивный метод задания такого 
функционала описан в~работах~[1, 2]. Отметим, что по своему экономическому 
содержанию такой функционал пред\-став\-ля\-ет собой накопленный случайный доход 
(прибыль), возникающий при эволюции сис\-те\-мы, математической моделью которой 
служит основной случайный процесс~$\xi(t)$.
Конкретное представление для характеристик~$c_j$ определяется формулами:
\begin{itemize}
\item  математическое ожидание дохода за время 
пребывания в~состоянии~$j$ при условии, что в~момент перехода в~данное состояние 
принято решение об управ\-ле\-ние $u\hm\in{U}$:
\begin{multline*}
c_j(u)=\mathsf{E}\left[\eta(t_{n+1})-\eta(t_n)| \xi_n=j, u_n=u\right] ={}\\
{}=\mathrm{E} \left[\Delta\eta_n| \xi_n=j, u_n=u\right];
\end{multline*}
\item математическое ожидание дохода (приращения значения рассматриваемого 
аддитивного функционала~$\eta(t)$) за время пребывания процесса~$\xi(t)$ 
в~фиксированном состоянии~$j$, которое определяется без условия на управление, $j\hm\in X$:
\begin{equation*}
c_j=\int\limits_{U}c_j(u)d\Psi_j(u)\,.
%\label{e8-sh}
\end{equation*}
\end{itemize}

Па\-ра\-мет\-ры~$m_j$, $j\hm\in X$, входящие в~(\ref{e7-sh}), определяются равенствами~(\ref{e3-sh}) и~(\ref{e6-sh}). 
Вектор $\Pi\hm=(\pi_1,\pi_2,\dots,\pi_N)$ пред\-став\-ля\-ет собой стационарное 
распределение вложенной цепи Маркова~$\{\xi_n\}$ и~зависит от заданной стратегии 
управления.

Теперь сформулируем основной результат данного раздела, связанный 
с~аналитическим пред\-став\-ле\-ни\-ем стационарного стоимостного показателя эф\-фек\-тив\-ности 
управ\-ле\-ния вида~(\ref{e7-sh}).

\smallskip

\noindent
\textbf{Теорема~1.}\
\textit{Стационарный стоимостный функционал~$I$ вида}~(\ref{e7-sh}), \textit{связанный с~управ\-ля\-емым 
полумарковским процессом~$\xi(t)$, пред\-став\-ля\-ет\-ся в~форме дроб\-но-ли\-ней\-но\-го 
интегрального функционала от \mbox{управля\-ющих} вероятностных распределений}:
\begin{multline}
I=I\left(\Psi_1,\Psi_2,\ldots ,\Psi_N\right)={}\\
{}=
\Bigg(\underset{U^{(N)}}{\int\!\!\!\int\cdots\!\int }
A(u_1,u_2,\ldots ,u_N)\,d\Psi_1(u_1)d\Psi_2(u_2)\cdots\\
\cdots d\Psi_N(u_N)\Bigg)\!\Bigg/\!
\Bigg(\underset{U^{(N)}}{\int\!\!\!\int\cdots
\int }B\left(u_1,u_2,\ldots \right.\\
\left.\ldots ,u_N\right)d\Psi_1(u_1)d\Psi_2(u_2)\cdots d\Psi_N(u_N)\Bigg),
\label{e9-sh}
\end{multline}
\textit{где подынтегральные функции числителя и~знаменателя задаются сле\-ду\-ющи\-ми 
выражениями}:
\begin{align}
A\left(u_1,u_2,\ldots ,u_N\right)&={}\notag\\
&\hspace*{-30mm}{}=\!\sum\limits_{j=1}^{N} \!
c_j(u_j)\widehat{D}^{(j)}\!\left(u_1,u_2,\ldots,u_{j-1},u_{j+1},\ldots ,u_N\!\right);
\label{e10-sh}
\\
B\left(u_1,u_2,\ldots ,u_N\right)&={}\notag\\
&\hspace*{-31mm}{}=\!\sum\limits_{j=1}^{N} \!
m_j(u_j)\widehat{D}^{(j)}\!\left(u_1,u_2,\ldots ,u_{j-1},u_{j+1},\ldots ,u_N\!\right)\!,
\label{e11-sh}
\end{align}
\textit{а функции $\widehat{D}^{(j)}(u_1,u_2,\ldots ,u_{j-1},u_{j+1},\ldots ,u_N)$, 
$j\hm=1,2,\ldots ,N$, в~свою очередь, определяются сле\-ду\-ющи\-ми соотношениями}:
\begin{multline}
\widehat{D}^{(j)}\left(u_{1},u_{2},\ldots ,u_{j-1},u_{j+1},\ldots,u_{N}\right)={}\\
{}= (-1)^{j} \sum\limits_{\alpha^{(N),j}} (-1)^{\delta{(\alpha^{(N),j})}} 
\widehat{D}_0^{(j)}\left(\alpha^{(N),j};u_{1},u_{2},\ldots\right.\\
\left. \ldots  ,u_{j-1},u_{j+1},\ldots ,u_{N}
\vphantom{\alpha^{(N),j};u_{1},u_{2}}
\right), 
\label{e12-sh}
\end{multline}
\textit{где} $\alpha^{(N),j} \hm= (\alpha_{1},\alpha_{2},\ldots,\alpha_{j-1},\alpha_{j+1},
\ldots ,\alpha_{N})$~--- 
\textit{произвольная перестановка чисел $(1,2,\ldots ,j-1,j+1,\ldots,N)$;
$\delta(\alpha^{(N),j})$~--- число инверсий в~перестановке~$\alpha^{(N),j}$,
причем суммирование в~правой части формулы}~(\ref{e12-sh}) \textit{проводится по всем возможным 
перестановкам набора чисел $(1,2,\ldots,j-1,j+1,\ldots ,N)$, т.\,е.\ число членов 
в~этой сумме составляет} $(N-1)!$;
\begin{multline}
\widehat{D}_0^{(j)}\left(\alpha^{(N),j};u_{1},u_{2},\ldots ,u_{j-1},u_{j+1},\ldots ,u_{N}\right)={}\\
{}=\widetilde{p}_{1,\alpha_1}\left(u_1\right)\widetilde{p}_{2,\alpha_2}\left(u_2\right)\cdots\\
\cdots
\widetilde{p}_{{j-1},\alpha_{j-1}}\left(u_{j-1}\right)
\widetilde{p}_{{j+1},\alpha_{j+1}}\left(u_{j+1}\right)\cdots\\
\cdots 
\widetilde{p}_{N,\alpha_N}\left(u_N\right),\label{e13-sh}
\end{multline}

\vspace*{-2pt}

\noindent
где

\noindent
\begin{equation}
\widetilde{p}_{i,\alpha_i}(u_i)
=\begin{cases}
p_{ii}(u_i)-1\,,&\mbox{если\ }\alpha_i=i;\\
p_{i,\alpha_i}\left(u_i\right),&\mbox{если\ } \alpha_i\neq i,\\
&\hspace*{-20mm} i=1,2,\ldots ,j-1,j+1,\ldots ,N.
\end{cases}
\label{e14-sh}
\end{equation}


\noindent
Д\,о\,к\,а\,з\,а\,т\,е\,л\,ь\,с\,т\,в\,о\  тео\-ре\-мы~1 приведено в~работе~[7].

\smallskip

\noindent
\textbf{Замечание~1.} Интегральные выражения в~числителе и~знаменателе 
функционала вила~(\ref{e9-sh}) представляют собой многомерные интегралы Ле\-бе\-га--Стилть\-еса, 
заданные на декартовых произведениях\linebreak \mbox{пространств} допустимых управ\-ле\-ний 
$U^{(N)}\hm=\linebreak = U\times U\times\dots\times U$. Вероятностные меры, на которых 
определены эти интегралы, по\-рож\-да\-ют\-ся произведениями вероятностных мер, 
образующих стратегии управ\-ле\-ния. Тео\-рия конечных произведений измеримых 
пространств, на которых заданы исходные меры, и~со\-от\-вет\-ст\-ву\-ющих произведений 
мер, опре\-де\-ля\-емых на произведении пространств, изложена в~из\-вест\-ной работе П.~Халмоша~[8].

\smallskip

\noindent
\textbf{Замечание~2.} Функции $c_j(u_j)$, $m_j(u_j)$ и~$p_{jk}(u_j)$, 
$j,k\hm=1,2,\dots,N$, входящие в~правые части соотношений~(\ref{e10-sh})--(\ref{e14-sh}), предполагаются 
известными. Таким образом, подынтегральные функции интегральных выражений 
в~числителе и~знаменателе функционала~(\ref{e9-sh}) имеют явные аналитические представления.

\section{Представление функционала математического ожидания времени пребывания 
процесса в~заданном подмножестве состояний}

Одна из важнейших вероятностных характеристик полумарковского процесса~--- 
распределение времени его пребывания в~заданном подмножестве множества со\-сто\-яний 
до первого выхода из этого подмножества. В~содержательных прикладных задачах, 
когда данный полумарковский процесс описывает эволюцию некоторой реальной 
системы, время пребывания в~заданном подмножестве состояний может совпадать, 
например, с~временем до первого отказа системы, или с~временем до первого 
наступления определенного <<нежелательного>> события. В~связи с~этим 
математическое ожидание времени пребывания управляемого процесса в~заданном 
подмножестве состояний может вполне естественно рассматриваться как один из 
важных показателей эффективности управления.

Рассмотрим вновь управляемый полумарковский процесс~$\xi(t)$ с~дискретным 
множеством состояний~$X$. Предположим дополнительно, что данный процесс может 
с~положительной вероятностью обрываться в~некоторый момент времени, совпадающий 
с~одним из моментов изменения состояний, причем событие, заключающееся в~обрыве 
процесса, не зависит от принимаемых решений. Обозначим через~$\zeta$ случайное 
число изменений состояний процесса до момента обрыва (включительно). Тогда 
основные вероятностные характеристики управ\-ля\-емо\-го полумарковского процесса~$\xi(t)$, 
за\-да\-ва\-емые соотношениями~(\ref{e1-sh})--(\ref{e6-sh}), будут определяться при 
дополнительном условии ($\zeta\hm>n$). Во введенных обозначениях~$t_\zeta$ 
представляет собой момент последнего скачка, или обрыва, процесса. В~стандартном 
варианте, когда процесс является необрывающимся, т.\,е.\ 
$\mathsf{P}(\zeta\hm=\infty)\hm=1$, будем полагать, что момент обрыва также равен бес\-ко\-неч\-ности 
с~ве\-ро\-ят\-ностью, равной единице.

Пусть $X_{0} \subset X$~--- заданное подмножество множества со\-сто\-яний 
полумарковского процесса~$\xi (t)$; $k \hm\in X_0$~--- фиксированное со\-сто\-яние 
подмножества $X_0$. Предположим, что в~некоторый случайный момент времени~$t_n$ 
процесс~$\xi(t)$ перешел в~со\-сто\-яние~$k$, т.\,е.\ значение процесса $\xi(t_n)\hm=k$. 
Обозначим через~$\zeta_k^{(n)}$ случайное время от момента~$t_n$ до момента 
первого выхода процесса из подмножества состояний~$X_0$.

Рассмотрим теперь математические ожидания случайных длительностей пребывания 
полумарковского процесса~$\xi(t)$ в~подмножестве состояний~$X_0$. Положим
\begin{equation*}
    a_k=\mathsf{E}\left(\zeta_k^{(n)}|t_\zeta>t_n+\zeta_k^{(n)}\right),
\end{equation*}
если процесс $\xi(t)$ обрывающийся. Если же процесс~$\xi(t)$ необрывающийся, 
т.\,е.\ с~вероятностью, равной единице, реализуется событие $(\zeta\hm=\infty)$, то 
будем полагать
\begin{equation*}
    a_k=\mathsf{E}\zeta_k^{(n)},\quad k \in X_0.
\end{equation*}
Поскольку распределение случайной величины~$\zeta_k^{(n)}$ не зависит от~$n$, 
в~дальнейшем индекс~$n$ в~обозначении данной величины будем опускать.

Как известно~[9, 10], математические ожидания~$a_k$, $k\hm\in X_0$, удовле\-тво\-ря\-ют 
сле\-ду\-ющей сис\-те\-ме линейных ал\-геб\-ра\-и\-че\-ских уравнений, связанных с~урав\-не\-ни\-ями 
марковского восстановления:
\begin{equation}
a_k=m_k+\sum\limits_{i \in X_0}p_{ki}a_i, \quad k\in X_0\,.
\label{e15-sh}
\end{equation}
Заметим, что величины $p_{ki}$, $k,i\hm\in X_0$, и~$m_k$, $k\hm\in X_0$, входящие 
в~соотношение~(\ref{e15-sh}), определяются равенствами~(\ref{e4-sh}) и~(\ref{e6-sh}) 
при дополнительном условии,  связанном с~возможным обрывом процесса~$\xi(t)$.

Теперь можно установить аналитическую структуру зависимости показателей 
$a_k$, $k\hm\in X_0$, от управ\-ля\-ющих вероятностных распределений.

\smallskip

\noindent
\textbf{Теорема~2.}\ \textit{Пусть $X_{0}$~--- конечное множество}: 
$X_{0}\hm=\{1,2,\dots,N_0 \}$. \textit{Тогда для любого фиксированного $k\hm\in \{1,2,\ldots 
,N_0\}$ функционал~$a_{k} $ среднего времени пребывания процесса~$\xi (t)$ 
в~множестве состояний~$X_{0} $ с~начальным состоянием~$k$ имеет сле\-ду\-ющий вид}:
\begin{multline}
\!a_{k} \left(\Psi_{1} ,\Psi_{2} ,\dots ,\Psi_{N_0} \right)\!=\!
\Bigg(\underset{U^{(N_0)}}{\int\!\!\!\int\cdots\int} A^{(0)}_{k} \left(u_{1} ,u_{2} 
,\ldots\right.\\
\left.\ldots ,u_{N_0} \right)\,d\Psi _{1} (u_{1} )d\Psi _{2} (u_{2} )\cdots\\
\cdots d\Psi _{N_0} (u_{N_0} )  
\Big)\!\!\Bigg/\!\!\Bigg(\underset{U^{(N_0)}}{\int\!\!\!\int\cdots\int} B^{(0)}\left(u_{1} ,u_{2} ,\ldots\right.\\
\left.\ldots ,u_{N_0} 
\right)\,d\Psi_{1} (u_{1} )d\Psi _{2} (u_{2} )\cdots d\Psi_{N_0} (u_{N_0})  \Bigg).
\label{e16-sh}
\end{multline}
\textit{Здесь}

\vspace*{-6pt}

\noindent
\begin{multline}
A^{(0)}_k\left(u_1, u_2,\ldots ,u_{N_0}\right)={}\\
{}=\sum\limits_{i^{(N_0)}}
(-1)^{\delta\left(i^{(N_0)}\right)+1}\widetilde{p}_{i_1,1}(u_{i_1})\widetilde{p}_{i_2,2}
(u_{i_2})\cdots\\
\cdots \widetilde{p}_{i_{k-1},k-1}(u_{i_{k-1}})m_{i_k}(u_{i_k})
\widetilde{p}_{i_{k+1},k+1}(u_{i_{k+1}})\cdots\\
\cdots \widetilde{p}_{i_{N_0},N_0}
(u_{i_{N_0}}),\enskip k=1,2,\ldots ,N_0;
\label{e17-sh}
\end{multline}

\vspace*{-14pt}

\noindent
\begin{multline}
B^{(0)}\left(u_1,u_2,\ldots ,u_{N_0}\right)={}\\
{}=\sum\limits_{j^{(N_0)}}
(-1)^{\delta(j^{(N_0)})}\widetilde{p}_{j_1,1}(u_{j_1})\widetilde{p}_{j_2,2}(u_{j_2})\cdots\\
\cdots 
\widetilde{p}_{j_{N_0},N_0}(u_{j_{N_0}}),
\label{e18-sh}
\end{multline}

\vspace*{-4pt}

\noindent
\textit{где}
$i^{(N_0)}=(i_1, i_2,\ldots ,i_{N_0})$ и~$j^{(N_0)}\hm=(j_1, j_2,\ldots\linebreak \ldots ,j_{N_0})$~--- 
\textit{произвольные перестановки набора чисел $(1,2,\ldots ,N_0)$;
$\delta(i^{(N_0)})$~--- число инверсий в~перестановке~$i^{(N_0)}$; суммирование 
в~формулах}~(\ref{e17-sh}) и~(\ref{e18-sh}) \textit{для функций
$A_k^{(0)}(u_1, u_2,\ldots ,u_{N_0})$, $k\hm=\overline{1,N_0}, B^{(0)}(u_1,u_2,\ldots,u_{N_0})$ проводится 
по всем возможным перестановкам чисел $(1,2,\ldots ,N_0)$, т.\,е.\ \mbox{число} членов 
в~указанных суммах рав\-но~$N_0!$.}

\textit{Величины, входящие в~правые части соотношений}~(\ref{e17-sh}) %%
\textit{и}~(\ref{e18-sh}), \textit{определяются 
равенством}
\begin{equation}
\widetilde{p}_{ij}(u_i)
=\begin{cases}
p_{ii}(u_{i})-1,&\mbox{если } j=i\,;\\
p_{ij}(u_i),&\mbox{если } j\neq{i}\,.
\end{cases}
\label{e19-sh}
\end{equation}


\noindent
Д\,о\,к\,а\,з\,а\,т\,е\,л\,ь\,с\,т\,в\,о\ \ тео\-ре\-мы~2 в~целом аналогично доказательству тео\-ре\-мы~1. 

\smallskip

\noindent
\textbf{Замечание 3.} Вероятностные характеристики полумарковского 
процесса~$p_{ij}(u_i)$, $m_k(u_k)$, $i,j,k\hm=1,2,\dots,N_0$, предполагаются 
заданными. Таким образом, подынтегральные функции интегральных выражений 
в~числителе и~знаменателе функционала~(\ref{e16-sh}) определяются аналитически соотношениями 
(\ref{e17-sh})--(\ref{e19-sh}).

\vspace*{-4pt}

\section{Аналитическое представление специального функционала, связанного 
с~предельным распределением времени пребывания полумарковского процесса в~заданном 
подмножестве состояний}

\vspace*{-2pt}

В данном разделе будет исследован показатель, также связанный с~распределением 
времени пребывания управ\-ля\-емо\-го полумарковского процесса в~заданном подмножестве 
состояний. В~отличие от предыдущего варианта предполагается, что веро-\linebreak\vspace*{-12pt}

\pagebreak

\noindent
ятностные 
характеристики управ\-ля\-емо\-го полумарковского процесса зависят от некоего малого 
параметра. При этом ве\-ро\-ят\-ность выхода процесса из\linebreak заданного подмножества 
состояний достаточно \mbox{мала} при стремлении этого па\-ра\-ме\-тра к~нулю. Прикладное 
содержание такого предположения \mbox{заключается} в~том, что ве\-ро\-ят\-ность отказа 
со\-от\-вет\-ст\-ву\-ющей сис\-те\-мы или вероятность иного <<нежелательного>> события мала, 
т.\,е.\ сис\-те\-ма считается высоконадежной. Исследование таких сис\-тем пред\-став\-ля\-ет 
значительный интерес.

Было установлено~[9], что предельное распределение времени до первого выхода 
полумарковского процесса из заданного подмножества состояний при стремлении 
малого параметра к~нулю в~некотором смысле близ\-ко к~экспоненциальному. 
Основываясь на этом результате, можно предложить еще один содержательный 
показатель эффективности управ\-ле\-ния полумарковским процессом в~модели с~малым 
параметром. Основное утверж\-де\-ние данного раздела со\-сто\-ит в~определении 
аналитической структуры этого показателя.

Построим модель управления полумарковским процессом, взяв за основу схему 
полумарковского процесса с~малым па\-ра\-мет\-ром, изложенную в~[9, гл.~6]. Пусть 
$\xi^{(\varepsilon )} (t)$~-- управ\-ля\-емый полумарковский процесс, зависящий от 
(малого) па\-ра\-ме\-тра~$\varepsilon$ и~принимающий значения в~дискретном фазовом 
пространстве  $X\hm=X_{0} \cup \{0\}$, где $X_0\hm=\{1,2,\dots,N_0\}$;  $\{0\}$~--- 
по\-гло\-ща\-ющее со\-сто\-яние. Как и~в~исходной полумарковской модели, основные 
вероятностные характеристики процесса~$\xi^{(\varepsilon)}(t)$ определяются 
равенствами~(\ref{e1-sh})--(\ref{e6-sh}). Предположим, что полумарковские 
функции~$Q_{ij}(t,u)$, $i,j\hm\in X$, зависят от~$\varepsilon$ сле\-ду\-ющим образом: 
$$
Q_{ij}^{\varepsilon } (t,u)=p_{ij}^{\varepsilon } (u)G_{ij}^{} (t,u), 
$$
где 
$$
G_{ij}(t,u)=\mathsf{P}\left(\theta_n<t| \xi_n=i, \xi_{n+1}\hm=j, u_n=u\right),
$$ 
причем условные 
распределения $G_{ij}(t,u)$, $i,j\hm\in X$, не зависят явно от па\-ра\-мет\-ра~$\varepsilon$. 
Ве\-ро\-ят\-ность перехода~$p_{ij}^{\varepsilon }(u)$ вложенной цепи 
Маркова исходного полумарковского процесса~$\xi ^{(\varepsilon)} (t)$ 
определяется соотношением:
\begin{equation}
p_{ij}^{\varepsilon } (u)=\begin{cases}
 p_{ij} (u)-\varepsilon b_{ij} (u), & i,j\in X_{0}; \\ 
\varepsilon q_{i} (u), & i\in X_{0},\ j=0; \\ 
1, &i=j=0. 
\end{cases}
\label{e20-sh}
\end{equation}
Функции $b_{ij}(u)$ и~$q_i(u)$, $i,j\hm\in X_0$, входящие в~соотношение~(\ref{e20-sh}), 
предполагаются известными, причем выполняется условие
$$
q_{i}(u)=\sum\limits_{j\in X_{0} }b_{ij} (u),
\enskip i\in X_0\,.
$$
 Указанные функции являются 
также ограниченными и~интегрируемыми по соответствующим управ\-ля\-ющим 
вероятностным распределениям $\Psi_i(u)$, $i\hm\in X_0$.

\smallskip

\noindent
\textbf{Определение~1.} Полумарковский процесс~$\xi ^{(\varepsilon )} (t)$, 
принимающий значения в~фазовом пространстве  $X\hm=X_{0} \cup \{0\}$, опре\-де\-ля\-емый 
начальным распределением $p^{(0)}$ (или заданным начальным состоянием $i_0\hm\in 
X_0$), совокупностью полумарковских функций $\{Q_{ij}^{\varepsilon } (t,u)$, $i,j 
\hm\in X\}$ и~набором управ\-ля\-ющих распределений (стратегией управ\-ле\-ния) 
$\Psi_{i}(u)$, $i\hm\in X_0$, будем называть возмущенным управ\-ля\-емым полумарковским 
процессом.

\smallskip

\noindent
\textbf{Определение 2.} Полумарковский процесс~$\xi^{(0)} (t)$ с~фазовым 
пространством~$X_{0}$, за\-да\-ва\-емый полумарковскими функциями $Q_{ij} (t,u)\hm=p_{ij} 
(u)G_{ij} (t,u)$, $i,j \hm\in X_{0}$, соответствующим начальным условием 
и~набором управ\-ля\-ющих вероятностных распределений~$\Psi_{i}(u)$, $i\hm\in X_{0} $, 
называется  невозмущенным  процессом, управ\-ля\-емым в~определенном выше смысле. 
Аналогично вложенные цепи Маркова $\{\xi_{n}^{(\varepsilon )} \}$ и~$\{\xi _{n}^{(0)} \}$ 
полумарковских процессов~$\xi ^{(\varepsilon )} (t)$ и~$\xi ^{(0)} (t)$ 
называются соответственно возмущенной и~невозмущенной вложенными 
цепями Маркова.

\smallskip

Как следует из~(\ref{e20-sh}), матрица переходных вероятностей ${\bf P}^{(\varepsilon )} 
(u)$ управ\-ля\-емой возмущенной цепи~$\{\xi _{n}^{(\varepsilon )} \}$, 
рас\-смат\-ри\-ва\-емой на множестве~$X_{0} $, представима в~виде
\begin{equation}
\mathbf{P}^{(\varepsilon )} (u)=\mathbf{P}^{(0)} (u)-\varepsilon \mathbf{B}(u), 
\label{e21-sh}
\end{equation}
где ${\bf P}^{(0)} (u)=\{p_{ij} (u)$, $i,j\hm\in X_{0} \}$~--- матрица 
переходных вероятностей управ\-ля\-емой невозмущенной цепи~$\{\xi_{n}^{(0)} \}$, 
$\varepsilon \mathbf{B}(u) \hm=\{\varepsilon b_{ij} (u)$, $i,j\hm\in X_{0} \}$~--- 
мат\-ри\-ца возмущений. Обозначим:
\begin{align}
p_{ij}&=\mathsf{P}\left(\xi_{n+1}^{(0)} =j |\xi_{n}^{(0)} =i\right)=\int\limits_{U}p_{ij}(u) \,
d\Psi_{i} (u);\label{e22-sh}\\
q_{i}^{}  &=\int\limits_{U}q_{i} (u)\,d\Psi_{i} (u) =\sum\limits_{j\in X_0} b_{ij},\enskip i\in X_{0}. \label{e23-sh}
\end{align}
где

\noindent
$$
b_{ij} =\int\limits_{U}b_{ij} (u)\,d\Psi_{i} (u).
$$


\vspace*{-2pt}

\noindent
Из равенств~(\ref{e21-sh})--(\ref{e23-sh}) следует мат\-рич\-ное соотношение
\begin{equation*}
\mathbf{P}^{(\varepsilon )}=\mathbf{P}^{(0)}-\varepsilon \mathbf{B},
%\eqno(24)
\end{equation*}
где ${\bf P}^{(\varepsilon )} \hm=\{p_{ij}^{(\varepsilon )}$, $i,j\hm\in X_{0} \}$, 
${\bf P}^{(0)} \hm=\{p_{ij}$, $i,j\hm\in X_{0} \}$, $\mathbf{B}\hm=\{b_{ij}$, 
$i,j\hm\in X_{0} \}$.
Рассмотрим также следующие характеристики невозмущенного управ\-ля\-емо\-го 
полумарковского процесса~$\xi^{(0)}(t)$:

\pagebreak

\noindent
\begin{equation*}
g_{ij}^{(1)} =\int\limits_{U}\left\{\int\limits_{0}^{\infty }td_{t} G_{ij} 
(t,u) \right\}d\Psi _{i} (u),\enskip i,j\in X_{0}; % \label{e24-sh}
\end{equation*}

\vspace*{-14pt}

\noindent
\begin{multline}
m_{i} (u)= \mathsf{E}\left(\theta_{n}^{(0)} | \xi_{n}^{(0)} =i, u_n=u 
\right)={}\\
{}=\int\limits_{0}^{\infty }\sum\limits_{j=1}^{N_0}td_{t} Q_{ij} (t,u), \enskip 
i=1,2,\dots ,N_0;
\label{e25-sh}
\end{multline}

\vspace*{-14pt}

\noindent
\begin{multline}
m_{i} =\mathsf{E}\left(\theta_{n}^{(0)} | \xi_{n}^{(0)} =i 
\right)=\int\limits_{U}m_i(u)d\Psi_{i} (u) ={}\\
{}=\sum\limits_{j \in X_{0} }p_{ij} 
g_{ij}^{(1)}. \label{e26-sh}
\end{multline}

\vspace*{-2pt}

\noindent
По аналогии с~[9] предположим, что для любых допустимых стратегий управления 
полумарковским процессом~$\xi^{(\varepsilon)} (t)$ выполнены следующие 
условия:
\begin{enumerate}[(1)]
\item  $Q_{ij}^{*} (s)=p_{ij}\hm - sp_{ij} g_{ij}^{(1)} \hm+o(s)$, где $Q_{ij}^{*} 
(s)$~--- преобразование Лап\-ла\-са--Стилть\-еса функции~$Q_{ij} (t)$, $i,j\hm\in X_0$;
\item  вложенная невозмущенная цепь Маркова~$\{\xi _{n}^{(0)} \}$ имеет 
ровно один класс положительно возвратных состояний;
\item  существует такое состояние $i \hm\in X_{0}$, которое возвратно 
в~невозмущенном управ\-ля\-емом  полумарковском процессе~$\xi ^{(0)} (t)$ и~для 
которого $p_{i0}^{\varepsilon }\hm >0$.
\end{enumerate}

Заметим, что для произвольной фиксированной до\-пус\-ти\-мой стратегии управ\-ле\-ния 
со\-сто\-яние~$i$ возвратно в~полумарковском процессе~$\xi^{(0)} (t)$ тогда и~только 
тогда, когда оно возвратно во вложенной цепи Маркова данного процесса~[9].

Обозначим через~$\zeta_{k}^{(\varepsilon )}$ время пребывания процесса~$\xi^{(\varepsilon )} (t)$ 
в~подмножестве состояний~$X_{0}$ до  поглощения  в~со\-сто\-янии~$\{0\}$, 
определяемое при условии, что в~начальный момент $t\hm=0$ 
процесс попадает в~состояние $k \hm\in X_{0}$. Тогда, согласно тео\-ре\-ме~6.1 из~[9], 
получаем следующее представление для предельного распределения случайного 
времени пребывания процесса~$\xi^{(\varepsilon)} (t)$ в~подмножестве состояний:
\begin{equation}
\lim\limits_{\varepsilon \to 0} \mathrm{P}\left(\varepsilon \zeta_{k}^{(\varepsilon )} 
\geq x\right)=e^{-\lambda x}.
\label{e27-sh}
\end{equation}
Здесь

\vspace*{-2pt}

\noindent
$$
\lambda=\fr {\sum\nolimits_{i\in X_{0} }q_{i} \pi_{i}}{\sum\nolimits_{i\in 
X_{0} }m_{i}\pi_{i}}\,,
$$ 

\vspace*{-2pt}

\noindent
где $q_{i},~i\in X_0$, определяются равенствами~(\ref{e23-sh}), 
параметры~$m_i$, $i \hm\in X_0$,~--- соотношениями~(\ref{e25-sh}) и~(\ref{e26-sh}); 
$\{\pi_i$, $i\hm\in X_0\}$~--- стационарное распределение невозмущенной цепи Маркова, определяемой матрицей 
переходных вероятностей~$\mathbf{P}^{(0)}$. Из соотноше-\linebreak\vspace*{-12pt}

\columnbreak

\noindent
ния~(\ref{e27-sh}) непосредственно 
следует, что при достаточно малых~$\varepsilon$ для любого фиксированного 
значения $y\hm>0$ можно приближенно полагать, что 
\begin{equation}
\mathsf{P}\left(\zeta^{(\varepsilon  )}\geq y\right) = e^{-\lambda \varepsilon y}.\label{e28-sh}
\end{equation}

Теперь рассмотрим величину~$T$, определяемую равенством

\noindent
\begin{equation}
T=\lambda^{-1} =\fr{\sum\nolimits_{i\in X_{0} }m_{i} 
\pi_{i}}{\sum\nolimits_{i\in X_{0} }q_{i} \pi_{i}}\,. \label{e29-sh}
\end{equation}
Параметр $T$ представляет собой предельное значение нормированного 
среднего времени пребывания полумарковского процесса~$\xi^{(\varepsilon)} (t)$ в~подмножестве состояний~$X_{0}$ до поглощения в~состоянии~$\{0\}$:
$$ 
T=\lim\limits_{\varepsilon \to 0} \varepsilon \mathsf{E}\zeta_{k}^{(\varepsilon )}.
$$

Отметим, что если $T^{*} \hm\geq T$, то для любого $x\hm\geq 0$ выполняется 
неравенство $e^{-\lambda ^{*} \varepsilon x} \hm\geq e^{-\lambda \varepsilon x} $, 
где $\lambda^{*} \hm=(T^{*} )^{-1} $. Учитывая вероятностное содержание функции~(\ref{e28-sh}), 
мож\-но утверж\-дать, что наилучшим предельным распределением случайной 
величины~$\zeta^{(\varepsilon)}$ будет распределение с~минимальным значением 
параметра~$\lambda$, т.\,е.\ с~максимальным значением па\-ра\-мет\-ра~$T$. Таким 
образом, величину~$T$ можно рас\-смат\-ри\-вать как содержательный показатель 
эффективности управ\-ле\-ния возмущенным полумарковским процессом. Как и~в~предыду\-щих 
разделах, выясним структуру зависимости показателя~$T$, опре\-де\-ля\-емо\-го 
равенством~(\ref{e29-sh}), от управ\-ля\-ющих вероятностных распределений.

\smallskip

\noindent
\textbf{Теорема~3.} \textit{Пусть $X_{0}\hm =\{1,2,\dots, N_0\}$~--- конечное 
множество. Тогда показатель~$T$, характеризующий предельное распределение 
времени до первого выхода процесса из данного подмножества, имеет сле\-ду\-ющее 
аналитическое представление}:

\vspace*{-4pt}

\noindent
\begin{multline}
T=\tilde{I}\left(\Psi_{1}, \Psi_{2}, \ldots , \Psi_{N_0} 
\right)={}\\
{}=\!\Bigg(\underset{U^{(N_0)}}{\int\! \cdots\! \int}\tilde{A} \left(u_{1} ,u_{2} ,\ldots ,u_{N_0} 
\right)\,d\Psi_{1} (u_{1})d\Psi_{2} (u_{2} )\cdots\\[-2pt]
\cdots d\Psi_{N_0} \left(u_{N_0} \right)  \Bigg)\!\!\Bigg/\!\!
\Bigg(\underset{U^{(N_0)}}{\int \!\cdots\! \int}\tilde{B}\left(u_{1} ,u_{2} ,\ldots\right.\\
\left.\ldots ,u_{N_0} 
\right)\,d\Psi_{1} (u_{1} )d\Psi_{2} (u_{2} )\cdots d\Psi_{N_0} (u_{N_0}) \Bigg). 
\label{e30-sh}
\end{multline}

\vspace*{-8pt}

\noindent
Здесь

\vspace*{-6pt}

\noindent
\begin{multline}
\tilde{A}(u_{1} ,u_{2} ,\ldots ,u_{N_0} )={}\\
{}=\sum\limits_{k=1}^{N_0} 
 \sum\limits_{i^{(N_0),k} } (-1)^{\delta (i^{(N_0),k} )+k}  
\tilde{p}_{1,i_{1} } (u_{1})\tilde{p}_{2,i_{2} } (u_{2} )\cdots\\
\cdots 
\tilde{p}_{k-1,i_{k-1} } (u_{k-1} )m_{k} (u_{k} )\tilde{p}_{k+1,i_{k+1} } 
(u_{k+1} )\cdots\\
\cdots \tilde{p}_{N_0,i_{N_0} } (u_{N_0} )\,;
\label{e31-sh}
\end{multline}

%\vspace*{-12pt}

\noindent
\begin{multline}
\tilde{B}(u_{1} ,u_{2} ,\ldots ,u_{N_0} )={}\\
{}=\sum\limits_{k=1}^{N_0}  
\sum\limits_{j^{(N_0),k} } (-1)^{\delta (j^{(N_0),k})+k} \, \tilde{p}_{1,j_{1} } 
(u_{1}) \tilde{p}_{2,j_{2} } (u_{2} )\cdots\\
\cdots \tilde{p}_{k-1,j_{k-1} } 
(u_{k-1} )q_{k} (u_{k} )\tilde{p}_{k+1,j_{k+1} } (u_{k+1} )\cdots\\
\cdots 
\tilde{p}_{N,j_{N_0} } (u_{N_0} ),\label{e32-sh}
\end{multline}
% m_{k} (u)= \int\limits_{0}^{\infty }\sum\limits_{j=1}^{N}td_{t} Q_{kj} (t,u), 
%\quad k=1,2,\dots ,N_0;\tag{30}\\
%\end{gather*}
где $i^{(N_0),k}=(i_1,\dots,i_{k-1},i_{k+1},\dots,i_{N_0})$~--- произвольная 
перестановка чисел $(1,2,\ldots,k-1,\linebreak k+1,\ldots,N_0)$; 
$j^{(N_0),k}\hm=(j_1,\ldots ,j_{k-1},j_{k+1},\ldots\linebreak
\ldots ,j_{N_0})$~--- произвольная 
перестановка чисел $(1,2,\ldots ,k-1,k+1,\ldots,N_0)$; $\delta(i^{(N_0),k})$ 
и~$\delta(j^{(N_0),k})$~--- число инверсий в~соответствующих перестановках. 
В~выражениях для функций $\tilde{A}(u_1,u_2,\ldots,u_{N_0})$ 
и~$\tilde{B}(u_1,u_2,\ldots,u_{N_0})$ суммирование проводится по всем возможным 
перестановкам.
Функции $\tilde{p}_{ij}(u)$,  входящие в~(\ref{e31-sh}) и~(\ref{e32-sh}), 
определяются со\-от\-но\-ше\-ни\-ями:
\begin{equation}
\tilde{p}_{ij} (u)=
\begin{cases}
  p_{ii} (u)- 1, &\hspace*{-12mm} i=j;\\
 p_{ij} (u), &\hspace*{-12mm}i\ne j, \\
i,j=1,2,\dots,N_0. 
\end{cases} 
\label{e33-sh}
\end{equation}

\noindent
\textbf{Замечание~4.} Исходное аналитическое выражение для показателя~$T$, 
определяемое формулой~(\ref{e29-sh}), имеет структуру, аналогичную стандартному 
стационарному стоимостному показателю~(\ref{e7-sh}). В~связи с~этим доказательство тео\-ре\-мы~3 
пол\-ностью аналогично доказательству тео\-ре\-мы~1.

\smallskip

\noindent
\textbf{Замечание~5.} Вероятностные характеристики полумарковского 
процесса~$p_{ij}(u_i)$, $m_k(u_k)$ и~$q_j(u_j)$, $i,j,k\hm=1,2,\ldots,N_0$, 
предполагаются заданными. Таким образом, подынтегральные функции интегральных 
выражений в~числителе и~знаменателе функционала~(\ref{e30-sh}) определяются аналитически 
соотношениями~(\ref{e31-sh})--(\ref{e33-sh}).

\section{Заключение}

В работах~[11--13] были получены исчерпывающие результаты решения задачи 
безусловного экстремума для дроб\-но-ли\-ней\-но\-го интегрального функционала, 
заданного на множестве вероятностных мер.
Тео\-ре\-ти\-че\-ское значение этих результатов для решения общей задачи управ\-ле\-ния 
марковскими и~полумарковскими случайными процессами можно описать следующим 
образом. Если удается установить, что некоторый показатель, связанный 
с~управ\-ля\-емым полумарковским процессом, может быть выражен в~указанной форме, 
причем известны явные аналитические пред\-став\-ле\-ния для подынтегральных функций 
его числителя и~знаменателя, то решение задачи оптимального управ\-ле\-ния пол\-ностью 
определяется экстремальными свойствами\linebreak так называемой основной функции дроб\-но-ли\-ней\-но\-го 
интегрального функционала, которая пред\-став\-ля\-ет собой отношение 
указанных подынтегральных функций. Таким образом, проб\-ле\-ма \mbox{существования} 
оптимальной детерминированной стратегии управ\-ле\-ния полумарковским процессом 
с~конечным множеством состояний и~проблема нахождения этой стратегии могут быть 
решены одновременно при помощи численного исследования на глобальный экстремум 
основной функции.
{\looseness=1

}

В настоящей работе установлено, что несколько конкретных содержательных 
показателей эффективности управления полумарковскими процессами с~конечными 
множествами состояний могут быть представлены в~форме дроб\-но-ли\-ней\-ных 
интегральных функционалов, заданных на множестве наборов вероятностных 
распределений, определяющих стратегию управления. При этом получены явные 
аналитические представления для основных функций этих функционалов. Таким 
образом, результаты данного исследования в~совокупности с~упомянутыми 
результатами работ~[11--13] создают теоретическую основу общего метода решения 
проб\-лем оптимального управ\-ле\-ния полумарковскими процессами с~конечными 
множествами состояний.


{\small\frenchspacing
 {%\baselineskip=10.8pt
 %\addcontentsline{toc}{section}{References}
 \begin{thebibliography}{99}
\bibitem{1} 
\Au{Майн Х., Осаки С.} Марковские процессы принятия решений~/ Пер. с~англ.~--- 
М.: Наука, 1977. 176~с. (\Au{Mine~H., Osaki~S.} Markovian decision processes.~--- New York, NY, USA: 
Elsevier, 1970. 142~p.)
\bibitem{2} 
\Au{Джевелл В.\,С.} Управ\-ля\-емые полумарковские процессы~// 
Кибернетический сборник. Новая серия.~--- М.: Мир, 1967. Вып.~4. С.~97--134. 
(\Au{Jewell~W.\,S.}  Markov-renewal programming~I, II~// Oper. Res., 
1963. Vol.~11. No.\,6. P.~938--971.)
\bibitem{3} 
\Au{Luque-Vasquez F., Herndndez-Lerma~О.} Semi-Markov control 
models with average costs~// Appl. Math., 1999. Vol.~26. No.\,3. P.~315--331.
\bibitem{4} 
\Au{Vega-Amaya O., Luque-Vasquez~F.} Sample-path average cost 
optimality for semi-Markov control processes on Borel spaces: Unbounded costs 
and mean holding times~// Appl. Math., 2000. Vol.~27. No.\,3. P.~343--367.
\bibitem{5} 
Вопросы математической теории надежности~/ Под ред. Б.\,В.~Гнеденко.~--- М.: Радио и~связь, 1983. 376~с.

\bibitem{6} 
\Au{Барзилович Е.\,Ю., Каштанов~В.\,А.} Некоторые математические 
вопросы теории обслуживания сложных сис\-тем.~--- М.: Сов. радио, 1971. 272~с.

\bibitem{7} 
\Au{Шнурков П.\,В., Иванов~А.\,В.} Анализ дискретной 
полумарковской модели управления запасом непрерывного продукта при периодическом 
прекращении по-\linebreak\vspace*{-12pt}

\pagebreak

\noindent
требления~// Дискретная математика, 2014. Т.~26. №\,1. С.~143--154.

\bibitem{halm}
 \Au{Халмош~П.} Теория меры~/
 Пер. с~англ.~--- М.: ИЛ, 1953. 282~с.
(\Au{Halmos~P.} Measure theory.~--- Princeton, NJ, USA: Van Nostrand, 1950. 
304~p.)

\bibitem{8} 
\Au{Королюк В.\,С., Турбин~А.\,Ф.} Полумарковские процессы и~их 
приложения.~--- Киев: Наукова думка, 1976.  184~с.
\bibitem{9} 
\Au{Janssen J., Manca~R.} Applied semi-Markov processes.~--- New 
York, NY, USA: Springer, 2006. 309~p.

\bibitem{10} 
\Au{Шнурков П.\,В.} О~решении задачи безусловного экстремума для 
дроб\-но-ли\-ней\-но\-го интегрального функционала на множестве вероятностных мер~// 
Докл. Акад. наук, 2016. Т.~470. №\,4. С.~387--392.

\bibitem{11} 
\Au{Шнурков П.\,В., Горшенин~А.\,К., Белоусов~В.\,В.} 
Аналитическое решение задачи оптимального управления полумарковским процессом 
с~конечным множеством состояний~// Информатика и~ее применения, 2016. Т.~10. 
Вып.~4. С.~72--88.
\bibitem{12} 
\Au{Shnurkov~P.\,V.} Solution of the unconditional extremum 
problem for a~linear-fractional integral functional on a~set of probability 
measures and its applications in the theory of optimal control of  semi-Markov 
processes. \mbox{arXiv}:2001.06424v1 [math.OC], 2020. 26~p.
\end{thebibliography}

 }
 }

\end{multicols}

\vspace*{-3pt}

\hfill{\small\textit{Поступила в~редакцию 17.04.22}}

\vspace*{8pt}

%\pagebreak

%\newpage

%\vspace*{-28pt}

\hrule

\vspace*{2pt}

\hrule

%\vspace*{-2pt}

\def\tit{ON THE ANALYTICAL STRUCTURE OF~SOME KINDS OF~TARGET FUNCTIONALS ASSOCIATED 
WITH~THE~CONTROL PROBLEMS OF~SEMI-MARKOV STOСHASTIC PROCESSES}


\def\titkol{On the analytical structure of~some kinds of~target functionals associated 
with~the~control problems of~semi-Markov % stoсhastic 
processes}


\def\aut{P.\,V.~Shnurkov}

\def\autkol{P.\,V.~Shnurkov}

\titel{\tit}{\aut}{\autkol}{\titkol}

\vspace*{-6pt}


\noindent
National Research University Higher School of Economics, 34~Tallinskaya Str., Moscow 123458, 
Russian Federation

\def\leftfootline{\small{\textbf{\thepage}
\hfill INFORMATIKA I EE PRIMENENIYA~--- INFORMATICS AND
APPLICATIONS\ \ \ 2022\ \ \ volume~16\ \ \ issue\ 2}
}%
 \def\rightfootline{\small{INFORMATIKA I EE PRIMENENIYA~---
INFORMATICS AND APPLICATIONS\ \ \ 2022\ \ \ volume~16\ \ \ issue\ 2
\hfill \textbf{\thepage}}}

\vspace*{6pt} 



\Abste{The present author investigates the analytical structure of three kinds 
of functionals from a~controllable semi-Markov process with a~finite set of states. 
It is proved that all these mathematical objects can be represented in the form of 
a~fractional-linear integral functional defined on a~finite set of probability measures that 
determine the control strategy of the corresponding semi-Markov process. 
For each of these functionals, explicit representations for the integrand functions of 
the numerator and denominator through the initial probabilistic characteristics 
of the controlled semi-Markov process are obtained. This result allows one to reduce the
 problem of optimal control of a~semi-Markov process with a~particular target functional to the 
 problem of investigation on the global extremum of a~given function of a~finite number of variables.}

\KWE{stochastic control models; optimal control of semi-Markovian processes; 
partial-linear integral functional; basic function of partial-linear integral functional}

\DOI{10.14357/19922264220210}

%\vspace*{-16pt}

%\Ack
%\noindent




\vspace*{6pt}

  \begin{multicols}{2}

\renewcommand{\bibname}{\protect\rmfamily References}
%\renewcommand{\bibname}{\large\protect\rm References}

{\small\frenchspacing
 {%\baselineskip=10.8pt
 \addcontentsline{toc}{section}{References}
 \begin{thebibliography}{99}
\bibitem{1-sh-1}
\Aue{Mine, H., and S.~Osaki.} 1970. \textit{Markovian decision processes}. New York, NY: Elsevier. 142~p.
\bibitem{2-sh-1}
\Aue{Jewell, W.\,S.}
 1963. Markov-renewal programming. I, II. \textit{Oper. Res.} 11(6):938--971.
\bibitem{3-sh-1}
\Aue{Luque-Vasquez, F., and О.~Herndndez-Lerma.}
 1999. Semi-Markov control models with average costs. \textit{Appl. Math.} 26(3):315--331.
\bibitem{4-sh-1}
\Aue{Vega-Amaya, O., and F.~Luque-Vasquez.}
 2000. Sample-path average cost optimality for semi-Markov control processes on Borel spaces: 
 Unbounded costs and mean holding times. \textit{Appl. Math.} 27(3):343--367.
\bibitem{5-sh-1}
Gnedenko, B.\,V., ed. 1983. 
\textit{Voprosy matematicheskoy teorii nadezhnosti}
 [Questions of mathematics reliability theory]. Moscow: Radio i~svyaz'. 376~p.

 \columnbreak
 
\bibitem{6-sh-1}
\Aue{Barzilovich, E.\,Yu., and V.\,A.~Kashtanov.}
 1971. \textit{Nekotorye matematicheskie voprosy teorii obsluzhivaniya slozhnykh sistem}
  [Some mathematical questions in theory of complex systems maintenance]. Moscow: Sovetskoe radio. 272~p.
\bibitem{7-sh-1}
\Aue{Snurkov, P.\,V., and A.\,V.~Ivanov.}
 2015. Analysis of a~discrete semi-Markov model of continuous inventory control with periodic 
 interruptions of consumption. \textit{Discrete Math.} 25(1):59--67.
\bibitem{8-sh-1}
\Aue{Halmos, P.} 1950. \textit{Measure theory}. Princeton, NJ:
Van Nostrand. 304~p.
\bibitem{9-sh-1}
\Aue{Korolyuk, V.\,S., and A.\,F.~Turbin.}
 1976. \textit{Polumarkovskie protsessy i~ikh prilozheniya} [Semi-Markov processes and their applications]. 
 Kiev: Naukova Dumka. 184~p.
\bibitem{10-sh-1}
\Aue{Janssen, J., and R.~Manca.} 2006. \textit{Applied semi-Markov processes}. New York, NY: Springer. 309~p.

\pagebreak

\bibitem{11-sh-1}
\Aue{Shnurkov, P.\,V.}
 2016. Solution of the unconditional extremum problem for a~linear-fractional integral functional on 
 a~set of probability measures. \textit{Dokl. Math.} 94(2):550--554.
 
% \pagebreak
 
\bibitem{12-sh-1}
\Aue{Shnurkov, P.\,V., A.\,K.~Gorshenin, and V.\,V.~Belousov.}
 2016. Analiticheskoe reshenie zadachi optimal'nogo upravleniya polumarkovskim protsessom 
 s~konechnym mnozhestvom sostoyaniy [An analytic solution of the optimal control problem for 
 a~semi-Markov process with a~finite set of states]. \textit{Informatika i~ee Primeneniya~--- Inform. Appl.}
  10(4):72--88.
\bibitem{13-sh-1}
\Aue{Shnurkov, P.} 2020. Solution of the unconditional extremum problem for 
a~linear-fractional integral functional on a~set of probability measures and its applications in
 the theory of optimal control of  semi-Markov processes. \mbox{arXiv}.org. 26~p. 
 Available at: {\sf https://arxiv.org/ abs/2001.06424} (accessed April~29, 2022).
 \end{thebibliography}

 }
 }

\end{multicols}

\vspace*{-6pt}

\hfill{\small\textit{Received April 17, 2022}}
 
\Contrl

\noindent
\textbf{Shnurkov Peter V.} (b.\ 1953)~--- 
Candidate of Science (PhD) in physics and mathematics, associate professor, National Research University 
Higher School of Economics, 34~Tallinskaya Str., Moscow 123458, Russian Federation; \mbox{pshnurkov@hse.ru}


\label{end\stat}

\renewcommand{\bibname}{\protect\rm Литература}    