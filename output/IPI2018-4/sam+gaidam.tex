\def\stat{gaid+sam}

\def\tit{ПРИМЕНЕНИЕ МОДЕЛЕЙ СЛУЧАЙНОГО БЛУЖДАНИЯ ПРИ~МОДЕЛИРОВАНИИ 
ПЕРЕМЕЩЕНИЯ УСТРОЙСТВ В~БЕСПРОВОДНОЙ СЕТИ$^*$}

\def\titkol{Применение моделей случайного блуждания при моделировании 
перемещения устройств в~беспроводной сети}

\def\aut{К.\,Е.~Самуйлов$^1$, Ю.\,В.~Гайдамака$^2$, С.\,Я.~Шоргин$^3$}

\def\autkol{К.\,Е.~Самуйлов, Ю.\,В.~Гайдамака, С.\,Я.~Шоргин}

\titel{\tit}{\aut}{\autkol}{\titkol}

\index{Самуйлов К.\,Е.}
\index{Гайдамака Ю.\,В.}
\index{Шоргин С.\,Я.}
\index{Samouylov K.\,E.}
\index{Gaidamaka Yu.\,V.} 
\index{Shorgin S.\,Ya.}




{\renewcommand{\thefootnote}{\fnsymbol{footnote}} \footnotetext[1]
{Исследование выполнено при финансовой поддержке Российского научного фонда (проект 16-11-10227).}}


\renewcommand{\thefootnote}{\arabic{footnote}}
\footnotetext[1]{Российский университет дружбы народов; Федеральный исследовательский центр <<Информатика 
и~управление>> Российской академии наук, 
\mbox{samouylov\_ke@rudn.university}}
\footnotetext[2]{Российский университет дружбы народов; Федеральный исследовательский центр <<Информатика 
и~управление>> Российской академии наук, \mbox{gaydamaka\_yuv@rudn.university}}
\footnotetext[3]{Институт проблем информатики Федерального исследовательского центра <<Информатика 
и~управление>> Российской академии наук, \mbox{ssorgin@ipiran.ru}}

%\vspace*{8pt}

   
  
  \Abst{Выполнен обзор моделей случайного блуж\-да\-ния объектов, применяемых при 
моделировании передвижения при\-емо\-пе\-ре\-да\-ющих устройств пользователей 
беспроводной сети пятого поколения (5G). Рассмотрены модели мо\-биль\-ности, характерные для 
имитационного моделирования поведения пользователей беспроводной 
самоорганизующейся сети. Обсуждаются особенности различных моделей 
индивидуального движения объектов, а~так\-же моделей движения групп объектов с~точки 
зрения применения к~анализу интерференции в~беспроводных сетях. Цель статьи~--- 
предложить ряд моделей мо\-биль\-ности для принятия обоснованного решения при выборе 
модели случайного блуж\-да\-ния для анализа качества предо\-став\-ле\-ния услуг 
в~беспроводных сетях. В~качестве иллюстрации применения разработанного авторами 
комплекса аналитических и~имитационных моделей проведен анализ отношения 
сигнал/\-ин\-тер\-фе\-рен\-ция, определяющего качество пред\-остав\-ле\-ния 
услуг в~сетях пятого 
поколения, для сценария случайного блуж\-да\-ния мобильных абонентов 
в~тор\-го\-во-раз\-вле\-ка\-тель\-ном цент\-ре при использовании 
модели случайного блуж\-да\-ния с~остановками Random Waypoint.}
  
  \KW{модель случайного блуждания; модель мобильности; отношение  
сиг\-нал/ин\-тер\-фе\-рен\-ция; отношение сиг\-нал/шум}

\DOI{10.14357/19922264180401}
  
\vspace*{5pt}


\vskip 10pt plus 9pt minus 6pt

\thispagestyle{headings}

\begin{multicols}{2}

\label{st\stat}

\section{Введение}

  В беспроводных сетях 5G интерференция служит одним из существенных 
источников помех, вли\-я\-ющим на показатели качества функционирования 
сети, к~которым относятся пиковые ско\-рости передачи данных между 
устройствами, задержка\linebreak начала передачи данных, отношение 
сигнал/\-ин\-тер\-фе\-рен\-ция (signal to interference ratio, SIR), энер\-го\-сбе\-ре\-же\-ние, 
эффективность использования частот\-но\-го спектра и~др.~[1]. При анализе 
интерференции\linebreak следует учитывать особенности современных беспроводных 
сетей, которые при использовании в~них технологии прямого взаимодействия 
оконечных устройств (device-to-device, D2D) образуют самоорганизующиеся 
сети (mobile ad hoc network, MANET) с~перемещающимися в~зоне покрытия 
узлами.\linebreak
 Относительно небольшие расстояния между подвижными узлами, 
соответствующими при\-емо\-пе\-ре\-да\-ющим устройствам, делают необходимым 
при анализе интерференции между соседними источниками сигнала в~таких 
сетях учитывать траектории перемещения узлов, которые фактически 
определяют динамику показателя SIR в~радиоканале между приемником 
и~передатчиком. 

В~работах~[2, 3] перемещение беспроводных устройств 
моделировалось с~по\-мощью ки\-не\-ти\-ческого уравнения  
Фок\-ке\-ра--Планка~[4], регулирование па\-ра\-мет\-ров (снос, диффузия) 
которого позволяет исследовать различные типы движения большого чис\-ла 
объектов, не строя индивидуальную траекторию перемещения каждого 
объекта. 

Однако такой подход не применим при анализе интерференции 
в~задачах, где необходимо принимать во внимание особенности 
предоставления услуг передачи данных в~сети, например учитывать 
препятствия в~зоне перемещения устройств, наличие нескольких сред 
распространения сигнала и~другие ограничения. Для решения таких задач 
необходимо детальное моделирование траектории движения каждого 
беспроводного устройства с~по\-мощью аналитических~[5--7] 
и~имитационных~[8, 9] моделей. 




В~статье проведен обзор моделей 
мо\-биль\-ности объектов, применяемых при моделировании перемещения 
устройств в~беспроводных самоорганизующихся сетях. В~разд.~2 
об\-суж\-да\-ют\-ся особенности различных моделей индивидуального движения 
объектов, а~также моделей движения\linebreak
 групп объектов с~точки зрения 
применения к~анализу интерференции в~беспроводных сетях. В~качест-\linebreak ве 
иллюстрации в~разд.~3 на примере прикладной\linebreak задачи анализа движения 
мобильных абонентов в~тор\-го\-во-раз\-вле\-ка\-тель\-ном центре с~по\-мощью 
разработанного авторами комплекса аналитических и~имитационных 
моделей на наборе исходных данных, близ\-ких к~реальным, проведен расчет 
показателя SIR, определяющего качество пред\-остав\-ле\-ния услуг в~сетях 
пятого поколения.

\vspace*{-8pt}

\section{Модели случайного блуждания}

\vspace*{-2pt}

  В настоящее время для имитационного моделирования передвижения 
беспроводных устройств в~сетях MANET традиционно используются как 
модели мо\-биль\-ности объектов, основанные на сборе и~анализе статистики 
движения абонентов в~реальных беспроводных сетях (traces), так 
и~синтетические (synthetic) модели~[8--10]. Первые построены на основе 
обработки данных от большого чис\-ла узлов сети, со\-бран\-ных в~течение 
длительного периода наблюдения, поэтому обеспечивают достоверное 
моделирование, однако их применение возможно лишь для анализа уже 
су\-ще\-ст\-ву\-ющих сетей. Поскольку сети~5G в~полной мере еще не 
реализованы, востребованными оказались синтетические модели, с~помощью 
которых мож\-но реалистично воспроизводить поведение абонентов 
беспроводной сети, регулируя правила изменения ско\-рости и~на\-прав\-ле\-ния 
движения мобильных узлов. Например, мобильные узлы не долж\-ны иметь 
прямую траекторию движения и~постоянную ско\-рость в~течение всего 
времени моделирования, потому что в~реальных сетях движение абонентов 
имеет более слож\-ный характер. Как правило, для моделирования движения 
абонентов беспроводной сети\linebreak
 используются как модели мо\-биль\-ности, 
опи\-сы\-ва\-ющие независимое друг от друга движение абонен\-тов беспроводной 
сети, модели так на\-зы\-ва\-емой <<индивидуальной>> мо\-биль\-ности (entity 
mobility models), так и~модели <<групповой>> мо\-биль\-ности (group mobility 
models), в~которых, например, движение группы основано на траектории 
логического цент\-ра~\cite{9-sg, 10-sg, 11-sg}.
  
  Примерами моделей индивидуальной мобильности, использующихся для 
описания перемещения независимых объектов, являются модель RW 
(Random Walk), для которой в~каж\-дой точке на\-прав\-ле\-ние и~ско\-рость 
движения разыгрываются случайным образом (рис.~1,\,\textit{а}), и~ее 
расширение~--- модель RWP (Random WayPoint), в~которой пред\-усмот\-ре\-но 
время остановки в~каж\-дой точке перед продолжением движения 
(рис.~1,\,\textit{б}). 

Модели RW и~RWP чаще других используются для 
моделирования движения мобильных устройств в~самоорганизующейся 
беспроводной сети. Особенностью обеих моделей является отсутствие 
<<памяти>>~--- текущие значения па\-ра\-мет\-ров модели (на\-прав\-ле\-ние, 
ско\-рость, длительность остановки) не зависят от значений этих па\-ра\-мет\-ров 
на прошлом шаге, что приводит к~генерации траекторий с~внезапными 
остановками и~резкими поворотами. При небольших значениях ско\-рости 
в~модели RW движение объектов становится броуновским; следовательно, 
эту модель мож\-но рекомендовать для анализа интерференции в~статической 
сети. 

Интересным развитием модели RWP является ее вероятностная версия, 
в~которой следующая позиция мобильного узла определяется в~соответствии 
с~заданными вероятностями. Недостатком модели RWP является замеченная 
осо\-бен\-ность~--- при достаточно длительном периоде работы имитационной 
модели плот\-ность объектов, в~начале моделирования распределенных 
равномерно, по краям об\-ласти моделирования становится заметно ниже, чем 
в~цент\-ре. Поскольку рас\-сто\-яние до передатчика является клю\-че\-вым 
фактором, оказывающим вли\-яние на мощ\-ность фик\-си\-ру\-емо\-го на приемнике 
сигнала, безосновательное увеличение чис\-ла ближайших интерферирующих 
передатчиков при оценке\linebreak
 показателя SIR исказит вывод о~качестве 
со\-еди\-не\-ния в~мо\-де\-ли\-ру\-емой сети. Однако при моделировании некоторых 
сценариев поведения пользователей, например осмот\-ра музея в~соответствии\linebreak 
с~пред\-ла\-га\-емой схемой знакомства с~экспозицией, модель случайного 
блуж\-да\-ния RWP за счет своей гиб\-кости создает реалистичные траектории 
движения объектов. Кроме того, описанный эффект скопления объектов 
в~цент\-ре об\-ласти моделирования практически исчезает для случая дол\-гих 
остановок даже при высоких значениях па\-ра\-мет\-ра ско\-рости~\cite{10-sg}.


  Указанного для модели RWP недостатка лишены модель RD (Random 
Direction), для которой на\-прав\-ле\-ние и~ско\-рость движения меняются при 
достижении объектом границы об\-ласти моделирования (рис.~1,\,\textit{в}), 
а~так\-же модель движения Гаус\-са--Мар\-ко\-ва (Gauss--Markov, GM), 
которая позволяет получить плав\-ную траекторию движения объекта 
(рис.~1,\,\textit{г}). В~\cite{10-sg} описан метод, который для модели GM 
принудительно меняет на\-прав\-ле\-ние движения объекта при при\-бли\-же\-нии 
к~границе об\-ласти моделирования, что позволяет избежать нежелатель-\linebreak\vspace*{-12pt}

\pagebreak

\end{multicols}

\begin{figure*} %fig1
\vspace*{1pt}
 \begin{center}
 \mbox{%
 \epsfxsize=163.242mm 
 \epsfbox{gai-1.eps}
 }
 \end{center}
\vspace*{-11pt}
\Caption{Примеры траекторий перемещения объекта при случайном блуж\-да\-нии: 
(\textit{а})~модель RW; (\textit{б})~модель RWP; (\textit{в})~модель RD; 
(\textit{г})~модель GM}
\vspace*{-4pt}
\end{figure*}

\begin{multicols}{2} 

\noindent
ных
эффектов <<прилипания>> объекта к~краю об\-ласти. 

Характерная для всех 
упомянутых выше моделей проб\-ле\-ма <<краевого эффекта>> при 
приближении объекта к~границе об\-ласти отсутствует в~модели Boundless 
Simulation Area (BSA), об\-ласть моделирования которой пред\-став\-ля\-ет собою 
тор. В~этой модели текущие значения на\-прав\-ле\-ния и~ско\-рости движения 
зависят от значений этих па\-ра\-мет\-ров на прошлом шаге, что создает 
реалистичную траекторию движения объекта. Однако при моделировании 
беспроводной сети с~по\-мощью модели BSA не избежать искажения 
динамики показателя SIR, поскольку движущееся по по\-верх\-ности тора 
беспроводное устройство регулярно становится соседом ка\-ко\-го-ли\-бо 
неподвижного беспроводного устройства. 

Еще одной моделью 
индивидуальной мо\-биль\-ности является так на\-зы\-ва\-емое <<вероятностное 
блуж\-да\-ние по сетке>> (Probabilistic Grid, PG)~--- модель в~дискретном 
времени, со\-глас\-но которой на каждом временн$\acute{\mbox{о}}$м такте объект делает шаг 
единичной длины, а~выбор одного из четырех на\-прав\-ле\-ний задается 
вероятностной мат\-ри\-цей~\cite{10-sg}. Благодаря простоте реализации эта 
модель так\-же широко применяется при моделировании движения, однако 
задание вероятностной мат\-ри\-цы для конкретного сценария поведения 
пользователей пред\-став\-ля\-ет определенную труд\-ность.
  
  К моделям групповой мо\-биль\-ности~\cite{10-sg, 11-sg} относятся модель 
ECRM (Exponential Correlated Random Mobility), основанная на 
экспоненциальной за\-ви\-си\-мости ско\-рости движения объектов; модель CM 
(Column Mobility), в~которой моделируется движение объектов, вы\-стро\-ен\-ных 
в~линию; модель перемещения кочевников NCM (Nomadic Community 
Mobility); модель преследования (Pursue Mobility),\linebreak
 в~которой группа следует 
за лидером, пе\-ре\-дви\-гающимся по заданной траектории, а~также на\-и\-более 
общая модель групповой мо\-биль\-ности\linebreak
 с~опорной точ\-кой RPGM (Reference 
Point Group Mobility), в~которой пред\-усмот\-ре\-но случайное движение группы 
с~одновременным случайным перемещением каждого отдельного объекта 
внут\-ри группы (рис.~2). Недостатком модели ECRM, которая позволяет 
описать практически все виды групповой мо\-биль\-ности, является 
существенная слож\-ность подбора па\-ра\-мет\-ров модели.

\begin{figure*} %fig2
\vspace*{1pt}
 \begin{center}
 \mbox{%
 \epsfxsize=158.36mm 
 \epsfbox{gai-2.eps}
 }
 \end{center}
\vspace*{-11pt}
\Caption{Пример траекторий для модели групповой мобильности с~опорной точ\-кой 
RPGM для трех объектов: (\textit{а})~перемещение опорной точ\-ки; 
(\textit{б})~траектории перемещения объектов}
\vspace*{-4pt}
\end{figure*}

  Перечисленные модели движения традиционно используются при 
исследовании про\-из\-во\-ди\-тель\-ности различных сетевых протоколов, 
при\-ме\-ня\-емых в~са\-мо\-ор\-га\-низу\-ющих\-ся беспроводных сетях, при этом 
сравнение проводится по таким показателям, как переданные полезная 
и~служебная на\-груз\-ка, джиттер, межконцевая за\-держ\-ка, за\-тра\-ты на 
маршрутизацию~\cite{11-sg}. При выборе модели мо\-биль\-ности с~целью 
исследования интерференции важно учитывать сценарий поведения 
пользователей. Наиболее универсальными моделями индивидуальной 
мо\-биль\-ности являются модель Random Waypoint и~модель  
Гаус\-са--Мар\-ко\-ва, настройка па\-ра\-мет\-ров которых позволяет гиб\-ко 
под\-стро\-ить\-ся под большинство сценариев. Для воспроизведения 
перемещения группы пользователей беспроводной сети рекомендуется 
использовать модель групповой мо\-биль\-ности с~опорной точ\-кой Reference 
Point Group Mobility, которая при со\-от\-вет\-ст\-ву\-ющих значениях па\-ра\-мет\-ров 
позволяет реализовать модели Column, Nomadic Community и~Pursue. 

\section{Пример анализа интерференции при~случайном~блуждании} 
  
  Одной из основных характеристик качества канала в~беспроводных сетях 
связи служит отношение уров\-ня сигнала к~уров\-ню интерференции и~шума 
(ОСШ, \textit{англ}.\ Signal to Interference and Noise Ratio, SINR), которое 
определяется отношением мощ\-ности принимаемого сигнала от 
соответствующего передатчика к~суммарной мощ\-ности шума 
и~при\-ни\-ма\-емо\-го сигнала от интерферирующих передатчиков~\cite{12-sg}. 
При этом мощ\-ности фик\-си\-ру\-емо\-го на приемнике сигнала как от целевого, 
так и~от каждого из интерферирующих передатчиков определяются 
с~соответствии с~классической моделью распространения сигнала, а~именно: 
прямо пропорционально базовой мощ\-ности сигнала передатчика и~обратно 
пропорционально рас\-сто\-янию меж\-ду передатчиком и~приемником 
в~некоторой по\-сто\-ян\-ной степени, зависящей от среды рас\-про\-стра\-не\-ния 
сигнала. Как и~в~\cite{12-sg, 13-sg}, для оценки отношения SIR далее 
используется формула 
$$
\mathrm{SIR}= \fr{r_0^{-\gamma_0}}{\sum\nolimits^N_{n=1} 
d_n^{-\gamma_n}}\,,
$$
 где $r_0$~--- рас\-сто\-яние между приемником 
и~передатчиком в~ис\-сле\-ду\-емой целевой паре; $d_n$~--- рассто\-яние между 
приемником целевой пары и~передатчиком $n$-й ин\-тер\-фе\-ри\-ру\-ющей пары; 
$\gamma$~--- коэффициент распространения сигнала, ха\-рак\-те\-ри\-зу\-ющий 
среду передачи (от~2 в~условиях прямой ви\-ди\-мости до~6 в~худшем случае, 
при котором возможна связь). Расчет проведен в~предположении о~рав\-ных 
из\-лу\-ча\-емых мощностях и~коэффициентах усиления приемной и~пе\-ре\-да\-ющей 
антенн для всех устройств. Для моделирования препятствий для 
про\-хож\-де\-ния сигнала использовались различные значения коэффициентов 
рас\-про\-стра\-не\-ния сигнала~$\gamma_n$, $n\hm= 0,1,\ldots, N$~\cite{13-sg}.
  
  На рис.~3 для одной из моделей индивидуальной мо\-биль\-ности, модели 
случайного блуж\-да\-ния RWP, приведена кривая, по\-ка\-зы\-ва\-ющая изменение 
показателя SIR в~течение~500~с. Для на\-гляд\-ности вы\-бран случай блуж\-да\-ния 
по сетке четырех мобильных \mbox{устройств}~--- целевой пары  
при\-ем\-ник-пе\-ре\-дат\-чик и~двух интерферирующих передатчиков, 
работающих на близ\-ких час\-тотах.

\setcounter{figure}{3}
\begin{figure*}[b] %fig4
\vspace*{1pt}
 \begin{center}
 \mbox{%
 \epsfxsize=162.925mm 
 \epsfbox{gai-4.eps}
 }
 \end{center}
\vspace*{-11pt}
 \Caption{Траектории и~взаимное расположение устройств: 
(\textit{а})~минимальное значение SIR (402-я секунда); (\textit{б})~максимальное 
значение SIR (449-я секунда)}
 \end{figure*}
  
  
     
  Моделировалось целенаправленное движение, когда пользователи, 
носители мобильных\linebreak\vspace*{-12pt}

{ \begin{center}  %fig3
 \vspace*{0.5pt}
  \mbox{%
 \epsfxsize=79mm 
 \epsfbox{gai-3.eps}
 }


\vspace*{6pt}


\noindent
{{\figurename~3}\ \ \small{Динамика SIR}}
\end{center}
}

\vspace*{6pt}

\addtocounter{figure}{1}

\noindent
 устройств, перемещались по кратчайшему пути между 
заранее выбранными точ\-ка\-ми своих маршрутов с~по\-сто\-ян\-ной 
ско\-ростью~1~м/с независимо друг от друга в~квад\-ра\-те $500\times500$~м. 
Такой сценарий характерен, например, для последовательного посещения 
магазинов тор\-го\-во-раз\-вле\-ка\-тель\-но\-го цент\-ра по заранее 
намеченному маршруту. Траектории устройств показаны на рис.~4, где 
перемещение целевого приемника, на котором оценивалось отношение 
сигнал/\-ин\-тер\-фе\-рен\-ция, показано сплош\-ной линией, перемещение 
передатчиков~--- пунктирными линиями, при этом целевому передатчику 
соответствует траектория с~самым длинным размером штриха. 
  
  На рис.~4 точки, отмеченные крестиками на соответствующих 
траекториях, позволяют судить о~взаимном расположении устройств. Для 
на\-гляд\-ности целевые передатчик и~приемник соединены показывающим 
на\-прав\-ле\-ние передачи в~радиоканале вектором, модуль которого равен 
рас\-сто\-янию в~целевой паре. Так, на рис.~4,\,\textit{а}, который соответствует 
402-й секунде моделирования, один из интерферирующих передатчиков 
расположен значительно ближе к~приемнику, чем целевой передатчик, а~на 
рис.~4,\,\textit{б} отражена обратная ситуация, когда на 449-й секунде 
расстояние в~целевой паре становится минимальным. Соответствующие 
локальные экстремумы показателя SIR в~указанные моменты отражены на 
рис.~3.
  
  
  Предложенный метод расчета SIR, основанный на моделировании 
траекторий движения устройств, поз\-во\-ля\-ет оценивать качество 
предо\-став\-ле\-ния услуг в~сети при заданных для каждой услуги требованиях 
к~минимальному допустимому значению этого показателя.

\section{Заключение}

  Проведенный в~работе обзор моделей случайного блуж\-да\-ния, традиционно 
применяемых для моделирования перемещения мобильных узлов 
в~беспроводных са\-мо\-ор\-га\-ни\-зу\-ющих\-ся сетях, поз\-во\-ля\-ет при выборе модели 
для проведения эксперимента учесть особенности каж\-дой модели, 
существенные с~точ\-ки зрения сценария поведения пользователей. 
Универсальной модели мо\-биль\-ности, способной воспроизвести поведение 
пользователя при любом сценарии, не существует, поэтому анализ 
интерференции рекомендуется проводить, применяя несколько моделей 
движения объектов. Также задачей дальнейших исследований может стать 
разработка новой комбинированной модели мо\-биль\-ности для 
воспроизведения перемещения пользователей беспроводной 
самоорганизующейся сети, сочетающей подход модели Gauss--Markov на 
границе об\-ласти моделирования и~принцип перемещения объектов модели 
Random Waypoint внутри об\-ласти, таким образом сохраняя преимущества 
и~компенсируя недостатки каж\-дой из этих моделей.

\bigskip

Авторы выражают благодарность магистрам кафедры прикладной 
информатики и~тео\-рии вероятностей РУДН А.~Жданкову и~О.~Крупко за 
подготовку иллюстраций к~статье по разработанному комплексу 
аналитических и~имитационных моделей.
  
{\small\frenchspacing
 {%\baselineskip=10.8pt
 \addcontentsline{toc}{section}{References}
 \begin{thebibliography}{99}
\bibitem{1-sg}
\Au{Andrews J.\,G., Buzzi~S., Choi~W., Hanly~S.\,V., Lozano~A., Soong~A.\,C., 
Zhang~J.\,C.} What will 5G be?~// IEEE J.~Sel. Area. Comm., 2014. 
Vol.~32. No.\,6. P.~1065--1082. doi: 10.1109/JSAC.2014.2328098.
\bibitem{2-sg}
\Au{Orlov Yu.\,N., Fedorov~S.\,L., Samuylov~A.\,K., Gaidamaka~Yu.\,V., 
Molchanov~D.\,A.} Simulation of devices mobility to estimate wireless channel quality 
metrics in 5G networks~// AIP Conf. Proc., 2017. Vol.~1863.  
P.~090005-1--090005-3. doi: 10.1063/1.4992270.
\bibitem{3-sg}
\Au{Гайдамака Ю.\,В., Орлов Ю.\,Н., Молчанов~Д.\,А., Самуйлов~А.\,К.} 
Моделирование отношения сиг\-нал/ин\-тер\-фе\-рен\-ция в~мобильной сети со 
случайным блужданием взаимодействующих устройств~// Информатика и~её 
применения, 2017. Т.~11. Вып.~2. С.~50--58. doi: 10.14357/19922264170206.
\bibitem{4-sg}
\Au{Risken~H., Frank T.} The Fokker--Planck equation: Methods of solution and 
applications.~--- Springer ser. in synergetics.~--- Berlin--Heidelberg: Springer-Verlag, 1996. 
Vol.~18. 486~p. doi: 10.1007/978-3-642-61544-3.

\bibitem{5-sg}
\Au{Тоффоли Т., Марголус~Н.} Машины клеточных автоматов~/ Пер. с~англ.~--- М.: 
Мир, 1991. 283~с. (\Au{Toffoli~T., Margolus~N.} Cellular automata machines.~--- 
The MIT Press, 1987. 276~p.)
\bibitem{6-sg}
\Au{Grewal M.\,S., Andrews A.\,P.} Kalman filtering: Theory and practice
using MATLAB.~--- 2nd ed.~--- John Wiley \& Sons, Inc., 2001. 410~p.
\bibitem{7-sg}
\Au{Семушин И.\,В., Цыганов~А.\,В., Цыганова~Ю.\,В., Голубков~А.\,В., 
Винокуров~С.\,Д.} Моделирование и~оцени\-ва\-ние траектории движущегося объекта~// 
Вестник\linebreak  ЮУрГУ. Сер. Матем. моделирование и~программирование, 2017. Т.~10. №\,3. 
С.~108--119. doi: 10.14529/ mmp170309.
\bibitem{8-sg}
The VINT Project (Virtual InterNetwork Testbed). The Network Simulator~--- ns-2. {\sf 
http://www.isi.edu/ nsnam/ns}.
\bibitem{9-sg}
Wolfram Demonstrations Project. Constrained Random Walk.
 {\sf http://demonstrations.wolfram.com}.
\bibitem{10-sg}
\Au{Camp T., Boleng J., Davies~V.} A~survey of mobility models for ad hoc network 
research~// Wirel. Commun. Mob. Com., 2002. No.\,2. P.~483--502. doi: 
10.1002/wcm.72.
\bibitem{11-sg}
\Au{Talwar G., Narang~H., Pandey~K., Singhal~P.} Analysis of different mobility models 
for ad hoc on-demand distance vector routing protocol and dynamic source routing 
protocol.~--- Lecture notes in electrical engineering ser.~--- New York, NY, USA: Springer, 
2013. Vol.~131. P.~579--588. doi: 10.1007/978-1-4614-6154-8\_57. 
\bibitem{12-sg}
\Au{Отт Г.} Методы подавления шумов и~помех в~электронных системах~/
Пер. с~англ.~--- М.: 
Мир, 1979. 318~с.
(\Au{Ott~G.} {Noise reduction techniques 
in electronic systems}.~--- 
New York, NY, USA: Wiley, 1976. 312~p.)
\bibitem{13-sg}
\Au{Гайдамака Ю.\,В., Андреев~С.\,Д., Сопин~Э.\,С., Самуйлов~К.\,Е., Шоргин~С.\,Я.} 
Анализ характеристик интерференции в~модели взаимодействия устройств с~учетом 
среды распространения сигнала~// Информатика и~её применения, 2016. Т.~10. 
Вып.~4. С.~2--10. doi: 10.14357/19922264160401.

 \end{thebibliography}

 }
 }

\end{multicols}

\vspace*{-3pt}

\hfill{\small\textit{Поступила в~редакцию 11.09.18}}

\vspace*{8pt}

%\pagebreak

%\newpage

%\vspace*{-28pt}

\hrule

\vspace*{2pt}

\hrule

%\vspace*{-2pt}

\def\tit{MODELING MOVEMENT OF DEVICES IN~A~WIRELESS NETWORK BY~RANDOM WALK MODELS}

\def\titkol{Modeling the movement of devices in a~wireless network by random 
walk models}

\def\aut{K.\,E.~Samouylov$^{1,2}$, Yu.\,V.~Gaidamaka$^{1,2}$, 
and~S.\,Ya.~Shorgin$^3$}

\def\autkol{K.\,E.~Samouylov, Yu.\,V.~Gaidamaka, 
and~S.\,Ya.~Shorgin}

\titel{\tit}{\aut}{\autkol}{\titkol}

\vspace*{-11pt}


\noindent
$^1$Peoples' Friendship University of Russia (RUDN University), 6~Miklukho-Maklaya Str., 
Moscow 117198, Russian\linebreak
$\hphantom{^1}$Federation

\noindent
$^2$Federal Research Center ``Computer Science and Control'' of the Russian Academy of 
Sciences, 44-2~Vavilov\linebreak
$\hphantom{^1}$Str., Moscow 119333, Russian Federation

\noindent
$^3$Institute of Informatics Problems, 
Federal Research Center ``Computer Science and Control'' of the Russian\linebreak
$\hphantom{^1}$Academy of Sciences, 44-2~Vavilov Str., Moscow 119333, Russian Federation


\def\leftfootline{\small{\textbf{\thepage}
\hfill INFORMATIKA I EE PRIMENENIYA~--- INFORMATICS AND
APPLICATIONS\ \ \ 2018\ \ \ volume~12\ \ \ issue\ 4}
}%
 \def\rightfootline{\small{INFORMATIKA I EE PRIMENENIYA~---
INFORMATICS AND APPLICATIONS\ \ \ 2018\ \ \ volume~12\ \ \ issue\ 4
\hfill \textbf{\thepage}}}

\vspace*{6pt}


\Abste{The authors overview mobility models which are applicable 
for simulation of  movement of users' devices in a~fifth generation (5G)
wireless network. Mobility patterns that are typical for simulating the 
behavior of users of a~wireless ad hoc network are considered. The features 
of the models are discussed, both for individual motion of objects and 
for motion of groups of objects, from the point of view of appliance to the 
analysis of interference in wireless networks. The purpose of the paper is 
to propose a~number of mobility models for making an informed decision when 
choosing a~model for evaluating the quality of service in 5G wireless networks. 
The authors present\linebreak\vspace*{-12pt}}

\Abstend{simulation results that illustrate the method of 
estimation of the key performance quality parameter, i.\,e., 
signal to interference ratio. For illustration, the developed complex of 
analytical and simulation models is used for simulation of movement of shopping moll 
customers with the help of the grid random walk mobility model.}

\KWE{entity mobility model; group mobility model; ad hoc network simulation; 
signal to interference and noise ratio; SINR}
  

  
\DOI{10.14357/19922264180401}

\vspace*{-16pt}

\Ack
\noindent
The reported study was supported by the Russian Science Foundation, 
research project  No.\,16-11-10227.



%\vspace*{-2pt}

  \begin{multicols}{2}

\renewcommand{\bibname}{\protect\rmfamily References}
%\renewcommand{\bibname}{\large\protect\rm References}

{\small\frenchspacing
 {%\baselineskip=10.8pt
 \addcontentsline{toc}{section}{References}
 \begin{thebibliography}{99}
\bibitem{1-sg-1}
\Aue{Andrews, J.\,G., S.~Buzzi, W.~Choi, S.\,V.~Hanly, A.~Lozano, A.\,C.~Soong., and 
J.\,C.~Zhang.} 2014. What will 5G be? \textit{IEEE J.~Sel. Area. Comm.} 
 32(6):1065--1082. doi: 10.1109/ JSAC.2014.2328098.
\bibitem{2-sg-1}
\Aue{Orlov, Yu.\,N., S.\,L.~Fedorov, A.\,K.~Sa\-muylov, Yu.\,V.~Gai\-da\-ma\-ka, and 
D.\,A.~Molchanov.} 2016. Simulation of devices mobility to estimate wireless channel 
quality metrics in 5G networks. \textit{AIP Conf. Proc.}  
1863:090005-1--090005-3. 2017. doi: 
10.1063/1.4992270.
\bibitem{3-sg-1}
\Aue{Gaidamaka, Yu.\,V., Yu.\,N.~Orlov, D.\,A.~Molchanov, and A.\,K.~Samuylov.} 
2017. Modelirovanie otnosheniya signal/\linebreak interferentsiya v~mobil'noy seti so sluchaynym 
bluzhdaniem vzaimodeystvuyushchikh ustroystv [Modeling the signal--interference ratio in 
a~mobile network with moving devices]. \textit{Informatika i~ee Primeneniya~--- Inform. 
Appl.} 11(2):50--58. doi: 10.14357/19922264170206.
\bibitem{4-sg-1}
\Aue{Risken, H., and T.~Frank.} 1996. 
\textit{The Fokker--Planck equation: Methods of solution 
and applications}. Springer ser. in synergetics.  Berlin--Heidelberg: 
Springer-Verlag.  Vol.~8.
 486~p. doi: 10.1007/978-3-642-61544-3.
\bibitem{5-sg-1}
\Aue{Toffoli, T., and N.~Margolus.} 1987. \textit{Cellular automata machines}. 
The MIT Press. 276~p.
\bibitem{6-sg-1}
\Aue{Grewal, M.\,S., and A.\,P.~Andrews.} 2001. \textit{Kalman filtering: 
Theory and practice using MATLAB}. 2nd ed.  John Wiley \& Sons, Inc. 410~p.

\bibitem{7-sg-1}
\Aue{Semushin, I.\,V., A.\,V.~Tsyganov, Yu.\,V.~Tsyganova, A.\,V.~Golubkov, and 
S.\,D.~Vinokurov.} 2017. Modelirovanie i~otsenivanie traektorii dvizhushchegosya ob''ekta 
[Modelling and estimation of a~moving object trajectory]. 
\textit{South Ural State University Bulletin. Ser. Mathematical Modelling, 
Programming \& Computer Software} 10(3):108--119. doi: 10.14529/mmp170309.
\bibitem{8-sg-1}
The VINT Project (Virtual InterNetwork Testbed). The Network Simulator ns-2. Available 
at: {\sf http://www.isi. edu/nsnam/ns/} (accessed September~10, 2018).
\bibitem{9-sg-1}
The Wolfram Demonstrations Project. Constrained Random Walk. Available at: {\sf 
http://demonstrations.wolfram.\linebreak com} (accessed September~10, 2018).
\bibitem{10-sg-1}
\Aue{Camp,~T., J.~Boleng, and V.~Davies.} 2002. A~survey of mobility models for ad hoc 
network research. \textit{Wirel. Commun. Com.} (2):483--502.
\bibitem{11-sg-1}
\Aue{Talwar, G., H.~Narang, K.~Pandey, and P.~Singhal.} 2013. Analysis of different 
mobility models for ad hoc on-demand distance vector routing protocol and dynamic 
source routing protocol. {Lecture notes in electrical engineering ser.} New York, NY: 
Springer. 131:579--588. doi: 10.1007/978-1-4614-6154-8\_57. 
\bibitem{12-sg-1}
\Aue{Ott, G.} 1976. \textit{Noise reduction techniques 
in electronic systems}. 
New York, NY: Wiley. 312~p.
\bibitem{13-sg-1}
\Aue{Gaidamaka, Yu.\,V., S.\,D.~Andreev, E.\,S.~Sopin, K.\,E.~Sa\-mouylov, and 
S.\,Ya.~Shorgin.} 2016. Analiz kharakteristik interferentsii v~modeli vzaimodeystviya 
ustroystv s~uche\-tom sredy rasprostraneniya signala [Analysis of the characteristics of the 
interference in the model of interaction between devices taking into account the signal 
propagation environment]. \textit{Informatika i~ee Primeneniya~--- Inform. Appl.}  
10(4):2--10. doi:10.14357/19922264160401.
\end{thebibliography}

 }
 }

\end{multicols}

\vspace*{-6pt}

\hfill{\small\textit{Received September 11, 2018}}

%\pagebreak

\vspace*{-18pt}

\Contr

\noindent
\textbf{Samouylov Konstantin E.} (b.\ 1955)~--- Doctor of Science in technology, professor, Head of Department, Director 
of Institute of Applied Mathematics and Telecommunications, Peoples' Friendship University of Russia (RUDN University), 
6~Miklukho-Maklaya Str., Moscow 117198, Russian Federation; senior scientist, Federal Research Center 
``Computer Science and Control'' of the Russian Academy of Sciences, 44-2~Vavilov Str., Moscow 119333, Russian Federation, 
\mbox{samuylov\_ke@rudn.university}

\vspace*{1pt}

\noindent
\textbf{Gaidamaka Yuliya V.} (b.\ 1971)~--- Doctor of Science in physics and 
mathematics, professor, Peoples' Friendship University of Russia (RUDN 
University), 6~Miklukho-Maklaya Str., Moscow 117198, Russian Federation; 
senior scientist, Federal Research Center ``Computer Science and Control'' of the 
Russian Academy of Sciences, 44-2~Vavilov Str., Moscow 119333, Russian 
Federation, \mbox{gaydamaka\_yuv@rudn.university}

\vspace*{1pt}

\noindent
\textbf{Shorgin Sergey Ya.} (b.\ 1952)~--- Doctor of Science in physics and 
mathematics, professor, principal scientist, Institute of Informatics Problems, 
Federal Research Center ``Computer Science and Control'' of the Russian 
Academy of Sciences, 44-2~Vavilov Str., Moscow 119333, Russian Federation; 
\mbox{sshorgin@ipiran.ru}
\label{end\stat}

\renewcommand{\bibname}{\protect\rm Литература}       