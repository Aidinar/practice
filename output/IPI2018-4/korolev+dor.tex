\def\stat{kor+dor}

\def\tit{О НЕРАВНОМЕРНЫХ ОЦЕНКАХ ТОЧНОСТИ НОРМАЛЬНОЙ АППРОКСИМАЦИИ 
ДЛЯ~РАСПРЕДЕЛЕНИЙ НЕКОТОРЫХ СЛУЧАЙНЫХ СУММ ПРИ~ОСЛАБЛЕННЫХ МОМЕНТНЫХ
УСЛОВИЯХ$^*$}

\def\titkol{О~неравномерных оценках точности нормальной аппроксимации для
распределений некоторых случайных сумм} % при ослабленных моментных условиях}

\def\aut{В.\,Ю.~Королев$^1$, А.\,В.~Дорофеева$^2$}

\def\autkol{В.\,Ю.~Королев, А.\,В.~Дорофеева}

\titel{\tit}{\aut}{\autkol}{\titkol}

\index{Королев В.\,Ю.}
\index{Дорофеева А.\,В.}
\index{Korolev V.\,Yu.}
\index{Dorofeeva A.\,V.}




{\renewcommand{\thefootnote}{\fnsymbol{footnote}} \footnotetext[1]
{Работа выполнена при поддержке РФФИ (проект 18-07-01405).}}


\renewcommand{\thefootnote}{\arabic{footnote}}
\footnotetext[1]{Факультет вычислительной математики 
и~кибернетики, Московский государственный университет им.\
М.\,В.~Ломоносова; Институт проб\-лем информатики Федерального
исследовательского центра <<Информатика и~управ\-ле\-ние>> Российской
академии наук; Hangzhou Dianzi University, China, \mbox{vkorolev@cs.msu.ru}}
\footnotetext[2]{Факультет вычислительной
математики и~кибернетики, Московский государственный университет
им.\ М.\,В.~Ломоносова, \mbox{alex.dorofeyeva@gmail.com}}

%\vspace*{-6pt}



\Abst{Представлены неравномерные оценки скорости
сходимости в~центральной предельной теореме для сумм случайного
числа независимых одинаково распределенных случайных величин для
случаев, когда индекс суммирования (число слагаемых в~сумме) имеет
биномиальное или пуассоновское распределение и~стохастически
независим от слагаемых. Рассматривается ситуация, 
в~которой доступна информация лишь о существовании моментов второго
порядка у~слагаемых. Указаны конкретные числовые значения абсолютных
констант, входящих в~оценки. Попутно анонсируется уточнение
абсолютной константы в~неравномерной оценке скорости сходимости 
в~центральной предельной теореме для сумм неслучайного числа
независимых одинаково распределенных случайных величин с~моментами
порядков не выше второго.}

\KW{центральная предельная теорема; нормальная
аппроксимация; случайная сумма; биномиальное распределение;
распределение Пуассона; теорема Пуассона}

\DOI{10.14357/19922264180412}
  
\vspace*{6pt}


\vskip 10pt plus 9pt minus 6pt

\thispagestyle{headings}

\begin{multicols}{2}

\label{st\stat}

\section{Введение}

Оценки точности нормальной аппроксимации для
распределений сумм случайных величин традиционно являются объектом
пристального внимания среди специалистов в~области тео\-рии
вероятностей, поскольку они играют важную роль во многих прикладных
задачах. Такие оценки помогают осознанно принимать решения об
адекватности или неадекватности нормальной модели для наблюдаемых
статистических закономерностей. При этом особый интерес представляет
ситуация, в~которой доступна лишь минимальная информация 
о~существовании моментов второго порядка у~слагаемых. Именно такой
случай и~рассматривается в~данной заметке.

Пусть $X_1,X_2,\ldots$~--- независимые случайные величины с~${\sf E}
X_i\hm=0$ и~$0\hm<{\sf E} X_i^2\hm\equiv\sigma_i^2\hm<\infty$, $i\hm=1,2,\ldots $
Существование моментов случайных\linebreak величин $X_1,X_2,\ldots$ порядков
выше второго не предполагается. Для $n\hm\in\mathbb{N}$ обозначим
$S_n\hm=X_1+\cdots +X_n$ и~$B_n^2\hm=\sigma_1^2+\cdots +\sigma_n^2$.
Стандартную нормальную функцию распределения обозначим~$\Phi(x)$:

\noindent
$$
\Phi(x)=\fr{1}{\sqrt{2\pi}}\int\limits_{-\infty}^{x}e^{-z^2/2}\,dz\,,\enskip
x\in\mathbb{R}\,.
$$
Обозначим
\begin{align*}
\Delta_n(x)&=\left\vert{\sf P}\left(S_n<xB_n\right)-\Phi(x)\right\vert\,;\\
\Delta_n&=\sup\limits_x\left\vert{\sf P}\left(S_n<xB_n\right)-\Phi(x)\right\vert\,.
\end{align*}
Всюду далее символ~$\mathbb{I}(A)$ будет обозначать индикаторную
функцию события~$A$. Для $\varepsilon\hm\in(0,\infty)$ обозначим:
\begin{align*}
L_n(\varepsilon)&=\fr{1}{B_n^2}\sum\limits_{i=1}^n{\sf E}
X_i^2\mathbb{I}\left(\left\vert X_i\right\vert \ge \varepsilon B_n\right)\,;\
L_n=L_n(1)\,;
\\
M_n(\varepsilon)&=\fr{1}{B_n^3}\sum\limits_{i=1}^n{\sf E}
\left\vert X_i\right\vert^3\mathbb{I}\left(\left\vert X_i\right\vert < 
\varepsilon B_n\right)\,; \\
&\hspace*{47mm} M_n=M_n(1)\,.
\end{align*}
Оценкам величины~$\Delta_n$ при указанных выше минимально возможных
моментных условиях посвящены работы~[1--13] (см.\ также книги~\cite{Petrov1972, 
Petrov1987}. В~частности, для любого
$\varepsilon\hm\in(0,\infty)$ справедлива оценка:
\begin{equation}
\Delta_n\le 1,86\left(L_n(\varepsilon)+M_n(\varepsilon)\right)\le
1{,}86\left(L_n(\varepsilon)+\varepsilon\right)\,.
\label{e1-kd}
\end{equation}
Детальная история уточнения верхних оценок величины~$\Delta_n$,
изобилующая интересными результатами и~курьезами, описана 
в~работах~\cite{KorolevDorofeevaLMJ, Shevtsova}, в~которых
подчеркнуто, что оценки типа~(\ref{e1-kd}) разумно считать \textit{естественными}, 
поскольку они связывают скорость сходимости 
в~центральной предельной тео\-ре\-ме с~критерием схо\-ди\-мости.

В данной работе сосредоточимся на верхних оценках величины~$\Delta_n(x)$. 
В~1979~г.\ В.\,В.~Петров~\cite{Petrov1979} показал,
что существует конечная положительная постоянная~$C$, гарантирующая
выполнение неравенства:
\begin{multline}
\Delta_n(x)\le C\sum\limits_{k=1}^n \left[
\fr{{\sf E} X_k^2\mathbf{1}\left(|X_k|\ge
(1+|x|)B_n\right)}{B_n^2(1+|x|)^2}+{}\right.\\
\left.{}+\fr{{\sf E}
|X_k|^3\mathbf{1}\left(|X_k|<(1+|x|)B_n\right)}{B_n^3(1+|x|)^3}\right]\,.
\label{e2-kd}
\end{multline}
В 2001 г.\ неравенство~(\ref{e2-kd}) было передоказано другим методом 
в~статье~\cite{ChenShao2001}. В~неcкольких работах предпринимались
попытки оценить значение константы~$C$ в~неравенстве~(\ref{e2-kd}). 
В~част\-ности, в~работах~\cite{TN2007, NT2007} была получена
оценка $C\hm\le 76{,}17$. Эта оценка была существенно уточнена в~работе~\cite{PopovDisser}, 
где было показано, что в~случае одинаково
распределенных слагаемых, рас\-смат\-ри\-ва\-емом в~на\-сто\-ящей статье,
константа не превосходит~39,25. Следующее утверждение содержит
уточненную оценку абсолютной константы.

\smallskip

\noindent
\textbf{Теорема~1.}\ \textit{Пусть $X_1,X_2,\ldots$~---
независимые одинаково распределенные случайные величины с~${\sf E} X_1\hm=0$
и~$0\hm<{\sf E} X_1^2\hm\equiv\sigma^2\hm<\infty$. Тогда для любого
$x\hm\in\mathbb{R}$ справедливо неравенство}:
\begin{multline*}
\Delta_n(x)\le 36{,}62\left[\fr{{\sf E}
X_1^2\mathbf{1}\left(|X_1|\ge(1+|x|)\sigma\sqrt{n}\right)}{\sigma^2(1+|x|)^2}+{}\right.\\
\left.{}+
\fr{{\sf E}
|X_1|^3\mathbf{1}\left(|X_1|<(1+|x|)\sigma\sqrt{n}\right)}
{\sigma^3\sqrt{n}(1+|x|)^3}\right]\,.
%\label{e3-kd}
\end{multline*}

%\smallskip

\noindent
Д\,о\,к\,а\,з\,а\,т\,е\,л\,ь\,с\,т\,в\,о\,.\ \  
Для уточнения абсолютной константы здесь
частично использовались методы,\linebreak
 описанные в~\cite{PopovDisser}, 
с~учетом текущих наилучших оценок констант в~неравенстве
Бер\-ри--Эс\-се\-ена~\cite{ShDAN}, его неравномерном аналоге~\cite{NSh} 
и~неравенстве~(\ref{e1-kd})~\cite{KorolevDorofeevaLMJ}. Подробное описание
алгоритма будет представлено в~одной из следующих статей.

\smallskip

Цель настоящей работы~--- распространить утверждение теоремы 1 на
случайные суммы, в~которых число слагаемых имеет биномиальное или
пуассоновское распределение. При этом будет существенно
использоваться подход, развитый в~работе~\cite{KorolevDorofeevaLMJ}.

\section{Неравномерные оценки для~биномиальных случайных~сумм}

Всюду далее рассматриваются независимые одинаково распределенные
случайные величины $X_1,X_2,\ldots$ с~${\sf E} X_i\hm=0$ и~$0\hm<{\sf E}
X_i^2\hm\equiv \sigma^2<\infty$. Пусть $p\hm\in(0,1]$~--- произвольно.
Пусть $\xi_1,\ldots,\xi_n$~--- независимые случайные величины, такие
что
$$
\xi_j=
\begin{cases}1 & \mbox{ с~вероятностью } p\,,\\
                    0 & \mbox{ с~вероятностью } 1-p\,,
      \end{cases}\enskip
       j=1,\ldots,n\,.
$$
Случайная величина $N_{n,p}\hm=\xi_1+\cdots+\xi_n$ может
интерпретироваться как число успехов в~схеме испытаний Бернулли 
с~вероятностью успеха~$p$. Cлучайная величина~$N_{n,p}$ имеет
биномиальное распределение с~параметрами~$n$ и~$p$:
$$
{\sf P}(N_{n,p}=k)=C_n^kp^k(1-p)^{n-k}\,,\enskip k=0,\ldots,n\,.
$$
Предположим, что при каждом $n\hm\in\mathbb{N}$ случайные величины
$N_{n,p},X_1,X_2,\ldots$ взаимно независимы. В~данном разделе
основным объектом изучения будут \textit{биномиальные случайные суммы}
вида
$$
S_{N_{n,p}}=X_1+\cdots+X_{N_{n,p}}.
$$
При этом если $N_{n,p}\hm=0$, то $S_{N_{n,p}}\hm=0$.

Для $j\in\mathbb{N}$ введем случайные величины~$\widetilde{X}_j$, полагая
$$
\widetilde{X}_j=
\begin{cases}X_j & \mbox{ с~вероятностью } p\,,\\
0 & \mbox{ с~вероятностью } 1-p\,.
\end{cases}
$$
Несложно видеть, что $\widetilde{X}_j \eqd \xi_jX_j$, где сомножители в~правой
части независимы (здесь и~далее символ~$\eqd$ обозначает совпадение
распределений).

Пусть $F(x)$~--- общая функция распределения случайных величин~$X_j$,
$E_0(x)$~--- функция распределения с~единственным единичным скачком 
в~нуле. Тогда, очевидно,
$$
{\sf P}\left(\widetilde{X}_j<x\right)=pF(x)+(1-p)E_0(x)\,,\enskip
x\in\mathbb{R}\,,\ j\in\mathbb{N}\,.
$$
При этом ${\sf E}\widetilde{X}_j\hm=0$,
\begin{equation}
{\sf D}\widetilde{X}_j={\sf E}\widetilde{X}_j^2=p\sigma^2\,.
\label{e4-kd}
\end{equation}


\smallskip

\noindent
\textbf{Лемма~1.}\ \textit{Для любых $n\hm\in\mathbb{N}$ и~$p_j\hm\in(0,1]$}
\begin{equation}
S_{N_{n,p}}\eqd \widetilde{X}_1+\cdots+\widetilde{X}_n\,,
\label{e5-kd}
\end{equation}
\textit{где случайные величины в~правой части}~(\ref{e5-kd}) \textit{не\-за\-ви\-симы.}

\smallskip

\noindent
Д\,о\,к\,а\,з\,а\,т\,е\,л\,ь\,с\,т\,в\,о\,.\ \
Доказательство представляет собой простое
упражнение на свойства характеристических функций.

\smallskip

С~учетом~(\ref{e4-kd}) и~(\ref{e5-kd}) легко заметить, что
$$
{\sf D}S_{N_{n,p}}=np\sigma^2\,.
$$
Обозначим
$$
\Delta_{n,p}(x)=\left \vert{\sf P}
\left(S_{N_{n,p}}<x\sigma\sqrt{np}\right)-\Phi(x)\right\vert\,.
$$

\smallskip

\noindent
\textbf{Теорема~2.}\ \textit{Для любых $n\hm\in\mathbb{N}$ и~$p\hm\in(0,1]$ справедливо
неравенство}:
\begin{multline*}
\Delta_{n,p}(x)\le 36{,}62\left[\fr{{\sf E}
X_1^2\mathbf{1}\left(|X_1|\ge(1+|x|)\sigma\sqrt{np}\right)}{\sigma^2(1+|x|)^2}+{}\right.
\hspace*{-1.22522pt}\\
\left.{}+
\fr{{\sf E}
|X_1|^3\mathbf{1}\left(|X_1|<(1+|x|)\sigma\sqrt{np}\right)}{\sigma^3\sqrt{np}(1+|x|)^3}
\right]\le{}
\\
{}\le \fr{36{,}62}{\sigma^2(1+|x|)^2}\cdot {\sf
E}X_1^2\min\left\{ 1,\,\fr{|X_1|}{\sigma\sqrt{np}(1+|x|)}\right\}.
\end{multline*}

\smallskip

\noindent
Д\,о\,к\,а\,з\,а\,т\,е\,л\,ь\,с\,т\,в\,о\,.\ \ Из леммы~1 и~соотношения~(\ref{e4-kd}) 
вытекает, что
$$
\Delta_{n,p}=\left\vert{\sf
P}\left(\widetilde{X}_1+\cdots+\widetilde{X}_n<x\sigma\sqrt{np}\right)-\Phi(x)\right\vert\,.
$$
Правую часть этого соотношения оценим с~по\-мощью теоремы~1 и~получим:
\begin{multline*}
\left\vert {\sf
P}\left(\widetilde{X}_1+\cdots+\widetilde{X}_n<
x\sigma\sqrt{\theta_n}\right)-\Phi(x)\right\vert \le{}
\\[1pt]
\le 36{,}62\left[\fr{{\sf
E}\widetilde{X}_1^2\mathbb{I}\left(|\widetilde{X}_1|\ge(1+|x|)\sigma\sqrt{np}\right)}
{p\sigma^2(1+|x|)^2}+{}\right.\\[1pt]
\left.{}+
\fr{{\sf E}|\widetilde{X}_1|^3\mathbb{I}\left(|\widetilde{X}_1|<(1+|x|)
\sigma\sqrt{np}\right)}{\sigma^3p^{3/2}\sqrt{n}(1+|x|)^3}\right]={}
\\[1pt]
{}\le 36{,}62\left[\fr{{\sf E}X_1^2\mathbb{I}\left(|X_1|
\ge(1+|x|)\sigma\sqrt{np}\right)}{\sigma^2(1+|x|)^2}+{}\right.\\[1pt]
\left.{}+
\fr{{\sf E}|X_1|^3\mathbb{I}\left(|X_1|<(1+|x|)\sigma\sqrt{np}\right)}
{\sigma^3\sqrt{np}(1+|x|)^3}\right]=
\\[1pt]
{}= \fr{36{,}62}{\sigma^2(1+|x|)^2}\,{\sf E}X_1^2\min
\left\{1,\,\fr{|X_1|}{\sigma\sqrt{np}(1+|x|)}\right\},
\end{multline*}
что и~требовалось доказать.

\section{Неравномерные оценки для~пуассоновских случайных~сумм}

Для $\lambda\hm>0$ пусть $N_{\lambda}$~--- случайная величина, имеющая
распределение Пуассона с~параметром~$\lambda$:
$$
{\sf P}\left(N_{\lambda}=k\right)=e^{-\lambda}\fr{\lambda^k}{k!}\,,\enskip
k\in\mathbb{N}\cup \{0\}\,.
$$
Предположим, что при каждом $\lambda\hm>0$ случайные величины
$N_{\lambda},X_1,X_2,\ldots$ независимы. Рассмотрим \textit{пуассоновскую случайную сумму}
$$
S_{N_{\lambda}}=X_1+\cdots+X_{N_{\lambda}}\,.
$$
Если $N_{\lambda}\hm=0$, то полагаем $S_{N_{\lambda}}\hm=0$. Несложно
убедиться, что ${\sf E}S_{N_{\lambda}}\hm=0$ и~${\sf D}S_{N_{\lambda}}\hm=
\lambda\sigma^2$. Точность нормальной
аппроксимации для рас\-пре\-де\-лений пуассоновских случайных сумм
изучалась многими\linebreak автора\-ми (см.\ исторические обзоры 
в~рабо\-тах~\cite{KorolevShevtsova, ShevtsovaPoisson}). Равномерные оценки
точности нормальной аппроксимации для распределений пуассоновских
случайных сумм при ослабленных моментных условиях получены 
в~статье~\cite{KorolevDorofeevaLMJ}. Насколько известно авторам,
неравномерные оценки для такой ситуации еще не проводились.

В данном разделе будут построены верхние оценки величины
$$
\Delta_{\lambda}(x)=\left\vert {\sf P}\left(S_{N_{\lambda}}<x\sigma
\sqrt{\lambda}\right)-\Phi(x)\right\vert\,.
$$
Зафиксируем $\lambda$ и~наряду с~$N_{\lambda}$ рассмотрим случайную
величину~$N_{n,p}$, имеющую биномиальное распределение с~\textit{произвольной} 
парой параметров~$n$ и~$p\hm\in(0,1]$, удовлетворяющей
условию $np=\lambda$. При этом
$$
{\sf D}S_{N_{\lambda}}={\sf D}S_{N_{n,p}}=\sigma^2\lambda=\sigma^2np\,.
$$
Поэтому для любого $x\hm\in\mathbb{R}$ по неравенству треугольника
\begin{multline}
\Delta_{\lambda}(x)\le{}\\
{}\le\Delta_{n,p}(x)+
\left\vert {\sf P}(S_{N_{\lambda}}<x)-{\sf P}(S_{N_{n,p}}<x)\right\vert\,.
\label{e6-kd}
\end{multline}
Первое слагаемое в~правой части~(\ref{e6-kd}) оценим с~помощью теоремы~2 и~получим:
\begin{multline}
\Delta_{n,p}(x)\le{}\\
{}\le \fr{36{,}62}{\sigma^2(1+|x|)^2}\, {\sf E}X_1^2\min
\left\{1,\,\fr{|X_1|}{\sigma\sqrt{\lambda}(1+|x|)}\right\}\,.
\label{e7-kd}
\end{multline}
Рассмотрим второе слагаемое в~правой части~(\ref{e6-kd}). Имеем:
\begin{multline}
\left\vert {\sf P}\left(S_{N_{\lambda}}<x\right)-{\sf P}
\left(S_{N_{n,p}}<x\right)\right\vert \le{}\\
{}\le\sup\limits_x
\sum\limits_{k=0}^{\infty}{\sf P}
\left(\sum\limits_{j=1}^kX_j<x\right)\left\vert 
{\sf P}(N_{n,p}=k)-{}\right.\\
\hspace*{-4.5mm}\left.{}-{\sf P}\left(N_{\lambda}=k\right)\right\vert \le
\sum\limits_{k=0}^{\infty}\left\vert {\sf P}(N_{n,p}=k)-{\sf P}
\left(N_{\lambda}=k\right)\right\vert\,.\!\!\!\!
\label{e8-kd}
\end{multline}
Правую часть неравенства~(\ref{e8-kd}) оценим с~помощью неравенства
Бар\-бу\-ра--Хол\-ла~\cite{BarbourHallPoisson}, в~соответствии с~которым
\begin{equation}
\sum\limits_{k=0}^{\infty}\left\vert {\sf P}(N_{n,p}=k)-{\sf P}
\left(N_{\lambda}=k\right)\right\vert \le 2p\min\{1,\lambda\}\,.
\label{e9-kd}
\end{equation}
Таким образом, из~(\ref{e6-kd}), (\ref{e7-kd}) и~(\ref{e9-kd}) 
вытекает, что \textit{для любых}~$n$ и~$p$, 
удовлетворяющих условию $np\hm=\lambda$, и~для любого
$x\hm\in\mathbb{R}$ справедливо неравенство:
\begin{multline}
\hspace*{-1.5mm}\Delta_{\lambda}(x)\le \fr{36{,}62}{\sigma^2(1+|x|)^2}\, {\sf E}
X_1^2\min\left\{\!1,\,\fr{|X_1|}{\sigma\sqrt{\lambda}(1+|x|)}\!\right\} +{}\hspace*{-0.44373pt}\\
{}+
\fr{2}{n}\,\lambda\min\{1,\lambda\}\,.
\label{e10-kd}
\end{multline}
Теперь, устремляя в~(\ref{e10-kd}) $n\hm\to\infty$, получаем окончательный
результат.

\smallskip

\noindent
\textbf{Теорема~3.}\ \textit{Для любых $\lambda\hm>0$ и~$x\hm\in\mathbb{R}$
справедлива оценка}:
$$
\Delta_{\lambda}(x)\le \fr{36{,}62}{\sigma^2(1+|x|)^2}\, {\sf E}
X_1^2\min\left\{1,\,\fr{|X_1|}{\sigma\sqrt{\lambda}(1+|x|)}\right\}\,.
$$

{\small\frenchspacing
 {%\baselineskip=10.8pt
 \addcontentsline{toc}{section}{References}
 \begin{thebibliography}{99}

\bibitem{Katz1963}
\Au{Katz~M.} Note on the Berry--Esseen theorem~// Ann. Math.
Stat., 1963. Vol.~39. No.\,4. P.~1348--1349.

\bibitem{Petrov1965}
\Au{Петров~В.\,В.} Одна оценка отклонения распределения суммы
независимых случайных величин от нормального закона~// Докл. АН
СССР, 1965. Т.~160. Вып.~5. С.~1013--1015.

\bibitem{Osipov1966}
\Au{Осипов~Л.\,В.} Уточнение теоремы Линдеберга~// Теория
вероятностей и~ее применения, 1966. Т.~11. Вып.~2. С.~339--342.

\bibitem{Feller1968}
\Au{Feller~W.} On the Berry--Esseen theorem~// Z.~Wahrscheinlichkeit., 1968. Bd.~10. S.~261--268.

\bibitem{Paditz1980}
\Au{Paditz~L.} Bemerkungen zu einer Fehlerabsch$\ddot{\mbox{a}}$tzung im
zentralen Grenzwertsatz~// Wiss. Z.~Hochsch. Verkehrswesen
Friedrich List Dres., 1980. Vol.~27. No.\,4. P.~829--837.

\bibitem{Paditz1984} %6
\Au{Paditz~L.} On error-estimates in the central limit theorem for
generalized linear discounting~// Math. Operationsforsch. Stat. 
Ser. Stat., 1984. Vol.~15. No.~4. P.~601--610.

\bibitem{BarbourHall1984} %7
\Au{Barbour~A.\,D., Hall~P.} Stein's method and the Berry--Esseen
theorem~// Aust. J.~Stat., 1984. Vol.~26.
P.~8--15.

\bibitem{Paditz1986} %8
{\it Paditz~L.} $\ddot{\mbox{U}}$ber eine Fehlerabsch$\ddot{\mbox{a}}$tzung im zentralen
Grenzwertsatz~// Wiss. Z. Hochsch. Verkehrswesen
Friedrich List Dres., 1986. Vol.~33. No.\,2. P.~399--404.



\bibitem{ChenShao2001} %9
\Au{Chen~L.\,H.\,Y., Shao~Q.\,M.} A~non-uniform Berry--Esseen bound
via Stein's method~// Probab. Theory Rel., 2001.
Vol.~120. P.~236--254.

\bibitem{KP2011_3}
\Au{Королев~В.\,Ю., Попов~С.\,В.} Уточнение оценок ско\-рости
сходимости в~центральной предельной теореме при отсутствии моментов
порядков, больших второго~// Тео\-рия вероятностей и~ее применения,
2011. Т.~56. Вып.~4. С.~797--805.

\bibitem{KorolevPopovDAN}
\Au{Королев~В.\,Ю., Попов~С.\,В.} Уточнение оценок ско\-рости
сходимости в~центральной предельной тео\-ре\-ме при ослабленных
моментных условиях~// Докл. РАН, 2012. Т.~445. Вып.~3. С.~265--270.

\bibitem{PopovDisser}
\Au{Попов~С.\,В.} Оценки скорости сходимости в~центральной
предельной теореме при ослабленных моментных условиях: Дис.\ \ldots\ канд.
физ.-мат. наук.~--- М.: МГУ, 2012.

\bibitem{KorolevDorofeevaLMJ}
\Au{Korolev~V., Dorofeeva~A.} Bounds of the accuracy of the normal
approximation to the distributions of random sums under relaxed
moment conditions~//  Lith. Math.~J., 2017. Vol.~57. No.\,1. P.~38--58.

\bibitem{Petrov1972}
\Au{Петров.~В.\,В.} Суммы независимых случайных величин.~--- М.:
Наука, 1972. 416~с.

\bibitem{Petrov1987}
\Au{Петров~В.\,В.} Предельные теоремы для сумм независимых
случайных величин.~--- М.: Наука, 1987. 320~с.

\bibitem{Shevtsova}
\Au{Шевцова~И.\,Г.} Моментное неравенство с~применением к~оценкам
скорости сходимости в~глобальной ЦПТ для пуас\-сон-би\-но\-ми\-аль\-ных
случайных сумм~// Тео\-рия вероятностей и~ее применения, 2017. Т.~62.
Вып.~2. С.~345--364.

\bibitem{Petrov1979}
\Au{Петров~В.\,В.} Одна предельная теорема для сумм независимых
неодинаково распределенных случайных величин~// Записки научных
семинаров ЛОМИ, 1979. Т.~85. С.~188--192.

\bibitem{TN2007}
\Au{Thongtha~P., Neammanee~K.} Refinement of the constants in the
non-uniform version of the Berry--Esseen theorem~// Thai J.~Math., 2007. Vol.~5. P.~1--13.

\bibitem{NT2007} %19
\Au{Neammanee~K., Thongtha~P.} Improvement of the non-uniform
version of the Berry--Esseen inequality via Paditz--Shiganov
theorems~// J.~Inequalities Pure  \mbox{Appl.} Math.,
2007. Vol.~8. No.\,4. Art.~92.

\bibitem{ShDAN}
\Au{Шевцова И.\,Г.} Об абсолютных константах в~неравенствах
Бер\-ри--Эс\-се\-ена~//
Докл.\ РАН, 2014. Т.~456. №\,6. С.~650--654.

\bibitem{NSh} %21
\Au{Нефедова~Ю.\,С., Шевцова~И.\,Г.} О~неравномерных
оценках ско\-рости схо\-ди\-мости в~цент\-раль\-ной предельной
%\linebreak\vspace*{-12pt}
%\pagebreak
%\noindent
теореме~// Тео\-рия вероятностей и~ее применения, 2012. Т.~57. №\,1. С.~62--97.

\pagebreak

\bibitem{KorolevShevtsova}
\Au{Korolev~V.\,Yu., Shevtsova~I.\,G.} An improvement of the
Berry--Esseen inequality with applications to Poisson and mixed
Poisson random sums~// Scand. Actuar.~J., 2012. No.\,2.
P.~81--105.

\bibitem{ShevtsovaPoisson}
\Au{Шевцова И.\,Г.} О~точ\-ности нормальной аппроксимации для
обобщенных пуассоновских распределений~// Теория вероятностей и~ее
применения, 2013.
Т.~58. №\,1. С.~152--176.

\bibitem{BarbourHallPoisson}
\Au{Barbour~A.\,D., Hall~P.} On the rate of Poisson convergence~// 
Math. Proc. Cambridge,
1984. Vol.~95. P.~473--480.
 \end{thebibliography}

 }
 }

\end{multicols}

\vspace*{-3pt}

\hfill{\small\textit{Поступила в~редакцию 15.10.18}}

\vspace*{8pt}

%\pagebreak

%\newpage

%\vspace*{-28pt}

\hrule

\vspace*{2pt}

\hrule

%\vspace*{-2pt}

\def\tit{ON NONUNIFORM ESTIMATES OF~ACCURACY OF~NORMAL
APPROXIMATION FOR DISTRIBUTIONS OF~SOME RANDOM SUMS UNDER~RELAXED
MOMENT CONDITIONS}

\def\titkol{On nonuniform estimates of~accuracy of~normal
approximation for distributions of~some random sums} % under~relaxed moment conditions}

\def\aut{V.\,Yu.~Korolev$^{1,2,3}$ and~A.\,V.~Dorofeeva$^1$}

\def\autkol{V.\,Yu.~Korolev and~A.\,V.~Dorofeeva}

\titel{\tit}{\aut}{\autkol}{\titkol}

\vspace*{-11pt}


\noindent
$^1$Faculty of Computational Mathematics
and Cybernetics, M.\,V.~Lomonosov Moscow State University, 
1-52~Lenin-\linebreak
$\hphantom{^1}$skiye Gory, GSP-1, Moscow 119991, Russian Federation

\noindent
$^2$Institute
of Informatics Problems, Federal Research Center ``Computer Science
and Control'' of the Russian\linebreak
 $\hphantom{^1}$Academy of Sciences, 44-2~Vavilov Str.,
Moscow 119333, Russian Federation

\noindent
$^3$Hangzhou Dianzi University, Xiasha Higher Education Zone, Hangzhou 310018, 
China


\def\leftfootline{\small{\textbf{\thepage}
\hfill INFORMATIKA I EE PRIMENENIYA~--- INFORMATICS AND
APPLICATIONS\ \ \ 2018\ \ \ volume~12\ \ \ issue\ 4}
}%
 \def\rightfootline{\small{INFORMATIKA I EE PRIMENENIYA~---
INFORMATICS AND APPLICATIONS\ \ \ 2018\ \ \ volume~12\ \ \ issue\ 4
\hfill \textbf{\thepage}}}

\vspace*{6pt}



\Abste{Nonuniform estimates are
presented for the rate of convergence in the central limit theorem
for sums of a random number of independent identically distributed
random variables. Two cases are studied in which the summation index
(the number of summands in the sum) has the binomial or Poisson
distribution. The index is assumed to be independent of the
summands. The situation is considered where the information that
only the second moments of the summands exist is available. Particular numerical
values of the absolute constants are presented explicitly. Also, the
sharpening of the absolute constant in the nonuniform estimate of
the rate of convergence in the central limit theorem for sums of 
a~nonrandom number of independent identically distributed random
variables is announced for the case where the summands possess only
second moments.}

\smallskip

\KWE{central limit theorem; normal approximation; random
sum; binomial distribution; Poisson distribution; Poisson theorem}


\DOI{10.14357/19922264180412}

\vspace*{-14pt}

\Ack
\noindent
This work was financially supported by the Russian Foundation for 
Basic Research (grant No.\,118-07-01405).


%\vspace*{6pt}

  \begin{multicols}{2}

\renewcommand{\bibname}{\protect\rmfamily References}
%\renewcommand{\bibname}{\large\protect\rm References}

{\small\frenchspacing
 {%\baselineskip=10.8pt
 \addcontentsline{toc}{section}{References}
 \begin{thebibliography}{99}

\bibitem{1-kd}
\Aue{Katz,~M.} 1963. Note on the Berry--Esseen theorem. 
\textit{Ann. Math. Stat.} 39(4):1348--1349.

\bibitem{2-kd}
\Aue{Petrov,~V.\,V.} 1965. Odna otsenka otkloneniya raspredeleniya summy
nezavisimykh sluchaynykh velichin ot normal'nogo zakona
[One estimate of the deviation of distribution of the sum of independent
random variables from the normal law]. 
\textit{Sov. Math.} 160(5):1013--1015.

\bibitem{3-kd}
\Aue{Osipov,~L.\,V.} 1966. Refinement of Lindeberg's theorem. 
\textit{Theor. Probab. Appl.} 11(2):299--302.

\bibitem{4-kd}
\Aue{Feller,~W.} 1968. On the Berry--Esseen theorem. 
\textit{Z.~Wahrscheinlichkeit.} 10:261--268.

\bibitem{5-kd}
\Aue{Paditz,~L.} 1980. Bemerkungen zu einer Fehlerabsch$\ddot{\mbox{a}}$tzung 
im zentralen Grenzwertsatz. \textit{Wiss. Z.~Hochsch. 
 Verkehrswesen Friedrich List Dres.} 27(4):829--837.

\bibitem{6-kd}
\Aue{Paditz,~L.} 1984. On error-estimates in the central limit theorem for
generalized linear discounting. \textit{Math. Operationsforsch. 
Stat. Ser. Stat.} 15(4):601--610.



\bibitem{8-kd} %7
\Aue{Barbour,~A.\,D., and P.~Hall.} 1984. Stein's method and the Berry--Esseen
theorem. \textit{Aust. J.~Stat}. 26:8--15.

\bibitem{7-kd} %8
\Aue{Paditz,~L.} 1986. $\ddot{\mbox{U}}$ber eine Fehlerabsch{\"a}tzung im zentralen
Grenzwertsatz. \textit{Wiss. Z.~Hochsch. Verkehrswesen
Friedrich List Dres.} 33(2):399--404.

\bibitem{9-kd}
\Aue{Chen,~L.\,H.\,Y., and Q.\,M.~Shao.} 2001. A~non-uniform Berry--Esseen bound via
Stein's method. \textit{Probab. Theory Rel.} 120:236--254.

\bibitem{10-kd}
\Aue{Korolev,~V.\,Yu., and S.\,V.~Popov.} 
2012. An improvement of convergence rate estimates 
in the central limit theorem under absence of moments higher than the
second. 
\textit{Theor. Probab. Appl.} 56(4):682--691.

\bibitem{11-kd}
\Aue{Korolev,~V.\,Yu., and S.\,V.~Popov.} 2012. 
Improvement of convergence rate estimates in the 
central limit theorem under weakened moment conditions.
\textit{Dokl. Math.} 86(1):506--511.

\bibitem{12-kd}
\Aue{Popov,~S.\,V.} 2012. Utochneniye ocnok skorosti skhodimosti 
v~tsentral'noy predel'noy teoreme pri  oslablennykh momentnykh usloviyakh
[Improvement of convergence rate estimates in the 
central limit theorem under weakened moment conditions]. 
 Moscow: MSU. PhD Diss.

\bibitem{13-kd}
\Aue{Korolev,~V., and A.~Dorofeeva.} 2017. Bounds of the accuracy of the
normal approximation to the distributions of random sums under
relaxed moment conditions. \textit{Lith. Math.~J.~} 57(1):38--58.

\bibitem{14-kd}
\Aue{Petrov,~V.\,V.} 1972. 
\textit{Summy nezavisimykh sluchaynykh velichin} 
[Sums of independent random variables]. Moscow: Nauka. 416~p.

\bibitem{15-kd}
\Aue{Petrov,~V.\,V.} 1987. 
\textit{Predel'nye teoremy dlya summ ne\-za\-vi\-si\-mykh sluchaynykh velichin} 
[Limit theorems for sums of independent random variables]. Moscow: Nauka. 320~p.

\bibitem{16-kd}
\Aue{Schevtsova,~I.\,G.} 2018. 
A~moment inequality with application to convergence rate estimates 
in the global CLT for Poisson-binomial random sums. 
\textit{Theor. Probab. Appl.} 62(2):278--294.

\bibitem{17-kd}
\Aue{Petrov,~V.\,V.} 1979. Odna predel'naya teorema dlya summ nezavisimykh
neodinakovo raspredelennykh sluchaynykh velichin 
[One limit theorem for sums of independent
unequally distributed random variables]. \textit{J.~Mathematical Sciences} 85:188--192.

\bibitem{18-kd}
\Aue{Thongtha,~P., and K.~Neammanee.} 2007. Refinement of the constants in the
non-uniform version of the Berry--Esseen theorem. \textit{Thai J.~Math.} 5:1--13.

\bibitem{19-kd}
\Aue{Neammanee,~K., and P.~Thongtha.} 2007. Improvement of the non-uniform version
of the Berry--Esseen inequality via Paditz--Shiganov theorems.
\textit{J.~Inequalities  Pure  Appl. Math.} 8(4): 92.

\bibitem{20-kd}
\Aue{Shevtsova,~I.\,G.} 2014. On the absolute constants in the Berry--Esseen-type
inequalities. \textit{Dokl. Math.} 89(3):378--381.

\bibitem{21-kd}
\Aue{Nefedova,~Yu.\,S., and I.\,G.~Shevtsova.} 2013. On nonuniform convergence rate
estimates in the central limit theorem. \textit{Theor. Probab. Appl.} 57(1):28--59.

\bibitem{22-kd}
\Aue{Korolev,~V.\,Yu., and  I.\,G.~Shevtsova.} 2012. An improvement of the
Berry--Esseen inequality with applications to Poisson and mixed
Poisson random sums. \textit{Scand. Actuar.~J.} 2:81--105.

\bibitem{23-kd}
\Aue{Shevtsova,~I.\,G.} 2014. On the accuracy of the normal approximation to
compound Poisson distributions. \textit{Theor. Probab. Appl.} 58(1):138--158.

\bibitem{24-kd}
\Aue{Barbour,~A.\,D., and P.~Hall.} 1984. On the rate of Poisson convergence.
\textit{Math. Proc. Cambridge} 95:473--480.

\end{thebibliography}

 }
 }

\end{multicols}

\vspace*{-6pt}

\hfill{\small\textit{Received October 15, 2018}}

%\pagebreak

%\vspace*{-18pt}

\Contr

\noindent
\textbf{Korolev Victor Yu.} (b.\ 1954)~--- Doctor of Science 
in physics and mathematics, professor, Head of the Department of Mathematical 
Statistics, Faculty of Computational Mathematics and Cybernetics, M.\,V.~Lomonosov 
Moscow State University, 1-52~Leninskiye Gory, GSP-1, Moscow 119991, 
Russian Federation; leading scientist, Institute of Informatics Problems, 
Federal Research Center ``Computer Science and Control'' 
of the Russian Academy of Sciences, 44-2~Vavilov Str., Moscow 119333, 
Russian Federation; professor, Hangzhou Dianzi University, Xiasha Higher 
Education Zone, Hangzhou 310018, China; \mbox{vkorolev@cs.msu.su}

\vspace*{3pt}

\noindent
\textbf{Dorofeeva Alexandra V.} (b.\ 1991)~--- 
PhD student, Faculty of Computational Mathematics and Cybernetics, 
M.\,V.~Lomonosov Moscow State University, 1-52~Leninskiye Gory, GSP-1, Moscow 119991, 
Russian Federation; \mbox{alex.dorofeyeva@gmail.com}
\label{end\stat}

\renewcommand{\bibname}{\protect\rm Литература}       