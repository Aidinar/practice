%\renewcommand{\figurename}{\protect\bf Figure}
\renewcommand{\tablename}{\protect\bf Table}

\def\stat{razum}


\def\tit{COMPARISON OF TWO ACTIVE QUEUE MANAGEMENT SCHEMES THROUGH THE~$M/D/1/N$ 
QUEUE}

\def\titkol{Comparison of two active queue management schemes through the $M/D/1/N$ 
queue}

\def\autkol{M.\,G.~Konovalov and R.\,V.~Razumchik}

\def\aut{M.\,G.~Konovalov$^1$ and R.\,V.~Razumchik$^2$}

\titel{\tit}{\aut}{\autkol}{\titkol}

%{\renewcommand{\thefootnote}{\fnsymbol{footnote}}
%\footnotetext[1] {The 
%research of Yuri Kabanov was done under partial financial support   of the grant 
%of  RSF No.\,14-49-00079.}}

\renewcommand{\thefootnote}{\arabic{footnote}}
\footnotetext[1]{Institute of Informatics Problems, Federal Research Center ``Computer Science and Control'' of the Russian Academy of Sciences,
44/2~Vavilov Str., Moscow 119333, Russian Federation, \mbox{mkonovalov@ipiran.ru}}
\footnotetext[2]{Institute of Informatics Problems, Federal Research Center 
``Computer Science and Control'' of the Russian Academy of Sciences,
44/2~Vavilov Str., Moscow 119333, Russian Federation; 
Peoples' Friendship University of Russia (RUDN University),
6~Miklukho-Maklaya Str., Moscow 117198, Russian 
Federation; 
\mbox{rrazumchik@ipiran.ru} %\mbox{razumchik\_rv@rudn.ru
}


\index{Konovalov M.\,G.}
\index{Razumchik R.\,V.}
\index{Коновалов М.\,Г.}
\index{Разумчик Р.\,В.}

\def\leftfootline{\small{\textbf{\thepage}
\hfill INFORMATIKA I EE PRIMENENIYA~--- INFORMATICS AND
APPLICATIONS\ \ \ 2018\ \ \ volume~12\ \ \ issue\ 4}
}%
 \def\rightfootline{\small{INFORMATIKA I EE PRIMENENIYA~---
INFORMATICS AND APPLICATIONS\ \ \ 2018\ \ \ volume~12\ \ \ issue\ 4
\hfill \textbf{\thepage}}}



\Abste{The paper focuses on giving the first in the literature numerical evidence
that the stationary performance characteristics of single-server queues
with the general renovation mechanism may be as good as of single-server queues
with the RED-type active queue management mechanisms
(AQM). Comparison is made in the queueing
theory context: the basic model is the $M/D/1/N$ queue. 
The characteristics reported are: the loss ratio, average system size, and 
average number of consecutive losses along 
with the standard deviations. Numerical results are based on the 
well-known facts and some new analytic results, presented in the paper.}

\KWE{queueing system; active queue management; RED; renovation}

\DOI{10.14357/19922264180402}


\vspace*{1pt}


\vskip 12pt plus 9pt minus 6pt

      \thispagestyle{myheadings}

      \begin{multicols}{2}

                  \label{st\stat}


\section{Introduction}

\noindent
A large number of AQM mechanisms have been developed
up to nowadays and quite a~lot of efforts have been devoted to the studies of
their efficiency. 
These mechanisms may be applicable in different contexts but historically, 
they are more often related to communication networks
in the context of mitigation of congestion and congestion avoidance.
This problem, as highlighted in the latest RFC~7567~\cite{RFC7567},
still does not have a~satisfying solution. 
An AQM mechanism is an advanced rejection discipline, 
when an arriving customer (packet, job, etc.) is lost randomly with a~probability 
that may depend on the (current, past, average, etc.) system state or performance.
The most popular class of AQM mechanisms seems to be the Random Early Detection (RED) and
its ramifications like GRED (Gentle RED), REM 
(Random Exponential Marking),
etc.\ (a~recent survey on the AQM can be found in~\cite{Adams}).
The goals of AQM are usually diverse and conflicting: 
prevent queues from growing too long, maintain high server (processor) utilization
and low variance of the queue size, ensure fairness among competing flows, 
and others. These are discussed in detail in~\cite{RFC7567} in the context
of communications network but most of the goals are applicable in other contexts as well
(buffer-bloat problems in data-center, etc.).

Besides simulation, analytic performance evaluation of systems with AQM is quite often
carried out in the queueing theory context (see,
for example,~\cite{Bonald,Chyd,Chyd2,oleg,hao,konnew} and references therein). 
Usually, the system with an AQM mechanism is modeled as a~queueing system or network
and then its performance characteristics are studied using known analytic techniques. 
Throughout the paper, we stay within the queueing theory context.

In the series of recent papers~\cite{Kreinin,Zaryadov2010,zarN1,zarN2,Zaryadov2009},
the authors have proposed the new type of AQM mechanism which they call 
\textit{renovation}. 
Roughly speaking, renovation implies that each customer, 
having received service, may remove some additional work from the system
(i.\,e., may renovate it). We will make this definition more precise in the next 
section 
but for now, note 
that queue management \textit{after service completions} is what makes the renovation
 different 
from the most known AQM schemes\footnote[3]{Indeed, renovation and most of the known AQM 
mechanisms
are conceptually different. One of the main goals of AQM mechanisms is to prevent 
queue from growing too large
leaving space for potential new arrivals.
In systems with renovation, the queue can become full (meaning that fewer customers are 
lost)
but after a~service completion, several customers may be removed from it. In this way, 
the content of the queue
can be preserved at a~certain average level but the loss pattern becomes intricate.}, 
in which the decisions are made \textit{upon arrivals}.  
To our best knowledge, there are no studies, 
which tell whether the performance of the systems with renovation is
better/same/worse than that of the same systems but with the implemented AQM mechanisms.
Thus, there is a~lack of bridge between available theoretical results for renovation and 
its practical perspective.

The scope of this paper is to give the first in the literature numerical evidence 
that the stationary performance of 
single-server queueing systems with the implemented renovation mechanism can
be as good as of 
the same single-server queues but the well-known packed dropping procedures like RED.
The emphasis is primarily on the reporting of this finding, complemented 
with some new insights into 
queueing systems with renovation. The relation to other 
AQM mechanisms like CoDel~\cite{RFC8289} is not discussed here. 
Moreover, in the numerical experiments presented here, 
we did not use any benchmarks to generate the traffic profiles 
but used the theoretical distributions instead.

The main stationary performance characteristics reported are: the loss ratio, 
the average number in the system
(average system size), and the average number of consecutive losses along with 
their standard deviations.
After introducing the renovation mechanism and the analytic setting,
in which  renovation mechanism is compared with RED, we give the new analytic results 
for computing system size moments and the loss ratio under the renovation mechanism.
The results presented in the numerical section are based on the analytic results. 
Monte-Carlo simulation is used only for the average (and standard deviation) 
number of consecutive losses in the system with renovation. 


\section{Settings and the Model}

\noindent
We follow the queueing theoretic approach and as the basic model, we use $M/D/1/N$ queue,
i.\,e., queue of finite capacity~$N$ fed at rate~$\lambda$ by a~Poisson flow of customers,
which are served on a~first-come-first-served basis  by a~single server with constant
service time $d>0$.
We assume that the system is in the steady state.
When an arriving customer sees that the queue is full,
it is lost. If no other type of losses occur in the system,
we say that the Tail Drop mechanism is implemented in it.

If an arriving customer is lost with probability~$d_n$
where~$n$ is the total number of customers 
it sees in the system on arrival, then we say that an AQM mechanism 
is implemented in the system. Various dropping functions can be obtained
by specifying the values of~$d_n$
(see, for example, RED dropping function in~\cite[Example 1]{Chyd}).
Important notice should be made here. In practice, RED-type mechanisms 
may use moving averages of the queue size instead of its instantaneous value. 
Thus, the way~$d_n$ introduced above is a~simplified way of thinking.
Yet, this trade-off is important because it allows to keep the mathematical 
models of RED-type AQM tractable.
Luckily, as noticed in~\cite[Section II.C]{Bonald}, such approximation may not
lead to significant bias, when the weight of the moving average scheme is small
(which is claimed to be the case sometimes in practice).

The renovation mechanism, which is implemented in a~system with Tail Drop,
works as follows. Define $N+1$ numbers, say, $q_i\ge 0$,
$0 \le i \le N$, satisfying \mbox{$\sum\nolimits_{i=0}^N q_i=1$}.
If upon service completion there are $i$, $1 \le i \le N$, customers
waiting in the queue, then the served customer leaves the system and
\begin{itemize}
\item with probability $q_0+Q_i$ nothing else happens, where 
$Q_i=q_i+q_{i+1}+\dots+ q_N$; and
\item with probability $q_j$, $0<j<i$, exactly $j$ customers
from the queue leave the system and those customers 
are chosen successively \textit{starting from the head of the queue}.
\end{itemize}
%\noindent
The served customer, which sees the empty queue, leaves the system.
Thus, after the renovation (if it happened), the system never becomes empty.
%what is appealing from the practical point of view. 

In the numerical section, we rank the systems with RED 
and renovation according to the stationary loss ratio, average system size,  
and average number of consecutive losses along with their standard deviations. 
The system with the Tail Drop is the standard $M/D/1/N$ queue, 
for which all these performance characteristics follow
from the classical results in queueing theory (see, for example,~\cite{Riordan1962}).
Analytic results for the systems of $M/G/1/N$ type with relatively 
arbitrary dropping functions are given in~\cite{Chyd}.
Yet, for the system with renovation, we need to derive these 
performance characteristics anew, since 
the available results in~\cite{Zaryadov2010,Zaryadov2009} 
are not valid for the renovation mechanism introduced above.
We briefly sketch the derivations in the next
section and omit most of the details
since they are based on the methodology, 
developed in~\cite{Zaryadov2010,Zaryadov2009},
and reviewed in~\cite{arxivRK}.

%Note that the above mentioned performance characteristics 
%do not depend on the order in which the customers
%are removed from the system; yet in the derivations we assume that the customers
%are chosen successively \textit{starting from the head of the queue}.

%The analytic results and parameters' values for 
%RED and REM are due to \cite{Chyd}.

\section{Performance Characteristics}

\noindent
Consider the $M/D/1/N$ queue with the renovation mechanism
introduced above. Since a~customer is served for constant service time
$d$, then for the cumulative distribution function $B(x)$ 
of its service time, one has: 
$$
B(x)=
\begin{cases}
0 & \mbox{if } x \le d\,;\\
1 & \mbox{if } x>d\,.
\end{cases}
$$
Let $N(t)$ be the total number of customers %\footnote{We assume that the system 
%starts empty, i.\,e., $N(0)=0$.} 
in the system at instant $t$ 
and $E(t)$ be the elapsed service time of the customer in server
(if there is one). 
In order to compute the stationary system size moments, 
one needs to know the stationary distribution:
\begin{equation*}
%\label{pn}
P_n=\lim\limits_{t \rightarrow \infty} \mathbf{P}\{ N(t)=n \},\enskip  0 \le n \le N+1\,.
\end{equation*} 
For the computation of the loss ratio,
due to the \mbox{PASTA} (Poisson Arrivals See Time Averages) 
property, it is sufficient to know
 the stationary probability densities
$p_n(x)=P'_n(x)$ where
\begin{multline*}
%\label{pnx}
P_n(x)=\lim\limits_{t \rightarrow \infty} \mathbf{P}\{ N(t)=n, E(t)<x \}, \\ 
1\le n \le N, \ x \in [0,d]\,.
\end{multline*}
Since we are dealing with the finite-capacity queue 
and work conserving service discipline, the
introduced stationary distributions exist.  
The probabilities~$P_n$ and 
the densities~$p_n(x)$ can be computed as follows. 
Let~$t_n$ denotes the $n$th service completion epoch 
and $N_n=N(t_n+0)$ denotes the total number of customers in the system. 
Clearly, $\{ N_n, \ n \ge 1\}$ is the finite-state Markov chain.
The entries of the transition probability matrix $\mathbb{P}=(p_{ij})$
of this chain have the form:
$$
p_{0j}=p_{1j}=
\begin{cases}
\beta_0, & \hspace*{-20mm}j=0;\\
\beta_j Q_j + \displaystyle\sum\limits_{k=j}^N \beta_k q_{k-j} +  B_N q_{N-j}, &\\
&\hspace*{-20mm} 1 \le j \le N-1\,;\\
(q_0 + q_N) B_{N-1}, & \hspace*{-20mm}j=N\,;
\end{cases}
$$
\begin{multline*}
\!\!p_{ij}=
\begin{cases}
0, & \hspace*{-38mm}j=0;\\
\sum\limits_{k=j-1}^{N-1} \beta_k q_{k-j+1} +  B_{N-1} q_{N-j}, & \\
& \hspace*{-38mm}1 \le j \le i-2\,;\\
\beta_{j-i+1} Q_j + \displaystyle\!\sum\limits_{k=j-1}^{N-1}\!\! \beta_k q_{k-j+1} + 
 B_{N-1} q_{N-j}, &\\
 &\hspace*{-38mm} i-1 \le j \le N-1;\\
(q_{0} +  q_{N})B_{N-i} , &\hspace*{-38mm} j=N\,,
\end{cases}
\\
 2 \le i \le N\,.
\end{multline*}
Here, $B_0=1-\beta_0$; $B_k=B_{k-1}-\beta_k$; and 
$\beta_k=[{(\lambda d)^k / k!}]e^{-\lambda d}$.
The matrix $\mathbb{P}$ does not have any special structure. 
So, the stationary distribution $\{P^+_n, \ 0 \le n\linebreak \le N\}$
may be found in a~straightforward manner by solving the system of linear algebraic 
equations 
$$
{\vec P}^+={\vec P}^+ \mathbb{P};\enskip 
{\vec P}^+ {\vec 1} =1
$$ 
where ${\vec P}^+= (P^+_0,\dots,P^+_N)$ and $\vec 1$ is the vector of ones. 
{\looseness=1

}

Once the probabilities $P^+_n$ are found,
the stationary distribution \mbox{$\{P_n, \ 0 \le n \le N+1\}$} 
is computed from the relation\footnote{This follows
from the well-known results for the Markov regenerative processes (see, for 
example,~\cite[Theorem 9.19]{kulk}).} 
$$
P_n=\sum\limits_{i=0}^N P^+_i \fr{f_{in}}{f^*}
$$
where $f_{in}$ is the average time during which there were $n$ customers in the system
provided that the system started with~$i$ customers in it; 
and~$f^*$ is the mean time between transitions of the embedded
Markov chain $\{ N_n, \ n \ge 1\}$.
{\looseness=1

}


Finally,the stationary probability densities $p_n(x)\linebreak =P'_n(x)$
can be computed using the fact that the relation for~$p_n(x)$ 
coincides with the relation for $p_n(x)$ in 
the standard $M/D/1/N$ queue.
Thus, $p_n(x)$ are given by (see, for example,~\cite[p.~72]{Riordan1962})

\noindent
\begin{multline}
\label{eq3}
p_n(x)=e^{-\lambda x} \left (1-B(x) \right ) 
\sum\limits_{k=0}^{n-1} p_{n-k}(0) 
\fr{(\lambda x)^k}{k!}\,, \\
1 \le n \le N\,,  \ x \in [0,d]\,.
\end{multline}
Even though~(\ref{eq3}) holds,
due to the presence of renovation, the boundary conditions $p_{n}(0)$ 
for the considered queue do not coincide 
with boundary conditions $p_{n}(0)$ for the standard $M/D/1/N$ queue.
By integration~(\ref{eq3}) from~0 to~$d$, one gets 
the following relation between~$p_{n}(0)$ and $P_n=\int\nolimits_0^d p_n(x) dx$: 
\begin{multline}
\label{eq3nn}
p_n(0)
= \fr{1}{B_0} \left (\lambda P_n- \sum\limits_{k=1}^{n-1} B_k p_{n-k}(0)
\right )\,, \\ 
1 \le n \le N\,.
\end{multline}
Since the stationary distribution \mbox{$\{P_n, \ 0 \le n \le N+1\}$} 
is already known, the values of $p_n(0)$ are computed recursively from~(\ref{eq3nn}).
The closed-form expressions for
 the average and the standard deviation of the system size are, in the most cases,
 not available and thus, they can be computed,
respectively, by $\sum\nolimits_{n=0}^{N+1} nP_n$ and  
$\sqrt{\sum\nolimits_{n=0}^{N+1} n^2P_n-(\sum_{n=0}^{N+1} nP_n)^2}$.

The computation of the loss ratio~$\pi$, i.\,e., the probability that the arriving customer is lost, 
is more involved. This is due to the fact that the accepted customer
may be lost either after the first service completion or the second, etc.
and the chance to be lost varies, depending on the number of
new customers that have arrived between successive service completions.
 
Let us introduce two quantities:
\begin{enumerate}[(1)]
\item $\gamma_{ij}$, $1 \le i \le N$, $j \ge 0$,~--- probability that the arriving customer
finds~$i$~customers in the system and until the next
service completion, exactly $j$ new customers arrive 
at the system; and
\item
$r_{ij}$, $0\le j \le N-1$, $0 \le i \le N-j-1$,~--- probability that the 
tagged customer waiting in the queue
\textit{will not} be served (i.\,e., will be lost), if there are~$j$~customers
 in front of it in the queue (excluding the one in server)
and~$i$ behind.
\end{enumerate}

Given that $\gamma_{ij}$ and $r_{ij}$ are known, the loss ratio~$\pi$  
can be computed as
\begin{multline*}
\pi =
P_{N+1}
 + 
\sum\limits_{i=1}^{N} \left [
\sum\limits_{j=0}^{N-i}
\gamma_{ij} \left ( \sum\limits_{k=0}^{i-2} q_k r_{j,i-2-k}
 +{}\right.\right.\\
\left. {}+ \sum\limits_{k=i}^{i+j-1} q_k  + Q_{j+i} r_{j,i-2} \right )
 + {}
\end{multline*}

\noindent
\begin{multline*}
{}+ \sum\limits_{j=N-i+1}^{\infty} \gamma_{ij} \left (
\sum\limits_{k=0}^{i-2} q_k r_{N-i,i-2-k}  + {}\right.\\
\left.\left.{}+\sum\limits_{k=i}^{N-1}
q_k  + Q_{N} r_{N-i,i-2} \right )
\right ]\,.
%\label{ploss2}
\end{multline*}

Due to the PASTA property of Poisson arrivals,
the expression for $\gamma_{ij}$ follows 
from the law of total probability:
\begin{multline*}
\gamma_{ij}
= \int\limits_0^d p_{i}(x) \fr{(\lambda (d- x))^j }{j!}\, e^{-\lambda (d-x)}\, dx\,,
\\
 1 \le i \le N\,, \ j \ge 0\,.
\end{multline*}
Again, by applying the law of total probability,
one gets the relations for the recursive computation of~$r_{ij}$, 
$0\le j \le N-1$, $0 \le i \le N-j-1$:
\begin{align*}
r_{i0}&= \sum\limits_{m=0}^{N-i-1}
\beta_m 
\sum\limits_{k=1}^{m+i} q_k +
B_{N-i-1} \sum\limits_{k=1}^{N-1} q_k\,;\\
r_{ij}&= \sum\limits_{m=i}^{N-1-j} \beta_{m-i} \left (
\vphantom{\sum\limits_{k=j+1}^{m+j} q_k + Q_{j+m+1} r_{m,j-1}}
\sum\limits_{k=0}^{j-1} q_k r_{m,j-1-k} +{}\right.\\
&\left.{}+
\sum\limits_{k=j+1}^{m+j} q_k + Q_{j+m+1} r_{m,j-1}
\right ) +{}
\\
&{}+ B_{N-j-i-1}
\left ( \vphantom{\sum\limits_{k=j+1}^{m+j} q_k + Q_{j+m+1} r_{m,j-1}}
\sum\limits_{k=0}^{j-1} q_k r_{N-j-1,j-1-k} +{}\right.\\
&\hspace*{16mm}\left.{}+\sum\limits_{k=j+1}^{N-1} q_k +Q_{N} r_{N-j-1,j-1}
\right ).
\end{align*}
The expressions above can be further simplified\footnote{There
are no principal difficulties in generalizing
the results for the $\mathrm{BMAP}/G/1/N$ queue.
Yet, this would obscure the goal of the paper and thus, 
we remain with the simple model.} by computing 
the integrals explicitly, but we do not dwell on it here.
For small and moderate values of~$d$, $N$, and~$\lambda$,
they can be directly used for numerical implementation.
In the numerical section, precisely these expressions are used to calculate
the loss ratio. The expressions for the average and the standard
deviation of consecutive losses are much harder to derive
and we leave it for a~separate study. The values of these 
two parameters were taken from the Monte-Carlo simulation. 

\section{Numerical Example}

\noindent
As the reference point, we have chosen the numerical results in~\cite{Chyd}
which are based on the analytic expressions and which show the 
performance characteristics 
of the $M/D/1/N$ queue with four different AQM mechanisms. 
Since RED scheme is one of the best among the four,
our goal here is to rank the $M/D/1/N$ queue with RED from~\cite{Chyd}
and the $M/D/1/N$ queue with renovation. Comparison is made 
according to the stationary loss ratio, average system size,  
and average number of consecutive losses along with 
their standard deviations.

The initial conditions are: 
the maximum queue size is $N=9$ and the service time is $d=1$. 
Thus, the offered load is $\rho=\lambda d$. 
The RED dropping function is given by (see~\cite[Eq. (59)]{Chyd}):
\begin{equation}
\label{df}
d_n=
\begin{cases}
0\,, & n\le 3\,;\\
0.11917n - 0.35752\,, & 4 \le n \le 9\,;\\
1\,, & n=10.
\end{cases}
\end{equation}
The performance of the $M/D/1/N$ queue with this RED dropping function 
is given in~\cite[Tables 1, 3, and~4]{Chyd}.
In order to find out whether there exists a~renovation 
mechanism under which the $M/D/1/N$ queue can perform at least as good as
under RED, one needs to perform exhaustive search over
the possible values of the renovation parameters $\{q_i, \ 0 \le i \le N\}$.
Since we are unaware of any analytic way of choosing these values,
adaptive search algorithms for partially observable
Markov decision processes from~\cite{kono1} were used instead.
Meta-heuristics (like particle swarm optimization), which are also applicable here,
were not used.

In Tables~1 and~2, one can see the numerical results for the four different
cases of the offered load\footnote{For the sake of reproducibility 
of the results 
presented in the paper, we also report the obtained values of the renovation 
probabilities: for $\rho=0.5$,
${\vec q}=(0.2544,0.0037,0.0053,0.0065,0.0122,0.0352,0.1108,0.1898,0.2129,0.1691)$;
for $\rho=1$, $q_0=0.0551$, $q_6=0.051$, $q_7=0.7166$, $q_8=0.0917$, and
$q_9=0.0856$;
for $\rho=2$, $q_0=0.1078$, $q_1=0.6374$, $q_4=0.0042$, $q_6=0.0084$, and
$q_9=0.2422$; and
for $\rho=3$, $q_1=0.4608$, and $q_2=0.5392$.}~$\rho$: $\rho=0.5$~--- underloaded system;
$\rho=1$~--- critically loaded system;
and $\rho=2$ and~3~--- overloaded system. The values displayed are the 
values from the numerical experiments rounded to three decimal digits.

%\noindent and in each case compute the stationary 
%loss ratio, average system content  
%and average number of consecutive losses along with
%their standard deviations. 


\begin{table*}\small
\begin{center}
\parbox{400pt}{\Caption{Performance characteristics of the $M/D/1/9$ system with the RED 
mechanism~(\ref{df}) and the $M/D/1/9$ the renovation mechanism (ren.)\
under the offered load $\rho=0.5$ and $\rho=1$}
}
%\label{my-label}
\vspace*{2ex}

\tabcolsep=8pt
\begin{tabular}{cc|c|c|c||c|c|c|}
\cline{3-8}
                                        &  & \multicolumn{3}{c||}{$\rho=0.5$} & \multicolumn{3}{c|}{$\rho=1$} \\ \cline{3-8} 
                                        &  &   Tail Drop      & RED     &    ren.   &    Tail Drop   &      RED &    ren.   \\ \hline
\multicolumn{2}{|c|}{loss ratio}    &   0    &   0.002        &   0.002  &    0.051   &    0.091        &   0.104    \\ \hline
\multicolumn{1}{|c|}{\textbf{system}} & average &   0.750    &   0.741        &    0.744    &    5.064   &      3.000       &   2.999  \\ \cline{2-8} 
\multicolumn{1}{|c|}{\textbf{size}} & standard deviation &   0.946    &    0.920      &    0.935      &    2.897   &   1.887       &   2.091      \\ \hline
\multicolumn{1}{|c|}{\textbf{consecutive}} &average  &  1.152     &  1.053        &   1.800     &    1.359   &    1.239       &  6.876         \\ \cline{2-8} 
\multicolumn{1}{|c|}{\textbf{losses}} & standard deviation &  0.403     &   0.240       &    1.260     &    0.647   &       0.561        &   0.852     \\ \hline
\end{tabular}
\end{center}
\vspace*{-6pt}
\end{table*}




\begin{table*}\small %tabl2
\begin{center}
\parbox{400pt}{\Caption{Performance characteristics of the $M/D/1/9$ system with the RED 
mechanism~(\ref{df}) and the $M/D/1/9$ the renovation mechanism (ren.)\
under the offered load $\rho=2$ and $\rho=3$}
}
%\label{my-label}
\vspace*{2ex}

\tabcolsep=8pt
\begin{tabular}{cc|c|c|c||c|c|c|}
\cline{3-8}
                                        &  & \multicolumn{3}{c||}{$\rho=2$} & \multicolumn{3}{c|}{$\rho=3$} \\ \cline{3-8} 
                                        &  &   Tail Drop      & RED     &    ren.   &    Tail Drop   &      RED &    ren.   \\ \hline
\multicolumn{2}{|c|}{loss ratio}    &   0.500    &    0.500        &   0.502     &    0.667   &    0.667       &     0.667      \\ \hline
\multicolumn{1}{|c|}{\textbf{system}} & average &   9.372    &    7.146       &   7.142      &  9.646     &      8.390        &    7.114  \\ \cline{2-8} 
\multicolumn{1}{|c|}{\textbf{size}} & standard deviation &   0.744    &     1.436       &  2.387      &    0.523   &     1.090        &    2.246  \\ \hline
\multicolumn{1}{|c|}{\textbf{consecutive}} &average  &  1.884     &   1.996         &  1.592       &     2.542  &     2.876        &   2.141    \\ \cline{2-8} 
\multicolumn{1}{|c|}{\textbf{losses}} & standard deviation &   1.069    &    1.366        &  1.100      &    1.454   &       2.064       &  1.236        \\ \hline
\end{tabular}
\end{center}
\end{table*}


Data is the tables show that with respect to the loss ratio,
renovation can perform as good as RED in the wide range of the offered load~$\rho$.
The only exception is the case $\rho=1$: here, renovation can keep
only the average system size at the same level as RED; other four 
performance characteristics are worse. 

If we fix the loss ratio, then the renovation mechanism
can guarantee at least the same value of the average system size as guaranteed by RED.
It is worth noticing that as the offered load increases, the average system size 
under the renovation mechanism becomes smaller than the average system size under RED.
Yet, renovation keeps the queue less stable than RED:
the standard deviation of system size is always smaller for RED.

If we fix the loss ratio and the average system size,
then the renovation mechanism spreads out the losses 
worse than RED when the system is underloaded and 
better than RED when it is overloaded.
This can be seen from the values of the averages and
standard deviations in the last two rows of Tables~1 and~2.

\vspace*{-6pt}

\section{Concluding Remarks}

\noindent
Even though the idea behind the renovation-type AQMs is completely 
different from the idea behind RED-type AQMs, 
renovation-type AQMs may allow one to achieve in some cases at least 
the same system performance level as guaranteed by RED-type AQMs. 
Although the comparison presented here is based only on a~single RED dropping 
function~(\ref{df}), 
our numerical experiments show that the results remain 
qualitatively the same for RED-type AQMs with other dropping functions.
Being defined by~$N$~parameters, the renovation mechanism is very flexible
and this constitutes its strength and weakness.
By varying the values of the renovation probabilities~$q_i$,
it is possible to carry out conditional optimization,
but good searching procedures are required here.

Implementation of the renovation as a~packet dropping mechanism
requires \textit{a~priori} tuning and/or operational configuration of its parameters.
Thus, whether it is appropriate to use renovation as a~packet dropping mechanism 
or not in practice heavily depends on the use case.
Although the tuning of the renovation parameters~$q_i$ can be made on the 
fly during operation, with respect to the recommendations of the RFC~7567~\cite{RFC7567},
renovation mechanism is not the proper choice for the network congestion control 
unless simple recommendations on how to set up the renovation parameters are given.
We believe this can be done based on more deep and insightful numerical experiments.

There remain a~large number of unresolved issues 
related to the renovation mechanism 
(e.\,g., can renovation ensure fairness among competing flows?
may the average queue size instead of its instantaneous value
increase the efficiency of the renovation mechanism?)
and this motivates its further analysis. 
Furthermore, evaluation of the renovation mechanism with parameters 
adapted to a~realistic router/switch use case
and/or evaluation which includes TCP feedback loops 
of several flows remains an open issue as well.

\vspace*{-6pt}


\Ack
  \noindent
   The reported study was partially funded by the Russian Foundation for 
Basic Research according to the research project No.\,18-07-00692.

The authors would like to thank the anonymous referees for their valuable comments 
which helped to improve the paper.
  
 \renewcommand{\bibname}{\protect\rmfamily References}

%\vspace*{-6pt}

\vspace*{-6pt}

{\small\frenchspacing
{\baselineskip=10.65pt
\begin{thebibliography}{99}
\bibitem{RFC7567} %1
\Aue{Baker, F., and G.~Fairhurst.} 2015.
IETF recommendations regarding active queue management.
Available at: {\sf https://tools.ietf.org/html/7567} (accessed October~4, 2018).



\bibitem{Adams} %2
\Aue{Adams, R.} 2013. 
Active queue management: A~survey. \textit{IEEE Commun. Surv.
Tut.} 15(3):1425--1476.

\bibitem{Bonald} %3
\Aue{Bonald, T., M.~May, and J.\,C.~Bolot.}
2000. Analytic evaluation of RED performance. 
\textit{IEEE Conference on Computer Communications Proceedings}
3:1415--1424.

\bibitem{hao} %4
\Aue{Hao, W., and Y.~Wei.} 2005.
An extended $GI^X/M/1/N$ queueing
model for evaluating the performance of AQM algorithms
with aggregate traffic.
\textit{Networking and mobile computing}.
Eds.\ Xicheng Lu and Wei Zhao.
{Lecture notes in computer science ser.} Springer. 3619:395--404.

\bibitem{Chyd} %5
\Aue{Chydzi$\acute{\mbox{n}}$nski, A., and L.~Chr$\acute{\mbox{o}}$st.} 
2011. Analysis of AQM queues with queue size based packet
dropping. \textit{Int. J.~Appl. Math. Comp.} 21(3):567--577.

\bibitem{Chyd2} %6
\Aue{Chydzi$\acute{\mbox{n}}$nski, A., and P.~Mrozowski.} 2016. 
Queues with dropping functions and general arrival
processes. \textit{PLoS ONE} 11(3):e0150702. Available at: 
{\sf https://\linebreak journals.plos.org/plosone/article?id=10.1371/journal.\linebreak pone.0150702} 
(accessed October~4, 2018).

\bibitem{oleg} %7
\Aue{Tikhonenko, O., and W.~Kempa.} 2016. Performance evaluation of 
an $M/G/n$-type queue
with bounded capacity and packet dropping. \textit{Int. J.~Appl.
Math. Comp.} 26(4):841--854.




\bibitem{konnew} %8
\Aue{Konovalov, M.\,G., and R.\,V.~Razumchik.} 2018. 
Numerical analysis of improved access restriction algorithms in a~$GI/G/1/N$
system. \textit{J.~Commun. Technol. El.} 63(6):616--625. 


\bibitem{Kreinin} %9
\Aue{Kreinin, A.\,Y.} 1997.
Queueing systems with renovation. 
\textit{J.~Appl. Math. Stochastic Analysis} 10(4):431--441.


\bibitem{zarN2} %10
\Aue{Zaryadov, I.\,S.} 2009. Queueing systems with general renovation.
\textit{Conference (International)
on Ultra Modern Telecommunications Proceedings}. 1--4.

\bibitem{Zaryadov2009} %11
\Aue{Zaryadov, I.\,S., and A.\,V.~Pechinkin.} 2009.
Stationary time characteristics of the ${GI/M/n/\infty}$
system with some variants of the generalized renovation discipline. \textit{Automat.
Rem. Contr.} 70(12):2085--2097.

\bibitem{Zaryadov2010} %12
\Aue{Zaryadov, I.\,S.}
2010. The ${GI/M/n/\infty}$ queuing system with generalized renovation.
\textit{Automat. Rem. Contr.} 71(4):663--671.

\bibitem{zarN1} %13
\Aue{Korolkova, A., and I.~Zaryadov.} 2010.
The mathematical model of the traffic transfer process with a~rate adjustable by {RED}.
\textit{Conference (International) on Ultra Modern Telecommunications Proceedings}. 
1046--1050.

\bibitem{RFC8289} %14
\Aue{Nichols, K., V.~Jacobson, A.~McGregor, and J.~Iyengar.} 2018.
Controlled delay active queue management.
Available at: {\sf https://datatracker.ietf.org/doc/rfc8289} (accessed October~4, 2018).

\bibitem{Riordan1962} %15
\Aue{Riordan, J.} 1962. \textit{Stochastic service systems}. 
SIAM ser. in applied mathematics. New York, NY: Wiley. 139~p.

\bibitem{arxivRK}  %16
\Aue{Konovalov, M.,  and R.~Razumchik.} 2017.
Queueing systems with renovation vs.\ queues with RED. Supplementary material. 
\textit{ArXiv e-prints}. Available at: {\sf https://arxiv.\linebreak org/abs/1709.01477/}
(accessed October~4, 2018).

\bibitem{kulk} %17
\Aue{Kulkarni, V.\,G.} 2016. \textit{Modeling and analysis of stochastic systems}. 
3rd ed. Chapman \&~Hall/CRC texts in statistical science ser.
Chapman \&~Hall/CRC. 606~p.

\bibitem{kono1} %18
\Aue{Konovalov, M.\,G.} 2007.
\textit{Metody adaptivnoy obrabotki informatsii i~ikh prilozheniya}
[Methods of adaptive information processing and their applications]. 
Moscow: Institute of Informatics Problems of RAS. 212~p.
\end{thebibliography} } }

\end{multicols}

\vspace*{-6pt}

\hfill{\small\textit{Received October 9, 2018}}

\vspace*{-18pt}
  

 \Contr

\noindent
\textbf{Konovalov Mikhail G.} (b.\ 1950)~--- 
Doctor of Science in technology, principal scientist, Institute of Informatics
Problems, Federal Research Center ``Computer Science and Control'' 
of the Russian Academy of Sciences, 44-2~Vavilov Str., Moscow 119333, 
Russian Federation; \mbox{mkonovalov@ipiran.ru}

\vspace*{3pt}


\noindent
\textbf{Razumchik Rostislav V.} (b.\ 1984)~--- 
Candidate of Science (PhD) in physics and mathematics, leading scientist,
Institute of Informatics Problems, Federal Research Center ``Computer 
Science and Control'' of the Russian
Academy of Sciences, 44-2~Vavilov Str., Moscow 119333, Russian Federation; 
associate professor, Peoples'
Friendship University of Russia (RUDN University), 
6~Miklukho-Maklaya Str., Moscow 117198, Russian
Federation; \mbox{rrazumchik@ipiran.ru} %; \mbox{razumchik\_rv@rudn.ru}

\vspace*{6pt}

\hrule

\vspace*{2pt}

\hrule

\vspace*{-2pt}

%\newpage

%\vspace*{-24pt}

\def\tit{СРАВНЕНИЕ ДВУХ МЕХАНИЗМОВ АКТИВНОГО УПРАВЛЕНИЯ ОЧЕРЕДЬЮ В~СИСТЕМЕ $M/D/1/N$$^*$}

\def\titkol{Сравнение двух механизмов активного управления очередью в~системе $M/D/1/N$}

\def\aut{М.\,Г.~Коновалов$^1$, Р.\,В.~Разумчик$^{1,2}$}

\def\autkol{М.\,Г.~Коновалов, Р.\,В.~Разумчик}

{\renewcommand{\thefootnote}{\fnsymbol{footnote}} \footnotetext[1]
{Исследование выполнено при частичной финансовой поддержке РФФИ (проект 18-07-00692).}}



\titel{\tit}{\aut}{\autkol}{\titkol}

\vspace*{-11pt}

\noindent
$^1$Институт проблем информатики Федерального исследовательского 
центра <<Информатика и управление>>\linebreak
$\hphantom{^1}$Российской академии наук

\noindent
$^2$Российский университет дружбы народов 

\vspace*{5pt}

\def\leftfootline{\small{\textbf{\thepage}
\hfill ИНФОРМАТИКА И ЕЁ ПРИМЕНЕНИЯ\ \ \ том\ 12\ \ \ выпуск\ 4\ \ \ 2018}
}%
 \def\rightfootline{\small{ИНФОРМАТИКА И ЕЁ ПРИМЕНЕНИЯ\ \ \ том\ 12\ \ \ выпуск\ 4\ \ \ 2018
\hfill \textbf{\thepage}}}

\vspace*{-3pt}


\Abst{Представлены некоторые результаты численных экспериментов, подтверждающие
следующее обстоятельство: параметры механизма обобщенного обновления
могут быть подобраны таким образом,\linebreak\vspace*{-12pt}}

\Abstend{что уровень производительности
однолинейных сис\-тем массового обслуживания с обобщенным обновлением
может быть не ниже уровня производительности
систем с RED-по\-доб\-ны\-ми механизмами активного управ\-ле\-ния очередями.
Механизмы сравниваются на примере сис\-те\-мы $M/D/1/N$
по стационарным значениям сле\-ду\-ющих характеристик:
вероятность потери заявки, среднее число заявок в сис\-те\-ме,
среднее чис\-ло последовательных потерь заявок 
и~их средние квадратические отклонения.
Расчеты основаны на известных фактах,
а~также на ряде новых аналитических результатов для систем
с~обобщенным обновлением, полученных в данной работе.}

\KW{система массового обслуживания; 
алгоритмы активного управления очередями; обобщенное обновление}

\DOI{10.14357/19922264180402}



%\vspace*{-3pt}


 \begin{multicols}{2}

\renewcommand{\bibname}{\protect\rmfamily Литература}
%\renewcommand{\bibname}{\large\protect\rm References}

{\small\frenchspacing
{%\baselineskip=10.8pt
\begin{thebibliography}{99}
%\vspace*{-3pt}

\bibitem{RFC7567-1} %1
\Au{Baker F., Fairhurst~G.}
IETF recommendations regarding active queue management, 2015.
{\sf https://tools.\linebreak ietf.org/html/7567}.



\bibitem{Adams-1} %2
\Au{Adams R.}
Active queue management: A~survey~// 
{IEEE Commun. Surv. Tut.}, 2013. Vol.~15. No.\,3. P.~1425--1476.

\bibitem{Bonald-1} %3
\Au{Bonald T., May M., Bolot~J.\,C.} Analytic evaluation of RED performance~//
{IEEE Conference on Computer Communications Proceedings}, 2000. 
Vol.~3. P.~1415--1424.

\bibitem{hao-1} %4
\Au{Hao W., Wei~Y.}
An extended $GI^X/M/1/N$ queueing
model for evaluating the performance of AQM algorithms
with aggregate traffic~// Networking and mobile computing~/
Eds. Xicheng Lu and Wei Zhao.~---
Lecture notes in computer science ser.~--- Springer, 2005. Vol.~3619. P.~395--404.

\bibitem{Chyd-1} %5
\Au{Chydzi$\acute{\mbox{n}}$ski A., Chr$\acute{\mbox{o}}$st~L.} 
Analysis of AQM queues with queue size based packet
dropping~// Int. J.~Appl. Math. Comp., 2011. Vol.~21. No.\,3. P.~567--577.

\bibitem{Chyd2-1} %6
\Au{Chydzi$\acute{\mbox{n}}$ski A.,  Mrozowski~P.}
 Queues with dropping functions and general arrival
processes~// PLoS ONE, 2016. Vol.~11. No.\,3. 
{\sf https://journals.plos.org/plosone/\linebreak article?id=10.1371/journal.pone.0150702}.

\bibitem{oleg-1} %7
\Au{Tikhonenko O., Kempa~W.} Performance evaluation of an $M/G/n$-type queue
with bounded capacity and packet dropping~// {Int. J.~Appl.
Math. Comp.}, 2016. Vol.~26. No.~4. P.~841--854.



\bibitem{konnew-1} %8
\Au{Konovalov M.\,G., Razumchik~R.\,V.}
Numerical analysis of improved access restriction algorithms in a~$GI/G/1/N$
system // {J.~Commun. Technol. El.}, 2018. Vol.~63. No.\,6. P.~616--625.

\bibitem{Kreinin-1} %9
\Au{Kreinin A.\,Y.}
Queueing systems with renovation //
{J.~Appl. Math. Stochastic Analysis}, 1997. Vol.~10. No.~4. P.~431--441.

\bibitem{zarN2-1} %10
\Au{Zaryadov~I.\,S.} Queueing systems with general renovation~//
{Conference (International) on Ultra Modern Telecommunications Proceedings}, 2009.
P.~1--4.

\bibitem{Zaryadov2009-1} %11
\Au{Зарядов И.\,С.,  Печинкин~А.\,В.}
Стационарные временные характеристики системы ${GI/M/n/\infty}$
с~некоторыми вариантами дисциплины обобщенного об\-нов\-ле\-ния~//
{Автоматика и~телемеханика}, 2009. Вып.~12. С.~161--174.

\bibitem{Zaryadov2010-1} %12
\Au{Зарядов И.\,С.} 
Система массового обслуживания $GI/M/n/\infty$ с~обобщенным об\-нов\-ле\-ни\-ем~//
{Автоматика и~телемеханика}, 2010. Вып.~4. С.~130--139.

\bibitem{zarN1-1} %13
\Au{Korolkova A., Zaryadov~I.}
The mathematical model of the traffic transfer process with a~rate adjustable by {RED}~//
{Conference (International) on Ultra Modern Telecommunications Proceedings}, 2010.
P.~1046--1050.

\bibitem{RFC8289-1} %14
\Au{Nichols K., Jacobson V., McGregor A., Iyengar J.}
Controlled delay active queue management, 2018.
{\sf https:// datatracker.ietf.org/doc/rfc8289}.


\bibitem{Riordan1962-1} %15
\Au{Riordan J.} {Stochastic service systems}.~--- 
SIAM ser. in applied mathematics.~--- New York, NY, USA: Wiley, 1962. 139~p.

\bibitem{arxivRK-1} %16
\Au{Konovalov M.,  Razumchik~R.}
Queueing systems with renovation vs.\ queues with RED. Supplementary material~//
{ArXiv e-prints}, 2017. {\sf https://arxiv.\linebreak org/abs/1709.01477/}.

\bibitem{kulk-1} %17
\Au{Kulkarni V.\,G.} Modeling and analysis of stochastic systems. 3rd ed.~--- 
Chapman \&~Hall/CRC texts in statistical science ser.~---
Chapman \& Hall/CRC, 2016. 606~p.

\bibitem{kono1-1}
\Au{Коновалов М.\,Г.} 
{Методы адаптивной обработки информации и~их приложения.}~--- 
М.: ИПИ РАН, 2007. 212~с.
\end{thebibliography}
} }

\end{multicols}

 \label{end\stat}

 \vspace*{-3pt}

\hfill{\small\textit{Поступила в~редакцию  09.10.2018}}


%\renewcommand{\bibname}{\protect\rm Литература}
%\renewcommand{\figurename}{\protect\bf Рис.}
\renewcommand{\tablename}{\protect\bf Таблица}