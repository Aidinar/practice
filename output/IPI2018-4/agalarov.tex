\def\stat{agalarov}

\def\tit{ОПТИМИЗАЦИЯ ОБЪЕМА БУФЕРНОЙ ПАМЯТИ УЗЛА КОММУТАЦИИ ПРИ~СХЕМЕ 
ПОЛНОГО РАЗДЕЛЕНИЯ ПАМЯТИ$^*$}

\def\titkol{Оптимизация объема буферной памяти узла коммутации при~схеме 
полного разделения памяти}

\def\aut{Я.\,М.~Агаларов$^1$}

\def\autkol{Я.\,М.~Агаларов}

\titel{\tit}{\aut}{\autkol}{\titkol}

\index{Агаларов Я.\,М.}
\index{Agalarov Yа.\,M.}




{\renewcommand{\thefootnote}{\fnsymbol{footnote}} \footnotetext[1]
{Работа выполнена при частичной финансовой поддержке РФФИ (проекты 18-07-00692, 17-07-000717  
и~17-07-00851).}}


\renewcommand{\thefootnote}{\arabic{footnote}}
\footnotetext[1]{Институт проблем информатики Федерального исследовательского центра 
<<Информатика 
и~управление>> Российской академии наук, \mbox{agglar@yandex.ru}}

%\vspace*{8pt}


  

\Abst{Рассматривается задача оптимизации объема буферной памяти (БП) узла коммутации 
(УК) сети 
с~коммутацией пакетов (КП), в~котором используется схема полного разделения памяти между выходными линиями 
передачи, с~учетом потерь пакетов из-за переполнения памяти и~средних задержек пакетов. 
Для модели УК с~пуассоновскими входящими потоками, общими 
распределениями времени обслуживания и~одноканальными выходными линиями передачи 
получены свойства решения задачи и~разработан алгоритм поиска оптимального объема. 
В~рамках указанной модели разработан алгоритм поиска оптимального плана распределения 
БП при заданном ее объеме. Приведены результаты вычислительных 
экспериментов с~использованием разработанных алгоритмов. Показана применимость 
полученных в~работе подходов и~алгоритмов к~решению рассматриваемой задачи в~рамках 
модели УК с~рекуррентными входящими потоками, экспоненциальными 
временами обслуживания и~одноканальными линиями передачи.}

\KW{узел коммутации; управление потоками; распределение буферной памяти; 
оптимальный объем буферной памяти}

\DOI{10.14357/19922264180404}
  
%\vspace*{4pt}


\vskip 10pt plus 9pt minus 6pt

\thispagestyle{headings}

\begin{multicols}{2}

\label{st\stat}

\section{Введение}

  Технико-экономические характеристики сети КП
в~значительной степени зависят от управления ресурсами и~потоками 
требований в~уз\-лах-ис\-точ\-ни\-ках и~УК сети. 
Управление потоками предназначено для эффективной загрузки основных 
ресурсов (БП и~каналов связи) в~условиях их ограниченного 
объема с~целью обеспечения требований к~качеству обслуживания (Quality of 
Service~--- QoS) в~сети. Зависимость качества обслуживания от объема памяти 
УК возникает из того, что при ограниченной пропускной способности каналов 
связи и~небольшом объеме памяти могут быть недопустимо большие потери 
пакетов, что вызывает рост числа повторных передач пакетов и, как следствие, 
резкий рост нагрузки на отдельных участках сети (возможно, и~в сети в~целом), 
а~при большом объеме памяти возможны недопустимо большие задержки 
пакетов в~УК. Отсюда и~возникает задача выбора оптимального объема БП УК с~точки зрения выполнения определенных требований к~качеству обслуживания 
пакетов и~затратам, связанным со стоимостью оборудования и~технической 
эксплуатации УК~[1, 2]. При этом следует учитывать зависимость качества 
обслуживания в~узле коммутации (и во всей сети в~целом) и~от схем (планов) 
распределения БП, из которых наиболее часто применяемыми на практике 
являются статические схемы управления~[3, 4]: CS (Complete Sharing); CP 
(Complete Partitioning); SMQ (Sharing with Maximum Queue Length); SMA 
(Sharing with Minimum Allocation); SMQMA (Sharing with Maximum Queue 
Length and Minimum Allocation). Зависимость качества обслуживания в~УК от 
схем распределения БП вызвана тем, что при бесконтрольном допуске пакетов 
в~УК происходит захват буферов пакетами более интенсивных потоков, что 
приводит к~неэффективному использованию выходных линий. 
  
  С момента зарождения сетей разработано великое множество алгоритмов 
локального управления потоками, многие из которых успешно применяются 
и~в~современных сетях. Основные результаты по управлению потоками в~УК 
получены методом математического моделирования с~использованием 
различных типов систем массового обслуживания (СМО). Некоторые из этих 
результатов описаны в~работах~\cite{4-ag, 5-ag, 6-ag}, которые посвящены 
сравнительному анализу различных схем распределения БП, 
и~в~работах~[7--13], в~которых получены явные или алгоритмические решения 
задач по оптимизации параметров управления потоками в~УК (объема БП, схем 
распределения БП~УК). 
  
  В работе~\cite{5-ag} приводится подробное описание методов анализа 
и~расчета некоторых наиболее известных схем распределения БП и~анализ 
известных результатов оптимизации схем CP и~SMQ в~предположении 
пуассоновских входящих потоков, экспоненциального времени обслуживания 
и~одноканальных линий передачи. 

В~работе~\cite{7-ag} при этом же 
предположении рас\-смот\-ре\-но динамическое управ\-ле\-ние рас\-пре\-де\-ле\-нием БП 
и~получен для случая трех каналов вид\linebreak допустимого пространства состояний, 
соответствующий оптимальному решению. В~работе~\cite{8-ag} рассмотрена 
двухступенчатая схема управления потоками в~УК и~приведен алгоритм 
оптимизации интенсивностей потоков пакетов, допускаемых на первой ступени 
в~УК, при схеме SMQ на второй ступени и~в предположении пуассоновских 
входных потоков, экспоненциальных времен обслуживания и~многоканальных 
выходных линий передачи. 

В~работе~\cite{9-ag} предложен эвристический 
алгоритм решения задачи совместного управления объемом БП
и~пропускной способностью линий передачи, который рассматривается 
в~следующих аспектах: 
\begin{enumerate}[(1)]
\item при заданной степени загрузки каналов и~заданной 
пропускной способности найти необходимый объем БП УК, 
обеспечивающий заданную среднюю задержку в~сети 
и~требуемую вероятность потери ячеек вследствие переполнения буферов; 
\item при заданной степени загрузки каналов и~известных объемах 
БП найти необходимую пропускную способность, обеспечивающую 
заданную среднюю задержку в~сети и~требуемую вероятность потери ячеек 
вследствие переполнения буферов.
\end{enumerate}

 В~работе~\cite{10-ag} авторы решили 
задачу выбора объема БП для СМО типа $M/M/1/r$, сформулировав ее как 
задачу нелинейного программирования:
\begin{enumerate}[(1)]
\item для фиксированной входной 
нагрузки найти объем накопителя (БП), при котором средняя задержка 
достигает минимума, а интенсивность потерь не превышает заданной 
величины; 
\item для фиксированного объема накопителя (БП) найти значение 
входной нагрузки, при которой средняя задержка достигает минимума, 
а~интенсивность потерь не превышает заданной величины.
\end{enumerate}

 Для СМО типа 
$G/M/1$ и~$M/G/1$ с~интегральной целевой функцией, учитывающей среднюю 
задержку в~БП и~вероятности потерь пакетов из-за переполнения БП, 
оптимизационная задача ограничения длины очереди решена 
в~работах~\cite{11-ag, 12-ag, 13-ag}. 
  
  Ниже рассматривается экономическая модель УК с~$L$ выходными линиями 
передачи, пред\-став\-лен\-ная в~виде~$L$~СМО типа 
$M/G/1/N$, в~которой качество обслуживания регулируется методом штрафных 
функций. В~рамках этой модели решается задача выбора оптимального объема 
БП УК и~оптимальной схемы распределения типа~CP. 

\section{Модель узла коммутации и~постановка задачи} 
  
  В качестве модели УК рассматривается многопотоковая СМО типа 
$M_L/G/L$ с~накопителем емкости~$K$ и~$L$~приборами, на которую 
поступает $L\hm\leq K$ пуассоновских потоков заявок $\lambda_j\hm> 0$, 
$j\hm=1,\ldots, L$, к~каждому из которых для обслуживания заявок прикреплен 
отдельный прибор, время обслуживания на котором является независимой 
случайной величиной с~заданной функцией распределения~$G_j(t)$, 
$j\hm=1,\ldots, L$. Накопитель СМО моделирует БП, приборы~--- линии 
передачи. Заявки каждого потока обслуживаются в~порядке поступления, места 
в~накопителе распределяются по схеме СР, т.\,е.\ за каждой $j$-й линией 
закрепляется $k_j\hm>0$ мест накопителя, где $\sum\nolimits_{i=1}^L k_i\hm=K$. 
В~дальнейшем вектор $\overline{k}\hm= (k_1,\ldots , k_L)$, удовлетворяющий 
указанному условию, будем называть планом распределения БП. Поступившая 
заявка $j$-го потока занимает любое свободное место в~накопителе, если 
в~момент ее поступления число занятых мест в~накопителе меньше~$k_j$. Если 
заявка поступила в~накопитель, она занимает любое свободное место 
в~накопителе и~обслуживается на приборе в~порядке поступления. Заявка 
покидает систему только при завершении обслуживания, освободив 
одновременно прибор и~накопитель, а~на освободившийся прибор поступает 
очередная заявка соответствующего типа (если таковая есть в~накопителе). 
Качество работы системы оценивается с~помощью интегральной целевой 
функции (функцией дохода системы) со стоимостными па\-ра\-мет\-ра\-ми (весовыми 
коэффициентами), зависящей от среднего времени ожидания и~вероятности 
потерь заявок. Функция дохода имеет следующие стоимостные параметры:
  \begin{description}
  \item[\,]
  $C_{0j}>0$~--- плата, получаемая системой за обслуживание единицы длины 
заявки $j$-го потока; 
  \item[\,]
  $C_{1j}>0$~--- величина штрафа, который платит сис\-те\-ма, если отклонена 
(потеряна) заявка $j$-го потока;
  \item[\,]
  $C_{2j}>0$~--- величина штрафа системы за единицу времени задержки 
заявки $j$-го потока;
  \item[\,]
  $C_{3j}>0$~--- величина штрафа системы за единицу времени простоя $j$-й 
линии; 
  \item[\,]
  $C_{4j}>0$~--- затраты системы в~единицу времени на техническое 
обслуживание $j$-й линии.
  \end{description}
  
  Считается, что плату за обслуживание система получает в~момент 
завершения обслуживания каж\-дой заявки в~зависимости от длины заявки 
(длительности занятия прибора).
  
  Как следует из определения схемы СР, для каж\-до\-го фиксированного 
значения~$K$ и~фиксированного плана распределения БП рассматриваемая 
СМО представляет собой совокупность~$L$~независимых СМО типа 
$M/G_j/1/k_j$ (в~работе используются обозначения СМО по классификации 
Кендалла). Это делает возможным использование в~дальнейшем результатов 
работ~\cite{11-ag, 12-ag, 13-ag}, полученных для систем $M/G/1$ и~$G/M/1$ 
с~ограничением на длину очереди.
  
  Введем обозначения:
  \begin{description}
  \item[\,]
  $\pi_{ji}^{k_j}$, $0\leq i\leq k_j\hm=1$,~--- стационарная вероятность того, 
что в~системе находится~$i$ заявок потока~$j$ в~момент завершения 
обслуживания заявки $j$-м прибором;
  \item[\,]
  $g_j^{k_j}$~--- значение предельного дохода подсистемы $M/G_j/1/k_j$, 
усредненного по числу обслуженных $j$-й линией заявок, $j\hm=1,\ldots, L$;
  \item[\,]
  $g^{\overline{k}}$~--- значение суммарного предельного дохода всей 
системы, усредненного по числу обслуженных заявок при плане 
$\overline{k}\hm= (k_1, \ldots, k_L)$;
  \item[\,]
  $Q_j^{k_j}$~--- предельный средний доход, получаемый системой 
$M/G_j/1/k_j$ в~единицу времени;
  \item[\,]
  $Q^{\overline{k}}$~--- предельный суммарный средний доход, получаемый 
всей системой в~единицу времени при заданном плане~$\overline{k}$;
  \item[\,]
  $\overline{v}_j=\int\nolimits_0^\infty t\,dG_j(t)$~--- среднее время 
обслуживания заявки потока~$j$, $0\hm< \overline{v}_j\hm<\infty$, 
$j\hm=1,\ldots, L$;
  \item[\,]
  $\rho_j=\lambda_j \overline{v}_j$~--- нагрузка на линию~$j$, $j\hm=1,\ldots, 
L$.
  \end{description}
  
  Пусть $A_K\hm= \left\{ \overline{k}>0:\ \sum\nolimits_{i=1}^L k_i\hm= K\right\}$, 
$U_{\mathrm{CP}}\hm= \left\{ A_K, K\hm\geq L\right\}$, $U_{\mathrm{CP},N}\hm= \left\{ 
\overline{k}:\ L\hm\leq \sum\nolimits^L_{j=1} k_j \hm\leq N\right\}$, 
$Q^k\hm=\max\nolimits_{\overline{k}:\ \overline{k}\in A_K} Q^{\overline{k}}$.
  
  Ставятся следующие две задачи: 
  \noindent
  \begin{enumerate}[1.]
\item Для множества планов~$U_{\mathrm{CP}}$ найти значение $K^*\hm\geq L$ 
такое, что 
\begin{equation}
\max\limits_{K\geq L} Q^K=Q^{K^*}\,.
\label{e1-ag}
\end{equation}
\item Найти $\overline{k}^0\hm\in U_{\mathrm{CP},N}$ такой, что
\begin{equation}
\max\limits_{\overline{k}\in U_{\mathrm{CP},N}} k^{\overline{k}}=g^{\overline{k}^0}\,.
\label{e2-ag}
\end{equation}
\end{enumerate}
  
  Ниже всюду через~$K^*$ будем обозначать решение задачи~(1), через 
$\overline{k}^0\hm\in U_{\mathrm{CP},N}$~--- решение задачи~(2).
     
\section{Решение задачи~(1)}

     Так как при схеме СР рассматриваемая модель УК есть совокупность 
независимых СМО типа $M/G_j/1/k_j$, $j\hm=1,\ldots ,L$, то ее доход при 
заданном $\overline{k}\hm\in U_{\mathrm{CP},K}$ представляется в~виде суммы:
     $$
     Q^{\overline{k}}=\sum\limits^L_{j=1} Q_j^{k_j}\,.
     $$
Тогда, очевидно, верны равенства:
\begin{multline}
\max\limits_{K\geq L} Q^K =\max\limits_{K\geq L} \max\limits_{\overline{k}:\ 
\overline{k}\in A_K} \!\! Q^{\overline{k}} ={}\\
{}=\max\limits_{K\geq L} 
\max\limits_{\overline{k}:\ \overline{k}\in A_K} \sum\limits^L_{j=1} Q_j^{k_j} 
=\sum\limits^L_{j=1} \max\limits_{k_j>0} Q_j^{k_j}\,,
\label{e3-ag}
\end{multline}
т.\,е.\ решением задачи~(1) является $K^*\hm= \sum\nolimits^L_{j=1} k_j^*$, 
где $k_j^*$~--- решение задачи
\begin{equation}
\max\limits_{k_j>0} Q_j^{k_j}\,,\enskip j=1,\ldots, L\,.
\label{e4-ag}
\end{equation}
Таким образом, задача~(1) сводится к~задаче, аналогичной~(\ref{e4-ag}), 
рассмотренной в~работе~\cite{12-ag} для СМО типа $M/G/1$ с~ограничением на 
длину очереди.

  В работе~\cite{12-ag} для дохода СМО типа $M/G/1$ при заданном 
пороговом значении длины очереди~$k$ получено выражение:
  $$
  Q^k=\lambda\left( 1-\theta^k\right) g^k\,,
  $$
где $g^k$~--- значение предельного дохода системы $M/G/1/k$, усредненного 
по числу обслуженных линией заявок; $\theta^k$~--- вероятность того, что 
поступившая заявка будет потеряна (для величин~$g^k$ и~$\theta^k$ 
в~\cite{12-ag} приведены явные аналитические выражения). Эта формула для 
системы $M/G_j/1/k_j$ ($j\hm=1,\ldots, L$) в~новых обозначениях записывается 
в~виде:
$$
Q_j^{k_j}=\lambda_j\left( 1-\theta_j^{k_j}\right) g_j^{k_j}\,.
$$
Здесь (см.~\cite{12-ag}) 
$$
\theta_{k_j}^{k_j} =1-\fr{1}{\pi_{j_0}^{k_j}+\rho}\,,\enskip 
g^{k_j}=\sum\limits_{i=0}^{k_j-1} q^{k_j}_{ji} \pi^{k_j}_{ji}\,,
$$
где
\begin{multline*}
q_{ji}^{k_j}=C_{0j} \overline{v}-C_{1j}\sum\limits^\infty_{m=k_j-i+1}\left( m-
k_j+i\right) r_{jm}-{}\\
{}-\fr{C_{2j}}{\lambda_j}\left[ \fr{1}{2}\sum\limits_{m=2}^{k_j-i+1}\! \!\!(m-1) 
mr_{jm} +\left( k_j-i\right)\!\!\!\!\!\sum\limits^\infty_{m=k_j-i+2}\!\!\!\!\!\!\! mr_{jm}-{}\right.\hspace*{-6.5886pt}\\
\left.{}-\fr{1}{2}\left( k_j-i\right) \left( k_j-i+1\right) \sum\limits^\infty_{m=k_j-
i+2} r_{jm}\right] -{}\\
{}- C_{2j} (i-1)\overline{v}_j -C_4\overline{v}_j\,,\enskip 1\leq 
i\leq k_j-1\,;
\end{multline*}

\noindent
\begin{align*}
q^{k_j}_{j_0}&=q^{k_j}_{j_1}-\fr{C_{3j}+C_{4j}}{\lambda_j}\,;
\\
r_{jm}&=\int\limits_0^\infty \fr{(\lambda_j v)^m}{m!}\,e^{-\lambda_j v}\,dG_j(v)\ \mbox{при}\ 
m\geq 0\,.
\end{align*}

Для системы $M/G_j/1/k_j$ функции $\tilde{G}(k)$ и~$\tilde{F}(k)$, введенные 
в~\cite{12-ag}, в~новых обозначениях примут вид:
\begin{multline}
H_j(k_j)=C_{0j}-\fr{C_{1j}}{\overline{v}_j} \left( \rho_j-1\right) -C_{4j} -{}\\
{}- \fr{C_{2j}}{\rho_j} \left[ \tilde{F}_j(k_j)+(k_j-1)\rho_j\right]\,;
\label{e5-ag}
\end{multline}

\vspace*{-12pt}

\noindent
\begin{multline*}
\tilde{F}(k_j)=\left( \sum\limits_{i=1}^{k_j} \pi_{ji}^{k_j+1} \!\!\!\!
\sum\limits^\infty_{m=k_j-i+1}\!\!\!\!\! \!\left[ m-(k_j-i+1)\right] k_m+{}\right.\\
\left.{}+ \pi_{j_0}^{k_j+1}\sum\limits^\infty_{m=k_j} \left(m-k_j\right)k_m
\vphantom{\sum\limits_{i=1}^{k_j}}
\right)\Bigg/ \pi_{jk_j}^{k_j+1}\,.
\end{multline*}
  
  Из~(\ref{e3-ag}), (\ref{e5-ag}) и~утверждений~1 и~2 работы~\cite{12-ag} 
вытекают следующие два свойства решения задачи~(1).
  
  \smallskip
  
  \noindent
   \textbf{Свойство~1.}\ При любых значениях параметров $0\hm\leq 
C_{ij}\hm<\infty$, $i\hm=0$, 1,  3, 4, и~$0\hm< C_{2j}\hm<\infty$, $j\hm=1,\ldots 
,L$, существует оптимальное значение $0\hm< K^*\hm<\infty$, причем 
$$
K^*= \sum\limits^L_{j=1} k_j^*\,,
$$ 
где $k_j^*\hm= \min\left\{ k_j>0:\ 
H_j(k_j)\hm\leq Q_j^{k_j}\right\}$. Eсли $C_{2j}\hm=0$ и~$Q_j^1\hm< H_j(1)$  
хотя бы для одного $j\hm\in \overline{1,L}$, то $k_j^*\hm=\infty$, 
$K^*\hm=\infty$.
   
   \smallskip
   
   \noindent
   \textbf{Свойство~2.}\ Значение $0\hm< K^*\hm<\infty$ удовлетворяет условию 
$\max\nolimits_{K>0} Q^K\hm= Q^{K^*}$ тогда и~только тогда, когда 
$$
K^*= \sum\limits^L_{j=1} 
k_j^*$$ 
и~$\overline{k}^*\hm= (k_1^*, \ldots, k_L^*)$ удовлетворяет одному из двух условий:
   \begin{itemize}
\item[(a)] для каждого $j\hm\in \overline{1,L}$
$$
k_j^*=\begin{cases}
\min\left\{ r_j>0:\ H_j(k_j)\leq Q_j^{k_j}\right\} &\\
& \hspace*{-40mm}\mbox{при}\ 
0<C_{2j}<\infty\ \mbox{и}\ H_j(1)>Q_j^1\,;\\
1 & \hspace*{-40mm}\mbox{при}\ H_j(1)\leq Q_j^1\,;
\end{cases}
$$
\item[(б)] для каждого $j\in \overline{1,L}$
$$
Q_j^{k_j^*-1}< Q_j^{k_j^*}\,;\enskip Q_j^{k_j^*+1}\leq Q_j^{k_j^*}\,.
$$ 
  \end{itemize}
  
  Из условия~(а) свойства~2 вытекает следующий алгоритм нахождения 
оптимальных значений~$\overline{k}^*$ и~$K^*$~\cite{12-ag}.
  \begin{enumerate}[1.]
\item Положить $j=1$.
\item Положить $k_j=1$.
\item Вычислить $H_j(k_j)$ и~$Q^{k_j}$.
\item Если $C_{2j}=0$ и~$Q^{k_j}\hm< H_j(k_j)$, то положить 
$k_j^*\hm=\infty$ и~перейти к~п.~9.
\item Если выполняется неравенство $Q^{k_j}\hm\geq H_j(k_j)$, то 
перейти к~п.~8.
\item Увеличить $k_j$ на единицу.
\item Вычислить $H_j(k_j)$ и~$Q^{k_j}$ и~перейти к~п.~4.
\item Положить $k_j^*\hm=k_j$.
\item Если $j<L$, то увеличить~$j$ на единицу и~перейти к~п.~2.
\item Положить $K^*\hm=\sum\nolimits^L_{j=1} k_j^*$.
\item  Конец алгоритма.
\end{enumerate}

  
  
  Приведем некоторые выражения, упрощающие расчет величин 
$\tilde{F}_j(k_j)$ и~$H_j(k_j)$, $j\hm=1,\ldots, L$, и~показывающие некоторые 
их свойства. Использовав соотношения для $\pi^{k_j}_{ji}$ 
и~$\overline{F}_j(k_j)$, полученные в~\cite{12-ag} (см.~(4) и~(12)), находим:
  \begin{equation}
  \tilde{F}_j(k_j)=\overline{F}_j(k_j)+\rho_j -1 =\fr{\pi_{j_0}^{k_j+1}+\rho_j-1} 
{\pi_{jk_j}^{k_j+1}}-r_{j_0}\,.
  \label{e6-ag}
  \end{equation}
Отсюда получаем:
\begin{multline*}
\tilde{F}_j(k_j) -\tilde{F}_j(k_j+1)={}\\
{}= \fr{\pi_{j_0}^{k_j+1}}{\pi_{jk_j}^{k_j+1}}+ 
\fr{\rho_j-1}{\pi^{k_j+1}_{jk_j}}- \fr{\pi_{j_0}^{k_j+2}}{\pi_{jk_j}^{k_j+2}}-
\fr{\rho_j-1}{\pi_{jk_j}^{k_j+2}}=
\fr{\pi_{j_0}^{k_j+1}}{\pi_{jk_j}^{k_j+1}} -{}\\
{}- \fr{\pi_{j_0}^{k_j+1}\left(1-
\pi_{jk_j}^{k_j+2}\right)}{\pi_{jk_j}^{k_j+2}}+
\left( \rho_j-1\right) \!\left(\! \fr{1}{\pi_{jk_j}^{k_j+1}}-
\fr{1} {\pi_{jk_j}^{k_j+2}}\!\right)={}\\
{}= \left( \pi_{j_0}^{k_j+1}+\rho_j -1\right) \left( \fr{1}{ \pi_{jk_j}^{k_j+1}}-
\fr{1}{\pi_{jk_j}^{k_j+2}}\right) +\pi_{j_0}^{k_j+1}\,.
\end{multline*}
Из правой части последнего равенства следует, например, что величины 
$\tilde{F}_j(k_j)\hm-\tilde{F}_j(k_j\hm+1)\hm< \rho_j$ (т.\,е.~$H_j(k_j)$~--- 
убывающая функция по~$k_j$) при условии $\pi_{j_0}^{k_j+1}\hm\leq \rho_j$ 
(заметим, что выполняются неравенства $\pi_{j_0}^{k_j+1} \hm+\rho_j\hm- 
1\hm>0$, $1/\pi_{jk_j}^{k_j+1} \hm- 1/\pi_{jk_j}^{k_j+2} \hm<0$, так как 
$\pi_{j_0}^{k_j+1}$ и~$\pi_{jk_j}^{k_j+1}$ убывают по $k_j\hm\geq 0$ 
и~$\lim\nolimits_{k_j\to\infty} \pi_{j_0}^{k_j+1} \hm= 1\hm- \rho_j$~\cite{14-ag}). 

Из формул~(\ref{e5-ag}), (\ref{e6-ag}) данной статьи и~(4) из~\cite{12-ag} 
следует, что при известных значениях величин~$r_{ji}$, $i\hm=1,\ldots, k_j$, 
$j\hm=1,\ldots , L$, для вычисления значений~$\overline{k}^*$ и~$K^*$ 
предложенный алгоритм требует порядка $\sum\nolimits^L_{j=1} k_j^{*\,2}$ 
арифметических операций.

 \section{Решение задачи (2)}

  Для решения задачи~(2) предлагается алгоритм, состоящий из следующих 
пунктов.
  \begin{enumerate}[1.]
\item Если $N=L$, то положить $k_j^0\hm= 1$, $j\hm=1,\ldots, L$, 
$K^0\hm=L$ и~перейти к~п.~12. 
\item Положить $K=L$, $m\hm=0$, $k_j\hm=1$, $\varphi_j\hm=0$, 
$j\hm=1,\ldots ,L$.
\item Вычислить $q_j^{k_j}$, $j\hm=1,\ldots ,L$.
\item Для $j=1,\ldots, L$ выполнить:
\begin{enumerate}[1.{1}]
\item Вычислить $g_j^{k_j+1}$, $\Delta_j(k_j)\hm= g_j^{k_j+1}\hm- 
g_j^{k_j}$;
\item Если $\Delta_j(k_j)\hm\leq 0$, то положить $\varphi_j\hm=1$, 
$m\hm= m\hm+1$. 
\end{enumerate}
\item Если $m=L$, то перейти к~п.~11.
\item Найти $j_0:\ \Delta_{j_0}(k_{j_0})\hm= \max \{ \Delta_j(k_j), 
\varphi_j\not=1$, $j\hm=1,\ldots ,L\}$.
\item Положить $K\hm=K\hm+1$, $k_j\hm= k_j\hm+1$ для $j\hm=j_0$.
\item Если  $K=N$, то перейти к~п.~11.
\item Вычислить $g_j^{k_j+1}$, $\Delta_j(k_j)\hm= g_j^{k_j+1}\hm- 
g_j^{k_j}$ для $j\hm=j_0$. 
\item Если $\Delta_{j_0}(k_{j_0})\hm\leq 0$ то положить 
$\varphi_{j_0}\hm=1$, $m\hm= m\hm+1$ и~перейти к~п.~5. 
\item Положить $k_j^0\hm=k_j$, $j\hm=1,\ldots ,L$, $K^0\hm= 
\sum\nolimits^L_{j=1} k_j^0$.
\item Конец алгоритма.
\end{enumerate}
  
  Справедливо следующее 
  
  \smallskip
  
  \noindent
  \textbf{Утверждение~1.}\ \textit{Результат работы алгоритма является 
решением задачи}~(2).
  
  \smallskip
  
  \noindent
  Д\,о\,к\,а\,з\,а\,т\,е\,л\,ь\,с\,т\,в\,о\,.\ \
  Справедливость утверждения при $N\hm=\infty$ следует непосредственно из 
условия~(б) свойства~2 решения задачи~(1). Действительно, согласно алгоритму 
в~этом случае для каждого $j\hm=1,\ldots ,L$ величина~$k_j$ увеличивается до 
тех пор, пока не выполнятся неравенства упомянутого выше условия 
свойства~2. В~этом случае $K^0\hm=K^*$.
  
  Докажем для случая $N\hm<\infty$. Если $K^*\hm= \sum\nolimits^L_{j=1} 
k_j^*\hm<N$, то, как легко видеть, алгоритм работает так же, как и~при 
$N\hm=\infty$, и,~следовательно, утверждение справедливо и~$K^0\hm=K^*$.
  
  Пусть $K^*\geq N$. В~работах~\cite{11-ag, 12-ag} для~$g_j^{k_j}$, 
$j\hm=1,\ldots ,L$, приводится равенство:
  \begin{equation}
  q_j^{k_j}-g_j^{k_j+1}=\pi_{jk_j}^{k_j+1} \left[ g_j^{k_j}-G_j(k_j)\right]\,,
  \label{e7-ag}
  \end{equation}
где $G_j(k_j)=H_j(k_j)\overline{v}_j$.

Согласно следствию теоремы~1 из~\cite{11-ag} (см.\ пп.~1 и~3) верны 
неравенства $g_j^{k_j}\hm< g_j^{k_j+1}$ при $0\hm< k_j\hm < k_j^*$, 
$j\hm=1,\ldots, L$. Тогда, так как $\pi_{jk_j}^{k_j+1}\hm<1$ 
и~$H_j(k_j)\overline{v}_j$~--- убывающие по~$k_j$ функции, из~(\ref{e6-ag}) 
получаем неравенства:
\begin{multline}
\Delta_j (k_j) <\Delta_j(k_j+1)\\
\mbox{для}\ 0<k_j\leq k_j^*-2\ \mbox{при}\ 
k_j^*>2\,.
\label{e8-ag}
\end{multline}

Рассмотрим последовательности $\{\Delta_j (k_j), k_j\hm>0\}$, $j\hm=1,\ldots ,L$, 
и~объединим их в~одну последовательность $\{ \Delta (k), k\hm>0\}$, 
упорядоченную по убыванию элементов. Обратим внимание, что алгоритм 
каждый раз, перед тем как увеличить~$K$ на единицу, в~п.~6 находит элемент 
$\Delta (K\hm+1)\hm= \Delta_{j_0} (k_{j_0})$ объединенной 
последовательности и~выполняет п.~7. Заметим также, что 
\begin{multline*}
g^{\overline{k}} 
= \sum\limits^L_{j=1} g_j^{k_j} = \sum\limits^L_{j=1} \left[ 
\sum\limits^{k_j-1}_{m=1} \Delta_j (m)+g_j^1\right] = {}\\
{}=
\sum\limits^L_{j=1} \sum\limits_{m=1}^{k_j-1} \Delta_j(m)+ 
\sum\limits^L_{j=1} g_j^1\,,
\end{multline*}
и,~если $\overline{k}^\prime\hm= (k^\prime_1, 
\ldots , k_L^\prime) \hm\in U_{\mathrm{CP},K}$ удовле\-тво\-ря\-ет условию 
$g^{\overline{k}^\prime}\hm= \max\nolimits_{\overline{k}\in U_{\mathrm{CP},K}} 
g^{\overline{k}}$, то $\max\nolimits_{\overline{k}\in U_{\mathrm{CP},K+1}} 
g^{\overline{k}^\prime}\hm= 
g^{\overline{k}^\prime} \hm+\Delta_{j_0}(k_{j_0}^\prime)$, где~$j_0$ 
вычисляется согласно п.~6 алгоритма. Тогда, так как $K^*\hm\geq N$ и~имеет 
место~(\ref{e8-ag}), согласно алгоритму по завершении п.~11 имеют место 
равенства:
$$
K^0= \sum\limits^L_{j=1} k_j^0 =N\,;
$$

\vspace*{-12pt}

\noindent
\begin{multline*}
g^{\overline{k}^0} 
= \sum\limits^L_{j=1} g_j^{k_j^0} = \sum\limits^L_{j=1} 
\sum\limits_{m=1}^{k_j^0-1}\Delta_j (m) + \sum\limits^L_{j=1} 
g_j^1={}\\
{}= \sum\limits^N_{k=L+1}\Delta(k) + \sum\limits^L_{j=1} 
g_j^1= {}\\
{}=\max\limits_{\overline{k}\in U_{\mathrm{CP},N}} \left[ \sum\limits^L_{j=1} 
\sum\limits_{m=1}^{k_j-1}\! \Delta_j(m)+ \sum\limits^L_{j=1} g_j^1\right]= 
\max\limits_{\overline{k}\in U_{\mathrm{CP},N}}\hspace*{-5pt} g^{\overline{k}}\,.
\end{multline*}

Следовательно,  утверждение~1 справедливо.

\begin{figure*} %fig1
\vspace*{1pt}
\begin{minipage}[t]{80mm}
 \begin{center}
 \mbox{%
 \epsfxsize=79.0mm 
 \epsfbox{aga-1.eps}
 }
 \end{center}
\vspace*{-9pt}
\Caption{Зависимость величины дохода системы от интенсивности входного потока: 
\textit{1}~--- оптимальная схема СР;
\textit{2}~--- схема квадратного корня; \textit{3}~--- схема CS}
\end{minipage}
%\end{figure*}
\hfill
%\begin{figure*} %fig2
\vspace*{1pt}
\begin{minipage}[t]{80mm}
 \begin{center}
 \mbox{%
 \epsfxsize=79.0mm 
 \epsfbox{aga-2.eps}
 }
 \end{center}
\vspace*{-9pt}
\Caption{Зависимость дохода системы от интенсивности входного потока и~времени 
обслуживания приборами:
\textit{1}~--- $\mu_{11}\hm 
= 1{,}5$, $\mu_{12}\hm=3$, $\mu_{21}\hm=2$ и~$\mu_{22}\hm=4$; \textit{2}~--- 
 $\mu_{11}\hm=1$, $\mu_{12}\hm=2$, $\mu_{21}\hm= 
1{,}5$ и~$\mu_{22}\hm=3$}
\end{minipage}
\end{figure*}

\section{Примеры}

  Рассмотрим СМО $M_2/G/2$ с~функциями распределения времени 
обслуживания 
\begin{align*}
G_1(t)&=f_{11}\left(1- e^{-\mu_{11}t}\right)+ f_{12}\left(1- e^{-
\mu_{12}t}\right)\,;\\
G_2(t)&= f_{21}\left(1- e^{-\mu_{21}t}\right)+ f_{22}\left(1- 
e^{-\mu_{22}t}\right)\,,
\end{align*}
 стоимостными параметрами $C_{10}\hm= C_{20}\hm= 30$, 
$C_{11}\hm= C_{21} \hm= 15$, $C_{12}\hm=C_{22} \hm=4$, $C_{13}\hm= 
C_{23}\hm=0$, $C_{14}\hm= C_{24}\hm=0{,}4$ и~одинаковыми входными 
потоками на приборы (выходные линии передачи) $\lambda_1\hm=\lambda_2$. 
На графиках рис.~1 проиллюстрированы за\-ви\-си\-мости величины дохода~$Q^k$ 
рас\-смат\-ри\-ва\-емо\-го СМО от интенсивности входного потока для трех схем 
распределения БП: оптимальной схемы СР, схемы квад\-рат\-но\-го корня при 
$K\hm=K^*$, схемы CS при $K\hm=K^*$. На всех трех графиках доход 
обозначен через~$Q^K$. Значения па\-ра\-мет\-ров функций~$G_1(t)$ и~$G_2(t)$ 
соответственно равны: $f_{11}\hm= f_{12}\hm= f_{21}\hm= f_{22}\hm= 0{,}5$; 
$\mu_{11}\hm= 1$; $\mu_{12}\hm= 2$; $\mu_{21}\hm= 1{,}5$; 
$\mu_{22}\hm=3$. 

На рис.~2 изображена зависимость значения 
дохода~$Q^{K^*}$ системы от интенсивности входного потока и~времени 
обслуживания приборами. 




На рис.~3 приведена диаграмма зависимости оптимального значения~$K^*$ от 
интенсивности входного потока. 

{ \begin{center}  %fig3
 \vspace*{20pt}
  \mbox{%
 \epsfxsize=75.551mm 
 \epsfbox{aga-3.eps}
 }


\end{center}


\noindent
{{\figurename~3}\ \ \small{Зависимость оптимального объема БП от интенсивности входного потока 
и~времени обслуживания приборами:
\textit{1}~---  
$\mu_{11}\hm=1{,}5$, $\mu_{12}\hm= 3$, $\mu_{21}\hm= 2$ 
и~$\mu_{22}\hm=4$; \textit{2}~--- $\mu_{11}\hm=1$, 
$\mu_{12}\hm=2$, $\mu_{21}\hm= 1{,}5$ и~$\mu_{22}\hm=3$}}
}

%\vspace*{9pt}

\addtocounter{figure}{1}

\section{Заключение}

  На основании проведенных выше исследований можно сделать следующие 
выводы:
  \begin{itemize}
\item при схеме СР, если параметр стоимости для времени задержки~--- 
положительная величина, то существует конечный объем БП, при котором 
функция дохода~$Q^K$ достигнет максимального значения на множестве 
планов~$U_{\mathrm{CP}}$; 
\item функция $Q^K$ является унимодальной функцией от переменной 
$K\hm>0$;
\item функция дохода УК зависит от объема БП, интенсивностей входных 
потоков и~распределения БП между выходными линиями (см.\ рис.~1 и~2), 
т.\,е.\ оптимальная процедура локального управления потоками в~УК должна 
управлять всеми тремя указанными параметрами; 
\item оптимальный объем БП существенно зависит от 
интенсивности входной нагрузки, механизма распределения БП 
и~пропускной спо\-соб\-ности каналов связи (см.\ рис.~1 и~3).
\end{itemize}

  Отметим, что соотношение вида~(\ref{e7-ag}), использованное при 
доказательстве приведенного выше утверждения~1, получено также 
и~в~работе~\cite{13-ag} для СМО типа $G/M/1$ с~пороговым значением длины 
очереди. Указанное соотношение (см.~(12) в~\cite{13-ag}) имеет вид:
  $$
  g^k-g^{k+1} =\pi_0^{k+1}\left[ g^k -G(k)\right]\,,\enskip k>0\,,
  $$
где $k$ соответствует объему БП; $\pi_0^{k+1}$ и~$G(k)$ обладают 
аналогичными свойствами величин $\pi_{jk_j}^{k_j+1}$ и~$G_j(k_j)$, 
использованным выше в~доказательстве утверждения~1. Следовательно, 
алгоритм и~утверж\-де\-ние~1, приведенные в~разд.~3, применимы для 
рассмотренной выше модели УК также при предположениях: входящие 
потоки~--- рекуррентные; времена обслуживания~--- экспоненциальные. 
Однако заметим, что в~этом случае, в~отличие от рас\-смот\-рен\-но\-го выше, 
согласно работе~\cite{13-ag} $g^k$~--- предельный доход, усредненный по 
числу поступивших в~систему заявок.

  Отметим также, что каждая из схем СР, CS, SMQ, SMA и~SMQMA 
предпочтительней в~смысле эффективности, чем остальные, только 
в~соответствующей области значений параметров системы и~выбранного 
критерия эффективности~\cite{5-ag}. Полученные с~помощью предложенных 
выше алгоритмов оптимальные при схеме СР значения объема БП и~плана 
распределения БП могут рассматриваться для других схем как оценочные. 

{\small\frenchspacing
 {%\baselineskip=10.8pt
 \addcontentsline{toc}{section}{References}
 \begin{thebibliography}{99}
\bibitem{1-ag}
\Au{Клейнрок Л.} Вычислительные системы с~очередями~/ Пер. с~англ.~--- М.: 
Мир, 1979. 600~с. 
(\Au{Kleinrock~L.} {Queueing systems. Vol.~II: Computer applications}.~--- 
New York, NY, USA: Wiley, 1976. 549~p.)
\bibitem{2-ag}
\Au{Дэвис Д., Барбер~Д., Прайс~У., Соломонидес~С}. 
Сети связи для вычислительных систем~/ Пер. с~англ.~--- 
М.: Мир, 1982. 563~с. 
(\Au{Davies~D.\,W., Barber D.\,L.\,A., Price~W.\,L., Solomonides~C.\,M.} 1979. 
{Computer networks and their protocols}.~--- New York, NY, USA: Wiley. 487~p.)
\bibitem{3-ag}
\Au{Irland M.} Buffer management in a~packet switch~// IEEE T.~Commun., 
1978. Vol.~26. No.\,3. P.~328--337.
\bibitem{4-ag}
\Au{Kamoun F., Kleinrock~L.} Analysis of shared finite storage in a~computer 
networks node environment under general traffic conditions~// IEEE 
T.~Commun., 1980. Vol.~28. No.\,7. P.~992--1003.
\bibitem{5-ag}
\Au{Башарин Г.\,П., Самуйлов~К.\,Е.} Об оптимальной структуре буферной 
памяти в~сетях передачи данных с~коммутацией пакетов.~--- М., 1982. 70~с. 
\bibitem{6-ag}
\Au{Михеев П.\,А.} Анализ стратегий разделения конечной буферной памяти 
маршрутизатора между выходными каналами~// Автоматика и~телемеханика, 
2014. №\,10. С.~125--128.
\bibitem{7-ag}
\Au{Forschini G., Gopinath~B.} Sharing memory optimally~// IEEE 
T.~Commun., 1983. Vol.~31. No.\,3. P.~352--359.
\bibitem{8-ag}
\Au{Агаларов Я.\,М.} Двухступенчатое адаптивное управление потоками в~узле 
коммутации телекоммуникационной сети~// Вестник РУДН. Сер.
Прикладная математика и~информатика, 2003. №\,1. С.~134--141.
\bibitem{9-ag}
\Au{Линец Г.\,И.} Управление объемом буферной памяти и~пропускной 
способностью каналов в~мультисервисных сетях~// 
Инфокоммуникационные технологии, 2008. Т. 6. №\,2. С.~62--64. 
\bibitem{10-ag}
\Au{Жерновый Ю.\,В.} Решение задач оптимального синтеза для некоторых 
марковских моделей обслуживания~// Информационные процессы, 2010. 
Т. 10. №\,3. C.~257--274.
\bibitem{11-ag}
\Au{Агаларов Я.\,М., Агаларов~М.\,Я., Шоргин~В.\,С.} Об оптимальном 
пороговом значении длины очереди в~одной задаче максимизации дохода 
СМО типа $M/G/1$~// Информатика и~её применения, 2016. Т.~10. Вып.~2. С.~70--79.
\bibitem{12-ag}
\Au{Агаларов Я.\,М.} Максимизация среднего стационарного дохода СМО типа 
$M/G/1$~// Информатика и~её применения, 2017. Т.~11. Вып.~2. С.~25--32. 
\bibitem{13-ag}
\Au{Агаларов Я.\,М., Шоргин~В.\,С.} Об одной задаче максимизации дохода 
системы массового обслуживания типа $G/M/1$ с~пороговым управлением 
очередью~// Информатика и~её применения, 2017. Т.~11. Вып.~4. С.~55--64.
\bibitem{14-ag}
\Au{Бочаров П.\,П., Печинкин~А.\,В.} Теория массового обслуживания.~--- 
М.: РУДН, 1995. 529~с.
 \end{thebibliography}

 }
 }

\end{multicols}

\vspace*{-3pt}

\hfill{\small\textit{Поступила в~редакцию 10.06.18}}

\vspace*{8pt}

%\pagebreak

%\newpage

%\vspace*{-28pt}

\hrule

\vspace*{2pt}

\hrule

%\vspace*{-2pt}

\def\tit{OPTIMIZATION OF~BUFFER MEMORY SIZE OF~SWITCHING NODE 
IN~MODE OF~FULL MEMORY SHARING}

\def\titkol{Optimization of~buffer memory size of~switching node 
in~mode of~full memory sharing}

\def\aut{Yа.\,M.~Agalarov}

\def\autkol{Yа.\,M.~Agalarov}

\titel{\tit}{\aut}{\autkol}{\titkol}

\vspace*{-11pt}


\noindent
Institute of Informatics Problems, Federal Research Center ``Computer Sciences and 
Control'' of the Russian Academy of Sciences; 44-2~Vavilov Str., Moscow 119133, 
Russian Federation


\def\leftfootline{\small{\textbf{\thepage}
\hfill INFORMATIKA I EE PRIMENENIYA~--- INFORMATICS AND
APPLICATIONS\ \ \ 2018\ \ \ volume~12\ \ \ issue\ 4}
}%
 \def\rightfootline{\small{INFORMATIKA I EE PRIMENENIYA~---
INFORMATICS AND APPLICATIONS\ \ \ 2018\ \ \ volume~12\ \ \ issue\ 4
\hfill \textbf{\thepage}}}

\vspace*{6pt}
  
\Abste{The problem of optimizing the volume of the buffer memory of the network 
switching node with the CP (Complete Partitioning)
is considered, which uses a~scheme for the complete 
memory sharing between the output transmission lines, taking into account packet 
loss due to memory overflow and average packet delays. The
properties of the 
solution of the problem are obtained, and an algorithm for searching for the optimal 
volume is developed for the switching node model with Poisson incoming flows, 
common service time distributions, and single-channel output transmission lines. 
Within the given model, an algorithm for finding the optimal buffer\linebreak\vspace*{-12pt}}

\Abstend{memory 
allocation plan for a~given volume is developed. The results of computational 
experiments using the developed algorithms are presented. The applicability of the 
proposed algorithms and approaches is obtained in the framework of the switching 
node model with recurrent incoming flows, exponential service times, and 
single-channel transmission lines.}

\KWE{switching node; flow management; buffer memory allocation; optimal buffer 
memory capacity}
  
  
\DOI{10.14357/19922264180404}

%\vspace*{-14pt}

\Ack
\noindent
The work was partly supported by
the Russian Foundation for Basic Research (projects 
18-07-00692, 17-07-000717,  
and 17-07-00851).



%\vspace*{6pt}

  \begin{multicols}{2}

\renewcommand{\bibname}{\protect\rmfamily References}
%\renewcommand{\bibname}{\large\protect\rm References}

{\small\frenchspacing
 {%\baselineskip=10.8pt
 \addcontentsline{toc}{section}{References}
 \begin{thebibliography}{99}

\bibitem{1-ag-1}
\Aue{Kleinrock, L.} 1976. \textit{Queueing systems. Vol.~II: Computer applications}. 
New York, NY: Wiley. 549~p. 
\bibitem{2-ag-1}
\Aue{Davies, D.\,W., D.\,L.\,A.~Barber, W.\,L.~Price, and C.\,M.~Solomonides.} 1979. 
\textit{Computer networks and their protocols}. New York, NY: Wiley. 487~p.
\bibitem{3-ag-1}
\Aue{Irland, M.} 1978. Buffer management in a~packet switch. 
\textit{IEEE T.~Commun.} 26(3):328--337.
\bibitem{4-ag-1}
\Aue{Kamoun, F., and L.~Kleinrock.} 1980. Analysis of shared finite storage in 
a~computer networks node environment under general traffic conditions. \textit{IEEE 
Trans. Commun.} 28(7):992--1003.
\bibitem{5-ag-1}
\Aue{Basharin, G.\,P., and K.\,E.~Samuilov.} 1982. \textit{Ob optimal'noy strukture bufernoy 
pamyati v~setyakh peredachi dannykh s~kommutatsiey paketov} [On the optimal 
structure of BP in data transmission networks with packet commutation].~--- 
Moscow. 70~p. 
\bibitem{6-ag-1}
\Aue{Mikheev, P.\,A.} 2014. Analyzing sharing strategies for finite buffer memory in 
a~router among outgoing channels. \textit{Automat. Rem. Contr.} 
75(10):1814--1825.
\bibitem{7-ag-1}
\Aue{Forschini, G., and B.~Gopinath.} 1983. Sharing memory optimally. 
\textit{IEEE T.~Commun.} 31(3):~352--359.
\bibitem{8-ag-1}
\Aue{Agalarov, Ya.\,M.} 2003. Dvukhstupenchatoe adaptivnoe upravlenie potokami 
v~uzle kommutatsii te\-le\-kom\-mu\-ni\-ka\-tsi\-on\-noy seti [Two-level flow control for 
a~switching node of communication network]. \textit{Vestnik RUDN. Ser. Prikladnaya 
matematika i~informatika} [PFUR Bull., Applied Mathematics and Computer 
Science] 1:134--141.
\bibitem{9-ag-1}
\Aue{Linets, G.\,I.} 2010. Upravlenie ob''emom bufernoy pa\-mya\-ti i~propusknoy 
sposobnost'yu kanalov v~mul'tiservisnykh setyakh [Volume management of 
buffer memory and throughput of channels in multiservice networks]. 
\textit{Infokommunikatsionnye tekhnologii} [Information  
Communication Technologies] 6(2):62--64. 
\bibitem{10-ag-1}
\Aue{Zhernovyy, Ju.\,V.} 2010. Reshenie zadach optimal'nogo sinteza dlya 
nekotorykh markovskikh modeley obsluzhivaniya [Solution of optimum 
synthesis problem for some Markov models of service]. \textit{Informatsionnye 
protsessy} [Information Processes] 10(3):257--274.
\bibitem{11-ag-1}
\Aue{Agalarov, Ya.\.M., M.\,Ya.~Agalarov, and V.\,S.~Shorgin.} 2016. Ob optimal'nom 
porogovom znachenii  dliny ocheredi v~od\-noy zadache maksimizatsii dokhoda 
SMO tipa $M/G/1$ [About the optimal threshold of queue length in particular 
problem of profit maximization in $M/G/1$ queueing system]. 
\textit{Informatika i~ee 
Primeneniya~--- Inform. Appl.} 10(2):70--79.
\bibitem{12-ag-1}
\Aue{Agalarov, Ya.\,M.} 2017. Maksimizatsiya srednego sta\-tsi\-o\-nar\-no\-go dokhoda 
SMO tipa  $M/G/1$ [Maximization of average stationary profit in $M/G/1$ queuing 
system]. \textit{Informatika i~ee Primeneniya~--- Inform. Appl.} 
11(2):25--32. 
\bibitem{13-ag-1}
\Aue{Agalarov, Ya.\,M., and V.\,S.~Shorgin.} 2016. Ob odnoy zadache maksimizatsii 
dokhoda sistemy massovogo obsluzhivaniya tipa $G/M/1$ s~porogovym 
upravleniem ochered'yu [Profit maximization in $G/M/1$ queuing system on a~set 
of threshold strategies with two switch points]. \textit{Informatika i~ee Primeneniya~--- 
Inform. Appl.} 11(4):\linebreak 55--64.
{ %\looseness=1

}
\bibitem{14-ag-1}
\Aue{Bocharov, P.\,P., and A.\,V.~Pechinkin.} 1995. \textit{Teoriya massovogo obsluzhivaniya} 
[Queueing theory]. Moscow: RUDN. 529~p.
\end{thebibliography}

 }
 }

\end{multicols}

\vspace*{-6pt}

\hfill{\small\textit{Received June 10, 2018}}

%\pagebreak

%\vspace*{-18pt}
  
  \Contrl
  
  \noindent
\textbf{Agalarov Yaver M.} (b.\ 1952)~--- Candidate of Science (PhD) in technology, associate 
professor; leading scientist, Institute of Informatics Problems, Federal Research Center ``Computer 
Science and Control'' of the Russian Academy of Sciences, 44-2~Vavilov Str., Moscow 119333, 
Russian Federation; \mbox{agglar@yandex.ru}
 
\label{end\stat}

\renewcommand{\bibname}{\protect\rm Литература}       