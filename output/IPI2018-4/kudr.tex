\def\stat{kudr}

\def\tit{ГАММА-ВЕЙБУЛЛОВСКИЙ СЛУЧАЙ АПРИОРНЫХ РАСПРЕДЕЛЕНИЙ 
В~БАЙЕСОВСКИХ МОДЕЛЯХ\\ МАССОВОГО ОБСЛУЖИВАНИЯ$^*$}

\def\titkol{Гамма-вейбулловский случай априорных распределений 
в~байесовских моделях массового обслуживания}

\def\aut{Е.\,Н.~Арутюнов$^1$, А.\,А.~Кудрявцев$^2$, А.\,И.~Титова$^3$}

\def\autkol{Е.\,Н.~Арутюнов, А.\,А.~Кудрявцев, А.\,И.~Титова}

\titel{\tit}{\aut}{\autkol}{\titkol}

\index{Арутюнов Е.\,Н.}
\index{Кудрявцев А.\,А.}
\index{Титова А.\,И.}
\index{Arutyunov E.\,N.}
\index{Kudryavtsev A.\,A.}
\index{Titova A.\,I.}




{\renewcommand{\thefootnote}{\fnsymbol{footnote}} \footnotetext[1]
{Работа выполнена при частичной финансовой поддержке РФФИ (проект 17-07-00577).}}


\renewcommand{\thefootnote}{\arabic{footnote}}
\footnotetext[1]{Институт проблем информатики Федерального исследовательского центра <<Информатика 
и~управление>> Российской академии наук,
\mbox{enapoleon@mail.ru}}
\footnotetext[2]{Факультет вычислительной математики и~кибернетики, 
Московский государственный университет им.~М.\,В.~Ломоносова, \mbox{nubigena@mail.ru}}
\footnotetext[3]{Факультет вычислительной математики и~кибернетики, 
Московский государственный университет им.~М.\,В.~Ломоносова, 
 \mbox{onkelskroot@gmail.com}}

%\vspace*{8pt}


\Abst{Статья посвящена байесовскому подходу к~задачам теории массового обслуживания 
и~надежности. В~байесовских моделях для классических постановок задач предполагается, 
что основные параметры системы, например интенсивности входящего потока и~обслуживания, 
являются случайными величинами с~известными априорными распределениями. 
Байесовский подход эффективен при изучении больших совокупностей однотипных 
систем или одной системы, характеристики которой меняются в~моменты времени, 
неизвестные исследователю. Рандомизация основных параметров системы приводит к~тому, 
что различные ее характеристики, такие как коэффициент загрузки, также становятся 
случайными. Приводятся аналитические результаты для вероятностных характеристик 
коэффициента загрузки, выражаемые в~терминах гам\-ма-экс\-по\-нен\-ци\-аль\-ной функции, 
в~случае, когда в~качестве пары априорных распределений параметров системы $M|M|1|0$ 
рассматриваются гам\-ма-рас\-пре\-де\-ле\-ние и~распределение Вейбулла.}

\KW{байесовский подход; системы массового обслуживания; смешанные
распределения; распределение Вейбулла; гам\-ма-рас\-пре\-де\-ле\-ние; 
гам\-ма-экс\-по\-нен\-ци\-аль\-ная функция}

\DOI{10.14357/19922264180413}
  
\vspace*{10pt}


\vskip 10pt plus 9pt minus 6pt

\thispagestyle{headings}

\begin{multicols}{2}

\label{st\stat}

\section{Введение}


При изучении и~математическом моделировании функционирования различных объектов 
среди параметров моделируемого объекта можно выделить <<способствующие>> 
и~<<препятствующие>> функционированию факторы. При применении такого 
разделения в~теории массового обслуживания\linebreak
 к~группе параметров, 
<<способствующих>> функционированию, можно отнести интенсивность обслуживания 
запросов, а~к~параметрам, <<пре\-пят\-ст\-ву\-ющим>> функционированию,~--- 
интенсивность входяще\-го потока требований. 

Зачастую в~силу структурной сложности 
моделируемых систем и~стохастичности среды функ\-ционирования определение точного 
значения характеристик каждого компонента системы или\linebreak
 со\-во\-куп\-ности сис\-тем 
представляется ресурсоемкой задачей. В~таких ситуациях целесообразно воспользоваться 
байесовским подходом. 
%
При байесовском подходе исходные характеристики системы 
считаются заданными в~определенном смысле неточно, однако предполагается, что 
имеется информация об их априорном распределении. За счет рандомизации исходных 
параметров системы происходит и~рандомизация зависящих от них показателей, которые, 
как правило, пред\-став\-ля\-ют больший интерес для исследования функционирования системы.

В данной работе рассматривается система массового обслуживания $M|M|1|0$, 
одним из основных показателей которой является коэффициент ее загрузки~$\rho$. 
Значение коэффициента загрузки определяется как отношение параметра входящего 
потока~$\lambda$ к~параметру обслуживания~$\mu$. Подробное описание предпосылок 
для исследования, особенностей и~библиографии байесовских моделей в~теории 
массового обслуживания и~надежности можно найти в~книге~\cite{KuSh2015}.

Ниже для модели $M|M|1|0$ приводятся вероятностные характеристики коэффициента 
загрузки~$\rho$ в~случае, когда в~качестве пары априорных распределений параметров 
системы~$\lambda$ и~$\mu$ рас\-смат\-ри\-ва\-ют\-ся гам\-ма-рас\-пре\-де\-ле\-ние 
и~распределение Вей\-булла.
{ %\looseness=1

}

\section{Основные результаты}

Обозначим через $G(q, \theta)$ гам\-ма-рас\-пре\-де\-ле\-ние с~плот\-ностью
$$
g_{q, \theta}(x) = \fr{\theta^q x^{q-1}e^{-\theta x}}{{\Gamma(q)}}\,, \enskip
 x>0\,,\enskip q>0\,,\enskip \theta>0\,,
 $$
а через $W(p,\alpha)$~--- распределение Вейбулла с~плот\-ностью
$$
w_{p,\alpha}(x) = \fr{px^{p-1}e^{-({x/\alpha})^{p}}}{{\alpha^{p}}}\,,  \enskip
 x>0\,,\enskip  p>0\,,\enskip \alpha>0\,.
 $$
Рассмотрим гамма-экспоненциальную функцию~\cite{KuTi2017}:

\noindent
$$
{\sf Ge}_{\alpha, \beta} (x) = \sum\limits_{k=0}^{\infty}
\fr{x^k}{k!}\, \Gamma(\alpha k + \beta)\,,\enskip \!
x\in\mathbb{R}\,, \enskip\! \alpha\ge0\,, \enskip\! \beta> 0\,.
$$

Хорошо известен следующий результат.

\smallskip

\noindent
\textbf{Лемма~1.}\
\textit{Для случайных величин $\xi$ и~$\eta$, имеющих 
гам\-ма-рас\-пре\-де\-ле\-ние~$G(q, \theta)$ и~распределение Вейбулла~$W(p,\alpha)$
 соответственно, для $z\hm\in\mathbb{R}$ выполняются следующие соотношения}:
\begin{alignat*}{2}
{\sf E}\,\xi^z &= \fr{\Gamma(q+z)}{\theta^z\Gamma(q)}\,, &\enskip z &> -q\,; \\
{\sf E}\,\eta^z &= \alpha^z \Gamma\left(1+\fr{z}{p}\right)\,,&\enskip
 z &> -p\,.
 \end{alignat*}


\smallskip

Для дальнейших вычислений потребуется следующее утверждение.

\smallskip

\noindent
\textbf{Лемма~2.}\
\textit{Для некоторых $q\hm>0$, $p\hm>0$, $a\hm>0$ и~$b\hm>0$ справедливо}:

\noindent
\begin{multline*}
\int\limits_0^{\infty}y^{q+p-1}e^{-ay}e^{-(y/b)^p} \, dy ={}\\[-4pt]
{}=
 \begin{cases}
   {b^{q+p}{p^{-1}}}{\sf Ge}_{1/p,\, q/p+1} (-ab)\,, &p > 1\,;\\
   {{a^{-q-p}}}{\sf Ge}_{p,\, q+p} \left(-{{(ab)^{-p}}}\right)\,, &p < 1\,;\\[5pt]
   {{\left(a+\fr{1}{b}\right)^{-q-1}}\Gamma(q+1)}\,, &p = 1\,.
 \end{cases}
\end{multline*}


\noindent
Д\,о\,к\,а\,з\,а\,т\,е\,л\,ь\,с\,т\,в\,о\,.\ \ 
Рассмотрим случай $p\hm>1$. Используя теорему Лебега о~предельном переходе, получаем:

\noindent
\begin{multline*}
\int\limits_0^{\infty}y^{q+p-1}e^{-ay}e^{-(y/b)^p} \,dy = {}\\[-5pt]
{}=
\int\limits_0^{\infty} y^{q+p-1}\sum\limits_{k=0}^\infty \fr{(-a)^k y^k}{k!}\,
 e^{-(y/b)^p} \, dy ={}\\
{}=\sum\limits_{k=0}^\infty 
\fr{(-ab)^k }{k!}\int\limits_0^{\infty} t^{k/p+q/p} e^{-t} \,dt= {}\\
{}=
\fr{b^{q+p}}{p}\sum\limits_{k=0}^\infty \fr{(-ab)^k}{k!}\, \Gamma\left(\fr{k}{p}+\fr{q}{p}+1\right)\,.
\end{multline*}
Случай $p<1$ рассматривается аналогично. Случай $p\hm=1$ напрямую следует 
из определения гам\-ма-функ\-ции.
Лемма доказана.

\smallskip

Леммы~1 и~2 дают возможность находить вероятностные характеристики коэффициента 
загрузки системы.

\smallskip

\noindent
\textbf{Теорема~1.}\
\textit{Пусть параметр входящего потока~$\lambda$ имеет гам\-ма-рас\-пре\-де\-ле\-ние 
$G(q, \theta)$, а параметр обслуживания~$\mu$ имеет распределение 
Вейбулла $W(p,\alpha)$, причем~$\lambda$ и~$\mu$ независимы. Тогда при $x\hm>0$ 
плотность, 
функция распределения и~моменты коэффициента загрузки $\rho\hm=\lambda/\mu$ имеют вид}:
\begin{align*}%\label{f_rho_GW}
f_{\rho}(x) &=
 \begin{cases}
  \fr{(\alpha \theta)^q x^{q-1}}{\Gamma(q)}\,
  {\sf Ge}_{1/p,\, q/p+1} (-\alpha \theta x), &\!\!\!p > 1;\\
   \fr{p}{x(\alpha \theta x)^{p}\Gamma(q)}\,{\sf Ge}_{p,\, q+p} 
   \left(-(\alpha \theta x)^{-p}\right), &\!\!\!p < 1;
 \end{cases}
\\
F_\rho(x) &=
 \begin{cases}
   \fr{(\alpha \theta x)^q }{p\Gamma(q)}\, {\sf Ge}_{1/p,\, q/p} 
   (-\theta \alpha x)\,, &p > 1\,;\\
  \fr{1}{\Gamma(q)}\int\limits_{(\alpha \theta x)^{-p}}^{\infty} 
  {\sf Ge}_{p,\, p+q}(-z) \,dz\,, &p < 1\,;
 \end{cases}
\\
{\sf E}\,\rho^z &= \fr{\Gamma(q+z)}{(\alpha\theta)^z\Gamma(q)}\,
 \Gamma\left(1-\fr{z}{p}\right)\,, \enskip z<p\,.
 \end{align*}
\textit{При $p=1$ распределение коэффициента загрузки~$\rho$ 
совпадает с~распределением Дагума}~\cite{Dagum1977} \textit{с~параметрами} 
$(1,(\alpha\theta)^{-1},q)$.


\smallskip

\noindent
Д\,о\,к\,а\,з\,а\,т\,е\,л\,ь\,с\,т\,в\,о\,.\ \ Поскольку
$$
f_\rho(x) = \int\limits_0^{\infty} \fr{p \theta^q x^{q-1} y^{p+q-1}}
{\alpha^p \Gamma(q)}\,e^{-\theta x y}e^{- (y/\alpha)^p} \, dy\,,
$$
вид плотности $f_\rho(x)$ вытекает из леммы~2 для всех $p\hm>0$.

Для функции распределения $\rho$ при $p\hm>1$ справедливо для $x\hm>0$:
\begin{multline*}
F_\rho(x) = \int\limits_{0}^{x} \fr{(\alpha \theta)^q u^{q-1}}{{\Gamma(q)}} 
\,{\sf Ge}_{1/p,\, q/p+1} (-\theta \alpha u) \, du ={}\\
{}= \fr{(\alpha \theta)^q }{{\Gamma(q)}} \sum\limits_{k=0}^\infty 
\fr{(-\alpha \theta )^k}{k!}\, \Gamma\left(\fr{k}{p} + \fr{q}{p} + 1\right) 
\!\int\limits_{0}^{x}\!u^{q+k-1}\,du ={}\hspace*{-4.39256pt}\\
{}= \fr{(\alpha \theta x)^q }{p\Gamma(q)} 
\sum\limits_{k=0}^\infty \fr{(-\alpha \theta x)^k}{k!}\, \Gamma\left(\fr{k+q}{p}\right)\,.
\end{multline*}
В случае $p\hm<1$ имеем:
\begin{multline*}
F_\rho(x) = \int\limits_{0}^{x}\fr{p{\sf Ge}_{p,\, q+p} 
(-(\alpha \theta u)^{-p})}{u(\alpha \theta u)^{p}\Gamma(q)} \, du={}\\
{} = \int\limits_{0}^{x} \sum\limits_{k=0}^\infty 
\fr{p(-(\alpha \theta u)^{-p})^k\Gamma(kp + q + p)}
{{(\alpha \theta)^{p}\Gamma(q)k!u^{p+1}}}  \, du={}\\
{}=\fr{1}{{\Gamma(q)}}\int\limits_{(\alpha \theta x)^{-p}}^{\infty} 
 \sum\limits_{k=0}^\infty \fr{(-z)^k}{k!}\, \Gamma(kp + q + p) \, dz\,.
 \end{multline*}

Для нахождения моментов~$\rho$ достаточно воспользоваться 
независимостью случайных величин~$\lambda$ и~$\mu$ и~леммой~1.
Теорема доказана.

\smallskip

Рассмотрим симметричный случай априорных распределений параметров входящего потока 
и~обслуживания.

\smallskip

\noindent
\textbf{Теорема~2.}\
\textit{Пусть параметр входящего потока~$\lambda$ имеет распределение 
Вейбулла $W(q,\theta)$, а~параметр обслуживания~$\mu$ имеет 
гам\-ма-рас\-пре\-де\-ле\-ние $G(p, \alpha)$, причем~$\lambda$ и~$\mu$ независимы. Тогда при $x\hm>0$ 
плотность, функция распределения и~моменты коэффициента загрузки $\rho\hm=\lambda/\mu$ 
имеют вид}:
\begin{align*}
f_\rho(x) &=
 \begin{cases}
   \fr{(\alpha \theta)^p}{{ x^{p+1}\Gamma(p)}}\,
   {\sf Ge}_{1/q,\, p/q+1} \left(-\fr{\alpha \theta}{x}\right)\,, &q > 1\,;\\[12pt]
  \fr{qx^{q-1}}{{(\alpha \theta)^{q}\Gamma(p)}}\,
  {\sf Ge}_{q,\, p+q} \left(-\left(\fr{x}{\alpha \theta}\right)^{q}\right)\,, &q < 1\,;
 \end{cases}
\\
F_\rho(x) &=
 \begin{cases}
   \fr{1}{{p\Gamma(p)}}\int\limits_{(x/(\alpha \theta))^{-p}}^{\infty} 
   {\sf Ge}_{1/q,\, p/q+1} (-z^{1/p}) \, dz\,, &\\
   &\hspace*{-12mm}q > 1\,;\\
   1-\fr{1}{{\Gamma(p)}}\,{\sf Ge}_{q,\, p}
   \left(-\left(\fr{x}{\theta \alpha}\right)^{q}\right)\,, &\hspace*{-12mm}q < 1\,;
 \end{cases}
\\
{\sf E}\,\rho^z &= (\alpha \theta)^z \fr{\Gamma(1+z/q) 
\Gamma(p-z)}{\Gamma(p)}\,, \enskip z<p\,.
\end{align*}
\textit{При $q=1$ распределение коэффициента загрузки~$\rho$ совпадает с~распределением 
Ломакса}~\cite{Lomax1954} \textit{с~па\-ра\-мет\-ра\-ми~$\alpha\theta$ и~$p$}.


\smallskip

\noindent
Д\,о\,к\,а\,з\,а\,т\,е\,л\,ь\,с\,т\,в\,о\,.\ \
 Аналогично предыдущей тео\-ре\-ме для получения выражения для плот\-ности~$\rho$ при всех 
 $q\hm >0$ достаточно воспользоваться леммой~2.

Найдем функцию распределения в~случае $q\hm>1$. Имеем:
\begin{multline*}
\!F_\rho(x) = \fr{(\alpha \theta)^p}{\Gamma(p)}\int\limits_{0}^{x}u^{-p-1}
{\sf Ge}_{1/q,\, p/q+1} \left(-\fr{\alpha \theta}{u}\right) \, du ={}\hspace*{-6.61475pt}
\end{multline*}

\noindent
\begin{multline*}
\hspace*{-9pt}\!\!{}=\fr{(\alpha \theta)^p}{\Gamma(p)}\!\int\limits_{0}^{x}\!u^{-p-1}
\sum\limits_{k=0}^\infty \fr{(-\alpha \theta/ u)^k }{k!}\, \Gamma
\left(\!\fr{k}{q}+\fr{p}{q}+1\!\right) \, du ={}\\
 {}= \fr{1}{ p\Gamma(p)} \int\limits_{(x/(\alpha \theta))^{-p}}^{\infty}
 \sum\limits_{k=0}^\infty \fr{(-z^{1/p})^k}{k!} \,\Gamma
 \left(\fr{k}{q}+\fr{p}{q}+1\right) \, dz.\hspace*{-4.34023pt}
 \end{multline*}
Для $q<1$ функция распределения~$\rho$ имеет вид:
\begin{multline*}
F_\rho(x) = \int\limits_{0}^{x}\fr{qu^{q-1}}{(\alpha \theta)^{q}\Gamma(p)}\,
{\sf Ge}_{q,\, p+q} \left(-\left(
\fr{u}{\alpha \theta}\right)^{q}\right) \,du ={}\\
{}=\int\limits_{0}^{x}\fr{qu^{q-1}}{(\alpha \theta)^{q}\Gamma(p)}
\sum\limits_{k=0}^\infty \fr{(-(u/(\theta \alpha))^q)^k}{k!}\, \Gamma(qk+q+p) \,du ={}\\
{}=-\fr{1}{\Gamma(p)}\sum\limits_{k=0}^\infty 
\fr{\left(-\left(x/(\theta \alpha\right)\right)^{q})^{k+1} }{(k+1)!}\, \Gamma(q(k+1)+p)={}\\
{}=1-\fr{1}{\Gamma(p)}\sum\limits_{m=0}^\infty 
\fr{\left(-\left(x/(\theta \alpha\right)\right)^{q})^m}{m!}\, \Gamma(qm+p)\,.
\end{multline*}

Соотношение для моментов следует из леммы~1 и~независимости случайных величин~$\lambda$ 
и~$\mu$. Теорема доказана.



\section{Заключение}

Результаты статьи в~очередной раз свидетельствуют о~том, 
что гам\-ма-экс\-по\-нен\-ци\-аль\-ная функция 
${\sf Ge}_{\alpha, \beta} (x)$ является удобным инструментом иссле\-до\-ва\-ния 
различных характеристик вероятностных смесей законов гам\-ма-типа.


{\small\frenchspacing
 {%\baselineskip=10.8pt
 \addcontentsline{toc}{section}{References}
 \begin{thebibliography}{9}

\bibitem{KuSh2015}
\Au{Кудрявцев~А.\,А., Шоргин~С.\,Я.}
Байесовские модели в~тео\-рии массового обслуживания и~надежности.~--- 
М.: ФИЦ ИУ РАН, 2015. 76~с.

\bibitem{KuTi2017}
\Au{Кудрявцев~А.\,А., Титова~А.\,И.}
Гам\-ма-экс\-по\-нен\-ци\-аль\-ная функция в~байесовских моделях массового обслуживания~// 
Информатика и~её применения, 2017. Т.~11. Вып.~4. С.~104--108.

\bibitem{Dagum1977}
\Au{Dagum~C.}
A~new model of personal income-distribution-specification and estimation~// 
Econ. Appl., 1977. Vol.~30. No.\,3. P.~413--437.

\bibitem{Lomax1954}
\textit{Lomax~K.\,S.\/}
Business failures: Another example of the analysis of failure data~// 
J.~Am. Stat. Assoc., 1954. Vol.~49. No.\,268. P.~847--852.


 \end{thebibliography}

 }
 }

\end{multicols}

\vspace*{-3pt}

\hfill{\small\textit{Поступила в~редакцию 19.08.18}}

%\vspace*{8pt}

%\pagebreak

\newpage

\vspace*{-28pt}

%\hrule

%\vspace*{2pt}

%\hrule

%\vspace*{-2pt}

\def\tit{GAMMA-WEIBULL \textit{A~PRIORI} DISTRIBUTIONS\\ IN~BAYESIAN QUEUING MODELS}

\def\titkol{Gamma-Weibull \textit{a~priori} distributions in~Bayesian queuing models}

\def\aut{E.\,N.~Arutyunov$^1$, A.\,A.~Kudryavtsev$^2$, and~A.\,I.~Titova$^2$}

\def\autkol{E.\,N.~Arutyunov, A.\,A.~Kudryavtsev, and~A.\,I.~Titova}

\titel{\tit}{\aut}{\autkol}{\titkol}

\vspace*{-11pt}


\noindent
$^1$Institute of Informatics Problems, Federal Research Center 
``Computer Science and Control'' of the Russian\linebreak
$\hphantom{^1}$Academy of Sciences, 
44-2~Vavilov Str., Moscow 119333, Russian Federation


\noindent
$^2$Faculty of Computational 
Mathematics and Cybernetics, M.\,V.~Lomonosov Moscow State University, 
1-52~Lenin-\linebreak
$\hphantom{^1}$skiye Gory, GSP-1, Moscow 119991, Russian Federation


\def\leftfootline{\small{\textbf{\thepage}
\hfill INFORMATIKA I EE PRIMENENIYA~--- INFORMATICS AND
APPLICATIONS\ \ \ 2018\ \ \ volume~12\ \ \ issue\ 4}
}%
 \def\rightfootline{\small{INFORMATIKA I EE PRIMENENIYA~---
INFORMATICS AND APPLICATIONS\ \ \ 2018\ \ \ volume~12\ \ \ issue\ 4
\hfill \textbf{\thepage}}}

\vspace*{6pt}



\Abste{This paper considers the Bayesian approach to the queueing 
theory and reliability theory. Within the framework of the Bayesian 
approach, it is assumed that the key parameters of classical systems, 
such as the input flow intensity and the service intensity, are random 
variables with known \textit{a~priori} distributions. It is reasonable to use the 
Bayesian approach in the studying of the systems which are of the same type 
as well as in the studying of one system where changes of its characteristics 
happen at unpredictable moments of time. The randomization of the system's 
key parameters leads to the randomization of the system's characteristics, 
for instance, the traffic intensity. In the paper, the analytical results 
for the traffic intensity probability characteristics in the case of the 
gamma and Weibull \textit{a~priori} distributions of $M/M/1/0$ system's 
parameters are presented. 
The obtained results are formulated using the gamma-exponential function.}

\KWE{Bayesian approach; queuing system; mixed distribution; Weibull distribution; 
gamma distribution; gamma-exponential function}




\DOI{10.14357/19922264180413}

%\vspace*{-14pt}

\Ack
\noindent
The work was partly supported by the Russian Foundation for Basic Research 
(project 17-07-00577).



%\vspace*{6pt}

  \begin{multicols}{2}

\renewcommand{\bibname}{\protect\rmfamily References}
%\renewcommand{\bibname}{\large\protect\rm References}

{\small\frenchspacing
 {%\baselineskip=10.8pt
 \addcontentsline{toc}{section}{References}
 \begin{thebibliography}{9}

\bibitem{1-kudr-1}
\Aue{Kudryavtsev, A.\,A., and S.\,Ya.~Shorgin.} 
2015. \textit{Bayesovskie modeli v~teorii massovogo obsluzhivaniya i~nadezhnosti} 
[Bayesian models in mass service and reliability theories]. Moscow: 
FRC CSC RAS. 76~p.

\bibitem{2-kudr-1}
\Aue{Kudryavtsev, A.\,A., and A.\,I.~Titova.} 2017. 
Gamma-eksponentsial'naya funktsiya v~bayesovskikh modelyakh massovogo obsluzhivaniya 
[Gamma-exponential function in Bayesian queuing models]. 
\textit{Informatika i~ee Primeneniya~--- Inform. Appl.} 11(4):104--108.

\bibitem{3-kudr-1}
\Aue{Dagum, C.} 1977. 
A~new model of personal income-distribution-specification and estimation. 
\textit{Econ. Appl.} 30(3):413--437.

\bibitem{4-kudr-1}
\Aue{Lomax, K.\,S.} 1954. Business failures: Another 
example of the analysis of failure data. 
\textit{J.~Am. Stat. Assoc.} 49(268):847--852.
\end{thebibliography}

 }
 }

\end{multicols}

\vspace*{-6pt}

\hfill{\small\textit{Received August 19, 2018}}

%\pagebreak

%\vspace*{-18pt}


\Contr

\noindent
\textbf{Arutyunov Evgeny N.} (b.\ 1952)~--- 
Candidate of Science (PhD) in physics and mathematics, senior scientist, 
Institute of Informatics Problems, Federal Research Center ``Computer 
Science and Control'' of the Russian Academy of Sciences, 44-2~Vavilov Str., 
Moscow 119333, Russian Federation; \mbox{enapoleon@mail.ru}

\vspace*{3pt}

\noindent
\textbf{Kudryavtsev Alexey A.} (b.\ 1978)~--- 
Candidate of Science (PhD) in physics and mathematics, associate professor, 
Department of Mathematical Statistics, Faculty of Computational Mathematics 
and Cybernetics, M.\,V.~Lomonosov Moscow State University, 1-52~Leninskiye Gory, 
GSP-1, Moscow 119991, Russian Federation; \mbox{nubigena@mail.ru}

\vspace*{3pt}

\noindent
\textbf{Titova Anastasiia I.} (b.\ 1995)~--- 
student, Department of Mathematical Statistics, Faculty of Computational Mathematics 
and Cybernetics, M.\,V.~Lomonosov Moscow State University, 1-52~Leninskiye Gory, GSP-1, 
Moscow 119991, Russian Federation; \mbox{onkelskroot@gmail.com}

\label{end\stat}

\renewcommand{\bibname}{\protect\rm Литература}       