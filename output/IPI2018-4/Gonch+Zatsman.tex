\def\stat{gonch+zatsman}

\def\tit{КОЛИЧЕСТВЕННЫЙ АНАЛИЗ РЕЗУЛЬТАТОВ\\ МАШИННОГО ПЕРЕВОДА 
С~ИСПОЛЬЗОВАНИЕМ\\ НАДКОРПУСНЫХ БАЗ ДАННЫХ$^*$\\[-5pt]}


\def\titkol{Количественный анализ результатов машинного 
перевода с~использованием надкорпусных баз данных}

\def\aut{Н.\,В.~Бунтман$^1$, А.\,А.~Гончаров$^2$, И.\,М.~Зацман$^3$, 
В.\,А.~Нуриев$^4$\\[-5pt]}

\def\autkol{Н.\,В.~Бунтман, А.\,А.~Гончаров, И.\,М.~Зацман, 
В.\,А.~Нуриев}

\titel{\tit}{\aut}{\autkol}{\titkol}

\index{Бунтман Н.\,В.}
\index{Гончаров А.\,А.}
\index{Зацман И.\,М.} 
\index{Нуриев В.\,А.}
\index{Buntman N.\,V.}
\index{Goncharov A.\,A.}
\index{Zatsman I.\,M.}
\index{Nuriev V.\,A.}




{\renewcommand{\thefootnote}{\fnsymbol{footnote}} \footnotetext[1]
{Работа выполнена в~Институте проблем информатики Федерального исследовательского 
центра <<Информатика и~управ\-ле\-ние>> Российской академии наук при поддержке 
РФФИ (проект №\,16-24-41002).}}


\renewcommand{\thefootnote}{\arabic{footnote}}
\footnotetext[1]{Факультет 
иностранных языков и~регионоведения, Московский государственный университет им.\ М.\,В.~Ломоносова,
\mbox{nabunt@hotmail.com}}
\footnotetext[2]{Институт проблем информатики Федерального исследовательского центра <<Информатика 
  и~управ\-ле\-ние>> Российской академии наук, \mbox{a.gonch48@gmail.com}}
  \footnotetext[3]{Институт проблем информатики Федерального исследовательского центра <<Информатика 
  и~управ\-ле\-ние>> Российской академии наук, \mbox{izatsman@yandex.ru}}
  \footnotetext[4]{Институт проблем информатики Федерального исследовательского центра <<Информатика 
  и~управ\-ле\-ние>> Российской академии наук, \mbox{nurieff.v@gmail.com}}

\vspace*{-18pt}




\Abst{Рассматривается информационная технология, которая поддерживает экспертизу 
результатов машинного перевода литературных текстов. Технология разрабатывалась 
и~апробировалась на примерах переводов коннекторов при следующих условиях.  
Во-первых, объектом исследования выступают переводы предложений как 
с~однословными (например, \textit{хотя}, \textit{а}, \textit{кстати} и~т.\,д.), так 
и~с~неоднословными коннекторами (например, \textit{да еще}, \textit{но зато}, 
\textit{и~вообще}, \textit{и~притом}, \textit{хотя~и} и~т.\,д.). Во-вторых, между словами, 
входящими в~состав коннекторов, может быть фрагмент текста, например: \textit{если} 
(расстояние) \textit{так}, \textit{когда} (расстояние) \textit{то}, \textit{не только} 
(расстояние) \textit{но~и}, \textit{так как} (расстояние) \textit{то} и~т.\,д. Технология 
поддержки экспертизы результатов машинного перевода охватывает три основные стадии: 
(1)~лингвистическое аннотирование результатов машинного перевода коннекторов и~их 
контекстов с~использованием надкорпусных баз данных (НБД); (2)~количественная обработка 
результатов аннотирования; (3)~линг\-ви\-сти\-че\-ский анализ сформированных аннотаций 
и~полученных числовых данных. Статья посвящена описанию технологических аспектов 
поддержки экспертизы, относящихся к~ее первым двум стадиям. Экспериментальный 
материал включает примеры только с~неоднословными коннекторами, части которых 
могут располагаться как дистантно, так и~контактно.}

\KW{ надкорпусная база данных; машинный перевод; классификация ошибок; технология 
поддержки экспертизы; лингвистическое аннотирование; корпусная лингвистика; 
коннекторы}

\DOI{10.14357/19922264180414}
  
\vspace*{-6pt}


\vskip 10pt plus 9pt minus 6pt

\thispagestyle{headings}

\begin{multicols}{2}

\label{st\stat}

\section{Введение}

  В 2018~г.\ в~издательстве <<Шпрингер>> началась публикация новой серии 
книг, посвященных технологиям и~применению систем машинного 
перевода~[1, 2]. Первый том этой серии полностью посвящен анализу 
результатов машинного перевода и~включает обзор подходов к~классификации 
и~аннотированию его ошибок~\cite{3-gz}, что говорит об актуальности этой 
проблематики.
  
  Одна из задач рос\-сий\-ско-швей\-цар\-ско\-го проекта <<Контрастивное 
корпусное исследование коннекторов русского языка>>\footnote[5]{Проект 
поддержан РФФИ (грант №\,16-24-41002) и~Швейцарским национальным научным фондом 
(грант №\,IZLRZ1\_164059).} заключается в~том, чтобы создать классификацию 
и~систему аннотирования ошибок машинного перевода коннекторов русского 
языка, функция которых состоит в~обеспечении связности текста~\cite{4-gz}.
  
  В качестве основного инструмента для решения этой задачи использовалась 
НБД, предназначенная для лингвистического 
аннотирования языковых единиц разных категорий, включая  
коннекторы~\cite{5-gz, 6-gz, 7-gz, 8-gz}. В~качестве исходного информационного ресурса для 
классификации и~аннотирования ошибок использовались параллельные 
выровненные тексты~\cite{9-gz}.
  
  Для поддержки процесса создания классификации и~аннотирования ошибок 
  в~рамках проекта была разработана информационная технология, которая 
охватывает три основные стадии этого процесса: 
\begin{enumerate}[(1)]
\item лингвистическое 
аннотирование результатов машинного перевода коннекторов и~их контекстов 
с~использованием НБД;
\item количественная обработка результатов 
аннотирования;
\item лингвистический анализ сформированных аннотаций 
и~полученных числовых данных.
\end{enumerate}

\begin{table*}\small %tabl1
\begin{center}
\Caption{Пример результата машинного перевода}
\vspace*{2ex}

\begin{tabular}{|p{70mm}|p{70mm}|}
\hline
\textbf{Когда} бываю в~таком 
состоянии, \textbf{то} становлюсь 
нахальным и~наглым до крайности. &
\textbf{Quand} je suis dans cet $\acute{\mbox{e}}$tat, 
je deviens impudent et insolent 
$\grave{\mbox{a}}$~l'extr$\hat{\mbox{e}}$me.\newline 
({\sf translate.google.com}, дата обращения 05.02.2018 
19:48)\\
\hline
\multicolumn{2}{p{142mm}}{\footnotesize \hspace*{2mm}\textbf{Примечание.} С~момента записи аннотации 
изменился вариант перевода, предлагаемый системой GNMT: <<Quand je suis dans un tel $\acute{\mbox{e}}$tat, 
je deviens impudent et impudent $\grave{\mbox{a}}$~l'extr$\hat{\mbox{e}}$me>> (11.10.2018 19:35).}
\end{tabular}
\end{center}
\end{table*}

 Цель статьи~--- описать технологические 
аспекты поддержки экспертизы результатов машинного перевода 
с~использованием сис\-те\-мы Google ({\sf translate.google.com}), которые 
относятся к~первым двум стадиям.
  
  Система машинного перевода Google при\-ме\-нялась для перевода с~русского 
языка на фран\-цузский фрагментов художественного текста, состоящих из 
одного или более предложений. На\linebreak момент начала эксперимента в~2018~г.\ 
переводчик Google уже работал по принципу нейронной сети, т.\,е.\ 
пред\-став\-лял собой систему нейронного машинного перевода (Google's Neural 
Machine Translation system~--- GNMT). Его архитектура отличается от 
устройства автоматических переводчиков предыду\-ще\-го поколения~--- 
статистических машинных переводчиков. Система со\-сто\-ит из рекуррентной 
нейросети с~долгой крат\-ко\-сроч\-ной памятью (long short-term memory). В~этой 
сети задействованы два восьмислойных сегмента~--- ко\-ди\-ру\-ющий (encoder) 
и~де\-ко\-ди\-ру\-ющий (decoder) в~сочетании с~механизмом внимания (attention 
mechanism). Свойство рекуррентности поз\-во\-ля\-ет вы\-чис\-лять значение слова или 
словосочетания с~учетом предыду\-щих значений в~последовательности, 
с~которыми сеть уже имела дело. Единицей перевода в~такой системе является 
не словосочетание, как в~статистических машинных переводчиках, 
а~предложение. Система GNMT умеет работать с~фрагментами слов: слова 
в~исходном тексте и~переводе представляются в~форме набора составных 
элементов (wordpieces), что направлено на улучшение точности при обработке 
редких слов (подробнее о~системе GNMT см.~\cite{10-gz, 11-gz}).
  
  Источником используемых в~эксперименте текстов послужил 
  рус\-ско-фран\-цуз\-ский параллельный подкорпус Национального корпуса русского языка 
(НКРЯ)~\cite{12-gz}, где переводы на французский язык выполнены 
профессиональными переводчиками. Поэтому кроме аннотирования 
и~экспертизы результатов машинного перевода была возможность провести 
сопоставление <<человеческих>> (далее~--- референтных) переводов 
с~машинным, что заслуживает освещения в~отдельной пуб\-ли\-кации.
{\looseness=1

}
  
  Приведем краткое описание используемых информационных ресурсов. Из 
НКРЯ были получены тексты нескольких десятков книг на русском языке 
объемом около~1,7~млн словоупотреблений и~их переводы на французский 
язык объемом около~2,3~млн словоупотреблений (существенное превышение 
объема перевода связано с~тем, что одно произведение может иметь несколько 
переводов). Параллельные тексты были выровнены, т.\,е.\ одному или 
нескольким предложениям на русском языке (объект перевода) соответствует 
одно или несколько предложений на французском (результат перевода).
  
  Использованные технологические решения были во многом обусловлены 
следующими фак\-то\-рами. Во-пер\-вых, предметом исследования вы\-сту\-пают 
переводы предложений с~неоднословными коннекторами (например, \textit{да 
еще}, \textit{но зато}, \textit{и~вообще}, \textit{и~притом}, \textit{хотя~и} 
и~т.\,д.). Во-вто\-рых, между словами, входящими в~состав коннекторов, может 
быть фрагмент текста, например: \textit{если} (расстояние) \textit{так}, 
\textit{когда} (расстояние) \textit{то}, \textit{не только} (расстояние) 
\textit{но~и}, \textit{так как} (расстояние) \textit{то} и~т.\,д.
  
  Из текстов НКРЯ был отобран экспериментальный массив 
объемом~1500~объектов перевода с~коннекторами русского языка, которые 
и~были переведены с~помощью системы GNMT (см.\ табл.~1 с~примером 
машинного перевода предложения из пьесы А.\,П.~Чехова <<Дядя Ваня>>). 
Критерии отбора объектов перевода будут описаны в~разд.~2.



   

  На первой стадии экспертизы было выполнено лингвистическое 
аннотирование результатов машинного перевода коннекторов и~их контекстов,\linebreak 
включая оценку их качества. Для этого использовался подход, основанный на 
анализе ошибок перевода, что подразумевает их локализацию 
и~классифи\-кацию~\cite{13-gz}. Ориентация проекта на исследование 
коннекторов не позволяла использовать уже существующие системы 
классификации ошибок перевода~\cite{3-gz, 14-gz}, поскольку в~них не 
учитывается специфика этих языковых единиц, в~част\-ности структура 
коннектора, его позиция (конечная, начальная, неначальная) и~расположение\linebreak 
частей коннектора (дистантное, контактное).

\section{Динамические фасетные классификации}

  Система классификации ошибок машинного перевода коннекторов, 
учитывающая их специфику, формировалась в~процессе лингвистического 
аннотирования на основе следующей категоризации коннекторов (в~конце 
каждого элемента списка указана доля коннекторов каждой категории, 
аннотированных в~НБД, в~процентах от общего их числа~1276 по состоянию на 
20.08.2018)~\cite{15-gz}:
  \begin{itemize}
\item одноэлементные (однословные)~--- состоят из одного элемента 
(например, \textit{и}, \textit{или}, \textit{но}, \textit{а}, \textit{хотя}, 
\textit{иначе} и~т.\,д.)~---~5\%;
\item многоэлементные~--- состоят из нескольких элементов (например, 
\textit{и}$\vert$\textit{вообще}, \textit{к~тому же}, \textit{а~особенно}, 
\textit{но при всем при том} и~т.\,д.)~---~55\%;
\item двухкомпонентные~--- коннекторы, части которых вводят два разных 
текстовых фрагмента (например, \textit{даже если бы}$\|$\textit{то}, \textit{как 
только}$\|$\textit{тут же}, \textit{не только}$\|$\textit{а~прежде всего}, 
\textit{раз уж}$\|$\textit{так уж} и~т.\,д.)\footnote{Двойная вертикальная 
черта~<<$\|$>> в~списке указывает на то, что коннектор состоит из двух или более 
компонентов, а~одиночная вертикальная черта~<<$\vert$>>~--- что языковые единицы, 
составляющие коннектор или его компонент, разделены текстом.}~---~35\%;
\item многокомпонентные~--- коннекторы, части которых вводят более двух 
разных текстовых фрагментов (например, \textit{не только}$\|$\textit{но 
даже~и}$\|$\textit{и~даже}, \textit{хотя}$\|$\textit{хотя}$\|$\textit{однако все 
же} и~т.\,д.)~---~5\%.
\end{itemize}

  Из определений этих четырех категорий сле\-дует, что неоднословные 
коннекторы относятся к~послед\-ним трем категориям и~только в~этих категориях 
фрагмент текста может находиться между частями коннекторов. В~качестве 
первоначальной версии классификации ошибок результатов машинного 
перевода использовался список из~8~руб\-рик~\cite{16-gz}, приведенный 
в~табл.~2.


  В процессе аннотирования результатов машинного перевода первоначальная 
классификация ошибок итерационно уточнялась. Техноло-\linebreak

\columnbreak

\noindent
гически это 
обеспечивалось НБД коннекторов за счет использования в~ней динамической 
фасетной классификации признаков (=\;рубрик) аннотирования~\cite{17-gz, 18-gz, 19-gz}. 
До  начала аннотирования в~нее был добавлен новый фасет для описания ошибок 
машинного перевода коннекторов, включавший~8~руб\-рик (см.\ табл.~2). 
В~процессе аннотирования НБД обеспечивала внесение необходимых 
лингвистам изменений в~классификацию непосредственно в~ходе этого 
процесса (включая переименование, удаление и~добавление, деление 
и~объединение руб\-рик), что иногда приводило к~переклассификации уже 
сформированных аннотаций (более подробно о~переклассификации см.~\cite{20-gz}).
  
  После завершения аннотирования обновленная система классификации 
ошибок увеличилась до~15~рубрик (табл.~3). Метод аннотирования 
с~использованием динамических фасетных классификаций описан 
в~работах~\cite{17-gz, 19-gz, 21-gz}.


  
  Отметим, что использование в~НБД метода аннотирования с~применением 
динамических фасетных классификаций позволяет начинать этот процесс, не 
имея в~фасете ошибок ни одной рубрики, т.\,е.\ можно формировать систему 
классификации с~нуля в~процессе аннотирования~\cite{22-gz}. В~любом\linebreak 
варианте формирования (с~нуля или с~исполь\-зо\-ванием первоначальной версии) 
на вариант клас\-сификации, получаемый после завершения ан\-нотирования, 
влияет степень представительности отобранного массива объектов перевода.
  
  Отбор объектов перевода осуществлялся с~учетом следующего условия, 
которое диктуется самой языковой реальностью: частотность употребления 
разных коннекторов непосредственно коррелирует с~их структурой. 
Согласно 
собранному экспериментальному материалу, на однословные коннекторы 
приходится~47,4\% всех зарегистрированных
случаев употребления, доля 
многоэлементных составляет~40,1\%, двухкомпонентных~--- 12\%, 
а~мно-\linebreak\vspace*{-12pt}

\end{multicols}

\begin{table*}[h]\small %tabl2
\vspace*{-12pt}
\begin{center}
\Caption{Версия классификации ошибок до начала эксперимента}
\vspace*{2ex}

\begin{tabular}{|l|c|}
\hline
\multicolumn{1}{|c|}{Название рубрики}&Код рубрики\\
\hline
1. Все предложение аграмматично&AgramTotal\\
2. Коннектор переведен несуществующей языковой единицей&ErrorTotal\\
3. Локальная аграмматичность во фрагменте текста, вводимом 
коннектором&AgramPostCNT\\
4. Локальная аграмматичность во фрагменте текста, не вводимом 
коннектором&AgramLocal\\
5. Орфографическая ошибка в~форме коннектора&ErrorOrth\\
6. Семантическая ошибка в~выборе коннектора&ErrorCNT\\
7. Слова кириллицей&Cyrillic\\
8. Часть неоднословного коннектора переведена ошибочно&ErrorPart\\
\hline
   \multicolumn{2}{l}{\footnotesize \hspace*{2mm}\textbf{Примечание.} В~работе~\cite{16-gz} рубрика 
<<Орфографическая ошибка в~форме коннектора>> имела код AgramOrth.}
   \end{tabular}
   \end{center}
   \vspace*{-12pt}
   \end{table*}
   
   \begin{table*}\small %tabl3
\begin{center}
\Caption{Обновленная классификация ошибок}
\vspace*{2ex}

\begin{tabular}{|l|c|}
\hline
\multicolumn{1}{|c|}{Название рубрики}&Код рубрики\\ 
\hline
\hphantom{9}1. Все предложение аграмматично&AgramTotal\\
\hphantom{9}2. Локальная аграмматичность во фрагменте текста, не вводимом 
коннектором&AgramLocal\\ 
\hphantom{9}3. Локальная аграмматичность во фрагменте текста, вводимом 
коннектором&AgramPostCNT\\ 
\hphantom{9}4. Лексическая ошибка во фрагменте текста&ErrorLex\\ 
\hphantom{9}5. Слова кириллицей&Cyrillic\\ 
\hphantom{9}6. Русское слово латинским шрифтом&Latin\\ 
\hphantom{9}7. Пропуск фрагмента текста&Lacuna\\ 
\hphantom{9}8. Переведена первая часть неоднословного коннектора&TradPartI\\ 
\hphantom{9}9. Переведена вторая часть неоднословного коннектора&TradPartII\\ 
10. Первая часть неоднословного коннектора переведена ошибочно&ErrorPartI\\ 
11. Вторая часть неоднословного коннектора переведена ошибочно&ErrorPartII\\ 
12. Коннектор переведен несуществующей единицей&ErrorTotal\\ 
13. Коннектор ошибочно заменен языковой единицей, не являющейся 
коннектором&ErrorMorph\\ 
14. Орфографическая ошибка в~форме коннектора&ErrorOrth\\ 
15. Семантическая ошибка в~выборе коннектора&ErrorCNT\\ 
\hline 
\end{tabular} 
\end{center} 
\end{table*}

\begin{multicols}{2}


  

\noindent
гокомпонентных~--- всего~0,5\%. Однако при
  формировании массива 
объектов перевода было %\linebreak
 ре-\linebreak шено только частично руководствоваться 
огра\-ничениями, которые накладывает естественный язык.
 Ис\-ходя из условий, 
обозначенных в~исследовательском проекте, объем массива должен был 
со\-став\-лять~1500~случаев употребления, т.\,е.~1500~русскоязычных 
контекстов\footnote{По факту объем массива объектов перевода 
составляет~1530~контекстов, т.\,е.\ на~30~контекстов больше запланированного по 
условиям проекта. Это увеличение объема обеспечивает свободу последующей коррекции 
и~отбора необходимых~1500~контекстов.}, где зафиксированы коннекторы разной 
структуры. 

Чтобы обеспечить репрезентативность экспериментального 
материала, с~одной стороны, было решено не рассматривать случаи 
употребления многокомпонентных коннекторов (в~силу недостаточной 
репрезентативности имеющейся выборки) и~отобрать по~500~контекстов для 
остальных трех категорий (более частотных по употреблению коннекторов), 
с~другой стороны, стояла задача обеспечить многообразие коннекторов каждой 
из этих трех категорий. При этом отбирались те коннекторы, которые 
характеризуются высокой долей употребления, что позволяет наблюдать, как 
система машинного перевода работает с~ними в~разных контекстах их 
употребления. \mbox{Третья} категория (двухкомпонентные коннекторы), в~отличие от 
первых двух, включает в~себя большее число коннекторов 
(табл.~4)\footnote{В~табл.~4 представлен массив объектов перевода в~его 
первоначальной~--- плановой форме (1530~объектов). После завершения аннотирования 
и~проверки аннотаций число отобранных объектов перевода по отдельным категориям 
незначительно изменилось, также менялось число отобранных коннекторов.}: поскольку 
эти языковые единицы гораздо менее частотны, приходится увеличивать 
количество коннекторов, чтобы обеспечить требуемое число контекстов~--- 
объектов перевода~(500). 

Следует отметить главный недостаток такого подхода к~отбору объектов перевода~--- 
в~массив не попадают низкочастотные 
коннекторы, однако особенности их машинного перевода могут стать 
предметом отдельного исследования.


  \begin{table*}\small %tabl4
  \begin{center}
  \Caption{Коннекторы в~массиве объектов машинного перевода}
  \vspace*{2ex}
  
  \begin{tabular}{|c|c|c|c|c|c|c|c|}
  \cline{1-2}
  \cline{4-5}
  \cline{7-8}
Одноэлементные&
\tabcolsep=0pt\begin{tabular}{c}Число\\ объектов\\ перевода\end{tabular}&&
Многоэлементные&\tabcolsep=0pt\begin{tabular}{c}Число\\ объектов\\ 
перевода\end{tabular}&&Двухкомпонентные&\tabcolsep=0pt\begin{tabular}{c}Число\\ 
объектов\\ перевода\end{tabular}\\
 \cline{1-2}
  \cline{4-5}
  \cline{7-8}
только&35&&то есть&41&&не только$\|$но и&63\\
 впрочем&40&&и вообще&41&&если$\|$то&63\\
%\hline
вообще&40&&одним словом&41&&не$\|$а&56\\
%\hline
а&40&&при этом&41&&когда$\|$то&47\\
%\hline
или&40&&да еще&41&&как$\|$так и&42\\
%\hline
также&25&&а впрочем&41&&не только$\|$но даже&30\\
%\hline
тут&29&&лишь только&41&&хотя$\|$но&27\\
%\hline
например&42&&как только&41&&если$\|$так&27\\
%\hline
хотя&42&&а также&37&&так как$\|$то&24\\
%\hline
кстати&40&&в общем&32&&не только$\|$но&20\\
%\hline
зато&42&&между прочим&26&&как$\|$так&15\\
%\hline
причем&42&&да и~вообще&25&&не успеть\;+\;Inf$\|$как&13\\
%\hline
притом&31&&и притом&23&&не только$\|$и&10\\
%\hline
если&22&&но зато&22&&не только$\|$$\varnothing$&16\\
\cline{1-2}
\multicolumn{3}{c|}{\ }&если только&17&&не 
только$\|$даже&\hphantom{9}8\\
\cline{4-5}
\multicolumn{6}{c|}{\ }&если б$\|$то$\vert$бы&\hphantom{9}9\\
%\cline{7-8}
\multicolumn{6}{c|}{\ }&не то что$\|$а&\hphantom{9}9\\
%\cline{5-6}
\multicolumn{6}{c|}{\ }&когда$\|$так&\hphantom{9}8\\
%\cline{5-6}
\multicolumn{6}{c|}{\ }&или$\|$или&\hphantom{9}8\\
%\cline{5-6}
\multicolumn{6}{c|}{\ }&не$\|$только&\hphantom{9}8\\
%\cline{5-6}
\multicolumn{6}{c|}{\ }&раз$\|$то&\hphantom{9}7\\
\cline{7-8}
\end{tabular}
\end{center}
\vspace*{-9pt}
\end{table*}

\vspace*{-9pt}

\section{Количественный анализ с~помощью надкорпусной базы~данных}
  
  По рубрикам, перечисленным в~табл.~3, НБД позволяет получать широкий 
спектр количественных характеристик, необходимых для проведения 
лингвистического анализа ошибок перевода. Отметим, что именно рубрики 
фасетной классификации, используемые в~процессе аннотирования, позволяют 
получать количественные характеристики ошибок машинного перевода 
коннекторов, что является отличительной чертой функциональных 
возможностей НБД. Проиллюстрируем это на примере рубрики TradPartI, 
наличие которой в~аннотации говорит о~том, что в~результате машинного 
перевода была переведена только первая часть неоднословного коннектора.
  
  Как отмечалось выше, в~процессе аннотирования использовались результаты 
машинного перевода сформированного экспериментального масси-\linebreak ва, который 
состоял из~1500~объектов перевода (см.\ табл.~4). В~итоге 
полученные~1500~аннотаций вклю\-ча\-ли~54~коннектора. Из 
них~15~относились к~категории одноэлементных, а~это значит, что аннотации 
с~ними не могли иметь рубрики TradPartI. Таким образом, 
оставалось~39~коннекторов, аннотации с~которыми могли иметь эту рубрику. 
Однако после завершения аннотирования~1500~объектов перевода эту рубрику 
имели только~22~коннектора (см.\ первый столбец табл.~5), 
а~именно: 8~многоэлементных и~14~двухкомпонентных.
  
  В табл.~5 эти коннекторы упорядочены в~соответствии с~числом (по мере 
убывания) сформированных для них аннотаций с~рубрикой TradPartI. Во 
втором столбце указано общее число аннотаций, сформированных для 
соответствующего коннектора из первого столбца, в~третьем~--- число тех из 
них, где была проставлена эта рубрика, а~в~чет\-вер\-том~--- процентное 
отношение числа аннотаций с~рубрикой TradPartI к~их общему числу. Всего 
для~22~коннекторов было сформировано~574~аннотации 
(т.\,е.~926~аннотаций были сформированы для других коннекторов, не 
имеющих этой рубрики), из которых~177~имеют эту рубрику.
  
  
  
  
 
  Отметим, что рубрика TradPartI не всегда говорит об ошибке машинного 
перевода. Строго говоря, она применяется в~тех аннотациях, где при переводе 
был опущен второй элемент многоэлементного коннектора или второй 
компонент двухкомпонентного коннектора. Поэтому в~интересах 
лингвистического анализа предлагается разделить случаи проставления этой 
рубрики на три группы:\\[-15pt]
  \begin{enumerate}[(1)]
\item потеря второго элемента (или компонента) коннектора при машинном 
переводе не влечет за собой изменения его семантики. В~массиве 
из~1500~аннотаций это наблюдается для следующих коннекторов: 
\textit{когда}$\|$\textit{то}, \textit{если}$\|$\textit{то}, \textit{так 
как}$\|$\textit{то}, \textit{если}$\|$\textit{так}, \textit{когда}$\|$\textit{так}, 
\textit{если~б}$\|$\textit{то}$\vert$\textit{бы}, \textit{раз}$\|$\textit{то}. 
Аннотация в~табл.~6 иллюстрирует случай, когда при машинном переводе не 
меняются ни семантика коннектора, ни смысл всего предложения;


\begin{table*}\small %tabl5
  \begin{center}
  \Caption{Количественные характеристики, полученные с~помощью НБД$^*$}
  \vspace*{2ex}
  
  \tabcolsep=7pt
  \begin{tabular}{|l|c|c|c|}
  \hline
Коннектор русского языка&\tabcolsep=0pt\begin{tabular}{c}Всего\\ аннотаций\end{tabular}&
\tabcolsep=0pt\begin{tabular}{c}Из них\\ число аннотаций\\  с~TradPartI\end{tabular}&
\tabcolsep=0pt\begin{tabular}{c}Доля аннотаций\\  с~TradPartI, \%\end{tabular}\\
\hline
\hphantom{9}1) когда$\|$то&47&39\hphantom{9}&82,98\\
\hphantom{9}2) если$\|$то&59&20\hphantom{9}&33,90\\
\hphantom{9}3) но зато&22&20\hphantom{9}&90,91\\
\hphantom{9}4) так как$\|$то&24&15\hphantom{9}&62,50\\
\hphantom{9}5) а впрочем&41&11\hphantom{9}&26,83\\
\hphantom{9}6) да еще&41&11\hphantom{9}&26,83\\
\hphantom{9}7) если$\|$так&25&9&36,00\\
\hphantom{9}8) а также&37&8&21,62\\
\hphantom{9}9) не успеть\;+\;Inf$\|$как&13&6&46,15\\
10) если б$\|$то$\vert$бы&\hphantom{9}9&5&55,56\\
11) как$\|$так и&37&5&13,51\\
12) когда$\|$так&\hphantom{9}8&5&62,50\\
13) и~вообще&44&4&\hphantom{9}9,09\\
14) и~притом&23&4&17,39\\
15) хотя и&\hphantom{9}3&3&100,00\hphantom{9}\\
16) хотя$\|$но&27&3&11,11\\
17) как$\|$так&13&2&15,38\\
18) не только$\|$и&10&2&20,00\\
19) не только$\|$но и&61&2&\hphantom{9}3,28\\
20) если только&17&1&\hphantom{9}5,88\\
21) не только$\|$даже&\hphantom{9}8&1&12,50\\
22) раз$\|$то&\hphantom{9}5&1&20,00\\
\hline
ИТОГО&574\hphantom{9}&177\hphantom{99}&---\\
\hline
\multicolumn{4}{l}{\footnotesize \hspace*{2mm}$^*$Авторы благодарят М.\,Г.~Кружкова за подготовку данной таблицы.}
\end{tabular}
\end{center}
%\end{table*}
%\begin{table*}\small %tabl6
\begin{center}
\Caption{Логико-семантическое отношение не изменяется при потере второй части 
коннектора в~процессе машинного перевода}
\vspace*{2ex}

\tabcolsep=8pt
\begin{tabular}{|p{57mm}|p{23mm}|p{38mm}|p{25mm}|}
\hline
\textbf{Когда} \textit{же} взгляды встретились, \textbf{то} дверь вдруг 
захлопнулась$\ldots$\newline
\newline
\newline
\newline
\newline
(Ф.\,М.~Достоевский <<Преступление и~на\-ка\-за\-ние>>)  &
 \textbf{когда}$\|$\textbf{то}\newline
$\langle$временные$\rangle$\newline
$\langle$CNT p CNT q $\rangle$\newline
$\langle$CNT $\rangle$\newline
$\langle$Дистант$\rangle$ &
\textbf{Quand} les regards se rencontr$\grave{\mbox{e}}$rent, la porte se referma 
brusquement$\ldots$\newline
\newline
\newline
({\sf translate.google.com}, дата обращения 05.02.2018 18:37) 
&  \textbf{quand}\newline
$\langle$временные$\rangle$\newline 
$\langle$с предикацией$\rangle$\newline
$\langle$начальная$\rangle$\newline
$\langle$CNT q p$\rangle$\newline 
$\langle$CNT$\rangle$\newline
$\langle$TradPartI$\rangle$\newline
$\langle$NoError$\rangle$\\
\hline
\multicolumn{4}{p{158mm}}{\footnotesize
    \hspace*{2mm}\textbf{Примечания:} (1)~с~момента записи аннотации изменился вариант перевода, 
предлагаемый системой GNMT: <<Quand les yeux se crois$\grave{\mbox{e}}$rent, la porte claqua 
soudainement$\ldots$>> (12.10.2018 19:22);
(2)~описание фасетов и~их рубрик во втором и~четвертом столбцах 
приведено в~работах~\cite{8-gz, 17-gz}.}
    \end{tabular}
    \end{center}
   % \end{table*}
   %\begin{table*}\small %tabl7
\begin{center}
\Caption{Логико-семантическое отношение незначительно изменяется при потере второй 
части коннектора в~процессе машинного перевода}
\vspace*{2ex}


\begin{tabular}{|p{50.2mm}|p{31mm}|p{42mm}|p{26mm}|}
\hline
Он дал ей понять, что догадался о~ее любви к~нему, 
\textbf{да еще}, \textit{может быть,} 
догадался невпопад.\newline
\newline
\newline
\newline
\newline
(И.\,А.~Гончаров. <<Обломов>>)&\textbf{да еще}\newline
$\langle$аддитивные\newline
 пропозициональные$\rangle$\newline
$\langle$с предикацией$\rangle$\newline
$\langle$начальная $\rangle$\newline
$\langle$p CNT q$\rangle$\newline
$\langle$CNT$\rangle$\newline
$\langle$Контакт$\rangle$
&Il lui fit savoir qu'il avait devin$\acute{\mbox{e}}$ son amour pour lui \textbf{et},  
\textit{peut-$\hat{\mbox{e}}$tre}, l'avait devin$\acute{\mbox{e}}$ au mauvais moment.\newline
\newline
\newline
(translate.google.com, дата об-\linebreak ращения 31.01.2018 01:27)&\textbf{et}\newline 
$\langle$соединительные$\rangle$\newline
$\langle$с предикацией$\rangle$\newline
$\langle$начальная$\rangle$\newline
$\langle$p CNT q$\rangle$\newline
$\langle$CNT$\rangle$\newline
$\langle$TradPartI $\rangle$\newline
$\langle$NoError$\rangle$\\
\hline
\multicolumn{4}{p{160mm}}{\footnotesize \hspace*{2mm}\textbf{Примечание.} С~момента записи 
аннотации изменился вариант перевода, предлагаемый системой GNMT: <<Il lui fit savoir qu'il avait 
devin$\acute{\mbox{e}}$ son amour pour lui et peut-$\hat{\mbox{e}}$tre aussi l'avait-il devin$\acute{\mbox{e}}$ au 
hasard>> (12.10.2018 19:38).}
\end{tabular}
\end{center}
\end{table*}

    
\item потеря второго элемента или компонента при переводе влечет 
незначительные изменения в~передаче семантики коннектора (табл.~7), 
когда в~оригинале второй элемент коннектора конкретизирует семантику 
первого (\textit{да еще}) или добавляет к~выражаемому им 
логико-семантическому отношению близкое по семантике (\textit{но зато});


\item потеря второго элемента или компонента коннектора при переводе ведет 
к~изменению вы\-ра\-жа\-емо\-го им ло\-ги\-ко-се\-ман\-ти\-че\-ско\-го отношения (см.\ 
аннотацию в~табл.~8), а~также может сопровождаться сбоями на уровне 
грамматики или смысла переводимого фрагмента (см.\ аннотацию в~табл.~9).
\end{enumerate}




  
  В заключение этого раздела отметим, что в~процессе лингвистического 
анализа используются таб\-ли\-цы с~более широким спектром количественных 
характеристик, чем табл.~5, так как одновременно рассматривается, как 
правило, сочетание несколь-\linebreak\vspace*{-12pt}

\pagebreak

\end{multicols}

\begin{table*}\small %tabl8
\begin{center}
\Caption{Логико-семантическое отношение значительно изменяется при потере второй 
части коннектора в~процессе машинного перевода}
\vspace*{2ex}


\tabcolsep=7.5pt
\begin{tabular}{|p{44mm}|p{32mm}|p{45mm}|p{24mm}|}
\hline
\textit{Вот}, \textbf{как} придет человек, \textbf{так} отдам.\newline
\newline
\newline
\newline
\newline
\newline
\newline
(И.\,А.~Гончаров. <<Обломов>>)&\textbf{как}$\|$\textbf{так}\newline
$\langle$временное/условное$\rangle$\newline
$\langle$CNT p CNT q$\rangle$\newline
$\langle$CNT$\rangle$\newline 
$\langle$Дистант$\rangle$
&\textit{Alors}, \textbf{comme} un homme vient, je vais le donner.\newline
\newline
\newline
\newline
\newline
\newline
(translate.google.com, дата об-\linebreak ра\-ще\-ния 27.02.2018 19:31)&\textbf{comme}\newline 
$\langle$причина$\rangle$\newline
$\langle$с предикацией$\rangle$\newline
$\langle$начальная$\rangle$\newline
$\langle$CNT q p$\rangle$\newline
$\langle$CNT$\rangle$\newline
$\langle$SubCNT$\rangle$\newline
$\langle$TradPartI$\rangle$\newline
$\langle$Error CNT$\rangle$\\
\hline
\multicolumn{4}{p{159mm}}{\footnotesize \hspace*{2mm}\textbf{Примечание.} С~момента записи 
аннотации изменился вариант перевода, предлагаемый системой GNMT: <<Voil$\grave{\mbox{a}}$ comment un homme 
vient, alors je vais donner>> (12.10.2018 19:58).}
\end{tabular}
\end{center}
%\end{table*}
 %\begin{table*}\small %tabl9
\begin{center}
\Caption{Логико-семантическое отношение значительно изменяется и~сопровождается 
аграмматичностью при потере второй части коннектора в~процессе машинного перевода}
\vspace*{2ex}

\tabcolsep=7.5pt
\begin{tabular}{|p{52mm}|p{23mm}|p{46mm}|p{24mm}|}
\hline
Целое отделение таких де\-во\-чек$\ldots$ \textbf{как} ехали в~вагоне, \textbf{так~и} 
лежали$\ldots$\newline
\newline
\newline
\newline
\newline
\newline
\newline
(С.\,А.~Алексиевич. <<Время секонд\linebreak хэнд>>)&
\textbf{как}$\|$\textbf{так и}\newline 
$\langle$аналогия$\rangle$\newline 
$\langle$CNT p CNT q$\rangle$\newline 
$\langle$CNT$\rangle$\newline 
$\langle$Дистант$\rangle$
&L'ensemble du d$\acute{\mbox{e}}$partement de ces filles \ldots \textbf{comme} ils sont 
mont$\acute{\mbox{e}}$s dans la voiture, ils $\acute{\mbox{e}}$taient$\ldots$
\newline
\newline
\newline
\newline
\newline
\newline
({\sf translate.google.com}, дата обра-\linebreak щения 06.02.2018 12:59)&\textbf{comme}\newline 
$\langle$сравнительные$\rangle$\newline 
$\langle$с предикацией$\rangle$\newline 
$\langle$начальная$\rangle$\newline 
$\langle$CNT q p$\rangle$\newline
$\langle$CNT$\rangle$\newline 
$\langle$AgramPostCNT$\rangle$\newline
$\langle$ErrorLex$\rangle$\newline
$\langle$TradPartI$\rangle$\newline
$\langle$Error CNT$\rangle$\\
\hline 
   \multicolumn{4}{p{159mm}}{\footnotesize \hspace*{2mm}\textbf{Примечание.} С~момента записи аннотации изменился 
вариант перевода, предлагаемый системой GNMT: <<Toute une branche de ces filles \ldots comme elles 
$\acute{\mbox{e}}$taient dans la voiture, elles gisaient comme 
\mbox{{\ptb{\normalsize \c{c}}}a}$\ldots$>> (13.10.2018 21:55).}
   \end{tabular}
   \end{center}
   \end{table*}
   
   \begin{multicols}{2}

\noindent
ких руб\-рик. Здесь была рассмотрена только одна 
руб\-ри\-ка TradPartI, чтобы продемонстрировать функциональные возможности 
НБД по аннотированию и~количественной обработке результатов машинного 
перевода.

\section{Заключение}

  Разработанная информационная технология, которая поддерживает 
экспертизу результатов машинного перевода, существенно расширяет спектр 
задач компьютерной лингвистики, которые можно решать с~использованием 
НБД. Сначала НБД использовались в~основном для лингвистического 
аннотирования языковых единиц в~параллельных текстах с~референтными 
переводами. Затем НБД была адаптирована для поддержки экспертизы 
результатов машинного перевода.
  
  В~процессе применения разработанной технологии с~использованием 
адаптированной НБД были\linebreak
 получены следующие новые результаты 
в~компьютерной лингвистике. Во-пер\-вых, появилась возмож\-ность 
интегрировать в~рамках единой фасетной классификации общеязыковые 
рубрики ошибок (например, пропуск слов, нарушение их порядка, ошибки 
в~пунктуации и~др.~\cite{3-gz}) и~рубрики, характеризующие ошибки перевода 
только одного вида языковых единиц. Такой подход позволяет выделить 
отдельный фасет классификации ошибок для каждого вида языковых единиц. 
Во-вто\-рых, появилась возможность непосредственно в~процессе 
аннотирования формировать единую систему классификации и~общеязыковых, и~видовых ошибок машинного перевода с~нуля или начиная с~некоторой ее 
первоначальной (базовой) версии.
  
   {\small\frenchspacing
 {%\baselineskip=10.8pt
 \addcontentsline{toc}{section}{References}
 \begin{thebibliography}{99}
\bibitem{1-gz}
Translation quality assessment: From principles to practice~/ 
Eds. J.~Moorkens, S.~Castilho, F.~Gaspari, S.~Doherty.~---
Machine translation: Technologies and applications ser.~--- 
Cham: Springer International Publishing, 
2018. Vol.~1. 299~p.
\bibitem{2-gz}
\Au{Scott B.} Translation, brains and the computer: 
A~neurolinguistic solution to ambiguity and 
complexity in machine translation.~--- Machine translation: Technologies and 
applications ser.~--- 
Cham: Springer International Publishing, 2018. Vol.~2. 241~p.
\bibitem{3-gz}
\Au{Popovi$\acute{\mbox{c}}$ M.} 
Error classification and analysis for machine translation quality 
assessment~//
Translation quality 
assessment: From principles to practice~/ 
Eds. J.~Moorkens, S.~Castilho, F.~Gaspari, S.~Doherty.~---
Machine translation: Technologies and applications ser.~--- 
Cham: Springer International Publishing, 2018. Vol.~1. P.~129--158.
\bibitem{4-gz}
Семантика коннекторов: контрастивное исследование~/ Под ред.\ О.\,Ю.~Иньковой.~--- М.: 
ТОРУС ПРЕСС, 2018. 368~с.
\bibitem{5-gz}
\Au{Кружков М.\,Г.} Информационные ресурсы конт\-растивных лингвистических 
исследований: электронные корпуса текстов~// Системы и~средства информатики, 2015. 
Т.~25. №\,2. С.~140--159.
\bibitem{6-gz}
\Au{Зализняк Анна А., Зацман~И.\,М., Инькова~О.\,Ю., Кружков~М.\,Г.} Надкорпусные базы 
данных как лингвистический ресурс~// Корпусная лингвистика-2015: Труды 7-й Междунар. 
конф.~--- СПб.: СПбГУ, 2015. С.~211--218.
\bibitem{7-gz}
\Au{Попкова Н.\,А., Инькова~О.\,Ю., Зацман~И.\,М., Кружков~М.\,Г.} Методика построения 
моноэквиваленций в~надкорпусной базе данных коннекторов~// Задачи современной 
информатики: Труды 2-й конф.~--- М.: ФИЦ ИУ РАН, 2015. С.~143--153.
\bibitem{8-gz}
\Au{Зацман И.\,М., Инькова~О.\,Ю., Кружков~М.\,Г., Попкова~Н.\,А.} Представление  
кросс-язы\-ко\-вых знаний о~коннекторах в~надкорпусных базах данных~// Информатика 
и~её применения, 2016. Т.~10. Вып.~1. С.~106--118.
\bibitem{9-gz}
\Au{Добровольский Д.\,О., Кретов~А.\,А., Шаров~С.\,А.} Корпус параллельных текстов: 
архитектура и~возможности использования~// Национальный корпус русского языка:  
2003--2005.~--- М.: Индрик, 2005. С.~263--296.
\bibitem{10-gz}
\Au{Wu Y., Schuster M., Chen~Z., %Le~Q.\,V., Norouzi~M., Macherey~W., Krikun~M., Cao~Y., 
%Gao~Q., Macherey~K., 
\textit{et al.}} Google's neural machine translation system: Bridging the 
gap between human and machine translation~// arXiv.org, 2016. {\sf 
https://arxiv.org/pdf/1609.08144.pdf}.
\bibitem{11-gz}
\Au{Johnson M., Schuster M., Le~Q.\,V., Krikun~M., Wu~Y., Chen~Zh., Thorat~N., 
Vi$\acute{\mbox{e}}$gas~F., Wattenberg~M., Corrado~G., Hughes~M., Dean~J.} Google's 
multilingual neural machine translation system: Enabling zero-shot translation~// 
T.~Assoc. Computational Linguistics, 2017. Vol.~5. P.~339--351.
\bibitem{12-gz}
Национальный корпус русского языка. {\sf http://\linebreak www.ruscorpora.ru}.
\bibitem{13-gz}
\Au{Улиткин~И.\,А.} Автоматическая оценка качества машинного перевода  
на\-уч\-но-тех\-ни\-че\-ско\-го текста~// Вестник МГОУ. Сер. Лингвистика, 2016. №\,4. 
С.~174--182.
\bibitem{14-gz}
\Au{Vilar D., Xu J., D'haro~L., Ney~H.} Error analysis of statistical machine translation output~// 
5th Conference (International) on Language Resources and Evaluation Proceedings.~--- Genoa, 
Italy: European Language Resources Association, 2006. P.~697--702. 
{\sf http://www.lrec-conf.org/proceedings/lrec2006/pdf/413\_pdf.pdf}.
\bibitem{15-gz}
\Au{Inkova~O.\,Yu., Kruzhkov~M.\,G.} Statistical analysis of language specificity of connectives 
based on parallel texts~// Информатика и~её применения, 2018. Т.~12. Вып.~3. С.~83--90.
\bibitem{16-gz}
\Au{Nuriev V., Buntman~N., Inkova~O.} Machine translation of Russian connectives into French: 
Errors and quality failures~// Информатика и~её применения, 2018. Т.~12. Вып.~2. С.~105--113.
\bibitem{17-gz}
\Au{Зализняк Анна А., Зацман~И.\,М., Инькова~О.\,Ю.} Надкорпусная база данных 
коннекторов: построение сис\-те\-мы терминов~// Информатика и~её применения, 2017. Т.~11. 
Вып.~1. С.~100--106.
\bibitem{18-gz}
\Au{Inkova O.\,Yu., Popkova~N.\,А.} Statistical data as information source for linguistic analysis of 
Russian connectors~// Информатика и~её применения, 2017. Т.~11. Вып.~3. С.~123--131.
\bibitem{19-gz}
\Au{Зацман И.\,М., Кружков~М.\,Г., Лощилова~Е.\,Ю.} Методы анализа частотности моделей 
перевода коннекторов и~обратимость генерализации статистических данных~// Системы 
и~средства информатики, 2017. Т.~27. №\,4. С.~164--176.
\bibitem{20-gz}
\Au{Зацман~И.\,М.} Стадии целенаправленного извлечения знаний, имплицированных 
в~параллельных текстах~// Системы и~средства информатики, 2018. Т.~28. №\,3.  
С.~169--182.
\bibitem{21-gz}
\Au{Дурново А.\,А., Зацман И.\,М., Лощилова~Е.\,Ю.} Кросс\-лингвистическая база данных для 
аннотирования ло\-ги\-ко-се\-ман\-ти\-че\-ских отношений в~тексте~// Системы и~средства 
информатики, 2016. Т.~26. №\,4. С.~124--137.
\bibitem{22-gz}
\Au{Zatsman~I.} Goal-oriented creation of individual knowledge: Model and information 
technology~// 19th European Conference on Knowledge Management Proceedings.~--- Reading: 
Academic Publishing International Ltd., 2018. Vol.~2. P.~947--956.
 \end{thebibliography}

 }
 }

\end{multicols}

\vspace*{-3pt}

\hfill{\small\textit{Поступила в~редакцию 15.10.18}}

%\vspace*{8pt}

%\pagebreak

\newpage

\vspace*{-28pt}

%\hrule

%\vspace*{2pt}

%\hrule

%\vspace*{-2pt}

\def\tit{USING SUPRACORPORA DATABASES FOR~QUANTITATIVE ANALYSIS  
OF~MACHINE TRANSLATIONS}

\def\titkol{Using supracorpora databases for~quantitative analysis  
of~machine translations}

\def\aut{N.\,V.~Buntman$^1$, A.\,A.~Goncharov$^2$, 
I.\,M.~Zatsman$^2$, and~V.\,A.~Nuriev$^2$}

\def\autkol{N.\,V.~Buntman, A.\,A.~Goncharov , I.\,M.~Zatsman, and~V.\,A.~Nuriev}

\titel{\tit}{\aut}{\autkol}{\titkol}

\vspace*{-11pt}


\noindent
$^1$M.\,V.~Lomonosov Moscow State University,  GSP-1, Leninskie Gory, Moscow 119991, 
Russian Federation

\noindent
$^2$Institute of Informatics Problems, Federal Research Center ``Computer Science and
Control'' of the Russian\linebreak
$\hphantom{^1}$Academy of Sciences, 44-2~Vavilov Str., Moscow 119333, Russian
Federation 




\def\leftfootline{\small{\textbf{\thepage}
\hfill INFORMATIKA I EE PRIMENENIYA~--- INFORMATICS AND
APPLICATIONS\ \ \ 2018\ \ \ volume~12\ \ \ issue\ 4}
}%
 \def\rightfootline{\small{INFORMATIKA I EE PRIMENENIYA~---
INFORMATICS AND APPLICATIONS\ \ \ 2018\ \ \ volume~12\ \ \ issue\ 4
\hfill \textbf{\thepage}}}

\vspace*{6pt}

      
\Abste{The paper discusses an information technology that supports expertise of machine 
translations. The technology has been developed to meet the following conditions: 
($i$)~there are 
connectives in all translated contexts; ($ii$)~the connectives may be both one-word 
(\textit{khotya} `although,' \textit{a} `and') and multiword (\textit{da esche} `and beside this,' 
\textit{no zato} `but instead'); and ($iii$)~between words making up a given connective, there may 
be a space (\textit{esli} (space) \textit{tak} `if (space) then'). With this technology, expertise of 
machine translations develops through three main stages: ($i$)~linguistic annotation of machine 
translations in a~supracorpora database; ($ii$)~quantitative processing of annotations; 
and ($iii$)~linguistic analysis of annotations and quantitative data. The paper describes technological 
aspects of the first two stages. The examples given are only those with multiword connectives. 
Source sentences chosen for machine translation have been collected from literary texts.}

\KWE{supracorpora database; machine translation; classification of errors; technology 
supporting expertise; linguistic annotation; corpus linguistics; connectives}

  \DOI{10.14357/19922264180414}

\vspace*{-14pt}

\Ack
\noindent
The work was fulfilled at the Institute of Informatics Problems of the
Federal Research Center ``Computer Science and
Control'' of the Russian Academy of Sciences and supported by the Russian
Foundation for Basic Research
(project No.\,16-24-41002).



%\vspace*{6pt}

  \begin{multicols}{2}

\renewcommand{\bibname}{\protect\rmfamily References}
%\renewcommand{\bibname}{\large\protect\rm References}

{\small\frenchspacing
 {%\baselineskip=10.8pt
 \addcontentsline{toc}{section}{References}
 \begin{thebibliography}{99}
\bibitem{1-gz-1}
Moorkens, J., S.~Castilho, F.~Gaspari, and S.~Doherty, eds. 
2018. \textit{Translation quality 
assessment: From principles to practice}. 
Machine translation: Technologies and applications ser.  Cham: Springer 
International Publishing. Vol.~1. 299~p.
\bibitem{2-gz-1}
\Aue{Scott, B.} 2018. \textit{Translation, brains and the computer:
A~neurolinguistic solution to ambiguity and 
complexity in machine translation.} Machine translation: 
Technologies and applications ser.  Cham: Springer International Publishing. 
Vol.~2. 241~p.
\bibitem{3-gz-1}
\Aue{Popovi$\acute{\mbox{c}}$.~M.} 2018. Error classification and analysis for machine 
translation quality assessment. \textit{Translation quality assessment:
From principles to practice}. 
Eds.\ J.~Moorkens, S.~Castilho, F.~Gaspari, and S.~Doherty. 
Machine translation:  Technologies and applications ser. 
Cham: Springer International Publishing. 1:129--158.
\bibitem{4-gz-1}
Inkova, O.\,Yu., ed. 2018. \textit{Semantika konnektorov: kontrastivnoe issledovanie} [Semantics 
of connectives: A~contrastive study]. Moscow: TORUS PRESS. 368~p.
\bibitem{5-gz-1}
\Aue{Kruzhkov, M.\,G.} 2015. Informatsionnye resursy kontrastivnykh lingvisticheskikh 
issledovaniy: elektronnye kor\-pu\-sa tekstov [Information resources for contrastive studies: Electronic 
text corpora]. \textit{Sistemy i~Sredstva Informatiki~--- Systems and Means of Informatics}  
25(2):140--159.
\bibitem{6-gz-1}
\Aue{Zaliznyak, Anna A., I.\,M.~Zatsman, O.\,Yu.~Inkova, and M.\,G.~Kruzhkov.} 2015. 
Nadkorpusnye bazy dannykh kak lingvisticheskiy resurs [Supracorpora databases as linguistic 
resource]. \textit{Conference 
(International) ``Corpus linguistics-2015'' Proceedings}. St.\ Petersburg: St.\ Petersburg State 
University. 211--218.
\bibitem{7-gz-1}
\Aue{Popkova, N.\,A., O.\,Yu.~Inkova, I.\,M.~Zatsman, and M.\,G.~Kruzhkov.} 2015. Metodika 
postroeniya monoekvivalentsiy v~nadkorpusnoy baze dannykh konnektorov [Methodology of 
constructing monoequivalences in the Supracorpora database of connectors]. \textit{Tr. 2-y nauchn. 
konf. ``Zadachi sovremennoy informatiki''} [2nd Scientific Conference ``Problems of Modern 
Informatics'' Proceedings]. Moscow: FRC CSC RAS. 143--153.
\bibitem{8-gz-1}
\Aue{Zatsman, I.\,M., O.\,Yu.~Inkova, M.\,G.~Kruzhkov, and N.\,A.~Popkova.} 2016. 
Predstavlenie kross-yazykovykh znaniy o~konnektorakh v~nadkorpusnykh bazakh dannykh 
[Representation of cross-lingual knowledge about connectors in Supracorpora databases]. 
\textit{Informatika i~ee Primeneniya~--- Inform. Appl.} 10(1):106--118.
\bibitem{9-gz-1}
\Aue{Dobrovol'skiy, D.\,O., A.\,A.~Kretov, and S.\,A.~Sharov.} 2005. Korpus parallel'nykh 
tekstov: arkhitektura i~voz\-mozh\-nosti ispol'zovaniya [Corpus of parallel texts: Architecture and 
applications]. \textit{Natsional'nyy korpus russkogo yazyka: 2003--2005} [Russian National 
Corpus: 2003--2005]. Moscow: Indrik. 263--296.
\bibitem{10-gz-1}
\Aue{Wu, Y., M.~Schuster, Z.~Chen, 
%Q.\,V.~Le, M.~Norouzi, W.~Macherey, M.~Krikun, Y.~Cao, 
%Q.~Gao, K.~Macherey, 
\textit{et al}}. 2016. Google's neural machine translation system: Bridging 
the gap between human and machine translation. Available at: {\sf 
https://arxiv.org/pdf/1609.08144.pdf} (accessed September~3, 2018).
\bibitem{11-gz-1}
\Aue{Johnson, M., M.~Schuster, Q.\,V.~Le, M.~Krikun, Y.~Wu, Zh.~Chen, N.~Thorat, 
F.~Vi$\acute{\mbox{e}}$gas, M.~Wattenberg, G.~Corrado, M.~Hughes, and J.~Dean.} 2017. 
Google's multilingual neural machine translation system: Enabling zero-shot translation. 
\textit{T.~Assoc. Computational Linguistics} 5:339--351.
\bibitem{12-gz-1}
Natsional'nyy korpus russkogo yazyka [Russian National Corpus]. 
Available at: {\sf http://www.ruscorpora.ru} (accessed November~30, 2018).
\bibitem{13-gz-1}
\Aue{Ulitkin, I.\,A.} 2016.  Avtomaticheskaya otsenka kachestva mashinnogo perevoda 
nauchno-tekhnicheskogo teksta [Automatic evaluation of machine translation quality of a scientific text]. 
\textit{B..~MRSU. Ser. 
Linguistics} 4:174--182.
\bibitem{14-gz-1}
\Aue{Vilar, D., J.~Xu, L.~D'haro, and H.~Ney}. 2006. Error analysis of statistical machine 
translation output. \textit{5th Conference (International) on Language Resources and Evaluation 
Proceedings}. Genoa, Italy: European Language Resources Association. Available at: {\sf 
http://www.lrec-conf.org/proceedings/lrec2006/pdf/413\_pdf.pdf} (accessed September~3, 2018).
\bibitem{15-gz-1}
\Aue{Inkova, O.\,Yu., and M.\,G.~Kruzhkov.} 2018. Statistical analysis of language specificity of 
connectives based on parallel texts. \textit{Informatika i~ee Primeneniya~--- Inform. Appl.}  
12(3):83--90.
\bibitem{16-gz-1}
\Aue{Nuriev, V., N.~Buntman, and O.~Inkova.} 2018. Machine translation of Russian connectives 
into French: Errors and quality failures. \textit{Informatika i~ee Primeneniya~--- Inform. Appl.} 
12(2):105--113.
\bibitem{17-gz-1}
\Aue{Zaliznyak, Anna~A., I.\,M.~Zatsman, and O.\,Yu.~Inkova.} 2017. Nadkorpusnaya baza 
dannykh konnektorov: postroenie sistemy terminov [Supracorpora database on connectives: Term 
system development]. \textit{Informatika i~ee Primeneniya~--- Inform. Appl.} 11(1):100--106.
\bibitem{18-gz-1}
\Aue{Inkova, O.\,Yu., and N.\,А.~Popkova.} 2017. Statistical data as information source for 
linguistic analysis of Russian connectors. \textit{Informatika i~ee Primeneniya~--- Inform. Appl.} 
11(3):123--131. 
\bibitem{19-gz-1}
\Aue{Zatsman, I.\,M., M.\,G.~Kruzhkov, and E.\,Yu.~Loshchilova}. 2017. Metody analiza 
chastotnosti modeley perevoda konnektorov i~obratimost' generalizatsii statisticheskikh dannykh 
[Methods of frequency analysis of connectives translations and reversibility of statistical data 
generalization]. \textit{Sistemy i~Sredstva Informatiki~--- Systems and Means of Informatics} 
27(4):164--176. 
\bibitem{20-gz-1}
\Aue{Zatsman, I.\,M.} 2018. Stadii tselenapravlennogo izvlecheniya znaniy, implitsirovannykh 
v~parallel'nykh tekstakh [Stages of goal-oriented discovery of knowledge implied in parallel texts]. 
\textit{Sistemy i~Sredstva Informatiki~--- Systems and Means of Informatics} 28(3):169--182.
\bibitem{21-gz-1}
\Aue{Durnovo, A.\,A., I.\,M.~Zatsman, and E.\,Yu.~Loshchilova.} 2016. Kross-lingvisticheskaya 
baza dannykh dlya annotirovaniya logiko-semanticheskikh otnosheniy v~tekste [Cross-lingual 
database for annotating logical-semantic relations in the text]. \textit{Sistemy i~Sredstva 
Informatiki~--- Systems and Means of Informatics} 26(4):124--137.
\bibitem{22-gz-1}
\Aue{Zatsman, I.} 2018. Goal-oriented creation of individual knowledge: Model and information 
technology. \textit{19th European Conference on Knowledge Management Proceedings}. Reading: 
Academic Publishing International Ltd. 2:947--956.
\end{thebibliography}

 }
 }

\end{multicols}

\vspace*{-6pt}

\hfill{\small\textit{Received October 15, 2018}}

%\pagebreak

%\vspace*{-18pt}  

  \Contr
  
  \noindent
  \textbf{Buntman Nadezhda V.} (b.\ 1957)~--- Candidate of Science (PhD) in philology, 
associate professor, M.\,V.~Lomonosov Moscow State University, GSP-1, Leninskie Gory, 
Moscow 119991, Russian Federation; \mbox{nabunt@hotmail.com}
  
  \vspace*{3pt}
  
  \noindent
  \textbf{Goncharov Alexander A.} (b.\ 1994)~--- engineer, Institute of Informatics Problems, 
Federal Research Center ``Computer Science and Control'' of the Russian Academy of Sciences, 
44-2~Vavilov Str., Moscow 119333, Russian Federation; \mbox{a.gonch48@gmail.com}
  
  \vspace*{3pt}
  
  
  \noindent
  \textbf{Zatsman Igor M.} (b.\ 1952)~--- Doctor of Science in technology, Head of Department, 
Institute of Informatics Problems, Federal Research Center ``Computer Science and Control'' of the 
Russian Academy of Sciences, 44-2~Vavilov Str., Moscow 119333, Russian Federation; 
\mbox{izatsman@yandex.ru}
  
  \vspace*{3pt}
  
  
  \noindent
  \textbf{Nuriev Vitaly A.} (b.\ 1980)~--- Candidate of Science (PhD) in philology, leading 
scientist, Institute of Informatics Problems, Federal Research Center ``Computer Science and 
Control'' of the Russian Academy of Sciences, 44-2~Vavilov Str., Moscow 119333, Russian 
Federation; \mbox{nurieff.v@gmail.com}
 
\label{end\stat}

\renewcommand{\bibname}{\protect\rm Литература} 