\def\stat{gorshenin-1}

\def\tit{РАЗВИТИЕ СЕРВИСОВ ЦИФРОВЫХ ПЛАТФОРМ ДЛЯ~ПРЕОДОЛЕНИЯ НЕФИНАНСОВЫХ БАРЬЕРОВ$^*$}

\def\titkol{Развитие сервисов цифровых платформ для преодоления нефинансовых барьеров}

\def\aut{А.\,К.~Горшенин$^1$}

\def\autkol{А.\,К.~Горшенин}

\titel{\tit}{\aut}{\autkol}{\titkol}

\index{Горшенин А.\,К.}
\index{Gorshenin A.\,K.}




{\renewcommand{\thefootnote}{\fnsymbol{footnote}} \footnotetext[1]
{Работа выполнена при поддержке РНФ (проект 18-71-00156).}}


\renewcommand{\thefootnote}{\arabic{footnote}}
\footnotetext[1]{Институт проблем информатики Федерального исследовательского
центра <<Информатика и~управ\-ле\-ние>> Российской академии наук; факультет
вычислительной математики и~кибернетики Московского государственного 
университета им.\ М.\,В.~Ломоносова, \mbox{agorshenin@frccsc.ru}}

\vspace*{8pt}




\Abst{Рассматриваются примеры различных нефинансовых барьеров, 
препятствующих эффективному развитию молодежи в~научной и~образовательной сферах. 
Их преодоление не всегда возможно исключительно за счет привлечения дополнительных 
ресурсов, так как зачастую требуется изменение среды.
Предложены концептуальные способы преодоления подобных барьеров за счет 
создания и~развития сервисов платформ цифровой экономики~--- одной из основных 
современных парадигм в~информационных технологиях (ИТ). На примере создаваемой в~Федеральном 
исследовательском центре <<Информатика и~управ\-ле\-ние>> Российской академии наук цифровой 
платформы <<Наука и~образование>> продемонстрировано  соответствие сервисов 
основным направлениям реализации Стратегии на\-уч\-но-тех\-но\-ло\-ги\-че\-ско\-го 
развития Российской Федерации (СНТР). 
Предложенная концепция является эффективной и~для решения более широкого класса задач.}

\KW{цифровые платформы; нефинансовые барьеры; стратегия научно-технологического развития; цифровая экономика; молодежная политика}

\DOI{10.14357/19922264180415}
  
%\vspace*{4pt}


\vskip 10pt plus 9pt minus 6pt

\thispagestyle{headings}

\begin{multicols}{2}

\label{st\stat}


\section{Введение}

Российская Федерация по относительному показателю~--- 
доле затрат на проведение научных исследований и~разработок от величины 
внутреннего валового продукта (ВВП)~--- уступает лидерам (США, Китай, Япония, 
Южная Корея), одна-\linebreak ко находится на приемлемом мировом уровне~\cite{GlobalRD2018}, 
опережая по ряду показателей страны БРИКС (вклю\-чая Индию и~Бразилию), 
а~также европейские государства (Италия, Испания). В~частности,\linebreak
 с~точки зрения 
такого важного наукометрического показателя, как число опубликованных\linebreak
 в~международных 
базах работ, согласно данным\linebreak \verb"The SCImago Journal & Country Rank" 
({\sf https://\linebreak www.scimagojr.com/countryrank.php}), Россия входит в~топ-15 стран мира 
(рис.~1).



В то же время влияние отечественных исследований на общемировые 
тенденции не столь велико. На рис.~\ref{FigCit} продемонстрирована 
за\-ви\-си\-мость\linebreak <<странового>> индекса Хирша от среднего 
чис\-ла цитирований документа ({\sf https://www.scimagojr.\linebreak com/worldreport.php}) 
для различных стран. Диа\-метр кругов соответствует числу работ, 
Россия выделена более крупным полужирным шрифтом.

\setcounter{figure}{1}
\begin{figure*} %fig2
\vspace*{1pt}
 \begin{center}
 \mbox{%
 \epsfxsize=149.748mm 
 \epsfbox{gor-2.eps}
 }
 \end{center}
\vspace*{-7pt}
\Caption{Зависимость <<странового>> индекса Хирша от среднего числа цитирований}
\label{FigCit}
\vspace*{14.5pt}
\end{figure*}



Указанное обстоятельство связано как с~дефицитом финансирования отрасли 
(только~1{,}52\%~ВВП\linebreak\vspace*{-12pt}

{ \begin{center}  %fig1
 \vspace*{1pt}
  \mbox{%
 \epsfxsize=79mm 
 \epsfbox{gor-1.eps}
 }


\end{center}


\noindent
{{\figurename~1}\ \ \small{Общее число опубликованных статей за 1996--2017~гг.\ в~стра\-нах-ли\-де\-рах}}
}

\vspace*{12pt}

\addtocounter{figure}{1}




\noindent
инвестируется в~научные исследования~\cite{GlobalRD2018}, 
в~то время как в~США~--- 2{,}83\%, при общем отличии отечественного ВВП в~6--8~раз 
по сравнению с~США и~Китаем), так и~с достаточно небольшим числом ученых и~инженеров 
в~пересчете на миллион населения (около~3300~человек)~--- 
по этому показателю Россия уступает Тайваню, Сингапуру, Дании, Норвегии, 
Финляндии, Швеции. Очевидно, что в~данной области необходимо достижение позитивной 
динамики, которая невозможна без привлечения новых молодых исследователей и~улучшения 
условий для уже задействованных в~на\-уч\-но-об\-ра\-зо\-ва\-тель\-ной сфере.


Развитие кадрового и~образовательного потенциала~--- один из ключевых драйверов 
роста согласно программе <<Цифровая экономика Российской Федерации>> 
(утверждена Распоряжением Правительства РФ от 28.07.2017 №\,1632-р). 

В~качестве барьеров для развития молодых исследователей могут выступать и~нормативные 
документы, и~сложившиеся практики, и~отсутствие должного уровня автоматизации 
процессов. Их преодоление не всегда возможно исключительно за счет привлечения 
дополнительных ресурсов, поэтому в~данном контексте они будут называться 
нефинансовыми барь\-е\-ра\-ми. 

В~настоящей статье рассматриваются соответствующие 
примеры по основным направлениям реализации СНТР (утверждена Указом Президента РФ №\,642 
от~01.12.2016) и~предлагаются возможные способы их преодоления путем создания 
и~использования специальных сервисов цифровых платформ. 

В~качестве конкретного примера рассмотрено соответствие между сервисами 
создаваемой в~настоящий момент Федеральным исследовательским центром 
<<Информатика и~управление>> Российской академии наук цифровой платформы 
<<Наука и~образование>>~\cite{Gorshenin2017} и~соответствующими направлениями СНТР.

\section{Концепция использования сервисов цифровых платформ}

\vspace*{-12pt}

Переход к~цифровой экономике и~поэтапное развитие подходов 
Индустрии~4.0~\cite{Schwab2016} приводят к~необходимости подготовки кадров, 
обладающих новым набором компетенций. Значительное влияние на данные процессы 
оказывает и~стремительное развитие ИТ, 
предоставляющих инструменты формирования единой цифровой институциональной 
среды для различных заинтересованных сторон. В~частности, одним из наиболее 
востребованных на сегодняшний день решений являются цифровые платформы, 
со\-став\-ля\-ющие ИТ-ба\-зис для задач современной цифровой экономики.

Преодоление нефинансовых барьеров в~определенной степени может осуществляться 
за счет нор\-ма\-тив\-но-пра\-во\-вых (внесение изменений в~действующие акты, положения) 
или финансовых (расширение инвестиций в~на\-уч\-но-об\-ра\-зо\-ва\-тель\-ную 
сферу целевым образом на поддержку и~развитие молодежных кадров) мер. Однако с~учетом 
экономической ситуации и~потенциальных сложностей при изменении законодательства на 
первый план должны выходить более гибкие инструменты решения задач. Поэтому в~данной 
работе предлагается подход на основе использования цифровых платформ, предполагающий 
создание или настройку сервисов, ориентированных на решение соответствующих задач 
(при этом, возможно, в~рамках расширения базового функционала). 

Безусловно, для 
создания таких решений требуется вовлечение широкого спектра 
материальных и~организационных ресурсов, однако развитие так называемой 
циф\-ро\-вой науки~--- одна из наиболее современных парадигм~\cite{Gorshenin2018a}. 
Это означает, что подобные решения будут так или иначе создаваться, а~значит, в~их 
рамках можно выделить необходимые инструменты, актуальные для развития молодежи.

Создание научно-об\-ра\-зо\-ва\-тель\-ных цифровых платформ для решения 
задач государственного уровня необходимо осуществлять на основе так называемых 
центров превосходства~\cite{Zaichenko2008}, осуществляющих прорывные фундаментальные 
и~прикладные исследования в~наиболее важных и~инновационных областях знания, 
обладающих уникальными интеллектуальными и~ма\-те\-ри\-аль\-но-тех\-ни\-че\-ски\-ми ресурсами. 
%
В~следующих разделах будут рассмотрены существующие барьеры в~соответствии 
с~направлениями СНТР, а~также предложено использование или развитие соответствующих 
сервисов цифровой платформы~\cite{Gorshenin2017} для преодоления существующих 
сложностей по каждому из них на примере ряда кейсов для научной молодежи.

\vspace*{-6pt}

\section{Кадры и~человеческий капитал}
\label{Staff}

На сегодняшний день в~научных и~образовательных организациях редко внедряются 
наиболее современные технологии развития молодежного персонала, включая 
инструментарий  формирования широкого спектра <<непрофильных>> компетенций 
(например, управленческих, юридических, финансовых) и~иных инструментов 
создания полноценного кадрового резерва. Для молодых сотрудников необходимо 
проводить различные мероприятия (от интенсивов до полноценных образовательных курсов) 
с~целью развития их компетенций в~административном, инновационном 
и~биз\-нес-на\-прав\-ле\-ни\-ях. Подобное дополнительное образование 
(в~том числе за счет организации) распространено в~коммерческом секторе, 
однако в~условиях формирования цифровой экономики  необходима реализация 
данного подхода в~более широком спектре направлений для формирования 
специалистов на стыке фундаментальной науки, инженерии и~технологического 
предпринимательства.

Решение данной задачи возможно в~рамках все\linebreak
 более распространенного в~мировой 
практике\linebreak
 подхода цифровизации образования~\cite{Paulsen2003} и~раз\-вер\-ты\-вания на 
базе цифровой платформы~\cite{Gorshenin2017} \textit{обра\-зо\-ва\-тельного} сервиса 
как площадки для электронного и~дис\-тан\-ци\-он\-но\-го обучения. Это позволит сформировать 
\textit{цифровые} кадры во взаимодействии с~университетами, федеральными 
органами исполнительной власти, институтами развития, государственными корпорациями. 
В~рамках такого серви\-са возможно внедрение в~образовательный процесс\linebreak
 технологий 
искусственного интеллекта. Уже сей-\linebreak час существуют решения для определения индивидуальных 
методов эффективного электронного\linebreak обучения~\cite{Villaverde2006}, планирования 
востребованности образовательных курсов~\cite{Kardan2013} 
и~самообучения~\cite{Kose2017}. Реализация и~развитие таких подходов на\-прав\-ле\-ны 
на создание современной образовательной \mbox{ИТ-эко}\-сис\-те\-мы~\cite{Gorshenin2018b}.

\vspace*{-6pt}

\section{Инфраструктура и~среда}
\label{Infrastructure}

Крайне важно, чтобы исследования молодых ученых находились в~русле мировых 
тенденций и~были ориентированы на достижение прорывных результатов.
 Для этого необходимо иметь возможность познакомиться с~накопленными 
 знаниями в~профильных областях, а~также смежных или даже просто потенциально 
 интересных. В~настоящий момент многие на\-уч\-но-об\-ра\-зо\-ва\-тель\-ные 
 организации получили доступ к~базам \verb"Web of Science" (преимущественно 
 к~\verb"Core Collection") и~\verb"Scopus", однако в~них представлены тексты 
 статей только для журналов с~открытым доступом. Таким образом, очевидна 
 необходимость внедрения инструментов доступа к~источникам научной информации, 
 например с~помощью подписки на ведущие мировые журналы. Сейчас такие механизмы 
 реализуются, в~частности, через конкурсы Российского фонда фундаментальных 
 исследований (РФФИ), однако максимальное удобство и~наиболее широкий 
 инструментарий для дальнейшего использования возможно обеспечить только 
 в~рамках \textit{информационного} сервиса цифровой платформы.

Еще один важный мировой тренд, ориентированный в~том числе и~на развитие 
национальных\linebreak
 инновационных систем~\cite{Autio2004},~--- 
использование в~научной деятельности наиболее современной ис\-следовательской 
инфраструктуры, в~том числе и~установок класса мегасайенс. Столь сложное оборудование 
должно быть задействовано и~в~целях методологического и~технологического обеспечения 
научно-образовательного процесса. В~России в~настоящий момент подобные объекты 
представлены центрами коллективного пользования (ЦКП) и~уникальными научными 
установками (УНУ). Они достаточно активно используются в~том чис\-ле и~коммерческими 
заказчиками. При этом для молодых исследователей существуют определенные 
сложности с~доступом к~со\-от\-вет\-ст\-ву\-ющим ресурсам, так как они не могут 
гарантировать оплату или получение прорывных результатов, пуб\-ли\-ка\-ции в~престижных 
международных журналах (один из отчетных показателей деятельности ЦКП и~УНУ). 
Более того, здесь важен и~образовательный эффект~--- молодым коллективам нужно 
научиться грамотно использовать подобные инфраструктурные решения, прежде 
чем будут получены результаты (мировой опыт по <<удаленным>> лабораторным работам 
описан, например, в~статье~\cite{Rapuano2006}). Таким образом, необходимо 
предусмотреть специальные механизмы поддержки исследований молодых ученых на ЦКП и~УНУ, 
например с~помощью введения специальных квот. Их выделение, использование, а~также 
наукометрический эффект могут быть учтены с~помощью \textit{сис\-те\-мы управ\-ле\-ния 
научными сервисами} (СУС)~--- специализированного сервиса 
платформы~\cite{Gorshenin2017}, который ориентирован на решение подобного 
класса задач для интегрированного оборудования ЦКП и~УНУ. Финансовые процедуры 
(например, выделение субсидий на оплату использования) реализуются 
в~\textit{обеспечивающих} сервисах, а~современная интеллектуальная обработка 
научных данных исследовательскими коллективами проводится с~по\-мощью 
специализированных \textit{научных} сервисов.

\vspace*{-9pt}

\section{Взаимодействие и~кооперация}
\label{Interaction}

В настоящее время для молодых исследователей доступны несколько 
потенциальных возможностей коммерциализации их научных исследований: создание 
малых инновационных предприятий (МИП), спин-офф компаний, стартапов в~рамках 
проектов институтов развития, включая технопарки и~биз\-нес-ин\-ку\-ба\-то\-ры. 
Однако вопросы привлечения дополнительного финансирования могут серь\-ез\-но отвлекать 
молодежные коллективы именно от содержательной составляющей научных и~инновационных 
исследований. В~подобных условиях достаточно трудно получить столь необходимый опыт 
в~рамках схемы <<идея\,--\,на\-уч\-ные ис\-сле\-до\-ва\-ния\,--\,раз\-ра\-бот\-ка 
про\-то\-ти\-па\,--\,за\-пуск в~производство>>, так как поддержку получают только 
отдельные отобранные коллективы. Поэтому оправдан проект по созданию массовых 
технопарков и~биз\-нес-ин\-ку\-ба\-то\-ров, в~которых одной из основных целей стало бы 
обучение механизму трансфера технологий~\cite{Grosse1996}, правильному 
распределению ролей в~команде, поддержке института техноброкерства. 
Такую практику можно было бы назвать \textit{научными стартапами}. Сами научные 
организации в~силу разных причин зачастую не заинтересованы в~создании 
МИП и~соответствующей поддержке. В~таких условиях кон\-суль\-та\-ци\-он\-но-ор\-га\-ни\-за\-ци\-он\-ная 
помощь от представителей бизнеса (менторов), а~также более опытных ученых чрезвычайно важна~--- 
и~может быть развернута в~рамках сервиса \textit{коммуникаций}. Полезным может 
оказаться и~предоставление базовой организацией льготного доступа к~уникальному 
дорогостоящему оборудованию, которое обсуждалось в~разд.~\ref{Infrastructure}.

\vspace*{-6pt}

\section{Управление и~инвестиции}

Даже в~рамках предложенного в~разд.~\ref{Interaction} формата научных 
стартапов и~инкубаторов существенное финансирование будет выделяться по-на\-сто\-яще\-му 
интересным и~действительно прорывным проектам. Однако при этом появится пространство 
для экспериментов, а~по итогам функционирования таких площадок вполне естественно 
ожидать и~увеличения общего числа качественных разработок, и~понимания молодежью 
соответствующих инструментов. Вопрос финансирования здесь может взять на себя как
 бизнес, так и~научные фонды, при этом цифровая платформа~\cite{Gorshenin2017} 
 предоставит инструменты эффективного взаимодействие различных субъектов цифровой 
 экономики, а также систему учета и~коммуникации лидеров и~энтузиастов направлений.
 {\looseness=-1
 
 }

Поддержка молодых ученых должна осуществляться в~том числе и~в~рамках целевого 
субсидирования их научных исследований в~рамках грантов и~программ. Известно, 
что практика сочетания индивидуальных исследовательских проектов с~решением 
прорывных задач на базе центров превосходства показала себя весьма 
эффективной с~различных точек зрения~\cite{Fortin2013}. Целевые молодежные 
проекты предлагаются в~настоящий момент как Российским научным фондом в~рамках 
Президентской программы, так и~РФФИ, включая и~инициативы сотрудничества с~региональными
 властями. Существенная сложность заключается в~том, что предпочтение зачастую 
 отдается заявкам с~хорошим научным (в~том числе и~публикационным) заделом~--- 
 и~начинающим исследователям достаточно сложно приобрести нужный опыт. Для 
 этого и~предложен инструмент научных стартапов, в~рамках которых вопрос 
 поддержки проектов может решаться с~учетом иных подходов. Вторая задача~--- 
 поиск молодых соисполнителей (или руководителя проекта). Эта проблема также может 
 быть решена с~помощью сервиса коммуникаций, а~\textit{аналитические} сервисы 
 обеспечат сбор, систематизацию и~эффективную обработку наукометрической 
 информации, а~также предоставят рекомендации по дальнейшим проектам активным 
 представителям научной молодежи.

 \setcounter{figure}{2}
\begin{figure*}[b] %fig3
\vspace*{1pt}
 \begin{center}
 \mbox{%
 \epsfxsize=157.723mm 
 \epsfbox{gor-3.eps}
 }
 \end{center}
\vspace*{-9pt}

\Caption{Концептуальное соответствие между направлениями СНТР и~сервисами 
цифровой платформы <<Наука и~образование>>}
\label{FigBarriersServices}
\end{figure*}

 
 \vspace*{-9pt}

\section{Сотрудничество и~интеграция}

Необходимо отметить значимый экономический эффект, возникающий на стыке 
на\-уч\-но-про\-мыш\-лен\-но\-го сотрудничества~\cite{Vuola2006}. 
Прорывные достижения в~рамках научных исследований стимулируют развитие 
технологических инновационных процессов, формирование и~задействование 
новых компетенций, проявляется эффект от трансфера технологий. 

Это позволяет сочетать вклад в~фундаментальную науку с~формированием 
новых ин\-ду\-ст\-ри\-аль\-но-ин\-но\-ва\-ци\-он\-ных отраслей бизнеса. 
Поэтому\linebreak
 необходима поддержка образовательной и~инновационной мобильности молодых 
ученых в~ведущих\linebreak
 отечественных и~зарубежных организациях, выполнение совместных 
научных исследований (напри-\linebreak мер, делегирование молодых исследователей для 
участия в~ме\-гай\-сай\-енс-про\-ек\-тах,
 таких как\linebreak \verb"Compact Linear Collider", 
\verb"European" \verb"XFEL" (X-ray Free Electron Laser), %\linebreak 
\verb"Facility" \verb"for" \verb"Antiproton" \verb"and" \verb"Ion Research", 
\verb"Human" \verb"Proteom" \verb"Project", \verb"ITER" (International Thermonuclear Experimental
Reactor) и~др.), их участие в~международных 
конференциях. Очевидно, что данный пункт существенным образом\linebreak
 связан с~дополнительным 
финансированием и~заключением партнерских соглашений, однако необходимо 
выделить и~сугубо организационные\linebreak моменты, связанные с~деталями оформления 
командировок. 

Данные процессы могут быть полностью автоматизированы в~рамках 
обеспечивающих сервисов, что упростит саму процедуру как для молодых ученых, 
так и~для участвующих организаций, а~также позволит избежать возможных ошибок 
при оформлении документов, финансовых и~временн$\acute{\mbox{ы}}$х 
затрат на их исправление и~пересогласование. Кроме того, циф\-ро\-вые платформы 
предоставляют эффективные инструменты для организации и~проведения совместных 
исследований распределенными научными коллективами, включая создание виртуальных 
лабораторий.

\section{Заключение}

В статье выделены некоторые нефинансовые барь\-е\-ры для развития молодежи в~сфере 
науки и~образования в~соответствии с~направлениями СНТР. Концептуальная схема 
соответствия представлена на рис.~\ref{FigBarriersServices}.


Данная схема отражает тесную взаимосвязь меж\-ду направлениями СНТР и~означает, 
что предложенное разделение сервисов цифровой платформы
 <<Наука и~образование>> 
по ним носит несколько условный характер. Некоторые сервисы могут\linebreak быть 
использованы для решения задач сразу из нескольких отраслей, при этом значительный 
эффект достигается именно при условии их комбинации. Часть рассмотренных барьеров 
может быть устранена и~с~по\-мощью дополнительных инвестиций, однако в~рамках 
идеологии <<больших вызовов>> СНТР такие меры не всегда оказываются однозначно 
эффективными, так как зачастую требуется изменение среды. Подход на основе сервисов 
цифровой платформы представляется наиболее универсальным и~современным. 
Кроме того, он является актуальным и~для решения задач, необязательно 
связанных непосредственно с~молодежной по\-ли\-тикой.

{\small\frenchspacing
 {%\baselineskip=10.8pt
 \addcontentsline{toc}{section}{References}
 \begin{thebibliography}{99}

\bibitem{GlobalRD2018} 
2018 Global R\&D Funding Forecast~// R\&D Mag., Winter 2018. 36~p.
\bibitem{Gorshenin2017} 
\Au{Зацаринный~А.\,А., Горшенин~А.\,К., Волович~К.\,И., Колин~К.\,К., 
Кондрашев~В.\,А., Степанов~П.\,В.} Управление научными сервисами 
как основа национальной циф\-ро\-вой платформы <<Наука и~образование>>~// 
Стратегические приоритеты, 2017. №\,2(14). С.~103--113.
\bibitem{Schwab2016} 
\Au{Шваб~К.} Четвертая промышленная революция.~---
М.: Эксмо, 2016. 208~с.
\bibitem{Gorshenin2018a} 
\Au{Зацаринный~А.\,А., Горшенин~А.\,К., Волович~К.\,И.,
Кондрашев~В.\,А.} Основные направления развития информационных технологий 
в~условиях вызовов цифровой экономики~// Цифровая обработка сигналов, 2018. №\,1. 
С.~3--7.
\bibitem{Zaichenko2008} 
\Au{Заиченко С.\,А.} Центры превосходства в~системе современной научной политики~// 
Форсайт, 2008. Т.~2. №\,1. С.~42--50.
\bibitem{Paulsen2003} 
\Au{Paulsen M.\,F.} Experiences with Learning Management Systems 
in~113~European institutions~// Educ. Technol. Soc., 2003. 
Vol.~6. Iss.~4. P.~134--148.
\bibitem{Villaverde2006} 
\Au{Villaverde~J.\,E., Godoy~D., Amandi~A.} 
Learning styles' recognition in e-learning environments 
with feed-forward neural networks~// J.~Comput. Assist. Lear., 2006. Vol.~22. Iss.~3. P.~197--206.
\bibitem{Kardan2013} 
\Au{Kardan~A.\,A., Sadeghi~H., Ghidary~S.\,S., Sani~M.\,R.\,F.} 
Prediction of student course selection in online higher education 
institutes using neural network~// Comput. Educ., 2013. Vol.~65. P.~1--11.
\bibitem{Kose2017} 
\Au{Kose~U., Arslan~A.} Optimization of self-learning in computer engineering courses: 
An intelligent software system supported by artificial neural network and vortex 
optimization algorithm~// Comput. Appl. Eng. Educ., 2017. Vol.~25. Iss.~1. P.~142--156.
\bibitem{Gorshenin2018b} 
\Au{Gorshenin~A.} Toward modern educational
IT-ecosystems: From learning management systems to digital platforms~// 
10th Congress (International)
on Ultra Modern Telecommunications and
Control Systems and Workshops Proceedings.~---
Piscataway, NJ, USA: IEEE, 2018.
P.~329--333.
\bibitem{Autio2004} 
\Au{Autio~E., Hameri~A.-P., Vuola~O.} 
A~framework of industrial knowledge spillovers in big-science centers~// 
Res. Policy, 2004. Vol.~33. Iss.~1. P.~107--126.
\bibitem{Rapuano2006} 
\Au{Rapuano~S., Zoino~F.} A~learning management system including laboratory 
experiments on measurement instrumentation~// IEEE T.~Instrum. 
Meas., 2006. Vol.~55. Iss.~5. P.~1757--1766.
\bibitem{Grosse1996} 
\Au{Grosse~R.} International technology transfer in services~// 
J.~Int. Bus. Stud., 1996. Vol.~27. Iss. 4. P. 781--800.
\bibitem{Fortin2013}
\Au{Fortin~J.-M., Currie~D.\,J.} Big science vs.\ little science: How 
scientific impact scales with funding~// PLoS One, 2013. Vol.~8. Iss.~6. Art.~No.~e65263.
\bibitem{Vuola2006} 
\Au{Vuola~O., Hameri~A.-P.} Mutually benefiting joint innovation process 
between industry and big-science~// Technovation, 2006. Vol.~26. Iss.~1. P.~3--12.
 \end{thebibliography}

 }
 }

\end{multicols}

\vspace*{-6pt}

\hfill{\small\textit{Поступила в~редакцию 20.09.18}}

\vspace*{6pt}

%\pagebreak

%\newpage

%\vspace*{-28pt}

\hrule

\vspace*{2pt}

\hrule

%\vspace*{-2pt}

\def\tit{DEVELOPMENT OF~SERVICES OF~DIGITAL PLATFORMS TO~OVERCOME NONFINANCIAL BARRIERS}

\def\titkol{Development of~services of~digital platforms to~overcome nonfinancial barriers}

\def\aut{A.\,K.~Gorshenin$^{1,2}$}

\def\autkol{A.\,K.~Gorshenin}

\titel{\tit}{\aut}{\autkol}{\titkol}

\vspace*{-15pt}


\noindent
$^1$Institute of Informatics Problems, Federal Research Center ``Computer Science and
Control'' of the Russian\linebreak
$\hphantom{^1}$Academy of Sciences, 44-2~Vavilov Str., Moscow 119333, Russian
Federation 

\noindent
$^2$Faculty of Computational Mathematics and Cybernetics, M.\,V.~Lomonosov Moscow
State University, GSP-1,\linebreak
$\hphantom{^1}$Leninskie Gory, Moscow 119991, Russian Federation


\def\leftfootline{\small{\textbf{\thepage}
\hfill INFORMATIKA I EE PRIMENENIYA~--- INFORMATICS AND
APPLICATIONS\ \ \ 2018\ \ \ volume~12\ \ \ issue\ 4}
}%
 \def\rightfootline{\small{INFORMATIKA I EE PRIMENENIYA~---
INFORMATICS AND APPLICATIONS\ \ \ 2018\ \ \ volume~12\ \ \ issue\ 4
\hfill \textbf{\thepage}}}

\vspace*{6pt}


\Abste{The article discusses examples of various nonfinancial barriers 
that impede the effective development of young people in science and 
education. Overcoming them is not always possible solely by attracting 
additional resources, since changes in the environment are often required. 
Conceptual ways to overcome such barriers through creation and development 
of digital platform services, one of key paradigms in modern information 
technologies, are proposed. Using the example of the digital platform 
``Science and Education'' created in the Federal Research Center ``Computer Science 
and Control'' of the Russian Academy of Sciences, the compliance of services with 
the main directions of the Scientific and Technological Development Strategy of 
the Russian Federation is demonstrated. 
The proposed concept is effective for solving a wider class of problems.}


\KWE{digital platforms; nonfinancial barriers; strategy of scientific and 
technological development; digital economy; youth policy}



\DOI{10.14357/19922264180415}

\vspace*{-14pt}

\Ack
\noindent
The research was supported by the Russian Science Foundation (project 18-71-00156).



%\vspace*{6pt}

  \begin{multicols}{2}

\renewcommand{\bibname}{\protect\rmfamily References}
%\renewcommand{\bibname}{\large\protect\rm References}

{\small\frenchspacing
 {%\baselineskip=10.8pt
 \addcontentsline{toc}{section}{References}
 \begin{thebibliography}{99}
\bibitem{1-gg}
2018 Global R\&D Funding Forecast. 
Winter 2018. \textit{R\&D Mag.} 36~p.
\bibitem{2-gg}
\Aue{Zatsarinnyy,~A.\,A., A.\,K.~Gorshenin, K.\,I.~Volovich, 
K.\,K.~Kolin, V.\,A.~Kondrashev, and P.\,V.~Stepanov.} 2017. 
Upravlenie nauchnymi servisami kak osnova na\-tsi\-onal'\-noy tsifrovoy platformy 
``Nauka i obrazovanie'' [Management of scientific services as the basis 
of the national digital platform ``Science and Education'']. 
\textit{Strategic Priorities} 2(14):103--113.
\bibitem{3-gg}
\Aue{Schwab, K.\,M.} 2017. \textit{The fourth industrial revolution}. 
London: Penguin U.K. 192~p.
\bibitem{4-gg}
\Aue{Zatsarinnyy,~A.\,A., A.\,K.~Gorshenin, K.\,I.~Volovich, and V.\,A.~Kondrashev.}
2018. Osnovnye napravleniya razvitiya informatsionnykh tekhnologiy 
v~usloviyakh vyzovov tsifrovoy ekonomiki [The main trends of the development 
of information technologies within the challenges of the digital economy]. 
\textit{Digital Signal Processing} 1:3--7.
\bibitem{5-gg}
\Aue{Zaichenko, S.} 2008. Tsentry prevoskhodstva v~sisteme sovremennoy nauchnoy 
politiki [Centres of excellence in the system of contemporary science policy]. 
\textit{Foresight and STI Governance} 2(1):42--50.
\bibitem{6-gg}
\Aue{Paulsen,~M.\,F.} 2003. Experiences with 
Learning Management Systems in~113~European institutions. 
\textit{Educ. Technol. Soc.} 6(4):134--148.
\bibitem{7-gg}
\Aue{Villaverde,~J.\,E., D.~Godoy, and A.~Amandi}. 2006. Learning 
styles' recognition in e-learning environments with feed-forward neural networks. 
\textit{J.~Comput. Assist. Lear.} 22(3):197--206.
\bibitem{8-gg}
\Aue{Kardan,~A.\,A., H.~Sadeghi, S.\,S.~Ghidary, and M.\,R.\,F.~Sani.}
 2013. Prediction of student course selection in online higher education 
 institutes using neural network. \textit{Comput. Educ.} 65:1--11.
\bibitem{9-gg}
\Aue{Kose,~U., and A.~Arslan.} 2017. Optimization of self-learning in computer 
engineering courses: An intelligent software system supported by artificial neural 
network and vortex optimization algorithm. \textit{Comput. Appl. 
Eng. Educ.} 25(1):142--156.
\bibitem{10-gg}
\Aue{Gorshenin,~A.} 2018. Toward modern educational IT-ecosystems: 
From learning management systems to digital platforms. 
\textit{10th Congress (International)
on Ultra Modern Telecommunications and
Control Systems and Workshops Proceedings}.
Piscataway, NJ: IEEE. 329--333.
 % Proceedings of the 10$^{th}$ International Congress on Ultra Modern Telecommunications and Control Systems  %and Workshops (ICUMT)-- Piscataway, NJ, USA: IEEE, 2018. -- P.~???--???.
\bibitem{11-gg}
\Aue{Autio,~E., A.-P.~Hameri, and O.~Vuola.} 2004. 
A~framework of industrial knowledge spillovers in big-science centers. 
\textit{Res. Policy} 33(1):107--126.
\bibitem{12-gg}
\Aue{Rapuano,~S., and F.~Zoino.} 2006. 
A~learning management system including laboratory experiments on measurement 
instrumentation. \textit{IEEE T.~Instrum. Meas.} 55(5):1757--1766.
\bibitem{13-gg}
\Aue{Grosse,~R.} 1996. International technology transfer in services. 
\textit{J.~Int. Bus. Stud.} 27(4):781--800.
\bibitem{14-gg}
\Aue{Fortin,~J.-M., and D.\,J.~Currie.} 2013. 
Big science vs.\ little science: How scientific impact scales with funding. 
\textit{PLoS One} 8(6):e65263.
\bibitem{15-gg}
\Aue{Vuola,~O., and A.-P.~Hameri.} 2006. 
Mutually benefiting joint innovation process between industry and big-science. 
\textit{Technovation} 26(1):3--12.
\end{thebibliography}

 }
 }

\end{multicols}

\vspace*{-6pt}

\hfill{\small\textit{Received September 20, 2018}}

%\pagebreak

%\vspace*{-18pt}

\Contrl

\noindent
\textbf{Gorshenin Andrey K.} (b.\ 1986)~--- Candidate of Science (PhD) in physics and
mathematics, associate professor, leading scientist, Institute of Informatics Problems,
Federal Research Center ``Computer Science and Control'' of the Russian Academy of
Sciences, 44-2~Vavilov Str., Moscow 119333, Russian Federation;  
leading scientist, Faculty
of Computational Mathematics and Cybernetics, M.\,V.~Lomonosov Moscow State 
University, GSP-1, Leninskie Gory, Moscow 119991, 
Russian Federation; \mbox{agorshenin@frccsc.ru}
\label{end\stat}

\renewcommand{\bibname}{\protect\rm Литература}       