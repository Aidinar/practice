%\renewcommand{\r}{\mathbb R}
%\newcommand{\eqd}{\stackrel{d}{=}}
%\newcommand{\pto}{\stackrel{P}{\longrightarrow}}

\def\stat{kor+gor+zei}

\def\tit{НОВЫЕ ПРЕДСТАВЛЕНИЯ ОБОБЩЕННОГО РАСПРЕДЕЛЕНИЯ МИТТАГ-ЛЕФФЛЕРА 
В~ВИДЕ СМЕСЕЙ И~ИХ ПРИЛОЖЕНИЯ$^*$}

\def\titkol{Новые представления обобщенного распределения Миттаг-Леффлера 
в~виде смесей и~их приложения}

\def\aut{В.\,Ю.~Королев$^1$, А.\,К.~Горшенин$^2$, А.\,И.~Зейфман$^3$}

\def\autkol{В.\,Ю.~Королев, А.\,К.~Горшенин, А.\,И.~Зейфман}

\titel{\tit}{\aut}{\autkol}{\titkol}

\index{Королев В.\,Ю.}
\index{Горшенин А.\,К.}
\index{Зейфман А.\,И.}
\index{Korolev V.\,Yu.}
\index{Gorshenin A.\,K.} 
\index{Zeifman A.\,I.}



{\renewcommand{\thefootnote}{\fnsymbol{footnote}} \footnotetext[1]
{Работа выполнена при поддержке РФФИ (проект 17-07-00717).}}


\renewcommand{\thefootnote}{\arabic{footnote}}
\footnotetext[1]{Факультет вычислительной математики и~кибернетики
Московского государственного университета им.\ М.\,В.~Ломоносова;
Институт проблем информатики Федерального исследовательского
центра <<Информатика и~управ\-ле\-ние>> Российской академии наук; 
Hangzhou Dianzi University, Китай, \mbox{vkorolev@cs.msu.ru}}
\footnotetext[2]{Институт проблем информатики Федерального исследовательского
центра <<Информатика и~управ\-ле\-ние>> Российской академии наук; факультет
вычислительной математики и~кибернетики Московского государственного университета им.\ 
М.\,В.~Ломоносова,
\mbox{agorshenin@frccsc.ru}}
\footnotetext[3]{Вологодский государственный университет; 
Институт проб\-лем информатики Федерального исследовательского
центра <<Информатика и~управление>> Российской академии наук;
Вологодский научный центр Российской академии наук, \mbox{a\_zeifman@mail.ru}}

%\vspace*{8pt}


\Abst{Приведены новые представления обобщенного
распределения Мит\-таг-Леф\-фле\-ра в~виде смесей. В~част\-ности, показано,
что при значениях <<обобщающего>> параметра, не превосходящих
единицы, обобщенное распределение Мит\-таг-Леф\-фле\-ра является
масштабной смесью полунормальных законов, масштабной смесью
<<обычных>> распределений Мит\-таг-Леф\-фле\-ра или масштабной смесью
обобщенных распределений Мит\-таг-Леф\-фле\-ра с~большими значениями
характеристического показателя. Во всех случаях приведены явные
выражения для смешивающих величин. Полученные представления
позволяют предложить новые алгоритмы моделирования случайных величин (с.в.)\
с обобщенным распределением Мит\-таг-Леф\-фле\-ра и~сформулировать новые
предельные теоремы, в~которых указанное распределение выступает в~качестве предельного.}

\KW{обобщенное распределение Мит\-таг-Леф\-фле\-ра;
масштабная смесь; обобщенное гам\-ма-рас\-пре\-де\-ле\-ние; полунормальное
распределение; устойчивое распределение}

\DOI{10.14357/19922264180411}
  
%\vspace*{4pt}


\vskip 10pt plus 9pt minus 6pt

\thispagestyle{headings}

\begin{multicols}{2}

\label{st\stat}

\section{Введение}

Данная статья продолжает исследования, начатые в~работах~[1--4]. В~статье 
приведены новые представления обобщенного распределения Мит\-таг-Леф\-фле\-ра 
в~виде смесей. Это распределение представляет особый интерес как 
<<тяжелохвостая>>\linebreak модель статистических закономерностей, при которых 
большие значения наблюдаемых характери-\linebreak стик встречаются намного 
чаще, чем предписывает классическая экспоненциальная модель. Оно возникает 
в~некоторых задачах, связанных с~дифференциальными уравнениями дробного порядка 
в физике, астрономии, финансовой математике и~других областях~[5--8]. 
<<Обычное>> распределение Мит\-таг-Леф\-фле\-ра традиционно рассматривается 
вместе с~распределением Линника, поскольку\linebreak характеристическая функция 
(х.ф.)\ распределения Линника имеет такой же аналитический вид, как 
преобразование Лап\-ла\-са--Стилть\-еса (п.~Л.--С.)\ распределения 
Мит\-таг-Леф\-фле\-ра. Поэтому эти распределения обладают многими сходными 
свойствами. В~частности, они геометрически устойчивы, так как являются 
предельными для геометрических случайных сумм независимых одинаково 
распределенных с.в.\ с~бесконечными дисперсиями и~потому представимы в~виде 
масштабных смесей устойчивых законов, в~которых смешивающим выступает 
распределение Вейбулла. Соответственно, обоб\-щен\-ные распределения Линника 
и~Мит\-таг-Леф\-фле\-ра представляют собой масштабные смеси устойчивых законов, 
в~которых смешивающим служит обобщенное гам\-ма-рас\-пре\-де\-ление.

В данной работе приведены альтернативные представления обобщенного
распределения Мит\-таг-Леф\-фле\-ра в~виде смесей. В~частности, показано,
что при значениях <<обобщающего>> параметра, не превосходящих
единицы, обобщенное распределение Мит\-таг-Леф\-фле\-ра~--- это масштабная смесь полунормальных законов, <<обычных>> распределений Миттаг-Леффлера или
обобщенных распределений Мит\-таг-Леф\-фле\-ра с~б$\acute{\mbox{о}}$льшими значениями
характеристического показателя. Во всех случаях приведены явные
выражения для смешивающих величин. Полученные представления
позволяют предложить новые алгоритмы моделирования с.в.\
с~обобщенным распределением Мит\-таг-Леф\-фле\-ра и~сформулировать новые
предельные теоремы, в~которых указанное распределение выступает 
в~качестве предельного.

Аналогичные результаты относительно обобщенного распределения
Линника приведены в~\cite{KorolevGorsheninZeifman2018}, где,
в~частности, показано, что обобщенное распределение Линника~--- масштабная 
смесь нормальных законов со смешивающим распределением типа обобщенного 
распределения Мит\-таг-Леф\-фле\-ра. Здесь этот результат будет использован 
для вывода некоторых свойств обобщенного распределения Мит\-таг-Леф\-флера.

\vspace*{-4pt}

\section{Распределения Миттаг-Леффлера и~Линника}

Пусть
$\alpha\in(0,1]$ и~$M_{\alpha}$~--- неотрицательная с.в.\ 
с~п.~Л.--С.:
\begin{equation}
\psi_{\alpha}(s)\equiv {\sf
E}\exp\{-sM_{\alpha}\}=\left(1+s^{\alpha}\right)^{-1},\enskip
s\ge0\,.\label{e1-gzk}
\end{equation}
Распределения с~п.~Л.--С.~(1) принято называть \textit{распределениями
Мит\-таг-Леф\-фле\-ра}. Происхождение этого названия связано с~тем, что
плотность, соответствующая п.~Л.--С.~(1), имеет вид:

\noindent
\begin{multline*}
f_{\alpha}^{M}(x)=\fr{1}{x^{1-\alpha}}\sum\limits_{n=0}^{\infty}\fr{(-1)^nx^{\alpha
n}}{\Gamma(\alpha n+1)}=-\fr{d}{dx}\,E_{\alpha}(-x^{\alpha})\,,\\
x\ge0,
\end{multline*}

\vspace*{-2pt}

\noindent
где $E_{\alpha}(z)$~--- функция Мит\-таг-Леф\-фле\-ра индекса~$\alpha$,
определяемая как степенной ряд
$$
E_{\alpha}(z)=\sum\limits_{n=0}^{\infty}\fr{z^n}{\Gamma(\alpha
n+1)}\,,\enskip \alpha>0\,,\ \ z\in\mathbb{Z}\,.
$$
Функция распределения (ф.р.), соответствующая плотности
$f_{\alpha}^{M}(x)$, будет обозначаться~$F_{\alpha}^{M}(x)$.

При $\alpha=1$ распределение Мит\-таг-Леф\-фле\-ра превращается 
в~стандартное показательное распределение: $M_1\hm\eqd W_1$. Но при
$\alpha\hm<1$ плот\-ность~(1) имеет хвост, убывающий степенн$\acute{\mbox{ы}}$м
образом: если $0\hm<\alpha\hm<1$, то
$\lim\nolimits_{x\to\infty}
x^{\alpha+1}f_\alpha^{M}(x)\hm=\pi^{-1}\Gamma(\alpha\hm+1)\sin\pi\alpha$
(см., например,~\cite{Kilbas2014}).

Моменты с.в.\ с~распределением Мит\-таг-Леф\-фле\-ра порядков
$\beta\hm\ge\alpha$ бесконечны, но если $0\hm<\beta\hm<\alpha\hm<1$, то ${\sf E}
M_{\alpha}^{\beta}\hm={\Gamma(1\hm+{\beta}/{\alpha})\Gamma(1\hm-{\beta}/{\alpha})}$.

Пусть $\nu>0$, $\alpha\in(0,1]$. Распределение неотрицательной с.в.\
$M_{\alpha,\,\nu}$, соответствующее п.~Л.--С.
$$
\psi_{\alpha,\nu}(s)\equiv {\sf E}
e^{-sM_{\alpha,\,\nu}}=\left(1+s^{\alpha}\right)^{-\nu}\,,\enskip s\ge0\,,
$$
называется \textit{обобщенным распределением Мит\-таг-Леф\-фле\-ра} 
(см.~\cite{MathaiHaubold2011, Joseetal} и~ссылки в~этих работах).

Распределения с~х.ф.~$\mathfrak{f}^{L}_{\alpha}(t)\hm=\left(1\hm+|t|^{\alpha}\right)^{-1}$,
$t\hm\in\mathbb{R}$, где $0\hm<\alpha\hm\le2$, принято называть 
\textit{распределениями Линника} (в~работе~\cite{Pillai1985} предложено
альтернативное менее употребительное название \textit{$\alpha$-Laplace
distribution}). Они были введены Ю.\,В.~Линником в~1953~г.~\cite{Linnik1953}. 
При $\alpha\hm=2$ распределение Линника
превращается в~распределение Лапласа, соответствующее плотности
\begin{equation}
f^{\Lambda}(x)=\fr{1}{2}\,e^{-|x|}\,,\enskip
x\in\mathbb{R}\,.
\label{e2-kgz}
\end{equation}
Лапласовская с.в.\ с~плотностью~(2) и~ее ф.р.\ будут соответственно
обозначаться~$\Lambda$ и~$F^{\Lambda}(x)$.

Случайная величина, имеющая распределение Линника с~параметром~$\alpha$, ее ф.р.\
и~плот\-ность будут соответственно обозначаться~$L_{\alpha}$,
$F_{\alpha}^{L}$ и~$f_{\alpha}^{L}$. При этом $F_2^{L}(x)\hm\equiv
F^{\Lambda}(x)$, $x\hm\in\mathbb{R}$.

Распределения Линника обладают многими интересными свойствами,
которые описаны, например, в~работах~\cite{KotzOstrovskiiHayfavi1995a,
KotzOstrovskiiHayfavi1995b, Lin1994, Anderson1992, Devroye1990}.
Абсолютные моменты порядков $\beta\hm<\alpha$ с.в.~$L_{\alpha}$ имеют
вид:
$$
{\sf E}\left\vert L_{\alpha}\right\vert^{\beta}=
\fr{2^{\beta}}{\sqrt{\pi}}\,\fr{\Gamma\left(1+{\beta}/{\alpha}\right)
\Gamma\left(({1+\beta})/{2}\right)\Gamma\left(1-{\beta}/{\alpha}\right)}
{\Gamma\left(1-{\beta}/{2}\right)}.
$$
В работе~\cite{Jacquesetal1999} показано, что при $0\hm<\alpha\hm<2$
хвосты распределения Линника убывают степенн$\acute{\mbox{ы}}$м образом:

\noindent
$$
\lim\limits_{x\to\infty}x^{\alpha}\left[1-F^{L}_{\alpha}(x)\right]=
\pi^{-1}\Gamma(\alpha)\sin\left(\fr{\pi\alpha}{2}\right)\,.
$$


В работах~\cite{Pakes1998, KotzOstrovskii1996, KorolevZeifman2016,
KorolevZeifmanKMJ} получены разнообразные представления
распределений Линника в~виде смесей. Некоторые из этих представлений
будут приведены и~использованы ниже.

В работе~\cite{Pakes1998} замечено, что \textit{обобщенные
распределения Линника}, задаваемые х.ф.

\vspace*{-6pt}

\noindent
\begin{multline}
\phi_{\alpha,\nu,\theta}(t)=
\left(1+e^{-i\theta\,{\mathrm{sgn}\,t}}|t|^{\alpha}\right)^{-\nu},\\
t\in\mathbb{R}\,,\enskip
|\theta|\le\min\left\{\fr{1}{2}\,\pi\alpha,\,\pi-\fr{1}{2}\,\pi\alpha\right\},\
\nu>0\,,
\label{e3-kgz}
\end{multline}

\vspace*{-1pt}

\noindent
играют видную роль в~некоторых характеризационных задачах
математической статистики. Среди работ, посвященных свойствам этих
распределений и~их применениям, следует упомянуть~[5, 6, 13, 19, 21--25].

В данной работе будут рассматриваться сим\-мет\-рич\-ные распределения,
для которых в~соотношении~(\ref{e3-kgz}) $\theta\hm=0$.

\vspace*{-6pt}

\section{Вспомогательные сведения}

\vspace*{-3pt}

В дальнейшем удобнее вести изложение
не в~терминах распределений, а~в~терминах с.в., предполагая, что
все они заданы на одном вероятностном пространстве
$(\Omega,\mathfrak{A}, {\sf P})$.

Случайная величина со стандартной показательной ф.р.\ будет 
обозначаться~$W_1$: 

\noindent
$$
{\sf P}\left(W_1\hm<x\right)=\left[1\hm-e^{-x}\right]
{\bf 1}(x\hm\ge0)
$$ 
(здесь и~далее символ~${\bf 1}(C)$ обозначает индикатор
множества~$C$). Случайная величина со стандартной нормальной ф.р.~$\Phi(x)$
будет обозначаться~$X$:
$$
{\sf P}(X<x)=\Phi(x)=\fr{1}{\sqrt{2\pi}}\int\limits_{-\infty}^{x}e^{-z^2/2}\,dz\,,
\enskip x\in\mathbb{R}\,.
$$
Функция распределения 
и~плот\-ность строго устойчивого распределения
с~характеристическим показателем~$\alpha$ и~параметром формы~$\theta$, 
определяемого характеристической функцией 
$$
\mathfrak{f}_{\alpha,\theta}(t)=
\exp\left\{-|t|^{\alpha}\exp\{-(1/2)i\pi\theta\alpha\,\mathrm{sign}\,t\}\right\},\
t\in\mathbb{R}\,,
$$
 где $0\hm<\alpha\hm\le2$,
$|\theta|\hm\le\min\left\{1,{2}/{\alpha}\hm-1\right\}$, будут
соответственно обозначаться $G_{\alpha,\theta}(x)$ 
и~$g_{\alpha,\theta}(x)$ (см., например,~\cite{Zolotarev1983}). Любую
с.в.\ с~ф.р.\ $G_{\alpha,\theta}(x)$ будем обозначать
$S_{\alpha,\theta}$. Симметричным строго устойчивым распределениям
соответствует значение $\theta\hm=0$ и~х.ф.~$\mathfrak{f}_{\alpha,0}(t)
\hm=e^{-|t|^{\alpha}}$, $t\hm\in\mathbb{R}$. Отсюда
несложно видеть, что $S_{2,0}\hm\eqd\sqrt{2}X$.

Односторонним строго устойчивым законам, сосредоточенным на
неотрицательной полуоси, соответствуют значения $\theta\hm=1$ 
и~$0\hm<\alpha\hm\le1$. Пары $\alpha\hm=1$, $\theta\hm=\pm1$ отвечают
распределениям, вы\-рож\-ден\-ным в~$\pm1$ соответственно. Остальные
устойчивые распределения абсолютно непрерывны. Явные выражения
устойчивых плотностей в~терминах элементарных функций отсутствуют за
четырьмя исключениями (нормальный закон ($\alpha\hm=2$, $\theta\hm=0$),
распределение Коши ($\alpha\hm=1$, $\theta\hm=0$), распределение Леви
($\alpha\hm=1/2$, $\theta\hm=1$) и~распределение, симметричное 
к~распределению Леви ($\alpha\hm=1/2$, $\theta\hm=-1$)). Выражения
устойчивых плотностей в~терминах функций Фокса (обобщенных
$G$-функ\-ций Мейера) можно найти в~\cite{Schneider1986, UchaikinZolotarev1999}.

Хорошо известно, что если $0\hm<\alpha\hm<2$, то ${\sf E}
|S_{\alpha,\theta}|^{\beta}\hm<\infty$ для любого
$\beta\in(0,\alpha)$, но моментов с.в.~$S_{\alpha,\theta}$ порядков
$\beta\hm>\alpha$ не существует (см., например,~\cite{Zolotarev1983}).
Несмотря на отсутствие явных выражений плотностей устойчивых
распределений в~терминах элементарных функций, можно показать~\cite{KorolevWeibull2016}, 
что для $0\hm<\beta\hm<\alpha\hm<2$
\begin{equation}
{\sf E}\left\vert S_{\alpha,0}\right\vert^{\beta}=\fr{2^{\beta}}{\sqrt{\pi}}\,
\fr{\Gamma\left(({\beta+1})/{2}\right)\Gamma\left(1-{\beta}/{\alpha}\right)}
{\Gamma\left({2}/{\beta}-1\right)}
\label{e4-kgz}
\end{equation}
и для $0<\beta<\alpha\hm< 1$
\begin{equation}
{\sf E}S_{\alpha,1}^{\beta}=
\fr{\Gamma\left(1-{\beta}/{\alpha}\right)}{\Gamma(1-\beta)}\,.
\label{e5-kgz}
\end{equation}

Символы $\eqd$ и~$\Longrightarrow$ будут соответственно обозначать
совпадение распределений и~сходимость по распределению.

\smallskip

\noindent
\textbf{Лемма~1}~\cite[теорема~3.3.1]{Zolotarev1983}. \textit{Пусть
$\alpha\hm\in(0,2]$, $\alpha'\hm\in(0,1]$. Тогда $S_{\alpha\alpha',0}\hm\eqd
S_{\alpha,0}S_{\alpha',1}^{1/\alpha}$, где с.в.\ в~правой части
независимы}.

\smallskip

\noindent
\textbf{Следствие~1.}\ Симметричное строго устойчивое распределение
с характеристическим показателем~$\alpha$ является масштабной смесью
нормальных законов, в~которой смешивающим служит одностороннее
строго устойчивое распределение с~характеристическим показателем
$\alpha/2$:
\begin{equation}
S_{\alpha,0}\eqd X\sqrt{2S_{\alpha/2,1}}\,, 
\label{e6-kgz}
\end{equation}
где с.в.\ в~правой части независимы.

\smallskip

Случайная величина, име\-ющая гам\-ма-рас\-пре\-де\-ле\-ние с~параметром формы $\nu\hm>0$ 
и~параметром масштаба $\lambda\hm>0$, будет обозначаться~$G_{\nu,\lambda}$,
\begin{equation*}
{\sf P}(G_{\nu,\lambda}<x)=\int\limits_{0}^{x}g(z;\nu,\lambda)\,dz\,,
\end{equation*}
где
$$
g(x;\nu,\lambda)=\fr{\lambda^\nu}{\Gamma(\nu)}\,x^{\nu-1}e^{-\lambda
x}\,,\enskip x\ge0\,,
$$
В этих обозначениях, очевидно, $G_{1,1}\hm\eqd W_1$.

Гамма-распределение~--- это частный случай обобщенных
гамма-распределений, введенных в~работе~\cite{Stacy1962} как единый
класс, одновременно содержащий гам\-ма-рас\-пре\-де\-ле\-ние и~распределение
Вейбулла. \textit{Обобщенное гам\-ма-рас\-пре\-де\-ле\-ние}~--- это абсолютно
непрерывное распределение, плотность которого имеет вид:
$$
\overline{g}(x;\nu,\alpha,\lambda)=\fr{|\alpha|\lambda^\nu}{\Gamma(\nu)}\,x^{\alpha
\nu-1}e^{-\lambda x^{\alpha}}\,,\enskip x\ge0\,,
$$
где $\alpha\in\mathbb{R}$, $\lambda\hm>0$, $\nu\hm>0$.

Случайная величина с~плот\-ностью $\overline{g}(x;\nu,\alpha,\lambda)$ будет
обозначаться $\overline{G}_{\nu,\alpha,\lambda}$. Легко видеть, что
$$
\overline{G}_{\nu,\alpha,\mu}\eqd G_{\nu,\mu}^{1/\alpha}\eqd
\mu^{-1/\alpha}G_{\nu,1}^{1/\alpha}
\eqd\mu^{-1/\alpha}\overline{G}_{\nu,\alpha,1}\,.
$$

Для с.в.\ с~\textit{распределением Вейбулла}, частным случаем
обобщенных гам\-ма-рас\-пре\-де\-ле\-ний, соответствующим плотности
$\overline{g}(x;1,\alpha,1)$ и~ф.р.\
$\left[1\hm-e^{-x^{\alpha}}\right]{\bf 1}(x\hm\ge0)$ с~$\alpha\hm>0$, будет
использовано особое обозначение~$W_{\alpha}$. Таким образом,
$G_{1,1}\hm\eqd W_1$. Очевидно, $W_1^{1/\alpha}\hm\eqd W_{\alpha}$.

Несложно убедиться, что если $\gamma\hm>0$ и~$\gamma'\hm>0$, то 
\begin{multline*}
{\sf P} \left(W_{\gamma'}^{1/\gamma}\ge x\right)=
{\sf P}\left(W_{\gamma'}\ge x^{\gamma}\right)=
e^{-x^{\gamma\gamma'}}={}\\
{}={\sf P}\left(W_{\gamma\gamma'}\ge x\right)\,,\enskip
x\ge 0\,,
\end{multline*}
 т.\,е.\ при любых $\gamma\hm>0$ и~$\gamma'\hm>0$
\begin{equation}
W_{\gamma\gamma'}\eqd W_{\gamma'}^{1/\gamma}\,.
\label{e7-kgz}
\end{equation}

В статье~\cite{Gleser1989} было показано, что каждое
гам\-ма-рас\-пре\-де\-ле\-ние с~параметром формы, не пре\-вос\-ходящим единицы,
является смешанным по\-ка\-зательным. Сформулируем этот результат в~виде\linebreak
сле\-ду\-ющей леммы.

\smallskip

\noindent
\textbf{Лемма~2}~\cite{Gleser1989}. \textit{Плотность гам\-ма-рас\-пре\-де\-ле\-ния
$g(x;\nu,\mu)$ с~$0\hm<\nu\hm<1$ может быть представлена в~виде}

\noindent
$$
g(x;\nu,\mu)=\int\limits_{0}^{\infty}ze^{-zx}p(z;\nu,\mu)\,dz\,,
$$

\vspace*{-2pt}

\noindent
\textit{где}

\noindent
\begin{equation}
p(z;\nu,\mu)=\fr{\mu^\nu}{\Gamma(1-\nu)\Gamma(\nu)}\,
\fr{\mathbf{1}(z\ge\mu)}{(z-\mu)^\nu z}\,.\label{e8-kgz}
\end{equation}
\textit{Более того, гам\-ма-рас\-пре\-де\-ле\-ние с~параметром формы $\nu\hm>1$ не может
быть представлено в~виде смешанного показательного закона.}

\smallskip

\noindent
\textbf{Лемма~3}~\cite{Korolev2017}. \textit{Для $\nu\hm\in(0,1)$ пусть
$G_{\nu,\,1}$ и~$G_{1-\nu,\,1}$~--- независимые гам\-ма-рас\-пре\-де\-лен\-ные
с.в. Пусть $\mu\hm>0$. Тогда плотность $p(z;\nu,\mu)$, определенная выражением}~(\ref{e8-kgz}), 
\textit{соответствует с.в.}
\begin{multline*}
Z_{\nu,\mu}=\fr{\mu(G_{\nu,\,1}+G_{1-\nu,\,1})}{G_{\nu,\,1}}\eqd{}\\
{}\eqd \mu
Z_{\nu,1}\eqd\mu\left(1+\fr{1-\nu}{\nu}\, Q_{1-\nu,\nu}\right)\,,
\end{multline*}
\textit{где $Q_{1-\nu,\nu}$~--- с.в.\ с~распределением Сне\-де\-ко\-ра--Фи\-ше\-ра,
соответствующим плотности}
\begin{multline*}
q(x;1-\nu,\nu)=\fr{(1-\nu)^{1-\nu}\nu^\nu}{\Gamma(1-\nu)\Gamma(\nu)}
\,\fr{1}{x^{\nu}[\nu+(1-\nu)x]},\\  x\ge0.
\end{multline*}


\smallskip

Несложно видеть, что $G_{\nu,1}\hm+G_{1-\nu,1}\hm\eqd W_1$. Однако
числитель и~знаменатель в~определении с.в.~$Z_{\nu,\mu}$ не
являются независимыми~с.в.

\smallskip

Фактически леммы~2 и~3 означают, что если $\nu\in(0,1)$, то
\begin{equation}
G_{\nu,\,\mu}\eqd W_1  Z_{\nu,\,\mu}^{-1}\,,\label{e9-kgz}
\end{equation}
где с.в.\ в~правой части независимы.

\smallskip

Следующее утверждение уже стало фольклором. Без претензий на
первенство его доказательство приведено в~\cite{KorolevZeifmanKMJ}
как упражнение.

\smallskip

\noindent
\textbf{Лемма~4.} \textit{При каждом $\delta\hm\in(0,1]$ распределение
Мит\-таг-Леф\-фле\-ра с~параметром~$\delta$ является масштабной смесью
одностороннего устойчивого закона, в~которой смешивающее
распределение~--- распределение Вейбулла с~параметром}
$\delta/2$$:$ 

\noindent
$$
M_{\delta}\eqd S_{\delta,1}W_{\delta}\eqd
S_{\delta,1}\sqrt{W_{\delta/2}}\,,
$$
 \textit{где с.в.\ в~правой части
независимы}.

\smallskip

Пусть $\rho\in(0,1)$. В~статье~\cite{Kozubowski1998} было показано,
что функция
\begin{equation}
f_{\rho}^{K}(x)=\fr{\sin(\pi\rho)}{\pi\rho[x^2+2x\cos(\pi\rho)+1]}\,,\enskip\!
 x\in(0,\infty),\!\!\!\label{e10-kgz}
\end{equation}
является плотностью вероятностей на $(0,\infty)$. Случайную величину 
с~плот\-ностью~(\ref{e10-kgz}) обозначим~$K_{\rho}$.

\smallskip

\noindent
\textbf{Лемма~5}~\cite{Kozubowski1998}. \textit{Пусть
$0\hm<\delta\hm<\delta'\hm\le1$ и~$\rho\hm=\delta/\delta'\hm<1$. Тогда
$M_{\delta}\hm\eqd M_{\delta'}K_{\rho}^{1/\delta}$, где с.в.\ в~правой
части независимы.}

\smallskip

В статье~\cite{KorolevZeifmanKMJ} было показано, что при любом
$\delta\hm\in(0,1)$

\noindent
\begin{equation}
K_{\delta}^{1/\delta}\eqd
\fr{S_{\delta,1}}{S'_{\delta,1}}\,,\label{e11-kgz}
\end{equation}

\vspace*{-1pt}

\noindent
где $S'_{\delta,1}\eqd S_{\delta,1}$ и~где с.в.\ в~правой части
независимы. Таким образом, при $\delta'\hm=1$ из леммы~5 вытекает

\smallskip

\noindent
\textbf{Следствие~2}~\cite{Kozubowski1998, KorolevZeifmanKMJ}. 
Пусть $0<\delta<1$. Тогда распределение Миттаг-Леффлера с~параметром
$\delta$ является смешанным показательным, т.\,е.\ справедливо представление:

\noindent
$$
M_{\delta}\eqd W_1 K_{\delta}^{1/\delta}\eqd
W_1 \fr{S_{\delta,1}}{S'_{\delta,1}},
$$

\vspace*{-2pt}

\noindent
где с.в.\ в~правой части независимы.

\smallskip

Пусть $0<\alpha\hm<\alpha'\hm\le2$. В~статье~\cite{KotzOstrovskii1996}
было показано, что функция
$$
f_{\alpha,\alpha'}^{Q}(x)= \fr{\alpha'\sin(\pi\alpha/\alpha')
x^{\alpha-1}}{\pi[1+x^{2\alpha}+2x^{\alpha}\cos(\pi\alpha/\alpha')]}\,,\enskip
 x>0\,,
$$
является плотностью вероятностей на $(0,\infty)$. Пусть
$Q_{\alpha,\alpha'}$~--- с.в.\ с~плот\-ностью
$f_{\alpha,\alpha'}^{Q}(x)$.

\smallskip

\noindent
\textbf{Лемма~6}~\cite{KotzOstrovskii1996}. \textit{Пусть
$0\hm<\alpha\hm<\alpha'\hm\le2$. Тогда $L_{\alpha}\hm\eqd
L_{\alpha'}Q_{\alpha,\alpha'}$, где с.в.\ в~правой части
независимы}.

\smallskip

При $\alpha'=2$ получаем

\smallskip

\noindent
\textbf{Следствие~3}~\cite{KotzOstrovskii1996}. Пусть
$0\hm<\alpha\hm<2$. Тогда распределение Линника с~параметром~$\alpha$
является масштабной смесью распределений Лапласа, соответствующих
плотности~(2): $L_{\alpha}\hm\eqd \Lambda Q_{\alpha,2}$, где с.в.\ 
в~правой части независимы.

\smallskip

В работе~\cite{Devroye1990} было доказано следующее утверждение.
Здесь оно уточнено с~учетом~(\ref{e7-kgz}).

\smallskip

\noindent
\textbf{Лемма~7}~\cite{Devroye1990}. \textit{При любом} $\alpha\hm\in(0,2]$
$$
L_{\alpha}\eqd S_{\alpha,0} W_1^{1/\alpha}\,,
$$ 
где с.в.\  \textit{в~правой части независимы}.

\smallskip

\noindent
\textbf{Лемма~8}~\cite{KorolevZeifmanKMJ}. \textit{Пусть $\alpha\hm\in(0,2]$,
$\alpha'\hm\in(0,1]$. Тогда}
$$
L_{\alpha\alpha'}\eqd S_{\alpha,0}M_{\alpha'}^{1/\alpha}\,.
$$

%\smallskip

\noindent
\textbf{Следствие~4}~\cite{KorolevZeifmanKMJ}.  При каждом
$\alpha\hm\in(0,2]$ распределение Линника с~параметром~$\alpha$
является масштабной смесью нормальных законов, в~которой сме\-ши\-ва\-ющее
распределение~---  распределение Мит\-таг-Леф\-фле\-ра с~параметром
$\alpha/2$:
\begin{equation}
L_{\alpha}\eqd X\sqrt{2M_{\alpha/2}}\,,
\label{e12-kgz}
\end{equation}
где с.в.\ в~правой части независимы.

\smallskip

\noindent
\textbf{Лемма~9}~\cite{KorolevZeifmanKMJ}. При каждом
$\alpha\hm\in(0,2]$ справедливо представление:
$$
L_{\alpha}\eqd \Lambda\sqrt{\fr{S_{\alpha/2,1}}{S'_{\alpha/2,1}}}\,,
$$
где с.в.\ в~правой части независимы.

\smallskip

В статье~\cite{KorolevZeifmanKMJ} показано, что если $S_{\alpha,1}$
и~$S'_{\alpha,1}$~--- независимые с.в.\ с~одним и~тем же
односторонним строго устойчивым распределением с~характеристическим
показателем $\alpha\hm\in(0,1)$, то $S_{\alpha,1}/S'_{\alpha,1}\hm\eqd
K_{\alpha}^{1/\alpha}\hm\eqd Q_{2\alpha,2}^2$, т.\,е.\ плот\-ность~$p_{\alpha}(x)$ 
отношения $S_{\alpha,1}/S'_{\alpha,1}$ имеет вид:
$$
p_{\alpha}(x)=f_{\alpha,1}^{Q}(x)=\fr{\sin(\pi\alpha)x^{\alpha-1}}
{\pi[1+x^{2\alpha}+2x^{\alpha}\cos(\pi\alpha)]}\,,\ 
 x>0\,.
$$

\smallskip

\noindent
\textbf{Лемма~10}~\cite{KorolevZeifmanKMJ}. \textit{При любом
$\delta\hm\in(0,1]$ распределение Мит\-таг-Леф\-фле\-ра с~параметром~$\delta$
является масштабной смесью полунормальных законов}:
$$
M_{\delta}\eqd
|X| \sqrt{2W_1}\, \fr{S_{\delta,1}}{S'_{\delta,1}},
$$
\textit{где с.в.\ в~правой части независимы}.

\section{Представления обобщенных распределений Миттаг-Леффлера 
и~Линника в~виде смесей}

Представленные здесь теоремы обобщают 
и~уточняют некоторые результаты работ~\cite{Pakes1998, LimTeo2009,
Mathai2010, MathaiHaubold2011, Joseetal}. Некоторые известные
результаты сформулируем в~виде лемм.

\smallskip

\noindent
\textbf{Лемма~11}~\cite{Devroye1990, Pakes1998}. \textit{Пусть
$\alpha\hm,\in(0,2]$, $\nu\hm>0$. Тогда} 
$$
L_{\alpha,\nu}\eqd S_{\alpha,0} G_{\nu,1}^{1/\alpha}\eqd
S_{\alpha,0}\overline{G}_{\nu,\alpha,1}\,.
$$


\smallskip

Из леммы~11 и~соотношения~(\ref{e4-kgz}) получаем

\smallskip

\noindent
\textbf{Следствие~5.} Для $0\hm<\beta\hm<\alpha\hm<2$
$$
{\sf E}\left\vert L_{\alpha,\nu}\right\vert^{\beta}=
\fr{2^{\beta}\Gamma(({\beta+1})/{2})\Gamma(1-{\beta}/{\alpha})
\Gamma(\nu+{\beta}/{\alpha})}{\sqrt{\pi}
\Gamma({2}/{\beta}-1)\Gamma(\nu)}\,.
$$

\smallskip

\noindent
\textbf{Лемма~12}~\cite{Mathai2010, MathaiHaubold2011, Joseetal}. 
\textit{Пусть $\delta\hm\in(0,1]$ и~$\nu\hm>0$. Тогда} 
$$
M_{\delta,\nu}\eqd S_{\delta,1}\overline{G}_{\nu,\delta,1}\,.
$$

\smallskip

Из леммы~12 и~соотношения~(\ref{e5-kgz}) получаем

\smallskip

\noindent
\textbf{Следствие~6.} Для $0<\beta<\delta<1$
$$
{\sf E}
M_{\delta,\nu}^{\beta}=\fr{\Gamma(1-{\beta}/{\delta})\Gamma
(\nu+{\beta}/{\delta})}{\Gamma(1-\beta)\Gamma(\nu)}\,.
$$

\smallskip

Из следствия~1 (см.\ соотношение~(\ref{e6-kgz})) вытекает, что для $\nu\hm>0$ 
и~$\alpha\hm\in(0,2]$
$$
L_{\alpha,\nu}\eqd X \sqrt{2S_{\alpha/2,1}}\,
G_{\nu,1}^{1/\alpha}\eqd
X \sqrt{2S_{\alpha/2,1} \overline{G}_{\nu,\alpha/2,1}}\,,
$$
т.\,е.\ обобщенное распределение Линника является масштабной смесью
нормальных законов. При этом согласно лемме~12 смешивающим
распределением служит обобщенное распределение Мит\-таг-Леф\-фле\-ра.
Таким образом, по аналогии со следствием~4 получаем следующее
утверждение.

\smallskip

\noindent
\textbf{Теорема~1.} \textit{Если $\alpha\hm\in(0,2]$ и~$\nu\hm>0$, то
$L_{\alpha,\nu}\eqd$\linebreak $\eqd X \sqrt{2M_{\alpha/2,\,\nu}}$, где с.в.\ 
в~правой части независимы}.

\smallskip

Пусть $\alpha\hm\in(0,2]$, $\alpha'\hm\in(0,1)$ и~$\nu\hm>0$. Используя леммы~1 и~11, 
получаем следующую цепочку соотношений:
\begin{multline*}
L_{\alpha\alpha',\,\nu}\eqd S_{\alpha\alpha',\,0}
G_{\nu,1}^{1/\alpha}\eqd S_{\alpha,\,0}
S_{\alpha',\,1}^{1/\alpha} G_{\nu,1}^{1/\alpha}\eqd{}\\
{}\eqd
L_{\alpha,\,\nu} S_{\alpha',\,1}^{1/\alpha}\,.
\end{multline*}
Следовательно, справедливо следующее утверждение.

\smallskip

\noindent
\textbf{Теорема~2.} \textit{Пусть $\alpha\hm\in(0,2]$, $\alpha'\hm\in(0,1)$ 
и~$\nu\hm>0$. Тогда обобщенное распределение Линника является масштабной
смесью обобщенных распределений Линника с~б$\acute{\mbox{о}}$льшим
характеристическим параметром}: 
$$
L_{\alpha\alpha',\,\nu}\eqd
L_{\alpha,\,\nu} S_{\alpha',\,1}^{1/\alpha}\,,
$$
 \textit{где с.в.\ 
в~правой части независимы}.

\smallskip

Пусть теперь $\nu\in(0,1]$. Из представления~(\ref{e9-kgz}) и~леммы~7 получаем
цепочку соотношений:
\begin{multline*}
L_{\alpha,\nu}\eqd S_{\alpha,0} G_{\nu,1}^{1/\alpha}\eqd
S_{\alpha,0} W_1^{1/\alpha} Z_{\nu,1}^{-1/\alpha}\eqd{}\\
{}\eqd
S_{\alpha,0} W_{\alpha} Z_{\nu,1}^{-1/\alpha}\eqd
L_{\alpha} Z_{\nu,1}^{-1/\alpha}\,,
\end{multline*}
из которой вытекает следующее утверждение, связывающее обобщенное 
и~<<обычное>> распределения Линника.

\smallskip

\noindent
\textbf{Теорема~3}. \textit{Если $\nu\hm\in(0,1)$ и~$\alpha\hm\in(0,2]$, то}
\begin{equation}
L_{\alpha,\nu}\eqd L_{\alpha} Z_{\nu,1}^{-1/\alpha}\,,\label{e13-kgz}
\end{equation}
\textit{где с.в.\ в~правой части независимы. Другими словами, при
$\nu\hm\in(0,1]$ и~$\alpha\hm\in(0,2]$ обобщенное распределение Линника
является масштабной смесью обычных распределений Линника.}

\smallskip

Из~(\ref{e13-kgz}) и~леммы~9 получаем следующее пред\-став\-ле\-ние обобщенного
распределения Линника в~виде масштабной смеси распределений Лапласа:
$$
L_{\alpha,\nu}\eqd \Lambda
Z_{\nu,1}^{-1/\alpha}\sqrt{\fr{S_{\alpha/2,1}}{S'_{\alpha/2,1}}}\,.
$$
Более того, из следствия~4 (см.\ соотношение~(\ref{e12-kgz})) вытекает, что при
$\nu\hm\in(0,1)$ и~$\alpha\hm\in(0,2]$
\begin{equation}
L_{\alpha,\nu}\eqd X
Z_{\nu,1}^{-1/\alpha}\sqrt{2M_{\alpha/2}}\,.\label{e14-kgz}
\end{equation}
Поскольку масштабные смеси нормальных законов идентифицируемы~\cite{Teicher1961}, 
из~(\ref{e14-kgz}) и~теоремы~1 получаем следующее
представление обобщенного распределения Мит\-таг-Леф\-фле\-ра в~виде
масштабной смеси <<обычных>> распределений Мит\-таг-Леф\-флера.

\smallskip

\noindent
\textbf{Теорема~4}. \textit{Пусть $\nu\hm\in(0,1)$ и~$\delta\hm\in(0,1]$. Тогда}
$$
M_{\delta,\nu}\eqd Z_{\nu,1}^{-1/\delta} M_{\delta}\,,
$$ \textit{где с.в.\  в~правой части независимы}.

\smallskip

Пусть $\delta\in(0,1]$. Из леммы~12 вытекает, что
$$
M_{\delta,\nu}\eqd S_{\delta,1} \overline{G}_{\nu,\delta,\,1}\eqd
S_{\delta,1} G_{\nu,1}^{1/\delta}\,.
$$

 Теперь есть все
инструменты, позволяющие получить аналог теоремы~2 для распределений
Мит\-таг-Леф\-флера.

Пусть $\alpha\in(0,2]$, $\alpha'\hm\in(0,1)$ и~$\nu\hm>0$. Из тео\-ремы~1
вытекает, что $L_{\alpha\alpha',\nu}\eqd
X\sqrt{2M_{\alpha\alpha'/2,\nu}}$ и~$L_{\alpha,\nu}\eqd
X\sqrt{2M_{\alpha/2,\nu}}$. Из теоремы~2 вытекает, что
\begin{multline*}
X\sqrt{2M_{\alpha\alpha'/2,\nu}}\eqd
L_{\alpha\alpha',\nu}\eqd{}\\
{}\eqd L_{\alpha,\nu}
S_{\alpha',1}^{1/\alpha}\eqd X\sqrt{2M_{\alpha/2,\nu}}
S_{\alpha',1}^{1/\alpha}.
\end{multline*}
Следовательно, в~силу идентифицируемости масштабных смесей
нормальных законов $M_{\alpha\alpha'/2,\nu}\hm\eqd
M_{\alpha/2,\nu} S_{\alpha',1}^{2/\alpha}$. Поэтому,
переобозначив $\alpha/2\hm=\delta$, $\alpha'\hm=\delta'$, получаем
следующий результат.

\smallskip

\noindent
\textbf{Теорема~5.} \textit{Пусть $\delta\hm\in(0,1]$, $\delta'\hm\in(0,1)$ 
и~$\nu\hm>0$. Тогда}
$$
M_{\delta\delta',\nu}\eqd M_{\delta,\nu}
S_{\delta',1}^{1/\delta}\,,
$$
\textit{где с.в.\ в~правой части независимы}.

\smallskip

Другими словами, любое обобщенное распределение Мит\-таг-Леф\-фле\-ра
является масштабной \mbox{смесью} обобщенных распределений Мит\-таг-Леф\-фле\-ра
с~б$\acute{\mbox{о}}$льшим характеристическим параметром.

\smallskip

Комбинируя утверждения теорем~4 и~5, по\-лу\-чаем

\smallskip

\noindent
\textbf{Следствие~7.} Пусть $\delta\hm\in(0,1]$, $\delta'\hm\in(0,1)$ 
и~$\nu\hm\in(0,1)$. Тогда
$$
M_{\delta\delta',\,\nu}\eqd M_{\delta}
\left(\fr{S_{\delta',\,1}}{Z_{\nu,\,1}}\right)^{1/\delta}\,,
$$
где с.в.\ в~правой части независимы.

\smallskip

Из теоремы~4, лемм~4 и~5 вытекает, что обобщенное распределение
Мит\-таг-Леф\-фле\-ра допускает представление в~виде смешанного
распределения Вейбулла.

\smallskip

\noindent
\textbf{Теорема~6.} \textit{Если $\nu\hm\in(0,1)$ и~$0\hm<\delta\hm<\delta'\hm\le1$,
то}

\noindent
$$
M_{\delta,\nu}\eqd W_{\delta'}
S_{\delta',1}\left(\fr{K_{\delta/\delta'}}{Z_{\nu,\,1}}\right)^{1/\delta}\,,
$$
\textit{где все с. в. в~правой части независимы.}

\smallskip

Из теоремы~4 и~леммы~10 вытекает, что обобщенное распределение
Мит\-таг-Леф\-фле\-ра допускает представление в~виде масштабной смеси
полунормальных законов.

\smallskip

\noindent
\textbf{Теорема~7}. \textit{Пусть $\nu\hm\in(0,1)$ и~$\delta\hm\in(0,1]$. Тогда}
$$
M_{\delta,\nu}\eqd |X|
\fr{\sqrt{2W_1}}{Z_{\nu,1}^{1/\delta}}\,
\fr{S_{\delta,1}}{S'_{\delta,1}},
$$
\textit{где с.в.\ в~правой части независимы.}

\vspace*{-6pt}

\section{Сходимость распределений экстремальных порядковых статистик 
в~выборках случайного объема к~обобщенному распределению Миттаг-Леффлера}

Хорошо известно, что при достаточно общих условиях
распределение Вейбулла может быть предельным для линейно
преобразованных экстремальных порядковых статистик. Этот факт вкупе
с теоремой~6 позволяет убедиться, что обобщенное распределение
Мит\-таг-Леф\-фле\-ра может служить предельным для экстремальных
порядковых статистик в~выборках случайного объема.

В книге~\cite{GnedenkoKorolev1996} предложено описывать эволюцию
неоднородных хаотических стохастических процессов при помощи моделей
вида обобщенных дважды стохастических пуассоновских процессов
(обобщенных процессов Кокса). В~соответствии с~таким\linebreak
 подходом поток
информативных событий, каж\-дое из которых генерирует очередное
наблюдение, описывается стохастическим точечным процессом~$P(U(t))$,
где $P(t)$, $t\hm\geq0$,~--- однородный\linebreak
 пуассоновский процесс с~единичной
интенсивностью, а~$U(t)$, $t\hm\geq0$,~--- независимый от~$P(t)$
случайный процесс, такой что $U(0)\hm=0$, ${\sf P}(U(t)\hm<\infty)\hm=1$ для
любого $t\hm>0$, траектории~$U(t)$ не убывают и~непрерывны справа.
Процесс~$P(U(t))$, $t\hm\geq0$, называется дважды стохастическим
пуассоновским процессом (процессом Кокса)~\cite{Grandell1976}.

В рамках такой модели при каждом~$t$ с.в.~$P(U(t))$ имеет смешанное
пуассоновское распределение. Для наглядности рассмотрим ситуацию 
с~дискретным временем~$t$: $U(t)\hm=U(n)\hm=U_n$, $n\hm\in\mathbb{N}$, где
$\{U_n\}_{n\ge1}$~--- неограниченно возрастающая последовательность
неотрицательных с.в.\ такая, что $U_{n+1}(\omega)\hm\ge U_{n}(\omega)$
для каждого $\omega\hm\in\Omega$, $n\hm\ge1$. При этом асимптотика
$n\to\infty$ может быть интерпретирована как то, что интенсивность
потока информативных событий неограниченно возрастает.

Из сделанных выше предположений вытекает, что с.в.~$U_n$ независима
от стандартного пуассоновского процесса~$P(t)$, $t\hm\ge0$. Для каждого
$n\hm\in\mathbb{N}$ положим $N_n\hm=P(U_n)$, $n\hm\ge1$. Очевидно, что так
определенная с.в.~$N_n$ имеет смешанное пуассоновское распределение
\begin{multline*}
{\sf P}(N_n=k)={\sf P}\left(P(U_n)=k\right)={}\\
{}=\int\limits_{0}^{\infty}
e^{-nz}\fr{(nz)^k}{k!}\,d{\sf P}(U_n<z)\,,\enskip
 k=0,1,\ldots
\end{multline*}

Пусть $X_1,X_2,\ldots $~--- независимые одинаково распределенные с.в.\ 
с~общей ф.р.\ $F(x)\hm={\sf P}(X_i\hm<x)$, $x\hm\in\mathbb{R}$, $i\hm\ge1$.
Обозначим $\mathrm{lext}(F)\hm=\inf\{x:\,F(x)>0\}$. Предположим, что
при\linebreak
 каждом $k\hm\in\mathbb{N}$ с.в.~$N_k$ независима от
последовательности $X_1,X_2,\ldots$ В~книге~\cite{KorolevSokolov2008} 
доказано сле\-ду\-ющее утверждение.

\smallskip

\noindent
\textbf{Лемма~13.} \textit{Предположим, что
существуют неограниченно возрастающая последовательность
положительных чисел $\{d_n\}_{n\ge1}$ и~неотрицательная с.в.~$U$
такие, что $U_n/d_n\hm\Longrightarrow U$. Также пусть $\mathrm{lext}
(F)\hm>-\infty$ и~ф.р.\ $A_F(x)\hm=F\left( \mathrm{lext}
(F)\hm-x^{-1}\right)$ удовлетворяет условию$:$ при каждом} $x\hm>0$

\noindent
\begin{equation}
\lim_{y\to\-\infty}\fr{A_F(yx)}{A_F(y)}=x^{-\delta'}\label{e15-kgz}
\end{equation}
\textit{для некоторого положительного числа~$\delta'$. Тогда существуют
числа~$a_n$ и~$b_n$ такие, что}
\begin{multline*}
{\sf P}\left(\min_{1\le j\le N_n}X_j-a_n<b_nx\right) \Longrightarrow{}\\
{} \Longrightarrow
\left[1-\int\limits_{0}^{\infty}e^{-u x^{\delta'}}d{\sf P}
(U<u)\right]\mathbf{1}(x\ge 0)\,.
\end{multline*}
\textit{При этом числа~$a_n$ и~$b_n$ можно определить как}
\begin{equation}
\left.
\begin{array}{l}
\hspace*{-2mm}a_n={\mathrm{lext}}(F)\,;\\[6pt]
  \hspace*{-2mm}b_n=\sup\left\{x:\ F(x)\le
d_n^{-1}\right\}-{\mathrm{lext}}(F)\,,\  n\ge1\,.
\end{array}\!\!
\right\}
\label{e16-kgz}
\end{equation}

\smallskip

\noindent
\textbf{Теорема~8.} \textit{Пусть $\nu\hm\in(0,1)$, $\delta\hm\in(0,1)$. Для
того чтобы существовали числа $a_n\hm\in\mathbb{R}$ и~$b_n\hm>0$ такие, что}
$$
\fr{1}{b_n}\left(\min\limits_{1\le j\le N_n}X_j-a_n\right)\Longrightarrow
M_{\delta,\nu}\,,
$$
\textit{достаточно, чтобы были выполнены следующие условия}:
\begin{enumerate}[(1)]
\item \textit{существует число $\delta'\hm\in(\delta,1]$ такое, что
ф.р.~$F$ принадлежит области $\min$-при\-тя\-же\-ния распределения
Вейбулла с~параметром формы $\delta'\hm\in(0,1]$, т.\,е.\ $\mathrm{lext}(F)
\hm>-\infty$ и~выполнено условие}~(\ref{e15-kgz});

\item \textit{существует неограниченно возрастающая
последовательность $\{d_n\}_{n\ge1}$ такая, что}
$U_n/d_n\hm\Longrightarrow S_{\delta',1}^{-\delta'}
\left(K_{\delta/\delta'}Z_{\nu,1}\right)^{\delta'/\delta}$.
\end{enumerate}

\textit{При этом числа~$a_n$ и~$b_n$ могут быть определены 
в~соответствии с}~(\ref{e16-kgz}).

\smallskip

\noindent
Д\,о\,к\,а\,з\,а\,т\,е\,л\,ь\,с\,т\,в\,о\,.\ \ Требуемое утверждение является
непосредственным следствием леммы~13 и~тео\-ре\-мы~6 с~учетом
соотношения $K_{\delta/\delta'}^{-1}\hm\eqd K_{\delta/\delta'}$,
вытекающего из~(\ref{e11-kgz}).

\vspace*{-6pt}

\section{Сходимость распределений максимальных случайных сумм 
к~обобщенному распределению Миттаг-Леффлера}

В этом разделе будет
показано, что обобщенное распределение Мит\-таг-Леф\-фле\-ра может быть
предельным для максимальных или минимальных\linebreak
 случайных сумм, а~также
абсолютных величин\linebreak случайных сумм независимых с.в.\ \textit{с~конечными\linebreak
дис\-пер\-си\-ями}. Основную роль будет играть теоре-\linebreak ма~7,
 уста\-нав\-ли\-ва\-ющая
возможность представления обобщен\-но\-го распределения Мит\-таг-Леф\-фле\-ра
в~виде масштабной смеси полунормальных распределений.

Рассмотрим независимые, не обязательно одинаково распределенные с.в.\
 $X_1,X_2,\ldots $ с~${\sf E}X_i\hm=0$ и~$0\hm<\sigma^2_i\hm=
 {\sf D}X_i\hm<\infty$, $i\hm\in\mathbb{N}$. Для $n\hm\ge1$ обозначим 
\begin{align*}
 \overline S^*_n&=\max\limits_{1\le i\le n}S^*_i;\enskip 
 \underline S^*_n=\min\limits_{1\le i\le
n}S^*_i;\\
B_n^2&=\sigma_1^2+\cdots+\sigma_n^2\,.
\end{align*}

Предположим, что с.в.\
 $X_1,X_2,\ldots $ удовлетворяют условию Линдеберга: для любого
$\tau\hm>0$
\begin{equation}
\lim\limits_{n\to\infty}\fr{1}{B^2_n}\sum\limits_{i=1}^{n}\int\limits_{|x|\ge\tau
B_n} x^2\,d{\sf P}(X_i<x)=0\,.\label{e17-kgz}
\end{equation}
Как известно, при таких условиях 
\begin{align*}
{\sf P}\big(\overline S^*_n<B_nx\big)&\Longrightarrow \Psi(x)\equiv{\sf P}
(|X|<x)=2\Phi(x)-1;\\
{\sf P}\big(\underline
S^*_n<B_nx\big)&\Longrightarrow 1-\Psi(-x)\,.
\end{align*}

Пусть $N_1,N_2,\ldots $~--- последовательность неотрицательных
целочисленных с.в.\ таких, что при каж\-дом $n\hm\in\mathbb{N}$ с.в.\
$N_n,X_1,X_2,\ldots $ независимы. Для $n\hm\in\mathbb{N}$ положим
\begin{align*}
S^*_{N_n}&=X_1+\cdots +X_{N_n}\,;\\
\overline S^*_{N_n}&= \max\limits_{1\le i\le N_n}S_i^*\,;\enskip 
\underline S^*_{N_n}\hm=\min\limits_{1\le i\le N_n}S_i^*
\end{align*}
(для определенности считаем, что $S^*_0\hm=\overline S^*_0\hm=\underline
S^*_0\hm=0$). 

Пусть $\{d_n\}_{n\ge1}$~--- неограниченно возрастающая
последовательность положительных чисел.

\smallskip

\noindent
\textbf{Лемма~14}~\cite{Korolev1994}. \textit{Предположим, что с.в.\
$X_1,X_2,\ldots$ и~$N_1,N_2,\ldots$ удовлетворяют приведенным выше
условиям. В частности, пусть выполнено условие Линдеберга}~(\ref{e17-kgz}).
\textit{Более того, пусть $N_n\pto\infty$. Тогда распределение нормированных
экстремальных случайных сумм и~абсолютных величин случайных сумм
сходятся к~некоторым распределениям, т.\,е.\ существуют с.в.~$Y$,
$\overline Y$ и~$\underline Y$ такие, что}
$$
\fr{\overline S^*_{N_n}}{d_n}\Longrightarrow \overline Y\,;\enskip 
\fr{\underline S^*_{N_n}}{d_n}\Longrightarrow \underline Y\,; \enskip
\fr{\left\vert S^*_{N_n}\right\vert}{d_n}\Longrightarrow |Y|
$$
\textit{тогда и~только тогда, когда существует неотрицательная с.в.~$U$
такая, что $d_n^{-2}B^2_{N_n}\hm\Longrightarrow U$. При этом}

\noindent
\begin{gather*}
{\sf P}\big(\overline Y<x\big)={\sf P}\left(|Y|<x\right)= {\sf E}
\Psi\left(\fr{x}{\sqrt{U}}\right)\,;\\ 
{\sf P}\big(\underline
Y<x\big) =1-{\sf E}\Psi\left(-\fr{x}{\sqrt{U}}\right),\enskip
x\in\mathbb{R}\,.
\end{gather*}

%\smallskip

Из леммы~14 и~теоремы~7 вытекает следующее утверждение.

\smallskip

\noindent
\textbf{Теорема~9.} \textit{Пусть $\delta\hm\in(0,1]$, $\nu\hm\in(0,1)$.
Предположим, что с.в.\ $X_1,X_2,\ldots$ и~$N_1,N_2,\ldots$
удовлетворяют приведенным выше условиям. В~частности, пусть
выполнено условие Линдеберга}~(\ref{e17-kgz}). \textit{Более того, пусть
$N_n\hm\pto\infty$. Тогда следующие утверждения эквивалентны}:
\begin{gather*}
\fr{\overline S^*_{N_n}}{d_n}\Longrightarrow M_{\delta,\nu}; \ \ 
\fr{\underline S^*_{N_n}}{d_n}\Longrightarrow
-M_{\delta,\nu};\ \ 
 \fr{|S^*_{N_n}|}{d_n}\Longrightarrow
M_{\delta,\nu};\\ 
\fr{B^2_{N_n}}{d^2_n}\Longrightarrow
\fr{2W_1}{Z_{\nu,1}^{2/\delta}}\left(\fr{S_{\delta,1}}{S'_{\delta,1}}\right)^2\,.
\end{gather*}

\vspace*{-12pt}


{\small\frenchspacing
 {%\baselineskip=10.8pt
 \addcontentsline{toc}{section}{References}
 \begin{thebibliography}{99}
\bibitem{KorolevZeifmanKorchagin}
\Au{Королев~В.\,Ю., Зейфман~А.\,И., Корчагин~А.\,Ю.} Несимметричные
дву\-сто\-ронние распределения Миттаг-Леффлера как предельные законы
для случайных сумм независимых случайных величин с~конечными
дисперсиями~// Статистические методы оценивания и~пpовеpки гипотез.~---
Пермь: Пермский гос. ун-т, 2016. 
Т.~27. С.~69--89.
%Информатика и~ее применения, 2016. Т.~10. Вып.~4. C.~21--33.

\bibitem{KorolevZeifman2016} 
\Au{Korolev~V.\,Yu., Zeifman~A.\,I.} A~note on mixture
representations for the Linnik and Mittag-Leffler distributions and
their applications~// J.~Math. Sci., 2017. Vol.~218. No.\,3. P.~314--327.

\bibitem{KorolevZeifmanKMJ} 
\Au{Korolev V.\,Yu., Zeifman A.\,I.} Convergence of statistics
constructed from samples with random sizes to the Linnik and
Mittag-Leffler distributions and their generalizations~// J.~Korean 
Stat. Soc., 2017. Vol.~46. No.\,2. P.~161--181.

\bibitem{KorolevGorsheninZeifman2018} 
\Au{Korolev~V.\,Yu., Gorshenin~A.\,K., Zeifman~A.\,I.} 
On mixture representations for the generalized Linnik distribution and their
applications in limit theorems~// arXiv, 2018.

\bibitem{MittnikRachev1993} 
\Au{Mittnik~S., Rachev~S.\,T.} Modeling asset returns with
alternative stable distributions~// Economet. Rev.,~1993. Vol.~12. P.~261--330.

\bibitem{Kotz2001} 
\Au{Kotz~S., Kozubowski~T.\,J., Podgorski~K.} The Laplace
distribution and generalizations: A~revisit with applications to communications, 
economics, engineering, and finance.~--- Boston, MA, USA: Birkhauser, 2001. 349~p.

\bibitem{GorenfloMainardi2006} 
\Au{Gorenflo R., Mainardi~F.} Continuous time random walk, Mittag-Leffler 
waiting time and fractional diffusion: Mathematical aspects~// Anomalous transport: 
Foundations and applications~/ Eds. R.~Klages, G.~Radons, I.\,M.\,Sokolov.~--- 
Weinheim, Germany: Wiley-VCH, 2008. P.~93--127.

\bibitem{Kilbas2014}  %8
\Au{Gorenflo~R., Kilbas~A.\,A., Mainardi~F., Rogosin~S.\,V.}
Mittag-Leffler functions, related topics and applications.~--- 
Berlin/New York: Springer, 2014. 443~p.



\bibitem{Joseetal} %9
\Au{Jose~K.\,K., Uma~P., Lekshmi~V.\,S., Haubold~H.\,J.} Generalized
Mittag-Leffler distributions and processes for applications in astrophysics and time 
series modeling~// Astrophysics Space, 2010. Iss.~202559. 
P.~79--92.

\bibitem{MathaiHaubold2011}  %10
\Au{Mathai A.\,M., Haubold~H.\,J.} Matrix-variate statistical
distributions and fractional calculus~// Fract. Calc.  Appl. Anal., 2011. 
Vol.~14. No.\,1. P.~138--155.

\bibitem{Pillai1985}  %11
\Au{Pillai R.\,N.} Semi-$\alpha$-Laplace distributions~//
Commun.  Stat. Theory, 1985. Vol.~14. P.~991--1000.

\bibitem{Linnik1953} %12
\Au{Линник Ю.\,В.} Линейные формы и~статистические критерии.  I, II~//
Украинский математический~ж., 1953.
Т.~5. Вып.~2. С.~207--243; Вып.~3. С.~247--290.



\bibitem{Devroye1990} %13
\Au{Devroye~L.} A~note on Linnik's distribution~// Stat. Probabil. Lett., 
1990. Vol.~9. P.~305--306.



\bibitem{Anderson1992} %14
\Au{Anderson~D.\,N.} A multivariate Linnik distribution~//
Stat.  Probabil. Lett., 1992. Vol.~14. P.~333--336.

\bibitem{Lin1994}  %15
\Au{Lin~G.\,D.} Characterizations of the Laplace and related
distributions via geometric compound~// Sankhya Ser.~A, 
1994. Vol.~56. P.~1--9.

\bibitem{KotzOstrovskiiHayfavi1995a}  %16
\Au{Kotz~S., Ostrovskii~I.\,V., Hayfavi~A.} Analytic and asymptotic properties 
of Linnik's probability densities, I~// J.~Math. Anal. Appl., 1995. Vol.~193. 
P.~353--371.

\bibitem{KotzOstrovskiiHayfavi1995b}  %17
\Au{Kotz~S., Ostrovskii~I.\,V., Hayfavi~A.} 
Analytic and asymptotic properties of Linnik's probability densities, II~// 
J.~Math. Anal.  Appl., 1995. Vol.~193. P.~497--521.

\bibitem{Jacquesetal1999}  %18
\Au{Jacques~C., R$\acute{\mbox{e}}$millard~B., Theodorescu~R.} Estimation of
Linnik law parameters~// Statistics Risk Modeling, 1999. Vol. ~17. No.\,3. P.~213--236.



\bibitem{KotzOstrovskii1996} %19
\Au{Kotz~S., Ostrovskii~I.\,V.} A~mixture representation of the
Linnik distribution~// Stat. Probabil. Lett., 1996. Vol.~26. P.~61--64.

\bibitem{Pakes1998}  %20
\Au{Pakes~A.\,G.} Mixture representations for symmetric generalized
Linnik laws~// Stat. Probabil. Lett., 1998. Vol.~37. P.~213--221.

\bibitem{Anderson1993} %21
\Au{Anderson~D.\,N., Arnold~B.\,C.} Linnik distributions and
processes~// J.~Appl. Probab., 1993. Vol.~30. P.~330--340.

\bibitem{Jayakumar1995}  %22
\Au{Jayakumar~K., Kalyanaraman~K., Pillai~R.\,N.} $\alpha$-Laplace processes~// 
Math. Comput. Model., 1995. Vol.~22. P.~109--116.

\bibitem{BaringhausGrubel1997}  %23
\Au{Baringhaus~L., Grubel~R.} On 
a~class of characterization problems for random convex combinations~// 
Ann. I.~Stat. Math., 1997. Vol.~49. P.~555--567.



\bibitem{Kozubowski1998} %24
\Au{Kozubowski~T.\,J.} Mixture representation of Linnik distribution
revisited~// Stat. Probabil. Lett., 1998. Vol.~38. P.~157--160.

\bibitem{Lin1998} %25
\Au{Lin G.\,D.} A~note on the Linnik distributions~// J.~Math. Anal. Appl., 1998. Vol.~217. P.~701--706.

\bibitem{Zolotarev1983} %26
\Au{Золотарев~В.\,М.} Одномерные устойчивые распределения.~--- 
Теория вероятностей и~математическая статистика сер.~---
М.: Наука, 1983. 304~c. 
%(Translation of Mathematical Monographs. Vol.~65).

\bibitem{Schneider1986}  %27
\Au{Schneider~W.\,R.} Stable distributions: Fox function
representation and generalization~//  Stochastic processes in classical and quantum
systems~/ Eds. S.~Albeverio, G.~Casati, D.~Merlini.~--- 
Berlin: Springer, 1986. P.~497--511.

\bibitem{UchaikinZolotarev1999} %28
\Au{Uchaikin~V.\,V., Zolotarev~V.\,M.} Chance and stability.~--- 
Utrecht: VSP, 1999. 596~p.

\bibitem{KorolevWeibull2016} 
\Au{Korolev~V.\,Yu.} Product representations for random variables with the
 Weibull distributions and their applications~// J.~Math. Sci., 2016. Vol.~218. No.\,3. P.~298--313.

\bibitem{Stacy1962} %30
\Au{Stacy E.\,W.} A generalization of the gamma distribution~//
Ann. Math. Stat., 1962. Vol.~33. P.~1187--1192.

\bibitem{Gleser1989} 
\Au{Gleser~L.\,J.} The gamma distribution as a mixture of
exponential distributions~// Am. Stat., 1989. Vol.~43. P.~115--117.

\bibitem{Korolev2017} 
\Au{Королев~В.\,Ю.} Аналоги теоремы Глезера для отрицательных биномиальных 
и~обобщенных гам\-ма-рас\-пре\-де\-ле\-ний и~некоторые их приложения~// 
Информатика и~её применения, 2017. Т.~11. Вып.~3. C.~2--17.

\bibitem{LimTeo2009} {\it Lim~S.\,C., Teo~L.\,P.} Analytic and asymptotic properties of
multivariate generalized Linnik's probability densities~// J.~Fourier Anal.
Appl., 2010. Vol.~16. Iss.~5. P.~715--747.

\bibitem{Mathai2010} 
\Au{Mathai~A.\,M.} Some properties of Mittag-Leffler functions and
matrix-variate analogues: A~statistical perspective~// Fract. Calc. 
Appl. Anal., 2010. Vol.~13. No.\,2. P.~113--132.

\bibitem{Teicher1961} 
\Au{Teicher~H.} Identifiability of mixtures~// Ann. Math. Stat., 1961. 
Vol.~32. P.~244--248.

\bibitem{GnedenkoKorolev1996}
\Au{Gnedenko~B.\,V., Korolev~V.\,Yu.} Random summation: Limit theorems and applications.~--- 
Boca Raton, FL, USA: CRC Press, 1996. 288~p.

\bibitem{Grandell1976} 
\Au{Grandell~J.} Doubly stochastic Poisson processes.~--- 
Lecture notes in mathematics book ser.~--- 
Berlin\,--\,Heidelberg\,--\,New York: Springer, 1976. 
Vol.~529. 244~p. 

\bibitem{KorolevSokolov2008}
\Au{Королев~В.\,Ю., Соколов~И.\,А.} Математические модели неоднородных 
потоков экстремальных событий.~--- М.: ТОРУС ПРЕСС, 2008. 192~c.

\bibitem{Korolev1994} 
\Au{Королев~В.\,Ю.} Сходимость случайных последовательностей с~независимыми
случайными индексами. I~// Теория вероятностей и~ее 
применения, 1994. Т.~39. №\,2. С.~313--333.
 \end{thebibliography}

 }
 }

\end{multicols}

\vspace*{-3pt}

\hfill{\small\textit{Поступила в~редакцию 15.10.18}}

%\vspace*{8pt}

%\pagebreak

\newpage

\vspace*{-28pt}

%\hrule

%\vspace*{2pt}

%\hrule

%\vspace*{-2pt}

\def\tit{NEW MIXTURE REPRESENTATIONS OF~THE~GENERALIZED MITTAG-LEFFLER DISTRIBUTION 
AND~THEIR APPLICATIONS}

\def\titkol{New mixture representations of~the~generalized Mittag-Leffler distribution 
and~their applications}

\def\aut{V.\,Yu.~Korolev$^{1,2,3}$, A.\,K.~Gorshenin$^{1,2}$, 
and~A.\,I.~Zeifman$^{2,4,5}$}

\def\autkol{V.\,Yu.~Korolev, A.\,K.~Gorshenin, and~A.\,I.~Zeifman}

\titel{\tit}{\aut}{\autkol}{\titkol}

\vspace*{-11pt}


\noindent
$^1$Faculty of Computational Mathematics and Cybernetics, M.\,V.~Lomonosov Moscow
State University, GSP-1,\linebreak
$\hphantom{^1}$Leninskie Gory, Moscow 119991, Russian Federation

\noindent
$^2$Institute of Informatics Problems, Federal Research Center 
``Computer Science and Control'' of the Russian\linebreak
$\hphantom{^1}$Academy of Sciences, 44-2~Vavilov Str., 
Moscow 119333, Russian Federation

\noindent
$^3$Hangzhou Dianzi University, Xiasha Higher Education Zone, Hangzhou 310018, China

\noindent
$^4$Vologda State University, 15~Lenin Str., Vologda 160000, Russian Federation

\noindent
$^5$Vologda Research Center of the Russian Academy of Sciences, 56-A~Gorky Str.,
Vologda 160001, Russian\linebreak
$\hphantom{^1}$Federation



\def\leftfootline{\small{\textbf{\thepage}
\hfill INFORMATIKA I EE PRIMENENIYA~--- INFORMATICS AND
APPLICATIONS\ \ \ 2018\ \ \ volume~12\ \ \ issue\ 4}
}%
 \def\rightfootline{\small{INFORMATIKA I EE PRIMENENIYA~---
INFORMATICS AND APPLICATIONS\ \ \ 2018\ \ \ volume~12\ \ \ issue\ 4
\hfill \textbf{\thepage}}}

\vspace*{6pt}


\Abste{The article provides new mixture represenations for the generalized 
Mittag-Leffler distribution. In particular, it is shown that for values of the 
``generalizing'' parameter not exceeding one, the generalized Mittag-Leffler 
distribution is a~scale mixture of the half-normal distribution laws, classic 
Mittag-Leffler distributions, or generalized Mittag-Leffler distributions with 
the larger values of the characteristic index. The explicit expressions for mixing 
quantities are given for all cases. The obtained representations allow proposing new 
algorithms for modeling random variables with the generalized Mittag-Leffler 
distribution and formulating new limit theorems in which such distributions appear 
as the limit ones.}


\KWE{generalized Mittag-Leffler distribution; scale mixture; generalized gamma distribution; 
half-normal distribution; stable distribution}




\DOI{10.14357/19922264180411}

\vspace*{-20pt}

\Ack
\noindent
The research is supported by the Russian Foundation for Basic Research 
(project~17-07-00717).


%\vspace*{6pt}

  \begin{multicols}{2}

\renewcommand{\bibname}{\protect\rmfamily References}
%\renewcommand{\bibname}{\large\protect\rm References}

{\small\frenchspacing
 {%\baselineskip=10.8pt
 \addcontentsline{toc}{section}{References}
 \begin{thebibliography}{99}
\bibitem{1-kgz}
\Aue{Korolev,~V.\,Yu., A.\,I.~Zeifman, and A.\,Yu.~Korchagin.}
 2016. Nesimmetrichnye dvustoronnie raspredeleniya Mittag-Lefflera 
 kak predel'nye zakony dlya sluchaynykh summ nezavisimykh sluchaynykh 
 velichin s~konechnymi dispersiyami [Nonsymmetric two-sided Mittag-Leffler 
 distributions as limit laws for random sums of independent random variables 
 with finite variances]. \textit{Statisticheskie metody otsenivaniya i~provepki gipotez}
 [Statistical methods for evaluating and testing hypothesis].
 Perm: Perm State University. 
 27:69--89. %Информатика и~ее применения, 2016. Т.~10. Вып.~4. C.~21--33.

\bibitem{2-kgz}
\Aue{Korolev,~V.\,Yu., and A.\,I.~Zeifman.} 2017. A~note on mixture
representations for the Linnik and Mittag-Leffler distributions and 
their applications. \textit{J.~Math. Sci.} 218(3):314--327.

\bibitem{3-kgz}
\Aue{Korolev,~V.\,Yu., and A.\,I.~Zeifman.} 2017. Convergence of statistics 
constructed from samples with random sizes to the Linnik and Mittag-Leffler 
distributions and their generalizations. \textit{J.~Korean Stat. Soc.} 
46(2):161--181.

\bibitem{4-kgz}
\Aue{Korolev,~V.\,Yu., A.\,K.~Gorshenin, and A.\,I.~Zeifman.} 2018. 
On mixture representations for the generalized Linnik distribution and their
applications in limit theorems. \textit{arXiv}.

\bibitem{5-kgz}
\Aue{Mittnik,~S., and S.\,T.~Rachev.} 
1993. Modeling asset returns with alternative stable distributions. 
\textit{Economet. Rev.} 12:261--330.

\bibitem{6-kgz}
\Aue{Kotz,~S., T.\,J.~Kozubowski, and K.~Podgorski.} 2001.
\textit{The Laplace
distribution and generalizations: A~revisit with applications to communications, 
economics, engineering, and finance}. Boston, MA: Birkhauser. 349~p.

\bibitem{7-kgz}
\Aue{Gorenflo, R., and F.~Mainardi.} 2008. 
{Continuous time random walk, Mittag-Leffler waiting time and fractional 
diffusion: Mathematical aspects}. 
\textit{Anomalous transport: Foundations and applications}.
Eds.\ R.~Klages, G.~Radons, and I.\,M.~Sokolov. 
 Weinheim, Germany: Wiley-VCH. 93--127.

\bibitem{8-kgz}
\Aue{Gorenflo,~R., A.\,A.~Kilbas, F.~Mainardi, and S.\,V.~Rogosin.} 2014.
\textit{Mittag-Leffler functions, related topics and applications}. 
Berlin--New York: Springer.  443~p.



\bibitem{10-kgz} %9
\Aue{Jose,~K.\,K., P.~Uma, V.\,S.~Lekshmi, and H.\,J.~Haubold.}  
2010. Generalized Mittag-Leffler distributions and processes 
for applications in astrophysics and time series modeling. 
\textit{Astrophysics Space} 202559:79--92.

\bibitem{9-kgz} %10
\Aue{Mathai,~A.\,M., and H.\,J.~Haubold.} 2011. Matrix-variate statistical 
distributions and fractional calculus. \textit{Fract. Calc. Appl. Anal.} 
14(1):138--155.

\bibitem{11-kgz}
\Aue{Pillai,~R.\,N.} 1985. Semi-$\alpha$-Laplace distributions.
\textit{Commun. Stat. Theory} 14:991--1000.

\bibitem{12-kgz}
\Aue{Linnik,~Yu.\,V.} 
1953. Lineynye formy i~statisticheskie kriterii.~I, II 
[Linear forms and statistical criteria.~I. II]. 
\textit{Ukr. Math.~J.} 5(2):207--243; 5(3):247--290.


\bibitem{17-kgz} %13
\Aue{Devroye,~L.} 1990. A~note on Linnik's distribution. 
\textit{Stat. Probabil. Lett.} 9:305--306.

\bibitem{16-kgz} %14
\Aue{Anderson,~D.\,N.} 1992. A~multivariate Linnik distribution.
\textit{Stat. Probabil. Lett.} 14:333--336.

\bibitem{15-kgz} %15
\Aue{Lin,~G.\,D.} 1994. Characterizations of the Laplace and related
distributions via geometric compound. \textit{Sankhya Ser.~A}
 56:1--9.


\bibitem{13-kgz} %16
\Aue{Kotz,~S., I.\,V.~Ostrovskii, and A.~Hayfavi.} 1995. 
Analytic and asymptotic properties of Linnik's probability densities,~I. 
\textit{J.~Math. Anal. Appl.} 193:353--371.

\bibitem{14-kgz} %17
\Aue{Kotz,~S., I.\,V.~Ostrovskii, and A.~Hayfavi.}
 1995. Analytic and asymptotic properties of Linnik's probability densities,~II. 
 \textit{J.~Math. Anal. Appl.} 193:497--521.


\bibitem{18-kgz} %18
\Aue{Jacques,~C., B.~R$\acute{\mbox{e}}$millard, and R.~Theodorescu.}
1999. Estimation of Linnik law parameters. 
\textit{Statistics Risk Modeling} 17(3):213--236.

\bibitem{20-kgz} %19
\Aue{Kotz,~S., and I.\,V.~Ostrovskii.} 1996. 
A~mixture representation of the Linnik distribution. 
\textit{Stat. Probabil. Lett.} 26:61--64.

\bibitem{19-kgz} %20
\Aue{Pakes,~A.\,G.} 1998. Mixture representations for symmetric generalized
Linnik laws. \textit{Stat. Probabil. Lett.} 37:213--221.



\bibitem{21-kgz} %21
\Aue{Anderson,~D.\,N., and B.\,C.~Arnold.} 1993. Linnik distributions and
processes. \textit{J.~Appl. Probab.} 30:330--340.

\bibitem{23-kgz} %22
\Aue{Jayakumar,~K., K.~Kalyanaraman, and R.\,N.~Pillai.}
 1995. $\alpha$-Laplace processes. \textit{Math. 
 Comput. Model.} 22:109--116.

\bibitem{22-kgz} %23
\Aue{Baringhaus,~L., and R.~Grubel.} 1997. 
On a~class of characterization problems for random convex combinations. 
\textit{Ann. I.~Stat. Math.} 49:555--567.



\bibitem{24-kgz}
\Aue{Kozubowski,~T.\,J.} 1998. Mixture representation of Linnik distribution. 
\textit{Stat. Probabil. Lett.} 38:157--160.

\bibitem{25-kgz}
\Aue{Lin, G.\,D.} 1998. A~note on the Linnik distributions. 
\textit{J.~Math. Anal. Appl.} 217:701--706.

\bibitem{26-kgz}
\Aue{Zolotarev,~V.\,M.} 1986. \textit{One-dimensional stable distributions}.
 Translation of mathematical monographs ser.
 Providence, RI:
 American Mathematical Society. Vol.~65. 284~p. 


\bibitem{27-kgz}
\Aue{Schneider,~W.\,R.} 1986. Stable distributions: Fox function
representation and generalization. 
\textit{Stochastic processes in classical and quantum
systems}. Eds. S.~Albeverio, G.~Casati, and D.~Merlini. 
Berlin: Springer. 497--511.

\bibitem{28-kgz}
\Aue{Uchaikin,~V.\,V., and V.\,M.~Zolotarev.} 1999. 
\textit{Chance and stability}. Utrecht: VSP. 596~p.

\bibitem{29-kgz}
\Aue{Korolev,~V.\,Yu.} 2016. Product representations for random variables with 
the Weibull distributions and their applications. 
\textit{J.~Math. Sci.} 218(3):298--313.

\bibitem{30-kgz}
\Aue{Stacy,~E.\,W.} 1962. A~generalization of the gamma distribution.
\textit{Ann. Math. Stat.} 33:1187--1192.

\bibitem{31-kgz}
\Aue{Gleser,~L.\,J.} 1989. The gamma distribution as a~mixture of
exponential distributions. \textit{Am. Stat.} 43:115--117.

\bibitem{32-kgz}
\Aue{Korolev,~V.\,Yu.} 2017. Analogi teoremy Glezera dlya ot\-ri\-tsa\-tel'\-nykh 
binomial'nykh i~obobshchennykh gamma-raspredeleniy i~nekotorye ikh prilozheniya 
[Analogs of Gleser's theorem for negative binomial and generalized
 gamma distributions and some their applications]. 
 \textit{Informatika i~ee Primeneniya~--- Inform. Appl.} 11(3):2--17.

\bibitem{33-kgz}
\Aue{Lim,~S.\,C., and L.\,P.~Teo.} 2010. Analytic and asymptotic properties 
of multivariate generalized Linnik's probability densities. 
\textit{J.~Fourier Anal. Appl.} 16(5):715--747.

\bibitem{34-kgz}
\Aue{Mathai,~A.\,M.} 2010. Some properties of Mittag-Leffler functions and
matrix-variate analogues: A~statistical perspective. 
\textit{Fract. Calc. Appl. Anal.} 13(2):113--132.

\bibitem{35-kgz}
\Aue{Teicher,~H.} 1961. Identifiability of mixtures. 
\textit{Ann. Math. Stat.} 32:244--248.

\bibitem{36-kgz}
\Aue{Gnedenko,~B.\,V., and V.\,Yu.~Korolev.} 1996. 
\textit{Random summation: Limit theorems and applications}. 
Boca Raton, FL: CRC Press. 288~p.

\bibitem{37-kgz}
\Aue{Grandell,~J.} 1976. \textit{Doubly stochastic poisson processes}. 
Lecture notes in mathematics book ser. Berlin\,--\,Heidelberg\,--\,New York: 
Springer.  Vol.~529. 244~p.

\bibitem{38-kgz}
\Aue{Korolev,~V.\,Yu., and I.\,A.~Sokolov.} 2008. \textit{Ma\-te\-ma\-ti\-che\-skie
modeli neodnorodnykh potokov ekstremal'nykh sobytiy}
[Mathematical 
models of nonhomogeneous flows of extremal events]. Moscow: TORUS PRESS. 192~p.  
%(in Russian)

\bibitem{39-kgz}
\Aue{Korolev,~V.\,Yu.} 1994. Convergence of random sequences with the
independent random indexes.~I. \textit{Theor. Probab. Appl.} 39(2):282--297.
\end{thebibliography}

 }
 }

\end{multicols}

\vspace*{-7pt}

\hfill{\small\textit{Received October 15, 2018}}

\vspace*{-16pt}

\Contr

\vspace*{-4pt}

\noindent
\textbf{Korolev Victor Yu.} (b.\ 1954)~--- 
Doctor of Science (PhD) in physics and
mathematics, professor, Head of Department, Faculty of Computational Mathematics 
and Cybernetics, M.\,V.~Lomonosov Moscow State University, GSP-1, Leninskie Gory, 
Moscow 119991, Russian Federation; leading scientist, 
Institute of Informatics Problems, Federal Research Center 
``Computer Science and Control'' of the Russian Academy of Sciences, 
44-2~Vavilov Str., Moscow 119333, Russian Federation; 
professor, Hangzhou Dianzi University, Xiasha Higher Education Zone, 
Hangzhou 310018, China; \mbox{vkorolev@cs.msu.ru}

\vspace*{1pt}

\noindent
\textbf{Gorshenin Andrey K.} (b.\ 1986)~--- Candidate of Science (PhD) in physics and
mathematics, associate professor, leading scientist, Institute of Informatics Problems,
Federal Research Center ``Computer Science and Control'' of the Russian Academy of
Sciences, 44-2~Vavilov Str., Moscow 119333, Russian Federation;  
leading scientist, Faculty of Computational Mathematics and Cybernetics, 
M.\,V.~Lomonosov Moscow State University, GSP-1, Leninskie Gory, Moscow 119991, 
Russian Federation; \mbox{agorshenin@frccsc.ru}

\vspace*{1pt}

\noindent
\textbf{Zeifman Alexander I.} (b.\ 1954)~--- 
Doctor of Science in physics and mathematics, professor, Head of Department, 
Vologda State University, 15~Lenin Str., Vologda 160000, Russian Federation; 
senior scientist, Institute of Informatics Problems, Federal Research Center 
``Computer Science and Control'' of the Russian Academy of Sciences, 
44-2~Vavilov Str.,Moscow 119333, Russian Federation; 
principal scientist, Vologda Research Center of the Russian Academy of Sciences, 
56-A~Gorky Str., Vologda 160001, Russian Federation; \mbox{a\_zeifman@mail.ru}
\label{end\stat}

\renewcommand{\bibname}{\protect\rm Литература}       