\def\stat{logachev}

\def\tit{ТЕОРЕТИКО-ИНФОРМАЦИОННАЯ ХАРАКТЕРИЗАЦИЯ СОВЕРШЕННО УРАВНОВЕШЕННЫХ 
ФУНКЦИЙ$^*$}

\def\titkol{Теоретико-информационная характеризация совершенно уравновешенных 
функций}

\def\aut{О.\,А.~Логачев$^1$}

\def\autkol{О.\,А.~Логачев}

\titel{\tit}{\aut}{\autkol}{\titkol}

\index{Логачев О.\,А.}
\index{Logachev O.\,A.}




{\renewcommand{\thefootnote}{\fnsymbol{footnote}} \footnotetext[1]
{Работа выполнена при поддержке РФФИ (проект 16-01-00470-А).}}


\renewcommand{\thefootnote}{\arabic{footnote}}
\footnotetext[1]{Московский государственный университет им.\ М.\,В.~Ломоносова, \mbox{logol@iisi.msu.ru}}

\vspace*{-6pt}


\Abst{Совершенно уравновешенные дискретные функции являются 
объектом исследований для ряда математических дисциплин, близких к~информатике, 
таких как комбинаторика, теория кодирования, криптография, символическая 
динамика, теория автоматов и~др. Данный класс дискретных функций оказался 
удобным математическим инструментом для синтеза и~описания сверточных кодов, 
некоторых криптографических примитивов, сюръективных эндоморфизмов дискретных 
динамических систем, а~также для конечных автоматов без потери информации. Ранее 
Хедлундом и~Сумароковым были доказаны критерии, связывающие свойство совершенной 
уравновешенности со свойствами функции быть дефекта нуль и~без потери 
информации. В~данной статье доказывается новый критерий совершенной 
уравновешенности функции, носящий тео\-ре\-ти\-ко-ин\-фор\-ма\-ци\-он\-ный характер, а~также 
рассмотрены некоторые алгоритмические свойства совершенно уравновешенных функций 
как преобразователей информации.}


\KW{конечный алфавит; дискретная функция; случайная 
величина; закон распределения; взаимная энтропия; совершенная уравновешенность}

\DOI{10.14357/19922264180410}
  
%\vspace*{-4pt}


\vskip 10pt plus 9pt minus 6pt

\thispagestyle{headings}

\begin{multicols}{2}

\label{st\stat}

\section{Введение}

В работах~\cite{Sumarokov,Hedlund} показана эквивалентность сле\-ду\-ющих свойств 
дискретных функций над конечным алфавитом:
\begin{itemize}
    \item совершенная уравновешенность функции (сильная равновероятность);
    \item свойство функции быть дефекта нуль (без запрета);
    \item свойство функции быть без потери информации.
\end{itemize}

В данной работе рассмотрен и~доказан критерий совершенной уравновешенности 
дискретных функций над конечным алфавитом, ис\-поль\-зу\-ющий понятия и~терминологию 
теории информации.\linebreak Получен\-ный результат характеризует процессы обработки 
информации, осуществляемые неавтономными регистрами сдвига с~фильтрующими 
функциями.

\vspace*{-4pt}


\section{Дискретные функции} % Раздел

\vspace*{-2pt}

Пусть $A=\{a_1,\ldots,a_p\}$~--- конечное множество символов (алфавит) мощности~$p$, 
$p \hm\geq 2$, и~$\mathbb{N}\hm=\{1,2,\ldots\}$~--- множество натуральных чисел. 
Для произвольного натурального числа~$n$ через $A^n \hm= \underbrace{A \times \dots 
\times A}_n$ будем обозначать множество слов длины~$n$ ($n$-на\-бо\-ров) в~алфавите~$A$. 
Для произвольных натуральных чисел~$n$ и~$m$ через $\mathcal{F}_{n, m, p}$ 
будем обозначать множество всех функций вида $f:~A^n \hm\rightarrow A^m$.  
Переменные~$x_1$ и~$x_n$ для функции $f(x_1, \dots, x_n)$ будем называть 
крайними. Функция $f(x_1,\dots,x_n)$ из  $\mathcal{F}_{n, 1, p}$ называется 
перестановочной по переменной $x_i$, $1 \hm\leq i \hm\leq n$, если для любого набора $c 
\hm= (c_1, \dots, c_{i-1}, c_{i+1}, \dots, c_n) \in A^{n-1}$ функция

\noindent
$$
\varphi_{c}(x_i)  = f(c_1, \dots, c_{i-1}, x_i, c_{i+1}, \dots, c_n)
$$
является перестановкой элементов алфавита~$A$.

Для произвольной функции $f \hm\in \mathcal{F}_{n, 1, p}$ и~произвольного 
натурального числа~$m$ через~$f_m$ обозначим функцию из  $\mathcal{F}_{n, m, p}$ 
вида:

\noindent
\begin{multline*}
f_m\left(x_1, \dots, x_{m+n-1}\right) ={}\\ 
{}=\left(f\left(x_1, \dots, x_n\right), f\left(x_2, \dots, x_{n+1}\right), 
\dots\right.\\
\left.\dots, f\left(x_m, \dots, x_{m+n-1}\right)\right).
\end{multline*}
Для конечного множества~$S$ через~$\#S$ будем обозначать его мощность.

\vspace*{-4pt}

\section{Взаимная энтропия} % Раздел

\vspace*{-2pt}

Для случайной величины~$\xi$, принимающей значения из~$A$ в~соответствии 
с~законом распределения~$\mathcal{P}$, будем пользоваться обозначением $\xi 
\hm\in_{\mathcal{P}} A$. В~частности, для равномерного распределения далее 
используется символ~$\mathcal{U}$.

Энтропия Шеннона (далее~--- энтропия, см.~\cite{Finstein}) случайной величины~$\xi$ 
(или распределения $\mathcal{P} \hm= \{\mathrm{Pr}\, [ \xi \hm= x], x \hm\in S\}$) 
определяется соотношением:
$$
H(\xi) = - \sum\limits_{x \in S} \mathrm{Pr}\, [\xi = x] \,\log_2 
\mathrm{Pr}\,[\xi = x]\,,
$$
где всегда полагают $0 \cdot \log_2 0 \hm= 0$. Для пары случайных величин 
$\xi$ и~$\eta$, принимающих значения в~множествах~$S$ и~$T$ соответственно, 
известны~\cite{Finstein} следующие виды энтропии:
\begin{itemize}
    \item совместная энтропия:
    
    \noindent
  \begin{multline*}
    H(\xi, \eta) = {}\\
    \hspace*{-16pt}{}=- \sum\limits_{x\in S} \sum\limits_{y\in T} 
    \mathrm{Pr}\,[\xi = x,\ \eta = y]\, 
\log_2 \mathrm{Pr}\,[\xi = x,\ \eta = y]\,;\hspace*{-8pt}
   \end{multline*}
    \item условная энтропия случайной величины~$\eta$ относительно 
случайной величины~$\xi$:

\noindent
 \begin{multline*}
    H(\eta|\xi) ={}\\
    \hspace*{-16pt}{}= - \sum\limits_{x\in S} \sum\limits_{y\in T} 
    \mathrm{Pr}\,[\xi = x,\ \eta = y]\, \log_2 
\mathrm{Pr}\,[\eta = y|\xi = x]\,.\hspace*{-5pt}
\end{multline*}
\end{itemize}

Далее понадобится величина, называемая средней взаимной информацией случайных 
величин~$\xi$ и~$\eta$:

\noindent
\begin{multline*}
I(\eta|\xi) = I(\xi|\eta) = H(\xi) + H(\eta) - H(\xi,~\eta) ={}\\
{}= H(\xi) - 
H(\xi|\eta) = H(\eta) - H(\eta|\xi)\,.
\end{multline*}

Средняя взаимная информация характеризует среднее значение информации 
о~случайной величине~$\eta$, содержащейся в~случайной величине~$\xi$ (и~на\-обо\-рот).

\vspace*{-4pt}

\section{Совершенно уравновешенные функции}

\vspace*{-2pt}

Свойство совершенной уравновешенности первоначально исследовалось для булева 
случая ($p \hm= 2$). Был установлен ряд свойств совершенно уравновешенных функций. 
Подробнее с~результатами и~библиографией в~данной области можно ознакомиться 
в~\cite{Logachev_Salnikov}.

\smallskip

\noindent
\textbf{Определение~1.}\
    Функция $f\hm\in \mathcal{F}_{n,1,p}$ называется совершенно уравновешенной, 
если для любого натурального~$m$ и~для любого набора $y \hm\in A^m$ выполнено 
равенство:
    $$
    \# f^{-1}_m(y) = p^{n-1}\,.
$$

\noindent
\textbf{Пример~1.}\
%        \label{expl_sov_ur}
        Функция  $f\hm\in \mathcal{F}_{n,1,p}$, перестановочная по переменной~$x_1$ 
        (либо перестановочная по переменной~$x_n$), является совершенно 
уравновешенной функцией.

\smallskip

Необходимо отметить, что свойство перестановочности функции по некоторой 
переменной\linebreak\vspace*{-12pt}

\columnbreak

\noindent
 в~булевом случае соответствует линейности булевой функции по данной 
переменной.

\vspace*{-9pt}

\section{Средняя на~символ взаимная~информация}

\vspace*{-2pt}

Пусть $f \hm\in \mathcal{F}_{n,1,p}$ и~последовательность случайных 
величин~$\{\xi_m\}_{m=1}^{\infty},$ $\xi_m\hm=(\xi_{m,1}, \dots, \xi_{m, m+n-1})$, 
удовлетворяет условию:

\noindent
$$
\xi_m \in_{\mathcal{U}} A^{m+n-1},\enskip m = 1,2,\dots
$$

Для каждой~$\xi_m$ определим случайную величину:

\noindent
\begin{equation}
\label{eta_m^f}
\eta _m^f = f_m\left(\xi_m\right)\,.
\end{equation}
Справедливо следующее утверждение.

\smallskip

\noindent
\textbf{Предложение~1.}\
   \textit{Для любого натурального числа~$m$ случайная величина~$\eta_m^f$ 
удовлетворяет условию}:

\noindent
$$
    \eta_m^f \in_{\mathcal{U}} A^m
  $$
    \textit{тогда и~только тогда, когда $f \hm\in \mathcal{F}_{n,1,p}$~--- совершенно 
уравновешенная функция}.

\smallskip

\noindent
Д\,о\,к\,а\,з\,а\,т\,е\,л\,ь\,с\,т\,в\,о\,.\ \
    Непосредственно вытекает из определения совершенно уравновешенной функции 
и~соотношения~(\ref{eta_m^f}).~\hfill$\square$

\smallskip

Определим среднюю на символ взаимную информацию для случайных величин~$\xi_m$ 
и~$\eta_m^f$ как
\begin{equation*}
i_m(f) = m^{-1} I\left(\eta_m^f|\xi_m\right)\,.
\end{equation*}

\vspace*{-2pt}

\noindent
\textbf{Теорема~1.}\
%    \label{thm_i_m(f)}
 \textit{Пусть $f \hm\in \mathcal{F}_{n,1,p}$, $\xi_m \hm\in_{\!\mathcal{U}} A^{m+n-1}$ 
 и~$\eta_m^f \hm= f_m(\xi_m)$, $m \hm= 1,2,\dots$
    Функция~$f$~--- совершенно уравновешенная тогда и~только тогда, когда для 
любого натурального~$m$ выполнено $i_m(f)\hm=\log_2 p$}.


\smallskip

\noindent
Д\,о\,к\,а\,з\,а\,т\,е\,л\,ь\,с\,т\,в\,о\,.\ \
    Заметим, что для любых $u \hm\in A^{m+n-1}$ и~$v \hm\in A^m$ справедливы 
следующие соотношения:

\noindent
    \begin{align*}
    \pi_u &= \mathrm{Pr}\, [\xi_m = u] = p^{-(m+n-1)}\,; \\
    \pi_{v/u} &= \mathrm{Pr}\left[\eta_m^f = v | \xi_m = u\right] ={}\\
    &\hspace*{28mm}{}=
        \begin{cases}
        1\,, &\mbox{ если } u \in f_m^{-1}(v)\,;\\
        0\,, &\mbox{ если } u \notin f_m^{-1}(v)\,;
        \end{cases}  \\
\pi_{v,u} &= \mathrm{Pr}\left[\eta_m^f = v,\ \xi_m = u\right] = \pi_u  \pi_{v/u} ={}\\
&\hspace*{14mm}{}=
    \begin{cases}
    p^{-(m+n-1)}\,, &\mbox{ если } u \in f_m^{-1}(v)\,;\\
    0\,, &\mbox{ если } u \notin f_m^{-1}(v)\,;
    \end{cases}   \\
    \widetilde{\pi}_v &= \sum\limits_{u \in A^{m+n-1}}\pi_{v,u} = p^{-(m+n-1)} 
 \#f^{-1}_m(v)\,. 
    \end{align*}
    
    \vspace*{-2pt}
    
    \noindent
    Воспользовавшись этими соотношениями, легко показать справедливость 
следующей цепочки равенств:

\noindent
    \begin{multline}
    \label{sums}
 i_m(f) = m^{-1}I\left(\eta_m^f|\xi_m\right) ={}\\
 {}= m^{-1}\left[H\left(\eta_m^f\right) - 
H\left(\eta_m^f|\xi_m\right)\right] ={} \\
    {}= m^{-1} \sum\limits_{v \in A^m} \sum\limits_{u \in f^{-1}_m(v)} 
\pi_{v,u} \left[\log_2 \pi_{v/u} - \log_2 \widetilde{\pi}_v\right] ={} \\
{}= m^{-1} \sum\limits_{v \in A^m} \sum\limits_{u \in f^{-1}_m(v)} 
p^{-(m+n-1)}\times{}\\
{}\times\left[(m+n-1)\log_2 p - \log_2\left(\#f^{-1}_m(v)\right)\right] = {}\\
{}= m^{-1} \sum\limits_{v \in A^m} \left(\#f^{-1}_m (v)\right) p^{-(m+n-1)} \times{}\\
{}\times\left[(m+n-
1)\log_2 p - \log_2 \left(\#f^{-1}_m(v)\right)\right]  ={} \\
   {}= m^{-1}\left[
   \vphantom{\sum\limits_{v \in A^m}}
(m+n-1)\log_2 p -{}\right.\\
\!\!\left.   {}- p^{-(m+n-1)} \sum\limits_{v \in 
A^m}\left(\#f^{-1}_m (v)\right) \log_2\left(\#f^{-1}_m (v)\right) \right]\!.
    \end{multline}
    
    Пусть $f$~--- совершенно уравновешенная функция из~$\mathcal{F}_{n,1,p}$. 
Тогда для любого натурального числа~$m$ и~любого набора $v \hm\in A^m$ справедливо 
равенство
    \begin{equation}
    \label{f_m}
    \#f^{-1}_m (v) = p^{n-1}\,.
    \end{equation}
    Воспользовавшись соотношениями~(\ref{sums}) и~(\ref{f_m}), для 
произвольного~$m$ получаем $i_m(f) \hm= \log_2 p$.
    
    Докажем теперь обратное утверждение. Пусть $f \hm\in \mathcal{F}_{n,1,p}$ 
    и~для любого натурального числа~$m$ выполняется равенство $i_m(f) \hm= \log_2 p$. 
Тогда из соотношения~(\ref{sums}) следует, что
    \begin{equation*}
    - \sum\limits_{v \in A^m} \fr{\#f_m^{-1} (v)}{p^{m+n-1}}\, \log_2 
\left(\fr{\#f_m^{-1} (v)}{p^{m+n-1}} \right) = m\,\log_2 p\,.
    \end{equation*}
    Кроме того,
   $$
    \sum\limits_{v \in A^m} \fr{\#f_m^{-1} (v)}{p^{m+n-1}} = 1\,.
$$
    Следовательно, энтропия распределения $\{\#f_m^{-1} 
(v)/p^{m+n-1} \}_{v \in A^m}$ имеет максимальное значение. Это возможно 
(см.~\cite{Finstein}) лишь в~случае, когда
    \begin{equation*}
    \fr{\# f^{-1}_m (v)  }{p^{m+n-1} }= p^{-m}
    \end{equation*}
    для любого $v \in A^m$, т.\,е.\ $\#f_m^{-1} (v) \hm= p^{n-1}$. Поскольку 
    в~приведенных рассуждениях~$m$~--- произвольное, то функция~$f$~--- совершенно 
уравновешенная.~\hfill$\square$

\smallskip

\noindent
\textbf{Следствие~1.}\
    Если функция $f \hm\in \mathcal{F}_{n,1,p}$ не является совершенно 
уравновешенной, то существует такое натуральное число~$m_0$, что $i_{m_0} (f)\hm < 
\log_2 p$.

\smallskip

\noindent
Д\,о\,к\,а\,з\,а\,т\,е\,л\,ь\,с\,т\,в\,о\,.\ \
Непосредственно вытекает из определения совершенно уравновешенной функции и~утверждения теоремы~1.~\hfill$\square$

\smallskip

Характеризации классов дискретных функций
 (т.\,е.\ необходимые и~достаточные 
условия при\-над\-леж\-ности функции данному классу) имеют\linebreak раз\-но\-об\-раз\-ную 
математическую природу. В~соответствующих формулировках используются 
алгебраические, комбинаторные, метрические, спект\-раль\-ные и~другие параметры этих 
функций.\linebreak Наличие нескольких характеризаций для конкретного класса дискретных 
функций позволяет рассматривать их с~различных позиций, а также оценивать их 
особенности с~точки зрения возможных приложений.


\section{Полиномиальный алгоритм обращения совершенно уравновешенных функций}

Пусть $f$~--- произвольная функция из $\mathcal{F}_{n,1,p}$ 
и~$F(f)\hm=\{f_m\in\mathcal{F}_{n,m,p}$, $m\hm\in\mathbb{N}\}$~--- счетное семейство 
функций, порожденное~$f$.
Задачу обращения (см.~\cite{Goldreich}) функций из семейства~$F(f)$ можно 
сформулировать следующим образом: для произвольного натурального~$m$ по 
известному набору $y\hm=(y_1,y_2,\ldots, y_m)$ из $f_m(A^{m+n-1})$ найти по крайней 
мере один набор $x\hm=(x_1,x_2,\ldots,x_{m+n-1})$ из~$f_m^{-1}(y)$.
Рассмотрим простейший алгоритм решения этой задачи.

\paragraph*{Алгоритм $\mathfrak{A}_f$ обращения функций семейства $F(f)$.}
Пусть $m\hm\in\mathbb{N}$, $y\hm=(y_1,y_2,\ldots,y_m)\in f_m(A^{m+n-1})$.
Строим последовательно~$m$~наборов из~$p$~непересекающихся подмножеств~$A^n$:
$$
\begin{array}{l@{\ }l@{\ }l@{\ }l}
  (S_1(1), &S_1(2), &\cdots, &S_1(p)),\\
  (S_2(1), &S_2(2), &\cdots, &S_2(p)),\\
   \quad\vdots  &\quad \vdots  &\ddots\,  &\quad\vdots \\
  (S_m(1), &S_m(2), &\cdots, &S_m(p));
\end{array}$$
если $i=1$, то
\begin{gather*}
\hspace*{-67mm}S_1(1)={}\\
\hspace*{-2.94907pt}{}=\{(u_1,\ldots,u_{n-1},a_1)\in A^n\colon \!f(u_1,\ldots,u_{n-
1},a_1)\!=\!y_1\};\\
\vdots\\
\hspace*{-67mm}S_1(p)={}\\
\hspace*{-2.94907pt}{}=\{(u_1,\ldots,u_{n-1},a_p)\in A^n\colon\! f(u_1,\ldots,u_{n-
1},a_p)\!=\!y_1\};\\
S_1=\bigcup\limits_{j=1}^p S_1(j)\neq\varnothing;
\end{gather*}
если набор подмножеств $(S_{i-1}(1),\ldots,S_{i-1}(p))$, $1\hm< i\hm\leq m$, построен, 
то

\noindent
\begin{gather*}
\hspace*{-30mm}S_i(1)=\{(v_2,\ldots,v_n,a_1)\colon\exists v={}\\
{}=(v_1,v_2,\ldots,v_n)\in S_{i-1}\  
\bigl(f(v_2,\ldots,v_n,a_1)=y_i\bigr)\};\\
\vdots\\
\hspace*{-30mm}S_i(p)=\{(v_2,\ldots,v_n,a_p)\colon\exists v={}\\
{}=(v_1,v_2,\ldots,v_n)\in S_{i-1}\  
\bigl(f(v_2,\ldots,v_n,a_p)=y_i\bigr)\};\\
S_i=\bigcup\limits_{j=1}^p S_i(j)\neq\varnothing,\quad i=2,3,\ldots,m;
\end{gather*}
вычисление набора $x\hm\in f_m^{-1}(y)$:
для~$y_m$, поскольку $S_m\hm\neq\varnothing$, существует
$$
\left(x_m,x_{m+1},\ldots,x_{m+n-1}\right)\in S_m\,,
$$
такой что $f(x_m,x_{m+1},\ldots,x_{m+n-1})\hm=y_m$;
для~$y_{m-1}$ множество $S_{m-1}\hm\neq\varnothing$ и~существует
$$
\left(x_{m-1},x_m,\ldots,x_{m+n-2}\right)\in S_{m-1},
$$
такой что $f(x_{m-1},x_m,\ldots,x_{m+n-2})\hm=y_{m-1}$
и т.\,д.\ для $i\hm=m-2,m-3,\ldots,1$;
полученные в~результате~$m$~наборов
\begin{gather*}
\left(x_m,x_{m+1},\ldots,x_{m+n-1}\right),\\
\left(x_{m-1},x_m,\ldots,x_{m+n-2}\right),\\
\vdots\\
\left(x_1,x_2,\ldots,x_n\right)
\end{gather*}
определяют набор $x\hm=(x_1,x_2,\ldots,x_{m+n-1})\hm\in A^{m+n-1}$, такой что 
$f_m(x)\hm=y$.

\smallskip

Остановимся подробнее на некоторых па\-ра\-мет\-рах описанного выше алгоритма.
Основными операциями, используемыми в~ходе реализации алгоритма~$\mathfrak{A}_f$, 
являются следующие:
\begin{enumerate}[(1)]
 \item запись в~память (считывание из памяти) $n$-на\-бо\-ров в~алфавите~$A$;
 \item переход от набора $(v_1,v_2,\ldots,v_n)$ к~набору 
$(v_2,\ldots,v_n,a)$, $v_1,v_2,\ldots,v_n$, $a\hm\in A$;
 \item проверка выполнения соотношения $f(v_2,\ldots,v_n,a)\hm=b$, 
$v_2,\ldots,v_n, a,b\hm\in A$.
\end{enumerate}
Будем называть их единичными операциями (ед.~оп.). Используемую память будем 
представлять в~виде массивов ячеек памяти, необходимых для хранения $n$-на\-бо\-ров 
в~алфавите~$A$, т.\,е.\ способных хранить $n\log_2 p$~бит информации.

Обозначим через $T_f\colon \mathbb{N}\hm\to\mathbb{N}$ и~$M_f\colon 
\mathbb{N}\hm\to\mathbb{N}$ трудоемкость алгоритма~$\mathfrak{A}_f$ и~память, 
необходимую для его реализации, соответственно.

Из описания алгоритма~$\mathfrak{A}_f$ легко видеть, что
\begin{align}
 T_f&\leq C_1(n,p) m\ [\mbox{ед.~оп.}];\label{algtime}\\
 M_f&\leq C_2(n,p) m\ [\mbox{бит}].\notag
\end{align}
Очевидно, что алгоритм~$\mathfrak{A}_f$ является детер\-ми\-ни\-рованным 
полиномиальным (по~$m$) и~улучшить асимптотическую оценку~\eqref{algtime} не 
представляется возможным.
Действительно, в~ходе реализации любого детерминированного алгоритма обращения 
функций из~$F(f)$ необходимо будет по крайней мере просмотреть набор 
$y\hm=(y_1,y_2,\ldots,y_m)$. Нетрудно заметить, что $C_1(n,p)\hm\leq 5p^n+1$ 
и~$C_2(n,p)\hm\leq p^n n \log_2 p$.

В случае если функция~$f$ является совершенно\linebreak уравновешенной, из утверждения 
теоремы~1 следует, что набор $y\hm=f_m(x)$ содержит в~среднем\linebreak 
максимально возможное количество информации о~наборе~$x$.
Естественно ожидать, что для совершенно уравновешенных функций величины 
$C_1(n,p)$ и~$C_2(n,p)$ могут быть значительно меньшими. Например, 
воспользовавшись эквивалентностью свойства функций быть без потери информации 
(см.~\cite{Logachev_Salnikov}) свойству совершенной уравновешенности, несложно 
получить оценку $C_1(n,p)\hm\leq 5p^{n-1}+1$, $C_2(n,p)\hm\leq p^{n-1} 
n \log_2 p$.

\smallskip

\noindent
\textbf{Замечание~1.}\
 Описание алгоритма~$\mathfrak{A}_f$ легко может быть перенесено на 
 тео\-ре\-ти\-ко-ав\-то\-мат\-ную модель с~соответствующей заменой единичных операций на операции, 
соответствующие функциям переходов и~выходов конечного автомата. Следовательно, 
для любой ограниченно-детерминированной функции над конечным алфавитом 
(см.~\cite{KAP}) существует полиномиальный алгоритм обращения.


\vspace*{-9pt}

{\small\frenchspacing
 {%\baselineskip=10.8pt
 \addcontentsline{toc}{section}{References}
 \begin{thebibliography}{9}



\bibitem{Hedlund}
\Au{Hedlund G.\,A.}
Endomorphisms and automorphisms of the shift dynamical system~//
Math. Syst. Theory,
1969. No.\,3. P.~320--375.

\bibitem{Sumarokov}
\Au{Сумароков С.\,Н.}
Запреты двоичных функций и~обратимость для одного класса кодирующих устройств~//
Обозрение прикладной и~промышленной математики,
1994. №\,1. С.~33--55.


\bibitem{Finstein}
\Au{Файнстейн А.}
Основы теории информации~/ Пер. с~англ.~--- М.: ИЛ, 1960. 140~с. 
(\Au{Feinstein~A.} Foundations of information theory.~--- New York, NY, USA: 
Mcgraw-Hill, 1958. 137~p.)

\bibitem{Logachev_Salnikov}
\Au{Логачев О.\,А., Сальников~А.\,А., Смышляев~С.\,В., Ященко~В.\,В.}
Булевы функции в~теории кодирования и~крип\-то\-ло\-гии.~--- М.: Ленанд, 2015. 576~с.


\bibitem{Goldreich}
\Au{Goldreich O.}
Foundations of cryptography.
Vol.~I: Basic tools.~--- Cambridge: Cambridge University Press, 2003. 372~p.

\bibitem{KAP}
\Au{Кудрявцев В.\,Б., Алешин~С.\,В., Подколзин~А.\,С.}
Введение в~теорию автоматов.~--- М.: Наука, 1985. 320~с.
 \end{thebibliography}

 }
 }

\end{multicols}

\vspace*{-7pt}

\hfill{\small\textit{Поступила в~редакцию 03.09.18}}

%\vspace*{8pt}

%\pagebreak

\newpage

\vspace*{-28pt}

%\hrule

%\vspace*{2pt}

%\hrule

%\vspace*{-2pt}

\def\tit{AN INFORMATION BASED CRITERION FOR~PERFECTLY BALANCED FUNCTIONS}

\def\titkol{An information based criterion for perfectly balanced functions}

\def\aut{O.\,A.~Logachev}

\def\autkol{O.\,A.~Logachev}

\titel{\tit}{\aut}{\autkol}{\titkol}

\vspace*{-11pt}


\noindent
Institute of Information Security Issues, 
M.\,V.~Lomonosov Moscow State University, 1~Michurinsky Pr., Moscow 119192, 
Russian Federation


\def\leftfootline{\small{\textbf{\thepage}
\hfill INFORMATIKA I EE PRIMENENIYA~--- INFORMATICS AND
APPLICATIONS\ \ \ 2018\ \ \ volume~12\ \ \ issue\ 4}
}%
 \def\rightfootline{\small{INFORMATIKA I EE PRIMENENIYA~---
INFORMATICS AND APPLICATIONS\ \ \ 2018\ \ \ volume~12\ \ \ issue\ 4
\hfill \textbf{\thepage}}}

\vspace*{6pt}







\Abste{The class of perfectly balanced functions is important for some areas 
of mathematics, e.\,g., combinatorics, coding theory, cryptography, symbolic 
dynamics, and automata theory. It turns out that perfectly balanced functions 
provide a~suitable mathematical tool for description and studying of convolutional 
codes, cryptographic primitives, surjective endomorphisms of discrete 
dynamical systems, and information-lossless finite-state automata. 
Previously, Hedlund and Sumarokov proved criteria of perfect balancedness 
of functions, which are related to the property of being defect zero and 
information-lossless. The present author proves a~new criterion of 
the perfect balancedness property in terms of average mutual information. 
The author also describes 
a~polinomial-time inverting algorithm for perfectly balanced functions.}

\KWE{finite alfabet; discrete function; averege mutual information; perfect balancedness; 
perfectly balanced function; function of defect zero}



\DOI{10.14357/19922264180410}

%\vspace*{-14pt}

\Ack
\noindent
The work was supported by the Russian Foundation for Basic Research 
(project 16-01-00470-A).



%\vspace*{6pt}

  \begin{multicols}{2}

\renewcommand{\bibname}{\protect\rmfamily References}
%\renewcommand{\bibname}{\large\protect\rm References}

{\small\frenchspacing
 {%\baselineskip=10.8pt
 \addcontentsline{toc}{section}{References}
 \begin{thebibliography}{9}

\bibitem{2-log}
\Aue{Hedlund, G.\,A.} 1969. Endomorphisms and automorphisms of the shift dynamical 
system. \textit{Math. Syst. Theory} 3:320--375.

\bibitem{1-log}
\Aue{Sumarokov, S.\,N.} 1994. Zaprety dvoichnykh funktsiy i~obratimost' dlya odnogo 
klassa kodiruyushchikh ustroystv [Prohibitions of Boolean functions and 
invertibility for a~coding devices class]. 
\textit{Obozrenie prikladnoy i~promyshlennoy matematiki} 
[Surveys in Applied and Industrial Mathematics] 1:33--55.


\bibitem{4-log}
\Aue{Feinstein, A.} 1958. \textit{Foundations of information theory}. 
New York, NY: Mcgraw-Hill. 137~p.

\bibitem{3-log}
\Aue{Logachev, O.\,A., A.\,A.~Sal'nikov, S.\,V.~Smyshlyaev, and V.\,V.~Yashchenko.}
 2015. \textit{Bulevy funktsii v~teorii kodirovaniya i~kriptologii} 
 [Boolean functions in coding theory and cryptology]. Moscow: Lenand. 576~p.
 
\bibitem{5-log}
\Aue{Goldreich, O.} 2003. 
\textit{Foundations of cryptography. Vol.~I: Basic tools.} Cambridge, U.K.: 
Cambridge University Press. 372~p.
\bibitem{6-log}
\Aue{Kudryavcev, V.\,B., S.\,V.~Aleshin, and A.\,S.~Podkolzin.}
1985. \textit{Vvedenie v~teoriyu avtomatov} [Introduction to automata theory]. 
Moscow: Nauka. 320~p.
\end{thebibliography}

 }
 }

\end{multicols}

\vspace*{-6pt}

\hfill{\small\textit{Received September 3, 2018}}

%\pagebreak

%\vspace*{-18pt}

\Contrl

\noindent
\textbf{Logachev Oleg A.} (b.\ 1950)~--- 
Candidate of Science(PhD) in physics and mathematics, Head of Department, 
Institute of Information Security Issues, M.\,V.~Lomonosov Moscow State University, 
1~Michurinsky Pr., Moscow 119192, Russian Federation; \mbox{logol@iisi.msu.ru}
\label{end\stat}

\renewcommand{\bibname}{\protect\rm Литература}       