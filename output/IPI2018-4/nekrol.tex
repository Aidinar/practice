   \vspace*{-48pt}

\begin{center}
\vspace*{6pt}
\mbox{%
\epsfxsize=79mm
\epsfbox{eme-1.eps}
}
\end{center}

\vspace*{6pt} %Академик


   \begin{center}
\fbox{\Large\textbf{Станислав Васильевич Емельянов
}}\\[12pt]
\textbf{\large 18.05.1929--15.11.2018}
   \end{center}


   %\vspace*{2.5mm}

   \vspace*{5mm}

   \thispagestyle{empty}

%\

%\vspace*{-12pt}


Редакционная коллегия журнала <<Информатика и~её применения>> 
с~глубоким прискорбием извещает, что 15~ноября 2018~года на 90-м году жизни 
скончался первый Главный редактор журнала <<Информатика и~её применения>>, 
сопредседатель Редакционного совета и член Редколлегии журнала Станислав Ва\-силь\-евич 
Емельянов~---  академик РАН, профессор, выдающийся ученый 
в~области автоматики, системного анализа и~информатики, научный руководитель 
ФИЦ ИУ РАН. 

Станислав Ва\-силь\-евич Емельянов~--- 
основоположник многих новых научных направлений, в том числе теории 
систем с переменной структурой, теории бинарного управления и теории 
новых типов обратной связи, ориентированных на решение задач управления в условиях 
неопределенности и сильной переменчивости характеристик объекта. 
Эти теории получили широкое мировое признание, активно развиваются и используются 
при решении актуальных практических задач в важнейших отраслях техники и промышленного 
производства. Он~--- основатель крупной научной школы. Им подготовлена большая группа 
докторов и кандидатов наук. Среди его учеников~--- 
академики и чле\-ны-кор\-рес\-пон\-ден\-ты РАН. С.\,В.~Емельянов~--- 
автор 25~книг, 278~статей в ведущих научных журналах, 72~авторских свидетельств 
на изобретения. 

Станислав Васильевич Емельянов~--- лауреат: Ленинской премии (1972), 
Государственной премии СССР (1980), Премии Совета министров СССР (1981), 
Государственной премии РФ (1994), Премии Правительства РФ в области 
науки и технологий (2009), Премии Правительства РФ в области образования (2012). 
Он награжден орденами: Октябрьской революции (1974), Дружбы народов (1979), 
За заслуги перед Отечеством IV~степени (1999) и~III~степени (2004), Почета (2010). 

Станислав Васильевич вложил много сил в создание, становление и развитие 
нашего журнала, активно работал в Редакционном совете и Редколлегии журнала. 
Глубокая научная эрудиция, разносторонние знания, высокий профессионализм 
снискали любовь и уважение к нему коллег и друзей.

\bigskip

Память о Станиславе Васильевиче Емельянове навсегда сохранится в наших сердцах.