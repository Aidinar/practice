\def\stat{gorsh+korolev}

\def\tit{ОПРЕДЕЛЕНИЕ ЭКСТРЕМАЛЬНОСТИ ОБЪЕМОВ ОСАДКОВ НА~ОСНОВЕ МОДИФИЦИРОВАННОГО 
МЕТОДА ПРЕВЫШЕНИЯ ПОРОГОВОГО ЗНАЧЕНИЯ$^*$}

\def\titkol{Определение экстремальности объемов осадков на~основе %модифицированного 
метода превышения порогового значения}

\def\aut{А.\,К.~Горшенин$^1$, В.\,Ю.~Королев$^2$}

\def\autkol{А.\,К.~Горшенин, В.\,Ю.~Королев}

\titel{\tit}{\aut}{\autkol}{\titkol}

\index{Горшенин А.\,К.}
\index{Королев В.\,Ю.}
\index{Gorshenin A.\,K.}
\index{Korolev V.\,Yu.}




{\renewcommand{\thefootnote}{\fnsymbol{footnote}} \footnotetext[1]
{Работа выполнена при поддержке РФФИ (проект 17-07-00851) 
и~Стипендии Президента Российской Федерации молодым ученым и~аспирантам 
(СП-538.2018.5).}}


\renewcommand{\thefootnote}{\arabic{footnote}}
\footnotetext[1]{Институт проблем информатики Федерального исследовательского
центра <<Информатика и~управ\-ле\-ние>> Российской академии наук; факультет
вычислительной математики и~кибернетики Московского государственного университета 
им.\ М.\,В.~Ломоносова,
\mbox{agorshenin@frccsc.ru}}
\footnotetext[2]{Факультет вычислительной математики и~кибернетики
Московского государственного университета им.\ М.\,В.~Ломоносова;
Институт проб\-лем информатики Федерального исследовательского
центра <<Информатика и~управ\-ле\-ние>> Российской академии наук,
\mbox{vkorolev@cs.msu.ru}}


\vspace*{-8pt}

\Abst{Задача корректного определения того, какие наблюдения следует признавать 
экстремальными, чрезвычайно важна при изучении метеорологических явлений. 
Предложены восходящий и~нисходящий методы определения порогового (экстремального) 
уровня на основе использования теорем Реньи для редеющих потоков и~результатов 
Пи\-канд\-са\,--\,Бал\-ке\-мы\,--\,Де Хаана. На примере данных за~60~лет 
наблюдений в~Потсдаме и~Элисте продемонстрировано, что восходящий метод 
показывает отличные результаты для суточных объемов осадков, однако для 
дождливых периодов необходимо использовать нисходящий метод. 
Приведено сравнение результатов подобного непараметрического подхода с~параметрическим 
критерием, который был предложен авторами в~предшествующих работах.}

\KW{осадки; дождливые периоды; экстремальные наблюдения; пороговые значения; 
теорема Реньи; теорема Пи\-канд\-са\,--\,Бал\-ке\-мы\,--\,Де Хаана; 
проверка статистических гипотез; анализ данных}

\DOI{10.14357/19922264180403}
  
\vspace*{-4pt}


\vskip 10pt plus 9pt minus 6pt

\thispagestyle{headings}

\begin{multicols}{2}

\label{st\stat}

\section{Введение}

Оценки закономерностей и~тенденций в~наблюдениях аномально
экстремальных метеорологических явлений важны для понимания
процесса изменения климата. Однако известно, что различные методы
оценки экстремальных осадков в~применении к~разным моделям дают
существенно отличающиеся результаты. Чаще всего решение основано
на выборе порогового значения, определяемого как квантиль 
некоторого распределения~\cite{Groisman1999}. Однако изменения 
в~максимальных значениях объемов осадков (и,~соответственно, большее число 
превышений порога) не всегда
ведут к~качественному изменению того, какие именно события
действительно должны быть признаны экстремальными. В~частности,
увеличение доли осадков подобного рода может быть связано 
с~изменениями суммарного объема или с~увеличением интенсивности
осадков в~сочетании с~уменьшением числа дождливых дней~\cite{Zolina2008}.

Для решения задачи определения аномальных наблюдений в~климатологических
 задачах часто используется подход теории экстремальных значений, 
 называемый \verb"Peaks over Threshold" (\verb"PoT"; пики, превышающие 
 порог)~\cite{Leadbetter1991}. В~частности, известно использование подобного 
 подхода в~задачах, связанных со штормовыми волнами~\cite{Mendez2006}, 
 дневными температурами~\cite{Kysely2010}, 
 осадками~\cite{Begueria2006,Begueria2011,Roth2012}. 
 Для выявления кри\-ти\-че\-ских/экстре\-маль\-ных наблюдений в~информационных 
 потоках различной природы авторами данной статьи ранее была предложена 
 методология поиска порогового значения на основе модификации \verb"PoT"-ме\-то\-да 
 с~использованием результатов теоремы Реньи для редеющих потоков~\cite{Gorshenin2016a}. 
 В~настоящей работе подобный подход будет развит для определения аномально 
 экстремальных значений объемов осадков, в~том числе выпавших в~течение так 
 называемого дождливого периода (подряд идущих дней, в~которые наблюдались осадки). 
 Кроме того, приведено сравнение результатов указанного непараметрического 
 подхода с~параметрическим критерием, который был ранее предложен авторами 
 в~статье~\cite{Gorshenin2018as} для решения подобного класса задач.
 
 \begin{figure*}[b] %fig1
\vspace*{9pt}
 \begin{center}
 \mbox{%
 \epsfxsize=161.412mm 
 \epsfbox{gok-1.eps}
 }
 \end{center}
\vspace*{-6pt}
\Caption{Пороговый уровень, полученный для дневных объемов восходящим методом, 
Потсдам: (\textit{а})~распределение интервалов между превышениями порога;
(\textit{б})~распределение превышений порога; (\textit{в})~суточные объемы;
\textit{1}~--- гистограмма; \textit{2}~--- экспоненциальное распределение 
с~параметром~0,002;
\textit{3}~--- обобщенное распределение Парето с~параметрами~0,226,
9,196 и~30,2;
\textit{4}~--- данные; \textit{5}~--- порог~30,2~мм}\label{FigPotsdamPOT}
\end{figure*}

\vspace*{-18pt}

\section{Использование PoT-методологии для~суточных~объемов~осадков}
\label{DailyData}

\vspace*{-1pt}

Воспользуемся подходом на основе двух фундаментальных результатов~--- 
тео\-ре\-мы Реньи для редеющих потоков~\cite{Gnedenko1996} и~классического 
результата тео\-рии экстремальных значений, связанного с~именами Пикандса, 
Балкемы и~Де Хаана~\cite{Balkema1974,Pickands1975}, которые позволяют 
избегать априорных предположений о~данных~\cite{Gorshenin2017a}. 
Из указанных выше тео\-рем следует, что распределение разностей моментов 
превышения порогового значения должно соответствовать экспоненциальному закону, 
а~величины превышений данного порога~--- обобщенному распределению Парето, 
которое имеет следующий вид ($\xi\hm\in\mathbb{R}$~--- параметр формы,  
$\mu\hm\in\mathbb{R}$~--- сдвига,  $\sigma\hm>0$~--- масштаба):
\begin{equation*}
F_{\xi, \sigma, \mu}(y)=
\begin{cases}
1-\left(1+\fr{\xi(y-\mu)}{\sigma}\right)^{-1/\xi}\,,& \!\!\mbox{если }\xi\neq 0\,;\\
1-e^{-({y-\mu})/{\sigma}}& \!\!\mbox{иначе.}
\end{cases}
\end{equation*}

Таким образом, пороговое значение может быть определено в~рамках статистической 
процедуры, в~которой для каждого уровня, начиная с~некоторого значения, 
например минимума в~данных, с~заранее заданным шагом должны проверяться 
последовательно две гипотезы об экспоненциальности и~паретовости описанных 
выше объектов. В~случае\linebreak\vspace*{-12pt}

\columnbreak

\noindent
 принятия обеих текущий уровень может считаться 
экстремальным пороговым значением. Назовем данный метод, следуя 
статьям~\cite{Gorshenin2016a,Gorshenin2017a}, \textit{восходящим} 
(в~соответствии с~направлением сдвига порогового значения в~процессе анализа данных).

Продемонстрируем описанный подход на примере наблюдений за объемом дневных 
осадков за почти 60-летний период (1950--2008~гг.)\ 
в~городах Потсдаме и~Элисте. Процедура поиска автоматизирована с~помощью 
программного решения, созданного на встроенном языке программирования 
пакета \verb"MATLAB". Получено, что для Потсдама критический уровень 
составляет~30{,}2~мм (шаг изменения уровня~--- 0{,}01~мм, 
уровень значимости статистического критерия при проверке по  $\chi^2$-тес\-ту 
выбран равным~0{,}01). Для экспоненциального распределения 
$p_{\mathrm{знач}}\hm=0{,}07$ (параметр оценивается значением~0{,}002), 
для обобщенного Парето  $p_{\mathrm{знач}}\hm=0{,}29$ (параметры: 0{,}226, 
9{,}196 и~30{,}2).


На рис.~\ref{FigPotsdamPOT} продемонстрировано визуальное соответствие 
между экспериментальными данными и~подогнанными распределениями (см.\
 гистограммы и~аппроксимирующие кривые на рис.~1,\,\textit{а} и~1,\,\textit{б}).\linebreak\vspace*{-12pt}
 
 \pagebreak
 
 \end{multicols}
 
 \begin{figure*} %fig2
\vspace*{1pt}
 \begin{center}
 \mbox{%
 \epsfxsize=160.191mm 
 \epsfbox{gok-2.eps}
 }
 \end{center}
\vspace*{-9pt}
\Caption{Пороговый уровень, полученный для дневных объемов восходящим методом, 
Элиста:
(\textit{а})~распределение интервалов между превышениями порога;
(\textit{б})~распределение превышений порога; (\textit{в})~суточные объемы;
\textit{1}~--- гистограмма; \textit{2}~--- экспоненциальное распределение 
с~параметром~0,002;
\textit{3}~--- обобщенное распределение Парето 
с~параметрами~0,038, 9,009 и~26,5;
\textit{4}~--- данные; \textit{5}~--- порог 26,5~мм
}\label{FigElistaPOT}
%\vspace*{3pt}
\end{figure*}

 
 \begin{multicols}{2}
 
 \noindent 
 На рис.~1,\,\textit{в} продемонстрировано, что полученное
 пороговое значение превышается относительно небольшое число раз за весь 
 период наблюдений; таким образом, данные пики могут  рассматриваться как 
 <<подозрительные на экстремальность>>.





Для Элисты критический уровень составляет~26{,}5~мм. 
Для экспоненциального распределения $p_{\mathrm{знач}}\hm=0{,}84$ 
(параметр оценивается значением~0{,}002), для обобщенного Парето 
 $p_{\mathrm{знач}}\hm=0{,}6$\linebreak
  (параметры: 0{,}038, 9{,}009 и~36{,}5). 
 На рис.~\ref{FigElistaPOT} отоб\-ра\-же\-но соответствие между экспериментальными 
 данными и~подогнанными распределениями (рис.~2,\,\textit{а} и~2,\,\textit{б}). 
 На рис.~2,\,\textit{в} продемонстрировано, что полученное 
 пороговое значение снова превышается относительно небольшое число раз за весь 
 период наблюдений.
 
 %\vspace*{-12pt}

\section{Использование PoT-методологии для~объемов осадков за~дождливые периоды}

Отметим, что описанный в~разд.~\ref{DailyData} восходящий метод может быть 
реализован и~в~обратном на\-прав\-ле\-нии (с~точки зрения смещения уровня в~процессе 
анализа данных). Для этого необходимо задать верхнюю границу (например, совпадающую 
с~максимумом наблюдений) и~последовательно проверять гипотезы об 
экспоненциальности и~па\-ре\-то\-вости. 

В~начале работы метода число превышений поро\-га 
будет недостаточным для построения гис\-то\-грамм и~корректного оценивания параметров, 
поэтому необходимо обеспечить минимально приемлемый объем соответствующей выборки. 

В~работе~\cite{Gorshenin2016a} описана схожая процедура, однако в~качестве 
примера рассмотрены кумулятивные данные. 
В~данном разделе откажемся от ограничения на тип рассматриваемых данных и~будем 
считать \textit{нисходящим} методом описанный выше алгоритм. 

В~качестве примера рассмотрим суммарные объемы осадков, зарегистрированные в~течение 
дождливого периода, однако данный метод может быть использован и~для исходных 
суточных на\-блю\-де\-ний.
{\looseness=1

}


Для объемов осадков за дождливые периоды в~Потсдаме верхний критический уровень 
(полу-\linebreak\vspace*{-12pt}

\pagebreak

\end{multicols}

\begin{figure*} %fig3
\vspace*{1pt}
 \begin{center}
 \mbox{%
 \epsfxsize=157.261mm 
 \epsfbox{gok-3.eps}
 }
 \end{center}
\vspace*{-9pt}
\Caption{Нисходящий, восходящий и~параметрический методы определения экстремальности:
 (\textit{а})~распределение интервалов между превышениями порога;
(\textit{б})~распределение превышений порога; (\textit{в})~объемы осадков за 
дождливые периоды, Потсдам; \textit{1}~--- гистограмма; 
\textit{2}~--- экспоненциальное распределение с~параметром~0,216;
\textit{3}~--- обобщенное распределение Парето 
с~параметрами~0,076,
13,406 и~14,41;
\textit{4}~--- данные; \textit{5}~--- порог~14,41~мм;
\textit{6}~--- порог~57,2~мм; \textit{7}~--- абсолютно экстремальные значения;
\textit{8}~--- промежуточные; \textit{9}~--- относительно экстремальные значения}
\label{FigPotsdamVolWet}
\vspace*{-6pt}
\end{figure*}

\begin{multicols}{2}

\noindent
чен нисходящим методом) составляет~57{,}2~мм (шаг
 изменения уровня~--- 0{,}01~мм, 
уровень зна\-чи\-мости
 статистического критерия при проверке по  $\chi^2$-тес\-ту 
выбран равным~0{,}01). Для экспоненциального\linebreak распределения $p_{\mathrm{знач}}\hm=0{,}1$ 
(параметр оценивается значением~0{,}014), для обобщенного Парето 
$p_{\mathrm{знач}}\hm=0{,}03$ (параметры: 0{,}097, 16{,}95 и~57{,}2). 
Нижний критический уровень (получен восходящим методом) составляет~14{,}41~мм. 
Для экспоненциального распределения $p_{\mathrm{знач}}\hm=0{,}058$ 
(параметр оценивается значением~0{,}216), для обобщенного Парето 
$p_{\mathrm{знач}}\hm=0{,}29$ (параметры: 0{,}076, 13{,}406 и~14{,}41).
{\looseness=1

}

На рис.~\ref{FigPotsdamVolWet} продемонстрировано визуальное соответствие 
между экспериментальными данными и~подогнанными распределениями (см.\
 гистограммы и~аппроксимирующие кривые на рис.~3,\,\textit{а}
 и~3,\,\textit{б}) для восходящего 
 метода. Рисунок~3,\,\textit{в} демонстрирует разницу пороговых значений, 
 полученных двумя способами. Очевидно, что в~отличие от случая разд.~\ref{DailyData} 
 (исходные данные) восходящий метод устанавливает критическую планку слишком низко 
 и~в~рассмотрении окажется избыточное количество пиков.

Для объемов осадков за дождливые периоды в~Элисте верхний критический уровень 
(получен нисходящим методом) составляет~28~мм (шаг изменения уровня~--- 0{,}01~мм, 
уровень значимости статистического критерия при проверке по  $\chi^2$-тес\-ту 
выбран равным~0{,}01). Для экспоненциального 
распределения $p_{\mathrm{знач}}\hm=0{,}082$ 
(параметр оценивается значением~0{,}029), для обобщенного 
Парето $p_{\mathrm{знач}}\hm=0{,}44$ 
(параметры: $-$0{,}095, 13{,}66 и~28). Нижний критический уровень 
(получен восходящим методом) составляет~10{,}71~мм.
 Для экспоненциального распределения $p_{\mathrm{знач}}\hm=0{,}062$ (параметр
  оценивается значением~0{,}183), для обобщенного Парето $p_{\mathrm{знач}}\hm=0{,}21$ 
  (параметры:  0{,}066, 9{,}586 и~10{,}71).



На рис.~\ref{FigElistaVolWet} продемонстрировано визуальное соответствие между 
экспериментальными данными и~подогнанными распределениями (см.\
 гистограммы и~аппроксимирующие кривые на рис.~4,\,\textit{а}
 и~4,\,\textit{б}) для восходящего метода. 
 Рисунок~4,\,\textit{в} демонстрирует разницу пороговых значений, 
 полученных двумя
 способами. Снова восходящий метод устанавливает критическую 
 планку слишком низко, и~более интересным представляется рассмотрение уровня 
 для нисходящего метода.
 
 \pagebreak
 
 \end{multicols}
 
 \begin{figure*} %fig4
\vspace*{1pt}
 \begin{center}
 \mbox{%
 \epsfxsize=157.761mm 
 \epsfbox{gok-4.eps}
 }
 \end{center}
\vspace*{-9pt}
\Caption{Нисходящий, восходящий и~параметрический методы определения экстремальности:
(\textit{а})~распределение интервалов между превышениями порога;
(\textit{б})~распределение превышений порога; 
(\textit{в})~объемы осадков за дождливые периоды, Элиста;
\textit{1}~--- гистограмма; 
\textit{2}~--- экспоненциальное распределение с~па\-ра\-мет\-ром~0,183;
\textit{3}~--- обобщенное распределение Парето с~па\-ра\-мет\-ра\-ми~0,066,
9,586 и~10,71;
\textit{4}~--- данные; \textit{5}~--- порог~10,71~мм;
\textit{6}~--- порог~28~мм; \textit{7}~--- абсолютно экстремальные значения;
\textit{8}~--- промежуточные; \textit{9}~--- относительно 
экстремальные значения}
\label{FigElistaVolWet}
%\vspace*{9pt}
\end{figure*}

\begin{multicols}{2}



\begin{figure*}[b] %fig5
\vspace*{4pt}
 \begin{center}
 \mbox{%
 \epsfxsize=106.504mm 
 \epsfbox{gok-5.eps}
 }
 \end{center}
\vspace*{-9pt}

\Caption{Алгоритм определения экстремальных значений}\label{FigMethods}
\end{figure*}


\section{Статистический тест экстремальности объемов, основанный 
на~распределении Снедекора--Фишера}
\label{StatExtremes}

На рис.~\ref{FigPotsdamVolWet},\,\textit{в} и~\ref{FigElistaVolWet},\,\textit{в} 
нанесены дополнительные маркеры (треугольники, круги и~квадраты), 
которыми отмечены экстремальные наблюдения, полученные в~соответствии 
со статистическим критерием, описанным в~работе~\cite{Gorshenin2018as}. 

Для перехода к~естественной временной шкале на со\-от\-вет\-ст\-ву\-ющие графики 
наносятся пики, по величине совпадающие с~объемами осадков, выпавших за 
дождливые периоды, а~их расположение на временн$\acute{\mbox{о}}$й шкале выбрано совпадающим с~днем 
начала выпадения осадков. Используемый для разметки параметрический критерий 
существенным образом использует установленные 
в~статьях~\cite{Gorshenin2017a,Gorshenin2017b,Gorshenin2017c,Gorshenin2017d,Gorshenin2018bs,Gorshenin2018c} 
факты о~соответствии распределений длительностей дождливых периодов отрицательному 
биномиальному распределению, а~их объемов~--- гам\-ма-рас\-пре\-де\-ле\-нию. 
%
Ниже опишем соответствующую процедуру, предложенную в~работе~\cite{Gorshenin2018as}.

Пусть $m\in\mathbb{N}$ и~$G^{(1)}_{r,\mu},G^{(2)}_{r,\mu},\ldots,G^{(m)}_{r,\mu}$~--- 
независимые случайные величины с~общим гам\-ма-рас\-пре\-де\-ле\-ни\-ем 
с~параметрами $r\hm>0$ и~$\mu\hm>0$. Рассмотрим статистику
\begin{equation*}
R_0=\fr{(m-1)G^{(1)}_{r,\mu}}{G^{(2)}_{r,\mu}+
\cdots+G^{(m)}_{r,\mu}}\eqd
\fr{k}{r}\,\fr{G_{r,\mu}}{G_{k,\mu}}\eqd\fr{k}{r}\,\fr{G_{r,1}}{G_{k,1}}\eqd
Q_{r,k}\,,
\end{equation*}
где $k=(m-1)r$; $Q_{k,r}$~---  случайная величина с~распределении 
Сне\-де\-ко\-ра--Фи\-ше\-ра с~параметрами $k\hm>0$ и~$r\hm>0$. Предположим, что 
$V_1,\ldots,V_m$~--- суммарные объемы осадков, выпавших за~$m$~дождли\-вых периодов. 
Рассмотрим следующий выборочный аналог величины~$R_0$:
\begin{equation*}
SR_{0,i}=\fr{(m-1)V_i}{\sum\nolimits_{j\neq i}V_j}\,.
\end{equation*}



Сформулируем гипотезу~$H_0$ в~следующем виде: <<объем осадков~$V_i$ 
не является аномально большим относительно остальных $m\hm-1$ величин>>.
 Тогда статистика~$SR_{0,i}$ имеет распределение Сне\-де\-ко\-ра--Фи\-ше\-ра 
 с~параметрами $r\hm>0$ и~$k\hm=(m\hm-1)r$. Обозначим через $q_{r,k}(1\hm-\alpha)$ 
 квантиль распределения Сне\-де\-ко\-ра--Фи\-ше\-ра уровня $(1\hm-\alpha)$, 
 $\alpha\hm\in(0,1)$. В~случае, если $SR_{0,i}\hm>q_{r,k}(1\hm-\alpha)$, гипотеза~$H_0$ 
 отвергается и~объем~$V_i$ должен быть признан экстремально большим. 
 Уровень значимости критерия установлен на уровне~$\alpha$.

Описанная процедура может быть расширена с~помощью метода скользящего окна. 
Задавая его ширину (т.\,е.\ число элементов в~выборке объемов дождливых периодов) 
равной~$m$ и~сдвигая каждый\linebreak
 раз на один элемент вправо (направление астрономического 
времени на графике с~экспериментальными данными), можно последовательно проверить, 
является ли каждый конкретный объем экстремаль\-ным относительно других в~смысле 
описанного выше критерия. Данная процедура может быть полезна в~ситуации, 
когда в~одно окно попадают достаточно близкие по абсолютной величине объемы, 
мало отличающиеся от максимального. 

Таким образом, в~рамках данного подхода 
каж\-дый элемент (начиная с~$m$) проверяется в~точ\-ности~$m$~раз. Тогда возможны 
сле\-ду\-ющие вари-\linebreak анты:\\[-15pt]
\begin{enumerate}[(1)]
\item элемент признается аномальным во всех~$m$~случаях 
и~соответствующее значение может считаться \textit{абсолютно} экстремальным;
\item 
элемент признается аномальным более чем в~половине случаев (т.\,е.\ 
не меньше чем на $\lceil  m/2\rceil$ положениях окна)  и~соответствующее значение 
может считаться \textit{промежуточным} экстремумом; 
\item элемент признается 
аномальным менее чем в~половине случаев и~соответствующее значение может 
считаться \textit{относительно} экстремальным.
\end{enumerate}

 Для приведения к~астрономическому 
времени необходимо найти среднюю продолжительность дождливых и~сухих периодов 
(используется модель на основе отрицательного биномиального распределения) и~умножить 
на размер окна. На рис.~\ref{FigPotsdamVolWet} и~\ref{FigElistaVolWet} 
расчеты проводились для периода в~360~дней (при этом $m\hm=62$).

\vspace*{-6pt}

\section{Алгоритм определения экстремальности значений~объемов}

Таким образом, алгоритм определения экстремальных значений 
(и~сравнения результатов, полученных разными способами) может быть пред\-став\-лен в~виде 
 блок-схе\-мы (рис.~\ref{FigMethods}).



К данным $\mathrm{Data}$ последовательно применяются восходящий, 
нисходящий и~параметрический методы поиска экстремальных значений. 
Для первых двух из них задается начальное значение ($\mathrm{level}\hm=0$ для нисходящего 
и~$\mathrm{LEVEL}\hm=\max(\mathrm{Data})$ для восходящего). Далее последовательно проверяются 
условия для теорем Реньи и~Пи\-канд\-са\,--\,Бал\-ке\-мы\,--\,Де Хаана при 
некотором заданном уровне зна\-чи\-мости~$\alpha$. Процедура \verb"StatExtremes()" 
реализует метод, описанный в~разд.~\ref{StatExtremes}. 
В~результате можно выявить наблюдения, которые признаются экстремальными с~помощью 
каждого из этих методов, как продемонстрировано на рис.~\ref{FigPotsdamVolWet} 
и~\ref{FigElistaVolWet}. Стоит отметить, что восходящий и~нисходящий методы 
могут быть использованы для любых данных, а~па\-ра\-мет\-ри\-че\-ский подход ориентирован 
на ряд дополнительных предположений о~распределениях характеристик наблюдений 
(а~также неотрицательность исходных значений).

%\vspace*{-6pt}

\section{Заключение}

\vspace*{-1pt}

Итак, в~работе рассмотрены восходящий и~нисходящий методы определения 
порогового (экстремального) уровня в~рамках идеологии \verb"PoT". 
Восходящий метод показывает отличные результаты для суточных объемов осадков, 
однако для дождливых периодов необходимо использовать нисходящий метод. 
Кроме того, предложено сравнение результатов со статистическим критерием, 
который построен в~рамках определенной (при этом весьма хорошо 
согласующейся с~реальными данными) вероятностной модели. 
Указанный метод позволяет улучшить полученные результаты, однако требует ряда 
априорных предположений о~распределениях характеристик исследуемых данных.

\vspace*{-6pt}

{\small\frenchspacing
 {%\baselineskip=10.8pt
 \addcontentsline{toc}{section}{References}
 \begin{thebibliography}{99}
 
% \vspace*{-1pt}
 
\bibitem{Groisman1999} 
\Au{Groisman~P.\,Y., Karl~T.\,R., Easterling~D.\,R., \textit{et al.}} Changes
in the probability of heavy precipitation: Important indicators of climatic change~// 
J.~Climate, 1999. Vol.~42. P.~243--285.

\bibitem{Zolina2008} 
\Au{Zolina~O., Simmer~C., Kapala~A., Bachner~S., Gulev~S., Maechel~H.} 
Seasonally dependent changes of precipitation extremes over Germany since~1950 
from a~very dense observational network~// J.~Geophys. Res., 2008. Vol.~113. Art. No.\,D06110.

\bibitem{Leadbetter1991} 
\Au{Leadbetter~M.\,R.} On a basis for ``Peaks over Threshold'' modeling~// 
Stat. Probabil. Lett., 1991. Vol.~12. Iss.~4. P.~357--362.

\bibitem{Mendez2006} 
\Au{Mendez~F.\,J., Menendez~M., Luceno~A., Losada~I.\,J.} 
Estimation of the long-term variability of extreme significant wave height 
using a~time-dependent Peak over Threshold (PoT) model~// 
J.~Geophys. Res. Oceans, 2006. Vol.~111. Iss.~C7. Art.\ No.\,C07024.

\bibitem{Kysely2010}
 \Au{Kysely~J., Picek~J., Beranova~R.} 
Estimating extremes in climate change simulations using the 
peaks-over-threshold method with a non-stationary threshold~// 
Global Planet. Change, 2010. Vol.~72. Iss.~1-2. P.~55--68.

\bibitem{Begueria2006} 
\Au{Begueria~S., Vicente-Serrano~S.\,M.} 
Mapping the hazard of extreme rainfall by peaks over threshold 
extreme value analysis and spatial regression techniques~// 
J.~Appl. Meteorol. Clim., 2006. Vol.~45. Iss.~1. P.~108--124.

\bibitem{Begueria2011} 
\Au{Begueria~S., Angulo-Martinez~M., Vicente-Serrano~S.\,M., 
Lopez-Moreno~I.\,J., El-Kenawy~A.} Assessing trends in extreme precipitation 
events intensity and magnitude using non-stationary peaks-over-threshold analysis: 
A~case study in northeast Spain from 1930 to 2006~// 
Int. J.~Climatol., 2011. Vol.~31. Iss.~142. P.~2102--2114.

\bibitem{Roth2012} 
\Au{Roth~M., Buishand~T.\,A., Jongbloed~G.,  Tank~A.\,M.\,G., 
van Zanten~J.\,H.} A~regional peaks-over-threshold model in a~nonstationary climate~// 
Water Resour. Res., 2012. Vol.~48. Art. No.\,W11533.

\bibitem{Gorshenin2016a} 
\Au{Gorshenin~A.\,K., Korolev~V.\,Yu.} 
A~methodology for the identification of extremal loading in data flows in 
information systems~// Comm. Com. Inf. Sc., 2016. Vol.~638. P.~94--103.

\bibitem{Gorshenin2018as} 
\Au{Korolev~V.\,Yu., Gorshenin~A.\,K., Belyaev~K.\,P.} 
Statistical tests for extreme precipitation volumes~// \mbox{ArXiv}:\linebreak
1802.02928v3 [stat.ME],  2018.

\bibitem{Gnedenko1996} 
\Au{Gnedenko~B.\,V., Korolev~V.\,Yu.} 
Random summation: Limit theorems and applications.~--- 
Boca Raton, FL, USA: CRC Press, 1996. 288~p.

\bibitem{Balkema1974} 
\Au{Balkema~A., de Haan~L.} 
Residual life time at great age~// Annals Probability, 1974. Vol.~2. No.\,5.
P.~792--804.

\bibitem{Pickands1975} 
\Au{Pickands~J.} 
Statistical inference using extreme order statistics~// 
Ann. Stat., 1975. Vol.~3. No.\,1. P.~119--131.

\bibitem{Gorshenin2017a} 
\Au{Горшенин~А.\,К.} О~некоторых математических
и программных методах построения структурных моделей информационных
потоков~// Информатика и~её применения, 2017. Т.~11. Вып.~1. C.~58--68.

\bibitem{Gorshenin2017b}  
\Au{Горшенин~А.\,К.} 
Анализ ве\-ро\-ят\-но\-ст\-но-ста\-ти\-сти\-че\-ских характеристик осадков на 
основе паттернов~// Информатика и~её применения, 2017. Т.~11. Вып.~4. C.~38--46.

\bibitem{Gorshenin2017c} 
\Au{Korolev~V.\,Yu., Gorshenin~A.\,K., Gulev~S.\,K.,
Belyaev~K.\,P., Grusho~A.\,A.} Statistical analysis of precipitation events~// AIP
Conf. Proc., 2017. Vol.~1863. P.~\mbox{090011-1}--\mbox{090011-4}.

\bibitem{Gorshenin2017d} 
\Au{Королев~В.\,Ю., Горшенин~А.\,К.} 
О~распределении вероятностей экстремальных осадков~// Докл. РАН,
2017. Т.~477. Вып.~5. C.~604--609.

\bibitem{Gorshenin2018bs} 
\Au{Gorshenin~A.\,K., Kuzmin~V.\,Yu.} Neural
network forecasting of precipitation volumes using patterns~// Pattern 
Recogn. Image
Anal., 2018. Vol.~28. No.\,3. P.~450--461.

\bibitem{Gorshenin2018c}
\Au{Gorshenin~A.\,K., Korolev~V.\,Yu.} Scale mixtures of
Frechet distributions as asymptotic approximations of extreme precipitation~// 
J.~Math. Sci., 2018. Vol.~234. No.\,6. P.~886--903.
 \end{thebibliography}

 }
 }

\end{multicols}

\vspace*{-11pt}

\hfill{\small\textit{Поступила в~редакцию 15.10.18}}

\vspace*{5pt}

%\pagebreak

%\newpage

%\vspace*{-28pt}

\hrule

\vspace*{2pt}

\hrule

\vspace*{-4pt}

\def\tit{DETERMINING THE~EXTREMES OF~PRECIPITATION VOLUMES BASED ON~THE~MODIFIED 
``PEAKS OVER THRESHOLD'' METHOD}

\def\titkol{Determining the~extremes of~precipitation volumes based on~the~modified 
``Peaks over Threshold'' method}


\def\aut{A.\,K.~Gorshenin$^{1,2}$ and V.\,Yu.~Korolev$^{1,2}$}

\def\autkol{A.\,K.~Gorshenin and V.\,Yu.~Korolev}

\titel{\tit}{\aut}{\autkol}{\titkol}

\vspace*{-11pt}




\noindent
$^1$Institute of Informatics Problems, Federal Research Center 
``Computer Science and Control'' of the Russian\linebreak
$\hphantom{^1}$Academy of Sciences,  44-2~Vavilov Str., Moscow 119333, Russian Federation

\noindent
$^2$Faculty of Computational Mathematics and Cybernetics, M.\,V.~Lomonosov Moscow
State University,  Leninskie\linebreak
$\hphantom{^1}$Gory,  Moscow 119991, GSP-1, Russian Federation

\def\leftfootline{\small{\textbf{\thepage}
\hfill INFORMATIKA I EE PRIMENENIYA~--- INFORMATICS AND
APPLICATIONS\ \ \ 2018\ \ \ volume~12\ \ \ issue\ 4}
}%
 \def\rightfootline{\small{INFORMATIKA I EE PRIMENENIYA~---
INFORMATICS AND APPLICATIONS\ \ \ 2018\ \ \ volume~12\ \ \ issue\ 4
\hfill \textbf{\thepage}}}

\vspace*{2pt}



\Abste{The problem of correct determination of extreme observations is 
very important when studying meteorological phenomena. The paper proposes 
ascending and descending methods for finding the threshold for extremes 
based on the R$\acute{\mbox{e}}$nyi theorem for thinning flows and the 
Pikands\,--\,Balkema\,--\,De Haan results. Using the observation data for~60~years 
for Potsdam and Elista, it is demonstrated that the ascending method can 
present excellent results for daily precipitation but for volumes 
of wet periods, the descending method should be used. The results of 
such nonparametric approaches are compared with the parametric criterion 
proposed in the previous papers by the authors.}


\KWE{precipitation; wet periods; extreme values; thresholds; R$\acute{\mbox{e}}$nyi theorem; 
Pickands\,--\,Balkema\,--\,de Haan theorem; testing statistical hypotheses;
 data analysis}


\DOI{10.14357/19922264180403}

\vspace*{-18pt}

\Ack
\noindent
The research is supported by the Russian Foundation 
for Basic Research (project~17-07-00851) and the 
RF Presidential scholarship program (No.\,538.2018.5).


%\vspace*{6pt}

  \begin{multicols}{2}

\renewcommand{\bibname}{\protect\rmfamily References}
%\renewcommand{\bibname}{\large\protect\rm References}

{\small\frenchspacing
 {%\baselineskip=10.8pt
 \addcontentsline{toc}{section}{References}
 \begin{thebibliography}{99}
\bibitem{1-gorkor}
\Aue{Groisman,~P.\,Y., T.\,R.~Karl, D.\,R.~Easterling, \textit{et al.}} 1999. Changes
in the probability of heavy precipitation: Important indicators of climatic change. 
\textit{J.~Climate} 42:243--285.

\bibitem{2-gorkor}
\Aue{Zolina,~O., C.~Simmer, A.~Kapala, S.~Bachner, S.~Gulev, and H.~Maechel.} 
2008. Seasonally dependent changes of precipitation extremes over Germany 
since 1950 from a~very dense observational network. 
\textit{J.~Geophys. Res.} 113: D06110.

\bibitem{3-gorkor}
\Aue{Leadbetter, M.\,R.} 1991. 
On a~basis for ``Peaks over Threshold'' modeling. 
\textit{Stat. Probabil. Lett.} 12(4):357--362.

\bibitem{4-gorkor}
\Aue{Mendez,~F.\,J., M.~Menendez, A.~Luceno, and I.\,J.~Losada.}
 2006. Estimation of the long-term variability of extreme significant wave 
 height using a time-dependent Peak over Threshold (PoT) model. \textit{J.~Geophys. 
 Res. Oceans} 111(C7):C07024.

\bibitem{5-gorkor}
\Aue{Kysely,~J., J.~Picek, and R.~Beranova.} 2010. 
Estimating extremes in climate change simulations using the peaks-over-threshold 
method with a non-stationary threshold. 
\textit{Global Planet. Change} 72(1--2):55--68.

\bibitem{6-gorkor}
\Aue{Begueria,~S., and S.\,M.~Vicente-Serrano.} 2006. Mapping the hazard 
of extreme rainfall by peaks over threshold extreme value analysis and 
spatial regression techniques. \textit{J.~Appl. Meteorol. Clim.} 
45(1):108--124.

\bibitem{7-gorkor}
\Aue{Begueria,~S., M.~Angulo-Martinez, S.\,M.~Vicente-Serrano, 
I.\,J.~Lopez-Moreno, and A.~El-Kenawy.} 2011. Assessing trends 
in extreme precipitation events intensity and magnitude using non-stationary 
peaks-over-threshold analysis: A~case study in northeast Spain from 1930 to~2006. 
\textit{Int. J.~Climatol.} 31(142):2102--2114.

\bibitem{8-gorkor}
\Aue{Roth,~M., T.\,A.~Buishand, G.~Jongbloed,  A.\,M.\,G.~Tank, and J.\,H.~van Zanten.}
 2012. A~regional peaks-over-threshold model in a~nonstationary climate. 
 \textit{Water Resour. Res.} 48:W11533.

\bibitem{9-gorkor}
\Aue{Gorshenin,~A.\,K., and V.\,Yu.~Korolev.} 2016. 
A~methodology for the identification of extremal loading in data flows in 
information systems. \textit{Comm. Com. Inf. Sc.} 638:94--103.

\bibitem{10-gorkor}
\Aue{Korolev,~V.\,Yu., A.\,K.~Gorshenin, and K.\,P.~Belyaev.} 
2018. Statistical tests for extreme precipitation volumes. 
1802.02928 [stat.ME]. Available at:
{\sf https://arxiv.org/\linebreak abs/1802.02928v3} (accessed December~3, 2018).

\bibitem{11-gorkor}
\Aue{Gnedenko,~B.\,V., and V.\,Yu.~Korolev.} 1996. \textit{Random summation: Limit 
theorems and applications}. Boca Raton, FL: CRC Press. 288~p.

\bibitem{12-gorkor}
\Aue{Balkema,~A., and L.~de Haan.} 1974. Residual life time at great age. 
\textit{Ann. Probab.} 2(5):792--804.

\bibitem{13-gorkor}
\Aue{Pickands,~J.} 1975. Statistical inference using extreme order statistics. 
\textit{Ann. Stat.} 3(1):119--131.

\bibitem{14-gorkor}
\Aue{Gorshenin,~A.\,K.} 2017. O~nekotorykh matematicheskikh i~programmnykh metodakh
postroeniya strukturnykh modeley informatsionnykh potokov [On some mathematical and
programming methods for construction of structural models of information flows]. 
\textit{Informatika i~ee Primeneniya ~--- Inform. Appl.} 11(1):58--68.

\bibitem{15-gorkor}
\Aue{Gorshenin,~A.\,K.} 2017. 
Analiz veroyatnostno-statisticheskikh kharakteristik osadkov 
na osnove patternov [Pattern-based analysis of probabilistic and 
statistical characteristics of precipitations]. 
\textit{Informatika i~ee Primeneniya~--- Inform. Appl.} 11(4):38--46.

\bibitem{16-gorkor}
\Aue{Korolev,~V.\,Yu., A.\,K.~Gorshenin, S.\,K.~Gulev, K.\,P.~Belyaev, and A.\,A.~Grusho.}
2017. Statistical analysis of precipitation events. \textit{AIP Conf. Proc.} 
1863:\mbox{090011-1}--\mbox{090011-4}.

\bibitem{17-gorkor}
\Aue{Korolev,~V.\,Yu., and A.\,K.~Gorshenin.} 2017. 
The probability distribution of extreme precipitation. \textit{Dokl. Earth Sci.} 
477(2):1461--1466.

\bibitem{18-gorkor}
\Aue{Gorshenin,~A.\,K., and V.\,Yu.~Kuzmin.}
 2018. Neural network forecasting of precipitation volumes using patterns. 
 \textit{Pattern Recogn. Image Anal.} 28(3):450--461.

\bibitem{19-gorkor}
\Aue{Gorshenin,~A.\,K., and V.\,Yu.~Korolev.} 2018. Scale mixtures of
Frechet distributions as asymptotic approximations of extreme precipitation. 
\textit{J.~Math. Sci.} 234(6):886--903.
\end{thebibliography}

 }
 }

\end{multicols}

\vspace*{-6pt}

\hfill{\small\textit{Received October 15, 2018}}

%\pagebreak

%\vspace*{-18pt}



\Contr

\noindent
\textbf{Gorshenin Andrey K.} (b.\ 1986)~--- Candidate of Science (PhD) in physics and
mathematics, associate professor, leading scientist, Institute of Informatics Problems,
Federal Research Center ``Computer Science and Control'' of the Russian Academy of
Sciences, 44-2~Vavilov Str., Moscow 119333, Russian Federation;  leading scientist, 
Faculty
of Computational Mathematics and Cybernetics, M.\,V.~Lomonosov Moscow State University, 
 Leninskie Gory,  Moscow 119991,  GSP-1, Russian Federation; \mbox{agorshenin@frccsc.ru}

\vspace*{3pt}

\noindent
\textbf{Korolev Victor Yu.} (b.\ 1954)~--- Doctor of Science (PhD) in physics and
mathematics, professor, Head of Department, 
Faculty of Computational Mathematics and Cybernetics, 
M.\,V.~Lomonosov Moscow State University,  Leninskie Gory,  Moscow 119991,  GSP-1,
Russian Federation; leading scientist, Institute of Informatics Problems, 
Federal Research Center ``Computer Science and Control'' of the 
Russian Academy of Sciences, 44-2~Vavilov Str., Moscow 119333, Russian Federation; 
professor, Hangzhou Dianzi University, Xiasha Higher Education Zone, Hangzhou 310018, 
China; \mbox{vkorolev@cs.msu.ru}
\label{end\stat}

\renewcommand{\bibname}{\protect\rm Литература}    