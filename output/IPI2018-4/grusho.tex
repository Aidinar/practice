\def\stat{grusho}

\def\tit{МЕТОДЫ ИДЕНТИФИКАЦИИ ЗАХВАТА ХОСТА В~РАСПРЕДЕЛЕННОЙ 
ИНФОРМАЦИОННО-ВЫЧИСЛИТЕЛЬНОЙ СИСТЕМЕ, ЗАЩИЩЕННОЙ С~ПОМОЩЬЮ 
МЕТАДАННЫХ$^*$}

\def\titkol{Методы идентификации захвата хоста в~РИВС, %распределенной 
%информационно-вычислительной системе, 
защищенной с~помощью  метаданных}

\def\aut{А.\,А.~Грушо$^1$, Н.\,А.~Грушо$^2$, М.\,В.~Левыкин$^3$, Е.\,Е.~Тимонина$^4$}

\def\autkol{А.\,А.~Грушо, Н.\,А.~Грушо, М.\,В.~Левыкин, Е.\,Е.~Тимонина}

\titel{\tit}{\aut}{\autkol}{\titkol}

\index{Грушо А.\,А.}
\index{Грушо Н.\,А.}
\index{Левыкин М.\,В.}
\index{Тимонина Е.\,Е.}
\index{Grusho A.\,A.}
\index{Grusho N.\,A.}
\index{Levykin M.\,V.}
\index{Timonina E.\,E.}




{\renewcommand{\thefootnote}{\fnsymbol{footnote}} \footnotetext[1]
{Работа поддержана РФФИ (проект 18-07-00274).}}


\renewcommand{\thefootnote}{\arabic{footnote}}
\footnotetext[1]{Институт проблем информатики Федерального исследовательского центра <<Информатика 
и~управление>> Российской академии наук,
\mbox{grusho@yandex.ru}}
\footnotetext[2]{Институт проблем информатики Федерального исследовательского центра <<Информатика 
и~управление>> Российской академии наук,
\mbox{info@itake.ru}}
\footnotetext[3]{Институт проблем информатики Федерального исследовательского центра <<Информатика 
и~управление>> Российской академии наук,
\mbox{de\_shiko@yahoo.com}}
\footnotetext[4]{Институт проблем информатики Федерального исследовательского центра <<Информатика 
и~управление>> Российской академии наук,
\mbox{eltimon@yandex.ru}}

%\vspace*{8pt}


  
    \Abst{Рассматривается модель распределенной ин\-фор\-ма\-ци\-он\-но-вы\-чис\-ли\-тель\-ной  
сис\-те\-мы (РИВС), в~которой разрешения на сетевые соединения определяются с~по\-мощью 
метаданных (МД). Метаданные служат упрощением моделей биз\-нес-про\-цес\-сов (БП). Доказано, что 
нарушитель информационной без\-опас\-ности (ИБ), захвативший хост и~аккуратно атакующий 
систему с~помощью изменений выходных данных решаемых на этом хосте задач, не 
выявляется на уровне МД. Проб\-ле\-ма связана с~тем, что модели  
БП и,~следовательно, МД оперируют переменными, для которых 
изменения конкретных значений не отражаются в~их описании. Исключение составляют 
случаи выхода на запрещенные значения, например за пределы областей определения 
и~множества значений функций, из которых построены информационные технологии (ИТ).
    Предложены дополнительные варианты защиты информации, учитывающие 
<<невидимость>> подобных нарушений ИБ на уровне 
МД.}

\KW{информационная безопасность; информационные технологии; распределенные  
ин\-фор\-ма\-ци\-он\-но-вы\-чис\-ли\-тель\-ные системы; метаданные; запреты; 
угроза захвата  хоста}

\DOI{10.14357/19922264180406}
  
%\vspace*{4pt}


\vskip 10pt plus 9pt minus 6pt

\thispagestyle{headings}

\begin{multicols}{2}

\label{st\stat}

\section{Введение }

  Концепция защиты сетевых соединений в~\mbox{РИВС} с~по\-мощью 
МД рассматривалась в~работах~[1--3]. Основная идея концепции 
состоит в~том, чтобы в~сети допускать только разрешенные соединения хостов. 
С~этой целью реализуемые в~РИВС информационные технологии  должны 
быть описаны математическими моделями (Unified Modeling Language
(UML), графы зависимостей и~др.)~[4], 
которые чаще всего сложны для управления соединениями. Упрощая эти модели 
с~помощью набора функций, можно получить быст\-ро\-вы\-чис\-ли\-мые ограничения 
на запрашиваемые соединения, которые называем МД. Такой подход 
к~организации взаимодействий в~сети решает целый ряд проблем ИБ, в~том числе 
переносит управление сетью на уровень выше обмена дан\-ными.
  
  Вместе с~тем угрозы захвата хоста в~сети не полностью исследованы~[5]. 
В~этой работе будет рассмотрено несколько не рассмотренных раньше вариантов 
действий нарушителя ИБ в~случае захвата им хоста. Оказалось, что некоторые 
виды нарушений ИБ при захвате хоста не могут быть выявлены на уровне МД по 
любым данным мониторинга. Необходимо получать дополнительные данные 
и~создавать дополнительные технологии для решения рассматриваемой задачи.

\vspace*{-4pt}

\section{Возможные стратегии нарушителя информационной 
безопасности}

\vspace*{-2pt}

  Некоторые стратегии нарушителя ИБ легко обнаружить на уровне МД. Получив 
информацию о~перечне задач, с~которыми нарушитель может установить 
соединение, но не зная скрытого экземпляра идентификатора текущей ИТ, которая 
должна исполняться на захваченном хосте, нарушитель может наудачу запросить 
соединение. В этом случае механизмы управления соединениями на уровне МД 
(задача~$\mathfrak{N}$, см.~[2]) сразу идентифицируют захваченный хост.
  
  В работе~[5] рассматривается более изощренная атака, создающая задержки 
в~выполнении всех действующих ИТ. Однако при правильном распределении 
задач по хостам (задача~$\mathfrak{M}$, см.~[2]) эта атака компенсируется.
  
  Наиболее сложные проблемы возникают, если нарушитель начинает аккуратно 
менять выходные данные задач, которые легально решаются на данном хосте. 
Простейший вариант этой задачи рассмотрен в~[5]. Если нарушитель меняет 
выходные данные решаемой задачи на уже использованные старые данные этой 
задачи, то возможно создание баз данных (БД) значений хеш-функ\-ций выходов всех 
задач. Это могут быть централизованные или локальные БД~[5]. 
Сравнение очередных значений хеш-функ\-ций выходных данных 
с~сохраняющимися в~БД позволяет выявить использование старых данных 
и~захваченный хост. Это возможно, так как повторение значений 
является маловероятным событием и~связано с~повторами значений других 
переменных. Появление одиночного повторения значения переменной 
и~отсутствие повторений в~других переменных указывает на атаку.

\section{Условия невидимости изменения данных на~уровне~метаданных}

  Рассмотрим проблему замены данных нарушителем в~более общей постановке 
и~определим условия, когда выявление подмены невозможно. 
  
  Существуют несколько хороших средств построения математических моделей 
БП, из упрощения которых получаются МД. Назовем 
некоторые из них~\cite{4-gr}:
  \begin{itemize}
\item UML;
\item BPEL (Business Processes Execution Language);
\item ARIS (Architecture of Integration Information Systems).
\end{itemize}

  Детальный анализ существующих средств описания моделей БП показал, что 
они не предполагают использования конкретных значений данных в~этих моделях 
и~связанных с~ними ИТ. В этих моделях всюду используются переменные 
величины и~описания доменов этих переменных, но нигде в~моделях не 
используются значения данных. Это легко понять, так как все БП и~ИТ 
предполагают многократное использование. Поэтому множественный характер 
значений данных не может входить в~описание преобразований, реализуемых БП 
и~ИТ, кроме как в~виде переменных, областей определения и~областей значений. 
  
  Отсюда можно сделать следующий вывод. Любые отображения 
преобразований, входящих в~модели БП и~ИТ, не зависят от значений данных. 
Результаты преобразований БП и~ИТ также описываются переменными, поэтому 
в~описание МД не входят конкретные значения переменных. 
  
  Отсюда вытекает следующее
  
  \smallskip
  
  \noindent
  \textbf{Утверждение.}\ \textit{Метаданные не могут отслеживать 
намеренные изменения данных в~задачах, кроме как с~помощью выхода значений 
переменных в~области запрещенных значений $($запреты}~\cite{6-gr}). 
  
  \smallskip
  
  \noindent
  Д\,о\,к\,а\,з\,а\,т\,е\,л\,ь\,с\,т\,в\,о\,.\ \ Изменение значений переменных 
злоумышленником соответствует изменению значений параметров, определенных 
с~по\-мощью некоторых переменных в~описаниях БП и~ИТ. Метаданные являются 
функциями от преобразований, участвующих в~описаниях БП и~ИТ. Поэтому все 
переменные, встречающиеся в~МД, не содержат конкретных значений, кроме как 
в~описаниях доменов или запретов. 
  
  \smallskip
  
  \noindent
  \textbf{Следствие.} Нарушитель ИБ, захвативший хост и~выбравший стратегию 
намеренного изменения данных, не может быть выявлен без дополнительной 
информации о доменах значений переменных или запретах, получаемых 
с~помощью наблюдений данных мониторинга. 
  
  Это означает, что для выявления нарушителя, выбравшего стратегию изменения 
данных (далее просто нарушителя), необходимо знать запреты на значения 
переменных и~надеяться, что изменения данных приведут к~появлению запретов. 
  

\section{Стратегия выявления нарушителя с~помощью~запретов}

  Можно предложить несколько способов описания доменов и~запретов при 
различных стратегиях изменения данных нарушителем. 
  
  Наиболее общий случай~--- это установление запретов на недопустимые 
значения выполняемых функций, когда появление запрета означает некорректное 
появление данных в~ка\-ких-то задачах. 
  
  Поясним этот метод на следующем примере. Пусть~$A$~--- задача, 
получающая исходные данные от задач~$B$ и~$C$. Пусть хост $H(B)$ захвачен 
и~изменяет данные, а хост $H(C)$ не захвачен. Тогда задача~$A$ получает от 
задачи~$B$ искаженные исходные данные~$x$, а~от задачи~$C$~--- правильные 
исходные данные~$y$. Выполнение задачи~$A$ означает вычисление некоторой 
функции~$f_A$. Положим  $f_A(x,y)\hm=z$. Значения~$x$ и~$y$ часто связаны 
между собой исходными условиями. Поэтому измененные данные~$x$, возможно, 
нарушают эти связи, что отражается на значениях~$z$. Если $z\hm\notin D_A$, где 
$D_A$~--- множество допустимых значений функции~$f_A$, то этот факт можно 
интерпретировать как сбой исходных данных. Отсюда следует, что необходимо 
исследовать хосты~$H(B)$ и~$H(C)$ на захват.
  
  Исследование множеств значений всех функций, используемых в~ИТ,~--- это 
сложная задача. Обычно происходит предварительное тестирование наиболее 
часто встречающихся исходных данных и~для ряда задач~$A$ строятся 
$D_A^\prime \hm\subseteq D_A$, где $D^\prime_A$~--- эмпирическая оценка 
множества~$D_A$~[7]. Использование множеств~$D^\prime_A$ вместо 
множеств~$D_A$ порождает ложные тревоги, когда исходные данные легально 
отличаются от часто встречающихся данных. Поэтому эта методика обычно 
применяется для агрегированных задач, а их крайние значения получаются при 
использовании нагрузочных тестов.

\section{Технология <<ловушек>>}

  <<Ловушки>> (Honey Pot) в~ИТ отличаются от <<ловушек>>, используемых 
для борьбы с~вредоносным кодом. Предлагается следующая идея выявления 
нарушителя, изменяющего данные. 
  
  На этапе опытной эксплуатации запоминаются все траектории процессов 
исполнения ИТ, т.\,е.\ входные и~выходные данные всех задач (множество~$U$). 
Элементы множества~$U$ считаются <<правильными>>. В~случае изменения 
данных хотя бы для одной из задач появляются аномалии~\cite{8-gr}. 
В~соответствии с~некоторым (случайным) расписанием вместо очередного 
экземпляра ИТ запускается экземпляр ИТ $u\hm\in U$. Если нарушитель 
постоянно заменяет данные на захваченном хосте, то этот хост выявляется на 
запуске тестовой технологии~$u$, так как на траектории~$u$ полностью 
определены выходные данные всех задач. Если нарушитель применяет 
<<византийскую>> стратегию, т.\,е.\ изменяет правильные данные не на каждом 
экземпляре ИТ, а на случайно выбранных, то возможность выявления 
захваченного хоста можно определить на основе следующей вероятностной 
модели.  
  
  Пусть $p$~--- вероятность запуска контрольного экземпляра ИТ 
в~последовательности запускаемых ИТ ($n\hm= 1, 2, \ldots$). Пусть~$q$~--- 
вероятность изменения данных в~рассматриваемом экземпляре ИТ. Тогда 
вероятность выявления изменения данных в~очередном экземпляре ИТ 
равна~$pq$. Согласно модели геометрического распределения вероятность 
выявления изменения данных за~$n$~шагов равна $1\hm- (1\hm- pq)^n$. 
Математическое ожидание числа шагов до выявления изменения данных 
рав\-но~$1/(pq)$. 
{\looseness=1

}
  
  В рассматриваемом методе предполагается, что повторений контрольных 
траекторий нет, иначе возможно выявление контроля  нарушителем, так как 
нарушитель запоминает проходящие через захваченный им хост значения 
па\-ра\-метров. 
  
  Реализация метода <<ловушек>> требует специальной архитектуры РИВС. Это 
может быть кли\-ент-сер\-вер\-ная архитектура или варианты, близкие 
к~ней~\cite{3-gr, 9-gr}. Однако в~кли\-ент-сер\-вер\-ной архитектуре исчезает 
необходимость непосредственного взаимодействия хостов между собой. 
  
\section{Заключение}

  Работа посвящена проблеме выявления нарушителя ИБ в~условиях, когда на 
захваченном хосте им выбирается стратегия изменения данных в~задачах ИТ, 
решаемых на этом хосте. Показано, что традиционные средства защиты 
информации не могут выявить нарушителя, кроме как путем выявления запретов 
значений вычисляемых функций. 
  
  Наиболее экономный и~эффективный способ выявления захваченного хоста 
представляет собой <<ловушку>>, которая основана на экземплярах правильных 
траекторий процессов исполнения ИТ. Однако реализация этого подхода требует 
ограничений на архитектуру РИВС. 
  
{\small\frenchspacing
 {%\baselineskip=10.8pt
 \addcontentsline{toc}{section}{References}
 \begin{thebibliography}{9}
\bibitem{1-gr}
\Au{Grusho A., Grusho~N., Zabezhailo~M., Zatsarinny~A., Timonina~E.} Information security of 
SDN on the basis of meta data~// Computer network security~/ Eds. J.~Rak, J.~Bay, I.\,V.~Kotenko, 
\textit{et al.}~--- Lecture notes in computer science ser.~--- Springer, 2017. Vol.~10446. P.~339--347. 
{\sf https://link.springer.com/chapter/10.1007/978-3-319-65127-9\_27}.
\bibitem{2-gr}
\Au{Grusho A.\,A., Timonina E.\,E., Shorgin~S.\,Ya.} Modelling for ensuring information security of 
the distributed information systems~// 31th European Conference on Modelling and Simulation 
Proceedings.~--- Dudweiler, Germany: Digitaldruck Pirrot GmbHP, 2017. P.~656--660. 
{\sf  
http://www.scs-europe.net/dlib/2017/ecms2017acceptedpapers/0656-probstat\_ECMS2017\_0026.pdf}.
\bibitem{3-gr}
\Au{Грушо А.\,А., Тимонина Е.\,Е., Шоргин~С.\,Я.} Иерархический метод порождения 
метаданных для управления сетевыми соединениями~// Информатика и~её применения, 2018. 
Т.~12. Вып.~2. С.~44--49.
\bibitem{4-gr}
\Au{Самуйлов К.\,Е., Чукарин А.\,В., Яркина~Н.\,В.} Биз\-нес-про\-цес\-сы и~информационные 
технологии в~управлении телекоммуникационными компаниями.~--- М.: Альпина Паблишерс, 
2009. 442~с. 
\bibitem{5-gr}
\Au{Grusho A., Timonina~E., Shorgin~S.} Security models based on stochastic meta data~// Analytical 
and computational methods in theory probability~/ Eds. V.~Rykov, N.~Singpurwalla, A.~Zubkov.~---
Lecture notes in computer science ser.~--- Springer, 2017. Vol.~10684: P.~388--400.
{\sf https://link.springer.com/chapter/10.1007/978-3-319-71504-9\_32}. 

\bibitem{6-gr}
\Au{Грушо А.\,А., Тимонина~Е.\,Е.} 
Запреты в~дискретных ве\-ро\-ят\-но\-ст\-но-ста\-ти\-сти\-че\-ских задачах~// 
Дискретная математика, 2011. Т.~23. №\,2. С.~53--58.
\bibitem{7-gr}
\Au{Грушо А.\,А., Грушо~Н.\,А., Тимонина~Е.\,Е.} Статистические методы определения запретов 
вероятностных мер на дискретных пространствах~// Информатика и~её применения, 2013. Т.~7. 
Вып.~1. С.~54--57.
\bibitem{8-gr}
\Au{Grusho A., Grusho~N., Timonina~E.} Detection of anomalies in non-numerical data~// 8th 
Congress (International) on Ultra Modern Telecommunications and Control Systems and Workshops 
Proceedings.~--- Piscataway, NJ, USA: IEEE, 2016. P.~273--276.
{\sf https://ieeexplore.\linebreak ieee.org/document/7765370/}.
\bibitem{9-gr}
\Au{Грушо А.\,А., Грушо~Н.\,А., Забежайло~М.\,И., Тимонина~Е.\,Е.} Защита ценной 
информации в~информационных технологиях~// Проблемы информационной\linebreak без\-опас\-ности. 
Компьютерные системы, 2018. №\,2. С.~22--26.

 \end{thebibliography}

 }
 }

\end{multicols}

\vspace*{-3pt}

\hfill{\small\textit{Поступила в~редакцию 24.09.18}}

\vspace*{8pt}

%\pagebreak

%\newpage

%\vspace*{-28pt}

\hrule

\vspace*{2pt}

\hrule

%\vspace*{-2pt}

\def\tit{METHODS OF IDENTIFICATION OF~HOST CAPTURE 
IN~A~DISTRIBUTED INFORMATION SYSTEM\\ WHICH IS PROTECTED ON~THE~BASIS OF~META DATA}

\def\titkol{Methods of identification of host capture in a distributed information system which is protected on the basis of meta data}

\def\aut{A.\,A.~Grusho, N.\,A.~Grusho, M.\,V.~Levykin, and~E.\,E.~Timonina}

\def\autkol{A.\,A.~Grusho, N.\,A.~Grusho, M.\,V.~Levykin, and~E.\,E.~Timonina}

\titel{\tit}{\aut}{\autkol}{\titkol}

\vspace*{-11pt}


\noindent
Institute of Informatics Problems, Federal Research Center ``Computer Sciences and 
Control'' of the Russian Academy of Sciences, 44-2~Vavilov Str., Moscow 119133, 
Russian Federation


\def\leftfootline{\small{\textbf{\thepage}
\hfill INFORMATIKA I EE PRIMENENIYA~--- INFORMATICS AND
APPLICATIONS\ \ \ 2018\ \ \ volume~12\ \ \ issue\ 4}
}%
 \def\rightfootline{\small{INFORMATIKA I EE PRIMENENIYA~---
INFORMATICS AND APPLICATIONS\ \ \ 2018\ \ \ volume~12\ \ \ issue\ 4
\hfill \textbf{\thepage}}}

\vspace*{6pt}


   
\Abste{The model of a distributed information system in which permissions 
on network connections are based on meta data is considered. Meta data 
are simplification of business process models. It is proved that the 
adversary of information security who captured a~host and accurately attacked 
a~system by means of changes of output data of tasks solved on this host cannot 
be detected at the level of meta data. The problem is connected with the fact that 
a~business process model and, therefore, meta data operate with variables for which 
changes of specific values are not reflected in their description. 
Exceptions are output cases on forbidden values, for example, out of limits of 
definition ranges and a~set of values of functions of which information technologies 
are constructed. Additional variants of information security measures which consider 
``invisibility'' of similar violations of information security at the level of meta 
data are suggested.}

\KWE{information security; information technologies; distributed information system; 
meta data; ban; threat of host capture}
   
     

\DOI{10.14357/19922264180406}

%\vspace*{-14pt}

\Ack
\noindent
The paper was supported by the Russian Foundation 
for Basic Research (project 18-07-00274).



%\vspace*{6pt}

  \begin{multicols}{2}

\renewcommand{\bibname}{\protect\rmfamily References}
%\renewcommand{\bibname}{\large\protect\rm References}

{\small\frenchspacing
 {%\baselineskip=10.8pt
 \addcontentsline{toc}{section}{References}
 \begin{thebibliography}{9}
\bibitem{1-gr-1}
\Aue{Grusho, A., N.~Grusho, M.~Zabezhailo, A.~Zatsarinny, and E.~Timonina.} 2017. Information 
security of SDN on the basis of meta data. \textit{Computer network security}. Eds. J.~Rak, J.~Bay, 
I.\,V.~Kotenko, \textit{et al.} Lecture notes in computer science ser. Springer. 10446:339--347. 
Available at: {\sf https://link.springer.com/chapter/10.1007/978-3-319-65127-9\_27} (September~23, 2018).
\bibitem{2-gr-1}
\Aue{Grusho, A.\,A., E.\,E.~Timonina, and S.\,Ya.~Shorgin.} 
2017. Modelling for ensuring 
information security of the distributed information systems. \textit{31th European Conference on 
Modelling and Simulation Proceedings}. 
Dudweiler, Germany: Digitaldruck Pirrot GmbHP. 656--660. 
Available at: {\sf  
http://www.scs-europe.net/dlib/2017/ecms2017acceptedpapers/0656-probstat\_ECMS2017\_0026.pdf} (accessed 
September~23, 2018).
\bibitem{3-gr-1}
\Aue{Grusho, A.\,A., E.\,E.~Timonina, and S.\,Ya.~Shorgin.} 2018. Ierarkhicheskiy metod 
porozhdeniya metadannykh dlya upravleniya setevymi soedineniyami [Hierarchical method of meta 
data generation for control of network connections]. \textit{Informatika i~ee Primeneniya~--- Inform. 
Appl.} 12(2):44--49.
\bibitem{4-gr-1}
\Aue{Samuylov, K.\,E., A.\,V.~Chukarin, and N.\,V.~Yarkina.}  2009. \textit{Biznes-protsessy 
i~informatsionnye tekhnologii v~upravlenii telekommunikatsionnymi kompaniyami} [Business 
processes and information technologies in management of the telecommunication companies]. 
Moscow: Alpina Pabls. 442~p. 
\bibitem{5-gr-1}
\Aue{Grusho, A., E.~Timonina, and S.~Shorgin.} 2017. Security models based on stochastic meta 
data. \textit{Analytical and computational methods in theory probability}. 
Eds. V.~Rykov, 
N.~Singpurwalla, and A.~Zubkov. Lecture notes in computer science ser. Springer. 10684:388--400. 
Available at: {\sf https://link.springer.com/chapter/10.1007/978-3-319-71504-9\_32} (accessed September~23, 
2018). 
\bibitem{6-gr-1}
\Aue{Grusho, A., and E.~Timonina.} 2011. Prohibitions in discrete probabilistic statistical problems. 
\textit{Discrete Mathematics Applications} 21(3):275--281.
\bibitem{7-gr-1}
\Aue{Grusho, A.\,A., N.\,A.~Grusho, and E.\,E.~Timonina.} 2013. Statisticheskie metody 
opredeleniya zapretov veroyatnostnykh mer na diskretnykh prostranstvakh [Statistical methods of 
definition of the bans of probability measures on discrete spaces]. \textit{Informatika i~ee 
Primeneniya~--- Inform. Appl.} 7(1):54--57.
\bibitem{8-gr-1}
\Aue{Grusho, A., N.~Grusho, and E.~Timonina.} 2016. Detection of anomalies in non-numerical data. 
\textit{8th Congress (International) on Ultra Modern Telecommunications and Control Systems and 
Workshops Proceedings}. Piscataway, NJ: IEEE. 273--276.
Available at: {\sf https://ieeexplore.ieee.\linebreak org/document/7765370/} (accessed September~23, 2018). 
\bibitem{9-gr-1}
\Aue{Grusho, A., N.~Grusho,  M.~Zabezhailo, and E.~Timonina.} 2018. Zashchita tsennoy 
informatsii v~informatsionnykh tekhnologiyakh [Protection of valuable information in information 
technologies]. \textit{Information Security Problems. Computer Systems} 2:22--26.
\end{thebibliography}

 }
 }

\end{multicols}

\vspace*{-6pt}

\hfill{\small\textit{Received September 24, 2018}}

%\pagebreak

%\vspace*{-18pt}

\Contr

\noindent
\textbf{Grusho Alexander A.} (b.\ 1946)~--- Doctor of Science in physics and 
mathematics, professor, principal scientist, Institute of Informatics Problems, Federal 
Research Center ``Computer Sciences and Control'' of the Russian Academy of 
Sciences, 44-2~Vavilov Str., Moscow 119133, Russian Federation; 
\mbox{grusho@yandex.ru}

\vspace*{3pt}

\noindent
\textbf{Grusho Nikolai A.} (b.\ 1982)~--- Candidate of Science (PhD) in physics 
and mathematics, senior scientist, Institute of Informatics Problems, Federal 
Research Center ``Computer Sciences and Control'' of the Russian Academy of 
Sciences, 44-2~Vavilov Str., Moscow 119133, Russian Federation; 
\mbox{info@itake.ru}

\vspace*{3pt}

\noindent
\textbf{Levykin Mikhail V.} (b.\ 1985)~--- Candidate of Science (PhD) in 
technology, senior scientist, Institute of Informatics Problems, Federal Research 
Center ``Computer Sciences and Control'' of the Russian Academy of Sciences,  
44-2~Vavilov Str., Moscow 119133, Russian Federation; 
\mbox{de\_shiko@yahoo.com}

\vspace*{3pt}

\noindent
\textbf{Timonina Elena E.} (b.\ 1952)~--- Doctor of Science in technology, 
professor, leading scientist, Institute of Informatics Problems, Federal Research 
Center ``Computer Sciences and Control'' of the Russian Academy of Sciences,  
44-2~Vavilov Str., Moscow 119133, Russian Federation; 
\mbox{eltimon@yandex.ru} 

\label{end\stat}

\renewcommand{\bibname}{\protect\rm Литература}       